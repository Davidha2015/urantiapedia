\begin{document}

\title{Exodus}

 
\chapter{1}

\par 1 و این است نامهای پسران اسرائیل که به مصر آمدند، هر کس با اهل خانه‌اش همراه یعقوب آمدند:
\par 2 روبین و شمعون و لاوی ویهودا،
\par 3 یساکار و زبولون و بنیامین،
\par 4 و دان ونفتالی، و جاد و اشیر.
\par 5 و همه نفوسی که از صلب یعقوب پدید آمدند هفتاد نفر بودند. و یوسف درمصر بود.
\par 6 و یوسف و همه برادرانش، و تمامی آن طبقه مردند.
\par 7 و بنی‌اسرائیل بارور و منتشر شدند، وکثیر و بی‌نهایت زورآور گردیدند. و زمین ازایشان پر گشت.
\par 8 اما پادشاهی دیگر بر مصربرخاست که یوسف را نشناخت.
\par 9 و به قوم خودگفت: «همانا قوم بنی‌اسرائیل از ما زیاده وزورآورترند.
\par 10 بیایید با ایشان به حکمت رفتارکنیم، مبادا که زیاد شوند. و واقع شود که چون جنگ پدید آید، ایشان نیز با دشمنان ماهمداستان شوند، و با ما جنگ کرده، از زمین بیرون روند.»
\par 11 پس سرکاران بر ایشان گماشتند، تا ایشان را به‌کارهای دشوار ذلیل سازند، و برای فرعون شهرهای خزینه، یعنی فیتوم و رعمسیس را بناکردند.
\par 12 لیکن چندانکه بیشتر ایشان را ذلیل ساختند، زیادتر متزاید و منتشر گردیدند، و ازبنی‌اسرائیل احتراز می‌نمودند.
\par 13 و مصریان ازبنی‌اسرائیل به ظلم خدمت گرفتند.
\par 14 و جانهای ایشان را به بندگی سخت، به گل کاری وخشت سازی و هر گونه عمل صحرایی، تلخ ساختندی. و هر خدمتی که بر ایشان نهادندی به ظلم می‌بود.
\par 15 و پادشاه مصر به قابله های عبرانی که یکی را شفره و دیگری را فوعه نام بود، امرکرده،
\par 16 گفت: «چون قابله گری برای زنان عبرانی بکنید، و بر سنگها نگاه کنید، اگر پسر باشد او رابکشید، و اگر دختر بود زنده بماند.»
\par 17 لکن قابله‌ها از خدا ترسیدند، و آنچه پادشاه مصربدیشان فرموده بود نکردند، بلکه پسران را زنده گذاردند.
\par 18 پس پادشاه مصر قابله‌ها را طلبیده، بدیشان گفت: «چرا این کار را کردید، و پسران را زنده گذاردید؟»
\par 19 قابله‌ها به فرعون گفتند: «از این سبب که زنان عبرانی چون زنان مصری نیستند، بلکه زورآورند، و قبل از رسیدن قابله می‌زایند.»
\par 20 و خدا با قابله‌ها احسان نمود، و قوم کثیرشدند، و بسیار توانا گردیدند.
\par 21 و واقع شدچونکه قابله‌ها از خدا ترسیدند، خانه‌ها برای ایشان بساخت.و فرعون قوم خود را امر کرده، گفت: «هر پسری که زاییده شود به نهر اندازید، وهر دختری را زنده نگاه دارید.
\par 22 و فرعون قوم خود را امر کرده، گفت: «هر پسری که زاییده شود به نهر اندازید، وهر دختری را زنده نگاه دارید.
 
\chapter{2}

\par 1 و شخصی از خاندان لاوی رفته، یکی ازدختران لاوی را به زنی گرفت.
\par 2 و آن زن حامله شده، پسری بزاد. و چون او را نیکومنظردید، وی را سه ماه نهان داشت.
\par 3 و چون نتوانست او را دیگر پنهان دارد، تابوتی از نی برایش گرفت، و آن را به قیر و زفت اندوده، طفل را در آن نهاد، وآن را در نیزار به کنار نهر گذاشت.
\par 4 و خواهرش ازدور ایستاد تا بداند او را چه می‌شود.
\par 5 و دخترفرعون برای غسل به نهر فرود آمد. و کنیزانش به کنار نهر می‌گشتند. پس تابوت را در میان نیزاردیده، کنیزک خویش را فرستاد تا آن را بگیرد.
\par 6 وچون آن را بگشاد، طفل را دید و اینک پسری گریان بود. پس دلش بر وی بسوخت و گفت: «این از اطفال عبرانیان است.»
\par 7 و خواهر وی به دخترفرعون گفت: «آیا بروم و زنی شیرده را از زنان عبرانیان نزدت بخوانم تا طفل را برایت شیردهد؟»
\par 8 دختر فرعون به وی گفت: «برو.» پس آن دختر رفته، مادر طفل را بخواند.
\par 9 و دختر فرعون گفت: «این طفل را ببر و او را برای من شیر بده ومزد تو را خواهم داد.» پس آن زن طفل رابرداشته، بدو شیر می‌داد.
\par 10 و چون طفل نموکرد، وی را نزد دختر فرعون برد، و او را پسر شد. و وی را موسی نام نهاد زیرا گفت: «او را از آب کشیدم.»
\par 11 و واقع شد در آن ایام که چون موسی بزرگ شد، نزد برادران خود بیرون آمد، و به‌کارهای دشوار ایشان نظر انداخته، شخصی مصری را دیدکه شخصی عبرانی را که از برادران او بود، می‌زند.
\par 12 پس به هر طرف نظر افکنده، چون کسی را ندید، آن مصری را کشت، و او را در ریگ پنهان ساخت.
\par 13 و روز دیگر بیرون آمد، که ناگاه دومرد عبرانی منازعه می‌کنند، پس به ظالم گفت: «چرا همسایه خود را می‌زنی.»
\par 14 گفت: «کیست که تو را بر ما حاکم یا داور ساخته است، مگر تومی خواهی مرا بکشی چنانکه آن مصری راکشتی؟» پس موسی ترسید و گفت: «یقین این امرشیوع یافته است.»
\par 15 و چون فرعون این ماجرا رابشنید، قصد قتل موسی کرد، و موسی از حضورفرعون فرار کرده، در زمین مدیان ساکن شد. و برسر چاهی بنشست.
\par 16 و کاهن مدیان را هفت دختر بود که آمدند و آب کشیده، آبخورها را پرکردند، تا گله پدر خویش را سیراب کنند.
\par 17 وشبانان نزدیک آمدند، تا ایشان را دور کنند. آنگاه موسی برخاسته، ایشان را مدد کرد، و گله ایشان راسیراب نمود.
\par 18 و چون نزد پدر خود رعوئیل آمدند، او گفت: «چگونه امروز بدین زودی برگشتید؟»
\par 19 گفتند: «شخصی مصری ما را ازدست شبانان رهایی داد، و آب نیز برای ما کشیده، گله را سیراب نمود.»
\par 20 پس به دختران خودگفت: «او کجاست؟ چرا آن مرد را ترک کردید؟ وی را بخوانید تا نان خورد.»
\par 21 و موسی راضی شد که با آن مرد ساکن شود، و او دختر خود، صفوره را به موسی داد.
\par 22 و آن زن پسری زایید، و (موسی ) او را جرشون نام نهاد، چه گفت: «در زمین بیگانه نزیل شدم.»
\par 23 و واقع شد بعد از ایام بسیار که پادشاه مصربمرد، و بنی‌اسرائیل به‌سبب بندگی آه کشیده، استغاثه کردند، و ناله ایشان به‌سبب بندگی نزد خدا برآمد.
\par 24 و خدا ناله ایشان را شنید، و خداعهد خود را با ابراهیم و اسحاق و یعقوب بیادآورد.و خدا بر بنی‌اسرائیل نظر کرد و خدادانست.
\par 25 و خدا بر بنی‌اسرائیل نظر کرد و خدادانست.
 
\chapter{3}

\par 1 و اما موسی گله پدر زن خود، یترون، کاهن مدیان را شبانی می‌کرد، و گله را بدان طرف صحرا راند و به حوریب که جبل الله باشد آمد.
\par 2 وفرشته خداوند در شعله آتش از میان بوته‌ای بروی ظاهر شد، و چون او نگریست، اینک آن بوته به آتش مشتعل است اما سوخته نمی شود.
\par 3 وموسی گفت: «اکنون بدان طرف شوم، و این امرغریب را ببینم، که بوته چرا سوخته نمی شود.»
\par 4 چون خداوند دید که برای دیدن مایل بدان سومی شود، خدا از میان بوته به وی ندا درداد و گفت: «ای موسی! ای موسی!» گفت: «لبیک.»
\par 5 گفت: «بدین جا نزدیک میا، نعلین خود را از پایهایت بیرون کن، زیرا مکانی که در آن ایستاده‌ای زمین مقدس است.»
\par 6 و گفت: «من هستم خدای پدرت، خدای ابراهیم، و خدای اسحاق، وخدای یعقوب.» آنگاه موسی روی خود راپوشانید، زیرا ترسید که به خدا بنگرد.
\par 7 وخداوند گفت: «هر آینه مصیبت قوم خود را که درمصرند دیدم، و استغاثه ایشان را از دست سرکاران ایشان شنیدم، زیرا غمهای ایشان رامی دانم.
\par 8 و نزول کردم تا ایشان را از دست مصریان خلاصی دهم، و ایشان را از آن زمین به زمین نیکو و وسیع برآورم، به زمینی که به شیر وشهد جاری است، به مکان کنعانیان و حتیان و اموریان و فرزیان و حویان و یبوسیان.
\par 9 و الان اینک استغاثه بنی‌اسرائیل نزد من رسیده است، وظلمی را نیز که مصریان بر ایشان می‌کنند، دیده‌ام.
\par 10 پس اکنون بیا تا تو را نزد فرعون بفرستم، و قوم من، بنی‌اسرائیل را از مصر بیرون آوری.»
\par 11 موسی به خدا گفت: «من کیستم که نزدفرعون بروم، و بنی‌اسرائیل را از مصر بیرون آورم؟»
\par 12 گفت: «البته با تو خواهم بود. و علامتی که من تو را فرستاده‌ام، این باشد که چون قوم را ازمصر بیرون آوردی، خدا را بر این کوه عبادت خواهید کرد.»
\par 13 موسی به خدا گفت: «اینک چون من نزد بنی‌اسرائیل برسم، و بدیشان گویم خدای پدران شما مرا نزد شما فرستاده است، و ازمن بپرسند که نام او چیست، بدیشان چه گویم؟»
\par 14 خدا به موسی گفت: «هستم آنکه هستم.» وگفت: «به بنی‌اسرائیل چنین بگو: اهیه (هستم ) مرانزد شما فرستاد.»
\par 15 و خدا باز به موسی گفت: «به بنی‌اسرائیل چنین بگو، یهوه خدای پدران شما، خدای ابراهیم و خدای اسحاق و خدای یعقوب، مرا نزد شما فرستاده، این است نام من تا ابدالاباد، و این است یادگاری من نسلا بعد نسل.
\par 16 برو ومشایخ بنی‌اسرائیل را جمع کرده، بدیشان بگو: یهوه خدای پدران شما، خدای ابراهیم و اسحاق و یعقوب، به من ظاهر شده، گفت: هر آینه از شماو از آنچه به شما در مصر کرده‌اند، تفقد کرده‌ام،
\par 17 و گفتم شما را از مصیبت مصر بیرون خواهم آورد، به زمین کنعانیان و حتیان و اموریان وفرزیان و حویان و یبوسیان، به زمینی که به شیر وشهد جاری است.
\par 18 و سخن تو را خواهندشنید، و تو با مشایخ اسرائیل، نزد پادشاه مصر بروید، و به وی گویید: یهوه خدای عبرانیان ما راملاقات کرده است. و الان سفر سه روزه به صحرابرویم، تا برای یهوه خدای خود قربانی بگذرانیم.
\par 19 و من می‌دانم که پادشاه مصر شما را نمی گذاردبروید، و نه هم به‌دست زورآور.
\par 20 پس دست خود را دراز خواهم کرد، و مصر را به همه عجایب خود که در میانش به ظهور می‌آورم خواهم زد، و بعد از آن شما را رها خواهد کرد.
\par 21 و این قوم را در نظر مصریان مکرم خواهم ساخت، و واقع خواهد شد که چون بروید تهی‌دست نخواهید رفت.بلکه هر زنی از همسایه خود و مهمان خانه خویش آلات نقره و آلات طلا و رخت خواهد خواست، و به پسران ودختران خود خواهید پوشانید، و مصریان راغارت خواهید نمود.»
\par 22 بلکه هر زنی از همسایه خود و مهمان خانه خویش آلات نقره و آلات طلا و رخت خواهد خواست، و به پسران ودختران خود خواهید پوشانید، و مصریان راغارت خواهید نمود.»
 
\chapter{4}

\par 1 تصدیق نخواهند کرد، و سخن مرانخواهند شنید، بلکه خواهند گفت یهوه بر توظاهر نشده است.»
\par 2 پس خداوند به وی گفت: «آن چیست در دست تو؟» گفت: «عصا.»
\par 3 گفت: «آن را بر زمین بینداز.» و چون آن را به زمین انداخت، ماری گردید و موسی از نزدش گریخت.
\par 4 پس خداوند به موسی گفت: «دست خود رادراز کن و دمش را بگیر.» پس دست خود را درازکرده، آن را بگرفت، که در دستش عصا شد.
\par 5 «تاآنکه باور کنند که یهوه خدای پدران ایشان، خدای ابراهیم، خدای اسحاق، و خدای یعقوب، به تو ظاهر شد.»
\par 6 و خداوند دیگرباره وی راگفت: «دست خود را در گریبان خود بگذار.» چون دست به گریبان خود برد، و آن را بیرون آورد، اینک دست او مثل برف مبروص شد.
\par 7 پس گفت: «دست خود را باز به گریبان خود بگذار.» چون دست به گریبان خود باز برد، و آن را بیرون آورد، اینک مثل سایر بدنش باز آمده بود.
\par 8 «وواقع خواهد شد که اگر تو را تصدیق نکنند، وآواز آیت نخستین را نشنوند، همانا آواز آیت دوم را باور خواهند کرد.
\par 9 و هر گاه این دو آیت راباور نکردند و سخن تو را نشنیدند، آنگاه از آب نهر گرفته، به خشکی بریز، و آبی که از نهر گرفتی بر روی خشکی به خون مبدل خواهد شد.»
\par 10 پس موسی به خداوند گفت: «ای خداوند، من مردی فصیح نیستم، نه در سابق و نه از وقتی که به بنده خود سخن گفتی، بلکه بطی الکلام و کندزبان.»
\par 11 خداوند گفت: «کیست که زبان به انسان داد، و گنگ و کر و بینا و نابینا را که آفرید؟ آیا نه من که یهوه هستم؟
\par 12 پس الان برو و من با زبانت خواهم بود، و هر‌چه باید بگویی تو را خواهم آموخت.»
\par 13 گفت: «استدعا دارم‌ای خداوند که بفرستی به‌دست هر‌که می‌فرستی.»
\par 14 آنگاه خشم خداوند بر موسی مشتعل شدو گفت: «آیا برادرت، هارون لاوی را نمی دانم که او فصیح الکلام است؟ و اینک او نیز به استقبال توبیرون می‌آید، و چون تو را ببیند، در دل خود شادخواهد گردید.
\par 15 و بدو سخن خواهی گفت وکلام را به زبان وی القا خواهی کرد، و من با زبان توو با زبان او خواهم بود، و آنچه باید بکنید شما راخواهم آموخت.
\par 16 و او برای تو به قوم سخن خواهد گفت، و او مر تو را به‌جای زبان خواهدبود، و تو او را به‌جای خدا خواهی بود. 
\par 17 و این عصا را به‌دست خود بگیر که به آن آیات را ظاهر سازی.»
\par 18 پس موسی روانه شده، نزد پدر زن خود، یترون، برگشت و به وی گفت: «بروم و نزد برادران خود که در مصرند برگردم، و ببینم که تا کنون زنده‌اند.» یترون به موسی گفت: «به سلامتی برو.»
\par 19 و خداوند در مدیان به موسی گفت: «روانه شده به مصر برگرد، زیرا آنانی که در قصد جان توبودند، مرده‌اند.»
\par 20 پس موسی زن خویش وپسران خود را برداشته، ایشان را بر الاغ سوارکرده، به زمین مصر مراجعت نمود، و موسی عصای خدا را به‌دست خود گرفت.
\par 21 و خداوندبه موسی گفت: «چون روانه شده، به مصرمراجعت کردی، آگاه باش که همه علاماتی را که به‌دستت سپرده‌ام به حضور فرعون ظاهر سازی، و من دل او را سخت خواهم ساخت تا قوم را رهانکند.
\par 22 و به فرعون بگو خداوند چنین می‌گوید: اسرائیل، پسر من و نخست زاده من است،
\par 23 و به تو می‌گویم پسرم را رها کن تا مرا عبادت نماید، واگر از رها کردنش ابا نمایی، همانا پسر تو، یعنی نخست زاده تو را می‌کشم.»
\par 24 و واقع شد در بین راه که خداوند در منزل بدو برخورده، قصد قتل وی نمود.
\par 25 آنگاه صفوره سنگی تیز گرفته، غلفه پسر خود را ختنه کرد و نزد پای وی انداخته، گفت: «تو مرا شوهرخون هستی.»
\par 26 پس او وی را رها کرد، آنگاه (صفوره ) گفت: «شوهر خون هستی، » به‌سبب ختنه.
\par 27 و خداوند به هارون گفت: «به سوی صحرا به استقبال موسی برو.» پس روانه شد و او را در جبل الله ملاقات کرده، او را بوسید.
\par 28 وموسی از جمیع کلمات خداوند که او را فرستاده بود، و از همه آیاتی که به وی امر فرموده بود، هارون را خبر داد.
\par 29 پس موسی و هارون رفته، کل مشایخ بنی‌اسرائیل را جمع کردند.
\par 30 وهارون همه سخنانی را که خداوند به موسی فرموده بود، باز‌گفت، و آیات را به نظر قوم ظاهرساخت.و قوم ایمان آوردند. و چون شنیدندکه خداوند از بنی‌اسرائیل تفقد نموده، و به مصیبت ایشان نظر انداخته است، به روی درافتاده، سجده کردند.
\par 31 و قوم ایمان آوردند. و چون شنیدندکه خداوند از بنی‌اسرائیل تفقد نموده، و به مصیبت ایشان نظر انداخته است، به روی درافتاده، سجده کردند.
 
\chapter{5}

\par 1 و بعد از آن موسی و هارون آمده، به فرعون گفتند: «یهوه خدای اسرائیل چنین می‌گوید: قوم مرا رها کن تا برای من در صحرا عیدنگاه دارند.»
\par 2 فرعون گفت: «یهوه کیست که قول او را بشنوم و اسرائیل را رهایی دهم؟ یهوه رانمی شناسم و اسرائیل را نیز رها نخواهم کرد.»
\par 3 گفتند: «خدای عبرانیان ما را ملاقات کرده است، پس الان سفر سه روزه به صحرا برویم، ونزد یهوه، خدای خود، قربانی بگذرانیم، مبادا مارا به وبا یا شمشیر مبتلا سازد.»
\par 4 پس پادشاه مصربدیشان گفت: «ای موسی و هارون چرا قوم را ازکارهای ایشان بازمی دارید؟ به شغلهای خودبروید!»
\par 5 و فرعون گفت: «اینک الان اهل زمین بسیارند، و ایشان را از شغلهای ایشان بیکارمی سازید.»
\par 6 و در آن روز، فرعون سرکاران و ناظران قوم خود را قدغن فرموده، گفت:
\par 7 «بعد از این، کاه برای خشت سازی مثل سابق بدین قوم مدهید. خود بروند و کاه برای خویشتن جمع کنند.
\par 8 وهمان حساب خشتهایی را که پیشتر می‌ساختند، برایشان بگذارید، و از آن هیچ کم مکنید، زیراکاهلند، و از این‌رو فریاد می‌کنند و می‌گویند: برویم تا برای خدای خود قربانی گذرانیم.
\par 9 وخدمت ایشان سخت‌تر شود تا در آن مشغول شوند، و به سخنان باطل اعتنا نکنند.»
\par 10 پس سرکاران و ناظران قوم بیرون آمده، قوم را خطاب کرده، گفتند: «فرعون چنین می‌فرماید که من کاه به شما نمی دهم.
\par 11 خود بروید و کاه برای خود ازهرجا که بیابید بگیرید، و از خدمت شما هیچ کم نخواهد شد.»
\par 12 پس قوم در تمامی زمین مصر پراکنده شدند تا خاشاک به عوض کاه جمع کنند.
\par 13 وسرکاران، ایشان را شتابانیده، گفتند: «کارهای خود را تمام کنید، یعنی حساب هر روز را درروزش، مثل وقتی که کاه بود.»
\par 14 و ناظران، بنی‌اسرائیل را که سرکاران فرعون بر ایشان گماشته بودند، می‌زدند و می‌گفتند: «چرا خدمت معین خشت سازی خود را در این‌روزها مثل سابق تمام نمی کنید؟»
\par 15 آنگاه ناظران بنی‌اسرائیل آمده، نزد فرعون فریاد کرده، گفتند: «چرا به بندگان خود چنین می‌کنی؟
\par 16 کاه به بندگانت نمی دهند و می‌گویند: خشت برای مابسازید! و اینک بندگانت را می‌زنند و اما خطا ازقوم تو می‌باشد.»
\par 17 گفت: «کاهل هستید. شماکاهلید! از این سبب شما می‌گویید: برویم و برای خداوند قربانی بگذرانیم.
\par 18 اکنون رفته، خدمت بکنید، و کاه به شما داده نخواهد شد، و حساب خشت را خواهید داد.»
\par 19 و ناظران بنی‌اسرائیل دیدند که در بدی گرفتار شده‌اند، زیرا گفت: «ازحساب یومیه خشتهای خود هیچ کم مکنید.»
\par 20 و چون از نزد فرعون بیرون آمدند، به موسی وهارون برخوردند، که برای ملاقات ایشان ایستاده بودند.
\par 21 و بدیشان گفتند: «خداوند بر شما بنگردو داوری فرماید! زیرا که رایحه ما را نزد فرعون وملازمانش متعفن ساخته‌اید، و شمشیری به‌دست ایشان داده‌اید تا ما را بکشند.»
\par 22 آنگاه موسی نزد خداوند برگشته، گفت: «خداوندا چرا بدین قوم بدی کردی؟ و برای چه مرا فرستادی؟زیرا از وقتی که نزد فرعون آمدم تا به نام تو سخن گویم، بدین قوم بدی کرده است و قوم خود را هرگز خلاصی ندادی.»
\par 23 زیرا از وقتی که نزد فرعون آمدم تا به نام تو سخن گویم، بدین قوم بدی کرده است و قوم خود را هرگز خلاصی ندادی.»
 
\chapter{6}

\par 1 خداوند به موسی گفت: «الان خواهی دیدآنچه به فرعون می‌کنم، زیرا که به‌دست قوی ایشان را رها خواهد کرد، و به‌دست زورآورایشان را از زمین خود خواهد راند.»
\par 2 و خدا به موسی خطاب کرده، وی را گفت: «من یهوه هستم.
\par 3 و به ابراهیم و اسحاق و یعقوب به نام خدای قادرمطلق ظاهر شدم، لیکن به نام خود، یهوه، نزد ایشان معروف نگشتم.
\par 4 و عهد خود رانیز با ایشان استوار کردم، که زمین کنعان را بدیشان دهم، یعنی زمین غربت ایشان را که در آن غریب بودند.
\par 5 و من نیز چون ناله بنی‌اسرائیل را که مصریان ایشان را مملوک خود ساخته‌اند، شنیدم، عهد خود را بیاد آوردم.
\par 6 بنابراین بنی‌اسرائیل رابگو، من یهوه هستم، و شما را از زیر مشقتهای مصریان بیرون خواهم آورد، و شما را از بندگی ایشان رهایی دهم، و شما را به بازوی بلند و به داوری های عظیم نجات دهم.
\par 7 و شما را خواهم گرفت تا برای من قوم شوید، و شما را خداخواهم بود، و خواهید دانست که من یهوه هستم، خدای شما، که شما را از مشقتهای مصریان بیرون آوردم.
\par 8 و شما را خواهم رسانید به زمینی که درباره آن قسم خوردم که آن را به ابراهیم واسحاق و یعقوب بخشم، پس آن را به ارثیت شماخواهم داد. من یهوه هستم.»
\par 9 و موسی بنی‌اسرائیل را بدین مضمون گفت، لیکن بسبب تنگی روح و سختی خدمت، او رانشنیدند.
\par 10 و خداوند موسی را خطاب کرده، گفت:
\par 11 «برو و به فرعون پادشاه مصر بگو که بنی‌اسرائیل را از زمین خود رهایی دهد.»
\par 12 وموسی به حضور خداوند عرض کرده، گفت: «اینک بنی‌اسرائیل مرا نمی شنوند، پس چگونه فرعون مرا بشنود، و حال آنکه من نامختون لب هستم؟»
\par 13 و خداوند به موسی و هارون تکلم نموده، ایشان را به سوی بنی‌اسرائیل و به سوی فرعون پادشاه مصر مامور کرد، تا بنی‌اسرائیل را از زمین مصر بیرون آورند.
\par 14 و اینانند روسای خاندانهای آبای ایشان: پسران روبین، نخست زاده اسرائیل، حنوک و فلو و حصرون و کرمی؛ اینانندقبایل روبین.
\par 15 و پسران شمعون: یموئیل و یامین و اوهد و یاکین و صوحر و شاول که پسر زن کنعانی بود؛ اینانند قبایل شمعون.
\par 16 و این است نامهای پسران لاوی به حسب پیدایش ایشان: جرشون و قهات و مراری. و سالهای عمر لاوی صد و سی و هفت سال بود.
\par 17 پسران جرشون: لبنی و شمعی، به حسب قبایل ایشان.
\par 18 و پسران قهات: عمرام و یصهار و حبرون و عزیئیل. وسالهای عمر قهات صد و سی و سه سال بود.
\par 19 وپسران مراری: محلی و موشی؛ اینانند قبایل لاویان به حسب پیدایش ایشان.
\par 20 و عمرام عمه خود، یوکابد را به زنی گرفت، و او برای وی هارون و موسی را زایید، و سالهای عمر عمرام صد و سی و هفت سال بود.
\par 21 و پسران یصهار: قورح و نافج و زکری.
\par 22 و پسران عزیئیل: میشائیل و ایلصافن و ستری.
\par 23 و هارون، الیشابع، دختر عمیناداب، خواهر نحشون را به زنی گرفت، و برایش ناداب و ابیهو و العازر وایتامر را زایید.
\par 24 و پسران قورح: اسیر و القانه وابیاساف؛ اینانند قبایل قورحیان.
\par 25 و العازر بن هارون یکی از دختران فوتیئیل را به زنی گرفت، وبرایش فینحاس را زایید؛ اینانند روسای آبای لاویان، بحسب قبایل ایشان.
\par 26 اینانند هارون وموسی که خداوند بدیشان گفت: «بنی‌اسرائیل رابا جنود ایشان از زمین مصر بیرون آورید.»
\par 27 اینانند که به فرعون پادشاه مصر سخن‌گفتند، برای بیرون آوردن بنی‌اسرائیل از مصر. اینان موسی و هارونند.
\par 28 و واقع شد در روزی که خداوند در زمین مصر موسی را خطاب کرد.
\par 29 که خداوند به موسی فرموده، گفت: «من یهوه هستم هر‌آنچه من به تو گویم آن را به فرعون پادشاه مصر بگو.»وموسی به حضور خداوند عرض کرد: «اینک من نامختون لب هستم، پس چگونه فرعون مرابشنود؟»
\par 30 وموسی به حضور خداوند عرض کرد: «اینک من نامختون لب هستم، پس چگونه فرعون مرابشنود؟»
 
\chapter{7}

\par 1 و خداوند به موسی گفت: «ببین تو را برفرعون خدا ساخته‌ام، و برادرت، هارون، نبی تو خواهد بود.
\par 2 هرآنچه به تو امر نمایم تو آن را بگو، و برادرت هارون، آن را به فرعون بازگوید، تا بنی‌اسرائیل را از زمین خود رهایی دهد.
\par 3 و من دل فرعون را سخت می‌کنم، و آیات وعلامات خود را در زمین مصر بسیار می‌سازم.
\par 4 وفرعون به شما گوش نخواهد گرفت، و دست خودرا بر مصر خواهم‌انداخت، تا جنود خود، یعنی قوم خویش بنی‌اسرائیل را از زمین مصر به داوریهای عظیم بیرون آورم.
\par 5 و مصریان خواهند دانست که من یهوه هستم، چون دست خود را بر مصر دراز کرده، بنی‌اسرائیل را از میان ایشان بیرون آوردم.»
\par 6 و موسی و هارون چنانکه خداوند بدیشان امر فرموده بود کردند، و هم چنین عمل نمودند.
\par 7 و موسی هشتاد ساله بود وهارون هشتاد و سه ساله، وقتی که به فرعون سخن‌گفتند.
\par 8 پس خداوند موسی و هارون را خطاب کرده، گفت:
\par 9 «چون فرعون شما را خطاب کرده، گوید معجزه‌ای برای خود ظاهر کنید، آنگاه به هارون بگو عصای خود را بگیر، و آن را پیش روی فرعون بینداز، تا اژدها شود.»
\par 10 آنگاه موسی و هارون نزد فرعون رفتند، و آنچه خداوندفرموده بود کردند. و هارون عصای خود را پیش روی فرعون و پیش روی ملازمانش انداخت، واژدها شد.
\par 11 و فرعون نیز حکیمان و جادوگران را طلبید و ساحران مصر هم به افسونهای خودچنین کردند،
\par 12 هر یک عصای خود را انداختند و اژدها شد، ولی عصای هارون عصاهای ایشان را بلعید.
\par 13 و دل فرعون سخت شد و ایشان رانشنید، چنانکه خداوند گفته بود.
\par 14 و خداوند موسی را گفت: «دل فرعون سخت شده، و از رها کردن قوم ابا کرده است.
\par 15 بامدادان نزد فرعون برو، اینک به سوی آب بیرون می‌آید، و برای ملاقات وی به کنار نهربایست، و عصا را که به مار مبدل گشت، بدست خود بگیر.
\par 16 و او را بگو: یهوه خدای عبرانیان مرا نزد تو فرستاده، گفت: قوم مرا رها کن تا مرا درصحرا عبادت نمایند و اینک تا بحال نشنیده‌ای،
\par 17 پس خداوند چنین می‌گوید، از این خواهی دانست که من یهوه هستم، همانا من به عصایی که در دست دارم آب نهر را می‌زنم و به خون مبدل خواهد شد.
\par 18 و ماهیانی که در نهرند خواهندمرد، و نهر گندیده شود و مصریان نوشیدن آب نهر را مکروه خواهند داشت.»
\par 19 و خداوند به موسی گفت: «به هارون بگوعصای خود را بگیر و دست خود را بر آبهای مصر دراز کن، بر نهرهای ایشان، و جویهای ایشان، و دریاچه های ایشان، و همه حوضهای آب ایشان، تا خون شود، و در تمامی زمین مصردر ظروف چوبی و ظروف سنگی، خون خواهدبود.» 
\par 20 و موسی و هارون چنانکه خداوند امرفرموده بود، کردند. و عصا را بلند کرده، آب نهر رابه حضور فرعون و به حضور ملازمانش زد، وتمامی آب نهر به خون مبدل شد.
\par 21 و ماهیانی که در نهر بودند، مردند. و نهر بگندید، و مصریان از آب نهر نتوانستند نوشید، و در تمامی زمین مصرخون بود.
\par 22 و جادوگران مصر به افسونهای خویش هم چنین کردند، و دل فرعون سخت شد، که بدیشان گوش نگرفت، چنانکه خداوند گفته بود.
\par 23 و فرعون برگشته، به خانه خود رفت و براین نیز دل خود را متوجه نساخت.
\par 24 و همه مصریان گرداگرد نهر برای آب خوردن حفره می‌زدند زیرا که از آب نهر نتوانستند نوشید.وبعد از آنکه خداوند نهر را زده بود، هفت روزسپری شد.
\par 25 وبعد از آنکه خداوند نهر را زده بود، هفت روزسپری شد.
 
\chapter{8}

\par 1 و خداوند موسی را گفت: «نزد فرعون برو، و به وی بگو خداوند چنین می‌گوید: قوم مرا رها کن تا مرا عبادت نمایند،
\par 2 و اگر تو از رهاکردن ایشان ابا می‌کنی، همانا من تمامی حدود تورا به وزغها مبتلا سازم.
\par 3 و نهر، وزغها را به کثرت پیدا نماید، به حدی که برآمده، به خانه ات وخوابگاهت و بسترت و خانه های بندگانت و برقومت و به تنورهایت و تغارهای خمیرت، درخواهند آمد،
\par 4 و بر تو و قوم تو و همه بندگان تو وزغها برخواهند آمد.»
\par 5 و خداوند به موسی گفت: «به هارون بگو: دست خود را با عصای خویش بر نهرها و جویها و دریاچه‌ها دراز کن، ووزغها را بر زمین مصر برآور.»
\par 6 پس چون هارون دست خود را بر آبهای مصر دراز کرد، وزغها برآمده، زمین مصر راپوشانیدند.
\par 7 و جادوگران به افسونهای خودچنین کردند، و وزغها بر زمین مصر برآوردند.
\par 8 آنگاه فرعون موسی و هارون را خوانده، گفت: «نزد خداوند دعا کنید، تا وزغها را از من و قوم من دور کند، و قوم را رها خواهم کرد تا برای خداوندقربانی گذرانند.»
\par 9 موسی به فرعون گفت: «وقتی را برای من معین فرما که برای تو و بندگانت وقومت دعا کنم تا وزغها از تو و خانه ات نابودشوند و فقط در نهر بمانند.»
\par 10 گفت: «فردا»، موسی گفت: «موافق سخن تو خواهد شد تا بدانی که مثل یهوه خدای ما دیگری نیست،
\par 11 و وزغهااز تو و خانه ات و بندگانت و قومت دور خواهندشد و فقط در نهر باقی خواهند ماند.»
\par 12 و موسی و هارون از نزد فرعون بیرون آمدند و موسی درباره وزغهایی که بر فرعون فرستاده بود، نزدخداوند استغاثه نمود.
\par 13 و خداوند موافق سخن موسی عمل نمود و وزغها از خانه‌ها و از دهات واز صحراها مردند،
\par 14 و آنها را توده توده جمع کردند و زمین متعفن شد.
\par 15 اما فرعون چون دیدکه آسایش پدید آمد، دل خود را سخت کرد وبدیشان گوش نگرفت، چنانکه خداوند گفته بود.
\par 16 و خداوند به موسی گفت: «به هارون بگو که عصای خود را دراز کن و غبار زمین را بزن تا درتمامی زمین مصر پشه‌ها بشود.»
\par 17 پس چنین کردند و هارون دست خود را با عصای خویش دراز کرد و غبار زمین را زد و پشه‌ها بر انسان وبهایم پدید آمد زیرا که تمامی غبار زمین در کل ارض مصر پشه‌ها گردید،
\par 18 و جادوگران به افسونهای خود چنین کردند تا پشه‌ها بیرون آورند اما نتوانستند و پشه‌ها بر انسان و بهایم پدیدشد.
\par 19 و جادوگران به فرعون گفتند: «این انگشت خداست.» اما فرعون را دل سخت شد که بدیشان گوش نگرفت، چنانکه خداوند گفته بود.
\par 20 و خداوند به موسی گفت: «بامدادان برخاسته پیش روی فرعون بایست. اینک بسوی آب بیرون می‌آید. و او را بگو: خداوند چنین می‌گوید: قوم مرا رها کن تا مرا عبادت نمایند،
\par 21 زیرا اگر قوم مرا رها نکنی، همانا من بر تو وبندگانت و قومت و خانه هایت انواع مگسهافرستم و خانه های مصریان و زمینی نیز که برآننداز انواع مگسها پر خواهد شد.
\par 22 و در آن روززمین جوشن را که قوم من در آن مقیمند، جداسازم که در آنجا مگسی نباشد تا بدانی که من درمیان این زمین یهوه هستم.
\par 23 و فرقی در میان قوم خود و قوم تو گذارم. فردا این علامت خواهدشد.»
\par 24 و خداوند چنین کرد و انواع مگسهای بسیار به خانه فرعون و به خانه های بندگانش و به تمامی زمین مصر آمدند و زمین از مگسها ویران شد.
\par 25 و فرعون موسی و هارون را خوانده گفت: «بروید و برای خدای خود قربانی در این زمین بگذرانید.»
\par 26 موسی گفت: «چنین کردن نشایدزیرا آنچه مکروه مصریان است برای یهوه خدای خود ذبح می‌کنیم. اینک چون مکروه مصریان را پیش روی ایشان ذبح نماییم، آیاما را سنگسار نمی کنند؟
\par 27 سفر سه روزه به صحرا برویم و برای یهوه خدای خود قربانی بگذرانیم چنانکه به ما امر خواهد فرمود.»
\par 28 فرعون گفت: «من شما را رهایی خواهم داد تا برای یهوه، خدای خود، در صحرا قربانی گذرانید لیکن بسیار دور مروید و برای من دعاکنید.»
\par 29 موسی گفت: «همانا من از حضورت بیرون می‌روم و نزد خداوند دعا می‌کنم و مگسهااز فرعون و بندگانش و قومش فردا دور خواهندشد اما زنهار فرعون بار دیگر حیله نکند که قوم رارهایی ندهد تا برای خداوند قربانی گذرانند.»
\par 30 پس موسی از حضور فرعون بیرون شده نزدخداوند دعا کرد،
\par 31 و خداوند موافق سخن موسی عمل کرد و مگسها را از فرعون و بندگانش و قومش دور کرد که یکی باقی نماند.اما در این مرتبه نیز فرعون دل خود را سخت ساخته، قوم رارهایی نداد.
\par 32 اما در این مرتبه نیز فرعون دل خود را سخت ساخته، قوم رارهایی نداد.
 
\chapter{9}

\par 1 و خداوند به موسی گفت: «نزد فرعون بروو به وی بگو: یهوه خدای عبرانیان چنین می‌گوید: قوم مرا رها کن تا مرا عبادت کنند.
\par 2 زیرا اگر تو از رهایی دادن ابا نمایی و ایشان را بازنگاه داری،
\par 3 همانا دست خداوند بر مواشی تو که در صحرایند خواهد شد، بر اسبان و الاغان وشتران و گاوان و گوسفندان، یعنی وبایی بسیارسخت.
\par 4 و خداوند در میان مواشی اسرائیلیان ومواشی مصریان فرقی خواهد گذاشت که از آنچه مال بنی‌اسرائیل است، چیزی نخواهد مرد.»
\par 5 وخداوند وقتی معین نموده، گفت: «فردا خداونداین کار را در این زمین خواهد کرد.»
\par 6 پس در فرداخداوند این کار را کرد و همه مواشی مصریان مردند و از مواشی بنی‌اسرائیل یکی هم نمرد.
\par 7 وفرعون فرستاد و اینک از مواشی اسرائیلیان یکی هم نمرده بود اما دل فرعون سخت شده، قوم رارهایی نداد.
\par 8 و خداوند به موسی و هارون گفت: «ازخاکستر کوره، مشتهای خود را پر کرده، برداریدو موسی آن را به حضور فرعون بسوی آسمان برافشاند،
\par 9 و غبار خواهد شد بر تمامی زمین مصر و سوزشی که دملها بیرون آورد بر انسان و بربهایم در تمامی زمین مصر خواهد شد.»
\par 10 پس ازخاکستر کوره گرفتند و به حضور فرعون ایستادندو موسی آن را بسوی آسمان پراکند، و سوزشی پدید شده، دملها بیرون آورد، در انسان و دربهایم.
\par 11 و جادوگران به‌سبب آن سوزش به حضور موسی نتوانستند ایستاد، زیرا که سوزش بر جادوگران و بر همه مصریان بود.
\par 12 و خداونددل فرعون را سخت ساخت که بدیشان گوش نگرفت، چنانکه خداوند به موسی گفته بود.
\par 13 و خداوند به موسی گفت: «بامدادان برخاسته، پیش روی فرعون بایست، و به وی بگو: یهوه خدای عبرانیان چنین می‌گوید: قوم مرا رهاکن تا مرا عبادت نمایند.
\par 14 زیرا در این دفعه تمامی بلایای خود را بر دل تو و بندگانت و قومت خواهم فرستاد، تا بدانی که در تمامی جهان مثل من نیست.
\par 15 زیرا اگر تاکنون دست خود را درازکرده، و تو را و قومت را به وبا مبتلا ساخته بودم، هرآینه از زمین هلاک می‌شدی.
\par 16 و لکن برای همین تو را برپا داشته‌ام تا قدرت خود را به تونشان دهم، و نام من در تمامی جهان شایع شود.
\par 17 و آیا تابحال خویشتن را بر قوم من برترمی سازی و ایشان را رهایی نمی دهی؟
\par 18 همانا فردا این وقت، تگرگی بسیار سخت خواهم بارانید، که مثل آن در مصر از روز بنیانش تاکنون نشده است.
\par 19 پس الان بفرست و مواشی خود وآنچه را در صحرا داری جمع کن، زیرا که بر هرانسان و بهایمی که در صحرا یافته شوند، و به خانه‌ها جمع نشوند، تگرگ فرود خواهد آمد وخواهند مرد.»
\par 20 پس هر کس از بندگان فرعون که از قول خداوند ترسید، نوکران و مواشی خود را به خانه‌ها گریزانید.
\par 21 اما هر‌که دل خود را به کلام خداوند متوجه نساخت، نوکران و مواشی خود رادر صحرا واگذاشت.
\par 22 و خداوند به موسی گفت: «دست خود را به سوی آسمان دراز کن، تادر تمامی زمین مصر تگرگ بشود، بر انسان و بربهایم و بر همه نباتات صحرا، در کل ارض مصر.»
\par 23 پس موسی عصای خود را به سوی آسمان دراز کرد، و خداوند رعد و تگرگ داد، و آتش برزمین فرود آمد، و خداوند تگرگ بر زمین مصربارانید.
\par 24 و تگرگ آمد و آتشی که در میان تگرگ آمیخته بود، و به شدت سخت بود، که مثل آن درتمامی زمین مصر از زمانی که امت شده بودند، نبود.
\par 25 و در تمامی زمین مصر، تگرگ آنچه را که در صحرا بود، از انسان و بهایم زد. و تگرگ همه نباتات صحرا را زد، و جمیع درختان صحرا راشکست.
\par 26 فقط در زمین جوشن، جایی که بنی‌اسرائیل بودند، تگرگ نبود.
\par 27 آنگاه فرعون فرستاده، موسی و هارون راخواند، و بدیشان گفت: «در این مرتبه گناه کرده‌ام، خداوند عادل است و من و قوم من گناهکاریم.
\par 28 نزد خداوند دعا کنید، زیرا کافی است تا رعدهای خدا و تگرگ دیگر نشود، و شما را رهاخواهم کرد، و دیگر درنگ نخواهید نمود.»
\par 29 موسی به وی گفت: «چون از شهر بیرون روم، دستهای خود را نزد خداوند خواهم افراشت، تارعدها موقوف شود، و تگرگ دیگر نیاید، تا بدانی جهان از آن خداوند است.
\par 30 و اما تو و بندگانت، می‌دانم که تابحال از یهوه خدا نخواهید ترسید.»
\par 31 و کتان و جو زده شد، زیرا که جو خوشه آورده بود، و کتان تخم داشته.
\par 32 و اما گندم و خلر زده نشد زیرا که متاخر بود.
\par 33 و موسی از حضورفرعون از شهر بیرون شده، دستهای خود را نزدخداوند برافراشت، و رعدها و تگرگ موقوف شد، و باران بر زمین نبارید.
\par 34 و چون فرعون دیدکه باران و تگرگ و رعدها موقوف شد، باز گناه ورزیده، دل خود را سخت ساخت، هم او و هم بندگانش.پس دل فرعون سخت شده، بنی‌اسرائیل را رهایی نداد، چنانکه خداوند به‌دست موسی گفته بود.
\par 35 پس دل فرعون سخت شده، بنی‌اسرائیل را رهایی نداد، چنانکه خداوند به‌دست موسی گفته بود.
 
\chapter{10}

\par 1 و خداوند به موسی گفت: «نزد فرعون برو زیرا که من دل فرعون و دل بندگانش را سخت کرده‌ام، تا این آیات خود را درمیان ایشان ظاهر سازم.
\par 2 و تا آنچه در مصر کردم و آیات خود را که در میان ایشان ظاهر ساختم، بگوش پسرت و پسر پسرت بازگویی تا بدانید که من یهوه هستم.»
\par 3 پس موسی و هارون نزد فرعون آمده، به وی گفتند: «یهوه خدای عبرانیان چنین می‌گوید: تا به کی از تواضع کردن به حضور من اباخواهی نمود؟ قوم مرا رها کن تا مرا عبادت کنند.
\par 4 زیرا اگر تو از رها کردن قوم من ابا کنی، هرآینه من فردا ملخها در حدود تو فرود آورم.
\par 5 که روی زمین را مستور خواهند ساخت، به حدی که زمین را نتوان دید، و تتمه آنچه رسته است که برای شمااز تگرگ باقی‌مانده، خواهند خورد، و هر درختی را که برای شما در صحرا روییده است، خواهندخورد.
\par 6 و خانه تو و خانه های بندگانت وخانه های همه مصریان را پر خواهند ساخت، به مرتبه‌ای که پدرانت و پدران پدرانت از روزی که بر زمین بوده‌اند تا الیوم ندیده‌اند.» پس روگردانیده، از حضور فرعون بیرون رفت.
\par 7 آنگاه بندگان فرعون به وی گفتند: «تا به کی برای ما این مرد دامی باشد؟ این مردمان را رها کن تا یهوه، خدای خود را عبادت نمایند. مگرتابحال ندانسته‌ای که مصر ویران شده است؟»
\par 8 پس موسی و هارون را نزد فرعون برگردانیدند، واو به ایشان گفت: «بروید و یهوه، خدای خود راعبادت کنید، لیکن کیستند که می‌روند؟»
\par 9 موسی گفت: «با جوانان و پیران خود خواهیم رفت، باپسران و دختران، و گوسفندان و گاوان خودخواهیم رفت، زیرا که ما را عیدی برای خداونداست.» 
\par 10 بدیشان گفت: «خداوند با شما چنین باشد، اگر شما را با اطفال شما رهایی دهم با حذرباشید زیرا که بدی پیش روی شماست!
\par 11 نه چنین! بلکه شما که بالغ هستید رفته، خداوند راعبادت کنید، زیرا که این است آنچه خواسته بودید.» پس ایشان را از حضور فرعون بیرون راندند.
\par 12 و خداوند به موسی گفت: «دست خود رابرای ملخها بر زمین مصر دراز کن، تا بر زمین مصربرآیند، و همه نباتات زمین را که از تگرگ مانده است، بخورند.»
\par 13 پس موسی عصای خود را برزمین مصر دراز کرد، و خداوند تمامی آن روز، وتمامی آن شب را بادی شرقی بر زمین مصر وزانید، و چون صبح شد، باد شرقی ملخها راآورد.
\par 14 و ملخها بر تمامی زمین مصر برآمدند، ودر همه حدود مصر نشستند، بسیار سخت که قبل از آن چنین ملخها نبود، و بعد از آن نخواهد بود.
\par 15 و روی تمامی زمین را پوشانیدند، که زمین تاریک شد و همه نباتات زمین و همه میوه درختان را که از تگرگ باقی‌مانده بود، خوردند، به حدی که هیچ سبزی بر درخت، و نبات صحرا درتمامی زمین مصر نماند.
\par 16 آنگاه فرعون، موسی و هارون را به زودی خوانده، گفت: «به یهوه خدای شما و به شما گناه کرده‌ام.
\par 17 و اکنون این مرتبه فقط گناه مرا عفو فرمایید، و از یهوه خدای خود استدعا نمایید تا این موت را فقط از من برطرف نماید.»
\par 18 پس از حضور فرعون بیرون شده، از خداوند استدعا نمود.
\par 19 و خداوند بادغربی‌ای بسیار سخت برگردانید، که ملخها رابرداشته، آنها را به دریای قلزم ریخت، و درتمامی حدود مصر ملخی نماند.
\par 20 اما خداونددل فرعون را سخت گردانید، که بنی‌اسرائیل رارهایی نداد.
\par 21 و خداوند به موسی گفت: «دست خود را به سوی آسمان برافراز، تا تاریکی‌ای بر زمین مصرپدید آید، تاریکی‌ای که بتوان احساس کرد.»
\par 22 پس موسی دست خود را به سوی آسمان برافراشت، و تاریکی غلیظ تا سه روز در تمامی زمین مصر پدید آمد.
\par 23 و یکدیگر را نمی دیدند. و تا سه روز کسی از جای خود برنخاست، لیکن برای جمیع بنی‌اسرائیل در مسکنهای ایشان روشنایی بود.
\par 24 و فرعون موسی را خوانده، گفت: «بروید خداوند را عبادت کنید، فقط گله هاو رمه های شما بماند، اطفال شما نیز با شمابروند.»
\par 25 موسی گفت: «ذبایح و قربانی های سوختنی نیز می‌باید به‌دست ما بدهی، تا نزدیهوه، خدای خود بگذرانیم.
\par 26 مواشی ما نیز با ماخواهد آمد، یک سمی باقی نخواهد ماند زیرا که از اینها برای عبادت یهوه، خدای خود می‌بایدگرفت، و تا بدانجا نرسیم، نخواهیم دانست به چه چیز خداوند را عبادت کنیم.»
\par 27 و خداوند، دل فرعون را سخت گردانید که از رهایی دادن ایشان ابا نمود.
\par 28 پس فرعون وی را گفت: «از حضورمن برو! و با حذر باش که روی مرا دیگر نبینی، زیرا در روزی که مرا ببینی خواهی مرد.»موسی گفت: «نیکو گفتی، روی تو را دیگرنخواهم دید.»
\par 29 موسی گفت: «نیکو گفتی، روی تو را دیگرنخواهم دید.»
 
\chapter{11}

\par 1 و خداوند به موسی گفت: «یک بلای دیگر بر فرعون و بر مصر می‌آورم، وبعد از آن شما را از اینجا رهایی خواهد داد، وچون شما را رها کند، البته شما را بالکلیه از اینجاخواهد راند.
\par 2 اکنون به گوش قوم بگو که هر مرداز همسایه خود، و هر زن از همسایه‌اش آلات نقره و آلات طلا بخواهند.»
\par 3 و خداوند قوم را درنظر مصریان محترم ساخت. و شخص موسی نیزدر زمین مصر، در نظر بندگان فرعون و در نظرقوم، بسیار بزرگ بود.
\par 4 و موسی گفت: «خداوندچنین می‌گوید: قریب به نصف شب در میان مصربیرون خواهم آمد.
\par 5 و هر نخست زاده‌ای که در زمین مصر باشد، از نخست زاده فرعون که برتختش نشسته است، تا نخست زاده کنیزی که درپشت دستاس باشد، و همه نخست زادگان بهایم خواهند مرد.
\par 6 و نعره عظیمی در تمامی زمین مصر خواهد بود که مثل آن نشده، و مانند آن دیگر نخواهد شد.
\par 7 اما بر جمیع بنی‌اسرائیل سگی زبان خود را تیز نکند، نه بر انسان و نه بربهایم، تا بدانید که خداوند در میان مصریان واسرائیلیان فرقی گذارده است.
\par 8 و این همه بندگان تو به نزد من فرود آمده، و مرا تعظیم کرده، خواهند گفت: تو و تمامی قوم که تابع تو باشند، بیرون روید! و بعد از آن بیرون خواهم رفت.»
\par 9 و خداوند به موسی گفت: «فرعون به شماگوش نخواهد گرفت، تا آیات من در زمین مصرزیاد شود.»و موسی و هارون جمیع این آیات را به حضور فرعون ظاهر ساختند. اما خداوند دل فرعون را سخت گردانید، و بنی‌اسرائیل را از زمین خود رهایی نداد.
\par 10 و موسی و هارون جمیع این آیات را به حضور فرعون ظاهر ساختند. اما خداوند دل فرعون را سخت گردانید، و بنی‌اسرائیل را از زمین خود رهایی نداد.
 
\chapter{12}

\par 1 و خداوند موسی و هارون را در زمین مصر مخاطب ساخته، گفت:
\par 2 «این ماه برای شما سر ماهها باشد، این اول از ماههای سال برای شماست.
\par 3 تمامی جماعت اسرائیل راخطاب کرده، گویید که در دهم این ماه هر یکی ازایشان بره‌ای به حسب خانه های پدران خودبگیرند، یعنی برای هر خانه یک بره.
\par 4 و اگر اهل خانه برای بره کم باشند، آنگاه او و همسایه‌اش که مجاور خانه او باشد آن را به حسب شماره نفوس بگیرند، یعنی هر کس موافق خوراکش بره راحساب کند.
\par 5 بره شما بی‌عیب، نرینه یکساله باشد، از گوسفندان یا از بزها آن را بگیرید.
\par 6 و آن را تا چهاردهم این ماه نگاه دارید، و تمامی انجمن جماعت بنی‌اسرائیل آن را در عصر ذبح کنند.
\par 7 واز خون آن بگیرند، و آن را بر هر دو قایمه، وسردر خانه که در آن، آن را می‌خورند، بپاشند.
\par 8 وگوشتش را در آن شب بخورند. به آتش بریان کرده، با نان فطیر و سبزیهای تلخ آن را بخورند.
\par 9 و از آن هیچ خام نخورید، و نه پخته با آب، بلکه به آتش بریان شده، کله‌اش و پاچه هایش واندرونش را.
\par 10 و چیزی از آن تا صبح نگاه مدارید، و آنچه تا صبح مانده باشد به آتش بسوزانید.
\par 11 و آن را بدین طور بخورید: کمر شمابسته، و نعلین بر پایهای شما، و عصا در دست شما، و آن را به تعجیل بخورید، چونکه فصح خداوند است.
\par 12 «و در آن شب از زمین مصر عبور خواهم کرد، و همه نخست زادگان زمین مصر را از انسان وبهایم خواهم زد، و بر تمامی خدایان مصر داوری خواهم کرد. من یهوه هستم.
\par 13 و آن خون، علامتی برای شما خواهد بود، بر خانه هایی که درآنها می‌باشید، و چون خون را ببینم، از شماخواهم گذشت و هنگامی که زمین مصر را می‌زنم، آن بلا برای هلاک شما بر شما نخواهد آمد.
\par 14 وآن روز، شما را برای یادگاری خواهد بود، و درآن، عیدی برای خداوند نگاه دارید، و آن را به قانون ابدی، نسلا بعد نسل عید نگاه دارید.
\par 15 هفت روز نان فطیر خورید، در روز اول خمیرمایه را از خانه های خود بیرون کنید، زیرا هر‌که از روز نخستین تا روز هفتمین چیزی خمیرشده بخورد، آن شخص از اسرائیل منقطع گردد.
\par 16 و در روز اول، محفل مقدس، و در روز هفتم، محفل مقدس برای شما خواهد بود. در آنها هیچ کار کرده نشود جز آنچه هر کس باید بخورد؛ آن فقط در میان شما کرده شود.
\par 17 پس عید فطیر رانگاه دارید، زیرا که در همان روز لشکرهای شمارا از زمین مصر بیرون آوردم. بنابراین، این‌روز رادر نسلهای خود به فریضه ابدی نگاه دارید.
\par 18 درماه اول در روز چهاردهم ماه، در شام، نان فطیربخورید، تا شام بیست و یکم ماه.
\par 19 هفت روزخمیرمایه در خانه های شما یافت نشود، زیرا هرکه چیزی خمیر شده بخورد، آن شخص ازجماعت اسرائیل منقطع گردد، خواه غریب باشدخواه بومی آن زمین.
\par 20 هیچ‌چیز خمیر شده مخورید، در همه مساکن خود فطیر بخورید.»
\par 21 پس موسی جمیع مشایخ اسرائیل راخوانده، بدیشان گفت: «بروید و بره‌ای برای خودموافق خاندانهای خویش بگیرید، و فصح را ذبح نمایید.
\par 22 و دسته‌ای از زوفا گرفته، در خونی که در طشت است فروبرید، و بر سر در و دو قایمه آن، از خونی که در طشت است بزنید، و کسی ازشما از در خانه خود تا صبح بیرون نرود.
\par 23 زیراخداوند عبور خواهد کرد تا مصریان را بزند وچون خون را بر سردر و دو قایمه‌اش بیند، هماناخداوند از در گذرد و نگذارد که هلاک کننده به خانه های شما درآید تا شما را بزند.
\par 24 و این امررا برای خود و پسران خود به فریضه ابدی نگاه دارید.
\par 25 و هنگامی که داخل زمینی شدید که خداوند حسب قول خود، آن را به شما خواهدداد. آنگاه این عبادت را مرعی دارید.
\par 26 و چون پسران شما به شما گویند که این عبادت شماچیست،
\par 27 گویید این قربانی فصح خداوند است، که از خانه های بنی‌اسرائیل در مصر عبور کرد، وقتی که مصریان را زد و خانه های ما را خلاصی داد.» پس قوم به روی درافتاده، سجده کردند.
\par 28 پس بنی‌اسرائیل رفته، آن را کردند، چنانکه خداوند به موسی و هارون امر فرموده بودهمچنان کردند.
\par 29 و واقع شد که در نصف شب، خداوند همه نخست زادگان زمین مصر را، ازنخست زاده فرعون که بر تخت نشسته بود تانخست زاده اسیری که در زندان بود، و همه نخست زاده های بهایم را زد.
\par 30 و در آن شب فرعون و همه بندگانش وجمیع مصریان برخاستند و نعره عظیمی در مصربرپا شد، زیرا خانه‌ای نبود که در آن میتی نباشد.
\par 31 و موسی و هارون را در شب طلبیده، گفت: «برخیزید! و از میان قوم من بیرون شوید، هم شماو جمیع بنی‌اسرائیل! و رفته، خداوند را عبادت نمایید، چنانکه گفتید.
\par 32 گله‌ها و رمه های خود رانیز چنانکه گفتید، برداشته، بروید و مرا نیز برکت دهید.»
\par 33 و مصریان نیز بر قوم الحاح نمودند تاایشان را بزودی از زمین روانه کنند، زیرا گفتند ماهمه مرده‌ایم.
\par 34 و قوم، آرد سرشته خود را پیش از آنکه خمیر شود برداشتند، و تغارهای خویش را در رختها بر دوش خود بستند.
\par 35 و بنی‌اسرائیل به قول موسی عمل کرده، از مصریان آلات نقره و آلات طلا و رختها خواستند.
\par 36 وخداوند قوم را در نظر مصریان مکرم ساخت، که هرآنچه خواستند بدیشان دادند. پس مصریان راغارت کردند.
\par 37 و بنی‌اسرائیل از رعمسیس به سکوت کوچ کردند، قریب ششصدهزار مردپیاده، سوای اطفال.
\par 38 و گروهی مختلفه بسیار نیزهمراه ایشان بیرون رفتند، و گله‌ها و رمه‌ها ومواشی بسیار سنگین.
\par 39 و از آرد سرشته، که ازمصر بیرون آورده بودند، قرصهای فطیر پختند، زیرا خمیر نشده بود، چونکه از مصر رانده شده بودند، و نتوانستند درنگ کنند، و زاد سفر نیزبرای خود مهیا نکرده بودند.
\par 40 و توقف بنی‌اسرائیل که در مصر کرده بودند، چهارصد وسی سال بود.
\par 41 و بعد از انقضای چهار صد وسی سال در همان روز به وقوع پیوست که جمیع لشکرهای خدا از زمین مصر بیرون رفتند.
\par 42 این است شبی که برای خداوند باید نگاه داشت، چون ایشان را از زمین مصر بیرون آورد. این همان شب خداوند است که بر جمیع بنی‌اسرائیل نسلابعد نسل واجب است که آن را نگاه دارند.
\par 43 و خداوند به موسی و هارون گفت: «این است فریضه فصح که هیچ بیگانه از آن نخورد.
\par 44 و اما هر غلام زرخرید، او را ختنه کن و پس آن را بخورد.
\par 45 نزیل و مزدور آن را نخورند.
\par 46 دریک خانه خورده شود، و چیزی از گوشتش ازخانه بیرون مبر، و استخوانی از آن مشکنید.
\par 47 تمامی جماعت بنی‌اسرائیل آن را نگاه بدارند.
\par 48 و اگر غریبی نزد تو نزیل شود، و بخواهد فصح را برای خداوند مرعی بدارد، تمامی ذکورانش مختون شوند، و بعد از آن نزدیک آمده، آن را نگاه دارد، و مانند بومی زمین خواهد بود و اما هرنامختون از آن نخورد.
\par 49 یک قانون خواهد بودبرای اهل وطن و بجهت غریبی که در میان شمانزیل شود.»
\par 50 پس تمامی بنی‌اسرائیل این راکردند، چنانکه خداوند به موسی و هارون امرفرموده بود، عمل نمودند.و واقع شد که خداوند در همان روز بنی‌اسرائیل را با لشکرهای ایشان از زمین مصر بیرون آورد.
\par 51 و واقع شد که خداوند در همان روز بنی‌اسرائیل را با لشکرهای ایشان از زمین مصر بیرون آورد. 
 
\chapter{13}

\par 1 و خداوند موسی را خطاب کرده، گفت:
\par 2 «هر نخست زاده‌ای را که رحم رابگشاید، در میان بنی‌اسرائیل، خواه از انسان خواه از بهایم، تقدیس نما؛ او از آن من است.»
\par 3 وموسی به قوم گفت: «این‌روز را که از مصر از خانه غلامی بیرون آمدید، یاد دارید، زیرا خداوندشما را به قوت دست، از آنجا بیرون آورد، پس نان خمیر، خورده نشود.
\par 4 این‌روز، در ماه ابیب بیرون آمدید.
\par 5 و هنگامی که خداوند تو را به زمین کنعانیان و حتیان و اموریان و حویان ویبوسیان داخل کند، که با پدران تو قسم خورد که آن را به تو بدهد، زمینی که به شیر و شهد جاری است، آنگاه این عبادت را در این ماه بجا بیاور.
\par 6 هفت روز نان فطیر بخور، و در روز هفتمین عیدخداوند است.
\par 7 هفت روز نان فطیر خورده شود، و هیچ‌چیز خمیر شده نزد تو دیده نشود، و خمیرمایه نزد تو در تمامی حدودت پیدا نشود.
\par 8 و در آن روز پسر خود را خبر داده، بگو: این است به‌سبب آنچه خداوند به من کرد، وقتی که از مصربیرون آمدم.
\par 9 و این برای تو علامتی بر دستت خواهد بود و تذکره‌ای در میان دو چشمت، تاشریعت خداوند در دهانت باشد. زیرا خداوند تورا به‌دست قوی از مصر بیرون آورد.
\par 10 و این فریضه را در موسمش سال به سال نگاه دار.
\par 11 «و هنگامی که خداوند تو را به زمین کنعانیان درآورد، چنانکه برای تو و پدرانت قسم خورد، و آن را به تو بخشد.
\par 12 آنگاه هر‌چه رحم را گشاید، آن را برای خدا جدا بساز، و هرنخست زاده‌ای از بچه های بهایم که از آن توست، نرینه‌ها از آن خداوند باشد.
\par 13 و هر نخست زاده الاغ را به بره‌ای فدیه بده، و اگر فدیه ندهی گردنش را بشکن، و هر نخست زاده انسان را ازپسرانت فدیه بده.
\par 14 و در زمان آینده چون پسرت از تو سوال کرده، گوید که این چیست، اورا بگو، یهوه ما را به قوت دست از مصر، از خانه غلامی بیرون آورد.
\par 15 و چون فرعون از رها کردن ما دل خود را سخت ساخت، واقع شد که خداوندجمیع نخست زادگان مصر را از نخست زاده انسان تا نخست زاده بهایم کشت. بنابراین من همه نرینه‌ها را که رحم را گشایند، برای خداوند ذبح می‌کنم، لیکن هر نخست زاده‌ای از پسران خود رافدیه می‌دهم.
\par 16 و این علامتی بر دستت وعصابه‌ای در میان چشمان تو خواهد بود، زیراخداوند ما را بقوت دست از مصر بیرون آورد.»
\par 17 و واقع شد که چون فرعون قوم را رها کرده بود، خدا ایشان را از راه زمین فلسطینیان رهبری نکرد، هرچند آن نزدیکتر بود. زیرا خدا گفت: «مبادا که چون قوم جنگ بینند، پشیمان شوند و به مصر برگردند.»
\par 18 اما خدا قوم را از راه صحرای دریای قلزم دور گردانید. پس بنی‌اسرائیل مسلح شده، از زمین مصر برآمدند.
\par 19 و موسی استخوانهای یوسف را با خود برداشت، زیرا که او بنی‌اسرائیل را قسم سخت داده، گفته بود: «هرآینه خدا از شما تفقد خواهد نمود واستخوانهای مرا از اینجا با خود خواهید برد.»
\par 20 و از سکوت کوچ کرده، در ایتام به کنار صحرااردو زدند.
\par 21 و خداوند در روز، پیش روی قوم در ستون ابر می‌رفت تا راه را به ایشان دلالت کند، و شبانگاه در ستون آتش، تا ایشان را روشنایی بخشد، و روز و شب راه روند.و ستون ابر را درروز و ستون آتش را در شب، از پیش روی قوم برنداشت.
\par 22 و ستون ابر را درروز و ستون آتش را در شب، از پیش روی قوم برنداشت.
 
\chapter{14}

\par 1 و خداوند موسی را خطاب کرده، گفت:
\par 2 «به بنی‌اسرائیل بگو که برگردیده، برابر فم الحیروت در میان مجدل ودریا اردو زنند. و در مقابل بعل صفون، در برابر آن به کنار دریا اردو زنید.
\par 3 و فرعون درباره بنی‌اسرائیل خواهد گفت: در زمین گرفتارشده‌اند، و صحرا آنها را محصور کرده است.
\par 4 ودل فرعون را سخت گردانم تا ایشان را تعاقب کند، و در فرعون و تمامی لشکرش جلال خود راجلوه دهم، تا مصریان بدانند که من یهوه هستم.» پس چنین کردند.
\par 5 و به پادشاه مصر گفته شد که قوم فرار کردند، و دل فرعون و بندگانش بر قوم متغیر شد، پس گفتند: «این چیست که کردیم که بنی‌اسرائیل را از بندگی خود رهایی دادیم؟»
\par 6 پس ارابه خود را بیاراست، و قوم خود را با خودبرداشت،
\par 7 و ششصد ارابه برگزیده برداشت، وهمه ارابه های مصر را و سرداران را بر جمیع آنها.
\par 8 و خداوند دل فرعون، پادشاه مصر را سخت ساخت تا بنی‌اسرائیل را تعاقب کرد، وبنی‌اسرائیل به‌دست بلند بیرون رفتند.
\par 9 و مصریان با تمامی اسبان و ارابه های فرعون و سوارانش و لشکرش در عقب ایشان تاخته، بدیشان دررسیدند، وقتی که به کنار دریا نزدفم الحیروت، برابر بعل صفون فرود آمده بودند.
\par 10 و چون فرعون نزدیک شد، بنی‌اسرائیل چشمان خود را بالا کرده، دیدند که اینک مصریان از عقب ایشان می‌آیند. پس بنی‌اسرائیل سخت بترسیدند، و نزد خداوند فریاد برآوردند.
\par 11 و به موسی گفتند: «آیا در مصر قبرها نبود که ما رابرداشته‌ای تا در صحرا بمیریم؟ این چیست به ماکردی که ما را از مصر بیرون آوردی؟
\par 12 آیا این آن سخن نیست که به تو در مصر گفتیم که ما رابگذار تا مصریان را خدمت کنیم؟ زیرا که ما راخدمت مصریان بهتر است از مردن در صحرا!»
\par 13 موسی به قوم گفت: «مترسید. بایستید و نجات خداوند را ببینید، که امروز آن را برای شماخواهد کرد، زیرا مصریان را که امروز دیدید تا به ابد دیگر نخواهید دید.
\par 14 خداوند برای شماجنگ خواهد کرد و شما خاموش باشید.»
\par 15 و خداوند به موسی گفت: «چرا نزد من فریاد می‌کنی؟ بنی‌اسرائیل را بگو که کوچ کنند.
\par 16 و اما تو عصای خود را برافراز و دست خود رابر دریا دراز کرده، آن را منشق کن، تا بنی‌اسرائیل از میان دریا بر خشکی راه سپر شوند.
\par 17 و اما من اینک، دل مصریان را سخت می‌سازم، تا از عقب ایشان بیایند، و از فرعون و تمامی لشکر او وارابه‌ها و سوارانش جلال خواهم یافت.
\par 18 ومصریان خواهند دانست که من یهوه هستم، وقتی که از فرعون و ارابه هایش و سوارانش جلال یافته باشم.»
\par 19 و فرشته خدا که پیش اردوی اسرائیل می‌رفت، حرکت کرده، از عقب ایشان خرامید، وستون ابر از پیش ایشان نقل کرده، در عقب ایشان بایستاد.
\par 20 و میان اردوی مصریان و اردوی اسرائیل آمده، از برای آنها ابر و تاریکی می‌بود، واینها را در شب روشنایی می‌داد که تمامی شب نزدیک یکدیگر نیامدند.
\par 21 پس موسی دست خود را بر دریا دراز کرد و خداوند دریا را به بادشرقی شدید، تمامی آن شب برگردانیده، دریا راخشک ساخت و آب منشق گردید.
\par 22 وبنی‌اسرائیل در میان دریا بر خشکی می‌رفتند وآبها برای ایشان بر راست و چپ، دیوار بود.
\par 23 ومصریان با تمامی اسبان و ارابه‌ها و سواران فرعون از عقب ایشان تاخته، به میان دریا درآمدند.
\par 24 ودر پاس سحری واقع شد که خداوند بر اردوی مصریان از ستون آتش و ابر نظر انداخت، واردوی مصریان را آشفته کرد.
\par 25 و چرخهای ارابه های ایشان را بیرون کرد، تا آنها را به سنگینی برانند و مصریان گفتند: «از حضور بنی‌اسرائیل بگریزیم! زیرا خداوند برای ایشان با مصریان جنگ می‌کند.»
\par 26 و خداوند به موسی گفت: «دست خود را بردریا دراز کن، تا آبها بر مصریان برگردد، و برارابه‌ها و سواران ایشان.»
\par 27 پس موسی دست خود را بر دریا دراز کرد، و به وقت طلوع صبح، دریا به جریان خود برگشت، و مصریان به مقابلش گریختند، و خداوند مصریان را در میان دریا به زیر انداخت.
\par 28 و آبها برگشته، عرابه‌ها و سواران و تمام لشکر فرعون را که از عقب ایشان به دریادرآمده بودند، پوشانید، که یکی از ایشان هم باقی نماند.
\par 29 اما بنی‌اسرائیل در میان دریا به خشکی رفتند، و آبها برای ایشان دیواری بود به طرف راست و به طرف چپ.
\par 30 و در آن روز خداونداسرائیل را از دست مصریان خلاصی داد واسرائیل مصریان را به کنار دریا مرده دیدند.واسرائیل آن کار عظیمی را که خداوند به مصریان کرده بود دیدند، و قوم از خداوند ترسیدند، و به خداوند و به بنده او موسی ایمان آوردند.
\par 31 واسرائیل آن کار عظیمی را که خداوند به مصریان کرده بود دیدند، و قوم از خداوند ترسیدند، و به خداوند و به بنده او موسی ایمان آوردند.
 
\chapter{15}

\par 1 آنگاه موسی و بنی‌اسرائیل این سرودرا برای خداوند سراییده، گفتند که «یهوه را سرود می‌خوانم زیرا که با جلال مظفرشده است.اسب و سوارش را به دریا انداخت.
\par 2 خداوند قوت و تسبیح من است.و او نجات من گردیده است.این خدای من است، پس او را تمجید می‌کنم.خدای پدر من است، پس او را متعال می‌خوانم.
\par 3 خداوند مرد جنگی است.نام او یهوه است.
\par 4 ارابه‌ها و لشکر فرعون را به دریا انداخت.مبارزان برگزیده او در دریای قلزم غرق شدند.
\par 5 لجه‌ها ایشان را پوشانید.مثل سنگ به ژرفیها فرو رفتند.
\par 6 دست راست تو‌ای خداوند، به قوت جلیل گردیده.دست راست تو‌ای خداوند، دشمن را خردشکسته است.
\par 7 و به کثرت جلال خود خصمان را منهدم ساخته‌ای.غضب خود را فرستاده، ایشان را چون خاشاک سوزانیده‌ای.
\par 8 و به نفخه بینی تو آبها فراهم گردید.و موجها مثل توده بایستاد و لجه‌ها در میان دریامنجمد گردید.
\par 9 دشمن گفت تعاقب می‌کنم و ایشان را فرومی گیرم.و غارت را تقسیم کرده، جانم از ایشان سیرخواهد شد.شمشیر خود را کشیده، دست من ایشان را هلاک خواهد ساخت.
\par 10 و چون به نفخه خود دمیدی، دریا ایشان راپوشانید.
\par 11 کیست مانند تو‌ای خداوند در میان خدایان؟کیست مانند تو جلیل در قدوسیت؟
\par 12 چون دست راست خود را دراز کردی، زمین ایشان را فرو برد.
\par 13 این قوم خویش را که فدیه دادی، به رحمانیت خود، رهبری نمودی.ایشان را به قوت خویش به سوی مسکن قدس خود هدایت کردی.
\par 14 امتها چون شنیدند، مضطرب گردیدند.لرزه بر سکنه فلسطین مستولی گردید.
\par 15 آنگاه امرای ادوم در حیرت افتادند.و اکابر موآب را لرزه فرو گرفت، و جمیع سکنه کنعان گداخته گردیدند.
\par 16 ترس و هراس، ایشان را فروگرفت.از بزرگی بازوی تو مثل سنگ ساکت شدند.تا قوم تو‌ای خداوند عبور کنند.تا این قومی که تو خریده‌ای، عبور کنند.
\par 17 ایشان را داخل ساخته، در جبل میراث خودغرس خواهی کرد، به مکانی که تو‌ای خداوندمسکن خود ساخته‌ای، یعنی آن مقام مقدسی که دستهای تو‌ای خداوند مستحکم کرده است.
\par 18 خداوند سلطنت خواهد کرد تا ابدالاباد.»
\par 19 زیرا که اسبهای فرعون با ارابه‌ها وسوارانش به دریا درآمدند، و خداوند آب دریا رابر ایشان برگردانید. اما بنی‌اسرائیل از میان دریا به خشکی رفتند.
\par 20 و مریم نبیه، خواهر هارون، دف را به‌دست خود گرفته، و همه زنان از عقب وی دفها گرفته، رقص‌کنان بیرون آمدند.
\par 21 پس مریم در جواب ایشان گفت: «خداوند را بسرایید، زیرا که با جلال مظفر شده است، اسب و سوارش را به دریاانداخت.»
\par 22 پس موسی اسرائیل را از بحر قلزم کوچانید، و به صحرای شور آمدند، و سه روز درصحرا می‌رفتند و آب نیافتند.
\par 23 پس به ماره رسیدند، و از آب ماره نتوانستند نوشید زیرا که تلخ بود. از این سبب، آن را ماره نامیدند.
\par 24 و قوم بر موسی شکایت کرده، گفتند: «چه بنوشیم؟»
\par 25 چون نزد خداوند استغاثه کرد، خداونددرختی بدو نشان داد، پس آن را به آب انداخت وآب شیرین گردید. و در آنجا فریضه‌ای وشریعتی برای ایشان قرار داد، و در آنجا ایشان راامتحان کرد.
\par 26 و گفت: «هرآینه اگر قول یهوه، خدای خود را بشنوی، و آنچه را در نظر او راست است بجا آوری، و احکام او را بشنوی، و تمامی فرایض او را نگاه داری، همانا هیچ‌یک از همه مرضهایی را که بر مصریان آورده‌ام بر تو نیاورم، زیرا که من یهوه، شفا دهنده تو هستم.»پس به ایلیم آمدند، و در آنجا دوازده چشمه آب و هفتاددرخت خرما بود، و در آنجا نزد آب خیمه زدند.
\par 27 پس به ایلیم آمدند، و در آنجا دوازده چشمه آب و هفتاددرخت خرما بود، و در آنجا نزد آب خیمه زدند. 
 
\chapter{16}

\par 1 پس تمامی جماعت بنی‌اسرائیل ازایلیم کوچ کرده، به صحرای سین که درمیان ایلیم و سینا است در روز پانزدهم از ماه دوم، بعد از بیرون آمدن ایشان از زمین مصر، رسیدند.
\par 2 و تمامی جماعت بنی‌اسرائیل در آن صحرا برموسی و هارون شکایت کردند.
\par 3 و بنی‌اسرائیل بدیشان گفتند: «کاش که در زمین مصر به‌دست خداوند مرده بودیم، وقتی که نزد دیگهای گوشت می‌نشستیم و نان را سیر می‌خوردیم، زیراکه ما را بدین صحرا بیرون آوردید، تا تمامی این جماعت را به گرسنگی بکشید.»
\par 4 آنگاه خداوند به موسی گفت: «همانا من نان از آسمان برای شما بارانم، و قوم رفته، کفایت هرروز را در روزش گیرند، تا ایشان را امتحان کنم که بر شریعت من رفتار می‌کنند یا نه.
\par 5 و واقع خواهد شد در روز ششم، که چون آنچه را که آورده باشند درست نمایند، همانا دوچندان آن خواهدبود که هر روز برمی چیدند.»
\par 6 و موسی و هارون به همه بنی‌اسرائیل گفتند: «شامگاهان خواهیددانست که خداوند شما را از زمین مصر بیرون آورده است.
\par 7 و بامدادان جلال خداوند راخواهید دید، زیرا که او شکایتی را که بر خداوندکرده‌اید شنیده است، و ما چیستیم که بر ماشکایت می‌کنید؟»
\par 8 و موسی گفت: «این خواهدبود چون خداوند، شامگاه شما را گوشت دهد تابخورید، و بامداد نان، تا سیر شوید، زیرا خداوندشکایتهای شما را که بر وی کرده‌اید شنیده است، و ما چیستیم؟ بر ما نی، بلکه بر خداوند شکایت نموده‌اید.»
\par 9 و موسی به هارون گفت: «به تمامی جماعت بنی‌اسرائیل بگو به حضور خداوندنزدیک بیایید، زیرا که شکایتهای شما را شنیده است.»
\par 10 و واقع شد که چون هارون به تمامی جماعت بنی‌اسرائیل سخن گفت، به سوی صحرانگریستند و اینک جلال خداوند در ابر ظاهر شد.
\par 11 و خداوند موسی را خطاب کرده، گفت:
\par 12 «شکایتهای بنی‌اسرائیل را شنیده‌ام، پس ایشان را خطاب کرده، بگو: در عصر گوشت خواهید خورد، و بامداد از نان سیر خواهید شد تابدانید که من یهوه خدای شما هستم.»
\par 13 و واقع شد که در عصر، سلوی برآمده، لشکرگاه راپوشانیدند، و بامدادان شبنم گرداگرد اردونشست.
\par 14 و چون شبنمی که نشسته بودبرخاست، اینک بر روی صحرا چیزی دقیق، مدور و خرد، مثل ژاله بر زمین بود.
\par 15 و چون بنی‌اسرائیل این را دیدند به یکدیگر گفتند که این من است، زیرا که ندانستند چه بود. موسی به ایشان گفت: «این آن نان است که خداوند به شمامی دهد تا بخورید.
\par 16 این است امری که خداوندفرموده است، که هر کس به قدر خوراک خود ازاین بگیرد، یعنی یک عومر برای هر نفر به حسب شماره نفوس خویش، هر شخص برای کسانی که در خیمه او باشند بگیرد.»
\par 17 پس بنی‌اسرائیل چنین کردند، بعضی زیاد و بعضی کم برچیدند.
\par 18 اما چون به عومر پیمودند، آنکه زیاد برچیده بود، زیاده نداشت، و آنکه کم برچیده بود، کم نداشت، بلکه هر کس به قدر خوراکش برچیده بود.
\par 19 و موسی بدیشان گفت: «زنهار کسی چیزی از این تا صبح نگاه ندارد.»
\par 20 لکن به موسی گوش ندادند، بلکه بعضی چیزی از آن تا صبح نگاه داشتند. و کرمها بهم رسانیده، متعفن گردید، و موسی بدیشان خشمناک شد.
\par 21 و هر صبح، هر کس به قدر خوراک خودبرمی چید، و چون آفتاب گرم می‌شد، می‌گداخت.
\par 22 و واقع شد در روز ششم که نان مضاعف، یعنی برای هر نفری دو عومر برچیدند. پس همه روسای جماعت آمده، موسی را خبردادند.
\par 23 او بدیشان گفت: «این است آنچه خداوند گفت، که فردا آرامی است، و سبت مقدس خداوند. پس آنچه بر آتش باید پخت بپزید، و آنچه در آب باید جوشانید بجوشانید، وآنچه باقی باشد، برای خود ذخیره کرده، بجهت صبح نگاه دارید.»
\par 24 پس آن را تا صبح ذخیره کردند، چنانکه موسی فرموده بود، و نه متعفن گردید و نه کرم در آن پیدا شد.
\par 25 و موسی گفت: «امروز این را بخورید زیرا که امروز سبت خداوند است، و در این‌روز آن را در صحرانخواهید یافت.
\par 26 شش روز آن را برچینید، وروز هفتمین، سبت است. در آن نخواهد بود.»
\par 27 و واقع شد که در روز هفتم، بعضی از قوم برای برچیدن بیرون رفتند، اما نیافتند.
\par 28 و خداوند به موسی گفت: «تا به کی از نگاه داشتن وصایا وشریعت من ابا می‌نمایید؟
\par 29 ببینید چونکه خداوند سبت را به شما بخشیده است، از این سبب در روز ششم، نان دو روز را به شما می‌دهد، پس هر کس در جای خود بنشیند و در روز هفتم هیچ‌کس از مکانش بیرون نرود.»
\par 30 پس قوم درروز هفتمین آرام گرفتند.
\par 31 و خاندان اسرائیل آن را من نامیدند، و آن مثل تخم گشنیز سفید بود، و طعمش مثل قرصهای عسلی.
\par 32 و موسی گفت: «این امری است که خداوند فرموده است که عومری از آن پرکنی، تا در نسلهای شما نگاه داشته شود، تا آن نان را ببینند که در صحرا، وقتی که شما را از زمین مصر بیرون آوردم، آن را به شما خورانیدم.»
\par 33 پس موسی به هارون گفت: «ظرفی بگیر، وعومری پر از من در آن بنه و آن را به حضورخداوند بگذار، تا در نسلهای شما نگاه داشته شود.»
\par 34 چنانکه خداوند به موسی‌امر فرموده بود، همچنان هارون آن را پیش (تابوت ) شهادت گذاشت تا نگاه داشته شود.
\par 35 و بنی‌اسرائیل مدت چهل سال من را می‌خوردند، تا به زمین آبادرسیدند، یعنی تا به‌سرحد زمین کنعان داخل شدند، خوراک ایشان من بود.و اما عومر، ده‌یک ایفه است.
\par 36 و اما عومر، ده‌یک ایفه است.
 
\chapter{17}

\par 1 و تمامی جماعت بنی‌اسرائیل به حکم خداوند طی منازل کرده، ازصحرای سین کوچ کردند، و در رفیدیم اردوزدند، و آب نوشیدن برای قوم نبود.
\par 2 و قوم باموسی منازعه کرده، گفتند: «ما را آب بدهید تابنوشیم.» موسی بدیشان گفت: «چرا با من منازعه می‌کنید، و چرا خداوند را امتحان می‌نمایید؟»
\par 3 ودر آنجا قوم تشنه آب بودند، و قوم بر موسی شکایت کرده، گفتند: «چرا ما را از مصر بیرون آوردی، تا ما و فرزندان و مواشی ما را به تشنگی بکشی؟»
\par 4 آنگاه موسی نزد خداوند استغاثه نموده، گفت: «با این قوم چه کنم؟ نزدیک است مرا سنگسار کنند.»
\par 5 خداوند به موسی گفت: «پیش روی قوم برو، و بعضی از مشایخ اسرائیل را با خود بردار، و عصای خود را که بدان نهر رازدی به‌دست خود گرفته، برو.
\par 6 همانا من در آنجاپیش روی تو بر آن صخره‌ای که در حوریب است، می‌ایستم، و صخره را خواهی زد تا آب ازآن بیرون آید، و قوم بنوشند.» پس موسی به حضور مشایخ اسرائیل چنین کرد.
\par 7 و آن موضع را مسه و مریبه نامید، به‌سبب منازعه بنی‌اسرائیل، و امتحان کردن ایشان خداوند را، زیرا گفته بودند: «آیا خداوند در میان ما هست یانه؟»
\par 8 پس عمالیق آمده، در رفیدیم با اسرائیل جنگ کردند.
\par 9 و موسی به یوشع گفت: «مردان برای ما برگزین و بیرون رفته، با عمالیق مقابله نما، و بامدادان من عصای خدا را به‌دست گرفته، بر قله کوه خواهم ایستاد.»
\par 10 پس یوشع بطوری که موسی او را امر فرموده بود کرد، تا با عمالیق محاربه کند. و موسی و هارون و حور بر قله کوه برآمدند.
\par 11 و واقع شد که چون موسی دست خود را برمی افراشت، اسرائیل غلبه می‌یافتند وچون دست خود را فرو می‌گذاشت، عمالیق چیره می‌شدند.
\par 12 و دستهای موسی سنگین شد. پس ایشان سنگی گرفته، زیرش نهادند که بر آن بنشیند. و هارون و حور، یکی از این طرف ودیگری از آن طرف، دستهای او را بر می‌داشتند، و دستهایش تا غروب آفتاب برقرار ماند.
\par 13 ویوشع، عمالیق و قوم او را به دم شمشیر منهزم ساخت.
\par 14 پس خداوند به موسی گفت: «این رابرای یادگاری در کتاب بنویس، و به سمع یوشع برسان که هرآینه ذکر عمالیق را از زیر آسمان محو خواهم ساخت.»
\par 15 و موسی مذبحی بنا کردو آن را یهوه نسی نامید.و گفت: «زیرا که دست بر تخت خداوند است، که خداوند را جنگ باعمالیق نسلا بعد نسل خواهد بود.»
\par 16 و گفت: «زیرا که دست بر تخت خداوند است، که خداوند را جنگ باعمالیق نسلا بعد نسل خواهد بود.»
 
\chapter{18}

\par 1 و چون یترون، کاهن مدیان، پدر زن موسی، آنچه را که خدا با موسی و قوم خود، اسرائیل کرده بود شنید که خداوند چگونه اسرائیل را از مصر بیرون آورده بود،
\par 2 آنگاه یترون پدرزن موسی، صفوره، زن موسی رابرداشت، بعد از آنکه او را پس فرستاده بود.
\par 3 ودو پسر او را که یکی را جرشون نام بود، زیراگفت: «در زمین بیگانه غریب هستم.»
\par 4 و دیگری را الیعازر نام بود، زیرا گفت: «که خدای پدرم مددکار من بوده، مرا از شمشیر فرعون رهانید.»
\par 5 پس یترون، پدر زن موسی، با پسران وزوجه‌اش نزد موسی به صحرا آمدند، در جایی که او نزد کوه خدا خیمه زده بود.
\par 6 و به موسی خبر داد که من یترون، پدر زن تو با زن تو و دوپسرش نزد تو آمده‌ایم.
\par 7 پس موسی به استقبال پدر زن خود بیرون آمد و او را تعظیم کرده، بوسیدو سلامتی یکدیگر را پرسیده، به خیمه درآمدند.
\par 8 و موسی پدر زن خود را از آنچه خداوند به فرعون و مصریان به‌خاطر اسرائیل کرده بود خبرداد، و از تمامی مشقتی که در راه بدیشان واقع شده، خداوند ایشان را از آن رهانیده بود.
\par 9 ویترون شاد گردید، به‌سبب تمامی احسانی که خداوند به اسرائیل کرده، و ایشان را از دست مصریان رهانیده بود.
\par 10 و یترون گفت: «متبارک است خداوند که شما را از دست مصریان و ازدست فرعون خلاصی داده است، و قوم خود را ازدست مصریان رهانیده.
\par 11 الان دانستم که یهوه ازجمیع خدایان بزرگتر است، خصوص در همان امری که بر ایشان تکبر می‌کردند.»
\par 12 و یترون، پدر زن موسی، قربانی سوختنی و ذبایح برای خدا گرفت، و هارون و جمیع مشایخ اسرائیل آمدند تا با پدر زن موسی به حضور خدا نان بخورند.
\par 13 بامدادان واقع شد که موسی برای داوری قوم بنشست، و قوم به حضور موسی از صبح تاشام ایستاده بودند.
\par 14 و چون پدر زن موسی آنچه را که او به قوم می‌کرد دید، گفت: «این چه‌کاراست که تو با قوم می‌نمایی؟ چرا تو تنها می نشینی و تمامی قوم نزد تو از صبح تا شام می‌ایستند؟»
\par 15 موسی به پدر زن خود گفت که «قوم نزد من می‌آیند تا از خدا مسالت نمایند.
\par 16 هرگاه ایشان را دعوی شود، نزد من می‌آیند، ومیان هر کس و همسایه‌اش داوری می‌کنم، وفرایض و شرایع خدا را بدیشان تعلیم می‌دهم.»
\par 17 پدر زن موسی به وی گفت: «کاری که تومی کنی، خوب نیست.
\par 18 هرآینه تو و این قوم نیزکه با تو هستند، خسته خواهید شد، زیرا که این امر برای تو سنگین است. تنها این را نمی توانی کرد.
\par 19 اکنون سخن مرا بشنو. تو را پند می‌دهم. وخدا با تو باد. و تو برای قوم به حضور خدا باش، وامور ایشان را نزد خدا عرضه دار.
\par 20 و فرایض وشرایع را بدیشان تعلیم ده، و طریقی را که بدان می‌باید رفتار نمود، و عملی را که می‌باید کرد، بدیشان اعلام نما.
\par 21 و از میان تمامی قوم، مردان قابل را که خداترس و مردان امین، که از رشوت نفرت کنند، جستجو کرده، بر ایشان بگمار، تاروسای هزاره و روسای صده و روسای پنجاه وروسای ده باشند.
\par 22 تا بر قوم پیوسته داوری نمایند، و هر امر بزرگ را نزد تو بیاورند، و هر امرکوچک را خود فیصل دهند. بدین طور بار خود راسبک خواهی کرد، و ایشان با تو متحمل آن خواهند شد.
\par 23 اگر این کار را بکنی و خدا تو راچنین امر فرماید، آنگاه یارای استقامت خواهی داشت، و جمیع این قوم نیز به مکان خود به سلامتی خواهند رسید.»
\par 24 پس موسی سخن پدر زن خود را اجابت کرده، آنچه او گفته بود به عمل آورد.
\par 25 و موسی مردان قابل از تمامی اسرائیل انتخاب کرده، ایشان را روسای قوم ساخت، روسای هزاره وروسای صده و روسای پنجاه و روسای ده.
\par 26 ودر داوری قوم پیوسته مشغول می‌بودند. هر امرمشکل را نزد موسی می‌آوردند، و هر دعوی کوچک را خود فیصل می‌دادند.و موسی پدرزن خود را رخصت داد و او به ولایت خود رفت.
\par 27 و موسی پدرزن خود را رخصت داد و او به ولایت خود رفت.
 
\chapter{19}

\par 1 و در ماه سوم از بیرون آمدن بنی‌اسرائیل از زمین مصر، در همان روزبه صحرای سینا آمدند، 
\par 2 و از رفیدیم کوچ کرده، به صحرای سینا رسیدند، و در بیابان اردو زدند، واسرائیل در آنجا در مقابل کوه فرود آمدند.
\par 3 وموسی نزد خدا بالا رفت، و خداوند از میان کوه اورا ندا درداد و گفت: «به خاندان یعقوب چنین بگو، و بنی‌اسرائیل را خبر بده:
\par 4 شما آنچه را که من به مصریان کردم، دیده‌اید، و چگونه شما را بربالهای عقاب برداشته، نزد خود آورده‌ام.
\par 5 واکنون اگر آواز مرا فی الحقیقه بشنوید، و عهد مرانگاه دارید، همانا خزانه خاص من از جمیع قومهاخواهید بود. زیرا که تمامی جهان، از آن من است.
\par 6 و شما برای من مملکت کهنه و امت مقدس خواهید بود. این است آن سخنانی که به بنی‌اسرائیل می‌باید گفت.»
\par 7 پس موسی آمده، مشایخ قوم را خواند، و همه این سخنان را که خداوند او را فرموده بود، بر ایشان القا کرد.
\par 8 وتمامی قوم به یک زبان در جواب گفتند: «آنچه خداوند امر فرموده است، خواهیم کرد.» و موسی سخنان قوم را باز به خداوند عرض کرد.
\par 9 و خداوند به موسی گفت: «اینک من در ابرمظلم نزد تو می‌آیم، تا هنگامی که به تو سخن گویم قوم بشنوند، و بر تو نیز همیشه ایمان داشته باشند.» پس موسی سخنان قوم را به خداوند بازگفت.
\par 10 خداوند به موسی گفت: «نزد قوم برو وایشان را امروز و فردا تقدیس نما، و ایشان رخت خود را بشویند.
\par 11 و در روز سوم مهیا باشید، زیرا که در روز سوم خداوند در نظر تمامی قوم برکوه سینا نازل شود.
\par 12 و حدود برای قوم از هرطرف قرار ده، و بگو: باحذر باشید از اینکه به فرازکوه برآیید، یا دامنه آن را لمس نمایید، زیرا هر‌که کوه را لمس کند، هرآینه کشته شود.
\par 13 دست برآن گذارده نشود بلکه یا سنگسار شود یا به تیرکشته شود، خواه بهایم باشد خواه انسان، زنده نماند. اما چون کرنا نواخته شود، ایشان به کوه برآیند.»
\par 14 پس موسی از کوه نزد قوم فرود آمده، قوم را تقدیس نمود و رخت خود را شستند.
\par 15 و به قوم گفت: «در روز سوم حاضر باشید، و به زنان نزدیکی منمایید.»
\par 16 و واقع شد در روز سوم به وقت طلوع صبح، که رعدها و برقها و ابر غلیظ برکوه پدید آمد، و آواز کرنای بسیار سخت، بطوری که تمامی قوم که در لشکرگاه بودند، بلرزیدند.
\par 17 و موسی قوم را برای ملاقات خدا ازلشکرگاه بیرون آورد، و در پایان کوه ایستادند.
\par 18 و تمامی کوه سینا را دود فرو گرفت، زیراخداوند در آتش بر آن نزول کرد، و دودش مثل دود کوره‌ای بالا می‌شد، و تمامی کوه سخت متزلزل گردید.
\par 19 و چون آواز کرنا زیاده و زیاده سخت نواخته می‌شد، موسی سخن گفت، و خدا او را به زبان جواب داد.
\par 20 و خداوند بر کوه سینا بر قله کوه نازل شد، و خداوند موسی را به قله کوه خواند، و موسی بالا رفت.
\par 21 و خداوند به موسی گفت: «پایین برو و قوم را قدغن نما، مبادا نزدخداوند برای نظر کردن، از حد تجاوز نمایند، که بسیاری از ایشان هلاک خواهند شد.
\par 22 و کهنه نیز که نزد خداوند می‌آیند، خویشتن را تقدیس نمایند، مبادا خداوند بر ایشان هجوم آورد.»
\par 23 موسی به خداوند گفت: «قوم نمی توانند به فرازکوه سینا آیند، زیرا که تو ما را قدغن کرده، گفته‌ای کوه را حدود قرار ده و آن را تقدیس نما.»
\par 24 خداوند وی را گفت: «پایین برو و تو و هارون همراهت برآیید، اما کهنه و قوم از حد تجاوزننمایند، تا نزد خداوند بالا بیایند، مبادا بر ایشان هجوم آورد.»پس موسی نزد قوم فرود شده، بدیشان سخن گفت.
\par 25 پس موسی نزد قوم فرود شده، بدیشان سخن گفت.
 
\chapter{20}

\par 1 و خدا تکلم فرمود و همه این کلمات را بگفت:
\par 2 «من هستم یهوه، خدای تو، که تو را از زمین مصر و از خانه غلامی بیرون آوردم.
\par 3 تو را خدایان دیگر غیر از من نباشد.
\par 4 صورتی تراشیده و هیچ تمثالی از آنچه بالا درآسمان است، و از آنچه پایین در زمین است، و ازآنچه در آب زیر زمین است، برای خود مساز.
\par 5 نزد آنها سجده مکن، و آنها را عبادت منما، زیرامن که یهوه، خدای تو می‌باشم، خدای غیورهستم، که انتقام گناه پدران را از پسران تا پشت سوم و چهارم از آنانی که مرا دشمن دارندمی گیرم.
\par 6 و تا هزار پشت بر آنانی که مرا دوست دارند و احکام مرا نگاه دارند، رحمت می‌کنم.
\par 7 نام یهوه، خدای خود را به باطل مبر، زیراخداوند کسی را که اسم او را به باطل برد، بی‌گناه نخواهد شمرد.
\par 8 روز سبت را یاد کن تا آن راتقدیس نمایی.
\par 9 شش روز مشغول باش و همه کارهای خود را بجا آور.
\par 10 اما روز هفتمین، سبت یهوه، خدای توست. در آن هیچ کار مکن، تو و پسرت و دخترت و غلامت و کنیزت وبهیمه ات و مهمان تو که درون دروازه های توباشد.
\par 11 زیرا که در شش روز، خداوند آسمان وزمین و دریا و آنچه را که در آنهاست بساخت، ودر روز هفتم آرام فرمود. از این سبب خداوند روزهفتم را مبارک خوانده، آن را تقدیس نمود.
\par 12 پدر و مادر خود را احترام نما، تا روزهای تو درزمینی که یهوه خدایت به تو می‌بخشد، درازشود.
\par 13 قتل مکن.
\par 14 زنا مکن.
\par 15 دزدی مکن.
\par 16 بر همسایه خود شهادت دروغ مده.
\par 17 به خانه همسایه خود طمع مورز، و به زن همسایه ات وغلامش و کنیزش و گاوش و الاغش و به هیچ‌چیزی که از آن همسایه تو باشد، طمع مکن.»
\par 18 و جمیع قوم رعدها و زبانه های آتش وصدای کرنا و کوه را که پر از دود بود دیدند، وچون قوم این را بدیدند لرزیدند، و از دوربایستادند.
\par 19 و به موسی گفتند: «تو به ما سخن بگو و خواهیم شنید، اما خدا به ما نگوید، مبادابمیریم.»
\par 20 موسی به قوم گفت: «مترسید زیراخدا برای امتحان شما آمده است، تا ترس اوپیش روی شما باشد و گناه نکنید.»
\par 21 پس قوم ازدور ایستادند و موسی به ظلمت غلیظ که خدا درآن بود، نزدیک آمد.
\par 22 و خداوند به موسی گفت: «به بنی‌اسرائیل چنین بگو: شما دیدید که ازآسمان به شما سخن گفتم:
\par 23 با من خدایان نقره مسازید و خدایان طلا برای خود مسازید.
\par 24 مذبحی از خاک برای من بساز، و قربانی های سوختنی خود و هدایای سلامتی خود را از گله ورمه خویش بر آن بگذران، در هر جایی که یادگاری برای نام خود سازم، نزد تو خواهم آمد، و تو را برکت خواهم داد.
\par 25 و اگر مذبحی ازسنگ برای من سازی، آن را از سنگهای تراشیده بنا مکن، زیرا اگر افزار خود را بر آن بلند کردی، آن را نجس خواهی ساخت.و بر مذبح من ازپله‌ها بالا مرو، مبادا عورت تو بر آن مکشوف شود.»
\par 26 و بر مذبح من ازپله‌ها بالا مرو، مبادا عورت تو بر آن مکشوف شود.»
 
\chapter{21}

\par 1 «و این است احکامی که پیش ایشان می گذاری:
\par 2 اگر غلام عبری بخری، شش سال خدمت کند، و در هفتمین، بی‌قیمت، آزاد بیرون رود.
\par 3 اگر تنها آمده، تنها بیرون رود. واگر صاحب زن بوده، زنش همراه او بیرون رود.
\par 4 اگر آقایش زنی بدو دهد و پسران یا دختران برایش بزاید، آنگاه زن و اولادش از آن آقایش باشند، و آن مرد تنها بیرون رود.
\par 5 لیکن هرگاه آن غلام بگوید که هرآینه آقایم و زن و فرزندان خودرا دوست می‌دارم و نمی خواهم که آزاد بیرون روم،
\par 6 آنگاه آقایش او را به حضور خدا بیاورد، واو را نزدیک در یا قایمه در برساند، و آقایش گوش او را با درفشی سوراخ کند، و او وی راهمیشه بندگی نماید.
\par 7 اما اگر شخصی، دخترخود را به کنیزی بفروشد، مثل غلامان بیرون نرود.
\par 8 هر گاه به نظر آقایش که او را برای خودنامزد کرده است ناپسند آید، بگذارد که او را فدیه دهند، اما هیچ حق ندارد که او را به قوم بیگانه بفروشد، زیرا که بدو خیانت کرده است.
\par 9 و هرگاه او را به پسر خود نامزد کند، موافق رسم دختران بااو عمل نماید.
\par 10 اگر زنی دیگر برای خود گیرد، آنگاه خوراک و لباس و مباشرت او را کم نکند.
\par 11 و اگر این سه چیز را برای او نکند، آنگاه بی‌قیمت و رایگان بیرون رود.
\par 12 «هر‌که انسانی را بزند و او بمیرد، هر آینه کشته شود.
\par 13 اما اگر قصد او نداشت، بلکه خداوی را بدستش رسانید، آنگاه مکانی برای تو معین کنم تا بدانجا فرار کند.
\par 14 لیکن اگر شخصی عمد بر همسایه خود آید، تا او را به مکر بکشد، آنگاه او را از مذبح من کشیده، به قتل برسان.
\par 15 و هر‌که پدر یا مادر خود را زند، هرآینه کشته شود.
\par 16 وهر‌که آدمی را بدزدد و او را بفروشد یا در دستش یافت شود، هرآینه کشته شود.
\par 17 و هر‌که پدر یامادر خود را لعنت کند، هرآینه کشته شود.
\par 18 واگر دو مرد نزاع کنند، و یکی دیگری را به سنگ یابه مشت زند، و او نمیرد لیکن بستری شود،
\par 19 اگربرخیزد و با عصا بیرون رود، آنگاه زننده او بی‌گناه شمرده شود، اما عوض بیکاریش را ادا نماید، وخرج معالجه او را بدهد.
\par 20 و اگر کسی غلام یاکنیز خود را به عصا بزند، و او زیر دست او بمیرد، هرآینه انتقام او گرفته شود.
\par 21 لیکن اگر یک دوروز زنده بماند، از او انتقام کشیده نشود، زیرا که زرخرید اوست.
\par 22 و اگر مردم جنگ کنند، و زنی حامله را بزنند، و اولاد او سقط گردد، و ضرری دیگر نشود، البته غرامتی بدهد موافق آنچه شوهرزن بدو گذارد، و به حضور داوران ادا نماید.
\par 23 واگر اذیتی دیگر حاصل شود، آنگاه جان به عوض جان بده.
\par 24 و چشم به عوض چشم، و دندان به عوض دندان، و دست به عوض دست، و پا به عوض پا.
\par 25 و داغ به عوض داغ، و زخم به عوض زخم، و لطمه به عوض لطمه.
\par 26 و اگر کسی چشم غلام یا چشم کنیز خود را بزند که ضایع شود، او را به عوض چشمش آزاد کند.
\par 27 و اگردندان غلام یا دندان کنیز خود را بیندازد او را به عوض دندانش آزاد کند.
\par 28 و هرگاه گاوی به شاخ خود مردی یا زنی را بزند که او بمیرد، گاو را البته سنگسار کنند، و گوشتش را نخورند و صاحب گاو بی‌گناه باشد.
\par 29 و لیکن اگر گاو قبل از آن شاخ زن می‌بود، و صاحبش آگاه بود، و آن را نگاه نداشت، و او مردی یا زنی را کشت، گاو راسنگسار کنند، و صاحبش را نیز به قتل رسانند.
\par 30 و اگر دیه بر او گذاشته شود، آنگاه برای فدیه جان خود هرآنچه بر او مقرر شود، ادا نماید.
\par 31 خواه پسر خواه دختر را شاخ زده باشد، به حسب این حکم با او عمل کنند.
\par 32 اگر گاو، غلامی یا کنیزی را بزند، سی مثقال نقره به صاحب او داده شود، و گاو سنگسار شود.
\par 33 واگر کسی چاهی گشاید یا کسی چاهی حفر کند، وآن را نپوشاند، و گاوی یا الاغی در آن افتد،
\par 34 صاحب چاه عوض او را بدهد، و قیمتش را به صاحبش ادا نماید، و میته از آن او باشد.
\par 35 و اگرگاو شخصی، گاو همسایه او را بزند، و آن بمیردپس گاو زنده را بفروشند، و قیمت آن را تقسیم کنند، و میته را نیز تقسیم نمایند،اما اگر معلوم بوده باشد که آن گاو قبل از آن شاخ زن می‌بود، وصاحبش آن را نگاه نداشت، البته گاو به عوض گاو بدهد و میته از آن او باشد.
\par 36 اما اگر معلوم بوده باشد که آن گاو قبل از آن شاخ زن می‌بود، وصاحبش آن را نگاه نداشت، البته گاو به عوض گاو بدهد و میته از آن او باشد.
 
\chapter{22}

\par 1 «اگر کسی گاوی یا گوسفندی بدزدد، و آن را بکشد یا بفروشد، به عوض گاوپنج گاو، و به عوض گوسفند چهار گوسفند بدهد.
\par 2 اگر دزدی در رخنه کردن گرفته شود، و او رابزنند بطوری که بمیرد، بازخواست خون برای اونباشد.
\par 3 اما اگر آفتاب بر او طلوع کرد، بازخواست خون برای او هست. البته مکافات باید داد، و اگر چیزی ندارد، به عوض دزدی که کرد، فروخته شود.
\par 4 اگر چیزی دزدیده شده، ازگاو یا الاغ یا گوسفند زنده در دست او یافت شود، دو مقابل آن را رد کند.
\par 5 اگر کسی مرتعی یاتاکستانی را بچراند، یعنی مواشی خود را براند تامرتع دیگری را بچراند، از نیکوترین مرتع و ازبهترین تاکستان خود عوض بدهد.
\par 6 اگر آتشی بیرون رود، و خارها را فراگیرد و بافه های غله یاخوشه های نادرویده یا مزرعه‌ای سوخته گردد، هر‌که آتش را افروخته است، البته عوض بدهد. 
\par 7 اگر کسی پول یا اسباب نزد همسایه خود امانت گذارد، و از خانه آن شخص دزدیده شود، هر گاه دزد پیدا شود، دو چندان رد نماید.
\par 8 و اگر دزدگرفته نشود، آنگاه صاحب‌خانه را به حضورحکام بیاورند، تا حکم شود که آیا دست خود رابر اموال همسایه خویش دراز کرده است یا نه.
\par 9 در هر خیانتی از گاو و الاغ و گوسفند و رخت وهر چیز گم شده، که کسی بر آن ادعا کند، امر هردو به حضور خدا برده شود، و بر گناه هر کدام که خدا حکم کند، دو چندان به همسایه خود ردنماید.
\par 10 اگر کسی الاغی یا گاوی یا گوسفندی یاجانوری دیگر به همسایه خود امانت دهد، و آن بمیرد یا پایش شکسته شود یا دزدیده شود، وشاهدی نباشد،
\par 11 قسم خداوند در میان هر دونهاده شود، که دست خود را به مال همسایه خویش دراز نکرده است. پس مالکش قبول بکندو او عوض ندهد.
\par 12 لیکن اگر از او دزدیده شد، به صاحبش عوض باید داد.
\par 13 و اگر دریده شد، آن را برای شهادت بیاورد، و برای دریده شده، عوض ندهد.
\par 14 و اگر کسی حیوانی از همسایه خود عاریت گرفت، و پای آن شکست یا مرد، وصاحبش همراهش نبود، البته عوض باید داد.
\par 15 اما اگر صاحبش همراهش بود، عوض نباید داد، و اگر کرایه شد، برای کرایه آمده بود.
\par 16 «اگر کسی دختری را که نامزد نبود فریب داده، با او هم بستر شد، البته می‌باید او را زن منکوحه خویش سازد.
\par 17 و هرگاه پدرش راضی نباشد که او را بدو دهد، موافق مهر دوشیزگان نقدی بدو باید داد.
\par 18 زن جادوگر را زنده مگذار.
\par 19 هر‌که با حیوانی مقاربت کند، هرآینه کشته شود.
\par 20 هر‌که برای خدای غیر از یهوه و بس قربانی گذراند، البته هلاک گردد.
\par 21 غریبی رااذیت مرسانید. و بر او ظلم مکنید، زیرا که درزمین مصر غریب بودید.
\par 22 بر بیوه‌زنی یا یتیمی ظلم مکنید.
\par 23 و هر گاه بر او ظلم کردی، و او نزدمن فریاد برآورد، البته فریاد او را مستجاب خواهم فرمود.
\par 24 و خشم من مشتعل شود، و شمارا به شمشیر خواهم کشت، و زنان شما بیوه شوندو پسران شما یتیم.
\par 25 اگر نقدی به فقیری ازقوم من که همسایه تو باشد قرض دادی، مثل رباخوار با او رفتار مکن و هیچ سود بر اومگذار.
\par 26 اگر رخت همسایه خود را به گروگرفتی، آن را قبل از غروب آفتاب بدو رد کن.
\par 27 زیرا که آن فقط پوشش او و لباس برای بدن اوست، پس در چه چیز بخوابد، و اگر نزد من فریاد برآورد، هرآینه اجابت خواهم فرمود، زیرا که من کریم هستم.
\par 28 به خدا ناسزا مگو ورئیس قوم خود را لعنت مکن.
\par 29 درآوردن نوبرغله و عصیر رز خود تاخیر منما. و نخست زاده پسران خود را به من بده.
\par 30 با گاوان وگوسفندان خود چنین بکن. هفت روز نزد مادرخود بماند و در روز هشتمین آن را به من بده.وبرای من مردان مقدس باشید، و گوشتی را که در صحرا دریده شود مخورید؛ آن را نزد سگان بیندازید.
\par 31 وبرای من مردان مقدس باشید، و گوشتی را که در صحرا دریده شود مخورید؛ آن را نزد سگان بیندازید.
 
\chapter{23}

\par 1 «خبر باطل را انتشار مده، و با شریران همداستان مشو، که شهادت دروغ دهی.
\par 2 پیروی بسیاری برای عمل بد مکن؛ و درمرافعه، محض متابعت کثیری، سخنی برای انحراف حق مگو.
\par 3 و در مرافعه فقیر نیزطرفداری او منما.
\par 4 اگر گاو یا الاغ دشمن خود رایافتی که گم شده باشد، البته آن را نزد او بازبیاور.
\par 5 اگر الاغ دشمن خود را زیر بارش خوابیده یافتی، و از گشادن او روگردان هستی، البته آن را همراه او باید بگشایی.
\par 6 حق فقیرخود را در دعوی او منحرف مساز.
\par 7 از امردروغ اجتناب نما و بی‌گناه و صالح را به قتل مرسان زیرا که ظالم را عادل نخواهم شمرد.
\par 8 ورشوت مخور زیرا که رشوت بینایان را کورمی کند و سخن صدیقان را کج می‌سازد.
\par 9 و برشخص غریب ظلم منما زیرا که از دل غریبان خبر دارید، چونکه در زمین مصر غریب بودید.
\par 10 «و شش سال مزرعه خود را بکار ومحصولش را جمع کن،
\par 11 لیکن در هفتمین آن رابگذار و ترک کن تا فقیران قوم تو از آن بخورند و آنچه از ایشان باقی ماند حیوانات صحرا بخورند. همچنین با تاکستان و درختان زیتون خود عمل نما.
\par 12 شش روز به شغل خود بپرداز و در روزهفتمین آرام کن تا گاوت و الاغت آرام گیرند وپسر کنیزت و مهمانت استراحت کنند.
\par 13 و آنچه را به شما گفته‌ام، نگاه دارید و نام خدایان غیر راذکر مکنید، از زبانت شنیده نشود.
\par 14 «در هر سال سه مرتبه عید برای من نگاه دار.
\par 15 عید فطیر را نگاه دار، و چنانکه تو را امرفرموده‌ام، هفت روز نان فطیر بخور در زمان معین در ماه ابیب، زیرا که در آن از مصر بیرون آمدی. وهیچ‌کس به حضور من تهی‌دست حاضر نشود.
\par 16 و عید حصاد نوبر غلات خود را که در مزرعه کاشته‌ای، و عید جمع را در آخر سال وقتی که حاصل خود را از صحرا جمع کرده‌ای.
\par 17 در هرسال سه مرتبه همه ذکورانت به حضور خداوندیهوه حاضر شوند.
\par 18 خون قربانی مرا با نان خمیرمایه دار مگذران و پیه عید من تا صبح باقی نماند.
\par 19 نوبر نخستین زمین خود را به خانه یهوه خدای خود بیاور و بزغاله را در شیر مادرش مپز.
\par 20 «اینک من فرشته‌ای پیش روی تومی فرستم تا تو را در راه محافظت نموده، بدان مکانی که مهیا کرده‌ام برساند.
\par 21 از او با حذر باش و آواز او را بشنو و از او تمرد منما زیرا گناهان شما را نخواهد آمرزید، چونکه نام من در اوست.
\par 22 و اگر قول او را شنیدی و به آنچه گفته‌ام عمل نمودی، هرآینه دشمن دشمنانت و مخالف مخالفانت خواهم بود،
\par 23 زیرا فرشته من پیش روی تو می‌رود و تو را به اموریان و حتیان وفرزیان و کنعانیان و حویان و یبوسیان خواهدرسانید و ایشان را هلاک خواهم ساخت.
\par 24 خدایان ایشان را سجده منما و آنها را عبادت مکن و موافق کارهای ایشان مکن، البته آنها رامنهدم ساز و بتهای ایشان را بشکن.
\par 25 و یهوه، خدای خود را عبادت نمایید تا نان و آب تو رابرکت دهد و بیماری را از میان تو دور خواهم کرد،
\par 26 و در زمینت سقط کننده و نازاد نخواهدبود و شماره روزهایت را تمام خواهم کرد.
\par 27 وخوف خود را پیش روی تو خواهم فرستاد و هرقومی را که بدیشان برسی متحیر خواهم ساخت و جمیع دشمنانت را پیش تو روگردان خواهم ساخت.
\par 28 و زنبورها پیش روی تو خواهم فرستاد تا حویان و کنعانیان و حتیان را ازحضورت برانند.
\par 29 ایشان را در یک سال ازحضور تو نخواهم راند، مبادا زمین ویران گردد وحیوانات صحرا بر تو زیاده شوند.
\par 30 ایشان را ازپیش روی تو به تدریج خواهم راند تا کثیر شوی و زمین را متصرف گردی.
\par 31 و حدود تو را از بحرقلزم تا بحر فلسطین، و از صحرا تا نهر فرات قراردهم زیرا ساکنان آن زمین را بدست شما خواهم سپرد و ایشان را از پیش روی خود خواهی راند.
\par 32 با ایشان و با خدایان ایشان عهد مبند.درزمین تو ساکن نشوند، مبادا تو را بر من عاصی گردانند و خدایان ایشان را عبادت کنی و دامی برای تو باشد.»
\par 33 درزمین تو ساکن نشوند، مبادا تو را بر من عاصی گردانند و خدایان ایشان را عبادت کنی و دامی برای تو باشد.»
 
\chapter{24}

\par 1 و به موسی گفت: «نزد خداوند بالابیا، تو و هارون و ناداب و ابیهو و هفتادنفر از مشایخ اسرائیل و از دور سجده کنید.
\par 2 وموسی تنها نزدیک خداوند بیاید و ایشان نزدیک نیایند و قوم همراه او بالا نیایند.»
\par 3 پس موسی آمده، همه سخنان خداوند و همه این احکام را به قوم باز‌گفت و تمامی قوم به یک زبان در جواب گفتند: «همه سخنانی که خداوند گفته است، بجاخواهیم آورد.»
\par 4 و موسی تمامی سخنان خداوند را نوشت و بامدادان برخاسته، مذبحی درپای کوه و دوازده ستون، موافق دوازده سبطاسرائیل بنا نهاد.
\par 5 و بعضی از جوانان بنی‌اسرائیل را فرستاد و قربانی های سوختنی گذرانیدند وقربانی های سلامتی از گاوان برای خداوند ذبح کردند.
\par 6 و موسی نصف خون را گرفته، در لگنهاریخت و نصف خون را بر مذبح پاشید،
\par 7 و کتاب عهد را گرفته، به سمع قوم خواند. پس گفتند: «هرآنچه خداوند گفته است، خواهیم کرد و گوش خواهیم گرفت.»
\par 8 و موسی خون را گرفت و بر قوم پاشیده، گفت: «اینک خون آن عهدی که خداوند بر جمیع این سخنان با شما بسته است.»
\par 9 و موسی با هارون و ناداب و ابیهو و هفتاد نفراز مشایخ اسرائیل بالا رفت.
\par 10 و خدای اسرائیل را دیدند، و زیر پایهایش مثل صنعتی از یاقوت کبود شفاف و مانند ذات آسمان در صفا.
\par 11 و برسروران بنی‌اسرائیل دست خود را نگذارد، پس خدا را دیدند و خوردند و آشامیدند.
\par 12 وخداوند به موسی گفت: «نزد من به کوه بالا بیا، وآنجا باش تا لوحهای سنگی و تورات و احکامی را که نوشته‌ام تا ایشان را تعلیم نمایی، به تو دهم.»
\par 13 پس موسی با خادم خود یوشع برخاست، وموسی به کوه خدا بالا آمد.
\par 14 و به مشایخ گفت: «برای ما در اینجا توقف کنید، تا نزد شمابرگردیم، همانا هارون و حور با شما می‌باشند. پس هر‌که امری دارد، نزد ایشان برود.»
\par 15 و چون موسی به فراز کوه برآمد، ابر کوه را فرو گرفت.
\par 16 و جلال خداوند بر کوه سینا قرار گرفت، وشش روز ابر آن را پوشانید، و روز هفتمین، موسی را از میان ابر ندا درداد.
\par 17 و منظر جلال خداوند، مثل آتش سوزنده در نظر بنی‌اسرائیل برقله کوه بود.و موسی به میان ابر داخل شده، به فراز کوه برآمد، و موسی چهل روز و چهل شب در کوه ماند.
\par 18 و موسی به میان ابر داخل شده، به فراز کوه برآمد، و موسی چهل روز و چهل شب در کوه ماند.
 
\chapter{25}

\par 1 و خداوند موسی را خطاب کرده، گفت:
\par 2 «به بنی‌اسرائیل بگو که برای من هدایا بیاورند؛ از هر‌که به میل دل بیاورد، هدایای مرا بگیرید.
\par 3 و این است هدایا که از ایشان می‌گیرید: طلا و نقره و برنج،
\par 4 و لاجورد وارغوان و قرمز و کتان نازک و پشم بز،
\par 5 و پوست قوچ سرخ شده و پوست خز و چوب شطیم،
\par 6 وروغن برای چراغها، و ادویه برای روغن مسح، وبرای بخور معطر،
\par 7 و سنگهای عقیق و سنگهای مرصعی برای ایفود و سینه بند.
\par 8 و مقامی ومقدسی برای من بسازند تا در میان ایشان ساکن شوم.
\par 9 موافق هر‌آنچه به تو نشان دهم از نمونه مسکن و نمونه جمیع اسبابش، همچنین بسازید.
\par 10 «و تابوتی از چوب شطیم بسازند که طولش دو ذراع و نیم، و عرضش یک ذراع و نیم وبلندیش یک ذراع و نیم باشد.
\par 11 و آن را به طلای خالص بپوشان. آن را از درون و بیرون بپوشان، وبر زبرش به هر طرف تاجی زرین بساز.
\par 12 وبرایش چهار حلقه زرین بریز، و آنها را بر چهارقایمه‌اش بگذار، دو حلقه بر یک طرفش و دوحلقه بر طرف دیگر.
\par 13 و دو عصا از چوب شطیم بساز، و آنها را به طلا بپوشان.
\par 14 و آن عصاها رادر حلقه هایی که بر طرفین تابوت باشد بگذران، تاتابوت را به آنها بردارند.
\par 15 و عصاها درحلقه های تابوت بماند و از آنها برداشته نشود.
\par 16 و آن شهادتی را که به تو می‌دهم، در تابوت بگذار.
\par 17 و تخت رحمت را از طلای خالص بساز. طولش دو ذراع و نیم، و عرضش یک ذراع ونیم.
\par 18 و دو کروبی از طلا بساز، آنها را ازچرخکاری از هر دو طرف تخت رحمت بساز.
\par 19 و یک کروبی در این سر و کروبی دیگر در آن سر بساز، کروبیان را از تخت رحمت بر هر دوطرفش بساز.
\par 20 و کروبیان بالهای خود را بر زبرآن پهن کنند، و تخت رحمت را به بالهای خودبپوشانند. و رویهای ایشان به سوی یکدیگرباشد، و رویهای کروبیان به طرف تخت رحمت باشد.
\par 21 و تخت رحمت را بر روی تابوت بگذارو شهادتی را که به تو می‌دهم در تابوت بنه.
\par 22 ودر آنجا با تو ملاقات خواهم کرد و از بالای تخت رحمت از میان دو کروبی که بر تابوت شهادت می‌باشند، با تو سخن خواهم گفت، درباره همه اموری که بجهت بنی‌اسرائیل تو را امر خواهم فرمود.
\par 23 «و خوانی از چوب شطیم بساز که طولش دو ذراع، و عرضش یک ذراع، و بلندیش یک ذراع و نیم باشد. 
\par 24 و آن را به طلای خالص بپوشان، و تاجی از طلا به هر طرفش بساز.
\par 25 وحاشیه‌ای به قدر چهار انگشت به اطرافش بساز، و برای حاشیه‌اش تاجی زرین از هر طرف بساز.
\par 26 و چهار حلقه زرین برایش بساز، و حلقه‌ها را برچهار گوشه چهار قایمه‌اش بگذار.
\par 27 و حلقه هادر برابر حاشیه باشد، تا خانه‌ها باشد بجهت عصاها برای برداشتن خوان.
\par 28 و عصاها را ازچوب شطیم بساز، و آنها را به طلا بپوشان تاخوان را بدانها بردارند.
\par 29 و صحنها و کاسه‌ها وجامها و پیاله هایش را که به آنها هدایای ریختنی می‌ریزند بساز، آنها را از طلای خالص بساز.
\par 30 ونان تقدمه را بر خوان، همیشه به حضور من بگذار.
\par 31 «و چراغدانی از طلای خالص بساز، و ازچرخکاری چراغدان ساخته شود، قاعده‌اش وپایه‌اش و پیاله هایش و سیبهایش و گلهایش ازهمان باشد.
\par 32 و شش شاخه از طرفینش بیرون آید، یعنی سه شاخه چراغدان از یک طرف و سه شاخه چراغدان از طرف دیگر.
\par 33 سه پیاله بادامی با سیبی و گلی در یک شاخه و سه پیاله بادامی با سیبی و گلی در شاخه دیگر و هم چنین در شش شاخه‌ای که از چراغدان بیرون می‌آید.
\par 34 و درچراغدان چهار پیاله بادامی با سیبها و گلهای آنهاباشد.
\par 35 و سیبی زیر دو شاخه آن و سیبی زیر دوشاخه آن و سیبی زیر دو شاخه آن بر شش شاخه‌ای که از چراغدان بیرون می‌آید.
\par 36 وسیبها و شاخه هایش از همان باشد، یعنی از یک چرخکاری طلای خالص.
\par 37 و هفت چراغ برای آن بساز، و چراغهایش را بر بالای آن بگذار تاپیش روی آن را روشنایی دهند.
\par 38 و گل گیرها وسینیهایش از طلای خالص باشد.
\par 39 خودش باهمه اسبابش از یک وزنه طلای خالص ساخته شود.و آگاه باش که آنها را موافق نمونه آنها که در کوه به تو نشان داده شد بسازی.
\par 40 و آگاه باش که آنها را موافق نمونه آنها که در کوه به تو نشان داده شد بسازی.
 
\chapter{26}

\par 1 «و مسکن را از ده پرده کتان نازک تابیده، و لاجورد و ارغوان و قرمز بساز. با کروبیان از صنعت نساج ماهر آنها را ترتیب نما.
\par 2 طول یک پرده بیست و هشت ذراع، و عرض یک پرده چهار ذراع، و همه پرده‌ها را یک اندازه باشد.
\par 3 پنج پرده با یکدیگر پیوسته باشد، و پنج پرده با یکدیگر پیوسته.
\par 4 و مادگیهای لاجورد برکنار هر پرده‌ای بر لب پیوستگی‌اش بساز، و برکنار پرده بیرونی در‌پیوستگی دوم چنین بساز.
\par 5 پنجاه مادگی در یک پرده بساز، و پنجاه مادگی در کنار پرده‌ای که در‌پیوستگی دوم است بساز، به قسمی که مادگیها مقابل یکدیگر باشد.
\par 6 و پنجاه تکمه زرین بساز و پرده‌ها را به تکمه‌ها با یکدیگرپیوسته ساز، تا مسکن یک باشد.
\par 7 و خیمه بالای مسکن را از پرده های پشم بز بساز، و برای آن یازده پرده درست کن.
\par 8 طول یک پرده سی ذراع، و عرض یک پرده چهار ذراع، و اندازه هر یازده پرده‌یک باشد.
\par 9 و پنج پرده را جدا و شش پرده راجدا، پیوسته ساز، و پرده ششم را پیش روی خیمه دولا کن.
\par 10 و پنجاه مادگی بر کنار پرده‌ای که در‌پیوستگی بیرون است بساز، و پنجاه مادگی بر کنار پرده‌ای که در‌پیوستگی دوم است.
\par 11 وپنجاه تکمه برنجین بساز، و تکمه‌ها را در مادگیهابگذران، و خیمه را با هم پیوسته ساز تا یک باشد.
\par 12 و زیادتی پرده های خیمه که باقی باشد، یعنی نصف پرده که زیاده است، از پشت خیمه آویزان شود.
\par 13 و ذراعی از این طرف و ذراعی از آن طرف که در طول پرده های خیمه زیاده باشد، برطرفین مسکن از هر دو جانب آویزان شود تا آن رابپوشد.
\par 14 و پوششی برای خیمه از پوست قوچ سرخ شده بساز، و پوششی از پوست خز بر زبرآن.
\par 15 «و تخته های قایم از چوب شطیم برای مسکن بساز.
\par 16 طول هر تخته ده ذراع، و عرض هر تخته یک ذراع و نیم.
\par 17 و در هر تخته دو زبانه قرینه یکدیگر باشد، و همه تخته های مسکن راچنین بساز.
\par 18 و تخته‌ها برای مسکن بساز، یعنی بیست تخته از طرف جنوب به سمت یمانی.
\par 19 وچهل پایه نقره در زیر آن بیست تخته بساز، یعنی دو پایه زیر یک تخته برای دو زبانه‌اش، و دو پایه زیر یک تخته برای دو زبانه‌اش.
\par 20 و برای جانب دیگر مسکن، از طرف شمال بیست تخته باشد.
\par 21 و چهل پایه نقره آنها، یعنی دو پایه زیر یک تخته و دو پایه زیر تخته دیگر.
\par 22 و برای موخر مسکن از جانب غربی شش تخته بساز.
\par 23 و برای گوشه های مسکن در موخرش دو تخته بساز.
\par 24 واز زیر وصل کرده شود، و تا بالا نیز در یک حلقه باهم پیوسته شود، و برای هر دو چنین بشود، در هردو گوشه باشد.
\par 25 و هشت تخته باشد و پایه های آنها از نقره شانزده پایه باشد، یعنی دو پایه زیریک تخته و دو پایه زیر تخته دیگر.
\par 26 «و پشت بندها از چوب شطیم بساز، پنج ازبرای تخته های یک طرف مسکن،
\par 27 و پنج پشت بند برای تخته های طرف دیگر مسکن، وپنج پشت بند برای تخته های طرف مسکن درموخرش به سمت مغرب.
\par 28 و پشت بند وسطی که میان تخته هاست، از این سر تا آن سر بگذرد.
\par 29 و تخته‌ها را به طلا بپوشان و حلقه های آنها رااز طلا بساز تا خانه های پشت بندها باشد وپشت بندها را به طلا بپوشان.
\par 30 «پس مسکن را برپا کن موافق نمونه‌ای که در کوه به تو نشان داده شد.
\par 31 و حجابی ازلاجورد و ارغوان و قرمز و کتان نازک تابیده شده بساز، از صنعت نساج ماهر با کروبیان ساخته شود.
\par 32 و آن را بر چهار ستون چوب شطیم پوشیده شده به طلا بگذار، و قلابهای آنها از طلاباشد و بر چهار پایه نقره قایم شود.
\par 33 و حجاب را زیر تکمه‌ها آویزان کن، و تابوت شهادت را درآنجا به اندرون حجاب بیاور، و حجاب، قدس رابرای شما از قدس‌الاقداس جدا خواهد کرد.
\par 34 وتخت رحمت را بر تابوت شهادت درقدس‌الاقداس بگذار.
\par 35 و خوان را بیرون حجاب و چراغدان را برابر خوان به طرف جنوبی مسکن بگذار، و خوان را به طرف شمالی آن برپا کن.
\par 36 و پرده‌ای برای دروازه مسکن از لاجورد و ارغوان وقرمز و کتان نازک تابیده شده از صنعت طرازبساز.و برای پرده پنج ستون از چوب شطیم بساز، و آنها را به طلا بپوشان، و قلابهای آنها ازطلا باشد، و برای آنها پنج پایه برنجین بریز.
\par 37 و برای پرده پنج ستون از چوب شطیم بساز، و آنها را به طلا بپوشان، و قلابهای آنها ازطلا باشد، و برای آنها پنج پایه برنجین بریز.
 
\chapter{27}

\par 1 «و مذبح را از چوب شطیم بساز، طولش پنج ذراع و عرضش پنج ذراع. ومذبح مربع باشد. و بلندی‌اش سه ذراع.
\par 2 وشاخه هایش را بر چهار گوشه‌اش بساز وشاخه هایش از همان باشد و آن را به برنج بپوشان.
\par 3 و لگنهایش را برای برداشتن خاکسترش بساز. وخاک اندازهایش و جامهایش و چنگالهایش ومجمرهایش و همه اسبابش را از برنج بساز.
\par 4 وبرایش آتش دانی مشبک برنجین بساز و بر آن شبکه چهار حلقه برنجین بر چهار گوشه‌اش بساز.
\par 5 و آن را در زیر، کنار مذبح بگذار تا شبکه به نصف مذبح برسد.
\par 6 و دو عصا برای مذبح بساز. عصاها از چوب شطیم باشد و آنها را به برنج بپوشان.
\par 7 و عصاها را در حلقه‌ها بگذرانند، وعصاها بر هر دو طرف مذبح باشد تا آن را بردارند.
\par 8 و آن را مجوف از تخته‌ها بساز، همچنانکه درکوه به تو نشان داده شد، به این طور ساخته شود.
\par 9 «و صحن مسکن را بساز به طرف جنوب به سمت یمانی. پرده های صحن از کتان نازک تابیده شده باشد، و طولش صد ذراع به یک طرف.
\par 10 وستونهایش بیست و پایه های آنها بیست، از برنج باشد و قلابهای ستونها و پشت بندهای آنها ازنقره باشد.
\par 11 و همچنین به طرف شمال، درطولش پرده‌ها باشد که طول آنها صد ذراع باشد وبیست ستون آن و بیست پایه آنها از برنج باشد وقلابهای ستونها و پشت بندهای آنها از نقره باشد.
\par 12 و برای عرض صحن به سمت مغرب، پرده های پنجاه ذراعی باشد. و ستونهای آنها ده وپایه های آنها ده.
\par 13 و عرض صحن به‌جانب مشرق از سمت طلوع پنجاه ذراع باشد.
\par 14 وپرده های یک طرف دروازه، پانزده ذراع وستونهای آنها سه و پایه های آنها سه.
\par 15 وپرده های طرف دیگر پانزده ذراعی و ستونهای آنها سه و پایه های آنها سه.
\par 16 و برای دروازه صحن، پرده بیست ذراعی از لاجورد و ارغوان وقرمز و کتان نازک تابیده شده از صنعت طرازباشد. و ستونهایش چهار و پایه هایش چهار.
\par 17 همه ستونهای گرداگرد صحن با پشت بندهای نقره پیوسته شود، و قلابهای آنها از نقره وپایه های آنها از برنج باشد.
\par 18 طول صحن صدذراع، و عرضش در هر جا پنجاه ذراع، وبلندی‌اش پنج ذراع از کتان نازک تابیده شده، وپایه هایش از برنج باشد.
\par 19 و همه اسباب مسکن برای هر خدمتی، و همه میخهایش، و همه میخهای صحن از برنج باشد.
\par 20 «و تو بنی‌اسرائیل را امر فرما که روغن زیتون مصفی و کوبیده شده برای روشنایی نزد توبیاورند تا چراغها دائم روشن شود.در خیمه اجتماع، بیرون پرده‌ای که در برابر شهادت است، هارون و پسرانش از شام تا صبح، به حضورخداوند آن را درست کنند. و این برای بنی‌اسرائیل نسلا بعد نسل فریضه ابدی باشد.
\par 21 در خیمه اجتماع، بیرون پرده‌ای که در برابر شهادت است، هارون و پسرانش از شام تا صبح، به حضورخداوند آن را درست کنند. و این برای بنی‌اسرائیل نسلا بعد نسل فریضه ابدی باشد.
 
\chapter{28}

\par 1 «و تو برادر خود، هارون و پسرانش را با وی از میان بنی‌اسرائیل نزد خودبیاور تا برای من کهانت بکند، یعنی هارون وناداب و ابیهو و العازار و ایتامار، پسران هارون.
\par 2 ورختهای مقدس برای برادرت، هارون، بجهت عزت و زینت بساز.
\par 3 و تو به جمیع دانادلانی که ایشان را به روح حکمت پر ساخته‌ام، بگو که رختهای هارون را بسازند برای تقدیس کردن او تابرای من کهانت کند.
\par 4 و رختهایی که می‌سازنداین است: سینه بند و ایفود و ردا و پیراهن مطرز وعمامه و کمربند. این رختهای مقدس را برای برادرت هارون و پسرانش بسازند تا بجهت من کهانت کند.
\par 5 و ایشان طلا و لاجورد و ارغوان وقرمز و کتان نازک را بگیرند،
\par 6 «و ایفود را از طلا و لاجورد و ارغوان و قرمزو کتان نازک تابیده شده، از صنعت نساج ماهربسازند.
\par 7 و دو کتفش را بر دو کناره‌اش بپیوندند تاپیوسته شود.
\par 8 و زنار ایفود که برآن است، ازهمان صنعت و از همان پارچه باشد، یعنی از طلاو لاجورد و ارغوان و قرمز و کتان نازک تابیده شده.
\par 9 و دو سنگ جزع بگیر و نامهای بنی‌اسرائیل را بر آنها نقش کن.
\par 10 شش نام ایشان را بر یک سنگ و شش نام باقی ایشان را بر سنگ دیگر موافق تولد ایشان.
\par 11 از صنعت نقاش سنگ مثل نقش خاتم نامهای بنی‌اسرائیل را بر هردو سنگ نقش نما و آنها را در طوقهای طلا نصب کن.
\par 12 و آن دو سنگ را بر کتفهای ایفود بگذار تاسنگهای یادگاری برای بنی‌اسرائیل باشد، وهارون نامهای ایشان را بر دو کتف خود، بحضور خداوند برای یادگاری بردارد.
\par 13 و دو طوق ازطلا بساز.
\par 14 و دو زنجیر از طلای خالص بسازمثل طناب بهم پیچیده شده، و آن دو زنجیر بهم پیچیده شده را در طوقها بگذار.
\par 15 «و سینه بند عدالت را از صنعت نساج ماهر، موافق کار ایفود بساز و آن را از طلا و لاجورد وارغوان و قرمز و کتان نازک تابیده شده بساز.
\par 16 ومربع و دولا باشد، طولش یک وجب و عرضش یک وجب.
\par 17 و آن را به ترصیع سنگها، یعنی به چهار رسته از سنگها مرصع کن که رسته اول عقیق احمر و یاقوت اصفر و زمرد باشد،
\par 18 و رسته دوم بهرمان و یاقوت کبود و عقیق سفید،
\par 19 ورسته سوم عین الهر و یشم و جمشت،
\par 20 و رسته چهارم زبرجد و جزع و یشب. و آنها دررسته های خود با طلا نشانده شود.
\par 21 و سنگهاموافق نامهای بنی‌اسرائیل مطابق اسامی ایشان، دوازده باشد، مثل نقش خاتم، و هر یک برای دوازده سبط موافق اسمش باشد.
\par 22 و بر سینه بند، زنجیرهای بهم پیچیده شده، مثل طناب از طلای خالص بساز.
\par 23 و بر سینه بند، دو حلقه از طلابساز و آن دو حلقه را بر دو طرف سینه بند بگذار. 
\par 24 و آن دو زنجیر طلا را بر آن دو حلقه‌ای که برسر سینه بند است بگذار.
\par 25 و دو سر دیگر آن دوزنجیر را در آن دو طوق ببند و بر دو کتف ایفودبطرف پیش بگذار.
\par 26 و دو حلقه زرین بساز و آنهارا بر دو سر سینه بند، به کنار آن که بطرف اندرون ایفود است، بگذار.
\par 27 و دو حلقه دیگر زرین بسازو آنها را بر دو کتف ایفود از پایین بجانب پیش، دربرابر پیوستگی آن، بر زبر زنار ایفود بگذار.
\par 28 وسینه بند را به حلقه هایش بر حلقه های ایفود به نوار لاجورد ببندند تا بالای زنار ایفود باشد و تاسینه بند از ایفود جدا نشود.
\par 29 و هارون نامهای بنی‌اسرائیل را بر سینه بند عدالت بر دل خود، وقتی که به قدس داخل شود، به حضور خداوندبجهت یادگاری دائم بردارد.
\par 30 و اوریم و تمیم را در سینه بند عدالت بگذار تا بر دل هارون باشد، وقتی که به حضور خداوند بیاید، و عدالت بنی‌اسرائیل را بر دل خود بحضور خداوند دائم متحمل شود.
\par 31 «و ردای ایفود را تمام از لاجورد بساز.
\par 32 و شکافی برای سر، در وسطش باشد. و حاشیه گرداگرد شکافش از کار نساج مثل گریبان زره، تادریده نشود.
\par 33 و در دامنش، انارها بساز ازلاجورد و ارغوان و قرمز، گرداگرد دامنش، وزنگوله های زرین در میان آنها به هر طرف.
\par 34 زنگوله زرین و اناری و زنگوله زرین و اناری گرداگرد دامن ردا.
\par 35 و در بر هارون باشد، هنگامی که خدمت می‌کند، تا آواز آنها شنیده شود، هنگامی که در قدس بحضور خداوند داخل می‌شود و هنگامی که بیرون می‌آید تا نمیرد.
\par 36 وتنکه از طلای خالص بساز و بر آن مثل نقش خاتم قدوسیت برای یهوه نقش کن.
\par 37 و آن را به نوارلاجوردی ببند تا بر عمامه باشد، بر پیشانی عمامه خواهد بود.
\par 38 و بر پیشانی هارون باشد تا هارون گناه موقوفاتی که بنی‌اسرائیل وقف می‌نمایند، درهمه هدایای مقدس ایشان متحمل شود. و آن دائم بر پیشانی او باشد تا بحضور خداوند مقبول شوند.
\par 39 و پیراهن کتان نازک را بباف و عمامه‌ای از کتان نازک بساز و کمربندی از صنعت طرازبساز.
\par 40 و برای پسران هارون پیراهنها بساز و بجهت ایشان کمربندها بساز و برای ایشان عمامه‌ها بساز بجهت عزت و زینت.
\par 41 و برادرخود هارون و پسرانش را همراه او به آنها آراسته کن، و ایشان را مسح کن و ایشان را تخصیص وتقدیس نما تا برای من کهانت کنند.
\par 42 و زیرجامه های کتان برای ستر عورت ایشان بساز که ازکمر تا ران برسد،و بر هارون و پسرانش باشد، هنگامی که به خیمه اجتماع داخل شوند یا نزدمذبح آیند تا در قدس خدمت نمایند، مبادامتحمل گناه شوند و بمیرند. این برای وی و بعد ازاو برای ذریتش فریضه ابدی است.
\par 43 و بر هارون و پسرانش باشد، هنگامی که به خیمه اجتماع داخل شوند یا نزدمذبح آیند تا در قدس خدمت نمایند، مبادامتحمل گناه شوند و بمیرند. این برای وی و بعد ازاو برای ذریتش فریضه ابدی است.
 
\chapter{29}

\par 1 «و این است کاری که بدیشان می‌کنی، برای تقدیس نمودن ایشان تا بجهت من کهانت کنند: یک گوساله و دو قوچ بی‌عیب بگیر،
\par 2 و نان فطیر و قرصهای فطیر سرشته به روغن ورقیقهای فطیر مسح شده به روغن. آنها را از آردنرم گندم بساز.
\par 3 و آنها را در یک سبد بگذار و آنهارا در سبد با گوساله و دو قوچ بگذران.
\par 4 و هارون و پسرانش را نزد دروازه خیمه اجتماع آورده، ایشان را به آب غسل ده،
\par 5 و آن رختها را گرفته، هارون را به پیراهن و ردای ایفود و ایفود وسینه بند آراسته کن و زنار ایفود را بر وی ببند.
\par 6 وعمامه را بر سرش بنه و افسر قدوسیت را برعمامه بگذار،
\par 7 و روغن مسح را گرفته، برسرش بریز و او را مسح کن.
\par 8 و پسرانش را نزدیک آورده، ایشان را به پیراهنها بپوشان.
\par 9 و بر ایشان، یعنی هارون و پسرانش کمربندها ببند و عمامه هارا بر ایشان بگذار و کهانت برای ایشان فریضه ابدی خواهد بود. پس هارون و پسرانش راتخصیص نما.
\par 10 و گوساله را پیش خیمه اجتماع برسان، و هارون و پسرانش دستهای خود را بر سرگوساله بگذارند.
\par 11 و گوساله را به حضورخداوند نزد در خیمه اجتماع ذبح کن.
\par 12 و ازخون گوساله گرفته، بر شاخهای مذبح به انگشت خود بگذار، و باقی خون را بر بنیان مذبح بریز.
\par 13 و همه پیه را که احشا را می‌پوشاند، و سفیدی که بر جگر است، و دو گرده را با پیهی که برآنهاست، گرفته، بر مذبح بسوزان.
\par 14 اما گوشت گوساله را و پوست و سرگینش را بیرون از اردو به آتش بسوزان، زیرا قربانی گناه است.
\par 15 «و یک قوچ بگیر و هارون و پسران، دستهای خود را بر سر قوچ بگذارند.
\par 16 و قوچ راذبح کرده، خونش را بگیر و گرداگرد مذبح بپاش.
\par 17 و قوچ را به قطعه هایش ببر، و احشا وپاچه هایش را بشوی، و آنها را بر قطعه‌ها و سرش بنه.
\par 18 و تمام قوچ را بر مذبح بسوزان، زیرا برای خداوند قربانی سوختنی است، و عطر خوشبو، وقربانی آتشین برای خداوند است.
\par 19 پس قوچ دوم را بگیر و هارون و پسرانش دستهای خود رابر سر قوچ بگذارند.
\par 20 و قوچ را ذبح کرده، ازخونش بگیر و به نرمه گوش راست هارون، و به نرمه گوش پسرانش، و به شست دست راست ایشان، و به شست پای راست ایشان، بگذار، وباقی خون را گرداگرد مذبح بپاش.
\par 21 و از خونی که بر مذبح است، و از روغن مسح گرفته، آن را برهارون و رخت وی و بر پسرانش و رخت پسرانش با وی بپاش، تا او و رختش و پسرانش ورخت پسرانش با وی تقدیس شوند.
\par 22 پس پیه قوچ را، و دنبه و پیهی که احشا را می‌پوشاند، وسفیدی جگر، و دو گرده و پیهی که بر آنهاست، وساق راست را بگیر، زیرا که قوچ، قربانی تخصیص است.
\par 23 و یک گرده نان و یک قرص نان روغنی، و یک رقیق از سبد نان فطیر را که به حضور خداوند است،
\par 24 و این همه را به‌دست هارون و به‌دست پسرانش بنه، و آنها را برای هدیه جنبانیدنی به حضور خداوند بجنبان.
\par 25 وآنها را از دست ایشان گرفته، برای قربانی سوختنی بر مذبح بسوزان، تا برای خداوند عطرخوشبو باشد، زیرا که این قربانی آتشین خداونداست.
\par 26 و سینه قوچ قربانی تخصیص را که برای هارون است گرفته، آن را برای هدیه جنبانیدنی به حضور خداوند بجنبان. و آن حصه تو می‌باشد.
\par 27 و سینه جنبانیدنی و ساق رفیعه را که از قوچ قربانی تخصیص هارون و پسرانش جنبانیده، وبرداشته شد، تقدیس نمای.
\par 28 و آن از آن هارون و پسرانش از جانب بنی‌اسرائیل به فریضه ابدی خواهد بود، زیرا که هدیه رفیعه است و هدیه رفیعه از جانب بنی‌اسرائیل از قربانی های سلامتی ایشان برای خداوند خواهد بود.
\par 29 ورخت مقدس هارون بعد از او، از آن پسرانش خواهد بود، تا در آنها مسح و تخصیص شوند.
\par 30 هفت روز، آن کاهن که جانشین او می‌باشد ازپسرانش و به خیمه اجتماع داخل شده، خدمت قدس را می‌کند، آنها را بپوشد.
\par 31 و قوچ قربانی تخصیص را گرفته، گوشتش را در قدس آب پزکن.
\par 32 و هارون و پسرانش گوشت قوچ را با نانی که در سبد است، به در خیمه اجتماع بخورند.
\par 33 و آنانی که برای تخصیص و تقدیس خود بدین چیزها کفاره کرده شدند، آنها را بخورند، لیکن شخص اجنبی نخورد زیرا که مقدس است.
\par 34 و اگر چیزی از گوشت هدیه تخصیص و از نان، تاصبح باقی ماند، آن باقی را به آتش بسوزان، و آن را نخورند، زیرا که مقدس است.
\par 35 «همچنان به هارون و پسرانش عمل نما، موافق آنچه به تو امر فرموده‌ام، هفت روز ایشان را تخصیص نما.
\par 36 و گوساله قربانی گناه را هرروز بجهت کفاره ذبح کن.
\par 37 و مذبح را طاهر سازبه کفاره‌ای که بر آن می‌کنی، و آن را مسح کن تامقدس شود.
\par 38 هفت روز برای مذبح کفاره کن، وآن را مقدس ساز، و مذبح، قدس اقداس خواهدبود. هر‌که مذبح را لمس کند، مقدس باشد.
\par 39 واین است قربانی هایی که بر مذبح باید گذرانید: دوبره یکساله. هر روز پیوسته
\par 40 یک بره را در صبح ذبح کن، و بره دیگر را در عصر ذبح نما.
\par 41 و ده‌یک از آرد نرم سرشته شده با یک ربع هین روغن کوبیده، و برای هدیه ریختنی، یک ربع هین شراب برای هر بره خواهد بود.
\par 42 و بره دیگر رادر عصر ذبح کن و برای آن موافق هدیه صبح وموافق هدیه ریختنی آن بگذران، تا عطر خوشبوو قربانی آتشین برای خداوند باشد.
\par 43 این قربانی سوختنی دائمی، در نسلهای شما نزد دروازه خیمه اجتماع خواهد بود، به حضور خداوند، درجایی که با شما ملاقات می‌کنم تا آنجا به تو سخن گویم.
\par 44 و در آنجا با بنی‌اسرائیل ملاقات می‌کنم، تا از جلال من مقدس شود.
\par 45 و خیمه اجتماع و مذبح را مقدس می‌کنم، و هارون وپسرانش را تقدیس می‌کنم تا برای من کهانت کنند.و در میان بنی‌اسرائیل ساکن شده، خدای ایشان می‌باشم. و خواهند دانست که من یهوه، خدای ایشان هستم، که ایشان را از زمین مصر بیرون آورده‌ام، تا در میان ایشان ساکن شوم. من یهوه خدای ایشان هستم.
\par 46 و در میان بنی‌اسرائیل ساکن شده، خدای ایشان می‌باشم. و خواهند دانست که من یهوه، خدای ایشان هستم، که ایشان را از زمین مصر بیرون آورده‌ام، تا در میان ایشان ساکن شوم. من یهوه خدای ایشان هستم.
 
\chapter{30}

\par 1 «و مذبحی برای سوزانیدن بخوربساز. آن را از چوب شطیم بساز.
\par 2 طولش ذراعی باشد، و عرضش ذراعی، یعنی مربع باشد، و بلندی‌اش دو ذراع، و شاخهایش ازخودش باشد.
\par 3 و آن را به طلای خالص بپوشان. سطحش و جانبهایش به هر طرف و شاخهایش راو تاجی از طلا گرداگردش بساز.
\par 4 و دو حلقه زرین برایش در زیر تاجش بساز، بر دو گوشه‌اش، بر هر دو طرفش، آنها را بساز. و آنها خانه‌ها باشدبرای عصاها، تا آن را بدانها بردارند.
\par 5 و عصاها رااز چوب شطیم بساز و آنها را به طلا بپوشان.
\par 6 وآن را پیش حجابی که روبروی تابوت شهادت است، در مقابل کرسی رحمت که بر زبر شهادت است، در جایی که با تو ملاقات می‌کنم، بگذار.
\par 7 و هر بامداد هارون بخور معطر بر روی آن بسوزاند، وقتی که چراغها را می‌آراید، آن رابسوزاند.
\par 8 و در عصر چون هارون چراغها راروشن می‌کند، آن را بسوزاند تا بخور دائمی به حضور خداوند در نسلهای شما باشد.
\par 9 هیچ بخور غریب و قربانی سوختنی و هدیه‌ای بر آن مگذرانید، و هدیه ریختنی بر آن مریزید.
\par 10 وهارون سالی یک مرتبه بر شاخهایش کفاره کند، به خون قربانی گناه که برای کفاره است، سالی یک مرتبه بر آن کفاره کند در نسلهای شما، زیرا که آن برای خداوند قدس اقداس است.»
\par 11 و خداوند به موسی خطاب کرده، گفت:
\par 12 «وقتی که شماره بنی‌اسرائیل را برحسب شمرده شدگان ایشان می‌گیری، آنگاه هرکس فدیه جان خود را به خداوند دهد، هنگامی که ایشان را می‌شماری، مبادا در حین شمردن ایشان، وبایی در ایشان حادث شود.
\par 13 هر‌که به سوی شمرده شدگان می‌گذرد، این را بدهد، یعنی نیم مثقال موافق مثقال قدس، که یک مثقال بیست قیراط است، و این نیم مثقال هدیه خداوند است.
\par 14 هرکس از بیست ساله و بالاتر که به سوی شمرده شدگان بگذرد، هدیه خداوند را بدهد.
\par 15 دولتمند از نیم مثقال زیاده ندهد، و فقیر کمترندهد، هنگامی که هدیه‌ای برای کفاره جانهای خود به خداوند می‌دهند.
\par 16 و نقد کفاره را ازبنی‌اسرائیل گرفته، آن را برای خدمت خیمه اجتماع بده، تا برای بنی‌اسرائیل یادگاری به حضور خداوند باشد، و بجهت جانهای ایشان کفاره کند.»
\par 17 و خداوند به موسی خطاب کرده، گفت:
\par 18 «حوضی نیز برای شستن از برنج بساز، وپایه‌اش از برنج و آن را در میان خیمه اجتماع ومذبح بگذار، و آب در آن بریز.
\par 19 و هارون وپسرانش دست و پای خود را از آن بشویند.
\par 20 هنگامی که به خیمه اجتماع داخل شوند، به آب بشویند، مبادا بمیرند. و وقتی که برای خدمت کردن و سوزانیدن قربانی های آتشین بجهت خداوند به مذبح نزدیک آیند،
\par 21 آنگاه دست و پای خود را بشویند، مبادا بمیرند. و این برای ایشان، یعنی برای او و ذریتش، نسلا بعدنسل فریضه ابدی باشد.» 
\par 22 و خداوند موسی را خطاب کرده، گفت:
\par 23 «و تو عطریات خاص بگیر، از مر چکیده پانصد مثقال، و از دارچینی معطر، نصف آن، دویست و پنجاه مثقال، و از قصب الذریره، دویست و پنجاه مثقال.
\par 24 و از سلیخه پانصدمثقال موافق مثقال قدس، و از روغن زیتون یک هین.
\par 25 و از آنها روغن مسح مقدس را بساز، عطری که از صنعت عطار ساخته شود، تا روغن مسح مقدس باشد.
\par 26 و خیمه اجتماع و تابوت شهادت را بدان مسح کن.
\par 27 و خوان را با تمامی اسبابش، و چراغدان را با اسبابش، و مذبح بخوررا،
\par 28 و مذبح قربانی سوختنی را با همه اسبابش، و حوض را با پایه‌اش.
\par 29 و آنها را تقدیس نما، تاقدس اقداس باشد. هر‌که آنها را لمس نمایدمقدس باشد.
\par 30 و هارون و پسرانش را مسح نموده، ایشان را تقدیس نما، تا برای من کهانت کنند.
\par 31 و بنی‌اسرائیل را خطاب کرده، بگو: این است روغن مسح مقدس برای من در نسلهای شما.
\par 32 و بر بدن انسان ریخته نشود، و مثل آن موافق ترکیبش مسازید، زیرا که مقدس است ونزد شما مقدس خواهد بود.
\par 33 هر‌که مثل آن ترکیب نماید، و هر‌که چیزی از آن بر شخصی بیگانه بمالد، از قوم خود منقطع شود.»
\par 34 و خداوند به موسی گفت: «عطریات بگیر، یعنی میعه و اظفار و قنه و از این عطریات با کندرصاف حصه‌ها مساوی باشد.
\par 35 و از اینها بخوربساز، عطری از صنعت عطار نمکین و مصفی و مقدس.
\par 36 و قدری از آن را نرم بکوب، و آن راپیش شهادت در خیمه اجتماع، جایی که با توملاقات می‌کنم بگذار، و نزد شما قدس اقداس باشد.
\par 37 و موافق ترکیب این بخور که می‌سازی، دیگری برای خود مسازید؛ نزد توبرای خداوند مقدس باشد.هر‌که مثل آن رابرای بوییدن بسازد، از قوم خود منقطع شود.»
\par 38 هر‌که مثل آن رابرای بوییدن بسازد، از قوم خود منقطع شود.»
 
\chapter{31}

\par 1 و خداوند موسی را خطاب کرده، گفت:
\par 2 «آگاه باش بصلئیل بن اوری بن حور را از سبط یهودا به نام خوانده‌ام.
\par 3 و او را به روح خدا پر ساخته‌ام، و به حکمت و فهم ومعرفت و هر هنری،
\par 4 برای اختراع مخترعات، تادر طلا و نقره و برنج کار کند.
\par 5 و برای تراشیدن سنگ و ترصیع آن و درودگری چوب، تا در هرصنعتی اشتغال نماید.
\par 6 و اینک من، اهولیاب بن اخیسامک را از سبط دان، انباز او ساخته‌ام، و دردل همه دانادلان حکمت بخشیده‌ام، تا آنچه را به تو امر فرموده‌ام، بسازند.
\par 7 خیمه اجتماع و تابوت شهادت و کرسی رحمت که بر آن است، و تمامی اسباب خیمه،
\par 8 و خوان و اسبابش و چراغدان طاهر و همه اسبابش و مذبح بخور،
\par 9 و مذبح قربانی سوختنی و همه اسبابش، و حوض وپایه‌اش،
\par 10 و لباس خدمت و لباس مقدس برای هارون کاهن، و لباس پسرانش بجهت کهانت،
\par 11 و روغن و مسح و بخور معطر بجهت قدس، موافق آنچه به تو امر فرموده‌ام، بسازند.»
\par 12 و خداوند موسی را خطاب کرده، گفت:
\par 13 «و تو بنی‌اسرائیل را مخاطب ساخته، بگو: البته سبت های مرا نگاه دارید. زیرا که این در میان من وشما در نسلهای شما آیتی خواهد بود تا بدانید که من یهوه هستم که شما را تقدیس می‌کنم.
\par 14 پس سبت را نگاه دارید، زیرا که برای شما مقدس است، هر‌که آن را بی‌حرمت کند، هرآینه کشته شود، و هر‌که در آن کار کند، آن شخص از میان قوم خود منقطع شود.
\par 15 شش روز کار کرده شود، و در روز هفتم سبت آرام و مقدس خداونداست. هر‌که در روز سبت کار کند، هرآینه کشته شود.
\par 16 پس بنی‌اسرائیل سبت را نگاه بدارند، نسلا بعد نسل سبت را به عهد ابدی مرعی دارند.
\par 17 این در میان من و بنی‌اسرائیل آیتی ابدی است، زیرا که در شش روز، خداوند آسمان و زمین راساخت و در روز هفتمین آرام فرموده، استراحت یافت.»و چون گفتگو را با موسی در کوه سینابپایان برد، دو لوح شهادت، یعنی دو لوح سنگ مرقوم به انگشت خدا را به وی داد.
\par 18 و چون گفتگو را با موسی در کوه سینابپایان برد، دو لوح شهادت، یعنی دو لوح سنگ مرقوم به انگشت خدا را به وی داد.
 
\chapter{32}

\par 1 و چون قوم دیدند که موسی در فرودآمدن از کوه تاخیر نمود، قوم نزدهارون جمع شده، وی را گفتند: «برخیز و برای ماخدایان بساز که پیش روی ما بخرامند، زیرا این مرد، موسی، که ما را از زمین مصر بیرون آورد، نمی دانیم او را چه شده است.»
\par 2 هارون بدیشان گفت: «گوشواره های طلا را که در گوش زنان و پسران و دختران شماست، بیرون کرده، نزد من بیاورید.»
\par 3 پس تمامی گوشواره های زرین را که در گوشهای ایشان بودبیرون کرده، نزد هارون آوردند.
\par 4 و آنها را ازدست ایشان گرفته، آن را با قلم نقش کرد، و از آن گوساله ریخته شده ساخت، و ایشان گفتند: «ای اسرائیل این خدایان تو می‌باشند، که تو را از زمین مصر بیرون آوردند.»
\par 5 و چون هارون این را بدید، مذبحی پیش آن بنا کرد و هارون ندا درداده، گفت: «فردا عید یهوه می‌باشد.»
\par 6 و بامدادان برخاسته، قربانی های سوختنی گذرانیدند، و هدایای سلامتی آوردند، و قوم برای خوردن و نوشیدن نشستند، و بجهت لعب برپا شدند.
\par 7 و خداوند به موسی گفت: «روانه شده، بزیر برو، زیرا که این قوم تو که از زمین مصر بیرون آورده‌ای، فاسدشده‌اند.
\par 8 و به زودی از آن طریقی که بدیشان امرفرموده‌ام، انحراف ورزیده، گوساله ریخته شده برای خویشتن ساخته‌اند، و نزد آن سجده کرده، وقربانی گذرانیده، می‌گویند که‌ای اسرائیل این خدایان تو می‌باشند که تو را از زمین مصر بیرون آورده‌اند.»
\par 9 و خداوند به موسی گفت: «این قوم را دیده‌ام و اینک قوم گردنکش می‌باشند.
\par 10 واکنون مرا بگذار تا خشم من بر ایشان مشتعل شده، ایشان را هلاک کنم و تو را قوم عظیم خواهم ساخت.»
\par 11 پس موسی نزد یهوه، خدای خود تضرع کرده، گفت: «ای خداوند چرا خشم تو بر قوم خود که با قوت عظیم و دست زورآور اززمین مصر بیرون آورده‌ای، مشتعل شده است؟
\par 12 چرا مصریان این سخن گویند که ایشان را برای بدی بیرون آورد، تا ایشان را در کوهها بکشد، و ازروی زمین تلف کند؟ پس از شدت خشم خودبرگرد، و از این قصد بدی قوم خویش رجوع فرما.
\par 13 بندگان خود ابراهیم و اسحاق و اسرائیل را بیادآور، که برای ایشان به ذات خود قسم خورده، بدیشان گفتی که ذریت شما را مثل ستارگان آسمان کثیر گردانم، و تمامی این زمین را که درباره آن سخن گفته‌ام به ذریت شما بخشم، تا آن را متصرف شوند تا ابدالاباد.»
\par 14 پس خداوند ازآن بدی که گفته بود که به قوم خود برساند، رجوع فرمود.
\par 15 آنگاه موسی برگشته، از کوه به زیر آمد، ودو لوح شهادت به‌دست وی بود، و لوحها به هردو طرف نوشته بود، بدین طرف و بدان طرف مرقوم بود.
\par 16 و لوح‌ها صنعت خدا بود، و نوشته، نوشته خدا بود، منقوش بر لوح‌ها.
\par 17 و چون یوشع آواز قوم را که می‌خروشیدند شنید، به موسی گفت: «در اردو صدای جنگ است.»
\par 18 گفت: «صدای خروش ظفر نیست، و صدای خروش شکست نیست، بلکه آواز مغنیان را من می‌شنوم.»
\par 19 و واقع شد که چون نزدیک به اردو رسید، وگوساله و رقص کنندگان را دید، خشم موسی مشتعل شد، و لوحها را از دست خود افکنده، آنها را زیر کوه شکست.
\par 20 و گوساله‌ای را که ساخته بودند گرفته، به آتش سوزانید، و آن راخرد کرده، نرم ساخت، و بر روی آب پاشیده، بنی‌اسرائیل را نوشانید.
\par 21 و موسی به هارون گفت: «این قوم به تو چه کرده بودند که گناه عظیمی بر ایشان آوردی؟»
\par 22 هارون گفت: «خشم آقایم افروخته نشود، تو این قوم رامی شناسی که مایل به بدی می‌باشند.
\par 23 و به من گفتند، برای ما خدایان بساز که پیش روی مابخرامند، زیرا که این مرد، موسی، که ما را از زمین مصر بیرون آورده است، نمی دانیم او را چه شده.
\par 24 بدیشان گفتم هر‌که را طلا باشد آن را بیرون کند، پس به من دادند، و آن را در آتش انداختم واین گوساله بیرون آمد.»
\par 25 و چون موسی قوم رادید که بی‌لگام شده‌اند، زیرا که هارون ایشان رابرای رسوایی ایشان در میان دشمنان ایشان بی‌لگام ساخته بود،
\par 26 آنگاه موسی به دروازه اردو ایستاده، گفت: «هر‌که به طرف خداوندباشد، نزد من آید.» پس جمیع بنی لاوی نزد وی جمع شدند.
\par 27 او بدیشان گفت: «یهوه، خدای اسرائیل، چنین می‌گوید: هر کس شمشیر خود رابر ران خویش بگذارد، و از دروازه تا دروازه اردوآمد و رفت کند، و هر کس برادر خود و دوست خویش و همسایه خود را بکشد.»
\par 28 و بنی لاوی موافق سخن موسی کردند. و در آن روز قریب سه هزار نفر از قوم افتادند.
\par 29 و موسی گفت: «امروزخویشتن را برای خداوند تخصیص نمایید. حتی هر کس به پسر خود و به برادر خویش، تا امروزشما را برکت دهد.»
\par 30 و بامدادان واقع شد که موسی به قوم گفت: «شما گناهی عظیم کرده‌اید. اکنون نزد خداوندبالا می‌روم، شاید گناه شما را کفاره کنم.»
\par 31 پس موسی به حضور خداوند برگشت و گفت: «آه، این قوم گناهی عظیم کرده، و خدایان طلا برای خویشتن ساخته‌اند.
\par 32 الان هرگاه گناه ایشان رامی آمرزی و اگرنه مرا از دفترت که نوشته‌ای، محو ساز.»
\par 33 خداوند به موسی گفت: «هر‌که گناه کرده است، او را از دفتر خود محو سازم.
\par 34 واکنون برو و این قوم را بدانجایی که به تو گفته‌ام، راهنمایی کن. اینک فرشته من پیش روی توخواهد خرامید، لیکن در یوم تفقد من، گناه ایشان را از ایشان بازخواست خواهم کرد.»وخداوند قوم را مبتلا ساخت زیرا گوساله‌ای را که هارون ساخته بود، ساخته بودند.
\par 35 وخداوند قوم را مبتلا ساخت زیرا گوساله‌ای را که هارون ساخته بود، ساخته بودند.
 
\chapter{33}

\par 1 و خداوند به موسی گفت: «روانه شده، از اینجا کوچ کن، تو و این قوم که اززمین مصر برآورده‌ای، بدان زمینی که برای ابراهیم، اسحاق و یعقوب قسم خورده، گفته‌ام آن را به ذریت تو عطا خواهم کرد.
\par 2 و فرشته‌ای پیش روی تو می‌فرستم، و کنعانیان و اموریان و حتیان و فرزیان و حویان و یبوسیان را بیرون خواهم کرد.
\par 3 به زمینی که به شیر و شهد جاری است، زیرا که در میان شما نمی آیم، چونکه قوم گردن کش هستی، مبادا تو را در بین راه هلاک سازم.»
\par 4 وچون قوم این سخنان بد را شنیدند، ماتم گرفتند، و هیچکس زیور خود را برخود ننهاد.
\par 5 وخداوند به موسی گفت: «بنی‌اسرائیل را بگو: شماقوم گردن کش هستید؛ اگر لحظه‌ای در میان توآیم، همانا تو را هلاک سازم. پس اکنون زیورخود را از خود بیرون کن تا بدانم با تو چه کنم.»
\par 6 پس بنی‌اسرائیل زیورهای خود را از جبل حوریب از خود بیرون کردند.
\par 7 و موسی خیمه خود را برداشته، آن را بیرون لشکرگاه، دور از اردو زد، و آن را «خیمه اجتماع» نامید. و واقع شد که هر‌که طالب یهوه می‌بود، به خیمه اجتماع که خارج لشکرگاه بود، بیرون می‌رفت.
\par 8 و هنگامی که موسی به سوی خیمه بیرون می‌رفت، تمامی قوم برخاسته، هر یکی به در خیمه خود می‌ایستاد، و در عقب موسی می‌نگریست تا داخل خیمه می‌شد.
\par 9 و چون موسی به خیمه داخل می‌شد، ستون ابر نازل شده، به در خیمه می‌ایستاد، و خدا با موسی سخن می‌گفت.
\par 10 و چون تمامی قوم، ستون ابر رابر در خیمه ایستاده می‌دیدند، همه قوم برخاسته، هر کس به در خیمه خود سجده می‌کرد.
\par 11 وخداوند با موسی روبرو سخن می‌گفت، مثل شخصی که با دوست خود سخن گوید. پس به اردو بر می‌گشت. اما خادم او یوشع بن نون جوان، از میان خیمه بیرون نمی آمد.
\par 12 و موسی به خداوند گفت: «اینک تو به من می‌گویی: این قوم را ببر. و تو مرا خبر نمی دهی که همراه من که را می‌فرستی. و تو گفته‌ای، تو را به نام می‌شناسم، و ایض در حضور من فیض یافته‌ای.
\par 13 الان اگر فی الحقیقه منظور نظر توشده‌ام، طریق خود را به من بیاموز تا تو رابشناسم، و در حضور تو فیض یابم، و ملاحظه بفرما که این طایفه، قوم تو می‌باشند.»
\par 14 گفت: «روی من خواهد آمد و تو را آرامی خواهم بخشید.»
\par 15 به وی عرض کرد: «هر گاه روی تونیاید، ما را از اینجا مبر. 
\par 16 زیرا به چه چیز معلوم می‌شود که من و قوم تو منظور نظر تو شده‌ایم، آیانه از آمدن تو با ما؟ پس من و قوم تو از جمیع قومهایی که بر روی زمینند، ممتاز خواهیم شد.»
\par 17 خداوند به موسی گفت: «این کار را نیز که گفته‌ای خواهم کرد، زیرا که در نظر من فیض یافته‌ای و تو را بنام می‌شناسم.»
\par 18 عرض کرد: «مستدعی آنکه جلال خود را به من بنمایی.»
\par 19 گفت: «من تمامی احسان خود را پیش روی تومی گذرانم و نام یهوه را پیش روی تو ندا می‌کنم، و رافت می‌کنم بر هر‌که رئوف هستم و رحمت خواهم کرد بر هر‌که رحیم هستم.
\par 20 و گفت روی مرا نمی توانی دید، زیرا انسان نمی تواند مرا ببیندو زنده بماند.»
\par 21 و خداوند گفت: «اینک مقامی نزد من است. پس بر صخره بایست.
\par 22 و واقع می‌شود که چون جلال من می‌گذرد، تو را درشکاف صخره می‌گذارم، و تو را به‌دست خودخواهم پوشانید تا عبور کنم.پس دست خودرا خواهم برداشت تا قفای مرا ببینی، اما روی من دیده نمی شود.»
\par 23 پس دست خودرا خواهم برداشت تا قفای مرا ببینی، اما روی من دیده نمی شود.»
 
\chapter{34}

\par 1 و خداوند به موسی گفت: «دو لوح سنگی مثل اولین برای خود بتراش، وسخنانی را که بر لوح های اول بود و شکستی براین لوح‌ها خواهم نوشت.
\par 2 و بامدادان حاضر شوو صبحگاهان به کوه سینا بالا بیا، و در آنجا نزد من بر قله کوه بایست.
\par 3 و هیچکس با تو بالا نیاید، وهیچکس نیز در تمامی کوه دیده نشود، و گله ورمه نیز به طرف این کوه چرا نکند.»
\par 4 پس موسی دو لوح سنگی مثل اولین تراشید و بامدادان برخاسته، به کوه سینا بالا آمد، چنانکه خداوند اورا امر فرموده بود، و دو لوح سنگی را به‌دست خود برداشت.
\par 5 و خداوند در ابر نازل شده، درآنجا با وی بایستاد، و به نام خداوند ندا درداد.
\par 6 وخداوند پیش روی وی عبور کرده، ندا درداد که «یهوه، یهوه، خدای رحیم و رئوف و دیرخشم وکثیر احسان و وفا،
\par 7 نگاه دارنده رحمت برای هزاران، و آمرزنده خطا و عصیان و گناه، لکن گناه را هرگز بی‌سزا نخواهد گذاشت، بلکه خطایای پدران را بر پسران و پسران پسران ایشان تا پشت سوم و چهارم خواهد گرفت.»
\par 8 و موسی به زودی رو به زمین نهاده، سجده کرد.
\par 9 و گفت: «ای خداوند اگر فی الحقیقه منظور نظر تو شده‌ام، مستدعی آنکه خداوند در میان ما بیاید، زیرا که این قوم گردنکش می‌باشند، پس خطا و گناه ما رابیامرز و ما را میراث خود بساز.»
\par 10 گفت: «اینک عهدی می‌بندم و در نظرتمامی قوم تو کارهای عجیب می‌کنم، که درتمامی جهان و در جمیع امتها کرده نشده باشد، وتمامی این قومی که تو در میان ایشان هستی، کارخداوند را خواهند دید، زیراکه این کاری که با توخواهم کرد، کاری هولناک است.
\par 11 آنچه را من امروز به تو امر می‌فرمایم، نگاه دار. اینک من ازپیش روی تو اموریان و کنعانیان و حتیان و فرزیان و حویان و یبوسیان را خواهم راند.
\par 12 با حذرباش که با ساکنان آن زمین که تو بدانجا می‌روی، عهد نبندی، مبادا در میان شما دامی باشد.
\par 13 بلکه مذبحهای ایشان را منهدم سازید، و بتهای ایشان را بشکنید و اشیریم ایشان را قطع نمایید.
\par 14 زنهارخدای غیر را عبادت منما، زیرا یهوه که نام اوغیور است، خدای غیور است.
\par 15 زنهار با ساکنان آن زمین عهد مبند، والا از عقب خدایان ایشان زنا می کنند، و نزد خدایان ایشان قربانی می‌گذرانند، و تو را دعوت می‌نمایند و از قربانی های ایشان می‌خوری.
\par 16 و از دختران ایشان برای پسران خود می‌گیری، و چون دختران ایشان از عقب خدایان خود زنا کنند، آنگاه پسران شما را درپیروی خدایان خود مرتکب زنا خواهند نمود.
\par 17 خدایان ریخته شده برای خویشتن مساز.
\par 18 عید فطیر را نگاه دار، و هفت روز نان فطیرچنانکه تو را امر فرمودم، در وقت معین در ماه ابیب بخور، زیراکه در ماه ابیب از مصر بیرون آمدی.
\par 19 هر‌که رحم را گشاید، از آن من است وهر‌که نخست زاده ذکور از مواشی تو، چه از گاوچه از گوسفند،
\par 20 و برای نخست زاده الاغ، بره‌ای فدیه بده، و اگر فدیه ندهی، گردنش را بشکن و هرنخست زاده‌ای از پسرانت را فدیه بده. و هیچکس به حضور من تهی‌دست حاضر نشود.
\par 21 شش روز مشغول باش، و روز هفتمین، سبت را نگاه دار. در وقت شیار و در حصاد، سبت را نگاه دار.
\par 22 و عید هفته‌ها را نگاه دار، یعنی عید نوبر حصادگندم و عید جمع در تحویل سال.
\par 23 سالی سه مرتبه همه ذکورانت به حضور خداوند یهوه، خدای اسرائیل، حاضر شوند.
\par 24 زیرا که امتها رااز پیش روی تو خواهم راند، و حدود تو را وسیع خواهم گردانید، و هنگامی که در هر سال سه مرتبه می‌آیی تا به حضور یهوه، خدای خودحاضر شوی، هیچکس زمین تو را طمع نخواهدکرد.
\par 25 خون قربانی مرا با خمیرمایه مگذران، و قربانی عید فصح تا صبح نماند.
\par 26 نخستین نوبرزمین خود را به خانه یهوه، خدای خود، بیاور. وبزغاله را در شیر مادرش مپز.»
\par 27 و خداوند به موسی گفت: «این سخنان را تو بنویس، زیرا که به حسب این سخنان، عهد با تو و با اسرائیل بسته‌ام.»
\par 28 و چهل روز و چهل شب آنجا نزدخداوند بوده، نان نخورد و آب ننوشید و اوسخنان عهد، یعنی ده کلام را بر لوحها نوشت.
\par 29 و چون موسی از کوه سینا بزیر می‌آمد، ودو لوح سنگی در دست موسی بود، هنگامی که ازکوه بزیر می‌آمد، واقع شد که موسی ندانست که به‌سبب گفتگوی با او پوست چهره وی می‌درخشید.
\par 30 اما هارون و جمیع بنی‌اسرائیل موسی را دیدند که اینک پوست چهره اومی درخشد. پس ترسیدند که نزدیک او بیایند.
\par 31 و موسی ایشان را خواند، و هارون و همه سرداران جماعت نزد او برگشتند، و موسی بدیشان سخن گفت.
\par 32 و بعد از آن همه بنی‌اسرائیل نزدیک آمدند، و آنچه خداوند درکوه سینا بدو گفته بود، بدیشان امر فرمود.
\par 33 وچون موسی از سخن‌گفتن با ایشان فارغ شد، نقابی بر روی خود کشید.
\par 34 و چون موسی به حضورخداوند داخل می‌شد که با وی گفتگو کند، نقاب را برمی داشت تا بیرون آمدن او. پس بیرون آمده، آنچه به وی امر شده بود، به بنی‌اسرائیل می‌گفت.و بنی‌اسرائیل روی موسی را می‌دیدند که پوست چهره او می‌درخشد. پس موسی نقاب را به روی خود باز می‌کشید، تا وقتی که برای گفتگوی او می‌رفت.
\par 35 و بنی‌اسرائیل روی موسی را می‌دیدند که پوست چهره او می‌درخشد. پس موسی نقاب را به روی خود باز می‌کشید، تا وقتی که برای گفتگوی او می‌رفت.
 
\chapter{35}

\par 1 و موسی تمام جماعت بنی‌اسرائیل را جمع کرده، بدیشان گفت: «این است سخنانی که خداوند امر فرموده است که آنها رابکنی:
\par 2 شش روز کار کرده شود، و در روزهفتمین، سبت آرامی مقدس خداوند برای شماست؛ هر‌که در آن کاری کند، کشته شود.
\par 3 درروز سبت آتش در همه مسکنهای خودمیفروزید.»
\par 4 و موسی تمامی جماعت بنی‌اسرائیل راخطاب کرده، گفت: «این است امری که خداوندفرموده، و گفته است:
\par 5 از خودتان هدیه‌ای برای خداوند بگیرید. هر‌که از دل راغب است، هدیه خداوند را از طلا و نقره و برنج بیاورد،
\par 6 و ازلاجورد و ارغوان و قرمز و کتان نازک و پشم بز،
\par 7 و پوست قوچ سرخ شده و پوست خز و چوب شطیم،
\par 8 و روغن برای روشنایی، و عطریات برای روغن مسح و برای بخور معطر،
\par 9 وسنگهای جزع و سنگهای ترصیع برای ایفود وسینه بند.»
\par 10 و همه دانادلان از شما بیایند و آنچه را خداوند امر فرموده است، بسازند.
\par 11 مسکن وخیمه‌اش و پوشش آن و تکمه هایش وتخته هایش و پشت بندهایش و ستونهایش وپایه هایش،
\par 12 و تابوت و عصاهایش و کرسی رحمت و حجاب ستر،
\par 13 و خوان و عصاهایش و کل اسبابش و نان تقدمه،
\par 14 و چراغدان برای روشنایی و اسبابش و چراغهایش و روغن برای روشنایی،
\par 15 و مذبح بخور و عصاهایش و روغن مسح و بخور معطر، و پرده دروازه برای درگاه مسکن،
\par 16 و مذبح قربانی سوختنی و شبکه برنجین آن، و عصاهایش و کل اسبابش و حوض و پایه‌اش،
\par 17 و پرده های صحن و ستونهایش وپایه های آنها و پرده دروازه صحن،
\par 18 و میخهای مسکن و میخهای صحن و طنابهای آنها،
\par 19 ورختهای بافته شده برای خدمت قدس، یعنی رخت مقدس هارون کاهن، و رختهای پسرانش راتا کهانت نمایند.»
\par 20 پس تمامی جماعت بنی‌اسرائیل از حضورموسی بیرون شدند.
\par 21 و هر‌که دلش او را ترغیب کرد، و هر‌که روحش او را با اراده گردانید، آمدندو هدیه خداوند را برای کار خیمه اجتماع، و برای تمام خدمتش و برای رختهای مقدس آوردند.
\par 22 مردان و زنان آمدند، هر‌که از دل راغب بود، وحلقه های بینی و گوشواره‌ها و انگشتریها وگردن بندها و هر قسم آلات طلا آوردند، و هر‌که هدیه طلا برای خداوند گذرانیده بود.
\par 23 و هرکسی‌که لاجورد و ارغوان و قرمز و کتان نازک وپشم بز و پوست قوچ سرخ شده و پوست خز نزداو یافت شد، آنها را آورد.
\par 24 هر‌که خواست هدیه نقره و برنج بیاورد، هدیه خداوند را آورد، وهر‌که چوب شطیم برای هر کار خدمت نزد اویافت شد، آن را آورد.
\par 25 و همه زنان دانادل به‌دستهای خود می‌رشتند، و رشته شده را ازلاجورد و ارغوان و قرمز و کتان نازک، آوردند.
\par 26 و همه زنانی که دل ایشان به حکمت مایل بود، پشم بز را می‌رشتند.
\par 27 و سروران، سنگهای جزع و سنگهای ترصیع برای ایفود و سینه بندآوردند.
\par 28 و عطریات و روغن برای روشنایی وبرای روغن مسح و برای بخور معطر.
\par 29 و همه مردان و زنان بنی‌اسرائیل که دل ایشان، ایشان راراغب ساخت که چیزی برای هر کاری که خداوند امر فرموده بود که به وسیله موسی ساخته شود، برای خداوند به اراده دل آوردند.
\par 30 و موسی بنی‌اسرائیل را گفت: «آگاه باشیدکه خداوند بصلئیل بن اوری بن حور را از سبطیهودا به نام دعوت کرده است.
\par 31 و او را به روح خدا از حکمت و فطانت و علم و هر هنری پرساخته،
\par 32 و برای اختراع مخترعات و برای کارکردن در طلا و نقره و برنج،
\par 33 و برای تراشیدن ومرصع ساختن سنگها، و برای درودگری چوب تاهر صنعت هنری را بکند.
\par 34 و در دل او تعلیم دادن را القا نمود، و همچنین اهولیاب بن اخیسامک را از سبط دان،و ایشان را به حکمت دلی پر ساخت، برای هر عمل نقاش ونساج ماهر و طراز در لاجورد و ارغوان و قرمز وکتان نازک، و در هر کار نساج تا صانع هر صنعتی ومخترع مخترعات بشوند.
\par 35 و ایشان را به حکمت دلی پر ساخت، برای هر عمل نقاش ونساج ماهر و طراز در لاجورد و ارغوان و قرمز وکتان نازک، و در هر کار نساج تا صانع هر صنعتی ومخترع مخترعات بشوند.
 
\chapter{36}

\par 1 «و بصلئیل و اهولیاب و همه دانادلانی که خداوند حکمت و فطانت بدیشان داده است، تا برای کردن هر صنعت خدمت قدس، ماهر باشند، موافق آنچه خداوند امرفرموده است، کار بکنند.»
\par 2 پس موسی، بصلئیل واهولیاب و همه دانادلانی را که خداوند در دل ایشان حکمت داده بود، و آنانی را که دل ایشان، ایشان را راغب ساخته بود که برای کردن کارنزدیک بیایند، دعوت کرد.
\par 3 و همه هدایایی را که بنی‌اسرائیل برای بجا آوردن کار خدمت قدس آورده بودند، از حضور موسی برداشتند، و هربامداد هدایای تبرعی دیگر نزد وی می‌آوردند.
\par 4 و همه دانایانی که هر گونه کار قدس رامی ساختند، هر یک از کار خود که در آن مشغول می‌بود، آمدند.
\par 5 و موسی را عرض کرده، گفتند: «قوم زیاده از آنچه لازم است برای عمل آن کاری که خداوند فرموده است که ساخته شود، می‌آورند.»
\par 6 و موسی فرمود تا در اردو ندا کرده، گویند که «مردان و زنان هیچ کاری دیگر برای هدایای قدس نکنند.» پس قوم از آوردن بازداشته شدند.
\par 7 و اسباب برای انجام تمام کار، کافی، بلکه زیاده بود.
\par 8 پس همه دانادلانی که در کار اشتغال داشتند، ده پرده مسکن را ساختند، از کتان نازک تابیده شده و لاجورد و ارغوان و قرمز، و آنها را باکروبیان از صنعت نساج ماهر ترتیب دادند.
\par 9 طول هر پرده بیست و هشت ذراع، و عرض هر پرده چهار ذراع. همه پرده‌ها را یک اندازه بود. 
\par 10 وپنج پرده را با یکدیگر بپیوست، و پنج پرده را بایکدیگر بپیوست،
\par 11 و بر لب یک پرده در کنارپیوستگی‌اش مادگیهای لاجورد ساخت، وهمچنین در لب پرده بیرونی در‌پیوستگی دوم ساخت.
\par 12 و در یک پرده، پنجاه مادگی ساخت، و در کنار پرده‌ای که در‌پیوستگی دومین بود، پنجاه مادگی ساخت. و مادگیها مقابل یکدیگربود.
\par 13 و پنجاه تکمه زرین ساخت، و پرده‌ها را به تکمه‌ها با یکدیگر بپیوست، تا مسکن یک باشد.
\par 14 و پرده‌ها از پشم بز ساخت بجهت خیمه‌ای که بالای مسکن بود؛ آنها را پانزده پرده ساخت.
\par 15 طول هر پرده سی ذراع، و عرض هر پرده چهار ذراع؛ و یازده پرده را یک اندازه بود.
\par 16 وپنج پرده را جدا پیوست، و شش پرده را جدا.
\par 17 وپنجاه مادگی بر کنار پرده‌ای که در‌پیوستگی بیرونی بود ساخت، و پنجاه مادگی در کنار پرده در‌پیوستگی دوم.
\par 18 و پنجاه تکمه برنجین برای پیوستن خیمه بساخت تا یک باشد.
\par 19 و پوششی از پوست قوچ سرخ شده برای خیمه ساخت، وپوششی بر زبر آن از پوست خز.
\par 20 و تخته های قایم از چوب شطیم برای مسکن ساخت.
\par 21 طول هر تخته ده ذراع، وعرض هر تخته یک ذراع و نیم.
\par 22 هر تخته را دوزبانه بود مقرون یکدیگر، و بدین ترکیب همه تخته های مسکن را ساخت.
\par 23 و تخته های مسکن را ساخت، بیست تخته به‌جانب جنوب به طرف یمانی،
\par 24 و چهل پایه نقره زیر بیست تخته ساخت، یعنی دو پایه زیر تخته‌ای برای دوزبانه‌اش، و دو پایه زیر تخته دیگر برای دوزبانه‌اش.
\par 25 و برای جانب دیگر مسکن به طرف شمال، بیست تخته ساخت.
\par 26 و چهل پایه نقره آنها را یعنی دو پایه زیر یک تخته‌ای و دو پایه زیرتخته دیگر.
\par 27 و برای موخر مسکن به طرف مغرب، شش تخته ساخت.
\par 28 و دو تخته برای گوشه های مسکن در هر دو جانبش ساخت.
\par 29 واز زیر با یکدیگر پیوسته شد، و تا سر آن با هم دریک حلقه تمام شد. و همچنین برای هر دو در هردو گوشه کرد.
\par 30 پس هشت تخته بود، و پایه های آنها از نقره شانزده پایه، یعنی دو پایه زیر هرتخته.
\par 31 و پشت بندها از چوب شطیم ساخت، یعنی پنج برای تخته های یک جانب مسکن،
\par 32 و پنج پشت بند برای تخته های جانب دیگر مسکن، وپنج پشت بند برای تخته های موخر جانب غربی مسکن.
\par 33 و پشت بند وسطی را ساخت تا در میان تخته‌ها از سر تا سر بگذرد.
\par 34 تخته‌ها را به طلاپوشانید، و حلقه های آنها را از طلا ساخت تابرای پشت بندها، خانه‌ها باشد، و پشت بندها را به طلا پوشانید.
\par 35 و حجاب را از لاجورد و ارغوان و قرمز و کتان نازک تابیده شده ساخت، و آن را باکروبیان از صنعت نساج ماهر ترتیب داد.
\par 36 وچهار ستون از چوب شطیم برایش ساخت، و آنهارا به طلا پوشانید و قلابهای آنها از طلا بود، وبرای آنها چهار پایه نقره ریخت.
\par 37 و پرده‌ای برای دروازه خیمه از لاجورد و ارغوان و قرمز وکتان نازک تابیده شده از صنعت طراز بساخت.و پنج ستون آن و قلابهای آنها را ساخت وسرها و عصاهای آنها را به طلا پوشانید و پنج پایه آنها از برنج بود.
\par 38 و پنج ستون آن و قلابهای آنها را ساخت وسرها و عصاهای آنها را به طلا پوشانید و پنج پایه آنها از برنج بود.
 
\chapter{37}

\par 1 و بصلئیل، تابوت را از چوب شطیم ساخت، طولش دو ذراع و نیم، وعرضش یک ذراع و نیم، و بلندیش یک ذراع ونیم.
\par 2 و آن را به طلای خالص از درون و بیرون پوشانید. و برای آن تاجی از طلا بر طرفش ساخت.
\par 3 و چهار حلقه زرین برای چهارقایمه‌اش بریخت، یعنی دو حلقه بر یک طرفش ودو حلقه بر طرف دیگر.
\par 4 و دو عصا از چوب شطیم ساخته، آنها را به طلا پوشانید.
\par 5 و عصاهارا در حلقه‌ها بر دو جانب تابوت گذرانید، برای برداشتن تابوت.
\par 6 و کرسی رحمت را از طلای خالص ساخت. طولش دو ذراع و نیم، و عرضش یک ذراع و نیم.
\par 7 و دو کروبی از طلا ساخت. وآنها را بر هر دو طرف کرسی رحمت ازچرخکاری ساخت.
\par 8 یک کروبی بر این طرف و کروبی دیگر بر آن طرف، و از کرسی رحمت، کروبیان را بر هر دو طرفش ساخت.
\par 9 و کروبیان بالهای خود را بر زبر آن پهن می‌کردند، و به بالهای خویش کرسی رحمت را می‌پوشانیدند، ورویهای ایشان به سوی یکدیگر می‌بود، یعنی رویهای کروبیان به‌جانب کرسی رحمت می‌بود.
\par 10 و خوان را از چوب شطیم ساخت. طولش دو ذراع، و عرضش یک ذراع، و بلندیش یک ذراع و نیم.
\par 11 و آن را به طلای خالص پوشانید، وتاجی زرین گرداگردش ساخت.
\par 12 و حاشیه‌ای به مقدار چهار انگشت گرداگردش ساخت، و تاجی زرین گرداگرد حاشیه ساخت.
\par 13 و چهار حلقه زرین برایش ریخت، و حلقه‌ها را بر چهارگوشه‌ای که بر چهار قایمه‌اش بود گذاشت.
\par 14 وحلقه‌ها مقابل حاشیه بود، تا خانه های عصاهاباشد، برای برداشتن خوان.
\par 15 و دو عصا را ازچوب شطیم ساخته، آنها را به طلا پوشانید، برای برداشتن خوان.
\par 16 و ظروفی را که بر خوان می‌بوداز صحنها و کاسه‌ها و پیاله‌ها و جامهایش که بدانها هدایای ریختنی می‌ریختند، از طلای خالص ساخت.
\par 17 و چراغدان را از طلای خالص ساخت. ازچرخکاری، چراغدان را ساخت، و پایه‌اش وشاخه هایش و پیاله هایش و سیبهایش و گلهایش از همین بود.
\par 18 و از دو طرفش شش شاخه بیرون آمد، یعنی سه شاخه چراغدان از یک طرف، و سه شاخه چراغدان از طرف دیگر.
\par 19 و سه پیاله بادامی با سیبی و گلی در یک شاخه، و سه پیاله بادامی و سیبی و گلی بر شاخه دیگر، و همچنین برای شش شاخه‌ای که از چراغدان بیرون می‌آمد.
\par 20 و بر چراغدان چهار پیاله بادامی باسیبها و گلهای آن.
\par 21 و سیبی زیر دو شاخه آن، وسیبی زیر دو شاخه آن، و سیبی زیر دو شاخه آن، برای شش شاخه‌ای که از آن بیرون می‌آمد.
\par 22 سیبهای آنها و شاخه های آنها از همین بود، یعنی همه از یک چرخکاری طلای خالص.
\par 23 وهفت چراغش و گلگیرهایش و سینیهایش را ازطلای خالص ساخت.
\par 24 از یک وزنه طلای خالص آن را با همه اسبابش ساخت.
\par 25 و مذبح بخور را از چوب شطیم ساخت، طولش یک ذراع، و عرضش یک ذراع مربع، وبلندیش دو ذراع، و شاخهایش از همان بود.
\par 26 وآن را به طلای خالص پوشانید، یعنی سطحش وطرفهای گرداگردش، و شاخهایش، و تاجی گرداگردش از طلای خالص ساخت.
\par 27 و دوحلقه زرین برایش زیر تاج بر دو گوشه‌اش بر دوطرفش ساخت، تا خانه های عصاها باشد برای برداشتنش به آنها.
\par 28 و عصاها را از چوب شطیم ساخته، آنها را به طلا پوشانید.و روغن مسح مقدس و بخور معطر طاهر را از صنعت عطارساخت.
\par 29 و روغن مسح مقدس و بخور معطر طاهر را از صنعت عطارساخت.
 
\chapter{38}

\par 1 و مذبح قربانی سوختنی را از چوب شطیم ساخت. طولش پنج ذراع، وعرضش پنج ذراع مربع، و بلندیش سه ذراع.
\par 2 وشاخهایش را بر چهار گوشه‌اش ساخت. شاخهایش از همان بود و آن را از برنج پوشانید.
\par 3 و همه اسباب مذبح را ساخت، یعنی: دیگها وخاک اندازها و کاسه‌ها و چنگالها و مجمرها وهمه ظروفش را از برنج ساخت.
\par 4 و برای مذبح، آتش دانی مشبک از برنج ساخت، که زیرحاشیه‌اش بطرف پایین تا نصفش برسد.
\par 5 و چهارحلقه برای چهار سر آتش دان برنجین ریخت، تاخانه های عصاها باشد.
\par 6 و عصاها را از چوب شطیم ساخته، آنها را به برنج بپوشانید.
\par 7 و عصاهارا در حلقه‌ها بر دو طرف مذبح گذرانید، برای برداشتنش به آنها، و مذبح را از چوبها مجوف ساخت.
\par 8 و حوض را از برنج ساخت، و پایه‌اش را از برنج از آینه های زنانی که نزد دروازه خیمه اجتماع برای خدمت جمع می‌شدند.
\par 9 و صحن را ساخت که برای طرف جنوبی به سمت یمانی. پرده های صحن از کتان نازک تابیده شده صد ذراعی بود.
\par 10 ستونهای آنها بیست بود، و پایه های آنها بیست بود، از برنج و قلابهای آنهاو پشت بندهای آنها از نقره.
\par 11 و برای طرف شمالی صد ذراعی بود، و ستونهای آنها بیست ازبرنج، و قلابهای ستونها و پشت بندهای آنها ازنقره بود.
\par 12 و برای طرف غربی، پرده های پنجاه ذراعی بود، و ستونهای آنها ده و پایه های آنها ده، و قلابها و پشت بندهای ستونها از نقره بود.
\par 13 وبرای طرف شرقی به سمت طلوع، پنجاه ذراعی بود.
\par 14 و پرده های یک طرف دروازه پانزده ذراعی بود، ستونهای آنها سه و پایه های آنها سه.
\par 15 و برای طرف دیگر دروازه صحن از این طرف واز آن طرف پرده‌ها پانزده ذراعی بود، ستونهای آنها سه و پایه های آنها سه.
\par 16 همه پرده های صحن به هر طرف از کتان نازک تابیده شده بود.
\par 17 و پایه های ستونها از برنج بود، و قلابها وپشت بندهای ستونها از نقره، و پوشش سرهای آنها از نقره، و جمیع ستونهای صحن به پشت بندهای نقره پیوسته شده بود.
\par 18 و پرده دروازه صحن از صنعت طراز از لاجورد و ارغوان و قرمز و کتان نازک تابیده شده بود. طولش بیست ذراع، و بلندیش به عرض پنج ذراع موافق پرده های صحن.
\par 19 و ستونهای آنها چهار، وپایه های برنجین آنها چهار، و قلابهای آنها ازنقره، و پوشش سرهای آنها و پشت بندهای آنهااز نقره بود.
\par 20 و همه میخهای مسکن و صحن، به هر طرف از برنج بود.
\par 21 این است حساب مسکن، یعنی مسکن شهادت، چنانکه حسب فرمان موسی به خدمت لاویان، به توسط ایتاماربن هارون کاهن حساب آن گرفته شد.
\par 22 و بصلئیل بن اوری بن حور از سبطیهودا، آنچه را که خداوند به موسی‌امر فرموده بود بساخت.
\par 23 و با وی اهولیاب بن اخیسامک ازسبط دان بود، نقاش و مخترع و طراز در لاجورد وارغوان و قرمز و کتان نازک.
\par 24 و تمام طلایی که در کار صرف شد، یعنی در همه کار قدس، ازطلای هدایا بیست و نه وزنه و هفتصد و سی مثقال موافق مثقال قدس بود.
\par 25 و نقره شمرده شدگان جماعت صد وزنه و هزار و هفتصد وهفتاد و پنج مثقال بود، موافق مثقال قدس.
\par 26 یک درهم یعنی نیم مثقال موافق مثقال قدس، برای هر نفری از آنانی که به سوی شمرده شدگان گذشتند، از بیست ساله و بالاتر، که ششصد و سه هزار وپانصد و پنجاه نفر بودند.
\par 27 و اما آن صد وزنه نقره برای ریختن پایه های قدس و پایه های پرده بود. صد پایه از صد وزنه یعنی یک وزنه برای یک پایه.
\par 28 و از آن هزار و هفتصد و هفتاد و پنج مثقال قلابها برای ستونها ساخت، و سرهای آنهارا پوشانید، و پشت بندها برای آنها ساخت.
\par 29 وبرنج هدایا هفتاد وزنه و دو هزار و چهارصد مثقال بود.
\par 30 و از آن پایه های دروازه خیمه اجتماع، ومذبح برنجین، و شبکه برنجین آن و همه اسباب مذبح را ساخت.و پایه های صحن را به هرطرف، و پایه های دروازه صحن و همه میخهای مسکن و همه میخهای گرداگرد صحن را.
\par 31 و پایه های صحن را به هرطرف، و پایه های دروازه صحن و همه میخهای مسکن و همه میخهای گرداگرد صحن را.
 
\chapter{39}

\par 1 و از لاجورد و ارغوان و قرمز رختهای بافته شده ساختند، برای خدمت کردن در قدس، و رختهای مقدس برای هارون ساختند، چنانکه خداوند به موسی‌امر نموده بود.
\par 2 و ایفود را از طلا و لاجورد و ارغوان و قرمزو کتان نازک تابیده شده، ساخت.
\par 3 و تنگه های نازک از طلا ساختند و تارها کشیدند تا آنها را درمیان لاجورد و ارغوان و قرمز و کتان نازک به صنعت نساج ماهر ببافند.
\par 4 و کتفهای پیوسته شده برایش ساختند، که بر دو کنار پیوسته شد.
\par 5 و زنار بسته شده‌ای که بر آن بود از همان پارچه واز همان صنعت بود، از طلا و لاجورد و ارغوان وقرمز و کتان نازک تابیده شده، چنانکه خداوند به موسی‌امر فرموده بود.
\par 6 و سنگهای جزع مرصع در دو طوق طلا، و منقوش به نقش خاتم، موافق نامهای بنی‌اسرائیل درست کردند.
\par 7 آنها را بر کتفهای ایفود نصب کرد، تا سنگهای یادگاری برای بنی‌اسرائیل باشد، چنانکه خداوند به موسی‌امر فرموده بود.
\par 8 و سینه بند را موافق کار ایفود از صنعت نساج ماهر ساخت، از طلا و لاجورد و ارغوان وقرمز و کتان نازک تابیده شده. 
\par 9 و آن مربع بود وسینه بند را دولا ساختند طولش یک وجب وعرضش یک وجب دولا.
\par 10 و در آن چهار رسته سنگ نصب کردند، رسته‌ای از عقیق سرخ ویاقوت زرد و زمرد. این بود رسته اول.
\par 11 و رسته دوم از بهرمان و یاقوت کبود و عقیق سفید.
\par 12 ورسته سوم از عین الهر و یشم و جمست.
\par 13 ورسته چهارم از زبرجد و جزع و یشب در ترصیعه خود که به دیوارهای طلا احاطه شده بود.
\par 14 وسنگها موافق نامهای بنی‌اسرائیل دوازده بود، مطابق اسامی ایشان، مثل نقش خاتم، هر یکی به اسم خود برای دوازده سبط.
\par 15 و بر سینه بندزنجیرهای تابیده شده، مثل کار طنابها از طلای خالص ساختند.
\par 16 و دو طوق زرین و دو حلقه زرین ساختند و دو حلقه را بر دو سر سینه بندگذاشتند.
\par 17 و آن دو زنجیر تابیده شده زرین رادر دو حلقه‌ای که بر سرهای سینه بند بود، گذاشتند.
\par 18 و دو سر دیگر آن دو زنجیر را بر دوطوق گذاشتند، و آنها را بر دو کتف ایفود در‌پیش نصب کردند.
\par 19 و دو حلقه زرین ساختند، آنها رابر دو سر سینه بند گذاشتند، بر کناری که بر طرف اندرونی‌ایفود بود.
\par 20 و دو حلقه زرین دیگرساختند، و آنها را بر دو کتف ایفود، به طرف پایین، از جانب پیش، مقابل پیوستگیش بالای زنار ایفود گذاشتند.
\par 21 و سینه بند را به حلقه هایش با حلقه های ایفود به نوار لاجوردی بستند، تا بالای زنار ایفود باشد. و سینه بند از ایفود جدا نشود، چنانکه خداوند به موسی‌امرفرموده بود.
\par 22 و ردای ایفود را از صنعت نساج، تمام لاجوردی ساخت.
\par 23 و دهنه‌ای در وسط ردابود، مثل دهنه زره با حاشیه‌ای گرداگرد دهنه تادریده نشود.
\par 24 و بر دامن ردا، انارها از لاجورد وارغوان و قرمز و کتان تابیده شده ساختند.
\par 25 وزنگوله‌ها از طلای خالص ساختند. و زنگوله‌ها رادر میان انارها بر دامن ردا گذاشتند، گرداگردش درمیان انارها.
\par 26 و زنگوله‌ای و اناری، و زنگوله‌ای واناری گرداگرد دامن ردا برای خدمت کردن، چنانکه خداوند به موسی‌امر فرموده بود.
\par 27 وپیراهنها را برای هارون و پسرانش از کتان نازک ازصنعت نساج ساختند.
\par 28 و عمامه را از کتان نازک و دستارهای زیبا را از کتان نازک، و زیرجامهای کتانی را از کتان نازک تابیده شده.
\par 29 و کمربند رااز کتان نازک تابیده شده، و لاجورد و ارغوان وقرمز از صنعت طراز، چنانکه خداوند به موسی‌امر فرموده بود.
\par 30 و تنگه افسر مقدس را ازطلای خالص ساختند، و بر آن کتابتی مثل نقش خاتم مرقوم داشتند: قدوسیت برای یهوه.
\par 31 و برآن نواری لاجوردی بستند تا آن را بالای عمامه ببندند، چنانکه خداوند به موسی‌امر فرموده بود.
\par 32 پس همه کار مسکن خیمه اجتماع تمام شد، و بنی‌اسرائیل ساختند. موافق آنچه خداوندبه موسی‌امر فرموده بود، عمل نمودند.
\par 33 ومسکن خیمه را نزد موسی آوردند، با همه اسبابش و تکمه‌ها و تخته‌ها و پشت بندها وستونها و پایه هایش.
\par 34 و پوشش از پوست قوچ سرخ شده و پوشش از پوست خز و حجاب ستر.
\par 35 و تابوت شهادت و عصاهایش و کرسی رحمت.
\par 36 و خوان و همه اسبابش و نان تقدمه.
\par 37 و چراغدان طاهر و چراغهایش، چراغهای آراسته شده و همه اسبابش، و روغن برای روشنایی.
\par 38 و مذبح زرین و روغن مسح و بخورمعطر و پرده برای دروازه خیمه.
\par 39 و مذبح برنجین و شبکه برنجین آن، و عصاهایش و همه اسبابش و حوض و پایه‌اش.
\par 40 و پرده های صحن و ستونها و پایه هایش و پرده دروازه صحن، وطنابهایش و میخهایش و همه اسباب خدمت مسکن برای خیمه اجتماع.
\par 41 و رختهای بافته شده برای خدمت قدس، و رخت مقدس برای هارون کاهن، و رختها برای پسرانش تا کهانت نمایند.
\par 42 موافق آنچه خداوند به موسی‌امرفرموده بود، بنی‌اسرائیل همچنین تمام کار راساختند.و موسی تمام کارها را ملاحظه کرد، و اینک موافق آنچه خداوند امر فرموده بودساخته بودند، همچنین کرده بودند. و موسی ایشان را برکت داد.
\par 43 و موسی تمام کارها را ملاحظه کرد، و اینک موافق آنچه خداوند امر فرموده بودساخته بودند، همچنین کرده بودند. و موسی ایشان را برکت داد.
 
\chapter{40}

\par 1 و خداوند موسی را خطاب کرده، گفت:
\par 2 «در غره ماه اول مسکن خیمه اجتماع را برپا نما.
\par 3 و تابوت شهادت را در آن بگذار. وحجاب را پیش تابوت پهن کن.
\par 4 و خوان رادرآورده، چیزهایی را که می‌باید، بر آن ترتیب نما. و چراغدان را درآور و چراغهایش را آراسته کن.
\par 5 و مذبح زرین را برای بخور پیش تابوت شهادت بگذار، و پرده دروازه را بر مسکن بیاویز.
\par 6 و مذبح قربانی سوختنی را پیش دروازه مسکن خیمه اجتماع بگذار.
\par 7 و حوض را در میان خیمه اجتماع و مذبح بگذار، و آب در آن بریز.
\par 8 و صحن را گرداگرد برپا کن. و پرده دروازه صحن رابیاویز.
\par 9 و روغن مسح را گرفته، مسکن را با آنچه در آن است مسح کن، و آن را با همه اسبابش تقدیس نما تا مقدس شود.
\par 10 و مذبح قربانی سوختنی را با همه اسبابش مسح کرده، مذبح راتقدیس نما. و مذبح، قدس اقداس خواهد بود.
\par 11 و حوض را با پایه‌اش مسح نموده، تقدیس کن.
\par 12 و هارون و پسرانش را نزد دروازه خیمه اجتماع آورده، ایشان را به آب غسل ده.
\par 13 وهارون را به رخت مقدس بپوشان، و او را مسح کرده، تقدیس نما، تا برای من کهانت کند.
\par 14 وپسرانش را نزدیک آورده، ایشان را به پیراهنهابپوشان.
\par 15 و ایشان را مسح کن، چنانکه پدرایشان را مسح کردی تا برای من کهانت نماید. ومسح ایشان هر آینه برای کهانت ابدی در نسلهای ایشان خواهد بود.»
\par 16 پس موسی موافق آنچه خداوند او را امر فرموده بود کرد، و همچنین به عمل آورد.
\par 17 و واقع شد در غره ماه اول از سال دوم که مسکن برپا شد،
\par 18 و موسی مسکن را برپا نمود، وپایه هایش را بنهاد و تخته هایش را قایم کرد، وپشت بندهایش را گذاشت، و ستونهایش را برپانمود،
\par 19 و خیمه را بالای مسکن کشید، و پوشش خیمه را بر زبر آن گسترانید، چنانکه خداوند به موسی‌امر نموده بود.
\par 20 و شهادت را گرفته، آن رادر تابوت نهاد، و عصاها را بر تابوت گذارد، وکرسی رحمت را بالای تابوت گذاشت.
\par 21 وتابوت را به مسکن درآورد، و حجاب ستر راآویخته، آن را پیش تابوت شهادت کشید. چنانکه خداوند به موسی‌امر فرموده بود.
\par 22 و خوان را در خیمه اجتماع به طرف شمالی مسکن، بیرون حجاب نهاد.
\par 23 و نان را به حضور خداوند بر آن ترتیب داد، چنانکه خداوند به موسی‌امر فرموده بود.
\par 24 و چراغدان را در خیمه اجتماع، مقابل خوان به طرف جنوبی مسکن نهاد.
\par 25 و چراغها رابه حضور خداوند گذاشت، چنانکه خداوند به موسی‌امر فرموده بود.
\par 26 و مذبح زرین را درخیمه اجتماع، پیش حجاب نهاد.
\par 27 و بخورمعطر بر آن سوزانید، چنانکه خداوند به موسی‌امر فرموده بود.
\par 28 و پرده دروازه مسکن راآویخت.
\par 29 و مذبح قربانی سوختنی را پیش دروازه مسکن خیمه اجتماع وضع کرد، و قربانی سوختنی و هدیه را بر آن گذرانید، چنانکه خداوند به موسی‌امر فرموده بود.
\par 30 و حوض رادر میان خیمه اجتماع و مذبح وضع کرده، آب برای شستن در آن بریخت.
\par 31 و موسی و هارون وپسرانش دست و پای خود را در آن شستند.
\par 32 وقتی که به خیمه اجتماع داخل شدند ونزد مذبح آمدند شست و شو کردند، چنانکه خداوند به موسی‌امر فرموده بود.
\par 33 و صحن راگرداگرد مسکن و مذبح برپا نمود، و پرده دروازه صحن را آویخت. پس موسی کار را به انجام رسانید.
\par 34 آنگاه ابر، خیمه اجتماع را پوشانید و جلال خداوند مسکن را پر ساخت.
\par 35 و موسی نتوانست به خیمه اجتماع داخل شود، زیراکه ابربر آن ساکن بود، و جلال خداوند مسکن را پرساخته بود.
\par 36 و چون ابر از بالای مسکن برمی خاست، بنی‌اسرائیل در همه مراحل خودکوچ می‌کردند.و هرگاه ابر برنمی خاست، تاروز برخاستن آن، نمی کوچیدند.
\par 37 و هرگاه ابر برنمی خاست، تاروز برخاستن آن، نمی کوچیدند.


\end{document}