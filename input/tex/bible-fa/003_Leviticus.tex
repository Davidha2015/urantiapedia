\begin{document}

\title{Leviticus}

 
\chapter{1}

\par 1 و خداوند موسی را خواند، و او را از خیمه اجتماع خطاب کرده، گفت:
\par 2 «بنی‌اسرائیل را خطاب کرده، به ایشان بگو: هرگاه کسی از شماقربانی نزد خداوند بگذراند، پس قربانی خود را ازبهایم یعنی از گاو یا از گوسفند بگذرانید.
\par 3 اگرقربانی او قربانی سوختنی از گاو باشد، آن را نربی عیب بگذارند، و آن را نزد در خیمه اجتماع بیاورد تا به حضور خداوند مقبول شود.
\par 4 و دست خود را بر سر قربانی سوختنی بگذارد، و برایش مقبول خواهد شد تا بجهت او کفاره کند.
\par 5 پس گاو را به حضور خداوند ذبح نماید، و پسران هارون کهنه خون را نزدیک بیاورند، و خون را براطراف مذبح که نزد در خیمه اجتماع است بپاشند.
\par 6 و پوست قربانی سوختنی را بکند و آن را قطعه قطعه کند.
\par 7 و پسران هارون کاهن آتش برمذبح بگذارند، و هیزم بر آتش بچینند.
\par 8 و پسران هارون کهنه قطعه‌ها و سر و پیه را بر هیزمی که برآتش روی مذبح است بچینند.
\par 9 و احشایش وپاچه هایش را به آب بشویند، و کاهن همه را برمذبح بسوزاند، برای قربانی سوختنی و هدیه آتشین و عطر خوشبو بجهت خداوند.
\par 10 و اگرقربانی او از گله باشد خواه از گوسفند خواه از بزبجهت قربانی سوختنی، آن را نر بی‌عیب بگذراند.
\par 11 و آن را به طرف شمالی مذبح به حضور خداوند ذبح نماید، و پسران هارون کهنه خونش را به اطراف مذبح بپاشند.
\par 12 و آن را باسرش و پیه‌اش قطعه قطعه کند، و کاهن آنها را برهیزمی که بر آتش روی مذبح است بچیند.
\par 13 واحشایش و پاچه هایش را به آب بشوید، و کاهن همه را نزدیک بیاورد و بر مذبح بسوزاند، که آن قربانی سوختنی و هدیه آتشین و عطر خوشبوبجهت خداوند است.
\par 14 «و اگر قربانی او بجهت خداوند قربانی سوختنی از مرغان باشد، پس قربانی خود را ازفاخته‌ها یا از جوجه های کبوتر بگذراند.
\par 15 وکاهن آن را نزد مذبح بیاورد و سرش را بپیچد و برمذبح بسوزاند، و خونش را بر پهلوی مذبح افشرده شود.
\par 16 و چینه دانش را با فضلات آن بیرون کرده، آن را بر جانب شرقی مذبح در جای خاکستر بیندازد.و آن را از میان بالهایش چاک کند و از هم جدا نکند، و کاهن آن را بر مذبح برهیزمی که بر آتش است بسوزاند، که آن قربانی سوختنی و هدیه آتشین و عطر خوشبو بجهت خداوند است.
\par 17 و آن را از میان بالهایش چاک کند و از هم جدا نکند، و کاهن آن را بر مذبح برهیزمی که بر آتش است بسوزاند، که آن قربانی سوختنی و هدیه آتشین و عطر خوشبو بجهت خداوند است.
 
\chapter{2}

\par 1 «و هرگاه کسی قربانی هدیه آردی بجهت خداوند بگذراند، پس قربانی او از آرد نرم باشد، و روغن بر آن بریزد و کندر بر آن بنهد.
\par 2 وآن را نزد پسران هارون کهنه بیاورد، و یک مشت از آن بگیرد یعنی از آرد نرمش و روغنش باتمامی کندرش و کاهن آن را برای یادگاری بسوزاند، تا هدیه آتشین و عطر خوشبو بجهت خداوند باشد.
\par 3 و بقیه هدیه آردی از آن هارون وپسرانش باشد. این از هدایای آتشین خداوند قدس اقداس است.
\par 4 و هرگاه قربانی هدیه آردی پخته شده‌ای در تنور بگذرانی، پس قرصهای فطیر از آرد نرم سرشته شده به روغن، یا گرده های فطیر مالیده شده به روغن باشد.
\par 5 و اگر قربانی توهدیه آردی بر ساج باشد، پس از آرد نرم فطیرسرشته شده به روغن باشد.
\par 6 و آن را پاره پاره کرده، روغن بر آن بریز. این هدیه آردی است.
\par 7 واگر قربانی تو هدیه آردی تابه باشد از آرد نرم باروغن ساخته شود.
\par 8 و هدیه آردی را که از این چیزها ساخته شود نزد خداوند بیاور، و آن راپیش کاهن بگذار، و او آن را نزد مذبح خواهدآورد.
\par 9 و کاهن از هدیه آردی یادگاری آن رابردارد و بر مذبح بسوزاند. این هدیه آتشین و عطرخوشبو بجهت خداوند است.
\par 10 و بقیه هدیه آردی از آن هارون و پسرانش باشد. این ازهدایای آتشین خداوند قدس اقداس است.
\par 11 «و هیچ هدیه آردی که بجهت خداوندمی گذرانید با خمیرمایه ساخته نشود، زیرا که هیچ خمیرمایه و عسل را برای هدیه آتشین بجهت خداوند نباید سوزانید.
\par 12 آنها را برای قربانی نوبرها بجهت خداوند بگذرانید، لیکن برای عطر خوشبو به مذبح برنیارند.
\par 13 و هرقربانی هدیه آردی خود را به نمک نمکین کن، ونمک عهد خدای خود را از هدیه آردی خودبازمدار، با هر قربانی خود نمک بگذران.
\par 14 و اگرهدیه آردی نوبرها بجهت خداوند بگذرانی، پس خوشه های برشته شده به آتش، یعنی بلغورحاصل نیکو بجهت هدیه آردی نوبرهای خودبگذران.
\par 15 و روغن بر آن بریز و کندر بر آن بنه. این هدیه آردی است.و کاهن یادگاری آن رایعنی قدری از بلغور آن و از روغنش با تمامی کندرش بسوزاند. این هدیه آتشین بجهت خداوند است.
\par 16 و کاهن یادگاری آن رایعنی قدری از بلغور آن و از روغنش با تمامی کندرش بسوزاند. این هدیه آتشین بجهت خداوند است.
 
\chapter{3}

\par 1 «و اگر قربانی او ذبیحه سلامتی باشد، اگر ازرمه بگذراند خواه نر و خواه ماده باشد، آن را بی‌عیب به حضور خداوند بگذراند.
\par 2 و دست خود را بر سر قربانی خویش بنهد، و آن را نزد درخیمه اجتماع ذبح نماید، و پسران هارون کهنه خون را به اطراف مذبح بپاشند.
\par 3 و از ذبیحه سلامتی، هدیه آتشین بجهت خداوند بگذراند، یعنی پیهی که احشا را می‌پوشاند و همه پیه را که بر احشاست.
\par 4 و دو گرده و پیه که بر آنهاست که بر دو تهیگاه است، و سفیدی را که بر جگر است، با گرده‌ها جدا کند.
\par 5 و پسران هارون آن را برمذبح با قربانی سوختنی بر هیزمی که بر آتش است بسوزانند. این هدیه آتشین و عطر خوشبوبجهت خداوند است.
\par 6 و اگر قربانی او برای ذبیحه سلامتی بجهت خداوند از گله باشد، آن رانر یا ماده بی‌عیب بگذراند.
\par 7 اگر بره‌ای برای قربانی خود بگذراند، آن را به حضور خداوندنزدیک بیاورد.
\par 8 و دست خود را بر سر قربانی خود بنهد، و آن را نزد در خیمه اجتماع ذبح نماید، و پسران هارون خونش را به اطراف مذبح بپاشند.
\par 9 و از ذبیحه سلامتی هدیه آتشین بجهت خداوند بگذراند، یعنی پیه‌اش و تمامی دنبه را وآن را از نزد عصعص جدا کند، و پیهی که احشا رامی پوشاند و همه پیه را که بر احشاست.
\par 10 و دوگرده و پیهی که بر آنهاست که بر دو تهیگاه است وسفیدی را که بر جگر است با گرده‌ها جدا کند.
\par 11 و کاهن آن را بر مذبح بسوزاند. این طعام هدیه آتشین بجهت خداوند است.
\par 12 و اگر قربانی او بز باشد پس آن را به حضور خداوند نزدیک بیاورد.
\par 13 و دست خود را برسرش بنهد و آن را پیش خیمه اجتماع ذبح نماید. و پسران هارون خونش را به اطراف مذبح بپاشند.
\par 14 و قربانی خود، یعنی هدیه آتشین را، بجهت خداوند از آن بگذراند، پیهی که احشا رامی پوشاند و تمامی پیهی که بر احشاست.
\par 15 و دوگرده و پیهی که بر آنهاست که بر دو تهیگاه است وسفیدی را که بر جگر است با گرده‌ها جدا کند.
\par 16 و کاهن آن را بر مذبح بسوزاند. این طعام هدیه آتشین برای عطر خوشبوست. تمامی پیه از آن خداوند است.این قانون ابدی در همه پشتهای شما در جمیع مسکنهای شما خواهد بود که هیچ خون و پیه را نخورید.»
\par 17 این قانون ابدی در همه پشتهای شما در جمیع مسکنهای شما خواهد بود که هیچ خون و پیه را نخورید.»
 
\chapter{4}

\par 1 و خداوند موسی را خطاب کرده، گفت:
\par 2 «بنی‌اسرائیل را خطاب کرده، بگو: اگرکسی سهو گناه کند، در هر کدام از نواهی خداوند که نباید کرد، و به خلاف هریک از آنهاعمل کند،
\par 3 اگر کاهن که مسح شده است گناهی ورزد و قوم را مجرم سازد، پس برای گناهی که کرده است، گوساله‌ای بی‌عیب از رمه برای قربانی گناه نزد خداوند بگذراند.
\par 4 و گوساله را به درخیمه اجتماع به حضور خداوند بیاورد، و دست خود را بر سر گوساله بنهد و گوساله را به حضورخداوند ذبح نماید.
\par 5 و کاهن مسح شده از خون گوساله گرفته، آن را به خیمه اجتماع درآورد.
\par 6 وکاهن انگشت خود را در خون فرو برد، و به حضور خداوند پیش حجاب قدس قدری ازخون را هفت مرتبه بپاشد.
\par 7 و کاهن قدری ازخون را بر شاخه های مذبح بخور معطر که درخیمه اجتماع است، به حضور خداوند بپاشد، و همه خون گوساله را بر بنیان مذبح قربانی سوختنی که به در خیمه اجتماع است بریزد.
\par 8 وتمامی پیه گوساله قربانی گناه، یعنی پیهی که احشا را می‌پوشاند و همه پیه را که بر احشاست ازآن بردارد.
\par 9 و دو گرده و پیهی که بر آنهاست که بردو تهیگاه است و سفیدی را که بر جگر است باگرده‌ها جدا کند.
\par 10 چنانکه از گاو ذبیحه سلامتی برداشته می‌شود و کاهن آنها را بر مذبح قربانی سوختنی بسوزاند.
\par 11 و پوست گوساله و تمامی گوشتش با سرش و پاچه هایش و احشایش وسرگینش،
\par 12 یعنی تمامی گوساله را بیرون لشکرگاه در مکان پاک جایی که خاکستر رامی ریزند ببرد، و آن را بر هیزم به آتش بسوزاند. در جایی که خاکستر را می‌ریزند سوخته شود.
\par 13 «و هرگاه تمامی جماعت اسرائیل سهو گناه کنند و آن امر از چشمان جماعت مخفی باشد، و کاری را که نباید کرد از جمیع نواهی خداوند کرده، مجرم شده باشند.
\par 14 چون گناهی که کردند معلوم شود، آنگاه جماعت گوساله‌ای ازرمه برای قربانی گناه بگذرانند و آن را پیش خیمه اجتماع بیاورند.
\par 15 و مشایخ جماعت دستهای خود را بر سر گوساله به حضور خداوند بنهند، وگوساله به حضور خداوند ذبح شود.
\par 16 و کاهن مسح شده، قدری از خون گوساله را به خیمه اجتماع درآورد.
\par 17 و کاهن انگشت خود را درخون فروبرد و آن را به حضور خداوند پیش حجاب هفت مرتبه بپاشد.
\par 18 و قدری از خون رابر شاخه های مذبح که به حضور خداوند درخیمه اجتماع است بگذارد، و همه خون را بربنیان مذبح قربانی سوختنی که نزد در خیمه اجتماع است بریزد.
\par 19 و همه پیه آن را از آن برداشته، بر مذبح بسوزاند.
\par 20 و با گوساله عمل نماید چنانکه با گوساله قربانی گناه عمل کرد، همچنان با این بکند و کاهن برای ایشان کفاره کند، و آمرزیده خواهند شد.
\par 21 و گوساله را بیرون لشکرگاه برده، آن را بسوزاند چنانکه گوساله اول را سوزانید. این قربانی گناه جماعت است.
\par 22 «و هرگاه رئیس گناه کند، و کاری را که نباید کرد از جمیع نواهی یهوه خدای خود سهو بکند و مجرم شود،
\par 23 چون گناهی که کرده است بر او معلوم شود، آنگاه بز نر بی‌عیب برای قربانی خود بیاورد.
\par 24 و دست خود را بر سر بز بنهد وآن را در جایی که قربانی سوختنی را ذبح کنند به حضور خداوند ذبح نماید. این قربانی گناه است.
\par 25 و کاهن قدری از خون قربانی گناه را به انگشت خود گرفته، بر شاخهای مذبح قربانی سوختنی بگذارد، و خونش را بر بنیان مذبح سوختنی بریزد.
\par 26 و همه پیه آن را مثل پیه ذبیحه سلامتی بر مذبح بسوزاند، و کاهن برای او گناهش را کفاره خواهد کرد و آمرزیده خواهد شد.
\par 27 «و هرگاه کسی از اهل زمین سهو گناه ورزدو کاری را که نباید کرد از همه نواهی خداوندبکند و مجرم شود،
\par 28 چون گناهی که کرده است بر او معلوم شود، آنگاه برای قربانی خود بز ماده بی‌عیب بجهت گناهی که کرده است بیاورد.
\par 29 ودست خود را بر سر قربانی گناه بنهد و قربانی گناه را در جای قربانی سوختنی ذبح نماید.
\par 30 و کاهن قدری از خونش را به انگشت خود گرفته، آن را برشاخهای مذبح قربانی سوختنی بگذارد، و همه خونش را بر بنیان مذبح بریزد.
\par 31 و همه پیه آن راجدا کند چنانکه پیه از ذبیحه سلامتی جدامی شود، و کاهن آن را بر مذبح بسوزاند برای عطر خوشبو بجهت خداوند و کاهن برای او کفاره خواهد کرد و آمرزیده خواهد شد.
\par 32 و اگر برای قربانی خود بره‌ای بجهت قربانی گناه بیاورد آن راماده بی‌عیب بیاورد.
\par 33 و دست خود را بر سرقربانی گناه بنهد و آن را برای قربانی گناه در جایی که قربانی سوختنی ذبح می‌شود ذبح نماید.
\par 34 وکاهن قدری از خون قربانی گناه را به انگشت خودگرفته، بر شاخهای مذبح قربانی سوختنی بگذاردو همه خونش را بر بنیان مذبح بریزد.و همه پیه آن را جدا کند، چنانکه پیه بره ذبیحه سلامتی جدا می‌شود، و کاهن آن را بر مذبح بر هدایای آتشین خداوند بسوزاند، و کاهن برای او بجهت گناهی که کرده است کفاره خواهد کرد و آمرزیده خواهد شد. 
\par 35 و همه پیه آن را جدا کند، چنانکه پیه بره ذبیحه سلامتی جدا می‌شود، و کاهن آن را بر مذبح بر هدایای آتشین خداوند بسوزاند، و کاهن برای او بجهت گناهی که کرده است کفاره خواهد کرد و آمرزیده خواهد شد.
 
\chapter{5}

\par 1 بشنود، و او شاهد باشد خواه دیده و خواه دانسته، اگر اطلاع ندهد گناه او را متحمل خواهدبود.
\par 2 یا کسی‌که هر چیز نجس را لمس کند، خواه لاش وحش نجس، خواه لاش بهیمه نجس، خواه لاش حشرات نجس، و از او مخفی باشد، پس نجس و مجرم می‌باشد.
\par 3 یا اگر نجاست آدمی را لمس کند، از هر نجاست او که به آن نجس می‌شود، و از وی مخفی باشد، چون معلوم شد آنگاه مجرم خواهد بود.
\par 4 و اگر کسی غفلت به لبهای خود قسم خورد برای کردن کار بد یا کارنیک، یعنی در هر چیزی که آدمی غفلت قسم خورد، و از او مخفی باشد، چون بر او معلوم شودآنگاه در هر کدام که باشد مجرم خواهد بود.
\par 5 وچون در هر کدام از اینها مجرم شد، آنگاه به آن چیزی که در آن گناه کرده است اعتراف بنماید.
\par 6 و قربانی جرم خود را برای گناهی که کرده است نزد خداوند بیاورد، یعنی ماده‌ای از گله بره‌ای یابزی بجهت قربانی گناه، و کاهن برای وی گناهش را کفاره خواهد کرد.
\par 7 و اگر دست او به قیمت بره نرسد، پس قربانی جرم خود را برای گناهی که کرده است دو فاخته یا دو جوجه کبوتر نزدخداوند بیاورد، یکی برای قربانی گناه و دیگری برای قربانی سوختنی.
\par 8 و آنها را نزد کاهن بیاورد، و او آن را که برای قربانی گناه است اول بگذراند وسرش را از گردنش بکند و آن را دو پاره نکند،
\par 9 وقدری از خون قربانی گناه را بر پهلوی مذبح بپاشد، و باقی خون بر بنیان مذبح افشرده شود. این قربانی گناه است.
\par 10 و دیگری را برای قربانی سوختنی موافق قانون بگذراند، و کاهن برای وی گناهش را که کرده است کفاره خواهد کرد وآمرزیده خواهدشد.
\par 11 و اگر دستش به دو فاخته یا دو جوجه کبوتر نرسد، آنگاه قربانی خود رابرای گناهی که کرده است ده‌یک ایفه آرد نرم بجهت قربانی گناه بیاورد، و روغن برآن ننهد وکندر برآن نگذارد زیرا قربانی گناه است.
\par 12 و آن را نزد کاهن بیاورد و کاهن یک مشت از آن را برای یادگاری گرفته، بر هدایای آتشین خداوند برمذبح بسوزاند. این قربانی گناه است.
\par 13 و کاهن برای وی گناهش را که در هرکدام از اینها کرده است کفاره خواهد کرد، و آمرزیده خواهد شد، ومثل هدیه آردی از آن کاهن خواهد بود.»
\par 14 و خداوند موسی را خطاب کرده، گفت:
\par 15 «اگر کسی خیانت ورزد، و درباره چیزهای مقدس خداوند سهو گناه کند، آنگاه قربانی جرم خود را قوچی بی‌عیب از گله نزد خداوند موافق برآورد، وبه مثقالهای نقره مطابق مثقال قدس بیاورد، واین قربانی جرم است.
\par 16 وبه عوض نقصانی که در چیز مقدس رسانیده است عوض بدهد، و پنج یک بر آن اضافه کرده، و آن را به کاهن بدهد و کاهن برای وی به قوچ قربانی جرم کفاره خواهد کرد، و آمرزیده خواهد شد.
\par 17 واگر کسی گناه کند و کاری از جمیع نواهی خداوند که نباید کرد بکند، و آن را نداند، پس مجرم است و متحمل گناه خود خواهد بود.
\par 18 وقوچی بی‌عیب از گله موافق برآورد و نزد کاهن بیاورد، و کاهن برای وی غفلت او را که کرده است کفاره خواهد کرد، و آمرزیده خواهد شد.این قربانی جرم است البته نزد خداوند مجرم می‌باشد.»
\par 19 این قربانی جرم است البته نزد خداوند مجرم می‌باشد.»
 
\chapter{6}

\par 1 و خداوند موسی را خطاب کرده، گفت:
\par 2 «اگر کسی گناه کند، و خیانت به خداوندورزد، و به همسایه خود دروغ گوید، درباره امانت یا رهن یا چیز دزدیده شده، یا مال همسایه خود را غصب نماید،
\par 3 یا چیز گمشده را یافته، درباره آن دروغ گوید، و قسم دروغ بخورد، در هرکدام از کارهایی که شخصی در آنها گناه کند.
\par 4 پس چون گناه ورزیده، مجرم شود، آنچه را که دزدیده یا آنچه را غصب نموده یا آنچه نزد او به امانت سپرده شده یا آن چیز گم شده را که یافته است، رد بنماید.
\par 5 یا هر‌آنچه را که درباره آن قسم دروغ خورده، هم اصل مال را رد بنماید، وهم پنج یک آن را برآن اضافه کرده، آن را به مالکش بدهد، در روزی که جرم او ثابت شده باشد.
\par 6 و قربانی جرم خود را نزد خداوند بیاورد، یعنی قوچ بی‌عیب از گله موافق برآورد تو برای قربانی جرم نزد کاهن.
\par 7 و کاهن برای وی به حضور خداوند کفاره خواهد کرد و آمرزیده خواهد شد، از هر کاری که کرده، و در آن مجرم شده است.»
\par 8 و خداوند موسی را خطاب کرده، گفت:
\par 9 «هارون و پسرانش را امر فرموده، بگو: این است قانون قربانی سوختنی: که قربانی سوختنی تمامی شب تا صبح بر آتشدان مذبح باشد، و آتش مذبح بر آن افروخته بماند.
\par 10 و کاهن لباس کتان خود رابپوشد، و زیرجامه کتان بر بدن خود بپوشد، وخاکستر قربانی سوختنی را که بر مذبح به آتش سوخته شده، بردارد و آن را به یک طرف مذبح بگذارد.
\par 11 و لباس خود را بیرون کرده، لباس دیگر بپوشد، و خاکستر را بیرون لشکرگاه به‌جای پاک ببرد.
\par 12 و آتشی که بر مذبح است افروخته باشد، و خاموش نشود و هر بامداد کاهن هیزم برآن بسوزاند، و قربانی سوختنی را بر آن مرتب سازد، و پیه ذبیحه سلامتی را بر آن بسوزاند،
\par 13 وآتش بر مذبح پیوسته افروخته باشد، و خاموش نشود.
\par 14 «و این است قانون هدیه آردی: پسران هارون آن را به حضور خداوند بر مذبح بگذرانند.
\par 15 و از آن یک مشت از آرد نرم هدیه آردی و ازروغنش با تمامی کندر که بر هدیه آردی است بردارد، و بر مذبح بسوزاند، برای عطر خوشبو ویادگاری آن نزد خداوند.
\par 16 و باقی آن را هارون و پسرانش بخورند. بی‌خمیرمایه در مکان قدس خورده شود، در صحن خیمه اجتماع آن رابخورند.
\par 17 با خمیرمایه پخته نشود، آن را ازهدایای آتشین برای قسمت ایشان داده‌ام، این قدس اقداس است مثل قربانی گناه و مثل قربانی جرم.
\par 18 جمیع ذکوران از پسران هارون آن رابخورند. این فریضه ابدی در نسلهای شما ازهدایای آتشین خداوند است، هر‌که آنها را لمس کند مقدس خواهد بود.»
\par 19 و خداوند موسی را خطاب کرده، گفت:
\par 20 «این است قربانی هارون و پسرانش که در روزمسح کردن او نزد خداوند بگذرانند، ده‌یک ایفه آرد نرم برای هدیه آردی دائمی، نصفش در صبح و نصفش در شام،
\par 21 و بر ساج با روغن ساخته شود و چون آمیخته شد آن را بیاور و آن را به پاره های برشته شده برای هدیه آردی بجهت عطر خوشبو نزد خداوند بگذران.
\par 22 و کاهن مسح شده که از پسرانش در جای او خواهد بودآن را بگذراند. این است فریضه ابدی که تمامش نزد خداوند سوخته شود.
\par 23 و هر هدیه آردی کاهن تمام سوخته شود و خورده نشود.»
\par 24 و خداوند موسی را خطاب کرده، گفت:
\par 25 «هارون و پسرانش را خطاب کرده، بگو: این است قانون قربانی گناه، در جایی که قربانی سوختنی ذبح می‌شود، قربانی گناه نیز به حضورخداوند ذبح شود. این قدس اقداس است.
\par 26 وکاهنی که آن را برای گناه می‌گذراند آن را بخورد، در مکان مقدس، در صحن خیمه اجتماع خورده شود.
\par 27 هر‌که گوشتش را لمس کند مقدس شود، و اگر خونش بر جامه‌ای پاشیده شود آنچه را که بر آن پاشیده شده است در مکان مقدس بشوی.
\par 28 و ظرف سفالین که در آن پخته شود شکسته شود و اگر در ظرف مسین پخته شود زدوده، و به آب شسته شود.
\par 29 و هر ذکوری از کاهنان آن رابخورد، این قدس اقداس است.و هیچ قربانی گناه که از خون آن به خیمه اجتماع درآورده شودتا در قدس کفاره نماید خورده نشود، به آتش سوخته شود.
\par 30 و هیچ قربانی گناه که از خون آن به خیمه اجتماع درآورده شودتا در قدس کفاره نماید خورده نشود، به آتش سوخته شود.
 
\chapter{7}

\par 1 «و این است قانون قربانی جرم؛ این اقدس اقداس است.
\par 2 در جایی که قربانی سوختنی را ذبح کنند، قربانی جرم را نیز ذبح بکنند، و خونش را به اطراف مذبح بپاشند.
\par 3 و ازآن همه پیه‌اش را بگذراند، دنبه و پیه که احشا رامی پوشاند.
\par 4 و دو گرده و پیهی که بر آنهاست که بر دو تهیگاه است، و سفیدی را که بر جگر است، با گرده‌ها جدا کند.
\par 5 و کاهن آنها را برای هدیه آتشین بجهت خداوند بسوزاند. این قربانی جرم است.
\par 6 و هر ذکوری از کاهنان آن را بخورد، درمکان مقدس خورده شود. این قدس اقداس است.
\par 7 «قربانی جرم مانند قربانی گناه است. آنها رایک قانون است. کاهنی که به آن کفاره کند از آن اوخواهد بود.
\par 8 و کاهنی که قربانی سوختنی کسی را گذراند، آن کاهن پوست قربانی سوختنی را که گذرانید برای خود نگه دارد.
\par 9 و هر هدیه آردی که در تنور پخته شود و هر‌چه بر تابه یا ساج ساخته شود از آن کاهن که آن راگذرانید خواهد بود.
\par 10 و هر هدیه آردی، خواه به روغن سرشته شده، خواه خشک، از آن همه پسران هارون بی‌تفاوت یکدیگر خواهدبود.
\par 11 «و این است قانون ذبیحه سلامتی که کسی نزد خداوند بگذراند.
\par 12 اگر آن را برای تشکربگذراند پس با ذبیحه تشکر، قرصهای فطیرسرشته شده به روغن، و نازکهای فطیر مالیده شده به روغن، و از آرد نرم آمیخته شده، قرصهای سرشته شده به روغن را بگذراند.
\par 13 با قرصهای نان خمیر مایه دار قربانی خود را همراه ذبیحه تشکر سلامتی خود بگذراند.
\par 14 و از آن از هرقربانی یکی را برای هدیه افراشتنی نزد خداوندبگذراند، و از آن آن کاهن که خون ذبیحه سلامتی را می‌پاشد خواهد بود.
\par 15 و گوشت ذبیحه تشکرسلامتی او در روز قربانی وی خورده شود، چیزی از آن را تا صبح نگذارد.
\par 16 و اگر ذبیحه قربانی او نذری یا تبرعی باشد، در روزی که ذبیحه خود را می‌گذراند خورده شود، و باقی آن در فردای آن روز خورده شود.
\par 17 و باقی گوشت ذبیحه در روز سوم به آتش سوخته شود.
\par 18 و اگرچیزی از گوشت ذبیحه سلامتی او در روز سوم خورده شود مقبول نخواهد شد و برای کسی‌که آن را گذرانید محسوب نخواهد شد، نجس خواهد بود. و کسی‌که آن را بخورد گناه خود رامتحمل خواهد شد.
\par 19 و گوشتی که به هر چیزنجس برخورد، خورده نشود، به آتش سوخته شود، و هر‌که طاهر باشد از آن گوشت بخورد.
\par 20 لیکن کسی‌که از گوشت ذبیحه سلامتی که برای خداوند است بخورد و نجاست او بر اوباشد، آن کس از قوم خود منقطع خواهد شد.
\par 21 وکسی‌که هر چیز نجس را خواه نجاست آدمی، خواه بهیمه نجس، خواه هر چیز مکروه نجس رالمس کند، و از گوشت ذبیحه سلامتی که برای خداوند است بخورد، آن کس از قوم خود منقطع خواهد شد.»
\par 22 و خداوند موسی را خطاب کرده، گفت:
\par 23 «بنی‌اسرائیل را خطاب کرده، بگو: هیچ پیه گاوو گوسفند و بز را مخورید.
\par 24 اما پیه مردار و پیه حیوان دریده شده برای هر کار استعمال می‌شود، لیکن هرگز خورده نشود.
\par 25 زیرا هر‌که پیه جانوری که از آن هدیه آتشین برای خداوندمی گذرانند بخورد، آن کس که خورد، از قوم خودمنقطع شود.
\par 26 و هیچ خون را خواه از مرغ خواه از بهایم در همه مسکنهای خود مخورید.
\par 27 هرکسی‌که از هر قسم خون بخورد، آن کس از قوم خود منقطع خواهد شد.»
\par 28 و خداوند موسی را خطاب کرده، گفت:
\par 29 «بنی‌اسرائیل را خطاب کرده، بگو: هر‌که ذبیحه سلامتی خود را برای خداوند بگذراند، قربانی خود را از ذبیحه سلامتی خود نزد خداوندبیاورد.
\par 30 به‌دستهای خود هدایای آتشین خداوند را بیاورد، پیه را با سینه بیاورد تا سینه بجهت هدیه جنبانیدنی به حضور خداوندجنبانیده شود.
\par 31 و کاهن پیه را بر مذبح بسوزاند، و سینه از آن هارون و پسرانش خواهد بود.
\par 32 وران راست را برای هدیه افراشتنی از ذبایح سلامتی خود به کاهن بدهید.
\par 33 آن کس از پسران هارون که خون ذبیحه سلامتی وپیه را گذرانید، ران راست حصه وی خواهد بود.
\par 34 زیرا سینه جنبانیدنی و ران افراشتنی را از بنی‌اسرائیل از ذبایح سلامتی ایشان گرفتم، و آنها را به هارون کاهن و پسرانش به فریضه ابدی از جانب بنی‌اسرائیل دادم.»
\par 35 این است حصه مسح هارون و حصه مسح پسرانش از هدایای آتشین خداوند، در روزی که ایشان را نزدیک آورد تابرای خداوند کهانت کنند. 
\par 36 که خداوند امرفرمود که به ایشان داده شود، در روزی که ایشان را از میان بنی‌اسرائیل مسح کرد، این فریضه ابدی در نسلهای ایشان است.
\par 37 این است قانون قربانی سوختنی و هدیه آردی و قربانی گناه و قربانی جرم و قربانی تقدیس و ذبیحه سلامتی،که خداوند به موسی در کوه سینا امر فرموده بود، درروزی که بنی‌اسرائیل را مامور فرمود تاقربانی های خود را نزد خداوند بگذرانند درصحرای سینا.
\par 38 که خداوند به موسی در کوه سینا امر فرموده بود، درروزی که بنی‌اسرائیل را مامور فرمود تاقربانی های خود را نزد خداوند بگذرانند درصحرای سینا.
 
\chapter{8}

\par 1 و خداوند موسی را خطاب کرده، گفت:
\par 2 «هارون و پسرانش را با او و رختها وروغن مسح و گوساله قربانی گناه و دو قوچ وسبدنان فطیر را بگیر.
\par 3 و تمامی جماعت را به درخیمه اجتماع جمع کن.»
\par 4 پس موسی چنانکه خداوند به وی امر فرموده بود به عمل آورد، وجماعت به در خیمه اجتماع جمع شدند.
\par 5 وموسی به جماعت گفت: «این است آنچه خداوندفرموده است که کرده شود.»
\par 6 پس موسی هارون و پسرانش را نزدیک آورد، و ایشان را به آب غسل داد.
\par 7 و پیراهن را بر او پوشانید و کمربند رابر او بست، و او را به ردا ملبس ساخت، و ایفود رابر او گذاشت و زنار ایفود را بر او بسته، آن را بر وی استوار ساخت
\par 8 و سینه بند را بر او گذاشت واوریم و تمیم را در سینه بند گذارد.
\par 9 و عمامه را برسرش نهاد، و بر عمامه در‌پیش آن تنکه زرین، یعنی افسر مقدس را نهاد، چنانکه خداوند موسی را امر فرموده بود.
\par 10 و موسی روغن مسح را گرفته، مسکن وآنچه را که در آن بود مسح کرده، آنها را تقدیس نمود.
\par 11 و قدری از آن را بر مذبح هفت مرتبه پاشید، و مذبح و همه اسبابش و حوض و پایه‌اش را مسح کرد، تا آنها را تقدیس نماید.
\par 12 و قدری از روغن مسح را بر سر هارون ریخته، او را مسح کرد تا او را تقدیس نماید.
\par 13 و موسی پسران هارون را نزدیک آورده، بر ایشان پیراهنها راپوشانید و کمربندها را بر ایشان بست و کلاهها رابر ایشان نهاد، چنانکه خداوند موسی را امرفرموده بود.
\par 14 پس گوساله قربانی گناه را آورد، و هارون وپسرانش دستهای خود را بر سر گوساله قربانی گناه نهادند.
\par 15 و آن را ذبح کرد، و موسی خون راگرفته، بر شاخهای مذبح به هر طرف به انگشت خود مالید، و مذبح را طاهر ساخت، و خون را بربنیان مذبح ریخته، آن را تقدیس نمود تا برایش کفاره نماید.
\par 16 و همه پیه را که بر احشا بود وسفیدی جگر و دو گرده و پیه آنها را گرفت، وموسی آنها را بر مذبح سوزانید
\par 17 و گوساله وپوستش و گوشتش و سرگینش را بیرون ازلشکرگاه به آتش سوزانید، چنانکه خداوندموسی را امر فرموده بود.
\par 18 پس قوچ قربانی سوختنی را نزدیک آورد، و هارون و پسرانش دستهای خود را بر سر قوچ نهادند.
\par 19 و آن را ذبح کرد، و موسی خون را به اطراف مذبح پاشید.
\par 20 وقوچ را قطعه قطعه کرد، و موسی سر و قطعه‌ها و چربی را سوزانید.
\par 21 و احشا و پاچه‌ها را به آب شست و موسی تمامی قوچ را بر مذبح سوزانید.
\par 22 پس قوچ دیگر یعنی قوچ تخصیص را نزدیک آورد، و هارون وپسرانش دستهای خود را بر سر قوچ نهادند.
\par 23 وآن را ذبح کرد، و موسی قدری از خونش را گرفته، بر نرمه گوش راست هارون و بر شست دست راست او، و بر شست پای راست او مالید.
\par 24 وپسران هارون را نزدیک آورد، و موسی قدری ازخون را بر نرمه گوش راست ایشان، و بر شست دست راست ایشان، و بر شست پای راست ایشان مالید، و موسی خون را به اطراف مذبح پاشید.
\par 25 و پیه و دنبه و همه پیه را که بر احشاست، وسفیدی جگر و دو گرده و پیه آنها و ران راست راگرفت.
\par 26 و از سبد نان فطیر که به حضور خداوندبود، یک قرص فطیر و یک قرص نان روغنی ویک نازک گرفت، و آنها را بر پیه و بر ران راست نهاد.
\par 27 و همه را بر دست هارون و بر دستهای پسرانش نهاد. و آنها را برای هدیه جنبانیدنی به حضور خداوند بجنبانید.
\par 28 و موسی آنها را ازدستهای ایشان گرفته، بر مذبح بالای قربانی سوختنی سوزانید. این هدیه تخصیص برای عطرخوشبو و قربانی آتشین بجهت خداوند بود.
\par 29 وموسی سینه را گرفته، آن را به حضور خداوندبرای هدیه جنبانیدنی جنبانید، و از قوچ تخصیص، این حصه موسی بود چنانکه خداوندموسی را امر فرموده بود.
\par 30 و موسی قدری از روغن مسح و از خونی که بر مذبح بود گرفته، آن را بر هارون و رختهایش و بر پسرانش و رختهای پسرانش با وی پاشید، و هارون و رختهایش و پسرانش و رختهای پسرانش را با وی تقدیس نمود.
\par 31 و موسی هارون و پسرانش را گفت: «گوشت را نزد درخیمه اجتماع بپزید و آن را با نانی که در سبدتخصیص است در آنجا بخورید، چنانکه امرفرموده، گفتم که هارون و پسرانش آن را بخورند.
\par 32 و باقی گوشت و نان را به آتش بسوزانید.
\par 33 واز در خیمه اجتماع هفت روز بیرون مروید تاروزی که ایام تخصیص شما تمام شود، زیرا که در هفت روز شما را تخصیص خواهد کرد.
\par 34 چنانکه امروز کرده شده است، همچنان خداوند امر فرمود که بشود تا برای شما کفاره گردد.
\par 35 پس هفت روز نزد در خیمه اجتماع روزو شب بمانید، و امر خداوند را نگاه دارید مبادابمیرید، زیرا همچنین مامور شده‌ام.»و هارون و پسرانش همه کارهایی را که خداوند به‌دست موسی‌امر فرموده بود بجا آوردند.
\par 36 و هارون و پسرانش همه کارهایی را که خداوند به‌دست موسی‌امر فرموده بود بجا آوردند.
 
\chapter{9}

\par 1 و واقع شد که در روز هشتم، موسی هارون و پسرانش و مشایخ اسرائیل را خواند.
\par 2 وهارون را گفت: «گوساله‌ای نرینه برای قربانی گناه، و قوچی بجهت قربانی سوختنی، هر دو رابی عیب بگیر، و به حضور خداوند بگذران.
\par 3 وبنی‌اسرائیل را خطاب کرده، بگو: بزغاله نرینه برای قربانی گناه، و گوساله و بره‌ای هر دو یک ساله و بی‌عیب برای قربانی سوختنی بگیرید.
\par 4 وگاوی و قوچی برای ذبیحه سلامتی، تا به حضورخداوند ذبح شود، و هدیه آردی سرشته شده به روغن را، زیرا که امروز خداوند بر شما ظاهر خواهد شد.»
\par 5 پس آنچه را که موسی‌امر فرموده بود پیش خیمه اجتماع آوردند. و تمامی جماعت نزدیک شده، به حضور خداوندایستادند.
\par 6 و موسی گفت: «این است کاری که خداوند امر فرموده است که بکنید، و جلال خداوند بر شما ظاهر خواهد شد.»
\par 7 و موسی هارون را گفت: «نزدیک مذبح بیا و قربانی گناه خود و قربانی سوختنی خود را بگذران، و برای خود و برای قوم کفاره کن، و قربانی قوم را بگذران و بجهت ایشان کفاره کن، چنانکه خداوند امرفرموده است.»
\par 8 و هارون به مذبح نزدیک آمده، گوساله قربانی گناه را که برای خودش بود ذبح کرد.
\par 9 وپسران هارون خون را نزد او آوردند و انگشت خود را به خون فرو برده، آن را بر شاخهای مذبح مالید و خون را بر بنیان مذبح ریخت.
\par 10 و پیه وگرده‌ها و سفیدی جگر از قربانی گناه را بر مذبح سوزانید، چنانکه خداوند موسی را امر فرموده بود.
\par 11 و گوشت و پوست را بیرون لشکرگاه به آتش سوزانید.
\par 12 و قربانی سوختنی را ذبح کرد، و پسران هارون خون را به او سپردند، و آن را به اطراف مذبح پاشید.
\par 13 و قربانی را به قطعه هایش و سرش به او سپردند، و آن را بر مذبح سوزانید.
\par 14 و احشا و پاچه‌ها را شست و آنها را بر قربانی سوختنی بر مذبح سوزانید.
\par 15 و قربانی قوم رانزدیک آورد، و بز قربانی گناه را که برای قوم بودگرفته، آن را ذبح کرد و آن را مثل اولین برای گناه گذرانید.
\par 16 و قربانی سوختنی را نزدیک آورده، آن را به حسب قانون گذرانید.
\par 17 و هدیه آردی رانزدیک آورده، مشتی از آن برداشت، و آن را علاوه بر قربانی سوختنی صبح بر مذبح سوزانید.
\par 18 و گاو و قوچ ذبیحه سلامتی را که برای قوم بودذبح کرد، و پسران هارون خون را به او سپردند وآن را به اطراف مذبح پاشید.
\par 19 و پیه گاو و دنبه قوچ و آنچه احشا را می‌پوشاند و گرده‌ها وسفیدی جگر را.
\par 20 و پیه را بر سینه‌ها نهادند، وپیه را بر مذبح سوزانید.
\par 21 و هارون سینه‌ها و ران راست را برای هدیه جنبانیدنی به حضور خداوندجنبانید، چنانکه موسی‌امر فرموده بود.
\par 22 پس هارون دستهای خود را به سوی قوم برافراشته، ایشان را برکت داد، و از گذرانیدن قربانی گناه وقربانی سوختنی و ذبایح سلامتی بزیر آمد.
\par 23 وموسی و هارون به خیمه اجتماع داخل شدند، وبیرون آمده، قوم را برکت دادند و جلال خداوندبر جمیع قوم ظاهر شد.و آتش از حضورخداوند بیرون آمده، قربانی سوختنی و پیه را برمذبح بلعید، و چون تمامی قوم این را دیدند، صدای بلند کرده، به روی در‌افتادند.
\par 24 و آتش از حضورخداوند بیرون آمده، قربانی سوختنی و پیه را برمذبح بلعید، و چون تمامی قوم این را دیدند، صدای بلند کرده، به روی در‌افتادند.
 
\chapter{10}

\par 1 و ناداب و ابیهو پسران هارون، هر یکی مجمره خود را گرفته، آتش بر آنهانهادند. و بخور بر آن گذارده، آتش غریبی که ایشان را نفرموده بود، به حضور خداوند نزدیک آوردند.
\par 2 و آتش از حضور خداوند به در شده، ایشان را بلعید، و به حضور خداوند مردند.
\par 3 پس موسی به هارون گفت: «این است آنچه خداوندفرموده، و گفته است که از آنانی که به من نزدیک آیند تقدیس کرده خواهم شد، و در نظر تمامی قوم جلال خواهم یافت.» پس هارون خاموش شد.
\par 4 و موسی میشائیل و الصافان، پسران عزیئیل عموی هارون را خوانده، به ایشان گفت: «نزدیک آمده، برادران خود را از پیش قدس بیرون لشکرگاه ببرید.»
\par 5 پس نزدیک آمده، ایشان را در‌پیراهنهای ایشان بیرون لشکرگاه بردند، چنانکه موسی گفته بود.
\par 6 و موسی هارون وپسرانش العازار و ایتامار را گفت: «مویهای سرخود را باز مکنید و گریبان خود را چاک مزنیدمبادا بمیرید. و غضب بر تمامی جماعت بشود. اما برادران شما یعنی تمام خاندان اسرائیل بجهت آتشی که خداوند افروخته است ماتم خواهند کرد.
\par 7 و از در خیمه اجتماع بیرون مروید مبادا بمیرید، زیرا روغن مسح خداوند برشماست.» پس به حسب آنچه موسی گفت، کردند.
\par 8 و خداوند هارون را خطاب کرده، گفت:
\par 9 «توو پسرانت با تو چون به خیمه اجتماع داخل شوید شراب و مسکری منوشید مبادا بمیرید. این است فریضه ابدی در نسلهای شما.
\par 10 و تا درمیان مقدس و غیرمقدس و نجس و طاهر تمیزدهید،
\par 11 و تا به بنی‌اسرائیل همه فرایضی را که خداوند به‌دست موسی برای ایشان گفته است، تعلیم دهید.»
\par 12 و موسی به هارون و پسرانش العازار و ایتامار که باقی بودند گفت: «هدیه آردی که از هدایای آتشین خداوند مانده است بگیرید، و آن را بی‌خمیرمایه نزد مذبح بخورید زیراقدس اقداس است.
\par 13 و آن را در مکان مقدس بخورید زیرا که از هدایای آتشین خداوند این حصه تو و حصه پسران توست چنانکه مامورشده‌ام.
\par 14 و سینه جنبانیدنی و ران افراشتنی را تو و پسرانت و دخترانت با تو در جای پاک بخورید، زیرا اینها از ذبایح سلامتی بنی‌اسرائیل برای حصه تو و حصه پسرانت داده شده است.
\par 15 ران افراشتنی و سینه جنبانیدنی را با هدایای آتشین پیه بیاورند، تا هدیه جنبانیدنی به حضور خداوندجنبانیده شود، و از آن تو و از آن پسرانت خواهدبود، به فریضه ابدی چنانکه خداوند امر فرموده است.»
\par 16 و موسی بز قربانی گناه را طلبید و اینک سوخته شده بود، پس بر العازار و ایتامار پسران هارون که باقی بودند خشم نموده، گفت:
\par 17 «چراقربانی گناه را در مکان مقدس نخوردید؟ زیرا که آن قدس اقداس است، و به شما داده شده بود تاگناه جماعت را برداشته، برای ایشان به حضورخداوند کفاره کنید.
\par 18 اینک خون آن به اندرون قدس آورده نشد، البته می‌بایست آن را در قدس خورده باشید، چنانکه امر کرده بودم.»
\par 19 هارون به موسی گفت: «اینک امروز قربانی گناه خود وقربانی سوختنی خود را به حضور خداوندگذرانیدند، و چنین چیزها بر من واقع شده است، پس اگر امروز قربانی گناه را می‌خوردم آیا منظورنظر خداوند می‌شد؟»چون موسی این راشنید، در نظرش پسند آمد. 
\par 20 چون موسی این راشنید، در نظرش پسند آمد.
 
\chapter{11}

\par 1 و خداوند موسی و هارون را خطاب کرده، به ایشان گفت:
\par 2 «بنی‌اسرائیل راخطاب کرده، بگویید: اینها حیواناتی هستند که می‌باید بخورید، از همه بهایمی که بر روی زمین‌اند.
\par 3 هر شکافته سم که شکاف تمام دارد ونشخوار کننده‌ای از بهایم، آن را بخورید.
\par 4 اما ازنشخوارکنندگان و شکافتگان سم اینها رامخورید، یعنی شتر، زیرا نشخوار می‌کند لیکن شکافته سم نیست، آن برای شما نجس است.
\par 5 وونک، زیرا نشخوار می‌کند اما شکافته سم نیست، این برای شما نجس است.
\par 6 و خرگوش، زیرانشخوار می‌کند ولی شکافته سم نیست، این برای شما نجس است.
\par 7 و خوک، زیرا شکافته سم است و شکاف تمام دارد لیکن نشخوار نمی کند، این برای شما نجس است.
\par 8 از گوشت آنهامخورید و لاش آنها را لمس مکنید، اینها برای شما نجس‌اند.
\par 9 از همه آنچه در آب است اینها رابخورید، هر‌چه پر و فلس دارد در آب خواه دردریا خواه در نهرها، آنها را بخورید.
\par 10 و هر‌چه پر و فلس ندارد در دریا یا در نهرها، از همه حشرات آب و همه جانورانی که در آب می‌باشند، اینها نزد شما مکروه باشند.
\par 11 البته نزدشما مکروه‌اند، از گوشت آنها مخورید و لاشهای آنها را مکروه دارید.
\par 12 هر‌چه در آبها پر و فلس ندارد نزد شما مکروه خواهد بود.
\par 13 و از مرغان اینها را مکروه دارید، خورده نشوند، زیرامکروه‌اند، عقاب و استخوان خوار و نسربحر.
\par 14 وکرکس و لاشخوار به اجناس آن.
\par 15 و غراب به اجناس آن.
\par 16 و شترمرغ و جغد و مرغ دریایی وباز به اجناس آن.
\par 17 و بوم و غواص و بوتیمار.
\par 18 و قاز و مرغ سقا و رخم.
\par 19 و لقلق و کلنگ به اجناس آن و هدهد و شبپره.
\par 20 و همه حشرات بالدار که بر چهارپا می‌روند برای شما مکروه‌اند.
\par 21 لیکن اینها را بخورید از همه حشرات بالدار که بر چهار پا می‌روند، هر کدام که بر پایهای خودساقها برای جستن بر زمین دارند.
\par 22 از آن قسم اینها را بخورید. ملخ به اجناس آن و دبا به اجناس آن و حرجوان به اجناس آن و حدب به اجناس آن.
\par 23 و سایر حشرات بالدار که چهار پا دارندبرای شما مکروه‌اند.
\par 24 از آنها نجس می‌شوید، هرکه لاش آنها را لمس کند تا شام نجس باشد.
\par 25 و هر‌که چیزی از لاش آنها را بردارد، رخت خود را بشوید و تا شام نجس باشد.
\par 26 و هربهیمه‌ای که شکافته سم باشد لیکن شکاف تمام ندارد و نشخوار نکند اینها برای شما نجسند، وهر‌که آنها را لمس کند نجس است.
\par 27 و هر‌چه برکف پا رود از همه جانورانی که بر چهار پامی روند، اینها برای شما نجس‌اند، هر‌که لاش آنها را لمس کند تا شام نجس باشد.
\par 28 و هر‌که لاش آنها را بردارد، رخت خود را بشوید و تا شام نجس باشد. اینها برای شما نجس‌اند.
\par 29 «و از حشراتی که بر زمین می‌خزند اینهابرای شما نجس‌اند: موش کور و موش وسوسمار به اجناس آن،
\par 30 و دله و ورل و چلپاسه و کرباسه و بوقلمون.
\par 31 از جمیع حشرات اینهابرای شما نجس‌اند: هر‌که لاش آنها را لمس کندتا شام نجس باشد،
\par 32 و بر هر چیزی که یکی ازاینها بعد از موتش بیفتد نجس باشد، خواه هرظرف چوبی، خواه رخت، خواه چرم، خواه جوال؛ هر ظرفی که در آن کار کرده شود در آب گذاشته شود و تا شام نجس باشد، پس طاهرخواهد بود.
\par 33 و هر ظرف سفالین که یکی از اینهادر آن بیفتد آنچه در آن است نجس باشد و آن رابشکنید.
\par 34 هر خوراک در آن که خورده شود، اگرآب بر آن ریخته شد نجس باشد، و هر مشروبی که آشامیده شود که در چنین ظرف است نجس باشد.
\par 35 و بر هر چیزی که پاره‌ای از لاش آنهابیفتد نجس باشد، خواه تنور، خواه اجاق، شکسته شود؛ اینها نجس‌اند و نزد شما نجس خواهند بود.
\par 36 و چشمه و حوض که مجمع آب باشد طاهر است لیکن هر‌که لاش آنها را لمس کند نجس خواهد بود.
\par 37 و اگر پاره‌ای از لاش آنها بر تخم کاشتنی که باید کاشته شود بیفتدطاهر است.
\par 38 لیکن اگر آب بر تخم ریخته شود وپاره‌ای از لاش آنها بر آن بیفتد، این برای شمانجس باشد.
\par 39 و اگر یکی از بهایمی که برای شماخوردنی است بمیرد، هر‌که لاش آن را لمس کندتا شام نجس باشد.
\par 40 و هر‌که لاش آن را بخوردرخت خود را بشوید و تا شام نجس باشد. و هر‌که لاش آن را بردارد، رخت خود را بشوید و تا شام نجس باشد.
\par 41 «و هر حشراتی که بر زمین می‌خزد مکروه است؛ خورده نشود.
\par 42 و هر‌چه بر شکم راه رودو هر‌چه بر چهارپا راه رود و هر‌چه پایهای زیاده دارد، یعنی همه حشراتی که بر زمین می‌خزند، آنها را مخورید زیرا که مکروه‌اند.
\par 43 خویشتن رابه هر حشراتی که می‌خزد مکروه مسازید، و خودرا به آنها نجس مسازید، مبادا از آنها ناپاک شوید.
\par 44 زیرا من یهوه خدای شما هستم، پس خود را تقدیس نمایید و مقدس باشید، زیرا من قدوس هستم پس خویشتن را به همه حشراتی که بر زمین می‌خزند نجس مسازید.
\par 45 زیرا من یهوه هستم که شما را از زمین مصر بیرون آوردم تا خدای شما باشم، پس مقدس باشید زیرامن قدوس هستم.
\par 46 این است قانون بهایم ومرغان و هر حیوانی که در آبها حرکت می‌کندو هر حیوانی که بر زمین می‌خزد.تا در میان نجس و طاهر و در میان حیواناتی که خورده شوند و حیواناتی که خورده نشوند امتیازبشود.»
\par 47 تا در میان نجس و طاهر و در میان حیواناتی که خورده شوند و حیواناتی که خورده نشوند امتیازبشود.»
 
\chapter{12}

\par 1 و خداوند موسی را خطاب کرده، گفت:
\par 2 «بنی‌اسرائیل را خطاب کرده، بگو: چون زنی آبستن شده، پسر نرینه‌ای بزاید، آنگاه هفت روز نجس باشد، موافق ایام طمث حیضش نجس باشد.
\par 3 و در روز هشتم گوشت غلفه اومختون شود.
\par 4 و سی و سه روز در خون تطهیرخود بماند، و هیچ‌چیز مقدس را لمس ننماید، وبه مکان مقدس داخل نشود، تا ایام طهرش تمام شود.
\par 5 و اگر دختری بزاید، دو هفته برحسب مدت طمث خود نجس باشد، و شصت و شش روز در خون تطهیر خود بماند.
\par 6 و چون ایام طهرش برای پسر یا دختر تمام شود، بره‌ای یک ساله برای قربانی سوختنی و جوجه کبوتر یافاخته‌ای برای قربانی گناه به در خیمه اجتماع نزدکاهن بیاورد.
\par 7 و او آن را به حضور خداوندخواهد گذرانید، و برایش کفاره خواهد کرد، تا ازچشمه خون خود طاهر شود. این است قانون آن که بزاید، خواه پسر خواه دختر.و اگر دست اوبه قیمت بره نرسد، آنگاه دو فاخته یا دو جوجه کبوتر بگیرد، یکی برای قربانی سوختنی ودیگری برای قربانی گناه. و کاهن برای وی کفاره خواهد کرد، و طاهر خواهد شد.»
\par 8 و اگر دست اوبه قیمت بره نرسد، آنگاه دو فاخته یا دو جوجه کبوتر بگیرد، یکی برای قربانی سوختنی ودیگری برای قربانی گناه. و کاهن برای وی کفاره خواهد کرد، و طاهر خواهد شد.»
 
\chapter{13}

\par 1 و خداوند موسی و هارون را خطاب کرده، گفت:
\par 2 «چون شخصی را درپوست بدنش آماس یا قوبا یا لکه‌ای براق بشود، وآن در پوست بدنش مانند بلای برص باشد، پس او را نزد هارون کاهن یا نزد یکی از پسرانش که کهنه باشند بیاورند.
\par 3 و کاهن آن بلا را که درپوست بدنش باشد ملاحظه نماید. اگر مو در بلاسفید گردیده است، و نمایش بلا از پوست بدنش گودتر باشد، بلای برص است، پس کاهن او راببیند و حکم به نجاست او بدهد.
\par 4 و اگر آن لکه براق در پوست بدنش سفید باشد، و از پوست گودتر ننماید، و موی آن سفید نگردیده، آنگاه کاهن آن مبتلا را هفت روز نگاه دارد.
\par 5 و روزهفتم کاهن او را ملاحظه نماید، و اگر آن بلا درنظرش ایستاده باشد، و بلا در پوست پهن نشده، پس کاهن او را هفت روز دیگر نگاه دارد.
\par 6 و درروز هفتم کاهن او را باز ملاحظه کند، و اگر بلا کم رنگ شده، و در پوست پهن نگشته است، کاهن حکم به طهارتش بدهد. آن قوبا است. رخت خودرا بشوید و طاهر باشد.
\par 7 و اگر قوبا در پوست پهن شود بعد از آن که خود را به کاهن برای تطهیرنمود، پس بار دیگر خود را به کاهن بنماید.
\par 8 وکاهن ملاحظه نماید، و هرگاه قوبا در پوست پهن شده باشد، حکم به نجاست او بدهد. این برص است.
\par 9 و چون بلای برص در کسی باشد او را نزدکاهن بیاورند.
\par 10 و کاهن ملاحظه نماید اگر آماس سفید در پوست باشد، و موی را سفید کرده، وگوشت خام زنده در آماس باشد،
\par 11 این درپوست بدنش برص مزمن است. کاهن به نجاستش حکم دهد و او را نگاه ندارد زیرا که نجس است.
\par 12 و اگر برص در پوست بسیار پهن شده باشد وبرص، تمامی پوست آن مبتلا را از سر تا پا هرجایی که کاهن بنگرد، پوشانیده باشد،
\par 13 پس کاهن ملاحظه نماید اگر برص تمام بدن را فروگرفته است، به تطهیر آن مبتلا حکم دهد. چونکه همه بدنش سفید شده است، طاهر است.
\par 14 لیکن هر وقتی که گوشت زنده در او ظاهر شود، نجس خواهد بود.
\par 15 و کاهن گوشت زنده را ببیند وحکم به نجاست او بدهد. این گوشت زنده نجس است زیرا که برص است.
\par 16 و اگر گوشت زنده به سفیدی برگردد نزد کاهن بیاید.
\par 17 و کاهن او راملاحظه کند و اگر آن بلا به سفیدی مبدل شده است، پس کاهن به طهارت آن مبتلا حکم دهدزیرا طاهر است.
\par 18 «و گوشتی که در پوست آن دمل باشد وشفا یابد،
\par 19 و در جای دمل آماس سفید یا لکه براق سفید مایل به‌سرخی پدید آید، آن را به کاهن بنماید.
\par 20 و کاهن آن را ملاحظه نماید و اگراز پوست گودتر بنماید و موی آن سفید شده، پس کاهن به نجاست او حکم دهد. این بلای برص است که از دمل درآمده است.
\par 21 و اگر کاهن آن راببیند و اینک موی سفید در آن نباشد و گودتر ازپوست هم نباشد و کم رنگ باشد، پس کاهن او راهفت روز نگاه دارد.
\par 22 و اگر در پوست پهن شده، کاهن به نجاست او حکم دهد. این بلا می‌باشد.
\par 23 و اگر آن لکه براق در جای خود مانده، پهن نشده باشد، این گری دمل است. پس کاهن به طهارت وی حکم دهد.
\par 24 یا گوشتی که درپوست آن داغ آتش باشد و از گوشت زنده آن داغ، لکه براق سفید مایل به‌سرخی یا سفید پدیدآید،
\par 25 پس کاهن آن را ملاحظه نماید. اگر مو درلکه براق سفید گردیده، و گودتر از پوست بنمایداین برص است که از داغ درآمده است. پس کاهن به نجاست او حکم دهد زیرا بلای برص است.
\par 26 و اگر کاهن آن را ملاحظه نماید و اینک در لکه براق موی سفید نباشد و گودتر از پوست نباشد وکم رنگ باشد، کاهن او را هفت روز نگه دارد.
\par 27 و در روز هفتم کاهن او را ملاحظه نماید. اگر درپوست پهن شده، کاهن به نجاست وی حکم دهد. این بلای برص است.
\par 28 و اگر لکه براق در جای خود مانده، در پوست پهن نشده باشد و کم رنگ باشد، این آماس داغ است. پس کاهن به طهارت وی حکم دهد. این گری داغ است.
\par 29 «و چون مرد یا زن، بلایی در سر یا در زنخ داشته باشد،
\par 30 کاهن آن بلا را ملاحظه نماید. اگرگودتر از پوست بنماید و موی زرد باریک در آن باشد، پس کاهن به نجاست او حکم دهد. این سعفه یعنی برص سر یا زنخ است.
\par 31 و چون کاهن بلای سعفه را ببیند، اگر گودتر از پوست ننماید و موی سیاه در آن نباشد، پس کاهن آن مبتلای سعفه را هفت روز نگاه دارد.
\par 32 و در روزهفتم کاهن آن بلا را ملاحظه نماید. اگر سعفه پهن نشده، و موی زرد در آن نباشد و سعفه گودتر ازپوست ننماید،
\par 33 آنگاه موی خود را بتراشد لیکن سعفه را نتراشد و کاهن آن مبتلای سعفه را بازهفت روز نگاه دارد.
\par 34 و در روز هفتم کاهن سعفه را ملاحظه نماید. اگر سعفه در پوست پهن نشده، و از پوست گودتر ننماید، پس کاهن حکم به طهارت وی دهد و او رخت خود را بشوید وطاهر باشد.
\par 35 لیکن اگر بعد از حکم به طهارتش سعفه در پوست پهن شود،
\par 36 پس کاهن او راملاحظه نماید. اگر سعفه در پوست پهن شده باشد، کاهن موی زرد را نجوید، او نجس است.
\par 37 اما اگر در نظرش سعفه ایستاده باشد، و موی سیاه از آن در‌آمده، پس سعفه شفا یافته است. اوطاهر است و کاهن حکم به طهارت وی بدهد. 
\par 38 «و چون مرد یا زن در پوست بدن خودلکه های براق یعنی لکه های براق سفید داشته باشد،
\par 39 کاهن ملاحظه نماید. اگر لکه‌ها در پوست بدن ایشان کم رنگ و سفید باشد، این بهق است که از پوست درآمده. او طاهر است.
\par 40 وکسی‌که موی سر او ریخته باشد، او اقرع است، وطاهر می‌باشد.
\par 41 و کسی‌که موی سر او از طرف پیشانی ریخته باشد، او اصلع است، و طاهرمی باشد.
\par 42 و اگر در سر کل او یا پیشانی کل اوسفید مایل به‌سرخی باشد، آن برص است که ازسر کل او یا پیشانی کل او در‌آمده است.
\par 43 پس کاهن او را ملاحظه کند. اگر آماس آن بلا در سرکل او یا پیشانی کل او سفید مایل به‌سرخی، مانندبرص در پوست بدن باشد،
\par 44 او مبروص است، ونجس می‌باشد. کاهن البته حکم به نجاست وی بدهد. بلای وی در سرش است.
\par 45 و اما مبروص که این بلا را دارد، گریبان او چاک شده، و موی سراو گشاده، و شاربهای او پوشیده شود، و ندا کندنجس نجس.
\par 46 و همه روزهایی که بلا دارد، البته نجس خواهد بود، و تنها بماند و مسکن او بیرون لشکرگاه باشد.
\par 47 «و رختی که بلای برص داشته باشد، خواه رخت پشمین خواه رخت پنبه‌ای،
\par 48 خواه در تارو خواه در پود، چه از پشم و چه از پنبه و چه ازچرم، یا از هر چیزی که از چرم ساخته شود،
\par 49 اگر آن بلا مایل به سبزی یا به‌سرخی باشد، دررخت یا در چرم، خواه در تار خواه در پود یا درهر ظرف چرمی، این بلای برص است. به کاهن نشان داده شود.
\par 50 و کاهن آن بلا را ملاحظه نماید و آن چیزی را که بلا دارد هفت روز نگاه دارد.
\par 51 و آن چیزی را که بلا دارد، در روز هفتم ملاحظه کند. اگر آن بلا در رخت پهن شده باشد، خواه در تار خواه در پود، یا در چرم در هر کاری که چرم برای آن استعمال می‌شود، این برص مفسد است و آن چیز نجس می‌باشد.
\par 52 پس آن رخت را بسوزاند، چه تار و چه پود، خواه در پشم خواه در پنبه، یا در هر ظرف چرمی که بلا در آن باشد، زیرا برص مفسد است. به آتش سوخته شود.
\par 53 اما چون کاهن آن را ملاحظه کند، اگر بلادر رخت، خواه در تار خواه در پود، یا در هرظرف چرمی پهن نشده باشد،
\par 54 پس کاهن امرفرماید تا آنچه را که بلا دارد بشویند، و آن راهفت روز دیگر نگاه دارد.
\par 55 و بعد از شستن آن چیز که بلا دارد کاهن ملاحظه نماید. اگر رنگ آن بلا تبدیل نشده، هر‌چند بلا هم پهن نشده باشد، این نجس است. آن را به آتش بسوزان. این خوره است، خواه فرسودگی آن در درون باشد یا دربیرون.
\par 56 و چون کاهن ملاحظه نماید، اگر بلا بعداز شستن آن کم رنگ شده باشد، پس آن را ازرخت یا از چرم خواه از تار خواه از پود، پاره کند.
\par 57 و اگر باز در آن رخت خواه در تار خواه در پود، یا در هر ظرف چرمی ظاهر شود، این برآمدن برص است. آنچه را که بلا دارد به آتش بسوزان.
\par 58 و آن رخت خواه تار و خواه پود، یا هر ظرف چرمی را که شسته‌ای و بلا از آن رفع شده باشد، دوباره شسته شود و طاهر خواهد بود.»این است قانون بلای برص در رخت پشمین یا پنبه‌ای خواه در تار خواه در پود، یا در هر ظرف چرمی برای حکم به طهارت یا نجاست آن.
\par 59 این است قانون بلای برص در رخت پشمین یا پنبه‌ای خواه در تار خواه در پود، یا در هر ظرف چرمی برای حکم به طهارت یا نجاست آن.
 
\chapter{14}

\par 1 و خداوند موسی را خطاب کرده، گفت:
\par 2 «این است قانون مبروص: در روزتطهیرش نزد کاهن آورده شود.
\par 3 و کاهن بیرون لشکرگاه برود و کاهن ملاحظه کند. اگر بلای برص از مبروص رفع شده باشد،
\par 4 کاهن حکم بدهد که برای آن کسی‌که باید تطهیر شود، دوگنجشک زنده طاهر، و چوب ارز و قرمز و زوفابگیرند.
\par 5 و کاهن امر کند که یک گنجشک را درظرف سفالین بر بالای آب روان بکشند.
\par 6 و اماگنجشک زنده را با چوب ارز و قرمز و زوفا بگیردو آنها را با گنجشک زنده به خون گنجشکی که برآب روان کشته شده، فرو برد.
\par 7 و بر کسی‌که ازبرص باید تطهیر شود هفت مرتبه بپاشد، و حکم به طهارتش بدهد. و گنجشک زنده را به سوی صحرا رها کند.
\par 8 و آن کس که باید تطهیر شودرخت خود را بشوید، و تمامی موی خود رابتراشد، و به آب غسل کند، و طاهر خواهد شد. وبعد از آن به لشکرگاه داخل شود، لیکن تا هفت روز بیرون خیمه خود بماند.
\par 9 و در روز هفتم تمامی موی خود را بتراشد از سر و ریش وآبروی خود، یعنی تمامی موی خود را بتراشد ورخت خود را بشوید و بدن خود را به آب غسل دهد. پس طاهر خواهد بود.
\par 10 «و در روز هشتم دو بره نرینه بی‌عیب، ویک بره ماده‌یک ساله بی‌عیب، و سه عشر آرد نرم سرشته شده به روغن، برای هدیه آردی، و یک لج روغن بگیرد.
\par 11 و آن کاهن که او را تطهیر می‌کند، آن کس را که باید تطهیر شود، با این چیزها به حضور خداوند نزد در خیمه اجتماع حاضر کند.
\par 12 و کاهن یکی از بره های نرینه را گرفته، آن را باآن لج روغن برای قربانی جرم بگذراند. و آنها رابرای هدیه جنبانیدنی به حضور خداوند بجنباند.
\par 13 و بره را در جایی که قربانی گناه و قربانی سوختنی را ذبح می‌کنند، در مکان مقدس ذبح کند، زیرا قربانی جرم مثل قربانی گناه از آن کاهن است. این قدس اقداس است.
\par 14 و کاهن از خون قربانی جرم بگیرد، و کاهن آن را بر نرمه گوش راست کسی‌که باید تطهیر شود، و بر شست دست راست و بر شست پای راست وی بمالد.
\par 15 وکاهن قدری از لج روغن گرفته، آن را در کف دست چپ خود بریزد.
\par 16 و کاهن انگشت راست خود را به روغنی که در کف چپ خود دارد فروبرد، و هفت مرتبه روغن را به حضور خداوندبپاشد.
\par 17 و کاهن از باقی روغن که در کف وی است بر نرمه گوش راست و بر شست دست راست و بر شست پای راست آن کس که بایدتطهیر شود، بالای خون قربانی جرم بمالد.
\par 18 وبقیه روغن را که در کف کاهن است بر سر آن کس که باید تطهیر شود بمالد و کاهن برای وی به حضور خداوند کفاره خواهد نمود.
\par 19 و کاهن قربانی گناه را بگذراند، و برای آن کس که بایدتطهیر شود نجاست او را کفاره نماید. و بعد از آن قربانی سوختنی را ذبح کند.
\par 20 و کاهن قربانی سوختنی و هدیه آردی را بر مذبح بگذراند، وبرای وی کفاره خواهد کرد، و طاهر خواهد بود.
\par 21 و اگر او فقیر باشد و دستش به اینها نرسد، پس یک بره نرینه برای قربانی جرم تا جنبانیده شود وبرای وی کفاره کند، بگیرد و یک عشر از آرد نرم سرشته شده به روغن برای هدیه آردی و یک لج روغن،
\par 22 و دو فاخته یا دو جوجه کبوتر، آنچه دستش به آن برسد، و یکی قربانی گناه و دیگری قربانی سوختنی بشود.
\par 23 «و در روز هشتم آنها را نزد کاهن به درخیمه اجتماع برای طهارت خود به حضور خداوند بیاورد.
\par 24 و کاهن بره قربانی جرم و لج روغن را بگیرد و کاهن آنها را برای هدیه جنبانیدنی به حضور خداوند بجنباند.
\par 25 وقربانی جرم را ذبح نماید و کاهن از خون قربانی جرم گرفته، بر نرمه گوش راست و شست دست راست و شست پای راست کسی‌که تطهیرمی شود بمالد.
\par 26 و کاهن قدری از روغن را به کف دست چپ خود بریزد.
\par 27 و کاهن از روغنی که دردست چپ خود دارد، به انگشت راست خودهفت مرتبه به حضور خداوند بپاشد.
\par 28 و کاهن ازروغنی که در دست دارد بر نرمه گوش راست و برشست دست راست و بر شست پای راست کسی‌که تطهیر می‌شود، بر جای خون قربانی جرم بمالد.
\par 29 و بقیه روغنی که در دست کاهن است آن را بر سر کسی‌که تطهیر می‌شود بمالد تا برای وی به حضور خداوند کفاره کند.
\par 30 و یکی از دوفاخته یا از دو جوجه کبوتر را از آنچه دستش به آن رسیده باشد بگذراند.
\par 31 یعنی هر‌آنچه دست وی به آن برسد، یکی را برای قربانی گناه ودیگری را برای قربانی سوختنی با هدیه آردی. وکاهن برای کسی‌که تطهیر می‌شود به حضورخداوند کفاره خواهد کرد.»
\par 32 این است قانون کسی‌که بلای برص دارد، و دست وی به تطهیرخود نمی رسد.
\par 33 و خداوند موسی و هارون را خطاب کرده، گفت:
\par 34 «چون به زمین کنعان که من آن را به شمابه ملکیت می‌دهم داخل شوید، و بلای برص رادر خانه‌ای از زمین ملک شما عارض گردانم،
\par 35 آنگاه صاحب‌خانه آمده، کاهن را اطلاع داده، بگوید که مرا به نظر می‌آید که مثل بلا در خانه است.
\par 36 و کاهن امر فرماید تا قبل از داخل شدن کاهن برای دیدن بلا، خانه را خالی کنند، مباداآنچه در خانه است نجس شود، و بعد از آن کاهن برای دیدن خانه داخل شود،
\par 37 و بلا را ملاحظه نماید. اگر بلا در دیوارهای خانه از خطهای مایل به سبزی یا سرخی باشد، و از سطح دیوار گودتربنماید،
\par 38 پس کاهن از خانه نزد در بیرون رود وخانه را هفت روز ببندد.
\par 39 و در روز هفتم کاهن باز بیاید و ملاحظه نماید اگر بلا در دیوارهای خانه پهن شده باشد،
\par 40 آنگاه کاهن امر فرماید تاسنگهایی را که بلا در آنهاست کنده، آنها را به‌جای ناپاک بیرون شهر بیندازند.
\par 41 و اندرون خانه را از هر طرف بتراشند و خاکی را که تراشیده باشند به‌جای ناپاک بیرون شهر بریزند.
\par 42 وسنگهای دیگر گرفته، به‌جای آن سنگها بگذارندو خاک دیگر گرفته، خانه را اندود کنند.
\par 43 و اگربلا برگردد و بعد از کندن سنگها و تراشیدن واندود کردن خانه باز در خانه بروز کند،
\par 44 پس کاهن بیاید و ملاحظه نماید. اگر بلا در خانه پهن شده باشد این برص مفسد در خانه است و آن نجس است.
\par 45 پس خانه را خراب کند باسنگهایش و چوبش و تمامی خاک خانه و به‌جای ناپاک بیرون شهر بیندازند.
\par 46 و هر‌که داخل خانه شود در تمام روزهایی که بسته باشد تا شام نجس خواهد بود.
\par 47 و هر‌که در خانه بخوابدرخت خود را بشوید و هر‌که در خانه چیزی خورد، رخت خود را بشوید.
\par 48 و چون کاهن بیاید و ملاحظه نماید اگر بعد از اندود کردن خانه بلا در خانه پهن نشده باشد، پس کاهن حکم به طهارت خانه بدهد، زیرا بلا رفع شده است.
\par 49 وبرای تطهیر خانه دو گنجشک و چوب ارز و قرمزو زوفا بگیرد.
\par 50 و یک گنجشک را در ظرف سفالین بر آب روان ذبح نماید،
\par 51 و چوب ارز وزوفا و قرمز و گنجشک زنده را گرفته، آنها را به خون گنجشک ذبح شده و آب روان فرو برد، وهفت مرتبه بر خانه بپاشد.
\par 52 و خانه را به خون گنجشک و به آب روان و به گنجشک زنده و به چوب ارز و زوفا و قرمز تطهیر نماید.
\par 53 وگنجشک زنده را بیرون شهر به سوی صحرا رهاکند، و خانه را کفاره نماید و طاهر خواهد بود.»
\par 54 این است قانون، برای هر بلای برص و برای سعفه،
\par 55 و برای برص رخت و خانه،
\par 56 و برای آماس و قوبا و لکه براق.و برای تعلیم دادن که چه وقت نجس می‌باشد و چه وقت طاهر. این قانون برص است.
\par 57 و برای تعلیم دادن که چه وقت نجس می‌باشد و چه وقت طاهر. این قانون برص است.
 
\chapter{15}

\par 1 و خداوند موسی و هارون را خطاب کرده، گفت:
\par 2 «بنی‌اسرائیل را خطاب کرده، به ایشان بگویید: مردی که جریان از بدن خود دارد او به‌سبب جریانش نجس است.
\par 3 واین است نجاستش، به‌سبب جریان او، خواه جریانش از گوشتش روان باشد خواه جریانش ازگوشتش بسته باشد. این نجاست اوست.
\par 4 هربستری که صاحب جریان بر آن بخوابد نجس است، و هر‌چه بر آن بنشیند نجس است.
\par 5 و هرکه بستر او را لمس نماید، رخت خود را بشوید، وبه آب غسل کند، و تا شام نجس باشد.
\par 6 و هر‌که بنشیند بر هر‌چه صاحب جریان بر آن نشسته بود، رخت خود را بشوید و به آب غسل کند، وتا شام نجس باشد.
\par 7 و هر‌که بدن صاحب جریان رالمس کند رخت خود را بشوید و به آب غسل کند و تا شام نجس باشد.
\par 8 و اگر صاحب جریان، بر شخص طاهر آب دهن اندازد، آن کس رخت خود را بشوید، و به آب غسل کند، و تا شام نجس باشد.
\par 9 و هر زینی که صاحب جریان بر آن سوارشود، نجس باشد.
\par 10 و هر‌که چیزی را که زیر اوبوده باشد لمس نماید تا شام نجس بوده، و هر‌که این چیزها را بردارد، رخت خود را بشوید، و به آب غسل کند و تا شام نجس باشد. 
\par 11 و هر کسی را که صاحب جریان لمس نماید، و دست خود رابه آب نشسته باشد، رخت خود را بشوید، وبه آب غسل کند و تا شام نجس باشد.
\par 12 و ظرف سفالین که صاحب جریان آن را لمس نماید، شکسته شود، و هر ظرف چوبین به آب شسته شود.
\par 13 و چون صاحب جریان از جریان خودطاهر شده باشد، آنگاه هفت روز برای تطهیرخود بشمارد، و رخت خود را بشوید و بدن خودرا به آب غسل دهد و طاهر باشد.
\par 14 و در روزهشتم دو فاخته یا دو جوجه کبوتر بگیرد، و به حضور خداوند به در خیمه اجتماع آمده، آنها رابه کاهن بدهد.
\par 15 و کاهن آنها را بگذراند، یکی برای قربانی گناه و دیگری برای قربانی سوختنی. و کاهن برای وی به حضور خداوند جریانش راکفاره خواهد کرد.
\par 16 و چون منی از کسی درآیدتمامی بدن خود را به آب غسل دهد، وتا شام نجس باشد.
\par 17 و هر رخت و هر چرمی که منی برآن باشد به آب شسته شود، و تا شام نجس باشد.
\par 18 و هر زنی که مرد با او بخوابد و انزال کند، به آب غسل کنند و تا شام نجس باشند.
\par 19 «و اگر زنی جریان دارد، و جریانی که دربدنش است خون باشد، هفت روز در حیض خودبماند. و هر‌که او را لمس نماید، تا شام نجس باشد.
\par 20 و بر هر چیزی که در حیض خود بخوابدنجس باشد، و بر هر چیزی که بنشیند نجس باشد.
\par 21 و هر‌که بستر او را لمس کند، رخت خود را بشوید، و به آب غسل کند و تا شام نجس باشد.
\par 22 و هر‌که چیزی را که او بر آن نشسته بود لمس نماید رخت خود را بشوید، و به آب غسل کند، وتا شام نجس باشد.
\par 23 و اگر آن بر بستر باشد یا برهر چیزی که او بر آن نشسته بود، چون آن چیز رالمس کند تا شام نجس باشد.
\par 24 و اگر مردی با اوهم بستر شود و حیض او بر وی باشد تا هفت روزنجس خواهد بود. و هر بستری که بر آن بخوابدنجس خواهد بود.
\par 25 و زنی که روزهای بسیار، غیر از زمان حیض خود جریان خون دارد، یازیاده از زمان حیض خود جریان دارد، تمامی روزهای جریان نجاستش مثل روزهای حیضش خواهد بود. او نجس است.
\par 26 و هر بستری که درروزهای جریان خود بر آن بخوابد، مثل بسترحیضش برای وی خواهد بود. و هر چیزی که برآن بنشیند مثل نجاست حیضش نجس خواهدبود.
\par 27 و هر‌که این چیزها را لمس نماید نجس می‌باشد. پس رخت خود را بشوید و به آب غسل کند و تا شام نجس باشد.
\par 28 و اگر از جریان خودطاهر شده باشد، هفت روز برای خود بشمارد، وبعد از آن طاهر خواهد بود.
\par 29 و در روز هشتم دوفاخته یا دو جوجه کبوتر بگیرد، و آنها را نزد کاهن به در خیمه اجتماع بیاورد.
\par 30 و کاهن یکی رابرای قربانی گناه و دیگری را برای قربانی سوختنی بگذراند. و کاهن برای وی نجاست جریانش را به حضور خداوند کفاره کند.
\par 31 پس بنی‌اسرائیل را از نجاست ایشان جدا خواهیدکرد، مبادا مسکن مرا که در میان ایشان است نجس سازند و در نجاست خود بمیرند.»
\par 32 این است قانون کسی‌که جریان دارد، و کسی‌که منی از وی درآید و از آن نجس شده باشد.و حایض درحیضش و هر‌که جریان دارد خواه مرد خواه زن، و مردی که با زن نجس همبستر شود.
\par 33 و حایض درحیضش و هر‌که جریان دارد خواه مرد خواه زن، و مردی که با زن نجس همبستر شود.
 
\chapter{16}

\par 1 و خداوند موسی را بعد از مردن دو پسرهارون، وقتی که نزد خداوند آمدند ومردند خطاب کرده، گفت:
\par 2 «پس خداوند به موسی گفت: برادر خود هارون را بگو که به قدس درون حجاب پیش کرسی رحمت که بر تابوت است همه وقت داخل نشود، مبادا بمیرد، زیرا که در ابر بر کرسی رحمت ظاهر خواهم شد.
\par 3 و بااین چیزها هارون داخل قدس بشود، با گوساله‌ای برای قربانی گناه، و قوچی برای قربانی سوختنی.
\par 4 و پیراهن کتان مقدس را بپوشد، و زیر جامه کتان بر بدنش باشد، و به کمربند کتان بسته شود، و به عمامه کتان معمم باشد. اینها رخت مقدس است. پس بدن خود را به آب غسل داده، آنها را بپوشد.
\par 5 و از جماعت بنی‌اسرائیل دو بز نرینه برای قربانی گناه، و یک قوچ برای قربانی سوختنی بگیرد.
\par 6 و هارون گوساله قربانی گناه را که برای خود اوست بگذراند، و برای خود و اهل خانه خود کفاره نماید.
\par 7 و دو بز را بگیرد و آنها را به حضور خداوند به در خیمه اجتماع حاضر سازد.
\par 8 و هارون بر آن دو بز قرعه اندازد، یک قرعه برای خداوند و یک قرعه برای عزازیل.
\par 9 و هارون بزی را که قرعه برای خداوند بر آن برآمد نزدیک بیاورد، و بجهت قربانی گناه بگذراند.
\par 10 و بزی که قرعه برای عزازیل بر آن برآمد به حضور خداوندزنده حاضر شود، و بر آن کفاره نماید و آن را برای عزازیل به صحرا بفرستد.
\par 11 «و هارون گاو قربانی گناه را که برای خوداوست نزدیک بیاورد، و برای خود و اهل خانه خود کفاره نماید، و گاو قربانی گناه را که برای خود اوست ذبح کند.
\par 12 و مجمری پر از زغال آتش از روی مذبح که به حضور خداوند است ودو مشت پر از بخور معطر کوبیده شده برداشته، به اندرون حجاب بیاورد.
\par 13 و بخور را بر آتش به حضور خداوند بنهد تا ابر بخور کرسی رحمت راکه بر تابوت شهادت است بپوشاند، مبادا بمیرد.
\par 14 و از خون گاو گرفته، بر کرسی رحمت به انگشت خود به طرف مشرق بپاشد، و قدری ازخون را پیش روی کرسی رحمت هفت مرتبه بپاشد.
\par 15 پس بز قربانی گناه را که برای قوم است ذبح نماید، و خونش را به اندرون حجاب بیاورد، وبا خونش چنانکه با خون گاو عمل کرده بودعمل کند، و آن را بر کرسی رحمت و پیش روی کرسی رحمت بپاشد.
\par 16 و برای قدس کفاره نماید به‌سبب نجاسات بنی‌اسرائیل، و به‌سبب تقصیرهای ایشان با تمامی گناهان ایشان، و برای خیمه اجتماع که با ایشان در میان نجاسات ایشان ساکن است، همچنین بکند.
\par 17 و هیچکس درخیمه اجتماع نباشد، و از وقتی که برای کردن کفاره داخل قدس بشود تا وقتی که بیرون آید، پس برای خود و برای اهل خانه خود و برای تمامی جماعت اسرائیل کفاره خواهد کرد.
\par 18 پس نزد مذبح که به حضور خداوند است بیرون آید، و برای آن کفاره نماید، و از خون گاو و ازخون بز گرفته، آن را بر شاخه های مذبح بهر طرف بپاشد.
\par 19 و قدری از خون را به انگشت خود هفت مرتبه بر آن بپاشد و آن را تطهیر کند، و آن رااز نجاسات بنی‌اسرائیل تقدیس نماید.
\par 20 «و چون از کفاره نمودن برای قدس و برای خیمه اجتماع و برای مذبح فارغ شود، آنگاه بززنده را نزدیک بیاورد.
\par 21 و هارون دو دست خودرا بر سر بز زنده بنهد، و همه خطایای بنی‌اسرائیل و همه تقصیرهای ایشان را با همه گناهان ایشان اعتراف نماید، و آنها را بر سر بز بگذارد و آن را به‌دست شخص حاضر به صحرا بفرستد.
\par 22 و بزهمه گناهان ایشان را به زمین ویران بر خودخواهد برد. پس بز را به صحرا رها کند.
\par 23 وهارون به خیمه اجتماع داخل شود، و رخت کتان را که در وقت داخل شدن به قدس پوشیده بودبیرون کرده، آنها را در آنجا بگذارد.
\par 24 و بدن خود را در جای مقدس به آب غسل دهد، ورخت خود را پوشیده، بیرون آید، و قربانی سوختنی خود و قربانی سوختنی قوم را بگذراند، و برای خود و برای قوم کفاره نماید.
\par 25 و پیه قربانی گناه را بر مذبح بسوزاند.
\par 26 و آنکه بز رابرای عزازیل رها کرد رخت خود را بشوید و بدن خود را به آب غسل دهد، و بعد از آن به لشکرگاه داخل شود.
\par 27 و گاو قربانی گناه و بز قربانی گناه را که خون آنها به قدس برای کردن کفاره آورده شد، بیرون لشکرگاه برده شود، و پوست و گوشت و سرگین آنها را به آتش بسوزانند.
\par 28 و آنکه آنهارا سوزانید رخت خود را بشوید و بدن خود را به آب غسل دهد، و بعد از آن به لشکرگاه داخل شود.
\par 29 «و این برای شما فریضه دائمی باشد، که درروز دهم از ماه هفتم جانهای خود را ذلیل سازید، و هیچ کار مکنید، خواه متوطن خواه غریبی که درمیان شما ماوا گزیده باشد.
\par 30 زیرا که در آن روزکفاره برای تطهیر شما کرده خواهد شد، و ازجمیع گناهان خود به حضور خداوند طاهرخواهید شد.
\par 31 این سبت آرامی برای شماست، پس جانهای خود را ذلیل سازید. این است فریضه دائمی.
\par 32 و کاهنی که مسح شده، و تخصیص شده باشد، تا در جای پدر خود کهانت نمایدکفاره را بنماید. و رختهای کتان یعنی رختهای مقدس را بپوشد.
\par 33 و برای قدس مقدس کفاره نماید، و برای خیمه اجتماع و مذبح کفاره نماید، و برای کهنه و تمامی جماعت قوم کفاره نماید.و این برای شما فریضه دائمی خواهد بود تابرای بنی‌اسرائیل از تمامی گناهان ایشان یک مرتبه هر سال کفاره شود.» پس چنانکه خداوندموسی را امر فرمود، همچنان بعمل آورد.
\par 34 و این برای شما فریضه دائمی خواهد بود تابرای بنی‌اسرائیل از تمامی گناهان ایشان یک مرتبه هر سال کفاره شود.» پس چنانکه خداوندموسی را امر فرمود، همچنان بعمل آورد.
 
\chapter{17}

\par 1 و خداوند موسی را خطاب کرده، گفت:
\par 2 «هارون و پسرانش و جمیع بنی‌اسرائیل را خطاب کرده، به ایشان بگو: این است کاری که خداوند می‌فرماید و می‌گوید:
\par 3 هرشخصی از خاندان اسرائیل که گاو یا گوسفند یا بزدر لشکرگاه ذبح نماید، یا آنکه بیرون لشکرگاه ذبح نماید،
\par 4 و آن را به در خیمه اجتماع نیاورد، تاقربانی برای خداوند پیش مسکن خداوندبگذراند، بر آن شخص خون محسوب خواهدشد. او خون ریخته است و آن شخص از قوم خود منقطع خواهد شد.
\par 5 تا آنکه بنی‌اسرائیل ذبایح خود را که در صحرا ذبح می‌کنند بیاورند، یعنی برای خداوند به در خیمه اجتماع نزد کاهن آنها را بیاورند، و آنها را بجهت ذبایح سلامتی برای خداوند ذبح نمایند.
\par 6 و کاهن خون را برمذبح خداوند نزد در خیمه اجتماع بپاشد، و پیه را بسوزاند تا عطر خوشبو برای خداوند شود.
\par 7 وبعد از این، ذبایح خود را برای دیوهایی که درعقب آنها زنا می‌کنند دیگر ذبح ننمایند. این برای ایشان در پشتهای ایشان فریضه دائمی خواهدبود.
\par 8 «و ایشان را بگو: هر کس از خاندان اسرائیل و از غریبانی که در میان شما ماوا گزینند که قربانی سوختنی یا ذبیحه بگذراند،
\par 9 و آن را به درخیمه اجتماع نیاورد، تا آن را برای خداوندبگذراند. آن شخص از قوم خود منقطع خواهدشد.
\par 10 و هر کس از خاندان اسرائیل یا از غریبانی که در میان شما ماوا گزینند که هر قسم خون رابخورد، من روی خود را بر آن شخصی که خون خورده باشد برمی گردانم، و او را از میان قومش منقطع خواهم ساخت.
\par 11 زیرا که جان جسد درخون است، و من آن را بر مذبح به شما داده‌ام تابرای جانهای شما کفاره کند، زیرا خون است که برای جان کفاره می‌کند.
\par 12 بنابراین بنی‌اسرائیل را گفته‌ام: هیچکس از شما خون نخورد و غریبی که در میان شما ماوا گزیند خون نخورد.
\par 13 و هرشخص از بنی‌اسرائیل یا از غریبانی که در میان شما ماوا گزینند، که هر جانور یا مرغی را که خورده می‌شود صید کند، پس خون آن را بریزد وبه خاک بپوشاند.
\par 14 زیرا جان هر ذی جسد خون آن و جان آن یکی است، پس بنی‌اسرائیل راگفته‌ام خون هیچ ذی جسد را مخورید، زیرا جان هر ذی جسد خون آن است، هر‌که آن را بخوردمنقطع خواهد شد.
\par 15 و هر کسی از متوطنان یا ازغریبانی که میته یا دریده شده‌ای بخورد، رخت خود را بشوید، و به آب غسل کند و تا شام نجس باشد. پس طاهر خواهد شد.و اگر آن را نشویدو بدن خود را غسل ندهد، متحمل گناه خودخواهد بود.»
\par 16 و اگر آن را نشویدو بدن خود را غسل ندهد، متحمل گناه خودخواهد بود.»
 
\chapter{18}

\par 1 و خداوند موسی را خطاب کرده، گفت:
\par 2 «بنی‌اسرائیل را خطاب کرده، به ایشان بگو: من یهوه خدای شما هستم.
\par 3 مثل اعمال زمین مصر که در آن ساکن می‌بودید عمل منمایید، و مثل اعمال زمین کنعان که من شما را به آنجا داخل خواهم کرد عمل منمایید، و برحسب فرایض ایشان رفتار مکنید.
\par 4 احکام مرا بجاآورید و فرایض مرا نگاه دارید تا در آنها رفتارنمایید، من یهوه خدای شما هستم.
\par 5 پس فرایض و احکام مرا نگاه دارید، که هر آدمی که آنها را بجاآورد در آنها زیست خواهد کرد، من یهوه هستم. 
\par 6 هیچ‌کس به احدی از اقربای خویش نزدیکی ننماید تا کشف عورت او بکند. من یهوه هستم.
\par 7 عورت پدر خود یعنی مادر خود را کشف منما؛ او مادر توست. کشف عورت او مکن.
\par 8 عورت زن پدر خود را کشف مکن. آن عورت پدر تو است.
\par 9 عورت خواهر خود، خواه دختر پدرت، خواه دختر مادرت چه مولود در خانه، چه مولودبیرون، عورت ایشان را کشف منما.
\par 10 عورت دختر پسرت و دختر دخترت، عورت ایشان را کشف مکن، زیرا که اینها عورت تو است.
\par 11 عورت دختر زن پدرت که از پدر تو زاییده شده باشد، او خواهر تو است کشف عورت او رامکن.
\par 12 عورت خواهر پدر خود را کشف مکن، او از اقربای پدر تو است.
\par 13 عورت خواهرمادر خود را کشف مکن، او از اقربای مادر تواست.
\par 14 عورت برادر پدر خود را کشف مکن، وبه زن او نزدیکی منما. او (به منزله ) عمه تو است.
\par 15 عورت عروس خود را کشف مکن، او زن پسرتو است. عورت او را کشف مکن.
\par 16 عورت زن برادر خود را کشف مکن. آن عورت برادر تواست.
\par 17 عورت زنی را با دخترش کشف مکن. ودختر پسر او یا دختر دختر او را مگیر، تا عورت او را کشف کنی. اینان از اقربای او می‌باشند و این فجور است.
\par 18 و زنی را با خواهرش مگیر، تاهیوی او بشود، و تا عورت او را با وی مادامی که او زنده است، کشف نمایی.
\par 19 و به زنی درنجاست حیضش نزدیکی منما، تا عورت او راکشف کنی.
\par 20 و با زن همسایه خود همبسترمشو، تا خود را با وی نجس سازی.
\par 21 و کسی ازذریت خود را برای مولک از آتش مگذران و نام خدای خود را بی‌حرمت مساز. من یهوه هستم.
\par 22 و با ذکور مثل زن جماع مکن، زیرا که این فجور است.
\par 23 و با هیچ بهیمه‌ای جماع مکن، تاخود را به آن نجس سازی، و زنی پیش بهیمه‌ای نایستد تا با آن جماع کند، زیرا که این فجوراست.
\par 24 «به هیچ کدام از اینها خویشتن را نجس مسازید، زیرا به همه اینها امتهایی که من پیش روی شما بیرون می‌کنم، نجس شده‌اند.
\par 25 وزمین نجس شده است، و انتقام گناهش را از آن خواهم کشید، و زمین ساکنان خود را قی خواهدنمود.
\par 26 پس شما فرایض و احکام مرا نگاه دارید، و هیچ کدام از این فجور را به عمل نیاورید، نه متوطن و نه غریبی که در میان شما ماواگزیند.
\par 27 زیرا مردمان آن زمین که قبل از شمابودند، جمیع این فجور را کردند، و زمین نجس شده است.
\par 28 مبادا زمین شما را نیز قی کند، اگرآن را نجس سازید، چنانکه امتهایی را که قبل ازشما بودند، قی کرده است.
\par 29 زیرا هر کسی‌که یکی از این فجور را بکند، همه کسانی که کرده باشند، از میان قوم خود منقطع خواهند شد.پس وصیت مرا نگاه دارید، و از این رسوم زشت که قبل از شما به عمل آورده شده است عمل منمایید، و خود را به آنها نجس مسازید. من یهوه خدای شما هستم.»
\par 30 پس وصیت مرا نگاه دارید، و از این رسوم زشت که قبل از شما به عمل آورده شده است عمل منمایید، و خود را به آنها نجس مسازید. من یهوه خدای شما هستم.»
 
\chapter{19}

\par 1 و خداوند موسی را خطاب کرده، گفت:
\par 2 «تمامی جماعت بنی‌اسرائیل راخطاب کرده، به ایشان بگو: مقدس باشید، زیرا که من یهوه خدای شما قدوس هستم.
\par 3 هر یکی ازشما مادر و پدر خود را احترام نماید و سبت های مرا نگاه دارید، من یهوه خدای شما هستم.
\par 4 به سوی بتها میل مکنید، و خدایان ریخته شده برای خود مسازید. من یهوه خدای شما هستم.
\par 5 وچون ذبیحه سلامتی نزد خداوند بگذرانید، آن رابگذرانید تا مقبول شوید،
\par 6 در روزی که آن را ذبح نمایید. و در فردای آن روز خورده شود، و اگرچیزی از آن تا روز سوم بماند به آتش سوخته شود.
\par 7 و اگر در روز سوم خورده شود، مکروه می‌باشد، مقبول نخواهد شد.
\par 8 و هر‌که آن رابخورد، متحمل گناه خود خواهد بود، زیرا چیزمقدس خداوند را بی‌حرمت کرده است، آن کس از قوم خود را منقطع خواهد شد.
\par 9 «و چون حاصل زمین خود را درو کنید، گوشه های مزرعه خود را تمام نکنید، و محصول خود را خوشه چینی مکنید.
\par 10 و تاکستان خود رادانه چینی منما، و خوشه های ریخته شده تاکستان خود را بر مچین، آنها را برای فقیر وغریب بگذار، من یهوه خدای شما هستم.
\par 11 دزدی مکنید، و مکر منمایید، و با یکدیگردروغ مگویید.
\par 12 و به نام من قسم دروغ مخورید، که نام خدای خود را بی‌حرمت نموده باشی، من یهوه هستم.
\par 13 مال همسایه خود را غصب منما، و ستم مکن، و مزد مزدور نزد تو تا صبح نماند.
\par 14 کر را لعنت مکن، و پیش روی کور سنگ لغزش مگذار، و از خدای خود بترس، من یهوه هستم.
\par 15 در داوری بی‌انصافی مکن، و فقیر راطرفداری منما، و بزرگ را محترم مدار، و درباره همسایه خود به انصاف داوری بکن؛
\par 16 در میان قوم خود برای سخن‌چینی گردش مکن، و بر خون همسایه خود مایست. من یهوه هستم.
\par 17 برادر خود را در دل خود بغض منما، البته همسایه خودرا تنبیه کن، و به‌سبب او متحمل گناه مباش.
\par 18 ازابنای قوم خود انتقام مگیر، و کینه مورز، وهمسایه خود را مثل خویشتن محبت نما. من یهوه هستم.
\par 19 فرایض مرا نگاه دارید. بهیمه خودرا با غیر جنس به جماع وامدار؛ و مزرعه خود رابه دو قسم تخم مکار؛ و رخت از دوقسم بافته شده در بر خود مکن.
\par 20 و مردی که با زنی همبستر شود و آن زن کنیز و نامزد کسی باشد، امافدیه نداده شده، و نه آزادی به او بخشیده، ایشان را سیاست باید کرد، لیکن کشته نشوند زیرا که اوآزاد نبود.
\par 21 و مرد برای قربانی جرم خود قوچ قربانی جرم را نزد خداوند به در خیمه اجتماع بیاورد.
\par 22 و کاهن برای وی به قوچ قربانی جرم رانزد خداوند گناهش را که کرده است کفاره خواهدکرد، و او از گناهی که کرده است آمرزیده خواهدشد.
\par 23 «و چون به آن زمین داخل شدید و هر قسم درخت را برای خوراک نشاندید، پس میوه آن رامثل نامختونی آن بشمارید، سه سال برای شمانامختون باشد؛ خورده نشود.
\par 24 و در سال چهارم همه میوه آن برای تمجید خداوند مقدس خواهدبود.
\par 25 و در سال پنجم میوه آن را بخورید تامحصول خود را برای شما زیاده کند. من یهوه خدای شما هستم.
\par 26 هیچ‌چیز را با خون مخورید و تفال مزنید و شگون مکنید.
\par 27 گوشه های سر خود را متراشید، و گوشه های ریش خود را مچینید.
\par 28 بدن خود را بجهت مرده مجروح مسازید، و هیچ نشان بر خود داغ مکنید. من یهوه هستم.
\par 29 دختر خود را بی‌عصمت مساز، و او را به فاحشگی وامدار، مبادا زمین مرتکب زنا شود و زمین پر از فجور گردد.
\par 30 سبت های مرا نگاه دارید، و مکان مقدس مرامحترم دارید. من یهوه هستم.
\par 31 به اصحاب اجنه توجه مکنید، و از جادوگران پرسش منمایید، تاخود را به ایشان نجس سازید. من یهوه خدای شما هستم.
\par 32 در‌پیش ریش سفید برخیز، و روی مرد پیر را محترم دار، و از خدای خود بترس. من یهوه هستم.
\par 33 و چون غریبی با تو در زمین شما ماواگزیند، او را میازارید.
\par 34 غریبی که در میان شماماوا گزیند، مثل متوطن از شما باشد. و او را مثل خود محبت نما، زیرا که شما در زمین مصر غریب بودید. من یهوه خدای شما هستم.
\par 35 در عدل هیچ بی‌انصافی مکنید، یعنی در‌پیمایش یا دروزن یا در‌پیمانه.
\par 36 ترازوهای راست و سنگهای راست و ایفه راست و هین راست بدارید. من یهوه خدای شما هستم که شما را از زمین مصر بیرون آوردم.پس جمیع فرایض مرا و احکام مرانگاه دارید و آنها را بجا آورید. من یهوه هستم.»
\par 37 پس جمیع فرایض مرا و احکام مرانگاه دارید و آنها را بجا آورید. من یهوه هستم.»
 
\chapter{20}

\par 1 و خداوند موسی را خطاب کرده، گفت:
\par 2 «بنی‌اسرائیل را بگو: هر کسی ازبنی‌اسرائیل یا از غریبانی که در اسرائیل ماوا گزینند، که از ذریت خود به مولک بدهد، البته کشته شود؛ قوم زمین او را با سنگ سنگسار کنند.
\par 3 و من روی خود را به ضد آن شخص خواهم گردانید، و او را از میان قومش منقطع خواهم ساخت، زیرا که از ذریت خود به مولک داده است، تا مکان مقدس مرا نجس سازد، و نام قدوس مرا بی‌حرمت کند.
\par 4 و اگر قوم زمین چشمان خود را از آن شخص بپوشانند، وقتی که از ذریت خود به مولک داده باشد، و او را نکشند،
\par 5 آنگاه من روی خود را به ضد آن شخص وخاندانش خواهم گردانید، و او را و همه کسانی راکه در عقب او زناکار شده، در‌پیروی مولک زناکرده‌اند، از میان قوم ایشان منقطع خواهم ساخت.
\par 6 و کسی‌که به سوی صاحبان اجنه وجادوگران توجه نماید، تا در عقب ایشان زنا کند، من روی خود را به ضد آن شخص خواهم گردانید، و او را از میان قومش منقطع خواهم ساخت.
\par 7 پس خود را تقدیس نمایید و مقدس باشید، زیرا من یهوه خدای شما هستم.
\par 8 وفرایض مرا نگاه داشته، آنها را بجا آورید. من یهوه هستم که شما را تقدیس می‌نمایم.
\par 9 و هر کسی‌که پدر یا مادر خود را لعنت کند، البته کشته شود، چونکه پدر و مادر خود را لعنت کرده است، خونش بر خود او خواهد بود.
\par 10 و کسی‌که با زن دیگری زنا کند یعنی هر‌که با زن همسایه خود زنانماید، زانی و زانیه البته کشته شوند.
\par 11 و کسی‌که با زن پدر خود بخوابد، و عورت پدر خود راکشف نماید، هر دو البته کشته شوند. خون ایشان بر خود ایشان است.
\par 12 و اگر کسی با عروس خودبخوابد، هر دو ایشان البته کشته شوند. فاحشگی کرده‌اند خون ایشان بر خود ایشان است.
\par 13 و اگرمردی با مردی مثل با زن بخوابد هر دو فجورکرده‌اند. هر دو ایشان البته کشته شوند. خون ایشان بر خود ایشان است.
\par 14 و اگر کسی زنی ومادرش را بگیرد، این قباحت است. او و ایشان به آتش سوخته شوند، تا در میان شما قباحتی نباشد.
\par 15 و مردی که با بهیمه‌ای جماع کند، البته کشته شود و آن بهیمه را نیز بکشید.
\par 16 و زنی که به بهیمه‌ای نزدیک شود تا با آن جماع کند، آن زن و بهیمه را بکش. البته کشته شوند خون آنها برخود آنهاست.
\par 17 و کسی‌که خواهر خود را خواه دختر پدرش خواه دختر مادرش باشد بگیرد، وعورت او را ببیند و او عورت وی را ببیند، این رسوایی است. در‌پیش چشمان پسران قوم خودمنقطع شوند، چون که عورت خواهر خود راکشف کرده است. متحمل گناه خود خواهد بود.
\par 18 و کسی‌که با زن حایض بخوابد و عورت او راکشف نماید، او چشمه او را کشف کرده است واو چشمه خون خود را کشف نموده است، هردوی ایشان از میان قوم خود منقطع خواهند شد.
\par 19 و عورت خواهر مادرت یا خواهر پدرت راکشف مکن؛ آن کس خویش خود را عریان ساخته است. ایشان متحمل گناه خود خواهند بود.
\par 20 وکسی‌که با زن عموی خود بخوابد، عورت عموی خود را کشف کرده است. متحمل گناه خودخواهند بود. بی‌کس خواهند بود.
\par 21 و کسی‌که زن برادر خود را بگیرد، این نجاست است. عورت برادر خود را کشف کرده است. بی‌کس خواهندبود.
\par 22 «پس جمیع فرایض مرا و جمیع احکام مرانگاه داشته، آنها را بجا آورید، تا زمینی که من شمارا به آنجا می‌آورم تا در آن ساکن شوید، شما راقی نکند.
\par 23 و به رسوم قومهایی که من آنها را ازپیش شما بیرون می‌کنم رفتار ننمایید، زیرا که جمیع این کارها را کردند پس ایشان را مکروه داشتم.
\par 24 و به شما گفتم شما وارث این زمین خواهید بود ومن آن را به شما خواهم داد و وارث آن بشوید، زمینی که به شیر و شهد جاری است. من یهوه خدای شما هستم که شما را از امتهاامتیاز کرده‌ام.
\par 25 پس در میان بهایم طاهر و نجس، و در میان مرغان نجس و طاهر امتیاز کنید، وجانهای خود را به بهیمه یا مرغ یا به هیچ‌چیزی که بر زمین می‌خزد مکروه مسازید، که آنها رابرای شما جدا کرده‌ام تا نجس باشند.
\par 26 و برای من مقدس باشید زیرا که من یهوه قدوس هستم، وشما را از امتها امتیاز کرده‌ام تا از آن من باشید.مرد و زنی که صاحب اجنه یا جادوگر باشد، البته کشته شوند؛ ایشان را به سنگ سنگسار کنید. خون ایشان بر خود ایشان است.»
\par 27 مرد و زنی که صاحب اجنه یا جادوگر باشد، البته کشته شوند؛ ایشان را به سنگ سنگسار کنید. خون ایشان بر خود ایشان است.» 
 
\chapter{21}

\par 1 و خداوند به موسی گفت: «به کاهنان یعنی پسران هارون خطاب کرده، به ایشان بگو: کسی از شما برای مردگان، خود رانجس نسازد،
\par 2 جز برای خویشان نزدیک خود، یعنی برای مادرش و پدرش و پسرش و دخترش و برادرش.
\par 3 و برای خواهر باکره خود که قریب او باشد و شوهر ندارد؛ برای او خود را نجس تواند کرد.
\par 4 چونکه در قوم خود رئیس است، خود را نجس نسازد، تا خویشتن را بی‌عصمت نماید.
\par 5 سر خود را بی‌مو نسازند، و گوشه های ریش خود را نتراشند، و بدن خود را مجروح ننمایند.
\par 6 برای خدای خود مقدس باشند، و نام خدای خود را بی‌حرمت ننمایند. زیرا که هدایای آتشین خداوند و طعام خدای خود را ایشان می‌گذرانند. پس مقدس باشند.
\par 7 زن زانیه یابی عصمت را نکاح ننمایند، و زن مطلقه ازشوهرش را نگیرند، زیرا او برای خدای خودمقدس است.
\par 8 پس او را تقدیس نما، زیرا که اوطعام خدای خود را می‌گذراند. پس برای تومقدس باشد، زیرا من یهوه که شما را تقدیس می‌کنم، قدوس هستم.
\par 9 و دختر هر کاهنی که خود را به فاحشگی بی‌عصمت ساخته باشد، پدرخود را بی‌عصمت کرده است. به آتش سوخته شود.
\par 10 «و آن که از میان برادرانش رئیس کهنه باشد، که بر سر او روغن مسح ریخته شده، وتخصیص گردیده باشد تا لباس را بپوشد، موی سر خود را نگشاید و گریبان خود را چاک نکند،
\par 11 و نزد هیچ شخص مرده نرود، و برای پدر خودو مادر خود خویشتن را نجس نسازد.
\par 12 و ازمکان مقدس بیرون نرود، و مکان مقدس خدای خود را بی‌عصمت نسازد، زیرا که تاج روغن مسح خدای او بر وی می‌باشد. من یهوه هستم.
\par 13 و اوزن باکره‌ای نکاح کند.
\par 14 و بیوه و مطلقه وبی عصمت و زانیه، اینها را نگیرد. فقط باکره‌ای از قوم خود را به زنی بگیرد.
\par 15 و ذریت خود را درمیان قوم خود بی‌عصمت نسازد. من یهوه هستم که او را مقدس می‌سازم.»
\par 16 و خداوند موسی را خطاب کرده، گفت:
\par 17 «هارون را خطاب کرده، بگو: هر کس از اولادتو در طبقات ایشان که عیب داشته باشد نزدیک نیاید، تا طعام خدای خود را بگذراند.
\par 18 پس هرکس که عیب دارد نزدیک نیاید، نه مرد کور و نه لنگ و نه پهن بینی و نه زایدالاعضا،
\par 19 و نه کسی‌که شکسته پا یا شکسته دست باشد،
\par 20 و نه گوژپشت و نه کوتاه قد و نه کسی‌که در چشم خودلکه دارد، و نه صاحب جرب و نه کسی‌که گری دارد و نه شکسته بیضه.
\par 21 هر کس از اولاد هارون کاهن که عیب داشته باشد نزدیک نیاید، تا هدایای آتشین خداوند را بگذراند، چونکه معیوب است، برای گذرانیدن طعام خدای خود نزدیک نیاید.
\par 22 طعام خدای خود را خواه از آنچه قدس اقداس است و خواه از آنچه مقدس است، بخورد.
\par 23 لیکن به حجاب داخل نشود و به مذبح نزدیک نیاید، چونکه معیوب است، تا مکان مقدس مرا بی‌حرمت نسازد. من یهوه هستم که ایشان را تقدیس می‌کنم.»پس موسی هارون وپسرانش و تمامی بنی‌اسرائیل را چنین گفت.
\par 24 پس موسی هارون وپسرانش و تمامی بنی‌اسرائیل را چنین گفت.
 
\chapter{22}

\par 1 و خداوند موسی را خطاب کرده، گفت:
\par 2 «هارون و پسرانش را بگو که ازموقوفات بنی‌اسرائیل که برای من وقف می‌کننداحتراز نمایند، و نام قدوس مرا بی‌حرمت نسازند. من یهوه هستم.
\par 3 به ایشان بگو: هر کس از همه ذریت شما در نسلهای شما که به موقوفاتی که بنی‌اسرائیل برای خداوند وقف نمایند نزدیک بیاید، و نجاست او بر وی باشد، آن کس از حضورمن منقطع خواهد شد. من یهوه هستم.
\par 4 هر کس از ذریت هارون که مبروص یا صاحب جریان باشد تا طاهر نشود، از چیزهای مقدس نخورد، وکسی‌که هر چیزی را که از میت نجس شود لمس نماید، و کسی‌که منی از وی درآید،
\par 5 و کسی‌که هر حشرات را که از آن نجس می‌شوند لمس نماید، یا آدمی را که از او نجس می‌شوند از هرنجاستی که دارد.
\par 6 پس کسی‌که یکی از اینها رالمس نماید تا شام نجس باشد، و تا بدن خود را به آب غسل ندهد از چیزهای مقدس نخورد.
\par 7 وچون آفتاب غروب کند، آنگاه طاهر خواهد بود، و بعد از آن از چیزهای مقدس بخورد چونکه خوراک وی است.
\par 8 میته یا دریده شده را نخوردتا از آن نجس شود. من یهوه هستم.
\par 9 پس وصیت مرا نگاه دارند مبادا به‌سبب آن متحمل گناه شوند. و اگر آن را بی‌حرمت نمایند بمیرند. من یهوه هستم که ایشان را تقدیس می‌نمایم.
\par 10 هیچ غریبی چیز مقدس نخورد، و مهمان کاهن ومزدور او چیز مقدس نخورد.
\par 11 اما اگر کاهن کسی را بخرد، زرخرید او می‌باشد. او آن رابخورد و خانه زاد او نیز. هر دو خوراک او رابخورند.
\par 12 و دختر کاهن اگر منکوحه مرد غریب باشد، از هدایای مقدس نخورد.
\par 13 و دختر کاهن که بیوه یا مطلقه بشود و اولاد نداشته، به خانه پدرخود مثل طفولیتش برگردد، خوراک پدر خود رابخورد، لیکن هیچ غریب از آن نخورد.
\par 14 و اگرکسی سهو چیز مقدس را بخورد، پنج یک بر آن اضافه کرده، آن چیز مقدس را به کاهن بدهد.
\par 15 وچیزهای مقدس بنی‌اسرائیل را که برای خداوندمی گذرانند، بی‌حرمت نسازند.
\par 16 و به خوردن چیزهای مقدس ایشان، ایشان را متحمل جرم گناه نسازند، زیرا من یهوه هستم که ایشان راتقدیس می‌نمایم.»
\par 17 و خداوند موسی را خطاب کرده، گفت:
\par 18 «هارون و پسرانش و جمیع بنی‌اسرائیل راخطاب کرده، به ایشان بگو: هر کس از خاندان اسرائیل و از غریبانی که در اسرائیل باشند که قربانی خود را بگذراند، خواه یکی از نذرهای ایشان، خواه یکی از نوافل ایشان، که آن را برای قربانی سوختنی نزد خداوند می‌گذرانند،
\par 19 تاشما مقبول شوید. آن را نر بی‌عیب از گاو یا ازگوسفند یا از بز بگذرانید.
\par 20 هر‌چه را که عیب دارد مگذرانید، برای شما مقبول نخواهد شد.
\par 21 و اگر کسی ذبیحه سلامتی برای خداوندبگذراند، خواه برای وفای نذر، خواه برای نافله، چه از رمه چه از گله، آن بی‌عیب باشد تا مقبول بشود، البته هیچ عیب در آن نباشد.
\par 22 کور یاشکسته یا مجروح یا آبله دار یا صاحب جرب یاگری، اینها را برای خداوند مگذرانید، و از اینهاهدیه آتشین برای خداوند بر مذبح مگذارید.
\par 23 اما گاو و گوسفند که زاید یا ناقص اعضا باشد، آن را برای نوافل بگذران، لیکن برای نذر قبول نخواهد شد.
\par 24 و آنچه را که بیضه آن کوفته یافشرده یا شکسته یا بریده باشد، برای خداوندنزدیک میاورید، و در زمین خود قربانی مگذرانید.
\par 25 و از دست غریب نیز طعام خدای خود را از هیچ‌یک از اینها مگذرانید، زیرا فسادآنها در آنهاست چونکه عیب دارند، برای شمامقبول نخواهند شد.»
\par 26 و خداوند موسی را خطاب کرده، گفت:
\par 27 «چون گاو یا گوسفند یا بز زاییده شود، هفت روز نزد مادر خود بماند و در روز هشتم و بعد برای قربانی هدیه آتشین نزد خداوند مقبول خواهد شد.
\par 28 اما گاو یا گوسفند آن را با بچه‌اش در یک روز ذبح منمایید.
\par 29 و چون ذبیحه تشکربرای خداوند ذبح نمایید، آن را ذبح کنید تا مقبول شوید.
\par 30 در همان روز خورده شود و چیزی ازآن را تا صبح نگاه ندارید. من یهوه هستم.
\par 31 پس اوامر مرا نگاه داشته، آنها را بجا آورید. من یهوه هستم.
\par 32 و نام قدوس مرا بی‌حرمت مسازید ودر میان بنی‌اسرائیل تقدیس خواهم شد. من یهوه هستم که شما را تقدیس می‌نمایم.و شما را اززمین مصر بیرون آوردم تا خدای شما باشم. من یهوه هستم.»
\par 33 و شما را اززمین مصر بیرون آوردم تا خدای شما باشم. من یهوه هستم.»
 
\chapter{23}

\par 1 و خداوند موسی را خطاب کرده، گفت:
\par 2 «بنی‌اسرائیل را خطاب کرده، به ایشان بگو: موسمهای خداوند که آنها رامحفلهای مقدس خواهید خواند، اینهاموسمهای من می‌باشند.
\par 3 «شش روز کار کرده شود و در روز هفتم سبت آرامی و محفل مقدس باشد. هیچ کارمکنید. آن در همه مسکنهای شما سبت برای خداوند است.
\par 4 «اینها موسمهای خداوند و محفلهای مقدس می‌باشد، که آنها را در وقتهای آنها اعلان باید کرد.
\par 5 در ماه اول، در روز چهاردهم ماه بین العصرین، فصح خداوند است.
\par 6 و در روزپانزدهم این ماه عید فطیر برای خداوند است، هفت روز فطیر بخورید.
\par 7 در روز اول محفل مقدس برای شما باشد، هیچ کار از شغل مکنید.
\par 8 هفت روز هدیه آتشین برای خداوند بگذرانید، و در روز هفتم، محفل مقدس باشد؛ هیچ کار ازشغل مکنید.»
\par 9 و خداوند موسی را خطاب کرده، گفت:
\par 10 «بنی‌اسرائیل را خطاب کرده، به ایشان بگو: چون به زمینی که من به شما می‌دهم داخل شوید، و محصول آن را درو کنید، آنگاه بافه نوبر خود رانزد کاهن بیاورید.
\par 11 و بافه را به حضور خداوندبجنباند تا شما مقبول شوید، در فردای بعد ازسبت کاهن آن را بجنباند.
\par 12 و در روزی که شمابافه را می‌جنبانید، بره یک ساله بی‌عیب برای قربانی سوختنی به حضور خداوند بگذرانید.
\par 13 وهدیه آردی آن دو عشر آرد نرم سرشته شده به روغن خواهد بود، تا هدیه آتشین و عطر خوشبوبرای خداوند باشد، و هدیه ریختنی آن چهار یک هین شراب خواهد بود.
\par 14 و نان و خوشه های برشته شده و خوشه های تازه مخورید، تا همان روزی که قربانی خدای خود را بگذرانید. این برای پشتهای شما در همه مسکنهای شمافریضه‌ای ابدی خواهد بود.
\par 15 و از فردای آن سبت، از روزی که بافه جنبانیدنی را آورده باشید، برای خود بشمارید تا هفت هفته تمام بشود.
\par 16 تا فردای بعد از سبت هفتم، پنجاه روز بشمارید، و هدیه آردی تازه برای خداوند بگذرانید.
\par 17 از مسکنهای خود دونان جنبانیدنی از دو عشر بیاورید از آرد نرم باشد، و با خمیر مایه پخته شود تا نوبر برای خداوندباشد.
\par 18 و همراه نان، هفت بره یک ساله بی‌عیب و یک گوساله و دو قوچ، و آنها با هدیه آردی وهدیه ریختنی آنها قربانی سوختنی برای خداوندخواهدبود، و هدیه آتشین و عطر خوشبو برای خداوند.
\par 19 و یک بز نر برای قربانی گناه، و دو بره نر یک ساله برای ذبیحه سلامتی بگذرانید.
\par 20 وکاهن آنها را با نان نوبر بجهت هدیه جنبانیدنی به حضور خداوند با آن دو بره بجنباند، تا برای خداوند بجهت کاهن مقدس باشد.
\par 21 و در همان روز منادی کنید که برای شما محفل مقدس باشد؛ و هیچ کار از شغل مکنید. در همه مسکنهای شما بر پشتهای شما فریضه ابدی باشد.
\par 22 و چون محصول زمین خود را دروکنید، گوشه های مزرعه خود را تمام درو مکن، و حصاد خود را خوشه چینی منما، آنها را برای فقیر و غریب بگذار. من یهوه خدای شماهستم.»
\par 23 و خداوند موسی را خطاب کرده، گفت:
\par 24 «بنی‌اسرائیل را خطاب کرده، بگو: در ماه هفتم در روز اول ماه، آرامی سبت برای شما خواهدبود، یعنی یادگاری نواختن کرناها و محفل مقدس.
\par 25 هیچ کار از شغل مکنید و هدیه آتشین برای خداوند بگذرانید.»
\par 26 و خداوند موسی را خطاب کرده، گفت:
\par 27 «در دهم این ماه هفتم، روز کفاره است. این برای شما محفل مقدس باشد. جانهای خود راذلیل سازید، و هدیه آتشین برای خداوندبگذرانید.
\par 28 و در همان روز هیچ کار مکنید، زیراکه روز کفاره است تا برای شما به حضوریهوه خدای شما کفاره بشود.
\par 29 و هر کسی‌که درهمان روز خود را ذلیل نسازد، از قوم خود منقطع خواهد شد.
\par 30 و هر کسی‌که در همان روزهرگونه کاری بکند، آن شخص را از میان قوم اومنقطع خواهم ساخت.
\par 31 هیچ کار مکنید. برای پشتهای شما در همه مسکنهای شما فریضه‌ای ابدی است.
\par 32 این برای شما سبت آرامی خواهدبود، پس جانهای خود را ذلیل سازید، درشام روز نهم، از شام تا شام، سبت خود را نگاه دارید.»
\par 33 و خداوند موسی را خطاب کرده، گفت:
\par 34 «بنی‌اسرائیل را خطاب کرده، بگو: در روزپانزدهم این ماه هفتم، عید خیمه‌ها، هفت روزبرای خداوند خواهد بود.
\par 35 در روز اول، محفل مقدس باشد؛ هیچ کار از شغل مکنید.
\par 36 هفت روز هدیه آتشین برای خداوند بگذرانید، و درروز هشتم جشن مقدس برای شما باشد، و هدیه آتشین برای خداوند بگذرانید. این تکمیل عیداست؛ هیچ کار از شغل مکنید. 
\par 37 این موسمهای خداوند است که در آنها محفلهای مقدس رااعلان بکنید تا هدیه آتشین برای خداوند بگذرانید، یعنی قربانی سوختنی و هدیه آردی وذبیحه و هدایای ریختنی. مال هر روز را درروزش،
\par 38 سوای سبت های خداوند و سوای عطایای خود و سوای جمیع نذرهای خود وسوای همه نوافل خود که برای خداوند می‌دهید.
\par 39 در روز پانزدهم ماه هفتم چون شما محصول زمین را جمع کرده باشید، عید خداوند را هفت روز نگاه دارید، در روز اول، آرامی سبت خواهدبود، و در روز هشتم آرامی سبت.
\par 40 و در روزاول میوه درختان نیکو برای خود بگیرید، وشاخه های خرما و شاخه های درختان پربرگ، وبیدهای نهر، و به حضور یهوه خدای خود هفت روز شادی نمایید.
\par 41 و آن را هر سال هفت روزبرای خداوند عید نگاه دارید، برای پشتهای شمافریضه‌ای ابدی است که در ماه هفتم آن را عیدنگاه دارید.
\par 42 هفت روز در خیمه‌ها ساکن باشید؛ همه متوطنان در اسرائیل در خیمه‌ها ساکن شوند.
\par 43 تا طبقات شما بدانند که من بنی‌اسرائیل راوقتی که ایشان را از زمین مصر بیرون آوردم درخیمه‌ها ساکن گردانیدم. من یهوه خدای شماهستم.»پس موسی بنی‌اسرائیل را ازموسمهای خداوند خبر داد.
\par 44 پس موسی بنی‌اسرائیل را ازموسمهای خداوند خبر داد.
 
\chapter{24}

\par 1 و خداوند موسی را خطاب کرده، گفت
\par 2 که «بنی‌اسرائیل را امر بفرما تا روغن زیتون صاف کوبیده شده برای روشنایی بگیرند، تا چراغ را دائم روشن کنند.
\par 3 هارون آن را بیرون حجاب شهادت در خیمه اجتماع از شام تا صبح به حضور خداوند پیوسته بیاراید. در پشتهای شما فریضه ابدی است.
\par 4 چراغها را بر چراغدان طاهر، به حضور خداوند پیوسته بیاراید.
\par 5 و آردنرم بگیر و از آن دوازده گرده بپز؛ برای هر گرده دوعشر باشد.
\par 6 و آنها را به دو صف، در هر صف شش، بر میز طاهر به حضور خداوند بگذار.
\par 7 و برهر صف بخور صاف بنه، تا بجهت یادگاری برای نان و هدیه آتشین باشد برای خداوند.
\par 8 در هرروز سبت آن را همیشه به حضور خداوند بیاراید. از جانب بنی‌اسرائیل عهد ابدی خواهد بود.
\par 9 واز آن هارون و پسرانش خواهد بود تا آن را درمکان مقدس بخورند، زیرا این از هدایای آتشین خداوند به فریضه ابدی برای وی قدس اقداس خواهد بود.»
\par 10 و پسر زن اسرائیلی که پدرش مرد مصری بود در میان بنی‌اسرائیل بیرون آمد، و پسر زن اسرائیلی با مرد اسرائیلی در لشکرگاه جنگ کردند.
\par 11 و پسر زن اسرائیلی اسم را کفر گفت ولعنت کرد. پس او را نزد موسی آوردند و نام مادراو شلومیت دختر دبری از سبط دان بود.
\par 12 و او رادر زندان انداختند تا از دهن خداوند اطلاع یابند.
\par 13 و خداوند موسی را خطاب کرده، گفت:
\par 14 «آن کس را که لعنت کرده است، بیرون لشکرگاه ببر، و همه آنانی که شنیدند دستهای خود را بر سر وی بنهند، و تمامی جماعت او راسنگسار کنند.
\par 15 و بنی‌اسرائیل را خطاب کرده، بگو: هر کسی‌که خدای خود را لعنت کندمتحمل گناه خود خواهد بود.
\par 16 و هر‌که اسم یهوه را کفر گوید هرآینه کشته شود، تمامی جماعت او را البته سنگسار کنند، خواه غریب خواه متوطن. چونکه اسم را کفر گفته است کشته شود.
\par 17 و کسی‌که آدمی را بزند که بمیرد، البته کشته شود.
\par 18 و کسی‌که بهیمه‌ای را بزند که بمیرد عوض آن را بدهد، جان به عوض جان.
\par 19 وکسی‌که همسایه خود را عیب رسانیده باشدچنانکه او کرده باشد، به او کرده خواهد شد.
\par 20 شکستگی عوض شکستگی، چشم عوض چشم، دندان عوض دندان، چنانکه به آن شخص عیب رسانیده، همچنان به او رسانیده شود.
\par 21 وکسی‌که بهیمه‌ای را کشت، عوض آن را بدهد، اماکسی‌که انسان را کشت، کشته شود.
\par 22 شما رایک حکم خواهد بود، خواه غریب خواه متوطن، زیرا که من یهوه خدای شما هستم.»و موسی بنی‌اسرائیل را خبر داد، و آن را که لعنت کرده بود، بیرون لشکرگاه بردند، و او را به سنگ سنگسار کردند. پس بنی‌اسرائیل چنان‌که خداوند به موسی‌امر فرموده بود به عمل آوردند.
\par 23 و موسی بنی‌اسرائیل را خبر داد، و آن را که لعنت کرده بود، بیرون لشکرگاه بردند، و او را به سنگ سنگسار کردند. پس بنی‌اسرائیل چنان‌که خداوند به موسی‌امر فرموده بود به عمل آوردند.
 
\chapter{25}

\par 1 و خداوند موسی را در کوه سینا خطاب کرده، گفت:
\par 2 «بنی‌اسرائیل را خطاب کرده، به ایشان بگو: چون شما به زمینی که من به شما می‌دهم، داخل شوید، آنگاه زمین، سبت خداوند را نگاه بدارد.
\par 3 شش سال مزرعه خود رابکار، و شش سال تاکستان خود را پازش بکن، ومحصولش را جمع کن.
\par 4 و در سال هفتم سبت آرامی برای زمین باشد، یعنی سبت برای خداوند. مزرعه خود را مکار و تاکستان خود راپازش منما.
\par 5 آنچه از مزرعه تو خودرو باشد، درو مکن، و انگورهای مو پازش ناکرده خود رامچین، سال آرامی برای زمین باشد.
\par 6 و سبت زمین، خوراک بجهت شما خواهد بود، برای تو وغلامت و کنیزت و مزدورت و غریبی که نزد توماوا گزیند.
\par 7 و برای بهایمت و برای جانورانی که در زمین تو باشند، همه محصولش خوراک خواهد بود.
\par 8 «و برای خود هفت سبت سالها بشمار، یعنی هفت در هفت سال و مدت هفت سبت سالها برای تو چهل و نه سال خواهد بود.
\par 9 و در روز دهم ازماه هفتم در روز کفاره، کرنای بلندآواز را بگردان؛ در تمامی زمین خود کرنا را بگردان.
\par 10 سال پنجاهم را تقدیس نمایید، و در زمین برای جمیع ساکنانش آزادی را اعلان کنید. این برای شمایوبیل خواهد بود، و هر کس از شما به ملک خودبرگردد، و هر کس از شما به قبیله خود برگردد.
\par 11 این سال پنجاهم برای شما یوبیل خواهد بود. زراعت مکنید و حاصل خودروی آن را مچینید، و انگورهای مو پازش ناکرده آن را مچینید.
\par 12 چونکه یوبیل است، برای شما مقدس خواهدبود؛ محصول آن را در مزرعه بخورید.
\par 13 در این سال یوبیل هر کس از شما به ملک خود برگردد.
\par 14 و اگر چیزی به همسایه خود بفروشی یا چیزی از دست همسایه ات بخری یکدیگر را مغبون مسازید.
\par 15 برحسب شماره سالهای بعد ازیوبیل، از همسایه خود بخر و برحسب سالهای محصولش به تو بفروشد.
\par 16 برحسب زیادتی سالها قیمت آن را زیاده کن، و برحسب کمی سالها قیمتش را کم نما، زیرا که شماره حاصلها رابه تو خواهد فروخت.
\par 17 و یکدیگر را مغبون مسازید، و از خدای خود بترس. من یهوه خدای شما هستم.
\par 18 پس فرایض مرا بجا آورید واحکام مرا نگاه داشته، آنها را به عمل آورید، تا درزمین به امنیت ساکن شوید.
\par 19 و زمین بار خود راخواهد داد و به سیری خواهید خورد، و به امنیت در آن ساکن خواهید بود.
\par 20 و اگر گویید در سال هفتم چه بخوریم، زیرا اینک نمی کاریم و حاصل خود را جمع نمی کنیم،
\par 21 پس در سال ششم برکت خود را بر شما خواهم فرمود، و محصول سه سال خواهد داد.
\par 22 و در سال هشتم بکارید واز محصول کهنه تا سال نهم بخورید. تا حاصل آن برسد، کهنه را بخورید.
\par 23 و زمین به فروش ابدی نرود زیرا زمین از آن من است، و شما نزد من غریب و مهمان هستید.
\par 24 و در تمامی زمین ملک خود برای زمین فکاک بدهید.
\par 25 اگر برادر تو فقیرشده، بعضی از ملک خود را بفروشد، آنگاه ولی او که خویش نزدیک او باشد بیاید، و آنچه را که برادرت می‌فروشد، انفکاک نماید.
\par 26 و اگر کسی ولی ندارد و برخوردار شده، قدر فکاک آن را پیدانماید.
\par 27 آنگاه سالهای فروش آن را بشمارد وآنچه را که زیاده است به آنکس که فروخته بود، رد نماید، و او به ملک خود برگردد.
\par 28 و اگرنتواند برای خود پس بگیرد، آنگاه آنچه فروخته است به‌دست خریدار آن تا سال یوبیل بماند، ودر یوبیل رها خواهد شد، و او به ملک خودخواهد برگشت.
\par 29 «و اگر کسی خانه سکونتی در شهرحصاردار بفروشد، تا یک سال تمام بعد ازفروختن آن حق انفکاک آن را خواهد داشت، مدت انفکاک آن یک سال خواهد بود.
\par 30 و اگر در مدت یک سال تمام آن را انفکاک ننماید، پس آن خانه‌ای که در شهر حصاردار است، برای خریدارآن نسلا بعد نسل برقرار باشد، در یوبیل رهانشود.
\par 31 لیکن خانه های دهات که حصار گردخود ندارد، با مزرعه های آن زمین شمرده شود. برای آنها حق انفکاک هست و در یوبیل رهاخواهد شد.
\par 32 و اما شهرهای لاویان، خانه های شهرهای ملک ایشان، حق انفکاک آنها همیشه برای لاویان است.
\par 33 و اگر کسی از لاویان بخرد، پس آنچه فروخته شده است از خانه یا از شهرملک او در یوبیل رها خواهد شد، زیرا خانه های شهرهای لاویان در میان بنی‌اسرائیل، ملک ایشان است.
\par 34 و مزرعه های حوالی شهرهای ایشان فروخته نشود، زیرا که این برای ایشان ملک ابدی است.
\par 35 «و اگر برادرت فقیر شده، نزد تو تهی‌دست باشد، او را مثل غریب و مهمان دستگیری نما تا باتو زندگی نماید.
\par 36 از او ربا و سود مگیر و ازخدای خود بترس، تا برادرت با تو زندگی نماید.
\par 37 نقد خود را به او به ربا مده و خوراک خود را به او به سود مده.
\par 38 من یهوه خدای شما هستم که شما را از زمین مصر بیرون آوردم تا زمین کنعان رابه شما دهم و خدای شما باشم.
\par 39 و اگر برادرت نزد تو فقیر شده، خود را به تو بفروشد، بر او مثل غلام خدمت مگذار.
\par 40 مثل مزدور و مهمان نزدتو باشد و تا سال یوبیل نزد تو خدمت نماید.
\par 41 آنگاه از نزد تو بیرون رود، خود او و پسرانش همراه وی، و به خاندان خود برگردد و به ملک پدران خود رجعت نماید.
\par 42 زیرا که ایشان بندگان منند که ایشان را از زمین مصر بیرون آوردم؛ مثل غلامان فروخته نشوند.
\par 43 بر او به سختی حکم رانی منما و از خدای خود بترس.
\par 44 و اما غلامانت و کنیزانت که برای توخواهندبود، از امتهایی که به اطراف تو می‌باشنداز ایشان غلامان و کنیزان بخرید.
\par 45 و هم ازپسران مهمانانی که نزد شما ماوا گزینند، و ازقبیله های ایشان که نزد شما باشند، که ایشان را درزمین شما تولید نمودند، بخرید و مملوک شماخواهندبود.
\par 46 و ایشان را بعد از خود برای پسران خود واگذارید، تا ملک موروثی باشند وایشان را تا به ابد مملوک سازید. و اما برادران شمااز بنی‌اسرائیل هیچکس بر برادر خود به سختی حکمرانی نکند.
\par 47 «و اگر غریب یا مهمانی نزد شما برخوردارگردد، و برادرت نزد او فقیر شده، به آن غریب یامهمان تو یا به نسل خاندان آن غریب، خود رابفروشد،
\par 48 بعد از فروخته شدنش برای وی حق انفکاک می‌باشد. یکی از برادرانش او را انفکاک نماید.
\par 49 یا عمویش یا پسر عمویش او را انفکاک نماید، یا یکی از خویشان او از خاندانش او راانفکاک نماید، یا خود او اگر برخوردار گردد، خویشتن را انفکاک نماید.
\par 50 و با آن کسی‌که او راخرید از سالی که خود را فروخت تا سال یوبیل حساب کند، و نقد فروش او برحسب شماره سالها باشد، موافق روزهای مزدور نزد او باشد.
\par 51 اگر سالهای بسیار باقی باشد، برحسب آنها نقدانفکاک خود را از نقد فروش خود، پس بدهد.
\par 52 و اگر تا سال یوبیل، سالهای کم باقی باشد باوی حساب بکند، و برحسب سالهایش نقدانفکاک خود را رد نماید.
\par 53 مثل مزدوری که سال به سال اجیر باشد نزد او بماند، و در نظر تو به سختی بر وی حکمرانی نکند.
\par 54 و اگر به اینهاانفکاک نشود پس در سال یوبیل رهاشود، هم خود او و پسرانش همراه وی.زیرا برای من بنی‌اسرائیل غلام‌اند، ایشان غلام من می‌باشند که ایشان را از زمین مصر بیرون آوردم. من یهوه خدای شما هستم.
\par 55 زیرا برای من بنی‌اسرائیل غلام‌اند، ایشان غلام من می‌باشند که ایشان را از زمین مصر بیرون آوردم. من یهوه خدای شما هستم.
 
\chapter{26}

\par 1 «برای خود بتها مسازید، و تمثال تراشیده و ستونی به جهت خود برپامنمایید، و سنگی مصور در زمین خود مگذاریدتا به آن سجده کنید، زیرا که من یهوه خدای شماهستم.
\par 2 سبت های مرا نگاه دارید، و مکان مقدس مرا احترام نمایید. من یهوه هستم.
\par 3 اگر درفرایض من سلوک نمایید و اوامر مرا نگاه داشته، آنها را بجا آورید، 
\par 4 آنگاه بارانهای شما را درموسم آنها خواهم داد، و زمین محصول خود راخواهد آورد، و درختان صحرا میوه خود راخواهد داد.
\par 5 و کوفتن خرمن شما تا چیدن انگورخواهد رسید، و چیدن انگور تا کاشتن تخم خواهد رسید، و نان خود را به سیری خورده، درزمین خود به امنیت سکونت خواهید کرد.
\par 6 و به زمین، سلامتی خواهم داد و خواهید خوابید وترساننده‌ای نخواهد بود، و حیوانات موذی را اززمین نابود خواهم ساخت، و شمشیر از زمین شماگذر نخواهد کرد.
\par 7 و دشمنان خود را تعاقب خواهید کرد، و ایشان پیش روی شما از شمشیرخواهند افتاد.
\par 8 و پنج نفر از شما صد را تعاقب خواهند کرد، و صد از شما ده هزار را خواهندراند، و دشمنان شما پیش روی شما از شمشیرخواهند افتاد.
\par 9 و بر شما التفات خواهم کرد، وشما را بارور گردانیده، شما را کثیر خواهم ساخت، و عهد خود را با شما استوار خواهم نمود.
\par 10 و غله کهنه پارینه را خواهید خورد، وکهنه را برای نو بیرون خواهید آورد.
\par 11 و مسکن خود را در میان شما برپا خواهم کرد و جانم شمارا مکروه نخواهد داشت.
\par 12 و در میان شماخواهم خرامید و خدای شما خواهم بود و شماقوم من خواهید بود.
\par 13 من یهوه خدای شماهستم که شما را از زمین مصر بیرون آوردم تاایشان را غلام نباشید، و بندهای یوغ شما راشکستم، و شما را راست روان ساختم.
\par 14 «و اگر مرا نشنوید و جمیع این اوامر را بجانیاورید،
\par 15 و اگر فرایض مرا رد نمایید و دل شمااحکام مرا مکروه دارد، تا تمامی اوامر مرا بجانیاورده، عهد مرا بشکنید،
\par 16 من این را به شماخواهم کرد که خوف و سل و تب را که چشمان رافنا سازد، و جان را تلف کند، بر شما مسلط خواهم ساخت، و تخم خود را بی‌فایده خواهید کاشت ودشمنان شما آن را خواهند خورد.
\par 17 و روی خود را به ضد شما خواهم داشت، و پیش روی دشمنان خود منهزم خواهید شد، و آنانی که ازشما نفرت دارند، بر شما حکمرانی خواهند کرد، و بدون تعاقب کننده‌ای فرار خواهید نمود.
\par 18 واگر با وجود این همه، مرا نشنوید، آنگاه شما را برای گناهان شما هفت مرتبه زیاده سیاست خواهم کرد.
\par 19 و فخر قوت شما را خواهم شکست، و آسمان شما را مثل آهن و زمین شما رامثل مس خواهم ساخت.
\par 20 و قوت شما دربطالت صرف خواهد شد، زیرا زمین شما حاصل خود را نخواهد داد، و درختان زمین میوه خود رانخواهد آورد.
\par 21 و اگر به خلاف من رفتار نموده، از شنیدن من ابا نمایید، آنگاه برحسب گناهانتان هفت چندان بلایای زیاده بر شما عارض گردانم.
\par 22 و وحوش صحرا را بر شما فرستم تا شما رابی اولاد سازند، و بهایم شما را هلاک کنند، و شمارا در شماره کم سازند، و شاهراههای شما ویران خواهد شد.
\par 23 و اگر با این همه از من متنبه نشده، به خلاف من رفتار کنید،
\par 24 آنگاه من نیز به خلاف شما رفتار خواهم کرد، و شما را برای گناهانتان هفت چندان سزا خواهم داد.
\par 25 و بر شماشمشیری خواهم آورد که انتقام عهد مرا بگیرد. وچون به شهرهای خود جمع شوید، وبا در میان شما خواهم فرستاد، و به‌دست دشمن تسلیم خواهید شد.
\par 26 و چون عصای نان شما را بشکنم، ده زن نان شما را در یک تنور خواهند پخت، و نان شما را به شما به وزن پس خواهند داد، و چون بخورید سیر نخواهید شد.
\par 27 و اگر با وجود این، مرا نشنوید و به خلاف من رفتار نمایید،
\par 28 آنگاه به غضب به خلاف شما رفتار خواهم کرد، و من نیز برای گناهانتان، شما را هفت چندان سیاست خواهم کرد.
\par 29 و گوشت پسران خود را خواهیدخورد، و گوشت دختران خود را خواهید خورد.
\par 30 و مکانهای بلند شما را خراب خواهم ساخت، و اصنام شما را قطع خواهم کرد، و لاشه های شمارا بر لاشه های بتهای شما خواهم افکند، و جان من شما را مکروه خواهد داشت.
\par 31 و شهرهای شما را خراب خواهم ساخت، و مکانهای مقدس شما را ویران خواهم کرد، و بوی عطرهای خوشبوی شما را نخواهم بویید.
\par 32 و من زمین راویران خواهم ساخت، به حدی که دشمنان شماکه در آن ساکن باشند، متحیر خواهند شد.
\par 33 وشما را در میان امتها پراکنده خواهم ساخت، وشمشیر را در عقب شما خواهم کشید، و زمین شما ویران و شهرهای شما خراب خواهد شد.
\par 34 آنگاه زمین در تمامی روزهای ویرانی‌اش، حینی که شما در زمین دشمنان خود باشید، ازسبت های خود تمتع خواهد برد. پس زمین آرامی خواهد یافت و از سبت های خود تمتع خواهدبرد.
\par 35 تمامی روزهای ویرانی‌اش آرامی خواهدیافت، یعنی آن آرامی که در سبت های شماحینی که در آن ساکن می‌بودید، نیافته بود.
\par 36 «و اما در دلهای بقیه شما در زمین دشمنان شما ضعف خواهم فرستاد، و آواز برگ رانده شده، ایشان را خواهد گریزانید، و بدون تعاقب کننده‌ای مثل کسی‌که از شمشیر فرار کند، خواهند گریخت و خواهند افتاد.
\par 37 و به روی یکدیگر مثل از دم شمشیر خواهند ریخت، باآنکه کسی تعاقب نکند، و شما را یارای مقاومت با دشمنان خود نخواهد بود.
\par 38 و در میان امتهاهلاک خواهید شد و زمین دشمنان شما، شما راخواهد خورد.
\par 39 و بقیه شما در زمین دشمنان خود در گناهان خود فانی خواهند شد، و درگناهان پدران خود نیز فانی خواهند شد.
\par 40 پس به گناهان خود و به گناهان پدران خود در خیانتی که به من ورزیده، و سلوکی که به خلاف من نموده‌اند، اعتراف خواهند کرد.
\par 41 از این سبب من نیز به خلاف ایشان رفتار نمودم، و ایشان را به زمین دشمنان ایشان آوردم. پس اگر دل نامختون ایشان متواضع شود و سزای گناهان خود رابپذیرند،
\par 42 آنگاه عهد خود را با یعقوب بیادخواهم آورد، و عهد خود را با اسحاق نیز و عهدخود را با ابراهیم نیز بیاد خواهم آورد، و آن زمین را بیاد خواهم آورد.
\par 43 و زمین از ایشان ترک خواهد شد و چون از ایشان ویران باشد ازسبت های خود تمتع خواهد برد، و ایشان سزای گناه خود را خواهند پذیرفت، به‌سبب اینکه احکام مرا رد کردند، و دل ایشان فرایض مرامکروه داشت.
\par 44 و با وجود این همه نیز چون درزمین دشمنان خود باشند، من ایشان را رد نخواهم کرد، و ایشان را مکروه نخواهم داشت تا ایشان راهلاک کنم، و عهد خود را با ایشان بشکنم، زیرا که من یهوه خدای ایشان هستم.
\par 45 بلکه برای ایشان عهد اجداد ایشان را بیاد خواهم آورد که ایشان رادر نظر امتها از زمین مصر بیرون آوردم، تا خدای ایشان باشم. من یهوه هستم.»این است فرایض و احکام و شرایعی که خداوند در میان خود وبنی‌اسرائیل در کوه سینا به‌دست موسی قرار داد.
\par 46 این است فرایض و احکام و شرایعی که خداوند در میان خود وبنی‌اسرائیل در کوه سینا به‌دست موسی قرار داد.
 
\chapter{27}

\par 1 و خداوند موسی را خطاب کرده، گفت:
\par 2 «بنی‌اسرائیل را خطاب کرده، به ایشان بگو: چون کسی نذر مخصوصی نماید، نفوس برحسب برآورد تو، از آن خداوند باشند.
\par 3 و اگر برآورد تو بجهت ذکور، از بیست ساله تاشصت ساله باشد، برآورد تو پنجاه مثقال نقره برحسب مثقال قدس خواهد بود.
\par 4 و اگر اناث باشد برآورد تو سی مثقال خواهد بود.
\par 5 و اگر ازپنج ساله تا بیست ساله باشد، برآورد تو بجهت ذکور، بیست مثقال و بجهت اناث ده مثقال خواهدبود.
\par 6 و اگر از یک ماهه تا پنج ساله باشد، برآوردتو بجهت ذکور پنج مثقال نقره، و بجهت اناث، برآورد تو سه مثقال نقره خواهد بود.
\par 7 و اگر ازشصت ساله و بالاتر باشد، اگر ذکور باشد، آنگاه برآورد تو پانزده مثقال، و برای اناث ده مثقال خواهد بود.
\par 8 و اگر از برآورد تو فقیرتر باشد، پس او را به حضور کاهن حاضر کنند، و کاهن برایش برآورد کند و کاهن به مقدار قوه آن که نذر کرده، برای وی برآورد نماید.
\par 9 و اگر بهیمه‌ای باشد ازآنهایی که برای خداوند قربانی می‌گذرانند، هرآنچه را که کسی از آنها به خداوند بدهد، مقدس خواهد بود.
\par 10 آن را مبادله ننماید و خوب را به بد یا بد را به خوب عوض نکند. و اگر بهیمه‌ای رابه بهیمه‌ای مبادله کند، هم آن و آنچه به عوض آن داده شود، هر دو مقدس خواهد بود.
\par 11 و اگر هرقسم بهیمه نجس باشد که از آن قربانی برای خداوند نمی گذرانند، آن بهیمه را پیش کاهن حاضر کند.
\par 12 و کاهن آن را چه خوب و چه بد، قیمت کند و برحسب برآورد تو‌ای کاهن، چنین باشد.
\par 13 و اگر آن را فدیه دهد، پنج یک بر برآوردتو زیاده دهد.
\par 14 و اگر کسی خانه خود را وقف نماید تا برای خداوند مقدس شود، کاهن آن راچه خوب و چه بد برآورد کند، و بطوری که کاهن آن را برآورد کرده باشد، همچنان بماند.
\par 15 و اگروقف کننده بخواهد خانه خود را فدیه دهد، پس پنج یک بر نقد برآورد تو زیاده کند و از آن اوخواهد بود.
\par 16 و اگر کسی قطعه‌ای از زمین ملک خود را برای خداوند وقف نماید، آنگاه برآوردتو موافق زراعت آن باشد، زراعت یک حومر جو به پنجاه مثقال نقره باشد.
\par 17 و اگر زمین خود را ازسال یوبیل وقف نماید، موافق برآورد تو برقرارباشد.
\par 18 و اگر زمین خود را بعد از یوبیل وقف نماید، آنگاه کاهن نقد آن را موافق سالهایی که تاسال یوبیل باقی می‌باشد برای وی بشمارد، و ازبرآورد تو تخفیف شود.
\par 19 و اگر آنکه زمین راوقف کرد بخواهد آن را فدیه دهد، پس پنج یک از نقد برآورد تو را بر آن بیفزاید و برای وی برقرارشود.
\par 20 و اگر نخواهد زمین را فدیه دهد، یا اگرزمین را به دیگری فروخته باشد، بعد از آن فدیه داده نخواهد شد.
\par 21 و آن زمین چون در یوبیل رها شود مثل زمین وقف برای خداوند، مقدس خواهد بود؛ ملکیت آن برای کاهن است.
\par 22 و اگرزمینی را که خریده باشد که از زمین ملک او نبود، برای خداوند وقف نماید،
\par 23 آنگاه کاهن مبلغ برآورد تو را تا سال یوبیل برای وی بشمارد، و درآن روز برآورد تو را مثل وقف خداوند به وی بدهد.
\par 24 و آن زمین در سال یوبیل به کسی‌که از اوخریده شده بود خواهد برگشت، یعنی به کسی‌که آن زمین ملک موروثی وی بود.
\par 25 و هر برآورد توموافق مثقال قدس باشد که بیست جیره یک مثقال است.
\par 26 «لیکن نخست زاده‌ای از بهایم که برای خداوند نخست زاده شده باشد، هیچکس آن راوقف ننماید، خواه گاو خواه گوسفند، از آن خداوند است.
\par 27 و اگر از بهایم نجس باشد، آنگاه آن را برحسب برآورد تو فدیه دهد، و پنج یک برآن بیفزاید، و اگر فدیه داده نشود پس موافق برآورد تو فروخته شود.
\par 28 اما هر چیزی که کسی برای خداوند وقف نماید، از کل مایملک خود، چه از انسان چه از بهایم چه از زمین ملک خود، نه فروخته شود و نه فدیه داده شود، زیرا هر‌چه وقف باشد برای خداوند قدس اقداس است.
\par 29 هر وقفی که از انسان وقف شده باشد، فدیه داده نشود. البته کشته شود.
\par 30 و تمامی ده‌یک زمین چه از تخم زمین چه از میوه درخت از آن خداوند است، و برای خداوند مقدس می‌باشد.
\par 31 و اگر کسی از ده‌یک خود چیزی فدیه دهدپنج یک آن را بر آن بیفزاید.
\par 32 و تمامی ده‌یک گاو و گوسفند یعنی هر‌چه زیر عصا بگذرد، دهم آن برای خداوند مقدس خواهد بود.در خوبی و بدی آن تفحص ننماید و آن را مبادله نکند، واگر آن را مبادله کند هم آن و هم بدل آن مقدس خواهد بود و فدیه داده نشود.»
\par 33 در خوبی و بدی آن تفحص ننماید و آن را مبادله نکند، واگر آن را مبادله کند هم آن و هم بدل آن مقدس خواهد بود و فدیه داده نشود.»


\end{document}