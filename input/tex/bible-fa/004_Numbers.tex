\begin{document}

\title{اعداد}

 
\chapter{1}

\par 1 و در روز اول ماه دوم از سال دوم از بیرون آمدن ایشان از زمین مصر، خداوند دربیابان سینا در خیمه اجتماع موسی را خطاب کرده، گفت:
\par 2 «حساب تمامی جماعت بنی‌اسرائیل را برحسب قبایل و خاندان آبای ایشان، به شماره اسم های همه ذکوران موافق سرهای ایشان بگیرید.
\par 3 از بیست ساله و زیاده، هر‌که از اسرائیل به جنگ بیرون می‌رود، تو وهارون ایشان را برحسب افواج ایشان بشمارید.
\par 4 و همراه شما یک نفر از هر سبط باشد که هر یک رئیس خاندان آبایش باشد.
\par 5 و اسم های کسانی که با شما باید بایستند، این است: از روبین، الیصوربن شدیئور.
\par 6 و از شمعون، شلومیئیل بن صوریشدای.
\par 7 و از یهودا، نحشون بن عمیناداب.
\par 8 و از یساکار، نتنائیل بن صوغر.
\par 9 و از زبولون، الیاب بن حیلون.
\par 10 و از بنی یوسف: از افرایم، الیشمع بن عمیهود. و از منسی، جملیئیل بن فدهصور.
\par 11 از بنیامین، ابیدان بن جدعونی.
\par 12 و از دان، اخیعزر بن عمیشدای.
\par 13 و از اشیر، فجعیئیل بن عکران.
\par 14 و از جاد، الیاساف بن دعوئیل.
\par 15 و از نفتالی، اخیرع بن عینان.»
\par 16 اینانند دعوت‌شدگان جماعت و سروران اسباط آبای ایشان، و روسای هزاره های اسرائیل.
\par 17 و موسی و هارون این کسان را که به نام، معین شدند، گرفتند.
\par 18 و در روز اول ماه دوم، تمامی جماعت را جمع کرده، نسب نامه های ایشان را برحسب قبایل و خاندان آبای ایشان، به شماره اسم‌ها از بیست ساله و بالاتر موافق سرهای ایشان خواندند.
\par 19 چنانکه خداوندموسی را امر فرموده بود، ایشان را در بیابان سینابشمرد.
\par 20 و اما انساب بنی روبین نخست زاده اسرائیل، برحسب قبایل و خاندان آبای ایشان، موافق نامها و سرهای ایشان این بود: هر ذکور ازبیست ساله و بالاتر، جمیع کسانی که برای جنگ بیرون می‌رفتند.
\par 21 شمرده شدگان ایشان از سبطروبین، چهل و شش هزار و پانصد نفر بودند.
\par 22 و انساب بنی شمعون برحسب قبایل وخاندان آبای ایشان، کسانی که از ایشان شمرده شدند، موافق شماره اسم‌ها و سرهای ایشان این بود: هر ذکور از بیست ساله و بالاتر، هر‌که برای جنگ بیرون می‌رفت.
\par 23 شمرده شدگان ایشان ازسبط شمعون، پنجاه و نه هزار و سیصد نفر بودند.
\par 24 و انساب بنی جاد برحسب قبایل و خاندان آبای ایشان، موافق شماره اسم‌ها، از بیست ساله وبالاتر، هر‌که برای جنگ بیرون می‌رفت.
\par 25 شمرده شدگان ایشان از سبط جاد، چهل و پنج هزار و ششصد و پنجاه نفر بودند.
\par 26 و انساب بنی یهودا برحسب قبایل وخاندان آبای ایشان، موافق شماره اسم‌ها ازبیست ساله و بالاتر، هر‌که برای جنگ بیرون می‌رفت.
\par 27 شمرده شدگان ایشان از سبط یهودا، هفتاد و چهار هزار و شش صد نفر بودند.
\par 28 و انساب بنی یساکار برحسب قبایل وخاندان آبای ایشان، موافق شماره اسم‌ها ازبیست ساله و بالاتر، هر‌که برای جنگ بیرون می‌رفت.
\par 29 شمرده شدگان ایشان از سبط یساکار، پنجاه و چهار هزار و چهارصد نفر بودند.
\par 30 و انساب بنی زبولون برحسب قبایل وخاندان آبای ایشان، موافق شماره اسم‌ها ازبیست ساله و بالاتر، هر‌که برای جنگ بیرون می‌رفت.
\par 31 شمرده شدگان ایشان از سبط زبولون پنجاه و هفت هزار و چهارصد نفر بودند.
\par 32 و انساب بنی یوسف از بنی افرایم برحسب قبایل و خاندان آبای ایشان، موافق شماره اسم هااز بیست ساله و بالاتر، هر‌که برای جنگ بیرون می‌رفت.
\par 33 شمرده شدگان ایشان از سبط افرایم، چهل هزار و پانصد نفر بودند.
\par 34 و انساب بنی منسی برحسب قبایل وخاندان آبای ایشان، موافق شماره اسم‌ها، ازبیست ساله و بالاتر، هر‌که برای جنگ بیرون می‌رفت.
\par 35 شمرده شدگان ایشان از سبط منسی، سی و دو هزار و دویست نفر بودند.
\par 36 و انساب بنی بنیامین برحسب قبایل وخاندان آبای ایشان، موافق شماره اسم‌ها، ازبیست ساله بالاتر، هر‌که برای جنگ بیرون می‌رفت.
\par 37 شمرده شدگان ایشان از سبط بنیامین، سی و پنج هزار و چهارصد نفر بودند.
\par 38 و انساب بنی دان برحسب قبایل و خاندان آبای ایشان، موافق شماره اسم‌ها از بیست ساله وبالاتر، هر‌که برای جنگ می‌رفت.
\par 39 شمرده شدگان ایشان از سبط دان، شصت و دو هزار و هفتصد نفر بودند.
\par 40 و انساب بنی اشیر برحسب قبایل و خاندان آبای ایشان، موافق شماره اسم‌ها از بیست ساله وبالاتر، هر‌که برای جنگ بیرون می‌رفت.
\par 41 شمرده شدگان ایشان از سبط اشیر، چهل ویک هزار و پانصد نفر بودند.
\par 42 و انساب بنی نفتالی برحسب قبایل وخاندان آبای ایشان، موافق شماره اسم‌ها ازبیست ساله و بالاتر، هر‌که برای جنگ بیرون می‌رفت.
\par 43 شمرده شدگان ایشان از سبط نفتالی، پنجاه و سه هزار و پانصد نفر بودند.
\par 44 اینانند شمرده شدگانی که موسی و هارون با دوازده نفر از سروران اسرائیل، که یک نفر برای هر خاندان آبای ایشان بود، شمردند.
\par 45 و تمامی شمرده شدگان بنی‌اسرائیل برحسب خاندان آبای ایشان، از بیست ساله و بالاتر، هر کس از اسرائیل که برای جنگ بیرون می‌رفت.
\par 46 همه شمرده شدگان، ششصد و سه هزار و پانصد وپنجاه نفر بودند.
\par 47 اما لاویان برحسب سبط آبای ایشان در میان آنها شمرده نشدند.
\par 48 زیرا خداوند موسی را خطاب کرده، گفت:
\par 49 «اما سبط لاوی را مشمار و حساب ایشان را درمیان بنی‌اسرائیل مگیر.
\par 50 لیکن لاویان را برمسکن شهادت و تمامی اسبابش و بر هرچه علاقه به آن دارد بگمار، و ایشان مسکن و تمامی اسبابش را بردارند، و ایشان آن را خدمت نمایندو به اطراف مسکن خیمه زنند.
\par 51 و چون مسکن روانه شود لاویان آن را پایین بیاورند، و چون مسکن افراشته شود لاویان آن را برپا نمایند، وغریبی که نزدیک آن آید، کشته شود.
\par 52 و بنی‌اسرائیل هر کس در محله خود و هر کس نزدعلم خویش برحسب افواج خود، خیمه زنند.
\par 53 و لاویان به اطراف مسکن شهادت خیمه زنند، مبادا غضب بر جماعت بنی‌اسرائیل بشود، ولاویان شعائر مسکن شهادت را نگاه دارند.»پس بنی‌اسرائیل چنین کردند، و برحسب آنچه خداوند موسی را امر فرموده بود، به عمل آوردند.
\par 54 پس بنی‌اسرائیل چنین کردند، و برحسب آنچه خداوند موسی را امر فرموده بود، به عمل آوردند.
 
\chapter{2}

\par 1 و خداوند موسی و هارون را خطاب کرده، گفت:
\par 2 «هر کس از بنی‌اسرائیل نزد علم ونشان خاندان آبای خویش خیمه زند، در برابر واطراف خیمه اجتماع خیمه زنند.
\par 3 و به‌جانب مشرق به سوی طلوع آفتاب اهل علم محله یهودابرحسب افواج خود خیمه زنند، و رئیس بنی یهودا نحشون بن عمیناداب باشد.
\par 4 و فوج اوکه از ایشان شمرده شدند هفتاد و چهار هزار وششصد نفر بودند.
\par 5 و سبط یساکار در پهلوی اوخیمه زنند، و رئیس بنی یساکار نتنائیل بن صوغرباشد.
\par 6 و فوج او که از ایشان شمرده شدند پنجاه و چهار هزار و چهارصد نفر بودند.
\par 7 و سبطزبولون و رئیس بنی زبولون الیاب بن حیلون باشد.
\par 8 و فوج او که از ایشان شمرده شدند، پنجاه وهفت هزار و چهارصد نفر بودند.
\par 9 جمیع شمرده شدگان محله یهودا برحسب افواج ایشان صد و هشتاد و شش هزار و چهارصد نفر بودند. وایشان اول کوچ کنند.
\par 10 و بر جانب جنوب، علم محله روبین برحسب افواج ایشان باشد، و رئیس بنی روبین الیصور بن شدیئور باشد.
\par 11 و فوج او که از ایشان شمرده شدند چهل و شش هزار و پانصد نفر بودند.
\par 12 و در پهلوی او سبط شمعون خیمه زنندو رئیس بنی شمعون شلومیئیل بن صوریشدای باشد.
\par 13 و فوج او که از ایشان شمرده شدند، پنجاه و نه هزار و سیصد نفر بودند.
\par 14 و سبط جادو رئیس بنی جاد الیاساف بن رعوئیل باشد.
\par 15 وفوج او که از ایشان شمرده شدند، چهل و پنج هزار و ششصد و پنجاه نفر بودند.
\par 16 جمیع شمرده شدگان محله روبین برحسب افواج ایشان صد و پنجاه و یک هزار و چهارصد و پنجاه نفربودند و ایشان دوم کوچ کنند.
\par 17 و بعد از آن خیمه اجتماع با محله لاویان درمیان محله‌ها کوچ کند، چنانکه خیمه می‌زنند، همچنان هر کس در جای خود نزد علمهای خویش کوچ کنند.
\par 18 و به طرف مغرب، علم محله افرایم برحسب افواج ایشان و رئیس بنی افرایم، الیشمع بن عمیهود باشد.
\par 19 و فوج او که از ایشان شمرده شدند، چهل هزار و پانصد نفر بودند.
\par 20 و درپهلوی او سبط منسی، و رئیس بنی منسی جملیئیل بن فدهصور باشد.
\par 21 و فوج او که ازایشان شمرده شدند، سی و دو هزار و دویست نفربودند.
\par 22 و سبط بنیامین و رئیس بنی بنیامین، ابیدان بن جدعونی باشد.
\par 23 و فوج او که از ایشان شمرده شدند، سی و پنج هزار و چهارصد نفربودند.
\par 24 جمیع شمرده شدگان محله افرایم برحسب افواج ایشان، صد و هشت هزار و یکصدنفر بودند، و ایشان سوم کوچ کنند.
\par 25 و به طرف شمال، علم محله دان، برحسب افواج ایشان، و رئیس بنی دان اخیعزر بن عمیشدای باشد.
\par 26 و فوج او که از ایشان شمرده شدند، شصت و دو هزاروهفت صد نفر بودند.
\par 27 ودر پهلوی ایشان سبط اشیر خیمه زنند، و رئیس بنی اشیر فجعیئیل بن عکران باشد.
\par 28 و فوج او که از ایشان شمرده شدند، چهل و یک هزار و پانصدنفر بودند.
\par 29 و سبط نفتالی و رئیس بنی نفتالی اخیرع بن عینان باشد.
\par 30 و فوج او که از ایشان شمرده شدند، پنجاه و سه هزار و چهارصد نفربودند.
\par 31 جمیع شمرده شدگان محله دان، صد وپنجاه و هفت هزار و ششصد نفر بودند. ایشان نزدعلمهای خود در عقب کوچ کنند.
\par 32 اینانند شمرده شدگان بنی‌اسرائیل برحسب خاندان آبای ایشان، جمیع شمرده شدگان محله‌ها موافق افواج ایشان شش صد و سه هزار وپانصد و پنجاه نفر بودند.
\par 33 اما لاویان چنانکه خداوند به موسی‌امر فرموده بود، در میان بنی‌اسرائیل شمرده نشدند.و بنی‌اسرائیل موافق هرچه خداوند به موسی‌امر فرموده بود، عمل نمودند، به اینطورنزد علمهای خود خیمه می‌زدند و به اینطور هرکس برحسب قبایل خود با خاندان آبای خودکوچ می‌کردند.
\par 34 و بنی‌اسرائیل موافق هرچه خداوند به موسی‌امر فرموده بود، عمل نمودند، به اینطورنزد علمهای خود خیمه می‌زدند و به اینطور هرکس برحسب قبایل خود با خاندان آبای خودکوچ می‌کردند.
 
\chapter{3}

\par 1 این است انساب هارون و موسی در روزی که خداوند در کوه سینا با موسی متکلم شد.
\par 2 و نامهای پسران هارون این است: نخست زاده‌اش ناداب و ابیهو و العازار و ایتامار.
\par 3 این است نامهای پسران هارون کهنه که مسح شده بودند که ایشان را برای کهانت تخصیص نمود.
\par 4 اما ناداب و ابیهو در حضور خداوندمردند، هنگامی که ایشان در بیابان سینا آتش غریب به حضور خداوند گذرانیدند، و ایشان را پسری نبود و العازار و ایتامار به حضور پدر خودهارون، کهانت می‌نمودند.
\par 5 و خداوند موسی را خطاب کرده، گفت:
\par 6 «سبط لاوی را نزدیک آورده، ایشان را پیش هارون کاهن حاضر کن تا او را خدمت نمایند.
\par 7 وایشان شعائر او و شعائر تمامی جماعت را پیش خیمه اجتماع نگاه داشته، خدمت مسکن را بجاآورند.
\par 8 و جمیع اسباب خیمه اجتماع و شعائربنی‌اسرائیل را نگاه داشته، خدمت مسکن را بجاآورند.
\par 9 و لاویان را به هارون و پسرانش بده، زیراکه ایشان از جانب بنی‌اسرائیل بالکل به وی داده شده‌اند.
\par 10 و هارون و پسرانش را تعیین نما تاکهانت خود را بجا بیاورند، و غریبی که نزدیک آید، کشته شود.»
\par 11 و خداوند موسی را خطاب کرده، گفت:
\par 12 «که اینک من لاویان را از میان بنی‌اسرائیل، به عوض هر نخست زاده‌ای از بنی‌اسرائیل که رحم را بگشاید گرفته‌ام، پس لاویان از آن من می‌باشند.
\par 13 زیرا جمیع نخست زادگان از آن منند، و درروزی که همه نخست زادگان زمین مصر را کشتم، جمیع نخست زادگان اسرائیل را خواه از انسان وخواه از بهایم برای خود تقدیس نمودم، پس ازآن من می‌باشند. من یهوه هستم.»
\par 14 و خداوند موسی را در بیابان سینا خطاب کرده، گفت:
\par 15 «بنی لاوی را برحسب خاندان آباو قبایل ایشان بشمار، هر ذکور ایشان را از یک ماهه و زیاده بشمار.» 
\par 16 پس موسی برحسب قول خداوند چنانکه مامور شد، ایشان را شمرد.
\par 17 وپسران لاوی موافق نامهای ایشان اینانند: جرشون و قهات و مراری.
\par 18 و نامهای بنی جرشون برحسب قبایل ایشان این است: لبنی و شمعی.
\par 19 و پسران قهات برحسب قبایل ایشان: عمرام ویصهار و حبرون و عزیئیل.
\par 20 و پسران مراری برحسب قبایل ایشان: محلی و موشی بودند. اینانند قبایل لاویان برحسب خاندان آبای ایشان.
\par 21 و از جرشون، قبیله لبنی و قبیله شمعی. اینانند قبایل جرشونیان.
\par 22 و شمرده شدگان ایشان به شماره همه ذکوران از یک ماهه و بالاتر، شمرده شدگان ایشان هفت هزار و پانصد نفربودند.
\par 23 و قبایل جرشونیان در عقب مسکن، به طرف مغرب خیمه زنند.
\par 24 و سرور خاندان آبای جرشونیان، الیاساف بن لایل باشد.
\par 25 و ودیعت بنی جرشون در خیمه اجتماع، مسکن و خیمه وپوشش آن و پرده دروازه خیمه اجتماع باشد.
\par 26 و تجیرهای صحن و پرده دروازه صحن که پیش روی مسکن و به اطراف مذبح است وطنابهایش با هر خدمت آنها.
\par 27 و از قهات، قبیله عمرامیان و قبیله یصهاریان و قبیله حبرونیان و قبیله عزیئیلیان، اینانند قبایل قهاتیان.
\par 28 به شماره همه ذکوران ازیک ماهه و بالاتر، هشت هزار و شش صد نفربودند که ودیعت قدس را نگاه می‌داشتند.
\par 29 وقبایل بنی قهات به طرف جنوب مسکن، خیمه بزنند.
\par 30 و سرور خاندان آبای قبایل قهاتیان، الیاصافان بن عزیئیل باشد.
\par 31 و ودیعت ایشان تابوت و میز و شمعدان و مذبح‌ها و اسباب قدس که با آنها خدمت می‌کنند، و حجاب و هر خدمت آن باشد.
\par 32 و سرور سروران لاویان، العازار بن هارون کاهن باشد، و نظارت نگهبانان خدمت قدس، او را خواهد بود.
\par 33 و از مراری، قبیله محلیان و قبیله موشیان؛ اینانند قبایل مراری.
\par 34 و شمرده شدگان ایشان وشماره همه ذکوران از یک ماهه و بالاتر، شش هزار و دویست نفر بودند.
\par 35 و سرور خاندان آبای قبایل مراری، صوریئیل بن ابیحایل باشد وایشان به طرف شمالی مسکن، خیمه بزنند.
\par 36 وودیعت معین بنی مراری، تختهای مسکن وپشت بندهایش و ستونهایش و پایه هایش وتمامی اسبابش با تمامی خدمتش باشد.
\par 37 وستونهای اطراف صحن و پایه های آنها و میخها وطنابهای آنها.
\par 38 و پیش مسکن به طرف مشرق و پیش روی خیمه اجتماع به طرف طلوع شمس، موسی وهارون و پسرانش خیمه بزنند و نگاهبانی قدس راو نگاهبانی بنی‌اسرائیل را بدارند. و هر غریبی که نزدیک آید، کشته شود.
\par 39 و جمع شمرده شدگان لاویان که موسی و هارون ایشان را برحسب قبایل ایشان و فرمان خداوند شمردند، همه ذکوران ازیک ماهه و بالاتر، بیست و دو هزار نفر بودند.
\par 40 و خداوند به موسی گفت: «جمیع نخست زادگان نرینه بنی‌اسرائیل را از یک ماهه وبالاتر بشمار، و حساب نامهای ایشان را بگیر.
\par 41 ولاویان را به عوض همه نخست زادگان بنی‌اسرائیل برای من که یهوه هستم بگیر، و بهایم لاویان را به عوض همه نخست زادگان بهایم بنی‌اسرائیل.»
\par 42 پس موسی چنانکه خداوند او را امرفرموده بود، همه نخست زادگان بنی‌اسرائیل راشمرد.
\par 43 و جمیع نخست زادگان نرینه، برحسب شماره اسم های شمرده شدگان ایشان از یک ماهه و بالاتر، بیست و دو هزار و دویست و هفتاد و سه نفر بودند.
\par 44 و خداوند موسی را خطاب کرده، گفت:
\par 45 «لاویان را به عوض جمیع نخست زادگان بنی‌اسرائیل، و بهایم لاویان را به عوض بهایم ایشان بگیر، و لاویان از آن من خواهند بود. من یهوه هستم.
\par 46 و اما درباره فدیه دویست و هفتادو سه نفر از نخست زادگان بنی‌اسرائیل که برلاویان زیاده‌اند،
\par 47 پنج مثقال برای هر سری بگیر، آن را موافق مثقال قدس که بیست جیره یک مثقال باشد، بگیر.
\par 48 و نقد فدیه آنانی که از ایشان زیاده‌اند به هارون و پسرانش بده.»
\par 49 پس موسی نقد فدیه را از آنانی که زیاده بودند، بر کسانی که لاویان فدیه آنها شده بودند، گرفت.
\par 50 و از نخست زادگان بنی‌اسرائیل نقد راکه هزار و سیصد و شصت و پنج مثقال موافق مثقال قدس باشد، گرفت.و موسی نقد فدیه رابرحسب قول خداوند چنانکه خداوند موسی راامر فرموده بود، به هارون و پسرانش داد.
\par 51 و موسی نقد فدیه رابرحسب قول خداوند چنانکه خداوند موسی راامر فرموده بود، به هارون و پسرانش داد.
 
\chapter{4}

\par 1 و خداوند موسی و هارون را خطاب کرده، گفت:
\par 2 «حساب بنی قهات را از میان بنی لاوی برحسب قبایل و خاندان آبای ایشان بگیر.
\par 3 از سی ساله و بالاتر تا پنجاه ساله، هر‌که داخل خدمت شود تا در خیمه اجتماع کار کند.
\par 4 «و خدمت بنی قهات در خیمه اجتماع، کارقدس‌الاقداس باشد.
\par 5 و هنگامی که اردو کوچ می‌کند هارون وپسرانش داخل شده، پوشش حجاب را پایین بیاورند، و تابوت شهادت را به آن بپوشانند.
\par 6 و برآن پوشش پوست خز آبی بگذارند و جامه‌ای که تمام آن لاجوردی باشد بالای آن پهن نموده، چوب دستهایش را بگذرانند.
\par 7 «و بر میز نان تقدمه، جامه لاجوردی بگسترانند و بر آن، بشقابها و قاشقها و کاسه‌ها وپیاله های ریختنی را بگذرانند و نان دائمی بر آن باشد.
\par 8 و جامه قرمز بر آنها گسترانیده، آن را به پوشش پوست خز بپوشانند و چوبدستهایش رابگذرانند.
\par 9 «و جامه لاجوردی گرفته، شمعدان روشنایی و چراغهایش و گلگیرهایش وسینی هایش و تمامی ظروف روغنش را که به آنهاخدمتش می‌کنند بپوشانند،
\par 10 و آن را و همه اسبابش را در پوشش پوست خز گذارده، برچوب دستی بگذارند.
\par 11 «و بر مذبح زرین، جامه لاجوردی گسترانیده، آن را به پوشش پوست خز بپوشانند، و چوب دستهایش را بگذرانند.
\par 12 «و تمامی اسباب خدمت را که به آنها درقدس خدمت می‌کنند گرفته، آنها را در جامه لاجوردی بگذرانند، و آنها را به پوشش پوست خز پوشانیده، بر چوب دست بنهند.
\par 13 «و مذبح را از خاکستر خالی کرده، جامه ارغوانی بر آن بگسترانند.
\par 14 و جمیع اسبابش راکه به آنها خدمت آن را می‌کنند یعنی مجمرها وچنگالها و خاک اندازها و کاسه‌ها، همه اسباب مذبح را بر روی آن بنهند، و بر آن پوشش، پوست خز گسترانیده، چوب دستهایش را بگذرانند.
\par 15 «و چون هارون و پسرانش در هنگام کوچ کردن اردو، از پوشانیدن قدس و تمامی اسباب قدس فارغ شوند، بعد از آن پسران قهات برای برداشتن آن بیایند، اما قدس را لمس ننمایند مبادابمیرند، این چیزها از خیمه اجتماع حمل بنی قهات می‌باشد.
\par 16 «و ودیعت العازار بن هارون کاهن، روغن بجهت روشنایی و بخور خوشبو و هدیه آردی دائمی و روغن مسح و نظارت تمامی مسکن می‌باشد، با هرآنچه در آن است، خواه از قدس وخواه از اسبابش.»
\par 17 و خداوند موسی و هارون را خطاب کرده، گفت:
\par 18 «سبط قبایل قهاتیان را از میان لاویان منقطع مسازید.
\par 19 بلکه با ایشان چنین رفتارنمایید تا چون به قدس‌الاقداس نزدیک آیند، زنده بمانند و نمیرند. هارون و پسرانش داخل آن بشوند، و هریک از ایشان را به خدمت و حمل خود بگمارند.
\par 20 و اما ایشان بجهت دیدن قدس لحظه‌ای هم داخل نشوند، مبادا بمیرند.»
\par 21 و خداوند موسی را خطاب کرده، گفت:
\par 22 «حساب بنی جرشون را نیز برحسب خاندان آباو قبایل ایشان بگیر.
\par 23 از سی ساله و بالاتر تاپنجاه ساله ایشان را بشمار، هر‌که داخل شود تادر خیمه اجتماع به شغل بپردازد و خدمت بنماید.
\par 24 «این است خدمت قبایل بنی جرشون درخدمت گذاری و حمل،
\par 25 که تجیرهای مسکن وخیمه اجتماع را با پوشش آن و پوشش پوست خز که بر بالای آن است، و پرده دروازه خیمه اجتماع را بردارند.
\par 26 و تجیرهای صحن و پرده مدخل دروازه صحن، که پیش مسکن و به اطراف مذبح است، و طنابهای آنها و همه اسباب خدمت آنها و هرچه به آنها باید کرده شود، ایشان بکنند.
\par 27 و تمامی خدمت بنی جرشون در هر حمل وخدمت ایشان، به فرمان هارون و پسران او بشود، و جمیع حملهای ایشان را بر ایشان ودیعت گذارید.
\par 28 این است خدمت قبایل بنی جرشون در خیمه اجتماع. و نظارت ایشان به‌دست ایتامار بن هارون کاهن باشد.
\par 29 «و بنی مراری را برحسب قبایل و خاندان آبای ایشان بشمار.
\par 30 از سی ساله و بالاتر تاپنجاه ساله هرکه به خدمت داخل شود، تا کارخیمه اجتماع را بنماید. ایشان را بشمار.
\par 31 این است ودیعت حمل ایشان، در تمامی خدمت ایشان در خیمه اجتماع، تختهای مسکن وپشت بندهایش و ستونهایش و پایه هایش
\par 32 وستونهای اطراف صحن و پایه های آنها و میخهای آنها و طنابهای آنها با همه اسباب آنها، و تمامی خدمت آنها، پس اسباب ودیعت حمل ایشان رابه نامها حساب کنید.
\par 33 این است خدمت قبایل بنی مراری در تمامی خدمت ایشان در خیمه اجتماع، زیردست ایتامار بن هارون کاهن.»
\par 34 و موسی و هارون و سروران جماعت، بنی قهات را برحسب قبایل و خاندان آبای ایشان شمردند.
\par 35 از سی ساله و بالاتر تا پنجاه ساله هرکه به خدمت داخل می‌شد تا در خیمه اجتماع مشغول شود.
\par 36 و شمرده شدگان ایشان برحسب قبایل ایشان، دو هزار و هفتصد و پنجاه نفر بودند.
\par 37 اینانند شمرده شدگان قبایل قهاتیان، هرکه درخیمه اجتماع کار می‌کرد که موسی و هارون ایشان را برحسب آنچه خداوند به واسطه موسی فرموده بود، شمردند.
\par 38 و شمرده شدگان بنی جرشون برحسب قبایل و خاندان آبای ایشان،
\par 39 از سی ساله وبالاتر تا پنجاه ساله، هرکه به خدمت داخل می‌شدتا در خیمه اجتماع کار کند.
\par 40 و شمرده شدگان ایشان برحسب قبایل و خاندان آبای ایشان، دوهزار و ششصد و سی نفر بودند.
\par 41 اینانند شمرده شدگان قبایل بنی جرشون، هرکه در خیمه اجتماع کار می‌کرد که موسی و هارون ایشان رابرحسب فرمان خداوند شمردند.
\par 42 و شمرده شدگان قبایل بنی مراری برحسب قبایل و خاندان آبای ایشان،
\par 43 از سی ساله وبالاتر تا پنجاه ساله، هرکه به خدمت داخل می‌شدتا در خیمه اجتماع کار کند.
\par 44 و شمرده شدگان ایشان برحسب قبایل ایشان سه هزار و دویست نفر بودند.
\par 45 اینانند شمرده شدگان قبایل بنی مراری که موسی و هارون ایشان را برحسب آنچه خداوند به واسطه موسی فرموده بود، شمردند.
\par 46 جمیع شمرده شدگان لاویان که موسی وهارون و سروران اسرائیل ایشان را برحسب قبایل و خاندان آبای ایشان شمردند،
\par 47 از سی ساله و بالاتر تا پنجاه ساله هرکه داخل می‌شد تاکار خدمت و کار حملها را در خیمه اجتماع بکند.
\par 48 شمرده شدگان ایشان هشت هزار وپانصد و هشتاد نفر بودند،برحسب فرمان خداوند به توسط موسی، هرکس موافق خدمتش و حملش شمرده شد.و چنانکه خداوند موسی را امر فرموده بود، اوایشان را شمرد.
\par 49 برحسب فرمان خداوند به توسط موسی، هرکس موافق خدمتش و حملش شمرده شد.و چنانکه خداوند موسی را امر فرموده بود، اوایشان را شمرد.
 
\chapter{5}

\par 1 و خداوند موسی را خطاب کرده، گفت:
\par 2 «بنی‌اسرائیل را امر فرما که مبروص را وهرکه جریان دارد و هرکه از میته نجس شود، ازاردو اخراج کنند.
\par 3 خواه مرد و خواه زن، ایشان رااخراج نمایید؛ بیرون از اردو ایشان را اخراج نمایید، تا اردوی خود را جایی که من در میان ایشان ساکن هستم، نجس نسازند.»
\par 4 وبنی‌اسرائیل چنین کردند، و آن کسان را بیرون ازاردو اخراج کردند. چنانکه خداوند به موسی گفته بود، بنی‌اسرائیل به آن طور عمل نمودند.
\par 5 و خداوند موسی را خطاب کرده، گفت:
\par 6 «بنی‌اسرائیل را بگو: هرگاه مردی یا زنی به هرکدام از جمیع گناهان انسان مرتکب شده، به خداوند خیانت ورزد، و آن شخص مجرم شود،
\par 7 آنگاه گناهی را که کرده است اعتراف بنماید، واصل جرم خود را رد نماید، و خمس آن را برآن مزید کرده، به کسی‌که بر او جرم نموده است، بدهد.
\par 8 و اگر آن کس را ولی‌ای نباشد که دیه جرم به او داده شود، آنگاه دیه جرمی که برای خداوندداده می‌شود، از آن کاهن خواهد بود، علاوه برقوچ کفاره که به آن درباره وی کفاره می‌شود.
\par 9 وهر هدیه افراشتنی از همه موقوفات بنی‌اسرائیل که نزد کاهن می‌آورند، از آن او باشد. 
\par 10 وموقوفات هر کس از آن او خواهد بود، و هرچه که کسی به کاهن بدهد، از آن او باشد.»
\par 11 و خداوند موسی را خطاب کرده، گفت:
\par 12 «بنی‌اسرائیل را خطاب کرده، به ایشان بگو: هرگاه زن کسی از او برگشته، به وی خیانت ورزد،
\par 13 و مردی دیگر با او همبستر شود، و این ازچشمان شوهرش پوشیده و مستور باشد، آن زن نجس می‌باشد، و اگر بر او شاهدی نباشد و در عین فعل گرفتار نشود،
\par 14 و روح غیرت بر او بیایدو به زن خود غیور شود، و آن زن نجس شده باشد، یا روح غیرت بر او بیاید و به زن خود غیورشود، و آن زن نجس نشده باشد،
\par 15 پس آن مردزن خود را نزد کاهن بیاورد، و بجهت او برای هدیه، یک عشر ایفه آرد جوین بیاورد، و روغن برآن نریزد، و کندر بر آن ننهد، زیرا که هدیه غیرت است و هدیه یادگار، که گناه را بیاد می‌آورد.
\par 16 «و کاهن او را نزدیک آورده، به حضورخداوند برپا دارد.
\par 17 و کاهن آب مقدس در ظرف سفالین بگیرد، و کاهن قدری از غباری که بر زمین مسکن باشد گرفته، بر آب بپاشد.
\par 18 و کاهن زن رابه حضور خداوند برپا داشته، موی سر او را بازکند و هدیه یادگار را که هدیه غیرت باشد بردست آن زن بگذارد، و آب تلخ لعنت بر دست کاهن باشد.
\par 19 و کاهن به زن قسم داده، به وی بگوید: اگر کسی با تو همبستر نشده، و اگر بسوی نجاست به کسی غیر از شوهر خود برنگشته‌ای، پس از این آب تلخ لعنت مبرا شوی.
\par 20 و لیکن اگر به غیر از شوهر خود برگشته، نجس شده‌ای، وکسی غیر از شوهرت با تو همبستر شده است،
\par 21 آنگاه کاهن زن را قسم لعنت بدهد و کاهن به زن بگوید: خداوند تو را در میان قومت موردلعنت و قسم بسازد به اینکه خداوند ران تو راساقط و شکم تو را منتفخ گرداند.
\par 22 و این آب لعنت در احشای تو داخل شده، شکم تو را منتفخ و ران تو را ساقط بسازد. و آن زن بگوید: آمین آمین.
\par 23 «و کاهن این لعنتها را در طوماری بنویسد، و آنها را در آب تلخ محو کند.
\par 24 و آن آب لعنت تلخ را به زن بنوشاند، و آن آب لعنت در او داخل شده، تلخ خواهد شد.
\par 25 و کاهن هدیه غیرت رااز دست زن گرفته، آن هدیه را به حضور خداوندبجنباند، و آن را نزد مذبح بیاورد.
\par 26 و کاهن مشتی از هدیه برای یادگاری آن گرفته، آن را برمذبح بسوزاند و بعد از آن، آن آب را به زن بنوشاند.
\par 27 و چون آب را به او نوشانید، اگرنجس شده و به شوهر خود خیانت ورزیده باشد، آن آب لعنت داخل او شده، تلخ خواهد شد، وشکم او منتفخ و ران او ساقط خواهد گردید، و آن زن در میان قوم خود مورد لعنت خواهدبود.
\par 28 واگر آن زن نجس نشده، طاهر باشد، آنگاه مبراشده، اولاد خواهد زایید.
\par 29 «این است قانون غیرت، هنگامی که زن ازشوهر خود برگشته، نجس شده باشد.
\par 30 یاهنگامی که روح غیرت بر مرد بیاید، و بر زنش غیور شود، آنگاه زن را به حضور خداوند برپابدارد، و کاهن تمامی این قانون را درباره او اجرادارد.پس آن مرد از گناه مبرا شود، و زن گناه خود را متحمل خواهد بود.»
\par 31 پس آن مرد از گناه مبرا شود، و زن گناه خود را متحمل خواهد بود.»
 
\chapter{6}

\par 1 و خداوند موسی را خطاب کرده، گفت:
\par 2 «بنی‌اسرائیل را خطاب کرده، به ایشان بگو: چون مرد یا زن نذر خاص، یعنی نذر نذیره بکند، و خود را برای خداوند تخصیص نماید،
\par 3 آنگاه از شراب و مسکرات بپرهیزد و سرکه شراب و سرکه مسکرات را ننوشد، و هیچ عصیرانگور ننوشد، و انگور تازه یا خشک نخورد.
\par 4 وتمام ایام تخصیصش از هر چیزی که از تاک انگور ساخته شود، از هسته تا پوست نخورد.
\par 5 «و تمام ایام نذر تخصیص او، استره بر سر اونیاید، و تا انقضای روزهایی که خود را برای خداوند تخصیص نموده است، مقدس شده، گیسهای موی سر خود را بلند دارد.
\par 6 و تمام روزهایی که خود را برای خداوندتخصیص نموده است، نزدیک بدن میت نیاید.
\par 7 برای پدر و مادر و برادر و خواهر خود، هنگامی که بمیرند خویشتن را نجس نسازد، زیرا که تخصیص خدایش بر سر وی می‌باشد.
\par 8 تمامی روزهای تخصیصش برای خداوند مقدس خواهد بود.
\par 9 «و اگر کسی دفعت ناگهان نزد او بمیرد، پس سر خود را در روز طهارت خویش بتراشد، یعنی در روز هفتم آن را بتراشد.
\par 10 و در روز هشتم دوفاخته یا دو جوجه کبوتر نزد کاهن به در خیمه اجتماع بیاورد.
\par 11 و کاهن یکی را برای قربانی گناه و دیگری را برای قربانی سوختنی گذرانیده، برای وی کفاره نماید، از آنچه به‌سبب میت، گناه کرده است و سر او را در آن روز تقدیس نماید.
\par 12 و روزهای تخصیص خود را برای خداوند (ازنو) تخصیص نماید، و بره نرینه یک ساله برای قربانی جرم بیاورد، لیکن روزهای اول ساقطخواهد بود، چونکه تخصیصش نجس شده است.
\par 13 «این است قانون نذیره، چون روزهای تخصیص او تمام شود، آنگاه او را نزد دروازه خیمه اجتماع بیاورند.
\par 14 و قربانی خود را برای خداوند بگذراند، یعنی یک بره نرینه یک ساله بی‌عیب بجهت قربانی سوختنی، و یک بره ماده‌یک ساله بی‌عیب، بجهت قربانی گناه، و یک قوچ بی عیب بجهت ذبیحه سلامتی.
\par 15 و یک سبد نان فطیر یعنی گرده های آرد نرم سرشته شده باروغن، و قرصهای فطیر مسح شده با روغن، وهدیه آردی آنها و هدیه ریختنی آنها.
\par 16 «و کاهن آنها را به حضور خداوند نزدیک آورده، قربانی گناه و قربانی سوختنی او رابگذراند.
\par 17 و قوچ را با سبد نان فطیر بجهت ذبیحه سلامتی برای خداوند بگذراند، و کاهن هدیه آردی و هدیه ریختنی او را بگذراند.
\par 18 «و آن نذیره سر تخصیص خود را نزد درخیمه اجتماع بتراشد، و موی سر تخصیص خودرا گرفته، آن را بر آتشی که زیر ذبیحه سلامتی است بگذراند.
\par 19 «و کاهن سردست بریان شده قوچ را با یک گرده فطیر از سبد و یک قرص فطیر گرفته، آن رابر دست نذیره، بعد از تراشیدن سر تخصیصش بگذارد.
\par 20 و کاهن آنها را بجهت هدیه جنبانیدنی به حضور خداوند بجنباند، این با سینه جنبانیدنی و ران افراشتنی برای کاهن، مقدس است. و بعد ازآن نذیره شراب بنوشد.
\par 21 «این است قانون نذیره‌ای که نذر بکند وقانون قربانی که بجهت تخصیص خود برای خداوند باید بگذراند، علاوه بر آنچه دستش به آن می‌رسد موافق نذری که کرده باشد، همچنین برحسب قانون تخصیص خود، باید بکند.»
\par 22 و خداوند موسی را خطاب کرده، گفت:
\par 23 «هارون و پسرانش را خطاب کرده، بگو: به اینطور بنی‌اسرائیل را برکت دهید و به ایشان بگویید:
\par 24 «یهوه تو را برکت دهد و تو را محافظت نماید.
\par 25 یهوه روی خود را بر تو تابان سازد و برتو رحمت کند.
\par 26 یهوه روی خود را بر توبرافرازد و تو را سلامتی بخشد،و نام مرابنی‌اسرائیل بگذارند، و من ایشان را برکت خواهم داد.»
\par 27 و نام مرابنی‌اسرائیل بگذارند، و من ایشان را برکت خواهم داد.»
 
\chapter{7}

\par 1 و در روزی که موسی از برپا داشتن مسکن فارغ شده و آن را مسح نموده و تقدیس کرده و تمامی اسبابش را و مذبح را با تمامی اسبابش مسح کرده و تقدیس نموده بود،
\par 2 سروران اسرائیل و روسای خاندان آبای ایشان هدیه گذرانیدند. و اینها روسای اسباط بودند که برشمرده شدگان گماشته شدند.
\par 3 پس ایشان بجهت هدیه خود، به حضور خداوند شش ارابه‌سرپوشیده و دوازده گاو آوردند، یعنی یک ارابه برای دو سرور، و برای هر نفری یک گاو، و آنها راپیش روی مسکن آوردند.
\par 4 و خداوند موسی را خطاب کرده، گفت:
\par 5 «اینها را از ایشان بگیر تا برای بجا آوردن خدمت خیمه اجتماع به‌کار آید، و به لاویان به هرکس به اندازه خدمتش تسلیم نما.»
\par 6 پس موسی ارابه‌ها و گاوها را گرفته، آنها را به لاویان تسلیم نمود.
\par 7 دو ارابه و چهار گاو به بنی جرشون، به اندازه خدمت ایشان تسلیم نمود.
\par 8 و چهار ارابه و هشت گاو به بنی مراری، به اندازه خدمت ایشان، به‌دست ایتامار بن هارون کاهن تسلیم نمود.
\par 9 اما به بنی قهات هیچ نداد، زیراخدمت قدس متعلق به ایشان بود و آن را بر دوش خود برمی داشتند.
\par 10 و سروران بجهت تبرک مذبح، در روز مسح کردن آن، هدیه گذرانیدند. وسروران هدیه خود را پیش مذبح آوردند.
\par 11 وخداوند به موسی گفت که هر سرور در روز نوبه خود هدیه خویش را بجهت تبرک مذبح بگذراند.
\par 12 و در روز اول، نحشون بن عمیناداب ازسبط یهودا هدیه خود را گذرانید.
\par 13 و هدیه اویک طبق نقره بود که وزنش صد و سی مثقال بود، و یک لگن نقره، هفتاد مثقال به مثقال قدس که هردوی آنها پر از آرد نرم مخلوط شده با روغن بودبجهت هدیه آردی.
\par 14 و یک قاشق طلا ده مثقال پر از بخور.
\par 15 و یک گاو جوان و یک قوچ و یک بره نرینه یک ساله بجهت قربانی سوختنی.
\par 16 ویک بز نر بجهت قربانی گناه.
\par 17 و بجهت ذبیحه سلامتی، دو گاو و پنج قوچ و پنج بز نر و پنج بره نرینه یک ساله، این بود هدیه نحشون بن عمیناداب.
\par 18 و در روز دوم، نتنائیل بن صوغر، سروریساکار هدیه گذرانید.
\par 19 و هدیه‌ای که او گذرانیدیک طبق نقره بود که وزنش صد و سی مثقال بود، و یک لگن نقره هفتاد مثقال، موافق مثقال قدس، هر دوی آنها پر از آرد نرم مخلوط با روغن بجهت هدیه آردی.
\par 20 و یک قاشق طلا ده مثقال پر ازبخور.
\par 21 و یک گاو جوان و یک قوچ و یک بره نرینه یک ساله، بجهت قربانی سوختنی.
\par 22 و یک بز نر بجهت قربانی گناه.
\par 23 و بجهت ذبیحه سلامتی، دو گاو و پنج قوچ و پنج بز نر و پنج بره نرینه یک ساله، این بود هدیه نتنائیل بن صوغر.
\par 24 و در روز سوم، الیاب بن حیلون سروربنی زبولون،
\par 25 هدیه او یک طبق نقره که وزنش صد و سی مثقال بود، و یک لگن نقره هفتاد مثقال، موافق مثقال قدس، هر دوی آنها پر از آرد نرم مخلوط با روغن بجهت هدیه آردی.
\par 26 و یک قاشق طلا ده مثقال پر از بخور،
\par 27 و یک گاوجوان و یک قوچ بره نرینه یک ساله بجهت قربانی سوختنی.
\par 28 و یک بز نر بجهت قربانی گناه.
\par 29 وبجهت ذبیحه سلامتی، دو گاو و پنج قوچ و پنج بزنر و پنج بره نرینه یک ساله. این بود هدیه الیاب بن حیلون.
\par 30 و در روز چهارم، الیصور بن شدیئور سروربنی روبین.
\par 31 هدیه او یک طبق نقره که وزنش صدو سی مثقال بود، و یک لگن نقره هفتاد مثقال، موافق مثقال قدس، هر دوی آنها پر از آرد نرم مخلوط با روغن بجهت هدیه آردی.
\par 32 و یک قاشق طلا ده مثقال پر از بخور.
\par 33 و یک گاو جوان و یک قوچ و یک بره نرینه یک ساله بجهت قربانی سوختنی.
\par 34 و یک بز نر بجهت قربانی گناه.
\par 35 وبجهت ذبیحه سلامتی، دو گاو و پنج قوچ و پنج بزنر و پنج بره نرینه یک ساله. این بود هدیه الیصوربن شدیئور.
\par 36 و در روز پنجم، شلومیئیل بن صوریشدای سرور بنی شمعون.
\par 37 هدیه او یک طبق نقره که وزنش صد و سی مثقال بود، و یک لگن نقره هفتادمثقال، موافق مثقال قدس، هر دوی آنها پر از آردنرم مخلوط با روغن بجهت هدیه آردی.
\par 38 و یک قاشق طلا ده مثقال پر از بخور.
\par 39 و یک گاو جوان و یک قوچ و یک بره نرینه یک ساله بجهت قربانی سوختنی.
\par 40 و یک بز نر بجهت قربانی گناه.
\par 41 وبجهت ذبیحه سلامتی، دو گاو و پنج قوچ و پنج بزنر و پنج بره نرینه یک ساله. این بود هدیه شلومیئیل بن صوریشدای.
\par 42 و در روز ششم، الیاساف بن دعوئیل سروربنی جاد.
\par 43 هدیه او یک طبق نقره که وزنش صدو سی مثقال بود، و یک لگن نقره هفتاد مثقال، موافق مثقال قدس، هر دوی آنها پر از آرد نرم مخلوط با روغن بجهت هدیه آردی.
\par 44 و یک قاشق طلا ده مثقال پر از بخور.
\par 45 و یک گاو جوان و یک قوچ و یک بره نرینه یک ساله بجهت قربانی سوختنی.
\par 46 و یک بز نر بجهت قربانی گناه.
\par 47 وبجهت ذبیحه سلامتی، دو گاو و پنج قوچ و پنج بزنر و پنج بره نرینه یک ساله. این بود هدیه الیاساف بن دعوئیل.
\par 48 و در روز هفتم، الیشمع بن عمیهود سروربنی افرایم.
\par 49 هدیه او یک طبق نقره که وزنش صد و سی مثقال بود، و یک لگن نقره هفتاد مثقال، موافق مثقال قدس، هر دوی آنها پر از آرد نرم مخلوط با روغن بجهت هدیه آردی. 
\par 50 و یک قاشق طلا ده مثقال پر از بخور.
\par 51 و یک گاو جوان و یک قوچ و یک بره نرینه یک ساله بجهت قربانی سوختنی.
\par 52 و یک بز نر بجهت قربانی گناه.
\par 53 وبجهت ذبیحه سلامتی، دو گاو و پنج قوچ و پنج بزنر و پنج بره نرینه یک ساله. این بود هدیه الیشمع بن عمیهود.
\par 54 و در روز هشتم، جملیئیل بن فدهصورسرور بنی منسی.
\par 55 هدیه او یک طبق نقره که وزنش صد و سی مثقال بود، و یک لگن نقره هفتادمثقال، موافق مثقال قدس، هر دوی آنها پر از آردنرم مخلوط با روغن بجهت هدیه آردی.
\par 56 و یک قاشق طلا ده مثقال پر از بخور.
\par 57 و یک گاوجوان و یک قوچ و یک بره نرینه یک ساله بجهت قربانی سوختنی.
\par 58 و یک بز نر بجهت قربانی گناه.
\par 59 و بجهت ذبیحه سلامتی، دو گاو و پنج قوچ و پنج بز نر و پنج بره نرینه یک ساله. این بودهدیه جملیئیل بن فدهصور.
\par 60 و در روز نهم، ابیدان بن جدعونی سروربنی بنیامین.
\par 61 هدیه او یک طبق نقره که وزنش صد و سی مثقال بود و یک لگن نقره هفتاد مثقال موافق مثقال قدس، هر دوی آنها پر از آرد نرم مخلوط با روغن، بجهت هدیه آردی.
\par 62 و یک قاشق طلا ده مثقال پر از بخور.
\par 63 و یک گاو جوان و یک قوچ و یک بره نرینه یک ساله بجهت قربانی سوختنی.
\par 64 و یک بز نر به جهت قربانی گناه
\par 65 به جهت ذبیحه سلامتی دو گاو و پنج قوچ و پنج بره نرینه یک ساله. این بود هدیه ابیدان بن جدعونی.
\par 66 و در روز دهم، اخیعزربن عمیشدای سروربنی دان.
\par 67 هدیه او یک طبق نقره که وزنش صد وسی مثقال بود، و یک لگن نقره، هفتاد مثقال موافق مثقال قدس، هر دوی آنها پر از آرد نرم مخلوط باروغن بجهت هدیه آردی.
\par 68 و یک قاشق طلا، ده مثقال پر از بخور.
\par 69 و یک گاو جوان و یک قوچ ویک بره نرینه یک ساله بجهت قربانی سوختنی.
\par 70 و یک بز نر بجهت قربانی گناه.
\par 71 و بجهت ذبیحه سلامتی، دو گاو و پنج قوچ و پنج بز نر وپنج بره نرینه یک ساله. این بود هدیه اخیعزر بن عمیشدای.
\par 72 و در روز یازدهم، فجعیئیل بن عکران سرور بنی اشیر.
\par 73 هدیه او یک طبق نقره که وزنش صد و سی مثقال بود، و یک لگن نقره، هفتاد مثقال موافق مثقال قدس؛ هردوی آنها پر ازآرد نرم مخلوط با روغن بجهت هدیه آردی.
\par 74 ویک قاشق طلا ده مثقال پر از بخور.
\par 75 و یک گاوجوان و یک قوچ و بره نرینه یک ساله بجهت قربانی سوختنی.
\par 76 و یک بز نر بجهت قربانی گناه.
\par 77 و بجهت ذبیحه سلامتی، دو گاو و پنج قوچ و پنج بز نر و پنج بره نرینه یک ساله. این بودهدیه فجعیئیل بن عکران.
\par 78 و در روز دوازدهم، اخیرع بن عینان، سروربنی نفتالی.
\par 79 هدیه او یک طبق نقره که وزنش صد و سی مثقال بود، و یک لگن نقره هفتاد مثقال موافق مثقال قدس، هردوی آنها پر از آرد نرم مخلوط با روغن بجهت هدیه آردی.
\par 80 و یک قاشق طلا، ده مثقال پر از بخور.
\par 81 و یک گاوجوان و یک قوچ و یک بره نرینه یک ساله بجهت قربانی سوختنی.
\par 82 و یک بز نر بجهت قربانی گناه.
\par 83 و بجهت ذبیحه سلامتی، دو گاو و پنج قوچ و پنج بز نر و پنج بره نرینه یک ساله. این بودهدیه اخیرع بن عینان.
\par 84 این بود تبرک مذبح در روزی که مسح شده بود، از جانب سروران اسرائیل دوازده طبق نقره و دوازده لگن نقره و دوازده قاشق طلا.
\par 85 هرطبق نقره صد و سی مثقال و هر لگن هفتاد، که تمامی نقره ظروف، دوهزار و چهارصد مثقال موافق مثقال قدس بود.
\par 86 و دوازده قاشق طلا پراز بخور هر کدام ده مثقال موافق مثقال قدس، که تمامی طلای قاشقها صد و بیست مثقال بود.
\par 87 تمامی گاوان بجهت قربانی سوختنی، دوازده گاو و دوازده قوچ و دوازده بره نرینه یک ساله. باهدیه آردی آنها و دوازده بز نر بجهت قربانی گناه.
\par 88 و تمامی گاوان بجهت ذبیحه سلامتی، بیست وچهار گاو و شصت قوچ و شصت بز نر و شصت بره نرینه یک ساله. این بود تبرک مذبح بعد از آنکه مسح شده بود.و چون موسی به خیمه اجتماع داخل شدتا با وی سخن گوید، آنگاه قول را می‌شنید که ازبالای کرسی رحمت که بر تابوت شهادت بود، ازمیان دو کروبی به وی سخن می‌گفت، پس با اوتکلم می‌نمود.
\par 89 و چون موسی به خیمه اجتماع داخل شدتا با وی سخن گوید، آنگاه قول را می‌شنید که ازبالای کرسی رحمت که بر تابوت شهادت بود، ازمیان دو کروبی به وی سخن می‌گفت، پس با اوتکلم می‌نمود.
 
\chapter{8}

\par 1 و خداوند موسی را خطاب کرده، گفت:
\par 2 «هارون را خطاب کرده، به وی بگو: هنگامی که چراغها را برافرازی، هفت چراغ پیش شمعدان روشنایی بدهد.»
\par 3 پس هارون چنین کرد، و چراغها را برافراشت تا پیش شمعدان روشنایی بدهد، چنانکه خداوند موسی را امرفرموده بود.
\par 4 و صنعت شمعدان این بود: ازچرخکاری طلا از ساق تا گلهایش چرخکاری بود، موافق نمونه‌ای که خداوند به موسی نشان داده بود، به همین طور شمعدان را ساخت.
\par 5 و خداوند موسی را خطاب کرده، گفت:
\par 6 «لاویان را از میان بنی‌اسرائیل گرفته، ایشان راتطهیر نما.
\par 7 و بجهت تطهیر ایشان، به ایشان چنین عمل نما، آن کفاره گناه را بر ایشان بپاش و بر تمام بدن خود استره بگذرانند، و رخت خود را شسته، خود را تطهیر نمایند.
\par 8 و گاوی جوان و هدیه آردی آن، یعنی آرد نرم مخلوط با روغن بگیرند، وگاو جوان دیگر بجهت قربانی گناه بگیر.
\par 9 ولاویان را پیش خیمه اجتماع نزدیک بیاور، وتمامی جماعت بنی‌اسرائیل را جمع کن.
\par 10 ولاویان را به حضور خداوند نزدیک بیاور، وبنی‌اسرائیل دستهای خود را بر لاویان بگذارند.
\par 11 و هارون لاویان را از جانب بنی‌اسرائیل به حضور خداوند هدیه بگذراند، تا خدمت خداوندرا بجا بیاورند.
\par 12 و لاویان دستهای خود را بر سرگاوان بنهند، و تو یکی را بجهت قربانی گناه ودیگری را بجهت قربانی سوختنی برای خداوندبگذران، تا بجهت لاویان کفاره شود.
\par 13 و لاویان را پیش هارون و پسرانش برپا بدار، و ایشان رابرای خداوند هدیه بگذران.
\par 14 و لاویان را از میان بنی‌اسرائیل جدا نما و لاویان از آن من خواهندبود.
\par 15 «و بعد از آن لاویان داخل شوند تا خدمت خیمه اجتماع را بجا آورند، و تو ایشان را تطهیرکرده، ایشان را هدیه بگذران.
\par 16 زیراکه ایشان ازمیان بنی‌اسرائیل به من بالکل داده شده‌اند، و به عوض هر گشاینده رحم، یعنی به عوض همه نخست زادگان بنی‌اسرائیل، ایشان را برای خودگرفته‌ام.
\par 17 زیرا که جمیع نخست زادگان بنی‌اسرائیل خواه از انسان و خواه از بهایم، از آن من‌اند، در روزی که جمیع نخست زادگان را درزمین مصر زدم، ایشان را برای خود تقدیس نمودم.
\par 18 پس لاویان را به عوض همه نخست زادگان بنی‌اسرائیل گرفتم.
\par 19 و لاویان رااز میان بنی‌اسرائیل به هارون و پسرانش پیشکش دادم تا خدمت بنی‌اسرائیل را در خیمه اجتماع بجا آورند، و بجهت بنی‌اسرائیل کفاره نمایند، وچون بنی‌اسرائیل به قدس نزدیک آیند، وبا به بنی‌اسرائیل عارض نشود.»
\par 20 پس موسی و هارون و تمامی جماعت بنی‌اسرائیل به لاویان چنین کردند، برحسب هرآنچه خداوند موسی را درباره لاویان امرفرمود، همچنان بنی‌اسرائیل به ایشان عمل نمودند.
\par 21 و لاویان برای گناه خود کفاره کرده، رخت خود را شستند، و هارون ایشان را به حضور خداوند هدیه گذرانید، و هارون برای ایشان کفاره نموده، ایشان را تطهیر کرد.
\par 22 و بعداز آن لاویان داخل شدند تا در خیمه اجتماع به حضور هارون و پسرانش به خدمت خودبپردازند، و چنانکه خداوند موسی را درباره لاویان امر فرمود، همچنان به ایشان عمل نمودند.
\par 23 و خداوند موسی را خطاب کرده، گفت:
\par 24 «این است قانون لاویان که از بیست و پنج ساله و بالاتر داخل شوند تا در کار خیمه اجتماع مشغول خدمت بشوند.
\par 25 و از پنجاه ساله از کارخدمت بازایستند، و بعد از آن خدمت نکنند.لیکن با برادران خود در خیمه اجتماع به نگاهبانی نمودن مشغول شوند، و خدمتی دیگرنکنند. بدین طور با لاویان درباره ودیعت ایشان عمل نما.»
\par 26 لیکن با برادران خود در خیمه اجتماع به نگاهبانی نمودن مشغول شوند، و خدمتی دیگرنکنند. بدین طور با لاویان درباره ودیعت ایشان عمل نما.»
 
\chapter{9}

\par 1 و در ماه اول سال دوم بعد از بیرون آمدن ایشان از زمین مصر، خداوند موسی را درصحرای سینا خطاب کرده، گفت:
\par 2 «بنی‌اسرائیل عید فصح را در موسمش بجا آورند.
\par 3 در روزچهاردهم این ماه آن را در وقت عصر در موسمش بجا آورید، برحسب همه فرایضش و همه احکامش آن را معمول دارید.»
\par 4 پس موسی به بنی‌اسرائیل گفت که فصح رابجا آورند.
\par 5 و فصح را در روز چهاردهم ماه اول، در وقت عصر در صحرای سینا بجا آوردند، برحسب هرچه خداوند به موسی‌امر فرموده بودبنی‌اسرائیل چنان عمل نمودند.
\par 6 اما بعضی اشخاص بودند که از میت آدمی نجس شده، فصح را در آن روز نتوانستند بجا آورند، پس درآن روز نزد موسی و هارون آمدند.
\par 7 و آن اشخاص وی را گفتند که «ما از میت آدمی نجس هستیم؛ پس چرا از گذرانیدن قربانی خداوند در موسمش در میان بنی‌اسرائیل ممنوع شویم؟»
\par 8 موسی ایشان را گفت: «بایستید تا آنچه خداوند در حق شما امر فرماید، بشنوم.»
\par 9 و خداوند موسی را خطاب کرده، گفت:
\par 10 «بنی‌اسرائیل را خطاب کرده، بگو: اگر کسی ازشما یا از اعقاب شما از میت نجس شود، یا درسفر دور باشد، مع هذا فصح را برای خداوند بجاآورد.
\par 11 در روز چهاردهم ماه دوم، آن را دروقت عصر بجا آورند، و آن را با نان فطیر و سبزی تلخ بخورند.
\par 12 چیزی از آن تا صبح نگذارند و ازآن استخوانی نشکنند؛ برحسب جمیع فرایض فصح آن را معمول دارند.
\par 13 اما کسی‌که طاهرباشد و در سفر نباشد و از بجا آوردن فصح بازایستد، آن کس از قوم خود منقطع شود، چونکه قربانی خداوند را در موسمش نگذرانیده است، آن شخص گناه خود را متحمل خواهدشد.
\par 14 و اگر غریبی در میان شما ماوا گزیند وبخواهد که فصح را برای خداوند بجا آورد، برحسب فریضه و حکم فصح عمل نماید، برای شما یک فریضه می‌باشد خواه برای غریب وخواه برای متوطن.»
\par 15 و در روزی که مسکن برپا شد، ابر مسکن خیمه شهادت را پوشانید، و از شب تا صبح مثل منظر آتش بر مسکن می‌بود.
\par 16 همیشه چنین بودکه ابر آن را می‌پوشانید و منظر آتش در شب.
\par 17 وهرگاه ابر از خیمه برمی خاست بعد از آن بنی‌اسرائیل کوچ می‌کردند و در هر جایی که ابرساکن می‌شد آنجا بنی‌اسرائیل اردو می‌زدند.
\par 18 به فرمان خداوند بنی‌اسرائیل کوچ می‌کردند وبه فرمان خداوند اردو می‌زدند، همه روزهایی که ابر بر مسکن ساکن می‌بود، در اردو می‌ماندند.
\par 19 وچون ابر، روزهای بسیار برمسکن توقف می‌نمود، بنی‌اسرائیل ودیعت خداوند را نگاه می‌داشتند وکوچ نمی کردند.
\par 20 و بعضی اوقات ابر ایام قلیلی بر مسکن می‌ماند، آنگاه به فرمان خداوند در اردومی ماندند و به فرمان خداوند کوچ می‌کردند.
\par 21 وبعضی اوقات، ابر از شام تا صبح می‌ماند و دروقت صبح ابر برمی خاست، آنگاه کوچ می‌کردند، یا اگر روز و شب می‌ماند چون ابر برمی خاست، می‌کوچیدند.
\par 22 خواه دو روز و خواه یک ماه وخواه یک سال، هر قدر ابر بر مسکن توقف نموده، بر آن ساکن می‌بود، بنی‌اسرائیل در اردومی ماندند، و کوچ نمی کردند و چون برمی خاست، می‌کوچیدند.به فرمان خداونداردو می‌زدند، و به فرمان خداوند کوچ می‌کردند، و ودیعت خداوند را برحسب آنچه خداوند به واسطه موسی فرموده بود، نگاه می‌داشتند.
\par 23 به فرمان خداونداردو می‌زدند، و به فرمان خداوند کوچ می‌کردند، و ودیعت خداوند را برحسب آنچه خداوند به واسطه موسی فرموده بود، نگاه می‌داشتند.
 
\chapter{10}

\par 1 و خداوند موسی را خطاب کرده، گفت:
\par 2 «برای خود دو کرنای نقره بساز، آنهارا از چرخکاری درست کن، و آنها را بجهت خواندن جماعت و کوچیدن اردو بکار ببر.
\par 3 وچون آنها را بنوازند تمامی جماعت نزد تو به درخیمه اجتماع جمع شوند.
\par 4 و چون یکی رابنوازند، سروران و روسای هزاره های اسرائیل نزد تو جمع شوند. 
\par 5 و چون تیز آهنگ بنوازیدمحله هایی که به طرف مشرق جا دارند، کوچ بکنند.
\par 6 و چون مرتبه دوم تیز آهنگ بنوازید، محله هایی که به طرف جنوب جا دارند کوچ کنند؛ بجهت کوچ دادن ایشان تیز آهنگ بنوازند.
\par 7 و بجهت جمع کردن جماعت بنوازید، لیکن تیزآهنگ منوازید.
\par 8 «و بنی هارون کهنه، کرناها را بنوازند. این برای شما در نسلهای شما فریضه ابدی باشد.
\par 9 وچون در زمین خود برای مقاتله با دشمنی که برشما تعدی می‌نمایند می‌روید، کرناها را تیزآهنگ بنوازید، پس به حضور یهوه خدای خودبیاد آورده خواهید شد، و از دشمنان خود نجات خواهید یافت.
\par 10 و در روز شادی خود و درعیدها و در اول ماه های خود کرناها را برقربانی های سوختنی و ذبایح سلامتی خودبنوازید، تا برای شما به حضور خدای شمایادگاری باشد. من یهوه خدای شما هستم.»
\par 11 و واقع شد در روز بیستم ماه دوم سال دوم که ابر از بالای خیمه شهادت برداشته شد،
\par 12 وبنی‌اسرائیل به مراحل خود از صحرای سینا کوچ کردند، و ابر در صحرای فاران ساکن شد،
\par 13 وایشان اول به فرمان خداوند به واسطه موسی کوچ کردند.
\par 14 و علم محله بنی یهودا، اول با افواج ایشان روانه شد، و بر فوج او نحشون بن عمیناداب بود.
\par 15 و بر فوج سبط بنی یساکار، نتنائیل بن صوغر.
\par 16 و بر فوج سبط بنی زبولون، الیاب بن حیلون.
\par 17 پس مسکن را پایین آوردند و بنی جرشون وبنی مراری که حاملان مسکن بودند، کوچ کردند.
\par 18 و علم محله روبین با افواج ایشان روانه شد، و بر فوج او الیصور بن شدیئور بود.
\par 19 و بر فوج سبط بنی شمعون، شلومیئیل بن صوریشدای.
\par 20 و بر فوج سبط بنی جاد، الیاساف بن دعوئیل.
\par 21 پس قهاتیان که حاملان قدس بودند، کوچ کردند و پیش از رسیدن ایشان، آنها مسکن را برپاداشتند.
\par 22 پس علم محله بنی افرایم با افواج ایشان روانه شد، و بر فوج او الیشمع بن عمیهود بود.
\par 23 و بر فوج سبط بنی منسی، جملیئیل بن فدهصور.
\par 24 و بر فوج سبط بنی بنیامین، ابیدان بن جدعونی.
\par 25 پس علم محله بنی دان که موخر همه محله‌ها بود با افواج ایشان روانه شد، و بر فوج اواخیعزر بن عمیشدای بود.
\par 26 و بر فوج سبطبنی اشیر، فجعیئیل بن عکران.
\par 27 و بر فوج سبطبنی نفتالی، اخیرع بن عینان.
\par 28 این بود مراحل بنی‌اسرائیل با افواج ایشان. پس کوچ کردند.
\par 29 و موسی به حوباب بن رعوئیل مدیانی که برادرزن موسی بود، گفت: «ما به مکانی که خداوند درباره آن گفته است که آن را به شماخواهم بخشید کوچ می‌کنیم، همراه ما بیا و بتواحسان خواهیم نمود، چونکه خداوند درباره اسرائیل نیکو گفته است.»
\par 30 او وی را گفت: «نمی آیم، بلکه به زمین و به خاندان خود خواهم رفت.»
\par 31 گفت: «ما را ترک مکن زیرا چونکه تو منازل ما را در صحرا می‌دانی، بجهت ما مثل چشم خواهی بود.
\par 32 و اگر همراه ما بیایی، هر احسانی که خداوند بر ما بنماید، همان را بر تو خواهیم نمود.»
\par 33 و از کوه خداوند سفر سه روزه کوچ کردند، و تابوت عهد خداوند سفر سه روزه پیش روی ایشان رفت تا آرامگاهی برای ایشان بطلبد.
\par 34 و ابر خداوند در روز بالای سر ایشان بود، و وقتی که از لشکرگاه روانه می‌شدند.
\par 35 و چون تابوت روانه می‌شد، موسی می‌گفت: «ای خداوند برخیز و دشمنانت پراکنده شوند و مبغضانت از حضور تو منهزم گردند.»و چون فرود می‌آمد، می‌گفت: «ای خداوند نزد هزاران هزار اسرائیل رجوع نما.»
\par 36 و چون فرود می‌آمد، می‌گفت: «ای خداوند نزد هزاران هزار اسرائیل رجوع نما.»
 
\chapter{11}

\par 1 و قوم شکایت‌کنان در گوش خداوند بدگفتند، و خداوند این را شنیده، غضبش افروخته شد، و آتش خداوند در میان ایشان مشتعل شده، در اطراف اردو بسوخت.
\par 2 و قوم نزد موسی فریاد برآورده، موسی نزد خداوند دعانمود و آتش خاموش شد.
\par 3 پس آن مکان راتبعیره نام نهادند، زیرا که آتش خداوند در میان ایشان مشتعل شد.
\par 4 و گروه مختلف که در میان ایشان بودند، شهوت‌پرست شدند، و بنی‌اسرائیل باز گریان شده، گفتند: «کیست که ما را گوشت بخوراند!
\par 5 ماهی‌ای را که در مصر مفت می‌خوردیم و خیارو خربوزه و تره و پیاز و سیر را بیاد می‌آوریم.
\par 6 والان جان ما خشک شده، و چیزی نیست و غیر ازاین من، در نظر ما هیچ نمی آید!»
\par 7 و من مثل تخم گشنیز بود و شکل آن مثل مقل.
\par 8 و قوم گردش کرده، آن را جمع می‌نمودند، و آن را در آسیا خرد می‌کردند یا در هاون می‌کوبیدند، و در دیگها پخته، گرده‌ها از آن می‌ساختند. و طعم آن مثل طعم قرصهای روغنی بود.
\par 9 و چون شبنم در وقت شب بر اردو می بارید، من نیز بر آن می‌ریخت.
\par 10 و موسی قوم را شنید که با اهل خانه خودهر یک به در خیمه خویش می‌گریستند، و خشم خداوند به شدت افروخته شد، و در نظر موسی نیز قبیح آمد.
\par 11 و موسی به خداوند گفت: «چرابه بنده خود بدی نمودی؟ و چرا در نظر تو التفات نیافتم که بار جمیع این قوم را بر من نهادی؟
\par 12 آیامن به تمامی این قوم حامله شده، یا من ایشان رازاییده‌ام که به من می‌گویی ایشان را در آغوش خود بردار، به زمینی که برای پدران ایشان قسم خوردی مثل لالا که طفل شیرخواره رابرمی دارد؟
\par 13 گوشت از کجا پیدا کنم تا به همه این قوم بدهم؟ زیرا نزد من گریان شده، می‌گویندما را گوشت بده تا بخوریم.
\par 14 من به تنهایی نمی توانم تحمل تمامی این قوم را بنمایم زیرا برمن زیاد سنگین است.
\par 15 و اگر با من چنین رفتارنمایی، پس هرگاه در نظر تو التفات یافتم مراکشته، نابود ساز تا بدبختی خود را نبینم.»
\par 16 پس خداوند موسی را خطاب کرده، گفت: «هفتاد نفر از مشایخ بنی‌اسرائیل که ایشان رامی دانی که مشایخ قوم و سروران آنها می‌باشندنزد من جمع کن، و ایشان را به خیمه اجتماع بیاورتا در آنجا با تو بایستند.
\par 17 و من نازل شده، درآنجا با تو سخن خواهم گفت، و از روحی که برتوست گرفته، بر ایشان خواهم نهاد تا با تومتحمل بار این قوم باشند و تو به تنهایی متحمل آن نباشی.
\par 18 «و قوم را بگو که برای فردا خود را تقدیس نمایید تا گوشت بخورید، چونکه در گوش خداوند گریان شده، گفتید، کیست که ما را گوشت بخوراند! زیرا که در مصر ما را خوش می‌گذشت! پس خداوند شما را گوشت خواهد داد تا بخورید.
\par 19 نه یک روز و نه دو روز خواهیدخورد، و نه پنج روز و نه ده روز و نه بیست روز،
\par 20 بلکه یک ماه تمام تا از بینی شما بیرون آید ونزد شما مکروه شود، چونکه خداوند را که درمیان شماست رد نمودید، و به حضور وی گریان شده، گفتید، چرا از مصر بیرون آمدیم.»
\par 21 موسی گفت: «قومی که من در میان ایشانم، ششصد هزار پیاده‌اند و تو گفتی ایشان را گوشت خواهم داد تا یک ماه تمام بخورند.
\par 22 آیا گله‌ها ورمه‌ها برای ایشان کشته شود تا برای ایشان کفایت کند؟ یا همه ماهیان دریا برای ایشان جمع شوند تا برای ایشان کفایت کند؟»
\par 23 خداوند موسی را گفت: «آیا دست خداوندکوتاه شده است؟ الان خواهی دید که کلام من برتو واقع می‌شود یا نه.»
\par 24 پس موسی بیرون آمده، سخنان خداوند رابه قوم گفت، و هفتاد نفر از مشایخ قوم را جمع کرده، ایشان را به اطراف خیمه برپا داشت.
\par 25 وخداوند در ابر نازل شده، با وی تکلم نمود، و ازروحی که بر وی بود، گرفته، بر آن هفتاد نفرمشایخ نهاد و چون روح بر ایشان قرار گرفت، نبوت کردند، لیکن مزید نکردند.
\par 26 اما دو نفر در لشکرگاه باقی ماندند که نام یکی الداد بود و نام دیگری میداد، و روح بر ایشان نازل شد و نامهای ایشان در ثبت بود، لیکن نزدخیمه نیامده، در لشکرگاه نبوت کردند.
\par 27 آنگاه جوانی دوید و به موسی خبر داده، گفت: «الداد ومیداد در لشکرگاه نبوت می‌کنند.»
\par 28 و یوشع بن نون خادم موسی که ازبرگزیدگان او بود، در جواب گفت: «ای آقایم موسی ایشان را منع نما!»
\par 29 موسی وی را گفت: «آیا تو برای من حسد می بری؟ کاشکه تمامی قوم خداوند نبی می‌بودندو خداوند روح خود را بر ایشان افاضه می‌نمود!»
\par 30 پس موسی با مشایخ اسرائیل به لشکرگاه آمدند.
\par 31 و بادی از جانب خداوند وزیده، سلوی رااز دریا برآورد و آنها را به اطراف لشکرگاه تخمین یک روز راه به این طرف و یک روز راه به آن طرف پراکنده ساخت، و قریب به دو ذراع از روی زمین بالا بودند.
\par 32 و قوم برخاسته تمام آن روز و تمام آن شب و تمام روز دیگر سلوی را جمع کردند وآنکه کمتر یافته بود، ده حومر جمع کرده بود، وآنها را به اطراف اردو برای خود پهن کردند.
\par 33 وگوشت هنوز در میان دندان ایشان می‌بود پیش ازآنکه خاییده شود، که غضب خداوند بر ایشان افروخته شده خداوند قوم را به بلای بسیارسخت مبتلا ساخت.
\par 34 و آن مکان را قبروت هتاوه نامیدند، زیرا قومی را که شهوت‌پرست شدند، در آنجا دفن کردند.و قوم از قبروت هتاوه به حضیروت کوچ کرده، در حضیروت توقف نمودند.
\par 35 و قوم از قبروت هتاوه به حضیروت کوچ کرده، در حضیروت توقف نمودند.
 
\chapter{12}

\par 1 و مریم و هارون درباره زن حبشی که موسی گرفته بود، بر او شکایت آوردند، زیرا زن حبشی گرفته بود.
\par 2 و گفتند: «آیا خداوندبا موسی به تنهایی تکلم نموده است، مگر به ما نیزتکلم ننموده؟» و خداوند این را شنید.
\par 3 و موسی مرد بسیار حلیم بود، بیشتر از جمیع مردمانی که در روی زمینند.
\par 4 در ساعت خداوند به موسی و هارون و مریم گفت: «شما هر سه نزد خیمه اجتماع بیرون آیید.» و هر سه بیرون آمدند.
\par 5 و خداوند در ستون ابرنازل شده، به در خیمه ایستاد، و هارون و مریم راخوانده، ایشان هر دو بیرون آمدند.
\par 6 و او گفت: «الان سخنان مرا بشنوید: اگر در میان شما نبی‌ای باشد، من که یهوه هستم، خود را در رویا بر اوظاهر می‌کنم و در خواب به او سخن می‌گویم.
\par 7 اما بنده من موسی چنین نیست. او در تمامی خانه من امین است.
\par 8 با وی روبرو و آشکارا و نه در رمزها سخن می‌گویم، و شبیه خداوند رامعاینه می‌بیند، پس چرا نترسیدید که بر بنده من موسی شکایت آوردید؟»
\par 9 و غضب خداوند برایشان افروخته شده، برفت.
\par 10 و چون ابر از روی خیمه برخاست، اینک مریم مثل برف مبروص بود، و هارون بر مریم نگاه کرد و اینک مبروص بود.
\par 11 و هارون به موسی گفت: «وای‌ای آقایم بار این گناه را بر ما مگذار زیرا که حماقت کرده، گناه ورزیده‌ایم.
\par 12 و او مثل میته‌ای نباشد که چون از رحم مادرش بیرون آید، نصف بدنش پوسیده باشد.»
\par 13 پس موسی نزد خداوند استغاثه کرده، گفت: «ای خدا او را شفا بده!»
\par 14 خداوند به موسی گفت: «اگر پدرش به روی وی فقط آب دهان می‌انداخت، آیا هفت روز خجل نمی شد؟ پس هفت روز بیرون لشکرگاه محبوس بشود، وبعد از آن داخل شود.»
\par 15 پس مریم هفت روزبیرون لشکرگاه محبوس ماند، و تا داخل شدن مریم، قوم کوچ نکردند.و بعد از آن، قوم از حضیروت کوچ کرده، در صحرای فاران اردو زدند.
\par 16 و بعد از آن، قوم از حضیروت کوچ کرده، در صحرای فاران اردو زدند.
 
\chapter{13}

\par 1 و خداوند موسی را خطاب کرده، گفت:
\par 2 «کسان بفرست تا زمین کنعان را که به بنی‌اسرائیل دادم، جاسوسی کنند؛ یک نفر را ازهر سبط آبای ایشان که هرکدام در میان ایشان سرور باشد، بفرستید.»
\par 3 پس موسی به فرمان خداوند، ایشان را ازصحرای فاران فرستاد، و همه ایشان از روسای بنی‌اسرائیل بودند.
\par 4 و نامهای ایشان اینهاست: ازسبط روبین، شموع بن زکور.
\par 5 از سبط شمعون، شافاط بن حوری.
\par 6 از سبط یهودا، کالیب بن یفنه.
\par 7 از سبط یساکار، یجال بن یوسف.
\par 8 از سبطافرایم، هوشع بن نون.
\par 9 از سبط بنیامین، فلطی بن رافو.
\par 10 از سبط زبولون، جدیئیل بن سودی.
\par 11 ازسبط یوسف از سبط بنی منسی، جدی بن سوسی.
\par 12 از سبط دان، عمیئیل بن جملی.
\par 13 از سبطاشیر، ستور بن میکائیل. 
\par 14 از سبط نفتالی، نحبی بن وفسی.
\par 15 از سبط جاد، جاوئیل بن ماکی.
\par 16 این است نامهای کسانی که موسی برای جاسوسی زمین فرستاد، و موسی هوشع بن نون رایهوشوع نام نهاد.
\par 17 و موسی ایشان را برای جاسوسی زمین کنعان فرستاده، به ایشان گفت: «از اینجا به جنوب رفته، به کوهستان برآیید.
\par 18 و زمین را ببینید که چگونه است و مردم را که در آن ساکنند که قوی‌اند یا ضعیف، قلیل‌اند یا کثیر.
\par 19 و زمینی که در آن ساکنند چگونه است، نیک یا بد؟ و در چه قسم شهرها ساکنند، در چادرها یا در قلعه‌ها؟
\par 20 و چگونه است زمین، چرب یا لاغر؟ درخت دارد یا نه؟ پس قوی‌دل شده، از میوه زمین بیاورید.» و آن وقت موسم نوبر انگور بود.
\par 21 پس رفته زمین را از بیابان سین تا رحوب، نزد مدخل حمات جاسوسی کردند.
\par 22 و به جنوب رفته، به حبرون رسیدند، و اخیمان وشیشای و تلمای بنی عناق در آنجا بودند، اماحبرون هفت سال قبل از صوعن مصر بنا شده بود.
\par 23 و به وادی اشکول آمدند، و شاخه‌ای با یک خوشه انگور بریده، آن را بر چوب دستی، میان دونفر با قدری از انار و انجیر برداشته، آوردند.
\par 24 وآن مکان به‌سبب خوشه انگور که بنی‌اسرائیل ازآنجا بریده بودند، به وادی اشکول نامیده شد.
\par 25 و بعد از چهل روز، از جاسوسی زمین برگشتند.
\par 26 و روانه شده، نزد موسی و هارون وتمامی جماعت بنی‌اسرائیل به قادش در بیابان فاران رسیدند، و برای ایشان و برای تمامی جماعت خبر‌آوردند، و میوه زمین را به ایشان نشان دادند.
\par 27 و برای او حکایت کرده، گفتند: «به زمینی که ما را فرستادی رفتیم، و به درستی که به شیر و شهد جاریست، و میوه‌اش این است.
\par 28 لیکن مردمانی که در زمین ساکنند زورآورند، وشهرهایش حصاردار و بسیار عظیم، و بنی عناق رانیز در آنجا دیدیم.
\par 29 و عمالقه در زمین جنوب ساکنند، و حتیان و یبوسیان و اموریان درکوهستان سکونت دارند. و کنعانیان نزد دریا و برکناره اردن ساکنند.»
\par 30 و کالیب قوم را پیش موسی خاموش ساخته، گفت: «فی الفور برویم و آن را در تصرف آریم، زیرا که می‌توانیم بر آن غالب شویم.»
\par 31 اماآن کسانی که با وی رفته بودند، گفتند: «نمی توانیم با این قوم مقابله نماییم زیرا که ایشان از ماقوی ترند.»
\par 32 و درباره زمینی که آن را جاسوسی کرده بودند، خبر بد نزد بنی‌اسرائیل آورده، گفتند: «زمینی که برای جاسوسی آن از آن گذشتیم زمینی است که ساکنان خود را می‌خورد، و تمامی قومی که در آن دیدیم، مردان بلند قدبودند.و در آنجا جباران بنی عناق را دیدیم که اولاد جبارانند، و ما در نظر خود مثل ملخ بودیم وهمچنین در نظر ایشان می‌نمودیم.»
\par 33 و در آنجا جباران بنی عناق را دیدیم که اولاد جبارانند، و ما در نظر خود مثل ملخ بودیم وهمچنین در نظر ایشان می‌نمودیم.»
 
\chapter{14}

\par 1 و تمامی جماعت آواز خود را بلندکرده، فریاد نمودند. و قوم در آن شب می‌گریستند.
\par 2 و جمیع بنی‌اسرائیل بر موسی وهارون همهمه کردند، و تمامی جماعت به ایشان گفتند: «کاش که در زمین مصر می‌مردیم یا در این صحرا وفات می‌یافتیم!
\par 3 و چرا خداوند ما را به این زمین می‌آورد تا به دم شمشیر بیفتیم، و زنان واطفال ما به یغما برده شوند، آیا برگشتن به مصربرای ما بهتر نیست؟»
\par 4 و به یکدیگر گفتند: «سرداری برای خود مقرر کرده، به مصربرگردیم.»
\par 5 پس موسی و هارون به حضور تمامی گروه جماعت بنی‌اسرائیل به رو افتادند.
\par 6 و یوشع بن نون و کالیب بن یفنه که از جاسوسان زمین بودند، رخت خود را دریدند.
\par 7 و تمامی جماعت بنی‌اسرائیل را خطاب کرده، گفتند: «زمینی که برای جاسوسی آن از آن عبور نمودیم، زمین بسیار بسیار خوبیست.
\par 8 اگر خداوند از ما راضی است ما را به این زمین آورده، آن را به ما خواهدبخشید، زمینی که به شیر و شهد جاری است.
\par 9 زنهار از خداوند متمرد نشوید، و از اهل زمین ترسان مباشید، زیراکه ایشان خوراک ما هستند، سایه ایشان از ایشان گذشته است، و خداوند با ماست، از ایشان مترسید.»
\par 10 لیکن تمامی جماعت گفتند که باید ایشان را سنگسار کنند. آنگاه جلال خداوند در خیمه اجتماع بر تمامی بنی‌اسرائیل ظاهر شد.
\par 11 وخداوند به موسی گفت: «تا به کی این قوم مرااهانت نمایند؟ و تا به کی با وجود همه آیاتی که در میان ایشان نمودم، به من ایمان نیاورند؟
\par 12 ایشان را به وبا مبتلا ساخته، هلاک می‌کنم و ازتو قومی بزرگ و عظیم تر از ایشان خواهم ساخت.»
\par 13 موسی به خداوند گفت: «آنگاه مصریان خواهند شنید، زیراکه این قوم را به قدرت خود ازمیان ایشان بیرون آوردی.
\par 14 و به ساکنان این زمین خبر خواهند داد و ایشان شنیده‌اند که تو‌ای خداوند، در میان این قوم هستی، زیراکه تو‌ای خداوند، معاینه دیده می‌شوی، و ابر تو بر ایشان قایم است، و تو پیش روی ایشان روز در ستون ابرو شب در ستون آتش می‌خرامی.
\par 15 پس اگر این قوم را مثل شخص واحد بکشی، طوایفی که آوازه تو را شنیده‌اند، خواهند گفت:
\par 16 چون که خداوند نتوانست این قوم را به زمینی که برای ایشان قسم خورده بود درآورد از این سبب ایشان را در صحرا کشت.
\par 17 پس الان قدرت خداوندعظیم بشود، چنانکه گفته بودی
\par 18 که یهوه دیرخشم و بسیار رحیم و آمرزنده گناه و عصیان است، لیکن مجرم را هرگز بی‌سزا نخواهدگذاشت بلکه عقوبت گناه پدران را بر پسران تاپشت سوم و چهارم می‌رساند.
\par 19 پس گناه این قوم را برحسب عظمت رحمت خود بیامرز، چنانکه این قوم را از مصر تا اینجا آمرزیده‌ای.»
\par 20 و خداوند گفت: «برحسب کلام توآمرزیدم.
\par 21 لیکن به حیات خودم قسم که تمامی زمین از جلال یهوه پر خواهد شد.
\par 22 چونکه جمیع مردانی که جلال و آیات مرا که در مصر وبیابان نمودم دیدند، مرا ده مرتبه امتحان کرده، آواز مرا نشنیدند.
\par 23 به درستی که ایشان زمینی راکه برای پدران ایشان قسم خوردم، نخواهند دید، و هرکه مرا اهانت کرده باشد، آن را نخواهد دید.
\par 24 لیکن بنده من کالیب چونکه روح دیگر داشت ومرا تمام اطاعت نمود، او را به زمینی که رفته بودداخل خواهم ساخت، و ذریت او وارث آن خواهند شد.
\par 25 و چونکه عمالیقیان و کنعانیان دروادی ساکنند، فردا رو گردانیده، از راه بحر قلزم به صحرا کوچ کنید.»
\par 26 و خداوند موسی و هارون را خطاب کرده، گفت:
\par 27 «تا به کی این جماعت شریر را که بر من همهمه می‌کنند متحمل بشوم؟ همهمه بنی‌اسرائیل را که بر من همهمه می‌کنند، شنیدم.
\par 28 به ایشان بگو خداوند می‌گوید: به حیات خودم قسم که چنانکه شما در گوش من گفتید، همچنان با شما عمل خواهم نمود.
\par 29 لاشه های شما دراین صحرا خواهد افتاد، و جمیع شمرده شدگان شما برحسب تمامی عدد شما، از بیست ساله وبالاتر که بر من همهمه کرده‌اید.
\par 30 شما به زمینی که درباره آن دست خود را بلند کردم که شما را درآن ساکن گردانم، هرگز داخل نخواهید شد، مگرکالیب بن یفنه و یوشع بن نون.
\par 31 اما اطفال شما که درباره آنها گفتید که به یغما برده خواهند شد، ایشان را داخل خواهم کرد و ایشان زمینی را که شما رد کردید، خواهند دانست.
\par 32 لیکن لاشه های شما در این صحرا خواهد افتاد.
\par 33 وپسران شما در این صحرا چهل سال آواره بوده، بار زناکاری شما را متحمل خواهند شد، تالاشه های شما در صحرا تلف شود.
\par 34 برحسب شماره روزهایی که زمین را جاسوسی می‌کردید، یعنی چهل روز. یک سال به عوض هر روز، بارگناهان خود را چهل سال متحمل خواهید شد، ومخالفت مرا خواهید دانست.
\par 35 من که یهوه هستم، گفتم که البته این را به تمامی این جماعت شریر که به ضد من جمع شده‌اند خواهم کرد، ودر این صحرا تلف شده، در اینجا خواهند مرد.»
\par 36 و اما آن کسانی که موسی برای جاسوسی زمین فرستاده بود. و ایشان چون برگشتند خبر بددرباره زمین آورده، تمام جماعت را از او گله‌مندساختند.
\par 37 آن کسانی که این خبر بد را درباره زمین آورده بودند، به حضور خداوند از وبامردند.
\par 38 اما یوشع بن نون و کالیب بن یفنه ازجمله آنانی که برای جاسوسی زمین رفته بودند، زنده ماندند.
\par 39 و چون موسی این سخنان را به جمیع بنی‌اسرائیل گفت، قوم بسیار گریستند.
\par 40 وبامدادان به زودی برخاسته، به‌سر کوه برآمده، گفتند: «اینک حاضریم و به مکانی که خداوندوعده داده است می‌رویم، زیرا گناه کرده‌ایم.»
\par 41 موسی گفت: «چرا از فرمان خداوند تجاوزمی نمایید؟ لیکن این کار به کام نخواهد شد!
\par 42 مروید زیرا خداوند در میان شما نیست، مبادااز پیش دشمنان خود منهزم شوید.
\par 43 زیراعمالیقیان و کنعانیان آنجا پیش روی شما هستند، پس به شمشیر خواهید افتاد، و چونکه از پیروی خداوند روگردانیده‌اید، لهذا خداوند با شمانخواهد بود.»
\par 44 لیکن ایشان از راه تکبر به‌سر کوه رفتند، اماتابوت عهد خداوند و موسی از میان لشکرگاه بیرون نرفتند.آنگاه عمالیقیان و کنعانیان که درآن کوهستان ساکن بودند فرودآمده، ایشان رازدند و تا حرما منهزم ساختند.
\par 45 آنگاه عمالیقیان و کنعانیان که درآن کوهستان ساکن بودند فرودآمده، ایشان رازدند و تا حرما منهزم ساختند.
 
\chapter{15}

\par 1 و خداوند موسی را خطاب کرده، گفت:
\par 2 «بنی‌اسرائیل را خطاب کرده، به ایشان بگو: چون به زمین سکونت خود که من آن را به شما می‌دهم داخل شوید،
\par 3 و می‌خواهیدهدیه آتشین برای خداوند بگذرانید، چه قربانی سوختنی و چه ذبیحه وفای نذر، یا برای نافله یادر عیدهای خود، برای گذرانیدن هدیه خوشبوبجهت خداوند، خواه از رمه و خواه از گله،
\par 4 آنگاه کسی‌که هدیه خود را می‌گذراند، برای هدیه آردی یک عشر ایفه آرد نرم مخلوط شده بایک ربع هین روغن بجهت خداوند بگذراند.
\par 5 وبرای هدیه ریختنی یک ربع هین شراب با قربانی سوختنی یا برای ذبیحه بجهت هر بره حاضرکن.
\par 6 «یا بجهت قوچ برای هدیه آردی دو عشرایفه آرد نرم مخلوط شده با یک ثلث هین روغن حاضرکن.
\par 7 و بجهت هدیه ریختنی یک ثلث هین شراب برای خوشبویی بجهت خداوندحاضرکن.
\par 8 «و چون گاوی برای قربانی سوختنی یاذبیحه‌ای برای ادای نذر یا برای ذبیحه سلامتی بجهت خداوند حاضر می‌کنی،
\par 9 آنگاه بجهت هدیه آردی، سه عشر آرد نرم مخلوط شده بانصف هین روغن با گاو بگذراند.
\par 10 و برای هدیه ریختنی نصف هین شراب بگذران تا هدیه آتشین خوشبو برای خداوند بشود.
\par 11 «همچنین برای هر گاو و برای هر قوچ وبرای هر بره نرینه و هر بزغاله کرده شود.
\par 12 برحسب شماره‌ای که حاضر کنید بدین قسم برای هریک، موافق شماره آنها عمل نمایید.
\par 13 «هر متوطن چون هدیه آتشین خوشبو برای خداوند می‌گذراند، این اوامر را به اینطور بجابیاورد.
\par 14 و اگر غریبی که در میان شما ماواگزیند، هرکه در قرنهای شما در میان شما باشد، می‌خواهد هدیه آتشین خوشبو برای خداوندبگذراند، به نوعی که شما عمل می‌نمایید، او نیزعمل نماید.
\par 15 برای شما که اهل جماعت هستیدو برای غریبی که نزد شما ماوا گزیند یک فریضه باشد، فریضه ابدی در نسلهای شما؛ مثل شما به حضور خداوند مثل غریب است.
\par 16 یک قانون ویک حکم برای شما و برای غریبی که در میان شما ماوا گزیند، خواهد بود.»
\par 17 و خداوند موسی را خطاب کرده، گفت:
\par 18 «بنی‌اسرائیل را خطاب کرده، به ایشان بگو: چون به زمینی که من شما را در آن درمی آورم داخل شوید،
\par 19 و از محصول زمین بخورید، آنگاه هدیه افراشتنی برای خداوند بگذرانید.
\par 20 از خمیر اول خود گرده‌ای بجهت هدیه افراشتنی بگذرانید؛ مثل هدیه افراشتنی خرمن، همچنان آن را بگذرانید.
\par 21 از خمیر اول خود، هدیه افراشتنی در قرنهای خود برای خداوند بگذرانید.
\par 22 «و هرگاه سهو خطا کرده، جمیع این اوامررا که خداوند به موسی گفته است، بجا نیاورده باشید،
\par 23 یعنی هرچه خداوند به واسطه موسی شما را امر فرمود، از روزی که خداوند امر فرمودو از آن به بعد در قرنهای شما. 
\par 24 پس اگر این کارسهو و بدون اطلاع جماعت کرده شد، آنگاه تمامی جماعت یک گاو جوان برای قربانی سوختنی و خوشبویی بجهت خداوند با هدیه آردی و هدیه ریختنی آن، موافق رسم بگذرانند، و یک بز نر بجهت قربانی گناه.
\par 25 و کاهن برای تمامی جماعت بنی‌اسرائیل کفاره نماید، و ایشان آمرزیده خواهندشد، زیراکه آن کار سهو شده است، و ایشان قربانی خود را بجهت هدیه آتشین خداوند و قربانی گناه خود را بجهت سهوخویش، به حضور خداوند گذرانیده‌اند.
\par 26 وتمامی جماعت بنی‌اسرائیل و غریبی که در میان ایشان ساکن باشد، آمرزیده خواهند شد، زیراکه به تمامی جماعت سهو شده بود.
\par 27 «و اگر یک نفر سهو خطا کرده باشد، آنگاه بز ماده‌یک ساله برای قربانی گناه بگذراند.
\par 28 وکاهن بجهت آن کسی‌که سهو کرده است چونکه خطای او از نادانستگی بود، به حضور خداوندکفاره کند تا بجهت وی کفاره بشود و آمرزیده خواهد شد.
\par 29 بجهت کسی‌که سهو خطا کند، خواه متوطنی از بنی‌اسرائیل و خواه غریبی که در میان ایشان ساکن باشد، یک قانون خواهدبود.
\par 30 «و اما کسی‌که به‌دست بلند عمل نماید، چه متوطن و چه غریب، او به خداوند کفر کرده باشد، پس آن شخص از میان قوم خود منقطع خواهد شد.
\par 31 چونکه کلام خداوند را حقیرشمرده، حکم او را شکسته است، آن کس البته منقطع شود و گناهش بر وی خواهد بود.»
\par 32 و چون بنی‌اسرائیل در صحرا بودند، کسی را یافتند که در روز سبت هیزم جمع می‌کرد.
\par 33 وکسانی که او را یافتند که هیزم جمع می‌کرد، او رانزد موسی و هارون و تمامی جماعت آوردند.
\par 34 و او را در حبس نگاه داشتند، زیراکه اعلام نشده بود که با وی چه باید کرد.
\par 35 و خداوند به موسی گفت: «این شخص البته کشته شود، تمامی جماعت او را بیرون از لشکرگاه با سنگها سنگسارکنند.»
\par 36 پس تمامی جماعت او را بیرون ازلشکرگاه آورده، او را سنگسار کردند و بمرد، چنانکه خداوند به موسی‌امر کرده بود.
\par 37 و خداوند موسی را خطاب کرده، گفت:
\par 38 «بنی‌اسرائیل را خطاب کرده، به ایشان بگو که: برای خود بر گوشه های رخت خویش در قرنهای خود صیصیت بسازند و رشته لاجوردی بر هرگوشه صیصیت بگذارند.
\par 39 و بجهت شماصیصیت خواهد بود تا برآن بنگرید و تمام اوامرخداوند را بیاد آورده، بجا آورید، و در‌پی دلها وچشمان خود که شما در‌پی آنها زنا می‌کنید، منحرف نشوید.
\par 40 تا تمامی اوامر مرا بیاد آورده، بجا آورید، و بجهت خدای خود مقدس باشید.من یهوه خدای شما هستم که شما را از زمین مصر بیرون آوردم تا خدای شما باشم. من یهوه خدای شما هستم.»
\par 41 من یهوه خدای شما هستم که شما را از زمین مصر بیرون آوردم تا خدای شما باشم. من یهوه خدای شما هستم.»
 
\chapter{16}

\par 1 و قورح بن یصهار بن قهات بن لاوی وداتان و ابیرام پسران الیاب و اون بن فالت پسران روبین (کسان ) گرفته،
\par 2 با بعضی ازبنی‌اسرائیل، یعنی دویست و پنجاه نفر ازسروران جماعت که برگزیدگان شورا و مردان معروف بودند، به حضور موسی برخاستند.
\par 3 و به مقابل موسی و هارون جمع شده، به ایشان گفتند: «شما از حد خود تجاوز می‌نمایید، زیرا تمامی جماعت هریک از ایشان مقدس‌اند، و خداوند درمیان ایشان است. پس چرا خویشتن را بر جماعت خداوند برمی افرازید؟»
\par 4 و چون موسی این راشنید به روی خود درافتاد.
\par 5 و قورح و تمامی جمعیت او را خطاب کرده، گفت: «بامدادان خداوند نشان خواهد داد که چه کس از آن وی وچه کس مقدس است، و او را نزد خود خواهدآورد، و هرکه را برای خود برگزیده است، او رانزد خود خواهد آورد.
\par 6 این را بکنید که مجمرهابرای خود بگیرید، ای قورح و تمامی جمعیت تو.
\par 7 و آتش در آنها گذارده، فردا به حضورخداوند بخور در آنها بریزید، و آن کس که خداوند برگزیده است، مقدس خواهد شد. ای پسران لاوی شما از حد خود تجاوز می‌نمایید!»
\par 8 و موسی به قورح گفت: «ای بنی لاوی بشنوید!
\par 9 آیا نزد شما کم است که خدای اسرائیل شما رااز جماعت اسرائیل ممتاز کرده است، تا شما رانزد خود بیاورد تا در مسکن خداوندخدمت نمایید، و به حضور جماعت برای خدمت ایشان بایستید؟
\par 10 و تو را و جمیع برادرانت بنی لاوی رابا تو نزدیک آورد، و آیا کهانت را نیز می‌طلبید؟
\par 11 از این جهت تو و تمامی جمعیت تو به ضدخداوند جمع شده‌اید، و اما هارون چیست که براو همهمه می‌کنید؟»
\par 12 و موسی فرستاد تا داتان و ابیرام پسران الیاب را بخواند، و ایشان گفتند: «نمی آییم!
\par 13 آیاکم است که ما را از زمینی که به شیر و شهد جاری است، بیرون آوردی تا ما را در صحرا نیز هلاک سازی که می‌خواهی خود را بر ما حکمران سازی؟
\par 14 و ما را هم به زمینی که به شیر و شهدجاری است درنیاوردی و ملکیتی از مزرعه‌ها وتاکستانها به ما ندادی. آیا چشمان این مردمان رامی کنی؟ نخواهیم آمد!»
\par 15 و موسی بسیار خشمناک شده، به خداوندگفت: «هدیه ایشان را منظور منما، یک خر ازایشان نگرفتم، و به یکی از ایشان زیان نرساندم.»
\par 16 و موسی به قورح گفت: «تو با تمامی جمعیت خود فردا به حضور خداوند حاضر شوید، تو وایشان و هارون.
\par 17 و هر کس مجمر خود راگرفته، بخور بر آنها بگذارد و شما هر کس مجمرخود، یعنی دویست و پنجاه مجمر به حضورخداوند بیاورید، تو نیز و هارون هر یک مجمرخود را بیاورید.»
\par 18 پس هر کس مجمر خود راگرفته، و آتش در آنها نهاده، و بخور بر آنهاگذارده، نزد دروازه خیمه اجتماع، با موسی وهارون ایستادند.
\par 19 و قورح تمامی جماعت را به مقابل ایشان نزد در خیمه اجتماع جمع کرد، وجلال خداوند بر تمامی جماعت ظاهرشد.
\par 20 و خداوند موسی و هارون را خطاب کرده، گفت:
\par 21 «خود را از این جماعت دور کنید تاایشان را در لحظه‌ای هلاک کنم.»
\par 22 پس ایشان به روی در‌افتاده، گفتند: «ای خدا که خدای روحهای تمام بشر هستی، آیا یک نفر گناه ورزد وبر تمام جماعت غضبناک شوی؟»
\par 23 و خداوند موسی را خطاب کرده، گفت:
\par 24 «جماعت را خطاب کرده، بگو از اطراف مسکن قورح و داتان و ابیرام دور شوید.»
\par 25 پس موسی برخاسته، نزد داتان و ابیرام رفت و مشایخ اسرائیل در عقب وی رفتند.
\par 26 و جماعت راخطاب کرده، گفت: «از نزد خیمه های این مردمان شریر دور شوید، و چیزی را که از آن ایشان است لمس منمایید، مبادا در همه گناهان ایشان هلاک شوید.»
\par 27 پس از اطراف مسکن قورح و داتان وابیرام دور شدند، و داتان و ابیرام بیرون آمده، بازنان و پسران و اطفال خود به در خیمه های خودایستادند.
\par 28 و موسی گفت: «از این خواهیددانست که خداوند مرا فرستاده است تا همه این کارها را بکنم و به اراده من نبوده است.
\par 29 اگر این کسان مثل موت سایر بنی آدم بمیرند و اگر مثل وقایع جمیع بنی آدم بر ایشان واقع شود، خداوندمرا نفرستاده است.
\par 30 و اما اگر خداوند چیزتازه‌ای بنماید و زمین دهان خود را گشاده، ایشان را با جمیع مایملک ایشان ببلعد که به گور زنده فرود روند، آنگاه بدانید که این مردمان خداوند رااهانت نموده‌اند.»
\par 31 و چون از گفتن همه این سخنان فارغ شد، زمینی که زیر ایشان بود، شکافته شد.
\par 32 و زمین دهان خود را گشوده، ایشان را و خانه های ایشان و همه کسان را که تعلق به قورح داشتند، با تمامی اموال ایشان بلعید.
\par 33 و ایشان با هرچه به ایشان تعلق داشت، زنده به گور فرورفتند، و زمین برایشان به هم آمد که از میان جماعت هلاک شدند.
\par 34 و جمیع اسرائیلیان که به اطراف ایشان بودند، از نعره ایشان گریختند، زیرا گفتند مبادا زمین ما رانیز ببلعد.
\par 35 و آتش از حضور خداوند بدر‌آمده، دویست و پنجاه نفر را که بخور می‌گذرانیدند، سوزانید.
\par 36 و خداوند موسی را خطاب کرده، گفت:
\par 37 «به العازار بن هارون کاهن بگو که مجمرها را ازمیان آتش بردار، و آتش را به آن طرف بپاش زیراکه آنها مقدس است.
\par 38 یعنی مجمرهای این گناهکاران را به ضد جان ایشان و از آنها تختهای پهن برای پوشش مذبح بسازند، زیرا چونکه آنهارا به حضور خداوند گذرانیده‌اند مقدس شده است، تا برای بنی‌اسرائیل آیتی باشد.»
\par 39 پس العازار کاهن مجمرهای برنجین را که سوخته شدگان گذرانیده بودند گرفته، از آنها پوشش مذبح ساختند.
\par 40 تا برای بنی‌اسرائیل یادگارباشد تا هیچ غریبی که از اولاد هارون نباشدبجهت سوزانیدن بخور به حضور خداوند نزدیک نیاید، مبادا مثل قورح و جمعیتش بشود، چنانکه خداوند به واسطه موسی او را امر فرموده بود.
\par 41 و در فردای آن روز تمامی جماعت بنی‌اسرائیل بر موسی و هارون همهمه کرده، گفتند که شما قوم خداوند را کشتید.
\par 42 و چون جماعت بر موسی و هارون جمع شدند، به سوی خیمه اجتماع نگریستند، و اینک ابر آن راپوشانید و جلال خداوند ظاهر شد.
\par 43 و موسی وهارون پیش خیمه اجتماع آمدند.
\par 44 و خداوندموسی را خطاب کرده، گفت:
\par 45 «از میان این جماعت دور شوید تا ایشان را ناگهان هلاک سازم.» و ایشان به روی خود درافتادند.
\par 46 و موسی به هارون گفت: «مجمر خود راگرفته، آتش از روی مذبح در آن بگذار، و بخور برآن بریز، و به زودی به سوی جماعت رفته، برای ایشان کفاره کن، زیرا غضب از حضور خداوندبرآمده، و وبا شروع شده است.»
\par 47 پس هارون به نحوی که موسی گفته بود آن را گرفته، در میان جماعت دوید و اینک وبا در میان قوم شروع شده بود، پس بخور را بریخت و بجهت قوم کفاره نمود.
\par 48 و او در میان مردگان و زندگان ایستاد ووبا بازداشته شد.
\par 49 و عدد کسانی که از وبا مردندچهارده هزار و هفتصد بود، سوای آنانی که درحادثه قورح هلاک شدند.پس هارون نزدموسی به در خیمه اجتماع برگشت و وبا رفع شد.
\par 50 پس هارون نزدموسی به در خیمه اجتماع برگشت و وبا رفع شد.
 
\chapter{17}

\par 1 و خداوند موسی را خطاب کرده، گفت:
\par 2 «به بنی‌اسرائیل سخن بگو و از ایشان عصاها بگیر، یک عصا از هر خاندان آبا، از جمیع سروران ایشان دوازده عصا برحسب خاندان آبای ایشان. و نام هرکس را بر عصای او بنویس.
\par 3 واسم هارون را بر عصای لاوی بنویس، زیراکه برای هر سرور خاندان آبای ایشان یک عصاخواهد بود.
\par 4 و آنها را در خیمه اجتماع پیش شهادت، جایی که من با شما ملاقات می‌کنم بگذار.
\par 5 و شخصی را که من اختیار می‌کنم عصای او شکوفه خواهدآورد، پس همهمه بنی‌اسرائیل را که بر شما می‌کنند از خود ساکت خواهم نمود.»
\par 6 و موسی این را به بنی‌اسرائیل گفت، پس جمیع سروران ایشان او را عصاها دادند، یک عصابرای هر سرور، یعنی دوازده عصا برحسب خاندان آبای ایشان، و عصاهای هارون در میان عصاهای آنها بود.
\par 7 و موسی عصاها را به حضور خداوند در خیمه شهادت گذارد.
\par 8 و در فردای آن روز چون موسی به خیمه شهادت داخل شد، اینک عصای هارون که بجهت خاندان لاوی بودشکفته بود، و شکوفه آورده و گل داده، و بادام رسانیده بود.
\par 9 و موسی همه عصاها را از حضورخداوند نزد جمیع بنی‌اسرائیل بیرون آورده، هریک نگاه کرده، عصای خود را گرفتند.
\par 10 و خداوند به موسی گفت: «عصای هارون را پیش روی شهادت باز بگذار تا بجهت علامت برای ابنای تمرد نگاه داشته شود، و همهمه ایشان را از من رفع نمایی تا نمیرند.»
\par 11 پس موسی چنان کرد، و به نحوی که خداوند او را امر فرموده بود، عمل نمود.
\par 12 وبنی‌اسرائیل به موسی عرض کرده، گفتند: «اینک فانی و هلاک می‌شویم. جمیع ما هلاک شده‌ایم!هرکه نزدیک می‌آید که به مسکن خداوندنزدیک می‌آید می‌میرد. آیا تمام فانی شویم؟»
\par 13 هرکه نزدیک می‌آید که به مسکن خداوندنزدیک می‌آید می‌میرد. آیا تمام فانی شویم؟»
 
\chapter{18}

\par 1 و خداوند به هارون گفت: «تو و پسرانت و خاندان آبایت با تو، گناه مقدس رامتحمل شوید، و تو و پسرانت با تو، گناه کهانت خود را متحمل شوید.
\par 2 و هم برادران خود یعنی سبط لاوی راکه سبط آبای توباشند باخود نزدیک بیاور تا با تو متفق شده، تو را خدمت نمایند، و اماتو با پسرانت پیش خیمه شهادت باشید. 
\par 3 وایشان ودیعت تو را و ودیعت تمامی مسکن رانگاه دارند، لیکن به اسباب قدس و به مذبح نزدیک نیایند مبادا بمیرند، ایشان و شما نیز.
\par 4 و ایشان باتو متفق شده، ودیعت خیمه اجتماع را با تمامی خدمت خیمه بجا آورند و غریبی به شما نزدیک نیاید.
\par 5 و ودیعت قدس و ودیعت مذبح را نگاه دارید تا غضب بر بنی‌اسرائیل دیگر مستولی نشود.
\par 6 و اما من اینک برادران شما لاویان را ازمیان بنی‌اسرائیل گرفتم، و برای شما پیشکش می‌باشند که به خداوند داده شده‌اند، تا خدمت خیمه اجتماع را بجا آورند.
\par 7 و اما تو با پسرانت، کهانت خود را بجهت هر کار مذبح و برای آنچه اندرون حجاب است نگاه دارید، و خدمت بکنید. کهانت را به شما دادم تا خدمت از راه بخشش باشد، و غریبی که نزدیک آید، کشته شود.»
\par 8 و خداوند به هارون گفت: «اینک من ودیعت هدایای افراشتنی خود را با همه‌چیزهای مقدس بنی‌اسرائیل به تو بخشیدم. آنها را به تو و پسرانت به‌سبب مسح شدن به فریضه ابدی دادم.
\par 9 ازقدس اقداس که از آتش نگاه داشته شود این از آن تو خواهد بود، هر هدیه ایشان یعنی هر هدیه آردی و هر قربانی گناه و هر قربانی جرم ایشان که نزد من بگذرانند، اینها برای تو و پسرانت قدس اقداس باشد.
\par 10 مثل قدس اقداس آنها را بخور. هر ذکور از آن بخورد، برای تو مقدس باشد.
\par 11 واین هم از آن تو باشد، هدیه افراشتنی از عطایای ایشان با هر هدیه جنبانیدنی بنی‌اسرائیل را به تو وبه پسرانت و دخترانت به فریضه ابدی دادم، هرکه در خانه تو طاهر باشد، از آن بخورد.
\par 12 تمامی بهترین روغن و تمامی بهترین حاصل مو و غله یعنی نوبرهای آنها را که به خداوند می‌دهند، به تو بخشیدم.
\par 13 و نوبرهای هرچه در زمین ایشان است که نزد خداوند می‌آورند از آن تو باشد، هرکه در خانه تو طاهر باشد، از آن بخورد.
\par 14 و هرچه در اسرائیل وقف بشود، از آن تو باشد.
\par 15 وهرچه رحم را گشاید از هر ذی جسدی که برای خداوند می‌گذرانند چه از انسان و چه از بهایم ازآن تو باشد، اما نخست زاده انسان را البته فدیه دهی، و نخست زاده بهایم ناپاک را فدیه‌ای بده.
\par 16 «و اما درباره فدیه آنها، آنها را از یک ماهه به حساب خود به پنج مثقال نقره، موافق مثقال قدس که بیست جیره باشد فدیه بده.
\par 17 ولی نخست زاده گاو یا نخست زاده گوسفند یانخست زاده بز را فدیه ندهی؛ آنها مقدسند، خون آنها را بر مذبح بپاش و پیه آنها را بجهت هدیه آتشین و عطر خوشبو برای خداوند بسوزان.
\par 18 وگوشت آنها مثل سینه جنبانیدنی، از آن تو باشد وران راست، از آن تو باشد.
\par 19 جمیع هدایای افراشتنی را از چیزهای مقدس که بنی‌اسرائیل برای خداوند می‌گذرانند به تو و پسرانت ودخترانت با تو به فریضه ابدی دادم، این به حضورخداوند برای تو و ذریت تو با تو عهد نمک تا به ابد خواهد بود.»
\par 20 و خداوند به هارون گفت: «تو در زمین ایشان هیچ ملک نخواهی یافت، و در میان ایشان برای تو نصیبی نخواهد بود، نصیب تو و ملک تودر میان بنی‌اسرائیل من هستم.
\par 21 «و به بنی لاوی‌اینک تمامی عشر اسرائیل را برای ملکیت دادم، به عوض خدمتی که می‌کنند یعنی خدمت خیمه اجتماع.
\par 22 و بعد ازاین بنی‌اسرائیل به خیمه اجتماع نزدیک نیایند، مبادا گناه را متحمل شده، بمیرند.
\par 23 اما لاویان خدمت خیمه اجتماع را بکنند و متحمل گناه ایشان بشوند، این در قرنهای شما فریضه‌ای ابدی خواهد بود، و ایشان در میان بنی‌اسرائیل ملک نخواهندیافت.
\par 24 زیراکه عشر بنی‌اسرائیل را که آن را نزد خداوند برای هدیه افراشتنی بگذرانندبه لاویان بجهت ملک بخشیدم، بنابراین به ایشان گفتم که در میان بنی‌اسرائیل ملک نخواهندیافت.»
\par 25 و خداوند موسی را خطاب کرده، گفت:
\par 26 «که لاویان را نیز خطاب کرده، به ایشان بگو: چون عشری را که از بنی‌اسرائیل به شما برای ملکیت دادم از ایشان بگیرید، آنگاه هدیه افراشتنی خداوند را از آن، یعنی عشری از عشربگذرانید.
\par 27 و هدیه افراشتنی شما برای شما، مثل غله خرمن و پری چرخشت حساب می‌شود.
\par 28 بدینطور شما نیز از همه عشرهایی که ازبنی‌اسرائیل می‌گیرید، هدیه افراشتنی برای خداوند بگذرانید، و از آنها هدیه افراشتنی خداوند را به هارون کاهن بدهید.
\par 29 از جمیع هدایای خود هر هدیه خداوند را از تمامی پیه آنها و از قسمت مقدس آنها بگذرانید.
\par 30 و ایشان را بگو هنگامی که پیه آنها را از آنها گذرانیده باشید، آنگاه برای لاویان مثل محصول خرمن وحاصل چرخشت حساب خواهد شد.
\par 31 و شماو خاندان شما آن را در هرجا بخورید زیراکه این مزد شما است، به عوض خدمتی که در خیمه اجتماع می‌کنید.و چون پیه آنها را از آنهاگذرانیده باشید، پس به‌سبب آنها متحمل گناه نخواهید بود، و چیزهای مقدس بنی‌اسرائیل راناپاک نکنید، مبادا بمیرند.»
\par 32 و چون پیه آنها را از آنهاگذرانیده باشید، پس به‌سبب آنها متحمل گناه نخواهید بود، و چیزهای مقدس بنی‌اسرائیل راناپاک نکنید، مبادا بمیرند.»
 
\chapter{19}

\par 1 و خداوند موسی و هارون را خطاب کرده، گفت:
\par 2 «این است فریضه شریعتی که خداوند آن را امر فرموده، گفت: به بنی‌اسرائیل بگو که گاو سرخ پاک که در آن عیب نباشد و یوغ بر گردنش نیامده باشد، نزد توبیاورند.
\par 3 و آن را به العازار کاهن بدهید، و آن رابیرون از لشکرگاه برده، پیش روی وی کشته شود.
\par 4 و العازار کاهن به انگشت خود از خون آن بگیرد، و به سوی پیشگاه خیمه اجتماع آن خون را هفت مرتبه بپاشد.
\par 5 و گاو در نظر او سوخته شود، پوست و گوشت و خون با سرگین آن سوخته شود.
\par 6 و کاهن چوب سرو با زوفا و قرمز گرفته، آنها را در میان آتش گاو بیندازد.
\par 7 پس کاهن رخت خود را بشوید و بدن خود را به آب غسل دهد، و بعد از آن در لشکرگاه داخل شود و کاهن تا شام نجس باشد.
\par 8 و کسی‌که آن را سوزانید، رخت خود را به آب بشوید و بدن خود را به آب غسل دهد، و تا شام نجس باشد.
\par 9 «و شخص طاهر، خاکستر گاو را جمع کرده، بیرون از لشکرگاه در جای پاک بگذارد. و آن بجهت جماعت بنی‌اسرائیل برای آب تنزیه نگاه داشته شود. آن قربانی گناه است.
\par 10 و کسی‌که خاکستر گاو را جمع کند، رخت خود را بشوید وتا شام نجس باشد. این برای بنی‌اسرائیل و غریبی که در میان ایشان ساکن باشد، فریضه‌ای ابدی خواهد بود.
\par 11 «هرکه میته هر آدمی را لمس نماید هفت روز نجس باشد.
\par 12 و آن شخص در روز سوم خویشتن را به آن پاک کند، و در روز هفتم طاهرباشد، و اگر خویشتن را در روز سوم پاک نکرده باشد، در روز هفتم طاهر نخواهد بود.
\par 13 و هرکه میته هر آدمی را که مرده باشد لمس نموده، و خود را به آن پاک نکرده باشد، او مسکن خداوندرا ملوث کرده است. و آن شخص از اسرائیل منقطع شود، چونکه آب تنزیه بر او پاشیده نشده است، نجس خواهد بود، و نجاستش بر وی باقی است.
\par 14 «این است قانون برای کسی‌که در خیمه‌ای بمیرد، هرکه داخل آن خیمه شود و هرکه در آن خیمه باشد هفت روز نجس خواهد بود.
\par 15 و هرظرف گشاده که سرپوش برآن بسته نباشد، نجس خواهد بود.
\par 16 و هرکه در بیابان کشته شمشیر یامیته یا استخوان آدمی یا قبری را لمس نماید، هفت روز نجس باشد.
\par 17 و برای شخص نجس ازخاکستر آتش آن قربانی گناه بگیرند و آب روان برآن در ظرفی بریزند.
\par 18 و شخص طاهر زوفاگرفته، درآن آب فرو برد و بر خیمه بر همه اسباب و کسانی که در آن بودند و بر شخصی که استخوان یا مقتول یا میته یا قبر را لمس کرده باشد، بپاشد.
\par 19 و آن شخص طاهر، آب را بر آن شخص نجس در روز سوم و در روز هفتم بپاشد، و در روزهفتم خویشتن را تطهیر کرده، رخت خود رابشوید و به آب غسل کند و در شام طاهرخواهد بود.
\par 20 و اما کسی‌که نجس شده، خویشتن را تطهیر نکند. آن شخص از میان جماعت منقطع شود، چونکه مقدس خداوند راملوث نموده، و آب تنزیه بر او پاشیده نشده است. او نجس است.
\par 21 «و برای ایشان فریضه ابدی خواهد بود. وکسی‌که آب تنزیه را بپاشد، رخت خود را بشویدو کسی‌که آب تنزیه را لمس کند تا شام نجس باشد.و هر چیزی را که شخص نجس لمس نماید نجس خواهد بود، و هر کسی‌که آن را لمس نماید تا شام نجس خواهد بود.»
\par 22 و هر چیزی را که شخص نجس لمس نماید نجس خواهد بود، و هر کسی‌که آن را لمس نماید تا شام نجس خواهد بود.»
 
\chapter{20}

\par 1 و تمامی جماعت بنی‌اسرائیل در ماه اول به بیابان صین رسیدند، و قوم درقادش اقامت کردند، و مریم در آنجا وفات یافته، دفن شد.
\par 2 و برای جماعت آب نبود. پس بر موسی وهارون جمع شدند.
\par 3 و قوم با موسی منازعت کرده، گفتند: «کاش که می‌مردیم وقتی که برادران ما در حضور خداوند مردند!
\par 4 و چرا جماعت خداوند را به این بیابان آوردید تا ما و بهایم ما، دراینجا بمیریم؟
\par 5 و ما را از مصر چرا برآوردید تا مارا به این‌جای بد بیاورید که جای زراعت و انجیرو مو و انار نیست؟ و آب هم نیست که بنوشیم!»
\par 6 و موسی و هارون از حضور جماعت نزد درخیمه اجتماع آمدند، و به روی خود درافتادند، وجلال خداوند بر ایشان ظاهر شد.
\par 7 و خداوندموسی را خطاب کرده، گفت:
\par 8 «عصا را بگیر و توو برادرت هارون جماعت را جمع کرده، در نظرایشان به این صخره بگویید که آب خود را بدهد، پس آب را برای ایشان از صخره بیرون آورده، جماعت و بهایم ایشان را خواهی نوشانید.»
\par 9 پس موسی عصا را از حضور خداوند، چنانکه او را فرموده بود، گرفت.
\par 10 و موسی وهارون، جماعت را پیش صخره جمع کردند، و به ایشان گفت: «ای مفسدان بشنوید، آیا از این صخره آب برای شما بیرون آوریم؟»
\par 11 و موسی دست خود را بلند کرده، صخره را دو مرتبه باعصای خود زد و آب بسیار بیرون آمد که جماعت و بهایم ایشان نوشیدند.
\par 12 و خداوند به موسی و هارون گفت: «چونکه مرا تصدیق ننمودید تا مرا در نظر بنی‌اسرائیل تقدیس نمایید، لهذا شما این جماعت را به زمینی که به ایشان داده‌ام داخل نخواهید ساخت.»
\par 13 این است آب مریبه‌جایی که بنی‌اسرائیل با خداوندمخاصمه کردند، و او خود را در میان ایشان تقدیس نمود.
\par 14 و موسی، رسولان از قادش نزد ملک ادوم فرستاد که «برادر تو اسرائیل چنین می‌گوید: که تمامی مشقتی را که بر ما واقع شده است، تومی دانی.
\par 15 که پدران ما به مصر فرود آمدند ومدت مدیدی در مصر ساکن می‌بودیم، و مصریان با ما و با پدران ما، بد سلوکی نمودند.
\par 16 و چون نزد خداوند فریاد برآوردیم، او آواز ما را شنیده، فرشته‌ای فرستاد و ما را از مصر بیرون آورد. واینک ما در قادش هستیم، شهری که در آخرحدود توست.
\par 17 تمنا اینکه از زمین تو بگذریم، از مزرعه و تاکستان نخواهیم گذشت، و آب ازچاهها نخواهیم نوشید، بلکه از شاهراه‌ها خواهیم رفت، و تا از حدود تو نگذشته باشیم، به طرف راست یا چپ انحراف نخواهیم کرد.»
\par 18 ادوم وی را گفت: «از من نخواهی گذشت والا به مقابله تو با شمشیر بیرون خواهم آمد.»
\par 19 بنی‌اسرائیل در جواب وی گفتند: «از راههای عام خواهیم رفت و هرگاه من و مواشیم از آب توبنوشیم قیمت آن را خواهم داد، فقط بر پایهای خود می‌گذرم و بس.»
\par 20 گفت: «نخواهی گذشت.» و ادوم با خلق بسیار و دست قوی به مقابله ایشان بیرون آمد.
\par 21 بدینطور ادوم راضی نشد که اسرائیل را از حدود خود راه دهد، پس اسرائیل از طرف او رو گردانید.
\par 22 پس تمامی جماعت بنی‌اسرائیل از قادش کوچ کرده، به کوه هور رسیدند.
\par 23 و خداوندموسی و هارون را در کوه هور نزد سرحد زمین ادوم خطاب کرده، گفت:
\par 24 «هارون به قوم خودخواهد پیوست، زیرا چونکه شما نزد آب مریبه از قول من عصیان ورزیدید، از این جهت او به زمینی که به بنی‌اسرائیل دادم، داخل نخواهد شد.
\par 25 پس هارون و پسرش العازار را برداشته، ایشان را به فراز کوه هور بیاور.
\par 26 و لباس هارون رابیرون کرده، بر پسرش العازار بپوشان، و هارون درآنجا وفات یافته، به قوم خود خواهد پیوست.» 
\par 27 پس موسی به طوری که خداوند او را امرفرموده بود، عمل نموده، ایشان درنظر تمامی جماعت به فراز کوه هور برآمدند.
\par 28 و موسی لباس هارون را بیرون کرده، به پسرش العازارپوشانید. و هارون در آنجا بر قله کوه وفات یافت، و موسی و العازار از کوه فرود آمدند.و چون تمامی جماعت دیدند که هارون مرد، جمیع خاندان اسرائیل برای هارون سی روز ماتم گرفتند.
\par 29 و چون تمامی جماعت دیدند که هارون مرد، جمیع خاندان اسرائیل برای هارون سی روز ماتم گرفتند.
 
\chapter{21}

\par 1 و چون کنعانی که ملک عراد و درجنوب ساکن بود، شنید که اسرائیل ازراه اتاریم می‌آید، با اسرائیل جنگ کرد و بعضی از ایشان را به اسیری برد.
\par 2 و اسرائیل برای خداوند نذر کرده گفت: «اگر این قوم را به‌دست من تسلیم نمایی، شهرهای ایشان را بالکل هلاک خواهم ساخت.»
\par 3 پس خداوند دعای اسرائیل را مستجاب فرموده، کنعانیان را تسلیم کرد، و ایشان و شهرهای ایشان را بالکل هلاک ساختند، و آن مکان حرمه نامیده شد.
\par 4 و از کوه هور به راه بحر قلزم کوچ کردند تازمین ادوم را دور زنند، و دل قوم به‌سبب راه، تنگ شد.
\par 5 و قوم بر خدا و موسی شکایت آورده، گفتند: «که ما را از مصر چرا برآوردید تا در بیابان بمیریم؟ زیرا که نان نیست و آب هم نیست! و دل ما از این خوراک سخیف کراهت دارد!»
\par 6 پس خداوند، مارهای آتشی در میان قوم فرستاده، قوم را گزیدند، و گروهی کثیر ازاسرائیل مردند.
\par 7 و قوم نزد موسی آمده، گفتند: «گناه کرده‌ایم زیرا که بر خداوند و بر تو شکایت آورده‌ایم، پس نزد خداوند دعا کن تا مارها را از مادور کند.» و موسی بجهت قوم استغاثه نمود.
\par 8 وخداوند به موسی گفت: «مار آتشینی بساز و آن رابر نیزه‌ای بردار، و هر گزیده شده‌ای که بر آن نظرکند، خواهد زیست.»
\par 9 پس موسی مار برنجینی ساخته، و بر سر نیزه‌ای بلند کرد، و چنین شد که اگر مار کسی را گزیده بود، به مجرد نگاه کردن برآن مار برنجین، زنده می‌شد.
\par 10 و بنی‌اسرائیل کوچ کرده، در اوبوت اردوزدند.
\par 11 و از اوبوت کوچ کرده، در عیی عباریم، در بیابانی که در مقابل موآب به طرف طلوع آفتاب است، اردو زدند.
\par 12 و از آنجا کوچ کرده، به وادی زارد اردو زدند.
\par 13 و از آنجا کوچ کرده، به آن طرف ارنون که در بیابان خارج از حدوداموریان می‌باشد اردو زدند، زیرا که ارنون حدموآب در میان موآب و اموریان است.
\par 14 از این جهت، در کتاب جنگهای خداوند گفته می‌شود: «واهیب در سوفه و وادیهای ارنون،
\par 15 و رودخانه وادیهایی که بسوی مسکن عار متوجه است، و برحدود موآب تکیه می‌زند.»
\par 16 و از آنجا به بئر کوچ کردند. این آن چاهی است که خداوند درباره‌اش به موسی گفت: «قوم را جمع کن تا به ایشان آب دهم.»
\par 17 آنگاه اسرائیل این سرود را سراییدند: «ای چاه بجوش آی، شما برایش سرودبخوانید،
\par 18 «چاهی که سروران حفره زدند، و نجبای قوم آن را کندند. به صولجان حاکم، به عصاهای خود آن را کندند.»
\par 19 و ازمتانه به نحلیئیل و از نحلیئیل به باموت.
\par 20 و ازباموت به دره‌ای که در صحرای موآب نزد قله فسجه که به سوی بیابان متوجه است.
\par 21 و اسرائیل، رسولان نزد سیحون ملک اموریان فرستاده، گفت:
\par 22 «مرا اجازت بده تا اززمین تو بگذرم، به سوی مزرعه یا تاکستان انحراف نخواهیم ورزید، و از آب چاه نخواهیم نوشید، و به شاهراه خواهیم رفت تا از سرحد توبگذریم.»
\par 23 اما سیحون، اسرائیل را از حدودخود راه نداد. و سیحون تمامی قوم خود را جمع نموده، به مقابله اسرائیل به بیابان بیرون آمد. و چون به یاهص رسید با اسرائیل جنگ کرد.
\par 24 واسرائیل او را به دم شمشیر زده، زمینش را ازارنون تا یبوق و تا حد بنی عمون به تصرف آورد، زیرا که حد بنی عمون مستحکم بود.
\par 25 واسرائیل تمامی آن شهرها را گرفت و اسرائیل درتمامی شهرهای اموریان در حشبون و در تمامی دهاتش ساکن شد.
\par 26 زیرا که حشبون، شهرسیحون، ملک اموریان بود، و او با ملک سابق موآب جنگ کرده، تمامی زمینش را تا ارنون ازدستش گرفته بود.
\par 27 بنابراین مثل آورندگان می‌گویند: «به حشبون بیایید تا شهر سیحون بنا کرده، و استوارشود.
\par 28 زیرا آتشی از حشبون برآمد و شعله‌ای از قریه سیحون. و عار، موآب را سوزانید وصاحبان بلندیهای ارنون را.
\par 29 وای بر تو‌ای موآب! ای قوم کموش، هلاک شدید! پسران خود را مثل گریزندگان تسلیم نمود، و دختران خود را به سیحون ملک اموریان به اسیری داد.
\par 30 به ایشان تیر انداختیم. حشبون تا به دیبون هلاک شد. و آن را تا نوفح که نزد میدباست ویران ساختیم.»
\par 31 و اسرائیل در زمین اموریان اقامت کردند.
\par 32 و موسی برای جاسوسی یعزیر فرستاد ودهات آن را گرفته، اموریان را که در آنجا بودند، بیرون کردند.
\par 33 پس برگشته، از راه باشان برآمدند. و عوج ملک باشان با تمامی قوم خود به مقابله ایشان ازبرای جنگ به ادرعی بیرون آمد.
\par 34 و خداوند به موسی گفت: «از او مترس زیرا که او را با تمامی قومش و زمینش به‌دست تو تسلیم نموده‌ام، و به نحوی که با سیحون ملک اموریان که در حشبون ساکن بود، عمل نمودی، با او نیز عمل خواهی نمود.»پس او را با پسرانش و تمامی قومش زدند، به حدی که کسی از برایش باقی نماند وزمینش را به تصرف آوردند.
\par 35 پس او را با پسرانش و تمامی قومش زدند، به حدی که کسی از برایش باقی نماند وزمینش را به تصرف آوردند.
 
\chapter{22}

\par 1 و بنی‌اسرائیل کوچ کرده، در عربات موآب به آنطرف اردن، در مقابل اریحااردو زدند.
\par 2 و چون بالاق بن صفور هر‌چه اسرائیل به اموریان کرده بودند دید،
\par 3 موآب ازقوم بسیار ترسید، زیرا که کثیر بودند. و موآب ازبنی‌اسرائیل مضطرب گردیدند.
\par 4 و موآب به مشایخ مدیان گفتند: «الان این گروه هر‌چه به اطراف ما هست خواهند لیسید، به نوعی که گاوسبزه صحرا را می‌لیسد.» و در آن زمان بالاق بن صفور، ملک موآب بود.
\par 5 پس رسولان به فتور که برکنار وادی است، نزد بلعام بن بعور، به زمین پسران قوم او فرستاد تااو را طلبیده، بگویند: «اینک قومی از مصر بیرون آمده‌اند و هان روی زمین را مستور می‌سازند، ودر مقابل من مقیم می‌باشند.
\par 6 پس الان بیا و این قوم را برای من لعنت کن، زیرا که از من قوی ترند، شاید توانایی یابم تا بر ایشان غالب آییم، و ایشان را از زمین خود بیرون کنم، زیرا می‌دانم هر‌که راتو برکت دهی مبارک است و هر‌که را لعنت نمایی، ملعون است.»
\par 7 پس مشایخ موآب و مشایخ مدیان، مزد فالگیری را به‌دست گرفته، روانه شدند، و نزدبلعام رسیده، سخنان بالاق را به وی گفتند.
\par 8 او به ایشان گفت: «این شب را در اینجا بمانید، تاچنانکه خداوند به من گوید، به شما باز گویم.» وسروران موآب نزد بلعام ماندند.
\par 9 و خدا نزد بلعام آمده، گفت: «این کسانی که نزد تو هستند، کیستند؟»
\par 10 بلعام به خدا گفت: «بالاق بن صفورملک موآب نزد من فرستاده است،
\par 11 که اینک این قومی که از مصر بیرون آمده‌اند روی زمین راپوشانیده‌اند. الان آمده، ایشان را برای من لعنت کن شاید که توانایی یابم تا با ایشان جنگ نموده، ایشان را دور سازم.»
\par 12 خدا به بلعام گفت: «باایشان مرو و قوم را لعنت مکن زیرا مبارک هستند.»
\par 13 پس بلعام بامدادان برخاسته، به‌سروران بالاق گفت: «به زمین خود بروید، زیراخداوند مرا اجازت نمی دهد که با شما بیایم.»
\par 14 و سروران موآب برخاسته، نزد بالاق برگشته، گفتند که «بلعام از آمدن با ما انکار نمود.»
\par 15 و بالاق بار دیگر سروران زیاده و بزرگتر ازآنان فرستاد.
\par 16 و ایشان نزد بلعام آمده، و وی راگفتند: «بالاق بن صفور چنین می‌گوید: تمنا اینکه از آمدن نزد من انکار نکنی.
\par 17 زیرا که البته تو رابسیار تکریم خواهم نمود، و هر‌آنچه به من بگویی بجا خواهم آورد، پس بیا و این قوم را برای من لعنت کن.»
\par 18 بلعام در جواب نوکران بالاق گفت: اگر بالاق خانه خود را پر از نقره و طلا به من بخشد، نمی توانم از فرمان یهوه خدای خودتجاوز نموده، کم یا زیاد به عمل آورم.
\par 19 پس الان شما نیز امشب در اینجا بمانید تا بدانم که خداوند به من دیگر‌چه خواهد گفت.»
\par 20 و خدادر شب نزد بلعام آمده، وی را گفت: «اگر این مردمان برای طلبیدن تو بیایند برخاسته، همراه ایشان برو، اما کلامی را که من به تو گویم به همان عمل نما.»
\par 21 پس بلعام بامدادان برخاسته، الاغ خود را بیاراست و همراه سروران موآب روانه شد.
\par 22 و غضب خدا به‌سبب رفتن او افروخته شده، فرشته خداوند در راه به مقاومت وی ایستاد، و او بر الاغ خود سوار بود، و دو نوکرش همراهش بودند.
\par 23 و الاغ، فرشته خداوند را باشمشیر برهنه به‌دستش، بر سر راه ایستاده دید. پس الاغ از راه به یک سو شده، به مزرعه‌ای رفت و بلعام الاغ را زد تا او را به راه برگرداند.
\par 24 پس فرشته خداوند در جای گود در میان تاکستان بایستاد، و به هر دو طرفش دیوار بود.
\par 25 و الاغ فرشته خداوند را دیده، خود را به دیوار چسبانید، و پای بلعام را به دیوار فشرد. پس او را بار دیگرزد.
\par 26 و فرشته خداوند پیش رفته، در مکانی تنگ بایستاد، که جایی بجهت برگشتن به طرف راست یا چپ نبود.
\par 27 و چون الاغ، فرشته خداوند رادید، در زیر بلعام خوابید. و خشم بلعام افروخته شده، الاغ را به عصای خود زد.
\par 28 آنگاه خداونددهان الاغ را باز کرد که بلعام را گفت: به تو چه کرده‌ام که مرا این سه مرتبه زدی.
\par 29 بلعام به الاغ گفت: «از این جهت که تو مرا استهزا نمودی! کاش که شمشیر در دست من می‌بود که الان تو رامی کشتم.»
\par 30 الاغ به بلعام گفت: «آیا من الاغ تونیستم که از وقتی که مال تو شده‌ام تا امروز بر من سوار شده‌ای، آیا هرگز عادت می‌داشتم که به اینطور با تو رفتار نمایم؟» او گفت: «نی»
\par 31 و خداوند چشمان بلعام را باز کرد تا فرشته خداوند را دید که با شمشیر برهنه در دستش، به‌سر راه ایستاده است پس خم شده، به روی درافتاد.
\par 32 و فرشته خداوند وی را گفت: «الاغ خود را این سه مرتبه چرا زدی؟ اینک من به مقاومت تو بیرون آمدم، زیرا که این سفر تو در نظرمن از روی تمرد است.
\par 33 و الاغ مرا دیده، این سه مرتبه از من کناره جست، و اگر از من کناره نمی جست یقین الان تو را می‌کشتم و او را زنده نگاه می‌داشتم.»
\par 34 بلعام به فرشته خداوند گفت: «گناه کردم زیرا ندانستم که تو به مقابل من در راه ایستاده‌ای. پس الان اگر در نظر تو ناپسند است برمی گردم.»
\par 35 فرشته خداوند به بلعام گفت: «همراه این اشخاص برو لیکن سخنی را که من به تو گویم، همان را فقط بگو». پس بلعام همراه سروران بالاق رفت.
\par 36 و چون بالاق شنید که بلعام آمده است، به استقبال وی تا شهر موآب که برحد ارنون و براقصای حدود وی بود، بیرون آمد.
\par 37 و بالاق به بلعام گفت: «آیا برای طلبیدن تو نزد تو نفرستادم، پس چرا نزد من نیامدی، آیا حقیقت قادر نیستم که تو را به عزت رسانم؟»
\par 38 بلعام به بالاق گفت: «اینک نزد تو آمده‌ام، آیا الان هیچ قدرتی دارم که چیزی بگویم؟ آنچه خدا به دهانم می‌گذارد همان را خواهم گفت.»
\par 39 پس بلعام همرا بالاق رفته، به قریت حصوت رسیدند.
\par 40 و بالاق گاوان وگوسفندان ذبح کرده، نزد بلعام و سرورانی که باوی بودند، فرستاد.و بامدادان بالاق بلعام رابرداشته، او را به بلندیهای بعل آورد، تا از آنجااقصای قوم خود را ملاحظه کند.
\par 41 و بامدادان بالاق بلعام رابرداشته، او را به بلندیهای بعل آورد، تا از آنجااقصای قوم خود را ملاحظه کند.
 
\chapter{23}

\par 1 و بلعام به بالاق گفت: «در اینجا برای من هفت مذبح بساز، و هفت گاو و هفت قوچ در اینجا برایم حاضر کن.»
\par 2 و بالاق به نحوی که بلعام گفته بود به عمل آورد، و بالاق و بلعام، گاوی و قوچی بر هر مذبح گذرانیدند.
\par 3 و بلعام به بالاق گفت: «نزد قربانی سوختنی خود بایست، تا من بروم؛ شاید خداوند برای ملاقات من بیاید، وهر‌چه او به من نشان دهد آن را به تو باز خواهم گفت.» پس به تلی برآمد. 
\par 4 و خدا بلعام را ملاقات کرد، و او وی را گفت: «هفت مذبح برپا داشتم و گاوی و قوچی بر هرمذبح قربانی کردم.»
\par 5 خداوند سخنی به دهان بلعام گذاشته، گفت: «نزد بالاق برگشته چنین بگو.»
\par 6 پس نزد او برگشت، و اینک او با جمیع سروران موآب نزد قربانی سوختنی خود ایستاده بود.
\par 7 و مثل خود را آورده، گفت: «بالاق ملک موآب مرا از ارام از کوههای مشرق آورد، که بیایعقوب را برای من لعنت کن، و بیا اسرائیل رانفرین نما.
\par 8 چگونه لعنت کنم آن را که خدا لعنت نکرده است؟ و چگونه نفرین نمایم آن را که خداوند نفرین ننموده است؟
\par 9 زیرا از سرصخره‌ها او را می‌بینم. و از کوهها او را مشاهده می‌نمایم. اینک قومی است که به تنهایی ساکن می‌شود، و در میان امتها حساب نخواهد شد.
\par 10 کیست که غبار یعقوب را تواند شمرد یا ربع اسرائیل را حساب نماید؟ کاش که من به وفات عادلان بمیرم و عاقبت من مثل عاقبت ایشان باشد.»
\par 11 پس بالاق به بلعام گفت: «به من چه کردی؟ تو را آوردم تا دشمنانم را لعنت کنی، و هان برکت تمام دادی!»
\par 12 او در جواب گفت: «آیا نمی بایدباحذر باشم تا آنچه را که خداوند به دهانم گذاردبگویم؟»
\par 13 بالاق وی را گفت: «بیا الان همراه من به‌جای دیگر که از آنجا ایشان را توانی دید، فقطاقصای ایشان را خواهی دید، و جمیع ایشان رانخواهی دید و از آنجا ایشان را برای من لعنت کن.»
\par 14 پس او را به صحرای صوفیم، نزد قله فسجه برد و هفت مذبح بنا نموده، گاوی و قوچی بر هر مذبح قربانی کرد.
\par 15 و او به بالاق گفت: «نزدقربانی سوختنی خود، اینجا بایست تا من در آنجا(خداوند را) ملاقات نمایم.»
\par 16 و خداوند بلعام را ملاقات نموده، و سخنی در زبانش گذاشته، گفت: «نزد بالاق برگشته، چنین بگو.»
\par 17 پس نزدوی آمد، و اینک نزد قربانی سوختنی خود باسروران موآب ایستاده بود، و بالاق از او پرسیدکه «خداوند چه گفت؟»
\par 18 آنگاه مثل خود راآورده، گفت: «ای بالاق برخیز و بشنو. و‌ای پسرصفور مرا گوش بگیر.
\par 19 خدا انسان نیست که دروغ بگوید. و از بنی آدم نیست که به اراده خودتغییر بدهد. آیا او سخنی گفته باشد و نکند؟ یاچیزی فرموده باشد و استوار ننماید؟
\par 20 اینک مامور شده‌ام که برکت بدهم. و او برکت داده است و آن را رد نمی توانم نمود.
\par 21 او گناهی دریعقوب ندیده، و خطایی در اسرائیل مشاهده ننموده است. یهوه خدای او با وی است. و نعره پادشاه در میان ایشان است.
\par 22 خدا ایشان را ازمصر بیرون آورد. او را شاخها مثل گاو وحشی است.
\par 23 به درستی که بر یعقوب افسون نیست وبر اسرائیل فالگیری نی. درباره یعقوب و درباره اسرائیل در وقتش گفته خواهد شد، که خدا چه کرده است.
\par 24 اینک قوم مثل شیر ماده خواهندبرخاست. و مثل شیر نر خویشتن را خواهندبرانگیخت، و تا شکار را نخورد، و خون کشتگان را ننوشد، نخواهد خوابید.»
\par 25 بالاق به بلعام گفت: «نه ایشان را لعنت کن ونه برکت ده.»
\par 26 بلعام در جواب بالاق گفت: «آیاتو را نگفتم که هر‌آنچه خداوند به من گوید، آن را باید بکنم؟»
\par 27 بالاق به بلعام گفت: «بیا تا تو را به‌جای دیگر ببرم، شاید در نظر خدا پسند آید که ایشان را برای من از آنجا لعنت نمایی.»
\par 28 پس بالاق بلعام را بر قله فغور که مشرف بر بیابان است، برد.
\par 29 بلعام به بالاق گفت: «در اینجا برای من هفت مذبح بساز و هفت گاو و هفت قوچ از برایم دراینجا حاضر کن.»و بالاق به طوری که بلعام گفته بود، عمل نموده، گاوی و قوچی بر هر مذبح قربانی کرد.
\par 30 و بالاق به طوری که بلعام گفته بود، عمل نموده، گاوی و قوچی بر هر مذبح قربانی کرد.
 
\chapter{24}

\par 1 و چون بلعام دید که اسرائیل را برکت دادن به نظر خداوند پسند می‌آید، مثل دفعه های پیش برای طلبیدن افسون نرفت، بلکه به سوی صحرا توجه نمود.
\par 2 و بلعام چشمان خودرا بلند کرده، اسرائیل را دید که موافق اسباط خودساکن می‌بودند. و روح خدا بر او نازل شد.
\par 3 پس مثل خود را آورده، گفت: «وحی بلعام بن بعور. وحی آن مردی که چشمانش باز شد.
\par 4 وحی آن کسی‌که سخنان خدا را شنید. و رویای قادرمطلق را مشاهده نمود. آنکه بیفتاد و چشمان اوگشاده گردید.
\par 5 چه زیباست خیمه های تو‌ای یعقوب! و مسکنهای تو‌ای اسرائیل!
\par 6 مثل وادیهای کشیده شده، مثل باغها بر کنار رودخانه، مثل درختان عود که خداوند غرس نموده باشد، ومثل سروهای آزاد نزد جویهای آب.
\par 7 آب ازدلوهایش ریخته خواهد شد. و بذر او در آبهای بسیار خواهد بود. و پادشاه او از اجاج بلندتر، ومملکت او برافراشته خواهد شد.
\par 8 خدا او را ازمصر بیرون آورد. او را شاخها مثل گاو وحشی است. امتهای دشمنان خود را خواهد بلعید و استخوانهای ایشان را خواهد شکست و ایشان رابه تیرهای خود خواهد دوخت.
\par 9 مثل شیر نرخود را جمع کرده، خوابید. و مثل شیر ماده کیست که او را برانگیزاند؟ مبارک باد هر‌که تو رابرکت دهد. و ملعون باد هر‌که تو را لعنت نماید!»
\par 10 پس خشم بالاق بر بلعام افروخته شده، هردو دست خود را بر هم زد و بالاق به بلعام گفت: «تو را خواندم تا دشمنانم را لعنت کنی و اینک این سه مرتبه ایشان را برکت تمام دادی.
\par 11 پس الان به‌جای خود فرار کن! گفتم که تو را احترام تمام نمایم. همانا خداوند تو را از احترام بازداشته است.»
\par 12 بلعام به بالاق گفت آیا به رسولانی که نزد من فرستاده بودی نیز نگفتم:
\par 13 که اگر بالاق خانه خود را پر از نقره و طلا به من بدهد، نمی توانم از فرمان خداوند تجاوز نموده، از دل خود نیک یا بد بکنم بلکه آنچه خداوند به من گوید آن را خواهم گفت؟
\par 14 و الان اینک نزدقوم خود می‌روم. بیا تا تو را اعلام نمایم که این قوم با قوم تو در ایام آخر چه خواهند کرد.
\par 15 پس مثل خود را آورده، گفت: «وحی بلعام بن بعور. وحی آن مردی که چشمانش بازشد.
\par 16 وحی آن کسی‌که سخنان خدا را شنید. ومعرفت حضرت اعلی را دانست. و رویای قادرمطلق را مشاهده نمود. آنکه بیفتاد و چشمان اوگشوده گردید.
\par 17 او را خواهم دید لیکن نه الان. اورا مشاهده خواهم نمود اما نزدیک نی. ستاره‌ای از یعقوب طلوع خواهد کرد و عصایی از اسرائیل خواهد برخاست و اطراف موآب را خواهدشکست. و جمیع ابنای فتنه را هلاک خواهدساخت.
\par 18 و ادوم ملک او خواهد شد ودشمنانش (اهل ) سعیر، مملوک او خواهندگردید. و اسرائیل به شجاعت عمل خواهد نمود.
\par 19 و کسی‌که از یعقوب ظاهر می‌شود، سلطنت خواهد نمود. و بقیه اهل شهر را هلاک خواهدساخت.»
\par 20 و به عمالقه نظر انداخته، مثل خود راآورده، گفت: «عمالیق اول امتها بود، اما آخر اومنتهی به هلاکت است.»
\par 21 و بر قینیان نظر انداخته، مثل خود را آورد وگفت: «مسکن تو مستحکم و آشیانه تو بر صخره نهاده (شده است ).
\par 22 لیکن قاین تباه خواهد شد، تا وقتی که آشور تو را به اسیری ببرد.»
\par 23 پس مثل خود را آورده، گفت: «وای! چون خدا این را می‌کند، کیست که زنده بماند؟
\par 24 وکشتیها از جانب کتیم آمده، آشور را ذلیل خواهند ساخت، و عابر را ذلیل خواهند گردانید، و او نیز به هلاکت خواهد رسید.»و بلعام برخاسته، روانه شده، به‌جای خودرفت و بالاق نیز راه خود را پیش گرفت.
\par 25 و بلعام برخاسته، روانه شده، به‌جای خودرفت و بالاق نیز راه خود را پیش گرفت.
 
\chapter{25}

\par 1 و اسرائیل در شطیم اقامت نمودند، وقوم با دختران موآب زنا کردن گرفتند.
\par 2 زیرا که ایشان قوم را به قربانی های خدایان خوددعوت نمودند، پس قوم می‌خوردند و به خدایان ایشان سجده می‌نمودند.
\par 3 و اسرائیل به بعل فغورملحق شدند، و غضب خداوند بر اسرائیل افروخته شد.
\par 4 و خداوند به موسی گفت که: «تمامی روسای قوم را گرفته، ایشان را برای خداوند پیش آفتاب به دار بکش، تا شدت خشم خداوند از اسرائیل برگردد.»
\par 5 و موسی به داوران اسرائیل گفت که «هر یکی از شما کسان خود راکه به بعل فغور ملحق شدند، بکشید.»
\par 6 و اینک مردی از بنی‌اسرائیل آمده، زن مدیانی‌ای را در نظر موسی و در نظر تمامی جماعت بنی‌اسرائیل نزد برادران خود آورد، وایشان به دروازه خیمه اجتماع گریه می‌کردند.
\par 7 وچون فینحاس بن العازار بن هارون کاهن، این رادید، از میان جماعت برخاسته، نیزه‌ای به‌دست خود گرفت،
\par 8 و از عقب آن مرد اسرائیلی به قبه داخل شده، هر دوی ایشان یعنی آن مرداسرائیلی و زن را به شکمش فرو برد، و وبا ازبنی‌اسرائیل رفع شد.
\par 9 و آنانی که از وبا مردند، بیست و چهار هزار نفر بودند.
\par 10 و خداوند موسی را خطاب کرده، گفت:
\par 11 «فینحاس بن العازار بن هارون کاهن، غضب مرااز بنی‌اسرائیل برگردانید، چونکه باغیرت من درمیان ایشان غیور شد، تا بنی‌اسرائیل را در غیرت خود هلاک نسازم.
\par 12 لهذا بگو اینک عهدسلامتی خود را به او می‌بخشم.
\par 13 و برای او وبرای ذریتش بعد از او این عهد کهانت جاودانی خواهد بود، زیرا که برای خدای خود غیور شد، وبجهت بنی‌اسرائیل کفاره نمود.»
\par 14 و اسم آن مرد اسرائیلی مقتول که با زن مدیانی کشته گردید، زمری ابن سالو رئیس خاندان آبای سبط شمعون بود.
\par 15 و اسم زن مدیانی که کشته شد، کزبی دختر صور بود و اورئیس قوم خاندان آبا در مدیان بود.
\par 16 و خداوند موسی را خطاب کرده، گفت:
\par 17 «مدیانیان را ذلیل ساخته، مغلوب سازید.زیرا که ایشان شما را به مکاید خود ذلیل ساختند، چونکه شما را در واقعه فغور و در امرخواهر خود کزبی، دختر رئیس مدیان، که در روزوبا در واقعه فغور کشته شد، فریب دادند.»
\par 18 زیرا که ایشان شما را به مکاید خود ذلیل ساختند، چونکه شما را در واقعه فغور و در امرخواهر خود کزبی، دختر رئیس مدیان، که در روزوبا در واقعه فغور کشته شد، فریب دادند.»
 
\chapter{26}

\par 1 و بعد از وبا، خداوند موسی و العازاربن هارون کاهن را خطاب کرده، گفت:
\par 2 «شماره تمامی بنی‌اسرائیل را برحسب خاندان آبای ایشان، از بیست ساله و بالاتر، یعنی جمیع کسانی را که از اسرائیل به جنگ بیرون می‌روند، بگیرید.»
\par 3 پس موسی و العازار کاهن ایشان را درعربات موآب، نزد اردن در مقابل اریحا خطاب کرده، گفتند:
\par 4 «قوم را از بیست ساله و بالاتربشمارید، چنانکه خداوند موسی و بنی‌اسرائیل را که از زمین مصر بیرون آمدند، امر فرموده بود.»
\par 5 روبین نخست زاده اسرائیل: بنی روبین: ازحنوک، قبیله حنوکیان. و از فلو، قبیله فلوئیان.
\par 6 واز حصرون، قبیله حصرونیان. و از کرمی، قبیله کرمیان.
\par 7 اینانند قبایل روبینیان و شمرده شدگان ایشان، چهل و سه هزار و هفتصد و سی نفر بودند.
\par 8 و بنی فلو: الیاب.
\par 9 و بنی الیاب: نموئیل و داتان و ابیرام. اینانند داتان و ابیرام که خوانده‌شدگان جماعت بوده، با موسی و هارون در جمعیت قورح مخاصمه کردند، چون با خداوند مخاصمه نمودند،
\par 10 و زمین دهان خود را گشوده، ایشان رابا قورح فرو برد، هنگامی که آن گروه مردند وآتش، آن دویست و پنجاه نفر را سوزانیده، عبرت گشتند.
\par 11 لکن پسران قورح نمردند.
\par 12 و بنی شمعون برحسب قبایل ایشان: ازنموئیل، قبیله نموئیلیان و از یامین، قبیله یامینیان و از یاکین، قبیله یاکینیان.
\par 13 و از زارح قبیله زارحیان و از شاول قبیله شاولیان.
\par 14 اینانندقبایل شمعونیان: بیست و دو هزار و دویست نفر.
\par 15 و بنی جاد برحسب قبایل ایشان: از صفون قبیله صفونیان و از حجی قبیله حجیان و از شونی قبیله شونیان.
\par 16 و از ازنی قبیله ازنیان و از عیری، قبیله عیریان.
\par 17 و از ارود قبیله ارودیان و ازارئیلی قبیله ارئیلیان.
\par 18 اینانند قبایل بنی جادبرحسب شماره ایشان، چهل هزار و پانصد نفر.
\par 19 و بنی یهودا عیر و اونان. و عیر و اونان درزمین کنعان مردند.
\par 20 و بنی یهودا برحسب قبایل ایشان اینانند: از شیله قبیله شیلئیان و از فارص قبیله فارصیان و از زارح قبیله زارحیان. 
\par 21 وبنی فارص اینانند: از حصرون قبیله حصرونیان واز حامول قبیله حامولیان.
\par 22 اینانند قبایل یهودابرحسب شمرده شدگان ایشان، هفتاد و شش هزارو پانصد نفر.
\par 23 و بنی یساکار برحسب قبایل ایشان: از تولع قبیله تولعیان و از فوه قبیله فوئیان.
\par 24 و از یاشوب قبیله یاشوبیان و از شمرون قبیله شمرونیان.
\par 25 اینانند قبایل یساکار برحسب شمرده شدگان ایشان، شصت و چهار هزار و سیصد نفر.
\par 26 و بنی زبولون برحسب قبایل ایشان: از ساردقبیله ساردیان و از ایلون قبیله ایلونیان و ازیحلیئیل قبیله یحلیئیلیان.
\par 27 اینانند قبایل زبولونیان برحسب شمرده شدگان ایشان، شصت هزار و پانصد نفر.
\par 28 و بنی یوسف برحسب قبایل ایشان: منسی و افرایم.
\par 29 و بنی منسی: از ماکیر قبیله ماکیریان وماکیر جلعاد را آورد و از جلعاد قبیله جلعادیان.
\par 30 اینانند بنی جلعاد: از ایعزر قبیله ایعزریان، ازحالق قبیله حالقیان.
\par 31 از اسریئیل قبیله اسریئیلیان، از شکیم قبیله شکیمیان.
\par 32 ازشمیداع قبیله شمیداعیان و از حافر قبیله حافریان.
\par 33 و صلحفاد بن حافر را پسری نبودلیکن دختران داشت و نامهای دختران صلحفاد محله و نوعه و حجله و ملکه و ترصه.
\par 34 اینانندقبایل منسی و شمرده شدگان ایشان، پنجاه ودوهزار و هفتصد نفر بودند.
\par 35 و اینانند بنی افرایم برحسب قبایل ایشان: از شوتالح قبیله شوتالحیان و از باکر قبیله باکریان و از تاحن قبیله تاحنیان.
\par 36 و بنی شوتالح اینانند: از عیران قبیله عیرانیان.
\par 37 اینانند قبایل بنی افرایم برحسب شمرده شدگان ایشان، سی و دو هزار وپانصد نفر. و بنی یوسف برحسب قبایل ایشان اینانند.
\par 38 و بنی بنیامین برحسب قبایل ایشان: از بالع قبیله بالعیان از اشبیل قبیله اشبیلیان و از احیرام قبیله احیرامیان.
\par 39 از شفوفام قبیله شفوفامیان ازحوفام قبیله حوفامیان.
\par 40 و بنی بالع: ارد و نعمان. از ارد قبیله اردیان و از نعمان قبیله نعمانیان.
\par 41 اینانند بنی بنیامین برحسب قبایل ایشان وشمرده شدگان ایشان، چهل و پنج هزار و ششصدنفر بودند.
\par 42 اینانند بنی دان برحسب قبایل ایشان: ازشوحام قبیله شوحامیان. اینانند قبایل دان برحسب قبایل ایشان.
\par 43 جمیع قبایل شوحامیان برحسب شمرده شدگان ایشان، شصت وچهارهزار و چهارصد نفر بودند.
\par 44 اینانند بنی اشیر برحسب قبایل ایشان: ازیمنه قبیله یمنئیان، از یشوی قبیله یشویان، ازبریعه قبیله بریعئیان،
\par 45 از بنی بریعه، از حابر قبیله حابریان، از ملکیئیل قبیله ملکیئیلیان.
\par 46 و نام دختر اشیر، ساره بود.
\par 47 اینانند قبایل بنی اشیربرحسب شمرده شدگان ایشان، پنجاه و سه هزار وچهارصد نفر.
\par 48 اینانند بنی نفتالی برحسب قبایل ایشان: ازیاحصئیل، قبیله یاحصئیلیان، از جونی قبیله جونیان.
\par 49 از یصر قبیله یصریان از شلیم قبیله شلیمیان.
\par 50 اینانند قبایل نفتالی برحسب قبایل ایشان و شمرده شدگان ایشان، چهل و پنج هزار وچهارصد نفر بودند.
\par 51 اینانند شمرده شدگان بنی‌اسرائیل: ششصدو یکهزار و هفتصد و سی نفر.
\par 52 و خداوند موسی را خطاب کرده، گفت:
\par 53 «برای اینان برحسب شماره نامها، زمین برای ملکیت تقسیم بشود.
\par 54 برای کثیر، نصیب او رازیاده کن و برای قلیل، نصیب او را کم نما، به هرکس برحسب شمرده شدگان او نصیبش داده شود.
\par 55 لیکن زمین به قرعه تقسیم شود، و برحسب نامهای اسباط آبای خود در آن تصرف نمایند.
\par 56 موافق قرعه، ملک ایشان در میان کثیر و قلیل تقسیم شود.»
\par 57 و اینانند شمرده شدگان لاوی برحسب قبایل ایشان: از جرشون قبیله جرشونیان، ازقهات قبیله قهاتیان، از مراری قبیله مراریان.
\par 58 اینانند قبایل لاویان: قبیله لبنیان و قبیله حبرونیان و قبیله محلیان و قبیله موشیان و قبیله قورحیان. اما قهات، عمرام را آورد.
\par 59 و نام زن عمرام، یوکابد بود، دختر لاوی که برای لاوی درمصر زاییده شد و او برای عمرام، هارون و موسی و خواهر ایشان مریم را زایید.
\par 60 و برای هارون ناداب و ابیهو و العازار و ایتامار زاییده شدند.
\par 61 ناداب و ابیهو چون آتش غریبی به حضورخداوند گذرانیده بودند، مردند.
\par 62 وشمرده شدگان ایشان یعنی همه ذکوران از یک ماهه و بالاتر، بیست و سه هزار نفر بودند زیرا که ایشان در میان بنی‌اسرائیل شمرده نشدند، چونکه نصیبی در میان بنی‌اسرائیل به ایشان داده نشد.
\par 63 اینانند آنانی که موسی و العازار کاهن شمردند، وقتی که بنی‌اسرائیل را در عربات موآب نزد اردن در مقابل اریحا شمردند.
\par 64 و درمیان ایشان کسی نبود از آنانی که موسی و هارون کاهن، شمرده بودند وقتی که بنی‌اسرائیل را دربیابان سینا شمردند.زیرا خداوند درباره ایشان گفته بود که البته در بیابان خواهند مرد، پس از آنهایک مرد سوای کالیب بن یفنه و یوشع بن نون باقی نماند.
\par 65 زیرا خداوند درباره ایشان گفته بود که البته در بیابان خواهند مرد، پس از آنهایک مرد سوای کالیب بن یفنه و یوشع بن نون باقی نماند.
 
\chapter{27}

\par 1 و دختران صلفحاد بن حافر بن جلعادبن ماکیر بن منسی، که از قبایل منسی ابن یوسف بود نزدیک آمدند، و اینهاست نامهای دخترانش: محله و نوعه و حجله و ملکه و ترصه.
\par 2 و به حضور موسی و العازار کاهن، و به حضورسروران و تمامی جماعت نزد در خیمه اجتماع ایستاده، گفتند:
\par 3 «پدر ما در بیابان مرد و او از آن گروه نبود که در جمعیت قورح به ضد خداوندهمداستان شدند، بلکه در گناه خود مرد و پسری نداشت.
\par 4 پس چرا نام پدر ما از این جهت که پسری ندارد از میان قبیله‌اش محو شود، لهذا ما رادر میان برادران پدر ما نصیبی بده.»
\par 5 پس موسی دعوی ایشان را به حضورخداوند آورد.
\par 6 و خداوند موسی را خطاب کرده، گفت:
\par 7 «دختران صلفحاد راست می‌گویند، البته در میان برادران پدر ایشان ملک موروثی به ایشان بده، و نصیب پدر ایشان را به ایشان انتقال نما.
\par 8 و بنی‌اسرائیل را خطاب کرده، بگو: اگر کسی بمیرد و پسری نداشته باشد، ملک او را به دخترش انتقال نمایید.
\par 9 و اگر او رادختری نباشد، ملک او را به برادرانش بدهید.
\par 10 واگر او را برادری نباشد، ملک او را به برادران پدرش بدهید.
\par 11 و اگر پدر او را برادری نباشد، ملک او را به هر کس از قبیله‌اش که خویش نزدیکتر او باشد بدهید، تا مالک آن بشود، پس این برای بنی‌اسرائیل فریضه شرعی باشد، چنانکه خداوند به موسی‌امر فرموده بود.»
\par 12 و خداوند به موسی گفت: «به این کوه عباریم برآی و زمینی را که به بنی‌اسرائیل داده‌ام، ببین.
\par 13 و چون آن را دیدی تو نیز به قوم خودملحق خواهی شد، چنانکه برادرت هارون ملحق شد.
\par 14 زیرا که در بیابان صین وقتی که جماعت مخامصه نمودند شما از قول من عصیان ورزیدید، و مرا نزد آب در نظر ایشان تقدیس ننمودید.» این است آب مریبه قادش، در بیابان صین.
\par 15 و موسی به خداوند عرض کرده، گفت:
\par 16 «ملتمس اینکه یهوه خدای ارواح تمامی بشر، کسی را بر این جماعت بگمارد
\par 17 که پیش روی ایشان بیرون رود، و پیش روی ایشان داخل شود، و ایشان را بیرون برد و ایشان را درآورد، تاجماعت خداوند مثل گوسفندان بی‌شبان نباشند.»
\par 18 و خداوند به موسی گفت: «یوشع بن نون را که مردی صاحب روح است گرفته، دست خود را بر او بگذار.
\par 19 و او را به حضورالعازار کاهن و به حضور تمامی جماعت برپاداشته، در نظر ایشان به وی وصیت نما.
\par 20 و ازعزت خود بر او بگذار تا تمامی جماعت بنی‌اسرائیل او را اطاعت نمایند.
\par 21 و او به حضور العازار کاهن بایستد تا از برای او به حکم اوریم به حضور خداوند سوال نماید، و به فرمان وی، او و تمامی بنی‌اسرائیل با وی و تمامی جماعت، بیرون روند، و به فرمان وی داخل شوند.»
\par 22 پس موسی به نوعی که خداوند او راامر فرموده بود عمل نموده، یوشع را گرفت و اورا به حضور العازار کاهن و به حضور تمامی جماعت برپا داشت.و دستهای خود را بر اوگذاشته، او را به طوری که خداوند به واسطه موسی گفته بود، وصیت نمود.
\par 23 و دستهای خود را بر اوگذاشته، او را به طوری که خداوند به واسطه موسی گفته بود، وصیت نمود.
 
\chapter{28}

\par 1 و خداوند موسی را خطاب کرده، گفت:
\par 2 «بنی‌اسرائیل را امر فرموده، به ایشان بگو: مراقب باشید تا هدیه طعام مرا از قربانی های آتشین عطر خوشبوی من در موسمش نزد من بگذرانید.
\par 3 و ایشان را بگو قربانی آتشین را که نزدخداوند بگذرانید، این است: دو بره نرینه یک ساله بی‌عیب، هر روز بجهت قربانی سوختنی دائمی.
\par 4 یک بره را در صبح قربانی کن و بره دیگررا در عصر قربانی کن.
\par 5 و یک عشر ایفه آرد نرم مخلوط شده با یک ربع هین روغن زلال برای هدیه آردی.
\par 6 این است قربانی سوختنی دائمی که در کوه سینا بجهت عطر خوشبو و قربانی آتشین خداوند معین شد.
\par 7 و هدیه ریختنی آن یک ربع هین بجهت هر بره‌ای باشد، این هدیه ریختنی مسکرات را برای خداوند در قدس بریز.
\par 8 و بره دیگر را در عصر قربانی کن، مثل هدیه آردی صبح و مثل هدیه ریختنی آن بگذران تاقربانی آتشین و عطر خوشبو برای خداوند باشد.
\par 9 «و در روز سبت دو بره یک ساله بی‌عیب، ودو عشر ایفه آرد نرم سرشته شده با روغن، بجهت هدیه آردی با هدیه ریختنی آن.
\par 10 این است قربانی سوختنی هر روز سبت سوای قربانی سوختنی دائمی با هدیه ریختنی آن.
\par 11 «و در اول ماههای خود قربانی سوختنی برای خداوند بگذرانید، دو گاو جوان و یک قوچ و هفت بره نرینه یک ساله بی‌عیب.
\par 12 و سه عشرایفه آرد نرم سرشته شده با روغن بجهت هدیه آردی برای هر گاو، و دو عشر آرد نرم سرشته شده با روغن، بجهت هدیه آردی برای هر قوچ.
\par 13 و یک عشر آرد نرم سرشته شده با روغن، بجهت هدیه آردی برای هر بره، تا قربانی سوختنی، عطر خوشبو و هدیه آتشین برای خداوند باشد.
\par 14 و هدایای ریختنی آنها نصف هین شراب برای هر گاو، و ثلث هین برای هر قوچ، و ربع هین برای هر بره باشد. این است قربانی سوختنی هر ماه از ماههای سال.
\par 15 و یک بز نربجهت قربانی گناه سوای قربانی سوختنی دائمی، با هدیه ریختنی آن برای خداوند قربانی بشود.
\par 16 «و در روز چهاردهم ماه اول، فصح خداوند است.
\par 17 و در روز پانزدهم این ماه، عید است که هفت روز نان فطیر خورده شود.
\par 18 در روز اول، محفل مقدس است که هیچ کار خدمت در آن نکنید.
\par 19 و بجهت هدیه آتشین و قربانی سوختنی برای خداوند، دو گاو جوان و یک قوچ و هفت بره نرینه یک ساله قربانی کنید، اینها برای شما بی‌عیب باشد.
\par 20 و بجهت هدیه آردی آنهاسه عشر آرد نرم سرشته شده با روغن برای هرگاو، و دو عشر برای هر قوچ بگذرانید.
\par 21 و یک عشر برای هر بره، از آن هفت بره بگذران.
\par 22 ویک بز نر بجهت قربانی گناه تا برای شما کفاره شود.
\par 23 اینها را سوای قربانی سوختنی صبح که قربانی سوختنی دائمی است، بگذرانید.
\par 24 به اینطور هر روز از آن هفت روز، طعام هدیه آتشین، عطر خوشبو برای خداوند بگذرانید، واین سوای قربانی سوختنی دائمی گذرانیده شود، با هدیه ریختنی آن.
\par 25 و در روز هفتم، برای شمامحفل مقدس باشد. هیچ کار خدمت در آن نکنید.
\par 26 «و در روز نوبرها چون هدیه آردی تازه درعید هفته های خود برای خداوند بگذرانید، محفل مقدس برای شما باشد و هیچ کار خدمت در آن مکنید.
\par 27 و بجهت قربانی سوختنی برای عطر خوشبوی خداوند دو گاو جوان و یک قوچ و هفت بره نرینه یک ساله قربانی کنید.
\par 28 و هدیه آردی آنها سه عشر آرد نرم سرشته شده با روغن برای هر گاو، و دو عشر برای هر قوچ.
\par 29 و یک عشر برای هر بره، از آن هفت بره.
\par 30 و یک بز نر تا برای شما کفاره شود.اینها را با هدیه آردی آنها و هدایای ریختنی آنها سوای قربانی سوختنی دائمی بگذرانید و برای شما بی‌عیب باشد. 
\par 31 اینها را با هدیه آردی آنها و هدایای ریختنی آنها سوای قربانی سوختنی دائمی بگذرانید و برای شما بی‌عیب باشد.
 
\chapter{29}

\par 1 «و در روز اول ماه هفتم، محفل مقدس برای شما باشد؛ در آن هیچ کار خدمت مکنید و برای شما روز نواختن کرنا باشد.
\par 2 وقربانی سوختنی بجهت عطر خوشبوی خداوندبگذرانید، یک گاو جوان و یک قوچ، و هفت بره نرینه یک ساله بی‌عیب.
\par 3 و هدیه آردی آنها، سه عشر آرد نرم سرشته شده با روغن برای هر گاو، ودو عشر برای هر قوچ.
\par 4 و یک عشر برای هر بره، از آن هفت بره.
\par 5 و یک بز نر بجهت قربانی گناه تابرای شما کفاره شود.
\par 6 سوای قربانی سوختنی اول ماه و هدیه آردی‌اش، و قربانی سوختنی دائمی با هدیه آردی‌اش، با هدایای ریختنی آنهابرحسب قانون آنها تا عطر خوشبو و هدیه آتشین خداوند باشد.
\par 7 «و در روز دهم این ماه هفتم، محفل مقدس برای شما باشد. جانهای خود را ذلیل سازید وهیچ کار مکنید.
\par 8 و قربانی سوختنی عطر خوشبوبرای خداوند بگذرانید، یک گاو جوان و یک قوچ و هفت بره نرینه یک ساله که برای شما بی‌عیب باشند.
\par 9 و هدیه آردی آنها سه عشر آرد نرم سرشته شده با روغن برای هر گاو، و دو عشر برای هر قوچ.
\par 10 و یک عشر برای هر بره، از آن هفت بره.
\par 11 و یک بز نر برای قربانی گناه سوای قربانی گناه کفاره‌ای و قربانی سوختنی دائمی با هدیه آردی‌اش و هدایای ریختنی آنها.
\par 12 «و در روز پانزدهم ماه هفتم، محفل مقدس برای شما باشد، هیچ کار خدمت مکنید و هفت روز برای خداوند عید نگاه دارید.
\par 13 و قربانی سوختنی هدیه آتشین عطر خوشبو برای خداوندبگذرانید. سیزده گاو جوان و دو قوچ و چهارده بره نرینه یک ساله که برای شما بی‌عیب باشند.
\par 14 و بجهت هدیه آردی آنها سه عشر آرد نرم سرشته شده با روغن برای هر گاو از آن سیزده گاو، و دو عشر برای هر قوچ از آن دو قوچ.
\par 15 ویک عشر برای هر بره از آن چهارده بره.
\par 16 و یک بز نر بجهت قربانی گناه، سوای قربانی سوختنی دائمی، با هدیه آردی و هدیه ریختنی آن.
\par 17 «و در روز دوم، دوازده گاو جوان و دو قوچ و چهارده بره نرینه یک ساله بی‌عیب.
\par 18 وهدایای آردی و هدایای ریختنی آنها برای گاوهاو قوچها و بره‌ها به شماره آنها برحسب قانون.
\par 19 و یک بز نر بجهت قربانی گناه، سوای قربانی سوختنی دائمی با هدیه آردی‌اش، و هدایای ریختنی آنها.
\par 20 «و در روز سوم، یازده گاو جوان و دو قوچ و چهارده بره نرینه یک ساله بی‌عیب.
\par 21 وهدایای آردی و هدایای ریختنی آنها برای گاوهاو قوچها و بره‌ها به شماره آنها برحسب قانون.
\par 22 و یک بز نر بجهت قربانی گناه سوای قربانی سوختنی دائمی با هدیه آردی‌اش و هدیه ریختنی آن.
\par 23 «و در روز چهارم ده گاو جوان و دو قوچ وچهارده بره نرینه یک ساله بی‌عیب.
\par 24 و هدایای آردی و هدایای ریختنی آنها برای گاوها وقوچها و بره‌ها به شماره آنها برحسب قانون.
\par 25 ویک بز نر بجهت قربانی گناه، سوای قربانی سوختنی دائمی، و هدیه آردی‌اش و هدیه ریختنی آن.
\par 26 «و در روز پنجم، نه گاو جوان و دو قوچ وچهارده بره نرینه یک ساله بی‌عیب.
\par 27 و هدایای آردی و هدایای ریختنی آنها برای گاوها وقوچها و بره‌ها به شماره آنها برحسب قانون.
\par 28 ویک بز نر بجهت قربانی گناه، سوای قربانی سوختنی دائمی و هدیه آردی‌اش و هدیه ریختنی آن.
\par 29 «و در روز ششم، هشت گاو جوان و دوقوچ و چهارده بره نرینه یک ساله بی‌عیب.
\par 30 وهدایای آردی و هدایای ریختنی آنها برای گاوهاو قوچها و بره‌ها به شماره آنها برحسب قانون.
\par 31 و یک بز نر بجهت قربانی گناه سوای قربانی سوختنی دائمی و هدیه آردی‌اش و هدایای ریختنی آن.
\par 32 «و در روز هفتم، هفت گاو جوان و دو قوچ و چهارده بره نرینه یک ساله بی‌عیب.
\par 33 وهدایای آردی و هدایای ریختنی آنها برای گاوهاو قوچها و بره‌ها به شماره آنها برحسب قانون.
\par 34 و یک بز نر بجهت قربانی گناه، سوای قربانی سوختنی دائمی و هدیه آردی‌اش و هدیه ریختنی آن.
\par 35 «و در روز هشتم، برای شما جشن مقدس باشد؛ هیچ کار خدمت مکنید.
\par 36 و قربانی سوختنی هدیه آتشین عطر خوشبو برای خداوندبگذرانید، یک گاو جوان و یک قوچ و هفت بره نرینه یک ساله بی‌عیب.
\par 37 و هدایای آردی وهدایای ریختنی آنها برای گاو و قوچ و بره‌ها به شماره آنها برحسب قانون.
\par 38 و یک بز نر برای قربانی گناه سوای قربانی سوختنی دائمی، باهدیه آردی‌اش و هدیه ریختنی آن.
\par 39 اینها را شما در موسمهای خود برای خداوند بگذرانید، سوای نذرها و نوافل خودبرای قربانی های سوختنی و هدایای آردی وهدایای ریختنی و ذبایح سلامتی خود.»پس برحسب هر‌آنچه خداوند به موسی‌امر فرموده بود، موسی بنی‌اسرائیل را اعلام نمود.
\par 40 پس برحسب هر‌آنچه خداوند به موسی‌امر فرموده بود، موسی بنی‌اسرائیل را اعلام نمود.
 
\chapter{30}

\par 1 و موسی سروران اسباط بنی‌اسرائیل راخطاب کرده، گفت: «این است کاری که خداوند امر فرموده است:
\par 2 چون شخصی برای خداوند نذر کند یا قسم خورد تا جان خود را به تکلیفی الزام نماید، پس کلام خود را باطل نسازد، بلکه برحسب هر‌آنچه از دهانش برآمد، عمل نماید.
\par 3 «و اما چون زن برای خداوند نذر کرده، خودرا در خانه پدرش در جوانی‌اش به تکلیفی الزام نماید،
\par 4 و پدرش نذر او و تکلیفی که خود را برآن الزام نموده، شنیده باشد، و پدرش درباره اوساکت باشد، آنگاه تمامی نذرهایش استوار، و هرتکلیفی که خود را به آن الزام نموده باشد، قایم خواهد بود.
\par 5 اما اگر پدرش در روزی که شنید اورا منع کرد، آنگاه هیچ کدام از نذرهایش و ازتکالیفش که خود را به آن الزام نموده باشد، استوار نخواهد بود، و از این جهت که پدرش او رامنع نموده است، خداوند او را خواهد آمرزید.
\par 6 «و اگر به شوهری داده شود، و نذرهای او یاسخنی که از لبهایش جسته، و جان خود را به آن الزام نموده، بر او باشد،
\par 7 و شوهرش شنید و درروز شنیدنش به وی هیچ نگفت، آنگاه نذرهایش استوار خواهد ماند. و تکلیفهایی که خویشتن را به آنها الزام نموده است، قایم خواهند ماند.
\par 8 لیکن اگر شوهرش در روزی که آن را شنید، او را منع نماید، و نذری را که بر او است یا سخنی را که ازلبهایش جسته، و خویشتن را به آن الزام نموده باشد، باطل سازد، پس خداوند او را خواهدآمرزید.
\par 9 اما نذر زن بیوه یا مطلقه، در هر‌چه خود را به آن الزام نموده باشد، بر وی استوارخواهد ماند.
\par 10 و اما اگر زنی در خانه شوهرش نذر کند، یا خویشتن را با قسم به تکلیفی الزام نماید،
\par 11 و شوهرش بشنود و او را هیچ نگوید ومنع ننماید، پس تمامی نذرهایش استوار، و هرتکلیفی که خویشتن را به آن الزام نموده باشد، قایم خواهد بود.
\par 12 و اما اگر شوهرش در روزی که بشنود، آنها را باطل سازد. پس هر‌چه ازلبهایش درآمده باشد درباره نذرهایش یا تکالیف خود، استوار نخواهد ماند. چونکه شوهرش آن راباطل نموده است، خداوند او را خواهد آمرزید.
\par 13 هر نذری و هر قسم الزامی را برای ذلیل ساختن جان خود، شوهرش آن را استوار نماید، وشوهرش آن را باطل سازد.
\par 14 اما اگر شوهرش روز به روز به او هیچ نگوید، پس همه نذرهایش وهمه تکالیفش را که بر وی باشد استوار نموده باشد، چونکه در روزی که شنید به وی هیچ نگفت، پس آنها را استوار نموده است.
\par 15 و اگربعد از شنیدن، آنها را باطل نمود، پس او گناه وی را متحمل خواهد بود.»این است فرایضی که خداوند به موسی‌امرفرمود، در میان مرد و زنش و در میان پدر ودخترش، در زمان جوانی او در خانه پدر وی.
\par 16 این است فرایضی که خداوند به موسی‌امرفرمود، در میان مرد و زنش و در میان پدر ودخترش، در زمان جوانی او در خانه پدر وی.
 
\chapter{31}

\par 1 و خداوند موسی را خطاب کرده، گفت:
\par 2 «انتقام بنی‌اسرائیل را از مدیانیان بگیر، و بعد از آن به قوم خود ملحق خواهی شد.»
\par 3 پس موسی قوم را مخاطب ساخته، گفت: «ازمیان خود مردان برای جنگ مهیا سازید تا به مقابله مدیان برآیند، و انتقام خداوند را از مدیان بکشند.
\par 4 هزار نفر از هر سبط از جمیع اسباطاسرائیل برای جنگ بفرستید.»
\par 5 پس از هزاره های اسرائیل، از هر سبط یک هزار، یعنی دوازده هزار نفر مهیا شده برای جنگ منتخب شدند.
\par 6 و موسی ایشان را هزار نفر از هرسبط به جنگ فرستاد، ایشان را با فینحاس بن العازار کاهن و اسباب قدس و کرناها برای نواختن در دستش به جنگ فرستاد.
\par 7 و با مدیان به طوری که خداوند موسی را امر فرموده بود، جنگ کرده، همه ذکوران را کشتند.
\par 8 و در میان کشتگان ملوک مدیان یعنی اوی و راقم و صور و حور و رابع، پنج پادشاه مدیان را کشتند، بلعام بن بعور را به شمشیر کشتند.
\par 9 و بنی‌اسرائیل زنان مدیان واطفال ایشان را به اسیری بردند، و جمیع بهایم وجمیع مواشی ایشان و همه املاک ایشان را غارت کردند.
\par 10 و تمامی شهرها و مساکن و قلعه های ایشان را به آتش سوزانیدند.
\par 11 و تمامی غنیمت و جمیع غارت را از انسان و بهایم گرفتند.
\par 12 واسیران و غارت و غنیمت را نزد موسی و العازارکاهن و جماعت بنی‌اسرائیل در لشکرگاه درعربات موآب، که نزد اردن در مقابل اریحاست، آوردند.
\par 13 و موسی و العازار کاهن و تمامی سروران جماعت بیرون از لشکرگاه به استقبال ایشان آمدند.
\par 14 و موسی بر روسای لشکر یعنی سرداران هزاره‌ها و سرداران صدها که از خدمت جنگ باز آمده بودند، غضبناک شد.
\par 15 و موسی به ایشان گفت: «آیا همه زنان را زنده نگاه داشتید؟
\par 16 اینک اینانند که برحسب مشورت بلعام، بنی‌اسرائیل را واداشتند تا در امر فغور به خداوندخیانت ورزیدند و در جماعت خداوند وباعارض شد.
\par 17 پس الان هر ذکوری از اطفال رابکشید، و هر زنی را که مرد را شناخته، با اوهمبستر شده باشد، بکشید.
\par 18 و از زنان هردختری را که مرد را نشناخته، و با او همبسترنشده برای خود زنده نگاه دارید.
\par 19 و شما هفت روز بیرون از لشکرگاه خیمه زنید، و هر شخصی را کشته و هر‌که کشته‌ای را لمس نموده باشد ازشما و اسیران شما در روز سوم و در روز هفتم، خود را تطهیر نماید.
\par 20 و هر جامه و هرظرف چرمی و هر‌چه از پشم بز ساخته شده باشد و هرظرف چوبین را تطهیر نمایید.»
\par 21 و العازار کاهن به مردان جنگی که به مقاتله رفته بودند، گفت: «این است قانون شریعتی که خداوند به موسی‌امر فرموده است:
\par 22 طلا و نقره و برنج و آهن و روی و سرب،
\par 23 یعنی هر‌چه متحمل آتش بشود، آن را از آتش بگذرانید و طاهر خواهد شد، و به آب تنزیه نیز آن را طاهرسازند و هر‌چه متحمل آتش نشود، آن را از آب بگذرانید.
\par 24 و در روز هفتم رخت خود رابشویید تا طاهر شوید، و بعد از آن به لشکرگاه داخل شوید.»
\par 25 و خداوند موسی را خطاب کرده، گفت:
\par 26 «تو و العازار کاهن و سروران خاندان آبای جماعت، حساب غنایمی که گرفته شده است، چه از انسان و چه از بهایم بگیرید.
\par 27 و غنیمت رادر میان مردان جنگی که به مقاتله بیرون رفته‌اند، وتمامی جماعت نصف نما.
\par 28 و از مردان جنگی که به مقاتله بیرون رفته‌اند زکات برای خداوندبگیر، یعنی یک نفر از پانصد چه از انسان و چه ازگاو و چه از الاغ و چه از گوسفند.
\par 29 از قسمت ایشان بگیر و به العازار کاهن بده تا هدیه افراشتنی برای خداوند باشد.
\par 30 و از قسمت بنی‌اسرائیل یکی که از هر پنجاه نفر گرفته شده باشد چه ازانسان و چه از گاو و چه از الاغ و چه از گوسفند وچه از جمیع بهایم بگیر، و آنها را به لاویانی که ودیعت مسکن خداوند را نگاه می‌دارند، بده.»
\par 31 پس موسی و العازار کاهن برحسب آنچه خداوند به موسی‌امر فرموده بود، عمل کردند.
\par 32 و غنیمت سوای آن غنیمتی که مردان جنگی گرفته بودند، از گوسفند ششصد و هفتاد و پنج هزار راس بود.
\par 33 و از گاو هفتاد و دو هزار راس. 
\par 34 و از الاغ شصت و یک هزار راس.
\par 35 و ازانسان از زنانی که مرد را نشناخته بودند، سی و دوهزار نفر بودند.
\par 36 و نصفه‌ای که قسمت کسانی بود که به جنگ رفته بودند، سیصد و سی و هفت هزار و پانصد گوسفند بود.
\par 37 و زکات خداوند ازگوسفند ششصد و هفتاد و پنج راس بود.
\par 38 وگاوان سی و شش هزار بود و از آنها زکات خداوند هفتاد و دو راس بود.
\par 39 و الاغها سی هزار و پانصد و از آنها زکات خداوند شصت ویک راس بود.
\par 40 و مردمان شانزده هزار و ازایشان زکات خداوند سی و دو نفر بودند.
\par 41 وموسی زکات را هدیه افراشتی خداوند بود به العازار کاهن داد، چنانکه خداوند به موسی‌امرفرموده بود.
\par 42 و از قسمت بنی‌اسرائیل که موسی آن را ازمردان جنگی جدا کرده بود،
\par 43 و قسمت جماعت از گوسفندان، سیصد و سی هزار وپانصد راس بود.
\par 44 و از گاوان سی و شش هزارراس.
\par 45 و از الاغها، سی هزار و پانصد راس.
\par 46 واز انسان، شانزده هزار نفر.
\par 47 و موسی از قسمت بنی‌اسرائیل یکی را که از هر پنجاه گرفته شده بود، چه از انسان و چه از بهایم گرفت، و آنها را به لاویانی که ودیعت مسکن خداوند را نگاه می‌داشتند، داد، چنانکه خداوند به موسی‌امرفرموده بود.
\par 48 و روسایی که بر هزاره های لشکر بودند، سرداران هزاره‌ها با سرداران صدها نزد موسی آمدند.
\par 49 و به موسی گفتند: «بندگانت حساب مردان جنگی را که زیردست ما می‌باشند گرفتیم، و از ما یک نفر مفقود نشده است.
\par 50 پس ما ازآنچه هر کس یافته است هدیه‌ای برای خداوندآورده‌ایم از زیورهای طلا و خلخالها ودست بندها و انگشترها و گوشواره‌ها وگردن بندها تا برای جانهای ما به حضور خداوند کفاره شود.»
\par 51 و موسی و العازار کاهن، طلا وهمه زیورهای مصنوعه را از ایشان گرفتند.
\par 52 وتمامی طلای هدیه‌ای که از سرداران هزاره‌ها وسرداران صدها برای خداوند گذرانیدند، شانزده هزار و هفتصد و پنجاه مثقال بود.
\par 53 زیرا که هریکی از مردان جنگی غنیمتی برای خود برده بودند.و موسی و العازار کاهن، طلا را ازسرداران هزاره‌ها و صدها گرفته، به خیمه اجتماع آوردند تا بجهت بنی‌اسرائیل، به حضور خداوندیادگار باشد.
\par 54 و موسی و العازار کاهن، طلا را ازسرداران هزاره‌ها و صدها گرفته، به خیمه اجتماع آوردند تا بجهت بنی‌اسرائیل، به حضور خداوندیادگار باشد.
 
\chapter{32}

\par 1 و بنی روبین و بنی جاد را مواشی بی نهایت بسیار و کثیر بود، پس چون زمین یعزیر و زمین جلعاد را دیدند که اینک این مکان، مکان مواشی است.
\par 2 بنی جاد و بنی روبین نزد موسی و العازار کاهن و سروران جماعت آمده، گفتند:
\par 3 «عطاروت و دیبون و یعزیر و نمره و حشبون و العاله و شبام و نبو و بعون،
\par 4 زمینی که خداوند پیش روی جماعت اسرائیل مفتوح ساخته است، زمین مواشی است، و بندگانت صاحب مواشی می‌باشیم.
\par 5 پس گفتند: اگر درنظر تو التفات یافته‌ایم، این زمین به بندگانت به ملکیت داده شود، و ما را از اردن عبور مده.»
\par 6 موسی به بنی جاد و بنی روبین گفت: «آیابرادران شما به جنگ روند و شما اینجا بنشینید؟
\par 7 چرا دل بنی‌اسرائیل را افسرده می‌کنید تا به زمینی که خداوند به ایشان داده است، عبورنکنند؟
\par 8 به همین طور پدران شما عمل نمودند، وقتی که ایشان را از قادش برنیع برای دیدن زمین فرستادم.
\par 9 به وادی اشکول رفته، زمین را دیدند ودل بنی‌اسرائیل را افسرده ساختند تا به زمینی که خداوند به ایشان داده بود، داخل نشوند.
\par 10 پس غضب خداوند در آن روز افروخته شد به حدی که قسم خورده، گفت:
\par 11 البته هیچکدام ازمردانی که از مصر بیرون آمدند از بیست ساله وبالاتر آن زمین را که برای ابراهیم و اسحاق ویعقوب قسم خوردم، نخواهند دید، چونکه ایشان مرا پیروی کامل ننمودند.
\par 12 سوای کالیب بن یفنه قنزی و یوشع بن نون، چونکه ایشان خداوند را پیروی کامل نمودند.
\par 13 پس غضب خداوند بر اسرائیل افروخته شده، ایشان را چهل سال در بیابان آواره گردانید، تا تمامی آن گروهی که این شرارت را در نظر خداوند ورزیده بودند، هلاک شدند.
\par 14 و اینک شما به‌جای پدران خودانبوهی از مردان خطاکار برپا شده‌اید تا شدت غضب خداوند را بر اسرائیل باز زیاده کنید؟
\par 15 زیرا اگر از پیروی او روبگردانید بار دیگرایشان را در بیابان ترک خواهد کرد و شما تمامی این قوم را هلاک خواهید ساخت.»
\par 16 پس ایشان نزد وی آمده، گفتند: «آغلها رااینجا برای مواشی خود و شهرها بجهت اطفال خویش خواهیم ساخت.
\par 17 و خود مسلح شده، حاضر می‌شویم و پیش روی بنی‌اسرائیل خواهیم رفت تا آنها را به مکان ایشان برسانیم. واطفال ما از ترس ساکنان زمین در شهرهای حصاردار خواهند ماند.
\par 18 و تا هر یکی ازبنی‌اسرائیل ملک خود را نگرفته باشد، به خانه های خود مراجعت نخواهیم کرد.
\par 19 زیرا که ما با ایشان در آن طرف اردن و ماورای آن ملک نخواهیم گرفت، چونکه نصیب ما به این طرف اردن به طرف مشرق به ما رسیده است.»
\par 20 و موسی به ایشان گفت: «اگر این کار رابکنید و خویشتن را به حضور خداوند برای جنگ مهیا سازید،
\par 21 و هر مرد جنگی از شما به حضورخداوند از اردن عبور کند تا او دشمنان خود را ازپیش روی خود اخراج نماید،
\par 22 و زمین به حضور خداوند مغلوب شود، پس بعد از آن برگردیده، به حضور خداوند و به حضور اسرائیل بی‌گناه خواهید شد، و این زمین از جانب خداوندملک شما خواهد بود.
\par 23 و اگر چنین نکنید اینک به خداوند گناه ورزیده‌اید، و بدانید که گناه شما، شما را درخواهد گرفت.
\par 24 پس شهرها برای اطفال و آغلها برای گله های خود بنا کنید، و به آنچه از دهان شما درآمد، عمل نمایید.»
\par 25 پس بنی جاد و بنی روبین موسی را خطاب کرده، گفتند: «بندگانت به طوری که آقای مافرموده است، خواهیم کرد.
\par 26 اطفال و زنان ومواشی و همه بهایم ما اینجا در شهرهای جلعادخواهند ماند.
\par 27 و جمیع بندگانت مهیای جنگ شده، چنانکه آقای ما گفته است به حضورخداوند برای مقاتله عبور خواهیم نمود.»
\par 28 پس موسی العازار کاهن، و یوشع بن نون، وروسای خاندان آبای اسباط بنی‌اسرائیل را درباره ایشان وصیت نمود.
\par 29 و موسی به ایشان گفت: «اگر جمیع بنی جاد و بنی روبین مهیای جنگ شده، همراه شما به حضور خداوند از اردن عبورکنند، و زمین پیش روی شما مغلوب شود، آنگاه زمین جلعاد را برای ملکیت به ایشان بدهید.
\par 30 واگر ایشان مهیا نشوند و همراه شما عبور ننمایند، پس در میان شما در زمین کنعان ملک بگیرند.»
\par 31 بنی جاد و بنی روبین در جواب وی گفتند: «چنانکه خداوند به بندگانت گفته است، همچنین خواهیم کرد.
\par 32 ما مهیای جنگ شده، پیش روی خداوند به زمین کنعان عبور خواهیم کرد، و ملک نصیب ما به این طرف اردن داده شود.»
\par 33 پس موسی به ایشان یعنی به بنی جاد وبنی روبین و نصف سبط منسی ابن یوسف، مملکت سیحون، ملک اموریان و مملکت عوج ملک باشان را داد، یعنی زمین را با شهرهایش وحدود شهرهایش، زمین را از هر طرف.
\par 34 وبنی جاد، دیبون و عطاروت و عروعیر.
\par 35 وعطروت، شوفان و یعزیز و یجبهه.
\par 36 و بیت نمره و بیت هاران را بنا کردند یعنی شهرهای حصارداررا با آغلهای گله‌ها.
\par 37 و بنی روبین حشبون والیعاله و قریتایم.
\par 38 و نبو و بعل معون که نام این دورا تغییر دادند و سبمه را بنا کردند و شهرهایی راکه بنا کردند به نامها مسمی ساختند.
\par 39 وبنی ماکیر بن منسی به جلعاد رفته، آن را گرفتند واموریان را که در آن بودند، اخراج نمودند.
\par 40 وموسی جلعاد را به ماکیر بن منسی داد و او در آن ساکن شد.
\par 41 و یائیر بن منسی رفته، قصبه هایش راگرفت، و آنها را حووت یائیر نامید.و نوبح رفته، قنات و دهاتش را گرفته، آنها را به اسم خودنوبح نامید.
\par 42 و نوبح رفته، قنات و دهاتش را گرفته، آنها را به اسم خودنوبح نامید.
 
\chapter{33}

\par 1 این است منازل بنی‌اسرائیل که از زمین مصر با افواج خود زیردست موسی وهارون کوچ کردند.
\par 2 و موسی به فرمان خداوندسفرهای ایشان را برحسب منازل ایشان نوشت، واین است منازل و مراحل ایشان:
\par 3 پس در ماه اول از رعمسیس، در روز پانزدهم از ماه اول کوچ کردند، و در فردای بعد از فصح بنی‌اسرائیل درنظر تمامی مصریان با دست بلند بیرون رفتند.
\par 4 ومصریان همه نخست زادگان خود را که خداوند ازایشان کشته بود دفن می‌کردند، و یهوه بر خدایان ایشان قصاص نموده بود.
\par 5 و بنی‌اسرائیل از رعمسیس کوچ کرده، درسکوت فرود آمدند.
\par 6 و از سکوت کوچ کرده، درایتام که به کنار بیابان است، فرود آمدند.
\par 7 و ازایتام کوچ کرده، به سوی فم الحیروت که در مقابل بعل صفون است، برگشتند، و پیش مجدل فرودآمدند.
\par 8 و از مقابل حیروت کوچ کرده، از میان دریا به بیابان عبور کردند و در بیابان ایتام سفر سه روزه کرده، در ماره فرود آمدند.
\par 9 و از ماره کوچ کرده، به ایلیم رسیدند و در ایلیم دوازده چشمه آب و هفتاد درخت خرما بود، و در آنجا فرودآمدند.
\par 10 و از ایلیم کوچ کرده، نزد بحر قلزم فرود آمدند.
\par 11 و از بحر قلزم کوچ کرده، در بیابان سین فرود آمدند.
\par 12 و از بیابان سین کوچ کرده، دردفقه فرود آمدند.
\par 13 و از دفقه کوچ کرده، درالوش فرود آمدند.
\par 14 و از الوش کوچ کرده، دررفیدیم فرود آمدند و در آنجا آب نبود که قوم بنوشند.
\par 15 و از رفیدیم کوچ کرده، در بیابان سینافرود آمدند.
\par 16 و از بیابان سینا کوچ کرده، درقبروت هتاوه فرود آمدند.
\par 17 و از قبروت هتاوه کوچ کرده، در حصیروت فرود آمدند.
\par 18 و ازحصیروت کوچ کرده، در رتمه فرود آمدند.
\par 19 واز رتمه کوچ کرده، و در رمون فارص فرود آمدند.
\par 20 و از رمون فارص کوچ کرده، در لبنه فرودآمدند.
\par 21 و از لبنه کوچ کرده، در رسه فرودآمدند.
\par 22 و از رسه کوچ کرده، در قهیلاته فرودآمدند.
\par 23 و از قهیلاته کوچ کرده، در جبل شافر فرود آمدند.
\par 24 و از جبل شافر کوچ کرده، درحراده فرود آمدند.
\par 25 و از حراده کوچ کرده، درمقهیلوت فرود آمدند.
\par 26 و از مقهیلوت کوچ کرده، در تاحت فرود آمدند.
\par 27 و از تاحت کوچ کرده، در تارح فرود آمدند.
\par 28 و از تارح کوچ کرده، در متقه فرود آمدند.
\par 29 و از متقه کوچ کرده، در حشمونه فرود آمدند.
\par 30 و از حشمونه کوچ کرده، در مسیروت فرود آمدند.
\par 31 و از مسیروت کوچ کرده، در بنی یعقان فرود آمدند.
\par 32 و ازبنی یعقان کوچ کرده، در حورالجدجاد فرودآمدند.
\par 33 و از حورالجدجاد کوچ کرده، دریطبات فرود آمدند.
\par 34 و از یطبات کوچ کرده، درعبرونه فرود آمدند.
\par 35 و از عبرونه کوچ کرده، درعصیون جابر فرود آمدند.
\par 36 و از عصیون جابرکوچ کرده، در بیابان صین که قادش باشد، فرودآمدند.
\par 37 و از قادش کوچ کرده، در جبل هور درسرحد زمین ادوم فرود آمدند.
\par 38 هارون کاهن برحسب فرمان خداوند به جبل هور برآمده، در سال چهلم خروج بنی‌اسرائیل از زمین مصر، در روز اول ماه پنجم وفات یافت.
\par 39 و هارون صد و بیست و سه ساله بود که در جبل هور مرد.
\par 40 و ملک عراد کنعانی که در جنوب زمین کنعان ساکن بود از آمدن بنی‌اسرائیل اطلاع یافت.
\par 41 پس از جبل هور کوچ کرده، در صلمونه فرود آمدند.
\par 42 و از صلمونه کوچ کرده در فونون فرود آمدند
\par 43 و از فونون کوچ کرده، در اوبوت فرود آمدند.
\par 44 و از اوبوت کوچ کرده، درعیی عباریم در حدود موآب فرود آمدند.
\par 45 و از عییم کوچ کرده، در دیبون جاد فرود آمدند.
\par 46 واز دیبون جاد کوچ کرده، در علمون دبلاتایم فرودآمدند.
\par 47 و از علمون دبلاتایم کوچ کرده، درکوههای عباریم در مقابل نبو فرود آمدند.
\par 48 و ازکوههای عباریم کوچ کرده، در عربات موآب نزداردن در مقابل اریحا فرود آمدند. 
\par 49 پس نزد اردن از بیت یشیموت تا آبل شطیم در عربات موآب اردو زدند.
\par 50 و خداوند موسی را در عربات مواب نزداردن، در مقابل اریحا خطاب کرده، گفت:
\par 51 «بنی‌اسرائیل را خطاب کرده، به ایشان بگو: چون شما از اردن به زمین کنعان عبور کنید،
\par 52 جمیع ساکنان زمین را از پیش روی خوداخراج نمایید، و تمامی صورتهای ایشان راخراب کنید، و تمامی بتهای ریخته شده ایشان رابشکنید، و همه مکانهای بلند ایشان را منهدم سازید.
\par 53 و زمین را به تصرف آورده، در آن ساکن شوید، زیرا که آن زمین را به شما دادم تا مالک آن باشید.
\par 54 و زمین را به حسب قبایل خود به قرعه تقسیم کنید، برای کثیر، نصیب او را کثیر بدهید، وبرای قلیل، نصیب او را قلیل بدهید، جایی که قرعه برای هر کس برآید از آن او باشد برحسب اسباط آبای شما آن را تقسیم نمایید.
\par 55 و اگرساکنان زمین را از پیش روی خود اخراج ننماییدکسانی را که از ایشان باقی می‌گذارید در چشمان شما خار خواهند بود، و در پهلوهای شما تیغ وشما را در زمینی که در آن ساکن شوید، خواهندرنجانید.و به همان طوری که قصد نمودم که با ایشان رفتار نمایم، با شما رفتار خواهم نمود.»
\par 56 و به همان طوری که قصد نمودم که با ایشان رفتار نمایم، با شما رفتار خواهم نمود.»
 
\chapter{34}

\par 1 و خداوند موسی را خطاب کرده، گفت:
\par 2 «بنی‌اسرائیل را امر فرموده، به ایشان بگو: چون شما به زمین کنعان داخل شوید، این است زمینی که به شما به ملکیت خواهد رسید، یعنی زمین کنعان با حدودش.
\par 3 آنگاه حد جنوبی شما از بیابان سین بر جانب ادوم خواهد بود، و سرحد جنوبی شما از آخر بحرالملح به طرف مشرق خواهد بود.
\par 4 و حد شما از جانب جنوب گردنه عقربیم دور خواهد زد و به سوی سین خواهدگذشت، و انتهای آن به طرف جنوب قادش برنیع خواهد بود، و نزد حصرادار بیرون رفته، تاعصمون خواهد گذشت.
\par 5 و این حد از عصمون تاوادی مصر دور زده، انتهایش نزد دریا خواهدبود.
\par 6 و اما حد غربی. دریای بزرگ حد شماخواهد بود. این است حد غربی شما.
\par 7 و حد شمالی شما این باشد، از دریای بزرگ برای خود جبل هور را نشان گیرید.
\par 8 و از جبل هور تا مدخل حمات را نشان گیرید. و انتهای این حد نزد صدد باشد.
\par 9 و این حد نزد زفرون بیرون رود و انتهایش نزد حصر عینان باشد، این حدشمالی شما خواهد بود.
\par 10 و برای حد مشرقی خود از حصر عینان تاشفام را نشان گیرید.
\par 11 و این حد از شفام تا ربله به طرف شرقی عین برود، پس این حد کشیده شده به‌جانب دریای کنرت به طرف مشرق برسد.
\par 12 واین حد تا به اردن برسد و انتهایش نزد بحرالملح باشد. این زمین برحسب حدودش به هر طرف زمین شما خواهد بود.»
\par 13 و موسی بنی‌اسرائیل را امر کرده، گفت: «این است زمینی که شما آن را به قرعه تقسیم خواهید کرد که خداوند امر فرموده است تا به نه سبط و نصف داده شود.
\par 14 زیرا که سبط بنی روبین برحسب خاندان آبای خود و سبط بنی جادبرحسب خاندان آبای خود، و نصف سبط منسی، نصیب خود را گرفته‌اند.
\par 15 این دو سبط و نصف به آن طرف اردن در مقابل اریحا به‌جانب مشرق به سوی طلوع آفتاب نصیب خود را گرفته‌اند.»
\par 16 و خداوند موسی را خطاب کرده، گفت:
\par 17 «این است نامهای کسانی که زمین را برای شماتقسیم خواهند نمود. العازار کاهن و یوشع بن نون.
\par 18 و یک سرور را از هر سبط برای تقسیم نمودن زمین بگیرید.
\par 19 و این است نامهای ایشان. ازسبط یهودا کالیب بن یفنه.
\par 20 و از سبطبنی شمعون شموئیل بن عمیهود.
\par 21 و از سبطبنیامین الیداد بن کسلون.
\par 22 و از سبط بنی دان رئیس بقی ابن یجلی.
\par 23 و از بنی یوسف از سبطبنی منسی رئیس حنیئیل بن ایفود.
\par 24 و از سبطبنی افرایم رئیس قموئیل بن شفطان.
\par 25 و از سبطبنی زبولون رئیس الیصافان بن فرناک.
\par 26 و از سبطبنی یساکار رئیس فلطیئیل بن عزان.
\par 27 و از سبطبنی اشیر رئیس اخیهود بن شلومی.
\par 28 و از سبطبنی نفتالی رئیس فدهئیل بن عمیهود.»اینانندکه خداوند مامور فرمود که ملک را در زمین کنعان برای بنی‌اسرائیل تقسیم نمایند.
\par 29 اینانندکه خداوند مامور فرمود که ملک را در زمین کنعان برای بنی‌اسرائیل تقسیم نمایند.
 
\chapter{35}

\par 1 و خداوند موسی را در عربات موآب نزد اردن در مقابل اریحا خطاب کرده، گفت:
\par 2 «بنی‌اسرائیل را امر فرما که از نصیب ملک خود شهرها برای سکونت به لاویان بدهند، ونواحی شهرها را از اطراف آنها به لاویان بدهید.
\par 3 و شهرها بجهت سکونت ایشان باشد، و نواحی آنها برای بهایم و اموال و سایر حیوانات ایشان باشد.
\par 4 و نواحی شهرها که به لاویان بدهید ازدیوار شهر بیرون از هر طرف هزار ذراع باشد.
\par 5 واز بیرون شهر به طرف مشرق دو هزار ذراع، و به طرف جنوب دو هزار ذراع، و به طرف مغرب دوهزار ذراع، و به طرف شمال دو هزار ذراع بپیمایید. و شهر در وسط باشد و این نواحی شهرها برای ایشان خواهد بود.
\par 6 «و از شهرها که به لاویان بدهید شش شهرملجا خواهد بود، و آنها را برای قاتل بدهید تا به آنجا فرار کند و سوای آنها چهل و دو شهربدهید.
\par 7 پس جمیع شهرها که به لاویان خواهیدداد چهل و هشت شهر با نواحی آنها خواهد بود.
\par 8 و اما شهرهایی که از ملک بنی‌اسرائیل می‌دهیداز کثیر، کثیر و از قلیل، قلیل بگیرید. هر کس به اندازه نصیب خود که یافته باشد از شهرهای خودبه لاویان بدهد.»
\par 9 و خداوند موسی را خطاب کرده، گفت:
\par 10 «بنی‌اسرائیل را خطاب کرده، به ایشان بگو: چون شما از اردن به زمین کنعان عبور کنید،
\par 11 آنگاه شهرها برای خود تعیین کنید تا شهرهای ملجا برای شما باشد، تا هر قاتلی که شخصی راسهو کشته باشد، به آنجا فرار کند.
\par 12 و این شهرها برای شما بجهت ملجا از ولی مقتول خواهد بود، تا قاتل پیش از آنکه به حضورجماعت برای داوری بایستد، نمیرد.
\par 13 «و از شهرهایی که می‌دهید، شش شهرملجا برای شما باشد.
\par 14 سه شهر از آنطرف اردن بدهید، و سه شهر در زمین کنعان بدهید تاشهرهای ملجا باشد.
\par 15 بجهت بنی‌اسرائیل وغریب و کسی‌که در میان شما وطن گزیند، این شش شهر برای ملجا باشد تا هر‌که شخصی راسهو کشته باشد به آنجا فرار کند.
\par 16 «و اگر او را به آلت آهنین زد که مرد، او قاتل است و قاتل البته کشته شود.
\par 17 و اگر او را بادست خود به سنگی که از آن کسی کشته شود، بزند تا بمیرد، او قاتل است و قاتل البته کشته شود.
\par 18 و اگر او را به چوب دستی که به آن کسی کشته شود، بزند تا بمیرد، او قاتل است و قاتل البته کشته شود.
\par 19 ولی خون، خود، قاتل رابکشد. هرگاه به او برخورد، او را بکشد.
\par 20 و اگراز روی بغض او را با تیغ زد یا قصد چیزی بر اوانداخت که مرد،
\par 21 یا از روی عداوت او را بادست خود زد که مرد، آن زننده چون که قاتل است البته کشته شود، ولی خون هرگاه به قاتل برخورد، او را بکشد.
\par 22 «لیکن اگر او را بدون عداوت سهو تیغ زندیا چیزی بدون قصد بر او اندازد،
\par 23 و اگر سنگی را که کسی به آن کشته شود نادیده بر او بیندازد که بمیرد و با وی دشمنی نداشته، و بداندیش اونبوده باشد،
\par 24 پس جماعت در میان قاتل و ولی خون برحسب این احکام داوری نمایند.
\par 25 وجماعت، قاتل را از دست ولی خون رهایی دهند، و جماعت، وی را به شهر ملجای او که به آن فرارکرده بود برگردانند، و او در آنجا تا موت رئیس کهنه که به روغن مقدس مسح شده است، ساکن باشد.
\par 26 و اگر قاتل وقتی از حدود شهر ملجای خود که به آن فرار کرده بود بیرون آید،
\par 27 و ولی خون، او را بیرون حدود شهر ملجایش بیابد، پس ولی خون قاتل را بکشد؛ قصاص خون برای اونشود.
\par 28 زیرا که می‌بایست تا وفات رئیس کهنه در شهر ملجای خود مانده باشد، و بعد ازوفات رئیس کهنه، قاتل به زمین ملک خودبرگردد.
\par 29 «و این احکام برای شما در قرنهای شما درجمیع مسکنهای شما فریضه عدالتی خواهد بود.
\par 30 «هر‌که شخصی را بکشد پس قاتل به گواهی شاهدان کشته شود، و یک شاهد برای کشته شدن کسی شهادت ندهد.
\par 31 و هیچ فدیه به عوض جان قاتلی که مستوجب قتل است، مگیرید بلکه او البته کشته شود.
\par 32 و از کسی‌که به شهر ملجای خود فرار کرده باشد فدیه مگیرید، که پیش از وفات کاهن برگردد و به زمین خودساکن شود.
\par 33 و زمینی را که در آن ساکنید ملوث مسازید، زیرا که خون، زمین را ملوث می‌کند، وزمین را برای خونی که در آن ریخته شود، کفاره نمی توان کرد مگر به خون کسی‌که آن را ریخته باشد.پس زمینی را که شما در آن ساکنید ومن در میان آن ساکن هستم نجس مسازید، زیرا من که یهوه هستم در میان بنی‌اسرائیل ساکن می‌باشم.»
\par 34 پس زمینی را که شما در آن ساکنید ومن در میان آن ساکن هستم نجس مسازید، زیرا من که یهوه هستم در میان بنی‌اسرائیل ساکن می‌باشم.»
 
\chapter{36}

\par 1 و روسای خاندان آبای قبیله بنی جلعادبن ماکیربن منسی که از قبایل بنی یوسف بودند نزدیک آمده به حضور موسی وبه حضور سروران و روسای خاندان آبای بنی‌اسرائیل عرض کرده،
\par 2 گفتند: «خداوند، آقای ما را امر فرمود که زمین را به قرعه تقسیم کرده، به بنی‌اسرائیل بدهد، و آقای ما از جانب خداوند مامور شده است که نصیب برادر ماصلفحاد را به دخترانش بدهد.
\par 3 پس اگر ایشان به یکی از پسران سایر اسباط بنی‌اسرائیل منکوحه شوند، ارث ما از میراث پدران ما قطع شده، به میراث سبطی که ایشان به آن داخل شوند، اضافه خواهد شد، و از بهره میراث ما قطع خواهد شد.
\par 4 و چون یوبیل بنی‌اسرائیل بشود ملک ایشان به ملک سبطی که به آن داخل شوند اضافه خواهدشد، و ملک ایشان از ملک پدران ما قطع خواهدشد.»
\par 5 پس موسی بنی‌اسرائیل را برحسب قول خداوند امر فرموده، گفت: «سبط بنی یوسف راست گفتند.
\par 6 این است آنچه خداوند درباره دختران صلفحاد امر فرموده، گفته است: به هر‌که در نظر ایشان پسند آید، به زنی داده شوند، لیکن در قبیله سبط پدران خود فقط به نکاح داده شوند.
\par 7 پس میراث بنی‌اسرائیل از سبط به سبط منتقل نشود، بلکه هر یکی از بنی‌اسرائیل به میراث سبطپدران خود ملصق باشند.
\par 8 و هر دختری که وارث ملکی از اسباط بنی‌اسرائیل بشود، به کسی از قبیله سبط پدر خود به زنی داده شود، تا هریکی از بنی‌اسرائیل وارث ملک آبای خود گردند.
\par 9 و ملک از یک سبط به سبط دیگر منتقل نشود، بلکه هرکس از اسباط بنی‌اسرائیل به میراث خودملصق باشند.»
\par 10 پس چنانکه خداوند موسی را امر فرمود، دختران صلفحاد چنان کردند.
\par 11 و دختران صلفحاد، محله و ترصه و حجله و ملکه و نوعه به پسران عموهای خود به زنی داده شدند.درقبایل بنی منسی ابن یوسف منکوحه شدند وملک ایشان در سبط قبیله پدر ایشان باقی ماند.
\par 12 درقبایل بنی منسی ابن یوسف منکوحه شدند وملک ایشان در سبط قبیله پدر ایشان باقی ماند.


\end{document}