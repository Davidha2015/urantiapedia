\begin{document}

\title{Deuteronomy}

 
\chapter{1}

\par 1 اردن، در بیابان عربه مقابل سوف، در میان فاران و توفل و لابان و حضیروت و دی ذهب باتمامی اسرائیل گفت.
\par 2 از حوریب به راه جبل سعیر تا قادش برنیع، سفر یازده روزه است.
\par 3 پس در روز اول ماه یازدهم سال چهلم، موسی بنی‌اسرائیل را برحسب هرآنچه خداوند او رابرای ایشان امر فرموده بود تکلم نمود،
\par 4 بعد ازآنکه سیحون ملک اموریان را که در حشبون ساکن بود و عوج ملک باشان را که در عشتاروت درادرعی ساکن بود، کشته بود.
\par 5 به آن طرف اردن درزمین موآب، موسی به بیان کردن این شریعت شروع کرده، گفت: دستور ترک حوریب
\par 6 یهوه خدای ما، ما را در حوریب خطاب کرده، گفت: «توقف شما در این کوه بس شده است.
\par 7 پس توجه نموده، کوچ کنید و به کوهستان اموریان، و جمیع حوالی آن از عربه و کوهستان وهامون و جنوب و کناره دریا، یعنی زمین کنعانیان و لبنان تا نهر بزرگ که نهر فرات باشد، داخل شوید.
\par 8 اینک زمین را پیش روی شما گذاشتم. پس داخل شده، زمینی را که خداوند برای پدران شما، ابراهیم و اسحاق و یعقوب، قسم خورد که به ایشان و بعد از آنها به ذریت ایشان بدهد، به تصرف آورید.»
\par 9 و در آن وقت به شما متکلم شده، گفتم: «من به تنهایی نمی توانم متحمل شما باشم.
\par 10 یهوه خدای شما، شما را افزوده است و اینک شماامروز مثل ستارگان آسمان کثیر هستید.
\par 11 یهوه خدای پدران شما، شما را هزار چندان‌که هستیدبیفزاید و شما را برحسب آنچه به شما گفته است، برکت دهد.
\par 12 لیکن من چگونه به تنهایی متحمل محنت و بار و منازعت شما بشوم.
\par 13 پس مردان حکیم و عاقل و معروف از اسباط خود بیاورید، تاایشان را بر شما روسا سازم.»
\par 14 و شما در جواب من گفتید: «سخنی که گفتی نیکو است که بکنیم.»
\par 15 پس روسای اسباط شما را که مردان حکیم ومعروف بودند گرفته، ایشان را بر شما روساساختم، تا سروران هزاره‌ها و سروران صدها وسروران پنجاهها و سروران دهها و ناظران اسباطشما باشند.
\par 16 و در آنوقت داوران شما را امرکرده، گفتم: دعوای برادران خود را بشنوید، و درمیان هرکس و برادرش و غریبی که نزد وی باشدبه انصاف داوری نمایید.
\par 17 و در داوری طرف داری مکنید، کوچک را مثل بزرگ بشنویدو از روی انسان مترسید، زیرا که داوری از آن خداست، و هر دعوایی که برای شما مشکل است، نزد من بیاورید تا آن را بشنوم.
\par 18 و آن وقت همه‌چیزهایی را که باید بکنید، برای شما امر فرمودم.
\par 19 پس از حوریب کوچ کرده، از تمامی این بیابان بزرگ و ترسناک که شما دیدید به راه کوهستان اموریان رفتیم، چنانکه یهوه خدای ما به ما امر فرمود و به قادش برنیع رسیدیم.
\par 20 و به شما گفتم: «به کوهستان اموریانی که یهوه خدای ما به ما می‌دهد، رسیده‌اید.
\par 21 اینک یهوه خدای تو، این زمین را پیش روی تو گذاشته است، پس برآی و چنانکه یهوه خدای پدرانت به تو گفته است، آن را به تصرف آور و ترسان و هراسان مباش.»
\par 22 آنگاه جمیع شما نزد من آمده، گفتید: «مردان چند، پیش روی خود بفرستیم تا زمین رابرای ما جاسوسی نمایند، و ما را از راهی که بایدبرویم و از شهرهایی که به آنها می‌رویم، خبربیاورند.»
\par 23 و این سخن مرا پسند آمد، پس دوازده نفر از شما، یعنی یکی را از هر سبطگرفتم،
\par 24 و ایشان متوجه راه شده، به کوه برآمدند و به وادی اشکول رسیده، آن راجاسوسی نمودند.
\par 25 و از میوه زمین به‌دست خود گرفته، آن را نزد ما آوردند، و ما را مخبرساخته، گفتند: «زمینی که یهوه خدای ما، به مامی دهد، نیکوست.»
\par 26 لیکن شما نخواستید که بروید، بلکه ازفرمان خداوند عصیان ورزیدید.
\par 27 و درخیمه های خود همهمه کرده، گفتید: «چونکه خداوند ما را دشمن داشت، ما را از زمین مصربیرون آورد، تا ما را به‌دست اموریان تسلیم کرده، هلاک سازد.
\par 28 و حال کجا برویم چونکه برادران ما دل ما را گداخته، گفتند که این قوم از ما بزرگتر وبلندترند و شهرهای ایشان بزرگ و تا آسمان حصاردار است، و نیز بنی عناق را در آنجادیده‌ایم.»
\par 29 پس من به شما گفتم: «مترسید و ازایشان هراسان مباشید.
\par 30 یهوه خدای شما که پیش روی شما می‌رود برای شما جنگ خواهدکرد، برحسب هرآنچه به نظر شما در مصر برای شما کرده است.»
\par 31 و هم در بیابان که در آنجادیدید چگونه یهوه خدای تو مثل کسی‌که پسرخود را می‌برد تو را در تمامی راه که می‌رفتیدبرمی داشت تا به اینجا رسیدید.
\par 32 لیکن با وجوداین، همه شما به یهوه خدای خود ایمان نیاوردید.
\par 33 که پیش روی شما در راه می‌رفت تاجایی برای نزول شما بطلبد، وقت شب در آتش تا راهی را که به آن بروید به شما بنماید و وقت روز در ابر.
\par 34 و خداوند آواز سخنان شما را شنیده، غضبناک شد، و قسم خورده، گفت:
\par 35 «هیچکدام از این مردمان و از این طبقه شریر، آن زمین نیکو راکه قسم خوردم که به پدران شما بدهم، هرگزنخواهند دید.
\par 36 سوای کالیب بن یفنه که آن راخواهد دید و زمینی را که در آن رفته بود، به وی وبه پسرانش خواهم داد، چونکه خداوند را پیروی کامل نمود.»
\par 37 و خداوند بخاطر شما برمن نیزخشم نموده، گفت که «تو هم داخل آنجا نخواهی شد.
\par 38 یوشع بن نون که بحضور تو می‌ایستدداخل آنجا خواهد شد، پس او را قوی گردان زیرا اوست که آن را برای بنی‌اسرائیل تقسیم خواهد نمود.
\par 39 و اطفال شما که درباره آنها گفتید که به یغما خواهند رفت، و پسران شماکه امروز نیک و بد را تمیز نمی دهند، داخل آنجاخواهند شد، و آن را به ایشان خواهم داد تا مالک آن بشوند.
\par 40 و اما شما روگردانیده از راه بحرقلزم به بیابان کوچ کنید.»
\par 41 و شما در جواب من گفتید که «به خداوندگناه ورزیده‌ایم، پس رفته، جنگ خواهیم کرد، موافق هرآنچه یهوه خدای ما به ما امر فرموده است، و همه شما اسلحه جنگ خود را بسته، عزیمت کردید که به کوه برآیید.
\par 42 آنگاه خداوندبه من گفت: «به ایشان بگو که نروند و جنگ منمایند زیرا که من در میان شما نیستم، مبادا ازحضور دشمنان خود مغلوب شوید.»
\par 43 پس به شما گفتم، لیکن نشنیدید، بلکه از فرمان خداوندعصیان ورزیدید، و مغرور شده، به فراز کوه برآمدید.
\par 44 و اموریانی که در آن کوه ساکن بودندبه مقابله شما بیرون آمده، شما را تعاقب نمودند، بطوری که زنبورها می‌کنند و شما را از سعیر تاحرما شکست دادند.
\par 45 پس برگشته، به حضورخداوند گریه نمودید، اما خداوند آواز شما رانشنید و به شما گوش نداد.و در قادش برحسب ایام توقف خود، روزهای بسیار ماندید.
\par 46 و در قادش برحسب ایام توقف خود، روزهای بسیار ماندید.
 
\chapter{2}

\par 1 پس برگشته، چنانکه خداوند به من گفته بود، از راه بحرقلزم در بیابان کوچ کردیم وروزهای بسیار کوه سعیر را دور زدیم.
\par 2 پس خداوند مرا خطاب کرده، گفت:
\par 3 «دور زدن شمابه این کوه بس است، بسوی شمال برگردید.
\par 4 وقوم را امر فرموده، بگو که شما از حدود برادران خود بنی عیسو که در سعیر ساکنند باید بگذرید، و ایشان از شما خواهند ترسید، پس بسیاراحتیاط کنید.
\par 5 و با ایشان منازعه مکنید، زیرا که از زمین ایشان بقدر کف پایی هم به شما نخواهم داد، چونکه کوه سعیر را به عیسو به ملکیت داده‌ام.
\par 6 خوراک را از ایشان به نقره خریده، بخورید و آب را نیز از ایشان به نقره خریده، بنوشید.»
\par 7 زیرا که یهوه خدای تو، تو را در همه کارهای دستت برکت داده است، او راه رفتنت رادر این بیابان بزرگ می‌داند، الان چهل سال است که یهوه خدایت با تو بوده است و به هیچ‌چیزمحتاج نشده‌ای.
\par 8 پس از برادران خود بنی عیسوکه در سعیر ساکنند، از راه عربه از ایلت و عصیون جابر عبور نمودیم.
\par 9 پس برگشته، از راه بیابان موآب گذشتیم، وخداوند مرا گفت: «موآب را اذیت مرسان و باایشان منازعت و جنگ منما، زیرا که از زمین ایشان هیچ نصیبی به شما نخواهم داد، چونکه عاررا به بنی لوط برای ملکیت داده‌ام.»
\par 10 ایمیان که قوم عظیم و کثیر و بلند قد مثل عناقیان بودند، پیش در آنجا سکونت داشتند.
\par 11 ایشان نیز مثل عناقیان از رفائیان محسوب بودند، لیکن موآبیان ایشان را ایمیان می‌خوانند.
\par 12 و حوریان در سعیرپیشتر ساکن بودند، و بنی عیسو ایشان را اخراج نموده، ایشان را از پیش روی خود هلاک ساختند، و در جای ایشان ساکن شدند، چنانکه اسرائیل به زمین میراث خود که خداوند به ایشان داده بود، کردند.
\par 13 الان برخیزید و از وادی زاردعبور نمایید. پس از وادی زارد عبور نمودیم.
\par 14 وایامی که از قادش برنیع راه می‌رفتیم تا از وادی زارد عبور نمودیم سی و هشت سال بود، تاتمامی آن طبقه مردان جنگی از میان اردو تمام شدند، چنانکه خداوند برای ایشان قسم خورده بود.
\par 15 و دست خداوند نیز بر ایشان می‌بود تاایشان را از میان اردو بالکل هلاک کند.
\par 16 پس چون جمیع مردان جنگی از میان قوم بالکل مردند،
\par 17 آنگاه خداوند مرا خطاب کرده، گفت:
\par 18 «تو امروز از عار که سرحد موآب باشد، باید بگذری.
\par 19 و چون به مقابل بنی عمون برسی ایشان را مرنجان و با ایشان منازعه مکن، زیرا که اززمین بنی عمون نصیبی به تو نخواهم داد چونکه آن را به بنی لوط به ملکیت داده‌ام.
\par 20 (آن نیز زمین رفائیان شمرده می‌شود و رفائیان پیشتر ساکن آنجا بودند، لیکن عمونیان ایشان را زمزمیان می‌خوانند.
\par 21 ایشان قومی عظیم و کثیر و بلند قدمثل عناقیان بودند، و خداوند آنها را از پیش روی ایشان هلاک کرد، پس ایشان را اخراج نموده، درجای ایشان ساکن شدند.
\par 22 چنانکه برای بنی عیسو که در سعیر ساکنند عمل نموده، حوریان را از حضور ایشان هلاک ساخته، آنها رااخراج نمودند، و تا امروز در جای ایشان ساکنند.
\par 23 و عویان را که در دهات تا به غزا ساکن بودندکفتوریان که از کفتور بیرون آمدند هلاک ساخته، در جای ایشان ساکن شدند. )
\par 24 پس برخیزید وکوچ کرده، از وادی ارنون عبور کنید، اینک سیحون اموری ملک حشبون و زمین او را به‌دست تو دادم، به تصرف آن شروع کن و با ایشان جنگ نما.
\par 25 امروز شروع کرده، خوف و ترس تورا بر قومهای زیر تمام آسمان مستولی می‌گردانم، و ایشان آوازه تو را شنیده، خواهندلرزید، و از ترس تو مضطرب خواهند شد.»
\par 26 پس قاصدان با سخنان صلح‌آمیز از بیابان قدیموت نزد سیحون ملک حشبون فرستاده، گفتم:
\par 27 «اجازت بده که از زمین تو بگذرم، به شاهراه خواهم رفت و به طرف راست یا چپ میل نخواهم کرد.
\par 28 خوراک را به نقره به من بفروش تابخورم، و آب را به نقره به من بده تا بنوشم، فقطاجازت بده تا بر پایهای خود بگذرم.
\par 29 چنانکه بنی عیسو که در سعیر ساکنند و موآبیان که در عارساکنند به من رفتار نمودند، تا از اردن به زمینی که یهوه خدای ما به ما می‌دهد، عبور نمایم.»
\par 30 اماسیحون ملک حشبون نخواست که ما را از سرحدخود راه بدهد، زیرا که یهوه خدای تو روح او رابه قساوت و دل او را به سختی واگذاشت، تا او راچنانکه امروز شده است، به‌دست تو تسلیم نماید.
\par 31 و خداوند مرا گفت: «اینک به تسلیم نمودن سیحون و زمین او به‌دست تو شروع کردم، پس بنابه تصرف آن بنما تا زمین او را مالک شوی.»
\par 32 آنگاه سیحون با تمامی قوم خود به مقابله مابرای جنگ کردن در یاهص بیرون آمدند.
\par 33 ویهوه خدای ما او را به‌دست ما تسلیم نموده، او رابا پسرانش و جمیع قومش زدیم.
\par 34 و تمامی شهرهای او را در آنوقت گرفته، مردان و زنان واطفال هر شهر را هلاک کردیم که یکی را باقی نگذاشتیم.
\par 35 لیکن بهایم را با غنیمت شهرهایی که گرفته بودیم، برای خود به غارت بردیم.
\par 36 ازعروعیر که برکناره وادی ارنون است، و شهری که در وادی است، تا جلعاد قریه‌ای نبود که به ماممتنع باشد، یهوه خدای ما همه را به ما تسلیم نمود.لیکن به زمین بنی عمون و به تمامی کناره وادی یبوق و شهرهای کوهستان، و به هر جایی که یهوه خدای ما نهی فرموده بود، نزدیک نشدیم.
\par 37 لیکن به زمین بنی عمون و به تمامی کناره وادی یبوق و شهرهای کوهستان، و به هر جایی که یهوه خدای ما نهی فرموده بود، نزدیک نشدیم. 
 
\chapter{3}

\par 1 پس برگشته، به راه باشان رفتیم، و عوج ملک باشان با تمامی قوم خود به مقابله مابیرون آمده، در ادرعی جنگ کرد.
\par 2 و خداوند مراگفت: «از او مترس زیرا که او و تمامی قومش وزمینش را به‌دست تو تسلیم نموده‌ام، تا بطوری که با سیحون ملک اموریان که در حشبون ساکن بود، عمل نمودی، با وی نیز عمل نمایی.»
\par 3 پس یهوه، خدای ما، عوج ملک باشان را نیز و تمامی قومش را به‌دست ما تسلیم نموده، او را به حدی شکست دادیم که احدی از برای وی باقی نماند.
\par 4 و در آنوقت همه شهرهایش را گرفتیم، وشهری نماند که از ایشان نگرفتیم، یعنی شصت شهر و تمامی مرزبوم ارجوب که مملکت عوج در باشان بود.
\par 5 جمیع اینها شهرهای حصاردار بادیوارهای بلند و دروازه‌ها و پشت بندها بود، سوای قرای بی‌حصار بسیار کثیر.
\par 6 و آنها رابالکل هلاک کردیم، چنانکه با سیحون، ملک حشبون کرده بودیم، هر شهر را با مردان و زنان واطفال هلاک ساختیم.
\par 7 و تمامی بهایم و غنیمت شهرها را برای خود به غارت بردیم.
\par 8 و در آن وقت زمین را از دست دو ملک اموریان که به آن طرف اردن بودند، از وادی ارنون تا جبل حرمون، گرفتیم.
\par 9 (و این حرمون راصیدونیان سریون می‌خوانند و اموریان آن راسنیر می‌خوانند. )
\par 10 تمام شهرهای هامون وتمامی جلعاد و تمامی باشان تا سلخه و ادرعی که شهرهای مملکت عوج در باشان بود.
\par 11 زیرا که عوج ملک باشان از بقیه رفائیان تنها باقی‌مانده بود. اینک تخت خواب او تخت آهنین است آیاآن در ربت بنی عمون نیست. و طولش نه ذراع وعرضش چهار ذراع برحسب ذراع آدمی می‌باشد.
\par 12 و این زمین را در آن وقت به تصرف آوردیم، و آن را از عروعیر که برکنار وادی ارنون است و نصف کوهستان جلعاد؛ و شهرهایش را به روبینیان و جادیان دادم.
\par 13 و بقیه جلعاد و تمامی باشان را که مملکت عوج باشد به نصف سبطمنسی دادم، یعنی تمامی مرزبوم ارجوب را باتمامی باشان که زمین رفائیان نامیده می‌شود.
\par 14 یائیر بن منسی تمامی مرزبوم ارجوب را تا حدجشوریان و معکیان گرفت، و آنها را تا امروز به اسم خود باشان، حووت یائیر نامید.
\par 15 و جلعادرا به ماکیر دادم.
\par 16 و به روبینیان و جادیان، ازجلعاد تا وادی ارنون، هم وسط وادی و هم کناره‌اش تا وادی یبوق را که حد بنی عمون باشد، دادم.
\par 17 و عربه را نیز و اردن و کناره‌اش را ازکنرت تا دریای عربه که بحرالملح باشد، زیردامنه های فسجه به طرف مشرق دادم.
\par 18 و در آن وقت به شما امر فرموده، گفتم: «یهوه خدای شما این زمین را به شما داده است تاآن را به تصرف آورید، پس جمیع مردان جنگی شما مهیا شده، پیش روی برادران خود، بنی‌اسرائیل، عبور کنید.
\par 19 لیکن زنان و اطفال ومواشی شما، چونکه می‌دانم مواشی بسیار دارید، در شهرهای شما که به شما دادم، بمانند.
\par 20 تاخداوند به برادران شما مثل شما آرامی دهد، وایشان نیز زمینی را که یهوه خدای شما به آنطرف اردن به ایشان می‌دهد، به تصرف آورند، آنگاه هریکی از شما به ملک خود که به شما دادم، برگردید.»
\par 21 و در آن وقت یوشع را امر فرموده، گفتم: «هرآنچه یهوه، خدای شما، به این دوپادشاه کرده است، چشمان تو دید، پس خداوندبا تمامی ممالکی که بسوی آنها عبور می‌کنی، چنین خواهد کرد.
\par 22 از ایشان مترسید زیرا که یهوه خدای شماست که برای شما جنگ می‌کند.»
\par 23 و در آنوقت نزد خداوند استغاثه کرده، گفتم:
\par 24 «ای خداوند یهوه تو به نشان دادن عظمت و دست قوی خود به بنده ات شروع کرده‌ای، زیرا کدام خداست در آسمان یا در زمین که مثل اعمال و جبروت تو می‌تواند عمل نماید.
\par 25 تمنا اینکه عبور نمایم و زمین نیکو را که به آنطرف اردن است و این کوه نیکو و لبنان را ببینم.»
\par 26 لیکن خداوند بخاطر شما با من غضبناک شده، مرا اجابت ننمود و خداوند مرا گفت: «تو را کافی است. بار دیگر درباره این امر با من سخن مگو.
\par 27 به قله فسجه برآی و چشمان خود را به طرف مغرب و شمال و جنوب و مشرق بلند کرده، به چشمان خود ببین، زیرا که از این اردن نخواهی گذشت.
\par 28 اما یوشع را امر فرموده، او را دلیر وقوی گردان، زیرا که او پیش این قوم عبور نموده، زمینی را که تو خواهی دید، برای ایشان تقسیم خواهد نمود.»پس در دره، در برابر بیت فغورتوقف نمودیم.
\par 29 پس در دره، در برابر بیت فغورتوقف نمودیم.
 
\chapter{4}

\par 1 پس الان‌ای اسرائیل، فرایض و احکامی راکه من به شما تعلیم می‌دهم تا آنها را بجاآورید بشنوید، تا زنده بمانید و داخل شده، زمینی را که یهوه، خدای پدران شما، به شمامی دهد به تصرف آورید.
\par 2 بر کلامی که من به شماامر می‌فرمایم چیزی میفزایید و چیزی از آن کم منمایید، تا اوامر یهوه خدای خود را که به شماامر می‌فرمایم، نگاه دارید.
\par 3 چشمان شما آنچه راخداوند در بعل فغور کرد دید، زیرا هرکه پیروی بعل فغور کرد، یهوه خدای تو، او را از میان توهلاک ساخت.
\par 4 اما جمیع شما که به یهوه خدای خود ملصق شدید، امروز زنده ماندید.
\par 5 اینک چنانکه یهوه، خدایم، مرا امر فرموده است، فرایض و احکام به شما تعلیم نمودم، تا درزمینی که شما داخل آن شده، به تصرف می‌آورید، چنان عمل نمایید.
\par 6 پس آنها را نگاه داشته، بجا آورید زیرا که این حکمت و فطانت شماست، در نظر قومهایی که چون این فرایض رابشنوند، خواهند گفت: «هرآینه این طایفه‌ای بزرگ، قوم حکیم، و فطانت پیشه‌اند.»
\par 7 زیرا کدام قوم بزرگ است که خدا نزدیک ایشان باشدچنانکه یهوه، خدای ما است، در هروقت که نزداو دعا می‌کنیم؟
\par 8 و کدام طایفه بزرگ است که فرایض و احکام عادله‌ای مثل تمام این شریعتی که من امروز پیش شما می‌گذارم، دارند؟
\par 9 لیکن احتراز نما و خویشتن را بسیار متوجه باش، مبادا این چیزهایی را که چشمانت دیده است فراموش کنی و مبادا اینها در تمامی ایام عمرت از دل تو محو شود، بلکه آنها را به پسرانت و پسران پسرانت تعلیم ده.
\par 10 در روزی که درحضور یهوه خدای خود در حوریب ایستاده بودی و خداوند به من گفت: «قوم را نزد من جمع کن تا کلمات خود را به ایشان بشنوانم، تا بیاموزندکه در تمامی روزهایی که برروی زمین زنده باشنداز من بترسند، و پسران خود را تعلیم دهند.»
\par 11 وشما نزدیک آمده، زیر کوه ایستادید، و کوه تا به وسط آسمان به آتش و تاریکی و ابرها و ظلمت غلیظ می‌سوخت.
\par 12 و خداوند با شما از میان آتش متکلم شد، و شما آواز کلمات را شنیدید، لیکن صورتی ندیدید، بلکه فقط آواز را شنیدید.
\par 13 و عهد خود را که شما را به نگاه داشتن آن مامور فرمود، برای شما بیان کرد، یعنی ده کلمه را و آنها را بر دو لوح سنگ نوشت.
\par 14 و خداوندمرا در آنوقت امر فرمود که فرایض و احکام را به شما تعلیم دهم، تا آنها را در زمینی که برای تصرفش به آن عبور می‌کنید، بجا آورید.
\par 15 پس خویشتن را بسیار متوجه باشید، زیرادر روزی که خداوند با شما در حوریب از میان آتش تکلم می‌نمود، هیچ صورتی ندیدید.
\par 16 مبادا فاسد شوید و برای خود صورت تراشیده، یا تمثال هر شکلی از شبیه ذکور یا اناث بسازید.
\par 17 یا شبیه هر بهیمه‌ای که بر روی زمین است، یا شبیه هر مرغ بالدار که در آسمان می‌پرد.
\par 18 یا شبیه هر خزنده‌ای بر زمین یا شبیه هرماهی‌ای که در آبهای زیر زمین است.
\par 19 و مباداچشمان خود را بسوی آسمان بلند کنی، و آفتاب و ماه و ستارگان و جمیع جنود آسمان را دیده، فریفته شوی و سجده کرده، آنها را که یهوه خدایت برای تمامی قومهایی که زیر تمام آسمانند، تقسیم کرده است، عبادت نمایی.
\par 20 لیکن خداوند شما را گرفته، از کوره آهن ازمصر بیرون آورد تا برای او قوم میراث باشید، چنانکه امروز هستید.
\par 21 و خداوند بخاطر شما بر من غضبناک شده، قسم خورد که از اردن عبور نکنم و به آن زمین نیکو که یهوه خدایت به تو برای ملکیت می‌دهد، داخل نشوم.
\par 22 بلکه من در این زمین خواهم مرد واز اردن عبور نخواهم کرد، لیکن شما عبورخواهید کرد، و آن زمین نیکو را به تصرف خواهیدآورد.
\par 23 پس احتیاط نمایید، مبادا عهد یهوه، خدای خود را که با شما بسته است فراموش نمایید، و صورت تراشیده یا شبیه هر چیزی که یهوه خدایت به تو نهی کرده است، برای خودبسازی.
\par 24 زیرا که یهوه خدایت آتش سوزنده وخدای غیور است.
\par 25 چون پسران و پسران پسران را تولیدنموده، و در زمین مدتی ساکن باشید، اگر فاسدشده، صورت تراشیده، و شبیه هرچیزی رابسازید و آنچه در نظر یهوه خدای شما بد است بجا آورده، او را غضبناک سازید،
\par 26 آسمان وزمین را امروز بر شما شاهد می‌آورم که از آن زمینی که برای تصرف آن از اردن بسوی آن عبورمی کنید البته هلاک خواهید شد. و روزهای خودرا در آن طویل نخواهید ساخت، بلکه بالکل هلاک خواهید شد.
\par 27 و خداوند شما را در میان قومها پراکنده خواهد نمود، و شما در میان طوایفی که خداوند شما را به آنجا می‌برد، قلیل العدد خواهید ماند.
\par 28 و در آنجا خدایان ساخته شده دست انسان از چوپ و سنگ راعبادت خواهید کرد، که نمی بینند و نمی شنوند ونمی خورند و نمی بویند.
\par 29 لیکن اگر از آنجا یهوه خدای خود را بطلبی، او را خواهی یافت. بشرطی که او را به تمامی دل و به تمامی جان خود تفحص نمایی.
\par 30 چون در تنگی گرفتار شوی، و جمیع این وقایع برتو عارض شود، در ایام آخر بسوی یهوه خدای خود برگشته، آواز او را خواهی شنید.
\par 31 زیرا که یهوه خدای تو خدای رحیم است، تو را ترک نخواهد کرد و تو را هلاک نخواهد نمود، و عهد پدرانت را که برای ایشان قسم خورده بود، فراموش نخواهد کرد.
\par 32 زیرا که از ایام پیشین که قبل از تو بوده است از روزی که خدا آدم را برزمین آفرید، و از یک کناره آسمان تا به کناره دیگر بپرس که آیا مثل این امر عظیم واقع شده یا مثل این شنیده شده است؟
\par 33 آیا قومی هرگز آواز خدا را که از میان آتش متکلم شود، شنیده باشند و زنده بمانند، چنانکه تو شنیدی؟
\par 34 و آیا خدا عزیمت کرد که برود وقومی برای خود از میان قوم دیگر بگیرد باتجربه‌ها و آیات و معجزات و جنگ و دست قوی و بازوی دراز شده و ترسهای عظیم، موافق هرآنچه یهوه خدای شما برای شما در مصر درنظر شما بعمل آورد؟
\par 35 این برتو ظاهر شد تابدانی که یهوه خداست و غیر از او دیگری نیست.
\par 36 از آسمان آواز خود را به تو شنوانید تا تو راتادیب نماید، و برزمین آتش عظیم خود را به تونشان داد و کلام او را از میان آتش شنیدی.
\par 37 و ازاین جهت که پدران تو را دوست داشته، ذریت ایشان را بعد از ایشان برگزیده بود، تو را به حضرت خود با قوت عظیم از مصر بیرون آورد.
\par 38 تا امتهای بزرگتر و عظیم تر از تو را پیش روی تو بیرون نماید و تو را درآورده، زمین ایشان رابرای ملکیت به تو دهد، چنانکه امروز شده است.
\par 39 لهذا امروز بدان و دردل خود نگاه دار که یهوه خداست، بالا در آسمان و پایین برروی زمین ودیگری نیست.
\par 40 و فرایض و اوامر او را که من امروز به تو امر می‌فرمایم نگاهدار، تا تو را و بعداز تو فرزندان تو را نیکو باشد و تا روزهای خود رابر زمینی که یهوه خدایت به تو می‌دهد تا به ابدطویل نمایی.
\par 41 آنگاه موسی سه شهر به آن طرف اردن بسوی مشرق آفتاب جدا کرد.
\par 42 تا قاتلی که همسایه خود را نادانسته کشته باشد و پیشتر باوی بغض نداشته به آنها فرار کند، و به یکی از این شهرها فرار کرده، زنده ماند.
\par 43 یعنی باصر دربیابان، در زمین همواری به جهت روبینیان، وراموت در جلعاد به جهت جادیان، و جولان درباشان به جهت منسیان.
\par 44 و این است شریعتی که موسی پیش روی بنی‌اسرائیل نهاد.
\par 45 این است شهادات و فرایض و احکامی که موسی به بنی‌اسرائیل گفت، وقتی که ایشان از مصر بیرون آمدند.
\par 46 به آنطرف اردن در دره مقابل بیت فغور در زمین سیحون، ملک اموریان که در حشبون ساکن بود، و موسی وبنی‌اسرائیل چون از مصر بیرون آمده بودند او رامغلوب ساختند.
\par 47 و زمین او را و زمین عوج ملک باشان را، دو ملک اموریانی که به آنطرف اردن بسوی مشرق آفتاب بودند، به تصرف آوردند. 
\par 48 از عروعیر که بر کناره وادی ارنون است تا جبل سیئون که حرمون باشد.و تمامی عربه به آنطرف اردن بسوی مشرق تا دریای عربه زیر دامنه های فسجه.
\par 49 و تمامی عربه به آنطرف اردن بسوی مشرق تا دریای عربه زیر دامنه های فسجه.
 
\chapter{5}

\par 1 و موسی تمامی بنی‌اسرائیل را خوانده، به ایشان گفت: ای اسرائیل فرایض و احکامی را که من امروز به گوش شما می‌گویم بشنوید، تاآنها را یاد گرفته، متوجه باشید که آنها را بجاآورید.
\par 2 یهوه خدای ما با ما در حوریب عهدبست.
\par 3 خداوند این عهد را با پدران ما نبست، بلکه با ما که جمیع امروز در اینجا زنده هستیم.
\par 4 خداوند در کوه از میان آتش با شما روبرومتکلم شد.
\par 5 (من در آن وقت میان خداوند و شماایستاده بودم، تا کلام خداوند را برای شما بیان کنم، زیرا که شما به‌سبب آتش می‌ترسیدید و به فراز کوه برنیامدید) و گفت:
\par 6 «من هستم یهوه، خدای تو، که تو را از زمین مصر از خانه بندگی بیرون آوردم.
\par 7 تو را به حضور من خدایان دیگر نباشند.
\par 8 «به جهت خود صورت تراشیده یا هیچ تمثالی از آنچه بالا در آسمان، یا از آنچه پایین درزمین، یا از آنچه در آبهای زیر زمین است مساز.
\par 9 آنها را سجده و عبادت منما. زیرا من که یهوه خدای تو هستم، خدای غیورم، و گناه پدران را برپسران تا پشت سوم و چهارم از آنانی که مرادشمن دارند، می‌رسانم.
\par 10 و رحمت می‌کنم تاهزار پشت برآنانی که مرا دوست دارند و احکام مرا نگاه دارند.
\par 11 «نام یهوه خدای خود را به باطل مبر، زیراخداوند کسی را که نام او را به باطل برد، بی‌گناه نخواهد شمرد.
\par 12 «روز سبت را نگاه دار و آن را تقدیس نما، چنانکه یهوه خدایت به تو امر فرموده است.
\par 13 شش روز مشغول باش و هرکار خود را بکن.
\par 14 اما روز هفتمین سبت یهوه خدای توست، درآن هیچکاری مکن، تو و پسرت و دخترت وغلامت و کنیزت و گاوت و الاغت و همه بهایمت و مهمانت که در اندرون دروازه های تو باشد، تاغلامت و کنیزت مثل تو آرام گیرند.
\par 15 و بیاد آورکه در زمین مصر غلام بودی، و یهوه خدایت تو رابه‌دست قوی و بازوی دراز از آنجا بیرون آورد، بنابراین یهوه، خدایت، تو را امر فرموده است که روز سبت را نگاه داری.
\par 16 «پدر و مادر خود را حرمت دار، چنانکه یهوه خدایت تو را امر فرموده است، تا روزهایت دراز شود و تو را در زمینی که یهوه خدایت به تومی بخشد، نیکویی باشد.
\par 17 «قتل مکن.
\par 18 و زنا مکن.
\par 19 و دزدی مکن.
\par 20 و بر همسایه خود شهادت دروغ مده.
\par 21 وبرزن هسمایه ات طمع مورز، و به خانه همسایه ات و به مزرعه او و به غلامش و کنیزش وگاوش و الاغش و به هرچه از آن همسایه توباشد، طمع مکن.»
\par 22 این سخنان را خداوند به تمامی جماعت شما در کوه از میان آتش و ابر و ظلمت غلیظ به آواز بلند گفت، و بر آنها چیزی نیفزود و آنها را بردو لوح سنگ نوشته، به من داد.
\par 23 و چون شما آن آواز را از میان تاریکی شنیدید، و کوه به آتش می‌سوخت، شما با جمیع روسای اسباط ومشایخ خود نزد من آمده،
\par 24 گفتید: اینک یهوه، خدای ما، جلال و عظمت خود را بر ما ظاهرکرده است، و آواز او را از میان آتش شنیدیم، پس امروز دیدیم که خدا با انسان سخن می‌گوید وزنده است.
\par 25 و اما الان چرا بمیریم زیرا که این آتش عظیم ما را خواهد سوخت، اگر آواز یهوه خدای خود را دیگر بشنویم، خواهیم مرد.
\par 26 زیرا از تمامی بشر کیست که مثل ما آوازخدای حی را که از میان آتش سخن گوید، بشنودو زنده ماند؟
\par 27 تو نزدیک برو و هرآنچه یهوه خدای ما بگوید، بشنو و هرآنچه یهوه خدای مابه تو بگوید برای ما بیان کن، پس خواهیم شنید وبه عمل خواهیم آورد.
\par 28 و خداوند آواز سخنان شما را که به من گفتید شنید، و خداوند مرا گفت: «آواز سخنان این قوم را که به تو گفتند، شنیدم؛ هرچه گفتندنیکو گفتند.
\par 29 کاش که دلی را مثل این داشتند تااز من می‌ترسیدند، و تمامی اوامر مرا در هر وقت بجا می‌آوردند، تا ایشان را و فرزندان ایشان را تابه ابد نیکو باشد.
\par 30 برو و ایشان را بگو به خیمه های خود برگردید.
\par 31 و اما تو در اینجاپیش من بایست، تا جمیع اوامر و فرایض واحکامی را که می‌باید به ایشان تعلیم دهی به توبگویم، و آنها را در زمینی که من به ایشان می‌دهم تا در آن تصرف نمایید، بجا آورند.»
\par 32 «پس توجه نمایید تا آنچه یهوه، خدای شما، به شما امر فرموده است، به عمل آورید، وبه راست و چپ انحراف منمایید.در تمامی آن طریقی که یهوه، خدای شما، به شما امرفرموده است، سلوک نمایید، تا برای شما نیکوباشد و ایام خود را در زمینی که به تصرف خواهیدآورد، طویل نمایید.
\par 33 در تمامی آن طریقی که یهوه، خدای شما، به شما امرفرموده است، سلوک نمایید، تا برای شما نیکوباشد و ایام خود را در زمینی که به تصرف خواهیدآورد، طویل نمایید.
 
\chapter{6}

\par 1 و این است اوامر و فرایض و احکامی که یهوه، خدای شما، امر فرمود که به شماتعلیم داده شود، تا آنها را در زمینی که شما بسوی آن برای تصرفش عبور می‌کنید، بجا آورید.
\par 2 و تااز یهوه خدای خود ترسان شده، جمیع فرایض واوامر او را که من به شما امر می‌فرمایم نگاه داری، تو و پسرت و پسر پسرت، در تمامی ایام عمرت و تا عمر تو دراز شود.
\par 3 پس‌ای اسرائیل بشنو، و به عمل نمودن آن متوجه باش، تا برای تو نیکوباشد، و بسیار افزوده شوی در زمینی که به شیر وشهد جاری است، چنانکه یهوه خدای پدرانت تورا وعده داده است.
\par 4 ‌ای اسرائیل بشنو، یهوه، خدای ما، یهوه واحد است.
\par 5 پس یهوه خدای خود را به تمامی جان و تمامی قوت خود محبت نما.
\par 6 و این سخنانی که من امروز تو را امر می‌فرمایم، بر دل توباشد.
\par 7 و آنها را به پسرانت به دقت تعلیم نما، وحین نشستنت در خانه، و رفتنت به راه، و وقت خوابیدن و برخاستنت از آنها گفتگو نما.
\par 8 و آنهارا بردست خود برای علامت ببند، و در میان چشمانت عصابه باشد.
\par 9 و آنها را بر باهوهای درخانه ات و بر دروازه هایت بنویس.
\par 10 و چون یهوه، خدایت، تو را به زمینی که برای پدرانت ابراهیم و اسحاق و یعقوب قسم خورد که به تو بدهد، درآورد، به شهرهای بزرگ وخوشنمایی که تو بنا نکرده‌ای،
\par 11 و به خانه های پر از هر چیز نیکو که پر نکرده‌ای، و حوضهای کنده شده‌ای که نکنده‌ای، و تاکستانها و باغهای زیتونی که غرس ننموده‌ای، و از آنها خورده، سیرشدی.
\par 12 آنگاه با حذر باش مبادا خداوند را که تورا از زمین مصر، از خانه بندگی بیرون آورد، فراموش کنی.
\par 13 از یهوه خدای خود بترس و اورا عبادت نما و به نام او قسم بخور.
\par 14 خدایان دیگر را از خدایان طوایفی که به اطراف تومی باشند، پیروی منمایید.
\par 15 زیرا یهوه خدای تودر میان تو خدای غیور است، مبادا غضب یهوه، خدایت، برتو افروخته شود، و تو را از روی زمین هلاک سازد.
\par 16 یهوه خدای خود را میازمایید، چنانکه او رادر مسا آزمودید.
\par 17 توجه نمایید تا اوامر یهوه خدای خود را و شهادات و فرایض او را که به شما امر فرموده است، نگاه دارید.
\par 18 و آنچه درنظر خداوند راست و نیکوست، به عمل آور تابرای تو نیکو شود، و داخل شده آن زمین نیکو راکه خداوند برای پدرانت قسم خورد به تصرف آوری.
\par 19 و تا جمیع دشمنانت را از حضورت اخراج نماید، چنانکه خداوند گفته است.
\par 20 چون پسر تو در ایام آینده از تو سوال نموده، گوید که مراد از این شهادات و فرایض واحکامی که یهوه خدای ما به شما امر فرموده است، چیست؟
\par 21 پس به پسر خود بگو: ما درمصر غلام فرعون بودیم، و خداوند ما را از مصر بادست قوی بیرون آورد.
\par 22 و خداوند آیات ومعجزات عظیم و ردی‌ای بر مصر و فرعون وتمامی اهل خانه او در نظر ما ظاهر ساخت.
\par 23 وما را از آنجا بیرون آورد تا ما را به زمینی که برای پدران ما قسم خورد که به ما بدهد، درآورد.
\par 24 وخداوند ما را مامور داشت که تمام این فرایض رابجا آورده، از یهوه خدای خود بترسیم، تا برای ماهمیشه نیکو باشد و ما را زنده نگاه دارد، چنانکه تاامروز شده است.و برای ما عدالت خواهد بودکه متوجه شویم که جمیع این اوامر را به حضوریهوه خدای خود بجا آوریم، چنانکه ما را امرفرموده است.
\par 25 و برای ما عدالت خواهد بودکه متوجه شویم که جمیع این اوامر را به حضوریهوه خدای خود بجا آوریم، چنانکه ما را امرفرموده است.
 
\chapter{7}

\par 1 چون یهوه، خدایت، تو را به زمینی که برای تصرفش به آنجا می‌روی درآورد، وامتهای بسیار را که حتیان و جرجاشیان و اموریان و کنعانیان و فرزیان و حویان و یبوسیان، هفت امت بزرگتر و عظیم تر از تو باشند، از پیش تواخراج نماید.
\par 2 و چون یهوه خدایت، ایشان را به‌دست تو تسلیم نماید، و تو ایشان را مغلوب سازی، آنگاه ایشان را بالکل هلاک کن، و با ایشان عهد مبند و بر ایشان ترحم منما.
\par 3 و با ایشان مصاهرت منما؛ دختر خود را به پسر ایشان مده، و دختر ایشان را برای پسر خود مگیر.
\par 4 زیراکه اولاد تو را از متابعت من برخواهند گردانید، تاخدایان غیر را عبادت نمایند، و غضب خداوندبرشما افروخته شده، شما را بزودی هلاک خواهد ساخت.
\par 5 بلکه با ایشان چنین عمل نمایید؛ مذبحهای ایشان را منهدم سازید، وتمثالهای ایشان را بشکنید و اشیریم ایشان را قطع نمایید، و بتهای تراشیده ایشان را به آتش بسوزانید.
\par 6 زیرا که تو برای یهوه، خدایت، قوم مقدس هستی. یهوه خدایت تو را برگزیده است تااز جمیع قومهایی که بر روی زمین‌اند، قوم مخصوص برای خود او باشی.
\par 7 خداوند دل خود را با شما نبست و شما رابرنگزید از این سبب که از سایر قومها کثیرتربودید، زیرا که شما از همه قومها قلیلتر بودید.
\par 8 لیکن از این جهت که خداوند شما را دوست می‌داشت، و می‌خواست قسم خود را که برای پدران شما خورده بود، بجا آورد، پس خداوند شما را با دست قوی بیرون آورد، و از خانه بندگی از دست فرعون، پادشاه مصر، فدیه داد.
\par 9 پس بدان که یهوه، خدای تو، اوست خدا، خدای امین که عهد و رحمت خود را با آنانی که او را دوست می‌دارند و اوامر او را بجا می‌آورند تا هزار پشت نگاه می‌دارد.
\par 10 و آنانی را که او را دشمن دارند، بر روی ایشان مکافات رسانیده، ایشان را هلاک می‌سازد. و به هرکه او را دشمن دارد، تاخیرننموده، او را بر رویش مکافات خواهد رسانید.
\par 11 پس اوامر و فرایض و احکامی را که من امروزبه جهت عمل نمودن به تو امر می‌فرمایم نگاه دار.
\par 12 پس اگر این احکام را بشنوید و آنها را نگاه داشته، بجا آورید، آنگاه یهوه خدایت عهد ورحمت را که برای پدرانت قسم خورده است، باتو نگاه خواهد داشت.
\par 13 و تو را دوست داشته، برکت خواهد داد، و خواهد افزود، و میوه بطن توو میوه زمین تو را و غله و شیره و روغن تو را ونتاج رمه تو را و بچه های گله تو را، در زمینی که برای پدرانت قسم خورد که به تو بدهد، برکت خواهد داد.
\par 14 از همه قومها مبارک تر خواهی شد، و در میان شما و بهایم شما، نر یا ماده، نازادنخواهد بود.
\par 15 و خداوند هر بیماری را از تو دورخواهد کرد، و هیچکدام از مرضهای بد مصر راکه می‌دانی، برتو عارض نخواهد گردانید، بلکه برتمامی دشمنانت آنها را خواهد آورد.
\par 16 وتمامی قومها را که یهوه بدست تو تسلیم می‌کندهلاک ساخته، چشم تو برآنها ترحم ننماید، وخدایان ایشان را عبادت منما، مبادا برای تو دام باشد.
\par 17 و اگر در دلت گویی که این قومها از من زیاده‌اند، چگونه توانم ایشان را اخراج نمایم؟
\par 18 از ایشان مترس بلکه آنچه را یهوه خدایت بافرعون و جمیع مصریان کرد، نیکو بیاد آور.
\par 19 یعنی تجربه های عظیمی را که چشمانت دیده است، و آیات و معجزات و دست قوی و بازوی دراز را که یهوه، خدایت، تو را به آنها بیرون آورد. پس یهوه، خدایت، با همه قومهایی که از آنهامی ترسی، چنین خواهد کرد.
\par 20 و یهوه خدایت نیز زنبورها در میان ایشان خواهد فرستاد، تاباقی ماندگان و پنهان شدگان ایشان از حضور توهلاک شوند. 
\par 21 از ایشان مترس زیرا یهوه خدایت که در میان توست، خدای عظیم و مهیب است.
\par 22 و یهوه، خدایت، این قومها را از حضورتو به تدریج اخراج خواهد نمود، ایشان را بزودی نمی توانی تلف نمایی مبادا وحوش صحرا برتوزیاد شوند.
\par 23 لیکن یهوه خدایت، ایشان را به‌دست تو تسلیم خواهد کرد، و ایشان را به اضطراب عظیمی پریشان خواهد نمود تا هلاک شوند.
\par 24 و ملوک ایشان را بدست تو تسلیم خواهد نمود، تا نام ایشان را از زیر آسمان محوسازی، و کسی یارای مقاومت با تو نخواهدداشت تا ایشان را هلاک سازی.
\par 25 و تمثالهای خدایان ایشان را به آتش بسوزانید، به نقره وطلایی که بر آنهاست، طمع مورز، و برای خودمگیر، مبادا از آنها به دام گرفتار شوی، چونکه نزد یهوه، خدای تو، مکروه است.و چیزمکروه را به خانه خود میاور، مبادا مثل آن حرام شوی، از آن نهایت نفرت و کراهت دار چونکه حرام است.
\par 26 و چیزمکروه را به خانه خود میاور، مبادا مثل آن حرام شوی، از آن نهایت نفرت و کراهت دار چونکه حرام است.
 
\chapter{8}

\par 1 تمامی اوامری را که من امروز به شما امرمی فرمایم، حفظ داشته، بجا آورید، تا زنده مانده، زیاد شوید، و به زمینی که خداوند برای پدران شما قسم خورده بود، داخل شده، در آن تصرف نمایید.
\par 2 و بیاد آور تمامی راه را که یهوه، خدایت، تو را این چهل سال در بیابان رهبری نمود تا تو را ذلیل ساخته، بیازماید، و آنچه را که در دل تو است بداند، که آیا اوامر او را نگاه خواهی داشت یا نه.
\par 3 و او تو را ذلیل و گرسنه ساخت و من را به تو خورانید که نه تو آن رامی دانستی و نه پدرانت می‌دانستند، تا تو رابیاموزاند که انسان نه به نان تنها زیست می‌کندبلکه به هر کلمه‌ای که از دهان خداوند صادرشود، انسان زنده می‌شود.
\par 4 در این چهل سال لباس تو در برت مندرس نشد، و پای تو آماس نکرد.
\par 5 پس در دل خود فکر کن که بطوری که پدر، پسر خود را تادیب می‌نماید، یهوه خدایت تو را تادیب کرده است.
\par 6 و اوامر یهوه خدای خود را نگاه داشته، در طریقهای او سلوک نما و ازاو بترس.
\par 7 زیرا که یهوه خدایت تو را به زمین نیکو درمی آورد؛ زمین پر از نهرهای آب و ازچشمه‌ها و دریاچه‌ها که از دره‌ها و کوهها جاری می‌شود.
\par 8 زمینی که پر از گندم و جو و مو و انجیرو انار باشد، زمینی که پر از زیتون زیت و عسل است.
\par 9 زمینی که در آن نان را به تنگی نخواهی خورد، و در آن محتاج به هیچ‌چیز نخواهی شد، زمینی که سنگهایش آهن است، و از کوههایش مس خواهی کند.
\par 10 و خورده، سیر خواهی شد، و یهوه خدای خود را به جهت زمین نیکو که به تو داده است، متبارک خواهی خواند.
\par 11 پس باحذر باش، مبادا یهوه خدای خود رافراموش کنی و اوامر و احکام و فرایض او را که من امروز به تو امر می‌فرمایم، نگاه نداری.
\par 12 مباداخورده، سیر شوی، و خانه های نیکو بنا کرده، درآن ساکن شوی.
\par 13 و رمه و گله تو زیاد شود، ونقره و طلا برای تو افزون شود، و مایملک توافزوده گردد.
\par 14 و دل تو مغرور شده، یهوه خدای خود را که تو را از زمین مصر از خانه بندگی بیرون آورد، فراموش کنی.
\par 15 که تو را در بیابان بزرگ وخوفناک که در آن مارهای آتشین و عقربها و زمین تشنه بی‌آب بود، رهبری نمود، که برای تو آب ازسنگ خارا بیرون آورد.
\par 16 که تو را در بیابان من راخورانید که پدرانت آن را ندانسته بودند، تا تو راذلیل سازد و تو را بیازماید و بر تو در آخرت احسان نماید.
\par 17 مبادا در دل خود بگویی که قوت من و توانایی دست من، این توانگری را ازبرایم پیدا کرده است.
\par 18 بلکه یهوه خدای خود رابیاد آور، زیرا اوست که به تو قوت می‌دهد تاتوانگری پیدا نمایی، تا عهد خود را که برای پدرانت قسم خورده بود، استوار بدارد، چنانکه امروز شده است.
\par 19 و اگر یهوه خدای خود رافراموش کنی و پیروی خدایان دیگر نموده، آنهارا عبادت و سجده نمایی، امروز برشما شهادت می‌دهم که البته هلاک خواهید شد.مثل قومهایی که خداوند پیش روی تو هلاک می‌سازد، شما همچنین هلاک خواهید شد از این جهت که قول یهوه خدای خود را نشنیدید.
\par 20 مثل قومهایی که خداوند پیش روی تو هلاک می‌سازد، شما همچنین هلاک خواهید شد از این جهت که قول یهوه خدای خود را نشنیدید.
 
\chapter{9}

\par 1 ای اسرائیل بشنو. تو امروز از اردن عبورمی کنی، تا داخل شده، قومهایی را که از تو عظیم تر و قوی تراند، و شهرهای بزرگ را که تا به فلک حصاردار است، به تصرف آوری،
\par 2 یعنی قوم عظیم و بلند قد بنی عناق را که می‌شناسی وشنیده‌ای که گفته‌اند کیست که یارای مقاومت بابنی عناق داشته باشد.
\par 3 پس امروز بدان که یهوه، خدایت، اوست که پیش روی تو مثل آتش سوزنده عبور می‌کند، و او ایشان را هلاک خواهدکرد، و پیش روی تو ذلیل خواهد ساخت، پس ایشان را اخراج نموده، بزودی هلاک خواهی نمود، چنانکه خداوند به تو گفته است.
\par 4 پس چون یهوه، خدایت، ایشان را از حضورتو اخراج نماید، در دل خود فکر مکن و مگو که به‌سبب عدالت من، خداوند مرا به این زمین درآوردتا آن را به تصرف آورم، بلکه به‌سبب شرارت این امتها، خداوند ایشان را از حضور تو اخراج می‌نماید.
\par 5 نه به‌سبب عدالت خود و نه به‌سبب راستی دل خویش داخل زمین ایشان برای تصرفش می‌شوی، بلکه به‌سبب شرارت این امتها، یهوه، خدایت، ایشان را از حضور تو اخراج می‌نماید، و تا آنکه کلامی را که خداوند برای پدرانت، ابراهیم و اسحاق و یعقوب، قسم خورده بود، استوار نماید.
\par 6 پس بدان که یهوه، خدایت، این زمین نیکو رابه‌سبب عدالت تو به تو نمی دهد تا در آن تصرف نمایی، زیرا که قومی گردن کش هستی.
\par 7 پس بیادآور و فراموش مکن که چگونه خشم یهوه خدای خود را در بیابان جنبش دادی و از روزی که از زمین مصر بیرون آمدی تا به اینجا رسیدی به خداوند عاصی می‌شدید.
\par 8 و در حوریب خشم خداوند را جنبش دادید، و خداوند بر شما غضبناک شد تا شما راهلاک نماید.
\par 9 هنگامی که من به کوه برآمدم تالوحهای سنگ یعنی لوحهای عهدی را که خداوند با شما بست، بگیرم، آنگاه چهل روز وچهل شب در کوه ماندم؛ نه نان خوردم و نه آب نوشیدم.
\par 10 و خداوند دو لوح سنگ مکتوب شده به انگشت خدا را به من داد و بر آنها موافق تمامی سخنانی که خداوند در کوه از میان آتش در روزاجتماع به شما گفته بود، نوشته شد.
\par 11 و واقع شدبعد از انقضای چهل روز و چهل شب که خداونداین دو لوح سنگ یعنی لوحهای عهد را به من داد،
\par 12 و خداوند مرا گفت: «برخاسته، از اینجا به زودی فرود شو زیرا قوم تو که از مصر بیرون آوردی فاسد شده‌اند، و از طریقی که ایشان را امرفرمودم به زودی انحراف ورزیده، بتی ریخته شده برای خود ساختند.»
\par 13 و خداوند مراخطاب کرده، گفت: «این قوم را دیدم و اینک قوم گردن کش هستند.
\par 14 مرا واگذار تا ایشان را هلاک سازم و نام ایشان را از زیر آسمان محو کنم و از توقومی قوی تر و کثیرتر از ایشان بوجود آورم.»
\par 15 پس برگشته، از کوه فرود آمدم و کوه به آتش می‌سوخت و دو لوح عهد در دو دست من بود.
\par 16 و نگاه کرده، دیدم که به یهوه خدای خودگناه ورزیده، گوساله‌ای ریخته شده برای خودساخته و از طریقی که خداوند به شما امر فرموده بود، به زودی برگشته بودید.
\par 17 پس دو لوح راگرفتم و آنها را از دو دست خود انداخته، در نظرشما شکستم.
\par 18 و مثل دفعه اول، چهل روز وچهل شب به حضور خداوند به روی درافتادم؛ نه نان خوردم و نه آب نوشیدم، به‌سبب همه گناهان شما که کرده و کار ناشایسته که در نظر خداوندعمل نموده، خشم او را به هیجان آوردید.
\par 19 زیرا که از غضب و حدت خشمی که خداوندبر شما نموده بود تا شما را هلاک سازد، می‌ترسیدم، و خداوند آن مرتبه نیز مرا اجابت نمود.
\par 20 و خداوند بر هارون بسیا غضبناک شده بود تا او را هلاک سازد، و برای هارون نیز در آن وقت دعا کردم.
\par 21 و اما گناه شما یعنی گوساله‌ای را که ساخته بودید، گرفتم و آن را به آتش سوزانیدم و آن را خرد کرده، نیکو ساییدم تا مثل غبار نرم شد، و غبارش را به نهری که از کوه جاری بود، پاشیدم.
\par 22 و در تبعیره و مسا و کبروت هتاوه خشم خداوند را به هیجان آوردید.
\par 23 و وقتی که خداوند شما را از قادش برنیع فرستاده، گفت: بروید و در زمینی که به شما داده‌ام تصرف نمایید، از قول یهوه خدای خود عاصی شدید و به اوایمان نیاورده، آواز او را نشنیدید.
\par 24 از روزی که شما را شناخته‌ام به خداوند عصیان ورزیده‌اید.
\par 25 پس به حضور خداوند به روی درافتادم درآن چهل روز و چهل شب که افتاده بودم، از این جهت که خداوند گفته بود که شما را هلاک سازد.
\par 26 و نزد خداوند استدعا نموده، گفتم: «ای خداوند یهوه، قوم خود و میراث خود را که به عظمت خود فدیه دادی و به‌دست قوی از مصربیرون آوردی، هلاک مساز.
\par 27 بندگان خودابراهیم و اسحاق و یعقوب را بیاد آور، و برسخت دلی این قوم و شرارت و گناه ایشان نظر منما.
\par 28 مبادا اهل زمینی که ما را از آن بیرون آوردی، بگویند چونکه خداوند نتوانست ایشان را به زمینی که به ایشان وعده داده بود درآورد، وچونکه ایشان را دشمن می‌داشت، از این جهت ایشان را بیرون آورد تا در بیابان هلاک سازد.لیکن ایشان قوم تو و میراث تو هستند که به قوت عظیم خود و به بازوی افراشته خویش بیرون آوردی.»
\par 29 لیکن ایشان قوم تو و میراث تو هستند که به قوت عظیم خود و به بازوی افراشته خویش بیرون آوردی.»
 
\chapter{10}

\par 1 و در آن وقت خداوند به من گفت: «دولوح سنگ موافق اولیه برای خودبتراش، و نزد من به کوه برآی، و تابوتی از چوب برای خود بساز.
\par 2 و بر این لوحها کلماتی را که برلوحهای اولین که شکستی بود، خواهم نوشت، وآنها را در تابوت بگذار.»
\par 3 پس تابوتی از چوب سنط ساختم، و دو لوح سنگ موافق اولین تراشیدم، و آن دو لوح را در دست داشته، به کوه برآمدم.
\par 4 و بر آن دو لوح موافق کتابت اولین، آن ده کلمه را که خداوند در کوه از میان آتش، درروز اجتماع به شما گفته بود نوشت، و خداوندآنها را به من داد.
\par 5 پس برگشته، از کوه فرود آمدم، و لوحها را در تابوتی که ساخته بودم گذاشتم، ودر آنجا هست، چنانکه خداوند مرا امر فرموده بود.
\par 6 (و بنی‌اسرائیل از بیروت بنی یعقان به موسیره کوچ کردند، و در آنجا هارون مرد و درآنجا دفن شد. و پسرش العازار در جایش به کهانت پرداخت.
\par 7 و از آنجا به جدجوده کوچ کردند، و از جدجوده به یطبات که زمین نهرهای آب است.
\par 8 در آنوقت خداوند سبط لاوی را جداکرد، تا تابوت عهد خداوند را بردارند، و به حضور خداوند ایستاده، او را خدمت نمایند، و به نام او برکت دهند، چنانکه تا امروز است.
\par 9 بنابراین لاوی را در میان برادرانش نصیب ومیراثی نیست؛ خداوند میراث وی است، چنانکه یهوه خدایت به وی گفته بود).
\par 10 و من در کوه مثل روزهای اولین، چهل روزو چهل شب توقف نمودم، و در آن دفعه نیزخداوند مرا اجابت نمود، و خداوند نخواست تورا هلاک سازد.
\par 11 و خداوند مرا گفت: «برخیز وپیش روی این قوم روانه شو تا به زمینی که برای پدران ایشان قسم خوردم که به ایشان بدهم داخل شده، آن را به تصرف آورند.»
\par 12 پس الان‌ای اسرائیل، یهوه خدایت از توچه می‌خواهد، جز اینکه از یهوه خدایت بترسی و در همه طریقهایش سلوک نمایی، و او رادوست بداری و یهوه خدای خود را به تمامی دل و به تمامی جان خود عبادت نمایی.
\par 13 و اوامرخداوند و فرایض او را که من امروز تو را برای خیریتت امر می‌فرمایم، نگاه داری.
\par 14 اینک فلک و فلک الافلاک از آن یهوه خدای توست، و زمین وهرآنچه در آن است.
\par 15 لیکن خداوند به پدران تورغبت داشته، ایشان را محبت می‌نمود، و بعد ازایشان ذریت ایشان، یعنی شما را از همه قومهابرگزید، چنانکه امروز شده است.
\par 16 پس غلفه دلهای خود را مختون سازید، و دیگر گردن کشی منمایید.
\par 17 زیرا که یهوه خدای شما خدای خدایان و رب‌الارباب، و خدای عظیم و جبار ومهیب است، که طرفداری ندارد و رشوه نمی گیرد. 
\par 18 یتیمان و بیوه‌زنان را دادرسی می‌کند، و غریبان را دوست داشته، خوراک وپوشاک به ایشان می‌دهد.
\par 19 پس غریبان رادوست دارید، زیرا که در زمین مصر غریب بودید.
\par 20 از یهوه خدای خود بترس، و او را عبادت نما وبه او ملصق شو و به نام او قسم بخور.
\par 21 او فخرتوست و او خدای توست که برای تو این اعمال عظیم و مهیبی که چشمانت دیده بجا آورده است.پدران تو با هفتاد نفر به مصر فرود شدندو الان یهوه خدایت، تو را مثل ستارگان آسمان کثیر ساخته است.
\par 22 پدران تو با هفتاد نفر به مصر فرود شدندو الان یهوه خدایت، تو را مثل ستارگان آسمان کثیر ساخته است.
 
\chapter{11}

\par 1 پس یهوه خدای خود را دوست بدار، وودیعت و فرایض و احکام و اوامر او رادر همه وقت نگاهدار.
\par 2 و امروز بدانید، زیرا که به پسران شما سخن نمی گویم که ندانسته‌اند، وتادیب یهوه خدای شما را ندیده‌اند، و نه عظمت و دست قوی و بازوی افراشته او را.
\par 3 و آیات واعمال او را که در میان مصر، به فرعون، پادشاه مصر، و به تمامی زمین او بظهور آورد.
\par 4 و آنچه راکه به لشکر مصریان، به اسبها و به ارابه های ایشان کرد، که چگونه آب بحر قلزم را برایشان جاری ساخت، وقتی که شما را تعاقب می‌نمودند، وچگونه خداوند، ایشان را تا به امروز هلاک ساخت.
\par 5 و آنچه را که برای شما در بیابان کرد تاشما به اینجا رسیدید.
\par 6 و آنچه را که به داتان و ابیرام پسران الیاب بن روبین کرد، که چگونه زمین دهان خود را گشوده، ایشان را و خاندان وخیمه های ایشان را، و هر ذی حیات را که همراه ایشان بود در میان تمامی اسرائیل بلعید.
\par 7 لیکن چشمان شما تمامی اعمال عظیمه خداوند را که کرده بود، دیدند.
\par 8 پس جمیع اوامری را که من امروز برای شماامر می‌فرمایم نگاه دارید، تا قوی شوید و داخل شده، زمینی را که برای گرفتن آن عبور می‌کنید، به تصرف آورید.
\par 9 و تا در آن زمینی که خداوندبرای پدران شما قسم خورد که آن را به ایشان وذریت ایشان بدهد، عمر دراز داشته باشید، زمینی که به شیر و شهد جاری است.
\par 10 زیرا زمینی که تو برای گرفتن آن داخل می‌شوی، مثل زمین مصرکه از آن بیرون آمدی نیست، که در آن تخم خودرا می‌کاشتی و آن را مثل باغ بقول به پای خودسیراب می‌کردی.
\par 11 لیکن زمینی که شما برای گرفتنش به آن عبور می‌کنید، زمین کوهها ودره هاست که از بارش آسمان آب می‌نوشد،
\par 12 زمینی است که یهوه خدایت برآن التفات داردو چشمان یهوه خدایت از اول سال تا آخر سال پیوسته بر آن است.
\par 13 و چنین خواهد شد که اگر اوامری را که من امروز برای شما امر می‌فرمایم، بشنوید، و یهوه خدای خود را دوست بدارید، و او را به تمامی دل و به تمامی جان خود عبادت نمایید،
\par 14 آنگاه باران زمین شما یعنی باران اولین و آخرین را درموسمش خواهم بخشید، تا غله و شیره و روغن خود را جمع نمایی.
\par 15 و در صحرای تو برای بهایمت علف خواهم داد تا بخوری و سیر شوی.
\par 16 باحذر باشید مبادا دل شما فریفته شود وبرگشته، خدایان دیگر را عبادت و سجده نمایید.
\par 17 و خشم خداوند برشما افروخته شود، تاآسمان را مسدود سازد، و باران نبارد، و زمین محصول خود را ندهد و شما از زمین نیکویی که خداوند به شما می‌دهد، بزودی هلاک شوید.
\par 18 پس این سخنان مرا در دل و جان خود جادهید، و آنها را بر دستهای خود برای علامت ببندید، و در میان چشمان شما عصابه باشد.
\par 19 وآنها را به پسران خود تعلیم دهید، و حین نشستنت در خانه خود، و رفتنت به راه، و وقت خوابیدن و برخاستنت از آنها گفتگو نمایید.
\par 20 وآنها را بر باهوهای در خانه خود و بر دروازه های خود بنویسید.
\par 21 تا ایام شما و ایام پسران شما برزمینی که خداوند برای پدران شما قسم خورد که به ایشان بدهد، کثیر شود، مثل ایام افلاک بر بالای زمین.
\par 22 زیرا اگر تمامی این اوامر را که من به جهت عمل نمودن به شما امر می‌فرمایم، نیکو نگاه دارید، تا یهوه خدای خود را دوست دارید، و درتمامی طریقهای او رفتار نموده، به او ملصق شوید،
\par 23 آنگاه خداوند جمیع این امتها را ازحضور شما اخراج خواهد نمود، و شما امتهای بزرگتر و قویتر از خود را تسخیر خواهید نمود.
\par 24 هرجایی که کف پای شما برآن گذارده شود، ازآن شما خواهد بود، از بیابان و لبنان و از نهر، یعنی نهر فرات تا دریای غربی، حدود شماخواهد بود.
\par 25 و هیچکس یارای مقاومت با شمانخواهد داشت، زیرا یهوه خدای شما ترس و خوف شما را بر تمامی زمین که به آن قدم می‌زنیدمستولی خواهد ساخت، چنانکه به شما گفته است.
\par 26 اینک من امروز برکت و لعنت پیش شمامی گذارم.
\par 27 اما برکت، اگر اوامر یهوه خدای خود را که من امروز به شما امر می‌فرمایم، اطاعت نمایید.
\par 28 و اما لعنت، اگر اوامر یهوه خدای خودرا اطاعت ننموده، از طریقی که من امروز به شماامر می‌فرمایم برگردید، و خدایان غیر را که نشناخته‌اید، پیروی نمایید.
\par 29 و واقع خواهد شدکه چون یهوه، خدایت، تو را به زمینی که به جهت گرفتنش به آن می‌روی داخل سازد، آنگاه برکت را بر کوه جرزیم و لعنت را بر کوه ایبال خواهی گذاشت.
\par 30 آیا آنها به آنطرف اردن نیستند پشت راه غروب آفتاب، در زمین کنعانیانی که در عربه ساکنند مقابل جلجال نزد بلوطهای موره.
\par 31 زیرا که شما از اردن عبور می‌کنید تا داخل شده، زمینی را که یهوه خدایت به تو می‌بخشد به تصرف آورید، و آن را خواهید گرفت و در آن ساکن خواهید شد.پس متوجه باشید تا جمیع این فرایض و احکامی را که من امروز پیش شمامی گذارم، به عمل آورید.
\par 32 پس متوجه باشید تا جمیع این فرایض و احکامی را که من امروز پیش شمامی گذارم، به عمل آورید.
 
\chapter{12}

\par 1 اینهاست فرایض و احکامی که شما درتمامی روزهایی که بر زمین زنده خواهید ماند، می‌باید متوجه شده، به عمل آرید، در زمینی که یهوه خدای پدرانت به تو داده است، تا در آن تصرف نمایی.
\par 2 جمیع اماکن امتهایی را که در آنها خدایان خود را عبادت می‌کنند و شما آنها را اخراج می‌نمایید خراب نمایید، خواه بر کوههای بلندخواه بر تلها و خواه زیر هر درخت سبز.
\par 3 مذبحهای ایشان را بشکنید و ستونهای ایشان راخرد کنید، و اشیره های ایشان را به آتش بسوزانید، و بتهای تراشیده شده خدایان ایشان راقطع نمایید، و نامهای ایشان را از آنجا محوسازید.
\par 4 با یهوه خدای خود چنین عمل منمایید.
\par 5 بلکه به مکانی که یهوه خدای شما از جمیع اسباط شما برگزیند تا نام خود را در آنجا بگذارد، یعنی مسکن او را بطلبید و به آنجا بروید.
\par 6 و به آنجا قربانی های سوختنی و ذبایح و عشرهای خود، و هدایای افراشتنی دستهای خویش، ونذرها و نوافل خود و نخست زاده های رمه و گله خویش را ببرید.
\par 7 و در آنجا بحضور یهوه خدای خود بخورید، و شما و اهل خانه شما در هر شغل دست خود که یهوه خدای شما، شما را در آن برکت دهد، شادی نمایید.
\par 8 موافق هرآنچه ما امروز در اینجا می‌کنیم، یعنی آنچه در نظر هرکس پسند آید، نکنید.
\par 9 زیرا که هنوز به آرامگاه و نصیبی که یهوه خدای شما، به شما می‌دهد داخل نشده‌اید.
\par 10 اما چون از اردن عبور کرده، در زمینی که یهوه، خدای شما، برای شما تقسیم می‌کند، ساکن شوید، و اوشما را از جمیع دشمنان شما از هرطرف آرامی دهد تا در امنیت سکونت نمایید.
\par 11 آنگاه به مکانی که یهوه خدای شما برگزیند تا نام خود رادر آن ساکن سازد، به آنجا هرچه را که من به شماامر فرمایم بیاورید، از قربانی های سوختنی وذبایح و عشرهای خود، و هدایای افراشتنی دستهای خویش، و همه نذرهای بهترین خود که برای خداوند نذر نمایید.
\par 12 و به حضور یهوه خدای خود شادی نمایید، شما با پسران ودختران و غلامان و کنیزان خود، و لاویانی که درون دروازه های شما باشند، چونکه ایشان را باشما حصه‌ای و نصیبی نیست.
\par 13 با حذر باش که در هر جایی که می‌بینی قربانی های سوختنی خود را نگذرانی،
\par 14 بلکه درمکانی که خداوند در یکی از اسباط تو برگزیند درآنجا قربانی های سوختنی خود را بگذرانی، و درآنجا هرچه من به تو امر فرمایم، به عمل آوری.
\par 15 لیکن گوشت را برحسب تمامی آرزوی دلت، موافق برکتی که یهوه خدایت به تو دهد، درهمه دروازه هایت ذبح کرده، بخور؛ اشخاص نجس و طاهر از آن بخورند چنانکه از غزال و آهومی خورند.
\par 16 ولی خون را نخور؛ آن را مثل آب بر زمین بریز.
\par 17 عشر غله و شیره و روغن ونخست زاده رمه و گله خود را در دروازه های خودمخور، و نه هیچ‌یک از نذرهای خود را که نذرمی کنی و از نوافل خود و هدایای افراشتنی دست خود را.
\par 18 بلکه آنها را به حضور یهوه خدایت درمکانی که یهوه خدایت برگزیند، بخور، تو وپسرت و دخترت و غلامت و کنیزت و لاویانی که درون دروازه های تو باشند، و به هرچه دست خود را برآن بگذاری به حضور یهوه خدایت شادی نما.
\par 19 با حذر باش که لاویان را در تمامی روزهایی که در زمین خود باشی، ترک ننمایی.
\par 20 چون یهوه، خدایت، حدود تو را بطوری که تو را وعده داده است، وسیع گرداند، و بگویی که گوشت خواهم خورد، زیرا که دل تو به گوشت خوردن مایل است، پس موافق همه آرزوی دلت گوشت را بخور.
\par 21 و اگر مکانی که یهوه، خدایت، برگزیند تا اسم خود را در آن بگذارد ازتو دور باشد، آنگاه از رمه و گله خود که خداوندبه تو دهد ذبح کن، چنانکه به تو امر فرموده‌ام و ازهرچه دلت بخواهد در دروازه هایت بخور.
\par 22 چنانکه غزال و آهو خورده شود، آنها را چنین بخور؛ شخص نجس و شخص طاهر از آن برابربخورند.
\par 23 لیکن هوشیار باش که خون را نخوری زیرا خون جان است و جان را با گوشت نخوری.
\par 24 آن را مخور، بلکه مثل آب برزمینش بریز.
\par 25 آن را مخور تا برای تو و بعد از تو برای پسرانت نیکو باشد هنگامی که آنچه در نظرخداوند راست است، بجا آوری.
\par 26 لیکن موقوفات خود را که داری و نذرهای خود رابرداشته، به مکانی که خداوند برگزیند، برو.
\par 27 وگوشت و خون قربانی های سوختنی خود را برمذبح یهوه خدایت بگذران و خون ذبایح توبرمذبح یهوه خدایت ریخته شود و گوشت رابخور.
\par 28 متوجه باش که همه این سخنانی را که من به تو امر می‌فرمایم بشنوی تا برای تو و بعد از توبرای پسرانت هنگامی که آنچه در نظر یهوه، خدایت، نیکو و راست است بجا آوری تا به ابدنیکو باشد.
\par 29 وقتی که یهوه، خدایت، امتهایی را که به جهت گرفتن آنها به آنجا می‌روی، از حضور تومنقطع سازد، و ایشان را اخراج نموده، در زمین ایشان ساکن شوی.
\par 30 آنگاه باحذر باش، مبادا بعداز آنکه از حضور تو هلاک شده باشند به دام گرفته شده، ایشان را پیروی نمایی و درباره خدایان ایشان دریافت کرده، بگویی که این امتها خدایان خود را چگونه عبادت کردند تا من نیز چنین کنم.
\par 31 با یهوه، خدای خود، چنین عمل منما، زیراهرچه را که نزد خداوند مکروه است و از آن نفرت دارد، ایشان برای خدایان خود می‌کردند، حتی اینکه پسران و دختران خود را نیز برای خدایان خود به آتش می‌سوزانیدند.هر‌آنچه من به شما امر می‌فرمایم متوجه شوید، تا آن را به عمل آورید، چیزی برآن میفزایید و چیزی از آن کم نکنید.
\par 32 هر‌آنچه من به شما امر می‌فرمایم متوجه شوید، تا آن را به عمل آورید، چیزی برآن میفزایید و چیزی از آن کم نکنید.
 
\chapter{13}

\par 1 اگر در میان تو نبی‌ای یا بیننده خواب ازمیان شما برخیزد، و آیت یا معجزه‌ای برای شما ظاهر سازد،
\par 2 و آن آیت یا معجزه واقع شود که از آن تو را خبر داده، گفت خدایان غیر راکه نمی شناسی پیروی نماییم، و آنها را عبادت کنیم،
\par 3 سخنان آن نبی یا بیننده خواب را مشنو، زیرا که یهوه، خدای شما، شما را امتحان می‌کند، تا بداند که آیا یهوه، خدای خود را به تمامی دل وبه تمامی جان خود محبت می‌نمایید؟
\par 4 یهوه خدای خود را پیروی نمایید و از او بترسید، واوامر او را نگاه دارید، و قول او را بشنوید و او راعبادت نموده، به او ملحق شوید. 
\par 5 و آن نبی یابیننده خواب کشته شود، زیرا که سخنان فتنه انگیز بر یهوه خدای شما که شما را از زمین مصر بیرون آورد، و تو را از خانه بندگی فدیه داد، گفته است تا تو را از طریقی که یهوه خدایت به توامر فرمود تا با آن سلوک نمایی، منحرف سازد، پس به این طور بدی را از میان خود دور خواهی کرد.
\par 6 و اگر برادرت که پسر مادرت باشد یا پسر یادختر تو یا زن هم آغوش تو یا رفیقت که مثل جان تو باشد، تو را در خفا اغوا کند، و گوید که برویم وخدایان غیر را که تو و پدران تو نشناختید عبادت نماییم،
\par 7 از خدایان امتهایی که به اطراف شمامی باشند، خواه به تو نزدیک و خواه از تو دورباشند، از اقصای زمین تا اقصای دیگر آن،
\par 8 او راقبول مکن و او را گوش مده، و چشم تو بر وی رحم نکند و بر او شفقت منما و او را پنهان مکن.
\par 9 البته او را به قتل رسان، دست تو اول به قتل اودراز شود و بعد دست تمامی قوم.
\par 10 و او را به سنگ سنگسار نما تا بمیرد، چونکه می‌خواست تو را از یهوه، خدایت، که تو را از زمین مصر ازخانه بندگی بیرون آورد، منحرف سازد.
\par 11 وجمیع اسرائیلیان چون بشنوند، خواهند ترسید وبار دیگر چنین امر زشت را در میان شما مرتکب نخواهند شد.
\par 12 اگر درباره یکی از شهرهایی که یهوه خدایت به تو به جهت سکونت می‌دهد خبر یابی،
\par 13 که بعضی پسران بلیعال از میان تو بیرون رفته، ساکنان شهر خود را منحرف ساخته، گفته اندبرویم و خدایان غیر را که نشناخته‌اید، عبادت نماییم،
\par 14 آنگاه تفحص و تجسس نموده، نیکواستفسار نما. و اینک اگر این امر، صحیح و یقین باشد که این رجاست در میان تو معمول شده است،
\par 15 البته ساکنان آن شهر را به دم شمشیربکش و آن را با هرچه در آن است و بهایمش را به دم شمشیر هلاک نما.
\par 16 و همه غنیمت آن را در میان کوچه‌اش جمع کن و شهر را با تمامی غنیمتش برای یهوه خدایت به آتش بالکل بسوزان، و آن تا به ابد تلی خواهد بود و بار دیگربنا نخواهد شد.
\par 17 و از چیزهای حرام شده چیزی به‌دستت نچسبد تا خداوند از شدت خشم خود برگشته، برتو رحمت و رافت بنماید، و تو را بیفزاید بطوری که برای پدرانت قسم خورده بود.هنگامی که قول یهوه خدای خودرا شنیده، و همه اوامرش را که من امروز به تو امرمی فرمایم نگاه داشته، آنچه در نظر یهوه خدایت راست است، بعمل آورده باشی.
\par 18 هنگامی که قول یهوه خدای خودرا شنیده، و همه اوامرش را که من امروز به تو امرمی فرمایم نگاه داشته، آنچه در نظر یهوه خدایت راست است، بعمل آورده باشی.
 
\chapter{14}

\par 1 شما پسران یهوه خدای خود هستید، پس برای مردگان، خویشتن را مجروح منمایید، و مابین چشمان خود را متراشید.
\par 2 زیراتو برای یهوه، خدایت، قوم مقدس هستی، وخداوند تو را برای خود برگزیده است تا از جمیع امتهایی که برروی زمین‌اند به جهت او قوم خاص باشی.
\par 3 هیچ‌چیز مکروه مخور.
\par 4 این است حیواناتی که بخورید: گاو و گوسفند و بز،
\par 5 و آهو و غزال وگور و بزکوهی و ریم و گاو دشتی و مهات.
\par 6 و هرحیوان شکافته سم که سم را به دو حصه شکافته دارد و نشخوار کند، آن را از بهایم بخورید.
\par 7 لیکن از نشخوارکنندگان و شکافتگان سم اینهارا مخورید: یعنی شتر و خرگوش و ونک، زیرا که نشخوار می‌کنند اما شکافته سم نیستند. اینهابرای شما نجس‌اند.
\par 8 و خوک زیرا شکافته سم است، لیکن نشخوار نمی کند، این برای شما نجس است. از گوشت آنها مخورید و لاش آنها را لمس مکنید.
\par 9 از همه آنچه در آب است اینها را بخورید: هرچه پر و فلس دارد، آنها را بخورید.
\par 10 و هرچه پر و فلس ندارد مخورید، برای شما نجس است.
\par 11 از همه مرغان طاهر بخورید.
\par 12 و این است آنهایی که نخورید: عقاب و استخوان خوار و نسربحر،
\par 13 و لاشخوار و شاهین و کرکس به اجناس آن؛
\par 14 و هر غراب به اجناس آن؛
\par 15 و شترمرغ وجغد و مرغ دریایی و باز، به اجناس آن؛
\par 16 و بوم و بوتیمار و قاز؛
\par 17 و قائت و رخم و غواص؛
\par 18 ولقلق و کلنک، به اجناس آن؛ و هدهد و شبپره.
\par 19 و همه حشرات بالدار برای شما نجس‌اند؛ خورده نشوند.
\par 20 اما از همه مرغان طاهربخورید.
\par 21 هیچ میته مخورید؛ به غریبی که درون دروازه های تو باشد بده تا بخورد، یا به اجنبی بفروش، زیرا که تو برای یهوه، خدایت، قوم مقدس هستی و بزغاله را در شیر مادرش مپز.
\par 22 عشر تمامی محصولات مزرعه خود را که سال به سال از زمین برآید، البته بده.
\par 23 و به حضور یهوه خدایت در مکانی که برگزیند تا نام خود را در آنجا ساکن سازد، عشر غله و شیره وروغن خود را و نخست زادگان رمه و گله خویش را بخور، تا بیاموزی که از یهوه خدایت همه اوقات بترسی.
\par 24 و اگر راه از برایت دور باشد که آن را نمی توانی برد، و آن مکانی که یهوه، خدایت، خواهد برگزید تا نام خود را در آن بگذارد، وقتی که یهوه، خدایت، تو را برکت دهد، از تو دور باشد.
\par 25 پس آن را به نقره بفروش و نقره را بدست خود گرفته، به مکانی که یهوه خدایت برگزیند، برو.
\par 26 و نقره را برای هرچه دلت می‌خواهد از گاو و گوسفند و شراب و مسکرات و هرچه دلت از تو بطلبد، بده، و در آنجا بحضوریهوه، خدایت، بخور و خودت با خاندانت شادی نما.
\par 27 و لاوی‌ای را که اندرون دروازه هایت باشد، ترک منما چونکه او را با تو حصه و نصیبی نیست.
\par 28 و در آخر هر سه سال تمام عشر محصول خود را در همان سال بیرون آورده، در اندرون دروازه هایت ذخیره نما.و لاوی چونکه با توحصه و نصیبی ندارد و غریب و یتیم و بیوه‌زنی که درون دروازه هایت باشند، بیایند و بخورند و سیرشوند، تا یهوه، خدایت، تو را در همه اعمال دستت که می‌کنی، برکت دهد.
\par 29 و لاوی چونکه با توحصه و نصیبی ندارد و غریب و یتیم و بیوه‌زنی که درون دروازه هایت باشند، بیایند و بخورند و سیرشوند، تا یهوه، خدایت، تو را در همه اعمال دستت که می‌کنی، برکت دهد.
 
\chapter{15}

\par 1 و در آخر هر هفت سال، انفکاک نمایی.
\par 2 و قانون انفکاک این باشد، هرطلبکاری قرضی را که به همسایه خود داده باشدمنفک سازد، و از همسایه و برادر خود مطالبه نکند، چونکه انفکاک خداوند اعلان شده است.
\par 3 از غریب مطالبه توانی کرد، اما هرآنچه از مال تو نزد برادرت باشد، دست تو آن را منفک سازد.
\par 4 تا نزد تو هیچ فقیر نباشد، زیرا که خداوند تو رادر زمینی که یهوه، خدایت، برای نصیب و ملک به تو می‌دهد، البته برکت خواهد داد،
\par 5 اگر قول یهوه، خدایت، را به دقت بشنوی تا متوجه شده، جمیع این اوامر را که من امروز به تو امر می‌فرمایم بجا آوری.
\par 6 زیرا که یهوه، خدایت، تو را چنانکه گفته است برکت خواهد داد، و به امتهای بسیارقرض خواهی داد، لیکن تو مدیون نخواهی شد، وبر امتهای بسیار تسلط خواهی نمود، و ایشان برتو مسلط نخواهند شد.
\par 7 اگر نزد تو در یکی از دروازه هایت، در زمینی که یهوه، خدایت، به تو می‌بخشد، یکی ازبرادرانت فقیر باشد، دل خود را سخت مساز، ودستت را بر برادر فقیر خود مبند.
\par 8 بلکه البته دست خود را بر او گشاده دار، و به قدر کفایت، موافق احتیاج او به او قرض بده.
\par 9 و باحذر باش مبادا در دل تو فکر زشت باشد، و بگویی سال هفتم یعنی سال انفکاک نزدیک است، و چشم توبر برادر فقیر خود بد شده، چیزی به او ندهی و اواز تو نزد خداوند فریاد برآورده، برایت گناه باشد.
\par 10 البته به او بدهی و دلت از دادنش آزرده نشود، زیرا که به عوض این کار یهوه، خدایت، تو را درتمامی کارهایت و هرچه دست خود را بر آن درازمی کنی، برکت خواهد داد.
\par 11 چونکه فقیر اززمینت معدوم نخواهد شد، بنابراین من تو را امرفرموده، می‌گویم البته دست خود را برای برادرمسکین و فقیر خود که در زمین تو باشند، گشاده دار.
\par 12 اگر مرد یا زن عبرانی از برادرانت به توفروخته شود، و او تو را شش سال خدمت نماید، پس در سال هفتم او را از نزد خود آزاد کرده، رها کن.
\par 13 و چون او را از نزد خود آزاد کرده، رهامی کنی، او را تهی‌دست روانه مساز.
\par 14 او را ازگله و خرمن و چرخشت خود البته زاد بده، به اندازه‌ای که یهوه، خدایت، تو را برکت داده باشدبه او بده.
\par 15 و بیادآور که تو در زمین مصر غلام بودی و یهوه، خدایت، تو را فدیه داد، بنابراین من امروز این را به تو امر می‌فرمایم،
\par 16 و اگر به توگوید از نزد تو بیرون نمی روم چونکه تو را وخاندان تو را دوست دارد و او را نزد تو خوش گذشته باشد،
\par 17 آنگاه درفشی گرفته، گوشش را باآن به دربدوز تا تو را غلام ابدی باشد، و با کنیزخود نیز چنین عمل نما.
\par 18 و چون او را از نزد خود آزاد کرده، رهامی کنی، بنظرت بد نیاید، زیرا که دو برابر اجرت اجیر، تو را شش سال خدمت کرده است. و یهوه خدایت در هرچه می‌کنی تو را برکت خواهد داد.
\par 19 همه نخست زادگان نرینه را که از رمه و گله تو زاییده شوند برای یهوه، خدای خود، تقدیس نما، و با نخست زاده گاو خود کار مکن ونخست زاده گوسفند خود را پشم مبر.
\par 20 آنها رابه حضور یهوه خدای خود در مکانی که خداوندبرگزیند، تو و اهل خانه ات سال به سال بخورید.
\par 21 لیکن اگر عیبی داشته باشد، مثلا لنگ یا کور یاهر عیب دیگر، آن را برای یهوه خدایت ذبح مکن.
\par 22 آن را در اندرون دروازه هایت بخور، شخص نجس و شخص طاهر، آن را برابر مثل غزال و آهوبخورند.اما خونش را مخور. آن را مثل آب بر زمین بریز.
\par 23 اما خونش را مخور. آن را مثل آب بر زمین بریز.
 
\chapter{16}

\par 1 ماه ابیب را نگاهدار و فصح را به جهت یهوه، خدایت، بجا آور، زیرا که در ماه ابیب یهوه، خدایت، تو را از مصر در شب بیرون آورد.
\par 2 پس فصح را از رمه و گله برای یهوه، خدایت، ذبح کن، در مکانی که خداوند برگزیند، تا نام خود را در آن ساکن سازد.
\par 3 با آن، خمیرمایه مخور، هفت روز نان فطیر یعنی نان مشقت را باآن بخور، زیرا که به تعجیل از زمین مصر بیرون آمدی، تا روز خروج خود را از زمین مصر درتمامی روزهای عمرت بیاد آوری.
\par 4 پس هفت روز هیچ خمیرمایه در تمامی حدودت دیده نشود، و از گوشتی که در شام روز اول، ذبح می‌کنی چیزی تا صبح باقی نماند.
\par 5 فصح را درهر یکی از دروازه هایت که یهوه خدایت به تومی دهد، ذبح نتوانی کرد.
\par 6 بلکه در مکانی که یهوه، خدایت، برگزیند تا نام خود را در آن ساکن سازد، در آنجا فصح را در شام، وقت غروب آفتاب، هنگام بیرون آمدنت از مصر ذبح کن.
\par 7 وآن را در مکانی که یهوه، خدایت، برگزیند بپز وبخور و بامدادان برخاسته، به خیمه هایت برو.
\par 8 شش روز نان فطیر بخور، و در روز هفتم، جشن مقدس برای یهوه خدایت باشد، در آن هیچ کارمکن.
\par 9 هفت هفته برای خود بشمار. از ابتدای نهادن داس در زرع خود، شمردن هفت هفته را شروع کن.
\par 10 و عید هفته‌ها را با هدیه نوافل دست خودنگاهدار و آن را به اندازه برکتی که یهوه خدایت به تو دهد، بده.
\par 11 و به حضور یهوه، خدایت، شادی نما تو و پسرت و دخترت و غلامت و کنیزت ولاوی که درون دروازه هایت باشد و غریب و یتیم و بیوه‌زنی که در میان تو باشند، در مکانی که یهوه خدایت برگزیند تا نام خود را در آن ساکن گرداند.
\par 12 و بیاد آور که در مصر غلام بودی، پس متوجه شده، این فرایض را بجا آور.
\par 13 عید خیمه‌ها را بعد از جمع کردن حاصل ازخرمن، و چرخشت خود هفت روز نگاهدار.
\par 14 ودر عید خود شادی نما، تو و پسرت و دخترت وغلامت و کنیزت و لاوی و غریب و یتیم وبیوه‌زنی که درون دروازه هایت باشند.
\par 15 هفت روز در مکانی که خداوند برگزیند، برای یهوه خدایت عید نگاه دار، زیرا که یهوه خدایت تو رادر همه محصولت و در تمامی اعمال دستت برکت خواهد داد، و بسیار شادمان خواهی بود.
\par 16 سه مرتبه در سال جمیع ذکورانت به حضوریهوه خدایت در مکانی که او برگزیند حاضرشوند، یعنی در عید فطیر و عید هفته‌ها و عیدخیمه‌ها و به حضور خداوند تهی‌دست حاضرنشوند. 
\par 17 هر کس به قدر قوه خود به اندازه برکتی که یهوه، خدایت، به تو عطا فرماید، بدهد.
\par 18 داوران و سروران در جمیع دروازه هایی که یهوه، خدایت، به تو می‌دهد برحسب اسباط خود برایت تعیین نما، تا قوم را به حکم عدل، داوری نمایند.
\par 19 داوری را منحرف مساز و طرفداری منما و رشوه مگیر، زیرا که رشوه چشمان حکمارا کور می‌سازد و سخنان عادلان را کج می‌نماید.
\par 20 انصاف کامل را پیروی نما تا زنده مانی وزمینی را که یهوه خدایت به تو می‌دهد، مالک شوی.
\par 21 اشیره‌ای از هیچ نوع درخت نزد مذبح یهوه، خدایت، که برای خود خواهی ساخت غرس منما.و ستونی برای خود نصب مکن زیرا یهوه خدایت آن را مکروه می‌دارد.
\par 22 و ستونی برای خود نصب مکن زیرا یهوه خدایت آن را مکروه می‌دارد.
 
\chapter{17}

\par 1 گاو یا گوسفندی که در آن عیب یا هیچ چیز بد باشد، برای یهوه خدای خودذبح منما، چونکه آن، نزد یهوه خدایت مکروه است.
\par 2 اگر در میان تو، در یکی از دروازه هایت که یهوه خدایت به تو می‌دهد، مرد یا زنی پیدا شودکه در نظر یهوه، خدایت، کار ناشایسته نموده، ازعهد او تجاوز کند،
\par 3 و رفته خدایان غیر را عبادت کرده، سجده نماید، خواه آفتاب یا ماه یا هر یک از جنود آسمان که من امر نفرموده‌ام،
\par 4 و از آن اطلاع یافته، بشنوی، پس نیکو تفحص کن. واینک اگر راست و یقین باشد که این رجاست دراسرائیل واقع شده است،
\par 5 آنگاه آن مرد یا زن راکه این کار بد را در دروازه هایت کرده است، بیرون آور، و آن مرد یا زن را با سنگها سنگسار کن تابمیرند.
\par 6 از گواهی دو یا سه شاهد، آن شخصی که مستوجب مرگ است کشته شود؛ از گواهی یک نفر کشته نشود.
\par 7 اولا دست شاهدان به جهت کشتنش بر او افراشته شود، و بعد از آن، دست تمامی قوم، پس بدی را از میان خود دورکرده‌ای.
\par 8 اگر در میان تو امری که حکم بر آن مشکل شود به ظهور آید، در میان خون و خون، و درمیان دعوی و دعوی، و در میان ضرب و ضرب، از مرافعه هایی که در دروازه هایت واقع شود، آنگاه برخاسته، به مکانی که یهوه، خدایت، برگزیند، برو.
\par 9 و نزد لاویان کهنه و نزد داوری که در آن روزها باشد رفته، مسالت نما و ایشان تو رااز فتوای قضا مخبر خواهند ساخت.
\par 10 وبرحسب فتوایی که ایشان از مکانی که خداوندبرگزیند، برای تو بیان می‌کنند، عمل نما. وهوشیار باش تا موافق هر‌آنچه به تو تعلیم دهند، عمل نمایی.
\par 11 موافق مضمون شریعتی که به توتعلیم دهند، و مطابق حکمی که به تو گویند، عمل نما، و از فتوایی که برای تو بیان می‌کنند به طرف راست یا چپ تجاوز مکن.
\par 12 و شخصی که از روی تکبر رفتار نماید، و کاهنی را که به حضوریهوه، خدایت، به جهت خدمت در آنجامی ایستد یا داور را گوش نگیرد، آن شخص کشته شود. پس بدی را از میان اسرائیل دور کرده‌ای.
\par 13 و تمامی قوم چون این را بشنوند، خواهندترسید و بار دیگر از روی تکبر رفتار نخواهندنمود.
\par 14 چون به زمینی که یهوه، خدایت، به تومی دهد، داخل شوی و در آن تصرف نموده، ساکن شوی و بگویی مثل جمیع امتهایی که به اطراف منند پادشاهی بر خود نصب نمایم،
\par 15 البته پادشاهی را که یهوه خدایت برگزیند برخود نصب نما. یکی از برادرانت را بر خود پادشاه بساز، و مرد بیگانه‌ای را که از برادرانت نباشد، نمی توانی بر خود مسلط نمایی.
\par 16 لکن او برای خود اسبهای بسیار نگیرد، و قوم را به مصر پس نفرستد، تا اسبهای بسیار برای خود بگیرد، چونکه خداوند به شما گفته است بار دیگر به آن راه برنگردید.
\par 17 و برای خود زنان بسیار نگیرد، مبادا دلش منحرف شود، و نقره و طلا برای خودبسیار زیاده نیندوزد.
\par 18 و چون بر تخت مملکت خود بنشیند، نسخه این شریعت را از آنچه از آن، نزد لاویان کهنه است برای خود در طوماری بنویسد.
\par 19 و آن نزد او باشد و همه روزهای عمرش آن را بخواند، تا بیاموزد که از یهوه خدای خود بترسد، و همه کلمات این شریعت و این فرایض را نگاه داشته، به عمل آورد.مبادا دل او بر برادرانش افراشته شود، و از این اوامر به طرف چپ یا راست منحرف شود، تا آنکه او وپسرانش در مملکت او در میان اسرائیل روزهای طویل داشته باشند.
\par 20 مبادا دل او بر برادرانش افراشته شود، و از این اوامر به طرف چپ یا راست منحرف شود، تا آنکه او وپسرانش در مملکت او در میان اسرائیل روزهای طویل داشته باشند.
 
\chapter{18}

\par 1 لاویان کهنه و تمامی سبط لاوی راحصه و نصیبی با اسرائیل نباشد.
\par 2 پس ایشان در میان برادران خود نصیب نخواهندداشت. خداوند نصیب ایشان است، چنانکه به ایشان گفته است.
\par 3 و حق کاهنان از قوم، یعنی از آنانی که قربانی، خواه از گاو و خواه از گوسفندمی گذرانند، این است که دوش و دو بنا گوش وشکنبه را به کاهن بدهند.
\par 4 و نوبر غله و شیره وروغن خود و اول چین پشم گوسفند خود را به اوبده،
\par 5 زیرا که یهوه، خدایت، او را از همه اسباطت برگزیده است، تا او و پسرانش همیشه بایستند و به نام خداوند خدمت نمایند.
\par 6 و اگر احدی از لاویان از یکی ازدروازه هایت از هر جایی در اسرائیل که در آنجاساکن باشد آمده، به تمامی آرزوی دل خود به مکانی که خداوند برگزیند، برسد،
\par 7 پس به نام یهوه خدای خود، مثل سایر برادرانش از لاویانی که در آنجا به حضور خداوند می‌ایستند، خدمت نماید.
\par 8 حصه های برابر بخورند، سوای آنچه ازارثیت خود بفروشد.
\par 9 چون به زمینی که یهوه، خدایت، به تومی دهد داخل شوی، یاد مگیر که موافق رجاسات آن امتها عمل نمایی.
\par 10 و در میان تو کسی یافت نشود که پسر یا دختر خود را از آتش بگذرانند، ونه فالگیر و نه غیب گو و نه افسونگر و نه جادوگر،
\par 11 و نه ساحر و نه سوال کننده از اجنه و نه رمال ونه کسی‌که از مردگان مشورت می‌کند.
\par 12 زیرا هرکه این کارها را کند، نزد خداوند مکروه است وبه‌سبب این رجاسات، یهوه، خدایت، آنها را ازحضور تو اخراج می‌کند.
\par 13 نزد یهوه، خدایت، کامل باش.
\par 14 زیرا این امتهایی که تو آنها را بیرون می‌کنی به غیب گویان و فالگیران گوش می‌گیرند، و اما یهوه، خدایت، تو را نمی گذارد که چنین بکنی.
\par 15 یهوه، خدایت، نبی‌ای را از میان تو ازبرادرانت، مثل من برای تو مبعوث خواهدگردانید، او را بشنوید.
\par 16 موافق هر‌آنچه درحوریب در روز اجتماع از یهوه خدای خودمسالت نموده، گفتی: «آواز یهوه خدای خود رادیگر نشنوم، و این آتش عظیم را دیگر نبینم، مبادابمیرم.»
\par 17 و خداوند به من گفت: «آنچه گفتندنیکو گفتند.
\par 18 نبی‌ای را برای ایشان از میان برادران ایشان مثل تو مبعوث خواهم کرد، و کلام خود را به دهانش خواهم گذاشت و هر‌آنچه به اوامر فرمایم به ایشان خواهد گفت.
\par 19 و هر کسی‌که سخنان مرا که او به اسم من گوید نشنود، من ازاو مطالبه خواهم کرد.
\par 20 و اما نبی‌ای که جسارت نموده، به اسم من سخن گوید که به گفتنش امرنفرمودم، یا به اسم خدایان غیر سخن گوید، آن نبی البته کشته شود.»
\par 21 و اگر در دل خود گویی: «سخنی را که خداوند نگفته است، چگونه تشخیص نماییم.»هنگامی که نبی به اسم خداوند سخن گوید، اگر آن چیز واقع نشود و به انجام نرسد، این امری است که خداوند نگفته است، بلکه آن نبی آن را از روی تکبر گفته است.پس از او نترس.
\par 22 هنگامی که نبی به اسم خداوند سخن گوید، اگر آن چیز واقع نشود و به انجام نرسد، این امری است که خداوند نگفته است، بلکه آن نبی آن را از روی تکبر گفته است.پس از او نترس.
 
\chapter{19}

\par 1 وقتی که خدایت این امتها را که یهوه، خدایت، زمین ایشان را به تو می‌دهدمنقطع سازد، و تو وارث ایشان شده، در شهرها وخانه های ایشان ساکن شوی،
\par 2 پس سه شهر رابرای خود در میان زمینی که یهوه، خدایت، به جهت ملکیت به تو می‌دهد، جدا کن.
\par 3 شاهراه رابرای خود درست کن، و حدود زمین خود را که یهوه خدایت برای تو تقسیم می‌کند، سه قسمت کن، تا هر قاتلی در آنجا فرار کند.
\par 4 و این است حکم قاتلی که به آنجا فرار کرده، زنده ماند، هر‌که همسایه خود را نادانسته بکشد، و قبل از آن از او بغض نداشت.
\par 5 مثل کسی‌که باهمسایه خود برای بریدن درخت در جنگل برود، و دستش برای قطع نمودن درخت تبر را بلند کند، و آهن از دسته بیرون رفته، به همسایه‌اش بخوردتا بمیرد، پس به یکی از آن شهرها فرار کرده، زنده ماند.
\par 6 مبادا ولی خون وقتی که دلش گرم است قاتل را تعاقب کند، و به‌سبب مسافت راه به وی رسیده، او را بکشد، و او مستوجب موت نباشد، چونکه او را پیشتر بغض نداشت.
\par 7 از این جهت من تو را امر فرموده، گفتم برای خود سه شهر جداکن.
\par 8 و اگر یهوه، خدایت، حدود تو را به طوری که به پدرانت قسم خورده است وسیع گرداند، وتمامی زمین را به تو عطا فرماید، که به دادن آن به پدرانت وعده داده است،
\par 9 و اگر تمامی این اوامررا که من امروز به تو می‌فرمایم نگاه داشته، بجا آوری که یهوه خدای خود را دوست داشته، به طریقهای او دائم سلوک نمایی، آنگاه سه شهردیگر بر این سه برای خود مزید کن.
\par 10 تا خون بی‌گناه در زمینی که یهوه خدایت برای ملکیت به تو می‌دهد، ریخته نشود، و خون بر گردن تونباشد.
\par 11 لیکن اگر کسی همسایه خود را بغض داشته، در کمین او باشد و بر او برخاسته، او راضرب مهلک بزند که بمیرد، و به یکی از این شهرها فرار کند،
\par 12 آنگاه مشایخ شهرش فرستاده، او را از آنجا بگیرند، و او را به‌دست ولی خون تسلیم کنند، تا کشته شود.
\par 13 چشم تو بر اوترحم نکند، تا خون بی‌گناهی را از اسرائیل دورکنی، و برای تو نیکو باشد.
\par 14 حد همسایه خود را که پیشینیان گذاشته‌اند، در ملک تو که به‌دست تو خواهد آمد، در زمینی که یهوه خدایت برای تصرفش به تومی دهد، منتقل مساز.
\par 15 یک شاهد بر کسی برنخیزد، به هر تقصیر وهر گناه از جمیع گناهانی که کرده باشد، به گواهی دو شاهد یا به گواهی سه شاهد هر امری ثابت شود.
\par 16 اگر شاهد کاذبی بر کسی برخاسته، به معصیتش شهادت دهد،
\par 17 آنگاه هر دو شخصی که منازعه در میان ایشان است، به حضور خداوندو به حضور کاهنان و داورانی که در آن زمان باشند، حاضر شوند.
\par 18 و داوران، نیکو تفحص نمایند، و اینک اگر شاهد، شاهد کاذب است و بربرادر خود شهادت دروغ داده باشد،
\par 19 پس به طوری که او خواست با برادر خود عمل نماید با او همان طور رفتار نمایند، تا بدی را از میان خوددور نمایی.
\par 20 و چون بقیه مردمان بشنوند، خواهند ترسید، و بعد از آن مثل این کار زشت درمیان شما نخواهند کرد.و چشم تو ترحم نکند، جان به عوض جان، و چشم به عوض چشم، و دندان به عوض دندان، و دست به عوض دست و پا به عوض پا.
\par 21 و چشم تو ترحم نکند، جان به عوض جان، و چشم به عوض چشم، و دندان به عوض دندان، و دست به عوض دست و پا به عوض پا.
 
\chapter{20}

\par 1 چون برای مقاتله با دشمن خود بیرون روی، و اسبها و ارابه‌ها و قومی را زیاده از خودبینی، از ایشان مترس زیرا یهوه خدایت که تو را از زمین مصر برآورده است، با توست.
\par 2 وچون به جنگ نزدیک شوید آنگاه کاهن پیش آمده، قوم را مخاطب سازد.
\par 3 و ایشان را گوید: «ای اسرائیل بشنوید شما امروز برای مقاتله بادشمنان خود پیش می‌روید، دل شما ضعیف نشود، و از ایشان ترسان و لرزان و هراسان مباشید.
\par 4 زیرا یهوه، خدای شما، با شما می‌رود، تا برای شما با دشمنان شما جنگ کرده شما رانجات دهد.»
\par 5 و سروران، قوم را خطاب کرده، گویند: «کیست که خانه نو بنا کرده، آن راتخصیص نکرده است؛ او روانه شده، به خانه خودبرگردد، مبادا در جنگ بمیرد و دیگری آن راتخصیص نماید.
\par 6 و کیست که تاکستانی غرس نموده، آن را حلال نکرده است؛ او روانه شده، به خانه خود برگردد، مبادا در جنگ بمیرد، ودیگری آن را حلال کند.
\par 7 و کیست که دختری نامزد کرده، به نکاح در نیاورده است، او روانه شده، به خانه خود برگردد، مبادا در جنگ بمیرد ودیگری او را به نکاح درآورد.» 
\par 8 و سروران نیزقوم را خطاب کرده، گویند: «کیست که ترسان وضعیف دل است؛ او روانه شده، به خانه‌اش برگردد، مبادا دل برادرانش مثل دل او گداخته شود.»
\par 9 و چون سروران از تکلم نمودن به قوم فارغ شوند، بر سر قوم، سرداران لشکر مقررسازند.
\par 10 چون به شهری نزدیک آیی تا با آن جنگ نمایی، آن را برای صلح ندا بکن.
\par 11 و اگر تو راجواب صلح بدهد، و دروازه‌ها را برای توبگشاید، آنگاه تمامی قومی که در آن یافت شوند، به تو جزیه دهند و تو را خدمت نمایند.
\par 12 و اگر باتو صلح نکرده، با تو جنگ نمایند، پس آن رامحاصره کن.
\par 13 و چون یهوه، خدایت، آن را به‌دست تو بسپارد، جمیع ذکورانش را به دم شمشیربکش.
\par 14 لیکن زنان و اطفال و بهایم و آنچه درشهر باشد، یعنی تمامی غنیمتش را برای خود به تاراج ببر، و غنایم دشمنان خود را که یهوه خدایت به تو دهد، بخور.
\par 15 به همه شهرهایی که از تو بسیار دورند که از شهرهای این امتها نباشند، چنین رفتار نما.
\par 16 اما از شهرهای این امتهایی که یهوه، خدایت، تو را به ملکیت می‌دهد، هیچ ذی نفس را زنده مگذار.
\par 17 بلکه ایشان را، یعنی حتیان و اموریان و کنعانیان و فرزیان و حویان ویبوسیان را، چنانکه یهوه، خدایت، تو را امرفرموده است، بالکل هلاک ساز.
\par 18 تا شما راتعلیم ندهند که موافق همه رجاساتی که ایشان باخدایان خود عمل می‌نمودند، عمل نمایید. و به یهوه، خدای خود، گناه کنید.
\par 19 چون برای گرفتن شهری با آن جنگ کنی، وآن را روزهای بسیار محاصره نمایی، تبر بر درختهایش مزن و آنها را تلف مساز. چونکه ازآنها می‌خوری پس آنها را قطع منما، زیرا آیادرخت صحرا انسان است تا آن را محاصره نمایی؟و اما درختی که میدانی درختی نیست که از آن خورده شود آن را تلف ساخته، قطع نما وسنگری بر شهری که با تو جنگ می‌کند، بنا کن تامنهدم شود.
\par 20 و اما درختی که میدانی درختی نیست که از آن خورده شود آن را تلف ساخته، قطع نما وسنگری بر شهری که با تو جنگ می‌کند، بنا کن تامنهدم شود.
 
\chapter{21}

\par 1 اگر در زمینی که یهوه، خدایت، برای تصرفش به تو می‌دهد مقتولی درصحرا افتاده، پیدا شود و معلوم نباشد که قاتل اوکیست،
\par 2 آنگاه مشایخ و داوران تو بیرون آمده، مسافت شهرهایی را که در اطراف مقتول است، بپیمایند.
\par 3 و اما شهری که نزدیک تر به مقتول است، مشایخ آن شهر گوساله رمه را که با آن خیش نزده، و یوغ به آن نبسته‌اند، بگیرند.
\par 4 ومشایخ آن شهر آن گوساله را در وادی‌ای که آب در آن همیشه جاری باشد و در آن خیش نزده، وشخم نکرده باشند، فرود آورند، و آنجا در وادی، گردن گوساله را بشکنند.
\par 5 و بنی لاوی کهنه نزدیک بیایند، چونکه یهوه خدایت ایشان رابرگزیده است تا او را خدمت نمایند، و به نام خداوند برکت دهند، و برحسب قول ایشان هرمنازعه و هر آزاری فیصل پذیرد.
\par 6 و جمیع مشایخ آن شهری که نزدیک تر به مقتول است، دستهای خود را بر گوساله‌ای که گردنش در وادی شکسته شده است، بشویند.
\par 7 و جواب داده، بگویند: «دستهای ما این خون را نریخته، وچشمان ما ندیده است.
\par 8 ‌ای خداوند قوم خوداسرائیل را که فدیه داده‌ای بیامرز، و مگذار که خون بی‌گناه در میان قوم تو اسرائیل بماند.» پس خون برای ایشان عفو خواهد شد.
\par 9 پس خون بی‌گناه را از میان خود رفع کرده‌ای هنگامی که آنچه در نظر خداوند راست است به عمل آورده‌ای.
\par 10 چون بیرون روی تا با دشمنان خود جنگ کنی، و یهوه خدایت ایشان را به‌دستت تسلیم نماید و ایشان را اسیر کنی،
\par 11 و در میان اسیران زن خوب صورتی دیده، عاشق او بشوی وبخواهی او را به زنی خود بگیری،
\par 12 پس او را به خانه خود ببر و او سر خود را بتراشد و ناخن خودرا بگیرد.
\par 13 و رخت اسیری خود را بیرون کرده، در خانه تو بماند، و برای پدر و مادر خود یک ماه ماتم گیرد، و بعد از آن به او درآمده، شوهر او بشوو او زن تو خواهد بود.
\par 14 و اگر از وی راضی نباشی او را به خواهش دلش رها کن، لیکن او را به نقره هرگز مفروش و به او سختی مکن چونکه اورا ذلیل کرده‌ای.
\par 15 و اگر مردی را دو زن باشد یکی محبوبه ویکی مکروهه، و محبوبه و مکروهه هر دو برایش پسران بزایند، و پسر مکروهه نخست زاده باشد،
\par 16 پس در روزی که اموال خود را به پسران خویش تقسیم نماید، نمی تواند پسر محبوبه را برپسر مکروهه که نخست زاده است، حق نخست زادگی دهد.
\par 17 بلکه حصه‌ای مضاعف ازجمیع اموال خود را به پسر مکروهه داده، او رانخست زاده خویش اقرار نماید، زیرا که او ابتدای قوت اوست و حق نخست زادگی از آن اومی باشد.
\par 18 اگر کسی را پسری سرکش و فتنه انگیزباشد، که سخن پدر و سخن مادر خود را گوش ندهد، و هر‌چند او را تادیب نمایند ایشان رانشنود،
\par 19 پدر و مادرش او را گرفته، نزد مشایخ شهرش به دروازه محله‌اش بیاورند.
\par 20 و به مشایخ شهرش گویند: «این پسر ما سرکش وفتنه انگیز است، سخن ما را نمی شنود و مسرف ومیگسار است.»
\par 21 پس جمیع اهل شهرش او را به سنگ سنگسار کنند تا بمیرد، پس بدی را از میان خود دور کرده‌ای و تمامی اسرائیل چون بشنوند، خواهند ترسید.
\par 22 و اگر کسی گناهی را که مستلزم موت است، کرده باشد و کشته شود، و او را به دارکشیده باشی،بدنش در شب بر دار نماند. او راالبته در همان روز دفن کن، زیرا آنکه بر دارآویخته شود ملعون خدا است تا زمینی را که یهوه، خدایت، تو را به ملکیت می‌دهد، نجس نسازی.
\par 23 بدنش در شب بر دار نماند. او راالبته در همان روز دفن کن، زیرا آنکه بر دارآویخته شود ملعون خدا است تا زمینی را که یهوه، خدایت، تو را به ملکیت می‌دهد، نجس نسازی.
 
\chapter{22}

\par 1 اگر گاو یا گوسفند برادر خود را گم شده بینی، از او رومگردان. آن را البته نزدبرادر خود برگردان.
\par 2 و اگر برادرت نزدیک تونباشد یا او را نشناسی، آن را به خانه خود بیاور ونزد تو بماند، تا برادرت آن را طلب نماید، آنگاه آن را به او رد نما.
\par 3 و به الاغ او چنین کن و به لباسش چنین عمل نما و به هر چیز گمشده برادرت که از او گم شود و یافته باشی چنین عمل نما، نمی توانی از او روگردانی.
\par 4 اگر الاغ یا گاو برادرت را در راه افتاده بینی، از آن رومگردان، البته آن را با او برخیزان.
\par 5 متاع مرد بر زن نباشد، و مرد لباس زن رانپوشد، زیرا هر‌که این کار را کند مکروه یهوه خدای توست.
\par 6 اگر اتفاق آشیانه مرغی در راه به نظر تو آید، خواه بر درخت یا بر زمین، و در آن بچه‌ها یاتخمها باشد، و مادر بر بچه‌ها یا تخمها نشسته، مادر را با بچه‌ها مگیر.
\par 7 مادر را البته رها کن وبچه‌ها را برای خود بگیر، تا برای تو نیکو شود وعمر دراز کنی.
\par 8 چون خانه نو بنا کنی، بر پشت بام خوددیواری بساز، مبادا کسی از آن بیفتد و خون برخانه خود بیاوری.
\par 9 در تاکستان خود دو قسم تخم مکار، مباداتمامی آن، یعنی هم تخمی که کاشته‌ای و هم محصول تاکستان، وقف شود.
\par 10 گاو و الاغ را با هم جفت کرده، شیار منما.
\par 11 پارچه مختلط از پشم و کتان با هم مپوش.
\par 12 بر چهار گوشه رخت خود که خود را به آن می‌پوشانی، رشته‌ها بساز.
\par 13 اگر کسی برای خود زنی گیرد و چون بدودرآید، او را مکروه دارد.
\par 14 و اسباب حرف بدونسبت داده، از او اسم بد شهرت دهد و گوید این زن را گرفتم و چون به او نزدیکی نمودم، او را باکره نیافتم.
\par 15 آنگاه پدر یا مادر آن دختر علامت بکارت دختر برداشته، نزد مشایخ شهر نزددروازه بیاورند.
\par 16 و پدر دختر به مشایخ بگوید: «دختر خود را به این مرد به زنی داده‌ام، و از اوکراهت دارد،
\par 17 و اینک اسباب حرف بدو نسبت داده، می‌گوید دختر تو را باکره نیافتم، و علامت بکارت دختر من این است.» پس جامه را پیش مشایخ شهر بگسترانند.
\par 18 پس مشایخ آن شهرآن مرد را گرفته، تنبیه کنند.
\par 19 و او را صد مثقال نقره جریمه نموده، به پدر دختر بدهند چونکه برباکره اسرائیل بدنامی آورده است. و او زن وی خواهد بود و در تمامی عمرش نمی تواند او رارها کند.
\par 20 لیکن اگر این سخن راست باشد، و علامت بکارت آن دختر پیدا نشود،
\par 21 آنگاه دختر را نزددر خانه پدرش بیرون آورند، و اهل شهرش او رابا سنگ سنگسار نمایند تا بمیرد، چونکه در خانه پدر خود زنا کرده، در اسرائیل قباحتی نموده است. پس بدی را از میان خود دور کرده‌ای.
\par 22 اگر مردی یافت شود که با زن شوهرداری همبستر شده باشد، پس هر دو یعنی مردی که بازن خوابیده است و زن، کشته شوند. پس بدی رااز اسرائیل دور کرده‌ای.
\par 23 اگر دختر باکره‌ای به مردی نامزد شود ودیگری او را در شهر یافته، با او همبستر شود،
\par 24 پس هر دو ایشان را نزد دروازه شهر بیرون آورده، ایشان را با سنگها سنگسار کنند تا بمیرند؛ اما دختر را چونکه در شهر بود و فریاد نکرد، ومرد را چونکه زن همسایه خود را ذلیل ساخت، پس بدی را از میان خود دور کرده‌ای.
\par 25 اما اگر آن مرد دختری نامزد را در صحرایابد و آن مرد به او زور آورده، با او بخوابد، پس آن مرد که با او خوابیده، تنها کشته شود.
\par 26 و اما بادختر هیچ مکن زیرا بر دختر، گناه مستلزم موت نیست، بلکه این مثل آن است که کسی بر همسایه خود برخاسته، او را بکشد.
\par 27 چونکه او را درصحرا یافت و دختر نامزد فریاد برآورد و برایش رهاننده‌ای نبود.
\par 28 و اگر مردی دختر باکره‌ای را که نامزدنباشد بیابد و او را گرفته، با او همبستر شود وگرفتار شوند،
\par 29 آنکه آن مرد که با او خوابیده است پنجاه مثقال نقره به پدر دختر بدهد و آن دختر زن او باشد، چونکه او را ذلیل ساخته است و در تمامی عمرش نمی تواند او را رها کند.هیچ‌کس زن پدر خود را نگیرد و دامن پدرخود را منکشف نسازد.
\par 30 هیچ‌کس زن پدر خود را نگیرد و دامن پدرخود را منکشف نسازد.
 
\chapter{23}

\par 1 جماعت خداوند شخصی که کوبیده بیضه و آلت بریده باشد داخل جماعت خداوند نشود.
\par 2 حرام زاده‌ای داخل جماعت خداوند نشود، حتی تا پشت دهم احدی از او داخل جماعت خداوند نشود.
\par 3 عمونی و موآبی داخل جماعت خداوند نشوند. حتی تا پشت دهم، احدی ازایشان هرگز داخل جماعت خداوند نشود.
\par 4 زیراوقتی که شما از مصر بیرون آمدید، شما را در راه به نان و آب استقبال نکردند، و از این جهت که بلعام بن بعور را از فتور ارام نهرین اجیر کردند تا تو را لعنت کند.
\par 5 لیکن یهوه خدایت نخواست بلعام را بشنود، پس یهوه خدایت لعنت را به جهت تو، به برکت تبدیل نمود، چونکه یهوه خدایت تو را دوست می‌داشت.
\par 6 ابد در تمامی عمرت جویای خیریت و سعادت ایشان مباش.
\par 7 ادومی را دشمن مدار چونکه برادر توست، ومصری را دشمن مدار چونکه در زمین وی غریب بودی.
\par 8 اولادی که از ایشان زاییده شوند درپشت سوم داخل جماعت خداوند شوند.
\par 9 چون در اردو به مقابله دشمنانت بیرون روی خویشتن را از هر چیز بد نگاه دار.
\par 10 اگر در میان شما کسی باشد که از احتلام شب نجس شود، از اردو بیرون رود و داخل اردونشود.
\par 11 چون شب نزدیک شود با آب غسل کند، و چون آفتاب غروب کند، داخل اردو شود.
\par 12 و تو را مکانی بیرون از اردو باشد تا به آنجابیرون روی.
\par 13 و در میان اسباب تو میخی باشد، و چون بیرون می‌نشینی با آن بکن و برگشته، فضله خود را از آن بپوشان.
\par 14 زیرا که یهوه خدایت درمیان اردوی تو می‌خرامد تا تو را رهایی داده، دشمنانت را به تو تسلیم نماید، پس اردوی تومقدس باشد، مبادا چیز پلید را در میان تو دیده، ازتو روگرداند.
\par 15 غلامی را که از آقای خود نزد تو بگریزد به آقایش مسپار.
\par 16 با تو در میان تو در مکانی که برگزینند در یکی از شهرهای تو که به نظرش پسند آید، ساکن شود. و بر او جفامنما.
\par 17 از دختران اسرائیل فاحشه‌ای نباشد و ازپسران اسرائیل لواطی نباشد.
\par 18 اجرت فاحشه وقیمت سگ را برای هیچ نذری به خانه یهوه خدایت میاور، زیرا که این هر دو نزد یهوه خدایت مکروه است. 
\par 19 برادر خود را به سود قرض مده نه به سودنقره و نه به سود آذوقه و نه به سود هر چیزی که به سود داده می‌شود.
\par 20 غریب را می‌توانی به سودقرض بدهی، اما برادر خود را به سود قرض مده تایهوه خدایت در زمینی که برای تصرفش داخل آن می‌شوی تو را به هر‌چه دستت را بر آن درازمی کنی، برکت دهد.
\par 21 چون نذری برای یهوه خدایت می‌کنی دروفای آن تاخیر منما، زیرا که یهوه خدایت البته آن را از تو مطالبه خواهد نمود، و برای تو گناه خواهد بود.
\par 22 اما اگر از نذر کردن ابا نمایی، تو راگناه نخواهد بود.
\par 23 آنچه را که از دهانت بیرون آید، هوشیار باش که بجا آوری، موافق آنچه برای یهوه خدایت از اراده خود نذر کرده‌ای و به زبان خود گفته‌ای.
\par 24 چون به تاکستان همسایه خود درآیی، ازانگور، هر‌چه می‌خواهی به سیری بخور، اما درظرف خود هیچ مگذار.چون به کشتزار همسایه خود داخل شوی، خوشه‌ها را به‌دست خود بچین، اما داس بر کشت همسایه خود مگذار.
\par 25 چون به کشتزار همسایه خود داخل شوی، خوشه‌ها را به‌دست خود بچین، اما داس بر کشت همسایه خود مگذار.
 
\chapter{24}

\par 1 چون کسی زنی گرفته، به نکاح خوددرآورد، اگر در نظر او پسند نیاید از این که چیزی ناشایسته در او بیابد آنگاه طلاق نامه‌ای نوشته، بدستش دهد، و او را از خانه‌اش رها کند.
\par 2 و از خانه او روانه شده، برود و زن دیگری شود.
\par 3 و اگر شوهر دیگر نیز او را مکروه دارد وطلاق نامه‌ای نوشته، به‌دستش بدهد و او را ازخانه‌اش رها کند، یا اگر شوهری دیگر که او را به زنی گرفت، بمیرد،
\par 4 شوهر اول که او را رها کرده بود، نمی تواند دوباره او را به نکاح خود درآورد. بعد از آن ناپاک شده است، زیرا که این به نظرخداوند مکروه است. پس بر زمینی که یهوه، خدایت، تو را به ملکیت می‌دهد، گناه میاور.
\par 5 چون کسی زن تازه‌ای بگیرد، در لشکر بیرون نرود، و هیچ کار به او تکلیف نشود، تا یک سال درخانه خود آزاد بماند، و زنی را که گرفته است، مسرور سازد.
\par 6 هیچکس آسیا یا سنگ بالایی آن را به گرونگیرد، زیرا که جان را به گرو گرفته است.
\par 7 اگر کسی یافت شود که یکی از برادران خوداز بنی‌اسرائیل را دزدیده، بر او ظلم کند یابفروشد، آن دزد کشته شود، پس بدی را از میان خود دور کرده‌ای.
\par 8 درباره بلای برص هوشیار باش که به هرآنچه لاویان کهنه شما را تعلیم دهند به دقت توجه نموده، عمل نمایید، و موافق آنچه به ایشان امر فرمودم، هوشیار باشید که عمل نمایید.
\par 9 بیادآور که یهوه خدایت در راه با مریم چه کرد، وقتی که شما از مصر بیرون آمدید.
\par 10 چون به همسایه خود هر قسم قرض دهی، برای گرفتن گرو به خانه‌اش داخل مشو.
\par 11 بلکه بیرون بایست تا شخصی که به او قرض می‌دهی گرو را نزد تو بیرون آورد.
\par 12 و اگر مرد فقیر باشددر گرو او مخواب.
\par 13 البته به وقت غروب آفتاب، گرو را به او پس بده، تا در رخت خود بخوابد و تورا برکت دهد و به حضور یهوه خدایت، عدالت شمرده خواهد شد.
\par 14 بر مزدوری که فقیر و مسکین باشد، خواه ازبرادرانت و خواه از غریبانی که در زمینت دراندرون دروازه های تو باشند ظلم منما.
\par 15 درهمان روز مزدش را بده، و آفتاب بر آن غروب نکند، چونکه او فقیر است و دل خود را به آن بسته است، مبادا بر تو نزد خداوند فریاد برآورد و برای تو گناه باشد.
\par 16 پدران به عوض پسران کشته نشوند، و نه پسران به عوض پدران خود کشته شوند. هر کس برای گناه خود کشته شود.
\par 17 داوری غریب و یتیم را منحرف مساز، وجامه بیوه را به گرو مگیر.
\par 18 و بیاد آور که در مصرغلام بودی و یهوه، خدایت، تو را از آنجا فدیه داد. بنابراین من تو را امر می‌فرمایم که این کار رامعمول داری.
\par 19 چون محصول خود را در مزرعه خویش درو کنی، و در مزرعه، بافه‌ای فراموش کنی، برای برداشتن آن برمگرد؛ برای غریب و یتیم و بیوه‌زن باشد تا یهوه خدایت تو را در همه کارهای دستت برکت دهد.
\par 20 چون زیتون خود را بتکانی باردیگر شاخه‌ها را متکان؛ برای غریب و یتیم و بیوه باشد.
\par 21 چون انگور تاکستان خود را بچینی باردیگر آن را مچین، برای غریب و یتیم و بیوه باشد.و بیاد آور که در زمین مصر غلام بودی. بنابراین تو را امر می‌فرمایم که این کار را معمول داری.
\par 22 و بیاد آور که در زمین مصر غلام بودی. بنابراین تو را امر می‌فرمایم که این کار را معمول داری.
 
\chapter{25}

\par 1 اگر در میان مردم مرافعه‌ای باشد و به محاکمه آیند و در میان ایشان داوری نمایند، آنگاه عادل را عادل شمارند، و شریر راملزم سازند.
\par 2 و اگر شریر مستوجب تازیانه باشدآنگاه داور، او را بخواباند و حکم دهد تا او راموافق شرارتش به حضور خود به شماره بزنند.
\par 3 چهل تازیانه او را بزند و زیاد نکند، مبادا اگر ازاین زیاده کرده، تازیانه بسیار زند، برادرت در نظرتو خوار شود.
\par 4 دهن گاو را هنگامی که خرمن را خرد می‌کند، مبند.
\par 5 اگر برادران با هم ساکن باشند و یکی از آنهابی اولاد بمیرد، پس زن آن متوفی، خارج به شخص بیگانه داده نشود، بلکه برادر شوهرش به او درآمده، او را برای خود به زنی بگیرد، و حق برادر شوهری را با او بجا آورد.
\par 6 ونخست زاده‌ای که بزاید به اسم برادر متوفای اووارث گردد، تا اسمش از اسرائیل محو نشود.
\par 7 واگر آن مرد به گرفتن زن برادرش راضی نشود، آنگاه زن برادرش به دروازه نزد مشایخ برود وگوید: «برادر شوهر من از برپا داشتن اسم برادرخود در اسرائیل انکار می‌کند، و از بجا آوردن حق برادر شوهری با من ابا می‌نماید.»
\par 8 پس مشایخ شهرش او را طلبیده، با وی گفتگو کنند، و اگر اصرار کرده، بگوید نمی خواهم او را بگیرم،
\par 9 آنگاه زن برادرش نزد وی آمده، به حضورمشایخ کفش او را از پایش بکند، و به رویش آب دهن اندازد، و در جواب گوید: «با کسی‌که خانه برادر خود را بنا نکند، چنین کرده شود.»
\par 10 ونام او در اسرائیل، خانه کفش کنده خوانده شود.
\par 11 و اگر دو شخص با یکدیگر منازعه نمایند، و زن یکی پیش آید تا شوهر خود را از دست زننده‌اش رها کند و دست خود را دراز کرده، عورت او را بگیرد،
\par 12 پس دست او را قطع کن. چشم تو بر او ترحم نکند.
\par 13 در کیسه تو وزنه های مختلف، بزرگ وکوچک نباشد.
\par 14 در خانه تو کیلهای مختلف، بزرگ و کوچک، نباشد.
\par 15 تو را وزن صحیح وراست باشد و تو را کیل صحیح و راست باشد، تاعمرت در زمینی که یهوه، خدایت، به تومی دهد دراز شود.
\par 16 زیرا هر‌که این کار کندیعنی هر‌که بی‌انصافی نماید، نزد یهوه خدایت مکروه است.
\par 17 بیاد آور آنچه عمالیق وقت بیرون آمدنت از مصر در راه به تو نمودند.
\par 18 که چگونه تو رادر راه، مقابله کرده، همه واماندگان را در عقب تو از موخرت قطع نمودند، در حالی که توضعیف و وامانده بودی و از خدا نترسیدند.پس چون یهوه خدایت تو را در زمینی که یهوه، خدایت، تو را برای تصرفش نصیب می‌دهد، از جمیع دشمنانت آرامی بخشد، آنگاه ذکر عمالیق را از زیر آسمان محو ساز و فراموش مکن.
\par 19 پس چون یهوه خدایت تو را در زمینی که یهوه، خدایت، تو را برای تصرفش نصیب می‌دهد، از جمیع دشمنانت آرامی بخشد، آنگاه ذکر عمالیق را از زیر آسمان محو ساز و فراموش مکن.
 
\chapter{26}

\par 1 و چون به زمینی که یهوه خدایت تو رانصیب می‌دهد داخل شدی، و در آن تصرف نموده، ساکن گردیدی،
\par 2 آنگاه نوبر تمامی حاصل زمین را که از زمینی که یهوه خدایت به تومی دهد، جمع کرده باشی بگیر، و آن را در سبدگذاشته، به مکانی که یهوه خدایت برگزیند تا نام خود را در آن ساکن گرداند، برو.
\par 3 و نزد کاهنی که در آن روزها باشد رفته، وی را بگو: «امروز برای یهوه خدایت اقرار می‌کنم که به زمینی که خداوندبرای پدران ما قسم خورد که به ما بدهد، داخل شده‌ام.»
\par 4 و کاهن سبد را از دستت گرفته، پیش مذبح یهوه خدایت بگذارد.
\par 5 پس تو به حضوریهوه خدای خود اقرار کرده، بگو: «پدر من ارامی آواره بود، و با عددی قلیل به مصر فرود شده، درآنجا غربت پذیرفت، و در آنجا امتی بزرگ وعظیم و کثیر شد.
\par 6 و مصریان با ما بدرفتاری نموده، ما را ذلیل ساختند، و بندگی سخت بر مانهادند.
\par 7 و چون نزد یهوه، خدای پدران خود، فریاد برآوردیم، خداوند آواز ما را شنید ومشقت و محنت و تنگی ما را دید.
\par 8 و خداوند مارا از مصر به‌دست قوی و بازوی افراشته و خوف عظیم، و با آیات و معجزات بیرون آورد.
\par 9 و ما رابه این مکان درآورده، این زمین را زمینی که به شیرو شهد جاری است به ما بخشید.
\par 10 و الان اینک نوبر حاصل زمینی را که تو‌ای خداوند به من دادی، آورده‌ام.» پس آن را به حضور یهوه خدای خود بگذار، و به حضور یهوه، خدایت، عبادت نما.
\par 11 و تو با لاوی و غریبی که در میان تو باشد ازتمامی نیکویی که یهوه، خدایت، به تو و به خاندانت بخشیده است، شادی خواهی نمود.
\par 12 و در سال سوم که سال عشر است، چون ازگرفتن تمامی عشر محصول خود فارغ شدی، آن را به لاوی و غریب و یتیم و بیوه‌زن بده، تا دراندرون دروازه های تو بخورند و سیر شوند.
\par 13 وبه حضور یهوه خدایت بگو: «موقوفات را از خانه خود بیرون کردم، و آنها را نیز به لاوی و غریب ویتیم و بیوه‌زن، موافق تمامی اوامری که به من امرفرمودی، دادم، و از اوامر تو تجاوز ننموده، فراموش نکردم.
\par 14 در ماتم خود از آنها نخوردم ودر نجاستی از آنها صرف ننمودم، و برای اموات از آنها ندادم، بلکه به قول یهوه، خدایم، گوش داده، موافق هر‌آنچه به من امر فرمودی، رفتارنمودم.
\par 15 از مسکن مقدس خود از آسمان بنگر، و قوم خود اسرائیل و زمینی را که به ما دادی چنانکه برای پدران ما قسم خوردی، زمینی که به شیر و شهد جاری است، برکت بده،.»
\par 16 امروز یهوه، خدایت، تو را امر می‌فرمایدکه این فرایض و احکام را بجا آوری، پس آنها رابه تمامی دل و تمامی جان خود نگاه داشته، بجاآور.
\par 17 امروز به یهوه اقرار نمودی که خدای توست، و اینکه به طریقهای او سلوک خواهی نمود، و فرایض و اوامر و احکام او را نگاه داشته، آواز او را خواهی شنید.
\par 18 و خداوند امروز به تواقرار کرده است که تو قوم خاص او هستی، چنانکه به تو وعده داده است، و تا تمامی اوامر اورا نگاه داری.و تا تو را در ستایش و نام و اکرام از جمیع امتهایی که ساخته است، بلند گرداند، وتا برای یهوه، خدایت، قوم مقدس باشی، چنانکه وعده داده است.
\par 19 و تا تو را در ستایش و نام و اکرام از جمیع امتهایی که ساخته است، بلند گرداند، وتا برای یهوه، خدایت، قوم مقدس باشی، چنانکه وعده داده است.
 
\chapter{27}

\par 1 و موسی و مشایخ اسرائیل، قوم را امرفرموده، گفتند: «تمامی اوامری را که من امروز به شما امر می‌فرمایم، نگاه دارید.
\par 2 و درروزی که از اردن به زمینی که یهوه، خدایت، به تومی دهد عبور کنید، برای خود سنگهای بزرگ برپاکرده، آنها را با گچ بمال.
\par 3 و بر آنها تمامی کلمات این شریعت را بنویس، هنگامی که عبور نمایی تابه زمینی که یهوه، خدایت، به تو می‌دهد، داخل شوی، زمینی که به شیر و شهد جاری است، چنانکه یهوه خدای پدرانت به تو وعده داده است.
\par 4 و چون از اردن عبور نمودی این سنگها راکه امروز به شما امر می‌فرمایم در کوه عیبال برپاکرده، آنها را با گچ بمال.
\par 5 و در آنجا مذبحی برای یهوه خدایت بنا کن، و مذبح از سنگها باشد و آلت آهنین بر آنها بکار مبر.
\par 6 مذبح یهوه خدای خودرا از سنگهای ناتراشیده بنا کن، و قربانی های سوختنی برای یهوه خدایت، بر آن بگذران.
\par 7 وذبایح سلامتی ذبح کرده، در آنجا بخور و به حضور یهوه خدایت شادی نما.
\par 8 و تمامی کلمات این شریعت را بر آن به خط روشن بنویس.»
\par 9 پس موسی و لاویان کهنه تمامی اسرائیل را خطاب کرده، گفتند:
\par 10 «ای اسرائیل خاموش باش و بشنو. امروز قوم یهوه خدایت شدی.
\par 11 پس آواز یهوه خدایت را بشنو و اوامر وفرایض او را که من امروز به تو امر می‌فرمایم، بجاآر.» 
\par 12 «چون از اردن عبور کردید، اینان یعنی شمعون و لاوی و یهودا و یساکار و یوسف وبنیامین بر کوه جرزیم بایستند تا قوم را برکت دهند.
\par 13 و اینان یعنی روبین و جاد و اشیر وزبولون و دان و نفتالی بر کوه عیبال بایستند تانفرین کنند.
\par 14 و لاویان جمیع مردان اسرائیل رابه آواز بلند خطاب کرده، گویند:
\par 15 «ملعون باد کسی‌که صورت تراشیده یاریخته شده از صنعت دست کارگر که نزد خداوندمکروه است، بسازد، و مخفی نگاه دارد.» و تمامی قوم در جواب بگویند: «آمین!»
\par 16 «ملعون باد کسی‌که با پدر و مادر خود به خفت رفتار نماید.» و تمامی قوم بگویند: «آمین!»
\par 17 «ملعون باد کسی‌که حد همسایه خود راتغییر دهد.» و تمامی قوم بگویند: «آمین!»
\par 18 «ملعون باد کسی‌که نابینا را از راه منحرف سازد.» و تمامی قوم بگویند: «آمین!»
\par 19 «ملعون باد کسی‌که داوری غریب و یتیم وبیوه را منحرف سازد.» و تمامی قوم بگویند: «آمین!»
\par 20 «ملعون باد کسی‌که با زن پدر خود همبسترشود، چونکه دامن پدر خود را کشف نموده است.» و تمامی قوم بگویند: «آمین!»
\par 21 «ملعون باد کسی‌که با هر قسم بهایمی بخوابد.» و تمامی قوم بگویند: «آمین!»
\par 22 «ملعون باد کسی‌که با خواهر خویش چه دختر پدر و چه دختر مادر خویش بخوابد.» وتمامی قوم بگویند: «آمین!»
\par 23 «ملعون باد کسی‌که با مادر‌زن خودبخوابد.» و تمامی قوم بگویند: «آمین!»
\par 24 «ملعون باد کسی‌که همسایه خود را درپنهانی بزند.» و تمامی قوم بگویند: «آمین!»
\par 25 «ملعون باد کسی‌که رشوه گیرد تا خون بی‌گناهی ریخته شود.» و تمامی قوم بگویند: «آمین!»«ملعون باد کسی‌که کلمات این شریعت رااثبات ننماید تا آنها را بجا نیاورد.» وتمامی قوم بگویند: «آمین!»
\par 26 «ملعون باد کسی‌که کلمات این شریعت رااثبات ننماید تا آنها را بجا نیاورد.» وتمامی قوم بگویند: «آمین!»
 
\chapter{28}

\par 1 «و اگر آواز یهوه خدای خود را به دقت بشنوی تا هوشیار شده، تمامی اوامر اورا که من امروز به تو امر می‌فرمایم بجا آوری، آنگاه یهوه خدایت تو را بر جمیع امتهای جهان بلند خواهد گردانید.
\par 2 و تمامی این برکتها به توخواهد رسید و تو را خواهد دریافت، اگر آوازیهوه خدای خود را بشنوی.
\par 3 در شهر، مبارک ودر صحرا، مبارک خواهی بود.
\par 4 میوه بطن تو ومیوه زمین تو و میوه بهایمت و بچه های گاو وبره های گله تو مبارک خواهند بود.
\par 5 سبد و ظرف خمیر تو مبارک خواهد بود.
\par 6 وقت درآمدنت مبارک، و وقت بیرون رفتنت مبارک خواهی بود.
\par 7 «و خداوند دشمنانت را که با تو مقاومت نمایند، از حضور تو منهزم خواهد ساخت، ازیک راه بر تو خواهند آمد، و از هفت راه پیش توخواهند گریخت.
\par 8 خداوند در انبارهای تو و به هر‌چه دست خود را به آن دراز می‌کنی بر توبرکت خواهد فرمود، و تو را در زمینی که یهوه خدایت به تو می‌دهد، مبارک خواهد ساخت.
\par 9 واگر اوامر یهوه خدای خود را نگاهداری، و درطریقهای او سلوک نمایی، خداوند تو را برای خود قوم مقدس خواهد گردانید، چنانکه برای تو قسم خورده است.
\par 10 و جمیع امتهای زمین خواهند دید که نام خداوند بر تو خوانده شده است، و از تو خواهند ترسید.
\par 11 و خداوندتو را در میوه بطنت و ثمره بهایمت ومحصول زمینت، در زمینی که خداوند برای پدرانت قسم خورد که به تو بدهد، به نیکویی خواهد افزود.
\par 12 و خداوند خزینه نیکوی خود، یعنی آسمان را برای تو خواهد گشود، تا باران زمین تو را در موسمش بباراند، و تو را درجمیع اعمال دستت مبارک سازد، و به امتهای بسیار قرض خواهی داد، و تو قرض نخواهی گرفت.
\par 13 و خداوند تو را سر خواهد ساخت نه دم، و بلند خواهی بود فقط نه پست، اگر اوامریهوه خدای خود را که من امروز به تو امرمی فرمایم بشنوی، و آنها را نگاه داشته، بجاآوری.
\par 14 و از همه سخنانی که من امروز به تو امرمی کنم به طرف راست یا چپ میل نکنی، تاخدایان غیر را پیروی نموده، آنها را عبادت کنی.
\par 15 «و اما اگر آواز یهوه خدای خود را نشنوی تا هوشیار شده، همه اوامر و فرایض او را که من امروز به تو امر می‌فرمایم بجا آوری، آنگاه جمیع این لعنتها به تو خواهد رسید، و تو را خواهددریافت.
\par 16 در شهر ملعون، و در صحرا ملعون خواهی بود.
\par 17 سبد و ظرف خمیر تو ملعون خواهد بود.
\par 18 میوه بطن تو و میوه زمین تو وبچه های گاو و بره های گله تو ملعون خواهد بود.
\par 19 وقت درآمدنت ملعون، و وقت بیرون رفتنت ملعون خواهی بود.
\par 20 و به هر‌چه دست خود رابرای عمل نمودن دراز می‌کنی خداوند بر تولعنت و اضطراب و سرزنش خواهد فرستاد تا به زودی هلاک و نابود شوی، به‌سبب بدی کارهایت که به آنها مرا ترک کرده‌ای،
\par 21 خداوندوبا را بر تو ملصق خواهد ساخت، تا تو را اززمینی که برای تصرفش به آن داخل می‌شوی، هلاک سازد.
\par 22 و خداوند تو را با سل و تب والتهاب و حرارت و شمشیر و باد سموم و یرقان خواهد زد، و تو را تعاقب خواهند نمود تا هلاک شوی.
\par 23 و فلک تو که بالای سر تو است مس خواهد شد، و زمینی که زیر تو است آهن.
\par 24 وخداوند باران زمینت را گرد و غبار خواهدساخت، که از آسمان بر تو نازل شود تا هلاک شوی.
\par 25 «و خداوند تو را پیش روی دشمنانت منهزم خواهد ساخت. از یک راه بر ایشان بیرون خواهی رفت، و از هفت راه از حضور ایشان خواهی گریخت، و در تمامی ممالک جهان به تلاطم خواهی افتاد.
\par 26 و بدن شما برای همه پرندگان هوا و بهایم زمین خوراک خواهد بود، وهیچ‌کس آنها را دور نخواهد کرد.
\par 27 خداوند تو را به دنبل مصر و خراج و جرب و خارشی که تواز آن شفا نتوانی یافت، مبتلا خواهد ساخت.
\par 28 خداوند تو را به دیوانگی و نابینایی و پریشانی دل مبتلا خواهد ساخت.
\par 29 و در وقت ظهر مثل کوری که در تاریکی لمس نماید کورانه راه خواهی رفت، و در راههای خود کامیاب نخواهی شد، بلکه در تمامی روزهایت مظلوم وغارت شده خواهی بود، و نجات‌دهنده‌ای نخواهد بود.
\par 30 زنی را نامزد خواهی کرد ودیگری با او خواهد خوابید. خانه‌ای بنا خواهی کرد و در آن ساکن نخواهی شد. تاکستانی غرس خواهی نمود و میوه‌اش را نخواهی خورد.
\par 31 گاوت در نظرت کشته شود و از آن نخواهی خورد. الاغت پیش روی تو به غارت برده شود وباز به‌دست تو نخواهند آمد. گوسفند تو به دشمنت داده می‌شود و برای تو رهاکننده‌ای نخواهد بود.
\par 32 پسران و دخترانت به امت دیگرداده می‌شوند، و چشمانت نگریسته از آرزوی ایشان تمامی روز کاهیده خواهد شد، و در دست تو هیچ قوه‌ای نخواهد بود.
\par 33 میوه زمینت ومشقت تو را امتی که نشناخته‌ای، خواهند خورد، و همیشه فقط مظلوم و کوفته شده خواهی بود.
\par 34 به حدی که از چیزهایی که چشمت می‌بیند، دیوانه خواهی شد.
\par 35 خداوند زانوها و ساقها واز کف پا تا فرق سر تو را به دنبل بد که از آن شفانتوانی یافت، مبتلا خواهد ساخت.
\par 36 خداوند تورا و پادشاهی را که بر خود نصب می‌نمایی، بسوی امتی که تو و پدرانت نشناخته‌اید، خواهدبرد، و در آنجا خدایان غیر را از چوب و سنگ عبادت خواهی کرد.
\par 37 و در میان تمامی امتهایی که خداوند شما را به آنجا خواهد برد، عبرت ومثل و سخریه خواهی شد.
\par 38 «تخم بسیار به مزرعه خواهی برد، و اندکی جمع خواهی کرد چونکه ملخ آن را خواهدخورد.
\par 39 تاکستانها غرس نموده، خدمت آنها راخواهی کرد، اما شراب را نخواهی نوشید و انگوررا نخواهی چید، زیرا کرم آن را خواهد خورد.
\par 40 تو را در تمامی حدودت درختان زیتون خواهد بود، لکن خویشتن را به زیت تدهین نخواهی کرد، زیرا زیتون تو نارس ریخته خواهدشد.
\par 41 پسران و دختران خواهی آورد، لیکن ازآن تو نخواهند بود، چونکه به اسیری خواهندرفت.
\par 42 تمامی درختانت و محصول زمینت راملخ به تصرف خواهد آورد.
\par 43 غریبی که در میان تو است بر تو به نهایت رفیع و برافراشته خواهدشد، و تو به نهایت پست و متنزل خواهی گردید.
\par 44 او به تو قرض خواهد داد و تو به او قرض نخواهی داد، او سر خواهد بود و تو دم خواهی بود.
\par 45 «و جمیع این لعنتها به تو خواهد رسید، وتو را تعاقب نموده، خواهد دریافت تا هلاک شوی، از این جهت که قول یهوه خدایت را گوش ندادی تا اوامر و فرایضی را که به تو امر فرموده بود، نگاه داری.
\par 46 و تو را و ذریت تو را تا به ابدآیت و شگفت خواهد بود.
\par 47 «از این جهت که یهوه خدای خود را به شادمانی و خوشی دل برای فراوانی همه‌چیزعبادت ننمودی.
\par 48 پس دشمنانت را که خداوندبر تو خواهد فرستاد در گرسنگی و تشنگی وبرهنگی و احتیاج همه‌چیز خدمت خواهی نمود، و یوغ آهنین بر گردنت خواهد گذاشت تا تورا هلاک سازد.
\par 49 و خداوند از دور، یعنی ازاقصای زمین، امتی را که مثل عقاب می‌پرد بر توخواهد آورد، امتی که زبانش را نخواهی فهمید.
\par 50 امتی مهیب صورت که طرف پیران را نگاه ندارد و بر جوانان ترحم ننماید.
\par 51 و نتایج بهایم و محصول زمینت را بخورد تا هلاک شوی. وبرای تو نیز غله و شیره و روغن و بچه های گاو وبره های گوسفند را باقی نگذارد تا تو را هلاک سازد.
\par 52 و تو را در تمامی دروازه هایت محاصره کند تا دیواره های بلند و حصین که بر آنها توکل داری در تمامی زمینت منهدم شود، و تو را درتمامی دروازه هایت، در تمامی زمینی که یهوه خدایت به تو می‌دهد، محاصره خواهد نمود.
\par 53 و میوه بطن خود، یعنی گوشت پسران ودخترانت را که یهوه خدایت به تو می‌دهد درمحاصره و تنگی که دشمنانت تو را به آن زبون خواهند ساخت، خواهی خورد.
\par 54 مردی که درمیان شما نرم و بسیار متنعم است، چشمش بربرادر خود و زن هم آغوش خویش و بقیه فرزندانش که باقی می‌مانند بد خواهد بود.
\par 55 به حدی که به احدی از ایشان از گوشت پسران خودکه می‌خورد نخواهد داد زیرا که در محاصره وتنگی که دشمنانت تو را در تمامی دروازه هایت به آن زبون سازند، چیزی برای او باقی نخواهد ماند.
\par 56 و زنی که در میان شما نازک و متنعم است که به‌سبب تنعم و نازکی خود جرات نمی کرد که کف پای خود را به زمین بگذارد، چشم او بر شوهرهم آغوش خود و پسر و دخترخویش بد خواهدبود.
\par 57 و بر مشیمه‌ای که از میان پایهای او درآیدو بر اولادی که بزاید زیرا که آنها را به‌سبب احتیاج همه‌چیز در محاصره و تنگی که دشمنانت در دروازه هایت به آن تو را زبون سازندبه پنهانی خواهد خورد.»
\par 58 اگر به عمل نمودن تمامی کلمات این شریعت که در این کتاب مکتوب است، هوشیارنشوی و از این نام مجید و مهیب، یعنی یهوه، خدایت، نترسی،
\par 59 آنگاه خداوند بلایای تو وبلایای اولاد تو را عجیب خواهد ساخت، یعنی بلایای عظیم و مزمن و مرضهای سخت و مزمن.
\par 60 و تمامی بیماریهای مصر را که از آنهامی ترسیدی بر تو باز خواهد آورد و به تو خواهدچسبید.
\par 61 و نیز همه مرضها و همه بلایایی که درطومار این شریعت مکتوب نیست آنها را خداوندبر تو مستولی خواهد گردانید تا هلاک شوی.
\par 62 وگروه قلیل خواهید ماند، برعکس آن که مثل ستارگان آسمان کثیر بودید، زیرا که آواز یهوه خدای خود را نشنیدید.
\par 63 و واقع می‌شودچنانکه خداوند بر شما شادی نمود تا به شمااحسان کرده، شما را بیفزاید همچنین خداوند برشما شادی خواهد نمود تا شما را هلاک و نابودگرداند، و ریشه شما از زمینی که برای تصرفش در آن داخل می‌شوید کنده خواهد شد.
\par 64 وخداوند تو را در میان جمیع امتها از کران زمین تاکران دیگرش پراکنده سازد و در آنجا خدایان غیر را از چوب و سنگ که تو و پدرانت نشناخته‌اید، عبادت خواهی کرد.
\par 65 و در میان این امتها استراحت نخواهی یافت و برای کف پایت آرامی نخواهد بود، و در آنجا یهوه تو را دل لرزان و کاهیدگی چشم و پژمردگی جان خواهدداد.
\par 66 و جان تو پیش رویت معلق خواهد بود، وشب و روز ترسناک شده، به‌جان خود اطمینان نخواهی داشت.
\par 67 بامدادان خواهی گفت: کاش که شام می‌بود، و شامگاهان خواهی گفت: کاش که صبح می‌بود، به‌سبب ترس دلت که به آن خواهی ترسید، و به‌سبب رویت چشمت که خواهی دید.و خداوند تو را در کشتیها ازراهی که به تو گفتم آن را دیگر نخواهی دید به مصر باز خواهد آورد، و خود را در آنجا به دشمنان خویش برای غلامی و کنیزی خواهیدفروخت و مشتری نخواهد بود.» 
\par 68 و خداوند تو را در کشتیها ازراهی که به تو گفتم آن را دیگر نخواهی دید به مصر باز خواهد آورد، و خود را در آنجا به دشمنان خویش برای غلامی و کنیزی خواهیدفروخت و مشتری نخواهد بود.»
 
\chapter{29}

\par 1 این است کلمات عهدی که خداوند درزمین موآب به موسی‌امر فرمود که بابنی‌اسرائیل ببندد، سوای آن عهد که با ایشان درحوریب بسته بود.
\par 2 و موسی تمامی اسرائیل را خطاب کرده، به ایشان گفت: «هر‌آنچه خداوند در زمین مصر بافرعون و جمیع بندگانش و تمامی زمینش عمل نمود، شما دیده‌اید.
\par 3 تجربه های عظیم که چشمان تو دید و آیات و آن معجزات عظیم.
\par 4 اماخداوند دلی را که بدانید و چشمانی را که ببینید وگوشهایی را که بشنوید تا امروز به شما نداده است.
\par 5 و شما را چهل سال در بیابان رهبری نمودم که لباس شما مندرس نگردید، و کفشها درپای شما پاره نشد.
\par 6 نان نخورده و شراب ومسکرت ننوشیده‌اید، تا بدانید که من یهوه خدای شما هستم.
\par 7 و چون به اینجا رسیدید، سیحون، ملک حشبون، و عوج، ملک باشان، به مقابله شمابرای جنگ بیرون آمدند و آنها را مغلوب ساختیم.
\par 8 و زمین ایشان را گرفته، به روبینیان وجادیان و نصف سبط منسی به ملکیت دادیم.
\par 9 پس کلمات این عهد را نگاه داشته، بجا آورید تا در هر‌چه کنید کامیاب شوید.
\par 10 «امروز جمیع شما به حضور یهوه، خدای خود حاضرید، یعنی روسای شما و اسباط شما ومشایخ شما و سروران شما و جمیع مردان اسرائیل.
\par 11 و اطفال و زنان شما و غریبی که درمیان اردوی شماست از هیزم شکنان تا آب کشان شما.
\par 12 تا در عهد یهوه خدایت و سوگند او که یهوه خدایت امروز با تو استوار می‌سازد، داخل شوی.
\par 13 تا تو را امروز برای خود قومی برقراردارد، و او خدای تو باشد چنانکه به تو گفته است، و چنانکه برای پدرانت، ابراهیم و اسحاق ویعقوب، قسم خورده است.
\par 14 و من این عهد و این قسم را با شما تنها استوار نمی نمایم.
\par 15 بلکه باآنانی که امروز با ما به حضور یهوه خدای ما دراینجا حاضرند، و هم با آنانی که امروز با ما دراینجا حاضر نیستند.
\par 16 زیرا شما می‌دانید که چگونه در زمین مصر سکونت داشتیم، و چگونه از میان امتهایی که عبور نمودید، گذشتیم.
\par 17 ورجاسات و بتهای ایشان را از چوب و سنگ ونقره و طلا که در میان ایشان بود، دیدید.
\par 18 تا درمیان شما مرد یا زن یا قبیله یا سبطی نباشد که دلش امروز از یهوه خدای ما منحرف گشته، برودو خدایان این طوایف را عبادت نماید، مبادا درمیان شما ریشه‌ای باشد که حنظل و افسنتین بارآورد.
\par 19 «و مبادا چون سخنان این لعنت را بشنود دردلش خویشتن را برکت داده، گوید: هر‌چند درسختی دل خود سلوک می‌نمایم تا سیراب و تشنه را با هم هلاک سازم، مرا سلامتی خواهد بود.
\par 20 خداوند او را نخواهد آمرزید، بلکه در آن وقت خشم و غیرت خداوند بر آن شخص دودافشان خواهد شد، و تمامی لعنتی که در این کتاب مکتوب است، بر آن کس نازل خواهد شد، و خداوند نام او را از زیر آسمان محو خواهدساخت.
\par 21 و خداوند او را از جمیع اسباطاسرائیل برای بدی جدا خواهد ساخت، موافق جمیع لعنتهای عهدی که در این طومار شریعت مکتوب است.
\par 22 و طبقه آینده یعنی فرزندان شما که بعد از شما خواهند برخاست، و غریبانی که از زمین دور می‌آیند، خواهند گفت: هنگامی که بلایای این زمین و بیماریهایی که خداوند به آن می‌رساند ببینند.
\par 23 و تمامی زمین آن را که کبریت و شوره و آتش شده، نه کاشته می‌شود و نه حاصل می‌روید و هیچ علف در آن نمو نمی کندو مثل انقلاب سدوم و عموره و ادمه صبوئیم که خداوند در غضب و خشم خود آنها را واژگون ساخت، گشته است.
\par 24 پس جمیع امتها خواهندگفت: چرا خداوند با این زمین چنین کرده است وشدت این خشم عظیم از چه سبب است؟
\par 25 آنگاه خواهند گفت: از این جهت که عهد یهوه خدای پدران خود را که به وقت بیرون آوردن ایشان از زمین مصر با ایشان بسته بود، ترک کردند.
\par 26 و رفته، خدایان غیر را عبادت نموده، به آنهاسجده کردند، خدایانی را که نشناخته بودند وقسمت ایشان نساخته بود.
\par 27 پس خشم خداوندبر این زمین افروخته شده، تمامی لعنت را که دراین کتاب مکتوب است، بر آن آورد.
\par 28 و خداوندریشه ایشان را به غضب و خشم و غیض عظیم، اززمین ایشان کند و به زمین دیگر انداخت، چنانکه امروز شده است.چیزهای مخفی از آن یهوه خدای ماست و اما چیزهای مکشوف تا به ابد ازآن ما و فرزندان ما است، تا جمیع کلمات این شریعت را به عمل آوریم.
\par 29 چیزهای مخفی از آن یهوه خدای ماست و اما چیزهای مکشوف تا به ابد ازآن ما و فرزندان ما است، تا جمیع کلمات این شریعت را به عمل آوریم.
 
\chapter{30}

\par 1 و چون جمیع این چیزها، یعنی برکت ولعنتی که پیش روی تو گذاشتم بر توعارض شود، و آنها را در میان جمیع امتهایی که یهوه، خدایت، تو را به آنجا خواهد راند، بیادآوری.
\par 2 و تو با فرزندانت با تمامی دل و تمامی جان خود به سوی یهوه خدایت بازگشت نموده، قول او را موافق هر‌آنچه که من امروز به تو امرمی فرمایم، اطاعت نمایی.
\par 3 آنگاه یهوه خدایت اسیری تو را برگردانیده، بر تو ترحم خواهد کرد، و رجوع کرده، تو را از میان جمیع امتهایی که یهوه، خدایت، تو را به آنجا پراکنده کرده است، جمع خواهد نمود.
\par 4 اگر آوارگی تو تا کران آسمان بشود، یهوه، خدایت، تو را از آنجا جمع خواهد کرد و تو را از آنجا خواهد آورد.
\par 5 ویهوه، خدایت، تو را به زمینی که پدرانت مالک آن بودند خواهد آورد، و مالک آن خواهی شد، و برتو احسان نموده، تو را بیشتر از پدرانت خواهدافزود.
\par 6 «و یهوه خدایت دل تو و دل ذریت تو رامختون خواهد ساخت تا یهوه خدایت را به تمامی دل و تمامی جان خود دوست داشته، زنده بمانی.
\par 7 و یهوه خدایت جمیع این لعنتها را بردشمنان و بر خصمانت که تو را آزردند، نازل خواهد گردانید.
\par 8 و تو بازگشت نموده، قول خداوند را اطاعت خواهی کرد، و جمیع اوامر او را که من امروز به تو امر می‌فرمایم، بجا خواهی آورد.
\par 9 و یهوه، خدایت، تو را در تمامی اعمال دستت و در میوه بطنت و نتایج بهایمت ومحصول زمینت به نیکویی خواهد افزود، زیراخداوند بار دیگر بر تو برای نیکویی شادی خواهد کرد، چنانکه بر پدران تو شادی نمود.
\par 10 اگر آواز یهوه خدای خود را اطاعت نموده، اوامر و فرایض او را که در طومار این شریعت مکتوب است، نگاه داری، و به سوی یهوه، خدای خود، با تمامی دل و تمامی جان بازگشت نمایی.
\par 11 «زیرا این حکمی که من امروز به تو امرمی فرمایم، برای تو مشکل نیست و از تو دورنیست.
\par 12 نه در آسمان است تا بگویی کیست که به آسمان برای ما صعود کرده، آن را نزد ما بیاوردو آن را به ما بشنواند تا به عمل آوریم.
\par 13 و نه آن طرف دریا که بگویی کیست که برای ما به آنطرف دریا عبور کرده، آن را نزد ما بیاورد و به ما بشنواندتا به عمل آوریم.
\par 14 بلکه این کلام، بسیار نزدیک توست و در دهان و دل توست تا آن را بجا آوری.
\par 15 «ببین امروز حیات و نیکویی و موت و بدی را پیش روی تو گذاشتم.
\par 16 چونکه من امروز تورا امر می‌فرمایم که یهوه خدای خود را دوست بداری و در طریقهای او رفتار نمایی، و اوامر وفرایض و احکام او را نگاه داری تا زنده مانده، افزوده شوی، و تا یهوه، خدایت، تو را در زمینی که برای تصرفش به آن داخل می‌شوی، برکت دهد.
\par 17 لیکن اگر دل تو برگردد و اطاعت ننمایی و فریفته شده، خدایان غیر را سجده و عبادت نمایی،
\par 18 پس امروز به شما اطلاع می‌دهم که البته هلاک خواهید شد، و در زمینی که از اردن عبور می‌کنید تا در آن داخل شده، تصرف نمایید، عمر طویل نخواهید داشت.
\par 19 امروز آسمان وزمین را بر شما شاهد می‌آورم که حیات و موت وبرکت و لعنت را پیش روی تو گذاشتم، پس حیات را برگزین تا تو با ذریتت زنده بمانی.و تایهوه خدای خود را دوست بداری و آواز او رابشنوی و با او ملصق شوی، زیرا که او حیات تو ودرازی عمر توست تا در زمینی که خداوند برای پدرانت، ابراهیم و اسحاق و یعقوب، قسم خوردکه آن را به ایشان بدهد، ساکن شوی.»
\par 20 و تایهوه خدای خود را دوست بداری و آواز او رابشنوی و با او ملصق شوی، زیرا که او حیات تو ودرازی عمر توست تا در زمینی که خداوند برای پدرانت، ابراهیم و اسحاق و یعقوب، قسم خوردکه آن را به ایشان بدهد، ساکن شوی.»
 
\chapter{31}

\par 1 و موسی رفته، این سخنان را به تمامی اسرائیل بیان کرد،
\par 2 و به ایشان گفت: «من امروز صد و بیست ساله هستم و دیگر طاقت خروج و دخول ندارم، و خداوند به من گفته است که از این اردن عبور نخواهی کرد.
\par 3 یهوه خدای تو، خود به حضور تو عبور خواهد کرد، و او این امتها را از حضور تو هلاک خواهد ساخت، تا آنهارا به تصرف آوری، و یوشع نیز پیش روی توعبور خواهد نمود چنانکه خداوند گفته است.
\par 4 وخداوند چنانکه به سیحون و عوج، دو پادشاه اموریان، که هلاک ساخت و به زمین ایشان عمل نمود به اینها نیز رفتار خواهد کرد.
\par 5 پس چون خداوند ایشان را به‌دست شما تسلیم کند شما باایشان موافق تمامی حکمی که به شما امرفرمودم، رفتار نمایید.
\par 6 قوی و دلیر باشید و ازایشان ترسان و هراسان مباشید، زیرا یهوه، خدایت، خود با تو می‌رود و تو را وا نخواهدگذاشت و ترک نخواهد نمود.»
\par 7 و موسی یوشع را خوانده، در نظر تمامی اسرائیل به او گفت: «قوی و دلیر باش زیرا که توبا این قوم به زمینی که خداوند برای پدران ایشان قسم خورد که به ایشان بدهد داخل خواهی شد، و تو آن را برای ایشان تقسیم خواهی نمود.
\par 8 و خداوند خود پیش روی تومی رود او با تو خواهد بود و تو را وانخواهدگذاشت و ترک نخواهد نمود پس ترسان وهراسان مباش.»
\par 9 و موسی این تورات را نوشته، آن را به بنی لاوی کهنه که تابوت عهد خداوند رابرمی داشتند و به جمیع مشایخ اسرائیل سپرد.
\par 10 و موسی ایشان را امر فرموده، گفت: «که درآخر هر هفت سال، در وقت معین سال انفکاک درعید خیمه‌ها،
\par 11 چون جمیع اسرائیل بیایند تا به حضور یهوه خدای تو در مکانی که او برگزیندحاضر شوند، آنگاه این تورات را پیش جمیع اسرائیل در سمع ایشان بخوان.
\par 12 قوم را از مردان و زنان و اطفال و غریبانی که در دروازه های توباشند جمع کن تا بشنوند، و تعلیم یافته، از یهوه خدای شما بترسند و به عمل نمودن جمیع سخنان این تورات هوشیار باشند.
\par 13 و تا پسران ایشان که ندانسته‌اند، بشنوند، و تعلیم یابند، تامادامی که شما بر زمینی که برای تصرفش از اردن عبور می‌کنید زنده باشید، از یهوه خدای شمابترسند.»
\par 14 و خداوند به موسی گفت: «اینک ایام مردن تو نزدیک است، یوشع را طلب نما و در خیمه اجتماع حاضر شوید تا او را وصیت نمایم.» پس موسی و یوشع رفته، در خیمه اجتماع حاضرشدند.
\par 15 و خداوند در ستون ابر در خیمه ظاهرشد و ستون ابر، بر در خیمه ایستاد.
\par 16 و خداوند به موسی گفت: «اینک با پدران خود می‌خوابی و این قوم برخاسته، در‌پی خدایان بیگانه زمینی که ایشان به آنجا در میان آنها می‌روند، زنا خواهند کرد و مرا ترک کرده، عهدی را که با ایشان بستم خواهند شکست.
\par 17 ودر آن روز، خشم من بر ایشان مشتعل شده، ایشان را ترک خواهم نمود، و روی خود را از ایشان پنهان کرده، تلف خواهند شد، و بدیها و تنگیهای بسیار به ایشان خواهد رسید، به حدی که در آن روز خواهند گفت: «آیا این بدیها به ما نرسید ازاین جهت که خدای ما در میان ما نیست؟»
\par 18 و به‌سبب تمامی بدی که کرده‌اند که به سوی خدایان غیر برگشته‌اند من در آن روز البته روی خود راپنهان خواهم کرد.
\par 19 پس الان این سرود را برای خود بنویسید و تو آن را به بنی‌اسرائیل تعلیم داده، آن را در دهان ایشان بگذار تا این سرود برای من بر بنی‌اسرائیل شاهد باشد.
\par 20 زیرا چون ایشان را به زمینی که برای پدران ایشان قسم خورده بودم که به شیر و شهد جاری است، درآورده باشم، و چون ایشان خورده، و سیر شده، و فربه گشته باشند، آنگاه ایشان به سوی خدایان غیر برگشته، آنها را عبادت خواهند نمود، و مرااهانت کرده، عهد مرا خواهند شکست. 
\par 21 و چون بدیها و تنگیهای بسیار بر ایشان عارض شده باشد، آنگاه این سرود مثل شاهد پیش روی ایشان شهادت خواهد داد، زیرا که از دهان ذریت ایشان فراموش نخواهد شد، زیرا خیالات ایشان را نیزکه امروز دارند می‌دانم، قبل از آن که ایشان را به زمینی که درباره آن قسم خوردم، درآورم.»
\par 22 پس موسی این سرود را در همان روز نوشته، به بنی‌اسرائیل تعلیم داد.
\par 23 و یوشع بن نون را وصیت نموده، گفت: «قوی و دلیر باش زیرا که تو بنی‌اسرائیل را به زمینی که برای ایشان قسم خوردم داخل خواهی ساخت، و من با تو خواهم بود.»
\par 24 و واقع شد که چون موسی نوشتن کلمات این تورات را در کتاب، تمام به انجام رسانید،
\par 25 موسی به لاویانی که تابوت عهد خداوند رابرمی داشتند وصیت کرده، گفت:
\par 26 «این کتاب تورات را بگیرید و آن را در پهلوی تابوت عهدیهوه، خدای خود، بگذارید تا در آنجا برای شماشاهد باشد.
\par 27 زیرا که من تمرد و گردن کشی شمارا می‌دانم، اینک امروز که من هنوز با شما زنده هستم بر خداوند فتنه انگیخته‌اید، پس چند مرتبه زیاده بعد از وفات من.
\par 28 جمیع مشایخ اسباط وسروران خود را نزد من جمع کنید تا این سخنان رادر گوش ایشان بگویم، و آسمان و زمین را برایشان شاهد بگیرم.
\par 29 زیرا می‌دانم که بعد ازوفات من خویشتن را بالکل فاسد گردانیده، ازطریقی که به شما امر فرمودم خواهید برگشت، ودر روزهای آخر بدی بر شما عارض خواهد شد، زیرا که آنچه در نظر خداوند بد است خواهیدکرد، و از اعمال دست خود، خشم خداوند را به هیجان خواهید آورد.»پس موسی کلمات این سرود را در گوش تمامی جماعت اسرائیل تمام گفت: سرود موسی ای آسمان گوش بگیر تا بگویم.
\par 30 پس موسی کلمات این سرود را در گوش تمامی جماعت اسرائیل تمام گفت: سرود موسی ای آسمان گوش بگیر تا بگویم.
 
\chapter{32}

\par 1 و زمین سخنان دهانم را بشنود.
\par 2 تعلیم من مثل باران خواهد بارید. و کلام من مثل شبنم خواهد ریخت. مثل قطره های باران بر سبزه تازه، و مثل بارشها بر نباتات.
\par 3 زیرا که نام یهوه را ندا خواهم کرد. خدای ما رابه عظمت وصف نمایید.
\par 4 او صخره است و اعمال او کامل. زیرا همه طریقهای او انصاف است. خدای امین و از ظلم مبرا. عادل و راست است او.
\par 5 ایشان خود را فاسد نموده، فرزندان او نیستندبلکه عیب ایشانند. طبقه کج و متمردند.
\par 6 آیا خداوند را چنین مکافات می‌دهید، ای قوم احمق و غیر حکیم. آیا او پدر و مالک تو نیست. او تو را آفرید و استوار نمود.
\par 7 ایام قدیم را بیاد آور. در سالهای دهر به دهرتامل نما. از پدر خود بپرس تا تو را آگاه سازد. واز مشایخ خویش تا تو را اطلاع دهند.
\par 8 چون حضرت اعلی به امتها نصیب ایشان را دادو بنی آدم را منتشر ساخت، آنگاه حدود امتها راقرار داد، برحسب شماره بنی‌اسرائیل.
\par 9 زیرا که نصیب یهوه قوم وی است. و یعقوب قرعه میراث اوست.
\par 10 او را در زمین ویران یافت. و در بیابان خراب وهولناک. او را احاطه کرده، منظور داشت. و او را مثل مردمک چشم خود محافظت نمود.
\par 11 مثل عقابی که آشیانه خود را حرکت دهد. وبچه های خود را فرو‌گیرد. و بالهای خود را پهن کرده، آنها را بردارد. و آنها را بر پرهای خود ببرد.
\par 12 همچنین خداوند تنها او را رهبری نمود. وهیچ خدای بیگانه با وی نبود.
\par 13 او را بر بلندیهای زمین سوار کرد تا ازمحصولات زمین بخورد. و شهد را از صخره به اوداد تا مکید. و روغن را از سنگ خارا.
\par 14 کره گاوان و شیر گوسفندان را با پیه بره‌ها وقوچها را از جنس باشان و بزها. و پیه گرده های گندم را. و شراب از عصیر انگور نوشیدی.
\par 15 لیکن یشورون فربه شده، لگد زد. تو فربه وتنومند و چاق شده‌ای. پس خدایی را که او راآفریده بود، ترک کرد. و صخره نجات خود راحقیر شمرد.
\par 16 او را به خدایان غریب به غیرت آوردند. وخشم او را به رجاسات جنبش دادند.
\par 17 برای دیوهایی که خدایان نبودند، قربانی گذرانیدند، برای خدایانی که نشناخته بودند، برای خدایان جدید که تازه به وجود آمده، وپدران ایشان از آنها نترسیده بودند.
\par 18 و به صخره‌ای که تو را تولید نمود، اعتناننمودی. و خدای آفریننده خود را فراموش کردی.
\par 19 چون یهوه این را دید ایشان را مکروه داشت. چونکه پسران و دخترانش خشم او را به هیجان آوردند.
\par 20 پس گفت روی خود را از ایشان خواهم پوشید. تا ببینم که عاقبت ایشان چه خواهد بود.زیرا طبقه بسیار گردن کشند. و فرزندانی که امانتی در ایشان نیست.
\par 21 ایشان مرا به آنچه خدا نیست به غیرت آوردند. و به اباطیل خود مرا خشمناک گردانیدند. و من ایشان را به آنچه قوم نیست به غیرت خواهم آورد. و به امت باطل، ایشان را خشمناک خواهم ساخت.
\par 22 زیرا آتشی در غضب من افروخته شده. و تاهاویه پایین‌ترین شعله‌ور شده است. و زمین را باحاصلش می‌سوزاند. و اساس کوهها را آتش خواهد زد.
\par 23 بر ایشان بلایا را جمع خواهم کرد. و تیرهای خود را تمام بر ایشان صرف خواهم نمود.
\par 24 از گرسنگی کاهیده، و از آتش تب، و از وبای تلخ تلف می‌شوند. و دندانهای وحوش را به ایشان خواهم فرستاد، با زهر خزندگان زمین.
\par 25 شمشیر از بیرون و دهشت از اندرون. ایشان رابی اولاد خواهد ساخت. هم جوان و هم دوشیزه را. شیرخواره را با ریش سفید هلاک خواهد کرد.
\par 26 می‌گفتم ایشان را پراکنده کنم و ذکر ایشان را ازمیان مردم، باطل سازم.
\par 27 اگر از کینه دشمن نمی ترسیدم که مبادامخالفان ایشان برعکس آن فکر کنند، و بگوینددست ما بلند شده، و یهوه همه این را نکرده است.
\par 28 زیرا که ایشان قوم گم کرده تدبیر هستند. و درایشان بصیرتی نیست.
\par 29 کاش که حکیم بوده، این را می‌فهمیدید. و درعاقبت خود تامل می‌نمودند.
\par 30 چگونه یک نفر هزار را تعاقب می‌کرد. و دو نفرده هزار را منهزم می‌ساختند. اگر صخره ایشان، ایشان را نفروخته. و خداوند، ایشان را تسلیم ننموده بود.
\par 31 زیرا که صخره ایشان مثل صخره ما نیست. اگرچه هم دشمنان ما خود، حکم باشند.
\par 32 زیرا که مو ایشان از موهای سدوم است، و ازتاکستانهای عموره. انگورهای ایشان انگورهای حنظل است، و خوشه های ایشان تلخ است.
\par 33 شراب ایشان زهر اژدرهاست. و سم قاتل افعی.
\par 34 آیا این نزد من مکنون نیست. و درخزانه های من مختوم نی.
\par 35 انتقام و جزا از آن من است، هنگامی که پایهای ایشان بلغزد، زیرا که روز هلاکت ایشان نزدیک است و قضای ایشان می‌شتابد.
\par 36 زیرا خداوند، قوم خود را داوری خواهدنمود. و بر بندگان خویش شفقت خواهد کرد. چون می‌بیند که قوت ایشان نابود شده، وهیچکس چه غلام و چه آزاد باقی نیست.
\par 37 و خواهد گفت: خدایان ایشان کجایند، وصخره‌ای که بر آن اعتماد می‌داشتند.
\par 38 که پیه قربانی های ایشان را می‌خوردند. و شراب هدایای ریختنی‌ایشان را می‌نوشیدند. آنهابرخاسته، شما را امداد کنند. و برای شما ملجاباشند.
\par 39 الان ببینید که من خود، او هستم. و با من خدای دیگری نیست. من می‌میرانم و زنده می‌کنم. مجروح می‌کنم و شفا می‌دهم. و از دست من رهاننده‌ای نیست.
\par 40 زیرا که دست خود را به آسمان برمی افرازم، ومی گویم که من تا ابدالاباد زنده هستم.
\par 41 اگر شمشیر براق خود را تیز کنم. و قصاص رابه‌دست خود گیرم. آنگاه از دشمنان خود انتقام خواهم کشید. و به خصمان خود مکافات خواهم رسانید.
\par 42 تیرهای خود را از خون مست خواهم ساخت. و شمشیر من گوشت را خواهد خورد. از خون کشتگان و اسیران، با روسای سروران دشمن.
\par 43 ‌ای امتها با قوم او آواز شادمانی دهید. زیراانتقام خون بندگان خود را گرفته است. و ازدشمنان خود انتقام کشیده و برای زمین خود وقوم خویش کفاره نموده است.
\par 44 و موسی آمده، تمامی سخنان این سرود رابه سمع قوم رسانید، او و یوشع بن نون.
\par 45 و چون موسی از گفتن همه این سخنان به تمامی اسرائیل فارغ شد،
\par 46 به ایشان گفت: «دل خود را به همه سخنانی که من امروز به شما شهادت می‌دهم، مشغول سازید، تا فرزندان خود را حکم دهید که متوجه شده، تمامی کلمات این تورات را به عمل آورند.
\par 47 زیرا که این برای شما امر باطل نیست، بلکه حیات شماست، و به واسطه این امر، عمرخود را در زمینی که شما برای تصرفش از اردن به آنجا عبور می‌کنید، طویل خواهید ساخت.»
\par 48 و خداوند در همان روز موسی را خطاب کرده، گفت:
\par 49 «به این کوه عباریم یعنی جبل نبوکه در زمین موآب در مقابل اریحاست برآی، وزمین کنعان را که من آن را به بنی‌اسرائیل به ملکیت می‌دهم ملاحظه کن.
\par 50 و تو در کوهی که به آن برمی آیی وفات کرده، به قوم خود ملحق شو، چنانکه برادرت هارون در کوه هور مرد و به قوم خود ملحق شد.
\par 51 زیرا که شما در میان بنی‌اسرائیل نزد آب مریبا قادش در بیابان سین به من تقصیر نمودید، چون که مرا در میان بنی‌اسرائیل تقدیس نکردید.پس زمین را پیش روی خود خواهی دید، لیکن به آنجا به زمینی که به بنی‌اسرائیل می‌دهم، داخل نخواهی شد.»
\par 52 پس زمین را پیش روی خود خواهی دید، لیکن به آنجا به زمینی که به بنی‌اسرائیل می‌دهم، داخل نخواهی شد.»
 
\chapter{33}

\par 1 و این است برکتی که موسی، مرد خدا، قبل از وفاتش به بنی‌اسرائیل برکت داده،
\par 2 گفت: «یهوه از سینا آمد، و از سعیر برایشان طلوع نمود. و از جبل فاران درخشان گردید. و با کرورهای مقدسین آمد، و از دست راست او برای ایشان شریعت آتشین پدید آمد.
\par 3 به درستی که قوم خود را دوست می‌دارد. وجمیع مقدسانش در دست تو هستند. و نزدپایهای تو نشسته، هر یکی از کلام تو بهره‌مندمی شوند.
\par 4 موسی برای ما شریعتی امر فرمود که میراث جماعت یعقوب است.
\par 5 و او در یشورون پادشاه بود هنگامی که روسای قوم اسباطاسرائیل با هم جمع شدند.
\par 6 روبین زنده بماند ونمیرد. و مردان او در شماره کم نباشند.»
\par 7 و این است درباره یهودا که گفت: «ای خداوند آواز یهودا رابشنو. و او را به قوم خودش برسان. به‌دستهای خود برای خویشتن جنگ می‌کند. و تو از دشمنانش معاون می‌باشی.»
\par 8 و درباره لاوی گفت: «تمیم و اوریم تو نزدمرد مقدس توست. که او را در مسا امتحان نمودی. و با او نزد آب مریبا منازعت کردی.
\par 9 که درباره پدر و مادر خود گفت که ایشان را ندیده‌ام و برادران خود را نشناخت. و پسران خود راندانست. زیرا که کلام تو را نگاه می‌داشتند. و عهدتو را محافظت می‌نمودند.
\par 10 احکام تو را به یعقوب تعلیم خواهند داد. و شریعت تو را به اسرائیل. بخور به حضور تو خواهند‌آورد. وقربانی های سوختنی بر مذبح تو.
\par 11 ‌ای خداونداموال او را برکت بده، و اعمال دستهای او را قبول فرما. کمرهای مقاومت کنندگانش را بشکن. کمرهای خصمان او را که دیگر برنخیزند.»
\par 12 و درباره بنیامین گفت: «حبیب خداوند نزدوی ایمن ساکن می‌شود. تمامی روز او را مستورمی سازد. و در میان کتفهایش ساکن می‌شود.»
\par 13 و درباره یوسف گفت: «زمینش از خداوندمبارک باد، از نفایس آسمان و از شبنم، و از لجه هاکه در زیرش مقیم است.
\par 14 از نفایس محصولات آفتاب و از نفایس نباتات ماه.
\par 15 از فخرهای کوههای قدیم، و از نفایس تلهای جاودانی.
\par 16 ازنفایس زمین و پری آن، و از رضامندی او که دربوته ساکن بود. برکت بر سر یوسف برسد. و برفرق سر آنکه از برادران خود ممتاز گردید.
\par 17 جاه او مثل نخست زاده گاوش باشد. و شاخهایش مثل شاخهای گاو وحشی. با آنها امتها را جمیع تا به اقصای زمین خواهد زد. و اینانند ده هزارهای افرایم و هزارهای منسی.»
\par 18 و درباره زبولون گفت: «ای زبولون در بیرون رفتنت شاد باش، و تو‌ای یساکار در خیمه های خویش.
\par 19 قومها را به کوه دعوت خواهند نمود. در آنجا قربانی های عدالت را خواهند گذرانید.زیرا که فراوانی دریا را خواهند مکید. وخزانه های مخفی ریگ را.»
\par 20 و درباره جاد گفت: «متبارک باد آنکه جاد را وسیع گرداند. مثل شیرماده ساکن باشد، وبازو و فرق را نیز می‌درد، 
\par 21 و حصه بهترین رابرای خود نگاه دارد، زیرا که در آنجا نصیب حاکم محفوظ است. و با روسای قوم می‌آید. وعدالت خداوند و احکامش را با اسرائیل بجامی آورد.»
\par 22 و درباره دان گفت: «دان بچه شیر است که ازباشان می‌جهد.»
\par 23 و درباره نفتالی گفت: «ای نفتالی ازرضامندی خداوند سیر شو. و از برکت او مملوگردیده، مغرب و جنوب را به تصرف آور.»
\par 24 و درباره اشیر گفت: «اشیر از فرزندان مبارک شود، و نزد برادران خود مقبول شده، پای خود را به روغن فرو برد.
\par 25 نعلین تو از آهن وبرنجست، و مثل روزهایت همچنان قوت توخواهد بود.
\par 26 ‌ای یشورون مثل خدا کسی نیست، که برای مدد تو بر آسمانها سوار شود. ودر کبریای خود برافلاک.
\par 27 خدای ازلی مسکن توست. و در زیر توبازوهای جاودانی است. و دشمن را از حضور تواخراج کرده، می‌گوید هلاک کن.
\par 28 پس اسرائیل در امنیت ساکن خواهد شد، و چشمه یعقوب به تنهایی. و در زمینی که پر از غله و شیره باشد. وآسمان آن شبنم می‌ریزد.خوشابه‌حال تو‌ای اسرائیل. کیست مانند تو! ای قومی که ازخداوند نجات یافته‌اید. که او سپر نصرت تو وشمشیر جاه توست. و دشمنانت مطیع تو خواهند شد. و تو بلندیهای ایشان را پایمال خواهی نمود.»
\par 29 خوشابه‌حال تو‌ای اسرائیل. کیست مانند تو! ای قومی که ازخداوند نجات یافته‌اید. که او سپر نصرت تو وشمشیر جاه توست. و دشمنانت مطیع تو خواهند شد. و تو بلندیهای ایشان را پایمال خواهی نمود.»
 
\chapter{34}

\par 1 و موسی از عربات موآب، به کوه نبو، برقله فسجه که در مقابل اریحاست برآمد، و خداوند تمامی زمین را، از جلعاد تا دان، به او نشان داد.
\par 2 و تمامی نفتالی و زمین افرایم ومنسی و تمامی زمین یهودا را تا دریای مغربی.
\par 3 وجنوب را و میدان دره اریحا را که شهر نخلستان است تا صوغر.
\par 4 و خداوند وی را گفت: «این است زمینی که برای ابراهیم و اسحاق و یعقوب قسم خورده، گفتم که این را به ذریت تو خواهم داد، تو را اجازت دادم که به چشم خود آن راببینی لیکن به آنجا عبور نخواهی کرد.»
\par 5 پس موسی بنده خداوند در آنجا به زمین موآب برحسب قول خداوند مرد.
\par 6 و او را در زمین موآب در مقابل بیت فعور، در دره دفن کرد، واحدی قبر او را تا امروز ندانسته است.
\par 7 و موسی چون وفات یافت، صد و بیست سال داشت، و نه چشمش تار، و نه قوتش کم شده بود.
\par 8 وبنی‌اسرائیل برای موسی در عربات موآب سی روز ماتم گرفتند. پس روزهای ماتم و نوحه گری برای موسی سپری گشت.
\par 9 و یوشع بن نون از روح حکمت مملو بود، چونکه موسی دستهای خود را بر او نهاده بود، وبنی‌اسرائیل او را اطاعت نمودند، و برحسب آنچه خداوند به موسی‌امر فرموده بود، عمل کردند.
\par 10 و نبی‌ای مثل موسی تا بحال در اسرائیل برنخاسته است که خداوند او را روبرو شناخته باشد.در جمیع آیات و معجزاتی که خداونداو را فرستاد تا آنها را در زمین مصر به فرعون و جمیع بندگانش و تمامی زمینش بنماید.
\par 11 در جمیع آیات و معجزاتی که خداونداو را فرستاد تا آنها را در زمین مصر به فرعون و جمیع بندگانش و تمامی زمینش بنماید.


\end{document}