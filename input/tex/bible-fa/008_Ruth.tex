\begin{document}

\title{روت}

 
\chapter{1}

\par 1 و واقع شد در ایام حکومت داوران که قحطی در زمین پیدا شد، و مردی ازبیت لحم یهودا رفت تا در بلاد موآب ساکن شود، او و زنش و دو پسرش.
\par 2 و اسم آن مرد الیملک بود، و اسم زنش نعومی، و پسرانش به محلون وکلیون مسمی و افراتیان بیت لحم یهودا بودند. پس به بلاد موآب رسیده، در آنجا ماندند.
\par 3 والیملک شوهر نعومی، مرد و او با دو پسرش باقی ماند.
\par 4 و ایشان از زنان موآب برای خود زن گرفتند که نام یکی عرفه و نام دیگری روت بود، ودر آنجا قریب به ده سال توقف نمودند.
\par 5 و هردوی ایشان محلون و کلیون نیز مردند، و آن زن ازدو پسر و شوهر خود محروم ماند.
\par 6 پس او با دو عروس خود برخاست تا از بلادموآب برگردد، زیرا که در بلاد موآب شنیده بودکه خداوند از قوم خود تفقد نموده، نان به ایشان داده است.
\par 7 و از مکانی که در آن ساکن بود بیرون آمد، و دو عروسش همراه وی بودند، و به راه روانه شدند تا به زمین یهودا مراجعت کنند.
\par 8 ونعومی به دو عروس خود گفت: «بروید و هر یکی از شما به خانه مادر خود برگردید، و خداوند برشما احسان کناد، چنانکه شما به مردگان و به من کردید.
\par 9 و خداوند به شما عطا کناد که هر یکی ازشما در خانه شوهر خود راحت یابید.» پس ایشان را بوسید و آواز خود را بلند کرده، گریستند.
\par 10 وبه او گفتند: «نی بلکه همراه تو نزد قوم تو خواهیم برگشت.»
\par 11 نعومی گفت: «ای دخترانم برگردید، چرا همراه من بیایید؟ آیا در رحم من هنوز پسران هستند که برای شما شوهر باشند؟
\par 12 ‌ای دخترانم برگشته، راه خود را پیش گیرید زیرا که برای شوهر گرفتن زیاده پیر هستم، و اگر گویم که امید دارم و امشب نیز به شوهر داده شوم وپسران هم بزایم،
\par 13 آیا تا بالغ شدن ایشان صبر خواهیدکرد، و به‌خاطر ایشان، خود را از شوهر گرفتن محروم خواهید داشت؟ نی‌ای دخترانم زیرا که جانم برای شما بسیار تلخ شده است چونکه دست خداوند بر من دراز شده است.»
\par 14 پس بار دیگر آواز خود را بلند کرده، گریستند و عرفه مادر شوهر خود را بوسید، اماروت به وی چسبید.
\par 15 و او گفت: «اینک زن برادر شوهرت نزد قوم خود و خدایان خویش برگشته است تو نیز در عقب زن برادر شوهرت برگرد.»
\par 16 روت گفت: «بر من اصرار مکن که تو راترک کنم و از نزد تو برگردم، زیرا هر جایی که روی می‌آیم و هر جایی که منزل کنی، منزل می‌کنم، قوم تو قوم من و خدای تو خدای من خواهد بود.
\par 17 جایی که بمیری، می‌میرم و درآنجا دفن خواهم شد. خداوند به من چنین بلکه زیاده بر این کند اگر چیزی غیر از موت، مرا از توجدا نماید.»
\par 18 پس چون دید که او برای رفتن همراهش مصمم شده است از سخن‌گفتن با وی باز ایستاد.
\par 19 و ایشان هر دو روانه شدند تا به بیت لحم رسیدند، و چون وارد بیت لحم گردیدند، تمامی شهر بر ایشان به حرکت آمده، زنان گفتند که آیا این نعومی است؟
\par 20 او به ایشان گفت: «مرا نعومی مخوانید بلکه مرا مره بخوانید زیرا قادر مطلق به من مرارت سخت رسانیده است.
\par 21 من پر بیرون رفتم و خداوند مرا خالی برگردانید پس برای چه مرا نعومی می‌خوانید چونکه خداوند مرا ذلیل ساخته است و قادرمطلق به من بدی رسانیده است.»و نعومی مراجعت کرد و عروسش روت موآبیه که از بلاد موآب برگشته بود، همراه وی آمد، و در ابتدای درویدن جو وارد بیت لحم شدند.
\par 22 و نعومی مراجعت کرد و عروسش روت موآبیه که از بلاد موآب برگشته بود، همراه وی آمد، و در ابتدای درویدن جو وارد بیت لحم شدند.
 
\chapter{2}

\par 1 و نعومی خویش شوهری داشت که مردی دولتمند، بوعز نام از خاندان الیملک بود.
\par 2 و روت موآبیه به نعومی گفت: «مرا اجازت ده که به کشتزارها بروم و در عقب هر کسی‌که در نظرش التفات یابم خوشه چینی نمایم. او وی را گفت: «برو‌ای دخترم.»
\par 3 پس روانه شده، به کشتزاردرآمد و در عقب دروندگان خوشه چینی می‌نمود، و اتفاق او به قطعه زمین بوعز که ازخاندان الیملک بود، افتاد.
\par 4 و اینک بوعز ازبیت لحم آمده، به دروندگان گفت: «خداوند باشما باد.» ایشان وی را گفتند: «خداوند تو رابرکت دهد.»
\par 5 و بوعز به نوکر خود که بر دروندگان گماشته بود، گفت: «این دختر از آن کیست؟»
\par 6 نوکر که بردروندگان گماشته شده بود، در جواب گفت: «این است دختر موآبیه که با نعومی از بلاد موآب برگشته است،
\par 7 و به من گفت: تمنا اینکه خوشه چینی نمایم و در عقب دروندگان در میان بافه‌ها جمع کنم، پس آمده، از صبح تا به حال مانده است، سوای آنکه اندکی در خانه توقف کرده است.»
\par 8 و بوعز به روت گفت: «ای دخترم مگرنمی شنوی، به هیچ کشت زار دیگر برای خوشه چینی مرو و از اینجا هم مگذر بلکه باکنیزان من در اینجا باش.
\par 9 و چشمانت به زمینی که می‌دروند نگران باشد و در عقب ایشان برو، آیاجوانان را حکم نکردم که تو را لمس نکنند، و اگرتشنه باشی، نزد ظروف ایشان برو و از آنچه جوانان می‌کشند، بنوش.»
\par 10 پس به روی درافتاده، او را تا به زمین تعظیم کرد و به او گفت: «برای چه در نظر تو التفات یافتم که به من توجه نمودی و حال آنکه غریب هستم.»
\par 11 بوعز در جواب او گفت: «از هر‌آنچه بعد ازمردن شوهرت به مادر شوهر خود کردی اطلاع تمام به من رسیده است، و چگونه پدر و مادر وزمین ولادت خود را ترک کرده، نزد قومی که پیشتر ندانسته بودی، آمدی.
\par 12 خداوند عمل تورا جزا دهد و از جانب یهوه، خدای اسرائیل، که در زیر بالهایش پناه بردی، اجر کامل به تو برسد.»
\par 13 گفت: «ای آقایم، در نظر تو التفات بیابم زیرا که مرا تسلی دادی و به کنیز خود سخنان دل آویزگفتی، اگر‌چه من مثل یکی از کنیزان تو نیستم.»
\par 14 بوعز وی را گفت: «در وقت چاشت اینجابیا و از نان بخور و لقمه خود را در شیره فرو بر.» پس نزد دروندگان نشست و غله برشته به او دادندو خورد و سیر شده، باقی‌مانده را واگذاشت.
\par 15 وچون برای خوشه چینی برخاست بوعز جوانان خود را امر کرده، گفت: «بگذارید که در میان بافه‌ها هم خوشه چینی نماید و او را زجرمنمایید.
\par 16 و نیز از دسته‌ها کشیده، برایش بگذارید تا برچیند و او را عتاب مکنید.»
\par 17 پس تا شام در آن کشتزار خوشه چینی نموده، آنچه را که برچیده بود، کوبید و به قدر یک ایفه جو بود.
\par 18 پس آن را برداشته، به شهردرآمد، و مادر شوهرش آنچه را که برچیده بود، دید، و آنچه بعد از سیرشدنش باقی‌مانده بود، بیرون آورده، به وی داد.
\par 19 و مادر شوهرش وی را گفت: «امروز کجا خوشه چینی نمودی و کجاکار کردی؟ مبارک باد آنکه بر تو توجه نموده است.» پس مادر شوهر خود را از کسی‌که نزد وی کار کرده بود، خبر داده، گفت: «نام آن شخص که امروز نزد او کار کردم، بوعز است.»
\par 20 و نعومی به عروس خود گفت: «او از جانب خداوند مبارک باد زیرا که احسان را بر زندگان ومردگان ترک ننموده است.» و نعومی وی را گفت: «این شخص، خویش ما و از ولی های ماست.»
\par 21 و روت موآبیه گفت که «او نیز مرا گفت باجوانان من باش تا همه درو مرا تمام کنند.»
\par 22 نعومی به عروس خود روت گفت که «ای دخترم خوب است که با کنیزان او بیرون روی وتو را در کشتزار دیگر نیابند.»پس با کنیزان بوعز برای خوشه چینی می‌ماند تا درو جو و دروگندم تمام شد، و با مادرشوهرش سکونت داشت.
\par 23 پس با کنیزان بوعز برای خوشه چینی می‌ماند تا درو جو و دروگندم تمام شد، و با مادرشوهرش سکونت داشت.
 
\chapter{3}

\par 1 و مادر شوهرش، نعومی وی را گفت: «ای دختر من، آیا برای تو راحت نجویم تا برایت نیکو باشد.
\par 2 و الان آیا بوعز که تو باکنیزانش بودی خویش ما نیست؟ و اینک اوامشب در خرمن خود، جو پاک می‌کند.
\par 3 پس خویشتن را غسل کرده، تدهین کن و رخت خود راپوشیده، به خرمن برو، اما خود را به آن مردنشناسان تا از خوردن و نوشیدن فارغ شود.
\par 4 وچون او بخوابد جای خوابیدنش را نشان کن، ورفته، پایهای او را بگشا و بخواب، و او تو راخواهد گفت که چه باید بکنی.»
\par 5 او وی را گفت: «هر‌چه به من گفتی، خواهم کرد.»
\par 6 پس به خرمن رفته، موافق هر‌چه مادرشوهرش او را امر فرموده بود، رفتار نمود.
\par 7 پس چون بوعز خورد و نوشید و دلش شاد شد ورفته، به کنار بافه های جو خوابید، آنگاه او آهسته آهسته آمده، پایهای او را گشود و خوابید.
\par 8 و درنصف شب آن مرد مضطرب گردید و به آن سمت متوجه شد که اینک زنی نزد پایهایش خوابیده است.
\par 9 و گفت: «تو کیستی»؟ او گفت: «من کنیزتو، روت هستم، پس دامن خود را بر کنیز خویش بگستران زیرا که تو ولی هستی.»
\par 10 او گفت: «ای دختر من! از جانب خداوندمبارک باش! زیرا که در آخر بیشتر احسان نمودی از اول، چونکه در عقب جوانان، چه فقیر و چه غنی، نرفتی.
\par 11 و حال‌ای دختر من، مترس! هرآنچه به من گفتی برایت خواهم کرد، زیرا که تمام شهر قوم من تو را زن نیکو می‌دانند.
\par 12 و الان راست است که من ولی هستم لیکن ولی‌ای نزدیکتر از من هست.
\par 13 امشب در اینجا بمان وبامدادان اگر او حق ولی را برای تو ادا نماید، خوب ادا نماید، و اگر نخواهد که برای تو حق ولی را ادا نماید، پس قسم به حیات خداوند که من آن را برای تو ادا خواهم نمود، الان تا صبح بخواب.»
\par 14 پس نزد پایش تا صبح خوابیده، پیش ازآنکه کسی همسایه‌اش را تشخیص دهد، برخاست، و بوعز گفت: «زنهار کسی نفهمد که این زن به خرمن آمده است.
\par 15 و گفت چادری که برتوست، بیاور و بگیر.» پس آن را بگرفت و او شش کیل جو پیموده، بر وی گذارد و به شهر رفت.
\par 16 وچون نزد مادر شوهر خود رسید، او وی را گفت: «ای دختر من، بر تو چه گذشت؟» پس او را از هرآنچه آن مرد با وی کرده بود، خبر داد.
\par 17 و گفت: «این شش کیل جو را به من داد زیرا گفت، نزدمادرشوهرت تهیدست مرو.»او وی را گفت: «ای دخترم آرام بنشین تا بدانی که این امر چگونه خواهد شد، زیرا که آن مرد تا این کار را امروزتمام نکند، آرام نخواهد گرفت.»
\par 18 او وی را گفت: «ای دخترم آرام بنشین تا بدانی که این امر چگونه خواهد شد، زیرا که آن مرد تا این کار را امروزتمام نکند، آرام نخواهد گرفت.»
 
\chapter{4}

\par 1 و بوعز به دروازه آمده، آنجا نشست و اینک آن ولی که بوعز درباره او سخن گفته بود می‌گذشت، و به او گفت: «ای فلان! به اینجابرگشته، بنشین.» و او برگشته، نشست.
\par 2 و ده نفراز مشایخ شهر را برداشته، به ایشان گفت: «اینجابنشینید.» و ایشان نشستند.
\par 3 و به آن ولی گفت: «نعومی که از بلاد موآب برگشته است قطعه زمینی را که از برادر ما الیملک بود، می‌فروشد.
\par 4 ومن مصلحت دیدم که تو را اطلاع داده، بگویم که آن را به حضور این مجلس و مشایخ قوم من بخر، پس اگر انفکاک می‌کنی، بکن، و اگر انفکاک نمی کنی مرا خبر بده تا بدانم، زیرا غیر از تو کسی نیست که انفکاک کند، و من بعد از تو هستم.» اوگفت: «من انفکاک می‌کنم.»
\par 5 بوعز گفت: «درروزی که زمین را از دست نعومی می‌خری، ازروت موآبیه، زن متوفی نیز باید خرید، تا نام متوفی را بر میراثش برانگیزانی.»
\par 6 آن ولی گفت: «نمی توانم برای خود انفکاک کنم مبادا میراث خود را فاسد کنم، پس تو حق انفکاک مرا بر ذمه خود بگیر زیرا نمی توانم انفکاک نمایم.»
\par 7 و رسم انفکاک و مبادلت در ایام قدیم دراسرائیل به جهت اثبات هر امر این بود که شخص کفش خود را بیرون کرده، به همسایه خود می‌داد. و این در اسرائیل قانون شده است.
\par 8 پس آن ولی به بوعز گفت: «آن را برای خود بخر.» و کفش خود را بیرون کرد.
\par 9 و بوعز به مشایخ و به تمامی قوم گفت: «شما امروز شاهد باشید که تمامی مایملک الیملک و تمامی مایملک کلیون ومحلون را از دست نعومی خریدم.
\par 10 و هم روت موآبیه زن محلون را به زنی خود خریدم تا نام متوفی را بر میراثش برانگیزانم، و نام متوفی ازمیان برادرانش و از دروازه محله‌اش منقطع نشود، شما امروز شاهد باشید.»
\par 11 و تمامی قوم که نزد دروازه بودند و مشایخ گفتند: «شاهد هستیم و خداوند این زن را که به خانه تو درآمد، مثل راحیل و لیه گرداند که خانه اسرائیل را بنا کردند، و تو در افراته کامیاب شو، ودر بیت لحم نامور باش. 
\par 12 و خانه تو مثل خانه فارص باشد که تامار برای یهودا زایید، ازاولادی که خداوند تو را از این دختر، خواهدبخشید.»
\par 13 پس بوعز روت را گرفت و او زن وی شد وبه او درآمد و خداوند او را حمل داد که پسری زایید.
\par 14 و زنان به نعومی گفتند: «متبارک بادخداوند که تو را امروز بی‌ولی نگذاشته است و نام او در اسرائیل بلند شود.
\par 15 و او برایت تازه کننده جان و پرورنده پیری تو باشد، زیرا که عروست که تو را دوست می‌دارد و برایت از هفت پسر بهتراست، او را زایید.»
\par 16 و نعومی پسر را گرفته، در آغوش خود گذاشت و دایه او شد.
\par 17 و زنان همسایه‌اش، او را نام نهاده، گفتند برای نعومی پسری زاییده شد، و نام اورا عوبید خواندند و او پدر یسی پدر داوداست.
\par 18 این است پیدایش فارص: فارص حصرون را آورد؛
\par 19 و حصرون، رام را آورد؛ و رام، عمیناداب را آورد؛
\par 20 و عمیناداب نحشون راآورد؛ و نحشون سلمون را آورد؛و سلمون بوعز را آورد؛ و بوعز عوبید را آورد؛
\par 21 و سلمون بوعز را آورد؛ و بوعز عوبید را آورد؛


\end{document}