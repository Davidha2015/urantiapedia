\begin{document}

\title{اول سموئيل}

 
\chapter{1}

\par 1 و مردی بود از رامه تایم صوفیم از کوهستان افرایم، مسمی به القانه بن یروحام بن الیهو بن توحو بن صوف. و او افرایمی بود.
\par 2 و او دو زن داشت. اسم یکی حنا و اسم دیگری فننه بود. و فننه اولاد داشت لیکن حنا رااولاد نبود.
\par 3 و آن مرد هر سال برای عبادت نمودن وقربانی گذرانیدن برای یهوه صبایوت از شهر خودبه شیلوه می‌آمد، و حفنی و فینحاس دو پسرعیلی، کاهنان خداوند در آنجا بودند.
\par 4 و چون روزی می‌آمد که القانه قربانی می‌گذرانید، به زن خود فننه و همه پسران و دختران خود قسمت هامی داد.
\par 5 و اما به حنا قسمت مضاعف می‌داد زیراکه حنا را دوست می‌داشت، اگر‌چه خداوند رحم او را بسته بود.
\par 6 و هئوی وی او را نیز سخت می‌رنجانید به حدی که وی را خشمناک می‌ساخت، چونکه خداوند رحم او را بسته بود.
\par 7 و همچنین سال به سال واقع می‌شد که چون حنابه خانه خدا می‌آمد، فننه همچنین او رامی رنجانید و او گریه نموده، چیزی نمی خورد.
\par 8 و شوهرش، القانه، وی را می‌گفت: «ای حنا چراگریانی و چرا نمی خوری و دلت چرا غمگین است؟ آیا من برای تو از ده پسر بهتر نیستم؟»
\par 9 و بعد از اکل و شرب نمودن ایشان در شیلوه، حنا برخاست و عیلی کاهن بر کرسی خود نزد ستونی در هیکل خدا نشسته بود.
\par 10 و او به تلخی جان نزد خداوند دعا کرد، و زارزار بگریست.
\par 11 ونذر کرده، گفت: «ای یهوه صبایوت اگر فی الواقع به مصیبت کنیز خود نظر کرده، مرا بیاد آوری وکنیزک خود را فراموش نکرده، اولاد ذکوری به کنیز خود عطا فرمایی، او را تمامی ایام عمرش به خداوند خواهم داد، و استره بر سرش نخواهدآمد.»
\par 12 و چون دعای خود را به حضور خداوندطول داد عیلی دهن او را ملاحظه کرد.
\par 13 و حنادر دل خود سخن می‌گفت، و لبهایش فقط، متحرک بود و آوازش مسموع نمی شد، و عیلی گمان برد که مست است.
\par 14 پس عیلی وی راگفت: «تا به کی مست می‌شوی؟ شرابت را از خوددور کن.»
\par 15 و حنا در جواب گفت: «نی آقایم، بلکه زن شکسته روح هستم، و شراب و مسکرات ننوشیده‌ام، بلکه جان خود را به حضور خداوندریخته‌ام.
\par 16 کنیز خود را از دختران بلیعال مشمار، زیرا که از کثرت غم و رنجیدگی خود تابحال می‌گفتم.»
\par 17 عیلی در جواب گفت: «به سلامتی برو و خدای اسرائیل مسالتی را که از اوطلب نمودی، تو را عطا فرماید.»
\par 18 گفت: «کنیزت در نظرت التفات یابد.» پس آن زن راه خود را پیش گرفت و می‌خورد و دیگر ترشرونبود.
\par 19 و ایشان بامدادان برخاسته، به حضور خداوند عبادت کردند و برگشته، به خانه خویش به رامه آمدند، و القانه زن خود حنا را بشناخت وخداوند او را به یاد آورد.
\par 20 و بعد از مرور ایام حنا حامله شده، پسری زایید و او را سموئیل نام نهاد، زیرا گفت: «او را از خداوند سوال نمودم.»
\par 21 و شوهرش القانه با تمامی اهل خانه‌اش رفت تا قربانی سالیانه و نذر خود را نزد خداوندبگذراند.
\par 22 و حنا نرفت زیرا که به شوهر خودگفته بود تا پسر از شیر باز داشته نشود، نمی آیم، آنگاه او را خواهم آورد و به حضور خداوندحاضر شده، آنجا دائم خواهد ماند.
\par 23 شوهرش القانه وی را گفت: «آنچه در نظرت پسند آید، بکن، تا وقت باز داشتنش از شیر بمان، لیکن خداوند کلام خود را استوار نماید.» پس آن زن ماند و تا وقت بازداشتن پسر خود از شیر، او راشیر می‌داد.
\par 24 و چون او را از شیر باز داشته بود، وی را باسه گاو و یک ایفه آرد و یک مشک شراب با خودآورده، به خانه خداوند در شیلوه رسانید و آن پسر کوچک بود.
\par 25 و گاو را ذبح نمودند، و پسررا نزد عیلی آوردند.
\par 26 و حنا گفت: «عرض می‌کنم‌ای آقایم! جانت زنده باد‌ای آقایم! من آن زن هستم که در اینجا نزد تو ایستاده، از خداوندمسئلت نمودم.
\par 27 برای این پسر مسالت نمودم وخداوند مسالت مرا که از او طلب نموده بودم، به من عطا فرموده است.و من نیز او را برای خداوند وقف نمودم؛ تمام ایامی که زنده باشدوقف خداوند خواهد بود.» پس در آنجا خداوند را عبادت نمودند.
\par 28 و من نیز او را برای خداوند وقف نمودم؛ تمام ایامی که زنده باشدوقف خداوند خواهد بود.» پس در آنجا خداوند را عبادت نمودند.
 
\chapter{2}

\par 1 و حنا دعا نموده، گفت: «دل من در خداوند وجد می‌نماید.و شاخ من در نزد خداوند برافراشته شده. و دهانم بر دشمنانم وسیع گردیده است. زیرا که در نجات تو شادمان هستم.
\par 2 مثل یهوه قدوسی نیست. زیرا غیر از تو کسی نیست. و مثل خدای ما صخره‌ای نیست.
\par 3 سخنان تکبرآمیز دیگر مگویید. و غرور از دهان شما صادر نشود. زیرا یهوه خدای علام است. وبه او اعمال، سنجیده می‌شود.
\par 4 کمان جباران را شکسته است. و آنانی که می‌لغزیدند، کمر آنها به قوت بسته شد.
\par 5 سیرشدگان، خویشتن را برای نان اجیر ساختند. و کسانی که گرسنه بودند، استراحت یافتند. بلکه زن نازا هفت فرزند زاییده است. و آنکه اولادبسیار داشت، زبون گردیده.
\par 6 خداوند می‌میراند و زنده می‌کند؛ به قبر فرودمی آورد و برمی خیزاند.
\par 7 خداوند فقیر می‌سازد و غنی می‌گرداند؛ پست می‌کند و بلند می‌سازد.
\par 8 فقیر را از خاک برمی افرازد. و مسکین را از مزبله برمی دارد تا ایشان را با امیران بنشاند. و ایشان راوارث کرسی جلال گرداند. زیرا که ستونهای زمین از آن خداوند است. و ربع مسکون را بر آنهااستوار نموده است.
\par 9 پایهای مقدسین خود را محفوظ می‌دارد. اماشریران در ظلمت خاموش خواهند شد، زیرا که انسان به قوت خود غالب نخواهد آمد.
\par 10 آنانی که با خداوند مخاصمه کنند، شکسته خواهند شد. او بر ایشان از آسمان صاعقه خواهدفرستاد. خداوند، اقصای زمین را داوری خواهدنمود، و به پادشاه خود قوت خواهد بخشید. وشاخ مسیح خود را بلند خواهد گردانید.»
\par 11 پس القانه به خانه خود به راما رفت و آن پسر به حضور عیلی کاهن، خداوند را خدمت می‌نمود.
\par 12 و پسران عیلی از بنی بلیعال بودند وخداوند را نشناختند.
\par 13 و عادت کاهنان با قوم این بود که چون کسی قربانی می‌گذرانید هنگامی که گوشت پخته می‌شد، خادم کاهن با چنگال سه دندانه در دست خود می‌آمد
\par 14 و آن را به تاوه یامرجل یا دیگ یا پاتیل فرو برده، هر‌چه چنگال برمی آورد کاهن آن را برای خود می‌گرفت، وهمچنین با تمامی اسرائیل که در آنجا به شیلوه می‌آمدند، رفتار می‌نمودند.
\par 15 و نیز قبل ازسوزانیدن پیه، خادم کاهن آمده، به کسی‌که قربانی می‌گذرانید، می‌گفت: «گوشت به جهت کباب برای کاهن بده، زیرا گوشت پخته از تونمی گیرد بلکه خام.»
\par 16 و آن مرد به وی می‌گفت: «پیه را اول بسوزانند و بعد هر‌چه دلت می‌خواهدبرای خود بگیر.» او می‌گفت: «نی، بلکه الان بده، والا به زور می‌گیرم.»
\par 17 پس گناهان آن جوانان به حضور خداوند بسیار عظیم بود، زیرا که مردمان هدایای خداوند را مکروه می‌داشتند.
\par 18 و اما سموئیل به حضور خداوند خدمت می کرد، و او پسر کوچک بود و بر کمرش ایفودکتان بسته بود.
\par 19 و مادرش برای وی جبه کوچک می‌ساخت، و آن را سال به سال همراه خودمی آورد، هنگامی که با شوهر خود برمی آمد تاقربانی سالیانه را بگذرانند.
\par 20 و عیلی القانه وزنش را برکت داده، گفت: «خداوند تو را از این زن به عوض عاریتی که به خداوند داده‌ای، اولادبدهد.» پس به مکان خود رفتند.
\par 21 و خداوند از حنا تفقد نمود و او حامله شده، سه پسر و دو دختر زایید، و آن پسر، سموئیل به حضور خداوند نمو می‌کرد.
\par 22 و عیلی بسیار سالخورده شده بود، و هر‌چه پسرانش با تمامی اسرائیل عمل می‌نمودند، می‌شنید، و اینکه چگونه با زنانی که نزد در خیمه اجتماع خدمت می‌کردند، می‌خوابیدند.
\par 23 پس به ایشان گفت: «چرا چنین کارها می‌کنید زیرا که اعمال بد شما را از تمامی این قوم می‌شنوم.
\par 24 چنین مکنید‌ای پسرانم زیرا خبری که می‌شنوم خوب نیست، شما باعث عصیان قوم خداوند می‌باشید.
\par 25 اگر شخصی بر شخصی گناه ورزد خدا او را داوری خواهد کرد اما اگرشخصی بر خداوند گناه ورزد، کیست که برای وی شفاعت نماید؟» اما ایشان سخن پدر خود رانشنیدند، زیرا خداوند خواست که ایشان راهلاک سازد.
\par 26 و آن پسر، سموئیل، نمو می‌یافت و هم نزدخداوند و هم نزد مردمان پسندیده می‌شد.
\par 27 و مرد خدایی نزد عیلی آمده، به وی گفت: «خداوند چنین می‌گوید: آیا خود را بر خاندان پدرت هنگامی که ایشان در مصر در خانه فرعون بودند، ظاهر نساختم؟
\par 28 و آیا او را از جمیع اسباط اسرائیل برنگزیدم تا کاهن من بوده، نزدمذبح من بیاید. و بخور بسوزاند و به حضور من ایفود بپوشد، وآیا جمیع هدایای آتشین بنی‌اسرائیل را به خاندان پدرت نبخشیدم؟
\par 29 پس چرا قربانی‌ها و هدایای مرا که در مسکن خود امر فرمودم، پایمال می‌کنید و پسران خود رازیاده از من محترم می‌داری، تا خویشتن را ازنیکوترین جمیع هدایای قوم من، اسرائیل فربه سازی.
\par 30 بنابراین یهوه، خدای اسرائیل می‌گوید: البته گفتم که خاندان تو و خاندان پدرت به حضور من تا به ابد سلوک خواهند نمود، لیکن الان خداوند می‌گوید: حاشا از من! زیرا آنانی که مرا تکریم نمایند، تکریم خواهم نمود و کسانی که مرا حقیر شمارند، خوار خواهند شد.
\par 31 اینک ایامی می‌آید که بازوی تو را و بازوی خاندان پدرتو را قطع خواهم نمود که مردی پیر در خانه تویافت نشود.
\par 32 و تنگی مسکن مرا خواهی دید، در هر احسانی که به اسرائیل خواهد شد، و مردی پیر در خانه تو ابد نخواهد بود.
\par 33 و شخصی رااز کسان تو که از مذبح خود قطع نمی نمایم برای کاهیدن چشم تو و رنجانیدن دلت خواهد بود، وجمیع ذریت خانه تو در جوانی خواهند مرد.
\par 34 واین برای تو علامت باشد که بر دو پسرت حفنی وفینحاس واقع می‌شود که هر دوی ایشان در یک روز خواهند مرد.
\par 35 و کاهن امینی به جهت خودبرپا خواهم داشت که موافق دل و جان من رفتارخواهد نمود، و برای او خانه مستحکمی بنا خواهم کرد، و به حضور مسیح من پیوسته سلوک خواهد نمود.و واقع خواهد شد که هر‌که درخانه تو باقی ماند، آمده، نزد او به جهت پاره‌ای نقره و قرص نانی تعظیم خواهد نمود و خواهدگفت: تمنا اینکه مرا به یکی از وظایف کهانت بگذار تا لقمه‌ای نان بخورم.»
\par 36 و واقع خواهد شد که هر‌که درخانه تو باقی ماند، آمده، نزد او به جهت پاره‌ای نقره و قرص نانی تعظیم خواهد نمود و خواهدگفت: تمنا اینکه مرا به یکی از وظایف کهانت بگذار تا لقمه‌ای نان بخورم.»
 
\chapter{3}

\par 1 و آن پسر، سموئیل، به حضور عیلی، خداوند را خدمت می‌نمود، و در آن روزهاکلام خداوند نادر بود و رویا مکشوف نمی شد.
\par 2 و در آن زمان واقع شد که چون عیلی در جایش خوابیده بود و چشمانش آغاز تار شدن نموده، نمی توانست دید،
\par 3 و چراغ خدا هنوز خاموش نشده، و سموئیل در هیکل خداوند، جایی که تابوت خدا بود، می‌خوابید،
\par 4 خداوند سموئیل را خواند و او گفت: «لبیک.»
\par 5 پس نزد عیلی شتافته، گفت: «اینک حاضرم زیرا مرا خواندی.» او گفت: «نخواندم، برگشته، بخواب.» و او برگشته، خوابید.
\par 6 و خداوند بار دیگر خواند: «ای سموئیل!» وسموئیل برخاسته، نزد عیلی آمده، گفت: «اینک حاضرم زیرا مرا خواندی.» او گفت: «ای پسرم تورا نخواندم، برگشته، بخواب.»
\par 7 و سموئیل، خداوند را هنوز نمی شناخت وکلام خداوند تا حال بر او منکشف نشده بود.
\par 8 وخداوند باز سموئیل را بار سوم خواند و اوبرخاسته، نزد عیلی آمده، گفت: «اینک حاضرم زیرا مرا خواندی.» آنگاه عیلی فهمید که یهوه، پسر را خوانده است.
\par 9 و عیلی به سموئیل گفت: «برو و بخواب و اگر تو را بخواند، بگو‌ای خداوندبفرما زیرا که بنده تو می‌شنود.» پس سموئیل رفته، در جای خود خوابید.
\par 10 و خداوند آمده، بایستاد و مثل دفعه های پیش خواند: «ای سموئیل! ای سموئیل!» سموئیل گفت: «بفرما زیرا که بنده تو می‌شنود.»
\par 11 و خداوند به سموئیل گفت: «اینک من کاری در اسرائیل می‌کنم که گوشهای هر‌که بشنود، صدا خواهد داد.
\par 12 در آن روز هر‌چه درباره خانه عیلی گفتم بر او اجرا خواهم داشت، و شروع نموده، به انجام خواهم رسانید. 
\par 13 زیرا به او خبردادم که من بر خانه او تا به ابد داوری خواهم نمودبه‌سبب گناهی که می‌داند، چونکه پسرانش برخود لعنت آوردند و او ایشان را منع ننمود.
\par 14 بنابراین برای خاندان عیلی قسم خوردم که گناه خاندان عیلی به قربانی و هدیه، تا به ابد کفاره نخواهد شد.»
\par 15 و سموئیل تا صبح خوابید و درهای خانه خداوند را باز کرد، و سموئیل ترسید که عیلی رااز رویا اطلاع دهد.
\par 16 اما عیلی سموئیل راخوانده، گفت: «ای پسرم سموئیل!» او گفت: «لبیک»
\par 17 گفت: «چه سخنی است که به تو گفته است؟ آن را از من مخفی مدار. خدا با تو چنین بلکه زیاده از این عمل نماید، اگر از هر‌آنچه به توگفته است چیزی از من مخفی داری.»
\par 18 پس سموئیل همه‌چیز را برای او بیان کرد و چیزی ازآن مخفی نداشت. و او گفت «خداوند است آنچه در نظر او پسند آید بکند.»
\par 19 و سموئیل بزرگ می‌شد و خداوند با وی می‌بود و نمی گذاشت که یکی از سخنانش بر زمین بیفتد.
\par 20 و تمامی اسرائیل از دان تا بئرشبع دانستند که سموئیل برقرار شده است تا نبی خداوند باشد.و خداوند بار دیگر در شیلوه ظاهر شد، زیرا که خداوند در شیلوه خود را برسموئیل به کلام خداوند ظاهر ساخت.
\par 21 و خداوند بار دیگر در شیلوه ظاهر شد، زیرا که خداوند در شیلوه خود را برسموئیل به کلام خداوند ظاهر ساخت.
 
\chapter{4}

\par 1 و کلام سموئیل به تمامی اسرائیل رسید. واسرائیل به مقابله فلسطینیان در جنگ بیرون آمده، نزد ابن عزر اردو زدند، و فلسطینیان در افیق فرود آمدند.
\par 2 و فلسطینیان در مقابل اسرائیل صف آرایی کردند، و چون جنگ درپیوستند، اسرائیل از حضور فلسطینیان شکست خوردند، و در معرکه به قدر چهار هزار نفر را درمیدان کشتند.
\par 3 و چون قوم به لشکرگاه رسیدند، مشایخ اسرائیل گفتند: «چرا امروز خداوند ما رااز حضور فلسطینیان شکست داد؟ پس تابوت عهد خداوند را از شیلوه نزد خود بیاوریم تا درمیان ما آمده، ما را از دست دشمنان ما نجات دهد.»
\par 4 و قوم به شیلوه فرستاده، تابوت عهد یهوه صبایوت را که در میان کروبیان ساکن است از آنجاآوردند، و دو پسر عیلی حفنی و فینحاس درآنجا با تابوت عهد خدا بودند.
\par 5 و چون تابوت عهد خداوند به لشکرگاه داخل شد، جمیع اسرائیل صدای بلند زدند به حدی که زمین متزلزل شد.
\par 6 و چون فلسطینیان آواز صدا را شنیدند، گفتند: «این آواز صدای بلنددر اردوی عبرانیان چیست؟» پس فهمیدند که تابوت خداوند به اردو آمده است.
\par 7 و فلسطینیان ترسیدند زیرا گفتند: «خدا به اردو آمده است» و گفتند: «وای بر ما، زیرا قبل از این‌چنین چیزی واقع نشده است!
\par 8 وای بر ما، کیست که ما را ازدست این خدایان زورآور رهایی دهد، همین خدایانند که مصریان را در بیابان به همه بلایا مبتلاساختند.
\par 9 ‌ای فلسطینیان خویشتن را تقویت داده، مردان باشید مبادا عبرانیان را بندگی کنید، چنانکه ایشان شما را بندگی نمودند، پس مردان شوید و جنگ کنید.»
\par 10 پس فلسطینیان جنگ کردند و اسرائیل شکست خورده، هر یک به خیمه خود فرار کردندو کشتار بسیار عظیمی شد، و از اسرائیل سی هزار پیاده کشته شدند.
\par 11 و تابوت خدا گرفته شد، و دو پسر عیلی حفنی و فینحاس کشته شدند.
\par 12 و مردی بنیامینی از لشکر دویده، در همان روز با جامه دریده و خاک بر سر ریخته، به شیلوه آمد.
\par 13 و چون وارد شد اینک عیلی به کنار راه برکرسی خود مراقب نشسته، زیرا که دلش درباره تابوت خدا مضطرب می‌بود. و چون آن مرد به شهر داخل شده، خبر داد، تمامی شهر نعره زدند.
\par 14 و چون عیلی آواز نعره را شنید، گفت: «این آواز هنگامه چیست؟» پس آن مرد شتافته، عیلی را خبر داد.
\par 15 و عیلی نود و هشت ساله بود وچشمانش تار شده، نمی توانست دید.
\par 16 پس آن مرد به عیلی گفت: «منم که از لشکرآمده، و من امروز از لشکر فرار کرده‌ام.» گفت: «ای پسرم کار چگونه گذشت؟»
\par 17 و آن خبرآورنده در جواب گفت: «اسرائیل از حضورفلسطینیان فرار کردند، و شکست عظیمی هم درقوم واقع شد، و نیز دو پسرت حفنی و فینحاس مردند و تابوت عهد خدا گرفته شد.»
\par 18 و چون ازتابوت خدا خبر داد، عیلی از کرسی خود به پهلوی دروازه به پشت افتاده، گردنش بشکست وبمرد، زیرا که مردی پیر و سنگین بود و چهل سال بر اسرائیل داوری کرده بود.
\par 19 و عروس او، زن فینحاس که حامله و نزدیک به زاییدن بود، چون خبر‌گرفتن تابوت خدا و مرگ پدر شوهرش وشوهرش را شنید، خم شده، زایید زیرا که درد زه او را بگرفت.
\par 20 و در وقت مردنش زنانی که نزدوی ایستاده بودند، گفتند: «مترس زیرا که پسرزاییدی، » اما او جواب نداد و اعتنا ننمود.
\par 21 وپسر را ایخابود نام نهاده، گفت: «جلال از اسرائیل زایل شد»، چونکه تابوت خدا گرفته شده بود و به‌سبب پدر شوهرش و شوهرش.پس گفت: «جلال از اسرائیل زایل شد زیرا که تابوت خداگرفته شده است.»
\par 22 پس گفت: «جلال از اسرائیل زایل شد زیرا که تابوت خداگرفته شده است.»
 
\chapter{5}

\par 1 و فلسطینیان تابوت خدا را گرفته، آن را ازابن عزر به اشدود آوردند.
\par 2 و فلسطینیان تابوت خدا را گرفته، آن را به خانه داجون درآورده، نزدیک داجون گذاشتند.
\par 3 و بامدادان چون اشدودیان برخاستند، اینک داجون به حضور تابوت خداوند رو به زمین افتاده بود. وداجون را برداشته، باز در جایش برپا داشتند.
\par 4 ودر فردای آن روز چون صبح برخاستند، اینک داجون به حضور تابوت خداوند رو به زمین افتاده، و سر داجون و دو دستش بر آستانه قطع شده، و تن داجون فقط از او باقی‌مانده بود.
\par 5 ازاین جهت کاهنان داجون و هر‌که داخل خانه داجون می‌شود تا امروز بر آستانه داجون دراشدود پا نمی گذرد.
\par 6 و دست خداوند بر اهل اشدود سنگین شده، ایشان را تباه ساخت و ایشان را هم اشدود و هم نواحی آن را به خراجها مبتلا ساخت.
\par 7 و چون مردان اشدود دیدند که چنین است گفتند تابوت خدای اسرائیل با ما نخواهد ماند، زیرا که دست او بر ما و بر خدای ما، داجون سنگین است.
\par 8 پس فرستاده، جمیع سروران فلسطینیان را نزد خودجمع کرده، گفتند: «با تابوت خدای اسرائیل چه کنیم؟» گفتند: «تابوت خدای اسرائیل به جت منتقل شود.» پس تابوت خدای اسرائیل را به آنجا بردند.
\par 9 و واقع شد بعد از نقل کردن آن که دست خداوند بر آن شهر به اضطراب بسیارعظیمی دراز شده، مردمان شهر را از خرد و بزرگ مبتلا ساخته، خراجها بر ایشان منتفخ شد.
\par 10 پس تابوت خدا را به عقرون بردند و به مجرد ورودتابوت خدا به عقرون، اهل عقرون فریاد کرده، گفتند: «تابوت خدای اسرائیل را نزد ما آوردند تاما را و قوم ما را بکشند.»
\par 11 پس فرستاده، جیمع سروران فلسطینیان را جمع کرده، گفتند: «تابوت خدای اسرائیل را روانه کنید تا به‌جای خودبرگردد و ما را و قوم ما را نکشد زیرا که در تمام شهر هنگامه مهلک بود، و دست خدا در آنجابسیار سنگین شده بود.و آنانی که نمردند به خراجها مبتلا شدند، و فریاد شهر تا به آسمان بالارفت.
\par 12 و آنانی که نمردند به خراجها مبتلا شدند، و فریاد شهر تا به آسمان بالارفت.
 
\chapter{6}

\par 1 و تابوت خداوند در ولایت فلسطینیان هفت ماه ماند.
\par 2 و فلسطینیان، کاهنان وفالگیران خود را خوانده، گفتند: «با تابوت خداوند چه کنیم؟ ما را اعلام نمایید که آن را به‌جایش با چه چیز بفرستیم.»
\par 3 گفتند: «اگر تابوت خدای اسرائیل را بفرستید آن را خالی مفرستید، بلکه قربانی جرم البته برای او بفرستید، آنگاه شفاخواهید یافت، و بر شما معلوم خواهد شد که ازچه سبب دست او از شما برداشته نشده است.»
\par 4 ایشان گفتند: «چه قربانی جرم برای اوبفرستیم؟»
\par 5 پس تماثیل خراجهای خود و تماثیل موشهای خود را که زمین را خراب می‌کنند بسازید، وخدای اسرائیل را جلال دهید که شاید دست خود را از شما و از خدایان شما و از زمین شمابردارد.
\par 6 و چرا دل خود را سخت سازید، چنانکه مصریان و فرعون دل خود را سخت ساختند؟ آیابعد از آنکه در میان ایشان کارهای عجیب کرده بود ایشان را رها نکردند که رفتند؟
\par 7 پس الان ارابه تازه بسازید و دو گاو شیرده را که یوغ برگردن ایشان نهاده نشده باشد بگیرید، و دو گاو رابه ارابه ببندید و گوساله های آنها را از عقب آنها به خانه برگردانید.
\par 8 و تابوت خداوند را گرفته، آن رابر ارابه بنهید و اسباب طلا را که به جهت قربانی جرم برای او می‌فرستید در صندوقچه‌ای به پهلوی آن بگذارید، و آن را رها کنید تا برود.
\par 9 ونظر کنید اگر به راه سرحد خود به سوی بیت شمس برود، بدانید اوست که این بلای عظیم را بر ما وارد گردانیده است، و اگرنه، پس خواهیددانست که دست او ما را لمس نکرده است، بلکه آنچه بر ما واقع شده است، اتفاقی است.»
\par 10 پس آن مردمان چنین کردند و دو گاوشیرده را گرفته، آنها را به ارابه بستند، وگوساله های آنها را در خانه نگاه داشتند.
\par 11 و تابوت خداوند و صندوقچه را با موشهای طلا وتماثیل خراجهای خود بر ارابه گذاشتند.
\par 12 وگاوان راه خود را راست گرفته، به راه بیت شمس روانه شدند و به شاهراه رفته، بانگ می‌زدند و به سوی چپ یا راست میل نمی نمودند، و سروران فلسطینیان در عقب آنها تا حد بیت شمس رفتند.
\par 13 و اهل بیت شمس در دره گندم را درومی کردند، و چشمان خود را بلند کرده، تابوت رادیدند و از دیدنش خوشحال شدند.
\par 14 و ارابه به مزرعه یهوشع بیت شمسی درآمده، در آنجابایستاد و سنگ بزرگی در آنجا بود. پس چوب ارابه را شکسته، گاوان را برای قربانی سوختنی به جهت خداوند گذرانیدند.
\par 15 و لاویان تابوت خداوند و صندوقچه‌ای را که با آن بود و اسباب طلا داشت، پایین آورده، آنها را بر آن سنگ بزرگ نهادند و مردان بیت شمس در همان روز برای خداوند قربانی های سوختنی گذرانیدند و ذبایح ذبح نمودند.
\par 16 و چون آن پنج سرور فلسطینیان این را دیدند، در همان روز به عقرون برگشتند.
\par 17 و این است خراجهای طلایی که فلسطینیان به جهت قربانی جرم نزد خداوند فرستادند: برای اشدود یک، و برای غزه یک، و برای اشقلون یک، و برای جت یک، و برای عقرون یک.
\par 18 وموشهای طلا بر‌حسب شماره جمیع شهرهای فلسطینیان که از املاک آن پنج سرور بود، چه ازشهرهای حصاردار و چه از دهات بیرون تا آن سنگ بزرگی که تابوت خداوند را بر آن گذاشتندکه تا امروز در مزرعه یهوشع بیت شمسی باقی است.
\par 19 و مردمان بیت شمس را زد، زیرا که به تابوت خداوند نگریستند، پس پنجاه هزار وهفتاد نفر از قوم را زد و قوم ماتم گرفتند، چونکه خداوند خلق را به بلای عظیم مبتلا ساخته بود.
\par 20 و مردمان بیت شمس گفتند: «کیست که به حضور این خدای قدوس یعنی یهوه می‌تواندبایستد و از ما نزد که خواهد رفت؟»پس رسولان نزد ساکنان قریه یعاریم فرستاده، گفتند: «فلسطینیان تابوت خداوند را پس فرستاده‌اند، بیایید و آن را نزد خود ببرید.»
\par 21 پس رسولان نزد ساکنان قریه یعاریم فرستاده، گفتند: «فلسطینیان تابوت خداوند را پس فرستاده‌اند، بیایید و آن را نزد خود ببرید.»
 
\chapter{7}

\par 1 خداوند را آوردند، و آن را به خانه ابیناداب در جبعه داخل کرده، پسرش العازار راتقدیس نمودند تا تابوت خداوند را نگاهبانی کند.
\par 2 و از روزی که تابوت در قریه یعاریم ساکن شد، وقت طول کشید تا بیست سال گذشت، و بعداز آن خاندان اسرائیل برای پیروی خداوند جمع شدند.
\par 3 و سموئیل تمامی خاندان اسرائیل را خطاب کرده، گفت: «اگر به تمامی دل به سوی خداوندبازگشت نمایید، و خدایان غیر و عشتاروت را ازمیان خود دور کنید، و دلهای خود را برای خداوند حاضر ساخته، او را تنها عبادت نمایید، پس او، شما را از دست فلسطینیان خواهدرهانید.»
\par 4 آنگاه بنی‌اسرائیل بعلیم و عشتاروت رادور کرده، خداوند را تنها عبادت نمودند.
\par 5 و سموئیل گفت: «تمامی اسرائیل را درمصفه جمع کنید تا درباره شما نزد خداوند دعانمایم.»
\par 6 و در مصفه جمع شدند و آب کشیده، آن را به حضور خداوند ریختند، و آن روز راروزه داشته، در آنجا گفتند که بر خداوند گناه کرده‌ایم، و سموئیل بنی‌اسرائیل را در مصفه داوری نمود. 
\par 7 و چون فلسطینیان شنیدند که بنی‌اسرائیل درمصفه جمع شده‌اند، سروران فلسطینیان براسرائیل برآمدند، و بنی‌اسرائیل چون این راشنیدند، از فلسطینیان ترسیدند.
\par 8 و بنی‌اسرائیل به سموئیل گفتند: «از تضرع نمودن برای ما نزدیهوه خدای ما ساکت مباش تا ما را از دست فلسطینیان برهاند.»
\par 9 و سموئیل بره شیرخواره گرفته، آن را به جهت قربانی سوختنی تمام برای خداوند گذرانید، و سموئیل درباره اسرائیل نزدخداوند تضرع نموده، خداوند او را اجابت نمود.
\par 10 و چون سموئیل قربانی سوختنی رامی گذرانید، فلسطینیان برای مقاتله اسرائیل نزدیک آمدند، و در آن روز خداوند به صدای عظیم بر فلسطینیان رعد کرده، ایشان را منهزم ساخت، و از حضور اسرائیل شکست یافتند.
\par 11 ومردان اسرائیل از مصفه بیرون آمدند و فلسطینیان را تعاقب نموده، ایشان را تا زیر بیت کار شکست دادند.
\par 12 و سموئیل سنگی گرفته، آن را میان مصفه وسن برپا داشت و آن را ابن عزر نامیده، گفت: «تابحال خداوند ما را اعانت نموده است.»
\par 13 پس فلسطینیان مغلوب شدند، و دیگر به حدوداسرائیل داخل نشدند، و دست خداوند در تمامی روزهای سموئیل بر فلسطینیان سخت بود.
\par 14 وشهرهایی که فلسطینیان از اسرائیل گرفته بودند، از عقرون تا جت، به اسرائیل پس دادند، و اسرائیل حدود آنها را از دست فلسطینیان رهانیدند، و در میان اسرائیل و اموریان صلح شد.
\par 15 و سموئیل در تمام روزهای عمر خود براسرائیل داوری می‌نمود.
\par 16 و هر سال رفته، به بیت ئیل و جلجال و مصفه گردش می‌کردند، و درتمامی این‌جاها بر اسرائیل داوری می‌نمود.وبه رامه بر می‌گشت زیرا خانه‌اش در آنجا بود و درآنجا بر اسرائیل داوری می‌نمود، و مذبحی درآنجا برای خداوند بنا کرد.
\par 17 وبه رامه بر می‌گشت زیرا خانه‌اش در آنجا بود و درآنجا بر اسرائیل داوری می‌نمود، و مذبحی درآنجا برای خداوند بنا کرد.
 
\chapter{8}

\par 1 و واقع شد که چون سموئیل پیر شد، پسران خود را بر اسرائیل داوران ساخت.
\par 2 و نام پسر نخست زاده‌اش یوئیل بود و نام دومینش ابیاه؛ و در بئرشبع داور بودند.
\par 3 اماپسرانش به راه او رفتار نمی نمودند بلکه در‌پی سود رفته، رشوه می‌گرفتند و داوری را منحرف می‌ساختند.
\par 4 پس جمیع مشایخ اسرائیل جمع شده، نزدسموئیل به رامه آمدند.
\par 5 و او را گفتند: «اینک توپیر شده‌ای و پسرانت به راه تو رفتار نمی نمایند، پس الان برای ما پادشاهی نصب نما تا مثل سایرامتها بر ما حکومت نماید.»
\par 6 و این امر در نظرسموئیل ناپسند آمد، چونکه گفتند: «ما راپادشاهی بده تا بر ما حکومت نماید.» و سموئیل نزد خداوند دعا کرد.
\par 7 و خداوند به سموئیل گفت: «آواز قوم را در هر‌چه به تو گفتند بشنو، زیرا که تو را ترک نکردند بلکه مرا ترک کردند تا برایشان پادشاهی ننمایم.
\par 8 بر‌حسب همه اعمالی که از روزی که ایشان را از مصر بیرون آوردم، بجا آوردند و مرا ترک نموده، خدایان غیر را عبادت نمودند پس با تو نیز همچنین رفتار می‌نمایند.
\par 9 پس الان آواز ایشان را بشنو لکن بر ایشان به تاکید شهادت بده، و ایشان را از رسم پادشاهی که بر ایشان حکومت خواهد نمود، مطلع ساز.»
\par 10 و سموئیل تمامی سخنان خداوند را به قوم که از او پادشاه خواسته بودند، بیان کرد.
\par 11 وگفت: «رسم پادشاهی که بر شما حکم خواهدنمود این است که پسران شما را گرفته، ایشان را برارابه‌ها و سواران خود خواهد گماشت و پیش ارابه هایش خواهند دوید.
\par 12 و ایشان را سرداران هزاره و سرداران پنجاهه برای خودخواهدساخت، و بعضی را برای شیار کردن زمینش و درویدن محصولش و ساختن آلات جنگش و اسباب ارابه هایش تعیین خواهد نمود.
\par 13 و دختران شما را برای عطرکشی و طباخی وخبازی خواهد گرفت.
\par 14 و بهترین مزرعه‌ها وتاکستانها و باغات زیتون شما را گرفته، به خادمان خود خواهد داد.
\par 15 و عشر زراعات وتاکستانهای شما را گرفته، به خواجه‌سرایان وخادمان خود خواهد داد.
\par 16 و غلامان و کنیزان ونیکوترین جوانان شما را و الاغهای شما را گرفته، برای کار خود خواهد گماشت.
\par 17 و عشرگله های شما را خواهد گرفت و شما غلام اوخواهید بود.
\par 18 و در آن روز از دست پادشاه خودکه برای خویشتن برگزیده‌اید فریاد خواهید کرد وخداوند در آن روز شما را اجابت نخواهد نمود.»
\par 19 اما قوم از شنیدن قول سموئیل ابا نمودند وگفتند: «نی بلکه می‌باید بر ما پادشاهی باشد.
\par 20 تاما نیز مثل سایر امتها باشیم و پادشاه ما بر ماداوری کند، و پیش روی ما بیرون رفته، درجنگهای ما برای ما بجنگد.»و سموئیل تمامی سخنان قوم را شنیده، آنها را به سمع خداوند رسانید. و خداوند به سموئیل گفت: «آواز ایشان را بشنو و پادشاهی بر ایشان نصب نما.» پس سموئیل به مردمان اسرائیل گفت: «شماهرکس به شهر خود بروید.»
\par 21 و سموئیل تمامی سخنان قوم را شنیده، آنها را به سمع خداوند رسانید. و خداوند به سموئیل گفت: «آواز ایشان را بشنو و پادشاهی بر ایشان نصب نما.» پس سموئیل به مردمان اسرائیل گفت: «شماهرکس به شهر خود بروید.»
 
\chapter{9}

\par 1 و مردی بود از بنیامین که اسمش قیس بن ابیئیل بن صرور بن بکورت بن افیح بود، و اوپسر مرد بنیامینی و مردی زورآور مقتدر بود.
\par 2 واو را پسری شاول نام، جوانی خوش اندام بود که در میان بنی‌اسرائیل کسی از او خوش اندام تر نبودکه از کتفش تا به بالا از تمامی قوم بلندتر بود.
\par 3 و الاغهای قیس پدر شاول گم شد. پس قیس به پسر خود شاول گفت: «الان یکی از جوانان خود را با خود گرفته، برخیز و رفته، الاغها راجستجو نما.»
\par 4 پس از کوهستان افرایم گذشته، واز زمین شلیشه عبور نموده، آنها را نیافتند. و اززمین شعلیم گذشتند و نبود و از زمین بنیامین گذشته، آنها را نیافتند.
\par 5 و چون به زمین صوف رسیدند، شاول به خادمی که همراهش بود، گفت: «بیا برگردیم مباداپدرم از فکر الاغها گذشته، به فکر ما افتد.»
\par 6 او درجواب وی گفت: «اینک مرد خدایی در این شهراست و او مردی مکرم است و هر‌چه می‌گویدالبته واقع می‌شود. الان آنجا برویم؛ شاید از راهی که باید برویم ما را اطلاع بدهد.»
\par 7 شاول به خادمش گفت: «اینک اگر برویم چه چیز برای آن مرد ببریم، زیرا نان از ظروف ما تمام شده، وهدیه‌ای نیست که به آن مرد خدا بدهیم. پس چه چیز داریم.»
\par 8 و آن خادم باز در جواب شاول گفت که «اینک در دستم ربع مثقال نقره است. آن را به مرد خدا می‌دهم تا راه ما را به ما نشان دهد.»
\par 9 در زمان سابق چون کسی در اسرائیل برای درخواست کردن از خدا می‌رفت، چنین می‌گفت: «بیایید تا نزد رائی برویم.» زیرا نبی امروز را سابق رائی می‌گفتند.
\par 10 و شاول به خادم خود گفت: «سخن تو نیکوست. بیا برویم.» پس به شهری که مرد خدا در آن بود، رفتند.
\par 11 و چون ایشان به فراز شهر بالا می‌رفتند، دختران چند یافتند که برای آب کشیدن بیرون می‌آمدند و به ایشان گفتند: «آیا رائی دراینجاست؟»
\par 12 در جواب ایشان گفتند: «بلی اینک پیش روی شماست. حال بشتابید زیراامروز به شهر آمده است چونکه امروز قوم را درمکان بلند قربانی هست.
\par 13 به مجرد ورود شما به شهر، قبل از آنکه به مکان بلند برای خوردن بیایید، به او خواهید برخورد زیرا که تا او نیایدقوم غذا نخواهند خورد، چونکه او می‌باید اول قربانی را برکت دهد و بعد از آن دعوت‌شدگان بخورند. پس اینک بروید زیرا که الان او راخواهید یافت.»
\par 14 پس به شهر رفتند و چون داخل شهر می‌شدند، اینک سموئیل به مقابل ایشان بیرون آمد تا به مکان بلند برود.
\par 15 و یک روز قبل از آمدن شاول خداوند برسموئیل کشف نموده، گفت:
\par 16 «فردا مثل این وقت شخصی را از زمین بنیامین نزد تو می‌فرستم، او را مسح نما تا بر قوم من اسرائیل رئیس باشد، وقوم مرا از دست فلسطینیان رهایی دهد. زیرا که برقوم خود نظر کردم چونکه تضرع ایشان نزد من رسید.»
\par 17 و چون سموئیل شاول را دید، خداوند او را گفت: «اینک این است شخصی که درباره‌اش به تو گفتم که بر قوم من حکومت خواهد نمود.»
\par 18 و شاول در میان دروازه به سموئیل نزدیک آمده، گفت: «مرا بگو که خانه رائی کجاست؟»
\par 19 سموئیل در جواب شاول گفت: «من رائی هستم. پیش من به مکان بلند برو زیرا که شماامروز با من خواهید خورد، و بامدادان تو را رهاکرده، هرچه در دل خود داری برای تو بیان خواهم کرد.
\par 20 و اما الاغهایت که سه روز قبل ازاین گم شده است، درباره آنها فکر مکن زیرا پیداشده است، و آرزوی تمامی اسرائیل بر کیست؟ آیا بر تو و بر تمامی خاندان پدر تو نیست؟»
\par 21 شاول در جواب گفت: «آیا من بنیامینی و ازکوچک ترین اسباط بنی‌اسرائیل نیستم؟ و آیاقبیله من از جمیع قبایل سبط بنیامین کوچکترنیست؟ پس چرا مثل این سخنان به من می‌گویی؟»
\par 22 و سموئیل شاول و خادمش را گرفته، ایشان را به مهمانخانه آورد و بر صدردعوت‌شدگان که قریب به سی نفر بودند، جا داد.
\par 23 و سموئیل به طباخ گفت: «قسمتی را که به تودادم و درباره‌اش به تو گفتم که پیش خودنگاهدار، بیاور.»
\par 24 پس طباخ ران را با هرچه برآن بود، گرفته، پیش شاول گذاشت و سموئیل گفت: «اینک آنچه نگاهداشته شده است، پیش خود بگذار و بخور زیرا که تا زمان معین برای تونگاه داشته شده است، از وقتی که گفتم از قوم وعده بخواهم.»
\par 25 وچون ایشان از مکان بلند به شهر آمدند، او با شاول بر پشت بام گفتگو کرد.
\par 26 و صبح زود برخاستندو نزد طلوع فجر سموئیل شاول را بر پشت بام خوانده، گفت: «برخیز تا تو را روانه نمایم.» پس شاول برخاست و هر دوی ایشان، او و سموئیل بیرون رفتند.و چون ایشان به کنار شهر رسیدند، سموئیل به شاول گفت: «خادم را بگو که پیش مابرود. (و او پیش رفت ) و اما تو الان بایست تا کلام خدا را به تو بشنوانم.»
\par 27 و چون ایشان به کنار شهر رسیدند، سموئیل به شاول گفت: «خادم را بگو که پیش مابرود. (و او پیش رفت ) و اما تو الان بایست تا کلام خدا را به تو بشنوانم.»
 
\chapter{10}

\par 1 پس سموئیل ظرف روغن را گرفته، برسر وی ریخت و او را بوسیده، گفت: «آیا این نیست که خداوند تو را مسح کرد تا برمیراث او حاکم شوی؟
\par 2 امروز بعد از رفتنت ازنزد من دو مرد، نزد قبر راحیل به‌سرحد بنیامین درصلصح خواهی یافت، و تو را خواهند گفت: الاغهایی که برای جستن آنها رفته بودی، پیداشده است و اینک پدرت فکر الاغها را ترک کرده، به فکر شما افتاده است، و می‌گوید به جهت پسرم چکنم.
\par 3 چون از آنجا پیش رفتی و نزد بلوط تابوررسیدی در آنجا سه مرد خواهی یافت که به حضور خدا به بیت ئیل می‌روند که یکی از آنها سه بزغاله دارد، و دیگری سه قرص نان، و سومی یک مشگ شراب.
\par 4 و سلامتی تو را خواهند پرسید ودو نان به تو خواهندداد که از دست ایشان خواهی گرفت.
\par 5 بعد از آن به جبعه خدا که در آنجا قراول فلسطینیان است خواهی آمد، و چون در آنجانزدیک شهر برسی گروهی از انبیا که از مکان بلندبه زیر می‌آیند و در‌پیش ایشان چنگ و دف و نای و بربط بوده، نبوت می‌کنند، به تو خواهندبرخورد.
\par 6 و روح خداوند بر تو مستولی شده، باایشان نبوت خواهی نمود، و به مرد دیگر متبدل خواهی شد.
\par 7 و هنگامی که این علامات به تورونماید، هرچه دستت یابد بکن زیرا خدا باتوست.
\par 8 و پیش من به جلجال برو و اینک من برای گذرانیدن قربانی های سوختنی و ذبح نمودن ذبایح سلامتی نزد تو می‌آیم، و هفت روز منتظرباش تا نزد تو بیایم و تو را اعلام نمایم که چه بایدکرد.»
\par 9 و چون رو گردانید تا از نزد سموئیل برود، خدا او را قلب دیگر داد. و در آن روز جمیع این علامات واقع شد.
\par 10 و چون آنجا به جبعه رسیدند، اینک گروهی از انبیا به وی برخوردند، وروح خدا بر او مستولی شده، در میان ایشان نبوت می‌کرد.
\par 11 و چون همه کسانی که او را پیشترمی شناختند، دیدند که اینک با انبیا نبوت می‌کند، مردم به یکدیگر گفتند: «این چیست که با پسرقیس واقع شده است، آیا شاول نیز از‌جمله انبیااست؟»
\par 12 و یکی از حاضرین در جواب گفت: «اما پدر ایشان کیست؟» از این جهت مثل شد که آیا شاول نیز از‌جمله انبیا است. 
\par 13 و چون ازنبوت کردن فارغ شد به مکان بلند آمد.
\par 14 و عموی شاول به او و به خادمش گفت: «کجا رفته بودید؟» او در جواب گفت: «برای جستن الاغها و چون دیدیم که نیستند، نزدسموئیل رفتیم.»
\par 15 عموی شاول گفت: «مرا بگوکه سموئیل به شما چه گفت؟»
\par 16 شاول به عموی خود گفت: «ما را واضح خبر داد که الاغها پیداشده است.» لیکن درباره امر سلطنت که سموئیل به او گفته بود، او را مخبر نساخت.
\par 17 و سموئیل قوم را در مصفه به حضورخداوند خواند
\par 18 و به بنی‌اسرائیل گفت: «یهوه، خدای اسرائیل، چنین می‌گوید: من اسرائیل را ازمصر برآوردم، و شما از دست مصریان و از دست جمیع ممالکی که بر شما ظلم نمودند، رهایی دادم.
\par 19 و شما امروز خدای خود را که شما را ازتمامی بدیها و مصیبت های شما رهانید، اهانت کرده، او را گفتید: پادشاهی بر ما نصب نما، پس الان با اسباط و هزاره های خود به حضور خداوندحاضر شوید.»
\par 20 و چون سموئیل جمیع اسباط اسرائیل راحاضر کرد، سبط بنیامین گرفته شد.
\par 21 و سبطبنیامین را با قبایل ایشان نزدیک آورد، و قبیله مطری گرفته شد. و شاول پسر قیس گرفته شد، وچون او را طلبیدند، نیافتند.
\par 22 پس بار دیگر ازخداوند سوال کردند که آیا آن مرد به اینجا دیگرخواهد آمد؟ خداوند در جواب گفت: «اینک اوخود را در میان اسبابها پنهان کرده است.»
\par 23 ودویده، او را از آنجا آوردند، و چون در میان قوم بایستاد، از تمامی قوم از کتف به بالا بلندتر بود.
\par 24 و سموئیل به تمامی قوم گفت: «آیا شخصی راکه خداوند برگزیده است ملاحظه نمودید که درتمامی قوم مثل او کسی نیست؟» و تمامی قوم صدا زده، گفتند: «پادشاه زنده بماند!»
\par 25 پس سموئیل رسوم سلطنت را به قوم بیان کرده، در کتاب نوشت، و آن را به حضور خداوندگذاشت. و سموئیل هرکس از تمامی قوم را به خانه‌اش روانه نمود.
\par 26 و شاول نیز به خانه خودبه جبعه رفت و فوجی از کسانی که خدا دل ایشان را برانگیخت همراه وی رفتند.اما بعضی پسران بلیعال گفتند: «این شخص چگونه ما رابرهاند؟» و او را حقیر شمرده، هدیه برایش نیاوردند. اما او هیچ نگفت.
\par 27 اما بعضی پسران بلیعال گفتند: «این شخص چگونه ما رابرهاند؟» و او را حقیر شمرده، هدیه برایش نیاوردند. اما او هیچ نگفت.
 
\chapter{11}

\par 1 و ناحاش عمونی برآمده، در برابر یابیش جلعاد اردو زد، و جمیع اهل یابیش به ناحاش گفتند: «با ما عهد ببند و تو را بندگی خواهیم نمود.»
\par 2 ناحاش عمونی به ایشان گفت: «به این شرط با شما عهد خواهم بست که چشمان راست جمیع شما کنده شود، و این را بر تمامی اسرائیل عار خواهم ساخت.»
\par 3 و مشایخ یابیش به وی گفتند: «ما را هفت روز مهلت بده تا رسولان به تمامی حدود اسرائیل بفرستیم، و اگر برای مارهاننده‌ای نباشد نزد تو بیرون خواهیم آمد.»
\par 4 پس رسولان به جبعه شاول آمده، این سخنان رابه گوش قوم رسانیدند، و تمامی قوم آواز خود رابلند کرده، گریستند.
\par 5 و اینک شاول در عقب گاوان از صحرامی آمد، و شاول گفت: «قوم را چه شده است که می‌گریند؟» پس سخنان مردان یابیش را به او بازگفتند.
\par 6 و چون شاول این سخنان را شنید روح خدا بر وی مستولی گشته، خشمش به شدت افروخته شد.
\par 7 پس یک جفت گاو را گرفته، آنهارا پاره پاره نمود و به‌دست قاصدان به تمامی حدود اسرائیل فرستاده، گفت: «هر‌که در عقب شاول و سموئیل بیرون نیاید، به گاوان او چنین کرده شود.» آنگاه ترس خداوند بر قوم افتاد که مثل مرد واحد بیرون آمدند.
\par 8 و ایشان را در بازق شمرد و بنی‌اسرائیل سیصد هزار نفر و مردان یهودا سی هزار بودند.
\par 9 پس به رسولانی که آمده بودند گفتند: «به مردمان یابیش جلعاد چنین گویید: فردا وقتی که آفتاب گرم شود برای شماخلاصی خواهد شد.» و رسولان آمده، به اهل یابیش خبر دادند، پس ایشان شاد شدند.
\par 10 ومردان یابیش گفتند: «فردا نزد شما بیرون خواهیم آمد تا هرچه در نظرتان پسند آید به ما بکنید.»
\par 11 و در فردای آن روز شاول قوم را به سه فرقه تقسیم نمود و ایشان در پاس صبح به میان لشکرگاه آمده، عمونیان را تا گرم شدن آفتاب می‌زدند، و باقی ماندگان پراکنده شدند به حدی که دو نفر از ایشان در یک جا نماندند.
\par 12 و قوم به سموئیل گفتند: «کیست که گفته است! آیا شاول بر ما سلطنت نماید؟ آن کسان رابیاورید تا ایشان را بکشیم.»
\par 13 اما شاول گفت: «کسی‌امروز کشته نخواهد شد زیرا که خداوندامروز در اسرائیل نجات به عمل آورده است.»
\par 14 و سموئیل به قوم گفت: «بیایید تا به جلجال برویم و سلطنت را در آنجا از سر نو برقرار کنیم.»پس تمامی قوم به جلجال رفتند، و آنجا درجلجال، شاول را به حضور خداوند پادشاه ساختند، و در آنجا ذبایح سلامتی به حضورخداوند ذبح نموده، شاول و تمامی مردمان اسرائیل در آنجا شادی عظیم نمودند.
\par 15 پس تمامی قوم به جلجال رفتند، و آنجا درجلجال، شاول را به حضور خداوند پادشاه ساختند، و در آنجا ذبایح سلامتی به حضورخداوند ذبح نموده، شاول و تمامی مردمان اسرائیل در آنجا شادی عظیم نمودند.
 
\chapter{12}

\par 1 و سموئیل به تمامی بنی‌اسرائیل گفت: «اینک قول شما را در هرآنچه به من گفتید، شنیدم و پادشاهی بر شما نصب نمودم.
\par 2 وحال اینک پادشاه پیش روی شما راه می‌رود و من پیر و مو سفید شده‌ام، و اینک پسران من با شمامی باشند، و من از جوانیم تا امروز پیش روی شماسلوک نموده‌ام.
\par 3 اینک من حاضرم، پس به حضور خداوند و مسیح او بر من شهادت دهید که گاو که را گرفتم؟ و الاغ که را گرفتم و بر که ظلم نموده، که را ستم کردم و از دست که رشوه گرفتم تا چشمان خود را به آن کور سازم و آن را به شمارد نمایم.»
\par 4 گفتند: «بر ما ظلم نکرده‌ای و بر ما ستم ننموده‌ای و چیزی از دست کسی نگرفته‌ای.»
\par 5 به ایشان گفت: «خداوند بر شماشاهد است و مسیح او امروز شاهد است که چیزی در دست من نیافته‌اید.» گفتند: «او شاهداست.»
\par 6 و سموئیل به قوم گفت: «خداوند است که موسی و هارون را مقیم ساخت و پدران شما را اززمین مصر برآورد.
\par 7 پس الان حاضر شوید تا به حضور خداوند با شما درباره همه اعمال عادله خداوند که با شما و با پدران شما عمل نمود، محاجه نمایم.
\par 8 چون یعقوب به مصر آمد وپدران شما نزد خداوند استغاثه نمودند، خداوندموسی و هارون را فرستاد که پدران شما را از مصربیرون آورده، ایشان را در این مکان ساکن گردانیدند.
\par 9 و چون یهوه خدای خود را فراموش کردند ایشان را به‌دست سیسرا، سردار لشکرحاصور، و به‌دست فلسطینیان و به‌دست پادشاه موآب فروخت که با آنها جنگ کردند.
\par 10 پس نزدخداوند فریاد برآورده، گفتند: «گناه کرده‌ایم زیراخداوند را ترک کرده بعلیم و عشتاروت را عبادت نموده‌ایم، و حال ما را از دست دشمنان ما رهایی ده و تو را عبادت خواهیم نمود.
\par 11 پس خداوندیربعل و بدان و یفتاح و سموئیل را فرستاده، شمارا از دست دشمنان شما که در اطراف شما بودند، رهانید و در اطمینان ساکن شدید.
\par 12 و چون دیدید که ناحاش، پادشاه بنی عمون، بر شمامی آید به من گفتید: نی بلکه پادشاهی بر ماسلطنت نماید، و حال آنکه یهوه، خدای شما، پادشاه شما بود.
\par 13 و الان اینک پادشاهی که برگزیدید و او را طلبیدید. و همانا خداوند بر شماپادشاهی نصب نموده است.
\par 14 اگر از خداوندترسیده، او را عبادت نمایید و قول او را بشنوید واز فرمان خداوند عصیان نورزید، و هم شما و هم پادشاهی که بر شما سلطنت می‌کند، یهوه، خدای خود را پیروی نمایید خوب.
\par 15 و اما اگر قول خداوند را نشنوید و از فرمان خداوند عصیان ورزید، آنگاه دست خداوند چنانکه به ضد پدران شما بود، به ضد شما نیز خواهدبود.
\par 16 پس الان بایستید و این کار عظیم را که خداوند به نظر شمابجا می‌آورد، ببینید.
\par 17 آیا امروز وقت درو گندم نیست؟ از خداوند استدعا خواهم نمود و اورعدها و باران خواهد فرستاد تا بدانید و ببینید که شرارتی که از طلبیدن پادشاه برای خود نمودیددر نظر خداوند عظیم است.»
\par 18 پس سموئیل ازخداوند استدعا نمود و خداوند در همان روزرعدها و باران فرستاد، و تمامی قوم از خداوند وسموئیل بسیار ترسیدند.
\par 19 و تمامی قوم به سموئیل گفتند: «برای بندگانت از یهوه، خدای خود استدعا نما تانمیریم، زیرا که بر تمامی گناهان خود این بدی راافزودیم که برای خود پادشاهی طلبیدیم.»
\par 20 وسموئیل به قوم گفت: «مترسید! شما تمامی این بدی را کرده‌اید، لیکن از پیروی خداوندبرنگردید، بلکه خداوند را به تمامی دل خودعبادت نمایید.
\par 21 و در عقب اباطیلی که منفعت ندارد و رهایی نتواند داد، چونکه باطل است، برنگردید.
\par 22 زیرا خداوند به‌خاطر نام عظیم خود قوم خود را ترک نخواهد نمود، چونکه خداوند را پسند آمد که شما را برای خود قومی سازد.
\par 23 و اما من، حاشا از من که به خداوند گناه ورزیده، ترک دعا کردن برای شما نمایم، بلکه راه نیکو و راست را به شما تعلیم خواهم داد.
\par 24 لیکن از خداوند بترسید و او را به راستی به تمامی دل خود عبادت نمایید و در کارهای عظیمی که برای شما کرده است، تفکر کنید.و اما اگر شرارت ورزید، هم شما و هم پادشاه شما، هلاک خواهیدشد.»
\par 25 و اما اگر شرارت ورزید، هم شما و هم پادشاه شما، هلاک خواهیدشد.»
 
\chapter{13}

\par 1 و شاول (سی ) ساله بود که پادشاه شد. و چون دو سال بر اسرائیل سلطنت نموده بود،
\par 2 شاول به جهت خود سه هزار نفر ازاسرائیل برگزید، و از ایشان دو هزار با شاول درمخماس و در کوه بیت ئیل بودند، و یک هزار بایوناتان در جبعه بنیامین. و اما هرکس از بقیه قوم رابه خیمه‌اش فرستاد.
\par 3 و یوناتان قراول فلسطینیان را که در جبعه بودند، شکست داد. و فلسطینیان این را شنیدند. و شاول در تمامی زمین کرنانواخته، گفت که «ای عبرانیان بشنوید!»
\par 4 و چون تمامی اسرائیل شنیدند که شاول قراول فلسطینیان را شکست داده است، و اینکه اسرائیل نزد فلسطینیان مکروه شده‌اند، قوم نزد شاول درجلجال جمع شدند.
\par 5 و فلسطینیان سی هزار ارابه و شش هزارسوار و خلقی را که مثل ریگ کناره دریا بیشماربودند، جمع کردند تا با اسرائیل جنگ نمایند، وبرآمده، در مخماس به طرف شرقی بیت آون اردوزدند.
\par 6 و چون اسرائیلیان را دیدند که در تنگی هستند زیرا که قوم مضطرب بودند، پس ایشان خود را در مغاره‌ها و بیشه‌ها و گریوه‌ها و حفره هاو صخره‌ها پنهان کردند.
\par 7 و بعضی از عبرانیان ازاردن به زمین جاد و جلعاد عبور کردند. و شاول هنوز در جلجال بود و تمامی قوم در عقب او لرزان بودند.
\par 8 پس هفت روز موافق وقتی که سموئیل تعیین نموده بود، درنگ کرد. اما سموئیل به جلجال نیامد و قوم از او پراکنده می‌شدند.
\par 9 وشاول گفت: «قربانی سوختنی و ذبایح سلامتی رانزد من بیاورید.» و قربانی سوختنی را گذرانید.
\par 10 و چون از گذرانیدن قربانی سوختنی فارغ شد، اینک سموئیل برسید و شاول به جهت تحیتش، به استقبال وی بیرون آمد.
\par 11 و سموئیل گفت: «چه کردی؟» شاول گفت: «چون دیدم که قوم ازنزد من پراکنده می‌شوند و تو در روزهای معین نیامدی و فلسطینیان در مخماس جمع شده‌اند،
\par 12 پس گفتم: الان فلسطینیان بر من در جلجال فرود خواهند آمد، و من رضامندی خداوند رانطلبیدم. پس خویشتن را مجبور ساخته، قربانی سوختنی را گذرانیدم.»
\par 13 و سموئیل به شاول گفت: «احمقانه عمل نمودی و امری که یهوه خدایت به تو امر فرموده است، بجا نیاوردی، زیرا که حال خداوندسلطنت تو را بر اسرائیل تا به ابد برقرار می‌داشت.
\par 14 لیکن الان سلطنت تو استوار نخواهد ماند وخداوند به جهت خویش مردی موافق دل خودطلب نموده است، و خداوند او را مامور کرده است که پیشوای قوم وی باشد، چونکه تو فرمان خداوند را نگاه نداشتی.»
\par 15 و سموئیل برخاسته، از جلجال به جبعه بنیامین آمد.
\par 16 و شاول و پسرش یوناتان و قومی که با ایشان حاضر بودند در جبعه بنیامین ماندند، و فلسطینیان در مخماس اردو زدند. 
\par 17 و تاراج کنندگان از اردوی فلسطینیان در سه فرقه بیرون آمدند که یک فرقه از ایشان به راه عفره به زمین شوعال توجه نمودند.
\par 18 و فرقه دیگر به راه بیت حورون میل کردند. و فرقه سوم به راه حدی که مشرف بر دره صبوعیم به‌جانب بیابان است، توجه نمودند.
\par 19 و در تمام زمین اسرائیل آهنگری یافت نمی شد، زیرا که فلسطینیان می‌گفتند: «مباداعبرانیان برای خود شمشیر یا نیزه بسازند.»
\par 20 وجمیع اسرائیلیان نزد فلسطینیان فرود می‌آمدند تاهر کس بیل و گاوآهن و تبر و داس خود را تیزکند.
\par 21 اما به جهت بیل و گاوآهن و چنگال سه دندانه و تبر و برای تیز کردن آهن گاوران سوهان داشتند.
\par 22 و در روز جنگ، شمشیر و نیزه دردست تمامی قومی که با شاول و یوناتان بودندیافت نشد، اما نزد شاول و پسرش یوناتان بود.و قراول فلسطینیان به معبر مخماس بیرون آمدند.
\par 23 و قراول فلسطینیان به معبر مخماس بیرون آمدند.
 
\chapter{14}

\par 1 و روزی واقع شد که یوناتان پسر شاول به جوان سلاح دار خود گفت: «بیا تا به قراول فلسطینیان که به آن طرفند بگذریم.» اماپدر خود را خبر نداد.
\par 2 و شاول در کناره جبعه زیر درخت اناری که در مغرون است، ساکن بود وقومی که همراهش بودند تخمین ششصد نفربودند.
\par 3 و اخیا ابن اخیطوب برادر ایخابودبن فینحاس بن عیلی، کاهن خداوند، در شیلوه باایفود ملبس شده بود، و قوم از رفتن یوناتان خبرنداشتند.
\par 4 و در میان معبرهایی که یوناتان می خواست از آنها نزد قراول فلسطینیان بگذرد، یک صخره تیز به این طرف و یک صخره تیز به آن طرف بود، که اسم یکی بوصیص و اسم دیگری سنه بود.
\par 5 و یکی از این صخره‌ها به طرف شمال در برابر مخماس ایستاده بود، و دیگری به طرف جنوب در برابر جبعه.
\par 6 و یوناتان به جوان سلاحدار خود گفت: «بیانزد قراول این نامختونان بگذریم شاید خداوندبرای ما عمل کند زیرا که خداوند را از رهانیدن باکثیر یا با قلیل مانعی نیست.»
\par 7 و سلاحدارش به وی گفت: «هر‌چه در دلت باشد، عمل نما. پیش برو؛ اینک من موافق رای تو با تو هستم.»
\par 8 ویوناتان گفت: «اینک ما به طرف این مردمان گذرنماییم و خود را به آنها ظاهر سازیم،
\par 9 اگر به ماچنین گویند: بایستید تا نزد شما برسیم، آنگاه درجای خود خواهیم ایستاد و نزد ایشان نخواهیم رفت.
\par 10 اما اگر چنین گویند که نزد ما برآیید، آنگاه خواهیم رفت زیرا خداوند ایشان را به‌دست ما تسلیم نموده است و به جهت ما، این علامت خواهد بود.»
\par 11 پس هر دوی ایشان خویشتن را به قراول فلسطینیان ظاهر ساختند و فلسطینیان گفتند: «اینک عبرانیان از حفره هایی که خود را در آنهاپنهان ساخته‌اند، بیرون می‌آیند.»
\par 12 و قراولان، یوناتان و سلاحدارش را خطاب کرده، گفتند: «نزد ما برآیید تا چیزی به شما نشان دهیم.» ویوناتان به سلاحدار خود گفت که «در عقب من بیازیرا خداوند ایشان را به‌دست اسرائیل تسلیم نموده است.»
\par 13 و یوناتان به‌دست و پای خود نزد ایشان بالارفت و سلاحدارش در عقب وی، و ایشان پیش روی یوناتان افتادند و سلاحدارش در عقب او می کشت.
\par 14 و این کشتار اول که یوناتان وسلاحدارش کردند به قدر بیست نفر بود در قریب نصف شیار یک جفت گاو زمین.
\par 15 و در اردو وصحرا و تمامی قوم تزلزل در‌افتاد و قراولان وتاراج کنندگان نیز لرزان شدند و زمین متزلزل شد، پس تزلزل عظیمی واقع گردید.
\par 16 و دیده بانان شاول در جبعه بنیامین نگاه کردند و اینک آن انبوه گداخته شده، به هر طرف پراکنده می‌شدند.
\par 17 و شاول به قومی که همراهش بودند، گفت: «الان تفحص کنید و ببینیداز ما که بیرون رفته است؟» پس تفحص کردند که اینک یوناتان و سلاحدارش حاضر نبودند.
\par 18 وشاول به اخیا گفت: «تابوت خدا را نزدیک بیاور.» زیرا تابوت خدا در آن وقت همراه بنی‌اسرائیل بود.
\par 19 و واقع شد چون شاول با کاهن سخن می‌گفت که اغتشاش در اردوی فلسطینیان زیاده وزیاده می‌شد، و شاول به کاهن گفت: «دست خودرا نگاهدار.»
\par 20 و شاول و تمامی قومی که با وی بودندجمع شده، به جنگ آمدند، و اینک شمشیر هرکس به ضد رفیقش بود و قتال بسیار عظیمی بود.
\par 21 و عبرانیانی که قبل از آن با فلسطینیان بودند وهمراه ایشان از اطراف به اردو آمده بودند، ایشان نیز نزد اسرائیلیانی که با شاول و یوناتان بودند، برگشتند.
\par 22 و تمامی مردان اسرائیل نیز که خودرا در کوهستان افرایم پنهان کرده بودند چون شنیدند که فلسطینیان منهزم شده‌اند، ایشان را درجنگ تعاقب نمودند.
\par 23 پس خداوند در آن روزاسرائیل را نجات داد و جنگ تا بیت آون رسید.
\par 24 و مردان اسرائیل آن روز در تنگی بودندزیرا که شاول قوم را قسم داده، گفته بود: «تا من ازدشمنان خود انتقام نکشیده باشم ملعون باد کسی که تا شام طعام بخورد.» و تمامی قوم طعام نچشیدند.
\par 25 و تمامی قوم به جنگلی رسیدند که در آنجا عسل بر روی زمین بود.
\par 26 و چون قوم به جنگل داخل شدند، اینک عسل می‌چکید امااحدی دست خود را به دهانش نبرد زیرا قوم ازقسم ترسیدند.
\par 27 لیکن یوناتان هنگامی که پدرش به قوم قسم می‌داد نشنیده بود، پس نوک عصایی را که در دست داشت دراز کرده، آن را به‌شان عسل فرو برد، و دست خود را به دهانش برده، چشمان او روشن گردید.
\par 28 و شخصی از قوم به او توجه نموده، گفت: «پدرت قوم را قسم سخت داده، گفت: ملعون بادکسی‌که امروز طعام خورد.» و قوم بی‌تاب شده بودند.
\par 29 و یوناتان گفت: «پدرم زمین را مضطرب ساخته است، الان ببینید که چشمانم چه قدرروشن شده است که اندکی از این عسل چشیده‌ام.
\par 30 و چه قدر زیاده اگر امروز قوم از غارت دشمنان خود که یافته‌اند بی‌ممانعت می‌خوردند، آیا قتال فلسطینیان بسیار زیاده نمی شد؟»
\par 31 و در آن روز فلسطینیان را از مخماس تاایلون منهزم ساختند و قوم بسیار بی‌تاب شدند.
\par 32 و قوم بر غنیمت حمله کرده، از گوسفندان وگاوان و گوساله‌ها گرفته، بر زمین کشتند و قوم آنهارا با خون خوردند.
\par 33 و شاول را خبر داده، گفتند: «اینک قوم به خداوند گناه ورزیده، با خون می‌خورند.» گفت: «شما خیانت ورزیده‌اید امروزسنگی بزرگ نزد من بغلطانید.»
\par 34 و شاول گفت: «خود را در میان قوم منتشر ساخته، به ایشان بگویید: هر کس گاو خود و هر کس گوسفند خود را نزد من بیاورد و در اینجا ذبح نموده، بخورید وبه خدا گناه نورزیده، با خون مخورید.» و تمامی قوم در آن شب هر کس گاوش را با خود آورده، در آنجا ذبح کردند.
\par 35 و شاول مذبحی برای خداوند بنا کرد و این مذبح اول بود که برای خداوند بنا نمود.
\par 36 و شاول گفت: «امشب در عقب فلسطینیان برویم و آنها را تا روشنایی صبح غارت کرده، ازایشان احدی را باقی نگذاریم.» ایشان گفتند: «هرچه در نظرت پسند آید بکن.» و کاهن گفت: «دراینجا به خدا تقرب بجوییم.»
\par 37 و شاول از خداسوال نمود که آیا از عقب فلسطینیان برویم و آیاایشان را به‌دست اسرائیل خواهی داد، اما در آن روز او را جواب نداد.
\par 38 آنگاه شاول گفت: «ای تمامی روسای قوم به اینجا نزدیک شوید و بدانیدو ببینید که امروز این گناه در چه چیز است.
\par 39 زیرا قسم به حیات خداوند رهاننده اسرائیل که اگر در پسرم یوناتان هم باشد، البته خواهدمرد.» لیکن از تمامی قوم احدی به او جواب نداد.
\par 40 پس به تمامی اسرائیل گفت: «شما به یک طرف باشید و من با پسر خود یوناتان به یک طرف باشیم.» و قوم به شاول گفتند: «هر‌چه در نظرت پسند آید، بکن.»
\par 41 و شاول به یهوه، خدای اسرائیل گفت: «قرعه‌ای راست بده.» پس یوناتان و شاول گرفته شدند و قوم رها گشتند.
\par 42 و شاول گفت: «در میان من و پسرم یوناتان قرعه بیندازید.» و یوناتان گرفته شد.
\par 43 و شاول به یوناتان گفت: «مرا خبر ده که چه کرده‌ای؟» و یوناتان به او خبر داده، گفت: «به نوک عصایی که در دست دارم اندکی عسل چشیدم واینک باید بمیرم؟»
\par 44 و شاول گفت: «خدا چنین بلکه زیاده از این بکند‌ای یوناتان! زیرا البته خواهی مرد.»
\par 45 اما قوم به شاول گفتند: «آیایوناتان که نجات عظیم را در اسرائیل کرده است، باید بمیرد؟ حاشا! قسم به حیات خداوند که مویی از سرش به زمین نخواهد افتاد زیرا که امروز با خدا عمل نموده است.» پس قوم یوناتان را خلاص نمودند که نمرد.
\par 46 و شاول از تعاقب فلسطینیان باز آمد و فلسطینیان به‌جای خودرفتند.
\par 47 و شاول عنان سلطنت اسرائیل را به‌دست گرفت و با جمیع دشمنان اطراف خود، یعنی باموآب و بنی عمون و ادوم و ملوک صوبه وفلسطینیان جنگ کرد و به هر طرف که توجه می‌نمود، غالب می‌شد.
\par 48 و به دلیری عمل می‌نمود و عمالیقیان را شکست داده، اسرائیل رااز دست تاراج کنندگان ایشان رهانید.
\par 49 و پسران شاول، یوناتان و یشوی و ملکیشوبودند. و اسمهای دخترانش این است: اسم نخست زاده‌اش میرب و اسم کوچک میکال.
\par 50 واسم زن شاول اخینوعام، دختر اخیمعاص، بود واسم سردار لشکرش ابنیر بن نیر، عموی شاول بود.
\par 51 و قیس پدر شاول بود و نیر پدر ابنیر و پسرابیئیل بود.و در تمامی روزهای شاول با فلسطینیان جنگ سخت بود و هر صاحب قوت و صاحب شجاعت که شاول می‌دید، او را نزد خودمی آورد.
\par 52 و در تمامی روزهای شاول با فلسطینیان جنگ سخت بود و هر صاحب قوت و صاحب شجاعت که شاول می‌دید، او را نزد خودمی آورد.
 
\chapter{15}

\par 1 و سموئیل به شاول گفت: «خداوند مرافرستاد که ترا مسح نمایم تا بر قوم اواسرائیل پادشاه شوی. پس الان آواز کلام خداوند را بشنو.
\par 2 یهوه صبایوت چنین می‌گوید: آنچه عمالیق به اسرائیل کرد، بخاطر داشته‌ام که چگونه هنگامی که از مصر برمی آمد، با او در راه مقاومت کرد.
\par 3 پس الان برو و عمالیق را شکست داده، جمیع مایملک ایشان را بالکل نابود ساز، وبر ایشان شفقت مفرما بلکه مرد و زن و طفل وشیرخواره و گاو و گوسفند و شتر و الاغ را بکش.»
\par 4 پس شاول قوم را طلبید و از ایشان دویست هزار پیاده و ده هزار مرد از یهودا در طلایم سان دید.
\par 5 و شاول به شهر عمالیق آمده، در وادی کمین گذاشت.
\par 6 و شاول به قینیان گفت: «بروید وبرگشته، از میان عمالقه دور شوید، مبادا شما را باایشان هلاک سازم و حال آنکه شما با همه بنی‌اسرائیل هنگامی که از مصر برآمدند، احسان نمودید.» پس قینیان از میان عمالقه دور شدند.
\par 7 وشاول عمالقه را از حویله تا شور که در برابر مصراست، شکست داد.
\par 8 و اجاج پادشاه عمالیق رازنده گرفت و تمامی خلق را به دم شمشیر، بالکل هلاک ساخت.
\par 9 و اما شاول و قوم اجاج را وبهترین گوسفندان و گاوان و پرواریها و بره‌ها و هرچیز خوب را دریغ نموده، نخواستند آنها راهلاک سازند. لیکن هر چیز خوار و بیقیمت رابالکل نابود ساختند.
\par 10 و کلام خداوند بر سموئیل نازل شده، گفت:
\par 11 «پشیمان شدم که شاول را پادشاه ساختم زیرا از پیروی من برگشته، کلام مرا بجا نیاورده است.» و سموئیل خشمناک شده، تمامی شب نزد خداوند فریاد برآورد.
\par 12 و بامدادان سموئیل برخاست تا شاول را ملاقات نماید وسموئیل را خبر داده، گفتند که «شاول به کرمل آمد و اینک به جهت خویشتن ستونی نصب نمودو دور زده، گذشت و در جلجال فرود آمده است.»
\par 13 و چون سموئیل نزد شاول رسید شاول به او گفت: «برکت خداوند بر تو باد! من فرمان خداوند را بجا آوردم.»
\par 14 سموئیل گفت: «پس این صدای گوسفندان در گوش من و بانگ گاوان که من می‌شنوم چیست؟»
\par 15 شاول گفت: «اینها رااز عمالقه آورده‌اند زیرا قوم بهترین گوسفندان وگاوان را دریغ داشتند تا برای یهوه خدایت قربانی نمایند، و بقیه را بالکل هلاک ساختیم.»
\par 16 سموئیل به شاول گفت: «تامل نما تا آنچه خداوند دیشب به من گفت به تو بگویم.» او وی راگفت: «بگو.»
\par 17 و سموئیل گفت: «هنگامی که تو در نظرخود کوچک بودی، آیا رئیس اسباط اسرائیل نشدی و آیا خداوند تو را مسح نکرد تا براسرائیل پادشاه شوی؟
\par 18 و خداوند تو را به راهی فرستاده، گفت: این عمالقه گناهکار را بالکل هلاک ساز و با ایشان جنگ کن تا نابود شوند. 
\par 19 پس چرا قول خداوند را نشنیدی بلکه برغنیمت هجوم آورده، آنچه را که در نظر خداوندبد است عمل نمودی؟»
\par 20 شاول به سموئیل گفت: «قول خداوند را استماع نمودم و به راهی که خداوند مرا فرستاد، رفتم و اجاج، پادشاه عمالقه را آوردم و عمالقه را بالکل هلاک ساختم.
\par 21 اما قوم از غنیمت، گوسفندان و گاوان، یعنی بهترین آنچه حرام شده بود، گرفتند تا برای یهوه خدایت در جلجال قربانی بگذرانند.»
\par 22 سموئیل گفت: «آیا خداوند به قربانی های سوختنی وذبایح خوشنود است یا به اطاعت فرمان خداوند؟ اینک اطاعت از قربانی‌ها و گوش گرفتن از پیه قوچها نیکوتر است.
\par 23 زیرا که تمرد مثل گناه جادوگری است وگردن کشی مثل بت‌پرستی و ترافیم است. چونکه کلام خداوند را ترک کردی او نیز تو را از سلطنت رد نمود.»
\par 24 و شاول به سموئیل گفت: «گناه کردم زیرااز فرمان خداوند و سخن تو تجاوز نمودم چونکه از قوم ترسیده، قول ایشان را شنیدم.
\par 25 پس حال تمنا اینکه گناه مرا عفو نمایی و با من برگردی تاخداوند را عبادت نمایم.»
\par 26 سموئیل به شاول گفت: «با تو برنمی گردم چونکه کلام خداوند راترک نموده‌ای. خداوند نیز تو را از پادشاه بودن براسرائیل رد نموده است.»
\par 27 و چون سموئیل برگشت تا روانه شود، اودامن جامه او را بگرفت که پاره شد.
\par 28 و سموئیل وی را گفت: «امروز خداوند سلطنت اسرائیل رااز تو پاره کرده، آن را به همسایه ات که از تو بهتراست، داده است.
\par 29 و نیز جلال اسرائیل دروغ نمی گوید، و تغییر به اراده خود نمی دهد زیرا اوانسان نیست که به اراده خود تغییر دهد.»
\par 30 گفت: «گناه کرده‌ام، حال تمنا اینکه مرا به حضورمشایخ قومم و به حضور اسرائیل محترم داری وهمراه من برگردی تا یهوه خدایت را عبادت نمایم.»
\par 31 پس سموئیل در عقب شاول برگشت، و شاول خداوند را عبادت نمود.
\par 32 و سموئیل گفت: «اجاج پادشاه عمالیق رانزد من بیاورید.» و اجاج به خرمی نزد او آمد واجاج گفت: «به درستی که تلخی موت گذشته است.»
\par 33 و سموئیل گفت: «چنانکه شمشیر توزنان را بی‌اولاد کرده است، همچنین مادر تو ازمیان زنان، بی‌اولاد خواهد شد.» و سموئیل اجاج را به حضور خداوند در جلجال پاره پاره کرد.
\par 34 و سموئیل به رامه رفت و شاول به خانه خود به جبعه شاول برآمد.و سموئیل برای دیدن شاول تا روز وفاتش دیگر نیامد. اماسموئیل برای شاول ماتم می‌گرفت، و خداوندپشیمان شده بود که شاول را بر اسرائیل پادشاه ساخته بود.
\par 35 و سموئیل برای دیدن شاول تا روز وفاتش دیگر نیامد. اماسموئیل برای شاول ماتم می‌گرفت، و خداوندپشیمان شده بود که شاول را بر اسرائیل پادشاه ساخته بود.
 
\chapter{16}

\par 1 و خداوند به سموئیل گفت: «تا به کی توبرای شاول ماتم می‌گیری چونکه من اورا از سلطنت نمودن بر اسرائیل رد نمودم. پس حقه خود را از روغن پر کرده، بیا تا تو را نزد یسای بیت لحمی بفرستم، زیرا که از پسرانش پادشاهی برای خود تعیین نموده‌ام.»
\par 2 سموئیل گفت: «چگونه بروم. اگر شاول بشنود مرا خواهدکشت.» خداوند گفت: «گوساله‌ای همراه خود ببرو بگو که به جهت گذرانیدن قربانی برای خداوندآمده‌ام.
\par 3 و یسا را به قربانی دعوت نما، و من تو رااعلام می‌نمایم که چه باید بکنی، و کسی را که به تو امر نمایم برای من مسح نما.»
\par 4 و سموئیل آنچه را که خداوند به او گفته بود بجا آورده، به بیت لحم آمد، و مشایخ شهر لرزان شده، به استقبال او آمدند، و گفتند: «آیا با سلامتی می‌آیی؟»
\par 5 گفت: «با سلامتی به جهت قربانی گذرانیدن برای خداوند آمده‌ام، پس خود را تقدیس نموده، همراه من به قربانی بیایید.» و اویسا و پسرانش را تقدیس نموده، ایشان را به قربانی دعوت نمود.
\par 6 و واقع شد که چون آمدند بر الیاب نظرانداخته، گفت: «یقین مسیح خداوند به حضوروی است.»
\par 7 اما خداوند به سموئیل گفت: «به چهره‌اش و بلندی قامتش نظر منما زیرا او را ردکرده‌ام، چونکه خداوند مثل انسان نمی نگرد، زیرا که انسان به ظاهر می‌نگرد و خداوند به دل می‌نگرد.»
\par 8 و یسا ابیناداب را خوانده، او را ازحضور سموئیل گذرانید، و او گفت: «خداوند این را نیز برنگزیده است.»
\par 9 و یسا شماه را گذرانید واو گفت: «خداوند این را نیز برنگزیده است.»
\par 10 ویسا هفت پسر خود را از حضور سموئیل گذرانیدو سموئیل به یسا گفت: «خداوند اینها رابرنگزیده است.»
\par 11 و سموئیل به یسا گفت: «آیا پسرانت تمام شدند.» گفت: «کوچکتر هنوز باقی است و اینک او گله را می‌چراند.» و سموئیل به یسا گفت: «بفرست و او را بیاور، زیرا که تا او به اینجا نیایدنخواهیم نشست.»
\par 12 پس فرستاده، او را آورد، واو سرخ رو و نیکوچشم و خوش منظر بود. وخداوند گفت: «برخاسته، او را مسح کن زیرا که همین است.»
\par 13 پس سموئیل حقه روغن راگرفته، او را در میان برادرانش مسح نمود، و از آن روز به بعد روح خداوند بر داود مستولی شد، وسموئیل برخاسته، به رامه رفت.
\par 14 و روح خداوند از شاول دور شد، و روح بداز جانب خداوند او را مضطرب می‌ساخت.
\par 15 وبندگان شاول وی را گفتند: «اینک روح بد از جانب خدا تو را مضطرب می‌سازد.
\par 16 پس آقای ما بندگان خود را که به حضورت هستند امرفرماید تا کسی را که بر بربط نواختن ماهر باشدبجویند، و چون روح بد از جانب خدا بر تو بیایدبه‌دست خود بنوازد، و تو را نیکو خواهد شد.»
\par 17 و شاول به بندگان خود گفت: «الان کسی را که به نواختن ماهر باشد برای من پیدا کرده، نزد من بیاورید.»
\par 18 و یکی از خادمانش در جواب وی گفت: «اینک پسر یسای بیت لحمی را دیدم که به نواختن ماهر و صاحب شجاعت و مرد جنگ آزموده و فصیح زبان و شخص نیکو صورت است و خداوند با وی می‌باشد.»
\par 19 پس شاول قاصدان نزد یسا فرستاده، گفت: «پسرت داود را که با گوسفندان است، نزد من بفرست.»
\par 20 آنگاه یسا یک بار الاغ از نان و یک مشگ شراب و یک بزغاله گرفته، به‌دست پسرخود داود نزد شاول فرستاد.
\par 21 و داود نزد شاول آمده، به حضور وی ایستاد و او وی را بسیاردوست داشت و سلاحدار او شد.
\par 22 و شاول نزدیسا فرستاده، گفت: «داود نزد من بماند زیرا که به نظرم پسند آمد.»و واقع می‌شد هنگامی که روح بد از جانب خدا بر شاول می‌آمد که داودبربط گرفته، به‌دست خود می‌نواخت، و شاول راراحت و صحت حاصل می‌شد و روح بد از اومی رفت.
\par 23 و واقع می‌شد هنگامی که روح بد از جانب خدا بر شاول می‌آمد که داودبربط گرفته، به‌دست خود می‌نواخت، و شاول راراحت و صحت حاصل می‌شد و روح بد از اومی رفت.
 
\chapter{17}

\par 1 و فلسطینیان لشکر خود را برای جنگ جمع نموده، در سوکوه که در یهودیه است، جمع شدند، و در میان سوکوه و عزیقه درافس دمیم اردو زدند.
\par 2 و شاول و مردان اسرائیل جمع شده، در دره ایلاه اردو زده، به مقابله فلسطینیان صف آرایی کردند.
\par 3 و فلسطینیان برکوه از یک طرف ایستادند، و اسرائیلیان بر کوه به طرف دیگر ایستادند، و دره در میان ایشان بود.
\par 4 و از اردوی فلسطینیان مرد مبارزی مسمی به جلیات که از شهر جت بود بیرون آمد، و قدش شش ذراع و یک وجب بود.
\par 5 و بر سر خود، خودبرنجینی داشت و به زره فلسی ملبس بود، و وزن زره‌اش پنج هزار مثقال برنج بود.
\par 6 و بر ساقهایش ساق بندهای برنجین و در میان کتفهایش مزراق برنجین بود.
\par 7 و چوب نیزه‌اش مثل نوردجولاهگان و سرنیزه‌اش ششصد مثقال آهن بود، و سپردارش پیش او می‌رفت.
\par 8 و او ایستاده، افواج اسرائیل را صدا زد و به ایشان گفت: «چرابیرون آمده، صف آرایی نمودید؟ آیا من فلسطینی نیستم و شما بندگان شاول؟ برای خودشخصی برگزینید تا نزد من درآید.
\par 9 اگر او بتواندبا من جنگ کرده، مرا بکشد، ما بندگان شماخواهیم شد، و اگر من بر او غالب آمده، او رابکشم شما بندگان ما شده، ما را بندگی خواهیدنمود.»
\par 10 و فلسطینی گفت: «من امروز فوجهای اسرائیل را به ننگ می‌آورم، شخصی به من بدهیدتا با هم جنگ نماییم.»
\par 11 و چون شاول و جمیع اسرائیلیان این سخنان فلسطینی را شنیدندهراسان شده، بسیار بترسیدند.
\par 12 و داود پسر آن مرد افراتی بیت لحم یهودابود که یسا نام داشت، و او را هشت پسر بود، و آن مرد در ایام شاول در میان مردمان پیر و سالخورده بود.
\par 13 و سه پسر بزرگ یسا روانه شده، در عقب شاول به جنگ رفتند و اسم سه پسرش که به جنگ رفته بودند: نخست زاده‌اش الیاب و دومش ابیناداب و سوم شماه بود.
\par 14 و داود کوچکتر بودو آن سه بزرگ در عقب شاول رفته بودند.
\par 15 و داود از نزد شاول آمد و رفت می‌کرد تا گوسفندان پدر خود را در بیت لحم بچراند.
\par 16 و آن فلسطینی صبح و شام می‌آمد و چهل روز خود راظاهر می‌ساخت.
\par 17 و یسا به پسر خود داود گفت: «الان به جهت برادرانت یک ایفه از این غله برشته و این ده قرص نان را بگیر و به اردو نزد برادرانت بشتاب.
\par 18 و این ده قطعه پنیر را برای سردار هزاره ایشان ببر و از سلامتی برادرانت بپرس و از ایشان نشانی‌ای بگیر.»
\par 19 و شاول و آنها و جمیع مردان اسرائیل دردره ایلاه بودند و با فلسطینیان جنگ می‌کردند.
\par 20 پس داود بامدادان برخاسته، گله را به‌دست چوپان واگذاشت و برداشته، چنانکه یسا او را امرفرموده بود برفت، و به سنگر اردو رسید وقتی که لشکر به میدان بیرون رفته، برای جنگ نعره می‌زدند.
\par 21 و اسرائیلیان و فلسطینیان لشکر به مقابل لشکر صف آرایی کردند.
\par 22 و داود اسبابی را که داشت به‌دست نگاهبان اسباب سپرد و به سوی لشکر دویده، آمد و سلامتی برادران خودرا بپرسید.
\par 23 و چون با ایشان گفتگو می‌کرد اینک آن مرد مبارز فلسطینی جتی که اسمش جلیات بود از لشکر فلسطینیان برآمده، مثل پیش سخن گفت و داود شنید.
\par 24 و جمیع مردان اسرائیل چون آن مرد رادیدند، از حضورش فرار کرده، بسیار ترسیدند.
\par 25 و مردان اسرائیل گفتند: «آیا این مرد را که برمی آید، دیدید؟ یقین برای به ننگ‌آوردن اسرائیل برمی آید و هر‌که او را بکشد، پادشاه اورا از مال فراوان دولتمند سازد، و دختر خود را به او دهد، و خانه پدرش را در اسرائیل آزاد خواهدساخت.»
\par 26 و داود کسانی را که نزد او ایستاده بودند خطاب کرده، گفت: «به شخصی که این فلسطینی را بکشد و این ننگ را از اسرائیل برداردچه خواهد شد؟ زیرا که این فلسطینی نامختون کیست که لشکرهای خدای حی را به ننگ‌آورد؟»
\par 27 و قوم او را به همین سخنان خطاب کرده، گفتند: «به شخصی که او را بکشد، چنین خواهد شد.»
\par 28 و چون با مردمان سخن می‌گفتند برادربزرگش الیاب شنید و خشم الیاب بر داودافروخته شده، گفت: «برای چه اینجا آمدی و آن گله قلیل را در بیابان نزد که گذاشتی؟ من تکبر وشرارت دل تو را می‌دانم زیرا برای دیدن جنگ آمده‌ای.»
\par 29 داود گفت: «الان چه کردم آیا سببی نیست؟»
\par 30 پس از وی به طرف دیگری روگردانیده، به همین طور گفت و مردمان او را مثل پیشتر جواب دادند.
\par 31 و چون سخنانی که داود گفت، مسموع شد، شاول را مخبر ساختند و او وی را طلبید.
\par 32 و داود به شاول گفت: «دل کسی به‌سبب اونیفتد. بنده ات می‌رود و با این فلسطینی جنگ می‌کند.»
\par 33 شاول به داود گفت: «تو نمی توانی به مقابل این فلسطینی بروی تا با وی جنگ نمایی زیرا که تو جوان هستی و او از جوانیش مردجنگی بوده است.»
\par 34 داود به شاول گفت: «بنده ات گله پدر خود را می‌چراند که شیر وخرسی آمده، بره‌ای از گله ربودند.
\par 35 و من آن راتعاقب نموده، کشتم و از دهانش رهانیدم و چون به طرف من بلند شد، ریش او را گرفته، او را زدم وکشتم.
\par 36 بنده ات هم شیر و هم خرس را کشت، واین فلسطینی نامختون مثل یکی از آنها خواهدبود، چونکه لشکرهای خدای حی را به تنگ آورده است.
\par 37 و داود گفت: خداوند که مرا از چنگ شیر و از چنگ خرس رهانید، مرا از دست این فلسطینی خواهد رهانید.» و شاول به داودگفت: «برو و خداوند با تو باد.» 
\par 38 و شاول لباس خود را به داود پوشانید وخود برنجینی بر سرش نهاد و زره‌ای به اوپوشانید.
\par 39 و داود شمشیرش را بر لباس خودبست و می‌خواست که برود زیرا که آنها رانیازموده بود و داود به شاول گفت: «با اینهانمی توانم رفت چونکه نیازموده‌ام.» پس داود آنهارا از بر خود بیرون آورد.
\par 40 و چوب دستی خودرا به‌دست گرفته، پنج سنگ مالیده، از نهر سواکرد، و آنها را در کیسه شبانی که داشت، یعنی درانبان خود گذاشت و فلاخنش را به‌دست گرفته، به آن فلسطینی نزدیک شد.
\par 41 و آن فلسطینی همی آمد تا به داود نزدیک شد و مردی که سپرش را برمی داشت پیش رویش می‌آمد.
\par 42 و فلسطینی نظر افکنده، داودرا دید و او را حقیر شمرد زیرا جوانی خوشرو ونیکومنظر بود.
\par 43 و فلسطینی به داود گفت: «آیامن سگ هستم که با چوب دستی نزد من می‌آیی؟» و فلسطینی داود را به خدایان خود لعنت کرد.
\par 44 و فلسطینی به داود گفت: «نزد من بیا تا گوشت تو را به مرغان هوا و درندگان صحرا بدهم.»
\par 45 داود به فلسطینی گفت: «تو با شمشیر و نیزه و مزراق نزد من می‌آیی اما من به اسم یهوه صبایوت، خدای لشکرهای اسرائیل که او را به ننگ‌آورده‌ای نزد تو می‌آیم.
\par 46 و خداوند امروزتو را به‌دست من تسلیم خواهد کرد و تو را زده، سر تو را از تنت جدا خواهم کرد، و لاشه های لشکر فلسطینیان را امروز به مرغان هوا و درندگان زمین خواهم داد تا تمامی زمین بدانند که دراسرائیل خدایی هست.
\par 47 و تمامی این جماعت خواهند دانست که خداوند به شمشیر و نیزه خلاصی نمی دهد زیرا که جنگ از آن خداونداست و او شما را به‌دست ما خواهد داد.»
\par 48 و چون فلسطینی برخاسته، پیش آمد و به مقابله داود نزدیک شد، داود شتافته، به مقابله فلسطینی به سوی لشکر دوید.
\par 49 و داود دست خود را به کیسه‌اش برد و سنگی از آن گرفته، ازفلاخن انداخت و به پیشانی فلسطینی زد، و سنگ به پیشانی او فرو رفت که بر روی خود بر زمین افتاد.
\par 50 پس داود بر فلسطینی با فلاخن و سنگ غالب آمده، فلسطینی را زد و کشت و در دست داود شمشیری نبود.
\par 51 و داود دویده، بر آن فلسطینی‌ایستاد، و شمشیر او را گرفته، ازغلافش کشید و او را کشته، سرش را با آن از تنش جدا کرد، و چون فلسطینیان، مبارز خود را کشته دیدند، گریختند.
\par 52 و مردان اسرائیل و یهودابرخاستند و نعره زده، فلسطینیان را تا جت و تادروازه های عقرون تعاقب نمودند و مجروحان فلسطینیان به راه شعریم تا به جت و عقرون افتادند.
\par 53 و بنی‌اسرائیل از تعاقب نمودن فلسطینیان برگشتند و اردوی ایشان را غارت نمودند.
\par 54 و داود سر فلسطینی را گرفته، به اورشلیم آورد اما اسلحه او را در خیمه خودگذاشت.
\par 55 و چون شاول داود را دید که به مقابله فلسطینی بیرون می‌رود، به‌سردار لشکرش ابنیرگفت: «ای ابنیر، این جوان پسر کیست؟» ابنیر گفت: «ای پادشاه به‌جان تو قسم که نمی دانم.»
\par 56 پادشاه گفت: «بپرس که این جوان پسر کیست.»
\par 57 و چون داود از کشتن فلسطینی برگشت، ابنیراو را گرفته، به حضور شاول آورد، و سر آن فلسطینی در دستش بود.و شاول وی را گفت: «ای جوان تو پسر کیستی؟» داود گفت: «پسربنده ات، یسای بیت لحمی، هستم.»
\par 58 و شاول وی را گفت: «ای جوان تو پسر کیستی؟» داود گفت: «پسربنده ات، یسای بیت لحمی، هستم.»
 
\chapter{18}

\par 1 و واقع شد که چون از سخن‌گفتن باشاول فارغ شد، دل یوناتان بر دل داودچسبید، و یوناتان او را مثل جان خویش دوست داشت.
\par 2 و در آن روز شاول وی را گرفته، نگذاشت که به خانه پدرش برگردد.
\par 3 و یوناتان باداود عهد بست چونکه او را مثل جان خوددوست داشته بود.
\par 4 و یوناتان ردایی را که دربرش بود، بیرون کرده، آن را به داود داد و رخت خود حتی شمشیر و کمان و کمربند خویش رانیز.
\par 5 و داود به هر جایی که شاول او را می‌فرستادبیرون می‌رفت، و عاقلانه حرکت می‌کرد و شاول او را بر مردان جنگی خود گماشت، و به نظرتمامی قوم و به نظر خادمان شاول نیز مقبول افتاد.
\par 6 و واقع شد هنگامی که داود از کشتن فلسطینی برمی گشت چون ایشان می‌آمدند که زنان از جمیع شهرهای اسرائیل با دفها و شادی وبا آلات موسیقی سرود و رقص‌کنان به استقبال شاول پادشاه بیرون آمدند.
\par 7 و زنان لهو و لعب کرده، به یکدیگر می‌سراییدند و می‌گفتند: «شاول هزاران خود را و داود ده هزاران خود را کشته است.»
\par 8 و شاول بسیار غضبناک شد، و این سخن در نظرش ناپسند آمده، گفت: «به داود ده هزاران دادند و به من هزاران دادند، پس غیر از سلطنت برایش چه باقی است.»
\par 9 و از آن روز به بعد شاول بر داود به چشم بد می‌نگریست.
\par 10 و در فردای آن روز روح بد از جانب خدا برشاول آمده، در میان خانه شوریده احوال گردید. و داود مثل هر روز به‌دست خود می‌نواخت ومزراقی در دست شاول بود.
\par 11 و شاول مزراق راانداخته، گفت: داود را تا به دیوار خواهم زد، اماداود دو مرتبه از حضورش خویشتن را به کنارکشید.
\par 12 و شاول از داود می‌ترسید زیرا خداوند با اوبود و از شاول دور شده.
\par 13 پس شاول وی را ازنزد خود دور کرد و او را سردار هزاره خود نصب نمود، و به حضور قوم خروج و دخول می‌کرد.
\par 14 و داود در همه رفتار خود عاقلانه حرکت می‌نمود، و خداوند با وی می‌بود.
\par 15 و چون شاول دید که او بسیار عاقلانه حرکت می‌کند به‌سبب او هراسان می‌بود.
\par 16 اما تمامی اسرائیل ویهودا داود را دوست می‌داشتند، زیرا که به حضور ایشان خروج و دخول می‌کرد.
\par 17 و شاول به داود گفت: «اینک دختر بزرگ خود میرب را به تو به زنی می‌دهم. فقط برایم شجاع باش و در جنگهای خداوند بکوش، زیراشاول می‌گفت: «دست من بر او دراز نشود بلکه دست فلسطینیان.»
\par 18 و داود به شاول گفت: من کیستم و جان من و خاندان پدرم در اسرائیل چیست تا داماد پادشاه بشوم.»
\par 19 و در وقتی که میرب دختر شاول می‌بایست به داود داده شود اوبه عدریئیل محولاتی به زنی داده شد.
\par 20 و میکال، دختر شاول، داود را دوست می‌داشت، و چون شاول را خبر دادند این امر وی را پسند آمد.
\par 21 و شاول گفت: «او را به وی می‌دهم تا برایش دام شود و دست فلسطینیان براو دراز شود.» پس شاول به داود بار دوم گفت: «امروز داماد من خواهی شد.»
\par 22 و شاول خادمان خود را فرمود که در خفا با داود متکلم شده، بگویید: «اینک پادشاه از تو راضی است وخادمانش تو را دوست می‌دارند؛ پس الان دامادپادشاه بشو.»
\par 23 پس خادمان شاول این سخنان را به سمع داود رسانیدند و داود گفت: «آیا در نظر شما دامادپادشاه شدن آسان است؟ و حال آنکه من مردمسکین و حقیرم.»
\par 24 و خادمان شاول او را خبرداده، گفتند که داود به این طور سخن گفته است.
\par 25 و شاول گفت: «به داود چنین بگویید که پادشاه مهر نمی خواهد جز صد قلفه فلسطینیان تا ازدشمنان پادشاه انتقام کشیده شود.» و شاول فکرکرد که داود را به‌دست فلسطینیان به قتل رساند.
\par 26 پس خادمانش داود را از این امر خبر دادندو این سخن به نظر داود پسند آمد که داماد پادشاه بشود، و روزهای معین هنوز تمام نشده بود.
\par 27 پس داود برخاسته، با مردان خود رفت ودویست نفر از فلسطینیان را کشته، داود قلفه های ایشان را آورد و آنها را تمام نزد پادشاه گذاشتند، تا داماد پادشاه بشود. و شاول دختر خود میکال رابه وی به زنی داد.
\par 28 و شاول دید و فهمید که خداوند با داود است. و میکال دختر شاول او رادوست می‌داشت.
\par 29 و شاول از داود باز بیشترترسید، و شاول همه اوقات دشمن داود بود.و بعد از آن سرداران فلسطینیان بیرون آمدند، و هر دفعه که بیرون می‌آمدند داود از جمیع خادمان شاول زیاده عاقلانه حرکت می‌کرد، و از این جهت اسمش بسیار شهرت یافت.
\par 30 و بعد از آن سرداران فلسطینیان بیرون آمدند، و هر دفعه که بیرون می‌آمدند داود از جمیع خادمان شاول زیاده عاقلانه حرکت می‌کرد، و از این جهت اسمش بسیار شهرت یافت.
 
\chapter{19}

\par 1 و شاول به پسر خود یوناتان و به جمیع خادمان خویش فرمود تا داود رابکشند.
\par 2 اما یوناتان پسر شاول به داود بسیار میل داشت، و یوناتان داود را خبر داده، گفت: «پدرم شاول قصد قتل تو دارد، پس الان تا بامدادان خویشتن را نگاهدار و در جایی مخفی مانده، خود را پنهان کن.
\par 3 و من بیرون آمده، به پهلوی پدرم در صحرایی که تو در آن می‌باشی خواهم ایستاد، و درباره تو با پدرم گفتگو خواهم کرد واگر چیزی ببینم تو را اطلاع خواهم داد.»
\par 4 و یوناتان درباره داود نزد پدر خود شاول به نیکویی سخن رانده، وی را گفت: «پادشاه بر بنده خود داود گناه نکند زیرا که او به تو گناه نکرده است بلکه اعمال وی برای تو بسیار نیکو بوده است.
\par 5 و جان خویش را به‌دست خود نهاده، آن فلسطینی را کشت و خداوند نجات عظیمی به جهت تمامی اسرائیل نمود و تو آن را دیده، شادمان شدی؛ پس چرا به خون بی‌تقصیری گناه کرده، داود را بی‌سبب بکشی.»
\par 6 و شاول به سخن یوناتان گوش گرفت، و شاول قسم خورد که به حیات خداوند او کشته نخواهد شد.
\par 7 آنگاه یوناتان داود را خواند و یوناتان او را از همه این سخنان خبر داد و یوناتان داود را نزد شاول آورده، او مثل ایام سابق در حضور وی می‌بود.
\par 8 و باز جنگ واقع شده، داود بیرون رفت و بافلسطینیان جنگ کرده، ایشان را به کشتار عظیمی شکست داد و از حضور وی فرار کردند.
\par 9 و روح بد از جانب خداوند بر شاول آمد و او در خانه خود نشسته، مزراق خویش را در دست داشت وداود به‌دست خود می‌نواخت.
\par 10 و شاول خواست که داود را با مزراق خود تا به دیوار بزند، اما او از حضور شاول بگریخت و مزراق را به دیوار زد و داود فرار کرده، آن شب نجات یافت.
\par 11 و شاول قاصدان به خانه داود فرستاد تا آن را نگاهبانی نمایند و در صبح او را بکشند. امامیکال، زن داود، او را خبر داده، گفت: «اگر امشب جان خود را خلاص نکنی، فردا کشته خواهی شد.»
\par 12 پس میکال داود را از پنجره فرو هشته، اوروانه شد و فرار کرده، نجات یافت.
\par 13 اما میکال ترافیم را گرفته، آن را در بستر نهاد و بالینی از پشم بز زیر سرش نهاده، آن را با رخت پوشانید.
\par 14 وچون شاول قاصدان فرستاده تا داود را بگیرند، گفت بیمار است.
\par 15 پس شاول قاصدان را فرستادتا داود را ببینند و گفت: «او را بر بسترش نزد من بیاورید تا او را بکشم.»
\par 16 و چون قاصدان داخل شدند، اینک ترافیم در بستر و بالین پشم بز زیرسرش بود.
\par 17 و شاول به میکال گفت: «برای چه مرا چنین فریب دادی و دشمنم را رها کردی تانجات یابد.» و میکال شاول را جواب داد که او به من گفت: «مرا رها کن؛ برای چه تو را بکشم؟»
\par 18 و داود فرار کرده، رهایی یافت و نزدسموئیل به رامه آمده، از هر‌آنچه شاول با وی کرده بود، او را مخبر ساخت، و او و سموئیل رفته، در نایوت ساکن شدند.
\par 19 پس شاول را خبرداده، گفتند: «اینک داود در نایوت رامه است.»
\par 20 و شاول قاصدان برای گرفتن داود فرستاد، وچون جماعت انبیا را دیدند که نبوت می‌کنند و سموئیل را که به پیشوایی ایشان ایستاده است، روح خدا بر قاصدان شاول آمده، ایشان نیز نبوت کردند.
\par 21 و چون شاول را خبر دادند قاصدان دیگر فرستاده، ایشان نیز نبوت کردند، و شاول بازقاصدان سوم فرستاده، ایشان نیز نبوت کردند.
\par 22 پس خود او نیز به رامه رفت و چون به چاه بزرگ که نزد سیخوه است رسید، سوال کرده، گفت: «سموئیل و داود کجا می‌باشند؟» و کسی گفت: «اینک در نایوت رامه هستند.»
\par 23 و به آنجابه نایوت رامه روانه شد و روح خدا بر او نیز آمد ودر حینی که می‌رفت نبوت می‌کرد تا به نایوت رامه رسید.و او نیز جامه خود را کنده، به حضور سموئیل نبوت می‌کرد و تمامی آن روز وتمامی آن شب برهنه افتاد، بنابراین گفتند: «آیاشاول نیز از‌جمله انبیاست؟»
\par 24 و او نیز جامه خود را کنده، به حضور سموئیل نبوت می‌کرد و تمامی آن روز وتمامی آن شب برهنه افتاد، بنابراین گفتند: «آیاشاول نیز از‌جمله انبیاست؟» 
 
\chapter{20}

\par 1 و داود از نایوت رامه فرار کرده، آمد وبه حضور یوناتان گفت: «چه کرده‌ام وعصیانم چیست و در نظر پدرت چه گناهی کرده‌ام که قصد جان من دارد؟»
\par 2 او وی را گفت: «حاشا! تو نخواهی مرد. اینک پدر من امری بزرگ وکوچک نخواهد کرد جز آنکه مرا اطلاع خواهدداد. پس چگونه پدرم این امر را از من مخفی بدارد؟ چنین نیست.»
\par 3 و داود نیز قسم خورده، گفت: «پدرت نیکومی داند که در نظر تو التفات یافته‌ام، و می‌گویدمبادا یوناتان این را بداند و غمگین شود، و لکن به حیات خداوند و به حیات تو که در میان من وموت یک قدم بیش نیست.»
\par 4 یوناتان به داود گفت: «هر‌چه دلت بخواهد آن را برای تو خواهم نمود.»
\par 5 داود به یوناتان گفت: «اینک فردا اول ماه است و من می‌باید با پادشاه به غذا بنشینم، پس مرا رخصت بده که تا شام سوم، خود را در صحراپنهان کنم.
\par 6 اگر پدرت مرا مفقود بیند بگو داود ازمن بسیار التماس نمود که به شهر خود به بیت لحم بشتابد، زیرا که تمامی قبیله او را آنجا قربانی سالیانه است.
\par 7 اگر گوید که خوب، آنگاه بنده ات را سلامتی خواهد بود، و اما اگر بسیار غضبناک شود بدانکه او به بدی جازم شده است.
\par 8 پس بابنده خود احسان نما چونکه بنده خویش را باخودت به عهد خداوند در‌آوردی و اگر عصیان در من باشد، خودت مرا بکش زیرا برای چه مرانزد پدرت ببری.»
\par 9 یوناتان گفت: «حاشا از تو! زیرا اگرمی دانستم بدی از جانب پدرم جزم شده است که بر تو بیاید، آیا تو را از آن اطلاع نمی دادم؟»
\par 10 داود به یوناتان گفت: «اگر پدرت تو را به درشتی جواب دهد کیست که مرا مخبر سازد؟»
\par 11 یوناتان به داود گفت: «بیا تا به صحرا برویم.» وهر دوی ایشان به صحرا رفتند.
\par 12 و یوناتان به داود گفت: «ای یهوه، خدای اسرائیل، چون فردا یا روز سوم پدر خود را مثل این وقت آزمودم و اینک اگر برای داود خیر باشد، اگر من نزد او نفرستم و وی را اطلاع ندهم،
\par 13 خداوند به یوناتان مثل این بلکه زیاده از این عمل نماید، و اما اگر پدرم ضرر تو را صواب بیند، پس تو را اطلاع داده، رها خواهم نمود تا به سلامتی بروی و خداوند همراه تو باشد چنانکه همراه پدر من بود.
\par 14 و نه‌تنها مادام حیاتم، لطف خداوند را با من بجا آوری تا نمیرم،
\par 15 بلکه لطف خود را از خاندانم تا به ابد قطع ننمایی، هم دروقتی که خداوند دشمنان داود را جمیع از روی زمین منقطع ساخته باشد.»
\par 16 پس یوناتان باخاندان داود عهد بست و گفت خداوند این را ازدشمنان داود مطالبه نماید.
\par 17 و یوناتان بار دیگربه‌سبب محبتی که با او داشت داود را قسم دادزیرا که او را دوست می‌داشت چنانکه جان خودرا دوست می‌داشت.
\par 18 و یوناتان او را گفت: «فردا اول ماه است وچونکه جای تو خالی می‌باشد، تو را مفقودخواهند یافت.
\par 19 و در روز سوم به زودی فرودشده، به‌جایی که خود را در آن در روز شغل پنهان کردی بیا و در جانب سنگ آزل بنشین.
\par 20 ومن سه تیر به طرف آن خواهم‌انداخت که گویا به هدف می‌اندازم.
\par 21 و اینک خادم خود رافرستاده، خواهم گفت برو و تیرها را پیدا کن و اگربه خادم گویم: اینک تیرها از این طرف تو است. آنها را بگیر. آنگاه بیا زیرا که برای تو سلامتی است و به حیات خداوند تو را هیچ ضرری نخواهد بود.
\par 22 اما اگر به خادم چنین بگویم که: اینک تیرها از آن طرف توست، آنگاه برو زیراخداوند تو را رها کرده است.
\par 23 و اما آن کاری که من و تو درباره آن گفتگو کردیم اینک خداوند درمیان من و تو تا به ابد خواهد بود.»
\par 24 پس داود خود را در صحرا پنهان کرد و چون اول ماه رسید، پادشاه برای غذا خوردن نشست.
\par 25 و پادشاه در جای خود بر‌حسب عادتش برمسند، نزد دیوار نشسته، و یوناتان ایستاده بود و ابنیر به پهلوی شاول نشسته، و جای داود خالی بود.
\par 26 و شاول در آن روز هیچ نگفت زیرا گمان می‌برد: «چیزی بر او واقع شده، طاهر نیست. البته طاهر نیست!»
\par 27 و در فردای اول ماه که روز دوم بود، جای داود نیز خالی بود. پس شاول به پسرخود یوناتان گفت: «چرا پسر یسا، هم دیروز و هم امروز به غذا نیامد؟»
\par 28 یوناتان در جواب شاول گفت: «داود از من بسیار التماس نمود تا به بیت لحم برود.
\par 29 و گفت: تمنا اینکه مرا رخصت بدهی زیرا خاندان ما را در شهر قربانی است وبرادرم مرا امر فرموده است، پس اگر الان در نظرتو التفات یافتم، مرخص بشوم تا برادران خودراببینم. از این جهت به سفره پادشاه نیامده است.»
\par 30 آنگاه خشم شاول بر یوناتان افروخته شده، او را گفت: «ای پسر زن کردنکش فتنه انگیز، آیانمی دانم که تو پسر یسا را به جهت افتضاح خود وافتضاح عورت مادرت اختیار کرده‌ای؟
\par 31 زیرامادامی که پسر یسا بر روی زمین زنده باشد تو وسلطنت تو پایدار نخواهید ماند. پس الان بفرست و او را نزد من بیاور زیرا که البته خواهد مرد.»
\par 32 یوناتان پدر خود شاول را جواب داده، وی راگفت: «چرا بمیرد؟ چه کرده است؟»
\par 33 آنگاه شاول مزراق خود را به او انداخت تااو را بزند. پس یوناتان دانست که پدرش بر کشتن داود جازم است.
\par 34 و یوناتان به شدت خشم، ازسفره برخاست و در روز دوم ماه، طعام نخوردچونکه برای داود غمگین بود زیرا پدرش او راخجل ساخته بود.
\par 35 و بامدادان یوناتان در وقتی که با داود تعیین کرده بود به صحرا بیرون رفت و یک پسر کوچک همراهش بود.
\par 36 و به خادم خود گفت: «بدو وتیرها را که می‌اندازم پیدا کن.» و چون پسرمی دوید تیر را چنان انداخت که از او رد شد.
\par 37 وچون پسر به مکان تیری که یوناتان انداخته بود، می‌رفت، یوناتان در عقب پسر آواز داده، گفت که: «آیا تیر به آن طرف تو نیست؟»
\par 38 و یوناتان درعقب پسر آواز داد که بشتاب و تعجیل کن ودرنگ منما. پس خادم یوناتان تیرها را برداشته، نزد آقای خود برگشت.
\par 39 و پسر چیزی نفهمید. اما یوناتان و داود این امر را می‌دانستند.
\par 40 ویوناتان اسلحه خود را به خادم خود داده، وی راگفت: «برو و آن را به شهر ببر.»
\par 41 و چون پسر رفته بود، داود از جانب جنوبی برخاست و بر روی خود بر زمین افتاده، سه مرتبه سجده کرد و یکدیگر را بوسیده، با هم گریه کردند تا داود از حد گذرانید.و یوناتان به داودگفت: «به سلامتی برو چونکه ما هر دو به نام خداوند قسم خورده، گفتیم که خداوند در میان من و تو و در میان ذریه من و ذریه تو تا به ابد باشد. پس برخاسته، برفت و یوناتان به شهر برگشت.
\par 42 و یوناتان به داودگفت: «به سلامتی برو چونکه ما هر دو به نام خداوند قسم خورده، گفتیم که خداوند در میان من و تو و در میان ذریه من و ذریه تو تا به ابد باشد. پس برخاسته، برفت و یوناتان به شهر برگشت.
 
\chapter{21}

\par 1 و داود به نوب نزد اخیملک کاهن رفت، و اخیملک لرزان شده، به استقبال داودآمده، گفت: «چرا تنها آمدی و کسی با تونیست؟»
\par 2 داود به اخیملک کاهن گفت: «پادشاه مرا به‌کاری مامور فرمود و مرا گفت: از این کاری که تو را می‌فرستم و از آنچه به تو امر فرمودم کسی اطلاع نیابد، و خادمان را به فلان و فلان جا تعیین نمودم.
\par 3 پس الان چه در دست داری، پنج قرص نان یا هر‌چه حاضر است به من بده.»
\par 4 کاهن در جواب داود گفت: «هیچ نان عام دردست من نیست، لیکن نان مقدس هست اگرخصوص خادمان، خویشتن را از زنان بازداشته باشند.»
\par 5 داود در جواب کاهن گفت: «به درستی که در این سه روز زنان از ما دور بوده‌اند و چون بیرون آمدم ظروف جوانان مقدس بود، و آن بطوری عام است خصوص چونکه امروز دیگری در ظرف مقدس شده است.»
\par 6 پس کاهن، نان مقدس را به او داد زیرا که در آنجا نانی نبود غیر ازنان تقدمه که از حضور خداوند برداشته شده بود، تا در روز برداشتنش نان گرم بگذارند.
\par 7 و در آن روز یکی از خادمان شاول که مسمی به دوآغ ادومی بود، به حضور خداوند اعتکاف داشت، و بزرگترین شبانان شاول بود.
\par 8 و داود به اخیملک گفت: «آیا اینجا دردستت نیزه یا شمشیر نیست زیرا که شمشیر وسلاح خویش را با خود نیاورده‌ام چونکه کارپادشاه به تعجیل بود.»
\par 9 کاهن گفت: «اینک شمشیر جلیات فلسطینی که در دره ایلاه کشتی، در پشت ایفود به‌جامه ملفوف است. اگرمی خواهی آن را بگیری بگیر، زیرا غیر از آن دراینجا نیست.» داود گفت: «مثل آن، دیگری نیست. آن را به من بده.»
\par 10 پس داود آن روز برخاسته، از حضورشاول فرار کرده، نزد اخیش، ملک جت آمد.
\par 11 و خادمان اخیش او را گفتند: «آیا این داود، پادشاه زمین نیست؟ و آیا در باره او رقص‌کنان سرودخوانده، نگفتند که شاول هزاران خود را و داود ده هزاران خود را کشت؟»
\par 12 و داود این سخنان رادر دل خود جا داده، از اخیش، ملک جت بسیاربترسید.
\par 13 و در نظر ایشان رفتار خود راتغییرداده، به حضور ایشان خویشتن را دیوانه نمود، وبر لنگه های در خط می‌کشید و آب دهنش را برریش خود می‌ریخت.
\par 14 و اخیش به خادمان خود گفت: «اینک این شخص را می‌بینید که دیوانه است، او را چرا نزد من آوردید؟آیامحتاج به دیوانگان هستم که این شخص راآوردید تا نزد من دیوانگی کند؟ و آیا این شخص داخل خانه من بشود؟»
\par 15 آیامحتاج به دیوانگان هستم که این شخص راآوردید تا نزد من دیوانگی کند؟ و آیا این شخص داخل خانه من بشود؟»
 
\chapter{22}

\par 1 و داود از آنجا رفته، به مغاره عدلام فرارکرد و چون برادرانش و تمامی خاندان پدرش شنیدند، آنجا نزد او فرود آمدند.
\par 2 و هرکه در تنگی بود و هر قرض دار و هر‌که تلخی جان داشت، نزد او جمع آمدند، و بر ایشان سردار شدو تخمین چهار صد نفر با او بودند.
\par 3 و داود از آنجا به مصفه موآب رفته، به پادشاه موآب گفت: «تمنا اینکه پدرم و مادرم نزد شمابیایند تا بدانم خدا برای من چه خواهد کرد.»
\par 4 پس ایشان را نزد پادشاه موآب برد و تمامی روزهایی که داود در آن ملاذ بود، نزد او ساکن بودند.
\par 5 و جاد نبی به داود گفت که «در این ملاذدیگر توقف منما بلکه روانه شده، به زمین یهودا برو.» پس داود رفت و به جنگل حارث درآمد.
\par 6 و شاول شنید که داود و مردمانی که با وی بودند پیدا شده‌اند، و شاول در جبعه، زیر درخت بلوط در رامه نشسته بود، و نیزه‌اش در دستش، وجمیع خادمانش در اطراف او ایستاده بودند.
\par 7 وشاول به خادمانی که در اطرافش ایستاده بودند، گفت: «حال‌ای بنیامینیان بشنوید! آیا پسر یسا به جمیع شما کشتزارها و تاکستانها خواهد داد و آیاهمگی شما را سردار هزاره‌ها و سردار صده هاخواهد ساخت؟
\par 8 که جمیع شما بر من فتنه انگیزشده، کسی مرا اطلاع ندهد که پسر من با پسر یساعهد بسته است و از شما کسی برای من غمگین نمی شود تا مرا خبر دهد که پسر من بنده مرابرانگیخته است تا در کمین بنشیند چنانکه امروزهست.»
\par 9 و دوآغ ادومی که با خادمان شاول ایستاده بود در جواب گفت: «پسر یسا را دیدم که به نوب نزد اخیملک بن اخیتوب درآمد.
\par 10 و او ازبرای وی از خداوند سوال نمود و توشه‌ای به اوداد و شمشیر جلیات فلسطینی را نیز به او داد.»
\par 11 پس پادشاه فرستاده، اخیملک بن اخیتوب کاهن و جمیع کاهنان خاندان پدرش را که در نوب بودند طلبید، و تمامی ایشان نزد پادشاه آمدند.
\par 12 و شاول گفت: «ای پسر اخیتوب بشنو.» اوگفت: «لبیک‌ای آقایم!»
\par 13 شاول به او گفت: «تو وپسر یسا چرا بر من فتنه انگیختید به اینکه به وی نان و شمشیر دادی و برای وی از خدا سوال نمودی تا به ضد من برخاسته، در کمین بنشیندچنانکه امروز شده است.»
\par 14 اخیملک در جواب پادشاه گفت: «کیست ازجمیع بندگانت که مثل داود امین باشد و او دامادپادشاه است و در مشورت شریک تو و در خانه تومکرم است.
\par 15 آیا امروز به سوال نمودن از خدابرای او شروع کردم، حاشا از من. پادشاه این کار رابه بنده خود و به جمیع خاندان پدرم اسناد ندهدزیرا که بنده ات از این چیزها کم یا زیاد ندانسته بود.»
\par 16 پادشاه گفت: «ای اخیملک تو و تمامی خاندان پدرت البته خواهید مرد.» 
\par 17 آنگاه پادشاه به شاطرانی که به حضورش ایستاده بودند، گفت: «برخاسته، کاهنان خداوندرا بکشید زیرا که دست ایشان نیز با داود است و بااینکه دانستند که او فرار می‌کند، مرا اطلاع ندادند.» اما خادمان پادشاه نخواستند که دست خود را دراز کرده، بر کاهنان خداوند هجوم آورند.
\par 18 پس پادشاه به دوآغ گفت: «تو برگرد وبر کاهنان حمله آور.» و دوآغ ادومی برخاسته، برکاهنان حمله آورد و هشتاد و پنج نفر را که ایفودکتان می‌پوشیدند در آن روز کشت.
\par 19 و نوب رانیز که شهر کاهنان است به دم شمشیر زد و مردان و زنان و اطفال و شیرخوارگان و گاوان و الاغان وگوسفندان را به دم شمشیر کشت.
\par 20 اما یکی از پسران اخیملک بن اخیتوب که ابیاتار نام داشت رهایی یافته، در عقب داود فرارکرد.
\par 21 و ابیاتار داود را مخبر ساخت که شاول کاهنان خداوند را کشت.
\par 22 داود به ابیاتار گفت: «روزی که دوآغ ادومی در آنجا بود، دانستم که اوشاول را البته مخبر‌خواهد ساخت، پس من باعث کشته شدن تمامی اهل خاندان پدرت شدم.نزد من بمان و مترس زیرا هر‌که قصد جان من دارد قصد جان تو نیز خواهد داشت. و لکن نزد من محفوظ خواهی بود.»
\par 23 نزد من بمان و مترس زیرا هر‌که قصد جان من دارد قصد جان تو نیز خواهد داشت. و لکن نزد من محفوظ خواهی بود.»
 
\chapter{23}

\par 1 و به داود خبر داده، گفتند: «اینک فلسطینیان با قعیله جنگ می‌کنند وخرمنها را غارت می‌نماید.»
\par 2 و داود از خداوندسوال کرده، گفت: «آیا بروم و این فلسطینیان راشکست دهم.» خداوند به داود گفت: «برو وفلسطینیان را شکست داده، قعیله را خلاص کن.»
\par 3 و مردمان داود وی را گفتند: «اینک اینجا دریهودا می‌ترسیم پس چند مرتبه زیاده اگر به مقابله لشکرهای فلسطینیان به قعیله برویم.»
\par 4 و داود بار دیگر از خداوند سوال نمود وخداوند او را جواب داده، گفت: «برخیز به قعیله برو زیرا که من فلسطینیان را به‌دست تو خواهم داد.»
\par 5 و داود با مردانش به قعیله رفتند و بافلسطینیان جنگ کرده، مواشی ایشان را بردند، وایشان را به کشتار عظیمی کشتند. پس داودساکنان قعیله را نجات داد.
\par 6 و هنگامی که ابیاتار بن اخیملک نزد داود به قعیله فرار کرد، ایفود را در دست خود آورد.
\par 7 وبه شاول خبر دادند که داود به قعیله آمده است وشاول گفت: «خدا او را به‌دست من سپرده است، زیرا به شهری که دروازه‌ها و پشت بندها داردداخل شده، محبوس گشته است.»
\par 8 و شاول جمیع قوم را برای جنگ طلبید تا به قعیله فرودشده، داود و مردانش را محاصره نماید.
\par 9 و چون داود دانست که شاول شرارت رابرای او اندیشیده است، به ابیاتار کاهن گفت: «ایفود را نزدیک بیاور،
\par 10 و داود گفت: «ای یهوه، خدای اسرائیل، بنده ات شنیده است که شاول عزیمت دارد که به قعیله بیاید تا به‌خاطر من شهررا خراب کند.
\par 11 آیا اهل قعیله مرا به‌دست اوتسلیم خواهند نمود؟ و آیا شاول چنانکه بنده ات شنیده است، خواهد آمد؟ ای یهوه، خدای اسرائیل، مسالت آنکه بنده خود را خبر دهی.» خداوند گفت که او خواهد آمد.
\par 12 داود گفت: «آیا اهل قعیله مرا و کسان مرا به‌دست شاول تسلیم خواهند نمود؟» خداوند گفت که «تسلیم خواهند نمود.»
\par 13 پس داود و مردانش که تخمین ششصد نفربودند، برخاسته، از قعیله بیرون رفتند و هر جایی که توانستند بروند، رفتند. و چون به شاول خبردادند که داود از قعیله فرار کرده است، از بیرون رفتن بازایستاد.
\par 14 و داود در بیابان در ملاذهانشست و در کوهی در بیابان زیف توقف نمود. وشاول همه روزه او را می‌طلبید لیکن خداوند او رابه‌دستش تسلیم ننمود.
\par 15 و داود دید که شاول به قصد جان او بیرون آمده است و داود در بیابان زیف در جنگل ساکن بود.
\par 16 و یوناتان، پسر شاول، به جنگل آمده، دست او را به خدا تقویت نمود.
\par 17 و او را گفت: «مترس زیرا که دست پدر من، شاول تو را نخواهدجست، و تو بر اسرائیل پادشاه خواهی شد، و من دومین تو خواهم بود و پدرم شاول نیز این رامی داند.»
\par 18 و هر دوی ایشان به حضور خداوندعهد بستند و داود به جنگل برگشت و یوناتان به خانه خود رفت.
\par 19 و زیفیان نزد شاول به جبعه آمده، گفتند: «آیا داود در ملاذهای جنگل در کوه حخیله که به طرف جنوب بیابان است، خود را نزد ما پنهان نکرده است؟
\par 20 پس‌ای پادشاه چنانکه دلت کمال آرزو برای آمدن دارد بیا و تکلیف ما این است که او را به‌دست پادشاه تسلیم نماییم.»
\par 21 شاول گفت: «شما از جانب خداوند مبارک باشید چونکه بر من دلسوزی نمودید.
\par 22 پس بروید و بیشتر تحقیق نموده، جایی را که آمد ورفت می‌کند ببینید و بفهمید، و دیگر اینکه کیست که او را در آنجا دیده است، زیرا به من گفته شد که بسیار با مکر رفتار می‌کند.
\par 23 پس ببینید و جمیع مکانهای مخفی را که خود را در آنها پنهان می‌کندبدانید و حقیقت حال را به من باز رسانید تا با شمابیایم و اگر در این زمین باشد او را از جمیع هزاره های یهودا پیدا خواهم کرد.»
\par 24 پس برخاسته، پیش روی شاول به زیف رفتند.
\par 25 و شاول و مردان اوبه تفحص او رفتند و چون داود را خبر دادند، اونزد صخره فرود آمده، در بیابان معون ساکن شد وشاول چون این را شنید، داود را در بیابان معون تعاقب نمود.
\par 26 و شاول به یک طرف کوه می‌رفت و داود و کسانش به طرف دیگر کوه، و داودمی شتافت که از حضور شاول بگریزد و شاول ومردانش داود و کسانش را احاطه نمودند تا ایشان را بگیرند.
\par 27 اما قاصدی نزد شاول آمده، گفت: «بشتاب و بیا زیرا که فلسطینیان به زمین حمله آورده‌اند.
\par 28 پس شاول از تعاقب نمودن داود برگشته، به مقابله فلسطینیان رفت، بنابراین آن مکان را صخره محلقوت نامیدند.و داود ازآنجا برآمده، در ملاذهای عین جدی ساکن شد.
\par 29 و داود ازآنجا برآمده، در ملاذهای عین جدی ساکن شد.
 
\chapter{24}

\par 1 و واقع شد بعد از برگشتن شاول از عقب فلسطینیان که او را خبر داده، گفتند: «اینک داود در بیابان عین جدی است.»
\par 2 و شاول سه هزار نفر برگزیده را از تمامی اسرائیل گرفته، برای جستجوی داود و کسانش بر صخره های بزهای کوهی رفت.
\par 3 و به‌سر راه به آغلهای گوسفندان که در آنجا مغاره‌ای بود، رسید. وشاول داخل آن شد تا پایهای خود را بپوشاند. وداود و کسان او در جانبهای مغاره نشسته بودند.
\par 4 و کسان داود وی را گفتند: «اینک روزی که خداوند به تو وعده داده است که همانا دشمن تورا به‌دستت تسلیم خواهم نمود تا هر‌چه درنظرت پسند آید به او عمل نمایی.» و داودبرخاسته، دامن ردای شاول را آهسته برید.
\par 5 وبعد از آن دل داود مضطرب شد از این جهت که دامن شاول را بریده بود.
\par 6 و به کسان خود گفت: «حاشا بر من از جانب خداوند که این امر را به آقای خود مسیح خداوند بکنم، و دست خود رابر او دراز نمایم چونکه او مسیح خداوند است.»
\par 7 پس داود کسان خود را به این سخنان توبیخ نموده، ایشان را نگذاشت که بر شاول برخیزند، وشاول از مغاره برخاسته، راه خود را پیش گرفت.
\par 8 و بعد از آن، داود برخاسته، از مغاره بیرون رفت و در عقب شاول صدا زده، گفت: «ای آقایم پادشاه.» و چون شاول به عقب خود نگریست داود رو به زمین خم شده، تعظیم کرد.
\par 9 و داود به شاول گفت: «چرا سخنان مردم را می‌شنوی که می‌گویند اینک داود قصد اذیت تو را دارد.
\par 10 اینک امروز چشمانت دیده است که چگونه خداوند تو را در مغاره امروز به‌دست من تسلیم نمود، و بعضی گفتند که تو را بکشم، اما چشمم برتو شفقت نموده، گفتم دست خود را بر آقای خویش دراز نکنم، زیرا که مسیح خداوند است.
\par 11 و‌ای پدرم ملاحظه کن و دامن ردای خود را دردست من ببین، زیرا از اینکه جامه تو را بریدم و تورا نکشتم بدان و ببین که بدی و خیانت در دست من نیست، و به تو گناه نکرده‌ام، اما تو جان مراشکار می‌کنی تا آن را گرفتار سازی.
\par 12 خداونددر میان من و تو حکم نماید، و خداوند انتقام مرااز تو بکشد اما دست من بر تو نخواهد شد.
\par 13 چنانکه مثل قدیمان می‌گوید که شرارت ازشریران صادر می‌شود اما دست من بر تو نخواهدشد.
\par 14 و در عقب کیست که پادشاه اسرائیل بیرون می‌آید و کیست که او را تعاقب می‌نمایی، در عقب سگ مرده‌ای بلکه در عقب یک کیک!
\par 15 پس خداوند داور باشد و میان من و تو حکم نماید و ملاحظه کرده دعوی مرا با تو فیصل کند ومرا از دست تو برهاند.»
\par 16 و چون داود از گفتن این سخنان به شاول فارغ شد، شاول گفت: «آیا این آواز توست‌ای پسر من داود.» و شاول آواز خود را بلند کرده، گریست.
\par 17 و به داود گفت: «تو از من نیکوترهستی زیرا که تو جزای نیکو به من رسانیدی و من جزای بد به تو رسانیدم.
\par 18 و تو امروز ظاهرکردی که چگونه به من احسان نمودی چونکه خداوند مرا به‌دست تو تسلیم کرده، و مرانکشتی.
\par 19 و اگر کسی دشمن خویش را بیابد، آیا او را به نیکویی رها نماید؟ پس خداوند تو را به نیکویی جزا دهد به‌سبب آنچه امروز به من کردی.
\par 20 و حال اینک می‌دانم که البته پادشاه خواهی شد و سلطنت اسرائیل در دست تو ثابت خواهد گردید.
\par 21 پس الان برای من قسم به خداوند بخور که بعد از من ذریه مرا منقطع نسازی، و اسم مرا از خاندان پدرم محو نکنی.»و داود برای شاول قسم خورد، و شاول به خانه خود رفت و داود و کسانش به مامن خویش آمدند.
\par 22 و داود برای شاول قسم خورد، و شاول به خانه خود رفت و داود و کسانش به مامن خویش آمدند.
 
\chapter{25}

\par 1 و سموئیل وفات نمود، و تمامی اسرائیل جمع شده، از برایش نوحه گری نمودند، و او را در خانه‌اش در رامه دفن نمودند و داود برخاسته، به بیابان فاران فرود شد.
\par 2 و در معون کسی بود که املاکش در کرمل بود و آن مرد بسیار بزرگ بود و سه هزار گوسفند وهزار بز داشت، و گوسفندان خود را در کرمل پشم می‌برید.
\par 3 و اسم آن شخص نابال بود و اسم زنش ابیجایل. و آن زن نیک فهم و خوش منظر بود. اماآن مرد سخت دل و بدرفتار و از خاندان کالیب بود.
\par 4 و داود در بیابان شنید که نابال گله خود راپشم می‌برد.
\par 5 پس داود ده خادم فرستاد و داود به خادمان خود گفت که «به کرمل برآیید و نزد نابال رفته، اززبان من سلامتی او را بپرسید.
\par 6 و چنین گویید: زنده باش و سلامتی بر تو باد و بر خاندان تو و برهرچه داری سلامتی باشد.
\par 7 و الان شنیده‌ام که پشم برندگان داری و به شبانان تو که در این اوقات نزد ما بودند، اذیت نرسانیدیم. همه روزهایی که در کرمل بودند چیزی از ایشان گم نشد.
\par 8 از خادمان خود بپرس و تو را خواهند گفت. پس خادمان در نظر تو التفات یابند زیرا که در روزسعادتمندی آمده‌ایم، تمنا اینکه آنچه دستت بیابد به بندگانت و پسرت داود بدهی.»
\par 9 پس خادمان داود آمدند و جمیع این سخنان را از زبان داود به نابال گفته، ساکت شدند.
\par 10 ونابال به خادمان داود جواب داده، گفت: «داودکیست و پسر یسا کیست؟ امروز بسا بندگان هریکی از آقای خویش می‌گریزند.
\par 11 آیا نان و آب خود را و گوشت را که برای پشم برندگان خودذبح نموده‌ام، بگیرم و به کسانی که نمی دانم از کجاهستند بدهم.»
\par 12 پس خادمان داود برگشته، مراجعت نمودند و آمده، داود را از جمیع این سخنان مخبر ساختند.
\par 13 و داود به مردان خودگفت: «هر یک از شما شمشیر خود را ببندد.» وهریک شمشیر خود را بستند، و داود نیز شمشیرخود را بست و تخمین چهارصد نفر از عقب داودرفتند، و دویست نفر نزد اسباب ماندند.
\par 14 و خادمی از خادمانش به ابیجایل، زن نابال، خبر داده، گفت: «اینک داود، قاصدان از بیابان فرستاد تا آقای مرا تحیت گویند و او ایشان رااهانت نمود.
\par 15 و آن مردمان احسان بسیار به مانمودند و همه روزهایی که در صحرا بودیم و باایشان معاشرت داشتیم اذیتی به ما نرسید وچیزی از ما گم نشد.
\par 16 و تمام روزهایی که باایشان گوسفندان را می‌چرانیدیم هم در شب وهم در روز برای ما مثل حصار بودند.
\par 17 پس الان بدان و ببین که چه باید بکنی زیرا که بدی برای آقای ما و تمامی خاندانش مهیاست، چونکه او به حدی پسر بلیعال است که احدی با وی سخن نتواند گفت.» 
\par 18 آنگاه ابیجایل تعجیل نموده، دویست گرده نان و دو مشگ شراب و پنج گوسفند مهیا شده، وپنج کیل خوشه برشته و صد قرص کشمش ودویست قرص انجیر گرفته، آنها را بر الاغهاگذاشت.
\par 19 و به خادمان خود گفت: «پیش من بروید و اینک من از عقب شما می‌آیم.» اما به شوهر خود نابال هیچ خبر نداد.
\par 20 و چون بر الاغ خود سوار شده، از سایه کوه به زیر می‌آمد، اینک داود و کسانش به مقابل او رسیدند و به ایشان برخورد.
\par 21 و داود گفته بود: «به تحقیق که تمامی مایملک این شخص را در بیابان عبث نگاه داشتم که از جمیع اموالش چیزی گم نشد و او بدی را به عوض نیکویی به من پاداش داده است.
\par 22 خدا به دشمنان داود چنین بلکه زیاده از این عمل نمایداگر از همه متعلقان او تا طلوع صبح ذکوری واگذارم.»
\par 23 و چون ابیجایل، داود را دید، تعجیل نموده، از الاغ پیاده شد و پیش داود به روی خودبه زمین افتاده، تعظیم نمود.
\par 24 و نزد پایهایش افتاده، گفت: «ای آقایم این تقصیر بر من باشد وکنیزت در گوش تو سخن بگوید، و سخنان کنیزخود را بشنو.
\par 25 و آقایم دل خود را بر این مردبلیعال، یعنی نابال مشغول نسازد، زیرا که اسمش مثل خودش است اسمش نابال است و حماقت بااوست، لیکن من کنیز تو خادمانی را که آقایم فرستاده بود، ندیدم.
\par 26 و الان‌ای آقایم به حیات خداوند و به حیات جان تو چونکه خداوند تو رااز ریختن خون و از انتقام کشیدن به‌دست خودمنع نموده است، پس الان دشمنانت و جویندگان ضرر آقایم مثل نابال بشوند.
\par 27 و الان این هدیه‌ای که کنیزت برای آقای خود آورده است به غلامانی که همراه آقایم می‌روند، داده شود.
\par 28 و تقصیر کنیز خود را عفو نما زیرا به درستی که خداوند برای آقایم خانه استوار بنا خواهد نمود، چونکه آقایم در جنگهای خداوند می‌کوشد وبدی در تمام روزهایت به تو نخواهد رسید.
\par 29 واگر‌چه کسی برای تعاقب تو و به قصد جانت برخیزد، اما جان آقایم در دسته حیات، نزد یهوه، خدایت، بسته خواهد شد. و اما جان دشمنانت راگویا از میان کفه فلاخن خواهد انداخت.
\par 30 وهنگامی که خداوند بر‌حسب همه احسانی که برای آقایم وعده داده است، عمل آورد، و تو راپیشوا بر اسرائیل نصب نماید.
\par 31 آنگاه این برای تو سنگ مصادم و به جهت آقایم لغزش دل نخواهد بود که خون بی‌جهت ریخته‌ای و آقایم انتقام خود را کشیده باشد، و چون خداوند به آقایم احسان نماید آنگاه کنیز خود را بیاد آور.»
\par 32 داود به ابیجایل گفت: «یهوه، خدای اسرائیل، متبارک باد که تو را امروز به استقبال من فرستاد.
\par 33 و حکمت تو مبارک و تو نیز مبارک باشی که امروز مرا از ریختن خون و از کشیدن انتقام خویش به‌دست خود منع نمودی.
\par 34 ولیکن به حیات یهوه، خدای اسرائیل، که مرا ازرسانیدن اذیت به تو منع نمود. اگر تعجیل ننموده، به استقبال من نمی آمدی البته تا طلوع صبح برای نابال ذکوری باقی نمی ماند.»
\par 35 پس داود آنچه را که به جهت او آورده بود، از دستش پذیرفته، به او گفت: «به سلامتی به خانه ات برو وببین که سخنت را شنیده، تو را مقبول داشتم.»
\par 36 پس ابیجایل نزد نابال برگشت و اینک اوضیافتی مثل ضیافت ملوکانه در خانه خودمی داشت و دل نابال در اندرونش شادمان بودچونکه بسیار مست بود و تا طلوع صبح چیزی کم یا زیاد به او خبر نداد.
\par 37 و بامدادان چون شراب از نابال بیرون رفت، زنش این چیزها را به او بیان کرد و دلش در اندرونش مرده گردید و خود مثل سنگ شد.
\par 38 و واقع شد که بعد از ده روز خداوندنابال را مبتلا ساخت که بمرد.
\par 39 و چون داود شنید که نابال مرده است، گفت: «مبارک باد خداوند که انتقام عار مرا ازدست نابال کشیده، و بنده خود را از بدی نگاه داشته است، زیرا خداوند شرارت نابال را به‌سرش رد نموده است و داود فرستاده، با ابیجایل سخن گفت تا او را به زنی خود بگیرد.»
\par 40 وخادمان داود نزد ابیجایل به کرمل آمده، با وی مکالمه کرده، گفتند: «داود ما را نزد تو فرستاده است تا تو را برای خویش به زنی بگیرد.»
\par 41 و اوبرخاسته، رو به زمین خم شد و گفت: «اینک کنیزت بنده است تا پایهای خادمان آقای خود رابشوید.»
\par 42 و ابیجایل تعجیل نموده، برخاست وبر الاغ خود سوار شد و پنج کنیزش همراهش روانه شدند و از عقب قاصدان داود رفته، زن اوشد.
\par 43 و داود اخینوعم یزرعیلیه را نیز گرفت وهردوی ایشان زن او شدند.و شاول دخترخود، میکال، زن داود را به فلطی ابن لایش که ازجلیم بود، داد.
\par 44 و شاول دخترخود، میکال، زن داود را به فلطی ابن لایش که ازجلیم بود، داد.
 
\chapter{26}

\par 1 پس زیفیان نزد شاول به جبعه آمده، گفتند: «آیا داود خویشتن را در تل حخیله که در مقابل بیابان است، پنهان نکرده است؟»
\par 2 آنگاه شاول برخاسته، به بیابان زیف فرود شد و سه هزار مرد از برگزیدگان اسرائیل همراهش رفتند تا داود را در بیابان زیف جستجونماید.
\par 3 و شاول در تل حخیله که در مقابل بیابان به‌سر راه است اردو زد، و داود در بیابان ساکن بود، و چون دید که شاول در عقبش در بیابان آمده است،
\par 4 داود جاسوسان فرستاده، دریافت کرد که شاول به تحقیق آمده است.
\par 5 و داود برخاسته، به‌جایی که شاول در آن اردو زده بود، آمد. و داود مکانی را که شاول وابنیر، پسر نیر، سردار لشکرش خوابیده بودند، ملاحظه کرد، و شاول در اندرون سنگرمی خوابید و قوم در اطراف او فرود آمده بودند.
\par 6 و داود به اخیملک حتی و ابیشای ابن صرویه برادر یوآب خطاب کرده، گفت: «کیست که همراه من نزد شاول به اردو بیاید؟» ابیشای گفت: «من همراه تو می‌آیم.»
\par 7 پس داود و ابیشای در شب به میان قوم آمدند و اینک شاول در اندرون سنگردراز شده، خوابیده بود، و نیزه‌اش نزد سرش درزمین کوبیده، و ابنیر و قوم در اطرافش خوابیده بودند.
\par 8 و ابیشای به داود گفت: «امروز خدا، دشمن تو را به‌دستت تسلیم نموده. پس الان اذن بده تا او را با نیزه یک دفعه به زمین بدوزم و او رادوباره نخواهم زد.»
\par 9 و داود به ابیشای گفت: «اورا هلاک مکن زیرا کیست که به مسیح خداونددست خود را دراز کرده، بی‌گناه باشد.
\par 10 و داودگفت: «به حیات یهوه قسم که یا خداوند او راخواهد زد یا اجلش رسیده، خواهد مرد یا به جنگ فرود شده، هلاک خواهد گردید.
\par 11 حاشابر من از خداوند که دست خود را بر مسیح خداوند دراز کنم اما الان نیزه‌ای را که نزد سرش است و سبوی آب را بگیر و برویم.»
\par 12 پس داودنیزه و سبوی آب را از نزد سر شاول گرفت و روانه شدند، و کسی نبود که ببیند و بداند یا بیدار شودزیرا جمیع ایشان در خواب بودند، چونکه خواب سنگین از خداوند بر ایشان مستولی شده بود.
\par 13 و داود به طرف دیگر گذشته، از دور به‌سرکوه بایستاد و مسافت عظیمی در میان ایشان بود.
\par 14 و داود قوم و ابنیر پسر نیر را صدا زده، گفت: «ای ابنیر جواب نمی دهی؟» و ابنیر جواب داده، گفت: «تو کیستی که پادشاه را می‌خوانی؟»
\par 15 داود به ابنیر گفت: «آیا تو مرد نیستی و دراسرائیل مثل تو کیست؟ پس چرا آقای خودپادشاه را نگاهبانی نمی کنی؟ زیرا یکی از قوم آمد تا آقایت پادشاه را هلاک کند.
\par 16 این کار که کردی خوب نیست، به حیات یهوه، شمامستوجب قتل هستید، چونکه آقای خود مسیح خداوند را نگاهبانی نکردید، پس الان ببین که نیزه پادشاه و سبوی آب که نزد سرش بود، کجاست؟»
\par 17 و شاول آواز داود را شناخته، گفت: «آیااین آواز توست‌ای پسر من داود؟» و داود گفت: «ای آقایم پادشاه آواز من است.»
\par 18 و گفت: «این از چه سبب است که آقایم بنده خود را تعاقب می‌کند؟ زیرا چه کردم و چه بدی در دست من است؟
\par 19 پس الان آقایم پادشاه سخنان بنده خودرا بشنود، اگر خداوند تو را بر من تحریک نموده است پس هدیه‌ای قبول نماید، و اگر بنی آدم باشند پس ایشان به حضور خداوند ملعون باشند، زیرا که امروز مرا از التصاق به نصیب خداوندمی رانند و می‌گویند برو و خدایان غیر را عبادت نما.
\par 20 و الان خون من از حضور خداوند به زمین ریخته نشود زیرا که پادشاه اسرائیل مثل کسی‌که کبک را بر کوه‌ها تعاقب می‌کند به جستجوی یک کیک بیرون آمده است.»
\par 21 شاول گفت: «گناه ورزیدم‌ای پسرم داود! برگرد و تو را دیگر اذیت نخواهم کرد، چونکه امروز جان من در نظر تو عزیز آمد اینک احمقانه رفتار نمودم و بسیار گمراه شدم.»
\par 22 داود درجواب گفت: «اینک نیزه پادشاه! پس یکی ازغلامان به اینجا گذشته، آن را بگیرد.
\par 23 و خداوندهر کس را بر‌حسب عدالت و امانتش پاداش دهد، چونکه امروز خداوند تو را به‌دست من سپرده بود. اما نخواستم دست خود را بر مسیح خداونددراز کنم.
\par 24 و اینک چنانکه جان تو امروز در نظرمن عظیم آمد جان من در نظر خداوند عظیم باشدو مرا از هر تنگی برهاند.»شاول به داود گفت: «مبارک باش‌ای پسرم داود، البته کارهای عظیم خواهی کرد و غالب خواهی شد.» پس داود راه خود را پیش گرفت و شاول به‌جای خودمراجعت کرد.
\par 25 شاول به داود گفت: «مبارک باش‌ای پسرم داود، البته کارهای عظیم خواهی کرد و غالب خواهی شد.» پس داود راه خود را پیش گرفت و شاول به‌جای خودمراجعت کرد.
 
\chapter{27}

\par 1 و داود در دل خود گفت: «الحال روزی به‌دست شاول هلاک خواهم شد. چیزی برای من از این بهتر نیست که به زمین فلسطینیان فرار کنم، و شاول از جستجوی من درتمامی حدود اسرائیل مایوس شود. پس از دست او نجات خواهم یافت.»
\par 2 پس داود برخاسته، باآن ششصد نفر که همراهش بودند نزد اخیش بن معوک، پادشاه جت گذشت.
\par 3 و داود نزد اخیش در جت ساکن شد، او و مردمانش هرکس با اهل خانه‌اش، و داود با دو زنش اخینوعم یزرعیلیه وابیجایل کرملیه زن نابال.
\par 4 و به شاول گفته شد که داود به جت فرار کرده است، پس او را دیگرجستجو نکرد.
\par 5 و داود به اخیش گفت: «الان اگر من در نظر توالتفات یافتم مکانی به من در یکی از شهرهای صحرا بدهند تا در آنجا ساکن شوم، زیرا که بنده تو چرا در شهر دارالسلطنه با تو ساکن شود.»
\par 6 پس اخیش در آن روز صقلغ را به او داد، لهذاصقلغ تا امروز از آن پادشاهان یهوداست.
\par 7 وعدد روزهایی که داود در بلاد فلسطینیان ساکن بود یک سال و چهار ماه بود.
\par 8 و داود و مردانش برآمده، بر جشوریان وجرزیان و عمالقه هجوم آوردند زیرا که این طوایف در ایام قدیم در آن زمین از شور تا به زمین مصر ساکن می‌بودند.
\par 9 و داود اهل آن زمین راشکست داده، مرد یا زنی زنده نگذاشت وگوسفندان و گاوان و الاغها و شتران و رخوت گرفته، برگشت و نزد اخیش آمد.
\par 10 و اخیش گفت: «امروز به کجا تاخت آوردید.» داود گفت: «بر جنوبی یهودا و جنوب یرحمئیلیان و به جنوب قینیان.
\par 11 و داود مرد یا زنی را زنده نگذاشت که به جت بیایند زیرا گفت مبادا درباره ما خبر‌آورده، بگویند که داود چنین کرده است وتمامی روزهایی که در بلاد فلسطینیان بماند، عادتش چنین خواهد بود.»و اخیش داود را تصدیق نموده، گفت: «خویشتن را نزد قوم خود اسرائیل بالکل مکروه نموده است، پس تا به ابد بنده من خواهد بود.»
\par 12 و اخیش داود را تصدیق نموده، گفت: «خویشتن را نزد قوم خود اسرائیل بالکل مکروه نموده است، پس تا به ابد بنده من خواهد بود.»
 
\chapter{28}

\par 1 و واقع شد در آن ایام که فلسطینیان لشکرهای خود را برای جنگ فراهم آوردند تا با اسرائیل مقاتله نمایند، و اخیش به داود گفت: «یقین بدان که تو و کسانت همراه من به اردو بیرون خواهید آمد.»
\par 2 داود به اخیش گفت: «به تحقیق خواهی دانست که بنده تو چه خواهدکرد.» اخیش به داود گفت: «از این جهت تو راهمیشه اوقات نگاهبان سرم خواهم ساخت.»
\par 3 و سموئیل وفات نموده بود، و جمیع اسرائیل به جهت او نوحه گری نموده، او را درشهرش رامه دفن کرده بودند، و شاول تمامی اصحاب اجنه و فالگیران را از زمین بیرون کرده بود.
\par 4 و فلسطینیان جمع شده، آمدند و در شونیم اردو زدند، و شاول تمامی اسرائیل را جمع کرده، در جلبوع اردو زدند.
\par 5 و چون شاول لشکرفلسطینیان را دید، بترسید و دلش بسیار مضطرب شد. 
\par 6 و شاول از خداوند سوال نمود و خداوند اورا جواب نداد، نه به خوابها و نه به اوریم و نه به انبیا.
\par 7 و شاول به خادمان خود گفت: «زنی را که صاحب اجنه باشد، برای من بطلبید تا نزد او رفته، از او مسالت نمایم.» خادمانش وی را گفتند: «اینک زنی صاحب اجنه در عین دور می‌باشد.»
\par 8 و شاول صورت خویش را تبدیل نموده، لباس دیگر پوشید و دو نفر همراه خود برداشته، رفت و شبانگاه نزد آن زن آمده، گفت: «تمنا اینکه به واسطه جن برای من فالگیری نمایی و کسی راکه به تو بگویم از برایم برآوری.»
\par 9 آن زن وی راگفت: «اینک آنچه شاول کرده است می‌دانی که چگونه اصحاب اجنه و فالگیران را از زمین منقطع نموده است، پس تو چرا برای جانم دام می‌گذاری تا مرا به قتل رسانی؟»
\par 10 و شاول برای وی به یهوه قسم خورده، گفت: «به حیات یهوه قسم که از این امر به تو هیچ بدی نخواهد رسید.»
\par 11 آن زن گفت: «از برایت که را برآورم؟» او گفت: «سموئیل را برای من برآور.»
\par 12 و چون آن زن سموئیل را دید به آواز بلند صدا زد و زن، شاول را خطاب کرده، گفت: «برای چه مرا فریب دادی زیرا تو شاول هستی.»
\par 13 پادشاه وی را گفت: «مترس! چه دیدی؟» آن زن در جواب شاول گفت: «خدایی را می‌بینم که از زمین بر می‌آید.»
\par 14 او وی را گفت: «صورت او چگونه است؟» زن گفت: «مردی پیر بر می‌آید و به ردایی ملبس است.» پس شاول دانست که سموئیل است و روبه زمین خم شده، تعظیم کرد.
\par 15 و سموئیل به شاول گفت: «چرا مرابرآورده، مضطرب ساختی؟» شاول گفت: «درشدت تنگی هستم چونکه فلسطینیان با من جنگ می‌نمایند و خدا از من دور شده، مرا نه به واسطه انبیا و نه به خوابها دیگر جواب می‌دهد، لهذا تو راخواندم تا مرا اعلام نمایی که چه باید بکنم.»
\par 16 سموئیل گفت: «پس چرا از من سوال می‌نمایی و حال آنکه خداوند از تو دور شده، دشمنت گردیده است.
\par 17 و خداوند به نحوی که به زبان من گفته بود، برای خود عمل نموده است، زیراخداوند سلطنت را از دست تو دریده، آن را به همسایه ات داود داده است.
\par 18 چونکه آوازخداوند را نشنیدی و شدت غضب او را برعمالیق به عمل نیاوردی، بنابراین خداوند امروزاین عمل را به تو نموده است.
\par 19 و خداونداسرائیل را نیز با تو به‌دست فلسطینیان خواهدداد، و تو و پسرانت فردا نزد من خواهید بود، و خداوند اردوی اسرائیل را نیز به‌دست فلسطینیان خواهد داد.»
\par 20 و شاول فور به تمامی قامتش بر زمین افتاد، و از سخنان سموئیل بسیار بترسید. وچونکه تمامی روز و تمامی شب نان نخورده بود، هیچ قوت نداشت.
\par 21 و چون آن زن نزد شاول آمده، دید که بسیار پریشان حال است وی راگفت: «اینک کنیزت آواز تو را شنید و جانم را به‌دست خود گذاشتم و سخنانی را که به من گفتی اطاعت نمودم.
\par 22 پس حال تمنا اینکه تو نیز آوازکنیز خود را بشنوی تا لقمه‌ای نان به حضورت بگذارم و بخوری تا قوت یافته، به راه خودبروی.»
\par 23 اما او انکار نموده، گفت: «نمی خورم.» لیکن چون خادمانش و آن زن نیز اصرار نمودند، آواز ایشان را بشنید و از زمین برخاسته، بر بسترنشست.
\par 24 و آن زن گوساله‌ای پرواری در خانه داشت. پس تعجیل نموده، آن را ذبح کرد و آردگرفته، خمیر ساخت و قرصهای نان فطیر پخت.و آنها را نزد شاول و خادمانش گذاشت که خوردند. پس برخاسته، در آن شب روانه شدند.
\par 25 و آنها را نزد شاول و خادمانش گذاشت که خوردند. پس برخاسته، در آن شب روانه شدند.
 
\chapter{29}

\par 1 و فلسطینیان همه لشکرهای خود را درافیق جمع کردند، و اسرائیلیان نزدچشمه‌ای که در یزرعیل است، فرود آمدند.
\par 2 وسرداران فلسطینیان صدها و هزارها می‌گذشتند، و داود و مردانش با اخیش در دنباله ایشان می‌گذشتند.
\par 3 و سرداران فلسطینیان گفتند که «این عبرانیان کیستند؟» و اخیش به جواب سرداران فلسطینیان گفت: «مگر این داود، بنده شاول، پادشاه اسرائیل نیست که نزد من این‌روزهایا این سالها بوده است و از روزی که نزد من آمد تاامروز در او عیبی نیافتم.»
\par 4 اما سرداران فلسطینیان بر وی غضبناک شدند، و سرداران فلسطینیان او را گفتند: «این مردرا باز گردان تا به‌جایی که برایش تعیین کرده‌ای برگردد، و با ما به جنگ نیاید، مبادا در جنگ دشمن ما بشود، زیرا این کس با چه چیز با آقای خود صلح کند آیا نه با سرهای این مردمان؟
\par 5 آیااین داود نیست که درباره او با یکدیگر رقص کرده، می‌سراییدند و می‌گفتند: «شاول هزارهای خود و داود ده هزارهای خویش را کشته است.»
\par 6 آنگاه اخیش داود را خوانده، او را گفت: «به حیات یهوه قسم که تو مرد راست هستی و خروج و دخول تو با من در اردو به نظر من پسند آمد، زیرا از روز آمدنت نزد من تا امروز از تو بدی ندیده‌ام لیکن در نظر سرداران پسند نیستی.
\par 7 پس الان برگشته، به سلامتی برو مبادا مرتکب عملی شوی که در نظر سرداران فلسطینیان ناپسند آید.»
\par 8 و داود به اخیش گفت: «چه کرده‌ام و از روزی که به حضور تو بوده‌ام تا امروز در بنده ات چه یافته‌ای تا آنکه به جنگ نیایم و با دشمنان آقایم پادشاه جنگ ننمایم؟»
\par 9 اخیش در جواب داود گفت: «می‌دانم که تودر نظر من مثل فرشته خدا نیکو هستی لیکن سرداران فلسطینیان گفتند که با ما به جنگ نیاید.
\par 10 پس الحال بامدادان با بندگان آقایت که همراه تو آمده‌اند، برخیز و چون بامدادان برخاسته باشید و روشنایی برای شما بشود، روانه شوید.»پس داود با کسان خود صبح زود برخاستند تا روانه شده، به زمین فلسطینیان برگردند وفلسطینیان به یزرعیل برآمدند.
\par 11 پس داود با کسان خود صبح زود برخاستند تا روانه شده، به زمین فلسطینیان برگردند وفلسطینیان به یزرعیل برآمدند.
 
\chapter{30}

\par 1 و واقع شد چون داود و کسانش در روزسوم به صقلغ رسیدند که عمالقه برجنوب و بر صقلغ هجوم آورده بودند، و صقلغ رازده آن را به آتش سوزانیده بودند.
\par 2 و زنان و همه کسانی را که در آن بودند از خرد و بزرگ اسیرکرده، هیچ‌کس را نکشته، بلکه همه را به اسیری برده، به راه خود رفته بودند.
\par 3 و چون داود وکسانش به شهر رسیدند، اینک به آتش سوخته، وزنان و پسران و دختران ایشان اسیر شده بودند.
\par 4 پس داود و قومی که همراهش بودند آواز خودرا بلند کرده، گریستند تا طاقت گریه کردن دیگرنداشتند.
\par 5 و دو زن داود اخینوعم یزرعیلیه وابیجایل، زن نابال کرملی، اسیر شده بودند.
\par 6 وداود بسیار مضطرب شد زیرا که قوم می‌گفتند که او را سنگسار کنند، چون جان تمامی قوم هر یک برای پسران و دختران خویش بسیار تلخ شده بود، اما داود خویشتن را از یهوه، خدای خود، تقویت نمود.
\par 7 و داود به ابیاتار کاهن، پسر اخیملک گفت: «ایفود را نزد من بیاور.» و ابیاتار ایفود را نزد داودآورد.
\par 8 و داود از خداوند سوال نموده، گفت: «اگر این فوج را تعاقب نمایم، آیا به آنها خواهم رسید؟» او وی را گفت: «تعاقب نما زیرا که به تحقیق خواهی رسید و رها خواهی کرد.»
\par 9 پس داود و ششصد نفر که همراهش بودند روانه شده، به وادی بسور آمدند و واماندگان در آنجا توقف نمودند.
\par 10 و داود با چهارصد نفر تعاقب نمود ودویست نفر توقف نمودند زیرا به حدی خسته شده بودند که از وادی بسور نتوانستند گذشت.
\par 11 پس شخصی مصری در صحرا یافته، او رانزد داود آوردند و به او نان دادند که خورد و او راآب نوشانیدند.
\par 12 و پاره‌ای از قرص انجیر و دوقرص کشمش به او دادند و چون خورد روحش به وی بازگشت زیرا که سه روز و سه شب نه نان خورده، و نه آب نوشیده بود.
\par 13 و داود او راگفت: «از آن که هستی و از کجا می‌باشی؟» اوگفت: «من جوان مصری و بنده شخص عمالیقی هستم، و آقایم مرا ترک کرده است زیرا سه روزاست که بیمار شده‌ام.
\par 14 ما به جنوب کریتیان و برملک یهودا و بر جنوب کالیب تاخت آوردیم. صقلغ را به آتش سوزانیدیم.»
\par 15 داود وی راگفت: «آیا مرا به آن گروه خواهی رسانید؟» اوگفت: «برای من به خدا قسم بخور که نه مرا بکشی و نه مرا به‌دست آقایم تسلیم کنی، پس تو را نزدآن گروه خواهم رسانید.»
\par 16 و چون او را به آنجا رسانید اینک بر روی تمامی زمین منتشر شده، می‌خوردند ومی نوشیدند و بزم می‌کردند، به‌سبب تمامی غنیمت عظیمی که از زمین فلسطینیان و از زمین یهودا آورده بودند.
\par 17 و داود ایشان را از وقت شام تا عصر روز دیگر می‌زد که از ایشان احدی رهایی نیافت جز چهارصد مرد جوان که بر شتران سوار شده، گریختند.
\par 18 و داود هرچه عمالقه گرفته بودند، بازگرفت و داود دو زن خود را بازگرفت.
\par 19 و چیزی از ایشان مفقود نشد از خرد وبزرگ و از پسران و دختران و غنیمت و از همه چیزهایی که برای خود گرفته بودند، بلکه داودهمه را باز آورد.
\par 20 و داود همه گوسفندان وگاوان خود را گرفت و آنها را پیش مواشی دیگرراندند و گفتند این است غنیمت داود.
\par 21 و داود نزد آن دویست نفر که از شدت خستگی نتوانسته بودند در عقب داود بروند وایشان را نزد وادی بسور واگذاشته بودند آمد، وایشان به استقبال داود و به استقبال قومی که همراهش بودند بیرون آمدند، و چون داود نزدقوم رسید از سلامتی ایشان پرسید.
\par 22 اما جمیع کسان شریر و مردان بلیعال از اشخاصی که با داودرفته بودند متکلم شده، گفتند: «چونکه همراه مانیامدند، از غنیمتی که باز آورده‌ایم چیزی به ایشان نخواهیم داد مگر به هر کس زن و فرزندان او را. پس آنها را برداشته، بروند.»
\par 23 لیکن داودگفت: «ای برادرانم چنین مکنید چونکه خداونداینها را به ما داده است و ما را حفظ نموده، آن فوج را که بر ما تاخت آورده بودند به‌دست ماتسلیم نموده است.
\par 24 و کیست که در این امر به شما گوش دهد زیرا قسمت آنانی که نزد اسباب می‌مانند مثل قسمت آنانی که به جنگ می‌روند، خواهد بود و هر دو قسمت مساوی خواهند برد.»
\par 25 و از آن روز به بعد چنین شد که این را قاعده وقانون در اسرائیل تا امروز قرار داد.
\par 26 و چون داود به صقلغ رسید، بعضی ازغنیمت را برای مشایخ یهودا و دوستان خودفرستاده، گفت: «اینک هدیه‌ای از غنیمت دشمنان خداوند برای شماست.»
\par 27 برای اهل بیت ئیل و اهل راموت جنوبی و اهل یتیر؛
\par 28 وبرای اهل عروعیر و اهل سفموت و اهل اشتموع؛
\par 29 و برای اهل راکال و اهل شهرهای یرحمئیلیان و اهل شهرهای قینیان؛
\par 30 و برای اهل حرما و اهل کورعاشان و اهل عتاق؛و برای اهل حبرون و جمیع مکانهایی که داود و کسانش درآنها آمد و رفت می‌کردند.
\par 31 و برای اهل حبرون و جمیع مکانهایی که داود و کسانش درآنها آمد و رفت می‌کردند.
 
\chapter{31}

\par 1 و فلسطینیان با اسرائیل جنگ کردند ومردان اسرائیل از حضور فلسطینیان فرار کردند، و در کوه جلبوع کشته شده، افتادند.
\par 2 و فلسطینیان، شاول و پسرانش را به سختی تعاقب نمودند، و فلسطینیان یوناتان و ابیناداب وملکیشوع پسران شاول را کشتند.
\par 3 و جنگ برشاول سخت شد، و تیراندازان دور او را گرفتند وبه‌سبب تیراندازان به غایت دلتنگ گردید.
\par 4 و شاول به سلاحدار خود گفت: «شمشیرخود را کشیده، آن را به من فرو بر مبادا این نامختونان آمده، مرا مجروح سازند و مرا افتضاح نمایند.» اما سلاحدارش نخواست زیرا که بسیاردر ترس بود. پس شاول شمشیر خود را گرفته، برآن افتاد.
\par 5 و هنگامی که سلاحدارش شاول را دید که مرده است، او نیز بر شمشیر خود افتاده، با اوبمرد.
\par 6 پس شاول و سه پسرش و سلاحدارش وجمیع کسانش نیز در آن روز با هم مردند.
\par 7 وچون مردان اسرائیل که به آن طرف دره و به آن طرف اردن بودند، دیدند که مردان اسرائیل فرارکرده‌اند و شاول و پسرانش مرده‌اند، شهرهای خود را ترک کرده، گریختند و فلسطینیان آمده، در آنها ساکن شدند.
\par 8 و در فردای آن روز، چون فلسطینیان برای برهنه کردن کشتگان آمدند، شاول و سه پسرش رایافتند که در کوه جلبوع افتاده بودند.
\par 9 پس سر اورا بریدند و اسلحه‌اش را بیرون کرده، به زمین فلسطینیان، به هر طرف فرستادند تا به بتخانه های خود و به قوم مژده برسانند. 
\par 10 و اسلحه او را درخانه عشتاروت نهادند و جسدش را بر حصاربیتشان آویختند.
\par 11 و چون ساکنان یابیش جلعاد، آنچه را که فلسطینیان به شاول کرده بودندشنیدند،جمیع مردان شجاع برخاسته، و تمامی شب سفر کرده، جسد شاول و اجسادپسرانش را از حصار بیتشان گرفتند، و به یابیش برگشته، آنها را در آنجا سوزانیدند.
\par 12 جمیع مردان شجاع برخاسته، و تمامی شب سفر کرده، جسد شاول و اجسادپسرانش را از حصار بیتشان گرفتند، و به یابیش برگشته، آنها را در آنجا سوزانیدند.


\end{document}