\begin{document}

\title{2 Samuel}

 
\chapter{1}

\par 1 و بعد از وفات شاول و مراجعت داود از مقاتله عمالقه واقع شد که داود دو روز درصقلغ توقف نمود.
\par 2 و در روز سوم ناگاه شخصی از نزد شاول با لباس دریده و خاک بر سرش ریخته از لشکر آمد، و چون نزد داود رسید، به زمین افتاده، تعظیم نمود.
\par 3 و داود وی را گفت: «ازکجا آمدی؟» او در جواب وی گفت: «از لشکراسرائیل فرار کرده‌ام.»
\par 4 داود وی را گفت: «مراخبر بده که کار چگونه شده است.» او گفت: «قوم از جنگ فرار کردند و بسیاری از قوم نیز افتادند ومردند، و هم شاول و پسرش، یوناتان، مردند.»
\par 5 پس داود به جوانی که او را مخبر ساخته بود، گفت: «چگونه دانستی که شاول و پسرش یوناتان مرده‌اند.»
\par 6 و جوانی که او را مخبر ساخته بود، گفت: «اتفاق مرا در کوه جلبوع گذر افتاد و اینک شاول بر نیزه خود تکیه می‌نمود، و اینک ارابه‌ها وسواران او را به سختی تعاقب می‌کردند.
\par 7 و به عقب نگریسته، مرا دید و مرا خواند و جواب دادم، لبیک.
\par 8 او مرا گفت: تو کیستی؟ وی را گفتم: عمالیقی هستم.
\par 9 او به من گفت: تمنا اینکه بر من بایستی و مرا بکشی زیرا که پریشانی مرا در‌گرفته است چونکه تمام جانم تا بحال در من است.
\par 10 پس بر او ایستاده، او را کشتم زیرا دانستم که بعد از افتادنش زنده نخواهد ماند و تاجی که برسرش و بازوبندی که بر بازویش بود، گرفته، آنها را اینجا نزد آقایم آوردم.»
\par 11 آنگاه داود جامه خود را گرفته، آن را دریدو تمامی کسانی که همراهش بودند، چنین کردند.
\par 12 و برای شاول و پسرش، یوناتان، و برای قوم خداوند و خاندان اسرائیل ماتم گرفتند و گریه کردند، و تا شام روزه داشتند، زیرا که به دم شمشیر افتاده بودند.
\par 13 و داود به جوانی که او رامخبر ساخت، گفت: «تو از کجا هستی؟» او گفت: «من پسر مرد غریب عمالیقی هستم.»
\par 14 داود وی را گفت: «چگونه نترسیدی که دست خود را بلندکرده، مسیح خداوند را هلاک ساختی؟»
\par 15 آنگاه داود یکی از خادمان خود را طلبیده، گفت: «نزدیک آمده، او را بکش.» پس او را زد که مرد.
\par 16 و داود او راگفت: «خونت بر سر خودت باشدزیرا که دهانت بر تو شهادت داده، گفت که من مسیح خداوند را کشتم.»
\par 17 و داود این مرثیه را درباره شاول و پسرش یوناتان انشا کرد.
\par 18 و امر فرمود که نشید قوس رابه بنی یهودا تعلیم دهند. اینک در سفر یاشرمکتوب است:
\par 19 «زیبایی تو‌ای اسرائیل در مکانهای بلندت کشته شد. جباران چگونه افتادند.
\par 20 در جت اطلاع ندهید و در کوچه های اشقلون خبر مرسانید، مبادا دختران فلسطینیان شادی کنند. و مبادا دختران نامختونان وجد نمایند.
\par 21 ‌ای کوههای جلبوع، شبنم و باران بر شما نبارد. و نه از کشتزارهایت هدایا بشود، زیرا در آنجاسپر جباران دور انداخته شد. سپر شاول که گویابه روغن مسح نشده بود.
\par 22 از خون کشتگان و از پیه جباران، کمان یوناتان برنگردید. و شمشیر شاول تهی برنگشت.
\par 23 شاول و یوناتان در حیات خویش محبوب نازنین بودند. و در موت خود از یکدیگر جدانشدند. از عقابها تیزپرتر و از شیران تواناتر بودند.
\par 24 ‌ای دختران اسرائیل برای شاول گریه کنید که شما را به قرمز و نفایس ملبس می‌ساخت وزیورهای طلا بر لباس شما می‌گذاشت.
\par 25 شجاعان در معرض جنگ چگونه افتادند. ای یوناتان بر مکان های بلند خود کشته شدی.
\par 26 ‌ای برادر من یوناتان برای تو دلتنگ شده‌ام. برای من بسیار نازنین بودی. محبت تو با من عجیب تر از محبت زنان بود.جباران چگونه افتادند. و چگونه اسلحه جنگ تلف شد.»
\par 27 جباران چگونه افتادند. و چگونه اسلحه جنگ تلف شد.»
 
\chapter{2}

\par 1 و بعد از آن واقع شد که داود از خداوندسوال نموده، گفت: «آیا به یکی ازشهرهای یهودا برآیم؟» خداوند وی را گفت: «برآی.» داود گفت: «کجا برآیم؟» گفت: «به حبرون.»
\par 2 پس داود به آنجا برآمد و دو زنش نیزاخینوعم یزرعیلیه و ابیجایل زن نابال کرملی.
\par 3 وداود کسانی را که با او بودند با خاندان هر یکی برد، و در شهرهای حبرون ساکن شدند.
\par 4 و مردان یهودا آمده، داود را در آنجا مسح کردند، تا برخاندان یهودا پادشاه شود. و به داود خبر داده، گفتند که «اهل یابیش جلعاد بودند که شاول رادفن کردند.»
\par 5 پس داود قاصدان نزد اهل یابیش جلعاد فرستاده، به ایشان گفت: «شما از جانب خداوند مبارک باشید زیرا که این احسان را به آقای خود شاول نمودید و او را دفن کردید.
\par 6 والان خداوند به شما احسان و راستی بنماید و من نیز جزای این نیکویی را به شما خواهم نمودچونکه این کار را کردید.
\par 7 و حال دستهای شماقوی باشد و شما شجاع باشید زیرا آقای شماشاول مرده است و خاندان یهودا نیز مرا بر خود به پادشاهی مسح نمودند.»
\par 8 اما ابنیر بن نیر سردار لشکر شاول، ایشبوشت بن شاول را گرفته، او را به محنایم برد.
\par 9 و او را بر جلعاد و بر آشوریان و بر یزرعیل و برافرایم و بر بنیامین و بر تمامی اسرائیل پادشاه ساخت.
\par 10 و ایشبوشت بن شاول هنگامی که براسرائیل پادشاه شد چهل ساله بود، و دو سال سلطنت نمود، اما خاندان یهودا، داود را متابعت کردند.
\par 11 و عدد ایامی که داود در حبرون برخاندان یهودا سلطنت نمود هفت سال و شش ماه بود.
\par 12 و ابنیر بن نیر و بندگان ایشبوشت بن شاول ازمحنایم به جبعون بیرون آمدند.
\par 13 و یوآب بن صرویه و بندگان داود بیرون آمده، نزد برکه جبعون با آنها ملتقی شدند، و اینان به این طرف برکه و آنان بر آن طرف برکه نشستند.
\par 14 و ابنیر به یوآب گفت: «الان جوانان برخیزند و در حضور مابازی کنند.» یوآب گفت: «برخیزید.»
\par 15 پس برخاسته، به شماره عبور کردند، دوازده نفر برای بنیامین و برای ایشبوشت بن شاول و دوازده نفر ازبندگان داود.
\par 16 و هر یک از ایشان سر حریف خود را گرفته، شمشیر خود را در پهلویش زد، پس با هم افتادند. پس آن مکان را که در جبعون است، حلقت هصوریم نامیدند.
\par 17 و آن روزجنگ بسیار سخت بود و ابنیر و مردان اسرائیل ازحضور بندگان داود منهزم شدند.
\par 18 و سه پسر صرویه، یوآب و ابیشای وعسائیل، در آنجا بودند، و عسائیل مثل غزال بری سبک پا بود.
\par 19 و عسائیل، ابنیر را تعاقب کرد ودر رفتن به طرف راست یا چپ از تعاقب ابنیرانحراف نورزید.
\par 20 و ابنیر به عقب نگریسته، گفت: «آیا تو عسائیل هستی؟» گفت: «من هستم.»
\par 21 ابنیر وی را گفت: «به طرف راست یا به طرف چپ خود برگرد و یکی از جوانان را گرفته، اسلحه او را بردار.» اما عسائیل نخواست که ازعقب او انحراف ورزد.
\par 22 پس ابنیر بار دیگر به عسائیل گفت: «از عقب من برگرد چرا تو را به زمین بزنم، پس چگونه روی خود را نزد برادرت یوآب برافرازم.»
\par 23 و چون نخواست که برگرددابنیر او را به موخر نیزه خود به شکمش زد که سرنیزه از عقبش بیرون آمد و در آنجا افتاده، درجایش مرد. و هر کس که به مکان افتادن و مردن عسائیل رسید، ایستاد.
\par 24 اما یوآب و ابیشای، ابنیر را تعاقب کردند وچون ایشان به تل امه که به مقابل جیح در راه بیابان جبعون است رسیدند، آفتاب فرو رفت.
\par 25 وبنی بنیامین بر عقب ابنیر جمع شده، یک گروه شدند و بر سر یک تل ایستادند.
\par 26 و ابنیر یوآب را صدا زده، گفت که «آیا شمشیر تا به ابد هلاک سازد؟ آیا نمی دانی که آخر به تلخی خواهدانجامید؟ پس تا به کی قوم را امر نمی کنی که ازتعاقب برادران خویش برگردند.»
\par 27 یوآب درجواب گفت: «به خدای حی قسم اگر سخن نگفته بودی هر آینه قوم در صبح از تعاقب برادران خودبرمی گشتند.»
\par 28 پس یوآب کرنا نواخته، تمامی قوم ایستادند و اسرائیل را باز تعاقب ننمودند ودیگر جنگ نکردند.
\par 29 و ابنیر و کسانش، تمامی آن شب را از راه عربه رفته، از اردن عبور کردند و از تمامی یترون گذشته، به محنایم رسیدند.
\par 30 و یوآب از عقب ابنیر برگشته، تمامی قوم را جمع کرد. و از بندگان داود سوای عسائیل نوزده نفر مفقود بودند.
\par 31 امابندگان داود، بنیامین و مردمان ابنیر را زدند که ازایشان سیصد و شصت نفر مردند.و عسائیل رابرداشته، او را در قبر پدرش که در بیت لحم است، دفن کردند و یوآب و کسانش، تمامی شب کوچ کرده، هنگام طلوع فجر به حبرون رسیدند.
\par 32 و عسائیل رابرداشته، او را در قبر پدرش که در بیت لحم است، دفن کردند و یوآب و کسانش، تمامی شب کوچ کرده، هنگام طلوع فجر به حبرون رسیدند.
 
\chapter{3}

\par 1 و جنگ در میان خاندان شاول و خاندان داود به طول انجامید و داود روز به روزقوت می‌گرفت و خاندان شاول روز به روزضعیف می‌شدند.
\par 2 و برای داود در حبرون پسران زاییده شدند، و نخست زاده‌اش، عمون، از اخینوعم یزرعیلیه بود.
\par 3 و دومش، کیلاب، از ابیجایل، زن نابال کرملی، و سوم، ابشالوم، پسر معکه، دختر تلمای پادشاه جشور.
\par 4 و چهارم ادونیا، پسر حجیت، وپنجم شفطیا پسر ابیطال،
\par 5 و ششم، یترعام ازعجله، زن داود. اینان برای داود در حبرون زاییده شدند.
\par 6 و هنگامی که جنگ در میان خاندان شاول وخاندان داود می‌بود، ابنیر، خاندان شاول راتقویت می‌نمود.
\par 7 و شاول را کنیزی مسمی به رصفه دختر ایه بود، و ایشبوشت به ابنیر گفت: «چرا به کنیز پدرم درآمدی؟»
\par 8 و خشم ابنیر به‌سبب سخن ایشبوشت بسیار افروخته شده، گفت: «آیا من سر سگ برای یهودا هستم و حال آنکه امروز به خاندان پدرت، شاول، و برادرانش واصحابش احسان نموده‌ام و تو را به‌دست داودتسلیم نکرده‌ام که به‌سبب این زن امروز گناه بر من اسناد می‌دهی؟
\par 9 خدا مثل این و زیاده از این به ابنیر بکند اگر من به طوری که خداوند برای داودقسم خورده است، برایش چنین عمل ننمایم.
\par 10 تا سلطنت را از خاندان شاول نقل نموده، کرسی داود را بر اسرائیل و یهودا از دان تا بئرشبع پایدار گردانم.»
\par 11 و او دیگر نتوانست در جواب ابنیر سخنی گوید زیرا که از او می‌ترسید.
\par 12 پس ابنیر در آن حین قاصدان نزد داودفرستاده، گفت: «این زمین مال کیست؟ و گفت تو بامن عهد ببند و اینک دست من با تو خواهد بود تاتمامی اسرائیل را به سوی تو برگردانم.»
\par 13 اوگفت: «خوب، من با تو عهد خواهم بست ولیکن یک چیز از تو می‌طلبم و آن این است که روی مرانخواهی دید، جز اینکه اول چون برای دیدن روی من بیایی میکال، دختر شاول را بیاوری.»
\par 14 پس داود رسولان نزد ایشبوشت بن شاول فرستاده، گفت: «زن من، میکال را که برای خود به صد قلفه فلسطینیان نامزد ساختم، نزد من بفرست.»
\par 15 پس ایشبوشت فرستاده، او را از نزد شوهرش فلطئیل بن لایش گرفت.
\par 16 و شوهرش همراهش رفت و در عقبش تا حوریم گریه می‌کرد. پس ابنیر وی را گفت: «برگشته، برو.» و اوبرگشت.
\par 17 و ابنیر با مشایخ اسرائیل تکلم نموده، گفت: «قبل از این داود را می‌طلبیدید تا بر شماپادشاهی کند.
\par 18 پس الان این را به انجام برسانیدزیرا خداوند درباره داود گفته است که به وسیله بنده خود، داود، قوم خویش، اسرائیل را از دست فلسطینیان و از دست جمیع دشمنان ایشان نجات خواهم داد.»
\par 19 و ابنیر به گوش بنیامینیان نیزسخن گفت. و ابنیر هم به حبرون رفت تا آنچه راکه در نظر اسرائیل و در نظر تمامی خاندان بنیامین پسند آمده بود، به گوش داود بگوید.
\par 20 پس ابنیر بیست نفر با خود برداشته، نزدداود به حبرون آمد و داود به جهت ابنیر ورفقایش ضیافتی برپا کرد.
\par 21 و ابنیر به داود گفت: «من برخاسته، خواهم رفت و تمامی اسرائیل رانزد آقای خود، پادشاه، جمع خواهم آورد تا با توعهد ببندند و به هر‌آنچه دلت می‌خواهد سلطنت نمایی. پس داود ابنیر را مرخص نموده، او به سلامتی برفت.
\par 22 و ناگاه بندگان داود و یوآب از غارتی بازآمده، غنیمت بسیار با خود آوردند. و ابنیر با داوددر حبرون نبود زیرا وی را رخصت داده، و او به سلامتی رفته بود.
\par 23 و چون یوآب و تمامی لشکری که همراهش بودند، برگشتند، یوآب راخبر داده، گفتند که «ابنیر بن نیر نزد پادشاه آمد واو را رخصت داده و به سلامتی رفت.»
\par 24 پس یوآب نزد پادشاه آمده، گفت: «چه کردی! اینک ابنیر نزد تو آمد. چرا او را رخصت دادی و رفت؟ 
\par 25 ابنیر بن نیر را می‌دانی که او آمد تا تو را فریب دهد و خروج و دخول تو را بداند و هر کاری را که می‌کنی دریافت کند.»
\par 26 و یوآب از حضور داود بیرون رفته، قاصدان در عقب ابنیر فرستاد که او را از چشمه سیره بازآوردند اما داود ندانست.
\par 27 و چون ابنیربه حبرون برگشت، یوآب او را در میان دروازه به کنار کشید تا با او به خفیه سخن گوید و به‌سبب خون برادرش عسائیل به شکم او زد که مرد.
\par 28 وبعد از آن چون داود این را شنید، گفت: «من وسلطنت من به حضور خداوند از خون ابنیر بن نیرتا به ابد بری هستیم.
\par 29 پس بر سر یوآب و تمامی خاندان پدرش قرار گیرد و کسی‌که جریان وبرص داشته باشد و بر عصا تکیه کند و به شمشیربیفتد و محتاج نان باشد، از خاندان یوآب منقطع نشود.»
\par 30 و یوآب و برادرش ابیشای، ابنیر راکشتند، به‌سبب این که برادر ایشان، عسائیل را درجبعون در جنگ کشته بود.
\par 31 و داود به یوآب و تمامی قومی که همراهش بودند، گفت: «جامه خود را بدرید وپلاس بپوشید و برای ابنیر نوحه کنید.» و داودپادشاه در عقب جنازه رفت.
\par 32 و ابنیر را درحبرون دفن کردند و پادشاه آواز خود را بلندکرده، نزد قبر ابنیر گریست و تمامی قوم گریه کردند.
\par 33 و پادشاه برای ابنیر مرثیه خوانده، گفت: «آیا باید ابنیر بمیرد به طوری که شخص احمق می‌میرد.
\par 34 دستهای تو بسته نشد وپایهایت در زنجیر گذاشته نشد. مثل کسی‌که پیش شریران افتاده باشد افتادی.» پس تمامی قوم بار دیگر برای او گریه کردند.
\par 35 و تمامی قوم چون هنوز روز بود آمدند تا داود را نان بخوراننداما داود قسم خورده، گفت: «خدا به من مثل این بلکه زیاده از این بکند اگر نان یا چیز دیگر پیش ازغروب آفتاب بچشم.»
\par 36 و تمامی قوم ملتفت شدند و به نظر ایشان پسند آمد. چنانکه هر‌چه پادشاه می‌کرد، در نظر تمامی قوم پسند می‌آمد.
\par 37 و جمیع قوم و تمامی اسرائیل در آن روزدانستند که کشتن ابنیر بن نیر از پادشاه نبود.
\par 38 وپادشاه به خادمان خود گفت: «آیا نمی دانید که سروری و مرد بزرگی امروز در اسرائیل افتاد؟و من امروز با آنکه به پادشاهی مسح شده‌ام ضعیف هستم و این مردان، یعنی پسران صرویه ازمن تواناترند. خداوند عامل شرارت را بر‌حسب شرارتش جزا دهد.»
\par 39 و من امروز با آنکه به پادشاهی مسح شده‌ام ضعیف هستم و این مردان، یعنی پسران صرویه ازمن تواناترند. خداوند عامل شرارت را بر‌حسب شرارتش جزا دهد.»
 
\chapter{4}

\par 1 و چون پسر شاول شنید که ابنیر در حبرون مرده است دستهایش ضعیف شد، وتمامی اسرائیل پریشان گردیدند.
\par 2 و پسر شاول دو مرد داشت که سردار فوج بودند؛ اسم یکی بعنه و اسم دیگری ریکاب بود، پسران رمون بئیروتی از بنی بنیامین، زیرا که بئیروت با بنیامین محسوب بود.
\par 3 و بئیروتیان به جتایم فرار کرده، در آنجا تا امروز غربت پذیرفتند.
\par 4 و یوناتان پسر شاول را پسری لنگ بود که هنگام رسیدن خبر شاول و یوناتان از یزرعیل، پنج ساله بود، و دایه‌اش او را برداشته، فرار کرد. وچون به فرار کردن تعجیل می‌نمود، او افتاد و لنگ شد و اسمش مفیبوشت بود.
\par 5 و ریکاب و بعنه، پسران رمون بئیروتی روانه شده، در وقت گرمای روز به خانه ایشبوشت داخل شدند و او به خواب ظهر بود.
\par 6 پس به بهانه‌ای که گندم بگیرند در میان خانه داخل شده، به شکم او زدند و ریکاب و برادرش بعنه فرارکردند.
\par 7 و چون به خانه داخل شدند و او بربسترش در خوابگاه خود می‌خوابید، او را زدند وکشتند و سرش را از تن جدا کردند و سرش راگرفته، از راه عربه تمامی شب کوچ کردند.
\par 8 و سرایشبوشت را نزد داود به حبرون آورده، به پادشاه گفتند: «اینک سر دشمنت، ایشبوشت، پسرشاول، که قصد جان تو می‌داشت. و خداوندامروز انتقام آقای ما پادشاه را از شاول و ذریه‌اش کشیده است.»
\par 9 و داود ریکاب و برادرش بعنه، پسران رمون بئیروتی را جواب داده، به ایشان گفت: «قسم به حیات خداوند که جان مرا از هر تنگی فدیه داده است.
\par 10 وقتی که کسی مرا خبر داده، گفت که اینک شاول مرده است و گمان می‌برد که بشارت می‌آورد، او را گرفته، در صقلغ کشتم و این اجرت بشارت بود که به او دادم.
\par 11 پس چندمرتبه زیاده چون مردان شریر، شخص صالح رادر خانه‌اش بر بسترش بکشند، آیا خون او را ازدست شما مطالبه نکنم؟ و شما را از زمین هلاک نسازم؟»پس داود خادمان خود را امر فرمودکه ایشان را کشتند و دست و پای ایشان را قطع نموده، بر برکه حبرون آویختند. اما سرایشبوشت را گرفته در قبر ابنیر در حبرون دفن کردند.
\par 12 پس داود خادمان خود را امر فرمودکه ایشان را کشتند و دست و پای ایشان را قطع نموده، بر برکه حبرون آویختند. اما سرایشبوشت را گرفته در قبر ابنیر در حبرون دفن کردند.
 
\chapter{5}

\par 1 و جمیع اسباط اسرائیل نزد داود به حبرون آمدند و متکلم شده، گفتند: «اینک مااستخوان و گوشت تو هستیم.
\par 2 و قبل از این نیزچون شاول بر ما سلطنت می‌نمود تو بودی که اسرائیل را بیرون می‌بردی و اندرون می‌آوردی، و خداوند تو را گفت که تو قوم من، اسرائیل رارعایت خواهی کرد و بر اسرائیل پیشوا خواهی بود.»
\par 3 و جمیع مشایخ اسرائیل نزد پادشاه به حبرون آمدند، و داود پادشاه در حبرون به حضورخداوند با ایشان عهد بست و داود را بر اسرائیل به پادشاهی مسح نمودند.
\par 4 و داود هنگامی که پادشاه شد سی ساله بود، و چهل سال سلطنت نمود.
\par 5 هفت سال و شش ماه در حبرون بر یهوداسلطنت نمود، و سی و سه سال در اورشلیم برتمامی اسرائیل و یهودا سلطنت نمود.
\par 6 و پادشاه با مردانش به اورشلیم به مقابله یبوسیان که ساکنان زمین بودند، رفت، و ایشان به داود متکلم شده، گفتند: «به اینجا داخل نخواهی شد جز اینکه کوران و لنگان را بیرون کنی.» زیراگمان بردند که داود به اینجا داخل نخواهد شد.
\par 7 و داود قلعه صهیون را گرفت که همان شهر داوداست.
\par 8 و در آن روز داود گفت: «هر‌که یبوسیان رابزند و به قنات رسیده، لنگان و کوران را که مبغوض جان داود هستند (بزند).» بنابرین می‌گویند کور و لنگ، به خانه داخل نخواهند شد.
\par 9 و داود در قلعه ساکن شد و آن را شهر داودنامید، و داود به اطراف ملو و اندرونش عمارت ساخت.
\par 10 و داود ترقی کرده، بزرگ می‌شد و یهوه، خدای صبایوت، با وی می‌بود.
\par 11 و حیرام، پادشاه صور، قاصدان و درخت سرو آزاد و نجاران و سنگ تراشان نزد داودفرستاده، برای داود خانه‌ای بنا نمودند.
\par 12 پس داود فهمید که خداوند او را بر اسرائیل به پادشاهی استوار نموده، و سلطنت او را به‌خاطرقوم خویش اسرائیل برافراشته است.
\par 13 و بعد از آمدن داود از حبرون کنیزان و زنان دیگر از اورشلیم گرفت، و باز برای داود پسران ودختران زاییده شدند.
\par 14 و نامهای آنانی که برای او در اورشلیم زاییده شدند، این است: شموع وشوباب و ناتان و سلیمان،
\par 15 و یبجار و الیشوع ونافج و یافیع،
\par 16 و الیشمع و الیداع و الیفلط.
\par 17 و چون فلسطینیان شنیدند که داود را به پادشاهی اسرائیل مسح نموده‌اند، جمیع فلسطینیان برآمدند تا داود را بطلبند، و چون داوداین را شنید به قلعه فرود آمد.
\par 18 و فلسطینیان آمده، در وادی رفائیان منتشر شدند.
\par 19 و داود ازخداوند سوال نموده، گفت: «آیا به مقابله فلسطینیان برآیم و ایشان را به‌دست من تسلیم خواهی نمود؟» خداوند به داود گفت: «برو زیراکه فلسطینیان را البته به‌دست تو خواهم داد.»
\par 20 وداود به بعل فراصیم آمد و داود ایشان را در آنجاشکست داده، گفت: «خداوند دشمنانم را ازحضور من رخنه کرد مثل رخنه آبها.» بنابرین آن مکان را بعل فراصیم نام نهادند.
\par 21 و بتهای خودرا در آنجا ترک کردند و داود و کسانش آنها رابرداشتند.
\par 22 و فلسطینیان بار دیگر برآمده، در وادی رفائیان منتشر شدند.
\par 23 و چون داود از خداوندسوال نمود، گفت: «برمیا، بلکه از عقب ایشان دورزده، پیش درختان توت بر ایشان حمله آور.
\par 24 وچون آواز صدای قدمها در سر درختان توت بشنوی، آنگاه تعجیل کن زیرا که در آن وقت خداوند پیش روی تو بیرون خواهد آمد تا لشکرفلسطینیان را شکست دهد.»پس داود چنانکه خداوند او را امر فرموده بود، کرد، و فلسطینیان رااز جبعه تا جازر شکست داد.
\par 25 پس داود چنانکه خداوند او را امر فرموده بود، کرد، و فلسطینیان رااز جبعه تا جازر شکست داد.
 
\chapter{6}

\par 1 و داود بار دیگر جمیع برگزیدگان اسرائیل، یعنی سی هزار نفر را جمع کرد.
\par 2 و داود باتمامی قومی که همراهش بودند برخاسته، ازبعلی یهودا روانه شدند تا تابوت خدا را که به اسم، یعنی به اسم یهوه صبایوت که بر کروبیان نشسته است، مسمی می‌باشد، از آنجا بیاورند.
\par 3 و تابوت خدا را بر ارابه‌ای نو گذاشتند و آن را از خانه ابیناداب که در جبعه است، برداشتند، و عزه واخیو، پسران ابیناداب، ارابه نو را راندند.
\par 4 و آن رااز خانه ابیناداب که در جبعه است با تابوت خداوند آوردند و اخیو پیش تابوت می‌رفت.
\par 5 وداود و تمامی خاندان اسرائیل با انواع آلات چوب سرو و بربط و رباب و دف و دهل و سنجهابه حضور خداوند بازی می‌کردند.
\par 6 و چون به خرمنگاه ناکون رسیدند عزه دست خود را به تابوت خداوند دراز کرده، آن را گرفت زیرا گاوان می‌لغزیدند.
\par 7 پس غضب خداوند برعزه افروخته شده، خدا او را در آنجا به‌سبب تقصیرش زد، و در آنجا نزد تابوت خدا مرد.
\par 8 وداود غمگین شد زیرا خداوند بر عزه رخنه کرده بود، و آن مکان را تا به امروز فارص عزه نام نهاد.
\par 9 و در آن روز، داود از خداوند ترسیده، گفت که تابوت خداوند نزد من چگونه بیاید.
\par 10 و داودنخواست که تابوت خداوند را نزد خود به شهرداود بیاورد. پس داود آن را به خانه عوبید ادوم جتی برگردانید.
\par 11 و تابوت خداوند در خانه عوبید ادوم جتی سه ماه ماند و خداوند عوبیدادوم و تمامی خاندانش را برکت داد.
\par 12 و داود پادشاه را خبر داده، گفتند که: خداوند خانه عوبید ادوم و جمیع مایملک او را به‌سبب تابوت خدا برکت داده است. پس داود رفت و تابوت خدا را از خانه عوبید ادوم به شهر داود به شادمانی آورد.
\par 13 و چون بردارندگان تابوت خداوند شش قدم رفته بودند، گاوان و پرواریهاذبح نمود.
\par 14 و داود با تمامی قوت خود به حضور خداوند رقص می‌کرد، و داود به ایفودکتان ملبس بود.
\par 15 پس داود و تمامی خاندان اسرائیل، تابوت خداوند را به آواز شادمانی وآواز کرنا آوردند.
\par 16 و چون تابوت خداوندداخل شهر داود می‌شد، میکال دختر شاول ازپنجره نگریسته، داود پادشاه را دید که به حضورخداوند جست وخیز و رقص می‌کند، پس او رادر دل خود حقیر شمرد.
\par 17 و تابوت خداوند را درآورده، آن را درمکانش در میان خیمه‌ای که داود برایش برپاداشته بود، گذاشتند، و داود به حضور خداوندقربانی های سوختنی و ذبایح سلامتی گذرانید.
\par 18 و چون داود از گذرانیدن قربانی های سوختنی و ذبایح سلامتی فارغ شد، قوم را به اسم یهوه صبایوت برکت داد.
\par 19 و به تمامی قوم، یعنی به جمیع گروه اسرائیل، مردان و زنان به هر یکی یک گرده نان و یک پاره گوشت و یک قرص کشمش بخشید، پس تمامی قوم هر یکی به خانه خودرفتند.
\par 20 اما داود برگشت تا اهل خانه خود را برکت دهد. و میکال دختر شاول به استقبال داود بیرون آمده، گفت: «پادشاه اسرائیل امروز چه قدرخویشتن را عظمت داد که خود را در نظر کنیزان بندگان خود برهنه ساخت، به طوری که یکی ازسفها خود را برهنه می‌کند.»
\par 21 و داود به میکال گفت: «به حضور خداوند بود که مرا بر پدرت و برتمامی خاندانش برتری داد تا مرا بر قوم خداوند، یعنی بر اسرائیل پیشوا سازد، از این جهت به حضور خداوند بازی کردم.
\par 22 و از این نیز خود رازیاده حقیر خواهم نمود و در نظر خود پست خواهم شد، لیکن در نظر کنیزانی که درباره آنهاسخن گفتی، معظم خواهم بود.»و میکال دخترشاول را تا روز وفاتش اولاد نشد.
\par 23 و میکال دخترشاول را تا روز وفاتش اولاد نشد.
 
\chapter{7}

\par 1 و واقع شد چون پادشاه در خانه خود نشسته، و خداوند او را از جمیع دشمنانش از هر طرف آرامی داده بود، 
\par 2 که پادشاه به ناتان نبی گفت: «الان مرا می‌بینی که در خانه سرو آزادساکن می‌باشم، و تابوت خدا در میان پرده‌ها ساکن است.»
\par 3 ناتان به پادشاه گفت: «بیا و هر‌آنچه دردلت باشد معمول دار زیرا خداوند با توست.»
\par 4 و در آن شب واقع شد که کلام خداوند به ناتان نازل شده، گفت:
\par 5 «برو و به بنده من داودبگو، خداوند چنین می‌گوید: آیا تو خانه‌ای برای سکونت من بنا می‌کنی؟
\par 6 زیرا از روزی که بنی‌اسرائیل را از مصر بیرون آوردم تا امروز، درخانه‌ای ساکن نشده‌ام بلکه در خیمه و مسکن گردش کرده‌ام.
\par 7 و به هر جایی که با جمیع بنی‌اسرائیل گردش کردم آیا به احدی از داوران اسرائیل که برای رعایت قوم خود، اسرائیل، مامور داشتم، سخنی گفتم که چرا خانه‌ای از سروآزاد برای من بنا نکردید؟
\par 8 و حال به بنده من، داود چنین بگو که یهوه صبایوت چنین می‌گوید: من تو را از چراگاه از عقب گوسفندان گرفتم تاپیشوای قوم من، اسرائیل، باشی.
\par 9 و هر جایی که می‌رفتی من با تو می‌بودم و جمیع دشمنانت را ازحضور تو منقطع ساختم، و برای تو اسم بزرگ مثل اسم بزرگانی که بر زمینند، پیدا کردم.
\par 10 و به جهت قوم خود، اسرائیل، مکانی تعیین کردم وایشان را غرس نمودم تا در مکان خویش ساکن شده، باز متحرک نشوند، و شریران، دیگر ایشان را مثل سابق ذلیل نسازند.
\par 11 و مثل روزهایی که داوران را بر قوم خود، اسرائیل، تعیین نموده بودم و تو را از جمیع دشمنانت آرامی دادم، و خداوندتو را خبر می‌دهد که خداوند برای تو خانه‌ای بناخواهد نمود.
\par 12 زیرا روزهای تو تمام خواهدشد و با پدران خود خواهی خوابید و ذریت تو راکه از صلب تو بیرون آید، بعد از تو استوار خواهم ساخت، و سلطنت او را پایدار خواهم نمود.
\par 13 او برای اسم من خانه‌ای بنا خواهد نمود وکرسی سلطنت او را تا به ابد پایدار خواهم ساخت.
\par 14 من او را پدر خواهم بود و او مرا پسرخواهد بود، و اگر او گناه ورزد، او را با عصای مردمان و به تازیانه های بنی آدم تادیب خواهم نمود.
\par 15 ولیکن رحمت من از او دور نخواهد شد، به طوری که آن را از شاول دور کردم که او را ازحضور تو رد ساختم.
\par 16 و خانه و سلطنت تو، به حضورت تا به ابد پایدار خواهد شد، و کرسی توتا به ابد استوار خواهد ماند.»
\par 17 بر‌حسب تمامی این کلمات و مطابق تمامی این‌رویا ناتان به داودتکلم نمود.
\par 18 و داود پادشاه داخل شده، به حضورخداوند نشست و گفت: «ای خداوند یهوه، من کیستم و خاندان من چیست که مرا به این مقام رسانیدی؟
\par 19 و این نیز در نظر تو‌ای خداوند یهوه امر قلیل نمود زیرا که درباره خانه بنده ات نیزبرای زمان طویل تکلم فرمودی. و آیا این‌ای خداوند یهوه عادت بنی آدم است؟
\par 20 و داوددیگر به تو چه تواند گفت زیرا که تو‌ای خداوندیهوه، بنده خود را می‌شناسی،
\par 21 و بر‌حسب کلام خود و موافق دل خود تمامی این کارهای عظیم را بجا آوردی تا بنده خود را تعلیم دهی.
\par 22 بنابرین‌ای یهوه خدا، تو بزرگ هستی زیراچنانکه به گوشهای خود شنیده‌ایم مثل تو کسی نیست و غیر از تو خدایی نیست.
\par 23 و مثل قوم تواسرائیل کدام‌یک امت بر روی زمین است که خدابیاید تا ایشان را فدیه داده، برای خویش قوم بسازد، و اسمی برای خود پیدا نماید، و چیزهای عظیم و مهیب برای شما و برای زمین خود بجاآورد به حضور قوم خویش که برای خود از مصرو از امتها و خدایان ایشان فدیه دادی.
\par 24 و قوم خود اسرائیل را برای خود استوار ساختی، تاایشان تا به ابد قوم تو باشند، و تو‌ای یهوه، خدای ایشان شدی.
\par 25 و الان‌ای یهوه خدا، کلامی را که درباره بنده خود و خانه‌اش گفتی تا به ابد استوارکن، و بر‌حسب آنچه گفتی، عمل نما.
\par 26 و اسم توتا به ابد معظم بماند، تا گفته شود که یهوه صبایوت، خدای اسرائیل است، و خاندان بنده ات داود به حضور تو پایدار بماند.
\par 27 زیرا توای یهوه صبایوت، خدای اسرائیل، به بنده خوداعلان نموده، گفتی که برای تو خانه‌ای بنا خواهم نموده، بنابرین بنده تو جرات کرده است که این دعا را نزد تو بگوید.
\par 28 و الان‌ای خداوند یهوه توخدا هستی و کلام تو صدق است و این نیکویی رابه بنده خود وعده داده‌ای.و الان احسان فرموده، خاندان بنده خود را برکت بده تا آنکه درحضورت تا به ابد بماند، زیرا که تو‌ای خداوندیهوه گفته‌ای و خاندان بنده ات از برکت تو تا به ابدمبارک خواهد بود.»
\par 29 و الان احسان فرموده، خاندان بنده خود را برکت بده تا آنکه درحضورت تا به ابد بماند، زیرا که تو‌ای خداوندیهوه گفته‌ای و خاندان بنده ات از برکت تو تا به ابدمبارک خواهد بود.»
 
\chapter{8}

\par 1 و بعد از این واقع شد که داود فلسطینیان راشکست داده، ایشان را ذلیل ساخت. وداود زمام‌ام البلاد را از دست فلسطینیان گرفت.
\par 2 و موآب را شکست داده، ایشان را به زمین خوابانیده، با ریسمانی پیمود و دو ریسمان برای کشتن پیمود، و یک ریسمان تمام برای زنده نگاه داشتن. و موآبیان بندگان داود شده، هدایا آوردند.
\par 3 و داود، هددعزر بن رحوب، پادشاه صوبه راهنگامی که می‌رفت تا استیلای خود را نزد نهر بازبه‌دست آورد، شکست داد.
\par 4 و داود هزار وهفتصد سوار و بیست هزار پیاده از او گرفت، وداود جمیع اسبهای ارابه هایش را پی کرد، اما ازآنها برای صد ارابه نگاه داشت.
\par 5 و چون ارامیان دمشق به مدد هددعزر، پادشاه صوبه، آمدند، داود بیست و دو هزار نفر از ارامیان را بکشت.
\par 6 وداود در ارام دمشق قراولان گذاشت، و ارامیان، بندگان داود شده، هدایا می‌آوردند، و خداوند، داود را در هر جا که می‌رفت، نصرت می‌داد.
\par 7 وداود سپرهای طلا را که بر خادمان هددعزر بودگرفته، آنها را به اورشلیم آورد.
\par 8 و از باته وبیروتای شهرهای هددعزر داود پادشاه، برنج ازحد افزون گرفت.
\par 9 و چون توعی، پادشاه حمات شنید که داودتمامی لشکر هددعزر را شکست داده است،
\par 10 توعی، یورام، پسر خود را نزد داود پادشاه فرستاد تا از سلامتی او بپرسد، و او را تهنیت گوید، از آن جهت که با هددعزر جنگ نموده، اورا شکست داده بود، زیرا که هددعزر با توعی مقاتله می‌نمود و یورام ظروف نقره و ظروف طلاو ظروف برنجین با خود آورد.
\par 11 و داود پادشاه آنها را نیز برای خداوند وقف نمود با نقره وطلایی که از جمیع امت هایی که شکست داده بود، وقف نموده بود،
\par 12 یعنی از ارام و موآب وبنی عمون و فلسطینیان و عمالقه و از غنیمت هددعزر بن رحوب پادشاه صوبه.
\par 13 و داود برای خویشتن تذکره‌ای برپا نمودهنگامی که از شکست دادن هجده هزار نفر از ارامیان در وادی ملح مراجعت نمود.
\par 14 و در ادوم قراولان گذاشت، بلکه در تمامی ادوم قراولان گذاشته، جمیع ادومیان بندگان داود شدند، وخداوند، داود را هر جا که می‌رفت، نصرت می‌داد.
\par 15 و داود بر تمامی اسرائیل سلطنت می‌نمود، و داود بر تمامی قوم خود داوری و انصاف را اجرامی داشت.
\par 16 و یوآب بن صرویه سردار لشکر بودو یهوشافات بن اخیلود وقایع نگار.
\par 17 و صادوق بن اخیطوب و اخیملک بن ابیاتار، کاهن بودند وسرایا کاتب بود.و بنایاهو بن یهویاداع برکریتیان و فلیتیان بود و پسران داود کاهن بودند.
\par 18 و بنایاهو بن یهویاداع برکریتیان و فلیتیان بود و پسران داود کاهن بودند.
 
\chapter{9}

\par 1 و داود گفت: «آیا از خاندان شاول کسی تابه حال باقی است تا به‌خاطر یوناتان او رااحسان نمایم؟»
\par 2 و از خاندان شاول خادمی مسمی به صیبا بود، پس او را نزد داود خواندند وپادشاه وی را گفت: «آیا تو صیبا هستی؟» گفت: «بنده تو هستم.»
\par 3 پادشاه گفت: «آیا تا به حال ازخاندان شاول کسی هست تا او را احسان خدایی نمایم؟» صیبا در جواب پادشاه گفت: «یوناتان راتا به حال پسری لنگ باقی است.»
\par 4 پادشاه از وی پرسید که «او کجاست؟» صیبا به پادشاه گفت: «اینک او در خانه ماکیر بن عمیئیل در لودباراست.»
\par 5 و داود پادشاه فرستاده، او را از خانه ماکیر بن عمیئیل از لودبار گرفت.
\par 6 پس مفیبوشت بن یوناتان بن شاول نزد داودآمده، به روی در‌افتاده، تعظیم نمود. و داود گفت: «ای مفیبوشت!» گفت: «اینک بنده تو.»
\par 7 داود وی را گفت: «مترس! زیرا به‌خاطر پدرت یوناتان برتو البته احسان خواهم نمود و تمامی زمین پدرت شاول را به تو رد خواهم کرد، و تو دائم بر سفره من نان خواهی خورد.»
\par 8 پس او تعظیم کرده، گفت که «بنده تو چیست که بر سگ مرده‌ای مثل من التفات نمایی؟»
\par 9 و پادشاه، صیبا، بنده شاول را خوانده، گفت: «آنچه را که مال شاول و تمام خاندانش بود به پسرآقای تو دادم.
\par 10 و تو و پسرانت و بندگانت به جهت او زمین را زرع نموده، محصول آن رابیاورید تا برای پسر آقایت به جهت خوردنش نان باشد، اما مفیبوشت، پسر آقایت همیشه بر سفره من نان خواهد خورد.» و صیبا پانزده پسر و بیست خادم داشت.
\par 11 و صیبا به پادشاه گفت: «موافق هر‌آنچه آقایم پادشاه به بنده‌اش فرموده است بهمین طور بنده ات عمل خواهد نمود.» و پادشاه گفت که مفیبوشت بر سفره من مثل یکی از پسران پادشاه خواهد خورد.
\par 12 و مفیبوشت را پسری کوچک بود که میکا نام داشت، و تمامی ساکنان خانه صیبا بنده مفیبوشت بودند.پس مفیبوشت در اورشلیم ساکن شد زیرا که همیشه بر سفره پادشاه می‌خورد و از هر دو پا لنگ بود.
\par 13 پس مفیبوشت در اورشلیم ساکن شد زیرا که همیشه بر سفره پادشاه می‌خورد و از هر دو پا لنگ بود.
 
\chapter{10}

\par 1 و بعد از آن واقع شد که پادشاه بنی عمون، مرد و پسرش، حانون، درجایش سلطنت نمود.
\par 2 و داود گفت: «به حانون بن ناحاش احسان نمایم چنانکه پدرش به من احسان کرد.» پس داود فرستاد تا او را به واسطه خادمانش درباره پدرش تعزیت گوید، و خادمان داود به زمین بنی عمون آمدند.
\par 3 و سروران بنی عمون به آقای خود حانون گفتند: «آیا گمان می‌بری که برای تکریم پدر توست که داود، رسولان به جهت تعزیت تو فرستاده است، آیاداود خادمان خود را نزد تو نفرستاده است تا شهررا تفحص و تجسس نموده، آن را منهدم سازد؟»
\par 4 پس حانون، خادمان داود را گرفت و نصف ریش ایشان را تراشید و لباسهای ایشان را از میان تا جای نشستن بدرید و ایشان را رها کرد.
\par 5 وچون داود را خبر دادند، به استقبال ایشان فرستادزیرا که ایشان بسیار خجل بودند، و پادشاه گفت: «در اریحا بمانید تا ریشهای شما درآید و بعد ازآن برگردید.»
\par 6 و چون بنی عمون دیدند که نزد داود مکروه شدند، بنی عمون فرستاده، بیست هزار پیاده ازارامیان بیت رحوب و ارامیان صوبه و پادشاه معکه را با هزار نفر و دوازده هزار نفر از مردان طوب اجیر کردند.
\par 7 و چون داود شنید، یوآب وتمامی لشکر شجاعان را فرستاد.
\par 8 و بنی عمون بیرون آمده، نزد دهنه دروازه برای جنگ صف آرایی نمودند، و ارامیان صوبه و رحوب ومردان طوب و معکه در صحرا علیحده بودند.
\par 9 و چون یوآب دید که روی صفوف جنگ، هم از پیش و هم از عقبش بود، از تمام برگزیدگان اسرائیل گروهی را انتخاب کرده، در مقابل ارامیان صف آرایی نمود.
\par 10 و بقیه قوم را به‌دست برادرش ابیشای سپرد تا ایشان را به مقابل بنی عمون صف آرایی کند.
\par 11 و گفت: «اگرارامیان بر من غالب آیند، به مدد من بیا، و اگربنی عمون بر تو غالب آیند، به جهت امداد تو خواهم آمد.
\par 12 دلیر باش و به جهت قوم خویش و به جهت شهرهای خدای خود مردانه بکوشیم، و خداوند آنچه را که در نظرش پسند آید بکند.»
\par 13 پس یوآب و قومی که همراهش بودند نزدیک شدند تا با ارامیان جنگ کنند و ایشان از حضوروی فرار کردند.
\par 14 و چون بنی عمون دیدند که ارامیان فرار کردند، ایشان نیز از حضور ابیشای گریخته، داخل شهر شدند و یوآب از مقابله بنی عمون برگشته، به اورشلیم آمد.
\par 15 و چون ارامیان دیدند که از حضور اسرائیل شکست یافته‌اند، با هم جمع شدند.
\par 16 و هددعزرفرستاده، ارامیان را که به آن طرف نهر بودند، آوردو ایشان به حیلام آمدند، و شوبک، سردار لشکرهددعزر، پیشوای ایشان بود.
\par 17 و چون به داودخبر رسید، جمیع اسرائیل را جمع کرده، از اردن عبور کرد و به حیلام آمد، و ارامیان به مقابل داودصف آرایی نموده، با او جنگ کردند. 
\par 18 و ارامیان از حضور اسرائیل فرار کردند، و داود از ارامیان، مردان هفتصد ارابه و چهل هزار سوار را کشت وشوبک سردار لشکرش را زد که در آنجا مرد.وچون جمیع پادشاهانی که بنده هددعزر بودند، دیدند که از حضور اسرائیل شکست خوردند، بااسرائیل صلح نموده، بنده ایشان شدند. و ارامیان پس از آن از امداد بنی عمون ترسیدند.
\par 19 وچون جمیع پادشاهانی که بنده هددعزر بودند، دیدند که از حضور اسرائیل شکست خوردند، بااسرائیل صلح نموده، بنده ایشان شدند. و ارامیان پس از آن از امداد بنی عمون ترسیدند.
 
\chapter{11}

\par 1 و واقع شد بعد از انقضای سال، هنگام بیرون رفتن پادشاهان که داود، یوآب رابا بندگان خویش و تمامی اسرائیل فرستاد، وایشان بنی عمون را خراب کرده، ربه را محاصره نمودند، اما داود در اورشلیم ماند.
\par 2 و واقع شد در وقت عصر که داود از بسترش برخاسته، بر پشت بام خانه پادشاه گردش کرد و ازپشت بام زنی را دید که خویشتن را شستشومی کند و آن زن بسیار نیکومنظر بود.
\par 3 پس داودفرستاده، درباره زن استفسار نمود و او را گفتند که «آیا این بتشبع، دختر الیعام، زن اوریای حتی نیست؟»
\par 4 و داود قاصدان فرستاده، او را گرفت واو نزد وی آمده، داود با او همبستر شد و او ازنجاست خود طاهر شده، به خانه خود برگشت.
\par 5 و آن زن حامله شد و فرستاده، داود را مخبرساخت و گفت که من حامله هستم.
\par 6 پس داود نزد یوآب فرستاد که اوریای حتی را نزد من بفرست و یوآب، اوریا را نزد داودفرستاد.
\par 7 و چون اوریا نزد وی رسید، داود ازسلامتی یوآب و از سلامتی قوم و از سلامتی جنگ پرسید.
\par 8 و داود به اوریا گفت: «به خانه ات برو و پایهای خود را بشو.» پس اوریا از خانه پادشاه بیرون رفت و از عقبش، خوانی از پادشاه فرستاده شد.
\par 9 اما اوریا نزد در خانه پادشاه با سایربندگان آقایش خوابیده، به خانه خود نرفت.
\par 10 وداود را خبر داده، گفتند که «اوریا به خانه خودنرفته است.» پس داود به اوریا گفت: «آیا تو ازسفر نیامده‌ای، پس چرا به خانه خود نرفته‌ای؟»
\par 11 اوریا به داود عرض کرد که «تابوت و اسرائیل و یهودا در خیمه‌ها ساکنند و آقایم، یوآب، وبندگان آقایم بر روی بیابان خیمه نشینند و آیا من به خانه خود بروم تا اکل و شرب بنمایم و با زن خود بخوابم؟ به حیات تو و به حیات جان تو قسم که این کار را نخواهم کرد.»
\par 12 و داود به اوریا گفت: «امروز نیز اینجا باش و فردا تو را روانه می‌کنم.» پس اوریا آن روز و فردایش را دراورشلیم ماند.
\par 13 و داود او را دعوت نمود که درحضورش خورد و نوشید و او را مست کرد، ووقت شام بیرون رفته، بر بسترش با بندگان آقایش خوابید و به خانه خود نرفت.
\par 14 و بامدادان داود مکتوبی برای یوآب نوشته، به‌دست اوریا فرستاد.
\par 15 و در مکتوب به این مضمون نوشت که «اوریا را در مقدمه جنگ سخت بگذارید، و از عقبش پس بروید تا زده شده، بمیرد.»
\par 16 و چون یوآب شهر را محاصره می‌کرد اوریا را در مکانی که می‌دانست که مردان شجاع در آنجا می‌باشند، گذاشت.
\par 17 و مردان شهر بیرون آمده، با یوآب جنگ کردند و بعضی از قوم، از بندگان داود، افتادند و اوریای حتی نیزبمرد.
\par 18 پس یوآب فرستاده، داود را از جمیع وقایع جنگ خبر داد.
\par 19 و قاصد را امر فرموده، گفت: «چون از تمامی وقایع جنگ به پادشاه خبرداده باشی،
\par 20 اگر خشم پادشاه افروخته شود وتو را گوید چرا برای جنگ به شهر نزدیک شدید، آیا نمی دانستید که از سر حصار، تیر خواهندانداخت؟
\par 21 کیست که ابیملک بن یربوشت راکشت؟ آیا زنی سنگ بالایین آسیایی را از روی حصار بر او نینداخت که در تاباص مرد؟ پس چرابه حصار نزدیک شدید؟ آنگاه بگو که «بنده ات، اوریای حتی نیز مرده است.»
\par 22 پس قاصد روانه شده، آمد و داود را از هرآنچه یوآب او را پیغام داده بود، مخبر ساخت.
\par 23 و قاصد به داود گفت که «مردان بر ما غالب شده، در عقب ما به صحرا بیرون آمدند، و ما بر ایشان تا دهنه دروازه تاختیم.
\par 24 و تیراندازان بربندگان تو از روی حصار تیر انداختند، و بعضی ازبندگان پادشاه مردند و بنده تو اوریای حتی نیزمرده است.»
\par 25 داود به قاصد گفت: «به یوآب چنین بگو: این واقعه در نظر تو بد نیاید زیرا که شمشیر، این و آن را بی‌تفاوت هلاک می‌کند. پس در مقاتله با شهر به سختی کوشیده، آن را منهدم بساز. پس او را خاطر جمعی بده.»
\par 26 و چون زن اوریا شنید که شوهرش اوریامرده است، برای شوهر خود ماتم گرفت.وچون ایام ماتم گذشت، داود فرستاده، او را به خانه خود آورد و او زن وی شد، و برایش پسری زایید، اما کاری که داود کرده بود، در نظر خداوندناپسند آمد.
\par 27 وچون ایام ماتم گذشت، داود فرستاده، او را به خانه خود آورد و او زن وی شد، و برایش پسری زایید، اما کاری که داود کرده بود، در نظر خداوندناپسند آمد.
 
\chapter{12}

\par 1 و خداوند ناتان را نزد داود فرستاد و نزدوی آمده، او را گفت که «در شهری دومرد بودند، یکی دولتمند و دیگری فقیر.
\par 2 ودولتمند را گوسفند و گاو، بی‌نهایت بسیار بود.
\par 3 وفقیر را جز یک ماده بره کوچک نبود که آن راخریده، و پرورش داده، همراه وی و پسرانش بزرگ می‌شد، از خوراک وی می‌خورد و از کاسه او می‌نوشید و در آغوشش می‌خوابید و برایش مثل دختر می‌بود.
\par 4 و مسافری نزد آن مرددولتمند آمد و او را حیف آمد که از گوسفندان وگاوان خود بگیرد تا به جهت مسافری که نزد وی آمده بود مهیا سازد، و بره آن مرد فقیر را گرفته، برای آن مرد که نزد وی آمده بود، مهیا ساخت.»
\par 5 آنگاه خشم داود بر آن شخص افروخته شده، به ناتان گفت: «به حیات خداوند قسم، کسی‌که این کار را کرده است، مستوجب قتل است.
\par 6 وچونکه این کار را کرده است و هیچ ترحم ننموده، بره را چهار چندان باید رد کند.»
\par 7 ناتان به داود گفت: «آن مرد تو هستی، ویهوه، خدای اسرائیل، چنین می‌گوید: من تو را براسرائیل به پادشاهی مسح نمودم و من تو را ازدست شاول رهایی دادم.
\par 8 و خانه آقایت را به تودادم و زنان آقای تو را به آغوش تو، و خاندان اسرائیل و یهودا را به تو عطا کردم. و اگر این کم می‌بود، چنین و چنان برای تو مزید می‌کردم.
\par 9 پس چرا کلام خداوند را خوار نموده، در نظروی عمل بد بجا آوردی و اوریای حتی را به شمشیر زده، زن او را برای خود به زنی گرفتی، واو را با شمشیر بنی عمون به قتل رسانیدی.
\par 10 پس حال شمشیر از خانه تو هرگز دور نخواهد شد به علت اینکه مرا تحقیر نموده، زن اوریای حتی راگرفتی تا زن تو باشد.
\par 11 خداوند چنین می‌گوید: اینک من از خانه خودت بدی را بر تو عارض خواهم گردانید و زنان تو را پیش چشم تو گرفته، به همسایه ات خواهم داد، و او در نظر این آفتاب، با زنان تو خواهد خوابید.
\par 12 زیرا که تو این کار رابه پنهانی کردی، اما من این کار را پیش تمام اسرائیل و در نظر آفتاب خواهم نمود.»
\par 13 و داودبا ناتان گفت: «به خداوند گناه کرده‌ام.» ناتان به داود گفت: «خداوند نیز گناه تو را عفو نموده است که نخواهی مرد.
\par 14 لیکن چون از این امرباعث کفر گفتن دشمنان خداوند شده‌ای، پسری نیز که برای تو زاییده شده است، البته خواهدمرد.»
\par 15 پس ناتان به خانه خود رفت.
\par 16 پس داود از خدا برای طفل استدعا نمود وداود روزه گرفت و داخل شده، تمامی شب برروی زمین خوابید.
\par 17 و مشایخ خانه‌اش بر اوبرخاستند تا او را از زمین برخیزانند، اما قبول نکرد و با ایشان نان نخورد.
\par 18 و در روز هفتم طفل بمرد و خادمان داود ترسیدند که از مردن طفل اورا اطلاع دهند، زیرا گفتند: «اینک چون طفل زنده بود، با وی سخن گفتیم و قول ما را نشنید، پس اگربه او خبر دهیم که طفل مرده است، چه قدر زیاده رنجیده می‌شود.»
\par 19 و چون داود دید که بندگانش با یکدیگر نجوی می‌کنند، داود فهمیدکه طفل مرده است، و داود به خادمان خود گفت: «آیا طفل مرده است؟» گفتند: «مرده است.»
\par 20 آنگاه داود از زمین برخاسته، خویشتن راشست و شو داده، تدهین کرد و لباس خود راعوض نموده، به خانه خداوند رفت و عبادت نمود و به خانه خود آمده، خوراک خواست که پیشش گذاشتند و خورد.
\par 21 و خادمانش به وی گفتند: «این چه‌کار است که کردی؟ وقتی که طفل زنده بود روزه گرفته، گریه نمودی و چون طفل مرد، برخاسته، خوراک خوردی؟»
\par 22 او گفت: «وقتی که طفل زنده بود روزه گرفتم و گریه نمودم زیرا فکر کردم کیست که بداند که شاید خداوند برمن ترحم فرماید تا طفل زنده بماند،
\par 23 اما الان که مرده است، پس چرا من روزه بدارم؛ آیا می‌توانم دیگر او را باز بیاورم؟ ! من نزد او خواهم رفت لیکن او نزد من باز نخواهد آمد.»
\par 24 و داود زن خود بتشبع را تسلی داد و نزدوی درآمده، با او خوابید و او پسری زاییده، او راسلیمان نام نهاد. و خداوند او را دوست داشت.
\par 25 و به‌دست ناتان نبی فرستاد و او را به‌خاطرخداوند یدیدیا نام نهاد.
\par 26 و یوآب با ربه بنی عمون جنگ کرده، شهرپادشاه نشین را گرفت.
\par 27 و یوآب قاصدان نزدداود فرستاده، گفت که «با ربه جنگ کردم و شهرآبها را گرفتم.
\par 28 پس الان بقیه قوم را جمع کن ودر برابر شهر اردو زده، آن را بگیر، مبادا من شهررا بگیرم و به اسم من نامیده شود.»
\par 29 پس داودتمامی قوم را جمع کرده، به ربه رفت و با آن جنگ کرده، آن را گرفت.
\par 30 و تاج پادشاه ایشان را ازسرش گرفت که وزنش یک وزنه طلا بود وسنگهای گرانبها داشت و آن را بر سر داودگذاشتند، و غنیمت از حد زیاده از شهر بردند.و خلق آنجا را بیرون آورده، ایشان را زیر اره هاو چومهای آهنین و تیشه های آهنین گذاشت وایشان را از کوره آجرپزی گذرانید، و به همین طور با جمیع شهرهای بنی عمون رفتار نمود. پس داود و تمامی قوم به اورشلیم برگشتند.
\par 31 و خلق آنجا را بیرون آورده، ایشان را زیر اره هاو چومهای آهنین و تیشه های آهنین گذاشت وایشان را از کوره آجرپزی گذرانید، و به همین طور با جمیع شهرهای بنی عمون رفتار نمود. پس داود و تمامی قوم به اورشلیم برگشتند.
 
\chapter{13}

\par 1 و بعد از این، واقع شد که ابشالوم بن داود را خواهری نیکو صورت مسمی به تامار بود، و امنون، پسر داود، او را دوست می‌داشت.
\par 2 و امنون به‌سبب خواهر خود تامارچنان گرفتار شد که بیمار گشت، زیرا که او باکره بود و به نظر امنون دشوار آمد که با وی کاری کند.
\par 3 و امنون رفیقی داشت که مسمی به یوناداب بن شمعی، برادر داود، بود، و یوناداب مردی بسیارزیرک بود.
\par 4 و او وی را گفت: «ای پسر پادشاه چرا روز به روز چنین لاغر می‌شوی و مرا خبرنمی دهی؟» امنون وی را گفت که «من تامار، خواهر برادر خود، ابشالوم را دوست می‌دارم.»
\par 5 و یوناداب وی را گفت: «بر بستر خود خوابیده، تمارض نما و چون پدرت برای عیادت تو بیاید، وی را بگو: تمنا این که خواهر من تامار بیاید و مراخوراک بخوراند و خوراک را در نظر من حاضرسازد تا ببینم و از دست وی بخورم.»
\par 6 پس امنون خوابید و تمارض نمود و چون پادشاه به عیادتش آمد، امنون به پادشاه گفت: «تمنا اینکه خواهرم تامار بیاید و دو قرص طعام پیش من بپزد تا ازدست او بخورم.»
\par 7 و داود نزد تامار به خانه‌اش فرستاده، گفت: «الان به خانه برادرت امنون برو و برایش طعام بساز.»
\par 8 و تامار به خانه برادر خود، امنون، رفت. واو خوابیده بود. و آرد گرفته، خمیر کرد، و پیش اوقرصها ساخته، آنها را پخت.
\par 9 و تابه را گرفته، آنها را پیش او ریخت. اما از خوردن ابا نمود وگفت: «همه کس را از نزد من بیرون کنید.» وهمگان از نزد او بیرون رفتند.
\par 10 و امنون به تامارگفت: «خوراک را به اطاق بیاور تا از دست توبخورم.» و تامار قرصها را که ساخته بود، گرفته، نزد برادر خود، امنون، به اطاق آورد.
\par 11 و چون پیش او گذاشت تا بخورد، او وی را گرفته، به اوگفت: «ای خواهرم بیا با من بخواب.»
\par 12 او وی راگفت: «نی‌ای برادرم، مرا ذلیل نساز زیرا که چنین کار در اسرائیل کرده نشود، این قباحت را به عمل میاور. 
\par 13 اما من کجا ننگ خود را ببرم و اما تو مثل یکی از سفها در اسرائیل خواهی شد، پس حال تمنا اینکه به پادشاه بگویی، زیرا که مرا از تو دریغ نخواهد نمود.»
\par 14 لیکن او نخواست سخن وی رابشنود، و بر او زورآور شده، او را مجبور ساخت و با او خوابید.
\par 15 آنگاه امنون با شدت بر وی بغض نمود، وبغضی که با او ورزید از محبتی که با وی می داشت، زیاده بود، پس امنون وی را گفت: «برخیز و برو.»
\par 16 او وی را گفت: «چنین مکن. زیرا این ظلم عظیم که در بیرون کردن من می‌کنی، بدتر است از آن دیگری که با من کردی.» لیکن اونخواست که وی را بشنود.
\par 17 پس خادمی را که اورا خدمت می‌کرد خوانده، گفت: «این دختر را ازنزد من بیرون کن و در را از عقبش ببند.»
\par 18 و اوجامه رنگارنگ دربر داشت زیرا که دختران باکره پادشاه به این‌گونه لباس، ملبس می‌شدند. وخادمش او را بیرون کرده، در را از عقبش بست.
\par 19 و تامار خاکستر بر سر خود ریخته، و جامه رنگارنگ که در برش بود، دریده، و دست خود رابر سر گذارده، روانه شد. و چون می‌رفت، فریادمی نمود.
\par 20 و برادرش، ابشالوم، وی را گفت: «که آیابرادرت، امنون، با تو بوده است؟ پس‌ای خواهرم اکنون خاموش باش. او برادر توست و از این کارمتفکر مباش.» پس تامار در خانه برادر خود، ابشالوم، در پریشان حالی ماند.
\par 21 و چون داودپادشاه تمامی این وقایع را شنید، بسیار غضبناک شد.
\par 22 و ابشالوم به امنون سخنی نیک یا بدنگفت، زیرا که ابشالوم امنون را بغض می‌داشت، به علت اینکه خواهرش تامار را ذلیل ساخته بود.
\par 23 و بعد از دو سال تمام، واقع شد که ابشالوم در بعل حاصور که نزد افرایم است، پشم برندگان داشت. و ابشالوم تمامی پسران پادشاه را دعوت نمود.
\par 24 و ابشالوم نزد پادشاه آمده، گفت: «اینک حال، بنده تو، پشم برندگان دارد. تمنا اینکه پادشاه با خادمان خود همراه بنده ات بیایند.»
\par 25 پادشاه به ابشالوم گفت: «نی‌ای پسرم، همه ما نخواهیم آمد مبادا برای تو بار سنگین باشیم.» وهر‌چند او را الحاح نمود لیکن نخواست که بیایدو او را برکت داد.
\par 26 و ابشالوم گفت: «پس تمنااینکه برادرم، امنون، با ما بیاید.» پادشاه او را گفت: «چرا با تو بیاید؟»
\par 27 اما چون ابشالوم او را الحاح نمود، امنون و تمامی پسران پادشاه را با او روانه کرد.
\par 28 و ابشالوم خادمان خود را امر فرموده، گفت: «ملاحظه کنید که چون دل امنون از شراب خوش شود، و به شما بگویم که امنون را بزنید. آنگاه او را بکشید، و مترسید، آیا من شما را امرنفرمودم. پس دلیر و شجاع باشید.»
\par 29 و خادمان ابشالوم با امنون به طوری که ابشالوم امر فرموده بود، به عمل آوردند، و جمیع پسران پادشاه برخاسته، هر کس به قاطر خود سوار شده، گریختند.
\par 30 و چون ایشان در راه می‌بودند، خبر به داودرسانیده، گفتند که «ابشالوم همه پسران پادشاه راکشته و یکی از ایشان باقی نمانده است.»
\par 31 پس پادشاه برخاسته، جامه خود را درید و به روی زمین دراز شد و جمیع بندگانش با جامه دریده دراطرافش ایستاده بودند.
\par 32 اما یوناداب بن شمعی برادر داود متوجه شده، گفت: «آقایم گمان نبرد که جمیع جوانان، یعنی پسران پادشاه کشته شده‌اند، زیرا که امنون تنها مرده است چونکه این، نزدابشالوم مقرر شده بود از روزی که خواهرش تاماررا ذلیل ساخته بود.
\par 33 و الان آقایم، پادشاه از این امر متفکر نشود، و خیال نکند که تمامی پسران پادشاه مرده‌اند زیرا که امنون تنها مرده است.»
\par 34 و ابشالوم گریخت، و جوانی که دیده بانی می‌کرد، چشمان خود را بلند کرده، نگاه کرد و اینک خلق بسیاری از پهلوی کوه که در عقبش بود، می‌آمدند.
\par 35 و یوناداب به پادشاه گفت: «اینک پسران پادشاه می‌آیند، پس به طوری که بنده ات گفت، چنان شد.»
\par 36 و چون از سخن‌گفتن فارغ شد، اینک پسران پادشاه رسیدند و آوازخود را بلند کرده، گریستند، و پادشاه نیز و جمیع خادمانش به آواز بسیار بلند گریه کردند.
\par 37 و ابشالوم فرار کرده، نزد تلمای ابن عمیهود، پادشاه جشور رفت، و داود برای پسرخود هر روز نوحه گری می‌نمود.
\par 38 و ابشالوم فرار کرده، به جشور رفت و سه سال در آنجا ماند.و داود آرزو می‌داشت که نزد ابشالوم بیرون رود، زیرا درباره امنون تسلی یافته بود، چونکه مرده بود.
\par 39 و داود آرزو می‌داشت که نزد ابشالوم بیرون رود، زیرا درباره امنون تسلی یافته بود، چونکه مرده بود.
 
\chapter{14}

\par 1 و یوآب بن صرویه فهمید که دل پادشاه به ابشالوم مایل است.
\par 2 پس یوآب به تقوع فرستاده، زنی دانشمند از آنجا آورد و به وی گفت: «تمنا اینکه خویشتن را مثل ماتم کننده ظاهر سازی، و لباس تعزیت پوشی و خود را به روغن تدهین نکنی و مثل زنی که روزهای بسیار به جهت مرده ماتم گرفته باشد، بشوی.
\par 3 و نزدپادشاه داخل شده، او را بدین مضمون بگویی.» پس یوآب سخنان را به دهانش گذاشت.
\par 4 و چون زن تقوعیه با پادشاه سخن گفت، به روی خود به زمین افتاده، تعظیم نمود و گفت: «ای پادشاه، اعانت فرما.»
\par 5 و پادشاه به او گفت: «تو راچه شده است؟» عرض کرد: «اینک من زن بیوه هستم و شوهرم مرده است.
\par 6 و کنیز تو را دو پسر بود و ایشان با یکدیگر در صحرا مخاصمه نمودند و کسی نبود که ایشان را از یکدیگر جداکند. پس یکی از ایشان دیگری را زد و کشت.
\par 7 واینک تمامی قبیله بر کنیز تو برخاسته، و می‌گویندقاتل برادر خود را بسپار تا او را به عوض جان برادرش که کشته شده است، به قتل برسانیم، ووارث را نیز هلاک کنیم. و به اینطور اخگر مرا که باقی‌مانده است، خاموش خواهند کرد، و برای شوهرم نه اسم و نه اعقاب بر روی زمین واخواهند گذاشت.»
\par 8 پادشاه به زن فرمود: «به خانه ات برو و من درباره ات حکم خواهم نمود.»
\par 9 و زن تقوعیه به پادشاه عرض کرد: «ای آقایم پادشاه، تقصیر بر من و بر خاندان من باشد و پادشاه و کرسی اوبی تقصیر باشند.»
\par 10 و پادشاه گفت: «هر‌که با توسخن گوید، او را نزد من بیاور، و دیگر به تو ضررنخواهد رسانید.»
\par 11 پس زن گفت: «ای پادشاه، یهوه، خدای خود را به یاد آور تا ولی مقتول، دیگر هلاک نکند، مبادا پسر مرا تلف سازند.» پادشاه گفت: «به حیات خداوند قسم که مویی ازسر پسرت به زمین نخواهد افتاد.»
\par 12 پس زن گفت: «مستدعی آنکه کنیزت باآقای خود پادشاه سخنی گوید.» گفت: «بگو.»
\par 13 زن گفت: «پس چرا درباره قوم خدا مثل این تدبیر کرده‌ای و پادشاه در گفتن این سخن مثل تقصیرکار است، چونکه پادشاه آواره شده خود راباز نیاورده است.
\par 14 زیرا ما باید البته بمیریم ومثل آب هستیم که به زمین ریخته شود، و آن رانتوان جمع کرد، و خدا جان را نمی گیرد بلکه تدبیرها می‌کند تا آواره شده‌ای از او آواره نشود.
\par 15 و حال که به قصد عرض کردن این سخن، نزدآقای خود، پادشاه، آمدم، سبب این بود که خلق، مرا ترسانیدند، و کنیزت فکر کرد که چون به پادشاه عرض کنم، احتمال دارد که پادشاه عرض کنیز خود را به انجام خواهد رسانید.
\par 16 زیراپادشاه اجابت خواهد نمود که کنیز خود را ازدست کسی‌که می‌خواهد مرا و پسرم را با هم ازمیراث خدا هلاک سازد، برهاند.
\par 17 و کنیز تو فکرکرد که کلام آقایم، پادشاه، باعث تسلی خواهدبود، زیرا که آقایم، پادشاه، مثل فرشته خداست تانیک و بد را تشخیص کند، و یهوه، خدای توهمراه تو باشد.»
\par 18 پس پادشاه در جواب زن فرمود: «چیزی راکه از تو سوال می‌کنم، از من مخفی مدار.» زن عرض کرد «آقایم پادشاه، بفرماید.»
\par 19 پادشاه گفت: «آیا دست یوآب در همه این کار با تونیست؟» زن در جواب عرض کرد: «به حیات جان تو، ای آقایم پادشاه که هیچ‌کس از هرچه آقایم پادشاه بفرماید به طرف راست یا چپ نمی تواندانحراف ورزد، زیرا که بنده تو یوآب، اوست که مرا امر فرموده است، و اوست که تمامی این سخنان را به دهان کنیزت گذاشته است.
\par 20 برای تبدیل صورت این امر، بنده تو، یوآب، این کار راکرده است، اما حکمت آقایم، مثل حکمت فرشته خدا می‌باشد تا هر‌چه بر روی زمین است، بداند.»
\par 21 پس پادشاه به یوآب گفت: «اینک این کار راکرده‌ام. حال برو و ابشالوم جوان را باز آور.»
\par 22 آنگاه یوآب به روی خود به زمین افتاده، تعظیم نمود، و پادشاه را تحسین کرد و یوآب گفت: «ای آقایم پادشاه امروز بنده ات می‌داند که در نظر تو التفات یافته‌ام چونکه پادشاه کار بنده خود را به انجام رسانیده است.»
\par 23 پس یوآب برخاسته، به جشور رفت و ابشالوم را به اورشلیم بازآورد.
\par 24 و پادشاه فرمود که به خانه خودبرگردد و روی مرا نبیند. پس ابشالوم به خانه خودرفت و روی پادشاه را ندید.
\par 25 و در تمامی اسرائیل کسی نیکو منظر وبسیار ممدوح مثل ابشالوم نبود که از کف پا تا فرق سرش در او عیبی نبود.
\par 26 و هنگامی که موی سرخود را می‌چید، (زیرا آن را در آخر هر سال می‌چید، چونکه بر او سنگین می‌شد و از آن سبب آن را می‌چید) موی سر خود را وزن نموده، دویست مثقال به وزن شاه می‌یافت.
\par 27 و برای ابشالوم سه پسر و یک دختر مسمی به تامارزاییده شدند. و او دختری نیکو صورت بود.
\par 28 و ابشالوم دو سال تمام در اورشلیم مانده، روی پادشاه را ندید.
\par 29 پس ابشالوم، یوآب راطلبید تا او را نزد پادشاه بفرستد. اما نخواست که نزد وی بیاید. و باز بار دیگر فرستاد و نخواست که بیاید.
\par 30 پس به خادمان خود گفت: «ببینید، مزرعه یوآب نزد مزرعه من است و در آنجا جودارد. بروید و آن را به آتش بسوزانید.» پس خادمان ابشالوم مزرعه را به آتش سوزانیدند.
\par 31 آنگاه یوآب برخاسته، نزد ابشالوم به خانه‌اش رفته، وی را گفت که «چرا خادمان تو مزرعه مراآتش زده‌اند؟»
\par 32 ابشالوم به یوآب گفت: «اینک نزد تو فرستاده، گفتم: اینجا بیا تا تو را نزد پادشاه بفرستم تا بگویی برای چه از جشور آمده‌ام؟ مرابهتر می‌بود که تابحال در آنجا مانده باشم، پس حال روی پادشاه را ببینم و اگر گناهی در من باشد، مرا بکشد.»پس یوآب نزد پادشاه رفته، او را مخبر ساخت. و او ابشالوم را طلبید که پیش پادشاه آمد و به حضور پادشاه رو به زمین افتاده، تعظیم کرده و پادشاه، ابشالوم را بوسید.
\par 33 پس یوآب نزد پادشاه رفته، او را مخبر ساخت. و او ابشالوم را طلبید که پیش پادشاه آمد و به حضور پادشاه رو به زمین افتاده، تعظیم کرده و پادشاه، ابشالوم را بوسید.
 
\chapter{15}

\par 1 و بعد از آن، واقع شد که ابشالوم ارابه‌ای و اسبان و پنجاه مرد که پیش اوبدوند، مهیا نمود.
\par 2 و ابشالوم صبح زود برخاسته، به کناره راه دروازه می‌ایستاد، و هر کسی‌که دعوایی می‌داشت و نزد پادشاه به محاکمه می‌آمد، ابشالوم او را خوانده، می‌گفت: «تو ازکدام شهر هستی؟» و او می‌گفت: «بنده ات از فلان سبط از اسباط اسرائیل هستم.»
\par 3 و ابشالوم او رامی گفت: «ببین، کارهای تو نیکو و راست است لیکن از جانب پادشاه کسی نیست که تو رابشنود.»
\par 4 و ابشالوم می‌گفت: «کاش که در زمین داور می‌شدم و هر کس که دعوایی یا مرافعه‌ای می‌داشت، نزد من می‌آمد و برای او انصاف می‌نمودم.»
\par 5 و هنگامی که کسی نزدیک آمده، اورا تعظیم می‌نمود، دست خود را دراز کرده، او رامی گرفت و می‌بوسید.
\par 6 و ابشالوم با همه اسرائیل که نزد پادشاه برای داوری می‌آمدند بدین منوال عمل می‌نمود، پس ابشالوم دل مردان اسرائیل رافریفت.
\par 7 و بعد از انقضای چهل سال، ابشالوم به پادشاه گفت: «مستدعی اینکه بروم تا نذری را که برای خداوند در حبرون کرده‌ام، وفا نمایم،
\par 8 زیراکه بنده ات وقتی که در جشور ارام ساکن بودم، نذرکرده، گفتم که اگر خداوند مرا به اورشلیم بازآورد، خداوند را عبادت خواهم نمود.»
\par 9 پادشاه وی را گفت: «به سلامتی برو.» پس او برخاسته، به حبرون رفت.
\par 10 و ابشالوم، جاسوسان به تمامی اسباط اسرائیل فرستاده، گفت: «به مجرد شنیدن آواز کرنا بگویید که ابشالوم در حبرون پادشاه شده است.»
\par 11 و دویست نفر که دعوت شده بودند، همراه ابشالوم از اورشلیم رفتند، و اینان به صافدلی رفته، چیزی ندانستند. 
\par 12 و ابشالوم اخیتوفل جیلونی را که مشیر داود بود، از شهرش، جیلوه، وقتی که قربانی‌ها می‌گذرانید، طلبید وفتنه سخت شد، و قوم با ابشالوم روزبه روز زیاده می‌شدند.
\par 13 و کسی نزد داود آمده، او را خبر داده، گفت که «دلهای مردان اسرائیل در عقب ابشالوم گرویده است.»
\par 14 و داود به تمامی خادمانی که بااو در اورشلیم بودند، گفت: «برخاسته، فرار کنیم والا ما را از ابشالوم نجات نخواهد بود. پس به تعجیل روانه شویم مبادا او ناگهان به ما برسد وبدی بر ما عارض شود و شهر را به دم شمشیربزند.
\par 15 و خادمان پادشاه، به پادشاه عرض کردند: «اینک بندگانت حاضرند برای هرچه آقای ماپادشاه اختیار کند.»
\par 16 پس پادشاه و تمامی اهل خانه‌اش با وی بیرون رفتند، و پادشاه ده زن را که متعه او بودند، برای نگاه داشتن خانه واگذاشت.
\par 17 و پادشاه و تمامی قوم با وی بیرون رفته، دربیت مرحق توقف نمودند.
\par 18 و تمامی خادمانش پیش او گذشتند و جمیع کریتیان و جمیع فلیتیان و جمیع جتیان، یعنی ششصد نفر که از جت درعقب او آمده بودند، پیش روی پادشاه گذشتند.
\par 19 و پادشاه به اتای جتی گفت: «تو نیز همراه ما چرا می‌آیی؟ برگرد و همراه پادشاه بمان زیراکه تو غریب هستی و از مکان خود نیز جلای وطن کرده‌ای.
\par 20 دیروز آمدی. پس آیا امروز تو را همراه ما آواره گردانم و حال آنکه من می‌روم به‌جایی که می‌روم. پس برگرد و برادران خود رابرگردان و رحمت و راستی همراه تو باد.»
\par 21 واتای در جواب پادشاه عرض کرد: «به حیات خداوند و به حیات آقایم پادشاه، قسم که هرجایی که آقایم پادشاه خواه در موت و خواه در زندگی، باشد، بنده تو در آنجا خواهد بود.»
\par 22 و داود به اتای گفت: «بیا و پیش برو.» پس اتای جتی با همه مردمانش و جمیع اطفالی که با اوبودند، پیش رفتند.
\par 23 و تمامی اهل زمین به آوازبلند گریه کردند، و جمیع قوم عبور کردند، وپادشاه از نهر قدرون عبور کرد و تمامی قوم به راه بیابان گذشتند.
\par 24 و اینک صادوق نیز و جمیع لاویان با وی تابوت عهد خدا را برداشتند، و تابوت خدا رانهادند و تا تمامی قوم از شهر بیرون آمدند، ابیاتارقربانی می‌گذرانید.
\par 25 و پادشاه به صادوق گفت: «تابوت خدا را به شهر برگردان. پس اگر در نظرخداوند التفات یابم مرا باز خواهد آورد، و آن راو مسکن خود را به من نشان خواهد داد.
\par 26 و اگرچنین گوید که از تو راضی نیستم، اینک حاضرم هرچه در نظرش پسند آید، به من عمل نماید.»
\par 27 و پادشاه به صادوق کاهن گفت: «آیا تو رایی نیستی؟ پس به شهر به سلامتی برگرد و هر دو پسرشما، یعنی اخیمعص، پسر تو، و یوناتان، پسرابیاتار، همراه شما باشند.
\par 28 بدانید که من درکناره های بیابان درنگ خواهم نمود تا پیغامی ازشما رسیده، مرا مخبر سازد.»
\par 29 پس صادوق وابیاتار تابوت خدا را به اورشلیم برگردانیده، درآنجا ماندند.
\par 30 و اما داود به فراز کوه زیتون برآمد و چون می‌رفت، گریه می‌کرد و با سر پوشیده و پای برهنه می‌رفت و تمامی قومی که همراهش بودند، هریک سر خود را پوشانیدند و گریه‌کنان می‌رفتند.
\par 31 و داود را خبر داده، گفتند: «که اخیتوفل، یکی از فتنه انگیزان، با ابشالوم شده است. و داود گفت: «ای خداوند، مشورت اخیتوفل را حماقت گردان.»
\par 32 و چون داود به فراز کوه، جایی که خدا راسجده می‌کنند رسید، اینک حوشای ارکی باجامه دریده و خاک بر سر ریخته او را استقبال کرد.
\par 33 و داود وی را گفت: «اگر همراه من بیایی برای من بار خواهی شد.
\par 34 اما اگر به شهربرگردی و به ابشالوم بگویی: ای پادشاه، من بنده تو خواهم بود، چنانکه پیشتر بنده تو بودم، الان بنده تو خواهم بود. آنگاه مشورت اخیتوفل رابرای من باطل خواهی گردانید.
\par 35 و آیا صادوق وابیاتار کهنه در آنجا همراه تو نیستند؟ پس هرچیزی را که از خانه پادشاه بشنوی، آن را به صادوق و ابیاتار کهنه اعلام نما.
\par 36 و اینک دوپسر ایشان اخیمعص، پسر صادوق، و یوناتان، پسر ابیاتار، در آنجا با ایشانند و هر خبری را که می‌شنوید، به‌دست ایشان، نزد من خواهیدفرستاد.»پس حوشای، دوست داود، به شهررفت و ابشالوم وارد اورشلیم شد.
\par 37 پس حوشای، دوست داود، به شهررفت و ابشالوم وارد اورشلیم شد.
 
\chapter{16}

\par 1 و چون داود از سر کوه اندکی گذشته بود، اینک صیبا، خادم مفیبوشت، بایک جفت الاغ آراسته که دویست قرص نان وصد قرص کشمش و صد قرص انجیر و یک مشک شراب بر آنها بود، به استقبال وی آمد.
\par 2 و پادشاه به صیبا گفت: «از این چیزها چه مقصودداری؟» صیبا گفت: «الاغها به جهت سوار شدن اهل خانه پادشاه، و نان و انجیر برای خوراک خادمان، و شراب به جهت نوشیدن خسته شدگان در بیابان است.»
\par 3 پادشاه گفت: «اما پسر آقایت کجا است؟» صیبا به پادشاه عرض کرد: «اینک دراورشلیم مانده است، زیرا فکر می‌کند که امروزخاندان اسرائیل سلطنت پدر مرا به من ردخواهند کرد.»
\par 4 پادشاه به صیبا گفت: «اینک کل مایملک مفیبوشت از مال توست.» پس صیباگفت: «اظهار بندگی می‌نمایم‌ای آقایم پادشاه، تمنا اینکه در نظر تو التفات یابم.»
\par 5 و چون داود پادشاه به بحوریم رسید، اینک شخصی از قبیله خاندان شاول مسمی به شمعی از آنجا بیرون آمد و چون می‌آمد، دشنام می‌داد.
\par 6 و به داود و به جمیع خادمان داود پادشاه سنگهامی انداخت، و تمامی قوم و جمیع شجاعان به طرف راست و چپ او بودند
\par 7 و شمعی دشنام داده، چنین می‌گفت: «دور شو دور شو‌ای مردخون ریز و‌ای مرد بلیعال!
\par 8 خداوند تمامی خون خاندان شاول را که در جایش سلطنت نمودی برتو رد کرده، و خداوند سلطنت را به‌دست پسر توابشالوم، تسلیم نموده است، و اینک چونکه مردی خون ریز هستی، به شرارت خود گرفتارشده‌ای.»
\par 9 و ابیشای ابن صرویه به پادشاه گفت که «چرااین سگ مرده، آقایم پادشاه را دشنام دهد؟ مستدعی آنکه بروم و سرش را از تن جدا کنم.»
\par 10 پادشاه گفت: «ای پسران صرویه مرا با شما چه‌کار است؟ بگذارید که دشنام دهد، زیرا خداونداو را گفته است که داود را دشنام بده، پس کیست که بگوید چرا این کار را می‌کنی؟»
\par 11 و داود به ابیشای و به تمامی خادمان گفت: «اینک پسر من که از صلب من بیرون آمد، قصد جان من دارد، پس حال چند مرتبه زیاده این بنیامینی، پس او رابگذارید که دشنام دهد زیرا خداوند او را امرفرموده است.
\par 12 شاید خداوند بر مصیبت من نگاه کند و خداوند به عوض دشنامی که او امروز به من می‌دهد، به من جزای نیکو دهد.»
\par 13 پس داود ومردانش راه خود را پیش گرفتند و اما شمعی دربرابر ایشان به‌جانب کوه می‌رفت و چون می‌رفت، دشنام داده، سنگها به سوی او می‌انداخت و خاک به هوا می‌پاشید.
\par 14 و پادشاه با تمامی قومی که همراهش بودند، خسته شده، آمدند و در آنجااستراحت کردند.
\par 15 و اما ابشالوم و تمامی گروه مردان اسرائیل به اورشلیم آمدند، و اخیتوفل همراهش بود،
\par 16 وچون حوشای ارکی، دوست داود، نزد ابشالوم رسید، حوشای به ابشالوم گفت: «پادشاه زنده بماند! پادشاه زنده بماند!»
\par 17 و ابشالوم به حوشای گفت: «آیا مهربانی تو با دوست خود این است؟ چرا با دوست خود نرفتی؟»
\par 18 و حوشای به ابشالوم گفت: «نی، بلکه هرکس را که خداوند واین قوم و جمیع مردان اسرائیل برگزیده باشند، بنده او خواهم بود و نزد او خواهم ماند.
\par 19 و ثانی که را می‌باید خدمت نمایم؟ آیا نه نزد پسر او؟ پس چنانکه به حضور پدر تو خدمت نموده‌ام، به همان طور در حضور تو خواهم بود.»
\par 20 و ابشالوم به اخیتوفل گفت: «شما مشورت کنید که چه بکنیم.»
\par 21 و اخیتوفل به ابشالوم گفت که «نزد متعه های پدر خود که به جهت نگاهبانی خانه گذاشته است، درآی، و چون تمامی اسرائیل بشنوند که نزد پدرت مکروه شده‌ای، آنگاه دست تمامی همراهانت قوی خواهد شد.»
\par 22 پس خیمه‌ای بر پشت بام برای ابشالوم برپاکردند و ابشالوم در نظر تمامی بنی‌اسرائیل نزدمتعه های پدرش درآمد.و مشورتی که اخیتوفل در آن روزها می‌داد، مثل آن بود که کسی از کلام خدا سوال کند. و هر مشورتی که اخیتوفل هم به داود و هم به ابشالوم می‌داد، چنین می‌بود.
\par 23 و مشورتی که اخیتوفل در آن روزها می‌داد، مثل آن بود که کسی از کلام خدا سوال کند. و هر مشورتی که اخیتوفل هم به داود و هم به ابشالوم می‌داد، چنین می‌بود.
 
\chapter{17}

\par 1 و اخیتوفل به ابشالوم گفت: «مرا اذن بده که دوازده هزار نفر را برگزیده، برخیزم و شبانگاه داود را تعاقب نمایم.
\par 2 پس در حالتی که او خسته و دستهایش سست است بر اورسیده، او را مضطرب خواهم ساخت و تمامی قومی که همراهش هستند، خواهند گریخت، وپادشاه را به تنهایی خواهم کشت.
\par 3 و تمامی قوم را نزد تو خواهم برگردانید زیرا شخصی که او رامی طلبی مثل برگشتن همه است، پس تمامی قوم در سلامتی خواهند بود.»
\par 4 و این سخن در نظرجمیع مشایخ اسرائیل پسند آمد.
\par 5 و ابشالوم گفت: «حوشای ارکی را نیزبخوانید تا بشنویم که او چه خواهد گفت.»
\par 6 وچون حوشای نزد ابشالوم آمد، ابشالوم وی راخطاب کرده، گفت: «اخیتوفل بدین مضمون گفته است، پس تو بگو که بر‌حسب رای او عمل نماییم یا نه.»
\par 7 حوشای به ابشالوم گفت: «مشورتی که اخیتوفل این مرتبه داده است، خوب نیست.»
\par 8 و حوشای گفت: «می‌دانی که پدرت ومردانش شجاع هستند و مثل خرسی که بچه هایش را در بیابان گرفته باشند، در تلخی جانند، و پدرت مرد جنگ آزموده است، و شب را در میان قوم نمی ماند.
\par 9 اینک او الان درحفره‌ای یا جای دیگر مخفی است، و واقع خواهد شد که چون بعضی از ایشان در ابتدابیفتند، هر کس که بشنود خواهد گفت: در میان قومی که تابع ابشالوم هستند، شکستی واقع شده است.
\par 10 و نیز شجاعی که دلش مثل دل شیرباشد، بالکل گداخته خواهد شد، زیرا جمیع اسرائیل می‌دانند که پدر تو جباری است ورفیقانش شجاع هستند.
\par 11 لهذا رای من این است که تمامی اسرائیل از دان تا بئرشبع که مثل ریگ کناره دریا بی‌شمارند، نزد تو جمع شوند، وحضرت تو همراه ایشان برود.
\par 12 پس در مکانی که یافت می‌شود بر او خواهیم رسید، و مثل شبنمی که بر زمین می‌ریزد بر او فرود خواهیم آمد، و از او و تمامی مردانی که همراه وی می‌باشند، یکی هم باقی نخواهد ماند.
\par 13 و اگر به شهری داخل شود، آنگاه تمامی اسرائیل طنابهابه آن شهر خواهند‌آورد و آن شهر را به نهرخواهند کشید تا یک سنگ ریزه‌ای هم در آن پیدانشود.»
\par 14 پس ابشالوم و جمیع مردان اسرائیل گفتند: «مشورت حوشای ارکی از مشورت اخیتوفل بهتر است.» زیرا خداوند مقدر فرموده بود که مشورت نیکوی اخیتوفل را باطل گرداند تا آنکه خداوند بدی را بر ابشالوم برساند.
\par 15 و حوشای به صادوق و ابیاتار کهنه گفت: «اخیتوفل به ابشالوم و مشایخ اسرائیل چنین و چنان مشورت داده، و من چنین و چنان مشورت داده‌ام.
\par 16 پس حال به زودی بفرستید و داود رااطلاع داده، گویید: امشب در کناره های بیابان توقف منما بلکه به هر طوری که توانی عبور کن، مبادا پادشاه و همه کسانی که همراه وی می‌باشند، بلعیده شوند.»
\par 17 و یوناتان و اخیمعص نزد عین روجل توقف می‌نمودند و کنیزی رفته، برای ایشان خبرمی آورد، و ایشان رفته، به داود پادشاه خبرمی رسانیدند، زیرا نمی توانستند به شهر داخل شوند که مبادا خویشتن را ظاهر سازند.
\par 18 اماغلامی ایشان را دیده، به ابشالوم خبر داد، و هردوی ایشان به زودی رفته، به خانه شخصی دربحوریم داخل شدند و در حیاط او چاهی بود که در آن فرود شدند.
\par 19 و زن، سرپوش چاه راگرفته، بر دهنه‌اش گسترانید و بلغور بر آن ریخت. پس چیزی معلوم نشد.
\par 20 و خادمان ابشالوم نزدآن زن به خانه درآمده، گفتند: «اخیمعص ویوناتان کجایند؟» زن به ایشان گفت: «از نهر آب عبور کردند.» پس چون جستجو کرده، نیافتند، به اورشلیم برگشتند.
\par 21 و بعد از رفتن آنها، ایشان از چاه برآمدند ورفته، داود پادشاه را خبر دادند و به داود گفتند: «برخیزید و به زودی از آب عبور کنید، زیرا که اخیتوفل درباره شما چنین مشورت داده است.» 
\par 22 پس داود و تمامی قومی که همراهش بودند، برخاستند و از اردن عبور کردند و تا طلوع فجریکی باقی نماند که از اردن عبور نکرده باشد.
\par 23 اما چون اخیتوفل دید که مشورت او بجاآورده نشد، الاغ خود را بیاراست و برخاسته، به شهر خود به خانه‌اش رفت و برای خانه خودتدارک دیده، خویشتن را خفه کرد و مرد و او را در قبر پدرش دفن کردند.
\par 24 اما داود به محنایم آمد و ابشالوم از اردن گذشت و تمامی مردان اسرائیل همراهش بودند.
\par 25 و ابشالوم، عماسا را به‌جای یوآب به‌سرداری لشکر نصب کرد، و عماسا پسر شخصی مسمی به یترای اسرائیلی بود که نزد ابیجایل، دخترناحاش، خواهر صرویه، مادر یوآب درآمده بود.
\par 26 پس اسرائیل و ابشالوم در زمین جلعاد اردوزدند.
\par 27 و واقع شد که چون داود به محنایم رسید، شوبی ابن ناحاش از ربت بنی عمون و ماکیر بن عمیئیل از لودبار و برزلائی جلعادی از روجلیم،
\par 28 بسترها و کاسه‌ها و ظروف سفالین و گندم وجو و آرد و خوشه های برشته و باقلا و عدس ونخود برشته،و عسل و کره و گوسفندان و پنیرگاو برای خوراک داود و قومی که همراهش بودندآوردند، زیرا گفتند که قوم در بیابان گرسنه وخسته و تشنه می‌باشند.
\par 29 و عسل و کره و گوسفندان و پنیرگاو برای خوراک داود و قومی که همراهش بودندآوردند، زیرا گفتند که قوم در بیابان گرسنه وخسته و تشنه می‌باشند.
 
\chapter{18}

\par 1 و داود قومی را که همراهش بودند، سان دید، و سرداران هزاره و سرداران صده برایشان تعیین نمود.
\par 2 و داود قوم را روانه نمود، ثلثی به‌دست یوآب و ثلثی به‌دست ابیشای ابن صرویه، برادر یوآب، و ثلثی به‌دست اتای جتی. و پادشاه به قوم گفت: «من نیز البته همراه شما می‌آیم.»
\par 3 اما قوم گفتند: «تو همراه مانخواهی آمد زیرا اگر ما فرار کنیم درباره ما فکرنخواهند کرد و اگر نصف ما بمیریم برای ما فکرنخواهند کرد و حال تو مثل ده هزار نفر ما هستی، پس الان بهتر این است که ما را از شهر امداد کنی.»
\par 4 پادشاه به ایشان گفت: «آنچه در نظر شما پسند آید، خواهم کرد.» و پادشاه به‌جانب دروازه ایستاده بود، و تمامی قوم با صده‌ها و هزاره هابیرون رفتند.
\par 5 و پادشاه یوآب و ابیشای و اتای راامر فرموده، گفت: «به‌خاطر من بر ابشالوم جوان به رفق رفتار نمایید. و چون پادشاه جمیع سرداران را درباره ابشالوم فرمان داد، تمامی قوم شنیدند.
\par 6 پس قوم به مقابله اسرائیل به صحرا بیرون رفتند و جنگ در جنگل افرایم بود.
\par 7 و قوم اسرائیل در آنجا از حضور بندگان داود شکست یافتند، و در آن روز کشتار عظیمی در آنجا شد وبیست هزار نفر کشته شدند.
\par 8 و جنگ در آنجا برروی تمامی زمین منتشر شد و در آن روز آنانی که از جنگل هلاک گشتند، بیشتر بودند از آنانی که به شمشیر کشته شدند.
\par 9 و ابشالوم به بندگان داود برخورد و ابشالوم بر قاطر سوار بود و قاطر زیر شاخه های پیچیده شده بلوط بزرگی درآمد، و سر او در میان بلوطگرفتار شد، به طوری که در میان آسمان و زمین آویزان گشت و قاطری که زیرش بود، بگذشت.
\par 10 و شخصی آن را دیده، به یوآب خبر رسانید وگفت: «اینک ابشالوم را دیدم که در میان درخت بلوط آویزان است.»
\par 11 و یوآب به آن شخصی که او را خبر داد، گفت: «هان تو دیده‌ای، پس چرا اورا در آنجا به زمین نزدی؟ و من ده مثقال نقره وکمربندی به تو می‌دادم.»
\par 12 آن شخص به یوآب گفت: «اگر هزار مثقال نقره به‌دست من می‌رسیددست خود را بر پسر پادشاه دراز نمی کردم، زیراکه پادشاه تو را و ابیشای و اتای را به سمع ما امرفرموده، گفت: «زنهار هر یکی از شما درباره ابشالوم جوان باحذر باشید.
\par 13 والا بر جان خودظلم می‌کردم چونکه هیچ امری از پادشاه مخفی نمی ماند، و خودت به ضد من بر می‌خاستی.»
\par 14 آنگاه یوآب گفت: «نمی توانم با تو به اینطورتاخیر نمایم.» پس سه تیر به‌دست خود گرفته، آنها را به دل ابشالوم زد حینی که او هنوز در میان بلوط زنده بود.
\par 15 و ده جوان که سلاحداران یوآب بودند دور ابشالوم را گرفته، او را زدند وکشتند.
\par 16 و چون یوآب کرنا را نواخت، قوم از تعاقب نمودن اسرائیل برگشتند، زیرا که یوآب قوم رامنع نمود.
\par 17 و ابشالوم را گرفته، او را در حفره بزرگ که در جنگل بود، انداختند، و بر او توده بسیار بزرگ از سنگها افراشتند، و جمیع اسرائیل هر یک به خیمه خود فرار کردند.
\par 18 اما ابشالوم در حین حیات خود بنایی را که در وادی ملک است برای خود برپا کرد، زیرا گفت پسری ندارم که از او اسم من مذکور بماند، و آن بنا را به اسم خود مسمی ساخت. پس تا امروز ید ابشالوم خوانده می‌شود.
\par 19 و اخیمعص بن صادوق گفت: «حال بروم ومژده به پادشاه برسانم که خداوند انتقام او را ازدشمنانش کشیده است.»
\par 20 یوآب او را گفت: «توامروز صاحب بشارت نیستی، اما روز دیگربشارت خواهی برد و امروز مژده نخواهی دادچونکه پسر پادشاه مرده است.»
\par 21 و یوآب به کوشی گفت: «برو و از آنچه دیده‌ای به پادشاه خبر برسان.» و کوشی یوآب را تعظیم نموده، دوید.
\par 22 و اخیمعص بن صادوق، بار دیگر به یوآب گفت: «هرچه بشود، ملتمس اینکه من نیزدر عقب کوشی بدوم.» یوآب گفت: «ای پسرم چرا باید بدوی چونکه بشارت نداری که ببری؟»
\par 23 گفت: «هرچه بشود، بدوم.» او وی را گفت: «بدو.» پس اخیمعص به راه وادی دویده، ازکوشی سبقت جست.
\par 24 و داود در میان دو دروازه نشسته بود ودیده بان بر پشت بام دروازه به حصار برآمد وچشمان خود را بلند کرده، مردی را دید که اینک به تنهایی می‌دود.
\par 25 و دیده بان آواز کرده، پادشاه را خبر داد و پادشاه گفت: «اگر تنهاست بشارت می‌آورد.» و او می‌آمد و نزدیک می‌شد.
\par 26 ودیده بان، شخص دیگر را دید که می‌دود ودیده بان به دربان آواز داده، گفت: «شخصی به تنهایی می‌دود.» و پادشاه گفت: «او نیز بشارت می‌آورد.»
\par 27 و دیده بان گفت: «دویدن اولی رامی بینم که مثل دویدن اخیمعص بن صادوق است.» پادشاه گفت: «او مرد خوبی است و خبرخوب می‌آورد.»
\par 28 و اخیمعص ندا کرده، به پادشاه گفت: «سلامتی است.» و پیش پادشاه رو به زمین افتاده، گفت: «یهوه خدای تو متبارک باد که مردمانی که دست خود را بر آقایم پادشاه بلند کرده بودند، تسلیم کرده است.»
\par 29 پادشاه گفت: «آیا ابشالوم جوان به سلامت است؟ و اخیمعص در جواب گفت: «چون یوآب، بنده پادشاه و بنده تو رافرستاد، هنگامه عظیمی دیدم که ندانستم که چه بود.»
\par 30 و پادشاه گفت: «بگرد و اینجا بایست.» واو به آن طرف شده، بایستاد.
\par 31 و اینک کوشی رسید و کوشی گفت: «برای آقایم، پادشاه، بشارت است، زیرا خداوند امروزانتقام تو را از هر‌که با تو مقاومت می‌نمود، کشیده است.»
\par 32 و پادشاه به کوشی گفت: «آیا ابشالوم جوان به سلامت است؟» کوشی گفت: «دشمنان آقایم، پادشاه، و هر‌که برای ضرر تو برخیزد، مثل آن جوان باشد.»پس پادشاه، بسیار مضطرب شده، به بالاخانه دروازه برآمد و می‌گریست و چون می‌رفت، چنین می‌گفت: «ای پسرم ابشالوم! ای پسرم! پسرم ابشالوم! کاش که به‌جای تومی مردم، ای ابشالوم پسرم‌ای پسر من!»
\par 33 پس پادشاه، بسیار مضطرب شده، به بالاخانه دروازه برآمد و می‌گریست و چون می‌رفت، چنین می‌گفت: «ای پسرم ابشالوم! ای پسرم! پسرم ابشالوم! کاش که به‌جای تومی مردم، ای ابشالوم پسرم‌ای پسر من!»
 
\chapter{19}

\par 1 و به یوآب خبر دادند که اینک پادشاه گریه می‌کند و برای ابشالوم ماتم گرفته است.
\par 2 و در آن روز برای تمامی قوم ظفر به ماتم مبدل گشت، زیرا قوم در آن روز شنیدند که پادشاه برای پسرش غمگین است.
\par 3 و قوم در آن روز دزدانه به شهر داخل شدند، مثل کسانی که ازجنگ فرار کرده، از روی خجالت دزدانه می‌آیند.
\par 4 و پادشاه روی خود را پوشانید و پادشاه به آوازبلند صدا زد که‌ای پسرم ابشالوم! ای ابشالوم! پسرم! ای پسر من!
\par 5 پس یوآب نزد پادشاه به خانه درآمده، گفت: «امروز روی تمامی بندگان خود راشرمنده ساختی که جان تو و جان پسرانت ودخترانت و جان زنانت و جان متعه هایت را امروزنجات دادند.
\par 6 چونکه دشمنان خود را دوست داشتی و محبان خویش را بغض نمودی، زیرا که امروز ظاهر ساختی که سرداران و خادمان نزد توهیچند و امروز فهمیدم که اگر ابشالوم زنده می‌ماند و جمیع ما امروز می‌مردیم آنگاه در نظرتو پسند می‌آمد.
\par 7 و الان برخاسته، بیرون بیا و به بندگان خود سخنان دل آویز بگو، زیرا به خداوندقسم می‌خورم که اگر بیرون نیایی، امشب برای توکسی نخواهد ماند، و این بلا برای تو بدتر خواهدبود از همه بلایایی که از طفولیتت تا این وقت به تو رسیده است.»
\par 8 پس پادشاه برخاست و نزددروازه بنشست و تمامی قوم را خبر داده، گفتند که «اینک پادشاه نزد دروازه نشسته است.» وتمامی قوم به حضور پادشاه آمدند.
\par 9 و جمیع قوم در تمامی اسباط اسرائیل منازعه کرده، می‌گفتند که «پادشاه ما را از دست دشمنان ما رهانیده است، و اوست که ما را ازدست فلسطینیان رهایی داده، و حال به‌سبب ابشالوم از زمین فرار کرده است.
\par 10 و ابشالوم که او را برای خود مسح نموده بودیم، در جنگ مرده است. پس الان شما چرا در بازآوردن پادشاه تاخیر می‌نمایید؟»
\par 11 و داود پادشاه نزد صادوق و ابیاتار کهنه فرستاده، گفت: «به مشایخ یهودا بگویید: شماچرا در بازآوردن پادشاه به خانه‌اش، آخر همه هستید، و حال آنکه سخن جمیع اسرائیل نزدپادشاه به خانه‌اش رسیده است.
\par 12 شما برادران من هستید و شما استخوانها و گوشت منید. پس چرا در بازآوردن پادشاه، آخر همه می‌باشید؟
\par 13 و به عماسا بگویید: آیا تو استخوان و گوشت من نیستی؟ خدا به من مثل این بلکه زیاده از این به عمل آورد اگر تو در حضور من در همه اوقات به‌جای یوآب، سردار لشکر، نباشی.»
\par 14 پس دل جمیع مردان یهودا را مثل یک شخص مایل گردانید که ایشان نزد پادشاه پیغام فرستادند که توو تمامی بندگانت برگردید.
\par 15 پس پادشاه برگشته، به اردن رسید و یهودا به استقبال پادشاه به جلجال آمدند تا پادشاه را از اردن عبور دهند.
\par 16 و شمعی بن جیرای بنیامینی که از بحوریم بود، تعجیل نموده، همراه مردان یهودا به استقبال داود پادشاه فرودآمد.
\par 17 و هزار نفر از بنیامینیان وصیبا، خادم خاندان شاول، با پانزده پسرش وبیست خادمش همراهش بودند، و ایشان پیش پادشاه از اردن عبور کردند.
\par 18 و معبر را عبوردادند تا خاندان پادشاه عبور کنند، و هر‌چه درنظرش پسند آید بجا آورند.
\par 19 و به پادشاه گفت: «آقایم گناهی بر من اسناد ندهد و خطایی را که بنده ات در روزی که آقایم پادشاه از اورشلیم بیرون می‌آمد ورزید بیاد نیاورد و پادشاه آن را به دل خود راه ندهد.
\par 20 زیرا که بنده تو می‌داند که گناه کرده‌ام و اینک امروز من از تمامی خاندان یوسف، اول آمده‌ام و به استقبال آقایم، پادشاه، فرود شده‌ام.»
\par 21 و ابیشای ابن صرویه متوجه شده، گفت: «آیا شمعی به‌سبب اینکه مسیح خداوند را دشنام داده است، کشته نشود؟»
\par 22 اماداود گفت: «ای پسران صرویه مرا با شما چه‌کاراست که امروز دشمن من باشید و آیا امروز کسی در اسرائیل کشته شود و آیا نمی دانم که من امروزبر اسرائیل پادشاه هستم؟»
\par 23 پس پادشاه به شمعی گفت: «نخواهی مرد.» و پادشاه برای وی قسم خورد.
\par 24 و مفیبوشت، پسر شاول، به استقبال پادشاه آمد و از روزی که پادشاه رفت تا روزی که به سلامتی برگشت نه پایهای خود را ساز داده، و نه ریش خویش را طراز نموده، و نه جامه خود راشسته بود.
\par 25 و چون برای ملاقات پادشاه به اورشلیم رسید، پادشاه وی را گفت: «ای مفیبوشت چرا با من نیامدی؟»
\par 26 او عرض کرد: «ای آقایم پادشاه، خادم من مرا فریب داد زیرابنده ات گفت که الاغ خود را خواهم آراست تا برآن سوار شده، نزد پادشاه بروم، چونکه بنده تولنگ است.
\par 27 و او بنده تو را نزد آقایم، پادشاه، متهم کرده است، لیکن آقایم، پادشاه، مثل فرشته خداست، پس هر‌چه در نظرت پسند آید، به عمل آور. 
\par 28 زیرا تمامی خاندان پدرم به حضورت آقایم، پادشاه، مثل مردمان مرده بودند، و بنده خود را در میان خورندگان سفره ات ممتازگردانیدی، پس من دیگر‌چه حق دارم که باز نزدپادشاه فریاد نمایم.»
\par 29 پادشاه وی را گفت: «چرادیگر از کارهای خود سخن می‌گویی؟ گفتم که توو صیبا، زمین را تقسیم نمایید.»
\par 30 مفیبوشت به پادشاه عرض کرد: «نی، بلکه او همه را بگیردچونکه آقایم، پادشاه، به خانه خود به سلامتی برگشته است.»
\par 31 و برزلائی جلعادی از روجلیم فرود آمد وبا پادشاه از اردن عبور کرد تا او را به آن طرف اردن مشایعت نماید.
\par 32 و برزلای مرد بسیار پیر هشتادساله بود، و هنگامی که پادشاه در محنایم توقف می‌نمود او را پرورش می‌داد زیرا مردی بسیاربزرگ بود.
\par 33 و پادشاه به برزلای گفت: «تو همراه من بیا و تو را در اورشلیم پرورش خواهم داد.»
\par 34 برزلای به پادشاه عرض کرد: «ایام سالهای زندگی من چند است که با پادشاه به اورشلیم بیایم؟
\par 35 من امروز هشتاد ساله هستم و آیامی توانم در میان نیک و بد تمیز بدهم و آیا بنده توطعم آنچه را که می‌خورم و می‌نوشم، توانم دریافت؟ یا دیگر آواز مغنیان و مغنیات را توانم شنید؟ پس چرا بنده ات دیگر برای آقایم پادشاه بار باشد؟
\par 36 لهذا بنده تو همراه پادشاه اندکی ازاردن عبور خواهد نمود و چرا پادشاه مرا چنین مکافات بدهد.
\par 37 بگذار که بنده ات برگردد تا درشهر خود نزد قبر پدر و مادر خویش بمیرم، واینک بنده تو، کمهام، همراه آقایم پادشاه برود وآنچه در نظرت پسند آید با او به عمل آور.»
\par 38 پادشاه گفت: «کمهام همراه من خواهد آمدو آنچه در نظر تو پسند آید، با وی به عمل خواهم آورد و هر‌چه از من خواهش کنی برای تو به انجام خواهم رسانید.»
\par 39 پس تمامی قوم از اردن عبور کردند و چون پادشاه عبور کرد، پادشاه برزلائی را بوسید و وی را برکت داد و او به مکان خود برگشت.
\par 40 و پادشاه به جلجال رفت وکمهام همراهش آمد و تمامی قوم یهودا و نصف قوم اسرائیل نیز پادشاه را عبور دادند.
\par 41 و اینک جمیع مردان اسرائیل نزد پادشاه آمدند و به پادشاه گفتند: «چرا برادران ما، یعنی مردان یهودا، تو را دزدیدند و پادشاه و خاندانش را و جمیع کسان داود را همراهش از اردن عبوردادند؟»
\par 42 و جمیع مردان یهودا به مردان اسرائیل جواب دادند: «از این سبب که پادشاه ازخویشان ماست، پس چرا از این امر حسدمی برید؟ آیا چیزی از پادشاه خورده‌ایم یا انعامی به ما داده است؟»و مردان اسرائیل در جواب مردان یهودا گفتند: «ما را در پادشاه ده حصه است و حق ما در داود از شما بیشتر است. پس چرا ما راحقیر شمردید؟ و آیا ما برای بازآوردن پادشاه خود، اول سخن نگفتیم؟» اما گفتگوی مردان یهودا از گفتگوی مردان اسرائیل سختتر بود.
\par 43 و مردان اسرائیل در جواب مردان یهودا گفتند: «ما را در پادشاه ده حصه است و حق ما در داود از شما بیشتر است. پس چرا ما راحقیر شمردید؟ و آیا ما برای بازآوردن پادشاه خود، اول سخن نگفتیم؟» اما گفتگوی مردان یهودا از گفتگوی مردان اسرائیل سختتر بود.
 
\chapter{20}

\par 1 و اتفاق مرد بلیعال، مسمی به شبع بن بکری بنیامینی در آنجا بود و کرنا رانواخته، گفت که «ما را در داود حصه‌ای نیست، وبرای ما در پسر یسا نصیبی نی، ای اسرائیل! هرکس به خیمه خود برود.»
\par 2 و تمامی مردان اسرائیل از متابعت داود به متابعت شبع ابن بکری برگشتند، اما مردان یهودا از اردن تا اورشلیم، پادشاه را ملازمت نمودند.
\par 3 و داود به خانه خود در اورشلیم آمد، وپادشاه ده زن متعه را که برای نگاهبانی خانه خودگذاشته بود، گرفت و ایشان را در خانه محروس نگاه داشته، پرورش داد، اما نزد ایشان داخل نشدو ایشان تا روز مردن در حالت بیوگی محبوس بودند.
\par 4 و پادشاه به عماسا گفت: «مردان یهودا را درسه روز نزد من جمع کن و تو در اینجا حاضر شو.»
\par 5 پس عماسا رفت تا یهودا را جمع کند، اما اززمانی که برایش تعیین نموده بود تاخیر کرد.
\par 6 وداود به ابیشای گفت: «الان شبع بن بکری بیشتر ازابشالوم به ما ضرر خواهد رسانید؛ پس بندگان آقایت را برداشته، او را تعاقب نما مبادا شهرهای حصاردار برای خود پیدا کند و از نظر ما رهایی یابد.»
\par 7 و کسان یوآب و کریتیان و فلیتیان و جمیع شجاعان از عقب او بیرون رفتند، و به جهت تعاقب نمودن شبع بن بکری از اورشلیم روانه شدند.
\par 8 و چون ایشان نزد سنگ بزرگی که درجبعون است رسیدند، عماسا به استقبال ایشان آمد. و یوآب ردای جنگی دربرداشت و بر آن بندشمشیری که در غلافش بود، بر کمرش بسته، و چون می‌رفت شمشیر از غلاف افتاد.
\par 9 و یوآب به عماسا گفت: «ای برادرم آیا به سلامت هستی؟» ویوآب ریش عماسا را به‌دست راست خود گرفت تا او را ببوسد.
\par 10 و عماسا به شمشیری که دردست یوآب بود، اعتنا ننمود. پس او آن را به شکمش فرو برد که احشایش به زمین ریخت و اورا دوباره نزد و مرد.
\par 11 و یکی از خادمان یوآب نزدوی ایستاده، گفت: «هرکه یوآب را می‌خواهد وهرکه به طرف داود است، در عقب یوآب بیاید.»
\par 12 و عماسا در میان راه در خونش می‌غلطید، وچون آن شخص دید که تمامی قوم می‌ایستند، عماسا را از میان راه در صحرا کشید و لباسی بر اوانداخت زیرا دید که هر‌که نزدش می‌آید، می‌ایستد.
\par 13 پس چون از میان راه برداشته شد، جمیع مردان در عقب یوآب رفتند تا شبع بن بکری را تعاقب نمایند.
\par 14 و او از جمیع اسباط اسرائیل تا آبل و تا بیت معکه و تمامی بیریان عبور کرد، و ایشان نیز جمع شده، او را متابعت کردند.
\par 15 و ایشان آمده، او رادر آبل بیت معکه محاصره نمودند و پشته‌ای دربرابر شهر ساختند که در برابر حصار برپا شد، وتمامی قوم که با یوآب بودند، حصار را می‌زدند تاآن را منهدم سازند.
\par 16 و زنی حکیم از شهر صدادرداد که بشنوید: «به یوآب بگویید: اینجا نزدیک بیا تا با تو سخن گویم.»
\par 17 و چون نزدیک وی شد، زن گفت که «آیا تو یوآب هستی؟» او گفت: «من هستم.» وی را گفت: «سخنان کنیز خود را بشنو.» او گفت: «می‌شنوم.»
\par 18 پس زن متکلم شده، گفت: «در زمان قدیم چنین می‌گفتند که هرآینه درآبل می‌باید مشورت بجویند و همچنین هر امری را ختم می‌کردند.
\par 19 من در اسرائیل سالم و امین هستم و تو می‌خواهی شهری و مادری را دراسرائیل خراب کنی، چرا نصیب خداوند را بالکل هلاک می‌کنی؟»
\par 20 پس یوآب در جواب گفت: «حاشا از من حاشا از من! که هلاک یا خراب نمایم.
\par 21 کار چنین نیست بلکه شخصی مسمی به شبع بن بکری از کوهستان افرایم دست خود را برداود پادشاه بلند کرده است. او را تنها بسپارید واز نزد شهر خواهم رفت.» زن در جواب یوآب گفت: «اینک سر او را از روی حصار نزد توخواهند انداخت.»
\par 22 پس آن زن به حکمت خودنزد تمامی قوم رفت و ایشان سر شبع بن بکری رااز تن جدا کرده، نزد یوآب انداختند و او کرنا رانواخته، ایشان از نزد شهر، هر کس به خیمه خودمتفرق شدند. و یوآب به اورشلیم نزد پادشاه برگشت.
\par 23 و یوآب، سردار تمامی لشکر اسرائیل بود، و بنایاهو ابن یهویاداع سردار کریتیان و فلیتیان بود.
\par 24 و ادورام سردار باجگیران و یهوشافاط بن اخیلود وقایع نگار،
\par 25 و شیوا کاتب و صادوق وابیاتار، کاهن بودند،و عیرای یائیری نیز کاهن داود بود.
\par 26 و عیرای یائیری نیز کاهن داود بود.
 
\chapter{21}

\par 1 و در ایام داود، سه سال علی الاتصال قحطی شد، و داود به حضور خداوندسوال کرد و خداوند گفت: «به‌سبب شاول وخاندان خون ریز او شده است زیرا که جبعونیان را کشت.»
\par 2 و پادشاه جبعونیان را خوانده، به ایشان گفت (اما جبعونیان از بنی‌اسرائیل نبودندبلکه از بقیه اموریان، و بنی‌اسرائیل برای ایشان قسم خورده بودند؛ لیکن شاول از غیرتی که برای اسرائیل و یهودا داشت، قصد قتل ایشان می‌نمود).
\par 3 و داود به جبعونیان گفت: «برای شماچه بکنم و با چه چیز کفاره نمایم تا نصیب خداوند را برکت دهید.»
\par 4 جبعونیان وی را گفتند: «از شاول و خاندانش، نقره و طلا نمی خواهیم ونه آنکه کسی در اسرائیل برای ما کشته شود.» اوگفت: «هر‌چه شما بگویید، برای شما خواهم کرد.»
\par 5 ایشان به پادشاه گفتند: «آن شخص که مارا تباه می‌ساخت و برای ما تدبیر می‌کرد که ما راهلاک سازد تا در هیچ کدام از حدود اسرائیل باقی نمانیم،
\par 6 هفت نفر از پسران او به ما تسلیم شوند تاایشان را در حضور خداوند در جبعه شاول که برگزیده خداوند بود به دار کشیم.» پادشاه گفت: «ایشان را به شما تسلیم خواهم کرد.»
\par 7 اما پادشاه، مفیبوشت بن یوناتان بن شاول رادریغ داشت، به‌سبب قسم خداوند که در میان ایشان، یعنی در میان داود و یوناتان بن شاول بود.
\par 8 و پادشاه ارمونی و مفیبوشت، دو پسر رصفه، دختر ایه که ایشان را برای شاول زاییده بود، وپنج پسر میکال، دختر شاول را که برای عدرئیل بن برزلای محولاتی زاییده بود، گرفت،
\par 9 و ایشان را به‌دست جبعونیان تسلیم نموده، آنها را در آن کوه به حضور خداوند به دار کشیدند و این هفت نفر با هم افتادند، و ایشان در ابتدای ایام حصاد دراول درویدن جو کشته شدند.
\par 10 و رصفه، دختر ایه، پلاسی گرفته، آن رابرای خود از ابتدای درو تا باران از آسمان برایشان بارانیده شد، بر صخره‌ای گسترانید، ونگذاشت که پرندگان هوا در روز، یا بهایم صحرادر شب بر ایشان بیایند.
\par 11 و داود را از آنچه رصفه، دختر ایه، متعه شاول کرده بود، خبردادند.
\par 12 پس داود رفته، استخوانهای شاول واستخوانهای پسرش، یوناتان را از اهل یابیش جلعاد گرفت که ایشان آنها را از شارع عام بیت‌شان دزدیده بودند، جایی که فلسطینیان آنهارا آویخته بودند در روزی که فلسطینیان شاول رادر جلبوع کشته بودند.
\par 13 و استخوانهای شاول واستخوانهای پسرش، یوناتان را از آنجا آورد واستخوانهای آنانی را که بر دار بودند نیز، جمع کردند.
\par 14 و استخوانهای شاول و پسرش یوناتان را در صیلع، در زمین بنیامین، در قبر پدرش قیس، دفن کردند و هرچه پادشاه امر فرموده بود، بجاآوردند. و بعد از آن، خدا به جهت زمین اجابت فرمود.
\par 15 و باز فلسطینیان با اسرائیل جنگ کردند وداود با بندگانش فرود آمده، با فلسطینیان مقاتله نمودند و داود وامانده شد.
\par 16 و یشبی بنوب که ازاولاد رافا بود و وزن نیزه او سیصد مثقال برنج بودو شمشیری نو بر کمر داشت، قصد کشتن داودنمود.
\par 17 اما ابیشای ابن صرویه او را مدد کرده، آن فلسطینی را زد و کشت. آنگاه کسان داود قسم خورده، به وی گفتند: «بار دیگر همراه ما به جنگ نخواهی آمد مبادا چراغ اسرائیل را خاموش گردانی.»
\par 18 و بعد از آن نیز، جنگی با فلسطینیان درجوب واقع شد که در آن سبکای حوشاتی، صاف را که او نیز از اولاد رافا بود، کشت.
\par 19 و باز جنگ با فلسطینیان در جوب واقع شد و الحانان بن یعری ارجیم بیت لحمی، جلیات جتی را کشت که چوب نیزه‌اش مثل نورد جولاهکان بود.
\par 20 ودیگر جنگی در جت واقع شد که در آنجا مردی بلند قد بود که دست و پای او هریک شش انگشت داشت که جمله آنها بیست و چهار باشد و او نیزبرای رافا زاییده شده بود.
\par 21 و چون اسرائیل را به ننگ‌آورد، یوناتان بن شمعی، برادر داود، او راکشت.این چهار نفر برای رافا در جت زاییده شده بودند و به‌دست داود و به‌دست بندگانش افتادند.
\par 22 این چهار نفر برای رافا در جت زاییده شده بودند و به‌دست داود و به‌دست بندگانش افتادند.
 
\chapter{22}

\par 1 و داود در روزی که خداوند او را ازدست جمیع دشمنانش و از دست شاول رهایی داد، کلمات این سرود را برای خداوند انشا نمود.
\par 2 و گفت: «خداوند صخره من و قلعه من و رهاننده من است.
\par 3 خدای صخره من که بر او توکل خواهم نمود، سپر من و شاخ نجاتم، برج بلند و ملجای من، ای نجات‌دهنده من، مرا از ظلم خواهی رهانید.
\par 4 خداوند را که سزاوار کل حمد است، خواهم خواند. پس از دشمنان خود خلاصی خواهم یافت.
\par 5 زیرا که موجهای موت مرا احاطه نموده، وسیلهای عصیان مرا ترسانیده بود. 
\par 6 رسنهای گور مرا احاطه نمودند. دامهای موت مرا دریافتند.
\par 7 در تنگی خود خداوند را خواندم. و نزد خدای خویش دعا نمودم. و او آواز مرا از هیکل خودشنید. و استغاثه من به گوش وی رسید.
\par 8 آنگاه زمین متزلزل و مرتعش گردید. واساسهای آسمان بلرزیدند. و از حدت خشم اومتحرک گردیدند.
\par 9 از بینی وی دود متصاعد شد. و از دهان او آتش سوزان درآمد و اخگرها از آن افروخته گردید.
\par 10 و او آسمانها را خم کرده، نزول فرمود. وتاریکی غلیظ زیر پایهایش بود.
\par 11 بر کروبین سوار شده، پرواز نمود. و بر بالهای باد نمایان گردید.
\par 12 ظلمت را به اطراف خود سایبانها ساخت. واجتماع آبها و ابرهای متراکم افلاک را.
\par 13 از درخشندگی‌ای که پیش روی وی بود، اخگرهای آتش افروخته گردید.
\par 14 خداوند از آسمان رعد نمود. و حضرت اعلی آواز خویش را مسموع گردانید.
\par 15 تیرها فرستاده، ایشان را پراکنده ساخت. و برق را جهانیده، ایشان را سراسیمه گردانید.
\par 16 پس عمق های دریا ظاهر شد. و اساسهای ربع مسکون منکشف گردید. از توبیخ خداوند و ازنفخه باد بینی وی.
\par 17 از اعلی علیین فرستاده، مرا گرفت. و از آبهای بسیار مرا بیرون کشید.
\par 18 مرا از دشمنان زورآورم رهایی داد. و ازمبغضانم، چونکه از من قویتر بودند.
\par 19 در روز شقاوت من، ایشان مرا دریافته بودند. لیکن خداوند تکیه گاه من بود.
\par 20 مرا به مکان وسیع بیرون آورد. و مرا خلاصی داد چونکه به من رغبت می‌داشت.
\par 21 پس خداوند مرا به حسب عدالتم جزا خواهدداد. و به حسب پاکیزگی دستم مرا مکافات خواهد رسانید.
\par 22 زیرا که طریق های خداوند را حفظ نمودم. و ازخدای خویش عصیان نورزیدم.
\par 23 چونکه جمیع احکام او در مد نظر من است. واز فرایض او انحراف نورزیدم.
\par 24 و به حضور او کامل شدم. و از عصیان ورزیدن، خویشتن را بازداشتم.
\par 25 بنابراین خداوند مرا به حسب عدالتم جزا داد. و بر‌حسب صداقتی که در نظر وی داشتم.
\par 26 با شخص رحیم، خویشتن را رحیم خواهی نمود. و با مرد کامل با کاملیت رفتار خواهی کرد.
\par 27 با شخص طاهر به طهارت عمل خواهی نمود. و با کج خلقان مخالفت خواهی کرد.
\par 28 و قوم مستمند را نجات خواهی داد. اماچشمان تو بر متکبران است تا ایشان را پست گردانی.
\par 29 زیرا که تو‌ای خداوند، نور من هستی. وخداوند، تاریکی مرا به روشنایی مبدل خواهدساخت.
\par 30 زیرا که به استعانت تو بر لشکری تاخت آوردم. و به مدد خدای خود بر حصارها جست وخیز نمودم.
\par 31 و اما خدا، طریق وی کامل است؛ و کلام خداوند مصفا؛ و او برای جمیع متوکلانش سپرمی باشد.
\par 32 زیرا کیست خدا غیر از یهوه؟ و کیست صخره غیر از خدای ما؟
\par 33 خدا قلعه استوار من است. و طریق مرا کامل می سازد.
\par 34 و پایهایم را مثل پای غزال می‌گرداند، و مرا برمکانهای بلندم برپا می‌دارد.
\par 35 دستهای مرا به جنگ تعلیم می‌دهد، و به بازوی خود کمان برنجین را می‌کشم.
\par 36 و سپر نجات خود را به من خواهی داد، و لطف تو مرا بزرگ خواهد ساخت.
\par 37 قدمهای مرا در زیر من وسعت دادی که پایهایم نلغزید.
\par 38 دشمنان خود را تعاقب نموده، ایشان را هلاک خواهم ساخت، و تا نابود نشوند بر نخواهم گشت.
\par 39 ایشان را خراب کرده، خرد خواهم ساخت تادیگر برنخیزند، و زیر پایهایم خواهند افتاد.
\par 40 زیرا کمر مرا برای جنگ به قوت خواهی بست، و آنانی را که به ضد من برخیزند در زیر من خم خواهی ساخت.
\par 41 و دشمنانم را پیش من منهزم خواهی کرد تاخصمان خود را منقطع سازم.
\par 42 فریاد برمی آورند، اما رهاننده‌ای نیست؛ و به سوی خداوند، لیکن ایشان را اجابت نخواهدکرد.
\par 43 پس ایشان را مثل غبار زمین نرم می‌کنم. و مثل گل کوچه‌ها کوبیده، پایمال می‌سازم.
\par 44 و تو مرا از مخاصمات قوم من خواهی رهانید، و مرا برای سرداری امت‌ها حفظ خواهی کرد، وقومی را که نشناخته بودم، مرا بندگی خواهندنمود.
\par 45 غریبان نزد من تذلل خواهند کرد و به مجردشنیدن من، مرا اطاعت خواهند نمود.
\par 46 غریبان پژمرده خواهند گردید و از مکان های مخفی خود با ترس بیرون خواهند آمد.
\par 47 خداوند زنده است و صخره من متبارک وخدای صخره نجات من متعال باد.
\par 48 ‌ای خدایی که برای من انتقام می‌کشی و قومهارا زیر من پست می‌سازی.
\par 49 و مرا از دست دشمنانم بیرون می‌آوری و برمقاومت کنندگانم مرا بلند می‌گردانی. تو مرا ازمرد ظالم خلاصی خواهی داد.
\par 50 بنابراین‌ای خداوند، تو را در میان امت‌ها حمدخواهم گفت. و به نام تو ترنم خواهم نمود.نجات عظیمی برای پادشاه خود می‌نماید. وبرای مسیح خویش رحمت را پدید می‌آورد. به جهت داود و ذریت وی تا ابدالاباد.»
\par 51 نجات عظیمی برای پادشاه خود می‌نماید. وبرای مسیح خویش رحمت را پدید می‌آورد. به جهت داود و ذریت وی تا ابدالاباد.»
 
\chapter{23}

\par 1 و این است سخنان آخر داود: «وحی داود بن یسا. و وحی مردی که بر مقام بلند ممتاز گردید، مسیح خدای یعقوب، و مغنی شیرین اسرائیل.
\par 2 روح خداوند به وسیله من متکلم شد و کلام او بر زبانم جاری گردید.
\par 3 خدای اسرائیل متکلم شد و صخره اسرائیل مراگفت: آنکه بر مردمان حکمرانی کند، عادل باشدو با خدا ترسی سلطنت نماید.
\par 4 و او خواهد بودمثل روشنایی صبح، وقتی که آفتاب طلوع نماید، یعنی صبح بی‌ابر، هنگامی که علف سبز از زمین می‌روید، به‌سبب درخشندگی بعد از باران.
\par 5 یقین خانه من با خدا چنین نیست. لیکن عهدجاودانی با من بسته است، که در همه‌چیز آراسته و مستحکم است. و تمامی نجات و تمامی مسرت من این است، هرچند آن را نمو نمی دهد.
\par 6 لیکن جمیع مردان بلیعال مثل خارهایند که دورانداخته می‌شوند. چونکه آنها را به‌دست نتوان گرفت.
\par 7 و کسی‌که ایشان را لمس نماید، می‌بایدبا آهن و نی نیزه مسلح شود. و ایشان در مسکن خود با آتش سوخته خواهند شد.»
\par 8 و نامهای شجاعانی که داود داشت این است: یوشیب بشبت تحکمونی که سردار شالیشیم بودکه همان عدینو عصنی باشد که بر هشتصد نفرتاخت آورد و ایشان را در یک وقت کشت.
\par 9 و بعد از او العازار بن دودو ابن اخوخی، یکی از آن سه مرد شجاع که با داود بودند، هنگامی که فلسطینیان را که در آنجا برای جنگ جمع شده، ومردان اسرائیل رفته بودند، به مقاتله طلبیدند.
\par 10 واما او برخاسته، با فلسطینیان جنگ کرد تا دستش خسته شد و دستش به شمشیر چسبید و خداونددر آن روز، ظفر عظیمی داد، و قوم در عقب اوفقط برای غارت کردن برگشتند.
\par 11 و بعد از او شمه بن آجی هراری بود وفلسطینیان، لشکری فراهم آوردند، در جایی که قطعه زمینی پر از عدس بود، و قوم از حضورفلسطینیان فرار می‌کردند.
\par 12 آنگاه او در میان آن قطعه زمین ایستاد و آن را نگاه داشته، فلسطینیان را شکست داد و خداوند ظفر عظیمی داد.
\par 13 و سه نفر از آن سی سردار فرود شده، نزدداود در وقت حصاد به مغاره عدلام آمدند، ولشکر فلسطینیان در وادی رفائیم اردو زده بودند.
\par 14 و داود در آن وقت در ملاذ خویش بود و قراول فلسطینیان در بیت لحم.
\par 15 و داود خواهش نموده، گفت: «کاش کسی مرا از چاهی که نزددروازه بیت لحم است آب بنوشاند.»
\par 16 پس آن سه مرد شجاع، لشکر فلسطینیان را از میان شکافته، آب را از چاهی که نزد دروازه بیت لحم است کشیده، برداشتند و آن را نزد داود آوردند، اما نخواست که آن را بنوشد و آن را به جهت خداوند ریخت.
\par 17 و گفت: «ای خداوند حاشا ازمن که این کار را بکنم، مگر این خون آن کسان نیست که به خطر جان خود رفتند؟» از این جهت نخواست که بنوشد. کاری که این سه مرد کردند، این است.
\par 18 و ابیشای، برادر یوآب بن صرویه، سردارسه نفر بود و نیزه خود را بر سیصد نفر حرکت داده، ایشان را کشت و در میان آن سه نفر اسم یافت.
\par 19 آیا از آن سه نفر مکرم تر نبود؟ پس سردار ایشان شد لیکن به سه نفر اول نرسید.
\par 20 و بنایاهو ابن یهویاداع، پسر مردی شجاع قبصئیلی، که کارهای عظیم کرده بود، دو پسراریئیل موآبی را کشت و در روز برف به حفره‌ای فرود شده، شیری را بکشت.
\par 21 و مرد خوش اندام مصری‌ای را کشت و آن مصری در دست خود نیزه‌ای داشت اما نزد وی با چوب دستی رفت و نیزه را از دست مصری ربود و وی را با نیزه خودش کشت.
\par 22 و بنایاهو ابن یهویاداع این کارها را کرد و در میان آن سه مرد شجاع اسم یافت.
\par 23 و از آن سی نفر مکرمتر شد لیکن به آن سه نفر اول نرسید و داود او را بر اهل مشورت خود گماشت.
\par 24 و عسائیل برادر یوآب یکی از آن سی نفربود و الحانان بن دودوی بیت لحمی،
\par 25 و شمه حرودی و الیقای حرودی،
\par 26 و حالص فلطی و عیرا ابن عقیش تقوعی،
\par 27 و ابیعزر عناتوتی ومبونای حوشاتی،
\par 28 و صلمون اخوخی و مهرای نطوفاتی،
\par 29 و حالب بن بعنه نطوفاتی و اتای بن ریبای از جبعه بنی بنیامین،
\par 30 و بنایای فرعاتونی و هدای از وادیهای جاعش،
\par 31 و ابوعلبون عرباتی و عزموت برحومی،
\par 32 و الیحبای شعلبونی و از بنی یاشن یوناتان،
\par 33 و شمه حراری و اخیام بن شارر اراری،
\par 34 و الیفلط بن احسبای ابن معکاتی و الیعام بن اخیتوفل جیلونی،
\par 35 وحصرای کرملی و فعرای اربی،
\par 36 و یجال بن ناتان از صوبه و بانی جادی،
\par 37 و صالق عمونی ونحرای بئیروتی که سلاحداران یوآب بن صرویه بودند،
\par 38 و عیرای یتری و جارب یتری،واوریای حتی، که جمیع اینها سی و هفت نفربودند.
\par 39 واوریای حتی، که جمیع اینها سی و هفت نفربودند.
 
\chapter{24}

\par 1 و خشم خداوند بار دیگر بر اسرائیل افروخته شد. پس داود را بر ایشان برانگیزانیده، گفت: «برو و اسرائیل و یهودا رابشمار.»
\par 2 و پادشاه به‌سردار لشکر خود یوآب که همراهش بود، گفت: «الان در تمامی اسباطاسرائیل از دان تا بئرشبع گردش کرده، قوم رابشمار تا عدد قوم را بدانم.»
\par 3 و یوآب به پادشاه گفت: «حال یهوه، خدای تو، عدد قوم را هر‌چه باشد، صد چندان زیاده کند، و چشمان آقایم، پادشاه، این را ببیند، لیکن چرا آقایم، پادشاه، خواهش این عمل دارد؟»
\par 4 اما کلام پادشاه بریوآب و سرداران لشکر غالب آمد و یوآب وسرداران لشکر از حضور پادشاه برای شمردن قوم اسرائیل بیرون رفتند.
\par 5 و از اردن عبور کرده، در عروعیر به طرف راست شهری که در وسط وادی جاد در مقابل یعزیر است، اردو زدند.
\par 6 و به جلعاد و زمین تحتیم حدشی آمدند و به دان یعن رسیده، به سوی صیدون دور زدند.
\par 7 و به قلعه صور و تمامی شهرهای حویان و کنعانیان آمدند و به جنوب یهودا تا بئرشبع گذشتند.
\par 8 وچون در تمامی زمین گشته بودند، بعد ازانقضای نه ماه و بیست روز به اورشلیم مراجعت کردند.
\par 9 و یوآب عدد شمرده شدگان قوم را به پادشاه داد: از اسرائیل هشتصد هزار مردجنگی شمشیرزن و از یهودا پانصد هزار مردبودند.
\par 10 و داود بعد از آنکه قوم را شمرده بود، دردل خود پشیمان گشت. پس داود به خداوندگفت: «در این کاری که کردم، گناه عظیمی ورزیدم و حال‌ای خداوند گناه بنده خود را عفوفرما زیرا که بسیار احمقانه رفتار نمودم.»
\par 11 وبامدادان چون داود برخاست، کلام خداوند به‌جاد نبی که رایی داود بود، نازل شده، گفت:
\par 12 «برو داود را بگو خداوند چنین می‌گوید: سه چیز پیش تو می‌گذارم پس یکی از آنها را برای خود اختیار کن تا برایت به عمل آورم.»
\par 13 پس جاد نزد داود آمده، او را مخبر ساخت و گفت: «آیا هفت سال قحط در زمینت برتو عارض شود، یا سه ماه از حضور دشمنان خود فرار نمایی وایشان تو را تعاقب کنند، یا وبا سه روز در زمین توواقع شود. پس الان تشخیص نموده، ببین که نزدفرستنده خود چه جواب ببرم.»
\par 14 داود به‌جادگفت: «در شدت تنگی هستم. تمنا اینکه به‌دست خداوند بیفتیم زیرا که رحمتهای او عظیم است وبه‌دست انسان نیفتم.» 
\par 15 پس خداوند وبا براسرائیل از آن صبح تاوقت معین فرستاد و هفتاد هزار نفر از قوم، از دان تا بئرشبع مردند.
\par 16 و چون فرشته، دست خود رابر اورشلیم دراز کرد تا آن را هلاک سازد، خداونداز آن بلا پشیمان شد و به فرشته‌ای که قوم راهلاک می‌ساخت گفت: «کافی است! حال دست خود را باز دار.» و فرشته خداوند نزد خرمنگاه ارونه یبوسی بود.
\par 17 و چون داود، فرشته‌ای را که قوم را هلاک می‌ساخت دید، به خداوند عرض کرده، گفت: «اینک من گناه کرده‌ام و من عصیان ورزیده‌ام اما این گوسفندان چه کرده‌اند؟ تمنااینکه دست تو بر من و برخاندان پدرم باشد.»
\par 18 و در آن روز جاد نزد داود آمده، گفت: «بروو مذبحی در خرمنگاه ارونه یبوسی برای خداوندبرپا کن.»
\par 19 پس داود موافق کلام جاد چنانکه خداوند امر فرموده بود، رفت.
\par 20 و چون ارونه نظر انداخته، پادشاه و بندگانش را دید که نزد وی می‌آیند، ارونه بیرون آمده، به حضور پادشاه به روی خود به زمین افتاده، تعظیم نمود.
\par 21 و ارونه گفت: «آقایم، پادشاه، چرا نزد بنده خود آمده است؟» داود گفت: «تا خرمنگاه را از تو بخرم ومذبحی برای خداوند بنا نمایم و تا وبا از قوم رفع شود.»
\par 22 و ارونه به داود عرض کرد: «آقایم پادشاه آنچه را که در نظرش پسند آیدگرفته، قربانی کند و اینک گاوان به جهت قربانی سوختنی و چومها و اسباب گاوان به جهت هیزم. این همه را‌ای پادشاه، ارونه به پادشاه می‌دهد. و ارونه به پادشاه گفت: «یهوه، خدایت، تو را قبول فرماید.»اما پادشاه به ارونه گفت: «نی، بلکه البته به قیمت از توخواهم گرفت، و برای یهوه، خدای خود، قربانی های سوختنی بی‌قیمت نخواهم گذرانید.» پس داود خرمنگاه و گاوان را به پنجاه مثقال نقره خرید.
\par 23 اما پادشاه به ارونه گفت: «نی، بلکه البته به قیمت از توخواهم گرفت، و برای یهوه، خدای خود، قربانی های سوختنی بی‌قیمت نخواهم گذرانید.» پس داود خرمنگاه و گاوان را به پنجاه مثقال نقره خرید.


\end{document}