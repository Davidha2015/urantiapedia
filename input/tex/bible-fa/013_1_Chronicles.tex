\begin{document}

\title{اول تواريخ}

 
\chapter{1}

\par 1 آدم، شِيث اَنوش،
\par 2 قِينان مهلَلئِيل يارَد،
\par 3 خَنُوخ مَتُوشالَح لَمَک،
\par 4 نُوح سام حام يافَث.
\par 5 پسران يافَث: جُومَر و ماجُوج و ماداي و ياوان و ماشَک و تيراس.
\par 6 وپسران جُومَر: اَشکَناز و ريفات و تُجَرمَه.
\par 7 و پسران ياوان: اَلِيشَه و تَرشِيش و کتيم و دُودانِيم.
\par 8 و پسران حام: کوش و مِصرايم و فُوت و کَنعان.
\par 9 و پسران کوش: سَبا و حَويلَه و سبتا و رَعما و سَبتَکا. و پسران رَعما: شَبا و دَدان.
\par 10 و کوش، نِمرود را آورد، و او به جبار شدن در جهان شروع نمود.
\par 11 و مِصرايم، لُوديم و عَنَاميم و لَهابيم و نَفتُوحيم را آورد،
\par 12 و فَتروسيم و کَسلُوحيم را که فَلَستيم و کَفتوريم از ايشان پديد آمدند.
\par 13 و کَنعان نُخست زاده خود، صيدون و حِتّ را آورد،
\par 14 و يبُوسي و اَمُوري و جَرجاشي،
\par 15 و حِوّي و عِرقي و سِيني،
\par 16 و اروادي و صَماري و حَماتي را.
\par 17 پسران سام: عيلام و آشُّور و اَرفَکشاد و لُود واَرام و عُوص و حُول و جاتَر و ماشَک.
\par 18 و اَرفَکشاد، شالَح را آورد و شالَح، عابَر را آورد.
\par 19 و براي عابَر، دو پسر متولد شدند که يکي را فالَج نام بود زيرا در ايام وي زمين منقسم شد و اسم برادرش يقطان بود.
\par 20 و يقطان، اَلمُوداد و شالَف و حَضَرموت و يارَح را آورد؛
\par 21 و هَدُورام و اُوزال و دِقلَه،
\par 22 و اِيبال و اَبيمايل و شَبا،
\par 23 و اُوفير و حَويلَه و يوباب را که جميع اينها پسران يقطان بودند.
\par 24 سام، اَرفَکشاد سالَح،
\par 25 عابَر فالَج رَعُو،
\par 26 سَروج ناحُور تارَح،
\par 27 اَبرام که همان ابراهيم باشد.
\par 28 پسران ابراهيم: اسحاق واسماعيل.
\par 29 اين است پيداش ايشان: نخستزاده اسماعيل: نَبايوت و قيدار و اَدَبئيل و مِبسام،
\par 30 و مِشماع و دُومَه و مَسّا و حَدَد و تيما،
\par 31 و يطُور و نافيش و قِدمَه که اينان پسران اسماعيل بودند.
\par 32 و پسران قَطُورَه که مُتعه ابراهيم بود، پس او زِمران و يقشان و مَدان و مِديان و يشباق و شُوحا را زاييد و پسران يقشان: شَبا و دَدان بودند.
\par 33 و پسران مِديان عِيفَه و عِيفَرو خَنُوح و اَبيداع و اَلدَعَه بودند. پس جميع اينها پسران قَطُورَه بودند.
\par 34 و ابراهيم اسحاق را آورد؛ و پسران اسحاق عِيسُو واسرائيل بودند.
\par 35 و پسران عِيسُو:اَليفاز و رَعُوئيل و يعُوش و يعلام وقُورَح.
\par 36 پسران اَليفاز: تيمان و اُومار و صَفي و جَعتام و قَناز و تِمناع و عَماليق.
\par 37 پسران رَعُوئيل: نَحَت و زارَح و شَمَّه و مِزَّه .
\par 38 و پسران سَعِير: لُوطان و شُوبال و صِبعون و عَنَه و ديشُون و اِيصر ودِيشان.
\par 39 و پسران لُوطان: حوري و هُومام و خواهر لُوطان تِمناع.
\par 40 پسران شُوبال: عَليان ومَنَاحَت وعِيبال وشَفي واُونام و پسران صِبعُون: اَيه و عَنَه.
\par 41 و پسران عَنَه: ديشون وپسران ديشون: حَمران و اِشبان و يتران و کَران.
\par 42 پسران ايصر: بِلهان و زَعوان و يعقان وپسران ديشان: عُوص و اَران.
\par 43 و پادشاهاني که در زمين اَدُوم سلطنت نمودند، پيش از آنکه پادشاهي بر بني اسرائيل سلطنت کند، اينانند: بالَع بن بَعُور و اسم شهر او دِنهابِه بود.
\par 44 و بالَع مُرد و يوباب بن زارَح از بُصرَه به جايش پادشاه شد.
\par 45 و يوباب مرد و حوشام از زمين تيماني به جايش سلطنت نمود.
\par 46 و حُوشام مُرد وهَدَد بن بَدَد که مِديان را در زمين موآب شکست داد در جايش پادشاه شد و اسم شهرش عَوِيت بود .
\par 47 و هَدَد مُرد و سَملَه از مَسريقَه به جايش پادشاه شد.
\par 48 و سَملَه مُرد و شاؤل از رَحُوبوت نهر به جايش پادشاه شد.
\par 49 و شاؤل مُرد و بَعل حانان بن عَکبور به جايش پادشاه شد.
\par 50 و بَعل حانان مُرد وهَدَد به جايش پادشاه شد؛ و اسم شهرش فاعي و اسم زنش مَهِيطَبئيل دختر مَطرِد دختر مَي ذَهَب بود.
\par 51 وهَدَد مُرد و اَميرانِ اَدوُم امير تِمناع وامير اَليه و امير يتِيت بودند.
\par 52 و اَمير اَهولِيبَامَه و امير اِيلَه و امير فِينُون؛
\par 53 وامير قَناز و اميرِتيمان وامير مِبصار؛وامير مَجدِيئيل وامير عيرام؛ اينان اميران اَدُوم بودند.
\par 54 وامير مَجدِيئيل وامير عيرام؛ اينان اميران اَدُوم بودند.
 
\chapter{2}

\par 1 پسران اسرائيل اينانند: رؤبين وشَمعون و لاوي يهودا ويسّاکار زبِولون
\par 2 و دان ويوسف و بنيامين و نَفتالي وجاد واَشِير.
\par 3 پسران يهودا: عِير واُونان وشيلَه؛ اين سه نفر از بَتشُوعِ کَنعانيه براي او زاييده شدند؛ وعير نخست زاده يهودا به نظر خداوند شرير بود؛ پس او را کُشت.
\par 4 و عروس وي تامار فارَص و زارَح را براي وي زاييد، و همه پسران يهودا پنج نفر بودند.
\par 5 و پسران فارَص: حَصرون وحامول.
\par 6 و پسران زارَح: زِمري و اِيتان وهِيمان و کَلکُول و دارَع که همگي ايشان پنج نفر بودند.
\par 7 واز پسران کَرمِي، عاکار مضطرب کننده اسرائيل بود که درباره چيز حرام خيانت ورزيد.
\par 8 و پسر اِيتان: عَزَريا بود.
\par 9 و پسران حَصرُون که براي وي زاييده شدند، يرحَمئيل ورام وکَلُوباي.
\par 10 و رام عَميناداب را آورد وعَميناداب نَحشُون را آورد که رئيس بني يهودا بود.
\par 11 و نَحشُون سَلما را آورد و سَلما بُوعَز را آورد.
\par 12 و بُوعَز عوبيد را آورد و عُوبيد يسي را آورد.
\par 13 يسي نخست زاده خويش اَلِيآب را آورد،و دومين ابيناداب را، و سومين شِمعي را،
\par 14 وچهارمين نَتَنئيل را و پنجمين رَدّاي را،
\par 15 وششمين اُوصَم را وهفتمين داود را آورد.
\par 16 و خواهران ايشان صَرُويه و اَبيحايل بودند. و پسران صَرُويه، اَبشاي و يوآب و عَسائيل، سه نفر بودند.
\par 17 و اَبيحايل عَماسا را زائيد و پدر عَماسا يتَرِ اِسماعيلي بود.
\par 18 و کاليب بن حَصرُون از زن خود عَزُوبَه واز يرِيعُوت اولاد به هم رسانيد و پسران وي اينانند: ياشَر و شُوباب و اَردُون.
\par 19 و عَزُوبَه مُرد وکاليب اَفرات را به زني گرفت و او حور را براي وي زاييد.
\par 20 وحُور، اُوري را آورد واُوري بَصَلئيل را آورد.
\par 21 وبعد از آن، حَصرُون به دختر ماکير پدر جِلعاد درآمده، او را به زني گرفت حيني که شصت ساله بود و او سَجُوب را براي وي زاييد.
\par 22 و سَجُوب يائير را آورد و او بيست و سه شهر در زمين جِلعاد داشت.
\par 23 واو جَشور و اَرام را که حَوُّوب يائير باشد، با قنات و دهات آنها که شصت شهر بود، از ايشان گرفت و جمييع اينها از آن بني ماکير پدر جِلعاد بودند.
\par 24 و بعد از آنکه حَصرُون در کاليب اَفراته وفات يافت، اَبِيه زن حَصرُون اَشحُور پدر تَقُوع را براي وي زاييد.
\par 25 و پسران يرحَمئيل نخست زاده حَصرُون نخست زاده اش: رام و بُونَه و اُورَن و اُوصَم و اَخيا بودند.
\par 26 و يرحَمئيل را زن ديگر مسمّاة به عطارَه بود که مادرِ اُونام باشد.
\par 27 و پسران رام نخست زاده يرحَمئيل مَعص ويامِين و عاقَر بودند.
\par 28 و پسران اُونام: شَمّاي وياداع بودند، و پسران شَمّاي ناداب و اَبيشور.
\par 29 و اسم زن ابيشور اَبِيحايل بود و او اَحبان و مُوليد را براي وي زاييد.
\par 30 ناداب سَلَد و اَفّايم بودند و سَلَد بي اولاد مُرد.
\par 31 و بني اَفّايم يشعي و بني يشعي شيشان و بني شيشان اَحلاي.
\par 32 و پسران ياداع برادر شَماي يتَر ويوناتان؛ و يتَر بي اولاد مُرد.
\par 33 و پسران يوناتان: فالَت وزازا. اينها پسران يرحَمئيل بودند.
\par 34 و شيشان را پسري نبود ليکن دختران داشت وشيشان را غلامي مصري بود که يرحاع نام داشت.
\par 35 و شيشان دختر خود را به غلام خويش يرحاع به زني داد و او عَتّاي را براي وي زاييد.
\par 36 وعتاي ناتان را آورد و ناتان زاباد را آورد.
\par 37 و زاباد اَفلال را آورد و اَفلال عوبيد را آورد.
\par 38 و عوبيد ييهُو را آورد، ييهُوعَزَريا را آورد.
\par 39 و عَزَريا حالَص را آورد و حالص اَلعاسَه را آورد.
\par 40 و اَلعاسَه سَسماي را آورد و سَسماي شَلُّوُم را آورد.
\par 41 و شَلوم يقَميا را آورد و يقَميا اَلِيشَمَع را آورد.
\par 42 و بني کاليب برادر يرحَمئيل نخست زاده اش ميشاع که پدر زِيف باشد و بني ماريشَه که پدرحَبرُون باشد بودند.
\par 43 و پسران حَبرُون: قُورَح و تَفُّوح و راقَم و شامَع.
\par 44 و شامَع راحَم پدر يرقَعام راآورد و راقَم شَمّاي را آورد.
\par 45 و پسر شَمّاي ماعئون و ماعون پدر بَيت صُور بود.
\par 46 عِيفَه مُتعه کاليب حاران و موصا و جازيز را زاييد و حاران جازيز را آورد.
\par 47 وپسران يهداي راجَم و يوتام و جيشان و فالَت و عِيفَه و شاعَف.
\par 48 و مَعکه مُتعه کاليب، شابَرو تِرحَنه را زاييد.
\par 49 و او نيز شاعَف، پدر مَدمَنَه و شوا، پدر مَکبينا پدر جِبعا را زاييد؛ و دختر کاليب عَکسَه بود.
\par 50 و پسران کاليب بن حُور نخست زاده اَفراته اينانند: شُوبال پدر قربه يعاريم،
\par 51 و سَلما پدر بيت لحم و حاريف پدر بيت جادَر.
\par 52 و پسران شوبال پدر قريه يعاريم اينانند: هَرُواه و نصف مَنُوحُوت.
\par 53 و قبايل قريه يعاريم اينانند: يترِيان و فُوتيان و شُوماتيان و مِشراعيان که از ايشان صارعاتيان و اِشطاوُليان پيدا شدند.
\par 54 وبني سَلما بيت لحم و نطوفاتيان و عَطروت بيت يوآب و نصف مانَحتيان و صُرعيان بودند.و قبايل کاتباني که در يعبيص ساکن بودند، تِرعاتيان و شِمعاتيان و سُوکاتيان بودند. اينان قينيان اند که از حَمَّت پدرِبَيت ريکاب بيرون آمدند.
\par 55 و قبايل کاتباني که در يعبيص ساکن بودند، تِرعاتيان و شِمعاتيان و سُوکاتيان بودند. اينان قينيان اند که از حَمَّت پدرِبَيت ريکاب بيرون آمدند.
 
\chapter{3}

\par 1 و پسران داود که براي او در حَبرُون زاييده شدند، اينانند: نخست زاده اش اَمنون از اَخينُوعِم يزرَعِيليه؛ و دومين دانيال از اَبِيجايلِ کَرمَلِيه؛
\par 2 و سومين ابشالوم پسر مَعکَه دختر تَلماي پادشاه جَشور؛ و چهارمين اُدُونيا پسر حَجّيت.
\par 3 و پنجمين شَفَطيا از اَبيطال و ششمين يتَرعام از زن او عجلَه.
\par 4 اين شش براي او در حَبرون زاييده شدند که در آنجا هفت سال و شش ماه سلطنت نمود و در اورشليم سي سه سال سلطنت کرد.
\par 5 و اينها براي وي در اورشليم زاييده شدند: شِمعي و شوباب و ناتان و سُليمان. اين چهار از بَتشُوع دخترعَمّيئيل بودند.
\par 6 ويبِحار و اَليشامَع و اَليفالَط.
\par 7 و نُوجَه و نافَج و يافيع.
\par 8 و اَلِشَمَع و اَلياداع و اَلِيفَلَط که نه نفر باشند.
\par 9 همه اينها پسران داود بودند سواي پسران مُتعه ها. و خواهر ايشان تامار بود.
\par 10 و پسرسُليمان، رَحَبعام و پسر او ابّيا و پسر او آسا و پسر اويهُوشافاط.
\par 11 و پسر او يورام و پسر او اَخَزيا و پسر او يوآش.
\par 12 و پسر او اَمَصيا و پسر او عَزَريا وپسر او يوتام.
\par 13 و پسر او آحاز و پسر او حِزقيا و پسر او مَنَّسي.
\par 14 و پسر او آمون و پسر او يوشيا.
\par 15 و پسران يوشيا نخست زاده اش يوحانان و دومين يهُوياقيم و سومين صِدقَيا و چهارمين شَلّوم.
\par 16 و پسران يهُوياقيم پسر او يکُنيا و پسر او صِدقَيا.
\par 17 و پسران يکُنيا اَشّير و پسر او شأَلِتيئيل.
\par 18 و مَلکيرام و فَدايا و شَنأَصَّر و يقَميا و هوشاماع و نَدَبيا.
\par 19 و پسران فَدايا زَرُبّابِل و شِمعي و پسران زَرُبّابِل مَشُلاّم وحَنَنيا و خواهرايشان شَلُوميت بود.
\par 20 و حَشُوبَه و اُوهَل و بَرَخِيا و حَسَديا و يوشَب حَسَد که پنج نفر باشند.
\par 21 و پسران حَنَنيا فَلَطيا و اِشعيا، بني رفايا و بني اَرنان و بني عُوبَديا و بني شَکُنيا.
\par 22 و شَکُنيا شَمَعيا و پسران شَمَعيا، حَطّوش و يبحآل و باريح و نَعَريا و شافاط که شش باشند.
\par 23 و پسران نَعَريا اَليوعيناي و حِزقيا و عَزريقام که سه باشند.و بني اَليوعيناي هُودايا و اَلياشيب و فَلايا و عَقُّوب و يوحانان و دَلاياع و عَناني که هفت باشند. 
\par 24 و بني اَليوعيناي هُودايا و اَلياشيب و فَلايا و عَقُّوب و يوحانان و دَلاياع و عَناني که هفت باشند.
 
\chapter{4}

\par 1 بني يهودا: فارَص و حَصرُون و کَرمي و حور و شوبال.
\par 2 و رَآيا ابن شوبال يحَت را آورد و يحَت اَخُوماي و لاهَد را آورد. اينانند قبايل صَرعاتيان.
\par 3 و اينان پسران پدر عيطام اند: يزرَعيل و يشما و يدباش و اسم خواهر ايشان هَصلَلفُوني بود.
\par 4 فَنُوئيل پدر جَدُور و عازَر پدر خُوشَه اينها پسران حور نخست زاده اَفراته پدر بيت لحم بودند.
\par 5 و اَشحُور پدر تَقّوع دو زن داشت: حَلا و نَعرَه.
\par 6 و نَعرَه، اَخُزّام و حافَر و تَيماني و اَخَشطاري را براي او زاييد؛ اينان پسران نَعرَه اند.
\par 7 و پسران حَلا: صَرَت و صُوحَر و اَتنان.
\par 8 و قُوس عانوب و صُوبِيبَه و قبايل اَخَرحيل بن هارُم را آورد.
\par 9 و يعبِيص از برادران خود شريف تر بود و مادرش او را يعبِيص نام نهاد و گفت:« از اين جهت که او را با حُزن زاييدم.»
\par 10 و يبِيص از خداي اسرائيل استدعا نموده گفت:« کاش که مرا برکت مي دادي و حدود مرا وسيع ميگردانيدي و دست تو با من مي بود و مرا از بلا نگاه مي داشتي تا محزون نشوم.» و خدا آنچه را که خواست به او بخشيد.
\par 11 وکَلوب برادر شُوحَه مَحِير را که پدر اَشتون باشد آورد.
\par 12 و اَشتون بيت رافا و فاسيح تَحِنّه پدر عِير ناحاش را آورد. اينان اهل ريقَه مي باشد.
\par 13 و پسران قَناز و عُتنِيئِيل و سَرايا بودند؛ و پسر عُتنيئيل حَتَات.
\par 14 و مَعُونُوتاي عُفرَه را آورد و سَرايا، يوآب پدرجيحَراشيم را آورد، زيرا که صنعتگر بودند.
\par 15 و پسران کاليب بن يفُنَّه، عيرُو و اِيلَه و ناعَم بودند؛ و پسر اِيلَه قَناز بود.
\par 16 و پسران يهلَلئيل، زيف و زيفَه و تِيريا و اَسَرئيل.
\par 17 و پسران عَزرَه يتَر و مَرَد و عافَر و يالون(و زنِ مَرَد) مريم و شَماي و يشبَح پدر اَشتَمُوع را زاييد.
\par 18 و زن يهوديه او يارَد، پدر جَدُور، و جابَر پدر سُوکُو و يقوتيئيل پدر زانوح را زاييد. اما آنان پسران بِتيه دختر فرعون که مَرَد او را به زني گرفته بود مي باشند.
\par 19 و پسران زن يهوديه او که خواهر نَحَم بود پدر قَعيلَه جَرمي و اَشتَمُوع مَعکاتي بودند.
\par 20 وپسران شيمون: اَمنون وَرِنَّه و بِنحانان وتيلون و پسران يشعِي زُنزُوحيت.
\par 21 و بني شيلَه بن يهودا، عير پدر ليکَه، و لَعدَه پدر مَريشَه و قبايل خاندان عاملان کتان نازک از خانواده اَشبيع بودند.
\par 22 و يوقيم و اهل کُوزِيبا و يوآش و ساراف که در موآب مِلک داشتند، و يشُوبي لَحمَ؛ و اين وقايع قديم است.
\par 23 و اينان کوزه گر بودند با ساکنان نتاعيم و جَديره که در آنجاها نزد پادشاه به جهت کار او سکونت داشتند.
\par 24 پسران شَمعون: نمُوئيل و يامين و ياريب و زارَح و شاؤل.
\par 25 و پسرش شَلّوُم و پسرش مِبسام و پسرش مِشماع.
\par 26 و بني مِشماع پسرش حموُئيل و پسرش زَکُّور و پسران شِمعي.
\par 27 و شِمعي را شانزده پسر و شش دختر بود ولکن برادرانش را پسران بسيار نبود و همه قبايل ايشان مثل بني يهودا زياد نشدند.
\par 28 و ايشان در بئرشَبَع و مُولادَه و حَصَر شُوآل،
\par 29 و در بِلهَه و عاصَم و تولاد،
\par 30 و در بَتوئيل و حَرمُه و صِقلَغ
\par 31 و دربيت مرکَبوت و حَصرسُوسيم و بيت بِرئِي و شَعَرايم ساکن بودند. اينها شهرهاي ايشان تا زمان سلطنت داود بود.
\par 32 و قريه هاي ايشان عيطام عين و رِمُّون و تُوکَن و عاشان، يعني پنج قريه بود،
\par 33 و جميع قريه هاي ايشان که در پيرامون آن شهرها تابَعل بود. پس مسکنهاي ايشان اين است و نسب نامه هاي خود را داشتند.
\par 34 و مَشوبات و يمليک و يوشَه بن اَمصيا،
\par 35 يوئيل و ييهُو ابن يوشِبيا ابن سَرايا ابن عَسِيئيل،
\par 36 و اَليوعيناي و يعکوبَه و يشوحايا وعَسايا و عِديئيل و يسيميئيل و بنايا،
\par 37 و زيزا ابن شِفعِي ابن اِلّوُن بن يدايا ابن شِمري ابن شَمَعيا،
\par 38 ايناني که اسم ايشان مذکور شد، در قبايل خود رؤسا بودند و خانه هاي آباي ايشان بسيار زياد شد.
\par 39 و به مدخل جَدُورتا طرف شرقي وادي رفتند تا براي گله هاي خويش چراگاه بجويند.
\par 40 پس برومند نيکو يافتند و آن زمين وسيع و آرام و امين بود، زيرا که آلِ حام در زمان قديم در آنجا ساکن بودند.
\par 41 و ايناني که اسم ايشان مذکور شد، در ايام حِزقيا پادشاه يهودا آمدند و خيمه هاي ايشان و معونيان را که در آنجا يافت شدند، شکست دادند و ايشان را تا به امروز تباه ساخته، در جاي ايشان ساکن شده اند زيرا که مرتع براي گله هاي ايشان در انجا بود.
\par 42 و بعضي از ايشان، يعني پانصد نفر از بني شَمعُون به کوه سَعير رفتند؛ و فَلطِيا و نَعرِيا و رَفايا و عُرِّيئيل پسران يشيع رؤساي ايشان بودند.و بقيه عَمالَقَه را که فرار کرده بودند، شکست داده، تا امروز در آنجا ساکن شده اند.
\par 43 و بقيه عَمالَقَه را که فرار کرده بودند، شکست داده، تا امروز در آنجا ساکن شده اند.
 
\chapter{5}

\par 1 و پسران رؤبين نخست زاده اسرائيل اينانند: (زيرا که او نخست زاده بود و اما به سبب بي عصمت ساختن بستر پدرخويش، حق نخست زادگي او به پسران يوسف بن اسرائيل داده شد. از اين جهت نسب نامه او بر حسب نخست زادگي ثبت نشده بود.
\par 2 زيرا يهُودا بر برادران خود برتري يافت و پادشاه از او بود؛ اما نخست زادگي از آن يوسف بود).
\par 3 پس پسران رؤبين نخست زاده اسرائيل: حَنوک و فَلُّو و حَصرون و کَرمي.
\par 4 و پسران يوئيل: پسرش شَمَعيا و پسرش جوج و پسرش شِمعِي؛
\par 5 و پسرش ميکا و پسرش رَآيا و پسرش بَعل؛
\par 6 و پسرش بَئيرَه که تِلغَت فِلناسَر پادشاه اَشُّور او را به اسيري بُرد و او رئيس رؤبينيان بود.
\par 7 و برادرانش بر حسب قبايل ايشان وقتي که نسب نامه مواليد ايشان ثبت گرديد، مقدم ايشان يعِيئيل بود و زَکريا،
\par 8 و بالع بن عَزاز بن شامع بن يوئيل که در عَرُوعير تانَبُو و بَعل مَعُون ساکن بود،
\par 9 و به طرف مشرق تا مدخل بيابان از نهر فرات سکنا گرفت، زيرا که مواشي ايشان در زمين جِلعاد زياده شد.
\par 10 و در ايام شاؤل ايشان با حاجريان جنگ کردند و آنها به دست ايشان افتادند و در خيمه هاي آنها در تمامي اطراف شرقي جِلعاد ساکن شدند.
\par 11 و بني جاد در مقابل ايشان در زمين باشان تا سَلخَه ساکن بودند.
\par 12 و مقدّم ايشان يوئيل بود و دومين شافام و يعناي و شافاط در باشان(ساکن بود).
\par 13 و برادران ايشان بر حسب خانه هاي آباي ايشان، ميکائيل و مَشُلام و شَبَع و يوراي و يعکان و زِيع و عابَر که هفت نفر باشند.
\par 14 اينانند پسران اَبيحايل بن حوري ابن ياروح بن جِلعاد بن ميکائيل بن يشيشاي بن يحدُو ابن بوز.
\par 15 اَخي ابن عَبديئيل بن جوني رئيس خاندان آباي ايشان.
\par 16 و ايشان در جِلعادِ باشان و قريه هايش و در تمامي نواحي شارون تا حدود آنها ساکن بودند.
\par 17 نسب نامه جميع اينها در ايام يوتام پادشاه يهودا و در ايام يرُبعام پادشاه اسرائيل ثبت گرديد.
\par 18 از بني رؤبين و جاديان و نصف سبط مَنَّسي شجاعان و مرداني که سپر شمشير برمي داشتند و تيراندازان و جنگ آزمودگان که به جنگ بيرون مي رفتند، چهل و هزار و هفت صد و شصت نفر بودند.
\par 19 و ايشان با حاجريان و يطُور و نافيش و نوداب مقاتله نمودند.
\par 20 و بر ايشان نصرت يافتند و حاجريان و جميع رفقاي آنها به دست ايشان تسليم شدند زيرا که در حين جنگ نزد خدا استغاثه نمودند و او ايشان را چونکه بر او توکل نمودند، اجابت فرمود.
\par 21 پس از مواشي ايشان، پنجاه هزار شتر و دويست و پنجاه هزار گوسفند و دو هزار الاغ و صد هزار مرد به تاراج بردند.
\par 22 زيرا چونکه جنگ از جانب خدا بود، بسياري مقتول گرديدند. پس ايشان به جاي آنها تا زمان اسيري ساکن شدند.
\par 23 و پسران نصف سبط مَنَّسي در آن زمين ساکن شده، از باشان تا بَعل حَرمون و سَنير و جَبَل حَرمون زياد شدند.
\par 24 و اينانند رؤساي خاندان آباي ايشان عافَر و يشعِي و اَليئيل وعَزريئيل و اِرميا و هُودَويا يحدييئل که مردان تنومند شجاع و ناموران و رؤساي خاندان آباي ايشان بودند.
\par 25 اما به خداي پدران خود خيانت ورزيده، در پي خدايان قومهاي آن زمين که خدا آنها را به حضور ايشان هلاک کرده بود، زنا کردند.پس خداي اسرائيل روح فُول پادشاه اَشُّور و روح تِلغَت فِلناسَر پادشاه اَشُّور را برانگيخت که رؤبينيان و جاديان و نصف سبط مَنَّسي را اسير کرده، ايشان را به حَلَح و خابور و هارا و نهر جوزان تا امروز بُرد.
\par 26 پس خداي اسرائيل روح فُول پادشاه اَشُّور و روح تِلغَت فِلناسَر پادشاه اَشُّور را برانگيخت که رؤبينيان و جاديان و نصف سبط مَنَّسي را اسير کرده، ايشان را به حَلَح و خابور و هارا و نهر جوزان تا امروز بُرد.
 
\chapter{6}

\par 1 بني لاوي: جَرشون و قَهات و مَراري.
\par 2 بني قَهات: عَمرام و يصهار و حَبرون و عُزّيئيل.
\par 3 وبني عَمرام: هارون و موسي و مريم. و بني هارون: ناداب و اَبِيهُو و اَليعازار و ايتامار.
\par 4 و اَليعازار فينَحاس را آورد و فينَحاس ابيشوع را آورد.
\par 5 و ابيشوع بُقِّي را آورد و بُقِّي عُزّي را آورد.
\par 6 و عُزّي زَرَحيا را آورد و زَرَحيا مَرايوت را آورد.
\par 7 ومَرايوت اَمَريا را آورد و اَمَريا اخيطوب را آورد.
\par 8 و اخيطوب صادوق را آورد و صادوق اَخِيمَعص را آورد.
\par 9 واَخِيمَعص عَزَريا را آورد و عَزَريا يوحانان را آورد.
\par 10 و يوحانان عَزَريا را آورد و او در خانه اي که سليمان در اورشليم بنا کرد، کاهن بود.
\par 11 و عَزَريا اَمَريا را آورد و اَمَريا اخيطوب را آورد.
\par 12 و اَخيطوب صادوق را آورد و صادوق شَلّوم را آورد.
\par 13 و شَلّوم حِلقيا را آورد و حِلقّيا عَزَريا را آورد
\par 14 و عَزَريا سَرايا را آورد و سرايا يهُوصاداق را آورد.
\par 15 و يهُوصاداق به اسيري رفت هنگامي که خداوند يهودا و اورشليم را به دست نَبُوکَدنَصَّر اسير ساخت.
\par 16 پسران لاوي: جرشوم و قَهات و مَراري.
\par 17 و اينها است اسمهاي پسران جَرشُوم: لِبنِي و شِمعي.
\par 18 و پسران قَهات: عَمرام و يصهار و حبرون و عُزّيئيل.
\par 19 و پسران مَراري: مَحلي و موشي پس اينها قبايل لاويان بر حسب اجداد ايشان است.
\par 20 از جَرشُوم پسرش لِبنِي، پسرش يحَت، پسرش زِمَّه.
\par 21 پسرش يوآخ پسرش عِدُّو پسرش زارَح پسرش ياتراي.
\par 22 پسران قَهات، پسرش عَمِّيناداب پسرش قُورَح پسرش اَسّير.
\par 23 پسرش اَلقانَه پسرش اَبيآ ساف پسرش اَسّير.
\par 24 پسرش تَحَت پسرش اُوريئيل پسرش عُزّيا، پسرش شاؤل.
\par 25 و پسران اَلقانَه عماساي و اَخيمُوت.
\par 26 و امّا اَلقانَه. پسران اَلقانَه پسرش صوفاي پسرش نَحَت.
\par 27 پسرش اَلِيآب پسرش يرُوحام پسرش اَلقانَه.
\par 28 و پسران سموئيل نخست زاده اش وَشنِي و دومش اَبِيا.
\par 29 پسران مَراري مَحلي و پسرش لِبنِي پسرش شِمعي پسرش عُزَّه.
\par 30 پسرش شِمعِي پسرش هَجيا پسرش عَسايا.
\par 31 و اينانند که داود ايشان را بر خدمت سرود در خانه خداوند تعيين نمود بعد از آنکه تابوت مستقر شد.
\par 32 و ايشان پيش مسکن خيمه اجتماع مشغول سراييدن مي شدند تا حيني که سليمان خانه خداوند را در اورشليم بنا کرد. پس بر حسب قانون خويش بر خدمت خود مواظب شدند.
\par 33 پس آنهايي که با پسران خود معين شدند، اينانند: از بني قَهاتيان هِمانِ مغنّي ابن يوئيل بن سموئيل.
\par 34 بن اَلقانَه بن يرُوحام بن اَليئيل بن نُوح،
\par 35 ابن صوف بن اَلقانَه بن مَهت بن عماساي،
\par 36 ابن اَلقانَه بن يوئيل بن عَزَرياء بن صَفَنيا،
\par 37 ابن تَحَت بن اَسّير بن اَبيآ ساف بن قُورح،
\par 38 ابن يصهار بن قَهات بن لاوي بن اسرائيل.
\par 39 و برادرش آساف که به دست راست وي مي ايستاد. آساف بن بَرَکيا ابن شِمعِي، 
\par 40 ابن ميکائيل بن بَعسِيا ابن مَلکِيا،
\par 41 ابن اَتنِي ابن زارَح بن عَدايا،
\par 42 ابن اِيتان بن زِمَّه بن شِمعِي،
\par 43 ابن يحت بن جَرشُوم بن لاوي.
\par 44 و به طرف چپ برادران ايشان که پسران مَراري بودند: اِيتان بن قيشي ابن عَبدِي ابن مَلّوک،
\par 45 ابن حَشَبيا ابن اَمَصيا ابن حِلقِيا،
\par 46 ابن اَمصِي ابن باني ابن شامَر،
\par 47 ابن مَحلِي ابن موشي ابن مَراري ابن لاوي.
\par 48 و لاوياني که برادران ايشان بودند، به تمامي خدمت مسکن خانه خدا گماشته شدند.
\par 49 و اما هارون و پسرانش بر مذبح قرباني سوختني و بر مذبح بخور به جهت تمامي عمل قدس الاقداس قرباني مي گذرانيدند تا به جهت اسرائيل موافق هر آنچه موسي بنده خدا امر فرموده بود، کفاره نمايند.
\par 50 و اينانند پسران هارون: پسرش اَلعازار، پسرش فينَحاس، پسرش اَبِيشُوع.
\par 51 پسرش بُقي، پسرش عُزّي، پسرش زَرَحيا،
\par 52 پسرش مَرايوت پسرش اَمَريا پسرش اَخيطوب،
\par 53 پسرش صادوق، پسرش اَخِيمَعص.
\par 54 و مسکن هاي ايشان بر حسب موضع ها و حدود ايشان اينها است: از پسران هارون به جهت قبايل قَهاتيان زيرا قرعه اوّل از آنِ ايشان بود.
\par 55 پس حَبرُون در زمين يهودا با حوالي آن به هر طرفش به ايشان داده شد.
\par 56 و اما زمينهاي آن شهر دهاتش را به کاليب بن يفُنَّه دادند.
\par 57 به پسران هارون به جهت شهرهاي ملجا حَبرُون و لِبنَه و حوالي آن، و يتّير و اَشتَموع و حوالي آن را دادند.
\par 58 و حِيلين و حوالي آن را و دَبير و حوالي آن را،
\par 59 و عاشان و حوالي آن را و بيت شمس و حوالي آن را،
\par 60 و از بسط بنيامين جَبَع و حوالي آن را و عَلَّمَت و حوالي آن را و عَناتوت و حوالي آن را. پس جميع شهرهاي ايشان بر حسب قبايل ايشان سيزده شهر بود.
\par 61 و به پسران قَهات که از قبايل آن سبط باقي ماندند، ده شهر از نصف سبط يعني از نصف مَنَّسي به قرعه داده شد.
\par 62 و به بني جَرشُوم بر حسب قبايل ايشان از بسط يسّاکار و از سبط اَشير و از سبط نَفتالي و از سبط مَنَّسي درباشان سيزده شهر.
\par 63 و به پسران مَراري بر حسب قبايل ايشان از بسط رؤبين و از سبط جاد و از سبط زبولون دوازده شهر به قرعه داده شد.
\par 64 پس بني اسرائيل اين شهرها را با حوالي آنها به لاويان دادند.
\par 65 و از بسط بني يهودا و از بسط بني شَمعون و از بسط بني بنيامين اين شهرها را که اسم آنها مذکور است به قرعه دادند.
\par 66 و بعضي از قبايل بني قَهات شهرهاي حدود خود را از سبط افرايم داشتند.
\par 67 پس شکيم را با حوالي آن در کوهستان افرام و جازَر را با حوالي آن به جهت شهرهاي ملجا به ايشان دادند.
\par 68 و يقمَعام را با حوالي آن و بيت حُورُون را با حوالي آن.
\par 69 و اَيلُون را با حوالي آن و جَتّ رِمُّون را با حوالي آن.
\par 70 و از نصف سبط مَنَّسي، عانير را با حوالي آن، و بِلعام را با حوالي آن، به قبايل باقي مانده بقي قَهات دادند.
\par 71 و به پسران جَرشُوم از قبايل نصف سبط مَنَّسي، جُولان را در باشان با حوالي آن و عَشتارُوت را با حوالي آن.
\par 72 و از سبط يسّاکار قادِش را با حوالي آن و دَبَرَه را با حوالي آن.
\par 73 و راموت را با حوالي آن و عانيم را با حوالي آن.
\par 74 و از بسط اَشير مَشآل را با حوالي آن عَبدُون را با حوالي آن.
\par 75 و حُقُوق را با حوالي آن و رَحُوب را با حوالي آن.
\par 76 و از سبط نَفتالي قادِش را در جَلِيل با حوالي آن و حَمّون را با حوالي آن و قِريتايم را با حوالي آن.
\par 77 و به پسران مَراري که از لاويان باقي مانده بودند، از سبط زبولون رِمّون را با حوالي آن و تابور را با حوالي آن.
\par 78 و از آن طرف اُردُن در برابر اريحا به جانب شرقي اُردُن از سبط رؤبين، باصَر را در بيابان با حوالي آن و يهصَه را با حوالي آن.
\par 79 و قديموت را با حوالي آن و مَيفَعَه را با حوالي آن.
\par 80 و از سبط جاد راموت را در جِلعاد با حوالي آن و مَحَنايم را با حوالي آن.و حَشبون را با حوالي آن و يعزير را با حوالي آن.
\par 81 و حَشبون را با حوالي آن و يعزير را با حوالي آن.
 
\chapter{7}

\par 1 و اما پسران يسّاکار: تولاع و فُوَه و ياشوب و شِمرُون چهار نفر بودند.
\par 2 و پسران تولاع: عُزِّي و رفايا و يربيئيل و يحماي و يبسام و سموئيل؛ ايشان رؤساي خاندان پدر خود تولاع و مردان قوي شجاع در انساب خود بودند، و عدد ايشان در داود بيست و دو هزار و ششصد بود.
\par 3 و پسر عُزّي، يزرَحيا و پسران يزرَحيا، ميکائيل و عُوبَديا و يوئيل و يشِيا که پنج نفر و جميع آنها رؤسا بودند.
\par 4 و با ايشان بر حسب انساب ايشان و خاندان آباي ايشان، فوجهاي لشکر جنگي سي و شش هزار نفر بودند، زيرا که زنان وپسران بسيار داشتند.
\par 5 و برادران ايشان از جميع قبايل يسّاکار مردان قوي شجاع هشتاد و هفت هزار نفر جميعاً در نسب نامه ثبت شدند.
\par 6 و پسران بنيامين: بالَع و باکَر ويدِيعَئيل، سه نفر بودند.
\par 7 و پسران بالَع: اَصبُون و عُزّي و عُزّيئيل و يريموت و عِيرِي، پنج نفر رؤساي خاندان آبا و مردان قوي شجاع که بيست ودو هزار و سي و چهار نفر از ايشان در نسب نامه ثبت گرديدند.
\par 8 و پسران باکَر: زَمِيرَه و يوعاش و اَلِيعازار و اَليوعِيناي و عُمرِي و يريموت و اَبِيا و عَناتُوت وعَلامَت. جميع اينها پسران باکَر بودند.
\par 9 و بيست هزار ودويست نفر از ايشان بر حسب انساب ايشان، رؤساي خاندان آباي ايشان مردان قوي شجاع در نسب نامه ثبت شدند.
\par 10 و پسر يدِيعَئيل: بِلهان و پسران بِلهان: يعِيش و بنيامين و اِيهُود و کَنَعنَه و زِيتان و تَرشِيش و اَخِيشاحَر.
\par 11 جميع اينها پسران يدِيعَئيل بر حسب رؤساي آبا و مردان جنگي شجاع هفده هزار و دويست نفر بودند که در لشکر براي جنگ بيرون مي رفتند.
\par 12 و پسران عِير: شُفّيم و حُفّيم و پسر اَحِير حُوشيم.
\par 13 و پسران نَفتالي: يحصيئيل و جُوني و يصَر و شَلُّوم از پسران بِلهَه بودند.
\par 14 پسران مَنَّسي اَسريئيل که زوجه اش او را زاييد، و ماکير پدر جِلعاد که مُتعه اَراميه وي او را زاييد.
\par 15 و ماکير خواهر حُفّيم و شُفّيم را که به مَعکَه مسماة بود، به زني گرفت؛ و اسم پسر دوم او صَلُفحاد بود؛ و صَلُفحاد دختران داشت.
\par 16 و مَعکَه زن ماکير پسري زاييده، او را فارَش نام نهاد و اسم برادرش شارَش بود و پسرانش، اُلام و راقَم بودند.
\par 17 و پسر اُولام بِدان بود. اينانند پسران جِلعاد بن ماکير بن مَنَّسي.
\par 18 و خواهر او هَمُولَکَه اِيشهُود و ابَيعَزَر و مَحلَه را زاييد.
\par 19 و پسران شَميداع اَخيان و شَکيم و لَقحِي و اَنيعام.
\par 20 و پسران افرايم شُوتالَح و پسرش بارَد و پسرش تَحَت و پسرش اَلِعادا و پسرش تَحَت.
\par 21 و پسرش زاباد و پسرش شُوتالَح و عازَر و اَلِعاد که مردان جَتّ که در آن زمين مولود شدند، ايشان را کشتند زيرا که براي گرفتن مواشي ايشان فرمود آمده بودند.
\par 22 و پدر ايشان افرايم به جهت ايشان روزها ي بسيار ماتم گرفت و برادرانش براي تعزيت وي آمدند.
\par 23 پس نزد زن خود درآمد و او حامله شده، پسري زاييد و او را بَرِيعَه نام نهاد، از اين جهت که در خاندان او بلايي عارض شده بود.
\par 24 و دخترش شيره بود که بيت حورون پايين و بالا را و اُزِّين شِيرَه را بنا کرد.
\par 25 و پسرش رافَح و راشَف، و پسرش تالَح، و پسرش تاحَن،
\par 26 و پسرش لَعَدان، و پسرش عَميهُود، و پسرش اَلِيشَمَع،
\par 27 و پسرش نون، و پسرش يهُوشوع،
\par 28 و املاک و مسکن هاي ايشان بيت ئيل و دهات آن بود و به طرف مشرق نَعران و به طرف مغرب جازَر و دهات آن و شکيم و دهات آن تا غَزَّه و دهات آن.
\par 29 و نزد حدود بني مَنَّسي بيت شان و دهات آن و تَعناک و دهات آن و مَجِدّو و دهات آن و دُور و دهات آن که در اينها پسران يوسف بن اسرائيل ساکن بودند.
\par 30 پسران اَشِير، يمنَه و يشوَه و يشوي و بَريعه بودند، و خواهر ايشان سارَح بود.
\par 31 و پسران بَريعَه، حابَر و مَلکيئيل که همان پدر بِرزاوَت باشد.
\par 32 حابَر، يفلِيط و شومير و حُوتام و خواهر ايشان شوعا را آورد.
\par 33 و پسران يفليط فاسَک و بِمهال و عَشوَت بودند. اينانند بني يفليط.
\par 34 و پسران شامَراَخي و رُهجَه و يحُبَّه و اَرام.
\par 35 و پسران هيلام برادر وي صُوفَح و يمناع و شالَش و عامال بودند.
\par 36 و پسران صُوفَح، سُوح و حَرنَفَر و شُوعال و بِيرِي و يمرَه.
\par 37 و باصَر و هُود و شَمَّا و شَلَشَه و يتران و بئِيرا.
\par 38 و پسران يتَر، يفُنَّه و فِسفا و آرا.
\par 39 و پسران عُلاّ، آرَح و حَنِّيئيل و رَصِيا.جميع اينها پسران اَشِير و رؤساي خاندان آباي ايشان و برگزيدگان مردان جنگي و رؤساي سرداران بودند. و شماره ايشان که در لشکر براي جنگ بر حسب نسب نامه ثبت گرديد، بيست و شش هزار نفر بود.
\par 40 جميع اينها پسران اَشِير و رؤساي خاندان آباي ايشان و برگزيدگان مردان جنگي و رؤساي سرداران بودند. و شماره ايشان که در لشکر براي جنگ بر حسب نسب نامه ثبت گرديد، بيست و شش هزار نفر بود.
 
\chapter{8}

\par 1 و بنيامين نخست زاده خود بالَع را آورد و دومين اَشبيل و سومش اَخرَخ،
\par 2 و چهارم نُوحَه و پنجم را فارا.
\par 3 و پسران بالَع: اَدّار و جِيرا و اَبِيهُود.
\par 4 و اَبيشُوع و نُعمان و اَخُوخ.
\par 5 و جيرا و شَفُوفان و حُورام بودند.
\par 6 و اينانند پسران اَحُود که رؤساي خاندان آباي ساکنان جَبَع بودند و ايشان را به مناحت کوچانيدند.
\par 7 و او نُعمان و اَخِيا و جيرا را کوچانيد و او عُزّا و اَخِيحُود را توليد نمود.
\par 8 و شَحرايم در بلاد موآب بعد از طلاق دادن زنان خود حُوشيم و بَعَرا فرزندان توليد نمود.
\par 9 پس از زن خويش که خُوداش نام داشت يوباب و ظبيا و ميشا و مَلکام را آورد.
\par 10 و يعُوض و شَکِيا و مِرمَه را که اينها پسران او و رؤساي خاندانهاي آبا بودند.
\par 11 و از حوشيم ابيطوب و اَلفَعل را آورد.
\par 12 و پسران اَلفَعل عابَر و مِشعام و شامَر که اُونُو و لُود و دهاتش را بنا نهاد بودند.
\par 13 و بَرِيعه و شامع که ايشان رؤساي خاندان آباي ساکنان اَيلُون بودند و ايشان ساکنان جَّت را اخراج نمودند.
\par 14 و اَخِيو و شاشَق و يرِيمُوت.
\par 15 و زَبَديا و عارَد و عادَر.
\par 16 و ميکائيل و يشفَه و يوخا پسران بَريعه بودند.
\par 17 و زَبَديا و مَشُلاّم و جِزقِي و حابَر،
\par 18 و يشمَراي و يزلِيآه و يوباب پسران اَلفَعل بودند.
\par 19 و يعقيم و زِکرِي و زَبدِي،
\par 20 و اَلِيعيناي و صِلَّتاي و ايليئيل،
\par 21 و اَدايا و بريا و شِمرَت پسران شِمعي،
\par 22 و يشفان و عابَر و ايليئيل.
\par 23 و عَبدون و زِکرِي و حانان،
\par 24 و حَنَنيا و عيلام و عَنتُوتِيا،
\par 25 و يفَديا و فَنُوئيل پسران شاشَق بودند.
\par 26 و شِمشَراي و شَحَريا و عَتَليا.
\par 27 و يعرَشيا و ايليا و زِکرِي پسران يرُحام بودند.
\par 28 اينان رؤساي خاندان آبا بر حسب انساب خود و سرداران بودند و ايشان در اورشليم سکونت داشتند.
\par 29 و در جِبعُون پدر جِبعُون سکونت داشت و اسم زنش مَعکَه بود.
\par 30 و نخست زاده اش عَبدُون بود، پس صور و قَيس و بَعل و ناداب،
\par 31 و جَدُور و اَخِيو و زاکَر؛
\par 32 و مِقلُوت شِمآه را آورد و ايشان نيز با برادران خود در اورشليم در مقابل برادران ايشان ساکن بودند.
\par 33 و نير قَيس را آورد و قَيس شاؤل را آورد و شاؤل يهُوناتان و مَلکيشوع و ابيناداب و اَشبَعل را آورد.
\par 34 و پسر يهُوناتان مَرِيب بَعل بود و مَرِيبِ بَعل ميکا را آورد.
\par 35 و پسران ميکا، فيتون و مالَک و تاريع و آحاز بودند.
\par 36 و آحاز يهُوعَدَه را آورد، يهُوعَدَه عَلمَت و عَزمُوت و زِمري را آورد و زِمري موصا را آورد.
\par 37 و موصا بِنعا را آورد و پسرش رافَه بود و پسرش اَلعاسَه و پسرش آصيل بود. 
\par 38 و آصيل را شش پسر بود و نامهاي ايشان اينها است: عَزرِيقام و بُکرُو و اِسمَعِيل و شَعريا و عُوبَديا و حانان. و جميع اينها پسران آصيل اند.
\par 39 و پسران عِيشَق برادر او نخست زاده اش اُولام و دومين يعُوش و سومين اَلِيفَلَط.و پسران اُولام، مردان زورآورِ شجاع و تيرانداز بودند؛ و پسران و پسرانِ پسران ايشان بسيار يعني صد و پنجاه نفر بودند. جميع اينها از بني بنيامين ميباشند.
\par 40 و پسران اُولام، مردان زورآورِ شجاع و تيرانداز بودند؛ و پسران و پسرانِ پسران ايشان بسيار يعني صد و پنجاه نفر بودند. جميع اينها از بني بنيامين ميباشند.
 
\chapter{9}

\par 1 و تمامي اسرائيل بر حسب نسب نامه هاي خود شمرده شدند، و اينک در کتاب پادشاهان اسرائيل مکتوب اند و يهودا به سبب خيانت خود به بابل اسيري رفتند.
\par 2 و کساني که اول در مُلکها و شهرهاي ايشان سکونت داشتند، اسرائيليان و کاهنان و لاويان و نَتينيم بودند.
\par 3 و در اورشليم بعضي از بني يهودا واز بني بنيامين و از بني افرايم و مَنَّسي ساکن بودند.
\par 4 عوتاي ابن عمَّيهُود بن عُمرِي ابن اِمرِي ابن باني از بني فارَص بن يهودا.
\par 5 و از شيلونيان نخست زاده اش عسايا و پسران او.
\par 6 و از بني زارَح يعُوئيل و براران ايشان ششصد و نود نفر.
\par 7 و از بني بنيامين سَلُّو ابن مَشُلاّمِ بن هُودُويا ابن هَسنُوآه.
\par 8 و يبنِيا ابن يرُوحام و اِيلَة بن عُزِّي ابن مِکرِي و مَشُلاّم بن شَفَطيا بن رَاؤئيل بن يبنِيا.
\par 9 و برادران ايشان بر حسب انساب ايشان نه صد و پنجاه و شش نفر. جميع اينها رؤساي اجداد بر حسب خاندانهاي آباي ايشان بودند.
\par 10 و از کاهنان، يدَعيا و يهُوياريب و ياکين،
\par 11 عَزَريا ابن حِلقيا ابن مَشُلاّم بن صادوق بن مَرايوت بن اَخيطُوب رئيس خانه خدا،
\par 12 و عَدايا ابن يرُوحام بن فَشحُور بن مَلکيا و مَعَساي ابن عَديئيل بن يحزيره بن مَشُلاّم بن مَشِليمِيت بن اِمّير.
\par 13 و برادران ايشان که رؤساي خاندان آباي ايشان بودند، هزار و هفتصد شصت نفر که مردان رشيد به جهت عمل خدمت خانه خدا بودند.
\par 14 و از لاويان شَمَعيا ابن حَشُّوب بن عَزريقام بن حَشَبيا از بني مَراري.
\par 15 و بَقبَقَّر و حارَش و جَلال و مَتَنيا ابن ميکا ابن زِکرِي ابن آساف.
\par 16 و عَوبَديا ابن شَمَعيا ابن جَلال و بن يدُوتُون و بَرخِيا ابن آسا ابن اَلقانَه که در دهات نَطُوفاتيان ساکن بود.
\par 17 و در بانان، شَلُّوم و عَقُّوب و طَلمُون و اَخِيمان و برادران ايشان. و شَلُّوم رئيس بود.
\par 18 و ايشان تا الآن بر دروازه شرقي پادشاه (مي باشند) و در بانانِ فرقه بني لاوي بودند.
\par 19 و شَلُّوم بن قُورِي ابن اَبيآ ساف بن قُورَح و برادرانش از خاندان پدرش يعني بني قُورَح که ناظران عمل خدمت و مستحفظان دروازه هاي خيمه بودند و پدران ايشان ناظران اُردوي خداوند و مستحفظان مدخل آن بودند.
\par 20 و فينَحاس بن اَلعازار، سابق رئيس ايشان بود. و خداوند با وي مي بود.
\par 21 و زکريا ابن مَشَلَمِيا دربان دروازه خيمه اجتماع بود.
\par 22 و جميع ايناني که براي درباني دروازه ها منتخب شدند، دويست دوازده نفر بودند و ايشان در دهات خود بر حسب نسب نامه هاي خود شمرده شدند که داود و سموئيلِ رائي ايشان را بر وظيفه هاي ايشان گماشته بودند.
\par 23 پس ايشان و پسران ايشان بر دروازه هاي خانه خداوند و خانه خيمه براي نگهباني آن گماشته شدند.
\par 24 و دربانان به هر چهار طرف يعني مشرق و مغرب و شمال و جنوب بودند.
\par 25 و برادران ايشان که در دهات خود بودند، هر هفت روز نوبت به نوبت با ايشان مي آمدند.
\par 26 زيرا چهار رئيس دربانان که لاويان بودند، منصب خاصّ داشتند و ناظرانِ حجره ها و خزانه هاي خانه خدا بودند.
\par 27 و به اطراف خانه خدا منزل داشتند زيرا که نگاه بانيش بر ايشان بود، و باز کردن آن هر صبح بر ايشان بود.
\par 28 و بعضي از ايشان بر آلات خدمت مأمور بودند، چونکه آنها را به شماره مي آوردند و به شماره بيرون مي بردند.
\par 29 از ايشان بر اسباب و جميع آلات قدس و آردِ نرم و شراب و روغن و بخور و عطريات مأمور بودند.
\par 30 و بعضي از پسران کاهنان، عطريات خوشبو را ترکيب مي کردند.
\par 31 و مَتَّتيا که از جمله لاويان و نخست زاده شَلُّوم قُورَحي بود، بر عمل مطبوخات گماشته شده بود.
\par 32 و بعضي از برادران ايشان از پسران قَهاتيان، بر نان تَقدِمه مأمور بودند تا آن را در هر روز سبت مهيا سازند.
\par 33 و مغنيان از رؤساي خاندان آباي لاويان در حجره ها سکونت داشتند و از کار ديگر فارغ بودند زيرا که روز و شب در کار خود مشغول مي بودند.
\par 34 اينان رؤساي خاندان آباي لاويان و بر حسب انساب خود رئيس بودند و در اورشليم سکونت داشتند.
\par 35 و در جِبعُون، پدر جِبعُون، يعوئيل سکونت داشت و اسم زنش مَعکَه بود.
\par 36 و نخست زاده اش عَبدون بود، پس صور و قَيس و بَعل و نير و ناداب،
\par 37 و جَدُور و اَخِيو و زَکريا و مِقلُوت؛
\par 38 و مِقلُوت شِمآم را آورد و ايشان نيز با برادران خود در اورشليم در مقابل برادران ايشان ساکن بودند.
\par 39 و نير قَيس را آورد و قَيس شاؤل را آورد و شاؤل يهُوناتان و مَلکيشوع و ابيناداب و اَشبَعل را آورد.
\par 40 و پسر يهُوناتان، مَريب بَعل بود و مَريب بَعل ميکا را آورد.
\par 41 و پسر ميکا، فيتون و مالَک و تَحريع و آحاز بودند.
\par 42 و آحاز يعرَه را آورد و يعره عَلمَت و عَزمُوت و زِمرِي را آورد و زِمرِي موصا را آورد.
\par 43 و موصا بِنعا را آورد و پسرش رفايا و پسرش اَلعاسَه و پسرش آصيل.و آصيل را شش پسر بود و اين است نامهاي ايشان: عَزرِيقام و بُکرُو و اسمَعيل و شَعَريا و عُوبَديا و حانان اينها پسران آصيل مي باشند.
\par 44 و آصيل را شش پسر بود و اين است نامهاي ايشان: عَزرِيقام و بُکرُو و اسمَعيل و شَعَريا و عُوبَديا و حانان اينها پسران آصيل مي باشند.
 
\chapter{10}

\par 1 و فلسطينيان با اسرائيل جنگ کردند، و مردان اسرائيل از حضور فلسطينيان فرار کردند و در کوه جِلبُوع کشته شده، افتادند.
\par 2 و فلسطينيان شاؤل و پسرانش را به سختي تعاقب نمودند، و فلسطينيان پسران شاؤل يوناتان و ابيناداب و مَلکيشوع را کشتند.
\par 3 و جنگ بر شاؤل سخت شد و تيراندازان او را دريافتند و از تيراندازان مجروح شد.
\par 4 و شاؤل به سلاحدار خود گفت: «شمشير را بکش و به من فرو بر، مبادا اين نامختونان بيايند و مرا افتضاح کنند.» اما سلاحدارش نخواست زيرا که بسيار مي ترسيد؛ پس شاؤل شمشير را گرفته بر آن افتاد.
\par 5 و سلاحدارش چون شاؤل را مرده ديد، او نيز بر شمشير افتاده، بمُرد.
\par 6 و شاؤل مُرد و سه پسرش و تمامي اهل خانه اش همراه وي مردند.
\par 7 و چون جميع مردان اسرائيل که در وادي بودند، اين را ديدند که لشکر منهزم شده، و شاؤل و پسرانش مرده اند، ايشان نيز شهرهاي خود را ترک کرده، گريختند و فلسطينيان آمده، در آنها قرار گرفتند.
\par 8 و روز ديگر واقع شد که چون فلسطينيان آمدند تا کشتگان را برهنه نمايند، شاؤل و پسرانش را در کوه جِلبُوع افتاده يافتند.
\par 9 پس او را برهنه ساخته، سَر و اسلحه اش را گرفتند و آنها را به زمين فلسطينيان به هر طرف فرستادند تا به بتها و قوم خود مژده برسانند.
\par 10 و اسلحه اش را در خانه خدايان خود گذاشتند و سرش را در خانه داجون به ديوار کوبيدند.
\par 11 و چون تمامي اهل يابيش جِلعاد آنچه را که فلسطينيان به شاؤل کرده بودند شنيدند،
\par 12 جميع شجاعان برخاسته، جسد شاؤل و جسد هاي پسرانش را برداشته، آنها را به يابيش آورده، استخوانهاي ايشان را زير درخت بلوط که در يابيش است، دفن کردند و هفت روز روزه داشتند.
\par 13 پس شاؤل به سبب خيانتي که به خداوند ورزيده بود مُرد، به جهت کلام خداوند که آن را نگاه نداشته بود، و از اين جهت نيز که از صاحبه اجنّه سؤال نموده بود.و چونکه خداوند را نطلبيده بود، او را کُشت و سلطنت او را به داود بن يسَّي برگردانيد.
\par 14 و چونکه خداوند را نطلبيده بود، او را کُشت و سلطنت او را به داود بن يسَّي برگردانيد.
 
\chapter{11}

\par 1 و تمامي اسرائيل نزد داود در حَبرُون جمع شده، گفتند: « اينک ما استخوانها و گوشت تو مي باشيم.
\par 2 و قبل از اين نيز هنگامي که شاؤل پادشاه مي بود، تو اسرائيل را بيرون مي بردي و درون مي آوردي؛ و يهُوَه خدايت تو را گفت که: تو قوم من اسرائيل را شباني خواهي نمود و تو بر قوم من اسرائيل پيشوا خواهي شد.»
\par 3 و جميع مشايخ اسرائيل نزد پادشاه به حَبرُون آمدند و داود با ايشان به حضور خداوند در حَبرُون عهد بست، و داود را بر حسب کلامي که خداوند به واسطه سموئيل گفته بود به پادشاهي اسرائيل مسح نمودند.
\par 4 و داود و تمامي اسرائيل به اورشليم که يبُوس باشد، آمدند و يبُوسيان در آن زمين ساکن بودند.
\par 5 و اهل يبُوس به داود گفتند:« به اينجا داخل نخواهي شد.» اما داود قلعه صَهيون را که شهر داود باشد بگرفت.
\par 6 و داود گفت:« هر که يبُوسيان را اول مغلوب سازد، رئيس و سردار خواهد شد.» پس يوآب بن صَرُويه اول بر آمد و رئيس شد.
\par 7 و داود در آن قلعه ساکن شد، از آن جهت آن را شهر داود ناميدند.
\par 8 و شهر را به اطراف آن و گرداگرد مِلُّوه بنا کرد و يوآب باقي شهر را تعمير نمود.
\par 9 و داود ترقي کرده، بزرگ مي شد و يهُوَه صبايوت با وي مي بود.
\par 10 و اينانند رؤساي شجاعاني که داود داشت که با تمامي اسرائيل او را در سلطنتش تقويت دادند تا او را بر حسب کلامي که خداوند درباره اسرائيل گفته بود پادشاه سازد.
\par 11 و عدد شجاعاني که داود داشت اين است: يشُبعام بن حَکوُني که سردار شليشيم بود که بر سيصد نفر نيزه خود را حرکت داد و ايشان را در يک وقت کشت.
\par 12 و بعد از او اَلِعازار بن دُودُوي اَخُوخي که يکي از آن سه شجاع بود.
\par 13 او با داود در فَسدَمِيم بود وقتي که فلسطينيان در آنجا براي جنگ جمع شده بودند، و قطعه زمين پُر از جو بود، و قوم از حضور فلسطينيان فرار مي کردند.
\par 14 و ايشان در ميان آن قطعه زمين ايستاده، آن را محافظت نمودند، و فلسطينيان را شکست دادند و خداوند نصرت عظيمي به ايشان داد.
\par 15 و سه نفر از آن سي سردار به صخره نزد داود به مغاره عَدُلاّم فرود شدند و لشکر فلسطينيان در وادي رفائيم اردو زده بودند.
\par 16 و داود در آن وقت در ملاذ خويش بود، و قراول فلسطينيان آن وقت در بيت لحم بودند.
\par 17 و داود خواهش نموده، گفت:« کاش کسي مرا از آب چاهي که نزد دروازه بيت لحم است بنوشاند.»
\par 18 پس آن سه مرد، لشکر فلسطينيان را از ميان شکافته، آب را از چاهي که نزد دروازه بيت لحم است کشيده، برداشتند و آن را نزد داود آوردند؛ اما داود نخواست که آن را بنوشد و آن را به جهت خداوند بريخت،
\par 19 و گفت:« اي خداوند من حاشا از من که اين کار را بکنم! آيا خون اين مردان را بنوشم که جان خود را به خطر انداختند زيرا به خطر جان خود آن را آوردند؟» پس نخواست که آن را بنوشد؛ کاري که اين سه مرد شجاع کردند اين است.
\par 20 و اَبِيشاي برادر يوآب سردار آن سه نفر بود و او نيز نيزه خود را بر سيصد نفر حرکت داده، ايشان را کشت و در ميان آن سه نفر اسم يافت.
\par 21 در ميان آن سه نفر از دو مکرّم تر بود؛ پس سردار ايشان شد، ليکن به سه نفر اول نرسيد.
\par 22 و بنايا ابن يهُوياداع پسر مردي شجاع قَبصيئيلي بود که کارهاي عظيم کرده بود، و پسر اَرِيئيل موآبي را کشت و در روز برف به حفره اي فرود شده، شيري را کشت.
\par 23 و مرد مصري بلند قد را که قامت او پنج ذارع بود کشت، و آن مصري در دست خود نيزه اي مثل نورد نساجان داشت؛ اما او نزد وي با چوب دستي رفت و نيزه را از دست مصري ربوده، وي را با نيزه خودش کشت.
\par 24 بنايا ابن يهُوياداع اين کارها را کرد و در ميان آن سه مرد شجاع اسم يافت.
\par 25 اينک اواز آن سي نفر مکرّم تر شد، ليکن به آن سه نفر اول نرسيد و داود او را بر اهل مشورت خود برگماشت. 
\par 26 و نيز از شجاعان لشکر، عسائيل برادر يوآب و اَلحانان بن دُودُوي بيت لحمي،
\par 27 و شَمُّوتِ هَرُورِي وحالَصِ فَلُوني،
\par 28 و عيرا ابن عِقّيشِ تَقُوعي و اَبيعَزَر عَناتُوتي،
\par 29 و سِبکاي حُوشاتي و عيلاي اَخُوخي،
\par 30 و مَهراي نَطُوفاتِي و خالَد بن بَعَنَه نَطُوفاتِي،
\par 31 و اِتّاي ابن ريباي از جِبعه بني بنيامين و بناياي فَرعاتُونِي،
\par 32 و حُوراي از واديهاي جاعَش و اَبيئيلِ عَرُباتِي،
\par 33 و عَزموتِ بَحرُومي و اَيحَباي شَعَلبُونِي.
\par 34 و از بني هاشَمِ جِزُوني يوناتان بن شاجاي هَراري،
\par 35 و اَخيام بن ساکارِ هَراري و اليفال بن اُور،
\par 36 و حافَر مَکيراتي و اَخِياي فَلوني،
\par 37 و حَصرُوي کَرمَلي و نَعراي ابن اَزباي.
\par 38 و يوئيل برادر ناتان و مَبحار بن هَجرِي،
\par 39 و صالَقِ عَمُّوني ونحراي بِيرُوتي که سلاحدار يوآب بن صَرُويه بود.
\par 40 و عيراي يترِي و جارَبِ يترِي،
\par 41 و اُورياي حِتِّي و زاباد بن اَحلاي،
\par 42 و عَدينا ابن شيزاي رؤبيني که سردار رؤبينيان بود و سي نفر همراهش بودند.
\par 43 و حانان بن مَعکَه و يوشافاط مِتنِي،
\par 44 و عُزّياي عَشتَرُوتِي و شاماع و يعُوئيل پسران حُوتامِ عَرُوعِيرِي،
\par 45 و يدِيعيئيل بن شِمرِي و برادرش يوخاي تِيصي،
\par 46 و اَلِيئيل از مَحُويم و يربياي يو شُويا پسران اَلناعَم و يتمَه موآبي،و اَليئيل و عَوبيد و يعسِيئيلِ مَصُوباتي.
\par 47 و اَليئيل و عَوبيد و يعسِيئيلِ مَصُوباتي.
 
\chapter{12}

\par 1 و اينانند که نزد داود به صِقلَغ آمدند، هنگامي که او هنوز از ترس شاؤل بن قَيس گرفتار بود، و ايشان از آن شجاعان بودند که در جنگ معاون او بودند.
\par 2 و به کمان مسلح بودند و سنگها و تيرها از کمانها از دست راست و دست چپ مي انداختند و از برادران شاؤل بنياميني بودند.
\par 3 سردار ايشان اَخيعَزَر بود، و بعد از او يوآش پسران شَماعَه جِبعاتي و يزيئيل و فالَط پسران عَزمُوت و بَراکَه و ييهُوي عناتوتي،
\par 4 و يشمَعياي جِبعُوني که در ميان آن سي نفر شجاع بود، و بر آن سي نفر برتري داشت و اِرميا و يحزيئيل و يوحانان و يوزابادِ جَديراتي،
\par 5 و اَلعُوزاي و يريموت وبَعلِيا و شَمَريا و شَفَطياي حَرُوفي،
\par 6 و اَلقانَه و يشَيا و عَزَرئيل و يوعَزَر و يشُبعام که از قُورَحيان بودند،
\par 7 ويوعيلَه و زَبديا پسران يرُوحام جَدوري.
\par 8 و بعضي از جاديان که مردان قوي شجاع و مردان جنگ آزموده و مسلح به سپر و تيراندازان که روي ايشان مثل روي شير و مانند غزال کوهي تيزرو بودند، خوشتن را نزد داود در ملاذ بيابان جدا ساختند،
\par 9 که رئيس ايشان عازَر و دومين عُوبَديا و سومين اَلِيآب بود،
\par 10 و چهارمين مِشمَنَّه و پنجمين اِرميا،
\par 11 و ششم عَتّاي و هفتم اَلِيئيل،
\par 12 و هشتم يوحانان و نهم اَلزاباد،
\par 13 و دهم اِرميا و يازدهم مَکبَنَّاي.
\par 14 اينان از بني جاد رؤساي لشکر بودند که کوچکتر ايشان برابر صد نفر و بزرگتر برابر هزار نفر مي بود.
\par 15 اينانند که در ماه اول از اُردُن عبور نمودند هنگامي که آن از تمامي حدودش سيلان کرده بود و جميع ساکنان و اديها را هم به طرف مشرق و هم به طرف مغرب منهزم ساختند.
\par 16 و بعضي از بني بنيامين و يهودا نزد داود به آن ملاذ آمدند.
\par 17 داود به استقبال ايشان بيرون آمده، ايشان را خطاب کرده، گفت:« اگر با سلامتي براي اعانت من نزد من آمديد، دل من با شما ملصق خواهد شد؛ و اگر براي تسليم نمودن من به دست دشمنانم آمديد، با آنکه ظلمي در دست من نيست، پس خداي پدران ما اين را ببيند و انصاف نمايد.»
\par 18 آنگاه روح بر عماساي که رئيس شلاشيم بود نازل شد (و او گفت): « اي داود ما از آن تو و اي پسر يسي ما با تو هستيم؛ سلامتي، سلامتي بر تو باد، و سلامتي بر انصار تو باد زيرا خداي تو نصرت دهنده تو است.» پس داود ايشان را پذيرفته، سرداران لشکر ساخت.
\par 19 و بعضي از مَنَّسي به داود ملحق شدند هنگامي که او همراه فلسطينيان براي مقاتله با شاؤل مي رفت؛ اما ايشان را مدد نکردند زيرا که سرداران فلسطينيان بعد از مشورت نمودن، او را پس فرستاده، گفتند که:« او با سرهاي ما به آقاي خود شاؤل ملحق خواهد شد.»
\par 20 و هنگامي که به صِقلَغ مي رفت، بعضي از مَنَّسي به او پيوستند يعني عَدناح و يوزاباد و يدِيعَئيل و ميکائيل و يوزاباد و اَلِيهُو و صِلتاي که سرداران هزارهاي مَنَّسي بودند.
\par 21 ايشان داود را به مقاومت فوجهاي (عَمالَقه) مدد کردند، زيرا جميع ايشان مردان قوي شجاع و سردار لشکر بودند.
\par 22 زيرا در آن وقت، روز به روز براي اعانت داود نزد وي مي آمدند تا لشکرِ بزرگ، مثل لشکر خدا شد.
\par 23 و اين است شماره افراد آناني که براي جنگ مسلح شده، نزد داود به حَبرُون آمدند تا سلطنت شاؤل را بر حسب فرمان خداوند به وي تحويل نمايند.
\par 24 از بني يهودا شش هزار و هشتصد نفر که سپر و نيزه داشتند و مسلح جنگ بودند.
\par 25 از بني شَمعون هفت هزار و يکصد نفر که مردان قوي شجاع براي جنگ بودند.
\par 26 از بني لاوي چهار هزار و ششصد نفر.
\par 27 و يهُوياداع رئيس بني هارون و سه هزار و هفتصد نفر همراه وي.
\par 28 و صادوق که جوان قوي و شجاع بود با بيست ودو سردار از خاندان پدرش.
\par 29 و از بني بنيامين سه هزار نفر از برادران شاؤل و تا آن وقت اکثر ايشان وفاي خاندان شاؤل را نگاه مي داشتند.
\par 30 و از بني افرايم بيست هزار و هشتصد نفر که مردان قوي و شجاع و در خاندان پدران خويش نامور بودند.
\par 31 و از نصف سبط مَنَّسي هجده هزار نفر که به نامهاي خود تعيين شده بودند که بيايند و داود را به پادشاهي نصب نمايند.
\par 32 و از بني يسّاکار کساني که از زمانها مخبر شده، مي فهميدند که اسرائيليان چه بايد بکنند، سرداران ايشان دوست نفر و جميع برادران ايشان فرمان بردار ايشان بودند.
\par 33 و از زبولون پنجاه هزار نفر که با لشکر بيرون رفته، مي توانستند جنگ را با همه آلات حرب بيارايند و صف آرايي کنند و دو دل نبودند.
\par 34 و از نَفتالي هزار سردار و با ايشان سي و هفت هزار نفر با سپر و نيزه.
\par 35 و از بني دان بيست و هشت هزار و ششصد نفر براي جنگ مهيا شدند.
\par 36 و از اَشِير چهل هزار نفر که با لشکر بيرون رفته، مي توانستند جنگ را مهيا سازند.
\par 37 و از آن طرف اُردُن از بني جاد و نصف سبط مَنَّسي صد و بيست هزار نفر که با جميع آلات لشکر براي جنگ (مهيا شدند).
\par 38 جميع اينها مردان جنگي بودند که بر صف آرائي قادر بودند با دل کامل به حَبرُون آمدند تا داود را بر تمامي اسرائيل به پادشاهي نصب نمايند، و تمامي بقيه اسرائيل نيز براي پادشاه ساختن داود يک دل بودند.
\par 39 و در آنجا با داود سه روز اکل و شرب نمودند زيرا که برادران ايشان به جهت ايشان تدارک ديده بودند.و مجاوران ايشان نيز تا يسّاکار و زبولون و نَفتالي نان بر الاغها و شتران و قاطران و گاوان آوردند و مأکولات از آرد و قرصهاي انجير و کشمش و شراب و روغن و گاوان و گوسفندان به فراواني آوردند چونکه در اسرائيل شادماني بود.
\par 40 و مجاوران ايشان نيز تا يسّاکار و زبولون و نَفتالي نان بر الاغها و شتران و قاطران و گاوان آوردند و مأکولات از آرد و قرصهاي انجير و کشمش و شراب و روغن و گاوان و گوسفندان به فراواني آوردند چونکه در اسرائيل شادماني بود.
 
\chapter{13}

\par 1 و داود با سرداران هزاره و صده و با جميع رؤسا مشورت کرد.
\par 2 و داود به تمامي جماعت اسرائيل گفت: « اگر شما مصلحت مي دانيد و اگر اين از جانب يهُوَه خداي ما باشد، نزد برادران خود که در همه زمينهاي اسرائيل باقي مانده اند، به هر طرف بفرستيم و با ايشان کاهنان و لاوياني که در شهرهاي خود و حوالي آنها مي باشند، نزد ما جمع شوند،
\par 3 و تابوت خداي خويش را باز نزد خود بياوريم چونکه در ايام شاؤل نزد آن مسألت ننموديم.»
\par 4 و تمامي جماعت گفتند که:« چنين بکنيم.» زيرا که اين امر به نظر تمامي قوم پسند آمد.
\par 5 پس داود تمامي اسرائيل را از شيحُورِ مصر تا مدخل حَمات جمع کرد تا تابوت خدا را از قريت يعاريم بياورند.
\par 6 و داود و تمامي اسرائيل به بَعلَه که همان قريت يعاريم است و از آنِ يهودا بود، بر آمدند تا تابوت خدا يهُوَه را که در ميان کروبيان در جايي که اسم او خوانده مي شود ساکن است، از آنجا بياورند.
\par 7 و تابوت خدا را بر ارابه اي تازه از خانه اَبِيناداب آوردند و عُزّا و اَخِيو ارابه را مي راندند.
\par 8 و داود و تمامي اسرائيل با سرود و بربط و عود و دف و سنج و کرنا به قوت تمام به حضور خدا وجد مي نمودند.
\par 9 و چون به خرمنگاه کيدون رسيدند عُزّا دست خود را دراز کرد تا تابوت را بگيرد زيرا گاوان مي لغزيدند.
\par 10 و خشم خداوند بر عُزّا افروخته شده، او را زد از آن جهت که دست خود را به تابوت دراز کرد و در آنجا به حضور خدا مرد.
\par 11 و داود محزون شد چونکه خداوند بر عُزّا رخنه نمود و آن مکان را تا امروز فارَص عُزّا ناميد.
\par 12 و در آن روز داود از خدا ترسان شده، گفت:« تابوت خدا را نزد خود چگونه بياورم؟»
\par 13 پس داود تابوت را نزد خود به شهر داود نياورد بلکه آن را به خانه عُوبيد اَدُوم جَتِّي برگردانيد.و تابوت خدا نزد خاندان عُوبيد اَدُوم در خانه اش سه ماه ماند و خداوند خانه عُوبيد اَدُوم و تمامي مايملک او را برکت داد.
\par 14 و تابوت خدا نزد خاندان عُوبيد اَدُوم در خانه اش سه ماه ماند و خداوند خانه عُوبيد اَدُوم و تمامي مايملک او را برکت داد.
 
\chapter{14}

\par 1 و حيرام پادشاه صور، قاصدان با چوب سرو آزاد و بنايان و نجاران نزد داود فرستاده تا خانه اي براي او بسازند.
\par 2 و داود دانست که خداوند او را به پادشاهي اسرائيل استوار داشته است، زيرا که سلطنتش به خاطر قوم او اسرائيل به درجه بلند برافراشته شده بود.
\par 3 و داود در اورشليم با زنان گرفت، و داود پسران و دختران ديگر توليد نمود.
\par 4 و اين است نامهاي فرزنداني که در اورشليم براي وي به هم رسيدند: شَمُّوع و شُوباب و ناتان و سُليمان،
\par 5 و یبحار و اَلِيشُوع و اَلِيفالَط،
\par 6 و نُوجَه و نافَج و يافِيع،
\par 7 و اَلِيشامَع و بَعلياداع و اَلِيفَلَط.
\par 8 و چون فلسطينيان شنيدند که داود به پادشاهي تمام اسرائيل مسح شده است، پس فلسطينيان برآمدند تا داود را (براي جنگ) بطلبند؛ و چون داود شنيد، به مقابله ايشان برآمد.
\par 9 و فلسطينيان آمده، در وادي رفائيم منتشر شدند.
\par 10 و داود از خدا مسألت نموده، گفت:« آيا به مقابله فلسطينيان برآيم و آيا ايشان را به دست من تسليم خواهي نمود؟» خداوند او را گفت:« برآي وايشان را به دست تو تسليم خواهم کرد.»
\par 11 پس به بَعل فَراصِيم برآمدند و داود ايشان را در آنجا شکست داد و داود گفت:« خدا بر دشمنان من به دست من مثل رخنه آب رخنه کرده است.» بنابراين آن مکان را بَعل فَراصِيم نام نهادند.
\par 12 و خدايان خود را در آنجا ترک کردند و داود امر فرمود که آنها را به آتش بسوزانند.
\par 13 و فلسطينيان بار ديگر در آن وادي منتشر شدند.
\par 14 و داود باز از خدا سؤال نمود و خدا او را گفت:« از عقب ايشان مرو بلکه از ايشان رو گردانيده، در مقابل درختان توت به ايشان نزديک شو.
\par 15 و چون در سر درختان توت آواز قدمها بشنوي، آنگاه براي جنگ بيرون شو، زيرا خدا پيش روي تو بيرون رفته است تا لشکر فلسطينيان را مغلوب سازد.»
\par 16 پس داود بر وفق آنچه خدا اورا امر فرموده بود، عمل نمود و لشکر فلسطينيان را از جِبعُون تا جارز شکست دادند.و اسم داود در جميع اراضي شيوع يافت و خداوند ترس او را بر تمامي امّت ها مستولي ساخت.
\par 17 و اسم داود در جميع اراضي شيوع يافت و خداوند ترس او را بر تمامي امّت ها مستولي ساخت. 
 
\chapter{15}

\par 1 و داود در شهر خود خانه ها بنا کرد و مکاني براي تابوت خدا فراهم ساخته، خيمه اي به جهت آن برپا نمود.
\par 2 آنگاه داود فرمود که غير از لاويان کسي تابوت خدا را برندارد زيرا خداوند ايشان را برگزيده بود تا تابوت خدا بردارند و او را هميشه خدمت نمايند.
\par 3 و داود تمامي اسرائيل را در اورشليم جمع کرد تا تابوت خداوند را به مکاني که برايش مهيا ساخته بود، بياورند.
\par 4 و داود پسران هارون و لاويان را جمع کرد.
\par 5 از بني قَهات اُوريئيلِ رئيس وصد بيست نفر برادرانش را.
\par 6 از بني مَراري، عَساياي رئيس ودوست بيست نفر برادرانش را.
\par 7 از بني جَرشُوم، يوئيل رئيس و صد وسي نفر برادرانش را.
\par 8 از بني اليصافان، شَمَعياي رئيس و دوست نفر برادرانش را.
\par 9 از بني حَبرُون، اِيلِيئيل رئيس و هشتاد نفر برادرانش را.
\par 10 از بني عُزّيئيل، عَمِّيناداب رئيس وصد و دوازده نفر برادرانش را.
\par 11 و داود صادوق و ابياتارِکَهَنَه و لاويان يعني اُرِيئيل و عَسايا و يوئيل و شَمَعيا و اِيليئيل و عَمِّيناداب را خوانده،
\par 12 به ايشان گفت: «شما رؤساي خاندانهاي آباي لاويان هستيد؛ پس شما و برادران شما خويشتن را تقديس نماييد تا تابوت يهُوَه خداي اسرائيل را به مکاني که برايش مهيا ساخته ام بياوريد.
\par 13 زيرا از اين سبب که شما دفعه اول آن را نياورديد، يهُوَه خداي ما بر ما رخنه کرد، چونکه او را بر حسب قانون نطلبيديم.»
\par 14 پس کاهنان و لاويان خويشتن را تقديس نمودند تا تابوت يهُوَه خداي اسرائيل را بياورند.
\par 15 و پسران لاويان بر وفق آنچه موسي بر حسب کلام خداوند امر فرموده بود، چوب دستيهاي تابوت خدا را بر کتفهاي خود گذاشته، آن را برداشتند.
\par 16 و داود رؤساي لاويان را فرمود تا برادران خود مغنيان را با آلات موسيقي از عودها و بربطها سنجها تعين نمايند، تا به آواز بلند و شادماني صدا زنند.
\par 17 پس لاويان هِيمان بن يوئيل و از برادران او آساف بن بَرَکيا و از برادران ايشان بني مَراري ايتان بن قُوشيا را تعيين نمودند.
\par 18 و با ايشان از برادران درجه دوم خود: زکريا و بَين و يعريئيل شَميراموت و يحيئيل و عُنِّي و اَلِيآب و بنايا و مَعَسيا و مَتِّتيا و اَليفَليا و مَقَنيا و عُوبيد اَدُوم و يعِيئِيل دربانان را.
\par 19 و از مغنيان: هِيمان و آساف و ايتان را با سنجهاي برنجين تا بنوازند.
\par 20 و زَکرّيا و عَزِيئيل و شَميراموت و يحيئيل و عُنِّي و اَليآب و مَعَسيا و بنايا را با عودها بر آلاموت.
\par 21 و مَتَّتيا و اَليفَليا و مَقَنيا و عُوبيد اَدُوم و يعِيئيل و عَزَريا را با بربطهاي بر ثَمانِي تا پيشرَوِي نمايند.
\par 22 و کَنَنيا رئيس لاويان بر نغمات بود و مغنّيان را تعليم مي داد زيرا که ماهر بود.
\par 23 و بَرَکيا و اَلقانَه دربانان تابوت بودند.
\par 24 و شَبَنيا و يوشافاط و نَتَنئيل و عَماساي و زَکريا و بَنايا و اَلَيعَزَر کَهَنَه پيش تابوت خدا کرنا مي نواختند، و عوبيد اَدُوم و يحيي دربانانِ تابوت بودند.
\par 25 و داود و مشايخ اسرائيل و سرداران هزاره رفتند تا تابوت عهد خداوند را از خانه عوبيد اَدُوم با شادماني بياورند.
\par 26 و چون خدا لاويان را که تابوت عهد خداوند را برمي داشتند اعانت کرد، ايشان هفت گاو و هفت قوچ ذبح کردند.
\par 27 و داود و جميع لاوياني که تابوت را برمي داشتند و مغنيان و کَنَنيا که رئيس نغمات مغنيان بود به کتان نازک ملبس بودند، و داود ايفود کتان دربرداشت.
\par 28 و تمامي اسرائيل تابوت عهد خداوند را به آواز شادماني و آواز بوق و کرّنا و سنج و عود و بربط مي نواختند.و چون تابوت عهد خداوند وارد شهر داود مي شد، ميکال دختر شاؤل از پنجره نگريست و داود پادشاه را ديد که رقص و وجد مي نمايد، او را در دل خود خوار شمرد.
\par 29 و چون تابوت عهد خداوند وارد شهر داود مي شد، ميکال دختر شاؤل از پنجره نگريست و داود پادشاه را ديد که رقص و وجد مي نمايد، او را در دل خود خوار شمرد.
 
\chapter{16}

\par 1 و تابوت خدا را آورده، آن را در خيمه اي که داود برايش برپا کرده بود، گذاشتند؛ و قرباني هاي سوختني و ذبايح سلامتي به حضور خدا گذرانيدند.
\par 2 و چون داود از گذرانيدن قرباني هاي سوختني و ذبايح سلامتي فارغ شد، قوم را به اسم خداوند برکت داد.
\par 3 و به جميع اسرائيليان به مردان و زنان به هر يکي يک گرده نان و يک پاره گوشت ويک قرص کشمش بخشيد.
\par 4 و بعضي لاويان براي خدمتگزاري پيش تابوت خداوند تعيين نمود تا يهُوَه خداي اسرائيل را ذکر نمايند و شکر گويند و تسبيح خوانند،
\par 5 يعني آساف رئيس و بعد از او زکريا و يعيئيل و شَمِيرامُوت و يحيئيل و مَتَّتيا و اَلِيآب و بنايا و عُوبيد اَدُوم و يعيئيل را با عودها و بربطها و آساف با سنجها مي نواخت.
\par 6 و بنايا و يحزِيئيل کهنه پيش تابوت عهد خدا با کرّناها دائماً (حاضر مي بودند).
\par 7 پس در همان روز داود اولاً (اين سرود را) به دست آساف و برادرانش داد تا خداوند را تسبيح بخوانند:
\par 8 يهُوَه را حمد گوييد و نام او را بخوانيد. اعمال او را در ميان قومها اعلام نماييد.
\par 9 اورا بسراييد براي او تسبيح بخوانيد. در تمامي کارهاي عجيب او تفکر نماييد.
\par 10 در نام قدّوس او فخر کنيد. دل طالبان خداوند شادمان باشد.
\par 11 خداوند و قوت او را بطلبيد. روي او را پيوسته طالب باشيد.
\par 12 کارهاي عجيب را که او کرده است، بياد آوريد، آيات او را و داوريهاي دهان وي را،
\par 13 اي ذريت بنده او اسرائيل! اي فرزندان يعقوب برگزيده او!
\par 14 يهُوَه خداي ما است. داوريهاي او در تمامي جهان است.
\par 15 عهد او را بياد آوريد تا ابدالآباد، و کلامي را که به هزاران پشت فرموده است،
\par 16 آن عهدي را که با ابراهيم بسته، و قَسَمي را که براي اسحاق خورده است،
\par 17 و آن را براي يعقوب فريضه قرار داد و براي اسرائيل عهد جاوداني؛
\par 18 و گفت زمين کَنعان را به تو خواهم داد، تا حصه ميراث شما شود،
\par 19 هنگامي که عددي معدود بوديد، قليل العدد و غربا در آنجا،
\par 20 و از اُمّتي تا اُمّتي سرگردان مي بودند، واز يک مملکت تا قوم ديگر.
\par 21 او نگذاشت که کسي بر ايشان ظلم کند، بلکه پادشاهان را به خاطر ايشان توبيخ نمود،
\par 22 که بر مسيحان من دست مگذاريد، و انبياي مرا ضرر مرسانيد.
\par 23 اي تمامي زمين يهُوَه را بسراييد. نجات او را روز به روز بشارت دهيد.
\par 24 در ميان امّت ها جلال او را ذکر کنيد، و کارهاي عجيب او را در جميع قوم ها.
\par 25 زيرا خداوند عظيم است و بي نهايت محمود؛ و او مُهيب است بر جميع خدايان.
\par 26 زيرا جميع خدايان امّت ها بتهايند. اما يهُوَه آسمانها را آفريد.
\par 27 مجد و جلال به حضور وي است؛ قوت و شادماني در مکان او است.
\par 28 اي قبايل قوم ها خداوند را توصيف نماييد. خداوند را به جلال و قوت توصيف نماييد.
\par 29 خداوند را به جلال اسم او توصيف نماييد. هدايا بياوريد و به حضور وي بياييد. خداوند را در زينت قدوسيت بپرستيد.
\par 30 اي تمامي زمين از حضور وي بلرزيد. ربع مسکون نيز پايدار شد و جنبش نخواهد خورد.
\par 31 آسمان شادي کند و زمين سرور نمايد، و در ميان اُمّت ها بگويند که يهُوَه سلطنت مي کند.
\par 32 دريا و پري آن غرش نمايد؛ و صحرا و هر چه در آن است به وجد آيد.
\par 33 آنگاه درختان جنگل ترنم خواهند نمود، به حضور خداوند زيرا که براي داوري جهان مي آيد.
\par 34 يهُوَه را حمد بگوييد زيرا که نيکو است. زيرا که رحمت او تا ابدالآد است.
\par 35 و بگوييد اي خداي نجات ما ما را نجات بده. و ما را جمع کرده، از ميان اُمت ها رهايي بخش. تا نام قدوس تو را حمد گوييم، و در تسبيح تو فخر نماييم.
\par 36 يهُوَه خداي اسرائيل متبارک باد. از ازل تا ابدالآباد.
\par 37 پس آساف و برادرانش را آنجا پيش تابوت عهد خداوند گذاشت تا هميشه پيش تابوت به خدمت هر روز در روزش مشغول باشند.
\par 38 و عُوبيد اَدُوم و شصت و هشت نفر برادران ايشان و عُوبيد اَدُوم بن يدِيتُون و حُوسَه دربانان را.
\par 39 و صادوقِ کاهن و کاهنان برادرانش را پيش مسکن خداوند در مکان بلندي که در جِبعُون بود،
\par 40 تا قرباني هاي سوختني براي خداوند بر مذبح قرباني سوختني دائماً صبح و شام بگذرانند بر حسب آنچه در شريعت خداوند که آن را به اسرائيل امر فرموده بود مکتوب است.
\par 41 و با ايشان هِيمان و يدُوتُون و ساير برگزيدگاني را که اسم ايشان ذکر شده بود تا خداوند را حمد گويند زيرا که رحمت او تا ابدالآباد است.
\par 42 و همراه ايشان هِيمان و يدُوتُون را با کّرناها و سنجها و آلات نغمات خدا به جهت نوازندگان و پسران يدُوتُون را تا نزد دروازه باشند.پس تمامي قوم هر يکي به خانه خود رفتند، اما داود برگشت تا خانه خود را تبرک نمايد.
\par 43 پس تمامي قوم هر يکي به خانه خود رفتند، اما داود برگشت تا خانه خود را تبرک نمايد.
 
\chapter{17}

\par 1 و واقع شد چون داود در خانه خود نشسته بود که داود به ناتان نبي گفت:« اينک من خانه سرو آزاد ساکن مي باشم و تابوت عهد خداوند زير پرده ها است.»
\par 2 ناتان به داود گفت:« هر آنچه در دلت باشد به عمل آور زيرا خدا با تو است.»
\par 3 و در آن شب واقع شد که کلام خدا به ناتان نازل شده،گفت:
\par 4 « برو و به بنده من داود بگو خداوند چنين مي فرمايد: تو خانه اي براي سکونت من بنا نخواهي کرد.
\par 5 زيرا از روزي که بني اسرائيل را بيرون آوردم تا امروز در خانه ساکن نشده ام بلکه از خيمه به خيمه و مسکن به مسکن گردش کرده ام .
\par 6 و به هر جايي که با تمامي اسرائيل گردش کرده ام، آيا به احدي از داوران اسرائيل که براي رعايت قوم خود مأمور داشتم، سخني گفتم که چرا خانه اي از سرو آزاد براي من بنا نکرديد؟
\par 7 و حال به بنده من داود چنين بگو: يهُوَه صبايوت چنين مي فرمايد: من تو را از چراگاه از عقب گوسفندان گرفتم تا پيشواي قوم من اسرائيل باشي.
\par 8 و هر جايي که مي رفتي، من با تو مي بودم و جميع دشمنانت را از حضور تو منقطع ساختم و براي تو اسمي مثل اسم بزرگاني که بر زمين اند پيدا کردم.
\par 9 و به جهت قوم خود اسرائيل مکاني تعيين نمودم و ايشان را غرس کردم تا در مکان خويش ساکن شده، باز متحرک نشوند، و شريران ايشان را ديگر مثل سابق ذليل نسازند.
\par 10 و از ايامي که داوران را بر قوم خود اسرائيل تعيين نمودم و تمامي دشمنانت را مغلوب ساختم، تو را خبر مي دادم که خداوند خانه اي براي تو بنا خواهد نمود.
\par 11 و چون روزهاي عمر تو تمام شود که نزد پدران خود رحلت کني، آنگاه ذريت تو را که پسران تو خواهد بود، بعد از تو خواهم برانگيخت و سلطنت او را پايدار خواهم نمود.
\par 12 او خانه اي براي من بنا خواهد کرد و من کرسي او را تا به ابد استوار خواهم ساخت.
\par 13 من او را پدر خواهم بود و او مرا پسر خواهد بود و رحمت خود را از او دور نخواهم کرد چنانکه آن را از کسي که قبل از تو بود دور کردم.
\par 14 و او را در خانه و سلطنت خودم تا به ابد پايدار خواهم ساخت و کرسي او استوار خواهد ماند تا ابدالآباد.»
\par 15 بر حسب تمامي اين کلمات و مطابق تمامي اين رؤيا ناتان به داود تکلم نمود.
\par 16 و داود پادشاه داخل شده، به حضور خداوند نشست و گفت:« اي يهُوَه خدا، من کيستم و خاندان من چيست که مرا به اين مقام رسانيدي؟
\par 17 و اين نيز در نظر تو اي خدا امر قليل نمود زيرا که درباره خانه بنده ات نيز براي زمان طويل تکلم نمودي و مرا اي يهُوَه خدا، مثل مرد بلند مرتبه منظور داشتي.
\par 18 و داود ديگر درباره اکرامي که به بنده خود کردي، نزد تو چه تواند افزود زيرا که تو بنده خود را مي شناسي.
\par 19 اي خداوند، به خاطر بنده خود و موافق دل خويش جميع اين کارهاي عظيم را به جا آوردي تا تمامي اين عظمت را ظاهر سازي. 
\par 20 اي يهُوَه مثل تو کسي نيست و غير از تو خدايي ني. موافق هر آنچه به گوشهاي خود شنيديم،
\par 21 و مثل قوم تو اسرائيل کدام امتي بر روي زمين است که خدا بيايد تا ايشان را فديه داده، براي خويش قوم بسازد، و به کارهاي عظيم و مهيب اسمي براي خود پيدا نمايي و امّت ها را از حضور قوم خود که ايشان را از مصر فديه دادي، اخراج نمايي.
\par 22 و قوم خود اسرائيل را براي خويش تا به ابد قوم ساختني و تو اي يهُوَه خداي ايشان شدي.
\par 23 « و الآن اي خداوند کلامي که درباره بنده ات و خانه اش گفتي تا به ابد استوار شود و بر حسب آنچه گفتي عمل نما.
\par 24 و اسم تو تا به ابد استوار و معظم بماند تا گفته شود که يهُوَه صبايوت خداي اسرائيل خداي اسرائيل است و خاندان بنده ات داود به حضور تو پايدار بماند.
\par 25 زيرا تو اي خداي من بر بنده خود کشف نمودي که خانه اي برايش بنا خواهي نمود؛ بنابراين بنده ات جرأت کرده است که اين دعا را نزد تو بگويد.
\par 26 و الآن اي يهُوَه، تو خدا هستي و اين احسان را به بنده خود وعده دادي.و الآن تو را پسند آمد که خانه بنده خود را برکت دهي تا در حضور تو تا به ابد بماند زيرا که تو اي خداوند برکت داده اي و مبارک خواهد بود تا ابدالآباد.»
\par 27 و الآن تو را پسند آمد که خانه بنده خود را برکت دهي تا در حضور تو تا به ابد بماند زيرا که تو اي خداوند برکت داده اي و مبارک خواهد بود تا ابدالآباد.»
 
\chapter{18}

\par 1 و بعد از اين واقع شد که داود فلسطينيان را شکست داده، مغلوب ساخت و جَتّ و قريه هايش را از دست فلسطينيان گرفت.
\par 2 و موآب را شکست داد و موآبيان بندگان داود شده، هدايا آوردند.
\par 3 و داود هَدَرعَزَر پادشاه صُوبَه را در حَمات هنگامي که مي رفت تا سلطنت خود را نزد نهر فرات استوار سازد، شکست داد.
\par 4 و داود هزار ارابه و هفت هزار سوار و بيست هزار پياده از او گرفت، و داود تمامي اسبان ارابه را پي کرد، اما از آنها براي صد ارابه نگاه داشت.
\par 5 و چون اَراميانِ دمشق به مدد هَدَرعَزَر پادشاه صُوبَه آمدند، داود بيست و دو هزار نفر از اَراميان را کشت.
\par 6 و داود در اَرامِ دمشق (قراولان) گذاشت و اَراميان بندگان داود شده، هدايا آوردند. و خداوند داود را در هر جا که مي رفت نصرت مي داد.
\par 7 و داود سپرهاي طلا را که بر خادمان هَدَرعَزَر بود گرفته، آنها را به اورشليم آورد.
\par 8 و داود از طِبحَت و کُون شهرهاي هَدَرعَزَر برنج از حد زياده گرفت که از آن سُليمان درياچه و ستونها و ظروف برنجين ساخت.
\par 9 و چون تُوعُو پادشاه حَمات شنيد که داود تمامي لشکر هَدَرعَزَر پادشاه صُوبَه را شکست داده است،
\par 10 پسر خود هَدُرام را نزد داود پادشاه فرستاد تا سلامتي او بپرسد و او را تَهنِيت گويد از آن جهت که با هَدَرعَزَر جنگ نموده او را شکست داده بود، زيرا هَدَرعَزَر با تُوعُو مقاتله مي نمود؛ و هر قِسم ظروف طلا و نقره و برنج (با خود آورد).
\par 11 و داود پادشاه آنها را نيز براي خداوند وقف نمود، با نقره و طلايي که از جميع امّت ها يعني از اَدُوم و موآب و بني عَمُّون و فلسطينيان و عَمالَقَه آورده بود.
\par 12 و اَبِشاي ابن صَرُويه هجده هزار نفر از اَدُوميان را در وادي مِلح کشت.
\par 13 و در اَدُوم قراولان قرار داد و جميع اَدُوميان بندگان داود شدند و خداوند داود را در هر جايي که مي رفت نصرت مي داد.
\par 14 و داود بر تمامي اسرائيل سلطنت نمود، انصاف و عدالت را بر تمامي قوم خود مجزا مي داشت.
\par 15 و يوآب بن صَرُويه سردار لشکر بود و يهوشافاط بن اَخِيلُود وقايع نگار.
\par 16 و صادوق بن اَخِيطُوب و ابيمَلِک بن اَبياتار کاهن بودند و شوشا کاتب بود.و بنايا ابن يهُوياداع رئيس کَريتيان و فَلِيتيان و پسران داود نزد پادشاه مقدم بودند.
\par 17 و بنايا ابن يهُوياداع رئيس کَريتيان و فَلِيتيان و پسران داود نزد پادشاه مقدم بودند.
 
\chapter{19}

\par 1 و بعد از اين واقع شد که ناحاش، پادشاه بني عَمُّون مرد و پسرش در جاي او سلطنت نمود.
\par 2 و داود گفت:« با حانُون بن ناحاش احسان نمايم چنانکه پدرش به من احسان کرد.» پس داود قاصدان فرستاد تا او را درباره پدرش تعزيت گويند. و خادمان داود به زمين بني عَمُّون نزد حانُون براي تعزيت وي آمدند.
\par 3 و سروران بني عَمُّون به حانُون گفتند:« آيا گمان مي بري که به جهت تکريم پدر تو است که داود تعزيت کنندگان نزد تو فرستاده است؟ ني بلکه بندگانش به جهت تفحص و انقلاب و جاسوسي زمين نزد تو آمدند.»
\par 4 پس حانُون خادمان داود را گرفته، ريش ايشان را تراشيد و لباسهاي ايشان را از ميان تا جاي نشستن دريده، ايشان را رها کرد.
\par 5 و چون بعضي آمده، داود را از حالت آن کسان خبر دادند، به استقبال ايشان فرستاد زيرا که ايشان بسيار خجل بودند، و پادشاه گفت:« در اَريحا بمانيد تا ريشهاي شما درآيد و بعد از آن برگرديد.»
\par 6 و چون بني عَمُّون ديدند که نزد داود مکروه شده اند، حانُون و بني عَمُّون هزار وزنه نقره فرستادند تا ارابه ها و سواران از اَرام نَهرَين و اَرام مَعکَه و صُوبَه براي خود اجير سازند.
\par 7 پس سي و دو هزار ارابه و پادشاه مَعکَه و جمعيت او را براي خود اجير کردند، و ايشان بيرون آمده، در مقابل مِيدَبا اُردو زدند، و بني عَمُّون از شهر هاي خود جمع شده، براي مقاتله آمدند.
\par 8 و چون داود اين را شنيد، يوآب و تمامي لشکر شجاعان را فرستاد.
\par 9 و بني عَمُّون بيرون آمده، نزد دروازه شهر براي جنگ صف آرايي نمودند. و پادشاهاني که آمده بودند، در صحرا عليحده بودند.
\par 10 و چون يوآب ديد که روي صفوف جنگ، هم از پيش و هم از عقبش بود، از تمامي برگزيدگان اسرائيل گروهي را انتخاب کرده، در مقابل اَرميان صف آرايي نمود.
\par 11 و بقيه قوم را به دست برادر خود اَبِشاي سپرده و به مقابل بني عَمُّون صف کشيدند.
\par 12 و گفت:« اگر اَرميان بر من غالب آيند، به مدد من بيا؛ و اگر بني عَمُّون بر تو غالب آيند، به جهت امداد تو خواهم آمد.
\par 13 دلير باش که به جهت قوم خويش و به جهت شهر هاي خداي خود مردانه بکوشيم و خداوند آنچه را در نظرش پسند آيد بکند.»
\par 14 پس يوآب و گروهي که همراهش بودند، نزديک شدند تا با اَرميان جنگ کنند و ايشان از حضور وي فرار کردند.
\par 15 و چون بني عَمُّون ديدند که اَراميان فرار کردند، ايشان نيز از حضور برادرش اَبِشاي گريخته، داخل شهر شدند؛ و يوآب به اورشليم برگشت.
\par 16 و چون اَراميان ديدند که از حضور اسرائيل شکست يافتند، ايشان قاصدان فرستاده، اَراميان را که به آن طرف نهر بودند، و شُوفَک سردار لشکر هَدَرعَزَر پيشواي ايشان بود.
\par 17 و چون خبر به داود رسيد، تمامي اسرائيل را جمع کرده، از اُردُن عبور نمود وبه ايشان رسيده، مقابل ايشان صف آرايي نمود. و چون داود جنگ را با اَراميان آراسته بود، ايشان با وي جنگ کردند.
\par 18 و اَراميان از حضور اسرائيل فرار کردند و داود مردان هفت هزار ارابه و چهل هزار پياده از اَراميان را کشت، و شُوفَک سردار لشکر را به قتل رسانيد.و چون بندگان هَدَرعَزَر ديدند که از حضور اسرائيل شکست خوردند، با داود صلح نموده، بنده او شدند، و اَراميان بعد از آن در اعانت بني عَمُّون اقدام ننمودند.
\par 19 و چون بندگان هَدَرعَزَر ديدند که از حضور اسرائيل شکست خوردند، با داود صلح نموده، بنده او شدند، و اَراميان بعد از آن در اعانت بني عَمُّون اقدام ننمودند.
 
\chapter{20}

\par 1 و واقع شد در وقت تحويل سال، هنگام بيرون رفتن پادشاهان، که يوآب قوت لشکر را بيرون آورد، و زمين بني عَمُّون را ويران ساخت و آمده، رَبَّه را محاصره نمود. اما داود در اورشليم ماند و يوآب رَبَّه را تسخير نموده، آن را منهدم ساخت.
\par 2 و داود تاج پادشاه ايشان را از سرش گرفت که وزنش يک وزنه طلا بود و سنگهاي گرانبها داشت و آن را بر سر داود گذاشتند و غنيمت از حد زياده از شهر بردند.
\par 3 و خلق آنجا را بيرون آورده، ايشان را به ارّه ها و چومهاي آهنين و تيشه ها پاره پاره کرد؛ و داود به همين طور با جميع شهرهاي بني عَمُّون رفتار نمود. پس داود و تمامي قوم به اورشليم برگشتند.
\par 4 و بعد از آن جنگي با فلسطينيان در جازَر، واقع شد که در آن سِبکاي حُوشاتي سِفّاي را که از اولاد رافا بود کشت و ايشان مغلوب شدند.
\par 5 و باز جنگ با فلسطينيان واقع شد و اَلحانان بن ياعير لحميرا که برادر جُليات جَتِّي بود کُشت که چوب نيزه اش مثل نورد جولاهکان بود.
\par 6 و باز جنگ در جَتّ واقع شد که در آنجا مردي بلند قد بود که بيست وچهار انگشت، شش بر هر دست و شش بر هر پا داشت و او نيز براي رافا زاييده شده بود.
\par 7 و چون او اسرائيل را به تنگ آورد، يهُوناتان بن شِمعا برادر داود او را کُشت.اينان براي رافا در جَتّ زاييده شدند و به دست داود و به دست بندگانش افتادند.
\par 8 اينان براي رافا در جَتّ زاييده شدند و به دست داود و به دست بندگانش افتادند.
 
\chapter{21}

\par 1 و شيطان بر ضد اسرائيل برخاسته، داود را اغوا نمود که اسرائيل را بشمارد.
\par 2 و داود به يوآب و سروران قوم گفت:« برويد و عدد اسرائيل را از بئرشَبَع تا دان گرفته، نزد من بياوريد تا آن را بدانم.»
\par 3 يوآب گفت:« خداوند بر قوم خود هر قدر که باشند صد چندان مزيد کند؛ و اي آقايم پادشاه آيا جميع ايشان بندگان آقايم نيستند؟ ليکن چرا آقايم خواهش اين عمل دارد و چرا بايد باعث گناه اسرائيل بشود؟»
\par 4 اما کلام پادشاه بر يوآب غالب آمد و يوآب در ميان تمامي اسرائيل گردش کرده، باز به اورشليم مراجعت نمود.
\par 5 و يوآب عدد شمرده شدگان قوم را به داود داد و جمله اسرائيليان هزار هزار و يکصد هزار مرد شمشير زن و از يهودا چهارصد و هفتاد و چهار هزار مرد شمشير زن بودند.
\par 6 ليکن لاويان و بنيامينيان را در ميان ايشان نشمرد زيرا که فرمان پادشاه نزد يوآب مکروه بود.
\par 7 و اين امر به نظر خدا ناپسند آمد، پس اسرائيل را مبتلا ساخت.
\par 8 و داود به خدا گفت:« در اين کاري که کردم، گناه عظيمي ورزيدم. و حال گناه بنده خود را عفو فرما زيرا که بسيار احمقانه رفتار نمودم.»
\par 9 و خداوند جاد را که رايي داود بود خطاب کرده، گفت:
\par 10 « برو و داود را اعلام کرده، بگو خداوند چنين مي فرمايد: من سه چيز پيش تو مي گذارم؛ پس يکي از آنها را براي خود اختيار کن تا برايت به عمل آورم.»
\par 11 پس جاد نزد داود آمده، وي را گفت: « خداوند چنين مي فرمايد براي خود اختيار کن:
\par 12 يا سه سال قحط بشود، يا سه ماه پيش روي خصمانت تلف شوي و شمشير دشمنانت تو را درگيرد، يا سه روز شمشير خداوند و وبا در زمين تو واقع شود، و فرشته خداوند تمامي حدود اسرائيل را ويران سازد. پس الآن ببين که نزد فرستنده خود چه جواب برم.»
\par 13 داود به جاد گفت: « در شدت تنگي هستم. تمنا اينکه به دست خداوند بيفتم زيرا که رحمت هاي او بسيار عظيم است و به دست انسان نيفتم.»
\par 14 پس خداوند وبا بر اسرائيل فرستاد و هفتاد هزار نفر از اسرائيل مردند.
\par 15 و خدا فرشته اي به اورشليم فرستاد تا آن را هلاک سازد. و چون مي خواست آن را هلاک کند، خداوند ملاحظه نمود و از آن بلا پشيمان شد و به فرشته اي که (قوم را) هلاک مي ساخت گفت: «کافي است، حال دست خود را باز دار.» و فرشته خداوند نزد خرمنگاه اَرنانِ يبُوسي ايستاده بود.
\par 16 و داود چشمان خود را بالا انداخته، فرشته خداوند را ديد که در ميان زمين و آسمان ايستاده است و شمشيري برهنه در دستش بر اورشليم برافراشته؛ پس داود و مشايخ به پلاس ملبس شده، به روي خود درافتادند.
\par 17 و داود به خدا گفت: «آيا من براي شمردن قوم امر نفرمودم و آيا من آن نيستم که گناه ورزيده، مرتکب شرارت زشت شدم؟ اما اين گوسفندان چه کرده اند؟ پس اي يهُوَه خدايم، مستدعي اينکه دست تو بر من و خاندان پدرم باشد و به قوم خود بلا مرساني.»
\par 18 و فرشته خداوند جاد را امر فرمود که به داود بگويد که داود برود و مذبحي به جهت خداوند در خرمنگاه اُرنان يبُوسي برپا کند. 
\par 19 پس داود بر حسب کلامي که جاد به اسم خداوند گفت برفت.
\par 20 و اُرنان روگردانيده، فرشته را ديد و چهار پسرش که همراهش بودند، خويشتن را پنهان کردند؛ و اُرنان گندم مي کوبيد.
\par 21 و چون داود نزد اُرنان آمد، اُرنان نگريسته، داود را ديد و از خرونگاه بيرون آمده، به حضور داود رو به زمين افتاد.
\par 22 و داود به اُرنان گفت: « جاي خرمنگاه را به من بده تا مذبحي به جهت خداوند برپا نمايم؛ آن را به قيمت تمام به من بده تا وبا از قوم رفع شود.»
\par 23 اُرنان به داود عرض کرد: « آن را براي خود بگير و آقايم پادشاه آنچه که در نظرش پسند آيد به عمل آورد؛ ببين گاوان را به جهت قرباني سوختني و چومها را براي هيزم و گندم را به جهت هديه آردي دادم و همه را به تو بخشيدم.»
\par 24 اما داود پادشاه به اُرنان گفت: «ني، بلکه آن را البته به قيمت تمام از تو خواهم خريد، زيرا که از اموال تو براي خداوند نخواهم گرفت و قرباني سوختني مجاناً نخواهم گذرانيد.»
\par 25 پس داود براي آن موضع ششصد مثقال طلا به وزن، به اُرنان داد.
\par 26 و داود در آنجا مذبحي به جهت خداوند بنا نموده، قرباني هاي سوختني و ذبايح سلامتي گذرانيد و نزد خداوند استدعا نمود؛ و او آتشي از آسمان بر مذبح قرباني سوختني (نازل کرده،) او را مستجاب فرمود.
\par 27 و خداوند فرشته را حکم داد تا شمشير خود را در غلافش برگردانيد.
\par 28 در آن زمان چون داود ديد که خداوند او را در خرمنگاه اُرنان يبُوسي مستجاب فرموده است، در آنجا قرباني ها گزرانيد.
\par 29 اما مسکن خداوند که موسي در بيابان ساخته بود و مذبح قرباني سوختني، در آن ايام در مکان بلند جِبعون بود.ليکن داود نتوانست نزد آن برود تا از خدا مسألت نمايد، چونکه از شمشير فرشته خداوند مي ترسيد.
\par 30 ليکن داود نتوانست نزد آن برود تا از خدا مسألت نمايد، چونکه از شمشير فرشته خداوند مي ترسيد.
 
\chapter{22}

\par 1 پس داود گفت: « اين است خانه يهُوَه خدا، و اين مذبح قرباني سوختني براي اسرائيل مي باشد.»
\par 2 و داود فرمود تا غريبان را که در زمين اسرائيل اند جمع کنند، و سنگ تراشان معين کرد تا سنگهاي مربع براي بناي خانه خدا بتراشند.
\par 3 و داود آهن بسياري به جهت ميخها براي لنگه هاي دروازه ها و براي وصلها حاضر ساخت و برنج بسيار که نتوان وزن نمود.
\par 4 و چوب سرو آزاد بيشمار زيرا اهل صَيدون و صور چوب سرو آزاد بسيار براي داود آوردند.
\par 5 و داود گفت: « پسر من سليمان صغير و نازک است و خانه اي براي يهُوَه بايد بنا نمود، مي بايست بسيار عظيم و نامي و جليل در تمامي زمينها بشود؛ لهذا حال برايش تهيه مي بينم.» پس داود قبل از وفاتش تهيه بسيار ديد.
\par 6 پس پسر خود سليمان را خوانده، او را وصيت نمود که خانه اي براي يهُوَه خداي اسرائيل بنا نمايد.
\par 7 و داود به سليمان گفت که : « اي پسرم! من اراده داشتم که خانه اي براي اسم يهُوَه خداي خود بنا نمايم.
\par 8 ليکن کلام خداوند بر من نازل شده، گفت: چونکه بسيار خون ريخته اي و جنگهاي عظيم کرده اي، پس خانه اي براي اسم من بنا نخواهي کرد، چونکه به حضور من بسيار خون بر زمين ريخته اي.
\par 9 اينک پسري براي تو متولد خواهد شد که مرد آرامي خواهد بود زيرا که من او را از جميع دشمنانش از هر طرف آرامي خواهم بخشيد، چونکه اسم او سليمان خواهد بود در ايام او اسرائيل را سلامتي و راحت عطا خواهم فرمود.
\par 10 او خانه اي براي اسم من بنا خواهد کرد و او پسر من خواهد بود و من پدر او خواهم بود. و کُرسي سلطنت او را بر اسرائيل تا ابدالآباد پايدار خواهم گردانيد.
\par 11 پس حال اي پسر من خداوند همراه تو باد تا کامياب شوي و خانه يهُوَه خداي خود را چنانکه درباره تو فرموده است بنا نمايي.
\par 12 اما خداوند تو را فطانت و فهم عطا فرمايد و تو را درباره اسرائيل وصيت نمايد تا شريعت يهُوَه خداي خود را نگاه داري.
\par 13 آنگاه اگر متوجه شده، فرايض و احکامي را که خداوند به موسي درباره اسرائيل امر فرموده است، به عمل آوري کامياب خواهي شد. پس قوي دلير باش و ترسان و هراسان مشو.
\par 14 و اينک من در تنگي خود صد هزار وزنه طلا و صد هزار وزنه نقره و برنج و آهن اينقدر زياده که به وزن نيايد، براي خانه خداوند حاضر کرده ام؛ و چوب و سنگ نيز مهيا ساخته ام و تو بر آنها مزيد کن.
\par 15 و نزد تو عمله بسيارند، از سنگ بران و سنگتراشان و نجاران و اشخاص هنرمند براي هر صنعتي.
\par 16 طلا و نقره و برنج و آهن بيشمار است پس برخيز و مشغول باش و خداوند همراه تو باد.»
\par 17 و داود تمامي سروران اسرائيل را امر فرمود که پسرش سليمان را اعانت نمايند.
\par 18 ( و گفت): « آيا يهُوَه خداي شما با شما نيست و آيا شما را از هر طرف آرامي نداده است؟ زيرا ساکنان زمين را به دست من تسليم کرده است و زمين به حضور خداوند و به حضور قوم او مغلوب شده است.پس حال دلها و جانهاي خود را متوجه سازيد تا يهُوَه خداي خويش را بطلبيد و برخاسته، مَقدس يهُوَه خداي خويش را بنا نمايد تا تابوت عهد خداوند و آلات مقدس خدا را به خانه اي که به جهت اسم يهُوَه بنا مي شود درآوريد.»
\par 19 پس حال دلها و جانهاي خود را متوجه سازيد تا يهُوَه خداي خويش را بطلبيد و برخاسته، مَقدس يهُوَه خداي خويش را بنا نمايد تا تابوت عهد خداوند و آلات مقدس خدا را به خانه اي که به جهت اسم يهُوَه بنا مي شود درآوريد.»
 
\chapter{23}

\par 1 و چون داود پير و سالخورده شد، پسر خود سُليمان را به پادشاهي اسرائيل نصب نمود.
\par 2 و تمامي سروران اسرائيل و کاهنان و لاويان را جمع کرد.
\par 3 و لاويان از سي ساله و بالاتر شمرده شدند و عدد ايشان بر حسب سرهاي مردان ايشان، سي و هشت هزار بود.
\par 4 از ايشان بيست و چهار هزار به جهت نظارت عمل خانه خداوند و شش هزار سروران و داوران بودند.
\par 5 و چهار هزار دربانان و چهار هزار نفر بودند که خداوند را به آلاتي که به جهت تسبيح ساخته شد، تسبيح خواندند.
\par 6 و داود ايشان را بر حسب پسران لاوي يعني جَرشون و قَهات و مَراري به فرقه ها تقسيم نمود.
\par 7 از جَرشونيان لَعدان و شِمعي.
\par 8 پسران لَعدان اول يحيئيل و زيتام و سومين يوئيل.
\par 9 پسران شِمعي شَلُومِيت و حَزِيئِيل و هاران سه نفر. اينان رؤساي خاندانهاي آباي لَعدان بودند.
\par 10 و پسران شِمعي يحَت و زينا و يعُوش و بَريعَه. اينان چهار پسر شِمعي بودند.
\par 11 و يحَت اولين و زيزا دومين و يعُوش و بَريعَه پسران بسيار نداشتند؛ از اين سبب يک خاندان آبا از ايشان شمرده شد.
\par 12 پسران قَهات عَمرام و يصهار و حَبرُون و عُزّيئيل چهار نفر.
\par 13 پسران عَمرام هارون و موسي و هارون ممتاز شد تا او و پسرانش قدس الاقداس را پيوسته تقديس نمايند و به حضور خداوند بخور بسوزانند و او را خدمت نمايند و به اسم او هميشه اوقات برکت دهند.
\par 14 و پسران موسي مرد خدا با سبط لاوي ناميده شدند.
\par 15 پسران موسي جَرشوم و اَلِعازار.
\par 16 از پسران جَرشُوم شَبُوئيل رئيس بود.
\par 17 و از پسران اَلِعازار رَحَبيا رئيس بود و اَلِعازار را پسر ديگر نبود؛ اما پسران رَحَبيا بسيار زياد بودند.
\par 18 از پسران يصهار شَلُوميت رئيس بود.
\par 19 پسران حَبرُون، اولين يرِيا و دومين اَمريا و سومين يحزِيئيل و چهارمين يقمَعام.
\par 20 پسران عُزّيئيل اولين ميکا و دومين يشِّيا.
\par 21 پسران مَراري مَحلي و مُوشِي و پسران مَحلِي اَلِعازار و قَيس.
\par 22 و اَلِعازار مُرد و او را پسري نبود؛ ليکن دختران داشت و برادران ايشان پسرانِ قَيس ايشان را به زني گرفتند.
\par 23 پسران مُوشِي مَحلي و عادَر و يريموت سه نفر بودند.
\par 24 اينان پسران لاوي موافق خاندانهاي آباي خود و رؤساي خاندانهاي آبا از آناني که شمرده شدند بر حسب شماره اسماي سرهاي خود بودند که از بيست ساله و بالاتر در عمل خدمت خانه خداوند مي پرداختند.
\par 25 زيرا که داود گفت: « يهُوَه خداي اسرائيل قوم خويش را آرامي داده است و او در اورشليم تا به ابد ساکن مي باشد.
\par 26 و نيز لاويان را ديگر لازم نيست که مسکن و همه اسباب خدمت را بردارند.»
\par 27 لهذا بر حسب فرمان آخر داود پسران لاوي از بيست ساله و بالاتر شمرده شدند.
\par 28 زيرا که منصب ايشان به طرف بني هارون بود تا خانه خداوند را خدمت نمايند، در صحن ها و حجره ها و براي تطهير همه چيزهاي مقدس و عمل خدمت خانه خدا.
\par 29 و بر نان تَقدِمه و آرد نرم به جهت هديه آردي و قرصهاي فطير و آنچه بر ساج پخته مي شود و رَبيکه ها و بر همه کيلها و وزنها.
\par 30 و تا هر صبح براي تسبيح و حمد خداوند حاضر شوند و همچنين هر شام.
\par 31 و به جهت گذرانيدن همه قرباني هاي سوختني براي خداوند در هر روز سبت و غرّه ها و عيدها بر حسب شماره و بر وفق قانون آنها دائماً به حضور خداوند.و براي نگاه داشتن وظيفه خيمه اجتماع و وظيفه قدس و وظيفه برادران خود بني هارون در خدمت خانه خداوند.
\par 32 و براي نگاه داشتن وظيفه خيمه اجتماع و وظيفه قدس و وظيفه برادران خود بني هارون در خدمت خانه خداوند.
 
\chapter{24}

\par 1 و اين است فرقه هاي بني هارون: پسران هارون، ناداب و اَبِيهُو و اَلِعازار و ايتامار.
\par 2 و ناداب و اَبِيهُو قبل از پدر خود مُردند و پسري نداشتند، پس اَلِعازار و ايتامار به کهانت پرداختند.
\par 3 و داود با صادوق که از بني اَلِعازار بود و اَخِيمَلَک که از بني ايتامار بود،
\par 4 و از پسران اَلِعازار مرداني که قابل رياست بودند، زياده از بني ايتامار يافت شدند. پس شانزده رئيس خاندان آبا از بني اَلِعازار و هشت رئيس خاندان آبا از بني ايتامار معين ايشان را بر حسب وکالت ايشان بر خدمت ايشان تقسيم کردند.
\par 5 پس اينان با آنان به حسب قرعه معين شدند زيرا که رؤساي قدس و رؤساي خانه خدا هم از بني اَلِعازار و هم از بني ايتامار بودند.
\par 6 و شَمَعيا ابن نَتَنئيل کاتب که از بني لاوي بود، اسمهاي ايشان را به حضور پادشاه و سروران و صادوق کاهن و اَخِيمَلَک بن ابياتار و رؤساي خاندان آباي کاهنان و لاويان نوشت و يک خاندان آبا به جهت اَلِعازار گرفته شد و يک به جهت ايتامار گرفته شد.
\par 7 و قرعه اول براي يهُوَياريب بيرون آمد و دوم براي يدَعيا،
\par 8 و سوم براي حاريم و چهارم براي سعُوريم،
\par 9 و پنجم براي مَلکيه و ششم براي مَيامين،
\par 10 و هفتم براي هَقُّوص و هشتم براي اَبِيا،
\par 11 و نهم براي يشُوع و دهم براي شَکُنيا،
\par 12 و يازدهم براي اَلياشيب و دوازدهم براي ياقيم،
\par 13 و سيزدهم براي حُفَّه و چهاردهم براي يشَبآب،
\par 14 و پانزدهم براي بِلجَه و شانزدهم براي اِمير،
\par 15 و هفدهم براي حيزير و هجدهم براي هِفصيص،
\par 16 و نوزدهم براي فَتَحيا و بيستم براي يحَزقيئيل،
\par 17 و بيست و يکم براي ياکين و بيست و دوم براي جامُول،
\par 18 و بيست و سوم براي دَلايا و بيست و چهارم براي مَعَزيا.
\par 19 پس اين است وظيفه ها و خدمت هاي ايشان به جهت داخل شدن در خانه خداوند بر حسب قانوني که به واسطه پدر ايشان هارون موافق فرمان يهُوَه خداي اسرائيل به ايشان داده شد.
\par 20 و اما درباره بقيه بني لاوي، از بني عَمرام شُوبائيل و از بني شوبائيل يحَديا.
\par 21 و اما رَحَبيا. از بني رَحَبيا يشِياي رئيس،
\par 22 و از بني يصهار شَلُومُوت و از بني شَلُومُوت يحَت.
\par 23 و از بني حَبرُون يريا و دومين اَمَريا و سومين يحزيئيل و چهارمين يقمَعام.
\par 24 از بني عُزّيئيل ميکا و از بني ميکا شامير.
\par 25 و برادر ميکا يشِيا و از بني يشِيا زکريا.
\par 26 و از بني مَراري مَحلي و مُوشِي و پسرِ يعزيابَنُو.
\par 27 و از بني مَراري پسران يعزيا بَنُو و شُوهَم و زَکُّور و عِبري.
\par 28 و پسر مَحلي اَلِعازار و او را فرزندي نَبُود.
\par 29 و اما قَيس، از بني قَيس يرَحميئيل، 
\par 30 و از بني مُوشِي مَحلي و عادَر و يريمُوت. اينان بر حسب خاندان آباي ايشان بني لاوي مي باشند.ايشان نيز مثل برادران خود بني هارون به حضور داود پادشاه و صادوق و اَخِيمَلَک و رؤساي آباي کَهَنَه و لاويان قرعه انداختند يعني خاندانهاي آباي برادر بزرگتر برابر خاندانهاي کوچکتر او بودند.
\par 31 ايشان نيز مثل برادران خود بني هارون به حضور داود پادشاه و صادوق و اَخِيمَلَک و رؤساي آباي کَهَنَه و لاويان قرعه انداختند يعني خاندانهاي آباي برادر بزرگتر برابر خاندانهاي کوچکتر او بودند.
 
\chapter{25}

\par 1 و داود و سرداران لشکر بعضي از پسران آساف و هِيمان و يدُوتُون را به جهت خدمت جدا ساختند تا با بربط و عود و سنج نبوت نمايند؛ و شماره آناني که بر حسب خدمت خود به کار مي پرداختند اين است:
\par 2 و اما از بني آساف، زَکُّور و يوسف و نَتَنيا و اَشرَئيلَه پسران آساف زير حکم آساف بودند که بر حسب فرمان پادشاه نبوت مي نمود.
\par 3 و از يدُوتُون، پسران يدُوتون جَدَليا و صَرِي و اَشعيا و حَشَبيا و مَتَّتيا شش نفر زير حکم پدر خويش يدُوتُون با بربطها بودند که با حمد و تسبيح خداوند نبوت مي نمود.
\par 4 و از هَيمان، پسران هَيمان بُقِّيا و مَتَنيا و عُزّيئيل و شَبُوئيل و يريموت و حَنَنيا و حَناني و اَلِيآتَه و جدَّلتِي و رُومَمتِي عَزَر و يشبِقاشَه و مَلُّوتِي و هُوتير و مَحزِيوت.
\par 5 جميع اينها پسران هِيمان بودند که در کلام خدا به جهت برافراشتن بوق رايي پادشاه بود. و خدا به هِيمان چهارده پسر و سه دختر داد .
\par 6 جميع اينها زير فرمان پدران خويش بودند تا در خانه خداوند با سنج و عود و بربط بسرايند و زير دست پادشاه و آساف و يدُوتُون و هِيمان به خدمت خانه خدا بپردازند.
\par 7 و شماره ايشان با برادران ايشان که سراييدن را به جهت خداوند آموخته بودند، يعني همه کسان ماهر دويست و هشتاد وهشت نفر بودند.
\par 8 و براي وظيفه هاي خود کوچک با بزرگ و معلم با تلميذ علي السويه قرعه انداختند.
\par 9 پس قرعه اولِ بني آساف براي يوسف بيرون آمد. و قرعه دوم براي جَدَليا و او و برادرانش و پسرانش دوازده نفر بودند.
\par 10 و سوم براي زَکّور و پسران و برادران او دوازده نفر.
\par 11 و چهارم براي يصرِي و پسران و برادران او دوازده نفر.
\par 12 و پنجم براي نَتَنيا و پسران و برادران او دوازده نفر.
\par 13 و ششم براي بُقِّيا و پسران و برادران او دوازده نفر.
\par 14 و هفتم براي يشَرئيله و پسران و برادران او دوازده نفر.
\par 15 و هشتم براي اِشَعيا و پسران و برادران او دوازده نفر.
\par 16 نهم براي مَتَنيا و پسران و برادران او دوازده نفر.
\par 17 و دهم براي شِمعي و پسران او و برادران او دوازده نفر.
\par 18 و يازدهم براي عَزَرئيل و پسران و برادران او دوازده نفر.
\par 19 و دوازدهم براي حَشَبيا و پسران و برادران او دوازده نفر.
\par 20 و سيزدهم براي شُوبائيل و پسران و برادران او دوازده نفر.
\par 21 و چهاردهم براي مَتَّتيا و پسران و برادران او دوازده نفر.
\par 22 و پانزدهم براي يريموت و پسران و برادران او دوازده نفر.
\par 23 و شانزدهم براي حَنَنيا و پسران و برادران او دوازده نفر.
\par 24 و هفدهم براي يشبَقاشه و پسران و برادران او دوازده نفر.
\par 25 و هجدهم براي حَنانِي و پسران و برادران او دوازده نفر.
\par 26 و نوزدهم براي مَلوتي و پسران و برادران او دوازده نفر.
\par 27 و بيستم براي اِيلِيآتَه و پسران و برادران او دوازده نفر.
\par 28 و بيست و يکم براي هُوتير و پسران و برادران او دوازده نفر.
\par 29 و بيست و دوم براي جِدَّلتِي و پسران و برادران او دوازده نفر.
\par 30 و بيست و سوم براي مَحزِيوت و پسران و برادران او دوازده نفر.و بيست و چهارم براي رُومَمتِي عَزَر و پسران و برادران او دوازده نفر.
\par 31 و بيست و چهارم براي رُومَمتِي عَزَر و پسران و برادران او دوازده نفر.
 
\chapter{26}

\par 1 و اما فرقه هاي دربانان: پس از قُورَحيان مَشَلَميا ابن قُورِي که از بني آساف بود.
\par 2 و مَشَلَميا را پسران بود. نخست زاده اش زکريا و دوم يديعيئيل و سوم زَبَديا و چهارم يتنِيئيل.
\par 3 و پنجم عِيلام و ششم يهوحانان و هفتم اَلِيهُوعِيناي.
\par 4 و عُوبِيد اَدُوم را پسران بود: نخست زاده اش، شَمَعيا و دوم يهُوزاباد و سوم يوآخ و چهارم ساکار و پنجم نَتَنئيل.
\par 5 و ششم عَمّيئيل و هفتم يسّاکار و هشتم فَعَلتاي زيرا خدا او را برکت داده بود.
\par 6 و براي پسرش شَمَعيا پسراني که بر خاندان آباي خويش تسلط يافتند، زاييده شدند زيرا که ايشان مردان قوي شجاع بودند.
\par 7 پسران شَمَعيا عُتنِي و رَفائيل و عُوبيد و اَلزاباد که برادران او مردان شجاع بودند و اَلِهُو و سَمَکيا.
\par 8 جميع اينها از بني عُوبيد اَدُوم بودند و ايشان با پسران و برادران ايشان در قوتِ خدمت مردان قابل بودند يعني شصت و دو نفر (از اولاد) عوبيد اَدُوم.
\par 9 و مَشَلَميا هجده نفر مردان قابل از پسران و برادران خود داشت.
\par 10 و حُوسَه که از بني مَراري بود پسران داشت که شِمرِي رئيس ايشان بود زيرا اگر چه نخست زاده نبود، پدرش او را رئيس ساخت.
\par 11 و دوم حِلقيا و سوم طَبَليا و چهارم زکريا و جميع پسران و برادران حُوسَه سيزده نفر بودند.
\par 12 و به اينان يعني به رؤساي ايشان فرقه هاي دربانان داده شد و وظيفه هاي ايشان مثل برادران ايشان بود تا در خانه خداوند خدمت نمايند.
\par 13 و ايشان از کوچک و بزرگ بر حسب خاندان آباي خويش براي هر دوازده قرعه انداختند.
\par 14 و قرعه شرقي به شَلَميا افتاد و بعد از او براي پسرش زکريا که مُشيرِ دانا بود، قرعه انداختند و قرعه او به سمت شمال بيرون آمد.
\par 15 و براي عُوبيد اَدُوم (قرعه) جنوبي و براي پسرانش (قرعه) بيت المال.
\par 16 و براي شُفّيم و حُوسَه قرعه مغربي نزد دروازه شَلَکَت در جاده اي که سر بالا مي رفت و محرس اين مقابل محرس آن بود.
\par 17 و به طرف شرقي شش نفر از لاويان بودند و به طرف شمال هر روزه چهار نفر و به طرف جنوب هر روزه چهار نفر و نزد بيت المال جفت جفت.
\par 18 و به طرف غربي فَروار براي جاده سربالا چهار نفر و براي فَروار دو نفر.
\par 19 اينها فرقه هاي دربانان از بني قُورَح و از بني مَراري بودند.
\par 20 و اما از لاويان اَخِيا بر خزانه خانه خدا و بر خزانه موقوفات بود.
\par 21 و اما بني لادان: از پسران لادان جَرشوني رؤساي خاندان آباي لادان يحيئيلي جَرشوني.
\par 22 پسران يحيئيلي زيتام و برادرش يوئيل بر خزانه خانه خداوند بودند.
\par 23 از عَمراميان و از يصهاريان و از حَبرُونيان و از عُزّيئيليان.
\par 24 و شَبُوئيل بن جَرشُوم بن موسي ناظر خزانه ها بود.
\par 25 و از برادرانش بني اَلِعازار، پسرش رَحَبيا و پسرش اَشعَيا و پسرش يورام و پسرش زِکرِي و پسرش شَلوميت.
\par 26 اين شَلُوميت و برادرانش بر جميع خزائن موقوفاتي که داود پادشاه وقف کرده بود و رؤساي آبا و رؤساي هزاره ها و صده ها و سرداران لشکر بودند.
\par 27 از جنگها و غنيمت ها وقف کردند تا خانه خداوند را تعمير نمايند.
\par 28 و هر آنچه سموئيل رايي و شاؤل بن قَيس و اَبنيرين نير و يوآب بن صَرُويه وقف کرده بودند و هر چه هر کس وقف کرده بود زير دست شَلُوميت و برادرانش بود.
\par 29 و از يصهاريان کَنَنيا و پسرانش براي اعمال خارجه اسرائيل صاحبان منصب و داوران بودند.
\par 30 و از حَبرُونيان حَشَبيا و برادرانش هزار و هفتصد نفر مردان شجاع به آن طرف اُردُن به سمت مغرب به جهت هر کار خداوند و به جهت خدمت پادشاه بر اسرائيل گماشته شده بودند.
\par 31 از حَبرُونيان: بر حسب انساب آباي ايشان يريا رئيس حَبرُونيان بود و در سال چهلم سلطنت داود طلبيده شدند و در ميان ايشان مردان شجاع در يعزير جِلعاد يافت شدند.و از برادرانش دو هزار و هفتصد مرد شجاع و رئيس آبا بودند. پس داود پادشاه ايشان را بر رؤبينيان و جاديان و نصف سبط مَنَّسي براي همه امور خدا و امور پادشاه گماشت.
\par 32 و از برادرانش دو هزار و هفتصد مرد شجاع و رئيس آبا بودند. پس داود پادشاه ايشان را بر رؤبينيان و جاديان و نصف سبط مَنَّسي براي همه امور خدا و امور پادشاه گماشت.
 
\chapter{27}

\par 1 و از بني اسرائيل بر حسب شماره ايشان از رؤساي آبا و رؤساي هزاره و صده و صاحبان منصب که پادشاه را در همه امور فرقه هاي داخله و خارجه ماه به ماه در همه ماههاي سال خدمت مي کردند، هر فرقه بيست و چهار هزار نفر بودند.
\par 2 و بر فرقه اول براي ماه اول يشُبعام بن زَبدِيئيل بود و در فرقه او بيست و چهار هزار نفر بودند.
\par 3 او از پسران فارَص رئيس جميع رؤساي لشکر، به جهت ماه اول بود.
\par 4 و بر فرقه ماه دوم دُوداي اَخُوخِي و از فرقه او مَقلُوت رئيس بود و در فرقه او بيست و چهار هزار نفر بودند.
\par 5 و رئيس لشکر سوم براي ماه سوم بنايا ابن يهُوياداع کاهن بزرگ بود و در فرقه او بيست چهار هزار نفر بودند.
\par 6 اين همان بنايا است که در ميان آن سي نفر بزرگ بود و بر آن سي نفر برتري داشت و از فرقه او پسرش عَمّيزاباد بود.
\par 7 و رئيس چهارم براي ماه چهارم عَسائيل برادر يوآب و بعد از او برادرش زَبَديا بود و در فرقه او بيست و چهار هزار نفر بودند.
\par 8 و رئيس پنجم براي ماه پنجم شَمهُوتِ يزراحي بود و در فرقه او بيست و چهار هزار نفر بودند.
\par 9 و رئيس ششم براي ماه ششم عيرا ابن عِقّيش تَقِّوعِي بود و در فرقه او بيست و چهار هزار نفر بودند.
\par 10 و رئيس هفتم براي ماه هفتم حالَصِ فَلُونِي از بني افرايم بود و در فرقه او بيست و چهار هزار نفر بودند.
\par 11 و رئيس هشتم براي ماه هشتم سِبکاي حُوشاتِي از زارَحيان بود و در فرقه او بيست و چهار هزار نفر بودند.
\par 12 و رئيس نهم براي ماه نهم اَبيعَزَرِ عَناتوتِي از بني بنيامين بود و در فرقه او بيست و چهار هزار نفر بودند.
\par 13 و رئيس دهم براي ماه دهم مَهراي نَطُوفاتي از زارَحيان بود و در فرقه او بيست و چهار هزار نفر بودند.
\par 14 و رئيس يازدهم براي ماه يازدهم بناياي فِرعاتوني از بني افرايم بود و در فرقه او بيست و چهار هزار نفر بودند.
\par 15 و رئيس دوازدهم براي ماه دوازدهم خَلداي نَطُوفاتي از بني عُتنِيئيل و در فرقه او بيست و چهار هزار نفر بودند.
\par 16 و اما رؤساي اسباط بني اسرائيل: رئيس رؤبينيان اَلِعازار بن زِکرِي، و رئيس شَمعونيان شَفَطيا ابن مَعکَه.
\par 17 و رئيس لاويان عَشَبيا ابن قَمُوئيل و رئيس بني هارون صادوق.
\par 18 و رئيس يهودا اَلِيهُو از برادران داود و رئيس يسّاکار عُمرِي ابن ميکائيل.
\par 19 و رئيس زبولون يشمَعيا ابن عُوبَديا و رئيس نَفتالي يريموت بن عَزريئيل.
\par 20 و رئيس بني افرايم هُوشَع بن عَزَريا و رئيس نصف سبط مَنَّسي يوئيل بن فَدايا.
\par 21 و رئيس نصف سبط مَنَّسي در جِلعاد يدُّو ابن زکريا و رئيس بنيامين يعسيئيل بن اَبنير.
\par 22 و رئيس دان عَزَرئيل بن يرُوحام. اينها رؤساي اسباط اسرائيل بودند.
\par 23 و داود شماره کساني که بيست ساله و کمتر بودند، نگرفت زيرا خداوند وعده داده بود که اسرائيل را مثل ستارگان آسمان کثير گرداند.
\par 24 و يوآب بن صَرُويه آغاز شمردن نمود، اما به اتمام نرسانيد و از اين جهت غضب بر اسرائيل وارد شد و شماره آنها در دفتر اخبار ايام پادشاه ثبت نشد.
\par 25 و ناظر انبارهاي پادشاه عَزمُوت بن عَدِيئيل بود و ناظر انبارهاي مزرعه ها که در شهر ها و در دهات و در قلعه ها بود، يهوناتان بن عُزّيا بود.
\par 26 و ناظر عملجات مزرعه ها که کار زمين مي کردند، عَزرِي ابن کَلُوب بود.
\par 27 و ناظر تاکستانها شِمعي راماتي بود و ناظر محصول تاکستانها و انبارهاي شراب زَبدِي شِفماتي بود.
\par 28 و ناظر درختان زيتون و افراغ که در همواري بود بَعل حانانِ جدِيري بود و ناظر انبارهاي روغن يوآش بود. 
\par 29 و ناظر رمه هايي که در شارون مي چريدند شِطراي شاروني بود. و ناظر رمه هايي که در واديها بودند شافاط بن عَدلائي بود.
\par 30 و ناظر شتران عُوبيلِ اسمعيلي بود و ناظر الاغها يحَدياي ميرونوتي بود.
\par 31 و ناظر گله ها يازيز هاجري بود. جميع اينان ناظران اندوخته هاي داود پادشاه بودند.
\par 32 و يهُوناتان عموي داود مشير و مرد دانا و فقيه بود و يحيئيل بن حَکمُوني همراه پسران پادشاه بود.
\par 33 اَخيتُوفَل مشير پادشاه و حوشاي اَرکِي دوست پادشاه بود.و بعد از اَخِيتُوفَل يهُوياداع بن بنايا و اَبياتار بودند و سردار لشکر پادشاه يوآب بود.
\par 34 و بعد از اَخِيتُوفَل يهُوياداع بن بنايا و اَبياتار بودند و سردار لشکر پادشاه يوآب بود.
 
\chapter{28}

\par 1 و داود جميع رؤساي اسرائيل را از رؤساي اسباط و رؤساي فرقه هايي که پادشاه را خدمت مي کردند و رؤساي هزاره و رؤساي صده و ناظران همه اندوخته ها و اموال پادشاه و پسرانش را با خواجه سرايان و شجاعان و جميع مردان جنگي در اورشليم جمع کرد.
\par 2 پس داود پادشاه برپا ايستاده، گفت: « اي برادرانم و اي قوم من! مرا بشنويد! من اراده داشتم خانه اي که آرامگاه تابوت عهد خداوند و پاي انداز پايهاي خداي ما باشد بنا نمايم، و براي بناي آن تدارک ديده بودم.
\par 3 ليکن خدا مرا گفت: تو خانه اي به جهت اسم من بنا نخواهي نمود، زيرا مرد جنگي هستي و خون ريخته اي.
\par 4 ليکن يهُوَه خداي اسرائيل مرا از تمامي خاندان پدرم برگزيده است که بر اسرائيل تا ابد پادشاه بشوم، زيرا که يهودا را براي رياست اختيار کرد و از خاندان يهودا خاندان پدر مرا و از فرزندان پدرم مرا پسند کرد تا مرا بر تمامي اسرائيل به پادشاهي نصب نمايد.
\par 5 و از جميع پسران من (زيرا خداوند پسران بسيار به من داده است)، پسرم سليمان را برگزيده است تا بر کرسي سلطنت خداوند بر اسرائيل بنشيند.
\par 6 و به من گفت: پسر تو سليمان، او است که خانه مرا و صحن هاي مرا بنا خواهد نمود، زيرا که او را برگزيده ام تا پسر من باشد و من پدر او خواهم بود.
\par 7 و اگر او به جهت بجا آوردن فرايض و احکام من مثل امروز ثابت بماند، آنگاه سلطنت او را تا به ابد استوار خواهم گردانيد.
\par 8 پس الآن در نظر تمامي اسرائيل که جماعت خداوند هستند و به سمع خداي ما متوجه شده، تمامي اوامر يهُوَه خداي خود را بطلبيد تا اين زمين نيکو را به تصرف آورده، آن را بعد از خودتان به پسران خويش تا به ابد به ارثيت واگذاريد.
\par 9 « و تو اي پسر من سليمان خداي پدر خود را بشناس و او را به دل کامل و به ارادت تمام عبادت نما زيرا خداوند همه دلها را تفتيش مي نمايد و هر تصور فکرها را ادراک مي کند؛ و اگر او را طلب نمايي، او را خواهي يافت؛ اما اگر او را ترک کني، تو را تا به ابد دور خواهد انداخت.
\par 10 حال با حذر باش زيرا خداوند تو را برگزيده است تا به خانه اي به جهت مَقدَسِ او بنا نمايي. پس قوي شده، مشغول باش.»
\par 11 و داود به پسر خود سليمان نمونه رواق و خانه ها و خزاين و بالاخانه ها و حُجره هاي اندروني آن و خانه کرسي رحمت،
\par 12 و نمونه هر آنچه را که از روح به او داده شده بود، براي صحن هاي خانه خداوند و براي همه حجره هاي گرداگردش و براي خزاين خانه خدا و خزاين موقوفات داد.
\par 13 و براي فرقه هاي کاهنان و لاويان و براي تمامي کار خدمت خانه خداوند و براي همه اسباب خدمت خانه خداوند.
\par 14 و از طلا به وزن براي همه آلات طلا به جهت هر نوع خدمتي و از نقره به وزن براي همه آلات نقره به جهت هر نوع خدمتي.
\par 15 و طلا را به وزن به جهت شمعدانهاي طلا و چراغهاي آنها به جهت هر شمعدان و چراغهايش، آن را به وزن داد و براي شمعدانهاي نقره نيز نقره را به وزن به جهت هر چراغدان موافق کار هر شمعدان و چراغهاي آن.
\par 16 و طلا را به وزن به جهت ميزهاي نان تَقدِمه براي هر ميز عليحده و نقره را براي ميزهاي نقره.
\par 17 و زر خالص را براي چنگالها و کاسها و پياله ها و به جهت طاسهاي طلا موافق وزن هر طاس و به جهت طاسهاي نقره موافق وزن هر طاس.
\par 18 و طلاي مصفّي را به وزن به جهت مذبح بخور و طلا را به جهت نمونه مرکب کروبيان که بالهاي خود را پهن کرده، تابوت عهد خداوند را مي پوشانيدند.
\par 19 (و داود گفت): «خداوند اين همه را يعني تمامي کارهاي اين نمونه را از نوشته دست خود که بر من بود به من فهمانيد.»
\par 20 و داود به پسر خود سليمان گفت: « قوي و دلير باش و مشغول شو و ترسان و هراسان مباش، زيرا يهُوَه خدا که خداي من مي باشد، با تو است و تا همه کار خدمت خانه خداوند تمام نشود، تو را وا نخواهد گذاشت و تو را ترک نخواهد نمود.اينک فرقه هاي کاهنان و لاويان براي تمام خدمت خانه خدا (حاضرند) و براي هر گونه عمل همه کسان دلگرم که براي هر صنعتي مهارت دارند، با تو هستند و سروران و تمامي قوم مطيع کامل اوامر تو مي باشند.»
\par 21 اينک فرقه هاي کاهنان و لاويان براي تمام خدمت خانه خدا (حاضرند) و براي هر گونه عمل همه کسان دلگرم که براي هر صنعتي مهارت دارند، با تو هستند و سروران و تمامي قوم مطيع کامل اوامر تو مي باشند.»
 
\chapter{29}

\par 1 و داود پادشاه به تمامي جماعت گفت: «پسرم سليمان که خدا او را به تنهايي براي خود برگزيده، جوان و لطيف است و اين مُهِمّ عظيمي است زيرا که هيکل به جهت انسان نيست بلکه براي يهُوَه خدا است.
\par 2 و من به جهت خانه خداي خود به تمامي قوتم تدارک ديده ام، طلا را به جهت چيزهاي طلايي ونقره را براي چيزهاي نقره اي و برنج را به جهت چيزهاي برنجين و آهن را براي چيزهاي آهنين و چوب را به جهت چيزهاي چوبين و سنگ را جزع و سنگهاي ترصيع و سنگهاي سياه و سنگهاي رنگارنگ و هر قسم سنگ گرانبها و سنگ مَرمَرِ فراوان.
\par 3 و نيز چونکه به خانه خداي خود رغبت داشتم و طلا و نقره از اموال خاص خود داشتم، آن را علاوه بر هر آنچه به جهت خانه قُدس تدارک ديدم براي براي خانه خداي خود دادم.
\par 4 يعني سه هزار وزنه طلا از طلاي اُوفير و هفت هزار وزنه نقره خالص به جهت پوشانيدن ديوارهاي خانه ها.
\par 5 طلا را به جهت چيزهاي طلا و نقرا را به جهت چيزهاي نقره وبه جهت تمامي کاري که به دست صنعتگران ساخته مي شود. پس کيست که به خوشي دل خوشتن را امروز براي خداوند وقف نمايد؟»
\par 6 آنگاه رؤساي خاندانهاي آبا و رؤساي اسباط اسرائيل و سرداران هزاره و صده با ناظرانِ کارهاي پادشاه به خوشي دل هدايا آوردند.
\par 7 و به جهت خدمت خانه خدا پنج هزار وزنه و ده هزار درهم طلا و ده هزار وزنه نقره و هجده هزار وزنه برنج و صد هزار وزنه آهن دادند.
\par 8 و هر کس که سنگهاي گرانبها نزد او يافت شد، آنها را به خزانه خانه خداوند به دست يحيئيلِ جَرشوني داد.
\par 9 آنگاه قوم از آن رو که به خوشي دل هديه آورده بودند شاد شدند زيرا به دل کامل هداياي تبرّعي براي خداوند آوردند و داود پادشاه نيز بسيار شاد و مسرور شد.
\par 10 و داود به حضور تمامي جماعت خداوند را متبارک خواند و داود گفت: « اي يهُوَه خداي پدر ما اسرائيل تو از ازل تا به ابد متبارک هستي.
\par 11 و اي خداوند عظمت و جبروت و جلال و قوت و کبريا از آن تو است زيرا هر چه در آسمان و زمين است از آنِ تو مي باشد. و اي خداوند ملکوت از آنِ تو است و تو بر همه سر و متعال هستي.
\par 12 و دولت از تو مي آيد و تو بر همه حاکمي، و کبريا و جبروت در دست تو است و عظمت دادن و قوت بخشيدن به همه کس در دست تو است.
\par 13 و الآن اي خداي ما تو را حمد مي گوييم و اسم مجيد تو را تسبيح مي خوانيم.
\par 14 ليکن من کيستم و قوم من کيستند که قابليت داشته باشيم که به خوشي دل اينطور هدايا بياوريم؟ زيرا که همه اين چيزها از آن تو است و از دست تو به تو داده ايم.
\par 15 زيرا که ما مثل همه اجداد خود به حضور تو غريب و نزيل مي باشيم و ايام ما بر زمين مثل سايه است و هيچ دوام ندارد.
\par 16 اي يهُوَه خداي ما تمامي اين اموال که به جهت ساختن خانه براي اسم قدوس تو مهيا ساخته ايم، از دست تو و تمامي آن از آن تو مي باشد.
\par 17 و مي دانم اي خدايم که دلها را مي آزمايي و استقامت را دوست مي داري و من به استقامت دل خود همه اين چيزها را به خوشي دادم و الآن قوم تو را که اينجا حاضرند ديدم که به شادماني و خوشي دل هدايا براي تو آوردند.
\par 18 اي يهُوَه خداي پدران ما ابراهيم و اسحاق و اسرائيل اين را هميشه در تصور فکرهاي دل قوم خود نگاه دار و دلهاي ايشان را به سوي خود ثابت گردان.
\par 19 و به پسر من سليمان دل کامل عطا فرما تا اوامر و شهادات و فرايض تو را نگاه دارد، و جميع اين کارها را به عمل آورد و هيکل را که من براي آن تدارک ديدم بنا نمايد.»
\par 20 پس داود به تمامي جماعت گفت: « يهُوَه خداي خود را متبارک خوانيد.» و تمامي جماعت يهُوَه خداي پدران خويش را متبارک خوانده، به رو افتاده، خداوند را سجده کردند و پادشاه را تعظيم نمودند.
\par 21 و در فرداي آن روز براي خداوند ذبايح ذبح کردند و قرباني هاي سوختني براي خداوند گذرانيدند يعني هزار گاو و هزار قوچ و هزار بره با هداياي ريختني و ذبايح بسيار به جهت تمامي اسرائيل.
\par 22 و در آن روز به حضور خداوند به شادي عظيم اکل و شرب نمودند، و سليمان پسر داود را دوباره به پادشاهي نصب نموده، او را به حضور خداوند به رياست و صادوق را به کهانت مسح نمودند.
\par 23 پس سليمان بر کرسي خداوند نشسته، به جاي پدرش داود پادشاهي کرد و کامياب شد و تمامي اسرائيل او را اطاعت کردند.
\par 24 و جميع سروران و شجاعان و همه پسران داود پادشاه نيز مطيع سليمان پادشاه شدند.
\par 25 و خداوند سليمان را در نظر تمام اسرائيل بسيار بزرگ گردانيد و او را جلالي شاهانه داد که به هيچ پادشاه اسرائيل قبل از او داده نشده بود.
\par 26 پس داود بن يسي بر تمامي اسرائيل سلطنت نمود.
\par 27 و مدت سلطنت او بر اسرائيل چهل سال بود، اما در حَبرُون هفت سال سلطنت کرد و در اورشليم سي وسه سال پادشاهي کرد.
\par 28 و در پيري نيکو از عمر و دولت و حشمت سير شده، وفات نمود و پسرش سليمان به جايش پادشاه شد.
\par 29 و اينک امور اول و آخر داود پادشاه در سِفرِ اخبار سموئيل رايي و اخبار ناتان نبي و اخبار جاد رايي،با تمامي سلطنت و جبروت او و روزگاري که بر وي و بر اسرائيل و بر جميع ممالک آن اراضي گذشت، مکتوب است.
\par 30 با تمامي سلطنت و جبروت او و روزگاري که بر وي و بر اسرائيل و بر جميع ممالک آن اراضي گذشت، مکتوب است.


\end{document}