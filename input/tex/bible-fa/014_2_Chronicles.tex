\begin{document}

\title{2 Chronicles}

 
\chapter{1}

\par 1 و سلیمان پسر داود در سلطنت خود قوی شد و یهوه خدایش با وی می‌بود و او راعظمت بسیار بخشید.
\par 2 و سلیمان تمامی اسرائیل و سرداران هزاره وصده و داوران و هر رئیسی را که در تمامی اسرائیل بود از روسای خاندانهای آبا خواند.
\par 3 وسلیمان با تمامی جماعت به مکان بلندی که درجبعون بود رفتند زیرا خیمه اجتماع خدا که موسی بنده خداوند آن را در بیابان ساخته بود درآنجا بود.
\par 4 لیکن داود تابوت خدا را از قریه یعاریم به‌جایی که داود برایش مهیا کرده بود بالاآورد و خیمه‌ای برایش در اورشلیم برپا نمود.
\par 5 ومذبح برنجینی که بصلئیل بن اوری ابن حورساخته بود، در آنجا پیش مسکن خداوند ماند وسلیمان و جماعت نزد آن مسالت نمودند.
\par 6 پس سلیمان به آنجا نزد مذبح برنجینی که در خیمه اجتماع بود به حضور خداوند برآمده، هزارقربانی سوختنی بر آن گذرانید.
\par 7 در همان شب خدا به سلیمان ظاهر شد و اورا گفت: «آنچه را که به تو بدهم طلب نما.»
\par 8 سلیمان به خدا گفت: «تو به پدرم داود احسان عظیم نمودی و مرا به‌جای او پادشاه ساختی.
\par 9 حال‌ای یهوه خدا به وعده خود که به پدرم داوددادی وفا نما زیرا که تو مرا بر قومی که مثل غبار زمین کثیرند پادشاه ساختی.
\par 10 الان حکمت ومعرفت را به من عطا فرما تا به حضور این قوم خروج و دخول نمایم زیرا کیست که این قوم عظیم تو را داوری تواند نمود؟»
\par 11 خدا به سلیمان گفت: «چونکه این در خاطرتو بود و دولت و توانگری و حشمت و جان دشمنانت را نطلبیدی و نیز طول ایام را نخواستی بلکه به جهت خود حکمت و معرفت رادرخواست کردی تا بر قوم من که تو را بر سلطنت ایشان نصب نموده‌ام داوری نمایی.
\par 12 لهذاحکمت و معرفت به تو بخشیده شد و دولت وتوانگری و حشمت را نیز به تو خواهم داد که پادشاهانی که قبل از تو بودند مثل آن را نداشتند وبعد از تو نیز مثل آن را نخواهند داشت.»
\par 13 پس سلیمان از مکان بلندی که در جبعون بود ازحضور خیمه اجتماع به اورشلیم مراجعت کرد وبر اسرائیل سلطنت نمود.
\par 14 و سلیمان ارابه‌ها و سواران جمع کرده، هزار و چهارصد ارابه و دوازده هزار سوار داشت، و آنها را در شهرهای ارابه‌ها و نزد پادشاه دراورشلیم گذاشت.
\par 15 و پادشاه نقره و طلا را دراورشلیم مثل سنگها و چوب سرو آزاد را مثل چوب افراغ که در همواری است فراوان ساخت.
\par 16 و اسبهای سلیمان از مصر آورده می‌شد، و تاجران پادشاه دسته های آنها را می‌خریدند هردسته را به قیمت معین.و یک ارابه را به قیمت ششصد مثقال نقره از مصر بیرون می‌آوردند ومی رسانیدند و یک اسب را به قیمت صد و پنجاه، و همچنین برای جمیع پادشاهان حتیان وپادشاهان ارام به توسط آنها بیرون می‌آوردند.
\par 17 و یک ارابه را به قیمت ششصد مثقال نقره از مصر بیرون می‌آوردند ومی رسانیدند و یک اسب را به قیمت صد و پنجاه، و همچنین برای جمیع پادشاهان حتیان وپادشاهان ارام به توسط آنها بیرون می‌آوردند.
 
\chapter{2}

\par 1 و سلیمان قصد نمود که خانه‌ای برای اسم یهوه و خانه‌ای به جهت سلطنت خودش بنا نماید.
\par 2 و سلیمان هفتاد هزار نفر برای حمل بارها، و هشتاد هزار نفر برای بریدن چوب درکوهها، و سه هزار و ششصد نفر برای نظارت آنهاشمرد.
\par 3 و سلیمان نزد حورام، پادشاه صور فرستاده، گفت: «چنانکه با پدرم داود رفتار نمودی و چوب سرو آزاد برایش فرستادی تا خانه‌ای به جهت سکونت خویش بنا نماید، (همچنین با من رفتارنما).
\par 4 اینک من خانه‌ای برای اسم یهوه خدای خود بنا می‌نمایم تا آن را برای او تقدیس کنم وبخور معطر در حضور وی بسوزانم، و به جهت نان تقدمه دائمی و قربانی های سوختنی صبح وشام، و به جهت سبت‌ها و غره‌ها و عیدهای یهوه خدای ما زیرا که این برای اسرائیل فریضه‌ای ابدی است.
\par 5 و خانه‌ای که من بنا می‌کنم عظیم است، زیرا که خدای ما از جمیع خدایان عظیم ترمی باشد.
\par 6 و کیست که خانه‌ای برای او تواندساخت؟ زیرا فلک و فلک الافلاک گنجایش او راندارد؛ و من کیستم که خانه‌ای برای وی بنا نمایم؟ نی بلکه برای سوزانیدن بخور در حضور وی.
\par 7 و حال کسی را برای من بفرست که در کار طلا و نقره و برنج و آهن و ارغوان و قرمز و آسمانجونی ماهرو در صنعت نقاشی دانا باشد، تا با صنعتگرانی که نزد من در یهودا و اورشلیم هستند که پدر من داودایشان را حاضر ساخت، باشد.
\par 8 و چوب سروآزاد و صنوبر و چوب صندل برای من از لبنان بفرست، زیرا بندگان تو را می‌دانم که در بریدن چوب لبنان ماهرند و اینک بندگان من با بندگان توخواهند بود.
\par 9 تا چوب فراوان برای من مهیاسازند زیرا خانه‌ای که من بنا می‌کنم عظیم وعجیب خواهد بود.
\par 10 و اینک به چوب بران که این چوب را می‌برند، من بیست هزار کر گندم کوبیده شده، و بیست هزار کر جو و بیست هزاربت شراب و بیست هزار بت روغن برای بندگانت خواهم داد.»
\par 11 و حورام پادشاه صور مکتوب جواب داده، آن را نزد سلیمان فرستاد که «چون خداوند قوم خود را دوست می‌دارد از این جهت تو را به پادشاهی‌ایشان نصب نموده است.»
\par 12 و حورام گفت: «متبارک باد یهوه خدای اسرائیل که آفریننده آسمان و زمین می‌باشد، زیرا که به داودپادشاه پسری حکیم و صاحب معرفت و فهم بخشیده است تا خانه‌ای برای خداوند و خانه‌ای برای سلطنت خودش بنا نماید.
\par 13 و الان حورام را که مردی حکیم و صاحب فهم از کسان پدر من است فرستادم.
\par 14 و او پسرزنی از دختران دان است، و پدرش مرد صوری بود و به‌کار طلا و نقره و برنج و آهن و سنگ وچوب و ارغوان و آسمانجونی و کتان نازک وقرمز و هر صنعت نقاشی و اختراع همه اختراعات ماهر است، تا برای او با صنعتگران توو صنعتگران آقایم پدرت داود کاری معین بشود.
\par 15 پس حال آقایم گندم و جو و روغن و شراب راکه ذکر نموده بود، برای بندگان خود بفرستد.
\par 16 وما چوب از لبنان به قدر احتیاج تو خواهیم برید، وآنها را بستنه ساخته، بروی دریا به یافا خواهیم آورد و تو آن را به اورشلیم خواهی برد.»
\par 17 پس سلیمان تمامی مردان غریب را که درزمین اسرائیل بودند، بعد از شماره‌ای که پدرش داود آنها را شمرده بود شمرد، و صد و پنجاه وسه هزار و ششصد نفر از آنها یافت شدند.و ازایشان هفتاد هزار نفر برای حمل بارها و هشتادهزار نفر برای بریدن چوب در کوهها و سه هزار وشش صد نفر برای نظارت تا از مردم کار بگیرند، تعیین نمود.
\par 18 و ازایشان هفتاد هزار نفر برای حمل بارها و هشتادهزار نفر برای بریدن چوب در کوهها و سه هزار وشش صد نفر برای نظارت تا از مردم کار بگیرند، تعیین نمود.
 
\chapter{3}

\par 1 و سلیمان شروع کرد به بنا نمودن خانه خداوند در اورشلیم بر کوه موریا، جایی که (خداوند) بر پدرش داود ظاهر شده بود، جایی که داود در خرمنگاه ارنون یبوسی تعیین کرده بود.
\par 2 و در روز دوم ماه دوم از سال چهارم سلطنت خود به بنا کردن شروع نمود.
\par 3 و این است اساس هایی که سلیمان برای بنا نمودن خانه خدا نهاد: طولش به ذراعها برحسب پیمایش اول شصت ذراع و عرضش بیست ذراع،
\par 4 و طول رواقی که پیش خانه بود مطابق عرض خانه بیست ذراع، و بلندیش صد و بیست ذراع و اندرونش رابه طلای خالص پوشانید.
\par 5 و خانه بزرگ را به چوب صنوبر پوشانید و آن را به زر خالص پوشانید، و بر آن درختان خرما و رشته‌ها نقش نمود.
\par 6 و خانه را به سنگهای گرانبها برای زیبایی مرصع ساخت، و طلای آن طلای فروایم بود.
\par 7 وتیرها و آستانه‌ها و دیوارها و درهای خانه را به طلا پوشانید و بر دیوارها کروبیان نقش نمود.
\par 8 و خانه قدس‌الاقداس را ساخت که طولش موافق عرض خانه بیست ذراع، و عرضش بیست ذراع بود، و آن را به زر خالص به مقدار ششصدوزنه پوشانید.
\par 9 و وزن میخها پنجاه مثقال طلابود، و بالاخانه‌ها را به طلا پوشانید.
\par 10 و در خانه قدس‌الاقداس دو کروبی مجسمه کاری ساخت و آنها را به طلا پوشانید.
\par 11 و طول بالهای کروبیان بیست ذراع بود که بال یکی پنج ذراع بوده، به دیوار خانه می‌رسید و بال دیگرش پنج ذراع بوده، به بال کروبی دیگربرمی خورد.
\par 12 و بال کروبی دیگر پنج ذراع بوده، به دیوار خانه می‌رسید وبال دیگرش پنج ذراع بوده، به بال کروبی دیگر ملصق می‌شد.
\par 13 وبالهای این کروبیان به قدر بیست ذراع پهن می‌بودو آنها بر پایهای خود ایستاده بودند، و رویهای آنها به سوی اندرون خانه می‌بود.
\par 14 و حجاب رااز آسمانجونی و ارغوان و قرمز و کتان نازک ساخت، و کروبیان بر آن نقش نمود.
\par 15 و پیش خانه دو ستون ساخت که طول آنهاسی و پنج ذراع بود و تاجی که بر سر هر یکی ازآنها بود پنج ذراع بود.
\par 16 و رشته‌ها مثل آنهایی که در محراب بود ساخته، آنها را بر سر ستونها نهاد وصد انار ساخته، بر رشته‌ها گذاشت.و ستونهارا پیش هیکل یکی به‌دست راست، و دیگری به طرف چپ برپا نمود، و آن را که به طرف راست بود یاکین و آن را که به طرف چپ بود بوعز نام نهاد.
\par 17 و ستونهارا پیش هیکل یکی به‌دست راست، و دیگری به طرف چپ برپا نمود، و آن را که به طرف راست بود یاکین و آن را که به طرف چپ بود بوعز نام نهاد.
 
\chapter{4}

\par 1 و مذبح برنجینی ساخت که طولش بیست ذراع، و عرضش بیست ذراع، و بلندیش ده ذراع بود.
\par 2 و دریاچه ریخته شده را ساخت که ازلب تا لبش ده ذراع بود، و از هر طرف مدور بود، وبلندیش پنج ذراع، و ریسمانی سی ذراعی آن راگرداگرد احاطه می‌داشت.
\par 3 و زیر آن از هر طرف صورت گاوان بود که آن را گرداگرد احاطه می‌داشتند، یعنی برای هر ذراع ده از آنها دریاچه را از هرجانب احاطه می‌داشتند، و آن گاوان دردو صف بودند و در حین ریخته شدن آن ریخته شدند.
\par 4 و آن بر دوازده گاو قایم بود که روی سه از آنها به سوی شمال و روی سه به سوی مغرب وروی سه به سوی جنوب و روی سه به سوی مشرق بود، و دریاچه بر فوق آنها و همه موخرهای آنها به طرف اندرون بود.
\par 5 و حجم آن یک وجب بود و لبش مثل لب کاسه مانند گل سوسن ساخته شده بود که گنجایش سه هزار بت به پیمایش داشت.
\par 6 و ده حوض ساخت و از آنهاپنج را به طرف راست و پنج را به طرف چپ گذاشت تا در آنها شست و شو نمایند، و آنچه راکه به قربانی های سوختنی تعلق داشت در آنهامی شستند، اما دریاچه برای شست و شوی کاهنان بود.
\par 7 و ده شمعدان طلا موافق قانون آنها ساخته، پنج را به طرف راست و پنج را به طرف چپ درهیکل گذاشت.
\par 8 و ده میز ساخته، پنج را به طرف راست و پنج را به طرف چپ در هیکل گذاشت، وصد کاسه طلا ساخت.
\par 9 و صحن کاهنان و صحن بزرگ و دروازه های صحن (بزرگ را) ساخت، ودرهای آنها را به برنج پوشانید.
\par 10 و دریاچه را به‌جانب راست خانه به سوی مشرق از طرف جنوب گذاشت.
\par 11 و حورام دیگها و خاکندازها و کاسه‌ها راساخت پس حورام تمام کاری که برای سلیمان پادشاه به جهت خانه خدا می‌کرد به انجام رسانید.
\par 12 دو ستون و پیاله های تاجهایی که بر سر دوستون بود و دو شبکه به جهت پوشانیدن دو پیاله تاجهایی که بر ستونها بود
\par 13 و چهارصد انار برای دو شبکه و دو صف انار برای هر شبکه بود تا دوپیاله تاجهایی را که بالای ستونها بود بپوشاند.
\par 14 و پایه‌ها را ساخت و حوضها را بر پایه هاساخت.
\par 15 و یک دریاچه و دوازده گاو را زیردریاچه (ساخت ).
\par 16 و دیگها و خاکندازها وچنگالها و تمامی اسباب آنها را پدرش حورام برای سلیمان پادشاه به جهت خانه خداوند ازبرنج صیقلی ساخت.
\par 17 آنها را پادشاه درصحرای اردن در گل رست که در میان سکوت وصرده بود ریخت.
\par 18 و سلیمان تمام این آلات رااز حد زیاده ساخت، چونکه وزن برنج دریافت نمی شد.
\par 19 و سلیمان تمامی آلات را که در خانه خدابود و مذبح طلا و میزها را که نان تقدمه بر آنها بودساخت.
\par 20 و شمعدانها و چراغهای آنها را ازطلای خالص تا برحسب معمول در مقابل محراب افروخته شود.
\par 21 و گلها و چراغها وانبرها را از طلا یعنی از زر خالص ساخت.و گلگیرها و کاسه‌ها و قاشقها و مجمرها را از طلای خالص (ساخت )، و دروازه خانه و درهای اندرونی آن به جهت قدس‌الاقداس و درهای خانه هیکل از طلا بود. 
\par 22 و گلگیرها و کاسه‌ها و قاشقها و مجمرها را از طلای خالص (ساخت )، و دروازه خانه و درهای اندرونی آن به جهت قدس‌الاقداس و درهای خانه هیکل از طلا بود.
 
\chapter{5}

\par 1 پس تمامی کاری که سلیمان به جهت خانه خداوند کرد تمام شد، و سلیمان موقوفات پدرش داود را داخل ساخت، و نقره و طلا و سایرآلات آنها را در خزاین خانه خدا گذاشت.
\par 2 آنگاه سلیمان مشایخ اسرائیل و جمیع روسای اسباط و سروران آبای بنی‌اسرائیل را دراورشلیم جمع کرد تا تابوت عهد خداوند را ازشهر داود که صهیون باشد، برآورند.
\par 3 و جمیع مردان اسرائیل در عید ماه هفتم نزد پادشاه جمع شدند.
\par 4 پس جمیع مشایخ اسرائیل آمدند ولاویان تابوت را برداشتند.
\par 5 و تابوت و خیمه اجتماع و همه آلات مقدس را که در خیمه بودبرآوردند، و لاویان کهنه آنها را برداشتند.
\par 6 وسلیمان پادشاه و تمامی جماعت اسرائیل که نزدوی جمع شده بودند پیش تابوت ایستادند، وآنقدر گوسفند و گاو ذبح کردند که به شماره وحساب نمی آمد.
\par 7 و کاهنان تابوت عهد خداوندرا به مکانش در محراب خانه، یعنی درقدس‌الاقداس زیر بالهای کروبیان درآوردند.
\par 8 وکروبیان بالهای خود را بر مکان تابوت پهن می‌کردند و کروبیان تابوت و عصاهایش را از بالامی پوشانیدند.
\par 9 و عصاها اینقدر دراز بود که سرهای عصاها از تابوت پیش محراب دیده می شد، اما از بیرون دیده نمی شد، و تا امروز درآنجا است.
\par 10 و در تابوت چیزی نبود سوای آن دو لوح که موسی در حوریب در آن گذاشت وقتی که خداوند با بنی‌اسرائیل در حین بیرون آمدن ایشان از مصر عهد بست.
\par 11 و واقع شد که چون کاهنان از قدس بیرون آمدند (زیرا همه کاهنانی که حاضر بودند بدون ملاحظه نوبتهای خود خویشتن را تقدیس کردند.
\par 12 و جمیع لاویانی که مغنی بودند یعنی آساف وهیمان و یدوتون و پسران و برادران ایشان به کتان نازک ملبس شده، با سنجها و بربطها و عودها به طرف مشرق مذبح ایستاده بودند، و با ایشان صدو بیست کاهن بودند که کرنا می‌نواختند).
\par 13 پس واقع شد که چون کرنانوازان و مغنیان مثل یک نفربه یک آواز در حمد و تسبیح خداوند به صداآمدند، و چون با کرناها و سنجها و سایر آلات موسیقی به آواز بلند خواندند و خداوند را حمدگفتند که او نیکو است زیرا که رحمت او تاابدالاباد است، آنگاه خانه یعنی خانه خداوند ازابر پر شد.و کاهنان به‌سبب ابر نتوانستند برای خدمت بایستند زیرا که جلال یهوه خانه خدا راپر کرده بود.
\par 14 و کاهنان به‌سبب ابر نتوانستند برای خدمت بایستند زیرا که جلال یهوه خانه خدا راپر کرده بود.
 
\chapter{6}

\par 1 آنگاه سلیمان گفت: «خداوند فرموده است که در تاریکی غلیظ ساکن می‌شوم.
\par 2 اما من خانه‌ای برای سکونت تو و مکانی را که تا به ابدساکن شوی بنا نموده‌ام.»
\par 3 و پادشاه روی خود را برگردانیده، تمامی جماعت اسرائیل را برکت داد، و تمامی جماعت اسرائیل بایستادند.
\par 4 پس گفت: «یهوه خدای اسرائیل متبارک باد که به دهان خود به پدرم داودوعده داده، و به‌دست خود آن را به‌جا آورده، گفت:
\par 5 از روزی که قوم خود را از زمین مصربیرون آوردم شهری از جمیع اسباط اسرائیل برنگزیدم تا خانه‌ای بنا نمایم که اسم من در آن باشد، و کسی را برنگزیدم تا پیشوای قوم من اسرائیل بشود.
\par 6 اما اورشلیم را برگزیدم تا اسم من در آنجا باشد و داود را انتخاب نمودم تاپیشوای قوم من اسرائیل بشود.
\par 7 و در دل پدرم داود بود که خانه‌ای برای اسم یهوه خدای اسرائیل بنا نماید.
\par 8 اما خداوند به پدرم داودگفت: چون در دل تو بود که خانه‌ای برای اسم من بنا نمایی نیکو کردی که این را در دل خود نهادی.
\par 9 لیکن تو خانه را بنا نخواهی نمود، بلکه پسر توکه از صلب تو بیرون آید او خانه را برای اسم من بنا خواهد کرد.
\par 10 پس خداوند کلامی را که گفته بود ثابت گردانید و من به‌جای پدرم داودبرخاسته، و بر وفق آنچه خداوند گفته بود برکرسی اسرائیل نشسته‌ام و خانه را به اسم یهوه خدای اسرائیل بنا نمودم.
\par 11 و تابوت را که عهدخداوند که آن را با بنی‌اسرائیل بسته بود در آن می‌باشد در آنجا گذاشته‌ام.»
\par 12 و او پیش مذبح خداوند به حضور تمامی جماعت اسرائیل ایستاده، دستهای خود رابرافراشت.
\par 13 زیرا که سلیمان منبر برنجینی را که طولش پنج ذراع، و عرضش پنج ذراع، و بلندیش سه ذراع بود ساخته، آن را در میان صحن گذاشت و بر آن ایستاده، به حضور تمامی جماعت اسرائیل زانو زد و دستهای خود را به سوی آسمان برافراشته،
\par 14 گفت: «ای یهوه خدای اسرائیل! خدایی مثل تو نه در آسمان و نه در زمین می‌باشد که با بندگان خود که به حضور تو به تمامی دل خویش سلوک می‌نمایند، عهد ورحمت را نگاه می‌داری.
\par 15 و آن وعده‌ای را که به بنده خود پدرم داود داده‌ای نگاه داشته‌ای زیرا به دهان خود وعده دادی و به‌دست خود آن را وفانمودی چنانکه امروز شده است.
\par 16 پس الان‌ای یهوه خدای اسرائیل با بنده خود پدرم داود آن وعده را نگاه دار که به او داده و گفته‌ای که به حضور من کسی‌که بر کرسی اسرائیل بنشیندبرای تو منقطع نخواهد شد، به شرطی که پسرانت طریقهای خود را نگاه داشته، به شریعت من سلوک نمایند چنانکه تو به حضور من رفتارنمودی.
\par 17 و الان‌ای یهوه خدای اسرائیل کلامی که به بنده خود داود گفته‌ای ثابت بشود.
\par 18 «اما آیا خدا فی الحقیقه در میان آدمیان برزمین ساکن خواهد شد؟ اینک فلک و فلک الافلاک تو را گنجایش ندارد تا چه رسد به این خانه‌ای که بنا کردم.
\par 19 لیکن‌ای یهوه خدای من به دعا وتضرع بنده خود توجه نما و استغاثه و دعایی راکه بنده ات به حضور تو می‌کند اجابت فرما.
\par 20 تاآنکه شب و روز چشمان تو بر این خانه باز شود وبر مکانی که درباره‌اش وعده داده‌ای که اسم خودرا در آنجا قرار خواهی داد تا دعایی را که بنده ات به سوی این مکان بنماید اجابت کنی.
\par 21 و تضرع بنده ات و قوم خود اسرائیل را که به سوی این مکان دعا می‌نمایند استماع نما و از آسمان مکان سکونت خود بشنو و چون شنیدی عفو فرما.
\par 22 «اگر کسی با همسایه خود گناه ورزد و قسم بر او عرضه شود که بخورد و او آمده، پیش مذبح تو در این خانه قسم خورد،
\par 23 آنگاه از آسمان بشنو و عمل نموده، به جهت بندگانت داوری کن و شریران را جزا داده، طریق ایشان را بسر ایشان برسان، و عادلان را عادل شمرده، ایشان را به حسب عدالت ایشان جزا بده.
\par 24 «و هنگامی که قوم تو اسرائیل به‌سبب گناهانی که به تو ورزیده باشند به حضور دشمنان خود مغلوب شوند، اگر به سوی تو بازگشت نموده، به اسم تو اعتراف نمایند و نزد تو در این خانه دعا و تضرع کنند،
\par 25 آنگاه از آسمان بشنو وگناه قوم خود اسرائیل را بیامرز و ایشان را به زمینی که به ایشان و به پدران ایشان داده‌ای بازآور.
\par 26 «هنگامی که آسمان بسته شود و به‌سبب گناهانی که به تو ورزیده باشند باران نبارد، اگر به سوی این مکان دعا کنند و به اسم تو اعتراف نمایند و به‌سبب مصیبتی که به ایشان رسانیده باشی از گناه خویش بازگشت کنند،
\par 27 آنگاه ازآسمان بشنو و گناه بندگانت و قوم خود اسرائیل را بیامرز و راه نیکو را که در آن باید رفت به ایشان تعلیم بده، و به زمین خود که آن را به قوم خویش برای میراث بخشیده‌ای باران بفرست.
\par 28 «اگر در زمین قحطی باشد و اگر وبا یا بادسموم یا یرقان باشد یا اگر ملخ یا کرم باشد و اگردشمنان ایشان، ایشان را در شهرهای زمین ایشان محاصره نمایند هر بلایی یا هر مرضی که بوده باشد.
\par 29 آنگاه هر دعا و هر استغاثه‌ای که از هرمرد یا از تمامی قوم تو اسرائیل کرده شود که هریک از ایشان بلا و غم دل خود را خواهد دانست، و دستهای خود را به سوی این خانه دراز خواهد کرد.
\par 30 آنگاه از آسمان که مکان سکونت تو باشدبشنو و بیامرز و به هر کس که دل او را می‌دانی به حسب راههایش جزا بده، زیرا که تو به تنهایی عارف قلوب جمیع بنی آدم هستی.
\par 31 تا آن که ایشان در تمامی روزهایی که بروی زمینی که به پدران ما داده‌ای زنده باشند از تو بترسند.
\par 32 «و نیز غریبی که از قوم تو اسرائیل نباشد وبه‌خاطر اسم عظیم تو و دست قوی و بازوی برافراشته تو از زمین بعید آمده باشد، پس چون بیاید و به سوی این خانه دعا نماید،
\par 33 آنگاه ازآسمان، مکان سکونت خود، بشنو و موافق هرآنچه آن غریب نزد تو استغاثه نماید به عمل آورتا جمیع قومهای جهان اسم تو را بشناسند و مثل قوم تو اسرائیل از تو بترسند و بدانند که اسم تو براین خانه‌ای که بنا کرده‌ام نهاده شده است.
\par 34 «اگر قوم تو برای مقاتله با دشمنان خود به راهی که ایشان را فرستاده باشی، بیرون روند و به سوی شهری که برگزیده‌ای و خانه‌ای که به جهت اسم تو بنا کرده‌ام، نزد تو دعا نمایند،
\par 35 آنگاه دعاو تضرع ایشان را از آسمان بشنو و حق ایشان را به‌جا آور.
\par 36 «و اگر به تو گناه ورزیده باشند زیرا انسانی نیست که گناه نکند، و بر ایشان غضبناک شده، ایشان را به‌دست دشمنان تسلیم کرده باشی واسیرکنندگان ایشان، ایشان را به زمین دور یانزدیک ببرند،
\par 37 پس اگر در زمینی که در آن اسیرباشند به خود آمده، بازگشت نمایند و در زمین اسیری خود نزد تو تضرع نموده، گویند که گناه کرده و عصیان ورزیده، و شریرانه رفتارنموده‌ایم،
\par 38 و در زمین اسیری خویش که ایشان را به آن به اسیری برده باشند، به تمامی دل وتمامی جان خود به تو بازگشت نمایند، و به سوی زمینی که به پدران ایشان داده‌ای و شهری که برگزیده‌ای و خانه‌ای که برای اسم تو بنا کرده‌ام دعا نمایند،
\par 39 آنگاه از آسمان، مکان سکونت خود، دعا و تضرع ایشان را بشنو و حق ایشان رابجا آور، و قوم خود را که به تو گناه ورزیده باشندبیامرز.
\par 40 پس الان‌ای خدای من چشمان تو بازشود و گوشهای تو به دعاهایی که در این مکان کرده شود شنوا باشد.
\par 41 و حال تو‌ای یهوه خدا، با تابوت قوت خود به سوی آرامگاه خویش برخیز. ای یهوه خدا کاهنان تو به نجات ملبس گردند و مقدسانت به نیکویی شادمان بشوند‌ای یهوه خدا روی مسیح خود را برنگردان ورحمتهای بنده خود داود را بیاد آور.»
\par 42 ‌ای یهوه خدا روی مسیح خود را برنگردان ورحمتهای بنده خود داود را بیاد آور.»
 
\chapter{7}

\par 1 و چون سلیمان از دعا کردن فارغ شد، آتش از آسمان فرود آمده، قربانی های سوختنی و ذبایح را سوزانید و جلال خداوند خانه را مملوساخت.
\par 2 و کاهنان به خانه خداوند نتوانستندداخل شوند، زیرا جلال یهوه خانه خداوند را پرکرده بود.
\par 3 و چون تمامی بنی‌اسرائیل آتش را که فرود می‌آمد و جلال خداوند را که بر خانه می‌بوددیدند، روی خود را به زمین بر سنگفرش نهادند وسجده نموده، خداوند را حمد گفتند که او نیکواست، زیرا که رحمت او تا ابدالاباد است.
\par 4 و پادشاه و تمامی قوم قربانی‌ها در حضورخداوند گذرانیدند.
\par 5 و سلیمان پادشاه بیست ودو هزار گاو و صد و بیست هزار گوسفند برای قربانی گذرانید و پادشاه و تمامی قوم، خانه خدا را تبریک نمودند.
\par 6 و کاهنان بر سر شغلهای مخصوص خود ایستاده بودند و لاویان، آلات نغمه خداوند را (به‌دست گرفتند) که داود پادشاه آنها را ساخته بود، تا خداوند را به آنها حمدگویند، زیرا که رحمت او تا ابدالاباد است، و داودبه وساطت آنها تسبیح می‌خواند و کاهنان پیش ایشان کرنا می‌نواختند و تمام اسرائیل ایستاده بودند.
\par 7 و سلیمان وسط صحنی را که پیش خانه خداوند بود، تقدیس نمود زیرا که در آنجاقربانی های سوختنی و پیه ذبایح سلامتی رامی گذرانید، چونکه مذبح برنجینی که سلیمان ساخته بود، قربانی های سوختنی وهدایای آردی و پیه ذبایح را گنجایش نداشت.
\par 8 و در آنوقت سلیمان و تمامی اسرائیل با وی هفت روز را عید نگاه داشتند و آن انجمن بسیاربزرگ از مدخل حمات تا نهر مصر بود.
\par 9 و درروز هشتم محفلی مقدس برپا داشتند، زیرا که برای تبریک مذبح هفت روز و برای عید هفت روز نگاه داشتند.
\par 10 و در روز بیست و سوم ماه هفتم قوم را به خیمه های ایشان مرخص فرمود وایشان به‌سبب احسانی که خداوند به داود وسلیمان و قوم خود اسرائیل کرده بود، شادمان وخوشدل بودند. 
\par 11 پس سلیمان خانه خداوند و خانه پادشاه راتمام کرد و هرآنچه سلیمان قصد نموده بود که درخانه خداوند و در خانه خود بسازد، آن را نیکو به انجام رسانید.
\par 12 و خداوند بر سلیمان در شب ظاهر شده، اورا گفت: «دعای تو را اجابت نمودم و این مکان رابرای خود برگزیدم تا خانه قربانی‌ها شود.
\par 13 اگرآسمان را ببندم تا باران نبارد و اگر امر کنم که ملخ، حاصل زمین را بخورد و اگر وبا در میان قوم خودبفرستم،
\par 14 و قوم من که به اسم من نامیده شده اندمتواضع شوند، و دعا کرده، طالب حضور من باشند، و از راههای بد خویش بازگشت نمایند، آنگاه من از آسمان اجابت خواهم فرمود، وگناهان ایشان را خواهم آمرزید و زمین ایشان راشفا خواهم داد.
\par 15 و از این به بعد چشمان من گشاده، و گوشهای من به دعایی که در این مکان کرده شود شنوا خواهد بود.
\par 16 و حال این خانه رااختیار کرده، و تقدیس نموده‌ام که اسم من تا به ابددر آن قرار گیرد و چشم و دل من همیشه بر آن باشد.
\par 17 و اگر تو به حضور من سلوک نمایی، چنانکه پدرت داود سلوک نمود و برحسب هرآنچه تو را امر فرمایم عمل نمایی و فرایض واحکام مرا نگاه داری،
\par 18 آنگاه کرسی سلطنت تورا استوار خواهم ساخت چنانکه با پدرت داودعهد بسته، گفتم کسی‌که بر اسرائیل سلطنت نماید از تو منقطع نخواهد شد.
\par 19 «لیکن اگر شما برگردید و فرایض و احکام مرا که پیش روی شما نهاده‌ام ترک نمایید و رفته، خدایان غیر را عبادت کنید، و آنها را سجده نمایید،
\par 20 آنگاه ایشان را از زمینی که به ایشان داده‌ام خواهم کند و این خانه را که برای اسم خودتقدیس نموده‌ام، از حضور خود خواهم افکند وآن را در میان جمیع قوم‌ها ضرب‌المثل و مسخره خواهم ساخت.
\par 21 و این خانه که اینقدر رفیع است هرکه از آن بگذرد متحیر شده، خواهدگفت: برای چه خداوند به این زمین و به این خانه چنین عمل نموده است؟و جواب خواهندداد: چونکه یهوه خدای پدران خود را که ایشان رااز زمین مصر بیرون آورد ترک کردند و به خدایان غیر متمسک شده، آنها را سجده و عبادت نمودند از این جهت تمامی این بلا را بر ایشان وارد آورده است.»
\par 22 و جواب خواهندداد: چونکه یهوه خدای پدران خود را که ایشان رااز زمین مصر بیرون آورد ترک کردند و به خدایان غیر متمسک شده، آنها را سجده و عبادت نمودند از این جهت تمامی این بلا را بر ایشان وارد آورده است.»
 
\chapter{8}

\par 1 و بعد از انقضای بیست سالی که سلیمان خانه خداوند و خانه خود را بنا می‌کرد،
\par 2 سلیمان شهرهایی را که حورام به سلیمان داده بود تعمیر نمود، و بنی‌اسرائیل را در آنها ساکن گردانید.
\par 3 و سلیمان به حمات صوبه رفته، آن راتسخیر نمود.
\par 4 و تدمور را در بیابان و همه شهرهای خزینه را که در حمات بنا کرده بود به اتمام رسانید.
\par 5 و بیت حورون بالا و بیت حورون پایین را بنا نمود که شهرهای حصاردار با دیوارهاو دروازه‌ها و پشت بندها بود.
\par 6 و بعله و همه شهرهای خزانه را که سلیمان داشت، و جمیع شهرهای ارابه‌ها و شهرهای سواران را و هرآنچه را که سلیمان می‌خواست در اورشلیم و لبنان وتمامی زمین مملکت خویش بنا نماید (بنا نمود).
\par 7 و تمامی کسانی که از حتیان و اموریان و فرزیان و حویان و یبوسیان باقی‌مانده، و از بنی‌اسرائیل نبودند،
\par 8 یعنی از پسران ایشان که در زمین بعد ازایشان باقی‌مانده بودند، و بنی‌اسرائیل ایشان را هلاک نکرده بودند، سلیمان از ایشان تا امروزسخره گرفت.
\par 9 اما از بنی‌اسرائیل سلیمان احدی را برای کار خود به غلامی نگرفت بلکه ایشان مردان جنگی و سرداران ابطال و سرداران ارابه هاو سواران او بودند.
\par 10 و سروران مقدم سلیمان پادشاه که برقوم حکمرانی می‌کردند دویست وپنجاه نفر بودند.
\par 11 و سلیمان دختر فرعون را از شهر داود به خانه‌ای که برایش بنا کرده بود آورد، زیرا گفت: «زن من در خانه داود پادشاه اسرائیل ساکن نخواهد شد، چونکه همه جایهایی که تابوت خداوند داخل آنها شده است مقدس می‌باشد.»
\par 12 آنگاه سلیمان قربانی های سوختنی برمذبح خداوند که آن را پیش رواق بنا کرده بودبرای خداوند گذرانید.
\par 13 یعنی قربانی های سوختنی قسمت هر روز در روزش برحسب فرمان موسی در روزهای سبت، و غره‌ها و سه مرتبه در هر سال در مواسم یعنی در عید فطیر وعید هفته‌ها و عید خیمه‌ها.
\par 14 و فرقه های کاهنان را برحسب امر پدر خود داود بر سر خدمت ایشان معین کرد و لاویان را بر سر شغلهای ایشان تا تسبیح بخوانند و به حضور کاهنان لوازم خدمت هر روز را در روزش بجا آورند و دربانان را برحسب فرقه های ایشان بر هر دروازه (قرارداد)، زیرا که داود مرد خدا چنین امر فرموده بود.
\par 15 و ایشان از حکمی که پادشاه درباره هر امری ودرباره خزانه‌ها به کاهنان و لاویان داده بود تجاوزننمودند.
\par 16 پس تمامی کار سلیمان از روزی که بنیادخانه خداوند نهاده شد تا روزی که تمام گشت، نیکو آراسته شد، و خانه خداوند به اتمام رسید.
\par 17 آنگاه سلیمان به عصیون جابر و به ایلوت که بر کنار دریا در زمین ادوم است، رفت.وحورام کشتیها و نوکرانی را که در دریا مهارت داشتند به‌دست خادمان خود برای وی فرستاد وایشان با بندگان سلیمان به اوفیر رفتند، وچهارصد و پنجاه وزنه طلا از آنجا گرفته، برای سلیمان پادشاه آوردند.
\par 18 وحورام کشتیها و نوکرانی را که در دریا مهارت داشتند به‌دست خادمان خود برای وی فرستاد وایشان با بندگان سلیمان به اوفیر رفتند، وچهارصد و پنجاه وزنه طلا از آنجا گرفته، برای سلیمان پادشاه آوردند.
 
\chapter{9}

\par 1 و چون ملکه سبا آوازه سلیمان را شنید باموکب بسیار عظیم و شترانی که به عطریات و طلای بسیار و سنگهای گرانبها بارشده بود به اورشلیم آمد، تا سلیمان را به مسائل امتحان کند. و چون نزد سلیمان رسید با وی ازهرچه در دلش بود گفتگو کرد.
\par 2 و سلیمان تمامی مسائلش را برای وی بیان نمود و چیزی ازسلیمان مخفی نماند که برایش بیان نکرد.
\par 3 وچون ملکه سبا حکمت سلیمان و خانه‌ای را که بناکرده بود،
\par 4 و طعام سفره او و مجلس بندگانش ونظم و لباس خادمانش را و ساقیانش و لباس ایشان و زینه‌ای را که به آن به خانه خداوندبرمی آمد دید، روح دیگر در او نماند.
\par 5 پس به پادشاه گفت: «آوازه‌ای را که درولایت خود درباره کارها و حکمت تو شنیدم راست بود.
\par 6 اما تا نیامدم و به چشمان خود ندیدم اخبار آنها را باور نکردم، و همانا نصف عظمت حکمت تو به من اعلام نشده بود، و از خبری که شنیده بودم افزوده‌ای.
\par 7 خوشابه‌حال مردان تو وخوشابه‌حال این خادمانت که به حضور تو همیشه می ایستند و حکمت تو را می‌شنوند.
\par 8 متبارک بادیهوه خدای تو که بر تو رغبت داشته، تو را برکرسی خود نشانید تا برای یهوه خدایت پادشاه بشوی. چونکه خدای تو اسرائیل را دوست می‌دارد تا ایشان را تا به ابد استوار نماید، از این جهت تو را بر پادشاهی‌ایشان نصب نموده است تا داوری و عدالت را بجا آوری.»
\par 9 و به پادشاه صد و بیست وزنه طلا و عطریات از حد زیاده، وسنگهای گرانبها داد و مثل این عطریات که ملکه سبا به سلیمان پادشاه داد هرگز دیده نشد.
\par 10 و نیز بندگان حورام و بندگان سلیمان که طلا از اوفیر می‌آوردند چوب صندل و سنگهای گرانبها آوردند.
\par 11 و پادشاه از این چوب صندل زینه‌ها به جهت خانه خداوند و خانه پادشاه وعودها و بربطها برای مغنیان ساخت، و مثل آنهاقبل از آن در زمین یهودا دیده نشده بود.
\par 12 و سلیمان پادشاه به ملکه سبا تمامی آرزوی او را که خواسته بود داد، سوای آنچه که او برای پادشاه آورده بود، پس با بندگانش به ولایت خود توجه نموده، برفت.
\par 13 و وزن طلایی که در یک سال به سلیمان رسید ششصد و شصت و شش وزنه طلا بود.
\par 14 سوای آنچه تاجران و بازرگانان آوردند وجمیع پادشاهان عرب و حاکمان کشورها طلا ونقره برای سلیمان می‌آوردند.
\par 15 و سلیمان پادشاه دویست سپر طلای چکشی ساخت که برای هر سپر ششصد مثقال طلا بکار برده شد.
\par 16 و سیصد سپر کوچک طلای چکشی ساخت که برای هر سپر سیصد مثقال طلا بکار برده شد، وپادشاه آنها را در خانه جنگل لبنان گذاشت.
\par 17 وپادشاه تخت بزرگی از عاج ساخت و آن را به زرخالص پوشانید.
\par 18 و تخت را شش پله وپایندازی زرین بود که به تخت پیوسته بود و به این طرف و آن طرف نزد جای کرسیش دستها بود، ودو شیر به پهلوی دستها ایستاده بودند.
\par 19 ودوازده شیر از این طرف و آن طرف، بر آن شش پله ایستاده بودند که در هیچ مملکت مثل این ساخته نشده بود.
\par 20 و تمامی ظروف نوشیدنی سلیمان پادشاه از طلا و تمامی ظروف خانه جنگل لبنان از زر خالص بود، و نقره در ایام سلیمان هیچ به حساب نمی آمد
\par 21 زیرا که پادشاه را کشتیها بود که با بندگان حورام به ترشیش می‌رفت، و کشتیهای ترشیشی هر سه سال یک مرتبه می‌آمد، و طلا و نقره و عاج و میمونها وطاوسها می‌آورد.
\par 22 پس سلیمان پادشاه در دولت و حکمت ازجمیع پادشاهان کشورها بزرگتر شد.
\par 23 و تمامی پادشاهان کشورها حضور سلیمان را می‌طلبیدندتا حکمتی را که خدا در دلش نهاده بود بشنوند.
\par 24 و هریکی از ایشان هدیه خود را از آلات نقره وآلات طلا و رخوت و اسلحه و عطریات و اسبهاو قاطرها یعنی قسمت هر سال را در سالش می‌آوردند.
\par 25 و سلیمان چهار هزار آخور به جهت اسبان و ارابه‌ها و دوازده هزار سوار داشت. و آنها را در شهرهای ارابه‌ها و نزد پادشاه دراورشلیم گذاشت.
\par 26 و بر جمیع پادشاهان از نهر(فرات ) تا زمین فلسطینیان و سرحد مصرحکمرانی می‌کرد.
\par 27 و پادشاه نقره را در اورشلیم مثل سنگها و چوب سرو آزاد را مثل چوب افراغ که در صحراست فراوان ساخت.
\par 28 و اسبها برای سلیمان از مصر و از جمیع ممالک می‌آوردند.
\par 29 و اما بقیه وقایع سلیمان از اول تا آخر آیاآنها در تواریخ ناتان نبی و در نبوت اخیای شیلونی و در رویای یعدوی رایی درباره یربعام بن نباط مکتوب نیست؟
\par 30 پس سلیمان چهل سال در اورشلیم بر تمامی اسرائیل سلطنت کرد.و سلیمان با پدران خود خوابید و او را در شهرپدرش داود دفن کردند و پسرش رحبعام در جای او پادشاه شد.
\par 31 و سلیمان با پدران خود خوابید و او را در شهرپدرش داود دفن کردند و پسرش رحبعام در جای او پادشاه شد.
 
\chapter{10}

\par 1 و رحبعام به شکیم رفت زیرا که تمامی اسرائیل به شکیم آمدند تا او را پادشاه سازند.
\par 2 و چون یربعام بن نباط این را شنید، (و اوهنوز در مصر بود که از حضور سلیمان پادشاه به آنجا فرار کرده بود)، یربعام از مصر مراجعت نمود.
\par 3 و ایشان فرستاده، او را خواندند، آنگاه یربعام و تمامی اسرائیل آمدند و به رحبعام عرض کرده، گفتند:
\par 4 «پدر تو یوغ ما را سخت ساخت اما تو الان بندگی سخت پدر خود را ویوغ سنگین او را که بر ما نهاد سبک ساز و تو راخدمت خواهیم نمود.»
\par 5 او به ایشان گفت: «بعداز سه روز باز نزد من بیایید.» و ایشان رفتند.
\par 6 و رحبعام پادشاه با مشایخی که در حین حیات پدرش سلیمان به حضور وی می‌ایستادندمشورت کرده، گفت: «شما چه صلاح می‌بینید که به این قوم جواب دهم؟»
\par 7 ایشان به او عرض کرده، گفتند: «اگر با این قوم مهربانی نمایی وایشان را راضی کنی و با ایشان سخنان دلاویزگویی، همانا همیشه اوقات بنده تو خواهند بود.»
\par 8 اما او مشورت مشایخ را که به وی دادند ترک کرد و با جوانانی که با او تربیت یافته بودند و به حضورش می‌ایستادند مشورت کرد.
\par 9 و به ایشان گفت: «شما چه صلاح می‌بینید که به این قوم جواب دهیم که به من عرض کرده، گفته‌اند: یوغی را که پدرت بر ما نهاده است سبک ساز.»
\par 10 وجوانانی که با او تربیت یافته بودند او را خطاب کرده، گفتند: «به این قوم که به تو عرض کرده، گفته‌اند پدرت یوغ ما را سنگین ساخته است و توآن را برای ما سبک ساز چنین بگو: انگشت کوچک من از کمر پدرم کلفت تر است.
\par 11 و حال پدرم یوغ سنگینی بر شما نهاده است اما من یوغ شما را زیاده خواهم گردانید، پدرم شما را باتازیانه‌ها تنبیه می‌نمود اما من شما را با عقربها.» 
\par 12 و در روز سوم، یربعام و تمامی قوم به نزدرحبعام بازآمدند، به نحوی که پادشاه گفته وفرموده بود که در روز سوم نزد من بازآیید.
\par 13 وپادشاه قوم را به سختی جواب داد، و رحبعام پادشاه مشورت مشایخ را ترک کرد.
\par 14 و موافق مشورت جوانان ایشان را خطاب کرده، گفت: «پدرم یوغ شما را سنگین ساخت، اما من آن رازیاده خواهم گردانید، پدرم شما را با تازیانه هاتنبیه می‌نمود اما من با عقربها.»
\par 15 پس پادشاه قوم را اجابت نکرد زیرا که این امر از جانب خدا شده بود تا خداوند کلامی را که به واسطه اخیای شیلونی به یربعام بن نباط گفته بود ثابت گرداند.
\par 16 و چون تمامی اسرائیل دیدند که پادشاه ایشان را اجابت نکرد آنگاه قوم، پادشاه را جواب داده، گفتند: «ما را در داود چه حصه است؟ درپسر یسی نصیبی نداریم. ای اسرائیل! به خیمه های خود بروید. حال‌ای داود به خانه خودمتوجه باش!» پس تمامی اسرائیل به خیمه های خویش رفتند.
\par 17 اما بنی‌اسرائیلی که درشهرهای یهودا ساکن بودند رحبعام بر ایشان سلطنت می‌نمود.
\par 18 پس رحبعام پادشاه هدرام راکه رئیس باجگیران بود فرستاد، و بنی‌اسرائیل اورا سنگسار کردند که مرد و رحبعام پادشاه تعجیل نموده، بر ارابه خود سوار شد و به اورشلیم فرارکرد.پس اسرائیل تا به امروز بر خاندان داودعاصی شده‌اند.
\par 19 پس اسرائیل تا به امروز بر خاندان داودعاصی شده‌اند.
 
\chapter{11}

\par 1 و چون رحبعام وارد اورشلیم شد، صدو هشتاد هزار نفر برگزیده جنگ آزموده را از خاندان یهودا و بنیامین جمع کرد تا بااسرائیل مقاتله نموده، سلطنت را به رحبعام برگرداند.
\par 2 اما کلام خداوند بر شمعیا مرد خدانازل شده، گفت:
\par 3 «به رحبعام بن سلیمان پادشاه یهودا و به تمامی اسرائیلیان که در یهودا و بنیامین می‌باشند خطاب کرده، بگو:
\par 4 خداوند چنین می‌گوید: برمیایید و با برادران خود جنگ منمایید. هرکس به خانه خود برگردد زیرا که این امر از جانب من شده است.» و ایشان کلام خداوندرا شنیدند و از رفتن به ضد یربعام برگشتند.
\par 5 و رحبعام در اورشلیم ساکن شد و شهرهای حصاردار در یهودا ساخت.
\par 6 پس بیت لحم و عیتام و تقوع
\par 7 و بیت صور و سوکو و عدلام،
\par 8 وجت و مریشه و زیف،
\par 9 و ادورایم و لاکیش وعزیقه،
\par 10 و صرعه و ایلون و حبرون را بنا کرد که شهرهای حصاردار در یهودا و بنیامین می‌باشند.
\par 11 و حصارها را محکم ساخت و در آنها سرداران و انبارهای ماکولات و روغن و شراب گذاشت.
\par 12 و در هر شهری سپرها و نیزه‌ها گذاشته، آنها رابسیار محکم گردانید، پس یهودا و بنیامین با اوماندند.
\par 13 و کاهنان و لاویانی که در تمامی اسرائیل بودند از همه حدود خود نزد او جمع شدند.
\par 14 زیراکه لاویان اراضی شهرها و املاک خود راترک کرده، به یهودا و اورشلیم آمدند چونکه یربعام و پسرانش ایشان را از کهانت یهوه اخراج کرده بودند.
\par 15 و او برای خود به جهت مکان های بلند و دیوها و گوساله هایی که ساخته بود کاهنان معین کرد.
\par 16 و بعد از ایشان آنانی که دلهای خودرا به طلب یهوه خدای اسرائیل مشغول ساخته بودند از تمامی اسباط اسرائیل به اورشلیم آمدندتا برای یهوه خدای پدران خود قربانی بگذرانند.
\par 17 پس سلطنت یهودا را مستحکم ساختند ورحبعام بن سلیمان را سه سال تقویت کردند، زیراکه سه سال به طریق داود و سلیمان سلوک نمودند.
\par 18 و رحبعام محله دختر یریموت بن داود وابیحایل دختر الیاب بن یسی را به زنی گرفت.
\par 19 واو برای وی پسران یعنی یعوش و شمریا و زهم رازایید.
\par 20 و بعد از او معکه دختر ابشالوم را گرفت و او برای وی ابیا و عتای و زبزا و شلومیت رازایید.
\par 21 و رحبعام، معکه دختر ابشالوم را از جمیع زنان و متعه های خود زیاده دوست می‌داشت، زیرا که هجده زن و شصت متعه گرفته بود و بیست و هشت پسر و شصت دختر تولیدنمود.
\par 22 و رحبعام ابیا پسر معکه را در میان برادرانش سرور و رئیس ساخت، زیراکه می‌خواست او را به پادشاهی نصب نماید.وعاقلانه رفتار نموده، همه پسران خود را درتمامی بلاد یهودا و بنیامین در جمیع شهرهای حصاردار متفرق ساخت، و برای ایشان آذوقه بسیار قرار داد و زنان بسیار خواست.
\par 23 وعاقلانه رفتار نموده، همه پسران خود را درتمامی بلاد یهودا و بنیامین در جمیع شهرهای حصاردار متفرق ساخت، و برای ایشان آذوقه بسیار قرار داد و زنان بسیار خواست.
 
\chapter{12}

\par 1 و چون سلطنت رحبعام استوار گردیدو خودش تقویت یافت، او با تمامی اسرائیل شریعت خداوند را ترک نمودند.
\par 2 و درسال پنجم سلطنت رحبعام، شیشق پادشاه مصربه اورشلیم برآمد زیراکه ایشان بر خداوند عاصی شده بودند.
\par 3 با هزار و دویست ارابه و شصت هزار سوار و خلقی که از مصریان و لوبیان وسکیان و حبشیان همراهش آمدند بیشمار بودند.
\par 4 پس شهرهای حصاردار یهودا را گرفت و به اورشلیم آمد.
\par 5 و شمعیای نبی نزد رحبعام وسروران یهودا که از ترس شیشق در اورشلیم جمع بودند آمده، به ایشان گفت: «خداوند چنین می‌گوید: شما مرا ترک کردید پس من نیز شما رابه‌دست شیشق ترک خواهم نمود.»
\par 6 آنگاه سروران اسرائیل و پادشاه تواضع نموده، گفتند: «خداوند عادل است.»
\par 7 و چون خداوند دید که ایشان متواضع شده‌اند کلام خداوند بر شمعیانازل شده، گفت: «چونکه ایشان تواضع نموده اند ایشان را هلاک نخواهم کرد بلکه ایشان را اندک زمانی خلاصی خواهم داد و غضب من به‌دست شیشق بر اورشلیم ریخته نخواهد شد.
\par 8 لیکن ایشان بنده او خواهند شد تا بندگی من و بندگی ممالک جهان را بدانند.»
\par 9 پس شیشق پادشاه مصر به اورشلیم برآمده، خزانه های خانه خداوند و خزانه های خانه پادشاه را گرفت و همه‌چیز را برداشت و سپرهای طلا راکه سلیمان ساخته بود برد.
\par 10 و رحبعام پادشاه به عوض آنها سپرهای برنجین ساخت و آنها را به‌دست سرداران شاطرانی که در خانه پادشاه رانگاهبانی می‌کردند سپرد.
\par 11 و هر وقتی که پادشاه به خانه خداوند داخل می‌شد شاطران آمده، آنهارا برمی داشتند و آنها را به حجره شاطران بازمی آوردند.
\par 12 و چون او متواضع شد خشم خداوند از او برگشت تا او را بالکل هلاک نسازد، و در یهودا نیز اعمال نیکو پیدا شد.
\par 13 و رحبعام پادشاه، خویشتن را در اورشلیم قوی ساخته، سلطنت نمود و رحبعام چون پادشاه شد چهل و یک ساله بود، و در شهراورشلیم که خداوند آن را از تمام اسباط اسرائیل برگزید تا اسم خود را در آن بگذارد، هفده سال پادشاهی کرد و اسم مادرش نعمه عمونیه بود.
\par 14 و او شرارت ورزید زیرا که خداوند را به تصمیم قلب طلب ننمود.
\par 15 و اما وقایع اول و آخر رحبعام آیا آنها درتواریخ شمعیای نبی و تواریخ انساب عدوی رایی مکتوب نیست؟ و در میان رحبعام و یربعام پیوسته جنگ می‌بود.پس رحبعام با پدران خود خوابید و در شهر داود دفن شد و پسرش ابیا به‌جایش سلطنت کرد.
\par 16 پس رحبعام با پدران خود خوابید و در شهر داود دفن شد و پسرش ابیا به‌جایش سلطنت کرد.
 
\chapter{13}

\par 1 در سال هجدهم سلطنت یربعام، ابیا بریهودا پادشاه شد.
\par 2 سه سال دراورشلیم پادشاهی کرد و اسم مادرش میکایادختر اوریئیل از جبعه بود.
\par 3 و ابیا بافوجی از شجاعان جنگ آزموده یعنی چهارصدهزار مرد برگزیده تدارک جنگ دید، و یربعام باهشتصد هزار مرد برگزیده که شجاعان قوی بودند با وی جنگ را صف آرایی نمود.
\par 4 و ابیا برکوه صمارایم که در کوهستان افرایم است برپاشده، گفت: «ای یربعام و تمامی اسرائیل مراگوش گیرید!
\par 5 آیا شما نمی دانید که یهوه خدای اسرائیل سلطنت اسرائیل را به داود و پسرانش باعهد نمکین تا به ابد داده است؟
\par 6 و یربعام بن نباطبنده سلیمان بن داود برخاست و بر مولای خودعصیان ورزید.
\par 7 و مردان بیهوده که پسران بلیعال بودند نزد وی جمع شده، خویشتن را به ضدرحبعام بن سلیمان تقویت دادند، هنگامی که رحبعام جوان و رقیق القلب بود و با ایشان مقاومت نمی توانست نمود.
\par 8 و شما الان گمان می‌برید که با سلطنت خداوند که در دست پسران داود است مقابله توانید نمود؟ و شما گروه عظیمی می‌باشید و گوساله های طلا که یربعام برای شما به‌جای خدایان ساخته است با شمامی باشد.
\par 9 آیا شما کهنه خداوند را از بنی هارون و لاویان را نیز اخراج ننمودید و مثل قومهای کشورها برای خود کاهنان نساختید؟ و هرکه بیاید و خویشتن را با گوساله‌ای و هفت قوچ تقدیس نماید، برای آنهایی که خدایان نیستند کاهن می‌شود.
\par 10 و اما ما یهوه خدای ماست و اورا ترک نکرده‌ایم و کاهنان از پسران هارون خداوند را خدمت می‌کنند و لاویان در کار خودمشغولند.
\par 11 و هر صبح و هر شام قربانی های سوختنی و بخور معطر برای خداوند می‌سوزانندو نان تقدمه بر میز طاهر می‌نهند و شمعدان طلا وچراغهایش را هر شب روشن می‌کنند زیرا که ماوصایای یهوه خدای خود را نگاه می‌داریم اماشما او را ترک کرده‌اید.
\par 12 و اینک با ما خدارئیس است و کاهنان او با کرناهای بلند آوازهستند تا به ضد شما بنوازند. پس‌ای بنی‌اسرائیل با یهوه خدای پدران خود جنگ مکنید زیراکامیاب نخواهید شد.»
\par 13 اما یربعام کمین گذاشت که از عقب ایشان بیایند و خود پیش روی یهودا بودند و کمین درعقب ایشان بود.
\par 14 و چون یهودا نگریستند، اینک جنگ هم از پیش و هم از عقب ایشان بود، پس نزد خداوند استغاثه نمودند و کاهنان کرناهارا نواختند.
\par 15 و مردان یهودا بانگ بلند برآوردند، و واقع شد که چون مردان یهودا بانگ برآوردند، خدا یربعام و تمامی اسرائیل را به حضور ابیا ویهودا شکست داد.
\par 16 و بنی‌اسرائیل از حضوریهودا فرار کردند و خدا آنها را به‌دست ایشان تسلیم نمود.
\par 17 و ابیا و قوم او آنها را به صدمه عظیمی شکست دادند، چنانکه پانصد هزار مردبرگزیده از اسرائیل مقتول افتادند.
\par 18 پس بنی‌اسرائیل در آن وقت ذلیل شدند و بنی یهوداچونکه بر یهوه خدای پدران خود توکل نمودند، قوی گردیدند.
\par 19 و ابیا یربعام را تعاقب نموده، شهرهای بیت ئیل را با دهاتش و یشانه را بادهاتش و افرون را با دهاتش از او گرفت.
\par 20 ویربعام در ایام ابیا دیگر قوت بهم نرسانید و خداوند او را زد که مرد.
\par 21 و ابیا قوی می‌شد وچهارده زن برای خود گرفت و بیست و دو پسر وشانزده دختر به وجود آورد.پس بقیه وقایع ابیااز رفتار و اعمال او در مدرس عدوی نبی مکتوب است.
\par 22 پس بقیه وقایع ابیااز رفتار و اعمال او در مدرس عدوی نبی مکتوب است.
 
\chapter{14}

\par 1 و ابیا با پدران خود خوابید و او را درشهر داود دفن کردند و پسرش آسا درجایش پادشاه شد و در ایام او زمین ده سال آرامی یافت.
\par 2 و آسا آنچه را که در نظر یهوه خدایش نیکو و راست بود به‌جا می‌آورد.
\par 3 و مذبح های غریب و مکانهای بلند را برداشت و بتها رابشکست و اشوریم را قطع نمود.
\par 4 و یهودا را امرفرمود که یهوه خدای پدران خود را بطلبند وشریعت و اوامر او را نگاه دارند.
\par 5 و مکانهای بلندو تماثیل شمس را از جمیع شهرهای یهودا دورکرد، پس مملکت به‌سبب او آرامی یافت.
\par 6 وشهرهای حصاردار در یهودا بنا نمود زیرا که زمین آرام بود و در آن سالها کسی با او جنگ نکردچونکه خداوند او را راحت بخشید.
\par 7 و به یهوداگفت: «این شهرها را بنا نماییم و دیوارها و برجهابا دروازه‌ها و پشت بندها به اطراف آنها بسازیم.
\par 8 زیرا چونکه یهوه خدای خود را طلبیده‌ایم زمین پیش روی ما است. او را طلب نمودیم و او ما را ازهر طرف راحت بخشیده است.» پس بنا نمودند وکامیاب شدند.
\par 9 و آسا لشکری از یهودا یعنی سیصد هزارسپردار و نیزه‌دار داشت و از بنیامین دویست وهشتاد هزار سپردار و تیرانداز که جمیع اینهامردان قوی جنگی بودند.
\par 10 پسی زارح حبشی باهزار هزار سپاه و سیصد ارابه به ضد ایشان بیرون آمده، به مریشه رسید.
\par 11 و آسا به مقابله ایشان بیرون رفت پس ایشان در وادی صفاته نزد مریشه جنگ را صف آرایی نمودنند. 
\par 12 و آسا یهوه خدای خود را خوانده، گفت: «ای خداوندنصرت دادن به زورآوران یا به بیچارگان نزد تویکسان است، پس‌ای یهوه خدای ما، ما را اعانت فرما زیرا که بر تو توکل می‌داریم و به اسم تو به مقابله این گروه عظیم آمده‌ایم، ای یهوه تو خدای ما هستی پس مگذار که انسان بر تو غالب آید.»
\par 13 آنگاه خداوند حبشیان را به حضور آسا ویهودا شکست داد و حبشیان فرار کردند.
\par 14 وآسا با خلقی که همراه او بودند آنها را تا جرارتعاقب نمودند و از حبشیان آنقدر‌افتادند که ازایشان کسی زنده نماند، زیرا که به حضور خداوندو به حضور لشکر او شکست یافتند و ایشان غنیمت از حد زیاده بردند.و تمام شهرها را که به اطراف جرار بود تسخیر نمودند زیرا ترس خداوند بر ایشان مستولی شده بود و شهرها راتاراج نمودند، زیرا که غنیمت بسیار در آنها بود. و خیمه های مواشی را نیز زدند و گوسفندان فراوان و شتران را برداشته، به اورشلیم مراجعت کردند.
\par 15 و تمام شهرها را که به اطراف جرار بود تسخیر نمودند زیرا ترس خداوند بر ایشان مستولی شده بود و شهرها راتاراج نمودند، زیرا که غنیمت بسیار در آنها بود. و خیمه های مواشی را نیز زدند و گوسفندان فراوان و شتران را برداشته، به اورشلیم مراجعت کردند.
 
\chapter{15}

\par 1 و روح خدا به عزریا ابن عودید نازل شد.
\par 2 و او برای ملاقات آسا بیرون آمده، وی را گفت: «ای آسا و تمامی یهودا وبنیامین از من بشنوید! خداوند با شما خواهد بودهر گاه شما با او باشید و اگر او را بطلبید او راخواهید یافت، اما اگر او را ترک کنید او شما راترک خواهد نمود.
\par 3 و اسرائیل مدت مدیدی بی‌خدای حق و بی‌کاهن معلم و بی‌شریعت بودند.
\par 4 اما چون در تنگیهای خود به سوی یهوه خدای اسرائیل بازگشت نموده، او را طلبیدند او رایافتند.
\par 5 و در آن زمان به جهت هر‌که خروج ودخول می‌کرد هیچ امنیت نبود بلکه اضطراب سخت بر جمیع سکنه کشورها می‌بود.
\par 6 و قومی از قومی و شهری از شهری هلاک می‌شدند، چونکه خدا آنها را به هر قسم بلا مضطرب می‌ساخت.
\par 7 اما شما قوی باشید و دستهای شماسست نشود زیرا که اجرت اعمال خود راخواهید یافت.»
\par 8 پس چون آسا این سخنان و نبوت (پسر)عودید نبی را شنید، خویشتن را تقویت نموده، رجاسات را از تمامی زمین یهودا و بنیامین و ازشهرهایی که در کوهستان افرایم گرفته بود دورکرد، و مذبح خداوند را که پیش روی رواق خداوند بود تعمیر نمود.
\par 9 و تمامی یهودا وبنیامین و غریبان را که از افرایم و منسی و شمعون در میان ایشان ساکن بودند جمع کرد زیرا گروه عظیمی از اسرائیل چون دیدند که یهوه خدای ایشان با او می‌بود به او پیوستند.
\par 10 پس در ماه سوم از سال پانزدهم سلطنت آسا در اورشلیم جمع شدند.
\par 11 و در آن روز هفتصد گاو و هفت هزار گوسفند از غنیمتی که آورده بودند برای خداوند ذبح نمودند.
\par 12 و به تمامی دل و تمامی جان خود عهد بستند که یهوه خدای پدران خودرا طلب نمایند.
\par 13 و هر کسی‌که یهوه خدای اسرائیل را طلب ننماید، خواه کوچک و خواه بزرگ، خواه مرد و خواه زن، کشته شود.
\par 14 و به صدای بلند و آواز شادمانی و کرناها و بوقها برای خداوند قسم خوردند.
\par 15 و تمامی یهودا به‌سبب این قسم شادمان شدند زیرا که به تمامی دل خودقسم خورده بودند، و چونکه او را به رضامندی تمام طلبیدند وی را یافتند و خداوند ایشان را ازهر طرف امنیت داد.
\par 16 و نیز آسا پادشاه مادر خود معکه را از ملکه بودن معزول کرد زیرا که او تمثالی به جهت اشیره ساخته بود و آسا تمثال او را قطع نمود و آن راخرد کرده، در وادی قدرون سوزانید.
\par 17 امامکانهای بلند از میان اسرائیل برداشته نشد. لیکن دل آسا در تمامی ایامش کامل می‌بود.
\par 18 وچیزهایی را که پدرش وقف کرده، و آنچه را که خودش وقف نموده بود از نقره و طلا و ظروف به خانه خداوند درآوردهو تا سال سی و پنجم سلطنت آسا جنگ نبود.
\par 19 و تا سال سی و پنجم سلطنت آسا جنگ نبود.
 
\chapter{16}

\par 1 اما در سال سی و ششم سلطنت آسا، بعشا پادشاه اسرائیل بر یهودا برآمد، ورامه را بنا کرد تا نگذارد که کسی نزد آساپادشاه یهودا رفت و آمد نماید.
\par 2 آنگاه آسا نقره وطلا را از خزانه های خداوند و خانه پادشاه گرفته، آن را نزد بنهدد پادشاه ارام که دردمشق ساکن بود فرستاده، گفت:
\par 3 «در میان من وتو و در میان پدر من و پدر تو عهد بوده است. اینک نقره و طلا نزد تو فرستادم پس عهدی را که با بعشا پادشاه اسرائیل داری بشکن تا او از نزد من برود.»
\par 4 و بنهدد آسا پادشاه را اجابت نموده، سرداران افواج خود را بر شهرهای اسرائیل فرستاد و ایشان عیون و دان و آبل مایم و جمیع شهرهای خزانه نفتالی را تسخیر نمودند.
\par 5 وچون بعشا این را شنید بنا نمودن رامه را ترک کرده، از کاری که می‌کرد باز ایستاد.
\par 6 و آساپادشاه، تمامی یهودا را جمع نموده، ایشان سنگهای رامه و چوبهای آن را که بعشا بنا می کرد برداشتند و او جبع و مصفه را با آنها بنانمود.
\par 7 و در آن زمان حنانی رایی نزد آساپادشاه یهودا آمده، وی را گفت: «چونکه تو برپادشاه ارام توکل نمودی و بر یهوه خدای خود توکل ننمودی، از این جهت لشکر پادشاه ارام از دست تو رهایی یافت.
\par 8 آیا حبشیان ولوبیان لشکر بسیار بزرگ نبودند؟ و ارابه‌ها وسواران از حد زیاده نداشتند؟ اما چونکه برخداوند توکل نمودی آنها را به‌دست توتسلیم نمود.
\par 9 زیرا که چشمان خداوند درتمام جهان تردد می‌کند تا قوت خویش را برآنانی که دل ایشان با او کامل است نمایان سازد. تو در اینکار احمقانه رفتار نمودی، لهذا ازاین ببعد در جنگها گرفتار خواهی شد.»
\par 10 اماآسا بر آن رایی غضب نموده، او را در زندان انداخت زیرا که از این امر خشم او بر وی افروخته شد و در همان وقت آسا بر بعضی از قوم ظلم نمود.
\par 11 و اینک وقایع اول و آخر آسا در تورایخ پادشاهان یهودا و اسرائیل مکتوب است.
\par 12 ودر سال سی و نهم سلطنت آسا مرضی درپایهای او عارض شد و مرض او بسیار سخت گردید و نیز در بیماری خود از خداوند مددنخواست بلکه از طبیبان.
\par 13 پس آسا با پدران خود خوابید و در سال چهل و یکم از سلطنت خود وفات یافت.و او را در مقبره‌ای که برای خود در شهر داود کنده بود دفن کردندو او را در دخمه‌ای که از عطریات و انواع حنوطکه به صنعت عطاران ساخته شده بودگذاشتند و برای وی آتشی بی‌نهایت عظیم برافروختند.
\par 14 و او را در مقبره‌ای که برای خود در شهر داود کنده بود دفن کردندو او را در دخمه‌ای که از عطریات و انواع حنوطکه به صنعت عطاران ساخته شده بودگذاشتند و برای وی آتشی بی‌نهایت عظیم برافروختند.
 
\chapter{17}

\par 1 و پسرش یهوشافاط در جای او پادشاه شد و خود را به ضد اسرائیل تقویت داد.
\par 2 و سپاهیان در تمامی شهرهای حصارداریهودا گذاشت و قراولان در زمین یهودا و درشهرهای افرایم که پدرش آسا گرفته بود قرار داد.
\par 3 و خداوند با یهوشافاط می‌بود زیرا که درطریقهای اول پدر خود داود سلوک می‌کرد و ازبعلیم طلب نمی نمود.
\par 4 بلکه خدای پدر خویش را طلبیده، در اوامر وی سلوک می‌نمود و نه موافق اعمال اسرائیل.
\par 5 پس خداوند سلطنت را دردستش استوار ساخت و تمامی یهودا هدایا برای یهوشافاط آوردند و دولت و حشمت عظیمی پیدا کرد.
\par 6 و دلش به طریقهای خداوند رفیع شد، و نیز مکانهای بلند و اشیره‌ها را از یهودا دور کرد.
\par 7 و در سال سوم از سلطنت خود، سروران خویش را یعنی بنحایل و عوبدیا و زکریا و نتنئیل و میکایا را فرستاد تا در شهرهای یهودا تعلیم دهند.
\par 8 و با ایشان بعضی از لاویان یعنی شمعیا ونتنیا و زبدیا و عسائیل و شمیراموت و یهوناتان وادنیا و طوبیا و توب ادنیا را که لاویان بودند، فرستاد و با ایشان الیشمع و یهورام کهنه را.
\par 9 پس ایشان در یهودا تعلیم دادند و سفر تورات خداوند را با خود داشتند، و در همه شهرهای یهودا گردش کرده، قوم را تعلیم می‌دادند.
\par 10 و ترس خداوند بر همه ممالک کشورها که در اطراف یهودا بودند مستولی گردید تا بایهوشافاط جنگ نکردند.
\par 11 و بعضی ازفلسطینیان، هدایا و نقره جزیه را برای یهوشافاطآوردند و عربها نیز از مواشی هفت هزار و هفتصدقوچ و هفت هزار و هفتصد بز نر برای او آوردند.
\par 12 پس یهوشافاط ترقی نموده، بسیار بزرگ شد و قلعه‌ها و شهرهای خزانه در یهودا بنا نمود.
\par 13 ودر شهرهای یهودا کارهای بسیار کرد و مردان جنگ آزموده و شجاعان قوی در اورشلیم داشت.
\par 14 و شماره ایشان برحسب خاندان آبای ایشان این است: یعنی از یهودا سرداران هزاره که رئیس ایشان ادنه بود و با او سیصد هزار شجاع قوی بودند.
\par 15 و بعد از، او یهوحانان رئیس بود وبا او دویست و هشتاد هزار نفر بودند.
\par 16 و بعد ازاو، عمسیا ابن زکری بود که خویشن را برای خداوند نذر کرده بود و با او دویست هزار شجاع قوی بودند.
\par 17 و از بنیامین، الیاداع که شجاع قوی بود و با او دویست هزار نفر مسلح به کمان و سپربودند.
\par 18 و بعد از او یهوزاباد بود و با او صد وهشتاد هزار مرد مهیای جنگ بودند.اینان خدام پادشاه بودند، سوای آنانی که پادشاه درتمامی یهودا در شهرهای حصاردار قرار داده بود.
\par 19 اینان خدام پادشاه بودند، سوای آنانی که پادشاه درتمامی یهودا در شهرهای حصاردار قرار داده بود.
 
\chapter{18}

\par 1 و یهوشافاط دولت و حشمت عظیمی داشت، و با اخاب مصاهرت نمود.
\par 2 وبعد از چند سال نزد اخاب به سامره رفت و اخاب برای او و قومی که همراهش بودند گوسفندان وگاوان بسیار ذبح نمود و او را تحریض نمود که همراه خودش به راموت جلعاد برآید.
\par 3 پس اخاب پادشاه اسرائیل به یهوشافاط پادشاه یهوداگفت: «آیا همراه من به راموت جلعاد خواهی آمد؟» او جواب داد که «من چون تو و قوم من چون قوم تو هستیم و همراه تو به جنگ خواهیم رفت.»
\par 4 و یهوشافاط به پادشاه اسرائیل گفت: «تمناآنکه امروز از کلام خداوند مسالت نمایی.»
\par 5 وپادشاه اسرائیل چهارصد نفر از انبیا جمع کرده، به ایشان گفت: «آیا به راموت جلعاد برای جنگ برویم یا من از آن باز ایستم؟» ایشان جواب دادند: «برآی و خدا آن را به‌دست پادشاه تسلیم خواهدنمود.»
\par 6 اما یهوشافاط گفت: «آیا در اینجا غیر ازاینها نبی‌ای از جانب یهوه نیست تا از او سوال نماییم؟»
\par 7 و پادشاه اسرائیل به یهوشافاط گفت: «یک مرد دیگر هست که به واسطه او از خداوندمسالت توان کرد لیکن من از او نفرت دارم زیرا که درباره من به نیکویی هرگز نبوت نمی کند بلکه همیشه اوقات به بدی، و او میکایا ابن یملامی باشد.» و یهوشافاط گفت: «پادشاه چنین نگوید.»
\par 8 پس پادشاه اسرائیل یکی از خواجه‌سرایان خود را خوانده، گفت: «میکایا ابن یملا را به زودی حاضر کن.»
\par 9 و پادشاه اسرائیل ویهوشافاط پادشاه یهودا هر یکی لباس خود راپوشیده، بر کرسی خویش در جای وسیع نزددهنه دروازه سامره شسته بودند و جمیع انبیا به حضور ایشان نبوت می‌کردند.
\par 10 و صدقیا ابن کنعنه شاخهای آهنین برای خود ساخته، گفت: «یهوه چنین می‌گوید: ارامیان را با اینها خواهی زدتا تلف شوند.»
\par 11 و جمیع انبیا نبوت کرده، می‌گفتند: «به راموت جلعاد برآی و فیروز شوزیرا که خداوند آن را به‌دست پادشاه تسلیم خواهد نمود.»
\par 12 و قاصدی که برای طلبیدن میکایا رفته بوداو را خطاب کرده، گفت: «اینک انبیا به یک زبان درباره پادشاه نیکو می‌گویند پس کلام تو مثل کلام یکی از ایشان باشد و سخن نیکو بگو.»
\par 13 میکایا جواب داد: «به حیات یهوه قسم که هر‌آنچه خدای من مرا گوید همان را خواهم گفت.»
\par 14 پس چون نزد پادشاه رسید، پادشاه وی راگفت: «ای میکایا، آیا به راموت جلعاد برای جنگ برویم یا من از آن بازایستم.» او گفت: «برآیید و فیروز شوید، و به‌دست شما تسلیم خواهند شد.»
\par 15 پادشاه وی را گفت: «من چندمرتبه تو را قسم بدهم که به اسم یهوه غیر از آنچه راست است به من نگویی.»
\par 16 او گفت: «تمامی اسرائیل را مثل گوسفندانی که شبان ندارند برکوهها پراکنده دیدم و خداوند گفت اینها صاحب ندارند پس هر کس به سلامتی به خانه خودبرگردد.» 
\par 17 و پادشاه اسرائیل به یهوشافاط گفت: «آیا تو را نگفتم که درباره من به نیکویی نبوت نمی کند بلکه به بدی.»
\par 18 او گفت: «پس کلام یهوه را بشنوید: من یهوه را بر کرسی خود نشسته دیدم، و تمامی لشکر آسمان را که به طرف راست و چپ وی ایستاده بودند.
\par 19 و خداوند گفت: "کیست که اخاب پادشاه اسرائیل را اغوا نماید تا برود و درراموت جلعاد بیفتد؟ یکی جواب داده به اینطورسخن راند و دیگری به آنطور تکلم نمود.
\par 20 و آن روح (پلید) بیرون آمده، به حضور خداوندبایستاد و گفت: من او را اغوا می‌کنم و خداوندوی را گفت: به چه چیز؟
\par 21 او جواب داد که من بیرون می‌روم و در دهان جمیع انبیایش روح کاذب خواهم بود. او فرمود: وی را اغوا خواهی کرد و خواهی توانست، پس برو و چنین بکن.
\par 22 پس الان هان، یهوه روحی کاذب در دهان این انبیای تو گذاشته است و خداوند درباره تو سخن بد گفته است.»
\par 23 آنگاه صدقیا ابن کنعنه نزدیک آمده، به رخسار میکایا زد و گفت: «به کدام راه روح خداوند از نزد من به سوی تو رفت تا با تو سخن گوید؟»
\par 24 میکایا جواب داد: «اینک در روزی که به حجره اندرونی داخل شده، خود را پنهان کنی آن را خواهی دید.»
\par 25 و پادشاه اسرائیل گفت: «میکایا را بگیرید و او را نزد آمون، حاکم شهر ویوآش، پسر پادشاه ببرید.
\par 26 و بگویید پادشاه چنین می‌فرماید: این شخص را در زندان بیندازیدو او را به نان تنگی و آب تنگی بپرورانید تا من به سلامتی برگردم.»
\par 27 میکایا گفت: «اگر فی الواقع به سلامتی مراجعت کنی، یهوه با من تکلم ننموده است؛ و گفت‌ای قوم همگی شما بشنوید.»
\par 28 پس پادشاه اسرائیل و یهوشافاط پادشاه یهودا به راموت جلعاد برآمدند.
\par 29 و پادشاه اسرائیل به یهوشافاط گفت: من خود را متنکرساخته، به جنگ می‌روم اما تو لباس خود رابپوش.» پس پادشاه اسرائیل خویشتن را متنکرساخت و ایشان به جنگ رفتند.
\par 30 و پادشاه ارام سرداران ارابه های خویش را امر فرموده، گفت: «نه با کوچک و نه با بزرگ بلکه با پادشاه اسرائیل فقط جنگ نمایید.»
\par 31 و چون سرداران ارابه هایهوشافاط را دیدند گمان بردند که این پادشاه اسرائیل است، پس مایل شدند تا با او جنگ نمایند و یهوشافاط فریاد برآورد و خداوند او رااعانت نمود و خدا ایشان را از او برگردانید.
\par 32 وچون سرداران ارابه‌ها را دیدند که پادشاه اسرائیل نیست، از تعاقب او برگشتند.
\par 33 اما کسی کمان خود را بدون غرض کشیده، پادشاه اسرائیل را میان وصله های زره زد، و او به ارابه ران خود گفت: «دست خود را بگردان و مرا از لشکر بیرون ببرزیرا که مجروح شدم.»و در آن روز جنگ سخت شد و پادشاه اسرائیل را در ارابه‌اش به مقابل ارامیان تا وقت عصر برپا داشتند و در وقت غروب آفتاب مرد.
\par 34 و در آن روز جنگ سخت شد و پادشاه اسرائیل را در ارابه‌اش به مقابل ارامیان تا وقت عصر برپا داشتند و در وقت غروب آفتاب مرد.
 
\chapter{19}

\par 1 و یهوشافاط پادشاه یهودا به خانه خودبه اورشلیم به سلامتی برگشت.
\par 2 و ییهوابن حنانی رایی برای ملاقات وی بیرون آمده، به یهوشافاط پادشاه گفت: «آیا شریران را می‌بایست اعانت نمایی و دشمنان خداوند را دوست داری؟ پس از این جهت غضب از جانب خداوند بر توآمده است.
\par 3 لیکن در تو اعمال نیکو یافت شده است چونکه اشیره‌ها را از زمین دور کرده، و دل خود را به طلب خدا تصمیم نموده‌ای.»
\par 4 و چون یهوشافاط در اورشلیم ساکن شد، باردیگر به میان قوم از بئرشبع تا کوهستان افرایم بیرون رفته، ایشان را به سوی یهوه خدای پدران ایشان برگردانید.
\par 5 و داوران در ولایت یعنی درتمام شهرهای حصاردار یهودا شهر به شهر قرارداد.
\par 6 و به داوران گفت: «باحذر باشید که به چه طور رفتار می‌نمایید زیرا که برای انسان داوری نمی نمایید بلکه برای خداوند، و او در حکم نمودن با شما خواهد بود.
\par 7 و حال خوف خداوندبر شما باشد و این را با احتیاط به عمل آورید زیراکه با یهوه خدای ما بی‌انصافی و طرفداری ورشوه خواری نیست.»
\par 8 و در اورشلیم نیز یهوشافاط بعضی از لاویان و کاهنان را و بعضی از روسای آبای اسرائیل را به جهت داوری خداوند و مرافعه‌ها قرار داد پس به اورشلیم برگشتند.
\par 9 و ایشان را امر فرموده، گفت: «شما بدینطور با امانت و دل کامل در ترس خداوند رفتار نمایید.
\par 10 و در هر دعوی‌ای که ازبرادران شما که ساکن شهرهای خود می‌باشند، میان خون و خون و میان شرایع و اوامر و فرایض و احکام پیش شما آید، ایشان را انذار نمایید تانزد خداوند مجرم نشوند، مبادا غضب بر شما و بربرادران شما بیاید. اگر به این طور رفتار نمایید، مجرم نخواهید شد.و اینک امریا، رئیس کهنه، برای همه امور خداوند و زبدیا ابن اسمعئیل که رئیس خاندان یهودا می‌باشد، برای همه امور پادشاه بر سر شما هستند و لاویان همراه شما در خدمت مشغولند. پس به دلیری عمل نمایید و خداوند با نیکان باشد.»
\par 11 و اینک امریا، رئیس کهنه، برای همه امور خداوند و زبدیا ابن اسمعئیل که رئیس خاندان یهودا می‌باشد، برای همه امور پادشاه بر سر شما هستند و لاویان همراه شما در خدمت مشغولند. پس به دلیری عمل نمایید و خداوند با نیکان باشد.»
 
\chapter{20}

\par 1 و بعد از این، بنی موآب و بنی عمون و باایشان بعضی از عمونیان، برای مقاتله بایهوشافاط آمدند.
\par 2 و بعضی آمده، یهوشافاط راخبر دادند و گفتند: «گروه عظیمی از آن طرف دریا از ارام به ضد تو می‌آیند و اینک ایشان درحصون تامار که همان عین جدی باشد، هستند.»
\par 3 پس یهوشافاط بترسید و در طلب خداوند جزم نمود و در تمامی یهودا به روزه اعلان کرد.
\par 4 ویهودا جمع شدند تا از خداوند مسالت نمایند واز تمامی شهرهای یهودا آمدند تا خداوند راطلب نمایند.
\par 5 و یهوشافاط در میان جماعت یهودا واورشلیم، در خانه خداوند، پیش روی صحن جدید بایستاد،
\par 6 و گفت: «ای یهوه، خدای پدران، ما آیا تو در آسمان خدا نیستی و آیا تو بر جمیع ممالک امت‌ها سلطنت نمی نمایی؟ و در دست تو قوت و جبروت است و کسی نیست که با تومقاومت تواند نمود.
\par 7 آیا تو خدای ما نیستی که سکنه این زمین را از حضور قوم خود اسرائیل اخراج نموده، آن را به ذریت دوست خویش ابراهیم تا ابدالاباد داده‌ای؟
\par 8 و ایشان در آن ساکن شده، مقدسی برای اسم تو در آن بنا نموده، گفتند:
\par 9 حینی که بلا یا شمشیر یا قصاص یا وبا یاقحطی بر ما عارض شود و ما پیش روی این خانه و پیش روی تو (زیرا که اسم تو در این خانه مقیم است ) بایستیم، و در وقت تنگی خود نزد تواستغاثه نماییم، آنگاه اجابت فرموده، نجات بده.
\par 10 و الان اینک بنی عمون و موآب و اهل کوه سعیر، که اسرائیل را وقتی که از مصر بیرون آمدنداجازت ندادی که به آنها داخل شوند، بلکه ازایشان اجتناب نمودند و ایشان را هلاک نساختند.
\par 11 اینک ایشان مکافات آن را به ما می‌رسانند، به اینکه می‌آیند تا ما را از ملک تو که آن را به تصرف ما داده‌ای، اخراج نمایند.
\par 12 ‌ای خدای ما آیا توبر ایشان حکم نخواهی کرد؟ زیرا که ما را به مقابل این گروه عظیمی که بر ما می‌آیند، هیچ قوتی نیست و ما نمی دانیم چه بکنیم اما چشمان ما به سوی تو است.»
\par 13 و تمامی یهودا با اطفال و زنان و پسران خود به حضور خداوند ایستاده بودند.
\par 14 آنگاه روح خداوند بر یحزئیل بن زکریا ابن بنایا ابن یعیئیل بن متنیای لاوی که از بنی آساف بود، درمیان جماعت نازل شد.
\par 15 و او گفت: «ای تمامی یهودا و ساکنان اورشلیم! و‌ای یهوشافاط پادشاه گوش گیرید! خداوند به شما چنین می‌گوید: ازاین گروه عظیم ترسان و هراسان مباشید زیرا که جنگ از آن شما نیست بلکه از آن خداست.
\par 16 فردا به نزد ایشان فرود آیید. اینک ایشان به فراز صیص برخواهند آمد و ایشان را در انتهای وادی در برابر بیابان یروئیل خواهید یافت.
\par 17 دراین وقت بر شما نخواهد بود که جنگ نمایید. بایستید و نجات خداوند را که با شما خواهد بودمشاهده نمایید. ای یهودا و اورشلیم ترسان وهراسان مباشید و فردا به مقابل ایشان بیرون رویدو خداوند همراه شما خواهد بود.»
\par 18 پس یهوشافاط رو به زمین افتاد و تمامی یهودا و ساکنان اورشلیم به حضور خداوندافتادند و خداوند را سجده نمودند.
\par 19 و لاویان ازبنی قهاتیان و از بنی قورحیان برخاسته، یهوه خدای اسرائیل را به آواز بسیار بلند تسبیح خواندند.
\par 20 و بامدادان برخاسته، به بیابان تقوع بیرون رفتند و چون بیرون می‌رفتند، یهوشافاط بایستادو گفت: «مرا بشنوید‌ای یهودا و سکنه اورشلیم! بر یهوه خدای خود ایمان آورید و استوارخواهید شد و به انبیای او ایمان آورید که کامیاب خواهید شد.»
\par 21 و بعد از مشورت کردن با قوم بعضی را معین کرد تا پیش روی مسلحان رفته، برای خداوند بسرایند و زینت قدوسیت راتسبیح خوانند و گویند خداوند را حمد گوییدزیرا که رحمت او تا ابدالاباد است.
\par 22 و چون ایشان به‌سراییدن و حمد گفتن شروع نمودند، خداوند به ضد بنی عمون و موآب و سکنه جبل سعیر که بر یهودا هجوم آوده بودند، کمین گذاشت و ایشان منکسر شدند.
\par 23 زیرا که بنی عمون و موآب بر سکنه جبل سعیر برخاسته، ایشان را نابود و هلاک ساختند، و چون از ساکنان سعیر فارغ شدند، یکدیگر را به‌کار هلاکت امدادکردند.
\par 24 و چون یهودا به دیده بانگاه بیابان رسیدند وبه سوی آن گروه نظر انداختند، اینک لاشه‌ها برزمین افتاده، و احدی رهایی نیافته بود.
\par 25 ویهوشافاط با قوم خود به جهت گرفتن غنیمت ایشان آمدند و در میان آنها اموال و رخوت وچیزهای گرانبها بسیار یافتند و برای خود آنقدرگرفتند که نتوانستند ببرند و غنیمت اینقدر زیادبود که سه روز مشغول غارت می‌بودند.
\par 26 و درروز چهارم در وادی برکه جمع شدند زیرا که درآنجا خداوند را متبارک خواندند، و از این جهت آن مکان را تا امروز وادی برکه می‌نامند.
\par 27 پس جمیع مردان یهودا و اورشلیم و یهوشافاط مقدم ایشان با شادمانی برگشته، به اورشلیم مراجعت کردند زیرا خداوند ایشان را بر دشمنانشان شادمان ساخته بود.
\par 28 و با بربطها و عودها وکرناها به اورشلیم به خانه خداوند آمدند.
\par 29 وترس خدا بر جمیع ممالک کشورها مستولی شدچونکه شنیدند که خداوند با دشمنان اسرائیل جنگ کرده است.
\par 30 و مملکت یهوشافاط آرام شد، زیرا خدایش او را از هر طرف رفاهیت بخشید.
\par 31 پس یهوشافاط بر یهودا سلطنت نمود وسی و پنج ساله بود که پادشاه شد و بیست و پنج سال در اورشلیم سلطنت کرد و اسم مادرش عزوبه دختر شلحی بود.
\par 32 و موافق رفتار پدرش آسا سلوک نموده، از آن انحراف نورزید و آنچه در نظر خداوند راست بود بجا می‌آورد.
\par 33 لیکن مکان های بلند برداشته نشد و قوم هنوز دلهای خود را به سوی خدای پدران خویش مصمم نساخته بودند.
\par 34 و بقیه وقایع یهوشافاط از اول تا آخر دراخبار ییهو ابن حنانی که در تواریخ پادشاهان اسرائیل مندرج می‌باشد، مکتوب است.
\par 35 و بعد از این، یهوشافاط پادشاه یهودا بااخزیا پادشاه اسرائیل که شریرانه رفتار می‌نمود، طرح آمیزش انداخت.
\par 36 و در ساختن کشتیهابرای رفتن به ترشیش با وی مشارکت نمود وکشتیها را در عصیون جابر ساختند.آنگاه العازر بن دوداواهوی مریشاتی به ضد یهوشافاطنبوت کرده، گفت: «چونکه تو با اخزیا متحدشدی، خداوند کارهای تو را تباه ساخته است.» پس آن کشتیها شکسته شدند و نتوانستند به ترشیش بروند.
\par 37 آنگاه العازر بن دوداواهوی مریشاتی به ضد یهوشافاطنبوت کرده، گفت: «چونکه تو با اخزیا متحدشدی، خداوند کارهای تو را تباه ساخته است.» پس آن کشتیها شکسته شدند و نتوانستند به ترشیش بروند.
 
\chapter{21}

\par 1 و یهوشافاط با پدران خود خوابید و درشهر داود با پدرانش دفن شد، و پسرش یهورام به‌جایش پادشاه شد.
\par 2 و پسران یهوشافاط عزریا و یحیئیل و زکریا و عزریاهو ومیکائیل و شفطیا برادران او بودند. این همه پسران یهوشافاط پادشاه اسرائیل بودند.
\par 3 و پدر ایشان عطایای بسیار از نقره و طلا و نفایس با شهرهای حصاردار در یهودا به ایشان داد و اما سلطنت را به یهورام عطا فرمود زیرا که نخست زاده بود. 
\par 4 و چون یهورام بر سلطنت پدرش مستقر شدخویشتن را تقویت نموده، همه برادران خود وبعضی از سروران اسرائیل را نیز به شمشیر کشت.
\par 5 یهورام سی و دو ساله بود که پادشاه شد و هشت سال در اورشلیم سلطنت کرد.
\par 6 و موافق رفتار پادشاهان اسرائیل به طوری که خاندان اخاب رفتار می‌کردند، سلوک نمود زیرا که دختر اخاب زن او بود و آنچه در نظر خداوند ناپسند بود، به عمل آورد.
\par 7 لیکن خداوند به‌سبب آن عهدی که با داود بسته بود و چونکه وعده داده بود که چراغی به وی و به پسرانش همیشه اوقات ببخشد، نخواست که خاندان داود را هلاک سازد.
\par 8 و در ایام او ادوم از زیردست یهودا عاصی شده، پادشاهی برای خود نصب نمودند.
\par 9 ویهورام با سرداران خود و تمامی ارابه هایش رفت، و شبانگاه برخاسته، ادومیان را که او را احاطه کرده بودند با سرداران ارابه های ایشان شکست داد.
\par 10 اما ادوم تا امروز از زیر دست یهودا عاصی شده‌اند، و در همان وقت لبنه نیز از زیر دست اوعاصی شد، زیرا که او یهوه خدای پدران خود راترک کرد.
\par 11 و او نیز مکان های بلند در کوههای یهوداساخت و ساکنان اورشلیم را به زنا کردن تحریض نموده، یهودا را گمراه ساخت.
\par 12 و مکتوبی ازایلیای نبی بدو رسیده، گفت که «یهوه، خدای پدرت داود، چنین می‌فرماید: چونکه به راههای پدرت یهوشافاط و به طریقهای آسا پادشاه یهوداسلوک ننمودی،
\par 13 بلکه به طریق پادشاهان اسرائیل رفتار نموده، یهودا و ساکنان اورشلیم رااغوا نمودی که موافق زناکاری خاندان اخاب مرتکب زنا بشوند و برادران خویش را نیز ازخاندان پدرت که از تو نیکوتر بودند به قتل رسانیدی،
\par 14 همانا خداوند قومت و پسرانت وزنانت و تمامی اموالت را به بلای عظیم مبتلاخواهد ساخت.
\par 15 و تو به مرض سخت گرفتار شده، در احشایت چنان بیماری‌ای عارض خواهد شد که احشایت از آن مرض روزبه روزبیرون خواهد آمد.»
\par 16 پس خداوند دل فلسطینیان و عربانی را که مجاور حبشیان بودند، به ضد یهورام برانگیزانید.
\par 17 و بر یهودا هجوم آورده، در آن ثلمه انداختند و تمامی اموالی که در خانه پادشاه یافت شد و پسران و زنان او را نیزبه اسیری بردند. و برای او پسری سوای پسرکهترش یهواخاز باقی نماند.
\par 18 و بعد از اینهمه خداوند احشایش را به مرض علاج ناپذیر مبتلا ساخت.
\par 19 و به مرور ایام بعد از انقضای مدت دو سال، احشایش از شدت مرض بیرون آمد و با دردهای سخت مرد، وقومش برای وی (عطریات ) نسوزانیدند، چنانکه برای پدرش می‌سوزانیدند.و او سی و دو ساله بود که پادشاه شد و هشت سال در اورشلیم سلطنت نمود، و بدون آنکه بر او رقتی شود، رحلت کرد و او را در شهر داود، اما نه در مقبره پادشاهان دفن کردند.
\par 20 و او سی و دو ساله بود که پادشاه شد و هشت سال در اورشلیم سلطنت نمود، و بدون آنکه بر او رقتی شود، رحلت کرد و او را در شهر داود، اما نه در مقبره پادشاهان دفن کردند.
 
\chapter{22}

\par 1 و ساکنان اورشلیم پسر کهترش اخزیارا در جایش به پادشاهی نصب کردند، زیرا گروهی که با عربان بر اردو هجوم آورده بودند، همه پسران بزرگش را کشته بودند. پس اخزیا ابن یهورام پادشاه یهودا سلطنت کرد.
\par 2 واخزیا چهل و دو ساله بود که پادشاه شد و یک سال در اورشلیم سلطنت کرد و اسم مادرش عتلیادختر عمری بود.
\par 3 و او نیز به طریق های خاندان اخاب سلوک نمود زیرا که مادرش ناصح او بود تا اعمال زشت بکند.
\par 4 و مثل خاندان اخاب آنچه در نظر خداوند ناپسند بود، بجا آورد زیرا که ایشان بعد از وفات پدرش، برای هلاکتش ناصح او بودند.
\par 5 پس برحسب مشورت ایشان رفتارنموده، با یهورام بن اخاب پادشاه اسرائیل نیزبرای جنگ با حزائیل پادشاه ارام به راموت جلعاد رفت و ارامیان یورام را مجروح نمودند.
\par 6 پس به یزرعیل مراجعت کرد تا از جراحاتی که در محاربه با حزائیل پادشاه ارام در رامه یافته بود، شفا یابد. و عزریا ابن یهورام پادشاه یهودا برای عیادت یهورام بن اخاب به یزرعیل فرود آمد زیراکه بیمار بود.
\par 7 و هلاکت اخزیا در اینکه نزد یورام رفت، ازجانب خدا بود زیرا چون به آنجا رسید، با یهورام به مقابله ییهو ابن نمشی که خداوند او را برای هلاک ساختن خاندان اخاب مسح کرده بود، بیرون رفت.
\par 8 و چون ییهو قصاص بر خاندان اخاب می‌رسانید، بعضی از سروران یهودا وپسران برادران اخزیا را که ملازمان اخزیا بودندیافته، ایشان را کشت.
\par 9 و اخزیا را طلبید و او رادر حالتی که در سامره پنهان شده بود، دستگیرنموده، نزد ییهو آوردند و او را به قتل رسانیده، دفن کردند زیرا گفتند: «پسر یهوشافاط است که خداوند را به تمامی دل خود طلبید.» پس، ازخاندان اخزیا، کسی‌که قادر بر سلطنت باشد، نماند.
\par 10 پس چون عتلیا مادر اخزیا دید که پسرش کشته شده است، برخاست و تمامی اولادپادشاهان از خاندان یهودا را هلاک کرد.
\par 11 لیکن یهوشبعه، دختر پادشاه، یوآش پسر اخزیا را گرفت و او را از میان پسران پادشاه که مقتول شدند دزدیده، او را با دایه‌اش در اطاق خوابگاه گذاشت و یهوشبعه، دختر یهورام پادشاه، زن یهویاداع کاهن که خواهر اخزیا بود، او را از عتلیاپنهان کرد که او را نکشت.و او نزد ایشان درخانه خدا مدت شش سال پنهان ماند. و عتلیا برزمین سلطنت می‌کرد.
\par 12 و او نزد ایشان درخانه خدا مدت شش سال پنهان ماند. و عتلیا برزمین سلطنت می‌کرد.
 
\chapter{23}

\par 1 و در سال هفتم، یهویاداع خویشتن راتقویت داده، بعضی از سرداران صده یعنی عزریا ابن یهورام و اسماعیل بن یهوحانان وعزریا ابن عوبید و معسیا ابن عدایا و الیشافاط بن زکری را با خود همداستان ساخت.
\par 2 و ایشان دریهودا گردش کردند و لاویان را از جمیع شهرهای یهودا و روسای آبای اسرائیل را جمع کرده، به اورشلیم آمدند.
\par 3 و تمامی جماعت با پادشاه درخانه خدا عهد بستند. و او به ایشان گفت: «هماناپسر پادشاه سلطنت خواهد کرد، چنانکه خداونددرباره پسران داود گفته است.
\par 4 و کاری که بایدبکنید این است: یک ثلث از شما که از کاهنان ولاویان در سبت داخل می‌شوید دربانهای آستانه‌ها باشید.
\par 5 و ثلث دیگر به خانه پادشاه وثلثی به دروازه اساس و تمامی قوم در صحنهای خانه خداوند حاضر باشند.
\par 6 و کسی غیر ازکاهنان و لاویانی که به خدمت مشغول می‌باشند، داخل خانه خداوند نشود، اما ایشان داخل بشوند زیرا که مقدسند و تمامی قوم (خانه ) خداوند را حراست نمایند.
\par 7 و لاویان هرکس سلاح خود را به‌دست گرفته، پادشاه را از هرطرف احاطه نمایند و هر‌که به خانه درآید، کشته شود و چون پادشاه داخل شود یا بیرون رود، شما نزد او بمانید.»
\par 8 پس لاویان و تمامی یهودا موافق هر‌چه یهویاداع کاهن امر فرمود عمل نمودند، و هر کدام کسان خود را خواه از آنانی که در روز سبت داخل می‌شدند و خواه از آنانی که در روز سبت بیرون می‌رفتند، برداشتند زیرا که یهویاداع کاهن فرقه هارا مرخص نفرمود.
\par 9 و یهویاداع کاهن نیزه‌ها ومجنها و سپرها را که از آن داود پادشاه و در خانه خدا بود، به یوزباشیها داد.
\par 10 و تمامی قوم را که هر یک از ایشان سلاح خود را به‌دست گرفته بودند، از طرف راست خانه تا طرف چپ خانه به پهلوی مذبح و خانه، به اطراف پادشاه قرار داد.
\par 11 و پسر پادشاه را بیرون آورده، تاج را بر سرش گذاشتند و شهادت نامه را به او داده، او را به پادشاهی نصب کردند، و یهویاداع و پسرانش، اورا مسح نموده، گفتند: «پادشاه زنده بماند.»
\par 12 اما چون عتلیا آواز قوم را که می‌دویدند وپادشاه را مدح می‌کردند شنید، نزد قوم به خانه خداوند داخل شد.
\par 13 و دید که اینک پادشاه به پهلوی ستون خود نزد مدخل ایستاده است، وسروران و کرنانوازان نزد پادشاه می‌باشند وتمامی قوم زمین شادی می‌کنند و کرناها رامی نوازند و مغنیان با آلات موسیقی و پیشوایان تسبیح. آنگاه عتلیا لباس خود را دریده، صدا زدکه «خیانت، خیانت!»
\par 14 و یهویاداع کاهن، یوزباشیها را که سرداران فوج بودند امر فرموده، به ایشان گفت: «او را از میان صفها بیرون کنید، وهر‌که از عقب او برود، به شمشیر کشته شود.» زیرا کاهن فرموده بود که او را در خانه خداوندمکشید.
\par 15 پس او را راه دادند و چون به دهنه دروازه اسبان، نزد خانه پادشاه رسید، او را درآنجا کشتند.
\par 16 و یهویاداع در میان خود و تمامی قوم وپادشاه، عهد بست تا قوم خداوند باشند.
\par 17 وتمامی قوم به خانه بعل رفته، آن را منهدم ساختندو مذبح هایش و تماثیلش را شکستند و کاهن بعل متان را روبه‌روی مذبحها کشتند.
\par 18 و یهویاداع باشادمانی و نغمه سرایی برحسب امر داود، وظیفه های خانه خداوند را به‌دست کاهنان ولاویان سپرد، چنانکه داود ایشان را بر خانه خداوند تقسیم کرده بود تا موافق آنچه در تواره موسی مکتوب است، قربانی های سوختنی خداوند را بگذرانند.
\par 19 و دربانان را به دروازه های خانه خداوند قرار داد تا کسی‌که به هر جهتی نجس باشد، داخل نشود.
\par 20 ویوزباشیها و نجبا و حاکمان قوم و تمامی قوم زمین را برداشت و پادشاه را از خانه خداوند به زیر آورد و او را از دروازه اعلی به خانه پادشاه درآورده، او را بر کرسی سلطنت نشانید.وتمامی قوم زمین شادی کردند و شهر آرامی یافت و عتلیا را به شمشیر کشتند.
\par 21 وتمامی قوم زمین شادی کردند و شهر آرامی یافت و عتلیا را به شمشیر کشتند.
 
\chapter{24}

\par 1 و یوآش هفت ساله بود که پادشاه شد وچهل سال در اورشلیم سلطنت نمود واسم مادرش ظبیه بئرشبعی بود.
\par 2 و یوآش درتمامی روزهای یهویاداع کاهن، آنچه را که در نظرخداوند راست بود، به عمل می‌آورد.
\par 3 ویهویاداع دو زن برایش گرفت و او پسران ودختران تولید نمود.
\par 4 و بعد از آن، یوآش اراده کرد که خانه خداوند را تعمیر نماید.
\par 5 و کاهنان و لاویان راجمع کرده، به ایشان گفت: «به شهرهای یهودابیرون روید و از تمامی اسرائیل نقره برای تعمیر خانه خدای خود، سال به سال جمع کنید، و دراین کار تعجیل نمایید.» اما لاویان تعجیل ننمودند.
\par 6 پس پادشاه، یهویاداع رئیس (کهنه ) راخوانده، وی را گفت: «چرا از لاویان بازخواست نکردی که جزیه‌ای را که موسی بنده خداوند وجماعت اسرائیل به جهت خیمه شهادت قرارداده‌اند، از یهودا و اورشلیم بیاورند؟»
\par 7 زیرا که پسران عتلیای خبیثه، خانه خدا را خراب کرده، وتمامی موقوفات خانه خداوند را صرف بعلیم کرده بودند.
\par 8 و پادشاه امر فرمود که صندوقی بسازند و آن را بیرون دروازه خانه خداوند بگذارند.
\par 9 و دریهودا و اورشلیم ندا دردادند که جزیه‌ای را که موسی بنده خدا در بیابان بر اسرائیل قرار داده بودبرای خداوند بیاورند.
\par 10 و جمیع سروران وتمامی قوم آن را به شادمانی آورده، در صندوق انداختند تا پر شد.
\par 11 و چون صندوق به‌دست لاویان، نزد وکلای پادشاه آورده می‌شد و ایشان می‌دیدند که نقره بسیار هست آنگاه کاتب پادشاه و وکیل رئیس کهنه آمده، صندوق را خالی می‌کردند و آن را برداشته، باز به‌جایش می‌گذاشتند. و روز به روز چنین کرده، نقره بسیارجمع کردند.
\par 12 و پادشاه و یهویاداع آن را به آنانی که در کار خدمت خانه خداوند مشغول بودنددادند، و ایشان بنایان و نجاران به جهت تعمیرخانه خداوند و آهنگران و مسگران برای مرمت خانه خداوند اجیر نمودند.
\par 13 پس عمله‌ها به‌کارپرداختند و کار از دست ایشان به انجام رسید وخانه خدا را به حالت اولش برپا داشته، آن رامحکم ساختند.
\par 14 و چون آن را تمام کرده بودند، بقیه نقره را نزد پادشاه و یهویاداع آوردند و از آن برای خانه خداوند اسباب یعنی آلات خدمت وآلات قربانی‌ها و قاشقها و ظروف طلا و نقره ساختند، و در تمامی روزهای یهویاداع، قربانی های سوختنی دائم در خانه خداوندمی گذرانیدند.
\par 15 اما یهویاداع پیر و سالخورده شده، بمرد وحین وفاتش صد و سی ساله بود.
\par 16 و او را درشهر داود با پادشاهان دفن کردند، زیرا که دراسرائیل هم برای خدا و هم برای خانه او نیکویی کرده بود. 
\par 17 و بعد از وفات یهویاداع، سروران یهوداآمدند و پادشاه را تعظیم نمودند و پادشاه در آن وقت به ایشان گوش گرفت.
\par 18 و ایشان خانه یهوه خدای پدران خود را ترک کرده، اشیریم و بتها راعبادت نمودند، و به‌سبب این عصیان ایشان، خشم بر یهودا و اورشلیم افروخته شد.
\par 19 و اوانبیاء نزد ایشان فرستاد تا ایشان را به سوی یهوه برگردانند و ایشان آنها را شهادت دادند، اما ایشان گوش نگرفتند.
\par 20 پس روح خدا زکریا ابن یهویاداع کاهن راملبس ساخت و او بالای قوم ایستاده، به ایشان گفت: «خدا چنین می‌فرماید: شما چرا از اوامریهوه تجاوز می‌نمایید؟ پس کامیاب نخواهیدشد. چونکه خداوند را ترک نموده‌اید، او شما راترک نموده است.»
\par 21 و ایشان بر او توطئه نموده، او را به حکم پادشاه در صحن خانه خداوندسنگسار کردند.
\par 22 پس یوآش پادشاه احسانی را که پدرش یهویاداع، به وی نموده بود، بیاد نیاورد، بلکه پسرش را به قتل رسانید. و چون او می‌مرد، گفت: «خداوند این را ببیند و بازخواست نماید.»
\par 23 و در وقت تحویل سال، لشکر ارامیان به ضد وی برآمده، به یهودا و اورشلیم داخل شده، جمیع سروران قوم را از میان قوم هلاک ساختند، و تمامی غنیمت ایشان را نزد پادشاه دمشق فرستادند.
\par 24 زیرا که لشکر ارام با جمعیت کمی آمدند و خداوند لشکر بسیار عظیمی به‌دست ایشان تسلیم نمود، چونکه یهوه خدای پدران خود را ترک کرده بودند، پس بر یوآش قصاص نمودند.
\par 25 و چون از نزد او رفتند (زیرا که او را درمرضهای سخت واگذاشتند)، بندگانش به‌سبب خون پسران یهویاداع کاهن، براو فتنه انگیخته، اورا بر بسترش کشتند. و چون مرد، او را در شهرداود دفن کردند، اما او را در مقبره پادشاهان دفن نکردند.
\par 26 و آنانی که بر او فتنه انگیختند، اینانند: زاباد، پسر شمعه عمونیه و یهوزاباد، پسرشمریت موآبیه.و اما حکایت پسرانش وعظمت وحی که بر او نازل شد و تعمیر خانه خدا، اینک در مدرس تواریخ پادشاهان مکتوب است، و پسرش امصیا در جایش پادشاه شد.
\par 27 و اما حکایت پسرانش وعظمت وحی که بر او نازل شد و تعمیر خانه خدا، اینک در مدرس تواریخ پادشاهان مکتوب است، و پسرش امصیا در جایش پادشاه شد.
 
\chapter{25}

\par 1 امصیا بیست و پنج ساله بود که پادشاه شد و بیست و نه سال در اورشلیم پادشاهی کرد و اسم مادرش یهوعدان اورشلیمی بود.
\par 2 و آنچه در نظر خداوند پسند بود، به عمل آورد، اما نه به دل کامل.
\par 3 و چون سلطنت در دستش مستحکم شد، خادمان خود را که پدرش پادشاه را کشته بودند، به قتل رسانید.
\par 4 اما پسران ایشان را نکشت به موجب نوشته کتاب تورات موسی که خداوند امر فرموده و گفته بود: «پدران به جهت پسران کشته نشوند و پسران به جهت پدران مقتول نگردند، بلکه هر کس به جهت گناه خود کشته شود.»
\par 5 و امصیا یهودا را جمع کرده، سرداران هزاره و سرداران صده از ایشان در تمامی یهودا وبنیامین مقرر فرمود و ایشان را از بیست ساله بالاترشمرده، سیصد هزار مرد برگزیده نیزه و سپردار راکه به جنگ بیرون می‌رفتند، یافت.
\par 6 و صد هزارمرد شجاع جنگ آزموده به صد وزنه نقره ازاسرائیل اجیر ساخت.
\par 7 اما مرد خدایی نزد وی آمده، گفت: «ای پادشاه، لشکر اسرائیل با تونروند زیرا خداوند با اسرائیل یعنی با تمامی بنی افرایم نیست.
\par 8 و اگر می‌خواهی بروی برو وبه جهت جنگ قوی شو لیکن خدا تو را پیش دشمنان مغلوب خواهد ساخت زیرا قدرت نصرت دادن و مغلوب ساختن با خدا است.»
\par 9 امصیا به مرد خدا گفت: «برای صد وزنه نقره که به لشکر اسرائیل داده‌ام، چه کنم؟» مرد خداجواب داد: «خداوند قادر است که تو را بیشتر ازاین بدهد.»
\par 10 پس امصیا لشکری را که از افرایم نزد او آمده بودند، جدا کرد که به‌جای خودبرگردند و از این سبب خشم ایشان بر یهودا به شدت افروخته شد و بسیار غضبناک گردیده، به‌جای خود رفتند.
\par 11 و امصیا خویشتن را تقویت نموده، قوم خود را بیرون برد و به وادی الملح رسیده، ده هزار نفر از بنی سعیر را کشت.
\par 12 و بنی یهودا ده هزار نفر دیگر را زنده اسیر کرد، و ایشان را به قله سالع برده، از قله سالع به زیر انداختند که جمیع خرد شدند.
\par 13 و اما مردان آن فوج که امصیا بازفرستده بود تا همراهش به جنگ نروند، برشهرهای یهودا از سامره تا بیت حورون تاختند وسه هزار نفر را کشته، غنیمت بسیار بردند.
\par 14 و بعد از مراجعت امصیا از شکست دادن ادومیان، او خدایان بنی سعیر را آورده، آنها راخدایان خود ساخت و آنها را سجده نموده، بخور برای آنها سوزانید.
\par 15 پس خشم خداوند برامصیا افروخته شد و نبی نزد وی فرستاد که او رابگوید: «چرا خدایان آن قوم را که قوم خود را ازدست تو نتوانستند رهانید، طلبیدی؟»
\par 16 و چون این سخن را به وی گفت، او را جواب داد: «آیا تورا مشیر پادشاه ساخته‌اند؟ ساکت شو! چرا تو رابکشند؟» پس نبی ساکت شده، گفت: «می‌دانم که خدا قصد نموده است که تو را هلاک کند، چونکه این کار را کردی و نصیحت مرا نشنیدی.»
\par 17 پس امصیا، پادشاه یهودا، مشورت کرده، نزد یوآش بن یهوآحاز بن ییهو پادشاه اسرائیل فرستاده، گفت: «بیا تا با یکدیگر مقابله نماییم.»
\par 18 و یوآش پادشاه اسرائیل نزد امصیا پادشاه یهودا فرستاده، گفت: «شترخار لبنان نزد سروآزاد لبنان فرستاده، گفت: دختر خود را به پسر من به زنی بده. اما حیوان وحشی که در لبنان بود گذرکرده، شترخار را پایمال نمود.
\par 19 می‌گویی، هان ادوم را شکست دادم و دلت تو را مغرور ساخته است که افتخار نمایی؟ حال به خانه خود برگرد. چرا بلا را برای خود برمی انگیزانی تا خودت ویهودا همراهت بیفتید؟»
\par 20 اما امصیا گوش نداد زیرا که این امر از جانب خدا بود تا ایشان را به‌دست دشمنان تسلیم نماید، چونکه خدایان ادوم را طلبیدند.
\par 21 پس یوآش پادشاه اسرائیل برآمد و او و امصیا پادشاه یهودا در بیت شمس که در یهودا است، با یکدیگرمقابله نمودند.
\par 22 و یهودا از حضور اسرائیل منهزم شده، هر کس به خیمه خود فرار کرد.
\par 23 ویوآش پادشاه اسرائیل امصیا ابن یوآش بن یهوآحاز پادشاه یهودا را در بیت شمس گرفت و او را به اورشلیم آورده، چهارصد ذراع حصاراورشلیم را از دروازه افرایم تا دروازه زاویه منهدم ساخت.
\par 24 و تمامی طلا و نقره و تمامی ظروفی را که در خانه خدا نزد (بنی ) عوبید ادوم ودر خزانه های خانه پادشاه یافت شد و یرغمالان را گرفته، به سامره مراجعت کرد.
\par 25 و امصیاابن یوآش پادشاه یهودا، بعد از وفات یوآش بن یهوآحاز پادشاه اسرائیل، پانزده سال زندگانی نمود.
\par 26 و بقیه وقایع اول و آخر امصیا، آیا درتواریخ پادشاهان یهودا و اسرائیل مکتوب نیست؟
\par 27 و از زمانی که امصیا از پیروی خداوندانحراف ورزید، بعضی در اورشلیم فتنه بر وی انگیختند. پس به لاکیش فرار کرد و از عقبش به لاکیش فرستادند و او را در آنجا کشتند.و او رابر اسبان آوردند و با پدرانش در شهر یهودا دفن کردند.
\par 28 و او رابر اسبان آوردند و با پدرانش در شهر یهودا دفن کردند.
 
\chapter{26}

\par 1 و تمامی قوم یهودا عزیا را که شانزده ساله بود گرفته، در جای پدرش امصیاپادشاه ساختند.
\par 2 و او بعد از آنکه پادشاه باپدرانش خوابیده بود، ایلوت را بنا کرد و آن رابرای یهودا استرداد نمود.
\par 3 و عزیا شانزده ساله بود که پادشاه شد و پنجاه و دو سال در اورشلیم پادشاهی نمود و اسم مادرش یکلیای اورشلیمی بود.
\par 4 و آنچه در نظر خداوند پسند بود، موافق هرچه پدرش امصیا کرده بود، بجا آورد.
\par 5 و درروزهای زکریا که در رویاهای خدا بصیر بود، خدا را می‌طلبید و مادامی که خداوند رامی طلبید، خدا او را کامیاب می‌ساخت.
\par 6 و او بیرون رفته، با فلسطینیان جنگ کرد وحصار جت و حصار یبنه و حصار اشدود رامنهدم ساخت و شهرها در زمین اشدود وفلسطینیان بنا نمود.
\par 7 و خدا او را بر فلسطینیان وعربانی که در جوربعل ساکن بودند و بر معونیان نصرت داد.
\par 8 و عمونیان به عزیا هدایا دادند و اسم او تا مدخل مصر شایع گردید، زیرا که بی‌نهایت قوی گشت.
\par 9 و عزیا برجها در اورشلیم نزددروازه زاویه و نزد دروازه وادی و نزد گوشه حصار بنا کرده، آنها را مستحکم گردانید.
\par 10 وبرجها در بیابان بنا نمود و چاههای بسیار کند زیراکه مواشی کثیر در همواری و در هامون داشت وفلاحان و باغبانان در کوهستان و در بوستانهاداشت، چونکه فلاحت را دوست می‌داشت.
\par 11 وعزیا سپاهیان جنگ آزموده داشت که برای محاربه‌دسته دسته بیرون می‌رفتند، برحسب تعداد ایشان که یعیئیل کاتب و معسیای رئیس زیردست حننیا که یکی از سرداران پادشاه بود، آنها را سان می‌دیدند.
\par 12 و عدد تمامی سرداران آبا که شجاعان جنگ آزموده بودند، دو هزار وششصد بود.
\par 13 و زیر دست ایشان، سیصد و هفت هزار و پانصد سپاه جنگ آزموده بودند که پادشاه را به ضد دشمنانش مساعدت نموده، با قوت تمام مقاتله می‌کردند.
\par 14 و عزیا برای ایشان یعنی برای تمامی لشکر سپرها و نیزه‌ها و خودها وزره‌ها و کمانها و فلاخنها مهیا ساخت.
\par 15 ومنجنیقهایی را که مخترع صنعتگران ماهر بود دراورشلیم ساخت تا آنها را بر برجها و گوشه های حصار برای انداختن تیرها و سنگهای بزرگ بگذارند. پس آوازه او تا جایهای دور شایع شدزیرا که نصرت عظیمی یافته، بسیار قوی گردید.
\par 16 لیکن چون زورآور شد، دل او برای هلاکتش متکبر گردید و به یهوه خدای خودخیانت ورزیده، به هیکل خداوند درآمد تا بخوربر مذبح بخور بسوزاند.
\par 17 و عزریای کاهن ازعقب او داخل شد و همراه او هشتاد مرد رشید ازکاهنان خداوند درآمدند.
\par 18 و ایشان با عزیاپادشاه مقاومت نموده، او را گفتند: «ای عزیاسوزانیدن بخور برای خداوند کار تو نیست بلکه کار کاهنان پسران هارون است که برای سوزانیدن بخور تقدیس شده‌اند. پس از مقدس بیرون شوزیرا خطا کردی و این کار از جانب یهوه خداموجب عزت تو نخواهد بود.»
\par 19 آنگاه عزیا که مجمری برای سوزانیدن بخور در دست خود داشت، غضبناک شد و چون خشمش بر کاهنان افروخته گردید، برص به حضور کاهنان در خانه خداوند به پهلوی مذبح بخور بر پیشانی‌اش پدید آمد.
\par 20 و عزریای رئیس کهنه و سایر کاهنان بر او نگریستند و اینک برص بر پیشانی‌اش ظاهر شده بود. پس او را ازآنجا به شتاب بیرون کردند و خودش نیز به تعجیل بیرون رفت، چونکه خداوند او را مبتلاساخته بود.
\par 21 و عزیا پادشاه تا روز وفاتش ابرص بود و در مریضخانه مبروص ماند، زیرا از خانه خداوند ممنوع بود و پسرش یوتام، ناظر خانه پادشاه و حاکم قوم زمین می‌بود.
\par 22 و اشعیا ابن آموص نبی بقیه وقایع اول وآخر عزیا را نوشت.پس عزیا با پدران خودخوابید و او را با پدرانش در زمین مقبره پادشاهان دفن کردند، زیرا گفتند که ابرص است و پسرش یوتام در جایش پادشاه شد.
\par 23 پس عزیا با پدران خودخوابید و او را با پدرانش در زمین مقبره پادشاهان دفن کردند، زیرا گفتند که ابرص است و پسرش یوتام در جایش پادشاه شد.
 
\chapter{27}

\par 1 و یوتام بیست و پنج ساله بود که پادشاه شد و شانزده سال در اورشلیم سلطنت نمود و اسم مادرش یروشه دختر صادوق بود.
\par 2 وآنچه در نظر خداوند پسند بود، موافق هرآنچه پدرش عزیا کرده بود، به عمل آورد، اما به هیکل خداوند داخل نشد لیکن قوم هنوز فسادمی کردند.
\par 3 و او دروازه اعلای خانه خداوند را بنانمود و بر حصار عوفل عمارت بسیار ساخت.
\par 4 وشهرها در کوهستان یهودا بنا نمود و قلعه‌ها وبرجها در جنگلها ساخت.
\par 5 و با پادشاه بنی عمون جنگ نموده، بر ایشان غالب آمد. پس بنی عمون در آن سال، صد وزنه نقره و ده هزار کر گندم و ده هزار کر جو به او دادند، و بنی عمون در سال دوم وسوم به همان مقدار به او دادند.
\par 6 پس یوتام زورآور گردید زیرا رفتار خود را به حضور یهوه خدای خویش راست ساخت.
\par 7 و بقیه وقایع یوتام و همه جنگهایش و رفتارش، اینک درتواریخ پادشاهان اسرائیل و یهودا مکتوب است.
\par 8 و او بیست و پنج ساله بود که پادشاه شد وشانزده سال در اورشلیم سلطنت کرد.پس یوتام با پدران خود خوابید و او را در شهر داود دفن کردند، و پسرش آحاز در جایش سلطنت نمود. 
\par 9 پس یوتام با پدران خود خوابید و او را در شهر داود دفن کردند، و پسرش آحاز در جایش سلطنت نمود.
 
\chapter{28}

\par 1 و آحاز بیست ساله بود که پادشاه شد وشانزده سال در اورشلیم پادشاهی کرد. اما آنچه در نظر خداوند پسند بود، موافق پدرش داود به عمل نیاورد.
\par 2 بلکه به طریق های پادشاهان اسرائیل سلوک نموده، تمثالها نیز برای بعلیم ریخت.
\par 3 و در وادی ابن هنوم بخورسوزانید، و پسران خود را برحسب رجاسات امت هایی که خداوند از حضور بنی‌اسرائیل اخراج نموده بود، سوزانید.
\par 4 و بر مکان های بلندو تلها و زیر هر درخت سبز قربانی‌ها گذرانید وبخور‌سوزانید.
\par 5 بنابراین، یهوه خدایش او را به‌دست پادشاه ارام تسلیم نمود که ایشان او را شکست داده، اسیران بسیاری از او گرفته، به دمشق بردند. و به‌دست پادشاه اسرائیل نیز تسلیم شد که او راشکست عظیمی داد.
\par 6 و فقح بن رملیا صد وبیست هزار نفر را که جمیع ایشان مردان جنگی بودند، در یک روز در یهودا کشت، چونکه یهوه خدای پدران خود را ترک نموده بودند.
\par 7 و زکری که مرد شجاع افرایمی بود، معسیا پسر پادشاه، عزریقام ناظر خانه، و القانه را که شخص اول بعداز پادشاه بود، کشت.
\par 8 پس بنی‌اسرائیل دویست هزار نفر زنان وپسران و دختران از برادران خود به اسیری بردند ونیز غنیمت بسیاری از ایشان گرفتند و غنیمت رابه سامره بردند.
\par 9 و در آنجا نبی از جانب خداوندعودید نام بود که به استقبال لشکری که به سامره برمی گشتند آمده، به ایشان گفت: «اینک از این جهت که یهوه خدای پدران شما بر یهوداغضبناک می‌باشد، ایشان را به‌دست شما تسلیم نمود و شما ایشان را با غضبی که به آسمان رسیده است، کشتید.
\par 10 و حال شما خیال می‌کنید که پسران یهودا و اورشلیم را به عنف غلامان وکنیزان خود سازید. و آیا با خود شما نیز تقصیرهابه ضد یهوه خدای شما نیست؟
\par 11 پس الان مرابشنوید و اسیرانی را که از برادران خود آورده‌اید، برگردانید زیرا که حدت خشم خداوند بر شمامی باشد.»
\par 12 آنگاه بعضی از روسای بنی افرایم یعنی عزریا ابن یهوحانان و برکیا ابن مشلیموت ویحزقیا ابن شلوم و عماسا ابن حدلای با آنانی که از جنگ می‌آمدند، مقاومت نمودند.
\par 13 و به ایشان گفتند که «اسیران را به اینجا نخواهید آوردزیرا که تقصیری به ضد خداوند بر ما هست وشما می‌خواهید که گناهان و تقصیرهای ما رامزید کنید زیرا که تقصیر ما عظیم است و حدت خشم بر اسرائیل وارد شده است.»
\par 14 پس لشکریان، اسیران و غنیمت را پیش روسا وتمامی جماعت واگذاشتند.
\par 15 و آنانی که اسم ایشان مذکور شد برخاسته، اسیران را گرفتند وهمه برهنگان ایشان را از غنیمت پوشانیده، ملبس ساختند و کفش به پای ایشان کرده، ایشان راخورانیدند و نوشانیدند و تدهین کرده، تمامی ضعیفان را بر الاغها سوار نموده، ایشان را به اریحا که شهر نخل باشد نزد برادرانشان رسانیده، خود به سامره مراجعت کردند.
\par 16 و در آن زمان، آحاز پادشاه نزد پادشاهان آشور فرستاد تا او را اعانت کنند.
\par 17 زیرا که ادومیان هنوز می‌آمدند و یهودا را شکست داده، اسیران می‌بردند.
\par 18 و فلسطینیان بر شهرهای هامون و جنوبی یهودا هجوم آوردند و بیت شمس و ایلون و جدیروت و سوکو را با دهاتش وتمنه را با دهاتش و جمزو را با دهاتش گرفته، درآنها ساکن شدند.
\par 19 زیرا خداوند یهودا را به‌سبب آحاز، پادشاه اسرائیل ذلیل ساخت، چونکه او یهودا را به‌سرکشی واداشت و به خداوندخیانت عظیمی ورزید.
\par 20 پس تلغت فلناسر، پادشاه آشور بر او برآمد و او را به تنگ آورد ووی را تقویت نداد.
\par 21 زیرا که آحاز خانه خداوندو خانه های پادشاه و سروران را تاراج کرده، به پادشاه آشور داد، اما او را اعانت ننمود.
\par 22 و چون او را به تنگ آورده بود، همین آحازپادشاه به خداوند بیشتر خیانت ورزید.
\par 23 زیرا که برای خدایان دمشق که او را شکست داده بودند، قربانی گذرانید و گفت: «چونکه خدایان پادشاهان ارام، ایشان را نصرت داده‌اند، پس من برای آنهاقربانی خواهم گذرانید تا مرا اعانت نمایند.» اماآنها سبب هلاکت وی و تمامی اسرائیل شدند.
\par 24 و آحاز اسباب خانه خدا را جمع کرد و آلات خانه خدا را خرد کرد و درهای خانه خداوند رابسته، مذبح‌ها برای خود در هر گوشه اورشلیم ساخت.
\par 25 و در هر شهری از شهرهای یهودا، مکان های بلند ساخت تا برای خدایان غریب بخور‌سوزانند. پس خشم یهوه خدای پدران خود را به هیجان آورد.
\par 26 و بقیه وقایع وی و همه طریق های اول و آخر او، اینک در تواریخ پادشاهان یهودا و اسرائیل مکتوب است.پس آحاز با پدران خود خوابید و او را در شهراورشلیم دفن کردند، اما او را به مقبره پادشاهان اسرائیل نیاوردند. و پسرش حزقیا به‌جایش پادشاه شد.
\par 27 پس آحاز با پدران خود خوابید و او را در شهراورشلیم دفن کردند، اما او را به مقبره پادشاهان اسرائیل نیاوردند. و پسرش حزقیا به‌جایش پادشاه شد.
 
\chapter{29}

\par 1 حزقیا بیست و پنج ساله بود که پادشاه شد و بیست و نه سال در اورشلیم سلطنت نمود، و اسم مادرش ابیه دختر زکریا بود.
\par 2 و او آنچه در نظر خداوند پسند بود، موافق هرآنچه پدرش داود کرده بود، به عمل آورد.
\par 3 و در ماه اول از سال اول سلطنت خوددرهای خانه خداوند را گشوده، آنها را تعمیرنمود.
\par 4 و کاهنان و لاویان را آورده، ایشان را درمیدان شرقی جمع کرد.
\par 5 و به ایشان گفت: «ای لاویان مرا بشنوید! الان خویشتن را تقدیس نمایید و خانه یهوه خدای پدران خود را تقدیس کرده، نجاسات را از قدس بیرون برید.
\par 6 زیرا که پدران ما خیانت ورزیده، آنچه در نظر یهوه خدای ما ناپسند بود به عمل آوردند و او را ترک کرده، روی خود را از مسکن خداوند تافتند وپشت به آن دادند.
\par 7 و درهای رواق را بسته، چراغها را خاموش کردند و بخور نسوزانیدند وقربانی های سوختنی در قدس خدای اسرائیل نگذرانیدند.
\par 8 پس خشم خداوند بر یهودا واورشلیم افروخته شد و ایشان را محل تشویش ودهشت و سخریه ساخت، چنانکه شما به چشمان خود می‌بینید.
\par 9 و اینک پدران ما به شمشیر افتادند و پسران و دختران و زنان ما از این سبب به اسیری رفتند.
\par 10 الان اراده دارم که بایهوه خدای اسرائیل عهد ببندم تا حدت خشم اواز ما برگردد.
\par 11 پس حال، ای پسران من، کاهلی مورزید زیرا خداوند شما را برگزیده است تا به حضور وی ایستاده، او را خدمت نمایید وخادمان او شده، بخور‌سوزانید.»
\par 12 آنگاه بعضی از لاویان برخاستند یعنی ازبنی قهاتیان محت بن عماسای و یوئیل بن عزریا واز بنی مراری قیس بن عبدی و عزریا ابن یهللئیل واز جرشونیان یوآخ بن زمه و عیدن بن یوآخ.
\par 13 واز بنی الیصافان شمری و یعیئیل و از بنی آساف زکریا و متنیا.
\par 14 و از بنی هیمان یحیئیل و شمعی و از بنی یدوتون شمعیا و عزیئیل.
\par 15 و برادران خود را جمع کرده، خویشتن را تقدیس نمودند وموافق فرمان پادشاه، برحسب کلام خداوند برای تطهیر خانه خداوند داخل شدند.
\par 16 و کاهنان به اندرون خانه خداوند رفته، آن را طاهر ساختند وهمه نجاسات را که در هیکل خداوند یافتند، به صحن خانه خداوند بیرون آوردند و لاویان آن راگرفته، خارج شهر به وادی قدرون بیرون بردند.
\par 17 و در غره ماه اول به تقدیس نمودنش شروع کردند، و در روز هشتم ماه به رواق خداوندرسیدند و در هشت روز خانه خداوند را تقدیس نموده، در روز شانزدهم ماه اول آن را به اتمام رسانیدند.
\par 18 پس نزد حزقیا پادشاه به اندرون قصر داخل شده، گفتند: «تمامی خانه خداوند ومذبح قربانی سوختنی و همه اسبابش و میز نان تقدمه را با همه آلاتش طاهر ساختیم.
\par 19 و تمامی اسبابی که آحاز پادشاه در ایام سلطنتش حینی که خیانت ورزید دور انداخت ما آنها را مهیا ساخته، تقدیس نمودیم و اینک پیش مذبح خداوندحاضر است.»
\par 20 پس حزقیا پادشاه صبح زود برخاست وروسای شهر را جمع کرده، به خانه خداوندبرآمد.
\par 21 و ایشان هفت گاو و هفت قوچ و هفت بره و هفت بز نر آوردند تا برای مملکت و قدس ویهودا قربانی گناه بشود و او پسران هارون کهنه رافرمود تا آنها را بر مذبح خداوند بگذرانند.
\par 22 پس گاوان را کشتند و کاهنان، خون را گرفته بر مذبح پاشیدند و قوچها را کشته خون را بر مذبح پاشیدند و بره‌ها را کشته خون را بر مذبح پاشیدند.
\par 23 پس بزهای قربانی گناه را به حضور پادشاه وجماعت نزدیک آورده، دستهای خود را بر آنهانهادند.
\par 24 و کاهنان آنها را کشته، خون را بر مذبح برای قربانی گناه گذرانیدند تا به جهت تمامی اسرائیل کفاره بشود زیرا که پادشاه فرموده بود که قربانی سوختنی و قربانی گناه به جهت تمامی اسرائیل بشود.
\par 25 و او لاویان را با سنجها و بربطها و عودهابرحسب فرمان داود و جاد، رایی پادشاه و ناتان نبی در خانه خداوند قرار داد زیرا که این حکم ازجانب خداوند به‌دست انبیای او شده بود.
\par 26 پس لاویان با آلات داود و کاهنان با کرناها ایستادند.
\par 27 و حزقیا امر فرمود که قربانی های سوختنی رابر مذبح بگذرانند و چون به گذرانیدن قربانی سوختنی شروع نمودند، سرودهای خداوند را بناکردند و کرناها در عقب آلات داود، پادشاه اسرائیل، نواخته شد.
\par 28 و تمامی جماعت سجده کردند و مغنیان سراییدند و کرنانوازان نواختند وهمه این کارها می‌شد تا قربانی سوختنی تمام گردید.
\par 29 و چون قربانی های سوختنی تمام شد، پادشاه و جمیع حاضرین با وی رکوع کرده، سجده نمودند.
\par 30 و حزقیا پادشاه و روسا لاویان را امر فرمودند که به کلمات داود و آساف رایی برای خداوند تسبیح بخوانند. پس با شادمانی تسبیح خواندند و رکوع نموده، سجده کردند.
\par 31 پس حزقیا جواب داده، گفت: «حال خویشتن را برای خداوند تقدیس نمودید. پس نزدیک آمده، قربانی‌ها و ذبایح تشکر به خانه خداوند بیاورید.» آنگاه جماعت قربانی‌ها وذبایح تشکر آوردند و هر‌که از دل راغب بودقربانی های سوختنی آورد.
\par 32 و عدد قربانی های سوختنی که جماعت آوردند، هفتاد گاو و صدقوچ و دویست بره بود. همه اینها قربانی های سوختنی برای خداوند بود.
\par 33 و عدد موقوفات ششصد گاو و سه هزار گوسفند بود.
\par 34 و چون کاهنان کم بودند و به پوست کندن همه قربانی های سوختنی قادر نبودند، برادران ایشان لاویان، ایشان را مدد کردند تا کار تمام شد و تاکاهنان، خود را تقدیس نمودند زیرا که دل لاویان از کاهنان برای تقدیس نمودن خود مستقیم تربود.
\par 35 و قربانی های سوختنی نیز با پیه ذبایح سلامتی و هدایای ریختنی برای هر قربانی سوختنی، بسیار بود. پس خدمت خانه خداوندآراسته شد.و حزقیا و تمامی قوم شادی کردند چونکه خدا قوم را مستعد ساخته بود زیرااین امر ناگهان واقع شد.
\par 36 و حزقیا و تمامی قوم شادی کردند چونکه خدا قوم را مستعد ساخته بود زیرااین امر ناگهان واقع شد.
 
\chapter{30}

\par 1 و حزقیا نزد تمامی اسرائیل و یهودافرستاد و مکتوبات نیز به افرایم و منسی نوشت تا به خانه خداوند به اورشلیم بیایند و عیدفصح را برای یهوه خدای اسرائیل نگاه دارند.
\par 2 زیرا که پادشاه و سرورانش و تمامی جماعت در اورشلیم مشورت کرده بودند که عید فصح را در ماه دوم نگاه دارند.
\par 3 چونکه در آنوقت نتوانستند آن را نگاه دارند زیرا کاهنان خود راتقدیس کافی ننموده و قوم در اورشلیم جمع نشده بودند.
\par 4 و این امر به نظر پادشاه و تمامی جماعت پسند آمد.
\par 5 پس قرار دادند که در تمامی اسرائیل از بئرشبع تا دان ندا نمایند که بیایند وفصح را برای یهوه خدای اسرائیل در اورشلیم برپا نمایند زیرا مدت مدیدی بود که آن را به طوری که مکتوب است، نگاه نداشته بودند.
\par 6 پس شاطران با مکتوبات از جانب پادشاه و سرورانش، برحسب فرمان پادشاه به تمامی اسرائیل و یهودارفته، گفتند: «ای بنی‌اسرائیل به سوی یهوه، خدای ابراهیم و اسحاق و اسرائیل باز گشت نمایید تا او به بقیه شما که از دست پادشاهان آشور رسته‌اید، رجوع نماید.
\par 7 و مثل پدران وبرادران خود که به یهوه خدای پدران خویش خیانت ورزیدند، مباشید که ایشان را محل دهشت چنانکه می‌بینید گردانیده است. 
\par 8 پس مثل پدران خود گردن خود را سخت مسازیدبلکه نزد خداوند تواضع نمایید و به قدس او که آن را تا ابدالاباد تقدیس نموده است داخل شده، یهوه خدای خود را عبادت نمایید تا حدت خشم او از شما برگردد.
\par 9 زیرا اگر به سوی خداوندبازگشت نمایید، برادران و پسران شما به نظرآنانی که ایشان را به اسیری برده‌اند، التفات خواهند یافت و به این زمین مراجعت خواهندنمود، زیرا که یهوه خدای شما مهربان و رحیم است و اگر به سوی او بازگشت نمایید روی خودرا از شما بر نخواهد گردانید.»
\par 10 پس شاطران شهر به شهر از زمین افرایم ومنسی تا زبولون گذشتند، لیکن بر ایشان استهزا وسخریه می‌نمودند.
\par 11 اما بعضی از اشیر و منسی و زبولون تواضع نموده، به اورشلیم آمدند.
\par 12 ودست خدا بر یهودا بود که ایشان را یک دل بخشدتا فرمان پادشاه و سرورانش را موافق کلام خداوند بجا آورند.
\par 13 پس گروه عظیمی در اورشلیم برای نگاه داشتن عید فطیر در ماه دوم جمع شدند وجماعت، بسیار بزرگ شد.
\par 14 و برخاسته، مذبح هایی را که در اورشلیم بود خراب کردند وهمه مذبح های بخور را خراب کرده، به وادی قدرون انداختند.
\par 15 و در چهاردهم ماه دوم فصح را ذبح کردند و کاهنان و لاویان خجالت کشیده، خود را تقدیس نمودند و قربانی های سوختنی به خانه خداوند آوردند.
\par 16 پس در جایهای خود به ترتیب خویش برحسب تورات موسی مرد خداایستادند و کاهنان خون را از دست لاویان گرفته، پاشیدند.
\par 17 زیرا چونکه بسیاری از جماعت بودند که خود را تقدیس ننموده بودند لاویان مامور شدند که قربانی های فصح را به جهت هرکه طاهر نشده بود ذبح نمایند و ایشان را برای خداوند تقدیس کنند.
\par 18 زیرا گروهی عظیم ازقوم یعنی بسیار از افرایم و منسی و یساکار وزبولون طاهر نشده بودند و معهذا فصح راخوردند لکن نه موافق آنچه نوشته شده بود، زیراحزقیا برای ایشان دعا کرده، گفت:
\par 19 «خداوندمهربان، هر کس را که دل خود را مهیا سازد تا خدایعنی یهوه خدای پدران خویش را طلب نمایدبیامرزد، اگرچه موافق طهارت قدس نباشد.»
\par 20 وخداوند حزقیا را اجابت نموده، قوم را شفا داد.
\par 21 پس بنی‌اسرائیل که در اورشلیم حاضر بودند، عید فطیر را هفت روز به شادی عظیم نگاه داشتندو لاویان و کاهنان خداوند را روز به روز به آلات تسبیح خداوند حمد می‌گفتند.
\par 22 و حزقیا به جمیع لاویانی که در خدمت خداوند نیکو ماهربودند، سخنان دلاویز گفت. پس هفت روزمرسوم عید را خوردند و ذبایح سلامتی گذرانیده، یهوه خدای پدران خود را تسبیح خواندند.
\par 23 و تمامی جماعت مشورت کردند که عید راهفت روز دیگر نگاه دارند. پس هفت روز دیگر رابا شادمانی نگاه داشتند.
\par 24 زیرا حزقیا، پادشاه یهودا هزار گاو و هفت هزار گوسفند به جماعت بخشید و سروران هزار گاو و ده هزار گوسفند به جماعت بخشیدند و بسیاری از کاهنان خویشتن را تقدیس نمودند.
\par 25 و تمامی جماعت یهودا وکاهنان و لاویان و تمامی گروهی که از اسرائیل آمدند و غریبانی که از زمین اسرائیل آمدند و(غریبانی که ) در یهودا ساکن بودند، شادی کردند.
\par 26 و شادی عظیمی در اورشلیم رخ نمود زیرا که از ایام سلیمان بن داود، پادشاه اسرائیل مثل این در اورشلیم واقع نشده بود.پس لاویان کهنه برخاسته، قوم را برکت دادند و آواز ایشان مستجاب گردید و دعای ایشان به مسکن قدس اوبه آسمان رسید.
\par 27 پس لاویان کهنه برخاسته، قوم را برکت دادند و آواز ایشان مستجاب گردید و دعای ایشان به مسکن قدس اوبه آسمان رسید.
 
\chapter{31}

\par 1 و چون این همه تمام شد، جمیع اسرائیلیانی که در شهرهای یهوداحاضر بودند بیرون رفته، تمثالها را شکستند واشیریم را قطع نمودند و مکانهای بلند و مذبحهارا از تمامی یهودا و بنیامین و افرایم و منسی بالکل منهدم ساختند. پس تمامی بنی‌اسرائیل هرکس به ملک خویش به شهرهای خود برگشتند.
\par 2 و حزقیا فرقه های کاهنان و لاویان، رابرحسب اقسام ایشان قرار داد که هر کس ازکاهنان و لاویان موافق خدمت خود برای قربانی های سوختنی و ذبایح سلامتی و خدمت وتشکر و تسبیح به دروازه های اردوی خداوندحاضر شوند.
\par 3 و حصه پادشاه را از اموال خاصش برای قربانی های سوختنی معین کرد، یعنی برای قربانی های سوختنی صبح و شام وقربانی های سوختنی سبت‌ها و هلالها و موسمهابرحسب آنچه در تورات خداوند مکتوب بود.
\par 4 وبه قومی که در اورشلیم ساکن بودند، امر فرمود که حصه کاهنان و لاویان را بدهند تا به شریعت خداوند مواظب باشند.
\par 5 و چون این امر شایع شد، بنی‌اسرائیل نوبر گندم و شیره و روغن وعسل و تمامی محصول زمین را به فراوانی دادند وعشر همه‌چیز را به کثرت آوردند.
\par 6 وبنی‌اسرائیل و یهودا که در شهرهای یهودا ساکن بودند نیز عشر گاوان و گوسفندان و عشرموقوفاتی که برای یهوه خدای ایشان وقف شده بود آورده، آنها را توده توده نمودند.
\par 7 و در ماه سوم به ساختن توده‌ها شروع نمودند، و در ماه هفتم آنها را تمام کردند.
\par 8 و چون حزقیا وسروران آمدند و توده‌ها را دیدند، خداوند رامتبارک خواندند و قوم او اسرائیل را مبارک خواندند.
\par 9 و حزقیا درباره توده‌ها از کاهنان ولاویان سوآل نمود.
\par 10 و عزریا رئیس کهنه که ازخاندان صادوق بود او را جواب داد و گفت: «ازوقتی که قوم به آوردن هدایا برای خانه خداوندشروع کردند، خوردیم و سیر شدیم و بسیاری باقی گذاشتیم، زیرا خداوند قوم خود را برکت داده است و آنچه باقی‌مانده است، این مقدارعظیم است.»
\par 11 پس حزقیا امر فرمود که انبارها در خانه خداوند مهیا سازند و مهیا ساختند.
\par 12 و هدایا وعشرها و موقوفات را در آنها در مکان امانت گذاشتند و کوننیای لاوی بر آنها رئیس بود وبرادرش شمعی ثانی اثنین.
\par 13 و یحیئیل و عزریاو نحت و عسائیل و یریموت و یوزاباد و ایللئیل ویسمخیا و محت و بنایا برحسب تعیین حزقیاپادشاه و عزریا رئیس خانه خدا زیر دست کوننیاو برادرش شمعی وکلاء شدند.
\par 14 و قوری ابن یمنه لاوی که دربان دروازه شرقی بود ناظر نوافل خدا شد تا هدایای خداوند و موقوفات مقدس راتقسیم نماید.
\par 15 و زیردست او عیدن و منیامین ویشوع و شمعیا و امریا و شکنیا در شهرهای کاهنان به وظیفه های امانتی مقرر شدند تا به برادران خود، خواه بزرگ و خواه کوچک، برحسب فرقه های ایشان برسانند.
\par 16 علاوه برحصه یومیه ذکوری که در نسب نامه‌ها شمرده شده بودند، از سه ساله و بالاتر یعنی همه آنانی که به خانه خداوند داخل می‌شدند، برای خدمت های ایشان در وظیفه های ایشان برحسب فرقه های ایشان.
\par 17 (و حصه ) آنانی که در نسب نامه‌ها شمرده شده بود، از کاهنان برحسب خاندان آبای ایشان و از لاویان از بیست ساله وبالاتر در وظیفه های ایشان برحسب فرقه های ایشان.
\par 18 و (حصه ) جمیع اطفال و زنان و پسران و دختران ایشان که در تمامی جماعت درنسب نامه‌ها شمرده شده بودند، پس دروظیفه های امانتی خود خویشتن را تقدیس نمودند.
\par 19 و نیز برای پسران هارون کهنه که درزمینهای حوالی شهرهای خود ساکن بودند، کسان، شهر به شهر به نامهای خود معین شدند تابه همه ذکوران کهنه و به همه لاویانی که درنسب نامه‌ها شمرده شده بودند حصه‌ها بدهند.
\par 20 پس حزقیا در تمامی یهودا به اینطور عمل نمود و آنچه در نظر یهوه خدایش نیکو و پسند وامین بود بجا آورد.و در هر کاری که در خدمت خانه خدا و در شرایع و اوامر برای طلبیدن خدای خود اقدام نمود آن را به تمامی دل خود به عمل آورد و کامیاب گردید.
\par 21 و در هر کاری که در خدمت خانه خدا و در شرایع و اوامر برای طلبیدن خدای خود اقدام نمود آن را به تمامی دل خود به عمل آورد و کامیاب گردید.
 
\chapter{32}

\par 1 و بعد از این امور و این امانت، سنخاریب، پادشاه آشور آمده، به یهودا داخل شد، و به ضد شهرهای حصارداراردو زده، خواست که آنها را برای خود مفتوح نماید.
\par 2 و چون حزقیا دید که سنخاریب آمده است و قصد مقاتله با اورشلیم دارد،
\par 3 آنگاه باسرداران و شجاعان خود مشورت کرد که آب چشمه های بیرون شهر را مسدود نماید. پس اورااعانت کردند.
\par 4 و خلق بسیاری جمع شده، همه چشمه‌ها و نهری را که از میان زمین جاری بودمسدود کردند، و گفتند: «چرا باید پادشاهان آشور بیایند و آب فراوان بیابند؟»
\par 5 پس خویشتن را تقویت داده، تمامی حصار را که شکسته بود، تعمیر نمود و آن را تا برجها بلند نمود و حصاردیگری بیرون آن بنا کرد و ملو را در شهر داودمستحکم نمود و اسلحه‌ها و سپرهای بسیاری ساخت.
\par 6 و سرداران جنگی بر قوم گماشت وایشان را در جای وسیع نزد دروازه شهر جمع کرده، سخنان دلاویز به ایشان گفت
\par 7 که «دلیر وقوی باشید! و از پادشاه آشور و تمامی جمیعتی که با وی هستند، ترسان و هراسان مشوید! زیرا آنکه با ماست از آنکه با وی است قوی تر می‌باشد.
\par 8 با او بازوی بشری است و با ما یهوه خدای مااست تا ما را نصرت دهد و در جنگهای ما جنگ کند.» پس قوم بر سخنان حزقیا پادشاه یهودااعتماد نمودند.
\par 9 و بعد از آن سنخاریب، پادشاه آشور، بندگان خود را به اورشلیم فرستاد و خودش با تمامی حشمتش در برابر لاکیش بودند که به حزقیاپادشاه یهودا و تمامی یهودا که در اورشلیم بودند، بگویند:
\par 10 «سنخاریب پادشاه آشورچنین می‌فرماید: بر چه چیز اعتماد دارید که درمحاصره در اورشلیم می‌مانید؟
\par 11 آیا حزقیا شمارا اغوا نمی کند تا شما را با قحط و تشنگی به موت تسلیم نماید که می‌گوید: یهوه خدای ما، مارا از دست پادشاه آشور رهایی خواهد داد؟
\par 12 آیا همین حزقیا مکانهای بلند و مذبحهای اورا منهدم نساخته، و به یهودا و اورشلیم امرنفرموده و نگفته است که پیش یک مذبح سجده نمایید و بر آن بخور بسوزانید؟
\par 13 آیا نمی دانیدکه من و پدرانم به همه طوایف کشورها چه کرده‌ایم؟ مگر خدایان امت های آن کشورها هیچ قدرتی داشتند که زمین خود را از دست من برهانند؟
\par 14 کدام‌یک از همه خدایان این امت هایی که پدران من آنها را هلاک ساخته‌اند، قادر بر رهانیدن قوم خود از دست من بود تاخدای شما قادر باشد که شما را از دست من رهایی دهد؟
\par 15 پس حال، حزقیا شما را فریب ندهد و شما را به اینطور اغوا ننماید و بر اواعتماد منمایید، زیرا هیچ خدا از خدایان جمیع امت‌ها و ممالک قادر نبوده است که قوم خود را ازدست من و از دست پدرانم رهایی دهد، پس به طریق اولی خدای شما شما را از دست من نخواهد رهانید.»
\par 16 و بندگانش سخنان زیاده به ضد یهوه خدا وبه ضد بنده‌اش حزقیا گفتند.
\par 17 و مکتوبی نیزنوشته، یهوه خدای اسرائیل را اهانت نمود و به ضد وی حرف زده، گفت: «چنانکه خدایان امت های کشورها قوم خود را از دست من رهایی ندادند، همچنین خدای حزقیا قوم خویش را ازدست من نخواهد رهانید.»
\par 18 و به آواز بلند به زبان یهود به اهل اورشلیم که بر دیوار بودند، ندادر‌دادند تا ایشان را ترسان و مشوش ساخته، شهررا بگیرند.
\par 19 و درباره خدای اورشلیم مثل خدایان امت های جهان که مصنوع دست آدمیان می‌باشند، سخن‌گفتند.
\par 20 پس حزقیا پادشاه و اشعیاء ابن آموص نبی درباره این دعا کردند و به سوی آسمان فریادبرآوردند.
\par 21 و خداوند فرشته‌ای فرستاده، همه شجاعان جنگی و روسا و سرداران را که دراردوی پادشاه آشور بودند، هلاک ساخت و او باروی شرمنده به زمین خود مراجعت نمود. وچون به خانه خدای خویش داخل شد، آنانی که از صلبش بیرون آمده بودند، او را در آنجا به شمشیر کشتند.
\par 22 پس خداوند حزقیا و سکنه اورشلیم را از دست سنحاریب پادشاه آشور و ازدست همه رهایی داده، ایشان را از هر طرف نگاهداری نمود.
\par 23 و بسیاری هدایا به اورشلیم برای خداوند و پیشکشها برای حزقیا پادشاه یهودا آوردند و او بعد از آن به نظر همه امت هامحترم شد.
\par 24 و در آن ایام حزقیا بیمار و مشرف به موت شد. اما چون نزد خداوند دعا نمود، او با وی تکلم کرد و وی را علامتی داد.
\par 25 لیکن حزقیا موافق احسانی که به وی داده شده بود، عمل ننمود زیرادلش مغرور شد و غضب بر او و یهودا و اورشلیم افروخته گردید. 
\par 26 اما حزقیا با ساکنان اورشلیم، از غرور دلش تواضع نمود، لهذا غضب خداونددر ایام حزقیا بر ایشان نازل نشد.
\par 27 و حزقیا دولت و حشمت بسیار عظیمی داشت و به جهت خود مخزنها برای نقره و طلا وسنگهای گرانبها و عطریات و سپرها و هر گونه اسباب نفیسه ساخت.
\par 28 و انبارها برای محصولات از گندم و شیره و روغن و آخرها برای انواع بهایم و آغلها به جهت گله‌ها.
\par 29 و به جهت خود شهرها ساخت و مواشی گله‌ها و رمه های بسیار تحصیل نمود زیرا خدا اندوخته های بسیارفراوان به او عطا فرمود.
\par 30 و همین حزقیا منبع عالی آب جیحون را مسدود ساخته، آن را به راه راست به طرف غربی شهر داود فرود آورد. پس حزقیا در تمامی اعمالش کامیاب شد.
\par 31 اما درامر ایلچیان سرداران بابل که نزد وی فرستاده شده بودند تا درباره آیتی که در زمین ظاهر شده بودپرسش نمایند، خدا او را واگذاشت تا او راامتحان نماید و هر‌چه در دلش بود بداند.
\par 32 و بقیه وقایع حزقیا و حسنات او اینک دررویای اشعیا ابن آموص نبی و در تواریخ پادشاهان یهودا و اسرائیل مکتوب است.پس حزقیا با پدران خود خوابید و او را در بلندی مقبره پسران داود دفن کردند، و تمامی یهودا وساکنان اورشلیم او را در حین وفاتش اکرام نمودند، و پسرش منسی در جایش سلطنت نمود.
\par 33 پس حزقیا با پدران خود خوابید و او را در بلندی مقبره پسران داود دفن کردند، و تمامی یهودا وساکنان اورشلیم او را در حین وفاتش اکرام نمودند، و پسرش منسی در جایش سلطنت نمود.
 
\chapter{33}

\par 1 منسی دوازده ساله بود که پادشاه شد وپنجاه و پنج سال در اورشلیم سلطنت نمود.
\par 2 و آنچه در نظر خداوند ناپسند بود، موافق رجاسات امت هایی که خداوند آنها را از حضوربنی‌اسرائیل اخراج کرده بود، عمل نمود.
\par 3 زیرامکانهای بلند را که پدرش حزقیا خراب کرده بود، بار دیگر بنا نمود و مذبحها برای بعلیم برپا کرد واشیره‌ها بساخت و به تمامی لشکر آسمان سجده نموده، آنها را عبادت کرد.
\par 4 و مذبح‌ها در خانه خداوند بنا نمود که درباره‌اش خداوند گفته بود: «اسم من در اورشلیم تا به ابد خواهد بود.»
\par 5 ومذبح‌ها برای تمامی لشکر آسمان در هر دوصحن خانه خداوند بنا نمود.
\par 6 و پسران خود رادر وادی ابن هنوم از آتش گذرانید و فالگیری وافسونگری و جادوگری می‌کرد و با اصحاب اجنه و جادوگران مراوده می‌نمود و در نظرخداوند شرارت بسیار ورزیده، خشم او را به هیجان آورد.
\par 7 و تمثال ریخته شده بت را که ساخته بود، در خانه خداوند برپا داشت که درباره‌اش خدا به داود و به پسرش سلیمان گفته بود: «در این خانه و در اورشلیم که آن را از تمامی اسباط بنی‌اسرائیل برگزیده‌ام، اسم خود را تا به ابد قرار خواهم داد.
\par 8 و پایهای اسرائیل را اززمینی که مقر پدران شما ساخته‌ام، بار دیگر آواره نخواهم گردانید، به شرطی که توجه نمایند تابرحسب هر‌آنچه به ایشان امر فرموده‌ام و برحسب تمامی شریعت و فرایض و احکامی که به‌دست موسی داده‌ام، عمل نمایند.»
\par 9 امامنسی، یهودا و ساکنان اورشلیم را اغوا نمود تا ازامت هایی که خداوند پیش بنی‌اسرائیل هلاک کرده بود، بدتر رفتار نمودند.
\par 10 و خداوند به منسی و به قوم او تکلم نمود، اما ایشان گوش نگرفتند.
\par 11 پس خداوندسرداران لشکر آشور را بر ایشان آورد و منسی رابا غلها گرفته، او را به زنجیرها بستند و به بابل بردند.
\par 12 و چون در تنگی بود یهوه خدای خود راطلب نمود و به حضور خدای پدران خویش بسیار تواضع نمود.
\par 13 و چون از او مسالت نمودوی را اجابت نموده، تضرع او را شنید و به مملکتش به اورشلیم باز آورد، آنگاه منسی دانست که یهوه خدا است.
\par 14 و بعد از این حصار بیرونی شهر داود را به طرف غربی جیحون در وادی تا دهنه دروازه ماهی بنا نمود و دیواری گرداگرد عوفل کشیده، آن را بسیار بلند ساخت و سرداران جنگی بر همه شهرهای حصاردار یهودا قرار داد.
\par 15 و خدایان بیگانه و بت را از خانه خداوند و تمامی مذبحها راکه در کوه خانه خداوند و در اورشلیم ساخته بودبرداشته، آنها را از شهر بیرون ریخت.
\par 16 و مذبح خداوند را تعمیر نموده، ذبایح سلامتی و تشکربر آن گذرانیدند و یهودا را امر فرمود که یهوه خدای اسرائیل را عبادت نمایند.
\par 17 اما قوم هنوزدر مکانهای بلند قربانی می‌گذرانیدند لیکن فقطبرای یهوه خدای خود.
\par 18 و بقیه وقایع منسی و دعایی که نزد خدای خود کرد و سخنان رائیانی که به اسم یهوه خدای اسرائیل به او گفتند، اینک در تواریخ پادشاهان اسرائیل مکتوب است.
\par 19 و دعای او ومستجاب شدنش و تمامی گناه و خیانتش وجایهایی که مکانهای بلند در آنها ساخت واشیره‌ها و بتهایی که قبل از متواضع شدنش برپانمود، اینک در اخبار حوزای مکتوب است.
\par 20 پس منسی با پدران خود خوابید و او را در خانه خودش دفن کردند و پسرش آمون در جایش پادشاه شد.
\par 21 آمون بیست و دو ساله بود که پادشاه شد ودو سال در اورشلیم پادشاهی کرد.
\par 22 و آنچه درنظر خداوند ناپسند بود، موافق آنچه پدرش منسی کرده بود، به عمل آورد و آمون برای جمیع بتهایی که پدرش منسی ساخته بود، قربانی گذرانیده، آنها را پرستش کرد.
\par 23 و به حضورخداوند تواضع ننمود، چنانکه پدرش منسی تواضع نموده بود بلکه این آمون زیاده و زیاده عصیان ورزید.
\par 24 پس خادمانش بر او شوریده، او را در خانه خودش کشتند.و اهل زمین همه کسانی را که بر آمون پادشاه شوریده بودند، به قتل رسانیدند و اهل زمین پسرش یوشیا را در جایش به پادشاهی نصب کردند.
\par 25 و اهل زمین همه کسانی را که بر آمون پادشاه شوریده بودند، به قتل رسانیدند و اهل زمین پسرش یوشیا را در جایش به پادشاهی نصب کردند.
 
\chapter{34}

\par 1 یوشیا هشت ساله بود که پادشاه شد ودر اورشلیم سی و یکسال سلطنت نمود.
\par 2 و آنچه در نظر خداوند پسند بود، به عمل آورد و به طریق های پدر خود داود سلوک نموده، به طرف راست یا چپ انحراف نورزید.
\par 3 و درسال هشتم سلطنت خود، حینی که هنوز جوان بود، به طلبیدن خدای پدر خود داود شروع کرد ودر سال دوازدهم به طاهر ساختن یهودا واورشلیم از مکان های بلند و اشیره‌ها و تمثالها وبتها آغاز نمود.
\par 4 و مذبح های بعلیم را به حضوروی منهدم ساختند، و تماثیل شمس را که بر آنهابود قطع نمود، و اشیره‌ها و تمثالها و بتهای ریخته شده را شکست، و آنها را خرد کرده، برروی قبرهای آنانی که برای آنها قربانی می‌گذرانیدند، پاشید.
\par 5 و استخوانهای کاهنان رابر مذبح های خودشان سوزانید. پس یهودا واورشلیم را طاهر نمود.
\par 6 و در شهرهای منسی وافرایم و شمعون حتی نفتالی نیز در خرابه هایی که به هر طرف آنها بود (همچنین کرد).
\par 7 و مذبح هارا منهدم ساخت و اشیره‌ها و تمثالها را کوبیده، نرم کرد و همه تمثالهای شمس را در تمامی زمین اسرائیل قطع نموده، به اورشلیم مراجعت کرد.
\par 8 و در سال هجدهم سلطنت خود، بعد از آنکه زمین و خانه را طاهر ساخته بود، شافان بن اصلیا ومعسیا رئیس شهر و یوآخ بن یوآحاز وقایع نگاررا برای تعمیر خانه یهوه خدای خود فرستاد.
\par 9 ونزد حلقیای رئیس کهنه آمدند و نقره‌ای را که به خانه خدا درآورده شده، و لاویان مستحفظان آستانه، آن را از دست منسی و افرایم و تمامی بقیه اسرائیل و تمامی یهودا و بنیامین و ساکنان اورشلیم جمع کرده بودند، به او تسلیم نمودند.
\par 10 و آن را به‌دست سرکارانی که بر خانه خداوندگماشته شده بودند، سپردند تا آن را به عمله هایی که در خانه خداوند کار می‌کردند، به جهت اصلاح و تعمیر خانه بدهند.
\par 11 پس آن را به نجاران و بنایان دادند تا سنگهای تراشیده وچوب به جهت اردیها و تیرها برای خانه هایی که پادشاهان یهودا آنها را خراب کرده بودند، بخرند.
\par 12 و آن مردان، کار را به امانت بجا می‌آوردند، وسرکاران ایشان که نظارت می‌کردند، یحت وعوبدیای لاویان از بنی مراری و زکریا و مشلام ازبنی قهاتیان بودند، و نیز از لاویان هر‌که به آلات موسیقی ماهر بود.
\par 13 و ایشان ناظران حمالان ووکلاء بر همه آنانی که در هر گونه‌ای خدمت، اشتغال داشتند بودند، و از لاویان کاتبان وسرداران و دربانان بودند.
\par 14 و چون نقره‌ای را که به خانه خداوند آورده شده بود، بیرون می‌بردند، حلقیای کاهن، کتاب تورات خداوند را که به واسطه موسی (نازل شده )بود پیدا کرد.
\par 15 و حلقیا شافان کاتب را خطاب کرده، گفت: «کتاب تورات را در خانه خداوندیافته‌ام.» و حلقیا آن کتاب را به شافان داد.
\par 16 وشافان آن کتاب را نزد پادشاه برد و نیز به پادشاه خبر رسانیده، گفت: «هر‌آنچه به‌دست بندگانت سپرده شده است آن را بجا می‌آورند.»
\par 17 ونقره‌ای را که در خانه خداوند یافت شد، بیرون آوردند و آن را به‌دست سرکاران و به‌دست عمله‌ها دادند.
\par 18 و شافان کاتب پادشاه را خبرداده، گفت: «حلقیای کاهن کتابی به من داده است.» پس شافان آن را به حضور پادشاه خواند.
\par 19 و چون پادشاه سخنان تورات را شنید، لباس خود را درید.
\par 20 و پادشاه، حلقیای کاهن واخیقام بن شافان و عبدون بن میکا و شافان کاتب وعسایا خادم پادشاه را امر فرموده، گفت:
\par 21 «بروید و از خداوند برای من و برای بقیه اسرائیل و یهودا درباره سخنانی که در این کتاب یافت می‌شود، مسئلت نمایید زیرا غضب خداوند که بر ما ریخته شده است، عظیم می‌باشدچونکه پدران ما کلام خداوند را نگاه نداشتند و به هر‌آنچه در این کتاب مکتوب است عمل ننمودند.»
\par 22 پس حلقیا و آنانی که پادشاه ایشان را امر فرمود، نزد حلده نبیه زن شلوم بن توقهه بن حسره لباس دار رفتند، و او در محله دوم اورشلیم ساکن بود و او را بدین مضمون سخن‌گفتند.
\par 23 واو به ایشان گفت: «یهوه خدای اسرائیل چنین می‌فرماید: به کسی‌که شما را نزد من فرستاده است بگویید:
\par 24 خداوند چنین می‌فرماید: اینک من بلایی بر این مکان و ساکنانش خواهم رسانید، یعنی همه لعنتهایی که در این کتاب که آن را به حضور پادشاه یهودا خواندند، مکتوب است.
\par 25 چونکه مرا ترک کرده، برای خدایان دیگربخور‌سوزانیدند تا به تمامی اعمال دستهای خودخشم مرا به هیجان بیاورند، پس غضب من بر این مکان افروخته شده، خاموشی نخواهد پذیرفت.
\par 26 لیکن به پادشاه یهودا که شما را به جهت مسئلت نمودن از خداوند فرستاده است، بگویید: یهوه خدای اسرائیل چنین می‌فرماید: درباره سخنانی که شنیده‌ای،
\par 27 چونکه دل تو نرم بود وهنگامی که کلام خداوند را درباره این مکان وساکنانش شنیدی، در حضور وی تواضع نمودی و به حضور من متواضع شده، لباس خود رادریدی و به حضور من گریستی، بنابراین خداوندمی گوید: من نیز تو را اجابت فرمودم.
\par 28 اینک من تو را نزد پدرانت جمع خواهم کرد و در قبر خودبه سلامتی گذارده خواهی شد، و چشمان توتمامی بلا را که من بر این مکان و ساکنانش می رسانم نخواهد دید.» پس ایشان نزد پادشاه جواب آوردند.
\par 29 و پادشاه فرستاد که تمامی مشایخ یهودا واورشلیم را جمع کردند.
\par 30 و پادشاه و تمامی مردان یهودا و ساکنان اورشلیم و کاهنان و لاویان و تمامی قوم، چه کوچک و چه بزرگ، به خانه خداوند برآمدند و او همه سخنان کتاب عهدی راکه در خانه خداوند یافت شد، در گوش ایشان خواند.
\par 31 و پادشاه بر منبر خود ایستاد و به حضور خداوند عهد بست که خداوند را پیروی نموده، اوامر و شهادات و فرایض او را به تمامی دل و تمامی جان نگاه دارند و سخنان این عهد راکه در این کتاب مکتوب است، بجا آورند.
\par 32 وهمه آنانی را که در اورشلیم و بنیامین حاضربودند، بر این متمکن ساخت و ساکنان اورشلیم، برحسب عهد خدا یعنی خدای پدران خود، عمل نمودند.و یوشیا همه رجاسات را ازتمامی زمینهایی که از آن بنی‌اسرائیل بودبرداشت، و همه کسانی را که در اسرائیل یافت شدند، تحریض نمود که یهوه خدای خود راعبادت نمایند و ایشان در تمامی ایام او از متابعت یهوه خدای پدران خود انحراف نورزیدند.
\par 33 و یوشیا همه رجاسات را ازتمامی زمینهایی که از آن بنی‌اسرائیل بودبرداشت، و همه کسانی را که در اسرائیل یافت شدند، تحریض نمود که یهوه خدای خود راعبادت نمایند و ایشان در تمامی ایام او از متابعت یهوه خدای پدران خود انحراف نورزیدند. 
 
\chapter{35}

\par 1 و یوشیا عید فصحی در اورشلیم برای خداوند نگاه داشت، و فصح را درچهاردهم ماه اول در اورشلیم ذبح نمودند.
\par 2 وکاهنان را بر وظایف ایشان قرار داده، ایشان رابرای خدمت خانه خداوند قوی‌دل ساخت.
\par 3 وبه لاویانی که تمامی اسرائیل را تعلیم می‌دادند وبرای خداوند تقدیس شده بودند، گفت: «تابوت مقدس را در خانه‌ای که سلیمان بن داود، پادشاه اسرائیل بنا کرده است بگذارید. و دیگر بر دوش شما بار نباشد. الان به خدمت یهوه خدای خود وبه قوم او اسرائیل بپردازید.
\par 4 و خویشتن رابرحسب خاندانهای آبای خود و فرقه های خویش بر وفق نوشته داود، پادشاه اسرائیل ونوشته پسرش سلیمان مستعد سازید.
\par 5 وبرحسب فرقه های خاندانهای آبای برادران خویش یعنی بنی قوم و موافق فرقه های خاندانهای آبای لاویان در قدس بایستید.
\par 6 وفصح را ذبح نمایید و خویشتن را تقدیس نموده، برای برادران خود تدارک بینید تا برحسب کلامی که خداوند به واسطه موسی گفته است عمل نمایند.»
\par 7 پس یوشیا به بنی قوم یعنی به همه آنانی که حاضر بودند، از گله بره‌ها و بزغاله‌ها به قدر سی هزار راس، همه آنها را به جهت قربانی های فصح داد و از گاوان سه هزار راس که همه اینها از اموال خاص پادشاه بود.
\par 8 و سروران او به قوم و به کاهنان و به لاویان هدایای تبرعی دادند. و حلقیاو زکریا و یحیئیل که روسای خانه خدا بودند، دوهزار و ششصد بره و سیصد گاو به جهت قربانی های فصح دادند.
\par 9 و کوننیا و شمعیا ونتنیئیل برادرانش و حشبا و یعیئیل و یوزاباد که روسای لاویان بودند، پنج هزار بره و پانصد گاو به لاویان به جهت قربانی های فصح دادند.
\par 10 پس آن خدمت مهیا شد و کاهنان درجایهای خود و لاویان در فرقه های خویش، برحسب فرمان پادشاه ایستادند.
\par 11 و فصح را ذبح کردند و کاهنان خون را از دست ایشان (گرفته )پاشیدند و لاویان پوست آنها را کندند.
\par 12 و قربانی های سوختنی را برداشتند تا آنها رابرحسب فرقه های خاندانهای آبا به پسران قوم بدهند تا ایشان آنها را برحسب آنچه در کتاب موسی نوشته بود، برای خداوند بگذرانند و باگاوان نیز چنین عمل نمودند.
\par 13 و فصح را موافق رسم به آتش پختند و هدایای مقدس را در دیگهاو پاتیلها و تابه‌ها پخته، آنها را به تمامی پسران قوم به زودی دادند.
\par 14 و بعد از آن برای خودشان و برای کاهنان مهیا ساختند زیرا که پسران هارون کهنه در گذرانیدن قربانی های سوختنی و پیه تاشام مشغول بودند. لهذا لاویان برای خودشان وبرای پسران هارون کهنه مهیا ساختند.
\par 15 ومغنیان از بنی آساف برحسب فرمان داود و آساف و هیمان و یدوتون که رایی پادشاه بود، به‌جای خود ایستادند و دربانان نزد هر دروازه و برای ایشان لازم نبود که از خدمت خود دور شوند زیراکه برادران ایشان لاویان به جهت ایشان مهیاساختند.
\par 16 پس تمامی خدمت خداوند در همان روز آماده شد تا فصح را نگاه دارند و قربانی های سوختنی را بر مذبح خداوند برحسب فرمان یوشیا پادشاه بگذرانند.
\par 17 پس بنی‌اسرائیل که حاضر بودند، در همان وقت، فصح و عید فطیر راهفت روز نگاه داشتند.
\par 18 و هیچ عید فصح مثل این از ایام سموئیل نبی در اسرائیل نگاه داشته نشده بود، و هیچ کدام از پادشاهان اسرائیل مثل این عید فصحی که یوشیا و کاهنان و لاویان وتمامی حاضران یهودا و اسرائیل و سکنه اورشلیم نگاه داشتند، نگاه نداشته بود.
\par 19 واین فصح در سال هجدهم سلطنت یوشیا واقع شد.
\par 20 بعد از همه این امور چون یوشیا هیکل راآماده کرده بود، نکو پادشاه مصر برآمد تا باکرکمیش نزد نهر فرات جنگ کند. و یوشیا به مقابله او بیرون رفت.
\par 21 و (نکو) قاصدان نزد اوفرستاده، گفت: «ای پادشاه یهودا مرا با تو چه‌کار است؟ من امروز به ضد تو نیامده‌ام بلکه به ضد خاندانی که با آن محاربه می‌نمایم. و خدامرا امر فرموده است که بشتابم. پس از آن خدایی که با من است، دست بردار مبادا تو را هلاک سازد.»
\par 22 لیکن یوشیا روی خود را از او برنگردانیدبلکه خویشتن را متنکر ساخت تا با وی جنگ کندو به کلام نکو که از جانب خدا بود گوش نگرفته، به قصد مقاتله به میدان مجدو درآمد.
\par 23 وتیراندازان بر یوشیا پادشاه تیر انداختند و پادشاه به خادمان خود گفت: «مرا بیرون برید زیرا که سخت مجروح شده‌ام.»
\par 24 پس خادمانش او را ازارابه‌اش گرفتند و بر ارابه دومین که داشت سوارکرده، به اورشلیم آوردند. پس وفات یافته، درمقبره پدران خود دفن شد، و تمامی یهودا واورشلیم برای یوشیا ماتم گرفتند.
\par 25 و ارمیا به جهت یوشیا مرثیه خواند و تمامی مغنیان ومغنیات یوشیا را در مراثی خویش تا امروز ذکرمی کنند و آن را فریضه‌ای در اسرائیل قرار دادند، چنانکه در سفر مراثی مکتوب است.
\par 26 و بقیه وقایع یوشیا و اعمال حسنه‌ای که مطابق نوشته تورات خداوند به عمل آورد،و امور اول وآخر او اینک در تواریخ پادشاهان اسرائیل ویهودا مکتوب است.
\par 27 و امور اول وآخر او اینک در تواریخ پادشاهان اسرائیل ویهودا مکتوب است.
 
\chapter{36}

\par 1 و قوم زمین، یهوآحاز بن یوشیا راگرفته، او را در جای پدرش در اورشلیم به پادشاهی نصب نمودند.
\par 2 یهوآحاز بیست و سه ساله بود که پادشاه شد و در اورشلیم سه ماه سلطنت نمود.
\par 3 و پادشاه مصر، او را در اورشلیم معزول نمود و زمین را به صد وزنه نقره و یک وزنه طلا جریمه کرد.
\par 4 و پادشاه مصر، برادرش الیاقیم را بر یهودا و اورشلیم پادشاه ساخت، و اسم او رابه یهویاقیم تبدیل نمود، و نکو برادرش یهوآحازرا گرفته، به مصر برد.
\par 5 یهویاقیم بیست و پنج ساله بود که پادشاه شد و یازده سال در اورشلیم سلطنت نمود، و درنظر یهوه خدای خود شرارت ورزید.
\par 6 ونبوکدنصر پادشاه بابل به ضد او برآمد و او را به زنجیرها بست تا او را به بابل ببرد.
\par 7 و نبوکدنصربعضی از ظروف خانه خداوند را به بابل آورده، آنها را در قصر خود در بابل گذاشت.
\par 8 و بقیه وقایع یهویاقیم و رجاساتی که به عمل آورد وآنچه در او یافت شد، اینک در تواریخ پادشاهان اسرائیل و یهودا مکتوب است، و پسرش یهویاکین در جایش پادشاهی کرد.
\par 9 یهویاکین هشت ساله بود که پادشاه شد و سه ماه و ده روز در اورشلیم سلطنت نمود و آنچه درنظر خداوند ناپسند بود، به عمل آورد.
\par 10 و در وقت تحویل سال، نبوکدنصر پادشاه فرستاد و اورا با ظروف گرانبهای خانه خداوند به بابل آورد، و برادرش صدقیا را بر یهودا و اورشلیم پادشاه ساخت.
\par 11 صدقیا بیست و یکساله بود که پادشاه شد ویازده سال در اورشلیم سلطنت نمود.
\par 12 و در نظریهوه خدای خود شرارت ورزیده، در حضورارمیای نبی که از زبان خداوند به او سخن گفت، تواضع ننمود.
\par 13 و نیز بر نبوکدنصر پادشاه که اورا به خدا قسم داده بود عاصی شد و گردن خود راقوی و دل خویش را سخت ساخته، به سوی یهوه خدای اسرائیل بازگشت ننمود.
\par 14 و تمامی روسای کهنه و قوم، خیانت بسیاری موافق همه رجاسات امت‌ها ورزیدند و خانه خداوند را که آن را در اورشلیم تقدیس نموده بود، نجس ساختند.
\par 15 و یهوه خدای پدر ایشان به‌دست رسولان خویش نزد ایشان فرستاد، بلکه صبح زودبرخاسته، ایشان را ارسال نمود زیرا که بر قوم خود و بر مسکن خویش شفقت نمود.
\par 16 اماایشان رسولان خدا را اهانت نمودند و کلام او راخوار شمرده، انبیایش را استهزا نمودند، چنانکه غضب خداوند بر قوم او افروخته شد، به حدی که علاجی نبود.
\par 17 پس پادشاه کلدانیان را که جوانان ایشان رادر خانه مقدس ایشان به شمشیر کشت و برجوانان و دوشیزگان و پیران و ریش سفیدان ترحم ننمود، بر ایشان آورد و همه را به‌دست او تسلیم کرد.
\par 18 و او سایر ظروف خانه خدا را از بزرگ وکوچک و خزانه های خانه خداوند و گنجهای پادشاه و سرورانش را تمام به بابل برد.
\par 19 و خانه خدا را سوزانیدند و حصار اورشلیم را منهدم ساختند و همه قصرهایش را به آتش سوزانیدندو جمیع آلات نفیسه آنها را ضایع کردند.
\par 20 وبقیه السیف را به بابل به اسیری برد که ایشان تازمان سلطنت پادشاهان فارس او را و پسرانش رابنده بودند.
\par 21 تا کلام خداوند به زبان ارمیا کامل شود و زمین از سبت های خود تمتع برد زیرا درتمامی ایامی که ویران ماند آرامی یافت، تا هفتادسال سپری شد.و در سال اول کورش، پادشاه فارس، تاکلام خداوند به زبان ارمیا کامل شود، خداوندروح کورش، پادشاه فارس را برانگیخت تا درتمامی ممالک خود فرمانی نافذ کرد و آن را نیزمرقوم داشت و گفت:
\par 22 و در سال اول کورش، پادشاه فارس، تاکلام خداوند به زبان ارمیا کامل شود، خداوندروح کورش، پادشاه فارس را برانگیخت تا درتمامی ممالک خود فرمانی نافذ کرد و آن را نیزمرقوم داشت و گفت:


\end{document}