\begin{document}

\title{عزرا}

 
\chapter{1}

\par 1 و در سال اول کورش، پادشاه فارس تا کلام خداوند به زبان ارمیا کامل شود، خداوندروح کورش پادشاه فارس را برانگیخت تا درتمامی ممالک خود فرمانی نافذ کرد و آن را نیزمرقوم داشت و گفت:
\par 2 «کورش پادشاه فارس چنین می‌فرماید: یهوه خدای آسمانها جمیع ممالک زمین را به من داده و مرا امر فرموده است که خانه‌ای برای وی در اورشلیم که در یهودااست بنا نمایم.
\par 3 پس کیست از شما از تمامی قوم او که خدایش با وی باشد؟ او به اورشلیم که دریهودا است، برود و خانه یهوه را که خدای اسرائیل و خدای حقیقی است، در اورشلیم بنانماید.
\par 4 و هر‌که باقی‌مانده باشد، در هر مکانی ازمکان هایی که در آنها غریب می‌باشد، اهل آن مکان او را به نقره و طلا و اموال و چهارپایان علاوه بر هدایای تبرعی به جهت خانه خدا که دراورشلیم است اعانت نمایند.»
\par 5 پس روسای آبای یهودا و بنیامین و کاهنان ولاویان با همه کسانی که خدا روح ایشان رابرانگیزانیده بود برخاسته، روانه شدند تا خانه خداوند را که در اورشلیم است بنا نمایند.
\par 6 وجمیع همسایگان ایشان، ایشان را به آلات نقره وطلا و اموال و چهارپایان و تحفه‌ها، علاوه بر همه هدایای تبرعی اعانت کردند.
\par 7 و کورش پادشاه ظروف خانه خداوند را که نبوکدنصر آنها را از اورشلیم آورده و در خانه خدایان خود گذاشته بود، بیرون آورد.
\par 8 و کورش پادشاه فارس، آنها رااز دست متردات، خزانه‌دار خود بیرون آورده، به شیشبصر رئیس یهودیان شمرد.
\par 9 و عدد آنها این است: سی طاس طلا و هزار طاس نقره و بیست ونه کارد،
\par 10 و سی جام طلا و چهارصد و ده جام نقره از قسم دوم و هزار ظرف دیگر.تمامی ظروف طلا و نقره پنجهزار و چهارصد بود وشیشبصر همه آنها را با اسیرانی که از بابل به اورشلیم می‌رفتند برد.
\par 11 تمامی ظروف طلا و نقره پنجهزار و چهارصد بود وشیشبصر همه آنها را با اسیرانی که از بابل به اورشلیم می‌رفتند برد.
 
\chapter{2}

\par 1 و اینانند اهل ولایتها که از اسیری آن اشخاصی که نبوکدنصر، پادشاه بابل، به بابل به اسیری برده بود برآمدند و هر کدام ازایشان به اورشلیم و یهودا و شهر خود برگشتند.
\par 2 اما آنانی که همراه زربابل آمدند، یشوع و نحمیاو سرایا و رعیلایا و مردخای و بلشان و مسفار وبغوای و رحوم و بعنه. و شماره مردان قوم اسرائیل:
\par 3 بنی فرعوش دو هزار و یکصد و هفتاد و دو.
\par 4 بنی شفطیا سیصد و هفتاد و دو.
\par 5 بنی آرح هفتصد و هفتاد و پنج.
\par 6 بنی فحت موآب ازبنی یشوع و یوآب دو هزار و هشتصد و دوازده.
\par 7 بنی عیلام هزار و دویست و پنجاه و چهار.
\par 8 بنی زتونه صد و چهل و پنج.
\par 9 بنی زکای هفتصد و شصت.
\par 10 بنی بانی ششصد و چهل و دو.
\par 11 بنی بابای ششصد و بیست و سه.
\par 12 بنی ازجدهزار و دویست و بیست و دو.
\par 13 بنی ادونیقام ششصد و شصت و شش.
\par 14 بنی بغوای دو هزار وپنجاه و شش.
\par 15 بنی عادین چهارصد و پنجاه وچهار.
\par 16 بنی آطیر (از خاندان ) یحزقیا نود وهشت.
\par 17 بنی بیصای سیصد و بیست و سه.
\par 18 بنی یوره صد و دوازده.
\par 19 بنی حاشوم دویست و بیست و سه.
\par 20 بنی جبار نود و پنج.
\par 21 بنی بیت لحم صد و بیست و سه.
\par 22 مردان نطوفه پنجاه و شش.
\par 23 مردان عناتوت صد وبیست و هشت.
\par 24 بنی عزموت چهل و دو.
\par 25 بنی قریه عاریم و کفیره و بئیروت هفتصد وچهل و سه.
\par 26 بنی رامه و جبع ششصد و بیست ویک.
\par 27 مردان مکماس صد و بیست و دو.
\par 28 مردان بیت ئیل و عای دویست و بیست و سه.
\par 29 بنی نبو پنجاه و دو.
\par 30 بنی مغبیش صد و پنجاه و شش.
\par 31 بنی عیلام دیگر، هزار و دویست وپنجاه چهار.
\par 32 بنی حاریم سیصد و بیست.
\par 33 بنی لود و حادید و ارنو هفتصد و بیست و پنج.
\par 34 بنی اریحا سیصد و چهل و پنج.
\par 35 بنی سنائه سه هزار و ششصد و سی.
\par 36 و اما کاهنان: بنی یدعیا از خاندان یشوع نه صد و هفتاد و سه.
\par 37 بنی امیر هزار و پنجاه و دو.
\par 38 بنی فشحور هزار و دویست و چهل و هفت.
\par 39 بنی حاریم هزار و هفده.
\par 40 و اما لاویان: بنی یشوع و قدمیئیل از نسل هودویا هفتاد و چهار.
\par 41 و مغنیان: بنی آساف صد و بیست و هشت.
\par 42 و پسران دربانان: بنی شلوم و بنی آطیر و بنی طلمون و بنی عقوب و بنی حطیطا وبنی شوبای جمیع اینها صد و سی و نه.
\par 43 و امانتینیم: بنی صیحا و بنی حسوفا و بنی طباعوت،
\par 44 و بنی قیروس و بنی سیعها و بنی فادوم،
\par 45 وبنی لبانه و بنی حجابه و بنی عقوب،
\par 46 وبنی حاجاب و بنی شملای و بنی حانان،
\par 47 وبنی جدیل و بنی جحر و بنی رآیا،
\par 48 و بنی رصین و بنی نقودا و بنی جزام،
\par 49 و بنی عزه و بنی فاسیح و بنی بیسای،
\par 50 و بنی اسنه و بنی معونیم وبنی نفوسیم،
\par 51 و بنی بقبوق و بنی حقوفا وبنی حرحور،
\par 52 و بنی بصلوت و بنی محیدا وبنی حرشا،
\par 53 و بنی برقوس و بنی سیسرا وبنی تامح،
\par 54 و بنی نصیح و بنی حطیفا.
\par 55 و پسران خادمان سلیمان: بنی سوطای وبنی هصوفرت و بنی فرودا،
\par 56 و بنی یعله وبنی درقون و بنی جدیل،
\par 57 و بنی شفطیا وبنی حطیل و بنی فوخره ظبائیم و بنی آمی.
\par 58 جمیع نتینیم و پسران خادمان سلیمان سیصد ونود و دو.
\par 59 و اینانند آنانی که از تل ملح و تل حرشابرآمدند یعنی کروب و ادان و امیر، اما خاندان پدران و عشیره خود را نشان نتوانستند داد که آیااز اسرائیلیان بودند یا نه.
\par 60 بنی دلایا و بنی طوبیا وبنی نقودا ششصد و پنجاه و دو.
\par 61 و از پسران کاهنان، بنی حبایا و بنی هقوص و بنی برزلای که یکی از دختران برزلایی جلعادی را به زنی گرفت، پس به نام ایشان مسمی شدند.
\par 62 اینان انساب خود را در میان آنانی که در نسب نامه هاثبت شده بودند طلبیدند، اما نیافتند، پس ازکهانت اخراج شدند.
\par 63 پس ترشاتا به ایشان امر فرمود که تا کاهنی با اوریم و تمیم برقرار نشودایشان از قدس اقداس نخورند.
\par 64 تمامی جماعت، با هم چهل و دو هزار و سیصد و شصت نفر بودند.
\par 65 سوای غلامان و کنیزان ایشان، که هفتهزار و سیصد و سی و هفت نفر بودند، ومغنیان و مغنیاه ایشان دویست نفر بودند.
\par 66 واسبان ایشان هفتصد و سی و شش، و قاطران ایشان دویست و چهل و پنج.
\par 67 و شتران ایشان چهارصد و سی و پنج و حماران ایشان ششهزار وهفتصد و بیست.
\par 68 و چون ایشان به خانه خداوند که دراورشلیم است رسیدند، بعضی از روسای آبا، هدایای تبرعی به جهت خانه خدا آوردند تا آن رادر جایش برپا نمایند.
\par 69 برحسب قوه خود، شصت و یک هزار درهم طلا و پنج هزار منای نقره و صد (دست ) لباس کهانت به خزانه به جهت کار دادند.پس کاهنان و لاویان و بعضی از قوم و مغنیان و دربانان و نتینیم در شهرهای خودساکن شدند و تمامی اسرائیل در شهرهای خودمسکن گرفتند.
\par 70 پس کاهنان و لاویان و بعضی از قوم و مغنیان و دربانان و نتینیم در شهرهای خودساکن شدند و تمامی اسرائیل در شهرهای خودمسکن گرفتند.
 
\chapter{3}

\par 1 و چون ماه هفتم رسید بنی‌اسرائیل درشهرهای خود مقیم بودند و تمامی قوم مثل یک مرد در اورشلیم جمع شدند.
\par 2 و یشوع بن یوصاداق و برادرانش که کاهنان بودند و زربابل بن شالتیئیل با برادران خود برخاستند و مذبح خدای اسرائیل را برپا کردند تا قربانی های سوختنی برحسب آنچه در تورات موسی، مرد خدا مکتوب است بر آن بگذرانند.
\par 3 پس مذبح را برجایش برپا کردند زیرا که به‌سبب قوم زمین، ترس بر ایشان مستولی می‌بود وقربانی های سوختنی برای خداوند یعنی قربانی های سوختنی، صبح و شام را بر آن گذرانیدند.
\par 4 و عید خیمه‌ها را به نحوی که مکتوب است نگاه داشتند و قربانی های سوختنی روز به روز، معتاد هر روز را در روزش، برحسب رسم و قانون گذرانیدند.
\par 5 و بعد از آن، قربانی های سوختنی دائمی را در غره های ماه و در همه مواسم مقدس خداوند و برای هر کس که هدایای تبرعی به جهت خداوند می‌آورد، می‌گذرانیدند.
\par 6 از روز اول ماه هفتم، حینی که بنیاد هیکل خداوند هنوز نهاده نشده بود، به گذرانیدن قربانی های سوختنی برای خداوند شروع کردند.
\par 7 و به حجاران و نجاران نقره دادند و به اهل صیدون و صور ماکولات و مشروبات و روغن (دادند) تا چوب سرو آزاد از لبنان از دریا به یافا، برحسب امری که کورش پادشاه فارس، به ایشان داده بود بیاورند.
\par 8 و در ماه دوم از سال دوم، بعد از رسیدن ایشان به خانه خدا در اورشلیم، زربابل بن شالتیئیل و یشوع بن یوصاداق و سایر برادران ایشان از کاهنان و لاویان و همه کسانی که ازاسیری به اورشلیم برگشته بودند، به نصب نمودن لاویان از بیست ساله و بالاتر بر نظارت عمل خانه خداوند شروع کردند.
\par 9 و یشوع با پسران وبرادران خود و قدمیئیل با پسرانش از بنی یهودا باهم ایستادند تا بر بنی حیناداد و پسران و برادران ایشان که از لاویان در کار خانه خدا مشغول می‌بودند، نظارت نمایند.
\par 10 و چون بنایان بنیاد هیکل خداوند را نهادند، کاهنان را با لباس خودشان با کرناها و لاویان بنی آساف را با سنجها قرار دادند تا خداوند رابرحسب رسم داود پادشاه اسرائیل، تسبیح بخوانند.
\par 11 و بر یکدیگر می‌سراییدند و خداوندرا تسبیح و حمد می‌گفتند، که «او نیکوست زیراکه رحمت او بر اسرائیل تا ابدالاباد است» وتمامی قوم به آواز بلند صدا زده، خداوند را به‌سبب بنیاد نهادن خانه خداوند، تسبیح می‌خواندند.
\par 12 و بسیاری از کاهنان و لاویان و روسای آباکه پیر بودند و خانه اولین را دیده بودند، حینی که بنیاد این خانه در نظر ایشان نهاده شد، به آواز بلندگریستند و بسیاری با آواز شادمانی صداهای خود را بلند کردند.چنانکه مردم نتوانستند درمیان صدای آواز شادمانی و آواز گریستن قوم تشخیص نمایند زیرا که خلق، صدای بسیار بلندمی دادند چنانکه آواز ایشان از دور شنیده می‌شد.
\par 13 چنانکه مردم نتوانستند درمیان صدای آواز شادمانی و آواز گریستن قوم تشخیص نمایند زیرا که خلق، صدای بسیار بلندمی دادند چنانکه آواز ایشان از دور شنیده می‌شد.
 
\chapter{4}

\par 1 و چون دشمنان یهودا و بنیامین شنیدند که اسیران، هیکل یهوه خدای اسرائیل را بنامی کنند،
\par 2 آنگاه نزد زربابل و روسای آبا آمده، به ایشان گفتند که «ما همراه شما بنا خواهیم کردزیرا که ما مثل شما از زمان اسرحدون، پادشاه آشور که ما را به اینجا آورد، خدای شما رامی طلبیم و برای او قربانی می‌گذرانیم.»
\par 3 اما زربابل و یشوع و سایر روسای آبای اسرائیل به ایشان گفتند: «شما را با ما در بنا کردن خانه خدای ما کاری نیست، بلکه ما تنها آن رابرای یهوه، خدای اسرائیل چنانکه کورش پادشاه، سلطان فارس به ما امر فرموده است، آن رابنا خواهیم نمود.»
\par 4 آنگاه اهل زمین دستهای قوم یهودا را سست کردند و ایشان را در بنا نمودن به تنگ می‌آوردند،
\par 5 و به ضد ایشان مدبران اجیر ساختند که در تمام ایام کورش پادشاه فارس، تا سلطنت داریوش، پادشاه فارس قصد ایشان را باطل ساختند.
\par 6 وچون اخشورش پادشاه شد، در ابتدای سلطنتش بر ساکنان یهودا و اورشلیم شکایت نوشتند.
\par 7 ودر ایام ارتحشستا، بشلام و متردات و طبئیل وسایر رفقای ایشان به ارتحشستا پادشاه فارس نوشتند و مکتوب به خط آرامی نوشته شد ومعنی‌اش در زبان ارامی.
\par 8 رحوم فرمان فرما وشمشائی کاتب رساله به ضد اورشلیم، به ارتحشستا پادشاه، بدین مضمون نوشتند:
\par 9 «پس رحوم فرمان فرما و شمشائی کاتب وسائر رفقای ایشان از دینیان و افرستکیان وطرفلیان و افرسیان و ارکیان و بابلیان و شوشنکیان و دهائیان و عیلامیان،
\par 10 و سایر امت هایی که اسنفر عظیم و شریف ایشان را کوچانیده، در شهرسامره ساکن گردانیده است و سایر ساکنان ماورای نهر و اما بعد.
\par 11 (این است سواد مکتوبی که ایشان نزد ارتحشستا پادشاه فرستادند. بندگانت که ساکنان ماورای نهر می‌باشیم و امابعد. )
\par 12 پادشاه را معلوم باد که یهودیانی که از جانب تو به نزد ما آمدند، به اورشلیم رسیده‌اند وآن شهر فتنه انگیز و بد را بنا می‌نمایند و حصارهارا برپا می‌کنند و بنیادها را مرمت می‌نمایند. 
\par 13 الان پادشاه را معلوم باد که اگر این شهر بنا شودو حصارهایش تمام گردد، جزیه و خراج و باج نخواهند داد و بالاخره به پادشاهان ضرر خواهدرسید.
\par 14 پس چونکه ما نمک خانه پادشاه رامی خوریم، ما را نشاید که ضرر پادشاه را ببینیم، لهذا فرستادیم تا پادشاه را اطلاع دهیم،
\par 15 تا درکتاب تواریخ پدرانت تفتیش کرده شود و از کتاب تواریخ دریافت نموده، بفهمی که این شهر، شهرفتنه انگیز است و ضرررساننده به پادشاهان وکشورها و در ایام قدیم در میانش فتنه می‌انگیختند. و از همین سبب این شهر خراب شد.
\par 16 بنابراین پادشاه را اطلاع می‌دهیم که اگراین شهر بنا شود و حصارهایش تمام گردد تو را به این طرف نهر نصیبی نخواهد بود.»
\par 17 پس پادشاه به رحوم فرمان فرما و شمشائی کاتب و سایر رفقای ایشان که در سامره ساکن بودند و سایر ساکنان ماورای نهر، جواب فرستادکه «سلامتی و اما بعد.
\par 18 مکتوبی که نزد مافرستادید، در حضور من واضح خوانده شد.
\par 19 وفرمانی از من صادر گشت و تفحص کرده، دریافت کردند که این شهر از ایام قدیم با پادشاهان مقاومت می‌نموده و فتنه و فساد در آن واقع می‌شده است.
\par 20 و پادشاهان قوی در اورشلیم بوده‌اند که بر تمامی ماورای نهر سلطنت می‌کردند و جزیه و خراج و باج به ایشان می‌دادند.
\par 21 پس فرمانی صادر کنید که آن مردان را از کار باز دارند و تا حکمی از من صادر نگردد این شهر بنا نشود.
\par 22 پس باحذر باشید که در این کار کوتاهی ننمایید زیرا که چرا این فساد برای ضرر پادشاهان پیش رود؟»
\par 23 پس چون نامه ارتحشستا پادشاه به حضوررحوم و شمشایی کاتب و رفقای ایشان خوانده شد، ایشان به تعجیل نزد یهودیان به اورشلیم رفتند و ایشان را با زور و جفا از کار باز داشتند.آنگاه کار خانه خدا که در اورشلیم است، تعویق افتاد و تا سال دوم سلطنت داریوش، پادشاه فارس معطل ماند.
\par 24 آنگاه کار خانه خدا که در اورشلیم است، تعویق افتاد و تا سال دوم سلطنت داریوش، پادشاه فارس معطل ماند.
 
\chapter{5}

\par 1 آنگاه دو نبی، یعنی حجی نبی و زکریا ابن عدو، برای یهودیانی که در یهودا واورشلیم بودند، به نام خدای اسرائیل که با ایشان می‌بود نبوت کردند.
\par 2 و در آن زمان زربابل بن شالتیئیل و یشوع بن یوصاداق برخاسته، به بنانمودن خانه خدا که در اورشلیم است شروع کردند و انبیای خدا همراه ایشان بوده، ایشان رامساعدت می‌نمودند.
\par 3 در آن وقت تتنائی، والی ماورای نهر و شتربوزنای و رفقای ایشان آمده، به ایشان چنین گفتند: «کیست که شما را امر فرموده است که این خانه را بنا نمایید و این حصار را برپاکنید؟»
\par 4 پس ایشان را بدین منوال از نامهای کسانی که این عمارت را بنا می‌کردند اطلاع دادیم.
\par 5 اما چشم خدای ایشان بر مشایخ یهودا بودکه ایشان را نتوانستند از کار بازدارند تا این امر به سمع داریوش برسد و جواب مکتوب درباره‌اش داده شود.
\par 6 سواد مکتوبی که تتنای، والی ماورای نهر و شتربوزنای و رفقای او افرسکیان که درماورای نهر ساکن بودند، نزد داریوش پادشاه فرستادند.
\par 7 مکتوب را نزد او فرستادند و در آن بدین مضمون مرقوم بود که «بر داریوش پادشاه سلامتی تمام باد.
\par 8 بر پادشاه معلوم باد که ما به بلاد یهودیان، به خانه خدای عظیم رفتیم و آن رااز سنگهای بزرگ بنا می‌کنند و چوبها در دیوارش می‌گذارند و این کار در دست ایشان به تعجیل، معمول و به انجام رسانیده می‌شود.
\par 9 آنگاه ازمشایخ ایشان پرسیده، چنین به ایشان گفتیم: کیست که شما را امر فرموده است که این خانه رابنا کنید و دیوارهایش را برپا نمایید؟
\par 10 و نیزنامهای ایشان را از ایشان پرسیدیم تا تو را اعلام نماییم و نامهای کسانی که روسای ایشانندنوشته‌ایم.
\par 11 و ایشان در جواب ما چنین گفتند که ما بندگان خدای آسمان و زمین هستیم و خانه‌ای را تعمیر می‌نماییم که چندین سال قبل از این بناشده و پادشاه بزرگ اسرائیل آن را ساخته و به انجام رسانیده بود.
\par 12 لیکن بعد از آن، پدران ماخشم خدای آسمان را به هیجان آوردند. پس اوایشان را به‌دست نبوکدنصر کلدانی، پادشاه بابل تسلیم نمود که این خانه را خراب کرد و قوم را به بابل به اسیری برد.
\par 13 اما در سال اول کورش پادشاه بابل، همین کورش پادشاه امر فرمود که این خانه خدا را بنا نمایند.
\par 14 و نیز ظروف طلا و نقره خانه خدا که نبوکدنصر آنها را از هیکل اورشلیم گرفته وبه هیکل بابل آورده بود، کورش پادشاه آنها را از هیکل بابل بیرون آورد و به شیشبصرنامی که او را والی ساخته بود، تسلیم نمود.
\par 15 واو را گفت که این ظروف را برداشته، برو و آنها را به هیکلی که در اورشلیم است ببر و خانه خدا درجایش بنا کرده شود.
\par 16 آنگاه این شیشبصر آمد وبنیاد خانه خدا را که در اورشلیم است نهاد و از آن زمان تا بحال بنا می‌شود و هنوز تمام نشده است.پس الان اگر پادشاه مصلحت داند، در خزانه پادشاه که در بابل است تفحص کنند که آیا چنین است یا نه که فرمانی از کورش پادشاه صادر شده بود که این خانه خدا در اورشلیم بنا شود و پادشاه مرضی خود را در این امر نزد ما بفرستد.»
\par 17 پس الان اگر پادشاه مصلحت داند، در خزانه پادشاه که در بابل است تفحص کنند که آیا چنین است یا نه که فرمانی از کورش پادشاه صادر شده بود که این خانه خدا در اورشلیم بنا شود و پادشاه مرضی خود را در این امر نزد ما بفرستد.»
 
\chapter{6}

\par 1 آنگاه داریوش پادشاه، فرمان داد تا در کتابخانه بابل که خزانه‌ها در آن موضوع بودتفحص کردند.
\par 2 و در قصر احمتا که در ولایت مادیان است، طوماری یافت شد و تذکره‌ای درآن بدین مضمون مکتوب بود:
\par 3 «در سال اول کورش پادشاه، همین کورش پادشاه درباره خانه خدا در اورشلیم فرمان داد که آن خانه‌ای که قربانی‌ها در آن می‌گذرانیدند، بنا شود و بنیادش تعمیر گردد و بلندی‌اش شصت ذراع و عرضش شصت ذراع باشد.
\par 4 با سه صف سنگهای بزرگ ویک صف چوب نو. و خرجش از خانه پادشاه داده شود.
\par 5 و نیز ظروف طلا و نقره خانه خدا راکه نبوکدنصر آنها را از هیکل اورشلیم گرفته، به بابل آورده بود پس بدهند و آنها را به‌جای خوددر هیکل اورشلیم باز برند و آنها را در خانه خدابگذارند.
\par 6 «پس حال‌ای تتنای، والی ماورای نهر وشتربوزنای و رفقای شما و افرسکیانی که به آنطرف نهر می‌باشید، از آنجا دور شوید.
\par 7 و به کار این خانه خدا متعرض نباشید. اما حاکم یهودو مشایخ یهودیان این خانه خدا را در جایش بنانمایند.
\par 8 و فرمانی نیز از من صادر شده است که شما با این مشایخ یهود به جهت بنا نمودن این خانه خدا چگونه رفتار نمایید. از مال خاص پادشاه، یعنی از مالیات ماورای نهر، خرج به این مردمان، بلا تاخیر داده شود تا معطل نباشند.
\par 9 ومایحتاج ایشان را از گاوان و قوچها و بره‌ها به جهت قربانی های سوختنی برای خدای آسمان وگندم و نمک و شراب و روغن، برحسب قول کاهنانی که در اورشلیم هستند، روز به روز به ایشان بی‌کم و زیاد داده شود.
\par 10 تا آنکه هدایای خوشبو برای خدای آسمان بگذرانند و به جهت عمر پادشاه و پسرانش دعا نمایند.
\par 11 و دیگرفرمانی از من صادر شد که هرکس که این حکم راتبدیل نماید، از خانه او تیری گرفته شود و او برآن آویخته و مصلوب گردد و خانه او به‌سبب این عمل مزبله بشود.
\par 12 و آن خدا که نام خود را درآنجا ساکن گردانیده است، هر پادشاه یا قوم را که دست خود را برای تبدیل این امر و خرابی این خانه خدا که در اورشلیم است دراز نماید، هلاک سازد. من داریوش این حکم را صادر فرمودم، پس این عمل بلا تاخیر کرده شود.»
\par 13 آنگاه تتنای، والی ماورای نهر و شتربوزنای و رفقای ایشان بروفق فرمانی که داریوش پادشاه فرستاده بود، بلاتاخیر عمل نمودند.
\par 14 و مشایخ یهود به بنا نمودن مشغول شدند و برحسب نبوت حجی نبی و زکریا ابن عدو کار را پیش بردند وبرحسب حکم خدای اسرائیل و فرمان کورش وداریوش و ارتحشستا، پادشاهان فارس آن را بنا نموده، به انجام رسانیدند.
\par 15 و این خانه، در روزسوم ماه اذار در سال ششم داریوش پادشاه، تمام شد.
\par 16 و بنی‌اسرائیل، یعنی کاهنان و لاویان وسایر آنانی که از اسیری برگشته بودند، این خانه خدا را با شادمانی تبریک نمودند.
\par 17 و برای تبریک این خانه خدا صد گاو و دویست قوچ وچهارصد بره و به جهت قربانی گناه برای تمامی اسرائیل، دوازده بز نر موافق شماره اسباطاسرائیل گذرانیدند.
\par 18 و کاهنان را درفرقه های ایشان و لاویان را در قسمتهای ایشان، بر خدمت خدایی که در اورشلیم است برحسب آنچه در کتاب موسی مکتوب است قراردادند.
\par 19 و آنانی که از اسیری برگشته بودند، عیدفصح را در روز چهاردهم ماه اول نگاه داشتند.
\par 20 زیرا که کاهنان و لاویان، جمیع خویشتن راطاهر ساختند و چون همه ایشان طاهر شدند، فصح را برای همه آنانی که از اسیری برگشته بودند و برای برادران خود کاهنان و برای خودشان ذبح کردند.
\par 21 و بنی‌اسرائیل که ازاسیری برگشته بودند با همه آنانی که خویشتن را از رجاسات امت های زمین جدا ساخته، به ایشان پیوسته بودند تا یهوه خدای اسرائیل را بطلبند، آن را خوردند.و عید فطیر راهفت روز با شادمانی نگاه داشتند، چونکه خداوند ایشان را مسرور ساخت از اینکه دل پادشاه آشور را به ایشان مایل گردانیده، دستهای ایشان را برای ساختن خانه خدای حقیقی که خدای اسرائیل باشد، قوی گردانید.
\par 22 و عید فطیر راهفت روز با شادمانی نگاه داشتند، چونکه خداوند ایشان را مسرور ساخت از اینکه دل پادشاه آشور را به ایشان مایل گردانیده، دستهای ایشان را برای ساختن خانه خدای حقیقی که خدای اسرائیل باشد، قوی گردانید.
 
\chapter{7}

\par 1 و بعد از این امور، در سلطنت ارتحشستاپادشاه فارس، عزرا ابن سرایا ابن عزریا ابن حلقیا،
\par 2 ابن شلوم بن صادوق بن اخیطوب،
\par 3 بن امریا ابن عزریا ابن مرایوت،
\par 4 بن زرحیا ابن عزی ابن بقی،
\par 5 ابن ابیشوع بن فینحاس بن العازار بن هارون رئیس کهنه،
\par 6 این عزرا از بابل برآمد و اودر شریعت موسی که یهوه خدای اسرائیل آن راداده بود، کاتب ماهر بود و پادشاه بروفق دست یهوه خدایش که با وی می‌بود، هر‌چه را که اومی خواست به وی می‌داد.
\par 7 و بعضی ازبنی‌اسرائیل و از کاهنان و لاویان و مغنیان ودربانان و نتینیم نیز در سال هفتم ارتحشستاپادشاه به اورشلیم برآمدند.
\par 8 و او در ماه پنجم سال هفتم پادشاه، به اورشلیم رسید.
\par 9 زیرا که درروز اول ماه اول، به بیرون رفتن از بابل شروع نمودو در روز اول ماه پنجم، بروفق دست نیکوی خدایش که با وی می‌بود، به اورشلیم رسید.
\par 10 چونکه عزرا دل خود را به طلب نمودن شریعت خداوند و به عمل آوردن آن و به تعلیم دادن فرایض و احکام به اسرائیل مهیا ساخته بود.
\par 11 و این است صورت مکتوبی که ارتحشستاپادشاه، به عزرای کاهن و کاتب داد که کاتب کلمات وصایای خداوند و فرایض او بر اسرائیل بود:
\par 12 «از جانب ارتحشستا شاهنشاه، به عزرای کاهن و کاتب کامل شریعت خدای آسمان، امابعد.
\par 13 فرمانی از من صادر شد که هر کدام از قوم اسرائیل و کاهنان و لاویان ایشان که در سلطنت من هستند و به رفتن همراه تو به اورشلیم راضی باشند، بروند.
\par 14 چونکه تو از جانب پادشاه و هفت مشیر او، فرستاده شده‌ای تا درباره یهودا واورشلیم بروفق شریعت خدایت که در دست تواست، تفحص نمایی.
\par 15 و نقره و طلایی را که پادشاه و مشیرانش برای خدای اسرائیل که مسکن او در اورشلیم می‌باشد بذل کرده‌اند، ببری.
\par 16 و نیز تمامی نقره و طلایی را که در تمامی ولایت بابل بیابی، با هدایای تبرعی که قوم وکاهنان برای خانه خدای خود که در اورشلیم است داده‌اند، (ببری ).
\par 17 لهذا با این گاوان وقوچها و بره‌ها و هدایای آردی و هدایای ریختنی آنها رابه اهتمام بخر و آنها را بر مذبح خانه خدای خودتان که در اورشلیم است، بگذران.
\par 18 و هر‌چه به نظر تو و برادرانت پسندآید که با بقیه نقره و طلا بکنید، برحسب اراده خدای خود به عمل آورید. 
\par 19 و ظروفی که به جهت خدمت خانه خدایت به تو داده شده است، آنها را به حضور خدای اورشلیم تسلیم نما.
\par 20 واما چیزهای دیگر که برای خانه خدایت لازم باشد، هر‌چه برای تو اتفاق افتد که بدهی، آن را ازخزانه پادشاه بده.
\par 21 و از من ارتحشستا پادشاه فرمانی به تمامی خزانه‌داران ماورای نهر صادرشده است که هر‌چه عزرای کاهن و کاتب شریعت خدای آسمان از شما بخواهد، به تعجیل کرده شود.
\par 22 تا صد وزنه نقره و تا صد کر گندم و تاصد بت شراب و تا صد بت روغن و از نمک، هرچه بخواهد.
\par 23 هر‌چه خدای آسمان فرموده باشد، برای خانه خدای آسمان بلاتاخیر کرده شود، زیرا چرا غضب بر ملک پادشاه و پسرانش وارد آید.
\par 24 و شما را اطلاع می‌دهیم که بر همه کاهنان و لاویان و مغنیان و دربانان و نتینیم وخادمان این خانه خدا جزیه و خراج و باج نهادن جایز نیست.
\par 25 و تو‌ای عزرا، موافق حکمت خدایت که در دست تو می‌باشد، قاضیان و داوران از همه آنانی که شرایع خدایت را می‌دانند نصب نما تا بر جمیع اهل ماورای نهر داوری نمایند وآنانی را که نمی دانند تعلیم دهید.
\par 26 و هر‌که به شریعت خدایت و به فرمان پادشاه عمل ننماید، بر او بی‌محابا حکم شود، خواه به قتل یا به جلای وطن یا به ضبط اموال یا به حبس.»
\par 27 متبارک باد یهوه خدای پدران ما که مثل این را در دل پادشاه نهاده است که خانه خداوند را که در اورشلیم است زینت دهد.و مرا در حضورپادشاه و مشیرانش و جمیع روسای مقتدر پادشاه منظور ساخت، پس من موافق دست یهوه خدایم که بر من می‌بود، تقویت یافتم و روسای اسرائیل را جمع کردم تا با من برآیند.
\par 28 و مرا در حضورپادشاه و مشیرانش و جمیع روسای مقتدر پادشاه منظور ساخت، پس من موافق دست یهوه خدایم که بر من می‌بود، تقویت یافتم و روسای اسرائیل را جمع کردم تا با من برآیند.
 
\chapter{8}

\par 1 و اینانند روسای آبای ایشان و این است نسب نامه آنانی که در سلطنت ارتحشستاپادشاه، با من از بابل برآمدند:
\par 2 از بنی فینحاس، جرشوم و از بنی‌ایتامار، دانیال و از بنی داود، حطوش.
\par 3 و از بنی شکنیا از بنی فروش، زکریا و بااو صد و پنجاه نفر از ذکوران به نسب نامه شمرده شدند.
\par 4 از بنی فحت، موآب الیهو عینای ابن زرحیا و با او دویست نفر از ذکور.
\par 5 از بنی شکنیا، ابن یحزیئیل و با او سیصد نفر از ذکور.
\par 6 ازبنی عادین، عابد بن یوناتان و با او پنجاه نفر ازذکور.
\par 7 از بنی عیلام، اشعیا ابن عتلیا و با او هفتادنفر از ذکور.
\par 8 از بنی شفطیا، زبدیا ابن میکائیل و بااو هشتاد نفر از ذکور.
\par 9 از بنی یوآب، عوبدیا ابن یحئیل و با او دویست و هجده نفر از ذکور.
\par 10 واز بنی شلومیت بن یوسفیا و با او صد وشصت نفراز ذکور.
\par 11 و از بنی بابای، زکریا ابن بابای و با او بیست و هشت نفر از ذکور.
\par 12 و از بنی عزجد، یوحانان بن هقاطان و با او صد و ده نفر از ذکور.
\par 13 و موخران از بنی ادونیقام بودند و این است نامهای ایشان: الیفلط ویعیئیل و شمعیا و با ایشان شصت نفر از ذکور.
\par 14 و از بنی بغوای، عوتای وزبود و با ایشان هفتاد نفر از ذکور.
\par 15 پس ایشان را نزد نهری که به اهوا می‌رودجمع کردم و در آنجا سه روز اردو زدیم و چون قوم و کاهنان را بازدید کردم، از بنی لاوی کسی رادر آنجا نیافتم.
\par 16 پس نزد الیعزر و اریئیل وشمعیا و الناتان و یاریب و الناتان و ناتان و زکریا ومشلام که روسا بودند و نزد یویاریب و الناتان که علما بودند، فرستادم.
\par 17 و پیغامی برای عدوی رئیس، در مکان کاسفیا به‌دست ایشان فرستادم وسخنانی که باید به عدو و برادرانش نتینیم که درمکان کاسقیا بودند بگویند، به ایشان القا کردم تاخادمان به جهت خانه خدای ما نزد ما بیاورند.
\par 18 و از دست نیکوی خدای ما که با ما می‌بود، شخصی دانشمند از پسران محلی ابن لاوی ابن اسرائیل برای ما آوردند، یعنی شربیا را با پسران وبرادرانش که هجده نفر بودند.
\par 19 و حشبیا را نیز وبا او از بنی مراری اشعیا را. و برادران او و پسران ایشان را که بیست نفر بودند.
\par 20 و از نتینیم که داودو سروران، ایشان را برای خدمت لاویان تعیین نموده بودند. از نتینیم دویست و بیست نفر که جمیع به نام ثبت شده بودند.
\par 21 پس من در آنجانزد نهر اهوا به روزه داشتن اعلان نمودم تاخویشتن را در حضور خدای خود متواضع نموده، راهی راست برای خود و عیال خویش وهمه اموال خود از او بطلبیم.
\par 22 زیرا خجالت داشتم که سپاهیان و سواران از پادشاه بخواهیم تاما را از دشمنان در راه اعانت کنند، چونکه به پادشاه عرض کرده، گفته بودیم که دست خدای ما بر هر‌که او را می‌طلبد، به نیکویی می‌باشد، اماقدرت و غضب او به ضد آنانی که او را ترک می‌کنند.
\par 23 پس روزه گرفته، خدای خود را برای این طلب نمودیم و ما را مستجاب فرمود.
\par 24 ودوازده نفر از روسای کهنه، یعنی شربیا و حشبیا وده نفر از برادران ایشان را با ایشان جدا کردم.
\par 25 ونقره و طلا و ظروف هدیه خدای ما را که پادشاه ومشیران و سرورانش و تمامی اسرائیلیانی که حضور داشتند داده بودند، به ایشان وزن نمودم.
\par 26 پس ششصد و پنجاه وزنه نقره و صد وزنه ظروف نقره و صد وزنه طلا به‌دست ایشان وزن نمودم.
\par 27 و بیست طاس طلا هزار درهم و دوظرف برنج صیقلی خالص که مثل طلا گرانبها بود.
\par 28 و به ایشان گفتم: «شما برای خداوند مقدس می‌باشید و ظروف نیز مقدس است و نقره و طلا به جهت یهوه خدای پدران شما هدیه تبرعی است.
\par 29 پس بیدار باشید و اینها را حفظ نمایید تا به حضور روسای کهنه و لاویان و سروران آبای اسرائیل در اورشلیم، به حجره های خانه خداوندبه وزن بسپارید.»
\par 30 آنگاه کاهنان و لاویان وزن طلا و نقره وظروف را گرفتند تا آنها را به خانه خدای ما به اورشلیم برسانند.
\par 31 پس در روز دوازدهم ماه اول از نهر اهوا کوچ کرده، متوجه اورشلیم شدیم و دست خدای ما با ما بود و ما را از دست دشمنان و کمین نشینندگان سر راه خلاصی داد.
\par 32 و چون به اورشلیم رسیدیم سه روز درآنجاتوقف نمودیم.
\par 33 و در روز چهارم، نقره و طلا و ظروف را در خانه خدای ما به‌دست مریموت بن اوریای کاهن وزن کردند و العازار بن فینحاس با اوبود و یوزاباد بن یشوع و نوعدیا ابن بنوی لاویان باایشان بودند.
\par 34 همه را به شماره و به وزن (حساب کردند) و وزن همه در آن وقت نوشته شد.
\par 35 و اسیرانی که از اسیری برگشته بودند، قربانی های سوختنی برای خدای اسرائیل گذرانیدند، یعنی دوازده گاو و نود و شش قوچ وهفتاد و هفت بره و دوازده بز نر، به جهت قربانی گناه، برای تمامی اسرائیل که همه اینها قربانی سوختنی برای خداوند بود.و چون فرمانهای پادشاه را به امرای پادشاه و والیان ماورای نهردادند، ایشان قوم و خانه خدا را اعانت نمودند.
\par 36 و چون فرمانهای پادشاه را به امرای پادشاه و والیان ماورای نهردادند، ایشان قوم و خانه خدا را اعانت نمودند.
 
\chapter{9}

\par 1 و بعد از تمام شدن این وقایع، سروران نزدمن آمده، گفتند: «قوم اسرائیل و کاهنان ولاویان خویشتن را از امت های کشورها جدانکرده‌اند بلکه موافق رجاسات ایشان، یعنی کنعانیان و حتیان و فرزیان و یبوسیان و عمونیان وموآبیان و مصریان و اموریان (رفتار نموده اند).
\par 2 زیرا که از دختران ایشان برای خود و پسران خویش زنان گرفته و ذریت مقدس را با امت های کشورها مخلوط کرده‌اند و دست روسا و حاکمان در این خیانت مقدم بوده است.»
\par 3 پس چون این سخن را شنیدم، جامه و ردای خود را چاک زدم و موی سر و ریش خود را کندم و متحیر نشستم.
\par 4 آنگاه، همه آنانی که به‌سبب این عصیان اسیران، از کلام خدای اسرائیل می‌ترسیدند، نزد من جمع شدند و من تا وقت هدیه شام، متحیر نشستم.
\par 5 و در وقت هدیه شام، از تذلل خود برخاستم وبا لباس و ردای دریده، به زانو درآمدم و دست خود را بسوی یهوه خدای خویش برافراشتم.
\par 6 و گفتم: «ای خدای من، خجالت دارم و از بلند کردن روی خود بسوی توای خدایم شرم دارم، زیرا گناهان ما بالای سر مازیاده شده، و تقصیرهای ما تا به آسمان عظیم گردیده است.
\par 7 ما از ایام پدران خود تا امروزمرتکب تقصیرهای عظیم شده‌ایم و ما وپادشاهان و کاهنان ما به‌سبب گناهان خویش، به‌دست پادشاهان کشورها به شمشیر و اسیری وتاراج و رسوایی تسلیم گردیده‌ایم، چنانکه امروزشده است.
\par 8 و حال اندک زمانی لطف از جانب یهوه خدای ما بر ما ظاهر شده، مفری برای ماواگذاشته است و ما را در مکان مقدس خودمیخی عطا فرموده است و خدای ما چشمان ما راروشن ساخته، اندک حیات تازه‌ای در حین بندگی ما به ما بخشیده است.
\par 9 زیرا که ما بندگانیم، لیکن خدای ما، ما را در حالت بندگی ترک نکرده است، بلکه ما را منظور پادشاهان فارس گردانیده، حیات تازه به ما بخشیده است تا خانه خدای خود را بنانماییم و خرابیهای آن را تعمیر کنیم و ما را دریهودا و اورشلیم قلعه‌ای بخشیده است.
\par 10 وحال‌ای خدای ما بعد از این چه گوییم، زیرا که اوامر تو را ترک نموده‌ایم.
\par 11 که آنها را به‌دست بندگان خود انبیا امر فرموده و گفته‌ای که آن زمینی که شما برای تصرف آن می‌روید، زمینی است که از نجاسات امت های کشورها نجس شده است و آن را به رجاسات و نجاسات خویش، از سر تا سر مملو ساخته‌اند.
\par 12 پس الان، دختران خود را به پسران ایشان مدهید ودختران ایشان را برای پسران خود مگیرید و سلامتی و سعادتمندی ایشان را تا به ابد مطلبید تاقوی شوید و نیکویی آن زمین را بخورید و آن رابرای پسران خود به ارثیت ابدی واگذارید.
\par 13 وبعد از همه این بلایایی که به‌سبب اعمال زشت وتقصیرهای عظیم ما بر ما وارد شده است، با آنکه تو‌ای خدای ما، ما را کمتر از گناهان ما عقوبت رسانیده‌ای و چنین خلاصی‌ای به ما داده‌ای،
\par 14 آیا می‌شود که ما بار دیگر اوامر تو را بشکنیم وبا امت هایی که مرتکب این رجاسات شده‌اند، مصاهرت نماییم؟ و آیا تو بر ما غضب نخواهی نمود و ما را چنان هلاک نخواهی ساخت که بقیتی و نجاتی باقی نماند؟‌ای یهوه خدای اسرائیل تو عادل هستی چونکه بقیتی از ما مثل امروزناجی شده‌اند، اینک ما به حضور تو درتقصیرهای خویش حاضریم، زیرا کسی نیست که به‌سبب این کارها، در حضور تو تواند ایستاد.»
\par 15 ‌ای یهوه خدای اسرائیل تو عادل هستی چونکه بقیتی از ما مثل امروزناجی شده‌اند، اینک ما به حضور تو درتقصیرهای خویش حاضریم، زیرا کسی نیست که به‌سبب این کارها، در حضور تو تواند ایستاد.»
 
\chapter{10}

\par 1 پس چون عزرا دعا و اعتراف می‌نمود وگریه‌کنان پیش خانه خدا رو به زمین نهاده بود، گروه بسیار عظیمی از مردان و زنان واطفال اسرائیل نزد وی جمع شدند، زیرا قوم زارزار می‌گریستند.
\par 2 و شکنیا ابن یحئیل که ازبنی عیلام بود جواب داد و به عزرا گفت: «ما به خدای خویش خیانت ورزیده، زنان غریب ازقومهای زمین گرفته‌ایم، لیکن الان امیدی برای اسرائیل در این باب باقی است.
\par 3 پس حال باخدای خویش عهد ببندیم که آن زنان و اولادایشان را برحسب مشورت آقایم و آنانی که از امرخدای ما می‌ترسند دور کنیم و موافق شریعت عمل نماییم.
\par 4 برخیز زیرا که این کار تو است و مابا تو می‌باشیم. پس قوی‌دل باش و به‌کار بپرداز.»
\par 5 آنگاه عزرا برخاسته، روسای کهنه و لاویان و تمامی اسرائیل را قسم داد که برحسب این سخن عمل نمایند، پس قسم خوردند.
\par 6 و عزرا ازپیش روی خانه خدا برخاسته، به حجره یهوحانان بن الیاشیب رفت و نان نخورده و آب ننوشیده، به آنجا رفت، زیرا که به‌سبب تقصیراسیران ماتم گرفته بود.
\par 7 و به همه اسیران در یهوداو اورشلیم ندا دردادند که به اورشلیم جمع شوند.
\par 8 و هر کسی‌که تا روز سوم، برحسب مشورت سروران و مشایخ حاضر نشود، اموال او ضبطگردد و خودش از جماعت اسیران جدا شود.
\par 9 پس در روز سوم که روز بیستم ماه نهم بود، همه مردان یهودا و بنیامین در اورشلیم جمع شدند وتمامی قوم در سعه خانه خدا نشستند. و به‌سبب این امر و به‌سبب باران، سخت می‌لرزیدند. 
\par 10 آنگاه عزرای کاهن برخاسته، به ایشان گفت: «شما خیانت ورزیده و زنان غریب گرفته، جرم اسرائیل را افزوده‌اید.
\par 11 پس الان یهوه خدای پدران خود را تمجید نمایید و به اراده او عمل کنید و خویشتن را از قومهای زمین و از زنان غریب جدا سازید.»
\par 12 تمامی جماعت به آواز بلند جواب دادند وگفتند: «چنانکه به ما گفته‌ای همچنان عمل خواهیم نمود.
\par 13 اما خلق بسیارند و وقت باران است و طاقت نداریم که بیرون بایستیم و این امرکار یک یا دو روز نیست، زیرا که در این باب گناه عظیمی کرده‌ایم.
\par 14 پس سروران ما برای تمامی جماعت تعیین بشوند و جمیع کسانی که درشهرهای ما زنان غریب گرفته‌اند، در وقت های معین بیایند و مشایخ و داوران هر شهر همراه ایشان بیایند، تا حدت خشم خدای ما درباره این امر از ما رفع گردد.»
\par 15 لهذا یوناتان بن عسائیل و یحزیا ابن تقوه براین امر معین شدند و مشلام و شبتائی لاوی، ایشان را اعانت نمودند.
\par 16 و اسیران چنین کردندو عزرای کاهن و بعضی از روسای آبا، برحسب خاندانهای آبای خود منتخب شدند و نامهای همه ایشان ثبت گردید. پس در روز اول ماه دهم، برای تفتیش این امر نشستند.
\par 17 و تا روز اول ماه اول، کار همه مردانی را که زنان غریب گرفته بودند، به اتمام رسانیدند.
\par 18 و بعضی از پسران کاهنان پیدا شدند که زنان غریب گرفته بودند. ازبنی یشوع بن یوصاداق و برادرانش معسیا و الیعزرو یاریب و جدلیا.
\par 19 و ایشان دست دادند که زنان خود را بیرون نمایند و قوچی به جهت قربانی جرم خود گذرانیدند.
\par 20 و از بنی امیر، حنانی و زبدیا.
\par 21 و ازبنی حاریم، معسیا و ایلیا و شمعیا و یحیئیل وعزیا.
\par 22 و از بنی فشحور، الیوعینای و معسیا واسمعیل و نتنئیل و یوزاباد و العاسه.
\par 23 و ازلاویان، یوزاباد و شمعی و قلایا که قلیطا باشد. وفتحیا و یهودا و الیعزر.
\par 24 و از مغنیان، الیاشیب واز دربانان، شلوم و طالم و اوری.
\par 25 و اما ازاسرائیلیان: از بنی فرعوش، رمیا و یزیا و ملکیا ومیامین و العازار و ملکیا و بنایا.
\par 26 و از بنی عیلام، متنیا و زکریا و یحیئیل و عبدی و یریموت و ایلیا.
\par 27 و از بنی زتو، الیوعینای و الیاشیب و متنیا ویریموت و زاباد و عزیزا.
\par 28 و از بنی بابای، یهوحانان و حننیا و زبای و عتلای.
\par 29 و ازبنی بانی، مشلام و ملوک و عدایا و یاشوب و شال و راموت.
\par 30 و از بنی فحت، موآب عدنا و کلال وبنایا و معسیا و متنیا و بصلئیل و بنوی و منسی.
\par 31 و از بنی حاریم، الیعزر و اشیا و ملکیا و شمعیا وشمعون.
\par 32 و بنیامین و ملوک و شمریا.
\par 33 ازبنی حاشوم، متنای و متاته و زاباد و الیفلط ویریمای و منسی و شمعی.
\par 34 از بنی بانی، معدای و عمرام و اوئیل.
\par 35 و بنایا و بیدیا وکلوهی.
\par 36 و ونیا و مریموت و الیاشیب.
\par 37 ومتنیا و متنای و یعسو.
\par 38 و بانی و بنوی و شمعی.
\par 39 و شلمیا و ناتان و عدایا.
\par 40 ومکندبای و شاشای و شارای.
\par 41 و عزرئیل وشلمیا و شمریا.
\par 42 و شلوم و امریا و یوسف.از بنی نبو، یعیئیل و متتیا و زاباد و زبینا و یدوو یوئیل و بنایا.
\par 43 از بنی نبو، یعیئیل و متتیا و زاباد و زبینا و یدوو یوئیل و بنایا.


\end{document}