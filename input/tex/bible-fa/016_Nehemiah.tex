\begin{document}

\title{نحيميا}

 
\chapter{1}

\par 1 کلام نحمیا ابن حکلیا: در ماه کسلو در سال بیستم هنگامی که من در دارالسلطنه شوشان بودم، واقع شد
\par 2 که حنانی، یکی ازبرادرانم با کسانی چند از یهودا آمدند و از ایشان درباره بقیه یهودی که از اسیری باقی‌مانده بودندو درباره اورشلیم سوال نمودم.
\par 3 ایشان مراجواب دادند: «آنانی که آنجا در بلوک از اسیری باقی‌مانده‌اند در مصیبت سخت و افتضاح می‌باشند و حصار اورشلیم خراب ودروازه هایش به آتش سوخته شده است.»
\par 4 و چون این سخنان را شنیدم، نشستم و گریه کرده، ایامی چند ماتم داشتم و به حضور خدای آسمانها روزه گرفته، دعا نمودم.
\par 5 و گفتم: «آه‌ای یهوه، خدای آسمانها، ای خدای عظیم و مهیب که عهد و رحمت را بر آنانی که تو را دوست می‌دارند و اوامر تو را حفظ می‌نمایند، نگاه می‌داری،
\par 6 گوشهای تو متوجه و چشمانت گشاده شود و دعای بنده خود را که من در این وقت نزد تو روز و شب درباره بندگانت بنی‌اسرائیل می‌نمایم، اجابت فرمایی و به گناهان بنی‌اسرائیل که به تو ورزیده‌ایم، اعتراف می‌نمایم، زیرا که هم من و هم خاندان پدرم گناه کرده‌ایم.
\par 7 به درستی که به تو مخالفت عظیمی ورزیده‌ایم و اوامر و فرایض و احکامی را که به بنده خود موسی فرموده بودی، نگاه نداشته‌ایم.
\par 8 پس حال، کلامی را که به بنده خود موسی‌امرفرمودی، بیاد آور که گفتی شما خیانت خواهیدورزید و من شما را در میان امت‌ها پراکنده خواهم ساخت.
\par 9 اما چون بسوی من بازگشت نمایید واوامر مرا نگاه داشته، به آنها عمل نمایید، اگر‌چه پراکندگان شما در اقصای آسمانها باشند، من ایشان را از آنجا جمع خواهم کرد و به مکانی که آن را برگزیده‌ام تا نام خود را در آن ساکن سازم درخواهم آورد.
\par 10 و ایشان بندگان و قوم تومی باشند که ایشان را به قوت عظیم خود و به‌دست قوی خویش فدیه داده‌ای.‌ای خداوند، گوش تو بسوی دعای بنده ات و دعای بندگانت که به رغبت تمام از اسم تو ترسان می‌باشند، متوجه بشود و بنده خود را امروز کامیاب فرمایی و او را به حضور این مرد مرحمت عطا کنی.» زیراکه من ساقی پادشاه بودم.
\par 11 ‌ای خداوند، گوش تو بسوی دعای بنده ات و دعای بندگانت که به رغبت تمام از اسم تو ترسان می‌باشند، متوجه بشود و بنده خود را امروز کامیاب فرمایی و او را به حضور این مرد مرحمت عطا کنی.» زیراکه من ساقی پادشاه بودم.
 
\chapter{2}

\par 1 و در ماه نیسان، در سال بیستم ارتحشستاپادشاه، واقع شد که شراب پیش وی بود ومن شراب را گرفته، به پادشاه دادم و قبل از آن من در حضورش ملول نبودم.
\par 2 و پادشاه مرا گفت: «روی تو چرا ملول است با آنکه بیمار نیستی؟ این غیر از ملالت دل، چیزی نیست.» پس من بی‌نهایت ترسان شدم.
\par 3 و به پادشاه گفتم: «پادشاه تا به ابد زنده بماند، رویم چگونه ملول نباشد و حال آنکه شهری که موضع قبرهای پدرانم باشد، خراب است و دروازه هایش به آتش سوخته شده.»
\par 4 پادشاه مرا گفت: «چه چیز می‌طلبی؟» آنگاه نزد خدای آسمانها دعا نمودم
\par 5 و به پادشاه گفتم: «اگر پادشاه را پسند آید و اگر بنده ات درحضورش التفات یابد، مرا به یهودا و شهرمقبره های پدرانم بفرستی تا آن را تعمیر نمایم.»
\par 6 پادشاه مرا گفت و ملکه به پهلوی او نشسته بود: «طول سفرت چه قدر خواهد بود و کی مراجعت خواهی نمود؟» پس پادشاه صواب دیدکه مرا بفرستد و زمانی برایش تعیین نمودم.
\par 7 و به پادشاه عرض کردم، اگر پادشاه مصلحت بیندمکتوبات برای والیان ماورای نهر به من عطا شودتا مرا بدرقه نمایند و به یهودا برسانند.
\par 8 ومکتوبی نیز به آساف که ناظر درختستانهای پادشاه است تا چوب برای سقف دروازه های قصر که متعلق به خانه است، به من داده شود و هم برای حصار شهر و خانه‌ای که من در آن ساکن شوم. پس پادشاه برحسب دست مهربان خدایم که بر من بود، اینها را به من عطا فرمود.
\par 9 پس چون نزد والیان ماورای نهر رسیدم، مکتوبات پادشاه را به ایشان دادم و پادشاه، سرداران سپاه و سواران نیز همراه من فرستاده بود.
\par 10 اما چون سنبلط حرونی و طوبیای غلام عمونی‌این را شنیدند، ایشان را بسیار ناپسند آمدکه کسی به جهت طلبیدن نیکویی بنی‌اسرائیل آمده است.
\par 11 پس به اورشلیم رسیدم و در آنجا سه روزماندم.
\par 12 و شبگاهان به اتفاق چند نفری که همراه من بودند، برخاستم و به کسی نگفته بودم که خدایم در دل من چه نهاده بود که برای اورشلیم بکنم؛ و چهارپایی به غیر از آن چهارپایی که بر آن سوار بودم با من نبود.
\par 13 پس شبگاهان از دروازه وادی در مقابل چشمه اژدها تا دروازه خاکروبه بیرون رفتم و حصار اورشلیم را که خراب شده بود و دروازه هایش را که به آتش سوخته شده بود، ملاحظه نمودم.
\par 14 و از دروازه چشمه، نزدبرکه پادشاه گذشتم و برای عبور چهارپایی که زیرمن بود، راهی نبود.
\par 15 و در آن شب به کنار نهربرآمده، حصار را ملاحظه نمودم و برگشته، ازدروازه وادی داخل شده، مراجعت نمودم.
\par 16 وسروران ندانستند که کجا رفته یا چه کرده بودم، زیرا به یهودیان و به کاهنان و به شرفا و سروران وبه دیگر کسانی که در کار مشغول می‌بودند، هنوزخبر نداده بودم.
\par 17 پس به ایشان گفتم: «شما بلایی را که در آن هستیم که اورشلیم چگونه خراب و دروازه هایش به آتش سوخته شده است، می‌بینید. بیایید وحصار اورشلیم را تعمیر نماییم تا دیگر رسوانباشیم.»
\par 18 و ایشان را از دست خدای خود که بر من مهربان می‌بود و نیز از سخنانی که پادشاه به من گفته بود خبر دادم. آنگاه گفتند: «برخیزیم وتعمیر نماییم.» پس دستهای خود را برای کارخوب قوی ساختند.
\par 19 اما چون سنبلط حرونی و طوبیای غلام عمونی و جشم عربی این راشنیدند، ما را استهزا نمودند و ما را حقیر شمرده، گفتند: «این چه‌کار است که شما می‌کنید؟ آیا برپادشاه فتنه می‌انگیزید؟»من ایشان را جواب داده، گفتم: «خدای آسمانها ما را کامیاب خواهدساخت. پس ما که بندگان او هستیم برخاسته، تعمیر خواهیم نمود. اما شما را در اورشلیم، نه نصیبی و نه حقی و نه ذکری می‌باشد.»
\par 20 من ایشان را جواب داده، گفتم: «خدای آسمانها ما را کامیاب خواهدساخت. پس ما که بندگان او هستیم برخاسته، تعمیر خواهیم نمود. اما شما را در اورشلیم، نه نصیبی و نه حقی و نه ذکری می‌باشد.»
 
\chapter{3}

\par 1 و الیاشیب، رئیس کهنه و برادرانش از کاهنان برخاسته، دروازه گوسفند را بناکردند. ایشان آن را تقدیس نموده، دروازه هایش را برپا نمودند و آن را تا برج میا و برج حننئیل تقدیس نمودند.
\par 2 و به پهلوی او، مردان اریحا بناکردند و به پهلوی ایشان، زکور بن امری بنا نمود.
\par 3 و پسران هسناه، دروازه ماهی را بنا کردند. ایشان سقف آن را ساختند و درهایش را با قفلها وپشت بندهایش برپا نمودند.
\par 4 و به پهلوی ایشان، مریموت بن اوریا ابن حقوص تعمیر نمود و به پهلوی ایشان، مشلام بن برکیا ابن مشیزبئیل تعمیرنمود و به پهلوی ایشان، صادوق بن بعنا تعمیرنمود.
\par 5 و به پهلوی ایشان، تقوعیان تعمیر کردند، اما بزرگان ایشان گردن خود را به خدمت خداوندخویش ننهادند.
\par 6 و یویاداع بن فاسیح و مشلام بن بسودیا، دروازه کهنه را تعمیر نمودند. ایشان سقف آن را ساختند و درهایش را با قفلها وپشت بندهایش برپا نمودند.
\par 7 و به پهلوی ایشان، ملتیای جبعونی و یادون میرونوتی و مردان جبعون و مصفه آنچه را که متعلق به کرسی والی ماورای نهر بود، تعمیر نمودند.
\par 8 و به پهلوی ایشان، عزیئیل بن حرهایا که از زرگران بود، تعمیرنمود و به پهلوی او حننیا که از عطاران بود تعمیرنمود، پس اینان اورشلیم را تا حصار عریض، مستحکم ساختند.
\par 9 و به پهلوی ایشان، رفایا ابن حور که رئیس نصف بلد اورشلیم بود، تعمیرنمود.
\par 10 و به پهلوی ایشان، یدایا ابن حروماف دربرابر خانه خود تعمیر نمود و به پهلوی اوحطوش بن حشبنیا، تعمیر نمود.
\par 11 و ملکیا ابن حاریم و حشوب بن فحت موآب، قسمت دیگر وبرج تنورها را تعمیر نمودند.
\par 12 و به پهلوی او، شلوم بن هلوحیش رئیس نصف بلد اورشلیم، او ودخترانش تعمیر نمودند.
\par 13 و حانون و ساکنان زانوح، دروازه وادی را تعمیر نمودند. ایشان آن را بنا کردند و درهایش را با قفلها و پشتبندهایش برپا نمودند و هزار ذراع حصار را تا دروازه خاکروبه.
\par 14 و ملکیا ابن رکاب رئیس بلدبیت هکاریم، دروازه خاکروبه را تعمیر نمود. اوآن را بنا کرد و درهایش را با قفلها و پشتبندهایش برپا نمود.
\par 15 و شلون بن کلخوزه رئیس بلد مصفه، دروازه چشمه را تعمیر نمود. او آن را بنا کرده، سقف آن را ساخت و درهایش را با قفلها وپشت بندهایش برپا نمود و حصار برکه شلح رانزد باغ پادشاه نیز تا زینه‌ای که از شهر داود فرودمی آمد.
\par 16 و بعد از او نحمیا ابن عزبوق رئیس نصف بلد بیت صور، تا برابر مقبره داود و تا برکه مصنوعه و تا بیت جباران را تعمیر نمود.
\par 17 و بعداز او لاویان، رحوم بن بانی تعمیر نمود و به پهلوی او حشبیا رئیس نصف بلد قعیله در حصه خودتعمیر نمود.
\par 18 و بعد از او برادران ایشان، بوای ابن حیناداد، رئیس نصف بلد قعیله تعمیر نمود.
\par 19 و به پهلوی او، عازر بن یشوع رئیس مصفه قسمت دیگر را در برابر فراز سلاح خانه نزدزاویه، تعمیر نمود.
\par 20 و بعد از او باروک بن زبای، به صمیم قلب قسمت دیگر را از زاویه تا دروازه الیاشیب، رئیس کهنه تعمیر نمود.
\par 21 و بعد از اومریموت بن اوریا ابن هقوص قسمت دیگر را ازدر خانه الیاشیب تا آخر خانه الیاشیب، تعمیرنمود.
\par 22 و بعد از او کاهنان، از اهل غور تعمیرنمودند.
\par 23 و بعد از ایشان، بنیامین و حشوب دربرابر خانه خود تعمیر نمودند و بعد از ایشان، عزریا ابن معسیا ابن عننیا به‌جانب خانه خودتعمیر نمود.
\par 24 و بعد از او، بنوی ابن حیناداد قسمت دیگر را از خانه عزریا تا زاویه و تا برجش تعمیر نمود.
\par 25 و فالال بن اوزای از برابر زاویه وبرجی که از خانه فوقانی پادشاه خارج و نزدزندانخانه است، تعمیر نمود و بعد از او فدایا ابن فرعوش،
\par 26 و نتینیم، در عوفل تا برابر دروازه آب بسوی مشرق و برج خارجی، ساکن بودند.
\par 27 وبعد از او، تقوعیان قسمت دیگر را از برابر برج خارجی بزرگ تا حصار عوفل تعمیر نمودند.
\par 28 وکاهنان، هر کدام در برابر خانه خود از بالای دروازه اسبان تعمیر نمودند.
\par 29 و بعد از ایشان صادوق بن امیر در برابر خانه خود تعمیر نمود وبعد از او شمعیا ابن شکنیا که مستحفظ دروازه شرقی بود، تعمیر نمود.
\par 30 و بعد از او حننیا ابن شلمیا و حانون پسر ششم صالاف، قسمت دیگررا تعمیر نمودند و بعد از ایشان مشلام بن برکیا دربرابر مسکن خود، تعمیر نمود.
\par 31 و بعد او ازملکیا که یکی از زرگران بود، تا خانه های نتینیم وتجار را در برابر دروازه مفقاد تا بالاخانه برج، تعمیر نمود.و میان بالاخانه برج و دروازه گوسفند را زرگران و تاجران، تعمیر نمودند.
\par 32 و میان بالاخانه برج و دروازه گوسفند را زرگران و تاجران، تعمیر نمودند.
 
\chapter{4}

\par 1 و هنگامی که سنبلط شنید که ما به بنای حصار مشغول هستیم، خشمش افروخته شده، بسیار غضبناک گردید و یهودیان را استهزانمود.
\par 2 و در حضور برادرانش و لشکر سامره متکلم شده، گفت: «این یهودیان ضعیف چه می‌کنند؟ آیا (شهر را) برای خود مستحکم خواهند ساخت و قربانی خواهند گذرانید و دریک روز کار را به انجام خواهند رسانید؟ و سنگهااز توده های خاکروبه، زنده خواهند ساخت؟ وحال آنکه سوخته شده است.»
\par 3 و طوبیای عمونی که نزد او بود گفت: «اگر شغالی نیز بر آنچه ایشان بنا می‌کنند بالا رود، حصار سنگی ایشان رامنهدم خواهد ساخت!»
\par 4 ‌ای خدای ما بشنو، زیرا که خوار شده‌ایم وملامت ایشان را بسر ایشان برگردان و ایشان را درزمین اسیری، به تاراج تسلیم کن.
\par 5 و عصیان ایشان را مستور منما و گناه ایشان را از حضورخود محو مساز زیرا که خشم تو را پیش روی بنایان به هیجان آورده‌اند.
\par 6 پس حصار را بنا نمودیم و تمامی حصار تانصف بلندی‌اش بهم پیوست، زیرا که دل قوم درکار بود.
\par 7 و چون سنبلط و طوبیا و اعراب وعمونیان و اشدودیان شنیدند که مرمت حصاراورشلیم پیش رفته است و شکافهایش بسته می‌شود، آنگاه خشم ایشان به شدت افروخته شد. 
\par 8 و جمیع ایشان توطئه نمودند که بیایند و بااورشلیم جنگ نمایند و به آن ضرر برسانند.
\par 9 پس نزد خدای خود دعا نمودیم و از ترس ایشان روز و شب پاسبانان در مقابل ایشان قراردادیم.
\par 10 و یهودیان گفتند که «قوت حمالان تلف شده است و هوار بسیار است که نمی توانیم حصار را بنا نماییم.»
\par 11 و دشمنان ما می‌گفتند: «آگاه نخواهند شد و نخواهند فهمید، تا ما درمیان ایشان داخل شده، ایشان را بکشیم و کار راتمام نماییم.»
\par 12 و واقع شد که یهودیانی که نزدایشان ساکن بودند آمده، ده مرتبه به ما گفتند: «چون شما برگردید ایشان از هر طرف بر ما(حمله خواهند‌آورد).»
\par 13 پس قوم را در جایهای پست، در عقب حصار و بر مکانهای خالی تعیین نمودم و ایشان رابرحسب قبایل ایشان، با شمشیرها و نیزه‌ها وکمانهای ایشان قرار دادم.
\par 14 پس نظر کرده، برخاستم و به بزرگان و سروران و بقیه قوم گفتم: «از ایشان مترسید، بلکه خداوند عظیم و مهیب رابیاد آورید و به جهت برادران و پسران و دختران وزنان و خانه های خود جنگ نمایید.»
\par 15 و چون دشمنان ما شنیدند که ما آگاه شده‌ایم و خدا مشورت ایشان را باطل کرده است، آنگاه جمیع ما هر کس به‌کار خود به حصاربرگشتیم.
\par 16 و از آن روز به بعد، نصف بندگان من به‌کار مشغول می‌بودند و نصف دیگر ایشان، نیزه‌ها و سپرها و کمانها و زره‌ها را می‌گرفتند وسروران در عقب تمام خاندان یهودا می‌بودند.
\par 17 و آنانی که حصار را بنا می‌کردند و آنانی که بارمی بردند و عمله‌ها هر کدام به یک دست کارمی کردند و به‌دست دیگر اسلحه می‌گرفتند.
\par 18 وبنایان هر کدام شمشیر بر کمر خود بسته، بنایی می‌کردند و کرنانواز نزد من ایستاده بود.
\par 19 و به بزرگان و سروران و بقیه قوم گفتم: «کار، بسیار وسیع است و ما بر حصار متفرق و ازیکدیگر دور می‌باشیم.
\par 20 پس هر جا که آواز کرنارا بشنوید در آنجا نزد ما جمع شوید و خدای مابرای ما جنگ خواهد نمود.»
\par 21 پس به‌کارمشغول شدیم و نصف ایشان از طلوع فجر تابیرون آمدن ستارگان، نیزه‌ها را می‌گرفتند.
\par 22 وهم در آن وقت به قوم گفتم: «هر کس با بندگانش در اورشلیم منزل کند تا در شب برای ما پاسبانی نماید و در روز به‌کار بپردازد.»و من و برادران و خادمان من و پاسبانی که در عقب من می‌بودند، هیچکدام رخت خود را نکندیم و هر کس بااسلحه خود به آب می‌رفت.
\par 23 و من و برادران و خادمان من و پاسبانی که در عقب من می‌بودند، هیچکدام رخت خود را نکندیم و هر کس بااسلحه خود به آب می‌رفت.
 
\chapter{5}

\par 1 و قوم و زنان ایشان، بر برادران یهود خودفریاد عظیمی برآوردند.
\par 2 و بعضی ازایشان گفتند که «ما و پسران و دختران ما بسیاریم. پس گندم بگیریم تا بخوریم و زنده بمانیم.»
\par 3 وبعضی گفتند: «مزرعه‌ها و تاکستانها و خانه های خود را گرو می‌دهیم تا به‌سبب قحط، گندم بگیریم.»
\par 4 و بعضی گفتند که «نقره را به عوض مزرعه‌ها و تاکستانهای خود برای جزیه پادشاه قرض گرفتیم.
\par 5 و حال جسد ما مثل جسدهای برادران ماست و پسران ما مثل پسران ایشان واینک ما پسران و دختران خود را به بندگی می‌سپاریم و بعضی از دختران ما کنیز شده‌اند ودر دست ما هیچ استطاعتی نیست زیرا که مزرعه‌ها و تاکستانهای ما از آن دیگران شده است.»
\par 6 پس چون فریاد ایشان و این سخنان را شنیدم بسیار غضبناک شدم.
\par 7 و با دل خود مشورت کرده، بزرگان و سروران را عتاب نمودم و به ایشان گفتم: «شما هر کس از برادر خود ربا می‌گیرید!» وجماعتی عظیم به ضد ایشان جمع نمودم.
\par 8 و به ایشان گفتم: «ما برادران یهود خود را که به امت هافروخته شده‌اند، حتی المقدور فدیه کرده‌ایم. وآیا شما برادران خود را می‌فروشید و آیا می‌شودکه ایشان به ما فروخته شوند؟» پس خاموش شده، جوابی نیافتند.
\par 9 و گفتم: «کاری که شما می‌کنید خوب نیست، آیا نمی باید شما به‌سبب ملامت امت هایی که دشمن ما می‌باشند، در ترس خدای ما سلوک نمایید؟
\par 10 و نیز من و برادران و بندگانم نقره و غله به ایشان قرض داده‌ایم. پس سزاواراست که این ربا را ترک نماییم.
\par 11 و الان امروزمزرعه‌ها و تاکستانها و باغات زیتون و خانه های ایشان و صد یک از نقره و غله و عصیر انگور وروغن که بر ایشان نهاده‌اید به ایشان رد کنید.»
\par 12 پس جواب دادند که «رد خواهیم کرد و ازایشان مطالبه نخواهیم نمود و چنانکه تو فرمودی به عمل خواهیم آورد.» آنگاه کاهنان را خوانده، به ایشان قسم دادم که بروفق این کلام رفتارنمایند.
\par 13 پس دامن خود را تکانیده گفتم: «خداهر کس را که این کلام را ثابت ننماید، از خانه وکسبش چنین بتکاند و به این قسم تکانیده و خالی بشود.» پس تمامی جماعت گفتند آمین وخداوند را تسبیح خواندند و قوم برحسب این کلام عمل نمودند.
\par 14 و نیز از روزی که به والی بودن زمین یهوه مامور شدم، یعنی از سال بیستم تا سال سی و دوم ارتحشستا پادشاه، که دوازده سال بود من وبرادرانم وظیفه والیگری را نخوردیم.
\par 15 اماوالیان اول که قبل از من بودند بر قوم بار سنگین نهاده، علاوه بر چهل مثقال نقره، نان و شراب نیزاز ایشان می‌گرفتند و خادمان ایشان بر قوم حکمرانی می‌کردند. لیکن من به‌سبب ترس خداچنین نکردم.
\par 16 و من نیز در ساختن حصارمشغول می‌بودم و هیچ مزرعه نخریدیم و همه بندگان من در آنجا به‌کار جمع بودند.
\par 17 و صد وپنجاه نفر از یهودیان و سروران، سوای آنانی که ازامت های مجاور ما نزد ما می‌آمدند، بر سفره من خوراک می‌خوردند.
\par 18 و آنچه برای هر روز مهیامی شد، یک گاو و شش گوسفند پرواری می‌بود ومرغها نیز برای من حاضر می‌کردند و هر ده روز مقداری کثیر از هر گونه شراب. اما معهذا وظیفه والیگری را نطلبیدم زیرا که بندگی سخت بر این قوم می‌بود.‌ای خدایم موافق هر‌آنچه به این قوم عمل نمودم مرا به نیکویی یاد آور.
\par 19 ‌ای خدایم موافق هر‌آنچه به این قوم عمل نمودم مرا به نیکویی یاد آور.
 
\chapter{6}

\par 1 بازسازی حصار و چون سنبلط و طوبیا و جشم عربی و سایر دشمنان ما شنیدند که حصار را بناکرده‌ام و هیچ رخنه‌ای در آن باقی نمانده است، باآنکه درهای دروازه هایش را هنوز برپا ننموده بودم،
\par 2 سنبلط و جشم نزد من فرستاده، گفتند: «بیاتا در یکی از دهات بیابان اونو ملاقات کنیم.» اماایشان قصد ضرر من داشتند.
\par 3 پس قاصدان نزد ایشان فرستاده گفتم: «من درمهم عظیمی مشغولم و نمی توانم فرود آیم، چراکار حینی که من آن را ترک کرده، نزد شما فرودآیم به تعویق افتد.»
\par 4 و ایشان چهار دفعه مثل این پیغام به من فرستادند و من مثل این جواب به ایشان پس فرستادم.
\par 5 پس سنبلط دفعه پنجم خادم خود رابه همین طور نزد من فرستاد و مکتوبی گشوده دردستش بود،
\par 6 که در آن مرقوم بود: «در میان امت‌ها شهرت یافته است و جشم این را می‌گویدکه تو و یهود قصد فتنه انگیزی دارید و برای همین حصار را بنا می‌کنی و تو بروفق این کلام، می‌خواهی که پادشاه ایشان بشوی.
\par 7 و انبیا نیزتعیین نموده تا درباره تو در اورشلیم ندا کرده گویند که در یهودا پادشاهی است. و حال بروفق این کلام، خبر به پادشاه خواهد رسید. پس بیا تا باهم مشورت نماییم.»
\par 8 آنگاه نزد او فرستاده گفتم: «مثل این کلام که تو می‌گویی واقع نشده است، بلکه آن را از دل خود ابداع نموده‌ای.»
\par 9 زیرا جمیع ایشان خواستند ما را بترسانند، به این قصد که دستهای ما را از کار باز دارند تا کرده نشود. پس حال‌ای خدا دستهای مرا قوی ساز.
\par 10 و به خانه شمعیا ابن دلایا ابن مهیطبئیل رفتم و او در را بر خود بسته بود، پس گفت: «در خانه خدا در هیکل جمع شویم و درهای هیکل راببندیم زیرا که به قصد کشتن تو خواهند آمد. شبانگاه برای کشتن تو خواهند آمد.»
\par 11 من گفتم: «آیا مردی چون من فرار بکند؟ وکیست مثل من که داخل هیکل بشود تا جان خودرا زنده نگاه دارد؟ من نخواهم آمد.»
\par 12 زیرا درک کردم که خدا او را هرگز نفرستاده است بلکه خودش به ضد من نبوت می‌کند و طوبیا و سنبلطاو را اجیر ساخته‌اند.
\par 13 و از این جهت او را اجیرکرده‌اند تا من بترسم و به اینطور عمل نموده، گناه ورزم و ایشان خبر بد پیدا نمایند که مرا مفتضح سازند.
\par 14 ‌ای خدایم، طوبیا و سنبلط را موافق این اعمال ایشان و همچنین نوعدیه نبیه و سایر انبیا راکه می‌خواهند مرا بترسانند، به یاد آور.
\par 15 پس حصار در بیست و پنجم ماه ایلول درپنجاه و دو روز به اتمام رسید.
\par 16 و واقع شد که چون جمیع دشمنان ما این را شنیدند و همه امت هایی که مجاور ما بودند، این را دیدند، درنظر خود بسیار پست شدند و دانستند که این کاراز جانب خدای ما معمول شده است.
\par 17 و در آن روزها نیز بسیاری از بزرگان یهودا مکتوبات نزد طوبیا می‌فرستادند و مکتوبات طوبیا نزد ایشان می‌رسید،
\par 18 زیرا که بسا از اهل یهودا با اوهمداستان شده بودند، چونکه او داماد شکنیا ابن آره بود و پسرش یهوحانان، دختر مشلام بن برکیارا به زنی گرفته بود،و درباره حسنات او به حضور من نیز گفتگو می‌کردند و سخنان مرا به اومی رسانیدند. و طوبیا مکتوبات می‌فرستاد تا مرابترساند.
\par 19 و درباره حسنات او به حضور من نیز گفتگو می‌کردند و سخنان مرا به اومی رسانیدند. و طوبیا مکتوبات می‌فرستاد تا مرابترساند.
 
\chapter{7}

\par 1 و چون حصار بنا شد و درهایش را برپانمودم و دربانان و مغنیان و لاویان ترتیب داده شدند،
\par 2 آنگاه برادر خود حنانی و حننیارئیس قصر را، زیرا که او مردی امین و بیشتر ازاکثر مردمان خداترس بود، بر اورشلیم فرمان دادم.
\par 3 و ایشان را گفتم دروازه های اورشلیم را تاآفتاب گرم نشود باز نکنند و مادامی که حاضرباشند، درها را ببندند و قفل کنند و از ساکنان اورشلیم پاسبانان قرار دهید که هر کس به پاسبانی خود و هر کدام به مقابل خانه خویش حاضرباشند.
\par 4 و شهر وسیع و عظیم بود و قوم در اندرونش کم و هنوز خانه‌ها بنا نشده بود.
\par 5 و خدای من دردلم نهاد که بزرگان و سروران و قوم را جمع نمایم تا برحسب نسب نامه‌ها ثبت کردند و نسب نامه آنانی را که مرتبه اول برآمده بودند یافتم و در آن بدین مضمون نوشته دیدم:
\par 6 اینانند اهل ولایتها که از اسیری آن اشخاصی که نبوکدنصر پادشاه بابل به اسیری برده بود، برآمده بودند و هر کدام از ایشان به اورشلیم و یهودا به شهر خود برگشته بودند.
\par 7 اما آنانی که همراه زربابل آمده بودند: یسوع و نحمیا و عزریا و رعمیا و نحمانی و مردخای و بلشان و مسفارت و بغوای و نحوم و بعنه. و شماره مردان قوم اسرائیل:
\par 8 بنی فرعوش، دوهزار و یک صد وهفتاد و دو.
\par 9 بنی شفطیا، سیصد و هفتاد و دو.
\par 10 بنی آرح، ششصد و پنجاه و دو.
\par 11 بنی فحت موآب از بنی یشوع و یوآب، دو هزار و هشتصد وهجده.
\par 12 بنی عیلام، هزار و دویست و پنجاه وچهار.
\par 13 بنی زتو، هشتصد و چهل و پنج.
\par 14 بنی زکای، هفتصد و شصت.
\par 15 بنی بنوی، ششصد و چهل و هشت.
\par 16 بنی بابای، ششصد وبیست و هشت.
\par 17 بنی عزجد، دو هزار و سیصد وبیست و دو.
\par 18 بنی ادونیقام، ششصد و شصت وهفت.
\par 19 بنی بغوای، دو هزار و شصت و هفت.
\par 20 بنی عادین، ششصد و پنجاه و پنج.
\par 21 بنی آطیراز (خاندان ) حزقیا، نود و هشت.
\par 22 بنی حاشوم، سیصد و بیست و هشت.
\par 23 بنی بیصای، سیصد وبیست و چهار.
\par 24 بنی حاریف، صد و دوازده.
\par 25 بنی جبعون، نود و پنج.
\par 26 مردمان بیت لحم ونطوفه، صد و هشتاد و هشت.
\par 27 مردمان عناتوت، صد و بیست و هشت.
\par 28 مردمان بیت عزموت، چهل و دو.
\par 29 مردمان قریه یعاریم و کفیره و بئیروت، هفتصد و چهل و سه.
\par 30 مردمان رامه و جبع، ششصد و بیست و یک.
\par 31 مردمان مکماس، صد و بیست و دو.
\par 32 مردمان بیت ایل و عای، صد و بیست و سه.
\par 33 مردمان نبوی دیگر، پنجاه و دو.
\par 34 بنی عیلام دیگر، هزارو دویست و پنجاه و چهار.
\par 35 بنی حاریم، سیصدو بیست.
\par 36 بنی اریحا، سیصد و چهل و پنج. 
\par 37 بنی لود و حادید و اونو، هفتصد و بیست و یک.
\par 38 بنی سنائه، سه هزار و نه صد و سی.
\par 39 و اماکاهنان: بنی یدعیا از خاندان یشوع، نه صد و هفتادو سه.
\par 40 بنی امیر، هزار و پنجاه و دو.
\par 41 بنی فشحور، هزار و دویست و چهل و هفت.
\par 42 بنی حاریم، هزار و هفده.
\par 43 و اما لاویان: بنی یشوع از (خاندان ) قدمیئیل و از بنی هودویا، هفتاد و چهار.
\par 44 و مغنیان: بنی آساف، صد وچهل و هشت.
\par 45 و دربانان: بنی شلوم و بنی آطیرو بنی طلمون و بنی عقوب و بنی حطیطه وبنی سوبای، صد و سی و هشت.
\par 46 و اما نتینیم: بنی صیحه، بنی حسوفا، بنی طبایوت.
\par 47 بنی فیروس، بنی سیعا، بنی فادون.
\par 48 بنی لبانه، بنی حجابه، بنی سلمای.
\par 49 بنی حانان، بنی جدیل، بنی جاحر.
\par 50 بنی رآیا، بنی رصین، بنی نقودا.
\par 51 بنی جزام، بنی عزا، بنی فاسیح.
\par 52 بنی بیسای، بنی معونیم، بنی نفیشسیم.
\par 53 بنی بقبوق، بنی حقوفا، بنی حرحور.
\par 54 بنی بصلیت، بنی محیده، بنی حرشا.
\par 55 بنی برقوس، بنی سیسرا، بنی تامح.
\par 56 بنی نصیح، بنی حطیفا.
\par 57 و پسران خادمان سلیمان: بنی سوطای، بنی سوفرت، بنی فریدا.
\par 58 بنی یعلا، بنی درقون، بنی جدیل.
\par 59 بنی شفطیا، بنی حطیل، بنی فوخره حظبائیم، بنی آمون.
\par 60 جمیع نتینیم و پسران خادمان سلیمان، سیصدو نود و دو.
\par 61 و اینانند آنانی که از تل ملح و تل حرشاکروب و ادون و امیر برآمده بودند، اماخاندان پدران و عشیره خود را نشان نتوانستندداد که آیا از اسرائیلیان بودند یا نه.
\par 62 بنی دلایا، بنی طوبیا، بنی نقوده، ششصد و چهل و دو.
\par 63 و ازکاهنان: بنی حبایا، بنی هقوص، بنی برزلای که یکی از دختران برزلایی جلعادی را به زنی گرفته بود، پس به نام ایشان مسمی شدند.
\par 64 اینان انساب خود را در میان آنانی که در نسب نامه هاثبت شده بودند طلبیدند، اما نیافتند، پس ازکهانت اخراج شدند.
\par 65 پس ترشاتا به ایشان امر فرمود که تا کاهنی با اوریم و تمیم برقرار نشود، از قدس اقداس نخورند.
\par 66 تمامی جماعت با هم چهل و دو هزارو سیصد و شصت نفر بودند.
\par 67 سوای غلامان وکنیزان ایشان که هفت هزار و سیصد و سی و هفت نفر بودند و مغنیان و مغنیات ایشان دویست وچهل و پنج نفر بودند.
\par 68 و اسبان ایشان، هفتصدو سی و شش و قاطران ایشان، دویست و چهل وپنج.
\par 69 و شتران، چهار صد و سی و پنج وحماران، ششهزار و هفتصد و بیست بود.
\par 70 و بعضی از روسای آبا هدایا به جهت کاردادند. اما ترشاتا هزار درم طلا و پنجاه قاب وپانصد و سی دست لباس کهانت به خزانه داد.
\par 71 وبعضی از روسای آبا، بیست هزار درم طلا و دوهزار و دویست منای نقره به خزینه به جهت کاردادند.
\par 72 و آنچه سایر قوم دادند این بود: بیست هزار درم طلا و دو هزار منای نقره و شصت وهفت دست لباس کهانت.پس کاهنان و لاویان و دربانان و مغنیان و بعضی از قوم و نتینیم و جمیع اسرائیل، در شهرهای خود ساکن شدند و چون ماه هفتم رسید، بنی‌اسرائیل در شهرهای خودمقیم بودند.
\par 73 پس کاهنان و لاویان و دربانان و مغنیان و بعضی از قوم و نتینیم و جمیع اسرائیل، در شهرهای خود ساکن شدند و چون ماه هفتم رسید، بنی‌اسرائیل در شهرهای خودمقیم بودند.
 
\chapter{8}

\par 1 و تمامی، قوم مثل یک مرد در سعه پیش دروازه آب جمع شدند و به عزرای کاتب گفتند که کتاب تورات موسی را که خداوند به اسرائیل امر فرموده بود، بیاورد.
\par 2 و عزرای کاهن، تورات را در روز اول ماه هفتم به حضور جماعت از مردان و زنان و همه آنانی که می‌توانستند بشنوند و بفهمند، آورد.
\par 3 و آن را در سعه پیش دروازه آب از روشنایی صبح تا نصف روز، درحضور مردان و زنان و هر‌که می‌توانست بفهمدخواند و تمامی قوم به کتاب تورات گوش فراگرفتند.
\par 4 و عزرای کاتب بر منبر چوبی که به جهت اینکار ساخته بودند، ایستاد و به پهلویش از دست راستش متتیا و شمع و عنایا و اوریا وحلقیا و معسیا ایستادند و از دست چپش، فدایا ومیشائیل و ملکیا و حاشوم و حشبدانه و زکریا ومشلام.
\par 5 و عزرا کتاب را در نظر تمامی قوم گشودزیرا که او بالای تمامی قوم بود و چون آن راگشود، تمامی قوم ایستادند.
\par 6 و عزرا، یهوه خدای عظیم را متبارک خواند و تمامی قوم دستهای خود را برافراشته، در جواب گفتند: «آمین، آمین!» و رکوع نموده، و رو به زمین نهاده، خداوند را سجده نمودند.
\par 7 و یشوع و بانی وشربیا و یامین و عقوب و شبتای و هودیا و معسیاو قلیطا و عزریا و یوزاباد و حنان و فلایا و لاویان، تورات را برای قوم بیان می‌کردند و قوم، در جای خود ایستاده بودند.
\par 8 پس کتاب تورات خدا را به صدای روشن خواندند و تفسیر کردند تا آنچه را که می‌خواندند، بفهمند.
\par 9 و نحمیا که ترشاتا باشد وعزرای کاهن و کاتب و لاویانی که قوم رامی فهمانیدند، به تمامی قوم گفتند: «امروز برای یهوه خدای شما روز مقدس است. پس نوحه گری منمایید و گریه مکنید.» زیرا تمامی قوم، چون کلام تورات را شنیدند گریستند.
\par 10 پس به ایشان گفت: «بروید و خوراکهای لطیف بخورید و شربتها بنوشید و نزد هر‌که چیزی برای او مهیا نیست حصه‌ها بفرستید، زیراکه امروز، برای خداوند ما روز مقدس است، پس محزون نباشید زیرا که سرور خداوند، قوت شمااست.»
\par 11 و لاویان تمامی قوم را ساکت ساختندو گفتند: «ساکت باشید زیرا که امروز روز مقدس است. پس محزون نباشید.»
\par 12 پس تمامی قوم رفته، اکل و شرب نمودند و حصه‌ها فرستادند وشادی عظیم نمودند زیرا کلامی را که به ایشان تعلیم داده بودند فهمیدند.
\par 13 و در روز دوم روسای آبای تمامی قوم وکاهنان و لاویان نزد عزرای کاتب جمع شدند، تاکلام تورات را اصغا نمایند.
\par 14 و در تورات چنین نوشته یافتند که خداوند به واسطه موسی‌امرفرموده بود که بنی‌اسرائیل در عید ماه هفتم، درسایبانها ساکن بشوند.
\par 15 و در تمامی شهرهای خود و در اورشلیم اعلان نمایند و ندا دهند که به کوهها بیرون رفته، شاخه های زیتون و شاخه های زیتون بری و شاخه های آس و شاخه های نخل وشاخه های درختان کشن بیاورند و سایه بانها، به نهجی که مکتوب است بسازند.
\par 16 پس قوم بیرون رفتند و هر کدام بر پشت بام خانه خود و در حیاط خود و در صحنهای خانه خدا و در سعه دروازه آب و در سعه دروازه افرایم، سایبانها برای خود ساختند.
\par 17 و تمامی جماعتی که از اسیری برگشته بودند، سایبانهاساختند و در سایبانها ساکن شدند، زیرا که از ایام یوشع بن نون تا آن روز بنی‌اسرائیل چنین نکرده بودند. پس شادی بسیار عظیمی رخ نمود.وهر روز از روز اول تا روز آخر، کتاب تورات خدارا می‌خواند و هفت روز عید را نگاه داشتند. و در روز هشتم، محفل مقدس برحسب قانون برپا شد.
\par 18 وهر روز از روز اول تا روز آخر، کتاب تورات خدارا می‌خواند و هفت روز عید را نگاه داشتند. و در روز هشتم، محفل مقدس برحسب قانون برپا شد.
 
\chapter{9}

\par 1 و در روز بیست و چهارم این ماه، بنی‌اسرائیل روزه‌دار و پلاس دربر و خاک برسر جمع شدند.
\par 2 و ذریت اسرائیل خویشتن رااز جمیع غربا جدا نموده، ایستادند و به گناهان خود و تقصیرهای پدران خویش اعتراف کردند.
\par 3 و در جای خود ایستاده، یک ربع روز کتاب تورات یهوه خدای خود را خواندند و ربع دیگراعتراف نموده، یهوه خدای خود را عبادت نمودند.
\par 4 و یشوع و بانی و قدمیئیل و شبنیا و بنی و شربیا و بانی و کنانی بر زینه لاویان ایستادند و به آواز بلند، نزد یهوه خدای خویش استغاثه نمودند.
\par 5 آنگاه لاویان، یعنی یشوع و قدمیئیل وبانی و حشبنیا و شربیا و هودیا و شبنیا و فتحیاگفتند: «برخیزید و یهوه خدای خود را از ازل تا به ابد متبارک بخوانید. و اسم جلیل تو که از تمام برکات و تسبیحات اعلی تر است متبارک باد.
\par 6 توبه تنهایی یهوه هستی. تو فلک و فلک الافلاک وتمامی جنود آنها را و زمین را و هر‌چه بر آن است و دریاها را و هر‌چه در آنها است، ساخته‌ای و توهمه اینها را حیات می‌بخشی و جنود آسمان تورا سجده می‌کنند.
\par 7 تو‌ای یهوه آن خدا هستی که ابرام را برگزیدی و او را از اور کلدانیان بیرون آوردی واسم او را به ابراهیم تبدیل نمودی.
\par 8 ودل او را به حضور خود امین یافته، با وی عهدبستی که زمین کنعانیان و حتیان و اموریان وفرزیان و یبوسیان و جرجاشیان را به او ارزانی داشته، به ذریت او بدهی و وعده خود را وفا نمودی، زیرا که عادل هستی.
\par 9 و مصیبت پدران ما را در مصر دیدی و فریاد ایشان را نزد بحر قلزم شنیدی.
\par 10 و آیات و معجزات بر فرعون و جمیع بندگانش و تمامی قوم زمینش ظاهر ساختی، چونکه می‌دانستی که بر ایشان ستم می‌نمودندپس به جهت خود اسمی پیدا کردی، چنانکه امروز شده است.
\par 11 و دریا را به حضور ایشان منشق ساختی تا از میان دریا به خشکی عبورنمودند و تعاقب کنندگان ایشان را به عمقهای دریا مثل سنگ در آب عمیق انداختی.
\par 12 وایشان را در روز، به ستون ابر و در شب، به ستون آتش رهبری نمودی تا راه را که در آن باید رفت، برای ایشان روشن سازی.
\par 13 و بر کوه سینا نازل شده، با ایشان از آسمان تکلم نموده و احکام راست و شرایع حق و اوامر و فرایض نیکو را به ایشان دادی.
\par 14 و سبت مقدس خود را به ایشان شناسانیدی و اوامر و فرایض و شرایع به واسطه بنده خویش موسی به ایشان امر فرمودی.
\par 15 و نان از آسمان برای گرسنگی ایشان دادی و آب ازصخره برای تشنگی ایشان جاری ساختی و به ایشان وعده دادی که به زمینی که دست خود رابرافراشتی که آن را به ایشان بدهی داخل شده، آن را به تصرف آورند.
\par 16 «لیکن ایشان و پدران ما متکبرانه رفتارنموده، گردن خویش را سخت ساختند و اوامر تورا اطاعت ننمودند.
\par 17 و از شنیدن ابا نمودند واعمال عجیبه‌ای را که در میان ایشان نمودی بیادنیاوردند، بلکه گردن خویش را سخت ساختند و فتنه انگیخته، سرداری تعیین نمودند تا (به زمین )بندگی خود مراجعت کنند. اما تو خدای غفار وکریم و رحیم و دیرغضب و کثیراحسان بوده، ایشان را ترک نکردی.
\par 18 بلکه چون گوساله ریخته شده‌ای برای خود ساختند و گفتند: (ای اسرائیل )! این خدای تو است که تو را از مصربیرون آورد. و اهانت عظیمی نمودند.
\par 19 آنگاه تونیز برحسب رحمت عظیم خود، ایشان را دربیابان ترک ننمودی، و ستون ابر در روز که ایشان را در راه رهبری می‌نمود از ایشان دور نشد و نه ستون آتش در شب که راه را که در آن باید بروندبرای ایشان روشن می‌ساخت.
\par 20 و روح نیکوی خود را به جهت تعلیم ایشان دادی و من خویش را از دهان ایشان باز نداشتی و آب برای تشنگی ایشان، به ایشان عطا فرمودی.
\par 21 و ایشان را دربیابان چهل سال پرورش دادی که به هیچ‌چیزمحتاج نشدند. لباس ایشان مندرس نگردید وپایهای ایشان ورم نکرد.
\par 22 و ممالک و قومها به ایشان ارزانی داشته، آنها را تا حدود تقسیم نمودی و زمین سیحون و زمین پادشاه حشبون وزمین عوج پادشاه باشان را به تصرف آوردند.
\par 23 وپسران ایشان را مثل ستارگان آسمان افزوده، ایشان را به زمینی که به پدران ایشان وعده داده بودی که داخل شده، آن را به تصرف آورند، درآوردی.
\par 24 «پس، پسران ایشان داخل شده، زمین را به تصرف آوردند و کنعانیان را که سکنه زمین بودند، به حضور ایشان مغلوب ساختی و آنها را با پادشاهان آنها و قومهای زمین، به‌دست ایشان تسلیم نمودی، تا موافق اراده خود با آنها رفتارنمایند.
\par 25 پس شهرهای حصاردار و زمینهای برومند گرفتند و خانه های پر از نفایس وچشمه های کنده شده و تاکستانها و باغات زیتون و درختان میوه دار بیشمار به تصرف آوردند وخورده و سیر شده و فربه گشته، از نعمتهای عظیم تو متلذذ گردیدند.
\par 26 و بر تو فتنه انگیخته و تمردنموده، شریعت تو را پشت سر خود انداختند وانبیای تو را که برای ایشان شهادت می‌آوردند تابسوی تو بازگشت نمایند، کشتند و اهانت عظیمی به عمل آوردند.
\par 27 آنگاه تو ایشان را به‌دست دشمنانشان تسلیم نمودی تا ایشان را به تنگ آورند و در حین تنگی خویش، نزد تواستغاثه نمودند و ایشان را از آسمان اجابت نمودی و برحسب رحمتهای عظیم خود، نجات دهندگان به ایشان دادی که ایشان را ازدست دشمنانشان رهانیدند. 
\par 28 «اما چون استراحت یافتند، بار دیگر به حضور تو شرارت ورزیدند و ایشان را به‌دست دشمنانشان واگذاشتی که بر ایشان تسلط نمودند. و چون باز نزد تو استغاثه نمودند، ایشان را ازآسمان اجابت نمودی و برحسب رحمتهای عظیمت، بارهای بسیار ایشان را رهایی دادی.
\par 29 و برای ایشان شهادت فرستادی تا ایشان را به شریعت خود برگردانی، اما ایشان متکبرانه رفتارنموده، اوامر تو را اطاعت نکردند و به احکام توکه هر‌که آنها را بجا آورد از آنها زنده میماند، خطاورزیدند و دوشهای خود را معاند و گردنهای خویش را سخت نموده، اطاعت نکردند.
\par 30 «معهذا سالهای بسیار با ایشان مدارانمودی و به روح خویش به واسطه انبیای خودبرای ایشان شهادت فرستادی، اما گوش نگرفتند. لهذا ایشان را به‌دست قوم های کشورهاتسلیم نمودی.
\par 31 اما برحسب رحمتهای عظیمت، ایشان را بالکل فانی نساختی و ترک ننمودی، زیراخدای کریم و رحیم هستی.
\par 32 و الان‌ای خدای ما، ای خدای عظیم و جبار و مهیب که عهد ورحمت را نگاه می‌داری، زنهار تمامی این مصیبتی که بر ما و بر پادشاهان و سروران و کاهنان و انبیا و پدران ما و بر تمامی قوم تو از ایام پادشاهان اشور تا امروز مستولی شده است، درنظر تو قلیل ننماید.
\par 33 و تو در تمامی این چیزهایی که بر ما وارد شده است عادل هستی، زیرا که تو به راستی عمل نموده‌ای، اما ما شرارت ورزیده‌ایم.
\par 34 و پادشاهان و سروران و کاهنان وپدران ما به شریعت تو عمل ننمودند و به اوامر وشهادات تو که به ایشان امر فرمودی، گوش ندادند.
\par 35 و در مملکت خودشان و در احسان عظیمی که به ایشان نمودی و در زمین وسیع وبرومند که پیش روی ایشان نهادی تو را عبادت ننمودند و از اعمال شنیع خویش بازگشت نکردند.
\par 36 «اینک ما امروز غلامان هستیم و در زمینی که به پدران ما دادی تا میوه و نفایس آن رابخوریم، اینک در آن غلامان هستیم.
\par 37 و آن، محصول فراوان خود را برای پادشاهانی که به‌سبب گناهان ما، بر ما مسلط ساخته‌ای می‌آورد و ایشان بر جسدهای ما و چهارپایان ما برحسب اراده خود حکمرانی می‌کنند و ما در شدت تنگی گرفتار هستیم.و به‌سبب همه این امور، ما عهدمحکم بسته، آن را نوشتیم و سروران و لاویان وکاهنان ما آن را مهر کردند.»
\par 38 و به‌سبب همه این امور، ما عهدمحکم بسته، آن را نوشتیم و سروران و لاویان وکاهنان ما آن را مهر کردند.»
 
\chapter{10}

\par 1 و کسانی که آن را مهر کردند اینانند: نحمیای ترشاتا ابن حکلیا و صدقیا.
\par 2 وسرایا و عزریا و ارمیا.
\par 3 و فشحور و امریا و ملکیا.
\par 4 و حطوش و شبنیا و ملوک.
\par 5 و حاریم ومریموت و عوبدیا.
\par 6 و دانیال و جنتون و باروک.
\par 7 و مشلام و ابیا و میامین.
\par 8 و معزیا و بلجای وشمعیا، اینها کاهنان بودند.
\par 9 و اما لاویان: یشوع بن ازنیا و بنوی از پسران حیناداد و قدمیئیل.
\par 10 وبرادران ایشان شبنیا و هودیا و قلیطا و فلایا وحانان.
\par 11 و میخا و رحوب و حشبیا.
\par 12 و زکور وشربیا و شبنیا.
\par 13 و هودیا و بانی و بنینو.
\par 14 وسروران قوم فرعوش و فحت موآب و عیلام وزتو و بانی.
\par 15 و بنی و عزجد و بابای.
\par 16 و ادونیا وبغوای و عودین.
\par 17 و عاطیر و حزقیا و عزور.
\par 18 وهودیا و حاشوم و بیصای.
\par 19 و حاریف وعناتوت و نیبای.
\par 20 و مجفیعاش و مشلام وحزیر.
\par 21 و مشیزبئیل و صادوق و یدوع.
\par 22 وفلطیا و حانان و عنایا.
\par 23 و هوشع و حننیا وحشوب.
\par 24 و هلوحیش و فلحا و شوبیق.
\par 25 ورحوم و حشبنا و معسیا.
\par 26 و اخیا و حانان وعانان.
\par 27 و ملوک و حاریم و بعنه.
\par 28 «و سایر قوم و کاهنان و لاویان و دربانان ومغنیان و نتینیم و همه کسانی که خویشتن را ازاهالی کشورها به تورات خدا جدا ساخته بودند با زنان و پسران و دختران خود و همه صاحبان معرفت و فطانت،
\par 29 به برادران و بزرگان خویش ملصق شدند و لعنت و قسم بر خود نهادند که به تورات خدا که به واسطه موسی بنده خدا داده شده بود، سلوک نمایند و تمامی اوامر یهوه خداوند ما و احکام و فرایض او را نگاه دارند و به عمل آورند.
\par 30 و اینکه دختران خود را به اهل زمین ندهیم و دختران ایشان را برای پسران خودنگیریم.
\par 31 و اگر اهل زمین در روز سبت، متاع یاهر گونه آذوقه به جهت فروختن بیاورند، آنها را ازایشان در روزهای سبت و روزهای مقدس نخریم و (حاصل ) سال هفتمین و مطالبه هر قرض راترک نماییم.
\par 32 و بر خود فرایض قرار دادیم که یک ثلث مثقال در هر سال، بر خویشتن لازم دانیم به جهت خدمت خانه خدای ما.
\par 33 برای نان تقدمه و هدیه آردی دایمی و قربانی سوختنی دایمی در سبت‌ها و هلالها و مواسم و به جهت موقوفات و قربانی های گناه تا کفاره به جهت اسرائیل بشود و برای تمامی کارهای خانه خدای ما.
\par 34 و ما کاهنان و لاویان و قوم، قرعه برای هدیه هیزم انداختیم، تا آن را به خانه خدای خودبرحسب خاندانهای آبای خویش، هر سال به وقتهای معین بیاوریم تا بر مذبح یهوه خدای ماموافق آنچه در تورات نوشته است سوخته شود.
\par 35 و تا آنکه نوبرهای زمین خود و نوبرهای همه میوه هر گونه درخت را سال به سال به خانه خداوند بیاوریم.
\par 36 و تا اینکه نخست زاده های پسران و حیوانات خود را موافق آنچه در تورات نوشته شده است و نخست زاده های گاوان و گوسفندان خود را به خانه خدای خویش، برای کاهنانی که در خانه خدای ما خدمت می‌کنندبیاوریم.
\par 37 و نیز نوبر خمیر خود را و هدایای افراشتنی خویش را و میوه هر گونه درخت وعصیر انگور و روغن زیتون را برای کاهنان به حجره های خانه خدای خود و عشر زمین خویش را به جهت لاویان بیاوریم، زیرا که لاویان عشر رادر جمیع شهرهای زراعتی ما می‌گیرند.
\par 38 وهنگامی که لاویان عشر می‌گیرند، کاهنی ازپسران هارون همراه ایشان باشد و لاویان عشرعشرها را به خانه خدای ما به حجره های بیت‌المال بیاورند.زیرا که بنی‌اسرائیل وبنی لاوی هدایای افراشتنی غله و عصیر انگور وروغن زیتون را به حجره‌ها می‌بایست بیاورند، جایی که آلات قدس و کاهنانی که خدمت می‌کنند و دربانان و مغنیان حاضر می‌باشند، پس خانه خدای خود را ترک نخواهیم کرد.»
\par 39 زیرا که بنی‌اسرائیل وبنی لاوی هدایای افراشتنی غله و عصیر انگور وروغن زیتون را به حجره‌ها می‌بایست بیاورند، جایی که آلات قدس و کاهنانی که خدمت می‌کنند و دربانان و مغنیان حاضر می‌باشند، پس خانه خدای خود را ترک نخواهیم کرد.»
 
\chapter{11}

\par 1 و سروران قوم در اورشلیم ساکن شدند وسایر قوم قرعه انداختند تا از هر ده نفریکنفر را به شهر مقدس اورشلیم، برای سکونت بیاورند و نه نفر باقی، در شهرهای دیگر ساکن شوند.
\par 2 و قوم، همه کسانی را که به خوشی دل برای سکونت در اورشلیم حاضر شدند، مبارک خواندند.
\par 3 و اینانند سروران بلدانی که دراورشلیم ساکن شدند، (و سایر اسرائیلیان وکاهنان و لاویان و نتینیم و پسران بندگان سلیمان، هر کس در ملک شهر خود، در شهرهای یهوداساکن شدند).
\par 4 پس در اورشلیم، بعضی از بنی یهودا وبنی بنیامین سکنی گرفتند. و اما از بنی یهودا، عنایاابن عزیا ابن زکریا ابن امریا ابن شفطیا ابن مهللئیل از بنی فارص.
\par 5 و معسیا ابن باروک بن کلحوزه ابن حزیا ابن عدایا ابن یویاریب بن زکریا ابن شیلونی.
\par 6 جمیع بنی فارص که در اورشلیم ساکن شدند، چهار صد و شصت و هشت مرد شجاع بودند.
\par 7 واینانند پسران بنیامین: سلو ابن مشلام بن یوعید بن فدایا ابن قولایا ابن معسیا ابن ایتئیل بن اشعیا.
\par 8 وبعد از او جبای و سلای، نه صد و بیست و هشت نفر.
\par 9 و یوئیل بن زکری، رئیس ایشان بود و یهوداابن هسنوآه، رئیس دوم شهر بود.
\par 10 و از کاهنان، یدعیا ابن یویاریب و یاکین.
\par 11 و سرایا ابن حلقیاابن مشلام بن صادوق بن مرایوت بن اخیطوب رئیس خانه خدا.
\par 12 و برادران ایشان که درکارهای خانه مشغول می‌بودند هشتصد و بیست و دو نفر. و عدایا ابن یروحام بن فللیا ابن امصی ابن زکریا ابن فشحور بن ملکیا.
\par 13 و برادران او که روسای آبا بودند، دویست و چهل و دو نفر. وعمشیسای بن عزرئیل بن اخزای بن مشلیموت بن امیر.
\par 14 و برادرانش که مردان جنگی بودند، صد وبیست و هشت نفر. و زبدیئیل بن هجدولیم رئیس ایشان بود.
\par 15 و از لاویان شمعیا ابن حشوب بن عزریقام بن حشبیا ابن بونی.
\par 16 و شبتای و یوزابادبر کارهای خارج خانه خدا از روسای لاویان بودند.
\par 17 و متنیا ابن میکا ابن زبدی بن آساف پیشوای تسبیح که در نماز، حمد بگوید و بقبقیاکه از میان برادرانش رئیس دوم بود و عبدا ابن شموع بن جلال بن یدوتون.
\par 18 جمیع لاویان درشهر مقدس دویست و هشتاد و چهار نفر بودند.
\par 19 و دربانان عقوب و طلمون و برادران ایشان که درها را نگاهبانی می‌کردند، صد و هفتاد و دو نفر.
\par 20 و سایر اسرائیلیان و کاهنان و لاویان هر کدام در ملک خویش در جمیع شهرهای یهودا (ساکن شدند).
\par 21 و نتینیم در عوفل سکنی گرفتند وصیحا و جشفا روسای نتینیم
\par 22 و رئیس لاویان در اورشلیم بر کارهای خانه خدا عزی ابن بانی ابن حشبیا ابن متنیا ابن میکا از پسران آساف که مغنیان بودند، می‌بود.
\par 23 زیرا که درباره ایشان حکمی ازپادشاه بود و فریضه‌ای به جهت مغنیان برای امرهر روز در روزش.
\par 24 و فتحیا ابن مشیزبئیل ازبنی زارح بن یهودا از جانب پادشاه برای جمیع امور قوم بود.
\par 25 و بعضی از بنی یهودا در قصبه هاو نواحی آنها ساکن شدند. در قریه اربع و دهات آن و دیبون و دهات آن و یقبصیئیل و دهات آن.
\par 26 و در یشوع و مولاده و بیت فالط.
\par 27 و در حصرشوعال و بئرشبع و دهات آن.
\par 28 و در صقلغ ومکونه و دهات آن.
\par 29 و در عین رمون و صرعه ویرموت.
\par 30 و زانوح و عدلام و دهات آنها ولاکیش و نواحی آن و عزیقه و دهات آن. پس ازبئرشبع تا وادی هنوم ساکن شدند.
\par 31 وبنی بنیامین از جبع تا مکماش ساکن شدند. در عیاو بیت یل و دهات آن.
\par 32 و عناتوت و نوب وعننیه،
\par 33 و حاصور و رامه و جتایم،
\par 34 و حادیدو صبوعیم و نبلاط،
\par 35 و لود و اونو و وادی حراشیم.و بعضی فرقه های لاویان در یهودا وبنیامین ساکن شدند.
\par 36 و بعضی فرقه های لاویان در یهودا وبنیامین ساکن شدند.
 
\chapter{12}

\par 1 و اینانند کاهنان و لاویانی که با زربابل بن شئلتیئیل و یشوع برآمدند. سرایا وارمیا و عزرا.
\par 2 امریا و ملوک و حطوش.
\par 3 و شکنیاو رحوم و مریموت.
\par 4 و عدو و جنتوی و ابیا.
\par 5 ومیامین و معدیا و بلجه.
\par 6 و شمعیا و یویاریب و یدعیا.
\par 7 و سلو و عاموق و حلقیا و یدعیا. اینان روسای کاهنان و برادران ایشان در ایام یشوع بودند.
\par 8 و لاویان: یشوع و بنوی و قدمیئیل وشربیا و یهودا و متنیا که او و برادرانش پیشوایان تسبیح خوانان بودند.
\par 9 و برادران ایشان بقبقیه وعنی در مقابل ایشان در جای خدمت خود بودند.
\par 10 و یشوع یویاقیم را تولید نمود و یویاقیم الیاشیب را آورد و الیاشیب یویاداع را آورد.
\par 11 ویویاداع یوناتان را آورد و یوناتان یدوع را آورد.
\par 12 و در ایام یویاقیم روسای خاندانهای آبای کاهنان اینان بودند. از سرایا مرایا و از ارمیا حننیا.
\par 13 و از عزرا، مشلام و از امریا، یهوحانان.
\par 14 و ازملیکو، یوناتان و از شبنیا، یوسف.
\par 15 و از حاریم، عدنا و از مرایوت، حلقای.
\par 16 و از عدو، زکریا واز جنتون، مشلام.
\par 17 و از ابیا، زکری و از منیامین و موعدیا، فلطای.
\par 18 و از بلجه، شموع و ازشمعیا، یهوناتان.
\par 19 و از یویاریب، متنای و ازیدعیا، عزی.
\par 20 و از سلای، قلای و از عاموق، عابر.
\par 21 و از حلقیا، حشبیا و از یدعیا، نتنئیل.
\par 22 و روسای آبای لاویان، در ایام الیاشیب ویهویاداع و یوحانان و یدوع ثبت شدند و کاهنان نیز در سلطنت داریوش فارسی. 
\par 23 و روسای آبای بنی لاوی در کتاب تواریخ ایام تا ایام یوحانان بن الیاشیب ثبت گردیدند.
\par 24 و روسای لاویان، حشبیا و شربیا و یشوع بن قدمیئیل وبرادرانشان در مقابل ایشان، تا موافق فرمان داودمرد خدا، فرقه برابر فرقه، حمد و تسبیح بخوانند.
\par 25 و متنیا و بقبقیا و عوبدیا و مشلام و طلمون وعقوب دربانان بودند که نزد خزانه های دروازه هاپاسبانی می‌نمودند.
\par 26 اینان در ایام یویاقیم بن یشوع بن یوصاداق و در ایام نحمیای والی وعزرای کاهن کاتب بودند.
\par 27 و هنگام تبریک نمودن حصار اورشلیم، لاویان را از همه مکان های ایشان طلبیدند تا ایشان را به اورشلیم بیاورند که با شادمانی و حمد و سرود بادف و بربط و عود آن را تبریک نمایند.
\par 28 پس پسران مغنیان، از دایره گرداگرد اورشلیم و ازدهات نطوفاتیان جمع شدند.
\par 29 و از بیت جلجال و از مزرعه های جبع و عزموت، زیرا که مغنیان به اطراف اورشلیم به جهت خود دهات بنا کرده بودند.
\par 30 و کاهنان و لاویان خویشتن را تطهیرنمودند و قوم و دروازه‌ها و حصار را نیز تطهیرکردند.
\par 31 و من روسای یهودا را بر سر حصارآوردم و دو فرقه بزرگ از تسبیح خوانان معین کردم که یکی از آنها به طرف راست بر سر حصارتا دروازه خاکروبه به هیئت اجماعی رفتند.
\par 32 و در عقب ایشان، هوشعیا و نصف روسای یهودا.
\par 33 و عزریا و عزرا و مشلام.
\par 34 و یهودا وبنیامین شمعیا و ارمیا.
\par 35 و بعضی از پسران کاهنان با کرناها یعنی زکریا ابن یوناتان بن شمعیاابن متنیا ابن میکایا ابن زکور بن آصاف.
\par 36 وبرادران او شمعیا و عزریئیل و مللای و جللای وماعای و نتنئیل و یهودا و حنانی با آلات موسیقی داود مرد خدا، و عزرای کاتب پیش ایشان بود.
\par 37 پس ایشان نزد دروازه چشمه که برابر ایشان بود، بر زینه شهر داود بر فراز حصار بالای خانه داود، تا دروازه آب به طرف مشرق رفتند.
\par 38 وفرقه دوم، تسبیح خوانان در مقابل ایشان به هیئت اجماعی رفتند و من و نصف قوم بر سر حصار، ازنزد برج تنور تا حصار عریض در عقب ایشان رفتیم.
\par 39 و ایشان از بالای دروازه افرایم و بالای دروازه کهنه و بالای دروازه ماهی و برج حننئیل وبرج مئه تا دروازه گوسفندان (رفته )، نزد دروازه سجن توقف نمودند.
\par 40 پس هر دو فرقه تسبیح خوانان در خانه خدا ایستادند و من و نصف سروران ایستادیم.
\par 41 و الیاقیم و معسیا ومنیامین و میکایا و الیوعینای و زکریا وحننیای کهنه با کرناها،
\par 42 و معسیا و شمعیا والعازار و عزی و یوحانان و ملکیا و عیلام و عازر، و مغنیان و یزرحیای وکیل به آواز بلندسراییدند.
\par 43 و در آن روز، قربانی های عظیم گذرانیده، شادی نمودند، زیرا خدا ایشان را بسیار شادمان گردانیده بود و زنان و اطفال نیز شادی نمودند. پس شادمانی اورشلیم از جایهای دور مسموع شد.
\par 44 و در آن روز، کسانی چند بر حجره‌ها به جهت خزانه‌ها و هدایا و نوبرها و عشرها تعیین شدند تا حصه های کاهنان و لاویان را ازمزرعه های شهرها برحسب تورات در آنها جمع کنند، زیرا که یهودا درباره کاهنان و لاویانی که به خدمت می‌ایستادند، شادی می‌نمودند.
\par 45 وایشان با مغنیان و دربانان، موافق حکم داود وپسرش سلیمان، ودیعت خدای خود و لوازم تطهیر را نگاه داشتند.
\par 46 زیرا که در ایام داود وآساف از قدیم، روسای مغنیان بودند وسرودهای حمد و تسبیح برای خدا(می خواندند).و تمامی اسرائیل در ایام زربابل و در ایام نحمیا، حصه های مغنیان ودربانان را روز به روز می‌دادند و ایشان وقف به لاویان می‌دادند و لاویان وقف به بنی هارون می‌دادند.
\par 47 و تمامی اسرائیل در ایام زربابل و در ایام نحمیا، حصه های مغنیان ودربانان را روز به روز می‌دادند و ایشان وقف به لاویان می‌دادند و لاویان وقف به بنی هارون می‌دادند.
 
\chapter{13}

\par 1 در آن روز، کتاب موسی را به سمع قوم خواندند و در آن نوشته‌ای یافت شد که عمونیان و موآبیان تا به ابد به جماعت خدا داخل نشوند.
\par 2 چونکه ایشان بنی‌اسرائیل را به نان و آب استقبال نکردند، بلکه بلعام را به ضد ایشان اجیرنمودند تا ایشان را لعنت نماید اما خدای ما لعنت را به برکت تبدیل نمود.
\par 3 پس چون تورات راشنیدند، تمامی گروه مختلف را از میان اسرائیل جدا کردند.
\par 4 و قبل از این الیاشیب کاهن که بر حجره های خانه خدای ما تعیین شده بود، با طوبیا قرابتی داشت.
\par 5 و برای او حجره بزرگ ترتیب داده بودکه در آن قبل از آن هدایای آردی و بخور وظروف را و عشر گندم و شراب و روغن را که فریضه لاویان و مغنیان و دربانان بود و هدایای افراشتنی کاهنان را می‌گذاشتند.
\par 6 و در همه آن وقت، من در اورشلیم نبودم زیرا در سال سی ودوم ارتحشستا پادشاه بابل، نزد پادشاه رفتم و بعداز ایامی چند از پادشاه رخصت خواستم.
\par 7 وچون به اورشلیم رسیدم، از عمل زشتی که الیاشیب درباره طوبیا کرده بود، از اینکه حجره‌ای برایش در صحن خانه خدا ترتیب نموده بود، آگاه شدم.
\par 8 و این امر به نظر من بسیار ناپسند آمده، پس تمامی اسباب خانه طوبیا را از حجره بیرون ریختم.
\par 9 و امر فرمودم که حجره را تطهیر نمایندو ظروف خانه خدا و هدایا و بخور را در آن بازآوردم.
\par 10 و فهمیدم که حصه های لاویان را به ایشان نمی دادند و از این جهت، هر کدام از لاویان ومغنیانی که مشغول خدمت می‌بودند، به مزرعه های خویش فرار کرده بودند.
\par 11 پس باسروران مشاجره نموده، گفتم چرا درباره خانه خدا غفلت می‌نمایند. و ایشان را جمع کرده، درجایهای ایشان برقرار نمودم.
\par 12 و جمیع یهودیان، عشر گندم و عصیر انگور و روغن را درخزانه‌ها آوردند.
\par 13 و شلمیای کاهن و صادوق کاتب و فدایا را که از لاویان بود، بر خزانه هاگماشتم و به پهلوی ایشان، حانان بن زکور بن متنیارا، زیرا که مردم ایشان را امین می‌پنداشتند و کارایشان این بود که حصه های برادران خود را به ایشان بدهند.
\par 14 ‌ای خدایم مرا درباره این کار بیاد آور وحسناتی را که برای خانه خدای خود و وظایف آن کرده‌ام محو مساز.
\par 15 در آن روزها، در یهودا بعضی را دیدم که چرخشتها را در روز سبت می‌فشردند و بافه هامی آوردند و الاغها را بار می‌کردند و شراب وانگور و انجیر و هر گونه حمل را نیز در روز سبت به اورشلیم می‌آوردند. پس ایشان را به‌سبب فروختن ماکولات در آن روز تهدید نمودم.
\par 16 وبعضی از اهل صور که در آنجا ساکن بودند، ماهی و هرگونه بضاعت می‌آوردند و در روز سبت، به بنی یهودا و اهل اورشلیم می‌فروختند.
\par 17 پس با بزرگان یهودا مشاجره نمودم و به ایشان گفتم: «این چه عمل زشت است که شمامی کنید و روز سبت را بی‌حرمت می‌نمایید؟
\par 18 آیا پدران شما چنین نکردند و آیا خدای ما تمامی این بلا را بر ما و بر این شهر وارد نیاورد؟ وشما سبت را بی‌حرمت نموده، غضب را براسرائیل زیاد می‌کنید.»
\par 19 و هنگامی که دروازه های اورشلیم قبل از سبت سایه می‌افکند، امر فرمودم که دروازه‌ها را ببندند و قدغن کردم که آنها را تا بعد از سبت نگشایند و بعضی از خادمان خود را بر دروازه‌ها قرار دادم که هیچ بار در روزسبت آورده نشود.
\par 20 پس سوداگران و فروشندگان هرگونه بضاعت، یک دو دفعه بیرون از اورشلیم شب رابسر بردند.
\par 21 اما من ایشان را تهدید کرده، گفتم: «شما چرا نزد دیوار شب را بسر می‌برید؟ اگر باردیگر چنین کنید، دست بر شما می‌اندازم.» پس ازآنوقت دیگر در روز سبت نیامدند.
\par 22 و لاویان را امر فرمودم که خویشتن راتطهیر نمایند و آمده، دروازه‌ها را نگاهبانی کنندتا روز سبت تقدیس شود. ای خدایم این را نیزبرای من بیاد آور و برحسب کثرت رحمت خود، بر من ترحم فرما.
\par 23 در آن روزها نیز بعضی یهودیان را دیدم، که زنان از اشدودیان و عمونیان و موآبیان گرفته بودند.
\par 24 و نصف کلام پسران ایشان، در زبان اشدود می‌بود و به زبان یهود نمی توانستند به خوبی تکلم نمایند، بلکه به زبان این قوم و آن قوم.
\par 25 بنابراین با ایشان مشاجره نموده، ایشان را ملامت کردم و بعضی از ایشان را زدم و موی ایشان را کندم و ایشان را به خدا قسم داده، گفتم: «دختران خود را به پسران آنها مدهید و دختران آنها را به جهت پسران خود و به جهت خویشتن مگیرید.
\par 26 آیا سلیمان پادشاه اسرائیل در همین امر گناه نورزید با آنکه در امت های بسیارپادشاهی مثل او نبود؟ و اگر‌چه او محبوب خدای خود می‌بود و خدا او را به پادشاهی تمامی اسرائیل نصب کرده بود، زنان بیگانه او را نیزمرتکب گناه ساختند.
\par 27 پس آیا ما به شما گوش خواهیم گرفت که مرتکب این شرارت عظیم بشویم و زنان بیگانه گرفته، به خدای خویش خیانت ورزیم؟»
\par 28 و یکی از پسران یهویاداع بن الیاشیب رئیس کهنه، داماد سنبلط حورونی بود. پس او رااز نزد خود راندم.
\par 29 ‌ای خدای من ایشان را بیاد آور، زیرا که کهانت و عهد کهانت و لاویان را بی‌عصمت کرده‌اند.پس من ایشان را از هر چیز بیگانه طاهرساختم و وظایف کاهنان و لاویان را برقرارنمودم که هر کس بر خدمت خود حاضر شود.
\par 30 پس من ایشان را از هر چیز بیگانه طاهرساختم و وظایف کاهنان و لاویان را برقرارنمودم که هر کس بر خدمت خود حاضر شود.


\end{document}