\begin{document}

\title{Proverbs}

 
\chapter{1}

\par 1 امثال سلیمان بن داود پادشاه اسرائیل
\par 2 به جهت دانستن حکمت و عدل، و برای فهمیدن کلمات فطانت.
\par 3 به جهت اکتساب ادب معرفت آمیز، و عدالت و انصاف و استقامت.
\par 4 تاساده دلان را زیرکی بخشد، و جوانان را معرفت وتمیز.
\par 5 تا مرد حکیم بشنود و علم را بیفزاید. ومرد فهیم تدابیر را تحصیل نماید.
\par 6 تا امثال وکنایات را بفهمند، کلمات حکیمان و غوامض ایشان را.
\par 7 ترس یهوه آغاز علم است. لیکن جاهلان حکمت و ادب را خوار می‌شمارند.
\par 8 ‌ای پسر من تادیب پدر خود را بشنو، و تعلیم مادر خویش را ترک منما.
\par 9 زیرا که آنها تاج زیبایی برای سر تو، و جواهر برای گردن توخواهد بود.
\par 10 ‌ای پسر من اگر گناهکاران تو رافریفته سازند، قبول منما.
\par 11 اگر گویند: «همراه مابیا تا برای خون در کمین بنشینیم، و برای بی‌گناهان بی‌جهت پنهان شویم،
\par 12 مثل هاویه ایشان را زنده خواهیم بلعید، و تندرست مانندآنانی که به گور فرو می‌روند.
\par 13 هر گونه اموال نفیسه را پیدا خواهیم نمود. و خانه های خود را ازغنیمت مملو خواهیم ساخت.
\par 14 قرعه خود رادر میان ما بینداز. و جمیع ما را یک کیسه خواهدبود.»
\par 15 ‌ای پسر من با ایشان در راه مرو. و پای خودرا از طریقهای ایشان باز دار
\par 16 زیرا که پایهای ایشان برای شرارت می‌دود و به جهت ریختن خون می‌شتابد.
\par 17 به تحقیق، گستردن دام در نظرهر بالداری بی‌فایده است.
\par 18 لیکن ایشان به جهت خون خود کمین می‌سازند، و برای جان خویش پنهان می‌شوند.
\par 19 همچنین است راههای هر کس که طماع سود باشد، که آن جان مالک خود را هلاک می‌سازد.
\par 20 حکمت در بیرون ندا می‌دهد و در شوارع عام آواز خود را بلند می‌کند.
\par 21 در سرچهارراهها در دهنه دروازه‌ها می‌خواند و در شهربه سخنان خود متکلم می‌شود
\par 22 که «ای جاهلان تا به کی جهالت را دوست خواهید داشت؟ و تا به کی مستهزئین از استهزا شادی می‌کنند و احمقان از معرفت نفرت می‌نمایند؟
\par 23 به‌سبب عتاب من بازگشت نمایید. اینک روح خود را بر شما افاضه خواهم نمود و کلمات خود را بر شما اعلام خواهم کرد.
\par 24 زیرا که چون خواندم، شما ابانمودید و دستهای خود را افراشتم و کسی اعتنانکرد.
\par 25 بلکه تمامی نصیحت مرا ترک نمودید وتوبیخ مرا نخواستید.
\par 26 پس من نیز در حین مصیبت شما خواهم خندید و چون ترس بر شمامستولی شود استهزا خواهم نمود.
\par 27 چون خوف مثل باد تند بر شما عارض شود، و مصیبت مثل گردباد به شما دررسد، حینی که تنگی وضیق بر شما آید.
\par 28 آنگاه مرا خواهند خواندلیکن اجابت نخواهم کرد، و صبحگاهان مراجستجو خواهند نمود اما مرا نخواهند یافت.
\par 29 چونکه معرفت را مکروه داشتند، و ترس خداوند را اختیار ننمودند،
\par 30 و نصیحت مراپسند نکردند، و تمامی توبیخ مرا خوار شمردند،
\par 31 بنابراین، از میوه طریق خود خواهند خورد، واز تدابیر خویش سیر خواهند شد.
\par 32 زیرا که ارتداد جاهلان، ایشان را خواهد کشت و راحت غافلانه احمقان، ایشان را هلاک خواهد ساخت.اما هر‌که مرا بشنود در امنیت ساکن خواهدبود، و از ترس بلا مستریح خواهد ماند.»
\par 33 اما هر‌که مرا بشنود در امنیت ساکن خواهدبود، و از ترس بلا مستریح خواهد ماند.»
 
\chapter{2}

\par 1 ای پسر من اگر سخنان مرا قبول می‌نمودی و اوامر مرا نزد خود نگاه می‌داشتی،
\par 2 تاگوش خود را به حکمت فرا گیری و دل خود را به فطانت مایل گردانی،
\par 3 اگر فهم را دعوت می‌کردی و آواز خود را به فطانت بلندمی نمودی،
\par 4 اگر آن را مثل نقره می‌طلبیدی ومانند خزانه های مخفی جستجو می‌کردی،
\par 5 آنگاه ترس خداوند را می‌فهمیدی، و معرفت خدا را حاصل می‌نمودی.
\par 6 زیرا خداوندحکمت را می‌بخشد، و از دهان وی معرفت وفطانت صادر می‌شود.
\par 7 به جهت مستقیمان، حکمت کامل را ذخیره می‌کند و برای آنانی که درکاملیت سلوک می‌نمایند، سپر می‌باشد،
\par 8 تاطریقهای انصاف را محافظت نماید و طریق مقدسان خویش را نگاه دارد.
\par 9 پس آنگاه عدالت و انصاف را می‌فهمیدی، و استقامت و هر طریق نیکو را.
\par 10 زیرا که حکمت به دل تو داخل می‌شد و معرفت نزد جان تو عزیزمی گشت.
\par 11 تمیز، تو را محافظت می‌نمود، وفطانت، تو را نگاه می‌داشت،
\par 12 تا تو را از راه شریر رهایی بخشد، و از کسانی که به سخنان کج متکلم می‌شوند.
\par 13 که راههای راستی را ترک می‌کنند، و به طریقهای تاریکی سالک می‌شوند.
\par 14 از عمل بد خشنودند، و از دروغهای شریرخرسندند.
\par 15 که در راههای خود معوجند، و درطریقهای خویش کج رو می‌باشند.
\par 16 تا تو را اززن اجنبی رهایی بخشد، و از زن بیگانه‌ای که سخنان تملق‌آمیز می‌گوید؛
\par 17 که مصاحب جوانی خود را ترک کرده، و عهد خدای خویش را فراموش نموده است.
\par 18 زیرا خانه او به موت فرو می‌رود و طریقهای او به مردگان.
\par 19 کسانی که نزد وی روند برنخواهند گشت، و به طریقهای حیات نخواهند رسید.
\par 20 تا به راه صالحان سلوک نمایی و طریقهای عادلان را نگاه داری.
\par 21 زیراکه راستان در زمین ساکن خواهند شد، و کاملان در آن باقی خواهند ماند.لیکن شریران از زمین منقطع خواهند شد، و ریشه خیانتکاران از آن کنده خواهد گشت.
\par 22 لیکن شریران از زمین منقطع خواهند شد، و ریشه خیانتکاران از آن کنده خواهد گشت.
 
\chapter{3}

\par 1 ای پسر من، تعلیم مرا فراموش مکن و دل تو اوامر مرا نگاه دارد،
\par 2 زیرا که طول ایام وسالهای حیات و سلامتی را برای تو خواهدافزود.
\par 3 زنهار که رحمت و راستی تو را ترک نکند. آنها را بر گردن خود ببند و بر لوح دل خودمرقوم دار.
\par 4 آنگاه نعمت و رضامندی نیکو، درنظر خدا و انسان خواهی یافت.
\par 5 به تمامی دل خود بر خداوند توکل نما و بر عقل خود تکیه مکن.
\par 6 در همه راههای خود او را بشناس، و اوطریقهایت را راست خواهد گردانید.
\par 7 خویشتن را حکیم مپندار، از خداوند بترس و از بدی اجتناب نما.
\par 8 این برای ناف تو شفا، وبرای استخوانهایت مغز خواهد بود.
\par 9 از مایملک خود خداوند را تکریم نما و از نوبرهای همه محصول خویش.
\par 10 آنگاه انبارهای تو به وفورنعمت پر خواهد شد، و چرخشتهای تو از شیره انگور لبریز خواهد گشت.
\par 11 ‌ای پسر من، تادیب خداوند را خوار مشمار، و توبیخ او را مکروه مدار.
\par 12 زیرا خداوند هر‌که را دوست داردتادیب می‌نماید، مثل پدر پسر خویش را که از اومسرور می‌باشد.
\par 13 خوشابحال کسی‌که حکمت را پیدا کند، و شخصی که فطانت را تحصیل نماید.
\par 14 زیرا که تجارت آن از تجارت نقره ومحصولش از طلای خالص نیکوتر است.
\par 15 ازلعلها گرانبهاتر است و جمیع نفایس تو با آن برابری نتواند کرد.
\par 16 به‌دست راست وی طول ایام است، و به‌دست چپش دولت و جلال.
\par 17 طریقهای وی طریقهای شادمانی است و همه راههای وی سلامتی می‌باشد.
\par 18 به جهت آنانی که او را به‌دست گیرند، درخت حیات‌است وکسی‌که به او متمسک می‌باشد خجسته است.
\par 19 خداوند به حکمت خود زمین را بنیاد نهاد، و به عقل خویش آسمان را استوار نمود.
\par 20 به علم او لجه‌ها منشق گردید، و افلاک شبنم رامی چکانید.
\par 21 ‌ای پسر من، این چیزها از نظر تودور نشود. حکمت کامل و تمیز را نگاه دار.
\par 22 پس برای جان تو حیات، و برای گردنت زینت خواهد بود.
\par 23 آنگاه در راه خود به امنیت سالک خواهی شد، و پایت نخواهد لغزید.
\par 24 هنگامی که بخوابی، نخواهی ترسید و چون دراز شوی خوابت شیرین خواهد شد.
\par 25 از خوف ناگهان نخواهی ترسید، و نه از خرابی شریران چون واقع شود.
\par 26 زیرا خداوند اعتماد تو خواهد بود و پای تو را از دام حفظ خواهد نمود.
\par 27 احسان را ازاهلش باز مدار، هنگامی که بجا آوردنش در قوت دست توست.
\par 28 به همسایه خود مگو برو وبازگرد، و فردا به تو خواهم داد، با آنکه نزد توحاضر است.
\par 29 بر همسایه ات قصد بدی مکن، هنگامی که او نزد تو در امنیت ساکن است.
\par 30 باکسی‌که به تو بدی نکرده است، بی‌سبب مخاصمه منما.
\par 31 بر مرد ظالم حسد مبر وهیچکدام از راههایش را اختیار مکن.
\par 32 زیراکج خلقان نزد خداوند مکروهند، لیکن سر او نزدراستان است،
\par 33 لعنت خداوند بر خانه شریران است. اما مسکن عادلان را برکت می‌دهد.
\par 34 یقین که مستهزئین را استهزا می‌نماید، اما متواضعان رافیض می‌بخشد.حکیمان وارث جلال خواهند شد، اما احمقان خجالت را خواهند برد.
\par 35 حکیمان وارث جلال خواهند شد، اما احمقان خجالت را خواهند برد.
 
\chapter{4}

\par 1 ای پسران، تادیب پدر را بشنوید و گوش دهید تا فطانت را بفهمید،
\par 2 چونکه تعلیم نیکو به شما می‌دهم. پس شریعت مرا ترک منمایید.
\par 3 زیرا که من برای پدر خود پسر بودم، ودر نظر مادرم عزیز و یگانه.
\par 4 و او مرا تعلیم داده، می‌گفت: «دل تو به سخنان من متمسک شود، واوامر مرا نگاه دار تا زنده بمانی.
\par 5 حکمت راتحصیل نما و فهم را پیدا کن. فراموش مکن و از کلمات دهانم انحراف مورز.
\par 6 آن را ترک منما که تو را محافظت خواهد نمود. آن را دوست دار که تو را نگاه خواهد داشت.
\par 7 حکمت از همه‌چیزافضل است. پس حکمت را تحصیل نما و به هرآنچه تحصیل نموده باشی، فهم را تحصیل کن.
\par 8 آن را محترم دار، و تو را بلند خواهد ساخت. واگر او را در آغوش بکشی تو را معظم خواهدگردانید.
\par 9 بر سر تو تاج زیبایی خواهد نهاد. وافسر جلال به تو عطا خواهد نمود.»
\par 10 ‌ای پسر من بشنو و سخنان مرا قبول نما، که سالهای عمرت بسیار خواهد شد.
\par 11 راه حکمت را به تو تعلیم دادم، و به طریقهای راستی تو راهدایت نمودم.
\par 12 چون در راه بروی قدمهای توتنگ نخواهد شد، و چون بدوی لغزش نخواهی خورد.
\par 13 ادب را به چنگ آور و آن را فرو مگذار. آن را نگاه دار زیرا که حیات تو است.
\par 14 به راه شریران داخل مشو، و در طریق گناهکاران سالک مباش.
\par 15 آن را ترک کن و به آن گذر منما، و از آن اجتناب کرده، بگذر.
\par 16 زیرا که ایشان تا بدی نکرده باشند، نمی خوابند و اگر کسی را نلغزانیده باشند، خواب از ایشان منقطع می‌شود.
\par 17 چونکه نان شرارت را می‌خورند، و شراب ظلم رامی نوشند.
\par 18 لیکن طریق عادلان مثل نور مشرق است که تا نهار کامل روشنایی آن در تزایدمی باشد.
\par 19 و اما طریق شریران مثل ظلمت غلیظاست، و نمی دانند که از چه چیز می‌لغزند.
\par 20 ‌ای پسر من، به سخنان من توجه نما و گوش خود را به کلمات من فرا گیر.
\par 21 آنها از نظر تو دورنشود. آنها را در اندرون دل خود نگاه دار.
\par 22 زیراهر‌که آنها را بیابد برای او حیات‌است، و برای تمامی جسد او شفا می‌باشد.
\par 23 دل خود را به حفظ تمام نگاه دار، زیرا که مخرج های حیات ازآن است.
\par 24 دهان دروغگو را از خود بینداز، ولبهای کج را از خویشتن دور نما.
\par 25 چشمانت به استقامت نگران باشد، و مژگانت پیش روی توراست باشد.
\par 26 طریق پایهای خود را همواربساز، تا همه طریقهای تو مستقیم باشد.به طرف راست یا چپ منحرف مشو، و پای خود رااز بدی نگاه دار.
\par 27 به طرف راست یا چپ منحرف مشو، و پای خود رااز بدی نگاه دار.
 
\chapter{5}

\par 1 ای پسر من، به حکمت من توجه نما، و گوش خود را به فطانت من فراگیر،
\par 2 تاتدابیر را محافظت نمایی، و لبهایت معرفت رانگاه دارد.
\par 3 زیرا که لبهای زن اجنبی عسل رامی چکاند، و دهان او از روغن ملایم تر است.
\par 4 لیکن آخر او مثل افسنتین تلخ است و برنده مثل شمشیر دودم.
\par 5 پایهایش به موت فرو می‌رود، وقدمهایش به هاویه متمسک می‌باشد.
\par 6 به طریق حیات هرگز سالک نخواهد شد. قدمهایش آواره شده است و او نمی داند.
\par 7 و الان‌ای پسرانم مرابشنوید، و از سخنان دهانم انحراف مورزید.
\par 8 طریق خود را از او دور ساز، و به در خانه اونزدیک مشو.
\par 9 مبادا عنفوان جوانی خود را به دیگران بدهی، و سالهای خویش را به ستم کیشان. 
\par 10 و غریبان از اموال تو سیر شوند، و ثمره محنت تو به خانه بیگانه رود.
\par 11 که در عاقبت خودنوحه گری نمایی، هنگامی که گوشت و بدنت فانی شده باشد،
\par 12 و گویی چرا ادب را مکروه داشتم، و دل من تنبیه را خوار شمرد،
\par 13 و آواز مرشدان خود را نشنیدم، و به معلمان خود گوش ندادم.
\par 14 نزدیک بود که هر گونه بدی را مرتکب شوم، در میان قوم و جماعت.
\par 15 آب را از منبع خود بنوش، و نهرهای جاری را از چشمه خویش.
\par 16 جویهای تو بیرون خواهد ریخت، و نهرهای آب در شوارع عام،
\par 17 و از آن خودت به تنهایی خواهد بود، و نه از آن غریبان با تو.
\par 18 چشمه تو مبارک باشد، و از زن جوانی خویش مسرور باش،
\par 19 مثل غزال محبوب و آهوی جمیل. پستانهایش تو را همیشه خرم سازد، و از محبت او دائم محفوظ باش.
\par 20 لیکن‌ای پسر من، چرا از زن بیگانه فریفته شوی؟ و سینه زن غریب را در بر‌گیری؟
\par 21 زیراکه راههای انسان در مدنظر خداوند است، وتمامی طریقهای وی را می‌سنجد.
\par 22 تقصیرهای شریر او را گرفتار می‌سازد، و به بندهای گناهان خود بسته می‌شود.او بدون ادب خواهد مرد، و به کثرت حماقت خویش تلف خواهد گردید.
\par 23 او بدون ادب خواهد مرد، و به کثرت حماقت خویش تلف خواهد گردید.
 
\chapter{6}

\par 1 ای پسرم، اگر برای همسایه خود ضامن شده، و به جهت شخص بیگانه دست داده باشی،
\par 2 و از سخنان دهان خود در دام افتاده، و ازسخنان دهانت گرفتار شده باشی،
\par 3 پس‌ای پسرمن، این را بکن و خویشتن را رهایی ده چونکه به‌دست همسایه ات افتاده‌ای. برو و خویشتن رافروتن ساز و از همسایه خود التماس نما.
\par 4 خواب را به چشمان خود راه مده، و نه پینکی رابه مژگان خویش.
\par 5 مثل آهو خویشتن را از کمندو مانند گنجشک از دست صیاد خلاص کن.
\par 6 ‌ای شخص کاهل نزد مورچه برو، و درراههای او تامل کن و حکمت را بیاموز،
\par 7 که وی را پیشوایی نیست و نه سرور و نه حاکمی.
\par 8 اماخوراک خود را تابستان مهیا می‌سازد و آذوقه خویش را در موسم حصاد جمع می‌کند.
\par 9 ‌ای کاهل، تا به چند خواهی خوابید و از خواب خودکی خواهی برخاست؟
\par 10 اندکی خفت و اندکی خواب، و اندکی بر هم نهادن دستها به جهت خواب.
\par 11 پس فقر مثل راهزن بر تو خواهد آمد، و نیازمندی بر تو مانند مرد مسلح.
\par 12 مرد لئیم و مرد زشت خوی، با اعوجاج دهان رفتار می‌کند.
\par 13 با چشمان خود غمزه می‌زند و با پایهای خویش حرف می‌زند. باانگشتهای خویش اشاره می‌کند.
\par 14 در دلش دروغها است و پیوسته شرارت را اختراع می‌کند. نزاعها را می‌پاشد.
\par 15 بنابراین مصیبت بر او ناگهان خواهد آمد. در لحظه‌ای منکسر خواهد شد وشفا نخواهد یافت.
\par 16 شش چیز است که خداوند از آنها نفرت دارد، بلکه هفت چیز که نزد جان وی مکروه است.
\par 17 چشمان متکبر و زبان دروغگو، ودستهایی که خون بی‌گناه را می‌ریزد.
\par 18 دلی که تدابیر فاسد را اختراع می‌کند. پایهایی که درزیان کاری تیزرو می‌باشند.
\par 19 شاهد دروغگو که به کذب متکلم شود. و کسی‌که در میان برادران نزاعها بپاشد.
\par 20 ‌ای پسر من اوامر پدر خود را نگاه دار وتعلیم مادر خویش را ترک منما.
\par 21 آنها را بر دل خود دائم ببند، و آنها را بر گردن خویش بیاویز.
\par 22 حینی که به راه می‌روی تو را هدایت خواهد نمود، و حینی که می‌خوابی بر تو دیده بانی خواهد کرد، و وقتی که بیدار شوی با تو مکالمه خواهد نمود.
\par 23 زیرا که احکام (ایشان ) چراغ وتعلیم (ایشان ) نور است، و توبیخ تدبیرآمیز طریق حیات‌است.
\par 24 تا تو را از زن خبیثه نگاه دارد، و ازچاپلوسی زبان زن بیگانه.
\par 25 در دلت مشتاق جمال وی مباش، و از پلکهایش فریفته مشو،
\par 26 زیرا که به‌سبب زن زانیه، شخص برای یک قرص نان محتاج می‌شود، و زن مرد دیگر، جان گرانبها را صید می‌کند.
\par 27 آیا کسی آتش را درآغوش خود بگیرد و جامه‌اش سوخته نشود؟
\par 28 یا کسی بر اخگرهای سوزنده راه رود وپایهایش سوخته نگردد؟
\par 29 همچنین است کسی‌که نزد زن همسایه خویش داخل شود، زیرا هر‌که او را لمس نماید بی‌گناه نخواهد ماند.
\par 30 دزد را اهانت نمی کنند اگر دزدی کند تاجان خود را سیر نماید وقتی که گرسنه باشد.
\par 31 لیکن اگر گرفته شود، هفت چندان رد خواهدنمود و تمامی اموال خانه خود را خواهد داد.
\par 32 اما کسی‌که با زنی زنا کند، ناقص العقل است وهر‌که چنین عمل نماید، جان خود را هلاک خواهد ساخت.
\par 33 او ضرب و رسوایی خواهدیافت، و ننگ او محو نخواهد شد.
\par 34 زیرا که غیرت، شدت خشم مرد است و در روز انتقام، شفقت نخواهد نمود.بر هیچ کفاره‌ای نظرنخواهد کرد و هر‌چند عطایا را زیاده کنی، قبول نخواهد نمود.
\par 35 بر هیچ کفاره‌ای نظرنخواهد کرد و هر‌چند عطایا را زیاده کنی، قبول نخواهد نمود.
 
\chapter{7}

\par 1 ای پسر من سخنان مرا نگاه دار، و اوامر مرانزد خود ذخیره نما.
\par 2 اوامر مرا نگاه دار تازنده بمانی، و تعلیم مرا مثل مردمک چشم خویش.
\par 3 آنها را بر انگشتهای خود ببند و آنها رابر لوح قلب خود مرقوم دار.
\par 4 به حکمت بگو که تو خواهر من هستی و فهم را دوست خویش بخوان
\par 5 تا تو را از زن اجنبی نگاه دارد، و از زن غریبی که سخنان تملق‌آمیز می‌گوید.
\par 6 زیرا که از دریچه خانه خود نگاه کردم، و ازپشت شبکه خویش.
\par 7 در میان جاهلان دیدم، ودر میان جوانان، جوانی ناقص العقل مشاهده نمودم،
\par 8 که در کوچه بسوی گوشه او می‌گذشت. و به راه خانه وی می‌رفت،
\par 9 در شام در حین زوال روز، در سیاهی شب و در ظلمت غلیظ،
\par 10 که اینک زنی به استقبال وی می‌آمد، در لباس زانیه ودر خباثت دل.
\par 11 زنی یاوه‌گو و سرکش که پایهایش در خانه‌اش قرار نمی گیرد.
\par 12 گاهی درکوچه‌ها و گاهی در شوارع عام، و نزد هر گوشه‌ای در کمین می‌باشد.
\par 13 پس او را بگرفت و بوسید وچهره خود را بی‌حیا ساخته، او را گفت:
\par 14 «نزدمن ذبایح سلامتی است، زیرا که امروز نذرهای خود را وفا نمودم.
\par 15 از این جهت به استقبال توبیرون آمدم، تا روی تو را به سعی تمام بطلبم وحال تو را یافتم.
\par 16 بر بستر خود دوشکهاگسترانیده‌ام، با دیباها از کتان مصری.
\par 17 بسترخود را با مر و عود و سلیخه معطر ساخته‌ام.
\par 18 بیاتا صبح از عشق سیر شویم، و خویشتن را ازمحبت خرم سازیم.
\par 19 زیرا صاحبخانه در خانه نیست، و سفر دور رفته است.
\par 20 کیسه نقره‌ای به‌دست گرفته و تا روز بدر تمام مراجعت نخواهدنمود.»
\par 21 پس او را از زیادتی سخنانش فریفته کرد، واز تملق لبهایش او را اغوا نمود.
\par 22 در ساعت ازعقب او مثل گاوی که به سلاخ خانه می‌رود، روانه شد و مانند احمق به زنجیرهای قصاص.
\par 23 تا تیر به جگرش فرو رود، مثل گنجشکی که به دام می‌شتابد و نمی داند که به خطر جان خود می‌رود.
\par 24 پس حال‌ای پسران مرا بشنوید، و به سخنان دهانم توجه نمایید.
\par 25 دل تو به راههایش مایل نشود، و به طریقهایش گمراه مشو،
\par 26 زیراکه او بسیاری را مجروح‌انداخته است، و جمیع کشتگانش زورآورانند.خانه او طریق هاویه است و به حجره های موت مودی می‌باشد.
\par 27 خانه او طریق هاویه است و به حجره های موت مودی می‌باشد.
 
\chapter{8}

\par 1 آیا حکمت ندا نمی کند، و فطانت آوازخود را بلند نمی نماید؟
\par 2 به‌سر مکان های بلند، به کناره راه، در میان طریقها می‌ایستد.
\par 3 به‌جانب دروازه‌ها به دهنه شهر، نزد مدخل دروازه‌ها صدا می‌زند.
\par 4 که شما را‌ای مردان می‌خوانم و آواز من به بنی آدم است.
\par 5 ‌ای جاهلان زیرکی را بفهمید و‌ای احمقان عقل رادرک نمایید.
\par 6 بشنوید زیرا که به امور عالیه تکلم می‌نمایم و گشادن لبهایم استقامت است.
\par 7 دهانم به راستی تنطق می‌کند و لبهایم شرارت را مکروه می‌دارد.
\par 8 همه سخنان دهانم بر‌حق است و درآنها هیچ‌چیز کج یا معوج نیست.
\par 9 تمامی آنهانزد مرد فهیم واضح است و نزد یابندگان معرفت مستقیم است.
\par 10 تادیب مرا قبول کنید و نه نقره را، و معرفت را بیشتر از طلای خالص.
\par 11 زیرا که حکمت از لعلها بهتر است، و جمیع نفایس را به او برابر نتوان کرد.
\par 12 من حکمتم و در زیرکی سکونت دارم، و معرفت تدبیر را یافته‌ام.
\par 13 ترس خداوند، مکروه داشتن بدی است. غرور و تکبر وراه بد و دهان دروغگو را مکروه می‌دارم.
\par 14 مشورت و حکمت کامل از آن من است. من فهم هستم و قوت از آن من است.
\par 15 به من پادشاهان سلطنت می‌کنند، و داوران به عدالت فتوا می‌دهند.
\par 16 به من سروران حکمرانی می‌نمایند و شریفان و جمیع داوران جهان.
\par 17 من دوست می‌دارم آنانی را که مرا دوست می‌دارند. وهر‌که مرا به جد و جهد بطلبد مرا خواهد یافت.
\par 18 دولت و جلال با من است. توانگری جاودانی وعدالت.
\par 19 ثمره من از طلا و زر ناب بهتر است، وحاصل من از نقره خالص.
\par 20 در طریق عدالت می‌خرامم، در میان راههای انصاف،
\par 21 تا مال حقیقی را نصیب محبان خود گردانم، وخزینه های ایشان را مملو سازم.
\par 22 خداوند مرا مبداء طریق خود داشت، قبل از اعمال خویش از ازل.
\par 23 من از ازل برقرار بودم، از ابتدا پیش از بودن جهان.
\par 24 هنگامی که لجه هانبود من مولود شدم، وقتی که چشمه های پر ازآب وجود نداشت.
\par 25 قبل از آنگاه کوهها برپاشود، پیش از تلها مولود گردیدم.
\par 26 چون زمین وصحراها را هنوز نساخته بود، و نه اول غبار ربع مسکون را.
\par 27 وقتی که او آسمان را مستحکم ساخت من آنجا بودم، و هنگامی که دایره را برسطح لجه قرار داد.
\par 28 وقتی که افلاک را بالااستوار کرد، و چشمه های لجه را استوار گردانید.
\par 29 چون به دریا حد قرار داد، تا آبها از فرمان اوتجاوز نکنند، و زمانی که بنیاد زمین را نهاد.
\par 30 آنگاه نزد او معمار بودم، و روزبروز شادی می‌نمودم، و همیشه به حضور او اهتزاز می‌کردم.
\par 31 و اهتزاز من در آبادی زمین وی، و شادی من بابنی آدم می‌بود.
\par 32 پس الان‌ای پسران مرا بشنوید، وخوشابحال آنانی که طریقهای مرا نگاه دارند.
\par 33 تادیب را بشنوید و حکیم باشید، و آن را ردمنمایید.
\par 34 خوشابحال کسی‌که مرا بشنود، و هرروز نزد درهای من دیده بانی کند، و باهوهای دروازه های مرا محافظت نماید.
\par 35 زیرا هر‌که مرایابد حیات را تحصیل کند، و رضامندی خداوندرا حاصل نماید.و اما کسی‌که مرا خطا کند، به‌جان خود ضرر رساند، و هر‌که مرا دشمن دارد، موت را دوست دارد.
\par 36 و اما کسی‌که مرا خطا کند، به‌جان خود ضرر رساند، و هر‌که مرا دشمن دارد، موت را دوست دارد.
 
\chapter{9}

\par 1 حکمت، خانه خود را بنا کرده، و هفت ستونهای خویش را تراشیده است.
\par 2 ذبایح خود را ذبح نموده و شراب خود را ممزوج ساخته و سفره خود را نیز آراسته است.
\par 3 کنیزان خود را فرستاده، ندا کرده است، بر پشتهای بلندشهر.
\par 4 هر‌که جاهل باشد به اینجا بیاید، و هر‌که ناقص العقل است او را می‌گوید.
\par 5 بیایید از غذای من بخورید، و از شرابی که ممزوج ساخته‌ام بنوشید.
\par 6 جهالت را ترک کرده، زنده بمانید، و به طریق فهم سلوک نمایید.
\par 7 هر‌که استهزاکننده راتادیب نماید، برای خویشتن رسوایی را تحصیل کند، و هر‌که شریر را تنبیه نماید برای او عیب می‌باشد.
\par 8 استهزاکننده را تنبیه منما مبادا از تونفرت کند، اما مرد حکیم را تنبیه نما که تو رادوست خواهد داشت.
\par 9 مرد حکیم را پند ده که زیاده حکیم خواهد شد. مرد عادل را تعلیم ده که علمش خواهد افزود.
\par 10 ابتدای حکمت ترس خداوند است، و معرفت قدوس فطانت می‌باشد.
\par 11 زیرا که به واسطه من روزهای تو کثیر خواهدشد، و سالهای عمر از برایت زیاده خواهدگردید.
\par 12 اگر حکیم هستی، برای خویشتن حکیم هستی. و اگر استهزا نمایی به تنهایی متحمل آن خواهی بود.
\par 13 زن احمق یاوه‌گو است، جاهل است و هیچ نمی داند، 
\par 14 و نزد در خانه خود می‌نشیند، درمکانهای بلند شهر بر کرسی،
\par 15 تا راه روندگان رابخواند، و آنانی را که به راههای خود براستی می‌روند.
\par 16 هر‌که جاهل باشد به اینجا برگردد، وبه ناقص العقل می‌گوید:
\par 17 «آبهای دزدیده شده شیرین است، و نان خفیه لذیذ می‌باشد.»و اونمی داند که مردگان در آنجا هستند، ودعوت‌شدگانش در عمقهای هاویه می‌باشند.
\par 18 و اونمی داند که مردگان در آنجا هستند، ودعوت‌شدگانش در عمقهای هاویه می‌باشند.
 
\chapter{10}

\par 1 امثال سلیمان: پسر حکیم پدر خود رامسرور می‌سازد، اما پسر احمق باعث حزن مادرش می‌شود.
\par 2 گنجهای شرارت منفعت ندارد، اما عدالت از موت رهایی می‌دهد.
\par 3 خداوند جان مرد عادل را نمی گذارد گرسنه بشود، اما آرزوی شریران را باطل می‌سازد.
\par 4 کسی‌که به‌دست سست کار می‌کند فقیرمی گردد، اما دست چابک غنی می‌سازد.
\par 5 کسی‌که در تابستان جمع کند پسر عاقل است، اما کسی‌که در موسم حصاد می‌خوابد، پسر شرم‌آورنده است.
\par 6 بر سر عادلان برکت‌ها است، اما ظلم دهان شریران را می‌پوشاند.
\par 7 یادگار عادلان مبارک است، اما اسم شریران خواهد گندید.
\par 8 دانادل، احکام را قبول می‌کند، اما احمق پرگو تلف خواهد شد.
\par 9 کسی‌که به راستی راه رود، در امنیت سالک گردد، و کسی‌که راه خود را کج می‌سازد آشکارخواهد شد.
\par 10 هر‌که چشمک می‌زند الم می‌رساند، امااحمق پرگو تلف می‌شود.
\par 11 دهان عادلان چشمه حیات‌است، اما ظلم دهان شریران را می‌پوشاند.
\par 12 بغض نزاعها می‌انگیزاند، اما محبت هر گناه را مستور می‌سازد.
\par 13 در لبهای فطانت پیشگان حکمت یافت می‌شود، اما چوب به جهت پشت مردناقص العقل است.
\par 14 حکیمان علم را ذخیره می‌کنند، اما دهان احمق نزدیک به هلاکت است.
\par 15 اموال دولتمندان شهر حصاردار ایشان می‌باشد، اما بینوایی فقیران هلاکت ایشان است.
\par 16 عمل مرد عادل مودی به حیات‌است، امامحصول شریر به گناه می‌انجامد.
\par 17 کسی‌که تادیب را نگاه دارد در طریق حیات‌است، اما کسی‌که تنبیه را ترک نماید گمراه می‌شود.
\par 18 کسی‌که بغض را می‌پوشاند دروغگومی باشد. کسی‌که بهتان را شیوع دهد احمق است.
\par 19 کثرت کلام از گناه خالی نمی باشد، اما آنکه لبهایش را ضبط نماید عاقل است.
\par 20 زبان عادلان نقره خالص است، اما دل شریران لاشی ء می‌باشد.
\par 21 لبهای عادلان بسیاری را رعایت می‌کند، اما احمقان از بی‌عقلی می‌میرند.
\par 22 برکت خداوند دولتمند می‌سازد، و هیچ زحمت بر آن نمی افزاید.
\par 23 جاهل در عمل بد اهتزاز دارد، و صاحب فطانت در حکمت.
\par 24 خوف شریران به ایشان می‌رسد، و آرزوی عادلان به ایشان عطا خواهد شد.
\par 25 مثل گذشتن گردباد، شریر نابود می‌شود، امامرد عادل بنیاد جاودانی است.
\par 26 چنانکه سرکه برای دندان و دود برای چشمان است، همچنین است مرد کاهل برای آنانی که او را می‌فرستند.
\par 27 ترس خداوند عمر را طویل می‌سازد، اماسالهای شریران کوتاه خواهد شد.
\par 28 انتظار عادلان شادمانی است، اما امیدشریران ضایع خواهد شد.
\par 29 طریق خداوند به جهت کاملان قلعه است، اما به جهت عاملان شر هلاکت می‌باشد.
\par 30 مرد عادل هرگز متحرک نخواهد شد، اماشریران در زمین ساکن نخواهند گشت.
\par 31 دهان صدیقان حکمت را می‌رویاند، امازبان دروغگویان از ریشه‌کنده خواهد شد.لبهای عادلان به امور مرضیه عارف است، اما دهان شریران پر از دروغ‌ها است.
\par 32 لبهای عادلان به امور مرضیه عارف است، اما دهان شریران پر از دروغ‌ها است.
 
\chapter{11}

\par 1 ترازوی با تقلب نزد خداوند مکروه است، اما سنگ تمام پسندیده او است.
\par 2 چون تکبر می‌آید خجالت می‌آید، اما حکمت با متواضعان است.
\par 3 کاملیت راستان ایشان را هدایت می‌کند، اماکجی خیانتکاران ایشان را هلاک می‌سازد.
\par 4 توانگری در روز غضب منفعت ندارد، اماعدالت از موت رهایی می‌بخشد.
\par 5 عدالت مرد کامل طریق او را راست می‌سازد، اما شریر از شرارت خود هلاک می‌گردد.
\par 6 عدالت راستان ایشان را خلاصی می‌بخشد، اما خیانتکاران در خیانت خود گرفتار می‌شوند.
\par 7 چون مرد شریر بمیرد امید او نابود می‌گردد، و انتظار زورآوران تلف می‌شود.
\par 8 مرد عادل از تنگی خلاص می‌شود و شریربه‌جای او می‌آید.
\par 9 مرد منافق به دهانش همسایه خود را هلاک می‌سازد، و عادلان به معرفت خویش نجات می‌یابند.
\par 10 از سعادتمندی عادلان، شهر شادی می‌کند، و از هلاکت شریران ابتهاج می‌نماید.
\par 11 از برکت راستان، شهر مرتفع می‌شود، اما ازدهان شریران منهدم می‌گردد.
\par 12 کسی‌که همسایه خود را حقیر شماردناقص العقل می‌باشد، اما صاحب فطانت ساکت می‌ماند.
\par 13 کسی‌که به نمامی گردش می‌کند، سرها رافاش می‌سازد، اما شخص امین دل، امر را مخفی می‌دارد.
\par 14 جایی که تدبیر نیست مردم می‌افتند، اماسلامتی از کثرت مشیران است.
\par 15 کسی‌که برای غریب ضامن شود البته ضررخواهد یافت، و کسی‌که ضمانت را مکروه داردایمن می‌باشد.
\par 16 زن نیکوسیرت عزت را نگاه می‌دارد، چنانکه زورآوران دولت را محافظت می‌نمایند.
\par 17 مرد رحیم به خویشتن احسان می‌نماید، امامرد ستم کیش جسد خود را می‌رنجاند.
\par 18 شریر اجرت فریبنده تحصیل می‌کند، اماکارنده عدالت مزد حقیقی را.
\par 19 چنانکه عدالت مودی به حیات‌است، همچنین هر‌که شرارت را پیروی نماید او را به موت می‌رساند.
\par 20 کج خلقان نزد خداوند مکروهند، اماکاملان طریق پسندیده او می‌باشند.
\par 21 یقین شریر مبرا نخواهد شد، اما ذریت عادلان نجات خواهند یافت.
\par 22 زن جمیله بی‌عقل حلقه زرین است در بینی گراز.
\par 23 آرزوی عادلان نیکویی محض است، اماانتظار شریران، غضب می‌باشد.
\par 24 هستند که می‌پاشند و بیشتر می‌اندوزند وهستند که زیاده از آنچه شاید نگاه می‌دارند اما به نیازمندی می‌انجامد.
\par 25 شخص سخی فربه می‌شود، و هر‌که سیراب می‌کند خود نیز سیراب خواهد گشت.
\par 26 هر‌که غله را نگاه دارد مردم او را لعنت خواهند کرد، اما بر سر فروشنده آن برکت خواهدبود.
\par 27 کسی‌که نیکویی را بطلبد رضامندی رامی جوید، و هر‌که بدی را بطلبد بر او عارض خواهد شد.
\par 28 کسی‌که بر توانگری خود توکل کند، خواهد افتاد، اما عادلان مثل برگ سبز شکوفه خواهند‌آورد.
\par 29 هر‌که اهل خانه خود را برنجاند نصیب او باد خواهد بود، و احمق بنده حکیم دلان خواهدشد.
\par 30 ثمره مرد عادل درخت حیات‌است، وکسی‌که جانها را صید کند حکیم است.اینک مرد عادل بر زمین جزا خواهد یافت، پس چند مرتبه زیاده مرد شریر و گناهکار.
\par 31 اینک مرد عادل بر زمین جزا خواهد یافت، پس چند مرتبه زیاده مرد شریر و گناهکار.
 
\chapter{12}

\par 1 هر‌که تادیب را دوست می‌دارد معرفت را دوست می‌دارد، اما هر‌که از تنبیه نفرت کند وحشی است.
\par 2 مرد نیکو رضامندی خداوند را تحصیل می‌نماید، اما او صاحب تدبیر فاسد را ملزم خواهد ساخت.
\par 3 انسان از بدی استوار نمی شود، اما ریشه عادلان جنبش نخواهد خورد.
\par 4 زن صالحه تاج شوهر خود می‌باشد، اما زنی که خجل سازد مثل پوسیدگی در استخوانهایش می‌باشد.
\par 5 فکرهای عادلان انصاف است، اما تدابیرشریران فریب است.
\par 6 سخنان شریران برای خون در کمین است، اما دهان راستان ایشان را رهایی می‌دهد.
\par 7 شریران واژگون شده، نیست می‌شوند، اماخانه عادلان برقرار می‌ماند.
\par 8 انسان برحسب عقلش ممدوح می‌شود، اماکج دلان خجل خواهند گشت.
\par 9 کسی‌که حقیر باشد و خادم داشته باشد، بهتراست از کسی‌که خویشتن را برافرازد و محتاج نان باشد.
\par 10 مرد عادل برای جان حیوان خود تفکرمی کند، اما رحمتهای شریران ستم کیشی است.
\par 11 کسی‌که زمین خود را زرع کند از نان سیرخواهد شد، اما هر‌که اباطیل را پیروی نمایدناقص العقل است.
\par 12 مرد شریر به شکار بدکاران طمع می‌ورزد، اما ریشه عادلان میوه می‌آورد.
\par 13 در تقصیر لبها دام مهلک است، اما مردعادل از تنگی بیرون می‌آید.
\par 14 انسان از ثمره دهان خود از نیکویی سیرمی شود، و مکافات دست انسان به او رد خواهدشد.
\par 15 راه احمق در نظر خودش راست است، اماهر‌که نصیحت را بشنود حکیم است.
\par 16 غضب احمق فور آشکار می‌شود، اماخردمند خجالت را می‌پوشاند.
\par 17 هر‌که به راستی تنطق نماید عدالت را ظاهرمی کند، و شاهد دروغ، فریب را.
\par 18 هستند که مثل ضرب شمشیر حرفهای باطل می‌زنند، اما زبان حکیمان شفا می‌بخشد.
\par 19 لب راستگو تا به ابد استوار می‌ماند، اما زبان دروغگو طرفه العینی است.
\par 20 در دل هر‌که تدبیر فاسد کند فریب است، اما مشورت دهندگان صلح را شادمانی است.
\par 21 هیچ بدی به مرد صالح واقع نمی شود، اماشریران از بلا پر خواهند شد.
\par 22 لبهای دروغگو نزد خداوند مکروه است، اما عاملان راستی پسندیده او هستند.
\par 23 مرد زیرک علم را مخفی می‌دارد، اما دل احمقان حماقت را شایع می‌سازد.
\par 24 دست شخص زرنگ سلطنت خواهدنمود، اما مرد کاهل بندگی خواهد کرد.
\par 25 کدورت دل انسان، او را منحنی می‌سازد، اما سخن نیکو او را شادمان خواهد گردانید.
\par 26 مرد عادل برای همسایه خود هادی می‌شود، اما راه شریران ایشان را گمراه می‌کند.
\par 27 مرد کاهل شکار خود را بریان نمی کند، امازرنگی، توانگری گرانبهای انسان است.در طریق عدالت حیات‌است، و درگذرگاههایش موت نیست.
\par 28 در طریق عدالت حیات‌است، و درگذرگاههایش موت نیست.
 
\chapter{13}

\par 1 پسر حکیم تادیب پدر خود را اطاعت می کند، اما استهزاکننده تهدید رانمی شنود.
\par 2 مرد از میوه دهانش نیکویی را می‌خورد، اماجان خیانتکاران، ظلم را خواهد خورد.
\par 3 هر‌که دهان خود را نگاه دارد جان خویش رامحافظت نماید، اما کسی‌که لبهای خود رابگشاید هلاک خواهد شد.
\par 4 شخص کاهل آرزو می‌کند و چیزی پیدانمی کند. اما شخص زرنگ فربه خواهد شد.
\par 5 مرد عادل از دروغ گفتن نفرت دارد، اما شریررسوا و خجل خواهد شد.
\par 6 عدالت کسی را که در طریق خود کامل است محافظت می‌کند، اما شرارت، گناهکاران را هلاک می‌سازد.
\par 7 هستند که خود را دولتمند می‌شمارند وهیچ ندارند، و هستند که خویشتن را فقیرمی انگارند و دولت بسیار دارند.
\par 8 دولت شخص فدیه جان او خواهد بود، امافقیر تهدید را نخواهد شنید.
\par 9 نور عادلان شادمان خواهد شد، اما چراغ شریران خاموش خواهد گردید.
\par 10 از تکبر جز نزاع چیزی پیدا نمی شود، اما باآنانی که پند می‌پذیرند حکمت است.
\par 11 دولتی که از بطالت پیدا شود در تناقص می‌باشد، اما هر‌که به‌دست خود اندوزد در تزایدخواهد بود.
\par 12 امیدی که در آن تعویق باشد باعث بیماری دل است، اما حصول مراد درخت حیات می‌باشد.
\par 13 هر‌که کلام را خوار شمارد خویشتن راهلاک می‌سازد، اما هر‌که از حکم می‌ترسد ثواب خواهد یافت.
\par 14 تعلیم مرد حکیم چشمه حیات‌است، تا ازدامهای مرگ رهایی دهد.
\par 15 عقل نیکو نعمت را می‌بخشد، اما راه خیانتکاران، سخت است.
\par 16 هر شخص زیرک با علم عمل می‌کند. امااحمق حماقت را منتشر می‌سازد.
\par 17 قاصد شریر در بلا گرفتار می‌شود، امارسول امین شفا می‌بخشد.
\par 18 فقر و اهانت برای کسی است که تادیب راترک نماید، اما هر‌که تنبیه را قبول کند محترم خواهد شد.
\par 19 آرزویی که حاصل شود برای جان شیرین است، اما اجتناب از بدی، مکروه احمقان می‌باشد.
\par 20 با حکیمان رفتار کن و حکیم خواهی شد، اما رفیق جاهلان ضرر خواهد یافت.
\par 21 بلا گناهکاران را تعاقب می‌کند، اما عادلان، جزای نیکو خواهند یافت.
\par 22 مرد صالح پسران پسران را ارث خواهدداد، و دولت گناهکاران برای عادلان ذخیره خواهد شد.
\par 23 در مزرعه فقیران خوراک بسیار است، اماهستند که از بی‌انصافی هلاک می‌شوند. 
\par 24 کسی‌که چوب را باز‌دارد، از پسر خویش نفرت می‌کند، اما کسی‌که او را دوست می‌دارد اورا به سعی تمام تادیب می‌نماید.مرد عادل برای سیری جان خود می‌خورد، اما شکم شریران محتاج خواهد بود.
\par 25 مرد عادل برای سیری جان خود می‌خورد، اما شکم شریران محتاج خواهد بود.
 
\chapter{14}

\par 1 هر زن حکیم خانه خود را بنا می‌کند، اما زن جاهل آن را با دست خود خراب می‌نماید.
\par 2 کسی‌که به راستی خود سلوک می‌نماید ازخداوند می‌ترسد، اما کسی‌که در طریق خودکج رفتار است او را تحقیر می‌نماید.
\par 3 در دهان احمق چوب تکبر است، اما لبهای حکیمان ایشان را محافظت خواهد نمود.
\par 4 جایی که گاو نیست، آخور پاک است، اما ازقوت گاو، محصول زیاد می‌شود.
\par 5 شاهد امین دروغ نمی گوید، اما شاهد دروغ به کذب تنطق می‌کند.
\par 6 استهزاکننده حکمت را می‌طلبد و نمی یابد. اما به جهت مرد فهیم علم آسان است.
\par 7 از حضور مرد احمق دور شو، زیرا لبهای معرفت را در او نخواهی یافت.
\par 8 حکمت مرد زیرک این است که راه خود رادرک نماید، اما حماقت احمقان فریب است.
\par 9 احمقان به گناه استهزا می‌کنند، اما در میان راستان رضامندی است.
\par 10 دل شخص تلخی خویشتن را می‌داند، وغریب در خوشی آن مشارکت ندارد.
\par 11 خانه شریران منهدم خواهد شد، اما خیمه راستان شکوفه خواهد آورد.
\par 12 راهی هست که به نظر آدمی مستقیم می‌نماید، اما عاقبت آن، طرق موت است.
\par 13 هم در لهو و لعب دل غمگین می‌باشد، وعاقبت این خوشی حزن است.
\par 14 کسی‌که در دل مرتد است از راههای خودسیر می‌شود، و مرد صالح به خود سیر است.
\par 15 مرد جاهل هر سخن را باور می‌کند، اما مردزیرک در رفتار خود تامل می‌نماید.
\par 16 مرد حکیم می‌ترسد و از بدی اجتناب می‌نماید، اما احمق از غرور خود ایمن می‌باشد.
\par 17 مرد کج خلق، احمقانه رفتار می‌نماید، و(مردم ) از صاحب سوظن نفرت دارند.
\par 18 نصیب جاهلان حماقت است، اما معرفت، تاج زیرکان خواهد بود.
\par 19 بدکاران در حضور نیکان خم می‌شوند، وشریران نزد دروازه های عادلان می‌ایستند.
\par 20 همسایه فقیر نیز از او نفرت دارد، امادوستان شخص دولتمند بسیارند.
\par 21 هر‌که همسایه خود را حقیر شمارد گناه می‌ورزد، اما خوشابحال کسی‌که بر فقیران ترحم نماید.
\par 22 آیا صاحبان تدبیر فاسد گمراه نمی شوند، اما برای کسانی که تدبیر نیکو می‌نمایند، رحمت و راستی خواهد بود.
\par 23 از هر مشقتی منفعت است، اما کلام لبها به فقر محض می‌انجامد.
\par 24 تاج حکیمان دولت ایشان است، اماحماقت احمقان حماقت محض است.
\par 25 شاهد امین جانها را نجات می‌بخشد، اما هرکه به دروغ تنطق می‌کند فریب محض است.
\par 26 در ترس خداوند اعتماد قوی است، وفرزندان او را ملجا خواهد بود.
\par 27 ترس خداوند چشمه حیات‌است، تا ازدامهای موت اجتناب نمایند.
\par 28 جلال پادشاه از کثرت مخلوق است، وشکستگی سلطان از کمی مردم است.
\par 29 کسی‌که دیرغضب باشد کثیرالفهم است، وکج خلق حماقت را به نصیب خود می‌برد.
\par 30 دل آرام حیات بدن است، اما حسدپوسیدگی استخوانها است.
\par 31 هر‌که بر فقیر ظلم کند آفریننده خود راحقیر می‌شمارد، و هر‌که بر مسکین ترحم کند اورا تمجید می‌نماید.
\par 32 شریر از شرارت خود به زیر افکنده می‌شود، اما مرد عادل چون بمیرد اعتماد دارد.
\par 33 حکمت در دل مرد فهیم ساکن می‌شود، امادر اندرون جاهلان آشکار می‌گردد.
\par 34 عدالت قوم را رفیع می‌گرداند، اما گناه برای قوم، عار است.رضامندی پادشاه بر خادم عاقل است، اماغضب او بر پست فطرتان.
\par 35 رضامندی پادشاه بر خادم عاقل است، اماغضب او بر پست فطرتان.
 
\chapter{15}

\par 1 جواب نرم خشم را برمی گرداند، اماسخن تلخ غیظ را به هیجان می‌آورد.
\par 2 زبان حکیمان علم را زینت می‌بخشد، امادهان احمقان به حماقت تنطق می‌نماید.
\par 3 چشمان خداوند در همه جاست، و بر بدان ونیکان می‌نگرد.
\par 4 زبان ملایم، درخت حیات‌است و کجی آن، شکستگی روح است.
\par 5 احمق تادیب پدر خود را خوار می‌شمارد، اما هر‌که تنبیه را نگاه دارد زیرک می‌باشد.
\par 6 در خانه مرد عادل گنج عظیم است، امامحصول شریران، کدورت است.
\par 7 لبهای حکیمان معرفت را منتشر می‌سازد، اما دل احمقان، مستحکم نیست.
\par 8 قربانی شریران نزد خداوند مکروه است، امادعای راستان پسندیده اوست.
\par 9 راه شریران نزد خداوند مکروه است، اماپیروان عدالت را دوست می‌دارد.
\par 10 برای هر‌که طریق را ترک نماید تادیب سخت است، و هر‌که از تنبیه نفرت کند خواهدمرد.
\par 11 هاویه و ابدون در حضور خداوند است، پس چند مرتبه زیاده دلهای بنی آدم.
\par 12 استهزاکننده تنبیه را دوست ندارد، و نزدحکیمان نخواهد رفت.
\par 13 دل‌شادمان چهره را زینت می‌دهد، اما ازتلخی دل روح منکسر می‌شود.
\par 14 دل مرد فهیم معرفت را می‌طلبد، اما دهان احمقان حماقت را می‌چرد.
\par 15 تمامی روزهای مصیبت کشان بد است، اماخوشی دل ضیافت دائمی است.
\par 16 اموال اندک با ترس خداوند بهتر است ازگنج عظیم با اضطراب.
\par 17 خوان بقول در جایی که محبت باشد بهتراست، از گاو پرواری که با آن عداوت باشد.
\par 18 مرد تندخو نزاع را برمی انگیزد، اما شخص دیرغضب خصومت را ساکن می‌گرداند.
\par 19 راه کاهلان مثل خاربست است، اما طریق راستان شاهراه است.
\par 20 پسر حکیم پدر را شادمان می‌سازد، اما مرداحمق مادر خویش را حقیر می‌شمارد.
\par 21 حماقت در نظر شخص ناقص العقل خوشی است، اما مرد فهیم به راستی سلوک می‌نماید.
\par 22 از عدم مشورت، تدبیرها باطل می‌شود، اما از کثرت مشورت دهندگان برقرار می‌ماند.
\par 23 برای انسان از جواب دهانش شادی حاصل می‌شود، و سخنی که در محلش گفته شودچه بسیار نیکو است.
\par 24 طریق حیات برای عاقلان به سوی بالااست، تا از هاویه اسفل دور شود.
\par 25 خداوند خانه متکبران را منهدم می‌سازد، اما حدود بیوه‌زن را استوار می‌نماید.
\par 26 تدبیرهای فاسد نزد خداوند مکروه است، اما سخنان پسندیده برای طاهران است.
\par 27 کسی‌که حریص سود باشد خانه خود رامکدر می‌سازد، اما هر‌که از هدیه‌ها نفرت داردخواهد زیست.
\par 28 دل مرد عادل در جواب دادن تفکر می‌کند، اما دهان شریران، چیزهای بد را جاری می‌سازد.
\par 29 خداوند از شریران دور است، اما دعای عادلان را می‌شنود.
\par 30 نور چشمان دل را شادمان می‌سازد، و خبرنیکو استخوانها را پر مغز می‌نماید.
\par 31 گوشی که تنبیه حیات را بشنود، در میان حکیمان ساکن خواهد شد.
\par 32 هر‌که تادیب را ترک نماید، جان خود راحقیر می‌شمارد، اما هر‌که تنبیه را بشنود عقل راتحصیل می‌نماید.ترس خداوند ادیب حکمت است، وتواضع پیشرو حرمت می‌باشد.
\par 33 ترس خداوند ادیب حکمت است، وتواضع پیشرو حرمت می‌باشد.
 
\chapter{16}

\par 1 تدبیرهای دل از آن انسان است، اماتنطق زبان از جانب خداوند می‌باشد.
\par 2 همه راههای انسان در نظر خودش پاک است، اما خداوند روحها را ثابت می‌سازد.
\par 3 اعمال خود را به خداوند تفویض کن، تافکرهای تو استوار شود.
\par 4 خداوند هر چیز را برای غایت آن ساخته است، و شریران را نیز برای روز بلا.
\par 5 هر‌که دل مغرور دارد نزد خداوند مکروه است، و او هرگز مبرا نخواهد شد.
\par 6 از رحمت و راستی، گناه کفاره می‌شود، و به ترس خداوند، از بدی اجتناب می‌شود.
\par 7 چون راههای شخص پسندیده خداوندباشد، دشمنانش را نیز با وی به مصالحه می‌آورد.
\par 8 اموال اندک که با انصاف باشد بهتر است، ازدخل فراوان بدون انصاف.
\par 9 دل انسان در طریقش تفکر می‌کند، اماخداوند قدمهایش را استوار می‌سازد.
\par 10 وحی بر لبهای پادشاه است، و دهان او درداوری تجاوز نمی نماید.
\par 11 ترازو و سنگهای راست از آن خداونداست و تمامی سنگهای کیسه صنعت وی می‌باشد.
\par 12 عمل بد نزد پادشاهان مکروه است، زیرا که کرسی ایشان از عدالت برقرار می‌ماند.
\par 13 لبهای راستگو پسندیده پادشاهان است، وراستگویان را دوست می‌دارند.
\par 14 غضب پادشاهان، رسولان موت است امامرد حکیم آن را فرو می‌نشاند.
\par 15 در نور چهره پادشاه حیات‌است، ورضامندی او مثل ابر نوبهاری است.
\par 16 تحصیل حکمت از زر خالص چه بسیاربهتر است، و تحصیل فهم از نقره برگزیده تر.
\par 17 طریق راستان، اجتناب نمودن از بدی است، و هر‌که راه خود را نگاه دارد جان خویش را محافظت می‌نماید.
\par 18 تکبر پیش رو هلاکت است، و دل مغرورپیش رو خرابی.
\par 19 با تواضع نزد حلیمان بودن بهتر است، ازتقسیم نمودن غنیمت با متکبران.
\par 20 هر‌که در کلام تعقل کند سعادتمندی خواهد یافت، و هر‌که به خداوند توکل نمایدخوشابحال او.
\par 21 هر‌که دل حکیم دارد فهیم خوانده می‌شود، و شیرینی لبها علم را می‌افزاید.
\par 22 عقل برای صاحبش چشمه حیات‌است، اما تادیب احمقان، حماقت است.
\par 23 دل مرد حکیم دهان او را عاقل می‌گرداند، و علم را بر لبهایش می‌افزاید.
\par 24 سخنان پسندیده مثل‌شان عسل است، برای جان شیرین است و برای استخوانهاشفادهنده.
\par 25 راهی هست که در نظر انسان راست است، اما عاقبت آن راه، موت می‌باشد.
\par 26 اشتهای کارگر برایش کار می‌کند، زیرا که دهانش او را بر آن تحریض می‌نماید.
\par 27 مرد لئیم شرارت را می‌اندیشد، و برلبهایش مثل آتش سوزنده است.
\par 28 مرد دروغگو نزاع می‌پاشد، و نمام دوستان خالص را از همدیگر جدا می‌کند.
\par 29 مرد ظالم همسایه خود را اغوا می‌نماید، واو را به راه غیر نیکو هدایت می‌کند.
\par 30 چشمان خود را بر هم می‌زند تا دروغ رااختراع نماید، و لبهایش را می‌خاید و بدی را به انجام می‌رساند.
\par 31 سفیدمویی تاج جمال است، هنگامی که درراه عدالت یافت شود.
\par 32 کسی‌که دیرغضب باشد از جبار بهتر است، و هر‌که بر روح خود مالک باشد از تسخیرکننده شهر افضل است.قرعه در دامن‌انداخته می‌شود، لیکن تمامی حکم آن از خداوند است.
\par 33 قرعه در دامن‌انداخته می‌شود، لیکن تمامی حکم آن از خداوند است.
 
\chapter{17}

\par 1 لقمه خشک با سلامتی، بهتر است ازخانه پر از ضیافت با مخاصمت.
\par 2 بنده عاقل بر پسر پست فطرت مسلط خواهدبود، و میراث را با برادران تقسیم خواهد نمود.
\par 3 بوته برای نقره و کوره به جهت طلا است، اماخداوند امتحان کننده دلها است.
\par 4 شریر به لبهای دروغگو اصغا می‌کند، و مردکاذب به زبان فتنه انگیز گوش می‌دهد.
\par 5 هر‌که فقیر را استهزا کند آفریننده خویش رامذمت می‌کند، و هر‌که از بلا خوش می‌شودبی سزا نخواهد ماند.
\par 6 تاج پیران، پسران پسرانند، و جلال فرزندان، پدران ایشانند.
\par 7 کلام کبرآمیز احمق را نمی شاید، و چندمرتبه زیاده لبهای دروغگو نجبا را.
\par 8 هدیه در نظر اهل آن سنگ گرانبها است که هر کجا توجه نماید برخوردار می‌شود.
\par 9 هر‌که گناهی را مستور کند طالب محبت می‌باشد، اما هر‌که امری را تکرار کند دوستان خالص را از هم جدا می‌سازد.
\par 10 یک ملامت به مرد فهیم اثر می‌کند، بیشتراز صد تازیانه به مرد جاهل.
\par 11 مرد شریر طالب فتنه است و بس. لهذاقاصد ستمکیش نزد او فرستاده می‌شود.
\par 12 اگر خرسی که بچه هایش کشته شود به انسان برخورد، بهتر است از مرد احمق در حماقت خود.
\par 13 کسی‌که به عوض نیکویی بدی می‌کند بلااز خانه او دور نخواهد شد.
\par 14 ابتدای نزاع مثل رخنه کردن آب است، پس مخاصمه را ترک کن قبل از آنکه به مجادله برسد.
\par 15 هر‌که شریر را عادل شمارد و هر‌که عادل را ملزم سازد، هر دوی ایشان نزد خداوندمکروهند.
\par 16 قیمت به جهت خریدن حکمت چرا به‌دست احمق باشد؟ و حال آنکه هیچ فهم ندارد.
\par 17 دوست خالص در همه اوقات محبت می‌نماید، و برادر به جهت تنگی مولود شده است.
\par 18 مرد ناقص العقل دست می‌دهد و در حضورهمسایه خود ضامن می‌شود. 
\par 19 هر‌که معصیت را دوست دارد منازعه رادوست می‌دارد، و هر‌که در خود را بلند سازدخرابی را می‌طلبد.
\par 20 کسی‌که دل کج دارد نیکویی را نخواهدیافت. و هر‌که زبان دروغگو دارد در بلا گرفتارخواهد شد.
\par 21 هر‌که فرزند احمق آورد برای خویشتن غم پیدا می‌کند، و پدر فرزند ابله شادی نخواهد دید.
\par 22 دل‌شادمان شفای نیکو می‌بخشد، اما روح شکسته استخوانها را خشک می‌کند.
\par 23 مرد شریر رشوه را از بغل می‌گیرد، تاراههای انصاف را منحرف سازد.
\par 24 حکمت در مد نظر مرد فهیم است، اماچشمان احمق در اقصای زمین می‌باشد.
\par 25 پسر احمق برای پدر خویش حزن است، وبه جهت مادر خویش تلخی است.
\par 26 عادلان را نیز سرزنش نمودن خوب نیست، و نه ضرب زدن به نجبا به‌سبب راستی ایشان.
\par 27 صاحب معرفت سخنان خود را بازمی دارد، و هر‌که روح حلیم دارد مرد فطانت پیشه است.مرد احمق نیز چون خاموش باشد او راحکیم می‌شمارند، و هر‌که لبهای خود را می‌بنددفهیم است.
\par 28 مرد احمق نیز چون خاموش باشد او راحکیم می‌شمارند، و هر‌که لبهای خود را می‌بنددفهیم است.
 
\chapter{18}

\par 1 می باشد، و به هر حکمت صحیح مجادله می‌کند.
\par 2 احمق از فطانت مسرور نمی شود، مگر تاآنکه عقل خود را ظاهر سازد.
\par 3 هنگامی که شریر می‌آید، حقارت هم می‌آید، و با اهانت، خجالت می‌رسد.
\par 4 سخنان دهان انسان آب عمیق است، وچشمه حکمت، نهر جاری است.
\par 5 طرفداری شریران برای منحرف ساختن داوری عادلان نیکو نیست.
\par 6 لبهای احمق به منازعه داخل می‌شود، ودهانش برای ضربها صدا می‌زند.
\par 7 دهان احمق هلاکت وی است، و لبهایش برای جان خودش دام است.
\par 8 سخنان نمام مثل لقمه های شیرین است، و به عمق شکم فرو می‌رود.
\par 9 او نیز که در کار خود اهمال می‌کند برادرهلاک کننده است.
\par 10 اسم خداوند برج حصین است که مرد عادل در آن می‌دود و ایمن می‌باشد.
\par 11 توانگری شخص دولتمند شهر محکم اواست، و در تصور وی مثل حصار بلند است.
\par 12 پیش از شکستگی، دل انسان متکبرمی گردد، و تواضع مقدمه عزت است.
\par 13 هر‌که سخنی را قبل از شنیدنش جواب دهد برای وی حماقت و عار می‌باشد.
\par 14 روح انسان بیماری او را متحمل می‌شود، اما روح شکسته را کیست که متحمل آن بشود.
\par 15 دل مرد فهیم معرفت را تحصیل می‌کند، وگوش حکیمان معرفت را می‌طلبد.
\par 16 هدیه شخص، از برایش وسعت پیدامی کند و او را به حضور بزرگان می‌رساند.
\par 17 هر‌که در دعوی خود اول آید صادق می‌نماید، اما حریفش می‌آید و او را می‌آزماید.
\par 18 قرعه نزاعها را ساکت می‌نماید و زورآوران را از هم جدا می‌کند.
\par 19 برادر رنجیده از شهر قوی سختتر است، ومنازعت با او مثل پشت بندهای قصر است.
\par 20 دل آدمی از میوه دهانش پر می‌شود و ازمحصول لبهایش، سیر می‌گردد.
\par 21 موت و حیات در قدرت زبان است، و آنانی که آن را دوست می‌دارند میوه‌اش را خواهندخورد.
\par 22 هر‌که زوجه‌ای یابد چیز نیکو یافته است، ورضامندی خداوند را تحصیل کرده است.
\par 23 مرد فقیر به تضرع تکلم می‌کند، اما شخص دولتمند به سختی جواب می‌دهد.کسی‌که دوستان بسیار دارد خویشتن راهلاک می‌کند، اما دوستی هست که از برادرچسبنده تر می‌باشد.
\par 24 کسی‌که دوستان بسیار دارد خویشتن راهلاک می‌کند، اما دوستی هست که از برادرچسبنده تر می‌باشد.
 
\chapter{19}

\par 1 فقیری که در کاملیت خود سالک است، از دروغگویی که احمق باشد بهتر است.
\par 2 دلی نیز که معرفت ندارد نیکو نیست و هر‌که به پایهای خویش می‌شتابد گناه می‌کند.
\par 3 حماقت انسان، راه او را کج می‌سازد، و دلش از خداوند خشمناک می‌شود.
\par 4 توانگری دوستان بسیار پیدا می‌کند، اما فقیراز دوستان خود جدا می‌شود.
\par 5 شاهد دروغگو بی‌سزا نخواهد ماند، و کسی‌که به دروغ تنطق کند رهایی نخواهد یافت.
\par 6 بسیاری پیش امیران تذلل می‌نمایند، و همه کس دوست بذل کننده است.
\par 7 جمیع برادران مرد فقیر از او نفرت دارند، وبه طریق اولی دوستانش از او دور می‌شوند، ایشان را به سخنان، تعاقب می‌کند و نیستند.
\par 8 هر‌که حکمت را تحصیل کند جان خود رادوست دارد. و هر‌که فطانت را نگاه دارد، سعادتمندی خواهد یافت.
\par 9 شاهد دروغگو بی‌سزا نخواهد ماند، و هر‌که به کذب تنطق نماید هلاک خواهد گردید.
\par 10 عیش و عشرت احمق را نمی شاید، تا چه رسد به غلامی که بر نجبا حکمرانی می‌کند.
\par 11 عقل انسان خشم او را نگاه می‌دارد، وگذشتن از تقصیر جلال او است.
\par 12 خشم پادشاه مثل غرش شیر است، ورضامندی او مثل شبنم بر گیاه است.
\par 13 پسر جاهل باعث الم پدرش است، ونزاعهای زن مثل آبی است که دائم در چکیدن باشد.
\par 14 خانه و دولت ارث اجدادی است، امازوجه عاقله از جانب خداوند است.
\par 15 کاهلی خواب سنگین می‌آورد، و شخص اهمال کار، گرسنه خواهد ماند.
\par 16 هر‌که حکم را نگاه دارد جان خویش رامحافظت می‌نماید، اما هر‌که طریق خود را سبک گیرد، خواهد مرد.
\par 17 هر‌که بر فقیر ترحم نماید به خداوند قرض می‌دهد، و احسان او را به او رد خواهد نمود.
\par 18 پسر خود را تادیب نما زیرا که امید هست، اما خود را به کشتن او وامدار.
\par 19 شخص تندخو متحمل عقوبت خواهدشد، زیرا اگر او را خلاصی دهی آن را باید مکرربجا آوری.
\par 20 پند را بشنو و تادیب را قبول نما، تا درعاقبت خود حکیم بشوی.
\par 21 فکرهای بسیار در دل انسان است، اما آنچه ثابت ماند مشورت خداوند است.
\par 22 زینت انسان احسان او است، و فقیر ازدروغگو بهتر است.
\par 23 ترس خداوند مودی به حیات‌است، و هرکه آن را دارد در سیری ساکن می‌ماند، و به هیچ بلاگرفتار نخواهد شد.
\par 24 مرد کاهل دست خود را در بغلش پنهان می‌کند، و آن را هم به دهان خود برنمی آورد.
\par 25 استهزاکننده را تادیب کن تا جاهلان زیرک شوند، و شخص فهیم را تنبیه نما و معرفت رادرک خواهد نمود.
\par 26 هر‌که بر پدر خود ستم کند و مادرش رابراند، پسری است که رسوایی و خجالت می‌آورد.
\par 27 ‌ای پسر من شنیدن تعلیمی را ترک نما، که تو را از کلام معرفت گمراه می‌سازد.
\par 28 شاهد لئیم انصاف را استهزا می‌کند، و دهان شریران گناه را می‌بلعد.قصاص به جهت استهزاکنندگان مهیااست، و تازیانه‌ها برای پشت احمقان.
\par 29 قصاص به جهت استهزاکنندگان مهیااست، و تازیانه‌ها برای پشت احمقان.
 
\chapter{20}

\par 1 شراب استهزا می‌کند و مسکرات عربده می‌آورد، و هر‌که به آن فریفته شود حکیم نیست.
\par 2 هیبت پادشاه مثل غرش شیر است، و هر‌که خشم او را به هیجان آورد، به‌جان خود خطامی ورزد.
\par 3 از نزاع دور شدن برای انسان عزت است، اماهر مرد احمق مجادله می‌کند.
\par 4 مرد کاهل به‌سبب زمستان شیار نمی کند، لهذا در موسم حصاد گدایی می‌کند و نمی یابد.
\par 5 مشورت در دل انسان آب عمیق است، امامرد فهیم آن را می‌کشد.
\par 6 بسا کسانند که هر یک احسان خویش رااعلام می‌کنند، اما مرد امین را کیست که پیدا کند.
\par 7 مرد عادل که به کاملیت خود سلوک نماید، پسرانش بعد از او خجسته خواهند شد.
\par 8 پادشاهی که بر کرسی داوری نشیند، تمامی بدی را از چشمان خود پراکنده می‌سازد.
\par 9 کیست که تواند گوید: «دل خود را طاهرساختم، و از گناه خویش پاک شدم.»
\par 10 سنگهای مختلف و پیمانه های مختلف، هر دوی آنها نزد خداوند مکروه است.
\par 11 طفل نیز از افعالش شناخته می‌شود، که آیااعمالش پاک و راست است یا نه.
\par 12 گوش شنوا و چشم بینا، خداوند هر دو آنهارا آفریده است.
\par 13 خواب را دوست مدار مبادا فقیر شوی. چشمان خود را باز کن تا از نان سیر گردی.
\par 14 مشتری می‌گوید بد است، بد است، اماچون رفت آنگاه فخر می‌کند.
\par 15 طلا هست و لعلها بسیار، اما لبهای معرفت جواهر گرانبها است.
\par 16 جامه آنکس را بگیر که به جهت غریب ضامن است، و او را به رهن بگیر که ضامن بیگانگان است.
\par 17 نان فریب برای انسان لذیذ است، اما بعد دهانش از سنگ ریزه‌ها پر خواهد شد.
\par 18 فکرها به مشورت محکم می‌شود، و باحسن تدبیر جنگ نما.
\par 19 کسی‌که به نمامی گردش کند اسرار را فاش می‌نماید، لهذا با کسی‌که لبهای خود را می‌گشایدمعاشرت منما.
\par 20 هر‌که پدر و مادر خود را لعنت کندچراغش در ظلمت غلیظ خاموش خواهد شد.
\par 21 اموالی که اولا به تعجیل حاصل می‌شود، عاقبتش مبارک نخواهد شد.
\par 22 مگو که از بدی انتقام خواهم کشید، بلکه برخداوند توکل نما و تو را نجات خواهد داد.
\par 23 سنگهای مختلف نزد خداوند مکروه است، و ترازوهای متقلب نیکو نیست.
\par 24 قدمهای انسان از خداوند است، پس مردراه خود را چگونه بفهمد؟
\par 25 شخصی که چیزی را به تعجیل مقدس می‌گوید، و بعد از نذر کردن استفسار می‌کند، دردام می‌افتد.
\par 26 پادشاه حکیم شریران را پراکنده می‌سازد وچوم را بر ایشان می‌گرداند.
\par 27 روح انسان، چراغ خداوند است که تمامی عمقهای دل را تفتیش می‌نماید.
\par 28 رحمت و راستی پادشاه را محافظت می‌کند، و کرسی او به رحمت پایدار خواهد ماند.
\par 29 جلال جوانان قوت ایشان است، و عزت پیران موی سفید.ضربهای سخت از بدی طاهر می‌کند وتازیانه‌ها به عمق دل فرو می‌رود.
\par 30 ضربهای سخت از بدی طاهر می‌کند وتازیانه‌ها به عمق دل فرو می‌رود.
 
\chapter{21}

\par 1 دل پادشاه مثل نهرهای آب در دست خداوند است، آن را به هر سو که بخواهد برمی گرداند.
\par 2 هر راه انسان در نظر خودش راست است، اماخداوند دلها را می‌آزماید.
\par 3 عدالت و انصاف را بجا آوردن، نزد خداونداز قربانی‌ها پسندیده تر است.
\par 4 چشمان بلند و دل متکبر و چراغ شریران، گناه است.
\par 5 فکرهای مرد زرنگ تمام به فراخی می‌انجامد، اما هر‌که عجول باشد برای احتیاج تعجیل می‌کند.
\par 6 تحصیل گنجها به زبان دروغگو، بخاری است بر هوا شده برای جویندگان موت.
\par 7 ظلم شریران ایشان را به هلاکت می‌اندازد، زیرا که از بجا آوردن انصاف ابا می‌نمایند.
\par 8 طریق مردی که زیر بار (گناه ) باشد بسیار کج است، اما اعمال مرد طاهر، مستقیم است.
\par 9 در زاویه پشت بام ساکن شدن بهتر است، ازساکن بودن با زن ستیزه گر در خانه مشترک.
\par 10 جان شریر مشتاق شرارت است، و برهمسایه خود ترحم نمی کند.
\par 11 چون استهزاکننده سیاست یابد جاهلان حکمت می‌آموزند، و چون مرد حکیم تربیت یابد معرفت را تحصیل می‌نماید.
\par 12 مرد عادل در خانه شریر تامل می‌کند که چگونه اشرار به تباهی واژگون می‌شوند.
\par 13 هر‌که گوش خود را از فریاد فقیر می‌بندد، او نیز فریاد خواهد کرد و مستجاب نخواهد شد.
\par 14 هدیه‌ای در خفا خشم را فرو می‌نشاند، ورشوه‌ای در بغل، غضب سخت را.
\par 15 انصاف کردن خرمی عادلان است، اما باعث پریشانی بدکاران می‌باشد.
\par 16 هر‌که از طریق تعقل گمراه شود، درجماعت مردگان ساکن خواهد گشت.
\par 17 هر‌که عیش را دوست دارد محتاج خواهدشد، و هر‌که شراب و روغن را دوست دارددولتمند نخواهد گردید.
\par 18 شریران فدیه عادلان می‌شوند وخیانتکاران به عوض راستان.
\par 19 در زمین بایر ساکن بودن بهتر است از بودن با زن ستیزه گر و جنگجوی.
\par 20 در منزل حکیمان خزانه مرغوب و روغن است، اما مرد احمق آنها را تلف می‌کند.
\par 21 هر‌که عدالت و رحمت را متابعت کند، حیات و عدالت و جلال خواهد یافت.
\par 22 مرد حکیم به شهر جباران برخواهد آمد، وقلعه اعتماد ایشان را به زیر می‌اندازد.
\par 23 هر‌که دهان و زبان خویش را نگاه دارد، جان خود را از تنگیها محافظت می‌نماید.
\par 24 مرد متکبر و مغرور مسمی به استهزاکننده می‌شود، و به افزونی تکبر عمل می‌کند.
\par 25 شهوت مرد کاهل او را می‌کشد، زیرا که دستهایش از کار کردن ابا می‌نماید.
\par 26 هستند که همه اوقات به شدت حریص می‌باشند، اما مرد عادل بذل می‌کند و امساک نمی نماید. 
\par 27 قربانی های شریران مکروه است، پس چندمرتبه زیاده هنگامی که به عوض بدی آنها رامی گذرانند.
\par 28 شاهد دروغگو هلاک می‌شود، اما کسی‌که استماع نماید به راستی تکلم خواهد کرد.
\par 29 مرد شریر روی خود را بی‌حیا می‌سازد، ومرد راست، طریق خویش را مستحکم می‌کند.
\par 30 حکمتی نیست و نه فطانتی و نه مشورتی که به ضد خداوند به‌کار آید.اسب برای روز جنگ مهیا است، اما نصرت از جانب خداوند است.
\par 31 اسب برای روز جنگ مهیا است، اما نصرت از جانب خداوند است.
 
\chapter{22}

\par 1 نیک نامی از کثرت دولتمندی افضل است، و فیض از نقره و طلا بهتر.
\par 2 دولتمند و فقیر با هم ملاقات می‌کنند، آفریننده هر دوی ایشان خداوند است.
\par 3 مرد زیرک، بلا را می‌بیند و خویشتن رامخفی می‌سازد و جاهلان می‌گذرند و در عقوبت گرفتار می‌شوند.
\par 4 جزای تواضع و خداترسی، دولت و جلال وحیات‌است.
\par 5 خارها و دامها در راه کجروان است، اما هرکه جان خود را نگاه دارد از آنها دور می‌شود.
\par 6 طفل را در راهی که باید برود تربیت نما، وچون پیر هم شود از آن انحراف نخواهد ورزید.
\par 7 توانگر بر فقیر تسلط دارد، و مدیون غلام طلب کار می‌باشد.
\par 8 هر‌که ظلم بکارد بلا خواهد دروید، وعصای غضبش زایل خواهد شد.
\par 9 شخصی که نظر او باز باشد مبارک خواهدبود، زیرا که از نان خود به فقرا می‌دهد.
\par 10 استهزاکننده را دور نما و نزاع رفع خواهدشد، و مجادله و خجالت ساکت خواهد گردید.
\par 11 هر‌که طهارت دل را دوست دارد، و لبهای ظریف دارد پادشاه دوست او می‌باشد.
\par 12 چشمان خداوند معرفت را نگاه می‌دارد وسخنان خیانتکاران را باطل می‌سازد.
\par 13 مرد کاهل می‌گوید شیر بیرون است، و درکوچه‌ها کشته می‌شوم.
\par 14 دهان زنان بیگانه چاه عمیق است، و هر‌که مغضوب خداوند باشد در آن خواهد افتاد.
\par 15 حماقت در دل طفل بسته شده است، اماچوب تادیب آن را از او دور خواهد کرد.
\par 16 هر‌که بر فقیر برای فایده خویش ظلم نماید، و هر‌که به دولتمندان ببخشد البته محتاج خواهد شد.
\par 17 گوش خود را فرا داشته، کلام حکما رابشنو، و دل خود را به تعلیم من مایل گردان،
\par 18 زیرا پسندیده است که آنها را در دل خودنگاه داری، و بر لبهایت جمیع ثابت ماند،
\par 19 تا اعتماد تو بر خداوند باشد. امروز تو راتعلیم دادم،
\par 20 آیا امور شریف را برای تو ننوشتم؟ شامل بر مشورت معرفت،
\par 21 تا قانون کلام راستی را اعلام نمایم، و توکلام راستی را نزد فرستندگان خود پس ببری؟
\par 22 فقیر را از آن جهت که ذلیل است تاراج منما، و مسکین را در دربار، ستم مرسان،
\par 23 زیرا خداوند دعوی ایشان را فیصل خواهدنمود، و جان تاراج کنندگان ایشان را به تاراج خواهد داد.
\par 24 با مرد تندخو معاشرت مکن، و با شخص کج خلق همراه مباش،
\par 25 مبادا راههای او را آموخته شوی و جان خود را در دام گرفتار سازی.
\par 26 از‌جمله آنانی که دست می‌دهند مباش و نه از آنانی که برای قرضها ضامن می‌شوند.
\par 27 اگر چیزی نداری که ادا نمایی پس چرابستر تو را از زیرت بردارد.
\par 28 حد قدیمی را که پدرانت قرار داده اندمنتقل مساز.آیا مردی را که در شغل خویش ماهر باشدمی بینی؟ او در حضور پادشاهان خواهد ایستاد، پیش پست فطرتان نخواهد ایستاد.
\par 29 آیا مردی را که در شغل خویش ماهر باشدمی بینی؟ او در حضور پادشاهان خواهد ایستاد، پیش پست فطرتان نخواهد ایستاد.
 
\chapter{23}

\par 1 چون با حاکم به غذا خوردن نشینی، درآنچه پیش روی تو است تامل نما.
\par 2 و اگر مرد اکول هستی کارد بر گلوی خودبگذار.
\par 3 به خوراکهای لطیف او حریص مباش، زیراکه غذای فریبنده است.
\par 4 برای دولتمند شدن خود را زحمت مرسان واز عقل خود باز ایست.
\par 5 آیا چشمان خود را بر آن خواهی دوخت که نیست می‌باشد؟ زیرا که دولت البته برای خودبالها می‌سازد، و مثل عقاب در آسمان می‌پرد.
\par 6 نان مرد تنگ نظر را مخور، و به جهت خوراکهای لطیف او حریص مباش.
\par 7 زیرا چنانکه در دل خود فکر می‌کند خود اوهمچنان است. تو را می‌گوید: بخور و بنوش، امادلش با تو نیست.
\par 8 لقمه‌ای را که خورده‌ای قی خواهی کرد، وسخنان شیرین خود را بر باد خواهی داد.
\par 9 به گوش احمق سخن مگو، زیرا حکمت کلامت را خوار خواهد شمرد.
\par 10 حد قدیم را منتقل مساز، و به مزرعه یتیمان داخل مشو،
\par 11 زیرا که ولی ایشان زورآور است، و با تو دردعوی ایشان مقاومت خواهد کرد.
\par 12 دل خود را به ادب مایل گردان، و گوش خود را به کلام معرفت.
\par 13 از طفل خود تادیب را باز مدار که اگر او رابا چوب بزنی نخواهد مرد،
\par 14 پس او را با چوب بزن، و جان او را از هاویه نجات خواهی داد.
\par 15 ‌ای پسر من اگر دل تو حکیم باشد، دل من (بلی دل ) من شادمان خواهد شد.
\par 16 و گرده هایم وجد خواهد نمود، هنگامی که لبهای تو به راستی متکلم شود.
\par 17 دل تو به جهت گناهکاران غیور نباشد، امابه جهت ترس خداوند تمامی روز غیور باش،
\par 18 زیرا که البته آخرت هست، و امید تومنقطع نخواهد شد.
\par 19 پس تو‌ای پسرم بشنو و حکیم باش، و دل خود را در طریق مستقیم گردان.
\par 20 از زمره میگساران مباش، و از آنانی که بدنهای خود را تلف می‌کنند.
\par 21 زیرا که میگسار و مسرف، فقیر می‌شود وصاحب خواب به خرقه‌ها ملبس خواهد شد.
\par 22 پدر خویش را که تو را تولید نمود گوش گیر، و مادر خود را چون پیر شود خوار مشمار.
\par 23 راستی را بخر و آن را مفروش، و حکمت وادب و فهم را.
\par 24 پدر فرزند عادل به غایت شادمان می‌شود، و والد پسر حکیم از او مسرور خواهد گشت.
\par 25 پدرت و مادرت شادمان خواهند شد، ووالده تو مسرور خواهد گردید.
\par 26 ‌ای پسرم دل خود را به من بده، و چشمان تو به راههای من شاد باشد،
\par 27 چونکه زن زانیه حفره‌ای عمیق است، و زن بیگانه چاه تنگ.
\par 28 او نیز مثل راهزن در کمین می‌باشد، وخیانتکاران را در میان مردم می‌افزاید.
\par 29 وای از آن کیست و شقاوت از آن که ونزاعها از آن کدام و زاری از آن کیست وجراحت های بی‌سبب از آن که و سرخی چشمان از آن کدام؟
\par 30 آنانی را است که شرب مدام می‌نمایند، و برای چشیدن شراب ممزوج داخل می‌شوند.
\par 31 به شراب نگاه مکن وقتی که سرخ‌فام است، حینی که حبابهای خود را در جام ظاهر می‌سازد، و به ملایمت فرو می‌رود.
\par 32 اما در آخر مثل مار خواهد گزید، و مانندافعی نیش خواهد زد.
\par 33 چشمان تو چیزهای غریب را خواهد دید، و دل تو به چیزهای کج تنطق خواهد نمود،
\par 34 و مثل کسی‌که در میان دریا می‌خوابدخواهی شد، یا مانند کسی‌که بر سر دکل کشتی می‌خسبد،و خواهی گفت: مرا زدند لیکن درد رااحساس نکردم، مرا زجر نمودند لیکن نفهمیدم. پس کی بیدار خواهم شد؟ همچنین معاودت می‌کنم و بار دیگر آن را می‌طلبم.
\par 35 و خواهی گفت: مرا زدند لیکن درد رااحساس نکردم، مرا زجر نمودند لیکن نفهمیدم. پس کی بیدار خواهم شد؟ همچنین معاودت می‌کنم و بار دیگر آن را می‌طلبم.
 
\chapter{24}

\par 1 بر مردان شریر حسد مبر، و آرزو مدارتا با ایشان معاشرت نمایی،
\par 2 زیرا که دل ایشان در ظلم تفکر می‌کند ولبهای ایشان درباره مشقت تکلم می‌نماید.
\par 3 خانه به حکمت بنا می‌شود، و با فطانت استوار می‌گردد،
\par 4 و به معرفت اطاقها پر می‌شود، از هر گونه اموال گرانبها و نفایس.
\par 5 مرد حکیم در قدرت می‌ماند، و صاحب معرفت در توانایی ترقی می‌کند،
\par 6 زیرا که با حسن تدبیر باید جنگ بکنی، و ازکثرت مشورت دهندگان نصرت است.
\par 7 حکمت برای احمق زیاده بلند است، دهان خود را در دربار باز نمی کند.
\par 8 هر‌که برای بدی تفکر می‌کند، او را فتنه انگیزمی گویند.
\par 9 فکر احمقان گناه است، و استهزاکننده نزد آدمیان مکروه است.
\par 10 اگر در روز تنگی سستی نمایی، قوت توتنگ می‌شود.
\par 11 آنانی را که برای موت برده شوند خلاص کن، و از رهانیدن آنانی که برای قتل مهیااندکوتاهی منما.
\par 12 اگر گویی که این را ندانستیم، آیا آزماینده دلها نمی فهمد؟ و حافظ جان تو نمی داند؟ و به هرکس برحسب اعمالش مکافات نخواهد داد؟
\par 13 ‌ای پسر من عسل را بخور زیرا که خوب است، و‌شان عسل را چونکه به کامت شیرین است.
\par 14 همچنین حکمت را برای جان خود بیاموز، اگر آن را بیابی آنگاه اجرت خواهد بود، و امید تومنقطع نخواهد شد.
\par 15 ‌ای شریر برای منزل مرد عادل در کمین مباش، و آرامگاه او را خراب مکن،
\par 16 زیرا مرد عادل اگر‌چه هفت مرتبه بیفتدخواهد برخاست، اما شریران در بلا خواهندافتاد.
\par 17 چون دشمنت بیفتد شادی مکن، و چون بلغزد دلت وجد ننماید،
\par 18 مبادا خداوند این را ببیند و در نظرش ناپسند آید، و غضب خود را از او برگرداند.
\par 19 خویشتن را به‌سبب بدکاران رنجیده مساز، و بر شریران حسد مبر،
\par 20 زیرا که به جهت بدکاران اجر نخواهد بود، و چراغ شریران خاموش خواهد گردید.
\par 21 ‌ای پسر من از خداوند و پادشاه بترس، و بامفسدان معاشرت منما،
\par 22 زیرا که مصیبت ایشان ناگهان خواهدبرخاست، و عاقبت سالهای ایشان را کیست که بداند؟
\par 23 اینها نیز از (سخنان ) حکیمان است طرفداری در داوری نیکو نیست.
\par 24 کسی‌که به شریر بگوید تو عادل هستی، امت‌ها او را لعنت خواهند کرد و طوایف از اونفرت خواهند نمود.
\par 25 اما برای آنانی که او را توبیخ نمایندشادمانی خواهد بود، و برکت نیکو به ایشان خواهد رسید.
\par 26 آنکه به کلام راست جواب گوید لبها رامی بوسد.
\par 27 کار خود را در خارج آراسته کن، و آن را درملک مهیا ساز، و بعد از آن خانه خویش را بنا نما.
\par 28 بر همسایه خود بی‌جهت شهادت مده، و بالبهای خود فریب مده،
\par 29 و مگو به طوری که او به من عمل کرد من نیزبا وی عمل خواهم نمود، و مرد را بر‌حسب اعمالش پاداش خواهم داد.
\par 30 از مزرعه مرد کاهل، و از تاکستان شخص ناقص العقل گذشتم.
\par 31 و اینک بر تمامی آن خارها می‌رویید، وخس تمامی روی آن را می‌پوشانید، و دیوارسنگیش خراب شده بود،
\par 32 پس من نگریسته متفکر شدم، ملاحظه کردم و ادب آموختم.
\par 33 اندکی خفت و اندکی خواب، و اندکی بر هم نهادن دستها به جهت خواب.پس فقر تو مثل راهزن بر تو خواهد آمد، ونیازمندی تو مانند مرد مسلح.
\par 34 پس فقر تو مثل راهزن بر تو خواهد آمد، ونیازمندی تو مانند مرد مسلح.
 
\chapter{25}

\par 1 اینها نیز از امثال سلیمان است که مردان حزقیا، پادشاه یهودا آنها را نقل نمودند.
\par 2 مخفی داشتن امر جلال خدا است، و تفحص نمودن امر جلال پادشاهان است.
\par 3 آسمان را در بلندیش و زمین را در عمقش، ودل پادشاهان را تفتیش نتوان نمود.
\par 4 درد را از نقره دور کن، تا ظرفی برای زرگربیرون آید.
\par 5 شریران را از حضور پادشاه دور کن، تاکرسی او در عدالت پایدار بماند.
\par 6 در حضور پادشاه خویشتن را برمیفراز، و درجای بزرگان مایست،
\par 7 زیرا بهتر است تو را گفته شود که اینجا بالابیا، از آنکه به حضور سروری که چشمانت او رادیده است تو را پایین برند.
\par 8 برای نزاع به تعجیل بیرون مرو، مبادا درآخرش چون همسایه ات تو را خجل سازد، ندانی که چه باید کرد.
\par 9 دعوی خود را با همسایه ات بکن، اما رازدیگری را فاش مساز،
\par 10 مبادا هر‌که بشنود تو را ملامت کند، وبدنامی تو رفع نشود.
\par 11 سخنی که در محلش گفته شود، مثل سیبهای طلا در مرصعکاری نقره است.
\par 12 مودب حکیم برای گوش شنوا، مثل حلقه طلا و زیور زر خالص است.
\par 13 رسول امین برای فرستندگان خود، چون خنکی یخ در موسم حصاد می‌باشد، زیرا که جان آقایان خود را تازه می‌کند.
\par 14 کسی‌که از بخششهای فریبنده خود فخرمی کند، مثل ابرها و باد بی‌باران است.
\par 15 با تحمل داور را به رای خود توان آورد، وزبان ملایم، استخوان را می‌شکند. 
\par 16 اگر عسل یافتی بقدر کفایت بخور، مبادا ازآن پر شده، قی کنی.
\par 17 پای خود را از زیاد رفتن به خانه همسایه ات باز دار، مبادا از تو سیر شده، از تونفرت نماید.
\par 18 کسی‌که درباره همسایه خود شهادت دروغ دهد، مثل تبرزین و شمشیر و تیر تیز است.
\par 19 اعتماد بر خیانتکار در روز تنگی، مثل دندان کرم زده و پای مرتعش می‌باشد.
\par 20 سراییدن سرودها برای دلتنگ، مثل کندن جامه در وقت سرما و ریختن سرکه بر شوره است.
\par 21 اگر دشمن تو گرسنه باشد او را نان بخوران، و اگر تشنه باشد او را آب بنوشان،
\par 22 زیرا اخگرها بر سرش خواهی انباشت، وخداوند تو را پاداش خواهد داد.
\par 23 چنانکه باد شمال باران می‌آورد، همچنان زبان غیبتگو چهره را خشمناک می‌سازد.
\par 24 ساکن بودن در گوشه پشت بام بهتر است ازبودن با زن جنگجو در خانه مشترک.
\par 25 خبر خوش از ولایت دور، مثل آب سردبرای جان تشنه است.
\par 26 مرد عادل که پیش شریر خم شود، مثل چشمه گل آلود و منبع فاسد است.
\par 27 زیاد عسل خوردن خوب نیست، همچنان طلبیدن جلال خود جلال نیست.کسی‌که بر روح خود تسلط ندارد، مثل شهر منهدم و بی‌حصار است.
\par 28 کسی‌که بر روح خود تسلط ندارد، مثل شهر منهدم و بی‌حصار است.
 
\chapter{26}

\par 1 چنانکه برف در تابستان و باران درحصاد، همچنین حرمت برای احمق شایسته نیست.
\par 2 لعنت، بی‌سبب نمی آید، چنانکه گنجشک در طیران و پرستوک در پریدن.
\par 3 شلاق به جهت اسب و لگام برای الاغ، وچوب از برای پشت احمقان است.
\par 4 احمق را موافق حماقتش جواب مده، مباداتو نیز مانند او بشوی.
\par 5 احمق را موافق حماقتش جواب بده، مباداخویشتن را حکیم بشمارد.
\par 6 هر‌که پیغامی به‌دست احمق بفرستد، پایهای خود را می‌برد و ضرر خود را می‌نوشد.
\par 7 ساقهای شخص لنگ بی‌تمکین است، ومثلی که از دهان احمق برآید همچنان است.
\par 8 هر‌که احمق را حرمت کند، مثل کیسه جواهر در توده سنگها است.
\par 9 مثلی که از دهان احمق برآید، مثل خاری است که در دست شخص مست رفته باشد.
\par 10 تیرانداز همه را مجروح می‌کند، همچنان است هر‌که احمق را به مزد گیرد و خطاکاران رااجیر نماید.
\par 11 چنانکه سگ به قی خود برمی گردد، همچنان احمق حماقت خود را تکرار می‌کند.
\par 12 آیا شخصی را می‌بینی که در نظر خودحکیم است، امید داشتن بر احمق از امید بر اوبیشتر است.
\par 13 کاهل می‌گوید که شیر در راه است، و اسددر میان کوچه‌ها است.
\par 14 چنانکه در بر پاشنه‌اش می‌گردد، همچنان کاهل بر بستر خویش.
\par 15 کاهل دست خود را در قاب فرو می‌برد و ازبرآوردن آن به دهانش خسته می‌شود.
\par 16 کاهل در نظر خود حکیمتر است از هفت مرد که جواب عاقلانه می‌دهند.
\par 17 کسی‌که برود و در نزاعی که به او تعلق ندارد متعرض شود، مثل کسی است که گوشهای سگ را بگیرد.
\par 18 آدم دیوانه‌ای که مشعلها و تیرها و موت رامی اندازد،
\par 19 مثل کسی است که همسایه خود را فریب دهد، و می‌گوید آیا شوخی نمی کردم؟
\par 20 از نبودن هیزم آتش خاموش می‌شود، و ازنبودن نمام منازعه ساکت می‌گردد.
\par 21 زغال برای اخگرها و هیزم برای آتش است، و مرد فتنه انگیز به جهت برانگیختن نزاع.
\par 22 سخنان نمام مثل خوراک لذیذ است، که به عمقهای دل فرو می‌رود.
\par 23 لبهای پرمحبت با دل شریر، مثل نقره‌ای پردرد است که بر ظرف سفالین اندوده شود.
\par 24 هر‌که بغض دارد با لبهای خود نیرنگ می‌نماید، و در دل خود فریب را ذخیره می‌کند.
\par 25 هنگامی که سخن نیکو گوید، او را باور مکن زیرا که در قلبش هفت چیز مکروه است.
\par 26 هر‌چند بغض او به حیله مخفی شود، اماخباثت او در میان جماعت ظاهر خواهد گشت.
\par 27 هر‌که حفره‌ای بکند در آن خواهد افتاد، وهر‌که سنگی بغلطاند بر او خواهد برگشت.زبان دروغگو از مجروح شدگان خود نفرت دارد، و دهان چاپلوس هلاکت را ایجاد می‌کند.
\par 28 زبان دروغگو از مجروح شدگان خود نفرت دارد، و دهان چاپلوس هلاکت را ایجاد می‌کند.
 
\chapter{27}

\par 1 درباره فردا فخر منما، زیرا نمی دانی که روز چه خواهد زایید.
\par 2 دیگری تو را بستاید و نه دهان خودت، غریبی و نه لبهای تو.
\par 3 سنگ سنگین است و ریگ ثقیل، اما خشم احمق از هر دوی آنها سنگینتر است.
\par 4 غضب ستم کیش است و خشم سیل، اماکیست که در برابر حسد تواند ایستاد.
\par 5 تنبیه آشکار از محبت پنهان بهتر است.
\par 6 جراحات دوست وفادار است، اما بوسه های دشمن افراط است.
\par 7 شکم سیر از‌شان عسل کراهت دارد، امابرای شکم گرسنه هر تلخی شیرین است.
\par 8 کسی‌که از مکان خود آواره بشود، مثل گنجشکی است که از آشیانه‌اش آواره گردد.
\par 9 روغن و عطر دل را شاد می‌کند، همچنان حلاوت دوست از مشورت دل.
\par 10 دوست خود و دوست پدرت را ترک منما، و در روز مصیبت خود به خانه برادرت داخل مشو، زیرا که همسایه نزدیک از برادر دور بهتراست.
\par 11 ‌ای پسر من حکمت بیاموز و دل مرا شادکن، تا ملامت کنندگان خود را مجاب سازم.
\par 12 مرد زیرک، بلا را می‌بیند و خویشتن رامخفی می‌سازد، اما جاهلان می‌گذرند و درعقوبت گرفتار می‌شوند.
\par 13 جامه آن کس را بگیر که به جهت غریب ضامن است، و او را به رهن بگیر که ضامن بیگانگان است.
\par 14 کسی‌که صبح زود برخاسته، دوست خودرا به آواز بلند برکت دهد، از برایش لعنت محسوب می‌شود.
\par 15 چکیدن دائمی آب در روز باران، و زن ستیزه‌جو مشابه‌اند.
\par 16 هرکه او را باز‌دارد مثل کسی است که باد رانگاه دارد، یا روغن را که در دست راست خودگرفته باشد.
\par 17 آهن، آهن را تیز می‌کند، همچنین مرد روی دوست خود را تیز می‌سازد.
\par 18 هر‌که درخت انجیر را نگاه دارد میوه‌اش راخواهد خورد، و هر‌که آقای خود را ملازمت نماید محترم خواهد شد،
\par 19 چنانکه در آب صورت به صورت است، همچنان دل انسان به انسان.
\par 20 هاویه و ابدون سیر نمی شوند، همچنان چشمان انسان سیر نخواهند شد.
\par 21 بوته برای نقره و کوره به جهت طلاست، همچنان انسان از دهان ستایش کنندگان خود(آزموده می‌شود).
\par 22 احمق را میان بلغور در هاون با دسته بکوب، و حماقتش از آن بیرون نخواهد رفت.
\par 23 به حالت گله خود نیکو توجه نما، و دل خود را به رمه خود مشغول ساز،
\par 24 زیرا که دولت دائمی نیست، و تاج هم نسلابعد نسل (پایدار) نی.
\par 25 علف را می‌برند و گیاه سبز می‌روید، وعلوفه کوهها جمع می‌شود،
\par 26 بره‌ها برای لباس تو، و بزها به جهت اجاره زمین به‌کار می‌آیند،و شیر بزها برای خوراک تو و خوراک خاندانت، و معیشت کنیزانت کفایت خواهد کرد.
\par 27 و شیر بزها برای خوراک تو و خوراک خاندانت، و معیشت کنیزانت کفایت خواهد کرد.
 
\chapter{28}

\par 1 تعاقب کننده‌ای نیست، اما عادلان مثل شیر شجاعند.
\par 2 از معصیت اهل زمین حاکمانش بسیارمی شوند، اما مرد فهیم و دانا استقامتش برقرارخواهد ماند.
\par 3 مرد رئیس که بر مسکینان ظلم می‌کند مثل باران است که سیلان کرده، خوراک باقی نگذارد.
\par 4 هر‌که شریعت را ترک می‌کند شریران رامی ستاید، اما هر‌که شریعت را نگاه دارد از ایشان نفرت دارد.
\par 5 مردمان شریر انصاف را درک نمی نمایند، اماطالبان خداوند همه‌چیز را می‌فهمند.
\par 6 فقیری که در کاملیت خود سلوک نماید بهتراست از کج رونده دو راه اگر‌چه دولتمند باشد.
\par 7 هر‌که شریعت را نگاه دارد پسری حکیم است، اما مصاحب مسرفان، پدر خویش را رسوامی سازد.
\par 8 هر‌که مال خود را به ربا و سود بیفزاید، آن رابرای کسی‌که بر فقیران ترحم نماید، جمع می‌نماید.
\par 9 هر‌که گوش خود را از شنیدن شریعت برگرداند، دعای او هم مکروه می‌شود.
\par 10 هر‌که راستان را به راه بد گمراه کند به حفره خود خواهد افتاد، اما صالحان نصیب نیکوخواهند یافت.
\par 11 مرد دولتمند در نظر خود حکیم است، امافقیر خردمند او را تفتیش خواهد نمود.
\par 12 چون عادلان شادمان شوند فخر عظیم است، اما چون شریران برافراشته شوند مردمان خود را مخفی می‌سازند.
\par 13 هر‌که گناه خود را بپوشاند برخوردارنخواهد شد، اما هر‌که آن را اعتراف کند و ترک نماید رحمت خواهد یافت.
\par 14 خوشابحال کسی‌که دائم می‌ترسد، اما هرکه دل خود را سخت سازد به بلا گرفتار خواهدشد.
\par 15 حاکم شریر بر قوم مسکین، مثل شیر غرنده و خرس گردنده است.
\par 16 حاکم ناقص العقل بسیار ظلم می‌کند، اماهر‌که از رشوه نفرت کند عمر خود را درازخواهد ساخت.
\par 17 کسی‌که متحمل بار خون شخصی شود، به هاویه می‌شتابد. زنهار کسی او را باز ندارد.
\par 18 هر‌که به استقامت سلوک نماید رستگارخواهد شد، اما هر‌که در دو راه کج رو باشد دریکی از آنها خواهد افتاد.
\par 19 هر‌که زمین خود را زرع نماید از نان سیرخواهد شد، اما هر‌که پیروی باطلان کند از فقرسیر خواهد شد.
\par 20 مرد امین برکت بسیار خواهد یافت، اماآنکه در‌پی دولت می‌شتابد بی‌سزا نخواهد ماند.
\par 21 طرفداری نیکو نیست، و به جهت لقمه‌ای نان، آدمی خطاکار می‌شود.
\par 22 مرد تنگ نظر در‌پی دولت می‌شتابد ونمی داند که نیازمندی او را درخواهد یافت.
\par 23 کسی‌که آدمی را تنبیه نماید، آخر شکرخواهد یافت، بیشتر از آنکه به زبان خودچاپلوسی می‌کند.
\par 24 کسی‌که پدر و مادر خود را غارت نماید وگوید گناه نیست، مصاحب هلاک کنندگان خواهدشد.
\par 25 مرد حریص نزاع را برمی انگیزاند، اما هر‌که بر خداوند توکل نماید قوی خواهد شد.
\par 26 آنکه بر دل خود توکل نماید احمق می‌باشد، اما کسی‌که به حکمت سلوک نمایدنجات خواهد یافت.
\par 27 هر‌که به فقرا بذل نماید محتاج نخواهدشد، اما آنکه چشمان خود را بپوشاند لعنت بسیارخواهد یافت.وقتی که شریران برافراشته شوند مردم خویشتن را پنهان می‌کنند، اما چون ایشان هلاک شوند عادلان افزوده خواهند شد.
\par 28 وقتی که شریران برافراشته شوند مردم خویشتن را پنهان می‌کنند، اما چون ایشان هلاک شوند عادلان افزوده خواهند شد.
 
\chapter{29}

\par 1 کسی‌که بعد از تنبیه بسیار گردنکشی می کند، ناگهان منکسر خواهد شد وعلاجی نخواهد بود.
\par 2 چون عادلان افزوده گردند قوم شادی می‌کنند، اما چون شریران تسلط یابند مردم ناله می‌نمایند.
\par 3 کسی‌که حکمت را دوست دارد پدر خویش را مسرور می‌سازد، اما کسی‌که با فاحشه هامعاشرت کند اموال را تلف می‌نماید.
\par 4 پادشاه ولایت را به انصاف پایدار می‌کند، امامرد رشوه خوار آن را ویران می‌سازد.
\par 5 شخصی که همسایه خود را چاپلوسی می‌کند دام برای پایهایش می‌گستراند.
\par 6 در معصیت مرد شریر دامی است، اما عادل ترنم و شادی خواهد نمود.
\par 7 مرد عادل دعوی فقیر را درک می‌کند، اماشریر برای دانستن آن فهم ندارد.
\par 8 استهزاکنندگان شهر را به آشوب می‌آورند، اما حکیمان خشم را فرومی نشانند.
\par 9 اگر مرد حکیم با احمق دعوی دارد، خواه غضبناک شود خواه بخندد او را راحت نخواهد بود.
\par 10 مردان خون ریز از مرد کامل نفرت دارند، اما راستان سلامتی جان او را طالبند.
\par 11 احمق تمامی خشم خود را ظاهر می‌سازد، اما مرد حکیم به تاخیر آن را فرومی نشاند.
\par 12 حاکمی که به سخنان دروغ گوش گیرد، جمیع خادمانش شریر خواهند شد.
\par 13 فقیر و ظالم با هم جمع خواهند شد، وخداوند چشمان هر دوی ایشان را روشن خواهدساخت.
\par 14 پادشاهی که مسکینان را به راستی داوری نماید، کرسی وی تا به ابد پایدار خواهد ماند.
\par 15 چوب و تنبیه، حکمت می‌بخشد، اماپسری که بی‌لگام باشد، مادر خود را خجل خواهد ساخت.
\par 16 چون شریران افزوده شوند تقصیر زیاده می‌گردد، اما عادلان، افتادن ایشان را خواهند دید.
\par 17 پسر خود را تادیب نما که تو را راحت خواهد رسانید، و به‌جان تو لذات خواهدبخشید.
\par 18 جایی که رویا نیست قوم گردنکش می‌شوند، اما خوشابحال کسی‌که شریعت را نگاه می‌دارد. 
\par 19 خادم، محض سخن متنبه نمی شود، زیرااگر‌چه بفهمد اجابت نمی نماید.
\par 20 آیا کسی را می‌بینی که در سخن‌گفتن عجول است، امید بر احمق زیاده است از امید براو.
\par 21 هر‌که خادم خود را از طفولیت به نازمی پرورد، آخر پسر او خواهد شد.
\par 22 مرد تندخو نزاع برمی انگیزاند، و شخص کج خلق در تقصیر می‌افزاید.
\par 23 تکبر شخص او را پست می‌کند، اما مردحلیم دل، به جلال خواهد رسید.
\par 24 هر‌که با دزد معاشرت کند خویشتن رادشمن دارد، زیرا که لعنت می‌شنود و اعتراف نمی نماید.
\par 25 ترس از انسان دام می‌گستراند، اما هر‌که برخداوند توکل نماید سرافراز خواهد شد.
\par 26 بسیاری لطف حاکم را می‌طلبند، اماداوری انسان از جانب خداوند است.مرد ظالم نزد عادلان مکروه است، و هر‌که در طریق، مستقیم است نزد شریران مکروه می‌باشد.
\par 27 مرد ظالم نزد عادلان مکروه است، و هر‌که در طریق، مستقیم است نزد شریران مکروه می‌باشد.
 
\chapter{30}

\par 1 کلمات و پیغام آکور بن یاقه. وحی آن مرد به ایتیئیل یعنی به ایتیئیل و اکال.
\par 2 یقین من از هر آدمی وحشی تر هستم، و فهم انسان را ندارم.
\par 3 من حکمت را نیاموخته‌ام و معرفت قدوس را ندانسته‌ام.
\par 4 کیست که به آسمان صعود نمود و از آنجانزول کرد؟ کیست که باد را در مشت خود جمع نمود؟ و کیست که آب را در جامه بند نمود؟ کیست که تمامی اقصای زمین را استوار ساخت؟ نام او چیست و پسر او چه اسم دارد؟ بگو اگراطلاع داری.
\par 5 تمامی کلمات خدا مصفی است. او به جهت متوکلان خود سپر است.
\par 6 به سخنان او چیزی میفزا، مبادا تو را توبیخ نماید و تکذیب شوی.
\par 7 دو چیز از تو درخواست نمودم، آنها را قبل از آنکه بمیرم از من بازمدار:
\par 8 بطالت و دروغ را از من دور کن، مرا نه فقر ده و نه دولت. به خوراکی که نصیب من باشد مرابپرور،
\par 9 مبادا سیر شده، تو را انکار نمایم و بگویم که خداوند کیست. و مبادا فقیر شده، دزدی نمایم، واسم خدای خود را به باطل برم.
\par 10 بنده را نزد آقایش متهم مساز، مبادا تو رالعنت کند و مجرم شوی.
\par 11 گروهی می‌باشند که پدر خود را لعنت می‌نمایند، و مادر خویش را برکت نمی دهند.
\par 12 گروهی می‌باشند که در نظر خود پاک‌اند، اما از نجاست خود غسل نیافته‌اند.
\par 13 گروهی می‌باشند که چشمانشان چه قدربلند است، و مژگانشان چه قدر برافراشته.
\par 14 گروهی می‌باشند که دندانهایشان شمشیرها است، و دندانهای آسیای ایشان کاردهاتا فقیران را از روی زمین و مسکینان را از میان مردمان بخورند.
\par 15 زالو را دو دختر است که بده بده می‌گویند. سه چیز است که سیر نمی شود، بلکه چهار چیزکه نمی گوید که کافی است:
\par 16 هاویه و رحم نازاد، و زمینی که از آب سیرنمی شود، و آتش که نمی گوید که کافی است.
\par 17 چشمی که پدر را استهزا می‌کند و اطاعت مادر را خوار می‌شمارد، غرابهای وادی آن راخواهند کند و بچه های عقاب آن را خواهندخورد.
\par 18 سه چیز است که برای من زیاده عجیب است، بلکه چهار چیز که آنها را نتوانم فهمید:
\par 19 طریق عقاب در هوا و طریق مار بر صخره، و راه کشتی در میان دریا و راه مرد با دخترباکره.
\par 20 همچنان است طریق زن زانیه، می‌خورد ودهان خود را پاک می‌کند و می‌گوید گناه نکردم.
\par 21 به‌سبب سه چیز زمین متزلزل می‌شود، و به‌سبب چهار که آنها را تحمل نتواند کرد:
\par 22 به‌سبب غلامی که سلطنت می‌کند، واحمقی که از غذا سیر شده باشد،
\par 23 به‌سبب زن مکروهه چون منکوحه شود، وکنیز وقتی که وارث خاتون خود گردد.
\par 24 چهار چیز است که در زمین بسیار کوچک است، لیکن بسیار حکیم می‌باشد:
\par 25 مورچه‌ها طایفه بی‌قوتند، لیکن خوراک خود را در تابستان ذخیره می‌کنند.
\par 26 ونکها طایفه ناتوانند، اما خانه های خود رادر صخره می‌گذارند.
\par 27 ملخها را پادشاهی نیست، اما جمیع آنهادسته دسته بیرون می‌روند.
\par 28 چلپاسه‌ها به‌دستهای خود می‌گیرند و درقصرهای پادشاهان می‌باشند.
\par 29 سه چیز است که خوش خرام است، بلکه چهار چیز که خوش قدم می‌باشد:
\par 30 شیر که در میان حیوانات تواناتر است، و ازهیچکدام روگردان نیست.
\par 31 تازی و بز نر، و پادشاه که با او مقاومت نتوان کرد.
\par 32 اگر از روی حماقت خویشتن رابرافراشته‌ای و اگر بد اندیشیده‌ای، پس دست بردهان خود بگذار،زیرا چنانکه از فشردن شیر، پنیر بیرون می‌آید، و از فشردن بینی، خون بیرون می‌آید، همچنان از فشردن غضب نزاع بیرون می‌آید.
\par 33 زیرا چنانکه از فشردن شیر، پنیر بیرون می‌آید، و از فشردن بینی، خون بیرون می‌آید، همچنان از فشردن غضب نزاع بیرون می‌آید.
 
\chapter{31}

\par 1 کلام لموئیل پادشاه، پیغامی که مادرش به او تعلیم داد.
\par 2 چه گویم‌ای پسر من، چه گویم‌ای پسر رحم من! و چه گویم‌ای پسر نذرهای من!
\par 3 قوت خود را به زنان مده، و نه طریقهای خویش را به آنچه باعث هلاکت پادشاهان است.
\par 4 پادشاهان را نمی شاید‌ای لموئیل، پادشاهان را نمی شاید که شراب بنوشند، و نه امیران را که مسکرات را بخواهند.
\par 5 مبادا بنوشند و فرایض را فراموش کنند، وداوری جمیع ذلیلان را منحرف سازند.
\par 6 مسکرات را به آنانی که مشرف به هلاکتندبده. و شراب را به تلخ جانان،
\par 7 تا بنوشند و فقر خود را فراموش کنند، ومشقت خویش را دیگر بیاد نیاورند.
\par 8 دهان خود را برای گنگان باز کن، و برای دادرسی جمیع بیچارگان.
\par 9 دهان خود را باز کرده، به انصاف داوری نما، و فقیر و مسکین را دادرسی فرما.
\par 10 زن صالحه را کیست که پیدا تواند کرد؟ قیمت او از لعلها گرانتر است.
\par 11 دل شوهرش بر او اعتماد دارد، و محتاج منفعت نخواهد بود.
\par 12 برایش تمامی روزهای عمر خود، خوبی خواهد کرد و نه بدی.
\par 13 پشم و کتان را می‌جوید. و به‌دستهای خودبا رغبت کار می‌کند.
\par 14 او مثل کشتیهای تجار است، که خوراک خود را از دور می‌آورد.
\par 15 وقتی که هنوز شب است برمی خیزد، و به اهل خانه‌اش خوراک و به کنیزانش حصه ایشان را می دهد.
\par 16 درباره مزرعه فکر کرده، آن را می‌خرد، و ازکسب دستهای خود تاکستان غرس می‌نماید.
\par 17 کمر خود را با قوت می‌بندد، و بازوهای خویش را قوی می‌سازد.
\par 18 تجارت خود را می‌بیند که نیکو است، وچراغش در شب خاموش نمی شود.
\par 19 دستهای خود را به دوک دراز می‌کند، وانگشتهایش چرخ را می‌گیرد.
\par 20 کفهای خود را برای فقیران مبسوطمی سازد، و دستهای خویش را برای مسکینان دراز می‌نماید.
\par 21 به جهت اهل خانه‌اش از برف نمی ترسد، زیرا که جمیع اهل خانه او به اطلس ملبس هستند.
\par 22 برای خود اسبابهای زینت می‌سازد. لباسش از کتان نازک و ارغوان می‌باشد.
\par 23 شوهرش در دربارها معروف می‌باشد، و در میان مشایخ ولایت می‌نشیند.
\par 24 جامه های کتان ساخته آنها را می‌فروشد، وکمربندها به تاجران می‌دهد.
\par 25 قوت و عزت، لباس او است، و درباره وقت آینده می‌خندد.
\par 26 دهان خود را به حکمت می‌گشاید، و تعلیم محبت‌آمیز بر زبان وی است.
\par 27 به رفتار اهل خانه خود متوجه می‌شود، وخوراک کاهلی نمی خورد.
\par 28 پسرانش برخاسته، او را خوشحال می‌گویند، و شوهرش نیز او را می‌ستاید.
\par 29 دختران بسیار اعمال صالحه نمودند، اما توبر جمیع ایشان برتری داری.جمال، فریبنده و زیبایی، باطل است، اما زنی که از خداوند می‌ترسد ممدوح خواهدشد.
\par 30 جمال، فریبنده و زیبایی، باطل است، اما زنی که از خداوند می‌ترسد ممدوح خواهدشد.

\end{document}