\begin{document}

\title{جامعه}

 
\chapter{1}

\par 1 کلام جامعه بن داود که در اورشلیم پادشاه بود:
\par 2 باطل اباطیل، جامعه می‌گوید، باطل اباطیل، همه‌چیز باطل است.
\par 3 انسان را ازتمامی مشقتش که زیر آسمان می‌کشد چه منفعت است؟
\par 4 یک طبقه می‌روند و طبقه دیگرمی آیند و زمین تا به ابد پایدار می‌ماند.
\par 5 آفتاب طلوع می‌کند و آفتاب غروب می‌کند وبه‌جایی که از آن طلوع نمود می‌شتابد.
\par 6 بادبطرف جنوب می‌رود و بطرف شمال دور می‌زند؛ دورزنان دورزنان می‌رود و باد به مدارهای خود برمی گردد.
\par 7 جمیع نهرها به دریاجاری می‌شود اما دریا پر نمی گردد؛ به مکانی که نهرها از آن جاری شد به همان جا بازبرمی گردد.
\par 8 همه‌چیزها پر از خستگی است که انسان آن را بیان نتواند کرد. چشم از دیدن سیر نمی شود و گوش از شنیدن مملونمی گردد.
\par 9 آنچه بوده است همان است که خواهد بود، و آنچه شده است همان است که خواهد شد و زیر آفتاب هیچ‌چیز تازه نیست.
\par 10 آیا چیزی هست که درباره‌اش گفته شود: ببین این تازه است. در دهرهایی که قبل از مابود آن چیز قدیم بود.
\par 11 ذکری از پیشینیان نیست، و از آیندگان نیز که خواهند آمد، نزدآنانی که بعد از ایشان خواهند آمد، ذکری نخواهد بود.
\par 12 من که جامعه هستم بر اسرائیل در اورشلیم پادشاه بودم،
\par 13 و دل خود را بر آن نهادم که در هرچیزی که زیر آسمان کرده می‌شود، با حکمت تفحص و تجسس نمایم. این مشقت سخت است که خدا به بنی آدم داده است که به آن زحمت بکشند.
\par 14 و تمامی کارهایی را که زیر آسمان کرده می‌شود، دیدم که اینک همه آنها بطالت و درپی باد زحمت کشیدن است.
\par 15 کج را راست نتوان کرد و ناقص را بشمار نتوان آورد.
\par 16 در دل خود تفکر نموده، گفتم: اینک من حکمت را به غایت افزودم، بیشتر از همگانی که قبل از من براورشلیم بودند؛ و دل من حکمت و معرفت رابسیار دریافت نمود؛
\par 17 و دل خود را بر دانستن حکمت و دانستن حماقت و جهالت مشغول ساختم. پس فهمیدم که این نیز در‌پی باد زحمت کشیدن است.زیرا که در کثرت حکمت کثرت غم است و هر‌که علم را بیفزاید، حزن رامی افزاید.
\par 18 زیرا که در کثرت حکمت کثرت غم است و هر‌که علم را بیفزاید، حزن رامی افزاید.
 
\chapter{2}

\par 1 من در دل خود گفتم: الان بیا تا تو را به عیش و عشرت بیازمایم؛ پس سعادتمندی راملاحظه نما. و اینک آن نیز بطالت بود.
\par 2 درباره خنده گفتم که مجنون است و درباره شادمانی که چه می‌کند.
\par 3 در دل خود غور کردم که بدن خود را با شراب بپرورم، با آنکه دل من مرا به حکمت (ارشاد نماید) و حماقت را بدست آورم تا ببینم که برای بنی آدم چه چیز نیکو است که آن را زیرآسمان در تمامی ایام عمر خود به عمل آورند.
\par 4 کارهای عظیم برای خود کردم و خانه‌ها برای خود ساختم و تاکستانها به جهت خود غرس نمودم.
\par 5 باغها و فردوسها به جهت خود ساختم ودر آنها هر قسم درخت میوه دار غرس نمودم.
\par 6 حوضهای آب برای خود ساختم تا درختستانی را که در آن درختان بزرگ می‌شود، آبیاری نمایم.
\par 7 غلامان و کنیزان خریدم و خانه زادان داشتم ومرا نیز بیشتر از همه کسانی که قبل از من دراورشلیم بودند اموال از رمه و گله بود.
\par 8 نقره وطلا و اموال خاصه پادشاهان و کشورها نیز برای خود جمع کردم؛ و مغنیان و مغنیات و لذات بنی آدم یعنی بانو و بانوان به جهت خود گرفتم.
\par 9 پس بزرگ شدم و بر تمامی کسانی که قبل از من در اورشلیم بودند برتری یافتم و حکمتم نیز با من برقرار ماند،
\par 10 و هر‌چه چشمانم آرزو می‌کرد ازآنها دریغ نداشتم، و دل خود را از هیچ خوشی بازنداشتم زیرا دلم در هر محنت من شادی می‌نمودو نصیب من از تمامی مشقتم همین بود.
\par 11 پس به تمامی کارهایی که دستهایم کرده بود و به مشقتی که در عمل نمودن کشیده بودم نگریستم؛ و اینک تمامی آن بطالت و در‌پی باد زحمت کشیدن بود ودر زیر آفتاب هیچ منفعت نبود.
\par 12 پس توجه نمودم تا حکمت و حماقت وجهالت را ملاحظه نمایم؛ زیرا کسی‌که بعد از پادشاه بیاید چه خواهد کرد؟ مگر نه آنچه قبل ازآن کرده شده بود؟
\par 13 و دیدم که برتری حکمت برحماقت مثل برتری نور بر ظلمت است.
\par 14 چشمان مرد حکیم در سر وی است اما احمق در تاریکی راه می‌رود. با وجود آن دریافت کردم که بهر دو ایشان یک واقعه خواهد رسید.
\par 15 پس در دل خود تفکر کردم که چون آنچه به احمق واقع می‌شود، به من نیز واقع خواهد گردید، پس من چرا بسیار حکیم بشوم؟ و در دل خود گفتم: این نیز بطالت است،
\par 16 زیرا که هیچ ذکری از مردحکیم و مرد احمق تا به ابد نخواهد بود. چونکه در ایام آینده همه‌چیز بالتمام فراموش خواهدشد. و مرد حکیم چگونه می‌میرد آیا نه مثل احمق؟
\par 17 لهذا من از حیات نفرت داشتم زیرا اعمالی که زیر آفتاب کرده می‌شود، در نظر من ناپسندآمد چونکه تمام بطالت و در‌پی باد زحمت کشیدن است.
\par 18 پس تمامی مشقت خود را که زیر آسمان کشیده بودم مکروه داشتم از اینجهت که باید آن را به کسی‌که بعد از من بیاید واگذارم.
\par 19 و کیست بداند که او حکیم یا احمق خواهدبود، و معهذا بر تمامی مشقتی که من کشیدم و برحکمتی که زیر آفتاب ظاهر ساختم، او تسلطخواهد یافت. این نیز بطالت است.
\par 20 پس من برگشته، دل خویش را از تمامی مشقتی که زیرآفتاب کشیده بودم مایوس ساختم.
\par 21 زیرامردی هست که محنت او با حکمت و معرفت وکامیابی است و آن را نصیب شخصی خواهد ساخت که در آن زحمت نکشیده باشد. این نیزبطالت و بلای عظیم است.
\par 22 زیرا انسان را ازتمامی مشقت و رنج دل خود که زیر آفتاب کشیده باشد چه حاصل می‌شود؟
\par 23 زیرا تمامی روزهایش حزن و مشقتش غم است، بلکه شبانگاه نیز دلش آرامی ندارد. این هم بطالت است.
\par 24 برای انسان نیکو نیست که بخورد و بنوشد وجان خود را از مشقتش خوش سازد. این را نیز من دیدم که از جانب خدا است.
\par 25 زیرا کیست که بتواند بدون او بخورد یا تمتع برد؟زیرا به کسی‌که در نظر او نیکو است، حکمت و معرفت وخوشی را می‌بخشد؛ اما به خطاکار مشقت اندوختن و ذخیره نمودن را می‌دهد تا آن را به کسی‌که در نظر خدا پسندیده است بدهد. این نیزبطالت و در‌پی باد زحمت کشیدن است.
\par 26 زیرا به کسی‌که در نظر او نیکو است، حکمت و معرفت وخوشی را می‌بخشد؛ اما به خطاکار مشقت اندوختن و ذخیره نمودن را می‌دهد تا آن را به کسی‌که در نظر خدا پسندیده است بدهد. این نیزبطالت و در‌پی باد زحمت کشیدن است.
 
\chapter{3}

\par 1 برای هر چیز زمانی است و هر مطلبی رازیر آسمان وقتی است.
\par 2 وقتی برای ولادت و وقتی برای موت. وقتی برای غرس نمودن و وقتی برای کندن مغروس.
\par 3 وقتی برای قتل و وقتی برای شفا. وقتی برای منهدم ساختن ووقتی برای بنا نمودن.
\par 4 وقتی برای گریه و وقتی برای خنده. وقتی برای ماتم و وقتی برای رقص.
\par 5 وقتی برای پراکنده ساختن سنگها و وقتی برای جمع ساختن سنگها. وقتی برای در آغوش کشیدن و وقتی برای اجتناب از در آغوش کشیدن.
\par 6 وقتی برای کسب و وقتی برای خسارت. وقتی برای نگاه داشتن و وقتی برای دورانداختن.
\par 7 وقتی برای دریدن و وقتی برای دوختن. وقتی برای سکوت و وقتی برای گفتن.
\par 8 وقتی برای محبت و وقتی برای نفرت. وقتی برای جنگ و وقتی برای صلح.
\par 9 پس کارکننده را از زحمتی که می‌کشد چه منفعت است؟
\par 10 مشقتی را که خدا به بنی آدم داده است تا در آن زحمت کشند، ملاحظه کردم.
\par 11 او هر چیز را در وقتش نیکو ساخته است و نیزابدیت را در دلهای ایشان نهاده، بطوری که انسان کاری را که خدا کرده است، از ابتدا تا انتها دریافت نتواند کرد.
\par 12 پس فهمیدم که برای ایشان چیزی بهتر از این نیست که شادی کنند و در حیات خودبه نیکویی مشغول باشند.
\par 13 و نیز بخشش خدااست که هر آدمی بخورد و بنوشد و از تمامی زحمت خود نیکویی بیند.
\par 14 و فهمیدم که هرآنچه خدا می‌کند تا ابدالاباد خواهد ماند، و بر آن چیزی نتوان افزود و از آن چیزی نتوان کاست وخدا آن را به عمل می‌آورد تا از او بترسند.
\par 15 آنچه هست از قدیم بوده است و آنچه خواهدشد قدیم است و آنچه را که گذشته است خدامی طلبد.
\par 16 و نیز مکان انصاف را زیر آسمان دیدم که در آنجا ظلم است و مکان عدالت را که در آنجا بی‌انصافی است.
\par 17 و در دل خود گفتم که خدا عادل و ظالم را داوری خواهد نمود زیرا که برای هر امر و برای هر عمل در آنجا وقتی است.
\par 18 و درباره امور بنی آدم در دل خود گفتم: این واقع می‌شود تا خدا ایشان را بیازماید و تا خودایشان بفهمند که مثل بهایم می‌باشند.
\par 19 زیرا که وقایع بنی آدم مثل وقایع بهایم است برای ایشان یک واقعه است؛ چنانکه این می‌میرد به همانطورآن نیز می‌میرد و برای همه یک نفس است و انسان بر بهایم برتری ندارد چونکه همه باطل هستند.
\par 20 همه به یکجا می‌روند و همه از خاک هستند وهمه به خاک رجوع می‌نمایند.
\par 21 کیست روح انسان را بداند که به بالا صعود می‌کند یا روح بهایم را که پایین بسوی زمین نزول می‌نماید؟لهذا فهمیدم که برای انسان چیزی بهتر از این نیست که از اعمال خود مسرور شود، چونکه نصیبش همین است. و کیست که او را بازآورد تاآنچه را که بعد از او واقع خواهد شد مشاهده نماید؟
\par 22 لهذا فهمیدم که برای انسان چیزی بهتر از این نیست که از اعمال خود مسرور شود، چونکه نصیبش همین است. و کیست که او را بازآورد تاآنچه را که بعد از او واقع خواهد شد مشاهده نماید؟
 
\chapter{4}

\par 1 پس من برگشته، تمامی ظلمهایی را که زیرآفتاب کرده می‌شود، ملاحظه کردم. واینک اشکهای مظلومان و برای ایشان تسلی دهنده‌ای نبود! و زور بطرف جفاکنندگان ایشان بود اما برای ایشان تسلی دهنده‌ای نبود!
\par 2 ومن مردگانی را که قبل از آن مرده بودند، بیشتر اززندگانی که تا بحال زنده‌اند آفرین گفتم.
\par 3 و کسی را که تا بحال بوجود نیامده است، از هر دو ایشان بهتر دانستم چونکه عمل بد را که زیر آفتاب کرده می‌شود، ندیده است.
\par 4 و تمامی محنت و هر کامیابی را دیدم که برای انسان باعث حسد از همسایه او می‌باشد. وآن نیز بطالت و در‌پی باد زحمت کشیدن است.
\par 5 مرد کاهل دستهای خود را بر هم نهاده، گوشت خویشتن را می‌خورد.
\par 6 یک کف پر از راحت ازدو کف پر از مشقت و در‌پی باد زحمت کشیدن بهتر است.
\par 7 پس برگشته، بطالت دیگر را زیر آسمان ملاحظه نمودم.
\par 8 یکی هست که ثانی ندارد و او راپسری یا برادری نیست و مشقتش را انتها نی و چشمش نیز از دولت سیر نمی شود. و می‌گوید ازبرای که زحمت کشیده، جان خود را از نیکویی محروم سازم؟ این نیز بطالت و مشقت سخت است.
\par 9 دو از یک بهترند چونکه ایشان را ازمشقتشان اجرت نیکو می‌باشد؛
\par 10 زیرا اگربیفتند، یکی از آنها رفیق خود را خواهدبرخیزانید. لکن وای بر آن یکی که چون بیفتددیگری نباشد که او را برخیزاند.
\par 11 و اگر دو نفرنیز بخوابند، گرم خواهند شد اما یک نفر چگونه گرم شود.
\par 12 و اگر کسی بر یکی از ایشان حمله آورد، هر دو با او مقاومت خواهند نمود. وریسمان سه لا بزودی گسیخته نمی شود.
\par 13 جوان فقیر و حکیم از پادشاه پیر و خرف که پذیرفتن نصیحت را دیگر نمی داند بهتر است.
\par 14 زیرا که او از زندان به پادشاهی بیرون می‌آید وآنکه به پادشاهی مولود شده است فقیر می‌گردد.
\par 15 دیدم که تمامی زندگانی که زیر آسمان راه می‌روند، بطرف آن پسر دوم که بجای او برخیزد، می‌شوند.و تمامی قوم یعنی همه کسانی را که او بر ایشان حاکم شود انتها نیست. لیکن اعقاب ایشان به او رغبت ندارند. به درستی که این نیزبطالت و در‌پی باد زحمت کشیدن است.
\par 16 و تمامی قوم یعنی همه کسانی را که او بر ایشان حاکم شود انتها نیست. لیکن اعقاب ایشان به او رغبت ندارند. به درستی که این نیزبطالت و در‌پی باد زحمت کشیدن است.
 
\chapter{5}

\par 1 چون به خانه خدا بروی، پای خود را نگاه دار زیرا تقرب جستن به جهت استماع، ازگذرانیدن قربانی های احمقان بهتر است، چونکه ایشان نمی دانند که عمل بد می‌کنند.
\par 2 با دهان خود تعجیل منما و دلت برای گفتن سخنی به حضور خدا نشتابد زیرا خدا در آسمان است و تو بر زمین هستی پس سخنانت کم باشد. 
\par 3 زیراخواب از کثرت مشقت پیدا می‌شود و آواز احمق از کثرت سخنان.
\par 4 چون برای خدا نذر نمایی دروفای آن تاخیر منما زیرا که او از احمقان خشنودنیست؛ پس به آنچه نذر کردی وفا نما.
\par 5 بهتراست که نذر ننمایی از اینکه نذر نموده، وفا نکنی.
\par 6 مگذار که دهانت جسد تو را خطاکار سازد و درحضور فرشته مگو که این سهو شده است. چراخدا به‌سبب قول تو غضبناک شده، عمل دستهایت را باطل سازد؟
\par 7 زیرا که این از کثرت خوابها و اباطیل و کثرت سخنان است؛ لیکن تو ازخدا بترس.
\par 8 اگر ظلم را بر فقیران و برکندن انصاف وعدالت را در کشوری بینی، از این امر مشوش مباش زیرا آنکه بالاتر از بالا است ملاحظه می‌کندو حضرت اعلی فوق ایشان است.
\par 9 و منفعت زمین برای همه است بلکه مزرعه، پادشاه را نیزخدمت می‌کند.
\par 10 آنکه نقره را دوست دارد ازنقره سیر نمی شود، و هر‌که توانگری را دوست دارد از دخل سیر نمی شود. این نیز بطالت است.
\par 11 چون نعمت زیاده شود، خورندگانش زیادمی شوند؛ و به جهت مالکش چه منفعت است غیراز آنکه آن را به چشم خود می‌بیند؟
\par 12 خواب عمله شیرین است خواه کم و خواه زیاد بخورد، اما سیری مرد دولتمند او رانمی گذارد که بخوابد.
\par 13 بلایی سخت بود که آن را زیر آفتاب دیدم یعنی دولتی که صاحبش آن را برای ضرر خود نگاه داشته بود.
\par 14 و آن دولت از حادثه بد ضایع شد و پسری آورد اما چیزی در دست خودنداشت.
\par 15 چنانکه از رحم مادرش بیرون آمد، همچنان برهنه به حالتی که آمد خواهد برگشت واز مشقت خود چیزی نخواهد یافت که به‌دست خود ببرد.
\par 16 و این نیز بلای سخت است که از هرجهت چنانکه آمد همچنین خواهد رفت؛ و او راچه منفعت خواهد بود از اینکه در‌پی باد زحمت کشیده است؟
\par 17 و تمامی ایام خود را در تاریکی می‌خورد و با بیماری و خشم، بسیار محزون می‌شود.
\par 18 اینک آنچه من دیدم که خوب و نیکومی باشد، این است که انسان در تمامی ایام عمرخود که خدا آن را به او می‌بخشد بخورد و بنوشدو از تمامی مشقتی که زیر آسمان می‌کشد به نیکویی تمتع ببرد زیرا که نصیبش همین است.
\par 19 و نیز هر انسانی که خدا دولت و اموال به اوببخشد و او را قوت عطا فرماید که از آن بخورد ونصیب خود را برداشته، از محنت خود مسرورشود، این بخشش خدا است.زیرا روزهای عمر خود را بسیار به یاد نمی آورد چونکه خدا اورا از شادی دلش اجابت فرموده است.
\par 20 زیرا روزهای عمر خود را بسیار به یاد نمی آورد چونکه خدا اورا از شادی دلش اجابت فرموده است.
 
\chapter{6}

\par 1 مصیبتی هست که زیر آفتاب دیدم و آن برمردمان سنگین است:
\par 2 کسی‌که خدا به اودولت و اموال و عزت دهد، به حدی که هر‌چه جانش آرزو کند برایش باقی نباشد، اما خدا او راقوت نداده باشد که از آن بخورد بلکه مرد غریبی از آن بخورد. این نیز بطالت و مصیبت سخت است.
\par 3 اگر کسی صد پسر بیاورد و سالهای بسیار زندگانی نماید، به طوری که ایام سالهایش بسیارباشد اما جانش از نیکویی سیر نشود و برایش جنازه‌ای برپا نکنند، می‌گویم که سقطشده از اوبهتر است.
\par 4 زیرا که این به بطالت آمد و به تاریکی رفت و نام او در ظلمت مخفی شد.
\par 5 و آفتاب رانیز ندید و ندانست. این بیشتر از آن آرامی دارد.
\par 6 و اگر هزار سال بلکه دو چندان آن زیست کند ونیکویی را نبیند، آیا همه به یکجا نمی روند؟
\par 7 تمامی مشقت انسان برای دهانش می‌باشد؛ ومعهذا جان او سیر نمی شود.
\par 8 زیرا که مرد حکیم را از احمق چه برتری است؟ و برای فقیری که می‌داند چه طور پیش زندگان سلوک نماید، چه فایده است؟
\par 9 رویت چشم از شهوت نفس بهتر است. این نیز بطالت و در‌پی باد زحمت کشیدن است.
\par 10 هرچه بوده است به اسم خود از زمان قدیم مسمی شده است و دانسته شده است که او آدم است و به آن کسی‌که از آن تواناتر است منازعه نتواند نمود.
\par 11 چونکه چیزهای بسیار هست که بطالت رامی افزاید. پس انسان را چه فضیلت است؟زیراکیست که بداند چه چیز برای زندگانی انسان نیکواست، در مدت ایام حیات باطل وی که آنها رامثل سایه صرف می‌نماید؟ و کیست که انسان را ازآنچه بعد از او زیر آفتاب واقع خواهد شد مخبرسازد؟
\par 12 زیراکیست که بداند چه چیز برای زندگانی انسان نیکواست، در مدت ایام حیات باطل وی که آنها رامثل سایه صرف می‌نماید؟ و کیست که انسان را ازآنچه بعد از او زیر آفتاب واقع خواهد شد مخبرسازد؟
 
\chapter{7}

\par 1 نیک نامی از روغن معطر بهتر است و روزممات از روز ولادت.
\par 2 رفتن به خانه ماتم از رفتن به خانه ضیافت بهتر است زیرا که این آخرت همه مردمان است و زندگان این را در دل خود می‌نهند.
\par 3 حزن از خنده بهتر است زیرا که ازغمگینی صورت دل اصلاح می‌شود.
\par 4 دل حکیمان در خانه ماتم است و دل احمقان در خانه شادمانی.
\par 5 شنیدن عتاب حکیمان بهتر است از شنیدن سرود احمقان،
\par 6 زیرا خنده احمقان مثل صدای خارها در زیر دیگ است و این نیز بطالت است.
\par 7 به درستی که ظلم، مرد حکیم را جاهل می‌گرداند و رشوه، دل را فاسد می‌سازد.
\par 8 انتهای امر از ابتدایش بهتر است و دل حلیم از دل مغرور نیکوتر.
\par 9 در دل خود به زودی خشمناک مشو زیرا خشم در سینه احمقان مستقرمی شود.
\par 10 مگو چرا روزهای قدیم از این زمان بهتربود زیرا که در این خصوص از روی حکمت سوال نمی کنی.
\par 11 حکمت مثل میراث نیکو است بلکه به جهت بینندگان آفتاب نیکوتر.
\par 12 زیرا که حکمت ملجایی است و نقره ملجایی؛ اما فضیلت معرفت این است که حکمت صاحبانش را زندگی می‌بخشد.
\par 13 اعمال خدا را ملاحظه نما زیرا کیست که بتواند آنچه را که او کج ساخته است راست نماید؟
\par 14 در روز سعادتمندی شادمان باش و درروز شقاوت تامل نما زیرا خدا این را به ازاء آن قرار داد که انسان هیچ‌چیز را که بعد از او خواهدشد دریافت نتواند کرد.
\par 15 این همه را در روزهای بطالت خود دیدم. مرد عادل هست که در عدالتش هلاک می‌شود و مرد شریر هست که در شرارتش عمر دراز دارد.
\par 16 پس گفتم به افراط عادل مباش و خود را زیاده حکیم مپندار مبادا خویشتن را هلاک کنی.
\par 17 و به افراط شریر مباش و احمق مشو مبادا پیش ازاجلت بمیری.
\par 18 نیکو است که به این متمسک شوی و از آن نیز دست خود را برنداری زیرا هرکه از خدا بترسد، از این هر دو بیرون خواهد آمد.
\par 19 حکمت مرد حکیم را توانایی می‌بخشدبیشتر از ده حاکم که در یک شهر باشند.
\par 20 زیرامرد عادلی در دنیا نیست که نیکویی ورزد و هیچ خطا ننماید.
\par 21 و نیز به همه سخنانی که گفته شود دل خودرا منه، مبادا بنده خود را که تو را لعنت می‌کندبشنوی.
\par 22 زیرا دلت می‌داند که تو نیز بسیار بارهادیگران را لعنت نموده‌ای.
\par 23 این همه را با حکمت آزمودم و گفتم به حکمت خواهم پرداخت اما آن از من دور بود.
\par 24 آنچه هست دور و بسیار عمیق است. پس کیست که آن را دریافت نماید؟
\par 25 پس برگشته دل خود را بر معرفت و بحث و طلب حکمت وعقل مشغول ساختم تا بدانم که شرارت حماقت است و حماقت دیوانگی است.
\par 26 و دریافتم که زنی که دلش دامها و تله‌ها است و دستهایش کمندها می‌باشد، چیز تلختر از موت است. هر‌که مقبول خدا است، از وی رستگار خواهد شد اماخطاکار گرفتار وی خواهد گردید.
\par 27 جامعه می‌گوید که اینک چون این را با آن مقابله کردم تا نتیجه را دریابم این را دریافتم،
\par 28 که جان من تا به حال آن را جستجو می‌کند ونیافتم. یک مرد از هزار یافتم اما از جمیع آنها زنی نیافتم.همانا این را فقط دریافتم که خدا آدمی را راست آفرید، اما ایشان مخترعات بسیار طلبیدند.
\par 29 همانا این را فقط دریافتم که خدا آدمی را راست آفرید، اما ایشان مخترعات بسیار طلبیدند.
 
\chapter{8}

\par 1 کیست که مثل مرد حکیم باشد وکیست که تفسیر امر را بفهمد؟ حکمت روی انسان راروشن می‌سازد و سختی چهره او تبدیل می‌شود.
\par 2 من تو را می‌گویم حکم پادشاه را نگاه دار واین را به‌سبب سوگند خدا.
\par 3 شتاب مکن تا ازحضور وی بروی و در امر بد جزم منما زیرا که اوهر‌چه می‌خواهد به عمل می‌آورد.
\par 4 جایی که سخن پادشاه است قوت هست و کیست که به اوبگوید چه می‌کنی؟
\par 5 هر‌که حکم را نگاه داردهیچ امر بد را نخواهد دید. و دل مرد حکیم وقت و قانون را می‌داند.
\par 6 زیرا که برای هر مطلب وقتی و قانونی است چونکه شرارت انسان بر وی سنگین است.
\par 7 زیرا آنچه را که واقع خواهد شداو نمی داند؛ و کیست که او را خبر دهد که چگونه خواهد شد؟
\par 8 کسی نیست که بر روح تسلطداشته باشد تا روح خود را نگاه دارد و کسی برروز موت تسلط ندارد؛ و در وقت جنگ مرخصی نیست و شرارت صاحبش را نجات نمی دهد.
\par 9 این همه را دیدم و دل خود را بر هر عملی که زیر آفتاب کرده شود مشغول ساختم، وقتی که انسان بر انسان به جهت ضررش حکمرانی می‌کند.
\par 10 و همچنین دیدم که شریران دفن شدند، و آمدند و از مکان مقدس رفتند و درشهری که در آن چنین عمل نمودند فراموش شدند. این نیز بطالت است.
\par 11 چونکه فتوی برعمل بد بزودی مجرا نمی شود، از این جهت دل بنی آدم در اندرون ایشان برای بدکرداری جازم می شود.
\par 12 اگر‌چه گناهکار صد مرتبه شرارت ورزد و عمر دراز کند، معهذا می‌دانم برای آنانی که از خدا بترسند و به حضور وی خائف باشندسعادتمندی خواهد بود.
\par 13 اما برای شریرسعادتمندی نخواهد بود و مثل سایه، عمر درازنخواهد کرد چونکه از خدا نمی ترسد.
\par 14 بطالتی هست که بر روی زمین کرده می‌شود، یعنی عادلان هستند که بر ایشان مثل عمل شریران واقع می‌شود و شریران‌اند که بر ایشان مثل عمل عادلان واقع می‌شود. پس گفتم که این نیز بطالت است.
\par 15 آنگاه شادمانی را مدح کردم زیرا که برای انسان زیر آسمان چیزی بهتر از این نیست که بخورد و بنوشد و شادی نماید و این در تمامی ایام عمرش که خدا در زیر آفتاب به وی دهد درمحنتش با او باقی ماند.
\par 16 چونکه دل خود را بر آن نهادم تا حکمت رابفهمم و تا شغلی را که بر روی زمین کرده شودببینم (چونکه هستند که شب و روز خواب را به چشمان خود نمی بینند)،آنگاه تمامی صنعت خدا را دیدم که انسان، کاری را که زیر آفتاب کرده می‌شود نمی تواند درک نماید و هر‌چند انسان برای تجسس آن زیاده تر تفحص نماید آن راکمتر درک می‌نماید و اگر‌چه مرد حکیم نیز گمان برد که آن را می‌داند اما آن را درک نخواهد نمود.
\par 17 آنگاه تمامی صنعت خدا را دیدم که انسان، کاری را که زیر آفتاب کرده می‌شود نمی تواند درک نماید و هر‌چند انسان برای تجسس آن زیاده تر تفحص نماید آن راکمتر درک می‌نماید و اگر‌چه مرد حکیم نیز گمان برد که آن را می‌داند اما آن را درک نخواهد نمود.
 
\chapter{9}

\par 1 زیرا که جمیع این مطالب را در دل خود نهادم و این همه را غور نمودم که عادلان وحکیمان و اعمال ایشان در دست خداست. خواه محبت و خواه نفرت، انسان آن را نمی فهمد. همه‌چیز پیش روی ایشان است.
\par 2 همه‌چیز برای همه کس مساوی است. برای عادلان و شریران یک واقعه است؛ برای خوبان و طاهران و نجسان؛ برای آنکه ذبح می‌کند و برای آنکه ذبح نمی کندواقعه یکی است. چنانکه نیکانند همچنان گناهکارانند؛ و آنکه قسم می‌خورد و آنکه از قسم خوردن می‌ترسد مساوی‌اند.
\par 3 در تمامی اعمالی که زیر آفتاب کرده می‌شود، از همه بدتر این است که یک واقعه برهمه می‌شود و اینکه دل بنی آدم از شرارت پراست و مادامی که زنده هستند، دیوانگی در دل ایشان است و بعد از آن به مردگان می‌پیوندند.
\par 4 زیرا برای آنکه با تمامی زندگان می‌پیوندد، امیدهست چونکه سگ زنده از شیر مرده بهتر است.
\par 5 زانرو که زندگان می‌دانند که باید بمیرند، امامردگان هیچ نمی دانند و برای ایشان دیگر اجرت نیست چونکه ذکر ایشان فراموش می‌شود.
\par 6 هم محبت و هم نفرت و حسد ایشان، حال نابود شده است و دیگر تا به ابد برای ایشان از هر‌آنچه زیرآفتاب کرده می‌شود، نصیبی نخواهد بود.
\par 7 پس رفته، نان خود را به شادی بخور وشراب خود را به خوشدلی بنوش چونکه خدااعمال تو را قبل از این قبول فرموده است. 
\par 8 لباس تو همیشه سفید باشد و بر سر تو روغن کم نشود.
\par 9 جمیع روزهای عمر باطل خود را که او تو را درزیر آفتاب بدهد با زنی که دوست می‌داری درجمیع روزهای بطالت خود خوش بگذران. زیراکه از حیات خود و از زحمتی که زیر آفتاب می‌کشی نصیب تو همین است.
\par 10 هر‌چه دستت به جهت عمل نمودن بیابد، همان را با توانایی خود به عمل آور چونکه در عالم اموات که به آن می‌روی نه کار و نه تدبیر و نه علم و نه حکمت است.
\par 11 برگشتم و زیر آفتاب دیدم که مسابقت برای تیزروان و جنگ برای شجاعان و نان نیز برای حکیمان و دولت برای فهیمان و نعمت برای عالمان نیست، زیرا که برای جمیع ایشان وقتی واتفاقی واقع می‌شود.
\par 12 و چونکه انسان نیز وقت خود را نمی داند، پس مثل ماهیانی که در تورسخت گرفتار و گنجشکانی که در دام گرفته می‌شوند، همچنان بنی آدم به وقت نامساعد هرگاه آن بر ایشان ناگهان بیفتد گرفتار می‌گردند.
\par 13 و نیز این حکم را در زیر آفتاب دیدم و آن نزد من عظیم بود:
\par 14 شهری کوچک بود که مردان در آن قلیل العدد بودند و پادشاهی بزرگ بر آن آمده، آن را محاصره نمود و سنگرهای عظیم برپاکرد.
\par 15 و در آن شهر مردی فقیر حکیم یافت شد، که شهر را به حکمت خود رهانید، اما کسی آن مرد فقیر را بیاد نیاورد.
\par 16 آنگاه من گفتم حکمت از شجاعت بهتر است، هر‌چند حکمت این فقیررا خوار شمردند و سخنانش را نشنیدند.
\par 17 سخنان حکیمان که به آرامی گفته شود، ازفریاد حاکمی که در میان احمقان باشد زیاده مسموع می‌گردد.حکمت از اسلحه جنگ بهتر است. اما یک خطاکار نیکویی بسیار را فاسدتواند نمود.
\par 18 حکمت از اسلحه جنگ بهتر است. اما یک خطاکار نیکویی بسیار را فاسدتواند نمود.
 
\chapter{10}

\par 1 فاسد می‌سازد، و اندک حماقتی از حکمت و عزت سنگینتر است.
\par 2 دل مرد حکیم بطرف راستش مایل است و دل احمق بطرف چپش.
\par 3 ونیز چون احمق به راه می‌رود، عقلش ناقص می‌شود و به هر کس می‌گوید که احمق هستم.
\par 4 اگر خشم پادشاه بر تو انگیخته شود، مکان خود را ترک منما زیرا که تسلیم، خطایای عظیم را می‌نشاند.
\par 5 بدی‌ای هست که زیر آفتاب دیده‌ام مثل سهوی که از جانب سلطان صادر شود.
\par 6 جهالت بر مکان های بلند برافراشته می‌شود و دولتمندان در مکان اسفل می‌نشینند.
\par 7 غلامان را بر اسبان دیدم و امیران را مثل غلامان بر زمین روان.
\par 8 آنکه چاه می‌کند در آن می‌افتد و آنکه دیواررا می‌شکافد، مار وی را می‌گزد.
\par 9 آنکه سنگها رامی کند، از آنها مجروح می‌شود و آنکه درختان رامی برد از آنها در خطر می‌افتد.
\par 10 اگر آهن کند باشد و دمش را تیز نکنند بایدقوت زیاده بکار آورد، اما حکمت به جهت کامیابی مفید است.
\par 11 اگر مار پیش از آنکه افسون کنند بگزد پس افسونگر‌چه فایده دارد؟
\par 12 سخنان دهان حکیم فیض بخش است، امالبهای احمق خودش را می‌بلعد.
\par 13 ابتدای سخنان دهانش حماقت است و انتهای گفتارش دیوانگی موذی می‌باشد.
\par 14 احمق سخنان بسیارمی گوید، اما انسان آنچه را که واقع خواهد شدنمی داند و کیست که او را از آنچه بعد از وی واقع خواهد شد مخبر سازد؟
\par 15 محنت احمقان ایشان را خسته می‌سازد چونکه نمی دانند چگونه به شهر باید رفت.
\par 16 وای بر تو‌ای زمین وقتی که پادشاه تو طفل است و سرورانت صبحگاهان می‌خورند.
\par 17 خوشابحال تو‌ای زمین هنگامی که پادشاه توپسر نجبا است و سرورانت در وقتش برای تقویت می‌خورند و نه برای مستی.
\par 18 از کاهلی سقف خراب می‌شود و از سستی دستها، خانه آب پس می‌دهد.
\par 19 بزم به جهت لهو و لعب می‌کنند و شراب زندگانی را شادمان می‌سازد، اما نقره همه‌چیز رامهیا می‌کند.پادشاه را در فکر خود نیز نفرین مکن ودولتمند را در اطاق خوابگاه خویش لعنت منمازیرا که مرغ هوا آواز تو را خواهد برد و بالدار، امررا شایع خواهد ساخت.
\par 20 پادشاه را در فکر خود نیز نفرین مکن ودولتمند را در اطاق خوابگاه خویش لعنت منمازیرا که مرغ هوا آواز تو را خواهد برد و بالدار، امررا شایع خواهد ساخت.
 
\chapter{11}

\par 1 نان خود را بروی آبها بینداز، زیرا که بعد از روزهای بسیار آن را خواهی یافت.
\par 2 نصیبی به هفت نفر بلکه به هشت نفر ببخش زیراکه نمی دانی چه بلا بر زمین واقع خواهد شد.
\par 3 اگرابرها پر از باران شود، آن را بر زمین می‌باراند و اگردرخت بسوی جنوب یا بسوی شمال بیفتد، درهمانجا که درخت افتاده است خواهد ماند.
\par 4 آنکه به باد نگاه می‌کند، نخواهد کشت و آنکه به ابرها نظر نماید، نخواهد دروید.
\par 5 چنانکه تونمی دانی که راه باد چیست یا چگونه استخوانهادر رحم زن حامله بسته می‌شود، همچنین عمل خدا را که صانع کل است نمی فهمی.
\par 6 بامدادان تخم خود را بکار و شامگاهان دست خود را بازمدار زیرا تو نمی دانی کدام‌یک از آنها این یا آن کامیاب خواهد شد یا هر دو آنها مثل هم نیکوخواهد گشت.
\par 7 البته روشنایی شیرین است ودیدن آفتاب برای چشمان نیکو است.
\par 8 هر‌چندانسان سالهای بسیار زیست نماید و در همه آنهاشادمان باشد، لیکن باید روزهای تاریکی را به یادآورد چونکه بسیار خواهد بود. پس هر‌چه واقع می‌شود بطالت است.
\par 9 ‌ای جوان در وقت شباب خود شادمان باش و در روزهای جوانی ات دلت تو را خوش سازد ودر راههای قلبت و بر وفق رویت چشمانت سلوک نما، لیکن بدان که به‌سبب این همه خدا تورا به محاکمه خواهد آورد.پس غم را از دل خود بیرون کن و بدی را از جسد خویش دور نمازیرا که جوانی و شباب باطل است.
\par 10 پس غم را از دل خود بیرون کن و بدی را از جسد خویش دور نمازیرا که جوانی و شباب باطل است.
 
\chapter{12}

\par 1 پس آفریننده خود را در روزهای جوانی ات بیاد آور قبل از آنکه روزهای بلا برسد و سالها برسد که بگویی مرا از اینهاخوشی نیست.
\par 2 قبل از آنکه آفتاب و نور و ماه وستارگان تاریک شود و ابرها بعد از باران برگردد؛
\par 3 در روزی که محافظان خانه بلرزند و صاحبان قوت، خویشتن را خم نمایند و دستاس کنندگان چونکه کم‌اند باز ایستند و آنانی که از پنجره هامی نگرند تاریک شوند.
\par 4 و درها در کوچه بسته شود و آواز آسیاب پست گردد و از صدای گنجشک برخیزد و جمیع مغنیات ذلیل شوند.
\par 5 واز هر بلندی بترسند و خوفها در راه باشد ودرخت بادام شکوفه آورد و ملخی بار سنگین باشد و اشتها بریده شود. چونکه انسان به خانه جاودانی خود می‌رود و نوحه‌گران در کوچه گردش می‌کنند.
\par 6 قبل از آنکه مفتول نقره گسیخته شود و کاسه طلا شکسته گردد و سبو نزدچشمه خرد شود و چرخ بر چاه منکسر گردد،
\par 7 وخاک به زمین برگردد به طوری که بود. و روح نزدخدا که آن را بخشیده بود رجوع نماید.
\par 8 باطل اباطیل جامعه می‌گوید همه‌چیز بطالت است.
\par 9 و دیگر چونکه جامعه حکیم بود باز هم معرفت را به قوم تعلیم می‌داد و تفکر نموده، غوررسی می‌کرد و مثل های بسیار تالیف نمود.
\par 10 جامعه تفحص نمود تا سخنان مقبول را پیداکند و کلمات راستی را که به استقامت مکتوب باشد.
\par 11 سخنان حکیمان مثل سکهای گاورانی است و کلمات ارباب جماعت مانند میخهای محکم شده می‌باشد، که از یک شبان داده شود.
\par 12 و علاوه بر اینها، ای پسر من پند بگیر. ساختن کتابهای بسیار انتها ندارد و مطالعه زیاد، تعب بدن است.پس ختم تمام امر را بشنویم. از خدابترس و اوامر او را نگاه دار چونکه تمامی تکلیف انسان این است.
\par 13 پس ختم تمام امر را بشنویم. از خدابترس و اوامر او را نگاه دار چونکه تمامی تکلیف انسان این است.

\end{document}