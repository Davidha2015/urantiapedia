\begin{document}

\title{Jeremiah}

 
\chapter{1}

\par 1 کلام ارمیا ابن حلقیا از کاهنانی که درعناتوت در زمین بنیامین بودند.
\par 2 که کلام خداوند در ایام یوشیا ابن آمون پادشاه یهودا درسال سیزدهم از سلطنت او بر وی نازل شد.
\par 3 و درایام یهویاقیم بن یوشیا پادشاه یهودا تا آخر سال یازدهم صدقیا ابن یوشیا پادشاه یهودا نازل می‌شد تا زمانی که اورشلیم در ماه پنجم به اسیری برده شد.
\par 4 پس کلام خداوند بر من نازل شده، گفت:
\par 5 «قبل از آنکه تو را در شکم صورت بندم تو راشناختم و قبل از بیرون آمدنت از رحم تو راتقدیس نمودم و تو را نبی امت‌ها قرار دادم.»
\par 6 پس گفتم: «آه‌ای خداوند یهوه اینک من تکلم کردن را نمی دانم چونکه طفل هستم.»
\par 7 اماخداوند مرا گفت: «مگو من طفل هستم، زیرا هرجایی که تو را بفرستم خواهی رفت و بهر‌چه تورا امر فرمایم تکلم خواهی نمود.
\par 8 از ایشان مترس زیرا خداوند می‌گوید: من با تو هستم و تورا رهایی خواهم داد.»
\par 9 آنگاه خداوند دست خود را دراز کرده، دهان مرا لمس کرد و خداوندبه من گفت: «اینک کلام خود را در دهان تو نهادم.
\par 10 بدان که تو را امروز بر امت‌ها و ممالک مبعوث کردم تا از ریشه برکنی و منهدم سازی و هلاک کنی و خراب نمایی و بنا نمایی و غرس کنی.»
\par 11 پس کلام خداوند بر من نازل شده، گفت: «ای ارمیا چه می‌بینی؟ گفتم: «شاخه‌ای از درخت بادام می‌بینم.»
\par 12 خداوند مرا گفت: «نیکو دیدی زیرا که من بر کلام خود دیده بانی می‌کنم تا آن رابه انجام رسانم.»
\par 13 پس کلام خداوند بار دیگر به من رسیده، گفت: «چه چیز می‌بینی؟» گفتم: «دیگی جوشنده می‌بینم که رویش از طرف شمال است.»
\par 14 و خداوند مرا گفت: «بلایی از طرف شمال بر جمیع سکنه این زمین منبسط خواهدشد.
\par 15 زیرا خداوند می‌گوید اینک من جمیع قبایل ممالک شمالی را خواهم خواند و ایشان آمده، هر کس کرسی خود را در دهنه دروازه اورشلیم و بر تمامی حصارهایش گرداگرد و به ضد تمامی شهرهای یهودا برپا خواهد داشت.
\par 16 و بر ایشان احکام خود را درباره همه شرارتشان جاری خواهم ساخت چونکه مرا ترک کردند و برای خدایان غیر بخور‌سوزانیدند واعمال دستهای خود را سجده نمودند.
\par 17 پس توکمر خود را ببند و برخاسته، هر‌آنچه را من به توامر فرمایم به ایشان بگو و از ایشان هراسان مباش، مبادا تو را پیش روی ایشان مشوش سازم.
\par 18 زیرااینک من تو را امروز شهر حصاردار و ستون آهنین و حصارهای برنجین به ضد تمامی زمین برای پادشاهان یهودا و سروران و کاهنانش و قوم زمین ساختم.و ایشان با تو جنگ خواهند کرداما بر تو غالب نخواهند آمد، زیرا خداوند می گوید: من با تو هستم و تو را رهایی خواهم داد.»
\par 19 و ایشان با تو جنگ خواهند کرداما بر تو غالب نخواهند آمد، زیرا خداوند می گوید: من با تو هستم و تو را رهایی خواهم داد.»
 
\chapter{2}

\par 1 و کلام خداوند بر من نازل شده، گفت:
\par 2 «برو و به گوش اورشلیم ندا کرده، بگوخداوند چنین می‌گوید: غیرت جوانی تو ومحبت نامزد شدن تو را حینی که از عقب من دربیابان و در زمین لم یزرع می‌خرامیدی برایت به‌خاطر می‌آورم.
\par 3 اسرائیل برای خداوند مقدس ونوبر محصول او بود. خداوند می‌گوید: آنانی که اورا بخورند مجرم خواهند شد و بلا بر ایشان مستولی خواهد گردید.»
\par 4 ‌ای خاندان یعقوب و جمیع قبایل خانواده اسرائیل کلام خداوند را بشنوید!
\par 5 خداوند چنین می‌گوید: «پدران شما در من چه بی‌انصافی یافتندکه از من دوری ورزیدند و اباطیل را پیروی کرده، باطل شدند؟
\par 6 و نگفتند: یهوه کجا است که ما رااز زمین مصر برآورد و ما را در بیابان و زمین ویران و پر از حفره‌ها و زمین خشک و سایه موت وزمینی که کسی از آن گذر نکند و آدمی در آن ساکن نشود رهبری نمود؟
\par 7 و من شما را به زمین بستانها آوردم تا میوه‌ها و طیبات آن را بخورید، اما چون داخل آن شدید زمین مرا نجس ساختیدومیراث مرا مکروه گردانیدید.
\par 8 کاهنان نگفتند: یهوه کجاست و خوانندگان تورات مرا نشناختندو شبانان بر من عاصی شدند و انبیا برای بعل نبوت کرده، در عقب چیزهای بی‌فایده رفتند.
\par 9 بنابراین خداوند می‌گوید: بار دیگر با شما مخاصمه خواهم نمود و با پسران پسران شما مخاصمه خواهم کرد.
\par 10 پس به جزیره های کتیم گذرکرده، ملاحظه نمایید و به قیدار فرستاده به دقت تعقل نمایید و دریافت کنید که آیا حادثه‌ای مثل این واقع شده باشد؟
\par 11 که آیا هیچ امتی خدایان خویش را عوض کرده باشند با آنکه آنها خدانیستند؟ اما قوم من جلال خویش را به آنچه فایده‌ای ندارد عوض نمودند.
\par 12 پس خداوندمی گوید: ای آسمانها از این متحیر باشید و به خود لرزیده، به شدت مشوش شوید!
\par 13 زیرا قوم من دو کار بد کرده‌اند. مرا که چشمه آب حیاتم ترک نموده و برای خود حوضها کنده‌اند، یعنی حوضهای شکسته که آب را نگاه ندارد.
\par 14 آیااسرائیل غلام یا خانه زاد است پس چرا غارت شده باشد؟
\par 15 شیران ژیان بر او غرش نموده، آواز خود را بلند کردند و زمین او را ویران ساختند و شهرهایش سوخته و غیرمسکون گردیده است.
\par 16 و پسران نوف و تحفنیس فرق تورا شکسته‌اند.
\par 17 آیا این را بر خویشتن واردنیاوردی چونکه یهوه خدای خود را حینی که تورا رهبری می‌نمود ترک کردی؟
\par 18 و الان تو را باراه مصر چه‌کار است تا آب شیحور را بنوشی؟ وتو را با راه آشور چه‌کار است تا آب فرات رابنوشی؟»
\par 19 خداوند یهوه صبایوت چنین می‌گوید: «شرارت تو، تو را تنبیه کرده و ارتداد تو، تو راتوبیخ نموده است پس بدان و ببین که این امرزشت و تلخ است که یهوه خدای خود را ترک نمودی و ترس من در تو نیست.
\par 20 زیرا از زمان قدیم یوغ تو را شکستم و بندهای تو را گسیختم وگفتی بندگی نخواهم نمود زیرا بر هر تل بلند وزیر هر درخت سبز خوابیده، زنا کردی.
\par 21 و من تو را مو اصیل و تخم تمام نیکو غرس نمودم پس چگونه نهال مو بیگانه برای من گردیده‌ای؟
\par 22 پس اگر‌چه خویشتن را با اشنان بشویی و صابون برای خود زیاده بکار بری، اما خداوند یهوه می‌گویدکه گناه تو پیش من رقم شده است.
\par 23 چگونه می‌گویی که نجس نشدم و در عقب بعلیم نرفتم؟ طریق خویش را در وادی بنگر و به آنچه کردی اعتراف نما‌ای شتر تیزرو که در راههای خودمی دوی!
\par 24 مثل گورخر هستی که به بیابان عادت داشته، در شهوت دل خود باد را بو می‌کشد. کیست که از شهوتش او را برگرداند؟ آنانی که اورا می‌طلبند خسته نخواهند شد و او را در ماهش خواهند یافت.
\par 25 پای خود را از برهنگی و گلوی خویش را از تشنگی باز دار. اما گفتی نی امیدنیست زیرا که غریبان را دوست داشتم و از عقب ایشان خواهم رفت.
\par 26 مثل دزدی که چون گرفتارشود خجل گردد. همچنین خاندان اسرائیل باپادشاهان و سروران و کاهنان و انبیای ایشان خجل خواهند شد.
\par 27 که به چوب می‌گویند توپدر من هستی و به سنگ که تو مرا زاییده‌ای زیراکه پشت به من دادند و نه رو. اما در زمان مصیبت خود می‌گویند: برخیز و ما را نجات ده.
\par 28 پس خدایان تو که برای خود ساختی کجایند؟ ایشان در زمان مصیتت برخیزند و تو را نجات دهند. زیرا که‌ای یهودا خدایان تو به شماره شهرهای تومی باشند.»
\par 29 خداوند می‌گوید: «چرا با من مخاصمه می‌نمایید جمیع شما بر من عاصی شده‌اید.
\par 30 پسران شما را عبث زده‌ام زیرا که تادیب رانمی پذیرند. شمشیر شما مثل شیر درنده انبیای شما را هلاک کرده است.
\par 31 ‌ای شما که اهل این عصر می‌باشید کلام خداوند را بفهمید! آیا من برای اسرائیل مثل بیابان یا زمین ظلمت غلیظشده‌ام؟ پس قوم من چرا می‌گویند که روسای خود شده‌ایم و بار دیگر نزد تو نخواهیم آمد.
\par 32 آیا دوشیزه زیور خود را یا عروس آرایش خود را فراموش کند؟ اما قوم من روزهای بیشمارمرا فراموش کرده‌اند.
\par 33 چگونه راه خود را مهیامی سازی تا محبت را بطلبی؟ بنابراین زنان بد رانیز به راههای خود تعلیم دادی.
\par 34 در دامنهای تونیز خون جان فقیران بی‌گناه یافته شد. آنها را درنقب زدن نیافتم بلکه بر جمیع آنها.
\par 35 و می‌گویی: چونکه بی‌گناه هستم غضب او از من برگردانیده خواهد شد. اینک به‌سبب گفتنت که گناه نکرده‌ام بر تو داوری خواهم نمود.
\par 36 چرا اینقدرمی شتابی تا راه خود را تبدیل نمایی؟ چنانکه ازآشور خجل شدی همچنین از مصر نیز خجل خواهی شد.از این نیز دستهای خود را برسرت نهاده، بیرون خواهی آمد. چونکه خداونداعتماد تو را خوار شمرده است پس از ایشان کامیاب نخواهی شد.»
\par 37 از این نیز دستهای خود را برسرت نهاده، بیرون خواهی آمد. چونکه خداونداعتماد تو را خوار شمرده است پس از ایشان کامیاب نخواهی شد.»
 
\chapter{3}

\par 1 و می‌گوید: «اگر مرد، زن خود را طلاق دهدو او از وی جدا شده، زن مرد دیگری بشودآیا بار دیگر به آن زن رجوع خواهد نمود؟ مگرآن زمین بسیار ملوث نخواهد شد؟ لیکن خداوندمی گوید: تو با یاران بسیار زنا کردی اما نزد من رجوع نما.
\par 2 چشمان خود را به بلندیها برافراز وببین که کدام جا است که در آن با تو همخواب نشده‌اند. برای ایشان بسر راهها مثل (زن ) عرب در بیابان نشستی و زمین را به زنا و بدرفتاری خودملوث ساختی.
\par 3 پس بارش‌ها بازداشته شد وباران بهاری نیامد و تو را جبین زن زانیه بوده، حیارا از خود دور کردی.
\par 4 آیا از این به بعد مرا صدانخواهی زد که‌ای پدر من، تو یار جوانی من بودی؟
\par 5 آیا غضب خود را تا به ابد خواهد نمود وآن را تا به آخر نگاه خواهد داشت؟ اینک این راگفتی اما اعمال بد را بجا آورده، کامیاب شدی.»
\par 6 و خداوند در ایام یوشیا پادشاه به من گفت: «آیا ملاحظه کردی که اسرائیل مرتد چه کرده است؟ چگونه به فراز هر کوه بلند و زیر هردرخت سبز رفته در آنجا زنا کرده است؟
\par 7 و بعداز آنکه همه این کارها را کرده بود من گفتم نزد من رجوع نما، اما رجوع نکرد و خواهر خائن اویهودا این را بدید.
\par 8 و من دیدم با آنکه اسرائیل مرتد زنا کرد و از همه جهات او را بیرون کردم وطلاق نامه‌ای به وی دادم لکن خواهر خائن اویهودا نترسید بلکه او نیز رفته، مرتکب زنا شد.
\par 9 وواقع شد که به‌سبب سهل انگاری او در زناکاریش زمین ملوث گردید و او با سنگها و چوبها زنانمود.
\par 10 و نیز خداوند می‌گوید: با وجود این همه، خواهر خائن او یهودا نزد من با تمامی دل خود رجوع نکرد بلکه با ریاکاری.»
\par 11 پس خداوند مرا گفت: «اسرائیل مرتدخویشتن را از یهودای خائن عادلتر نموده است.
\par 12 لهذا برو و این سخنان را بسوی شمال نداکرده، بگو: خداوند می‌گوید: ای اسرائیل مرتدرجوع نما! و بر تو غضب نخواهم نمود زیراخداوند می‌گوید: من روف هستم و تا به ابد خشم خود را نگاه نخواهم داشت.
\par 13 فقط به گناهانت اعتراف نما که بر یهوه خدای خویش عاصی شدی و راههای خود را زیر هر درخت سبز برای بیگانگان منشعب ساختی و خداوند می‌گوید که شما آواز مرا نشنیدید.
\par 14 پس خداوند می‌گوید: ای پسران مرتد رجوع نمایید زیرا که من شوهرشما هستم و از شما یک نفر از شهری و دو نفر ازقبیله‌ای گرفته، شما را به صهیون خواهم آورد.
\par 15 و به شما شبانان موافق دل خود خواهم داد که شما را به معرفت و حکمت خواهند چرانید.
\par 16 «و خداوند می‌گوید که چون در زمین افزوده و بارور شوید در آن ایام بار دیگر تابوت عهد یهوه را به زبان نخواهند‌آورد و آن به‌خاطرایشان نخواهد آمد و آن را ذکر نخواهند کرد و آن را زیارت نخواهند نمود و بار دیگر ساخته نخواهد شد.
\par 17 زیرا در آن زمان اورشلیم راکرسی یهوه خواهند نامید و تمامی امت‌ها به آنجابه جهت اسم یهوه به اورشلیم جمع خواهند شدو ایشان بار دیگر پیروی سرکشی دلهای شریرخود را نخواهند نمود.
\par 18 و در آن ایام خاندان یهودا با خاندان اسرائیل راه خواهند رفت وایشان از زمین شمال به آن زمینی که نصیب پدران ایشان ساختم با هم خواهند آمد.
\par 19 و گفتم که من تو را چگونه در میان پسران قرار دهم و زمین مرغوب و میراث زیباترین امت‌ها را به تو دهم؟ پس گفتم که مرا پدر خواهی خواند و از من دیگرمرتد نخواهی شد. 
\par 20 خداوند می‌گوید: هر آینه مثل زنی که به شوهر خود خیانت ورزد همچنین شما‌ای خاندان اسرائیل به من خیانت ورزیدید.
\par 21 آواز گریه و تضرعات بنی‌اسرائیل از بلندیهاشنیده می‌شود زیرا که راههای خود را منحرف ساخته و یهوه خدای خود را فراموش کرده‌اند.
\par 22 ‌ای فرزندان مرتد بازگشت نمایید و من ارتدادهای شما را شفا خواهم داد.» (ومی گویند): «اینک نزد تو می‌آییم زیرا که تو یهوه خدای ما هستی.
\par 23 به درستی که ازدحام کوههااز تلها باطل می‌باشد. زیرا به درستی که نجات اسرائیل در یهوه خدای ما است.
\par 24 و خجالت مشقت پدران ما، یعنی رمه و گله و پسران ودختران ایشان را از طفولیت ما تلف کرده است.در خجالت خود می‌خوابیم و رسوایی ما، ما رامی پوشاند زیرا که هم ما و هم پدران ما از طفولیت خود تا امروز به یهوه خدای خویش گناه ورزیده و آواز یهوه خدای خویش را نشنیده‌ایم.»
\par 25 در خجالت خود می‌خوابیم و رسوایی ما، ما رامی پوشاند زیرا که هم ما و هم پدران ما از طفولیت خود تا امروز به یهوه خدای خویش گناه ورزیده و آواز یهوه خدای خویش را نشنیده‌ایم.»
 
\chapter{4}

\par 1 بازگشت نمایی، اگر نزد من بازگشت نمایی و اگر رجاسات خود را از خود دور نمایی پراکنده نخواهی شد.
\par 2 و به راستی و انصاف و عدالت به حیات یهوه قسم خواهی خورد و امت هاخویشتن را به او مبارک خواهند خواند و به وی فخر خواهند کرد.»
\par 3 زیرا خداوند به مردان یهودا و اورشلیم چنین می‌گوید: «زمینهای خود را شیار کنید و درمیان خارها مکارید.
\par 4 ‌ای مردان یهودا و ساکنان اورشلیم خویشتن را برای خداوند مختون سازیدو غلفه دلهای خود را دور کنید مبادا حدت خشم من به‌سبب بدی اعمال شما مثل آتش صادرشده، افروخته گردد و کسی آن را خاموش نتواندکرد.
\par 5 در یهودا اخبار نمایید و در اورشلیم اعلان نموده، بگویید و در زمین کرنا بنوازید و به آوازبلند ندا کرده، بگویید که جمع شوید تا به شهرهای حصاردار داخل شویم.
\par 6 علمی بسوی صهیون برافرازید و برای پناه فرار کرده، توقف منمایید زیرا که من بلایی و شکستی عظیم ازطرف شمال می‌آورم.
\par 7 شیری از بیشه خودبرآمده و هلاک کننده امت‌ها حرکت کرده، از مکان خویش درآمده است تا زمین تو را ویران سازد و شهرهایت خراب شده، غیرمسکون گردد.
\par 8 از این جهت پلاس پوشیده، ماتم گیرید و ولوله کنید زیرا که حدت خشم خداوند از ما برنگشته است.
\par 9 و خداوند می‌گوید که در آن روز دل پادشاه و دل سروران شکسته خواهد شد و کاهنان متحیر و انبیا مشوش خواهند گردید.»
\par 10 پس گفتم: «آه‌ای خداوند یهوه! به تحقیق این قوم و اورشلیم را بسیار فریب دادی زیرا گفتی شما را سلامتی خواهد بود و حال آنکه شمشیر به‌جان رسیده است.»
\par 11 در آن زمان به این قوم و به اورشلیم گفته خواهد شد که باد سموم ازبلندیهای بیابان بسوی دختر قوم من خواهد وزیدنه برای افشاندن و پاک کردن خرمن.
\par 12 باد شدیداز اینها برای من خواهد وزید و من نیز الان برایشان داوری‌ها خواهم فرمود.
\par 13 اینک او مثل ابر می‌آید و ارابه های او مثل گردباد و اسبهای اواز عقاب تیزروترند. وای بر ما زیرا که غارت شده‌ایم.
\par 14 ‌ای اورشلیم دل خود را از شرارت شست وشو کن تا نجات یابی! تا به کی خیالات فاسد تو در دلت بماند؟
\par 15 زیرا آوازی از دان اخبار می‌نماید و از کوهستان افرایم به مصیبتی اعلان می‌کند.
\par 16 امت‌ها را اطلاع دهید، هان به ضد اورشلیم اعلان کنید که محاصره کنندگان از ولایت بعیدمی آیند و به آواز خود به ضد شهرهای یهودا ندامی کنند.
\par 17 خداوند می‌گوید که مثل دیده بانان مزرعه او را احاطه می‌کنند چونکه بر من فتنه انگیخته است.
\par 18 راه تو و اعمال تو این چیزها رابر تو وارد آورده است. این شرارت تو به حدی تلخ است که به دلت رسیده است.
\par 19 احشای من احشای من، پرده های دل من ازدرد سفته شد و قلب من در اندرونم مشوش گردیده، ساکت نتوانم شد چونکه تو‌ای جان من آواز کرنا و نعره جنگ را شنیده‌ای.
\par 20 شکستگی بر شکستگی اعلان شده زیرا که تمام زمین غارت شده است و خیمه های من بغته و پرده هایم ناگهان به تاراج رفته است.
\par 21 تا به کی علم را ببینم و آوازکرنا را بشنوم؟
\par 22 چونکه قوم من احمقند و مرانمی شناسند و ایشان، پسران ابله هستند و هیچ فهم ندارند. برای بدی کردن ماهرند لیکن به جهت نیکوکاری هیچ فهم ندارند.
\par 23 بسوی زمین نظر انداختم و اینک تهی وویران بود و بسوی آسمان و هیچ نور نداشت.
\par 24 بسوی کوهها نظر انداختم و اینک متزلزل بود وتمام تلها از جا متحرک می‌شد.
\par 25 نظر کردم واینک آدمی نبود و تمامی مرغان هوا فرار کرده بودند.
\par 26 نظر کردم و اینک بوستانها بیابان گردیده و همه شهرها از حضور خداوند و از حدت خشم وی خراب شده بود.
\par 27 زیرا خداوند چنین می‌گوید: «تمامی زمین خراب خواهد شد لیکن آن را بالکل فانی نخواهم ساخت.
\par 28 از این سبب جهان ماتم خواهد گرفت و آسمان از بالا سیاه خواهد شد زیرا که این راگفتم و اراده نمودم و پشیمان نخواهم شد و از آن بازگشت نخواهم نمود.»
\par 29 از آواز سواران وتیراندازان تمام اهل شهر فرار می‌کنند و به جنگلها داخل می‌شوند و بر صخره‌ها برمی آیندو تمامی شهرها ترک شده، احدی در آنها ساکن نمی شود.
\par 30 و تو حینی که غارت شوی چه خواهی کرد؟ اگر‌چه خویشتن را به قرمز ملبس سازی و به زیورهای طلا بیارایی و چشمان خود را از سرمه جلا دهی لیکن خود را عبث زیبایی داده‌ای چونکه یاران تو تو را خوار شمرده، قصدجان تو دارند.زیرا که آوازی شنیدم مثل آواززنی که درد زه دارد و تنگی مثل زنی که نخست زاده خویش را بزاید یعنی آواز دخترصهیون را که آه می‌کشد و دستهای خود را درازکرده، می‌گوید: وای بر من زیرا که جان من به‌سبب قاتلان بیهوش شده است.
\par 31 زیرا که آوازی شنیدم مثل آواززنی که درد زه دارد و تنگی مثل زنی که نخست زاده خویش را بزاید یعنی آواز دخترصهیون را که آه می‌کشد و دستهای خود را درازکرده، می‌گوید: وای بر من زیرا که جان من به‌سبب قاتلان بیهوش شده است.
 
\chapter{5}

\par 1 «در کوچه های اورشلیم گردش کرده، ببینید و بفهمید و در چهارسوهایش تفتیش نمایید که آیا کسی را که به انصاف عمل نماید و طالب راستی باشد توانید یافت تا من آن رابیامرزم؟
\par 2 و اگر‌چه بگویند: قسم به حیات یهوه، لیکن به دروغ قسم می‌خورند.»
\par 3 ‌ای خداوند آیا چشمان تو براستی نگران نیست؟ ایشان را زدی اما محزون نشدند. و ایشان را تلف نمودی اما نخواستند تادیب را بپذیرند. رویهای خود را از صخره سختتر گردانیدند ونخواستند بازگشت نمایند.
\par 4 و من گفتم: «به درستی که اینان فقیرند و جاهل هستند که راه خداوند و احکام خدای خود را نمی دانند.
\par 5 پس نزد بزرگان می‌روم و با ایشان تکلم خواهم نمودزیرا که ایشان طریق خداوند و احکام خدای خود را می‌دانند.» لیکن ایشان متفق یوغ راشکسته و بندها را گسیخته‌اند.
\par 6 از این جهت شیری از جنگل ایشان را خواهد کشت و گرگ بیابان ایشان را تاراج خواهد کرد و پلنگ برشهرهای ایشان در کمین خواهد نشست و هر‌که از آنها بیرون رود دریده خواهد شد، زیرا که تقصیرهای ایشان بسیار و ارتدادهای ایشان عظیم است.
\par 7 «چگونه تو را برای این بیامرزم که پسرانت مرا ترک کردند و به آنچه خدا نیست قسم خوردند و چون من ایشان را سیر نمودم مرتکب زنا شدند و در خانه های فاحشه‌ها ازدحام نمودند.
\par 8 مثل اسبان پرورده شده مست شدند که هر یکی از ایشان برای زن همسایه خود شیهه می‌زند.
\par 9 و خداوند می‌گوید: «آیا به‌سبب این کارهاعقوبت نخواهم رسانید و آیا جان من از چنین طایفه‌ای انتقام نخواهد کشید؟»
\par 10 برحصارهایش برآیید و آنها را خراب کنید امابالکل هلاک مکنید و شاخه هایش را قطع نماییدزیرا که از آن خداوند نیستند.
\par 11 خداوند می‌گوید: «هر آینه خاندان اسرائیل و خاندان یهودا به من به شدت خیانت ورزیده‌اند.»
\par 12 خداوند را انکار نموده، می‌گویندکه او نیست و بلا به ما نخواهد رسید و شمشیر وقحط را نخواهیم دید.
\par 13 و انبیا باد می‌شوند وکلام در ایشان نیست پس به ایشان چنین واقع خواهد شد.
\par 14 بنابراین یهوه خدای صبایوت چنین می‌گوید: «چونکه این کلام را گفتید همانامن کلام خود را در دهان تو آتش و این قوم راهیزم خواهم ساخت و ایشان را خواهد سوزانید.»
\par 15 خداوند می‌گوید: «ای خاندان اسرائیل، اینک من امتی را از دور بر شما خواهم آورد. امتی که زورآورند و امتی که قدیمند و امتی که زبان ایشان را نمی دانی و گفتار ایشان را نمی فهمی.
\par 16 ترکش ایشان قبر گشاده است و جمیع ایشان جبارند.
\par 17 و خرمن و نان تو را که پسران و دخترانت آن را می باید بخورند خواهند خورد و گوسفندان وگاوان تو را خواهند خورد و انگورها و انجیرهای تو را خواهند خورد و شهرهای حصاردار تو را که به آنها توکل می‌نمایی با شمشیر هلاک خواهندساخت.»
\par 18 لیکن خداوند می‌گوید: «در آن روزها نیز شما را بالکل هلاک نخواهم ساخت.
\par 19 و چون شما گویید که یهوه خدای ما چراتمامی این بلاها را بر ما وارد آورده است آنگاه توبه ایشان بگو از این جهت که مرا ترک کردید وخدایان غیر را در زمین خویش عبادت نمودید. پس غریبان را در زمینی که از آن شما نباشدبندگی خواهید نمود.
\par 20 «این را به خاندان یعقوب اخبار نمایید و به یهودا اعلان کرده، گویید
\par 21 که‌ای قوم جاهل وبی فهم که چشم دارید اما نمی بینید و گوش داریداما نمی شنوید این را بشنوید.
\par 22 خداوندمی گوید آیا از من نمی ترسید و آیا از حضور من نمی لرزید که ریگ را به قانون جاودانی، حد دریاگذاشته‌ام که از آن نتواند گذشت و اگر‌چه امواجش متلاطم شود غالب نخواهد آمد و هرچند شورش نماید اما از آن تجاوز نمی تواندکرد؟
\par 23 اما این قوم، دل فتنه انگیز و متمرد دارند. ایشان فتنه انگیخته و رفته‌اند.
\par 24 و در دلهای خودنمی گویند که از یهوه خدای خود بترسیم که باران اول و آخر را در موسمش می‌بخشد و هفته های معین حصاد را به جهت ما نگاه می‌دارد.
\par 25 خطایای شما این چیزها را دور کرده و گناهان شما نیکویی را از شما منع نموده است.
\par 26 زیرادر میان قوم من شریران پیدا شده‌اند که مثل کمین نشستن صیادان در کمین می‌نشینند. دامها گسترانیده، مردم را صید می‌کنند.
\par 27 مثل قفسی که پر از پرندگان باشد، همچنین خانه های ایشان پر از فریب است و از این جهت بزرگ و دولتمندشده‌اند.
\par 28 فربه و درخشنده می‌شوند و در اعمال زشت هم از حد تجاوز می‌کنند. دعوی یعنی دعوی یتیمان را فیصل نمی دهند و با وجود آن کامیاب می‌شوند و فقیران را دادرسی نمی کنند.
\par 29 و خداوند می‌گوید: آیا به‌سبب این کارهاعقوبت نخواهم رسانید و آیا جان من از چنین طایفه‌ای انتقام نخواهد کشید؟
\par 30 امری عجیب و هولناک در زمین واقع شده است.انبیا به دروغ نبوت می‌کنند و کاهنان به واسطه ایشان حکمرانی می‌نمایند و قوم من این حالت را دوست می‌دارند و شما در آخر این چه خواهید کرد؟»
\par 31 انبیا به دروغ نبوت می‌کنند و کاهنان به واسطه ایشان حکمرانی می‌نمایند و قوم من این حالت را دوست می‌دارند و شما در آخر این چه خواهید کرد؟»
 
\chapter{6}

\par 1 ای بنی بنیامین از اورشلیم فرار کنید و کرنارا در تقوع بنوازید و علامتی بر بیت هکاریم برافرازید زیرا که بلایی از طرف شمال وشکست عظیمی رو خواهد داد.
\par 2 و من آن دخترجمیل و لطیف یعنی دختر صهیون را منقطع خواهم ساخت.
\par 3 و شبانان با گله های خویش نزدوی خواهند آمد و خیمه های خود را گرداگرد اوبرپا نموده، هر یک در جای خود خواهند چرانید.
\par 4 با او جنگ را مهیا سازید و برخاسته، در وقت ظهر برآییم. وای بر ما زیرا که روز رو به زوال نهاده است و سایه های عصر دراز می‌شود.
\par 5 برخیزید! و در شب برآییم تا قصرهایش رامنهدم سازیم.
\par 6 زیرا که یهوه صبایوت چنین می فرماید: «درختان را قطع نموده، مقابل اورشلیم سنگر برپا نمایید. زیرا این است شهری که سزاوار عقوبت است چونکه اندرونش تمام ظلم است. 
\par 7 مثل چشمه‌ای که آب خود رامی جوشاند همچنان او شرارت خویش رامی جوشاند. ظلم و تاراج در اندرونش شنیده می‌شود و بیماریها و جراحات دایم در نظر من است.
\par 8 ‌ای اورشلیم، تادیب را بپذیر مبادا جان من از تو بیزار شود و تو را ویران و زمین غیرمسکون گردانم.»
\par 9 یهوه صبایوت چنین می‌گوید که «بقیه اسرائیل را مثل مو خوشه چینی خواهند کرد پس مثل کسی‌که انگور می‌چینددست خود را بر شاخه هایش برگردان.»
\par 10 کیستند که به ایشان تکلم نموده، شهادت دهم تا بشنوند. هان گوش ایشان نامختون است که نتوانند شنید. اینک کلام خداوند برای ایشان عارگردیده است و در آن رغبت ندارند.
\par 11 و من ازحدت خشم خداوند پر شده‌ام و از خودداری خسته گردیده‌ام پس آن را در کوچه‌ها بر اطفال وبر مجلس جوانان با هم بریز. زیرا که شوهر و زن هر دو گرفتار خواهند شد و شیخ با دیرینه روز.
\par 12 و خانه‌ها و مزرعه‌ها و زنان ایشان با هم از آن دیگران خواهند شد زیرا خداوند می‌گوید که «دست خود را به ضد ساکنان این زمین درازخواهم کرد.
\par 13 چونکه جمیع ایشان چه خرد وچه بزرگ، پر از طمع شده‌اند و همگی ایشان چه نبی و چه کاهن، فریب را بعمل می‌آورند.
\par 14 وجراحت قوم مرا اندک شفایی دادند، چونکه می‌گویند سلامتی است، سلامتی است با آنکه سلامتی نیست.»
\par 15 آیا چون مرتکب رجاسات شدند خجل گردیدند؟ نی ابد خجل نشدند بلکه حیا را احساس ننمودند. بنابراین خداوندمی گوید که «در میان افتادگان خواهند افتاد وحینی که من به ایشان عقوبت رسانم خواهندلغزید.»
\par 16 خداوند چنین می‌گوید: «بر طریق هابایستید و ملاحظه نمایید و درباره طریق های قدیم سوآل نمایید که طریق نیکو کدام است تا درآن سلوک نموده، برای جان خود راحت بیابید، لیکن ایشان جواب دادند که در آن سلوک نخواهیم کرد.
\par 17 و من پاسبانان بر شما گماشتم (که می‌گفتند): به آواز کرنا گوش دهید، اما ایشان گفتند گوش نخواهیم داد.
\par 18 پس‌ای امت هابشنوید و‌ای جماعت آنچه را که در میان ایشان است بدانید!
\par 19 ‌ای زمین بشنو اینک من بلایی براین قوم می‌آورم که ثمره خیالات ایشان خواهدبود زیرا که به کلام من گوش ندادند و شریعت مرانیز ترک نمودند.
\par 20 چه فایده دارد که بخور از سباو قصب الذریره از زمین بعید برای من آورده می‌شود. قربانی های سوختنی شما مقبول نیست و ذبایح شما پسندیده من نی.»
\par 21 بنابراین خداوند چنین می‌گوید: «اینک من پیش روی این قوم سنگهای لغزش دهنده خواهم نهاد و پدران وپسران با هم از آنها لغزش خواهند خورد و ساکن زمین با همسایه‌اش هلاک خواهند شد.»
\par 22 خداوند چنین می‌گوید: «اینک قومی اززمین شمال می‌آورم و امتی عظیم از اقصای زمین خواهند برخاست.
\par 23 و کمان و نیزه خواهندگرفت. ایشان مردان ستمکیش می‌باشند که ترحم ندارند. به آواز خود مثل دریا شورش خواهندنمود و بر اسبان سوار شده، مثل مردان جنگی به ضد تو‌ای دختر صهیون صف آرایی خواهندکرد.»
\par 24 آوازه این را شنیدیم و دستهای ما سست گردید. تنگی و درد مثل زنی که می‌زاید ما را درگرفته است.
\par 25 به صحرا بیرون مشوید و به راه مروید زیرا که شمشیر دشمنان و خوف از هرطرف است.
\par 26 ‌ای دختر قوم من پلاس بپوش وخویشتن را در خاکستر بغلطان. ماتم پسر یگانه ونوحه گری تلخ برای خود بکن زیرا که تاراج کننده ناگهان بر ما می‌آید.
\par 27 تو را در میان قوم خودامتحان کننده و قلعه قرار دادم تا راههای ایشان رابفهمی و امتحان کنی.
\par 28 همه ایشان سخت متمرد شده‌اند و برای نمامی کردن گردش می‌کنند. برنج و آهن می‌باشند و جمیع ایشان فساد کننده‌اند.
\par 29 دم پر زور می‌دمد و سرب درآتش فانی می‌گردد و قالگر عبث قال می‌گذاردزیرا که شریران جدا نمی شوند.نقره ترک شده نامیده می‌شوند زیرا خداوند ایشان را ترک کرده است.
\par 30 نقره ترک شده نامیده می‌شوند زیرا خداوند ایشان را ترک کرده است.
 
\chapter{7}

\par 1 کلامی که از جانب خداوند به ارمیا نازل شده، گفت:
\par 2 «به دروازه خانه خداوندبایست و این کلام را در آنجا ندا کرده، بگو: ای تمامی یهودا که به این دروازه‌ها داخل شده، خداوند را سجده می‌نمایید کلام خداوند رابشنوید.
\par 3 یهوه صبایوت خدای اسرائیل چنین می‌گوید: طریق‌ها و اعمال خود را اصلاح کنید ومن شما را در این مکان ساکن خواهم گردانید.
\par 4 به سخنان دروغ توکل منمایید و مگویید که هیکل یهوه، هیکل یهوه، هیکل یهوه این است.
\par 5 زیرا اگر به تحقیق طریق‌ها و اعمال خود را اصلاح کنید و انصاف را در میان یکدیگر بعمل آورید،
\par 6 و بر غریبان و یتیمان و بیوه‌زنان ظلم ننمایید وخون بی‌گناهان را در این مکان نریزید و خدایان غیر را به جهت ضرر خویش پیروی ننمایید،
\par 7 آنگاه شما را در این مکان در زمینی که به پدران شما از ازل تا به ابد داده‌ام ساکن خواهم گردانید.
\par 8 اینک شما به سخنان دروغی که منفعت نداردتوکل می‌نمایید.
\par 9 آیا مرتکب دزدی و زنا و قتل نمی شوید و به دروغ قسم نمی خورید و برای بعل بخور نمی سوزانید؟ و آیا خدایان غیر را که نمی شناسید پیروی نمی نمایید؟
\par 10 و داخل شده، به حضور من در این خانه‌ای که به اسم من مسمی است می‌ایستید و می‌گویید که به گردن تمام این رجاسات سپرده شده‌ایم.
\par 11 آیا این خانه‌ای که به اسم من مسمی است در نظر شمامغاره دزدان شده است؟ و خداوند می‌گوید: اینک من نیز این را دیده‌ام.
\par 12 لکن به مکان من که در شیلو بود و نام خود را اول در آنجا قرار داده بودم بروید و آنچه را که به‌سبب شرارت قوم خود اسرائیل به آنجا کرده‌ام ملاحظه نمایید.
\par 13 پس حال خداوند می‌گوید: از آنرو که تمام این اعمال را بجا آوردید با آنکه من صبح زودبرخاسته، به شما تکلم نموده، سخن راندم امانشنیدید و شما را خواندم اما جواب ندادید.
\par 14 ازاین جهت به این خانه‌ای که به اسم من مسمی است و شما به آن توکل دارید و به مکانی که به شما و به پدران شما دادم به نوعی که به شیلو عمل نمودم عمل خواهم کرد.
\par 15 و شما را از حضورخود خواهم راند به نوعی که جمیع برادران شما یعنی تمام ذریت افرایم را راندم.
\par 16 پس تو برای این قوم دعا مکن و به جهت ایشان آواز تضرع واستغاثه بلند منما و نزد من شفاعت مکن زیرا که من تو را اجابت نخواهم نمود.
\par 17 آیا آنچه را که ایشان در شهرهای یهودا و کوچه های اورشلیم می‌کنند نمی بینی؟
\par 18 پسران، هیزم جمع می‌کنندو پدران، آتش می‌افروزند و زنان، خمیرمی سرشند تا قرصها برای ملکه آسمان بسازند وهدایای ریختنی برای خدایان غیر ریخته مرامتغیر سازند.
\par 19 اما خداوند می‌گوید آیا مرامتغیر می‌سازند؟ نی بلکه خویشتن را تا رویهای خود را رسوا سازند.
\par 20 بنابراین خداوند یهوه چنین می‌گوید: اینک خشم و غضب من بر این مکان برانسان و بر بهایم و بر درختان صحرا و برمحصول زمین ریخته خواهد شد و افروخته شده، خاموش نخواهد گردید.
\par 21 «یهوه صبایوت خدای اسرائیل چنین می‌گوید: قربانی های سوختنی خود را بر ذبایح خویش مزید کنید و گوشت بخورید.
\par 22 زیرا که به پدران شما سخن نگفتم و در روزی که ایشان رااز زمین مصر بیرون آوردم آنها را درباره قربانی های سوختنی و ذبایح امر نفرمودم.
\par 23 بلکه ایشان را به این چیز امر فرموده، گفتم که قول مرابشنوید ومن خدای شما خواهم بود و شما قوم من خواهید بود و بهر طریقی که به شما حکم نمایم سلوک نمایید تا برای شما نیکو باشد.
\par 24 اماایشان نشنیدند و گوش خود را فرا نداشتند بلکه برحسب مشورتها و سرکشی دل شریر خود رفتارنمودند و به عقب افتادند و پیش نیامدند.
\par 25 ازروزی که پدران شما از زمین مصر بیرون آمدند تا امروز جمیع بندگان خود انبیا را نزد شما فرستادم بلکه هر روز صبح زود برخاسته، ایشان را ارسال نمودم.
\par 26 اما ایشان نشنیدند و گوش خود را فرانداشتند بلکه گردن خویش را سخت نموده، ازپدران خود بدتر عمل نمودند.
\par 27 پس تو تمامی این سخنان را به ایشان بگو اما تو را نخواهند شنیدو ایشان را بخوان اما ایشان تو را جواب نخواهندداد.
\par 28 و به ایشان بگو: اینان قومی می‌باشند که قول یهوه خدای خویش را نمی شنوند و تادیب نمی پذیرند زیرا راستی نابود گردیده و از دهان ایشان قطع شده است.
\par 29 (ای اورشلیم ) موی خود را تراشیده، دور بینداز و بر بلندیها آوازنوحه برافراز زیرا خداوند طبقه مغضوب خود رارد و ترک نموده است.
\par 30 «چونکه خداوند می‌گوید بنی یهودا آنچه را که در نظر من ناپسند است بعمل آوردند ورجاسات خویش را در خانه‌ای که به اسم من مسمی است برپانموده، آن را نجس ساختند.
\par 31 ومکان های بلند خود را در توفت که در وادی ابن حنوم است بنا نمودند تا پسران و دختران خویش را در آتش بسوزانند که من اینکار را امر نفرموده بودم و بخاطر خویش نیاورده.
\par 32 بنابراین خداوند می‌گوید: اینک روزها می‌آید که آن باردیگر به توفت و وادی ابن حنوم مسمی نخواهدشد بلکه به وادی قتل و در توفت دفن خواهند کردتا جایی باقی نماند.
\par 33 و لاشهای این قوم خوراک مرغان هوا و جانوران زمین خواهد بود وکسی آنهارا نخواهد ترسانید.و از شهرهای یهودا و کوچه های اورشلیم آواز شادمانی و آواز خوشی و صدای داماد و صدای عروس را نابود خواهم ساخت زیرا که آن زمین ویران خواهد شد.»
\par 34 و از شهرهای یهودا و کوچه های اورشلیم آواز شادمانی و آواز خوشی و صدای داماد و صدای عروس را نابود خواهم ساخت زیرا که آن زمین ویران خواهد شد.»
 
\chapter{8}

\par 1 استخوانهای پادشاهان یهودا واستخوانهای سرورانش و استخوانهای کهنه واستخوانهای انبیا و استخوانهای سکنه اورشلیم را از قبرهای ایشان بیرون خواهند‌آورد.
\par 2 و آنهارا پیش آفتاب و ماه و تمامی لشکر آسمان که آنهارا دوست داشته و عبادت کرده و پیروی نموده وجستجو و سجده کرده‌اند پهن خواهند کرد و آنهارا جمع نخواهند نمود و دفن نخواهند کرد بلکه برروی زمین سرگین خواهد بود.
\par 3 و یهوه صبایوت می‌گوید که تمامی بقیه این قبیله شریر که باقی می‌مانند در هر مکانی که باقی‌مانده باشند و من ایشان را بسوی آن رانده باشم مرگ را بر حیات ترجیح خواهند داد.
\par 4 «و ایشان را بگو خداوند چنین می‌فرماید: اگر کسی بیفتد آیا نخواهد برخاست و اگر کسی مرتد شود آیا بازگشت نخواهد نمود؟
\par 5 پس چرااین قوم اورشلیم به ارتداد دایمی مرتد شده‌اند وبه فریب متمسک شده، از بازگشت نمودن ابامی نمایند؟
\par 6 من گوش خود را فرا داشته، شنیدم اما براستی تکلم ننمودند و کسی از شرارت خویش توبه نکرده و نگفته است چه کرده‌ام بلکه هر یک مثل اسبی که به جنگ می‌دود به راه خودرجوع می‌کند.
\par 7 لقلق نیز در هوا موسم خود رامی داند و فاخته و پرستوک و کلنک زمان آمدن خود را نگاه می‌دارند لیکن قوم من احکام خداوندرا نمی دانند.
\par 8 چگونه می‌گویید که ما حکیم هستیم و شریعت خداوند با ما است. به تحقیق قلم کاذب کاتبان به دروغ عمل می‌نماید.
\par 9 حکیمان شرمنده و مدهوش و گرفتار شده‌اند. اینک کلام خداوند را ترک نموده‌اند پس چه نوع حکمتی دارند.
\par 10 بنابراین زنان ایشان را به دیگران خواهم داد و مزرعه های ایشان را به مالکان دیگر. زیرا که جمیع ایشان چه خرد و چه بزرگ پر از طمع می‌باشند و همگی ایشان چه نبی و چه کاهن به فریب عمل می‌نمایند.
\par 11 وجراحات قوم مرا اندک شفایی داده‌اند چونکه می‌گویند سلامتی است، سلامتی است، با آنکه سلامتی نیست.
\par 12 آیا چون مرتکب رجاسات شدند خجل گردیدند؟ نی ابد خجل نشدند بلکه حیا را احساس ننمودند بنابراین خداوندمی گوید: در میان افتادگان خواهند افتاد و حینی که من به ایشان عقوبت رسانم خواهند لغزید.»
\par 13 خداوند می‌گوید: «ایشان را بالکل تلف خواهم نمود که نه انگور بر مو و نه انجیر بردرخت انجیر یافت شود و برگها پژمرده خواهدشد و آنچه به ایشان بدهم از ایشان زایل خواهدشد.»
\par 14 پس ما چرا می‌نشینیم؟ جمع بشوید تا به شهرهای حصاردار داخل شویم و در آنها ساکت باشیم. زیرا که یهوه خدای ما ما را ساکت گردانیده و آب تلخ به ما نوشانیده است زانرو که به خداوندگناه ورزیده‌ایم. 
\par 15 برای سلامتی انتظار کشیدیم اما هیچ خیر حاصل نشد و برای زمان شفا و اینک آشفتگی پدید آمد.
\par 16 صهیل اسبان او از دان شنیده شد و از صدای شیهه زورآورانش تمامی زمین متزلزل گردید زیرا که آمده‌اند و زمین و هرچه در آن است و شهر و ساکنانش را خورده‌اند.
\par 17 زیرا خداوند می‌گوید: «اینک من در میان شمامارها و افعیها خواهم فرستاد که آنها را افسون نتوان کرد و شما را خواهند گزید.»
\par 18 کاش که ازغم خود تسلی می‌یافتم. دل من در اندرونم ضعف بهم رسانیده است.
\par 19 اینک آواز تضرع دختر قوم من از زمین دور می‌آید که آیا خداوند در صهیون نیست و مگر پادشاهش در آن نیست پس چراخشم مرا به بتهای خود و اباطیل بیگانه به هیجان آوردند؟
\par 20 موسم حصاد گذشت و تابستان تمام شد و ما نجات نیافتیم.
\par 21 به‌سبب جراحت دخترقوم خود مجروح شده و ماتم گرفته‌ام و حیرت مرا فرو گرفته است.آیا بلسان در جلعاد نیست و طبیبی در آن نی؟ پس دختر قوم من چرا شفانیافته است؟
\par 22 آیا بلسان در جلعاد نیست و طبیبی در آن نی؟ پس دختر قوم من چرا شفانیافته است؟
 
\chapter{9}

\par 1 چشمه اشک. تا روز و شب برای کشتگان دختر قوم خود گریه می‌کردم.
\par 2 کاش که در بیابان منزل مسافران می‌داشتم تا قوم خود را ترک کرده، از نزد ایشان می‌رفتم چونکه همگی ایشان زناکارو جماعت خیانت کارند.
\par 3 زبان خویش را مثل کمان خود به دروغ می‌کشند. در زمین قوی شده‌اند اما نه برای راستی زیرا خداوند می‌گوید: «از شرارت به شرارت ترقی می‌کنند و مرانمی شناسند.»
\par 4 هر یک از همسایه خویش باحذر باشید و به هیچ برادر اعتماد منمایید زیراهر برادر از پا درمی آورد و هر همسایه به نمامی گردش می‌کند.
\par 5 و هر کس همسایه خود را فریب می‌دهد و ایشان براستی تکلم نمی نمایند و زبان خود را به دروغگویی آموخته‌اند و از کج رفتاری خسته شده‌اند.
\par 6 خداوند می‌گوید که «مسکن تودر میان فریب است و از مکر خویش نمی خواهندکه مرا بشناسند.»
\par 7 بنابراین یهوه صبایوت چنین می‌گوید: «اینک من ایشان را قال گذاشته، امتحان خواهم نمود. زیرا به‌خاطر دختر قوم خود چه توانم کرد؟
\par 8 زبان ایشان تیر مهلک است که به فریب سخن می‌راند. به زبان خود با همسایه خویش سخنان صلح‌آمیز می‌گویند، اما در دل خودبرای او کمین می‌گذارند.»
\par 9 پس خداوندمی گوید: «آیا به‌سبب این چیزها ایشان راعقوبت نرسانم و آیا جانم از چنین قومی انتقام نکشد؟»
\par 10 برای کوهها گریه و نوحه گری و برای مرتعهای بیابان ماتم برپا می‌کنم زیرا که سوخته شده است و احدی از آنها گذر نمی کند و صدای مواشی شنیده نمی شود. هم مرغان هوا و هم بهایم فرار کرده و رفته‌اند.
\par 11 و اورشلیم را به توده‌ها و ماوای شغالها مبدل می‌کنم و شهرهای یهودا را ویران و غیرمسکون خواهم ساخت.
\par 12 کیست مرد حکیم که این را بفهمد و کیست که دهان خداوند به وی سخن گفته باشد تا از این چیزها اخبار نماید که چرا زمین خراب و مثل بیابان سوخته شده است که احدی از آن گذرنمی کند.
\par 13 پس خداوند می‌گوید: «چونکه شریعت مرا که پیش روی ایشان گذاشته بودم ترک کردند وآواز مرا نشنیدند و در آن سلوک ننمودند،
\par 14 بلکه پیروی سرکشی دل خود را نمودند، و از عقب بعلیم که پدران ایشان آنها را به ایشان آموختندرفتند.»
\par 15 از این جهت یهوه صبایوت خدای اسرائیل چنین می‌گوید: «اینک من افسنتین راخوراک این قوم خواهم ساخت و آب تلخ به ایشان خواهم نوشانید.
\par 16 و ایشان را در میان امت هایی که ایشان و پدران ایشان آنها رانشناختند پراکنده خواهم ساخت و شمشیر را درعقب ایشان خواهم فرستاد تا ایشان را هلاک نمایم.»
\par 17 یهوه صبایوت چنین می‌گوید: «تفکر کنیدو زنان نوحه گر را بخوانید تا بیایند و در‌پی زنان حکیم بفرستید تا بیایند.»
\par 18 و ایشان تعجیل نموده، برای ما ماتم برپا کنند تا چشمان ما اشکهابریزد و مژگان ما آبها جاری سازد.
\par 19 زیرا که آواز نوحه گری از صهیون شنیده می‌شود که چگونه غارت شدیم و چه بسیار خجل گردیدیم چونکه زمین را ترک کردیم و مسکن های ما ما رابیرون انداخته‌اند.
\par 20 پس‌ای زنان، کلام خداوند را بشنوید و گوشهای شما کلام دهان اورا بپذیرد و شما به دختران خود نوحه گری راتعلیم دهید و هر زن به همسایه خویش ماتم را.
\par 21 زیرا موت به پنجره های ما برآمده، به قصرهای ما داخل شده است تا اطفال را از بیرون و جوانان را از چهارسوها منقطع سازد.
\par 22 خداوند چنین می‌گوید: «بگو که لاشهای مردمان مثل سرگین بر روی صحرا و مانند بافه درعقب دروگر افتاده است و کسی نیست که آن رابرچیند.»
\par 23 خداوند چنین می‌گوید: «حکیم، ازحکمت خود فخر ننماید و جبار، از تنومندی خویش مفتخر نشود و دولتمند از دولت خودافتخار نکند.
\par 24 بلکه هر‌که فخر نماید از این فخربکند که فهم دارد و مرا می‌شناسد که من یهوه هستم که رحمت و انصاف و عدالت را در زمین بجا می‌آورم زیرا خداوند می‌گوید در این چیزهامسرور می‌باشم.»
\par 25 خداوند می‌گوید: «اینک ایامی می‌آید که نامختونان را با مختونان عقوبت خواهم رسانید.یعنی مصر و یهودا و ادوم و بنی عمون وموآب و آنانی را که گوشه های موی خود رامی تراشند و در صحرا ساکنند. زیرا که جمیع این امت‌ها نامختونند و تمامی خاندان اسرائیل در دل نامختونند.»
\par 26 یعنی مصر و یهودا و ادوم و بنی عمون وموآب و آنانی را که گوشه های موی خود رامی تراشند و در صحرا ساکنند. زیرا که جمیع این امت‌ها نامختونند و تمامی خاندان اسرائیل در دل نامختونند.»
 
\chapter{10}

\par 1 ای خاندان اسرائیل کلامی را که خداوند به شما می‌گوید بشنوید!
\par 2 خداوند چنین می‌گوید: «طریق امت‌ها را یادمگیرید و از علامات افلاک مترسید زیرا که امت‌ها از آنها می‌ترسند.
\par 3 چونکه رسوم قومهاباطل است که ایشان درختی از جنگل با تبرمی برند که صنعت دستهای نجار می‌باشد.
\par 4 و آن را به نقره و طلا زینت داده، با میخ و چکش محکم می‌کنند تا متحرک نشود.
\par 5 و آنها مثل مترس دربوستان خیار می‌باشند که سخن نمی توانند گفت وآنها را می‌باید برداشت چونکه راه نمی توانندرفت. از آنها مترسید زیرا که ضرر نتوانند رسانیدو قوت نفع رسانیدن هم ندارند.»
\par 6 ‌ای یهوه مثل تو کسی نیست! تو عظیم هستی و اسم تو در قوت عظیم است!
\par 7 ‌ای پادشاه امت هاکیست که از تو نترسد زیرا که این به تو می‌شایدچونکه در جمیع حکیمان امت‌ها و در تمامی ممالک ایشان مانند تو کسی نیست.
\par 8 جمیع ایشان وحشی و احمق می‌باشند تادیب اباطیل چوب (بت ) است.
\par 9 نقره کوبیده شده از ترشیش و طلا از اوفاز که صنعت صنعتگر و عمل دستهای زرگر باشد می‌آورند. لاجورد و ارغوان لباس آنها و همه اینها عمل حکمت پیشگان است.
\par 10 اما یهوه خدای حق است و او خدای حی و پادشاه سرمدی می‌باشد. از غضب او زمین متزلزل می‌شود و امت‌ها قهر او را متحمل نتوانندشد.
\par 11 به ایشان چنین بگویید: «خدایانی که آسمان و زمین را نساخته‌اند از روی زمین و از زیرآسمان تلف خواهند شد.»
\par 12 او زمین را به قوت خود ساخت و ربع مسکون را به حکمت خویش استوار نمود وآسمان را به عقل خود گسترانید.
\par 13 چون آوازمی دهد غوغای آبها در آسمان پدید می‌آید. ابرها از اقصای زمین برمی آورد و برقها برای باران می‌سازد و باد را از خزانه های خود بیرون می‌آورد.
\par 14 جمیع مردمان وحشی‌اند و معرفت ندارند و هر‌که تمثالی می‌سازد خجل خواهدشد. زیرا که بت ریخته شده او دروغ است و در آن هیچ نفس نیست.
\par 15 آنها باطل و کار مسخرگی می‌باشد در روزی که به محاکمه می‌آیند تلف خواهند شد.
\par 16 او که نصیب یعقوب است مثل آنها نمی باشد. زیرا که او سازنده همه موجودات است و اسرائیل عصای میراث وی است و اسم اویهوه صبایوت می‌باشد.
\par 17 ‌ای که در تنگی ساکن هستی، بسته خود رااز زمین بردار!
\par 18 زیرا خداوند چنین می‌گوید: «اینک من این مرتبه ساکنان این زمین را از فلاخن خواهم‌انداخت و ایشان را به تنگ خواهم آورد تابفهمند.»
\par 19 وای بر من به‌سبب صدمه من.
\par 20 خیمه من خراب شد و تمامی طنابهای من گسیخته گردید، پسرانم از من بیرون رفته، نایاب شدند. کسی نیست که خیمه مرا پهن کند و پرده های مرابرپا نماید.
\par 21 زیرا که شبانان وحشی شده‌اند وخداوند را طلب نمی نمایند بنابراین کامیاب نخواهند شد و همه گله های ایشان پراکنده خواهد گردید.
\par 22 اینک صدای خبری می‌آید واضطراب عظیمی از دیار شمال. تا شهرهای یهودا را ویران و ماوای شغالها سازد.
\par 23 ‌ای خداوند می‌دانم که طریق انسان از آن اونیست و آدمی که راه می‌رود قادر بر هدایت قدمهای خویش نمی باشد.
\par 24 ‌ای خداوند مراتادیب نما اما به انصاف و نه به غضب خود مبادامرا ذلیل سازی.غضب خویش را بر امت هایی که تو را نمی شناسند بریز. و بر قبیله هایی که اسم تو را نمی خوانند، زیرا که ایشان یعقوب راخوردند و او را بلعیده، تباه ساختند و مسکن او راخراب نمودند.
\par 25 غضب خویش را بر امت هایی که تو را نمی شناسند بریز. و بر قبیله هایی که اسم تو را نمی خوانند، زیرا که ایشان یعقوب راخوردند و او را بلعیده، تباه ساختند و مسکن او راخراب نمودند.
 
\chapter{11}

\par 1 این است کلامی که از جانب خداوند به ارمیا نازل شده، گفت:
\par 2 «کلام این عهدرا بشنوید و به مردان یهودا و ساکنان اورشلیم بگویید.
\par 3 و تو به ایشان بگو یهوه خدای اسرائیل چنین می‌گوید: ملعون باد کسی‌که کلام این عهدرا نشنود.
\par 4 که آن را به پدران شما در روزی که ایشان را از زمین مصر از کوره آهنین بیرون آوردم امر فرموده، گفتم قول مرا بشنوید و موافق هر آنچه به شما امر بفرمایم آن را بجا بیاورید تا شماقوم من باشید و من خدای شما باشم.
\par 5 و تا قسمی را که برای پدران شما خوردم وفا نمایم که زمینی را که به شیر و عسل جاری است چنانکه امروزشده است به ایشان بدهم.» پس من در جواب گفتم: «ای خداوند آمین.»
\par 6 پس خداوند مرا گفت: «تمام این سخنان رادر شهرهای یهودا و کوچه های اورشلیم نداکرده، بگو که سخنان این عهد را بشنوید و آنها رابجا آورید.
\par 7 زیرا از روزی که پدران شما را اززمین مصر برآوردم تا امروز ایشان را تاکید سخت نمودم و صبح زود برخاسته، تاکید نموده، گفتم قول مرا بشنوید.
\par 8 اما نشنیدند و گوش خود را فرانداشتند بلکه پیروی سرکشی دل شریر خود رانمودند. پس تمام سخنان این عهد را بر ایشان وارد آوردم چونکه امر فرموده بودم که آن را وفانمایند اما وفا ننمودند.»
\par 9 و خداوند مرا گفت: «فتنه‌ای در میان مردان یهودا و ساکنان اورشلیم پیدا شده است.
\par 10 به خطایای پدران پیشین خود که از شنیدن این سخنان ابا نمودند برگشتند و ایشان خدایان غیر راپیروی نموده، آنها را عبادت نمودند. و خاندان اسرائیل و خاندان یهودا عهدی را که با پدران ایشان بسته بودم شکستند.»
\par 11 بنابراین خداوندچنین می‌گوید: «اینک من بلایی را که از آن نتوانندرست بر ایشان خواهم آورد. و نزد من استغاثه خواهند کرد اما ایشان را اجابت نخواهم نمود.
\par 12 و شهرهای یهودا و ساکنان اورشلیم رفته، نزدخدایانی که برای آنها بخور می‌سوزانیدند فریادخواهند کرد اما آنها در وقت مصیبت ایشان هرگزایشان را نجات نخواهند داد.
\par 13 زیرا که‌ای یهوداشماره خدایان تو بقدر شهرهای تو می‌باشد و برحسب شماره کوچه های اورشلیم مذبح های رسوایی برپا داشتید یعنی مذبح‌ها به جهت بخورسوزانیدن برای بعل.
\par 14 پس تو برای این قوم دعامکن و به جهت ایشان آواز تضرع و استغاثه بلندمنما زیرا که چون در وقت مصیبت خویش مرابخوانند ایشان را اجابت نخواهم نمود.
\par 15 محبوبه مرا در خانه من چه‌کار است چونکه شرارت ورزیده است. آیا تضرعات و گوشت مقدس می‌تواند گناه تو را از تو دور بکند؟ آنگاه می‌توانستی وجد نمایی.»
\par 16 خداوند تو را زیتون شاداب که به میوه نیکوخوشنما باشد مسمی نموده. اما به آواز غوغای عظیم آتش در آن افروخته است که شاخه هایش شکسته گردید. 
\par 17 زیرا یهوه صبایوت که تو راغرس نموده بود بلایی بر تو فرموده است به‌سبب شرارتی که خاندان اسرائیل و خاندان یهودا به ضد خویشتن کردند و برای بعل بخور‌سوزانیده، خشم مرا به هیجان آوردند.
\par 18 و خداوند مرا تعلیم داد پس دانستم. آنگاه اعمال ایشان را به من نشان دادی.
\par 19 و من مثل بره دست آموز که به مذبح برند بودم. و نمی دانستم که تدبیرات به ضد من نموده، می‌گفتند: «درخت را بامیوه‌اش ضایع سازیم و آن را از زمین زندگان قطع نماییم تا اسمش دیگر مذکور نشود.»
\par 20 اما‌ای یهوه صبایوت که داور عادل و امتحان کننده باطن و دل هستی، بشود که انتقام کشیدن تو را از ایشان ببینم زیرا که دعوی خود را نزد تو ظاهر ساختم.
\par 21 لهذا خداوند چنین می‌گوید: «درباره اهل عناتوت که قصد جان تو دارند و می‌گویند به نام یهوه نبوت مکن مبادا از دست ما کشته شوی.
\par 22 از این جهت یهوه صبایوت چنین می‌گوید: اینک بر ایشان عقوبت خواهم رسانید. و جوانان ایشان به شمشیر خواهند مرد و پسران و دختران ایشان از گرسنگی هلاک خواهند شد.و برای ایشان بقیه‌ای نخواهد ماند زیرا که من بر اهل عناتوت در سال عقوبت ایشان بلایی خواهم رسانید.»
\par 23 و برای ایشان بقیه‌ای نخواهد ماند زیرا که من بر اهل عناتوت در سال عقوبت ایشان بلایی خواهم رسانید.»
 
\chapter{12}

\par 1 ای خداوند تو عادل تر هستی از اینکه من با تو محاجه نمایم. لیکن درباره احکامت با تو سخن خواهم راند. چرا راه شریران برخوردار می‌شود و جمیع خیانتکاران ایمن می‌باشند؟
\par 2 تو ایشان را غرس نمودی پس ریشه زدند و نمو کرده، میوه نیز آوردند. تو به دهان ایشان نزدیکی، اما از قلب ایشان دور.
\par 3 اما تو‌ای خداوند مرا می‌شناسی و مرا دیده، دل مرا نزدخود امتحان کرده‌ای. ایشان را مثل گوسفندان برای ذبح بیرون کش و ایشان را به جهت روز قتل تعیین نما.
\par 4 زمین تا به کی ماتم خواهد نمود و گیاه تمامی صحرا خشک خواهد ماند. حیوانات ومرغان به‌سبب شرارت ساکنانش تلف شده اندزیرا می‌گویند که او آخرت ما را نخواهد دید.
\par 5 اگر وقتی که با پیادگان دویدی تو را خسته کردند پس چگونه با اسبان می‌توانی برابری کنی؟ و هر‌چند در زمین سالم، ایمن هستی در طغیان اردن چه خواهی کرد؟
\par 6 زیرا که هم برادرانت وهم خاندان پدرت به تو خیانت نمودند و ایشان نیزدر عقب تو صدای بلند می‌کنند پس اگر‌چه سخنان نیکو به تو بگویند ایشان را باور مکن.
\par 7 من خانه خود را ترک کرده، میراث خویش رادور انداختم. و محبوبه خود را به‌دست دشمنانش تسلیم نمودم.
\par 8 و میراث من مثل شیرجنگل برای من گردید. و به ضد من آواز خود رابلند کرد از این جهت از او نفرت کردم.
\par 9 آیامیراث من برایم مثل مرغ شکاری رنگارنگ که مرغان دور او را گرفته باشند شده است؟ بروید وجمیع حیوانات صحرا را جمع کرده، آنها رابیاورید تا بخورند.
\par 10 شبانان بسیار تاکستان مراخراب کرده، میراث مرا پایمال نمودند. و میراث مرغوب مرا به بیابان ویران مبدل ساختند.
\par 11 آن را ویران ساختند و آن ویران شده نزد من ماتم گرفته است. تمامی زمین ویران شده، چونکه کسی این را در دل خود راه نمی دهد.
\par 12 بر تمامی بلندیهای صحرا، تاراج کنندگان هجوم آوردندزیرا که شمشیر خداوند از کنار زمین تا کناردیگرش هلاک می‌کند و برای هیچ بشری ایمنی نیست.
\par 13 گندم کاشتند و خار درویدند، خویشتن را به رنج آورده، نفع نبردند. و از محصول شما به‌سبب حدت خشم خداوند خجل گردیدند.
\par 14 خداوند درباره جمیع همسایگان شریرخود که ضرر می‌رسانند به ملکی که قوم خوداسرائیل را مالک آن ساخته است چنین می‌گوید: «اینک ایشان را از آن زمین برمی کنم و خاندان یهودا را از میان ایشان برمی کنم.
\par 15 و بعد ازبرکندن ایشان رجوع خواهم کرد و بر ایشان ترحم خواهم نمود و هر کس از ایشان را به ملک خویش و هر کس را به زمین خود باز خواهم آورد.
\par 16 و اگر ایشان طریق های قوم مرا نیکو یادگرفته، به اسم من یعنی به حیات یهوه قسم خورندچنانکه ایشان قوم مرا تعلیم دادند که به بعل قسم خورند، آنگاه ایشان در میان قوم من بنا خواهند شد.اما اگر نشنوند آنگاه آن امت را بالکل برکنده، هلاک خواهم ساخت.» کلام خداوند این است.
\par 17 اما اگر نشنوند آنگاه آن امت را بالکل برکنده، هلاک خواهم ساخت.» کلام خداوند این است.
 
\chapter{13}

\par 1 خداوند به من چنین گفت که «برو وکمربند کتانی برای خود بخر و آن را به کمر خود ببند و آن را در آب فرو مبر.»
\par 2 پس کمربند را موافق کلام خداوند خریدم و به کمرخود بستم.
\par 3 و کلام خداوند بار دیگر به من نازل شده، گفت:
\par 4 «این کمربند را که خریدی و به کمرخود بستی بگیر و به فرات رفته، آن را در شکاف صخره پنهان کن.»
\par 5 پس رفتم و آن را در فرات برحسب آنچه خداوند به من فرموده بود پنهان کردم.
\par 6 و بعد ازمرور ایام بسیار خداوند مرا گفت: «برخاسته، به فرات برو و کمربندی را که تو را امر فرمودم که درآنجا پنهان کنی از آنجا بگیر.»
\par 7 پس به فرات رفتم و کنده کمربند را از جایی که آن را پنهان کرده بودم گرفتم و اینک کمربند پوسیده و لایق هیچکار نبود.
\par 8 و کلام خداوند به من نازل شده، گفت:
\par 9 «خداوند چنین می‌فرماید: تکبر یهودا و تکبرعظیم اورشلیم را همچنین تباه خواهم ساخت.
\par 10 و این قوم شریری که از شنیدن قول من ابانموده، سرکشی دل خود را پیروی می‌نمایند و درعقب خدایان غیر رفته، آنها را عبادت و سجده می‌کنند، مثل این کمربندی که لایق هیچکارنیست خواهند شد.
\par 11 زیرا خداوند می‌گوید: چنانکه کمربند به کمر آدمی می‌چسبد، همچنان تمامی خاندان اسرائیل و تمامی خاندان یهودا رابه خویشتن چسبانیدم تا برای من قوم و اسم و فخر و زینت باشند اما نشنیدند.
\par 12 پس این کلام را به ایشان بگو: یهوه خدای اسرائیل چنین می‌گوید: هر مشک از شراب پر خواهد شد وایشان به تو خواهند گفت: مگر مانمی دانیم که هرمشک از شراب پر خواهد شد؟
\par 13 پس به ایشان بگو: خداوند چنین می‌گوید: جمیع ساکنان این زمین را با پادشاهانی که بر کرسی داود می‌نشینندو کاهنان و انبیا و جمیع سکنه اورشلیم را به مستی پر خواهم ساخت.
\par 14 و خداوند می‌گوید: ایشان را یعنی پدران و پسران را با یکدیگر بهم خواهم‌انداخت. از هلاک ساختن ایشان شفقت و رافت ورحمت نخواهم نمود.»
\par 15 بشنوید و گوش فرا‌گیرید و مغرور مشویدزیرا خداوند تکلم می‌نماید.
\par 16 برای یهوه خدای خود جلال را توصیف نمایید قبل از آنکه تاریکی را پدید آورد و پایهای شما بر کوههای ظلمت بلغزد. و چون منتظر نور باشید آن را به سایه موت مبدل ساخته، به ظلمت غلیظ تبدیل نماید.
\par 17 واگر این را نشنوید، جان من در خفا به‌سبب تکبرشما گریه خواهد کرد و چشم من زارزار گریسته، اشکها خواهد ریخت از این جهت که گله خداوندبه اسیری برده شده است.
\par 18 به پادشاه و ملکه بگو: «خویشتن را فروتن ساخته، بنشینید زیرا که افسرها یعنی تاجهای جلال شما افتاده است.
\par 19 شهرهای جنوب مسدود شده، کسی نیست که آنها را مفتوح سازد. و تمامی یهودا اسیرشده، بالکل به اسیری رفته است.
\par 20 چشمان خود را بلند کرده، آنانی را که از طرف شمال می‌آیند بنگرید. گله‌ای که به تو داده شد وگوسفندان زیبایی تو کجا است؟
\par 21 اما چون او یارانت را به حکمرانی تو نصب کند چه خواهی گفت؟ چونکه تو ایشان را بر ضرر خود آموخته کرده‌ای. آیا دردها مثل زنی که می‌زاید تو را فرونخواهد گرفت؟
\par 22 و اگر در دل خود گویی این چیزها چرا به من واقع شده است، (بدانکه ) به‌سبب کثرت گناهانت دامنهایت گشاده شده وپاشنه هایت به زور برهنه گردیده است.
\par 23 آیاحبشی، پوست خود را تبدیل تواند نمود یاپلنگ، پیسه های خویش را؟ آنگاه شما نیز که به بدی کردن معتاد شده‌اید نیکویی توانید کرد؟
\par 24 ومن ایشان را مثل کاه که پیش روی باد صحرا رانده شود پراکنده خواهم ساخت.»
\par 25 خداوند می‌گوید: «قرعه تو و نصیبی که ازجانب من برای تو پیموده شده این است، چونکه مرا فراموش کردی و به دروغ اعتماد نمودی.
\par 26 پس من نیز دامنهایت را پیش روی تو منکشف خواهم ساخت و رسوایی تو دیده خواهد شد.فسق و شیهه های تو و زشتی زناکاری تو ورجاسات تو را بر تلهای بیابان مشاهده نمودم. وای بر تو‌ای اورشلیم تا به کی دیگر طاهرنخواهی شد!»
\par 27 فسق و شیهه های تو و زشتی زناکاری تو ورجاسات تو را بر تلهای بیابان مشاهده نمودم. وای بر تو‌ای اورشلیم تا به کی دیگر طاهرنخواهی شد!»
 
\chapter{14}

\par 1 کلام خداوند که درباره خشک سالی به ارمیا نازل شد.
\par 2 «یهودا نوحه گری می‌کند و دروازه هایش کاهیده شده، ماتم‌کنان بر زمین می‌نشینند و فریاداورشلیم بالا می‌رود.
\par 3 و شرفای ایشان صغیران ایشان را برای آب می‌فرستند و نزد حفره هامی روند و آب نمی یابند و با ظرفهای خالی برگشته، خجل و رسوا می‌شوند و سرهای خود رامی پوشانند.
\par 4 به‌سبب اینکه زمین منشق شده است چونکه باران بر جهان نباریده است. فلاحان خجل شده، سرهای خود را می‌پوشانند.
\par 5 بلکه غزالها نیز در صحرا می‌زایند و (اولاد خود را)ترک می‌کنند چونکه هیچ گیاه نیست.
\par 6 وگورخران بر بلندیها ایستاده، مثل شغالها برای باددم می‌زنند و چشمان آنها کاهیده می‌گردد چونکه هیچ علفی نیست.»
\par 7 ‌ای خداوند اگر‌چه گناهان ما بر ما شهادت می‌دهد اما به‌خاطر اسم خود عمل نما زیرا که ارتدادهای ما بسیار شده است و به تو گناه ورزیده‌ایم.
\par 8 ‌ای تو که امید اسرائیل ونجات‌دهنده او در وقت تنگی می‌باشی چرا مثل غریبی در زمین و مانند مسافری که برای شبی خیمه می‌زند شده‌ای؟
\par 9 چرا مثل شخص متحیرو مانند جباری که نمی تواند نجات دهد هستی؟ اما تو‌ای خداوند در میان ما هستی و ما به نام تونامیده شده‌ایم پس ما را ترک منما.
\par 10 خداوند به این قوم چنین می‌گوید: «ایشان به آواره گشتن چنین مایل بوده‌اند و پایهای خودرا باز نداشتند. بنابراین خداوند ایشان را مقبول ننمود و حال عصیان ایشان را به یاد آورده، گناه ایشان را جزا خواهد داد.»
\par 11 و خداوند به من گفت: «برای خیریت این قوم دعا منما!
\par 12 چون روزه گیرند ناله ایشان را نخواهم شنید و چون قربانی سوختنی و هدیه آردی گذرانند ایشان راقبول نخواهم فرمود بلکه من ایشان را به شمشیر وقحط و وبا هلاک خواهم ساخت.»
\par 13 پس گفتم: «آه‌ای خداوند یهوه اینک انبیا به ایشان می‌گویند که شمشیر را نخواهید دید وقحطی به شما نخواهد رسید بلکه شما را در این مکان سلامتی پایدار خواهم داد.»
\par 14 پس خداوند مرا گفت: «این انبیا به اسم من به دروغ نبوت می‌کنند. من ایشان را نفرستادم و به ایشان امری نفرمودم و تکلم ننمودم، بلکه ایشان به رویاهای کاذب و سحر و بطالت و مکر دلهای خویش برای شما نبوت می‌کنند.
\par 15 بنابراین خداوند درباره این انبیا که به اسم من نبوت می‌کنند و من ایشان را نفرستاده‌ام و می‌گویند که شمشیر و قحط در این زمین نخواهد شد می‌گویدکه این انبیا به شمشیر و قحط کشته خواهند شد.
\par 16 و این قومی که برای ایشان نبوت می‌کنند درکوچه های اورشلیم به‌سبب قحط و شمشیرانداخته خواهند شد و کسی نخواهد بود که ایشان و زنان ایشان و پسران و دختران ایشان را دفن کندزیرا که شرارت ایشان را بر ایشان خواهم ریخت.
\par 17 پس این کلام را به ایشان بگو: چشمان من شبانه‌روز اشک می‌ریزد و آرامی ندارد زیرا که آن دوشیزه یعنی دختر قوم من به شکستگی عظیم وصدمه بینهایت سخت شکسته شده است.
\par 18 اگربه صحرا بیرون روم اینک کشتگان شمشیر و اگربه شهر داخل شوم اینک بیماران از گرسنگی. زیراکه هم انبیا و کهنه در زمین تجارت می‌کنند و هیچ نمی دانند.»
\par 19 آیا یهودا را بالکل ترک کرده‌ای و آیا جانت صهیون را مکروه داشته است؟ چرا ما را چنان زده‌ای که برای ما هیچ علاجی نیست؟ برای سلامتی انتظار کشیدیم اما هیچ خیری نیامد وبرای زمان شفا و اینک اضطراب پدید آمد. 
\par 20 ‌ای خداوند به شرارت خود و به عصیان پدران خویش اعتراف می‌نماییم زیرا که به تو گناه ورزیده‌ایم.
\par 21 به‌خاطر اسم خود ما را رد منما. کرسی جلال خویش را خوار مشمار. عهد خودرا که با ما بستی به یاد آورده، آن را مشکن.آیادر میان اباطیل امت‌ها هستند که باران ببارانند و آیا آسمان می‌تواند بارش بدهد؟ مگر تو‌ای یهوه خدای ما همان نیستی و به تو امیدوار هستیم چونکه تو فاعل همه اینکارها می‌باشی.
\par 22 آیادر میان اباطیل امت‌ها هستند که باران ببارانند و آیا آسمان می‌تواند بارش بدهد؟ مگر تو‌ای یهوه خدای ما همان نیستی و به تو امیدوار هستیم چونکه تو فاعل همه اینکارها می‌باشی.
 
\chapter{15}

\par 1 و خداوند مرا گفت: «اگر‌چه هم موسی و سموئیل به حضور من می‌ایستادندجان من به این قوم مایل نمی شد. ایشان را ازحضور من دور انداز تا بیرون روند.
\par 2 و اگر به توبگویند به کجا بیرون رویم، به ایشان بگو: خداوندچنین می‌فرماید: آنکه مستوجب موت است به موت و آنکه مستحق شمشیر است به شمشیر وآنکه سزاوار قحط است به قحط و آنکه لایق اسیری است به اسیری.
\par 3 و خداوند می‌گوید: برایشان چهار قسم خواهم گماشت: یعنی شمشیربرای کشتن و سگان برای دریدن و مرغان هوا وحیوانات صحرا برای خوردن و هلاک ساختن.
\par 4 وایشان را در تمامی ممالک جهان مشوش خواهم ساخت. به‌سبب منسی ابن حزقیا پادشاه یهودا وکارهایی که او در اورشلیم کرد.
\par 5 زیرا‌ای اورشلیم کیست که بر تو ترحم نماید و کیست که برای تو ماتم گیرد و کیست که یکسو برود تا ازسلامتی تو بپرسد؟
\par 6 خداوند می‌گوید: چونکه تومرا ترک کرده، به عقب برگشتی من نیز دست خودرا بر تو دراز کرده، تو را هلاک ساختم زیرا که ازپشیمان شدن بیزار گشتم.
\par 7 و ایشان را دردروازه های زمین با غربال خواهم بیخت و قوم خود را بی‌اولاد ساخته، هلاک خواهم نمودچونکه از راههای خود بازگشت نکردند.
\par 8 بیوه‌زنان ایشان برایم از ریگ دریا زیاده شده‌اند، پس بر ایشان در وقت ظهر بر مادر جوانان تاراج کننده‌ای خواهم آورد و ترس و آشفتگی رابر شهر ناگهان مستولی خواهم گردانید.
\par 9 زاینده هفت ولد زبون شده، جان بداد و آفتاب او که هنوز روز باقی بود غروب کرد و او خجل و رسواگردید. و خداوند می‌گوید: من بقیه ایشان را پیش روی دشمنان ایشان به شمشیر خواهم سپرد.»
\par 10 وای بر من که تو‌ای مادرم مرا مرد جنگجوو نزاع کننده‌ای برای تمامی جهان زاییدی. نه به ربوا دادم ونه به ربوا گرفتم. معهذا هر یک از ایشان مرا لعنت می‌کنند.
\par 11 خداوند می‌گوید: «البته تورا برای نیکویی رها خواهم ساخت و هر آینه دشمن را در وقت بلا و در زمان تنگی نزد تومتذلل خواهم گردانید.
\par 12 آیا آهن می‌تواند آهن شمالی و برنج را بشکند؟
\par 13 توانگری وخزینه هایت را نه به قیمت، بلکه به همه گناهانت و در تمامی حدودت به تاراج خواهم داد.
\par 14 و تورا همراه دشمنانت به زمینی که نمی دانی خواهم کوچانید زیرا که ناری در غضب من افروخته شده شما را خواهد سوخت.»
\par 15 ‌ای خداوند تو این را می‌دانی پس مرا بیادآورده، از من تفقد نما و انتقام مرا از ستمکارانم بگیر و به دیرغضبی خویش مرا تلف منما و بدان که به‌خاطر تو رسوایی را کشیده‌ام.
\par 16 سخنان تویافت شد و آنها را خوردم و کلام تو شادی وابتهاج دل من گردید. زیرا که به نام تو‌ای یهوه خدای صبایوت نامیده شده‌ام.
\par 17 در مجلس عشرت کنندگان ننشستم و شادی ننمودم. به‌سبب دست تو به تنهایی نشستم زیرا که مرا ازخشم مملو ساختی.
\par 18 درد من چرا دایمی است و جراحت من چرا مهلک و علاج ناپذیر می‌باشد؟ آیا تو برای من مثل چشمه فریبنده و آب ناپایدارخواهی شد؟
\par 19 بنابراین خداوند چنین می‌گوید: «اگربازگشت نمایی من بار دیگر تو را به حضور خودقایم خواهم ساخت و اگر نفایس را از رذایل بیرون کنی، آنگاه تو مثل دهان من خواهی بود وایشان نزد تو خواهند برگشت و تو نزد ایشان بازگشت نخواهی نمود.
\par 20 و من تو را برای این قوم دیوار برنجین حصاردار خواهم ساخت و باتو جنگ خواهند نمود، اما بر تو غالب نخواهندآمد زیرا خداوند می‌گوید: من برای نجات دادن ورهانیدن تو با تو هستم.و تو را از دست شریران خواهم رهانید و تو را از کف ستمکیشان فدیه خواهم نمود.»
\par 21 و تو را از دست شریران خواهم رهانید و تو را از کف ستمکیشان فدیه خواهم نمود.»
 
\chapter{16}

\par 1 و کلام خداوند بر من نازل شده، گفت:
\par 2 «برای خود زنی مگیر و تو را در این مکان پسران و دختران نباشد.
\par 3 زیرا خداونددرباره پسران و دخترانی که در این مکان مولودشوند و درباره مادرانی که ایشان را بزایند وپدرانی که ایشان را در این زمین تولید نمایند چنین می‌گوید:
\par 4 به بیماریهای مهلک خواهند مرد. برای ایشان ماتم نخواهند گرفت و دفن نخواهندشد بلکه بر روی زمین سرگین خواهند بود. و به شمشیر و قحط تباه خواهند شد و لاشهای ایشان غذای مرغان هوا و وحوش زمین خواهد بود.
\par 5 زیرا خداوند چنین می‌گوید: به خانه نوحه گری داخل مشو و برای ماتم گرفتن نرو و برای ایشان تعزیت منما زیرا خداوند می‌گوید که سلامتی خود یعنی احسان و مراحم خویش را از این قوم خواهم برداشت.
\par 6 هم بزرگ و هم کوچک در این زمین خواهند مرد و دفن نخواهند شد. و برای ایشان ماتم نخواهند گرفت و خویشتن را مجروح نخواهند ساخت و موی خود را نخواهند تراشید.
\par 7 و برای ماتم گری نان را پاره نخواهند کرد تاایشان را برای مردگان تعزیت نمایند و کاسه تعزیت را با ایشان برای پدر یا مادر ایشان هم نخواهند نوشید.
\par 8 و تو به خانه بزم داخل مشو و باایشان برای اکل و شرب منشین.
\par 9 زیرا که یهوه صبایوت خدای اسرائیل چنین می‌گوید: اینک من در ایام شما و در نظر شما آواز خوشی و آوازشادمانی و آواز داماد و آواز عروس را از این مکان خواهم برداشت.
\par 10 و هنگامی که همه این سخنان را به این قوم بیان کنی و ایشان از توبپرسند که خداوند از چه سبب تمامی این بلای عظیم را به ضد ما گفته است و عصیان و گناهی که به یهوه خدای خود ورزیده‌ایم چیست؟
\par 11 آنگاه تو به ایشان بگو: خداوند می‌گوید: از این جهت که پدران شما مرا ترک کردند و خدایان غیر را پیروی نموده، آنها راعبادت و سجده نمودند و مرا ترک کرده، شریعت مرا نگاه نداشتند.
\par 12 و شما ازپدران خویش زیاده شرارت ورزیدید چونکه هریک از شما سرکشی دل شریر خود را پیروی نمودید و به من گوش نگرفتید.
\par 13 بنابراین من شمارا از این زمین به زمینی که شما و پدران شماندانسته‌اید خواهم‌انداخت و در آنجا شبانه‌روزخدایان غیر را عبادت خواهید نمود زیرا که من برشما ترحم نخواهم نمود.
\par 14 «بنابراین خداوند می‌گوید: اینک ایامی می‌آید که بار دیگر گفته نخواهد شد قسم به حیات یهوه که بنی‌اسرائیل را از زمین مصر بیرون آورد.
\par 15 بلکه قسم به حیات یهوه که بنی‌اسرائیل را از زمین شمال و همه زمینهایی که ایشان را به آنها رانده بود برآورد. زیرا من ایشان را به زمینی که به پدران ایشان داده‌ام باز خواهم آورد.
\par 16 خداوند می‌گوید: اینک ماهی گیران بسیار راخواهم فرستاد تا ایشان را صید نمایند و بعد از آن صیادان بسیار را خواهم فرستاد تا ایشان را از هرکوه و هر تل و از سوراخهای صخره‌ها شکارکنند.
\par 17 زیرا چشمانم بر همه راههای ایشان است و آنها از نظر من پنهان نیست و عصیان ایشان از چشمان من مخفی نی.
\par 18 و من اول عصیان وگناهان ایشان را مکافات مضاعف خواهم رسانیدچونکه زمین مرا به لاشهای رجاسات خود ملوث نموده و میراث مرا به مکروهات خویش مملوساخته‌اند.»
\par 19 ‌ای خداوند که قوت من و قلعه من و در روزتنگی پناهگاه من هستی! امت‌ها از کرانهای زمین نزد تو آمده، خواهند گفت: پدران ما جز دروغ واباطیل و چیزهایی را که فایده نداشت وارث هیچ نشدند.
\par 20 آیا می‌شود که انسان برای خودخدایان بسازد و حال آنکه آنها خدا نیستند؟«بنابراین هان این مرتبه ایشان را عارف خواهم گردانید بلی دست خود و جبروت خویش را معروف ایشان خواهم ساخت وخواهند دانست که اسم من یهوه است.»
\par 21 «بنابراین هان این مرتبه ایشان را عارف خواهم گردانید بلی دست خود و جبروت خویش را معروف ایشان خواهم ساخت وخواهند دانست که اسم من یهوه است.»
 
\chapter{17}

\par 1 «گناه یهودا به قلم آهنین و نوک الماس مرقوم است. و بر لوح دل ایشان و بر شاخهای مذبح های شما منقوش است.
\par 2 مادامی که پسران ایشان مذبح های خود و اشیریم خویش را نزد درختان سبز و بر تلهای بلند یاد می‌دارند،
\par 3 ‌ای کوه من که در صحرا هستی توانگری وتمامی خزاین تو را به تاراج خواهم داد ومکان های بلند تو را نیز به‌سبب گناهی که در همه حدود خود ورزیده‌ای.
\par 4 و تو از خودت نیز ملک خویش را که به تو داده‌ام بی‌زرع خواهی گذاشت و دشمنانت را در زمینی که نمی دانی خدمت خواهی نمود زیرا آتشی در غضب من افروخته‌اید که تا به ابد مشتعل خواهد بود.»
\par 5 و خداوند چنین می‌گوید: «ملعون باد کسی‌که بر انسان توکل دارد و بشر را اعتماد خویش سازد و دلش از یهوه منحرف باشد.
\par 6 و او مثل درخت عرعر در بیابان خواهد بود و چون نیکویی آید آن را نخواهد دید بلکه در مکان های خشک بیابان در زمین شوره غیرمسکون ساکن خواهدشد.»
\par 7 مبارک باد کسی‌که بر خداوند توکل دارد وخداوند اعتماد او باشد.
\par 8 او مثل درخت نشانده بر کنار آب خواهد بود که ریشه های خویش رابسوی نهر پهن می‌کند و چون گرما بیاید نخواهدترسید و برگش شاداب خواهد ماند و درخشکسالی اندیشه نخواهد داشت و از آوردن میوه باز نخواهد ماند.
\par 9 دل از همه‌چیز فریبنده تراست و بسیار مریض است کیست که آن را بداند؟
\par 10 «من یهوه تفتیش کننده دل و آزماینده گرده هاهستم تا بهر کس بر‌حسب راههایش و بر وفق ثمره اعمالش جزا دهم.»
\par 11 مثل کبک که بر تخمهایی که ننهاده باشدبنشیند، همچنان است کسی‌که مال را به بی انصافی جمع کند. در نصف روزهایش آن راترک خواهد کرد و در آخرت خود احمق خواهدبود.
\par 12 موضع قدس ما کرسی جلال و از ازل مرتفع است.
\par 13 ‌ای خداوند که امید اسرائیل هستی همگانی که تو را ترک نمایند خجل خواهند شد. آنانی که از من منحرف شوند درزمین مکتوب خواهند شد چونکه خداوند را که چشمه آب حیات‌است ترک نموده‌اند.
\par 14 ‌ای خداوند مرا شفا بده، پس شفا خواهم یافت. مرانجات بده، پس ناجی خواهم شد زیرا که توتسبیح من هستی.
\par 15 اینک ایشان به من می‌گویند: «کلام خداوند کجاست؟ الان واقع بشود.»
\par 16 واما من از بودن شبان برای پیروی تو تعجیل ننمودم و تو می‌دانی که یوم بلا را نخواستم. آنچه از لبهایم بیرون آمد به حضور تو ظاهر بود.
\par 17 برای من باعث ترس مباش که در روز بلاملجای من تویی.
\par 18 ستمکاران من خجل شونداما من خجل نشوم. ایشان هراسان شوند اما من هراسان نشوم. روز بلا را بر ایشان بیاور و ایشان رابه هلاکت مضاعف هلاک کن.
\par 19 خداوند به من چنین گفت که «برو و نزددروازه پسران قوم که پادشاهان یهودا از آن داخل می‌شوند و از آن بیرون می‌روند و نزد همه دروازه های اورشلیم بایست.
\par 20 و به ایشان بگو: ای پادشاهان یهودا و تمامی یهودا و جمیع سکنه اورشلیم که از این دروازه‌ها داخل می‌شوید کلام خداوند را بشنوید!
\par 21 خداوند چنین می‌گوید: برخویشتن با حذر باشید و در روز سبت هیچ باری حمل نکنید و آن را داخل دروازه های اورشلیم مسازید.
\par 22 و در روز سبت هیچ باری ازخانه های خود بیرون میاورید و هیچکار مکنیدبلکه روز سبت را تقدیس نمایید چنانکه به پدران شما امر فرمودم.»
\par 23 اما ایشان نشنیدند و گوش خود را فرانداشتند بلکه گردنهای خود را سخت ساختند تانشنوند و تادیب را نپذیرند.
\par 24 و خداوندمی گوید: «اگر مرا حقیقت بشنوید و در روز سبت، هیچ باری از دروازه های این شهر داخل نسازید وروز سبت را تقدیس نموده، هیچکار در آن نکنید،
\par 25 آنگاه پادشاهان و سروران بر کرسی داود نشسته و بر ارابه‌ها و اسبان سوار شده، ایشان و سروران ایشان مردان یهودا و ساکنان اورشلیم از دروازه های این شهر داخل خواهند شد و این شهر تا به ابد مسکون خواهد بود. 
\par 26 و ازشهرهای یهودا و از نواحی اورشلیم و از زمین بنیامین و از همواری و کوهستان و جنوب خواهند آمد و قربانی های سوختنی و ذبایح وهدایای آردی و بخور خواهند‌آورد و ذبایح تشکر را به خانه خداوند خواهند‌آورد.و اگرمرا نشنیده روز سبت را تقدیس ننمایید و در روزسبت باری برداشته، به شهرهای اورشلیم داخل سازید آنگاه در دروازه هایش آتشی خواهم افروخت که قصرهای اورشلیم را خواهد سوخت و خاموش نخواهد شد.»
\par 27 و اگرمرا نشنیده روز سبت را تقدیس ننمایید و در روزسبت باری برداشته، به شهرهای اورشلیم داخل سازید آنگاه در دروازه هایش آتشی خواهم افروخت که قصرهای اورشلیم را خواهد سوخت و خاموش نخواهد شد.»
 
\chapter{18}

\par 1 کلامی که از جانب خداوند به ارمیا نازل شده، گفت:
\par 2 «برخیز و به خانه کوزه‌گر فرود آی که در آنجا کلام خود را به تو خواهم شنوانید.»
\par 3 پس به خانه کوزه‌گر فرود شدم و اینک او برچرخها کار می‌کرد.
\par 4 و ظرفی که از گل می‌ساخت در دست کوزه‌گر ضایع شد پس دوباره ظرفی دیگر از آن ساخت بطوری که به نظرکوزه‌گر پسند آمد که بسازد.
\par 5 آنگاه کلام خداوندبه من نازل شده، گفت:
\par 6 «خداوند می‌گوید: ای خاندان اسرائیل آیا من مثل این کوزه‌گر با شماعمل نتوانم نمود زیرا چنانکه گل در دست کوزه‌گر است، همچنان شما‌ای خاندان اسرائیل در دست من می‌باشید.
\par 7 هنگامی که درباره امتی یا مملکتی برای کندن و منهدم ساختن و هلاک نمودن سخنی گفته باشم،
\par 8 اگر آن امتی که درباره ایشان گفته باشم از شرارت خویش بازگشت نمایند، آنگاه از آن بلایی که به آوردن آن قصدنموده‌ام خواهم برگشت.
\par 9 و هنگامی که درباره امتی یامملکتی به جهت بنا کردن و غرس نمودن سخن گفته باشم،
\par 10 اگر ایشان در نظر من شرارت ورزند و قول مرا نشنوند آنگاه از آن نیکویی که گفته باشم که برای ایشان بکنم خواهم برگشت.
\par 11 الان مردان یهودا و ساکنان اورشلیم را خطاب کرده، بگو که خداوند چنین می‌گوید: اینک من به ضد شما بلایی مهیا می‌سازم و قصدی به خلاف شما می‌نمایم. پس شما هر کدام از راه زشت خودبازگشت نمایید و راهها و اعمال خود را اصلاح کنید.
\par 12 اما ایشان خواهند گفت: امید نیست زیراکه افکار خود را پیروی خواهیم نمود و هر کدام موافق سرکشی دل شریر خود رفتار خواهیم کرد.
\par 13 بنابراین خداوند چنین می‌گوید: در میان امت‌ها سوال کنید کیست که مثل این چیزها راشنیده باشد؟ دوشیزه اسرائیل کار بسیار زشت کرده است.
\par 14 آیا برف لبنان از صخره صحرا بازایستد یا آبهای سرد که از جای دور جاری می‌شود خشک گردد؟
\par 15 زیرا که قوم من مرافراموش کرده برای اباطیل بخور می‌سوزانند وآنها ایشان را از راههای ایشان یعنی از طریق های قدیم می‌لغزانند تا در کوره راهها به راههایی که ساخته نشده است راه بروند.
\par 16 تا زمین خود رامایه حیرت و سخریه ابدی بگردانند به حدی که هر‌که از آن گذر کند متحیر شده، سر خود راخواهد جنبانید.
\par 17 من مثل باد شرقی ایشان را ازحضور دشمنان پراکنده خواهم ساخت و در روزمصیبت ایشان پشت را به ایشان نشان خواهم دادو نه رو را.»
\par 18 آنگاه گفتند: «بیایید تا به ضد ارمیا تدبیرهانماییم زیرا که شریعت از کاهنان و مشورت ازحکیمان و کلام از انبیا ضایع نخواهد شد پس بیایید تا او را به زبان خود بزنیم و هیچ سخنش راگوش ندهیم.»
\par 19 ‌ای خداوند مرا گوش بده و آواز دشمنان مرا بشنو!
\par 20 آیا بدی به عوض نیکویی ادا خواهدشد زیرا که حفره‌ای برای جان من کنده‌اند. بیادآور که چگونه به حضور تو ایستاده بودم تا درباره ایشان سخن نیکو گفته، حدت خشم تو را ازایشان بگردانم.
\par 21 پس پسران ایشان را به قحطبسپار و ایشان را به دم شمشیر تسلیم نما و زنان ایشان، بی‌اولاد و بیوه گردند و مردان ایشان به سختی کشته شوند و جوانان ایشان، در جنگ به شمشیر مقتول گردند.
\par 22 و چون فوجی بر ایشان ناگهان بیاوری نعره‌ای از خانه های ایشان شنیده شود زیرا به جهت گرفتار کردنم حفره‌ای کنده اندو دامها برای پایهایم پنهان نموده.اما تو‌ای خداوند تمامی مشورتهایی را که ایشان به قصد جان من نموده‌اند می‌دانی. پس عصیان ایشان رامیامرز و گناه ایشان را از نظر خویش محو مسازبلکه ایشان به حضور تو لغزانیده شوند و در حین غضب خویش، با ایشان عمل نما.
\par 23 اما تو‌ای خداوند تمامی مشورتهایی را که ایشان به قصد جان من نموده‌اند می‌دانی. پس عصیان ایشان رامیامرز و گناه ایشان را از نظر خویش محو مسازبلکه ایشان به حضور تو لغزانیده شوند و در حین غضب خویش، با ایشان عمل نما.
 
\chapter{19}

\par 1 خداوند چنین گفت: «برو و کوزه سفالین از کوزه‌گر بخر و بعضی از مشایخ قوم ومشایخ کهنه را همراه خود بردار.
\par 2 و به وادی ابن هنوم که نزد دهنه دروازه کوزه‌گران است بیرون رفته، سخنانی را که به تو خواهم گفت در آنجا نداکن.
\par 3 و بگو: ای پادشاهان یهودا و سکنه اورشلیم کلام خداوند را بشنوید! یهوه صبایوت خدای اسرائیل چنین می‌گوید: اینک بر این مکان چنان بلایی خواهم آورد که گوش هر کس که آن رابشنود صدا خواهد کرد.
\par 4 زانرو که مرا ترک کردندو این مکان را خوار شمردند و بخور در آن برای خدایان غیر که نه خود ایشان و نه پدران ایشان ونه پادشاهان یهودا آنها را شناخته بودندسوزانیدند و این مکان را از خون بی‌گناهان مملوساختند.
\par 5 و مکان های بلند برای بعل بنا کردند تاپسران خود را به‌جای قربانی های سوختنی برای بعل بسوزانند که من آن را امر نفرموده و نگفته ودر دلم نگذشته بود.
\par 6 بنابراین خداوند می‌گوید: اینک ایامی می‌آید که این مکان به توفت یا به وادی ابن هنوم دیگر نامیده نخواهد شد بلکه به وادی قتل.
\par 7 و مشورت یهودا و اورشلیم را در این مکان باطل خواهم گردانید و ایشان را از حضوردشمنان ایشان و به‌دست آنانی که قصد جان ایشان دارند خواهم‌انداخت و لاشهای ایشان راخوراک مرغان هوا و حیوانات زمین خواهم ساخت.
\par 8 و این شهر را مایه حیرت و سخریه خواهم گردانید به حدی که هر‌که از آن عبور کندمتحیر شده، به‌سبب جمیع بلایایش سخریه خواهد نمود.
\par 9 و گوشت پسران ایشان و گوشت دختران ایشان را به ایشان خواهم خورانید و درمحاصره و تنگی که دشمنان ایشان و جویندگان جان ایشان بر ایشان خواهند‌آورد، هر کس گوشت همسایه خود را خواهد خورد.
\par 10 آنگاه کوزه را به نظر آنانی که همراه تو می‌روند بشکن.
\par 11 و ایشان را بگو: یهوه صبایوت چنین می‌گوید: به نوعی که کسی کوزه کوزه‌گر را می‌شکند و آن رادیگر اصلاح نتوان کرد همچنان این قوم و این شهر را خواهم شکست و ایشان را در توفت دفن خواهند کرد تا جایی برای دفن کردن باقی نماند.
\par 12 خداوند می‌گوید: به این مکان و به ساکنانش چنین عمل خواهم نمود و این شهر را مثل توفت خواهم ساخت.
\par 13 و خانه های اورشلیم وخانه های پادشاهان یهودا مثل مکان توفت نجس خواهد شد یعنی همه خانه هایی که بر بامهای آنهابخور برای تمامی لشکر آسمان سوزانیدند وهدایای ریختنی برای خدایان غیر ریختند.»
\par 14 پس ارمیا از توفت که خداوند او را به جهت نبوت کردن به آنجا فرستاده بود باز آمد و درصحن خانه خداوند ایستاده، به تمامی قوم گفت:«یهوه صبایوت خدای اسرائیل چنین می‌گوید: اینک من بر این شهر و بر همه قریه هایش، تمامی بلایا را که درباره‌اش گفته‌ام وارد خواهم آورد زیرا که گردن خود را سخت گردانیده، کلام مرا نشنیدند.»
\par 15 «یهوه صبایوت خدای اسرائیل چنین می‌گوید: اینک من بر این شهر و بر همه قریه هایش، تمامی بلایا را که درباره‌اش گفته‌ام وارد خواهم آورد زیرا که گردن خود را سخت گردانیده، کلام مرا نشنیدند.»
 
\chapter{20}

\par 1 و فشحور بن امیر کاهن که ناظر اول خانه خداوند بود، ارمیا نبی را که به این امورنبوت می‌کرد شنید.
\par 2 پس فشحور ارمیای نبی رازده، او را در کنده‌ای که نزد دروازه عالی بنیامین که نزد خانه خداوند بود گذاشت.
\par 3 و در فردای آن روز فشحور ارمیا را از کنده بیرون آورد و ارمیاوی را گفت: «خداوند اسم تو را نه فشحور بلکه ماجور مسابیب خوانده است.
\par 4 زیرا خداوندچنین می‌گوید: اینک من تو را مورث ترس خودت و جمیع دوستانت می‌گردانم و ایشان به شمشیر دشمنان خود خواهند افتاد و چشمانت خواهد دید و تمامی یهودا را به‌دست پادشاه بابل تسلیم خواهم کرد که او ایشان را به بابل به اسیری برده، ایشان را به شمشیر به قتل خواهد رسانید.
\par 5 و تمامی دولت این شهر و تمامی مشقت آن را وجمیع نفایس آن را تسلیم خواهم کرد و همه خزانه های پادشاهان یهودا را به‌دست دشمنان ایشان خواهم سپرد که ایشان را غارت کرده وگرفتار نموده، به بابل خواهند برد.
\par 6 و تو‌ای فشحور با جمیع سکنه خانه ات به اسیری خواهیدرفت. و تو با جمیع دوستانت که نزد ایشان به دروغ نبوت کردی، به بابل داخل شده، در آنجاخواهید مرد و در آنجا دفن خواهید شد.»
\par 7 ‌ای خداوند مرا فریفتی پس فریفته شدم. ازمن زورآورتر بودی و غالب شدی. تمامی روزمضحکه شدم و هر کس مرا استهزا می‌کند.
\par 8 زیراهر گاه می‌خواهم تکلم نمایم ناله می‌کنم و به ظلم و غارت ندا می‌نمایم. زیرا کلام خداوند تمامی روز برای من موجب عار و استهزا گردیده است.
\par 9 پس گفتم که او را ذکر نخواهم نمود و بار دیگر به اسم او سخن نخواهم گفت، آنگاه در دل من مثل آتش افروخته شد و در استخوانهایم بسته گردیدو از خودداری خسته شده، باز نتوانستم ایستاد.
\par 10 زیرا که از بسیاری مذمت شنیدم و از هر جانب خوف بود و جمیع اصدقای من گفتند بر اوشکایت کنید و ما شکایت خواهیم نمود و مراقب لغزیدن من می‌باشند (و می‌گویند) که شاید اوفریفته خواهد شد تا بر وی غالب آمده، انتقام خود را از او بکشیم.
\par 11 لیکن خداوند با من مثل جبار قاهر است از این جهت ستمکاران من خواهند لغزید و غالب نخواهند آمد و چونکه به فطانت رفتار ننمودند به رسوایی ابدی که فراموش نخواهند شد بی‌نهایت خجل خواهندگردید.
\par 12 اما‌ای یهوه صبایوت که عادلان رامی آزمایی و گرده‌ها و دلها را مشاهده می‌کنی، بشود که انتقام تو را از ایشان ببینم زیرا که دعوی خویش را نزد تو کشف نمودم.
\par 13 برای خداوندبسرایید و خداوند را تسبیح بخوانید زیرا که جان مسکینان را از دست شریران رهایی داده است.
\par 14 ملعون باد روزی که در آن مولود شدم ومبارک مباد روزی که مادرم مرا زایید.
\par 15 ملعون باد کسی‌که پدر مرا مژده داد و گفت که برای توولد نرینه‌ای زاییده شده است و او را بسیارشادمان گردانید.
\par 16 و آنکس مثل شهرهایی که خداوند آنها را شفقت ننموده واژگون ساخت بشود و فریادی در صبح و نعره‌ای در وقت ظهربشنود.
\par 17 زیرا که مرا از رحم نکشت تا مادرم قبر من باشد و رحم او همیشه آبستن ماند.چرا ازرحم بیرون آمدم تا مشقت و غم را مشاهده نمایم و روزهایم در خجالت تلف شود؟
\par 18 چرا ازرحم بیرون آمدم تا مشقت و غم را مشاهده نمایم و روزهایم در خجالت تلف شود؟
 
\chapter{21}

\par 1 کلامی که به ارمیا از جانب خداوند نازل شد وقتی که صدقیا پادشاه، فشحور بن ملکیا و صفنیا ابن معسیای کاهن را نزد وی فرستاده، گفت:
\par 2 «برای ما از خداوند مسالت نمازیرا که نبوکدرصر پادشاه بابل با ما جنگ می‌کندشاید که خداوند موافق کارهای عجیب خود با ماعمل نماید تا او از ما برگردد.»
\par 3 و ارمیا به ایشان گفت: «به صدقیا چنین بگویید:
\par 4 یهوه خدای اسرائیل چنین می‌فرماید: اینک من اسلحه جنگ را که به‌دست شماست و شما با آنها با پادشاه بابل و کلدانیانی که شما را از بیرون دیوارها محاصره نموده‌اند جنگ می‌کنید برمی گردانم و ایشان رادر اندرون این شهر جمع خواهم کرد.
\par 5 و من به‌دست دراز و بازوی قوی و به غضب و حدت وخشم عظیم با شما مقاتله خواهم نمود.
\par 6 وساکنان این شهر را هم از انسان و هم از بهایم خواهم زد که به وبای سخت خواهند مرد.
\par 7 وخداوند می‌گوید که بعد از آن صدقیا پادشاه یهودا و بندگانش و این قوم یعنی آنانی را که از وباو شمشیر و قحط در این شهر باقی‌مانده باشند به‌دست نبوکدرصر پادشاه بابل و به‌دست دشمنان ایشان و به‌دست جویندگان جان ایشان تسلیم خواهم نمود تا ایشان را به دم شمشیر بکشد و اوبر ایشان رافت و شفقت و ترحم نخواهد نمود. 
\par 8 و به این قوم بگو که خداوند چنین می‌فرماید: اینک من طریق حیات و طریق موت را پیش شمامی گذارم؛
\par 9 هرکه در این شهر بماند از شمشیر وقحط و وبا خواهد مرد، اما هر‌که بیرون رود و به‌دست کلدانیانی که شما را محاصره نموده اندبیفتد، زنده خواهد ماند و جانش برای او غنیمت خواهد شد.
\par 10 زیرا خداوند می‌گوید: من روی خود را بر این شهر به بدی و نه به نیکویی برگردانیدم و به‌دست پادشاه بابل تسلیم شده، آن را به آتش خواهد سوزانید.
\par 11 «و درباره خاندان پادشاه یهودا بگو کلام خداوند را بشنوید:
\par 12 ‌ای خاندان داود خداوندچنین می‌فرماید: بامدادان به انصاف داوری نمایید و مغصوبان را از دست ظالمان برهانیدمبادا حدت خشم من به‌سبب بدی اعمال شمامثل آتش صادر گردد و مشتعل شده، خاموش کننده‌ای نباشد.
\par 13 خداوند می‌گوید: ای ساکنه وادی و‌ای صخره هامون که می‌گوییدکیست که به ضد ما فرود آید و کیست که به مسکنهای ما داخل شود اینک من به ضد توهستم.و خداوند می‌گوید بر‌حسب ثمره اعمال شما به شما عقوبت خواهم رسانید وآتشی در جنگل این (شهر) خواهم افروخت که تمامی حوالی آن را خواهد سوزانید.»
\par 14 و خداوند می‌گوید بر‌حسب ثمره اعمال شما به شما عقوبت خواهم رسانید وآتشی در جنگل این (شهر) خواهم افروخت که تمامی حوالی آن را خواهد سوزانید.»
 
\chapter{22}

\par 1 خداوند چنین گفت: «به خانه پادشاه یهودا فرود آی و در آنجا به این کلام متکلم شو
\par 2 و بگو: ای پادشاه یهودا که بر کرسی داود نشسته‌ای، تو و بندگانت و قومت که به این دروازه‌ها داخل می‌شوید کلام خداوند را بشنوید:
\par 3 خداوند چنین می‌گوید: انصاف وعدالت را اجرا دارید و مغصوبان را از دست ظالمان برهانید و بر غربا و یتیمان و بیوه‌زنان ستم و جور منمایید و خون بی‌گناهان را در این مکان مریزید.
\par 4 زیرا اگر این کار را بجا آورید هماناپادشاهانی که بر کرسی داود بنشینند، ازدروازه های این خانه داخل خواهند شد و هر یک با بندگان و قوم خود بر ارابه‌ها و اسبان سوارخواهند گردید.
\par 5 اما اگر این سخنان را نشنویدخداوند می‌گوید که به ذات خود قسم خوردم که این خانه خراب خواهد شد.
\par 6 زیرا خداونددرباره خاندان پادشاه یهودا چنین می‌گوید: اگرچه تو نزد من جلعاد و قله لبنان می‌باشی لیکن من تو را به بیابان و شهرهای غیرمسکون مبدل خواهم ساخت.
\par 7 و بر تو خراب کنندگان که هریک با آلاتش باشد معین می‌کنم و ایشان بهترین سروهای آزاد تو را قطع نموده، به آتش خواهندافکند.
\par 8 و امت های بسیار چون از این شهر عبورنمایند به یکدیگر خواهند گفت که خداوند به این شهر عظیم چرا چنین کرده است.
\par 9 و جواب خواهند داد از این سبب که عهد یهوه خدای خودرا ترک کردند و خدایان غیر را سجده و عبادت نمودند.
\par 10 «برای مرده گریه منمایید و برای او ماتم مگیرید. زارزار بگریید برای او که می‌رود زیرا که دیگر مراجعت نخواهد کرد و زمین مولد خویش را نخواهد دید.
\par 11 زیرا خداوند درباره شلوم بن یوشیا پادشاه یهودا که بجای پدر خود یوشیاپادشاه شده و از این مکان بیرون رفته است چنین می گوید که دیگر به اینجا برنخواهد گشت.
\par 12 بلکه در مکانی که او را به اسیری برده اندخواهد مرد و این زمین را باز نخواهد دید.
\par 13 «وای بر آن کسی‌که خانه خود را به بی‌انصافی و کوشکهای خویش را به ناحق بنامی کند که از همسایه خود مجان خدمت می‌گیردو مزدش را به او نمی دهد.
\par 14 که می‌گوید خانه وسیع و اطاقهای مروح برای خود بنا می‌کنم وپنجره‌ها برای خویشتن می‌شکافد و (سقف ) آن را از سرو آزاد می‌پوشاند و با شنجرف رنگ می‌کند.
\par 15 آیا از این جهت که با سروهای آزادمکارمت می‌نمایی، سلطنت خواهی کرد؟ آیاپدرت اکل وشرب نمی نمود و انصاف و عدالت رابجا نمی آورد، آنگاه برایش سعادتمندی می‌بود؟
\par 16 فقیر و مسکین را دادرسی می‌نمود، آنگاه سعادتمندی می‌شد. مگر شناختن من این نیست؟ خداوند می‌گوید:
\par 17 اما چشمان و دل تو نیست جز برای حرص خودت و برای ریختن خون بی‌گناهان و برای ظلم و ستم تا آنها را بجا آوری.
\par 18 بنابراین خداوند درباره یهویاقیم بن یوشیاپادشاه یهودا چنین می‌گوید: که برایش ماتم نخواهند گرفت و نخواهند گفت: آه‌ای برادر من یا آه‌ای خواهر و نوحه نخواهند کرد و نخواهندگفت: آه‌ای آقا یا آه‌ای جلال وی.
\par 19 کشیده شده و بیرون از دروازه های اورشلیم بجای دورانداخته شده به دفن الاغ مدفون خواهد گردید.
\par 20 «به فراز لبنان برآمده، فریاد برآور و آوازخود را در باشان بلندکن. و از عباریم فریاد کن زیرا که جمیع دوستانت تلف شده‌اند.
\par 21 در حین سعادتمندی تو به تو سخن گفتم، اما گفتی گوش نخواهم گرفت. همین از طفولیتت عادت تو بوده است که به آواز من گوش ندهی.
\par 22 باد تمامی شبانانت را خواهد چرانید و دوستانت به اسیری خواهند رفت. پس در آن وقت به‌سبب تمامی شرارتت خجل و رسوا خواهی شد.
\par 23 ‌ای که درلبنان ساکن هستی و آشیانه خویش را درسروهای آزاد می‌سازی! هنگامی که المها و دردمثل زنی که می‌زاید تو را فرو‌گیرد چه قدر بر توافسوس خواهند کرد؟
\par 24 یهوه می‌گوید: به حیات من قسم که اگر‌چه کنیاهو ابن یهویاقیم پادشاه یهودا خاتم بر دست راست من می‌بود هرآینه تو را از آنجا می‌کندم.
\par 25 و تو را به‌دست آنانی که قصد جان تو دارند و به‌دست آنانی که ازایشان ترسانی و به‌دست نبوکدرصر پادشاه بابل وبه‌دست کلدانیان تسلیم خواهم نمود.
\par 26 و تو ومادر تو را که تو را زایید، به زمین غریبی که در آن تولد نیافتید خواهم‌انداخت که در آنجا خواهیدمرد.
\par 27 اما به زمینی که ایشان بسیار آرزو دارند که به آن برگردند مراجعت نخواهند نمود.»
\par 28 آیا این مرد کنیاهو ظرفی خوار شکسته می‌باشد و یا ظرفی ناپسندیده است؟ چرا او بااولادش به زمینی که آن را نمی شناسند انداخته وافکنده شده‌اند؟
\par 29 ‌ای زمین‌ای زمین‌ای زمین، کلام خداوند رابشنو!خداوند چنین می‌فرماید: «این شخص را بی‌اولاد و کسی‌که در روزگار خود کامیاب نخواهد شد بنویس، زیرا که هیچکس از ذریت وی کامیاب نخواهد شد و بر کرسی داود نخواهدنشست، و بار دیگر در یهودا سلطنت نخواهد نمود.»
\par 30 خداوند چنین می‌فرماید: «این شخص را بی‌اولاد و کسی‌که در روزگار خود کامیاب نخواهد شد بنویس، زیرا که هیچکس از ذریت وی کامیاب نخواهد شد و بر کرسی داود نخواهدنشست، و بار دیگر در یهودا سلطنت نخواهد نمود.»
 
\chapter{23}

\par 1 خداوند می‌گوید: «وای بر شبانانی که گله مرتع مرا هلاک و پراکنده می‌سازند.»
\par 2 بنابراین، یهوه خدای اسرائیل درباره شبانانی که قوم مرا می‌چرانند چنین می‌گوید: «شما گله مرا پراکنده ساخته و رانده ایدو به آنها توجه ننموده‌اید. پس خداوند می‌گویداینک من عقوبت بدی اعمال شما را بر شماخواهم رسانید.
\par 3 و من بقیه گله خویش را از همه زمینهایی که ایشان را به آنها رانده‌ام جمع خواهم کرد و ایشان را به آغلهای ایشان باز خواهم آوردکه بارور و بسیار خواهند شد.
\par 4 وبرای ایشان شبانانی که ایشان را بچرانند برپا خواهم نمود که بار دیگر ترسان و مشوش نخواهند شد و مفقودنخواهند گردید.» قول خداوند این است.
\par 5 خداوند می‌گوید: «اینک ایامی می‌آید که شاخه‌ای عادل برای داود برپا می‌کنم و پادشاهی سلطنت نموده، به فطانت رفتار خواهد کرد وانصاف و عدالت را در زمین مجرا خواهد داشت.
\par 6 در ایام وی یهودا نجات خواهد یافت و اسرائیل با امنیت ساکن خواهد شد و اسمی که به آن نامیده می‌شود این است: یهوه صدقینو (یهوه عدالت ما).
\par 7 بنابراین خداوند می‌گوید: اینک ایامی می‌آید که دیگر نخواهند گفت قسم به حیات یهوه که بنی‌اسرائیل را از زمین مصر برآورد.
\par 8 بلکه قسم به حیات یهوه که ذریت خاندان اسرائیل را از زمین شمال و از همه زمینهایی که ایشان را به آنها رانده بودم بیرون آورده، رهبری نموده است و در زمین خود ساکن خواهند شد.»
\par 9 به‌سبب انبیا دل من در اندرونم شکسته وهمه استخوانهایم مسترخی شده است، مثل شخص مست و مانند مرد مغلوب شراب ازجهت خداوند و از جهت کلام مقدس اوگردیده‌ام.
\par 10 زیرا که زمین پر از زناکاران است و به‌سبب لعنت زمین ماتم می‌کند و مرتع های بیابان خشک شده است زیرا که طریق ایشان بد وتوانایی ایشان باطل است.
\par 11 چونکه هم انبیا و هم کاهنان منافق‌اند و خداوند می‌گوید: شرارت ایشان را هم در خانه خود یافته‌ام.
\par 12 بنابراین طریق ایشان مثل جایهای لغزنده در تاریکی غلیظ برای ایشان خواهد بود که ایشان رانده شده در آن خواهند افتاد. زیرا خداوند می‌گوید که «درسال عقوبت ایشان بلا بر ایشان عارض خواهم گردانید.
\par 13 و در انبیای سامره حماقتی دیده‌ام که برای بعل نبوت کرده، قوم من اسرائیل را گمراه گردانیده‌اند.
\par 14 و در انبیای اورشلیم نیز چیزهولناک دیدم. مرتکب زنا شده، به دروغ سلوک می‌نمایند و دستهای شریران را تقویت می‌دهندمبادا هر یک از ایشان از شرارت خویش بازگشت نماید. و جمیع ایشان برای من مثل سدوم وساکنان آن مانند عموره گردیده‌اند.»
\par 15 بنابراین یهوه صبایوت درباره آن انبیا چنین می‌گوید: «اینک من به ایشان افسنتین خواهم خورانید و آب تلخ به ایشان خواهم نوشانید زیراکه از انبیای اورشلیم نفاق در تمامی زمین منتشرشده است.»
\par 16 یهوه صبایوت چنین می‌گوید: «به سخنان این انبیایی که برای شما نبوت می‌کنندگوش مدهید زیرا شما را به بطالت تعلیم می‌دهندو رویای دل خود را بیان می‌کنند و نه از دهان خداوند.
\par 17 و به آنانی که مرا حقیر می‌شمارندپیوسته می‌گویند: خداوند می‌فرماید که برای شما سلامتی خواهد بود و به آنانی که به‌سرکشی دل خود سلوک می‌نمایند می‌گویند که بلا به شمانخواهد رسید.
\par 18 زیرا کیست که به مشورت خداوند واقف شده باشد تا ببیند و کلام او رابشنود و کیست که به کلام او گوش فرا داشته، استماع نموده باشد.
\par 19 اینک باد شدید غضب خداوند صادر شده و گردبادی دور می‌زند و برسر شریران فرود خواهد آمد.
\par 20 غضب خداوندتا مقاصد دل او را بجا نیاورد و به انجام نرساندبرنخواهد گشت. در ایام آخر این را نیکو خواهیدفهمید.
\par 21 من این انبیا را نفرستادم لیکن دویدند. به ایشان سخن نگفتم اما ایشان نبوت نمودند.
\par 22 امااگر در مشورت من قایم می‌ماندند، کلام مرا به قوم من بیان می‌کردند و ایشان را از راه بد و از اعمال شریر ایشان برمی گردانیدند.
\par 23 یهوه می‌گوید: آیا من خدای نزدیک هستم و خدای دور نی؟
\par 24 و خداوند می‌گوید: آیا کسی خویشتن را درجای مخفی پنهان تواند نمود که من او را نبینم مگر من آسمان و زمین را مملو نمی سازم؟ کلام خداوند این است.
\par 25 سخنان انبیا را که به اسم من کاذبانه نبوت کردند شنیدم که گفتند خواب دیدم خواب دیدم.
\par 26 این تا به کی در دل انبیایی که کاذبانه نبوت می‌کنند خواهد بود که انبیای فریب دل خودشان می‌باشند،
\par 27 که به خوابهای خویش که هر کدام از ایشان به همسایه خود باز می‌گویند خیال دارند که اسم مرا از یاد قوم من ببرند، چنانکه پدران ایشان اسم مرا برای بعل فراموش کردند.
\par 28 آن نبی‌ای که خواب دیده است خواب را بیان کند و آن که کلام مرا دارد کلام مرا براستی بیان نماید. خداوند می‌گوید کاه را با گندم چه‌کاراست؟»
\par 29 و خداوند می‌گوید: «آیا کلام من مثل آتش نیست و مانند چکشی که صخره را خردمی کند؟»
\par 30 لهذا خداوند می‌گوید: «اینک من به ضد این انبیایی که کلام مرا از بکدیگر می‌دزدندهستم.»
\par 31 و خداوند می‌گوید: «اینک من به ضداین انبیا هستم که زبان خویش را بکار برده، می‌گویند: او گفته است.»
\par 32 و خداوند می‌گوید: «اینک من به ضد اینان هستم که به خوابهای دروغ نبوت می‌کنند و آنها را بیان کرده، قوم مرا به دروغها و خیالهای خود گمراه می‌نمایند. و من ایشان را نفرستادم و مامور نکردم پس خداوندمی گوید که به این قوم هیچ نفع نخواهند رسانید. 
\par 33 و چون این قوم یا نبی یا کاهنی از تو سوال نموده، گویند که وحی خداوند چیست؟ پس به ایشان بگو: کدام وحی؟ قول خداوند این است که شما را ترک خواهم نمود.
\par 34 و آن نبی یا کاهن یاقومی که گویند وحی یهوه، همانا بر آن مرد و برخانه‌اش عقوبت خواهم رسانید.
\par 35 و هر کدام ازشما به همسایه خویش و هر کدام به برادر خودچنین گویید که خداوند چه جواب داده است وخداوند چه گفته است؟
\par 36 لیکن وحی یهوه رادیگر ذکر منمایید زیرا کلام هر کس وحی اوخواهد بود چونکه کلام خدای حی یعنی یهوه صبایوت خدای ما را منحرف ساخته‌اید.
\par 37 و به نبی چنین بگو که خداوند به تو چه جواب داده وخداوند به تو چه گفته است؟
\par 38 و اگر می‌گویید: وحی یهوه، پس یهوه چنین می‌فرماید چونکه این سخن یعنی وحی یهوه را گفتید با آنکه نزد شمافرستاده، فرمودم که وحی یهوه را مگویید،
\par 39 لهذا اینک من شما را بالکل فراموش خواهم کرد و شما را با آن شهری که به شما و به پدران داده بودم از حضور خود دور خواهم‌انداخت.و عار ابدی و رسوایی جاودانی را که فراموش نخواهد شد بر شما عارض خواهم گردانید.»
\par 40 و عار ابدی و رسوایی جاودانی را که فراموش نخواهد شد بر شما عارض خواهم گردانید.»
 
\chapter{24}

\par 1 و بعد از آنکه نبوکدرصر پادشاه بابل یکنیا ابن یهویاقیم پادشاه یهودا را باروسای یهودا و صنعتگران و آهنگران ازاورشلیم اسیر نموده، به بابل برد، خداوند دو سبدانجیر را که پیش هیکل خداوند گذاشته شده بودبه من نشان داد
\par 2 که در سبد اول، انجیر بسیار نیکومثل انجیر نوبر بود و در سبد دیگر انجیر بسیار بدبود که چنان زشت بود که نمی شود خورد.
\par 3 وخداوند مرا گفت: «ای ارمیا چه می‌بینی؟» گفتم: «انجیر. اما انجیرهای نیکو، بسیار نیکو است وانجیرهای بد بسیار بد است که از بدی آن رانمی توان خورد.»
\par 4 و کلام خداوند به من نازل شده، گفت:
\par 5 «یهوه خدای اسرائیل چنین می‌گوید: مثل این انجیرهای خوب همچنان اسیران یهودا را که ایشان را از اینجا به زمین کلدانیان برای نیکویی فرستادم منظور خواهم داشت.
\par 6 و چشمان خودرا بر ایشان به نیکویی خواهم‌انداخت و ایشان را به این زمین باز خواهم آورد و ایشان را بنا کرده، منهدم نخواهم ساخت و غرس نموده، ریشه ایشان را نخواهم کند.
\par 7 و دلی به ایشان خواهم بخشید تا مرا بشناسند که من یهوه هستم و ایشان قوم من خواهند بود و من خدای ایشان خواهم بود، زیرا که به تمامی دل بسوی من بازگشت خواهند نمود.»
\par 8 خداوند چنین می‌گوید: «مثل انجیرهای بدکه چنان بد است که نمی توان خورد، البته همچنان صدقیا پادشاه یهودا و روسای او و بقیه اورشلیم را که در این زمین باقی‌مانده‌اند و آنانی راکه در مصر ساکن‌اند تسلیم خواهم نمود.
\par 9 وایشان را در تمامی ممالک زمین مایه تشویش وبلا و در تمامی مکان هایی که ایشان را رانده‌ام عارو ضرب‌المثل و مسخره و لعنت خواهم ساخت.و در میان ایشان شمشیر و قحط و وبا خواهم فرستاد تا از زمینی که به ایشان و به پدران ایشان داده‌ام نابود شوند.»
\par 10 و در میان ایشان شمشیر و قحط و وبا خواهم فرستاد تا از زمینی که به ایشان و به پدران ایشان داده‌ام نابود شوند.»
 
\chapter{25}

\par 1 کلامی که در سال چهارم یهویاقیم بن یوشیا، پادشاه یهودا که سال اول نبوکدرصر پادشاه بابل بود بر ارمیا درباره تمامی قوم یهودا نازل شد.
\par 2 و ارمیای نبی تمامی قوم یهودا و جمیع سکنه اورشلیم را به آن خطاب کرده، گفت:
\par 3 «از سال سیزدهم یوشیا ابن آمون پادشاه یهودا تا امروز که بیست و سه سال باشدکلام خداوند بر من نازل می‌شد و من به شما سخن می‌گفتم و صبح زود برخاسته، تکلم می‌نمودم اما شما گوش نمی دادید.
\par 4 و خداوند جمیع بندگان خود انبیا را نزد شما فرستاد و صبح زودبرخاسته، ایشان را ارسال نمود اما نشنیدید وگوش خود را فرا نگرفتید تا استماع نمایید.
\par 5 وگفتند: هر یک از شما از راه بد خود و اعمال شریرخویش بازگشت نمایید و در زمینی که خداوند به شما و به پدران شما از ازل تا به ابد بخشیده است ساکن باشید.
\par 6 و از عقب خدایان غیر نروید و آنهارا عبادت و سجده منمایید و به اعمال دستهای خود غضب مرا به هیجان میاورید مبادا بر شما بلابرسانم.»
\par 7 اما خداوند می‌گوید: «مرا اطاعت ننمودید بلکه خشم مرا به اعمال دستهای خویش برای بلای خود به هیجان آوردید.»
\par 8 بنابراین یهوه صبایوت چنین می‌گوید: «چونکه کلام مرا نشنیدید،
\par 9 خداوند می‌گوید: اینک من فرستاده، تمامی قبایل شمال را با بنده خود نبوکدرصر پادشاه بابل گرفته، ایشان را بر این زمین و بر ساکنانش و بر همه امت هایی که به اطراف آن می‌باشند خواهم آورد و آنها را بالکل هلاک کرده، دهشت و مسخره و خرابی ابدی خواهم ساخت.
\par 10 و از میان ایشان آواز شادمانی و آواز خوشی و صدای داماد و صدای عروس وصدای آسیا و روشنایی چراغ را نابود خواهم گردانید.
\par 11 و تمامی این زمین خراب و ویران خواهد شد و این قوم‌ها هفتاد سال پادشاه بابل رابندگی خواهند نمود.»
\par 12 و خداوند می‌گوید که «بعد از انقضای هفتاد سال من بر پادشاه بابل و برآن امت و بر زمین کلدانیان عقوبت گناه ایشان راخواهم رسانید و آن را به خرابی ابدی مبدل خواهم ساخت.
\par 13 و بر این زمین تمامی سخنان خود را که به ضد آن گفته‌ام یعنی هر‌چه در این کتاب مکتوب است که ارمیا آن را درباره جمیع امت‌ها نبوت کرده است خواهم آورد.
\par 14 زیرا که امت های بسیار و پادشاهان عظیم ایشان را بنده خود خواهند ساخت و ایشان را موافق افعال ایشان و موافق اعمال دستهای ایشان مکافات خواهم رسانید.»
\par 15 زانرو که یهوه خدای اسرائیل به من چنین گفت که «کاسه شراب این غضب را از دست من بگیر و آن را به جمیع امت هایی که تو را نزد آنهامی فرستم بنوشان.
\par 16 تا بیاشامند و به‌سبب شمشیری که من در میان ایشان می‌فرستم نوان شوند و دیوانه گردند.»
\par 17 پس کاسه را از دست خداوند گرفتم و به جمیع امت هایی که خداوند مرا نزد آنها فرستادنوشانیدم.
\par 18 یعنی به اورشلیم و شهرهای یهوداو پادشاهانش و سرورانش تا آنها را خرابی ودهشت و سخریه و لعنت چنانکه امروز شده است گردانم.
\par 19 و به فرعون پادشاه مصر وبندگانش و سرورانش و تمامی قومش.
\par 20 و به جمیع امت های مختلف و به جمیع پادشاهان زمین عوص و به همه پادشاهان زمین فلسطینیان یعنی اشقلون و غزه و عقرون و بقیه اشدود.
\par 21 وبه ادوم و موآب و بنی عمون.
\par 22 و به جمیع پادشاهان صور و همه پادشاهان صیدون و به پادشاهان جزایری که به آن طرف دریا می‌باشند.
\par 23 و به ددان و تیما و بوز و به همگانی که گوشه های موی خود را می‌تراشند.
\par 24 و به همه پادشاهان عرب و به جمیع پادشاهان امت های مختلف که در بیابان ساکنند.
\par 25 و به جمیع پادشاهان زمری و همه پادشاهان عیلام و همه پادشاهان مادی.
\par 26 و به جمیع پادشاهان شمال خواه قریب و خواه بعید هر یک با مجاور خود وبه تمامی ممالک جهان که بر روی زمینند. وپادشاه شیشک بعد از ایشان خواهد آشامید.
\par 27 و به ایشان بگو: یهوه صبایوت خدای اسرائیل چنین می‌فرماید: «بنوشید و مست شویدو قی کنید تا از شمشیری که من در میان شمامی فرستم بیفتید و برنخیزید.
\par 28 و اگر از گرفتن کاسه از دست تو و نوشیدنش ابا نمایند آنگاه به ایشان بگو: یهوه صبایوت چنین می‌گوید: البته خواهید نوشید.
\par 29 زیرا اینک من به رسانیدن بلا براین شهری که به اسم من مسمی است شروع خواهم نمود و آیا شما بالکل بی‌عقوبت خواهیدماند؟ بی‌عقوبت نخواهید ماند زیرا یهوه صبایوت می‌گوید که من شمشیری بر جمیع ساکنان جهان مامور می‌کنم.
\par 30 پس تو به تمامی این سخنان برایشان نبوت کرده، به ایشان بگو: خداوند از اعلی علیین غرش می‌نماید و از مکان قدس خویش آواز خود را می‌دهد و به ضد مرتع خویش به شدت غرش می‌نماید و مثل آنانی که انگور رامی افشرند، بر تمامی ساکنان جهان نعره می‌زند،
\par 31 و صدا به کرانهای زمین خواهد رسید زیراخداوند را با امت‌ها دعوی است و او بر هرذی جسد داوری خواهد نمود و شریران را به شمشیر تسلیم خواهد کرد.» قول خداوند این است.
\par 32 یهوه صبایوت چنین گفت: «اینک بلا ازامت به امت سرایت می‌کند و باد شدید عظیمی ازکرانهای زمین برانگیخته خواهد شد.»
\par 33 و در آن روز کشتگان خداوند از کران زمین تا کران دیگرش خواهند بود. برای ایشان ماتم نخواهندگرفت و ایشان را جمع نخواهند کرد و دفن نخواهند نمود بلکه بر روی زمین سرگین خواهندبود.
\par 34 ‌ای شبانان ولوله نمایید و فریاد برآورید. وای روسای گله بغلطید زیرا که ایام کشته شدن شما رسیده است و من شما را پراکنده خواهم ساخت و مثل ظرف مرغوب خواهید افتاد.
\par 35 وملجا برای شبانان و مفر برای روسای گله نخواهدبود.
\par 36 هین فریاد شبانان و نعره روسای گله! زیراخداوند مرتعهای ایشان را ویران ساخته است.
\par 37 و مرتعهای سلامتی به‌سبب حدت خشم خداوند خراب شده است.مثل شیر بیشه خودرا ترک کرده است زیرا که زمین ایشان به‌سبب خشم هلاک کننده و به‌سبب حدت غضبش ویران شده است.
\par 38 مثل شیر بیشه خودرا ترک کرده است زیرا که زمین ایشان به‌سبب خشم هلاک کننده و به‌سبب حدت غضبش ویران شده است.
 
\chapter{26}

\par 1 در ابتدای سلطنت یهویاقیم بن یوشیاپادشاه یهودا این کلام از جانب خداوندنازل شده، گفت:
\par 2 «خداوند چنین می‌گوید: درصحن خانه خداوند بایست و به ضد تمامی شهرهای یهودا که به خانه خداوند برای عبادت می‌آیند همه سخنانی را که تو را امر فرمودم که به ایشان بگویی بگو و سخنی کم مکن.
\par 3 شایدبشنوند و هر کس از راه بد خویش برگردد تا ازبلایی که من قصد نموده‌ام که به‌سبب اعمال بدایشان به ایشان برسانم پشیمان گردم.
\par 4 پس ایشان را بگو: خداوند چنین می‌فرماید: اگر به من گوش ندهید و در شریعت من که پیش شما نهاده‌ام سلوک ننمایید،
\par 5 و اگر کلام بندگانم انبیا را که من ایشان را نزد شما فرستادم اطاعت ننمایید با آنکه من صبح زود برخاسته، ایشان را ارسال نمودم اماشما گوش نگرفتید.
\par 6 آنگاه این خانه را مثل شیلوه خواهم ساخت و این شهر را برای جمیع امت های زمین لعنت خواهم گردانید.»
\par 7 و کاهنان و انبیا و تمامی قوم، این سخنان راکه ارمیا در خانه خداوند گفت شنیدند.
\par 8 و چون ارمیا از گفتن هر‌آنچه خداوند او را مامور فرموده بود که به تمامی قوم بگوید فارغ شد، کاهنان وانبیا و تمامی قوم او را گرفته، گفتند: «البته خواهی مرد.
\par 9 چرا به اسم یهوه نبوت کرده، گفتی که این خانه مثل شیلوه خواهد شد و این شهر خراب وغیرمسکون خواهد گردید؟» پس تمامی قوم درخانه خداوند نزد ارمیا جمع شدند.
\par 10 و چون روسای یهودا این چیزها را شنیدنداز خانه پادشاه به خانه خداوند برآمده، به دهنه دروازه جدید خانه خداوند نشستند.
\par 11 پس کاهنان و انبیا، روسا وتمامی قوم را خطاب کرده، گفتند: «این شخص مستوجب قتل است زیراچنانکه به گوشهای خود شنیدید به خلاف این شهر نبوت کرد.»
\par 12 پس ارمیا جمیع سروران و تمامی قوم رامخاطب ساخته، گفت: «خداوند مرا فرستاده است تا همه سخنانی را که شنیدید به ضد این خانه و به ضد این شهر نبوت نمایم.
\par 13 پس الان راهها و اعمال خود را اصلاح نمایید و قول یهوه خدای خود را بشنوید تا خداوند از این بلایی که درباره شما فرموده است پشیمان شود.
\par 14 اما من اینک در دست شما هستم موافق آنچه در نظرشما پسند و صواب آید بعمل آرید.
\par 15 لیکن اگر شما مرا به قتل رسانید یقین بدانید که خون بی‌گناهی را بر خویشتن و بر این شهر و ساکنانش وارد خواهید آورد. زیرا حقیقت خداوند مرا نزدشما فرستاده است تا همه این سخنان را به گوش شما برسانم.»
\par 16 آنگاه روسا و تمامی قوم به کاهنان و انبیاگفتند که «این مرد مستوجب قتل نیست زیرا به اسم یهوه خدای ما به ما سخن گفته است.»
\par 17 وبعضی از مشایخ زمین برخاسته، تمامی جماعت قوم را خطاب کرده، گفتند 
\par 18 که «میکای مورشتی در ایام حزقیا پادشاه یهودا نبوت کرد وبه تمامی قوم یهودا تکلم نموده، گفت: یهوه صبایوت چنین می‌گوید که صهیون را مثل مزرعه شیار خواهند کرد و اورشلیم خراب شده، کوه این خانه به بلندیهای جنگل مبدل خواهد گردید.
\par 19 آیا حزقیا پادشاه یهودا و تمامی یهودا او راکشتند؟ نی بلکه از خداوند بترسید و نزد خداونداستدعا نمود و خداوند از آن بلایی که درباره ایشان گفته بود پشیمان گردید. پس ما بلای عظیمی بر جان خود وارد خواهیم آورد.»
\par 20 و نیز شخصی اوریا نام ابن شمعیا از قریت یعاریم بود که به نام یهوه نبوت کرد و او به ضد این شهر و این زمین موافق همه سخنان ارمیا نبوت کرد.
\par 21 و چون یهویاقیم پادشاه و جمیع شجاعانش و تمامی سرورانش سخنان او راشنیدند پادشاه قصد جان او نمود و چون اوریا این را شنید بترسید و فرار کرده، به مصر رفت.
\par 22 ویهویاقیم پادشاه کسان به مصر فرستاد یعنی الناتان بن عکبور و چند نفر را با او به مصر(فرستاد).
\par 23 و ایشان اوریا را از مصر بیرون آورده، او را نزد یهویاقیم پادشاه رسانیدند و او رابه شمشیر کشته، بدن او را به قبرستان عوام الناس انداخت.لیکن دست اخیقام بن شافان با ارمیابود تا او را به‌دست قوم نسپارند که او را به قتل رسانند.
\par 24 لیکن دست اخیقام بن شافان با ارمیابود تا او را به‌دست قوم نسپارند که او را به قتل رسانند.
 
\chapter{27}

\par 1 در ابتدای سلطنت یهویاقیم بن یوشیاپادشاه یهودا این کلام از جانب خداوندبر ارمیا نازل شده، گفت:
\par 2 خداوند به من چنین گفت: «بندها و یوغها برای خود بساز و آنها را برگردن خود بگذار.
\par 3 و آنها را نزد پادشاه ادوم وپادشاه موآب و پادشاه بنی عمون و پادشاه صور وپادشاه صیدون به‌دست رسولانی که به اورشلیم نزد صدقیا پادشاه یهودا خواهند آمد بفرست.
\par 4 وایشان را برای آقایان ایشان امر فرموده، بگو یهوه صبایوت خدای اسرائیل چنین می‌گوید: به آقایان خود بدین مضمون بگویید:
\par 5 من جهان وانسان و حیوانات را که بر روی زمینند به قوت عظیم و بازوی افراشته خود آفریدم و آن را بهر‌که در نظر من پسند آمد بخشیدم.
\par 6 و الان من تمامی این زمینها را به‌دست بنده خود نبوکدنصر پادشاه بابل دادم و نیز حیوانات صحرا را به او بخشیدم تااو را بندگی نمایند.
\par 7 و تمامی امت‌ها او را وپسرش و پسر پسرش را خدمت خواهند نمود تاوقتی که نوبت زمین او نیز برسد. پس امت های بسیار و پادشاهان عظیم او را بنده خود خواهندساخت.
\par 8 و واقع خواهد شد که هر امتی ومملکتی که نبوکدنصر پادشاه بابل را خدمت ننمایند و گردن خویش را زیر یوغ پادشاه بابل نگذارند خداوند می‌گوید: که آن امت را به شمشیر و قحط و وبا سزا خواهم داد تا ایشان را به‌دست او هلاک کرده باشم.
\par 9 و اما شما به انبیا وفالگیران و خواب بینندگان و ساحران و جادوگران خود که به شما حرف می‌زنند و می‌گویند پادشاه بابل را خدمت منمایید گوش مگیرید.
\par 10 زیرا که ایشان برای شما کاذبانه نبوت می‌کنند تا شما را اززمین شما دور نمایند و من شما را پراکنده سازم تاهلاک شوید.
\par 11 اما آن امتی که گردن خود را زیریوغ پادشاه بابل بگذارند و او را خدمت نمایندخداوند می‌گوید که آن امت را در زمین خودایشان مقیم خواهم ساخت و آن را زرع نموده، درآن ساکن خواهند شد.»
\par 12 و به صدقیا پادشاه یهودا همه این سخنان رابیان کرده، گفتم: «گردنهای خود را زیر یوغ پادشاه بابل بگذارید و او را و قوم او را خدمت نمایید تازنده بمانید.
\par 13 چرا تو و قومت به شمشیر و قحطو وبا بمیرید چنانکه خداوند درباره قومی که پادشاه بابل را خدمت ننمایند گفته است.
\par 14 وگوش مگیرید به سخنان انبیایی که به شمامی گویند: پادشاه بابل را خدمت ننمایید زیرا که ایشان برای شما کاذبانه نبوت می‌کنند.
\par 15 زیراخداوند می‌گوید: من ایشان را نفرستادم بلکه ایشان به اسم من به دروغ نبوت می‌کنند تا من شمارا اخراج کنم و شما با انبیایی که برای شما نبوت می‌نمایند هلاک شوید.»
\par 16 و به کاهنان و تمامی این قوم نیز خطاب کرده، گفتم: «خداوند چنین می‌گوید: گوش مگیرید به سخنان انبیایی که برای شما نبوت کرده، می‌گویند اینک ظروف خانه خداوند بعد از اندک مدتی از بابل باز آورده خواهد شد زیرا که ایشان کاذبانه برای شما نبوت می‌کنند.
\par 17 ایشان را گوش مگیریدبلکه پادشاه بابل را خدمت نمایید تا زنده بمانید. چرا این شهر خراب شود؟
\par 18 و اگر ایشان انبیا می‌باشند و کلام خداوند باایشان است پس الان از یهوه صبایوت استدعابکنند تا ظروفی که در خانه خداوند و در خانه پادشاه یهودا و اورشلیم باقی است به بابل برده نشود.
\par 19 زیرا که یهوه صبایوت چنین می‌گوید: درباره ستونها و دریاچه و پایه هاو سایر ظروفی که در این شهر باقی‌مانده است،
\par 20 و نبوکدنصرپادشاه بابل آنها را حینی که یکنیا ابن یهویاقیم پادشاه یهودا و جمیع شرفا یهودا و اورشلیم را ازاورشلیم به بابل برد نگرفت.
\par 21 به درستی که یهوه صبایوت خدای اسرائیل درباره این ظروفی که درخانه خداوند و در خانه پادشاه یهودا و اورشلیم باقی‌مانده است چنین می‌گوید:که آنها به بابل برده خواهد شد و خداوند می‌گوید تا روزی که ازایشان تفقد نمایم در آنجا خواهد ماند و بعد از آن آنها را بیرون آورده، به این مکان باز خواهم آورد.»
\par 22 که آنها به بابل برده خواهد شد و خداوند می‌گوید تا روزی که ازایشان تفقد نمایم در آنجا خواهد ماند و بعد از آن آنها را بیرون آورده، به این مکان باز خواهم آورد.»
 
\chapter{28}

\par 1 و در همان سال در ابتدای سلطنت صدقیا پادشاه یهودا در ماه پنجم ازسال چهارم واقع شد که حننیا ابن عزور نبی که ازجبعون بود مرا در خانه خداوند در حضور کاهنان و تمامی قوم خطاب کرده، گفت:
\par 2 «یهوه صبایوت خدای اسرائیل بدین مضمون تکلم نموده و گفته است من یوغ پادشاه بابل را شکسته‌ام.
\par 3 بعد از انقضای دو سال من همه ظرف های خانه خداوند را که نبوکدنصر پادشاه بابل از این مکان گرفته، به بابل برد به اینجا بازخواهم آورد.
\par 4 و خداوند می‌گوید من یکنیا ابن یهویاقیم پادشاه یهودا و جمیع اسیران یهودا را که به بابل رفته‌اند به اینجا باز خواهم آورد زیرا که یوغ پادشاه بابل را خواهم شکست.»
\par 5 آنگاه ارمیا نبی به حننیا نبی در حضورکاهنان و تمامی قومی که در خانه خداوند حاضربودند گفت؛
\par 6 پس ارمیا نبی گفت: «آمین خداوندچنین بکند و خداوند سخنانت را که به آنها نبوت کردی استوار نماید و ظروف خانه خداوند وجمیع اسیران را از بابل به اینجا باز بیاورد.
\par 7 لیکن این کلام را که من به گوش تو و به سمع تمامی قوم می‌گویم بشنو:
\par 8 انبیایی که از زمان قدیم قبل ازمن و قبل از تو بوده‌اند درباره زمینهای بسیاروممالک عظیم به جنگ و بلا و وبا نبوت کرده‌اند.
\par 9 اما آن نبی‌ای که بسلامتی نبوت کند اگر کلام آن نبی واقع گردد، آنگاه آن نبی معروف خواهد شدکه خداوند فی الحقیقه او را فرستاده است.»
\par 10 پس حننیا نبی یوغ را از گردن ارمیا نبی گرفته، آن را شکست.
\par 11 و حننیا به حضور تمامی قوم خطاب کرده، گفت: «خداوند چنین می‌گوید: بهمین طور یوغ نبوکدنصر پادشاه بابل را بعد ازانقضای دو سال از گردن جمیع امت‌ها خواهم شکست.» و ارمیا نبی به راه خود رفت.
\par 12 و بعد از آنکه حننیا نبی یوغ را از گردن ارمیانبی شکسته بود کلام خداوند بر ارمیا نازل شده، گفت:
\par 13 «برو و حننیا نبی را بگو: خداوند چنین می‌گوید: یوغهای چوبی را شکستی اما بجای آنها یوغهای آهنین را خواهی ساخت.
\par 14 زیرا که یهوه صبایوت خدای اسرائیل چنین می‌گوید: من یوغی آهنین بر گردن جمیع این امت‌ها نهادم تانبوکدنصر پادشاه بابل را خدمت نمایند پس او راخدمت خواهند نمود و نیز حیوانات صحرا را به او دادم.»
\par 15 آنگاه ارمیا نبی به حننیا نبی گفت: «ای حننیا بشنو! خداوند تو را نفرستاده است بلکه تواین قوم را وامیداری که به دروغ توکل نمایند.
\par 16 بنابراین خداوند چنین می‌گوید: اینک من تو رااز روی این زمین دور می‌اندازم و تو امسال خواهی مرد زیرا که سخنان فتنه انگیز به ضدخداوند گفتی.»پس در ماه هفتم همانسال حننیا نبی مرد.
\par 17 پس در ماه هفتم همانسال حننیا نبی مرد.
 
\chapter{29}

\par 1 این است سخنان رساله‌ای که ارمیانبی از اورشلیم نزد بقیه مشایخ اسیران وکاهنان و انبیا و تمامی قومی که نبوکدنصر ازاورشلیم به بابل به اسیری برده بود فرستاد.
\par 2 بعداز آنکه یکنیا پادشاه و ملکه و خواجه‌سرایان وسروران یهودا و اورشلیم و صنعتگران و آهنگران از اورشلیم بیرون رفته بودند.
\par 3 (پس آن را) به‌دست العاسه بن شافان و جمریا ابن حلقیا که صدقیا پادشاه یهودا ایشان را نزد نبوکدنصرپادشاه بابل به بابل فرستاد (ارسال نموده )، گفت:
\par 4 «یهوه صبایوت خدای اسرائیل به تمامی اسیرانی که من ایشان را از اورشلیم به بابل به اسیری فرستادم، چنین می‌گوید:
\par 5 خانه‌ها ساخته در آنها ساکن شوید و باغها غرس نموده، میوه آنها را بخورید.
\par 6 زنان گرفته، پسران و دختران به هم رسانید و زنان برای پسران خود بگیرید ودختران خود را به شوهر بدهید تا پسران ودختران بزایند و در آنجا زیاد شوید و کم نگردید.
\par 7 و سلامتی آن شهر را که شما را به آن به اسیری فرستاده‌ام بطلبید و برایش نزد خداوند مسالت نمایید زیرا که در سلامتی آن شما را سلامتی خواهد بود.
\par 8 زیرا که یهوه صبایوت خدای اسرائیل چنین می‌گوید: مگذارید که انبیای شماکه در میان شمااند و فالگیران شما شما را فریب دهند و به خوابهایی که شما ایشان را وامی داریدکه آنها را ببینند، گوش مگیرید.
\par 9 زیرا خداوندمی گوید که ایشان برای شما به اسم من کاذبانه نبوت می‌کنند و من ایشان را نفرستاده‌ام.
\par 10 وخداوند می‌گوید: چون مدت هفتاد سال بابل سپری شود من از شما تفقد خواهم نمود وسخنان نیکو را که برای شما گفتم انجام خواهم داد؛ به اینکه شما را به این مکان باز خواهم آورد.
\par 11 زیرا خداوند می‌گوید: فکرهایی را که برای شما دارم می‌دانم که فکرهای سلامتی می‌باشد ونه بدی تا شما را در آخرت امید بخشم.
\par 12 و مراخواهید خواند و آمده، نزد من تضرع خواهید کردو من شما را اجابت خواهم نمود.
\par 13 و مراخواهید طلبید و چون مرا به تمامی دل خودجستجو نمایید، مرا خواهید یافت.
\par 14 و خداوندمی گوید که مرا خواهید یافت و اسیران شما را بازخواهم آورد. و خداوند می‌گوید که شما را ازجمیع امت‌ها و از همه مکان هایی که شما را درآنها رانده‌ام، جمع خواهم نمود و شما را از جایی که به اسیری فرستاده‌ام، باز خواهم آورد.
\par 15 از آن رو که گفتید خداوند برای ما در بابل انبیا مبعوث نموده است.
\par 16 «پس خداوند به پادشاهی که بر کرسی داودنشسته است و به تمامی قومی که در این شهرساکنند، یعنی برادران شما که همراه شما به اسیری نرفته‌اند، چنین می‌گوید:
\par 17 بلی یهوه صبایوت چنین می‌گوید: اینک من شمشیر وقحط و وبا را بر ایشان خواهم فرستاد و ایشان رامثل انجیرهای بد که آنها را از بدی نتوان خورد، خواهم ساخت.
\par 18 و ایشان را به شمشیر و قحط ووبا تعاقب خواهم نمود و در میان جمیع ممالک جهان مشوش خواهم ساخت تا برای همه امت هایی که ایشان را در میان آنها رانده‌ام، لعنت و دهشت و مسخره و عار باشند.
\par 19 چونکه خداوند می‌گوید: کلام مرا که به واسطه بندگان خود انبیا نزد ایشان فرستادم نشنیدند با آنکه صبح زود برخاسته، آن را فرستادم اما خداوند می‌گویدکه شما نشنیدید.
\par 20 و شما‌ای جمیع اسیرانی که از اورشلیم به بابل فرستادم کلام خداوند رابشنوید.
\par 21 یهوه صبایوت خدای اسرائیل درباره اخاب بن قولایا و درباره صدقیا ابن معسیا که برای شما به اسم من کاذبانه نبوت می‌کنند، چنین می‌گوید: اینک من ایشان را به‌دست نبوکدنصرپادشاه بابل تسلیم خواهم کرد و او ایشان را درحضور شما خواهد کشت.
\par 22 و از ایشان برای تمامی اسیران یهودا که در بابل می‌باشند لعنت گرفته، خواهند گفت که خداوند تو را مثل صدقیاو اخاب که پادشاه بابل ایشان را در آتش سوزانید، بگرداند. 
\par 23 چونکه ایشان در اسرائیل حماقت نمودند و با زنان همسایگان خود زنا کردند و به اسم من کلامی را که به ایشان امرنفرموده بودم کاذبانه گفتند و خداوند می‌گوید که من عارف و شاهد هستم.
\par 24 «و شمعیای نحلامی را خطاب کرده، بگو:
\par 25 یهوه صبایوت خدای اسرائیل تکلم نموده، چنین می‌گوید: از آنجایی که تو رسایل به اسم خود نزد تمامی قوم که در اورشلیم‌اند و نزدصفنیا ابن معسیا کاهن و نزد جمیع کاهنان فرستاده، گفتی:
\par 26 که خداوند تو را به‌جای یهویاداع کاهن به کهانت نصب نموده است تا برخانه خداوند وکلا باشید. برای هر شخص مجنون که خویشتن را نبی می‌نماید تا او را درکنده‌ها و زنجیرها ببندی.
\par 27 پس الان چرا ارمیاعناتوتی را که خود را برای شما نبی می‌نمایدتوبیخ نمی کنی؟
\par 28 زیرا که او نزد ما به بابل فرستاده، گفت که این اسیری بطول خواهدانجامید پس خانه‌ها بنا کرده، ساکن شوید و باغهاغرس نموده، میوه آنها را بخورید.»
\par 29 و صفنیای کاهن این رساله را به گوش ارمیانبی خواند.
\par 30 پس کلام خداوند بر ارمیا نازل شده، گفت:
\par 31 «نزد جمیع اسیران فرستاده، بگوکه خداوند درباره شمعیای نحلامی چنین می‌گوید: چونکه شمعیا برای شما نبوت می‌کند ومن او را نفرستاده‌ام و او شما را وامیدارد که به دروغ اعتماد نمایید،بنابراین خداوند چنین می‌گوید: اینک من بر شمعیای نحلامی و ذریت وی عقوبت خواهم رسانید و برایش کسی‌که درمیان این قوم ساکن باشد، نخواهد ماند و خداوندمی گوید او آن احسانی را که من برای قوم خودمی کنم نخواهد دید، زیرا که درباره خداوند سخنان فتنه انگیز گفته است.»
\par 32 بنابراین خداوند چنین می‌گوید: اینک من بر شمعیای نحلامی و ذریت وی عقوبت خواهم رسانید و برایش کسی‌که درمیان این قوم ساکن باشد، نخواهد ماند و خداوندمی گوید او آن احسانی را که من برای قوم خودمی کنم نخواهد دید، زیرا که درباره خداوند سخنان فتنه انگیز گفته است.»
 
\chapter{30}

\par 1 کلامی که از جانب خداوند بر ارمیا نازل شده، گفت:
\par 2 «یهوه خدای اسرائیل تکلم نموده، چنین می‌گوید: تمامی سخنانی را که من به تو گفته‌ام، در طوماری بنویس.
\par 3 زیراخداوند می‌گوید: اینک ایامی می‌آید که اسیران قوم خود اسرائیل و یهودا را باز خواهم آورد وخداوند می‌گوید: ایشان را به زمینی که به پدران ایشان داده‌ام، باز خواهم رسانید تا آن را به تصرف آورند.»
\par 4 و این است کلامی که خداوند درباره اسرائیل و یهودا گفته است.
\par 5 زیرا خداوند چنین می‌گوید: «صدای ارتعاش شنیدیم. خوف است وسلامتی نی.
\par 6 سوال کنید و ملاحظه نمایید که آیاذکور اولاد می‌زاید؟ پس چرا هر مرد را می‌بینم که مثل زنی که می‌زاید دست خود را بر کمرش نهاده و همه چهره‌ها به زردی مبدل شده است؟»
\par 7 وای بر ما زیرا که آن روز، عظیم است و مثل آن دیگری نیست و آن زمان تنگی یعقوب است اما از آن نجات خواهد یافت.
\par 8 و یهوه صبایوت می‌گوید: «هر آینه در آن روز یوغ او را از گردنت خواهم شکست وبندهای تو را خواهم گسیخت. و غریبان بار دیگراو را بنده خود نخواهند ساخت.
\par 9 و ایشان خدای خود یهوه و پادشاه خویش داود را که برای ایشان برمی انگیزانم خدمت خواهند کرد.
\par 10 پس خداوند می‌گوید که‌ای بنده من یعقوب مترس وای اسرائیل هراسان مباش زیرا اینک من تو را از جای دور و ذریت تو را از زمین اسیری ایشان خواهم رهانید و یعقوب مراجعت نموده، دررفاهیت و امنیت خواهد بود و کسی او را نخواهدترسانید.
\par 11 زیرا خداوند می‌گوید: من با تو هستم تا تو را نجات‌بخشم و جمیع امت‌ها را که تو را درمیان آنها پراکنده ساختم، تلف خواهم کرد. اما تورا تلف نخواهم نمود، بلکه تو را به انصاف تادیب خواهم کرد و تو را بی‌سزا نخواهم گذاشت.
\par 12 زیرا خداوند چنین می‌گوید: جراحت توعلاج ناپذیر و ضربت تو مهلک می‌باشد.
\par 13 کسی نیست که دعوی تو را فیصل دهد تا التیام یابی وبرایت دواهای شفابخشنده‌ای نیست.
\par 14 جمیع دوستانت تو را فراموش کرده، درباره تواحوال پرسی نمی نمایند زیرا که تو را به صدمه دشمن و به تادیب بیرحمی به‌سبب کثرت عصیانت و زیادتی گناهانت مبتلا ساخته‌ام.
\par 15 چرا درباره جراحت خود فریاد می‌نمایی؟ دردتو علاج ناپذیر است. به‌سبب زیادتی عصیانت وکثرت گناهانت این کارها را به تو کرده‌ام.
\par 16 «بنابراین آنانی که تو را می‌بلعند، بلعیده خواهند شد و آنانی که تو را به تنگ می‌آورند، جمیع به اسیری خواهند رفت. و آنانی که تو راتاراج می‌کنند، تاراج خواهند شد و همه غارت کنندگانت را به غارت تسلیم خواهم کرد.
\par 17 زیرا خداوند می‌گوید: به تو عافیت خواهم رسانید و جراحات تو را شفا خواهم داد، از این جهت که تو را (شهر) متروک می‌نامند (ومی گویند) که این صهیون است که احدی درباره آن احوال پرسی نمی کند.
\par 18 خداوند چنین می‌گوید: اینک خیمه های اسیری یعقوب را بازخواهم آورد و به مسکنهایش ترحم خواهم نمودو شهر بر تلش بنا شده، قصرش برحسب عادت خود مسکون خواهد شد.
\par 19 و تسبیح و آوازمطربان از آنها بیرون خواهد آمد و ایشان راخواهم افزود و کم نخواهند شد و ایشان را معززخواهم ساخت و پست نخواهند گردید.
\par 20 وپسرانش مانند ایام پیشین شده، جماعتش درحضور من برقرار خواهند ماند و بر جمیع ستمگرانش عقوبت خواهم رسانید.
\par 21 و حاکم ایشان از خود ایشان بوده، سلطان ایشان از میان ایشان بیرون خواهد آمد و او را مقرب می‌گردانم تا نزدیک من بیاید زیرا خداوند می‌گوید: کیست که جرات کند نزد من آید؟
\par 22 و شما قوم من خواهید بود و من خدای شما خواهم بود.
\par 23 اینک باد شدید خداوند با حدت غضب وگردبادهای سخت بیرون می‌آید که بر سر شریران هجوم آورد.تا خداوند تدبیرات دل خود رابجا نیاورد و استوار نفرماید، حدت خشم اونخواهد برگشت. در ایام آخر این را خواهیدفهمید.»
\par 24 تا خداوند تدبیرات دل خود رابجا نیاورد و استوار نفرماید، حدت خشم اونخواهد برگشت. در ایام آخر این را خواهیدفهمید.»
 
\chapter{31}

\par 1 خداوند می‌گوید: «در آن زمان من خدای تمامی قبایل اسرائیل خواهم بود و ایشان قوم من خواهند بود.
\par 2 خداوند چنین می‌گوید: قومی که از شمشیر رستند در بیابان فیض یافتند، هنگامی که من رفتم تا برای اسرائیل آرامگاهی پیدا کنم.»
\par 3 خداوند از جای دور به من ظاهر شد (وگفت ): «با محبت ازلی تو را دوست داشتم، از این جهت تو را به رحمت جذب نمودم.
\par 4 ‌ای باکره اسرائیل تو را بار دیگر بنا خواهم کرد و تو بناخواهی شد و بار دیگر با دفهای خود خویشتن راخواهی آراست و در رقصهای مطربان بیرون خواهی آمد.
\par 5 بار دیگر تاکستانها بر کوههای سامره غرس خواهی نمود و غرس کنندگان غرس نموده، میوه آنها را خواهند خورد.
\par 6 زیرا روزی خواهد بود که دیده بانان بر کوهستان افرایم نداخواهند کرد که برخیزید و نزد یهوه خدای خودبه صهیون برآییم.»
\par 7 زیرا خداوند چنین می‌گوید: «به جهت یعقوب به شادمانی ترنم نمایید و به جهت سر امت‌ها آواز شادی دهید. اعلام نماییدو تسبیح بخوانید و بگویید: ای خداوند قوم خودبقیه اسرائیل را نجات بده!
\par 8 اینک من ایشان را اززمین شمال خواهم آورد و از کرانه های زمین جمع خواهم نمود و با ایشان کوران و لنگان وآبستنان و زنانی که می‌زایند با هم گروه عظیمی به اینجا باز خواهند آمد.
\par 9 با گریه خواهند آمد و من ایشان را با تضرعات خواهم آورد. نزد نهرهای آب ایشان را به راه صاف که در آن نخواهند لغزیدرهبری خواهم نمود زیرا که من پدر اسرائیل هستم و افرایم نخست زاده من است.
\par 10 ‌ای امت‌ها کلام خداوند را بشنوید و در میان جزایربعیده اخبار نمایید و بگویید آنکه اسرائیل راپراکنده ساخت ایشان را جمع خواهد نمود وچنانکه شبان گله خود را (نگاه دارد) ایشان رامحافظت خواهد نمود.
\par 11 زیرا خداوند یعقوب را فدیه داده و او را از دست کسی‌که از او قوی تربود رهانیده است.
\par 12 و ایشان آمده، بر بلندی صهیون خواهند سرایید و نزد احسان خداوندیعنی نزد غله و شیره و روغن و نتاج گله و رمه روان خواهند شد و جان ایشان مثل باغ سیرآب خواهد شد و بار دیگر هرگز غمگین نخواهندگشت.
\par 13 آنگاه باکره‌ها به رقص شادی خواهندکرد و جوانان و پیران با یکدیگر. زیرا که من ماتم ایشان را به شادمانی مبدل خواهم کرد و ایشان را از المی که کشیده‌اند تسلی داده، فرحناک خواهم گردانید.»
\par 14 و خداوند می‌گوید: «جان کاهنان رااز پیه تر و تازه خواهم ساخت و قوم من از احسان من سیر خواهند شد.»
\par 15 خداوند چنین می‌گوید: «آوازی در رامه شنیده شد ماتم و گریه بسیار تلخ که راحیل برای فرزندان خود گریه می‌کند و برای فرزندان خود تسلی نمی پذیرد زیرا که نیستند.»
\par 16 خداوند چنین می‌گوید: «آواز خود را ازگریه و چشمان خویش را از اشک باز دار. زیراخداوند می‌فرماید که برای اعمال خود اجرت خواهی گرفت و ایشان از زمین دشمنان مراجعت خواهند نمود.
\par 17 و خداوند می‌گوید که به جهت عاقبت تو امید هست و فرزندانت به حدودخویش خواهند برگشت.
\par 18 به تحقیق افرایم راشنیدم که برای خود ماتم گرفته، می‌گفت: مراتنبیه نمودی و متنبه شدم مثل گوساله‌ای که کارآزموده نشده باشد. مرا برگردان تا برگردانیده شوم زیرا که تو یهوه خدای من هستی.
\par 19 به درستی که بعد از آنکه برگردانیده شدم پشیمان گشتم و بعد از آنکه تعلیم یافتم بر ران خود زدم. خجل شدم و رسوایی هم کشیدم چونکه عارجوانی خویش را متحمل گردیدم.
\par 20 آیا افرایم پسر عزیز من یا ولد ابتهاج من است؟ زیرا هر گاه به ضد او سخن می‌گویم او را تا بحال به یادمی آورم. بنابراین خداوند می‌گوید که احشای من برای او به حرکت می‌آید و هر آینه بر او ترحم خواهم نمود.
\par 21 نشانه‌ها برای خود نصب نما وعلامت‌ها به جهت خویشتن برپا کن و دل خود رابسوی شاه راه به راهی که رفته‌ای متوجه ساز. ای باکره اسرائیل برگرد و به این شهرهای خود مراجعت نما.
\par 22 ‌ای دختر مرتد تا به کی به اینطرف و به آنطرف گردش خواهی نمود؟ زیراخداوند امر تازه‌ای در جهان ابداع نموده است که زن مرد را احاطه خواهد کرد.»
\par 23 یهوه صبایوت خدای اسرائیل چنین می‌گوید: «بار دیگر هنگامی که اسیران ایشان رابرمی گردانم، این کلام را در زمین یهودا وشهرهایش خواهند گفت که‌ای مسکن عدالت وای کوه قدوسیت، خداوند تو را مبارک سازد.
\par 24 ویهودا و تمامی شهرهایش با هم و فلاحان و آنانی که با گله‌ها گردش می‌کنند، در آن ساکن خواهندشد.
\par 25 زیرا که جان خستگان را تازه ساخته‌ام وجان همه محزونان را سیر کرده‌ام.»
\par 26 در این حال بیدار شدم و نگریستم و خوابم برای من شیرین بود.
\par 27 اینک خداوند می‌گوید: «ایامی می‌آید که خاندان اسرائیل و خاندان یهودا را به بذر انسان وبذر حیوان خواهم کاشت.
\par 28 و واقع خواهد شدچنانکه بر ایشان برای کندن و خراب نمودن ومنهدم ساختن و هلاک کردن و بلا رسانیدن مراقبت نمودم، به همینطور خداوند می‌گوید برایشان برای بنا نمودن و غرس کردن مراقب خواهم شد.
\par 29 و در آن ایام بار دیگر نخواهندگفت که پدران انگور ترش خوردند و دندان پسران کند گردید.
\par 30 بلکه هر کس به گناه خودخواهد مرد و هر‌که انگور ترش خورد دندان وی کند خواهد شد.»
\par 31 خداوند می‌گوید: «اینک ایامی می‌آید که باخاندان اسرائیل و خاندان یهودا عهد تازه‌ای خواهم بست.
\par 32 نه مثل آن عهدی که با پدران ایشان بستم در روزی که ایشان را دستگیری نمودم تا از زمین مصر بیرون آورم زیرا که ایشان عهد مرا شکستند، با آنکه خداوند می‌گوید من شوهر ایشان بودم.»
\par 33 اما خداوند می‌گوید: «اینست عهدی که بعد از این ایام با خاندان اسرائیل خواهم بست. شریعت خود را در باطن ایشان خواهم نهاد و آن را بر دل ایشان خواهم نوشت و من خدای ایشان خواهم بود و ایشان قوم من خواهند بود.
\par 34 و بار دیگر کسی به همسایه‌اش و شخصی به برادرش تعلیم نخواهدداد و نخواهد گفت خداوند را بشناس. زیراخداوند می‌گوید: جمیع ایشان از خرد و بزرگ مرا خواهند شناخت، چونکه عصیان ایشان راخواهم آمرزید و گناه ایشان را دیگر به یادنخواهم آورد.» 
\par 35 خداوند که آفتاب را به جهت روشنایی روز و قانونهای ماه و ستارگان را برای روشنایی شب قرار داده است و دریا را به حرکت می‌آورد تاامواجش خروش نمایند و اسم او یهوه صبایوت می‌باشد، چنین می‌گوید.
\par 36 پس خداوندمی گوید: «اگر این قانونها از حضور من برداشته شود، آنگاه ذریت اسرائیل نیز زایل خواهند شدتا به حضور من قوم دایمی نباشند.»
\par 37 خداوندچنین می‌گوید: «اگر آسمانهای علوی پیموده شوند و اساس زمین سفلی را تفحص توان نمود، آنگاه من نیز تمامی ذریت اسرائیل را به‌سبب آنچه عمل نمودند ترک خواهم کرد. کلام خداوند این است.»
\par 38 یهوه می‌گوید: «اینک ایامی می‌آید که این شهر از برج حننئیل تا دروازه زاویه بنا خواهدشد.
\par 39 و ریسمان کار به خط مستقیم تا تل جارب بیرون خواهد رفت و بسوی جوعت دور خواهدزد.و تمامی وادی لاشها و خاکستر و تمامی زمینها تا وادی قدرون و بطرف مشرق تا زاویه دروازه اسبان، برای خداوند مقدس خواهد شد وبار دیگر تا ابدالاباد کنده و منهدم نخواهدگردید.»
\par 40 و تمامی وادی لاشها و خاکستر و تمامی زمینها تا وادی قدرون و بطرف مشرق تا زاویه دروازه اسبان، برای خداوند مقدس خواهد شد وبار دیگر تا ابدالاباد کنده و منهدم نخواهدگردید.»
 
\chapter{32}

\par 1 کلامی که در سال دهم صدقیا پادشاه یهودا که سال هجدهم نبوکدرصر باشداز جانب خداوند بر ارمیا نازل شد.
\par 2 و در آن وقت لشکر پادشاه بابل اورشلیم را محاصره کرده بودندو ارمیا نبی در صحن زندانی که در خانه پادشاه یهودا بود محبوس بود.
\par 3 زیرا صدقیا پادشاه یهودا او را به زندان انداخته، گفت: «چرا نبوت می‌کنی و می‌گویی که خداوند چنین می‌فرماید. اینک من این شهر را به‌دست پادشاه بابل تسلیم خواهم کرد و آن را تسخیر خواهد نمود.
\par 4 وصدقیا پادشاه یهودا از دست کلدانیان نخواهدرست بلکه البته به‌دست پادشاه بابل تسلیم شده، دهانش با دهان وی تکلم خواهد نمود و چشمش چشم وی را خواهد دید.
\par 5 و خداوند می‌گوید که صدقیا را به بابل خواهد برد و او در آنجا تا حینی که از او تفقد نمایم خواهد ماند. زیرا که شما باکلدانیان جنگ خواهید کرد، اما کامیاب نخواهیدشد.»
\par 6 و ارمیا گفت: «کلام خداوند بر من نازل شده، گفت:
\par 7 اینک حنمئیل پسر عموی تو شلوم نزد تو آمده، خواهد گفت مزرعه مرا که در عناتوت است برای خود بخر زیرا حق انفکاک از آن تواست که آن را بخری.»
\par 8 پس حنمئیل پسر عموی من بر وفق کلام خداوند نزد من در صحن زندان آمده، مرا گفت: «تمنا اینکه مزرعه مرا که در عناتوت در زمین بنیامین است بخری زیرا که حق ارثیت و حق انفکاکش از آن تو است پس آن را برای خودبخر.» آنگاه دانستم که این کلام از جانب خداونداست.
\par 9 پس مرزعه‌ای را که در عناتوت بود ازحنمئیل پسر عموی خود خریدم و وجه آن راهفده مثقال نقره برای وی وزن نمودم.
\par 10 و قباله را نوشته، مهر کردم و شاهدان گرفته، نقره را درمیزان وزن نمودم.
\par 11 پس قباله های خرید را هم آن را که برحسب شریعت و فریضه مختوم بود وهم آن را که باز بود گرفتم.
\par 12 و قباله خرید را به باروک بن نیریا ابن محسیا به حضور پسر عموی خود حنمئیل و به حضور شهودی که قباله خریدرا امضا کرده بودند و به حضور همه یهودیانی که در صحن زندان نشسته بودند، سپردم.
\par 13 و باروک را به حضور ایشان وصیت کرده، گفتم:
\par 14 «یهوه صبایوت خدای اسرائیل چنین می‌گوید: این قباله‌ها یعنی قباله این خرید را، هم آن را که مختوم است و هم آن را که باز است، بگیرو آنها را در ظرف سفالین بگذار تا روزهای بسیاربماند.
\par 15 زیرا یهوه صبایوت خدای اسرائیل چنین می‌گوید: دیگرباره خانه‌ها و مزرعه‌ها وتاکستانها در این زمین خریده خواهد شد.»
\par 16 و بعد از آنکه قباله خرید را به باروک بن نیریا داده بودم، نزد خداوند تضرع نموده، گفتم:
\par 17 «آه‌ای خداوند یهوه اینک تو آسمان و زمین رابه قوت عظیم و بازوی بلند خود آفریدی وچیزی برای تو مشکل نیست
\par 18 که به هزاران احسان می‌نمایی و عقوبت گناه پدران را به آغوش پسرانشان بعد از ایشان می‌رسانی! خدای عظیم جبار که اسم تو یهوه صبایوت می‌باشد.
\par 19 عظیم المشورت و قوی العمل که چشمانت برتمامی راههای بنی آدم مفتوح است تا بهر کس برحسب راههایش و بر وفق ثمره اعمالش جزادهی.
\par 20 که آیات و علامات در زمین مصر و دراسرائیل و در میان مردمان تا امروز قرار دادی و ازبرای خود مثل امروز اسمی پیدا نمودی.
\par 21 وقوم خود اسرائیل را به آیات و علامات و به‌دست قوی و بازوی بلند و هیبت عظیم از زمین مصربیرون آوردی.
\par 22 و این زمین را که برای پدران ایشان قسم خوردی که به ایشان بدهی به ایشان دادی. زمینی که به شیر و شهد جاری است.
\par 23 وایشان چون داخل شده، آن را به تصرف آوردندکلام تو را نشنیدند و به شریعت تو سلوک ننمودند و به آنچه ایشان را امر فرمودی که بکنندعمل ننمودند. بنابراین تو تمام این بلا را به ایشان وارد آوردی.
\par 24 اینک سنگرها به شهر رسیده است تا آن را تسخیر نمایند و شهر به‌دست کلدانیانی که با آن جنگ می‌کنند به شمشیر وقحط و وبا تسلیم می‌شود و آنچه گفته بودی واقع شده است و اینک تو آن را می‌بینی.
\par 25 و تو‌ای خداوند یهوه به من گفتی که این مزرعه را برای خود به نقره بخر و شاهدان بگیر و حال آنکه شهربه‌دست کلدانیان تسلیم شده است.»
\par 26 پس کلام خداوند به ارمیا نازل شده، گفت:
\par 27 «اینک من یهوه خدای تمامی بشر هستم. آیاهیچ امر برای من مشکل می‌باشد؟
\par 28 بنابراین خداوند چنین می‌گوید: اینک من این شهر را به‌دست کلدانیان و به‌دست نبوکدرصر پادشاه بابل تسلیم می‌کنم و او آن را خواهد گرفت.
\par 29 وکلدانیانی که با این شهر جنگ می‌کنند آمده، این شهر را آتش خواهند زد و آن را با خانه هایی که بربامهای آنها برای بعل بخور‌سوزانیدند و هدایای ریختنی برای خدایان غیر ریخته خشم مرابهیجان آوردند، خواهند سوزانید.
\par 30 زیرا که بنی‌اسرائیل و بنی یهودا از طفولیت خود پیوسته شرارت ورزیدند و خداوند می‌گوید که بنی‌اسرائیل به اعمال دستهای خود خشم مرادایم بهیجان آوردند.
\par 31 زیرا که این شهر ازروزی که آن را بنا کردند تا امروز باعث هیجان خشم و غضب من بوده است تا آن را از حضورخود دور اندازم.
\par 32 به‌سبب تمام شرارتی که بنی‌اسرائیل و بنی یهودا، ایشان و پادشاهان وسروران و کاهنان و انبیای ایشان و مردان یهودا وساکنان اورشلیم کرده، خشم مرا بهیجان آورده‌اند.
\par 33 و پشت به من داده‌اند و نه رو. و هرچند ایشان را تعلیم دادم بلکه صبح زودبرخاسته، تعلیم دادم لیکن گوش نگرفتند وتادیب نپذیرفتند.
\par 34 بلکه رجاسات خود را درخانه‌ای که به اسم من مسمی است برپا کرده، آن رانجس ساختند.
\par 35 و مکان های بلند بعل را که دروادی ابن هنوم است بنا کردند تا پسران و دختران خود را برای مولک از آتش بگذرانند. عملی که ایشان را امر نفرمودم و به‌خاطرم خطور ننمود که چنین رجاسات را بجا آورده، یهودا را مرتکب گناه گردانند.
\par 36 پس الان از این سبب یهوه خدای اسرائیل در حق این شهر‌که شما درباره‌اش می‌گویید که به‌دست پادشاه بابل به شمشیر و قحط و وبا تسلیم شده است، چنین می‌فرماید:
\par 37 «اینک من ایشان را از همه زمینهایی که ایشان را در خشم و حدت و غضب عظیم خود رانده‌ام جمع خواهم کرد وایشان را به این مکان باز آورده، به اطمینان ساکن خواهم گردانید.
\par 38 و ایشان قوم من خواهند بود ومن خدای ایشان خواهم بود.
\par 39 و ایشان را یک دل و یک طریق خواهم داد تا به جهت خیریت خویش و پسران خویش که بعد از ایشان خواهندبود همیشه اوقات از من بترسند.
\par 40 و عهدجاودانی با ایشان خواهم بست که از احسان نمودن به ایشان برنخواهم گشت و ترس خود رادر دل ایشان خواهم نهاد تا از من دوری نورزند.
\par 41 و از احسان نمودن به ایشان مسرور خواهم شد و ایشان را براستی و به تمامی دل و جان خوددر این زمین غرس خواهم نمود.
\par 42 زیرا خداوندچنین می‌گوید: به نوعی که تمامی این بلای عظیم را به این قوم رسانیدم، همچنان تمامی احسانی راکه به ایشان وعده داده‌ام به ایشان خواهم رسانید.
\par 43 و در این زمین که شما درباره‌اش می‌گویید که ویران و از انسان و بهایم خالی شده و به‌دست کلدانیان تسلیم گردیده است، مزرعه‌ها خریده خواهد شد.و مزرعه‌ها به نقره خریده، قباله هاخواهند نوشت و مختوم خواهند نمود و شاهدان خواهند گرفت، در زمین بنیامین و حوالی اورشلیم و شهرهای یهودا و در شهرهای کوهستان و شهرهای همواری و شهرهای جنوب زیرا خداوند می‌گوید اسیران ایشان را باز خواهم آورد.»
\par 44 و مزرعه‌ها به نقره خریده، قباله هاخواهند نوشت و مختوم خواهند نمود و شاهدان خواهند گرفت، در زمین بنیامین و حوالی اورشلیم و شهرهای یهودا و در شهرهای کوهستان و شهرهای همواری و شهرهای جنوب زیرا خداوند می‌گوید اسیران ایشان را باز خواهم آورد.»
 
\chapter{33}

\par 1 و هنگامی که ارمیا در صحن زندان محبوس بود کلام خداوند بار دیگر براو نازل شده، گفت:
\par 2 «خداوند که این کار رامی کند و خداوند که آن را مصور ساخته، مستحکم می‌سازد و اسم او یهوه است چنین می‌گوید:
\par 3 مرا بخوان و تو را اجابت خواهم نمودو تو را از چیزهای عظیم و مخفی که آنها راندانسته‌ای مخبر خواهم ساخت.
\par 4 زیرا که یهوه خدای اسرائیل درباره خانه های این شهر و درباره خانه های پادشاهان یهودا که در مقابل سنگرها ومنجنیقها منهدم شده است،
\par 5 و می‌آیند تا باکلدانیان مقاتله نمایند و آنها را به لاشهای کسانی که من در خشم و غضب خود ایشان را کشتم پرمی کنند. زیرا که روی خود را از این شهر به‌سبب تمامی شرارت ایشان مستور ساخته‌ام.
\par 6 اینک به این شهر عافیت و علاج باز خواهم داد و ایشان راشفا خواهم بخشید و فراوانی سلامتی و امانت رابه ایشان خواهم رسانید.
\par 7 و اسیران یهودا واسیران اسرائیل را باز آورده، ایشان را مثل اول بناخواهم نمود.
\par 8 و ایشان را از تمامی گناهانی که به من ورزیده‌اند، طاهر خواهم ساخت و تمامی تقصیرهای ایشان را که بدانها بر من گناه ورزیده واز من تجاوز کرده‌اند، خواهم آمرزید.
\par 9 و این شهر برای من اسم شادمانی و تسبیح و جلال خواهد بود نزد جمیع امت های زمین که چون آنهاهمه احسانی را که به ایشان نموده باشم بشنوندخواهند ترسید. و به‌سبب تمام این احسان وتمامی سلامتی که من به ایشان رسانیده باشم خواهند لرزید.
\par 10 خداوند چنین می‌گوید که دراین مکان که شما درباره‌اش می‌گویید که آن ویران و خالی از انسان و بهایم است یعنی در شهرهای یهودا و کوچه های اورشلیم که ویران و خالی ازانسان و ساکنان و بهایم است،
\par 11 در آنها آوازشادمانی و آواز سرور و آواز داماد و آواز عروس و آواز کسانی که می‌گویند یهوه صبایوت راتسبیح بخوانید زیرا خداوند نیکو است و رحمت او تا ابدالاباد است، بار دیگر شنیده خواهد شد وآواز آنانی که هدایای تشکر به خانه خداوندمی آورند. زیرا خداوند می‌گوید اسیران این زمین را مثل سابق باز خواهم آورد.
\par 12 یهوه صبایوت چنین می‌گوید: در اینجایی که ویران و از انسان وبهایم خالی است و در همه شهرهایش بار دیگرمسکن شبانانی که گله‌ها را می‌خوابانند خواهدبود.
\par 13 و خداوند می‌گوید که در شهرهای کوهستان و شهرهای همواری و شهرهای جنوب و در زمین بنیامین و در حوالی اورشلیم وشهرهای یهودا گوسفندان بار دیگر از زیردست شمارندگان خواهند گذشت.
\par 14 اینک خداوندمی گوید: ایامی می‌آید که آن وعده نیکو را که درباره خاندان اسرائیل و خاندان یهودا دادم وفاخواهم نمود.
\par 15 در آن ایام و در آن زمان شاخه عدالت برای داود خواهم رویانید و او انصاف وعدالت را در زمین جاری خواهد ساخت.
\par 16 درآن ایام یهودا نجات خواهد یافت و اورشلیم به امنیت مسکون خواهد شد و اسمی که به آن نامیده می‌شود این است: یهوه صدقینو.
\par 17 زیرا خداوند چنین می‌گوید که از داود کسی‌که برکرسی خاندان اسرائیل بنشیند کم نخواهد شد. 
\par 18 و از لاویان کهنه کسی‌که قربانی های سوختنی بگذراند و هدایای آردی بسوزاند وذبایح همیشه ذبح نماید از حضور من کم نخواهدشد.»
\par 19 و کلام خداوند بر ارمیا نازل شده، گفت:
\par 20 «خداوند چنین می‌گوید: اگر عهد مرا با روز وعهد مرا با شب باطل توانید کرد که روز و شب دروقت خود نشود،
\par 21 آنگاه عهد من با بنده من داودباطل خواهد شد تا برایش پسری که بر کرسی اوسلطنت نماید نباشد و با لاویان کهنه که خادم من می‌باشند.
\par 22 چنانکه لشکر آسمان را نتوان شمردو ریگ دریا را قیاس نتوان کرد، همچنان ذریت بنده خود داود و لاویان را که مرا خدمت می‌نمایند زیاده خواهم گردانید.»
\par 23 و کلام خداوند بر ارمیا نازل شده، گفت:
\par 24 «آیا نمی بینی که این قوم چه حرف می‌زنند؟ می‌گویند که خداوند آن دو خاندان را که برگزیده بود ترک نموده است. پس قوم مرا خوارمی شمارند که در نظر ایشان دیگر قومی نباشند.
\par 25 خداوند چنین می‌گوید: اگر عهد من با روز وشب نمی بود و قانون های آسمان و زمین را قرارنمی دادم،آنگاه نیز ذریت یعقوب ونسل بنده خود داود را ترک می‌نمودم و از ذریت او بر اولادابراهیم و اسحاق و یعقوب حاکمان نمی گرفتم. زیرا که اسیران ایشان را باز خواهم آورد و برایشان ترحم خواهم نمود.»
\par 26 آنگاه نیز ذریت یعقوب ونسل بنده خود داود را ترک می‌نمودم و از ذریت او بر اولادابراهیم و اسحاق و یعقوب حاکمان نمی گرفتم. زیرا که اسیران ایشان را باز خواهم آورد و برایشان ترحم خواهم نمود.»
 
\chapter{34}

\par 1 کلامی که از جانب خداوند در حینی که نبوکدنصر پادشاه بابل و تمامی لشکرش و جمیع ممالک جهان که زیر حکم اوبودند و جمیع قومها با اورشلیم و تمامی شهرهایش جنگ می‌نمودند بر ارمیا نازل شده، گفت:
\par 2 «یهوه خدای اسرائیل چنین می‌گوید: بروو صدقیا پادشاه یهودا را خطاب کرده، وی را بگوخداوند چنین می‌فرماید: اینک من این شهر را به‌دست پادشاه بابل تسلیم می‌کنم و او آن را به آتش خواهد سوزانید.
\par 3 و تو از دستش نخواهی رست. بلکه البته گرفتار شده، به‌دست او تسلیم خواهی گردید و چشمان تو چشمان پادشاه بابل راخواهد دید و دهانش با دهان تو گفتگو خواهدکرد و به بابل خواهی رفت.
\par 4 لیکن‌ای صدقیاپادشاه یهودا کلام خداوند را بشنو. خداونددرباره تو چنین می‌گوید: به شمشیر نخواهی مرد،
\par 5 بلکه بسلامتی خواهی مرد. و چنانکه برای پدرانت یعنی پادشاهان پیشین که قبل از تو بودند(عطریات ) سوزانیدند، همچنان برای تو خواهندسوزانید و برای تو ماتم گرفته، خواهند گفت: آه‌ای آقا. زیرا خداوند می‌گوید: من این سخن راگفتم.»
\par 6 پس ارمیا نبی تمامی این سخنان را به صدقیاپادشاه یهودا در اورشلیم گفت،
\par 7 هنگامی که لشکر پادشاه بابل با اورشلیم و با همه شهرهای باقی یهود یعنی با لاکیش و عزیقه جنگ می‌نمودند. زیرا که این دو شهر از شهرهای حصاردار یهودا فقط باقی‌مانده بود.
\par 8 کلامی که ازجانب خداوند بر ارمیا نازل شد بعد از آنکه صدقیا پادشاه با تمامی قومی که در اورشلیم بودند عهد بست که ایشان به آزادی ندا نمایند،
\par 9 تا هر کس غلام عبرانی خود و هر کس کنیزعبرانیه خویش را به آزادی رها کند و هیچکس برادر یهود خویش را غلام خود نسازد.
\par 10 پس جمیع سروران و تمامی قومی که داخل این عهدشدند اطاعت نموده، هر کدام غلام خود را و هرکدام کنیز خویش را به آزادی رها کردند و ایشان را دیگر به غلامی نگاه نداشتند بلکه اطاعت نموده، ایشان را رهایی دادند.
\par 11 لکن بعد از آن ایشان برگشته، غلامان و کنیزان خود را که به آزادی رها کرده بودند، باز آوردند و ایشان را به عنف به غلامی و کنیزی خود گرفتند.
\par 12 و کلام خداوند بر ارمیا از جانب خداوندنازل شده، گفت:
\par 13 «یهوه خدای اسرائیل چنین می‌گوید: من با پدران شما در روزی که ایشان را اززمین مصر از خانه بندگی بیرون آوردم عهد بسته، گفتم
\par 14 که در آخر هر هفت سال هر کدام از شمابرادر عبرانی خود را که خویشتن را به تو فروخته باشد رها کنید. و چون تو را شش سال بندگی کرده باشد، او را از نزد خود به آزادی رهایی دهی. اما پدران شما مرا اطاعت ننمودند و گوش خود را به من فرا نداشتند.
\par 15 و شما در این زمان بازگشت نمودید و آنچه در نظر من پسند است بجا آوردید. و هر کس برای همسایه خود به آزادی ندا نموده، در خانه‌ای که به اسم من نامیده شده است عهد بستید.
\par 16 اما از آن روتافته اسم مرا بی‌عصمت کردید و هر کدام از شما غلام خودرا و هر کس کنیز خویش را که ایشان را برحسب میل ایشان به آزادی رها کرده بودید، باز آوردید و ایشان را به عنف به غلامی و کنیزی خودگرفتید.
\par 17 بنابراین خداوند چنین می‌گوید: چونکه شما مرا اطاعت ننمودید و هر کس برای برادر خود و هر کدام برای همسایه خویش به آزادی ندا نکردید، اینک خداوند می‌گوید: من برای شما آزادی را به شمشیر و وبا و قحط ندامی کنم و شما را در میان تمامی ممالک جهان مشوش خواهم گردانید.
\par 18 و تسلیم خواهم کردکسانی را که از عهد من تجاوز نمودند و وفاننمودند به کلام عهدی که به حضور من بستند، حینی که گوساله را دو پاره کرده، در میان پاره هایش گذاشتند.
\par 19 یعنی سروران یهودا وسروران اورشلیم و خواجه‌سرایان و کاهنان وتمامی قوم زمین را که در میان پاره های گوساله گذر نمودند.
\par 20 و ایشان را به‌دست دشمنان ایشان و به‌دست آنانی که قصد جان ایشان دارند، خواهم سپرد و لاشهای ایشان خوراک مرغان هواو حیوانات زمین خواهد شد.
\par 21 و صدقیا پادشاه یهودا و سرورانش را به‌دست دشمنان ایشان و به‌دست آنانی که قصد جان ایشان دارند و به‌دست لشکر پادشاه بابل که از نزد شما رفته‌اند تسلیم خواهم کرد.اینک خداوند می‌گوید من امرمی فرمایم و ایشان را به این شهر باز خواهم آوردو با آن جنگ کرده، آن را خواهند گرفت و به آتش خواهند سوزانید و شهرهای یهودا را ویران وغیرمسکون خواهم ساخت.»
\par 22 اینک خداوند می‌گوید من امرمی فرمایم و ایشان را به این شهر باز خواهم آوردو با آن جنگ کرده، آن را خواهند گرفت و به آتش خواهند سوزانید و شهرهای یهودا را ویران وغیرمسکون خواهم ساخت.»
 
\chapter{35}

\par 1 کلامی که از جانب خداوند در ایام یهویاقیم بن یوشیا پادشاه یهودا بر ارمیا نازل شده، گفت:
\par 2 «به خانه رکابیان برو و به ایشان سخن گفته، ایشان را به یکی از حجره های خانه خداوند بیاور و به ایشان شراب بنوشان.»
\par 3 پس یازنیا ابن ارمیا ابن حبصنیا و برادرانش وجمیع پسرانش و تمامی خاندان رکابیان رابرداشتم،
\par 4 و ایشان را به خانه خداوند به حجره پسران حانان بن یجدلیا مرد خدا که به پهلوی حجره سروران و بالای حجره معسیا ابن شلوم، مستحفظ آستانه بود آوردم.
\par 5 و کوزه های پر ازشراب و پیاله‌ها پیش رکابیان نهاده، به ایشان گفتم: «شراب بنوشید.»
\par 6 ایشان گفتند: «شراب نمی نوشیم زیرا که پدرما یوناداب بن رکاب ما را وصیت نموده، گفت که شما و پسران شما ابد شراب ننوشید.
\par 7 و خانه هابنا مکنید و کشت منمایید و تاکستانها غرس مکنید و آنها را نداشته باشید بلکه تمامی روزهای خود را در خیمه‌ها ساکن شوید تا روزهای بسیاربه روی زمینی که شما در آن غریب هستید زنده بمانید.
\par 8 و ما به سخن پدر خود یوناداب بن رکاب و بهر‌چه او به ما امر فرمود اطاعت نموده، درتمامی عمر خود شراب ننوشیدیم، نه ما و نه زنان ما و نه پسران ما و نه دختران ما.
\par 9 و خانه‌ها برای سکونت خود بنا نکردیم و تاکستانها و املاک ومزرعه‌ها برای خود نگرفتیم.
\par 10 و در خیمه هاساکن شده، اطاعت نمودیم و به آنچه پدر مایوناداب ما را امر فرمود عمل نمودیم.
\par 11 لیکن وقتی که نبوکدرصر پادشاه بابل به زمین برآمدگفتیم: بیایید از ترس لشکر کلدانیان و لشکرارامیان به اورشلیم داخل شویم پس در اورشلیم ساکن شدیم.»
\par 12 پس کلام خداوند بر ارمیا نازل شده، گفت:
\par 13 «یهوه صبایوت خدای اسرائیل چنین می‌گوید: برو و به مردان یهودا و ساکنان اورشلیم بگو که خداوند می‌گوید: آیا تادیب نمی پذیرید وبه کلام من گوش نمی گیرید؟
\par 14 سخنان یوناداب بن رکاب که به پسران خود وصیت نمود که شراب ننوشید استوار گردیده است و تا امروز شراب نمی نوشند و وصیت پدر خود را اطاعت می‌نمایند، اما من به شما سخن گفتم و صبح زودبرخاسته، تکلم نمودم و مرا اطاعت نکردید.
\par 15 وبندگان خود انبیا را نزد شما فرستادم و صبح زودبرخاسته، ایشان را ارسال نموده، گفتم هر کدام ازراه بد خود بازگشت نمایید و اعمال خود رااصلاح کنید و خدایان غیر را پیروی منمایید وآنها را عبادت مکنید تا در زمینی که به شما و به پدران شما داده‌ام ساکن شوید. اما شما گوش نگرفتید و مرا اطاعت ننمودید.
\par 16 پس چونکه پسران یوناداب بن رکاب وصیت پدر خویش راکه به ایشان فرموده است اطاعت می‌نمایند و این قوم مرا اطاعت نمی کنند،
\par 17 بنابراین یهوه خدای صبایوت خدای اسرائیل چنین می‌گوید: اینک من بر یهودا و بر جمیع سکنه اورشلیم تمامی آن بلا را که درباره ایشان گفته‌ام وارد خواهم آوردزیرا که به ایشان سخن گفتم و نشنیدند و ایشان راخواندم و اجابت ننمودند.»
\par 18 و ارمیا به خاندان رکابیان گفت: «یهوه صبایوت خدای اسرائیل چنین می‌گوید: چونکه شما وصیت پدر خود یوناداب را اطاعت نمودیدو جمیع اوامر او را نگاه داشته، بهر‌آنچه او به شماامر فرمود عمل نمودید،بنابراین یهوه صبایوت خدای اسرائیل چنین می‌گوید: ازیوناداب بن رکاب کسی‌که دایم به حضور من بایستد کم نخواهد شد.»
\par 19 بنابراین یهوه صبایوت خدای اسرائیل چنین می‌گوید: ازیوناداب بن رکاب کسی‌که دایم به حضور من بایستد کم نخواهد شد.»
 
\chapter{36}

\par 1 و در سال چهارم یهویاقیم بن یوشیاپادشاه یهودا واقع شد که این کلام ازجانب خداوند بر ارمیا نازل شده، گفت:
\par 2 «طوماری برای خود گرفته، تمامی سخنانی راکه من درباره اسرائیل و یهودا و همه امت‌ها به توگفتم از روزی که به تو تکلم نمودم یعنی از ایام یوشیا تا امروز در آن بنویس.
\par 3 شاید که خاندان یهودا تمامی بلا را که من می‌خواهم بر ایشان واردبیاورم گوش بگیرند تا هر کدام از ایشان از راه بدخود بازگشت نمایند و من عصیان و گناهان ایشان را بیامرزم.»
\par 4 پس ارمیا باروک بن نیریا را خواند و باروک ازدهان ارمیا تمامی کلام خداوند را که به او گفته بوددر آن طومار نوشت.
\par 5 و ارمیا باروک را امرفرموده، گفت: «من محبوس هستم و نمی توانم به خانه خداوند داخل شوم.
\par 6 پس تو برو و سخنان خداوند را از طوماری که از دهان من نوشتی درروز صوم در خانه خداوند در گوش قوم بخوان ونیز آنها را در گوش تمامی یهودا که از شهرهای خود می‌آیند بخوان.
\par 7 شاید که به حضورخداوند استغاثه نمایند و هر کدام از ایشان از راه بد خود بازگشت کنند زیرا که خشم و غضبی که خداوند درباره این قوم فرموده است عظیم می‌باشد.»
\par 8 پس باروک بن نیریا بهر‌آنچه ارمیا نبی او را امر فرموده بود عمل نمود و کلام خداوند را درخانه خداوند از آن طومار خواند.
\par 9 و در ماه نهم از سال پنجم یهویاقیم بن یوشیا پادشاه یهودابرای تمامی اهل اورشلیم و برای همه کسانی که از شهرهای یهودا به اورشلیم می‌آمدند برای روزه به حضور خداوند ندا کردند.
\par 10 و باروک سخنان ارمیا را از آن طومار در خانه خداوند درحجره جمریا ابن شافان کاتب در صحن فوقانی نزد دهنه دروازه جدید خانه خداوند به گوش تمامی قوم خواند.
\par 11 و چون میکایا ابن جمریا ابن شافان تمامی سخنان خداوند را از آن طومارشنید،
\par 12 به خانه پادشاه به حجره کاتب آمد واینک جمیع سروران در آنجا نشسته بودند یعنی الیشاماع کاتب و دلایا ابن شمعیا و الناتان بن عکبور و جمریا ابن شافان و صدقیا ابن حننیا وسایر سروران.
\par 13 پس میکایا تمامی سخنانی راکه از باروک وقتی که آنها را به گوش خلق ازطومار می‌خواند شنید برای ایشان باز‌گفت.
\par 14 آنگاه تمامی سروران یهودی ابن نتنیا ابن شلمبا ابن کوشی را نزد باروک فرستادند تا بگوید: «آن طوماری را که به گوش قوم خواندی به‌دست خود گرفته، بیا.» پس باروک بن نیریا طومار را به‌دست خود گرفته، نزد ایشان آمد. 
\par 15 و ایشان وی را گفتند: «بنشین و آن را به گوشهای ما بخوان.» و باروک به گوش ایشان خواند.
\par 16 و واقع شد که چون ایشان تمامی این سخنان را شنیدند با ترس به یکدیگر نظر افکندندو به باروک گفتند: «البته تمامی این سخنان را به پادشاه بیان خواهیم کرد.»
\par 17 و از باروک سوال کرده، گفتند: «ما را خبر بده که تمامی این سخنان را چگونه از دهان او نوشتی.»
\par 18 باروک به ایشان گفت: «او تمامی این سخنان را از دهان خود برای من می‌خواند و من با مرکب در طومار می‌نوشتم.»
\par 19 سروران به باروک گفتند: «تو و ارمیا رفته، خویشتن را پنهان کنید تا کسی نداند که کجامی باشید.»
\par 20 پس طومار را در حجره الیشاماع کاتب گذاشته، به‌سرای پادشاه رفتند وتمامی این سخنان را به گوش پادشاه باز‌گفتند.
\par 21 و پادشاه یهودی را فرستاد تا طومار را بیاورد و یهودی آن را از حجره الیشاماع کاتب آورده، در گوش پادشاه و در گوش تمامی سرورانی که به حضورپادشاه حاضر بودند خواند.
\par 22 و پادشاه در ماه نهم در خانه زمستانی نشسته و آتش پیش وی برمنقل افروخته بود.
\par 23 و واقع شد که چون یهودی سه چهار ورق خوانده بود، (پادشاه ) آن را باقلمتراش قطع کرده، در آتشی که بر منقل بودانداخت تا تمامی طومار در آتشی که در منقل بودسوخته شد.
\par 24 و پادشاه و همه بندگانش که تمامی این سخنان را شنیدند نه ترسیدند و نه جامه خود را چاک زدند.
\par 25 لیکن الناتان و دلایا و جمریا از پادشاه التماس کردند که طومار را نسوزاند اما به ایشان گوش نگرفت.
\par 26 بلکه پادشاه یرحمیئیل شاهزاده و سرایا ابن عزرئیل و شلمیا ابن عبدئیل را امرفرمود که باروک کاتب و ارمیا نبی را بگیرند. اماخداوند ایشان را مخفی داشت.
\par 27 و بعد از آنکه پادشاه طومار و سخنانی را که باروک از دهان ارمیا نوشته بود سوزانید کلام خداوند بر ارمیا نازل شده، گفت:
\par 28 «طوماری دیگر برای خود باز گیر و همه سخنان اولین را که در طومار نخستین که یهویاقیم پادشاه یهودا آن راسوزانید بر آن بنویس.
\par 29 و به یهویاقیم پادشاه یهودا بگو خداوند چنین می‌فرماید: تو این طومار را سوزانیدی و گفتی چرا در آن نوشتی که پادشاه بابل البته خواهد آمد و این زمین را خراب کرده، انسان و حیوان را از آن نابود خواهد ساخت.
\par 30 بنابراین خداوند درباره یهویاقیم پادشاه یهوداچنین می‌فرماید که برایش کسی نخواهد بود که برکرسی داود بنشیند و لاش او روز در گرما و شب در سرما بیرون افکنده خواهد شد.
\par 31 و بر او و برذریتش و بر بندگانش عقوبت گناه ایشان راخواهم آورد و بر ایشان و بر سکنه اورشلیم ومردان یهودا تمامی آن بلا را که درباره ایشان گفته‌ام خواهم رسانید زیرا که مرا نشنیدند.»پس ارمیا طوماری دیگر گرفته، به باروک بن نیریای کاتب سپرد و او تمامی سخنان طوماری راکه یهویاقیم پادشاه یهودا به آتش سوزانیده بود ازدهان ارمیا در آن نوشت و سخنان بسیاری نیز مثل آنها بر آن افزوده شد.
\par 32 پس ارمیا طوماری دیگر گرفته، به باروک بن نیریای کاتب سپرد و او تمامی سخنان طوماری راکه یهویاقیم پادشاه یهودا به آتش سوزانیده بود ازدهان ارمیا در آن نوشت و سخنان بسیاری نیز مثل آنها بر آن افزوده شد.
 
\chapter{37}

\par 1 و صدقیا ابن یوشیا پادشاه به‌جای کنیاهو ابن یهویاقیم که نبوکدرصرپادشاه بابل او را بر زمین یهودا به پادشاهی نصب کرده بود سلطنت نمود.
\par 2 و او و بندگانش و اهل زمین به کلام خداوند که به واسطه ارمیا نبی گفته بود گوش ندادند.
\par 3 و صدقیا پادشاه، یهوکل بن شلمیا و صفنیا ابن معسیا کاهن را نزد ارمیای نبی فرستاد که بگویند: «نزد یهوه خدای ما به جهت مااستغاثه نما.»
\par 4 و ارمیا در میان قوم آمد و شدمی نمود زیرا که او را هنوز در زندان نینداخته بودند.
\par 5 و لشکر فرعون از مصر بیرون آمدند وچون کلدانیانی که اورشلیم را محاصره کرده بودند خبر ایشان را شنیدند از پیش اورشلیم رفتند.
\par 6 آنگاه کلام خداوند بر ارمیا نبی نازل شده، گفت:
\par 7 «یهوه خدای اسرائیل چنین می‌فرماید: به پادشاه یهودا که شما را نزد من فرستاد تا از من مسالت نمایید چنین بگویید: اینک لشکر فرعون که به جهت اعانت شما بیرون آمده‌اند، به ولایت خود به مصر مراجعت خواهند نمود.
\par 8 و کلدانیان خواهند برگشت و با این شهر جنگ خواهند کردو آن را تسخیر نموده، به آتش خواهند سوزانید.
\par 9 و خداوند چنین می‌گوید که خویشتن را فریب ندهید و مگویید که کلدانیان از نزد ما البته خواهندرفت زیرا که نخواهند رفت.
\par 10 بلکه اگر تمامی لشکر کلدانیانی را که با شما جنگ می‌نمایندچنان شکست می‌دادید که از ایشان غیر ازمجروح شدگان کسی نمی ماند، باز هر کدام ازایشان از خیمه خود برخاسته، این شهر را به آتش می‌سوزانیدند.»
\par 11 و بعد از آنکه لشکر کلدانیان از ترس لشکرفرعون از اورشلیم کوچ کرده بودند، واقع شد
\par 12 که ارمیا از اورشلیم بیرون می‌رفت تا به زمین بنیامین برود و در آنجا از میان قوم نصیب خود رابگیرد.
\par 13 و چون به دروازه بنیامین رسید رئیس کشیکچیان مسمی به یرئیا ابن شلمیا ابن حننیا درآنجا بود و او ارمیای نبی را گرفته، گفت: «نزدکلدانیان می‌روی؟»
\par 14 ارمیا گفت: «دروغ است نزد کلدانیان نمی روم.» لیکن یرئیا به وی گوش نداد و ارمیا راگرفته او را نزد سروران آورد.
\par 15 و سروران برارمیا خشم نموده، او را زدند و او را در خانه یوناتان کاتب به زندان انداختند زیرا آن را زندان ساخته بودند.
\par 16 و چون ارمیا در سیاه چال به یکی از حجره‌ها داخل شده بود و ارمیا روزهای بسیاردر آنجا مانده بود،
\par 17 آنگاه صدقیا پادشاه فرستاده، او را آورد و پادشاه در خانه خود خفیه از او سوال نموده، گفت که «آیا کلامی از جانب خداوند هست؟» ارمیا گفت: «هست و گفت به‌دست پادشاه بابل تسلیم خواهی شد.»
\par 18 و ارمیا به صدقیا پادشاه گفت: «و به تو وبندگانت و این قوم چه گناه کرده‌ام که مرا به زندان انداخته‌اید؟
\par 19 و انبیای شما که برای شما نبوت کرده، گفتند که پادشاه بابل بر شما و بر این زمین نخواهد آمد کجا می‌باشند؟
\par 20 پس الان‌ای آقایم پادشاه بشنو: تمنا اینکه استدعای من نزد توپذیرفته شود که مرا به خانه یوناتان کاتب پس نفرستی مبادا در آنجا بمیرم.»پس صدقیا پادشاه امر فرمود که ارمیا را درصحن زندان بگذارند. و هر روز قرص نانی ازکوچه خبازان به او دادند تا همه نان از شهر تمام شد. پس ارمیا در صحن زندان ماند.
\par 21 پس صدقیا پادشاه امر فرمود که ارمیا را درصحن زندان بگذارند. و هر روز قرص نانی ازکوچه خبازان به او دادند تا همه نان از شهر تمام شد. پس ارمیا در صحن زندان ماند.
 
\chapter{38}

\par 1 و شفطیا ابن متان و جدلیا ابن فشحور ویوکل بن شلمیا و فشحور بن ملکیاسخنان ارمیا را شنیدند که تمامی قوم را بدانهامخاطب ساخته، گفت:
\par 2 «خداوند چنین می‌گوید: هر‌که در این شهر بماند از شمشیر وقحط و وبا خواهد مرد اما هر‌که نزد کلدانیان بیرون رود خواهد زیست و جانش برای اوغنیمت شده، زنده خواهد ماند.
\par 3 خداوند چنین می‌گوید: این شهر البته به‌دست لشکر پادشاه بابل تسلیم شده، آن را تسخیر خواهد نمود.»
\par 4 پس آن سروران به پادشاه گفتند: «تمنا اینکه این مرد کشته شود زیرا که بدین منوال دستهای مردان جنگی را که در این شهر باقی‌مانده‌اند ودستهای تمامی قوم را سست می‌کند چونکه مثل این سخنان به ایشان می‌گوید. زیرا که این مردسلامتی این قوم را نمی طلبد بلکه ضرر ایشان را.»
\par 5 صدقیا پادشاه گفت: «اینک او در دست شمااست زیرا پادشاه به خلاف شما کاری نمی تواندکرد.»
\par 6 پس ارمیا را گرفته او را در سیاه چال ملکیا ابن ملک که در صحن زندان بود انداختند و ارمیا را به ریسمانها فرو هشتند و در آن سیاه چال آب نبودلیکن گل بود و ارمیا به گل فرو رفت.
\par 7 و چون عبدملک حبشی که یکی از خواجه‌سرایان و درخانه پادشاه بود شنید که ارمیا را به سیاه چال انداختند (و به دروازه بنیامین نشسته بود)،
\par 8 آنگاه عبدملک از خانه پادشاه بیرون آمد و به پادشاه عرض کرده، گفت:
\par 9 «که‌ای آقایم پادشاه این مردان در آنچه به ارمیای نبی کرده و او را به سیاه چال انداخته‌اند شریرانه عمل نموده‌اند و اودر جایی که هست از گرسنگی خواهد مرد زیراکه در شهر هیچ نان باقی نیست.»
\par 10 پس پادشاه به عبدملک حبشی امر فرموده، گفت: «سی نفر از اینجا همراه خود بردار وارمیای نبی را قبل از آنکه بمیرد از سیاه چال برآور.»
\par 11 پس عبدملک آن کسان را همراه خودبرداشته، به خانه پادشاه از زیر خزانه داخل شد واز آنجا پارچه های مندرس و رقعه های پوسیده گرفته، آنها را با ریسمانها به سیاه چال نزد ارمیافروهشت.
\par 12 و عبدملک حبشی به ارمیا گفت: «این پارچه های مندرس و رقعه های پوسیده رازیر بغل خود در زیر ریسمانها بگذار.» و ارمیا چنین کرد.
\par 13 پس ارمیا را با ریسمانها کشیده، اورا از سیاه چال برآوردند و ارمیا در صحن زندان ساکن شد.
\par 14 و صدقیا پادشاه فرستاده، ارمیا نبی را به مدخل سومی که در خانه خداوند بود نزد خودآورد و پادشاه به ارمیا گفت: «من از تو مطلبی می‌پرسم، از من چیزی مخفی مدار.»
\par 15 ارمیا به صدقیا گفت: «اگر تو را خبر دهم آیا هر آینه مرانخواهی کشت و اگر تو را پند دهم مرا نخواهی شنید؟»
\par 16 آنگاه صدقیا پادشاه برای ارمیا خفیه قسم خورده، گفت: «به حیات یهوه که این‌جان رابرای ما آفرید قسم که تو را نخواهم کشت و تو رابه‌دست این کسانی که قصد جان تو دارند تسلیم نخواهم کرد.»
\par 17 پس ارمیا به صدقیا گفت: «یهوه خدای صبایوت خدای اسرائیل چنین می‌گوید: اگرحقیقت نزد سروران پادشاه بابل بیرون روی، جان تو زنده خواهد ماند و این شهر به آتش سوخته نخواهد شد بلکه تو و اهل خانه ات زنده خواهیدماند.
\par 18 اما اگر نزد سروران پادشاه بابل بیرون نروی این شهر به‌دست کلدانیان تسلیم خواهدشد و آن را به آتش خواهند سوزانید و تو از دست ایشان نخواهی رست.»
\par 19 اما صدقیا پادشاه به ارمیا گفت: «من ازیهودیانی که بطرف کلدانیان شده‌اند می‌ترسم، مبادا مرا به‌دست ایشان تسلیم نموده، ایشان مراتفضیح نمایند.»
\par 20 ارمیا در جواب گفت: «تو راتسلیم نخواهند کرد. مستدعی آنکه کلام خداوندرا که به تو می‌گویم اطاعت نمایی تا تو را خیریت شود و جان تو زنده بماند.
\par 21 اما اگر از بیرون رفتن ابا نمایی کلامی که خداوند بر من کشف نموده این است:
\par 22 اینک تمامی زنانی که در خانه پادشاه یهودا باقی‌مانده‌اند نزد سروران پادشاه بابل بیرون برده خواهند شدو ایشان خواهند گفت: اصدقای تو تو را اغوا نموده، بر تو غالب آمدند و الان چونکه پایهای تو در لجن فرو رفته است ایشان به عقب برگشته‌اند.
\par 23 و جمیع زنانت و فرزندانت رانزد کلدانیان بیرون خواهند برد و تو از دست ایشان نخواهی رست بلکه به‌دست پادشاه بابل گرفتار خواهی شد و این شهر را به آتش خواهی سوزانید.»
\par 24 آنگاه صدقیا به ارمیا گفت: «زنهار کسی ازاین سخنان اطلاع نیابد و نخواهی مرد.
\par 25 و اگرسروران بشنوند که با تو گفتگو کرده‌ام و نزدتو آمده، تو را گویند تمنا اینکه ما را از آنچه به پادشاه گفتی و آنچه پادشاه به تو گفت اطلاع دهی و آن را از ما مخفی نداری تا تو رابه قتل نرسانیم،
\par 26 آنگاه به ایشان بگو: من عرض خود را به حضور پادشاه رسانیدم تا مرابه خانه یوناتان باز نفرستد تا در آنجانمیرم.»
\par 27 پس جمیع سروران نزد ارمیا آمده، از اوسوال نمودند و او موافق همه این سخنانی که پادشاه به او امر فرموده بود به ایشان گفت. پس ازسخن‌گفتن با او باز ایستادند چونکه مطلب فهمیده نشد.و ارمیا در صحن زندان تا روز فتح شدن اورشلیم ماند و هنگامی که اورشلیم گرفته شد در آنجا بود.
\par 28 و ارمیا در صحن زندان تا روز فتح شدن اورشلیم ماند و هنگامی که اورشلیم گرفته شد در آنجا بود.
 
\chapter{39}

\par 1 در ماه دهم از سال نهم صدقیاپادشاه یهودا، نبوکدرصر پادشاه بابل با تمامی لشکر خود بر اورشلیم آمده، آن را محاصره نمودند. 
\par 2 و در روز نهم ماه چهارم از سال یازدهم صدقیا در شهر رخنه کردند.
\par 3 و تمام سروران پادشاه بابل داخل شده، در دروازه وسطی نشستند یعنی نرجل شراصر و سمجرنبو وسرسکیم رئیس خواجه‌سرایان و نرجل شراصررئیس مجوسیان و سایر سرداران پادشاه بابل.
\par 4 وچون صدقیا پادشاه یهودا و تمامی مردان جنگی این را دیدند فرار کرده، به راه باغ شاه از دروازه‌ای که در میان دو حصار بود در وقت شب از شهربیرون رفتند و (پادشاه ) به راه عربه رفت.
\par 5 ولشکر کلدانیان ایشان را تعاقب نموده، در عربه اریحا به صدقیا رسیدند و او را گرفتار کرده، نزدنبوکدرصر پادشاه بابل به ربله در زمین حمات آوردند و او بر وی فتوی داد.
\par 6 و پادشاه بابل پسران صدقیا را پیش رویش در ربله به قتل رسانید و پادشاه بابل تمامی شرفای یهودا راکشت.
\par 7 و چشمان صدقیا را کور کرد و او را به زنجیرها بسته، به بابل برد.
\par 8 و کلدانیان خانه پادشاه و خانه های قوم را به آتش سوزانیدند وحصارهای اورشلیم را منهدم ساختند.
\par 9 ونبوزردان رئیس جلادان، بقیه قوم را که در شهرباقی‌مانده بودند و خارجین را که بطرف او شده بودند و بقیه قوم را که مانده بودند به بابل به اسیری برد.
\par 10 لیکن نبوزردان رئیس جلادان فقیران قوم را که چیزی نداشتند در زمین یهودا واگذاشت وتاکستانها و مزرعه‌ها در آن روز به ایشان داد.
\par 11 و نبوکدرصر پادشاه بابل درباره ارمیا به نبوزردان رئیس جلادان امر فرموده، گفت:
\par 12 «او را بگیر و به او نیک متوجه شده، هیچ اذیتی به وی مرسان بلکه هر‌چه به تو بگوید برایش بعمل آور.»
\par 13 پس نبوزردان رئیس جلادان و نبوشزبان رئیس خواجه‌سرایان و نرجل شراصر رئیس مجوسیان و سایر سروران پادشاه بابل فرستادند.
\par 14 و ارسال نموده، ارمیا را از صحن زندان برداشتند و او را به جدلیا ابن اخیقام بن شافان سپردند تا او را به خانه خود ببرد. پس در میان قوم ساکن شد.
\par 15 و چون ارمیا هنوز در صحن زندان محبوس بود، کلام خداوند بر وی نازل شده، گفت:
\par 16 «بروو عبدملک حبشی را خطاب کرده، بگو: یهوه صبایوت خدای اسرائیل چنین می‌فرماید: اینک کلام خود را بر این شهر به بلا وارد خواهم آورد ونه بخوبی و در آن روز در نظر تو واقع خواهد شد.
\par 17 لیکن خداوند می‌گوید: من تو را در آن روزنجات خواهم داد و به‌دست کسانی که از ایشان می‌ترسی تسلیم نخواهی شد.زیرا خداوندمی گوید که تو را البته رهایی خواهم داد و به شمشیر نخواهی افتاد، بلکه از این جهت که بر من توکل نمودی جان تو برایت غنیمت خواهد شد.»
\par 18 زیرا خداوندمی گوید که تو را البته رهایی خواهم داد و به شمشیر نخواهی افتاد، بلکه از این جهت که بر من توکل نمودی جان تو برایت غنیمت خواهد شد.»
 
\chapter{40}

\par 1 کلامی که از جانب خداوند بر ارمیا نازل شد بعد از آنکه نبوزردان رئیس جلادان او را از رامه رهایی داد و وی را از میان تمامی اسیران اورشلیم و یهودا که به بابل جلای وطن می‌شدند و او در میان ایشان به زنجیرها بسته شده بود برگرفت.
\par 2 و رئیس جلادان ارمیا را گرفته، وی را گفت: «یهوه خدایت این بلا را درباره این مکان فرموده است.
\par 3 و خداوند برحسب کلام خود این را به وقوع آورده، عمل نموده است.
\par 4 وحال اینک من امروز تو را از زنجیرهایی که بردستهای تو است رها می‌کنم. پس اگر در نظرت پسند آید که با من به بابل بیایی بیا و تو را نیکومتوجه خواهم شد. و اگر در نظرت پسند نیاید که همراه من به بابل آیی، پس میا و بدان که تمامی زمین پیش تو است هر جایی که در نظرت خوش و پسند آید که بروی به آنجا برو.»
\par 5 و وقتی که او هنوز برنگشته بود (وی راگفت ): «نزد جدلیا ابن اخیقام بن شافان که پادشاه بابل او را بر شهرهای یهودا نصب کرده است برگرد و نزد او در میان قوم ساکن شو یا هر جایی که می‌خواهی بروی برو.» پس رئیس جلادان اورا توشه راه و هدیه داد و او را رها نمود.
\par 6 و ارمیانزد جدلیا ابن اخیقام به مصفه آمده، نزد او در میان قومی که در زمین باقی‌مانده بودند ساکن شد.
\par 7 و چون تمامی سرداران لشکر که در صحرابودند و مردان ایشان شنیدند که پادشاه بابل جدلیاابن اخیقام را بر زمین نصب کرده و مردان و زنان واطفال و فقیران زمین را که به بابل برده نشده بودندبه او سپرده است،
\par 8 آنگاه ایشان نزد جدلیا به مصفه آمدند یعنی اسماعیل بن نتنیا و یوحانان ویوناتان پسران قاریح و سرایا ابن تنحومت وپسران عیفای نطوفاتی و یزنیا پسر معکاتی ایشان و مردان ایشان.
\par 9 و جدلیا ابن اخیقام بن شافان برای ایشان و کسان ایشان قسم خورده، گفت: «ازخدمت نمودن به کلدانیان مترسید. در زمین ساکن شوید و پادشاه بابل را بندگی نمایید و برای شمانیکو خواهد شد.
\par 10 و اما من اینک در مصفه ساکن خواهم شد تا به حضور کلدانیانی که نزد ما آیندحاضر شوم و شما شراب و میوه جات و روغن جمع کرده، در ظروف خود بگذارید و درشهرهایی که برای خود گرفته‌اید ساکن باشید.»
\par 11 و نیز چون تمامی یهودیانی که در موآب ودر میان بنی عمون و در ادوم و سایر ولایات بودندشنیدند که پادشاه بابل و بقیه‌ای از یهود راواگذاشته و جدلیا ابن اخیقام بن شافان را بر ایشان گماشته است،
\par 12 آنگاه جمیع یهودیان از هرجایی که پراکنده شده بودند مراجعت کردند و به زمین یهودا نزد جدلیا به مصفه آمدند و شراب ومیوه جات بسیار و فراوان جمع نمودند.
\par 13 و یوحانان بن قاریح و همه سرداران لشکری که در بیابان بودند نزد جدلیا به مصفه آمدند،
\par 14 و او را گفتند: «آیا هیچ می‌دانی که بعلیس پادشاه بنی عمون اسماعیل بن نتنیا رافرستاده است تا تو را بکشد؟» اما جدلیا ابن اخیقام ایشان را باور نکرد.
\par 15 پس یوحانان بن قاریح جدلیا را در مصفه خفیه خطاب کرده، گفت: «اذن بده که بروم و اسماعیل بن نتنیا رابکشم و کسی آگاه نخواهد شد. چرا او تو رابکشد و جمیع یهودیانی که نزد تو فراهم آمده اندپراکنده شوند و بقیه یهودیان تلف گردند؟»اماجدلیا ابن اخیقام به یوحانان بن قاریح گفت: «این کار را مکن زیرا که درباره اسماعیل دروغ می‌گویی.»
\par 16 اماجدلیا ابن اخیقام به یوحانان بن قاریح گفت: «این کار را مکن زیرا که درباره اسماعیل دروغ می‌گویی.»
 
\chapter{41}

\par 1 و در ماه هفتم واقع شد که اسماعیل بن نتنیا ابن الیشاماع که از نسل پادشاهان بود با بعضی از روسای پادشاه و ده نفر همراهش نزد جدلیا ابن اخیقام به مصفه آمدند و آنجا درمصفه با هم نان خوردند.
\par 2 و اسماعیل بن نتنیا وآن ده نفر که همراهش بودند برخاسته، جدلیا ابن اخیقام بن شافان را به شمشیر زدند و او را که پادشاه بابل به حکومت زمین نصب کرده بودکشت.
\par 3 و اسماعیل تمامی یهودیانی را که همراه او یعنی با جدلیا در مصفه بودند و کلدانیانی را که در آنجا یافت شدند و مردان جنگی را کشت.
\par 4 و در روز دوم بعد از آنکه جدلیا را کشته بودو کسی از آن اطلاع نیافته بود،
\par 5 هشتاد نفر با ریش تراشیده و گریبان دریده و بدن خراشیده هدایا وبخور با خود آورده، از شکیم و شیلوه و سامره آمدند تا به خانه خداوند ببرند.
\par 6 و اسماعیل بن نتنیا به استقبال ایشان از مصفه بیرون آمد و دررفتن گریه می‌کرد و چون به ایشان رسید گفت: «نزد جدلیا ابن اخیقام بیایید.»
\par 7 و هنگامی که ایشان به میان شهر رسیدند اسماعیل بن نتنیا وکسانی که همراهش بودند ایشان را کشته، درحفره انداختند.
\par 8 اما در میان ایشان ده نفر پیداشدند که به اسماعیل گفتند: «ما را مکش زیرا که ما را ذخیره‌ای از گندم و جو و روغن و عسل درصحرا می‌باشد.» پس ایشان را واگذاشته، در میان برادران ایشان نکشت.
\par 9 و حفره‌ای که اسماعیل بدنهای همه کسانی را که به‌سبب جدلیا کشته در آن انداخته بود همان است که آسا پادشاه به‌سبب بعشا پادشاه اسرائیل ساخته بود و اسماعیل بن نتنیا آن را از کشتگان پرکرد.
\par 10 پس اسماعیل تمامی بقیه قوم را که درمصفه بودند با دختران پادشاه و جمیع کسانی که در مصفه باقی‌مانده بودند که نبوزردان رئیس جلادان به جدلیا ابن اخیقام سپرده بود، اسیر ساخت و اسماعیل بن نتنیا ایشان را اسیر ساخته، می‌رفت تا نزد بنی عمون بگذرد.
\par 11 اما چون یوحانان بن قاریح و تمامی سرداران لشکری که همراهش بودند از تمامی فتنه‌ای که اسماعیل بن نتنیا کرده بود خبر یافتند،
\par 12 آنگاه جمیع کسان خود را برداشتند و به قصد مقاتله با اسماعیل بن نتنیا روانه شده، او را نزد دریاچه بزرگ که درجبعون است یافتند.
\par 13 و چون جمیع کسانی که بااسماعیل بودند یوحانان بن قاریح و تمامی سرداران لشکر را که همراهش بودند دیدندخوشحال شدند.
\par 14 و تمامی کسانی که اسماعیل از مصفه به اسیری می‌برد روتافته، برگشتند و نزدیوحانان بن قاریح آمدند.
\par 15 اما اسماعیل بن نتنیابا هشت نفر از دست یوحانان فرار کرد و نزدبنی عمون رفت.
\par 16 و یوحانان بن قاریح با همه سرداران لشکرکه همراهش بودند، تمامی بقیه قومی را که ازدست اسماعیل بن نتنیا از مصفه بعد از کشته شدن جدلیا ابن اخیقام خلاصی داده بود بگرفت، یعنی مردان دلیر جنگی و زنان و اطفال و خواجه‌سرایان را که ایشان را در جبعون خلاصی داده بود؛
\par 17 و ایشان رفته، در جیروت کمهام که نزدبیت لحم است منزل گرفتند تا بروند و به مصرداخل شوند،به‌سبب کلدانیان زیرا که از ایشان می‌ترسیدند چونکه اسماعیل بن نتنیا جدلیا ابن اخیقام را که پادشاه بابل او را حاکم زمین قرار داده بود کشته بود.
\par 18 به‌سبب کلدانیان زیرا که از ایشان می‌ترسیدند چونکه اسماعیل بن نتنیا جدلیا ابن اخیقام را که پادشاه بابل او را حاکم زمین قرار داده بود کشته بود.
 
\chapter{42}

\par 1 پس تمامی سرداران لشکر و یوحانان بن قاریح و یزنیا ابن هوشعیا و تمامی خلق از خرد و بزرگ پیش آمدند،
\par 2 و به ارمیا نبی گفتند: «تمنا اینکه التماس ما نزد تو پذیرفته شودو به جهت ما و به جهت تمامی این بقیه نزد یهوه خدای خود مسالت نمایی زیرا که ما قلیلی ازکثیر باقی‌مانده‌ایم چنانکه چشمانت ما را می‌بیند.
\par 3 تا یهوه خدایت ما را به راهی که باید برویم و به‌کاری که باید بکنیم اعلام نماید.»
\par 4 پس ارمیای نبی به ایشان گفت: «شنیدم. اینک من برحسب آنچه به من گفته‌اید، نزد یهوه خدای شما مسالت خواهم نمود و هر‌چه خداوند در جواب شما بگوید به شما اطلاع خواهم داد و چیزی از شما باز نخواهم داشت.»
\par 5 ایشان به ارمیا گفتند: «خداوند در میان ماشاهد راست و امین باشد که برحسب تمامی کلامی که یهوه خدایت به واسطه تو نزد ما بفرستدعمل خواهیم نمود.
\par 6 خواه نیکو باشد و خواه بد، کلام یهوه خدای خود را که تو را نزد اومی فرستیم اطاعت خواهیم نمود تا آنکه قول یهوه خدای خود را اطاعت نموده، برای ماسعادتمندی بشود.»
\par 7 و بعد از ده روز واقع شد که کلام خداوند برارمیا نازل شد.
\par 8 پس یوحانان بن قاریح و همه سرداران لشکر که همراهش بودند و تمامی قوم رااز کوچک و بزرگ خطاب کرده،
\par 9 به ایشان گفت: «یهوه خدای اسرائیل که شما مرا نزد وی فرستادید تا دعای شما را به حضور او برسانم چنین می‌فرماید:
\par 10 اگر فی الحقیقه در این زمین بمانید آنگاه شما را بنا نموده، منهدم نخواهم ساخت و غرس کرده، نخواهم کند، زیرا از بلایی که به شما رسانیدم پشیمان شدم.
\par 11 از پادشاه بابل که از او بیم دارید ترسان مباشید. بلی خداوندمی گوید از او ترسان مباشید زیرا که من با شماهستم تا شما را نجات‌بخشم و شما را از دست او رهایی دهم.
\par 12 و من بر شما رحمت خواهم فرمود تا او بر شما لطف نماید و شما را به زمین خودتان پس بفرستد.
\par 13 اما اگر گویید که در این زمین نخواهیم ماند و اگر سخن یهوه خدای خودرا گوش نگیرید،
\par 14 و بگویید نی بلکه به زمین مصر خواهیم رفت زیرا که در آنجا جنگ نخواهیم دید و آواز کرنا نخواهیم شنید و برای نان گرسنه نخواهیم شد و در آنجا ساکن خواهیم شد،
\par 15 پس حال بنابراین‌ای بقیه یهودا کلام خداوند را بشنوید: یهوه صبایوت خدای اسرائیل چنین می‌گوید: اگر به رفتن به مصر جازم می‌باشید و اگر در آنجا رفته، ساکن شوید، 
\par 16 آنگاه شمشیری که از آن می‌ترسید البته آنجادر مصر به شما خواهد رسید و قحطی که از آن هراسان هستید آنجا در مصر شما را خواهددریافت و در آنجا خواهید مرد.
\par 17 و جمیع کسانی که برای رفتن به مصر و سکونت در آنجاجازم شده‌اند، از شمشیر و قحط و وبا خواهندمرد و احدی از ایشان از آن بلایی که من بر ایشان می‌رسانم باقی نخواهد ماند و خلاصی نخواهدیافت.
\par 18 زیرا که یهوه صبایوت خدای اسرائیل چنین می‌گوید: چنانکه خشم و غضب من برساکنان اورشلیم ریخته شد، همچنان غضب من به مجرد ورود شما به مصر بر شما ریخته خواهدشد و شما مورد نفرین و دهشت و لعنت و عارخواهید شد و این مکان را دیگر نخواهید دید.»
\par 19 ‌ای بقیه یهودا خداوند به شما می‌گوید به مصرمروید، یقین بدانید که من امروز شما را تهدیدنمودم.
\par 20 زیرا خویشتن را فریب دادید چونکه مرا نزد یهوه خدای خود فرستاده، گفتید که برای ما نزد یهوه خدای ما مسالت نما و ما را موافق هرآنچه یهوه خدای ما بگوید، مخبر ساز و آن را بعمل خواهیم آورد.
\par 21 پس امروز شما را مخبرساختم اما شما نه به قول یهوه خدای خود و نه بهیچ‌چیزی که به واسطه من نزد شما فرستادگوش گرفتید.پس الان یقین بدانید که شما درمکانی که می‌خواهید بروید و در آن ساکن شویداز شمشیر و قحط و وبا خواهید مرد.»
\par 22 پس الان یقین بدانید که شما درمکانی که می‌خواهید بروید و در آن ساکن شویداز شمشیر و قحط و وبا خواهید مرد.»
 
\chapter{43}

\par 1 و چون ارمیا فارغ شد از گفتن به تمامی قوم، تمامی کلام یهوه خدای ایشان راکه یهوه خدای ایشان آن را به واسطه او نزد ایشان فرستاده بود یعنی جمیع این سخنان را،
\par 2 آنگاه عزریا ابن هوشعیا و یوحانان بن قاریح و جمیع مردان متکبر، ارمیا را خطاب کرده، گفتند: «تودروغ می‌گویی، یهوه خدای ما تو را نفرستاده است تا بگویی به مصر مروید و در آنجا سکونت منمایید.
\par 3 بلکه باروک بن نیریا تو را بر مابرانگیخته است تا ما را به‌دست کلدانیان تسلیم نموده، ایشان ما را بکشند و به بابل به اسیری ببرند.»
\par 4 و یوحانان بن قاریح و همه سرداران لشکر وتمامی قوم فرمان خداوند را که در زمین یهودابمانند، اطاعت ننمودند.
\par 5 بلکه یوحانان بن قاریح و همه سرداران لشکر، بقیه یهودا را که از میان تمامی امت هایی که در میان آنها پراکنده شده بودند برگشته، در زمین یهودا ساکن شده بودندگرفتند.
\par 6 یعنی مردان و زنان و اطفال و دختران پادشاه و همه کسانی را که نبوزردان رئیس جلادان، به جدلیا ابن اخیقام بن شافان سپرده بود وارمیای نبی و باروک بن نیریا را.
\par 7 و به زمین مصررفتند زیرا که قول خداوند را گوش نگرفتند و به تحفنحیس آمدند.
\par 8 پس کلام خداوند در تحفنحیس بر ارمیانازل شده، گفت:
\par 9 «سنگهای بزرگ به‌دست خودبگیر و آنها را در نظر مردان یهودا در سعه‌ای که نزد دروازه خانه فرعون در تحفنحیس است با گچ بپوشان.
\par 10 و به ایشان بگو که یهوه صبایوت خدای اسرائیل چنین می‌گوید: اینک من فرستاده، بنده خود نبوکدرصر پادشاه بابل را خواهم گرفت و کرسی او را بر این سنگهایی که پوشانیدم خواهم نهاد و او سایبان خود را بر آنها خواهدبرافراشت.
\par 11 و آمده، زمین مصر را خواهد زد وآنانی را که مستوجب موت‌اند به موت و آنانی راکه مستوجب اسیری‌اند به اسیری و آنانی را که مستوجب شمشیرند به شمشیر (خواهد سپرد).
\par 12 و آتشی در خانه های خدایان مصر خواهم افروخت و آنها را خواهد سوزانید و به اسیری خواهد برد و خویشتن را به زمین مصر ملبس خواهد ساخت مثل شبانی که خویشتن را به‌جامه خود ملبس سازد و از آنجا به سلامتی بیرون خواهد رفت.و تمثالهای بیت شمس را که درزمین مصر است خواهد شکست و خانه های خدایان مصر را به آتش خواهد سوزانید.»
\par 13 و تمثالهای بیت شمس را که درزمین مصر است خواهد شکست و خانه های خدایان مصر را به آتش خواهد سوزانید.»
 
\chapter{44}

\par 1 کلامی که درباره تمامی یهود که درزمین مصر ساکن بودند و در مجدل وتحفنحیس و نوف و زمین فتروس سکونت داشتند، به ارمیا نازل شده، گفت:
\par 2 «یهوه صبایوت خدای اسرائیل چنین می‌فرماید: شماتمامی بلایی را که من بر اورشلیم و تمامی شهرهای یهودا وارد آوردم دیدید که اینک امروز خراب شده است و ساکنی در آنها نیست.
\par 3 به‌سبب شرارتی که کردند و خشم مرا بهیجان آوردند از اینکه رفته، بخور‌سوزانیدند و خدایان غیر را که نه ایشان و نه شما و نه پدران شما آنها راشناخته بودید عبادت نمودند.
\par 4 و من جمیع بندگان خود انبیا را نزد شما فرستادم و صبح زودبرخاسته، ایشان را ارسال نموده، گفتم این رجاست را که من از آن نفرت دارم بعمل نیاورید.
\par 5 اما ایشان نشنیدند و گوش خود را فرا نداشتند تااز شرارت خود بازگشت نمایند و برای خدایان غیر بخور نسوزانند.
\par 6 بنابراین خشم و غضب من ریخته و بر شهرهای یهودا و کوچه های اورشلیم افروخته گردید که آنها مثل امروز خراب و ویران گردیده است.
\par 7 پس حال یهوه خدای صبایوت خدای اسرائیل چنین می‌گوید: شما چرا این شرارت عظیم را بر جان خود وارد می‌آورید تاخویشتن را از مرد و زن و طفل و شیرخواره ازمیان یهودا منقطع سازید و از برای خود بقیه‌ای نگذارید؟
\par 8 زیرا که در زمین مصر که به آنجا برای سکونت رفته‌اید برای خدایان غیر بخورسوزانیده، خشم مرا به اعمال دستهای خود به هیجان می‌آورید تا من شما را منقطع سازم و شمادر میان تمامی امت های زمین مورد لعنت و عاربشوید.
\par 9 آیا شرارت پدران خود و شرارت پادشاهان یهودا و شرارت زنان ایشان و شرارت خود و شرارت زنان خویش را که در زمین یهودا وکوچه های اورشلیم بعمل آوردید، فراموش کرده‌اید؟
\par 10 و تا امروز متواضع نشده و ترسان نگشته‌اند و به شریعت و فرایض من که به حضورشما و به حضور پدران شما گذاشته‌ام، سالک نگردیده‌اند.
\par 11 «بنابراین یهوه صبایوت خدای اسرائیل چنین می‌گوید: اینک من روی خود را بر شما به بلا می‌گردانم تا تمامی یهودا را هلاک کنم.
\par 12 وبقیه یهودا را که رفتن به مصر و ساکن شدن در آنجارا جزم نموده‌اند، خواهم گرفت تا جمیع ایشان در زمین مصر هلاک شوند. و ایشان به شمشیر وقحط خواهند افتاد و از خرد و بزرگ به شمشیر وقحط تلف شده، خواهند مرد و مورد نفرین ودهشت و لعنت و عار خواهند گردید.
\par 13 و به آنانی که در زمین مصر ساکن شوند به شمشیروقحط و وبا عقوبت خواهم رسانید. چنانکه به اورشلیم عقوبت رسانیدم.
\par 14 و از بقیه یهودا که به زمین مصر رفته، در آنجا سکونت پذیرند احدی خلاصی نخواهد یافت و باقی نخواهد ماند تا به زمین یهودا که ایشان مشتاق برگشتن و ساکن شدن در آنجا خواهند شد مراجعت نماید. زیرا احدی از ایشان غیر از ناجیان مراجعت نخواهد کرد.»
\par 15 آنگاه تمامی مردانی که آگاه بودند که زنان ایشان برای خدایان غیر بخور می‌سوزانند وجمیع زنانی که حاضر بودند با گروهی عظیم وتمامی کسانی که در زمین مصر در فتروس ساکن بودند، در جواب ارمیا گفتند:
\par 16 «ما تو را در این کلامی که به اسم خداوند به ما گفتی گوش نخواهیم گرفت.
\par 17 بلکه بهر چیزی که از دهان ماصادر شود البته عمل خواهیم نمود و برای ملکه آسمان بخور‌سوزانیده، هدیه ریختنی به جهت او خواهیم ریخت چنانکه خود ما و پدران ما وپادشاهان و سروران ما در شهرهای یهودا وکوچه های اورشلیم می‌کردیم. زیرا که در آن زمان از نان سیر شده، سعادتمند می‌بودیم و بلا رانمی دیدیم.
\par 18 اما از زمانی که بخور‌سوزانیدن رابرای ملکه آسمان و ریختن هدایای ریختنی را به جهت او ترک نمودیم، محتاج همه‌چیز شدیم و به شمشیر و قحط هلاک گردیدیم.
\par 19 و چون به جهت ملکه آسمان بخور می‌سوزانیدیم و هدیه ریختنی برای او می‌ریختیم، آیا بی‌اطلاع شوهران خویش قرصها به شبیه او می‌پختیم وهدیه ریختنی به جهت او می‌ریختیم؟»
\par 20 پس ارمیا تمامی قوم را از مردان و زنان وهمه کسانی که این جواب را بدو داده بودندخطاب کرده، گفت:
\par 21 «آیا خداوند بخوری را که شما و پدران شما و پادشاهان و سروران شما واهل ملک در شهرهای یهودا و کوچه های اورشلیم سوزانیدند، بیاد نیاورده و آیا به‌خاطر اوخطور نکرده است؟
\par 22 چنانکه خداوند به‌سبب شرارت اعمال شما و رجاساتی که بعمل آورده بودید، دیگر نتوانست تحمل نماید. لهذا زمین شما ویران و مورد دهشت و لعنت و غیرمسکون گردیده، چنانکه امروز شده است.
\par 23 چونکه بخور‌سوزانیدید و به خداوند گناه ورزیده، به قول خداوند گوش ندادید و به شریعت و فرایض و شهادات او سلوک ننمودید، بنابراین این بلا مثل امروز بر شما وارد شده است.»
\par 24 و ارمیا به تمامی قوم و به جمیع زنان گفت: «ای تمامی یهودا که در زمین مصر هستید کلام خداوند را بشنوید!
\par 25 یهوه صبایوت خدای اسرائیل چنین می‌گوید: شما و زنان شما هم بادهان خود تکلم می‌نمایید و هم با دستهای خودبجا می‌آورید و می‌گویید نذرهایی را که کردیم البته وفا خواهیم نمود و بخور برای ملکه آسمان خواهیم سوزانید و هدایای ریختنی برای اوخواهیم ریخت. پس نذرهای خود را استوارخواهید کرد و نذرهای خود را وفا خواهید نمود.
\par 26 بنابراین‌ای تمامی یهودا که در زمین مصرساکن هستید کلام خداوند را بشنوید. اینک خداوند می‌گوید: من به اسم عظیم خود قسم خوردم که اسم من بار دیگر به دهان هیچکدام ازیهود در تمامی زمین مصر آورده نخواهد شد ونخواهند گفت: به حیات خداوند یهوه قسم.
\par 27 اینک من بر ایشان به بدی مراقب خواهم بود ونه به نیکویی تا جمیع مردان یهودا که در زمین مصر می‌باشند به شمشیر و قحط هلاک شده، تمام شوند.
\par 28 لیکن عدد قلیلی از شمشیر رهایی یافته، از زمین مصر به زمین یهودا مراجعت خواهند نمود و تمامی بقیه یهودا که به جهت سکونت آنجا در زمین مصر رفته‌اند، خواهنددانست که کلام کدام‌یک از من و ایشان استوارخواهد شد.
\par 29 و خداوند می‌گوید: این است علامت برای شما که من در اینجا به شما عقوبت خواهم رسانید تا بدانید که کلام من درباره شماالبته به بدی استوار خواهد شد.خداوند چنین می‌گوید: اینک من فرعون حفرع پادشاه مصر را به‌دست دشمنانش و به‌دست آنانی که قصد جان اودارند تسلیم خواهم کرد. چنانکه صدقیا پادشاه یهودا را به‌دست دشمنش نبوکدرصر پادشاه بابل که قصد جان او می‌داشت، تسلیم نمودم.»
\par 30 خداوند چنین می‌گوید: اینک من فرعون حفرع پادشاه مصر را به‌دست دشمنانش و به‌دست آنانی که قصد جان اودارند تسلیم خواهم کرد. چنانکه صدقیا پادشاه یهودا را به‌دست دشمنش نبوکدرصر پادشاه بابل که قصد جان او می‌داشت، تسلیم نمودم.»
 
\chapter{45}

\par 1 کلامی که ارمیا نبی به باروک بن نیریاخطاب کرده، گفت، هنگامی که این سخنان را از دهان ارمیا در سال چهارم یهویاقیم بن یوشیا پادشاه یهودا در طومار نوشت:
\par 2 «ای باروک یهوه خدای اسرائیل به تو چنین می فرماید:
\par 3 تو گفته‌ای وای بر من زیرا خداوند بردرد من غم افزوده است. از ناله کشیدن خسته شده‌ام و استراحت نمی یابم.
\par 4 او را چنین بگو، خداوند چنین می‌فرماید: آنچه بنا کرده‌ام، منهدم خواهم ساخت و آنچه غرس نموده‌ام یعنی تمامی این زمین را، از ریشه خواهم کند.و آیا توچیزهای بزرگ برای خویشتن می‌طلبی؟ آنها راطلب منما زیرا خداوند می‌گوید: اینک من برتمامی بشر بلا خواهم رسانید. اما در هر جایی که بروی جانت را به تو به غنیمت خواهم بخشید.»
\par 5 و آیا توچیزهای بزرگ برای خویشتن می‌طلبی؟ آنها راطلب منما زیرا خداوند می‌گوید: اینک من برتمامی بشر بلا خواهم رسانید. اما در هر جایی که بروی جانت را به تو به غنیمت خواهم بخشید.»
 
\chapter{46}

\par 1 کلام خداوند درباره امت‌ها که به ارمیانازل شد؛
\par 2 درباره مصر و لشکر فرعون نکو که نزد نهر فرات در کرکمیش بودند ونبوکدرصر پادشاه بابل ایشان را در سال چهارم یهویاقیم بن یوشیا پادشاه یهودا شکست داد:
\par 3 «مجن و سپر را حاضر کنید و برای جنگ نزدیک آیید.
\par 4 ‌ای سواران اسبان را بیارایید و سوار شویدو با خودهای خود بایستید. نیزه‌ها را صیقل دهیدو زره‌ها را بپوشید. 
\par 5 خداوند می‌گوید: چرا ایشان را می‌بینم که هراسان شده، به عقب برمی گردند وشجاعان ایشان خرد شده، بالکل منهزم می‌شوندو به عقب نمی نگرند، زیرا که خوف از هر طرف می‌باشد.
\par 6 تیزروان فرار نکنند و زورآوران رهایی نیابند. بطرف شمال به کنار نهر فرات می‌لغزند ومی افتند.
\par 7 این کیست که مانند رود نیل سیلان کرده است و آبهای او مثل نهرهایش متلاطم می‌گردد؟
\par 8 مصر مانند رود نیل سیلان کرده است و آبهایش مثل نهرها متلاطم گشته، می‌گوید: من سیلان کرده، زمین را خواهم پوشانید و شهر وساکنانش را هلاک خواهم ساخت.
\par 9 ‌ای اسبان، برآیید و‌ای ارابه‌ها تند بروید و شجاعان بیرون بروند. ای اهل حبش و فوت که سپرداران هستیدو‌ای لودیان که کمان را می‌گیرید و آن رامی کشید.
\par 10 زیرا که آن روز روز انتقام خداوندیهوه صبایوت می‌باشد که از دشمنان خود انتقام بگیرد. پس شمشیر هلاک کرده، سیر می‌شود و ازخون ایشان مست می‌گردد. زیرا خداوند یهوه صبایوت در زمین شمال نزد نهر فرات ذبحی دارد.
\par 11 ‌ای باکره دختر مصر به جلعاد برآی وبلسان بگیر. درمانهای زیاد را عبث به‌کار می‌بری. برای تو علاج نیست.
\par 12 امت‌ها رسوایی تو رامی شنوند و جهان از ناله تو پر شده است زیرا که شجاع بر شجاع می‌لغزد و هر دوی ایشان با هم می‌افتند.»
\par 13 کلامی که خداوند درباره آمدن نبوکدرصرپادشاه بابل و مغلوب ساختن زمین مصر به ارمیانبی گفت:
\par 14 «به مصر خبر دهید و به مجدل اعلام نمایید و به نوف و تحفنحیس اطلاع دهید. بگویید برپا شوید و خویشتن را آماده سازید زیراکه شمشیر مجاورانت را هلاک کرده است.
\par 15 زورآورانت چرا به زیر افکنده می‌شوند ونمی توانند ایستاد؟ زیرا خداوند ایشان را پراکنده ساخته است.
\par 16 بسیاری را لغزانیده است و ایشان بر یکدیگر می‌افتند، و می‌گویند: برخیزید و ازشمشیر بران نزد قوم خود و به زمین مولد خویش برگردیم.
\par 17 در آنجا فرعون، پادشاه مصر را هالک می‌نامند و فرصت را از دست داده است.
\par 18 پادشاه که نام او یهوه صبایوت می‌باشد می گوید به حیات خودم قسم که او مثل تابور، درمیان کوهها و مانند کرمل، نزد دریا خواهد آمد.
\par 19 ‌ای دختر مصر که در (امنیت ) ساکن هستی، اسباب جلای وطن را برای خود مهیا ساز زیرا که نوف ویران و سوخته و غیرمسکون گردیده است.
\par 20 مصر گوساله بسیار نیکو منظر است اما هلاکت از طرف شمال می‌آید و می‌آید.
\par 21 سپاهیان به مزد گرفته او در میانش مثل گوساله های پرواری می‌باشند. زیرا که ایشان نیز روتافته، با هم فرارمی کنند و نمی ایستند. چونکه روز هلاکت ایشان و وقت عقوبت ایشان بر ایشان رسیده است.
\par 22 آوازه آن مثل مار می‌رود زیرا که آنها با قوت می‌خرامند و با تبرها مثل چوب بران بر اومی آیند.
\par 23 خداوند می‌گوید که جنگل او را قطع خواهند نمود اگر‌چه لایحصی می‌باشد. زیرا که ایشان از ملخها زیاده و از حد شماره افزونند.
\par 24 دختر مصر خجل شده، به‌دست قوم شمالی تسلیم گردیده است.
\par 25 یهوه صبایوت خدای اسرائیل می‌گوید: اینک من بر آمون نو و فرعون ومصر و خدایانش و پادشاهانش یعنی بر فرعون وآنانی که بر وی توکل دارند، عقوبت خواهم رسانید.
\par 26 و خداوند می‌گوید که ایشان را به‌دست آنانی که قصد جان ایشان دارند، یعنی به‌دست نبوکدرصر پادشاه بابل و به‌دست بندگانش تسلیم خواهم کرد و بعد از آن، مثل ایام سابق مسکون خواهد شد.
\par 27 اما تو‌ای بنده من یعقوب مترس و‌ای اسرائیل هراسان مشو زیرا اینک من تو را از جای دور و ذریت تو را از زمین اسیری ایشان نجات خواهم داد و یعقوب برگشته، درامنیت و استراحت خواهد بود و کسی او را نخواهد ترسانید.و خداوند می‌گوید: ای بنده من یعقوب مترس زیرا که من با تو هستم و اگر‌چه تمام امت‌ها را که تو را در میان آنها پراکنده ساخته‌ام بالکل هلاک سازم لیکن تو را بالکل هلاک نخواهم ساخت. بلکه تو را به انصاف تادیب خواهم نمود و تو را هرگز بی‌سزا نخواهم گذاشت.»
\par 28 و خداوند می‌گوید: ای بنده من یعقوب مترس زیرا که من با تو هستم و اگر‌چه تمام امت‌ها را که تو را در میان آنها پراکنده ساخته‌ام بالکل هلاک سازم لیکن تو را بالکل هلاک نخواهم ساخت. بلکه تو را به انصاف تادیب خواهم نمود و تو را هرگز بی‌سزا نخواهم گذاشت.»
 
\chapter{47}

\par 1 کلام خداوند درباره فلسطینیان که برارمیا نبی نازل شد قبل از آنکه فرعون غزه را مغلوب بسازد.
\par 2 خداوند چنین می‌گوید: «اینک آبها از شمال برمی آید و مثل نهری سیلان می‌کند و زمین را با آنچه در آن است و شهر وساکنانش را درمی گیرد. و مردمان فریادبرمی آورند و جمیع سکنه زمین ولوله می‌نمایند
\par 3 از صدای سمهای اسبان زورآورش و ازغوغای ارابه هایش و شورش چرخهایش. وپدران به‌سبب سستی دستهای خود به فرزندان خویش اعتنا نمی کنند.
\par 4 به‌سبب روزی که برای هلاکت جمیع فلسطینیان می‌آید که هرنصرت کننده‌ای را که باقی می‌ماند از صور وصیدون منقطع خواهد ساخت. زیرا خداوندفلسطینیان یعنی بقیه جزیره کفتور را هلاک خواهد ساخت.
\par 5 اهل غزه بریده مو گشته‌اند واشقلون و بقیه وادی ایشان هلاک شده است. تا به کی بدن خود را خواهی خراشید؟
\par 6 آه‌ای شمشیر خداوند تا به کی آرام نخواهی گرفت؟ به غلاف خود برگشته، مستریح و آرام شو.چگونه می‌توانی آرام بگیری، با آنکه خداوند تو را بر اشقلون و بر ساحل دریا مامور فرموده و تو را به آنجا تعیین نموده است؟»
\par 7 چگونه می‌توانی آرام بگیری، با آنکه خداوند تو را بر اشقلون و بر ساحل دریا مامور فرموده و تو را به آنجا تعیین نموده است؟»
 
\chapter{48}

\par 1 درباره موآب، یهوه صبایوت خدای اسرائیل چنین می‌گوید: «وای بر نبوزیرا که خراب شده است. قریه تایم خجل وگرفتار گردیده است. و مسجاب رسوا و منهدم گشته است.
\par 2 فخر موآب زایل شده، در حشبون برای وی تقدیرهای بد کردند. بیایید و او رامنقطع سازیم تا دیگر قوم نباشد. تو نیز‌ای مدمین ساکت خواهی شد و شمشیر تو را تعاقب خواهدنمود.
\par 3 آواز ناله از حورونایم مسموع می‌شود. هلاکت و شکستگی عظیم.
\par 4 موآب بهم شکسته است و صغیرهای او فریاد برمی آورند.
\par 5 زیرا که به فراز لوحیت با گریه سخت برمی آیند و ازسرازیری حورونایم صدای شکست یافتن ازدشمنان شنیده می‌شود.
\par 6 بگریزید و جانهای خود را برهانید و مثل درخت عرعر در بیابان باشید.
\par 7 زیرا از این جهت که به اعمال و گنجهای خویش توکل نمودی تو نیز گرفتار خواهی شد. وکموش با کاهنان و سرورانش با هم به اسیری خواهند رفت.
\par 8 و غارت کننده به همه شهرهاخواهد آمد و هیچ شهر خلاصی نخواهد یافت وبرحسب فرمان خداوند اهل وادی تلف خواهندشد و اهل همواری هلاک خواهند گردید.
\par 9 بالهابه موآب بدهید تا پرواز نموده، بگریزد وشهرهایش خراب و غیرمسکون خواهد شد.
\par 10 ملعون باد کسی‌که کار خداوند را با غفلت عمل نماید و ملعون باد کسی‌که شمشیر خود را از خون باز‌دارد.
\par 11 موآب از طفولیت خودمستریح بوده و بر دردهای خود نشسته است و ازظرف به ظرف ریخته نشده و به اسیری نرفته است. از این سبب طعمش در او مانده است وخوشبویی او تغییر نیافته است.
\par 12 بنابراین اینک خداوند می‌گوید: روزها می‌آید که من ریزندگان می‌فرستم که او را بریزند و ظروف او را خالی کرده، مشکهایش را پاره خواهند نمود.
\par 13 وموآب از کموش شرمنده خواهد شد چنانکه خاندان اسرائیل از بیت ئیل که اعتماد ایشان بود، شرمنده شده‌اند.
\par 14 چگونه می‌گویید که ماشجاعان و مردان قوی برای جنگ می‌باشیم؟
\par 15 موآب خراب شده دود شهرهایش متصاعدمی شود و جوانان برگزیده‌اش به قتل فرودمی آیند. پادشاه که نام او یهوه صبایوت می‌باشداین را می‌گوید:
\par 16 رسیدن هلاکت موآب نزدیک است و بلای او بزودی هر‌چه تمامتر می‌آید.
\par 17 ‌ای جمیع مجاورانش و همگانی که نام او رامی دانید برای وی ماتم گیرید. بگویید عصای قوت و چوبدستی زیبایی چگونه شکسته شده است!
\par 18 ‌ای دختر دیبون که (در امنیت ) ساکن هستی از جلال خود فرود آی و در جای خشک بنشین زیرا که غارت کننده موآب بر تو هجوم می‌آورد و قلعه های تو را منهدم می‌سازد.
\par 19 ‌ای تو که در عروعیر ساکن هستی به‌سر راه بایست ونگاه کن و از فراریان و ناجیان بپرس و بگو که چه شده است؟
\par 20 موآب خجل شده، زیرا که شکست یافته است پس ولوله و فریاد برآورید. در ارنون اخبار نمایید که موآب هلاک گشته است.
\par 21 و داوری بر زمین همواری رسیده است.
\par 22 و بر دیبون و نبوو بیت دبلتایم،
\par 23 و بر قریه تایم و بیت جامول وبیت معون،
\par 24 و بر قریوت و بصره و بر تمامی شهرهای بعید و قریب زمین موآب.
\par 25 «خداوند می‌گوید که شاخ موآب بریده وبازویش شکسته شده است.
\par 26 او را مست سازیدزیرا به ضد خداوند تکبر می‌نماید. و موآب درقی خود غوطه می‌خورد و او نیز مضحکه خواهدشد.
\par 27 آیا اسرائیل برای تو مضحکه نبود؟ و آیااو در میان دزدان یافت شد به حدی که هر وقت که درباره او سخن می‌گفتی سر خود رامی جنبانیدی؟
\par 28 ‌ای ساکنان موآب شهرها راترک کرده، در صخره ساکن شوید و مثل فاخته‌ای باشید که آشیانه خود را در کنار دهنه مغاره می‌سازد.
\par 29 غرور موآب و بسیاری تکبر او را وعظمت و خیلا و کبر و بلندی دل او را شنیدیم.
\par 30 خداوند می‌گوید: خشم او را می‌دانم که هیچ است و فخرهای او را که از آنها هیچ برنمی آید.
\par 31 بنابراین برای موآب ولوله خواهم کرد و به جهت تمامی موآب فریاد برخواهم آورد. برای مردان قیرحارس ماتم گرفته خواهد شد.
\par 32 برای تو‌ای مو سبمه به گریه یعزیر خواهم گریست. شاخه های تو از دریا گذشته بود و به دریاچه یعزیر رسیده، بر میوه‌ها و انگورهایت غارت کننده هجوم آورده است.
\par 33 شادی وابتهاج از بستانها و زمین موآب برداشته شد وشراب را از چرخشتها زایل ساختم و کسی آنهارا به صدای شادمانی به پا نخواهد فشرد. صدای شادمانی صدای شادمانی نیست.
\par 34 به فریادحشبون آواز خود را تا العاله و یاهص بلند کردندو از صوغر تا حورونایم عجلت شلیشیا، زیرا که آبهای نمریم نیز خرابه شده است.
\par 35 و خداوند می گوید من آنانی را که در مکان های بلند قربانی می‌گذرانند و برای خدایان خود بخورمی سوزانند از موآب نابود خواهم گردانید.
\par 36 لهذا دل من به جهت موآب مثل نای صدامی کند و دل من به جهت مردان قیرحارس مثل نای صدا می‌کند، چونکه دولتی که تحصیل نمودند تلف شده است.
\par 37 و هر سر بی‌مو گشته وهر ریش تراشیده شده است و همه دستهاخراشیده و بر هر کمر پلاس است.
\par 38 بر همه پشت بامهای موآب و در جمیع کوچه هایش ماتم است زیرا خداوند می‌گوید موآب را مثل ظرف ناپسند شکسته‌ام.
\par 39 چگونه منهدم شده و ایشان چگونه ولوله می‌کنند؟ و موآب چگونه به رسوایی پشت داده است؟ پس موآب برای جمیع مجاوران خود مضحکه و باعث ترس شده است.
\par 40 زیرا خداوند چنین می‌گوید: او مثل عقاب پرواز خواهد کرد و بالهای خویش را بر موآب پهن خواهد نمود.
\par 41 شهرهایش گرفتار وقلعه هایش تسخیرشده، و دل شجاعان موآب درآن روز مثل دل زنی که درد زه داشته باشد خواهدشد.
\par 42 و موآب خراب شده، دیگر قوم نخواهدبود چونکه به ضد خداوند تکبر نموده است.
\par 43 خداوند می‌گوید: ای ساکن موآب خوف وحفره و دام پیش روی تو است.
\par 44 آنکه از ترس بگریزد در حفره خواهد افتاد و آنکه از حفره برآید گرفتار دام خواهد شد، زیرا خداوندفرموده است که سال عقوبت ایشان را بر ایشان یعنی بر موآب خواهم آورد.
\par 45 فراریان بی‌تاب شده، در سایه حشبون ایستاده‌اند زیرا که آتش ازحشبون و نار از میان سیحون بیرون آمده، حدودموآب و فرق سر فتنه انگیزان را خواهد سوزانید.
\par 46 وای بر تو‌ای موآب! قوم کموش هلاک شده‌اند زیرا که پسرانت به اسیری و دخترانت به جلای وطن گرفتار گردیده‌اند.لیکن خداوندمی گوید که در ایام آخر، اسیران موآب را بازخواهم آورد. حکم درباره موآب تا اینجاست.» 
\par 47 لیکن خداوندمی گوید که در ایام آخر، اسیران موآب را بازخواهم آورد. حکم درباره موآب تا اینجاست.»
 
\chapter{49}

\par 1 درباره بنی عمون، خداوند چنین می گوید: «آیا اسرائیل پسران ندارد وآیا او را وارثی نیست؟ پس چرا ملکم جاد را به تصرف آورده و قوم او در شهرهایش ساکن شده‌اند؟
\par 2 لهذا اینک خداوند می‌گوید: ایامی می‌آید که نعره جنگ را در ربه بنی عمون خواهم شنوانید و تل ویران خواهد گشت و دهاتش به آتش سوخته خواهد شد. و خداوند می‌گوید که اسرائیل متصرفان خویش را به تصرف خواهدآورد.»
\par 3 ‌ای حشبون ولوله کن، زیرا که عای خراب شده است. ای دهات ربه فریاد برآورید وپلاس پوشیده، ماتم گیرید و بر حصارها گردش نمایید. زیرا که ملکم با کاهنان و سروران خود باهم به اسیری می‌روند.
\par 4 ‌ای دختر مرتد چرا ازوادیها یعنی وادیهای برومند خود فخرمی نمایی؟ ای تو که به خزاین خود توکل می‌نمایی (و می‌گویی ) کیست که نزد من تواندآمد؟
\par 5 اینک خداوند یهوه صبایوت می‌گوید: «من از جمیع مجاورانت خوف بر تو خواهم آوردو هر یکی از شما پیش روی خود پراکنده خواهدشد و کسی نخواهد بود که پراکندگان را جمع نماید.
\par 6 لیکن خداوند می‌گوید: بعد از این اسیران بنی عمون را باز خواهم آورد.»
\par 7 درباره ادوم یهوه صبایوت چنین می‌گوید: «آیا دیگر حکمت در تیمان نیست؟ و آیامشورت از فهیمان زایل شده و حکمت ایشان نابود گردیده است؟
\par 8 ‌ای ساکنان ددان بگریزید ورو تافته در جایهای عمیق ساکن شوید. زیرا که بلای عیسو و زمان عقوبت وی را بر او خواهم آورد.
\par 9 اگر انگورچینان نزد تو آیند، آیا بعضی خوشه‌ها را نمی گذارند؟ و اگر دزدان در شب (آیند)، آیا به قدر کفایت غارت نمی نمایند؟
\par 10 اما من عیسو را برهنه ساخته و جایهای مخفی او را مکشوف گردانیده‌ام که خویشتن را نتواندپنهان کرد. ذریت او و برادران و همسایگانش هلاک شده‌اند و خودش نابود گردیده است.
\par 11 یتیمان خود را ترک کن و من ایشان را زنده نگاه خواهم داشت و بیوه‌زنانت بر من توکل بنمایند.
\par 12 زیرا خداوند چنین می‌گوید: اینک آنانی که رسم ایشان نبود که این‌جام را بنوشند، البته خواهند نوشید و آیا تو بی‌سزا خواهی ماند؟ بی‌سزا نخواهی ماند بلکه البته خواهی نوشید.
\par 13 زیرا خداوند می‌گوید به ذات خودم قسم می‌خورم که بصره مورد دهشت و عار و خرابی ولعنت خواهد شد و جمیع شهرهایش خرابه ابدی خواهد گشت.
\par 14 از جانب خداوند خبری شنیدم که رسولی نزد امت‌ها فرستاده شده، (می گوید): «جمع شوید و بر او هجوم آورید و برای جنگ برخیزید!
\par 15 زیرا که هان من تو را کوچکترین امت‌ها و در میان مردم خوار خواهم گردانید.
\par 16 ‌ای که در شکافهای صخره ساکن هستی و بلندی تلها را گرفته‌ای، هیبت تو و تکبر دلت تو رافریب داده است اگر‌چه مثل عقاب آشیانه خود رابلند بسازی، خداوند می‌گوید که من تو را از آنجافرود خواهم آورد.
\par 17 و ادوم محل تعجب خواهد گشت به حدی که هرکه از آن عبور نمایدمتحیر شده، به‌سبب همه صدماتش صفیر خواهدزد.
\par 18 خداوند می‌گوید: چنانکه سدوم و عموره و شهرهای مجاور آنها واژگون شده است، همچنان کسی در آنجا ساکن نخواهد شد واحدی از بنی آدم در آن ماوا نخواهد گزید.
\par 19 اینک او مثل شیر از طغیان اردن به آن مسکن منیع برخواهد آمد، زیرا که من وی را در لحظه‌ای از آنجا خواهم راند. و کیست آن برگزیده‌ای که اورا بر آن بگمارم؟ زیرا کیست که مثل من باشد وکیست که مرا به محاکمه بیاورد و کیست آن شبانی که به حضور من تواند ایستاد؟»
\par 20 بنابراین مشورت خداوند را که درباره ادوم نموده است و تقدیرهای او را که درباره ساکنان تیمان فرموده است بشنوید. البته ایشان صغیران گله را خواهند ربود و هر آینه مسکن ایشان رابرای ایشان خراب خواهد ساخت.
\par 21 از صدای افتادن ایشان زمین متزلزل گردید و آواز فریادایشان تا به بحر قلزم مسموع شد.
\par 22 اینک او مثل عقاب برآمده، پرواز می‌کند و بالهای خویش را بربصره پهن می‌نماید و دل شجاعان ادوم در آن روزمثل دل زنی که درد زه داشته باشد خواهد شد.
\par 23 درباره دمشق: «حمات و ارفاد خجل گردیده‌اند زیرا که خبر بد شنیده، گداخته شده‌اند.
\par 24 دمشق ضعیف شده، روبه فرار نهاده و لرزه او رادرگرفته است. آلام و دردها او را مثل زنی که می‌زاید گرفته است.
\par 25 چگونه شهر نامور و قریه ابتهاج من متروک نشده است؟
\par 26 لهذا یهوه صبایوت می‌گوید: جوانان او در کوچه هایش خواهند افتاد و همه مردان جنگی او در آن روزهلاک خواهند شد.
\par 27 و من آتش در حصارهای دمشق خواهم افروخت و قصرهای بنهدد راخواهد سوزانید.»
\par 28 درباره قیدار و ممالک حاصور که نبوکدرصر پادشاه بابل آنها را مغلوب ساخت، خداوند چنین می‌گوید: «برخیزید و برقیدارهجوم آورید و بنی مشرق را تاراج نمایید.
\par 29 خیمه‌ها و گله های ایشان را خواهند گرفت. پرده‌ها و تمامی اسباب و شتران ایشان را برای خویشتن خواهند برد و بر ایشان ندا خواهند دادکه خوف از هر طرف!
\par 30 بگریزید و به زودی هرچه تمامتر فرار نمایید. ای ساکنان حاصور درجایهای عمیق ساکن شوید.» زیرا خداوندمی گوید: «نبوکدرصر پادشاه بابل به ضد شمامشورتی کرده و به خلاف شما تدبیری نموده است.
\par 31 خداوند فرموده است که برخیزید و برامت مطمئن که در امنیت ساکن‌اند هجوم آورید. ایشان را نه دروازه‌ها و نه پشت بندها است و به تنهایی ساکن می‌باشند.
\par 32 خداوند می‌گوید که شتران ایشان تاراج و کثرت مواشی ایشان غارت خواهد شد و آنانی را که گوشه های موی خود را می تراشند بسوی هر باد پراکنده خواهم ساخت وهلاکت ایشان را از هر طرف ایشان خواهم آورد.
\par 33 و حاصور مسکن شغالها و ویرانه ابدی خواهدشد به حدی که کسی در آن ساکن نخواهد گردیدو احدی از بنی آدم در آن ماوا نخواهد گزید.»
\par 34 کلام خداوند درباره عیلام که بر ارمیا نبی در ابتدای سلطنت صدقیا پادشاه یهودا نازل شده، گفت:
\par 35 «یهوه صبایوت چنین می‌گوید: اینک من کمان عیلام و مایه قوت ایشان را خواهم شکست.
\par 36 و چهار باد را از چهار سمت آسمان بر عیلام خواهم وزانید و ایشان را بسوی همه این بادهاپراکنده خواهم ساخت به حدی که هیچ امتی نباشد که پراکندگان عیلام نزد آنها نیایند.
\par 37 واهل عیلام را به حضور دشمنان ایشان و به حضورآنانی که قصد جان ایشان دارند مشوش خواهم ساخت. و خداوند می‌گوید که بر ایشان بلا یعنی حدت خشم خویش را وارد خواهم آورد وشمشیر را در عقب ایشان خواهم فرستاد تا ایشان را بالکل هلاک سازم.
\par 38 و خداوند می‌گوید: من کرسی خود را در عیلام برپا خواهم نمود وپادشاه و سروران را از آنجا نابود خواهم ساخت.لیکن خداوند می‌گوید: در ایام آخر اسیران عیلام را باز خواهم آورد.»
\par 39 لیکن خداوند می‌گوید: در ایام آخر اسیران عیلام را باز خواهم آورد.»
 
\chapter{50}

\par 1 کلامی که خداوند درباره بابل و زمین کلدانیان به واسطه ارمیا نبی گفت:
\par 2 «درمیان امت‌ها اخبار و اعلام نمایید، علمی برافراشته، اعلام نمایید و مخفی مدارید. بگوییدکه بابل گرفتار شده، و بیل خجل گردیده است. مرودک خرد شده و اصنام او رسوا و بتهایش شکسته گردیده است
\par 3 زیرا که امتی از طرف شمال بر او می‌آید و زمینش را ویران خواهدساخت به حدی که کسی در آن ساکن نخواهد شدو هم انسان و هم بهایم فرار کرده، خواهند رفت.
\par 4 خداوند می‌گوید که در آن ایام و در آن زمان بنی‌اسرائیل و بنی یهودا با هم خواهند آمد. ایشان گریه‌کنان خواهند آمد و یهوه خدای خود راخواهند طلبید.
\par 5 و رویهای خود را بسوی صهیون نهاده، راه آن را خواهند پرسید و خواهندگفت بیایید و به عهد ابدی که فراموش نشود به خداوند ملصق شویم.
\par 6 «قوم من گوسفندان گم شده بودند و شبانان ایشان ایشان را گمراه کرده، بر کوهها آواره ساختند. از کوه به تل رفته، آرامگاه خود رافراموش کردند.
\par 7 هر‌که ایشان را می‌یافت ایشان را می‌خورد و دشمنان ایشان می‌گفتند که گناه نداریم زیرا که به یهوه که مسکن عدالت است و به یهوه که امید پدران ایشان بود، گناه ورزیدند.
\par 8 ازمیان بابل فرار کنید و از زمین کلدانیان بیرون آیید. و مانند بزهای نر پیش روی گله راه روید.
\par 9 زیرااینک من جمعیت امت های عظیم را از زمین شمال برمی انگیزانم و ایشان را بر بابل می‌آورم وایشان در برابر آن صف آرایی خواهند نمود و درآنوقت گرفتار خواهد شد. تیرهای ایشان مثل تیرهای جبار هلاک کننده که یکی از آنها خالی برنگردد خواهد بود.
\par 10 خداوند می‌گوید که کلدانیان تاراج خواهند شد و هر‌که ایشان را غارت نماید سیر خواهد گشت.
\par 11 زیرا شما‌ای غارت کنندگان میراث من شادی و وجد کردید ومانند گوساله‌ای که خرمن را پایمال کند، جست وخیز نمودید و مانند اسبان زورآور شیهه زدید.
\par 12 مادر شما بسیار خجل خواهد شد و والده شمارسوا خواهد گردید. هان او موخر امت‌ها و بیابان و زمین خشک و عربه خواهد شد.
\par 13 به‌سبب خشم خداوند مسکون نخواهد شد بلکه بالکل ویران خواهد گشت. و هر‌که از بابل عبور نمایدمتحیر شده، به جهت تمام بلایایش صفیر خواهدزد.
\par 14 ‌ای جمیع کمان داران در برابر بابل از هرطرف صف آرایی نمایید. تیرها بر او بیندازید ودریغ منمایید زیرا به خداوند گناه ورزیده است.
\par 15 از هر طرف بر او نعره زنید چونکه خویشتن راتسلیم نموده است. حصارهایش افتاده ودیوارهایش منهدم شده است زیرا که این انتقام خداوند است. پس از او انتقام بگیرید و بطوری که او عمل نموده است همچنان با او عمل نمایید.
\par 16 و از بابل، برزگران و آنانی را که داس را در زمان درو بکار می‌برند منقطع سازید. و از ترس شمشیر برنده هرکس بسوی قوم خود توجه نماید و هر کس به زمین خویش بگریزد.
\par 17 «اسرائیل مثل گوسفند، پراکنده گردید. شیران او را تعاقب کردند. اول پادشاه آشور او راخورد و آخر این نبوکدرصر پادشاه بابل استخوانهای او را خرد کرد.»
\par 18 بنابراین یهوه صبایوت خدای اسرائیل چنین می‌گوید: «اینک من بر پادشاه بابل و بر زمین او عقوبت خواهم رسانید چنانکه بر پادشاه آشور عقوبت رسانیدم.
\par 19 و اسرائیل را به مرتع خودش باز خواهم آورد و در کرمل و باشان خواهد چرید و بر کوهستان افرایم و جلعاد جان او سیر خواهد شد.
\par 20 خداوند می‌گوید که در آن ایام و در آن زمان عصیان اسرائیل را خواهند جست و نخواهد بودو گناه یهودا را اما پیدا نخواهد شد، زیرا آنانی راکه باقی می‌گذارم خواهم آمرزید.
\par 21 «بر زمین مراتایم برآی یعنی بر آن و برساکنان فقود. خداوند می‌گوید: بکش و ایشان راتعاقب نموده، بالکل هلاک کن و موافق هر‌آنچه من تو را امر فرمایم عمل نما.
\par 22 آواز جنگ وشکست عظیم در زمین است.
\par 23 کوپال تمام جهان چگونه بریده و شکسته شده و بابل در میان امت‌ها چگونه ویران گردیده است.
\par 24 ‌ای بابل ازبرای تو دام گستردم و تو نیز گرفتار شده، اطلاع نداری. یافت شده، تسخیر گشته‌ای چونکه باخداوند مخاصمه نمودی.
\par 25 خداونداسلحه خانه خود را گشوده، اسلحه خشم خویش را بیرون آورده است. زیرا خداوند یهوه صبایوت با زمین کلدانیان کاری دارد.
\par 26 بر او از همه اطراف بیایید و انبارهای او را بگشایید، او را مثل توده های انباشته بالکل هلاک سازید و چیزی ازاو باقی نماند.
\par 27 همه گاوانش را به سلاخ خانه فرود آورده، بکشید. وای بر ایشان! زیرا که یوم ایشان و زمان عقوبت ایشان رسیده است.
\par 28 آوازفراریان و نجات‌یافتگان از زمین بابل مسموع می‌شود که از انتقام یهوه خدای ما و انتقام هیکل او در صهیون اخبار می‌نمایند.
\par 29 تیراندازان را به ضد بابل جمع کنید. ای همگانی که کمان را زه می‌کنید، در برابر او از هر طرف اردو زنید تااحدی رهایی نیابد و بر وفق اعمالش او را جزا دهید و مطابق هر‌آنچه کرده است به او عمل نمایید. زیرا که به ضد خداوند و به ضد قدوس اسرائیل تکبر نموده است. 
\par 30 لهذا خداوندمی گوید که جوانانش در کوچه هایش خواهندافتاد و جمیع مردان جنگیش در آن روز هلاک خواهند شد.
\par 31 «اینک خداوند یهوه صبایوت می‌گوید: ای متکبر من بر‌ضد تو هستم. زیرا که یوم تو و زمانی که به تو عقوبت برسانم رسیده است.
\par 32 و آن متکبر لغزش خورده، خواهد افتاد و کسی او رانخواهد برخیزانید و آتش در شهرهایش خواهم افروخت که تمامی حوالی آنها را خواهدسوزانید.
\par 33 یهوه صبایوت چنین می‌گوید: بنی‌اسرائیل و بنی یهودا با هم مظلوم شدند و همه آنانی که ایشان را اسیر کردند ایشان را محکم نگاه می‌دارند و از رها کردن ایشان ابا می‌نمایند.
\par 34 اماولی ایشان که اسم او یهوه صبایوت می‌باشدزورآور است و دعوی ایشان را البته انجام خواهدداد و زمین را آرامی خواهد بخشید و ساکنان بابل را بی‌آرام خواهد ساخت.
\par 35 خداوند می‌گوید: شمشیری بر کلدانیان است و بر ساکنان بابل وسرورانش و حکیمانش.
\par 36 شمشیری بر کاذبان است و احمق خواهند گردید. شمشیری برجباران است و مشوش خواهند شد.
\par 37 شمشیری بر اسبانش و بر ارابه هایش می‌باشدو بر تمامی مخلوق مختلف که در میانش هستند ومثل زنان خواهند شد. شمشیری بر خزانه هایش است و غارت خواهد شد.
\par 38 خشکسالی برآبهایش می‌باشد و خشک خواهد شد زیرا که آن زمین بتها است و بر اصنام دیوانه شده‌اند.
\par 39 بنابراین وحوش صحرا با گرگان ساکن خواهندشد و شترمرغ در آن سکونت خواهد داشت و بعداز آن تا به ابد مسکون نخواهد شد و نسلا بعدنسل معمور نخواهد گردید.»
\par 40 خداوندمی گوید: «چنانکه خدا سدوم و عموره وشهرهای مجاور آنها را واژگون ساخت، همچنان کسی آنجا ساکن نخواهد شد و احدی از بنی آدم در آن ماوا نخواهد گزید.
\par 41 اینک قومی از طرف شمال می‌آیند و امتی عظیم و پادشاهان بسیار ازکرانه های جهان برانگیخته خواهند شد.
\par 42 ایشان کمان و نیزه خواهند گرفت. ایشان ستم پیشه هستند و ترحم نخواهند نمود. آواز ایشان مثل شورش دریا است و بر اسبان سوار شده، در برابرتو‌ای دختر بابل مثل مردان جنگی صف آرایی خواهند نمود.
\par 43 پادشاه بابل آوازه ایشان راشنید و دستهایش سست گردید. و الم و درد او رامثل زنی که می‌زاید در‌گرفته است.
\par 44 اینک اومثل شیر از طغیان اردن به آن مسکن منیع برخواهد آمد زیرا که من ایشان را در لحظه‌ای ازآنجا خواهم راند. و کیست آن برگزیده‌ای که او رابر آن بگمارم؟ زیرا کیست که مثل من باشد وکیست که مرا به محاکمه بیاورد و کیست آن شبانی که به حضور من تواند ایستاد؟»
\par 45 بنابراین مشورت خداوند را که درباره بابل نموده است وتقدیرهای او را که درباره زمین کلدانیان فرموده است بشنوید. البته ایشان صغیران گله را خواهندربود و هر آینه مسکن ایشان را برای ایشان خراب خواهد ساخت.از صدای تسخیر بابل زمین متزلزل شد و آواز آن در میان امت‌ها مسموع گردید.
\par 46 از صدای تسخیر بابل زمین متزلزل شد و آواز آن در میان امت‌ها مسموع گردید.
 
\chapter{51}

\par 1 «خداوند چنین می‌گوید: «اینک من بربابل و بر ساکنان وسطمقاومت کنندگانم بادی مهلک برمی انگیزانم.
\par 2 ومن بر بابل خرمن کوبان خواهم فرستاد و آن راخواهند کوبید و زمین آن را خالی خواهندساخت زیرا که ایشان در روز بلا آن را از هر طرف احاطه خواهند کرد.
\par 3 تیرانداز بر تیرانداز و برآنکه به زره خویش مفتخر می‌باشد تیر خود رابیندازد. و بر جوانان آن ترحم منمایید بلکه تمام لشکر آن را بالکل هلاک سازید.
\par 4 و ایشان بر زمین کلدانیان مقتول و در کوچه هایش مجروح خواهند افتاد.
\par 5 زیرا که اسرائیل و یهودا از خدای خویش یهوه صبایوت متروک نخواهند شد، اگرچه زمین ایشان از گناهی که به قدوس اسرائیل ورزیده‌اند پر شده است.
\par 6 از میان بابل بگریزید وهر کس جان خود را برهاند مبادا در گناه آن هلاک شوید. زیرا که این زمان انتقام خداوند است و اومکافات به آن خواهد رسانید.
\par 7 بابل در دست خداوند جام طلایی است که تمام جهان را مست می‌سازد. امت‌ها از شرابش نوشیده، و از این جهت امت‌ها دیوانه گردیده‌اند.
\par 8 بابل به ناگهان افتاده و شکسته شده است برای آن ولوله نمایید. بلسان به جهت جراحت آن بگیرید که شاید شفایابد.
\par 9 بابل را معالجه نمودیم اما شفا نپذیرفت. پس آن را ترک کنید و هر کدام از ما به زمین خودبرویم زیرا که داوری آن به آسمانها رسیده و به افلاک بلند شده است.
\par 10 خداوند عدالت ما رامکشوف خواهد ساخت. پس بیایید و اعمال یهوه خدای خویش را در صهیون اخبار نماییم.
\par 11 تیرها را تیز کنید و سپرها را به‌دست گیرید زیرا خداوند روح پادشاهان مادیان را برانگیخته است و فکر او به ضد بابل است تا آن را هلاک سازد. زیرا که این انتقام خداوند و انتقام هیکل اومی باشد.
\par 12 بر حصارهای بابل، علمها برافرازیدو آن را نیکو حراست نموده، کشیکچیان قراردهید و کمین بگذارید. زیرا خداوند قصد نموده وهم آنچه را که درباره ساکنان بابل گفته به عمل آورده است.
\par 13 ‌ای که بر آبهای بسیار ساکنی و ازگنجها معمور می‌باشی! عاقبت تو و نهایت طمع تو رسیده است!
\par 14 یهوه صبایوت به ذات خودقسم خورده است که من تو را از مردمان مثل ملخ پر خواهم ساخت و بر تو گلبانگ خواهند زد.
\par 15 «او زمین را به قوت خود ساخت و ربع مسکون را به حکمت خویش استوار نمود. وآسمانها را به عقل خود گسترانید.
\par 16 چون آوازمی دهد غوغای آبها در آسمان پدید می‌آید. ابرها از اقصای زمین برمی آورد و برقها برای باران می‌سازد و باد را از خزانه های خود بیرون می‌آورد.
\par 17 جمیع مردمان وحشی‌اند و معرفت ندارند و هر‌که تمثالی می‌سازد خجل خواهدشد. زیرا که بت ریخته شده او دروغ است و در آن هیچ نفس نیست.
\par 18 آنها باطل و کار مسخره می‌باشد. در روزی که به محاکمه می‌آیند تلف خواهند شد.
\par 19 او که نصیب یعقوب است مثل آنها نمی باشد. زیرا که او سازنده همه موجودات است و (اسرائیل ) عصای میراث وی است و اسم او یهوه صبایوت می‌باشد.
\par 20 «تو برای من کوپال و اسلحه جنگ هستی. پس از تو امت‌ها را خرد خواهم ساخت و از توممالک را هلاک خواهم نمود.
\par 21 و از تو اسب و سوارش را خرد خواهم ساخت و از تو ارابه وسوارش را خرد خواهم ساخت.
\par 22 و از تو مرد وزن را خرد خواهم ساخت و از تو پیر و طفل راخرد خواهم ساخت و از تو جوان و دوشیزه راخرد خواهم ساخت.
\par 23 و از تو شبان و گله‌اش راخرد خواهم ساخت. و از تو خویشران و گاوانش را خرد خواهم ساخت. و از تو حاکمان و والیان راخرد خواهم ساخت.
\par 24 و خداوند می‌گوید: به بابل و جمیع سکنه زمین کلدانیان جزای تمامی بدی را که ایشان به صهیون کرده‌اند در نظر شماخواهم رسانید.
\par 25 اینک خداوند می‌گوید: ای کوه مخرب که تمامی جهان را خراب می‌سازی من به ضد تو هستم! و دست خود را بر تو بلندکرده، تو را از روی صخره‌ها خواهم غلطانید و تورا کوه سوخته شده خواهم ساخت!
\par 26 و از توسنگی به جهت سر زاویه یا سنگی به جهت بنیادنخواهند گرفت، بلکه خداوند می‌گوید که توخرابی ابدی خواهی شد.
\par 27 «علمها در زمین برافرازید و کرنا در میان امت‌ها بنوازید. امت‌ها را به ضد او حاضر سازیدو ممالک آرارات و منی و اشکناز را بر وی جمع کنید. سرداران به ضد وی نصب نمایید و اسبان رامثل ملخ مودار برآورید.
\par 28 امت‌ها را به ضد وی مهیا سازید. پادشاهان مادیان و حاکمانش وجمیع والیانش و تمامی اهل زمین سلطنت او را.
\par 29 و جهان متزلزل و دردناک خواهد شد. زیرا که فکرهای خداوند به ضد بابل ثابت می‌ماند تا زمین بابل را ویران و غیرمسکون گرداند.
\par 30 و شجاعان بابل از جنگ دست برمی دارند و در ملاذهای خویش می‌نشینند و جبروت ایشان زایل شده، مثل زن گشته‌اند و مسکنهایش سوخته وپشت بندهایش شکسته شده است.
\par 31 قاصد برابر قاصد و پیک برابر پیک خواهد دوید تا پادشاه بابل را خبر دهد که شهرش از هر طرف گرفته شد.
\par 32 معبرها گرفتار شد و نی‌ها را به آتش سوختندو مردان جنگی مضطرب گردیدند.
\par 33 «زیرا که یهوه صبایوت خدای اسرائیل چنین می‌گوید: دختر بابل مثل خرمن در وقت کوبیدنش شده است و بعد از اندک زمانی وقت درو بدو خواهد رسید.
\par 34 نبوکدرصر پادشاه بابل مرا خورده و تلف کرده است و مرا ظرف خالی ساخته مثل اژدها مرا بلعیده، شکم خود را ازنفایس من پر کرده و مرا مطرود نموده است.
\par 35 وساکنه صهیون خواهد گفت ظلمی که بر من و برجسد من شده بر بابل فرود شود. و اورشلیم خواهد گفت: خون من بر ساکنان زمین کلدانیان وارد آید.
\par 36 بنابراین خداوند چنین می‌گوید: اینک من دعوی تو را به انجام خواهم رسانید وانتقام تو را خواهم کشید و نهر او را خشک ساخته، چشمه‌اش را خواهم خشکانید.
\par 37 و بابل به تلها و مسکن شغالها و محل دهشت و مسخره مبدل شده، احدی در آن ساکن نخواهد شد.
\par 38 مثل شیران با هم غرش خواهند کرد و مانندشیربچگان نعره خواهند زد.
\par 39 و خداوندمی گوید: هنگامی که گرم شوند برای ایشان بزمی برپا کرده، ایشان را مست خواهم ساخت تا وجدنموده، به خواب دایمی بخوابند که از آن بیدارنشوند.
\par 40 و ایشان را مثل بره‌ها و قوچها و بزهای نر به مسلخ فرود خواهم آورد.
\par 41 چگونه شیشک گرفتار شده و افتخار تمامی جهان تسخیر گردیده است! چگونه بابل در میان امت‌ها محل دهشت گشته است!
\par 42 دریا بر بابل برآمده و آن به کثرت امواجش مستور گردیده است.
\par 43 شهرهایش خراب شده، به زمین خشک و بیابان مبدل گشته.
\par 44 و من بیل را در بابل سزا خواهم داد و آنچه را که بلعیده است ازدهانش بیرون خواهم آورد. و امت‌ها بار دیگر به زیارت آن نخواهند رفت و حصار بابل خواهدافتاد.
\par 45 «ای قوم من از میانش بیرون آیید و هر کدام جان خود را از حدت خشم خداوند برهانید.
\par 46 ودل شما ضعف نکند و از آوازه‌ای که در زمین مسموع شود مترسید. زیرا که درآن سال آوازه‌ای شنیده خواهد شد و در سال بعد از آن آوازه‌ای دیگر. و در زمین ظلم خواهد شد و حاکم به ضدحاکم (خواهد برآمد).
\par 47 بنابراین اینک ایامی می‌آید که به بتهای بابل عقوبت خواهم رسانید وتمامی زمینش خجل خواهد شد و جمیع مقتولانش در میانش خواهند افتاد.
\par 48 اما آسمانهاو زمین و هر‌چه در آنها باشد بر بابل ترنم خواهندنمود. زیرا خداوند می‌گوید که غارت کنندگان ازطرف شمال بر او خواهند آمد.
\par 49 چنانکه بابل باعث افتادن مقتولان اسرائیل شده است، همچنین مقتولان تمامی جهان دربابل خواهندافتاد.
\par 50 ‌ای کسانی که از شمشیر رستگار شده ایدبروید و توقف منمایید و خداوند را از جای دورمتذکر شوید و اورشلیم را به‌خاطر خود آورید.»
\par 51 ما خجل گشته‌ایم زانرو که عار را شنیدیم ورسوایی چهره ما را پوشانیده است. زیرا که غریبان به مقدسهای خانه خداوند داخل شده‌اند.
\par 52 بنابراین خداوند می‌گوید: «اینک ایامی می‌آیدکه به بتهایش عقوبت خواهم رسانید و در تمامی زمینش مجروحان ناله خواهند کرد.
\par 53 اگر‌چه بابل تا به آسمان خویشتن را برافرازد و اگر‌چه بلندی قوت خویش را حصین نماید، لیکن خداوند می‌گوید: غارت کنندگان از جانب من براو خواهند آمد.
\par 54 صدای غوغا از بابل می‌آید وآواز شکست عظیمی از زمین کلدانیان.
\par 55 زیراخداوند بابل را تاراج می‌نماید و صدای عظیم رااز میان آن نابود می‌کند و امواج ایشان مثل آبهای بسیار شورش می‌نماید و صدای آواز ایشان شنیده می‌شود.
\par 56 زیرا که بر آن یعنی بر بابل غارت کننده برمی آید و جبارانش گرفتار شده، کمانهای ایشان شکسته می‌شود. چونکه یهوه خدای مجازات است و البته مکافات خواهدرسانید.
\par 57 و پادشاه که اسم او یهوه صبایوت است می‌گوید که من سروران و حکیمان وحاکمان و والیان و جبارانش را مست خواهم ساخت و به خواب دایمی که از آن بیدار نشوند، خواهند خوابید.
\par 58 یهوه صبایوت چنین می‌گویدکه حصارهای وسیع بابل بالکل سرنگون خواهدشد و دروازه های بلندش به آتش سوخته خواهدگردید و امت‌ها به جهت بطالت مشقت خواهندکشید و قبایل به جهت آتش خویشتن را خسته خواهند کرد.» 
\par 59 کلامی که ارمیا نبی به‌سرایا ابن نیریا ابن محسیا مرا فرمود هنگامی که او با صدقیاپادشاه یهودا در سال چهارم سلطنت وی به بابل می‌رفت. و سرایا رئیس دستگاه بود.
\par 60 و ارمیاتمام بلا را که بر بابل می‌بایست بیاید در طوماری نوشت یعنی تمامی این سخنانی را که درباره بابل مکتوب است.
\par 61 و ارمیا به‌سرایا گفت: «چون به بابل داخل شوی، آنگاه ببین و تمامی این سخنان را بخوان.
\par 62 و بگو: ای خداوند تو درباره این مکان فرموده‌ای که آن را هلاک خواهی ساخت به حدی که احدی از انسان یا از بهایم در آن ساکن نشود بلکه خرابه ابدی خواهد شد.
\par 63 و چون ازخواندن این طومار فارغ شدی، سنگی به آن ببندو آن را به میان فرات بینداز.و بگو همچنین بابل به‌سبب بلایی که من بر او وارد می‌آورم، غرق خواهد گردید و دیگر برپا نخواهد شد و ایشان خسته خواهند شد.»تا اینجا سخنان ارمیا است.
\par 64 و بگو همچنین بابل به‌سبب بلایی که من بر او وارد می‌آورم، غرق خواهد گردید و دیگر برپا نخواهد شد و ایشان خسته خواهند شد.»تا اینجا سخنان ارمیا است.
 
\chapter{52}

\par 1 صدقیا بیست و یکساله بود که آغازسلطنت نمود و یازده سال در اورشلیم پادشاهی کرد و اسم مادرش حمیطل دختر ارمیااز لبنه بود.
\par 2 و آنچه در نظر خداوند ناپسند بودموافق هر‌آنچه یهویاقیم کرده بود، بعمل آورد.
\par 3 زیرا به‌سبب غضبی که خداوند بر اورشلیم ویهودا داشت به حدی که آنها را از نظر خودانداخت واقع شد که صدقیا بر پادشاه بابل عاصی گشت.
\par 4 و واقع شد که نبوکدرصر پادشاه بابل باتمامی لشکر خود در روز دهم ماه دهم سال نهم سلطنت خویش بر اورشلیم برآمد و در مقابل آن اردو زده، سنگری گرداگردش بنا نمودند.
\par 5 وشهر تا سال یازدهم صدقیا پادشاه در محاصره بود.
\par 6 و در روز نهم ماه چهارم قحطی در شهرچنان سخت شد که برای اهل زمین نان نبود.
\par 7 پس در شهر رخنه‌ای ساختند و تمام مردان جنگی درشب از راه دروازه‌ای که در میان دو حصار نزد باغ پادشاه بود فرار کردند. و کلدانیان شهر را احاطه نموده بودند. و ایشان به راه عربه رفتند.
\par 8 و لشکرکلدانیان پادشاه را تعاقب نموده، در بیابان اریحا به صدقیا رسیدند و تمامی لشکرش از او پراکنده شدند.
\par 9 پس پادشاه را گرفته، او را نزد پادشاه بابل به ربله در زمین حمات آوردند و او بر وی فتوی داد.
\par 10 و پادشاه بابل پسران صدقیا را پیش رویش به قتل رسانید و جمیع سروران یهودا را نیز درربله کشت.
\par 11 و چشمان صدقیا را کور کرده، او رابدو زنجیر بست. و پادشاه بابل او را به بابل برده، وی را تا روز وفاتش در زندان انداخت.
\par 12 و در روز دهم ماه پنجم از سال نوزدهم سلطنت نبوکدرصر ملک پادشاه بابل، نبوزردان رئیس جلادان که به حضور پادشاه بابل می‌ایستادبه اورشلیم آمد.
\par 13 و خانه خداوند و خانه پادشاه را سوزانید و همه خانه های اورشلیم و هر خانه بزرگ را به آتش سوزانید.
\par 14 و تمامی لشکرکلدانیان که همراه رئیس جلادان بودند تمامی حصارهای اورشلیم را بهر طرف منهدم ساختند.
\par 15 و نبوزردان رئیس جلادان بعضی از فقیران خلق و بقیه قوم را که در شهر باقی‌مانده بودند وخارجین را که بطرف پادشاه بابل شده بودند و بقیه جمعیت را به اسیری برد.
\par 16 امانبوزردان رئیس جلادان بعضی از مسکینان زمین را برای باغبانی وفلاحی واگذاشت.
\par 17 و کلدانیان ستونهای برنجینی که در خانه خداوند بود و پایه‌ها ودریاچه برنجینی که در خانه خداوند بود، شکستند و تمامی برنج آنها را به بابل بردند.
\par 18 ودیگها و خاکندازها و گلگیرها و کاسه‌ها و قاشقهاو تمامی اسباب برنجینی را که به آنها خدمت می کردند بردند.
\par 19 و رئیس جلادان پیاله‌ها ومجمرها و کاسه‌ها و دیگها و شمعدانها و قاشقهاو لگنها را یعنی طلای آنچه را که از طلا بود ونقره آنچه را که از نقره بود برد.
\par 20 اما دو ستون ویک دریاچه و دوازده گاو برنجینی را که زیرپایه‌ها بود و سلیمان پادشاه آنها را برای خانه خداوند ساخته بود. برنج همه این اسباب بی‌اندازه بود.
\par 21 و اما ستونها، بلندی یکستون هجده ذراع و ریسمان دوازده ذراعی آنها رااحاطه داشت و حجم آن چهار انگشت بود و تهی بود.
\par 22 و تاج برنجین بر سرش و بلندی یکتاج پنج ذراع بود. و شبکه وانارها گرداگرد تاج همه ازبرنج بود. و ستون دوم مثل اینها و انارها داشت.
\par 23 و بهر طرف نود و شش انار بود. و تمام انارها به اطراف شبکه یکصد بود.
\par 24 و رئیس جلادان، سرایا رئیس کهنه، و صفنیای کاهن دوم و سه مستحفظ در را گرفت.
\par 25 و سرداری را که برمردان جنگی گماشته شده بود و هفت نفر از آنانی را که روی پادشاه را می‌دیدند و در شهر یافت شدند و کاتب سردار لشکر را که اهل ولایت راسان می‌دید و شصت نفر از اهل زمین را که درشهر یافت شدند، از شهر گرفت.
\par 26 و نبوزردان رئیس جلادان ایشان را برداشته، نزد پادشاه بابل به ربله برد.
\par 27 و پادشاه بابل ایشان را در ربله درزمین حمات زده، به قتل رسانید پس یهودا ازولایت خود به اسیری رفتند.
\par 28 و این است گروهی که نبوکدرصر به اسیری برد. در سال هفتم سه هزار و بیست و سه نفر از یهود را.
\par 29 و در سال هجدهم نبوکدرصر هشتصد و سی و دو نفر ازاورشلیم به اسیری برد.
\par 30 و در سال بیست و سوم نبوکدرصر نبوزردان رئیس جلادان هفتصد وچهل و پنج نفر از یهودا را به اسیری برد. پس جمله کسان چهار هزار و ششصد نفربودند.
\par 31 و در روز بیست و پنجم ماه دوازدهم ازسال سی و هفتم اسیری یهویاقیم پادشاه یهوداواقع شد که اویل مرودک پادشاه بابل در سال اول سلطنت خود سر یهویاقیم پادشاه یهودا را اززندان برافراشت.
\par 32 و با او سخنان دلاویز گفت وکرسی او را بالاتر از کرسیهای سایر پادشاهانی که با او در بابل بودند گذاشت.و لباس زندانی او راتبدیل نمود و او در تمامی روزهای عمرش همیشه نزد وی نان می‌خورد.
\par 33 و لباس زندانی او راتبدیل نمود و او در تمامی روزهای عمرش همیشه نزد وی نان می‌خورد.


\end{document}