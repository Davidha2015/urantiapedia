\begin{document}

\title{حزقيال}


\chapter{1}

\par 1 و روز پنجم ماه چهارم سال سی‌ام، چون من در میان اسیران نزد نهر خابور بودم، واقع شد که آسمان گشوده گردید و رویاهای خدارا دیدم.
\par 2 در پنجم آن ماه که سال پنجم اسیری یهویاکین پادشاه بود،
\par 3 کلام یهوه بر حزقیال بن بوزی کاهن نزد نهر خابور در زمین کلدانیان نازل شد و دست خداوند در آنجا بر او بود.
\par 4 پس نگریستم و اینک باد شدیدی از طرف شمال برمی آید و ابر عظیمی و آتش جهنده ودرخشندگی‌ای گرداگردش و از میانش یعنی ازمیان آتش، مثل منظر برنج تابان بود.
\par 5 و از میانش شبیه چهار حیوان پدید آمد و نمایش ایشان این بود که شبیه انسان بودند.
\par 6 و هریک از آنها چهاررو داشت و هریک از آنها چهار بال داشت.
\par 7 وپایهای آنها پایهای مستقیم و کف پای آنها مانندکف پای گوساله بود و مثل منظر برنج صیقلی درخشان بود.
\par 8 و زیر بالهای آنها از چهار طرف آنها دستهای انسان بود و آن چهار رویها و بالهای خود را چنین داشتند.
\par 9 و بالهای آنها به یکدیگرپیوسته بود و چون می‌رفتند رو نمی تافتند، بلکه هریک به راه مستقیم می‌رفتند.
\par 10 و اما شباهت رویهای آنها (این بود که ) آنها روی انسان داشتندو آن چهار روی شیر بطرف راست داشتند و آن چهار روی گاو بطرف چپ داشتند و آن چهار روی عقاب داشتند.
\par 11 و رویها و بالهای آنها ازطرف بالا از یکدیگر جدا بود و دو بال هریک به همدیگر پیوسته و دو بال دیگر بدن آنها رامی پوشانید.
\par 12 و هر یک از آنها به راه مستقیم می‌رفتند و به هر جایی که روح می‌رفت آنهامی رفتند و درحین رفتن رو نمی تافتند.
\par 13 و اما شباهت این حیوانات (این بود که )صورت آنها مانند شعله های اخگرهای آتش افروخته شده، مثل صورت مشعلها بود. و آن آتش در میان آن حیوانات گردش می‌کرد ودرخشان می‌بود و از میان آتش برق می‌جهید.
\par 14 و آن حیوانات مثل صورت برق می‌دویدند وبرمی گشتند.
\par 15 و چون آن حیوانات را ملاحظه می‌کردم، اینک یک چرخ به پهلوی آن حیوانات برای هر روی (هرکدام از) آن چهار بر زمین بود.
\par 16 و صورت چرخها و صنعت آنها مثل منظرزبرجد بود و آن چهار یک شباهت داشتند. وصورت و صنعت آنها مثل چرخ در میان چرخ بود.
\par 17 و چون آنها می‌رفتند، بر چهار جانب خودمی رفتند و در حین رفتن به هیچ طرف میل نمی کردند.
\par 18 و فلکه های آنها بلند و مهیب بود وفلکه های آن چهار از هر طرف از چشمها پر بود.
\par 19 و چون آن حیوانات می‌رفتند، چرخها درپهلوی آنها می‌رفت و چون آن حیوانات از زمین بلند می‌شدند، چرخها بلند می‌شد.
\par 20 و هر جایی که روح می‌رفت آنها می‌رفتند، به هر جا که روح سیر می‌کرد و چرخها پیش روی آنها بلندمی شد، زیرا که روح حیوانات در چرخها بود.
\par 21 و چون آنها می‌رفتند، اینها می‌رفت و چون آنها می‌ایستادند، اینها می‌ایستاد. و چون آنها اززمین بلند می‌شدند، چرخها پیش روی آنها اززمین بلند می‌شد، زیرا روح حیوانات در چرخهابود.
\par 22 و شباهت فلکی که بالای سر حیوانات بودمثل منظر بلور مهیب بود و بالای سر آنها پهن شده بود.
\par 23 و بالهای آنها زیر فلک بسوی یکدیگر مستقیم بود و دو بال هریک از این طرف می‌پوشانید و دو بال هر یک از آن طرف بدنهای آنها را می‌پوشانید.
\par 24 و چون می‌رفتند، من صدای بالهای آنها را مانند صدای آبهای بسیار، مثل آواز حضرت اعلی و صدای هنگامه را مثل صدای فوج شنیدم. زیرا که چون می‌ایستادندبالهای خویش را فرو می‌هشتند.
\par 25 و چون درحین ایستادن بالهای خود را فرو می‌هشتند، صدایی از فلکی که بالای سر آنها بود مسموع می‌شد.
\par 26 و بالای فلکی که بر سر آنها بودشباهت تختی مثل صورت یاقوت کبود بود وبرآن شباهت تخت، شباهتی مثل صورت انسان برفوق آن بود.
\par 27 و از منظر کمر او بطرف بالا مثل منظر برنج تابان، مانند نمایش آتش در اندرون آن و گرداگردش دیدم. و از منظر کمر او به طرف پایین مثل نمایش آتشی که از هر طرف درخشان بود دیدم.مانند نمایش قوس قزح که در روزباران در ابر می‌باشد، همچنین آن درخشندگی گرداگرد آن بود. این منظر شباهت جلال یهوه بود و چون آن را دیدم، به روی خود در‌افتادم و آوازقائلی را شنیدم،
\par 28 مانند نمایش قوس قزح که در روزباران در ابر می‌باشد، همچنین آن درخشندگی گرداگرد آن بود. این منظر شباهت جلال یهوه بود و چون آن را دیدم، به روی خود در‌افتادم و آوازقائلی را شنیدم،

\chapter{2}

\par 1 که مرا گفت: «ای پسر انسان بر پایهای خودبایست تا با تو سخن گویم.»
\par 2 و چون این رابه من گفت، روح داخل من شده، مرا بر پایهایم برپا نمود. و او را که با من متکلم نمود شنیدم
\par 3 که مرا گفت: «ای پسر انسان من تو را نزد بنی‌اسرائیل می‌فرستم، یعنی نزد امت فتنه انگیزی که به من فتنه انگیخته‌اند. ایشان و پدران ایشان تا به امروزبر من عصیان ورزیده‌اند.
\par 4 و پسران ایشان سخت رو و قسی القلب هستند و من تو را نزد ایشان می‌فرستم تا به ایشان بگویی: خداوند یهوه چنین می‌فرماید.
\par 5 و ایشان خواه بشنوند و خواه نشنوند، زیرا خاندان فتنه انگیز می‌باشند، خواهنددانست که نبی‌ای در میان ایشان هست.
\par 6 و تو‌ای پسر انسان از ایشان مترس و از سخنان ایشان بیم مکن اگرچه خارها و شوکها با تو باشد و در میان عقربها ساکن باشی، اما از سخنان ایشان مترس واز رویهای ایشان هراسان مشو، زیرا که ایشان خاندان فتنه انگیز می‌باشند.
\par 7 پس کلام مرا به ایشان بگو، خواه بشنوند و خواه نشنوند، چونکه فتنه انگیز هستند.
\par 8 و تو‌ای پسر انسان آنچه را که من به تو می‌گویم بشنو و مثل این خاندان فتنه انگیزعاصی مشو بلکه دهان خود را گشوده، آنچه را که من به تو می‌دهم بخور.»
\par 9 پس نگریستم و اینک دستی بسوی من درازشد و در آن طوماری بود.و آن را پیش من بگشود که رو و پشتش هر دو نوشته بود و نوحه و ماتم و وای بر آن مکتوب بود.
\par 10 و آن را پیش من بگشود که رو و پشتش هر دو نوشته بود و نوحه و ماتم و وای بر آن مکتوب بود.

\chapter{3}

\par 1 پس مرا گفت: «ای پسر انسان آنچه را که می یابی بخور. این طومار را بخور و رفته، باخاندان اسرائیل متکلم شو.»
\par 2 آنگاه دهان خود راگشودم و او آن طومار را به من خورانید.
\par 3 و مراگفت: «ای پسر انسان شکم خود را بخوران واحشای خویش را از این طوماری که من به تومی دهم پر کن.» پس آن را خوردم و در دهانم مثل عسل شیرین بود.
\par 4 و مرا گفت: «ای پسر انسان بیا و نزد خاندان اسرائیل رفته، کلام مرا برای ایشان بیان کن.
\par 5 زیراکه نزد امت غامض زبان و ثقیل لسان فرستاده نشدی، بلکه نزد خاندان اسرائیل.
\par 6 نه نزدقوم های بسیار غامض زبان و ثقیل لسان که سخنان ایشان را نتوانی فهمید. یقین اگر تو را نزدآنها می‌فرستادم به تو گوش می‌گرفتند.
\par 7 اماخاندان اسرائیل نمی خواهند تو را بشنوند زیرا که نمی خواهند مرا بشنوند. چونکه تمامی خاندان اسرائیل سخت پیشانی و قسی القلب هستند.
\par 8 هان من روی تو را در مقابل روی ایشان سخت خواهم ساخت و پیشانی تو را در مقابل پیشانی‌ایشان سخت خواهم گردانید.
\par 9 بلکه پیشانی تو رامثل الماس از سنگ خارا سخت‌تر گردانیدم. پس از ایشان مترس و از رویهای ایشان هراسان مباش، زیرا که خاندان فتنه انگیز می‌باشند.»
\par 10 و مرا گفت: «ای پسر انسان تمام کلام مرا که به تو می‌گویم در دل خود جا بده و به گوشهای خود استماع نما.
\par 11 و بیا و نزد اسیرانی که ازپسران قوم تو می‌باشند رفته، ایشان را خطاب کن و خواه بشنوند و خواه نشنوند. به ایشان بگو: خداوند یهوه چنین می‌فرماید.»
\par 12 آنگاه روح، مرا برداشت و از عقب خود صدای گلبانگ عظیمی شنیدم که «جلال یهوه از مقام او متبارک باد.»
\par 13 و صدای بالهای آن حیوانات را که به همدیگر برمی خوردند و صدای چرخها را که پیش روی آنها بود و صدای گلبانگ عظیمی راشنیدم.
\par 14 آنگاه روح مرا برداشت و برد و با تلخی در حرارت روح خود رفتم و دست خداوند بر من سنگین می‌بود.
\par 15 پس به تل ابیب نزد اسیرانی که نزد نهر خابور ساکن بودند، رسیدم و در مکانی که ایشان نشسته بودند، در آنجا به میان ایشان هفت روز متحیر نشستم.
\par 16 و بعد از انقضای هفت روز واقع شد که کلام خداوند بر من نازل شده، گفت:
\par 17 «ای پسر انسان تو را برای خاندان اسرائیل دیده بان ساختم، پس کلام را از دهان من بشنو و ایشان را از جانب من تهدید کن.
\par 18 و حینی که من به مرد شریر گفته باشم که البته خواهی مرد، اگر تو او را تهدید نکنی و سخن نگویی تا آن شریر را از طریق زشت اوتهدید نموده، او را زنده سازی، آنگاه آن شریر درگناهش خواهد مرد، اما خون او را از دست توخواهم طلبید.
\par 19 لیکن اگر تو مرد شریر را تهدیدکنی و او از شرارت خود و طریق بد خویش بازگشت نکند او در گناه خود خواهد مرد، اما توجان خود را نجات داده‌ای.
\par 20 و اگر مرد عادل ازعدالت خود برگردد و گناه ورزد و من سنگی مصادم پیش وی بنهم تا بمیرد، چونکه تو او راتهدید ننمودی، او در گناه خود خواهد مرد و عدالتی که بعمل آورده بود به یاد آورده نخواهدشد. لیکن خون او را از دست تو خواهم طلبید.
\par 21 و اگر تو مرد عادل را تهدید کنی که آن مردعادل گناه نکند و او خطا نورزد البته زنده خواهدماند، چونکه تهدید پذیرفته است و تو جان خودرا نجات داده‌ای.»
\par 22 و دست خداوند در آنجا بر من نهاده شد واو مرا گفت: «برخیز و به هامون بیرون شو که درآنجا با تو سخن خواهم گفت.»
\par 23 پس برخاسته، به هامون بیرون رفتم و اینک جلال خداوند مثل جلالی که نزد نهر خابور دیده بودم، در آنجا برپاشد و من به روی خود درافتادم.
\par 24 و روح داخل من شده، مرا بر پایهایم برپا داشت و او مرا خطاب کرده، گفت: «برو و خویشتن را در خانه خود ببند.
\par 25 و اما تو‌ای پسر انسان اینک بندها بر توخواهند نهاد و تو را به آنها خواهند بست اما درمیان ایشان بیرون مرو.
\par 26 و من زبان تو را به کامت خواهم چسبانید تا گنگ شده، برای ایشان ناصح نباشی. زیرا که ایشان خاندان فتنه انگیز می‌باشند.اما وقتی که من با تو تکلم نمایم، آنگاه دهان تورا خواهم گشود و به ایشان خواهی گفت: خداوندیهوه چنین می‌فرماید. آنگاه آنکه شنوا باشدبشنود و آنکه ابا نماید ابا کند. زیرا که ایشان خاندان فتنه انگیز می‌باشند.»
\par 27 اما وقتی که من با تو تکلم نمایم، آنگاه دهان تورا خواهم گشود و به ایشان خواهی گفت: خداوندیهوه چنین می‌فرماید. آنگاه آنکه شنوا باشدبشنود و آنکه ابا نماید ابا کند. زیرا که ایشان خاندان فتنه انگیز می‌باشند.»

\chapter{4}

\par 1 «و تو نیز‌ای پسر انسان آجری بگیر و آن راپیش روی خود بگذار و شهر اورشلیم را برآن نقش نما.
\par 2 و آن را محاصره کن و در برابرش برجها ساخته، سنگری در مقابلش برپا نما و به اطرافش اردو زده، منجنیقها به هر سوی آن برپاکن.
\par 3 و تابه آهنین برای خود گرفته، آن را در میان خود و شهر، دیواری آهنین بگذار و روی خود رابر آن بدار و محاصره خواهد شد و تو آن رامحاصره کن تا آیتی به جهت خاندان اسرائیل بشود.
\par 4 پس تو بر پهلوی چپ خود بخواب و گناه خاندان اسرائیل را بر آن بگذار. موافق شماره روزهایی که بر آن بخوابی، گناه ایشان را متحمل خواهی شد.
\par 5 و من سالهای گناه ایشان را مطابق شماره روزها یعنی سیصد و نود روز بر تونهاده‌ام. پس متحمل گناه خاندان اسرائیل خواهی شد.
\par 6 و چون اینها را به انجام رسانیده باشی، بازبه پهلوی راست خود بخواب و چهل روزمتحمل گناه خاندان یهودا خواهی شد. هر روزی را به جهت سالی برای تو قرار داده‌ام.
\par 7 و بازوی خود را برهنه کرده، روی به محاصره اورشلیم بدار و به ضد آن نبوت کن.
\par 8 و اینک بندها بر تومی نهم و تا روزهای محاصره ات را به اتمام نرسانیده باشی از پهلو به پهلوی دیگرت نخواهی غلطید.
\par 9 پس گندم و جو و باقلا و عدس و ارزن وجلبان برای خود گرفته، آنها را در یک ظرف بریزو خوراکی از آنها برای خود بپز و تمامی روزهایی که به پهلوی خود می‌خوابی، یعنی سیصد و نود روز آن را خواهی خورد.
\par 10 وغذایی که می‌خوری به وزن خواهد بود، یعنی بیست مثقال برای هر روز. وقت به وقت آن راخواهی خورد.
\par 11 و آب را به پیمایش یعنی سدس یک هین خواهی نوشید. آن را وقت به وقت خواهی نوشید.
\par 12 و قرصهای نان جو که می‌خوری، آنها را بر سرگین انسان در نظر ایشان خواهی پخت.
\par 13 و خداوند فرمود به همین منوال بنی‌اسرائیل نان نجس در میان امت هایی که من ایشان را به میان آنها پراکنده می‌سازم خواهندخورد.»
\par 14 پس گفتم: «آه‌ای خداوند یهوه اینک جان من نجس نشده و از طفولیت خود تا به حال میته یادریده شده را نخورده‌ام و خوراک نجس به دهانم نرفته است.»
\par 15 آنگاه به من گفت: «بدان که سرگین گاو را به عوض سرگین انسان به تو دادم، پس نان خود را بر آن خواهی پخت.»
\par 16 و مرا گفت: «ای پسر انسان اینک من عصای نان را در اورشلیم خواهم شکست و نان را به وزن و عسرت خواهندخورد و آب را به پیمایش و حیرت خواهندنوشید.زیرا که محتاج نان و آب خواهند شد وبه حیرت بر یکدیگر نظر خواهند انداخت و به‌سبب گناهان خود گداخته خواهند شد.
\par 17 زیرا که محتاج نان و آب خواهند شد وبه حیرت بر یکدیگر نظر خواهند انداخت و به‌سبب گناهان خود گداخته خواهند شد.

\chapter{5}

\par 1 بگیر و آن را مثل استره حجام به جهت خود بکار برده، آن را بر سر و ریش خود بگذران وترازویی گرفته، مویها را تقسیم کن.
\par 2 و چون روزهای محاصره را به اتمام رسانیده باشی، یک ثلث را در میان شهر به آتش بسوزان و یک ثلث راگرفته، اطراف آن را با تیغ بزن و ثلث دیگر را به بادها بپاش و من در عقب آنها شمشیری خواهم فرستاد.
\par 3 و اندکی از آن را گرفته، آنها را در دامن خود ببند.
\par 4 و باز قدری از آنها را بگیر و آنها را درمیان آتش انداخته، آنها را به آتش بسوزان وآتشی برای تمام خاندان اسرائیل از آن بیرون خواهد آمد.»
\par 5 خداوند یهوه چنین می‌گوید: «من این اورشلیم را در میان امت‌ها قرار دادم و کشورها رابهر طرف آن.
\par 6 و او از احکام من بدتر از امت‌ها واز فرایض من بدتر از کشورهایی که گرداگرد اومی باشد، عصیان ورزیده است زیرا که اهل او احکام مرا ترک کرده، به فرایض من سلوک ننموده‌اند.»
\par 7 بنابراین خداوند یهوه چنین می‌گوید: «چونکه شما زیاده از امت هایی که گرداگرد شما می‌باشند غوغا نمودید و به فرایض من سلوک نکرده، احکام مرا بعمل نیاوردید، بلکه موافق احکام امت هایی که گرداگرد شما می‌باشندنیز عمل ننمودید،
\par 8 لهذا خداوند یهوه چنین می‌گوید: من اینک من به ضد تو هستم و در میان تو به نظر امت‌ها داوریها خواهم نمود.
\par 9 و با تو به‌سبب جمیع رجاساتت کارها خواهم کرد که قبل از این نکرده باشم و مثل آنها هم دیگر نخواهم کرد.
\par 10 بنابراین پدران در میان تو پسران راخواهند خورد و پسران پدران خویش را خواهندخورد و بر تو داوریها نموده، تمامی بقیت تو رابسوی هر باد پراکنده خواهم ساخت.»
\par 11 لهذاخداوند یهوه می‌گوید: «به حیات خودم قسم چونکه تو مقدس مرا بتمامی رجاسات و جمیع مکروهات خویش نجس ساختی، من نیز البته تورا منقطع خواهم ساخت و چشم من شفقت نخواهد نمود و من نیز رحمت نخواهم فرمود.
\par 12 یک ثلث تو در میانت از وبا خواهند مرد و ازگرسنگی تلف خواهند شد. و یک ثلث به اطرافت به شمشیر خواهند افتاد و ثلث دیگر را بسوی هرباد پراکنده ساخته، شمشیر را در عقب ایشان خواهم فرستاد.
\par 13 پس چون غضب من به اتمام رسیده باشد و حدت خشم خویش را بر ایشان ریخته باشم، آنگاه پشیمان خواهم شد. و چون حدت خشم خویش را بر ایشان به اتمام رسانده باشم، آنگاه خواهند دانست که من یهوه این را درغیرت خویش گفته‌ام.
\par 14 و تو را در در نظر همه رهگذریان در میان امت هایی که به اطراف تومی باشند، به خرابی و رسوایی تسلیم خواهم نمود.
\par 15 و چون بر تو به خشم و غضب وسرزنشهای سخت داوری کرده باشم، آنگاه این موجد عار و مذمت و عبرت و دهشت برای امت هایی که به اطراف تو می‌باشند خواهد بود. من که یهوه هستم این را گفتم.
\par 16 و چون تیرهای بد قحطی را که برای هلاکت می‌باشد و من آنها رابه جهت خرابی شما می‌فرستم در میان شماانداخته باشم، آنگاه قحط را بر شما سخت‌ترخواهم گردانید و عصای نان شما را خواهم شکست.و قحط و حیوانات درنده در میان توخواهم فرستاد تا تو را بی‌اولاد گردانند و وبا وخون از میان تو عبور خواهد کرد و شمشیری برتو وارد خواهم آورد. من که یهوه هستم این راگفتم.»
\par 17 و قحط و حیوانات درنده در میان توخواهم فرستاد تا تو را بی‌اولاد گردانند و وبا وخون از میان تو عبور خواهد کرد و شمشیری برتو وارد خواهم آورد. من که یهوه هستم این راگفتم.»

\chapter{6}

\par 1 و کلام خداوند بر من نازل شده، گفت:
\par 2 «ای پسر انسان نظر خود را بر کوههای اسرائیل بدوز و درباره آنها نبوت کن.
\par 3 و بگو: ای کوههای اسرائیل کلام خداوند یهوه را بشنوید! خداوند یهوه به کوهها و تلها و وادیها و دره هاچنین می‌فرماید: اینک من شمشیری بر شمامی آورم و مکان های بلند شما را خراب خواهم کرد.
\par 4 و مذبح های شما منهدم و تمثالهای شمسی شما شکسته خواهد شد و کشتگان شمارا پیش بتهای شما خواهم‌انداخت. و لاشهای بنی‌اسرائیل را پیش بتهای ایشان خواهم گذاشت و استخوانهای شما را گرداگرد مذبح های شما خواهم پاشید.
\par 5 و در جمیع مساکن شما شهرهاخراب و مکان های بلند ویران خواهد شد تا آنکه مذبح های شما خراب و ویران شود و بتهای شماشکسته و نابود گردد و تمثالهای شمسی شمامنهدم و اعمال شمامحو شود.
\par 6 و چون کشتگان شما در میان شما بیفتند، آنگاه خواهی دانست که من یهوه هستم.
\par 7 اما بقیتی نگاه خواهم داشت. وچون در میان کشورها پراکنده شوید، بقیه السیف شما در میان امت‌ها ساکن خواهند شد.
\par 8 ونجات‌یافتگان شما در میان امت‌ها در جایی که ایشان را به اسیری برده‌اند مرا یاد خواهند داشت. چونکه دل زناکار ایشان را که از من دور شده است خواهم شکست و چشمان ایشان را که در عقب بتهای ایشان زنا کرده است - پس خویشتن را به‌سبب اعمال زشتی که در همه رجاسات خودنموده‌اند مکروه خواهند داشت.
\par 9 و خواهنددانست که من یهوه هستم و عبث نگفتم که این بلارا بر ایشان وارد خواهم آورد.»
\par 10 خداوند یهوه چنین می‌گوید: «به‌دست خود بزن و پای خود را بر زمین بکوب و بگو: وای بر تمامی رجاسات و شریر خاندان اسرائیل زیراکه به شمشیر و قحط و وباخواهد افتاد.
\par 11 آنکه دور باشد به وبا خواهد مرد و آنکه نزدیک است به شمشیر خواهد افتادو آنکه باقی‌مانده و درمحاصره باشد از گرسنگی خواهد مرد و من حدت خشم خود را بر ایشان به اتمام خواهم رسانید.
\par 12 و خواهید دانست که من یهوه هستم، هنگامی که کشتگان ایشان در میان بتهای ایشان به اطراف مذبح های ایشان، بر هر تل بلند و برقله های تمام کوهها و زیر هر درخت سبز و زیر هر بلوط کشن، در جایی که هدایای خوشبو برای همه بتهای خود می‌گذرانیدند یافت خواهند شد.و دست خود را بر ایشان دراز کرده، زمین را درتمام مسکن های ایشان خرابتر و ویرانتر از بیابان دبله خواهم ساخت. پس خواهند دانست که من یهوه هستم.»
\par 13 و دست خود را بر ایشان دراز کرده، زمین را درتمام مسکن های ایشان خرابتر و ویرانتر از بیابان دبله خواهم ساخت. پس خواهند دانست که من یهوه هستم.»

\chapter{7}

\par 1 و کلام خداوند بر من نازل شده، گفت:
\par 2 «وتو‌ای پسر انسان (بگو): خداوند یهوه به زمین اسرائیل چنین می‌گوید: انتهایی بر چهارگوشه زمین انتها رسیده است.
\par 3 الان انتها بر تورسیده است و من خشم خود را بر تو واردآورده‌ام و بر وفق راههایت ترا داوری نموده، تمامی رجاساتت را بر تو خواهم نهاد.
\par 4 و چشم من برتو شفقت نخواهد کرد و رحمت نخواهم فرمود بلکه راههای تو را بر تو خواهم نهاد ورجاسات تو در میانت خواهد بود. پس خواهی دانست که من یهوه هستم.»
\par 5 خداوند یهوه چنین می‌گوید: «بلا هان بلای واحد می‌آید!
\par 6 انتهایی می‌آید، انتهایی می‌آید وبه ضد تو بیدار شده است. هان می‌آید.
\par 7 ‌ای ساکن زمین اجل تو بر تو می‌آید. وقت معین می‌آید و آن روز نزدیک است. روز هنگامه خواهد شد و نه روز آواز شادمانی بر کوهها.
\par 8 الان عنقریب غضب خود را بر تو خواهم ریخت و خشم خویش را بر تو به اتمام رسانیده، تو را موافق راههایت داوری خواهم نمود و جمیع رجاساتت را بر تو خواهم نهاد.
\par 9 و چشم من شفقت نخواهدکرد و رحمت نخواهم فرمود، بلکه مکافات راههایت را به تو خواهم رسانید و رجاسات تو درمیانت خواهد بود و خواهید دانست که زننده تو من یهوه هستم.
\par 10 اینک آنروز هان می‌آید! اجل تو بیرون آمده و عصا شکوفه آورده و تکبر، گل کرده است.
\par 11 ظلم عصای شرارت گشته است. ازایشان و از جمعیت ایشان و از ازدحام ایشان چیزی باقی نیست و در میان ایشان حشمتی نمانده است.
\par 12 وقت می‌آید و آنروز نزدیک است. پس مشتری شادی نکند و فروشنده ماتم نگیرد، زیرا که خشم بر تمامی جمعیت ایشان قرار گرفته است.
\par 13 زیرا که فروشندگان اگر‌چه در میان زندگان زنده مانند، به آنچه فروخته باشندنخواهند برگشت، چونکه غضب بر تمامی جمعیت ایشان قرار گرفته است. ایشان نخواهندبرگشت و هیچکس به گناه خویش زندگی خود راتقویت نخواهد داد.
\par 14 کرنا را نواخته و همه‌چیزرا مهیا ساخته‌اند، اما کسی به جنگ نمی رود. زیراکه غضب من بر تمامی جمعیت ایشان قرار گرفته است.
\par 15 شمشیر در بیرون است و وبا و قحط دراندرون. آنکه در صحرا است به شمشیر می‌میرد وآنکه در شهر است قحط و وبا او را هلاک می‌سازد.
\par 16 و رستگاران ایشان فرار می‌کنند ومثل فاخته های دره‌ها بر کوهها می‌باشند. و هرکدام از ایشان به‌سبب گناه خود ناله می‌کنند.
\par 17 همه دستها سست شده و جمیع زانوها مثل آب بیتاب گردیده است.
\par 18 و پلاس در برمی کنند و وحشت ایشان را می‌پوشاند و بر همه چهره‌ها خجلت و بر جمیع سرها گری می‌باشد.
\par 19 نقره خود را در کوچه‌ها می‌ریزند و طلای ایشان مثل چیز نجس می‌باشد. نقره و طلای ایشان در روز غضب خداوند ایشان را نتواندرهانید. جانهای خود را سیر نمی کنند و بطنهای خویش را پر نمی سازند زیرا گناه ایشان سنگ مصادم آنها شده است.
\par 20 «و او زیبایی زینت خود را در کبریایی قرارداده بود، اما ایشان بتهای مکروهات و رجاسات خویش را در آن ساختند. بنابراین آن را برای ایشان مثل چیز نجس خواهم گردانید.
\par 21 و آن رابه‌دست غریبان به تاراج و به شریران جهان به غارت خواهم داد و آن را بی‌عصمت خواهندساخت.
\par 22 و روی خود را از ایشان خواهم برگردانید و مکان مستور مرا بی‌عصمت خواهندنمود و ظالمان داخل آن شده، آن را بی‌عصمت خواهند ساخت.
\par 23 زنجیر را بساز، زیرا که زمین از جرمهای خونریزی پر است و شهر از ظلم مملو.
\par 24 و اشرار امت‌ها را خواهم آورد و درخانه های ایشان تصرف خواهند نمود و تکبرزورآوران را زایل خواهم ساخت و آنهامکان های مقدس ایشان را بی‌عصمت خواهندنمود.
\par 25 هلاکت می‌آید و سلامتی را خواهندطلبید، اما یافت نخواهد شد.
\par 26 مصیبت برمصیبت می‌آید و آوازه بر آوازه مسموع می‌شود. رویا از نبی می‌طلبند، اما شریعت از کاهنان ومشورت از مشایخ نابود شده است.پادشاه ماتم می‌گیرد و رئیس به حیرت ملبس می‌شود ودستهای اهل زمین می‌لرزد. و موافق راههای ایشان با ایشان عمل خواهم نمود و بر وفق استحقاق ایشان ایشان را داوری خواهم نمود. پس خواهند دانست که من یهوه هستم؟
\par 27 پادشاه ماتم می‌گیرد و رئیس به حیرت ملبس می‌شود ودستهای اهل زمین می‌لرزد. و موافق راههای ایشان با ایشان عمل خواهم نمود و بر وفق استحقاق ایشان ایشان را داوری خواهم نمود. پس خواهند دانست که من یهوه هستم؟

\chapter{8}

\par 1 و در سال ششم در روز پنجم از ماه ششم، چون من در خانه خود نشسته بودم ومشایخ یهودا پیش من نشسته بودند، آنگاه دست خداوند یهوه در آنجا بر من فرود آمد.
\par 2 و دیدم که اینک شبیهی مثل صورت آتش بود یعنی ازنمایش کمر او تا پایین آتش و از کمر او تا بالا مثل منظر درخشندگی مانند صورت برنج لامع ظاهرشد.
\par 3 و شبیه دستی دراز کرده، موی پیشانی مرابگرفت و روح، مرا در میان زمین و آسمان برداشت و مرا در رویاهای خدا به اورشلیم نزددهنه دروازه صحن اندرونی که بطرف شمال متوجه است برد که در آنجا نشیمن تمثال غیرت غیرت انگیز می‌باشد.
\par 4 و اینک جلال خدای اسرائیل مانند آن رویایی که در هامون دیده بودم ظاهر شد.
\par 5 و اومرا گفت: «ای پسر انسان چشمان خود را بسوی راه شمال برافراز!» و چون چشمان خود را بسوی راه شمال برافراشتم، اینک بطرف شمالی دروازه مذبح این تمثال غیرت در مدخل ظاهر شد.
\par 6 و اومرا گفت: «ای پسر انسان آیا تو آنچه را که ایشان می‌کنند می‌بینی؟ یعنی رجاسات عظیمی که خاندان اسرائیل در اینجا می‌کنند تا از مقدس خود دور بشوم؟ اما باز رجاسات عظیم تر خواهی دید.»
\par 7 پس مرا به دروازه صحن آورد و دیدم که اینک سوراخی در دیوار است.
\par 8 و او مرا گفت: «ای پسر انسان دیوار را بکن.» و چون دیوار راکندم، اینک دروازه‌ای پدید آمد.
\par 9 و او مرا گفت: «داخل شو و رجاسات شنیعی را که ایشان دراینجا می‌کنند ملاحظه نما.»
\par 10 پس چون داخل شدم، دیدم که هرگونه حشرات و حیوانات نجس و جمیع بتهای خاندان اسرائیل بر دیوار از هر طرف نقش شده بود.
\par 11 و هفتاد نفر از مشایخ خاندان اسرائیل پیش آنهاایستاده بودند و یازنیا ابن شافان در میان ایشان ایستاده بود و هرکس مجمره‌ای در دست خودداشت و بوی ابر بخور بالا می‌رفت.
\par 12 و او مراگفت: «ای پسر انسان آیا آنچه را که مشایخ خاندان اسرائیل در تاریکی و هرکس درحجره های بتهای خویش می‌کنند دیدی؟ زیرامی گویند که خداوند ما را نمی بیند و خداوند این زمین را ترک کرده است.
\par 13 و به من گفت که بازرجاسات عظیم تر از اینهایی که اینان می‌کنندخواهی دید.»
\par 14 پس مرا به دهنه دروازه خانه خداوند که بطرف شمال بود آورد. و اینک در آنجا بعضی زنان نشسته، برای تموز می‌گریستند.
\par 15 و او مراگفت: «ای پسر انسان آیا این را دیدی؟ بازرجاسات عظیم تر از اینها را خواهی دید.»
\par 16 پس مرا به صحن اندرونی خانه خداوندآورد. و اینک نزد دروازه هیکل خداوند در میان رواق و مذبح به قدر بیست و پنج مرد بودند که پشتهای خود را بسوی هیکل خداوند و رویهای خویش را بسوی مشرق داشتند و آفتاب را بطرف مشرق سجده می‌نمودند.
\par 17 و به من گفت: «ای پسر انسان این را دیدی؟ آیا برای خاندان یهودابجا آوردن این رجاسات که در اینجا بجامی آورند سهل است؟ زیرا که زمین را از ظلم مملو ساخته‌اند و برای هیجان خشم من برمی گردند و هان شاخه را به بینی خودمی گذارند.بنابراین من نیز در غضب، عمل خواهم نمود و چشم من شفقت نخواهد کرد ورحمت نخواهم فرمود و اگرچه به آواز بلند به گوش من بخوانند، ایشان را اجابت نخواهم نمود.»
\par 18 بنابراین من نیز در غضب، عمل خواهم نمود و چشم من شفقت نخواهد کرد ورحمت نخواهم فرمود و اگرچه به آواز بلند به گوش من بخوانند، ایشان را اجابت نخواهم نمود.»

\chapter{9}

\par 1 و او به آواز بلند به گوش من ندا کرده، گفت: «وکلای شهر را نزدیک بیاور و هرکس آلت خراب کننده خود را در دست خود بدارد.»
\par 2 و اینک شش مرد از راه دروازه بالایی که بطرف شمال متوجه است آمدند و هرکس تبرخود را در دستش داشت. و در میان ایشان یک مرد ملبس شده به کتان بود و دوات کاتب درکمرش. و ایشان داخل شده، نزد مذبح برنجین ایستادند.
\par 3 و جلال خدای اسرائیل از روی آن کروبی که بالای آن بود به آستانه خانه برآمد و به آن مردی که به کتان ملبس بود و دوات کاتب را درکمر داشت خطاب کرد.
\par 4 و خداوند به او گفت: «از میان شهر یعنی از میان اورشلیم بگذر و برپیشانی کسانی که به‌سبب همه رجاساتی که در آن کرده می‌شود آه و ناله می‌کنند نشانی بگذار.
\par 5 و به آنان به سمع من گفت که در عقب او از شهربگذرید و هلاک سازید و چشمان شما شفقت نکند و ترحم منمایید.
\par 6 پیران و جوانان و دختران و اطفال و زنان را تمام به قتل رسانید، اما به هرکسی‌که این نشان را دارد نزدیک مشوید و ازقدس من شروع کنید.» پس از مردان پیری که پیش خانه بودند شروع کردند.
\par 7 و به ایشان فرمود: «خانه را نجس سازید وصحنها را از کشتگان پر ساخته، بیرون آیید.» پس بیرون آمدند و در شهر به کشتن شروع کردند.
\par 8 وچون ایشان می‌کشتند و من باقی‌مانده بودم به روی خود در‌افتاده، استغاثه نمودم و گفتم: «آه‌ای خداوند یهوه آیا چون غضب خود را براورشلیم می‌ریزی تمامی بقیه اسرائیل را هلاک خواهی ساخت؟»
\par 9 او مرا جواب داد: «گناه خاندان اسرائیل و یهودا بی‌نهایت عظیم است وزمین از خون مملو و شهر از ستم پر است. زیرامی گویند: خداوند زمین را ترک کرده است وخداوند نمی بیند.
\par 10 پس چشم من نیز شفقت نخواهد کرد و من رحمت نخواهم فرمود، بلکه رفتار ایشان را بر سر ایشان خواهم آورد.»واینک آن مردی که به کتان ملبس بود و دوات را درکمر داشت، جواب داد و گفت: «به نهجی که مراامر فرمودی عمل نمودم.»
\par 11 واینک آن مردی که به کتان ملبس بود و دوات را درکمر داشت، جواب داد و گفت: «به نهجی که مراامر فرمودی عمل نمودم.»

\chapter{10}

\par 1 پس نگریستم و اینک بر فلکی که بالای سر کروبیان بود چیزی مثل سنگ یاقوت کبود و مثل نمایش شبیه تخت بر زبر آنهاظاهر شد.
\par 2 و آن مرد را که به کتان ملبس بودخطاب کرده گفت: «در میان چرخها در زیرکروبیان برو و دستهای خود را از اخگرهای آتشی که در میان کروبیان است پر کن و بر شهربپاش.» و او در نظر من داخل شد.
\par 3 و چون آن مردداخل شد، کروبیان بطرف راست خانه ایستاده بودند و ابر، صحن اندرونی را پر کرد.
\par 4 و جلال خداوند از روی کروبیان به آستانه خانه برآمد وخانه از ابر پر شد و صحن از فروغ جلال خداوندمملو گشت.
\par 5 و صدای بالهای کروبیان تا به صحن بیرونی مثل آواز خدای قادر مطلق حینی که تکلم می کند، مسموع شد.
\par 6 و چون آن مرد را که ملبس به کتان بود امرفرموده، گفت که «آتش را از میان چرخها از میان کروبیان بردار.» آنگاه داخل شده، نزد چرخهاایستاد.
\par 7 و یکی از کروبیان دست خود را از میان کروبیان به آتشی که در میان کروبیان بود درازکرده، آن را برداشت و به‌دست آن مردی که به کتان، ملبس بود نهاد و او آن را گرفته، بیرون رفت.
\par 8 و در کروبیان شبیه صورت دست انسان زیربالهای ایشان ظاهر شد.
\par 9 و نگریستم و اینک چهار چرخ به پهلوی کروبیان یعنی یک چرخ به پهلوی یک کروبی وچرخ دیگر به پهلوی کروبی دیگر ظاهر شد. ونمایش چرخها مثل صورت سنگ زبرجد بود.
\par 10 و اما نمایش ایشان چنین بود. آن چهار را یک شباهت بود که گویا چرخ در میان چرخ باشد.
\par 11 وچون آنها می‌رفت بر چهار جانب خود می‌رفت وحینی که می‌رفت به هیچ سو میل نمی کرد، بلکه به‌جایی که سر به آن متوجه می‌شد از عقب آن می‌رفت. و چون می‌رفت به هیچ سو میل نمی کرد.
\par 12 و تمامی بدن و پشتها و دستها وبالهای ایشان و چرخها یعنی چرخهایی که آن چهار داشتند از هر طرف پر از چشمها بود.
\par 13 وبه سمع من به آن چرخها ندا در‌دادند که «ای چرخها!»
\par 14 و هر یک را چهار رو بود. روی اول روی کروبی بود و روی دوم روی انسان و سوم روی شیر و چهارم روی عقاب.
\par 15 پس کروبیان صعود کردند. این همان حیوان است که نزد نهر خابور دیده بودم.
\par 16 و چون کروبیان می‌رفتند، چرخها به پهلوی ایشان می‌رفت و چون کروبیان بالهای خود را برافراشته، از زمین صعود می‌کردند، چرخها نیز از پهلوی ایشان برنمی گشت.
\par 17 چون ایشان می‌ایستادندآنها می‌ایستاد و چون ایشان صعود می‌نمودند، آنها با ایشان صعود می‌نمود، زیرا که روح حیوان در آنها بود.
\par 18 و جلال خداوند از بالای آستانه خانه بیرون آمد و بر زبر کروبیان قرار گرفت.
\par 19 وچون کروبیان بیرون رفتند بالهای خود رابرافراشته، به نظر من از زمین صعود نمودند. وچرخها پیش روی ایشان بود و نزد دهنه دروازه شرقی خانه خداوند ایستادند. و جلال خدای اسرائیل از طرف بالا بر ایشان قرار گرفت.
\par 20 این همان حیوان است که زیر خدای اسرائیل نزد نهرخابور دیده بودم، پس فهمیدم که اینان کروبیانند.
\par 21 هر یک را چهار روی و هر یک را چهار بال بودو زیر بالهای ایشان شبیه دستهای انسان بود.واما شبیه رویهای ایشان چنین بود. همان رویها بودکه نزد نهر خابور دیده بودم. هم نمایش ایشان وهم خود ایشان (چنان بودند) و هر یک به راه مستقیم می‌رفت.
\par 22 واما شبیه رویهای ایشان چنین بود. همان رویها بودکه نزد نهر خابور دیده بودم. هم نمایش ایشان وهم خود ایشان (چنان بودند) و هر یک به راه مستقیم می‌رفت.

\chapter{11}

\par 1 و روح مرا برداشته، به دروازه شرقی خانه خداوند که بسوی مشرق متوجه است آورد. و اینک نزد دهنه دروازه بیست و پنج مرد بودند و در میان ایشان یازنیا ابن عزور و فلطیاابن بنایا روسای قوم را دیدم.
\par 2 و او مرا گفت: «ای پسر انسان اینها آن کسانی می‌باشند که تدابیرفاسد می‌کنند و در این شهر مشورتهای قبیح می‌دهند.
\par 3 و می‌گویند وقت نزدیک نیست که خانه‌ها را بنا نماییم، بلکه این شهر دیگ است و ما گوشت می‌باشیم.
\par 4 بنابراین برای ایشان نبوت کن. ای پسر انسان نبوت کن.»
\par 5 آنگاه روح خداوند بر من نازل شده، مرافرمود: «بگو که خداوند چنین می‌فرماید: ای خاندان اسرائیل شما به اینطور سخن می‌گویید واما من خیالات دل شما را می‌دانم.
\par 6 بسیاری را دراین شهر کشته‌اید و کوچه هایش را از کشتگان پرکرده‌اید.
\par 7 لهذا خداوند یهوه چنین می‌گوید: کشتگان شما که در میانش گذاشته‌اید، گوشت می‌باشند و شهر دیگ است. لیکن شما را ازمیانش بیرون خواهم برد.
\par 8 شما از شمشیرمی ترسید، اما خداوند یهوه می‌گوید شمشیر را برشما خواهم آورد.
\par 9 و شما را از میان شهر بیرون برده، شما را به‌دست غریبان تسلیم خواهم نمودو بر شما داوری خواهم کرد.
\par 10 به شمشیرخواهید افتاد و در حدود اسرائیل بر شما داوری خواهم نمود و خواهید دانست که من یهوه هستم.
\par 11 این شهر برای شما دیگ نخواهد بود و شما درآن گوشت نخواهید بود، بلکه در حدود اسرائیل بر شما داوری خواهم نمود.
\par 12 و خواهید دانست که من آن یهوه هستم که در فرایض من سلوک ننمودید و احکام مرا بجا نیاوردید، بلکه برحسب احکام امت هایی که به اطراف شمامی باشند عمل نمودید.»
\par 13 و واقع شد که چون نبوت کردم، فلطیا ابن بنایا مرد. پس به روی خوددر‌افتاده، به آواز بلند فریاد نمودم و گفتم: «آه‌ای خداوند یهوه آیا تو بقیه اسرائیل را تمام هلاک خواهی ساخت؟»
\par 14 و کلام خداوند بر من نازل شده، گفت:
\par 15 «ای پسر انسان برادران تو یعنی برادرانت که از اهل خاندان تو می‌باشند و تمامی خاندان اسرائیل جمیع کسانی می‌باشند که سکنه اورشلیم به ایشان می‌گویند: شما از خداوند دورشوید و این زمین به ما به ملکیت داده شده است.
\par 16 بنابراین بگو: خداوند یهوه چنین می‌گوید: اگرچه ایشان را در میان امت‌ها دور کنم و ایشان را درمیان کشورها پراکنده سازم، اما من برای ایشان درآن کشورهایی که به آنها رفته باشند اندک زمانی مقدس خواهم بود.
\par 17 پس بگو خداوند یهوه چنین می‌فرماید: شما را از میان امت‌ها جمع خواهم کرد و شما را از کشورهایی که در آنهاپراکنده شده‌اید فراهم خواهم آورد و زمین اسرائیل را به شما خواهم داد.
\par 18 و به آنجا داخل شده، تمامی مکروهات و جمیع رجاسات آن رااز میانش دور خواهند کرد.
\par 19 و ایشان را یکدل خواهم داد و در اندرون ایشان روح تازه خواهم نهاد و دل سنگی را از جسد ایشان دور کرده، دل گوشتی به ایشان خواهم بخشید.
\par 20 تا در فرایض من سلوک نمایند و احکام مرا نگاه داشته، آنها رابجا آورند. و ایشان قوم من خواهند بود و من خدای ایشان خواهم بود.
\par 21 اما آنانی که دل ایشان از عقب مکروهات و رجاسات ایشان می‌رود، پس خداوند یهوه می‌گوید: من رفتارایشان را بر سر ایشان وارد خواهم آورد.»
\par 22 آنگاه کروبیان بالهای خود را برافراشتند وچرخها به پهلوی ایشان بود و جلال خدای اسرائیل از طرف بالا بر ایشان قرار گرفت.
\par 23 وجلال خداوند از بالای میان شهر صعود نموده، برکوهی که بطرف شرقی شهر است قرار گرفت.
\par 24 و روح مرا برداشت و در عالم رویا مرا به روح خدا به زمین کلدانیان نزد اسیران برد و آن رویایی که دیده بودم از نظر من مرتفع شد.و تمامی کلام خداوند را که به من نشان داده بود، برای اسیران بیان کردم.
\par 25 و تمامی کلام خداوند را که به من نشان داده بود، برای اسیران بیان کردم.

\chapter{12}

\par 1 و کلام خداوند بر من نازل شده، گفت:
\par 2 «ای پسر انسان تو در میان خاندان فتنه انگیز ساکن می‌باشی که ایشان را چشمها به جهت دیدن هست اما نمی بینند و ایشان را گوشهابه جهت شنیدن هست اما نمی شنوند، چونکه خاندان فتنه انگیز می‌باشند.
\par 3 اما تو‌ای پسر انسان اسباب جلای وطن را برای خود مهیا ساز. و درنظر ایشان در وقت روز کوچ کن و از مکان خود به مکان دیگر به حضور ایشان نقل کن، شایدبفهمند. اگرچه خاندان فتنه انگیز می‌باشند.
\par 4 واسباب خود را مثل اسباب جلای وطن در وقت روز به نظر ایشان بیرون آور. و شامگاهان مثل کسانی که برای جلای وطن بیرون می‌روند بیرون شو.
\par 5 و شکافی برای خود در دیوار به حضورایشان کرده، از آن بیرون ببر.
\par 6 و در حضور ایشان آن را بر دوش خود بگذار و در تاریکی بیرون ببر وروی خود را بپوشان تا زمین را نبینی. زیرا که تو راعلامتی برای خاندان اسرائیل قرار داده‌ام.»
\par 7 پس به نهجی که مامور شدم، عمل نمودم واسباب خود را مثل اسباب جلای وطن در وقت روز بیرون آوردم. و شبانگاه شکافی برای خود به‌دست خویش در دیوار کردم و آن را در تاریکی بیرون برده، به حضور ایشان بر دوش برداشتم.
\par 8 وبامدادان کلام خداوند بر من نازل شده، گفت:
\par 9 «ای پسر انسان، آیا خاندان اسرائیل یعنی‌این خاندان فتنه انگیز به تو نگفتند: این چه‌کار است که می‌کنی؟
\par 10 پس به ایشان بگو خداوند یهوه چنین می‌گوید: این وحی اشاره به رئیسی است که دراورشلیم می‌باشد و به تمامی خاندان اسرائیل که ایشان در میان آنها می‌باشند
\par 11 بگو: من علامت برای شما هستم. به نهجی که من عمل نمودم، همچنان به ایشان کرده خواهد شد و جلای وطن شده، به اسیری خواهند رفت.
\par 12 و رئیسی که درمیان ایشان است (اسباب خود را) در تاریکی بردوش نهاده، بیرون خواهد رفت. و شکافی دردیوار خواهند کرد تا از آن بیرون ببرند. و او روی خود را خواهد پوشانید تا زمین را به چشمان خودنبیند.
\par 13 و من دام خود را بر او خواهم گسترانید ودر کمند من گرفتار خواهد شد. و او را به بابل به زمین کلدانیان خواهم برد و اگرچه در آنجاخواهد مرد، ولی آن را نخواهد دید.
\par 14 و جمیع مجاوران و معاونانش و تمامی لشکر او را بسوی هر باد پراکنده ساخته، شمشیری در عقب ایشان برهنه خواهم ساخت.
\par 15 و چون ایشان را در میان امت‌ها پراکنده ساخته و ایشان را در میان کشورهامتفرق نموده باشم، آنگاه خواهند دانست که من یهوه هستم.
\par 16 لیکن عدد قلیلی از میان ایشان ازشمشیر و قحط و وبا باقی خواهم گذاشت تا همه رجاسات خود را در میان امت هایی که به آنهامی روند، بیان نمایند. پس خواهند دانست که من یهوه هستم.»
\par 17 و کلام خداوند بر من نازل شده، گفت:
\par 18 «ای پسر انسان! نان خود را با ارتعاش بخور وآب خویش را با لرزه و اضطراب بنوش.
\par 19 و به اهل زمین بگو خداوند یهوه درباره سکنه اورشلیم و اهل زمین اسرائیل چنین می‌فرماید: که نان خودرا با اضطراب خواهند خورد و آب خود را باحیرت خواهند نوشید. زیرا که زمین آنها به‌سبب ظلم جمیع ساکنانش از هر‌چه در آن است تهی خواهد شد.
\par 20 و شهرهای مسکون ایشان خراب شده، زمین ویران خواهد شد. پس خواهیددانست که من یهوه هستم.»
\par 21 و کلام خداوند بر من نازل شده، گفت:
\par 22 «ای پسر انسان این مثل شما چیست که درزمین اسرائیل می‌زنید و می‌گویید: ایام طویل می‌شود و هر رویا باطل می‌گردد.
\par 23 لهذا به ایشان بگو، خداوند یهوه چنین می‌گوید: این مثل را باطل خواهم ساخت و آن را بار دیگر دراسرائیل نخواهند‌آورد. بلکه به ایشان بگو: ایام، نزدیک است و انجام هر رویا، قریب.
\par 24 زیرا که هیچ‌رویای باطل و غیب گویی تملق‌آمیز در میان خاندان اسرائیل بار دیگر نخواهد بود.
\par 25 زیرا من که یهوه هستم سخن خواهم گفت و سخنی که من می‌گویم، واقع خواهد شد و بار دیگر تاخیرنخواهد افتاد. زیرا خداوند یهوه می‌گوید: ای خاندان فتنه انگیز در ایام شما سخنی خواهم گفت و آن را به انجام خواهم رسانید.»
\par 26 و کلام خداوند بر من نازل شده، گفت:
\par 27 «ای پسر انسان! هان خاندان اسرائیل می‌گویندرویایی که او می‌بیند، به جهت ایام طویل است واو برای زمانهای بعیده نبوت می‌نماید.بنابراین به ایشان بگو: خداوند یهوه چنین می‌فرماید که هیچ کلام من بعد از این تاخیرنخواهد افتاد. و خداوند یهوه می‌فرماید: کلامی که من می‌گویم واقع خواهد شد.»
\par 28 بنابراین به ایشان بگو: خداوند یهوه چنین می‌فرماید که هیچ کلام من بعد از این تاخیرنخواهد افتاد. و خداوند یهوه می‌فرماید: کلامی که من می‌گویم واقع خواهد شد.»

\chapter{13}

\par 1 و کلام خداوند بر من نازل شده، گفت:
\par 2 «ای پسر انسان به ضد انبیای اسرائیل که نبوت می‌نمایند، نبوت نما. و به آنانی که ازافکار خود نبوت می‌کنند، بگو کلام خداوند رابشنوید!
\par 3 خداوند یهوه چنین می‌گوید: وای برانبیاء احمق که تابع روح خویش می‌باشند و هیچ ندیده‌اند.
\par 4 ‌ای اسرائیل انبیای تو مانند روباهان درخرابه‌ها بوده‌اند.
\par 5 شما به رخنه‌ها برنیامدید ودیوار را برای خاندان اسرائیل تعمیر نکردید تاایشان در روز خداوند به جنگ بتوانند ایستاد.
\par 6 رویای باطل و غیب گویی کاذب می‌بینند ومی گویند: خداوند می‌فرماید، با آنکه خداوندایشان را نفرستاده است و مردمان را امیدوارمی سازند به اینکه کلام ثابت خواهد شد.
\par 7 آیارویای باطل ندیدید و غیب گویی کاذب را ذکرنکردید چونکه گفتید خداوند می‌فرماید با آنکه من تکلم ننمودم؟»
\par 8 بنابراین خداوند یهوه چنین می‌فرماید: «چونکه سخن باطل گفتید و رویای کاذب دیدید، اینک خداوند یهوه می‌فرماید من به ضد شماخواهم بود.
\par 9 پس دست من بر انبیایی که رویای باطل دیدند و غیب گویی کاذب کردند درازخواهد شد و ایشان در مجلس قوم من داخل نخواهند شد و در دفتر خاندان اسرائیل ثبت نخواهند گردید و به زمین اسرائیل وارد نخواهندگشت و شما خواهید دانست که من خداوند یهوه می باشم.
\par 10 و از این جهت که قوم مرا گمراه کرده، گفتند که سلامتی است در حینی که سلامتی نبودو یکی از ایشان دیوار را بنا نمود و سایرین آن رابه گل ملاط مالیدند.
\par 11 پس به آنانی که گل ملاطرا مالیدند بگو که آن خواهد افتاد. باران سیال خواهد بارید و شما‌ای تگرگهای سخت خواهیدآمد و باد شدید آن را خواهد شکافت.
\par 12 و هان چون دیوار بیفتد، آیا شما را نخواهند گفت: کجااست آن اندودی که به آن اندود گردید؟»
\par 13 لهذا خداوند یهوه چنین می‌گوید: «من آن را به باد شدید در غضب خود خواهم شکافت وباران سیال در خشم من خواهد بارید و تگرگهای سخت برای فانی ساختن آن در غیظ من خواهدآمد.
\par 14 و آن دیوار را که شما به گل ملاط اندودکردید منهدم نموده، به زمین یکسان خواهم ساخت و پی آن منکشف خواهد شد. و چون آن بیفتد شما در میانش هلاک خواهید شد و خواهیددانست که من یهوه هستم.
\par 15 پس چون خشم خود را بر دیوار و بر آنانی که آن را به گل ملاطاندود کردند به اتمام رسانیده باشم، آنگاه به شماخواهم گفت: دیوار نیست شده و آنانی که آن رااندود کردند نابود گشته‌اند.
\par 16 یعنی انبیای اسرائیل که درباره اورشلیم نبوت می‌نمایند وبرایش رویای سلامتی را می‌بینند با آنکه خداوندیهوه می‌گوید که سلامتی نیست.
\par 17 و تو‌ای پسرانسان نظر خود را بر دختران قوم خویش که ازافکار خود نبوت می‌نمایند بدار و بر ایشان نبوت نما،
\par 18 و بگو خداوند یهوه چنین می‌فرماید: وای بر آنانی که بالشها برای مفصل هر بازویی می‌دوزند و مندیلها برای سر هر قامتی می‌سازند تا جانها را صید کنند! آیا جانهای قوم مرا صیدخواهید کرد و جانهای خود را زنده نگاه خواهیدداشت؟
\par 19 و مرا در میان قوم من برای مشت جویی و لقمه نانی بی‌حرمت می‌کنید چونکه به قوم من که به دروغ شما گوش می‌گیرند دروغ گفته، جانهایی را که مستوجب موت نیستندمی کشید و جانهایی را که مستحق حیات نمی باشند زنده نگاه می‌دارید.
\par 20 لهذا خداوندیهوه چنین می‌گوید: اینک من به ضد بالشهای شما هستم که به واسطه آنها جانها را مثل مرغان صید می‌کنید. و آنها را از بازوهای شما خواهم درید و کسانی را که جانهای ایشان را مثل مرغان صید می‌کنید، رهایی خواهم داد.
\par 21 و مندیلهای شما را خواهم درید و قوم خود را از دست شماخواهم رهانید و دیگر در دست شما نخواهند بودتا ایشان را صید کنید پس خواهید دانست که من یهوه هستم.
\par 22 چونکه شما به دروغ خود، دل مرد عادل را که من محزون نساختم، محزون ساخته‌اید و دستهای مرد شریر را تقویت داده ایدتا از رفتار قبیح خود بازگشت ننماید و زنده نشود.لهذا بار دیگر رویای باطل نخواهید دید وغیب گویی نخواهید نمود. و چون قوم خود را ازدست شما رهایی دهم، آنگاه خواهید دانست که من یهوه می‌باشم.»
\par 23 لهذا بار دیگر رویای باطل نخواهید دید وغیب گویی نخواهید نمود. و چون قوم خود را ازدست شما رهایی دهم، آنگاه خواهید دانست که من یهوه می‌باشم.»

\chapter{14}

\par 1 و کسانی چند از مشایخ اسرائیل نزد من آمده، پیش رویم نشستند.
\par 2 آنگاه کلام خداوند بر من نازل شده، گفت:
\par 3 «ای پسر انسان این اشخاص، بتهای خویش را در دلهای خودجای دادند و سنگ مصادم گناه خویش را پیش روی خود نهادند. پس آیا ایشان از من مسالت نمایند؟
\par 4 لهذا ایشان را خطاب کن و به ایشان بگو: خداوند یهوه چنین می‌فرماید: هرکسی ازخاندان اسرائیل که بتهای خویش را در دل خودجای دهد و سنگ مصادم گناه خویش را پیش روی خود بنهد و نزد نبی بیاید، من که یهوه هستم آن را که می‌آید موافق کثرت بتهایش اجابت خواهم نمود
\par 5 تا خاندان اسرائیل را در افکارخودشان گرفتار سازم چونکه جمیع ایشان به‌سبب بتهای خویش از من مرتد شده‌اند.
\par 6 بنابراین به خاندان اسرائیل بگو خداوند یهوه چنین می‌فرماید: توبه کنید و از بتهای خود بازگشت نمایید و رویهای خویش را از همه رجاسات خود برگردانید.
\par 7 زیرا هر کس چه از خاندان اسرائیل و چه از غریبانی که در اسرائیل ساکن باشند که از پیروی من مرتد شده، بتهای خویش را در دلش جای دهد و سنگ مصادم گناه خود راپیش رویش نهاده، نزد نبی آید تا به واسطه او ازمن مسالت نماید، من که یهوه هستم خود او راجواب خواهم داد.
\par 8 و من نظر خود را بر آن شخص دوخته، او را مورد دهشت خواهم ساخت تا علامتی و ضرب‌المثلی بشود و او را ازمیان قوم خود منقطع خواهم ساخت و خواهیددانست که من یهوه هستم.
\par 9 و اگر نبی فریب خورده، سخنی گوید، من که یهوه هستم آن نبی رافریب داده‌ام و دست خود را بر او دراز کرده، او رااز میان قوم خود اسرائیل منقطع خواهم ساخت.
\par 10 و ایشان بار گناهان خود را متحمل خواهندشد و گناه مسالت کننده مثل گناه آن نبی خواهدبود.
\par 11 تا خاندان اسرائیل دیگر از پیروی من گمراه نشوند و باز به تمامی تقصیرهای خویش نجس نگردند. بلکه خداوند یهوه می‌گوید: ایشان قوم من خواهند بود و من خدای ایشان خواهم بود.»
\par 12 و کلام خداوند بر من نازل شده، گفت:
\par 13 «ای پسر انسان اگر زمینی خیانت کرده، به من خطا ورزد و اگر من دست خود را بر آن دراز کرده، عصای نانش را بشکنم و قحطی در آن فرستاده، انسان و بهایم را از آن منقطع سازم،
\par 14 اگر‌چه این سه مرد یعنی نوح و دانیال و ایوب در آن باشند، خداوند یهوه می‌گوید که ایشان (فقط) جانهای خود را به عدالت خویش خواهند رهانید.
\par 15 واگر حیوانات درنده به آن زمین بیاورم که آن را ازاهل آن خالی سازند و چنان ویران شود که ازترس آن حیوانات کسی از آن گذر نکند،
\par 16 اگرچه این سه مرد در میانش باشند، خداوند یهوه می‌گوید: به حیات خودم قسم که ایشان پسران ودختران را رهایی نخواهند داد. ایشان به تنهایی رهایی خواهند یافت ولی زمین ویران خواهدشد.
\par 17 یا اگر شمشیری به آن زمین آورم و بگویم: ای شمشیر از این زمین بگذر. و اگر انسان و بهایم را از آن منقطع سازم،
\par 18 اگر‌چه این سه مرد درمیانش باشند، خداوند یهوه می‌گوید: به حیات خودم قسم که پسران و دختران را رهایی نخواهند داد بلکه ایشان به تنهایی رهایی خواهندیافت.
\par 19 یا اگر وبا در آن زمین بفرستم و خشم خود را بر آن با خون بریزم و انسان و بهایم را از آن منقطع بسازم،
\par 20 اگر‌چه نوح و دانیال و ایوب درمیانش باشند خداوند یهوه می‌گوید: به حیات خودم قسم که نه پسری و نه دختری را رهایی خواهند داد بلکه ایشان (فقط) جانهای خود را به عدالت خویش خواهند رهانید.
\par 21 پس خداوندیهوه چنین می‌گوید: چه قدر زیاده حینی که چهارعذاب سخت خود یعنی شمشیر و قحط و حیوان درنده و وبا را بر اورشلیم بفرستم تا انسان و بهایم را از آن منقطع سازم.
\par 22 لیکن اینک بقیتی ازپسران و دخترانی که بیرون آورده می‌شوند در آن واگذاشته خواهد شد. هان ایشان را نزد شمابیرون خواهند‌آورد و رفتار و اعمال ایشان راخواهید دید و از بلایی که بر اورشلیم وارد آورده و هر‌آنچه بر آن رسانیده باشم، تسلی خواهیدیافت.و چون رفتار و اعمال ایشان را ببینیدشما را تسلی خواهند داد و خداوند یهوه می‌گوید: شما خواهید دانست که هر‌آنچه به آن کردم بی‌سبب بجا نیاوردم.»
\par 23 و چون رفتار و اعمال ایشان را ببینیدشما را تسلی خواهند داد و خداوند یهوه می‌گوید: شما خواهید دانست که هر‌آنچه به آن کردم بی‌سبب بجا نیاوردم.»

\chapter{15}

\par 1 و کلام خداوند بر من نازل شده، گفت:
\par 2 «ای پسر انسان درخت مو در میان سایر درختان چیست و شاخه مو در میان درختان جنگل چه می‌باشد؟
\par 3 آیا چوب از آن برای کردن هیچ کاری گرفته می‌شود؟ یا میخی از آن برای آویختن هیچ ظرفی می‌گیرند؟
\par 4 هان آن را برای هیزم در آتش می‌اندازند و آتش هر دو طرفش رامی سوزاند و میانش نیم‌سوخته می‌شود پس آیابرای کاری مفید است؟
\par 5 اینک چون تمام بودبرای هیچ کار مصرف نداشت. چند مرتبه زیاده وقتی که آتش آن را سوزانیده و نیم‌سوخته باشد، دیگر برای هیچ کاری مصرف نخواهد داشت.»
\par 6 بنابراین خداوند یهوه چنین می‌گوید: «مثل درخت مو که آن را از میان درختان جنگل برای هیزم و آتش تسلیم کرده‌ام، همچنان سکنه اورشلیم را تسلیم خواهم نمود.
\par 7 و نظر خود را برایشان خواهم دوخت. از یک آتش بیرون می‌آیندو آتشی دیگر ایشان را خواهد سوزانید. پس چون نظر خود را بر ایشان دوخته باشم، خواهیددانست که من یهوه هستم.»و خداوند یهوه می‌گوید: «به‌سبب خیانتی که ورزیده‌اند زمین را ویران خواهم ساخت.»
\par 8 و خداوند یهوه می‌گوید: «به‌سبب خیانتی که ورزیده‌اند زمین را ویران خواهم ساخت.»

\chapter{16}

\par 1 و کلام خداوند بر من نازل شده، گفت:
\par 2 «ای پسر انسان اورشلیم را ازرجاساتش آگاه ساز!
\par 3 و بگو خداوند یهوه به اورشلیم چنین می‌فرماید: اصل و ولادت تو اززمین کنعان است. پدرت اموری و مادرت حتی بود.
\par 4 و اما ولادت تو. در روزی که متولد شدی نافت را نبریدند و تو را به آب غسل ندادند و طاهرنساختند و نمک نمالیدند و به قنداقه نپیچیدند.
\par 5 چشمی بر تو شفقت ننمود و بر تو مرحمت نفرمود تا یکی از اینکارها را برای تو بعمل آورد. بلکه در روز ولادتت جان تو را خوار شمرده، تورا بر روی صحرا انداختند.
\par 6 و من از نزد تو گذرنمودم و تو را در خونت غلطان دیدم. پس تو راگفتم: ای که به خونت آلوده هستی زنده شو! بلی گفتم: ای که به خونت آلوده هستی، زنده شو!
\par 7 وتو را مثل نباتات صحرا بسیار افزودم تا نمو کرده، بزرگ شدی و به زیبایی کامل رسیدی. پستانهایت برخاسته و مویهایت بلند شد، لیکن برهنه و عریان بودی.
\par 8 «و چون از تو گذر کردم برتو نگریستم واینک زمان تو زمان محبت بود. پس دامن خود رابر تو پهن کرده، عریانی تو را مستور ساختم و خداوند یهوه می‌گوید که با تو قسم خوردم و با توعهد بستم و از آن من شدی.
\par 9 و تو را به آب غسل داده، تو را از خونت طاهر ساختم و تو را به روغن تدهین کردم.
\par 10 و تو را به لباس قلابدوزی ملبس ساختم و نعلین پوست خز به پایت کردم و تو را به کتان نازک آراسته و به ابریشم پیراسته ساختم.
\par 11 و تو را به زیورها زینت داده، دستبندها بردستت و گردن بندی بر گردنت نهادم.
\par 12 وحلقه‌ای در بینی و گوشواره‌ها در گوشهایت و تاج جمالی بر سرت نهادم.
\par 13 پس با طلا و نقره آرایش یافتی و لباست از کتان نازک و ابریشم قلابدوزی بود و آرد میده و عسل و روغن خوردی و بی‌نهایت جمیل شده، به درجه ملوکانه ممتاز گشتی.
\par 14 و آوازه تو به‌سبب زیباییت درمیان امت‌ها شایع شد. زیرا خداوند یهوه می‌گویدکه آن زیبایی از جمال من که بر تو نهاده بودم کامل شد.
\par 15 «اما بر زیبایی خود توکل نمودی و به‌سبب آوازه خویش زناکار گردیدی و زنای خویش را برهر رهگذری ریختی و از آن او شد.
\par 16 و ازلباسهای خود گرفتی و مکان های بلند رنگارنگ برای خود ساخته، بر آنها زنا نمودی که مثل اینکارها واقع نشده و نخواهد شد.
\par 17 و زیورهای زینت خود را از طلا و نقره (من که به تو داده بودم )گرفته، تمثالهای مردان را ساخته با آنها زنانمودی.
\par 18 و لباس قلابدوزی خود را گرفته، به آنها پوشانیدی و روغن و بخور مرا پیش آنهاگذاشتی.
\par 19 و نان مرا که به تو داده بودم و آردمیده و روغن و عسل را که رزق تو ساخته بودم، پیش آنها برای هدیه خوشبویی نهادی و چنین شد. قول خداوند یهوه این است.
\par 20 و پسران ودخترانت را که برای من زاییده بودی گرفته، ایشان را به جهت خوراک آنها ذبح نمودی. آیا زنا کاری تو کم بود
\par 21 که پسران مرا نیز کشتی و ایشان راتسلیم نمودی که برای آنها از آتش گذرانیده شوند؟
\par 22 و در تمامی رجاسات و زنای خود ایام جوانی خود را حینی که عریان و برهنه بودی و درخون خود می‌غلطیدی بیاد نیاوردی.»
\par 23 و خداوند یهوه می‌گوید: «وای بر تو! وای بر تو! زیرا بعد از تمامی شرارت خود،
\par 24 خراباتها برای خود بنا نمودی و عمارات بلنددر هر کوچه برای خود ساختی.
\par 25 بسر هر راه عمارتهای بلند خود را بنا نموده، زیبایی خود رامکروه ساختی و برای هر راهگذری پایهای خویش را گشوده، زناکاریهای خود را افزودی.
\par 26 و با همسایگان خود پسران مصر که بزرگ گوشت می‌باشند، زنا نمودی و زناکاری خود راافزوده، خشم مرا بهیجان آوردی.
\par 27 لهذا اینک من دست خود را بر تو دراز کرده، وظیفه تو را قطع نمودم و تو را به آرزوی دشمنانت یعنی دختران فلسطینیان که از رفتار قبیح تو خجل بودند، تسلیم نمودم.
\par 28 و چونکه سیر نشدی با بنی آشور نیز زنا نمودی و با ایشان نیز زنا نموده، سیرنگشتی.
\par 29 و زناکاریهای خود را از زمین کنعان تازمین کلدانیان زیاد نمودی و از این هم سیرنشدی.»
\par 30 خداوند یهوه می‌گوید: «دل تو چه قدرضعیف است که تمامی این اعمال را که کار زن زانیه سلیطه می‌باشد، بعمل آوردی.
\par 31 که بسرهر راه خرابات خود را بنا نمودی و در هر کوچه عمارات بلند خود را ساختی و مثل فاحشه های دیگر نبودی چونکه اجرت را خوار شمردی.
\par 32 ‌ای زن زانیه که غریبان را به‌جای شوهر خودمی گیری!
\par 33 به جمیع فاحشه‌ها اجرت می‌دهند.
\par 34 و عادت تو درزناکاریت برعکس سایر زنان است. چونکه کسی به جهت زناکاری از عقب تو نمی آید و تو اجرت می‌دهی و کسی به تو اجرت نمی دهد. پس عادت تو بر عکس دیگران است.»
\par 35 بنابراین‌ای زانیه! کلام خداوند را بشنو!
\par 36 خداوند یهوه چنین می‌گوید: «چونکه نقد توریخته شد و عریانی تو از زناکاریت با عاشقانت وبا همه بتهای رجاساتت و از خون پسرانت که به آنها دادی مکشوف گردید،
\par 37 لهذا هان من جمیع عاشقانت را که به ایشان ملتذ بودی و همه آنانی راکه دوست داشتی، با همه کسانی که از ایشان نفرت داشتی جمع خواهم نمود. و ایشان را از هرطرف نزد تو فراهم آورده، برهنگی تو را به ایشان مکشوف خواهم ساخت، تا تمامی عریانیت راببینند.
\par 38 و بر تو فتوای زنانی را که زنا می‌کنند وخونریز می‌باشند، خواهم داد. و خون غضب وغیرت را بر تو وارد خواهم آورد.
\par 39 و تو را به‌دست ایشان تسلیم نموده، خراباتهای تو راخراب و عمارات بلند تو را منهدم خواهندساخت. و لباست را از تو خواهند کند و زیورهای قشنگ تو را خواهند گرفت و تو را عریان و برهنه خواهند گذاشت.
\par 40 و گروهی بر تو آورده، تو رابه سنگها سنگسار خواهند کرد و به شمشیرهای خود تو را پاره پاره خواهند نمود.
\par 41 و خانه های تو را به آتش سوزانیده، در نظر زنان بسیار بر توعقوبت خواهند رسانید. پس من تو را از زنا کاری بازخواهم داشت و بار دیگر اجرت نخواهی داد.
\par 42 و حدت خشم خود را بر تو فرو خواهم نشانیدو غیرت من از تو خواهد برگشت و آرام گرفته، بار دیگر غضب نخواهم نمود.
\par 43 چونکه ایام جوانی خود را به یاد نیاورده، مرا به همه اینکارهارنجانیدی، از این جهت خداوند یهوه می‌گوید که اینک نیز رفتار تو را بر سرت وارد خواهم آورد وعلاوه بر تمامی رجاساتت دیگر این عمل قبیح رامرتکب نخواهی شد.
\par 44 «اینک هر‌که مثل می‌آورد این مثل را بر توآورده، خواهد گفت که مثل مادر، مثل دخترش می‌باشد.
\par 45 تو دختر مادر خود هستی که ازشوهر و پسران خود نفرت می‌داشت. و خواهرخواهران خود هستی که از شوهران و پسران خویش نفرت می‌دارند. مادر شما حتی بود و پدرشما اموری.
\par 46 و خواهر بزرگ تو سامره است که با دختران خود بطرف چپ تو ساکن می‌باشد. وخواهر کوچک تو سدوم است که با دختران خودبطرف راست تو ساکن می‌باشد.
\par 47 اما تو درطریق های ایشان سلوک نکردی و مثل رجاسات ایشان عمل ننمودی. بلکه گویا این سهل بود که تو در همه رفتار خود از ایشان زیاده فاسد شدی.»
\par 48 پس خداوند یهوه می‌گوید: «به حیات خودم قسم که خواهر تو سدوم و دخترانش موافق اعمال تو و دخترانت عمل ننمودند.
\par 49 اینک گناه خواهرت سدوم این بود که تکبر وفراوانی نان و سعادتمندی رفاهیت برای او ودخترانش بود و فقیران و مسکینان را دستگیری ننمودند.
\par 50 و مغرور شده، در حضور من مرتکب رجاسات گردیدند. لهذا چنانکه صلاح دیدم ایشان را از میان برداشتم.
\par 51 و سامره نصف گناهانت را مرتکب نشد، بلکه تو رجاسات خودرا از آنها زیاده نمودی و خواهران خود را به تمامی رجاسات خویش که بعمل آوردی مبری ساختی.
\par 52 پس تو نیز که بر خواهران خود حکم دادی خجالت خود را متحمل بشو. زیرا به گناهانت که در آنها بیشتر از ایشان رجاسات نمودی ایشان از تو عادلتر گردیدند. لهذا تو نیزخجل شو و رسوایی خود را متحمل باش چونکه خواهران خود را مبری ساختی.
\par 53 و من اسیری ایشان یعنی اسیری سدوم و دخترانش و اسیری سامره و دخترانش و اسیری اسیران تو را در میان ایشان خواهم برگردانید.
\par 54 تا خجالت خود رامتحمل شده، از هر‌چه کرده‌ای شرمنده شوی چونکه ایشان را تسلی داده‌ای.
\par 55 و خواهرانت یعنی سدوم و دخترانش به حالت نخستین خودخواهند برگشت. و سامره و دخترانش به حالت نخستین خود خواهند برگشت. و تو و دخترانت به حالت نخستین خود خواهید برگشت.
\par 56 اماخواهر تو سدوم در روز تکبر تو به زبانت آورده نشد.
\par 57 قبل از آنکه شرارت تو مکشوف بشود. مثل آن زمانی که دختران ارام مذمت می‌کردند وجمیع مجاورانش یعنی دختران فلسطینیان که تورا از هر طرف خوار می‌شمردند.»
\par 58 پس خداوند می‌فرماید که «تو قباحت ورجاسات خود را متحمل خواهی شد.
\par 59 زیراخداوند یهوه چنین می‌گوید: به نهجی که تو عمل نمودی من با تو عمل خواهم نمود، زیرا که قسم را خوار شمرده، عهد را شکستی.
\par 60 لیکن من عهد خود را که در ایام جوانیت با تو بستم به یادخواهم آورد و عهد جاودانی با تو استوار خواهم داشت.
\par 61 و هنگامی که خواهران بزرگ و کوچک خود را پذیرفته باشی، آنگاه راههای خود را به یاد آورده، خجل خواهی شد. و من ایشان را به‌جای دختران به تو خواهم داد، لیکن نه از عهد تو.
\par 62 و من عهد خود را با تو استوار خواهم ساخت وخواهی دانست که من یهوه هستم.تا آنکه به یاد آورده، خجل شوی. و خداوند یهوه می‌فرماید که چون من همه کارهای تو را آمرزیده باشم، بار دیگر به‌سبب رسوایی خویش دهان خود را نخواهی گشود.»
\par 63 تا آنکه به یاد آورده، خجل شوی. و خداوند یهوه می‌فرماید که چون من همه کارهای تو را آمرزیده باشم، بار دیگر به‌سبب رسوایی خویش دهان خود را نخواهی گشود.»

\chapter{17}

\par 1 و کلام خداوند بر من نازل شده، گفت:
\par 2 «ای پسر انسان، معمایی بیاور و مثلی درباره خاندان اسرائیل بزن.
\par 3 و بگو خداوند یهوه چنین می‌فرماید: عقاب بزرگ که بالهای سترگ ونیهای دراز پر از پرهای رنگارنگ دارد به لبنان آمد و سر سرو آزاد را گرفت.
\par 4 و سر شاخه هایش را کنده، آن را به زمین تجارت آورده، در شهرسوداگران گذاشت.
\par 5 و از تخم آن زمین گرفته، آن را در زمین باروری نهاد و نزد آبهای بسیارگذاشته، آن را مثل درخت بید، غرس نمود.
\par 6 وآن نمو کرده، مو وسیع کوتاه قد گردید که شاخه هایش بسوی او مایل شد و ریشه هایش درزیر وی می‌بود. پس موی شده شاخه‌ها رویانید ونهالها آورد.
\par 7 و عقاب بزرگ دیگری با بالهای سترگ و پرهای بسیار آمد و اینک این موریشه های خود را بسوی او برگردانید وشاخه های خویش را از کرته های بستان خودبطرف او بیرون کرد تا او وی را سیراب نماید.
\par 8 درزمین نیکو نزد آبهای بسیار کاشته شد تا شاخه هارویانیده، میوه بیاورد و مو قشنگ گردد.
\par 9 بگو که خداوند یهوه چنین می‌فرماید: پس آیا کامیاب خواهد شد؟ آیا او ریشه هایش را نخواهد کند ومیوه‌اش را نخواهد چید تا خشک شود؟ تمامی برگهای تازه‌اش خشک خواهد شد و بدون قوت عظیم و خلق بسیاری از ریشه‌ها کنده خواهد شد.
\par 10 اینک غرس شده است اما کامیاب نخواهدشد. بلکه چون باد شرقی بر آن بوزد بالکل خشک خواهد شد و در بوستانی که در آن رویید پژمرده خواهد گردید.»
\par 11 و کلام خداوند بر من نازل شده، گفت:
\par 12 «به این خاندان متمرد بگو که آیا معنی‌این چیزها را نمی دانید؟ بگو که اینک پادشاه بابل به اورشلیم آمده، پادشاه و سرورانش را گرفت وایشان را نزد خود به بابل برد.
\par 13 و از ذریه ملوک گرفته، با او عهد بست و او را قسم داد و زورآوران زمین را برد.
\par 14 تا آنکه مملکت پست شده، سربلند نکند اما عهد او را نگاه داشته، استواربماند.
\par 15 و لیکن او از وی عاصی شده، ایلچیان خود را به مصر فرستاد تا اسبان و خلق بسیاری به او بدهند. آیا کسی‌که اینکارها را کرده باشد، کامیاب شود یا رهایی یابد؟ و یا کسی‌که عهد راشکسته است خلاصی خواهد یافت؟»
\par 16 خداوند یهوه می‌گوید: «به حیات خودم قسم که البته در مکان آن پادشاه که او را به پادشاهی نصب کرد و او قسم وی را خوارشمرده، عهد او را شکست یعنی نزد وی در میان بابل خواهد مرد.
\par 17 و چون سنگرها بر پا سازند وبرجها بنا نمایند تا جانهای بسیاری را منقطع سازند، آنگاه فرعون با لشکر عظیم و گروه کثیر اورا در جنگ اعانت نخواهد کرد.
\par 18 چونکه قسم راخوار شمرده، عهد را شکست و بعد از آنکه دست خود را داده بود همه اینکارها را بعمل آورد، پس رهایی نخواهی یافت.»
\par 19 بنابراین خداوند یهوه چنین می‌گوید: «به حیات خودم قسم که سوگند مرا که او خوار شمرده و عهد مراکه شکسته است البته آنها را بر سر او وارد خواهم آورد.
\par 20 و دام خود را بر او خواهم گسترانید و اودر کمند من گرفتار خواهد شد و او را به بابل آورده، در آنجا بر وی درباره خیانتی که به من ورزیده است محاکمه خواهم نمود.
\par 21 و تمامی فراریانش با جمیع افواجش از شمشیر خواهندافتاد و بقیه ایشان بسوی هر باد پراکنده خواهندشد و خواهید دانست که من که یهوه می‌باشم این را گفته‌ام.»
\par 22 خداوند یهوه چنین می‌فرماید: «من سر بلندسرو آزاد را گرفته، آن را خواهم کاشت و از سراغصانش شاخه تازه کنده، آن را بر کوه بلند و رفیع غرس خواهم نمود.
\par 23 آن را بر کوه بلند اسرائیل خواهم کاشت و آن شاخه‌ها رویانیده، میوه خواهد آورد. و سرو آزاد قشنگ خواهد شد که هر قسم مرغان بالدار زیر آن ساکن شده، در سایه شاخه هایش آشیانه خواهند گرفت.و تمامی در ختان صحرا خواهند دانست که من یهوه درخت بلند را پست می‌کنم و درخت پست رابلند می‌سازم و درخت سبز را خشک و درخت خشک را بارور می‌سازم. من که یهوه هستم این راگفته‌ام و بجا خواهم آورد.»
\par 24 و تمامی در ختان صحرا خواهند دانست که من یهوه درخت بلند را پست می‌کنم و درخت پست رابلند می‌سازم و درخت سبز را خشک و درخت خشک را بارور می‌سازم. من که یهوه هستم این راگفته‌ام و بجا خواهم آورد.»

\chapter{18}

\par 1 و کلام خداوند بر من نازل شده، گفت:
\par 2 «شما چه‌کار دارید که این مثل رادرباره زمین اسرائیل می‌زنید و می‌گویید: پدران انگور ترش خوردند و دندانهای پسران کندگردید.»
\par 3 خداوند یهوه می‌گوید: «به حیات خودم قسم که بعد از این این مثل را در اسرائیل نخواهیدآورد.
\par 4 اینک همه جانها از آن منند چنانکه جان پدر است، همچنین جان پسر نیز، هردوی آنها ازآن من می‌باشند. هر کسی‌که گناه ورزد او خواهدمرد.
\par 5 و اگر کسی عادل باشد و انصاف و عدالت رابعمل آورد،
\par 6 و بر کوهها نخورد و چشمان خودرا بسوی بتهای خاندان اسرائیل برنیفرازد و زن همسایه خود را بی‌عصمت نکند و به زن حایض نزدیکی ننماید،
\par 7 و بر کسی ظلم نکند و گروقرضدار را به او رد نماید و مال کسی را به غصب نبرد، بلکه نان خود را به گرسنگان بدهد وبرهنگان را به‌جامه بپوشاند،
\par 8 و نقد را به سودندهد و ربح نگیرد، بلکه دست خود را از ستم برداشته، انصاف حقیقی را در میان مردمان اجرادارد،
\par 9 و به فرایض من سلوک نموده و احکام مرانگاه داشته، به راستی عمل نماید. خداوند یهوه می‌فرماید که آن شخص عادل است و البته زنده خواهد ماند.
\par 10 «اما اگر او پسری ستم پیشه و خونریز تولیدنماید که یکی از این کارها را بعمل آورد،
\par 11 وهیچکدام از آن اعمال نیکو را بعمل نیاورد بلکه برکوهها نیز بخورد و زن همسایه خود را بی‌عصمت سازد،
\par 12 و بر فقیران و مسکینان ظلم نموده، مال مردم را به غصب ببرد و گرو را پس ندهد، بلکه چشمان خود را بسوی بتها برافراشته، مرتکب رجاسات بشود،
\par 13 و نقد را به سود داده، ربح گیرد، آیا او زنده خواهد ماند؟ البته او زنده نخواهد ماند و به‌سبب همه رجاساتی که بجاآورده است خواهد مرد و خونش بر سرش خواهد بود.
\par 14 «و اگر پسری تولید نماید که تمامی گناهان را که پدرش بجا می‌آورد دیده، بترسد و مثل آنهاعمل ننماید،
\par 15 و بر کوهها نخورد و چشمان خود را بسوی بتهای خاندان اسرائیل برنیفرازد وزن همسایه خویش را بی‌عصمت نکند،
\par 16 و برکسی ظلم نکند و گرو نگیرد و مال احدی را به غصب نبرد بلکه نان خود را به گرسنگان دهد وبرهنگان را به‌جامه پوشاند،
\par 17 و دست خود را ازفقیران برداشته، سود و ربح نگیرد و احکام مرا بجاآورده، به فرایض من سلوک نماید، او به‌سبب گناه پدرش نخواهد مرد بلکه البته زنده خواهد ماند.
\par 18 و اما پدرش چونکه با برادران خود به شدت ظلم نموده، مال ایشان را غصب نمود و اعمال شنیع را در میان قوم خود بعمل آورد او البته به‌سبب گناهانش خواهد مرد.
\par 19 «لیکن شما می‌گویید چرا چنین است؟ آیاپسر متحمل گناه پدرش نمی باشد؟ اگر پسرانصاف و عدالت را بجا آورده، تمامی فرایض مرانگاه دارد و به آنها عمل نماید، او البته زنده خواهد ماند.
\par 20 هر‌که گناه کند او خواهد مرد. پسر متحمل گناه پدرش نخواهد بود و پدرمتحمل گناه پسرش نخواهد بود. عدالت مردعادل بر خودش خواهد بود و شرارت مرد شریربر خودش خواهد بود.
\par 21 «و اگر مرد شریر از همه گناهانی که ورزیده باشد بازگشت نماید و جمیع فرایض مرا نگاه داشته، انصاف و عدالت را بجا آورد او البته زنده مانده نخواهد مرد.
\par 22 تمامی تقصیرهایی که کرده باشد به ضد او به یاد آورده نخواهد شد بلکه در عدالتی که کرده باشد زنده خواهد ماند.»
\par 23 خداوند یهوه می‌فرماید: «آیا من از مردن مرد شریر مسرور می‌باشم؟ نی بلکه از اینکه ازرفتار خود بازگشت نموده، زنده ماند.
\par 24 و اگرمرد عادل از عدالتش برگردد و ظلم نموده، موافق همه رجاساتی که شریران می‌کنند عمل نماید آیااو زنده خواهد ماند؟ نی بلکه تمامی عدالت او که کرده است به یاد آورده نخواهد شد و در خیانتی که نموده و در گناهی که ورزیده است خواهدمرد.
\par 25 «اما شما می‌گویید که طریق خداوندموزون نیست. پس حال‌ای خاندان اسرائیل بشنوید: آیا طریق من غیر موزون است و آیاطریق شما غیر موزون نیست؟
\par 26 چونکه مردعادل از عدالتش برگردد و ظلم کند در آن خواهدمرد. به‌سبب ظلمی که کرده است خواهد مرد.
\par 27 و چون مرد شریر را از شرارتی که کرده است بازگشت نماید و انصاف و عدالت را بجا آورد، جان خود را زنده نگاه خواهد داشت.
\par 28 چونکه تعقل نموده، از تمامی تقصیرهایی که کرده بودبازگشت کرد البته زنده خواهد ماند و نخواهدمرد.
\par 29 لیکن شما‌ای خاندان اسرائیل می‌گوییدکه طریق خداوند موزون نیست. ای خاندان اسرائیل آیا طریق من غیر موزون است و آیاطریق شما غیر موزون نیست؟»
\par 30 بنابراین خداوند یهوه می‌گوید: «ای خاندان اسرائیل من برهریک از شما موافق رفتارش داوری خواهم نمود. پس توبه کنید و از همه تقصیرهای خودبازگشت نمایید تا گناه موجب هلاکت شما نشود.
\par 31 تمامی تقصیرهای خویش را که مرتکب آنها شده‌اید از خود دور اندازید و دل تازه و روح تازه‌ای برای خود ایجاد کنید. زیرا که‌ای خاندان اسرائیل برای چه بمیرید؟زیرا خداوند یهوه می‌گوید: من از مرگ آنکس که می‌میرد مسرورنمی باشم. پس بازگشت نموده، زنده مانید.»
\par 32 زیرا خداوند یهوه می‌گوید: من از مرگ آنکس که می‌میرد مسرورنمی باشم. پس بازگشت نموده، زنده مانید.»

\chapter{19}

\par 1 «پس تو این مرثیه را برای سروران اسرائیل بخوان
\par 2 و بگو: مادر تو چه بود. او در میان شیران شیر ماده می‌خوابید وبچه های خود را در میان شیران ژیان می‌پرورد.
\par 3 و یکی از بچه های خود را تربیت نمود که شیرژیان گردید و به دریدن شکار آموخته شد ومردمان را خورد.
\par 4 و چون امت‌ها خبر او راشنیدند، در حفره ایشان گرفتار گردید و او را درغلها به زمین مصر بردند.
\par 5 و چون مادرش دید که بعد از انتظار کشیدن امیدش بریده شد، پس ازبچه هایش دیگری را گرفته، او را شیری ژیان ساخت.
\par 6 و او در میان شیران گردش کرده، شیرژیان گردید و به دریدن شکار آموخته شده، مردمان را خورد.
\par 7 و قصرهای ایشان را ویران وشهرهای ایشان را خراب نمود و زمین و هرچه درآن بود از آواز غرش او تهی گردید.
\par 8 و امت‌ها ازکشورها از هر طرف بر او هجوم آورده، دام خودرا بر او گسترانیدند که به حفره ایشان گرفتار شد.
\par 9 و او را در غلها کشیده، در قفس گذاشتند و نزدپادشاه بابل بردند واو را در قلعه‌ای نهادند تا آوازاو دیگر بر کوههای اسرائیل مسموع نشود.
\par 10 «مادر تو مثل درخت مو مانند خودت نزدآبها غرس شده، به‌سبب آبهای بسیار میوه آورد و شاخه بسیار داشت.
\par 11 و شاخه های قوی برای عصاهای سلاطین داشت. و قد آن در میان شاخه های پر برگ به حدی بلند شد که از کثرت اغصانش ارتفاعش نمایان گردید.
\par 12 اما به غضب کنده و به زمین انداخته شد. و باد شرقی میوه‌اش را خشک ساخت و شاخه های قویش شکسته وخشک گردیده، آتش آنها را سوزانید.
\par 13 و الان در بیابان در زمین خشک و تشنه مغروس است.و آتش از عصاهای شاخه هایش بیرون آمده، میوه‌اش را سوزانید. به نوعی که یک شاخه قوی برای عصای سلاطین نمانده است. این مرثیه است و مرثیه خواهد بود.»
\par 14 و آتش از عصاهای شاخه هایش بیرون آمده، میوه‌اش را سوزانید. به نوعی که یک شاخه قوی برای عصای سلاطین نمانده است. این مرثیه است و مرثیه خواهد بود.»

\chapter{20}

\par 1 و در روز دهم ماه پنجم از سال هفتم بعضی از مشایخ اسرائیل به جهت طلبیدن خداوند آمدند و پیش من نشستند.
\par 2 آنگاه کلام خداوند بر من نازل شده، گفت:
\par 3 «ای پسر انسان مشایخ اسرائیل را خطاب کرده، به ایشان بگو: خداوند یهوه چنین می‌فرماید: آیاشما برای طلبیدن من آمدید؟ خداوند یهوه می‌گوید: به حیات خودم قسم که از شما طلبیده نخواهم شد.
\par 4 ‌ای پسر انسان آیا بر ایشان حکم خواهی کرد؟ آیا بر ایشان حکم خواهی کرد؟ پس رجاسات پدران ایشان را بدیشان بفهمان.
\par 5 وبه ایشان بگو: خداوند یهوه چنین می‌فرماید درروزی که اسرائیل را برگزیدم و دست خود رابرای ذریت خاندان یعقوب برافراشتم و خود را به ایشان در زمین مصر معروف ساختم و دست خودرا برای ایشان برافراشته، گفتم: من یهوه خدای شما هستم،
\par 6 در همان روز دست خود را برای ایشان برافراشتم که ایشان را از زمین مصر به زمینی که برای ایشان بازدید کرده بودم بیرون آورم. زمینی که به شیر و شهد جاری است و فخرهمه زمینها می‌باشد.
\par 7 و به ایشان گفتم: هر کس ازشما رجاسات چشمان خود را دور کند وخویشتن را به بتهای مصر نجس نسازد، زیرا که من یهوه خدای شما هستم.
\par 8 اما ایشان از من عاصی شده، نخواستند که به من گوش گیرند. وهر کس از ایشان رجاسات چشمان خود را دورنکرد و بتهای مصر را ترک ننمود. آنگاه گفتم که خشم خود را بر ایشان خواهم ریخت و غضب خویش را در میان زمین مصر بر ایشان به اتمام خواهم رسانید.
\par 9 لیکن محض خاطر اسم خودعمل نمودم تا آن در نظر امت هایی که ایشان درمیان آنها بودند و در نظر آنها خود را به بیرون آوردن ایشان از زمین مصر، به ایشان شناسانیدم، بی‌حرمت نشود.
\par 10 پس ایشان را از زمین مصربیرون آورده، به بیابان رسانیدم.
\par 11 و فرایض خویش را به ایشان دادم و احکام خود را که هر‌که به آنها عمل نماید به آنها زنده خواهد ماند، به ایشان تعلیم دادم.
\par 12 و نیز سبت های خود را به ایشان عطا فرمودم تا علامتی در میان من و ایشان بشود و بدانند که من یهوه هستم که ایشان راتقدیس می‌نمایم.
\par 13 «لیکن خاندان اسرائیل در بیابان از من عاصی شده، در فرایض من سلوک ننمودند. واحکام مرا که هر‌که به آنها عمل نماید از آنها زنده ماند، خوار شمردند و سبت هایم را بسیاربی حرمت نمودند. آنگاه گفتم که خشم خود را بر ایشان ریخته، ایشان را در بیابان هلاک خواهم ساخت.
\par 14 لیکن محض خاطر اسم خود عمل نمودم تا آن به نظر امت هایی که ایشان را به حضور آنها بیرون آوردم بی‌حرمت نشود.
\par 15 ومن نیز دست خود را برای ایشان در بیابان برافراشتم که ایشان را به زمینی که به ایشان داده بودم، داخل نسازم. زمینی که به شیر و شهدجاری است و فخر تمامی زمینها می‌باشد.
\par 16 زیرا که احکام مرا خوار شمردند وبه فرایضم سلوک ننمودند و سبت های مرا بی‌حرمت ساختند، چونکه دل ایشان به بتهای خود مایل می‌بود.
\par 17 لیکن خشم من بر ایشان رقت نموده، ایشان را هلاک نساختم و ایشان را در بیابان، نابودننمودم.
\par 18 و به پسران ایشان در بیابان گفتم: به فرایض پدران خود سلوک منمایید و احکام ایشان را نگاه مدارید و خویشتن را به بتهای ایشان نجس مسازید.
\par 19 من یهوه خدای شما هستم. پس به فرایض من سلوک نمایید و احکام مرا نگاه داشته، آنها را بجا آورید.
\par 20 و سبت های مراتقدیس نمایید تا در میان من و شما علامتی باشدو بدانید که من یهوه خدای شما هستم.
\par 21 «لیکن پسران از من عاصی شده، به فرایض من سلوک ننمودند و احکام مرا که هر‌که آنها رابجا آورد از آنها زنده خواهد ماند، نگاه نداشتند وبه آنها عمل ننمودند و سبت های مرا بی‌حرمت ساختند. آنگاه گفتم که خشم خود را بر ایشان ریخته، غضب خویش را بر ایشان در بیابان به اتمام خواهم رسانید.
\par 22 لیکن دست خود رابرگردانیده، محض خاطر اسم خود عمل نمودم تا آن به نظر امت هایی که ایشان را به حضور آنها بیرون آوردم بی‌حرمت نشود.
\par 23 و من نیز دست خود را برای ایشان در بیابان برافراشتم که ایشان را در میان امت‌ها پراکنده نمایم و ایشان را درکشورها متفرق سازم.
\par 24 زیرا که احکام مرا بجانیاوردند و فرایض مرا خوار شمردند و سبت های مرا بی‌حرمت ساختند و چشمان ایشان بسوی بتهای پدران ایشان نگران می‌بود.
\par 25 بنابراین من نیز فرایضی را که نیکو نبود و احکامی را که از آنهازنده نمانند به ایشان دادم.
\par 26 و ایشان را به هدایای ایشان که هر کس را که رحم را می‌گشوداز آتش می‌گذرانیدند، نجس ساختم تا ایشان راتباه سازم و بدانند که من یهوه هستم.
\par 27 «بنابراین‌ای پسر انسان خاندان اسرائیل راخطاب کرده، به ایشان بگو: خداوند یهوه چنین می‌فرماید: در این دفعه نیز پدران شما خیانت کرده، به من کفر ورزیدند.
\par 28 زیرا که چون ایشان را به زمینی که دست خود را برافراشته بودم که آن را به ایشان بدهم در‌آوردم، آنگاه به هر تل بلند وهر درخت کشن نظر انداختند و ذبایح خود را درآنجا ذبح نمودند و قربانی های غضب انگیزخویش را گذرانیدند. و در آنجا هدایای خوشبوی خود را آوردند و در آنجا هدایای ریختنی خود را ریختند.
\par 29 و به ایشان گفتم: این مکان بلند که شما به آن می‌روید چیست؟ پس اسم آن تا امروز بامه خوانده می‌شود.
\par 30 «بنابراین به خاندان اسرائیل بگو: خداوندیهوه چنین می‌فرماید: آیا شما به رفتار پدران خود خویشتن را نجس می‌سازید و رجاسات ایشان را پیروی نموده، زنا می‌کنید؟
\par 31 و هدایای خود را آورده، پسران خویش را از آتش می‌گذرانید و خویشتن را از تمامی بتهای خود تاامروز نجس می‌سازید؟ پس‌ای خاندان اسرائیل آیا من از شما طلبیده بشوم؟ خداوند یهوه می‌فرماید به حیات خودم قسم که از شما طلبیده نخواهم شد.
\par 32 و آنچه به‌خاطر شما خطورمی کند هرگز واقع نخواهد شد که خیال می‌کنید. مثل امت‌ها و مانند قبایل کشورها گردیده، (بتهای ) چوب و سنگ را عبادت خواهید نمود.
\par 33 زیرا خداوند یهوه می‌فرماید: به حیات خودم قسم که هرآینه با دست قوی و بازوی برافراشته وخشم ریخته شده بر شما سلطنت خواهم نمود.
\par 34 و شما را از میان امت‌ها بیرون آورده، به‌دست قوی و بازوی برافراشته و خشم ریخته شده اززمینهایی که در آنها پراکنده شده‌اید جمع خواهم نمود.
\par 35 و شما را به بیابان امت‌ها در‌آورده، درآنجا بر شما روبرو داوری خواهم نمود.
\par 36 وخداوند یهوه می‌گوید: چنانکه بر پدران شما دربیابان زمین مصر داوری نمودم، همچنین بر شماداوری خواهم نمود.
\par 37 و شما را زیر عصاگذرانیده، به بند عهد درخواهم آورد.
\par 38 و آنانی را که متمرد شده و از من عاصی گردیده‌اند، ازمیان شما جدا خواهم نمود و ایشان را از زمین غربت ایشان بیرون خواهم آورد. لیکن به زمین اسرائیل داخل نخواهند شد و خواهید دانست که من یهوه هستم.»
\par 39 اما به شما‌ای خاندان اسرائیل خداوند یهوه چنین می‌گوید: «همه شما نزدبتهای خود رفته، آنها را عبادت کنید. لیکن بعد ازاین البته مرا گوش خواهید داد. و اسم قدوس مرا دیگر با هدایا و بتهای خود بی‌عصمت نخواهیدساخت.
\par 40 زیرا خداوند یهوه می‌فرماید: در کوه مقدس من بر کوه بلند اسرائیل تمام خاندان اسرائیل جمیع در آنجا مرا عبادت خواهند کردو در آنجا از ایشان راضی شده، ذبایح جنبانیدنی شما و نوبرهای هدایای شما را با تمامی موقوفات شما خواهم طلبید.
\par 41 و چون شما را ازامت‌ها بیرون آورم و شما را از زمینهایی که درآنها پراکنده شده‌اید جمع نمایم، آنگاه هدایای خوشبوی شما را از شما قبول خواهم کرد و به نظر امت‌ها در میان شما تقدیس کرده خواهم شد.
\par 42 و چون شما را به زمین اسرائیل یعنی به زمینی که درباره‌اش دست خود را برافراشتم که آن را به پدران شما بدهم بیاورم، آنگاه خواهید دانست که من یهوه هستم.
\par 43 و در آنجا طریق های خود وتمامی اعمال خویش را که خویشتن را به آنهانجس ساخته‌اید، به یاد خواهید آورد. و از همه اعمال قبیح که کرده‌اید، خویشتن را به نظر خودمکروه خواهید داشت.
\par 44 و‌ای خاندان اسرائیل خداوند یهوه می‌فرماید: هنگامی که با شمامحض خاطر اسم خود و نه به سزای رفتار قبیح شما و نه موافق اعمال فاسد شما عمل نموده باشم، آنگاه خواهید دانست که من یهوه هستم.»
\par 45 و کلام خداوند بر من نازل شده، گفت:
\par 46 «ای پسر انسان روی خود را بسوی جنوب متوجه ساز و به سمت جنوب تکلم نما و برجنگل صحرای جنوب نبوت کن.
\par 47 و به آن جنگل جنوب بگو: کلام خداوند را بشنو. خداوند یهوه چنین می‌فرماید: اینک من آتشی درتو می‌افروزم که هر درخت سبز و هر درخت خشک را در تو خواهد سوزانید. و لهیب ملتهب آن خاموش نخواهد شد و همه رویها از جنوب تا شمال از آن سوخته خواهد شد.
\par 48 و تمامی بشرخواهند فهمید که من یهوه آن را افروخته‌ام تاخاموشی نپذیرد.»و من گفتم: «آه‌ای خداوند یهوه ایشان درباره من می‌گویند آیا او مثلها نمی آورد؟»
\par 49 و من گفتم: «آه‌ای خداوند یهوه ایشان درباره من می‌گویند آیا او مثلها نمی آورد؟»

\chapter{21}

\par 1 و کلام خداوند بر من نازل شده، گفت:
\par 2 «ای پسر انسان روی خود را بسوی اورشلیم بدار و به مکان های بلند مقدس تکلم نما. و به زمین اسرائیل نبوت کن.
\par 3 و به زمین اسرائیل بگو: خداوند چنین می‌فرماید: اینک من به ضد توهستم. و شمشیر خود را از غلافش کشیده، عادلان و شریران را از میان تو منقطع خواهم ساخت.
\par 4 و چونکه عادلان و شریران را از میان تومنقطع می‌سازم، بنابراین شمشیر من بر تمامی بشر از جنوب تا شمال از غلافش بیرون خواهدآمد.
\par 5 و تمامی بشر خواهند فهمید که من یهوه شمشیر خود را از غلافش بیرون کشیدم تا باز به آن برنگردد.
\par 6 پس تو‌ای پسر انسان آه بکش! باشکستگی کمر و مرارت سخت به نظر ایشان آه بکش.
\par 7 و اگر به تو گویند که چرا آه می‌کشی؟ بگو: به‌سبب آوازه‌ای که می‌آید. زیرا که همه دلهاگداخته و تمامی دستها سست گردیده و همه جانها کاهیده و جمیع زانوها مثل آب بیتاب خواهد شد. خداوند یهوه می‌گوید: همانا آن می‌آید و به وقوع خواهد پیوست.»
\par 8 و کلام خداوند بر من نازل شده، گفت:
\par 9 «ای پسر انسان نبوت کرده، بگو: خداوند چنین می‌فرماید: بگو که شمشیر، شمشیر تیز شده و نیزصیقلی گردیده است.
\par 10 تیز شده است تا کشتارنماید و صیقلی گردیده تا براق شود. پس آیا ما شادی نماییم؟ عصای پسر من همه درختان راخوار می‌شمارد.
\par 11 و آن برای صیقلی شدن داده شد تا آن را به‌دست گیرند. و این شمشیر تیز شده و صیقلی گردیده است تا به‌دست قاتل داده شود.
\par 12 ‌ای پسر انسان فریاد برآور و ولوله نما زیرا که این بر قوم من و بر جمیع سروران اسرائیل واردمی آید. ترسها به‌سبب شمشیر بر قوم من عارض شده است. لهذا بر ران خود دست بزن.
\par 13 زیرا که امتحان است. و چه خواهد بود اگر عصایی که (دیگران را) خوار می‌شمارد، دیگر نباشد. قول خداوند یهوه این است:
\par 14 و تو‌ای پسر انسان نبوت کن و دستهای خود را بهم بزن و شمشیردفعه سوم تکرار بشود. شمشیر مقتولان است. شمشیر آن مقتول عظیم که ایشان را احاطه می‌کند.
\par 15 شمشیر برنده‌ای به ضد همه دروازه های ایشان قرار دادم تا دلها گداخته شود وهلاکت‌ها زیاده شود. آه (شمشیر) براق گردیده وبرای کشتار تیز شده است.
\par 16 جمع شده، به‌جانب راست برو و آراسته گردیده، به‌جانب چپ توجه نما. بهر طرف که رخسارهایت متوجه می‌باشد.
\par 17 و من نیز دستهای خود را بهم خواهم زد و حدت خشم خویش را ساکن خواهم گردانید. من یهوه هستم که تکلم نموده‌ام.»
\par 18 و کلام خداوند بر من نازل شده، گفت:
\par 19 «وتو‌ای پسر انسان دو راه به جهت خود تعیین نما تاشمشیر پادشاه بابل از آنها بیاید. هر دوی آنها ازیک زمین بیرون می‌آید. و علامتی بر پا کن. آن رابر سر راه شهر بر پا نما.
\par 20 راهی تعیین نما تاشمشیر به ربه بنی عمون و به یهودا در اورشلیم منیع بیاید.
\par 21 زیرا که پادشاه بابل بر شاهراه، به‌سر دو راه ایستاده است تا تفال زند و تیرها را بهم زده، از ترافیم سوال می‌کند و به جگر می‌نگرد.
\par 22 به‌دست راستش تفال اورشلیم است تامنجنیقها بر پا کند و دهان را برای کشتار بگشایدو آواز را به گلبانگ بلند نماید و منجنیقها بردروازه‌ها بر پا کند و سنگرها بسازد و برجها بنانماید.
\par 23 لیکن در نظر ایشان که قسم برای آنهاخورده‌اند، تفال باطل می‌نماید. و او گناه ایشان رابه یاد می‌آورد تا گرفتار شوند.»
\par 24 بنابراین خداوند یهوه چنین می‌گوید: «چونکه شما تقصیرهای خویش را منکشف ساخته و خطایای خود را در همه اعمال خویش ظاهر نموده، عصیان خود را یاد آورانیدید، پس چون به یاد آورده شدید دستگیر خواهید شد.
\par 25 و تو‌ای رئیس شریر اسرائیل که به زخم مهلک مجروح شده‌ای و اجل تو در زمان عقوبت آخررسیده است،
\par 26 خداوند یهوه چنین می‌گوید: عمامه را دور کن و تاج را بردار. چنین نخواهدماند. آنچه را که پست است بلند نما و آنچه را که بلند است پست کن.
\par 27 و من آن را سرنگون، سرنگون، سرنگون خواهم ساخت. و این دیگرواقع نخواهد شد تا آنکس بیاید که حق اومی باشد. و من آن را به وی عطا خواهم نمود.
\par 28 «و تو‌ای پسر انسان نبوت کرده، بگو: خداوند یهوه درباره بنی عمون و سرزنش ایشان چنین می‌فرماید: بگو که شمشیر، شمشیر برای کشتار کشیده شده است و به غایت صیقلی گردیده تا براق بشود.
\par 29 چونکه برای تو رویای باطل دیده‌اند و برای تو تفال دروغ زده‌اند تا تو رابر گردنهای مقتولان شریر بگذارند که اجل ایشان در زمان عقوبت آخر رسیده است.
\par 30 لهذا آن رابه غلافش برگردان و بر تو در مکانی که آفریده شده‌ای و در زمینی که تولد یافته‌ای داوری خواهم نمود.
\par 31 و خشم خود را بر تو خواهم ریخت و آتش غیظ خود را بر تو خواهم دمید. وتو را به‌دست مردان وحشی که برای هلاک نمودن چالاکند تسلیم خواهم نمود.و تو برای آتش هیزم خواهی شد و خونت در آن زمین خواهدماند. پس به یاد آورده نخواهی شد زیرا من که یهوه هستم تکلم نموده‌ام.»
\par 32 و تو برای آتش هیزم خواهی شد و خونت در آن زمین خواهدماند. پس به یاد آورده نخواهی شد زیرا من که یهوه هستم تکلم نموده‌ام.»

\chapter{22}

\par 1 و کلام خداوند بر من نازل شده، گفت:
\par 2 «ای پسر انسان آیا داوری خواهی نمود؟ آیا بر شهر خونریز داوری خواهی نمود؟ پس آن را از همه رجاساتش آگاه ساز.
\par 3 و بگوخداوند یهوه چنین می‌فرماید: ای شهری که خون را در میان خودت می‌ریزی تا اجل توبرسد! ای که بتها را به ضد خود ساخته، خویشتن را نجس نموده‌ای!
\par 4 به‌سبب خونی که ریخته‌ای مجرم شده‌ای و به‌سبب بتهایی که ساخته‌ای نجس گردیده‌ای. لهذا اجل خویش را نزدیک آورده، به انتهای سالهای خود رسیده‌ای. لهذا تورا نزد امت‌ها عار و نزد جمیع کشورها مسخره گردانیده‌ام.
\par 5 ‌ای پلید نام! و‌ای پر فتنه! آنانی که به تو نزدیک و آنانی که از تو دورند بر تو سخریه خواهند نمود.
\par 6 اینک سروران اسرائیل، هر کس به قدر قوت خویش مرتکب خونریزی در میان تومی بودند.
\par 7 پدر و مادر را در میان تو اهانت نمودند. و غریبان را در میان تو مظلوم ساختند وبر یتیمان و بیوه‌زنان در میان تو ستم نمودند.
\par 8 وتو مقدس های مرا خوار شمرده، سبت های مرابی عصمت نمودی.
\par 9 و بعضی در میان تو به جهت ریختن خون، نمامی می‌نمودند. و بر کوهها درمیان تو غذا می‌خوردند. و در میان تو مرتکب قباحت می‌شدند.
\par 10 و عورت پدران را در میان تو منکشف می‌ساختند. و زنان حایض را در میان توبی عصمت می‌نمودند.
\par 11 یکی در میان تو با زن همسایه خود عمل زشت نمود. و دیگری عروس خویش را به جور بی‌عصمت کرد. و دیگری خواهرش، یعنی دختر پدر خود را ذلیل ساخت.
\par 12 و در میان تو به جهت ریختن خون رشوه خوردند و سود و ربح گرفتند. و تو مال همسایه خود را به زور غصب کردی و مرا فراموش نمودی. قول خداوند یهوه این است.
\par 13 لهذا هان من به‌سبب حرص تو که مرتکب آن شده‌ای و به‌سبب خونی که در میان خودت ریخته‌ای، دستهای خود را بهم می‌زنم.
\par 14 پس در ایامی که من به تو مکافات رسانم آیا دلت قوی و دستهایت محکم خواهد بود؟ من که یهوه هستم تکلم نمودم و بعمل خواهم آورد.
\par 15 و تو را در میان امت هاپراکنده و در میان کشورها متفرق ساخته، نجاسات تو را از میانت نابود خواهم ساخت.
\par 16 وبه نظر امت‌ها بی‌عصمت خواهی شد و خواهی دانست که من یهوه هستم.»
\par 17 و کلام خداوند بر من نازل شده، گفت:
\par 18 «ای پسر انسان خاندان اسرائیل نزد من دردشده‌اند و جمیع ایشان مس و روی و آهن و سرب در میان کوره و درد نقره شده‌اند.
\par 19 بنابراین خداوند یهوه چنین می‌گوید: چونکه همگی شمادرد شده‌اید، لهذا من شما را در میان اورشلیم جمع خواهم نمود.
\par 20 چنانکه نقره و مس و آهن و سرب و روی را در میان کوره جمع کرده، آتش بر آنها می‌دمند تا گداخته شود، همچنان من شمارا در غضب و حدت خشم خویش جمع کرده، در آن خواهم نهاد و شما را خواهم گداخت.
\par 21 وشما را جمع کرده، آتش غضب خود را بر شماخواهم دمید که در میانش گداخته شوید.
\par 22 چنانکه نقره در میان کوره گداخته می‌شود، همچنان شما در میانش گداخته خواهید شد وخواهید دانست که من یهوه حدت خشم خویش را بر شما ریخته‌ام.»
\par 23 و کلام خداوند بر من نازل شده، گفت:
\par 24 «ای پسر انسان او را بگو: تو زمینی هستی که طاهر نخواهی شد. و باران در روز غضب بر تونخواهد بارید.
\par 25 فتنه انبیای آن در میانش می‌باشد. ایشان مثل شیر غران که شکار رامی درد، جانها را می‌خورند. و گنجها و نفایس رامی برند. و بیوه‌زنان را در میانش زیاد می‌سازند.
\par 26 کاهنانش به شریعت من مخالفت ورزیده، موقوفات مرا حلال می‌سازند. و در میان مقدس وغیر مقدس تمیز نمی دهند و در میان نجس وطاهر فرق نمی گذارند. و چشمان خود را ازسبت های من می‌پوشانند و من در میان ایشان بی‌حرمت گردیده‌ام.
\par 27 سرورانش مانند گرگان درنده خون می‌ریزند و جانها را هلاک می‌نمایندتا سود ناحق ببرند.
\par 28 و انبیایش ایشان را به گل ملاط اندود نموده، رویاهای باطل می‌بینند وبرای ایشان تفال دروغ زده، می‌گویند که خداوندیهوه چنین گفته است با آنکه یهوه تکلم ننموده.
\par 29 و قوم زمین به شدت ظلم نموده و مال یکدیگررا غصب کرده‌اند. و بر فقیران و مسکینان جفانموده، غریبان را به بی‌انصافی مظلوم ساخته‌اند.
\par 30 و من در میان ایشان کسی را طلبیدم که دیوار رابنا نماید و برای زمین به حضور من در شکاف بایستد تا آن را خراب ننمایم، اما کسی را نیافتم.پس خداوند یهوه می‌گوید: خشم خود را برایشان ریخته‌ام و ایشان را به آتش غضب خویش هلاک ساخته، طریق ایشان را بر سر ایشان واردآورده‌ام.»
\par 31 پس خداوند یهوه می‌گوید: خشم خود را برایشان ریخته‌ام و ایشان را به آتش غضب خویش هلاک ساخته، طریق ایشان را بر سر ایشان واردآورده‌ام.»

\chapter{23}

\par 1 و کلام خداوند بر من نازل شده، گفت:
\par 2 «ای پسر انسان دو زن دختر یک مادربودند.
\par 3 و ایشان در مصر زنا کرده، در جوانی خودزناکار شدند. در آنجا سینه های ایشان را مالیدند وپستانهای بکارت ایشان را افشردند.
\par 4 و نامهای ایشان بزرگتر اهوله و خواهر او اهولیبه بود. وایشان از آن من بوده، پسران و دختران زاییدند. واما نامهای ایشان اهوله، سامره می‌باشد و اهولیبه، اورشلیم.
\par 5 و اهوله از من رو تافته، زنا نمود و برمحبان خود یعنی بر آشوریان که مجاور او بودندعاشق گردید؛
\par 6 کسانی که به آسمانجونی ملبس بودند؛ حاکمان و سرداران که همه ایشان جوانان دلپسند و فارسان اسب‌سوار بودند.
\par 7 و به ایشان یعنی به جمیع برگزیدگان بنی آشور فاحشگی خود را بذل نمود و خود را از جمیع بتهای آنانی که بر ایشان عاشق می‌بود نجس می‌ساخت.
\par 8 وفاحشگی خود را که در مصر می‌نمود، ترک نکرد. زیرا که ایشان در ایام جوانیش با او همخواب می‌شدند و پستانهای بکارت او را افشرده، زناکاری خود را بر وی می‌ریختند.
\par 9 لهذا من او رابه‌دست عاشقانش یعنی به‌دست بنی آشور که اوبر ایشان عشق می‌ورزید، تسلیم نمودم.
\par 10 که ایشان عورت او را منکشف ساخته، پسران ودخترانش را گرفتند. و او را به شمشیر کشتند که در میان زنان عبرت گردید و بر وی داوری نمودند.
\par 11 «و چون خواهرش اهولیبه این را دید، درعشقبازی خویش از او زیادتر فاسد گردید وبیشتر از زناکاری خواهرش زنا نمود.
\par 12 و بر بنی آشور عاشق گردید که جمیع ایشان حاکمان وسرداران مجاور او بودند و ملبس به آسمانجونی و فارسان اسب‌سوار و جوانان دلپسند بودند.
\par 13 ودیدم که او نیز نجس گردیده و طریق هردوی ایشان یک بوده است.
\par 14 پس زناکاری خود رازیاد نمود، زیرا صورتهای مردان که بر دیوارهانقش شده بود یعنی تصویرهای کلدانیان را که به شنجرف کشیده شده بود، دید.
\par 15 که کمرهای ایشان به کمربندها بسته و عمامهای رنگارنگ برسر ایشان پیچیده بود. و جمیع آنها مانند سرداران و به شبیه اهل بابل که مولد ایشان زمین کلدانیان است بودند.
\par 16 و چون چشم او بر آنها افتاد، عاشق ایشان گردید. و رسولان نزد ایشان به زمین کلدانیان فرستاد.
\par 17 و پسران بابل نزد وی در بسترعشق بازی درآمده، او را از زناکاری خود نجس ساختند. پس چون خود را از ایشان نجس یافت، طبع وی از ایشان متنفر گردید.
\par 18 و چون که زناکاری خود را آشکار کرد و عورت خود رامنکشف ساخت، جان من از او متنفر گردید، چنانکه جانم از خواهرش متنفر شده بود.
\par 19 امااو ایام جوانی خود را که در آنها در زمین مصر زناکرده بود به یاد آورده، باز زناکاری خود را زیادنمود.
\par 20 و بر معشوقه های ایشان عشق ورزید که گوشت ایشان، مثل گوشت الاغان و نطفه ایشان چون نطفه اسبان می‌باشد.
\par 21 و قباحت جوانی خود را حینی که مصریان پستانهایت را به‌خاطرسینه های جوانیت افشردند به یاد آوردی.
\par 22 «بنابراین‌ای اهولیبه خداوند یهوه چنین می‌فرماید: اینک من عاشقانت را که جانت ازایشان متنفر شده است به ضد تو برانگیزانیده، ایشان را از هر طرف برتو خواهم آورد.
\par 23 یعنی پسران بابل و همه کلدانیان را از فقود و شوع وقوع. و همه پسران آشور را همراه ایشان که جمیع ایشان جوانان دلپسند و حاکمان و والیان و سرداران و نامداران هستند و تمامی ایشان اسب‌سوارند.
\par 24 و با اسلحه و کالسکه‌ها و ارابه‌ها وگروه عظیمی بر تو خواهند آمد و با مجن‌ها وسپرها و خودها تو را احاطه خواهند نمود. و من داوری تو را به ایشان خواهم سپرد تا تو را برحسب احکام خود داوری نمایند.
\par 25 و من غیرت خود را به ضد تو خواهم برانگیخت تا با تو به غضب عمل نمایند. و بینی و گوشهایت راخواهند برید و بقیه تو با شمشیر خواهند افتاد وپسران و دخترانت را خواهند گرفت و بقیه تو به آتش سوخته خواهند شد.
\par 26 و لباس تو را از توکنده، زیورهای زیبایی تو را خواهند برد.
\par 27 پس قباحت تو و زناکاریت را که از زمین مصرآورده‌ای، از تو نابود خواهم ساخت. و چشمان خود را بسوی ایشان بر نخواهی افراشت و دیگرمصر را به یاد نخواهی آورد.
\par 28 زیرا خداوندیهوه چنین می‌گوید: اینک تو را به‌دست آنانی که از ایشان نفرت داری و به‌دست آنانی که جانت ازایشان متنفر است، تسلیم خواهم نمود.
\par 29 و با تواز راه بغض رفتار نموده، تمامی حاصل تو راخواهند گرفت و تو را عریان و برهنه وا خواهم گذاشت. تا آنکه برهنگی زناکاری تو و قباحت وفاحشه گری تو ظاهر شود.
\par 30 و این کارها را به توخواهم کرد، از این جهت که در عقب امت‌ها زنانموده، خویشتن را از بتهای ایشان نجس ساخته‌ای.
\par 31 و چونکه به طریق خواهر خودسلوک نمودی، جام او را به‌دست تو خواهم داد.
\par 32 و خداوند یهوه چنین می‌فرماید: جام عمیق وبزرگ خواهر خود را خواهی نوشید. و محل سخریه و استهزا خواهی شد که متحمل آن نتوانی شد.
\par 33 و از مستی و حزن پر خواهی شد. از جام حیرت و خرابی یعنی از جام خواهرت سامره.
\par 34 و آن را خواهی نوشید و تا ته خواهی آشامید و خورده های آن را خواهی خایید وپستانهای خود را خواهی کند، زیرا خداوند یهوه می‌گوید که من این را گفته‌ام.
\par 35 بنابراین خداوندیهوه چنین می‌فرماید: چونکه مرا فراموش کردی و مرا پشت سر خود انداختی، لهذا تو نیز متحمل قباحت و زناکاری خود خواهی شد.»
\par 36 و خداوند مرا گفت: «ای پسر انسان! آیابراهوله و اهولیبه داوری خواهی نمود؟ بلکه ایشان را از رجاسات ایشان آگاه ساز.
\par 37 زیرا که زنا نموده‌اند و دست ایشان خون آلود است و بابتهای خویش مرتکب زنا شده‌اند. و پسران خودرا نیز که برای من زاییده بودند، به جهت آنها ازآتش گذرانیده‌اند تا سوخته شوند.
\par 38 و علاوه برآن این را هم به من کرده‌اند که در همانروز مقدس مرا بی‌عصمت کرده، سبت های مرا بی‌حرمت نموده‌اند.
\par 39 زیرا چون پسران خود را برای بتهای خویش ذبح نموده بودند، در همان روز به مقدس داخل شده، آن را بی‌عصمت نمودند و هان این عمل را در خانه من بجا آوردند.
\par 40 بلکه نزدمردانی که از دور آمدند، فرستادید که نزد ایشان قاصدی فرستاده شد. و چون ایشان رسیدند، خویشتن را برای ایشان غسل دادی و سرمه به چشمانت کشیدی و خود را به زیورهایت آرایش دادی.
\par 41 و بر بستر فاخری که سفره پیش آن آماده بود نشسته، بخور و روغن مرا بر آن نهادی.
\par 42 و در آن آواز گروه عیاشان مسموع شد. وهمراه آن گروه عظیم صابیان از بیابان آورده شدندکه دستبندها بر دستها و تاجهای فاخر بر سر هردوی آنها گذاشتند.
\par 43 و من درباره آن زنی که در زناکاری فرسوده شده بود گفتم: آیا ایشان الان بااو زنا خواهند کرد و او با ایشان؟
\par 44 و به اودرآمدند به نهجی که نزد فاحشه‌ها درمی آیند. همچنان به آن دو زن قباحت پیشه یعنی اهوله واهولیبه درآمدند.
\par 45 پس مردان عادل بر ایشان قصاص زنان زناکار و خونریز را خواهند رسانید، زیرا که ایشان زانیه می‌باشند و دست ایشان خون آلود است.
\par 46 زیرا خداوند یهوه چنین می‌فرماید: من گروهی به ضد ایشان خواهم برانگیخت. و ایشان را مشوش ساخته، به تاراج تسلیم خواهم نمود.
\par 47 و آن گروه ایشان را به سنگها سنگسار نموده، به شمشیرهای خود پاره خواهند کرد. و پسران و دختران ایشان را کشته، خانه های ایشان را به آتش خواهند سوزانید.
\par 48 وقباحت را از زمین نابود خواهم ساخت. پس جمیع زنان متنبه خواهند شد که مثل شما مرتکب قباحت نشوند.و سزای قباحت شما را بر شماخواهند رسانید. و متحمل گناهان بتهای خویش خواهید شد و خواهید دانست که من خداوندیهوه می‌باشم.»
\par 49 و سزای قباحت شما را بر شماخواهند رسانید. و متحمل گناهان بتهای خویش خواهید شد و خواهید دانست که من خداوندیهوه می‌باشم.»

\chapter{24}

\par 1 و در روز دهم ماه دهم از سال نهم کلام خداوند بر من نازل شده، گفت:
\par 2 «ای پسر انسان اسم امروز را برای خود بنویس، اسم همین روز را، زیرا که در همین روز پادشاه بابل براورشلیم هجوم آورد.
\par 3 و برای این خاندان فتنه انگیز مثلی آورده، به ایشان بگو: خداوند یهوه چنین می‌گوید: دیگ را بگذار. آن را بگذار و آب نیز در آن بریز.
\par 4 قطعه هایش یعنی هر قطعه نیکو و ران و دوش را در میانش جمع کن و از بهترین استخوانها آن را پر ساز.
\par 5 و بهترین گوسفندان رابگیر و استخوانها را زیرش دسته کرده، آن راخوب بجوشان تا استخوانهایی که در اندرونش هست پخته شود.
\par 6 بنابراین خداوند یهوه چنین می‌گوید: وای برآن شهر خونریز! وای بر آن دیگی که زنگش در میانش است و زنگش ازمیانش در نیامده است! آن را به قطعه هایش بیرون آور و قرعه بر آن انداخته نشود.
\par 7 زیرا خونی که ریخت در میانش می‌باشد. آن را بر صخره صاف نهاد و بر زمین نریخت تا از خاک پوشانیده شود.
\par 8 من خون او را بر صخره صاف نهادم که پنهان نشود تا آنکه حدت خشم را برانگیخته انتقام بکشم.
\par 9 بنابراین خداوند یهوه چنین می‌گوید: وای بر آن شهر خونریز! من نیز توده هیزم را بزرگ خواهم ساخت.
\par 10 هیزم زیاد بیاور و آتش بیفروزو گوشت را مهیا ساز و ادویه جات در آن بریز واستخوانها سوخته بشود.
\par 11 پس آن را بر اخگرخالی بگذار تا تابیده شده مسش سوخته گردد ونجاستش در آن گداخته شود و زنگش نابود گردد.
\par 12 او از مشقت‌ها خسته گردید، اما زنگ بسیارش از وی بیرون نیامد. پس زنگش در آتش بشود.
\par 13 در نجاسات تو قباحت است چونکه تو راتطهیر نمودم. اما طاهر نشدی. لهذا تا غضب خودرا بر تو به اتمام نرسانم، دیگر از نجاست خود طاهر نخواهی شد.
\par 14 من که یهوه هستم این را گفته‌ام و به وقوع خواهد پیوست وآن را بجا خواهم آورد. پس خداوند یهوه می‌گوید: دست نخواهم برداشت و شفقت نخواهم نمود و پشیمان نخواهم شد و بر‌حسب رفتارت و بر وفق اعمالت بر تو داوری خواهندکرد.»
\par 15 و کلام خداوند بر من نازل شده، گفت:
\par 16 «ای پسر انسان اینک من آرزوی چشمانت رابغته از تو خواهم گرفت. ماتم و گریه منما و اشک از چشمت جاری نشود.
\par 17 آه بکش و خاموش شو و برای مرده ماتم مگیر. بلکه عمامه بر سرت بپیچ و کفش به پایت بکن و شاربهایت را مپوشان وطعام مرده را مخور.»
\par 18 پس بامدادان با قوم تکلم نمودم و وقت عصر زن من مرد و صبحگاهان به نهجی که مامورشده بودم عمل نمودم.
\par 19 و قوم به من گفتند: «آیاما را خبر نمی دهی که این کارهایی که می‌کنی به ما چه نسبت دارد؟»
\par 20 ایشان را جواب دادم که کلام خداوند بر من نازل شده، گفت:
\par 21 «به خاندان اسرائیل بگو: خداوند یهوه چنین می‌فرماید که هان من مقدس خود را که فخر جلال شما و آرزوی چشمان شماو لذت جانهای شما است، بی‌عصمت خواهم نمود. و پسران و دختران شما که ایشان را ترک خواهید کرد، به شمشیر خواهند افتاد.
\par 22 و به نهجی که من عمل نمودم، شما عمل خواهیدنمود. شاربهای خود را نخواهید پوشانید و طعام مردگان را نخواهید خورد.
\par 23 عمامه های شما برسر و کفشهای شما در پایهای شما بوده، ماتم وگریه نخواهید کرد. بلکه به‌سبب گناهان خودکاهیده شده، بسوی یکدیگر آه خواهید کشید.
\par 24 و حزقیال برای شما آیتی خواهد بود موافق هر‌آنچه او کرد، شما عمل خواهید نمود. و حینی که این واقع شود خواهید دانست که من خداوندیهوه می‌باشم.
\par 25 «و اما تو‌ای پسر انسان! در روزی که من قوت و سرور فخر و آرزوی چشمان و رفعت جانهای ایشان یعنی پسران و دختران ایشان را ازایشان گرفته باشم، آیا واقع نخواهد شد
\par 26 که درآن روز هر‌که رهایی یابد نزد تو آمده، این را به سمع تو خواهد رسانید؟پس در آن روزدهانت برای آنانی که رهایی یافته‌اند باز خواهدشد و متکلم شده، دیگر گنگ نخواهی بود و برای ایشان آیتی خواهی بود و خواهند دانست که من یهوه می‌باشم.»
\par 27 پس در آن روزدهانت برای آنانی که رهایی یافته‌اند باز خواهدشد و متکلم شده، دیگر گنگ نخواهی بود و برای ایشان آیتی خواهی بود و خواهند دانست که من یهوه می‌باشم.»

\chapter{25}

\par 1 و کلام خداوند بر من نازل شده، گفت:
\par 2 «ای پسر انسان نظر خود را بربنی عمون بدار و به ضد ایشان نبوت نما.
\par 3 و به بنی عمون بگو: کلام خداوند یهوه را بشنوید! خداوند یهوه چنین می‌فرماید: چونکه درباره مقدس من حینی که بی‌عصمت شد و درباره زمین اسرائیل، حینی که ویران گردید و درباره خاندان یهودا، حینی که به اسیری رفتند هه گفتی،
\par 4 بنابراین همانا من تو را به بنی مشرق تسلیم می‌کنم تا در تو تصرف نمایند. و خیمه های خودرا در میان تو زده، مسکن های خویش را در تو بر پاخواهند نمود. و ایشان میوه تو را خواهند خوردو شیر تو را خواهند نوشید. و ربه را آرامگاه شتران و (زمین ) بنی عمون را خوابگاه گله هاخواهم گردانید و خواهید دانست که من یهوه هستم.
\par 5 زیرا خداوند یهوه چنین می‌گوید: چونکه تو بر زمین اسرائیل دستک می‌زنی و پا برزمین می‌کوبی و به تمامی کینه دل خود شادی می‌نمایی،
\par 6 بنابراین هان من دست خود را بر تو دراز خواهم کرد و تو را تاراج امت‌ها خواهم ساخت. و تو را از میان قوم‌ها منقطع ساخته، ازمیان کشورها نابود خواهم ساخت. و چون تو راهلاک ساخته باشم، آنگاه خواهی دانست که من یهوه هستم.»
\par 7 خداوند یهوه چنین می‌گوید: «چونکه موآب و سعیر گفته‌اند که اینک خاندان اسرائیل مانند جمیع امت‌ها می‌باشند،
\par 8 بنابراین اینک من حدود موآب را از شهرها یعنی از شهرهای حدودش که فخر زمین می‌باشد یعنی بیت یشیموت و بعل معون و قریه تایم مفتوح خواهم ساخت.
\par 9 برای بنی مشرق آن را با بنی عمون (مفتوح خواهم ساخت ) و به تصرف ایشان خواهم داد تا بنی عمون دیگر در میان امت هامذکور نشوند.
\par 10 و بر موآب داوری خواهم نمودو خواهند دانست که من یهوه می‌باشم.»
\par 11 خداوند یهوه چنین می‌گوید: «از این جهت که ادوم از خاندان یهودا انتقام کشیده‌اند و درانتقام کشیدن از ایشان خطایی عظیم ورزیده‌اند،
\par 12 بنابراین خداوند یهوه چنین می‌فرماید: دست خود را برادوم دراز کرده، انسان و بهایم را از آن منقطع خواهم ساخت و آن را ویران کرده، ازتیمان تا ددان به شمشیر خواهند افتاد.
\par 13 و به‌دست قوم خود اسرائیل انتقام خود را از ادوم خواهم کشید و موافق خشم و غضب من به ادوم عمل خواهند نمود. و خداوند یهوه می‌گوید که انتقام مرا خواهند فهمید.»
\par 14 خداوند یهوه چنین می‌گوید: «چونکه فلسطینیان انتقام کشیدند و با کینه دل خود انتقام سخت کشیدند تا آن را به عداوت ابدی خراب نمایند،
\par 15 بنابراین خداوند یهوه چنین می‌فرماید: اینک من دست خود را بر فلسطینیان دراز نموده، کریتیان را منقطع خواهم ساخت و باقی ماندگان ساحل دریا را هلاک خواهم نمود.و باسرزنش غضب آمیز انتقام سخت از ایشان خواهم گرفت. پس چون انتقام خود را از ایشان کشیده باشم، آنگاه خواهند دانست که من یهوه هستم.»
\par 16 و باسرزنش غضب آمیز انتقام سخت از ایشان خواهم گرفت. پس چون انتقام خود را از ایشان کشیده باشم، آنگاه خواهند دانست که من یهوه هستم.»

\chapter{26}

\par 1 و در سال یازدهم در غره ماه واقع شدکه کلام خداوند بر من نازل شده، گفت:
\par 2 «ای پسر انسان چونکه صور درباره اورشلیم می‌گوید هه، دروازه امت‌ها شکسته شد و حال به من منتقل گردیده است. و چون او خراب گردیدمن توانگر خواهم شد.
\par 3 بنابراین خداوند یهوه چنین می‌گوید: هان‌ای صور من به ضد تومی باشم و امت های عظیم بر تو خواهم برانگیخت به نهجی که دریا امواج خود رابرمی انگیزاند.
\par 4 و حصار صور را خراب کرده، برجهایش را منهدم خواهند ساخت و غبارش رااز آن خواهم رفت و آن را به صخره‌ای صاف تبدیل خواهم نمود.
\par 5 و او محل پهن کردن دامها در میان دریا خواهد شد، زیرا خداوند یهوه می‌فرماید که من این را گفته‌ام. و آن تاراج امت هاخواهد گردید.
\par 6 و دخترانش که در صحرامی باشند به شمشیر کشته خواهند شد. پس ایشان خواهند دانست که من یهوه هستم.»
\par 7 زیرا خداوند یهوه چنین می‌فرماید: «اینک من نبوکدرصر پادشاه بابل، پادشاه پادشاهان را ازطرف شمال بر صور با اسبان و ارابه‌ها و سواران وجمعیت و خلق عظیمی خواهم آورد.
\par 8 و اودختران تو را در صحرا به شمشیر خواهد کشت. و برجها به ضد تو بنا خواهد نمود. و سنگرها دربرابر تو خواهد ساخت. و مترسها در برابر تو برپاخواهد داشت.
\par 9 و منجنیقهای خود را برحصارهایت آورده، برجهایت را با تبرهای خودمنهدم خواهد ساخت.
\par 10 و اسبانش آنقدر زیادخواهد بود که گرد آنها تو را خواهد پوشانید. وچون به دروازه هایت داخل شود چنانکه به شهررخنه دار درمی آیند، حصارهایت از صدای سواران و ارابه‌ها و کالسکه‌ها متزلزل خواهدگردید.
\par 11 و به سم اسبان خود همه کوچه هایت راپایمال کرده، اهل تو را به شمشیر خواهد کشت. و بناهای فخر تو به زمین خواهد افتاد.
\par 12 وتوانگری تو را تاراج نموده، تجارت تو را به یغماخواهند برد. و حصارهایت را خراب نموده، خانه های مرغوب تو را منهدم خواهند نمود. وسنگها و چوب و خاک تو را در آب خواهندریخت.
\par 13 و آواز نغمات تو را ساکت خواهم گردانید که صدای عودهایت دیگر مسموع نشود.
\par 14 و تو را به صخره‌ای صاف مبدل خواهم گردانید تا محل پهن کردن دامها بشوی و بار دیگربنا نخواهی شد. زیرا خداوند یهوه می‌فرماید: من که یهوه هستم این را گفته‌ام.»
\par 15 خداوند یهوه به صور چنین می‌گوید: «آیاجزیره‌ها از صدای انهدام تو متزلزل نخواهد شدهنگامی که مجروحان ناله کشند و در میان توکشتار عظیمی بشود؟
\par 16 و جمیع سروران دریا ازکرسیهای خود فرود آیند. و رداهای خود را ازخود بیرون کرده، رخوت قلابدوزی خویش رابکنند. و به ترسها ملبس شده، بر زمین بنشینند وآن فان لرزان گردیده، درباره تو متحیر شوند.
\par 17 پس برای تو مرثیه خوانده، تو را خواهند گفت: ای که از دریا معمور بودی چگونه تباه گشتی! آن شهر نامداری که در دریا زورآور می‌بود که باساکنان خود هیبت خویش را بر جمیع سکنه دریامستولی می‌ساخت.
\par 18 الان در روز انهدام توجزیره‌ها می‌لرزند. و جزایری که در دریامی باشد، از رحلت تو مدهوش می‌شوند.
\par 19 زیراخداوند یهوه چنین می‌گوید: چون تو را شهرمخروب مثل شهرهای غیرمسکون گردانم ولجه‌ها را بر تو برآورده، تو را به آبهای بسیارمستور سازم،
\par 20 آنگاه تو را با آنانی که به هاویه فرو می‌روند، نزد قوم قدیم فرود آورده، تو را دراسفلهای زمین در خرابه های ابدی با آنانی که به هاویه فرو می‌روند ساکن خواهم گردانید تا دیگرمسکون نشوی و دیگر جلال تو را در زمین زندگان جای نخواهم داد.و خداوند یهوه می‌گوید: تو را محل وحشت خواهم ساخت که نابود خواهی شد و تو را خواهند طلبید اما تاابدالاباد یافت نخواهی شد.»
\par 21 و خداوند یهوه می‌گوید: تو را محل وحشت خواهم ساخت که نابود خواهی شد و تو را خواهند طلبید اما تاابدالاباد یافت نخواهی شد.»

\chapter{27}

\par 1 و کلام خداوند بر من نازل شده، گفت:
\par 2 «اما تو‌ای پسر انسان برای صور مرثیه بخوان!
\par 3 و به صور بگو: ای که نزد مدخل دریاساکنی و برای جزیره های بسیار تاجر طوایف می‌باشی! خداوند یهوه چنین می‌گوید: ای صورتو گفته‌ای که من کمال زیبایی هستم.
\par 4 حدود تودر وسط دریا است و بنایانت زیبایی تو را کامل ساخته‌اند.
\par 5 همه تخته هایت را از صنوبر سنیرساختند و سرو آزاد لبنان را گرفتند تا دکلها برای تو بسازند.
\par 6 پاروهایت را از بلوطهای باشان ساختند و نشیمنهایت را از شمشاد جزایر کتیم که به عاج منبط شده بود ترتیب دادند.
\par 7 کتان مطرز مصری بادبان تو بود تا برای تو علمی بشود. و شراع تو از آسمانجونی و ارغوان از جزایرالیشه بود.
\par 8 اهل سیدون و ارواد پاروزن تو بودندو حکمای تو‌ای صور که در تو بودند ناخدایان توبودند.
\par 9 مشایخ جبیل و حکمایش در تو بوده، قلافان تو بودند. تمامی کشتیهای دریا و ملاحان آنها در تو بودند تا برای تو تجارت نمایند.
\par 10 فارس و لود و فوط در افواجت مردان جنگی تو بودند. سپرها و خودها بر تو آویزان کرده، ایشان تو را زینت دادند.
\par 11 بنی ارواد با سپاهیانت بر حصارهایت از هر طرف و جمادیان بربرجهایت بودند. و سپرهای خود را برحصارهایت از هر طرف آویزان کرده، ایشان زیبایی تو را کامل ساختند.
\par 12 ترشیش به فراوانی هر قسم اموال سوداگران تو بودند. نقره و آهن وروی و سرب به عوض بضاعت تو می‌دادند.
\par 13 یاوان و توبال و ماشک سوداگران تو بودند.
\par 14 اهل خاندان توجرمه اسبان و سواران و قاطران به عوض بضاعت تو می‌دادند.
\par 15 بنی ددان سوداگران تو و جزایر بسیار بازارگانان دست تو بودند. شاخهای عاج و آبنوس را با تومعاوضت می‌کردند.
\par 16 ارام به فراوانی صنایع توسوداگران تو بودند. بهرمان و ارغوان و پارچه های قلابدوزی و کتان نازک و مرجان و لعل به عوض بضاعت تو می‌دادند.
\par 17 یهودا و زمین اسرائیل سوداگران تو بودند، گندم منیت و حلوا و عسل وروغن و بلسان به عوض متاع تو می‌دادند.
\par 18 دمشق به فراوانی صنایع تو و کثرت هر قسم اموال با شراب حلبون و پشم سفید با تو سودامی کردند.
\par 19 ودان و یاوان ریسمان به عوض بضاعت تو می‌دادند. آهن مصنوع و سلیخه وقصب الذریره از متاعهای تو بود.
\par 20 ددان با زین پوشهای نفیس به جهت سواری سوداگران توبودند.
\par 21 عرب و همه سروران قیدار بازارگانان دست تو بودند. با بره‌ها و قوچها و بزها با تو داد وستد می‌کردند.
\par 22 تجار شبا و رعمه سوداگران توبودند. بهترین همه ادویه جات و هرگونه سنگ گرانبها و طلا به عوض بضاعت تو می‌دادند.
\par 23 حران و کنه و عدن و تجار شبا و آشور و کلمدسوداگران تو بودند.
\par 24 اینان با نفایس و رداهای آسمانجونی و قلابدوزی و صندوقهای پر ازرختهای فاخر ساخته شده از چوب سرو آزاد وبسته شده با ریسمانها در بازارهای تو سوداگران تو بودند.
\par 25 کشتیهای ترشیش قافله های متاع توبودند. پس در وسط دریا توانگر و بسیار معززگردیدی.
\par 26 پاروزنانت تو را به آبهای عظیم بردند و باد شرقی تو را در میان دریا شکست.
\par 27 اموال و بضاعت و متاع و ملاحان و ناخدایان وقلافان و سوداگران و جمیع مردان جنگی که درتو بودند، با تمامی جمعیتی که در میان تو بودنددر روز انهدام تو در وسط دریا افتادند.
\par 28 «از آواز فریاد ناخدایانت ساحلها متزلزل گردید.
\par 29 و جمیع پاروزنان و ملاحان و همه ناخدایان دریا از کشتیهای خود فرود آمده، درزمین می‌ایستند.
\par 30 و برای تو آواز خود را بلندکرده، به تلخی ناله می‌کنند و خاک بر سر خودریخته، در خاکستر می‌غلطند.
\par 31 و برای تو موی خود را کنده، پلاس می‌پوشند و با مرارت جان ونوحه تلخ برای تو گریه می‌کنند.
\par 32 و در نوحه خود برای تو مرثیه می‌خوانند. و بر تو نوحه گری نموده، می‌گویند: کیست مثل صور و کیست مثل آن شهری که در میان دریا خاموش شده است؟
\par 33 هنگامی که بضاعت تو از دریا بیرون می‌آمد، قومهای بسیاری را سیر می‌گردانیدی و پادشاهان جهان را به فراوانی اموال و متاع خود توانگرمی ساختی.
\par 34 اما چون از دریا، در عمق های آبهاشکسته شدی، متاع و تمامی جمعیت تو درمیانت تلف شد.
\par 35 جمیع ساکنان جزایر به‌سبب تو متحیر گشته و پادشاهان ایشان به شدت دهشت زده و پریشان حال گردیده‌اند.تجارقوم‌ها بر تو صفیر می‌زنند و تو محل دهشت گردیده، دیگر تا به ابد نخواهی بود.»
\par 36 تجارقوم‌ها بر تو صفیر می‌زنند و تو محل دهشت گردیده، دیگر تا به ابد نخواهی بود.»

\chapter{28}

\par 1 و کلام خداوند بر من نازل شده، گفت:
\par 2 «ای پسر انسان به رئیس صور بگو: خداوند یهوه چنین می‌فرماید: چونکه دلت مغرور شده است و می‌گویی که من خدا هستم وبر کرسی خدایان در وسط دریا نشسته‌ام، و هرچند انسان هستی و نه خدا لیکن دل خود را ماننددل خدایان ساخته‌ای.
\par 3 اینک تو از دانیال حکیم تر هستی و هیچ سری از تو مخفی نیست؟
\par 4 و به حکمت و فطانت خویش توانگری برای خود اندوخته و طلا و نقره در خزاین خود جمع نموده‌ای.
\par 5 به فراوانی حکمت و تجارت خویش دولت خود را افزوده‌ای پس به‌سبب توانگریت دلت مغرور گردیده است.»
\par 6 بنابراین خداوندیهوه چنین می‌فرماید: «چونکه تو دل خود را مثل دل خدایان گردانیده‌ای،
\par 7 پس اینک من غریبان وستم کیشان امت‌ها را بر تو خواهم آورد که شمشیرهای خود را به ضد زیبایی حکمت توکشیده، جمال تو را ملوث سازند.
\par 8 و تو را به هاویه فرود آورند. پس به مرگ آنانی که در میان دریا کشته شوند خواهی مرد.
\par 9 آیا به حضورقاتلان خود خواهی گفت که من خدا هستم؟ نی بلکه در دست قاتلانت انسان خواهی بود و نه خدا.
\par 10 از دست غریبان به مرگ نامختونان کشته خواهی شد، زیرا خداوند یهوه می‌فرماید که من این را گفته‌ام.»
\par 11 و کلام خداوند بر من نازل شده، گفت:
\par 12 «ای پسر انسان برای پادشاه صور مرثیه بخوان و وی را بگو خداوند یهوه چنین می‌فرماید: توخاتم کمال و مملو حکمت و کامل جمال هستی.
\par 13 در عدن در باغ خدا بودی و هر گونه سنگ گرانبها از عقیق احمر و یاقوت اصفر و عقیق سفید و زبرجد و جزع و یشب و یاقوت کبود و بهرمان و زمرد پوشش تو بود. و صنعت دفها ونایهایت در تو از طلا بود که در روز خلقت تو آنهامهیا شده بود.
\par 14 تو کروبی مسح شده سایه‌گستربودی. و تو را نصب نمودم تا بر کوه مقدس خدابوده باشی. و در میان سنگهای آتشین می‌خرامیدی.
\par 15 از روزی که آفریده شدی تاوقتی که بی‌انصافی در تو یافت شد به رفتار خودکامل بودی.
\par 16 اما از کثرت سوداگریت بطن تو رااز ظلم پر ساختند. پس خطا ورزیدی و من تو رااز کوه خدا بیرون انداختم. و تو را‌ای کروبی سایه‌گستر، از میان سنگهای آتشین تلف نمودم.
\par 17 دل تو از زیباییت مغرور گردید و به‌سبب جمالت حکمت خود را فاسد گردانیدی. لهذا تورا بر زمین می‌اندازم و تو را پیش روی پادشاهان می‌گذارم تا بر تو بنگرند.
\par 18 به کثرت گناهت وبی انصافی تجارتت مقدس های خویش رابی عصمت ساختی. پس آتشی از میانت بیرون می‌آورم که تو را بسوزاند و تو را به نظر جمیع بینندگانت بر روی زمین خاکستر خواهم ساخت.
\par 19 و همه آشنایانت از میان قوم‌ها بر تو متحیرخواهند شد. و تو محل دهشت شده، دیگر تا به ابد نخواهی بود.»
\par 20 و کلام خداوند بر من نازل شده، گفت:
\par 21 «ای پسر انسان نظر خود را بر صیدون بدار و به ضدش نبوت نما.
\par 22 و بگو خداوند یهوه چنین می‌فرماید: اینک‌ای صیدون من به ضد تو هستم وخویشتن را در میان تو تمجید خواهم نمود. وحینی که بر او داوری کرده و خویشتن را در وی تقدیس نموده باشم، آنگاه خواهند دانست که من یهوه هستم.
\par 23 و وبا در او و خون در کوچه هایش خواهم فرستاد. و مجروحان به شمشیری که ازهر طرف بر او می‌آید در میانش خواهند افتاد. پس خواهند دانست که من یهوه هستم.
\par 24 و باردیگر برای خاندان اسرائیل از جمیع مجاوران ایشان که ایشان را خوار می‌شمارند، خاری خلنده و شوک رنج آورنده نخواهند بود. پس خواهند دانست که من خداوند یهوه می‌باشم.»
\par 25 خداوند یهوه چنین می‌گوید: «هنگامی که خاندان اسرائیل را از قوم هایی که در میان ایشان پراکنده شده‌اند جمع نموده، خویشتن را از ایشان به نظر امت‌ها تقدیس کرده باشم، آنگاه در زمین خودشان که به بنده خود یعقوب داده‌ام ساکن خواهند شد.و در آن به امنیت ساکن شده، خانه‌ها بنا خواهند نمود و تاکستانها غرس خواهند ساخت. و چون بر جمیع مجاوران ایشان که ایشان را حقیر می‌شمارند داوری کرده باشم، آنگاه به امنیت ساکن شده، خواهند دانست که من یهوه خدای ایشان می‌باشم.»
\par 26 و در آن به امنیت ساکن شده، خانه‌ها بنا خواهند نمود و تاکستانها غرس خواهند ساخت. و چون بر جمیع مجاوران ایشان که ایشان را حقیر می‌شمارند داوری کرده باشم، آنگاه به امنیت ساکن شده، خواهند دانست که من یهوه خدای ایشان می‌باشم.»

\chapter{29}

\par 1 و در روز دوازدهم ماه دهم از سال دهم کلام خداوند بر من نازل شده، گفت:
\par 2 «ای پسر انسان نظر خود را به طرف فرعون پادشاه مصر بدار و به ضد او و تمامی مصر نبوت نما.
\par 3 و متکلم شده، بگو: خداوند یهوه چنین می‌فرماید: اینک‌ای فرعون پادشاه مصر من به ضد تو هستم. ای اژدهای بزرگ که در میان نهرهایت خوابیده‌ای و می‌گویی نهر من از آن من است و من آن را به جهت خود ساخته‌ام!
\par 4 لهذاقلابها در چانه ات می‌گذارم و ماهیان نهرهایت رابه فلسهایت خواهم چسپانید و تو را از میان نهرهایت بیرون خواهم کشید و تمامی ماهیان نهرهایت به فلسهای تو خواهند چسپید.
\par 5 و تو رابا تمامی ماهیان نهرهایت در بیابان پراکنده خواهم ساخت و به روی صحرا افتاده، بار دیگرتو را جمع نخواهند کرد و فراهم نخواهند‌آورد. و تو را خوراک حیوانات زمین و مرغان هواخواهم ساخت.
\par 6 و جمیع ساکنان مصر خواهنددانست که من یهوه هستم چونکه ایشان برای خاندان اسرائیل عصای نئین بودند.
\par 7 چون دست تو را گرفتند، خرد شدی. و کتفهای جمیع ایشان را چاک زدی. و چون بر تو تکیه نمودند، شکسته شدی. و کمرهای جمیع ایشان را لرزان گردانیدی.»
\par 8 بنابراین خداوند یهوه چنین می‌فرماید: «اینک من بر تو شمشیری آورده، انسان و بهایم رااز تو منقطع خواهم ساخت.
\par 9 و زمین مصر ویران و خراب خواهد شد. پس خواهند دانست که من یهوه هستم، چونکه می‌گفت: نهر از آن من است ومن آن را ساخته‌ام.
\par 10 بنابراین اینک من به ضد توو به ضد نهرهایت هستم و زمین مصر را از مجدل تا اسوان و تا حدود حبشستان بالکل خراب وویران خواهم ساخت.
\par 11 که پای انسان از آن عبور ننماید و پای حیوان از آن گذر نکند و مدت چهل سال مسکون نشود.
\par 12 و زمین مصر را درمیان زمینهای ویران ویران خواهم ساخت وشهرهایش در میان شهرهای مخروب مدت چهل سال خراب خواهد ماند. و مصریان را در میان امت‌ها پراکنده و در میان کشورها متفرق خواهم ساخت.»
\par 13 زیرا خداوند یهوه چنین می‌فرماید: «بعد ازانقضای چهل سال مصریان را از قوم هایی که درمیان آنها پراکنده شوند، جمع خواهم نمود.
\par 14 واسیران مصر را باز آورده، ایشان را به زمین فتروس یعنی به زمین مولد ایشان راجع خواهم گردانید و در آنجا مملکت پست خواهند بود.
\par 15 و آن پست‌ترین ممالک خواهد بود. و بار دیگربر طوایف برتری نخواهد نمود. و من ایشان راقلیل خواهم ساخت تا برامت‌ها حکمرانی ننمایند.
\par 16 و آن بار دیگر برای خاندان اسرائیل محل اعتماد نخواهد بود تا بسوی ایشان متوجه شده، گناه را به یاد آورند. پس خواهند دانست که من خداوند یهوه هستم.»
\par 17 و در روز اول ماه اول از سال بیست و هفتم کلام خداوند بر من نازل شده، گفت:
\par 18 «ای پسرانسان نبوکدرصر پادشاه بابل از لشکر خود به ضدصور خدمت عظیمی گرفت که سرهای همه بی‌مو گردید و دوشهای همه پوست کنده شد. لیکن از صور به جهت خدمتی که به ضد آن نموده بود، خودش و لشکرش هیچ مزد نیافتند.»
\par 19 لهذا خداوند یهوه چنین می‌فرماید: «اینک من زمین مصر را به نبوکدرصر پادشاه بابل خواهم بخشید. و جمعیت آن را گرفتار کرده، غنیمتش رابه یغما و اموالش را به تاراج خواهد برد تا اجرت لشکرش بشود.
\par 20 و خداوند یهوه می‌گوید: زمین مصر را به جهت خدمتی که کرده است، اجرت اوخواهم داد چونکه این کار را برای من کرده‌اند.و در آن روز شاخی برای خاندان اسرائیل خواهم رویانید. و دهان تو را در میان ایشان خواهم گشود، پس خواهند دانست که من یهوه هستم.»
\par 21 و در آن روز شاخی برای خاندان اسرائیل خواهم رویانید. و دهان تو را در میان ایشان خواهم گشود، پس خواهند دانست که من یهوه هستم.»

\chapter{30}

\par 1 و کلام خداوند بر من نازل شده، گفت:
\par 2 «ای پسر انسان نبوت کرده، بگو: خداوند یهوه چنین می‌فرماید: ولوله کنید وبگویید وای برآن روز!
\par 3 زیرا که آن روز نزدیک است و روز خداوند نزدیک است! روز ابرها وزمان امت‌ها خواهد بود!
\par 4 و شمشیری بر مصرفرود می‌آید. و چون کشتگان در مصر بیفتند، آنگاه درد شدیدی بر حبش مستولی خواهد شد. و جمعیت آن را گرفتار خواهد کرد و اساسهایش منهدم خواهد گردید.
\par 5 و حبش و فوط ولود وتمامی قومهای مختلف و کوب و اهل زمین عهدهمراه ایشان به شمشیر خواهند افتاد.»
\par 6 وخداوند چنین می‌فرماید: «معاونان مصر خواهندافتاد و فخر قوت آن فرود خواهد آمد. و ازمجدل تا اسوان در میان آن به شمشیر خواهندافتاد. قول خداوند یهوه این است.
\par 7 و در میان زمینهای ویران ویران خواهند شد و شهرهایش در میان و شهرهای مخروب خواهد بود.
\par 8 وچون آتشی در مصر افروخته باشم و جمیع انصارش شکسته شوند، آنگاه خواهند دانست که من یهوه هستم.
\par 9 در آن روز قاصدان از حضور من به کشتیها بیرون رفته، حبشیان مطمئن را خواهندترسانید. و بر ایشان درد شدیدی مثل روز مصرمستولی خواهد شد، زیرا اینک آن می‌آید.»
\par 10 وخداوند یهوه چنین می‌گوید: «من جمعیت مصررا به‌دست نبوکدرصر پادشاه بابل تباه خواهم ساخت.
\par 11 او با قوم خود و ستمکیشان امت هاآورده خواهند شد تا آن زمین ویران را سازند. وشمشیرهای خود را بر مصر کشیده، زمین را از کشتگان پر خواهند ساخت.
\par 12 و نهرها را خشک گردانیده، زمین را به‌دست اشرار خواهم فروخت. و زمین را با هرچه در آن است، به‌دست غریبان ویران خواهم ساخت. من که یهوه هستم گفته‌ام.»
\par 13 و خداوند یهوه چنین می‌فرماید: «بتها رانابود ساخته، اصنام را از نوف تلف خواهم نمود. و بار دیگر رئیسی از زمین مصر نخواهدبرخاست. و خوف بر زمین مصر مستولی خواهم ساخت.
\par 14 و فترس را خراب نموده، آتشی درصوعن خواهم افروخت. و بر نو داوری خواهم نمود.
\par 15 و غضب خود را برسین که ملاذ مصراست ریخته، جمعیت نو را منقطع خواهم ساخت.
\par 16 و چون آتشی در مصر افروخته باشم، سین به درد سخت مبتلا و نومفتوح خواهد شد. وخصمان در وقت روز بر نوف خواهند آمد.
\par 17 جوانان آون و فیبست به شمشیر خواهند افتادو اهل آنها به اسیری خواهند رفت.
\par 18 و روزدرتحفنحیس تاریک خواهد شد حینی که یوغهای مصر را در آنجا شکسته باشم و فخرقوتش در آن تلف شده باشد. و ابرها آن را خواهدپوشانید و دخترانش به اسیری خواهند رفت.
\par 19 پس چون بر مصر داوری کرده باشم، آنگاه خواهند دانست که من یهوه هستم.»
\par 20 و در روز هفتم ماه اول از سال یازدهم، کلام خداوند بر من نازل شده، گفت:
\par 21 «ای پسر انسان بازوی فرعون پادشاه مصر را خواهم شکست. واینک شکسته بندی نخواهد شد و بر آن مرهم نخواهند گذارد و کرباس نخواهند بست تا قادر برگرفتن شمشیر بشود.
\par 22 بنابراین خداوند یهوه چنین می‌گوید: هان من به ضد فرعون پادشاه مصرهستم و هر دو بازوی او هم درست و هم شکسته را خرد خواهم کرد و شمشیر را از دستش خواهم انداخت.
\par 23 و مصریان را در میان امت‌ها پراکنده و در میان کشورها متفرق خواهم ساخت.
\par 24 وبازوهای پادشاه بابل را تقویت نموده، شمشیرخود را به‌دست او خواهم داد. و بازوهای فرعون را خواهم شکست که به حضور وی به ناله کشتگان ناله خواهد کرد.
\par 25 پس بازوهای پادشاه بابل را تقویت خواهم نمود. و بازوهای فرعون خواهد افتاد و چون شمشیر خود را به‌دست پادشاه بابل داده باشم و او آن را بر زمین مصر درازکرده باشد، آنگاه خواهند دانست که من یهوه هستم.و چون مصریان را در میان امت هاپراکنده و ایشان را در کشورها متفرق ساخته باشم، ایشان خواهند دانست که من یهوه هستم.»
\par 26 و چون مصریان را در میان امت هاپراکنده و ایشان را در کشورها متفرق ساخته باشم، ایشان خواهند دانست که من یهوه هستم.»

\chapter{31}

\par 1 و در روز اول ماه سوم از سال یازدهم، کلام خداوند بر من نازل شده، گفت:
\par 2 «ای پسر انسان به فرعون پادشاه مصر و به جمعیت او بگو: کیست که در بزرگیت به اوشباهت داری؟
\par 3 اینک آشور سرو آزاد لبنان باشاخه های جمیل و برگهای سایه‌گستر و قد بلندمی بود و سر او به ابرها می‌بود.
\par 4 آبها او را نمو داد. و لجه او را بلند ساخت که نهرهای آنها بهر طرف بوستان آن جاری می‌شد و جویهای خویش رابطرف همه درختان صحرا روان می‌ساخت.
\par 5 ازاین جهت قد او از جمیع درختان صحرا بلندترشده، شاخه هایش زیاده گردید و اغصان خود رانمو داده، آنها از کثرت آبها بلند شد.
\par 6 و همه مرغان هوا در شاخه هایش آشیانه ساختند. وتمامی حیوانات صحرا زیر اغصانش بچه آوردند. و جمیع امت های عظیم در سایه‌اش سکنی گرفتند.
\par 7 پس در بزرگی خود و در درازی شاخه های خویش خوشنما شد چونکه ریشه‌اش نزد آبهای بسیار بود.
\par 8 سروهای آزاد باغ خدا آن را نتوانست پنهان کرد. و صنوبرها به شاخه هایش مشابهت نداشت. و چنارها مثل اغصانش نبودبلکه هیچ درخت در باغ خدا به زیبایی او مشابه نبود.
\par 9 من او را به کثرت شاخه هایش به حدی زیبایی دادم که همه درختان عدن که در باغ خدابود بر او حسد بردند.»
\par 10 بنابراین خداوند یهوه چنین می‌فرماید: «چونکه قد تو بلند شده است، و او سر خود را درمیان ابرها برافراشته و دلش از بلندیش مغرورگردیده است،
\par 11 از این جهت من او را به‌دست قوی ترین (پادشاه ) امت‌ها تسلیم خواهم نمود واو آنچه را که می‌باید به وی خواهد کرد. و من او رابه‌سبب شرارتش بیرون خواهم‌انداخت.
\par 12 وغریبان یعنی ستمکیشان امت‌ها او را منقطع ساخته، ترک خواهند نمود. و شاخه هایش برکوهها و در جمیع دره‌ها خواهد افتاد و اغصان اونزد همه وادیهای زمین شکسته خواهد شد. وجمیع قوم های زمین از زیر سایه او فرود آمده، اورا ترک خواهند نمود.
\par 13 و همه مرغان هوا بر تنه افتاده او آشیانه گرفته، تمامی حیوانات صحرا برشاخه هایش ساکن خواهند شد.
\par 14 تا آنکه هیچکدام از درختانی که نزد آبها می‌باشند قدخود را بلند نکنند و سرهای خود را در میان ابرهابرنیفرازند. و زورآوران آنها از همگانی که سیراب می‌باشند، در بلندی خود نایستند. زیرا که جمیع آنها در اسفلهای زمین در میان پسران انسانی که به هاویه فرود می‌روند به مرگ تسلیم شده‌اند.»
\par 15 و خداوند یهوه چنین می‌گوید: «در روزی که او به عالم اموات فرود می‌رود، من ماتمی برپامی نمایم و لجه را برای وی پوشانیده، نهرهایش را باز خواهم داشت. و آبهای عظیم باز داشته خواهد شد و لبنان را برای وی سوگوار خواهم کرد. و جمیع درختان صحرا برایش ماتم خواهندگرفت.
\par 16 و چون او را با آنانی که به هاویه فرودمی روند به عالم اموات فرود آورم، آنگاه امت هارا از صدای انهدامش متزلزل خواهم ساخت. وجمیع درختان عدن یعنی برگزیده و نیکوترین لبنان از همگانی که سیراب می‌شوند، در اسفلهای زمین تسلی خواهند یافت.
\par 17 و ایشان نیز بامقتولان شمشیر و انصارش که در میان امت‌ها زیرسایه او ساکن می‌بودند، همراه وی به عالم اموات فرود خواهند رفت.به کدام‌یک از درختان عدن در جلال و عظمت چنین شباهت داشتی؟ اما با درختان عدن به اسفلهای زمین تو را فرودخواهند‌آورد و در میان نامختونان با مقتولان شمشیر خواهی خوابید. خداوند یهوه می‌گویدکه فرعون و تمامی جماعتش این است.»
\par 18 به کدام‌یک از درختان عدن در جلال و عظمت چنین شباهت داشتی؟ اما با درختان عدن به اسفلهای زمین تو را فرودخواهند‌آورد و در میان نامختونان با مقتولان شمشیر خواهی خوابید. خداوند یهوه می‌گویدکه فرعون و تمامی جماعتش این است.»

\chapter{32}

\par 1 و در روز اول ماه دوازدهم از سال دوازدهم واقع شد که کلام خداوند برمن نازل شده، گفت:
\par 2 «ای پسر انسان برای فرعون پادشاه مصر مرثیه بخوان و او را بگو تو به شیرژیان امت‌ها مشابه می‌بودی، اما مانند اژدها دردریا هستی و آب را از بینی خود می‌جهانی و آبهارا به پایهای خود حرکت داده، نهرهای آنها راگل آلود می‌سازی.»
\par 3 خداوند یهوه چنین می‌گوید: «دام خود را به واسطه گروهی از قوم های عظیم بر تو خواهم گسترانید و ایشان تورا در دام من بر خواهند کشید.
\par 4 و تو را بر زمین ترک نموده، بر روی صحرا خواهم‌انداخت وهمه مرغان هوا را بر تو فرود آورده، جمیع حیوانات زمین را از تو سیر خواهم ساخت.
\par 5 وگوشت تو را بر کوهها نهاده، دره‌ها را از لاش تو پرخواهم کرد.
\par 6 و زمینی را که در آن شنا می‌کنی ازخون تو تا به کوهها سیراب می‌کنم که وادیها از توپر خواهد شد.
\par 7 و هنگامی که تو را منطفی گردانم، آسمان را خواهم پوشانید و ستارگانش راتاریک کرده، آفتاب را به ابرها مستور خواهم ساخت و ماه روشنایی خود را نخواهد داد.
\par 8 وخداوند یهوه می‌فرماید، که تمامی نیرهای درخشنده آسمان را برای تو سیاه کرده، تاریکی بر زمینت خواهم آورد.
\par 9 و چون هلاکت تو را درمیان امت‌ها بر زمینهایی که ندانسته‌ای آورده باشم، آنگاه دلهای قوم های عظیم را محزون خواهم ساخت.
\par 10 و قوم های عظیم را بر تومتحیر خواهم ساخت. و چون شمشیر خود راپیش روی ایشان جلوه دهم، پادشاهان ایشان به شدت دهشتناک خواهند شد. و در روز انهدام توهر یک از ایشان برای جان خود هر لحظه‌ای خواهند لرزید.»
\par 11 زیرا خداوند یهوه چنین می‌گوید: «شمشیرپادشاه بابل بر تو خواهد آمد.
\par 12 و به شمشیرهای جباران که جمیع ایشان از ستمکیشان امت هامی باشند، جمعیت تو را به زیر خواهم‌انداخت. وایشان غرور مصر را نابود ساخته، تمامی جمعیتش هلاک خواهند شد.
\par 13 و تمامی بهایم او را از کنارهای آبهای عظیم هلاک خواهم ساخت. و پای انسان دیگر آنها را گل آلودنخواهد ساخت. و سم بهایم آنها را گل آلودنخواهد ساخت.
\par 14 آنگاه خداوند یهوه می‌گوید: آبهای آنها را ساکت گردانیده، نهرهای آنها رامانند روغن جاری خواهم ساخت.
\par 15 و چون زمین مصر را ویران کنم و آن زمین از هرچه در آن باشد خالی شود و چون جمیع ساکنانش را هلاک کنم، آنگاه خواهند دانست که من یهوه هستم.»
\par 16 و خداوند یهوه می‌گوید: «مرثیه‌ای که ایشان خواهند خواند همین است. دختران امت‌ها این مرثیه را خواهند خواند. برای مصر و تمامی جمعیتش این مرثیه را خواهند خواند.»
\par 17 و در روز پانزدهم ماه از سال دوازدهم واقع شد که کلام خداوند بر من نازل شده، گفت:
\par 18 «ای پسر انسان برای جمعیت مصر ولوله نما و هم اورا و هم دختران امت های عظیم را با آنانی که به هاویه فرود می‌روند، به اسفلهای زمین فرود آور.
\par 19 از چه کس زیباتر هستی؟ فرود بیا و بانامختونان بخواب.
\par 20 ایشان در میان مقتولان شمشیر خواهند افتاد. (مصر) به شمشیر تسلیم شده است. پس او را و تمامی جمعیتش رابکشید.
\par 21 اقویای جباران از میان عالم اموات اورا و انصار او را خطاب خواهند کرد. ایشان نامختون به شمشیر کشته شده، فرود آمده، خواهند خوابید.
\par 22 «در آنجا آشور و تمامی جمعیت اوهستند. قبرهای ایشان گرداگرد ایشان است وجمیع ایشان کشته شده از شمشیر افتاده‌اند.
\par 23 که قبرهای ایشان به اسفلهای هاویه قرار داده شد وجمعیت ایشان به اطراف قبرهای ایشان‌اند.
\par 24 در آنجاعیلام و تمامی جمعیتش هستند. قبرهای ایشان گرداگرد ایشان است و جمیع ایشان مقتول و ازشمشیر افتاده‌اند و به اسفلهای زمین نامختون فرود رفته‌اند، زیرا که در زمین زندگان باعث هیبت بوده‌اند. پس با آنانی که به هاویه فرود می‌روند، متحمل خجالت خویش خواهند بود.
\par 25 بستری برای او و تمامی جمعیتش در میان مقتولان قرارداده‌اند. قبرهای ایشان گرداگرد ایشان است وجمیع ایشان نامختون و مقتول شمشیرند. زیرا که در زمین زندگان باعث هیبت بودند. پس با آنانی که به هاویه فرود می‌روند، متحمل خجالت خویش خواهند بود. در میان کشتگان قرار داده شد.
\par 26 درآنجا ماشک و توبال و تمامی جمعیت آنهاهستند. قبرهای ایشان گرداگرد ایشان است وجمیع ایشان نامختون و مقتول شمشیرند. زیرا که در زمین زندگان باعث هیبت بودند.
\par 27 پس ایشان با جباران و نامختونانی که افتاده‌اند که با اسلحه جنگ خویش به هاویه فرود رفته‌اند، نخواهندخوابید. و ایشان شمشیرهای خود را زیر سرهای خود نهادند. و گناه ایشان بر استخوانهای ایشان خواهد بود. زیرا که در زمین زندگان باعث هیبت جباران بودند.
\par 28 و اما تو در میان نامختونان شکسته شده، با مقتولان شمشیر خواهی خوابید.
\par 29 در آنجا ادوم و پادشاهانش و جمیع سرورانش هستند که در جبروت خود با مقتولان شمشیرقرار داده شدند. و ایشان با نامختونان و آنانی که به هاویه فرود می‌روند خواهند خوابید.
\par 30 در آنجاجمیع روسای شمال و همه صیدونیان هستند که با مقتولان فرود رفتند. از هیبتی که به جبروت خویش باعث آن بودند، خجل خواهند شد. پس با مقتولان شمشیر نامختون خواهند خوابید و باآنانی که به هاویه فرود می‌روند، متحمل خجالت خود خواهند شد.
\par 31 و خداوند یهوه می‌گوید که فرعون چون این را بیند درباره تمامی جمعیت خود خویشتن را تسلی خواهد داد و فرعون وتمامی لشکر او به شمشیر کشته خواهند شد.زیرا خداوند یهوه می‌گوید: من او را در زمین زندگان باعث هیبت گردانیدم. پس فرعون وتمامی جمعیت او را با مقتولان شمشیر در میان نامختونان خواهند خوابانید.»
\par 32 زیرا خداوند یهوه می‌گوید: من او را در زمین زندگان باعث هیبت گردانیدم. پس فرعون وتمامی جمعیت او را با مقتولان شمشیر در میان نامختونان خواهند خوابانید.»

\chapter{33}

\par 1 و کلام خداوند بر من نازل شده، گفت:
\par 2 «ای پسر انسان پسران قوم خود راخطاب کرده، به ایشان بگو: اگر من شمشیری برزمینی آورم و اهل آن زمین کسی را از میان خودگرفته، او را به جهت خود به دیده بانی تعیین کنند،
\par 3 و او شمشیر را بیند که بر آن زمین می‌آید وکرنارا نواخته، آن قوم را متنبه سازد،
\par 4 و اگر کسی صدای کرنا را شنیده، متنبه نشود، آنگاه شمشیرآمده، او را گرفتار خواهد ساخت و خونش برگردنش خواهد بود.
\par 5 چونکه صدای کرنا را شنیدو متنبه نگردید، خون او بر خودش خواهد بود واگر متنبه می‌شد جان خود را می‌رهانید.
\par 6 و اگردیده بان شمشیر را بیند که می‌آید و کرنا راننواخته قوم را متنبه نسازد و شمشیر آمده، کسی را از میان ایشان گرفتار سازد، آن شخص در گناه خود گرفتار شده است، اما خون او را از دست دیده بان خواهم طلبید.
\par 7 و من تو را‌ای پسر انسان برای خاندان اسرائیل به دیده بانی تعیین نموده‌ام تا کلام را از دهانم شنیده، ایشان را از جانب من متنبه سازی.
\par 8 حینی که من به مرد شریر گویم: ای مرد شریر البته خواهی مرد! اگر تو سخن نگویی تاآن مرد شریر را از طریقش متنبه سازی، آنگاه آن مرد شریر در گناه خود خواهد مرد، اما خون او رااز دست تو خواهم طلبید.
\par 9 اما اگر تو آن مردشریر را از طریقش متنبه سازی تا از آن بازگشت نماید و او از طریق خود بازگشت نکند، آنگاه اودر گناه خود خواهد مرد، اما تو جان خود رارستگار ساخته‌ای.
\par 10 پس تو‌ای پسر انسان به خاندان اسرائیل بگو: شما بدین مضمون می‌گویید: چونکه عصیان و گناهان ما بر گردن مااست و به‌سبب آنها کاهیده شده‌ایم، پس چگونه زنده خواهیم ماند؟
\par 11 «به ایشان بگو: خداوند یهوه می‌فرماید: به حیات خودم قسم که من از مردن مرد شریرخوش نیستم بلکه (خوش هستم ) که شریر ازطریق خود بازگشت نموده، زنده ماند. ای خاندان اسرائیل بازگشت نمایید! از طریق های بد خویش بازگشت نمایید زیرا چرا بمیرید؟
\par 12 و تو‌ای پسرانسان به پسران قوم خود بگو: عدالت مرد عادل در روزی که مرتکب گناه شود، او را نخواهدرهانید. و شرارت مرد شریر در روزی که او ازشرارت خود بازگشت نماید، باعث هلاکت وی نخواهد شد. و مرد عادل در روزی که گناه ورزد، به عدالت خود زنده نتواند ماند.
\par 13 حینی که به مرد عادل گویم که البته زنده خواهی ماند، اگر اوبه عدالت خود اعتماد نموده، عصیان ورزد، آنگاه عدالتش هرگز به یاد آورده نخواهد شد بلکه درعصیانی که ورزیده است خواهد مرد.
\par 14 وهنگامی که به مرد شریر گویم: البته خواهی مرد! اگر او از گناه خود بازگشت نموده، انصاف و عدالت بجا آورد،
\par 15 و اگر آن مرد شریر رهن راپس دهد و آنچه دزدیده بود رد نماید و به فرایض حیات سلوک نموده، مرتکب بی‌انصافی نشود، اوالبته زنده خواهد ماند و نخواهد مرد.
\par 16 تمام گناهی که ورزیده بود بر او به یاد آورده نخواهدشد. چونکه انصاف و عدالت را بجا آورده است، البته زنده خواهد ماند.
\par 17 اما پسران قوم تومی گویند که طریق خداوند موزون نیست، بلکه طریق خود ایشان است که موزون نیست.
\par 18 هنگامی که مرد عادل از عدالت خود برگشته، عصیان ورزد، به‌سبب آن خواهد مرد.
\par 19 و چون مرد شریر از شرارت خود بازگشت نموده، انصاف و عدالت را بجا آورد به‌سبب آن زنده خواهدماند.
\par 20 اما شما می‌گویید که طریق خداوندموزون نیست. ای خاندان اسرائیل من بر هر یکی از شما موافق طریق هایش داوری خواهم نمود.»
\par 21 و در روز پنجم ماه دهم از سال دوازدهم اسیری ما واقع شد که کسی‌که از اورشلیم فرارکرده بود نزد من آمده، خبر داد که شهر تسخیرشده است.
\par 22 و در وقت شام قبل از رسیدن آن فراری دست خداوند بر من آمده، دهان مراگشود. پس چون او در وقت صبح نزد من رسید، دهانم گشوده شد و دیگر گنگ نبودم.
\par 23 و کلام خداوند بر من نازل شده، گفت:
\par 24 «ای پسر انسان ساکنان این خرابه های زمین اسرائیل می‌گویند: ابراهیم یک نفر بود حینی که وارث این زمین شد وما بسیار هستیم که زمین به ارث به ما داده شده است.
\par 25 بنابراین به ایشان بگو: خداوند یهوه چنین می‌فرماید: (گوشت را) با خونش می‌خورید و چشمان خود را بسوی بتهای خویش برمی افرازید و خون می‌ریزید. پس آیاشما وارث زمین خواهید شد؟
\par 26 «بر شمشیرهای خود تکیه می‌کنید ومرتکب رجاسات شده، هر کدام از شما زن همسایه خود را نجس می‌سازید. پس آیا وارث این زمین خواهید شد؟
\par 27 بدینطور به ایشان بگوکه خداوند یهوه چنین می‌فرماید: به حیات خودم قسم البته آنانی که در خرابه‌ها هستند به شمشیرخواهند افتاد. و آنانی که بر روی صحرااند برای خوراک به حیوانات خواهم داد. و آنانی که درقلعه‌ها و مغارهایند از وبا خواهند مرد.
\par 28 و این زمین را ویران و محل دهشت خواهم ساخت وغرور قوتش نابود خواهد شد. و کوههای اسرائیل به حدی ویران خواهد شد که رهگذری نباشد.
\par 29 و چون این زمین را به‌سبب همه رجاساتی که ایشان بعمل آورده‌اند ویران و محل دهشت ساخته باشم، آنگاه خواهند دانست که من یهوه هستم.
\par 30 اما تو‌ای پسر انسان پسران قومت به پهلوی دیوارها و نزد درهای خانه‌ها درباره توسخن می‌گویند. و هر یک به دیگری و هر کس به برادرش خطاب کرده، می‌گوید بیایید و بشنوید! چه کلام است که از جانب خداوند صادرمی شود.
\par 31 و نزد تو می‌آیند بطوری که قوم (من )می آیند. و مانند قوم من پیش تو نشسته، سخنان تو را می‌شنوند، اما آنها را بجا نمی آورند. زیرا که ایشان به دهان خود سخنان شیرین می‌گویند. لیکن دل ایشان در‌پی حرص ایشان می‌رود.
\par 32 واینک تو برای ایشان مثل سرود شیرین مطرب خوشنوا و نیک نواز هستی. زیرا که سخنان تو رامی شنوند، اما آنها را بجا نمی آورند.و چون این واقع می‌شود و البته واقع خواهد شد، آنگاه خواهند دانست که نبی در میان ایشان بوده است.»
\par 33 و چون این واقع می‌شود و البته واقع خواهد شد، آنگاه خواهند دانست که نبی در میان ایشان بوده است.»

\chapter{34}

\par 1 و کلام خداوند بر من نازل شده، گفت:
\par 2 «ای پسر انسان به ضد شبانان اسرائیل نبوت نما و نبوت کرده، به ایشان یعنی به شبانان بگو: خداوند یهوه چنین می‌فرماید: وای بر شبانان اسرائیل که خویشتن را می‌چرانند. آیا نمی بایدشبانان گله را بچرانند؟
\par 3 شما پیه را می‌خورید وپشم را می‌پوشید و پرواریها را می‌کشید، اما گله را نمی چرانید.
\par 4 ضعیفان را تقویت نمی دهید وبیماران را معالجه نمی نمایید و شکسته‌ها راشکسته بندی نمی کنید و رانده شدگان را پس نمی آورید و گم شدگان را نمی طلبید، بلکه بر آنهابا جور و ستم حکمرانی می‌نمایید.
\par 5 پس بدون شبان پراکنده می‌شوند و خوراک همه حیوانات صحرا گردیده، آواره می‌گردند.
\par 6 گوسفندان من بر جمیع کوهها و بر همه تلهای بلند آواره شده‌اند. و گله من بر روی تمامی زمین پراکنده گشته، کسی ایشان را نمی طلبد و برای ایشان تفحص نمی نماید.
\par 7 پس‌ای شبانان کلام خداوندرا بشنوید!
\par 8 خداوند یهوه می‌فرماید: به حیات خودم قسم هر آینه چونکه گله من به تاراج رفته وگوسفندانم خوراک همه حیوانات صحرا گردیده، شبانی ندارند. و شبانان من گوسفندانم رانطلبیدند. بلکه شبانان خویشتن را چرانیدند و گله مرا رعایت ننمودند.
\par 9 «بنابراین‌ای شبانان! کلام خداوند رابشنوید!
\par 10 خداوند یهوه چنین می‌فرماید: اینک من به ضد شبانان هستم. و گوسفندان خود را ازدست ایشان خواهم طلبید. و ایشان را از چرانیدن گله معزول خواهم ساخت تا شبانان خویشتن رادیگر نچرانند. و گوسفندان خود را از دهان ایشان خواهم رهانید تا خوراک ایشان نباشند.
\par 11 زیراخداوند یهوه چنین می‌گوید: هان من خودم گوسفندان خویش را طلبیده، آنها را تفقد خواهم نمود.
\par 12 چنانکه شبان حینی که در میان گوسفندان پراکنده خود می‌باشد، گله خویش راتفقد می‌نماید، همچنان من گوسفندان خویش راتفقد نموده، ایشان را از هر جایی که در روز ابرهاو تاریکی غلیظ پراکنده شده بودند خواهم رهانید.
\par 13 و ایشان را از میان قوم‌ها بیرون آورده، از کشورها جمع خواهم نمود. و به زمین خودشان درآورده، بر کوههای اسرائیل و در وادیها وجمیع معمورات زمین ایشان را خواهم چرانید.
\par 14 ایشان را بر مرتع نیکو خواهم چرانید و آرام گاه ایشان بر کوههای بلند اسرائیل خواهد بود. وآنجا در آرام گاه نیکو و مرتع پرگیاه خواهندخوابید. و بر کوههای اسرائیل خواهند چرید.
\par 15 خداوند یهوه می‌گوید که من گوسفندان خودرا خواهم چرانید و من ایشان را خواهم خوابانید.
\par 16 گم شدگان را خواهم طلبید و رانده شدگان راباز خواهم آورد و شکسته‌ها را شکسته بندی نموده، بیماران را قوت خواهم داد. لیکن فربهان وزورآوران را هلاک ساخته، بر ایشان به انصاف رعایت خواهم نمود.
\par 17 و اما به شما‌ای گوسفندان من، خداوند یهوه چنین می‌فرماید: هان من در میان گوسفند و گوسفند و در میان قوچهای و بزهای نر داوری خواهم نمود.
\par 18 آیا برای شما کم بود که مرتع نیکو راچرانیدید بلکه بقیه مرتع خود را نیز به پایهای خویش پایمال ساختید؟ و آب زلال را نوشیدیدبلکه باقی‌مانده را به پایهای خویش گل آلودساختید؟
\par 19 و گوسفندان من آنچه را که از پای شما پایمال شده است، می‌چرند و آنچه را که به پای شما گل آلود گشته است، می‌نوشند.
\par 20 بنابراین خداوند یهوه به ایشان چنین می‌گوید: هان من خودم در میان گوسفندان فربه وگوسفندان لاغر داوری خواهم نمود.
\par 21 چونکه شما به پهلو و کتف خود تنه می‌زنید و همه ضعیفان را به شاخهای خود می‌زنید، حتی اینکه ایشان را بیرون پراکنده ساخته‌اید،
\par 22 پس من گله خود را نجات خواهم داد که دیگر به تاراج برده نشوند و در میان گوسفند و گوسفند داوری خواهم نمود.
\par 23 و یک شبان بر ایشان خواهم گماشت که ایشان را بچراند یعنی بنده خود داودرا که ایشان را رعایت بنماید و او شبان ایشان خواهد بود.
\par 24 و من یهوه خدای ایشان خواهم بود و بنده من داود در میان ایشان رئیس خواهدبود. من که یهوه هستم گفته‌ام.
\par 25 و عهد سلامتی را با ایشان خواهم بست. و حیوانات موذی را اززمین نابود خواهم ساخت و ایشان در بیابان به امنیت ساکن شده، در جنگلها خواهند خوابید.
\par 26 و ایشان را و اطراف کوه خود را برکت خواهم ساخت. و باران را در موسمش خواهم بارانید وبارشهای برکت خواهد بود.
\par 27 و درختان صحرامیوه خود را خواهند‌آورد و زمین حاصل خویش را خواهد داد. و ایشان در زمین خود به امنیت ساکن خواهند شد. و حینی که چوبهای یوغ ایشان را شکسته و ایشان را از دست آنانی که ایشان را مملوک خود ساخته بودند رهانیده باشم، آنگاه خواهند دانست که من یهوه هستم.
\par 28 ودیگر در میان امت‌ها به تاراج نخواهد رفت وحیوانات صحرا ایشان را نخواهند خورد بلکه به امنیت، بدون ترساننده‌ای ساکن خواهند شد.
\par 29 وبرای ایشان درختستان ناموری بر پا خواهم داشت. و دیگر از قحط در زمین تلف نخواهندشد. و بار دیگر متحمل سرزنش امت‌ها نخواهندگردید.
\par 30 و خداوند یهوه می‌گوید: خاندان اسرائیل خواهند دانست که من یهوه خدای ایشان با ایشان هستم و ایشان قوم من می‌باشند.وخداوند یهوه می‌گوید: شما‌ای گله من و‌ای گوسفندان مرتع من، انسان هستید و من خدای شما می‌باشم.»
\par 31 وخداوند یهوه می‌گوید: شما‌ای گله من و‌ای گوسفندان مرتع من، انسان هستید و من خدای شما می‌باشم.»

\chapter{35}

\par 1 و کلام خداوند بر من نازل شده، گفت:
\par 2 «ای پسر انسان نظر خود را بر کوه سعیر بدار و به ضد آن نبوت نما!
\par 3 و آن را بگوخداوند یهوه چنین می‌فرماید: اینک‌ای کوه سعیر من به ضد تو هستم. و دست خود را بر تودراز کرده، تو را ویران و محل دهشت خواهم ساخت.
\par 4 شهرهایت را خراب خواهم نمود تاویران شده، بدانی که من یهوه هستم.
\par 5 چونکه عداوت دائمی داشتی و بنی‌اسرائیل را در زمان مصیبت ایشان و هنگام عقوبت آخر به دم شمشیرتسلیم نمودی،
\par 6 لهذا خداوند یهوه چنین می گوید: به حیات خودم قسم که تو را به خون تسلیم خواهم نمود که خون تو را تعاقب نماید. چون از خون نفرت نداشتی، خون تو را تعاقب خواهد نمود.
\par 7 و کوه سعیر را محل دهشت وویران ساخته، روندگان و آیندگان را از آن منقطع خواهم ساخت.
\par 8 و کوههایش را از کشتگانش مملو می‌کنم که مقتولان شمشیر بر تلها و دره‌ها وهمه وادیهای تو بیفتند.
\par 9 و تو را خرابه های دائمی می‌سازم که شهرهایت دیگر مسکون نشودو بدانید که من یهوه هستم.
\par 10 چونکه گفتی این دوامت و این دو زمین از آن من می‌شود و آن را به تصرف خواهیم آورد با آنکه یهوه در آنجا است.»
\par 11 بنابراین خداوند یهوه چنین می‌گوید: «به حیات خودم قسم که موافق خشم و حسدی که به ایشان نمودی، از کینه‌ای که با ایشان داشتی با توعمل خواهم نمود. و چون بر تو داوری کرده باشم، خویشتن را بر تو در میان ایشان معروف خواهم گردانید.
\par 12 و خواهی دانست که من یهوه تمامی سخنان کفرآمیز را که به ضد کوههای اسرائیل گفته‌ای، شنیده‌ام. چونکه گفتی: خراب گردید و برای خوراک ما داده شد.
\par 13 و شما به دهان خود به ضد من تکبر نموده، سخنان خویش را بر من افزودید و من آنها را شنیدم.
\par 14 خداوندیهوه چنین می‌گوید: حینی که تمامی جهان شادی کنند من تو را ویران خواهم ساخت.وچنانکه بر میراث خاندان اسرائیل حینی که ویران شد شادی نمودی، همچنان با تو عمل خواهم نمود. و تو‌ای کوه سعیر و تمام ادوم جمیع ویران خواهید شد. پس خواهند دانست که من یهوه هستم.»
\par 15 وچنانکه بر میراث خاندان اسرائیل حینی که ویران شد شادی نمودی، همچنان با تو عمل خواهم نمود. و تو‌ای کوه سعیر و تمام ادوم جمیع ویران خواهید شد. پس خواهند دانست که من یهوه هستم.»

\chapter{36}

\par 1 «و تو‌ای پسر انسان به کوههای اسرائیل نبوت کرده، بگو: ای کوههای اسرائیل کلام خداوند را بشنوید!
\par 2 خداوند یهوه چنین می‌گوید: چونکه دشمنان درباره شما گفته‌اند هه این بلندیهای دیرینه میراث ما شده است،
\par 3 لهذانبوت کرده، بگو که خداوند یهوه چنین می‌فرماید: از آن جهت که ایشان شما را از هرطرف خراب کرده و بلعیده‌اند تا میراث بقیه امت‌ها بشوید و بر لبهای حرف گیران برآمده، مورد مذمت طوایف گردیده‌اید،
\par 4 لهذا‌ای کوههای اسرائیل کلام خداوند یهوه را بشنوید! خداوند یهوه به کوهها و تلها و وادیها و دره‌ها وخرابه های ویران و شهرهای متروکی که تاراج شده و مورد سخریه بقیه امت های مجاور گردیده است، چنین می‌گوید:
\par 5 بنابراین خداوند یهوه چنین می‌فرماید: هر آینه به آتش غیرت خود به ضد بقیه امت‌ها و به ضد تمامی ادوم تکلم نموده‌ام که ایشان زمین مرا به شادی تمام دل وکینه قلب، ملک خود ساخته‌اند تا آن را به تاراج واگذارند.
\par 6 پس درباره زمین اسرائیل نبوت نماوبه کوهها و تلها و وادیها و دره‌ها بگو که خداوندیهوه چنین می‌فرماید: چونکه شما متحمل سرزنش امت‌ها شده‌اید، لهذا من در غیرت وخشم خود تکلم نمودم.»
\par 7 و خداوند یهوه چنین می‌گوید: «من دست خود را برافراشته‌ام که امت هایی که به اطراف شمایند البته سرزنش خود را متحمل خواهندشد.
\par 8 و شما‌ای کوههای اسرائیل شاخه های خود را خواهید رویانید و میوه خود را برای قوم من اسرائیل خواهید آورد زیرا که ایشان به زودی خواهند آمد.
\par 9 زیرا اینک من بطرف شما هستم. و بر شما نظر خواهم داشت و شیار شده، کاشته خواهید شد.
\par 10 و بر شما مردمان را خواهم افزودیعنی تمامی خاندان اسرائیل را جمیع. و شهرهامسکون و خرابه‌ها معمور خواهد شد.
\par 11 و برشما انسان و بهایم بسیار خواهم آورد که ایشان افزوده شده، بارور خواهند شد. و شما را مثل ایام قدیم معمور خواهم ساخت. بلکه بر شما بیشتر ازاول شما احسان خواهم نمود و خواهید دانست که من یهوه هستم.
\par 12 و مردمان یعنی قوم خوداسرائیل را بر شما خرامان خواهم ساخت تا تو رابه تصرف آورند. و میراث ایشان بشوی و ایشان رادیگر بی‌اولاد نسازی.»
\par 13 و خداوند یهوه چنین می‌گوید: «چونکه ایشان درباره تو می‌گویند که مردمان را می‌بلعی وامت های خویش را بی‌اولاد می‌گردانی،
\par 14 پس خداوند یهوه می‌گوید: مردمان را دیگر نخواهی بلعید و امت های خویش را دیگر بی‌اولادنخواهی ساخت.
\par 15 و سرزنش امت‌ها را دیگر درتو مسموع نخواهم گردانید. و دیگر متحمل مذمت طوایف نخواهی شد و امت های خویش رادیگر نخواهی لغزانید. خداوند یهوه این رامی گوید.»
\par 16 و کلام خداوند بر من نازل شده، گفت:
\par 17 «ای پسر انسان، هنگامی که خاندان اسرائیل درزمین خود ساکن می‌بودند آن را به راهها و به اعمال خود نجس نمودند. و طریق ایشان به نظرمن مثل نجاست زن حایض می‌بود.
\par 18 لهذا به‌سبب خونی که بر زمین ریختند و آن را به بتهای خود نجس ساختند، من خشم خود را بر ایشان ریختم.
\par 19 و ایشان را در میان امت‌ها پراکنده ساختم و در کشورها متفرق گشتند. و موافق راهها و اعمال ایشان، بر ایشان داوری نمودم.
\par 20 و چون به امت هایی که بطرف آنها رفتندرسیدند، آنگاه اسم قدوس مرا بی‌حرمت ساختند. زیرا درباره ایشان گفتند که اینان قوم یهوه می‌باشند و از زمین او بیرون آمده‌اند.
\par 21 لیکن من بر اسم قدوس خود که خاندان اسرائیل آن را در میان امت هایی که بسوی آنهارفته بودند بی‌حرمت ساختند شفقت نمودم.
\par 22 «بنابراین به خاندان اسرائیل بگو: خداوندیهوه چنین می‌فرماید: ای خاندان اسرائیل من این را نه به‌خاطر شما بلکه بخاطر اسم قدوس خودکه آن را در میان امت هایی که به آنها رفته، بی‌حرمت نموده‌اید بعمل می‌آورم.
\par 23 و اسم عظیم خود را که در میان امت‌ها بی‌حرمت شده است و شما آن را در میان آنها بی‌عصمت ساخته‌اید، تقدیس خواهم نمود. و خداوند یهوه می‌گوید: حینی که بنظر ایشان در شما تقدیس کرده شوم، آنگاه امت‌ها خواهند دانست که من یهوه هستم.
\par 24 و شما را از میان امت‌ها می‌گیرم واز جمیع کشورها جمع می‌کنم و شما را در زمین خود در خواهم آورد.
\par 25 و آب پاک بر شماخواهم پاشید و طاهر خواهید شد. و شما را ازهمه نجاسات واز همه بتهای شما طاهر خواهم ساخت.
\par 26 و دل تازه به شما خواهم داد و روح تازه در اندرون شما خواهم نهاد. و دل سنگی را ازجسد شما دور کرده، دل گوشتین به شما خواهم داد.
\par 27 و روح خود را در اندرون شما خواهم نهاد و شما را به فرایض خود سالک خواهم گردانید تا احکام مرا نگاه داشته، آنها را بجاآورید.
\par 28 و در زمینی که به پدران شما دادم ساکن شده، قوم من خواهید بود و من خدای شما خواهم بود.
\par 29 و شما را از همه نجاسات شمانجات خواهم داد. و غله را خوانده، آن را فراوان خواهم ساخت و دیگر قحط بر شما نخواهم فرستاد.
\par 30 و میوه درختان و حاصل زمین رافراوان خواهم ساخت تا دیگر در میان امت هامتحمل رسوایی قحط نشوید.
\par 31 و چون راههای قبیح و اعمال ناپسند خود را به یاد آورید، آنگاه به‌سبب گناهان و رجاسات خود خویشتن را درنظر خود مکروه خواهید داشت.»
\par 32 و خداوندیهوه می‌گوید: «بدانید که من این را به‌خاطر شمانکرده‌ام. پس‌ای خاندان اسرائیل به‌سبب راههای خود خجل و رسوا شوید.»
\par 33 خداوند یهوه چنین می‌فرماید: «در روزی که شما را از تمامی گناهانتان طاهر سازم، شهرها را مسکون خواهم ساخت و خرابه‌ها معمور خواهد شد.
\par 34 و زمین ویران که به نظر جمیع رهگذریان خراب می‌بود، شیار خواهد شد.
\par 35 و خواهند گفت این زمینی که ویران بود، مثل باغ عدن گردیده است و شهرهایی که خراب و ویران و منهدم بود، حصاردار ومسکون شده است.
\par 36 و امت هایی که به اطراف شما باقی‌مانده باشند، خواهند دانست که من یهوه مخروبات را بنا کرده و ویرانه‌ها را غرس نموده‌ام. من که یهوه هستم تکلم نموده و بعمل آورده‌ام.»
\par 37 خداوند یهوه چنین می‌گوید: «برای این باردیگر خاندان اسرائیل از من مسالت خواهند نمودتا آن را برای ایشان بعمل آورم. من ایشان را بامردمان مثل گله کثیر خواهم گردانید.مثل گله های قربانی یعنی گله اورشلیم در موسمهایش همچنان شهرهای مخروب از گله های مردمان پر خواهد شد و ایشان خواهند دانست که من یهوه هستم.»
\par 38 مثل گله های قربانی یعنی گله اورشلیم در موسمهایش همچنان شهرهای مخروب از گله های مردمان پر خواهد شد و ایشان خواهند دانست که من یهوه هستم.»

\chapter{37}

\par 1 دست خداوند بر من فرود آمده، مرا درروح خداوند بیرون برد و در همواری قرار داد و آن از استخوانها پر بود.
\par 2 و مرا به هرطرف آنها گردانید. و اینک آنها بر روی همواری بی‌نهایت زیاده و بسیار خشک بود.
\par 3 و او مراگفت: «ای پسر انسان آیا می‌شود که این استخوانها زنده گردد؟» گفتم: «ای خداوند یهوه تو می‌دانی.»
\par 4 پس مرا فرمود: «براین استخوانها نبوت نموده، به اینها بگو: ای استخوانهای خشک کلام خداوند را بشنوید!
\par 5 خداوند یهوه به این استخوانها چنین می‌گوید: اینک من روح به شمادرمی آورم تا زنده شوید.
\par 6 و پیه‌ها بر شما خواهم نهاد و گوشت بر شما خواهم آورد و شما را به پوست خواهم پوشانید و در شما روح خواهم نهاد تا زنده شوید. پس خواهید دانست که من یهوه هستم.»
\par 7 پس من چنانکه مامور شدم نبوت کردم. و چون نبوت نمودم، آوازی مسموع گردید. و اینک تزلزلی واقع شد و استخوانها به یکدیگر یعنی هر استخوانی به استخوانش نزدیک شد.
\par 8 و نگریستم و اینک پیه‌ها و گوشت به آنها برآمد و پوست آنها را از بالا پوشانید. امادر آنها روح نبود.
\par 9 پس او مرا گفت: «بر روح نبوت نما! ای پسر انسان بر روح نبوت کرده، بگو: خداوند یهوه چنین می‌فرماید که‌ای روح ازبادهای اربع بیا و به این کشتگان بدم تا ایشان زنده شوند.»
\par 10 پس چنانکه مرا امر فرمود، نبوت نمودم. وروح به آنها داخل شد و آنها زنده گشته، بر پایهای خود لشکر بی‌نهایت عظیمی ایستادند.
\par 11 و اومرا گفت: «ای پسر انسان این استخوانها تمامی خاندان اسرائیل می‌باشند. اینک ایشان می‌گویند: استخوانهای ما خشک شد و امید ما ضایع گردیدو خودمان منقطع گشتیم.
\par 12 لهذا نبوت کرده، به ایشان بگو: خداوند یهوه چنین می‌فرماید: اینک من قبرهای شما را می‌گشایم. و شما را‌ای قوم من از قبرهای شما درآورده، به زمین اسرائیل خواهم آورد.
\par 13 و‌ای قوم من چون قبرهای شما رابگشایم و شما را از قبرهای شما بیرون آورم، آنگاه خواهید دانست که من یهوه هستم.
\par 14 وروح خود را در شما خواهم نهاد تا زنده شوید. وشما را در زمین خودتان مقیم خواهم ساخت. پس خواهید دانست که من یهوه تکلم نموده و بعمل آورده‌ام. قول خداوند این است.»
\par 15 و کلام خداوند بر من نازل شده، گفت:
\par 16 «وتو‌ای پسر انسان یک عصا برای خود بگیر و بر آن بنویس برای یهودا و برای بنی‌اسرائیل رفقای وی. پس عصای دیگر بگیر و بر آن بنویس برای یوسف عصای افرایم و تمامی خاندان اسرائیل رفقای وی.
\par 17 و آنها را برای خودت با یکدیگریک عصا ساز تا در دستت یک باشد.
\par 18 و چون ابناء قومت تو را خطاب کرده، گویند: آیا ما راخبر نمی دهی که از این کارها مقصود تو چیست؟
\par 19 آنگاه به ایشان بگو: خداوند یهوه چنین می‌فرماید: اینک من عصای یوسف را که در دست افرایم است و اسباط اسرائیل را که رفقای وی‌اند، خواهم گرفت و آنها را با وی یعنی با عصای یهودا خواهم پیوست و آنها را یک عصا خواهم ساخت و در دستم یک خواهد شد.
\par 20 پس آن عصاها که بر آنها نوشتی در دست تو در نظر ایشان باشد.
\par 21 و به ایشان بگو: خداوند یهوه چنین می‌فرماید: اینک من بنی‌اسرائیل را از میان امت هایی که به آنها رفته‌اند گرفته، ایشان را از هرطرف جمع خواهم کرد و ایشان را به زمین خودشان خواهم آورد.
\par 22 و ایشان را در آن زمین بر کوههای اسرائیل یک امت خواهم ساخت. ویک پادشاه بر جمیع ایشان سلطنت خواهد نمودو دیگر دو امت نخواهند بود و دیگر به دومملکت تقسیم نخواهند شد.
\par 23 و خویشتن رادیگر به بتها و رجاسات و همه معصیت های خودنجس نخواهند ساخت. بلکه ایشان را از جمیع مساکن ایشان که در آنها گناه ورزیده‌اند نجات داده، ایشان را طاهر خواهم ساخت. و ایشان قوم من خواهند بود و من خدای ایشان خواهم بود.
\par 24 و بنده من داود، پادشاه ایشان خواهد بود. ویک شبان برای جمیع ایشان خواهد بود. و به احکام من سلوک نموده و فرایض مرا نگاه داشته، آنها را بجا خواهند‌آورد.
\par 25 و در زمینی که به بنده خود یعقوب دادم و پدران ایشان در آن ساکن می‌بودند، ساکن خواهند شد. و ایشان و پسران ایشان و پسران پسران ایشان تا به ابد در آن سکونت خواهند نمود و بنده من داود تا ابدالابادرئیس ایشان خواهد بود.
\par 26 و با ایشان عهدسلامتی خواهم بست که برای ایشان عهدجاودانی خواهد بود و ایشان را مقیم ساخته، خواهم افزود و مقدس خویش را تا ابدالاباد در میان ایشان قرار خواهم داد.
\par 27 و مسکن من برایشان خواهد بود و من خدای ایشان خواهم بود وایشان قوم من خواهند بود.پس چون مقدس من در میان ایشان تا به ابد بر قرار بوده باشد، آنگاه امت‌ها خواهند دانست که من یهوه هستم که اسرائیل را تقدیس می‌نمایم.»
\par 28 پس چون مقدس من در میان ایشان تا به ابد بر قرار بوده باشد، آنگاه امت‌ها خواهند دانست که من یهوه هستم که اسرائیل را تقدیس می‌نمایم.»

\chapter{38}

\par 1 و کلام خداوند بر من نازل شده، گفت:
\par 2 «ای پسر انسان نظر خود را بر جوج که از زمین ماجوج و رئیس روش و ماشک و توبال است بدار و بر او نبوت نما.
\par 3 و بگو خداوند یهوه چنین می‌فرماید: اینک من‌ای جوج رئیس روش و ماشک و توبال به ضد تو هستم.
\par 4 و تو را برگردانیده، قلاب خود را به چانه ات می‌گذارم و تورا با تمامی لشکرت بیرون می‌آورم. اسبان وسواران که جمیع ایشان با اسلحه تمام آراسته، جمعیت عظیمی با سپرها و مجنها و همگی ایشان شمشیرها به‌دست گرفته،
\par 5 فارس و کوش و فوط با ایشان و جمیع ایشان با سپر و خود،
\par 6 جومر و تمامی افواجش و خاندان توجرمه ازاطراف شمال با تمامی افواجش و قوم های بسیارهمراه تو.
\par 7 پس مستعد شو و تو و تمامی جمعیتت که نزد تو جمع شده‌اند، خویشتن رامهیا سازید و تو مستحفظ ایشان باش.
\par 8 بعد ازروزهای بسیار از تو تفقد خواهد شد. و درسالهای آخر به زمینی که از شمشیر استرداد شده است، خواهی آمد که آن از میان قوم های بسیار برکوههای اسرائیل که به خرابه دایمی تسلیم شده بود، جمع شده است و آن از میان قوم‌ها بیرون آورده شده و تمامی اهلش به امنیت ساکن می‌باشند.
\par 9 اما تو بر آن خواهی برآمد و مثل بادشدید داخل آن خواهی شد و مانند ابرها زمین راخواهی پوشانید. تو و جمیع افواجت و قوم های بسیار که همراه تو می‌باشند.»
\par 10 خداوند یهوه چنین می‌فرماید: «در آن روزچیزها در دل تو خطور خواهد کرد و تدبیری زشت خواهی نمود.
\par 11 و خواهی گفت: به زمین بی‌حصار برمی آیم. بر کسانی که به اطمینان وامنیت ساکنند می‌آیم که جمیع ایشان بی‌حصارندو پشت بندها و دروازه‌ها ندارند.
\par 12 تا تاراج نمایی و غنیمت را ببری و دست خود را به خرابه هایی که معمور شده است و به قومی که ازمیان امت‌ها جمع شده‌اند، بگردانی که ایشان مواشی و اموال اندوخته‌اند و در وسط جهان ساکنند.
\par 13 شبا و ددان و تجار ترشیش و جمیع شیران ژیان ایشان تو را خواهند گفت: آیا به جهت گرفتن غارت آمده‌ای؟ و آیا به جهت بردن غنیمت جمعیت خود را جمع کرده‌ای تا نقره وطلا برداری و مواشی و اموال را بربایی و غارت عظیمی ببری؟
\par 14 «بنابراین‌ای پسر انسان نبوت نموده، جوج را بگو که خداوند یهوه چنین می‌فرماید: در آن روز حینی که قوم من اسرائیل به امنیت ساکن باشند آیا تو نخواهی فهمید؟
\par 15 و از مکان خویش از اطراف شمال خواهی آمد تو وقوم های بسیار همراه تو که جمیع ایشان اسب‌سوار و جمعیتی عظیم و لشکری کثیر می‌باشند،
\par 16 و بر قوم من اسرائیل مثل ابری که زمین راپوشاند خواهی برآمد. در ایام بازپسین این به وقوع خواهد پیوست که تو را به زمین خودخواهم آورد تا آنکه امت‌ها حینی که من خویشتن را در تو‌ای جوج به نظر ایشان تقدیس کرده باشم مرا بشناسند.»
\par 17 خداوند یهوه چنین می‌گوید: «آیا توآنکس نیستی که در ایام سلف به واسطه بندگانم انبیای اسرائیل که در آن ایام درباره سالهای بسیارنبوت نمودند در خصوص تو گفتم که تو را برایشان خواهم آورد؟
\par 18 خداوند یهوه می‌گوید: در آن روز یعنی در روزی که جوج به زمین اسرائیل برمی آید، همانا حدت خشم من به بینی‌ام خواهد برآمد.
\par 19 زیرا در غیرت و آتش خشم خود گفته‌ام که هرآینه در آن روز تزلزل عظیمی در زمین اسرائیل خواهد شد.
\par 20 و ماهیان دریا و مرغان هوا و حیوانات صحرا و همه حشراتی که بر زمین می‌خزند و همه مردمانی که بر روی جهانند به حضور من خواهند لرزید وکوهها سرنگون خواهد شد و صخره‌ها خواهدافتاد و جمیع حصارهای زمین منهدم خواهدگردید.
\par 21 و خداوند یهوه می‌گوید: من شمشیری بر جمیع کوههای خود به ضد او خواهم خواند وشمشیر هر کس بر برادرش خواهد بود.
\par 22 و با وباو خون بر او عقوبت خواهم رسانید. و باران سیال و تگرگ سخت و آتش و گوگرد بر او و برافواجش و بر قوم های بسیاری که با وی می‌باشندخواهم بارانید.و خویشتن را در نظر امت های بسیار معظم و قدوس و معروف خواهم نمود وخواهند دانست که من یهوه هستم.
\par 23 و خویشتن را در نظر امت های بسیار معظم و قدوس و معروف خواهم نمود وخواهند دانست که من یهوه هستم.

\chapter{39}

\par 1 «پس تو‌ای پسر انسان درباره جوج نبوت کرده، بگو خداوند یهوه چنین می‌فرماید که اینک‌ای جوج رئیس روش وماشک و توبال من به ضد تو هستم.
\par 2 و تو رابرمی گردانم و رهبری می‌نمایم و تو را از اطراف شمال برآورده، بر کوههای اسرائیل خواهم آورد.
\par 3 و کمان تو را از دست چپت انداخته، تیرهای تو را از دست راستت خواهم افکند.
\par 4 وتو و همه افواجت و قوم هایی که همراه تو هستندبر کوههای اسرائیل خواهید افتاد و تو را به هرجنس مرغان شکاری و به حیوانات صحرا به جهت خوراک خواهم داد.
\par 5 خداوند یهوه می‌گوید که به روی صحرا خواهی افتاد زیرا که من تکلم نموده‌ام.
\par 6 و آتشی بر ماجوج و بر کسانی که در جزایر به امنیت ساکنند خواهم فرستاد تابدانند که من یهوه هستم.
\par 7 و نام قدوس خود را درمیان قوم خویش اسرائیل، معروف خواهم ساخت و دیگر نمی گذارم که اسم قدوس من بی‌حرمت شود تا امت‌ها بدانند که من یهوه قدوس اسرائیل می‌باشم.
\par 8 اینک خداوند یهوه می‌گوید: آن می‌آید و به وقوع خواهد پیوست. واین همان روز است که درباره‌اش تکلم نموده‌ام.
\par 9 و ساکنان شهرهای اسرائیل بیرون خواهند آمدو اسلحه یعنی مجن و سپر و کمان و تیرها وچوب دستی و نیزه‌ها را آتش زده، خواهندسوزانید. و مدت هفت سال آتش را به آنها زنده نگاه خواهند داشت.
\par 10 و هیزم از صحرا نخواهندآورد و چوب از جنگلها نخواهند برید زیرا که اسلحه‌ها را به آتش خواهند سوزانید. و خداوندیهوه می‌گوید که غارت کنندگان خود را غارت خواهند کرد و تاراج کنندگان خویش را تاراج خواهند نمود.
\par 11 و در آن روز موضعی برای قبردر اسرائیل یعنی وادی عابریم را بطرف مشرق دریا به جوج خواهم داد. و راه عبور کنندگان رامسدود خواهد ساخت. و در آنجا جوج و تمامی جمعیت، او را دفن خواهند کرد و آن را وادی هامون جوج خواهند نامید.
\par 12 و خاندان اسرائیل مدت هفت ماه ایشان را دفن خواهند کرد تا زمین را طاهر سازند.
\par 13 و تمامی اهل زمین ایشان رادفن خواهند کرد. و خداوند یهوه می‌گوید: روزتمجید من نیکنامی ایشان خواهد بود.
\par 14 وکسانی را معین خواهند کرد که پیوسته در زمین گردش نمایند. و همراه عبورکنندگان آنانی را که بر روی زمین باقی‌مانده باشند دفن کرده، آن راطاهر سازند. بعد از انقضای هفت ماه آنها راخواهند طلبید.
\par 15 و عبورکنندگان در زمین گردش خواهند کرد. و اگر کسی استخوان آدمی بیند نشانی نزد آن برپا کند تا دفن کنندگان آن را دروادی هامون جوج مدفون سازند.
\par 16 و اسم شهرنیز هامونه خواهد بود. پس زمین را طاهر خواهندساخت.
\par 17 «و اما تو‌ای پسر انسان! خداوند یهوه چنین می‌فرماید که بهر جنس مرغان و به همه حیوانات صحرا بگو: جمع شوید و بیایید و نزد قربانی من که آن را برای شما ذبح می‌نمایم، فراهم آیید. قربانی عظیمی بر کوههای اسرائیل تا گوشت بخورید و خون بنوشید.
\par 18 گوشت جباران راخواهید خورد و خون روسای جهان را خواهیدنوشید. از قوچها و بره‌ها و بزها و گاوها که همه آنها از پرواریهای باشان می‌باشند.
\par 19 و از قربانی من که برای شما ذبح می‌نمایم، پیه خواهید خوردتا سیر شوید و خون خواهید نوشید تا مست شوید.
\par 20 و خداوند یهوه می‌گوید که بر سفره من از اسبان و سواران و جباران و همه مردان جنگی سیر خواهید شد.
\par 21 و من جلال خود را در میان امت‌ها قرار خواهم داد و جمیع امت‌ها داوری مرا که آن را اجرا خواهم داشت و دست مرا که برایشان فرود خواهم آورد، مشاهده خواهند نمود.
\par 22 و خاندان اسرائیل از آن روز و بعد خواهنددانست که یهوه خدای ایشان من هستم.
\par 23 وامت‌ها خواهند دانست که خاندان اسرائیل به‌سبب گناه خودشان جلای وطن گردیدند. زیراچونکه به من خیانت ورزیدند، من روی خود را ازایشان پوشانیدم و ایشان را به‌دست ستم کاران ایشان تسلیم نمودم که جمیع ایشان به شمشیرافتادند.
\par 24 برحسب نجاسات و تقصیرات ایشان به ایشان عمل نموده، روی خود را از ایشان پوشانیدم.»
\par 25 بنابراین خداوند یهوه چنین می‌گوید: «الان اسیران یعقوب را باز آورده، بر تمامی خاندان اسرائیل رحمت خواهم فرمود و بر اسم قدوس خود غیرت خواهم نمود.
\par 26 و حینی که ایشان درزمین خود به امنیت ساکن شوند و ترساننده‌ای نباشد، آنگاه خجالت خود را و خیانتی را که به من ورزیده‌اند متحمل خواهند شد.
\par 27 و چون ایشان را از میان امت‌ها برگردانم و ایشان را از زمین دشمنانشان جمع نمایم، آنگاه در نظر امت های بسیار در ایشان تقدیس خواهم شد.
\par 28 و خواهنددانست که من یهوه خدای ایشان هستم، از آن روکه من ایشان را در میان امت‌ها جلای وطن ساختم و ایشان را به زمین خودشان جمع کردم و بار دیگرکسی را از ایشان در آنجا باقی نخواهم گذاشت.و خداوند یهوه می‌گوید که من بار دیگرروی خود را از ایشان نخواهم پوشانید زیراکه روح خویش را بر خاندان اسرائیل خواهم ریخت.»
\par 29 و خداوند یهوه می‌گوید که من بار دیگرروی خود را از ایشان نخواهم پوشانید زیراکه روح خویش را بر خاندان اسرائیل خواهم ریخت.»

\chapter{40}

\par 1 در سال بیست و پنجم اسیری ما درابتدای سال، در دهم ماه که سال چهاردهم بعد از تسخیر شهر بوده، در همان روزدست خداوند بر من نازل شده، مرا به آنجا برد.
\par 2 در رویاهای خدا مرا به زمین اسرائیل آورد. ومرا بر کوه بسیار بلند قرار داد که بطرف جنوب آن مثل بنای شهر بود.
\par 3 و چون مرا به آنجا آورد، اینک مردی که نمایش او مثل نمایش برنج بود ودر دستش ریسمانی از کتان و نی برای پیمودن بود و نزد دروازه ایستاده بود.
\par 4 آن مرد مرا گفت: «ای پسر انسان به چشمان خود ببین و به گوشهای خویش بشنو و دل خود را به هرچه به تو نشان دهم، مشغول ساز زیرا که تو را در اینجا آوردم تااین چیزها را به تو نشان دهم. پس خاندان اسرائیل را از هر‌چه می‌بینی آگاه ساز.»
\par 5 و اینک حصاری بیرون خانه گرداگردش بود. و به‌دست آن مرد نی پیمایش شش ذراعی بود که هر ذراعش یک ذراع و یک قبضه بود. پس عرض بنا را یک نی و بلندیش را یک نی پیمود.
\par 6 پس نزددروازه‌ای که بسوی مشرق متوجه بود آمده، به پله هایش برآمد. و آستانه دروازه را پیمود که عرضش یک نی بود و عرض آستانه دیگر را که یک نی بود.
\par 7 و طول هر غرفه یک نی بود وعرضش یک نی. و میان غرفه‌ها مسافت پنج ذراع. و آستانه دروازه نزد رواق دروازه از طرف اندرون یک نی بود.
\par 8 و رواق دروازه را از طرف اندرون یک نی پیمود.
\par 9 پس رواق دروازه را هشت ذراع واسبرهایش را دو ذراع پیمود. و رواق دروازه بطرف اندرون بود.
\par 10 و حجره های دروازه بطرف شرقی، سه از اینطرف و سه از آنطرف بود. و هرسه را یک پیمایش و اسبرها را از اینطرف وآنطرف یک پیمایش بود.
\par 11 و عرض دهنه دروازه را ده ذراع و طول دروازه را سیزده ذراع پیمود.
\par 12 و محجری پیش روی حجره‌ها ازاینطرف یک ذراع و محجری از آنطرف یک ذراع و حجره‌ها از این طرف شش ذراع و از آنطرف شش ذراع بود.
\par 13 و عرض دروازه را از سقف یک حجره تا سقف دیگری بیست و پنج ذراع پیمود. و دروازه در مقابل دروازه بود.
\par 14 و اسبرهاراشصت ذراع ساخت و رواق گرداگرد دروازه به اسبرها رسید.
\par 15 و پیش دروازه مدخل تا پیش رواق دروازه اندرونی پنجاه ذراع بود.
\par 16 وحجره‌ها و اسبرهای آنها را به اندرون دروازه پنجره های مشبک بهر طرف بود و همچنین رواقها را. و پنجره‌ها بطرف اندرون گرداگرد بود وبر اسبرها نخلها بود.
\par 17 پس مرا به صحن بیرونی آورد و اینک اطاقها و سنگ فرشی که برای صحن از هر طرفش ساخته شده بود. و سی اطاق بر آن سنگ فرش بود.
\par 18 و سنگ فرش یعنی سنگ فرش پائینی به‌جانب دروازه‌ها یعنی به اندازه طول دروازه‌ها بود.
\par 19 و عرضش را از برابر دروازه پایینی تا پیش صحن اندرونی از طرف بیرون صد ذراع به سمت مشرق و سمت شمال پیمود.
\par 20 و طول و عرض دروازه‌ای را که رویش بطرف شمال صحن بیرونی بود پیمود.
\par 21 و حجره هایش سه ازاینطرف و سه از آنطرف و اسبرهایش ورواقهایش موافق پیمایش دروازه اول بود. طولش پنجاه ذراع و عرضش بیست و پنج ذراع.
\par 22 و پنجره هایش و رواقهایش و نخلهایش موافق پیمایش دروازه‌ای که رویش به سمت مشرق است بود. و به هفت پله به آن برمی آمدند ورواقهایش پیش روی آنها بود.
\par 23 و صحن اندرونی را دروازه‌ای در مقابل دروازه دیگربطرف شمال و بطرف مشرق بود. و از دروازه تادروازه صد ذراع پیمود.
\par 24 پس مرا بطرف جنوب برد. و اینک دروازه‌ای به سمت جنوب و اسبرهایش ورواقهایش را مثل این پیمایشها پیمود.
\par 25 و برای آن و برای رواقهایش پنجره‌ها مثل آن پنجره هاگرداگردش بود. و طولش پنجاه ذراع و عرضش بیست و پنج ذراع بود.
\par 26 وزینه های آن هفت پله داشت. و رواقش پیش آنها بود. و آن را نخلهایکی از اینطرف و دیگری از آنطرف براسبرهایش بود.
\par 27 و صحن اندرونی بطرف جنوب دروازه‌ای داشت و از دروازه تا دروازه به سمت جنوب صد ذراع پیمود.
\par 28 و مرا از دروازه جنوبی به صحن اندرونی آورد. و دروازه جنوبی را مثل این پیمایشها پیمود.
\par 29 و حجره هایش واسبرهایش و رواقهایش موافق این پیمایشها بود. و در آن و در رواقهایش پنجره‌ها گرداگردش بودو طولش پنجاه ذراع و عرضش بیست و پنج ذراع بود.
\par 30 و طول رواقی که گرداگردش بود بیست وپنج ذراع و عرضش پنج ذراع بود.
\par 31 و رواقش به صحن بیرونی می‌رسید. و نخلها بر اسبرهایش بودو زینه‌اش هشت پله داشت.
\par 32 پس مرا به صحن اندرونی به سمت مشرق آورد. و دروازه را مثل این پیمایشها پیمود.
\par 33 وحجره هایش و اسبرهایش و رواقهایش موافق این پیمایشها بود. و درآن و در رواقهایش پنجره‌ها برهر طرفش بود و طولش پنجاه ذراع و عرضش بیست و پنج ذراع بود.
\par 34 و رواقهایش بسوی صحن بیرونی و نخلها بر اسبرهایش از این طرف وآنطرف بود و زینه‌اش هفت پله داشت.
\par 35 و مرا به دروازه شمالی آورد و آن را مثل این پیمایشهاپیمود.
\par 36 و حجره هایش و اسبرهایش ورواقهایش را نیز. و پنجره‌ها گرداگردش بود وطولش پنجاه ذراع و عرضش بیست و پنج ذراع بود.
\par 37 و اسبرهایش بسوی صحن بیرونی بود. ونخلها بر اسبرهایش از اینطرف و از آنطرف بود وزینه‌اش هشت پله داشت.
\par 38 و نزد اسبرهای دروازه‌ها اطاقی بادروازه‌اش بود که در آن قربانی های سوختنی رامی شستند.
\par 39 و در رواق دروازه دو میز ازاینطرف و دو میز از آن طرف بود تا بر آنهاقربانی های سوختنی و قربانی های گناه وقربانی های جرم را ذبح نمایند.
\par 40 و به یک جانب از طرف بیرون نزد زینه دهنه دروازه شمالی دومیز بود. و به‌جانب دیگر که نزد رواق دروازه بوددو میز بود.
\par 41 چهار میز از اینطرف و چهار میز ازآنطرف به پهلوی دروازه بود یعنی هشت میز که بر آنها ذبح می‌کردند.
\par 42 و چهار میز برای قربانی های سوختنی از سنگ تراشیده بود که طول هر یک یک ذراع و نیم و عرضش یک ذراع ونیم و بلندیش یک ذراع بود و بر آنها آلاتی را که به آنها قربانی های سوختنی و ذبایح را ذبح می‌نمودند، می‌نهادند.
\par 43 و کناره های یک قبضه قد در اندرون از هر طرف نصب بود و گوشت قربانی‌ها بر میزها بود.
\par 44 و بیرون دروازه اندرونی، اطاقهای مغنیان در صحن اندرونی به پهلوی دروازه شمالی بود وروی آنها به سمت جنوب بود و یکی به پهلوی دروازه مشرقی که رویش بطرف شمال می‌بودبود.
\par 45 و او مرا گفت: «این اطاقی که رویش به سمت جنوب است، برای کاهنانی که ودیعت خانه را نگاه می‌دارند می‌باشد.
\par 46 و اطاقی که رویش به سمت شمال است، برای کاهنانی که ودیعت مذبح را نگاه می‌دارند می‌باشد. اینانندپسران صادوق از بنی لاوی که نزدیک خداوندمی آیند تا او را خدمت نمایند.»
\par 47 و طول صحن را صد ذراع پیمود و عرضش را صد ذراع و آن مربع بود و مذبح در برابر خانه بود.
\par 48 و مرا به رواق خانه آورد. و اسبرهای رواق را پنج ذراع از اینطرف و پنج ذراع از آنطرف پیمود. و عرض دروازه را سه ذراع از اینطرف وسه ذراع از آنطرف.و طول رواق بیست ذراع وعرضش یازده ذراع. و نزد زینه‌اش که از آن برمی آمدند، دو ستون نزد اسبرها یکی از اینطرف و دیگری از آنطرف بود.
\par 49 و طول رواق بیست ذراع وعرضش یازده ذراع. و نزد زینه‌اش که از آن برمی آمدند، دو ستون نزد اسبرها یکی از اینطرف و دیگری از آنطرف بود.

\chapter{41}

\par 1 و مرا به هیکل آورد و عرض اسبرها راشش ذراع از اینطرف و عرض آنها راشش ذراع از آنطرف که عرض خیمه بود پیمود.
\par 2 و عرض مدخل ده ذراع بود و جانبهای مدخل از اینطرف پنج ذراع و از آنطرف پنج ذراع بود وطولش را چهل ذراع و عرضش را بیست ذراع پیمود.
\par 3 و به اندرون داخل شده، اسبرهای مدخل را دو ذراع و مدخل را شش ذراع و عرض مدخل را هفت ذراع پیمود.
\par 4 و طولش را بیست ذراع و عرضش را بیست ذراع پیش روی هیکل پیمود و مرا گفت: «این قدس‌الاقداس است.»
\par 5 و دیوار خانه را شش ذراع پیمود. و عرض غرفه‌ها که گرداگرد خانه بهر طرف می‌بود چهارذراع بود.
\par 6 و غرفه‌ها روی همدیگر سه طبقه بودو در هر رسته‌ای سی و در دیواری که به جهت غرفه‌ها گرداگرد خانه بود، داخل می‌شد تا (درآن ) متمکن شود و در دیوار خانه متمکن نشود.
\par 7 و غرفه‌ها خانه را بالاتر و بالاتر احاطه کرده، وسیعتر می‌شد، زیرا که خانه را بالاتر و بالاترگرداگرد خانه احاطه می‌کرد و از این جهت خانه بسوی بالا وسیعتر می‌بود، و همچنین از طبقه تحتانی به طبقه وسطی تا طبقه فوقانی بالامی رفتند.
\par 8 و بلندی خانه را از هر طرف ملاحظه نمودم و اساس های غرفه‌ها یک نی تمام، یعنی شش ذراع بزرگ بود.
\par 9 و بطرف بیرون عرض دیواری که به جهت غرفه‌ها بود پنج ذراع بود و فسحت باقی‌مانده مکان غرفه های خانه بود.
\par 10 و در میان حجره‌ها، عرض بیست ذراعی گرداگرد خانه بهرطرفش بود.
\par 11 و درهای غرفه هابسوی فسحت بود. یک در بسوی شمال و در دیگر به سوی جنوب و عرض مکان فسحت پنج ذراع گرداگرد.
\par 12 و عرض بنیانی که رو‌به‌روی مکان منفصل بوددر گوشه سمت مغرب هفتاد ذراع و عرض دیوارگرداگرد بنیان پنج ذراع و طولش نود ذراع بود.
\par 13 و طول خانه را صد ذراع پیمود و طول مکان منفصل و بنیان و دیوارهایش را صد ذراع.
\par 14 و عرض جلو خانه و مکان منفصل به سمت مشرق صد ذراع بود.
\par 15 و طول بنیان را تا پیش مکان منفصل که در عقبش بود با ایوانهایش ازاینطرف و آنطرف صد ذراع پیمود و هیکل اندرونی و رواقهای صحنها را.
\par 16 و آستانه‌ها وپنجره های مشبک و ایوانها گرداگرد در سه طبقه مقابل آستانه از زمین تا پنجره‌ها از هر طرف چوب پوش بود و پنجره‌ها هم پوشیده بود.
\par 17 تابالای درها و تا خانه اندرونی و بیرونی و بر تمامی دیوار گرداگرد از اندرون و بیرون به همین پیمایشها.
\par 18 و کروبیان و نخلها در آن ساخته شده بود ودر میان هر دو کروبی یک نخل بود و هرکروبی دو رو داشت.
\par 19 یعنی روی انسان بسوی نخل از اینطرف و روی شیر بسوی نخل ازآنطرف بر تمامی خانه بهر طرفش ساخته شده بود.
\par 20 و از زمین تا بالای درها کروبیان و نخلهامصور بود و بر دیوار هیکل هم چنین.
\par 21 و باهوهای هیکل مربع بود و منظر جلوقدس مثل منظر آن بود.
\par 22 و مذبح چوبین بود. بلندی‌اش سه ذراع و طولش دو ذراع وگوشه هایش و طولش و دیوارهایش از چوب بود. و او مرا گفت: «میزی که در حضور خداوندمی باشد این است.»
\par 23 و هیکل و قدس را دو در بود.
\par 24 و هر در رادو لنگه بود و این دو لنگه تا می‌شد. یک در را دولنگه و در دیگر را دو لنگه.
\par 25 و بر آنها یعنی بردرهای هیکل کروبیان و نخلها مصور بود بطوری که در دیوارها مصور بود و آستانه چوبین پیش روی رواق بطرف بیرون بود.بر جانب رواق پنجره های مشبک به اینطرف و به آنطرف بود وهمچنین بر غرفه های خانه و بر آستانه‌ها.
\par 26 بر جانب رواق پنجره های مشبک به اینطرف و به آنطرف بود وهمچنین بر غرفه های خانه و بر آستانه‌ها.

\chapter{42}

\par 1 و مرا به صحن بیرونی از راه سمت شمالی بیرون برد و مرا به حجره‌ای که مقابل مکان منفصل و روبروی بنیان بطرف شمال بود آورد.
\par 2 جلو طول صد ذراعی درشمالی بود و عرضش پنجاه ذراع بود.
\par 3 مقابل بیست ذراع که از آن صحن اندرونی بود و مقابل سنگفرشی که از صحن بیرونی بود دهلیزی روبروی دهلیزی در سه طبقه بود.
\par 4 و پیش روی حجره‌ها بطرف اندرون خرندی به عرض ده ذراع بود و راهی یک ذراع و درهای آنها بطرف شمال بود.
\par 5 و حجره های فوقانی کوتاه بود زیرا که دهلیزها از آنها می‌گرفتند بیشتر از آنچه آنها ازحجره های تحتانی و وسطی بنیان می‌گرفتند.
\par 6 چونکه سه طبقه بود و ستونها مثل ستونهای صحن‌ها نداشت و از این سبب، طبقه فوقانی ازطبقات تحتانی و وسطی از زمین تنگتر می‌شد.
\par 7 وطول دیواری که بطرف بیرون مقابل حجره هابسوی صحن بیرونی روبروی حجره‌ها بود پنجاه ذراع بود.
\par 8 زیرا طول حجره هایی که در صحن بیرونی بود پنجاه ذراع بود و اینک جلو هیکل صدذراع بود.
\par 9 و زیر این حجره‌ها از طرف شرقی مدخلی بود که از آن به آنها از صحن بیرونی داخل می‌شدند.
\par 10 و در حجم دیوار صحن که بطرف مشرق بود پیش روی مکان منفصل و مقابل بنیان حجره‌ها بود.
\par 11 و راه مقابل آنها مثل نمایش راه حجره های سمت شمال بود، عرض آنها مطابق طول آنها بود و تمامی مخرج های اینها مثل رسم آنها و درهای آنها.
\par 12 و مثل درهای حجره های سمت جنوب دری بر سر راه بود یعنی بر راهی که راست پیش روی دیوار مشرقی بود جایی که به آنها داخل می‌شدند.
\par 13 و مرا گفت: «حجره های شمالی وحجره های جنوبی که پیش روی مکان منفصل است، حجره های مقدس می‌باشد که کاهنانی که به خداوند نزدیک می‌آیند قدس اقداس را درآنها می‌خورند و قدس اقداس و هدایای آردی وقربانی های گناه و قربانی های جرم را در آنهامی گذارند زیرا که این مکان مقدس است.
\par 14 وچون کاهنان داخل آنها می‌شوند دیگر از قدس به صحن بیرونی بیرون نمی آیند بلکه لباسهای خودرا که در آنها خدمت می‌کنند در آنها می‌گذارندزیرا که آنها مقدس می‌باشد و لباس دیگرپوشیده، به آنچه به قوم تعلق دارد نزدیک می‌آیند.»
\par 15 و چون پیمایشهای خانه اندرونی را به اتمام رسانید، مرا بسوی دروازه‌ای که رویش به سمت مشرق بود بیرون آورد و آن را از هر طرف پیمود.
\par 16 جانب شرقی آن را به نی پیمایش، پانصد نی پیمود یعنی به نی پیمایش آن را از هرطرف (پیمود).
\par 17 و جانب شمالی را به نی پیمایش از هر طرف پانصد نی پیمود.
\par 18 و جانب جنوبی را به نی پیمایش، پانصد نی پیمود.
\par 19 پس به سوی جانب غربی برگشته، آن را به نی پیمایش پانصد نی پیمود.هر‌چهار جانب آن را پیمود وآن را دیواری بود که طولش پانصد و عرضش پانصد (نی ) بود تا در میان مقدس و غیرمقدس فرق گذارد.
\par 20 هر‌چهار جانب آن را پیمود وآن را دیواری بود که طولش پانصد و عرضش پانصد (نی ) بود تا در میان مقدس و غیرمقدس فرق گذارد.

\chapter{43}

\par 1 و مرا نزد دروازه آورد، یعنی به دروازه‌ای که به سمت مشرق متوجه بود.
\par 2 واینک جلال خدای اسرائیل از طرف مشرق آمد و آواز او مثل آبهای بسیار بود و زمین از جلال او منور گردید.
\par 3 و مثل منظر آن رویایی بود که دیده بودم یعنی مثل آن رویا که در وقت آمدن من، برای تخریب شهر دیده بودم و رویاهامثل آن رویا بود که نزد نهر خابور مشاهده نموده بودم. پس به روی خود در‌افتادم.
\par 4 پس جلال خداوند از راه دروازه‌ای که رویش به سمت مشرق بود به خانه درآمد.
\par 5 وروح مرا برداشته، به صحن اندرونی آورد و اینک جلال خداوند خانه را مملو ساخت.
\par 6 و هاتفی راشنیدم که از میان خانه به من تکلم می‌نماید ومردی پهلوی من ایستاده بود.
\par 7 و مرا گفت: «ای پسر انسان این است مکان کرسی من و مکان کف پایهایم که در آن در میان بنی‌اسرائیل تا به ابدساکن خواهم شد و خاندان اسرائیل هم خودایشان و هم پادشاهان ایشان بار دیگر به زناها ولاشهای پادشاهان خود در مکان های بلند خویش نام قدوس مرا بی‌حرمت نخواهند ساخت.
\par 8 ازاینکه آستانه های خود را نزد آستانه من وباهوهای خویش را به پهلوی باهوهای من برپاکرده‌اند و در میان من و ایشان فقط دیواری است، پس اسم قدوس مرا به رجاسات خویش که آنهارا بعمل آورده‌اند بی‌حرمت ساخته‌اند، لهذا من در خشم خود ایشان را تلف نموده‌ام.
\par 9 حال زناهای خود و لاشهای پادشاهان خویش را از من دور بنمایند و من در میان ایشان تا به ابد سکونت خواهم نمود.
\par 10 و تو‌ای پسر انسان خاندان اسرائیل را از این خانه مطلع ساز تا از گناهان خودخجل شوند و ایشان نمونه آن را بپیمایند.
\par 11 واگر از هر‌چه بعمل آورده‌اند خجل شوند، آنگاه صورت خانه را و نمونه و مخرجها و مدخلها و تمامی شکلها و همه فرایض و جمیع صورتها وتمامی قانونهایش را برای ایشان اعلام نما و به نظر ایشان بنویس تا تمامی صورتش و همه فرایضش را نگاه داشته، به آنها عمل نمایند.
\par 12 وقانون خانه این است که تمامی حدودش بر سرکوه از همه اطرافش قدس اقداس باشد. اینک قانون خانه همین است.
\par 13 و پیمایشهای مذبح به ذراعها که هر ذراع یک ذراع و یک قبضه باشد این است. سینه‌اش یک ذراع و عرضش یک ذراع و حاشیه‌ای که گرداگرد لبش می‌باشد یک وجب و این پشت مذبح می‌باشد.
\par 14 و از سینه روی زمین تا خروج پایینی دو ذراع و عرضش یک ذراع و از خروج کوچک تا خروج بزرگ چهار ذراع و عرضش یک ذراع.
\par 15 و آتش دانش چهار ذراع و از آتش دان چهار شاخ برآمده بود.
\par 16 و طول آتش دان دوازده و عرضش دوازده و از هر‌چهار طرف مربع بود.
\par 17 و طول خروج چهارده و عرضش چهارده برچهار طرفش بود و حاشیه‌ای که گرداگردش بودنیم ذراع و دایره سینه‌اش یک ذراع و پله هایش به سمت مشرق متوجه بود.
\par 18 و او مرا گفت: «ای پسر انسان خداوند یهوه چنین می‌فرماید: این است قانون های مذبح درروزی که آن را بسازند تا قربانی های سوختی برآن بگذرانند و خون بر آن بپاشند.
\par 19 و خداوندیهوه می‌فرماید که به لاویان کهنه که از ذریت صادوق می‌باشند و به جهت خدمت من به من نزدیک می‌آیند یک گوساله به جهت قربانی گناه بده.
\par 20 و از خونش گرفته، بر چهار شاخش و برچهار گوشه خروج و بر حاشیه‌ای که گرداگردش است بپاش و آن را طاهر ساخته، برایش کفاره کن.
\par 21 گوساله قربانی گناه را بگیر و آن را در مکان معین خانه بیرون از مقدس بسوزانند.
\par 22 و در روزدوم بز نر بی‌عیبی برای قربانی گناه بگذران تامذبح را به آن طاهر سازند چنانکه آن را به گوساله طاهر ساختند.
\par 23 و چون از طاهر ساختن آن فارغ شدی گوساله بی‌عیب و قوچی بی‌عیب از گله بگذران.
\par 24 تو آن را به حضور خداوند نزدیک بیاور و کاهنان نمک بر آنها پاشیده، آنها را به جهت قربانی سوختنی برای خداوند بگذرانند.
\par 25 هر روز از هفت روز تو بز نری برای قربانی گناه بگذران و ایشان گوساله‌ای و قوچی از گله هر دوبی عیب بگذرانند.
\par 26 هفت روز ایشان کفاره برای مذبح نموده، آن را طاهر سازند و تخصیص کنند.و چون این‌روزها را به اتمام رسانیدند، پس درروز هشتم و بعد از آن کاهنان قربانی های سوختنی و ذبایح سلامتی شما را بر مذبح بگذرانند و من شما را قبول خواهم کرد. قول خداوند یهوه این است.»
\par 27 و چون این‌روزها را به اتمام رسانیدند، پس درروز هشتم و بعد از آن کاهنان قربانی های سوختنی و ذبایح سلامتی شما را بر مذبح بگذرانند و من شما را قبول خواهم کرد. قول خداوند یهوه این است.»

\chapter{44}

\par 1 و مرا به راه دروازه مقدس بیرونی که به سمت مشرق متوجه بود، باز آورد و آن بسته شده بود.
\par 2 و خداوند مرا گفت: «این دروازه بسته بماند و گشوده نشود وهیچ‌کس از آن داخل نشود زیرا که یهوه خدای اسرائیل از آن داخل شده، لهذا بسته بماند.
\par 3 و اما رئیس، چونکه اورئیس است در آن به جهت خوردن غذا به حضور خداوند بنشیند و از راه رواق دروازه داخل شودو از همان راه بیرون رود.»
\par 4 پس مرا از راه دروازه شمالی پیش روی خانه آورد و نگریستم و اینک جلال خداوند خانه خداوند را مملو ساخته بود و بروی خوددرافتادم.
\par 5 و خداوند مرا گفت: «ای پسر انسان دل خود را به هرچه تو را گویم درباره تمامی قانون های خانه خداوند و همه قواعدش مشغول ساز و به چشمان خود ببین و به گوشهای خودبشنو و دل خویش را به مدخل خانه و به همه مخرج های مقدس مشغول ساز.
\par 6 و به این متمردین یعنی به خاندان اسرائیل بگو: خداوندیهوه چنین می‌فرماید: ای خاندان اسرائیل ازتمامی رجاسات خویش باز ایستید.
\par 7 زیرا که شما اجنبیان نامختون دل و نامختون گوشت راداخل ساختید تا در مقدس من بوده، خانه مراملوث سازند. و چون شما غذای من یعنی پیه وخون را گذرانیدید، ایشان علاوه بر همه رجاسات شما عهد مرا شکستند.
\par 8 و شما ودیعت اقداس مرا نگاه نداشتید، بلکه کسان به جهت خویشتن تعیین نمودید تا ودیعت مرا در مقدس من نگاه دارند.
\par 9 «خداوند یهوه چنین می‌فرماید: هیچ شخص غریب نامختون دل و نامختون گوشت از همه غریبانی که در میان بنی‌اسرائیل باشند به مقدس من داخل نخواهد شد.
\par 10 بلکه آن لاویان نیز که در حین آواره شدن بنی‌اسرائیل از من دوری ورزیده، از عقب بتهای خویش آواره گردیدند، متحمل گناه خود خواهند شد،
\par 11 زیرا خادمان مقدس من و مستحفظان دروازه های خانه و ملازمان خانه هستند و ایشان قربانی های سوختنی و ذبایح قوم را ذبح می‌نمایند و به حضور ایشان برای خدمت ایشان می‌ایستند.
\par 12 واز این جهت که به حضور بتهای خویش ایشان راخدمت نمودند و برای خاندان اسرائیل سنگ مصادم گناه شدند. بنابراین خداوند یهوه می‌گوید: دست خود را به ضد ایشان برافراشتم که متحمل گناه خود خواهند شد.
\par 13 و به من نزدیک نخواهند آمد و به کهانت من نخواهند پرداخت وبه هیچ‌چیز مقدس در قدس‌الاقداس نزدیک نخواهند آمد، بلکه خجالت خویش و رجاسات خود را که بعمل آوردند متحمل خواهند شد.
\par 14 لیکن ایشان را به جهت تمامی خدمت خانه وبرای هر کاری که در آن کرده می‌شود، مستحفظان ودیعت آن خواهم ساخت.
\par 15 «لیکن لاویان کهنه از بنی صادوق که درحینی که بنی‌اسرائیل از من آواره شدند ودیعت مقدس مرا نگاه داشتند، خداوند یهوه می‌گوید که ایشان به جهت خدمت من نزدیک خواهند آمد وبه حضور من ایستاده پیه و خون را برای من خواهند گذرانید.
\par 16 و ایشان به مقدس من داخل خواهند شد و به جهت خدمت من به خوان من نزدیک خواهند آمد و ودیعت مرا نگاه خواهندداشت.
\par 17 و هنگامی که به دروازه های صحن اندرونی داخل شوند لباس کتانی خواهند پوشیدو چون در دروازه های صحن اندرونی و در خانه مشغول خدمت باشند، هیچ لباس پشمین نپوشند.
\par 18 عمامه های کتانی بر سر ایشان وزیرجامه کتانی بر کمرهای ایشان باشد و هیچ چیزی که عرق آورد در بر نکنند.
\par 19 و چون به صحن بیرونی یعنی به صحن بیرونی نزد قوم بیرون روند، آنگاه لباس خویش را که در آن خدمت می‌کنند بیرون کرده، آن را در حجره های مقدس بگذارند و به لباس دیگر ملبس شوند وقوم را در لباس خویش تقدیس ننمایند.
\par 20 وایشان سر خود را نتراشند و گیسوهای بلندنگذارند بلکه موی سر خود را بچینند.
\par 21 و کاهن وقت درآمدنش در صحن اندرونی شراب ننوشد.
\par 22 و زن بیوه یا مطلقه را به زنی نگیرند، بلکه باکره‌ای که از ذریت خاندان اسرائیل باشد یابیوه‌ای را که بیوه کاهن باشد بگیرند.
\par 23 و فرق میان مقدس و غیرمقدس را به قوم من تعلیم دهندو تشخیص میان طاهر و غیرطاهر را به ایشان اعلام نمایند.
\par 24 و چون در مرافعه‌ها به جهت محاکمه بایستند، بر‌حسب احکام من داوری بنمایند و شرایع و فرایض مرا در جمیع مواسم من نگاه دارند و سبت های مرا تقدیس نمایند.
\par 25 واحدی از ایشان به میته آدمی نزدیک نیامده، خویشتن را نجس نسازد مگر اینکه به جهت پدریا مادر یا پسر یا دختر یا برادر یا خواهری که شوهر نداشته باشد، جایز است که خویشتن رانجس سازد.
\par 26 و بعد از آنکه طاهر شود هفت روز برای وی بشمارند.
\par 27 و خداوند یهوه می‌فرماید در روزی که به صحن اندرونی قدس داخل شود تا در قدس خدمت نماید آنگاه قربانی گناه خود را بگذراند.
\par 28 «و ایشان را نصیبی خواهد بود. من نصیب ایشان خواهم بود. پس ایشان را در میان اسرائیل ملک ندهید زیرا که من ملک ایشان خواهم بود.
\par 29 و ایشان هدایای آردی و قربانی های گناه وقربانی های جرم را بخورند و همه موقوفات اسرائیل از آن ایشان خواهد بود.
\par 30 و اول تمامی نوبرهای همه‌چیز و هر هدیه‌ای از همه‌چیزها ازجمیع هدایای شما از آن کاهنان خواهد بود وخمیر اول خود را به کاهن بدهید تا برکت بر خانه خود فرود آورید.و کاهن هیچ میته یا دریده شده‌ای را از مرغ یا بهایم نخورد.
\par 31 و کاهن هیچ میته یا دریده شده‌ای را از مرغ یا بهایم نخورد.

\chapter{45}

\par 1 «و چون زمین را به جهت ملکیت به قرعه تقسیم نمایید، حصه مقدس را که طولش بیست و پنج هزار (نی ) و عرضش ده هزار(نی ) باشد هدیه‌ای برای خداوند بگذرانید و این به تمامی حدودش از هر طرف مقدس خواهدبود.
\par 2 و از این پانصد در پانصد (نی ) از هر طرف مربع برای قدس خواهد بود و نواحی آن از هرطرفش پنجاه ذراع.
\par 3 و از این پیمایش طول بیست و پنج هزار و عرض ده هزار (نی ) خواهی پیمود تا در آن جای مقدس قدس‌الاقداس باشد.
\par 4 و این برای کاهنانی که خادمان مقدس باشند و به جهت خدمت خداوند نزدیک می‌آیند، حصه مقدس از زمین خواهد بود تا جای خانه‌ها به جهت ایشان و جای مقدس به جهت قدس باشد.
\par 5 و طول بیست و پنج هزار و عرض ده هزار (نی )به جهت لاویانی که خادمان خانه باشند خواهدبود تا ملک ایشان برای بیست خانه باشد.
\par 6 و ملک شهر را که عرضش پنجهزار و طولش بیست وپنجهزار (نی ) باشد موازی آن هدیه مقدس قرار خواهید داد و این از آن تمامی خاندان اسرائیل خواهد بود.
\par 7 و از اینطرف و از آنطرف هدیه مقدس و ملک شهر مقابل هدیه مقدس و مقابل ملک شهر از جانب غربی به سمت مغرب و ازجانب شرقی به سمت مشرق حصه رئیس خواهدبود و طولش موازی یکی از قسمت‌ها از حدمغرب تا حد مشرق خواهد بود.
\par 8 و این در آن زمین در اسرائیل ملک او خواهد بود تا روسای من بر قوم من دیگر ستم ننمایند و ایشان زمین را به خاندان اسرائیل بر‌حسب اسباط ایشان خواهندداد.
\par 9 «خداوند یهوه چنین می‌گوید: ای سروران اسرائیل باز ایستید و جور و ستم را دور کنید وانصاف و عدالت را بجا آورید و ظلم خود را ازقوم من رفع نمایید. قول خداوند یهوه این است:
\par 10 میزان راست و ایفای راست و بت راست برای شما باشد
\par 11 و ایفا و بت یکمقدار باشد به نوعی که بت به عشر حومر و ایفا به عشر حومر مساوی باشد. مقدار آنها بر‌حسب حومر باشد.
\par 12 و مثقال بیست جیره باشد. و منای شما بیست مثقال وبیست و پنج مثقال و پانزده مثقال باشد.
\par 13 «و هدیه‌ای که بگذرانید این است: یک سدس ایفا از هر حومر گندم و یک سدس ایفا ازهر حومر جو بدهید.
\par 14 و قسمت معین روغن برحسب بت روغن یک عشر بت از هر کر یا حومرده بت باشد زیرا که ده بت یک حومر می‌باشد.
\par 15 و یک گوسفند از دویست گوسفند از مرتع های سیراب اسرائیل برای هدیه آردی و قربانی سوختنی و ذبایح سلامتی بدهند تا برای ایشان کفاره بشود. قول خداوند یهوه این است.
\par 16 وتمامی قوم زمین این هدیه را برای رئیس دراسرائیل بدهند.
\par 17 و رئیس قربانی های سوختنی و هدایای آردی و هدایای ریختنی را در عیدها وهلال‌ها و سبت‌ها و همه مواسم خاندان اسرائیل بدهد واو قربانی گناه و هدیه آردی و قربانی سوختنی و ذبایح سلامتی را به جهت کفاره برای خاندان اسرائیل بگذراند.»
\par 18 خداوند یهوه چنین می‌گوید: «در غره ماه اول، گاوی جوان بی‌عیب گرفته، مقدس را طاهرخواهی نمود.
\par 19 و کاهن قدری از خون قربانی گناه گرفته، آن را بر چهار چوب خانه و بر چهارگوشه خروج مذبح و بر چهار چوب دروازه صحن اندرونی خواهد پاشید.
\par 20 و همچنین درروز هفتم ماه برای هر‌که سهو یا غفلت خطا ورزدخواهی کرد و شما برای خانه کفاره خواهیدنمود.
\par 21 و در روز چهاردهم ماه اول برای شماهفت روز عید فصح خواهد بود که در آنها نان فطیر خورده شود.
\par 22 و در آن روز رئیس، گاوقربانی گناه را برای خود و برای تمامی اهل زمین بگذراند.
\par 23 و در هفت روز عید، یعنی در هر روزاز آن هفت روز، هفت گاو و هفت قوچ بی‌عیب به جهت قربانی سوختنی برای خداوند و هر روزیک بز نر به جهت قربانی گناه بگذارند.
\par 24 و هدیه آردیش را یک ایفا برای هر گاو و یک ایفا برای هرقوچ و یک هین روغن برای هر ایفا بگذراند.واز روز پانزدهم ماه هفتم، در وقت عید موافق اینهایعنی موافق قربانی گناه و قربانی سوختنی و هدیه آردی و روغن تا هفت روز خواهد گذرانید.»
\par 25 واز روز پانزدهم ماه هفتم، در وقت عید موافق اینهایعنی موافق قربانی گناه و قربانی سوختنی و هدیه آردی و روغن تا هفت روز خواهد گذرانید.»

\chapter{46}

\par 1 خداوند یهوه چنین می‌گوید: «دروازه صحن اندرونی که به سمت مشرق متوجه است در شش روز شغل بسته بماند و درروز سبت مفتوح شود و در روز اول ماه گشاده گردد.
\par 2 و رئیس از راه رواق دروازه بیرونی داخل شود و نزد چهار چوب دروازه بایستد و کاهنان قربانی سوختنی و ذبیحه سلامتی او را بگذرانند واو بر آستانه دروازه سجده نماید، پس بیرون بروداما دروازه تا شام بسته نشود.
\par 3 و اهل زمین درسبت‌ها و هلال‌ها نزد دهنه آن دروازه به حضورخداوند سجده نمایند.
\par 4 و قربانی سوختنی که رئیس در روز سبت برای خداوند بگذراند، شش بره بی‌عیب و یک قوچ بی‌عیب خواهد بود.
\par 5 وهدیه آردی‌اش یک ایفا برای هر قوچ باشد وهدیه‌اش برای بره‌ها هر‌چه از دستش برآید و یک هین روغن برای هر ایفا.
\par 6 و در غره ماه یک گاوجوان بی‌عیب و شش بره و یک قوچ که بی‌عیب باشد.
\par 7 و هدیه آردی‌اش یک ایفا برای هر گاو ویک ایفا برای هر قوچ و هر‌چه از دستش برآیدبرای بره‌ها و یک هین روغن برای هر ایفا بگذراند.
\par 8 و هنگامی که رئیس داخل شود از راه رواق دروازه درآید و از همان راه بیرون رود.
\par 9 وهنگامی که اهل زمین در مواسم به حضورخداوند داخل شوند، آنگاه هر‌که از راه دروازه شمالی به جهت عبادت داخل شود، از راه دروازه جنوبی بیرون رود. و هرکه از راه دروازه جنوبی داخل شود، از راه دروازه شمالی بیرون رود و ازآن دروازه که از آن داخل شده باشد، برنگرددبلکه پیش روی خود بیرون رود.
\par 10 و چون ایشان داخل شوند رئیس در میان ایشان داخل شود و چون بیرون روند با هم بیرون روند.
\par 11 و هدیه آردی‌اش در عیدها و مواسم یک ایفا برای هر گاوو یک ایفا برای هر قوچ و هر‌چه از دستش برآیدبرای بره‌ها و یک هین روغن برای هر ایفا خواهدبود.
\par 12 و چون رئیس هدیه تبرعی را خواه قربانی سوختنی یا ذبایح سلامتی به جهت هدیه تبرعی برای خداوند بگذراند، آنگاه دروازه‌ای را که به سمت مشرق متوجه است بگشایند و او قربانی سوختنی و ذبایح سلامتی خود را بگذراند به طوری که آنها را در روز سبت می‌گذراند. پس بیرون رود و چون بیرون رفت دروازه را ببندند.
\par 13 و یک بره یک ساله بی‌عیب هر روز به جهت قربانی سوختنی برای خداوند خواهی گذرانید، هر صبح آن را بگذران.
\par 14 و هر بامداد هدیه آردی آن را خواهی گذرانید، یعنی یک سدس ایفا و یک ثلث هین روغن که بر آرد نرم پاشیده شود که هدیه آردی دایمی برای خداوند به فریضه ابدی خواهد بود.
\par 15 پس بره و هدیه آردی‌اش و روغنش را هر صبح به جهت قربانی سوختنی دایمی خواهند گذرانید.»
\par 16 خداوند یهوه چنین می‌گوید: «چون رئیس بخششی به یکی از پسران خود بدهد، حق ارثیت آن از آن پسرانش خواهد بود و ملک ایشان به رسم ارثیت خواهد بود.
\par 17 لیکن اگر بخششی ازملک موروث خویش به یکی از بندگان خودبدهد، تا سال انفکاک از آن او خواهد بود، پس به رئیس راجع خواهد شد و میراث او فقط از آن پسرانش خواهد بود.
\par 18 و رئیس از میراث قوم نگیرد و ملک ایشان را غصب ننماید بلکه پسران خود را از ملک خویش میراث دهد تا قوم من هر کس از ملک خویش پراکنده نشوند.
\par 19 پس مرا ازمدخلی که به پهلوی دروازه بود به حجره های مقدس کاهنان که به سمت شمال متوجه بوددرآورد. و اینک در آنجا بهر دو طرف به سمت مغرب مکانی بود.»
\par 20 و مرا گفت: «این است مکانی که کاهنان، قربانی جرم و قربانی گناه را طبخ می‌نمایند وهدیه آردی را می‌پزند تا آنها را به صحن بیرونی به جهت تقدیس نمودن قوم بیرون نیاورند.»
\par 21 پس مرا به صحن بیرونی آورد و مرا به چهار زاویه صحن گردانید و اینک در هر زاویه صحن صحنی بود.
\par 22 یعنی در چهار گوشه صحن صحنهای محوطه‌ای بود که طول هر یک چهل وعرضش سی (ذراع ) بود. این چهار را که درزاویه‌ها بود یک مقدار بود.
\par 23 و به گرداگرد آنهابطرف آن چهار طاقها بود و مطبخ‌ها زیر آن طاقهااز هر طرفش ساخته شده بود.و مرا گفت: «اینها مطبخ‌ها می‌باشد که خادمان خانه در آنهاذبایح قوم را طبخ می‌نمایند.»
\par 24 و مرا گفت: «اینها مطبخ‌ها می‌باشد که خادمان خانه در آنهاذبایح قوم را طبخ می‌نمایند.»

\chapter{47}

\par 1 و مرا نزد دروازه خانه آورد و اینک آبهااز زیر آستانه خانه بسوی مشرق جاری بود، زیرا که روی خانه به سمت مشرق بود وآن آبها اززیر جانب راست خانه از طرف جنوب مذبح جاری بود.
\par 2 پس مرا از راه دروازه شمالی بیرون برده، از راه خارج به دروازه بیرونی به راهی که به سمت مشرق متوجه است گردانید و اینک آبها از جانب راست جاری بود.
\par 3 و چون آن مردبسوی مشرق بیرون رفت، ریسمانکاری در دست داشت و هزار ذراع پیموده، مرا از آب عبور داد وآبها به قوزک می‌رسید.
\par 4 پس هزار ذراع پیمود ومرا از آبها عبور داد و آب به زانو می‌رسید و بازهزار ذراع پیموده، مرا عبور داد و آب به کمرمی رسید.
\par 5 پس هزار ذراع پیمود و نهری بود که از آن نتوان عبور کرد زیرا که آب زیاده شده بود، آبی که در آن می‌شود شنا کرد نهری که از آن عبور نتوان کرد.
\par 6 و مرا گفت: «ای پسر انسان آیااین را دیدی؟» پس مرا از آنجا برده، به کنار نهربرگردانید.
\par 7 و چون برگشتم اینک بر کنار نهر از اینطرف و از آنطرف درختان بی‌نهایت بسیار بود.
\par 8 و مراگفت: «این آبها بسوی ولایت شرقی جاری می‌شود و به عربه فرود شده، به دریا می‌رود وچون به دریا داخل می‌شود آبهایش شفا می‌یابد.
\par 9 و واقع خواهد شد که هر ذی حیات خزنده‌ای در هر جایی که آن نهر داخل شود، زنده خواهدگشت و ماهیان از حد زیاده پیدا خواهد شد، زیراچون این آبها به آنجا می‌رسد، آن شفا خواهدیافت و هر جایی که نهر جاری می‌شود، همه‌چیززنده می‌گردد.
\par 10 و صیادان بر کنار آن خواهندایستاد و از عین جدی تا عین عجلایم موضعی برای پهن کردن دامها خواهد بود و ماهیان آنها به حسب جنسها، مثل ماهیان دریای بزرگ از حدزیاده خواهند بود.
\par 11 اما خلابها و تالابهایش شفانخواهد یافت بلکه به نمک تسلیم خواهد شد.
\par 12 و بر کنار نهر به اینطرف و آنطرف هر قسم درخت خوراکی خواهد رویید که برگهای آنهاپژمرده نشود و میوه های آنها لاینقطع خواهد بودو هر ماه میوه تازه خواهد آورد زیرا که آبش از مقدس جاری می‌شود و میوه آنها برای خوراک وبرگهای آنها به جهت علاج خواهد بود.»
\par 13 خداوند یهوه چنین می‌گوید: «این است حدودی که زمین را برای دوازده سبط اسرائیل به آنها تقسیم خواهید نمود. برای یوسف دوقسمت.
\par 14 و شما هر کس مثل دیگری آن را به تصرف خواهید آورد زیرا که من دست خود رابرافراشتم که آن را به پدران شما بدهم پس این زمین به قرعه به شما به ملکیت داده خواهد شد.
\par 15 و حدود زمین این است. بطرف شمال از دریای بزرگ بطرف حتلون تا مدخل صدد.
\par 16 حمات وبیروته و سبرایم که در میان سرحد دمشق وسرحد حمات است و حصر وسطی که نزدسرحد حوران است.
\par 17 و حد از دریا حصر عینان نزد سرحد دمشق و بطرف سرحد حمات خواهدبود. و این است جانب شمالی.
\par 18 و بطرف شرقی در میان حوران و دمشق و در میان جلعاد و زمین اسرائیل اردن خواهد بود و از این حد تا دریای شرقی خواهی پیمود و این حد شرقی می‌باشد.
\par 19 و طرف جنوبی به‌جانب راست از تامار تا آب مریبوت قادش و نهر (مصر) و دریای بزرگ و این طرف جنوبی به‌جانب راست خواهد بود.
\par 20 وطرف غربی دریای بزرگ ازحدی که مقابل مدخل حمات است خواهد بود و این‌جانب غربی باشد.
\par 21 پس این زمین را برای خود برحسب اسباط اسرائیل تقسیم خواهید نمود.
\par 22 وآن را برای خود و برای غریبانی که در میان شماماوا گزینند و در میان شما اولاد بهم رسانند به قرعه تقسیم خواهید کرد و ایشان نزد شما مثل متوطنان بنی‌اسرائیل خواهند بود و با شما درمیان اسباط اسرائیل میراث خواهند یافت.وخداوند یهوه می‌فرماید: در هر سبط که شخصی غریب در آن ساکن باشد، در همان ملک خود راخواهد یافت.
\par 23 وخداوند یهوه می‌فرماید: در هر سبط که شخصی غریب در آن ساکن باشد، در همان ملک خود راخواهد یافت.

\chapter{48}

\par 1 «و این است نامهای اسباط: از طرف شمال تا جانب حتلون و مدخل حمات و حصر عینان نزد سرحد شمالی دمشق تا جانب حمات حد آنها از مشرق تا مغرب. برای دان یک قسمت.
\par 2 و نزد حد دان از طرف مشرق تا طرف مغرب برای اشیر یک قسمت.
\par 3 و نزد حد اشیر ازطرف مشرق تا طرف مغرب برای نفتالی یک قسمت.
\par 4 و نزد حد نفتالی از طرف مشرق تاطرف مغرب برای منسی یک قسمت.
\par 5 و نزد حدمنسی از طرف مشرق تا طرف مغرب برای افرایم یک قسمت.
\par 6 و نزد حد افرایم از طرف مشرق تاطرف مغرب برای رئوبین یک قسمت.
\par 7 و نزد حدرئوبین از طرف مشرق تا طرف مغرب برای یهودایک قسمت.
\par 8 و نزد حد یهودا از طرف مشرق تاطرف مغرب هدیه‌ای که می‌گذرانید خواهد بودکه عرضش بیست و پنجهزار (نی ) و طولش ازجانب مشرق تا جانب مغرب موافق یکی از این قسمت‌ها باشد و مقدس در میانش خواهد بود.
\par 9 و طول این هدیه‌ای که برای خداوند می‌گذرانیدبیست و پنج هزار(نی ) و عرضش ده هزار (نی )خواهد بود.
\par 10 و این هدیه مقدس برای اینان یعنی برای کاهنان می‌باشد و طولش بطرف شمال بیست و پنجهزار و عرضش بطرف مغرب ده هزارو عرضش بطرف مشرق ده هزار و طولش بطرف جنوب بیست و پنجهزار (نی ) می‌باشد و مقدس خداوند در میانش خواهد بود.
\par 11 و این برای کاهنان مقدس از بنی صادوق که ودیعت مرا نگاه داشته‌اند خواهد بود، زیرا ایشان هنگامی که بنی‌اسرائیل گمراه شدند و لاویان نیز ضلالت ورزیدند، گمراه نگردیدند.
\par 12 لهذا این برای ایشان از هدیه زمین، هدیه قدس اقداس به پهلوی سرحد لاویان خواهد بود.
\par 13 و مقابل حد کاهنان حصه‌ای که طولش بیست و پنجهزار و عرضش ده هزار (نی ) باشد برای لاویان خواهد بود، پس طول تمامش بیست و پنجهزار و عرضش ده هزار(نی ) خواهد بود.
\par 14 و از آن چیزی نخواهندفروخت و مبادله نخواهند نمود و نوبرهای زمین صرف دیگران نخواهد شد زیرا که برای خداوندمقدس می‌باشد.
\par 15 و پنجهزار (نی ) که ازعرضش مقابل آن بیست و پنجهزار (نی ) باقی می‌ماند عام خواهد بود، به جهت شهر و مسکن هاو نواحی شهر. و شهر در وسطش خواهد بود.
\par 16 و پیمایشهای آن این است: بطرف شمال چهارهزار و پانصد و بطرف جنوب چهار هزار و پانصدو به طرف مشرق چهار هزار و پانصد و به طرف مغرب چهار هزار و پانصد (ذراع ).
\par 17 و نواحی شهر بطرف شمال دویست و پنجاه و بطرف جنوب دویست و پنجاه و بطرف مشرق دویست و پنجاه و بطرف مغرب دویست و پنجاه خواهدبود.
\par 18 و آنچه از طولش مقابل هدیه مقدس باقی می‌ماند بطرف مشرق ده هزار و بطرف مغرب ده هزار (نی ) خواهد بود و این مقابل هدیه مقدس باشد و محصولش خوراک آنانی که در شهر کارمی کنند خواهد بود.
\par 19 و کارکنان شهر از همه اسباط اسرائیل آن را کشت خواهند کرد.
\par 20 پس تمامی هدیه بیست و پنجهزار در بیست و پنجهزار (نی ) باشد این هدیه مقدس را با ملک شهر مربع خواهید گذرانید.
\par 21 و بقیه آن بهر دوطرف هدیه مقدس و ملک شهر از آن رئیس خواهد بود؛ و این حصه رئیس نزد حد شرقی دربرابر آن بیست و پنجهزار (نی ) هدیه و نزد حدغربی هم برابر بیست و پنجهزار (نی هدیه )خواهد بود؛ و هدیه مقدس و مقدس خانه درمیانش خواهد بود.
\par 22 و از ملک لاویان و از ملک شهر‌که در میان ملک رئیس است، حصه‌ای درمیان حد یهودا و حد بنیامین از آن رئیس خواهدبود.
\par 23 و اما برای بقیه اسباط از طرف مشرق تاطرف مغرب برای بنیامین یک قسمت.
\par 24 و نزدحد بنیامین از طرف مشرق تا طرف مغرب برای شمعون یک قسمت.
\par 25 و نزد حد شمعون ازطرف مشرق تا طرف مغرب برای یساکار یک قسمت.
\par 26 و نزد حد یساکار از طرف مشرق تاطرف مغرب برای زبولون یک قسمت.
\par 27 و نزدحد زبولون از طرف مشرق تا طرف مغرب برای جاد یک قسمت.
\par 28 و نزد حد جاد بطرف جنوب به‌جانب راست حد (زمین ) از تامار تا آب مریبه قادش و نهر (مصر) و دریای بزرگ خواهدبود.»
\par 29 خداوند یهوه می‌گوید: «این است زمینی که برای اسباط اسرائیل به ملکیت تقسیم خواهیدکرد و قسمت های ایشان این می‌باشد.
\par 30 و این است مخرج های شهر بطرف شمال چهار هزار وپانصد پیمایش.
\par 31 و دروازه های شهر موافق نامهای اسباط اسرائیل باشد یعنی سه دروازه بطرف شمال. دروازه رئوبین یک و دروازه یهودایک و دروازه لاوی یک.
\par 32 و بطرف مشرق چهارهزار و پانصد (نی ) و سه دروازه یعنی دروازه یوسف یک و دروازه بنیامین یک و دروازه دان یک.
\par 33 و بطرف جنوب چهار هزار و پانصدپیمایش و سه دروازه یعنی دروازه شمعون یک ودروازه یساکار یک و دروازه زبولون یک.وبطرف مغرب چهار هزار و پانصد (نی ) و سه دروازه یعنی دروازه جاد یک و دروازه اشیر یک ودروازه نفتالی یک.
\par 34 وبطرف مغرب چهار هزار و پانصد (نی ) و سه دروازه یعنی دروازه جاد یک و دروازه اشیر یک ودروازه نفتالی یک.





\end{document}