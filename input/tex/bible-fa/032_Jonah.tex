\begin{document}

\title{يونس}


\chapter{1}

\par 1 و کلام خداوند بر یونس بن امتای نازل شده، گفت:
\par 2 «برخیز و به نینوا شهر بزرگ برو و بر آن ندا کن زیرا که شرارت ایشان به حضورمن برآمده است.»
\par 3 اما یونس برخاست تا از حضور خداوند به ترشیش فرار کند و به یافا فرود آمده، کشتی‌ای یافت که عازم ترشیش بود. پس کرایه‌اش را داده، سوار شد تا همراه ایشان از حضور خداوند به ترشیش برود.
\par 4 و خداوند باد شدیدی بر دریاوزانید که تلاطم عظیمی در دریا پدید آمدچنانکه نزدیک بود که کشتی شکسته شود.
\par 5 وملاحان ترسان شده، هر کدام نزد خدای خوداستغاثه نمودند و اسباب را که در کشتی بود به دریا ریختند تا آن را برای خود سبک سازند. امایونس در اندرون کشتی فرود شده، دراز شد وخواب سنگینی او را در ربود.
\par 6 و ناخدای کشتی نزد او آمده، وی را گفت: «ای که خفته‌ای تو را چه شده است؟ برخیز وخدای خود را بخوان شاید که خدا ما را بخاطرآورد تا هلاک نشویم.»
\par 7 و به یکدیگر گفتند: «بیایید قرعه بیندازیم تا بدانیم که این بلا به‌سبب چه کس بر ما وارد شده است؟» پس چون قرعه انداختند، قرعه به نام یونس درآمد.
\par 8 پس او راگفتند: «ما را اطلاع ده که این بلا به‌سبب چه کس بر ما عارض شده؟ شغل تو چیست و از کجا آمده‌ای و وطنت کدام است و از چه قوم هستی؟»
\par 9 او ایشان را جواب داد که: «من عبرانی هستم و ازیهوه خدای آسمان که دریا و خشکی را آفریده است ترسان می‌باشم.»
\par 10 پس آن مردمان سخت‌ترسان شدند و او را گفتند: «چه کرده‌ای؟» زیرا که ایشان می‌دانستند که از حضور خداوند فرار کرده است چونکه ایشان را اطلاع داده بود.
\par 11 و او راگفتند: «با تو چه کنیم تا دریا برای ما ساکن شود؟» زیرا دریا در تلاطم همی افزود.
\par 12 او به ایشان گفت: «مرا برداشته، به دریا بیندازید ودریا برای شما ساکن خواهد شد، زیرا می‌دانم این تلاطم عظیم به‌سبب من بر شما وارد آمده است.
\par 13 اما آن مردمان سعی نمودند تا کشتی را به خشکی برسانند اما نتوانستند زیرا که دریا به ضدایشان زیاده و زیاده تلاطم می‌نمود.
\par 14 پس نزدیهوه دعا کرده، گفتند: «آه‌ای خداوند به‌خاطرجان این شخص هلاک نشویم و خون بی‌گناه را برما مگذار زیرا تو‌ای خداوند هر‌چه می‌خواهی می‌کنی.»
\par 15 پس یونس را برداشته، در دریاانداختند و دریا از تلاطمش آرام شد.
\par 16 و آن مردمان از خداوند سخت‌ترسان شدند و برای خداوند قربانی‌ها گذرانیدند و نذرها نمودند.واما خداوند ماهی بزرگی پیدا کرد که یونس را فروبرد و یونس سه روز و سه شب در شکم ماهی ماند.
\par 17 واما خداوند ماهی بزرگی پیدا کرد که یونس را فروبرد و یونس سه روز و سه شب در شکم ماهی ماند.

\chapter{2}

\par 1 و یونس از شکم ماهی نزد یهوه خدای خود دعا نمود
\par 2 و گفت: «در تنگی خودخداوند را خواندم و مرا مستجاب فرمود. ازشکم هاویه تضرع نمودم و آواز مرا شنیدی.
\par 3 زیرا که مرا به ژرفی در دل دریاها انداختی وسیلها مرا احاطه نمود. جمیع خیزابها و موجهای تو بر من گذشت.
\par 4 و من گفتم از پیش چشم توانداخته شدم. لیکن هیکل قدس تو را باز خواهم دید.
\par 5 آبها مرا تا به‌جان احاطه نمود و لجه دورمرا گرفت و علف دریا بسر من پیچیده شد.
\par 6 به بنیان کوهها فرود رفتم و زمین به بندهای خود تا به ابد مرا در‌گرفت. اما تو‌ای یهوه خدایم حیات مرااز حفره برآوردی.
\par 7 چون جان من در اندرونم بی‌تاب شد، خداوند را بیاد آوردم و دعای من نزدتو به هیکل قدست رسید.
\par 8 آنانی که اباطیل دروغ را منظور می‌دارند، احسان های خویش را ترک می‌نمایند.
\par 9 اما من به آواز تشکر برای تو قربانی خواهم گذرانید، و به آنچه نذر کردم وفا خواهم نمود. نجات از آن خداوند است.»پس خداوند ماهی را امر فرمود و یونس را بر خشکی قی کرد.
\par 10 پس خداوند ماهی را امر فرمود و یونس را بر خشکی قی کرد.

\chapter{3}

\par 1 پس کلام خداوند بار دوم بر یونس نازل شده، گفت:
\par 2 «برخیز و به نینوا شهر بزرگ برو و آن وعظ را که من به تو خواهم گفت به ایشان ندا کن.»
\par 3 آنگاه یونس برخاسته، برحسب فرمان خداوند به نینوا رفت و نینوا شهر بزرگ بود که مسافت سه روز داشت.
\par 4 و یونس به مسافت یک روز داخل شهر شده، به ندا کردن شروع نمود وگفت بعد از چهل روز نینوا سرنگون خواهد شد.
\par 5 و مردمان نینوا به خدا ایمان آوردند و روزه راندا کرده، از بزرگ تا کوچک پلاس پوشیدند.
\par 6 و چون پادشاه نینوا از این امر اطلاع یافت، ازکرسی خود برخاسته، ردای خود را از برکند وپلاس پوشیده، بر خاکستر نشست.
\par 7 و پادشاه واکابرش فرمان دادند تا در نینوا ندا در‌دادند وامرفرموده، گفتند که «مردمان و بهایم و گاوان وگوسفندان چیزی نخورند و نچرند و آب ننوشند.
\par 8 و مردمان و بهایم به پلاس پوشیده شوند و نزدخدا بشدت استغاثه نمایند و هرکس از راه بدخود و از ظلمی که در دست او است بازگشت نماید.
\par 9 کیست بداند که شاید خدا برگشته، پشیمان شود و از حدت خشم خود رجوع نمایدتا هلاک نشویم.»پس چون خدا اعمال ایشان را دید که از راه زشت خود بازگشت نمودند، آنگاه خدا از بلایی که گفته بود که به ایشان برساند پشیمان گردید وآن را بعمل نیاورد.
\par 10 پس چون خدا اعمال ایشان را دید که از راه زشت خود بازگشت نمودند، آنگاه خدا از بلایی که گفته بود که به ایشان برساند پشیمان گردید وآن را بعمل نیاورد.

\chapter{4}

\par 1 اما این امر یونس را به غایت ناپسند آمد وغیظش افروخته شد،
\par 2 و نزد خداوند دعانموده، گفت: «آه‌ای خداوند، آیا این سخن من نبود حینی که در ولایت خود بودم و از این سبب به فرار کردن به ترشیش مبادرت نمودم زیرامی دانستم که تو خدای کریم و رحیم و دیر غضب و کثیر احسان هستی و از بلا پشیمان می‌شوی؟
\par 3 پس حال‌ای خداوند جانم را از من بگیر زیرا که مردن از زنده ماندن برای من بهتر است.»
\par 4 خداوندگفت: «آیا صواب است که خشمناک شوی؟»
\par 5 و یونس از شهر بیرون رفته، بطرف شرقی شهر نشست و در آنجا سایه بانی برای خودساخته زیر سایه‌اش نشست تا ببیند بر شهر‌چه واقع خواهد شد.
\par 6 و یهوه خدا کدویی رویانید وآن را بالای یونس نمو داد تا بر سر وی سایه افکنده، او را از حزنش آسایش دهد و یونس ازکدو بی‌نهایت شادمان شد.
\par 7 اما در فردای آن روزدر وقت طلوع فجر خدا کرمی پیدا کرد که کدو رازد و خشک شد.
\par 8 و چون آفتاب برآمد خدا بادشرقی گرم وزانید و آفتاب بر سر یونس تابید به حدی که بیتاب شده، برای خود مسالت نمود که بمیرد و گفت: «مردن از زنده ماندن برای من بهتراست.»
\par 9 خدا به یونس جواب داد: «آیا صواب است که به جهت کدو غضبناک شوی؟» او گفت: «صواب است که تا به مرگ غضبناک شوم.»خداوند گفت: «دل تو برای کدو بسوخت که برای آن زحمت نکشیدی و آن را نمو ندادی که در یک شب بوجود آمد و در یک شب ضایع گردید.
\par 10 خداوند گفت: «دل تو برای کدو بسوخت که برای آن زحمت نکشیدی و آن را نمو ندادی که در یک شب بوجود آمد و در یک شب ضایع گردید.



\end{document}