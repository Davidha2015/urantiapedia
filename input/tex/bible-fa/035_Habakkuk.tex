\begin{document}

\title{حبقوق}


\chapter{1}

\par 1 وحی که حبقوق نبی آن را دید.
\par 2 ‌ای خداوند تا به کی فریاد برمی آورم ونمی شنوی؟ تا به کی نزد تو از ظلم فریادبرمی آورم و نجات نمی دهی؟
\par 3 چرا بی‌انصافی رابه من نشان می‌دهی و بر ستم نظر می‌نمایی وغضب و ظلم پیش روی من می‌باشد؟ منازعه پدید می‌آید و مخاصمت سر خود را بلندمی کند.
\par 4 از این سبب، شریعت سست شده است و عدالت هرگز صادر نمی شود. چونکه شریران عادلان را احاطه می‌نمایند. بنابراین عدالت معوج شده صادر می‌گردد.
\par 5 در میان امت‌ها نظر کنید و ملاحظه نمایید وبشدت متحیر شوید. زیرا که در ایام شما کاری می‌کنم که اگر شما را هم از آن مخبر سازند، باورنخواهید کرد.
\par 6 زیرا که اینک آن امت تلخ وتندخو، یعنی کلدانیان را برمی انگیزانم که دروسعت جهان می‌خرامند تا مسکن هایی را که ازآن ایشان نیست به تصرف آورند.
\par 7 ایشان هولناک و مهیب می‌باشند. حکم و جلال ایشان از خودایشان صادر می‌شود.
\par 8 اسبان ایشان از پلنگهاچالاکتر و از گرگان شب تیزروترند و سواران ایشان جست و خیز می‌کنند. و سواران ایشان ازجای دور آمده، مثل عقابی که برای خوراک بشتابد می‌پرند.
\par 9 جمیع ایشان برای ظلم می‌آیند. عزیمت روی ایشان بطرف پیش است و اسیران رامثل ریگ جمع می‌کنند.
\par 10 و ایشان پادشاهان رااستهزا می‌نمایند و سروران مسخره ایشان می‌باشند. بر همه قلعه‌ها می‌خندند و خاک راتوده نموده، آنها را مسخر می‌سازند.
\par 11 پس مثل باد بشتاب رفته، عبور می‌کنند و مجرم می‌شوند. این قوت ایشان خدای ایشان است.
\par 12 ‌ای یهوه خدای من! ای قدوس من! آیا تو ازازل نیستی؟ پس نخواهیم مرد. ای خداوند ایشان را برای داوری معین کرده‌ای و‌ای صخره، ایشان را برای تادیب تاسیس نموده‌ای.
\par 13 چشمان توپاکتر است از اینکه به بدی بنگری و به بی‌انصافی نظر نمی توانی کرد. پس چرا خیانتکاران راملاحظه می‌نمایی و حینی که شریر کسی را که ازخودش عادل تر است می‌بلعد، خاموش می‌مانی؟
\par 14 و مردمان را مثل ماهیان دریا و مانندحشراتی که حاکمی ندارند می‌گردانی؟
\par 15 اوهمگی ایشان را به قلاب برمی کشد و ایشان را به دام خود می‌گیرد و در تور خویش آنها را جمع می‌نماید. از اینجهت، مسرور و شادمان می‌شود.
\par 16 بنابراین، برای دام خود قربانی می‌گذراند و برای تور خویش بخور می‌سوزاند. چونکه نصیب او از آنها فربه و خوراک وی لذیذ می‌شود.آیااز اینجهت دام خود را خالی خواهد کرد و ازپیوسته کشتن امت‌ها دریغ نخواهد نمود؟
\par 17 آیااز اینجهت دام خود را خالی خواهد کرد و ازپیوسته کشتن امت‌ها دریغ نخواهد نمود؟

\chapter{2}

\par 1 بر دیده بانگاه خود می‌ایستم و بر برج برپامی شوم. و مراقب خواهم شد تا ببینم که اوبه من چه خواهد گفت و درباره شکایتم چه جواب خواهد داد.
\par 2 پس خداوند مرا جواب دادو گفت: رویا را بنویس و آن را بر لوحها چنان نقش نما که دونده آن را بتواند خواند.
\par 3 زیرا که رویا هنوز برای وقت معین است و به مقصدمی شتابد و دروغ نمی گوید. اگر‌چه تاخیر نمایدبرایش منتظر باش زیرا که البته خواهد آمد ودرنگ نخواهد نمود.
\par 4 اینک جان مرد متکبر در اوراست نمی باشد، اما مرد عادل به ایمان خودزیست خواهد نمود.
\par 5 به درستی که شراب فریبنده است و مرد مغرور آرامی نمی پذیرد، که شهوت خود را مثل عالم اموات می‌افزاید وخودش مثل موت، سیر نمی شود. بلکه جمیع امت‌ها را نزد خود جمع می‌کند و تمامی قوم‌ها رابرای خویشتن فراهم می‌آورد.
\par 6 پس آیا جمیع ایشان بر وی مثلی نخواهند زد و معمای طعن آمیز بر وی (نخواهند‌آورد)؟ و نخواهندگفت: وای بر کسی‌که آنچه را که از آن وی نیست می‌افزاید؟ تا به کی؟ و خویشتن را زیر بار رهنهامی نهد.
\par 7 آیا گزندگان بر تو ناگهان برنخواهندخاست و آزارندگانت بیدار نخواهند شد و تو راتاراج نخواهند نمود؟
\par 8 چونکه تو امت های بسیاری را غارت کرده‌ای، تمامی بقیه قوم‌ها تو راغارت خواهند نمود، به‌سبب خون مردمان وظلمی که بر زمین و شهر و جمیع ساکنانش نموده‌ای.
\par 9 وای بر کسی‌که برای خانه خود بدی راکسب نموده است تا آشیانه خود را بر جای بلندساخته، خویشتن را از دست بلا برهاند.
\par 10 رسوایی را به جهت خانه خویش تدبیرکرده‌ای به اینکه قوم های بسیار را قطع نموده و برضد جان خویش گناه ورزیده‌ای.
\par 11 زیراکه سنگ از دیوار فریاد برخواهد آورد و تیر از میان چوبها آن را جواب خواهد داد.
\par 12 وای بر کسی‌که شهری به خون بنا می‌کند و قریه‌ای به بی‌انصافی استوار می‌نماید.
\par 13 آیا این از جانب یهوه صبایوت نیست که قوم‌ها برای آتش مشقت می‌کشند و طوایف برای بطالت خویشتن راخسته می‌نمایند؟
\par 14 زیرا که جهان از معرفت جلال خداوند مملو خواهد شد به نحوی که آبهادریا را مستور می‌سازد.
\par 15 وای بر کسی‌که همسایه خود را می‌نوشاندو بر تو که زهر خویش را ریخته، او را نیز مست می‌سازی تا برهنگی او را بنگری.
\par 16 تو ازرسوایی به عوض جلال سیر خواهی شد. تو نیزبنوش و غلفه خویش را منکشف ساز. کاسه دست راست خداوند بر تو وارد خواهد آمد و قی رسوایی بر جلال تو خواهد بود.
\par 17 زیرا ظلمی که بر لبنان نمودی و هلاکت حیوانات که آنها راترسانیده بود، تو را خواهد پوشانید. به‌سبب خون مردمان و ظلمی که بر زمین و شهر و بر جمیع ساکنانش رسانیدی.
\par 18 از بت تراشیده چه فایده است که سازنده آن، آن را بتراشد یا از بت ریخته شده و معلم دروغ، که سازنده آن بر صنعت خودتوکل بنماید و بتهای گنگ را بسازد.
\par 19 وای برکسی‌که به چوب بگوید بیدار شو و به سنگ گنگ که برخیز! آیا می‌شود که آن تعلیم دهد؟ اینک به طلا و نقره پوشیده می‌شود لکن در اندرونش مطلق روح نیست.اما خداوند در هیکل قدس خویش است پس تمامی جهان به حضور وی خاموش باشد.
\par 20 اما خداوند در هیکل قدس خویش است پس تمامی جهان به حضور وی خاموش باشد.

\chapter{3}

\par 1 دعای حبقوق نبی بر شجونوت.
\par 2 ای خداوند چون خبر تو را شنیدم ترسان گردیدم. ای خداوند عمل خویش را در میان سالها زنده کن! در میان سالها آن را معروف ساز ودر حین غضب رحمت را بیاد آور.
\par 3 خدا از تیمان آمد و قدوس از جبل فاران، سلاه. جلال اوآسمانها را پوشانید و زمین از تسبیح او مملوگردید.
\par 4 پرتو او مثل نور بود و از دست وی شعاع ساطع گردید. و ستر قوت او در آنجا بود.
\par 5 پیش روی وی وبا می‌رفت و آتش تب نزد پایهای اومی بود.
\par 6 او بایستاد و زمین را پیمود. او نظر افکندو امت‌ها را پراکنده ساخت و کوههای ازلی جستند و تلهای ابدی خم شدند. طریق های اوجاودانی است.
\par 7 خیمه های کوشان را در بلادیدم. و چادرهای زمین مدیان لرزان شد.
\par 8 ‌ای خداوند آیا بر نهرها غضب تو افروخته شد یا خشم تو بر نهرها و غیظ تو بر دریا واردآمد، که بر اسبان خود و ارابه های فتح مندی خویش سوار شدی؟
\par 9 کمان تو تمام برهنه شد، موافق قسمهایی که در کلام خود برای اسباط خورده‌ای، سلاه. زمین را به نهرها منشق ساختی.
\par 10 کوهها تو را دیدند و لرزان گشتند و سیلاب هاجاری شد. لجه آواز خود را داد و دستهای خویش را به بالا برافراشت.
\par 11 آفتاب و ماه در برجهای خود ایستادند. ازنور تیرهایت و از پرتو نیزه براق تو برفتند.
\par 12 باغضب در جهان خرامیدی، و با خشم امت‌ها راپایمال نمودی.
\par 13 برای نجات قوم خویش وخلاصی مسیح خود بیرون آمدی. سر را ازخاندان شریران زدی و اساس آن را تا به گردن عریان نمودی، سلاه.
\par 14 سر سرداران ایشان را به عصای خودشان مجروح ساختی، حینی که مثل گردباد آمدند تامرا پراکنده سازند. خوشی ایشان در این بود که مسکینان را در خفیه ببلعند.
\par 15 با اسبان خود بردریا و بر انبوه آبهای بسیار خرامیدی.
\par 16 چون شنیدم احشایم بلرزید و از آواز آن لبهایم بجنبید، و پوسیدگی به استخوانهایم داخل شده، در جای خود لرزیدم، که در روز تنگی استراحت یابم هنگامی که آن که قوم را ذلیل خواهدساخت، بر ایشان حمله آورد.
\par 17 اگرچه انجیر شکوفه نیاورد و میوه در موهایافت نشود و حاصل زیتون ضایع گردد ومزرعه‌ها آذوقه ندهد، و گله‌ها از آغل منقطع شودو رمه‌ها در طویله‌ها نباشد،لیکن من درخداوند شادمان خواهم شد و در خدای نجات خویش وجد خواهم نمود.
\par 18 لیکن من درخداوند شادمان خواهم شد و در خدای نجات خویش وجد خواهم نمود.


\end{document}