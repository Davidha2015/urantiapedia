\begin{document}

\title{ملکی}


\chapter{1}

\par 1 واسطه ملاکی.
\par 2 خداوند می‌گوید که شما را دوست داشته‌ام. اما شما می‌گویید: چگونه ما را دوست داشته‌ای؟ آیا عیسو برادر یعقوب نبود و خداوند می‌گویدکه یعقوب را دوست داشتم،
\par 3 و از عیسو نفرت نمودم و کوههای او را ویران و میراث وی رانصیب شغالهای بیابان گردانیدم.
\par 4 چونکه ادوم می‌گوید: منهدم شده‌ایم اما خواهیم برگشت ومخروبه‌ها را بنا خواهیم نمود. یهوه صبایوت چنین می‌فرماید: ایشان بنا خواهند نمود اما من منهدم خواهم ساخت و ایشان را به‌سرحدشرارت و به قومی که خداوند بر ایشان تا به ابدغضبناک می‌باشد مسمی خواهند ساخت.
\par 5 وچون چشمان شما این را بیند خواهید گفت: خداوند از حدود اسرائیل متعظم باد!
\par 6 پسر، پدر خود و غلام، آقای خویش رااحترام می‌نماید. پس اگر من پدر هستم احترام من کجا است؟ و اگر من آقا هستم هیبت من کجااست؟ یهوه صبایوت به شما تکلم می‌کند. ای کاهنانی که اسم مرا حقیر می‌شمارید و می‌گویید چگونه اسم تو را حقیر شمرده‌ایم؟
\par 7 نان نجس برمذبح من می‌گذرانید و می‌گویید چگونه تو رابی حرمت نموده‌ایم؟ از اینکه می‌گویید خوان خداوند محقر است.
\par 8 و چون کور را برای قربانی می‌گذرانید، آیا قبیح نیست؟ و چون لنگ یا بیماررا می‌گذرانید، آیا قبیح نیست؟ آن را به حاکم خود هدیه بگذران و آیا او از تو راضی خواهدشد یا تو را مقبول خواهد داشت؟ قول یهوه صبایوت این است.
\par 9 و الان از خدا مسالت نما تابر ما ترحم نماید. یهوه صبایوت می‌گوید این ازدست شما واقع شده است، پس آیا هیچ کدام ازشما را مستجاب خواهد فرمود؟
\par 10 کاش که یکی از شما می‌بود که درها راببندد تا آتش بر مذبح من بیجا نیفروزید. یهوه صبایوت می‌گوید: در شما هیچ خوشی ندارم وهیچ هدیه از دست شما قبول نخواهم کرد.
\par 11 زیرا که از مطلع آفتاب تا مغربش اسم من درمیان امت‌ها عظیم خواهد بود؛ و بخور و هدیه طاهر در هر جا به اسم من گذرانیده خواهد شد، زیرا یهوه صبایوت می‌گوید که اسم من در میان امت‌ها عظیم خواهد بود. و بخور و هدیه طاهر درهر جا به اسم من گذرانیده خواهد شد، زیرا یهوه صبایوت می‌گوید که اسم من در میان امت هاعظیم خواهد بود.
\par 12 اما شما آن را بی‌حرمت می‌سازید چونکه می‌گویید که خوان خداوند نجس است و ثمره آن یعنی طعامش محقر است.
\par 13 و یهوه صبایوت می‌فرماید که شما می‌گوییداینک این چه زحمت است و آن را اهانت می‌کنیدو چون (حیوانات ) دریده شده و لنگ و بیمار راآورده، آنها را برای هدیه می‌گذرانید آیا من آنهارا از دست شما قبول خواهم کرد؟ قول خداونداین است.پس ملعون باد هر‌که فریب دهد و باآنکه نرینه‌ای در گله خود دارد معیوبی برای خداوند نذر کرده، آن را ذبح نماید. زیرا که یهوه صبایوت می‌گوید: من پادشاه عظیم می‌باشم واسم من در میان امت‌ها مهیب خواهد بود.
\par 14 پس ملعون باد هر‌که فریب دهد و باآنکه نرینه‌ای در گله خود دارد معیوبی برای خداوند نذر کرده، آن را ذبح نماید. زیرا که یهوه صبایوت می‌گوید: من پادشاه عظیم می‌باشم واسم من در میان امت‌ها مهیب خواهد بود.

\chapter{2}

\par 1 و الان‌ای کاهنان این وصیت برای شما است!
\par 2 یهوه صبایوت می‌گوید که اگرنشنوید و آن را در دل خود جا ندهید تا اسم مراتمجید نمایید، من بر شما لعن خواهم فرستاد و بربرکات شما لعن خواهم کرد، بلکه آنها را لعن کرده‌ام چونکه آن را در دل خود جا ندادید.
\par 3 اینک من زراعت را به‌سبب شما نهیب خواهم نمود و بر رویهای شما سرگین یعنی سرگین عیدهای شما را خواهم پاشید و شما را با آن خواهند برداشت.
\par 4 و خواهید دانست که من این وصیت را بر شما فرستاده‌ام تا عهد من با لاوی باشد. قول یهوه صبایوت این است.
\par 5 عهد من باوی عهد حیات و سلامتی می‌بود و آنها را به‌سبب ترسی که از من می‌داشت به وی دادم و به‌سبب آنکه از اسم من هراسان می‌بود.
\par 6 شریعت حق در دهان او می‌بود و بی‌انصافی بر لبهایش یافت نمی شد بلکه در سلامتی و استقامت با من سلوک می‌نمود و بسیاری را از گناه برمی گردانید.
\par 7 زیرا که لبهای کاهن می‌باید معرفت را حفظنماید تا شریعت را از دهانش بطلبند چونکه اورسول یهوه صبایوت می‌باشد.
\par 8 اما یهوه صبایوت می‌گوید که شما از طریق تجاوز نموده، بسیاری را در شریعت لغزش دادید و عهد لاوی را شکستید.
\par 9 بنابراین من نیز شما را نزد تمامی این قوم خوار و پست خواهم ساخت زیرا که طریق مرا نگاه نداشته و در اجرای شریعت طرفداری نموده‌اید.
\par 10 آیا جمیع ما را یک پدر نیست و آیا یک خدا ما را نیافریده است؟ پس چرا عهد پدران خود را بی‌حرمت نموده، با یکدیگر خیانت می‌ورزیم؟
\par 11 یهودا خیانت ورزیده است ورجاسات را در اسرائیل و اورشلیم بعمل آورده‌اند زیرا که یهودا مقدس خداوند را که او آن را دوست می‌داشت بی‌حرمت نموده، دخترخدای بیگانه را به زنی گرفته است.
\par 12 پس خداوند هر کس را که چنین عمل نماید، هم خواننده و هم جواب دهنده را از خیمه های یعقوب منقطع خواهد ساخت و هر کس را نیز که برای یهوه صبایوت هدیه بگذراند.
\par 13 و این را نیزبار دیگر بعمل آورده‌اید که مذبح خداوند را بااشکها و گریه و ناله پوشانیده‌اید و از این جهت هدیه را باز منظور نمی دارد و آن را از دست شمامقبول نمی فرماید.
\par 14 اما شما می‌گویید سبب این چیست؟ سبب این است که خداوند در میان تو وزوجه جوانی ات شاهد بوده است و تو به وی خیانت ورزیده‌ای، با آنکه او یار تو و زوجه هم عهد تو می‌بود.
\par 15 و آیا او یکی را نیافرید با آنکه بقیه روح را می‌داشت و ازچه سبب یک را (فقطآفرید)؟ از این جهت که ذریت الهی را طلب می‌کرد. پس از روحهای خود باحذر باشید وزنهار احدی به زوجه جوانی خود خیانت نورزد.
\par 16 زیرا یهوه خدای اسرائیل می‌گوید که از طلاق نفرت دارم و نیز از اینکه کسی ظلم را به لباس خود بپوشاند. قول یهوه صبایوت این است پس ازروحهای خود با حذر بوده، زنهار خیانت نورزید.شما خداوند را به سخنان خود خسته نموده‌اید و می‌گویید: چگونه او را خسته نموده‌ایم؟ از اینکه گفته‌اید همه بدکاران به نظرخداوند پسندیده می‌باشند واو از ایشان مسروراست یا اینکه خدایی که داوری کند کجا است؟
\par 17 شما خداوند را به سخنان خود خسته نموده‌اید و می‌گویید: چگونه او را خسته نموده‌ایم؟ از اینکه گفته‌اید همه بدکاران به نظرخداوند پسندیده می‌باشند واو از ایشان مسروراست یا اینکه خدایی که داوری کند کجا است؟

\chapter{3}

\par 1 اینک من رسول خود را خواهم فرستاد و اوطریق را پیش روی من مهیا خواهد ساخت و خداوندی که شما طالب او می‌باشید، ناگهان به هیکل خود خواهد آمد، یعنی آن رسول عهدی که شما از او مسرور می‌باشید. هان او می‌آید! قول یهوه صبایوت این است.
\par 2 اما کیست که روزآمدن او را متحمل تواند شد؟ و کیست که در حین ظهور وی تواند ایستاد؟ زیرا که او مثل آتش قالگر و مانند صابون گازران خواهد بود.
\par 3 و مثل قالگر و مصفی کننده نقره خواهد نشست وبنی لاوی را طاهر ساخته، ایشان را مانند طلا ونقره مصفی خواهد گردانید تا ایشان هدیه‌ای برای خداوند به عدالت بگذرانند.
\par 4 آنگاه هدیه یهودا واورشلیم پسندیده خداوند خواهد شد چنانکه درایام قدیم و سالهای پیشین می‌بود.
\par 5 و من برای داوری نزد شما خواهم آمد و به ضد جادوگران وزناکاران و آنانی که قسم دروغ می‌خورند و کسانی که بر مزدور در مزدش و بیوه‌زنان و یتیمان ظلم می‌نمایند و غریب را از حق خودش دورمی سازند و از من نمی ترسند، بزودی شهادت خواهم داد. قول یهوه صبایوت این است.
\par 6 زیرامن که یهوه می‌باشم، تبدیل نمی پذیرم و از این سبب شما‌ای پسران یعقوب هلاک نمی شوید.
\par 7 شما از ایام پدران خود از فرایض من تجاوزنموده، آنها را نگاه نداشته‌اید. اما یهوه صبایوت می‌گوید: بسوی من بازگشت نمایید و من بسوی شما بازگشت خواهم کرد، اما شما می‌گویید به چه چیز بازگشت نماییم.
\par 8 آیا انسان خدا را گول زند؟ اما شما مرا گول زده‌اید و می‌گویید در چه چیز تو را گول زده‌ایم؟ در عشرها و هدایا.
\par 9 شماسخت ملعون شده‌اید زیرا که شما یعنی تمامی این امت مرا گول زده‌اید.
\par 10 تمامی عشرها را به مخزنهای من بیاورید تا در خانه من خوراک باشدو یهوه صبایوت می‌گوید مرا به اینطور امتحان نمایید که آیا روزنه های آسمان را برای شمانخواهم گشاد و چنان برکتی بر شما نخواهم ریخت که گنجایش آن نخواهد بود؟
\par 11 و یهوه صبایوت می‌گوید: خورنده را به جهت شما منع خواهم نمود تا ثمرات زمین شما را ضایع نسازد و مو شما در صحرا بی‌بار نشود.
\par 12 و همه امت هاشما را خوشحال خواهند خواند زیرا یهوه صبایوت می‌گوید که شما زمین مرغوب خواهیدبود.
\par 13 خداوند می‌گوید: به ضد من سخنان سخت گفته‌اید و می‌گویید به ضد تو چه گفته‌ایم؟
\par 14 گفته‌اید: بی‌فایده است که خدا را عبادت نماییم و چه سود از اینکه اوامر او را نگاه داریم وبحضور یهوه صبایوت با حزن سلوک نماییم؟
\par 15 و حال متکبران را سعادتمند می‌خوانیم وبدکاران نیز فیروز می‌شوند و ایشان خدا راامتحان می‌کنند و (معهذا) ناجی می‌گردند.
\par 16 آنگاه ترسندگان خداوند با یکدیگر مکالمه کردند و خداوند گوش گرفته، ایشان را استماع نمود و کتاب یادگاری به جهت ترسندگان خداوند و به جهت آنانی که اسم او را عزیزداشتند مکتوب شد.
\par 17 و یهوه صبایوت می‌گویدکه ایشان در آن روزی که من تعیین نموده‌ام، ملک خاص من خواهند بود و بر ایشان ترحم خواهم نمود، چنانکه کسی بر پسرش که او راخدمت می‌کند ترحم می‌نماید.و شمابرگشته، در میان عادلان و شریران و در میان کسانی که خدا را خدمت می‌نمایند و کسانی که او را خدمت نمی نمایند، تشخیص خواهید نمود.
\par 18 و شمابرگشته، در میان عادلان و شریران و در میان کسانی که خدا را خدمت می‌نمایند و کسانی که او را خدمت نمی نمایند، تشخیص خواهید نمود.

\chapter{4}

\par 1 زیرا اینک آن روزی که مثل تنور مشتعل می باشد، خواهد آمد و جمیع متکبران وجمیع بدکاران کاه خواهند بود و یهوه صبایوت می‌گوید: آن روز که می‌آید ایشان را چنان خواهدسوزانید که نه ریشه و نه شاخه‌ای برای ایشان باقی خواهد گذاشت.
\par 2 اما برای شما که از اسم من می‌ترسید آفتاب عدالت طلوع خواهد کرد و بربالهای وی شفا خواهد بود و شما بیرون آمده، مانند گوساله های پرواری جست و خیز خواهیدکرد.
\par 3 و یهوه صبایوت می‌گوید: شریران راپایمال خواهید نمود زیرا در آن روزی که من تعیین نموده‌ام، ایشان زیر کف پایهای شماخاکستر خواهند بود.
\par 4 تورات بنده من موسی راکه آن را با فرایض و احکام به جهت تمامی اسرائیل در حوریب امر فرمودم بیاد آورید.اینک من ایلیای نبی را قبل از رسیدن روز عظیم و مهیب خداوند نزد شما خواهم فرستاد.
\par 5 اینک من ایلیای نبی را قبل از رسیدن روز عظیم و مهیب خداوند نزد شما خواهم فرستاد.



\end{document}