\begin{document}

\title{Gospel of Matthew}


\chapter{1}

\par 1 کتاب نسب نامه عیسی مسیح بن داود بن ابراهیم:
\par 2 ابراهیم اسحاق را آورد و اسحاق یعقوب را آورد و یعقوب یهودا و برادران او راآورد.
\par 3 و یهودا، فارص و زارح را از تامار آورد وفارص، حصرون را آورد و حصرون، ارام را آورد.
\par 4 و ارام، عمیناداب را آورد و عمیناداب، نحشون را آورد و نحشون، شلمون را آورد.
\par 5 وشلمون، بوعز را از راحاب آورد و بوعز، عوبیدرا از راعوت آورد و عوبید، یسا را آورد.
\par 6 ویسا داود پادشاه را آورد و داود پادشاه، سلیمان را از زن اوریا آورد.
\par 7 و سلیمان، رحبعام را آورد و رحبعام، ابیا را آورد و ابیا، آسا راآورد.
\par 8 و آسا، یهوشافاط را آورد و یهوشافاط، یورام را آورد و یورام، عزیا را آورد.
\par 9 و عزیا، یوتام را آورد و یوتام، احاز را آورد و احاز، حزقیا را آورد.
\par 10 و حزقیا، منسی را آورد ومنسی، آمون را آورد و آمون، یوشیا را آورد.
\par 11 ویوشیا، یکنیا و برادرانش را در زمان جلای بابل آورد.
\par 12 و بعد از جلای بابل، یکنیا، سالتیئل را آورد و سالتیئیل، زروبابل را آورد.
\par 13 زروبابل، ابیهود را آورد و ابیهود، ایلیقایم را آورد و ایلیقایم، عازور را آورد.
\par 14 و عازور، صادوق را آورد و صادوق، یاکین را آورد و یاکین، ایلیهود را آورد.
\par 15 و ایلیهود، ایلعازر را آوردو ایلعازر، متان را آورد و متان، یعقوب راآورد.
\par 16 و یعقوب، یوسف شوهر مریم راآورد که عیسی مسمی به مسیح از او متولدشد.
\par 17 پس تمام طبقات، از ابراهیم تا داودچهارده طبقه است، و از داود تا جلای بابل چهارده طبقه، و از جلای بابل تا مسیح چهارده طبقه.
\par 18 اما ولادت عیسی مسیح چنین بود که چون مادرش مریم به یوسف نامزد شده بود، قبل ازآنکه با هم آیند، او را از روح‌القدس حامله یافتند.
\par 19 و شوهرش یوسف چونکه مرد صالح بود ونخواست او را عبرت نماید، پس اراده نمود او رابه پنهانی رها کند.
\par 20 اما چون او در این چیزهاتفکر می‌کرد، ناگاه فرشته خداوند در خواب بروی ظاهر شده، گفت: «ای یوسف پسر داود، ازگرفتن زن خویش مریم مترس، زیرا که آنچه دروی قرار گرفته است، از روح‌القدس است،
\par 21 و اوپسری خواهد زایید و نام او را عیسی خواهی نهاد، زیرا که او امت خویش را از گناهانشان خواهد رهانید.»
\par 22 و این همه برای آن واقع شد تاکلامی که خداوند به زبان نبی گفته بود، تمام گردد
\par 23 «که اینک باکره آبستن شده پسری خواهدزایید و نام او را عمانوئیل خواهند خواند که تفسیرش این است: خدا با ما.»
\par 24 پس چون یوسف از خواب بیدار شد، چنانکه فرشته خداوند بدو امر کرده بود، بعمل آورد و زن خویش را گرفتو تا پسر نخستین خود رانزایید، او را نشناخت؛ و او را عیسی نام نهاد.
\par 25 و تا پسر نخستین خود رانزایید، او را نشناخت؛ و او را عیسی نام نهاد.

\chapter{2}

\par 1 و چون عیسی در ایام هیرودیس پادشاه دربیت لحم یهودیه تولد یافت، ناگاه مجوسی چند از مشرق به اورشلیم آمده، گفتند:
\par 2 «کجاست آن مولود که پادشاه یهود است زیراکه ستاره او را در مشرق دیده‌ایم و برای پرستش او آمده‌ایم؟»
\par 3 اما هیرودیس پادشاه چون این راشنید، مضطرب شد و تمام اورشلیم با وی.
\par 4 پس همه روسای کهنه و کاتبان قوم را جمع کرده، ازایشان پرسید که «مسیح کجا باید متولد شود؟»
\par 5 بدو گفتند: «در بیت لحم یهودیه زیرا که از نبی چنین مکتوب است:
\par 6 و تو‌ای بیت لحم، در زمین یهودا از سایر سرداران یهودا هرگز کوچکترنیستی، زیرا که از تو پیشوایی به ظهور خواهدآمد که قوم من اسرائیل را رعایت خواهد نمود.»
\par 7 آنگاه هیرودیس مجوسیان را در خلوت خوانده، وقت ظهور ستاره را از ایشان تحقیق کرد.
\par 8 پس ایشان را به بیت لحم روانه نموده، گفت: «بروید و از احوال آن طفل بتدقیق تفحص کنید و چون یافتید مرا خبر دهید تا من نیز آمده، او را پرستش نمایم.»
\par 9 چون سخن پادشاه راشنیدند، روانه شدند که ناگاه آن ستاره‌ای که در مشرق دیده بودند، پیش روی ایشان می‌رفت تافوق آنجایی که طفل بود رسیده، بایستاد.
\par 10 وچون ستاره را دیدند، بی‌نهایت شاد و خوشحال گشتند
\par 11 و به خانه درآمده، طفل را با مادرش مریم یافتند و به روی در‌افتاده، او را پرستش کردند و ذخایر خود را گشوده، هدایای طلا وکندر و مر به وی گذرانیدند.
\par 12 و چون در خواب وحی بدیشان در‌رسید که به نزد هیرودیس بازگشت نکنند، پس از راه دیگر به وطن خویش مراجعت کردند.
\par 13 و چون ایشان روانه شدند، ناگاه فرشته خداوند در خواب به یوسف ظاهر شده، گفت: «برخیز و طفل و مادرش را برداشته به مصر فرارکن و در آنجا باش تا به تو خبر دهم، زیرا که هیرودیس طفل را جستجو خواهد کرد تا او راهلاک نماید.»
\par 14 پس شبانگاه برخاسته، طفل ومادر او را برداشته، بسوی مصر روانه شد
\par 15 و تاوفات هیرودیس در آنجا بماند، تا کلامی که خداوند به زبان نبی گفته بود تمام گردد که «ازمصر پسر خود را خواندم.»
\par 16 چون هیرودیس دید که مجوسیان او را سخریه نموده‌اند، بسیارغضبناک شده، فرستاد و جمیع اطفالی را که دربیت لحم و تمام نواحی آن بودند، از دو ساله وکمتر موافق وقتی که از مجوسیان تحقیق نموده بود، به قتل رسانید.
\par 17 آنگاه کلامی که به زبان ارمیای نبی گفته شده بود، تمام شد: «آوازی دررامه شنیده شد، گریه و زاری و ماتم عظیم که راحیل برای فرزندان خود گریه می‌کند و تسلی نمی پذیرد زیرا که نیستند.»
\par 18 اما چون هیرودیس وفات یافت، ناگاه فرشته خداوند در مصر به یوسف در خواب ظاهرشده، گفت:
\par 19 «برخیز و طفل و مادرش رابرداشته، به زمین اسرائیل روانه شو زیرا آنانی که قصد جان طفل داشتند فوت شدند.»
\par 20 پس برخاسته، طفل و مادر او را برداشت و به زمین اسرائیل آمد.
\par 21 اما چون شنید که ارکلاوس به‌جای پدر خود هیرودیس بر یهودیه پادشاهی می‌کند، از رفتن بدان سمت ترسید و در خواب وحی یافته، به نواحی جلیل برگشت.و آمده در بلده‌ای مسمی به ناصره ساکن شد، تا آنچه به زبان انبیا گفته شده بود تمام شود که «به ناصری خوانده خواهد شد.»
\par 22 و آمده در بلده‌ای مسمی به ناصره ساکن شد، تا آنچه به زبان انبیا گفته شده بود تمام شود که «به ناصری خوانده خواهد شد.»

\chapter{3}

\par 1 و در آن ایام، یحیی تعمید‌دهنده در بیابان یهودیه ظاهر شد و موعظه کرده، می‌گفت:
\par 2 «توبه کنید، زیرا ملکوت آسمان نزدیک است.»
\par 3 زیرا همین است آنکه اشعیای نبی از او خبرداده، می‌گوید: «صدای ندا کننده‌ای در بیابان که راه خداوند را مهیا سازید و طرق او را راست نمایید.»
\par 4 و این یحیی لباس از پشم شترمی داشت، و کمربند چرمی بر کمر و خوراک او ازملخ و عسل بری می‌بود.
\par 5 در این وقت، اورشلیم و تمام یهودیه وجمیع حوالی اردن نزد او بیرون می‌آمدند،
\par 6 و به گناهان خود اعتراف کرده، در اردن از وی تعمیدمی یافتند.
\par 7 پس چون بسیاری از فریسیان و صدوقیان رادید که بجهت تعمید وی می‌آیند، بدیشان گفت: «ای افعی‌زادگان، که شما را اعلام کرد که از غضب آینده بگریزید؟
\par 8 اکنون ثمره شایسته توبه بیاورید،
\par 9 و این سخن را به‌خاطر خود راه مدهیدکه پدر ما ابراهیم است، زیرا به شما می‌گویم خداقادر است که از این سنگها فرزندان برای ابراهیم برانگیزاند.
\par 10 و الحال تیشه بر ریشه درختان نهاده شده است، پس هر درختی که ثمره نیکونیاورد، بریده و در آتش افکنده شود.
\par 11 من شمارا به آب به جهت توبه تعمید می‌دهم. لکن او که بعد از من می‌آید از من تواناتر است که لایق برداشتن نعلین او نیستم؛ او شما را به روح‌القدس و آتش تعمید خواهد داد.
\par 12 او غربال خود را دردست دارد و خرمن خود را نیکو پاک کرده، گندم خویش را در انبار ذخیره خواهد نمود، ولی کاه رادر آتشی که خاموشی نمی پذیرد خواهدسوزانید.»
\par 13 آنگاه عیسی از جلیل به اردن نزد یحیی آمدتا از او تعمید یابد.
\par 14 اما یحیی او را منع نموده، گفت: «من احتیاج دارم که از تو تعمید یابم و تونزد من می‌آیی؟»
\par 15 عیسی در جواب وی گفت: «الان بگذارزیرا که ما را همچنین مناسب است تا تمام عدالت را به‌کمال رسانیم.»
\par 16 اما عیسی چون تعمیدیافت، فور از آب برآمد که در ساعت آسمان بروی گشاده شد و روح خدا را دید که مثل کبوتری نزول کرده، بر وی می‌آید.آنگاه خطابی ازآسمان در‌رسید که «این است پسر حبیب من که ازاو خشنودم.»
\par 17 آنگاه خطابی ازآسمان در‌رسید که «این است پسر حبیب من که ازاو خشنودم.»

\chapter{4}

\par 1 آنگاه عیسی به‌دست روح به بیابان برده شدتا ابلیس او را تجربه نماید.
\par 2 و چون چهل شبانه‌روز روزه داشت، آخر گرسنه گردید.
\par 3 پس تجربه کننده نزد او آمده، گفت: «اگر پسر خداهستی، بگو تا این سنگها نان شود.»
\par 4 در جواب گفت: «مکتوب است انسان نه محض نان زیست می‌کند، بلکه به هر کلمه‌ای که از دهان خدا صادرگردد.»
\par 5 آنگاه ابلیس او را به شهر مقدس برد و برکنگره هیکل برپا داشته،
\par 6 به وی گفت: «اگر پسرخدا هستی، خود را به زیر انداز، زیرا مکتوب است که فرشتگان خود را درباره تو فرمان دهد تاتو را به‌دستهای خود برگیرند، مبادا پایت به سنگی خورد.»
\par 7 عیسی وی را گفت: «و نیزمکتوب است خداوند خدای خود را تجربه مکن.»
\par 8 پس ابلیس او را به کوهی بسیار بلند برد وهمه ممالک جهان و جلال آنها را بدو نشان داده،
\par 9 به وی گفت: «اگر افتاده مرا سجده کنی، همانا این همه را به تو بخشم.»
\par 10 آنگاه عیسی وی راگفت: «دور شو‌ای شیطان، زیرا مکتوب است که خداوند خدای خود را سجده کن و او را فقطعبادت نما.»
\par 11 در ساعت ابلیس او را رها کرد واینک فرشتگان آمده، او را پرستاری می‌نمودند.
\par 12 و چون عیسی شنید که یحیی گرفتار شده است، به جلیل روانه شد،
\par 13 و ناصره را ترک کرده، آمد و به کفرناحوم، به کناره دریا در حدودزبولون و نفتالیم ساکن شد.
\par 14 تا تمام گردد آنچه به زبان اشعیای نبی گفته شده بود
\par 15 که «زمین زبولون و زمین نفتالیم، راه دریا آن طرف اردن، جلیل امت‌ها؛
\par 16 قومی که در ظلمت ساکن بودند، نوری عظیم دیدند و برنشینندگان دیار موت وسایه آن نوری تابید.»
\par 17 از آن هنگام عیسی به موعظه شروع کرد و گفت: «توبه کنید زیراملکوت آسمان نزدیک است.»
\par 18 و چون عیسی به کناره دریای جلیل می‌خرامید، دو برادر یعنی شمعون مسمی به پطرس و برادرش اندریاس را دید که دامی دردریا می‌اندازند، زیرا صیاد بودند.
\par 19 بدیشان گفت: «از عقب من آیید تا شما را صیاد مردم گردانم.»
\par 20 در ساعت دامها را گذارده، از عقب او روانه شدند.
\par 21 و چون از آنجا گذشت دو برادردیگر یعنی یعقوب، پسر زبدی و برادرش یوحنارا دید که در کشتی با پدر خویش زبدی، دامهای خود را اصلاح می‌کنند؛ ایشان را نیز دعوت نمود.
\par 22 در حال، کشتی و پدر خود را ترک کرده، از عقب او روانه شدند.
\par 23 و عیسی در تمام جلیل می‌گشت و درکنایس ایشان تعلیم داده، به بشارت ملکوت موعظه همی نمود و هر مرض و هر درد قوم راشفا می‌داد.
\par 24 و اسم او در تمام سوریه شهرت یافت، و جمیع مریضانی که به انواع امراض ودردها مبتلا بودند و دیوانگان و مصروعان ومفلوجان را نزد او آوردند، و ایشان را شفا بخشید.و گروهی بسیار از جلیل و دیکاپولس واورشلیم و یهودیه و آن طرف اردن در عقب اوروانه شدند.
\par 25 و گروهی بسیار از جلیل و دیکاپولس واورشلیم و یهودیه و آن طرف اردن در عقب اوروانه شدند.

\chapter{5}

\par 1 خوشابحالها و گروهی بسیار دیده، بر فراز کوه آمد و وقتی که او بنشست شاگردانش نزد اوحاضر شدند.
\par 2 آنگاه دهان خود را گشوده، ایشان را تعلیم داد و گفت:
\par 3 «خوشابحال مسکینان در روح، زیرا ملکوت آسمان از آن ایشان است.
\par 4 خوشابحال ماتمیان، زیرا ایشان تسلی خواهند یافت.
\par 5 خوشابحال حلیمان، زیرا ایشان وارث زمین خواهند شد.
\par 6 خوشابحال گرسنگان و تشنگان عدالت، زیراایشان سیر خواهندشد.
\par 7 خوشابحال رحم کنندگان، زیرا بر ایشان رحم کرده خواهدشد.
\par 8 خوشابحال پاک دلان، زیرا ایشان خدا راخواهند دید.
\par 9 خوشابحال صلح کنندگان، زیراایشان پسران خدا خوانده خواهند شد.
\par 10 خوشابحال زحمت کشان برای عدالت، زیراملکوت آسمان از آن ایشان است.
\par 11 خوشحال باشید چون شما را فحش گویند و جفا رسانند، وبخاطر من هر سخن بدی بر شما کاذبانه گویند.
\par 12 خوش باشید و شادی عظیم نمایید، زیرا اجر شما در آسمان عظیم است زیرا که به همینطور بر انبیای قبل از شما جفامی رسانیدند.
\par 13 «شما نمک جهانید! لیکن اگر نمک فاسدگردد، به کدام چیز باز نمکین شود؟ دیگر مصرفی ندارد جز آنکه بیرون افکنده، پایمال مردم شود.
\par 14 شما نور عالمید. شهری که بر کوهی بنا شود، نتوان پنهان کرد.
\par 15 و چراغ را نمی افروزند تا آن را زیر پیمانه نهند، بلکه تا بر چراغدان گذارند؛ آنگاه به همه کسانی که در خانه باشند، روشنایی می‌بخشد.
\par 16 همچنین بگذارید نور شما بر مردم بتابد تا اعمال نیکوی شما را دیده، پدر شما را که در آسمان است تمجید نمایند.
\par 17 «گمان مبرید که آمده‌ام تا تورات یا صحف انبیا را باطل سازم. نیامده‌ام تا باطل نمایم بلکه تاتمام کنم.
\par 18 زیرا هر آینه به شما می‌گویم، تاآسمان و زمین زایل نشود، همزه یا نقطه‌ای ازتورات هرگز زایل نخواهد شد تا همه واقع شود.
\par 19 پس هر‌که یکی از این احکام کوچکترین رابشکند و به مردم چنین تعلیم دهد، در ملکوت آسمان کمترین شمرده شود. اما هر‌که بعمل آورد و تعلیم نماید، او در ملکوت آسمان بزرگ خوانده خواهد شد.
\par 20 زیرا به شما می‌گویم، تاعدالت شما بر عدالت کاتبان و فریسیان افزون نشود، به ملکوت آسمان هرگز داخل نخواهیدشد.
\par 21 «شنیده‌اید که به اولین گفته شده است "قتل مکن و هر‌که قتل کند سزاوار حکم شود."
\par 22 لیکن من به شما می‌گویم، هر‌که به برادر خود بی‌سبب خشم گیرد، مستوجب حکم باشد و هر‌که برادرخود را راقا گوید، مستوجب قصاص باشد و هرکه احمق گوید، مستحق آتش جهنم بود.
\par 23 پس هرگاه هدیه خود را به قربانگاه ببری و آنجا به‌خاطرت آید که برادرت بر تو حقی دارد،
\par 24 هدیه خود را پیش قربانگاه واگذار و رفته، اول با برادرخویش صلح نما و بعد آمده، هدیه خود رابگذران.
\par 25 با مدعی خود مادامی که با وی در راه هستی صلح کن، مبادا مدعی، تو را به قاضی سپارد و قاضی، تو را به داروغه تسلیم کند و در زندان افکنده شوی.
\par 26 هرآینه به تو می‌گویم، که تا فلس آخر را ادا نکنی، هرگز از آنجا بیرون نخواهی آمد.
\par 27 «شنیده‌اید که به اولین گفته شده است "زنامکن."
\par 28 لیکن من به شما می‌گویم، هر کس به زنی نظر شهوت اندازد، همان دم در دل خود با او زناکرده است.
\par 29 پس اگر چشم راستت تو رابلغزاند، قلعش کن و از خود دور انداز زیرا تو رابهتر آن است که عضوی از اعضایت تباه گردد، ازآنکه تمام بدنت در جهنم افکنده شود.
\par 30 و اگردست راستت تو را بلغزاند، قطعش کن و از خوددور انداز، زیرا تو را مفیدتر آن است که عضوی ازاعضای تو نابود شود، از آنکه کل جسدت دردوزخ افکنده شود.
\par 31 «و گفته شده است هر‌که از زن خودمفارقت جوید، طلاق نامه‌ای بدو بدهد.
\par 32 لیکن من به شما می‌گویم، هر کس بغیر علت زنا، زن خود را از خود جدا کند باعث زنا کردن اومی باشد، و هر‌که زن مطلقه را نکاح کند، زنا کرده باشد.
\par 33 «باز شنیده‌اید که به اولین گفته شده است که "قسم دروغ مخور، بلکه قسم های خود را به خداوند وفا کن."
\par 34 لیکن من به شما می‌گویم، هرگز قسم مخورید، نه به آسمان زیرا که عرش خداست،
\par 35 و نه به زمین زیرا که پای انداز اواست، و نه به ارشلیم زیرا که شهر پادشاه عظیم است،
\par 36 و نه به‌سر خود قسم یاد کن، زیرا که مویی را سفید یا سیاه نمی توانی کرد.
\par 37 بلکه سخن شما بلی بلی و نی نی باشد زیرا که زیاده براین از شریر است.
\par 38 «شنیده‌اید که گفته شده است."چشمی به چشمی و دندانی به دندانی "
\par 39 لیکن من به شمامی گویم، با شریر مقاومت مکنید بلکه هر‌که بررخساره راست تو طپانچه زند، دیگری را نیز به سوی او بگردان،
\par 40 و اگر کسی خواهد با تو دعواکند و قبای تو را بگیرد، عبای خود را نیز بدوواگذار،
\par 41 و هرگاه کسی تو را برای یک میل مجبور سازد، دو میل همراه او برو.
\par 42 هر کس ازتو سوال کند، بدو ببخش و از کسی‌که قرض از توخواهد، روی خود را مگردان.»
\par 43 «شنیده‌اید که گفته شده است "همسایه خورا محبت نما و با دشمن خود عداوت کن."
\par 44 امامن به شما می‌گویم که دشمنان خود را محبت نمایید و برای لعن کنندگان خود برکت بطلبید و به آنانی که از شما نفرت کنند، احسان کنید و به هرکه به شما فحش دهد و جفا رساند، دعای خیرکنید،
\par 45 تا پدر خود را که در آسمان است پسران شوید، زیرا که آفتاب خود را بر بدان و نیکان طالع می‌سازد و باران بر عادلان و ظالمان می‌باراند.
\par 46 زیرا هرگاه آنانی را محبت نمایید که شما رامحبت می‌نمایند، چه اجر دارید؟ آیا باجگیران چنین نمی کنند؟
\par 47 و هرگاه برادران خود را فقطسلام گویید چه فضیلت دارید؟ آیا باجگیران چنین نمی کنند؟پس شما کامل باشید چنانکه پدر شما که در آسمان است کامل است.
\par 48 پس شما کامل باشید چنانکه پدر شما که در آسمان است کامل است.

\chapter{6}

\par 1 «زنهار عدالت خود را پیش مردم به‌جامیاورید تا شما را ببینند و الا نزد پدر خودکه در آسمان است، اجری ندارید.
\par 2 پس چون صدقه دهی، پیش خود کرنا منواز چنانکه ریاکاران در کنایس و بازارها می‌کنند، تا نزد مردم اکرام یابند. هرآینه به شما می‌گویم اجر خود رایافته‌اند.
\par 3 بلکه تو چون صدقه دهی، دست چپ تو از آنچه دست راستت می‌کند مطلع نشود،
\par 4 تاصدقه تو در نهان باشد و پدر نهان بین تو، تو راآشکارا اجر خواهد داد.
\par 5 «و چون عبادت کنی، مانند ریاکاران مباش زیرا خوش دارند که در کنایس و گوشه های کوچه‌ها ایستاده، نماز گذارند تا مردم ایشان راببینند. هرآینه به شما می‌گویم اجر خود راتحصیل نموده‌اند.
\par 6 لیکن تو چون عبادت کنی، به حجره خود داخل شو و در را بسته، پدر خود راکه در نهان است عبادت نما؛ و پدر نهان بین تو، تو را آشکارا جزا خواهد داد.
\par 7 و چون عبادت کنید، مانند امت‌ها تکرار باطل مکنید زیرا ایشان گمان می‌برند که به‌سبب زیاد گفتن مستجاب می‌شوند.
\par 8 پس مثل ایشان مباشید زیرا که پدر شماحاجات شما را می‌داند پیش از آنکه از او سوال کنید.
\par 9 «پس شما به اینطور دعا کنید: "ای پدر ما که در آسمانی، نام تو مقدس باد.
\par 10 ملکوت تو بیاید. اراده تو چنانکه در آسمان است، بر زمین نیز کرده شود.
\par 11 نان کفاف ما را امروز به ما بده.
\par 12 و قرض های ما را ببخش چنانکه ما نیزقرضداران خود را می‌بخشیم.
\par 13 و ما را در آزمایش میاور، بلکه از شریر ما رارهایی ده.
\par 14 «زیرا هرگاه تقصیرات مردم را بدیشان بیامرزید، پدر آسمانی شما، شما را نیز خواهدآمرزید.
\par 15 اما اگر تقصیرهای مردم را نیامرزید، پدر شما هم تقصیرهای شما را نخواهد آمرزید.
\par 16 «اما چون روزه‌دارید، مانند ریاکاران ترشرو مباشید زیرا که صورت خویش را تغییرمی دهند تا در نظر مردم روزه‌دار نمایند. هرآینه به شما می‌گویم اجر خود را یافته‌اند.
\par 17 لیکن توچون روزه‌داری، سر خود را تدهین کن و روی خود را بشوی
\par 18 تا در نظر مردم روزه‌دار ننمایی، بلکه در حضور پدرت که در نهان است؛ و پدر نهان بین تو، تو را آشکارا جزا خواهد داد.
\par 19 «گنجها برای خود بر زمین نیندوزید، جایی که بید و زنگ زیان می‌رساند و جایی که دزدان نقب می‌زنند و دزدی می‌نمایند.
\par 20 بلکه گنجهابجهت خود در آسمان بیندوزید، جایی که بید وزنگ زیان نمی رساند و جایی که دزدان نقب نمی زنند و دزدی نمی کنند.
\par 21 زیرا هرجا گنج تواست دل تو نیز در آنجا خواهد بود.
\par 22 «چراغ بدن چشم است؛ پس هرگاه چشمت بسیط باشد تمام بدنت روشن بود؛
\par 23 امااگر چشم تو فاسد است، تمام جسدت تاریک می‌باشد. پس اگر نوری که در تو است ظلمت باشد، چه ظلمت عظیمی است!
\par 24 «هیچ‌کس دو آقا را خدمت نمی تواند کرد، زیرا یا از یکی نفرت دارد و با دیگری محبت، و یابه یکی می‌چسبد و دیگر را حقیر می‌شمارد. محال است که خدا و ممونا را خدمت کنید.
\par 25 «بنابراین به شما می‌گویم، از بهر جان خوداندیشه مکنید که چه خورید یا چه آشامید و نه برای بدن خود که چه بپوشید. آیا جان، از خوراک و بدن از پوشاک بهتر نیست؟
\par 26 مرغان هوا را نظرکنید که نه می‌کارند و نه می‌دروند و نه در انبارهاذخیره می‌کنند و پدر آسمانی شما آنها رامی پروراند. آیا شما از آنها بمراتب بهتر نیستید؟
\par 27 و کیست از شما که به تفکر بتواند ذراعی برقامت خود افزاید؟
\par 28 و برای لباس چرا می اندیشید؟ در سوسنهای چمن تامل کنید، چگونه نمو می‌کنند! نه محنت می‌کشند و نه می‌ریسند!
\par 29 لیکن به شما می‌گویم سلیمان هم باهمه جلال خود چون یکی از آنها آراسته نشد.
\par 30 پس اگر خدا علف صحرا را که امروز هست وفردا در تنور افکنده می‌شود چنین بپوشاند، ای کم‌ایمانان آیا نه شما را از طریق اولی؟
\par 31 پس اندیشه مکنید و مگویید چه بخوریم یا چه بنوشیم یا چه بپوشیم.
\par 32 زیرا که در طلب جمیع این چیزها امت‌ها می‌باشند. اما پدر آسمانی شمامی داند که بدین همه‌چیز احتیاج دارید.
\par 33 لیکن اول ملکوت خدا و عدالت او را بطلبید که این همه برای شما مزید خواهد شد.پس در اندیشه فردا مباشید زیرا فردا اندیشه خود را خواهد کرد. بدی امروز برای امروز کافی است.
\par 34 پس در اندیشه فردا مباشید زیرا فردا اندیشه خود را خواهد کرد. بدی امروز برای امروز کافی است.

\chapter{7}

\par 1 «حکم مکنید تا بر شما حکم نشود.
\par 2 زیرابدان طریقی که حکم کنید بر شما نیز حکم خواهد شد و بدان پیمانه‌ای که پیمایید برای شماخواهند پیمود.
\par 3 و چون است که خس را درچشم برادر خود می‌بینی و چوبی را که چشم خود داری نمی یابی؟
\par 4 یا چگونه به برادر خودمی گویی "اجازت ده تا خس را از چشمت بیرون کنم " و اینک چوب در چشم تو است؟
\par 5 ‌ای ریاکار، اول چوب را از چشم خود بیرون کن، آنگاه نیک خواهی دید تا خس را از چشم برادرت بیرون کنی!
\par 6 «آنچه مقدس است، به سگان مدهید و نه مرواریدهای خود را پیش گرازان اندازید، مباداآنها را پایمال کنند و برگشته، شما را بدرند.
\par 7 «سوال کنید که به شما داده خواهد شد؛ بطلبید که خواهید یافت؛ بکوبید که برای شما بازکرده خواهد شد.
\par 8 زیرا هر‌که سوال کند، یابد وکسی‌که بطلبد دریافت کند و هر‌که بکوبد برای اوگشاده خواهد شد.
\par 9 و کدام آدمی است از شما که پسرش نانی از او خواهد و سنگی بدو دهد؟
\par 10 یااگر ماهی خواهد ماری بدو بخشد؟
\par 11 پس هرگاه شما که شریر هستید، دادن بخشش های نیکو را به اولاد خود می‌دانید، چقدر زیاده پدر شما که درآسمان است چیزهای نیکو را به آنانی که از اوسوال می‌کنند خواهد بخشید!
\par 12 لهذا آنچه خواهید که مردم به شما کنند، شما نیز بدیشان همچنان کنید، زیرا این است تورات و صحف انبیا.
\par 13 «از در تنگ داخل شوید. زیرا فراخ است آن در و وسیع است آن طریقی که مودی به هلاکت است و آنانی که بدان داخل می‌شوندبسیارند.
\par 14 زیرا تنگ است آن در و دشوار است آن طریقی که مودی به حیات‌است و یابندگان آن کمند.
\par 15 «اما از انبیای کذبه احتراز کنید، که به لباس میشها نزد شما می‌آیند ولی در باطن، گرگان درنده می‌باشند.
\par 16 ایشان را از میوه های ایشان خواهید شناخت. آیا انگور را از خار و انجیر را ازخس می‌چینند؟
\par 17 همچنین هر درخت نیکو، میوه نیکو می‌آورد و درخت بد، میوه بد می‌آورد.
\par 18 نمی تواند درخت خوب میوه بد آورد، و نه درخت بد میوه نیکو آورد.
\par 19 هر درختی که میوه نیکو نیاورد، بریده و در آتش افکنده شود.
\par 20 لهذا از میوه های ایشان، ایشان را خواهیدشناخت.
\par 21 «نه هر‌که مرا "خداوند، خداوند" گویدداخل ملکوت آسمان گردد، بلکه آنکه اراده پدرمرا که در آسمان است به‌جا آورد.
\par 22 بسا در آن روز مرا خواهند گفت: "خداوندا، خداوندا، آیا به نام تو نبوت ننمودیم و به اسم تو دیوها را اخراج نکردیم و به نام تو معجزات بسیار ظاهرنساختیم؟"
\par 23 آنگاه به ایشان صریح خواهم گفت که "هرگز شما را نشناختم! ای بدکاران از من دور شوید!"
\par 24 «پس هر‌که این سخنان مرا بشنود و آنها رابه‌جا آرد، او را به مردی دانا تشبیه می‌کنم که خانه خود را بر سنگ بنا کرد.
\par 25 و باران باریده، سیلابهاروان گردید و بادها وزیده، بدان خانه زورآور شدو خراب نگردید زیرا که بر سنگ بنا شده بود.
\par 26 وهر‌که این سخنان مرا شنیده، به آنها عمل نکرد، به مردی نادان ماند که خانه خود را بر ریگ بنا نهاد.
\par 27 و باران باریده، سیلابها جاری شد و بادهاوزیده، بدان خانه زور آورد و خراب گردید وخرابی آن عظیم بود.»
\par 28 و چون عیسی این سخنان را ختم کرد آن گروه از تعلیم او در حیرت افتادند،زیرا که ایشان را چون صاحب قدرت تعلیم می‌داد و نه مثل کاتبان.
\par 29 زیرا که ایشان را چون صاحب قدرت تعلیم می‌داد و نه مثل کاتبان.

\chapter{8}

\par 1 و چون او از کوه به زیر آمد، گروهی بسیاراز عقب او روانه شدند.
\par 2 که ناگاه ابرصی آمد و او را پرستش نموده، گفت: «ای خداوند اگربخواهی، می‌توانی مرا طاهر سازی.»
\par 3 عیسی دست آورده، او را لمس نمود و گفت: «می‌خواهم؛ طاهر شو!» که فور برص او طاهرگشت.
\par 4 عیسی بدو گفت: «زنهار کسی را اطلاع ندهی بلکه رفته، خود را به کاهن بنما و آن هدیه‌ای را که موسی فرمود، بگذران تا بجهت ایشان شهادتی باشد.»
\par 5 و چون عیسی وارد کفرناحوم شد، یوزباشی‌ای نزد وی آمد و بدو التماس نموده،
\par 6 گفت: «ای خداوند، خادم من مفلوج در خانه خوابیده و بشدت متالم است.»
\par 7 عیسی بدو گفت: «من آمده، او را شفا خواهم داد.»
\par 8 یوزباشی در جواب گفت: «خداوندا، لایق آن نی‌ام که زیرسقف من آیی. بلکه فقط سخنی بگو و خادم من صحت خواهد یافت.
\par 9 زیرا که من نیز مردی زیرحکم هستم و سپاهیان را زیر دست خود دارم؛ چون به یکی گویم برو، می‌رود و به دیگری بیا، می‌آید و به غلام خود فلان کار را بکن، می‌کند.»
\par 10 عیسی چون این سخن را شنید، متعجب شده، به همراهان خود گفت: «هرآینه به شما می‌گویم که چنین ایمانی در اسرائیل هم نیافته‌ام.
\par 11 و به شما می‌گویم که بسا از مشرق و مغرب آمده، درملکوت آسمان با ابراهیم و اسحاق و یعقوب خواهند نشست؛
\par 12 اما پسران ملکوت بیرون افکنده خواهند شد، در ظلمت خارجی جایی که گریه و فشار دندان باشد.
\par 13 پس عیسی به یوزباشی گفت: «برو، بر وفق ایمانت تو را عطاشود، » که در ساعت خادم او صحت یافت.
\par 14 و چون عیسی به خانه پطرس آمد، مادر‌زن او را دید که تب کرده، خوابیده است.
\par 15 پس دست او را لمس کرد و تب او را رها کرد. پس برخاسته، به خدمت گذاری ایشان مشغول گشت.
\par 16 اما چون شام شد، بسیاری از دیوانگان را به نزداو آوردند و محض سخنی ارواح را بیرون کرد وهمه مریضان را شفا بخشید.
\par 17 تا سخنی که به زبان اشعیای نبی گفته شده بود تمام گردد که «او ضعف های ما را گرفت و مرض های ما رابرداشت.»
\par 18 چون عیسی جمعی کثیر دور خود دید، فرمان داد تا به کناره دیگر روند.
\par 19 آنگاه کاتبی پیش آمده، بدو گفت: «استادا هرجا روی، تو رامتابعت کنم.»
\par 20 عیسی بدو گفت: «روباهان راسوراخها و مرغان هوا را آشیانه‌ها است. لیکن پسر انسان را جای سر نهادن نیست.»
\par 21 و دیگری از شاگردانش بدو گفت: «خداوندا اول مرارخصت ده تا رفته، پدر خود را دفن کنم.»
\par 22 عیسی وی را گفت: «مرا متابعت کن و بگذار که مردگان، مردگان خود را دفن کنند.»
\par 23 چون به کشتی سوار شد، شاگردانش ازعقب او آمدند.
\par 24 ناگاه اضطراب عظیمی در دریاپدید آمد، بحدی که امواج، کشتی را فرومی گرفت؛ و او در خواب بود.
\par 25 پس شاگردان پیش آمده، او را بیدار کرده، گفتند: «خداوندا، مارا دریاب که هلاک می‌شویم!»
\par 26 بدیشان گفت: «ای کم‌ایمانان، چرا ترسان هستید؟» آنگاه برخاسته، بادها و دریا را نهیب کرد که آرامی کامل پدید آمد.
\par 27 اما آن اشخاص تعجب نموده، گفتند: «این چگونه مردی است که بادها و دریا نیزاو را اطاعت می‌کنند!»
\par 28 و چون به آن کناره در زمین جرجسیان رسید، دو شخص دیوانه از قبرها بیرون شده، بدوبرخوردند و بحدی تندخوی بودند که هیچ‌کس از آن راه نتوانستی عبور کند.
\par 29 در ساعت فریادکرده، گفتند: «یا عیسی ابن‌الله، ما را با تو جه کار است؟ مگر در اینجا آمده‌ای تا ما را قبل از وقت عذاب کنی؟"
\par 30 و گلۀگراز بسیاری دور از ایشان می‌چرید.
\par 31 دیوها از وی استدعا نموده، گفتند: "هرگاه ما را بیرون کنی، در گلۀ گرازان ما را بفرست."
\par 32 ایشان را گفت: "بروید!" در حال بیرون شده، داخل گلۀ گرازان گردیدند که فی‌الفور همۀ آن گرازان از بلندی به دریا جسته، در آب هلاک شدند.
\par 33 اما شبانان گریخته، به شهر رفتند و تمام آن حادثه و ماجرای دیوانگان را شهرت دادند.واینک تمام شهر برای ملاقات عیسی بیرون آمد. چون او را دیدند، التماس نمودند که از حدودایشان بیرون رود.
\par 34 واینک تمام شهر برای ملاقات عیسی بیرون آمد. چون او را دیدند، التماس نمودند که از حدودایشان بیرون رود.

\chapter{9}

\par 1 پس به کشتی سوار شده، عبور کرد و به شهر خویش آمد.
\par 2 ناگاه مفلوجی را بر بستر خوابانیده، نزد وی آوردند. چون عیسی ایمان ایشان را دید، مفلوج را گفت: «ای فرزند، خاطر جمع دار که گناهانت آمرزیده شد.»
\par 3 آنگاه بعضی از کاتبان با خود گفتند: «این شخص کفر می‌گوید.»
\par 4 عیسی خیالات ایشان رادرک نموده، گفت: «از بهر‌چه خیالات فاسد به‌خاطر خود راه می‌دهید؟
\par 5 زیرا کدام سهل تراست، گفتن اینکه گناهان تو آمرزیده شد یا گفتن آنکه برخاسته بخرام؟
\par 6 لیکن تا بدانید که پسرانسان را قدرت آمرزیدن گناهان بر روی زمین هست...» آنگاه مفلوج را گفت: «برخیز و بسترخود را برداشته، به خانه خود روانه شو!»
\par 7 درحال برخاسته، به خانه خود رفت!
\par 8 و آن گروه چون این عمل را دیدند، متعجب شده، خدایی راکه این نوع قدرت به مردم عطا فرموده بود، تمجیدنمودند.
\par 9 چون عیسی از آنجا می‌گذشت، مردی رامسمی به متی به باج گاه نشسته دید. بدو گفت: «مرا متابعت کن.» در حال برخاسته، از عقب وی روانه شد.
\par 10 و واقع شد چون او در خانه به غذا نشسته بود که جمعی از باجگیران وگناهکاران آمده، با عیسی و شاگردانش بنشستند.
\par 11 «و فریسیان چون دیدند، به شاگردان اوگفتند: «چرا استاد شما با باجگیران وگناهکاران غذا می‌خورد؟»
\par 12 عیسی چون شنید، گفت: «نه تندرستان بلکه مریضان احتیاج به طبیب دارند.
\par 13 لکن رفته، این را دریافت کنید که "رحمت می‌خواهم نه قربانی "، زیرا نیامده‌ام تاعادلان را بلکه گناهکاران را به توبه دعوت نمایم.»
\par 14 آنگاه شاگردان یحیی نزد وی آمده، گفتند: «چون است که ما و فریسیان روزه بسیارمی داریم، لکن شاگردان تو روزه نمی دارند؟»
\par 15 عیسی بدیشان گفت: «آیا پسران خانه عروسی، مادامی که داماد با ایشان است، می‌توانند ماتم کنند؟ و لکن ایامی می‌آید که داماد از ایشان گرفته شود؛ در آن هنگام روزه خواهند داشت.
\par 16 وهیچ‌کس بر جامه کهنه پاره‌ای از پارچه نو وصله نمی کند زیرا که آن وصله از جامه جدا می‌گردد ودریدگی بدتر می‌شود.
\par 17 و شراب نو را درمشکهای کهنه نمی ریزند والا مشکها دریده شده، شراب ریخته و مشکها تباه گردد. بلکه شراب نو را در مشکهای نو می‌ریزند تا هر دومحفوظ باشد.»
\par 18 او هنوز این سخنان را بدیشان می‌گفت که ناگاه رئیسی آمد و او را پرستش نموده، گفت: «اکنون دختر من مرده است. لکن بیا و دست خودرا بر وی گذار که زیست خواهد کرد.»
\par 19 پس عیسی به اتفاق شاگردان خود برخاسته، از عقب او روان شد.
\par 20 و اینک زنی که مدت دوازده سال به مرض استحاضه مبتلا می‌بود، از عقب او آمده، دامن ردای او را لمس نمود،
\par 21 زیرا با خود گفته بود: «اگر محض ردایش را لمس کنم، هرآینه شفایابم.»
\par 22 عیسی برگشته، نظر بر وی انداخته، گفت: «ای دختر، خاطرجمع باش زیرا که ایمانت تو را شفا داده است!» در ساعت آن زن رستگارگردید.
\par 23 و چون عیسی به خانه رئیس در‌آمد، نوحه‌گران و گروهی از شورش کنندگان را دیده،
\par 24 بدیشان گفت: «راه دهید، زیرا دختر نمرده بلکه در خواب است.» ایشان بر وی سخریه کردند.
\par 25 اما چون آن گروه بیرون شدند، داخل شده، دست آن دختر را گرفت که در ساعت برخاست.
\par 26 و این کار در تمام آن مرز وبوم شهرت یافت.
\par 27 و چون عیسی از آن مکان می‌رفت، دو کورفریاد‌کنان در عقب او افتاده، گفتند: «پسر داودا، بر ما ترحم کن!»
\par 28 و چون به خانه در‌آمد، آن دوکور نزد او آمدند. عیسی بدیشان گفت: «آیا ایمان دارید که این کار را می‌توانم کرد؟» گفتندش: «بلی خداوندا.»
\par 29 در ساعت چشمان ایشان را لمس کرده، گفت: «بر وفق ایمانتان به شما بشود.»
\par 30 درحال چشمانشان باز شد و عیسی ایشان را به تاکیدفرمود که «زنهار کسی اطلاع نیابد.»
\par 31 اما ایشان بیرون رفته، او را در تمام آن نواحی شهرت دادند.
\par 32 و هنگامی که ایشان بیرون می‌رفتند، ناگاه دیوانه‌ای گنگ را نزد او آوردند.
\par 33 و چون دیوبیرون شد، گنگ، گویا گردید و همه در تعجب شده، گفتند: «در اسرائیل چنین امر هرگز دیده نشده بود.»
\par 34 لیکن فریسیان گفتند: «به واسطه رئیس دیوها، دیوها را بیرون می‌کند.»
\par 35 و عیسی در همه شهرها و دهات گشته، در کنایس ایشان تعلیم داده، به بشارت ملکوت موعظه می‌نمود و هر مرض و رنج مردم را شفامی داد.
\par 36 و چون جمع کثیر دید، دلش بر ایشان بسوخت زیرا که مانند گوسفندان بی‌شبان، پریشان حال و پراکنده بودند.آنگاه به شاگردان خود گفت: «حصاد فراوان است لیکن عمله کم. پس از صاحب حصاد استدعا نمایید تاعمله در حصاد خود بفرستد.»
\par 37 آنگاه به شاگردان خود گفت: «حصاد فراوان است لیکن عمله کم. پس از صاحب حصاد استدعا نمایید تاعمله در حصاد خود بفرستد.»

\chapter{10}

\par 1 و دوازده شاگرد خود را طلبیده، ایشان را بر ارواح پلید قدرت داد که آنها رابیرون کنند و هر بیماری و رنجی را شفا دهند.
\par 2 ونامهای دوازده رسول این است: اول شمعون معروف به پطرس و برادرش اندریاس؛ یعقوب بن زبدی و برادرش یوحنا؛
\par 3 فیلپس و برتولما؛ توماو متای باجگیر؛ یعقوب بن حلفی و لبی معروف به تدی؛
\par 4 شمعون قانوی و یهودای اسخریوطی که او را تسلیم نمود.
\par 5 این دوازده را عیسی فرستاده، بدیشان وصیت کرده، گفت: «از راه امت‌ها مروید و دربلدی از سامریان داخل مشوید،
\par 6 بلکه نزدگوسفندان گم شده اسرائیل بروید.
\par 7 و چون می‌روید، موعظه کرده، گویید که ملکوت آسمان نزدیک است.
\par 8 بیماران را شفا دهید، ابرصان راطاهر سازید، مردگان را زنده کنید، دیوها را بیرون نمایید. مفت یافته‌اید، مفت بدهید.
\par 9 طلا یا نقره یا مس در کمرهای خود ذخیره مکنید،
\par 10 و برای سفر، توشه‌دان یا دو پیراهن یا کفشها یا عصابرندارید، زیرا که مزدور مستحق خوراک خوداست.
\par 11 و در هر شهری یا قریه‌ای که داخل شوید، بپرسید که در آنجا که لیاقت دارد؛ پس درآنجا بمانید تا بیرون روید.
\par 12 و چون به خانه‌ای درآیید، بر آن سلام نمایید؛
\par 13 پس اگر خانه لایق باشد، سلام شما بر آن واقع خواهد شد و اگرنالایق بود، سلام شما به شما خواهد برگشت.
\par 14 و هر‌که شما را قبول نکند یا به سخن شماگوش ندهد، از آن خانه یا شهر بیرون شده، خاک پایهای خود را برافشانید.
\par 15 هرآینه به شمامی گویم که در روز جزا حالت زمین سدوم وغموره از آن شهر سهل تر خواهد بود.
\par 16 هان، من شما را مانند گوسفندان در میان گرگان می‌فرستم؛ پس مثل مارها هوشیار و چون کبوتران ساده باشید.
\par 17 اما از مردم برحذر باشید، زیرا که شما را به مجلسها تسلیم خواهند کرد ودر کنایس خود شما را تازیانه خواهند زد،
\par 18 ودر حضور حکام و سلاطین، شما را بخاطر من خواهند برد تا بر ایشان و بر امت‌ها شهادتی شود.
\par 19 اما چون شما را تسلیم کنند، اندیشه مکنید که چگونه یا چه بگویید زیرا در همان ساعت به شماعطا خواهد شد که چه باید گفت،
\par 20 زیرا گوینده شما نیستید بلکه روح پدر شما، در شما گوینده است.
\par 21 و برادر، برادر را و پدر، فرزند را به موت تسلیم خواهند کرد و فرزندان بر والدین خودبرخاسته، ایشان را به قتل خواهند رسانید؛
\par 22 و به جهت اسم من، جمیع مردم از شما نفرت خواهندکرد. لیکن هر‌که تا به آخر صبر کند، نجات یابد.
\par 23 و وقتی که در یک شهر بر شما جفا کنند، به دیگری فرار کنید زیرا هرآینه به شما می‌گویم تاپسر انسان نیاید، از همه شهرهای اسرائیل نخواهید پرداخت.
\par 24 شاگرد از معلم خود افضل نیست و نه غلام از آقایش برتر.
\par 25 کافی است شاگرد را که چون استاد خویش گردد و غلام را که چون آقای خودشود. پس اگر صاحب‌خانه را بعلزبول خواندند، چقدر زیادتر اهل خانه‌اش را.
\par 26 لهذا از ایشان مترسید زیرا چیزی مستور نیست که مکشوف نگردد و نه مجهولی که معلوم نشود.
\par 27 آنچه درتاریکی به شما می‌گویم، در روشنایی بگویید، وآنچه در گوش شنوید بر بامها موعظه کنید.
\par 28 واز قاتلان جسم که قادر بر کشتن روح نی‌اند، بیم مکنید بلکه از او بترسید که قادر است بر هلاک کردن روح و جسم را نیز در جهنم.
\par 29 آیا دوگنجشک به یک فلس فروخته نمی شود؟ و حال آنکه یکی از آنها جز به حکم پدر شما به زمین نمی افتد.
\par 30 لیکن همه مویهای سر شما نیزشمرده شده است.
\par 31 پس ترسان مباشید زیراشما از گنجشکان بسیار افضل هستید.
\par 32 پس هر‌که مرا پیش مردم اقرار کند، من نیزدر حضور پدر خود که در آسمان است، او رااقرار خواهم کرد.
\par 33 اما هر‌که مرا پیش مردم انکار نماید، من هم در حضور پدر خود که در آسمان است او را انکار خواهم نمود.
\par 34 گمان مبرید که آمده‌ام تا سلامتی بر زمین بگذارم. نیامده‌ام تا سلامتی بگذارم بلکه شمشیر را.
\par 35 زیرا که آمده‌ام تا مرد را از پدر خود و دختر رااز مادر خویش و عروس را از مادر شوهرش جداسازم.
\par 36 و دشمنان شخص، اهل خانه او خواهندبود.
\par 37 و هر‌که پدر یا مادر را بیش از من دوست دارد؛ لایق من نباشد و هر‌که پسر یا دختر را از من زیاده دوست دارد، لایق من نباشد.
\par 38 و هر‌که صلیب خود را برنداشته، از عقب من نیاید، لایق من نباشد.
\par 39 هر‌که جان خود را دریابد، آن راهلاک سازد و هر‌که جان خود را بخاطر من هلاک کرد، آن را خواهد دریافت.
\par 40 هر‌که شما را قبول کند، مرا قبول کرده و کسی‌که مرا قبول کرده، فرستنده مرا قبول کرده باشد.
\par 41 و آنکه نبی‌ای رابه اسم نبی پذیرد، اجرت نبی یابد و هر‌که عادلی را به اسم عادلی پذیرفت، مزد عادل را خواهدیافت.و هر‌که یکی از این صغار را کاسه‌ای ازآب سرد محض نام شاگرد نوشاند، هرآینه به شمامی گویم اجر خودرا ضایع نخواهد ساخت.»
\par 42 و هر‌که یکی از این صغار را کاسه‌ای ازآب سرد محض نام شاگرد نوشاند، هرآینه به شمامی گویم اجر خودرا ضایع نخواهد ساخت.»

\chapter{11}

\par 1 و چون عیسی این وصیت را با دوازده شاگرد خود به اتمام رسانید، از آنجا روانه شد تا در شهرهای ایشان تعلیم دهد و موعظه نماید.
\par 2 و چون یحیی در زندان، اعمال مسیح راشنید، دو نفر از شاگردان خود را فرستاده،
\par 3 بدو گفت: «آیا آن آینده تویی یا منتظر دیگری باشیم؟»
\par 4 عیسی در جواب ایشان گفت: «بروید ویحیی را از آنچه شنیده و دیده‌اید، اطلاع دهید
\par 5 که کوران بینا می‌گردند و لنگان به رفتار می‌آیندو ابرصان طاهر و کران شنوا و مردگان زنده می‌شوند و فقیران بشارت می‌شنوند؛
\par 6 وخوشابحال کسی‌که در من نلغزد.»
\par 7 و چون ایشان می‌رفتند، عیسی با آن جماعت درباره یحیی آغاز سخن کرد که «بجهت دیدن چه چیز بیرون شدید؟ آیا نی را که از باد درجنبش است؟
\par 8 بلکه بجهت دیدن چه چیز بیرون شدید؟ آیا مردی را که لباس فاخر در بر دارد؟ اینک آنانی که رخت فاخر می‌پوشند در خانه های پادشاهان می‌باشند.
\par 9 لیکن بجهت دیدن چه چیزبیرون رفتید؟ آیا نبی را؟ بلی به شما می‌گویم ازنبی افضلی را!
\par 10 زیرا همان است آنکه درباره اومکتوب است: "اینک من رسول خود را پیش روی تو می‌فرستم تا راه تو را پیش روی تو مهیا سازد."
\par 11 هرآینه به شما می‌گویم که از اولاد زنان، بزرگتری از یحیی تعمید‌دهنده برنخاست، لیکن کوچکتر در ملکوت آسمان از وی بزرگ تر است.
\par 12 و از ایام یحیی تعمید‌دهنده تا الان، ملکوت آسمان محبور می‌شود و جباران آن را به زورمی ربایند.
\par 13 زیرا جمیع انبیا و تورات تا یحیی اخبار می‌نمودند.
\par 14 و اگر خواهید قبول کنید، همان است الیاس که باید بیاید.
\par 15 هر‌که گوش شنوا دارد بشنود.
\par 16 لیکن این طایفه را به چه چیزتشبیه نمایم؟ اطفالی را مانند که در کوچه هانشسته، رفیقان خویش را صدا زده،
\par 17 می‌گویند: "برای شما نی نواختیم، رقص نکردید؛ نوحه گری کردیم، سینه نزدید."
\par 18 زیرا که یحیی آمد، نه می‌خورد و نه می‌آشامید، می‌گویند دیو دارد.
\par 19 پسر انسان آمد که می‌خورد و می‌نوشد، می‌گویند اینک مردی پرخور و میگسار و دوست باجگیران و گناهکاران است. لیکن حکمت ازفرزندان خود تصدیق کرده شده است."» وعده آرامش درون
\par 20 آنگاه شروع به ملامت نمود بر آن شهرهایی که اکثر از معجزات وی در آنها ظاهرشد زیرا که توبه نکرده بودند:
\par 21 «وای بر تو‌ای خورزین! وای بر تو‌ای بیت صیدا! زیرا اگرمعجزاتی که در شما ظاهر گشت، در صور وصیدون ظاهر می‌شد، هرآینه مدتی در پلاس وخاکستر توبه می‌نمودند.
\par 22 لیکن به شما می‌گویم که در روز جزا حالت صور و صیدون از شماسهلتر خواهد بود.
\par 23 و تو‌ای کفرناحوم که تا به فلک سرافراشته‌ای، به جهنم سرنگون خواهی شد زیرا هرگاه معجزاتی که در تو پدید آمد درسدوم ظاهر می‌شد، هرآینه تا امروز باقی می‌ماند.
\par 24 لیکن به شما می‌گویم که در روز جزاحالت زمین سدوم از تو سهل تر خواهد بود.»
\par 25 در آن وقت، عیسی توجه نموده، گفت: «ای پدر، مالک آسمان و زمین، تو را ستایش می‌کنم که این امور را از دانایان و خردمندان پنهان داشتی و به کودکان مکشوف فرمودی!
\par 26 بلی‌ای پدر، زیرا که همچنین منظور نظر تو بود.
\par 27 پدر همه‌چیز را به من سپرده است و کسی پسر را نمی شناسد بجز پدر و نه پدر را هیچ‌کس می‌شناسد غیر از پسر و کسی‌که پسر بخواهد بدومکشوف سازد.
\par 28 بیاید نزد من‌ای تمام زحمتکشان و گرانباران و من شما را آرامی خواهم بخشید.
\par 29 یوغ مرا بر خود گیرید و از من تعلیم یابید زیرا که حلیم و افتاده‌دل می‌باشم و درنفوس خود آرامی خواهید یافت؛زیرا یوغ من خفیف است و بار من سبک.»
\par 30 زیرا یوغ من خفیف است و بار من سبک.»

\chapter{12}

\par 1 در آن زمان، عیسی در روز سبت از میان کشتزارها می‌گذشت و شاگردانش چون گرسنه بودند، به چیدن و خوردن خوشه هاآغاز کردند.
\par 2 اما فریسیان چون این را دیدند، بدوگفتند: «اینک شاگردان تو عملی می‌کنند که کردن آن در سبت جایز نیست.»
\par 3 ایشان را گفت: «مگرنخوانده‌اید آنچه داود و رفیقانش کردند، وقتی که گرسنه بودند؟
\par 4 چه طور به خانه خدا در‌آمده، نانهای تقدمه را خورد که خوردن آن بر او ورفیقانش حلال نبود بلکه بر کاهنان فقط.
\par 5 یا درتورات نخوانده‌اید که در روزهای سبت، کهنه درهیکل سبت را حرمت نمی دارند و بی‌گناه هستند؟
\par 6 لیکن به شما می‌گویم که در اینجاشخصی بزرگتر از هیکل است!
\par 7 و اگر این معنی را درک می‌کردید که رحمت می‌خواهم نه قربانی، بی‌گناهان را مذمت نمی نمودید.
\par 8 زیرا که پسرانسان مالک روز سبت نیز ا§ست.»
\par 9 و از آنجا رفته، به کنیسه ایشان درآمد،
\par 10 که ناگاه شخص دست خشکی حاضر بود. پس ازوی پرسیده، گفتند: «آیا در روز سبت شفا دادن جایز است یا نه؟» تا ادعایی بر او وارد آورند.
\par 11 وی به ایشان گفت: «کیست از شما که یک گوسفند داشته باشد و هرگاه آن در روز سبت به حفره‌ای افتد، او را نخواهد گرفت و بیرون آورد؟
\par 12 پس چقدر انسان از گوسفند افضل است. بنابراین در سبت‌ها نیکویی‌کردن روا است.»
\par 13 آنگاه آن مرد را گفت: «دست خود را دراز کن!» پس دراز کرده، مانند دیگری صحیح گردید.
\par 14 اما فریسیان بیرون رفته، بر او شورا نمودند که چطور او را هلاک کنند.
\par 15 عیسی این را درک نموده، از آنجا روانه شدو گروهی بسیار از عقب او آمدند. پس جمیع ایشان را شفا بخشید،
\par 16 و ایشان را قدغن فرمودکه او را شهرت ندهند.
\par 17 تا تمام گردد کلامی که به زبان اشعیای نبی گفته شده بود:
\par 18 «اینک بنده من که او را برگزیدم و حبیب من که خاطرم از وی خرسند است. روح خود را بر وی خواهم نهاد تاانصاف را بر امت‌ها اشتهار نماید.
\par 19 نزاع و فغان نخواهد کرد و کسی آواز او را در کوچه هانخواهد شنید.
\par 20 نی خرد شده را نخواهد شکست و فتیله نیم‌سوخته را خاموش نخواهدکرد تا آنکه انصاف را به نصرت برآورد.
\par 21 و به نام او امت‌ها امید خواهند داشت.»
\par 22 آنگاه دیوانه‌ای کور و گنگ را نزد او آوردندو او را شفا داد چنانکه آن کور و گنگ، گویا و بیناشد.
\par 23 و تمام آن گروه در حیرت افتاده، گفتند: «آیا این شخص پسر داود نیست؟»
\par 24 لیکن فریسیان شنیده، گفتند: «این شخص دیوها رابیرون نمی کند مگر به یاری بعلزبول، رئیس دیوها!»
\par 25 عیسی خیالات ایشان را درک نموده، بدیشان گفت: «هر مملکتی که بر خود منقسم گردد، ویران شود و هر شهری یا خانه‌ای که برخود منقسم گردد، برقرار نماند.
\par 26 لهذا اگرشیطان، شیطان را بیرون کند، هرآینه بخلاف خودمنقسم گردد. پس چگونه سلطنتش پایدار ماند؟
\par 27 و اگر من به وساطت بعلزبول دیوها را بیرون می‌کنم، پسران شما آنها را به یاری که بیرون می‌کنند؟ از این جهت ایشان بر شما داوری خواهند کرد.
\par 28 لیکن هرگاه من به روح خدادیوها را اخراج می‌کنم، هرآینه ملکوت خدا برشما رسیده است.
\par 29 و چگونه کسی بتواند درخانه شخصی زورآور درآید و اسباب او را غارت کند، مگر آنکه اول آن زورآور را ببندد و پس خانه او را تاراج کند؟
\par 30 هر‌که با من نیست، برخلاف من است و هر‌که با من جمع نکند، پراکنده سازد.
\par 31 از این‌رو، شما را می‌گویم هر نوع گناه و کفر از انسان آمرزیده می‌شود، لیکن کفر به روح‌القدس از انسان عفو نخواهد شد.
\par 32 وهرکه برخلاف پسر انسان سخنی گوید، آمرزیده شود اما کسی‌که برخلاف روح‌القدس گوید، دراین عالم و در عالم آینده، هرگز آمرزیده نخواهدشد.
\par 33 یا درخت را نیکو گردانید و میوه‌اش رانیکو، یا درخت را فاسد سازید و میوه‌اش رافاسد، زیرا که درخت از میوه‌اش شناخته می‌شود.
\par 34 ‌ای افعی‌زادگان، چگونه می‌توانیدسخن نیکو گفت و حال آنکه بد هستید زیرا که زبان از زیادتی دل سخن می‌گوید.
\par 35 مرد نیکو ازخزانه نیکوی دل خود، چیزهای خوب برمی آورد و مرد بد از خزانه بد، چیزهای بدبیرون می‌آورد.
\par 36 لیکن به شما می‌گویم که هرسخن باطل که مردم گویند، حساب آن را در روزداوری خواهند داد.
\par 37 زیرا که از سخنان خودعادل شمرده خواهی شد و از سخنهای تو بر توحکم خواهد شد.»
\par 38 آنگاه بعضی از کاتبان و فریسیان در جواب گفتند: «ای استاد می‌خواهیم از تو آیتی ببینیم.»
\par 39 او در جواب ایشان گفت: «فرقه شریر و زناکارآیتی می‌طلبند و بدیشان جز آیت یونس نبی داده نخواهد شد.
\par 40 زیرا همچنانکه یونس سه شبانه‌روز در شکم ماهی ماند، پسر انسان نیز سه شبانه‌روز در شکم زمین خواهد بود.
\par 41 مردمان نینوا درروز داوری با این طایفه برخاسته، بر ایشان حکم خواهند کرد زیرا که به موعظه یونس توبه کردند واینک بزرگتری از یونس در اینجا است.
\par 42 ملکه جنوب در روز داوری با این فرقه برخاسته، برایشان حکم خواهد کرد زیرا که از اقصای زمین آمد تا حکمت سلیمان را بشنود، و اینک شخصی بزرگتر از سلیمان در اینجا است.
\par 43 و وقتی که روح پلید از آدمی بیرون آید، درطلب راحت به‌جایهای بی‌آب گردش می‌کند ونمی یابد.
\par 44 پس می‌گوید "به خانه خود که از آن بیرون آمدم برمی گردم، " و چون آید، آن را خالی و جاروب شده و آراسته می‌بیند.
\par 45 آنگاه می‌رود و هفت روح دیگر بدتر از خود رابرداشته، می‌آورد و داخل گشته، ساکن آنجامی شوند و انجام آن شخص بدتر از آغازش می‌شود. همچنین به این فرقه شریر خواهد شد.»
\par 46 او با آن جماعت هنوز سخن می‌گفت که ناگاه مادر و برادرانش در طلب گفتگوی وی بیرون ایستاده بودند.
\par 47 و شخصی وی را گفت: «اینک مادر تو و برادرانت بیرون ایستاده، می‌خواهند باتو سخن گویند.»
\par 48 در جواب قایل گفت: «کیست مادر من و برادرانم کیانند؟»
\par 49 و دست خود را به سوی شاگردان خود دراز کرده، گفت: «اینانند مادر من و برادرانم.زیرا هر‌که اراده پدر مرا که در آسمان است به‌جا آورد، همان برادر و خواهر و مادر من است.»
\par 50 زیرا هر‌که اراده پدر مرا که در آسمان است به‌جا آورد، همان برادر و خواهر و مادر من است.»

\chapter{13}

\par 1 و در همان روز، عیسی از خانه بیرون آمده، به کناره دریا نشست
\par 2 و گروهی بسیار بر وی جمع آمدند، بقسمی که او به کشتی سوار شده، قرار گرفت و تمامی آن گروه بر ساحل ایستادند؛
\par 3 و معانی بسیار به مثلها برای ایشان گفت: «وقتی برزگری بجهت پاشیدن تخم بیرون شد.
\par 4 و چون تخم می‌پاشید، قدری در راه افتاد ومرغان آمده، آن را خوردند.
\par 5 و بعضی برسنگلاخ جایی که خاک زیاد نداشت افتاده، بزودی سبز شد، چونکه زمین عمق نداشت،
\par 6 وچون آفتاب برآمد بسوخت و چون ریشه نداشت خشکید.
\par 7 و بعضی در میان خارها ریخته شد وخارها نمو کرده، آن را خفه نمود.
\par 8 و برخی درزمین نیکو کاشته شده، بار آورد، بعضی صد وبعضی شصت و بعضی سی.
\par 9 هر‌که گوش شنوادارد بشنود.»
\par 10 آنگاه شاگردانش آمده، به وی گفتند: «ازچه جهت با اینها به مثلها سخن می‌رانی؟»
\par 11 درجواب ایشان گفت: «دانستن اسرار ملکوت آسمان به شما عطا شده است، لیکن بدیشان عطانشده،
\par 12 زیرا هر‌که دارد بدو داده شود و افزونی یابد. اما کسی‌که ندارد آنچه دارد هم از او گرفته خواهد شد.
\par 13 از این جهت با اینها به مثلها سخن می گویم که نگرانند و نمی بینند و شنوا هستند ونمی شنوند و نمی فهمند.
\par 14 و در حق ایشان نبوت اشعیا تمام می‌شود که می‌گوید: "به سمع خواهید شنید و نخواهید فهمید و نظر کرده، خواهید نگریست و نخواهید دید.
\par 15 زیرا قلب این قوم سنگین شده و به گوشها به سنگینی شنیده‌اند و چشمان خود را بر هم نهاده‌اند، مبادابه چشمها ببینند و به گوشها بشنوند و به دلهابفهمند و بازگشت کنند و من ایشان را شفا دهم."
\par 16 لیکن خوشابه‌حال چشمان شما زیرا که می‌بینند و گوشهای شما زیرا که می‌شنوند
\par 17 زیرا هرآینه به شما می‌گویم بسا انبیا وعادلان خواستند که آنچه شما می‌بینید، ببینندو ندیدند و آنچه می‌شنوید، بشنوند ونشنیدند.
\par 18 پس شما مثل برزگر را بشنوید.
\par 19 کسی‌که کلمه ملکوت را شنیده، آن را نفهمید، شریرمی آید و آنچه در دل او کاشته شده است می‌رباید، همان است آنکه در راه کاشته شده است.
\par 20 و آنکه بر سنگلاخ ریخته شده، اوست که کلام را شنیده، فی الفور به خشنودی قبول می‌کند،
\par 21 و لکن ریشه‌ای در خود ندارد، بلکه فانی است و هرگاه سختی یا صدمه‌ای به‌سبب کلام بر او وارد آید، در ساعت لغزش می‌خورد.
\par 22 و آنکه در میان خارها ریخته شد، آن است که کلام را بشنود واندیشه این جهان و غرور دولت، کلام را خفه کند و بی‌ثمر گردد.
\par 23 و آنکه درزمین نیکو کاشته شد، آن است که کلام را شنیده، آن را می‌فهمد و بارآور شده، بعضی صد و بعضی شصت و بعضی سی ثمر می‌آورد.»
\par 24 و مثلی دیگر بجهت ایشان آورده، گفت: «ملکوت آسمان مردی را ماند که تخم نیکو درزمین خود کاشت:
\par 25 و چون مردم در خواب بودند دشمنش آمده، در میان گندم، کرکاس ریخته، برفت.
\par 26 و وقتی که گندم رویید و خوشه برآورد، کرکاس نیز ظاهر شد.
\par 27 پس نوکران صاحب‌خانه آمده، به وی عرض کردند: "ای آقامگر تخم نیکو در زمین خویش نکاشته‌ای؟ پس ازکجا کرکاس بهم رسانید؟"
\par 28 ایشان را فرمود: "این کار دشمن است." عرض کردند: "آیامی خواهی برویم آنها را جمع کنیم؟"
\par 29 فرمود: "نی، مبادا وقت جمع کردن کرکاس، گندم را باآنها برکنید.
\par 30 بگذارید که هر دو تا وقت حصاد باهم نمو کنند و در موسم حصاد، دروگران راخواهم گفت که اول کرکاسها را جمع کرده، آنهارا برای سوختن بافه‌ها ببندید اما گندم را در انبارمن ذخیره کنید."» مثل دانه خردل
\par 31 بار دیگر مثلی برای ایشان زده، گفت: «ملکوت آسمان مثل دانه خردلی است که شخصی گرفته، در مزرعه خویش کاشت.
\par 32 وهرچند از سایر دانه‌ها کوچکتر است، ولی چون نمو کند بزرگترین بقول است و درختی می‌شودچنانکه مرغان هوا آمده در شاخه هایش آشیانه می گیرند.»
\par 33 و مثلی دیگر برای ایشان گفت که ملکوت آسمان خمیرمایه‌ای را ماند که زنی آن را گرفته، در سه کیل خمیر پنهان کرد تا تمام، مخمر گشت.
\par 34 همه این معانی را عیسی با آن گروه به مثلهاگفت و بدون مثل بدیشان هیچ نگفت،
\par 35 تا تمام گردد کلامی که به زبان نبی گفته شد: «دهان خودرا به مثلها باز می‌کنم و به چیزهای مخفی شده ازبنای عالم تنطق خواهم کرد.»
\par 36 آنگاه عیسی آن گروه را مرخص کرده، داخل خانه گشت و شاگردانش نزد وی آمده، گفتند: «مثل کرکاس مزرعه را بجهت ما شرح فرما.»
\par 37 در جواب ایشان گفت: «آنکه بذر نیکومی کارد پسر انسان است،
\par 38 و مزرعه، این جهان است و تخم نیکو ابنای ملکوت و کرکاسها، پسران شریرند.
\par 39 و دشمنی که آنها را کاشت، ابلیس است و موسم حصاد، عاقبت این عالم ودروندگان، فرشتگانند.
\par 40 پس همچنان‌که کرکاسها را جمع کرده، در آتش می‌سوزانند، همانطور در عاقبت این عالم خواهد شد،
\par 41 که پسر انسان ملائکه خود را فرستاده، همه لغزش دهندگان و بدکاران را جمع خواهند کرد،
\par 42 و ایشان را به تنور آتش خواهند انداخت، جایی که گریه و فشار دندان بود.
\par 43 آنگاه عادلان در ملکوت پدر خود مثل آفتاب، درخشان خواهند شد. هر‌که گوش شنوا دارد بشنود.
\par 44 «و ملکوت آسمان گنجی را ماند، مخفی شده در زمین که شخصی آن را یافته، پنهان نمودو از خوشی آن رفته، آنچه داشت فروخت و آن زمین را خرید.
\par 45 «باز ملکوت آسمان تاجری را ماند که جویای مرواریدهای خوب باشد،
\par 46 و چون یک مروارید گرانبها یافت، رفت و مایملک خود رافروخته، آن را خرید.
\par 47 «ایض ملکوت آسمان مثل دامی است که به دریا افکنده شود و از هر جنسی به آن درآید،
\par 48 وچون پر شود، به کناره‌اش کشند و نشسته، خوبهارا در ظروف جمع کنند و بدها را دور اندازند.
\par 49 بدینطور در آخر این عالم خواهد شد. فرشتگان بیرون آمده، طالحین را از میان صالحین جدا کرده،
\par 50 ایشان را در تنور آتش خواهندانداخت، جایی که گریه و فشار دندان می‌باشد.»
\par 51 عیسی ایشان را گفت: «آیا همه این امور رافهمیده‌اید؟» گفتندش: «بلی خداوندا.»
\par 52 به ایشان گفت: «بنابراین، هر کاتبی که در ملکوت آسمان تعلیم یافته است، مثل صاحب‌خانه‌ای است که از خزانه خویش چیزهای نو و کهنه بیرون می‌آورد.»
\par 53 و چون عیسی این مثلها را به اتمام رسانید، از آن موضع روانه شد.
\par 54 و چون به وطن خویش آمد، ایشان را در کنیسه ایشان تعلیم داد، بقسمی که متعجب شده، گفتند: «از کجا این شخص چنین حکمت و معجزات را بهم رسانید؟
\par 55 آیا این پسر نجار نمی باشد؟ و آیا مادرش مریم نامی نیست؟ و برادرانش یعقوب و یوسف و شمعون ویهودا؟
\par 56 و همه خواهرانش نزد ما نمی باشند؟ پس این همه را از کجا بهم رسانید؟»
\par 57 و درباره او لغزش خوردند. لیکن عیسی بدیشان گفت: «نبی بی‌حرمت نباشد مگر در وطن و خانه خویش.»و به‌سبب بی‌ایمانی‌ایشان معجزه بسیار در آنجا ظاهر نساخت.
\par 58 و به‌سبب بی‌ایمانی‌ایشان معجزه بسیار در آنجا ظاهر نساخت.

\chapter{14}

\par 1 در آن هنگام هیرودیس تیترارخ چون شهرت عیسی را شنید،
\par 2 به خادمان خود گفت: «این است یحیی تعمید‌دهنده که ازمردگان برخاسته است، و از این جهت معجزات از او صادر می‌گردد.»
\par 3 زیرا که هیرودیس یحیی را بخاطر هیرودیا، زن برادر خود فیلپس گرفته، در بند نهاده و در زندان انداخته بود؛
\par 4 چون که یحیی بدو همی گفت: «نگاه داشتن وی بر تو حلال نیست.»
\par 5 و وقتی که قصد قتل او کرد، از مردم ترسید زیرا که او را نبی می‌دانستند.
\par 6 اما چون بزم میلاد هیرودیس را می‌آراستند، دختر هیرودیا درمجلس رقص کرده، هیرودیس را شاد نمود.
\par 7 از این‌رو قسم خورده، وعده داد که آنچه خواهدبدو بدهد.
\par 8 و او از ترغیب مادر خود گفت که «سریحیی تعمید‌دهنده را الان در طبقی به من عنایت فرما.»
\par 9 آنگاه پادشاه برنجید، لیکن بجهت پاس قسم و خاطر همنشینان خود، فرمود که بدهند.
\par 10 و فرستاده، سر یحیی را در زندان از تن جدا کرد،
\par 11 و سر او را در طشتی گذارده، به دختر تسلیم نمودند و او آن را نزد مادر خود برد.
\par 12 پس شاگردانش آمده، جسد او را برداشته، به خاک سپردند و رفته، عیسی را اطلاع دادند.
\par 13 و چون عیسی این را شنید، به کشتی سوارشده، از آنجا به ویرانه‌ای به خلوت رفت. و چون مردم شنیدند، از شهرها به راه خشکی از عقب وی روانه شدند.
\par 14 پس عیسی بیرون آمده، گروهی بسیار دیده، بر ایشان رحم فرمود وبیماران ایشان را شفا داد.
\par 15 و در وقت عصر، شاگردانش نزد وی آمده، گفتند: «این موضع ویرانه است و وقت الان گذشته. پس این گروه رامرخص فرما تا به دهات رفته بجهت خود غذابخرند.»
\par 16 عیسی ایشان را گفت: «احتیاج به رفتن ندارند. شما ایشان را غذا دهید.»
\par 17 بدو گفتند: «در اینجا جز پنج نان و دو ماهی نداریم!»
\par 18 گفت: «آنها را اینجا به نزد من بیاورید!»
\par 19 وبدان جماعت فرمود تا بر سبزه نشستند و پنج نان و دو ماهی را گرفته، به سوی آسمان نگریسته، برکت داد و نان را پاره کرده، به شاگردان سپرد وشاگردان بدان جماعت.
\par 20 و همه خورده، سیر شدند و از پاره های باقی‌مانده دوازده سبد پرکرده، برداشتند.
\par 21 و خورندگان سوای زنان واطفال قریب به پنج هزار مرد بودند.
\par 22 بی‌درنگ عیسی شاگردان خود را اصرارنمود تا به کشتی سوار شده، پیش از وی به کناره دیگر روانه شوند تا آن گروه را رخصت دهد.
\par 23 وچون مردم را روانه نمود، به خلوت برای عبادت بر فراز کوهی برآمد. و وقت شام در آنجا تنها بود.
\par 24 اما کشتی در آن وقت در میان دریا به‌سبب بادمخالف که می‌وزید، به امواج گرفتار بود.
\par 25 و درپاس چهارم از شب، عیسی بر دریا خرامیده، به سوی ایشان روانه گردید.
\par 26 اما چون شاگردان، او را بر دریا خرامان دیدند، مضطرب شده، گفتندکه خیالی است؛ و از خوف فریاد برآوردند.
\par 27 اماعیسی ایشان را بی‌تامل خطاب کرده، گفت: «خاطر جمع دارید! منم ترسان مباشید!»
\par 28 پطرس در جواب او گفت: «خداوندا، اگر تویی مرا بفرما تا بر روی آب، نزد تو آیم.»
\par 29 گفت: «بیا!» در ساعت پطرس از کشتی فرود شده، برروی آب روانه شد تا نزد عیسی آید.
\par 30 لیکن چون باد را شدید دید، ترسان گشت و مشرف به غرق شده، فریاد برآورده، گفت: «خداوندا مرادریاب.»
\par 31 عیسی بی‌درنگ دست آورده، او رابگرفت و گفت: «ای کم‌ایمان، چرا شک آوردی؟»
\par 32 و چون به کشتی سوار شدند، بادساکن گردید.
\par 33 پس اهل کشتی آمده، او راپرستش کرده، گفتند: «فی الحقیقه تو پسر خدا هستی!»
\par 34 آنگاه عبور کرده، به زمین جنیسره آمدند،
\par 35 و اهل آن موضع او را شناخته، به همگی آن نواحی فرستاده، همه بیماران را نزد او آوردند،و از او اجازت خواستند که محض دامن ردایش را لمس کنند و هر‌که لمس کرد، صحت کامل یافت.
\par 36 و از او اجازت خواستند که محض دامن ردایش را لمس کنند و هر‌که لمس کرد، صحت کامل یافت.

\chapter{15}

\par 1 آنگاه کاتبان و فریسیان اورشلیم نزدعیسی آمده، گفتند:
\par 2 «چون است که شاگردان تو از تقلید مشایخ تجاوز می‌نمایند، زیرا هرگاه نان می‌خورند دست خود رانمی شویند؟»
\par 3 او در جواب ایشان گفت: «شمانیز به تقلید خویش، از حکم خدا چرا تجاوزمی کنید؟
\par 4 زیرا خدا حکم داده است که مادر وپدر خود را حرمت دار و هرکه پدر یا مادر رادشنام دهد البته هلاک گردد.
\par 5 لیکن شمامی گویید هر‌که پدر یا مادر خود را گوید آنچه ازمن به تو نفع رسد هدیه‌ای است،
\par 6 و پدر یا مادرخود را بعد از آن احترام نمی نماید. پس به تقلیدخود، حکم خدا را باطل نموده‌اید.
\par 7 ‌ای ریاکاران، اشعیاء در باره شما نیکو نبوت نموده است که گفت:
\par 8 این قوم به زبانهای خود به من تقرب می‌جویند و به لبهای خویش مرا تمجیدمی نمایند، لیکن دلشان از من دور است.
\par 9 پس عبادت مرا عبث می‌کنند زیرا که احکام مردم رابمنزله فرایض تعلیم می‌دهند.»
\par 10 و آن جماعت را خوانده، بدیشان گفت: «گوش داده، بفهمید؛
\par 11 نه آنچه به دهان فرومی رود انسان را نجس می‌سازد بلکه آنچه از دهان بیرون می‌آید انسان را نجس می‌گرداند.»
\par 12 آنگاه شاگردان وی آمده، گفتند: «آیا می‌دانی که فریسیان چون این سخن را شنیدند، مکروهش داشتند؟»
\par 13 او در جواب گفت: «هر نهالی که پدرآسمانی من نکاشته باشد، کنده شود.
\par 14 ایشان راواگذارید، کوران راهنمایان کورانند و هرگاه کور، کور را راهنما شود، هر دو در چاه افتند.
\par 15 پطرس در جواب او گفت: «این مثل را برای ما شرح فرما.»
\par 16 عیسی گفت: «آیا شما نیز تا به حال بی‌ادراک هستید؟
\par 17 یا هنوز نیافته‌اید که آنچه از دهان فرو می‌رود، داخل شکم می‌گردد ودر مبرز افکنده می‌شود؟
\par 18 لیکن آنچه از دهان برآید، از دل صادر می‌گردد و این چیزها است که انسان را نجس می‌سازد.
\par 19 زیرا که از دل برمی آید، خیالات بد و قتلها و زناها و فسقها ودزدیها و شهادات دروغ و کفرها.
\par 20 اینها است که انسان را نجس می‌سازد، لیکن خوردن به‌دستهای ناشسته، انسان را نجس نمی گرداند.»
\par 21 پس عیسی از آنجا بیرون شده، به دیارصور و صیدون رفت.
\par 22 ناگاه زن کنعانیه‌ای از آن حدود بیرون آمده، فریاد‌کنان وی را گفت: «خداوندا، پسر داودا، بر من رحم کن زیرا دختر من سخت دیوانه است.»
\par 23 لیکن هیچ جوابش نداد تا شاگردان او پیش آمده، خواهش نمودند که «او را مرخص فرمای زیرا در عقب ما شورش می‌کند.»
\par 24 او در جواب گفت: «فرستاده نشده‌ام مگر بجهت گوسفندان گم شده خاندان اسرائیل.»
\par 25 پس آن زن آمده، او را پرستش کرده، گفت: «خداوندا مرا یاری کن.»
\par 26 در جواب گفت که «نان فرزندان را گرفتن و نزد سگان انداختن جایزنیست.»
\par 27 عرض کرد: «بلی خداوندا، زیرا سگان نیز از پاره های افتاده سفره آقایان خویش می‌خورند.»
\par 28 آنگاه عیسی در جواب او گفت: «ای زن! ایمان تو عظیم است! تو را برحسب خواهش تو بشود.» که در همان ساعت، دخترش شفا یافت.
\par 29 عیسی از آنجا حرکت کرده، به کناره دریای جلیل آمد و برفراز کوه برآمده، آنجا بنشست.
\par 30 و گروهی بسیار، لنگان و کوران و گنگان وشلان و جمعی از دیگران را با خود برداشته، نزداو آمدند و ایشان را بر پایهای عیسی افکندند وایشان را شفا داد،
\par 31 بقسمی که آن جماعت، چون گنگان را گویا و شلان را تندرست و لنگان راخرامان و کوران را بینا دیدند، متعجب شده، خدای اسرائیل را تمجید کردند.
\par 32 عیسی شاگردان خود را پیش طلبیده، گفت: «مرا بر این جماعت دل بسوخت زیرا که الحال سه روز است که با من می‌باشند و هیچ‌چیز برای خوراک ندارند و نمی خواهم ایشان را گرسنه برگردانم مبادا در راه ضعف کنند.»
\par 33 شاگردانش به او گفتند: «از کجا در بیابان ما را آنقدر نان باشدکه چنین انبوه را سیر کند؟»
\par 34 عیسی ایشان راگفت: «چند نان دارید؟» گفتند: «هفت نان و قدری از ماهیان کوچک.»
\par 35 پس مردم را فرمود تا برزمین بنشینند.
\par 36 و آن هفت نان و ماهیان را گرفته، شکر نمود و پاره کرده، به شاگردان خود داد وشاگردان به آن جماعت.
\par 37 و همه خورده، سیرشدند و از خرده های باقی‌مانده هفت زنبیل پربرداشتند.
\par 38 و خورندگان، سوای زنان و اطفال چهار هزار مرد بودند.پس آن گروه را رخصت داد و به کشتی سوار شده، به حدود مجدل آمد.
\par 39 پس آن گروه را رخصت داد و به کشتی سوار شده، به حدود مجدل آمد.

\chapter{16}

\par 1 آنگاه فریسیان و صدوقیان نزد او آمده، از روی امتحان از وی خواستند که آیتی آسمانی برای ایشان ظاهر سازد.
\par 2 ایشان راجواب داد که «در وقت عصر می‌گویید هوا خوش خواهد بود زیرا آسمان سرخ است؛
\par 3 وصبحگاهان می‌گویید امروز هوا بد خواهد شدزیرا که آسمان سرخ و گرفته است. ای ریاکاران می‌دانید صورت آسمان را تمییز دهید، اماعلامات زمانها را نمی توانید!
\par 4 فرقه شریر زناکار، آیتی می‌طلبند و آیتی بدیشان عطا نخواهد شدجز آیت یونس نبی.» پس ایشان را رها کرده، روانه شد.
\par 5 و شاگردانش چون بدان طرف می‌رفتند، فراموش کردند که نان بردارند.
\par 6 عیسی ایشان راگفت: «آگاه باشید که از خمیرمایه فریسیان وصدوقیان احتیاط کنید!»
\par 7 پس ایشان در خودتفکر نموده، گفتند: «از آن است که نان برنداشته‌ایم.»
\par 8 عیسی این را درک نموده، بدیشان گفت: «ای سست‌ایمانان، چرا در خود تفکرمی کنید از آنجهت که نان نیاورده‌اید؟
\par 9 آیا هنوزنفهمیده و یاد نیاورده‌اید آن پنج نان و پنج هزارنفر و چند سبدی را که برداشتید؟
\par 10 و نه آن هفت نان و چهار هزار نفر و چند زنبیلی را که برداشتید؟»
\par 11 پس چرا نفهمیدید که درباره نان شما را نگفتم که از خمیرمایه فریسیان وصدوقیان احتیاط کنید؟»
\par 12 آنگاه دریافتند که نه از خمیرمایه نان بلکه از تعلیم فریسیان وصدوقیان حکم به احتیاط فرموده است.
\par 13 و هنگامی که عیسی به نواحی قیصریه فیلپس آمد، از شاگردان خود پرسیده، گفت: «مردم مرا که پسر انسانم چه شخص می‌گویند؟»
\par 14 گفتند: «بعضی یحیی تعمید‌دهنده و بعضی الیاس و بعضی ارمیا یا یکی از انبیا.»
\par 15 ایشان راگفت: «شما مرا که می‌دانید؟»
\par 16 شمعون پطرس در جواب گفت که «تویی مسیح، پسر خدای زنده!»
\par 17 عیسی در جواب وی گفت: «خوشابحال تو‌ای شمعون بن یونا! زیرا جسم وخون این را بر تو کشف نکرده، بلکه پدر من که درآسمان است.
\par 18 و من نیز تو را می‌گویم که تویی پطرس و بر این صخره کلیسای خود را بنا می‌کنم و ابواب جهنم بر آن استیلا نخواهد یافت.
\par 19 وکلیدهای ملکوت آسمان را به تو می‌سپارم؛ وآنچه بر زمین ببندی در آسمان بسته گردد و آنچه در زمین گشایی در آسمان گشاده شود.
\par 20 آنگاه شاگردان خود را قدغن فرمود که به هیچ‌کس نگویند که او مسیح است.
\par 21 و از آن زمان عیسی به شاگردان خودخبردادن آغاز کرد که رفتن او به اورشلیم وزحمت بسیار کشیدن از مشایخ و روسای کهنه وکاتبان و کشته شدن و در روز سوم برخاستن ضروری است.
\par 22 و پطرس او را گرفته، شروع کرد به منع نمودن و گفت: «حاشا از تو‌ای خداوندکه این بر تو هرگز واقع نخواهد شد!»
\par 23 اما اوبرگشته، پطرس را گفت: «دور شو از من‌ای شیطان زیرا که باعث لغزش من می‌باشی، زیرا نه امورالهی را بلکه امور انسانی را تفکر می‌کنی!»
\par 24 آنگاه عیسی به شاگردان خود گفت: «اگر کسی خواهدمتابعت من کند، باید خود را انکار کرده وصلیب خود را برداشته، از عقب من آید.
\par 25 زیرا هر کس بخواهد جان خود را برهاند، آن را هلاک سازد؛ اما هر‌که جان خود را بخاطر من هلاک کند، آن را دریابد.
\par 26 زیرا شخص را چه سود دارد که تمام دنیا را ببرد و جان خود را ببازد؟ یا اینکه آدمی چه چیز را فدای جان خود خواهدساخت؟
\par 27 زیرا که پسر انسان خواهد آمد درجلال پدر خویش به اتفاق ملائکه خود و در آن وقت هر کسی را موافق اعمالش جزا خواهد داد.هرآینه به شما می‌گویم که بعضی در اینجاحاضرند که تا پسر انسان را نبینند که در ملکوت خود می‌آید، ذائقه موت را نخواهند چشید.»
\par 28 هرآینه به شما می‌گویم که بعضی در اینجاحاضرند که تا پسر انسان را نبینند که در ملکوت خود می‌آید، ذائقه موت را نخواهند چشید.»

\chapter{17}

\par 1 و بعد از شش روز، عیسی، پطرس ویعقوب و برادرش یوحنا را برداشته، ایشان را در خلوت به کوهی بلند برد.
\par 2 و در نظرایشان هیات او متبدل گشت و چهره‌اش چون خورشید، درخشنده و جامه‌اش چون نور، سفیدگردید.
\par 3 که ناگاه موسی و الیاس بر ایشان ظاهرشده، با او گفتگو می‌کردند.
\par 4 اما پطرس به عیسی متوجه شده، گفت که «خداوندا، بودن ما در اینجانیکو است! اگر بخواهی، سه سایبان در اینجابسازیم، یکی برای تو و یکی بجهت موسی ودیگری برای الیاس.»
\par 5 و هنوز سخن بر زبانش بود که ناگاه ابری درخشنده بر ایشان سایه افکند واینک آوازی از ابر در‌رسید که «این است پسرحبیب من که از وی خشنودم. او را بشنوید!»
\par 6 وچون شاگردان این را شنیدند، به روی در‌افتاده، بینهایت ترسان شدند.
\par 7 عیسی نزدیک آمده، ایشان را لمس نمود و گفت: «برخیزید و ترسان مباشید!»
\par 8 و چشمان خود را گشوده، هیچ‌کس راجز عیسی تنها ندیدند.
\par 9 و چون ایشان از کوه به زیر می‌آمدند، عیسی ایشان را قدغن فرمود که «تاپسر انسان از مردگان برنخیزد، زنهار این‌رویا را به کسی باز نگویید.»
\par 10 شاگردانش از او پرسیده، گفتند: «پس کاتبان چرا می‌گویند که می‌بایدالیاس اول آید؟»
\par 11 او در جواب گفت: «البته الیاس می‌آید و تمام چیزها را اصلاح خواهدنمود.
\par 12 لیکن به شما می‌گویم که الحال الیاس آمده است و او را نشناختند بلکه آنچه خواستندبا وی کردند؛ به همانطور پسر انسان نیز از ایشان زحمت خواهد دید.
\par 13 آنگاه شاگردان دریافتندکه درباره یحیی تعمیددهنده بدیشان سخن می‌گفت.
\par 14 و چون به نزد جماعت رسیدند، شخصی پیش آمده، نزد وی زانو زده، عرض کرد:
\par 15 «خداوندا، بر پسر من رحم کن زیرا مصروع و به شدت متالم است، چنانکه بارها در آتش و مکرر در آب می‌افتد.
\par 16 و او را نزد شاگردان تو آوردم، نتوانستند او را شفا دهند.»
\par 17 عیسی در جواب گفت: «ای فرقه بی‌ایمان کج رفتار، تا به کی با شماباشیم و تا چند متحمل شما گردم؟ او را نزد من آورید.»
\par 18 پس عیسی او را نهیب داده، دیو ازوی بیرون شد و در ساعت، آن پسر شفا یافت.
\par 19 اما شاگردان نزد عیسی آمده، در خلوت از اوپرسیدند: «چرا ما نتوانستیم او را بیرون کنیم؟
\par 20 عیسی ایشان را گفت: «به‌سبب بی‌ایمانی شما. زیرا هرآینه به شما می‌گویم، اگر ایمان به قدر دانه خردلی می‌داشتید، بدین کوه می‌گفتید از اینجابدانجا منتقل شو، البته منتقل می‌شد و هیچ امری بر شما محال نمی بود.
\par 21 لیکن این جنس جز به دعا و روزه بیرون نمی رود.»
\par 22 و چون ایشان در جلیل می‌گشتند، عیسی بدیشان گفت: «پسر انسان بدست مردم تسلیم کرده خواهد شد،
\par 23 و او را خواهند کشت و درروز سوم خواهد برخاست.» پس بسیار محزون شدند.
\par 24 و چون ایشان وارد کفرناحوم شدند، محصلان دو درهم نزد پطرس آمده، گفتند: «آیااستاد شما دو درهم را نمی دهد؟»
\par 25 گفت: «بلی.» و چون به خانه درآمده، عیسی بر او سبقت نموده، گفت: «ای شمعون، چه گمان داری؟ پادشاهان جهان از چه کسان عشر و جزیه می‌گیرند؟ از فرزندان خویش یا از بیگانگان؟»
\par 26 پطرس به وی گفت: «از بیگانگان.» عیسی بدوگفت: «پس یقین پسران آزادند!لیکن مبادا که ایشان را برنجانیم، به کناره دریا رفته، قلابی بینداز و اول ماهی که بیرون می‌آید، گرفته و دهانش راباز کرده، مبلغ چهار درهم خواهی یافت. آن رابرداشته، برای من و خود بدیشان بده!»
\par 27 لیکن مبادا که ایشان را برنجانیم، به کناره دریا رفته، قلابی بینداز و اول ماهی که بیرون می‌آید، گرفته و دهانش راباز کرده، مبلغ چهار درهم خواهی یافت. آن رابرداشته، برای من و خود بدیشان بده!»

\chapter{18}

\par 1 در همان ساعت، شاگردان نزد عیسی آمده، گفتند: «چه کس در ملکوت آسمان بزرگتر است؟»
\par 2 آنگاه عیسی طفلی طلب نموده، در میان ایشان برپا داشت
\par 3 و گفت: «هرآینه به شما می‌گویم تا بازگشت نکنید و مثل طفل کوچک نشوید، هرگز داخل ملکوت آسمان نخواهید شد.
\par 4 پس هر‌که مثل این بچه کوچک خود را فروتن سازد، همان در ملکوت آسمان بزرگتر است.
\par 5 و کسی‌که چنین طفلی را به اسم من قبول کند، مرا پذیرفته است.
\par 6 و هر‌که یکی ازاین صغار را که به من ایمان دارند، لغزش دهد او رابهتر می‌بود که سنگ آسیایی بر گردنش آویخته، در قعر دریا غرق می‌شد!
\par 7 وای بر این جهان به‌سبب لغزشها؛ زیرا که لابد است از وقوع لغزشها، لیکن وای بر کسی‌که سبب لغزش باشد.
\par 8 پس اگر دستت یا پایت تو رابلغزاند، آن را قطع کرده، از خود دور انداز زیرا تورا بهتر است که لنگ یا شل داخل حیات شوی ازآنکه با دو دست یا دو پا در نار جاودانی افکنده شوی.
\par 9 و اگر چشمت تو را لغزش دهد، آن را قلع کرده، از خود دور انداز زیرا تو را بهتر است بایک چشم وارد حیات شوی، از اینکه با دو چشم در آتش جهنم افکنده شوی.
\par 10 زنهار یکی از این صغار را حقیر مشمارید، زیرا شما را می‌گویم که ملائکه ایشان دائم درآسمان روی پدر مرا که در آسمان است می‌بینند.
\par 11 زیرا که پسر انسان آمده است تا گم شده رانجات‌بخشد.
\par 12 شما چه گمان می‌برید، اگرکسی را صد گوسفند باشد و یکی از آنها گم شود، آیا آن نود و نه را به کوهسار نمی گذارد و به جستجوی آن گم شده نمی رود؟
\par 13 و اگر اتفاق آن را دریابد، هرآینه به شما می‌گویم بر آن یکی بیشتر شادی می‌کند از آن نود و نه که گم نشده‌اند.
\par 14 همچنین اراده پدر شما که در آسمان است این نیست که یکی از این کوچکان هلاک گردد.
\par 15 «و اگر برادرت به تو گناه کرده باشد، برو واو را میان خود و او در خلوت الزام کن. هرگاه سخن تو را گوش گرفت، برادر خود را دریافتی؛
\par 16 و اگر نشنود، یک یا دو نفر دیگر با خود بردار تااز زبان دو یا سه شاهد، هر سخنی ثابت شود.
\par 17 واگر سخن ایشان را رد کند، به کلیسا بگو. و اگرکلیسا را قبول نکند، در نزد تو مثل خارجی یاباجگیر باشد.
\par 18 هرآینه به شما می‌گویم آنچه برزمین بندید، در آسمان بسته شده باشد و آنچه برزمین گشایید، در آسمان گشوده شده باشد.
\par 19 باز به شما می‌گویم هر گاه دو نفر از شما در زمین درباره هر‌چه که بخواهند متفق شوند، هرآینه ازجانب پدر من که در آسمان است برای ایشان کرده خواهد شد.
\par 20 زیرا جایی که دو یا سه نفر به اسم من جمع شوند، آنجا درمیان ایشان حاضرم.»
\par 21 آنگاه پطرس نزد او آمده، گفت: «خداوندا، چند مرتبه برادرم به من خطا ورزد، می‌باید او راآمرزید؟ آیا تا هفت مرتبه؟»
\par 22 عیسی بدو گفت: «تو را نمی گویم تا هفت مرتبه، بلکه تا هفتاد هفت مرتبه!
\par 23 از آنجهت ملکوت آسمان پادشاهی راماند که با غلامان خود اراده محاسبه داشت.
\par 24 وچون شروع به حساب نمود، شخصی را نزد اوآوردند که ده هزار قنطار به او بدهکار بود.
\par 25 وچون چیزی نداشت که ادا نماید، آقایش امر کردکه او را با زن و فرزندان و تمام مایملک اوفروخته، طلب را وصول کنند.
\par 26 پس آن غلام روبه زمین نهاده او را پرستش نمود و گفت: "ای آقامرا مهلت ده تا همه را به تو ادا کنم."
\par 27 آنگاه آقای آن غلام بر وی ترحم نموده، او را رها کرد وقرض او را بخشید.
\par 28 لیکن چون آن غلام بیرون رفت، یکی از همقطاران خود را یافت که از او صددینار طلب داشت. او را بگرفت و گلویش رافشرده، گفت: "طلب مرا ادا کن!"
\par 29 پس آن همقطار بر پایهای او افتاده، التماس نموده، گفت: "مرا مهلت ده تا همه را به تو رد کنم."
\par 30 اما اوقبول نکرد بلکه رفته، او را در زندان انداخت تاقرض را ادا کند.
\par 31 چون همقطاران وی این وقایع را دیدند، بسیار غمگین شده، رفتند و آنچه شده بود به آقای خود باز‌گفتند.
\par 32 آنگاه مولایش او راطلبیده، گفت: "ای غلام شریر، آیا تمام آن قرض را محض خواهش تو به تو نبخشیدم؟
\par 33 پس آیاتو را نیز لازم نبود که بر همقطار خود رحم کنی چنانکه من بر تو رحم کردم؟"
\par 34 پس مولای او درغضب شده، او را به جلادان سپرد تا تمام قرض رابدهد.به همینطور پدر آسمانی من نیز با شماعمل خواهد نمود، اگر هر یکی از شما برادر خودرا از دل نبخشد.»
\par 35 به همینطور پدر آسمانی من نیز با شماعمل خواهد نمود، اگر هر یکی از شما برادر خودرا از دل نبخشد.»

\chapter{19}

\par 1 و چون عیسی این سخنان را به اتمام رسانید، از جلیل روانه شده، به حدودیهودیه از آن طرف اردن آمد.
\par 2 و گروهی بسیار ازعقب او آمدند و ایشان را در آنجا شفا بخشید.
\par 3 پس فریسیان آمدند تا او را امتحان کنند وگفتند: «آیا جایز است مرد، زن خود را به هر علتی طلاق دهد؟»
\par 4 او در جواب ایشان گفت: «مگرنخوانده‌اید که خالق در ابتدا ایشان را مرد و زن آفرید،
\par 5 و گفت از این جهت مرد، پدر و مادرخود را رها کرده، به زن خویش بپیوندد و هر دویک تن خواهند شد؟
\par 6 بنابراین بعد از آن دونیستند بلکه یک تن هستند. پس آنچه را خداپیوست انسان جدا نسازد.»
\par 7 به وی گفتند: «پس از بهر‌چه موسی‌امر فرمود که زن را طلاقنامه دهند و جدا کنند؟»
\par 8 ایشان را گفت: «موسی به‌سبب سنگدلی شما، شما را اجازت داد که زنان خود را طلاق دهید. لیکن از ابتدا چنین نبود.
\par 9 و به شما می‌گویم هر‌که زن خود را بغیر علت زناطلاق دهد و دیگری را نکاح کند، زانی است و هرکه زن مطلقه‌ای را نکاح کند، زنا کند.»
\par 10 شاگردانش بدو گفتند: «اگر حکم شوهر بازن چنین باشد، نکاح نکردن بهتر است!»
\par 11 ایشان را گفت: «تمامی خلق این کلام را نمی پذیرند، مگر به کسانی که عطا شده است.
\par 12 زیرا که خصی‌ها می‌باشند که از شکم مادر چنین متولدشدند و خصی‌ها هستند که از مردم خصی شده‌اند و خصی‌ها می‌باشند که بجهت ملکوت خدا خود را خصی نموده‌اند. آنکه توانایی قبول دارد بپذیرد.»
\par 13 آنگاه چند بچه کوچک را نزد او آوردند تادستهای خود را بر ایشان نهاده، دعا کند. اماشاگردان، ایشان را نهیب دادند.
\par 14 عیسی گفت: «بچه های کوچک را بگذارید و از آمدن نزد من، ایشان را منع مکنید، زیرا ملکوت آسمان از مثل اینها است.»
\par 15 و دستهای خود را بر ایشان گذارده از آن جا روانه شد.
\par 16 ناگاه شخصی آمده، وی را گفت: «ای استادنیکو، چه عمل نیکو کنم تا حیات جاودانی یابم؟»
\par 17 وی را گفت: «از چه سبب مرا نیکو گفتی و حال آنکه کسی نیکو نیست، جز خدا فقط. لیکن اگربخواهی داخل حیات شوی، احکام را نگاه دار.»
\par 18 بدو گفت: «کدام احکام؟» عیسی گفت: «قتل مکن، زنا مکن، دزدی مکن، شهادت دروغ مده،
\par 19 و پدر و مادر خود را حرمت دار و همسایه خود را مثل نفس خود دوست دار.»
\par 20 جوان وی را گفت: «همه اینها را از طفولیت نگاه داشته‌ام. دیگر مرا چه ناقص است؟»
\par 21 عیسی بدو گفت: «اگر بخواهی کامل شوی، رفته مایملک خود رابفروش و به فقراء بده که در آسمان گنجی خواهی داشت؛ و آمده مرا متابعت نما.»
\par 22 چون جوان این سخن را شنید، دل تنگ شده، برفت زیرا که مال بسیار داشت.
\par 23 عیسی به شاگردان خود گفت: «هرآینه به شما می‌گویم که شخص دولتمند به ملکوت آسمان به دشواری داخل می‌شود.
\par 24 و باز شمارا می‌گویم که گذشتن شتر از سوراخ سوزن، آسانتر است از دخول شخص دولتمند درملکوت خدا.»
\par 25 شاگردان چون شنیدند، بغایت متحیر گشته، گفتند: «پس که می‌تواند نجات یابد؟»
\par 26 عیسی متوجه ایشان شده، گفت: «نزدانسان این محال است لیکن نزد خدا همه‌چیزممکن است.»
\par 27 آنگاه پطرس در جواب گفت: «اینک ما همه‌چیزها را ترک کرده، تو را متابعت می‌کنیم. پس ما را چه خواهد بود؟»
\par 28 «عیسی ایشان را گفت: «هرآینه به شما می‌گویم شما که مرا متابعت نموده‌اید، در معاد وقتی که پسر انسان بر کرسی جلال خود نشیند، شما نیز به دوازده کرسی نشسته، بر دوازده سبط اسرائیل داوری خواهید نمود.
\par 29 و هر‌که بخاطر اسم من، خانه هایا برادران یا خواهران یا پدر یا مادر یا زن یافرزندان یا زمینها را ترک کرد، صد چندان خواهدیافت و وارث حیات جاودانی خواهد گشت.لیکن بسا اولین که آخرین می‌گردند و آخرین، اولین!
\par 30 لیکن بسا اولین که آخرین می‌گردند و آخرین، اولین!

\chapter{20}

\par 1 «زیرا ملکوت آسمان صاحب‌خانه‌ای را ماند که بامدادان بیرون رفت تا عمله بجهت تاکستان خود به مزد بگیرد.
\par 2 پس با عمله، روزی یک دینار قرار داده، ایشان را به تاکستان خود فرستاد.
\par 3 و قریب به ساعت سوم بیرون رفته، بعضی دیگر را در بازار بیکار ایستاده دید.
\par 4 ایشان را نیز گفت: "شما هم به تاکستان بروید وآنچه حق شما است به شما می‌دهم." پس رفتند.
\par 5 باز قریب به ساعت ششم و نهم رفته، همچنین کرد.
\par 6 و قریب به ساعت یازدهم رفته، چند نفردیگر بیکار ایستاده یافت. ایشان را گفت: "از بهرچه تمامی روز در اینجا بیکار ایستاده‌اید؟"
\par 7 گفتندش: "هیچ‌کس ما را به مزد نگرفت." بدیشان گفت: "شما نیز به تاکستان بروید و حق خویش راخواهید یافت."
\par 8 و چون وقت شام رسید، صاحب تاکستان به ناظر خود گفت: "مزدوران راطلبیده، از آخرین گرفته تا اولین مزد ایشان را اداکن."
\par 9 پس یازده ساعتیان آمده، هر نفری دیناری یافتند.
\par 10 و اولین آمده، گمان بردند که بیشترخواهند یافت. ولی ایشان نیز هر نفری دیناری یافتند.
\par 11 اما چون گرفتند، به صاحب‌خانه شکایت نموده،
\par 12 گفتند که "این آخرین، یک ساعت کار کردند و ایشان را با ما که متحمل سختی و حرارت روز گردیده‌ایم مساوی ساخته‌ای؟"
\par 13 او در جواب یکی از ایشان گفت: "ای رفیق بر تو ظلمی نکردم. مگر به دیناری با من قرار ندادی؟
\par 14 حق خود را گرفته برو. می‌خواهم بدین آخری مثل تو دهم.
\par 15 آیا مرا جایز نیست که از مال خود آنچه خواهم بکنم؟ مگر چشم توبد است از آن رو که من نیکو هستم؟"
\par 16 بنابراین اولین آخرین و آخرین اولین خواهند شد، زیراخوانده‌شدگان بسیارند و برگزیدگان کم.»
\par 17 و چون عیسی به اورشلیم می‌رفت، دوازده شاگرد خود را در اثنای راه به خلوت طلبیده بدیشان گفت:
\par 18 «اینک به سوی اورشلیم می‌رویم و پسر انسان به روسای کهنه و کاتبان تسلیم کرده خواهد شد و حکم قتل او را خواهندداد،
\par 19 و او را به امت‌ها خواهند سپرد تا او رااستهزا کنند و تازیانه زنند و مصلوب نمایند و درروز سوم خواهد برخاست.»
\par 20 آنگاه مادر دو پسر زبدی با پسران خود نزدوی آمده و پرستش نموده، از او چیزی درخواست کرد.
\par 21 بدو گفت: «چه خواهش داری؟» گفت: «بفرما تا این دو پسر من در ملکوت تو، یکی بر دست راست و دیگری بر دست چپ تو بنشینند.»
\par 22 عیسی در جواب گفت: «نمی دانید چه می‌خواهید. آیا می‌توانید از آن کاسه‌ای که من می‌نوشم، بنوشید و تعمیدی را که من می‌یابم، بیابید؟» بدو گفتند: «می‌توانیم.»
\par 23 ایشان را گفت: «البته از کاسه من خواهیدنوشید و تعمیدی را که من می‌یابم، خواهید یافت. لیکن نشستن به‌دست راست و چپ من، از آن من نیست که بدهم، مگر به کسانی که از جانب پدرم برای ایشان مهیا شده است.»
\par 24 اما چون آن ده شاگرد شنیدند، بر آن دوبرادر به دل رنجیدند.
\par 25 عیسی ایشان را پیش طلبیده، گفت: «آگاه هستید که حکام امت‌ها برایشان سروری می‌کنند و روسا بر ایشان مسلطند.
\par 26 لیکن در میان شما چنین نخواهد بود، بلکه هرکه در میان شما می‌خواهد بزرگ گردد، خادم شماباشد.
\par 27 و هر‌که می‌خواهد در میان شما مقدم بود، غلام شما باشد.
\par 28 چنانکه پسر انسان نیامدتا مخدوم شود بلکه تا خدمت کند و جان خود رادر راه بسیاری فدا سازد.»
\par 29 و هنگامی که از اریحا بیرون می‌رفتند، گروهی بسیار از عقب او می‌آمدند.
\par 30 که ناگاه دومرد کور کنار راه نشسته، چون شنیدند که عیسی در گذر است، فریاد کرده، گفتند: «خداوندا، پسرداودا، بر ما ترحم کن!»
\par 31 و هر‌چند خلق ایشان را نهیب می‌دادند که خاموش شوند، بیشتر فریادکنان می‌گفتند: «خداوندا، پسر داودا، بر ما ترحم فرما!»
\par 32 پس عیسی ایستاده، به آواز بلند گفت: «چه می‌خواهید برای شما کنم؟»
\par 33 به وی گفتند: «خداوندا، اینکه چشمان ما باز‌گردد!»پس عیسی ترحم نموده، چشمان ایشان را لمس نمودکه در ساعت بینا گشته، از عقب او روانه شدند.
\par 34 پس عیسی ترحم نموده، چشمان ایشان را لمس نمودکه در ساعت بینا گشته، از عقب او روانه شدند.

\chapter{21}

\par 1 و چون نزدیک به اورشلیم رسیده، واردبیت‌فاجی نزد کوه زیتون شدند. آنگاه عیسی دو نفر از شاگردان خود را فرستاده،
\par 2 بدیشان گفت: «در این قریه‌ای که پیش روی شمااست بروید و در حال، الاغی با کره‌اش بسته خواهید یافت. آنها را باز کرده، نزد من آورید.
\par 3 وهرگاه کسی به شما سخنی گوید، بگویید خداوندبدینها احتیاج دارد که فی الفور آنها را خواهدفرستاد.»
\par 4 و این همه واقع شد تا سخنی که نبی گفته است تمام شود
\par 5 که «دختر صهیون را گوییداینک پادشاه تو نزد تو می‌آید با فروتنی و سواره بر حمار و بر کره الاغ.»
\par 6 پس شاگردان رفته، آنچه عیسی بدیشان امر فرمود، بعمل آوردند
\par 7 و الاغ را با کره آورده، رخت خود را بر آنها انداختند واو بر آنها سوار شد.
\par 8 و گروهی بسیار، رختهای خود را در راه گسترانیدند و جمعی از درختان شاخه‌ها بریده، در راه می‌گستردند.
\par 9 و جمعی ازپیش و پس او رفته، فریادکنان می‌گفتند: «هوشیعانا پسر داودا، مبارک باد کسی‌که به اسم خداوند می‌آید! هوشیعانا در اعلی علیین!»
\par 10 وچون وارد اورشلیم شد، تمام شهر به آشوب آمده، می‌گفتند: «این کیست؟»
\par 11 آن گروه گفتند: «این است عیسی نبی از ناصره جلیل.»
\par 12 پس عیسی داخل هیکل خدا گشته، جمیع کسانی را که در هیکل خرید و فروش می‌کردند، بیرون نمود و تختهای صرافان و کرسیهای کبوترفروشان را واژگون ساخت.
\par 13 و ایشان را گفت: «مکتوب است که خانه من خانه دعا نامیده می‌شود. لیکن شما مغاره دزدانش ساخته‌اید.»
\par 14 و کوران و شلان در هیکل، نزد او آمدند وایشان را شفا بخشید.
\par 15 اما روسای کهنه و کاتبان چون عجایبی که از او صادر می‌گشت و کودکان را که در هیکل فریاد برآورده، «هوشیعانا پسر داودا» می‌گفتنددیدند، غضبناک گشته،
\par 16 به وی گفتند: «نمی شنوی آنچه اینها می‌گویند؟» عیسی بدیشان گفت: «بلی مگر نخوانده‌اید این که از دهان کودکان و شیرخوارگان حمد را مهیا ساختی؟»
\par 17 پس ایشان را واگذارده، از شهر بسوی بیت عنیارفته، در آنجا شب را بسر برد.
\par 18 بامدادان چون به شهر مراجعت می‌کرد، گرسنه شد.
\par 19 و در کناره راه یک درخت انجیر دیده، نزد آن آمد و جز برگ بر آن هیچ نیافت. پس آن را گفت: «از این به بعد میوه تا به ابد بر تونشود!» که در ساعت درخت انجیر خشکید!
\par 20 چون شاگردانր´ این را دیدند، متعجب شده، گفتند: «چه بسیار زود درخت انحیر خشک شده است!»
\par 21 عیسی در جواب ایشان گفت: «هرآینه به شما می‌گویم اگر ایمان می‌داشتید و شک نمی نمودید، نه همین را که به درخت انجیر شدمی کردید، بلکه هر گاه بدین کوه می‌گفتید "منتقل شده به دریا افکنده شو" چنین می‌شد.
\par 22 و هرآنچه با‌ایمان به دعا طلب کنید، خواهید یافت.»
\par 23 و چون به هیکل درآمده، تعلیم می‌داد، روسای کهنه و مشایخ قوم نزد او آمده، گفتند: «به چه قدرت این اعمال را می‌نمایی و کیست که این قدرت را به تو داده است؟»
\par 24 عیسی در جواب ایشان گفت: «من نیز از شما سخنی می‌پرسم. اگرآن را به من گویید، من هم به شما گویم که این اعمال را به چه قدرت می‌نمایم:
\par 25 تعمید یحیی از کجا بود؟ از آسمان یا از انسان؟» ایشان با خودتفکر کرده، گفتند که «اگر گوییم از آسمان بود، هرآینه گوید پس چرا به وی ایمان نیاوردید.
\par 26 واگر گوییم از انسان بود، از مردم می‌ترسیم زیراهمه یحیی را نبی می‌دانند.»
\par 27 پس در جواب عیسی گفتند: «نمی دانیم.» بدیشان گفت: «من هم شما را نمی گویم که به چه قدرت این کارها رامی کنم.
\par 28 لیکن چه گمان دارید؟ شخصی را دو پسربود. نزد نخستین آمده، گفت: "ای فرزند امروز به تاکستان من رفته، به‌کار مشغول شو."
\par 29 درجواب گفت: "نخواهم رفت." اما بعد پشیمان گشته، برفت.
\par 30 و به دومین نیز همچنین گفت. اودر جواب گفت: "ای آقا من می‌روم." ولی نرفت.
\par 31 کدام‌یک از این دو خواهش پدر را به‌جاآورد؟» گفتند: «اولی.» عیسی بدیشان گفت: «هرآینه به شما می‌گویم که باجگیران و فاحشه هاقبل از شما داخل ملکوت خدا می‌گردند،
\par 32 زانرو که یحیی از راه عدالت نزد شما آمد وبدو ایمان نیاوردید، اما باجگیران و فاحشه‌ها بدوایمان آوردند و شما چون دیدید آخر هم پشیمان نشدید تا بدو ایمان آورید.
\par 33 و مثلی دیگر بشنوید: صاحب‌خانه‌ای بودکه تاکستانی غرس نموده، خطیره‌ای گردش کشید و چرخشتی در آن کند و برجی بنا نمود. پس آن را به دهقانان سپرده، عازم سفر شد.
\par 34 وچون موسم میوه نزدیک شد، غلامان خود را نزددهقانان فرستاد تا میوه های او را بردارند.
\par 35 امادهقانان غلامانش را گرفته، بعضی را زدند وبعضی را کشتند و بعضی را سنگسار نمودند.
\par 36 باز غلامان دیگر، بیشتر از اولین فرستاده، بدیشان نیز به همانطور سلوک نمودند.
\par 37 بالاخره پسر خود را نزد ایشان فرستاده، گفت: "پسر مرا حرمت خواهند داشت."
\par 38 اما دهقانان چون پسررا دیدند با خود گفتند: "این وارث است. بیایید اورا بکشیم و میراثش را ببریم."
\par 39 آنگاه او راگرفته، بیرون تاکستان افکنده، کشتند.
\par 40 پس چون مالک تاکستان آید، به آن دهقانان چه خواهد کرد؟»
\par 41 گفتند: «البته آن بدکاران را به سختی هلاک خواهد کرد و باغ را به باغبانان دیگرخواهد سپرد که میوه هایش را در موسم بدودهند.»
\par 42 عیسی بدیشان گفت: «مگر در کتب هرگزنخوانده‌اید این که سنگی را که معمارانش ردنمودند، همان سر زاویه شده است. این از جانب خداوند آمد و در نظر ما عجیب است.
\par 43 از این جهت شما را می‌گویم که ملکوت خدا از شماگرفته شده، به امتی که میوه‌اش را بیاورند، عطاخواهد شد.
\par 44 و هر‌که بر آن سنگ افتد، منکسرشود و اگر آن بر کسی افتد، نرمش سازد.»
\par 45 وچون روسای کهنه و فریسیان مثلهایش راشنیدند، دریافتند که درباره ایشان می‌گوید.وچون خواستند او را گرفتار کنند، از مردم ترسیدند زیرا که او را نبی می‌دانستند.
\par 46 وچون خواستند او را گرفتار کنند، از مردم ترسیدند زیرا که او را نبی می‌دانستند.

\chapter{22}

\par 1 و عیسی توجه نموده، باز به مثلها ایشان را خطاب کرده، گفت:
\par 2 «ملکوت آسمان پادشاهی را ماند که برای پسر خویش عروسی کرد.
\par 3 و غلامان خود را فرستاد تا دعوت‌شدگان را به عروسی بخوانند و نخواستند بیایند.
\par 4 باز غلامان دیگر روانه نموده، فرمود: "دعوت‌شدگان را بگویید که اینک خوان خود را حاضر ساخته‌ام و گاوان و پرواریهای من کشته شده و همه‌چیز آماده است، به عروسی بیایید."
\par 5 ولی ایشان بی‌اعتنایی نموده، راه خود را گرفتند، یکی به مزرعه خود و دیگری به تجارت خویش رفت.
\par 6 و دیگران غلامان او را گرفته، دشنام داده، کشتند.
\par 7 پادشاه چون شنید، غضب نموده، لشکریان خود را فرستاده، آن قاتلان را به قتل رسانید و شهر ایشان را بسوخت.
\par 8 آنگاه غلامان خود را فرمود: "عروسی حاضر است؛ لیکن دعوت‌شدگان لیاقت نداشتند.
\par 9 الان به شوارع عامه بروید و هر‌که را بیابید به عروسی بطلبید."
\par 10 پس آن غلامان به‌سر راهها رفته، نیک و بد هرکه را یافتند جمع کردند، چنانکه خانه عروسی ازمجلسیان مملو گشت.
\par 11 آنگاه پادشاه بجهت دیدن اهل مجلس داخل شده، شخصی را درآنجا دید که جامه عروسی در بر ندارد.
\par 12 بدوگفت: "ای عزیز چطور در اینجا آمدی و حال آنکه جامه عروسی در بر نداری؟" او خاموش شد.
\par 13 آنگاه پادشاه خادمان خود را فرمود: "این شخص را دست و پا بسته بردارید و در ظلمت خارجی اندازید، جایی که گریه و فشار دندان باشد."
\par 14 زیرا طلبیدگان بسیارند و برگزیدگان کم.»
\par 15 پس فریسیان رفته، شورا نمودند که چطوراو را در گفتگو گرفتار سازند.
\par 16 و شاگردان خودرا با هیرودیان نزد وی فرستاده، گفتند: «استادامی دانیم که صادق هستی و طریق خدا را به راستی تعلیم می‌نمایی و از کسی باک نداری زیراکه به ظاهر خلق نمی نگری.
\par 17 پس به ما بگو رای تو چیست. آیا جزیه دادن به قیصر رواست یا نه؟»
\par 18 عیسی شرارت ایشان را درک کرده، گفت: «ای ریاکاران، چرا مرا تجربه می‌کنید؟
\par 19 سکه جزیه را به من بنمایید.» ایشان دیناری نزد وی آوردند.
\par 20 بدیشان گفت: «این صورت و رقم از آن کیست؟»
\par 21 بدو گفتند: «از آن قیصر.» بدیشان گفت: «مال قیصر را به قیصر ادا کنید و مال خدا رابه خدا!»
\par 22 چون ایشان شنیدند، متعجب شدندو او را واگذارده، برفتند.
\par 23 و در همان روز، صدوقیان که منکر قیامت هستند نزد او آمده، سوال نموده،
\par 24 گفتند: «ای استاد، موسی گفت اگر کسی بی‌اولاد بمیرد، می‌باید برادرش زن او را نکاح کند تا نسلی برای برادر خود پیدا نماید.
\par 25 باری در میان ما هفت برادر بودند که اول زنی گرفته، بمرد و چون اولادی نداشت زن را به برادر خود ترک کرد.
\par 26 وهمچنین دومین و سومین تا هفتمین.
\par 27 و آخر ازهمه آن زن نیز مرد.
\par 28 پس او در قیامت، زن کدام‌یک از آن هفت خواهد بود زیرا که همه او راداشتند؟»
\par 29 عیسی در جواب ایشان گفت: «گمراه هستید از این‌رو که کتاب و قوت خدا رادر نیافته‌اید،
\par 30 زیرا که در قیامت، نه نکاح می‌کنند و نه نکاح کرده می‌شوند بلکه مثل ملائکه خدا در آسمان می‌باشند.
\par 31 اما درباره قیامت مردگان، آیا نخوانده‌اید کلامی را که خدا به شما گفته است،
\par 32 من هستم خدای ابراهیم و خدای اسحاق و خدای یعقوب؟ خدا، خدای مردگان نیست بلکه خدای زندگان است.»
\par 33 و آن گروه چون شنیدند، از تعلیم وی متحیر شدند.
\par 34 اما چون فریسیان شنیدند که صدوقیان را مجاب نموده است، با هم جمع شدند.
\par 35 و یکی از ایشان که فقیه بود، از وی به طریق امتحان سوال کرده، گفت:
\par 36 «ای استاد، کدام حکم در شریعت بزرگتر است؟»
\par 37 عیسی وی را گفت: «اینکه خداوند خدای خود را به همه دل و تمامی نفس و تمامی فکر خود محبت نما.
\par 38 این است حکم اول و اعظم.
\par 39 و دوم مثل آن است یعنی همسایه خود را مثل خود محبت نما.
\par 40 بدین دو حکم، تمام تورات و صحف انبیامتعلق است.»
\par 41 و چون فریسیان جمع بودند، عیسی ازایشان پرسیده،
\par 42 گفت: «درباره مسیح چه گمان می‌برید؟ او پسر کیست؟» بدو گفتند: «پسر داود.»
\par 43 ایشان را گفت: «پس چطور داود در روح، او راخداوند می‌خواند؟ چنانکه می‌گوید:
\par 44 "خداوندبه خداوند من گفت، به‌دست راست من بنشین تادشمانان تو را پای انداز تو سازم."
\par 45 پس هرگاه داود او را خداوند می‌خواند، چگونه پسرش می‌باشد؟»و هیچ‌کس قدرت جواب وی هرگز نداشت و نه کسی از آن روز دیگرجرات سوال کردن از او نمود.
\par 46 و هیچ‌کس قدرت جواب وی هرگز نداشت و نه کسی از آن روز دیگرجرات سوال کردن از او نمود.

\chapter{23}

\par 1 آنگاه عیسی آن جماعت و شاگردان خود را خطاب کرده،
\par 2 گفت: «کاتبان وفریسیان بر کرسی موسی نشسته‌اند.
\par 3 پس آنچه به شما گویند، نگاه دارید و به‌جا آورید، لیکن مثل اعمال ایشان مکنید زیرا می‌گویند ونمی کنند.
\par 4 زیرا بارهای گران و دشوار رامی بندند و بر دوش مردم می‌نهند و خودنمی خواهند آنها را به یک انگشت حرکت دهند.
\par 5 و همه کارهای خود را می‌کنند تا مردم، ایشان را ببینند. حمایلهای خود را عریض و دامنهای قبای خود را پهن می‌سازند،
\par 6 و بالا نشستن درضیافتها و کرسیهای صدر در کنایس را دوست می‌دارند،
\par 7 و تعظیم در کوچه‌ها را و اینکه مردم ایشان را آقا آقا بخوانند.
\par 8 لیکن شماآقا خوانده مشوید، زیرا استاد شما یکی است یعنی مسیح و جمیع شما برادرانید.
\par 9 و هیچ‌کس را بر زمین، پدر خود مخوانید زیرا پدر شمایکی است که در آسمان است.
\par 10 و پیشواخوانده مشوید، زیرا پیشوای شما یکی است یعنی مسیح.
\par 11 و هر‌که از شما بزرگتر باشد، خادم شما بود.
\par 12 و هر‌که خود را بلند کند، پست گردد و هر‌که خود را فروتن سازد سرافرازگردد.
\par 13 وای بر شما‌ای کاتبان و فریسیان ریاکار که در ملکوت آسمان را به روی مردم می‌بندید، زیراخود داخل آن نمی شوید و داخل‌شوندگان را ازدخول مانع می‌شوید.
\par 14 وای بر شما‌ای کاتبان وفریسیان ریاکار، زیرا خانه های بیوه‌زنان رامی بلعید و از روی ریا نماز را طویل می‌کنید؛ ازآنرو عذاب شدیدتر خواهید یافت.
\par 15 وای برشما‌ای کاتبان و فریسیان ریاکار، زیرا که بر و بحررا می‌گردید تا مریدی پیدا کنید و چون پیدا شداو را دو مرتبه پست‌تر از خود، پسر جهنم می‌سازید!
\par 16 وای بر شما‌ای راهنمایان کور که می‌گویید "هر‌که به هیکل قسم خورد باکی نیست لیکن هر‌که به طلای هیکل قسم خورد باید وفاکند."
\par 17 ‌ای نادانان و نابینایان، آیا کدام افضل است؟ طلا یا هیکلی که طلا را مقدس می‌سازد؟
\par 18 "و هر‌که به مذبح قسم خورد باکی نیست لیکن هر‌که به هدیه‌ای که بر آن است قسم خورد، بایدادا کند."
\par 19 ‌ای جهال و کوران، کدام افضل است؟ هدیه یا مذبح که هدیه را تقدیس می‌نماید؟
\par 20 پس هر‌که به مذبح قسم خورد، به آن و به هرچه بر آن است قسم خورده است؛
\par 21 و هر‌که به هیکل قسم خورد، به آن و به او که در آن ساکن است، قسم خورده است؛
\par 22 و هر‌که به آسمان قسم خورد، به کرسی خدا و به او که بر آن نشسته است، قسم خورده باشد.
\par 23 وای بر شما‌ای کاتبان و فریسیان ریاکار که نعناع و شبت و زیره را عشر می‌دهید و اعظم احکام شریعت، یعنی عدالت و رحمت و ایمان را ترک کرده‌اید! می‌بایست آنها را به‌جا آورده، اینها را نیز ترک نکرده باشید.
\par 24 ‌ای رهنمایان کورکه پشه را صافی می‌کنید و شتر را فرو می‌برید!
\par 25 وای بر شما‌ای کاتبان و فریسیان ریاکار، از آن رو که بیرون پیاله و بشقاب را پاک می‌نمایید ودرون آنها مملو از جبر و ظلم است.
\par 26 ‌ای فریسی کور، اول درون پیاله و بشقاب را طاهرساز تا بیرونش نیز طاهر شود!
\par 27 وای بر شما‌ای کاتبان و فریسیان ریاکار که چون قبور سفید شده می‌باشید که از بیرون، نیکو می‌نماید لیکن درون آنها از استخوانهای مردگان و سایر نجاسات پراست!
\par 28 همچنین شما نیز ظاهر به مردم عادل می‌نمایید، لیکن باطن از ریاکاری و شرارت مملوهستید.
\par 29 وای بر شما‌ای کاتبان و فریسیان ریاکار که قبرهای انبیا را بنا می‌کنید و مدفنهای صادقان رازینت می‌دهید،
\par 30 و می‌گویید: "اگر در ایام پدران خود می‌بودیم، در ریختن خون انبیا با ایشان شریک نمی شدیم!"
\par 31 پس بر خود شهادت می‌دهید که فرزندان قاتلان انبیا هستید.
\par 32 پس شما پیمانه پدران خود را لبریز کنید!
\par 33 ‌ای ماران و افعی‌زادگان! چگونه از عذاب جهنم فرارخواهید کرد؟
\par 34 لهذا الحال انبیا و حکماء وکاتبان نزد شما می‌فرستم و بعضی را خواهیدکشت و به دار خواهید کشید و بعضی را درکنایس خود تازیانه زده، از شهر به شهر خواهیدراند،
\par 35 تا همه خونهای صادقان که بر زمین ریخته شد بر شما وارد آید، از خون هابیل صدیق تا خون زکریا ابن برخیا که او را در میان هیکل ومذبح کشتید.
\par 36 هرآینه به شما می‌گویم که این همه بر این طایفه خواهد آمد!
\par 37 «ای اورشلیم، اورشلیم، قاتل انبیا وسنگسار کننده مرسلان خود! چند مرتبه‌خواستم فرزندان تو را جمع کنم، مثل مرغی که جوجه های خود را زیر بال خود جمع می‌کند ونخواستید!
\par 38 اینک خانه شما برای شما ویران گذارده می‌شود.زیرا به شما می‌گویم از این پس مرا نخواهید دید تا بگویید مبارک است او که به نام خداوند می‌آید.»
\par 39 زیرا به شما می‌گویم از این پس مرا نخواهید دید تا بگویید مبارک است او که به نام خداوند می‌آید.»

\chapter{24}

\par 1 پس عیسی از هیکل بیرون شده، برفت. و شاگردانش پیش آمدند تا عمارتهای هیکل را بدو نشان دهند.
\par 2 عیسی ایشان را گفت: «آیا همه این چیزها را نمی بینید؟ هرآینه به شمامی گویم در اینجا سنگی بر سنگی گذارده نخواهد شد که به زیر افکنده نشود!»
\par 3 و چون به کوه زیتون نشسته بود، شاگردانش در خلوت نزدوی آمده، گفتند: «به ما بگو که این امور کی واقع می‌شود و نشان آمدن تو و انقضای عالم چیست.»
\par 4 عیسی در جواب ایشان گفت: «زنهار کسی شما را گمراه نکند!
\par 5 زآنرو که بسا به نام من آمده خواهند گفت که من مسیح هستم و بسیاری را گمراه خواهند کرد.
\par 6 و جنگها و اخبار جنگها راخواهید شنید. زنهار مضطرب مشوید زیرا که وقوع این همه لازم است، لیکن انتها هنوز نیست.
\par 7 زیرا قومی با قومی و مملکتی با مملکتی مقاومت خواهند نمود و قحطیها و وباها وزلزله‌ها در جایها پدید آید.
\par 8 اما همه اینها آغازدردهای زه است.
\par 9 آنگاه شما را به مصیبت سپرده، خواهند کشت و جمیع امت‌ها بجهت اسم من از شما نفرت کنند.
\par 10 و در آن زمان، بسیاری لغزش خورده، یکدیگر را تسلیم کنند واز یکدیگر نفرت گیرند.
\par 11 و بسا انبیای کذبه ظاهر شده، بسیاری را گمراه کنند.
\par 12 و بجهت افزونی گناه محبت بسیاری سرد خواهد شد.
\par 13 لیکن هر‌که تا به انتها صبر کند، نجات یابد.
\par 14 و به این بشارت ملکوت در تمام عالم موعظه خواهد شد تا بر جمیع امت‌ها شهادتی شود؛ آنگاه انتها خواهد رسید.
\par 15 «پس چون مکروه ویرانی را که به زبان دانیال نبی گفته شده است، در مقام مقدس بر پاشده بینید هر‌که خواند دریافت کند
\par 16 آنگاه هر‌که در یهودیه باشد به کوهستان بگریزد؛
\par 17 وهر‌که بر بام باشد، بجهت برداشتن چیزی از خانه به زیر نیاید؛
\par 18 و هر‌که در مزرعه است، بجهت برداشتن رخت خود برنگردد.
\par 19 لیکن وای برآبستنان و شیر دهندگان در آن ایام!
\par 20 پس دعاکنید تا فرار شما در زمستان یا در سبت نشود،
\par 21 زیرا که در آن زمان چنان مصیبت عظیمی ظاهر می‌شود که از ابتدا عالم تا کنون نشده ونخواهد شد!
\par 22 و اگر آن ایام کوتاه نشدی، هیچ بشری نجات نیافتی، لیکن بخاطر برگزیدگان، آن روزها کوتاه خواهد شد.
\par 23 آنگاه اگر کسی به شما گوید: "اینک مسیح در اینجا یا در آنجا است " باور مکنید،
\par 24 زیرا که مسیحیان کاذب و انبیا کذبه ظاهر شده، علامات ومعجزات عظیمه چنان خواهند نمود که اگرممکن بودی برگزیدگان را نیز گمراه کردندی.
\par 25 اینک شما را پیش خبر دادم.
\par 26 «پس اگر شما را گویند: اینک درصحراست، بیرون مروید یا آنکه در خلوت است، باور مکنید،
\par 27 زیرا همچنان‌که برق ازمشرق ساطع شده، تا به مغرب ظاهر می‌شود، ظهور پسر انسان نیز چنین خواهد شد.
\par 28 و هرجا که مرداری باشد، کرکسان در آنجا جمع شوند.
\par 29 و فور بعد از مصیبت آن ایام، آفتاب تاریک گردد و ماه نور خود را ندهد و ستارگان ازآسمان فرو ریزند و قوتهای افلاک متزلزل گردد.
\par 30 آنگاه علامت پسر انسان در آسمان پدید گرددو در آن وقت، جمیع طوایف زمین سینه زنی کنندو پسر انسان را بینند که بر ابرهای آسمان، با قوت و جلال عظیم می‌آید؛
\par 31 و فرشتگان خود را باصور بلند آواز فرستاده، برگزیدگان او را ازبادهای اربعه از کران تا بکران فلک فراهم خواهندآورد.
\par 32 «پس از درخت انجیر مثلش را فرا‌گیرید که چون شاخه‌اش نازک شده، برگها می‌آورد، می فهمید که تابستان نزدیک است.
\par 33 همچنین شما نیز چون این همه را بینید، بفهمید که نزدیک بلکه بر در است.
\par 34 هرآینه به شما می‌گویم تا این همه واقع نشود، این طایفه نخواهد گذشت.
\par 35 آسمان و زمین زایل خواهد شد، لیکن سخنان من هرگز زایل نخواهد شد.
\par 36 «اما از آن روز و ساعت هیچ‌کس اطلاع ندارد، حتی ملائکه آسمان جز پدر من و بس.
\par 37 لیکن چنانکه ایام نوح بود، ظهور پسر انسان نیزچنان خواهد بود.
\par 38 زیرا همچنان‌که در ایام قبل از طوفان می‌خوردند و می‌آشامیدند و نکاح می‌کردند و منکوحه می‌شدند تا روزی که نوح داخل کشتی گشت،
\par 39 و نفهمیدند تا طوفان آمده، همه را ببرد، همچنین ظهور پسر انسان نیزخواهد بود.
\par 40 آنگاه دو نفری که در مزرعه‌ای می‌باشند، یکی گرفته و دیگری واگذارده شود.
\par 41 و دو زن که دستاس می‌کنند، یکی گرفته ودیگری رها شود.
\par 42 پس بیدار باشید زیرا که نمی دانید در کدام ساعت خداوند شما می‌آید.
\par 43 لیکن این را بدانید که اگر صاحب‌خانه می‌دانست در چه پاس از شب دزد می‌آید، بیدارمی ماند و نمی گذاشت که به خانه‌اش نقب زند.
\par 44 لهذا شما نیز حاضر باشید، زیرا در ساعتی که گمان نبرید، پسر انسان می‌آید.
\par 45 «پس آن غلام امین و دانا کیست که آقایش او را بر اهل خانه خود بگمارد تا ایشان را در وقت معین خوراک دهد؟
\par 46 خوشابحال آن غلامی که چون آقایش آید، او را در چنین کار مشغول یابد.
\par 47 هرآینه به شما می‌گویم که او را بر تمام مایملک خود خواهد گماشت.
\par 48 لیکن هرگاه آن غلام شریر با خود گوید که آقای من در‌آمدن تاخیر می‌نماید،
\par 49 و شروع کند به زدن همقطاران خود و خوردن و نوشیدن با میگساران،
\par 50 هرآینه آقای آن غلام آید، در روزی که منتظرنباشد و در ساعتی که نداند،و او را دو پاره کرده، نصیبش را با ریاکاران قرار دهد در مکانی که گریه و فشار دندان خواهد بود.
\par 51 و او را دو پاره کرده، نصیبش را با ریاکاران قرار دهد در مکانی که گریه و فشار دندان خواهد بود.

\chapter{25}

\par 1 «در آن زمان ملکوت آسمان مثل ده باکره خواهد بود که مشعلهای خود رابرداشته، به استقبال داماد بیرون رفتند.
\par 2 و ازایشان پنج دانا و پنج نادان بودند.
\par 3 اما نادانان مشعلهای خود را برداشته، هیچ روغن با خودنبردند.
\par 4 لیکن دانایان، روغن در ظروف خود بامشعلهای خویش برداشتند.
\par 5 و چون آمدن دامادبطول انجامید، همه پینکی زده، خفتند.
\par 6 و درنصف شب صدایی بلند شد که "اینک دامادمی آید به استقبال وی بشتابید."
\par 7 پس تمامی آن باکره‌ها برخاسته، مشعلهای خود را اصلاح نمودند.
\par 8 و نادانان، دانایان را گفتند: "از روغن خود به ما دهید زیرا مشعلهای ما خاموش می‌شود."
\par 9 اما دانایان در جواب گفتند: "نمی شود، مبادا ما و شما را کفاف ندهد. بلکه نزدفروشندگان رفته، برای خود بخرید."
\par 10 و درحینی که ایشان بجهت خرید می‌رفتند، داماد برسید و آنانی که حاضر بودند، با وی به عروسی داخل شده، در بسته گردید.
\par 11 بعد از آن، باکره های دیگر نیز آمده، گفتند: "خداوندا برای ماباز کن."
\par 12 او در جواب گفت: "هرآینه به شمامی گویم شما را نمی شناسم."
\par 13 پس بیدار باشیدزیرا که آن روز و ساعت را نمی دانید.
\par 14 «زیرا چنانکه مردی عازم سفر شده، غلامان خود را طلبید و اموال خود را بدیشان سپرد،
\par 15 یکی را پنج قنطار و دیگری را دو وسومی را یک داد؛ هر یک را بحسب استعدادش. و بی‌درنگ متوجه سفر شد.
\par 16 پس آنکه پنج قنطار یافته بود، رفته و با آنها تجارت نموده، پنج قنطار دیگر سود کرد.
\par 17 و همچنین صاحب دوقنطار نیز دو قنطار دیگر سود گرفت.
\par 18 اما آنکه یک قنطار گرفته بود، رفته زمین را کند و نقد آقای خود را پنهان نمود.
\par 19 «و بعد از مدت مدیدی، آقای آن غلامان آمده، از ایشان حساب خواست.
\par 20 پس آنکه پنج قنطار یافته بود، پیش آمده، پنج قنطار دیگرآورده، گفت: خداوندا پنج قنطار به من سپردی، اینک پنج قنطار دیگر سود کردم."
\par 21 آقای او به وی گفت: آفرین‌ای غلام نیک متدین! بر چیزهای اندک امین بودی، تو را بر چیزهای بسیار خواهم گماشت. به شادی خداوند خود داخل شو!
\par 22 وصاحب دو قنطار نیز آمده، گفت: ای آقا دو قنطارتسلیم من نمودی، اینک دو قنطار دیگر سودیافته‌ام.
\par 23 آقایش وی را گفت: آفرین‌ای غلام نیک متدین! بر چیزهای کم امین بودی، تو را بر چیزهای بسیار می‌گمارم. در خوشی خداوندخود داخل شو!
\par 24 پس آنکه یک قنطار گرفته بود، پیش آمده، گفت: ای آقا چون تو رامی شناختم که مرد درشت خویی می‌باشی، ازجایی که نکاشته‌ای می‌دروی و از جایی که نیفشانده‌ای جمع می‌کنی،
\par 25 پس ترسان شده، رفتم و قنطار تو را زیر زمین نهفتم. اینک مال توموجود است.
\par 26 آقایش در جواب وی گفت: ای غلام شریر بیکاره! دانسته‌ای که از جایی که نکاشته‌ام میدروم و از مکانی که نپاشیده‌ام، جمع می‌کنم.
\par 27 از همین جهت تو را می‌بایست نقد مرابه صرافان بدهی تا وقتی که بیایم مال خود را باسود بیابم.
\par 28 الحال آن قنطار را از او گرفته، به صاحب ده قنطار بدهید.
\par 29 زیرا به هر‌که داردداده شود و افزونی یابد و از آنکه ندارد آنچه داردنیز گرفته شود.
\par 30 و آن غلام بی‌نفع را در ظلمت خارجی اندازید، جایی که گریه و فشار دندان خواهد بود.
\par 31 «اما چون پسر انسان در جلال خود باجمیع ملائکه مقدس خویش آید، آنگاه بر کرسی جلال خود خواهد نشست،
\par 32 و جمیع امت‌ها درحضور او جمع شوند و آنها را از همدیگر جدامی کند به قسمی که شبان میشها را از بزها جدامی کند.
\par 33 و میشها را بر دست راست و بزها را برچپ خود قرار دهد.
\par 34 آنگاه پادشاه به اصحاب طرف راست گوید: بیایید‌ای برکت یافتگان از پدر من و ملکوتی را که از ابتدای عالم برای شما آماده‌شده است، به میراث گیرید.
\par 35 زیرا چون گرسنه بودم مرا طعام دادید، تشنه بودم سیرآبم نمودید، غریب بودم مرا جا دادید،
\par 36 عریان بودم مراپوشانیدید، مریض بودم عیادتم کردید، در حبس بودم دیدن من آمدید.
\par 37 آنگاه عادلان به پاسخ گویند: ای خداوند، کی گرسنه ات دیدیم تاطعامت دهیم، یا تشنه ات یافتیم تا سیرآبت نماییم،
\par 38 یا کی تو را غریب یافتیم تا تو را جادهیم یا عریان تا بپوشانیم،
\par 39 و کی تو را مریض یامحبوس یافتیم تا عیادتت کنیم؟
\par 40 پادشاه درجواب ایشان گوید: هرآینه به شما می‌گویم، آنچه به یکی از این برادران کوچکترین من کردید، به من کرده‌اید.
\par 41 «پس اصحاب طرف چپ را گوید: ای ملعونان، از من دور شوید در آتش جاودانی که برای ابلیس و فرشتگان او مهیا شده است.
\par 42 زیراگرسنه بودم مرا خوراک ندادید، تشنه بودم مراآب ندادید،
\par 43 غریب بودم مرا جا ندادید، عریان بودم مرا نپوشانیدید، مریض و محبوس بودم عیادتم ننمودید.
\par 44 پس ایشان نیز به پاسخ گویند: ای خداوند، کی تو را گرسنه یا تشنه یاغریب یا برهنه یا مریض یا محبوس دیده، خدمتت نکردیم؟
\par 45 آنگاه در جواب ایشان گوید: هرآینه به شما می‌گویم، آنچه به یکی از این کوچکان نکردید، به من نکرده‌اید.و ایشان درعذاب جاودانی خواهند رفت، اما عادلان درحیات جاودانی.»
\par 46 و ایشان درعذاب جاودانی خواهند رفت، اما عادلان درحیات جاودانی.»

\chapter{26}

\par 1 و چون عیسی همه این سخنان را به اتمام رسانید، به شاگردان خود گفت:
\par 2 «می‌دانید که بعد از دو روز عید فصح است که پسر انسان تسلیم کرده می‌شود تا مصلوب گردد.»
\par 3 آنگاه روسای کهنه و کاتبان و مشایخ قوم دردیوانخانه رئیس کهنه که قیافا نام داشت جمع شده،
\par 4 شورا نمودند تا عیسی را به حیله گرفتارساخته، به قتل رسانند.
\par 5 اما گفتند: «نه در وقت عید مبادا آشوبی در قوم بر پا شود.»
\par 6 و هنگامی که عیسی در بیت عنیا در خانه شمعون ابرص شد،
\par 7 زنی با شیشه‌ای عطر گرانبهانزد او آمده، چون بنشست بر سر وی ریخت.
\par 8 اماشاگردانش چون این را دیدند، غضب نموده، گفتند: «چرا این اسراف شده است؟
\par 9 زیرا ممکن بود این عطر به قیمت گران فروخته و به فقرا داده شود.»
\par 10 عیسی این را درک کرده، بدیشان گفت: «چرا بدین زن زحمت می‌دهید؟ زیرا کار نیکو به من کرده است.
\par 11 زیرا که فقرا را همیشه نزد خوددارید اما مرا همیشه ندارید.
\par 12 و این زن که این عطر را بر بدنم مالید، بجهت دفن من کرده است.
\par 13 هرآینه به شما می‌گویم هر جایی که درتمام عالم بدین بشارت موعظه کرده شود، کاراین زن نیز بجهت یادگاری او مذکور خواهدشد.»
\par 14 «آنگاه یکی از آن دوازده که به یهودای اسخریوطی مسمی بود، نزد روسای کهنه رفته،
\par 15 گفت: «مرا چند خواهید داد تا او را به شماتسلیم کنم؟ «ایشان سی پاره نقره با وی قراردادند.
\par 16 و از آن وقت در صدد فرصت شد تا اورا بدیشان تسلیم کند.
\par 17 پس در روز اول عید فطیر، شاگردان نزدعیسی آمده، گفتند: «کجا می‌خواهی فصح راآماده کنیم تا بخوری؟»
\par 18 گفت: «به شهر، نزدفلان کس رفته، بدو گویید: "استاد می‌گوید وقت من نزدیک شد و فصح را در خانه تو با شاگردان خود صرف می‌نمایم."»
\par 19 شاگردان چنانکه عیسی ایشان را امر فرمود کردند و فصح را مهیاساختند.
\par 20 چون وقت شام رسید با آن دوازده بنشست.
\par 21 و وقتی که ایشان غذا می‌خوردند، اوگفت: «هرآینه به شما می‌گویم که یکی از شما مراتسلیم می‌کند!»
\par 22 پس بغایت غمگین شده، هریک از ایشان به وی سخن آغاز کردند که «خداوندا آیا من آنم؟»
\par 23 او در جواب گفت: «آنکه دست با من در قاب فرو برد، همان کس مرا تسلیم نماید!
\par 24 هرآینه پسر انسان به همانطور که درباره او مکتوب است رحلت می‌کند. لیکن وای بر آنکسی‌که پسر انسان بدست او تسلیم شود! آن شخص را بهتر بودی که تولد نیافتی!»
\par 25 و یهوداکه تسلیم‌کننده وی بود، به جواب گفت: «ای استاد آیا من آنم؟» به وی گفت: «تو خود گفتی!»
\par 26 و چون ایشان غذا می‌خوردند، عیسی نان راگرفته، برکت داد و پاره کرده، به شاگردان داد وگفت: «بگیرید و بخورید، این است بدن من.»
\par 27 وپیاله را گرفته، شکر نمود و بدیشان داده، گفت: «همه شما از این بنوشید،
\par 28 زیرا که این است خون من در عهد جدید که در راه بسیاری بجهت آمرزش گناهان ریخته می‌شود.
\par 29 اما به شمامی گویم که بعد از این از میوه مو دیگر نخواهم نوشید تا روزی که آن را با شما در ملکوت پدرخود، تازه آشامم.»
\par 30 پس تسبیح خواندند و به سوی کوه زیتون روانه شدند.
\par 31 آنگاه عیسی بدیشان گفت: «همه شما امشب درباره من لغزش می‌خورید چنانکه مکتوب است که شبان را می‌زنم و گوسفندان گله پراکنده می‌شوند.
\par 32 لیکن بعد از برخاستنم، پیش از شما به جلیل خواهم رفت.»
\par 33 پطرس درجواب وی گفت: «هر گاه همه درباره تو لغزش خورند، من هرگز نخورم.»
\par 34 عیسی به وی گفت: «هرآینه به تو می‌گویم که در همین شب قبل ازبانک زدن خروس، سه مرتبه مرا انکار خواهی کرد!»
\par 35 پطرس به وی گفت: «هرگاه مردنم با تو لازم شود، هرگز تو را انکار نکنم!» و سایرشاگردان نیز همچنان گفتند.
\par 36 آنگاه عیسی با ایشان به موضعی که مسمی به جتسیمانی بود رسیده، به شاگردان خود گفت: «در اینجا بنشینید تا من رفته، در آنجا دعا کنم.»
\par 37 و پطرس و دو پسر زبدی را برداشته، بی‌نهایت غمگین و دردناک شد.
\par 38 پس بدیشان گفت: «نفس من از غایت الم مشرف به موت شده است. در اینجا مانده با من بیدار باشید.»
\par 39 پس قدری پیش رفته، به روی در‌افتاد و دعا کرده، گفت: «ای پدر من، اگر ممکن باشد این پیاله از من بگذرد؛ لیکن نه به خواهش من، بلکه به اراداه تو.»
\par 40 و نزدشاگردان خود آمده، ایشان را در خواب یافت. وبه پطرس گفت: «آیا همچنین نمی توانستید یک ساعت با من بیدار باشید؟
\par 41 بیدار باشید و دعاکنید تا در معرض آزمایش نیفتید! روح راغب است، لیکن جسم ناتوان.»
\par 42 و بار دیگر رفته، بازدعا نموده، گفت: «ای پدر من، اگر ممکن نباشد که این پیاله بدون نوشیدن از من بگذرد، آنچه اراده تواست بشود.»
\par 43 و آمده، باز ایشان را در خواب یافت زیرا که چشمان ایشان سنگین شده بود.
\par 44 پس ایشان را ترک کرده، رفت و دفعه سوم به همان کلام دعا کرد.
\par 45 آنگاه نزد شاگردان آمده، بدیشان گفت: «مابقی را بخوابید واستراحت کنید. الحال ساعت رسیده است که پسر انسان به‌دست گناهکاران تسلیم شود.
\par 46 برخیزید برویم. اینک تسلیم‌کننده من نزدیک است!»
\par 47 و هنوز سخن می‌گفت که ناگاه یهودا که یکی از آن دوازده بود با جمعی کثیر با شمشیرهاو چوبها از جانب روساء کهنه و مشایخ قوم آمدند.
\par 48 و تسلیم‌کننده او بدیشان نشانی داده، گفته بود: «هر‌که را بوسه زنم، همان است. او رامحکم بگیرید.»
\par 49 در ساعت نزد عیسی آمده، گفت: «سلام یا سیدی!» و او را بوسید.
\par 50 عیسی وی را گفت: «ای رفیق، از بهر‌چه آمدی؟» آنگاه پیش آمده، دست بر عیسی انداخته، او را گرفتند.
\par 51 و ناگاه یکی از همراهان عیسی دست آورده، شمشیر خود را از غلاف کشیده، بر غلام رئیس کهنه زد و گوشش را از تن جدا کرد.
\par 52 آنگاه عیسی وی را گفت: «شمشیر خود را غلاف کن، زیرا هر‌که شمشیر گیرد، به شمشیر هلاک گردد.
\par 53 آیا گمان می‌بری که نمی توانم الحال از پدرخود درخواست کنم که زیاده از دوازده فوج ازملائکه برای من حاضر سازد؟
\par 54 لیکن در این صورت کتب چگونه تمام گردد که همچنین می‌بایست بشود؟»
\par 55 در آن ساعت، به آن گروه گفت: گویا بر دزد بجهت گرفتن من با تیغها وچوبها بیرون آمدید! هر روز با شما در هیکل نشسته، تعلیم می‌دادم و مرا نگرفتید.
\par 56 لیکن این همه شد تا کتب انبیا تمام شود.» در آن وقت جمیع شاگردان او را واگذارده، بگریختند.
\par 57 و آنانی که عیسی را گرفته بودند، او را نزدقیافا رئیس کهنه جایی که کاتبان و مشایخ جمع بودند، بردند.
\par 58 اما پطرس از دور در عقب اوآمده، به خانه رئیس کهنه در‌آمد و با خادمان بنشست تا انجام کار را ببیند.
\par 59 پس روسای کهنه و مشایخ و تمامی اهل شورا طلب شهادت دروغ بر عیسی می‌کردند تا او را بقتل رسانند،
\par 60 لیکن نیافتند. با آنکه چند شاهد دروغ پیش آمدند، هیچ نیافتند. آخر دو نفر آمده،
\par 61 گفتند: «این شخص گفت: "می توانم هیکل خدا را خراب کنم و در سه روزش بنا نمایم."»
\par 62 پس رئیس کهنه برخاسته، بدو گفت: «هیچ جواب نمی دهی؟ چیست که اینها بر تو شهادت می‌دهند؟»
\par 63 اما عیسی خاموش ماند! تا آنکه رئیس کهنه روی به وی کرده، گفت: «تو را به خدای حی قسم می‌دهم مارا بگوی که تو مسیح پس خدا هستی یا نه؟»
\par 64 عیسی به وی گفت: «تو گفتی! و نیز شما رامی گویم بعد از این پسر انسان را خواهید دید که بر دست راست قوت نشسته، بر ابرهای آسمان می‌آید!»
\par 65 در ساعت رئیس کهنه رخت خود راچاک زده، گفت: «کفر گفت! دیگر ما را چه حاجت به شهود است؟ الحال کفرش را شنیدید!
\par 66 چه مصلحت می‌بینید؟» ایشان در جواب گفتند: «مستوجب قتل است!»
\par 67 آنگاه آب دهان بر رویش انداخته، او را طپانچه می‌زدند و بعضی سیلی زده،
\par 68 می‌گفتند: «ای مسیح، به ما نبوت کن! کیست که تو را زده است؟»
\par 69 اما پطرس در ایوان بیرون نشسته بود که ناگاه کنیزکی نزد وی آمده، گفت: «تو هم با عیسی جلیلی بودی!»
\par 70 او روبروی همه انکارنموده، گفت: «نمی دانم چه می‌گویی!»
\par 71 و چون به دهلیز بیرون رفت، کنیزی دیگر او را دیده، به حاضرین گفت: «این شخص نیز از رفقای عیسی ناصری است!»
\par 72 باز قسم خورده، انکار نمود که «این مرد را نمی شناسم.»
\par 73 بعد از چندی، آنانی که ایستاده بودند پیش آمده، پطرس را گفتند: «البته تو هم از اینها هستی زیرا که لهجه تو بر تودلالت می‌نماید!»
\par 74 پس آغاز لعن کردن و قسم خوردن نمود که «این شخص را نمی شناسم.» ودر ساعت خروس بانگ زدآنگاه پطرس سخن عیسی را به یاد آورد که گفته بود: قبل از بانگ زدن خروس، سه مرتبه مرا انکار خواهی کرد.» پس بیرون رفته زار‌زار بگریست.
\par 75 آنگاه پطرس سخن عیسی را به یاد آورد که گفته بود: قبل از بانگ زدن خروس، سه مرتبه مرا انکار خواهی کرد.» پس بیرون رفته زار‌زار بگریست.

\chapter{27}

\par 1 و چون صبح شد، همه روسای کهنه ومشایخ قوم بر عیسی شورا کردند که اورا هلاک سازند.
\par 2 پس او را بند نهاده، بردند و به پنطیوس پیلاطس والی تسلیم نمودند.
\par 3 در آن هنگام، چون یهودا تسلیم‌کننده او دید که بر او فتوا دادند، پشیمان شده، سی پاره نقره را به روسای کهنه و مشایخ رد کرده،
\par 4 گفت: «گناه کردم که خون بیگناهی را تسلیم نمودم.» گفتند: «ما را چه، خود دانی!»
\par 5 پس آن نقره را درهیکل انداخته، روانه شد و رفته خود را خفه نمود.
\par 6 اما روسای کهنه نقره را بر داشته، گفتند: «انداختن این در بیت‌المال جایز نیست زیرا خونبها است.»
\par 7 پس شورا نموده، به آن مبلغ، مزرعه کوزه‌گر را بجهت مقبره غرباءخریدند.
\par 8 از آن جهت، آن مزرعه تا امروزبحقل الدم مشهور است.
\par 9 آنگاه سخنی که به زبان ارمیای نبی گفته شده بود تمام گشت که «سی پاره نقره را برداشتند، بهای آن قیمت کرده شده‌ای که بعضی از بنی‌اسرائیل بر او قیمت گذاردند.
\par 10 و آنها را بجهت مزرعه کوزه‌گر دادند، چنانکه خداوند به من گفت.»
\par 11 اما عیسی در حضور والی ایستاده بود. پس والی از او پرسیده، گفت: «آیا تو پادشاه یهودهستی؟» عیسی بدو گفت: «تو می‌گویی!»
\par 12 وچون روسای کهنه و مشایخ از او شکایت می‌کردند، هیچ جواب نمی داد.
\par 13 پس پیلاطس وی را گفت: «نمی شنوی چقدر بر تو شهادت می‌دهند؟»
\par 14 اما در جواب وی، یک سخن هم نگفت، بقسمی که والی بسیار متعجب شد.
\par 15 و در هر عیدی، رسم والی این بود که یک زندانی، هر‌که را می‌خواستند، برای جماعت آزاد می‌کرد.
\par 16 و در آن وقت، زندانی مشهور، برابا نام داشت.
\par 17 پس چون مردم جمع شدند، پیلاطس ایشان را گفت: «که را می‌خواهید برای شما آزاد کنم؟ برابا یا عیسی مشهور به مسیح را؟»
\par 18 زیرا که دانست او را از حسد تسلیم کرده بودند.
\par 19 چون بر مسند نشسته بود، زنش نزد اوفرستاده، گفت: «با این مرد عادل تو را کاری نباشد، زیرا که امروز در خواب درباره او زحمت بسیار بردم.»
\par 20 اما روسای کهنه و مشایخ، قوم را بر این ترغیب نمودند که برابا را بخواهند و عیسی راهلاک سازند.
\par 21 پس والی بدیشان متوجه شده، گفت: «کدام‌یک از این دو نفر را می‌خواهیدبجهت شما رها کنم؟ گفتند: «برابا را.»
\par 22 پیلاطس بدیشان گفت: «پس با عیسی مشهور به مسیح چه کنم؟» جمیع گفتند: «مصلوب شود!»
\par 23 والی گفت: «چرا؟ چه بدی کرده است؟» ایشان بیشترفریاد زده، گفتند: «مصلوب شود!»
\par 24 چون پیلاطس دید که ثمری ندارد بلکه آشوب زیاده می‌گردد، آب طلبیده، پیش مردم دست خود راشسته گفت: «من بری هستم از خون این شخص عادل. شما ببینید.»
\par 25 تمام قوم در جواب گفتند: «خون او بر ما و فرزندان ما باد!»
\par 26 آنگاه برابا رابرای ایشان آزاد کرد و عیسی را تازیانه زده، سپرد تا او را مصلوب کنند.
\par 27 آنگاه سپاهیان والی، عیسی را به دیوانخانه برده، تمامی فوج را گرد وی فراهم آوردند.
\par 28 واو را عریان ساخته، لباس قرمزی بدو پوشانیدند،
\par 29 و تاجی از خار بافته، بر سرش گذاردند و نی بدست راست او دادند و پیش وی زانو زده، استهزاکنان او را می‌گفتند: «سلام‌ای پادشاه یهود!»
\par 30 و آب دهان بر وی افکنده، نی را گرفته بر سرش می‌زدند.
\par 31 و بعد از آنکه او را استهزاکرده بودند، آن لباس را از وی کنده، جامه خودش را پوشانیدند و او را بجهت مصلوب نمودن بیرون بردند.
\par 32 و چون بیرون می‌رفتند، شخصی قیروانی شمعون نام را یافته، او را بجهت بردن صلیب مجبور کردند.
\par 33 و چون به موضعی که به جلجتایعنی کاسه سر مسمی بود رسیدند،
\par 34 سرکه ممزوج به مر بجهت نوشیدن بدو دادند. اما چون چشید، نخواست که بنوشد.
\par 35 پس او را مصلوب نموده، رخت او راتقسیم نمودند و بر آنها قرعه انداختند تا آنچه به زبان نبی گفته شده بود تمام شود که «رخت مرا درمیان خود تقسیم کردند و بر لباس من قرعه انداختند.»
\par 36 و در آنجا به نگاهبانی او نشستند.
\par 37 و تقصیر نامه او را نوشته، بالای سرش آویختند که «این است عیسی، پادشاه یهود!»
\par 38 آنگاه دو دزد یکی بر دست راست و دیگری برچپش با وی مصلوب شدند.
\par 39 و راهگذاران سرهای خود را جنبانیده، کفر گویان
\par 40 می‌گفتند: «ای کسی‌که هیکل را خراب می‌کنی و در سه روزآن را می‌سازی، خود را نجات ده. اگر پسر خداهستی، از صلیب فرود بیا!»
\par 41 همچنین نیزروسای کهنه با کاتبان و مشایخ استهزاکنان می‌گفتند:
\par 42 «دیگران را نجات داد، اما نمی تواندخود را برهاند. اگر پادشاه اسرائیل است، اکنون ازصلیب فرود آید تا بدو ایمان آوریم!
\par 43 بر خداتوکل نمود، اکنون او را نجات دهد، اگر بدورغبت دارد زیرا گفت پسر خدا هستم!»
\par 44 وهمچنین آن دو دزد نیز که با وی مصلوب بودند، او را دشنام می‌دادند.
\par 45 و از ساعت ششم تا ساعت نهم، تاریکی تمام زمین را فرو گرفت.
\par 46 و نزدیک به ساعت نهم، عیسی به آواز بلند صدا زده گفت: «ایلی ایلی لما سبقتنی.» یعنی الهی الهی مرا چرا ترک کردی.
\par 47 اما بعضی از حاضرین چون این را شنیدند، گفتند که او الیاس را می‌خواند.
\par 48 در ساعت یکی از آن میان دویده، اسفنجی را گرفت و آن را پر ازسرکه کرده، بر سر نی گذارد و نزد او داشت تابنوشد.
\par 49 و دیگران گفتند: «بگذار تا ببینیم که آیاالیاس می‌آید او را برهاند.»
\par 50 عیسی باز به آوازبلند صیحه زده، روح را تسلیم نمود.
\par 51 که ناگاه پرده هیکل از سر تا پا دو پاره شد و زمین متزلزل وسنگها شکافته گردید،
\par 52 و قبرها گشاده شد وبسیاری از بدنهای مقدسین که آرامیده بودندبرخاستند،
\par 53 و بعد از برخاستن وی، از قبوربرآمده، به شهر مقدس رفتند و بر بسیاری ظاهرشدند.
\par 54 اما یوزباشی و رفقایش که عیسی رانگاهبانی می‌کردند، چون زلزله و این وقایع رادیدند، بینهایت ترسان شده، گفتند: «فی الواقع این شخص پسر خدا بود.»
\par 55 و در آنجا زنان بسیاری که از جلیل در عقب عیسی آمده بودند تا او راخدمت کنند، از دور نظاره می‌کردند،
\par 56 که از آن جمله، مریم مجدلیه بود و مریم مادر یعقوب ویوشاء و مادر پسران زبدی.
\par 57 اما چون وقت عصر رسید، شخصی دولتمند از اهل رامه، یوسف نام که او نیز ازشاگردان عیسی بود آمد،
\par 58 و نزد پیلاطس رفته، جسد عیسی را خواست. آنگاه پیلاطس فرمان داد که داده شود.
\par 59 پس یوسف جسد را برداشته، آن را در کتان پاک پیچیده،
\par 60 او را در قبری نو که برای خود از سنگ تراشیده بود، گذارد و سنگی بزرگ بر سر آن غلطانیده، برفت.
\par 61 و مریم مجدلیه و مریم دیگر در آنجا، در مقابل قبرنشسته بودند.
\par 62 و در فردای آن روز که بعد از روز تهیه بودروسای کهنه و فریسیان نزد پیلاطس جمع شده،
\par 63 گفتند: «ای آقا ما را یاد است که آن گمراه‌کننده وقتی که زنده بود گفت: "بعد از سه روزبرمی خیزم."
\par 64 پس بفرما قبر را تا سه روزنگاهبانی کنند مبادا شاگردانش در شب آمده، اورا بدزدند و به مردم گویند که از مردگان برخاسته است و گمراهی آخر، از اول بدتر شود.»
\par 65 پیلاطس بدیشان فرمود: «شما کشیکچیان دارید. بروید چنانکه دانید، محافظت کنید.»پس رفتند و سنگ را مختوم ساخته، قبر را باکشیکچیان محافظت نمودند.
\par 66 پس رفتند و سنگ را مختوم ساخته، قبر را باکشیکچیان محافظت نمودند.

\chapter{28}

\par 1 و بعد از سبت، هنگام فجر، روز اول هفته، مریم مجدلیه و مریم دیگربجهت دیدن قبر‌آمدند.
\par 2 که ناگاه زلزله‌ای عظیم حادث شد از آنرو که فرشته خداوند از آسمان نزول کرده، آمد و سنگ را از در قبر غلطانیده، برآن بنشست.
\par 3 و صورت او مثل برق و لباسش چون برف سفید بود.
\par 4 و از ترس او کشیکچیان به لرزه درآمده، مثل مرده گردیدند.
\par 5 اما فرشته به زنان متوجه شده، گفت: شما ترسان مباشید!
\par 6 دراینجا نیست زیرا چنانکه گفته بود برخاسته است. بیایید جایی که خداوند خفته بود ملاحظه کنید،
\par 7 و به زودی رفته شاگردانش را خبر دهید که ازمردگان برخاسته است. اینک پیش از شما به جلیل می‌رود. در آنجا او را خواهید دید. اینک شما را گفتم.»
\par 8 پس، از قبر با ترس و خوشی عظیم به زودی روانه شده، رفتند تا شاگردان او را اطلاع دهند.
\par 9 ودر هنگامی که بجهت اخبار شاگردان او می‌رفتند، ناگاه عیسی بدیشان برخورده، گفت: «سلام برشما باد!» پس پیش آمده، به قدمهای اوچسبیده، او را پرستش کردند.
\par 10 آنگاه عیسی بدیشان گفت: «مترسید! رفته، برادرانم رابگویید که به جلیل بروند که در آنجا مرا خواهنددید.»
\par 11 و چون ایشان می‌رفتند، ناگاه بعضی ازکشیکچیان به شهر شده، روسای کهنه را از همه این وقایع مطلع ساختند.
\par 12 ایشان با مشایخ جمع شده، شورا نمودند و نقره بسیار به سپاهیان داده،
\par 13 گفتند: «بگویید که شبانگاه شاگردانش آمده، وقتی که ما در خواب بودیم او را دزدیدند.
\par 14 وهرگاه این سخن گوش زد والی شود، همانا ما او رابرگردانیم و شما را مطمئن سازیم.»
\par 15 ایشان پول را گرفته، چنانکه تعلیم یافتند کردند و این سخن تاامروز در میان یهود منتشر است.
\par 16 اما یازده رسول به جلیل، بر کوهی که عیسی ایشان را نشان داده بود رفتند.
\par 17 و چون او را دیدند، پرستش نمودند. لیکن بعضی شک کردند.
\par 18 پس عیسی پیش آمده، بدیشان خطاب کرده، گفت: «تمامی قدرت در آسمان و بر زمین به من داده شده است.پس رفته، همه امت‌ها را شاگرد سازید و ایشان رابه اسم اب و ابن و روح‌القدس تعمید دهید.
\par 19 پس رفته، همه امت‌ها را شاگرد سازید و ایشان رابه اسم اب و ابن و روح‌القدس تعمید دهید.



\end{document}