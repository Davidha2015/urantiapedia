\begin{document}

\title{John}


\chapter{1}

\par 1 در ابتدا کلمه بود و کلمه نزد خدا بود وکلمه خدا بود.
\par 2 همان در ابتدا نزد خدا بود.
\par 3 همه‌چیز به واسطه او آفریده شد و به غیر از اوچیزی از موجودات وجود نیافت.
\par 4 در او حیات بود و حیات نور انسان بود.
\par 5 و نور در تاریکی می‌درخشد و تاریکی آن را درنیافت.
\par 6 شخصی از جانب خدا فرستاده شد که اسمش یحیی بود؛
\par 7 او برای شهادت آمد تا بر نورشهادت دهد تا همه به وسیله او ایمان آورند.
\par 8 اوآن نور نبود بلکه آمد تا بر نور شهادت دهد.
\par 9 آن نور حقیقی بود که هر انسان را منور می‌گرداند ودر جهان آمدنی بود.
\par 10 او در جهان بود و جهان به واسطه او آفریده شد و جهان او را نشناخت.
\par 11 به نزد خاصان خود آمد و خاصانش او را نپذیرفتند؛
\par 12 و اما به آن کسانی که او را قبول کردند قدرت داد تا فرزندان خدا گردند، یعنی به هر‌که به اسم اوایمان آورد،
\par 13 که نه از خون و نه از خواهش جسد و نه از خواهش مردم، بلکه از خدا تولدیافتند.
\par 14 و کلمه جسم گردید و میان ما ساکن شد، پراز فیض و راستی و جلال او را دیدیم، جلالی شایسته پسر یگانه پدر.
\par 15 و یحیی بر او شهادت داد و ندا کرده، می‌گفت: «این است آنکه درباره اوگفتم آنکه بعد از من می‌آید، پیش از من شده است زیرا که بر من مقدم بود.»
\par 16 و از پری او جمیع ما بهره یافتیم و فیض به عوض فیض،
\par 17 زیراشریعت به وسیله موسی عطا شد، اما فیض وراستی به وسیله عیسی مسیح رسید.
\par 18 خدا راهرگز کسی ندیده است؛ پسر یگانه‌ای که درآغوش پدر است، همان او را ظاهر کرد.
\par 19 و این است شهادت یحیی در وقتی که یهودیان از اورشلیم کاهنان و لاویان را فرستادندتا از او سوال کنند که تو کیستی،
\par 20 که معترف شدو انکار ننمود، بلکه اقرار کرد که من مسیح نیستم.
\par 21 آنگاه از او سوال کردند: «پس چه؟ آیا توالیاس هستی؟ گفت: «نیستم.»
\par 22 آنگاه بدو گفتند: «پس کیستی تا به آن کسانی که ما را فرستادند جواب بریم؟ درباره خود چه می‌گویی؟»
\par 23 گفت: «من صدای ندا کننده‌ای دربیابانم که راه خداوند را راست کنید، چنانکه اشعیانبی گفت.»
\par 24 و فرستادگان از فریسیان بودند.
\par 25 پس از اوسوال کرده، گفتند: «اگر تو مسیح و الیاس و آن نبی نیستی، پس برای چه تعمید می‌دهی؟»
\par 26 یحیی در جواب ایشان گفت: «من به آب تعمیدمی دهم و در میان شما کسی ایستاده است که شمااو را نمی شناسید.
\par 27 و او آن است که بعد از من می آید، اما پیش از من شده است، که من لایق آن نیستم که بند نعلینش را باز کنم.»
\par 28 و این دربیت عبره که آن طرف اردن است، در جایی که یحیی تعمید می‌داد واقع گشت.
\par 29 و در فردای آن روز یحیی عیسی را دید که به‌جانب او می‌آید. پس گفت: «اینک بره خدا که گناه جهان را برمی دارد!
\par 30 این است آنکه من درباره او گفتم که مردی بعد از من می‌آید که پیش از من شده است زیرا که بر من مقدم بود.
\par 31 و من اورا نشناختم، لیکن تا او به اسرائیل ظاهر گردد، برای همین من آمده به آب تعمید می‌دادم.»
\par 32 پس یحیی شهادت داده، گفت: «روح را دیدم که مثل کبوتری از آسمان نازل شده، بر او قرارگرفت.
\par 33 و من او را نشناختم، لیکن او که مرافرستاد تا به آب تعمید دهم، همان به من گفت برهر کس بینی که روح نازل شده، بر او قرار گرفت، همان است او که به روح‌القدس تعمید می‌دهد.
\par 34 و من دیده شهادت می‌دهم که این است پسرخدا.»
\par 35 و در روز بعد نیز یحیی با دو نفر از شاگردان خود ایستاده بود.
\par 36 ناگاه عیسی را دید که راه می‌رود؛ و گفت: «اینک بره خدا.»
\par 37 و چون آن دو شاگرد کلام او را شنیدند، از پی عیسی روانه شدند.
\par 38 پس عیسی روی گردانیده، آن دو نفر رادید که از عقب می‌آیند. بدیشان گفت:
\par 39 «چه می‌خواهید؟» بدو گفتند: «ربی (یعنی‌ای معلم )در کجا منزل می‌نمایی؟»
\par 40 بدیشان گفت: «بیایید و ببینید.» آنگاه آمده، دیدند که کجا منزل دارد، و آن روز را نزد او بماندند و قریب به ساعت دهم بود.
\par 41 و یکی از آن دو که سخن یحیی را شنیده، پیروی او نمودند، اندریاس برادر شمعون پطرس بود.
\par 42 او اول برادر خود شمعون را یافته، به اوگفت: «مسیح را (که ترجمه آن کرستس است )یافتیم.» و چون او را نزد عیسی آورد، عیسی بدونگریسته، گفت: «تو شمعون پسر یونا هستی؛ واکنون کیفا خوانده خواهی شد (که ترجمه آن پطرس است ).»
\par 43 بامدادان چون عیسی خواست به سوی جلیل روانه شود، فیلپس را یافته، بدو گفت: «ازعقب من بیا.»
\par 44 و فیلپس از بیت صیدا از شهراندریاس وپطرس بود.
\par 45 فیلیپس نتنائیل را یافته، بدو گفت: «آن کسی را که موسی در تورات و انبیامذکور داشته‌اند، یافته‌ایم که عیسی پسر یوسف ناصری است.»
\par 46 نتنائیل بدو گفت: «مگرمی شود که از ناصره چیزی خوب پیدا شود؟» فیلپس بدو گفت: «بیا و ببین.»
\par 47 و عیسی چون دید که نتنائیل به سوی او می‌آید، درباره اوگفت: «اینک اسرائیلی حقیقی که در او مکری نیست.»
\par 48 نتنائیل بدو گفت: «مرا از کجامی شناسی؟» عیسی در جواب وی گفت: «قبل ازآنکه فیلپس تو را دعوت کند، در حینی که زیردرخت انجیر بودی تو را دیدم.»
\par 49 نتنائیل درجواب او گفت: «ای استاد تو پسر خدایی! توپادشاه اسرائیل هستی!»
\par 50 عیسی در جواب اوگفت: «آیا از اینکه به تو گفتم که تو را زیر درخت انجیر دیدم، ایمان آوردی؟ بعد از این چیزهای بزرگتر از این خواهی دید.»پس بدو گفت: «آمین آمین به شما می‌گویم که از کنون آسمان را گشاده، و فرشتگان خدا را که بر پسر انسان صعودو نزول می‌کنند خواهید دید.»
\par 51 پس بدو گفت: «آمین آمین به شما می‌گویم که از کنون آسمان را گشاده، و فرشتگان خدا را که بر پسر انسان صعودو نزول می‌کنند خواهید دید.»

\chapter{2}

\par 1 و در روز سوم، در قانای جلیل عروسی بودو مادر عیسی در آنجا بود.
\par 2 و عیسی وشاگردانش را نیز به عروسی دعوت کردند.
\par 3 وچون شراب تمام شد، مادر عیسی بدو گفت: «شراب ندارند.»
\par 4 عیسی به وی گفت: «ای زن مرابا تو چه‌کار است؟ ساعت من هنوز نرسیده است.»
\par 5 مادرش به نوکران گفت: «هر‌چه به شماگوید بکنید.»
\par 6 و در آنجا شش قدح سنگی برحسب تطهیریهود نهاده بودند که هر یک گنجایش دو یا سه کیل داشت.
\par 7 عیسی بدیشان گفت: «قدحها را ازآب پر کنید.» و آنها را لبریز کردند.
\par 8 پس بدیشان گفت: «الان بردارید و به نزد رئیس مجلس ببرید.» پس بردند؛
\par 9 و چون رئیس مجلس آن آب را که شراب گردیده بود، بچشید و ندانست که از کجااست، لیکن نوکرانی که آب را کشیده بودند، می‌دانستند، رئیس مجلس داماد را مخاطب ساخته، بدو گفت:
\par 10 «هرکسی شراب خوب رااول می‌آورد و چون مست شدند، بدتر از آن. لیکن تو شراب خوب را تا حال نگاه داشتی؟»
\par 11 و این ابتدای معجزاتی است که از عیسی درقانای جلیل صادر گشت و جلال خود را ظاهرکرد و شاگردانش به او ایمان آوردند.
\par 12 و بعد ازآن او با مادر و برادران و شاگردان خود به کفرناحوم آمد و در آنجا ایامی کم ماندند.
\par 13 و چون عید فصح نزدیک بود، عیسی به اورشلیم رفت،
\par 14 و در هیکل، فروشندگان گاو وگوسفند و کبوتر و صرافان را نشسته یافت.
\par 15 پس تازیانه‌ای از ریسمان ساخته، همه را از هیکل بیرون نمود، هم گوسفندان و گاوان را، و نقودصرافان را ریخت و تختهای ایشان را واژگون ساخت،
\par 16 و به کبوترفروشان گفت: «اینها را ازاینجا بیرون برید و خانه پدر مرا خانه تجارت مسازید.»
\par 17 آنگاه شاگردان او را یاد آمد که مکتوب است: «غیرت خانه تو مرا خورده است.»
\par 18 پس یهودیان روی به او آورده، گفتند: «به ما چه علامت می‌نمایی که این کارها را می‌کنی؟»
\par 19 عیسی در جواب ایشان گفت: «این قدس راخراب کنید که در سه روز آن را برپا خواهم نمود.»
\par 20 آنگاه یهودیان گفتند: «در عرصه چهل و شش سال این قدس را بنا نموده‌اند؛ آیا تو درسه روز آن را برپا می‌کنی؟»
\par 21 لیکن او درباره قدس جسد خود سخن می‌گفت.
\par 22 پس وقتی که از مردگان برخاست شاگردانش را به‌خاطر آمد که این را بدیشان گفته بود. آنگاه به کتاب و به کلامی که عیسی گفته بود، ایمان آوردند.
\par 23 و هنگامی که در عید فصح در اورشلیم بودبسیاری چون معجزاتی را که از او صادر می‌گشت دیدند، به اسم او ایمان آوردند.
\par 24 لیکن عیسی خویشتن را بدیشان موتمن نساخت، زیرا که اوهمه را می‌شناخت.و از آنجا که احتیاج نداشت که کسی درباره انسان شهادت دهد، زیراخود آنچه در انسان بود می‌دانست.
\par 25 و از آنجا که احتیاج نداشت که کسی درباره انسان شهادت دهد، زیراخود آنچه در انسان بود می‌دانست.

\chapter{3}

\par 1 و شخصی از فریسیان نیقودیموس نام ازروسای یهود بود.
\par 2 او در شب نزد عیسی آمده، به وی گفت: «ای استاد می‌دانیم که تو معلم هستی که از جانب خدا آمده‌ای زیرا هیچ‌کس نمی تواند معجزاتی را که تو می‌نمایی بنماید، جزاینکه خدا با وی باشد.»
\par 3 عیسی در جواب اوگفت: «آمین آمین به تو می‌گویم اگر کسی از سر نومولود نشود، ملکوت خدا را نمی تواند دید.»
\par 4 نیقودیموس بدو گفت: «چگونه ممکن است که انسانی که پیر شده باشد، مولود گردد؟ آیا می‌شودکه بار دیگر داخل شکم مادر گشته، مولود شود؟»
\par 5 عیسی در جواب گفت: «آمین، آمین به تومی گویم اگر کسی از آب و روح مولود نگردد، ممکن نیست که داخل ملکوت خدا شود.
\par 6 آنچه از جسم مولود شد، جسم است و آنچه از روح مولود گشت روح است.
\par 7 عجب مدار که به توگفتم باید شما از سر نو مولود گردید.
\par 8 باد هرجاکه می‌خواهد می‌وزد و صدای آن را می‌شنوی لیکن نمی دانی از کجا می‌آید و به کجا می‌رود. همچنین است هر‌که از روح مولود گردد.»
\par 9 نیقودیموس در جواب وی گفت: «چگونه ممکن است که چنین شود؟»
\par 10 عیسی در جواب وی گفت: «آیا تو معلم اسرائیل هستی و این رانمی دانی؟
\par 11 آمین، آمین به تو می‌گویم آنچه می‌دانیم، می‌گوییم و به آنچه دیده‌ایم، شهادت می‌دهیم و شهادت ما را قبول نمی کنید.
\par 12 چون شما را از امور زمینی سخن گفتم، باور نکردید. پس هرگاه به امور آسمانی با شما سخن رانم چگونه تصدیق خواهید نمود؟
\par 13 و کسی به آسمان بالا نرفت مگر آن کس که از آسمان پایین آمد یعنی پسر انسان که در آسمان است.
\par 14 وهمچنان‌که موسی مار را در بیابان بلند نمود، همچنین پسر انسان نیز باید بلند کرده شود،
\par 15 تاهر‌که به او ایمان آرد هلاک نگردد، بلکه حیات جاودانی یابد.
\par 16 زیرا خدا جهان را اینقدر محبت نمود که پسر یگانه خود را داد تا هر‌که بر او ایمان آورد هلاک نگردد بلکه حیات جاودانی یابد.
\par 17 زیرا خدا پسر خود را در جهان نفرستاد تا برجهان داوری کند، بلکه تا به وسیله او جهان نجات یابد.
\par 18 آنکه به او ایمان آرد، بر او حکم نشود؛ اما هر‌که ایمان نیاورد الان بر او حکم شده است، بجهت آنکه به اسم پسر یگانه خدا ایمان نیاورده.
\par 19 و حکم این است که نور در جهان آمدو مردم ظلمت را بیشتر از نور دوست داشتند، ازآنجا که اعمال ایشان بد است.
\par 20 زیرا هر‌که عمل بد می‌کند، روشنی را دشمن دارد و پیش روشنی نمی آید، مبادا اعمال او توبیخ شود.
\par 21 و لیکن کسی‌که به راستی عمل می‌کند پیش روشنی می‌آید تا آنکه اعمال او هویدا گردد که در خداکرده شده است.»
\par 22 و بعد از آن عیسی با شاگردان خود به زمین یهودیه آمد و با ایشان در آنجا به‌سر برده، تعمیدمی داد.
\par 23 و یحیی نیز در عینون، نزدیک سالیم تعمید می‌داد زیرا که در آنجا آب بسیار بود ومردم می‌آمدند و تعمید می‌گرفتند،
\par 24 چونکه یحیی هنوز در زندان حبس نشده بود.
\par 25 آنگاه در خصوص تطهیر، در میان شاگردان یحیی ویهودیان مباحثه شد.
\par 26 پس به نزد یحیی آمده، به او گفتند: «ای استاد، آن شخصی که با تو درآنطرف اردن بود و تو برای او شهادت دادی، اکنون او تعمید می‌دهد و همه نزد اومی آیند.»
\par 27 یحیی در جواب گفت: «هیچ‌کس چیزی نمی تواند یافت، مگر آنکه از آسمان بدو داده شود.
\par 28 شما خود بر من شاهد هستید که گفتم من مسیح نیستم بلکه پیش روی او فرستاده شدم.
\par 29 کسی‌که عروس دارد داماد است، اما دوست داماد که ایستاده آواز او را می‌شنود، از آواز دامادبسیار خشنود می‌گردد. پس این خوشی من کامل گردید.
\par 30 می‌باید که او افزوده شود و من ناقص گردم.
\par 31 او که از بالا می‌آید، بالای همه است وآنکه از زمین است زمینی است و از زمین تکلم می‌کند؛ اما او که از آسمان می‌آید، بالای همه است.
\par 32 و آنچه را دید و شنید، به آن شهادت می‌دهد و هیچ‌کس شهادت او را قبول نمی کند.
\par 33 و کسی‌که شهادت اورا قبول کرد، مهر کرده است بر اینکه خدا راست است.
\par 34 زیرا آن کسی را که خدا فرستاد، به کلام خدا تکلم می‌نماید، چونکه خدا روح را به میزان عطا نمی کند.
\par 35 پدرپسر را محبت می‌نماید و همه‌چیز را بدست اوسپرده است.آنکه به پسر ایمان آورده باشد، حیات جاودانی دارد و آنکه به پسر ایمان نیاوردحیات را نخواهد دید، بلکه غضب خدا بر اومی ماند.»
\par 36 آنکه به پسر ایمان آورده باشد، حیات جاودانی دارد و آنکه به پسر ایمان نیاوردحیات را نخواهد دید، بلکه غضب خدا بر اومی ماند.»

\chapter{4}

\par 1 و چون خداوند دانست که فریسیان مطلع شده‌اند که عیسی بیشتر از یحیی شاگردپیدا کرده، تعمید می‌دهد،
\par 2 با اینکه خود عیسی تعمید نمی داد بلکه شاگردانش،
\par 3 یهودیه راگذارده، باز به‌جانب جلیل رفت.
\par 4 و لازم بود که از سامره عبور کند
\par 5 پس به شهری از سامره که سوخار نام داشت، نزدیک به آن موضعی که یعقوب به پسر خود یوسف داده بود رسید.
\par 6 و در آنجا چاه یعقوب بود. پس عیسی از سفر خسته شده، همچنین بر سر چاه نشسته بود و قریب به ساعت ششم بود.
\par 7 که زنی سامری بجهت آب کشیدن آمد. عیسی بدو گفت: «جرعه‌ای آب به من بنوشان.»
\par 8 زیرا شاگردانش بجهت خریدن خوراک به شهر رفته بودند.
\par 9 زن سامری بدو گفت: «چگونه تو که یهود هستی ازمن آب می‌خواهی و حال آنکه زن سامری می‌باشم؟» زیرا که یهود با سامریان معاشرت ندارند.
\par 10 عیسی در جواب او گفت: «اگر بخشش خدا را می‌دانستی و کیست که به تو می‌گوید آب به من بده، هرآینه تو از او خواهش می‌کردی و به تو آب زنده عطا می‌کرد.
\par 11 زن بدو گفت: «ای آقادلو نداری و چاه عمیق است. پس از کجا آب زنده داری؟
\par 12 آیا تو از پدر ما یعقوب بزرگترهستی که چاه را به ما داد و خود و پسران و مواشی او از آن می‌آشامیدند؟»
\par 13 عیسی در جواب اوگفت: «هر‌که از این آب بنوشد باز تشنه گردد،
\par 14 لیکن کسی‌که از آبی که من به او می‌دهم بنوشد، ابد تشنه نخواهد شد، بلکه آن آبی که به او می‌دهم در او چشمه آبی گردد که تا حیات جاودانی می‌جوشد.»
\par 15 زن بدو گفت: «ای آقا آن آب را به من بده تا دیگر تشنه نگردم و به اینجابجهت آب کشیدن نیایم.»
\par 16 عیسی به او گفت: «برو و شوهر خود رابخوان و در اینجا بیا.»
\par 17 زن در جواب گفت: «شوهر ندارم.» عیسی بدو گفت: «نیکو گفتی که شوهر نداری!
\par 18 زیرا که پنج شوهر داشتی وآنکه الان داری شوهر تو نیست! این سخن راراست گفتی!»
\par 19 زن بدو گفت: «ای آقا می‌بینم که تو نبی هستی!
\par 20 پدران ما در این کوه پرستش می‌کردند و شما می‌گویید که در اورشلیم جایی است که در آن عبادت باید نمود.»
\par 21 عیسی بدوگفت: «ای زن مرا تصدیق کن که ساعتی می‌آید که نه در این کوه و نه در اورشلیم پدر را پرستش خواهید کرد.
\par 22 شما آنچه را که نمی دانیدمی پرستید اما ما آنچه را که می‌دانیم عبادت می‌کنیم زیرا نجات از یهود است.
\par 23 لیکن ساعتی می‌آید بلکه الان است که در آن پرستندگان حقیقی پدر را به روح و راستی پرستش خواهندکرد زیرا که پدر مثل این پرستندگان خود را طالب است.
\par 24 خدا روح است و هر‌که او را پرستش کند می‌باید به روح و راستی بپرستد.»
\par 25 زن بدو گفت: «می‌دانم که مسیح یعنی کرستس می‌آید. پس هنگامی که او آید از هر چیزبه ما خبر‌خواهد داد.»
\par 26 عیسی بدو گفت: «من که با تو سخن می‌گویم همانم.»
\par 27 و در همان وقت شاگردانش آمده، تعجب کردند که با زنی سخن می‌گوید و لکن هیچ‌کس نگفت که چه می‌طلبی یا برای چه با او حرف می‌زنی.
\par 28 آنگاه زن سبوی خود را گذارده، به شهر رفت و مردم را گفت:
\par 29 «بیایید و کسی راببینید که هرآنچه کرده بودم به من گفت. آیا این مسیح نیست؟»
\par 30 پس از شهر بیرون شده، نزد اومی آمدند.
\par 31 و در اثنا آن شاگردان او خواهش نموده، گفتند: «ای استاد بخور.»
\par 32 بدیشان گفت: «من غذایی دارم که بخورم و شما آن را نمی دانید.»
\par 33 شاگردان به یکدیگر گفتند: «مگر کسی برای اوخوراکی آورده باشد!»
\par 34 عیسی بدیشان گفت: «خوراک من آن است که خواهش فرستنده خودرا به عمل آورم و کار او را به انجام رسانم.
\par 35 آیاشما نمی گویید که چهار ماه دیگر موسم درواست؟ اینک به شما می‌گویم چشمان خود را بالاافکنید و مزرعه‌ها را ببینید زیرا که الان بجهت درو سفید شده است.
\par 36 و دروگر اجرت می‌گیردو ثمری بجهت حیات جاودانی جمع می‌کند تاکارنده و درو‌کننده هر دو با هم خشنود گردند.
\par 37 زیرا این کلام در اینجا راست است که یکی می‌کارد و دیگری درو می‌کند.
\par 38 من شما رافرستادم تا چیزی را که در آن رنج نبرده‌اید دروکنید. دیگران محنت کشیدند و شما در محنت ایشان داخل شده‌اید.»
\par 39 پس در آن شهر بسیاری از سامریان بواسطه سخن آن زن که شهادت داد که هر‌آنچه کرده بودم به من باز‌گفت بدو ایمان آوردند.
\par 40 و چون سامریان نزد او آمدند، از او خواهش کردند که نزدایشان بماند و دو روز در آنجا بماند.
\par 41 و بسیاری دیگر بواسطه کلام او ایمان آوردند.
\par 42 و به زن گفتند که «بعد از این بواسطه سخن تو ایمان نمی آوریم زیرا خود شنیده و دانسته‌ایم که او درحقیقت مسیح و نجات‌دهنده عالم است.»
\par 43 اما بعد از دو روز از آنجا بیرون آمده، به سوی جلیل روانه شد.
\par 44 زیرا خود عیسی شهادت داد که هیچ نبی را در وطن خود حرمت نیست.
\par 45 پس چون به جلیل آمد، جلیلیان او راپذیرفتند زیرا هر‌چه در اورشلیم در عید کرده بود، دیدند، چونکه ایشان نیز در عید رفته بودند.
\par 46 پس عیسی به قانای جلیل آنجایی که آب را شراب ساخته بود، بازآمد. و یکی از سرهنگان ملک بود که پسر او در کفرناحوم مریض بود.
\par 47 وچون شنید که عیسی از یهودیه به جلیل آمده است، نزد او آمده، خواهش کرد که فرود بیاید وپسر او را شفا دهد، زیرا که مشرف به موت بود.
\par 48 عیسی بدو گفت: «اگر آیات و معجزات نبینید، همانا ایمان نیاورید.»
\par 49 سرهنگ بدو گفت: «ای آقا قبل از آنکه پسرم بمیرد فرود بیا.»
\par 50 عیسی بدو گفت: «برو که پسرت زنده است.» آن شخص به سخنی که عیسی بدو گفت، ایمان آورده، روانه شد.
\par 51 و در وقتی که او می‌رفت، غلامانش او راستقبال نموده، مژده دادند و گفتند که پسر توزنده است.
\par 52 پس از ایشان پرسید که در چه ساعت عافیت یافت. گفتند: «دیروز، در ساعت هفتم تب از او زایل گشت.»
\par 53 آنگاه پدر فهمیدکه در همان ساعت عیسی گفته بود: «پسر تو زنده است.» پس او و تمام اهل خانه او ایمان آوردند.و این نیز معجزه دوم بود که از عیسی در وقتی که از یهودیه به جلیل آمد، به ظهور رسید.
\par 54 و این نیز معجزه دوم بود که از عیسی در وقتی که از یهودیه به جلیل آمد، به ظهور رسید.

\chapter{5}

\par 1 و بعد از آن یهود را عیدی بود و عیسی به اورشلیم آمد.
\par 2 و در اورشلیم نزدباب الضان حوضی است که آن را به عبرانی بیت حسدا می‌گویند که پنج رواق دارد.
\par 3 و درآنجا جمعی کثیر از مریضان و کوران و لنگان وشلان خوابیده، منتظر حرکت آب می‌بودند.
\par 4 و در آنجا مردی بود که سی و هشت سال به مرضی مبتلا بود.
\par 5 چون عیسی او را خوابیده دیدو دانست که مرض او طول کشیده است، بدوگفت: «آیا می‌خواهی شفا یابی؟»
\par 6 مریض او راجواب داد که «ای آقا کسی ندارم که چون آب به حرکت آید، مرا در حوض بیندازد، بلکه تا وقتی که می‌آیم، دیگری پیش از من فرو رفته است.
\par 7 عیسی بدو گفت: «برخیز و بستر خود رابرداشته، روانه شو!»
\par 8 که در حال آن، مرد شفایافت و بستر خود را برداشته، روانه گردید. و آن روز سبت بود.
\par 9 پس یهودیان به آن کسی‌که شفا یافته بود، گفتند: «روز سبت است و بر تو روا نیست که بسترخود را برداری.»
\par 10 او در جواب ایشان گفت: «آن کسی‌که مرا شفا داد، همان به من گفت بستر خود را بردار و برو.»
\par 11 پس از او پرسیدند: «کیست آنکه به تو گفت، بستر خود را بردار وبرو؟»
\par 12 لیکن آن شفا یافته نمی دانست که بود، زیرا که عیسی ناپدید شد چون در آنجا ازدحامی بود.
\par 13 و بعد از آن، عیسی او را در هیکل یافته بدو گفت: «اکنون شفا یافته‌ای. دیگر خطا مکن تابرای تو بدتر نگردد.»
\par 14 آن مرد رفت و یهودیان را خبر داد که «آنکه مرا شفا داد، عیسی است.»
\par 15 و از این سبب یهودیان بر عیسی تعدی می‌کردند، زیرا که این کار را در روز سبت کرده بود.
\par 16 عیسی در جواب ایشان گفت که «پدر من تاکنون کار می‌کند و من نیز کار می‌کنم.»
\par 17 پس ازاین سبب، یهودیان بیشتر قصد قتل او کردند زیراکه نه‌تنها سبت را می‌شکست بلکه خدا را نیز پدرخود گفته، خود را مساوی خدا می‌ساخت.
\par 18 آنگاه عیسی در جواب ایشان گفت: «آمین آمین به شما می‌گویم که پسر از خود هیچ نمی تواند کرد مگر آنچه بیند که پدر به عمل آرد، زیرا که آنچه او می‌کند، همچنین پسر نیز می‌کند.
\par 19 زیرا که پدر پسر را دوست می‌دارد و هرآنچه خود می‌کند بدو می‌نماید و اعمال بزرگتر از این بدو نشان خواهد داد تا شما تعجب نمایید.
\par 20 زیرا همچنان‌که پدر مردگان را برمی خیزاند وزنده می‌کند، همچنین پسر نیز هر‌که رامی خواهد زنده می‌کند.
\par 21 زیرا که پدر برهیچ‌کس داوری نمی کند بلکه تمام داوری را به پسر سپرده است.
\par 22 تا آنکه همه پسر را حرمت بدارند، همچنان‌که پدر را حرمت می‌دارند؛ وکسی‌که به پسر حرمت نکند، به پدری که او رافرستاد احترام نکرده است.
\par 23 آمین آمین به شمامی گویم هر‌که کلام مرا بشنود و به فرستنده من ایمان آورد، حیات جاودانی دارد و در داوری نمی آید، بلکه از موت تا به حیات منتقل گشته است.
\par 24 آمین آمین به شما می‌گویم که ساعتی می‌آید بلکه اکنون است که مردگان آواز پسر خدارا می‌شنوند و هر‌که بشنود زنده گردد.
\par 25 زیراهمچنان‌که پدر در خود حیات دارد، همچنین پسر را نیز عطا کرده است که در خود حیات داشته باشد.
\par 26 و بدو قدرت بخشیده است که داوری هم بکند زیرا که پسر انسان است.
\par 27 و از این تعجب مکنید زیرا ساعتی می‌آید که در آن جمیع کسانی که در قبور می‌باشند، آواز او را خواهندشنید،
\par 28 و بیرون خواهند آمد؛ هر‌که اعمال نیکوکرد، برای قیامت حیات و هر‌که اعمال بد کرد، بجهت قیامت داوری.
\par 29 من از خود هیچ نمی توانم کرد بلکه چنانکه شنیده‌ام داوری می‌کنم و داوری من عادل است زیرا که اراده خود را طالب نیستم بلکه اراده پدری که مرا فرستاده است.
\par 30 اگر من بر خود شهادت دهم شهادت من راست نیست.
\par 31 دیگری هست که بر من شهادت می‌دهد ومی دانم که شهادتی که او بر من می‌دهد راست است.
\par 32 شما نزد یحیی فرستادید و او به راستی شهادت داد.
\par 33 اما من شهادت انسان را قبول نمی کنم ولیکن این سخنان را می‌گویم تا شمانجات یابید.
\par 34 او چراغ افروخته و درخشنده‌ای بود و شما خواستید که ساعتی به نور او شادی کنید.
\par 35 و اما من شهادت بزرگتر از یحیی دارم زیرا آن کارهایی که پدر به من عطا کرد تا کامل کنم، یعنی‌این کارهایی که من می‌کنم، بر من شهادت می‌دهد که پدر مرا فرستاده است.
\par 36 وخود پدر که مرا فرستاد، به من شهادت داده است که هرگز آواز او را نشنیده و صورت او راندیده‌اید،
\par 37 و کلام او را در خود ثابت نداریدزیرا کسی را که پدر فرستاد شما بدو ایمان نیاوردید.
\par 38 کتب را تفتیش کنید زیرا شما گمان می‌بریدکه در آنها حیات جاودانی دارید و آنها است که به من شهادت می‌دهد.
\par 39 و نمی خواهید نزد من آیید تا حیات یابید.
\par 40 جلال را از مردم نمی پذیرم.
\par 41 ولکن شما را می‌شناسم که در نفس خود محبت خدا را ندارید.
\par 42 من به اسم پدرخود آمده‌ام و مرا قبول نمی کنید، ولی هرگاه دیگری به اسم خود آید، او را قبول خواهید کرد.
\par 43 شما چگونه می‌توانید ایمان آرید و حال آنکه جلال از یکدیگر می‌طلبید و جلالی را که ازخدای واحد است طالب نیستید؟
\par 44 گمان مبریدکه من نزد پدر بر شما ادعا خواهم کرد. کسی هست که مدعی شما می‌باشد و آن موسی است که بر او امیدوار هستید.
\par 45 زیرا اگر موسی راتصدیق می‌کردید، مرا نیز تصدیق می‌کردیدچونکه او درباره من نوشته است.اما چون نوشته های او را تصدیق نمی کنید، پس چگونه سخنهای مرا قبول خواهید کرد.»
\par 46 اما چون نوشته های او را تصدیق نمی کنید، پس چگونه سخنهای مرا قبول خواهید کرد.»

\chapter{6}

\par 1 و بعد از آن عیسی به آن طرف دریای جلیل که دریای طبریه باشد، رفت.
\par 2 و جمعی کثیر از عقب او آمدند زیرا آن معجزاتی را که به مریضان می‌نمود، می‌دیدند.
\par 3 آنگاه عیسی به کوهی برآمده، با شاگردان خود در آنجا بنشست.
\par 4 و فصح که عید یهود باشد، نزدیک بود.
\par 5 پس عیسی چشمان خود را بالا انداخته، دیدکه جمعی کثیر به طرف او می‌آیند. به فیلپس گفت: «از کجا نان بخریم تا اینها بخورند؟»
\par 6 و این را از روی امتحان به او گفت، زیرا خود می‌دانست چه باید کرد.
\par 7 فیلپس او را جواب داد که «دویست دینار نان، اینها را کفایت نکند تا هر یک اندکی بخورند!»
\par 8 یکی از شاگردانش که اندریاس برادر شمعون پطرس باشد، وی را گفت:
\par 9 «دراینجا پسری است که پنج نان جو و دو ماهی دارد. و لیکن این از برای این گروه چه می‌شود؟»
\par 10 عیسی گفت: «مردم را بنشانید.» و در آن مکان، گیاه بسیار بود، و آن گروه قریب به پنج هزار مردبودند که نشستند.
\par 11 عیسی نانها را گرفته و شکرنموده، به شاگردان داد و شاگردان به نشستگان دادند؛ و همچنین از دو ماهی نیز به قدری که خواستند.
\par 12 و چون سیر گشتند، به شاگردان خود گفت: «پاره های باقی‌مانده را جمع کنید تاچیزی ضایع نشود.»
\par 13 پس جمع کردند و ازپاره های پنج نان جو که از خورندگان زیاده آمده بود، دوازده سبد پر کردند.
\par 14 و چون مردمان این معجزه را که از عیسی صادر شده بود دیدند، گفتند که «این البته همان نبی است که باید در جهان بیاید!»
\par 15 و اما عیسی چون دانست که می‌خواهند بیایند و او را به زور برده پادشاه سازند، باز تنها به کوه برآمد.
\par 16 و چون شام شد، شاگردانش به‌جانب دریاپایین رفتند،
\par 17 و به کشتی سوار شده به آن طرف دریا به کفرناحوم روانه شدند. و چون تاریک شدعیسی هنوز نزد ایشان نیامده بود.
\par 18 و دریابواسطه وزیدن باد شدید به تلاطم آمد.
\par 19 پس وقتی که قریب به بیست و پنج یا سی تیر پرتاپ رانده بودند، عیسی را دیدند که بر روی دریاخرامان شده، نزدیک کشتی می‌آید. پس ترسیدند.
\par 20 او بدیشان گفت: «من هستم، مترسید!»
\par 21 و چون می‌خواستند او را در کشتی بیاورند، در ساعت کشتی به آن زمینی که عازم آن بودند رسید.
\par 22 بامدادان گروهی که به آن طرف دریا ایستاده بودند، دیدند که هیچ زورقی نبود غیر از آن که شاگردان او داخل آن شده بودند و عیسی باشاگردان خود داخل آن زورق نشده، بلکه شاگردانش تنها رفته بودند.
\par 23 لیکن زورقهای دیگر از طبریه آمد، نزدیک به آنجایی که نان خورده بودند بعد از آنکه خداوند شکر گفته بود.
\par 24 پس چون آن گروه دیدند که عیسی وشاگردانش در آنجا نیستند، ایشان نیز به کشتیهاسوار شده، در طلب عیسی به کفرناحوم آمدند.
\par 25 و چون او را در آن طرف دریا یافتند، بدوگفتند: «ای استاد کی به اینجا آمدی؟»
\par 26 عیسی در جواب ایشان گفت: «آمین آمین به شمامی گویم که مرا می‌طلبید نه بسبب معجزاتی که دیدید، بلکه بسبب آن نان که خوردید و سیرشدید.
\par 27 کار بکنید نه برای خوراک فانی بلکه برای خوراکی که تا حیات جاودانی باقی است که پسر انسان آن را به شما عطا خواهد کرد، زیراخدای پدر بر او مهر زده است.»
\par 28 بدو گفتند: «چه کنیم تا اعمال خدا را به‌جا آورده باشیم؟»
\par 29 عیسی در جواب ایشان گفت: «عمل خدا این است که به آن کسی‌که او فرستاد، ایمان بیاورید.»
\par 30 بدو گفتند: «چه معجزه می‌نمایی تا آن را دیده به تو ایمان آوریم؟ چکار می‌کنی؟
\par 31 پدران ما دربیابان من را خوردند، چنانکه مکتوب است که ازآسمان بدیشان نان عطا کرد تا بخورند.»
\par 32 عیسی بدیشان گفت: «آمین آمین به شما می‌گویم که موسی نان را از آسمان به شما نداد، بلکه پدر من نان حقیقی را از آسمان به شما می‌دهد.
\par 33 زیرا که نان خدا آن است که از آسمان نازل شده، به جهان حیات می‌بخشد.»
\par 34 آنگاه بدو گفتند: «ای خداوند این نان را پیوسته به ما بده.»
\par 35 عیسی بدیشان گفت: «من نان حیات هستم. کسی‌که نزد من آید، هرگز گرسنه نشود و هر‌که به من ایمان آرد هرگز تشنه نگردد.
\par 36 لیکن به شماگفتم که مرا هم دیدید و ایمان نیاوردید.
\par 37 هرآنچه پدر به من عطا کند، به‌جانب من آید و هر‌که به‌جانب من آید، او را بیرون نخواهم نمود.
\par 38 زیرا از آسمان نزول کردم نه تا به اراده خودعمل کنم، بلکه به اراده فرستنده خود.
\par 39 و اراده پدری که مرا فرستاد این است که از آنچه به من عطا کرد چیزی تلف نکنم بلکه در روز بازپسین آن را برخیزانم.
\par 40 و اراده فرستنده من این است که هر‌که پسر را دید و بدو ایمان آورد، حیات جاودانی داشته باشد و من در روز بازپسین او راخواهم برخیزانید.»
\par 41 پس یهودیان درباره او همهمه کردند زیراگفته بود: «من هستم آن نانی که از آسمان نازل شد.»
\par 42 و گفتند: «آیا این عیسی پسر یوسف نیست که ما پدر و مادر او را می‌شناسیم؟ پس چگونه می‌گوید که از آسمان نازل شدم؟»
\par 43 عیسی در جواب ایشان گفت: «با یکدیگرهمهمه مکنید.
\par 44 کسی نمی تواند نزد من آید، مگر آنکه پدری که مرا فرستاد او را جذب کند ومن در روز بازپسین او را خواهم برخیزانید.
\par 45 درانبیا مکتوب است که همه از خدا تعلیم خواهندیافت. پس هر‌که از پدر شنید و تعلیم یافت نزد من می‌آید.
\par 46 نه اینکه کسی پدر را دیده باشد، جزآن کسی‌که از جانب خداست، او پدر را دیده است.
\par 47 آمین آمین به شما می‌گویم هر‌که به من ایمان آرد، حیات جاودانی دارد.
\par 48 من نان حیات هستم.
\par 49 پدران شما در بیابان من را خوردند ومردند.
\par 50 این نانی است که از آسمان نازل شد تاهر‌که از آن بخورد نمیرد.
\par 51 من هستم آن نان زنده که از آسمان نازل شد. اگر کسی از این نان بخورد تا به ابد زنده خواهد ماند و نانی که من عطامی کنم جسم من است که آن را بجهت حیات جهان می‌بخشم.»
\par 52 پس یهودیان با یکدیگر مخاصمه کرده، می‌گفتند: «چگونه این شخص می‌تواند جسدخود را به ما دهد تا بخوریم؟»
\par 53 عیسی بدیشان گفت: «آمین آمین به شما می‌گویم اگر جسد پسرانسان را نخورید و خون او را ننوشید، در خودحیات ندارید.
\par 54 و هر‌که جسد مرا خورد و خون مرا نوشید، حیات جاودانی دارد و من در روزآخر او را خواهم برخیزانید.
\par 55 زیرا که جسد من، خوردنی حقیقی و خون من، آشامیدنی حقیقی است.
\par 56 پس هر‌که جسد مرا می‌خورد و خون مرا می‌نوشد، در من می‌ماند و من در او.
\par 57 چنانکه پدر زنده مرا فرستاد و من به پدر زنده هستم، همچنین کسی‌که مرا بخورد او نیز به من زنده می‌شود.
\par 58 این است نانی که از آسمان نازل شد، نه همچنان‌که پدران شما من را خوردند و مردند؛ بلکه هر‌که این نان را بخورد تا به ابد زنده ماند.»
\par 59 این سخن را وقتی که در کفرناحوم تعلیم می‌داد، در کنیسه گفت.
\par 60 آنگاه بسیاری از شاگردان او چون این راشنیدند گفتند: «این کلام سخت است! که می‌تواندآن را بشنود؟»
\par 61 چون عیسی در خود دانست که شاگردانش در این امر همهمه می‌کنند، بدیشان گفت: «آیا این شما را لغزش می‌دهد؟
\par 62 پس اگرپسر انسان را بینید که به‌جایی که اول بود صعودمی کند چه؟
\par 63 روح است که زنده می‌کند و اما ازجسد فایده‌ای نیست. کلامی را که من به شمامی گویم روح و حیات‌است.
\par 64 ولیکن بعضی از شما هستند که ایمان نمی آورند.» زیرا که عیسی از ابتدا می‌دانست کیانند که ایمان نمی آورند وکیست که او را تسلیم خواهد کرد.
\par 65 پس گفت: «از این سبب به شما گفتم که کسی نزد من نمی تواند آمد مگر آنکه پدر من، آن را بدو عطاکند.»
\par 66 در همان وقت بسیاری از شاگردان اوبرگشته، دیگر با او همراهی نکردند.
\par 67 آنگاه عیسی به آن دوازده گفت: «آیا شمانیز می‌خواهید بروید؟»
\par 68 شمعون پطرس به اوجواب داد: «خداوندا نزد که برویم؟ کلمات حیات جاودانی نزد تو است.
\par 69 و ما ایمان آورده و شناخته‌ایم که تو مسیح پسر خدای حی هستی.»
\par 70 عیسی بدیشان جواب داد: «آیا من شما دوازده را برنگزیدم و حال آنکه یکی از شماابلیسی است.»و این را درباره یهودا پسرشمعون اسخریوطی گفت، زیرا او بود که می‌بایست تسلیم‌کننده وی بشود و یکی از آن دوازده بود.
\par 71 و این را درباره یهودا پسرشمعون اسخریوطی گفت، زیرا او بود که می‌بایست تسلیم‌کننده وی بشود و یکی از آن دوازده بود.

\chapter{7}

\par 1 و بعد از آن عیسی در جلیل می‌گشت زیرانمی خواست در یهودیه راه رود چونکه یهودیان قصد قتل او می‌داشتند.
\par 2 و عید یهود که عید خیمه‌ها باشد نزدیک بود.
\par 3 پس برادرانش بدو گفتند: «از اینجا روانه شده، به یهودیه برو تاشاگردانت نیز آن اعمالی را که تو می‌کنی ببینند،
\par 4 زیرا هر‌که می‌خواهد آشکار شود در پنهانی کارنمی کند. پس اگر این کارها را می‌کنی خود را به جهان بنما.»
\par 5 زیرا که برادرانش نیز به او ایمان نیاورده بودند.
\par 6 آنگاه عیسی بدیشان گفت: «وقت من هنوز نرسیده، اما وقت شما همیشه حاضراست.
\par 7 جهان نمی تواند شما را دشمن دارد ولیکن مرا دشمن می‌دارد زیرا که من بر آن شهادت می‌دهم که اعمالش بد است.
\par 8 شما برای این عیدبروید. من حال به این عید نمی آیم زیرا که وقت من هنوز تمام نشده است.»
\par 9 چون این را بدیشان گفت، در جلیل توقف نمود.
\par 10 لیکن چون برادرانش برای عید رفته بودند، او نیز آمد، نه آشکار بلکه در خفا.
\par 11 اما یهودیان در عید او را جستجو نموده، می‌گفتند که او کجااست.
\par 12 و در میان مردم درباره او همهمه بسیاربود. بعضی می‌گفتند که مردی نیکو است ودیگران می‌گفتند نی بلکه گمراه‌کننده قوم است.
\par 13 و لیکن بسبب ترس از یهود، هیچ‌کس درباره اوظاهر حرف نمی زد.
\par 14 و چون نصف عید گذشته بود، عیسی به هیکل آمده، تعلیم می‌داد.
\par 15 و یهودیان تعجب نموده، گفتند: «این شخص هرگز تعلیم نیافته، چگونه کتب را می‌داند؟»
\par 16 عیسی در جواب ایشان گفت: «تعلیم من از من نیست، بلکه ازفرستنده من.
\par 17 اگر کسی بخواهد اراده او را به عمل آرد، درباره تعلیم خواهد دانست که از خدااست یا آنکه من از خود سخن می‌رانم.
\par 18 هر‌که ازخود سخن گوید، جلال خود را طالب بود و اماهر‌که طالب جلال فرستنده خود باشد، او صادق است و در او ناراستی نیست.
\par 19 آیا موسی تورات را به شما نداده است؟ و حال آنکه کسی از شما نیست که به تورات عمل کند. از برای چه می‌خواهید مرا به قتل رسانید؟»
\par 20 آنگاه همه درجواب گفتند: «تو دیو داری. که اراده دارد تو رابکشد؟»
\par 21 عیسی در جواب ایشان گفت: «یک عمل نمودم وهمه شما از آن متعجب شدید.
\par 22 موسی ختنه را به شما داد نه آنکه از موسی باشد بلکه از اجداد و در روز سبت مردم را ختنه می‌کنید.
\par 23 پس اگر کسی در روز سبت مختون شود تا شریعت موسی شکسته نشود، چرا بر من خشم می‌آورید از آن سبب که در روز سبت شخصی را شفای کامل دادم؟
\par 24 بحسب ظاهرداوری مکنید بلکه به راستی داوری نمایید.»
\par 25 پس بعضی از اهل اورشلیم گفتند: «آیا این آن نیست که قصد قتل او دارند؟
\par 26 و اینک آشکارا حرف می‌زند و بدو هیچ نمی گویند. آیاروسا یقین می‌دانند که او در حقیقت مسیح است؟
\par 27 لیکن این شخص را می‌دانیم از کجا است، امامسیح چون آید هیچ‌کس نمی شناسد که از کجااست.»
\par 28 و عیسی چون در هیکل تعلیم می‌داد، ندا کرده، گفت: «مرا می‌شناسید و نیز می‌دانید ازکجا هستم و از خود نیامده‌ام بلکه فرستنده من حق است که شما او را نمی شناسید.
\par 29 اما من اورا می‌شناسم زیرا که از او هستم و او مرا فرستاده است.»
\par 30 آنگاه خواستند او را گرفتار کنند ولیکن کسی بر او دست نینداخت زیرا که ساعت او هنوزنرسیده بود.
\par 31 آنگاه بسیاری از آن گروه بدوایمان آوردند و گفتند: «آیا چون مسیح آید، معجزات بیشتر از اینها که این شخص می‌نماید، خواهد نمود؟»
\par 32 چون فریسیان شنیدند که خلق درباره اواین همهمه می‌کنند، فریسیان و روسای کهنه خادمان فرستادند تا او را بگیرند.
\par 33 آنگاه عیسی گفت: «اندک زمانی دیگر با شما هستم، بعد نزدفرستنده خود می‌روم.
\par 34 و مرا طلب خواهید کردو نخواهید یافت و آنجایی که من هستم شمانمی توانید آمد.»
\par 35 پس یهودیان با یکدیگرگفتند: «او کجا می‌خواهد برود که ما او رانمی یابیم؟ آیا اراده دارد به سوی پراکندگان یونانیان رود و یونانیان را تعلیم دهد؟
\par 36 این چه کلامی است که گفت مرا طلب خواهید کرد ونخواهید یافت و جایی که من هستم شمانمی توانید آمد؟»
\par 37 و در روز آخر که روز بزرگ عید بود، عیسی ایستاده، ندا کرد و گفت: «هر‌که تشنه باشدنزد من آید و بنوشد.
\par 38 کسی‌که به من ایمان آورد، چنانکه کتاب می‌گوید، از بطن او نهرهای آب زنده جاری خواهد شد.»
\par 39 اما این را گفت درباره روح که هر‌که به او ایمان آرد او را خواهدیافت زیرا که روح‌القدس هنوز عطا نشده بود، چونکه عیسی تا به حال جلال نیافته بود.
\par 40 آنگاه بسیاری از آن گروه، چون این کلام راشنیدند، گفتند: «در حقیقت این شخص همان نبی است.»
\par 41 و بعضی گفتند: «او مسیح است.» وبعضی گفتند: «مگر مسیح از جلیل خواهد آمد؟
\par 42 آیا کتاب نگفته است که از نسل داود و ازبیت لحم، دهی که داود در آن بود، مسیح ظاهر خواهد شد؟»
\par 43 پس درباره او در میان مردم اختلاف افتاد.
\par 44 و بعضی از ایشان خواستند او رابگیرند و لکن هیچ‌کس بر او دست نینداخت.
\par 45 پس خادمان نزد روسای کهنه و فریسیان آمدند. آنها بدیشان گفتند: «برای چه او رانیاوردید؟»
\par 46 خادمان در جواب گفتند: «هرگزکسی مثل این شخص سخن نگفته است!»
\par 47 آنگاه فریسیان در جواب ایشان گفتند: «آیاشما نیز گمراه شده‌اید؟
\par 48 مگر کسی از سرداران یا از فریسیان به او ایمان آورده است؟
\par 49 ولیکن این گروه که شریعت را نمی دانند، ملعون می‌باشند.»
\par 50 نیقودیموس، آنکه در شب نزد اوآمده و یکی از ایشان بود بدیشان گفت:
\par 51 «آیاشریعت ما بر کسی فتوی می‌دهد، جز آنکه اول سخن او را بشنوند و کار او را دریافت کنند؟»ایشان در جواب وی گفتند: «مگر تو نیزجلیلی هستی؟ تفحص کن و ببین زیرا که هیچ نبی از جلیل برنخاسته است.» پس هر یک به خانه خود رفتند.
\par 52 ایشان در جواب وی گفتند: «مگر تو نیزجلیلی هستی؟ تفحص کن و ببین زیرا که هیچ نبی از جلیل برنخاسته است.» پس هر یک به خانه خود رفتند.

\chapter{8}

\par 1 اما عیسی به کوه زیتون رفت.
\par 2 و بامدادان باز به هیکل آمد و چون جمیع قوم نزد اوآمدند نشسته، ایشان را تعلیم می‌داد.
\par 3 که ناگاه کاتبان و فریسیان زنی را که در زنا گرفته شده بود، پیش او آوردند و او را در میان برپا داشته،
\par 4 بدوگفتند: «ای استاد، این زن در عین عمل زنا گرفته شد؛
\par 5 و موسی در تورات به ما حکم کرده است که چنین زنان سنگسار شوند. اما تو چه می‌گویی؟»
\par 6 و این را از روی امتحان بدو گفتند تا ادعایی بر او پیدا کنند. اما عیسی سر به زیرافکنده، به انگشت خود بر روی زمین می‌نوشت.
\par 7 و چون در سوال کردن الحاح می‌نمودند، راست شده، بدیشان گفت: «هر‌که از شما گناه ندارد اول بر او سنگ اندازد.»
\par 8 و باز سر به زیر افکنده، برزمین می‌نوشت.
\par 9 پس چون شنیدند، از ضمیرخود ملزم شده، از مشایخ شروع کرده تا به آخر، یک یک بیرون رفتند و عیسی تنها باقی ماند با آن زن که در میان ایستاده بود.
\par 10 پس عیسی چون راست شد و غیر از زن کسی را ندید، بدو گفت: «ای زن آن مدعیان تو کجا شدند؟ آیا هیچ‌کس برتو فتوا نداد؟»
\par 11 گفت: «هیچ‌کس‌ای آقا.» عیسی گفت: «من هم بر تو فتوا نمی دهم. برو دیگر گناه مکن.»
\par 12 پس عیسی باز بدیشان خطاب کرده، گفت: «من نور عالم هستم. کسی‌که مرا متابعت کند، درظلمت سالک نشود بلکه نور حیات را یابد.»
\par 13 آنگاه فریسیان بدو گفتند: «تو بر خود شهادت می‌دهی، پس شهادت تو راست نیست.»
\par 14 عیسی در جواب ایشان گفت: «هرچند من برخود شهادت می‌دهم، شهادت من راست است زیرا که می‌دانم از کجا آمده‌ام و به کجا خواهم رفت، لیکن شما نمی دانید از کجا آمده‌ام و به کجامی روم.
\par 15 شما بحسب جسم حکم می‌کنید امامن بر هیچ‌کس حکم نمی کنم.
\par 16 و اگر من حکم دهم، حکم من راست است، از آنرو که تنها نیستم بلکه من و پدری که مرا فرستاد.
\par 17 و نیز درشریعت شما مکتوب است که شهادت دو کس حق است.
\par 18 من بر خود شهادت می‌دهم و پدری که مرا فرستاد نیز برای من شهادت می‌دهد.»
\par 19 بدو گفتند: «پدر تو کجا است؟» عیسی جواب داد که «نه مرا می‌شناسید و نه پدر مرا. هرگاه مرامی شناختید پدر مرا نیز می‌شناختید.»
\par 20 و این کلام را عیسی در بیت‌المال گفت، وقتی که درهیکل تعلیم می‌داد و هیچ‌کس او را نگرفت بجهت آنکه ساعت او هنوز نرسیده بود.
\par 21 باز عیسی بدیشان گفت: «من می‌روم و مراطلب خواهید کرد و در گناهان خود خواهید مردو جایی که من می‌روم شما نمی توانید آمد.»
\par 22 یهودیان گفتند: «آیا اراده قتل خود دارد که می‌گوید به‌جایی خواهم رفت که شما نمی توانیدآمد؟»
\par 23 ایشان را گفت: «شما از پایین می‌باشیداما من از بالا. شما از این جهان هستید، لیکن من ازاین جهان نیستم.
\par 24 از این جهت به شما گفتم که در گناهان خود خواهید مرد، زیرا اگر باور نکنیدکه من هستم در گناهان خود خواهید مرد.»
\par 25 بدوگفتند: «تو کیستی؟» عیسی بدیشان گفت: «همانم که از اول نیز به شما گفتم.
\par 26 من چیزهای بسیاردارم که درباره شما بگویم و حکم کنم؛ لکن آنکه مرا فرستاد حق است و من آنچه از او شنیده‌ام به جهان می‌گویم.»
\par 27 ایشان نفهمیدند که بدیشان درباره پدر سخن می‌گوید.
\par 28 عیسی بدیشان گفت: «وقتی که پسر انسان را بلند کردید، آن وقت خواهید دانست که من هستم و از خود کاری نمی کنم بلکه به آنچه پدرم مرا تعلیم داد تکلم می‌کنم.
\par 29 و او که مرا فرستاد، با من است و پدر مرا تنها نگذارده است زیرا که من همیشه کارهای پسندیده او را به‌جا می‌آورم.»
\par 30 چون این را گفت، بسیاری بدو ایمان آوردند.
\par 31 پس عیسی به یهودیانی که بدو ایمان آوردند گفت: «اگر شما در کلام من بمانیدفی الحقیقه شاگرد من خواهید شد،
\par 32 و حق راخواهید شناخت و حق شما را آزاد خواهد کرد.»
\par 33 بدو جواب دادند که «اولاد ابراهیم می‌باشیم وهرگز هیچ‌کس را غلام نبوده‌ایم. پس چگونه تومی گویی که آزاد خواهید شد؟»
\par 34 عیسی درجواب ایشان گفت: «آمین آمین به شما می‌گویم هر‌که گناه می‌کند غلام گناه است.
\par 35 و غلام همیشه در خانه نمی ماند، اما پسر همیشه می‌ماند.
\par 36 پس اگر پسر شما را آزاد کند، در حقیقت آزادخواهید بود.
\par 37 می‌دانم که اولاد ابراهیم هستید، لیکن می‌خواهید مرا بکشید زیرا کلام من در شماجای ندارد.
\par 38 من آنچه نزد پدر خود دیده‌ام می‌گویم و شما آنچه نزد پدر خود دیده ایدمی کنید.»
\par 39 در جواب او گفتند که «پدر ماابراهیم است.» عیسی بدیشان گفت: «اگر اولادابراهیم می‌بودید، اعمال ابراهیم را به‌جامی آوردید.
\par 40 ولیکن الان می‌خواهید مرا بکشیدو من شخصی هستم که با شما به راستی که از خداشنیده‌ام تکلم می‌کنم. ابراهیم چنین نکرد.
\par 41 شمااعمال پدر خود را به‌جا می‌آورید.»
\par 42 عیسی به ایشان گفت: «اگر خدا پدر شما می‌بود، مرا دوست می داشتید، زیرا که من از جانب خدا صادر شده وآمده‌ام، زیرا که من از پیش خود نیامده‌ام بلکه اومرا فرستاده است.
\par 43 برای چه سخن مرانمی فهمید؟ از آنجهت که کلام مرا نمی توانیدبشنوید.
\par 44 شما از پدر خود ابلیس می‌باشید وخواهشهای پدر خود را می‌خواهید به عمل آرید. او از اول قاتل بود و در راستی ثابت نمی باشد، از آنجهت که در او راستی نیست. هرگاه به دروغ سخن می‌گوید، از ذات خودمی گوید زیرا دروغگو و پدر دروغگویان است.
\par 45 و اما من از این سبب که راست می‌گویم، مراباور نمی کنید.
\par 46 کیست از شما که مرا به گناه ملزم سازد؟ پس اگر راست می‌گویم، چرا مرا باورنمی کنید؟
\par 47 کسی‌که از خدا است، کلام خدا رامی شنود و از این سبب شما نمی شنوید که از خدانیستید.»
\par 48 پس یهودیان در جواب او گفتند: «آیا ماخوب نگفتیم که تو سامری هستی و دیو داری؟»
\par 49 عیسی جواب داد که «من دیو ندارم، لکن پدرخود را حرمت می‌دارم و شما مرا بی‌حرمت می‌سازید.
\par 50 من جلال خود را طالب نیستم، کسی هست که می‌طلبد و داوری می‌کند.
\par 51 آمین آمین به شما می‌گویم، اگر کسی کلام مرا حفظکند، موت را تا به ابد نخواهد دید.»
\par 52 پس یهودیان بدو گفتند: «الان دانستیم که دیو داری! ابراهیم و انبیا مردند و تو می‌گویی اگر کسی کلام مرا حفظ کند، موت را تا به ابد نخواهد چشید
\par 53 آیا تو از پدر ما ابراهیم که مرد و انبیایی که مردند بزرگتر هستی؟ خود را که می‌دانی؟»
\par 54 عیسی در جواب داد: «اگر خود را جلال دهم، جلال من چیزی نباشد. پدر من آن است که مرا جلال می‌بخشد، آنکه شما می‌گویید خدای ما است.
\par 55 و او را نمی شناسید، اما من او رامی شناسم و اگر گویم او را نمی شناسم مثل شمادروغگو می‌باشم. لیکن او را می‌شناسم و قول اورا نگاه می‌دارم.
\par 56 پدر شما ابراهیم شادی کرد براینکه روز مرا ببیند و دید و شادمان گردید.»
\par 57 یهودیان بدو گفتند: «هنوز پنجاه سال نداری وابراهیم را دیده‌ای؟»
\par 58 عیسی بدیشان گفت: «آمین آمین به شما می‌گویم که پیش از آنکه ابراهیم پیدا شود من هستم.»آنگاه سنگهابرداشتند تا او را سنگسار کنند. اما عیسی خود رامخفی ساخت و از میان گذشته، از هیکل بیرون شد و همچنین برفت.
\par 59 آنگاه سنگهابرداشتند تا او را سنگسار کنند. اما عیسی خود رامخفی ساخت و از میان گذشته، از هیکل بیرون شد و همچنین برفت.

\chapter{9}

\par 1 و وقتی که می‌رفت کوری مادرزاد دید.
\par 2 وشاگردانش از او سوال کرده، گفتند: «ای استاد گناه که کرد، این شخص یا والدین او که کورزاییده شد؟»
\par 3 عیسی جواب داد که «گناه نه این شخص کرد و نه پدر و مادرش، بلکه تا اعمال خدادر وی ظاهر شود.
\par 4 مادامی که روز است، مرا بایدبه‌کارهای فرستنده خود مشغول باشم. شب می‌آید که در آن هیچ‌کس نمی تواند کاری کند.
\par 5 مادامی که در جهان هستم، نور جهانم.»
\par 6 این راگفت و آب دهان بر زمین انداخته، از آب گل ساخت و گل را به چشمان کور مالید،
\par 7 و بدوگفت: «برو در حوض سیلوحا (که به‌معنی مرسل است ) بشوی.» پس رفته شست و بینا شده، برگشت.
\par 8 پس همسایگان و کسانی که او را پیش از آن در حالت کوری دیده بودند، گفتند: «آیا این آن نیست که می‌نشست و گدایی می‌کرد؟»
\par 9 بعضی گفتند: «همان است.» و بعضی گفتند: «شباهت بدودارد.» او گفت: «من همانم.»
\par 10 بدو گفتند: «پس چگونه چشمان تو بازگشت؟»
\par 11 او جواب داد: «شخصی که او را عیسی می‌گویند، گل ساخت وبر چشمان من مالیده، به من گفت به حوض سیلوحا برو و بشوی. آنگاه رفتم و شسته بیناگشتم.»
\par 12 به وی گفتند: «آن شخص کجا است؟» گفت: «نمی دانم.»
\par 13 پس او را که پیشتر کور بود، نزد فریسیان آوردند.
\par 14 و آن روزی که عیسی گل ساخته، چشمان او را باز کرد روز سبت بود.
\par 15 آنگاه فریسیان نیز از او سوال کردند که «چگونه بیناشدی؟» بدیشان گفت: «گل به چشمهای من گذارد. پس شستم و بینا شدم.»
\par 16 بعضی ازفریسیان گفتند: «آن شخص از جانب خدا نیست، زیرا که سبت را نگاه نمی دارد.» دیگران گفتند: «چگونه شخص گناهکار می‌تواند مثل این معجزات ظاهر سازد.» و در میان ایشان اختلاف افتاد.
\par 17 باز بدان کور گفتند: «تو درباره او چه می‌گویی که چشمان تو را بینا ساخت؟» گفت: «نبی است.»
\par 18 لیکن یهودیان سرگذشت او را باور نکردندکه کور بوده و بینا شده است، تا آنکه پدر و مادرآن بینا شده را طلبیدند.
\par 19 و از ایشان سوال کرده، گفتند: «آیا این است پسر شما که می‌گویید کورمتولد شده؟ پس چگونه الحال بینا گشته است؟»
\par 20 پدر و مادر او در جواب ایشان گفتند: «می‌دانیم که این پسر ما است و کور متولد شده.
\par 21 لیکن الحال چطور می‌بیند، نمی دانیم ونمی دانیم که چشمان او را باز نموده. او بالغ است از وی سوال کنید تا او احوال خود را بیان کند.»
\par 22 پدر و مادر او چنین گفتند زیرا که از یهودیان می‌ترسیدند، از آنرو که یهودیان با خود عهد کرده بودند که هر‌که اعتراف کند که او مسیح است، ازکنیسه بیرونش کنند.
\par 23 و از اینجهت والدین اوگفتند: «او بالغ است از خودش بپرسید.»
\par 24 پس آن شخص را که کور بود، باز خوانده، بدو گفتند: «خدا را تمجید کن. ما می‌دانیم که این شخص گناهکار است.»
\par 25 او جواب داد اگرگناهکار است نمی دانم. یک چیز می‌دانم که کوربودم و الان بینا شده‌ام.»
\par 26 باز بدو گفتند: «با توچه کرد و چگونه چشمهای تو را باز کرد؟»
\par 27 ایشان را جواب داد که «الان به شما گفتم. نشنیدید؟ و برای چه باز می‌خواهید بشنوید؟ آیاشما نیز اراده دارید شاگرد او بشوید؟»
\par 28 پس اورا دشنام داده، گفتند: «تو شاگرد او هستی. ماشاگرد موسی می‌باشیم.
\par 29 ما می‌دانیم که خدا باموسی تکلم کرد. اما این شخص را نمی دانیم ازکجا است.»
\par 30 آن مرد جواب داده، بدیشان گفت: «این عجب است که شما نمی دانید از کجا است وحال آنکه چشمهای مرا باز کرد.
\par 31 و می‌دانیم که خدا دعای گناهکاران را نمی شنود؛ و لیکن اگرکسی خداپرست باشد و اراده او را به‌جا آرد، او رامی شنود.
\par 32 از ابتدای عالم شنیده نشده است که کسی چشمان کور مادرزاد را باز کرده باشد.
\par 33 اگر این شخص از خدا نبودی، هیچ کارنتوانستی کرد.»
\par 34 در جواب وی گفتند: «تو به کلی با گناه متولد شده‌ای. آیا تو ما را تعلیم می‌دهی؟» پس او را بیرون راندند.
\par 35 عیسی چون شنید که او را بیرون کرده‌اند، وی را جسته گفت: «آیا تو به پسر خدا ایمان داری؟»
\par 36 او در جواب گفت: «ای آقا کیست تا به او ایمان آورم؟»
\par 37 عیسی بدو گفت: «تو نیز او رادیده‌ای و آنکه با تو تکلم می‌کند همان است.»
\par 38 گفت: «ای خداوند ایمان آوردم.» پس او راپرستش نمود.
\par 39 آنگاه عیسی گفت: «من در این جهان بجهت داوری آمدم تا کوران بینا و بینایان، کور شوند.»
\par 40 بعضی از فریسیان که با او بودند، چون این کلام را شنیدند گفتند: «آیا ما نیز کورهستیم؟»عیسی بدیشان گفت: «اگر کورمی بودید گناهی نمی داشتید و لکن الان می‌گوییدبینا هستیم. پس گناه شما می‌ماند.
\par 41 عیسی بدیشان گفت: «اگر کورمی بودید گناهی نمی داشتید و لکن الان می‌گوییدبینا هستیم. پس گناه شما می‌ماند.

\chapter{10}

\par 1 «آمین آمین به شما می‌گویم هر‌که از دربه آغل گوسفند داخل نشود، بلکه از راه دیگر بالا رود، او دزد و راهزن است.
\par 2 و اما آنکه از در داخل شود، شبان گوسفندان است.
\par 3 دربان بجهت او می‌گشاید و گوسفندان آواز او رامی شنوند و گوسفندان خود را نام بنام می‌خواند وایشان را بیرون می‌برد.
\par 4 و وقتی که گوسفندان خود را بیرون برد، پیش روی ایشان می‌خرامد وگوسفندان از عقب او می‌روند، زیرا که آواز او رامی شناسند.
\par 5 لیکن غریب را متابعت نمی کنند، بلکه از او می‌گریزند زیرا که آواز غریبان رانمی شناسند.»
\par 6 و این مثل را عیسی برای ایشان آورد، اماایشان نفهمیدند که چه چیز بدیشان می‌گوید.
\par 7 آنگاه عیسی بدیشان باز‌گفت: «آمین آمین به شما می‌گویم که من در گوسفندان هستم.
\par 8 جمیع کسانی که پیش از من آمدند، دزد و راهزن هستند، لیکن گوسفندان سخن ایشان را نشنیدند.
\par 9 من درهستم هر‌که از من داخل گردد نجات یابد و بیرون و درون خرامد و علوفه یابد.
\par 10 دزد نمی آید مگرآنکه بدزدد و بکشد و هلاک کند. من آمدم تاایشان حیات یابند و آن را زیادتر حاصل کنند.
\par 11 «من شبان نیکو هستم. شبان نیکو جان خودرا در راه گوسفندان می‌نهد.
\par 12 اما مزدوری که شبان نیست و گوسفندان از آن او نمی باشند، چون بیند که گرگ می‌آید، گوسفندان را گذاشته، فرار می‌کند و گرگ گوسفندان را می‌گیرد وپراکنده می‌سازد.
\par 13 مزدور می‌گریزد چونکه مزدور است و به فکر گوسفندان نیست.
\par 14 من شبان نیکو هستم و خاصان خود را می‌شناسم وخاصان من مرا می‌شناسند.
\par 15 چنانکه پدر مرامی شناسد و من پدر را می‌شناسم و جان خود رادر راه گوسفندان می‌نهم.
\par 16 و مرا گوسفندان دیگر هست که از این آغل نیستند. باید آنها را نیزبیاورم و آواز مرا خواهند شنید و یک گله و یک شبان خواهند شد.
\par 17 و از این سبب پدر مرادوست می‌دارد که من جان خود را می‌نهم تا آن راباز گیرم.
\par 18 کسی آن را از من نمی گیرد، بلکه من خود آن را می‌نهم. قدرت دارم که آن را بنهم وقدرت دارم آن را باز گیرم. این حکم را از پدرخود یافتم.»
\par 19 باز به‌سبب این کلام، در میان یهودیان اختلاف افتاد.
\par 20 بسیاری از ایشان گفتند که «دیودارد و دیوانه است. برای چه بدو گوش می‌دهید؟»
\par 21 دیگران گفتند که «این سخنان دیوانه نیست. آیا دیو می‌تواند چشم کوران را بازکند؟»
\par 22 پس در اورشلیم، عید تجدید شد و زمستان بود.
\par 23 و عیسی در هیکل، در رواق سلیمان می‌خرامید.
\par 24 پس یهودیان دور او را گرفته، بدوگفتند: «تا کی ما را متردد داری؟ اگر تو مسیح هستی، آشکارا به ما بگو.
\par 25 عیسی بدیشان جواب داد: «من به شما گفتم و ایمان نیاوردید. اعمالی را که به اسم پدر خود به‌جا می‌آورم، آنهابرای من شهادت می‌دهد.
\par 26 لیکن شما ایمان نمی آورید زیرا از گوسفندان من نیستید، چنانکه به شما گفتم.
\par 27 گوسفندان من آواز مرا می‌شنوندو من آنها را می‌شناسم و مرا متابعت می‌کنند.
\par 28 ومن به آنها حیات جاودانی می‌دهم و تا به ابد هلاک نخواهند شد و هیچ‌کس آنها را از دست من نخواهد گرفت.
\par 29 پدری که به من داد از همه بزرگتر است و کسی نمی تواند از دست پدر من بگیرد.
\par 30 من و پدر یک هستیم.»
\par 31 آنگاه یهودیان باز سنگها برداشتند تا او راسنگسار کنند.
\par 32 عیسی بدیشان جواب داد: «ازجانب پدر خود بسیار کارهای نیک به شما نمودم. به‌سبب کدام‌یک از آنها مرا سنگسار می‌کنید؟»
\par 33 یهودیان در جواب گفتند: «به‌سبب عمل نیک، تو را سنگسار نمی کنیم، بلکه به‌سبب کفر، زیرا توانسان هستی و خود را خدا می‌خوانی.»
\par 34 عیسی در جواب ایشان گفت: «آیا در تورات شما نوشته نشده است که من گفتم شما خدایان هستید؟
\par 35 پس اگر آنانی را که کلام خدا بدیشان نازل شد، خدایان خواند و ممکن نیست که کتاب محو گردد،
\par 36 آیا کسی را که پدر تقدیس کرده، به جهان فرستاد، بدو می‌گویید کفر می‌گویی، از آن سبب که گفتم پسر خدا هستم؟
\par 37 اگر اعمال پدرخود را به‌جا نمی آورم، به من ایمان میاورید.
\par 38 ولکن چنانچه به‌جا می‌آورم، هرگاه به من ایمان نمی آورید، به اعمال ایمان آورید تا بدانید و یقین کنید که پدر در من است و من در او.»
\par 39 پس دیگرباره خواستند او را بگیرند، اما از دستهای ایشان بیرون رفت.
\par 40 و باز به آن طرف اردن، جایی که اول یحیی تعمید می‌داد، رفت و در آنجا توقف نمود.
\par 41 وبسیاری نزد او آمده، گفتند که یحیی هیچ معجزه ننمود و لکن هر‌چه یحیی درباره این شخص گفت راست است.پس بسیاری در آنجا به او ایمان آوردند.
\par 42 پس بسیاری در آنجا به او ایمان آوردند.

\chapter{11}

\par 1 و شخصی ایلعازر نام، بیمار بود، از اهل بیت عنیا که ده مریم و خواهرش مرتابود.
\par 2 و مریم آن است که خداوند را به عطر، تدهین ساخت و پایهای او را به موی خودخشکانید که برادرش ایلعازر بیمار بود.
\par 3 پس خواهرانش نزد او فرستاده، گفتند: «ای آقا، اینک آن که او را دوست می‌داری مریض است.»
\par 4 چون عیسی این را شنید گفت: «این مرض تا به موت نیست بلکه برای جلال خدا تا پسر خدا از آن جلال یابد.»
\par 5 و عیسی مرتا و خواهرش و ایلعازررا محبت می‌نمود.
\par 6 پس چون شنید که بیمار است در جایی که بود دو روز توقف نمود.
\par 7 و بعد از آن به شاگردان خود گفت: «باز به یهودیه برویم.»
\par 8 شاگردان او راگفتند: «ای معلم، الان یهودیان می‌خواستند تو راسنگسار کنند؛ و آیا باز می‌خواهی بدانجابروی؟»
\par 9 عیسی جواب داد: «آیا ساعتهای روزدوازده نیست؟ اگر کسی در روز راه رود لغزش نمی خورد زیرا که نور این جهان را می‌بیند.
\par 10 ولیکن اگر کسی در شب راه رود لغزش خورد زیراکه نور در او نیست.»
\par 11 این را گفت و بعد از آن به ایشان فرمود: «دوست ما ایلعازر در خواب است. اما می‌روم تا او را بیدار کنم.»
\par 12 شاگردان اوگفتند: «ای آقا اگر خوابیده است، شفا خواهدیافت.»
\par 13 اما عیسی درباره موت او سخن گفت وایشان گمان بردند که از آرامی خواب می‌گوید.
\par 14 آنگاه عیسی علانیه بدیشان گفت: «ایلعازرمرده است.
\par 15 و برای شما خشنود هستم که درآنجا نبودم تا ایمان آرید ولکن نزد او برویم.»
\par 16 پس توما که به‌معنی توام باشد به همشاگردان خود گفت: «ما نیز برویم تا با او بمیریم.»
\par 17 پس چون عیسی آمد، یافت که چهار روزاست در قبر می‌باشد.
\par 18 و بیت عنیا نزدیک اورشلیم بود، قریب به پانزده تیر پرتاب.
\par 19 وبسیاری از یهود نزد مرتا و مریم آمده بودند تابجهت برادرشان، ایشان را تسلی دهند.
\par 20 وچون مرتا شنید که عیسی می‌آید، او را استقبال کرد. لیکن مریم در خانه نشسته ماند.
\par 21 پس مرتابه عیسی گفت: «ای آقا اگر در اینجا می‌بودی، برادر من نمی مرد.
\par 22 ولیکن الان نیز می‌دانم که هرچه از خدا طلب کنی، خدا آن را به تو خواهد داد.
\par 23 عیسی بدو گفت: «برادر تو خواهد برخاست.»
\par 24 مرتا به وی گفت: «می‌دانم که در قیامت روزبازپسین خواهد برخاست.»
\par 25 عیسی بدو گفت: «من قیامت و حیات هستم. هر‌که به من ایمان آورد، اگر مرده باشد، زنده گردد.
\par 26 و هر‌که زنده بود و به من ایمان آورد، تا به ابد نخواهد مرد. آیااین را باور می‌کنی؟»
\par 27 او گفت: «بلی‌ای آقا، من ایمان دارم که تویی مسیح پسر خدا که در جهان آینده است.»
\par 28 و چون این را گفت، رفت و خواهر خودمریم را در پنهانی خوانده، گفت: «استاد آمده است و تو را می‌خواند.»
\par 29 او چون این را بشنید، بزودی برخاسته، نزد او آمد.
\par 30 و عیسی هنوزوارد ده نشده بود، بلکه در جایی بود که مرتا او راملاقات کرد.
\par 31 و یهودیانی که در خانه با او بودندو او را تسلی می‌دادند، چون دیدند که مریم برخاسته، به تعجیل بیرون می‌رود، از عقب اوآمده، گفتند: «به‌سر قبر می‌رود تا در آنجا گریه کند.»
\par 32 و مریم چون به‌جایی که عیسی بودرسید، او را دیده، بر قدمهای او افتاد و بدو گفت: «ای آقا اگر در اینجا می‌بودی، برادر من نمی مرد.»
\par 33 عیسی چون او را گریان دید و یهودیان را هم که با او آمده بودند گریان یافت، در روح خودبشدت مکدر شده، مضطرب گشت.
\par 34 و گفت: «او را کجا گذارده‌اید؟» به او گفتند: «ای آقا بیا وببین.»
\par 35 عیسی بگریست.
\par 36 آنگاه یهودیان گفتند: «بنگرید چقدر او را دوست می‌داشت!»
\par 37 بعضی از ایشان گفتند: «آیا این شخص که چشمان کور را باز کرد، نتوانست امر کند که این مرد نیز نمیرد؟»
\par 38 پس عیسی باز بشدت در خود مکدر شده، نزد قبر‌آمد و آن غاره‌ای بود، سنگی بر سرش گذارده.
\par 39 عیسی گفت: «سنگ را بردارید.» مرتاخواهر میت بدو گفت: «ای آقا الان متعفن شده، زیرا که چهار روز گذشته است.»
\par 40 عیسی به وی گفت: «آیا به تو نگفتم اگر ایمان بیاوری، جلال خدا را خواهی دید؟»
\par 41 پس سنگ را از جایی که میت گذاشته شده بود برداشتند. عیسی چشمان خود را بالا انداخته، گفت: «ای پدر، تو را شکرمی کنم که سخن مرا شنیدی.
\par 42 و من می‌دانستم که همیشه سخن مرا می‌شنوی؛ و لکن بجهت خاطر این گروه که حاضرند گفتم تا ایمان بیاورندکه تو مرا فرستادی.»
\par 43 چون این را گفت، به آوازبلند ندا کرد: «ای ایلعازر، بیرون بیا.»
\par 44 در حال آن مرده دست و پای به کفن بسته بیرون آمد وروی او به‌دستمالی پیچیده بود. عیسی بدیشان گفت: «او را باز کنید و بگذارید برود.»
\par 45 آنگاه بسیاری از یهودیان که با مریم آمده بودند، چون آنچه عیسی کرد دیدند، بدو ایمان آوردند.
\par 46 ولیکن بعضی از ایشان نزد فریسیان رفتند و ایشان را از کارهایی که عیسی کرده بودآگاه ساختند.
\par 47 پس روسای کهنه و فریسیان شورا نموده، گفتند: «چه کنیم زیرا که این مرد، معجزات بسیارمی نماید؟
\par 48 اگر او را چنین واگذاریم، همه به اوایمان خواهند‌آورد و رومیان آمده، جا و قوم مارا خواهند گرفت.»
\par 49 یکی از ایشان، قیافا نام که در آن سال رئیس کهنه بود، بدیشان گفت: «شماهیچ نمی دانید
\par 50 و فکر نمی کنید که بجهت مامفید است که یک شخص در راه قوم بمیرد وتمامی طائفه هلاک نگردند.»
\par 51 و این را از خودنگفت بلکه چون در آن سال رئیس کهنه بود، نبوت کرد که می‌بایست عیسی در راه آن طایفه بمیرد؛
\par 52 و نه در راه آن طایفه تنها بلکه تا فرزندان خدا را که متفرقند در یکی جمع کند.
\par 53 و از همان روز شورا کردند که او را بکشند.
\par 54 پس بعداز آن عیسی در میان یهود آشکارا راه نمی رفت بلکه از آنجا روانه شد به موضعی نزدیک بیابان به شهری که افرایم نام داشت و با شاگردان خود درآنجا توقف نمود.
\par 55 و چون فصح یهود نزدیک شد، بسیاری ازبلوکات قبل از فصح به اورشلیم آمدند تا خود راطاهر سازند
\par 56 و در طلب عیسی می‌بودند و درهیکل ایستاده، به یکدیگر می‌گفتند: « «چه گمان می‌برید؟ آیا برای عید نمی آید؟»اما روسای کهنه و فریسیان حکم کرده بودند که اگر کسی بداند که کجا است اطلاع دهد تا او را گرفتارسازند.
\par 57 اما روسای کهنه و فریسیان حکم کرده بودند که اگر کسی بداند که کجا است اطلاع دهد تا او را گرفتارسازند.

\chapter{12}

\par 1 پس شش روز قبل از عید فصح، عیسی به بیت عنیا آمد، جایی که ایلعازر مرده را از مردگان برخیزانیده بود.
\par 2 و برای او در آنجاشام حاضر کردند و مرتا خدمت می‌کرد و ایلعازریکی از مجلسیان با او بود.
\par 3 آنگاه مریم رطلی ازعطر سنبل خالص گرانبها گرفته، پایهای عیسی راتدهین کرد و پایهای او را از مویهای خودخشکانید، چنانکه خانه از بوی عطر پر شد.
\par 4 پس یکی از شاگردان او یعنی یهودای اسخریوطی، پسر شمعون که تسلیم‌کننده وی بود، گفت:
\par 5 «برای چه این عطر به سیصد دینار فروخته نشدتا به فقرا داده شود؟»
\par 6 و این را نه از آنرو گفت که پروای فقرا می‌داشت، بلکه از آنرو که دزد بود وخریطه در حواله او و از آنچه در آن انداخته می‌شد برمی داشت.
\par 7 عیسی گفت: «او را واگذارزیرا که بجهت روز تکفین من این را نگاه داشته است.
\par 8 زیرا که فقرا همیشه با شما می‌باشند و امامن همه وقت با شما نیستم.»
\par 9 پس جمعی کثیر از یهود چون دانستند که عیسی در آنجا است آمدند نه برای عیسی و بس بلکه تا ایلعازر را نیز که از مردگانش برخیزانیده بود ببینند.
\par 10 آنگاه روسای کهنه شورا کردند که ایلعازر را نیز بکشند.
\par 11 زیرا که بسیاری از یهودبه‌سبب او می‌رفتند و به عیسی ایمان می‌آوردند.
\par 12 فردای آن روز چون گروه بسیاری که برای عید آمده بودند شنیدند که عیسی به اورشلیم می‌آید،
\par 13 شاخه های نخل را گرفته به استقبال اوبیرون آمدند و ندا می‌کردند هوشیعانا مبارک بادپادشاه اسرائیل که به اسم خداوند می‌آید.
\par 14 وعیسی کره الاغی یافته، بر آن سوار شد چنانکه مکتوب است
\par 15 که «ای دختر صهیون مترس، اینک پادشاه تو سوار بر کره الاغی می‌آید.»
\par 16 و شاگردانش اولا این چیزها رانفهمیدند، لکن چون عیسی جلال یافت، آنگاه به‌خاطر آوردند که این چیزها درباره او مکتوب است و همچنان با او کرده بودند.
\par 17 و گروهی که با او بودند شهادت دادند که ایلعازر را از قبرخوانده، او را از مردگان برخیزانیده است.
\par 18 وبجهت همین نیز آن گروه او را استقبال کردند، زیرا شنیده بودند که آن معجزه را نموده بود.
\par 19 پس فریسیان به یکدیگر گفتند: «نمی بینید که هیچ نفع نمی برید؟ اینک تمام عالم از پی او رفته‌اند!»
\par 20 و از آن کسانی که در عید بجهت عبادت آمده بودند، بعضی یونانی بودند.
\par 21 ایشان نزدفیلپس که از بیت صیدای جلیل بود آمدند و سوال کرده، گفتند: «ای آقا می‌خواهیم عیسی را ببینیم.»
\par 22 فیلپس آمد و به اندریاس گفت و اندریاس وفیلپس به عیسی گفتند.
\par 23 عیسی در جواب ایشان گفت: «ساعتی رسیده است که پسر انسان جلال یابد.
\par 24 آمین آمین به شما می‌گویم اگر دانه گندم که در زمین می‌افتد نمیرد، تنها ماند لیکن اگربمیرد ثمر بسیار آورد.
\par 25 کسی‌که جان خود رادوست دارد آن را هلاک کند و هر‌که در این جهان جان خود را دشمن دارد تا حیات جاودانی آن رانگاه خواهد داشت.
\par 26 اگر کسی مرا خدمت کند، مرا پیروی بکند و جایی که من می‌باشم آنجاخادم من نیز خواهد بود؛ و هر‌که مرا خدمت کندپدر او را حرمت خواهد داشت.
\par 27 الان جان من مضطرب است و چه بگویم؟ ای پدر مرا از این ساعت رستگار کن. لکن بجهت همین امر تا این ساعت رسیده‌ام.
\par 28 ‌ای پدر اسم خود را جلال بده!» ناگاه صدایی از آسمان در‌رسید که جلال دادم و باز جلال خواهم داد.
\par 29 پس گروهی که حاضر بودند این را شنیده، گفتند: «رعد شد!» ودیگران گفتند: «فرشته‌ای با او تکلم کرد!»
\par 30 عیسی در جواب گفت: «این صدا از برای من نیامد، بلکه بجهت شما.
\par 31 الحال داوری این جهان است و الان رئیس این جهان بیرون افکنده می‌شود.
\par 32 و من اگر از زمین بلند کرده شوم، همه را به سوی خود خواهم کشید.»
\par 33 و این را گفت کنایه از آن قسم موت که می‌بایست بمیرد.
\par 34 پس همه به او جواب دادند: «ما از تورات شنیده‌ایم که مسیح تا به ابد باقی می‌ماند. پس چگونه تو می‌گویی که پسر انسان باید بالا کشیده شود؟ کیست این پسر انسان؟»
\par 35 آنگاه عیسی بدیشان گفت: «اندک زمانی نور با شماست. پس مادامی که نور با شماست، راه بروید تا ظلمت شما را فرو نگیرد؛ و کسی‌که در تاریکی راه می‌رود نمی داند به کجا می‌رود.
\par 36 مادامی که نوربا شماست به نور ایمان آورید تا پسران نورگردید.» عیسی چون این را بگفت، رفته خود را ازایشان مخفی ساخت.
\par 37 و با اینکه پیش روی ایشان چنین معجزات بسیار نموده بود، بدو ایمان نیاوردند.
\par 38 تا کلامی که اشعیا نبی گفت به اتمام رسد: «ای خداوندکیست که خبر ما را باور کرد و بازوی خداوند به که آشکار گردید؟»
\par 39 و از آنجهت نتوانستندایمان آورد، زیرا که اشعیا نیز گفت:
\par 40 «چشمان ایشان را کور کرد و دلهای ایشان را سخت ساخت تا به چشمان خود نبینند و به دلهای خود نفهمندو برنگردند تا ایشان را شفا دهم.»
\par 41 این کلام رااشعیا گفت وقتی که جلال او را دید و درباره اوتکلم کرد.
\par 42 لکن با وجود این، بسیاری ازسرداران نیز بدو ایمان آوردند، اما به‌سبب فریسیان اقرار نکردند که مبادا از کنیسه بیرون شوند.
\par 43 زیرا که جلال خلق را بیشتر از جلال خدا دوست می‌داشتند.
\par 44 آنگاه عیسی ندا کرده، گفت: «آنکه به من ایمان آورد، نه به من بلکه به آنکه مرا فرستاده است، ایمان آورده است.
\par 45 و کسی‌که مرا دید فرستنده مرا دیده است.
\par 46 من نوری در جهان آمدم تا هر‌که به من ایمان آورد در ظلمت نماند.
\par 47 و اگر کسی کلام مرا شنید و ایمان نیاورد، من براو داوری نمی کنم زیرا که نیامده‌ام تا جهان راداوری کنم بلکه تا جهان را نجات‌بخشم.
\par 48 هرکه مرا حقیر شمارد و کلام مرا قبول نکند، کسی هست که در حق او داوری کند، همان کلامی که گفتم در روز بازپسین بر او داوری خواهد کرد.
\par 49 زآنرو که من از خود نگفتم، لکن پدری که مرافرستاد، به من فرمان داد که چه بگویم و به چه چیزتکلم کنم.و می‌دانم که فرمان او حیات جاودانی است. پس آنچه من می‌گویم چنانکه پدربمن گفته است، تکلم می‌کنم.»
\par 50 و می‌دانم که فرمان او حیات جاودانی است. پس آنچه من می‌گویم چنانکه پدربمن گفته است، تکلم می‌کنم.»

\chapter{13}

\par 1 و قبل از عید فصح، چون عیسی دانست که ساعت او رسیده است تا از این جهان به‌جانب پدر برود، خاصان خود را که در این جهان محبت می‌نمود، ایشان را تا به آخر محبت نمود.
\par 2 و چون شام می‌خوردند و ابلیس پیش ازآن در دل یهودا پسر شمعون اسخریوطی نهاده بود که او را تسلیم کند،
\par 3 عیسی با اینکه می‌دانست که پدر همه‌چیز را به‌دست او داده است و از نزد خدا آمده و به‌جانب خدا می‌رود،
\par 4 از شام برخاست و جامه خود را بیرون کرد ودستمالی گرفته، به کمر بست.
\par 5 پس آب در لگن ریخته، شروع کرد به شستن پایهای شاگردان وخشکانیدن آنها با دستمالی که بر کمر داشت.
\par 6 پس چون به شمعون پطرس رسید، او به وی گفت: «ای آقا تو پایهای مرا می‌شویی؟»
\par 7 عیسی در جواب وی گفت: «آنچه من می‌کنم الان تونمی دانی، لکن بعد خواهی فهمید.»
\par 8 پطرس به اوگفت: «پایهای مرا هرگز نخواهی شست.» عیسی او را جواب داد: «اگر تو را نشویم تو را با من نصیبی نیست.»
\par 9 شمعون پطرس بدو گفت: «ای آقا نه پایهای مرا و بس، بلکه دستها و سر مرا نیز.»
\par 10 عیسی بدو گفت: «کسی‌که غسل یافت محتاج نیست مگر به شستن پایها، بلکه تمام او پاک است. و شما پاک هستید لکن نه همه.»
\par 11 زیرا که تسلیم‌کننده خود را می‌دانست و از این جهت گفت: «همگی شما پاک نیستید.»
\par 12 و چون پایهای ایشان را شست، رخت خودرا گرفته، باز بنشست و بدیشان گفت: «آیافهمیدید آنچه به شما کردم؟
\par 13 شما مرا استاد وآقا می‌خوانید و خوب می‌گویید زیرا که چنین هستم.
\par 14 پس اگر من که آقا و معلم هستم، پایهای شما را شستم، بر شما نیز واجب است که پایهای یکدیگر را بشویید.
\par 15 زیرا به شما نمونه‌ای دادم تا چنانکه من با شما کردم، شما نیز بکنید.
\par 16 آمین آمین به شما می‌گویم غلام بزرگتر از آقای خودنیست و نه رسول از فرستنده خود.
\par 17 هرگاه این را دانستید، خوشابحال شما اگر آن را به عمل آرید.
\par 18 درباره جمیع شما نمی گویم؛ من آنانی راکه برگزیده‌ام می‌شناسم، لیکن تا کتاب تمام شود"آنکه با من نان می‌خورد، پاشنه خود را بر من بلندکرده است."
\par 19 الان قبل از وقوع به شما می‌گویم تا زمانی که واقع شود باور کنید که من هستم.
\par 20 آمین آمین به شما می‌گویم هر‌که قبول کندکسی را که می‌فرستم، مرا قبول کرده؛ و آنکه مرا قبول کند، فرستنده مرا قبول کرده باشد.»
\par 21 چون عیسی این را گفت، در روح مضطرب گشت و شهادت داده، گفت: «آمین آمین به شمامی گویم که یکی از شما مرا تسلیم خواهد کرد.»
\par 22 پس شاگردان به یکدیگر نگاه می‌کردند وحیران می‌بودند که این را درباره که می‌گوید.
\par 23 ویکی از شاگردان او بود که به سینه عیسی تکیه می‌زد و عیسی او را محبت می‌نمود؛
\par 24 شمعون پطرس بدو اشاره کرد که بپرسد درباره که این راگفت.
\par 25 پس او در آغوش عیسی افتاده، بدوگفت: «خداوندا کدام است؟»
\par 26 عیسی جواب داد: «آن است که من لقمه را فرو برده، بدومی دهم.» پس لقمه را فرو برده، به یهودای اسخریوطی پسر شمعون داد.
\par 27 بعد از لقمه، شیطان در او داخل گشت. آنگاه عیسی وی راگفت، «آنچه می‌کنی، به زودی بکن.»
\par 28 اما این سخن را احدی از مجلسیان نفهمید که برای چه بدو گفت.
\par 29 زیرا که بعضی گمان بردند که چون خریطه نزد یهودا بود، عیسی وی را فرمود تامایحتاج عید را بخرد یا آنکه چیزی به فقرا بدهد.
\par 30 پس او لقمه را گرفته، در ساعت بیرون رفت و شب بود.
\par 31 چون بیرون رفت عیسی گفت: «الان پسر انسان جلال یافت و خدا در او جلال یافت.
\par 32 و اگر خدا در او جلال یافت، هرآینه خدا او را در خود جلال خواهد داد و به زودی اورا جلال خواهد داد.
\par 33 ‌ای فرزندان، اندک زمانی دیگر با شما هستم و مرا طلب خواهید کرد؛ وهمچنان‌که به یهود گفتم جایی که می‌روم شمانمی توانید آمد، الان نیز به شما می‌گویم.
\par 34 به شما حکمی تازه می‌دهم که یکدیگر را محبت نمایید، چنانکه من شما را محبت نمودم تا شمانیز یکدیگر را محبت نمایید.
\par 35 به همین همه خواهند فهمید که شاگرد من هستید اگر محبت یکدیگر را داشته باشید.»
\par 36 شمعون پطرس به وی گفت: «ای آقا کجا می‌روی؟» عیسی جواب داد: «جایی که می‌روم، الان نمی توانی از عقب من بیایی و لکن در آخر از عقب من خواهی آمد.»
\par 37 پطرس بدو گفت: «ای آقا برای چه الان نتوانم از عقب تو بیایم؟ جان خود را در راه تو خواهم نهاد.»عیسی به او جواب داد: «آیا جان خود رادر راه من می‌نهی؟ آمین آمین به تو می‌گویم تا سه مرتبه مرا انکار نکرده باشی، خروس بانگ نخواهد زد.
\par 38 عیسی به او جواب داد: «آیا جان خود رادر راه من می‌نهی؟ آمین آمین به تو می‌گویم تا سه مرتبه مرا انکار نکرده باشی، خروس بانگ نخواهد زد.

\chapter{14}

\par 1 «دل شما مضطرب نشود! به خدا ایمان آورید به من نیز ایمان آورید.
\par 2 در خانه پدر من منزل بسیار است والا به شما می‌گفتم. می‌روم تا برای شما مکانی حاضر کنم،
\par 3 و اگربروم و از برای شما مکانی حاضر کنم، بازمی آیم و شما را برداشته با خود خواهم برد تا جایی که من می‌باشم شما نیز باشید.
\par 4 و جایی که من می‌روم می‌دانید و راه را می‌دانید.»
\par 5 توما بدو گفت: «ای آقا نمی دانیم کجا می‌روی. پس چگونه راه را توانیم دانست؟»
\par 6 عیسی بدو گفت: «من راه و راستی و حیات هستم. هیچ‌کس نزد پدر جز به وسیله من نمی آید.
\par 7 اگر مرا می‌شناختید، پدر مرانیز می‌شناختید و بعد از این او را می‌شناسید و اورا دیده‌اید.»
\par 8 فیلپس به وی گفت: «ای آقا پدر رابه ما نشان ده که ما را کافی است.»
\par 9 عیسی بدوگفت: «ای فیلیپس در این مدت با شما بوده‌ام، آیامرا نشناخته‌ای؟ کسی‌که مرا دید، پدر را دیده است. پس چگونه تو می‌گویی پدر را به ما نشان ده؟
\par 10 آیا باور نمی کنی که من در پدر هستم و پدردر من است؟ سخنهایی که من به شما می‌گویم ازخود نمی گویم، لکن پدری که در من ساکن است، او این اعمال را می‌کند.
\par 11 مرا تصدیق کنید که من در پدر هستم و پدر در من است، والا مرا به‌سبب آن اعمال تصدیق کنید.
\par 12 آمین آمین به شمامی گویم هر‌که به من ایمان آرد، کارهایی را که من می‌کنم او نیز خواهد کرد و بزرگتر از اینها نیزخواهد کرد، زیرا که من نزد پدر می‌روم.
\par 13 «و هر چیزی را که به اسم من سوال کنید به‌جا خواهم آورد تا پدر در پسر جلال یابد.
\par 14 اگرچیزی به اسم من طلب کنید من آن را به‌جا خواهم آورد.
\par 15 اگر مرا دوست دارید، احکام مرا نگاه دارید.
\par 16 و من از پدر سوال می‌کنم و تسلی دهنده‌ای دیگر به شما عطا خواهد کرد تا همیشه باشما بماند،
\par 17 یعنی روح راستی که جهان نمی تواند او را قبول کند زیرا که او را نمی بیند و نمی شناسد و اما شما او را می‌شناسید، زیرا که باشما می‌ماند و در شما خواهد بود.
\par 18 «شما را یتیم نمی گذارم نزد شما می‌آیم.
\par 19 بعد از اندک زمانی جهان دیگر مرا نمی بیند واما شما مرا می‌بینید و از این جهت که من زنده‌ام، شما هم خواهید زیست.
\par 20 و در آن روز شماخواهید دانست که من در پدر هستم و شما در من و من در شما.
\par 21 هر‌که احکام مرا دارد و آنها راحفظ کند، آن است که مرا محبت می‌نماید؛ وآنکه مرا محبت می‌نماید، پدر من او را محبت خواهد نمود و من او را محبت خواهم نمود وخود را به او ظاهر خواهم ساخت.»
\par 22 یهودا، نه آن اسخریوطی، به وی گفت: «ای آقا چگونه می‌خواهی خود را بما بنمایی و نه بر جهان؟»
\par 23 عیسی در جواب او گفت: «اگر کسی مرا محبت نماید، کلام مرا نگاه خواهد داشت و پدرم او رامحبت خواهد نمود و به سوی او آمده، نزد وی مسکن خواهیم گرفت.
\par 24 و آنکه مرا محبت ننماید، کلام مرا حفظ نمی کند؛ و کلامی که می‌شنوید از من نیست بلکه از پدری است که مرافرستاد.
\par 25 این سخنان را به شما گفتم وقتی که باشما بودم.
\par 26 لیکن تسلی دهنده یعنی روح‌القدس که پدر او را به اسم من می‌فرستد، اوهمه‌چیز را به شما تعلیم خواهد داد و آنچه به شما گفتم به یاد شما خواهد آورد.
\par 27 سلامتی برای شما می‌گذارم، سلامتی خود را به شمامی دهم. نه‌چنانکه جهان می‌دهد، من به شمامی دهم. دل شما مضطرب و هراسان نباشد.
\par 28 شنیده‌اید که من به شما گفتم می‌روم و نزد شمامی آیم. اگر مرا محبت می‌نمودید، خوشحال می‌گشتید که گفتم نزد پدر می‌روم، زیرا که پدربزرگتر از من است.
\par 29 و الان قبل از وقوع به شما گفتم تا وقتی که واقع گردد ایمان آورید.
\par 30 بعد ازاین بسیار با شما نخواهم گفت، زیرا که رئیس این جهان می‌آید و در من چیزی ندارد.لیکن تاجهان بداند که پدر را محبت می‌نمایم، چنانکه پدر به من حکم کرد همانطور می‌کنم. برخیزید ازاینجا برویم.
\par 31 لیکن تاجهان بداند که پدر را محبت می‌نمایم، چنانکه پدر به من حکم کرد همانطور می‌کنم. برخیزید ازاینجا برویم.

\chapter{15}

\par 1 «من تاک حقیقی هستم و پدر من باغبان است.
\par 2 هر شاخه‌ای در من که میوه نیاورد، آن را دور می‌سازد و هر‌چه میوه آرد آن راپاک می‌کند تا بیشتر میوه آورد.
\par 3 الحال شما به‌سبب کلامی که به شما گفته‌ام پاک هستید.
\par 4 در من بمانید و من در شما. همچنانکه شاخه از خودنمی تواند میوه آورد اگر در تاک نماند، همچنین شما نیز اگر در من نمانید.
\par 5 من تاک هستم و شماشاخه‌ها. آنکه در من می‌ماند و من در او، میوه بسیار می‌آورد زیرا که جدا از من هیچ نمی توانیدکرد.
\par 6 اگر کسی در من نماند، مثل شاخه بیرون انداخته می‌شود و می‌خشکد و آنها را جمع کرده، در آتش می‌اندازند و سوخته می‌شود.
\par 7 اگر در من بمانید و کلام من در شما بماند، آنچه خواهیدبطلبید که برای شما خواهد شد.
\par 8 جلال پدر من آشکارا می‌شود به اینکه میوه بسیار بیاورید وشاگرد من بشوید.
\par 9 همچنان‌که پدر مرا محبت نمود، من نیز شما را محبت نمودم؛ در محبت من بمانید.
\par 10 اگر احکام مرا نگاه دارید، در محبت من خواهید ماند، چنانکه من احکام پدر خود را نگاه داشته‌ام و در محبت او می‌مانم.
\par 11 این را به شماگفتم تا خوشی من در شما باشد و شادی شماکامل گردد.
\par 12 «این است حکم من که یکدیگر را محبت نمایید، همچنان‌که شما را محبت نمودم.
\par 13 کسی محبت بزرگتر از این ندارد که جان خودرا بجهت دوستان خود بدهد.
\par 14 شما دوست من هستید اگر آنچه به شما حکم می‌کنم به‌جا آرید.
\par 15 دیگر شما را بنده نمی خوانم زیرا که بنده آنچه آقایش می‌کند نمی داند؛ لکن شما را دوست خوانده‌ام زیرا که هرچه از پدر شنیده‌ام به شمابیان کردم.
\par 16 شما مرا برنگزیدید، بلکه من شما رابرگزیدم و شما را مقرر کردم تا شما بروید و میوه آورید و میوه شما بماند تا هر‌چه از پدر به اسم من طلب کنید به شما عطا کند.
\par 17 به این چیزها شما را حکم می‌کنم تایکدیگر را محبت نمایید.
\par 18 «اگر جهان شما رادشمن دارد، بدانید که پیشتر از شما مرا دشمن داشته است.
\par 19 اگر از جهان می‌بودید، جهان خاصان خود را دوست می‌داشت. لکن چونکه ازجهان نیستید بلکه من شما را از جهان برگزیده‌ام، از این سبب جهان با شما دشمنی می‌کند.
\par 20 به‌خاطر آرید کلامی را که به شما گفتم: غلام بزرگتراز آقای خود نیست. اگر مرا زحمت دادند، شما رانیز زحمت خواهند داد، اگر کلام مرا نگاه داشتند، کلام شما را هم نگاه خواهند داشت.
\par 21 لکن بجهت اسم من جمیع این کارها را به شما خواهندکرد زیرا که فرستنده مرا نمی شناسند.
\par 22 اگرنیامده بودم و به ایشان تکلم نکرده، گناه نمی داشتند؛ و اما الان عذری برای گناه خودندارند.
\par 23 هر‌که مرا دشمن دارد پدر مرا نیز دشمن دارد.
\par 24 و اگر در میان ایشان کارهایی نکرده بودم که غیر از من کسی هرگز نکرده بود، گناه نمی داشتند. ولیکن اکنون دیدند و دشمن داشتند مرا و پدر مرا نیز.
\par 25 بلکه تا تمام شودکلامی که در شریعت ایشان مکتوب است که "مرابی سبب دشمن داشتند."
\par 26 لیکن چون تسلی دهنده که او را از جانب پدر نزد شما می‌فرستم آید، یعنی روح راستی که از پدر صادر می‌گردد، او بر من شهادت خواهد داد.و شما نیزشهادت خواهید داد زیرا که از ابتدا با من بوده‌اید.
\par 27 و شما نیزشهادت خواهید داد زیرا که از ابتدا با من بوده‌اید.

\chapter{16}

\par 1 این را به شما گفتم تا لغزش نخورید.
\par 2 شما را از کنایس بیرون خواهندنمود؛ بلکه ساعتی می‌آید که هر‌که شما را بکشد، گمان برد که خدا را خدمت می‌کند.
\par 3 و این کارهارا با شما خواهند کرد، بجهت آنکه نه پدر راشناخته‌اند و نه مرا.
\par 4 لیکن این را به شما گفتم تاوقتی که ساعت آید به‌خاطر آورید که من به شماگفتم. و این را از اول به شما نگفتم، زیرا که با شمابودم.
\par 5 «اما الان نزد فرستنده خود می‌روم و کسی ازشما از من نمی پرسد به کجا می‌روی.
\par 6 ولیکن چون این را به شما گفتم، دل شما از غم پر شده است.
\par 7 و من به شما راست می‌گویم که رفتن من برای شما مفید است، زیرا اگر نروم تسلی دهنده نزد شما نخواهد آمد. اما اگر بروم او را نزد شمامی فرستم.
\par 8 و چون او آید، جهان را بر گناه وعدالت و داوری ملزم خواهد نمود.
\par 9 اما بر گناه، زیرا که به من ایمان نمی آورند.
\par 10 و اما بر عدالت، از آن سبب که نزد پدر خود می‌روم و دیگر مرانخواهید دید.
\par 11 و اما بر داوری، از آنرو که بررئیس این جهان حکم شده است.
\par 12 «و بسیار چیزهای دیگر نیز دارم به شمابگویم، لکن الان طاقت تحمل آنها را ندارید.
\par 13 ولیکن چون او یعنی روح راستی آید، شما را به جمیع راستی هدایت خواهد کرد زیرا که از خودتکلم نمی کند بلکه به آنچه شنیده است سخن خواهد گفت و از امور آینده به شما خبر‌خواهدداد.
\par 14 او مرا جلال خواهد داد زیرا که از آنچه آن من است خواهد گرفت و به شما خبر‌خواهد داد.
\par 15 هر‌چه از آن پدر است، از آن من است. از این جهت گفتم که از آنچه آن من است، می‌گیرد و به شما خبر‌خواهد داد.
\par 16 «بعد از اندکی مرانخواهید دید و بعد از اندکی باز مرا خواهید دیدزیرا که نزد پدر می‌روم.»
\par 17 آنگاه بعضی از شاگردانش به یکدیگرگفتند: «چه چیز است اینکه به ما می‌گوید که اندکی مرا نخواهید دید و بعد از اندکی باز مراخواهید دید و زیرا که نزد پدر می‌روم؟»
\par 18 پس گفتند: «چه چیز است این اندکی که می‌گوید؟ نمی دانیم چه می‌گوید.»
\par 19 عیسی چون دانست که می‌خواهند از او سوال کنند، بدیشان گفت: «آیا در میان خود از این سوال می‌کنید که گفتم اندکی دیگر مرا نخواهید دید پس بعد از اندکی بازمرا خواهید دید.
\par 20 آمین آمین به شما می‌گویم که شما گریه و زاری خواهید کرد و جهان شادی خواهد نمود. شما محزون می‌شوید لکن حزن شما به خوشی مبدل خواهد شد.
\par 21 زن در حین زاییدن محزون می‌شود، زیرا که ساعت او رسیده است. و لیکن چون طفل را زایید، آن زحمت رادیگر یاد نمی آورد به‌سبب خوشی از اینکه انسانی در جهان تولد یافت.
\par 22 پس شما همچنین الان محزون می‌باشید، لکن باز شما را خواهم دیدو دل شما خوش خواهد گشت و هیچ‌کس آن خوشی را از شما نخواهد گرفت.
\par 23 و در آن روزچیزی از من سوال نخواهید کرد. آمین آمین به شما می‌گویم که هر‌آنچه از پدر به اسم من طلب کنید به شما عطا خواهد کرد.
\par 24 تا کنون به اسم من چیزی طلب نکردید، بطلبید تا بیابید و خوشی شما کامل گردد.
\par 25 این چیزها را به مثلها به شماگفتم، لکن ساعتی می‌آید که دیگر به مثلها به شماحرف نمی زنم بلکه از پدر به شما آشکارا خبرخواهم داد.
\par 26 «در آن روز به اسم من طلب خواهید کرد وبه شما نمی گویم که من بجهت شما از پدر سوال می‌کنم،
\par 27 زیرا خود پدر شما را دوست می‌دارد، چونکه شما مرا دوست داشتید و ایمان آوردیدکه من از نزد خدا بیرون آمدم.
\par 28 از نزد پدر بیرون آمدم و در جهان وارد شدم، و باز جهان را گذارده، نزد پدر می‌روم.»
\par 29 شاگردانش بدو گفتند: «هان اکنون علانیه سخن می‌گویی و هیچ مثل نمی گویی.
\par 30 الان دانستیم که همه‌چیز رامی دانی و لازم نیست که کسی از تو بپرسد. بدین جهت باور می‌کنیم که از خدا بیرون آمدی.»
\par 31 عیسی به ایشان جواب داد: «آیا الان باورمی کنید؟
\par 32 اینک ساعتی می‌آید بلکه الان آمده است که متفرق خواهید شد هریکی به نزدخاصان خود و مرا تنها خواهید گذارد. لیکن تنهانیستم زیرا که پدر با من است.بدین چیزها به شما تکلم کردم تا در من سلامتی داشته باشید. در جهان برای شما زحمت خواهد شد. و لکن خاطرجمع دارید زیرا که من بر جهان غالب شده‌ام.»
\par 33 بدین چیزها به شما تکلم کردم تا در من سلامتی داشته باشید. در جهان برای شما زحمت خواهد شد. و لکن خاطرجمع دارید زیرا که من بر جهان غالب شده‌ام.»

\chapter{17}

\par 1 عیسی چون این را گفت، چشمان خودرا به طرف آسمان بلند کرده، گفت: «ای پدر ساعت رسیده است. پسر خود را جلال بده تاپسرت نیز تو را جلال دهد.
\par 2 همچنان‌که او را برهر بشری قدرت داده‌ای تا هر‌چه بدو داده‌ای به آنها حیات جاودانی بخشد.
\par 3 و حیات جاودانی‌این است که تو را خدای واحد حقیقی و عیسی مسیح را که فرستادی بشناسند.
\par 4 من بر روی زمین تو را جلال دادم و کاری را که به من سپردی تابکنم، به‌کمال رسانیدم.
\par 5 و الان تو‌ای پدر مرا نزدخود جلال ده، به همان جلالی که قبل از آفرینش جهان نزد تو داشتم.
\par 6 «اسم تو را به آن مردمانی که از جهان به من عطا کردی ظاهر ساختم. از آن تو بودند و ایشان را به من دادی و کلام تو را نگاه داشتند.
\par 7 و الان دانستند آنچه به من داده‌ای از نزد تو می‌باشد.
\par 8 زیرا کلامی را که به من سپردی، بدیشان سپردم و ایشان قبول کردند و از روی یقین دانستند که ازنزد تو بیرون آمدم و ایمان آوردند که تو مرافرستادی.
\par 9 من بجهت اینها سوال می‌کنم و برای جهان سوال نمی کنم، بلکه از برای کسانی که به من داده‌ای، زیرا که از آن تو می‌باشند.
\par 10 و آنچه ازآن من است از آن تو است و آنچه از آن تو است از آن من است و در آنها جلال یافته‌ام.
\par 11 بعد از این در جهان نیستم اما اینها در جهان هستند و من نزدتو می‌آیم. ای پدر قدوس اینها را که به من داده‌ای، به اسم خود نگاه دار تا یکی باشندچنانکه ما هستیم.
\par 12 مادامی که با ایشان در جهان بودم، من ایشان را به اسم تو نگاه داشتم، و هر کس را که به من داده‌ای حفظ نمودم که یکی از ایشان هلاک نشد، مگر پسر هلاکت تا کتاب تمام شود.
\par 13 و اما الان نزد تو می‌آیم. و این را در جهان می‌گویم تا خوشی مرا در خود کامل داشته باشند.
\par 14 من کلام تو را به ایشان دادم و جهان ایشان رادشمن داشت زیرا که از جهان نیستند، همچنان‌که من نیز از جهان نیستم.
\par 15 خواهش نمی کنم که ایشان را از جهان ببری، بلکه تا ایشان را از شریرنگاه داری.
\par 16 ایشان از جهان نیستند چنانکه من از جهان نمی باشم.
\par 17 ایشان را به راستی خودتقدیس نما. کلام تو راستی است.
\par 18 همچنان‌که مرا در جهان فرستادی، من نیز ایشان را در جهان فرستادم.
\par 19 و بجهت ایشان من خود را تقدیس می‌کنم تا ایشان نیز در راستی، تقدیس کرده شوند.
\par 20 «و نه برای اینها فقط سوال می‌کنم، بلکه برای آنها نیز که به وسیله کلام ایشان به من ایمان خواهند‌آورد.
\par 21 تا همه یک گردند چنانکه تو‌ای پدر، در من هستی و من در تو، تا ایشان نیز در مایک باشند تا جهان ایمان آرد که تو مرا فرستادی.
\par 22 و من جلالی را که به من دادی به ایشان دادم تا یک باشند چنانکه ما یک هستیم.
\par 23 من در ایشان و تو در من، تا در یکی کامل گردند و تا جهان بداندکه تو مرا فرستادی و ایشان را محبت نمودی چنانکه مرا محبت نمودی.
\par 24 ‌ای پدر می‌خواهم آنانی که به من داده‌ای با من باشند در جایی که من می‌باشم تا جلال مرا که به من داده‌ای ببینند، زیراکه مرا پیش از بنای جهان محبت نمودی.
\par 25 ‌ای پدر عادل، جهان تو را نشناخت، اما من تو راشناختم؛ و اینها شناخته‌اند که تو مرا فرستادی.و اسم تو را به ایشان شناسانیدم و خواهم شناسانید تا آن محبتی که به من نموده‌ای در ایشان باشد و من نیز در ایشان باشم.»
\par 26 و اسم تو را به ایشان شناسانیدم و خواهم شناسانید تا آن محبتی که به من نموده‌ای در ایشان باشد و من نیز در ایشان باشم.»

\chapter{18}

\par 1 چون عیسی این را گفت، با شاگردان خود به آن طرف وادی قدرون رفت ودر آنجا باغی بود که با شاگردان خود به آن درآمد.
\par 2 و یهودا که تسلیم‌کننده وی بود، آن موضع را می‌دانست، چونکه عیسی در آنجا با شاگردان خود بارها انجمن می‌نمود.
\par 3 پس یهودا لشکریان و خادمان از نزد روسای کهنه و فریسیان برداشته، با چراغها و مشعلها و اسلحه به آنجا آمد.
\par 4 آنگاه عیسی با اینکه آگاه بود از آنچه می‌بایست بر اوواقع شود، بیرون آمده، به ایشان گفت: «که رامی طلبید؟»
\par 5 به او جواب دادند: «عیسی ناصری را!» عیسی بدیشان گفت: «من هستم!» و یهودا که تسلیم‌کننده او بود نیز با ایشان ایستاده بود.
\par 6 پس چون بدیشان گفت: «من هستم، » برگشته، بر زمین افتادند.
\par 7 او باز از ایشان سوال کرد: «که را می طلبید؟» گفتند: «عیسی ناصری را!»
\par 8 عیسی جواب داد: «به شما گفتم من هستم. پس اگر مرامی خواهید، اینها را بگذارید بروند.»
\par 9 تا آن سخنی که گفته بود تمام گردد که «از آنانی که به من داده‌ای یکی را گم نکرده‌ام.»
\par 10 آنگاه شمعون پطرس شمشیری که داشت کشیده، به غلام رئیس کهنه که ملوک نام داشت زده، گوش راستش را برید.
\par 11 عیسی به پطرس گفت: «شمشیر خود را غلاف کن. آیا جامی را که پدر به من داده است ننوشم؟»
\par 12 انگاه سربازان و سرتیبان و خادمان یهود، عیسی را گرفته، او را بستند.
\par 13 و اول او را نزدحنا، پدر زن قیافا که در همان سال رئیس کهنه بود، آوردند.
\par 14 و قیافا همان بود که به یهود اشاره کرده بود که «بهتر است یک شخص در راه قوم بمیرد.»
\par 15 اما شمعون پطرس و شاگردی دیگر ازعقب عیسی روانه شدند، و چون آن شاگرد نزدرئیس کهنه معروف بود، با عیسی داخل خانه رئیس کهنه شد.
\par 16 اما پطرس بیرون در ایستاده بود. پس آن شاگرد دیگر که آشنای رئیس کهنه بود، بیرون آمده، با دربان گفتگو کرد و پطرس را به اندرون برد.
\par 17 آنگاه آن کنیزی که دربان بود، به پطرس گفت: «آیا تو نیز از شاگردان این شخص نیستی؟» گفت: «نیستم.»
\par 18 و غلامان و خدام آتش افروخته، ایستاده بودند و خود را گرم می‌کردند چونکه هوا سرد بود؛ و پطرس نیز باایشان خود را گرم می‌کرد.
\par 19 پس رئیس کهنه از عیسی درباره شاگردان و تعلیم او پرسید.
\par 20 عیسی به او جواب داد که «من به جهان آشکارا سخن گفته‌ام. من هر وقت درکنیسه و در هیکل، جایی که همه یهودیان پیوسته جمع می‌شدند، تعلیم می‌دادم و در خفا چیزی نگفته‌ام.
\par 21 چرا از من سوال می‌کنی؟ از کسانی که شنیده‌اند بپرس که چه چیز بدیشان گفتم. اینک ایشان می‌دانند آنچه من گفتم.»
\par 22 و چون این راگفت، یکی از خادمان که در آنجا ایستاده بود، طپانچه بر عیسی زده، گفت: «آیا به رئیس کهنه چنین جواب می‌دهی؟»
\par 23 عیسی بدو جواب داد: «اگر بد گفتم، به بدی شهادت ده؛ و اگرخوب، برای چه مرا می‌زنی؟»
\par 24 پس حنا او رابسته، به نزد قیافا رئیس کهنه فرستاد.
\par 25 و شمعون پطرس ایستاده، خود را گرم می‌کرد. بعضی بدو گفتند: «آیا تو نیز از شاگردان او نیستی؟» او انکار کرده، گفت: «نیستم!»
\par 26 پس یکی از غلامان رئیس کهنه که از خویشان آن کس بود که پطرس گوشش را بریده بود، گفت: «مگرمن تو را با او در باغ ندیدم؟»
\par 27 پطرس باز انکارکرد که در حال خروس بانگ زد.
\par 28 بعد عیسی را از نزد قیافا به دیوانخانه آوردند و صبح بود و ایشان داخل دیوانخانه نشدند مبادا نجس بشوند بلکه تا فصح را بخورند.
\par 29 پس پیلاطس به نزد ایشان بیرون آمده، گفت: «چه دعوی بر این شخص دارید؟»
\par 30 در جواب او گفتند: «اگر او بدکار نمی بود، به تو تسلیم نمی کردیم.»
\par 31 پیلاطس بدیشان گفت: «شما اورا بگیرید و موافق شریعت خود بر او حکم نمایید.» یهودیان به وی گفتند: «بر ما جایز نیست که کسی را بکشیم.»
\par 32 تا قول عیسی تمام گرددکه گفته بود، اشاره به آن قسم موت که باید بمیرد.
\par 33 پس پیلاطس باز داخل دیوانخانه شد وعیسی را طلبیده، به او گفت: «آیا تو پادشاه یهودهستی؟»
\par 34 عیسی به او جواب داد: «آیا تو این رااز خود می‌گویی یا دیگران درباره من به توگفتند؟»
\par 35 پیلاطس جواب داد: «مگر من یهودهستم؟ امت تو و روسای کهنه تو را به من تسلیم کردند. چه کرده‌ای؟»
\par 36 عیسی جواب داد که «پادشاهی من از این جهان نیست. اگر پادشاهی من از این جهان می‌بود، خدام من جنگ می‌کردند تا به یهود تسلیم نشوم. لیکن اکنون پادشاهی من از این جهان نیست.»
\par 37 پیلاطس به او گفت: «مگر توپادشاه هستی؟» عیسی جواب داد: «تو می‌گویی که من پادشاه هستم. از این جهت من متولد شدم وبجهت این در جهان آمدم تا به راستی شهادت دهم، و هر‌که از راستی است سخن مرا می‌شنود.»
\par 38 پیلاطس به او گفت: «راستی چیست؟» و چون این را بگفت، باز به نزد یهودیان بیرون شده، به ایشان گفت: «من در این شخص هیچ عیبی نیافتم.
\par 39 و قانون شما این است که در عید فصح بجهت شما یک نفر آزاد کنم. پس آیا می‌خواهید بجهت شما پادشاه یهود را آزاد کنم؟»باز همه فریاد برآورده، گفتند: «او را نی بلکه برابا را.» وبرابا دزد بود.
\par 40 باز همه فریاد برآورده، گفتند: «او را نی بلکه برابا را.» وبرابا دزد بود.

\chapter{19}

\par 1 پس پیلاطس عیسی را گرفته، تازیانه زد.
\par 2 و لشکریان تاجی از خار بافته برسرش گذاردند و جامه ارغوانی بدو پوشانیدند
\par 3 ومی گفتند: «سلام‌ای پادشاه یهود!» و طپانچه بدومی زدند.
\par 4 باز پیلاطس بیرون آمده، به ایشان گفت: «اینک او را نزد شما بیرون آوردم تا بدانیدکه در او هیچ عیبی نیافتم.»
\par 5 آنگاه عیسی با تاجی از خار و لباس ارغوانی بیرون آمد. پیلاطس بدیشان گفت: «اینک آن انسان.»
\par 6 و چون روسای کهنه و خدام او را دیدند، فریاد برآورده، گفتند: «صلیبش کن! صلیبش کن!» پیلاطس بدیشان گفت: «شما او را گرفته، مصلوبش سازید زیرا که من در او عیبی نیافتم.»
\par 7 یهودیان بدو جواب دادند که «ما شریعتی داریم و موافق شریعت ماواجب است که بمیرد زیرا خود را پسر خداساخته است.»
\par 8 پس چون پیلاطس این را شنید، خوف بر اوزیاده مستولی گشت.
\par 9 باز داخل دیوانخانه شده، به عیسی گفت: «تو از کجایی؟» اما عیسی بدوهیچ جواب نداد.
\par 10 پیلاطس بدو گفت: «آیا به من سخن نمی گویی؟ نمی دانی که قدرت دارم تو راصلیب کنم و قدرت دارم آزادت نمایم؟»
\par 11 عیسی جواب داد: «هیچ قدرت بر من نمی داشتی اگر از بالا به تو داده نمی شد. و از این جهت آن کس که مرا به تو تسلیم کرد، گناه بزرگتردارد.»
\par 12 و از آن وقت پیلاطس خواست او راآزاد نماید، لیکن یهودیان فریاد برآورده، می‌گفتند که «اگر این شخص را رها کنی، دوست قیصر نیستی. هر‌که خود را پادشاه سازد، برخلاف قیصر سخن گوید.»
\par 13 پس چون پیلاطس این را شنید، عیسی رابیرون آورده، بر مسند حکومت، در موضعی که به بلاط و به عبرانی جباتا گفته می‌شد، نشست.
\par 14 ووقت تهیه فصح و قریب به ساعت ششم بود. پس به یهودیان گفت: «اینک پادشاه شما.»
\par 15 ایشان فریاد زدند: «او را بر دار، بر دار! صلیبش کن!» پیلاطس به ایشان گفت: «آیا پادشاه شما رامصلوب کنم؟» روسای کهنه جواب دادند که «غیر از قیصر پادشاهی نداریم.»
\par 16 آنگاه او را بدیشان تسلیم کرد تا مصلوب شود. پس عیسی را گرفته بردند
\par 17 و صلیب خودرا برداشته، بیرون رفت به موضعی که به جمجمه مسمی بود و به عبرانی آن را جلجتا می‌گفتند.
\par 18 او را در آنجا صلیب نمودند و دو نفر دیگررا از این طرف و آن طرف و عیسی را در میان.
\par 19 وپیلاطس تقصیرنامه‌ای نوشته، بر صلیب گذارد؛ ونوشته این بود: «عیسی ناصری پادشاه یهود.»
\par 20 واین تقصیر نامه را بسیاری از یهود خواندند، زیراآن مکانی که عیسی را صلیب کردند، نزدیک شهربود و آن را به زبان عبرانی و یونانی و لاتینی نوشته بودند.
\par 21 پس روسای کهنه یهود به پیلاطس گفتند: «منویس پادشاه یهود، بلکه که او گفت منم پادشاه یهود.»
\par 22 پیلاطس جواب داد: «آنچه نوشتم، نوشتم.»
\par 23 پس لشکریان چون عیسی را صلیب کردند، جامه های او را برداشته، چهار قسمت کردند، هرسپاهی را یک قسمت؛ و پیراهن را نیز، اما پیراهن درز نداشت، بلکه تمام از بالا بافته شده بود.
\par 24 پس به یکدیگر گفتند: «این را پاره نکنیم، بلکه قرعه بر آن بیندازیم تا از آن که شود.» تا تمام گردد کتاب که می‌گوید: «در میان خود جامه های مرا تقسیم کردند و بر لباس من قرعه افکندند.» پس لشکریان چنین کردند.
\par 25 و پای صلیب عیسی، مادر او و خواهر مادرش مریم زن، کلوپا ومریم مجدلیه ایستاده بودند.
\par 26 چون عیسی مادرخود را با آن شاگردی که دوست می‌داشت ایستاده دید، به مادر خود گفت: «ای زن، اینک پسر تو.»
\par 27 و به آن شاگرد گفت: «اینک مادر تو.» و در همان ساعت آن شاگرد او را به خانه خودبرد.
\par 28 و بعد چون عیسی دید که همه‌چیز به انجام رسیده است تا کتاب تمام شود، گفت: «تشنه‌ام.»
\par 29 و در آنجا ظرفی پر از سرکه گذارده بود. پس اسفنجی را از سرکه پر ساخته، و بر زوفا گذارده، نزدیک دهان او بردند.
\par 30 چون عیسی سرکه راگرفت، گفت: «تمام شد.» و سر خود را پایین آورده، جان بداد.
\par 31 پس یهودیان تا بدنها در روز سبت برصلیب نماند، چونکه روز تهیه بود و آن سبت، روز بزرگ بود، از پیلاطس درخواست کردند که ساق پایهای ایشان را بشکنند و پایین بیاورند.
\par 32 آنگاه لشکریان آمدند و ساقهای آن اول ودیگری را که با او صلیب شده بودند، شکستند.
\par 33 اما چون نزد عیسی آمدند و دیدند که پیش ازآن مرده است، ساقهای او را نشکستند.
\par 34 لکن یکی از لشکریان به پهلوی او نیزه‌ای زد که در آن ساعت خون و آب بیرون آمد.
\par 35 و آن کسی‌که دید شهادت داد و شهادت او راست است و اومی داند که راست می‌گوید تا شما نیز ایمان آورید.
\par 36 زیرا که این واقع شد تا کتاب تمام شودکه می‌گوید: «استخوانی از او شکسته نخواهدشد.»
\par 37 و باز کتاب دیگر می‌گوید: «آن کسی راکه نیزه زدند خواهند نگریست.»
\par 38 و بعد از این، یوسف که از اهل رامه وشاگرد عیسی بود، لیکن مخفی به‌سبب ترس یهود، از پیلاطس خواهش کرد که جسد عیسی رابردارد. پیلاطس اذن داد. پس آمده، بدن عیسی رابرداشت.
\par 39 و نیقودیموس نیز که اول در شب نزدعیسی آمده بود، مر مخلوط با عود قریب به صدرطل با خود آورد.
\par 40 آنگاه بدن عیسی را بداشته، در کفن با حنوط به رسم تکفین یهود پیچیدند.
\par 41 و در موضعی که مصلوب شد باغی بود و درباغ، قبر تازه‌ای که هرگز هیچ‌کس در آن دفن نشده بود.پس به‌سبب تهیه یهود، عیسی را در آنجاگذاردند، چونکه آن قبر نزدیک بود.
\par 42 پس به‌سبب تهیه یهود، عیسی را در آنجاگذاردند، چونکه آن قبر نزدیک بود.

\chapter{20}

\par 1 بامدادان در اول هفته، وقتی که هنوز تاریک بود، مریم مجدلیه به‌سر قبر‌آمدو دید که سنگ از قبر برداشته شده است.
\par 2 پس دوان دوان نزد شمعون پطرس و آن شاگرد دیگرکه عیسی او را دوست می‌داشت آمده، به ایشان گفت: «خداوند را از قبر برده‌اند و نمی دانیم او راکجا گذارده‌اند.»
\par 3 آنگاه پطرس و آن شاگرد دیگربیرون شده، به‌جانب قبر رفتند.
\par 4 و هر دو با هم می‌دویدند، اما آن شاگرد دیگر از پطرس پیش افتاده، اول به قبر رسید،
\par 5 و خم شده، کفن راگذاشته دید، لیکن داخل نشد.
\par 6 بعد شمعون پطرس نیز از عقب او آمد و داخل قبرگشته، کفن را گذاشته دید،
\par 7 و دستمالی را که بر سر او بود، نه با کفن نهاده، بلکه در جای علیحده پیچیده.
\par 8 پس آن شاگرد دیگر که اول به‌سر قبر‌آمده بود نیزداخل شده، دید و ایمان آورد.
\par 9 زیرا هنوز کتاب را نفهمیده بودند که باید او از مردگان برخیزد.
\par 10 پس آن دو شاگرد به مکان خود برگشتند.
\par 11 اما مریم بیرون قبر، گریان ایستاده بود وچون می‌گریست به سوی قبر خم شده،
\par 12 دوفرشته را که لباس سفید در بر داشتند، یکی به طرف سر و دیگری به‌جانب قدم، در جایی که بدن عیسی گذارده بود، نشسته دید.
\par 13 ایشان بدوگفتند: «ای زن برای چه گریانی؟» بدیشان گفت: «خداوند مرا برده‌اند و نمی دانم او را کجاگذارده‌اند.»
\par 14 چون این را گفت، به عقب ملتفت شده، عیسی را ایستاده دید لیکن نشناخت که عیسی است.
\par 15 عیسی بدو گفت: «ای زن برای چه گریانی؟ که را می‌طلبی؟» چون او گمان کردکه باغبان است، بدو گفت: «ای آقا اگر تو او رابرداشته‌ای، به من بگو او را کجا گذارده‌ای تا من اورا بردارم.»
\par 16 عیسی بدو گفت: «ای مریم!» اوبرگشته، گفت: «ربونی (یعنی‌ای معلم ).»
\par 17 عیسی بدو گفت: «مرا لمس مکن زیرا که هنوزنزد پدر خود بالا نرفته‌ام. و لیکن نزد برادران من رفته، به ایشان بگو که نزد پدر خود و پدر شما وخدای خود و خدای شما می‌روم.»
\par 18 مریم مجدلیه آمده، شاگردان را خبر داد که «خداوند رادیدم و به من چنین گفت.»
\par 19 و در شام همان روز که یکشنبه بود، هنگامی که درها بسته بود، جایی که شاگردان به‌سبب ترس یهود جمع بودند، ناگاه عیسی آمده، در میان ایستاد و بدیشان گفت: «سلام بر شما باد!»
\par 20 و چون این را گفت، دستها و پهلوی خود را به ایشان نشان داد و شاگردان چون خداوند رادیدند، شاد گشتند.
\par 21 باز عیسی به ایشان گفت: «سلام بر شما باد. چنانکه پدر مرا فرستاد، من نیزشما را می‌فرستم.»
\par 22 و چون این را گفت، دمید وبه ایشان گفت: «روح‌القدس را بیابید.
\par 23 گناهان آنانی را که آمرزیدید، برای ایشان آمرزیده شد وآنانی را که بستید، بسته شد.»
\par 24 اما توما که یکی از آن دوازده بود و او را توام می‌گفتند، وقتی که عیسی آمد با ایشان نبود.
\par 25 پس شاگردان دیگر بدو گفتند: «خداوند رادیده‌ایم.» بدیشان گفت: «تا در دو دستش جای میخها را نبینم و انگشت خود را در جای میخهانگذارم و دست خود را بر پهلویش ننهم، ایمان نخواهم آورد.»
\par 26 و بعد از هشت روز باز شاگردان با توما درخانه‌ای جمع بودند و درها بسته بود که ناگاه عیسی آمد و در میان ایستاده، گفت: «سلام بر شماباد.»
\par 27 پس به توما گفت: «انگشت خود را به اینجا بیاور و دستهای مرا ببین و دست خود رابیاور و بر پهلوی من بگذار و بی‌ایمان مباش بلکه ایمان دار.»
\par 28 توما در جواب وی گفت: «ای خداوند من و‌ای خدای من.»
\par 29 عیسی گفت: «ای توما، بعد از دیدنم ایمان آوردی؟ خوشابحال آنانی که ندیده ایمان آورند.»
\par 30 و عیسی معجزات دیگر بسیار نزد شاگردان نمود که در این کتاب نوشته نشد.لیکن این قدرنوشته شد تا ایمان آورید که عیسی، مسیح و پسرخدا است و تا ایمان آورده به اسم او حیات یابید.
\par 31 لیکن این قدرنوشته شد تا ایمان آورید که عیسی، مسیح و پسرخدا است و تا ایمان آورده به اسم او حیات یابید.

\chapter{21}

\par 1 بعد از آن عیسی باز خود را در کناره دریای طبریه، به شاگردان ظاهر ساخت و بر اینطور نمودار گشت:
\par 2 شمعون پطرس وتومای معروف به توام و نتنائیل که از قانای جلیل بود و دو پسر زبدی و دو نفر دیگر از شاگردان اوجمع بودند.
\par 3 شمعون پطرس به ایشان گفت: «می‌روم تا صید ماهی کنم.» به او گفتند: «مانیز باتو می‌آییم.» پس بیرون آمده، به کشتی سوارشدند و در آن شب چیزی نگرفتند.
\par 4 و چون صبح شد، عیسی بر ساحل ایستاده بود لیکن شاگردان ندانستند که عیسی است.
\par 5 عیسی بدیشان گفت: «ای بچه‌ها نزد شماخوراکی هست؟» به او جواب دادند که «نی.»
\par 6 بدیشان گفت: «دام را به طرف راست کشتی بیندازید که خواهید یافت.» پس انداختند و ازکثرت ماهی نتوانستند آن را بکشند.
\par 7 پس آن شاگردی که عیسی او را محبت می‌نمود به پطرس گفت: «خداوند است.» چون شمعون پطرس شنید که خداوند است، جامه خود را به خویشتن پیچید چونکه برهنه بود و خود را در دریاانداخت.
\par 8 اما شاگردان دیگر در زورق آمدند زیرااز خشکی دور نبودند، مگر قریب به دویست ذراع و دام ماهی را می‌کشیدند.
\par 9 پس چون به خشکی آمدند، آتشی افروخته و ماهی بر آن گذارده و نان دیدند.
\par 10 عیسی بدیشان گفت: «از ماهی‌ای که الان گرفته‌اید، بیاورید.»
\par 11 پس شمعون پطرس رفت و دام را برزمین کشید، پر از صد و پنجاه و سه ماهی بزرگ وبا وجودی که اینقدر بود، دام پاره نشد.
\par 12 عیسی بدیشان گفت: «بیایید بخورید.» ولی احدی ازشاگردان جرات نکرد که از او بپرسد «تو کیستی»، زیرا می‌دانستند که خداوند است.
\par 13 آنگاه عیسی آمد و نان را گرفته، بدیشان داد و همچنین ماهی را.
\par 14 و این مرتبه سوم بود که عیسی بعد ازبرخاستن از مردگان، خود را به شاگردان ظاهرکرد.
\par 15 و بعد از غذا خوردن، عیسی به شمعون پطرس گفت: «ای شمعون، پسر یونا، آیا مرابیشتر از اینها محبت می‌نمایی؟» بدو گفت: «بلی خداوندا، تو می‌دانی که تو را دوست می‌دارم.» بدو گفت: «بره های مرا خوراک بده.»
\par 16 باز درثانی به او گفت: «ای شمعون، پسر یونا، آیا مرامحبت می‌نمایی؟» به او گفت: «بلی خداوندا، تومی دانی که تو را دوست می‌دارم.» بدو گفت: «گوسفندان مرا شبانی کن.»
\par 17 مرتبه سوم بدوگفت: «ای شمعون، پسر یونا، مرا دوست می‌داری؟» پطرس محزون گشت، زیرا مرتبه سوم بدو گفت «مرا دوست می‌داری؟» پس به اوگفت: «خداوندا، تو بر همه‌چیز واقف هستی. تومی دانی که تو را دوست می‌دارم.» عیسی بدوگفت: «گوسفندان مرا خوراک ده.
\par 18 آمین آمین به تو می‌گویم وقتی که جوان بودی، کمر خود رامی بستی و هر جا می‌خواستی می‌رفتی ولکن زمانی که پیر شوی دستهای خود را دراز خواهی کرد و دیگران تو را بسته به‌جایی که نمی خواهی تو را خواهند برد.»
\par 19 و بدین سخن اشاره کردکه به چه قسم موت خدا را جلال خواهد داد وچون این را گفت، به او فرمود: «از عقب من بیا.»
\par 20 پطرس ملتفت شده، آن شاگردی که عیسی او را محبت می‌نمود دید که از عقب می‌آید؛ وهمان بود که بر سینه وی، وقت عشا تکیه می‌زد وگفت: «خداوندا کیست آن که تو را تسلیم می‌کند.»
\par 21 پس چون پطرس او را دید، به عیسی گفت: «ای خداوند و او چه شود؟»
\par 22 عیسی بدوگفت: «اگر بخواهم که او بماند تا باز آیم تو را چه؟ تو از عقب من بیا.»
\par 23 پس این سخن در میان برادران شهرت یافت که آن شاگرد نخواهد مرد. لیکن عیسی بدو نگفت که نمی میرد، بلکه «اگربخواهم که او بماند تا باز آیم تو را چه.»و این شاگردی است که به این چیزها شهادت داد و اینهارا نوشت و می‌دانیم که شهادت او راست است.
\par 24 و این شاگردی است که به این چیزها شهادت داد و اینهارا نوشت و می‌دانیم که شهادت او راست است.


\end{document}