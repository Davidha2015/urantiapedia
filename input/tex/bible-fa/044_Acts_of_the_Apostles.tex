\begin{document}

\title{اعمال رسولن}


\chapter{1}

\par 1 صحیفه اول را انشا نمودم، ای تیوفلس، درباره همه اموری که عیسی به عمل نمودن و تعلیم دادن آنها شروع کرد.
\par 2 تا آن روزی که رسولان برگزیده خود را به روح‌القدس حکم کرده، بالا برده شد.
\par 3 که بدیشان نیز بعد از زحمت کشیدن خود، خویشتن را زنده ظاهر کرد به دلیلهای بسیار که در مدت چهل روز بر ایشان ظاهر می‌شد و درباره امور ملکوت خدا سخن می‌گفت.
\par 4 و چون با ایشان جمع شد، ایشان راقدغن فرمود که «از اورشلیم جدا مشوید، بلکه منتظر آن وعده پدر باشید که از من شنیده‌اید.
\par 5 زیرا که یحیی به آب تعمید می‌داد، لیکن شمابعد از اندک ایامی، به روح‌القدس تعمید خواهیدیافت.»
\par 6 پس آنانی که جمع بودند، از او سوال نموده، گفتند: «خداوندا آیا در این وقت ملکوت را براسرائیل باز برقرار خواهی داشت؟»
\par 7 بدیشان گفت: «از شما نیست که زمانها و اوقاتی را که پدردر قدرت خود نگاه داشته است بدانید.
\par 8 لیکن چون روح‌القدس بر شما می‌آید، قوت خواهیدیافت و شاهدان من خواهید بود، در اورشلیم وتمامی یهودیه و سامره و تا اقصای جهان.»
\par 9 و چون این را گفت، وقتی که ایشان همی نگریستند، بالا برده شد و ابری او را از چشمان ایشان در ربود.
\par 10 و چون به سوی آسمان چشم دوخته می‌بودند، هنگامی که او می‌رفت، ناگاه دومرد سفیدپوش نزد ایشان ایستاده،
\par 11 گفتند: «ای مردان جلیلی چرا ایستاده، به سوی آسمان نگرانید؟ همین عیسی که از نزد شما به آسمان بالابرده شد، باز خواهد آمد به همین طوری که او رابه سوی آسمان روانه دیدید.»
\par 12 آنگاه به اورشلیم مراجعت کردند، از کوه مسمی به زیتون که نزدیک به اورشلیم به مسافت سفر یک روز سبت است.
\par 13 و چون داخل شدند، به بالاخانه‌ای برآمدند که در آنجا پطرس و یوحناو یعقوب و اندریاس و فیلپس و توما و برتولما ومتی و یعقوب بن حلفی و شمعون غیور و یهودای برادر بعقوب مقیم بودند.
\par 14 و جمیع اینها با زنان و مریم مادر عیسی و برادران او به یکدل درعبادت و دعا مواظب می‌بودند.
\par 15 و در آن ایام، پطرس در میان برادران که عدد اسامی ایشان جمله قریب به صد و بیست بود برخاسته، گفت:
\par 16 «ای برادران، می‌بایست آن نوشته تمام شود که روح‌القدس از زبان داودپیش گفت درباره یهودا که راهنما شد برای آنانی که عیسی را گرفتند.
\par 17 که او با ما محسوب شده، نصیبی در این خدمت یافت.
\par 18 پس او از اجرت ظلم خود، زمینی خریده، به روی درافتاده، ازمیان پاره شد و تمامی امعایش ریخته گشت.
\par 19 وبر تمام سکنه اورشلیم معلوم گردید چنانکه آن زمین در لغت ایشان به حقل دما، یعنی زمین خون نامیده شد.
\par 20 زیرا در کتاب زبور مکتوب است که خانه او خراب بشود و هیچ‌کس در آن مسکن نگیرد و نظارتش را دیگری ضبط نماید.
\par 21 الحال می‌باید از آن مردمانی که همراهان ما بودند، درتمام آن مدتی که عیسی خداوند با ما آمد و رفت می‌کرد،
\par 22 از زمان تعمید یحیی، تا روزی که ازنزد ما بالا برده شد، یکی از ایشان با ما شاهدبرخاستن او بشود.»
\par 23 آنگاه دو نفر یعنی یوسف مسمی به برسباکه به یوستس ملقب بود و متیاس را برپا داشتند،
\par 24 و دعا کرده، گفتند: «تو‌ای خداوند که عارف قلوب همه هستی، بنما کدام‌یک از این دو رابرگزیده‌ای
\par 25 تا قسمت این خدمت و رسالت رابیابد که یهودا از آن باز افتاده، به مکان خودپیوست.»پس قرعه به نام ایشان افکندند وقرعه به نام متیاس برآمد و او با یازده رسول محسوب گشت.
\par 26 پس قرعه به نام ایشان افکندند وقرعه به نام متیاس برآمد و او با یازده رسول محسوب گشت.

\chapter{2}

\par 1 و چون روز پنطیکاست رسید، به یک دل در یکجا بودند.
\par 2 که ناگاه آوازی چون صدای وزیدن باد شدید از آسمان آمد و تمام آن خانه را که در آنجا نشسته بودند پر ساخت.
\par 3 وزبانه های منقسم شده، مثل زبانه های آتش بدیشان ظاهر گشته، بر هر یکی از ایشان قرار گرفت.
\par 4 و همه از روح‌القدس پر گشته، به زبانهای مختلف، به نوعی که روح بدیشان قدرت تلفظ بخشید، به سخن‌گفتن شروع کردند.
\par 5 و مردم یهود دین دار از هر طایفه زیر فلک دراورشلیم منزل می‌داشتند.
\par 6 پس چون این صدابلند شد گروهی فراهم شده در حیرت افتادندزیرا هر کس لغت خود را از ایشان شنید.
\par 7 و همه مبهوت و متعجب شده به یکدیگر می‌گفتند: «مگر همه اینها که حرف می‌زنند جلیلی نیستند؟
\par 8 پس چون است که هر یکی از ما لغت خود را که در آن تولد یافته‌ایم می‌شنویم؟
\par 9 پارتیان و مادیان و عیلامیان و ساکنان جزیره و یهودیه و کپدکیا وپنطس و آسیا
\par 10 و فریجیه و پمفلیه و مصر ونواحی لبیا که متصل به قیروان است و غربا از روم یعنی یهودیان و جدیدان
\par 11 و اهل کریت و عرب اینها را می‌شنویم که به زبانهای ما ذکر کبریایی خدا می‌کنند.»
\par 12 پس همه در حیرت و شک افتاده، به یکدیگر گفتند: «این به کجا خواهدانجامید؟»
\par 13 اما بعضی استهزاکنان گفتند که «ازخمر تازه مست شده‌اند!»
\par 14 پس پطرس با آن یازده برخاسته، آواز خودرا بلند کرده، بدیشان گفت: «ای مردان یهود وجمیع سکنه اورشلیم، این را بدانید و سخنان مرافرا‌گیرید.
\par 15 زیرا که اینها مست نیستند چنانکه شما گمان می‌برید، زیرا که ساعت سوم از روزاست.
\par 16 بلکه این همان است که یوئیل نبی گفت
\par 17 که "خدا می‌گوید در ایام آخر چنین خواهدبود که از روح خود بر تمام بشر خواهم ریخت وپسران و دختران شما نبوت کنند و جوانان شما رویاها و پیران شما خوابها خواهند دید؛
\par 18 و برغلامان و کنیزان خود در آن ایام از روح خودخواهم ریخت و ایشان نبوت خواهند نمود.
\par 19 واز بالا در افلاک، عجایب و از پایین در زمین، آیات را از خون و آتش و بخار دود به ظهور آورم.
\par 20 خورشید به ظلمت و ماه به خون مبدل گرددقبل از وقوع روز عظیم مشهور خداوند.
\par 21 وچنین خواهد بود که هر‌که نام خداوند را بخواند، نجات یابد."
\par 22 «ای مردان اسرائیلی این سخنان را بشنوید. عیسی ناصری مردی که نزد شما از جانب خدامبرهن گشت به قوات و عجایب و آیاتی که خدادر میان شما از او صادر گردانید، چنانکه خودمی دانید،
\par 23 این شخص چون برحسب اراده مستحکم و پیشدانی خدا تسلیم شد، شما به‌دست گناهکاران بر صلیب کشیده، کشتید،
\par 24 که خدا دردهای موت را گسسته، او را برخیزانیدزیرا محال بود که موت او را در بند نگاه دارد،
\par 25 زیرا که داود درباره وی می‌گوید: "خداوند راهمواره پیش روی خود دیده‌ام که به‌دست راست من است تا جنبش نخورم؛
\par 26 از این سبب دلم شادگردید و زبانم به وجد آمد بلکه جسدم نیز در امیدساکن خواهد بود؛
\par 27 زیرا که نفس مرا در عالم اموات نخواهی گذاشت و اجازت نخواهی داد که قدوس تو فساد را ببیند.
\par 28 طریقهای حیات را به من آموختی و مرا از روی خود به خرمی سیرگردانیدی."
\par 29 «ای برادران، می‌توانم درباره داود پطریارخ با شما بی‌محابا سخن گویم که او وفات نموده، دفن شد و مقبره او تا امروز در میان ماست.
\par 30 پس چون نبی بود و دانست که خدا برای او قسم خوردکه از ذریت صلب او بحسب جسد، مسیح رابرانگیزاند تا بر تخت او بنشیند،
\par 31 درباره قیامت مسیح پیش دیده، گفت که نفس او در عالم اموات گذاشته نشود و جسد او فساد را نبیند.
\par 32 پس همان عیسی را خدا برخیزانید و همه ما شاهد برآن هستیم.
\par 33 پس چون به‌دست راست خدا بالابرده شد، روح‌القدس موعود را از پدر یافته، این را که شما حال می‌بینید و می‌شنوید ریخته است.
\par 34 زیرا که داود به آسمان صعود نکرد لیکن خودمی گوید "خداوند به خداوند من گفت بر دست راست من بنشین
\par 35 تا دشمنانت را پای انداز توسازم."
\par 36 پس جمیع خاندان اسرائیل یقین بدانندکه خدا همین عیسی را که شما مصلوب کردیدخداوند و مسیح ساخته است.»
\par 37 چون شنیدند دلریش گشته، به پطرس وسایر رسولان گفتند: «ای برادران چه کنیم؟»
\par 38 پطرس بدیشان گفت: «توبه کنید و هر یک ازشما به اسم عیسی مسیح بجهت آمرزش گناهان تعمید گیرید و عطای روح‌القدس را خواهیدیافت.
\par 39 زیرا که این وعده است برای شما وفرزندان شما و همه آنانی که دورند یعنی هرکه خداوند خدای ما او را بخواند.»
\par 40 و به سخنان بسیار دیگر، بدیشان شهادت داد و موعظه نموده، گفت که «خود را از این فرقه کجرو رستگارسازید.»
\par 41 پس ایشان کلام او را پذیرفته، تعمیدگرفتند و در همان روز تخمین سه هزار نفر بدیشان پیوستند
\par 42 و در تعلیم رسولان ومشارکت ایشان و شکستن نان و دعاها مواظبت می‌نمودند.
\par 43 و همه خلق ترسیدند و معجزات وعلامات بسیار از دست رسولان صادر می‌گشت.
\par 44 و همه ایمانداران با هم می‌زیستند و درهمه‌چیز شریک می‌بودند
\par 45 و املاک و اموال خود را فروخته، آنها را به هر کس به قدراحتیاجش تقسیم می‌کردند.
\par 46 و هر روزه درهیکل به یکدل پیوسته می‌بودند و در خانه‌ها نان را پاره می‌کردند و خوراک را به خوشی وساده دلی می‌خوردند.و خدا را حمد می‌گفتندو نزد تمامی خلق عزیز می‌گردیدند و خداوند هرروزه ناجیان را بر کلیسا می‌افزود.
\par 47 و خدا را حمد می‌گفتندو نزد تمامی خلق عزیز می‌گردیدند و خداوند هرروزه ناجیان را بر کلیسا می‌افزود.

\chapter{3}

\par 1 و در ساعت نهم، وقت نماز، پطرس و یوحنا با هم به هیکل می‌رفتند.
\par 2 ناگاه مردی را که لنگ مادرزاد بود می‌بردند که او را هرروزه بر آن در هیکل که جمیل نام داردمی گذاشتند تا از روندگان به هیکل صدقه بخواهد.
\par 3 آن شخص چون پطرس و یوحنا رادید که می‌خواهند به هیکل داخل شوند، صدقه خواست.
\par 4 اما پطرس با یوحنا بر وی نیک نگریسته، گفت: «به ما بنگر.»
\par 5 پس بر ایشان نظرافکنده، منتظر بود که از ایشان چیزی بگیرد.
\par 6 آنگاه پطرس گفت: «مرا طلا و نقره نیست، اماآنچه دارم به تو می‌دهم. به نام عیسی مسیح ناصری برخیز و بخرام!»
\par 7 و دست راستش راگرفته او را برخیزانید که در ساعت پایها و ساقهای او قوت گرفت
\par 8 و برجسته بایستاد وخرامید و با ایشان خرامان و جست و خیزکنان وخدا را حمدگویان داخل هیکل شد.
\par 9 و جمیع قوم او را خرامان و خدا راتسبیح خوانان دیدند.
\par 10 و چون او را شناختند که همان است که به در جمیل هیکل بجهت صدقه می‌نشست، به‌سبب این امر که بر او واقع شد، متعجب و متحیر گردیدند.
\par 11 و چون آن لنگ شفایافته به پطرس و یوحنا متمسک بود، تمامی قوم در رواقی که به سلیمانی مسمی است، حیرت زده بشتاب گرد ایشان جمع شدند.
\par 12 آنگاه پطرس ملتفت شده، بدان جماعت خطاب کرد که «ای مردان اسرائیلی، چرا از این کار تعجب دارید و چرا بر ما چشم دوخته‌اید که گویا به قوت و تقوای خود این شخص را خرامان ساختیم؟
\par 13 خدای ابراهیم و اسحاق و یعقوب، خدای اجداد ما، بنده خود عیسی را جلال داد که شما تسلیم نموده، او را در حضور پیلاطس انکارکردید، هنگامی که او حکم به رهانیدنش داد.
\par 14 اما شما آن قدوس و عادل را منکر شده، خواستید که مردی خون ریز به شما بخشیده شود.
\par 15 و رئیس حیات را کشتید که خدا او را ازمردگان برخیزانید و ما شاهد بر او هستیم.
\par 16 و به‌سبب ایمان به اسم او، اسم او این شخص را که می‌بینید و می‌شناسید قوت بخشیده است. بلی آن ایمانی که به وسیله اوست این کس را پیش روی همه شما این صحت کامل داده است.
\par 17 «و الحال‌ای برادران، می‌دانم که شما وچنین حکام شما این را به‌سبب ناشناسایی کردید.
\par 18 و لیکن خدا آن اخباری را که به زبان جمیع انبیای خود، پیش گفته بود که مسیح باید زحمت بیند، همینطور به انجام رسانید.
\par 19 پس توبه وبازگشت کنید تا گناهان شما محو گردد و تا اوقات استراحت از حضور خداوند برسد.
\par 20 و عیسی مسیح را که از اول برای شما اعلام شده بودبفرستد،
\par 21 که می‌باید آسمان او را پذیرد تا زمان معاد همه‌چیز که خدا از بدوعالم به زبان جمیع انبیای مقدس خود، از آن اخبار نمود.
\par 22 زیراموسی به اجداد گفت که خداوند خدای شما نبی مثل من، از میان برادران شما برای شما برخواهدانگیخت. کلام او را در هر‌چه به شما تکلم کندبشنوید؛
\par 23 و هر نفسی که آن نبی را نشنود، از قوم منقطع گردد.
\par 24 و جمیع انبیا نیز از سموئیل وآنانی که بعد از او تکلم کردند از این ایام اخبارنمودند.
\par 25 شما هستید اولاد پیغمبران و آن عهدی که خدا با اجداد ما بست، وقتی که به ابراهیم گفت از ذریت تو جمیع قبایل زمین، برکت خواهند یافتبرای شما اولا خدا بنده خودعیسی را برخیزانیده، فرستاد تا شما را برکت دهدبه برگردانیدن هر یکی از شما از گناهانش.»
\par 26 برای شما اولا خدا بنده خودعیسی را برخیزانیده، فرستاد تا شما را برکت دهدبه برگردانیدن هر یکی از شما از گناهانش.»

\chapter{4}

\par 1 و چون ایشان با قوم سخن می‌گفتند، کهنه وسردار سپاه هیکل و صدوقیان بر سر ایشان تاختند،
\par 2 چونکه مضطرب بودند از اینکه ایشان قوم را تعلیم می‌دادند و در عیسی به قیامت ازمردگان اعلام می‌نمودند.
\par 3 پس دست بر ایشان انداخته، تا فردا محبوس نمودند زیرا که آن، وقت عصر بود.
\par 4 اما بسیاری از آنانی که کلام را شنیدند ایمان آوردند و عدد ایشان قریب به پنج هزاررسید.
\par 5 بامدادان روسا و مشایخ و کاتبان ایشان دراورشلیم فراهم آمدند،
\par 6 با حنای رئیس کهنه وقیافا و یوحنا و اسکندر و همه کسانی که از قبیله رئیس کهنه بودند.
\par 7 و ایشان را در میان بداشتند واز ایشان پرسیدند که «شما به کدام قوت و به چه نام این کار را کرده‌اید؟»
\par 8 آنگاه پطرس ازروح‌القدس پر شده، بدیشان گفت: «ای روسای قوم و مشایخ اسرائیل،
\par 9 اگر امروز از ما بازپرس می‌شود درباره احسانی که بدین مرد ضعیف شده، یعنی به چه سبب او صحت یافته است،
\par 10 جمیع شما و تمام قوم اسرائیل را معلوم باد که به نام عیسی مسیح ناصری که شما مصلوب کردید و خدا او را از مردگان برخیزانید، در او این کس به حضور شما تندرست ایستاده است.
\par 11 این است آن سنگی که شما معماران آن را رد کردید والحال سر زاویه شده است.
\par 12 و در هیچ‌کس غیراز او نجات نیست زیرا که اسمی دیگر زیر آسمان به مردم عطانشده که بدان باید ما نجات یابیم.»
\par 13 پس چون دلیری پطرس و یوحنا را دیدندو دانستند که مردم بی‌علم و امی هستند، تعجب کردند و ایشان را شناختند که از همراهان عیسی بودند.
\par 14 و چون آن شخص را که شفا یافته بود باایشان ایستاده دیدند، نتوانستند به ضد ایشان چیزی گویند.
\par 15 پس حکم کردند که ایشان ازمجلس بیرون روند و با یکدیگر مشورت کرده، گفتند
\par 16 که «با این دو شخص چه کنیم؟ زیرا که برجمیع سکنه اورشلیم واضح شد که معجزه‌ای آشکار از ایشان صادر گردید و نمی توانیم انکار کرد.
\par 17 لیکن تا بیشتر در میان قوم شیوع نیابد، ایشان را سخت تهدید کنیم که دیگر با هیچ‌کس این اسم را به زبان نیاورند.»
\par 18 پس ایشان راخواسته قدغن کردند که هرگز نام عیسی را بر زبان نیاورند و تعلیم ندهند.
\par 19 اما پطرس و یوحنا درجواب ایشان گفتند: «اگر نزد خدا صواب است که اطاعت شما را بر اطاعت خدا ترجیح دهیم حکم کنید.
\par 20 زیرا که ما را امکان آن نیست که آنچه دیده و شنیده‌ایم، نگوییم.»
\par 21 و چون ایشان رازیاد تهدید نموده بودند، آزاد ساختند چونکه راهی نیافتند که ایشان را معذب سازند به‌سبب قوم زیرا همه به واسطه آن ماجرا خدا را تمجیدمی نمودند،
\par 22 زیرا آن شخص که معجزه شفا دراو پدید گشت، بیشتر از چهل ساله بود.
\par 23 و چون رهایی یافتند، نزد رفقای خودرفتند و ایشان را از آنچه روسای کهنه و مشایخ بدیشان گفته بودند، مطلع ساختند.
\par 24 چون این راشنیدند، آواز خود را به یکدل به خدا بلند کرده، گفتند: «خداوندا، تو آن خدا هستی که آسمان وزمین و دریا و آنچه در آنها است آفریدی،
\par 25 که بوسیله روح‌القدس به زبان پدر ما و بنده خودداود گفتی "چرا امت‌ها هنگامه می‌کنند و قومها به باطل می‌اندیشند؛
\par 26 سلاطین زمین برخاستند وحکام با هم مشورت کردند، برخلاف خداوند وبرخلاف مسیحش."
\par 27 زیرا که فی الواقع بر بنده قدوس تو عیسی که او را مسح کردی، هیرودیس و پنطیوس پیلاطس با امت‌ها و قومهای اسرائیل با هم جمع شدند،
\par 28 تا آنچه را که دست و رای تو از قبل مقدر فرموده بود، به‌جا آورند.
\par 29 و الان ای خداوند، به تهدیدات ایشان نظر کن و غلامان خود را عطا فرما تا به دلیری تمام به کلام تو سخن گویند،
\par 30 به دراز کردن دست خود، بجهت شفادادن و جاری کردن آیات و معجزات به نام بنده قدوس خود عیسی.»
\par 31 و چون ایشان دعا کرده بودند، مکانی که درآن جمع بودند به حرکت آمد و همه به روح‌القدس پر شده، کلام خدا را به دلیری می‌گفتند.
\par 32 و جمله مومنین را یک دل و یک جان بود، بحدی که هیچ‌کس چیزی از اموال خودرا از آن خود نمی دانست، بلکه همه‌چیز رامشترک می‌داشتند.
\par 33 و رسولان به قوت عظیم به قیامت عیسی خداوند شهادت می‌دادند و فیضی عظیم برهمگی ایشان بود.
\par 34 زیرا هیچ‌کس از آن گروه محتاج نبود زیرا هر‌که صاحب زمین یا خانه بود، آنها را فروختند و قیمت مبیعات را آورده،
\par 35 به قدمهای رسولان می‌نهادند و به هر یک بقدراحتیاجش تقسیم می‌نمودند.
\par 36 و یوسف که رسولان او را برنابا یعنی ابن الوعظ لقب دادند، مردی از سبط لاوی و از طایفه قپرسی،زمینی را که داشت فروخته، قیمت آن را آورد و پیش قدمهای رسولان گذارد.
\par 37 زمینی را که داشت فروخته، قیمت آن را آورد و پیش قدمهای رسولان گذارد.

\chapter{5}

\par 1 اما شخصی حنانیا نام، با زوجه‌اش سفیره ملکی فروخته،
\par 2 قدری از قیمت آن را به اطلاع زن خود نگاه داشت و قدری از آن راآورده، نزد قدمهای رسولان نهاد.
\par 3 آنگاه پطرس گفت: «ای حنانیا چرا شیطان دل تو را پر ساخته است تا روح‌القدس را فریب دهی و مقداری ازقیمت زمین را نگاه داری؟
\par 4 آیا چون داشتی از آن تو نبود و چون فروخته شد در اختیار تو نبود؟ چرا این را در دل خود نهادی؟ به انسان دروغ نگفتی بلکه به خدا.»
\par 5 حنانیا چون این سخنان راشنید افتاده، جان بداد و خوفی شدید بر همه شنوندگان این چیزها مستولی گشت.
\par 6 آنگاه جوانان برخاسته، او را کفن کردند و بیرون برده، دفن نمودند.
\par 7 و تخمین سه ساعت گذشت که زوجه‌اش ازماجرا مطلع نشده درآمد.
\par 8 پطرس بدو گفت: «مرابگو که آیا زمین را به همین قیمت فروختید؟» گفت: «بلی، به همین.»
\par 9 پطرس به وی گفت: «برای چه متفق شدید تا روح خداوند را امتحان کنید؟ اینک پایهای آنانی که شوهر تو را دفن کردند، بر آستانه است و تو را هم بیرون خواهندبرد.»
\par 10 در ساعت پیش قدمهای او افتاده، جان بداد و جوانان داخل شده، او را مرده یافتند. پس بیرون برده، به پهلوی شوهرش دفن کردند.
\par 11 وخوفی شدید تمامی کلیسا و همه آنانی را که این را شنیدند، فرو گرفت.
\par 12 و آیات و معجزات عظیمه از دستهای رسولان در میان قوم به ظهور می‌رسید و همه به یکدل در رواق سلیمان می‌بودند.
\par 13 اما احدی ازدیگران جرات نمی کرد که بدیشان ملحق شود، لیکن خلق، ایشان را محترم می‌داشتند.
\par 14 وبیشتر ایمانداران به خداوند متحد می‌شدند، انبوهی از مردان و زنان،
\par 15 بقسمی که مریضان رادر کوچه‌ها بیرون آوردند و بر بسترها و تختهاخوابانیدند تا وقتی که پطرس آید، اقلا سایه او بربعضی از ایشان بیفتد.
\par 16 و گروهی از بلدان اطراف اورشلیم، بیماران و رنج دیدگان ارواح پلیده را آورده، جمع شدند و جمیع ایشان شفایافتند.
\par 17 اما رئیس کهنه و همه رفقایش که از طایفه صدوقیان بودند، برخاسته، به غیرت پر گشتند
\par 18 و بر رسولان دست انداخته، ایشان را در زندان عام انداختند.
\par 19 شبانگاه فرشته خداوند درهای زندان را باز کرده و ایشان را بیرون آورده، گفت:
\par 20 «بروید و در هیکل ایستاده، تمام سخنهای این حیات را به مردم بگویید.
\par 21 چون این را شنیدند، وقت فجر به هیکل درآمده، تعلیم دادند.
\par 22 پس خادمان رفته، ایشان را در زندان نیافتند و برگشته، خبر داده،
\par 23 گفتند که «زندان را به احتیاط تمام بسته یافتیم و پاسبانان را بیرون درها ایستاده، لیکن چون باز کردیم، هیچ‌کس را در آن نیافتیم.»
\par 24 چون کاهن و سردار سپاه هیکل و روسای کهنه این سخنان را شنیدند، درباره ایشان در حیرت افتادند که «این چه خواهد شد؟»
\par 25 آنگاه کسی آمده ایشان را آگاهانید که اینک آن کسانی که محبوس نمودید، در هیکل ایستاده، مردم راتعلیم می‌دهند.
\par 26 پس سردار سپاه با خادمان رفته ایشان را آوردند، لیکن نه به زور زیرا که ازقوم ترسیدند که مبادا ایشان را سنگسار کنند.
\par 27 و چون ایشان را به مجلس حاضر کرده، برپابداشتند، رئیس کهنه از ایشان پرسیده، گفت:
\par 28 «مگر شما را قدغن بلیغ نفرمودیم که بدین اسم تعلیم مدهید؟ همانا اورشلیم را به تعلیم خود پرساخته‌اید و می‌خواهید خون این مرد را به گردن ما فرود آرید.»
\par 29 پطرس و رسولان در جواب گفتند: «خدا را می‌باید بیشتر از انسان اطاعت نمود.
\par 30 خدای پدران ما، آن عیسی را برخیزانیدکه شما به صلیب کشیده، کشتید.
\par 31 او را خدا بردست راست خود بالا برده، سرور و نجات‌دهنده ساخت تا اسرائیل را توبه و آمرزش گناهان بدهد.
\par 32 و ما هستیم شاهدان او بر این امور، چنانکه روح‌القدس نیز است که خدا او را به همه مطیعان او عطا فرموده است.»
\par 33 چون شنیدند دلریش گشته، مشورت کردند که ایشان را به قتل رسانند.
\par 34 اما شخصی فریسی، غمالائیل نام که مفتی و نزد تمامی خلق محترم بود، در مجلس برخاسته، فرمود تارسولان را ساعتی بیرون برند.
\par 35 پس ایشان راگفت: «ای مردان اسرائیلی، برحذر باشید از آنچه می‌خواهید با این اشخاص بکنید.
\par 36 زیرا قبل ازاین ایام، تیودا نامی برخاسته، خود را شخصی می‌پنداشت و گروهی قریب به چهار صد نفر بدوپیوستند. او کشته شد و متابعانش نیز پراکنده ونیست گردیدند.
\par 37 و بعد از او یهودای جلیلی درایام اسم نویسی خروج کرد و جمعی را در عقب خود کشید. او نیز هلاک شد و همه تابعان اوپراکنده شدند.
\par 38 الان به شما می‌گویم از این مردم دست بردارید و ایشان را واگذارید زیرا اگراین رای و عمل از انسان باشد، خود تباه خواهد شد.
\par 39 ولی اگر از خدا باشد، نمی توانید آن رابرطرف نمود مبادا معلوم شود که با خدا منازعه می‌کنید.»
\par 40 پس به سخن او رضا دادند و رسولان را حاضر ساخته، تازیانه زدند و قدغن نمودند که دیگر به نام عیسی حرف نزنند پس ایشان رامرخص کردند.
\par 41 و ایشان از حضور اهل شوراشادخاطر رفتند از آنرو که شایسته آن شمرده شدند که بجهت اسم او رسوایی کشندو هرروزه در هیکل و خانه‌ها از تعلیم و مژده دادن که عیسی مسیح است دست نکشیدند.
\par 42 و هرروزه در هیکل و خانه‌ها از تعلیم و مژده دادن که عیسی مسیح است دست نکشیدند.

\chapter{6}

\par 1 و در آن ایام چون شاگردان زیاد شدند، هلینستیان از عبرانیان شکایت بردند که بیوه‌زنان ایشان در خدمت یومیه بی‌بهره می‌ماندند.
\par 2 پس آن دوازده، جماعت شاگردان راطلبیده، گفتند: «شایسته نیست که ما کلام خدا راترک کرده، مائده‌ها را خدمت کنیم.
\par 3 لهذا‌ای برادران هفت نفر نیک نام و پر از روح‌القدس وحکمت را از میان خود انتخاب کنید تا ایشان را براین مهم بگماریم.
\par 4 اما ما خود را به عبادت وخدمت کلام خواهیم سپرد.»
\par 5 پس تمام جماعت بدین سخن رضا دادند و استیفان مردی پر از ایمان و روح‌القدس و فیلپس و پروخرس و نیکانور وتیمون و پرمیناس و نیقولاوس جدید، از اهل انطاکیه را انتخاب کرده،
\par 6 ایشان را در حضوررسولان برپا بداشتند و دعا کرده، دست بر ایشان گذاشتند.
\par 7 و کلام خدا ترقی نمود و عددشاگردان در اورشلیم بغایت می‌افزود و گروهی عظیم از کهنه مطیع ایمان شدند.
\par 8 اما استیفان پر از فیض و قوت شده، آیات ومعجزات عظیمه در میان مردم از او ظاهر می‌شد.
\par 9 و تنی چند از کنیسه‌ای که مشهور است به کنیسه لیبرتینیان و قیروانیان و اسکندریان و ازاهل قلیقیا و آسیا برخاسته، با استیفان مباحثه می‌کردند،
\par 10 و با آن حکمت و روحی که او سخن می‌گفت، یارای مکالمه نداشتند.
\par 11 پس چند نفررا بر این داشتند که بگویند: «این شخص راشنیدیم که به موسی و خدا سخن کفرآمیزمی گفت.»
\par 12 پس قوم و مشایخ و کاتبان راشورانیده، بر سر وی تاختند و او را گرفتار کرده، به مجلس حاضر ساختند.
\par 13 و شهود کذبه برپاداشته، گفتند که «این شخص از گفتن سخن کفرآمیز بر این مکان مقدس و تورات دست برنمی دارد.
\par 14 زیرا او را شنیدیم که می‌گفت این عیسی ناصری این مکان را تباه سازد و رسومی راکه موسی به ما سپرد، تغییر خواهد داد.»و همه کسانی که در مجلس حاضر بودند، بر او چشم دوخته، صورت وی را مثل صورت فرشته دیدند.
\par 15 و همه کسانی که در مجلس حاضر بودند، بر او چشم دوخته، صورت وی را مثل صورت فرشته دیدند.

\chapter{7}

\par 1 آنگاه رئیس کهنه گفت: «آیا این امور چنین است؟»
\par 2 او گفت: «ای برادران و پدران، گوش دهید. خدای ذوالجلال بر پدر ما ابراهیم ظاهر شد وقتی که در جزیره بود قبل از توقفش در حران.
\par 3 و بدوگفت: "از وطن خود و خویشانت بیرون شده، به زمینی که تو را نشان دهم برو."
\par 4 پس از دیارکلدانیان روانه شده، در حران درنگ نمود؛ و بعداز وفات پدرش، او را کوچ داد به سوی این زمین که شما الان در آن ساکن می‌باشید.
\par 5 و او را در این زمین میراثی، حتی بقدر جای پای خود نداد، لیکن وعده داد که آن را به وی و بعد از او به ذریتش به ملکیت دهد، هنگامی که هنوز اولادی نداشت.
\par 6 و خدا گفت که "ذریت تو در ملک بیگانه، غریب خواهند بود و مدت چهار صد سال ایشان را به بندگی کشیده، معذب خواهندداشت."
\par 7 و خدا گفت: "من بر آن طایفه‌ای که ایشان را مملوک سازند داوری خواهم نمود و بعداز آن بیرون آمده، در این مکان مرا عبادت خواهند نمود."
\par 8 و عهد ختنه را به وی داد که بنابراین چون اسحاق را آورد، در روز هشتم او رامختون ساخت و اسحاق یعقوب را و یعقوب دوازده پطریارخ را.
\par 9 «و پطریارخان به یوسف حسد برده، او را به مصر فروختند. اما خدا با وی می‌بود
\par 10 و او را ازتمامی زحمت او رستگار نموده، در حضورفرعون، پادشاه مصر توفیق و حکمت عطا فرمودتا او را بر مصر و تمام خاندان خود فرمان فرما قرارداد.
\par 11 پس قحطی و ضیقی شدید بر همه ولایت مصر و کنعان رخ نمود، بحدی که اجداد ما قوتی نیافتند.
\par 12 اما چون یعقوب شنید که در مصر غله یافت می‌شود، بار اول اجداد ما را فرستاد.
\par 13 ودر کرت دوم یوسف خود را به برادران خودشناسانید و قبیله یوسف به نظر فرعون رسیدند.
\par 14 پس یوسف فرستاده، پدر خود یعقوب و سایرعیالش را که هفتاد و پنج نفر بودند، طلبید.
\par 15 پس یعقوب به مصر فرود آمده، او و اجداد ما وفات یافتند.
\par 16 و ایشان را به شکیم برده، در مقبره‌ای که ابراهیم از بنی حمور، پدر شکیم به مبلغی خریده بود، دفن کردند.
\par 17 «و چون هنگام وعده‌ای که خدا با ابراهیم قسم خورده بود نزدیک شد، قوم در مصر نموکرده، کثیر می‌گشتند.
\par 18 تا وقتی که پادشاه دیگرکه یوسف را نمی شناخت برخاست.
\par 19 او با قوم ما حیله نموده، اجداد ما را ذلیل ساخت تا اولادخود را بیرون انداختند تا زیست نکنند.
\par 20 در آن وقت موسی تولد یافت وبغایت جمیل بوده، مدت سه ماه در خانه پدر خود پرورش یافت.
\par 21 و چون او را بیرون افکندند، دختر فرعون او رابرداشته، برای خود به فرزندی تربیت نمود.
\par 22 وموسی در تمامی حکمت اهل مصر تربیت یافته، در قول و فعل قوی گشت.
\par 23 چون چهل سال ازعمر وی سپری گشت، به‌خاطرش رسید که ازبرادران خود، خاندان اسرائیل تفقد نماید.
\par 24 وچون یکی را مظلوم دید او را حمایت نمود وانتقام آن عاجز را کشیده، آن مصری را بکشت.
\par 25 پس گمان برد که بردرانش خواهند فهمید که خدا به‌دست او ایشان را نجات خواهد داد. امانفهمیدند.
\par 26 و در فردای آن روز خود را به دو نفراز ایشان که منازعه می‌نمودند، ظاهر کرد وخواست مابین ایشان مصالحه دهد. پس گفت: "ای مردان، شما برادر می‌باشید. به یکدیگر چرا ظلم می‌کنید؟"
\par 27 آنگاه آنکه بر همسایه خودتعدی می‌نمود، او را رد کرده، گفت: "که تو را بر ماحاکم و داور ساخت؟
\par 28 آیا می‌خواهی مرابکشی چنانکه آن مصری را دیروز کشتی؟"
\par 29 پس موسی از این سخن فرار کرده، در زمین مدیان غربت اختیار کرد و در آنجا دو پسر آورد.
\par 30 و چون چهل سال گذشت، در بیابان کوه سینا، فرشته خداوند در شعله آتش از بوته به وی ظاهر شد.
\par 31 موسی چون این را دید از آن رویا درعجب شد و چون نزدیک می‌آمد تا نظر کند، خطاب از خداوند به وی رسید
\par 32 که "منم خدای پدرانت، خدای ابراهیم و خدای اسحاق و خدای یعقوب." آنگاه موسی به لرزه درآمده، جسارت نکرد که نظر کند.
\par 33 خداوند به وی گفت: "نعلین از پایهایت بیرون کن زیرا جایی که در آن ایستاده‌ای، زمین مقدس است.
\par 34 همانا مشقت قوم خود را که در مصرند دیدم و ناله ایشان راشنیدم و برای رهانیدن ایشان نزول فرمودم. الحال بیا تا تو را به مصر فرستم."
\par 35 همان موسی را که رد کرده، گفتند: "که تو را حاکم و داورساخت؟" خدا حاکم و نجات‌دهنده مقرر فرموده، به‌دست فرشته‌ای که در بوته بر وی ظاهر شد، فرستاد.
\par 36 او با معجزات و آیاتی که مدت چهل سال در زمین مصر و بحر قلزم و صحرا به ظهورمی آورد، ایشان را بیرون آورد.
\par 37 این همان موسی است که به بنی‌اسرائیل گفت: "خدا نبی‌ای را مثل من از میان برادران شما برای شما مبعوث خواهد کرد. سخن او را بشنوید."
\par 38 همین است آنکه در جماعت در صحرا باآن فرشته‌ای که در کوه سینا بدو سخن می‌گفت وبا پدران ما بود و کلمات زنده را یافت تا به مارساند،
\par 39 که پدران ما نخواستند او را مطیع شوندبلکه او را رد کرده، دلهای خود را به سوی مصرگردانیدند،
\par 40 و به هارون گفتند: "برای ما خدایان ساز که در‌پیش ما بخرامند زیرا این موسی که ما رااز زمین مصر برآورد، نمی دانیم او را چه شده است."
\par 41 پس در آن ایام گوساله‌ای ساختند وبدان بت قربانی گذرانیده به اعمال دستهای خودشادی کردند.
\par 42 از این جهت خدا رو گردانیده، ایشان را واگذاشت تا جنود آسمان را پرستش نمایند، چنانکه در صحف انبیا نوشته شده است که "ای خاندان اسرائیل، آیا مدت چهل سال دربیابان برای من قربانی‌ها و هدایا گذرانیدید؟
\par 43 وخیمه ملوک و کوکب، خدای خود رمفان رابرداشتید یعنی اصنامی را که ساختید تا آنها راعبادت کنید. پس شما را بدان طرف بابل منتقل سازم."
\par 44 و خیمه شهادت با پدران ما در صحرابود چنانکه امر فرموده، به موسی گفت: "آن رامطابق نمونه‌ای که دیده‌ای بساز."
\par 45 و آن را اجداد ما یافته، همراه یوشع درآوردند به ملک امت هایی که خدا آنها را ازپیش روی پدران ما بیرون افکند تا ایام داود.
\par 46 که او در حضور خدا مستفیض گشت و درخواست نمود که خود مسکنی برای خدای یعقوب پیدانماید.
\par 47 اما سلیمان برای او خانه‌ای بساخت.
\par 48 و لیکن حضرت اعلی در خانه های مصنوع دستها ساکن نمی شود چنانکه نبی گفته است
\par 49 که "خداوند می‌گوید آسمان کرسی من است وزمین پای انداز من. چه خانه‌ای برای من بنامی کنید و محل آرامیدن من کجاست؟
\par 50 مگردست من جمیع این چیزها را نیافرید."
\par 51 ‌ای گردنکشان که به دل و گوش نامختونید، شما پیوسته با روح‌القدس مقاومت می‌کنید، چنانکه پدران شما همچنین شما.
\par 52 کیست ازانبیا که پدران شما بدو جفا نکردند؟ و آنانی راکشتند که از آمدن آن عادلی که شما بالفعل تسلیم کنندگان و قاتلان او شدید، پیش اخبارنمودند.
\par 53 شما که به توسط فرشتگان شریعت رایافته، آن را حفظ نکردید!»
\par 54 چون این را شنیدند دلریش شده، بر وی دندانهای خود را فشردند.
\par 55 اما او از روح‌القدس پر بوده، به سوی آسمان نگریست و جلال خدا رادید و عیسی را بدست راست خدا ایستاده وگفت:
\par 56 «اینک آسمان را گشاده، و پسر انسان رابه‌دست راست خدا ایستاده می‌بینم.»
\par 57 آنگاه به آواز بلند فریاد برکشیدند و گوشهای خود راگرفته، به یکدل بر او حمله کردند،
\par 58 و از شهربیرون کشیده، سنگسارش کردند. و شاهدان، جامه های خود را نزد پایهای جوانی که سولس نام داشت گذاردند.
\par 59 و چون استیفان را سنگسارمی کردند، او دعا نموده، گفت: «ای عیسی خداوند، روح مرا بپذیر.»پس زانو زده، به آوازبلند ندا در‌داد که «خداوندا این گناه را بر اینهامگیر.» این را گفت و خوابید.
\par 60 پس زانو زده، به آوازبلند ندا در‌داد که «خداوندا این گناه را بر اینهامگیر.» این را گفت و خوابید.

\chapter{8}

\par 1 می بود.
\par 2 و مردان صالح استیفان را دفن کرده، برای وی ماتم عظیمی برپا داشتند.
\par 3 اما سولس کلیسا رامعذب می‌ساخت و خانه به خانه گشته، مردان وزنان را برکشیده، به زندان می‌افکند.
\par 4 پس آنانی که متفرق شدند، به هر جایی که می‌رسیدند به کلام بشارت می‌دادند.
\par 5 اما فیلپس به بلدی از سامره درآمده، ایشان را به مسیح موعظه می‌نمود.
\par 6 و مردم به یکدل به سخنان فیلپس گوش دادند، چون معجزاتی را که از اوصادر می‌گشت، می‌شنیدند و می‌دیدند،
\par 7 زیرا که ارواح پلید از بسیاری که داشتند نعره زده، بیرون می‌شدند ومفلوجان و لنگان بسیار شفا می‌یافتند.
\par 8 و شادی عظیم در آن شهر روی نمود.
\par 9 اما مردی شمعون نام قبل از آن در آن قریه بود که جادوگری می‌نمود و اهل سامره را متحیرمی ساخت و خود را شخصی بزرگ می‌نمود،
\par 10 بحدی که خرد و بزرگ گوش داده، می‌گفتند: «این است قوت عظیم خدا.»
\par 11 و بدو گوش دادنداز آنرو که مدت مدیدی بود از جادوگری اومتحیر می‌شدند.
\par 12 لیکن چون به بشارت فیلپس که به ملکوت خدا و نام عیسی مسیح می‌داد، ایمان آوردند، مردان و زنان تعمید یافتند.
\par 13 وشمعون نیز خود ایمان آورد و چون تعمید یافت همواره با فیلپس می‌بود و از دیدن آیات و قوات عظیمه که از او ظاهر می‌شد، در حیرت افتاد.
\par 14 اما رسولان که در اورشلیم بودند، چون شنیدند که اهل سامره کلام خدا را پذیرفته‌اند، پطرس و یوحنا را نزد ایشان فرستادند.
\par 15 وایشان آمده، بجهت ایشان دعا کردند تاروح‌القدس را بیابند،
\par 16 زیرا که هنوز بر هیچ‌کس از ایشان نازل نشده بود که به نام خداوند عیسی تعمید یافته بودند و بس.
\par 17 پس دستها بر ایشان گذارده، روح‌القدس را یافتند.
\par 18 اما شمعون چون دید که محض گذاردن دستهای رسولان روح‌القدس عطا می‌شود، مبلغی پیش ایشان آورده،
\par 19 گفت: «مرا نیز این قدرت دهید که به هرکس دست گذارم، روح‌القدس را بیابد.»
\par 20 پطرس بدو گفت: «زرت با تو هلاک باد، چونکه پنداشتی که عطای خدا به زر حاصل می‌شود.
\par 21 تو را دراین امر، قسمت و بهره‌ای نیست زیرا که دلت درحضور خدا راست نمی باشد.
\par 22 پس از این شرارت خود توبه کن و از خدا درخواست کن تاشاید این فکر دلت آمرزیده شود،
\par 23 زیرا که تو رامی بینم در زهره تلخ و قید شرارت گرفتاری.»
\par 24 شمعون در جواب گفت: «شما برای من به خداوند دعا کنید تا چیزی از آنچه گفتید بر من عارض نشود.»
\par 25 پس ارشاد نموده و به کلام خداوند تکلم کرده، به اورشلیم برگشتند و دربسیاری از بلدان اهل سامره بشارت دادند.
\par 26 اما فرشته خداوند به فیلپس خطاب کرده، گفت: «برخیز و به‌جانب جنوب، به راهی که ازاورشلیم به سوی غزه می‌رود که صحراست، روانه شو.»
\par 27 پس برخاسته، روانه شد که ناگاه شخصی حبشی که خواجه‌سرا و مقتدر نزدکنداکه، ملکه حبش، و بر تمام خزانه او مختاربود، به اورشلیم بجهت عبادت آمده بود.
\par 28 و درمراجعت بر ارابه خود نشسته، صحیفه اشعیای نبی را مطالعه می‌کند
\par 29 آنگاه روح به فیلپس گفت: «پیش برو و با آن ارابه همراه باش.»
\par 30 فیلپس پیش دویده، شنید که اشعیای نبی رامطالعه می‌کند. گفت: «آیا می‌فهمی آنچه رامی خوانی؟»
\par 31 گفت: «چگونه می‌توانم؟ مگرآنکه کسی مرا هدایت کند.» و از فیلپس خواهش نمود که سوار شده، با او بنشیند.
\par 32 و فقره‌ای ازکتاب که می‌خواند این بود که «مثل گوسفندی که به مذبح برند و چون بره‌ای خاموش نزد پشم برنده خود، همچنین دهان خود را نمی گشاید.
\par 33 درفروتنی او انصاف از او منقطع شد و نسب او را که می‌تواند تقریر کرد؟ زیرا که حیات او از زمین برداشته می‌شود.»
\par 34 پس خواجه‌سرا به فیلپس ملتفت شده، گفت: «از تو سوال می‌کنم که نبی این را درباره که می‌گوید؟ درباره خود یا درباره کسی دیگر؟»
\par 35 آنگاه فیلپس زبان خود را گشود و ازآن نوشته شروع کرده، وی را به عیسی بشارت داد.
\par 36 و چون در عرض راه به آبی رسیدند، خواجه گفت: «اینک آب است! از تعمید یافتنم چه چیز مانع می‌باشد؟»
\par 37 فیلپس گفت: «هر گاه به تمام دل ایمان آوردی، جایز است.» او درجواب گفت: «ایمان آوردم که عیسی مسیح پسرخداست.»
\par 38 پس حکم کرد تا ارابه را نگاه دارندو فیلپس با خواجه‌سرا هر دو به آب فرود شدند. پس او را تعمید داد.
\par 39 و چون از آب بالا آمدند، روح خداوند فیلپس را برداشته، خواجه‌سرا دیگراو را نیافت زیرا که راه خود را به خوشی پیش گرفت.اما فیلپس در اشدود پیدا شد و در همه شهرها گشته بشارت می‌داد تا به قیصریه رسید.
\par 40 اما فیلپس در اشدود پیدا شد و در همه شهرها گشته بشارت می‌داد تا به قیصریه رسید.

\chapter{9}

\par 1 اما سولس هنوز تهدید و قتل بر شاگردان خداوند همی دمید و نزد رئیس کهنه آمد،
\par 2 و از او نامه‌ها خواست به سوی کنایسی که دردمشق بود تا اگر کسی را از اهل طریقت خواه مردو خواه زن بیابد، ایشان را بند برنهاده، به اورشلیم بیاورد.
\par 3 و در اثنای راه، چون نزدیک به دمشق رسید، ناگاه نوری از آسمان دور او درخشید
\par 4 و به زمین افتاده، آوازی شنید که بدو گفت: «ای شاول، شاول، برای چه بر من جفا می‌کنی؟»
\par 5 گفت: «خداوندا تو کیستی؟» خداوند گفت: «من آن عیسی هستم که تو بدو جفا می‌کنی.
\par 6 لیکن برخاسته، به شهر برو که آنجا به تو گفته می‌شودچه باید کرد.»
\par 7 اما آنانی که همسفر او بودند، خاموش ایستادند چونکه آن صدا را شنیدند، لیکن هیچ‌کس را ندیدند.
\par 8 پس سولس از زمین برخاسته، چون چشمان خود را گشود، هیچ‌کس را ندید و دستش را گرفته، او را به دمشق بردند،
\par 9 و سه روز نابینا بوده، چیزی نخورد و نیاشامید.
\par 10 و در دمشق، شاگردی حنانیا نام بود که خداوند در رویا بدو گفت: «ای حنانیا!» عرض کرد: «خداوندا لبیک!»
\par 11 خداوند وی را گفت: «برخیز و به کوچه‌ای که آن را راست می‌نامند بشتاب و در خانه یهودا، سولس نام طرسوسی راطلب کن زیرا که اینک دعا می‌کند،
\par 12 و شخصی حنانیا نام را در خواب دیده است که آمده، بر اودست گذارد تا بینا گردد.»
\par 13 حنانیا جواب داد که «ای خداوند، درباره این شخص از بسیاری شنیده‌ام که به مقدسین تو در اورشلیم چه مشقتهارسانید،
\par 14 و در اینجا نیز از روسای کهنه قدرت دارد که هر‌که نام تو را بخواند، او را حبس کند.»
\par 15 خداوند وی را گفت: «برو زیرا که او ظرف برگزیده من است تا نام مرا پیش امت‌ها و سلاطین و بنی‌اسرائیل ببرد.
\par 16 زیرا که من او را نشان خواهم داد که چقدر زحمتها برای نام من بایدبکشد.»
\par 17 پس حنانیا رفته، بدان خانه درآمد و دستهابر وی گذارده، گفت: «ای برادر شاول، خداوندیعنی عیسی که در راهی که می‌آمدی بر تو ظاهرگشت، مرا فرستاد تا بینایی بیابی و از روح‌القدس پر شوی.»
\par 18 در ساعت از چشمان او چیزی مثل فلس افتاده، بینایی یافت و برخاسته، تعمیدگرفت.
\par 19 و غذا خورده، قوت گرفت و روزی چند با شاگردان در دمشق توقف نمود.
\par 20 وبی درنگ، در کنایس به عیسی موعظه می‌نمود که او پسر خداست.
\par 21 و آنانی که شنیدند تعجب نموده، گفتند: «مگر این آن کسی نیست که خوانندگان این اسم را در اورشلیم پریشان می‌نمود و در اینجا محض این آمده است تا ایشان را بند نهاده، نزد روسای کهنه برد؟»
\par 22 اما سولس بیشتر تقویت یافته، یهودیان ساکن دمشق رامجاب می‌نمود و مبرهن می‌ساخت که همین است مسیح.
\par 23 اما بعد از مرور ایام چند یهودیان شورا نمودند تا او را بکشند.
\par 24 ولی سولس ازشورای ایشان مطلع شد و شبانه‌روز به دروازه هاپاسبانی می‌نمودند تا او را بکشند.
\par 25 پس شاگردان او را در شب در زنبیلی گذارده، از دیوارشهر پایین کردند.
\par 26 و چون سولس به اورشلیم رسید، خواست به شاگردان ملحق شود، لیکن همه از او بترسیدندزیرا باور نکردند که از شاگردان است.
\par 27 اما برنابااو را گرفته، به نزد رسولان برد و برای ایشان حکایت کرد که چگونه خداوند را در راه دیده وبدو تکلم کرده و چطور در دمشق به نام عیسی به دلیری موعظه می‌نمود.
\par 28 و در اورشلیم با ایشان آمد و رفت می‌کرد و به نام خداوند عیسی به دلیری موعظه می‌نمود.
\par 29 و با هلینستیان گفتگوو مباحثه می‌کرد. اما درصدد کشتن او برآمدند.
\par 30 چون برادران مطلع شدند، او را به قیصریه بردند و از آنجا به طرسوس روانه نمودند.
\par 31 آنگاه کلیسا در تمامی یهودیه و جلیل و سامره آرامی یافتند و بنا می‌شدند و در ترس خداوند وبه تسلی روح‌القدس رفتار کرده، همی افزودند.
\par 32 اما پطرس در همه نواحی گشته، نزدمقدسین ساکن لده نیز فرود آمد.
\par 33 و در آنجاشخصی اینیاس نام یافت که مدت هشت سال ازمرض فالج بر تخت خوابیده بود.
\par 34 پطرس وی را گفت: «ای اینیاس، عیسی مسیح تو را شفامی دهد. برخیز و بستر خود را برچین که او درساعت برخاست.»
\par 35 و جمیع سکنه لده و سارون او را دیده، به سوی خداوند بازگشت کردند.
\par 36 و در یافا، تلمیذه‌ای طابیتا نام بود که معنی آن غزال است. وی از اعمال صالحه و صدقاتی که می‌کرد، پر بود.
\par 37 از قضا در آن ایام او بیمارشده، بمرد و او را غسل داده، در بالاخانه‌ای گذاردند.
\par 38 و چونکه لده نزدیک به یافا بود وشاگردان شنیدند که پطرس در آنجا است، دو نفرنزد او فرستاده، خواهش کردند که «در‌آمدن نزدما درنگ نکنی.»
\par 39 آنگاه پطرس برخاسته، باایشان آمد و چون رسید او را بدان بالاخانه بردندو همه بیوه‌زنان گریه‌کنان حاضر بودند و پیراهنهاو جامه هایی که غزال وقتی که با ایشان بود دوخته بود، به وی نشان می‌دادند.
\par 40 اما پطرس همه رابیرون کرده، زانو زد و دعا کرده، به سوی بدن توجه کرد و گفت: «ای طابیتا، برخیز!» که درساعت چشمان خود را باز کرد و پطرس را دیده، بنشست.
\par 41 پس دست او را گرفته، برخیزانیدش و مقدسان و بیوه‌زنان را خوانده، او را بدیشان زنده سپرد.
\par 42 چون این مقدمه در تمامی یافا شهرت یافت، بسیاری به خداوند ایمان آوردند.و دریافا نزد دباغی شمعون نام روزی چند توقف نمود.
\par 43 و دریافا نزد دباغی شمعون نام روزی چند توقف نمود.

\chapter{10}

\par 1 و در قیصریه مردی کرنیلیوس نام بود، یوزباشی فوجی که به ایطالیانی مشهوراست.
\par 2 و او با تمامی اهل بیتش متقی و خداترس بود که صدقه بسیار به قوم می‌داد و پیوسته نزدخدا دعا می‌کرد.
\par 3 روزی نزدیک ساعت نهم، فرشته خدا را در عالم رویا آشکارا دید که نزد اوآمده، گفت: «ای کرنیلیوس!»
\par 4 آنگاه او بر وی نیک نگریسته و ترسان گشته، گفت: «چیست‌ای خداوند؟» به وی گفت: «دعاها و صدقات تو بجهت یادگاری به نزد خدا برآمد.
\par 5 اکنون کسانی به یافا بفرست و شمعون ملقب به پطرس را طلب کن
\par 6 که نزد دباغی شمعون نام که خانه‌اش به کناره دریا است، مهمان است. او به تو خواهد گفت که تو را چه باید کرد.
\par 7 و چون فرشته‌ای که به وی سخن می‌گفت غایب شد، دو نفر از نوکران خود ویک سپاهی متقی از ملازمان خاص خویشتن راخوانده،
\par 8 تمامی ماجرا را بدیشان باز‌گفته، ایشان را به یافا فرستاد.
\par 9 روز دیگر چون از سفر نزدیک به شهرمی رسیدند، قریب به ساعت ششم، پطرس به بام خانه برآمد تا دعا کند.
\par 10 و واقع شد که گرسنه شده، خواست چیزی بخورد. اما چون برای اوحاضر می‌کردند، بی‌خودی او را رخ نمود.
\par 11 پس آسمان را گشاده دید و ظرفی را چون چادری بزرگ به چهار گوشه بسته، به سوی زمین آویخته بر او نازل می‌شود،
\par 12 که در آن هر قسمی ازدواب و وحوش و حشرات زمین و مرغان هوابودند.
\par 13 و خطابی به وی رسید که «ای پطرس برخاسته، ذبح کن و بخور.»
\par 14 پطرس گفت: «حاشا خداوندا زیرا چیزی ناپاک یا حرام هرگزنخورده‌ام.»
\par 15 بار دیگر خطاب به وی رسید که «آنچه خدا پاک کرده است، تو حرام مخوان.»
\par 16 واین سه مرتبه واقع شد که در ساعت آن ظرف به آسمان بالا برده شد.
\par 17 و چون پطرس در خود بسیار متحیر بود که این‌رویایی که دید چه باشد، ناگاه فرستادگان کرنیلیوس خانه شمعون را تفحص کرده، بر درگاه رسیدند،
\par 18 و ندا کرده، می‌پرسیدند که «شمعون معروف به پطرس در اینجا منزل دارد؟»
\par 19 وچون پطرس در رویا تفکر می‌کرد، روح وی راگفت: «اینک سه مرد تو را می‌طلبند.
\par 20 پس برخاسته، پایین شو و همراه ایشان برو و هیچ شک مبر زیرا که من ایشان را فرستادم.»
\par 21 پس پطرس نزد آنانی که کرنیلیوس نزد وی فرستاده بود، پایین آمده، گفت: «اینک منم آن کس که می‌طلبید. سبب آمدن شما چیست؟»
\par 22 گفتند: «کرنیلیوس یوزباشی، مرد صالح و خداترس و نزد تمامی طایفه یهود نیکنام، از فرشته مقدس الهام یافت که تو را به خانه خود بطلبد و سخنان از تو بشنود.»
\par 23 پس ایشان را به خانه برده، مهمانی نمود. وفردای آن روز پطرس برخاسته، همراه ایشان روانه شد و چند نفر از برادران یافا همراه او رفتند.
\par 24 روز دیگر وارد قیصریه شدند و کرنیلیوس خویشان و دوستان خاص خود را خوانده، انتظارایشان می‌کشید.
\par 25 چون پطرس داخل شد، کرنیلیوس او را استقبال کرده، بر پایهایش افتاده، پرستش کرد.
\par 26 اما پطرس او را برخیزانیده، گفت: «برخیز، من خود نیز انسان هستم.»
\par 27 و با اوگفتگوکنان به خانه درآمده، جمعی کثیر یافت.
\par 28 پس بدیشان گفت: «شما مطلع هستید که مردیهودی را با شخص اجنبی معاشرت کردن یا نزداو آمدن حرام است. لیکن خدا مرا تعلیم داد که هیچ‌کس را حرام یا نجس نخوانم.
\par 29 از این جهت به مجرد خواهش شما بی‌تامل آمدم و الحال می‌پرسم که از برای چه مرا خواسته‌اید.»
\par 30 کرنیلیوس گفت: «چهار روز قبل از این، تااین ساعت روزه‌دار می‌بودم؛ و در ساعت نهم درخانه خود دعا می‌کردم که ناگاه شخصی با لباس نورانی پیش من بایستاد
\par 31 و گفت: "ای کرنیلیوس دعای تو مستجاب شد و صدقات تو در حضورخدا یادآور گردید.
\par 32 پس به یافا بفرست و شمعون معروف به پطرس را طلب نما که در خانه شمعون دباغ به کناره دریا مهمان است. او چون بیاید با تو سخن خواهد راند."
\par 33 پس بی‌تامل نزدتو فرستادم و تو نیکو کردی که آمدی. الحال همه در حضور خدا حاضریم تا آنچه خدا به توفرموده است بشنویم.»
\par 34 پطرس زبان را گشوده، گفت: «فی الحقیقت یافته‌ام که خدا را نظر به ظاهر نیست،
\par 35 بلکه ازهر امتی، هر‌که از او ترسد و عمل نیکو کند، نزداو مقبول گردد.
\par 36 کلامی را که نزد بنی‌اسرائیل فرستاد، چونکه به وساطت عیسی مسیح که خداوند همه است به سلامتی بشارت می‌داد،
\par 37 آن سخن را شما می‌دانید که شروع آن از جلیل بود و در تمامی یهودیه منتشر شد، بعد از آن تعمیدی که یحیی بدان موعظه می‌نمود،
\par 38 یعنی عیسی ناصری را که خدا او را چگونه به روح‌القدس و قوت مسح نمود که او سیر کرده، اعمال نیکو به‌جا می‌آورد و همه مقهورین ابلیس را شفا می‌بخشید زیرا خدا با وی می‌بود.
\par 39 و ماشاهد هستیم بر جمیع کارهایی که او در مرزوبوم یهود و در اورشلیم کرد که او را نیز بر صلیب کشیده، کشتند.
\par 40 همان کس را خدا در روز سوم برخیزانیده، ظاهر ساخت.
\par 41 لیکن نه بر تمامی قوم بلکه بر شهودی که خدا پیش برگزیده بود، یعنی مایانی که بعد از برخاستن او از مردگان با اوخورده و آشامیده‌ایم.
\par 42 و ما را مامور فرمود که به قوم موعظه و شهادت دهیم بدین که خدا او رامقرر فرمود تا داور زندگان و مردگان باشد.
\par 43 وجمیع انبیا بر او شهادت می‌دهند که هر‌که به وی ایمان آورد، به اسم او آمرزش گناهان را خواهد یافت.»
\par 44 این سخنان هنوز بر زبان پطرس بود که روح‌القدس بر همه آنانی که کلام را شنیدند، نازل شد.
\par 45 و مومنان از اهل ختنه که همراه پطرس آمده بودند، در حیرت افتادند از آنکه بر امت هانیز عطای روح‌القدس افاضه شد،
\par 46 زیرا که ایشان را شنیدند که به زبانها متکلم شده، خدا راتمجید می‌کردند.
\par 47 آنگاه پطرس گفت: «آیاکسی می‌تواند آب را منع کند، برای تعمید دادن اینانی که روح‌القدس را چون ما نیز یافته‌اند.»پس فرمود تا ایشان را به نام عیسی مسیح تعمید دهند. آنگاه از او خواهش نمودند که روزی چند توقف نماید.
\par 48 پس فرمود تا ایشان را به نام عیسی مسیح تعمید دهند. آنگاه از او خواهش نمودند که روزی چند توقف نماید.

\chapter{11}

\par 1 پس رسولان و برادرانی که در یهودیه بودند، شنیدند که امت‌ها نیز کلام خدارا پذیرفته‌اند.
\par 2 و چون پطرس به اورشلیم آمد، اهل ختنه با وی معارضه کرده،
\par 3 گفتند که «بامردم نامختون برآمده، با ایشان غذا خوردی!»
\par 4 پطرس از اول مفصلا بدیشان بیان کرده، گفت:
\par 5 «من در شهر یافا دعا می‌کردم که ناگاه درعالم رویا ظرفی را دیدم که نازل می‌شود مثل چادری بزرگ به چهار گوشه از آسمان آویخته که بر من می‌رسد.
\par 6 چون بر آن نیک نگریسته، تامل کردم، دواب زمین و وحوش و حشرات و مرغان هوا را دیدم.
\par 7 و آوازی را شنیدم که به من می‌گوید: "ای پطرس برخاسته، ذبح کن و بخور."
\par 8 گفتم: "حاشا خداوندا، زیرا هرگز چیزی حرام یا ناپاک به دهانم نرفته است."
\par 9 بار دیگر خطاب از آسمان در‌رسید که "آنچه خدا پاک نموده، توحرام مخوان."
\par 10 این سه کرت واقع شد که همه باز به سوی آسمان بالا برده شد.
\par 11 و اینک در همان ساعت سه مرد از قیصریه نزد من فرستاده شده، به خانه‌ای که در آن بودم، رسیدند.
\par 12 و روح مرا گفت که «با ایشان بدون شک برو.» و این شش برادر نیز همراه من آمدند تابه خانه آن شخص داخل شدیم.
\par 13 و ما راآگاهانید که چطور فرشته‌ای را در خانه خود دیدکه ایستاده به وی گفت "کسان به یافا بفرست وشمعون معروف به پطرس را بطلب
\par 14 که با توسخنانی خواهد گفت که بدانها تو و تمامی اهل خانه تو نجات خواهید یافت."
\par 15 و چون شروع به سخن‌گفتن می‌کردم، روح‌القدس بر ایشان نازل شد، همچنانکه نخست بر ما.
\par 16 آنگاه بخاطرآوردم سخن خداوند را که گفت: "یحیی به آب تعمید داد، لیکن شما به روح‌القدس تعمیدخواهید یافت."
\par 17 پس چون خدا همان عطا رابدیشان بخشید، چنانکه به ما محض ایمان آوردن به عیسی مسیح خداوند، پس من که باشم که بتوانم خدا را ممانعت نمایم؟»
\par 18 چون این را شنیدند، ساکت شدند و خدا را تمجیدکنان گفتند: «فی الحقیقت، خدا به امت‌ها نیز توبه حیات‌بخش را عطا کرده است!»
\par 19 و آنانی که به‌سبب اذیتی که در مقدمه استیفان برپا شد متفرق شدند، تا فینیقیا و قپرس وانطاکیه می‌گشتند و به هیچ‌کس به غیر از یهود وبس کلام را نگفتند.
\par 20 لیکن بعضی از ایشان که از اهل قپرس و قیروان بودند، چون به انطاکیه رسیدند با یونانیان نیز تکلم کردند و به خداوندعیسی بشارت می‌دادند،
\par 21 و دست خداوند باایشان می‌بود و جمعی کثیر ایمان آورده، به سوی خداوند بازگشت کردند.
\par 22 اما چون خبر ایشان به سمع کلیسای اورشلیم رسید، برنابا را به انطاکیه فرستادند
\par 23 و چون رسید و فیض خدا را دید، شادخاطر شده، همه را نصیحت نمود که ازتصمیم قلب به خداوند بپیوندند.
\par 24 زیرا که مردی صالح و پر از روح‌القدس و ایمان بود وگروهی بسیار به خداوند ایمان آوردند.
\par 25 و برنابابه طرسوس برای طلب سولس رفت و چون او رایافت به انطاکیه آورد.
\par 26 و ایشان سالی تمام درکلیسا جمع می‌شدند و خلقی بسیار را تعلیم می‌دادند و شاگردان نخست در انطاکیه به مسیحی مسمی شدند.
\par 27 و در آن ایام انبیایی چند از اورشلیم به انطاکیه آمدند
\par 28 که یکی از ایشان اغابوس نام برخاسته، به روح اشاره کرد که قحطی شدید درتمامی ربع مسکون خواهد شد و آن در ایام کلودیوس قیصر پدید آمد.
\par 29 و شاگردان مصمم آن شدند که هر یک برحسب مقدور خود، اعانتی برای برادران ساکن یهودیه بفرستند.پس چنین کردند و آن را به‌دست برنابا و سولس نزد کشیشان روانه نمودند.
\par 30 پس چنین کردند و آن را به‌دست برنابا و سولس نزد کشیشان روانه نمودند.

\chapter{12}

\par 1 و در آن زمان هیرودیس پادشاه، دست تطاول بر بعضی از کلیسا دراز کرد
\par 2 ویعقوب برادر یوحنا را به شمشیر کشت.
\par 3 و چون دید که یهود را پسند افتاد، بر آن افزوده، پطرس رانیز گرفتار کرد و ایام فطیر بود.
\par 4 پس او را گرفته، در زندان انداخت و به چهار دسته رباعی سپاهیان سپرد که او را نگاهبانی کنند و اراده داشت که بعداز فصح او را برای قوم بیرون آورد.
\par 5 پس پطرس را در زندان نگاه می‌داشتند.
\par 6 و در شبی که هیرودیس قصد بیرون آوردن وی داشت، پطرس به دو زنجیر بسته، درمیان دو سپاهی خفته بود و کشیکچیان نزد درزندان را نگاهبانی می‌کردند.
\par 7 ناگاه فرشته خداوند نزد وی حاضر شد و روشنی در آن خانه درخشید. پس به پهلوی پطرس زده، او را بیدارنمود و گفت: «بزودی برخیز.» که در ساعت زنجیرها از دستش فرو ریخت.
\par 8 و فرشته وی راگفت: «کمر خود را ببند و نعلین برپا کن.» پس چنین کرد و به وی گفت: «ردای خود را بپوش و ازعقب من بیا.»
\par 9 پس بیرون شده، از عقب او روانه گردید و ندانست که آنچه از فرشته روی نمودحقیقی است بلکه گمان برد که خواب می‌بیند.
\par 10 پس از قراولان اول و دوم گذشته، به دروازه آهنی که به سوی شهر می‌رود رسیدند و آن خودبخود پیش روی ایشان باز شد؛ و از آن بیرون رفته، تا آخر یک کوچه برفتند که در ساعت فرشته از او غایب شد.
\par 11 آنگاه پطرس به خودآمده گفت: «اکنون به تحقیق دانستم که خداوندفرشته خود را فرستاده، مرا از دست هیرودیس واز تمامی انتظار قوم یهود رهانید.»
\par 12 چون این را دریافت، به خانه مریم مادریوحنای ملقب به مرقس آمد و در آنجا بسیاری جمع شده، دعا می‌کردند.
\par 13 چون او در خانه راکوبید، کنیزی رودا نام آمد تا بفهمد.
\par 14 چون آوازپطرس را شناخت، از خوشی در را باز نکرده، به اندرون شتافته، خبر داد که «پطرس به درگاه ایستاده است.»
\par 15 وی را گفتند: «دیوانه‌ای.» وچون تاکید کرد که چنین است، گفتند که فرشته اوباشد.»
\par 16 اما پطرس پیوسته در را می‌کوبید. پس در را گشوده، او را دیدند و در حیرت افتادند.
\par 17 اما او به‌دست خود به سوی ایشان اشاره کردکه خاموش باشند و بیان نمود که چگونه خدا او رااز زندان خلاصی داد و گفت: «یعقوب و سایربرادران را از این امور مطلع سازید.» پس بیرون شده، به‌جای دیگر رفت
\par 18 و چون روز شداضطرابی عظیم در سپاهیان افتاد که پطرس را چه شد.
\par 19 و هیرودیس چون او را طلبیده نیافت، کشیکچیان را بازخواست نموده، فرمود تا ایشان را به قتل رسانند؛ و خود از یهودیه به قیصریه کوچ کرده، در آنجا اقامت نمود.
\par 20 اما هیرودیس با اهل صور و صیدون خشمناک شد. پس ایشان به یکدل نزد او حاضرشدند و بلاستس ناظر خوابگاه پادشاه را با خودمتحد ساخته، طلب مصالحه کردند زیرا که دیارایشان از ملک پادشاه معیشت می‌یافت.
\par 21 و درروزی معین، هیرودیس لباس ملوکانه در بر کرد وبر مسند حکومت نشسته، ایشان را خطاب می‌کرد.
\par 22 و خلق ندا می‌کردند که آواز خداست نه آواز انسان.
\par 23 که در ساعت فرشته خداوند اورا زد زیرا که خدا را تمجید ننمود و کرم او راخورد که بمرد.
\par 24 اما کلام خدا نمو کرده، ترقی یافت.و برنابا و سولس چون آن خدمت را به انجام رسانیدند، از اورشلیم مراجعت کردند و یوحنای ملقب به مرقس را همراه خود بردند.
\par 25 و برنابا و سولس چون آن خدمت را به انجام رسانیدند، از اورشلیم مراجعت کردند و یوحنای ملقب به مرقس را همراه خود بردند.

\chapter{13}

\par 1 و در کلیسایی که در انطاکیه بود انبیا ومعلم چند بودند: برنابا و شمعون ملقب به نیجر و لوکیوس قیروانی و مناحم برادررضاعی هیرودیس تیترارخ و سولس.
\par 2 چون ایشان در عبادت خدا و روزه مشغول می‌بودند، روح‌القدس گفت: «برنابا و سولس را برای من جداسازید از بهر آن عمل که ایشان را برای آن خوانده‌ام.»
\par 3 آنگاه روزه گرفته و دعا کرده ودستها بر ایشان گذارده، روانه نمودند.
\par 4 پس ایشان از جانب روح‌القدس فرستاده شده، به سلوکیه رفتند و از آنجا از راه دریا به قپرس آمدند.
\par 5 و وارد سلامیس شده، در کنایس یهود به کلام خدا موعظه کردند و یوحنا ملازم ایشان بود.
\par 6 و چون در تمامی جزیره تا به پافس گشتند، در آنجا شخص یهودی را که جادوگر ونبی کاذب بود یافتند که نام او باریشوع بود.
\par 7 اورفیق سرجیوس پولس والی بود که مردی فهیم بود. همان برنابا و سولس را طلب نموده، خواست کلام خدا را بشنود.
\par 8 اما علیما یعنی آن جادوگر، زیرا ترجمه اسمش همچنین می‌باشد، ایشان رامخالفت نموده، خواست والی را از ایمان برگرداند.
\par 9 ولی سولس که پولس باشد، پر ازروح‌القدس شده، بر او نیک نگریسته،
\par 10 گفت: «ای پر از هر نوع مکر و خباثت، ای فرزند ابلیس ودشمن هر راستی، باز نمی ایستی از منحرف ساختن طرق راست خداوند؟
\par 11 الحال دست خداوند بر توست و کور شده، آفتاب را تا مدتی نخواهی دید.» که در همان ساعت، غشاوه وتاریکی او را فرو گرفت و دور زده، راهنمایی طلب می‌کرد.
\par 12 پس والی چون آن ماجرا را دید، از تعلیم خداوند متحیر شده، ایمان آورد.
\par 13 آنگاه پولس و رفقایش از پافس به کشتی سوار شده، به پرجه پمفلیه آمدند. اما یوحنا ازایشان جدا شده، به اورشلیم برگشت.
\par 14 و ایشان از پرجه عبور نموده، به انطاکیه پیسیدیه آمدند ودر روز سبت به کنیسه درآمده، بنشستند.
\par 15 وبعد از تلاوت تورات و صحف انبیا، روسای کنیسه نزد ایشان فرستاده، گفتند: «ای برادران عزیز، اگر کلامی نصیحت‌آمیز برای قوم دارید، بگویید.»
\par 16 پس پولس برپا ایستاده، به‌دست خوداشاره کرده، گفت: «ای مردان اسرائیلی وخداترسان، گوش دهید!
\par 17 خدای این قوم، اسرائیل، پدران ما را برگزیده، قوم را در غربت ایشان در زمین مصر سرافراز نمود و ایشان را به بازوی بلند از آنجا بیرون آورد؛
\par 18 و قریب به چهل سال در بیابان متحمل حرکات ایشان می‌بود.
\par 19 و هفت طایفه را در زمین کنعان هلاک کرده، زمین آنها را میراث ایشان ساخت تا قریب چهار صد و پنجاه سال.
\par 20 و بعد از آن بدیشان داوران داد تا زمان سموئیل نبی.
\par 21 و از آن وقت پادشاهی خواستند و خدا شاول بن قیس را ازسبط بنیامین تا چهل سال به ایشان داد.
\par 22 پس اورا از میان برداشته، داود را برانگیخت تا پادشاه ایشان شود و در حق او شهادت داد که "داود بن یسی را مرغوب دل خود یافته‌ام که به تمامی اراده من عمل خواهد کرد."
\par 23 و از ذریت او خدابرحسب وعده، برای اسرائیل نجات‌دهنده‌ای یعنی عیسی را آورد،
\par 24 چون یحیی پیش ازآمدن او تمام قوم اسرائیل را به تعمید توبه موعظه نموده بود.
\par 25 پس چون یحیی دوره خودرا به پایان برد، گفت: "مرا که می‌پندارید؟ من اونیستم، لکن اینک بعد از من کسی می‌آید که لایق گشادن نعلین او نیم."
\par 26 «ای برادران عزیز و ابنای آل ابراهیم وهرکه از شما خداترس باشد، مر شما را کلام این نجات فرستاده شد.
\par 27 زیرا سکنه اورشلیم وروسای ایشان، چونکه نه او را شناختند و نه آوازهای انبیا را که هر سبت خوانده می‌شود، بروی فتوی دادند و آنها را به اتمام رسانیدند.
\par 28 وهر‌چند هیچ علت قتل در وی نیافتند، از پیلاطس خواهش کردند که او کشته شود.
\par 29 پس چون آنچه درباره وی نوشته شده بود تمام کردند، او رااز صلیب پایین آورده، به قبر سپردند.
\par 30 لکن خدااو را از مردگان برخیزانید.
\par 31 و او روزهای بسیار ظاهر شد بر آنانی که همراه او از جلیل به اورشلیم آمده بودند که الحال نزد قوم شهود او می‌باشند.
\par 32 پس ما به شما بشارت می‌دهیم، بدان وعده‌ای که به پدران ما داده شد،
\par 33 که خدا آن را به ما که فرزندان ایشان می‌باشیم وفا کرد، وقتی که عیسی را برانگیخت، چنانکه در زبور دوم مکتوب است که "تو پسر من هستی، من امروز تو را تولیدنمودم."
\par 34 و در آنکه او را از مردگان برخیزانید تادیگر هرگز راجع به فساد نشود چنین گفت که "به برکات قدوس و امین داود برای شما وفا خواهم کرد."
\par 35 بنابراین در جایی دیگر نیز می‌گوید: "توقدوس خود را نخواهی گذاشت که فساد را بیند."
\par 36 زیرا که داود چونکه در زمان خود اراده خدا راخدمت کرده بود، به خفت و به پدران خود ملحق شده، فساد را دید.
\par 37 لیکن آن کس که خدا او رابرانگیخت، فساد را ندید.
\par 38 «پس‌ای برادران عزیز، شما را معلوم باد که به وساطت او به شما از آمرزش گناهان اعلام می‌شود.
\par 39 و به وسیله او هر‌که ایمان آورد، عادل شمرده می‌شود، از هر چیزی که به شریعت موسی نتوانستید عادل شمرده شوید.
\par 40 پس احتیاط کنید، مبادا آنچه در صحف انبیامکتوب است، بر شما واقع شود،
\par 41 که "ای حقیرشمارندگان، ملاحظه کنید و تعجب نمایید و هلاک شوید زیرا که من عملی را درایام شما پدید آرم، عملی که هر‌چند کسی شما را از آن اعلام نماید، تصدیق نخواهیدکرد.»
\par 42 پس چون از کنیسه بیرون می‌رفتند، خواهش نمودند که در سبت آینده هم این سخنان را بدیشان بازگویند.
\par 43 و چون اهل کنیسه متفرق شدند، بسیاری از یهودیان و جدیدان خداپرست از عقب پولس و برنابا افتادند؛ و آن دو نفر به ایشان سخن گفته، ترغیب می‌نمودند که به فیض خداثابت باشید.
\par 44 اما در سبت دیگر قریب به تمامی شهر فراهم شدند تا کلام خدا را بشنوند.
\par 45 ولی چون یهود ازدحام خلق را دیدند، از حسد پرگشتند و کفر گفته، با سخنان پولس مخالفت کردند.
\par 46 آنگاه پولس و برنابا دلیر شده، گفتند: «واجب بود کلام خدا نخست به شما القا شود. لیکن چون آن را رد کردید و خود را ناشایسته حیات جاودانی شمردید، همانا به سوی امت هاتوجه نماییم.
\par 47 زیرا خداوند به ما چنین امرفرمود که "تو را نور امت‌ها ساختم تا الی اقصای زمین منشا نجات باشی."»
\par 48 چون امت‌ها این راشنیدند، شادخاطر شده، کلام خداوند را تمجیدنمودند و آنانی که برای حیات جاودانی مقرربودند، ایمان آوردند.
\par 49 و کلام خدا در تمام آن نواحی منتشرگشت.
\par 50 اما یهودیان چند زن دیندار و متشخص و اکابر شهر را بشورانیدند و ایشان را به زحمت رسانیدن بر پولس و برنابا تحریض نموده، ایشان را از حدود خود بیرون کردند.
\par 51 و ایشان خاک پایهای خود را بر ایشان افشانده، به ایقونیه آمدند.و شاگردان پر از خوشی و روح‌القدس گردیدند.
\par 52 و شاگردان پر از خوشی و روح‌القدس گردیدند.

\chapter{14}

\par 1 اما در ایقونیه، ایشان با هم به کنیسه یهود در‌آمده، به نوعی سخن‌گفتند که جمعی کثیر از یهود و یونانیان ایمان آوردند.
\par 2 لیکن یهودیان بی‌ایمان دلهای امت‌ها را اغوانمودند و با برادران بداندیش ساختند.
\par 3 پس مدت مدیدی توقف نموده، به نام خداوندی که به کلام فیض خود شهادت می‌داد، به دلیری سخن می‌گفتند و او آیات و معجزات عطا می‌کرد که ازدست ایشان ظاهر شود.
\par 4 و مردم شهر دو فرقه شدند، گروهی همداستان یهود و جمعی با رسولان بودند.
\par 5 وچون امت‌ها و یهود با روسای خود بر ایشان هجوم می‌آوردند تا ایشان را افتضاح نموده، سنگسار کنند،
\par 6 آگاهی یافته، به سوی لستره ودربه شهرهای لیکاونیه و دیار آن نواحی فرارکردند.
\par 7 و در آنجا بشارت می‌دادند.
\par 8 و در لستره مردی نشسته بود که پایهایش بی‌حرکت بود و از شکم مادر، لنگ متولد شده، هرگز راه نرفته بود.
\par 9 چون او سخن پولس رامی شنید، او بر وی نیک نگریسته، دید که ایمان شفا یافتن را دارد.
\par 10 پس به آواز بلند بدو گفت: «بر پایهای خود راست بایست.» که در ساعت برجسته، خرامان گردید.
\par 11 اما خلق چون این عمل پولس را دیدند، صدای خود را به زبان لیکاونیه بلند کرده، گفتند: «خدایان به صورت انسان نزد مانازل شده‌اند.»
\par 12 پس برنابا را مشتری و پولس راعطارد خواندند زیرا که او در سخن‌گفتن مقدم بود.
\par 13 پس کاهن مشتری که پیش شهر ایشان بود، گاوان و تاجها با گروه هایی از خلق به دروازه هاآورده، خواست که قربانی گذراند.
\par 14 اما چون آن دو رسول یعنی برنابا و پولس شنیدند، جامه های خود را دریده، در میان مردم افتادند و ندا کرده،
\par 15 گفتند: «ای مردمان، چرا چنین می‌کنید؟ ما نیزانسان و صاحبان علتها مانند شما هستیم و به شمابشارت می‌دهیم که از این اباطیل رجوع کنید به سوی خدای حی که آسمان و زمین و دریا و آنچه را که در آنها است آفرید،
\par 16 که در طبقات سلف همه امت‌ها را واگذاشت که در طرق خود رفتارکنند،
\par 17 با وجودی که خود را بی‌شهادت نگذاشت، چون احسان می‌نمود و از آسمان باران بارانیده و فصول بارآور بخشیده، دلهای ما را ازخوراک و شادی پر می‌ساخت.»
\par 18 و بدین سخنان خلق را از گذرانیدن قربانی برای ایشان به دشواری باز داشتند.
\par 19 اما یهودیان از انطاکیه و ایقونیه آمده، مردم را با خود متحد ساختند و پولس را سنگسارکرده، از شهر بیرون کشیدند و پنداشتند که مرده است.
\par 20 اما چون شاگردان گرد او ایستادندبرخاسته، به شهر درآمد و فردای آن روز با برنابابه سوی دربه روانه شد
\par 21 و در آن شهر بشارت داده، بسیاری را شاگرد ساختند. پس به لستره وایقونیه و انطاکیه مراجعت کردند.
\par 22 و دلهای شاگردان را تقویت داده، پند می‌دادند که در ایمان ثابت بمانند و اینکه با مصیبتهای بسیار می‌بایدداخل ملکوت خدا گردیم.
\par 23 و در هر کلیسابجهت ایشان کشیشان معین نمودند و دعا و روزه داشته، ایشان را به خداوندی که بدو ایمان آورده بودند، سپردند.
\par 24 و از پیسیدیه گذشته به پمفلیه آمدند.
\par 25 و در پرجه به کلام موعظه نمودند و به اتالیه فرود آمدند.
\par 26 و از آنجا به کشتی سوار شده، به انطاکیه آمدند که از همان جا ایشان را به فیض خدا سپرده بودند برای آن کاری که به انجام رسانیده بودند.
\par 27 و چون وارد شهر شدند کلیسا را جمع کرده، ایشان را مطلع ساختند از آنچه خدا با ایشان کرده بود و چگونه دروازه ایمان را برای امت‌ها باز کرده بود.پس مدت مدیدی با شاگردان بسر بردند.
\par 28 پس مدت مدیدی با شاگردان بسر بردند.

\chapter{15}

\par 1 و تنی چند از یهودیه آمده، برادران راتعلیم می‌دادند که «اگر برحسب آیین موسی مختون نشوید، ممکن نیست که نجات یابید.»
\par 2 چون پولس و برنابا را منازعه و مباحثه بسیار با ایشان واقع شد، قرار بر این شد که پولس و برنابا و چند نفر دیگر از ایشان نزد رسولان وکشیشان در اورشلیم برای این مساله بروند.
\par 3 پس کلیسا ایشان را مشایعت نموده از فینیقیه و سامره عبور کرده، ایمان آوردن امت‌ها را بیان کردند وهمه برادران را شادی عظیم دادند.
\par 4 و چون وارد اورشلیم شدند، کلیسا ورسولان و کشیشان ایشان را پذیرفتند و آنها را ازآنچه خدا با ایشان کرده بود، خبر دادند.
\par 5 آنگاه بعضی از فرقه فریسیان که ایمان آورده بودند، برخاسته، گفتند: «اینها را باید ختنه نمایند و امر کنند که سنن موسی را نگاه دارند.»
\par 6 پس رسولان و کشیشان جمع شدند تا در این امر مصلحت بینند.
\par 7 و چون مباحثه سخت شد، پطرس برخاسته، بدیشان گفت: «ای برادران عزیز، شما آگاهید که از ایام اول، خدا از میان شمااختیار کرد که امت‌ها از زبان من کلام بشارت رابشنوند و ایمان آورند.
\par 8 و خدای عارف‌القلوب بر ایشان شهادت داد بدین که روح‌القدس رابدیشان داد، چنانکه به ما نیز.
\par 9 و در میان ما وایشان هیچ فرق نگذاشت، بلکه محض ایمان دلهای ایشان را طاهر نمود.
\par 10 پس اکنون چراخدا را امتحان می‌کنید که یوغی بر گردن شاگردان می‌نهید که پدران ما و ما نیز طاقت تحمل آن رانداشتیم،
\par 11 بلکه اعتقاد داریم که محض فیض خداوند عیسی مسیح نجات خواهیم یافت، همچنان‌که ایشان نیز.»
\par 12 پس تمام جماعت ساکت شده، به برنابا وپولس گوش گرفتند چون آیات و معجزات رابیان می‌کردند که خدا در میان امت‌ها به وساطت ایشان ظاهر ساخته بود.
\par 13 پس چون ایشان ساکت شدند، یعقوب رو آورده، گفت: «ای برادران عزیز، مرا گوش گیرید.
\par 14 شمعون بیان کرده است که چگونه خدا اول امت‌ها را تفقدنمود تا قومی از ایشان به نام خود بگیرد.
\par 15 وکلام انبیا در این مطابق است چنانکه مکتوب است
\par 16 که «بعد از این رجوع نموده، خیمه داود را که افتاده است باز بنا می‌کنم و خرابیهای آن را باز بنامی کنم و آن را برپا خواهم کرد،
\par 17 تا بقیه مردم طالب خداوند شوند و جمیع امت هایی که بر آنهانام من نهاده شده است.»
\par 18 این را می‌گوید خداوندی که این چیزها را از بدو عالم معلوم کرده است.
\par 19 پس رای من این است: کسانی را که ازامت‌ها به سوی خدا بازگشت می‌کنند زحمت نرسانیم،
\par 20 مگر اینکه ایشان را حکم کنیم که ازنجاسات بتها و زنا و حیوانات خفه شده و خون بپرهیزند.
\par 21 زیرا که موسی از طبقات سلف در هرشهر اشخاصی دارد که بدو موعظه می‌کنند، چنانکه در هر سبت در کنایس او را تلاوت می‌کنند.»
\par 22 آنگاه رسولان و کشیشان با تمامی کلیسابدین رضا دادند که چند نفر از میان خود انتخاب نموده، همراه پولس و برنابا به انطاکیه بفرستند، یعنی یهودای ملقب به برسابا و سیلاس که ازپیشوایان برادران بودند.
\par 23 و بدست ایشان نوشتند که «رسولان و کشیشان و برادران، به برادران از امت‌ها که در انطاکیه و سوریه و قیلیقیه می‌باشند، سلام می‌رسانند.
\par 24 چون شنیده شدکه بعضی از میان ما بیرون رفته، شما را به سخنان خود مشوش ساخته، دلهای شما رامنقلب می‌نمایند و می‌گویند که می‌باید مختون شده، شریعت را نگاه بدارید و ما به ایشان هیچ امر نکردیم.
\par 25 لهذا ما به یک دل مصلحت دیدیم که چند نفر را اختیار نموده، همراه عزیزان خود برنابا و پولس به نزد شمابفرستیم،
\par 26 اشخاصی که جانهای خود را در راه نام خداوند ما عیسی مسیح تسلیم کرده‌اند.
\par 27 پس یهودا و سیلاس را فرستادیم و ایشان شمارا از این امور زبانی خواهند آگاهانید.
\par 28 زیراکه روح‌القدس و ما صواب دیدیم که باری بر شماننهیم جز این ضروریات
\par 29 که از قربانی های بتها و خون و حیوانات خفه شده و زنا بپرهیزید که هرگاه از این امور خود را محفوظ دارید به نیکویی خواهید پرداخت والسلام.»
\par 30 پس ایشان مرخص شده، به انطاکیه آمدندو جماعت را فراهم آورده، نامه را رسانیدند.
\par 31 چون مطالعه کردند، از این تسلی شادخاطرگشتند.
\par 32 و یهودا و سیلاس چونکه ایشان هم نبی بودند، برادران را به سخنان بسیار، نصیحت وتقویت نمودند.
\par 33 پس چون مدتی در آنجا بسربردند به سلامتی از برادران رخصت گرفته، به سوی فرستندگان خود توجه نمودند.
\par 34 اماپولس و برنابا در انطاکیه توقف نموده،
\par 35 بابسیاری دیگر تعلیم و بشارت به کلام خدامی دادند.
\par 36 و بعد از ایام چند پولس به برنابا گفت: «برگردیم و برادران را در هر شهری که در آنها به کلام خداوند اعلام نمودیم، دیدن کنیم که چگونه می‌باشند.»
\par 37 اما برنابا چنان مصلحت دید که یوحنای ملقب به مرقس را همراه نیز بردارد.
\par 38 لیکن پولس چنین صلاح دانست که شخصی راکه از پمفلیه از ایشان جدا شده بود و با ایشان درکار همراهی نکرده بود، با خود نبرد.
\par 39 پس نزاعی سخت شد بحدی که از یکدیگر جدا شده، برنابا مرقس را برداشته، به قپرس از راه دریا رفت.
\par 40 اما پولس سیلاس را اختیار کرد و از برادران به فیض خداوند سپرده شده، رو به سفر نهاد.و ازسوریه و قیلیقیه عبور کرده، کلیساها را استوارمی نمود.
\par 41 و ازسوریه و قیلیقیه عبور کرده، کلیساها را استوارمی نمود.

\chapter{16}

\par 1 و به دربه و لستره آمد که اینک شاگردی تیموتاوس نام آنجا بود، پسر زن یهودیه مومنه لیکن پدرش یونانی بود.
\par 2 که برادران در لستره و ایقونیه بر او شهادت می‌دادند.
\par 3 چون پولس خواست او همراه وی بیاید، او راگرفته مختون ساخت، به‌سبب یهودیانی که در آن نواحی بودند زیرا که همه پدرش را می‌شناختندکه یونانی بود.
\par 4 و در هر شهری که می‌گشتند، قانونها را که رسولان و کشیشان در اورشلیم حکم فرموده بودند، بدیشان می‌سپردند تا حفظ نمایند.
\par 5 پس کلیساها در ایمان استوار می‌شدند و روزبروز در شماره افزوده می‌گشتند.
\par 6 و چون از فریجیه و دیار غلاطیه عبورکردند، روح‌القدس ایشان را از رسانیدن کلام به آسیا منع نمود.
\par 7 پس به میسیا آمده، سعی نمودندکه به بطینیا بروند، لیکن روح عیسی ایشان رااجازت نداد.
\par 8 و از میسیا گذشته به ترواس رسیدند.
\par 9 شبی پولس را رویایی رخ نمود که شخصی از اهل مکادونیه ایستاده بود التماس نموده گفت: «به مکادونیه آمده، ما را امداد فرما.»
\par 10 چون این‌رویا را دید، بی‌درنگ عازم سفرمکادونیه شدیم، زیرا به یقین دانستیم که خداوندما را خوانده است تا بشارت بدیشان رسانیم.
\par 11 پس از ترواس به کشتی نشسته، به راه مستقیم به ساموتراکی رفتیم و روز دیگر به نیاپولیس.
\par 12 و از آنجا به فیلپی رفتیم که شهر اول ازسرحد مکادونیه و کلونیه است و در آن شهر‌چندروز توقف نمودیم.
\par 13 و در روز سبت از شهربیرون شده و به کنار رودخانه جایی که نماز می گذاردند، نشسته با زنانی که در آنجا جمع می‌شدند سخن راندیم.
\par 14 و زنی لیدیه نام، ارغوان فروش، که از شهر طیاتیرا و خداپرست بود، می‌شنید که خداوند دل او را گشود تا سخنان پولس را بشنود.
\par 15 و چون او و اهل خانه‌اش تعمید یافتند، خواهش نموده، گفت: «اگر شما رایقین است که به خداوند ایمان آوردم، به خانه من درآمده، بمانید.» و ما را الحاح نمود.
\par 16 و واقع شد که چون ما به محل نمازمی رفتیم، کنیزی که روح تفال داشت و ازغیب گویی منافع بسیار برای آقایان خود پیدامی نمود، به ما برخورد.
\par 17 و از عقب پولس و ماآمده، ندا کرده، می‌گفت که «این مردمان خدام خدای تعالی می‌باشند که شما را از طریق نجات اعلام می‌نمایند.»
\par 18 و چون این کار را روزهای بسیار می‌کرد، پولس دلتنگ شده، برگشت و به روح گفت: «تو را می‌فرمایم به نام عیسی مسیح ازاین دختر بیرون بیا.» که در ساعت از او بیرون شد.
\par 19 اما چون آقایانش دیدند که از کسب خودمایوس شدند، پولس و سیلاس را گرفته، در بازارنزد حکام کشیدند.
\par 20 و ایشان را نزد والیان حاضرساخته، گفتند: «این دو شخص شهر ما را به شورش آورده‌اند و از یهود هستند،
\par 21 و رسومی را اعلام می‌نمایند که پذیرفتن و به‌جا آوردن آنهابر ما که رومیان هستیم، جایز نیست.»
\par 22 پس خلق بر ایشان هجوم آوردند و والیان جامه های ایشان را کنده، فرمودند ایشان را چوب بزنند.
\par 23 و چون ایشان را چوب بسیار زدند، به زندان افکندند و داروغه زندان را تاکید فرمودند که ایشان را محکم نگاه دارد.
\par 24 و چون او بدینطورامر یافت، ایشان را به زندان درونی انداخت وپایهای ایشان را در کنده مضبوط کرد.
\par 25 اما قریب به نصف شب، پولس و سیلاس دعا کرده، خدا را تسبیح می‌خواندند و زندانیان ایشان را می‌شنیدند.
\par 26 که ناگاه زلزله‌ای عظیم حادث گشت بحدی که بیناد زندان به جنبش درآمد و دفعه همه درها باز شد و زنجیرها از همه فرو ریخت.
\par 27 اما داروغه بیدار شده، چون درهای زندان را گشوده دید، شمشیر خود راکشیده، خواست خود را بکشد زیرا گمان برد که زندانیان فرار کرده‌اند.
\par 28 اما پولس به آواز بلندصدا زده، گفت: «خود را ضرری مرسان زیرا که ماهمه در اینجا هستیم.»
\par 29 پس چراغ طلب نموده، به اندرون جست و لرزان شده، نزد پولس وسیلاس افتاد.
\par 30 و ایشان را بیرون آورده، گفت: «ای آقایان، مرا چه باید کرد تا نجات یابم؟»
\par 31 گفتند: «به خداوند عیسی مسیح ایمان آور که تو و اهل خانه ات نجات خواهید یافت.»
\par 32 آنگاه کلام خداوند را برای او وتمامی اهل بیتش بیان کردند.
\par 33 پس ایشان را برداشته، در همان ساعت شب زخمهای ایشان را شست و خود و همه کسانش فی الفور تعمید یافتند.
\par 34 و ایشان را به خانه خود درآورده، خوانی پیش ایشان نهاد و باتمامی عیال خود به خدا ایمان آورده، شادگردیدند.
\par 35 اما چون روز شد، والیان فراشان فرستاده، گفتند: «آن دو شخص را رها نما.»
\par 36 آنگاه داروغه پولس را از این سخنان آگاهانیدکه «والیان فرستاده‌اند تا رستگار شوید. پس الان بیرون آمده، به سلامتی روانه شوید.»
\par 37 لیکن پولس بدیشان گفت: «ما را که مردمان رومی می‌باشیم، آشکارا و بی‌حجت زده، به زندان انداختند. آیا الان ما را به پنهانی بیرون می‌نمایند؟ نی بلکه خود آمده، ما را بیرون بیاورند.
\par 38 پس فراشان این سخنان را به والیان گفتند و چون شنیدند که رومی هستند بترسیدند
\par 39 و آمده، بدیشان التماس نموده، بیرون آوردندو خواهش کردند که از شهر بروند.آنگاه اززندان بیرون آمده، به خانه لیدیه شتافتند و بابرادران ملاقات نموده و ایشان را نصیحت کرده، روانه شدند.
\par 40 آنگاه اززندان بیرون آمده، به خانه لیدیه شتافتند و بابرادران ملاقات نموده و ایشان را نصیحت کرده، روانه شدند.

\chapter{17}

\par 1 و از امفپولس و اپلونیه گذشته، به تسالونیکی رسیدند که در آنجا کنیسه یهود بود.
\par 2 پس پولس برحسب عادت خود، نزدایشان داخل شده، در سه سبت با ایشان از کتاب مباحثه می‌کرد
\par 3 و واضح و مبین می‌ساخت که «لازم بود مسیح زحمت بیند و از مردگان برخیزدو عیسی که خبر او را به شما می‌دهم، این مسیح است.»
\par 4 و بعضی از ایشان قبول کردند و با پولس و سیلاس متحد شدند و از یونانیان خداترس، گروهی عظیم و از زنان شریف، عددی کثیر.
\par 5 امایهودیان بی‌ایمان حسد برده، چند نفر اشرار ازبازاریها را برداشته، خلق را جمع کرده، شهر را به شورش آوردند و به خانه یاسون تاخته، خواستندایشان را در میان مردم ببرند.
\par 6 و چون ایشان رانیافتند، یاسون و چند برادر را نزد حکام شهرکشیدند و ندا می‌کردند که «آنانی که ربع مسکون را شورانیده‌اند، حال بدینجا نیزآمده‌اند.
\par 7 و یاسون ایشان را پذیرفته است وهمه اینها برخلاف احکام قیصر عمل می‌کنند وقایل بر این هستند که پادشاهی دیگر هست یعنی عیسی.»
\par 8 پس خلق و حکام شهر را ازشنیدن این سخنان مضطرب ساختند
\par 9 و ازیاسون و دیگران کفالت گرفته، ایشان را رهاکردند.
\par 10 اما برادران بی‌درنگ در شب پولس وسیلاس را به سوی بیریه روانه کردند و ایشان بدانجا رسیده، به کنیسه یهود درآمدند.
\par 11 و اینهااز اهل تسالونیکی نجیب‌تر بودند، چونکه درکمال رضامندی کلام را پذیرفتند و هر روز کتب را تفتیش می‌نمودند که آیا این همچنین است.
\par 12 پس بسیاری از ایشان ایمان آوردند و از زنان شریف یونانیه و از مردان، جمعی عظیم.
\par 13 لیکن چون یهودیان تسالونیکی فهمیدند که پولس در بیریه نیز به کلام خدا موعظه می‌کند، درآنجا هم رفته، خلق را شورانیدند.
\par 14 در ساعت برادران پولس را به سوی دریا روانه کردند ولی سیلاس با تیموتاوس در آنجا توقف نمودند.
\par 15 ورهنمایان پولس او را به اطینا آوردند و حکم برای سیلاس و تیموتاوس گرفته که به زودی هر‌چه تمام تر به نزد او آیند، روانه شدند.
\par 16 اما چون پولس در اطینا انتظار ایشان رامی کشید، روح او در اندرونش مضطرب گشت چون دید که شهر از بتها پر است.
\par 17 پس درکنیسه با یهودیان و خداپرستان و در بازار، هرروزه با هر‌که ملاقات می‌کرد، مباحثه می‌نمود.
\par 18 اما بعضی از فلاسفه اپیکوریین و رواقیین با اوروبرو شده، بعضی می‌گفتند: «این یاوه‌گو چه می‌خواهد بگوید؟» و دیگران گفتند: «ظاهر واعظ به خدایان غریب است.» زیرا که ایشان را به عیسی و قیامت بشارت می‌داد.
\par 19 پس او راگرفته، به کوه مریخ بردند و گفتند: «آیا می‌توانیم یافت که این تعلیم تازه‌ای که تو می‌گویی چیست؟
\par 20 چونکه سخنان غریب به گوش مامی رسانی. پس می‌خواهیم بدانیم از اینها چه مقصود است.»
\par 21 اما جمیع اهل اطینا و غربای ساکن آنجا جز برای گفت و شنید درباره چیزهای تازه فراغتی نمی داشتند.
\par 22 پس پولس در وسط کوه مریخ ایستاده، گفت: «ای مردان اطینا، شما را از هر جهت بسیاردیندار یافته‌ام،
\par 23 زیرا چون سیر کرده، معابدشما را نظاره می‌نمودم، مذبحی یافتم که بر آن، نام خدای ناشناخته نوشته بود. پس آنچه را شماناشناخته می‌پرستید، من به شما اعلام می‌نمایم.
\par 24 خدایی که جهان و آنچه در آن است آفرید، چونکه او مالک آسمان و زمین است، درهیکلهای ساخته شده به‌دستها ساکن نمی باشد
\par 25 و از دست مردم خدمت کرده نمی شود که گویامحتاج چیزی باشد، بلکه خود به همگان حیات ونفس و جمیع چیزها می‌بخشد.
\par 26 و هر امت انسان را از یک خون ساخت تا بر تمامی روی زمین مسکن گیرند و زمانهای معین و حدودمسکنهای ایشان را مقرر فرمود
\par 27 تا خدا را طلب کنند که شاید او را تفحص کرده، بیابند، با آنکه ازهیچ‌یکی از ما دور نیست.
\par 28 زیرا که در او زندگی و حرکت و وجود داریم چنانکه بعضی از شعرای شما نیز گفته‌اند که از نسل او می‌باشیم.
\par 29 پس چون از نسل خدا می‌باشیم، نشاید گمان برد که الوهیت شباهت دارد به طلا یا نقره یا سنگ منقوش به صنعت یا مهارت انسان.
\par 30 پس خدا اززمانهای جهالت چشم پوشیده، الان تمام خلق رادر هر جا حکم می‌فرماید که توبه کنند.
\par 31 زیرا روزی را مقرر فرمود که در آن ربع مسکون را به انصاف داوری خواهد نمود به آن مردی که معین فرمود و همه را دلیل داد به اینکه او را از مردگان برخیزانید.»
\par 32 چون ذکر قیامت مردگان شنیدند، بعضی استهزا نمودند و بعضی گفتند مرتبه دیگر در این امر از تو خواهیم شنید.
\par 33 و همچنین پولس ازمیان ایشان بیرون رفت.لیکن چند نفر بدوپیوسته ایمان آوردند که از‌جمله ایشان دیونیسیوس آریوپاغی بود و زنی که دامرس نام داشت و بعضی دیگر با ایشان.
\par 34 لیکن چند نفر بدوپیوسته ایمان آوردند که از‌جمله ایشان دیونیسیوس آریوپاغی بود و زنی که دامرس نام داشت و بعضی دیگر با ایشان.

\chapter{18}

\par 1 و بعد از آن پولس از اطینا روانه شده، به قرنتس آمد.
\par 2 و مرد یهودی اکیلا نام راکه مولدش پنطس بود و از ایطالیا تازه رسیده بودو زنش پرسکله را یافت زیرا کلودیوس فرمان داده بود که همه یهودیان از روم بروند. پس نزدایشان آمد.
\par 3 و چونکه با ایشان همپیشه بود، نزدایشان مانده، به‌کار مشغول شد و کسب ایشان خیمه‌دوزی بود.
\par 4 و هر سبت در کنیسه مکالمه کرده، یهودیان و یونانیان را مجاب می‌ساخت.
\par 5 اما چون سیلاس و تیموتاوس از مکادونیه آمدند، پولس در روح مجبور شده، برای یهودیان شهادت می‌داد که عیسی، مسیح است.
\par 6 ولی چون ایشان مخالفت نموده، کفر می‌گفتند، دامن خود را بر ایشان افشانده، گفت: «خون شما بر سرشما است. من بری هستم. بعد از این به نزد امت هامی روم.»
\par 7 پس از آنجا نقل کرده، به خانه شخصی یوستس نام خداپرست آمد که خانه او متصل به کنیسه بود.
\par 8 اما کرسپس، رئیس کنیسه با تمامی اهل بیتش به خداوند ایمان آوردند و بسیاری ازاهل قرنتس چون شنیدند، ایمان آورده، تعمیدیافتند.
\par 9 شبی خداوند در رویا به پولس گفت: «ترسان مباش، بلکه سخن بگو و خاموش مباش
\par 10 زیرا که من با تو هستم و هیچ‌کس تو را اذیت نخواهد رسانید زیرا که مرا در این شهر خلق بسیار است.»
\par 11 پس مدت یک سال و شش ماه توقف نموده، ایشان را به کلام خدا تعلیم می‌داد.
\par 12 اماچون غالیون والی اخائیه بود، یهودیان یکدل شده، بر سر پولس تاخته، او را پیش مسند حاکم بردند
\par 13 و گفتند: «این شخص مردم را اغوامی کند که خدا را برخلاف شریعت عبادت کنند.»
\par 14 چون پولس خواست حرف زند، غالیون گفت: «ای یهودیان اگر ظلمی یا فسقی فاحش می‌بود، هر آینه شرط عقل می‌بود که متحمل شما بشوم.
\par 15 ولی چون مساله‌ای است درباره سخنان ونامها و شریعت شما، پس خود بفهمید. من درچنین امور نمی خواهم داوری کنم.»
\par 16 پس ایشان را از پیش مسند براند.
\par 17 و همه سوستانیس رئیس کنسیه را گرفته او را در مقابل مسند والی بزدند و غالیون را از این امور هیچ پروانبود.
\par 18 اما پولس بعد از آن روزهای بسیار در آنجاتوقف نمود پس برادران را وداع نموده، به سوریه از راه دریا رفت و پرسکله و اکیلا همراه او رفتند. و درکنخریه موی خود را چید چونکه نذر کرده بود.
\par 19 و چون به افسس رسید آن دو نفر را درآنجا رها کرده، خود به کنیسه درآمده، با یهودیان مباحثه نمود.
\par 20 و چون ایشان خواهش نمودندکه مدتی با ایشان بماند، قبول نکرد
\par 21 بلکه ایشان را وداع کرده، گفت که «مرا به هر صورت باید عیدآینده را در اورشلیم صرف کنم. لیکن اگر خدابخواهد، باز به نزد شما خواهم برگشت.» پس ازافسس روانه شد.
\par 22 و به قیصریه فرود آمده (به اورشلیم ) رفت و کلیسا را تحیت نموده، به انطاکیه آمد.
\par 23 و مدتی در آنجا مانده، باز به سفر توجه نمود و در ملک غلاطیه و فریجیه جابجا می‌گشت و همه شاگردان را استوار می‌نمود.
\par 24 اما شخصی یهود اپلس نام از اهل اسکندریه که مردی فصیح و در کتاب توانا بود، به افسس رسید.
\par 25 او درطریق خداوند تربیت یافته و در روح سرگرم بوده، درباره خداوند به دقت تکلم و تعلیم می‌نمود هر‌چند چز از تعمید یحیی اطلاعی نداشت.
\par 26 همان شخص در کنیسه به دلیری سخن آغاز کرد اما چون پرسکله و اکیلا او راشنیدند، نزد خود آوردند و به دقت تمام طریق خدا را بدو آموختند.
\par 27 پس چون اوعزیمت سفر اخائیه کرد، برادران او راترغیب نموده، به شاگردان سفارش نامه‌ای نوشتند که او را بپذیرند. و چون بدانجا رسیدآنانی را که به وسیله فیض ایمان آورده بودند، اعانت بسیار نمود،زیرا به قوت تمام بر یهوداقامه حجت می‌کرد و از کتب ثابت می‌نمود که عیسی، مسیح است.
\par 28 زیرا به قوت تمام بر یهوداقامه حجت می‌کرد و از کتب ثابت می‌نمود که عیسی، مسیح است.

\chapter{19}

\par 1 و چون اپلس در قرنتس بود، پولس درنواحی بالا گردش کرده، به افسس رسید. و در آنجا شاگرد چند یافته،
\par 2 بدیشان گفت: «آیا هنگامی که ایمان آوردید، روح‌القدس را یافتید؟» به وی گفتند: «بلکه نشنیدیم که روح‌القدس هست!»
\par 3 بدیشان گفت: «پس به چه چیز تعمید یافتید؟» گفتند: «به تعمید یحیی.»
\par 4 پولس گفت: «یحیی البته تعمید توبه می‌داد و به قوم می‌گفت به آن کسی‌که بعد از من می‌آید ایمان بیاورید یعنی به مسیح عیسی.»
\par 5 چون این راشنیدند به نام خداوند عیسی تعمید گرفتند،
\par 6 وچون پولس دست بر ایشان نهاد، روح‌القدس برایشان نازل شد و به زبانها متکلم گشته، نبوت کردند.
\par 7 و جمله آن مردمان تخمین دوازده نفربودند.
\par 8 پس به کنیسه درآمده، مدت سه ماه به دلیری سخن می‌راند و در امور ملکوت خدا مباحثه می‌نمود و برهان قاطع می‌آورد.
\par 9 اما چون بعضی سخت دل گشته، ایمان نیاوردند و پیش روی خلق، طریقت را بد می‌گفتند، از ایشان کناره گزیده، شاگردان را جدا ساخت و هر روزه درمدرسه شخصی طیرانس نام مباحثه می‌نمود.
\par 10 و بدینطور دو سال گذشت بقسمی که تمامی اهل آسیا چه یهود و چه یونانی کلام خداوندعیسی را شنیدند.
\par 11 و خداوند از دست پولس معجزات غیرمعتاد به ظهور می‌رسانید،
\par 12 بطوری که از بدن او دستمالها و فوطه‌ها برده، بر مریضان می‌گذاردند و امراض از ایشان زایل می شد و ارواح پلید از ایشان اخراج می‌شدند.
\par 13 لیکن تنی چند از یهودیان سیاح عزیمه خوان بر آنانی که ارواح پلید داشتند، نام خداوند عیسی را خواندن گرفتند و می‌گفتند: «شما را به آن عیسی که پولس به او موعظه می‌کندقسم می‌دهیم!»
\par 14 و هفت نفر پسران اسکیوارئیس کهنه یهود این کار می‌کردند.
\par 15 اما روح خبیث در جواب ایشان گفت: «عیسی رامی شناسم و پولس را می‌دانم. لیکن شماکیستید؟»
\par 16 و آن مرد که روح پلید داشت برایشان جست و بر ایشان زورآور شده، غلبه یافت بحدی که از آن خانه عریان و مجروح فرار کردند.
\par 17 چون این واقعه بر جمیع یهودیان و یونانیان ساکن افسس مشهور گردید، خوف بر همه ایشان طاری گشته، نام خداوند عیسی را مکرم می‌داشتند.
\par 18 و بسیاری از آنانی که ایمان آورده بودند آمدند و به اعمال خود اعتراف کرده، آنها رافاش می‌نمودند.
\par 19 و جمعی از شعبده بازان کتب خویش را آورده، در حضور خلق سوزانیدند وچون قیمت آنها را حساب کردند، پنجاه هزاردرهم بود
\par 20 بدینطور کلام خداوند ترقی کرده قوت می‌گرفت.
\par 21 و بعد از تمام شدن این مقدمات، پولس درروح عزیمت کرد که از مکادونیه و اخائیه گذشته، به اورشلیم برود و گفت: «بعد از رفتنم به آنجا روم را نیز باید دید.»
\par 22 پس دو نفر از ملازمان خودیعنی تیموتاوس و ارسطوس را به مکادونیه روانه کرد و خود در آسیا چندی توقف نمود.
\par 23 در آن زمان هنگامه‌ای عظیم درباره طریقت بر پا شد.
\par 24 زیرا شخصی دیمیتریوس نام زرگر که تصاویربتکده ارطامیس از نقره می‌ساخت و بجهت صنعتگران نفع خطیر پیدا می‌نمود، ایشان را و دیگرانی که در چنین پیشه اشتغال می‌داشتند،
\par 25 فراهم آورده، گفت: «ای مردمان شما آگاه هستید که از این شغل، فراخی رزق ما است.
\par 26 ودیده و شنیده‌اید که نه‌تنها در افسس، بلکه تقریب در تمام آسیا این پولس خلق بسیاری را اغوانموده، منحرف ساخته است و می‌گوید اینهایی که به‌دستها ساخته می‌شوند، خدایان نیستند.
\par 27 پس خطر است که نه فقط کسب ما از میان رودبلکه این هیکل خدای عظیم ارطامیس نیز حقیرشمرده شود و عظمت وی که تمام آسیا و ربع مسکون او را می‌پرستند برطرف شود.»
\par 28 چون این را شنیدند، از خشم پر گشته، فریاد کرده، می‌گفتند که «بزرگ است ارطامیس افسسیان.»
\par 29 و تمامی شهر به شورش آمده، همه متفق به تماشاخانه تاختند و غایوس و ارسترخس را که از اهل مکادونیه و همراهان پولس بودند با خودمی کشیدند.
\par 30 اما چون پولس اراده نمود که به میان مردم درآید، شاگردان او را نگذاشتند.
\par 31 وبعضی از روسای آسیا که او را دوست می‌داشتند، نزد او فرستاده، خواهش نمودند که خود را به تماشاخانه نسپارد.
\par 32 و هر یکی صدایی علیحده می‌کردند زیرا که جماعت آشفته بود و اکثرنمی دانستند که برای چه جمع شده‌اند.
\par 33 پس اسکندر را از میان خلق کشیدند که یهودیان او راپیش انداختند و اسکندر به‌دست خود اشاره کرده، خواست برای خود پیش مردم حجت بیاورد.
\par 34 لیکن چون دانستند که یهودی است همه به یک آواز قریب به دو ساعت ندا می‌کردندکه «بزرگ است ارطامیس افسسیان.»
\par 35 پس از آن مستوفی شهر خلق را ساکت گردانیده، گفت: «ای مردان افسسی، کیست که نمی داند که شهر افسسیان ارطامیس خدای عظیم و آن صنمی را که از مشتری نازل شد پرستش می‌کند؟
\par 36 پس چون این امور را نتوان انکار کرد، شما می‌باید آرام باشید و هیچ کاری به تعجیل نکنید.
\par 37 زیرا که این اشخاص را آوردید که نه تاراج کنندگان هیکل‌اند و نه به خدای شما بدگفته‌اند.
\par 38 پس هر گاه دیمیتریوس و همکاران وی ادعایی بر کسی دارند، ایام قضا مقرر است وداوران معین هستند. با همدیگر مرافعه باید کرد.
\par 39 و اگر در امری دیگر طالب چیزی باشید، درمحکمه شرعی فیصل خواهد پذیرفت.
\par 40 زیرادر خطریم که در خصوص فتنه امروز از مابازخواست شود چونکه هیچ علتی نیست که درباره آن عذری برای این ازدحام توانیم آورد.»این را گفته، جماعت را متفرق ساخت.
\par 41 این را گفته، جماعت را متفرق ساخت.

\chapter{20}

\par 1 و بعد از تمام شدن این هنگامه، پولس شاگردان را طلبیده، ایشان را وداع نمودو به سمت مکادونیه روانه شد.
\par 2 و در آن نواحی سیر کرده، اهل آنجا را نصیحت بسیار نمود و به یونانستان آمد.
\par 3 و سه ماه توقف نمود و چون عزم سفر سوریه کرد و یهودیان در کمین وی بودند، اراده نمود که از راه مکادونیه مراجعت کند.
\par 4 و سوپاترس از اهل بیریه و ارسترخس وسکندس از اهل تسالونیکی و غایوس از دربه وتیموتاوس و از مردم آسیا تیخیکس و تروفیمس تا به آسیا همراه او رفتند.
\par 5 و ایشان پیش رفته، درترواس منتظر ما شدند.
\par 6 و اما ما بعد از ایام فطیراز فیلپی به کشتی سوار شدیم و بعد از پنج روز به ترواس نزد ایشان رسیده، در آنجا هفت روزماندیم.
\par 7 و در اول هفته چون شاگردان بجهت شکستن نان جمع شدند و پولس در فردای آن روز عازم سفر بود، برای ایشان موعظه می‌کرد و سخن او تانصف شب طول کشید.
\par 8 و در بالاخانه‌ای که جمع بودیم چراغ بسیار بود.
\par 9 ناگاه جوانی که افتیخس نام داشت، نزد دریچه نشسته بود که خواب سنگین او را درربود و چون پولس کلام راطول می‌داد، خواب بر او مستولی گشته، از طبقه سوم به زیر افتاد و او را مرده برداشتند.
\par 10 آنگاه پولس به زیر آمده، بر او افتاد و وی را در آغوش کشیده، گفت: «مضطرب مباشید زیرا که جان اودر اوست.»
\par 11 پس بالا رفته و نان را شکسته، خورد و تا طلوع فجر گفتگوی بسیار کرده، همچنین روانه شد.
\par 12 و آن جوان را زنده بردند وتسلی عظیم پذیرفتند.
\par 13 اما ما به کشتی سوار شده، به اسوس پیش رفتیم که از آنجا می‌بایست پولس را برداریم که بدینطور قرار داد زیرا خواست تا آنجا پیاده رود.
\par 14 پس چون در اسوس او را ملاقات کردیم، او رابرداشته، به متیلینی آمدیم.
\par 15 و از آنجا به دریاکوچ کرده، روز دیگر به مقابل خیوس رسیدیم وروز سوم به ساموس وارد شدیم و در تروجیلیون توقف نموده، روز دیگر وارد میلیتس شدیم.
\par 16 زیرا که پولس عزیمت داشت که از محاذی افسس بگذرد، مبادا او را در آسیا درنگی پیداشود، چونکه تعجیل می‌کرد که اگر ممکن شود تاروز پنطیکاست به اورشلیم برسد.
\par 17 پس از میلیتس به افسس فرستاده، کشیشان کلیسا را طلبید.
\par 18 و چون به نزدش حاضر شدند، ایشان را گفت: «بر شما معلوم است که از روز اول که وارد آسیا شدم، چطور هر وقت با شما بسرمی بردم؛
\par 19 که با کمال فروتنی و اشکهای بسیار وامتحانهایی که از مکاید یهود بر من عارض می‌شد، به خدمت خداوند مشغول می‌بودم.
\par 20 وچگونه چیزی را از آنچه برای شما مفید باشد، دریغ نداشتم بلکه آشکارا و خانه به خانه شما رااخبار و تعلیم می‌نمودم.
\par 21 و به یهودیان ویونانیان نیز از توبه به سوی خدا و ایمان به خداوند ما عیسی مسیح شهادت می‌دادم.
\par 22 واینک الان در روح بسته شده، به اورشلیم می‌روم و از آنچه در آنجا بر من واقع خواهد شد، اطلاعی ندارم.
\par 23 جز اینکه روح‌القدس در هر شهرشهادت داده، می‌گوید که بندها و زحمات برایم مهیا است.
\par 24 لیکن این چیزها را به هیچ می‌شمارم، بلکه جان خود را عزیز نمی دارم تادور خود را به خوشی به انجام رسانم و آن خدمتی را که از خداوند عیسی یافته‌ام که به بشارت فیض خدا شهادت دهم.
\par 25 و الحال این رامی دانم که جمیع شما که در میان شما گشته و به ملکوت خدا موعظه کرده‌ام، دیگر روی مرانخواهید دید.
\par 26 پس امروز از شما گواهی می‌طلبم که من از خون همه بری هستم،
\par 27 زیراکه از اعلام نمودن شما به تمامی اراده خداکوتاهی نکردم.
\par 28 پس نگاه دارید خویشتن وتمامی آن گله را که روح‌القدس شما را بر آن اسقف مقرر فرمود تا کلیسای خدا را رعایت کنیدکه آن را به خون خود خریده است.
\par 29 زیرا من می‌دانم که بعد از رحلت من، گرگان درنده به میان شما درخواهند آمد که بر گله ترحم نخواهند نمود،
\par 30 و از میان خود شما مردمانی خواهندبرخاست که سخنان کج خواهند گفت تا شاگردان را در عقب خود بکشند.
\par 31 لهذا بیدار باشید و به یاد آورید که مدت سه سال شبانه‌روز از تنبیه نمودن هر یکی از شما با اشکها باز نایستادم.
\par 32 والحال‌ای برادران شما را به خدا و به کلام فیض اومی سپارم که قادر است شما را بنا کند و در میان جمیع مقدسین شما را میراث بخشد.
\par 33 نقره یاطلا یا لباس کسی را طمع نورزیدم،
\par 34 بلکه خودمی دانید که همین دستها در رفع احتیاج خود ورفقایم خدمت می‌کرد.
\par 35 این همه را به شمانمودم که می‌باید چنین مشقت کشیده، ضعفا رادستگیری نمایید و کلام خداوند عیسی را به‌خاطر دارید که او گفت دادن از گرفتن فرخنده‌تراست.»
\par 36 این بگفت و زانو زده، با همگی ایشان دعاکرد.
\par 37 و همه گریه بسیار کردند و بر گردن پولس آویخته، او را می‌بوسیدند.و بسیار متالم شدندخصوص بجهت آن سخنی که گفت: «بعد از این‌روی مرا نخواهید دید.» پس او را تا به کشتی مشایعت نمودند.
\par 38 و بسیار متالم شدندخصوص بجهت آن سخنی که گفت: «بعد از این‌روی مرا نخواهید دید.» پس او را تا به کشتی مشایعت نمودند.

\chapter{21}

\par 1 و چون از ایشان هجرت نمودیم، سفردریا کردیم و به راه راست به کوس آمدیم و روز دیگر به رودس و از آنجا به پاترا.
\par 2 وچون کشتی‌ای یافتیم که عازم فینیقیه بود، بر آن سوار شده، کوچ کردیم.
\par 3 و قپرس را به نظرآورده، آن را به طرف چپ رها کرده، به سوی سوریه رفتیم و در صور فرود آمدیم زیرا که در آنجا می‌بایست بار کشتی را فرود آورند.
\par 4 پس شاگردی چند پیدا کرده، هفت روز در آنجاماندیم و ایشان به الهام روح به پولس گفتند که به اورشلیم نرود.
\par 5 و چون آن روزها را بسر بردیم، روانه گشتیم و همه با زنان و اطفال تا بیرون شهر مارا مشایعت نمودند و به کناره دریا زانو زده، دعاکردیم.
\par 6 پس یکدیگر را وداع کرده، به کشتی سوار شدیم و ایشان به خانه های خود برگشتند.
\par 7 و ما سفر دریا را به انجام رسانیده، از صور به پتولامیس رسیدیم و برادران را سلام کرده، باایشان یک روز ماندیم.
\par 8 در فردای آن روز، ازآنجا روانه شده، به قیصریه آمدیم و به خانه فیلپس مبشر که یکی از آن هفت بود درآمده، نزداو ماندیم.
\par 9 و او را چهار دختر باکره بود که نبوت می‌کردند.
\par 10 و چون روز چند در آنجا ماندیم، نبی‌ای آغابوس نام از یهودیه رسید،
\par 11 و نزد ما آمده، کمربند پولس را گرفته و دستها و پایهای خود رابسته، گفت: «روح‌القدس می‌گوید که یهودیان دراورشلیم صاحب این کمربند را به همینطور بسته، او را به‌دستهای امت‌ها خواهند سپرد.»
\par 12 پس چون این را شنیدیم، ما و اهل آنجا التماس نمودیم که به اورشلیم نرود.
\par 13 پولس جواب داد: «چه می‌کنید که گریان شده، دل مرا می‌شکنیدزیرا من مستعدم که نه فقط قید شوم بلکه تا دراورشلیم بمیرم به‌خاطر نام خداوند عیسی.»
\par 14 چون او نشنید خاموش شده، گفتیم: «آنچه اراده خداوند است بشود.»
\par 15 و بعد از آن ایام تدارک سفر دیده، متوجه اورشلیم شدیم.
\par 16 و تنی چند از شاگردان قیصریه همراه آمده، ما را نزد شخصی مناسون نام که از اهل قپرس و شاگرد قدیمی بود، آوردند تانزد او منزل نماییم.
\par 17 و چون وارد اورشلیم گشتیم، برادران ما رابه خشنودی پذیرفتند.
\par 18 و در روز دیگر، پولس ما را برداشته، نزد یعقوب رفت و همه کشیشان حاضر شدند.
\par 19 پس ایشان را سلام کرده، آنچه خدا بوسیله خدمت او در میان امت‌ها به عمل آورده بود، مفصلا گفت.
\par 20 ایشان چون این راشنیدند، خدا را تمجید نموده، به وی گفتند: «ای برادر، آگاه هستی که چند هزارها از یهودیان ایمان آورده‌اند و جمیع در شریعت غیورند:
\par 21 ودرباره تو شنیده‌اند که همه یهودیان را که در میان امت‌ها می‌باشند، تعلیم می‌دهی که از موسی انحراف نمایند و می‌گویی نباید اولاد خود رامختون ساخت و به سنن رفتار نمود.
\par 22 پس چه باید کرد؟ البته جماعت جمع خواهند شد زیراخواهند شنید که تو آمده‌ای.
\par 23 پس آنچه به توگوییم به عمل آور: چهار مرد نزد ما هستند که برایشان نذری هست.
\par 24 پس ایشان را برداشته، خود را با ایشان تطهیر نما و خرج ایشان را بده که سر خود را بتراشند تا همه بدانند که آنچه درباره تو شنیده‌اند اصلی ندارد بلکه خود نیز درمحافظت شریعت سلوک می‌نمایی.
\par 25 لیکن درباره آنانی که از امت‌ها ایمان آورده‌اند، مافرستادیم و حکم کردیم که از قربانی های بت وخون و حیوانات خفه شده و زنا پرهیز نمایند.
\par 26 پس پولس آن اشخاص را برداشته، روز دیگر با ایشان طهارت کرده، به هیکل درآمد و از تکمیل ایام طهارت اطلاع داد تا هدیه‌ای برای هر یک ازایشان بگذرانند.
\par 27 و چون هفت روز نزدیک به انجام رسید، یهودی‌ای چند از آسیا او را در هیکل دیده، تمامی قوم را به شورش آوردند و دست بر اوانداخته،
\par 28 فریاد برآوردند که «ای مردان اسرائیلی، امداد کنید! این است آن کس که برخلاف امت و شریعت و این مکان در هر جاهمه را تعلیم می‌دهد. بلکه یونانی‌ای چند را نیز به هیکل درآورده، این مکان مقدس را ملوث نموده است.»
\par 29 زیرا قبل از آن تروفیمس افسسی را باوی در شهر دیده بودند و مظنه داشتند که پولس او را به هیکل آورده بود.
\par 30 پس تمامی شهر به حرکت آمد و خلق ازدحام کرده، پولس را گرفتند و از هیکل بیرون کشیدند و فی الفور درها را بستند.
\par 31 و چون قصدقتل او می‌کردند، خبر به مین باشی سپاه رسید که «تمامی اورشلیم به شورش آمده است.»
\par 32 اوبی درنگ سپاه و یوزباشی‌ها را برداشته، بر سرایشان تاخت. پس ایشان به مجرد دیدن مین باشی و سپاهیان، از زدن پولس دست برداشتند.
\par 33 چون مین باشی رسید، او را گرفته، فرمان داد تا او را بدو زنجیر ببندند و پرسید که «این کیست و چه کرده است؟»
\par 34 اما بعضی از آن گروه به سخنی و بعضی به سخنی دیگر صدا می‌کردند. و چون او به‌سبب شورش، حقیقت امر رانتوانست فهمید، فرمود تا او را به قلعه بیاورند.
\par 35 و چون به زینه رسید، اتفاق افتاد که لشکریان به‌سبب ازدحام مردم او را برگرفتند،
\par 36 زیرا گروهی کثیر از خلق از عقب او افتاده، صدا می‌زدند که «اورا هلاک کن!»
\par 37 چون نزدیک شد که پولس را به قلعه درآورند، او به مین باشی گفت: آیا اجازت است که به تو چیزی گویم؟ گفت: «آیا زبان یونانی رامی دانی؟
\par 38 مگر تو آن مصری نیستی که چندی پیش از این فتنه برانگیخته، چهار هزار مرد قتال رابه بیابان برد؟»
\par 39 پولس گفت: «من مرد یهودی هستم از طرسوس قیلیقیه، شهری که بی‌نام ونشان نیست و خواهش آن دارم که مرا اذن فرمایی تا به مردم سخن گویم.»چون اذن یافت، بر زینه ایستاده، به‌دست خود به مردم اشاره کرد؛ و چون آرامی کامل پیدا شد، ایشان را به زبان عبرانی مخاطب ساخته، گفت.
\par 40 چون اذن یافت، بر زینه ایستاده، به‌دست خود به مردم اشاره کرد؛ و چون آرامی کامل پیدا شد، ایشان را به زبان عبرانی مخاطب ساخته، گفت.

\chapter{22}

\par 1 «ای برادران عزیز و پدران، حجتی را که الان پیش شما می‌آورم بشنوید.»
\par 2 چون شنیدند که به زبان عبرانی با ایشان تکلم می‌کند، بیشتر خاموش شدند. پس گفت:
\par 3 «من مرد یهودی هستم، متولد طرسوس قیلیقیه، اما تربیت یافته بودم در این شهر درخدمت غمالائیل و در دقایق شریعت اجدادمتعلم شده، درباره خدا غیور می‌بودم، چنانکه همگی شما امروز می‌باشید.
\par 4 و این طریقت را تابه قتل مزاحم می‌بودم به نوعی که مردان و زنان رابند نهاده، به زندان می‌انداختم،
\par 5 چنانکه رئیس کهنه و تمام اهل شورا به من شهادت می‌دهند که از ایشان نامه‌ها برای برادران گرفته، عازم دمشق شدم تا آنانی را نیز که در آنجا باشند قید کرده، به اورشلیم آورم تا سزا یابند.
\par 6 و در اثنای راه، چون نزدیک به دمشق رسیدم، قریب به ظهر ناگاه نوری عظیم از آسمان گرد من درخشید.
\par 7 پس بر زمین افتاده، هاتفی را شنیدم که به من می‌گوید: "ای شاول، ای شاول، چرا بر من جفا می‌کنی؟"
\par 8 من جواب دادم: "خداوندا تو کیستی؟" او مرا گفت: "من آن عیسی ناصری هستم که تو بر وی جفامی کنی."
\par 9 و همراهان من نور را دیده، ترسان گشتند ولی آواز آن کس را که با من سخن گفت نشنیدند.
\par 10 گفتم: "خداوندا چه کنم؟" خداوندمرا گفت: "برخاسته، به دمشق برو که در آنجا تو رامطلع خواهند ساخت از آنچه برایت مقرر است که بکنی."
\par 11 پس چون از سطوت آن نور نابیناگشتم، رفقایم دست مرا گرفته، به دمشق رسانیدند.
\par 12 آنگاه شخصی متقی بحسب شریعت، حنانیا نام که نزد همه یهودیان ساکن آنجا نیکنام بود،
\par 13 به نزد من آمده و ایستاده، به من گفت: "ای برادر شاول، بینا شو" که در همان ساعت بر وی نگریستم.
\par 14 او گفت: "خدای پدران ما تو را برگزید تا اراده او را بدانی و آن عادل را ببینی و از زبانش سخنی بشنوی.
\par 15 زیرااز آنچه دیده و شنیده‌ای نزد جمیع مردم شاهد براو خواهی شد.
\par 16 و حال چرا تاخیر می‌نمایی؟ برخیز و تعمید بگیر و نام خداوند را خوانده، خود را از گناهانت غسل ده."
\par 17 و چون به اورشلیم برگشته، در هیکل دعا می‌کردم، بیخودشدم.
\par 18 پس او را دیدم که به من می‌گوید: "بشتاب و از اورشلیم به زودی روانه شو زیرا که شهادت تو را در حق من نخواهند پذیرفت."
\par 19 من گفتم: "خداوندا، ایشان می‌دانند که من در هرکنیسه مومنین تو را حبس کرده، می‌زدم؛
\par 20 وهنگامی که خون شهید تو استیفان را می‌ریختند، من نیز ایستاده، رضا بدان دادم و جامه های قاتلان او را نگاه می‌داشتم."
\par 21 او به من گفت: "روانه شوزیرا که من تو را به سوی امت های بعیدمی فرستم.»
\par 22 پس تا این سخن بدو گوش گرفتند؛ آنگاه آواز خود را بلند کرده، گفتند: «چنین شخص را ازروی زمین بردار که زنده ماندن او جایز نیست!»
\par 23 و چون غوغا نموده و جامه های خود راافشانده، خاک به هوا می‌ریختند،
\par 24 مین باشی فرمان داد تا او را به قلعه درآوردند و فرمود که اورا به تازیانه امتحان کنند تا بفهمد که به چه سبب اینقدر بر او فریاد می‌کردند.
\par 25 و وقتی که او را به ریسمانها می‌بستند، پولس به یوزباشی‌ای که حاضر بود گفت: «آیا بر شما جایز است که مردی رومی را بی‌حجت هم تازیانه زنید؟»
\par 26 چون یوزباشی این را شنید، نزد مین باشی رفته، او راخبر داده، گفت: «چه می‌خواهی بکنی زیرا این شخص رومی است؟»
\par 27 پس مین باشی آمده، به وی گفت: «مرا بگو که تو رومی هستی؟» گفت: «بلی!»
\par 28 مین باشی جواب داد: «من این حقوق رابه مبلغی خطیر تحصیل کردم!» پولس گفت: «امامن در آن مولود شدم.»
\par 29 در ساعت آنانی که قصد تفتیش او داشتند، دست از او برداشتند ومین باشی ترسان گشت چون فهمید که رومی است از آن سبب که او را بسته بود.بامدادان چون خواست درست بفهمد که یهودیان به چه علت مدعی او می‌باشند، او را از زندان بیرون آورده، فرمود تا روسای کهنه و تمامی اهل شوراحاضر شوند و پولس را پایین آورده، در میان ایشان برپا داشت.
\par 30 بامدادان چون خواست درست بفهمد که یهودیان به چه علت مدعی او می‌باشند، او را از زندان بیرون آورده، فرمود تا روسای کهنه و تمامی اهل شوراحاضر شوند و پولس را پایین آورده، در میان ایشان برپا داشت.

\chapter{23}

\par 1 پس پولس به اهل شورا نیک نگریسته، گفت: «ای برادران، من تا امروز با کمال ضمیر صالح در خدمت خدا رفتار کرده‌ام.»
\par 2 آنگاه حنانیا، رئیس کهنه، حاضران را فرمودتا به دهانش زنند.
\par 3 پولس بدو گفت: «خدا تو راخواهد زد، ای دیوار سفیدشده! تو نشسته‌ای تامرا برحسب شریعت داوری کنی و به ضدشریعت حکم به زدنم می‌کنی؟»
\par 4 حاضران گفتند: «آیا رئیس کهنه خدا را دشنام می‌دهی؟»
\par 5 پولس گفت: «ای برادران، ندانستم که رئیس کهنه است، زیرا مکتوب است حاکم قوم خود رابد مگوی.»
\par 6 چون پولس فهمید که بعضی از صدوقیان وبعضی از فریسیانند، در مجلس ندا در‌داد که «ای برادران، من فریسی، پسر فریسی هستم و برای امید و قیامت مردگان از من بازپرس می‌شود.»
\par 7 چون این را گفت، در میان فریسیان و صدوقیان منازعه برپا شد و جماعت دو فرقه شدند،
\par 8 زیراکه صدوقیان منکر قیامت و ملائکه و ارواح هستند لیکن فریسیان قائل به هر دو.
\par 9 پس غوغای عظیم برپا شد و کاتبان از فرقه فریسیان برخاسته مخاصمه نموده، می‌گفتند که «در این شخص هیچ بدی نیافته‌ایم و اگر روحی یا فرشته‌ای با او سخن گفته باشد با خدا جنگ نبایدنمود.»
\par 10 و چون منازعه زیادتر می‌شد، مین باشی ترسید که مبادا پولس را بدرند. پس فرمود تاسپاهیان پایین آمده، او را از میانشان برداشته، به قلعه درآوردند.
\par 11 و در شب همان روز خداوند نزد او آمده، گفت: «ای پولس خاطر جمع باش زیرا چنانکه دراورشلیم در حق من شهادت دادی، همچنین بایددر روم نیز شهادت دهی.»
\par 12 و چون روز شد، یهودیان با یکدیگر عهدبسته، بر خویشتن لعن کردند که تا پولس رانکشند، نخورند و ننوشند.
\par 13 و آنانی که درباره این، همقسم شدند، زیاده از چهل نفر بودند.
\par 14 اینها نزد روسای کهنه و مشایخ رفته، گفتند: «بر خویشتن لعنت سخت کردیم که تا پولس رانکشیم چیزی نچشیم.
\par 15 پس الان شما با اهل شورا، مین باشی را اعلام کنید که او را نزد شمابیاورد که گویا اراده دارید در احوال او نیکوترتحقیق نمایید؛ و ما حاضر هستیم که قبل ازرسیدنش او را بکشیم.»
\par 16 اما خواهرزاده پولس از کمین ایشان اطلاع یافته، رفت و به قلعه درآمده، پولس را آگاهانید.
\par 17 پولس یکی ازیوزباشیان را طلبیده، گفت: «این جوان را نزدمین باشی ببر زیرا خبری دارد که به او بگوید.»
\par 18 پس او را برداشته، به حضور مین باشی رسانیده، گفت: «پولس زندانی مرا طلبیده، خواهش کرد که این جوان را به خدمت تو بیاورم، زیرا چیزی داردکه به تو عرض کند.»
\par 19 پس مین باشی دستش راگرفته، به خلوت برد و پرسید: «چه چیز است که می خواهی به من خبر دهی؟»
\par 20 عرض کرد: «یهودیان متفق شده‌اند که از تو خواهش کنند تاپولس را فردا به مجلس شورا درآوری که گویااراده دارند در حق او زیادتر تفتیش نمایند.
\par 21 پس خواهش ایشان را اجابت مفرما زیرا که بیشتر از چهل نفر از ایشان در کمین وی‌اند و به سوگند عهد بسته‌اند که تا او را نکشند چیزی نخورند و نیاشامند و الان مستعد و منتظر وعده تو می‌باشند.»
\par 22 مین باشی آن جوان را مرخص فرموده، قدغن نمود که «به هیچ‌کس مگو که مرا ازاین راز مطلع ساختی.»
\par 23 پس دو نفر از یوزباشیان را طلبیده، فرمودکه «دویست سپاهی و هفتاد سوار و دویست نیزه‌دار در ساعت سوم از شب حاضر سازید تا به قیصریه بروند؛
\par 24 و مرکبی حاضر کنید تا پولس را سوار کرده، او را به سلامتی به نزد فیلکس والی برسانند.»
\par 25 و نامه‌ای بدین مضمون نوشت:
\par 26 «کلودیوس لیسیاس، به والی گرامی فیلکس سلام می‌رساند.
\par 27 یهودیان این شخص را گرفته، قصد قتل او داشتند. پس با سپاه رفته، او را ازایشان گرفتم، چون دریافت کرده بودم که رومی است.
\par 28 و چون خواستم بفهمم که به چه سبب بروی شکایت می‌کنند، او را به مجلس ایشان درآوردم.
\par 29 پس یافتم که در مسائل شریعت خود از او شکایت می‌دارند، ولی هیچ شکوه‌ای مستوجب قتل یا بند نمی دارند.
\par 30 و چون خبریافتم که یهودیان قصد کمین سازی برای او دارند، بی‌درنگ او را نزد تو فرستادم و مدعیان او را نیزفرمودم تا در حضور تو بر او ادعا نمایندوالسلام.»
\par 31 پس سپاهیان چنانکه مامور شدند، پولس را در شب برداشته، به انتیپاتریس رسانیدند.
\par 32 و بامدادان سواران را گذاشته که با او بروند، خود به قلعه برگشتند.
\par 33 و چون ایشان وارد قیصریه شدند، نامه را به والی سپردند و پولس را نیز نزد اوحاضر ساختند.
\par 34 پس والی نامه را ملاحظه فرموده، پرسید که از کدام ولایت است. چون دانست که از قیلیقیه است،گفت: «چون مدعیان تو حاضر شوند، سخن تو را خواهم شنید.» و فرمود تا او را در سرای هیرودیس نگاه دارند.
\par 35 گفت: «چون مدعیان تو حاضر شوند، سخن تو را خواهم شنید.» و فرمود تا او را در سرای هیرودیس نگاه دارند.

\chapter{24}

\par 1 و بعد از پنج روز، حنانیای رئیس کهنه با مشایخ و خطیبی ترتلس نام رسیدندو شکایت از پولس نزد والی آوردند.
\par 2 و چون اورا احضار فرمود، ترتلس آغاز ادعا نموده، گفت: چون از وجود تو در آسایش کامل هستیم واحسانات عظیمه از تدابیر تو بدین قوم رسیده است، ای فیلکس گرامی،
\par 3 در هر جا و در هروقت این را در کمال شکرگذاری می‌پذیریم.
\par 4 ولیکن تا تو را زیاده مصدع نشوم، مستدعی هستم که از راه نوازش مختصر عرض ما را بشنوی.
\par 5 زیرا که این شخص را مفسد و فتنه انگیز یافته‌ایم در میان همه یهودیان ساکن ربع مسکون و ازپیشوایان بدعت نصاری.
\par 6 و چون او خواست هیکل را ملوث سازد، او را گرفته، اراده داشتیم که به قانون شریعت خود بر او داوری نماییم.
\par 7 ولی لیسیاس مین باشی آمده، او را به زور بسیار ازدستهای ما بیرون آورد،
\par 8 و فرمود تا مدعیانش نزد تو حاضر شوند؛ و از او بعد از امتحان می‌توانی دانست حقیقت همه این اموری که ما براو ادعا می‌کنیم.»
\par 9 و یهودیان نیز با او متفق شده گفتند که چنین است.
\par 10 چون والی به پولس اشاره نمود که سخن بگوید، او جواب داد: «از آن رو که می‌دانم سالهای بسیار است که تو حاکم این قوم می‌باشی، به خشنودی وافر حجت درباره خود می‌آورم.
\par 11 زیرا تو می‌توانی دانست که زیاده از دوازده روز نیست که من برای عبادت به اورشلیم رفتم،
\par 12 و مرا نیافتند که در هیکل با کسی مباحثه کنم ونه در کنایس یا شهر‌که خلق را به شورش آورم.
\par 13 و هم آنچه الان بر من ادعا می‌کنند، نمی تواننداثبات نمایند.
\par 14 لیکن این را نزد تو اقرار می‌کنم که به طریقتی که بدعت می‌گویند، خدای پدران را عبادت می‌کنم و به آنچه در تورات و انبیامکتوب است معتقدم،
\par 15 و به خدا امیدوارم چنانکه ایشان نیز قبول دارند که قیامت مردگان ازعادلان و ظالمان نیز خواهد شد.
\par 16 و خود را دراین امر ریاضت می‌دهم تا پیوسته ضمیر خود رابه سوی خدا و مردم بی‌لغزش نگاه دارم.
\par 17 و بعداز سالهای بسیار آمدم تا صدقات و هدایا برای قوم خود بیاورم.
\par 18 و در این امور چند نفر ازیهودیان آسیا مرا در هیکل مطهر یافتند بدون هنگامه یا شورشی.
\par 19 و ایشان می‌بایست نیز دراینجا نزد تو حاضر شوند تا اگر حرفی بر من دارندادعا کنند.
\par 20 یا اینان خود بگویند اگر گناهی ازمن یافتند وقتی که در حضور اهل شورا ایستاده بودم،
\par 21 مگر آن یک سخن که در میان ایشان ایستاده، بدان ندا کردم که درباره قیامت مردگان ازمن امروز پیش شما بازپرس می‌شود.»
\par 22 آنگاه فیلکس چون از طریقت نیکوترآگاهی داشت، امر ایشان را تاخیر انداخته، گفت: «چون لیسیاس مین باشی آید، حقیقت امر شما را دریافت خواهم کرد.»
\par 23 پس یوزباشی را فرمان داد تا پولس را نگاه دارد و او را آزادی دهد واحدی از خویشانش را از خدمت و ملاقات اومنع نکند.
\par 24 و بعد از روزی چند فیلکس با زوجه خود درسلا که زنی یهودی بود، آمده پولس راطلبیده، سخن او را درباره ایمان مسیح شنید.
\par 25 وچون او درباره عدالت و پرهیزکاری وداوری آینده خطاب می‌کرد، فیلکس ترسان گشته، جواب داد که «الحال برو چون فرصت کنم تو راباز خواهم خواند.»
\par 26 و نیز امید می‌داشت که پولس او را نقدی بدهد تا او را آزاد سازد و از این جهت مکرر وی را خواسته، با او گفتگو می‌کرد.اما بعد از انقضای دو سال، پورکیوس فستوس، خلیفه ولایت فیلکس شد و فیلکس چون خواست بر یهود منت نهد، پولس را در زندان گذاشت.
\par 27 اما بعد از انقضای دو سال، پورکیوس فستوس، خلیفه ولایت فیلکس شد و فیلکس چون خواست بر یهود منت نهد، پولس را در زندان گذاشت.

\chapter{25}

\par 1 پس چون فستوس به ولایت خودرسید، بعد از سه روز از قیصریه به اورشلیم رفت.
\par 2 و رئیس کهنه و اکابر یهود نزد اوبر پولس ادعا کردند و بدو التماس نموده،
\par 3 منتی بر وی خواستند تا او را به اورشلیم بفرستد و درکمین بودند که او را در راه بکشند.
\par 4 اما فستوس جواب داد که «پولس را باید در قیصریه نگاه داشت»، زیرا خود اراده داشت به زودی آنجابرود.
\par 5 و گفت: «پس کسانی از شما که می‌توانندهمراه بیایند تا اگر چیزی در این شخص یافت شود، بر او ادعا نمایند.»
\par 6 و چون بیشتر از ده روز در میان ایشان توقف کرده بود، به قیصریه آمد و بامدادان بر مسندحکومت برآمده، فرمود تا پولس را حاضر سازند.
\par 7 چون او حاضر شد، یهودیانی که از اورشلیم آمده بودند، به گرد او ایستاده، شکایتهای بسیار وگران بر پولس آوردند ولی اثبات نتوانستند کرد.
\par 8 او جواب داد که «نه به شریعت یهود و نه به هیکل و نه به قیصر هیچ گناه کرده‌ام.»
\par 9 اما چون فستوس خواست بر یهود منت نهد، در جواب پولس گفت: «آیا می‌خواهی به اورشلیم آیی تا درآنجا در این امور به حضور من حکم شود؟»
\par 10 پولس گفت: «در محکمه قیصر ایستاده‌ام که درآنجا می‌باید محاکمه من بشود. به یهود هیچ ظلمی نکرده‌ام، چنانکه تو نیز نیکو می‌دانی.
\par 11 پس هر گاه ظلمی یا عملی مستوجب قتل کرده باشم، از مردن دریغ ندارم. لیکن اگر هیچ‌یک ازاین شکایتهایی که اینها بر من می‌آورند اصلی ندارد، کسی نمی تواند مرا به ایشان سپارد. به قیصر رفع دعوی می‌کنم.»
\par 12 آنگاه فستوس بعداز مکالمه با اهل شورا جواب داد: «آیا به قیصررفع دعوی کردی؟ به حضور قیصر خواهی رفت.»
\par 13 و بعد از مرور ایام چند، اغریپاس پادشاه وبرنیکی برای تحیت فستوس به قیصریه آمدند.
\par 14 و چون روزی بسیار در آنجا توقف نمودند، فستوس برای پادشاه، مقدمه پولس را بیان کرده، گفت: «مردی است که فیلکس او را در بند گذاشته است،
\par 15 که درباره او وقتی که به اورشلیم آمدم، روسای کهنه و مشایخ یهود مرا خبر دادند وخواهش نمودند که بر او داوری شود.
\par 16 درجواب ایشان گفتم که رومیان را رسم نیست که احدی را بسپارند قبل از آنکه مدعی علیه مدعیان خود را روبرو شود و او را فرصت دهند که ادعای ایشان را جواب گوید.
\par 17 پس چون ایشان دراینجا جمع شدند، بی‌درنگ در روز دوم بر مسندنشسته، فرمودم تا آن شخص را حاضر کردند.
\par 18 و مدعیانش برپا ایستاده، از آنچه من گمان می‌بردم هیچ ادعا بر وی نیاوردند.
\par 19 بلکه مساله‌ای چند بر او ایراد کردند درباره مذهب خود و در حق عیسی نامی که مرده است و پولس می‌گوید که او زنده است.
\par 20 و چون من در این‌گونه مسایل شک داشتم، از او پرسیدم که "آیامی خواهی به اورشلیم بروی تا در آنجا این مقدمه فیصل پذیرد؟"
\par 21 ولی چون پولس رفع دعوی کرد که برای محاکمه اوغسطس محفوظ ماند، فرمان دادم که او را نگاه بدارند تا او را به حضورقیصر روانه نمایم.»
\par 22 اغریپاس به فستوس گفت: «من نیزمی خواهم این شخص را بشنوم.» گفت: «فردا اورا خواهی شنید.»
\par 23 پس بامدادان چون اغریپاس و برنیکی باحشمتی عظیم آمدند و به دارالاستماع بامین باشیان و بزرگان شهر داخل شدند، به فرمان فستوس پولس را حاضر ساختند.
\par 24 آنگاه فستوس گفت: «ای اغریپاس پادشاه، و‌ای همه مردمانی که نزد ما حضور دارید، این شخص رامی بینید که درباره او تمامی جماعت یهود چه دراورشلیم و چه در اینجا فریاد کرده، از من خواهش نمودند که دیگر نباید زیست کند.
\par 25 ولیکن چون من دریافتم که او هیچ عملی مستوجب قتل نکرده است و خود به اوغسطس رفع دعوی کرد، اراده کردم که او را بفرستم.
\par 26 و چون چیزی درست ندارم که درباره او به خداوندگار مرقوم دارم، از این جهت او را نزد شما و علی الخصوص در حضور تو‌ای اغریپاس پادشاه آوردم تا بعد از تفحص شاید چیزی یافته بنگارم.زیرا مراخلاف عقل می‌نماید که اسیری را بفرستم وشکایتهایی که بر اوست معروض ندارم.»
\par 27 زیرا مراخلاف عقل می‌نماید که اسیری را بفرستم وشکایتهایی که بر اوست معروض ندارم.»

\chapter{26}

\par 1 اغریپاس به پولس گفت: «مرخصی که کیفیت خود را بگویی.»
\par 2 که «ای اغریپاس پادشاه، سعادت خود را در این می‌دانم که امروز درحضور تو حجت بیاورم، درباره همه شکایتهایی که یهود از من می‌دارند.
\par 3 خصوص چون تو درهمه رسوم و مسایل یهود عالم هستی، پس از تومستدعی آنم که تحمل فرموده، مرا بشنوی.
\par 4 رفتار مرا از جوانی چونکه از ابتدا در میان قوم خود در اورشلیم بسر می‌بردم، تمامی یهودمی دانند
\par 5 و مرا از اول می‌شناسند هر گاه بخواهند شهادت دهند که به قانون پارساترین فرقه دین خود فریسی می‌بودم.
\par 6 والحال به‌سبب امید آن وعده‌ای که خدا به اجداد ما داد، بر من ادعا می‌کنند.
\par 7 و حال آنکه دوازده سبط ماشبانه‌روز بجد و جهد عبادت می‌کنند محض امید تحصیل همین وعده که بجهت همین امید، ای اغریپاس پادشاه، یهود بر من ادعا می‌کنند.
\par 8 «شما چرا محال می‌پندارید که خدا مردگان را برخیزاند؟
\par 9 من هم در خاطر خود می‌پنداشتم که به نام عیسی ناصری مخالفت بسیار کردن واجب است،
\par 10 چنانکه در اورشلیم هم کردم واز روسای کهنه قدرت یافته، بسیاری از مقدسین را در زندان حبس می‌کردم و چون ایشان را می کشتند، در فتوا شریک می‌بودم.
\par 11 و در همه کنایس بارها ایشان را زحمت رسانیده، مجبورمی ساختم که کفر گویند و بر ایشان به شدت دیوانه گشته تا شهرهای بعید تعاقب می‌کردم.
\par 12 در این میان هنگامی که با قدرت و اجازت ازروسای کهنه به دمشق می‌رفتم،
\par 13 در راه، ای پادشاه، در وقت ظهر نوری را از آسمان دیدم، درخشنده تر از خورشید که در دور من و رفقایم تابید.
\par 14 و چون همه بر زمین افتادیم، هاتفی راشنیدم که مرا به زبان عبرانی مخاطب ساخته، گفت: "ای شاول، شاول، چرا بر من جفا می‌کنی؟ تو را بر میخها لگد زدن دشوار است."
\par 15 من گفتم: "خداوندا تو کیستی؟" گفت: "من عیسی هستم که تو بر من جفا می‌کنی.
\par 16 و لیکن برخاسته، بر پابایست زیرا که بر تو ظاهر شدم تا تو را خادم وشاهد مقرر گردانم بر آن چیزهایی که مرا در آنهادیده‌ای و بر آنچه به تو در آن ظاهر خواهم شد.
\par 17 و تو را رهایی خواهم داد از قوم و از امت هایی که تو را به نزد آنها خواهم فرستاد،
\par 18 تا چشمان ایشان را باز کنی تا از ظلمت به سوی نور و ازقدرت شیطان به‌جانب خدا برگردند تا آمرزش گناهان و میراثی در میان مقدسین بوسیله ایمانی که بر من است بیابند."
\par 19 آن وقت‌ای اغریپاس پادشاه، رویای آسمانی را نافرمانی نورزیدم.
\par 20 بلکه نخست آنانی را که در دمشق بودند و در اورشلیم و درتمامی مرز و بوم یهودیه و امت‌ها را نیز اعلام می‌نمودم که توبه کنند و به سوی خدا بازگشت نمایند و اعمال لایقه توبه را به‌جا آورند.
\par 21 به‌سبب همین امور یهود مرا در هیکل گرفته، قصد قتل من کردند.
\par 22 اما از خدا اعانت یافته، تا امروزباقی ماندم و خرد و بزرگ را اعلام می‌نمایم وحرفی نمی گویم، جز آنچه انبیا و موسی گفتند که می‌بایست واقع شود،
\par 23 که مسیح می‌بایست زحمت بیند و نوبر قیامت مردگان گشته، قوم وامت‌ها را به نور اعلام نماید.»
\par 24 چون او بدین سخنان، حجت خود رامی آورد، فستوس به آواز بلند گفت: «ای پولس دیوانه هستی! کثرت علم تو را دیوانه کرده است!»
\par 25 گفت: «ای فستوس گرامی، دیوانه نیستم بلکه سخنان راستی و هوشیاری را می‌گویم.
\par 26 زیراپادشاهی که در حضور او به دلیری سخن می‌گویم، از این امور مطلع است، چونکه مرا یقین است که هیچ‌یک از این مقدمات بر او مخفی نیست، زیرا که این امور در خلوت واقع نشد.
\par 27 ‌ای اغریپاس پادشاه، آیا به انبیا ایمان آورده‌ای؟ می‌دانم که ایمان داری!»
\par 28 اغریپاس به پولس گفت: «به قلیل ترغیب می‌کنی که من مسیحی بگردم؟»
\par 29 پولس گفت: «از خداخواهش می‌داشتم یا به قلیل یا به کثیر، نه‌تنها توبلکه جمیع این اشخاصی که امروز سخن مرامی شنوند مثل من گردند، جز این زنجیرها!»
\par 30 چون این را گفت، پادشاه و والی و برنیکی وسایر مجلسیان برخاسته،
\par 31 رفتند و با یکدیگرگفتگو کرده، گفتند: «این شخص هیچ عملی مستوجب قتل یا حبس نکرده است.»واغریپاس به فستوس گفت: «اگر این مرد به قیصررفع دعوی خود نمی کرد، او را آزاد کردن ممکن می‌بود.»
\par 32 واغریپاس به فستوس گفت: «اگر این مرد به قیصررفع دعوی خود نمی کرد، او را آزاد کردن ممکن می‌بود.»

\chapter{27}

\par 1 چون مقرر شد که به ایطالیا برویم، پولس و چند زندانی دیگر را به یوزباشی از سپاه اغسطس که یولیوس نام داشت، سپردند.
\par 2 و به کشتی ادرامیتینی که عازم بنادرآسیا بود، سوار شده، کوچ کردیم و ارسترخس ازاهل مکادونیه از تسالونیکی همراه ما بود.
\par 3 روزدیگر به صیدون فرود آمدیم و یولیوس با پولس ملاطفت نموده، او را اجازت داد که نزد دوستان خود رفته، از ایشان نوازش یابد.
\par 4 و از آنجا روانه شده، زیر قپرس گذشتیم زیرا که باد مخالف بود.
\par 5 و از دریای کنار قیلیقیه و پمفلیه گذشته، به میرای لیکیه رسیدیم
\par 6 در آنجا یوزباشی کشتی اسکندریه را یافت که به ایطالیا می‌رفت و ما را برآن سوار کرد.
\par 7 و چند روز به آهستگی رفته، به قنیدس به مشقت رسیدیم و چون باد مخالف مامی بود، در زیر کریت نزدیک سلمونی راندیم،
\par 8 وبه دشواری از آنجا گذشته، به موضعی که به بنادرحسنه مسمی و قریب به شهر لسائیه است رسیدیم.
\par 9 و چون زمان منقضی شد و در این وقت سفردریا خطرناک بود، زیرا که ایام روزه گذشته بود،
\par 10 پولس ایشان را نصیحت کرده، گفت: «ای مردمان، می‌بینم که در این سفر ضرر و خسران بسیار پیدا خواهد شد، نه فقط بار و کشتی را بلکه جانهای ما را نیز.»
\par 11 ولی یوزباشی ناخدا وصاحب کشتی را بیشتر از قول پولس اعتنا نمود.
\par 12 و چون آن بندر نیکو نبود که زمستان را در آن بسر برند، اکثر چنان مصلحت دانستند که از آنجا نقل کنند تا اگر ممکن شود خود را به فینیکسس رسانیده، زمستان را در آنجا بسر برند که آن بندری است از کریت مواجه مغرب جنوبی ومغرب شمالی.
\par 13 و چون نسیم جنوبی وزیدن گرفت، گمان بردند که به مقصد خویش رسیدند. پس لنگر برداشتیم و از کناره کریت گذشتیم.
\par 14 لیکن چیزی نگذشت که بادی شدید که آن رااورکلیدون می‌نامند از بالای آن زدن گرفت.
\par 15 درساعت کشتی ربوده شده، رو به سوی باد نتوانست نهاد. پس آن را از دست داده، بی‌اختیار رانده شدیم.
\par 16 پس در زیر جزیره‌ای که کلودی نام داشت، دوان دوان رفتیم و به دشواری زورق را درقبض خود آوردیم.
\par 17 و آن را برداشته و معونات را استعمال نموده، کمر کشتی را بستند و چون ترسیدند که به ریگزار سیرتس فرو روند، حبال کشتی را فرو کشیدند و همچنان رانده شدند.
\par 18 وچون طوفان بر ما غلبه می‌نمود، روز دیگر، بارکشتی را بیرون انداختند.
\par 19 و روز سوم به‌دستهای خود آلات کشتی را به دریا انداختیم.
\par 20 و چون روزهای بسیار آفتاب و ستارگان راندیدند و طوفانی شدید بر ما می‌افتاد، دیگر هیچ امید نجات برای ما نماند.
\par 21 و بعد از گرسنگی بسیار، پولس در میان ایشان ایستاده، گفت: «ای مردمان، نخست می‌بایست سخن مرا پذیرفته، از کریت نقل نکرده باشید تا این ضرر و خسران را نبینید.
\par 22 اکنون نیزشما را نصیحت می‌کنم که خاطرجمع باشید زیراکه هیچ ضرری به‌جان یکی از شما نخواهد رسیدمگر به کشتی.
\par 23 زیرا که دوش، فرشته آن خدایی که از آن او هستم و خدمت او را می‌کنم، به من ظاهر شده،
\par 24 گفت: "ای پولس ترسان مباش زیراباید تو در حضور قیصر حاضر شوی. و اینک خدا همه همسفران تو را به تو بخشیده است."
\par 25 پس‌ای مردمان خوشحال باشید زیرا ایمان دارم که به همانطور که به من گفت، واقع خواهدشد.
\par 26 لیکن باید در جزیره‌ای بیفتیم.»
\par 27 و چون شب چهاردهم شد و هنوز دردریای ادریا به هر سو رانده می‌شدیم، در نصف شب ملاحان گمان بردند که خشکی نزدیک است.
\par 28 پس پیمایش کرده، بیست قامت یافتند. وقدری پیشتر رفته، باز پیمایش کرده، پانزده قامت یافتند.
\par 29 و چون ترسیدند که به صخره‌ها بیفتیم، از پشت کشتی چهار لنگر انداخته، تمنا می‌کردندکه روز شود.
\par 30 اما چون ملاحان قصد داشتند که از کشتی فرار کنند و زورق را به دریا انداختند به بهانه‌ای که لنگرها را از پیش کشتی بکشند،
\par 31 پولس یوزباشی و سپاهیان را گفت: «اگر اینهادر کشتی نمانند، نجات شما ممکن نباشد.»
\par 32 آنگاه سپاهیان ریسمانهای زورق را بریده، گذاشتند که بیفتد.
\par 33 چون روز نزدیک شد، پولس از همه خواهش نمود که چیزی بخورند. پس گفت: «امروز روز چهاردهم است که انتظار کشیده وچیزی نخورده، گرسنه مانده‌اید.
\par 34 پس استدعای من این است که غذا بخورید که عافیت برای شما خواهد بود، زیرا که مویی از سر هیچ‌یک از شما نخواهد افتاد.»
\par 35 این بگفت و درحضور همه نان گرفته، خدا را شکر گفت و پاره کرده، خوردن گرفت.
\par 36 پس همه قوی‌دل گشته نیز غذا خوردند.
\par 37 و جمله نفوس در کشتی دویست و هفتاد و شش بودیم.
\par 38 چون از غذاسیر شدند، گندم را به دریا ریخته، کشتی را سبک کردند.
\par 39 اما چون روز، روشن شد، زمین را نشناختند؛ لیکن خلیجی دیدند که شاطی‌ای داشت. پس رای زدند که اگر ممکن شود، کشتی را بر آن برانند.
\par 40 و بند لنگرها را بریده، آنها را دردریا گذاشتند و بندهای سکان را باز کرده، وبادبان را برای باد گشاده، راه ساحل را پیش گرفتند.
\par 41 اما کشتی را درمجمع بحرین به پایاب رانده، مقدم آن فرو شده، بی‌حرکت ماند ولی موخرش از لطمه امواج درهم شکست.
\par 42 آنگاه سپاهیان قصد قتل زندانیان کردند که مبادا کسی شنا کرده، بگریزد.
\par 43 لیکن یوزباشی چون خواست پولس را برهاند، ایشان را از این اراده بازداشت و فرمود تا هر‌که شناوری داند، نخست خویشتن را به دریا انداخته به ساحل رساند.وبعضی بر تختها و بعضی بر چیزهای کشتی وهمچنین همه به سلامتی به خشکی رسیدند.
\par 44 وبعضی بر تختها و بعضی بر چیزهای کشتی وهمچنین همه به سلامتی به خشکی رسیدند.

\chapter{28}

\par 1 و چون رستگار شدند، یافتند که جزیره ملیطه نام دارد.
\par 2 و آن مردمان بربری باما کمال ملاطفت نمودند، زیرا به‌سبب باران که می‌بارید و سرما آتش افروخته، همه ما راپذیرفتند.
\par 3 چون پولس مقداری هیزم فراهم کرده، بر آتش می‌نهاد، به‌سبب حرارت، افعی‌ای بیرون آمده، بر دستش چسپید.
\par 4 چون بربریان جانور را از دستش آویخته دیدند، با یکدیگرمی گفتند: «بلاشک این شخص، خونی است که بااینکه از دریا رست، عدل نمی گذارد که زیست کند.»
\par 5 اما آن جانور را در آتش افکنده، هیچ ضرر نیافت.
\par 6 پس منتظر بودند که او آماس کند یابغته افتاده، بمیرد. ولی چون انتظار بسیار کشیدندو دیدند که هیچ ضرری بدو نرسید، برگشته گفتندکه خدایی است.
\par 7 و در آن نواحی، املاک رئیس جزیره که پوبلیوس نام داشت بود که او ما را به خانه خودطلبیده، سه روز به مهربانی مهمانی نمود.
\par 8 ازقضا پدر پوبلیوس را رنج تب و اسهال عارض شده، خفته بود. پس پولس نزد وی آمده و دعاکرده ودست بر او گذارده، او را شفا داد.
\par 9 و چون این امر واقع شد، سایر مریضانی که در جزیره بودند آمده، شفا یافتند.
\par 10 و ایشان ما را اکرام بسیار نمودند و چون روانه می‌شدیم، آنچه لازم بود برای ما حاضر ساختند.
\par 11 و بعد از سه ماه به کشتی اسکندریه که علامت جوزا داشت و زمستان را در جزیره بسربرده بود، سوار شدیم.
\par 12 و به‌سراکوس فرودآمده، سه روز توقف نمودیم.
\par 13 و از آنجا دورزده، به ریغیون رسیدیم و بعد از یک روز بادجنوبی وزیده، روز دوم وارد پوطیولی شدیم.
\par 14 و در آنجا برادران یافته، حسب خواهش ایشان هفت روز ماندیم و همچنین به روم آمدیم.
\par 15 وبرادران آنجا چون از احوال ما مطلع شدند، به استقبال ما بیرون آمدند تا فورن اپیوس و سه دکان. و پولس چون ایشان را دید، خدا را شکر نموده، قوی‌دل گشت.
\par 16 و چون به روم رسیدیم، یوزباشی زندانیان را به‌سردار افواج خاصه سپرد. اما پولس را اجازت دادند که با یک سپاهی که محافظت او می‌کرد، در منزل خودبماند.
\par 17 و بعد از سه روز، پولس بزرگان یهود راطلبید و چون جمع شدند به ایشان گفت: «ای برادران عزیز، با وجودی که من هیچ عملی خلاف قوم و رسوم اجداد نکرده بودم، همانا مرا در اورشلیم بسته، به‌دستهای رومیان سپردند.
\par 18 ایشان بعد از تفحص چون در من هیچ علت قتل نیافتند، اراده کردند که مرا رها کنند.
\par 19 ولی چون یهود مخالفت نمودند، ناچار شده به قیصررفع دعوی کردم، نه تا آنکه از امت خود شکایت کنم.
\par 20 اکنون بدین جهت خواستم شما راملاقات کنم و سخن گویم زیرا که بجهت امیداسرائیل، بدین زنجیر بسته شدم.»
\par 21 وی راگفتند: «ما هیچ نوشته در حق تو از یهودیه نیافته‌ایم و نه کسی از برادرانی که از آنجا آمدند، خبری یا سخن بدی درباره تو گفته است.
\par 22 لیکن مصلحت دانستیم از تو مقصود تو را بشنویم زیراما را معلوم است که این فرقه را در هر جا بدمی گویند.»
\par 23 پس چون روزی برای وی معین کردند، بسیاری نزد او به منزلش آمدند که برای ایشان به ملکوت خدا شهادت داده، شرح می‌نمود و ازتورات موسی و انبیا از صبح تا شام درباره عیسی اقامه حجت می‌کرد.
\par 24 پس بعضی به سخنان اوایمان آوردند و بعضی ایمان نیاوردند.
\par 25 و چون با یکدیگر معارضه می‌کردند، از او جدا شدند بعداز آنکه پولس این یک سخن را گفته بود که «روح‌القدس به وساطت اشعیای نبی به اجداد مانیکو خطاب کرده،
\par 26 گفته است که "نزد این قوم رفته بدیشان بگو به گوش خواهید شنید ونخواهید فهمید و نظر کرده خواهید نگریست ونخواهید دید؛
\par 27 زیرا دل این قوم غلیظ شده و به گوشهای سنگین می‌شنوند و چشمان خود را برهم نهاده‌اند، مبادا به چشمان ببینند و به گوشهابشنوند و به دل بفهمند و بازگشت کنند تا ایشان راشفا بخشم."
\par 28 پس بر شما معلوم باد که نجات خدا نزد امت‌ها فرستاده می‌شود و ایشان خواهندشنید.»
\par 29 چون این را گفت یهودیان رفتند و بایکدیگر مباحثه بسیار می‌کردند.اما پولس دوسال تمام در خانه اجاره‌ای خود ساکن بود و هرکه به نزد وی می‌آمد، می‌پذیرفت.
\par 30 اما پولس دوسال تمام در خانه اجاره‌ای خود ساکن بود و هرکه به نزد وی می‌آمد، می‌پذیرفت.



\end{document}