\begin{document}

\title{2 Corinthians}


\chapter{1}

\par 1 پولس به اراده خدا رسول عیسی مسیح وتیموتاوس برادر، به کلیسای خدا که در قرنتس می‌باشد با همه مقدسینی که در تمام اخائیه هستند،
\par 2 فیض و سلامتی از پدر ما خدا و عیسی مسیح خداوند به شما باد.
\par 3 متبارک باد خدا و پدر خداوند ما عیسی مسیح که پدر رحمتها و خدای جمیع تسلیات است،
\par 4 که ما را در هر تنگی ما تسلی می‌دهد تا مابتوانیم دیگران را در هر مصیبتی که باشد تسلی نماییم، به آن تسلی که خود از خدا یافته‌ایم.
\par 5 زیرا به اندازه‌ای که دردهای مسیح در ما زیاده شود، به همین قسم تسلی ما نیز بوسیله مسیح می‌افزاید.
\par 6 اما خواه زحمت کشیم، این است برای تسلی و نجات شما، و خواه تسلی پذیریم این هم بجهت تسلی و نجات شما است که میسرمی شود از صبر داشتن در همین دردهایی که ما هم می‌بینیم.
\par 7 و امید ما برای شما استوار می‌شودزیرا می‌دانیم که چنانکه شما شریک دردهاهستید، همچنین شریک تسلی نیز خواهید بود.
\par 8 زیرا‌ای برادران نمی خواهیم شما بی‌خبرباشید از تنگی‌ای که در آسیا به ما عارض گردید که بینهایت و فوق از طاقت بار کشیدیم، بحدی که از جان هم مایوس شدیم.
\par 9 لکن در خود فتوای موت داشتیم تا بر خود توکل نکنیم، بلکه بر خداکه مردگان را برمی خیزاند،
\par 10 که ما را از چنین موت رهانید و می‌رهاند و به او امیدواریم که بعداز این هم خواهد رهانید.
\par 11 و شما نیز به دعا درحق ما اعانت می‌کنید تا آنکه برای آن نعمتی که ازاشخاص بسیاری به ما رسید، شکرگزاری هم بجهت ما از بسیاری به‌جا آورده شود.
\par 12 زیرا که فخر ما این است یعنی شهادت ضمیر ما که به قدوسیت و اخلاص خدایی، نه به حکمت جسمانی، بلکه به فیض الهی در جهان رفتار نمودیم و خصوص نسبت به شما.
\par 13 زیراچیزی به شما نمی نویسیم مگر آنچه می‌خوانید وبه آن اعتراف می‌کنید و امیدوارم که تا به آخراعتراف هم خواهید کرد.
\par 14 چنانکه به مافی الجمله اعتراف گردید که محل فخر شماهستیم، چنانکه شما نیز ما را می‌باشید در روزعیسی خداوند.
\par 15 و بدین اعتماد قبل از این خواستم به نزدشما آیم تا نعمتی دیگر بیابید،
\par 16 و از راه شما به مکادونیه بروم و باز از مکادونیه نزد شما بیایم و شما مرا به سوی یهودیه مشایعت کنید.
\par 17 پس چون این را خواستم، آیا سهل انگاری کردم یاعزیمت من عزیمت بشری باشد تا آنکه به نزد من بلی بلی و نی نی باشد.
\par 18 لیکن خدا امین است که سخن ما با شما بلی و نی نیست.
\par 19 زیرا که پسرخدا عیسی مسیح که ما یعنی من و سلوانس وتیموتاوس در میان شما به وی موعظه کردیم، بلی و نی نشد بلکه در او بلی شده است.
\par 20 زیراچندان‌که وعده های خدا است، همه در او بلی واز این جهت در او امین است تا خدا از ما تمجیدیابد.
\par 21 اما او که ما را با شما در مسیح استوارمی گرداند و ما را مسح نموده است، خداست.
\par 22 که او نیز ما را مهر نموده و بیعانه روح را دردلهای ما عطا کرده است.
\par 23 لیکن من خدا را بر جان خود شاهدمی خوانم که برای شفقت بر شما تا بحال به قرنتس نیامدم،نه آنکه بر ایمان شما حکم کرده باشیم بلکه شادی شما را مددکار هستیم زیرا که به ایمان قایم هستید.
\par 24 نه آنکه بر ایمان شما حکم کرده باشیم بلکه شادی شما را مددکار هستیم زیرا که به ایمان قایم هستید.

\chapter{2}

\par 1 اما در دل خود عزیمت داشتم که دیگر باحزن به نزد شما نیایم،
\par 2 زیرا اگر من شما رامحزون سازم، کیست که مرا شادی دهد جز او که از من محزون گشت؟
\par 3 و همین را نوشتم که مباداوقتی که بیایم محزون شوم از آنانی که می‌بایست سبب خوشی من بشوند، چونکه بر همه شمااعتماد می‌دارم که شادی من، شادی جمیع شمااست.
\par 4 زیرا که از حزن و دلتنگی سخت و بااشکهای بسیار به شما نوشتم، نه تا محزون شویدبلکه تا بفهمید چه محبت بینهایتی با شما دارم.
\par 5 و اگر کسی باعث غم شد، مرا محزون نساخت بلکه فی الجمله جمیع شما را تا بار زیاده ننهاده باشم.
\par 6 کافی است آن کس را این سیاستی که از اکثر شما بدو رسیده است.
\par 7 پس برعکس شما باید او را عفو نموده، تسلی دهید که مباداافزونی غم چنین شخص را فرو برد.
\par 8 بنابراین، به شما التماس می‌دارم که با او محبت خود رااستوار نمایید.
\par 9 زیرا که برای همین نیز نوشتم تادلیل شما را بدانم که در همه‌چیز مطیع می‌باشید.
\par 10 اما هرکه را چیزی عفو نمایید، من نیز می‌کنم زیرا که آنچه من عفو کرده‌ام، هرگاه چیزی را عفوکرده باشم، به‌خاطر شما به حضور مسیح کرده‌ام،
\par 11 تا شیطان بر ما برتری نیابد، زیرا که از مکاید اوبی خبر نیستیم.
\par 12 اما چون به تروآس بجهت بشارت مسیح آمدم و دروازه‌ای برای من در خداوند باز شد،
\par 13 در روح خود آرامی نداشتم، از آن رو که برادرخود تیطس را نیافتم، بلکه ایشان را وداع نموده، به مکادونیه آمدم.
\par 14 لیکن شکر خدا راست که مارا در مسیح، دائم در موکب ظفر خود می‌برد وعطر معرفت خود را در هرجا بوسیله ما ظاهرمی کند.
\par 15 زیرا خدا را عطر خوشبوی مسیح می‌باشیم هم در ناجیان و هم در هالکان.
\par 16 امااینها را عطر موت، الی موت و آنها را عطر حیات الی حیات. و برای این امور کیست که کافی باشد؟زیرا مثل بسیاری نیستیم که کلام خدا را مغشوش سازیم، بلکه از ساده دلی و از جانب خدا در حضور خدا در مسیح سخن می‌گوییم.
\par 17 زیرا مثل بسیاری نیستیم که کلام خدا را مغشوش سازیم، بلکه از ساده دلی و از جانب خدا در حضور خدا در مسیح سخن می‌گوییم.

\chapter{3}

\par 1 آیا باز به سفارش خود شروع می‌کنیم؟ وآیا مثل بعضی احتیاج به سفارش نامه جات به شما یا از شما داشته باشیم؟
\par 2 شما رساله ماهستید، نوشته شده در دلهای ما، معروف وخوانده شده جمیع آدمیان.
\par 3 چونکه ظاهرشده‌اید که رساله مسیح می‌باشید، خدمت کرده شده از ما و نوشته شده نه به مرکب بلکه به روح خدای حی، نه بر الواح سنگ، بلکه بر الواح گوشتی دل.
\par 4 اما بوسیله مسیح چنین اعتماد به خدا داریم.
\par 5 نه آنکه کافی باشیم که چیزی را به خود تفکر کنیم که گویا از ما باشد، بلکه کفایت مااز خداست.
\par 6 که او ما را هم کفایت داد تا عهدجدید را خادم شویم، نه حرف را بلکه روح رازیرا که حرف می‌کشد لیکن روح زنده می‌کند.
\par 7 اما اگر خدمت موت که در حرف بود و برسنگها تراشیده شده با جلال می‌بود، بحدی که بنی‌اسرائیل نمی توانستند صورت موسی را نظاره کنند به‌سبب جلال چهره او که فانی بود،
\par 8 چگونه خدمت روح بیشتر با جلال نخواهد بود؟
\par 9 زیراهرگاه خدمت قصاص با جلال باشد، چند مرتبه زیادتر خدمت عدالت در جلال خواهد افزود.
\par 10 زیرا که آنچه جلال داده شده بود نیز بدین نسبت جلالی نداشت به‌سبب این جلال فایق.
\par 11 زیرا اگر آن فانی با جلال بودی، هرآینه این باقی از طریق اولی در جلال خواهد بود.
\par 12 پس چون چنین امید داریم، با کمال دلیری سخن می‌گوییم.
\par 13 و نه مانند موسی که نقابی برچهره خود کشید تا بنی‌اسرائیل، تمام شدن این فانی را نظر نکنند،
\par 14 بلکه ذهن ایشان غلیظ شدزیرا که تا امروز همان نقاب در خواندن عهد عتیق باقی است و کشف نشده است، زیرا که فقط درمسیح باطل می‌گردد.
\par 15 بلکه تا امروز وقتی که موسی را می‌خوانند، نقاب بر دل ایشان برقرارمی ماند.
\par 16 لیکن هرگاه به سوی خداوند رجوع کند، نقاب برداشته می‌شود.
\par 17 اما خداوند روح است و جایی که روح خداوند است، آنجا آزادی است.لیکن همه ما چون با چهره بی‌نقاب جلال خداوند را در آینه می‌نگریم، از جلال تاجلال به همان صورت متبدل می‌شویم، چنانکه از خداوند که روح است.
\par 18 لیکن همه ما چون با چهره بی‌نقاب جلال خداوند را در آینه می‌نگریم، از جلال تاجلال به همان صورت متبدل می‌شویم، چنانکه از خداوند که روح است.

\chapter{4}

\par 1 بنابراین چون این خدمت را داریم، چنانکه رحمت یافته‌ایم، خسته خاطر نمی شویم.
\par 2 بلکه خفایای رسوایی را ترک کرده، به مکر رفتارنمی کنیم و کلام خدا را مغشوش نمی سازیم، بلکه به اظهار راستی، خود را به ضمیر هرکس درحضور خدا مقبول می‌سازیم.
\par 3 لیکن اگر بشارت ما مخفی است، بر هالکان مخفی است،
\par 4 که درایشان خدای این جهان فهم های بی‌ایمانشان راکور گردانیده است که مبادا تجلی بشارت جلال مسیح که صورت خداست، ایشان را روشن سازد.
\par 5 زیرا به خویشتن موعظه نمی کنیم بلکه به مسیح عیسی خداوند، اما به خویشتن که غلام شما هستیم بخاطر عیسی.
\par 6 زیرا خدایی که گفت تانور از ظلمت درخشید، همان است که در دلهای ما درخشید تا نور معرفت جلال خدا در چهره عیسی مسیح از ما بدرخشد.
\par 7 لیکن این خزینه را در ظروف خاکی داریم تابرتری قوت از آن خدا باشد نه از جانب ما.
\par 8 درهرچیز زحمت کشیده، ولی در شکنجه نیستیم؛ متحیر ولی مایوس نی؛
\par 9 تعاقب کرده شده، لیکن نه متروک؛ افکنده شده، ولی هلاک شده نی؛
\par 10 پیوسته قتل عیسی خداوند را در جسد خودحمل می‌کنیم تا حیات عیسی هم در بدن ما ظاهرشود.
\par 11 زیرا ما که زنده‌ایم، دائم بخاطر عیسی به موت سپرده می‌شویم تا حیات عیسی نیز درجسد فانی ما پدید آید.
\par 12 پس موت در ما کارمی کند ولی حیات در شما.
\par 13 اما چون همان روح ایمان را داریم، بحسب آنچه مکتوب است «ایمان آوردم پس سخن گفتم»، ما نیز چون ایمان داریم، از اینرو سخن می‌گوییم.
\par 14 چون می‌دانیم او که عیسی خداوندرا برخیزانید، ما را نیز با عیسی خواهد برخیزانیدو با شما حاضر خواهد ساخت.
\par 15 زیرا که همه‌چیز برای شما است تا آن فیضی که بوسیله بسیاری افزوده شده است، شکرگزاری را برای تمجید خدا بیفزاید.
\par 16 از این جهت خسته خاطرنمی شویم، بلکه هرچند انسانیت ظاهری ما فانی می‌شود، لیکن باطن روز بروز تازه می‌گردد.
\par 17 زیرا که این زحمت سبک ما که برای لحظه‌ای است، بار جاودانی جلال را برای ما زیاده و زیاده پیدا می‌کند.در حالی که ما نظر نمی کنیم به چیزهای دیدنی، بلکه به چیزهای نادیدنی، زیرا که آنچه دیدنی است، زمانی است و نادیدنی جاودانی.
\par 18 در حالی که ما نظر نمی کنیم به چیزهای دیدنی، بلکه به چیزهای نادیدنی، زیرا که آنچه دیدنی است، زمانی است و نادیدنی جاودانی.

\chapter{5}

\par 1 زیرا می‌دانیم که هرگاه این خانه زمینی خیمه ما ریخته شود، عمارتی از خداداریم، خانه‌ای ناساخته شده به‌دستها و جاودانی در آسمانها.
\par 2 زیرا که در این هم آه می‌کشیم، چونکه مشتاق هستیم که خانه خود را که ازآسمان است بپوشیم،
\par 3 اگر فی الواقع پوشیده و نه عریان یافت شویم.
\par 4 از آنرو که ما نیز که در این خیمه هستیم، گرانبار شده، آه می‌کشیم، از آن جهت که نمی خواهیم این را بیرون کنیم، بلکه آن را بپوشیم تا فانی در حیات غرق شود.
\par 5 اما او که ما را برای این درست ساخت خدا است که بیعانه روح را به ما می‌دهد.
\par 6 پس دائم خاطرجمع هستیم و می‌دانیم که مادامی که در بدن متوطنیم، از خداوند غریب می‌باشیم.
\par 7 (زیرا که به ایمان رفتار می‌کنیم نه به دیدار)
\par 8 پس خاطرجمع هستیم و این را بیشترمی پسندیم که از بدن غربت کنیم و به نزد خداوندمتوطن شویم.
\par 9 لهذا حریص هستیم بر اینکه خواه متوطن و خواه غریب، پسندیده او باشیم.
\par 10 زیرا لازم است که همه ما پیش مسند مسیح حاضر شویم تا هرکس اعمال بدنی خود را بیابد، بحسب آنچه کرده باشد، چه نیک چه بد.
\par 11 پس چون ترس خدا را دانسته‌ایم، مردم رادعوت می‌کنیم. اما به خدا ظاهر شده‌ایم و امیدوارم به ضمایر شما هم ظاهر خواهیم شد.
\par 12 زیرا بار دیگر برای خود به شما سفارش نمی کنیم، بلکه سبب افتخار درباره خود به شمامی دهیم تا شما را جوابی باشد برای آنانی که درظاهر نه در دل فخر می‌کنند.
\par 13 زیرا اگر بی‌خودهستیم برای خداست و اگر هشیاریم برای شمااست.
\par 14 زیرا محبت مسیح ما را فرو گرفته است، چونکه این را دریافتیم که یک نفر برای همه مردپس همه مردند.
\par 15 و برای همه مرد تا آنانی که زنده‌اند، از این به بعد برای خویشتن زیست نکنندبلکه برای او که برای ایشان مرد و برخاست.
\par 16 بنابراین، ما بعد از این هیچ‌کس را بحسب جسم نمی شناسیم، بلکه هرگاه مسیح را هم بحسب جسم شناخته بودیم، الان دیگر او رانمی شناسیم.
\par 17 پس اگر کسی در مسیح باشد، خلقت تازه‌ای است؛ چیزهای کهنه درگذشت، اینک همه‌چیز تازه شده است.
\par 18 و همه‌چیز ازخدا که ما را بواسطه عیسی مسیح با خود مصالحه داده و خدمت مصالحه را به ما سپرده است.
\par 19 یعنی‌اینکه خدا در مسیح بود و جهان را باخود مصالحه می‌داد و خطایای ایشان را بدیشان محسوب نداشت و کلام مصالحه را به ما سپرد.
\par 20 پس برای مسیح ایلچی هستیم که گویا خدا به زبان ما وعظ می‌کند. پس بخاطر مسیح استدعامی کنیم که با خدا مصالحه کنید.زیرا او را که گناه نشناخت در راه ما گناه ساخت تا ما در وی عدالت خدا شویم.
\par 21 زیرا او را که گناه نشناخت در راه ما گناه ساخت تا ما در وی عدالت خدا شویم.

\chapter{6}

\par 1 پس چون همکاران او هستیم، التماس می نماییم که فیض خدا را بی‌فایده نیافته باشید.
\par 2 زیرا می‌گوید: «در وقت مقبول تو را مستجاب فرمودم و در روز نجات تو را اعانت کردم.» اینک الحال زمان مقبول است؛ اینک الان روز نجات است.
\par 3 در هیچ‌چیز لغزش نمی دهیم که مباداخدمت ما ملامت کرده شود،
\par 4 بلکه در هر امری خود را ثابت می‌کنیم که خدام خدا هستیم: درصبر بسیار، در زحمات، در حاجات در تنگیها،
\par 5 در تازیانه‌ها، در زندانها، در فتنه‌ها، در محنتها، در بی‌خوابیها، در گرسنگیها،
\par 6 در طهارت، درمعرفت، در حلم، در مهربانی، در روح‌القدس، درمحبت بی‌ریا،
\par 7 در کلام حق، در قوت خدا بااسلحه عدالت بر طرف راست و چپ،
\par 8 به عزت وذلت و بدنامی و نیکنامی. چون گمراه کنندگان واینک راستگو هستیم؛
\par 9 چون مجهول و اینک معروف؛ چون در حالت موت و اینک زنده هستیم؛ چون سیاست کرده شده، اما مقتول نی؛
\par 10 چون محزون، ولی دائم شادمان؛ چون فقیر واینک بسیاری را دولتمند می‌سازیم؛ چون بی‌چیز، اما مالک همه‌چیز.
\par 11 ‌ای قرنتیان، دهان ما به سوی شما گشاده ودل ما وسیع شده است.
\par 12 در ما تنگ نیستید لیکن در احشای خود تنگ هستید.
\par 13 پس در جزای این، زیرا که به فرزندان خود سخن می‌گویم، شمانیز گشاده شوید.
\par 14 زیر یوغ ناموافق با بی‌ایمانان مشوید، زیرا عدالت را با گناه چه رفاقت و نور را باظلمت چه شراکت است؟
\par 15 و مسیح را با بلیعال چه مناسبت و مومن را با کافر چه نصیب است؟
\par 16 و هیکل خدا را با بتها چه موافقت؟ زیرا شماهیکل خدای حی می‌باشید، چنانکه خدا گفت که «در ایشان ساکن خواهم بود و در ایشان راه خواهم رفت و خدای ایشان خواهم بود، و ایشان قوم من خواهند بود.»
\par 17 پس خداوند می‌گوید: «از میان ایشان بیرون آیید و جدا شوید و چیزناپاک را لمس مکنید تا من شما را مقبول بدارم،و شما را پدر خواهم بود و شما مرا پسران ودختران خواهید بود؛ خداوند قادر مطلق می‌گوید.»
\par 18 و شما را پدر خواهم بود و شما مرا پسران ودختران خواهید بود؛ خداوند قادر مطلق می‌گوید.»

\chapter{7}

\par 1 پس‌ای عزیزان، چون این وعده‌ها راداریم، خویشتن را از هر نجاست جسم وروح طاهر بسازیم و قدوسیت را در خدا ترسی به‌کمال رسانیم.
\par 2 ما را در دلهای خود جا دهید. بر هیچ‌کس ظلم نکردیم و هیچ‌کس را فاسد نساختیم وهیچ‌کس را مغبون ننمودیم.
\par 3 این را از روی مذمت نمی گویم، زیرا پیش گفتم که در دل ماهستید تا در موت و حیات با هم باشیم.
\par 4 مرا برشما اعتماد کلی و درباره شما فخر کامل است. ازتسلی سیر گشته‌ام و در هر زحمتی که بر مامی آید، شادی وافر می‌کنم.
\par 5 زیرا چون به مکادونیه هم رسیدیم، جسم ماآرامی نیافت، بلکه در هرچیز زحمت کشیدیم؛ در ظاهر، نزاعها و در باطن، ترسها بود.
\par 6 لیکن خدایی که تسلی دهنده افتادگان است، ما را به آمدن تیطس تسلی بخشید.
\par 7 و نه از آمدن او تنهابلکه به آن تسلی نیز که او در شما یافته بود، چون ما را مطلع ساخت از شوق شما و نوحه گری شماو غیرتی که درباره من داشتید، به نوعی که بیشترشادمان گردیدم.
\par 8 زیرا که هرچند شما را به آن رساله محزون ساختم، پشیمان نیستم، اگرچه پشیمان هم بودم زیرا یافتم که آن رساله شما رااگر هم به ساعتی، غمگین ساخت.
\par 9 الحال شادمانم، نه از آنکه غم خوردید بلکه از اینکه غم شما به توبه انجامید، زیرا که غم شما برای خدابود تا به هیچ وجه زیانی از ما به شما نرسد.
\par 10 زیرا غمی که برای خداست منشا توبه می‌باشد به جهت نجات که از آن پشیمانی نیست؛ اما غم دنیوی منشا موت است.
\par 11 زیرا اینک همین‌که غم شما برای خدا بود، چگونه کوشش، بل احتجاج، بل خشم، بل ترس، بل اشتیاق، بل غیرت، بل انتقام را در شما پدید آورد. در هر چیزخود را ثابت کردید که در این امر مبرا هستید.
\par 12 باری هرگاه به شما نوشتم، بجهت آن ظالم یا مظلوم نبود، بلکه تا غیرت ما درباره شما به شمادر حضور خدا ظاهر شود.
\par 13 و از این جهت تسلی یافتیم لیکن در تسلی خود شادی ما ازخوشی تیطس بینهایت زیاده گردید چونکه روح او از جمیع شما آرامی یافته بود.
\par 14 زیرا اگردرباره شما بدو فخر کردم، خجل نشدم بلکه چنانکه همه سخنان را به شما به راستی گفتیم، همچنین فخر ما به تیطس راست شد.
\par 15 و خاطراو به سوی شما زیادتر مایل گردید، چونکه اطاعت جمیع شما را به یاد می‌آورد که چگونه به ترس و لرز او را پذیرفتید.شادمانم که درهرچیز بر شما اعتماد دارم.
\par 16 شادمانم که درهرچیز بر شما اعتماد دارم.

\chapter{8}

\par 1 لیکن‌ای برادران شما را مطلع می‌سازیم ازفیض خدا که به کلیساهای مکادونیه عطاشده است.
\par 2 زیرا در امتحان شدید زحمت، فراوانی خوشی ایشان ظاهر گردید و از زیادتی فقر ایشان، دولت سخاوت ایشان افزوده شد.
\par 3 زیرا که شاهد هستم که بحسب طاقت بلکه فوق از طاقت خویش به رضامندی تمام،
\par 4 التماس بسیار نموده، این نعمت و شراکت در خدمت مقدسین را از ما طلبیدند.
\par 5 و نه‌چنانکه امیدداشتیم، بلکه اول خویشتن را به خداوند و به مابرحسب اراده خدا دادند.
\par 6 و از این سبب از تیطس استدعا نمودیم که همچنانکه شروع این نعمت را در میان شما کرد، آن را به انجام هم برساند.
\par 7 بلکه چنانکه درهرچیز افزونی دارید، در ایمان و کلام و معرفت وکمال اجتهاد و محبتی که با ما می‌دارید، در این نعمت نیز بیفزایید.
\par 8 این را به طریق حکم نمی گویم بلکه به‌سبب اجتهاد دیگران و تااخلاص محبت شما را بیازمایم.
\par 9 زیرا که فیض خداوند ما عیسی مسیح را می‌دانید که هرچنددولتمند بود، برای شما فقیر شد تا شما از فقر اودولتمند شوید.
\par 10 و در این، رای می‌دهم زیرا که این شما را شایسته است، چونکه شما در سال گذشته، نه در عمل فقط بلکه در اراده نیز اول ازهمه شروع کردید.
\par 11 اما الحال عمل را به انجام رسانید تا چنانکه دلگرمی در اراده بود، انجام عمل نیز برحسب آنچه دارید بشود.
\par 12 زیراهرگاه دلگرمی باشد، مقبول می‌افتد، بحسب آنچه کسی دارد نه بحسب آنچه ندارد.
\par 13 و نه اینکه دیگران را راحت و شما را زحمت باشد، بلکه به طریق مساوات تا در حال، زیادتی شمابرای کمی ایشان بکار آید؛
\par 14 و تا زیادتی ایشان بجهت کمی شما باشد و مساوات بشود.
\par 15 چنانکه مکتوب است: «آنکه بسیار جمع کرد، زیادتی نداشت و آنکه اندکی جمع کرد، کمی نداشت.
\par 16 اما شکر خداراست که این اجتهاد را برای شما در دل تیطس نهاد.
\par 17 زیرا او خواهش ما رااجابت نمود، بلکه بیشتر با اجتهاد بوده، به رضامندی تمام به سوی شما روانه شد.
\par 18 و باوی آن برادری را فرستادیم که مدح او در انجیل در تمامی کلیساها است.
\par 19 و نه همین فقط بلکه کلیساها نیز او را اختیار کردند تا در این نعمتی که خدمت آن را برای تمجید خداوند و دلگرمی شما می‌کنیم، هم‌سفر مابشود.
\par 20 چونکه اجتناب می‌کنیم که مبادا کسی ما را ملامت کند درباره این سخاوتی که خادمان آن هستیم.
\par 21 زیرا که نه درحضور خداوند فقط، بلکه در نظر مردم نیزچیزهای نیکو را تدارک می‌بینیم.
\par 22 و با ایشان برادر خود را نیز فرستادیم که مکرر در اموربسیار او را با اجتهاد یافتیم و الحال به‌سبب اعتماد کلی که بر شما می‌دارد، بیشتر با اجتهاداست.
\par 23 هرگاه درباره تیطس (بپرسند)، او درخدمت شما رفیق و همکار من است؛ و اگر درباره برادران ما، ایشان رسل کلیساها و جلال مسیح می‌باشند.پس دلیل محبت خود و فخر ما را درباره شما در حضور کلیساها به ایشان ظاهرنمایید.
\par 24 پس دلیل محبت خود و فخر ما را درباره شما در حضور کلیساها به ایشان ظاهرنمایید.

\chapter{9}

\par 1 زیرا که در خصوص این خدمت مقدسین، زیادتی می‌باشد که به شما بنویسم.
\par 2 چونکه دلگرمی شما را می‌دانم که درباره آن بجهت شما به اهل مکادونیه فخر می‌کنم که ازسال گدشته اهل اخائیه مستعد شده‌اند و غیرت شما اکثر ایشان را تحریض نموده است.
\par 3 امابرادران را فرستادم که مبادا فخر ما درباره شما دراین خصوص باطل شود تا چنانکه گفته‌ام، مستعدشوید.
\par 4 مبادا اگر اهل مکادونیه با من آیند و شمارا نامستعد یابند، نمی گویم شما بلکه ما از این اعتمادی که به آن فخر کردیم، خجل شویم،
\par 5 پس لازم دانستم که برادران را نصیحت کنم تاقبل از ما نزد شما آیند و برکت موعود شما را مهیاسازند تا حاضر باشد، از راه برکت نه از راه طمع.
\par 6 اما خلاصه این است، هرکه با بخیلی کارد، بابخیلی هم درو کند و هرکه با برکت کارد، با برکت نیز درو کند.
\par 7 اما هرکس بطوری که در دل خوداراده نموده است بکند، نه به حزن و اضطرار، زیراخدا بخشنده خوش را دوست می‌دارد.
\par 8 ولی خدا قادر است که هر نعمتی را برای شما بیفزایدتا همیشه در هر امری کفایت کامل داشته، برای هر عمل نیکو افزوده شوید.
\par 9 چنانکه مکتوب است که «پاشید و به فقرا داد و عدالتش تا به ابدباقی می‌ماند».
\par 10 اما او که برای برزگر بذر و برای خورنده نان را آماده می‌کند، بذر شما را آماده کرده، خواهد افزود و ثمرات عدالت شما را مزید خواهد کرد.
\par 11 تا آنکه در هرچیز دولتمند شده، کمال سخاوت را بنمایید که آن منشا شکر خدابوسیله ما می‌باشد.
\par 12 زیرا که به‌جا آوردن این خدمت، نه فقط حاجات مقدسین را رفع می‌کند، بلکه سپاس خدا را نیز بسیار می‌افزاید.
\par 13 و ازدلیل این خدمت، خدا را تمجید می‌کنند به‌سبب اطاعت شما در اعتراف انجیل مسیح و سخاوت بخشش شما برای ایشان و همگان.
\par 14 و ایشان به‌سبب افزونی فیض خدایی که بر شماست، دردعای خود مشتاق شما می‌باشند.خدا را برای عطای ما لاکلام او شکر باد.
\par 15 خدا را برای عطای ما لاکلام او شکر باد.

\chapter{10}

\par 1 اما من خود، پولس، که چون در میان شما حاضر بودم، فروتن بودم، لیکن وقتی که غایب هستم، با شما جسارت می‌کنم، ازشما به حلم و رافت مسیح استدعا دارم
\par 2 والتماس می‌کنم که چون حاضر شوم، جسارت نکنم بدان اعتمادی که گمان می‌برم که جرات خواهم کرد با آنانی که می‌پندارند که ما به طریق جسم رفتار می‌کنیم.
\par 3 زیرا هرچند در جسم رفتار می‌کنیم، ولی به قانون جسمی جنگ نمی نماییم.
\par 4 زیرا اسلحه جنگ ما جسمانی نیست بلکه نزد خدا قادر است برای انهدام قلعه‌ها،
\par 5 که خیالات و هر بلندی را که خود را به خلاف معرفت خدا می‌افرازد، به زیر می‌افکنیم وهر فکری را به اطاعت مسیح اسیر می‌سازیم.
\par 6 ومستعد هستیم که از هر معصیت انتقام جوییم وقتی که اطاعت شما کامل شود.
\par 7 آیا به صورت ظاهری نظر می‌کنید؟ اگر کسی بر خود اعتماد دارد که از آن مسیح است، این را نیز از خود بداند که چنانکه او از آن مسیح است، ما نیز همچنان از آن مسیح هستیم.
\par 8 زیراهرچند زیاده هم فخر بکنم درباره اقتدار خود که خداوند آن را برای بنا نه برای خرابی شما به ماداده است، خجل نخواهم شد،
\par 9 که مبادا معلوم شود که شما را به رساله‌ها می‌ترسانم.
\par 10 زیرامی گویند: «رساله های او گران و زورآور است، لیکن حضور جسمی او ضعیف و سخنش حقیر.»
\par 11 چنین شخص بداند که چنانکه در کلام به رساله‌ها در غیاب هستیم، همچنین نیز در فعل درحضور خواهیم بود.
\par 12 زیرا جرات نداریم که خود را از کسانی که خویشتن را مدح می‌کنند بشماریم، یا خود را باایشان مقابله نماییم؛ بلکه ایشان چون خود را باخود می‌پیمایند و خود را به خود مقابله می‌نمایند، دانا نیستند.
\par 13 اما ما زیاده از اندازه فخر نمی کنیم، بلکه بحسب اندازه آن قانونی که خدا برای ما پیمود، و آن اندازه‌ای است که به شمانیز می‌رسد.
\par 14 زیرا از حد خود تجاوز نمی کنیم که گویا به شما نرسیده باشیم، چونکه در انجیل مسیح به شما هم رسیده‌ایم.
\par 15 و از اندازه خودنگذشته در محنتهای دیگران فخر نمی نماییم، ولی امید داریم که چون ایمان شما افزون شود، در میان شما بحسب قانون خود بغایت افزوده خواهیم شد.
\par 16 تا اینکه در مکانهای دورتر ازشما هم بشارت دهیم و در امور مهیا شده به قانون دیگران فخر نکنیم.
\par 17 اما هرکه فخر نماید، به خداوند فخر بنماید.زیرا نه آنکه خود را مدح کند مقبول افتد بلکه آن را که خداوند مدح نماید.
\par 18 زیرا نه آنکه خود را مدح کند مقبول افتد بلکه آن را که خداوند مدح نماید.

\chapter{11}

\par 1 کاشکه مرا در اندک جهالتی متحمل شوید و متحمل من هم می‌باشید.
\par 2 زیراکه من بر شما غیور هستم به غیرت الهی؛ زیرا که شما را به یک شوهر نامزد ساختم تا باکره‌ای عفیفه به مسیح سپارم.
\par 3 لیکن می‌ترسم که چنانکه مار به مکر خود حوا را فریفت، همچنین خاطرشما هم از سادگی‌ای که در مسیح است، فاسدگردد.
\par 4 زیرا هرگاه آنکه آمد، وعظ می‌کرد به عیسای دیگر، غیر از آنکه ما بدو موعظه کردیم، یا شما روحی دیگر را جز آنکه یافته بودید، یاانجیلی دیگر را سوای آنچه قبول کرده بودیدمی پذیرفتید، نیکو می‌کردید که متحمل می‌شدید.
\par 5 زیرا مرا یقین است که از بزرگترین رسولان هرگز کمتر نیستم.
\par 6 اما هرچند در کلام نیز امی باشم، لیکن در معرفت نی. بلکه در هر امری نزدهمه کس به شما آشکار گردیدیم.
\par 7 آیا گناه کردم که خود را ذلیل ساختم تا شما سرافراز شوید دراینکه به انجیل خدا شما را مفت بشارت دادم؟
\par 8 کلیساهای دیگر را غارت نموده، اجرت گرفتم تا شما را خدمت نمایم و چون به نزد شما حاضربوده، محتاج شدم، بر هیچ‌کس بار ننهادم.
\par 9 زیرابرادرانی که از مکادونیه آمدند، رفع حاجت مرانمودند و در هرچیز از بار نهادن بر شما خود رانگاه داشته و خواهم داشت.
\par 10 به راستی مسیح که در من است قسم که این فخر در نواحی اخائیه از من گرفته نخواهد شد.
\par 11 از چه سبب؟ آیا ازاینکه شما را دوست نمی دارم؟ خدا می‌داند!
\par 12 لیکن آنچه می‌کنم هم خواهم کرد تا از جویندگان فرصت، فرصت را منقطع سازم تا درآنچه فخر می‌کنند، مثل ما نیز یافت شوند.
\par 13 زیرا که چنین اشخاص رسولان کذبه و عمله مکار هستند که خویشتن را به رسولان مسیح مشابه می‌سازند.
\par 14 و عجب نیست، چونکه خودشیطان هم خویشتن را به فرشته نور مشابه می‌سازد.
\par 15 پس امر بزرگ نیست که خدام وی خویشتن را به خدام عدالت مشابه سازند که عاقبت ایشان برحسب اعمالشان خواهد بود.
\par 16 باز می‌گویم، کسی مرا بی‌فهم نداند والا مراچون بی‌فهمی بپذیرید تا من نیز اندکی افتخارکنم.
\par 17 آنچه می‌گویم از جانب خداوندنمی گویم، بلکه از راه بی‌فهمی در این اعتمادی که فخر ما است.
\par 18 چونکه بسیاری از طریق جسمانی فخر می‌کنند، من هم فخر می‌نمایم.
\par 19 زیرا چونکه خود فهیم هستید، بی‌فهمان را به خوشی متحمل می‌باشید.
\par 20 زیرا متحمل می‌شوید هرگاه کسی شما را غلام سازد، یا کسی شما را فرو خورد، یا کسی شما را گرفتار کند، یاکسی خود را بلند سازد، یا کسی شما را بر رخسارطپانچه زند.
\par 21 از روی استحقار می‌گویم که گویاما ضعیف بوده‌ایم.
\par 22 آیاعبرانی هستند؟ من نیز هستم! اسرائیلی هستند؟ من نیز هستم! از ذریت ابراهیم هستند؟ من نیزمی باشم!
\par 23 آیا خدام مسیح هستند؟ چون دیوانه حرف می‌زنم، من بیشتر هستم! در محنتهاافزونتر، در تازیانه‌ها زیادتر، در زندانها بیشتر، در مرگها مکرر.
\par 24 از یهودیان پنج مرتبه از چهل یک کم تازیانه خوردم.
\par 25 سه مرتبه مرا چوب زدند؛ یک دفعه سنگسار شدم؛ سه کرت شکسته کشتی شدم؛ شبانه‌روزی در دریا بسر بردم؛
\par 26 درسفرها بارها؛ در خطرهای نهرها؛ در خطرهای دزدان؛ در خطرها از قوم خود و در خطرها ازامت‌ها؛ در خطرها در شهر؛ در خطرها دربیابان؛ در خطرها در دریا؛ در خطرها در میان برادران کذبه؛
\par 27 در محنت و مشقت، در بی‌خوابیها بارها؛ در گرسنگی و تشنگی، در روزه‌ها بارها؛ در سرماو عریانی.
\par 28 بدون آنچه علاوه بر اینها است، آن باری که هر روزه بر من است، یعنی اندیشه برای همه کلیساها.
\par 29 کیست ضعیف که من ضعیف نمی شوم؟ که لغزش می‌خورد که من نمی سوزم؟
\par 30 اگر فخر می‌باید کرد از آنچه به ضعف من تعلق دارد، فخر می‌کنم.
\par 31 خدا و پدر عیسی مسیح خداوند که تا به ابد متبارک است، می‌داند که دروغ نمی گویم.
\par 32 در دمشق، والی حارث پادشاه، شهر دمشقیان را برای گرفتن من محافظت می‌نمود.و مرا از دریچه‌ای در زنبیلی از باره قلعه پایین کردند و از دستهای وی رستم.
\par 33 و مرا از دریچه‌ای در زنبیلی از باره قلعه پایین کردند و از دستهای وی رستم.

\chapter{12}

\par 1 لابد است که فخر کنم، هرچند شایسته من نیست. لیکن به رویاها و مکاشفات خداوند می‌آیم.
\par 2 شخصی را در مسیح می‌شناسم، چهارده سال قبل از این. آیا در جسم؟ نمی دانم! و آیا بیرون از جسم؟ نمی دانم! خدامی داند. چنین شخصی که تا آسمان سوم ربوده شد.
\par 3 و چنین شخص را می‌شناسم، خواه درجسم و خواه جدا از جسم، نمی دانم، خدامی داند،
\par 4 که به فردوس ربوده شد و سخنان ناگفتنی شنید که انسان را جایز نیست به آنها تکلم کند.
\par 5 از چنین شخص فخر خواهم کرد، لیکن ازخود جز از ضعفهای خویش فخر نمی کنم.
\par 6 زیرااگر بخواهم فخر بکنم، بی‌فهم نمی باشم چونکه راست می‌گویم. لیکن اجتناب می‌کنم مبادا کسی در حق من گمانی برد فوق از آنچه در من بیند یا ازمن شنود.
\par 7 و تا آنکه از زیادتی مکاشفات زیاده سرافرازی ننمایم، خاری در جسم من داده شد، فرشته شیطان، تا مرا لطمه زند، مبادا زیاده سرافرازی نمایم.
\par 8 و درباره آن از خداوند سه دفعه استدعا نمودم تا از من برود.
\par 9 مرا گفت: «فیض من تو را کافی است، زیرا که قوت من درضعف کامل می‌گردد.» پس به شادی بسیار ازضعفهای خود بیشتر فخر خواهم نمود تا قوت مسیح در من ساکن شود.
\par 10 بنابراین، از ضعفها ورسوایی‌ها و احتیاجات و زحمات و تنگیهابخاطر مسیح شادمانم، زیرا که چون ناتوانم، آنگاه توانا هستم.
\par 11 بی‌فهم شده‌ام. شما مرا مجبور ساختید. زیرا می‌بایست شما مرا مدح کرده باشید، از آنروکه من از بزرگترین رسولان به هیچ وجه کمترنیستم، هرچند هیچ هستم.
\par 12 بدرستی که علامات رسول در میان شما با کمال صبر از آیات و معجزات و قوات پدید گشت.
\par 13 زیرا کدام چیزاست که در آن از سایر کلیساها قاصر بودید؟ مگراینکه من بر شما بار ننهادم. این بی‌انصافی را از من ببخشید!
\par 14 اینک مرتبه سوم مهیا هستم که نزد شمابیایم و بر شما بار نخواهم نهاد از آنرو که نه مال شما بلکه خود شما را طالبم، زیرا که نمی بایدفرزندان برای والدین ذخیره کنند، بلکه والدین برای فرزندان.
\par 15 اما من به‌کمال خوشی برای جانهای شما صرف می‌کنم و صرف کرده خواهم شد. و اگر شما را بیشتر محبت نمایم، آیا کمترمحبت بینم؟
\par 16 اما باشد، من بر شما بار ننهادم بلکه چون حیله گر بودم، شما را به مکر به چنگ آوردم.
\par 17 آیا به یکی از آنانی که نزد شمافرستادم، نفع از شما بردم؟
\par 18 به تیطس التماس نمودم و با وی برادر را فرستادم. آیا تیطس از شمانفع برد؟ مگر به یک روح و یک روش رفتارننمودیم؟
\par 19 آیا بعد از این مدت، گمان می‌کنید که نزدشما حجت می‌آوریم؟ به حضور خدا در مسیح سخن می‌گوییم. لیکن همه‌چیز‌ای عزیزان برای بنای شما است.
\par 20 زیرا می‌ترسم که چون آیم شما را نه‌چنانکه می‌خواهم بیابم و شما مرا بیابیدچنانکه نمی خواهید که مبادا نزاع و حسد وخشمها و تعصب و بهتان و نمامی و غرور وفتنه‌ها باشد.و چون بازآیم، خدای من مرا نزدشما فروتن سازد و ماتم کنم برای بسیاری از آنانی که پیشتر گناه کردند و از ناپاکی و زنا و فجوری که کرده بودند، توبه ننمودند.
\par 21 و چون بازآیم، خدای من مرا نزدشما فروتن سازد و ماتم کنم برای بسیاری از آنانی که پیشتر گناه کردند و از ناپاکی و زنا و فجوری که کرده بودند، توبه ننمودند.

\chapter{13}

\par 1 این مرتبه سوم نزد شما می‌آیم. به گواهی دو سه شاهد، هر سخن ثابت خواهد شد.
\par 2 پیش گفتم و پیش می‌گویم که گویادفعه دوم حاضر بوده‌ام، هرچند الان غایب هستم، آنانی را که قبل از این گناه کردند و همه دیگران را که اگر بازآیم، مسامحه نخواهم نمود.
\par 3 چونکه دلیل مسیح را که در من سخن می‌گویدمی جویید که او نزد شما ضعیف نیست بلکه درشما تواناست.
\par 4 زیرا هرگاه از ضعف مصلوب گشت، لیکن از قوت خدا زیست می‌کند. چونکه ما نیز در وی ضعیف هستیم، لیکن با او از قوت خدا که به سوی شما است، زیست خواهیم کرد.
\par 5 خود را امتحان کنید که در ایمان هستید یا نه. خود را باز‌یافت کنید. آیا خود را نمی شناسید که عیسی مسیح در شما است اگر مردود نیستید؟
\par 6 اما امیدوارم که خواهید دانست که ما مردودنیستیم.
\par 7 و از خدا مسالت می‌کنم که شما هیچ بدی نکنید، نه تا ظاهر شود که ما مقبول هستیم، بلکه تا شما نیکویی کرده باشید، هرچند ما گویامردود باشیم.
\par 8 زیرا که هیچ نمی توانیم به خلاف راستی عمل نماییم بلکه برای راستی.
\par 9 وشادمانیم وقتی که ما ناتوانیم و شما توانایید. و نیزبرای این دعا می‌کنیم که شما کامل شوید.
\par 10 ازاینجهت این را در غیاب می‌نویسم تا هنگامی که حاضر شوم، سختی نکنم بحسب آن قدرتی که خداوند بجهت بنا نه برای خرابی به من داده است.
\par 11 خلاصه‌ای برادران شاد باشید؛ کامل شوید؛ تسلی پذیرید؛ یک رای و با سلامتی بوده باشید و خدای محبت و سلامتی با شما خواهدبود.
\par 12 یکدیگر را به بوسه مقدسانه تحیت نمایید.جمیع مقدسان به شما سلام می‌رسانند.
\par 13 جمیع مقدسان به شما سلام می‌رسانند.



\end{document}