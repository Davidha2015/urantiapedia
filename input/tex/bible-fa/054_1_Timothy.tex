\begin{document}

\title{اول تيموتيوس}


\chapter{1}

\par 1 پولس، رسول عیسی مسیح به حکم نجات‌دهنده ما خدا و مسیح عیسی خداوند که امید ما است،
\par 2 به فرزند حقیقی خود در ایمان، تیموتاوس.فیض و رحم و سلامتی از جانب خدای پدر وخداوند ما مسیح عیسی بر تو باد.
\par 3 چنانکه هنگامی که عازم مکادونیه بودم، به شما التماس نمودم که در افسس بمانی تا بعضی را امر کنی که تعلیمی دیگر ندهند،
\par 4 و افسانه‌ها ونسب نامه های نامتناهی را اصغا ننمایند که اینهامباحثات را نه آن تعمیر الهی را که در ایمان است پدید می‌آورد.
\par 5 اما غایت حکم، محبت است ازدل پاک و ضمیر صالح و ایمان بی‌ریا.
\par 6 که از این امور بعضی منحرف گشته به بیهوده‌گویی توجه نموده‌اند،
\par 7 و می‌خواهند معلمان شریعت بشوندو حال آنکه نمی فهمند آنچه می‌گویند و نه آنچه به تاکید اظهار می‌نمایند.
\par 8 لیکن می‌دانیم که شریعت نیکو است اگرکسی آن را برحسب شریعت بکار برد.
\par 9 و این بداند که شریعت بجهت عادل موضوع نمی شود، بلکه برای سرکشان و طاغیان و بی‌دینان وگناهکاران و ناپاکان و حرامکاران و قاتلان پدر و قاتلان مادر و قاتلان مردم
\par 10 و زانیان و لواطان ومردم دزدان و دروغ‌گویان و قسم دروغ خوران وبرای هر عمل دیگری که برخلاف تعلیم صحیح باشد،
\par 11 برحسب انجیل جلال خدای متبارک که به من سپرده شده است.
\par 12 و شکر می‌کنم خداوند خود مسیح عیسی را که مرا تقویت داد، چونکه امین شمرده، به این خدمتم ممتاز فرمود،
\par 13 که سابق کفرگو و مضر وسقطگو بودم، لیکن رحم یافتم از آنرو که ازجهالت در بی‌ایمانی کردم.
\par 14 اما فیض خداوندما بی‌نهایت افزود با‌ایمان و محبتی که در مسیح عیسی است.
\par 15 این سخن امین است و لایق قبول تام که مسیح عیسی به‌دنیا آمد تا گناهکاران را نجات‌بخشد که من بزرگترین آنها هستم.
\par 16 بلکه از این جهت بر من رحم شد تا اول درمن، مسیح عیسی کمال حلم را ظاهر سازد تا آنانی را که بجهت حیات جاودانی به وی ایمان خواهند‌آورد، نمونه باشم.
\par 17 باری پادشاه سرمدی و باقی ونادیده را، خدای حکیم وحید را اکرام و جلال تاابدالاباد باد. آمین.
\par 18 ‌ای فرزند تیموتاوس، این وصیت را به تو می سپارم برحسب نبوتهایی که سابق بر تو شد تادر آنها جنگ نیکو کنی،
\par 19 و ایمان و ضمیرصالح را نگاه داری که بعضی این را از خود دورانداخته، مر ایمان را شکسته کشتی شدند.که از آن جمله هیمیناوس و اسکندر می‌باشند که ایشان را به شیطان سپردم تا تادیب شده، دیگرکفر نگویند.
\par 20 که از آن جمله هیمیناوس و اسکندر می‌باشند که ایشان را به شیطان سپردم تا تادیب شده، دیگرکفر نگویند.

\chapter{2}

\par 1 پس از همه‌چیز اول، سفارش می‌کنم که صلوات و دعاها و مناجات و شکرها رابرای جمیع مردم به‌جا آورند؛
\par 2 بجهت پادشاهان و جمیع صاحبان منصب تا به آرامی و استراحت و با کمال دینداری و وقار، عمر خود را بسر بریم.
\par 3 زیرا که این نیکو و پسندیده است، در حضورنجات‌دهنده ما خدا
\par 4 که می‌خواهد جمیع مردم نجات یابند و به معرفت راستی گرایند.
\par 5 زیراخدا واحد است و در میان خدا و انسان یک متوسطی است یعنی انسانی که مسیح عیسی باشد،
\par 6 که خود را در راه همه فدا داد، شهادتی درزمان معین.
\par 7 و برای این، من واعظ و رسول ومعلم امتها در ایمان و راستی مقرر شدم. در مسیح راست می‌گویم و دروغ نی.
\par 8 پس آرزوی این دارم که مردان، دست های مقدس را بدون غیظ و جدال برافراخته، در هر جادعا کنند.
\par 9 و همچنین زنان خویشتن را بیارایند به لباس مزین به حیا و پرهیز نه به زلفها و طلا ومروارید و رخت گرانبها؛
\par 10 بلکه چنانکه زنانی رامی شاید که دعوای دینداری می‌کنند به اعمال صالحه.
\par 11 زن با سکوت، به‌کمال اطاعت تعلیم گیرد.
\par 12 و زن را اجازت نمی دهم که تعلیم دهد یابر شوهر مسلط شود بلکه در سکوت بماند.
\par 13 زیرا که آدم اول ساخته شد و بعد حوا.
\par 14 وآدم فریب نخورد بلکه زن فریب خورده، درتقصیر گرفتار شد.اما به زاییدن رستگارخواهد شد، اگر در ایمان و محبت و قدوسیت وتقوا ثابت بمانند.
\par 15 اما به زاییدن رستگارخواهد شد، اگر در ایمان و محبت و قدوسیت وتقوا ثابت بمانند.

\chapter{3}

\par 1 این سخن امین است که اگر کسی منصب اسقفی را بخواهد، کار نیکو می‌طلبد.
\par 2 پس اسقف باید بی‌ملامت و صاحب یک زن و هشیارو خردمند و صاحب نظام و مهمان‌نواز و راغب به تعلیم باشد؛
\par 3 نه میگسار یا زننده یا طماع سودقبیح بلکه حلیم و نه جنگجو و نه زرپرست.
\par 4 مدبر اهل خانه خود، به نیکویی و فرزندان خویش را در کمال وقار مطیع گرداند،
\par 5 زیراهرگاه کسی نداند که اهل خانه خود را تدبیر کند، چگونه کلیسای خدا را نگاهبانی می‌نماید؟
\par 6 و نه جدیدالایمان که مبادا غرور کرده، به حکم ابلیس بیفتد.
\par 7 اما لازم است که نزد آنانی که خارجند هم نیک نام باشد که مبادا در رسوایی و دام ابلیس گرفتار شود.
\par 8 همچنین شماسان باوقار باشند، نه دو زبان ونه راغب به شراب زیاده و نه طماع سود قبیح؛
\par 9 دارندگان سر ایمان در ضمیر پاک.
\par 10 اما بایداول ایشان آزموده شوند و چون بی‌عیب یافت شدند، کار شماسی را بکنند.
\par 11 و به همینطورزنان نیز باید باوقار باشند و نه غیبت‌گو بلکه هشیارو در هر امری امین.
\par 12 و شماسان صاحب یک زن باشند و فرزندان و اهل خانه خویش را نیکو تدبیرنمایند،
\par 13 زیرا آنانی که کار شماسی را نیکو کرده باشند، درجه خوب برای خویشتن تحصیل می‌کنند و جلادت کامل در ایمانی که به مسیح عیسی است.
\par 14 این به تو می‌نویسم به امید آنکه به زودی نزد تو آیم.
\par 15 لیکن اگر تاخیر اندازم، تا بدانی که چگونه باید در خانه خدا رفتار کنی که کلیسای خدای حی و ستون و بنیاد راستی است.وبالاجماع سر دینداری عظیم است که خدا درجسم ظاهر شد و در روح، تصدیق کرده شد و به فرشتگان، مشهود گردید و به امتها موعظه کرده ودر دنیا ایمان آورده و به جلال بالا برده شد.
\par 16 وبالاجماع سر دینداری عظیم است که خدا درجسم ظاهر شد و در روح، تصدیق کرده شد و به فرشتگان، مشهود گردید و به امتها موعظه کرده ودر دنیا ایمان آورده و به جلال بالا برده شد.

\chapter{4}

\par 1 و لیکن روح صریح می‌گوید که در زمان آخر بعضی از ایمان برگشته، به ارواح مضل و تعالیم شیاطین اصغا خواهند نمود،
\par 2 به ریاکاری دروغگویان که ضمایر خود را داغ کرده‌اند؛
\par 3 که از مزاوجت منع می‌کنند و حکم می‌نمایند به احتراز از خوراک هایی که خدا آفریدبرای مومنین و عارفین حق تا آنها را به شکرگزاری بخورند.
\par 4 زیرا که هر مخلوق خدا نیکو است وهیچ‌چیز را رد نباید کرد، اگر به شکرگزاری پذیرند،
\par 5 زیرا که از کلام خدا و دعا تقدیس می‌شود.
\par 6 اگر این امور را به برادران بسپاری، خادم نیکوی مسیح عیسی خواهی بود، تربیت یافته درکلام ایمان و تعلیم خوب که پیروی آن راکرده‌ای.
\par 7 لیکن از افسانه های حرام عجوزهااحتراز نما و در دینداری ریاضت بکش.
\par 8 که ریاضت بدنی اندک فایده‌ای دارد، لیکن دینداری برای هر چیز مفید است که وعده زندگی حال وآینده را دارد.
\par 9 این سخن امین است و لایق قبول تام،
\par 10 زیراکه برای این زحمت و بی‌احترامی می‌کشیم، زیراامید داریم به خدای زنده که جمیع مردمان علی الخصوص مومنین را نجات‌دهنده است.
\par 11 این امور را حکم و تعلیم فرما.
\par 12 هیچ‌کس جوانی تو را حقیر نشمارد، بلکه مومنین را درکلام و سیرت و محبت و ایمان و عصمت، نمونه باش.
\par 13 تا مادامی که نه آیم، خود را به قرائت ونصیحت و تعلیم بسپار.
\par 14 زنهار از آن کرامتی که در تو است که بوسیله نبوت با نهادن دستهای کشیشان به تو داده شد، بی‌اعتنایی منما.
\par 15 در این امور تامل نما و در اینها راسخ باش تا ترقی تو برهمه ظاهر شود.خویشتن را و تعلیم را احتیاطکن و در این امور قائم باش که هرگاه چنین کنی، خویشتن را و شنوندگان خویش را نیز نجات خواهی داد.
\par 16 خویشتن را و تعلیم را احتیاطکن و در این امور قائم باش که هرگاه چنین کنی، خویشتن را و شنوندگان خویش را نیز نجات خواهی داد.

\chapter{5}

\par 1 مرد پیر را توبیخ منما بلکه چون پدر او رانصیحت کن، و جوانان را چون برادران؛
\par 2 زنان پیر را چون مادران؛ و زنان جوان را مثل خواهران با کمال عفت؛
\par 3 بیوه‌زنان را اگرفی الحقیقت بیوه باشند، محترم دار.
\par 4 اما اگربیوه‌زنی فرزندان یا نواده‌ها دارد، آموخته بشوندکه خانه خود را با دینداری نگاه دارند و حقوق اجداد خود را ادا کنند که این در حضور خدا نیکوو پسندیده است.
\par 5 اما زنی که فی الحقیقت بیوه وبی کس است، به خدا امیدوار است و در صلوات ودعاها شبانه‌روز مشغول می‌باشد.
\par 6 لیکن زن عیاش در حال حیات مرده است.
\par 7 و به این معانی امر فرما تا بی‌ملامت باشند.
\par 8 ولی اگر کسی برای خویشان و علی الخصوص اهل خانه خود تدبیرنکند، منکر ایمان و پست‌تر از بی‌ایمان است.
\par 9 بیوه‌زنی که کمتر از شصت ساله نباشد و یک شوهر کرده باشد، باید نام او ثبت گردد،
\par 10 که دراعمال صالح نیک نام باشد، اگر فرزندان را پرورده و غربا را مهمانی نموده و پایهای مقدسین راشسته و زحمت کشان را اعانتی نموده و هر کارنیکو را پیروی کرده باشد.
\par 11 اما بیوه های جوانتراز این را قبول مکن، زیرا که چون از مسیح سرکش شوند، خواهش نکاح دارند
\par 12 و ملزم می‌شوند ازاینکه ایمان نخست را برطرف کرده‌اند؛
\par 13 وعلاوه بر این خانه به خانه گردش کرده، آموخته می‌شوند که بی‌کار باشند؛ و نه فقط بی‌کار بلکه بیهوده گو و فضول هم که حرفهای ناشایسته می‌زنند.
\par 14 پس رای من این است که زنان جوان نکاح شوند و اولاد بزایند و کدبانو شوند و خصم را مجال مذمت ندهند؛
\par 15 زیرا که بعضی برگشتندبه عقب شیطان.
\par 16 اگر مرد یا زن مومن، بیوه هادارد ایشان را بپرورد و بار بر کلیسا ننهد تا آنانی راکه فی الحقیقت بیوه باشند، پرورش نماید.
\par 17 کشیشانی که نیکو پیشوایی کرده‌اند، مستحق حرمت مضاعف می‌باشند، علی الخصوص آنانی که در کلام و تعلیم محنت می‌کشند.
\par 18 زیرا کتاب می‌گوید: «گاو را وقتی که خرمن را خرد می‌کند، دهن مبند» و «مزدورمستحق اجرت خود است».
\par 19 ادعایی بر یکی ازکشیشان جز به زبان دو یا سه شاهد مپذیر.
\par 20 آنانی که گناه کنند، پیش همه توبیخ فرما تا دیگران بترسند.
\par 21 در حضور خدا و مسیح عیسی و فرشتگان برگزیده تو را قسم می‌دهم که این امور را بدون غرض نگاه داری و هیچ کاری از روی طرفداری مکن.
\par 22 و دستها به زودی بر هیچ‌کس مگذار و درگناهان دیگران شریک مشو بلکه خود را طاهرنگاه دار.
\par 23 دیگر آشامنده آب فقط مباش، بلکه بجهت شکمت و ضعفهای بسیار خود شرابی کم میل فرما.
\par 24 گناهان بعضی آشکار است و پیش روی ایشان به داوری می‌خرامد، اما بعضی را تعاقب می‌کند.و همچنین اعمال نیکو واضح است وآنهایی که دیگرگون باشد، نتوان مخفی داشت.
\par 25 و همچنین اعمال نیکو واضح است وآنهایی که دیگرگون باشد، نتوان مخفی داشت.

\chapter{6}

\par 1 آنانی که غلامان زیر یوغ می‌باشند، آقایان خویش را لایق کمال احترام بدانند که مبادانام و تعلیم خدا بد گفته شود.
\par 2 اما کسانی که آقایان مومن دارند، ایشان را تحقیر ننمایند، ازآنجا که برادرانند بلکه بیشتر خدمت کنند از آنروکه آنانی که در این احسان مشارکند، مومن ومحبوبند.
\par 3 و اگرکسی بطور دیگر تعلیم دهد و کلام صحیح خداوند ما عیسی مسیح و آن تعلیمی را که به طریق دینداری است قبول ننماید،
\par 4 از غرورمست شده، هیچ نمی داند بلکه در مباحثات ومجادلات دیوانه گشته است که از آنها پدیدمی آید حسد و نزاع و کفر و ظنون شر
\par 5 و منازعات مردم فاسدالعقل و مرتد از حق که می‌پندارند دینداری سود است. از چنین اشخاص اعراض نما.
\par 6 لیکن دینداری با قناعت سود عظیمی است.
\par 7 زیرا که در این دنیا هیچ نیاوردیم و واضح است که از آن هیچ نمی توانیم برد.
\par 8 پس اگر خوراک وپوشاک داریم، به آنها قانع خواهیم بود.
\par 9 اماآنانی که می‌خواهند دولتمند شوند، گرفتارمی شوند در تجربه و دام و انواع شهوات بی‌فهم ومضر که مردم را به تباهی و هلاکت غرق می‌سازند.
\par 10 زیرا که طمع ریشه همه بدیهااست که بعضی چون در‌پی آن می‌کوشیدند، ازایمان گمراه گشته، خود را به اقسام دردهاسفتند.
\par 11 ولی تو‌ای مرد خدا، از اینها بگریز وعدالت و دینداری و ایمان و محبت و صبر وتواضع را پیروی نما.
\par 12 و جنگ نیکوی ایمان رابکن و بدست آور آن حیات جاودانی را که برای آن دعوت شدی و اعتراف نیکو کردی در حضورگواهان بسیار.
\par 13 تو را وصیت می‌کنم به حضور آن خدایی که همه را زندگی می‌بخشد و مسیح عیسی که در‌پیش پنطیوس پیلاطس اعتراف نیکونمود،
\par 14 که تو وصیت را بی‌داغ و ملامت حفظکن تا به ظهور خداوند ما عیسی مسیح.
\par 15 که آن را آن متبارک و قادر وحید و ملک الملوک ورب‌الارباب در زمان معین به ظهور خواهد آورد.
\par 16 که تنها لایموت و ساکن در نوری است که نزدیک آن نتوان شد و احدی از انسان او را ندیده و نمی تواند دید. او را تا ابدالاباد اکرام و قدرت باد. آمین.
\par 17 دولتمندان این جهان را امر فرما که بلندپروازی نکنند و به دولت ناپایدار امید ندارند، بلکه به خدای زنده که همه‌چیز را دولتمندانه برای تمتع به ما عطا می‌کند؛
\par 18 که نیکوکار بوده، در اعمال صالحه دولتمند و سخی و گشاده‌دست باشند؛
\par 19 و برای خود اساس نیکو بجهت عالم آینده نهند تا حیات جاودانی را بدست آرند.‌ای تیموتاوس تو آن امانت را محفوظ دار واز بیهوده‌گویی های حرام و از مباحثات معرفت دروغ اعراض نما،
\par 20 ‌ای تیموتاوس تو آن امانت را محفوظ دار واز بیهوده‌گویی های حرام و از مباحثات معرفت دروغ اعراض نما،



\end{document}