\begin{document}

\title{دوم تيموتيوس}


\chapter{1}

\par 1 پولس به اراده خدا رسول مسیح عیسی، برحسب وعده حیاتی که در مسیح عیسی است،
\par 2 فرزند حبیب خود تیموتاوس را.فیض و رحمت و سلامتی از جانب خدای پدرو خداوند ما عیسی مسیح باد.
\par 3 شکر می‌کنم آن خدایی را که از اجداد خودبه ضمیر خالص بندگی او را می‌کنم، چونکه دائم در دعاهای خود تو را شبانه‌روز یاد می‌کنم،
\par 4 و مشتاق ملاقات تو هستم چونکه اشکهای تورا بخاطر می‌دارم تا از خوشی سیر شوم.
\par 5 زیرا که یاد می‌دارم ایمان بی‌ریای تو را که نخست درجده ات لوئیس و مادرت افنیکی ساکن می‌بود ومرا یقین است که در تو نیز هست.
\par 6 لهذا بیاد تومی آورم که آن عطای خدا را که بوسیله گذاشتن دستهای من بر تو است برافروزی.
\par 7 زیرا خداروح جبن را به ما نداده است بلکه روح قوت ومحبت و تادیب را.
\par 8 پس از شهادت خداوند ما عار مدار و نه ازمن که اسیر او می‌باشم، بلکه در زحمات انجیل شریک باش برحسب قوت خدا،
\par 9 که ما را نجات داد و به دعوت مقدس خواند نه به حسب اعمال ما بلکه برحسب اراده خود و آن فیضی که قبل ازقدیم الایام در مسیح عیسی به ما عطا شد.
\par 10 اماالحال آشکار گردید به ظهور نجات‌دهنده ماعیسی مسیح که موت را نیست ساخت و حیات وبی فسادی را روشن گردانید بوسیله انجیل،
\par 11 که برای آن من واعظ و رسول و معلم امت‌ها مقررشده‌ام.
\par 12 و از این جهت این زحمات را می‌کشم بلکه عار ندارم چون می‌دانم به که ایمان آوردم ومرا یقین است که او قادر است که امانت مرا تا به آن روز حفظ کند.
\par 13 نمونه‌ای بگیر از سخنان صحیح که از من شنیدی در ایمان و محبتی که در مسیح عیسی است.
\par 14 آن امانت نیکو را بوسیله روح‌القدس که در ما ساکن است، حفظ کن.
\par 15 از این آگاه هستی که همه آنانی که در آسیا هستند، از من رخ تافته‌اند که از آنجمله فیجلس و هرموجنس می‌باشند.
\par 16 خداوند اهل خانه انیسیفورس را ترحم کناد زیرا که او بارها دل مرا تازه کرد و از زنجیر من عار نداشت،
\par 17 بلکه چون به روم رسید، مرا به کوشش بسیار تفحص کرده، پیدا نمود.(خداوند بدو عطا کناد که در آن روز در حضورخداوند رحمت یابد. ) و خدمتهایی را که درافسس کرد تو بهتر می‌دانی.
\par 18 (خداوند بدو عطا کناد که در آن روز در حضورخداوند رحمت یابد. ) و خدمتهایی را که درافسس کرد تو بهتر می‌دانی.

\chapter{2}

\par 1 پس تو‌ای فرزند من، در فیضی که در مسیح عیسی است زورآور باش.
\par 2 و آنچه به شهود بسیار از من شنیدی، به مردمان امین بسپارکه قابل تعلیم دیگران هم باشند.
\par 3 چون سپاهی نیکوی مسیح عیسی در تحمل زحمات شریک باش.
\par 4 هیچ سپاهی خود را در امور روزگار گرفتارنمی سازد تا رضایت آنکه او را سپاهی ساخت بجوید.
\par 5 و اگر کسی نیز پهلوانی کند، تاج را بدونمی دهند اگر به قانون پهلوانی نکرده باشد.
\par 6 برزگری که محنت می‌کشد، باید اول نصیبی ازحاصل ببرد.
\par 7 در آنچه می‌گویم تفکر کن زیراخداوند تو را در همه‌چیز فهم خواهد بخشید.
\par 8 عیسی مسیح را بخاطر دار که از نسل داودبوده، از مردگان برخاست برحسب بشارت من،
\par 9 که در آن چون بدکار تا به بندها زحمت می‌کشم، لیکن کلام خدا بسته نمی شود.
\par 10 و ازاین جهت همه زحمات را بخاطر برگزیدگان متحمل می‌شوم تا ایشان نیز نجاتی را که درمسیح عیسی است با جلال جاودانی تحصیل کنند.
\par 11 این سخن امین است زیرا اگر با وی مردیم، با او زیست هم خواهیم کرد.
\par 12 و اگرتحمل کنیم، با او سلطنت هم خواهیم کرد؛ وهرگاه او را انکار کنیم او نیز ما را انکار خواهدکرد.
\par 13 اگر بی‌ایمان شویم، او امین می‌ماند زیراخود را انکار نمی تواند نمود.
\par 14 این چیزها را به یاد ایشان آور و در حضورخداوند قدغن فرما که مجادله نکنند، زیرا هیچ سود نمی بخشد بلکه باعث هلاکت شنوندگان می‌باشد.
\par 15 و سعی کن که خود را مقبول خداسازی، عاملی که خجل نشود و کلام خدا رابخوبی انجام دهد.
\par 16 و از یاوه‌گویی های حرام اعراض نما زیرا که تا به فزونی بی‌دینی ترقی خواهد کرد.
\par 17 و کلام ایشان، چون آکله می‌خورد و از آنجمله هیمیناوس و فلیطس می‌باشند
\par 18 که ایشان از حق برگشته، می‌گویندکه قیامت الان شده است و بعضی را از ایمان منحرف می‌سازند.
\par 19 و لیکن بنیاد ثابت خدا قائم است و این مهر را دارد که «خداوند کسان خود رامی شناسد» و «هرکه نام مسیح را خواند، ازناراستی کناره جوید.»
\par 20 اما در خانه بزرگ نه فقطظروف طلا و نقره می‌باشد، بلکه چوبی و گلی نیز؛ اما آنها برای عزت و اینها برای ذلت.
\par 21 پس اگرکسی خویشتن را از اینها طاهر سازد، ظرف عزت خواهد بود مقدس و نافع برای مالک خود ومستعد برای هر عمل نیکو.
\par 22 اما از شهوات جوانی بگریز و با آنانی که ازقلب خالص نام خداوند را می‌خوانند، عدالت وایمان و محبت و سلامتی را تعاقب نما.
\par 23 لیکن ازمسائل بیهوده و بی‌تادیب اعراض نما چون می‌دانی که نزاعها پدید می‌آورد.
\par 24 اما بنده خدانباید نزاع کند، بلکه با همه کس ملایم و راغب به تعلیم و صابر در مشقت باشد،
\par 25 و با حلم مخالفین را تادیب نماید که شاید خدا ایشان راتوبه بخشد تا راستی را بشناسند.تا از دام ابلیس باز به هوش آیند که به حسب اراده او صیداو شده‌اند.
\par 26 تا از دام ابلیس باز به هوش آیند که به حسب اراده او صیداو شده‌اند.

\chapter{3}

\par 1 اما این را بدان که در ایام آخر زمانهای سخت پدید خواهد آمد،
\par 2 زیرا که مردمان، خودپرست خواهند بود و طماع ولاف‌زن و متکبر و بدگو و نامطیع والدین و ناسپاس و ناپاک
\par 3 و بی‌الفت و کینه دل و غیبت‌گو و ناپرهیزو بی‌مروت و متنفر از نیکویی
\par 4 و خیانت کار وتندمزاج و مغرور که عشرت را بیشتر از خدادوست می‌دارند؛
\par 5 که صورت دینداری دارند، لیکن قوت آن را انکار می‌کنند. از ایشان اعراض نما.
\par 6 زیرا که از اینها هستند آنانی که به حیله داخل خانه‌ها گشته، زنان کم عقل را اسیر می‌کنندکه بار گناهان را می‌کشند و به انواع شهوات ربوده می‌شوند.
\par 7 و دائم تعلیم می‌گیرند، لکن هرگز به معرفت راستی نمی توانند رسید.
\par 8 و همچنان‌که ینیس و یمبریس با موسی مقاومت کردند، ایشان نیز با راستی مقاومت می‌کنند که مردم فاسدالعقل و مردود از ایمانند.
\par 9 لیکن بیشتر ترقی نخواهندکرد زیرا که حماقت ایشان بر جمیع مردم واضح خواهد شد، چنانکه حماقت آنها نیز شد.
\par 10 لیکن تو تعلیم و سیرت و قصد و ایمان وحلم و محبت و صبر مرا پیروی نمودی،
\par 11 وزحمات و آلام مرا مثل آنهایی که در انطاکیه وایقونیه و لستره بر من واقع شد، چگونه زحمات راتحمل می‌نمودم و خداوند مرا از همه رهایی داد.
\par 12 و همه کسانی که می‌خواهند در مسیح عیسی به دینداری زیست کنند، زحمت خواهند کشید.
\par 13 لیکن مردمان شریر و دغاباز در بدی ترقی خواهند کرد که فریبنده و فریب خورده می‌باشند.
\par 14 اما تو در آنچه آموختی و ایمان آوردی قایم باش چونکه می‌دانی از چه کسان تعلیم یافتی،
\par 15 و اینکه از طفولیت کتب مقدسه را دانسته‌ای که می‌تواند تو را حکمت آموزدبرای نجات بوسیله ایمانی که بر مسیح عیسی است.
\par 16 تمامی کتب از الهام خداست و بجهت تعلیم و تنبیه و اصلاح و تربیت در عدالت مفیداست،تا مرد خدا کامل و بجهت هر عمل نیکوآراسته بشود.
\par 17 تا مرد خدا کامل و بجهت هر عمل نیکوآراسته بشود.

\chapter{4}

\par 1 تو را در حضور خدا و مسیح عیسی که برزندگان و مردگان داوری خواهد کرد قسم می‌دهم و به ظهور وملکوت او
\par 2 که به کلام موعظه کنی و در فرصت و غیر فرصت مواظب باشی و تنبیه و توبیخ و نصیحت نمایی باکمال تحمل و تعلیم.
\par 3 زیرا ایامی می‌آید که تعلیم صحیح را متحمل نخواهند شد، بلکه برحسب شهوات خود خارش گوشها داشته، معلمان را برخود فراهم خواهند‌آورد،
\par 4 و گوشهای خود رااز راستی برگردانیده، به سوی افسانه‌ها خواهندگرایید.
\par 5 لیکن تو در همه‌چیز هشیار بوده، متحمل زحمات باش و عمل مبشر را بجا آور وخدمت خود را به‌کمال رسان.
\par 6 زیرا که من الان ریخته می‌شوم و وقت رحلت من رسیده است.
\par 7 به جنگ نیکو جنگ کرده‌ام و دوره خود را به‌کمال رسانیده، ایمان رامحفوظ داشته‌ام.
\par 8 بعد از این تاج عدالت برای من حاضر شده است که خداوند داور عادل در آن روز به من خواهد داد؛ و نه به من فقط بلکه نیز به همه کسانی که ظهور او را دوست می‌دارند.
\par 9 سعی کن که به زودی نزد من آیی،
\par 10 زیرا که دیماس برای محبت این جهان حاضر مرا ترک کرده، به تسالونیکی رفته است و کریسکیس به غلاطیه و تیطس به دلماطیه.
\par 11 لوقا تنها با من است. مرقس را برداشته، با خود بیاور زیرا که مرابجهت خدمت مفید است.
\par 12 اما تیخیکس را به افسس فرستادم.
\par 13 ردایی را که در تروآس نزدکرپس گذاشتم، وقت آمدنت بیاور و کتب را نیز وخصوص رقوق را.
\par 14 اسکندر مسکر با من بسیاربدیها کرد. خداوند او را بحسب افعالش جزاخواهد داد.
\par 15 و تو هم از او باحذر باش زیرا که باسخنان ما بشدت مقاومت نمود.
\par 16 در محاجه اول من، هیچ‌کس با من حاضر نشد بلکه همه مراترک کردند. مباد که این بر ایشان محسوب شود.
\par 17 لیکن خداوند با من ایستاده، به من قوت داد تاموعظه بوسیله من به‌کمال رسد و تمامی امت هابشنوند و از دهان شیر رستم.
\par 18 و خداوند مرا ازهر کار بد خواهد رهانید و تا به ملکوت آسمانی خود نجات خواهد داد. او را تا ابدالاباد جلال باد. آمین.
\par 19 فرسکا و اکیلا و اهل خانه انیسیفورس راسلام رسان.
\par 20 ارستس در قرنتس ماند؛ اماترفیمس را در میلیتس بیمار واگذاردم.سعی کن که قبل از زمستان بیایی. افبولس و پودیس ولینس و کلادیه و همه برادران تو را سلام می‌رسانند.
\par 21 سعی کن که قبل از زمستان بیایی. افبولس و پودیس ولینس و کلادیه و همه برادران تو را سلام می‌رسانند.



\end{document}