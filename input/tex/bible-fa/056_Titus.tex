\begin{document}

\title{Titus}


\chapter{1}

\par 1 پولس، غلام خدا و رسول عیسی مسیح برحسب ایمان برگزیدگان خدا و معرفت آن راستی که در دینداری است،
\par 2 به امید حیات جاودانی که خدایی که دروغ نمی تواند گفت، اززمانهای ازلی وعده آن را داد،
\par 3 اما در زمان معین، کلام خود را ظاهر کرد به موعظه‌ای که برحسب حکم نجات‌دهنده ما خدا به من سپرده شد،
\par 4 تیطس را که فرزند حقیقی من برحسب ایمان عام است، فیض و رحمت و سلامتی از جانب خدای پدرو نجات‌دهنده ما عیسی مسیح خداوند باد.
\par 5 بدین جهت تو را در کریت واگذاشتم تا آنچه را که باقی‌مانده است اصلاح نمایی و چنانکه من به تو امر نمودم، کشیشان در هر شهر مقرر کنی.
\par 6 اگر کسی بی‌ملامت و شوهر یک زن باشد که فرزندان مومن دارد، بری از تهمت فجور و تمرد،
\par 7 زیرا که اسقف می‌باید چون وکیل خدابی ملامت باشد و خودرای یا تندمزاج یا میگساریا زننده یا طماع سود قبیح نباشد،
\par 8 بلکه مهمان دوست و خیردوست و خرداندیش و عادل ومقدس و پرهیزکار؛
\par 9 و متمسک به کلام امین برحسب تعلیمی که یافته تا بتواند به تعلیم صحیح نصیحت کند و مخالفان را توبیخ نماید.
\par 10 زیرا که یاوه‌گویان و فریبندگان، بسیار ومتمرد می‌باشند، علی الخصوص آنانی که از اهل ختنه هستند؛
\par 11 که دهان ایشان را باید بست زیراخانه‌ها را بالکل واژگون می‌سازند و برای سودقبیح، تعالیم ناشایسته می‌دهند.
\par 12 یکی از ایشان که نبی خاص ایشان است، گفته است که «اهل کریت همیشه دروغگو و وحوش شریر وشکم پرست بی‌کاره می‌باشند.»
\par 13 این شهادت راست است؛ از این جهت ایشان را به سختی توبیخ فرما تا در ایمان، صحیح باشند،
\par 14 و گوش نگیرند به افسانه های یهود و احکام مردمانی که ازراستی انحراف می‌جویند.
\par 15 هرچیز برای پاکان پاک است، لیکن آلودگان و بی‌ایمانان را هیچ‌چیزپاک نیست، بلکه فهم و ضمیر ایشان نیز ملوث است؛مدعی معرفت خدا می‌باشند، اما به افعال خود او را انکار می‌کنند، چونکه مکروه ومتمرد هستند و بجهت هر عمل نیکو مردود.
\par 16 مدعی معرفت خدا می‌باشند، اما به افعال خود او را انکار می‌کنند، چونکه مکروه ومتمرد هستند و بجهت هر عمل نیکو مردود.

\chapter{2}

\par 1 اما تو سخنان شایسته تعلیم صحیح را بگو:
\par 2 که مردان پیر، هشیار و باوقار وخرداندیش و در ایمان و محبت و صبر، صحیح باشند.
\par 3 همچنین زنان پیر، در سیرت متقی باشند و نه غیبت‌گو و نه بنده شراب زیاده بلکه معلمات تعلیم نیکو،
\par 4 تا زنان جوان را خرد بیاموزند که شوهردوست و فرزنددوست باشند،
\par 5 وخرداندیش و عفیفه و خانه نشین و نیکو و مطیع شوهران خود که مبادا کلام خدا متهم شود.
\par 6 و به همین نسق جوانان را نصیحت فرما تا خرداندیش باشند.
\par 7 و خود را در همه‌چیز نمونه اعمال نیکو بسازو در تعلیم خود صفا و وقار و اخلاص را بکار بر،
\par 8 و کلام صحیح بی‌عیب را تا دشمن چونکه فرصت بد‌گفتن در حق ما نیابد، خجل شود.
\par 9 غلامان را نصیحت نما که آقایان خود رااطاعت کنند و در هر امر ایشان را راضی سازند ونقیض گو نباشند؛
\par 10 و دزدی نکنند بلکه کمال دیانت را ظاهر سازند تا تعلیم نجات‌دهنده ما خدارا در هر چیز زینت دهند.
\par 11 زیرا که فیض خدا که برای همه مردم نجات‌بخش است، ظاهر شده،
\par 12 ما را تادیب می‌کند که بی‌دینی و شهوات دنیوی را ترک کرده، با خرداندیشی و عدالت و دینداری در این جهان زیست کنیم.
\par 13 و آن امید مبارک و تجلی جلال خدای عظیم و نجات‌دهنده خود ما عیسی مسیح را انتظار کشیم،
\par 14 که خود را در راه ما فدا ساخت تا ما را از هر ناراستی برهاند و امتی برای خودطاهر سازد که ملک خاص او و غیور در اعمال نیکو باشند.این را بگو و نصیحت فرما و درکمال اقتدار توبیخ نما و هیچ‌کس تو را حقیرنشمارد.
\par 15 این را بگو و نصیحت فرما و درکمال اقتدار توبیخ نما و هیچ‌کس تو را حقیرنشمارد.

\chapter{3}

\par 1 بیاد ایشان آور که حکام و سلاطین رااطاعت کنند و فرمانبرداری نمایند و برای هرکار نیکو مستعد باشند،
\par 2 و هیچ‌کس را بدنگویند و جنگجو نباشند بلکه ملایم و کمال حلم را با جمیع مردم به‌جا آورند.
\par 3 زیرا که ما نیز سابق بی‌فهم و نافرمانبردار وگمراه و بنده انواع شهوات و لذات بوده، در خبث و حسد بسر می‌بردیم که لایق نفرت بودیم و بریکدیگر بغض می‌داشتیم.
\par 4 لیکن چون مهربانی ولطف نجات‌دهنده ما خدا ظاهر شد،
\par 5 نه به‌سبب اعمالی که ما به عدالت کرده بودیم، بلکه محض رحمت خود ما را نجات داد به غسل تولد تازه وتازگی‌ای که از روح‌القدس است؛
\par 6 که او را به ما به دولتمندی افاضه نمود، به توسط نجات‌دهنده ماعیسی مسیح،
\par 7 تا به فیض او عادل شمرده شده، وارث گردیم بحسب امید حیات جاودانی.
\par 8 این سخن امین است و در این امور می‌خواهم تو قدغن بلیغ فرمایی تا آنانی که به خدا ایمان آورند، بکوشند که در اعمال نیکو مواظبت نمایند، زیرا که این امور برای انسان نیکو و مفیداست.
\par 9 و از مباحثات نامعقول و نسب نامه‌ها و نزاعهاو جنگهای شرعی اعراض نما زیرا که بی‌ثمر وباطل است.
\par 10 و از کسی‌که از اهل بدعت باشد، بعد از یک دو نصیحت اجتناب نما،
\par 11 چون می‌دانی که چنین کس مرتد و از خود ملزم شده درگناه رفتار می‌کند.
\par 12 وقتی که ارتیماس یا تیخیکس را نزد توفرستم، سعی کن که در نیکوپولیس نزد من آیی زیرا که عزیمت دارم زمستان را در آنجا بسر برم.
\par 13 زیناس خطیب و اپلس را در سفر ایشان به سعی امداد کن تا محتاج هیچ‌چیز نباشند.و کسان ما نیز تعلیم بگیرند که در کارهای نیکومشغول باشند برای رفع احتیاجات ضروری، تابی ثمر نباشند.
\par 14 و کسان ما نیز تعلیم بگیرند که در کارهای نیکومشغول باشند برای رفع احتیاجات ضروری، تابی ثمر نباشند.


\end{document}