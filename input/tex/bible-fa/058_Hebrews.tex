\begin{document}

\title{Hebrews}


\chapter{1}

\par 1 خدا که در زمان سلف به اقسام متعدد و طریق های مختلف بوساطت انبیا به پدران ما تکلم نمود،
\par 2 در این ایام آخر به ما بوساطت پسر خود متکلم شد که او را وارث جمیع موجودات قرار داد و بوسیله او عالمها راآفرید؛
\par 3 که فروغ جلالش و خاتم جوهرش بوده و به کلمه قوت خود حامل همه موجودات بوده، چون طهارت گناهان را به اتمام رسانید، به‌دست راست کبریا در اعلی علیین بنشست،
\par 4 و از فرشتگان افضال گردید، بمقدارآنکه اسمی بزرگتر از ایشان به میراث یافته بود.
\par 5 زیرا به کدام‌یک از فرشتگان هرگز گفت که «تو پسر من هستی. من امروز تو را تولید نمودم» وایض «من او را پدر خواهم بود و او پسر من خواهدبود»؟
\par 6 و هنگامی که نخست زاده را باز به جهان می‌آورد می‌گوید که «جمیع فرشتگان خدا او راپرستش کنند.»
\par 7 و در حق فرشتگان می‌گوید که «فرشتگان خود را بادها می‌گرداند و خادمان خودرا شعله آتش.»
\par 8 اما در حق پسر: «ای خدا تخت تو تا ابدالاباد است و عصای ملکوت تو عصای راستی است.
\par 9 عدالت را دوست و شرارت رادشمن می‌داری. بنابراین خدا، خدای تو، تو را به روغن شادمانی بیشتر از رفقایت مسح کرده است.»
\par 10 و (نیز می‌گوید: ) «تو‌ای خداوند، درابتدا زمین را بنا کردی و افلاک مصنوع دستهای تواست.
\par 11 آنها فانی، لکن تو باقی هستی و جمیع آنها چون جامه، مندرس خواهد شد،
\par 12 و مثل ردا آنها را خواهی پیچید و تغییر خواهند یافت. لکن تو همان هستی و سالهای تو تمام نخواهدشد.»
\par 13 و به کدام‌یک از فرشتگان هرگز گفت: «بنشین به‌دست راست من تا دشمنان تو راپای انداز تو سازم»؟آیا همگی ایشان روح های خدمتگذار نیستند که برای خدمت آنانی که وارث نجات خواهند شد، فرستاده می‌شوند؟
\par 14 آیا همگی ایشان روح های خدمتگذار نیستند که برای خدمت آنانی که وارث نجات خواهند شد، فرستاده می‌شوند؟

\chapter{2}

\par 1 لهذا لازم است که به دقت بلیغ تر آنچه راشنیدیم گوش دهیم، مبادا که از آن ربوده شویم.
\par 2 زیرا هر گاه کلامی که بوساطت فرشتگان گفته شد برقرار گردید، بقسمی که هر تجاوز وتغافلی را جزای عادل می‌رسید،
\par 3 پس ما چگونه رستگار گردیم اگر از چنین نجاتی عظیم غافل باشیم؟ که در ابتدا تکلم به آن از خداوند بود و بعدکسانی که شنیدند، بر ما ثابت گردانیدند؛
\par 4 درحالتی که خدا نیز با ایشان شهادت می‌داد به آیات و معجزات و انواع قوات و عطایای روح‌القدس برحسب اراده خود.
\par 5 زیرا عالم آینده‌ای را که ذکر آن را می‌کنیم مطیع فرشتگان نساخت.
\par 6 لکن کسی در موضعی شهادت داده، گفت: «چیست انسان که او را بخاطرآوری یا پسر انسان که از او تفقد نمایی؟
\par 7 او را ازفرشتگان اندکی پست‌تر قرار دادی و تاج جلال واکرام را بر سر او نهادی و او را بر اعمال دستهای خود گماشتی.
\par 8 همه‌چیز را زیر پایهای اونهادی.» پس چون همه‌چیز را مطیع او گردانید، هیچ‌چیز را نگذاشت که مطیع او نباشد. لکن الان هنوز نمی بینیم که همه‌چیز مطیع وی شده باشد.
\par 9 اما او را که اندکی از فرشتگان کمتر شد می‌بینیم، یعنی عیسی را که به زحمت موت تاج جلال واکرام بر سر وی نهاده شد تا به فیض خدا برای همه ذائقه موت را بچشد.
\par 10 زیرا او را که بخاطروی همه و از وی همه‌چیز می‌باشد، چون فرزندان بسیار را وارد جلال می‌گرداند، شایسته بود که رئیس نجات ایشان را به دردها کامل گرداند.
\par 11 زانرو که چون مقدس کننده و مقدسان همه از یک می‌باشند، از این جهت عار ندارد که ایشان را برادر بخواند.
\par 12 چنانکه می‌گوید: «اسم تو را به برادران خود اعلام می‌کنم و در میان کلیساتو را تسبیح خواهم خواند.»
\par 13 و ایض: «من بروی توکل خواهم نمود.» و نیز: «اینک من وفرزندانی که خدا به من عطا فرمود.»
\par 14 پس چون فرزندان در خون و جسم شراکت دارند، او نیز همچنان در این هر دو شریک شد تابوساطت موت، صاحب قدرت موت یعنی ابلیس را تباه سازد،
\par 15 و آنانی را که از ترس موت، تمام عمر خود را گرفتار بندگی می‌بودند، آزاد گرداند.
\par 16 زیرا که در حقیقت فرشتگان را دستگیری نمی نماید بلکه نسل ابراهیم را دستگیری می‌نماید.
\par 17 از این جهت می‌بایست در هر امری مشابه برادران خود شود تا در امور خدا رئیس کهنه‌ای کریم و امین شده، کفاره گناهان قوم رابکند.زیرا که چون خود عذاب کشیده، تجربه دید استطاعت دارد که تجربه شدگان را اعانت فرماید.
\par 18 زیرا که چون خود عذاب کشیده، تجربه دید استطاعت دارد که تجربه شدگان را اعانت فرماید.

\chapter{3}

\par 1 بنابراین، ای برادران مقدس که در دعوت سماوی شریک هستید، در رسول و رئیس کهنه اعتراف ما یعنی عیسی تامل کنید،
\par 2 که نزداو که وی را معین فرمود امین بود، چنانکه موسی نیز در تمام خانه او بود.
\par 3 زیرا که این شخص لایق اکرامی بیشتر از موسی شمرده شد به آن اندازه‌ای که سازنده خانه را حرمت بیشتر از خانه است.
\par 4 زیرا هر خانه‌ای بدست کسی بنا می‌شود، لکن بانی همه خداست.
\par 5 و موسی مثل خادم در تمام خانه او امین بود تا شهادت دهد بر چیزهایی که می‌بایست بعد گفته شود.
\par 6 و اما مسیح مثل پسربر خانه او. و خانه او ما هستیم بشرطی که تا به انتهابه دلیری و فخر امید خود متمسک باشیم.
\par 7 پس چنانکه روح‌القدس می‌گوید: «امروزاگر آواز او را بشنوید،
\par 8 دل خود را سخت مسازید، چنانکه در وقت جنبش دادن خشم او درروز امتحان در بیابان،
\par 9 جایی که پدران شما مراامتحان و آزمایش کردند و اعمال مرا تا مدت چهل سال می‌دیدند.
\par 10 از این جهت به آن گروه خشم گرفته، گفتم ایشان پیوسته در دلهای خودگمراه هستند و راههای مرا نشناختند.
\par 11 تا درخشم خود قسم خوردم که به آرامی من داخل نخواهند شد.»
\par 12 ‌ای برادران، باحذر باشید مبادا در یکی ازشما دل شریر و بی‌ایمان باشد که از خدای حی مرتد شوید،
\par 13 بلکه هر روزه همدیگر رانصیحت کنید مادامی که «امروز» خوانده می‌شود، مبادا احدی از شما به فریب گناه سخت دل گردد.
\par 14 از آنرو که در مسیح شریک گشته‌ایم اگر به ابتدای اعتماد خود تا به انتهاسخت متمسک شویم.
\par 15 چونکه گفته می‌شود: «امروز اگر آواز او را بشنوید، دل خود را سخت مسازید، چنانکه در وقت جنبش دادن خشم او.»
\par 16 پس که بودند که شنیدند و خشم او راجنبش دادند؟ آیا تمام آن گروه نبودند که بواسطه موسی از مصر بیرون آمدند؟
\par 17 و به که تا مدت چهل سال خشمگین می‌بود؟ آیا نه به آن عاصیانی که بدنهای ایشان در صحرا ریخته شد؟
\par 18 و درباره که قسم خورد که به آرامی من داخل نخواهند شد، مگر آنانی را که اطاعت نکردند؟پس دانستیم که به‌سبب بی‌ایمانی نتوانستندداخل شوند.
\par 19 پس دانستیم که به‌سبب بی‌ایمانی نتوانستندداخل شوند.

\chapter{4}

\par 1 پس بترسیم مبادا با آنکه وعده دخول درآرامی وی باقی می‌باشد، ظاهر شود که احدی از شما قاصر شده باشد.
\par 2 زیرا که به ما نیز به مثال ایشان بشارت داده شد، لکن کلامی که شنیدند بدیشان نفع نبخشید، از اینرو که باشنوندگان به ایمان متحد نشدند.
\par 3 زیرا ما که ایمان آوردیم، داخل آن آرامی می‌گردیم، چنانکه گفته است: «در خشم خود قسم خوردم که به آرامی من داخل نخواهند شد.» و حال آنکه اعمال او از آفرینش عالم به اتمام رسیده بود.
\par 4 ودر مقامی درباره روز هفتم گفت که «در روز هفتم خدا از جمیع اعمال خود آرامی گرفت.»
\par 5 و بازدر این مقام که «به آرامی من داخل نخواهند شد.»
\par 6 پس چون باقی است که بعضی داخل آن بشوند و آنانی که پیش بشارت یافتند، به‌سبب نافرمانی داخل نشدند،
\par 7 باز روزی معین می‌فرماید چونکه به زبان داود بعد از مدت مدیدی «امروز» گفت، چنانکه پیش مذکور شدکه «امروز اگر آواز او را بشنوید، دل خود راسخت مسازید.»
\par 8 زیرا اگر یوشع ایشان را آرامی داده بود، بعد از آن دیگر را ذکر نمی کرد.
\par 9 پس برای قوم خدا آرامی سبت باقی می‌ماند.
\par 10 زیراهر‌که داخل آرامی او شد، او نیز از اعمال خودبیارامید، چنانکه خدا از اعمال خویش.
\par 11 پس جد و جهد بکنیم تا به آن آرامی داخل شویم، مبادا کسی در آن نافرمانی عبرت آمیز بیفتد.
\par 12 زیرا کلام خدا زنده و مقتدر و برنده تر است ازهر شمشیر دودم و فرورونده تا جدا کند نفس وروح و مفاصل و مغز را و ممیز افکار و نیتهای قلب است،
\par 13 و هیچ خلقت از نظر او مخفی نیست بلکه همه‌چیز در چشمان او که کار ما باوی است، برهنه و منکشف می‌باشد.
\par 14 پس چون رئیس کهنه عظیمی داریم که ازآسمانها درگذشته است یعنی عیسی، پسر خدا، اعتراف خود را محکم بداریم.
\par 15 زیرا رئیس کهنه‌ای نداریم که نتواند همدرد ضعفهای مابشود، بلکه آزموده شده در هر چیز به مثال مابدون گناه.پس با دلیری نزدیک به تخت فیض بیاییم تا رحمت بیابیم و فیضی را حاصل کنیم که در وقت ضرورت (ما را) اعانت کند.
\par 16 پس با دلیری نزدیک به تخت فیض بیاییم تا رحمت بیابیم و فیضی را حاصل کنیم که در وقت ضرورت (ما را) اعانت کند.

\chapter{5}

\par 1 زیرا که هر رئیس کهنه از میان آدمیان گرفته شده، برای آدمیان مقرر می‌شود در امورالهی تا هدایا و قربانی‌ها برای گناهان بگذراند؛
\par 2 که با جاهلان و گمراهان می‌تواند ملایمت کند، چونکه او نیز در کمزوری گرفته شده است.
\par 3 و به‌سبب این کمزوری، او را لازم است چنانکه برای قوم، همچنین برای خویشتن نیز قربانی برای گناهان بگذراند.
\par 4 و کسی این مرتبه را برای خودنمی گیرد، مگر وقتی که خدا او را بخواند، چنانکه هارون را.
\par 5 و همچنین مسیح نیز خود را جلال نداد که رئیس کهنه بشود، بلکه او که به وی گفت: «تو پسر من هستی؛ من امروز تو را تولید نمودم.»
\par 6 چنانکه در مقام دیگر نیز می‌گوید: «تو تا به ابدکاهن هستی بر رتبه ملکیصدق.»
\par 7 و او در ایام بشریت خود، چونکه با فریادشدید و اشکها نزد او که به رهانیدنش از موت قادر بود، تضرع و دعای بسیار کرد و به‌سبب تقوای خویش مستجاب گردید،
\par 8 هر‌چند پسربود، به مصیبتهایی که کشید، اطاعت را آموخت
\par 9 و کامل شده، جمیع مطیعان خود را سبب نجات جاودانی گشت.
\par 10 و خدا او را به رئیس کهنه مخاطب ساخت به رتبه ملکیصدق.
\par 11 که درباره او ما را سخنان بسیار است که شرح آنها مشکل می‌باشد چونکه گوشهای شماسنگین شده است.
\par 12 زیرا که هر‌چند با این طول زمان شما را می‌باید معلمان باشید، باز محتاجیدکه کسی اصول و مبادی الهامات خدا را به شمابیاموزد و محتاج شیر شدید نه غذای قوی.
\par 13 زیرا هر‌که شیرخواره باشد، در کلام عدالت ناآزموده است، چونکه طفل است.اما غذای قوی از آن بالغان است که حواس خود را به موجب عادت، ریاضت داده‌اند تا تمییز نیک و بدرا بکنند.
\par 14 اما غذای قوی از آن بالغان است که حواس خود را به موجب عادت، ریاضت داده‌اند تا تمییز نیک و بدرا بکنند.

\chapter{6}

\par 1 بنابراین، از کلام ابتدای مسیح درگذشته، به سوی کمال سبقت بجوییم و بار دیگر بنیادتوبه از اعمال مرده و ایمان به خدا ننهیم،
\par 2 وتعلیم تعمیدها و نهادن دستها و قیامت مردگان وداوری جاودانی را.
\par 3 و این را به‌جا خواهیم آوردهر گاه خدا اجازت دهد.
\par 4 زیرا آنانی که یک بار منور گشتند و لذت عطای سماوی را چشیدند و شریک روح‌القدس گردیدند
\par 5 و لذت کلام نیکوی خدا و قوات عالم آینده را چشیدند،
\par 6 اگر بیفتند، محال است که ایشان را بار دیگر برای توبه تازه سازند، در حالتی که پسر خدا را برای خود باز مصلوب می‌کنند واو را بی‌حرمت می‌سازند.
\par 7 زیرا زمینی که بارانی را که بارها بر آن می‌افتد، می‌خورد و نباتات نیکو برای فلاحان خود می‌رویاند، از خدا برکت می‌یابد.
\par 8 لکن اگر خار و خسک می‌رویاند، متروک و قرین به لعنت و در آخر، سوخته می‌شود.
\par 9 اما‌ای عزیزان در حق شما چیزهای بهتر وقرین نجات را یقین می‌داریم، هر‌چند بدینطورسخن می‌گوییم.
\par 10 زیرا خدا بی‌انصاف نیست که عمل شما و آن محبت را که به اسم او از خدمت مقدسین که در آن مشغول بوده و هستید ظاهرکرده‌اید، فراموش کند.
\par 11 لکن آرزوی این داریم که هر یک از شما همین جد و جهد را برای یقین کامل امید تا به انتها ظاهر نمایید،
\par 12 و کاهل مشوید بلکه اقتدا کنید آنانی را که به ایمان و صبروارث وعده‌ها می‌باشند.
\par 13 زیرا وقتی که خدا به ابراهیم وعده داد، چون به بزرگتر از خود قسم نتوانست خورد، به خود قسم خورده، گفت:
\par 14 «هرآینه من تو رابرکت عظیمی خواهم داد و تو را بی‌نهایت کثیرخواهم گردانید.»
\par 15 و همچنین چون صبر کرد، آن وعده را یافت.
\par 16 زیرا مردم به آنکه بزرگتراست، قسم می‌خورند و نهایت هر مخاصمه ایشان قسم است تا اثبات شود.
\par 17 از اینرو، چون خدا خواست که عدم تغییر اراده خود را به وارثان وعده به تاکید بی‌شمار ظاهر سازد، قسم در میان آورد.
\par 18 تا به دو امر بی‌تغییر که ممکن نیست خدادر مورد آنها دروغ گوید، تسلی قوی حاصل شود برای ما که پناه بردیم تا به آن امیدی که در پیش ما گذارده شده است تمسک جوییم،
\par 19 وآن را مثل لنگری برای جان خود ثابت و پایدارداریم که در درون حجاب داخل شده است،جایی که آن پیشرو برای ما داخل شد یعنی عیسی که بر رتبه ملکیصدق، رئیس کهنه گردید تاابدالاباد.
\par 20 جایی که آن پیشرو برای ما داخل شد یعنی عیسی که بر رتبه ملکیصدق، رئیس کهنه گردید تاابدالاباد.

\chapter{7}

\par 1 زیرا این ملکیصدق، پادشاه سالیم و کاهن خدای تعالی، هنگامی که ابراهیم ازشکست دادن ملوک، مراجعت می‌کرد، او رااستقبال کرده، بدو برکت داد.
\par 2 و ابراهیم نیز ازهمه‌چیزها ده‌یک بدو داد؛ که او اول ترجمه شده «پادشاه عدالت» است و بعد ملک سالیم نیز یعنی «پادشاه سلامتی».
\par 3 بی‌پدر و بی‌مادر وبی نسب نامه و بدون ابتدای ایام و انتهای حیات بلکه به شبیه پسر خدا شده، کاهن دایمی می‌ماند.
\par 4 پس ملاحظه کنید که این شخص چقدربزرگ بود که ابراهیم پاتریارخ نیز از بهترین غنایم، ده‌یک بدو داد.
\par 5 و اما از اولاد لاوی کسانی که کهانت را می‌یابند، حکم دارند که از قوم بحسب شریعت ده‌یک بگیرند، یعنی از برادران خود، باآنکه ایشان نیز از صلب ابراهیم پدید آمدند.
\par 6 لکن آن کس که نسبتی بدیشان نداشت، ازابراهیم ده‌یک گرفته و صاحب وعده‌ها را برکت داده است.
\par 7 و بدون هر شبهه، کوچک از بزرگ برکت داده می‌شود.
\par 8 و در اینجا مردمان مردنی ده‌یک می‌گیرند، اما در آنجا کسی‌که بر زنده بودن وی شهادت داده می‌شود.
\par 9 حتی آنکه گویا می توان گفت که بوساطت ابراهیم از همان لاوی که ده‌یک می‌گیرد، ده‌یک گرفته شد،
\par 10 زیرا که هنوز در صلب پدر خود بود هنگامی که ملکیصدق او را استقبال کرد.
\par 11 و دیگر اگر از کهانت لاوی، کمال حاصل می‌شد (زیرا قوم شریعت را بر آن یافتند)، باز چه احتیاج می‌بود که کاهنی دیگر بر رتبه ملکیصدق مبعوث شود و مذکور شود که بر رتبه هارون نیست؟
\par 12 زیرا هر گاه کهانت تغییر می‌پذیرد، البته شریعت نیز تبدیل می‌یابد.
\par 13 زیرا او که این سخنان در حق وی گفته می‌شود، از سبط دیگرظاهر شده است که احدی از آن، خدمت قربانگاه را نکرده است.
\par 14 زیرا واضح است که خداوند مااز سبط یهودا طلوع فرمود که موسی در حق آن سبط از جهت کهانت هیچ نگفت.
\par 15 و نیز بیشتر مبین است از اینکه به مثال ملکیصدق کاهنی بطور دیگر باید ظهور نماید
\par 16 که به شریعت و احکام جسمی مبعوث نشودبلکه به قوت حیات غیرفانی.
\par 17 زیرا شهادت داده شد که «تو تا به ابد کاهن هستی بر رتبه ملکیصدق.»
\par 18 زیرا که حاصل می‌شود هم نسخ حکم سابق بعلت ضعف و عدم فایده آن
\par 19 (از آن جهت که شریعت هیچ‌چیز را کامل نمی گرداند)، و هم برآوردن امید نیکوتر که به آن تقرب به خدامی جوییم.
\par 20 و بقدر آنکه این بدون قسم نمی باشد.
\par 21 زیرا ایشان بی‌قسم کاهن شده‌اند ولیکن این با قسم از او که به وی می‌گوید: «خداوند قسم خورد و تغییر اراده نخواهد داد که تو کاهن ابدی هستی بر رتبه ملکیصدق.»
\par 22 به همین قدرنیکوتر است آن عهدی که عیسی ضامن آن گردید.
\par 23 و ایشان کاهنان بسیار می‌شوند، ازجهت آنکه موت از باقی بودن ایشان مانع است.
\par 24 لکن وی چون تا به ابد باقی است، کهانت بی‌زوال دارد.
\par 25 از این جهت نیز قادر است که آنانی را که بوسیله وی نزد خدا آیند، نجات بی‌نهایت بخشد، چونکه دائم زنده است تاشفاعت ایشان را بکند.
\par 26 زیرا که ما را چنین رئیس کهنه شایسته است، قدوس و بی‌آزار و بی‌عیب و از گناهکاران جدا شده و از آسمانها بلندتر گردیده
\par 27 که هرروز محتاج نباشد به مثال آن روسای کهنه که اول برای گناهان خود و بعد برای قوم قربانی بگذراند، چونکه این را یک بار فقط به‌جا آورد هنگامی که خود را به قربانی گذرانید.از آنرو که شریعت مردمانی را که کمزوری دارند کاهن می‌سازد، لکن کلام قسم که بعد از شریعت است، پسر را که تاابدالاباد کامل شده است.
\par 28 از آنرو که شریعت مردمانی را که کمزوری دارند کاهن می‌سازد، لکن کلام قسم که بعد از شریعت است، پسر را که تاابدالاباد کامل شده است.

\chapter{8}

\par 1 پس مقصود عمده از این کلام این است که برای ما چنین رئیس کهنه‌ای هست که درآسمانها به‌دست راست تخت کبریا نشسته است،
\par 2 که خادم مکان اقدس و آن خیمه حقیقی است که خداوند آن را برپا نمود نه انسان.
\par 3 زیرا که هررئیس کهنه مقرر می‌شود تا هدایا و قربانی هابگذراند؛ و از این جهت واجب است که او را نیزچیزی باشد که بگذراند.
\par 4 پس اگر بر زمین می بود، کاهن نمی بود چون کسانی هستند که به قانون شریعت هدایا را می‌گذرانند.
\par 5 و ایشان شبیه و سایه چیزهای آسمانی را خدمت می‌کنند، چنانکه موسی ملهم شد هنگامی که عازم بود که خیمه را بسازد، زیرا بدو می‌گوید: «آگاه باش که همه‌چیز را به آن نمونه‌ای که در کوه به تو نشان داده شد بسازی.»
\par 6 لکن الان او خدمت نیکوتر یافته است، به مقداری که متوسط عهدنیکوتر نیز هست که بر وعده های نیکوتر مرتب است.
\par 7 زیرا اگر آن اول بی‌عیب می‌بود، جایی برای دیگری طلب نمی شد.
\par 8 چنانکه ایشان را ملامت کرده، می‌گوید: «خداوند می‌گوید اینک ایامی می‌آید که با خاندان اسرائیل و خاندان یهوداعهدی تازه استوار خواهم نمود.
\par 9 نه مثل آن عهدی که با پدران ایشان بستم، در روزی که من ایشان را دستگیری نمودم تا از زمین مصربرآوردم، زیرا که ایشان در عهد من ثابت نماندند. پس خداوند می‌گوید من ایشان را واگذاردم.
\par 10 وخداوند می‌گوید این است آن عهدی که بعد از آن ایام با خاندان اسرائیل استوار خواهم داشت که احکام خود را در خاطر ایشان خواهم نهاد و بردل ایشان مرقوم خواهم داشت و ایشان را خداخواهم بود و ایشان مرا قوم خواهند بود.
\par 11 ودیگر کسی همسایه و برادر خود را تعلیم نخواهدداد و نخواهد گفت خداوند را بشناس زیرا که همه از خرد و بزرگ مرا خواهند شناخت.
\par 12 زیرابر تقصیرهای ایشان ترحم خواهم فرمود وگناهانشان را دیگر به یاد نخواهم آورد.»پس چون «تازه» گفت، اول را کهنه ساخت؛ و آنچه کهنه و پیر شده است، مشرف بر زوال است.
\par 13 پس چون «تازه» گفت، اول را کهنه ساخت؛ و آنچه کهنه و پیر شده است، مشرف بر زوال است.

\chapter{9}

\par 1 خلاصه آن عهد اول را نیز فرایض خدمت و قدس دنیوی بود.
\par 2 زیرا خیمه اول نصب شد که در آن بود چراغدان و میز و نان تقدمه، و آن به قدس مسمی گردید.
\par 3 و در پشت پرده دوم بودآن خیمه‌ای که به قدس‌الاقداس مسمی است،
\par 4 که در آن بود مجمره زرین و تابوت شهادت که همه اطرافش به طلا آراسته بود؛ و در آن بود حقه طلا که پر از من بود و عصای هارون که شکوفه آورده بود و دو لوح عهد.
\par 5 و بر زبر آن کروبیان جلال که بر تخت رحمت سایه‌گستر می‌بودند والان جای تفصیل آنها نیست.
\par 6 پس چون این چیزها بدینطور آراسته شد، کهنه بجهت ادای لوازم خدمت، پیوسته به خیمه اول درمی آیند.
\par 7 لکن در دوم سالی یک مرتبه رئیس کهنه‌تنها داخل می‌شود؛ و آن هم نه بدون خونی که برای خود و برای جهالات قوم می‌گذراند.
\par 8 که به این همه روح‌القدس اشاره می‌نماید بر اینکه مادامی که خیمه اول برپاست، راه مکان اقدس ظاهر نمی شود.
\par 9 و این مثلی است برای زمان حاضر که بحسب آن هدایا وقربانی‌ها را می‌گذرانند که قوت ندارد که عبادت‌کننده را از جهت ضمیر کامل گرداند،
\par 10 چونکه اینها با چیزهای خوردنی و آشامیدنی و طهارات مختلفه، فقط فرایض جسدی است که تا زمان اصلاح مقرر شده است.
\par 11 لیکن مسیح چون ظاهر شد تا رئیس کهنه نعمتهای آینده باشد، به خیمه بزرگتر و کاملتر وناساخته شده به‌دست یعنی که از این خلقت نیست،
\par 12 و نه به خون بزها و گوساله‌ها، بلکه به خون خود، یک مرتبه فقط به مکان اقدس داخل شد و فدیه ابدی را یافت.
\par 13 زیرا هر گاه خون بزها و گاوان و خاکستر گوساله چون بر آلودگان پاشیده می‌شود، تا به طهارت جسمی مقدس می‌سازد،
\par 14 پس آیا چند مرتبه زیاده، خون مسیح که به روح ازلی خویشتن را بی‌عیب به خداگذرانید، ضمیر شما را از اعمال مرده طاهرنخواهد ساخت تا خدای زنده را خدمت نمایید؟
\par 15 و از این جهت او متوسط عهد تازه‌ای است تا چون موت برای کفاره تقصیرات عهد اول بوقوع آمد، خوانده‌شدگان وعده میراث ابدی رابیابند.
\par 16 زیرا در هر جایی که وصیتی است، لابداست که موت وصیت‌کننده را تصور کنند،
\par 17 زیرا که وصیت بعد از موت ثابت می‌شود؛ زیرامادامی که وصیت‌کننده زنده است، استحکامی ندارد.
\par 18 و از اینرو، آن اول نیز بدون خون برقرارنشد.
\par 19 زیرا که چون موسی تمامی احکام رابحسب شریعت، به سمع قوم رسانید، خون گوساله‌ها و بزها را با آب و پشم قرمز و زوفاگرفته، آن را بر خود کتاب و تمامی قوم پاشید؛
\par 20 و گفت: «این است خون آن عهدی که خداباشما قرار داد.»
\par 21 و همچنین خیمه و جمیع آلات خدمت را نیز به خون بیالود.
\par 22 و بحسب شریعت، تقریب همه‌چیز به خون طاهر می‌شود وبدون ریختن خون، آمرزش نیست.
\par 23 پس لازم بود که مثل های چیزهای سماوی به اینها طاهر شود، لکن خود سماویات به قربانی های نیکوتر از اینها.
\par 24 زیرا مسیح به قدس ساخته شده به‌دست داخل نشد که مثال مکان حقیقی است؛ بلکه به خود آسمان تا آنکه الان درحضور خدا بجهت ما ظاهر شود.
\par 25 و نه آنکه جان خود را بارها قربانی کند، مانند آن رئیس کهنه که هر سال با خون دیگری به مکان اقدس داخل می‌شود؛
\par 26 زیرا در این صورت می‌بایست که او از بنیاد عالم بارها زحمت کشیده باشد. لکن الان یک مرتبه در اواخر عالم ظاهر شد تا به قربانی خود، گناه را محو سازد.
\par 27 و چنانکه مردم را یک بار مردن و بعد از آن جزا یافتن مقرر است،همچنین مسیح نیز چون یک بار قربانی شد تاگناهان بسیاری را رفع نماید، بار دیگر بدون گناه، برای کسانی که منتظر او می‌باشند، ظاهر خواهدشد بجهت نجات.
\par 28 همچنین مسیح نیز چون یک بار قربانی شد تاگناهان بسیاری را رفع نماید، بار دیگر بدون گناه، برای کسانی که منتظر او می‌باشند، ظاهر خواهدشد بجهت نجات.

\chapter{10}

\par 1 زیرا که چون شریعت را سایه نعمتهای آینده است، نه نفس صورت آن چیزها، آن هرگز نمی تواند هر سال به همان قربانی هایی که پیوسته می‌گذرانند، تقرب جویندگان را کامل گرداند.
\par 2 والا آیا گذرانیدن آنها موقوف نمی شدچونکه عبادت‌کنندگان، بعد از آنکه یک بار پاک شدند، دیگر حس گناهان را در ضمیر نمی داشتند؟
\par 3 بلکه در اینها هر سال یادگاری گناهان می‌شود.
\par 4 زیرا محال است که خون گاوهاو بزها رفع گناهان را بکند.
\par 5 لهذا هنگامی که داخل جهان می‌شود، می‌گوید: «قربانی و هدیه را نخواستی، لکن جسدی برای من مهیا ساختی.
\par 6 به قربانی های سوختنی و قربانی های گناه رغبت نداشتی.
\par 7 آنگاه گفتم، اینک می‌آیم (در طومار کتاب درحق من مکتوب است ) تا اراده تو را‌ای خدا بجاآورم.»
\par 8 چون پیش می‌گوید: «هدایا و قربانی‌ها وقربانی های سوختنی و قربانی های گناه رانخواستی و به آنها رغبت نداشتی، » که آنها رابحسب شریعت می‌گذرانند،
\par 9 بعد گفت که «اینک می‌آیم تا اراده تو را‌ای خدا بجا آورم.» پس اول را برمی دارد، تا دوم را استوار سازد.
\par 10 و به این اراده مقدس شده‌ایم، به قربانی جسد عیسی مسیح، یک مرتبه فقط.
\par 11 و هر کاهن هر روزه به خدمت مشغول بوده، می‌ایستد وهمان قربانی‌ها را مکرر می‌گذراند که هرگز رفع گناهان را نمی تواند کرد.
\par 12 لکن او چون یک قربانی برای گناهان گذرانید، به‌دست راست خدابنشست تا ابدالاباد.
\par 13 و بعد از آن منتظر است تادشمنانش پای انداز او شوند.
\par 14 از آنرو که به یک قربانی مقدسان را کامل گردانیده است تا ابدالاباد.
\par 15 و روح‌القدس نیزبرای ما شهادت می‌دهد، زیرا بعد از آنکه گفته بود:
\par 16 «این است آن عهدی که بعد از آن ایام باایشان خواهم بست، خداوند می‌گوید احکام خود را در دلهای ایشان خواهم نهاد و بر ذهن ایشان مرقوم خواهم داشت،
\par 17 (باز می‌گوید) وگناهان و خطایای ایشان را دیگر به یاد نخواهم آورد.»
\par 18 اما جایی که آمرزش اینها هست، دیگرقربانی گناهان نیست.
\par 19 پس‌ای برادران، چونکه به خون عیسی دلیری داریم تا به مکان اقدس داخل شویم
\par 20 ازطریق تازه و زنده که آن را بجهت ما از میان پرده یعنی جسم خود مهیا نموده است،
\par 21 و کاهنی بزرگ را بر خانه خدا داریم،
\par 22 پس به دل راست، در یقین ایمان، دلهای خود را از ضمیر بد پاشیده و بدنهای خود را به آب پاک غسل داده، نزدیک بیاییم؛
\par 23 و اعتراف امید را محکم نگاه داریم زیراکه وعده‌دهنده امین است.
\par 24 و ملاحظه یکدیگررا بنماییم تا به محبت و اعمال نیکو ترغیب نماییم.
\par 25 و از با هم آمدن در جماعت غافل نشویم چنانکه بعضی را عادت است، بلکه یکدیگر را نصیحت کنیم و زیادتر به اندازه‌ای که می‌بینید که آن روز نزدیک می‌شود.
\par 26 زیرا که بعد از پذیرفتن معرفت راستی اگرعمد گناهکار شویم، دیگر قربانی گناهان باقی نیست،
\par 27 بلکه انتظار هولناک عذاب و غیرت آتشی که مخالفان را فرو خواهد برد.
\par 28 هر‌که شریعت موسی را خوار شمرد، بدون رحم به دویا سه شاهد کشته می‌شود.
\par 29 پس به چه مقدارگمان می‌کنید که آن کس، مستحق عقوبت سخت‌تر شمرده خواهد شد که پسر خدا راپایمال کرد و خون عهدی را که به آن مقدس گردانیده شد، ناپاک شمرد و روح نعمت رابی حرمت کرد؟
\par 30 زیرا می‌شناسیم او را که گفته است: «خداوند می‌گوید انتقام از آن من است؛ من مکافات خواهم داد.» و ایض: «خداوند قوم خودرا داوری خواهد نمود.»
\par 31 افتادن به‌دستهای خدای زنده چیزی هولناک است.
\par 32 ولیکن ایام سلف را به یاد آورید که بعد ازآنکه منور گردیدید، متحمل مجاهده‌ای عظیم ازدردها شدید،
\par 33 چه از اینکه از دشنامها وزحمات تماشای مردم می‌شدید، و چه از آنکه شریک با کسانی می‌بودید که در چنین چیزها بسرمی بردند.
\par 34 زیرا که با اسیران نیز همدردمی بودید و تاراج اموال خود را نیز به خوشی می‌پذیرفتید، چون دانستید که خود شما را درآسمان مال نیکوتر و باقی است.
\par 35 پس ترک مکنید دلیری خود را که مقرون به مجازات عظیم می‌باشد.
\par 36 زیرا که شما را صبر لازم است تااراده خدا را بجا آورده، وعده را بیابید.
\par 37 زیرا که «بعد از اندک زمانی، آن آینده خواهد آمد وتاخیر نخواهد نمود.
\par 38 لکن عادل به ایمان زیست خواهد نمود و اگر مرتد شود نفس من باوی خوش نخواهد شد.»لکن ما از مرتدان نیستیم تا هلاک شویم، بلکه از ایمانداران تا جان خود را دریابیم.
\par 39 لکن ما از مرتدان نیستیم تا هلاک شویم، بلکه از ایمانداران تا جان خود را دریابیم.

\chapter{11}

\par 1 پس ایمان، اعتماد بر چیزهای امید داشته شده است و برهان چیزهای نادیده.
\par 2 زیرا که به این، برای قدما شهادت داده شد.
\par 3 به ایمان فهمیده‌ایم که عالم‌ها به کلمه خدامرتب گردید، حتی آنکه چیزهای دیدنی ازچیزهای نادیدنی ساخته شد.
\par 4 به ایمان هابیل قربانی نیکوتر از قائن را به خدا گذرانید و به‌سبب آن شهادت داده شد که عادل است، به آنکه خدا به هدایای او شهادت می‌دهد؛ و به‌سبب همان بعد از مردن هنوزگوینده است.
\par 5 به ایمان خنوخ منتقل گشت تا موت را نبیندو نایاب شد چرا‌که خدا او را منتقل ساخت زیراقبل از انتقال وی شهادت داده شد که رضامندی خدا را حاصل کرد.
\par 6 لیکن بدون ایمان تحصیل رضامندی او محال است، زیرا هر‌که تقرب به خدا جوید، لازم است که ایمان آورد بر اینکه اوهست و جویندگان خود را جزا می‌دهد.
\par 7 به ایمان نوح چون درباره اموری که تا آن وقت دیده نشده، الهام یافته بود، خداترس شده، کشتی‌ای بجهت اهل خانه خود بساخت و به آن، دنیا را ملزم ساخته، وارث آن عدالتی که از ایمان است گردید.
\par 8 به ایمان ابراهیم چون خوانده شد، اطاعت نمود و بیرون رفت به سمت آن مکانی که می‌بایست به میراث یابد. پس بیرون آمد ونمی دانست به کجا می‌رود.
\par 9 و به ایمان در زمین وعده مثل زمین بیگانه غربت پذیرفت و درخیمه‌ها با اسحاق و یعقوب که در میراث همین وعده شریک بودند مسکن نمود.
\par 10 زانرو که مترقب شهری بابنیاد بود که معمار و سازنده آن خداست.
\par 11 به ایمان خود ساره نیز قوت قبول نسل یافت و بعد از انقضای وقت زایید، چونکه وعده‌دهنده را امین دانست.
\par 12 و از این سبب، ازیک نفر و آن هم مرده، مثل ستارگان آسمان، کثیرو مانند ریگهای کنار دریا، بی‌شمار زاییده شدند.
\par 13 در ایمان همه ایشان فوت شدند، درحالیکه وعده‌ها را نیافته بودند، بلکه آنها را از دوردیده، تحیت گفتند و اقرار کردند که بر روی زمین، بیگانه و غریب بودند.
\par 14 زیرا کسانی که همچنین می‌گویند، ظاهر می‌سازند که در جستجوی وطنی هستند.
\par 15 و اگر جایی را که از آن بیرون آمدند، بخاطر می‌آوردند، هرآینه فرصت می‌داشتند که (بدانجا) برگردند.
\par 16 لکن الحال مشتاق وطنی نیکوتر یعنی (وطن ) سماوی هستند و از اینرو خدا از ایشان عار ندارد که خدای ایشان خوانده شود، چونکه برای ایشان شهری مهیا ساخته است.
\par 17 به ایمان ابراهیم چون امتحان شد، اسحاق را گذرانید و آنکه وعده‌ها را پذیرفته بود، پسریگانه خود را قربانی می‌کرد؛
\par 18 که به او گفته شده بود که «نسل تو به اسحاق خوانده خواهد شد.»
\par 19 چونکه یقین دانست که خدا قادر بربرانگیزانیدن از اموات است و همچنین او را درمثلی از اموات نیز باز‌یافت.
\par 20 به ایمان اسحاق نیز یعقوب و عیسو را درامور آینده برکت داد.
\par 21 به ایمان یعقوب در وقت مردن خود، هریکی از پسران یوسف را برکت داد و بر سر عصای خود سجده کرد.
\par 22 به ایمان یوسف در حین وفات خود، ازخروج بنی‌اسرائیل اخبار نمود و درباره استخوانهای خود وصیت کرد.
\par 23 به ایمان موسی چون متولد شد، والدینش او را طفلی جمیل یافته، سه ماه پنهان کردند و ازحکم پادشاه بیم نداشتند.
\par 24 به ایمان چون موسی بزرگ شد، ابا نمود ازاینکه پسر دختر فرعون خوانده شود،
\par 25 و ذلیل بودن با قوم خدا را پسندیده تر داشت از آنکه لذت اندک زمانی گناه را ببرد؛
\par 26 و عار مسیح رادولتی بزرگتر از خزائن مصر پنداشت زیرا که به سوی مجازات نظر می‌داشت.
\par 27 به ایمان، مصررا ترک کرد و از غضب پادشاه نترسید زیرا که چون آن نادیده را بدید، استوار ماند.
\par 28 به ایمان، عید فصح و پاشیدن خون را بعمل آورد تاهلاک کننده نخستزادگان، بر ایشان دست نگذارد.
\par 29 به ایمان، از بحر قلزم به خشکی عبورنمودند واهل مصر قصد آن کرده، غرق شدند.
\par 30 به ایمان حصار اریحا چون هفت روز آن راطواف کرده بودند، به زیر افتاد.
\par 31 به ایمان، راحاب فاحشه با عاصیان هلاک نشد زیرا که جاسوسان را به سلامتی پذیرفته بود.
\par 32 و دیگر‌چه گویم؟ زیرا که وقت مرا کفاف نمی دهد که از جدعون و باراق و شمشون و یفتاح و داود و سموئیل و انبیا اخبار نمایم،
\par 33 که ازایمان، تسخیر ممالک کردند و به اعمال صالحه پرداختند و وعده‌ها را پذیرفتند و دهان شیران رابستند،
\par 34 سورت آتش را خاموش کردند و از دم شمشیرها رستگار شدند و از ضعف، توانایی یافتند و در جنگ شجاع شدند و لشکرهای غربارا منهزم ساختند.
\par 35 زنان، مردگان خود را به قیامتی باز‌یافتند، لکن دیگران معذب شدند وخلاصی را قبول نکردند تا به قیامت نیکوتربرسند.
\par 36 و دیگران از استهزاها و تازیانه‌ها بلکه از بندها و زندان آزموده شدند.
\par 37 سنگسارگردیدند و با اره دوپاره گشتند. تجربه کرده شدندو به شمشیر مقتول گشتند. در پوستهای گوسفندان و بزها محتاج و مظلوم و ذلیل و آواره شدند.
\par 38 آنانی که جهان لایق ایشان نبود، درصحراها و کوهها و مغاره‌ها و شکافهای زمین پراکنده گشتند.
\par 39 پس جمیع ایشان با اینکه از ایمان شهادت داده شدند، وعده را نیافتند.زیرا خدا برای ماچیزی نیکوتر مهیا کرده است تا آنکه بدون ماکامل نشوند.
\par 40 زیرا خدا برای ماچیزی نیکوتر مهیا کرده است تا آنکه بدون ماکامل نشوند.

\chapter{12}

\par 1 تادیب الهی بنابراین چونکه ما نیز چنین ابر شاهدان را گرداگرد خود داریم، هر بار گران وگناهی را که ما را سخت می‌پیچد دور بکنیم و باصبر در آن میدان که پیش روی ما مقرر شده است بدویم،
\par 2 و به سوی پیشوا و کامل کننده ایمان یعنی عیسی نگران باشیم که بجهت آن خوشی که پیش او موضوع بود، بی‌حرمتی را ناچیز شمرده، متحمل صلیب گردید و به‌دست راست تخت خدا نشسته است.
\par 3 پس تفکر کنید در او که متحمل چنین مخالفتی بود که از گناهکاران به اوپدید آمد، مبادا در جانهای خود ضعف کرده، خسته شوید.
\par 4 هنوز در جهاد با گناه تا به حدخون مقاومت نکرده‌اید،
\par 5 و نصیحتی را فراموش نموده‌اید که با شما چون با پسران مکالمه می‌کندکه «ای پسر من تادیب خداوند را خوار مشمار ووقتی که از او سرزنش یابی، خسته خاطر مشو.
\par 6 زیرا هر‌که را خداوند دوست می‌دارد، توبیخ می‌فرماید و هر فرزند مقبول خود را به تازیانه می‌زند.»
\par 7 اگر متحمل تادیب شوید، خدا با شما مثل باپسران رفتار می‌نماید. زیرا کدام پسر است که پدرش او را تادیب نکند؟
\par 8 لکن اگر بی‌تادیب می‌باشید، که همه از آن بهره یافتند، پس شماحرام زادگانید نه پسران.
\par 9 و دیگر پدران جسم خود را وقتی داشتیم که ما را تادیب می‌نمودند وایشان را احترام می‌نمودیم، آیا از طریق اولی پدرروحها را اطاعت نکنیم تا زنده شویم؟
\par 10 زیرا که ایشان اندک زمانی، موافق صوابدید خود ما راتادیب کردند، لکن او بجهت فایده تا شریک قدوسیت او گردیم.
\par 11 لکن هر تادیب در حال، نه از خوشیها بلکه از دردها می‌نماید، اما در آخرمیوه عدالت سلامتی را برای آنانی که از آن ریاضت یافته‌اند بار می‌آورد.
\par 12 لهذا دستهای افتاده و زانوهای سست شده را استوار نمایید،
\par 13 و برای پایهای خود راههای راست بسازید تاکسی‌که لنگ باشد، از طریق منحرف نشود، بلکه شفا یابد.
\par 14 و در‌پی سلامتی با همه بکوشید و تقدسی که بغیر از آن هیچ‌کس خداوند را نخواهد دید.
\par 15 و مترصد باشید مبادا کسی از فیض خدامحروم شود و ریشه مرارت نمو کرده، اضطراب بار آورد و جمعی از آن آلوده گردند.
\par 16 مباداشخصی زانی یا بی‌مبالات پیدا شود، مانند عیسوکه برای طعامی نخستزادگی خود را بفروخت.
\par 17 زیرا می‌دانید که بعد از آن نیز وقتی که خواست وارث برکت شود مردود گردید (زیرا که جای توبه پیدا ننمود) با آنکه با اشکها در جستجوی آن بکوشید.
\par 18 زیرا تقرب نجسته‌اید به کوهی که می‌توان لمس کرد و به آتش افروخته و نه به تاریکی وظلمت و باد سخت،
\par 19 و نه به آواز کرنا و صدای کلامی که شنوندگان، التماس کردند که آن کلام، دیگر بدیشان گفته نشود.
\par 20 زیرا که متحمل آن قدغن نتوانستند شد که اگر حیوانی نیز کوه رالمس کند، سنگسار یا به نیزه زده شود.
\par 21 و آن رویت بحدی ترسناک بود که موسی گفت: «بغایت ترسان و لرزانم.»
\par 22 بلکه تقرب جسته ایدبه جبل صهیون و شهر خدای حی یعنی اورشلیم سماوی و به جنود بی‌شماره از محفل فرشتگان
\par 23 و کلیسای نخستزادگانی که در آسمان مکتوبندو به خدای داور جمیع و به ارواح عادلان مکمل
\par 24 و به عیسی متوسط عهد جدید و به خون پاشیده شده که متکلم است به‌معنی نیکوتر ازخون هابیل.
\par 25 زنهار از آنکه سخن می‌گوید رو مگردانیدزیرا اگر آنانی که از آنکه بر زمین سخن گفت روگردانیدند، نجات نیافتند، پس ما چگونه نجات خواهیم یافت اگر از او که از آسمان سخن می‌گوید روگردانیم؟
\par 26 که آواز او در آن وقت زمین را جنبایند، لکن الان وعده داده است که «یک مرتبه دیگر نه فقط زمین بلکه آسمان را نیزخواهم جنبانید.»
\par 27 و این قول او یک مرتبه دیگراشاره است از تبدیل چیزهایی که جنبانیده می‌شود، مثل آنهایی که ساخته شد، تا آنهایی که جنبانیده نمی شود باقی ماند.
\par 28 پس چون ملکوتی را که نمی توان جنبانیدمی یابیم، شکر به‌جا بیاوریم تا به خشوع و تقواخدا را عبادت پسندیده نماییم.زیرا خدای ماآتش فروبرنده است.
\par 29 زیرا خدای ماآتش فروبرنده است.

\chapter{13}

\par 1 محبت برادرانه برقرار باشد؛
\par 2 و ازغریب‌نوازی غافل مشوید زیرا که به آن بعضی نادانسته فرشتگان را ضیافت کردند.
\par 3 اسیران را بخاطر آرید مثل همزندان ایشان، ومظلومان را چون شما نیز در جسم هستید.
\par 4 نکاح به هر وجه محترم باشد و بسترش غیرنجس زیرا که فاسقان و زانیان را خدا داوری خواهد فرمود.
\par 5 سیرت شما از محبت نقره خالی باشد و به آنچه دارید قناعت کنید زیرا که او گفته است: «تو را هرگز رها نکنم و تو را ترک نخواهم نمود.»
\par 6 بنابراین ما با دلیری تمام می‌گوییم: «خداوند مددکننده من است و ترسان نخواهم بود. انسان به من چه می‌کند؟»
\par 7 مرشدان خود را که کلام خدا را به شما بیان کردند بخاطر دارید و انجام سیرت ایشان راملاحظه کرده، به ایمان ایشان اقتدا نمایید.
\par 8 عیسی مسیح دیروز و امروز و تا ابدالاباد همان است.
\par 9 از تعلیمهای مختلف و غریب از جا برده مشوید، زیرا بهتر آن است که دل شما به فیض استوار شود و نه به خوراکهایی که آنانی که در آنهاسلوک نمودند، فایده نیافتند.
\par 10 مذبحی داریم که خدمت گذاران آن خیمه، اجازت ندارند که از آن بخورند.
\par 11 زیرا که جسدهای آن حیواناتی که رئیس کهنه خون آنهارا به قدس‌الاقداس برای گناه می‌برد، بیرون ازلشکرگاه سوخته می‌شود.
\par 12 بنابراین، عیسی نیزتا قوم را به خون خود تقدیس نماید، بیرون دروازه عذاب کشید.
\par 13 لهذا عار او را برگرفته، بیرون از لشکرگاه به سوی او برویم.
\par 14 زانرو که در اینجا شهری باقی نداریم بلکه آینده را طالب هستیم.
\par 15 پس بوسیله او قربانی تسبیح را به خدابگذرانیم، یعنی ثمره لبهایی را که به اسم او معترف باشند.
\par 16 لکن از نیکوکاری و خیرات غافل مشوید، زیرا خدا به همین قربانی‌ها راضی است.
\par 17 مرشدان خود را اطاعت و انقیاد نمایید زیراکه ایشان پاسبانی جانهای شما را می‌کنند، چونکه حساب خواهند داد تا آن را به خوشی نه به ناله به‌جا آورند، زیرا که این شما را مفید نیست.
\par 18 برای ما دعا کنید زیرا ما را یقین است که ضمیر خالص داریم و می‌خواهیم در هر امر رفتارنیکو نماییم.
\par 19 و بیشتر التماس دارم که چنین کنید تا زودتر به نزد شما باز آورده شوم.
\par 20 پس خدای سلامتی که شبان اعظم گوسفندان یعنی خداوند ما عیسی را به خون عهدابدی از مردگان برخیزانید،
\par 21 شما را در هر عمل نیکو کامل گرداناد تا اراده او را به‌جا آورید وآنچه منظور نظر او باشد، در شما بعمل آوردبوساطت عیسی مسیح که او را تا ابدالاباد جلال باد. آمین.
\par 22 لکن‌ای برادران از شما التماس دارم که این کلام نصیحت‌آمیز را متحمل شوید زیرامختصری نیز به شما نوشته‌ام.
\par 23 بدانید که برادرما تیموتاوس رهایی یافته است و اگر زود آید، به اتفاق او شما را ملاقات خواهم نمود.همه مرشدان خود و جمیع مقدسین راسلام برسانید؛ و آنانی که از ایتالیا هستند، به شماسلام می‌رسانند.
\par 24 همه مرشدان خود و جمیع مقدسین راسلام برسانید؛ و آنانی که از ایتالیا هستند، به شماسلام می‌رسانند.



\end{document}