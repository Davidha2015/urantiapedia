\begin{document}

\title{يعقوب}


\chapter{1}

\par 1 یعقوب که غلام خدا و عیسی مسیح خداوند است، به دوازده سبط که پراکنده هستند.خوش باشید.
\par 2 ‌ای برادران من، وقتی که در تجربه های گوناگون مبتلا شوید، کمال خوشی دانید.
\par 3 چونکه می‌دانید که امتحان ایمان شما صبر راپیدا می‌کند.
\par 4 لکن صبر را عمل تام خود باشد تاکامل و تمام شوید و محتاج هیچ‌چیز نباشید.
\par 5 واگر از شما کسی محتاج به حکمت باشد، سوال بکند از خدایی که هر کس را به سخاوت عطامی کند و ملامت نمی نماید و به او داده خواهدشد.
\par 6 لکن به ایمان سوال بکند و هرگز شک نکندزیرا هرکه شک کند، مانند موج دریاست که از بادرانده و متلاطم می‌شود.
\par 7 زیرا چنین شخص گمان نبرد که از خداوند چیزی خواهد یافت.
\par 8 مرد دودل در تمام رفتار خود ناپایدار است.
\par 9 لکن برادر مسکین به‌سرافرازی خود فخربنماید،
\par 10 و دولتمند از مسکنت خود، زیرا مثل گل علف در گذر است.
\par 11 از آنرو که آفتاب باگرمی طلوع کرده، علف را خشکانید و گلش به زیر افتاده، حسن صورتش زایل شد. به همینطورشخص دولتمند نیز در راههای خود، پژمرده خواهد گردید.
\par 12 خوشابحال کسی‌که متحمل تجربه شود، زیرا که چون آزموده شد، آن تاج حیاتی را که خداوند به محبان خود وعده فرموده است خواهد یافت.
\par 13 هیچ‌کس چون در تجربه افتد، نگوید: «خدا مرا تجربه می‌کند»، زیرا خدا هرگزاز بدیها تجربه نمی شود و او هیچ‌کس را تجربه نمی کند.
\par 14 لکن هرکس در تجربه می‌افتد وقتی که شهوت وی او را می‌کشد و فریفته می‌سازد.
\par 15 پس شهوت آبستن شده، گناه را می‌زاید و گناه به انجام رسیده، موت را تولید می‌کند.
\par 16 ‌ای برادران عزیز من، گمراه مشوید!
\par 17 هر بخشندگی نیکو و هر بخشش کامل از بالا است و نازل می‌شود از پدر نورها که نزد او هیچ تبدیل و سایه گردش نیست.
\par 18 او محض اراده خود ما رابوسیله کلمه حق تولید نمود تا ما چون نوبرمخلوقات او باشیم.
\par 19 بنابراین، ای برادران عزیز من، هرکس درشنیدن تند و در گفتن آهسته و در خشم سست باشد.
\par 20 زیرا خشم انسان عدالت خدا را به عمل نمی آورد.
\par 21 پس هر نجاست و افزونی شر را دورکنید و با فروتنی، کلام کاشته شده را بپذیرید که قادر است که جانهای شما را نجات‌بخشد.
\par 22 لکن کنندگان کلام باشید نه فقط شنوندگان که خود را فریب می‌دهند.
\par 23 زیرا اگر کسی کلام را بشنود و عمل نکند، شخصی را ماند که صورت طبیعی خود را در آینه می‌نگرد.
\par 24 زیرا خود رانگریست و رفت و فور فراموش کرد که چطورشخصی بود.
\par 25 لکن کسی‌که بر شریعت کامل آزادی چشم دوخت و در آن ثابت ماند، او چون شنونده فراموشکار نمی باشد، بلکه کننده عمل پس او در عمل خود مبارک خواهد بود.
\par 26 اگرکسی از شما گمان برد که پرستنده خدا است وعنان زبان خود را نکشد بلکه دل خود را فریب دهد، پرستش او باطل است.پرستش صاف وبی عیب نزد خدا و پدر این است که یتیمان وبیوه‌زنان را در مصیبت ایشان تفقد کنند و خود رااز آلایش دنیا نگاه دارند.
\par 27 پرستش صاف وبی عیب نزد خدا و پدر این است که یتیمان وبیوه‌زنان را در مصیبت ایشان تفقد کنند و خود رااز آلایش دنیا نگاه دارند.

\chapter{2}

\par 1 ای برادران من، ایمان خداوند ما عیسی مسیح، رب الجلال را با ظاهربینی مدارید.
\par 2 زیرا اگر به کنیسه شما شخصی با انگشتری زرین و لباس نفیس داخل شود و فقیری نیز باپوشاک ناپاک درآید،
\par 3 و به صاحب لباس فاخرمتوجه شده، گویید: «اینجا نیکو بنشین» و به فقیرگویید: «تو در آنجا بایست یا زیر پای انداز من بنشین»،
\par 4 آیا در خود متردد نیستید و داوران خیالات فاسد نشده‌اید؟
\par 5 ‌ای برادران عزیز، گوش دهید. آیا خدا فقیران این جهان را برنگزیده است تا دولتمند در ایمان و وارث آن ملکوتی که به محبان خود وعده فرموده است بشوند؟
\par 6 لکن شما فقیر را حقیر شمرده‌اید. آیا دولتمندان برشما ستم نمی کنند و شما را در محکمه هانمی کشند؟
\par 7 آیا ایشان به آن نام نیکو که بر شما نهاده شده است کفر نمی گویند؟
\par 8 اما اگر آن شریعت ملوکانه را برحسب کتاب به‌جا آورید یعنی «همسایه خود را مثل نفس خود محبت نما» نیکو می‌کنید.
\par 9 لکن اگرظاهربینی کنید، گناه می‌کنید و شریعت شما را به خطاکاری ملزم می‌سازد.
\par 10 زیرا هرکه تمام شریعت را نگاه دارد و در یک جزو بلغزد، ملزم همه می‌باشد.
\par 11 زیرا او که گفت: «زنا مکن»، نیزگفت: «قتل مکن». پس هرچند زنا نکنی، اگر قتل کردی، از شریعت تجاوز نمودی.
\par 12 همچنین سخن گویید و عمل نمایید مانند کسانی که برایشان داوری به شریعت آزادی خواهد شد.
\par 13 زیرا آن داوری بی‌رحم خواهد بود برکسی‌که رحم نکرده است و رحم بر داوری مفتخرمی شود.
\par 14 ‌ای برادران من، چه سود دارد اگر کسی گوید: «ایمان دارم» وقتی که عمل ندارد؟ آیاایمان می‌تواند او را نجات‌بخشد؟
\par 15 پس اگربرادری یا خواهری برهنه و محتاج خوراک روزینه باشد،
\par 16 و کسی از شما بدیشان گوید: «به سلامتی بروید و گرم و سیر شوید»، لیکن مایحتاج بدن را بدیشان ندهد، چه نفع دارد؟
\par 17 همچنین ایمان نیز اگر اعمال ندارد، در خودمرده است.
\par 18 بلکه کسی خواهد گفت: «تو ایمان داری و من اعمال دارم. ایمان خود را بدون اعمال به من بنما و من ایمان خود را از اعمال خود به توخواهم نمود.»
\par 19 تو ایمان داری که خدا واحداست؟ نیکو می‌کنی! شیاطین نیز ایمان دارند و می لرزند!
\par 20 و لیکن‌ای مرد باطل، آیا می‌خواهی دانست که ایمان بدون اعمال، باطل است؟
\par 21 آیا پدر ما ابراهیم به اعمال، عادل شمرده نشد وقتی که پسر خود اسحاق را به قربانگاه گذرانید؟
\par 22 می‌بینی که ایمان با اعمال او عمل کرد و ایمان از اعمال، کامل گردید.
\par 23 و آن نوشته تمام گشت که می‌گوید: «ابراهیم به خداایمان آورد و برای او به عدالت محسوب گردید»، و دوست خدا نامیده شد.
\par 24 پس می‌بینید که انسان از اعمال عادل شمرده می‌شود، نه از ایمان تنها.
\par 25 و همچنین آیا راحاب فاحشه نیز ازاعمال عادل شمرده نشد وقتی که قاصدان راپذیرفته، به راهی دیگر روانه نمود؟زیراچنانکه بدن بدون روح مرده است، همچنین ایمان بدون اعمال نیز مرده است.
\par 26 زیراچنانکه بدن بدون روح مرده است، همچنین ایمان بدون اعمال نیز مرده است.

\chapter{3}

\par 1 ای برادران من، بسیار معلم نشوید چونکه می دانید که بر ما داوری سخت‌تر خواهدشد.
\par 2 زیرا همگی ما بسیار می‌لغزیم. و اگر کسی در سخن‌گفتن نلغزد، او مرد کامل است و می‌تواندعنان تمام جسد خود را بکشد.
\par 3 و اینک لگام رابر دهان اسبان می‌زنیم تا مطیع ما شوند و تمام بدن آنها را برمی گردانیم.
\par 4 اینک کشتیها نیز چقدربزرگ است و از بادهای سخت رانده می‌شود، لکن با سکان کوچک به هر طرفی که اراده ناخدا باشد، برگردانیده می‌شود.
\par 5 همچنان زبان نیز عضوی کوچک است و سخنان کبرآمیز می‌گوید. اینک آتش کمی چه جنگل عظیمی را می‌سوزاند.
\par 6 وزبان آتشی است! آن عالم ناراستی در میان اعضای ما زبان است که تمام بدن را می‌آلاید ودایره کائنات را می‌سوزاند و از جهنم سوخته می‌شود!
\par 7 زیرا که هر طبیعتی از وحوش و طیور وحشرات و حیوانات بحری از طبیعت انسان رام می‌شود و رام شده است.
\par 8 لکن زبان را کسی ازمردمان نمی تواند رام کند. شرارتی سرکش و پر اززهر قاتل است!
\par 9 خدا و پدر را به آن متبارک می‌خوانیم و به همان مردمان را که به صورت خداآفریده شده‌اند، لعن می‌گوییم.
\par 10 از یک دهان برکت و لعنت بیرون می‌آید! ای برادران، شایسته نیست که چنین شود.
\par 11 آیا چشمه از یک شکاف آب شیرین و شور جاری می‌سازد؟
\par 12 یا می‌شودای برادران من که درخت انجیر، زیتون یا درخت مو، انجیر بار آورد؟ و چشمه شور نمی تواند آب شیرین را موجود سازد.
\par 13 کیست در میان شما که حکیم و عالم باشد؟ پس اعمال خود را از سیرت نیکو به تواضع حکمت ظاهر بسازد.
\par 14 لکن اگر در دل خودحسد تلخ و تعصب دارید، فخر مکنید و به ضدحق دروغ مگویید.
\par 15 این حکمت از بالا نازل نمی شود، بلکه دنیوی و نفسانی و شیطانی است.
\par 16 زیرا هرجایی که حسد و تعصب است، درآنجا فتنه و هر امر زشت موجود می‌باشد.
\par 17 لکن آن حکمت که از بالا است، اول طاهر است و بعدصلح‌آمیز و ملایم و نصیحت‌پذیر و پر از رحمت و میوه های نیکو و بی‌تردد و بی‌ریا.و میوه عدالت در سلامتی کاشته می‌شود برای آنانی که سلامتی را بعمل می‌آورند.
\par 18 و میوه عدالت در سلامتی کاشته می‌شود برای آنانی که سلامتی را بعمل می‌آورند.

\chapter{4}

\par 1 از کجا در میان شما جنگها و از کجا نزاعهاپدید می‌آید؟ آیا نه از لذت های شما که دراعضای شما جنگ می‌کند؟
\par 2 طمع می‌ورزید وندارید؛ می‌کشید و حسد می‌نمایید و نمی توانیدبه چنگ آرید؛ و جنگ و جدال می‌کنید و نداریداز این جهت که سوال نمی کنید.
\par 3 و سوال می‌کنیدو نمی یابید، از اینرو که به نیت بد سوال می‌کنید تادر لذات خود صرف نمایید.
\par 4 ‌ای زانیات، آیا نمی دانید که دوستی دنیا، دشمنی خداست؟ پس هرکه می‌خواهد دوست دنیا باشد، دشمن خدا گردد.
\par 5 آیا گمان دارید که کتاب عبث می‌گوید روحی که او را در ما ساکن کرده است، تا به غیرت بر ما اشتیاق دارد؟
\par 6 لیکن او فیض زیاده می‌بخشد. بنابراین می‌گوید: «خدامتکبران را مخالفت می‌کند، اما فروتنان را فیض می‌بخشد.»
\par 7 پس خدا را اطاعت نمایید و با ابلیس مقاومت کنید تا از شما بگریزد.
\par 8 و به خدا تقرب جویید تا به شما نزدیکی نماید. دستهای خود راطاهر سازید، ای گناهکاران و دلهای خود را پاک کنید، ای دودلان.
\par 9 خود را خوار سازید و ناله وگریه نمایید و خنده شما به ماتم و خوشی شما به غم مبدل شود.
\par 10 در حضور خدا فروتنی کنید تاشما را سرافراز فرماید.
\par 11 ‌ای برادران، یکدیگررا ناسزا مگویید زیرا هرکه برادر خود را ناسزاگوید و بر او حکم کند، شریعت را ناسزا گفته و برشریعت حکم کرده باشد. لکن اگر بر شریعت حکم کنی، عامل شریعت نیستی بلکه داورهستی.
\par 12 صاحب شریعت و داور، یکی است که بر رهانیدن و هلاک کردن قادر می‌باشد. پس تو کیستی که بر همسایه خود داوری می‌کنی؟
\par 13 هان، ای کسانی که می‌گویید: «امروز و فردابه فلان شهر خواهیم رفت و در آنجا یک سال بسرخواهیم برد و تجارت خواهیم کرد و نفع خواهیم برد»،
\par 14 و حال آنکه نمی دانید که فردا چه می‌شود؛ از آنرو که حیات شما چیست؟ مگربخاری نیستید که اندک زمانی ظاهر است و بعدناپدید می‌شود؟
\par 15 به عوض آنکه باید گفت که «اگر خدا بخواهد، زنده می‌مانیم و چنین و چنان می‌کنیم.»
\par 16 اما الحال به عجب خود فخرمی کنید و هر چنین فخر بد است.پس هرکه نیکویی‌کردن بداند و بعمل نیاورد، او را گناه است.
\par 17 پس هرکه نیکویی‌کردن بداند و بعمل نیاورد، او را گناه است.

\chapter{5}

\par 1 هان‌ای دولتمندان، بجهت مصیبتهایی که بر شما وارد می‌آید، گریه و ولوله نمایید.
\par 2 دولت شما فاسد و رخت شما بیدخورده می‌شود.
\par 3 طلا و نقره شما را زنگ می‌خورد وزنگ آنها بر شما شهادت خواهد داد و مثل آتش، گوشت شما را خواهد خورد. شما در زمان آخرخزانه اندوخته‌اید.
\par 4 اینک مزد عمله هایی که کشته های شما را درویده‌اند و شما آن را به فریب نگاه داشته‌اید، فریاد برمی آورد و ناله های دروگران، به گوشهای رب الجنود رسیده است.
\par 5 بر روی زمین به ناز و کامرانی مشغول بوده، دلهای خود را در یوم قتل پروردید.
\par 6 بر مردعادل فتوی دادید و او را به قتل رسانیدید و با شمامقاومت نمی کند.
\par 7 پس‌ای برادران، تا هنگام آمدن خداوند صبرکنید. اینک دهقان انتظار می‌کشد برای محصول گرانبهای زمین و برایش صبر می‌کند تا باران اولین و آخرین را بیابد.
\par 8 شما نیز صبر نمایید و دلهای خود را قوی سازید زیرا که آمدن خداوند نزدیک است.
\par 9 ‌ای برادران، از یکدیگر شکایت مکنید، مبادا بر شما حکم شود. اینک داور بر در ایستاده است.
\par 10 ‌ای برادران، نمونه زحمت و صبر رابگیرید از انبیایی که به نام خداوند تکلم نمودند.
\par 11 اینک صابران را خوشحال می‌گوییم و صبرایوب را شنیده‌اید و انجام کار خداوند رادانسته‌اید، زیرا که خداوند بغایت مهربان و کریم است.
\par 12 لکن اول همه‌ای برادران من، قسم مخورید نه به آسمان و نه به زمین و نه به هیچ سوگند دیگر، بلکه بلی شما بلی باشد و نی شما نی، مبادا درتحکم بیفتید.
\par 13 اگر کسی از شما مبتلای بلایی باشد، دعابنماید و اگر کسی خوشحال باشد، سرود بخواند.
\par 14 و هرگاه کسی از شما بیمار باشد، کشیشان کلیسا را طلب کند تا برایش دعا نمایند و او را به نام خداوند به روغن تدهین کنند.
\par 15 و دعای ایمان، مریض شما را شفا خواهد بخشید وخداوند او را خواهد برخیزانید، و اگر گناه کرده باشد، از او آمرزیده خواهد شد.
\par 16 نزد یکدیگربه گناهان خود اعتراف کنید و برای یکدیگر دعاکنید تا شفا یابید، زیرا دعای مرد عادل در عمل، قوت بسیار دارد.
\par 17 الیاس مردی بود صاحب حواس مثل ما و به تمامی دل دعا کرد که باران نبارد و تا مدت سه سال و شش ماه نبارید.
\par 18 و بازدعا کرد و آسمان بارید و زمین ثمر خود رارویانید.‌ای برادران من، اگر کسی از شما از راستی منحرف شود و شخصی او را بازگرداند،
\par 19 ‌ای برادران من، اگر کسی از شما از راستی منحرف شود و شخصی او را بازگرداند،



\end{document}