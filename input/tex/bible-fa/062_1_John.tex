\begin{document}

\title{1 John}


\chapter{1}

\par 1 آنچه از ابتدا بود و آنچه شنیده‌ایم و به چشم خود دیده، آنچه بر آن نگریستیم ودستهای ما لمس کرد، درباره کلمه حیات.
\par 2 وحیات ظاهر شد و آن را دیده‌ایم و شهادت می‌دهیم و به شما خبر می‌دهیم از حیات جاودانی که نزد پدر بود و برما ظاهر شد.
\par 3 ازآنچه دیده و شنیده‌ایم شما را اعلام می‌نماییم تاشما هم با ما شراکت داشته باشید. و اما شراکت مابا پدر و با پسرش عیسی مسیح است.
\par 4 و این را به شما می‌نویسم تا خوشی ما کامل گردد.
\par 5 و این است پیغامی که از او شنیده‌ایم و به شما اعلام می‌نماییم، که خدا نور است و هیچ ظلمت در وی هرگز نیست.
\par 6 اگر گوییم که با وی شراکت داریم، در حالیکه در ظلمت سلوک می‌نماییم، دروغ می‌گوییم و براستی عمل نمی کنیم.
\par 7 لکن اگر در نور سلوک می‌نماییم، چنانکه او در نور است، با یکدیگر شراکت داریم و خون پسر او عیسی مسیح ما را از هر گناه پاک می‌سازد.
\par 8 اگر گوییم که گناه نداریم خود را گمراه می‌کنیم و راستی در ما نیست.
\par 9 اگر به گناهان خود اعتراف کنیم، او امین و عادل است تا گناهان ما را بیامرزد و ما را از هر ناراستی پاک سازد.اگر گوییم که گناه نکرده‌ایم، او را دروغگو می شماریم و کلام او در ما نیست.
\par 10 اگر گوییم که گناه نکرده‌ایم، او را دروغگو می شماریم و کلام او در ما نیست.

\chapter{2}

\par 1 ای فرزندان من، این را به شما می‌نویسم تاگناه نکنید؛ و اگر کسی گناهی کند، شفیعی داریم نزد پدر یعنی عیسی مسیح عادل.
\par 2 واوست کفاره بجهت گناهان ما و نه گناهان ما فقطبلکه بجهت تمام جهان نیز.
\par 3 و از این می‌دانیم که او را می‌شناسیم، اگر احکام او را نگاه داریم.
\par 4 کسی‌که گوید او را می‌شناسم و احکام او رانگاه ندارد، دروغگوست و در وی راستی نیست.
\par 5 لکن کسی‌که کلام او را نگاه دارد، فی الواقع محبت خدا در وی کامل شده است و از این می‌دانیم که در وی هستیم.
\par 6 هرکه گوید که در وی می‌مانم، به همین طریقی که او سلوک می‌نمود، اونیز باید سلوک کند.
\par 7 ‌ای حبیبان، حکمی تازه به شما نمی نویسم، بلکه حکمی کهنه که آن را از ابتدا داشتید؛ و حکم کهنه آن کلام است که از ابتدا شنیدید.
\par 8 و نیزحکمی تازه به شما می‌نویسم که آن در وی و درشما حق است، زیرا که تاریکی درگذر است و نورحقیقی الان می‌درخشد.
\par 9 کسی‌که می‌گوید که در نور است و از برادرخود نفرت دارد، تا حال در تاریکی است.
\par 10 وکسی‌که برادر خود را محبت نماید، در نور ساکن است و لغزش در وی نیست.
\par 11 اما کسی‌که ازبرادر خود نفرت دارد، در تاریکی است و درتاریکی راه می‌رود و نمی داند کجا می‌رود زیرا که تاریکی چشمانش را کور کرده است.
\par 12 ‌ای فرزندان، به شما می‌نویسم زیرا که گناهان شما بخاطر اسم او آمرزیده شده است.
\par 13 ‌ای پدران، به شما می‌نویسم زیرا او را که ازابتدا است می‌شناسید. ای جوانان، به شمامی نویسم از آنجا که بر شریر غالب شده‌اید. ای بچه‌ها به شما نوشتم زیرا که پدر را می‌شناسید.
\par 14 ‌ای پدران، به شما نوشتم زیرا او را که ازابتداست می‌شناسید. ای جوانان، به شما نوشتم از آن جهت که توانا هستید و کلام خدا در شماساکن است وبر شریر غلبه یافته‌اید.
\par 15 دنیا را وآنچه در دنیاست دوست مدارید زیرا اگر کسی دنیا را دوست دارد، محبت پدر در وی نیست.
\par 16 زیرا که آنچه در دنیاست، از شهوت جسم وخواهش چشم و غرور زندگانی از پدر نیست بلکه از جهان است.
\par 17 و دنیا و شهوات آن در گذراست لکن کسی‌که به اراده خدا عمل می‌کند، تا به ابد باقی می‌ماند.
\par 18 ‌ای بچه‌ها، این ساعت آخر است و چنانکه شنیده‌اید که دجال می‌آید، الحال هم دجالان بسیار ظاهر شده‌اند و از این می‌دانیم که ساعت آخر است.
\par 19 از ما بیرون شدند، لکن از ما نبودند، زیرا اگر از ما می‌بودند با ما می‌ماندند؛ لکن بیرون رفتند تا ظاهر شود که همه ایشان از ما نیستند.
\par 20 و اما شما از آن قدوس، مسح را یافته‌اید وهرچیز را می‌دانید.
\par 21 ننوشتم به شما از این جهت که راستی را نمی دانید، بلکه از اینرو که آن رامی دانید و اینکه هیچ دروغ از راستی نیست.
\par 22 دروغگو کیست جز آنکه مسیح بودن عیسی راانکار کند. آن دجال است که پدر و پسر را انکارمی نماید.
\par 23 کسی‌که پسر را انکار کند، پدر را هم ندارد و کسی‌که اعتراف به پسر نماید، پدر را نیزدارد.
\par 24 و اما شما آنچه از ابتدا شنیدید در شماثابت بماند، زیرا اگر آنچه از اول شنیدید، در شماثابت بماند، شما نیز در پسر و در پدر ثابت خواهید ماند.
\par 25 و این است آن وعده‌ای که او به ما داده است، یعنی حیات جاودانی.
\par 26 و این را به شما نوشتم درباره آنانی که شما را گمراه می‌کنند.
\par 27 و اما در شما آن مسح که از او یافته‌اید ثابت است و حاجت ندارید که کسی شما را تعلیم دهدبلکه چنانکه خود آن مسح شما را از همه‌چیزتعلیم می‌دهد و حق است و دروغ نیست، پس بطوری که شما را تعلیم داد در او ثابت می‌مانید.
\par 28 الان‌ای فرزندان در او ثابت بمانید تا چون ظاهر شود، اعتماد داشته باشیم و در هنگام ظهورش از وی خجل نشویم.اگر فهمیده ایدکه او عادل است، پس می‌دانید که هر‌که عدالت رابه‌جا آورد، از وی تولد یافته است.
\par 29 اگر فهمیده ایدکه او عادل است، پس می‌دانید که هر‌که عدالت رابه‌جا آورد، از وی تولد یافته است.

\chapter{3}

\par 1 ملاحظه کنید چه نوع محبت پدر به ما داده است تا فرزندان خدا خوانده شویم؛ وچنین هستیم و از این جهت دنیا ما را نمی شناسدزیرا که او را نشناخت.
\par 2 ‌ای حبیبان، الان فرزندان خدا هستیم و هنوز ظاهر نشده است آنچه خواهیم بود؛ لکن می‌دانیم که چون او ظاهر شود، مانند او خواهیم بود زیرا او را چنانکه هست خواهیم دید.
\par 3 و هرکس که این امید را بر وی دارد، خود را پاک می‌سازد چنانکه او پاک است.
\par 4 و هرکه گناه را بعمل می‌آورد، برخلاف شریعت عمل می‌کند زیرا گناه مخالف شریعت است.
\par 5 و می‌دانید که او ظاهر شد تا گناهان رابردارد و در وی هیچ گناه نیست.
\par 6 هرکه در وی ثابت است گناه نمی کند و هرکه گناه می‌کند او راندیده است و نمی شناسد.
\par 7 ‌ای فرزندان، کسی شما را گمراه نکند؛ کسی‌که عدالت را به‌جا می‌آورد، عادل است چنانکه اوعادل است.
\par 8 و کسی‌که گناه می‌کند از ابلیس است زیرا که ابلیس از ابتدا گناهکار بوده است. واز این جهت پسر خدا ظاهر شد تا اعمال ابلیس راباطل سازد.
\par 9 هر‌که از خدا مولود شده است، گناه نمی کند زیرا تخم او در وی می‌ماند و اونمی تواند گناهکار بوده باشد زیرا که از خدا تولدیافته است.
\par 10 فرزندان خدا و فرزندان ابلیس ازاین ظاهر می‌گردند. هر‌که عدالت را به‌جانمی آورد از خدا نیست و همچنین هر‌که برادرخود را محبت نمی نماید.
\par 11 زیرا همین است آن پیغامی که از اول شنیدید که یکدیگر را محبت نماییم.
\par 12 نه مثل قائن که از آن شریر بود و برادر خود را کشت؛ و ازچه سبب او را کشت؟ از این سبب که اعمال خودش قبیح بود و اعمال برادرش نیکو.
\par 13 ‌ای برادران من، تعجب مکنید اگر دنیا از شما نفرت گیرد.
\par 14 ما می‌دانیم که از موت گذشته، داخل حیات گشته‌ایم از اینکه برادران را محبت می نماییم. هرکه برادر خود را محبت نمی نمایددر موت ساکن است.
\par 15 هر‌که از برادر خود نفرت نماید، قاتل است و می‌دانید که هیچ قاتل حیات جاودانی در خود ثابت ندارد.
\par 16 از این امر محبت را دانسته‌ایم که او جان خود را در راه ما نهاد و ما باید جان خود را در راه برادران بنهیم.
\par 17 لکن کسی‌که معیشت دنیوی دارد و برادر خود را محتاج بیند و رحمت خود رااز او باز‌دارد، چگونه محبت خدا در او ساکن است؟
\par 18 ‌ای فرزندان، محبت را به‌جا آریم نه درکلام و زبان بلکه در عمل و راستی.
\par 19 و از این خواهیم دانست که از حق هستیم و دلهای خود رادر حضور او مطمئن خواهیم ساخت،
\par 20 یعنی درهرچه دل ما، ما را مذمت می‌کند، زیرا خدا از دل ما بزرگتر است و هرچیز را می‌داند.
\par 21 ‌ای حبیبان، هرگاه دل ما ما را مذمت نکند، در حضور خدا اعتماد داریم.
\par 22 و هرچه سوآل کنیم، از او می‌پاییم، از آنجهت که احکام او رانگاه می‌داریم و به آنچه پسندیده اوست، عمل می‌نماییم. و این است حکم او که به اسم پسر اوعیسی مسیح ایمان آوریم و یکدیگر را محبت نماییم، چنانکه به ما امر فرمود.و هرکه احکام او را نگاه دارد، در او ساکن است و او در وی؛ و ازاین می‌شناسیم که در ما ساکن است، یعنی از آن روح که به ما داده است.
\par 23 و هرکه احکام او را نگاه دارد، در او ساکن است و او در وی؛ و ازاین می‌شناسیم که در ما ساکن است، یعنی از آن روح که به ما داده است.

\chapter{4}

\par 1 ای حبیبان، هر روح را قبول مکنید بلکه روح‌ها را بیازمایید که از خدا هستند یا نه. زیرا که انبیای کذبه بسیار به جهان بیرون رفته‌اند.
\par 2 به این، روح خدا را می‌شناسیم: هر روحی که به عیسی مسیح مجسم شده اقرار نماید از خداست،
\par 3 و هر روحی که عیسی مسیح مجسم شده راانکار کند، از خدا نیست. و این است روح دجال که شنیده‌اید که او می‌آید و الان هم در جهان است.
\par 4 ‌ای فرزندان، شما از خدا هستید و بر ایشان غلبه یافته‌اید زیرا او که در شماست، بزرگتر است از آنکه در جهان است.
\par 5 ایشان از دنیا هستند ازاین جهت سخنان دنیوی می‌گویند و دنیا ایشان رامی شنود.
\par 6 ما از خدا هستیم و هرکه خدا رامی شناسد ما را می‌شنود و آنکه از خدا نیست مارا نمی شنود. روح حق و روح ضلالت را از این تمییز می‌دهیم.
\par 7 ‌ای حبیبان، یکدیگر را محبت بنماییم زیراکه محبت از خداست و هرکه محبت می‌نماید ازخدا مولود شده است و خدا را می‌شناسد،
\par 8 وکسی‌که محبت نمی نماید، خدا را نمی شناسدزیرا خدا محبت است.
\par 9 و محبت خدا به ما ظاهرشده است به اینکه خدا پسر یگانه خود را به جهان فرستاده است تا به وی زیست نماییم.
\par 10 ومحبت در همین است، نه آنکه ما خدا را محبت نمودیم، بلکه اینکه او ما را محبت نمود و پسرخود را فرستاد تاکفاره گناهان ما شود.
\par 11 ‌ای حبیبان، اگر خدا با ما چنین محبت نمود، ما نیزمی باید یکدیگر را محبت نماییم.
\par 12 کسی هرگزخدا را ندید؛ اگر یکدیگر را محبت نماییم، خدادر ما ساکن است و محبت او در ما کامل شده است.
\par 13 از این می‌دانیم که در وی ساکنیم و او در مازیرا که از روح خود به ما داده است.
\par 14 و مادیده‌ایم و شهادت می‌دهیم که پدر پسر را فرستادتا نجات‌دهنده جهان بشود.
\par 15 هرکه اقرار می‌کندکه عیسی پسر خداست، خدا در وی ساکن است واو در خدا.
\par 16 و ما دانسته و باور کرده‌ایم آن محبتی را که خدا با ما نموده است. خدا محبت است و هرکه در محبت ساکن است در خدا ساکن است و خدا در وی.
\par 17 محبت در همین با ما کامل شده است تا درروز جزا ما را دلاوری باشد، زیرا چنانکه اوهست، ما نیز در این جهان همچنین هستیم.
\par 18 درمحبت خوف نیست بلکه محبت کامل خوف رابیرون می‌اندازد؛ زیرا خوف عذاب دارد و کسی‌که خوف دارد، در محبت کامل نشده است.
\par 19 مااو را محبت می‌نماییم زیرا که او اول ما را محبت نمود.
\par 20 اگر کسی گوید که خدا را محبت می‌نمایم و از برادر خود نفرت کند، دروغگوست، زیرا کسی‌که برادری را که دیده است محبت ننماید، چگونه ممکن است خدایی را که ندیده است محبت نماید؟و این حکم رااز وی یافته‌ایم که هرکه خدا را محبت می‌نماید، برادر خود را نیز محبت بنماید.
\par 21 و این حکم رااز وی یافته‌ایم که هرکه خدا را محبت می‌نماید، برادر خود را نیز محبت بنماید.

\chapter{5}

\par 1 هرکه ایمان دارد که عیسی، مسیح است، ازخدا مولود شده است؛ و هرکه والد رامحبت می‌نماید، مولود او را نیز محبت می‌نماید.
\par 2 از این می‌دانیم که فرزندان خدا را محبت می نماییم، چون خدا را محبت می‌نماییم واحکام او را به‌جا می‌آوریم.
\par 3 زیرا همین است محبت خدا که احکام او را نگاه داریم و احکام اوگران نیست.
\par 4 زیرا آنچه از خدا مولود شده است، بر دنیا غلبه می‌یابد؛ و غلبه‌ای که دنیا را مغلوب ساخته است، ایمان ماست.
\par 5 کیست آنکه بر دنیاغلبه یابد؟ جز آنکه ایمان دارد که عیسی پسرخداست.
\par 6 همین است او که به آب و خون آمد یعنی عیسی مسیح. نه به آب فقط بلکه به آب و خون. وروح است آنکه شهادت می‌دهد، زیرا که روح حق است.
\par 7 زیرا سه هستند که شهادت می‌دهند،
\par 8 یعنی روح و آب و خون؛ و این سه یک هستند.
\par 9 اگر شهادت انسان را قبول کنیم، شهادت خدابزرگتر است؛ زیرا این است شهادت خدا که درباره پسر خود شهادت داده است.
\par 10 آنکه به پسر خدا ایمان آورد، در خود شهادت دارد وآنکه به خدا ایمان نیاورد، او را دروغگو شمرده است، زیرا به شهادتی که خدا درباره پسر خودداده است، ایمان نیاورده است.
\par 11 و آن شهادت این است که خدا حیات جاودانی به ما داده است واین حیات، در پسر اوست.
\par 12 آنکه پسر را داردحیات را دارد و آنکه پسر خدا را ندارد، حیات رانیافته است.
\par 13 این را نوشتم به شما که به اسم خدا ایمان آورده‌اید تا بدانید که حیات جاودانی دارید و تا به اسم پسر خدا ایمان بیاورید.
\par 14 و این است آن دلیری که نزد وی داریم که هرچه برحسب اراده او سوال نماییم، ما را می‌شنود.
\par 15 و اگر دانیم که هرچه سوال کنیم ما را می‌شنود، پس می‌دانیم که آنچه از او درخواست کنیم می‌یابیم.
\par 16 اگر کسی برادر خود را بیند که گناهی را که منتهی به موت نباشد می‌کند، دعا بکند و او را حیات خواهدبخشید، به هرکه گناهی منتهی به موت نکرده باشد. گناهی منتهی به موت هست؛ بجهت آن نمی گویم که دعا باید کرد.
\par 17 هر ناراستی گناه است، ولی گناهی هست که منتهی به موت نیست.
\par 18 و می‌دانیم که هرکه از خدا مولود شده است، گناه نمی کند بلکه کسی‌که از خدا تولد یافت خودرا نگاه می‌دارد و آن شریر او را لمس نمی کند.
\par 19 و می‌دانیم که از خدا هستیم و تمام دنیا درشریر خوابیده است.اما آگاه هستیم که پسرخدا آمده است و به ما بصیرت داده است تا حق رابشناسیم و در حق یعنی در پسر او عیسی مسیح هستیم. اوست خدای حق و حیات جاودانی.
\par 20 اما آگاه هستیم که پسرخدا آمده است و به ما بصیرت داده است تا حق رابشناسیم و در حق یعنی در پسر او عیسی مسیح هستیم. اوست خدای حق و حیات جاودانی.



\end{document}