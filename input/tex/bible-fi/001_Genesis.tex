\begin{document}

\title{Ensimmäinen Mooseksen kirja}


\chapter{1}

\par 1 Alussa loi Jumala taivaan ja maan.
\par 2 Ja maa oli autio ja tyhjä, ja pimeys oli syvyyden päällä, ja Jumalan Henki liikkui vetten päällä.
\par 3 Ja Jumala sanoi: "Tulkoon valkeus". Ja valkeus tuli.
\par 4 Ja Jumala näki, että valkeus oli hyvä; ja Jumala erotti valkeuden pimeydestä.
\par 5 Ja Jumala kutsui valkeuden päiväksi, ja pimeyden hän kutsui yöksi. Ja tuli ehtoo, ja tuli aamu, ensimmäinen päivä.
\par 6 Ja Jumala sanoi: "Tulkoon taivaanvahvuus vetten välille erottamaan vedet vesistä".
\par 7 Ja Jumala teki taivaanvahvuuden ja erotti vedet, jotka olivat taivaanvahvuuden alla, vesistä, jotka olivat taivaanvahvuuden päällä; ja tapahtui niin.
\par 8 Ja Jumala kutsui vahvuuden taivaaksi. Ja tuli ehtoo, ja tuli aamu, toinen päivä.
\par 9 Ja Jumala sanoi: "Kokoontukoot vedet, jotka ovat taivaan alla, yhteen paikkaan, niin että kuiva tulee näkyviin". Ja tapahtui niin.
\par 10 Ja Jumala kutsui kuivan maaksi, ja paikan, mihin vedet olivat kokoontuneet, hän kutsui mereksi. Ja Jumala näki, että se oli hyvä.
\par 11 Ja Jumala sanoi: "Kasvakoon maa vihantaa, ruohoja, jotka tekevät siementä, ja hedelmäpuita, jotka lajiensa mukaan kantavat maan päällä hedelmää, jossa niiden siemen on". Ja tapahtui niin:
\par 12 maa tuotti vihantaa, ruohoja, jotka tekivät siementä lajiensa mukaan, ja puita, jotka lajiensa mukaan kantoivat hedelmää, jossa niiden siemen oli. Ja Jumala näki, että se oli hyvä.
\par 13 Ja tuli ehtoo, ja tuli aamu, kolmas päivä.
\par 14 Ja Jumala sanoi: "Tulkoot valot taivaanvahvuuteen erottamaan päivää yöstä, ja olkoot ne merkkeinä osoittamassa aikoja, päiviä ja vuosia,
\par 15 ja olkoot valoina taivaanvahvuudella paistamassa maan päälle". Ja tapahtui niin:
\par 16 Jumala teki kaksi suurta valoa, suuremman valon hallitsemaan päivää ja pienemmän valon hallitsemaan yötä, sekä tähdet.
\par 17 Ja Jumala pani ne taivaanvahvuuteen, paistamaan maan päälle
\par 18 ja hallitsemaan päivää ja yötä ja erottamaan valon pimeästä. Ja Jumala näki, että se oli hyvä.
\par 19 Ja tuli ehtoo, ja tuli aamu, neljäs päivä.
\par 20 Ja Jumala sanoi: "Viliskööt vedet eläviä olentoja, ja lentäkööt linnut maan päällä, taivaanvahvuuden alla".
\par 21 Ja Jumala loi suuret merieläimet ja kaikkinaiset liikkuvat, vesissä vilisevät elävät olennot, kunkin lajinsa mukaan, ja kaikkinaiset siivekkäät linnut, kunkin lajinsa mukaan. Ja Jumala näki, että se oli hyvä.
\par 22 Ja Jumala siunasi ne sanoen: "Olkaa hedelmälliset ja lisääntykää ja täyttäkää meren vedet, ja linnut lisääntykööt maan päällä".
\par 23 Ja tuli ehtoo, ja tuli aamu, viides päivä.
\par 24 Ja Jumala sanoi: "Tuottakoon maa elävät olennot, kunkin lajinsa mukaan, karjaeläimet ja matelijat ja metsäeläimet, kunkin lajinsa mukaan". Ja tapahtui niin:
\par 25 Jumala teki metsäeläimet, kunkin lajinsa mukaan, ja karjaeläimet, kunkin lajinsa mukaan, ja kaikki maan matelijat, kunkin lajinsa mukaan. Ja Jumala näki, että se oli hyvä.
\par 26 Ja Jumala sanoi: "Tehkäämme ihminen kuvaksemme, kaltaiseksemme; ja vallitkoot he meren kalat ja taivaan linnut ja karjaeläimet ja koko maan ja kaikki matelijat, jotka maassa matelevat".
\par 27 Ja Jumala loi ihmisen omaksi kuvaksensa, Jumalan kuvaksi hän hänet loi; mieheksi ja naiseksi hän loi heidät.
\par 28 Ja Jumala siunasi heidät, ja Jumala sanoi heille: "Olkaa hedelmälliset ja lisääntykää ja täyttäkää maa ja tehkää se itsellenne alamaiseksi; ja vallitkaa meren kalat ja taivaan linnut ja kaikki maan päällä liikkuvat eläimet".
\par 29 Ja Jumala sanoi: "Katso, minä annan teille kaikkinaiset siementä tekevät ruohot, joita kasvaa kaikkialla maan päällä, ja kaikki puut, joissa on siementä tekevä hedelmä; olkoot ne teille ravinnoksi.
\par 30 Ja kaikille metsäeläimille ja kaikille taivaan linnuille ja kaikille, jotka maassa matelevat ja joissa on elävä henki, minä annan kaikkinaiset viheriät ruohot ravinnoksi". Ja tapahtui niin.
\par 31 Ja Jumala katsoi kaikkea, mitä hän tehnyt oli, ja katso, se oli sangen hyvää. Ja tuli ehtoo, ja tuli aamu, kuudes päivä.

\chapter{2}

\par 1 Niin tulivat valmiiksi taivas ja maa kaikkine joukkoinensa.
\par 2 Ja Jumala päätti seitsemäntenä päivänä työnsä, jonka hän oli tehnyt, ja lepäsi seitsemäntenä päivänä kaikesta työstänsä, jonka hän oli tehnyt.
\par 3 Ja Jumala siunasi seitsemännen päivän ja pyhitti sen, koska hän sinä päivänä lepäsi kaikesta luomistyöstänsä, jonka hän oli tehnyt.
\par 4 Tämä on kertomus taivaan ja maan synnystä, kun ne luotiin. Siihen aikaan kun Herra Jumala teki maan ja taivaan,
\par 5 ei ollut vielä yhtään kedon pensasta maan päällä, eikä vielä kasvanut mitään ruohoa kedolla, koska Herra Jumala ei vielä ollut antanut sataa maan päälle eikä ollut ihmistä maata viljelemässä,
\par 6 vaan sumu nousi maasta ja kasteli koko maan pinnan.
\par 7 Silloin Herra Jumala teki maan tomusta ihmisen ja puhalsi hänen sieramiinsa elämän hengen, ja niin ihmisestä tuli elävä sielu.
\par 8 Ja Herra Jumala istutti paratiisin Eedeniin, itään, ja asetti sinne ihmisen, jonka hän oli tehnyt.
\par 9 Ja Herra Jumala kasvatti maasta kaikkinaisia puita, ihania nähdä ja hyviä syödä, ja elämän puun keskelle paratiisia, niin myös hyvän- ja pahantiedon puun.
\par 10 Ja Eedenistä lähti joki, joka kasteli paratiisia, ja se jakaantui sieltä neljään haaraan.
\par 11 Ensimmäisen nimi on Piison; se kiertää koko Havilan maan, jossa on kultaa;
\par 12 ja sen maan kulta on hyvää. Siellä on myös bedellion-pihkaa ja onyks-kiveä.
\par 13 Toisen virran nimi on Giihon; se kiertää koko Kuusin maan.
\par 14 Kolmannen virran nimi on Hiddekel; se juoksee Assurin editse. Ja neljäs virta on Eufrat.
\par 15 Ja Herra Jumala otti ihmisen ja pani hänet Eedenin paratiisiin viljelemään ja varjelemaan sitä.
\par 16 Ja Herra Jumala käski ihmistä sanoen: "Syö vapaasti kaikista muista paratiisin puista,
\par 17 mutta hyvän- ja pahantiedon puusta älä syö, sillä sinä päivänä, jona sinä siitä syöt, pitää sinun kuolemalla kuoleman".
\par 18 Ja Herra Jumala sanoi: "Ei ole ihmisen hyvä olla yksinänsä, minä teen hänelle avun, joka on hänelle sopiva".
\par 19 Ja Herra Jumala teki maasta kaikki metsän eläimet ja kaikki taivaan linnut ja toi ne ihmisen eteen nähdäkseen, kuinka hän ne nimittäisi; ja niinkuin ihminen nimitti kunkin elävän olennon, niin oli sen nimi oleva.
\par 20 Ja ihminen antoi nimet kaikille karjaeläimille ja taivaan linnuille ja kaikille metsän eläimille. Mutta Aadamille ei löytynyt apua, joka olisi hänelle sopinut.
\par 21 Niin Herra Jumala vaivutti ihmisen raskaaseen uneen, ja kun hän nukkui, otti hän yhden hänen kylkiluistaan ja täytti sen paikan lihalla.
\par 22 Ja Herra Jumala rakensi vaimon siitä kylkiluusta, jonka hän oli ottanut miehestä, ja toi hänet miehen luo.
\par 23 Ja mies sanoi: "Tämä on nyt luu minun luistani ja liha minun lihastani; hän kutsuttakoon miehettäreksi, sillä hän on miehestä otettu".
\par 24 Sentähden mies luopukoon isästänsä ja äidistänsä ja liittyköön vaimoonsa, ja he tulevat yhdeksi lihaksi.
\par 25 Ja he olivat molemmat, mies ja hänen vaimonsa, alasti eivätkä hävenneet toisiansa.

\chapter{3}

\par 1 Mutta käärme oli kavalin kaikista kedon eläimistä, jotka Herra Jumala oli tehnyt; ja se sanoi vaimolle: "Onko Jumala todellakin sanonut: 'Älkää syökö kaikista paratiisin puista'?"
\par 2 Niin vaimo vastasi käärmeelle: "Me saamme syödä muiden puiden hedelmiä paratiisissa,
\par 3 mutta sen puun hedelmästä, joka on keskellä paratiisia, on Jumala sanonut: 'Älkää syökö siitä älkääkä koskeko siihen, ettette kuolisi'."
\par 4 Niin käärme sanoi vaimolle: "Ette suinkaan kuole;
\par 5 vaan Jumala tietää, että sinä päivänä, jona te siitä syötte, aukenevat teidän silmänne, ja te tulette niinkuin Jumala tietämään hyvän ja pahan".
\par 6 Ja vaimo näki, että siitä puusta oli hyvä syödä ja että se oli ihana katsella ja suloinen puu antamaan ymmärrystä; ja hän otti sen hedelmästä ja söi ja antoi myös miehellensä, joka oli hänen kanssansa, ja hänkin söi.
\par 7 Silloin aukenivat heidän molempain silmät, ja he huomasivat olevansa alasti; ja he sitoivat yhteen viikunapuun lehtiä ja tekivät itselleen vyöverhot.
\par 8 Ja he kuulivat, kuinka Herra Jumala käyskenteli paratiisissa illan viileydessä. Ja mies vaimoineen lymysi Herran Jumalan kasvojen edestä paratiisin puiden sekaan.
\par 9 Mutta Herra Jumala huusi miestä ja sanoi hänelle: "Missä olet?"
\par 10 Hän vastasi: "Minä kuulin sinun askeleesi paratiisissa ja pelkäsin, sillä minä olen alasti, ja sentähden minä lymysin".
\par 11 Ja hän sanoi: "Kuka sinulle ilmoitti, että olet alasti? Etkö syönyt siitä puusta, josta minä kielsin sinua syömästä?"
\par 12 Mies vastasi: "Vaimo, jonka annoit olemaan minun kanssani, antoi minulle siitä puusta, ja minä söin".
\par 13 Niin Herra Jumala sanoi vaimolle: "Mitäs olet tehnyt?" Vaimo vastasi: "Käärme petti minut, ja minä söin".
\par 14 Ja Herra Jumala sanoi käärmeelle: "Koska tämän teit, kirottu ole sinä kaikkien karjaeläinten ja kaikkien metsän eläinten joukossa. Vatsallasi sinun pitää käymän ja tomua syömän koko elinaikasi.
\par 15 Ja minä panen vainon sinun ja vaimon välille ja sinun siemenesi ja hänen siemenensä välille; se on polkeva rikki sinun pääsi, ja sinä olet pistävä sitä kantapäähän."
\par 16 Ja vaimolle hän sanoi: "Minä teen suuriksi sinun raskautesi vaivat, kivulla sinun pitää synnyttämän lapsia; mutta mieheesi on sinun halusi oleva, ja hän on sinua vallitseva".
\par 17 Ja Aadamille hän sanoi: "Koska kuulit vaimoasi ja söit puusta, josta minä kielsin sinua sanoen: 'Älä syö siitä', niin kirottu olkoon maa sinun tähtesi. Vaivaa nähden sinun pitää elättämän itseäsi siitä koko elinaikasi;
\par 18 orjantappuroita ja ohdakkeita se on kasvava sinulle, ja kedon ruohoja sinun on syötävä.
\par 19 Otsasi hiessä sinun pitää syömän leipäsi, kunnes tulet maaksi jälleen, sillä siitä sinä olet otettu. Sillä maasta sinä olet, ja maaksi pitää sinun jälleen tuleman."
\par 20 Ja mies antoi vaimolleen nimen Eeva, sillä hänestä tuli kaiken elävän äiti.
\par 21 Ja Herra Jumala teki Aadamille ja hänen vaimollensa puvut nahasta ja puki ne heidän yllensä.
\par 22 Ja Herra Jumala sanoi: "Katso, ihminen on tullut sellaiseksi kuin joku meistä, niin että hän tietää hyvän ja pahan. Kun ei hän nyt vain ojentaisi kättänsä ja ottaisi myös elämän puusta ja söisi ja eläisi iankaikkisesti!"
\par 23 Niin Herra Jumala ajoi hänet pois Eedenin paratiisista viljelemään maata, josta hän oli otettu.
\par 24 Ja hän karkoitti ihmisen ja asetti Eedenin paratiisin itäpuolelle kerubit ynnä välkkyvän, leimuavan miekan vartioitsemaan elämän puun tietä.

\chapter{4}

\par 1 Ja mies yhtyi vaimoonsa Eevaan; ja tämä tuli raskaaksi ja synnytti Kainin ja sanoi: "Minä olen saanut pojan Herran avulla".
\par 2 Ja taas hän synnytti pojan, veljen Kainille, Aabelin. Ja Aabelista tuli lampuri, mutta Kainista peltomies.
\par 3 Ja jonkun ajan kuluttua tapahtui, että Kain toi maan hedelmistä uhrilahjan Herralle;
\par 4 ja myöskin Aabel toi uhrilahjan laumansa esikoisista, niiden rasvoista. Ja Herra katsoi Aabelin ja hänen uhrilahjansa puoleen;
\par 5 mutta Kainin ja hänen uhrilahjansa puoleen hän ei katsonut. Silloin Kain vihastui kovin, ja hänen hahmonsa synkistyi.
\par 6 Ja Herra sanoi Kainille: "Miksi olet vihastunut, ja miksi hahmosi synkistyy?
\par 7 Eikö niin: jos teet hyvin, voit kohottaa katseesi; mutta jos et hyvin tee, niin väijyy synti ovella, ja sen halu on sinuun, mutta hallitse sinä sitä!"
\par 8 Ja Kain sanoi veljellensä Aabelille: "Menkäämme kedolle". Ja heidän kedolla ollessansa Kain karkasi veljensä Aabelin kimppuun ja tappoi hänet.
\par 9 Niin Herra sanoi Kainille: "Missä on veljesi Aabel?" Hän vastasi: "En tiedä; olenko minä veljeni vartija?"
\par 10 Ja hän sanoi: "Mitä olet tehnyt? Kuule, veljesi veri huutaa minulle maasta.
\par 11 Ja nyt ole kirottu ja karkoitettu pois tältä vainiolta, joka avasi suunsa ottamaan veljesi veren sinun kädestäsi.
\par 12 Kun maata viljelet, ei se ole enää sinulle satoansa antava; kulkija ja pakolainen pitää sinun oleman maan päällä."
\par 13 Ja Kain sanoi Herralle: "Syyllisyyteni on suurempi, kuin että sen kantaa voisin.
\par 14 Katso, sinä karkoitat minut nyt pois vainiolta, ja minun täytyy lymytä sinun kasvojesi edestä ja olla kulkija ja pakolainen maan päällä; ja kuka ikinä minut kohtaa, se tappaa minut."
\par 15 Mutta Herra sanoi hänelle: "Sentähden, kuka ikinä tappaa Kainin, hänelle se pitää seitsenkertaisesti kostettaman". Ja Herra pani Kainiin merkin, ettei kukaan, joka hänet kohtaisi, tappaisi häntä.
\par 16 Niin Kain poistui Herran kasvojen edestä ja asettui asumaan Noodin maahan, itään päin Eedenistä.
\par 17 Ja Kain yhtyi vaimoonsa, ja tämä tuli raskaaksi ja synnytti Hanokin. Ja hän rakensi kaupungin ja antoi sille kaupungille poikansa nimen Hanok.
\par 18 Ja Hanokille syntyi Iirad, Iiradille syntyi Mehujael, Mehujaelille syntyi Metusael, Metusaelille syntyi Lemek.
\par 19 Lemek otti itselleen kaksi vaimoa, toisen nimi oli Aada ja toisen nimi Silla.
\par 20 Ja Aada synnytti Jaabalin; hänestä tuli niiden kantaisä, jotka teltoissa asuvat ja karjanhoitoa harjoittavat.
\par 21 Ja hänen veljensä nimi oli Juubal; hänestä tuli kaikkien niiden kantaisä, jotka kannelta ja huilua soittavat.
\par 22 Myöskin Silla synnytti pojan, Tuubal-Kainin; hänestä tuli kaikkinaisten vaski- ja rauta-aseiden takoja. Ja Tuubal-Kainin sisar oli Naema.
\par 23 Ja Lemek lausui vaimoillensa: "Aada ja Silla, kuulkaa puhettani, te Lemekin emännät, ottakaa sanani korviinne: minä surmaan miehen haavastani ja nuorukaisen mustelmastani.
\par 24 Niin, Kain kostetaan seitsenkertaisesti, mutta Lemek seitsemänkymmentä seitsemän kertaa."
\par 25 Ja Aadam yhtyi taas vaimoonsa, ja tämä synnytti pojan ja antoi hänelle nimen Seet, sanoen: "Jumala on suonut minulle toisen pojan Aabelin sijaan, koska Kain hänet surmasi".
\par 26 Ja myöskin Seetille syntyi poika, ja hän antoi hänelle nimen Enos. Siihen aikaan ruvettiin avuksi huutamaan Herran nimeä.

\chapter{5}

\par 1 Tämä on Aadamin sukuluettelo. Kun Jumala loi ihmisen, teki hän hänet Jumalan kaltaiseksi.
\par 2 Mieheksi ja naiseksi hän heidät loi ja siunasi heidät ja antoi heille nimen ihminen, silloin kun heidät luotiin.
\par 3 Kun Aadam oli sadan kolmenkymmenen vuoden vanha, syntyi hänelle poika, joka oli hänen kaltaisensa, hänen kuvansa, ja hän antoi hänelle nimen Seet.
\par 4 Ja Aadam eli Seetin syntymän jälkeen kahdeksansataa vuotta, ja hänelle syntyi poikia ja tyttäriä.
\par 5 Niin oli Aadamin koko elinaika yhdeksänsataa kolmekymmentä vuotta; sitten hän kuoli.
\par 6 Kun Seet oli sadan viiden vuoden vanha, syntyi hänelle Enos.
\par 7 Ja Seet eli Enoksen syntymän jälkeen kahdeksansataa seitsemän vuotta, ja hänelle syntyi poikia ja tyttäriä.
\par 8 Niin oli Seetin koko elinaika yhdeksänsataa kaksitoista vuotta; sitten hän kuoli.
\par 9 Kun Enos oli yhdeksänkymmenen vuoden vanha, syntyi hänelle Keenan.
\par 10 Ja Enos eli Keenanin syntymän jälkeen kahdeksansataa viisitoista vuotta, ja hänelle syntyi poikia ja tyttäriä.
\par 11 Niin oli Enoksen koko elinaika yhdeksänsataa viisi vuotta; sitten hän kuoli.
\par 12 Kun Keenan oli seitsemänkymmenen vuoden vanha, syntyi hänelle Mahalalel.
\par 13 Ja Keenan eli Mahalalelin syntymän jälkeen kahdeksansataa neljäkymmentä vuotta, ja hänelle syntyi poikia ja tyttäriä.
\par 14 Niin oli Keenanin koko elinaika yhdeksänsataa kymmenen vuotta; sitten hän kuoli.
\par 15 Kun Mahalalel oli kuudenkymmenen viiden vuoden vanha, syntyi hänelle Jered.
\par 16 Ja Mahalalel eli Jeredin syntymän jälkeen kahdeksansataa kolmekymmentä vuotta, ja hänelle syntyi poikia ja tyttäriä.
\par 17 Niin oli Mahalalelin koko elinaika kahdeksansataa yhdeksänkymmentä viisi vuotta; sitten hän kuoli.
\par 18 Kun Jered oli sadan kuudenkymmenen kahden vuoden vanha, syntyi hänelle Hanok.
\par 19 Ja Jered eli Hanokin syntymän jälkeen kahdeksansataa vuotta, ja hänelle syntyi poikia ja tyttäriä.
\par 20 Niin oli Jeredin koko elinaika yhdeksänsataa kuusikymmentä kaksi vuotta; sitten hän kuoli.
\par 21 Kun Hanok oli kuudenkymmenen viiden vuoden vanha, syntyi hänelle Metusalah.
\par 22 Ja Hanok vaelsi Metusalahin syntymän jälkeen Jumalan yhteydessä kolmesataa vuotta, ja hänelle syntyi poikia ja tyttäriä.
\par 23 Niin oli Hanokin koko elinaika kolmesataa kuusikymmentä viisi vuotta.
\par 24 Ja kun Hanok oli vaeltanut Jumalan yhteydessä, ei häntä enää ollut, sillä Jumala oli ottanut hänet pois.
\par 25 Kun Metusalah oli sadan kahdeksankymmenen seitsemän vuoden vanha, syntyi hänelle Lemek.
\par 26 Ja Metusalah eli Lemekin syntymän jälkeen seitsemänsataa kahdeksankymmentä kaksi vuotta, ja hänelle syntyi poikia ja tyttäriä.
\par 27 Niin oli Metusalahin koko elinaika yhdeksänsataa kuusikymmentä yhdeksän vuotta; sitten hän kuoli.
\par 28 Kun Lemek oli sadan kahdeksankymmenen kahden vuoden vanha, syntyi hänelle poika.
\par 29 Ja hän antoi hänelle nimen Nooa, sanoen: "Tämä lohduttaa meitä työssämme ja kättemme vaivoissa viljellessämme maata, jonka Herra on kironnut".
\par 30 Ja Lemek eli Nooan syntymän jälkeen viisisataa yhdeksänkymmentä viisi vuotta, ja hänelle syntyi poikia ja tyttäriä.
\par 31 Niin oli Lemekin koko elinaika seitsemänsataa seitsemänkymmentä seitsemän vuotta; sitten hän kuoli.
\par 32 Kun Nooa oli viidensadan vuoden vanha, syntyivät hänelle Seem, Haam ja Jaafet.

\chapter{6}

\par 1 Kun ihmiset alkoivat lisääntyä maan päällä ja heille syntyi tyttäriä,
\par 2 huomasivat Jumalan pojat ihmisten tyttäret ihaniksi ja ottivat vaimoikseen kaikki, jotka he parhaiksi katsoivat.
\par 3 Silloin Herra sanoi: "Minun Henkeni ei ole vallitseva ihmisessä iankaikkisesti, koska hän on liha. Niin olkoon hänen aikansa sata kaksikymmentä vuotta."
\par 4 Siihen aikaan eli maan päällä jättiläisiä, ja myöhemminkin, kun Jumalan pojat yhtyivät ihmisten tyttäriin ja nämä synnyttivät heille lapsia; nämä olivat noita muinaisajan kuuluisia sankareita.
\par 5 Mutta kun Herra näki, että ihmisten pahuus oli suuri maan päällä ja että kaikki heidän sydämensä aivoitukset ja ajatukset olivat kaiken aikaa ainoastaan pahat,
\par 6 niin Herra katui tehneensä ihmiset maan päälle, ja hän tuli murheelliseksi sydämessänsä.
\par 7 Ja Herra sanoi: "Minä hävitän maan päältä ihmiset, jotka minä loin, sekä ihmiset että karjan, matelijat ja taivaan linnut; sillä minä kadun ne tehneeni".
\par 8 Mutta Nooa sai armon Herran silmien edessä.
\par 9 Tämä on kertomus Nooan suvusta. Nooa oli aikalaistensa keskuudessa hurskas ja nuhteeton mies ja vaelsi Jumalan yhteydessä.
\par 10 Ja Nooalle syntyi kolme poikaa, Seem, Haam ja Jaafet.
\par 11 Mutta maa turmeltui Jumalan edessä, ja maa tuli täyteen väkivaltaa.
\par 12 Niin Jumala näki, että maa oli turmeltunut; sillä kaikki liha oli turmellut vaelluksensa maan päällä.
\par 13 Silloin Jumala sanoi Nooalle: "Minä olen päättänyt tehdä lopun kaikesta lihasta, sillä maa on heidän tähtensä täynnä väkivaltaa; katso, minä hävitän heidät ynnä maan.
\par 14 Tee itsellesi arkki honkapuista, rakenna arkki täyteen kammioita, ja tervaa se sisältä ja ulkoa.
\par 15 Ja näin on sinun se rakennettava: kolmesataa kyynärää olkoon arkin pituus, viisikymmentä kyynärää sen leveys ja kolmekymmentä kyynärää sen korkeus.
\par 16 Tee arkkiin valoaukko, ja tee se kyynärän korkuiseksi, ja sijoita arkin ovi sen kylkeen; rakenna siihen kolme kerrosta, alimmainen, keskimmäinen ja ylimmäinen.
\par 17 Sillä katso, minä annan vedenpaisumuksen tulla yli maan hävittämään taivaan alta kaiken lihan, kaiken, jossa on elämän henki; kaikki, mikä on maan päällä, on hukkuva.
\par 18 Mutta sinun kanssasi minä teen liiton, ja sinun on mentävä arkkiin, sinun ja sinun poikiesi, vaimosi ja miniäisi sinun kanssasi.
\par 19 Ja kaikista eläimistä, kaikesta lihasta, sinun on vietävä arkkiin kaksi kutakin lajia säilyttääksesi ne hengissä kanssasi; niitä olkoon koiras ja naaras.
\par 20 Lintuja lajiensa mukaan, karjaeläimiä lajiensa mukaan ja kaikkia maan matelijoita lajiensa mukaan tulkoon kaksi kutakin lajia sinun luoksesi, säilyttääksesi ne hengissä.
\par 21 Ja hanki itsellesi kaikkinaista ravintoa, syötäväksi kelpaavaa, ja kokoa sitä talteesi, että se olisi ruuaksi sinulle ja heille."
\par 22 Ja Nooa teki näin; aivan niin kuin Jumala hänen käski tehdä, niin hän teki.

\chapter{7}

\par 1 Ja Herra sanoi Nooalle: "Mene sinä ja koko perheesi arkkiin, sillä sinut minä olen tässä sukukunnassa havainnut hurskaaksi edessäni.
\par 2 Kaikkia puhtaita eläimiä ota luoksesi seitsemän paria, koiraita ja naaraita, mutta epäpuhtaita eläimiä kutakin yksi pari, koiras ja naaras.
\par 3 Niin myös taivaan lintuja seitsemän paria, koiraita ja naaraita, että siemen säilyisi elossa koko maan päällä.
\par 4 Sillä seitsemän päivän kuluttua minä annan sataa maan päälle neljäkymmentä päivää ja neljäkymmentä yötä ja hävitän maan päältä kaikki olennot, jotka olen tehnyt."
\par 5 Ja Nooa teki näin, aivan niin kuin Herra oli hänen käskenyt tehdä.
\par 6 Nooa oli kuudensadan vuoden vanha, kun vedenpaisumus tuli maan päälle.
\par 7 Ja Nooa ja hänen poikansa, vaimonsa ja miniänsä hänen kanssaan menivät arkkiin vedenpaisumusta pakoon.
\par 8 Puhtaita eläimiä ja epäpuhtaita eläimiä, lintuja ja kaikkia, joita maassa matelee,
\par 9 meni Nooan luo arkkiin kaksittain, koiras ja naaras, niinkuin Jumala oli Nooalle käskyn antanut.
\par 10 Ja niiden seitsemän päivän kuluttua tuli vedenpaisumus maan päälle.
\par 11 Ja sinä vuonna, jona Nooa oli kuudensadan vuoden vanha, sen toisena kuukautena, kuukauden seitsemäntenätoista päivänä, sinä päivänä puhkesivat kaikki suuren syvyyden lähteet, ja taivaan akkunat aukenivat.
\par 12 Ja satoi rankasti maan päälle neljäkymmentä päivää ja neljäkymmentä yötä.
\par 13 Juuri sinä päivänä menivät Nooa sekä Seem, Haam ja Jaafet, Nooan pojat, ja Nooan vaimo ja hänen kolme miniäänsä heidän kanssaan arkkiin;
\par 14 he ja kaikki metsäeläimet lajiensa mukaan ja kaikki karjaeläimet lajiensa mukaan ja kaikki matelijat, jotka maan päällä matelevat, lajiensa mukaan ja kaikki linnut lajiensa mukaan, kaikki lintuset, kaikki, mikä siivekästä on.
\par 15 Ja ne menivät Nooan luo arkkiin kaksittain, kaikki liha, kaikki, jossa oli elämän henki.
\par 16 Ja ne, jotka menivät sisälle, olivat koiras ja naaras kaikesta lihasta, niinkuin Jumala oli hänelle käskyn antanut. Ja Herra sulki oven hänen jälkeensä.
\par 17 Silloin tuli vedenpaisumus maan päälle, tuli neljänäkymmenenä päivänä; ja vedet paisuivat ja nostivat arkin, niin että se kohosi korkealle maasta.
\par 18 Ja vedet saivat vallan ja paisuivat suuresti maan päällä, ja arkki ajelehti veden pinnalla.
\par 19 Ja vedet nousivat nousemistaan maan päällä, niin että kaikki korkeat vuoret kaiken taivaan alla peittyivät.
\par 20 Viisitoista kyynärää vesi nousi vuorten yli, niin että ne peittyivät.
\par 21 Silloin hukkui kaikki liha, joka maan päällä liikkui: linnut, karjaeläimet, metsäeläimet ja kaikki pikkueläimet, joita maassa vilisi, sekä kaikki ihmiset.
\par 22 Kaikki, joiden sieramissa oli elämän hengen henkäys, kaikki, jotka elivät kuivalla maalla, kuolivat.
\par 23 Niin Herra hävitti kaikki olennot, jotka maan päällä olivat, niin hyvin ihmiset kuin myös karjaeläimet, matelijat ja taivaan linnut; ne hävisivät maan päältä. Ainoastaan Nooa sekä ne, jotka olivat hänen kanssansa arkissa, jäivät henkiin.
\par 24 Ja vedet vallitsivat maan päällä sata viisikymmentä päivää.

\chapter{8}

\par 1 Silloin Jumala muisti Nooaa ja kaikkia metsäeläimiä ja kaikkia karjaeläimiä, jotka olivat hänen kanssansa arkissa. Ja Jumala nosti tuulen puhaltamaan yli maan, niin että vesi laskeutui.
\par 2 Ja syvyyden lähteet ja taivaan akkunat sulkeutuivat, ja sade taivaasta taukosi.
\par 3 Ja vesi väistyi väistymistään maan päältä; sadan viidenkymmenen päivän kuluttua alkoi vesi vähentyä.
\par 4 Niin arkki pysähtyi seitsemäntenä kuukautena, kuukauden seitsemäntenätoista päivänä, Araratin vuorille.
\par 5 Ja vesi väheni vähenemistään aina kymmenenteen kuukauteen asti. Kymmenentenä kuukautena, kuukauden ensimmäisenä päivänä, tulivat vuorten huiput näkyviin.
\par 6 Neljänkymmenen päivän kuluttua Nooa avasi arkin ikkunan, jonka hän oli tehnyt,
\par 7 ja laski kaarneen lentoon, ja se lenteli edestakaisin, kunnes vesi maan päältä kuivui.
\par 8 Sitten hän laski luotansa kyyhkysen nähdäksensä, oliko vesi vähentynyt maan pinnalta.
\par 9 Mutta kyyhkynen ei löytänyt paikkaa, missä lepuuttaa jalkaansa, vaan palasi hänen luoksensa arkkiin, sillä koko maa oli vielä veden peitossa; niin hän ojensi kätensä ja otti sen luoksensa arkkiin.
\par 10 Ja hän odotti vielä toiset seitsemän päivää ja laski taas kyyhkysen arkista.
\par 11 Ja kyyhkynen tuli hänen luoksensa ehtoopuolella, ja katso, sen suussa oli tuore öljypuun lehti. Niin Nooa ymmärsi, että vesi oli vähentynyt maan päältä.
\par 12 Mutta hän odotti vielä toiset seitsemän päivää ja laski kyyhkysen lentoon, eikä se enää palannut hänen luoksensa.
\par 13 Ja Nooan kuudentenasadantena yhdentenä ikävuotena, vuoden ensimmäisenä kuukautena, kuukauden ensimmäisenä päivänä, oli vesi kuivunut maan päältä. Niin Nooa poisti katon arkista ja katseli; ja katso, maan pinta oli kuivunut.
\par 14 Ja toisena kuukautena, kuukauden kahdentenakymmenentenä seitsemäntenä päivänä, oli maa aivan kuiva.
\par 15 Ja Jumala puhui Nooalle sanoen:
\par 16 "Lähde arkista, sinä ja vaimosi, poikasi ja miniäsi sinun kanssasi.
\par 17 Ja kaikki eläimet, jotka ovat luonasi, kaikki liha, linnut ja karjaeläimet ja kaikki matelijat, jotka maan päällä matelevat, vie ne ulos kanssasi. Niitä vilisköön maassa, ja ne olkoot hedelmälliset ja lisääntykööt maan päällä."
\par 18 Ja Nooa ja hänen poikansa, vaimonsa ja miniänsä hänen kanssaan lähtivät ulos,
\par 19 niin myös kaikki metsäeläimet, kaikki matelijat ja kaikki linnut, kaikki, mikä liikkuu maan päällä, lähtivät arkista suvuittain.
\par 20 Ja Nooa rakensi alttarin Herralle ja otti kaikkia puhtaita karjaeläimiä ja kaikkia puhtaita lintuja ja uhrasi polttouhreja alttarilla.
\par 21 Ja Herra tunsi suloisen tuoksun ja sanoi sydämessänsä: "En minä koskaan enää kiroa maata ihmisen tähden; sillä ihmisen sydämen aivoitukset ovat pahat nuoruudesta saakka. Enkä minä koskaan enää tuhoa kaikkea, mikä elää, niinkuin nyt olen tehnyt.
\par 22 Niin kauan kuin maa pysyy, ei lakkaa kylväminen eikä leikkaaminen, ei vilu eikä helle, ei kesä eikä talvi, ei päivä eikä yö."

\chapter{9}

\par 1 Ja Jumala siunasi Nooan ja hänen poikansa ja sanoi heille: "Olkaa hedelmälliset ja lisääntykää ja täyttäkää maa.
\par 2 Ja peljätkööt ja vaviskoot teitä kaikki eläimet maan päällä ja kaikki taivaan linnut ja kaikki, jotka maassa matelevat, ja kaikki meren kalat; ne olkoot teidän valtaanne annetut.
\par 3 Kaikki, mikä liikkuu ja elää, olkoon teille ravinnoksi; niinkuin minä olen antanut teille viheriäiset kasvit, niin minä annan teille myös tämän kaiken.
\par 4 Älkää vain syökö lihaa, jossa sen sielu, sen veri, vielä on.
\par 5 Mutta teidän oman verenne minä kostan; jokaiselle eläimelle minä sen kostan, ja myöskin ihmisille minä kostan ihmisen sielun, toiselle toisen sielun.
\par 6 Joka ihmisen veren vuodattaa, hänen verensä on ihminen vuodattava, sillä Jumala on tehnyt ihmisen kuvaksensa.
\par 7 Ja te olkaa hedelmälliset ja lisääntykää, enentykää maassa ja lisääntykää siinä."
\par 8 Ja Jumala puhui Nooalle ja hänen pojillensa, jotka olivat hänen kanssansa, sanoen:
\par 9 "Katso, minä teen liiton teidän ja teidän jälkeläistenne kanssa
\par 10 ja kaikkien elävien olentojen kanssa, jotka luonanne ovat, lintujen, karjaeläinten ja kaikkien metsäeläinten kanssa, jotka luonanne ovat, kaikkien kanssa, jotka arkista lähtivät, kaikkien maan eläinten kanssa.
\par 11 Minä teen liiton teidän kanssanne: ei koskaan enää pidä kaikkea lihaa hukutettaman vedenpaisumuksella, eikä vedenpaisumus koskaan enää maata turmele."
\par 12 Ja Jumala sanoi: "Tämä on sen liiton merkki, jonka minä ikuisiksi ajoiksi teen itseni ja teidän ja kaikkien elävien olentojen välillä, jotka teidän luonanne ovat:
\par 13 minä panen kaareni pilviin, ja se on oleva liiton merkkinä minun ja maan välillä.
\par 14 Ja kun minä kokoan pilviä maan päälle ja kaari näkyy pilvissä,
\par 15 muistan minä liittoni, joka on minun ja teidän ja kaikkien elävien olentojen, kaiken lihan välillä, eikä vesi enää paisu tulvaksi hävittämään kaikkea lihaa.
\par 16 Niin kaari on oleva pilvissä, ja minä katselen sitä muistaakseni iankaikkista liittoa Jumalan ja kaikkien elävien olentojen, kaiken lihan välillä, joka maan päällä on."
\par 17 Ja Jumala sanoi Nooalle: "Tämä on sen liiton merkki, jonka minä olen tehnyt itseni ja kaiken lihan välillä, joka maan päällä on".
\par 18 Ja Nooan pojat, jotka lähtivät arkista, olivat Seem, Haam ja Jaafet. Ja Haam oli Kanaanin isä.
\par 19 Nämä kolme ovat Nooan pojat, ja heistä kaikki maan asukkaat polveutuvat.
\par 20 Ja Nooa oli peltomies ja ensimmäinen, joka istutti viinitarhan.
\par 21 Mutta kun hän joi viiniä, niin hän juopui ja makasi alasti majassansa.
\par 22 Ja Haam, Kanaanin isä, näki isänsä hävyn ja kertoi siitä molemmille veljillensä ulkona.
\par 23 Niin Seem ja Jaafet ottivat vaipan ja panivat molemmat sen hartioilleen ja menivät selin sisään ja peittivät isänsä hävyn; ja heidän kasvonsa olivat käännetyt toisaalle, niin etteivät he nähneet isänsä häpyä.
\par 24 Kun Nooa heräsi päihtymyksestänsä ja sai tietää, mitä hänen nuorin poikansa oli hänelle tehnyt,
\par 25 niin hän sanoi: "Kirottu olkoon Kanaan, olkoon hän veljiensä orjain orja".
\par 26 Vielä hän sanoi: "Kiitetty olkoon Herra, Seemin Jumala, ja olkoon Kanaan heidän orjansa.
\par 27 Jumala laajentakoon Jaafetin, ja asukoon hän Seemin majoissa, ja Kanaan olkoon heidän orjansa."
\par 28 Ja Nooa eli vedenpaisumuksen jälkeen kolmesataa viisikymmentä vuotta.
\par 29 Niin Nooan koko ikä oli yhdeksänsataa viisikymmentä vuotta; sitten hän kuoli.

\chapter{10}

\par 1 Tämä on kertomus Nooan poikien, Seemin, Haamin ja Jaafetin, suvusta. Heille syntyi poikia vedenpaisumuksen jälkeen.
\par 2 Jaafetin pojat olivat Goomer, Maagog, Maadai, Jaavan, Tuubal, Mesek ja Tiiras.
\par 3 Ja Goomerin pojat olivat Askenas, Riifat ja Toogarma.
\par 4 Ja Jaavanin pojat olivat Elisa, Tarsis, kittiläiset ja doodanilaiset;
\par 5 heistä haarautuivat pakanoiden saarten asukkaat maittensa, eri kieltensä, heimojensa ja kansakuntiensa mukaan.
\par 6 Ja Haamin pojat olivat Kuus, Misraim, Puut ja Kanaan.
\par 7 Ja Kuusin pojat olivat Seba, Havila, Sabta, Raema ja Sabteka. Ja Raeman pojat olivat Saba ja Dedan.
\par 8 Ja Kuusille syntyi Nimrod. Hän oli ensimmäinen valtias maan päällä.
\par 9 Hän oli mahtava metsämies Herran edessä. Siitä on sananparsi: "Mahtava metsämies Herran edessä niinkuin Nimrod".
\par 10 Ja hänen valtakuntansa alkuna olivat Baabel, Erek, Akkad ja Kalne Sinearin maassa.
\par 11 Siitä maasta hän lähti Assuriin ja rakensi Niiniven, Rehobot-Iirin ja Kelahin,
\par 12 sekä Resenin Niiniven ja Kelahin välille; se on tuo suuri kaupunki.
\par 13 Ja Misraimille syntyivät luudilaiset, anamilaiset, lehabilaiset, naftuhilaiset,
\par 14 patrokselaiset, kasluhilaiset, joista filistealaiset ovat lähteneet, sekä kaftorilaiset.
\par 15 Ja Kanaanille syntyivät Siidon, hänen esikoisensa, ja Heet,
\par 16 sekä jebusilaiset, amorilaiset, girgasilaiset,
\par 17 hivviläiset, arkilaiset, siiniläiset,
\par 18 arvadilaiset, semarilaiset ja hamatilaiset. Sittemmin hajaantuivat kanaanilaisten heimot.
\par 19 Ja kanaanilaisten alue ulottui Siidonista Gerariin päin aina Gassaan asti sekä Sodomaan, Gomorraan, Admaan ja Seboimiin päin aina Lesaan asti.
\par 20 Nämä olivat Haamin pojat heimojensa, kieltensä, maittensa ja kansakuntiensa mukaan.
\par 21 Myöskin Seemille, Jaafetin vanhimmalle veljelle, josta tuli kaikkien Eeberin poikien kantaisä, syntyi poikia.
\par 22 Seemin pojat olivat Eelam, Assur, Arpaksad, Luud ja Aram.
\par 23 Ja Aramin pojat olivat Uus, Huul, Geter ja Mas.
\par 24 Arpaksadille syntyi Selah, ja Selahille syntyi Eeber.
\par 25 Ja Eeberille syntyi kaksi poikaa: toisen nimi oli Peleg, sillä hänen aikanansa jakaantuivat maan asukkaat, ja hänen veljensä nimi oli Joktan.
\par 26 Ja Joktanille syntyi Almodad, Selef, Hasarmavet, Jerah,
\par 27 Hadoram, Uusal, Dikla,
\par 28 Oobal, Abimael, Saba,
\par 29 Oofir, Havila ja Joobab. Kaikki nämä olivat Joktanin poikia.
\par 30 Ja heidän asumasijansa ulottuivat Meesasta aina Sefariin, Itävuorelle asti.
\par 31 Nämä olivat Seemin pojat heimojensa, kieltensä, maittensa ja kansakuntiensa mukaan.
\par 32 Nämä olivat Nooan poikien heimot sukukuntiensa ja kansakuntiensa mukaan; ja niistä haarautuivat kansat maan päälle vedenpaisumuksen jälkeen.

\chapter{11}

\par 1 Ja kaikessa maassa oli yksi kieli ja yksi puheenparsi.
\par 2 Kun he lähtivät liikkeelle itään päin, löysivät he lakeuden Sinearin maassa ja asettuivat sinne.
\par 3 Ja he sanoivat toisillensa: "Tulkaa, tehkäämme tiiliä ja polttakaamme ne koviksi". Ja tiiltä he käyttivät kivenä, ja maapihkaa he käyttivät laastina.
\par 4 Ja he sanoivat: "Tulkaa, rakentakaamme itsellemme kaupunki ja torni, jonka huippu ulottuu taivaaseen, ja tehkäämme itsellemme nimi, ettemme hajaantuisi yli kaiken maan".
\par 5 Niin Herra astui alas katsomaan kaupunkia ja tornia, jonka ihmislapset olivat rakentaneet.
\par 6 Ja Herra sanoi: "Katso, he ovat yksi kansa, ja heillä kaikilla on yksi kieli, ja tämä on heidän ensimmäinen yrityksensä. Ja nyt ei heille ole mahdotonta mikään, mitä aikovatkin tehdä.
\par 7 Tulkaa, astukaamme alas ja sekoittakaamme siellä heidän kielensä, niin ettei toinen ymmärrä toisen kieltä."
\par 8 Ja niin Herra hajotti heidät sieltä yli kaiken maan, niin että he lakkasivat kaupunkia rakentamasta.
\par 9 Siitä tuli sen nimeksi Baabel, koska Herra siellä sekoitti kaiken maan kielen; ja sieltä Herra hajotti heidät yli kaiken maan.
\par 10 Tämä on kertomus Seemin suvusta. Kun Seem oli sadan vuoden vanha, syntyi hänelle Arpaksad kaksi vuotta vedenpaisumuksen jälkeen.
\par 11 Ja Seem eli Arpaksadin syntymän jälkeen viisisataa vuotta, ja hänelle syntyi poikia ja tyttäriä.
\par 12 Kun Arpaksad oli kolmenkymmenen viiden vuoden vanha, syntyi hänelle Selah.
\par 13 Ja Arpaksad eli Selahin syntymän jälkeen neljäsataa kolme vuotta, ja hänelle syntyi poikia ja tyttäriä.
\par 14 Kun Selah oli kolmenkymmenen vuoden vanha, syntyi hänelle Eeber.
\par 15 Ja Selah eli Eeberin syntymän jälkeen neljäsataa kolme vuotta, ja hänelle syntyi poikia ja tyttäriä.
\par 16 Kun Eeber oli kolmenkymmenen neljän vuoden vanha, syntyi hänelle Peleg.
\par 17 Ja Eeber eli Pelegin syntymän jälkeen neljäsataa kolmekymmentä vuotta, ja hänelle syntyi poikia ja tyttäriä.
\par 18 Kun Peleg oli kolmenkymmenen vuoden vanha, syntyi hänelle Regu.
\par 19 Ja Peleg eli Regun syntymän jälkeen kaksisataa yhdeksän vuotta, ja hänelle syntyi poikia ja tyttäriä.
\par 20 Kun Regu oli kolmenkymmenen kahden vuoden vanha, syntyi hänelle Serug.
\par 21 Ja Regu eli Serugin syntymän jälkeen kaksisataa seitsemän vuotta, ja hänelle syntyi poikia ja tyttäriä.
\par 22 Kun Serug oli kolmenkymmenen vuoden vanha, syntyi hänelle Naahor.
\par 23 Ja Serug eli Naahorin syntymän jälkeen kaksisataa vuotta, ja hänelle syntyi poikia ja tyttäriä.
\par 24 Kun Naahor oli kahdenkymmenen yhdeksän vuoden vanha, syntyi hänelle Terah.
\par 25 Ja Naahor eli Terahin syntymän jälkeen sata yhdeksäntoista vuotta, ja hänelle syntyi poikia ja tyttäriä.
\par 26 Kun Terah oli seitsemänkymmenen vuoden vanha, syntyivät hänelle Abram, Naahor ja Haaran.
\par 27 Ja tämä on kertomus Terahin suvusta. Terahille syntyivät Abram, Naahor ja Haaran. Ja Haaranille syntyi Loot.
\par 28 Ja Haaran kuoli ennen isäänsä Terahia synnyinmaassansa, Kaldean Uurissa.
\par 29 Ja Abram ja Naahor ottivat itsellensä vaimot; Abramin vaimon nimi oli Saarai, ja Naahorin vaimon nimi oli Milka, Haaranin tytär; Haaran oli Milkan ja Jiskan isä.
\par 30 Mutta Saarai oli hedelmätön, hänellä ei ollut lasta.
\par 31 Ja Terah otti poikansa Abramin ja poikansa pojan Lootin, Haaranin pojan, ja miniänsä Saarain, poikansa Abramin vaimon, ja he lähtivät heidän kanssaan Kaldean Uurista mennäksensä Kanaanin maahan, ja he saapuivat Harraniin asti ja asuivat siellä.
\par 32 Ja Terahin ikä oli kaksisataa viisi vuotta; sitten Terah kuoli Harranissa.

\chapter{12}

\par 1 Ja Herra sanoi Abramille: "Lähde maastasi, suvustasi ja isäsi kodista siihen maahan, jonka minä sinulle osoitan.
\par 2 Niin minä teen sinusta suuren kansan, siunaan sinut ja teen sinun nimesi suureksi, ja sinä olet tuleva siunaukseksi.
\par 3 Ja minä siunaan niitä, jotka sinua siunaavat, ja kiroan ne, jotka sinua kiroavat, ja sinussa tulevat siunatuiksi kaikki sukukunnat maan päällä."
\par 4 Niin Abram lähti, niinkuin Herra oli hänelle puhunut, ja Loot meni hänen kanssansa. Abram oli Harranista lähtiessänsä seitsemänkymmenen viiden vuoden vanha.
\par 5 Ja Abram otti vaimonsa Saarain ja veljensä pojan Lootin sekä kaiken omaisuuden, jonka he olivat koonneet, ja ne palvelijat, jotka he olivat hankkineet Harranissa, ja he lähtivät menemään Kanaanin maahan. Ja he tulivat Kanaanin maahan.
\par 6 Ja Abram kulki maan läpi aina Sikemin paikkakunnalle, Mooren tammelle asti. Ja siihen aikaan kanaanilaiset asuivat siinä maassa.
\par 7 Silloin Herra ilmestyi Abramille ja sanoi: "Sinun jälkeläisillesi minä annan tämän maan". Niin hän rakensi sinne alttarin Herralle, joka oli hänelle ilmestynyt.
\par 8 Sieltä hän siirtyi edemmäksi vuoristoon, itään päin Beetelistä, ja pystytti telttansa, niin että Beetel oli lännessä ja Ai idässä, ja hän rakensi sinne alttarin Herralle ja huusi avuksi Herran nimeä.
\par 9 Ja sieltä Abram lähti ja vaelsi yhä edemmäksi Etelämaahan päin.
\par 10 Niin tuli nälänhätä maahan, ja Abram meni Egyptiin, asuakseen siellä jonkun aikaa, sillä nälänhätä maassa oli kova.
\par 11 Ja kun hän lähestyi Egyptiä, puhui hän vaimollensa Saaraille: "Katso, minä tiedän, että sinä olet kaunis nainen.
\par 12 Kun egyptiläiset saavat nähdä sinut, niin he sanovat: 'Hän on hänen vaimonsa', ja tappavat minut, mutta antavat sinun elää.
\par 13 Sano siis olevasi minun sisareni, että minun kävisi hyvin sinun tähtesi ja minä sinun takiasi saisin jäädä henkiin."
\par 14 Kun Abram tuli Egyptiin, näkivät egyptiläiset, että Saarai oli hyvin kaunis nainen.
\par 15 Ja kun faraon ruhtinaat olivat nähneet hänet, ylistelivät he häntä faraolle; ja vaimo vietiin faraon hoviin.
\par 16 Ja Abramia hän kohteli hyvin hänen tähtensä. Ja hän sai pikkukarjaa, raavaskarjaa ja aaseja, palvelijoita ja palvelijattaria, aasintammoja ja kameleja.
\par 17 Mutta Herra antoi kovien vitsausten kohdata faraota ja hänen hoviansa Saarain, Abramin vaimon, tähden.
\par 18 Silloin farao kutsui Abramin luoksensa ja sanoi: "Mitä olet minulle tehnyt? Miksi et ilmoittanut minulle, että hän on sinun vaimosi?
\par 19 Miksi sanoit: 'Hän on minun sisareni', niin että minä otin hänet vaimokseni? Katso, tässä on vaimosi, ota hänet ja mene."
\par 20 Ja farao antoi hänestä käskyn miehillensä, että he saattaisivat hänet pois, hänet ja hänen vaimonsa sekä kaiken, mitä hänellä oli.

\chapter{13}

\par 1 Niin Abram lähti pois Egyptistä, hän ja hänen vaimonsa ja kaikki, mitä hänellä oli, ja Loot hänen kanssaan, Etelämaahan.
\par 2 Abram oli hyvin rikas: hänellä oli karjaa, hopeata ja kultaa.
\par 3 Ja hän vaelsi, kulkien levähdyspaikasta toiseen, Etelämaasta Beeteliin asti, aina siihen paikkaan, missä hänen majansa oli ensi kerralla ollut, Beetelin ja Ain välillä,
\par 4 siihen paikkaan, johon hän ennen oli rakentanut alttarin; ja Abram huusi siinä avuksi Herran nimeä.
\par 5 Ja myöskin Lootilla, joka vaelsi Abramin kanssa, oli pikkukarjaa, raavaskarjaa ja telttoja.
\par 6 Eikä maa riittänyt heidän asuakseen yhdessä, sillä heillä oli paljon omaisuutta, niin etteivät voineet yhdessä asua.
\par 7 Niin syntyi riitaa Abramin karjapaimenten ja Lootin karjapaimenten välillä. Ja siihen aikaan asuivat siinä maassa kanaanilaiset ja perissiläiset.
\par 8 Silloin Abram sanoi Lootille: "Älköön olko riitaa meidän välillämme, minun ja sinun, älköönkä minun paimenteni ja sinun paimentesi välillä, sillä olemmehan veljeksiä.
\par 9 Eikö koko maa ole avoinna edessäsi? Eroa minusta. Jos sinä menet vasemmalle, niin minä menen oikealle, tahi jos sinä menet oikealle, niin minä menen vasemmalle."
\par 10 Ja Loot nosti silmänsä ja näki koko Jordanin lakeuden olevan runsasvetistä seutua; ennenkuin Herra hävitti Sodoman ja Gomorran, oli se Sooariin saakka niinkuin Herran puutarha, niinkuin Egyptin maa.
\par 11 Niin Loot valitsi itselleen koko Jordanin lakeuden ja siirtyi itään päin, ja he erkanivat toisistaan.
\par 12 Abram asettui Kanaanin maahan, Loot asettui lakeuden kaupunkeihin ja siirtyi siirtymistään telttoineen Sodomaan asti.
\par 13 Mutta Sodoman kansa oli kovin pahaa ja syntistä Herran edessä.
\par 14 Ja Herra sanoi Abramille, sen jälkeen kuin Loot oli hänestä eronnut: "Nosta silmäsi ja katso siitä paikasta, missä olet, pohjoiseen, etelään, itään ja länteen.
\par 15 Sillä kaiken maan, jonka näet, minä annan sinulle ja sinun jälkeläisillesi ikuisiksi ajoiksi.
\par 16 Ja minä teen sinun jälkeläistesi luvun paljoksi kuin maan tomun. Jos voidaan lukea maan tomu, niin voidaan lukea myöskin sinun jälkeläisesi.
\par 17 Nouse ja vaella maata pitkin ja poikin, sillä sinulle minä sen annan."
\par 18 Ja Abram siirtyi siirtymistään telttoineen ja tuli ja asettui Mamren tammistoon, joka on Hebronin luona, ja rakensi sinne alttarin Herralle.

\chapter{14}

\par 1 Ja tapahtui siihen aikaan, kun Amrafel oli Sinearin kuninkaana, Arjok Ellasarin kuninkaana, Kedorlaomer Eelamin kuninkaana ja Tidal Goojimin kuninkaana,
\par 2 että he alottivat sodan Beraa, Sodoman kuningasta, Birsaa, Gomorran kuningasta, Sinabia, Adman kuningasta, Semeberiä, Seboimin kuningasta, ja Belan, se on Sooarin, kuningasta vastaan.
\par 3 Nämä kaikki liittoutuivat kokoontuen Siddimin laaksoon, jossa Suolameri nyt on.
\par 4 Kaksitoista vuotta he olivat olleet Kedorlaomerille alamaiset, mutta kolmantenatoista vuotena he tekivät kapinan.
\par 5 Neljäntenätoista vuotena tulivat Kedorlaomer ja ne kuninkaat, jotka olivat hänen kanssaan; ja he voittivat refalaiset Astarot-Karnaimissa ja suusilaiset Haamissa, niin myös eemiläiset Kirjataimin tasangolla
\par 6 ja hoorilaiset heidän vuoristossaan, Seirissä, aina Eel-Paaraniin asti, joka on erämaan laidassa.
\par 7 Ja he palasivat ja tulivat Mispatin lähteelle, se on Kaadekseen, ja valloittivat koko amalekilaisten maan ja voittivat myöskin amorilaiset, jotka asuivat Hasason-Taamarissa.
\par 8 Silloin lähtivät Sodoman kuningas, Gomorran kuningas, Adman kuningas, Seboimin kuningas ja Belan, se on Sooarin, kuningas ja asettuivat Siddimin laaksossa sotarintaan heitä vastaan -
\par 9 Eelamin kuningasta Kedorlaomeria, Goojimin kuningasta Tidalia, Sinearin kuningasta Amrafelia ja Ellasarin kuningasta Arjokia vastaan, neljä kuningasta viittä vastaan.
\par 10 Mutta Siddimin laakso oli täynnä maapihkakuoppia. Ja Sodoman ja Gomorran kuninkaat pakenivat ja putosivat niihin; mutta henkiin jääneet pakenivat vuoristoon.
\par 11 Ja he ottivat Sodomasta ja Gomorrasta kaiken tavaran ja kaikki ruokavarat ja menivät matkaansa.
\par 12 Ja lähtiessään he ottivat mukaansa myöskin Lootin, Abramin veljenpojan, ja hänen omaisuutensa; hän näet asui Sodomassa.
\par 13 Mutta muuan pakolainen tuli ja ilmoitti siitä Abramille, hebrealaiselle; tämä asui tammistossa, joka oli amorilaisen Mamren, Eskolin ja Aanerin veljen, oma, ja he olivat Abramin liittolaisia.
\par 14 Kun Abram kuuli, että hänen sukulaisensa oli otettu vangiksi, aseisti hän luotettavimmat palvelijansa, jotka olivat hänen kodissaan syntyneet, kolmesataa kahdeksantoista miestä, ja ajoi vihollisia takaa aina Daaniin saakka.
\par 15 Ja hän jakoi väkensä ja hyökkäsi palvelijoineen yöllä vihollisten kimppuun ja voitti heidät ja ajoi heitä takaa aina Hoobaan saakka, joka on Damaskosta pohjoiseen.
\par 16 Ja hän toi takaisin kaiken tavaran; myöskin sukulaisensa Lootin ja hänen tavaransa hän toi takaisin, niin myös vaimot ja muun väen.
\par 17 Kun hän oli paluumatkalla, voitettuaan Kedorlaomerin ja ne kuninkaat, jotka olivat tämän kanssa, meni Sodoman kuningas häntä vastaan Saaven laaksoon, jota sanotaan "Kuninkaan laaksoksi".
\par 18 Ja Melkisedek, Saalemin kuningas, toi leipää ja viiniä; hän oli Jumalan, Korkeimman, pappi.
\par 19 Ja hän siunasi hänet sanoen: "Siunatkoon Abramia Jumala, Korkein, taivaan ja maan luoja.
\par 20 Ja kiitetty olkoon Jumala, Korkein, joka antoi vihollisesi sinun käsiisi." Ja Abram antoi hänelle kymmenykset kaikesta.
\par 21 Ja Sodoman kuningas sanoi Abramille: "Anna minulle väki ja pidä sinä tavara".
\par 22 Mutta Abram sanoi Sodoman kuninkaalle: "Minä nostan käteni Herran, Jumalan, Korkeimman, taivaan ja maan luojan, puoleen ja vannon:
\par 23 En totisesti ota, en langan päätä, en kengän paulaa enkä mitään muuta, mikä on sinun, ettet sanoisi: 'Minä olen tehnyt Abramin rikkaaksi'.
\par 24 En tahdo mitään, paitsi mitä palvelijat ovat kuluttaneet ja mikä on niille miehille tuleva, jotka minua seurasivat, Aanerille, Eskolille ja Mamrelle; he saakoot osansa."

\chapter{15}

\par 1 Näiden tapausten jälkeen tuli Abramille näyssä tämä Herran sana: "Älä pelkää, Abram! Minä olen sinun kilpesi; sinun palkkasi on oleva sangen suuri."
\par 2 Mutta Abram sanoi: "Oi Herra, Herra, mitä sinä minulle annat? Minä lähden täältä lapsetonna, ja omaisuuteni haltijaksi tulee damaskolainen mies, Elieser."
\par 3 Ja Abram sanoi vielä: "Sinä et ole antanut minulle jälkeläistä; katso, talossani syntynyt palvelija on minut perivä".
\par 4 Mutta katso, hänelle tuli tämä Herran sana: "Hän ei ole sinua perivä, vaan joka lähtee sinun omasta ruumiistasi, hän on sinut perivä".
\par 5 Ja hän vei hänet ulos ja sanoi: "Katso taivaalle ja lue tähdet, jos ne taidat lukea". Ja hän sanoi hänelle: "Niin paljon on sinulla oleva jälkeläisiä".
\par 6 Ja Abram uskoi Herraan, ja Herra luki sen hänelle vanhurskaudeksi.
\par 7 Ja hän sanoi hänelle: "Minä olen Herra, joka toin sinut Kaldean Uurista, antaakseni sinulle tämän maan omaksesi".
\par 8 Mutta hän sanoi: "Oi Herra, Herra, mistä minä tiedän, että saan sen omakseni?"
\par 9 Ja hän sanoi hänelle: "Tuo minulle kolmivuotias hieho, kolmivuotias vuohi ja kolmivuotias oinas sekä metsäkyyhkynen ja nuori kyyhkynen".
\par 10 Ja hän toi nämä kaikki ja halkaisi ne ja asetti puolikkaat vastakkain; lintuja hän ei kuitenkaan halkaissut.
\par 11 Niin laskeutui petolintuja ruumiiden päälle, mutta Abram karkoitti ne pois.
\par 12 Kun aurinko oli laskemaisillaan, valtasi raskas uni Abramin, ja katso, kauhu ja suuri pimeys valtasi hänet.
\par 13 Ja Herra sanoi Abramille: "Niin tiedä totisesti, että sinun jälkeläisesi tulevat elämään muukalaisina maassa, joka ei ole heidän omansa, ja heidän on niitä palveleminen, ja ne sortavat heitä neljäsataa vuotta.
\par 14 Mutta myös sen kansan, jota he palvelevat, minä tuomitsen; ja sitten he pääsevät lähtemään, mukanaan paljon tavaraa.
\par 15 Mutta sinä saat mennä isiesi tykö rauhassa, ja sinut haudataan päästyäsi korkeaan ikään.
\par 16 Ja neljännessä polvessa sinun jälkeläisesi palaavat tänne takaisin; sillä amorilaisten syntivelka ei ole vielä täysi."
\par 17 Ja kun aurinko oli laskenut ja oli tullut pilkkopimeä, näkyi suitsuava pätsi ja liekehtivä tuli, joka liikkui uhrikappaleiden välissä.
\par 18 Sinä päivänä Herra teki Abramin kanssa liiton, sanoen: "Sinun jälkeläisillesi minä annan tämän maan, Egyptin virrasta aina suureen virtaan, Eufrat-virtaan saakka:
\par 19 keeniläiset, kenissiläiset, kadmonilaiset,
\par 20 heettiläiset, perissiläiset, refalaiset,
\par 21 amorilaiset, kanaanilaiset, girgasilaiset ja jebusilaiset".

\chapter{16}

\par 1 Saarai, Abramin vaimo, ei synnyttänyt hänelle lasta. Mutta Saarailla oli egyptiläinen orjatar, jonka nimi oli Haagar.
\par 2 Ja Saarai sanoi Abramille: "Katso, Herra on sulkenut minut synnyttämästä; yhdy siis minun orjattareeni, ehkä minä saisin lapsia hänestä". Ja Abram kuuli Saaraita.
\par 3 Ja Saarai, Abramin vaimo, otti egyptiläisen orjattarensa Haagarin, sitten kuin Abram oli asunut kymmenen vuotta Kanaanin maassa, ja antoi hänet miehellensä Abramille vaimoksi.
\par 4 Ja hän yhtyi Haagariin, ja Haagar tuli raskaaksi. Kun hän huomasi olevansa raskaana, tuli hänen emäntänsä halvaksi hänen silmissään.
\par 5 Silloin Saarai sanoi Abramille: "Minun kärsimäni vääryys kohdatkoon sinua; minä annoin orjattareni sinun syliisi, mutta kun hän huomasi olevansa raskaana, tulin minä halvaksi hänen silmissään. Herra tuomitkoon meidän välillämme, minun ja sinun."
\par 6 Abram sanoi Saaraille: "Katso, orjattaresi on sinun vallassasi, tee hänelle, mitä tahdot". Niin Saarai kuritti häntä, ja hän pakeni hänen luotaan.
\par 7 Ja Herran enkeli tapasi hänet vesilähteeltä erämaassa, sen lähteen luota, joka on Suurin tien varressa.
\par 8 Ja hän sanoi: "Haagar, Saarain orjatar, mistä tulet ja mihin menet?" Hän vastasi: "Olen paossa emäntääni Saaraita".
\par 9 Ja Herran enkeli sanoi hänelle: "Palaa emäntäsi tykö ja nöyrry hänen kätensä alle".
\par 10 Ja Herran enkeli sanoi hänelle: "Minä teen sinun jälkeläistesi luvun niin suureksi, ettei heitä voida lukea heidän paljoutensa tähden".
\par 11 Vielä Herran enkeli puhui hänelle: "Katso, sinä olet raskaana ja synnytät pojan ja kutsut hänet Ismaeliksi, sillä Herra on kuullut sinun hätäsi.
\par 12 Hänestä tulee mies kuin villiaasi: hänen kätensä on kaikkia vastaan, ja kaikkien käsi on häntä vastaan, ja hän on kaikkien veljiensä niskassa."
\par 13 Ja Haagar nimitti Herraa, joka oli häntä puhutellut, nimellä: "Sinä olet ilmestyksen Jumala". Sillä hän sanoi: "Olenko minä tässä vilaukselta saanut nähdä hänet, joka minut näkee?"
\par 14 Sentähden kutsutaan kaivoa nimellä Lahai-Roin kaivo; se on Kaadeksen ja Beredin välillä.
\par 15 Ja Haagar synnytti Abramille pojan, ja Abram antoi pojallensa, jonka Haagar oli hänelle synnyttänyt, nimen Ismael.
\par 16 Ja Abram oli kahdeksankymmenen kuuden vuoden vanha, kun Haagar synnytti hänelle Ismaelin.

\chapter{17}

\par 1 Kun Abram oli yhdeksänkymmenen yhdeksän vuoden vanha, ilmestyi Herra hänelle ja sanoi hänelle: "Minä olen Jumala, Kaikkivaltias; vaella minun edessäni ja ole nuhteeton.
\par 2 Ja minä teen liittoni meidän välillemme, minun ja sinun, ja lisään sinut ylen runsaasti."
\par 3 Ja Abram lankesi kasvoilleen, ja Jumala puhui hänelle sanoen:
\par 4 "Katso, tämä on minun liittoni sinun kanssasi: sinusta tulee kansojen paljouden isä.
\par 5 Niin älköön sinua enää kutsuttako Abramiksi, vaan nimesi olkoon Aabraham, sillä minä teen sinusta kansojen paljouden isän.
\par 6 Minä teen sinut sangen hedelmälliseksi ja annan sinusta tulla kansoja, ja sinusta on polveutuva kuninkaita.
\par 7 Ja minä teen liiton sinun kanssasi ja sinun jälkeläistesi kanssa, sukupolvesta sukupolveen, iankaikkisen liiton, ollakseni sinun ja sinun jälkeläistesi Jumala,
\par 8 ja minä annan sinulle ja sinun jälkeläisillesi sen maan, jossa sinä muukalaisena asut, koko Kanaanin maan, ikuiseksi omaisuudeksi; ja minä olen heidän Jumalansa."
\par 9 Ja Jumala sanoi Aabrahamille: "Mutta sinä pidä minun liittoni, sinä ja sinun jälkeläisesi, sukupolvesta sukupolveen.
\par 10 Ja tämä on minun liittoni sinun ja sinun jälkeläistesi kanssa; pitäkää se: ympärileikatkaa jokainen miehenpuoli keskuudessanne.
\par 11 Ympärileikatkaa esinahkanne liha, ja se olkoon liiton merkki meidän välillämme, minun ja teidän.
\par 12 Sukupolvesta sukupolveen ympärileikattakoon jokainen poikalapsi teidän keskuudessanne kahdeksan päivän vanhana, niin hyvin kotona syntynyt palvelija kuin muukalaiselta, keneltä tahansa, rahalla ostettu, joka ei ole sinun jälkeläisiäsi.
\par 13 Ympärileikattakoon sekä kotonasi syntynyt että rahalla ostamasi; ja niin minun liittoni on oleva merkitty teidän lihaanne iankaikkiseksi liitoksi.
\par 14 Mutta ympärileikkaamaton miehenpuoli, jonka esinahan liha ei ole ympärileikattu, hävitettäköön kansastansa; hän on rikkonut minun liittoni."
\par 15 Ja Jumala sanoi Aabrahamille: "Älä kutsu vaimoasi Saaraita enää Saaraiksi, vaan Saara olkoon hänen nimensä.
\par 16 Ja minä siunaan häntä ja annan hänestäkin sinulle pojan; niin, minä siunaan häntä, ja hänestä tulee kansakuntia, polveutuu kansojen kuninkaita."
\par 17 Aabraham lankesi kasvoillensa ja naurahti, sillä hän ajatteli sydämessään: "Voiko satavuotiaalle syntyä poika, ja voiko Saara, joka on yhdeksänkymmenen vuoden vanha, vielä synnyttää?"
\par 18 Ja Aabraham sanoi Jumalalle: "Kunpa edes Ismael saisi elää sinun edessäsi!"
\par 19 Niin Jumala sanoi: "Totisesti, sinun vaimosi Saara synnyttää sinulle pojan, ja sinun on pantava hänen nimekseen Iisak, ja minä teen liittoni hänen kanssaan iankaikkiseksi liitoksi hänen jälkeläisillensä.
\par 20 Mutta myös Ismaelista minä olen kuullut sinun rukouksesi; katso, minä siunaan häntä ja teen hänet hedelmälliseksi ja annan hänen lisääntyä ylen runsaasti. Hänestä on polveutuva kaksitoista ruhtinasta, ja minä teen hänestä suuren kansan.
\par 21 Mutta liittoni minä teen Iisakin kanssa, jonka Saara sinulle synnyttää tähän aikaan tulevana vuonna."
\par 22 Kun Jumala oli lakannut puhumasta Aabrahamin kanssa, kohosi hän ylös hänen luotansa.
\par 23 Ja Aabraham otti poikansa Ismaelin sekä kaikki kotona syntyneet ja kaikki rahalla ostamansa palvelijat, kaikki Aabrahamin perheen miehenpuolet, ja ympärileikkasi sinä samana päivänä heidän esinahkansa lihan, niinkuin Jumala oli hänelle sanonut.
\par 24 Aabraham oli yhdeksänkymmenen yhdeksän vuoden vanha, kun hänen esinahkansa liha ympärileikattiin.
\par 25 Ja hänen poikansa Ismael oli kolmentoista vuoden vanha, kun hänen esinahkansa liha ympärileikattiin.
\par 26 Aabraham ja hänen poikansa Ismael ympärileikattiin sinä samana päivänä;
\par 27 ja kaikki hänen perheensä miehet, sekä kotona syntyneet että muukalaisilta rahalla ostetut, ympärileikattiin hänen kanssaan.

\chapter{18}

\par 1 Ja Herra ilmestyi hänelle Mamren tammistossa, jossa hän istui telttamajansa ovella päivän ollessa palavimmillaan.
\par 2 Kun hän nosti silmänsä ja katseli, niin katso, kolme miestä seisoi hänen edessänsä; nähdessään heidät hän riensi heitä vastaan majan ovelta ja kumartui maahan
\par 3 ja sanoi: "Herrani, jos olen saanut armon sinun silmiesi edessä, älä mene palvelijasi ohitse.
\par 4 Sallikaa tuoda vähän vettä pestäksenne jalkanne ja levätkää puun siimeksessä.
\par 5 Minä tuon palasen leipää virkistääksenne itseänne, ennenkuin jatkatte matkaanne, sillä kaiketi sitä varten olette poikenneet palvelijanne luo." He sanoivat: "Tee, niinkuin olet puhunut".
\par 6 Ja Aabraham kiiruhti majaan Saaran tykö ja sanoi: "Hae joutuin kolme vakallista lestyjä jauhoja, sotke ja leivo kaltiaisia".
\par 7 Sitten Aabraham riensi karjaan, otti nuoren ja kauniin vasikan ja antoi palvelijalle, joka ryhtyi nopeasti sitä valmistamaan.
\par 8 Ja hän otti voita ja maitoa sekä vasikan, jonka hän oli valmistuttanut, ja pani ne heidän eteensä; itse hän seisoi heidän luonansa puun alla sillä aikaa, kuin he söivät.
\par 9 Ja he kysyivät häneltä: "Missä vaimosi Saara on?" Hän vastasi: "Tuolla majassa".
\par 10 Ja hän sanoi: "Minä palaan luoksesi tulevana vuonna tähän aikaan, ja katso, vaimollasi Saaralla on silloin oleva poika". Mutta Saara kuunteli majan ovella hänen takanansa.
\par 11 Mutta Aabraham ja Saara olivat iäkkäät, eikä Saaran enää ollut, niinkuin naisten tavallisesti on.
\par 12 Sentähden Saara naurahti itseksensä ja ajatteli: "Heräisikö minussa, näin kuihduttuani, vielä halu? Ja myös minun herrani on vanha."
\par 13 Mutta Herra sanoi Aabrahamille: "Miksi Saara nauroi ajatellen: 'Synnyttäisinkö minä todella, minä, joka olen näin vanha?'
\par 14 Onko mikään Herralle mahdotonta? Tähän aikaan minä palaan luoksesi tulevana vuonna, ja Saaralla on silloin poika."
\par 15 Ja Saara kielsi sanoen: "En minä nauranut"; sillä hän pelkäsi. Mutta hän sanoi: "Ei ole niin; sinä nauroit".
\par 16 Silloin miehet nousivat siitä ja kääntyivät Sodomaan päin, ja Aabraham kulki heidän kanssaan saattaaksensa heitä.
\par 17 Ja Herra sanoi: "Salaisinko minä Aabrahamilta, mitä olen tekevä?
\par 18 Onhan Aabrahamista tuleva suuri ja väkevä kansa, ja kaikki kansakunnat maan päällä tulevat hänessä siunatuiksi.
\par 19 Sillä minä olen valinnut hänet, että hän käskisi lapsiansa ja perhettänsä, joka jää hänen jälkeensä, noudattamaan Herran tietä ja tekemään sitä, mikä vanhurskaus ja oikeus on, jotta Herra antaisi Aabrahamille tapahtua, mitä hän on hänelle luvannut."
\par 20 Niin Herra sanoi: "Valitushuuto Sodoman ja Gomorran tähden on suuri, ja heidän syntinsä ovat ylen raskaat.
\par 21 Sentähden minä menen alas katsomaan, ovatko he todella tehneet kaiken sen, josta huuto on minun eteeni tullut, vai eivätkö; minä tahdon sen tietää."
\par 22 Ja miehet kääntyivät sieltä ja kulkivat Sodomaan päin, mutta Aabraham jäi vielä seisomaan Herran eteen.
\par 23 Ja Aabraham lähestyi häntä ja sanoi: "Aiotko siis hukuttaa vanhurskaan yhdessä jumalattoman kanssa?
\par 24 Entä jos kaupungissa on viisikymmentä vanhurskasta; aiotko hukuttaa heidät etkä säästä paikkaa siellä olevain viidenkymmenen vanhurskaan tähden?
\par 25 Pois se, että sinä näin tekisit: surmaisit vanhurskaan yhdessä jumalattoman kanssa, niin että vanhurskaan kävisi samoin kuin jumalattoman! Pois se sinusta! Eikö kaiken maan tuomari tekisi oikeutta?"
\par 26 Ja Herra sanoi: "Jos löydän Sodoman kaupungista viisikymmentä vanhurskasta, niin minä heidän tähtensä säästän koko paikan".
\par 27 Aabraham vastasi ja sanoi: "Katso, olen rohjennut puhua Herralleni, vaikka olen tomu ja tuhka.
\par 28 Entä jos viidestäkymmenestä vanhurskaasta puuttuu viisi; hävitätkö viiden tähden koko kaupungin?" Hän sanoi: "En hävitä, jos löydän sieltä neljäkymmentä viisi".
\par 29 Ja hän puhui vielä hänelle sanoen: "Entä jos siellä on neljäkymmentä?" Hän vastasi: "Niiden neljänkymmenen tähden jätän sen tekemättä".
\par 30 Aabraham sanoi: "Älköön Herrani vihastuko, että vielä puhun. Entä jos siellä on kolmekymmentä?" Hän vastasi: "En tee sitä, jos löydän sieltä kolmekymmentä".
\par 31 Mutta hän sanoi: "Katso, minä olen rohjennut puhua Herralleni. Entä jos siellä on kaksikymmentä?" Hän vastasi: "Niiden kahdenkymmenen tähden jätän hävittämättä".
\par 32 Ja hän sanoi: "Älköön Herrani vihastuko, että puhun vielä tämän ainoan kerran. Entä jos siellä on kymmenen?" Hän vastasi: "Niiden kymmenen tähden jätän hävittämättä".
\par 33 Ja Herra lähti pois, senjälkeen kuin hän oli lakannut puhumasta Aabrahamin kanssa, ja Aabraham palasi kotiinsa.

\chapter{19}

\par 1 Ja ne kaksi enkeliä tulivat Sodomaan illalla, ja Loot istui Sodoman portissa; ja nähtyänsä heidät Loot nousi heitä vastaan ja kumartui maahan kasvoillensa.
\par 2 Ja hän sanoi: "Oi herrani, poiketkaa palvelijanne taloon yöksi ja peskää jalkanne! Aamulla varhain voitte jatkaa matkaanne." He sanoivat: "Emme, vaan me jäämme yöksi taivasalle".
\par 3 Mutta hän pyysi heitä pyytämällä, ja he poikkesivat hänen luoksensa ja tulivat hänen taloonsa. Ja hän valmisti heille aterian ja leipoi happamattomia leipiä, ja he söivät.
\par 4 Ennenkuin he olivat laskeutuneet levolle, piirittivät kaupungin miehet, sodomalaiset, sekä nuoret että vanhat, koko kansa kaikkialta, talon.
\par 5 Ja he huusivat Lootia sanoen hänelle: "Missä ne miehet ovat, jotka tulivat luoksesi yöllä? Tuo heidät tänne meidän luoksemme, ryhtyäksemme heihin."
\par 6 Silloin Loot meni ulos heidän luokseen portille ja sulki oven jälkeensä
\par 7 ja sanoi: "Älkää, veljeni, tehkö niin pahoin.
\par 8 Katsokaa, minulla on kaksi tytärtä, jotka eivät vielä miehestä tiedä. Ne minä tuon teille, tehkää heille, mitä tahdotte. Älkää vain tehkö näille miehille mitään pahaa, sillä he ovat tulleet minun kattoni suojaan."
\par 9 Mutta he vastasivat: "Mene tiehesi!" Ja he sanoivat: "Tuo yksi on tullut tänne asumaan muukalaisena, ja yhtäkaikki hän alati pyrkii hallitsemaan. Nytpä me pitelemmekin sinua pahemmin kuin heitä." Ja he tunkeutuivat väkivaltaisesti miehen, Lootin, kimppuun ja kävivät murtamaan ovea.
\par 10 Silloin miehet ojensivat kätensä, vetivät Lootin luoksensa huoneeseen ja sulkivat oven.
\par 11 Ja he sokaisivat ne miehet, jotka olivat talon ovella, sekä nuoret että vanhat, niin että he turhaan koettivat löytää ovea.
\par 12 Ja miehet sanoivat Lootille: "Vieläkö sinulla on ketään omaista täällä? Vie pois täältä vävysi, poikasi, tyttäresi ja kaikki, keitä sinulla kaupungissa on,
\par 13 sillä me hävitämme tämän paikan. Koska huuto heistä on käynyt suureksi Herran edessä, lähetti Herra meidät hävittämään sen."
\par 14 Silloin Loot meni puhuttelemaan vävyjänsä, joiden oli aikomus ottaa hänen tyttärensä, ja sanoi: "Nouskaa, lähtekää pois tästä paikasta, sillä Herra hävittää tämän kaupungin". Mutta hänen vävynsä luulivat hänen laskevan leikkiä.
\par 15 Aamun sarastaessa enkelit kiirehtivät Lootia sanoen: "Nouse, ota vaimosi ja molemmat tyttäresi, jotka ovat luonasi, ettet hukkuisi kaupungin syntivelan tähden".
\par 16 Ja kun hän vielä vitkasteli, tarttuivat miehet hänen käteensä sekä hänen vaimonsa ja molempien tyttäriensä käteen, sillä Herra tahtoi säästää hänet, ja veivät hänet ulos ja jättivät hänet ulkopuolelle kaupunkia.
\par 17 Ja viedessään heitä ulos sanoi mies: "Pakene henkesi tähden, älä katso taaksesi äläkä pysähdy mihinkään lakeudella. Pakene vuorille, ettet hukkuisi."
\par 18 Mutta Loot sanoi heille: "Oi herrani, ei niin!
\par 19 Katso, palvelijasi on saanut armon sinun silmiesi edessä, ja suuri on sinun laupeutesi, jota olet minulle osoittanut pelastaaksesi henkeni, mutta minä en voi päästä pakoon vuorille; pelkään, että onnettomuus saavuttaa minut ja minä kuolen.
\par 20 Katso, tuolla on kaupunki lähellä, vähän matkan päässä, paetakseni sinne; salli minun pelastua sinne - onhan se vähän matkan päässä - jäädäkseni eloon."
\par 21 Ja hän sanoi hänelle: "Katso, minä teen sinulle mieliksi tässäkin kohden; en hävitä kaupunkia, josta puhut.
\par 22 Pakene nopeasti sinne, sillä minä en voi tehdä mitään, ennenkuin olet sinne saapunut." Siitä kaupunki sai nimekseen Sooar.
\par 23 Aurinko oli noussut, kun Loot saapui Sooariin.
\par 24 Ja Herra antoi sataa Sodoman ja Gomorran päälle tulikiveä ja tulta, Herran tyköä taivaasta,
\par 25 ja hävitti nämä kaupungit ynnä koko lakeuden sekä kaikki niiden kaupunkien asukkaat ja maan kasvullisuuden.
\par 26 Ja Lootin vaimo, joka tuli hänen jäljessään, katsoi taaksensa, ja niin hän muuttui suolapatsaaksi.
\par 27 Aabraham nousi varhain aamulla ja meni siihen paikkaan, jossa hän oli seisonut Herran edessä,
\par 28 katseli Sodomaan ja Gomorraan päin ja yli koko lakeuden, ja katso, maasta nousi savu niinkuin pätsin savu.
\par 29 Kun Jumala tuhosi sen lakeuden kaupungit, muisti Jumala Aabrahamia ja johdatti Lootin pois hävityksen keskeltä, hävittäessään ne kaupungit, joissa Loot oli asunut.
\par 30 Ja Loot lähti Sooarista ja asettui vuoristoon molempien tyttäriensä kanssa, sillä hän pelkäsi asua Sooarissa; ja hän asui luolassa, hän ja hänen molemmat tyttärensä.
\par 31 Niin vanhempi sanoi nuoremmalle: "Isämme on vanha, eikä tässä maassa ole ketään miestä, joka voisi tulla luoksemme siten, kuin on kaiken maan tapa.
\par 32 Tule, juottakaamme isällemme viiniä ja maatkaamme hänen kanssaan, saadaksemme isästämme jälkeläisen."
\par 33 Niin he juottivat sinä yönä isällensä viiniä. Ja vanhempi meni ja makasi hänen kanssaan, eikä tämä huomannut, milloin hän tuli hänen viereensä ja milloin hän nousi.
\par 34 Seuraavana päivänä sanoi vanhempi nuoremmalle: "Katso, minä makasin viime yönä isäni kanssa; juottakaamme hänelle tänäkin yönä viiniä, ja mene sinä ja makaa hänen kanssaan, saadaksemme isästämme jälkeläisen".
\par 35 Niin he juottivat sinäkin yönä isällensä viiniä; ja nuorempi meni ja makasi hänen kanssaan, eikä tämä huomannut, milloin hän tuli hänen viereensä ja milloin hän nousi.
\par 36 Ja niin Lootin molemmat tyttäret tulivat isästänsä raskaiksi.
\par 37 Ja vanhempi synnytti pojan ja antoi hänelle nimen Mooab; hänestä polveutuvat mooabilaiset aina tähän päivään saakka.
\par 38 Ja myöskin nuorempi synnytti pojan ja antoi hänelle nimen Ben-Ammi; hänestä polveutuvat ammonilaiset aina tähän päivään saakka.

\chapter{20}

\par 1 Ja Aabraham siirtyi sieltä Etelämaahan ja asettui Kaadeksen ja Suurin välimaille; ja hän asui jonkun aikaa Gerarissa.
\par 2 Ja Aabraham sanoi vaimostansa Saarasta: "Hän on minun sisareni". Niin Abimelek, Gerarin kuningas, lähetti noutamaan Saaran luoksensa.
\par 3 Mutta Jumala tuli Abimelekin tykö yöllä unessa ja sanoi hänelle: "Katso, sinun on kuoltava sen vaimon tähden, jonka olet ottanut, sillä hän on toisen miehen aviovaimo".
\par 4 Mutta Abimelek ei ollut ryhtynyt häneen, ja hän sanoi: "Herra, surmaatko syyttömänkin?
\par 5 Eikö hän itse sanonut minulle: 'Hän on minun sisareni'? Ja samoin tämäkin sanoi: 'Hän on minun veljeni'. Olen tehnyt tämän vilpittömin sydämin ja viattomin käsin."
\par 6 Ja Jumala sanoi hänelle unessa: "Niin, minä tiedän, että sinä teit sen vilpittömin sydämin. Sentähden minä estinkin sinut tekemästä syntiä minua vastaan enkä sallinut sinun kajota häneen.
\par 7 Niin anna nyt miehelle hänen vaimonsa takaisin, sillä hän on profeetta, ja hän on rukoileva sinun puolestasi, että saisit elää. Mutta jollet anna häntä takaisin, niin tiedä, että olet kuolemalla kuoleva, sinä ja kaikki sinun omaisesi."
\par 8 Niin Abimelek nousi varhain aamulla ja kutsui kaikki palvelijansa ja kertoi heille tämän kaiken; ja miehet olivat kovin peloissaan.
\par 9 Ja Abimelek kutsui Aabrahamin ja sanoi hänelle: "Mitä oletkaan meille tehnyt! Mitä minä olen rikkonut sinua vastaan, koska olet saattanut minut ja minun valtakuntani syypääksi näin suureen rikokseen? Sinä olet tehnyt minulle, mitä ei sovi tehdä."
\par 10 Ja Abimelek sanoi vielä Aabrahamille: "Mikä oli tarkoituksesi, kun tämän teit?"
\par 11 Aabraham vastasi: "Minä ajattelin näin: 'Tällä paikkakunnalla ei varmaankaan peljätä Jumalaa; he surmaavat minut vaimoni tähden'.
\par 12 Ja hän onkin todella minun sisareni, isäni tytär, vaikkei olekaan äitini tytär; ja niin hänestä tuli minun vaimoni.
\par 13 Mutta kun Jumala lähetti minut kulkemaan pois isäni kodista, sanoin minä hänelle: 'Osoita minulle se rakkaus, että, mihin paikkaan ikinä tulemmekin, sanot minusta: hän on minun veljeni'."
\par 14 Silloin Abimelek otti pikkukarjaa ja raavaskarjaa, palvelijoita ja palvelijattaria ja antoi ne Aabrahamille. Ja hän lähetti hänelle takaisin hänen vaimonsa Saaran.
\par 15 Ja Abimelek sanoi: "Katso, minun maani on avoinna edessäsi; asu, missä mielesi tekee!"
\par 16 Ja Saaralle hän sanoi: "Katso, minä annan veljellesi tuhat sekeliä hopeata. Olkoot ne sinulle hyvitykseksi kaikkien niiden edessä, jotka sinun kanssasi ovat; niin olet kaikkien edessä todistettu viattomaksi."
\par 17 Mutta Aabraham rukoili Jumalaa, ja Jumala paransi Abimelekin ja hänen vaimonsa ja hänen orjattarensa, niin että he synnyttivät lapsia.
\par 18 Sillä Herra oli sulkenut jokaisen kohdun Abimelekin huoneessa Saaran, Aabrahamin vaimon, tähden.

\chapter{21}

\par 1 Ja Herra piti Saarasta huolen, niinkuin oli luvannut; ja Herra teki Saaralle, niinkuin oli puhunut.
\par 2 Ja Saara tuli raskaaksi ja synnytti Aabrahamille pojan hänen vanhoilla päivillään, juuri sinä aikana, jonka Jumala oli hänelle sanonut.
\par 3 Ja Aabraham nimitti poikansa, joka hänelle oli syntynyt, sen, jonka Saara oli hänelle synnyttänyt, Iisakiksi.
\par 4 Ja Aabraham ympärileikkasi poikansa Iisakin, tämän ollessa kahdeksan päivän vanha, niinkuin Jumala oli hänen käskenyt tehdä.
\par 5 Aabraham oli sadan vuoden vanha, kun hänen poikansa Iisak syntyi hänelle.
\par 6 Ja Saara sanoi: "Jumala on saattanut minut naurunalaiseksi; kuka ikinä saa tämän kuulla, se nauraa minulle".
\par 7 Ja hän sanoi vielä: "Kuka olisi tiennyt sanoa Aabrahamille: Saara on imettävä lapsia? Ja nyt minä kuitenkin olen synnyttänyt hänelle pojan hänen vanhoilla päivillään."
\par 8 Ja poika kasvoi, ja hänet vieroitettiin. Ja Aabraham laittoi suuret pidot siksi päiväksi, jona Iisak vieroitettiin.
\par 9 Ja Saara näki egyptiläisen Haagarin pojan, jonka tämä oli Aabrahamille synnyttänyt, ilvehtivän
\par 10 ja sanoi Aabrahamille: "Aja pois tuo orjatar poikinensa, sillä ei tuon orjattaren poika saa periä minun poikani, Iisakin, kanssa".
\par 11 Aabraham pahastui suuresti tästä puheesta poikansa tähden.
\par 12 Mutta Jumala sanoi Aabrahamille: "Älä pahastu siitä poikasi ja orjattaresi tähden. Kuule Saaraa kaikessa, mitä hän sinulle sanoo, sillä ainoastaan Iisakista sinä saat nimellesi jälkeläiset.
\par 13 Mutta myöskin orjattaren pojasta minä teen suuren kansan, koska hän on sinun jälkeläisesi."
\par 14 Varhain seuraavana aamuna Aabraham otti leipää ja vesileilin ja antoi ne Haagarille, pannen ne hänen olalleen, sekä pojan, ja lähetti hänet menemään. Hän lähti ja harhaili Beerseban erämaassa.
\par 15 Mutta kun vesi loppui leilistä, heitti hän pojan pensaan alle,
\par 16 meni ja istui syrjään jousenkantaman päähän, sillä hän ajatteli: "En voi nähdä pojan kuolevan". Ja istuessaan siinä syrjässä hän korotti äänensä ja itki.
\par 17 Silloin Jumala kuuli pojan valituksen, ja Jumalan enkeli huusi taivaasta Haagarille sanoen: "Mikä sinun on, Haagar? Älä pelkää, sillä Jumala on kuullut pojan valituksen, siinä missä hän makaa.
\par 18 Nouse, nosta poika maasta ja tartu hänen käteensä, sillä minä teen hänestä suuren kansan."
\par 19 Ja Jumala avasi hänen silmänsä, niin että hän huomasi vesikaivon. Ja hän meni ja täytti leilin vedellä ja antoi pojan juoda.
\par 20 Ja Jumala oli pojan kanssa, ja hän kasvoi ja asui erämaassa, ja hänestä tuli jousimies.
\par 21 Ja hän asui Paaranin erämaassa; ja hänen äitinsä otti hänelle vaimon Egyptin maasta.
\par 22 Siihen aikaan puhui Abimelek ja hänen sotapäällikkönsä Piikol Aabrahamille sanoen: "Jumala on sinun kanssasi kaikessa, mitä teet.
\par 23 Vanno nyt tässä minulle Jumalan kautta, ettet ole oleva petollinen minulle etkä minun suvulleni etkä jälkeläisilleni, vaan tee sinäkin laupeus minulle ja sille maalle, jossa muukalaisena asut, niinkuin minä olen sinulle tehnyt."
\par 24 Aabraham sanoi: "Minä vannon".
\par 25 Aabraham nuhteli kuitenkin Abimelekia vesikaivon tähden, jonka Abimelekin palvelijat olivat vallanneet.
\par 26 Mutta Abimelek vastasi: "En tiedä, kuka sen on tehnyt; et ole itse minulle mitään ilmoittanut, enkä ole siitä kuullut ennen kuin tänään".
\par 27 Silloin Aabraham otti pikkukarjaa ja raavaskarjaa ja antoi Abimelekille; ja he tekivät molemmat keskenänsä liiton.
\par 28 Ja Aabraham asetti laumasta seitsemän uuhikaritsaa erilleen muista.
\par 29 Silloin Abimelek sanoi Aabrahamille: "Mitä tarkoittavat nuo seitsemän karitsaa, jotka olet asettanut tuonne erilleen?"
\par 30 Hän vastasi: "Nämä seitsemän karitsaa on sinun otettava minun kädestäni, todistukseksi minulle siitä, että tämä kaivo on minun kaivamani".
\par 31 Siitä kutsuttiin paikkaa Beersebaksi, koska he molemmat vannoivat siinä toisilleen valan.
\par 32 Niin he tekivät liiton Beersebassa. Ja Abimelek nousi ja Piikol, hänen sotapäällikkönsä, ja he palasivat filistealaisten maahan.
\par 33 Ja Aabraham istutti tamariskipuun Beersebaan ja huusi siinä avuksi Herran, iankaikkisen Jumalan, nimeä.
\par 34 Ja Aabraham asui kauan muukalaisena filistealaisten maassa.

\chapter{22}

\par 1 Näiden tapausten jälkeen Jumala koetteli Aabrahamia ja sanoi hänelle: "Aabraham!" Hän vastasi: "Tässä olen".
\par 2 Ja hän sanoi: "Ota Iisak, ainokainen poikasi, jota rakastat, ja mene Moorian maahan ja uhraa hänet siellä polttouhriksi vuorella, jonka minä sinulle sanon".
\par 3 Varhain seuraavana aamuna Aabraham satuloi aasinsa ja otti mukaansa kaksi palvelijaansa sekä poikansa Iisakin, ja halottuaan polttouhripuita hän lähti menemään siihen paikkaan, jonka Jumala oli hänelle sanonut.
\par 4 Kolmantena päivänä Aabraham nosti silmänsä ja näki sen paikan kaukaa.
\par 5 Silloin Aabraham sanoi palvelijoilleen: "Jääkää tähän aasin kanssa. Minä ja poika menemme tuonne rukoilemaan; sitten me palaamme luoksenne."
\par 6 Ja Aabraham otti polttouhripuut ja sälytti ne poikansa Iisakin selkään; itse hän otti käteensä tulen ja veitsen, ja niin he astuivat molemmat yhdessä.
\par 7 Iisak puhui isällensä Aabrahamille sanoen: "Isäni!" Tämä vastasi: "Tässä olen, poikani". Ja hän sanoi: "Katso, tässä on tuli ja halot, mutta missä on lammas polttouhriksi?"
\par 8 Aabraham vastasi: "Jumala on katsova itselleen lampaan polttouhriksi, poikani". Ja he astuivat molemmat yhdessä.
\par 9 Ja kun he olivat tulleet siihen paikkaan, jonka Jumala oli hänelle sanonut, rakensi Aabraham siihen alttarin, latoi sille halot, sitoi poikansa Iisakin ja pani hänet alttarille halkojen päälle.
\par 10 Ja Aabraham ojensi kätensä ja tarttui veitseen teurastaakseen poikansa.
\par 11 Silloin Herran enkeli huusi hänelle taivaasta sanoen: "Aabraham, Aabraham!" Hän vastasi: "Tässä olen".
\par 12 Niin hän sanoi: "Älä satuta kättäsi poikaan äläkä tee hänelle mitään, sillä nyt minä tiedän, että sinä pelkäät Jumalaa, kun et kieltänyt minulta ainokaista poikaasi".
\par 13 Niin Aabraham nosti silmänsä ja huomasi takanansa oinaan, joka oli sarvistaan takertunut pensaikkoon. Ja Aabraham meni, otti oinaan ja uhrasi sen polttouhriksi poikansa sijasta.
\par 14 Ja Aabraham pani sen paikan nimeksi "Herra näkee". Niinpä vielä tänä päivänä sanotaan: "Vuorella, missä Herra ilmestyy".
\par 15 Ja Herran enkeli huusi Aabrahamille toistamiseen taivaasta
\par 16 ja sanoi: "Minä vannon itse kauttani, sanoo Herra: Sentähden että tämän teit etkä kieltänyt minulta ainokaista poikaasi,
\par 17 minä runsaasti siunaan sinua ja teen sinun jälkeläistesi luvun paljoksi kuin taivaan tähdet ja hiekka, joka on meren rannalla, ja sinun jälkeläisesi valtaavat vihollistensa portit.
\par 18 Ja sinun siemenessäsi tulevat siunatuiksi kaikki kansakunnat maan päällä, sentähden että olit minun äänelleni kuuliainen."
\par 19 Sitten Aabraham palasi palvelijainsa luo, ja he nousivat ja kulkivat yhdessä Beersebaan. Ja Aabraham jäi asumaan Beersebaan.
\par 20 Näiden tapausten jälkeen ilmoitettiin Aabrahamille: "Katso, myöskin Milka on synnyttänyt poikia sinun veljellesi Naahorille".
\par 21 Nämä olivat Uus, hänen esikoisensa, tämän veli Buus, Kemuel, Aramin isä,
\par 22 Kesed, Haso, Pildas, Jidlaf ja Betuel;
\par 23 ja Betuelille syntyi Rebekka. Nämä kahdeksan synnytti Milka Naahorille, Aabrahamin veljelle.
\par 24 Ja hänen sivuvaimonsa, nimeltä Reuma, synnytti myös lapsia: Tebahin, Gahamin, Tahaan ja Maakan.

\chapter{23}

\par 1 Ja Saara eli sadan kahdenkymmenen seitsemän vuoden vanhaksi; niin vanhaksi eli Saara.
\par 2 Ja Saara kuoli Kirjat-Arbassa, se on Hebronissa, Kanaanin maassa. Ja Aabraham meni murehtimaan Saaraa ja itkemään häntä.
\par 3 Senjälkeen Aabraham nousi ja lähti vainajan luota ja puhui heettiläisille sanoen:
\par 4 "Minä olen muukalainen ja vieras teidän keskuudessanne; antakaa minulle perintöhauta maassanne, haudatakseni ja kätkeäkseni siihen vainajani".
\par 5 Heettiläiset vastasivat Aabrahamille, sanoen hänelle:
\par 6 "Kuule meitä, herra. Sinä olet Jumalan ruhtinas meidän keskuudessamme; hautaa vainajasi parhaaseen hautaamme. Ei kukaan meistä kiellä sinua hautaamasta vainajaasi hautaansa."
\par 7 Niin Aabraham nousi ja kumarsi maan kansalle, heettiläisille,
\par 8 ja puhui heille sanoen: "Jos teidän mieleenne on, että minä hautaan ja kätken vainajani, niin kuulkaa minua ja taivuttakaa Efron, Sooharin poika,
\par 9 antamaan minulle Makpelan luola, joka on hänen omansa ja on hänen vainionsa perällä. Täydestä hinnasta hän antakoon sen minulle perintöhaudaksi teidän keskuudessanne."
\par 10 Mutta Efron istui siellä heettiläisten joukossa. Ja Efron, heettiläinen, vastasi Aabrahamille heettiläisten kuullen, kaikkien, jotka kulkivat hänen kaupunkinsa portista, sanoen:
\par 11 "Ei, herrani, vaan kuule minua. Vainion minä lahjoitan sinulle ja myöskin luolan, joka siinä on, minä lahjoitan sinulle; kansalaisteni nähden minä sen sinulle lahjoitan. Hautaa vainajasi."
\par 12 Ja Aabraham kumarsi maan kansalle
\par 13 ja puhui Efronille, maan kansan kuullen, sanoen: "Jospa kuitenkin - oi, kuule minua! Minä maksan vainion hinnan; ota se minulta ja anna minun haudata siihen vainajani."
\par 14 Efron vastasi Aabrahamille, sanoen hänelle:
\par 15 "Herrani, kuule minua. Neljänsadan hopeasekelin maa, mitä se minulle ja sinulle merkitsee? Hautaa vainajasi."
\par 16 Kuultuaan Efronin sanat Aabraham punnitsi Efronille sen rahasumman, jonka tämä oli maininnut heettiläisten kuullen, neljäsataa hopeasekeliä, kaupassa käypää.
\par 17 Niin Efronin vainio, joka on Makpelassa itään päin Mamresta, sekä vainio että siellä oleva luola ynnä kaikki puut, jotka kasvoivat vainiolla, koko sillä alueella,
\par 18 joutuivat Aabrahamin omaksi kaikkien heettiläisten nähden, jotka kulkivat hänen kaupunkinsa portista.
\par 19 Senjälkeen Aabraham hautasi vaimonsa Saaran luolaan, joka on Makpelan vainiolla, itään päin Mamresta, se on Hebronista, Kanaanin maassa.
\par 20 Niin vainio ja siellä oleva luola siirtyi heettiläisiltä Aabrahamille, perintöhaudaksi.

\chapter{24}

\par 1 Ja Aabraham oli vanha ja iälliseksi tullut, ja Herra oli siunannut Aabrahamia kaikessa.
\par 2 Niin Aabraham sanoi palvelijallensa, talonsa vanhimmalle, joka hallitsi kaikkea, mitä hänellä oli: "Pane kätesi kupeeni alle.
\par 3 Minä vannotan sinua Herran, taivaan ja maan Jumalan, kautta, ettet ota pojalleni vaimoa kanaanilaisten tyttäristä, joiden keskuudessa minä asun,
\par 4 vaan menet minun omaan maahani ja sukuni luo ja otat sieltä vaimon pojalleni Iisakille."
\par 5 Palvelija sanoi hänelle: "Entä jos tyttö ei tahdo seurata minua tähän maahan, onko minun silloin vietävä poikasi takaisin siihen maahan, josta olet lähtenyt?"
\par 6 Aabraham vastasi hänelle: "Varo, ettet vie poikaani takaisin sinne.
\par 7 Herra, taivaan Jumala, joka otti minut pois isäni kodista ja synnyinmaastani, hän, joka puhui minulle ja vannoi minulle sanoen: 'Sinun jälkeläisillesi minä annan tämän maan', hän lähettää enkelinsä sinun edelläsi, niin että saat sieltä vaimon pojalleni.
\par 8 Mutta jos tyttö ei tahdo seurata sinua, niin olet tästä valasta vapaa. Älä vain vie poikaani sinne takaisin."
\par 9 Niin palvelija pani kätensä herransa Aabrahamin kupeen alle ja lupasi sen hänelle valalla vannoen.
\par 10 Ja palvelija otti herransa kameleista kymmenen sekä kaikkinaisia kalleuksia herransa tavaroista; ja hän nousi ja lähti Mesopotamiaan, Naahorin kaupunkiin.
\par 11 Ja hän antoi kamelien asettua kaupungin ulkopuolelle vesikaivon ääreen lepäämään, illan suussa, jolloin naiset tulivat vettä ammentamaan.
\par 12 Ja hän sanoi: "Herra, minun herrani Aabrahamin Jumala, suo minulle tänään menestystä ja tee laupeus herralleni Aabrahamille.
\par 13 Katso, minä seison tässä lähteellä, ja kaupunkilaisten tyttäret tulevat ammentamaan vettä.
\par 14 Niin tapahtukoon, että se tyttö, jolle minä sanon: 'Kallista tänne vesiastiaasi juodakseni', ja joka vastaa: 'Juo, minä juotan kamelisikin', on juuri se, jonka sinä olet määrännyt palvelijallesi Iisakille; ja siitä minä tiedän, että olet tehnyt laupeuden herralleni."
\par 15 Tuskin hän oli lakannut puhumasta, niin tuli siihen Rebekka, joka oli Aabrahamin veljen Naahorin ja hänen vaimonsa Milkan pojan, Betuelin, tytär; ja hänellä oli vesiastia olallaan.
\par 16 Ja tyttö oli näöltään hyvin ihana, neitsyt, johon ei mies ollut koskenut. Ja hän astui alas lähteelle, täytti astiansa ja tuli ylös.
\par 17 Silloin palvelija riensi häntä vastaan ja sanoi: "Anna minun juoda vähän vettä astiastasi".
\par 18 Hän vastasi: "Juo, herrani". Ja hän laski nopeasti astiansa alas kädelleen ja antoi hänen juoda.
\par 19 Ja annettuaan hänen juoda hän sanoi: "Minä ammennan vettä myös sinun kameleillesi, kunnes ne ovat juoneet kyllikseen".
\par 20 Niin hän tyhjensi nopeasti astiansa kaukaloon ja riensi jälleen kaivolle ammentamaan vettä ja ammensi kaikille hänen kameleilleen.
\par 21 Ja mies katseli häntä ääneti saadakseen tietää, oliko Herra antanut hänen matkansa onnistua vai ei.
\par 22 Kun kamelit olivat juoneet, otti mies kultaisen nenärenkaan, joka painoi puolen sekeliä, ja hänen käsiinsä kaksi rannerengasta, jotka painoivat kymmenen kultasekeliä.
\par 23 Ja hän kysyi: "Kenenkä tytär olet? Sano minulle. Onko isäsi talossa tilaa yötä ollaksemme?"
\par 24 Hän vastasi hänelle: "Olen Betuelin, Naahorin ja Milkan pojan, tytär".
\par 25 Ja hän sanoi hänelle vielä: "Meillä on runsaasti olkia ja rehua; myöskin yösijaa on meillä antaa".
\par 26 Silloin mies kumartui maahan ja rukoili Herraa
\par 27 ja sanoi: "Kiitetty olkoon Herra, minun herrani Aabrahamin Jumala, joka ei ole ottanut pois armoansa ja totuuttansa minun herraltani. Herra on johdattanut minut tätä tietä herrani veljen kotiin."
\par 28 Mutta tyttö riensi ilmoittamaan äitinsä perheelle, mitä oli tapahtunut.
\par 29 Rebekalla oli veli, jonka nimi oli Laaban; ja Laaban riensi ulos miehen luo lähteelle.
\par 30 Sillä kun hän näki nenärenkaan ja rannerenkaat sisarensa käsissä ja kuuli sisarensa Rebekan kertovan ja sanovan: "Näin mies puhui minulle", meni hän miehen luo; ja katso, hän seisoi vielä kamelien luona lähteellä.
\par 31 Ja hän sanoi hänelle: "Tule sisään, sinä Herran siunattu. Minkätähden seisot ulkona? Minä olen valmistanut tilaa talossa ja sijaa kameleille."
\par 32 Niin mies tuli taloon, ja Laaban riisui kamelit ja antoi kameleille olkia ja rehua sekä hänelle ja hänen seuralaisilleen vettä jalkojen pesemiseksi.
\par 33 Sitten pantiin ruokaa hänen eteensä, mutta hän sanoi: "En syö, ennenkuin olen puhunut asiani". Laaban vastasi: "Puhu".
\par 34 Hän sanoi: "Minä olen Aabrahamin palvelija.
\par 35 Herra on suuresti siunannut minun herraani, niin että hänestä on tullut mahtava mies; hän on antanut hänelle pikkukarjaa ja raavaskarjaa, hopeata ja kultaa, palvelijoita ja palvelijattaria, kameleja ja aaseja.
\par 36 Ja Saara, herrani vaimo, on vanhalla iällänsä synnyttänyt herralleni pojan; ja tälle hän on antanut kaiken omaisuutensa.
\par 37 Ja herrani vannotti minua sanoen: 'Älä ota pojalleni vaimoa kanaanilaisten tyttäristä, joiden maassa minä asun,
\par 38 vaan mene minun isäni kotiin ja sukuni luo ja ota sieltä pojalleni vaimo'.
\par 39 Silloin minä sanoin herralleni: 'Entä jos tyttö ei seuraa minua?'
\par 40 Hän vastasi minulle: 'Herra, jonka edessä minä olen vaeltanut, lähettää enkelinsä sinun kanssasi ja antaa matkasi onnistua, niin että saat pojalleni vaimon minun suvustani ja isäni perheestä.
\par 41 Silloin, kun saavut sukuni luo, olet vapaa valasta, jonka minulle vannoit; jos he eivät häntä sinulle anna, olet valasta vapaa'.
\par 42 Niin minä tänään tulin lähteelle ja sanoin: 'Herra, minun herrani Aabrahamin Jumala, jos annat onnistua sen matkan, jolla olen,
\par 43 salli tapahtua niin, että se neitonen, joka minun seisoessani tässä vesilähteellä tulee ammentamaan vettä ja sanoessani hänelle: 'Anna minun juoda vähän vettä astiastasi',
\par 44 vastaa minulle: 'Juo itse, ja minä ammennan myös sinun kameleillesi' - että hän on se vaimo, jonka Herra on määrännyt minun herrani pojalle.
\par 45 Tuskin olin lakannut näin puhumasta itsekseni, niin katso, Rebekka tuli sinne, vesiastia olallansa, astui alas lähteelle ja ammensi. Ja minä sanoin hänelle: 'Anna minun juoda'.
\par 46 Hän laski nopeasti astian alas olaltansa ja sanoi: 'Juo, minä juotan myös sinun kamelisi'. Niin minä join, ja hän juotti myös kamelit.
\par 47 Ja minä kysyin häneltä sanoen: 'Kenenkä tytär sinä olet?' Hän vastasi: 'Olen Betuelin, Naahorin ja Milkan pojan, tytär'. Niin minä panin nenärenkaan hänen nenäänsä ja rannerenkaat hänen käsiinsä.
\par 48 Ja minä kumarruin maahan ja rukoilin Herraa, kiittäen Herraa, minun herrani Aabrahamin Jumalaa, joka oli johdattanut minut oikeata tietä saamaan herrani veljen tyttären hänen pojalleen.
\par 49 Ja jos nyt tahdotte osoittaa suosiota ja uskollisuutta minun herralleni, niin ilmoittakaa se minulle; jollette, niin ilmoittakaa minulle sekin, kääntyäkseni toisaalle, oikealle tai vasemmalle."
\par 50 Laaban ja Betuel vastasivat sanoen: "Herralta tämä on tullut; emme voi tässä asiassa puhua sinulle hyvää emmekä pahaa.
\par 51 Katso, siinä on Rebekka edessäsi, ota hänet ja mene. Tulkoon hän herrasi pojan vaimoksi, niinkuin Herra on sanonut."
\par 52 Kuultuaan heidän sanansa Aabrahamin palvelija kumartui maahan Herran eteen.
\par 53 Sitten palvelija otti esille hopea- ja kultakaluja sekä vaatteita ja antoi ne Rebekalle. Myöskin hänen veljelleen ja äidilleen hän antoi kallisarvoisia lahjoja.
\par 54 Ja he söivät ja joivat, hän ja hänen seuralaisensa, ja olivat siellä yötä. Mutta kun he olivat nousseet seuraavana aamuna, sanoi hän: "Päästäkää minut menemään herrani luo".
\par 55 Tytön veli ja äiti vastasivat: "Anna tytön viipyä luonamme vielä joku aika, edes kymmenen päivää. Sitten saat lähteä."
\par 56 Mutta hän sanoi heille: "Älkää viivyttäkö minua, koska Herra on antanut matkani onnistua. Päästäkää minut menemään, tahdon lähteä herrani luo."
\par 57 He vastasivat: "Kutsukaamme tänne tyttö ja kysykäämme häneltä itseltään".
\par 58 Ja he kutsuivat Rebekan ja sanoivat hänelle: "Tahdotko lähteä tämän miehen kanssa?" Hän vastasi: "Tahdon".
\par 59 Niin he lähettivät sisarensa Rebekan imettäjineen matkalle Aabrahamin palvelijan ja hänen miestensä mukana.
\par 60 Ja he siunasivat Rebekan ja lausuivat hänelle: "Oi sisaremme, tulkoon sinusta tuhat kertaa kymmenentuhatta, ja vallatkoot sinun jälkeläisesi vihamiestensä portit!"
\par 61 Ja Rebekka nousi palvelijattarineen, ja he istuivat kamelien selkään ja seurasivat miestä. Niin palvelija otti Rebekan mukaansa ja lähti matkalle.
\par 62 Ja Iisak oli tulossa Lahai-Roin kaivon tienoilta; hän asui näet Etelämaassa.
\par 63 Ja Iisak oli illan suussa lähtenyt kedolle käyskentelemään, ja kun hän nosti silmänsä, näki hän kamelien lähestyvän.
\par 64 Kun Rebekka nosti silmänsä ja näki Iisakin, laskeutui hän nopeasti maahan kamelin selästä
\par 65 ja kysyi palvelijalta: "Kuka on tuo mies, joka tulee kedolla meitä vastaan?" Palvelija vastasi: "Hän on minun herrani". Niin hän otti hunnun ja verhoutui siihen.
\par 66 Ja palvelija kertoi Iisakille kaikki, mitä hän oli toimittanut.
\par 67 Ja Iisak vei Rebekan äitinsä Saaran majaan ja otti hänet luokseen, ja hänestä tuli hänen vaimonsa, ja hän rakasti häntä. Niin Iisak sai lohdutuksen äitinsä kuoltua.

\chapter{25}

\par 1 Ja Aabraham otti vielä vaimon, ja hänen nimensä oli Ketura.
\par 2 Ja hän synnytti hänelle Simranin ja Joksanin, Medanin ja Midianin, Jisbakin ja Suuahin.
\par 3 Mutta Joksanille syntyi Seba ja Dedan. Dedanin jälkeläisiä olivat: assurilaiset, letusilaiset ja leummilaiset.
\par 4 Ja Midianin pojat olivat Eefa, Eefer, Hanok, Abida ja Eldaa. Kaikki nämä ovat Keturan jälkeläisiä.
\par 5 Ja Aabraham antoi kaiken omaisuutensa Iisakille.
\par 6 Mutta sivuvaimojensa pojille Aabraham antoi lahjoja; ja hän lähetti heidät vielä eläessänsä pois poikansa Iisakin luota itään päin, Itäiselle maalle.
\par 7 Tämä on Aabrahamin elinvuosien luku: sata seitsemänkymmentä viisi vuotta.
\par 8 Ja Aabraham vaipui kuolemaan korkeassa iässä, vanhana ja elämästä kyllänsä saaneena, ja tuli otetuksi heimonsa tykö.
\par 9 Ja hänen poikansa Iisak ja Ismael hautasivat hänet Makpelan luolaan, heettiläisen Efronin, Sooharin pojan, vainiolle, joka on itään päin Mamresta,
\par 10 sille vainiolle, jonka Aabraham oli ostanut heettiläisiltä; siihen haudattiin Aabraham ja hänen vaimonsa Saara.
\par 11 Ja Aabrahamin kuoltua Jumala siunasi hänen poikaansa Iisakia. Ja Iisak asui Lahai-Roin kaivon tienoilla.
\par 12 Ja tämä on kertomus Ismaelin suvusta, Aabrahamin pojan, jonka Saaran egyptiläinen orjatar Haagar synnytti Aabrahamille.
\par 13 Nämä ovat Ismaelin poikien nimet heidän nimiensä ja polveutumisensa mukaan: Nebajot, Ismaelin esikoinen, Keedar, Adbeel, Mibsam,
\par 14 Misma, Duuma, Massa,
\par 15 Hadad, Teema, Jetur, Naafis ja Keedma.
\par 16 Nämä ovat Ismaelin pojat ja nämä heidän nimensä heidän kyliensä ja leiripaikkojensa mukaan, kaksitoista ruhtinasta heimokuntineen.
\par 17 Ja tämä on Ismaelin elinvuosien luku: sata kolmekymmentä seitsemän vuotta; ja hän vaipui kuolemaan ja tuli otetuksi heimonsa tykö.
\par 18 Ja he asuivat Havilasta aina Suuriin asti, joka on Egyptistä itään päin Assyriaan mentäessä. Hän kävi kaikkien veljiensä kimppuun.
\par 19 Ja tämä on kertomus Iisakin, Aabrahamin pojan, suvusta. Aabrahamille syntyi Iisak.
\par 20 Ja Iisak oli neljänkymmenen vuoden vanha, kun hän otti vaimokseen Rebekan, joka oli aramilaisen Betuelin tytär Mesopotamiasta ja aramilaisen Laabanin sisar.
\par 21 Ja Iisak rukoili Herraa vaimonsa puolesta, sillä tämä oli hedelmätön. Ja Herra kuuli hänen rukouksensa, ja hänen vaimonsa Rebekka tuli raskaaksi.
\par 22 Ja lapset sysäsivät toisiaan hänen kohdussansa. Niin hän sanoi: "Jos näin käy, minkätähden minä elän?" Ja hän meni kysymään Herralta.
\par 23 Ja Herra sanoi hänelle: "Kaksi kansaa on sinun kohdussasi, kaksi heimoa erkanee sinun ruumiistasi, toinen heimo on toista voimakkaampi, vanhempi palvelee nuorempaa".
\par 24 Kun hänen synnyttämisensä aika oli tullut, katso, hänen kohdussaan oli kaksoiset.
\par 25 Joka ensiksi tuli hänen kohdustaan, oli ruskea ja yliyltään niinkuin karvainen vaippa; sentähden pantiin hänelle nimeksi Eesau.
\par 26 Senjälkeen tuli ulos hänen veljensä, ja hän piti kädellään Eesaun kantapäästä. Ja hänelle pantiin nimeksi Jaakob. Iisak oli kuudenkymmenen vuoden vanha heidän syntyessänsä.
\par 27 Ja pojat kasvoivat suuriksi, ja Eesausta tuli taitava metsästäjä, aron mies; Jaakob sitä vastoin oli hiljainen mies, joka pysyi kotosalla.
\par 28 Iisak rakasti enemmän Eesauta, sillä hän söi mielellänsä metsänriistaa, mutta Rebekka rakasti enemmän Jaakobia.
\par 29 Kerran, kun Jaakob oli keittänyt itselleen keiton, tuli Eesau kedolta nälästä nääntyneenä.
\par 30 Ja Eesau sanoi Jaakobille: "Anna minun särpiä tuota ruskeata, tuota ruskeata keittoa, sillä minä olen nälästä nääntynyt". Sentähden hän sai nimen Edom.
\par 31 Mutta Jaakob sanoi: "Myy minulle ensin esikoisuutesi".
\par 32 Eesau vastasi: "Katso, minä kuolen kuitenkin, mitä minä esikoisuudellani teen?"
\par 33 Jaakob sanoi: "Vanno minulle ensin". Ja hän vannoi hänelle ja myi esikoisuutensa Jaakobille.
\par 34 Ja Jaakob antoi Eesaulle leipää ja hernekeittoa. Ja hän söi ja joi, nousi ja meni matkoihinsa. Niin halpana Eesau piti esikoisuutensa.

\chapter{26}

\par 1 Mutta maahan tuli nälänhätä, toinen kuin se nälänhätä, joka oli ollut aikaisemmin, Aabrahamin päivinä. Silloin Iisak meni Abimelekin, filistealaisten kuninkaan, luo Gerariin.
\par 2 Ja Herra ilmestyi hänelle ja sanoi: "Älä mene Egyptiin, vaan jää asumaan maahan, jonka minä sinulle sanon.
\par 3 Oleskele muukalaisena tässä maassa, ja minä olen sinun kanssasi ja siunaan sinua. Sillä sinulle ja sinun jälkeläisillesi minä annan kaikki nämä maat ja pidän valan, jonka minä olen vannonut sinun isällesi Aabrahamille.
\par 4 Ja minä teen sinun jälkeläistesi luvun paljoksi kuin taivaan tähdet ja annan jälkeläisillesi kaikki nämä maat, ja sinun siemenessäsi tulevat siunatuiksi kaikki kansakunnat maan päällä,
\par 5 sentähden että Aabraham kuuli minua ja noudatti, mitä minä noudatettavaksi annoin, minun käskyjäni, säädöksiäni ja opetuksiani."
\par 6 Niin Iisak asettui Gerariin.
\par 7 Ja kun sen paikkakunnan miehet kysyivät hänen vaimostansa, sanoi hän: "Hän on minun sisareni". Hän näet pelkäsi sanoa: "Hän on minun vaimoni", sillä hän ajatteli: "Nuo miehet voivat tappaa minut Rebekan tähden, sillä hän on näöltään kaunis".
\par 8 Kun hän oli viipynyt siellä jonkun aikaa, niin tapahtui, että Abimelek, filistealaisten kuningas, kerran katsellessaan ulos akkunasta näki Iisakin hyväilevän vaimoansa.
\par 9 Niin Abimelek kutsutti Iisakin ja sanoi: "Katso, hänhän on sinun vaimosi! Miksi olet sanonut: 'Hän on minun sisareni'?" Iisak vastasi hänelle: "Minä ajattelin, että minut muuten ehkä tapetaan hänen tähtensä".
\par 10 Abimelek sanoi: "Mitä oletkaan meille tehnyt! Kuinka helposti olisikaan voinut tapahtua, että joku kansasta olisi maannut sinun vaimosi kanssa, ja niin sinä olisit saattanut meidät syyhyn!"
\par 11 Ja Abimelek antoi käskyn kaikelle kansalle, sanoen: "Joka koskee tähän mieheen tai hänen vaimoonsa, se on kuolemalla rangaistava".
\par 12 Ja Iisak kylvi siinä maassa ja sai sinä vuonna satakertaisesti, sillä Herra siunasi häntä.
\par 13 Ja hän rikastui ja hyötyi hyötymistään, kunnes hänestä tuli hyvin rikas.
\par 14 Ja hänellä oli laumoittain sekä pikkukarjaa että raavaskarjaa ja paljon palvelijoita, niin että filistealaiset alkoivat kadehtia häntä.
\par 15 Ja filistealaiset tukkivat kaikki ne kaivot, jotka hänen isänsä palvelijat olivat kaivaneet hänen isänsä Aabrahamin päivinä, ja täyttivät ne mullalla.
\par 16 Silloin Abimelek sanoi Iisakille: "Mene pois luotamme, sillä sinä olet tullut paljon väkevämmäksi meitä".
\par 17 Niin Iisak lähti sieltä ja leiriytyi Gerarin laaksoon ja asui siellä.
\par 18 Ja Iisak kaivatti uudelleen auki ne vesikaivot, jotka hänen isänsä Aabrahamin päivinä olivat kaivetut, mutta jotka filistealaiset olivat Aabrahamin kuoltua tukkineet, ja antoi niille jälleen ne nimet, jotka hänen isänsä oli niille antanut.
\par 19 Ja Iisakin palvelijat kaivoivat laaksossa ja löysivät sieltä kaivon, jossa oli juoksevaa vettä.
\par 20 Mutta Gerarin paimenet alkoivat riidellä Iisakin paimenten kanssa, sanoen: "Vesi on meidän". Niin hän kutsui kaivon Eesekiksi, koska he olivat riidelleet hänen kanssaan.
\par 21 Senjälkeen he kaivoivat toisen kaivon ja joutuivat riitaan siitäkin; niin hän antoi sille nimen Sitna.
\par 22 Sitten hän muutti sieltä pois ja kaivoi vielä yhden kaivon. Siitä ei syntynyt riitaa. Sentähden hän nimitti sen Rehobotiksi ja sanoi: "Nyt Herra on antanut meille tilaa, niin että voimme lisääntyä maassa".
\par 23 Sieltä hän meni Beersebaan.
\par 24 Ja Herra ilmestyi hänelle sinä yönä ja sanoi: "Minä olen sinun isäsi Aabrahamin Jumala; älä pelkää, sillä minä olen sinun kanssasi ja siunaan sinua ja teen sinun jälkeläistesi luvun suureksi palvelijani Aabrahamin tähden".
\par 25 Silloin hän rakensi sinne alttarin, huusi avuksi Herran nimeä ja pystytti sinne telttansa. Ja Iisakin palvelijat kaivoivat siihen kaivon.
\par 26 Niin Abimelek lähti hänen luokseen Gerarista ystävänsä Ahusatin ja sotapäällikkönsä Piikolin seuraamana.
\par 27 Mutta Iisak sanoi heille: "Minkätähden tulette minun luokseni, vaikka olette minulle vihamieliset ja karkoititte minut luotanne?"
\par 28 He vastasivat: "Me olemme selvästi nähneet, että Herra on sinun kanssasi. Sentähden ajattelimme: 'Olkoon sinun ja meidän välillämme valallinen sopimus; me tahdomme tehdä sinun kanssasi liiton,
\par 29 ettet tee meille mitään pahaa, niinkuin emme mekään ole sinuun koskeneet, vaan olemme tehneet sinulle ainoastaan hyvää ja sallineet sinun lähteä rauhassa'. Sinä olet nyt Herran siunattu."
\par 30 Silloin hän laittoi heille pidot, ja he söivät ja joivat.
\par 31 Varhain seuraavana aamuna he vannoivat valan toisillensa; senjälkeen Iisak päästi heidät menemään, ja he lähtivät hänen luotaan rauhassa.
\par 32 Samana päivänä Iisakin palvelijat tulivat ja kertoivat hänelle kaivosta, jonka olivat kaivaneet, sanoen hänelle:
\par 33 "Me löysimme vettä". Ja hän antoi sille nimen Siba. Sentähden on kaupungin nimi vielä tänäkin päivänä Beerseba.
\par 34 Kun Eesau oli neljänkymmenen vuoden vanha, otti hän vaimoikseen Jehuditin, heettiläisen Beerin tyttären, ja Baasematin, heettiläisen Eelonin tyttären.
\par 35 Näistä tuli Iisakille ja Rebekalle katkera suru.

\chapter{27}

\par 1 Kun Iisak oli tullut vanhaksi ja hänen silmänsä olivat hämärtyneet, niin ettei hän enää voinut nähdä, kutsui hän Eesaun, vanhemman poikansa, ja sanoi hänelle: "Poikani!" Tämä vastasi hänelle: "Tässä olen".
\par 2 Niin hän sanoi: "Katso, minä olen tullut vanhaksi enkä tiedä, milloin kuolen.
\par 3 Ota siis aseesi, viinesi ja jousesi, ja mene kedolle ja pyydystä minulle riistaa.
\par 4 Ja laita minulle herkkuruoka, minun mieliruokani, ja tuo se syödäkseni, että minä siunaisin sinut, ennenkuin kuolen."
\par 5 Mutta Rebekka kuuli, kuinka Iisak puhui pojallensa Eesaulle. Ja kun Eesau oli lähtenyt kedolle pyydystämään riistaa, tuodakseen isällensä,
\par 6 puhui Rebekka pojalleen Jaakobille sanoen: "Katso, minä kuulin sinun isäsi puhuvan veljellesi Eesaulle ja sanovan:
\par 7 'Tuo minulle riistaa ja laita minulle herkkuruoka syödäkseni, että siunaisin sinut Herran edessä, ennenkuin kuolen'.
\par 8 Kuule siis, poikani, mitä sanon, ja tee, mitä minä käsken:
\par 9 mene laumaan ja ota sieltä minulle kaksi hyvää vohlaa laittaakseni niistä isällesi herkkuruuan, hänen mieliruokansa.
\par 10 Ja sinun on vietävä se isäsi syödä, että hän siunaisi sinut, ennenkuin kuolee."
\par 11 Mutta Jaakob sanoi äidillensä Rebekalle: "Katso, veljeni Eesau on karvainen, mutta minä olen sileäihoinen.
\par 12 Entä jos isäni tunnustelee minua? Silloin minä joudun hänen silmissään pilkkaajaksi ja hankin itselleni kirouksen enkä siunausta."
\par 13 Hänen äitinsä sanoi hänelle: "Kohdatkoon se kirous minua, poikani; kuule vain, mitä minä sanon, mene ja nouda".
\par 14 Niin hän meni noutamaan ne ja toi ne äidilleen; ja hänen äitinsä laittoi herkkuruuan, hänen isänsä mieliruuan.
\par 15 Ja Rebekka otti vanhemman poikansa Eesaun parhaat vaatteet, jotka olivat hänen hallussaan talossa, ja puki ne nuoremman poikansa Jaakobin ylle.
\par 16 Mutta vohlain nahat hän kääri hänen käsiinsä ja paljaaseen kaulaansa.
\par 17 Sitten hän antoi herkkuruuan ynnä leipomansa leivän poikansa Jaakobin käteen.
\par 18 Ja Jaakob meni isänsä luo ja sanoi: "Isäni!" Hän vastasi: "Tässä olen; kuka sinä olet, poikani?"
\par 19 Jaakob sanoi isällensä: "Minä olen Eesau, esikoisesi. Olen tehnyt, niinkuin käskit minun tehdä; nouse istumaan ja syö riistaani, siunataksesi minut."
\par 20 Mutta Iisak sanoi pojalleen: "Kuinka olet, poikani, niin pian löytänyt?" Hän vastasi: "Herra, sinun Jumalasi, johdatti sen minun eteeni".
\par 21 Niin Iisak sanoi Jaakobille: "Tulehan lähemmä, poikani, tunnustellakseni, oletko sinä poikani Eesau vai etkö".
\par 22 Ja Jaakob astui isänsä Iisakin luo, ja tämä tunnusteli häntä ja sanoi: "Ääni on Jaakobin ääni, mutta kädet ovat Eesaun kädet".
\par 23 Eikä Iisak tuntenut häntä, sillä hänen kätensä olivat karvaiset, niinkuin hänen veljensä Eesaun kädet, ja hän siunasi hänet.
\par 24 Vielä hän kysyi: "Oletko sinä todella minun poikani Eesau?" Hän vastasi: "Olen".
\par 25 Silloin hän sanoi: "Tuo ruoka minulle, syödäkseni poikani riistaa, että siunaisin sinut". Niin hän toi hänelle sen, ja hän söi; ja hän tarjosi hänelle viiniä, ja hän joi.
\par 26 Senjälkeen hänen isänsä Iisak sanoi hänelle: "Tule tänne ja suutele minua, poikani".
\par 27 Hän astui hänen luokseen ja suuteli häntä. Niin Iisak tunsi hänen vaatteidensa hajun ja siunasi hänet, sanoen: "Katso, minun poikani tuoksu on kuin kedon tuoksu, jonka Herra on siunannut.
\par 28 Jumala antakoon sinulle taivaan kastetta ja maan lihavuutta, jyviä ja viiniä yllin kyllin.
\par 29 Kansat palvelkoot sinua, kansakunnat sinua kumartakoot. Ole veljiesi herra, ja äitisi pojat kumartakoot sinua. Kirottu olkoon, joka sinua kiroaa, siunattu, joka sinua siunaa."
\par 30 Kun Iisak oli ehtinyt siunata Jaakobin ja tämä juuri oli lähtenyt isänsä Iisakin luota, niin hänen veljensä Eesau tuli kotiin metsästämästä.
\par 31 Ja hänkin laittoi herkkuruuan, vei sen isälleen ja sanoi isälleen: "Nouse, isäni, ja syö poikasi riistaa, siunataksesi minut".
\par 32 Hänen isänsä Iisak kysyi häneltä: "Kuka olet?" Hän vastasi: "Minä olen poikasi Eesau, sinun esikoisesi".
\par 33 Silloin Iisak säikähtyi kovin ja sanoi: "Kuka sitten oli se metsästäjä, joka toi minulle riistaa, niin että minä, ennenkuin sinä tulit, söin kaikkea ja siunasin hänet? Siunattu hän myös on oleva."
\par 34 Kun Eesau kuuli isänsä sanat, puhkesi hän valittamaan äänekkäästi ja haikeasti ja sanoi isälleen: "Siunaa minutkin, isäni!"
\par 35 Mutta hän vastasi: "Veljesi tuli kavalasti ja riisti sinulta siunauksen".
\par 36 Niin hän sanoi: "Oikeinpa häntä kutsutaankin Jaakobiksi. Sillä hän on nyt kahdesti minut pettänyt: esikoisuuteni hän on minulta vienyt, ja katso, nyt hän riisti minulta myöskin siunauksen." Ja hän kysyi: "Eikö sinulla ole mitään siunausta minun varalleni?"
\par 37 Ja Iisak vastasi ja sanoi Eesaulle: "Katso, minä olen asettanut hänet sinun herraksesi ja antanut kaikki hänen veljensä hänelle palvelijoiksi sekä varustanut hänet jyvillä ja viinillä; mitä voisin enää tehdä sinun hyväksesi, poikani?"
\par 38 Eesau sanoi isällensä: "Tuo yksi ainoa siunausko sinulla vain onkin, isäni? Siunaa myöskin minut, isäni!" Ja Eesau korotti äänensä ja itki.
\par 39 Niin hänen isänsä Iisak vastasi ja sanoi hänelle: "Katso, sinun asuinsijasi on oleva kaukana lihavasta maasta ja vailla taivaan kastetta ylhäältä.
\par 40 Miekkasi varassa sinä olet elävä ja palveleva veljeäsi. Mutta valtoimena kierrellen sinä riisut hänen ikeensä niskaltasi."
\par 41 Ja Eesau alkoi vihata Jaakobia siunauksen tähden, jolla hänen isänsä oli hänet siunannut; ja Eesau ajatteli itsekseen: "Pian tulee aika, jolloin suremme isäämme; silloin minä tapan veljeni Jaakobin".
\par 42 Mutta Rebekalle ilmoitettiin hänen vanhemman poikansa Eesaun aikeista; ja hän kutsutti luokseen nuoremman poikansa Jaakobin ja sanoi hänelle: "Katso, veljesi Eesau uhkaa kostaa sinulle ja tappaa sinut.
\par 43 Kuule siis, mitä sanon, poikani: nouse ja pakene minun veljeni Laabanin luo Harraniin
\par 44 ja jää hänen luokseen joksikin aikaa, kunnes veljesi kiukku asettuu,
\par 45 kunnes veljesi lakkaa sinua vihaamasta ja unhottaa, mitä olet hänelle tehnyt. Sitten minä lähetän noutamaan sinut sieltä. Minkätähden menettäisin teidät molemmat samana päivänä!"
\par 46 Ja Rebekka sanoi Iisakille: "Minä olen kyllästynyt elämääni Heetin tyttärien tähden. Jos Jaakobkin ottaa vaimon Heetin tyttäristä, tässä maassa syntyneen, sellaisen kuin nämä, niin mitä varten minä enää elän?"

\chapter{28}

\par 1 Silloin Iisak kutsui Jaakobin ja siunasi hänet; hän käski häntä ja sanoi hänelle: "Älä ota vaimoksesi ketään Kanaanin tyttäristä,
\par 2 vaan nouse ja mene Mesopotamiaan, isoisäsi Betuelin kotiin, ja ota sieltä itsellesi vaimo enosi Laabanin tyttäristä.
\par 3 Ja Jumala, Kaikkivaltias, siunatkoon sinua ja antakoon sinun tulla hedelmälliseksi ja lisääntyä, niin että sinusta tulee suuri kansojen joukko.
\par 4 Ja hän suokoon sinulle Aabrahamin siunauksen, sinulle ynnä sinun jälkeläisillesi, että saisit omaksesi maan, jossa sinä muukalaisena asut ja jonka Jumala on antanut Aabrahamille."
\par 5 Niin Iisak lähetti Jaakobin matkalle, ja hän lähti Mesopotamiaan Laabanin luo, joka oli aramilaisen Betuelin poika ja Rebekan, Jaakobin ja Eesaun äidin, veli.
\par 6 Kun Eesau näki, että Iisak oli siunannut Jaakobin ja lähettänyt hänet Mesopotamiaan ottamaan sieltä itselleen vaimoa - sillä hän oli siunannut hänet, käskenyt häntä ja sanonut: "Älä ota vaimoa Kanaanin tyttäristä" -
\par 7 ja että Jaakob oli totellut isäänsä ja äitiänsä ja lähtenyt Mesopotamiaan,
\par 8 silloin hän huomasi, että Kanaanin tyttäret olivat hänen isälleen Iisakille vastenmieliset;
\par 9 ja niin Eesau meni Ismaelin luo ja otti Mahalatin, Aabrahamin pojan Ismaelin tyttären, Nebajotin sisaren, vaimokseen entisten lisäksi.
\par 10 Niin Jaakob lähti Beersebasta mennäksensä Harraniin.
\par 11 Ja hän osui erääseen paikkaan, johon yöpyi, sillä aurinko oli laskenut; ja hän otti sen paikan kivistä yhden, pani sen päänsä alaiseksi ja asettui nukkumaan siihen paikkaan.
\par 12 Niin hän näki unta, ja katso, maan päälle oli asetettu tikapuut, joiden pää ulottui taivaaseen, ja katso, Jumalan enkelit kulkivat niitä myöten ylös ja alas.
\par 13 Ja katso, Herra seisoi hänen edessään ja sanoi: "Minä olen Herra, sinun Isäsi Aabrahamin Jumala ja Iisakin Jumala. Tämän maan, jonka päällä sinä makaat, minä annan sinulle ja sinun jälkeläisillesi.
\par 14 Ja sinun jälkeläistesi paljous on oleva kuin maan tomu, ja sinä leviät länteen ja itään, pohjoiseen ja etelään, ja sinussa ja sinun siemenessäsi tulevat siunatuiksi kaikki sukukunnat maan päällä.
\par 15 Ja katso, minä olen sinun kanssasi ja varjelen sinua, missä ikinä kuljet, ja saatan sinut takaisin tähän maahan; sillä minä en jätä sinua, ennenkuin olen toteuttanut sen, minkä minä olen sinulle puhunut."
\par 16 Silloin Jaakob heräsi unestansa ja sanoi: "Herra on totisesti tässä paikassa, enkä minä sitä tiennyt".
\par 17 Ja pelko valtasi hänet, ja hän sanoi: "Kuinka peljättävä onkaan tämä paikka! Tässä on varmasti Jumalan huone ja taivaan portti."
\par 18 Ja Jaakob nousi varhain aamulla, otti kiven, jonka hän oli pannut päänsä alaiseksi, ja nosti sen pystyyn patsaaksi ja vuodatti öljyä sen päälle.
\par 19 Ja hän kutsui paikan Beeteliksi; mutta ennen oli kaupungin nimenä Luus.
\par 20 Ja Jaakob teki lupauksen, sanoen: "Jos Jumala on minun kanssani ja varjelee minut sillä tiellä, jota nyt kuljen, ja antaa minulle leipää syödäkseni ja vaatteita pukeutuakseni,
\par 21 niin että saan palata rauhassa isäni kotiin, niin on Herra oleva minun Jumalani;
\par 22 ja tämä kivi, jonka olen patsaaksi pystyttänyt, on oleva Jumalan huone, ja kaikesta, mitä minulle suot, minä totisesti annan sinulle kymmenykset".

\chapter{29}

\par 1 Ja Jaakob lähti matkaan ja tuli Idän miesten maalle.
\par 2 Ja katso, hän huomasi kedolla kaivon, ja katso, sen ääressä makasi kolme lammaslaumaa, sillä siitä kaivosta oli tapana juottaa laumoja. Ja kaivon suulla oli suuri kivi.
\par 3 Sentähden annettiin kaikkien laumojen kokoontua sinne ja sitten vieritettiin kivi kaivon suulta ja juotettiin lampaat, jonka jälkeen kivi taas pantiin paikoilleen kaivon suulle.
\par 4 Ja Jaakob sanoi heille: "Veljeni, mistä te olette?" He vastasivat: "Harranista olemme".
\par 5 Hän kysyi heiltä: "Tunnetteko Laabanin, Naahorin pojan?" He vastasivat: "Tunnemme".
\par 6 Ja hän kysyi heiltä: "Voiko hän hyvin?" He vastasivat: "Hyvin hän voi; ja katso, Raakel, hänen tyttärensä, tulee tuolla lammasten kanssa".
\par 7 Hän sanoi: "Vielä on täysi päivä ja liian aikaista koota lauma; juottakaa lampaat ja viekää ne takaisin laitumelle".
\par 8 He vastasivat: "Emme voi, ennenkuin kaikki laumat ovat koolla ja kivi on vieritetty kaivon suulta; sitten juotamme lampaat".
\par 9 Hänen vielä puhuessaan heidän kanssaan tuli Raakel isänsä lammasten kanssa, sillä hänen oli tapana olla paimenessa.
\par 10 Kun Jaakob näki Raakelin, enonsa Laabanin tyttären, ja enonsa Laabanin lampaat, astui hän esiin ja vieritti kiven kaivon suulta ja juotti enonsa Laabanin lampaat.
\par 11 Ja Jaakob suuteli Raakelia ja korotti äänensä ja itki.
\par 12 Niin Jaakob ilmoitti Raakelille, että hän oli hänen isänsä sukulainen ja Rebekan poika; ja Raakel riensi pois ja ilmoitti sen isällensä.
\par 13 Kun Laaban kuuli kerrottavan Jaakobista, sisarensa pojasta, riensi hän häntä vastaan, syleili ja suuteli häntä ja toi hänet kotiinsa; ja hän kertoi Laabanille kaikki, mitä oli tapahtunut.
\par 14 Ja Laaban sanoi hänelle: "Niin, sinä olet minun luutani ja lihaani". Ja hän asui hänen luonaan kuukauden päivät.
\par 15 Ja Laaban sanoi Jaakobille: "Olet tosin sukulaiseni, mutta palvelisitko silti minua palkatta? Sano minulle, mikä on palkkasi oleva."
\par 16 Mutta Laabanilla oli kaksi tytärtä; vanhemman nimi oli Leea, nuoremman nimi oli Raakel.
\par 17 Ja Leealla oli sameat silmät, mutta Raakelilla oli kaunis vartalo ja kauniit kasvot.
\par 18 Ja Jaakob rakasti Raakelia; niin hän sanoi: "Minä palvelen sinua seitsemän vuotta saadakseni Raakelin, nuoremman tyttäresi".
\par 19 Laaban vastasi: "Parempi on, että annan hänet sinulle, kuin että antaisin hänet jollekin toiselle; jää luokseni".
\par 20 Jaakob palveli siis seitsemän vuotta saadakseen Raakelin, ja ne tuntuivat hänestä muutamilta päiviltä; niin hän rakasti häntä.
\par 21 Senjälkeen Jaakob sanoi Laabanille: "Anna minulle vaimoni, sillä aikani on nyt kulunut umpeen, mennäkseni hänen tykönsä".
\par 22 Silloin Laaban kutsui kokoon kaikki sen paikkakunnan asukkaat ja laittoi pidot;
\par 23 mutta illalla hän otti tyttärensä Leean ja vei hänet hänen luokseen; ja hän yhtyi häneen.
\par 24 Ja Laaban antoi orjattarensa Silpan tyttärellensä Leealle orjattareksi.
\par 25 Aamulla hän näki, että se oli Leea. Ja hän sanoi Laabanille: "Mitä oletkaan minulle tehnyt? Olenhan palvellut sinua saadakseni Raakelin! Miksi petit minut?"
\par 26 Laaban vastasi: "Ei ole meidän maassamme tapana antaa nuorempaa ennen vanhempaa.
\par 27 Vietä nyt vain tämä hääviikko loppuun, niin annetaan sinulle toinenkin, kunhan olet palveluksessani vielä toiset seitsemän vuotta."
\par 28 Ja Jaakob suostui siihen, ja kun hän oli viettänyt sen viikon loppuun, antoi Laaban myöskin tyttärensä Raakelin hänelle vaimoksi.
\par 29 Ja Laaban antoi orjattarensa Bilhan tyttärellensä Raakelille orjattareksi.
\par 30 Ja hän yhtyi myös Raakeliin, ja Raakel oli hänelle rakkaampi kuin Leea; ja hän palveli Laabanin luona vielä toiset seitsemän vuotta.
\par 31 Ja kun Herra näki, että Leeaa hyljittiin, avasi hän hänen kohtunsa; mutta Raakel oli hedelmätön.
\par 32 Niin Leea tuli raskaaksi ja synnytti pojan ja antoi hänelle nimen Ruuben, sanoen: "Herra on nähnyt minun kurjuuteni; nyt on mieheni rakastava minua".
\par 33 Ja hän tuli jälleen raskaaksi ja synnytti pojan ja sanoi: "Herra on kuullut, että minua hyljitään, ja on antanut minulle myös tämän". Niin hän antoi hänelle nimen Simeon.
\par 34 Ja hän tuli jälleen raskaaksi ja synnytti pojan ja sanoi: "Nyt kai mieheni vihdoinkin kiintyy minuun, sillä olenhan synnyttänyt hänelle kolme poikaa". Sentähden hän antoi hänelle nimen Leevi.
\par 35 Ja hän tuli vieläkin raskaaksi ja synnytti pojan ja sanoi: "Nyt minä kiitän Herraa". Sentähden hän antoi hänelle nimen Juuda. Sitten hän lakkasi synnyttämästä.

\chapter{30}

\par 1 Kun Raakel näki, ettei hän synnyttänyt Jaakobille, kadehti hän sisartaan ja sanoi Jaakobille: "Hanki minulle lapsia, muuten minä kuolen".
\par 2 Niin Jaakob vihastui Raakeliin ja sanoi: "Minäkö olen Jumala, joka on kieltänyt sinulta kohdun hedelmän?"
\par 3 Mutta Raakel sanoi: "Tässä on orjattareni Bilha; yhdy häneen, että hän synnyttäisi minun helmaani ja minäkin siten saisin lapsia hänestä".
\par 4 Ja hän antoi hänelle orjattarensa Bilhan vaimoksi ja Jaakob yhtyi häneen.
\par 5 Ja Bilha tuli raskaaksi ja synnytti Jaakobille pojan.
\par 6 Niin Raakel sanoi: "Jumala hankki minulle oikeuden, ja hän kuuli minun ääneni ja antoi minulle pojan". Sentähden hän antoi hänelle nimen Daan.
\par 7 Ja Bilha, Raakelin orjatar, tuli jälleen raskaaksi ja synnytti Jaakobille toisen pojan.
\par 8 Niin Raakel sanoi: "Jumalan taisteluja minä olen taistellut sisareni kanssa ja olen voittanut". Ja hän antoi hänelle nimen Naftali.
\par 9 Kun Leea näki lakanneensa synnyttämästä, otti hän orjattarensa Silpan ja antoi hänet Jaakobille vaimoksi.
\par 10 Ja Silpa, Leean orjatar, synnytti Jaakobille pojan.
\par 11 Niin Leea sanoi: "Onneksi!" Ja hän antoi hänelle nimen Gaad.
\par 12 Ja Silpa, Leean orjatar, synnytti Jaakobille toisen pojan.
\par 13 Niin Leea sanoi: "Onnellista minua! Niin, naiset ylistävät minua onnelliseksi." Ja hän antoi hänelle nimen Asser.
\par 14 Mutta Ruuben meni kerran ulos nisunleikkuun aikana ja löysi lemmenmarjoja vainiolta ja toi ne äidillensä Leealle. Niin Raakel sanoi Leealle: "Anna minulle poikasi lemmenmarjoja".
\par 15 Leea vastasi hänelle: "Eikö riitä, että olet vienyt minulta mieheni, koska tahdot ottaa vielä poikani lemmenmarjatkin?" Raakel sanoi: "Olkoon, maatkoon hän tämän yön sinun kanssasi, kunhan saan poikasi lemmenmarjat".
\par 16 Kun Jaakob illalla palasi vainiolta, meni Leea häntä vastaan ja sanoi: "Minun luokseni sinun on tultava, sillä minä olen ostanut sinut poikani lemmenmarjoilla". Ja hän makasi sen yön hänen kanssaan.
\par 17 Ja Jumala kuuli Leeaa, ja Leea tuli raskaaksi ja synnytti Jaakobille viidennen pojan.
\par 18 Niin Leea sanoi: "Jumala on palkinnut minulle sen, että annoin orjattareni miehelleni". Ja hän antoi hänelle nimen Isaskar.
\par 19 Ja Leea tuli jälleen raskaaksi ja synnytti Jaakobille kuudennen pojan.
\par 20 Silloin Leea sanoi: "Jumala on antanut minulle hyvän lahjan. Nyt mieheni on asuva minun luonani, sillä minä olen synnyttänyt hänelle kuusi poikaa." Ja hän antoi hänelle nimen Sebulon.
\par 21 Sitten hän synnytti tyttären ja antoi hänelle nimen Diina.
\par 22 Mutta Jumala muisti Raakeliakin, ja Jumala kuuli häntä ja avasi hänen kohtunsa.
\par 23 Niin hän tuli raskaaksi ja synnytti pojan ja sanoi: "Jumala on ottanut pois minun häpeäni".
\par 24 Ja hän antoi hänelle nimen Joosef, sanoen: "Herra antakoon minulle vielä toisen pojan".
\par 25 Ja kun Raakel oli synnyttänyt Joosefin, sanoi Jaakob Laabanille: "Päästä minut menemään kotiini ja omaan maahani.
\par 26 Anna minulle vaimoni ja lapseni, joiden vuoksi olen sinua palvellut, mennäkseni pois; sillä tiedäthän itse, kuinka olen sinua palvellut."
\par 27 Laaban vastasi hänelle: "Jospa saisin armon silmiesi edessä! Merkkini ilmoittavat, että Herra sinun tähtesi on siunannut minua."
\par 28 Ja hän sanoi vielä: "Määrää palkka, joka minun on sinulle maksettava, niin minä sen annan".
\par 29 Hän vastasi hänelle: "Itsehän tiedät, kuinka minä olen sinua palvellut ja millaiseksi karjasi on tullut minun hoidossani.
\par 30 Sillä vähän sinulla oli ennen minun tuloani, mutta sitten se on karttunut suureksi, ja Herra on siunannut sinua, missä vain minä liikuin. Mutta milloin saan ruveta tekemään työtä minäkin oman perheeni hyväksi?"
\par 31 Hän vastasi: "Mitä minun on sinulle annettava?" Jaakob sanoi: "Ei sinun tarvitse antaa minulle mitään. Jos myönnät minulle tämän, niin minä yhä edelleen paimennan ja vartioitsen sinun laumojasi:
\par 32 minä tarkastan tänään kaiken laumasi; erota siitä pois kaikki pilkulliset ja kirjavat lampaat sekä karitsoista kaikki mustat ja vuohista kirjavat ja pilkulliset. Ja minun palkkani on sitten oleva tämä,
\par 33 ja siinä minun rehellisyyteni tulee toteennäytetyksi: kun vasta tulet omin silmin katsomaan minun palkkaani, niin kaikki vuohet, jotka eivät ole pilkullisia eivätkä kirjavia, ja kaikki karitsat, jotka eivät ole mustia, katsottakoon minun varastamikseni."
\par 34 Laaban vastasi: "Hyvä, olkoon, niinkuin olet puhunut".
\par 35 Ja samana päivänä hän erotti pois juovikkaat ja kirjavat vuohipukit ja kaikki pilkulliset ja kirjavat vuohet - kaikki, joissa oli jotakin valkoista - sekä kaikki mustat karitsat ja jätti ne poikiensa hoitoon.
\par 36 Ja hän asetti niin, että oli kolmen päivän välimatka hänen ja Jaakobin välillä; ja Jaakob paimensi Laabanin muuta karjaa.
\par 37 Mutta Jaakob otti itselleen tuoreita haavan, mantelipuun ja plataanin oksia ja kuori niihin valkeita juovia, paljastaen oksien valkoisen rungon.
\par 38 Ja kuorimansa oksat hän pani eläinten eteen vesikaukaloihin eli juoma-astioihin, joista ne tulivat juomaan; ja ne olivat kiimallaan tullessansa juomaan.
\par 39 Ja eläimet pariutuivat oksien edessä ja synnyttivät juovikkaita, pilkullisia ja kirjavia karitsoita.
\par 40 Sitten Jaakob erotti karitsat; ja hän asetti eläinten päät niihin päin, jotka olivat pilkullisia, ja kaikkiin niihin päin, jotka olivat mustia Laabanin laumassa; siten hän hankki itselleen eri laumansa eikä päästänyt niitä Laabanin laumaan.
\par 41 Ja joka kerta kun voimakkaat eläimet olivat kiimallaan, pani Jaakob oksat eläinten silmien eteen vesikaukaloihin, niin että ne pariutuivat oksien edessä.
\par 42 Mutta heikkojen eläinten eteen hän ei niitä pannut. Niin joutuivat heikot Laabanille ja voimakkaat Jaakobille.
\par 43 Ja siten mies tuli ylen rikkaaksi; hän sai paljon pikkukarjaa sekä palvelijattaria, palvelijoita, kameleja ja aaseja.

\chapter{31}

\par 1 Mutta hän sai kuulla, että Laabanin pojat puhuivat näin: "Jaakob on anastanut kaikki, mitä isämme omisti; isämme omaisuudesta hän on hankkinut itselleen kaiken tämän rikkauden".
\par 2 Ja Jaakob huomasi Laabanin kasvoista, ettei hän ollut häntä kohtaan niinkuin ennen.
\par 3 Ja Herra sanoi Jaakobille: "Palaja isiesi maahan ja sukusi tykö. Minä olen sinun kanssasi."
\par 4 Niin Jaakob kutsutti Raakelin ja Leean kedolle laumansa luo,
\par 5 ja hän sanoi heille: "Minä huomaan isänne kasvoista, ettei hän ole minua kohtaan niinkuin ennen, vaikka isäni Jumala on ollut minun kanssani.
\par 6 Te tiedätte, että minä olen palvellut isäänne kaikin voimin,
\par 7 mutta isänne on kohdellut minua petollisesti ja muuttanut palkkaani kymmenen kertaa. Jumala ei kuitenkaan ole sallinut hänen tehdä minulle mitään vahinkoa.
\par 8 Kun hän sanoi: 'Pilkulliset olkoot sinun palkkasi', niin koko lauma kantoi pilkullisia; ja kun hän sanoi: 'Juovikkaat olkoot sinun palkkasi', niin koko lauma kantoi juovikkaita.
\par 9 Niin on Jumala ottanut isänne omaisuuden ja antanut minulle.
\par 10 Mutta lauman pariutumisen aikana minä nostin silmäni ja näin unessa, että vuohipukit, jotka astuivat laumaa, olivat juovikkaita, pilkullisia ja kirjavia.
\par 11 Ja Jumalan enkeli sanoi minulle unessa: 'Jaakob!' Minä vastasin: 'Tässä olen'.
\par 12 Niin hän sanoi: 'Nosta silmäsi ja katso, kaikki vuohipukit, jotka astuvat laumaa, ovat juovikkaita, pilkullisia ja kirjavia, sillä minä olen nähnyt kaiken, mitä Laaban sinulle tekee.
\par 13 Minä olen Jumala, joka ilmestyin Beetelissä, jossa sinä voitelit patsaan ja jossa teit minulle lupauksen. Nouse nyt ja lähde tästä maasta ja palaja synnyinmaahasi.'"
\par 14 Silloin Raakel ja Leea vastasivat ja sanoivat hänelle: "Ei meillä ole enää osaa eikä perintöä isämme talossa.
\par 15 Eikö hän pitänyt meitä kuin vieraita, koska myi meidät ja söi suuhunsa meistä saamansa hinnan.
\par 16 Niin, koko se rikkaus, jonka Jumala on ottanut isältämme, meidän se on ja meidän lastemme. Tee siis nyt kaikki, mitä Jumala on sinulle sanonut."
\par 17 Silloin Jaakob nousi ja nosti lapsensa ja vaimonsa kamelien selkään
\par 18 ja kuljetti pois kaiken karjansa ja kaiken omaisuutensa, jonka hän oli koonnut, kaiken omistamansa ja Mesopotamiassa hankkimansa karjan, mennäkseen isänsä Iisakin luo Kanaanin maahan.
\par 19 Mutta Laaban oli mennyt keritsemään lampaitaan. Silloin Raakel varasti isänsä kotijumalat.
\par 20 Ja Jaakob lähti varkain aramilaisen Laabanin luota eikä ilmaissut hänelle pakenemisaiettaan.
\par 21 Niin hän pakeni kaikkinensa, lähti ja meni virran yli ja suuntasi kulkunsa Gileadin vuorille päin.
\par 22 Mutta kolmantena päivänä Laabanille ilmoitettiin, että Jaakob oli paennut.
\par 23 Silloin hän otti mukaansa heimonsa miehet ja ajoi häntä takaa seitsemän päivänmatkaa ja saavutti hänet Gileadin vuorilla.
\par 24 Mutta Jumala tuli aramilaisen Laabanin luo unessa yöllä ja sanoi hänelle: "Varo, ettet puhu Jaakobille hyvää etkä pahaa".
\par 25 Ja Laaban saavutti Jaakobin. Jaakob oli pystyttänyt telttansa vuorelle, Laaban taas pystytti telttansa Gileadin vuorelle.
\par 26 Ja Laaban sanoi Jaakobille: "Mitä oletkaan tehnyt? Sinä olet pettänyt minut ja kuljettanut pois minun tyttäreni niinkuin miekalla otetut!
\par 27 Minkätähden pakenit salaa, läksit luotani varkain? Et ilmoittanut minulle mitään, ja niin minä en saanut saattaa sinua matkalle iloiten ja laulaen, vaskirummuin ja kantelein,
\par 28 etkä suonut minun suudella lasteni lapsia ja tyttäriäni. Tyhmästi sinä olet menetellyt.
\par 29 Minulla olisi valta tehdä teille pahaa, mutta teidän isänne Jumala sanoi minulle viime yönä näin: 'Varo, ettet puhu Jaakobille hyvää etkä pahaa'.
\par 30 Olkoon: sinä olet nyt lähtenyt matkallesi, koska niin suuresti ikävöit isäsi kotiin, mutta minkätähden olet varastanut minun jumalani?"
\par 31 Jaakob vastasi ja sanoi Laabanille: "Minä pelkäsin, sillä ajattelin, että sinä riistäisit minulta tyttäresi.
\par 32 Mutta se, jonka hallusta löydät jumalasi, menettäköön henkensä. Heimojemme miesten läsnäollessa tutki, mitä minulla on, ja ota pois, mikä on omaasi." Sillä Jaakob ei tiennyt, että Raakel oli ne varastanut.
\par 33 Ja Laaban meni Jaakobin telttaan, Leean telttaan ja molempien orjatarten telttaan, mutta ei löytänyt mitään. Ja tultuaan ulos Leean teltasta hän meni Raakelin telttaan.
\par 34 Mutta Raakel oli ottanut kotijumalat, pannut ne kamelin satulaan ja istunut niiden päälle. Ja Laaban penkoi koko teltan, mutta ei löytänyt mitään.
\par 35 Ja hän sanoi isällensä: "Älä vihastu, herrani, siitä etten voi nousta sinun edessäsi, sillä minun on, niinkuin naisten tavallisesti on". Ja Laaban etsi, mutta ei löytänyt kotijumalia.
\par 36 Silloin Jaakob vihastui ja soimasi Laabania; Jaakob puhkesi puhumaan ja sanoi Laabanille: "Mitä minä olen rikkonut, mitä pahaa minä olen tehnyt, että näin minua ahdistat?
\par 37 Nyt olet penkonut kaikki minun tavarani; mitä olet löytänyt sellaista, joka olisi sinun talosi tavaraa? Tuo se tähän minun heimoni ja sinun heimosi miesten eteen, että he ratkaisisivat meidän molempien välin.
\par 38 Jo kaksikymmentä vuotta minä olen ollut sinun luonasi; sinun uuhesi ja vuohesi eivät ole synnyttäneet keskoisia, enkä minä ole oinaita sinun laumastasi syönyt.
\par 39 Pedon haaskaamaa en ole sinulle tuonut, se oli minun itseni korvattava; minulta sinä sen vaadit, olipa se viety päivällä tai viety yöllä.
\par 40 Päivällä vaivasi minua helle, yöllä vilu, ja uni pakeni silmistäni.
\par 41 Jo kaksikymmentä vuotta minä olen ollut sinun talossasi; neljätoista vuotta minä palvelin sinua saadakseni molemmat tyttäresi ja kuusi vuotta saadakseni sinulta karjaa, mutta kymmenen kertaa sinä muutit minun palkkani.
\par 42 Jos minun isäni Jumala, Aabrahamin Jumala, jota myöskin Iisak pelkää, ei olisi ollut minun puolellani, niin sinä olisit nyt lähettänyt minut tyhjänä tieheni. Jumala on nähnyt minun kurjuuteni ja kätteni vaivannäön, ja hän ratkaisi viime yönä asian."
\par 43 Niin Laaban vastasi ja sanoi Jaakobille: "Tyttäret ovat minun tyttäriäni, ja lapset ovat minun lapsiani, ja karja on minun karjaani, ja kaikki, mitä näet, on minun omaani. Mutta minkä minä nyt mahdan näille tyttärilleni tai lapsille, jotka he ovat synnyttäneet!
\par 44 Tule siis, tehkäämme liitto keskenämme, ja olkoon se todistuksena meidän välillämme, minun ja sinun."
\par 45 Silloin Jaakob otti kiven ja nosti sen pystyyn patsaaksi.
\par 46 Ja Jaakob sanoi heimonsa miehille: "Kootkaa kiviä". Ja he ottivat kiviä ja rakensivat roukkion ja aterioivat siinä sen kiviroukkion päällä.
\par 47 Ja Laaban antoi sille nimen Jegar-Saahaduta, mutta Jaakob antoi sille nimen Galed.
\par 48 Ja Laaban sanoi: "Tämä roukkio olkoon tänään todistajana meidän välillämme, minun ja sinun"; sentähden hän antoi sille nimen Galed
\par 49 ja myöskin nimen Mispa, sillä hän sanoi: "Herra olkoon vartija meidän välillämme, minun ja sinun, kun joudumme loitolle toistemme näkyvistä.
\par 50 Jos sinä kohtelet pahasti minun tyttäriäni tahi otat toisia vaimoja tyttärieni lisäksi, niin tiedä, että vaikkei ketään ihmistä olekaan läsnä, Jumala kuitenkin on todistajana meidän välillämme, minun ja sinun."
\par 51 Ja Laaban sanoi vielä Jaakobille: "Katso, tämä roukkio ja tämä patsas, jonka minä olen pystyttänyt meidän välillemme, minun ja sinun
\par 52 - tämä roukkio olkoon todistuksena, ja myöskin tämä patsas olkoon todistuksena siitä, etten minä kulje tämän roukkion ohi sinun luoksesi ja ettet sinäkään kulje tämän roukkion ja tämän patsaan ohi minun luokseni paha mielessä.
\par 53 Aabrahamin Jumala ja Naahorin Jumala, heidän isiensä Jumala, olkoon tuomarina meidän välillämme." Ja Jaakob vannoi valansa Jumalan kautta, jota hänen isänsä Iisak pelkäsi.
\par 54 Ja Jaakob uhrasi vuorella teurasuhrin ja kutsui heimonsa miehet aterioimaan, ja he aterioivat ja olivat yötä vuorella.
\par 55 Mutta varhain seuraavana aamuna Laaban nousi, suuteli lastensa lapsia ja tyttäriään ja hyvästeli heidät; sitten hän lähti matkaan ja palasi kotiinsa.

\chapter{32}

\par 1 Mutta Jaakob kulki tietänsä; ja Jumalan enkeleitä tuli häntä vastaan.
\par 2 Ja nähdessään heidät Jaakob sanoi: "Tämä on Jumalan sotajoukkoa". Ja hän antoi sille paikalle nimen Mahanaim.
\par 3 Sitten Jaakob lähetti sanansaattajat edellään veljensä Eesaun luo Seirin maahan, Edomin alueelle.
\par 4 Ja hän käski heitä sanoen: "Sanokaa herralleni Eesaulle näin: 'Sinun palvelijasi Jaakob sanoo: Minä olen oleskellut Laabanin luona ja viipynyt siellä tähän saakka;
\par 5 ja minä olen saanut raavaita, aaseja, pikkukarjaa, palvelijoita ja palvelijattaria ja lähetän nyt sanan herralleni, että saisin armon sinun silmiesi edessä'."
\par 6 Sanansaattajat palasivat Jaakobin luo ja sanoivat: "Me tulimme veljesi Eesaun luo; hän on jo matkalla sinua vastaan, neljäsataa miestä mukanaan".
\par 7 Silloin valtasi Jaakobin suuri pelko ja ahdistus. Ja hän jakoi väen, joka oli hänen kanssansa, ja pikkukarjan ja raavaskarjan ja kamelit kahteen joukkoon.
\par 8 Sillä hän ajatteli: "Jos Eesau hyökkää toisen joukon kimppuun ja tuhoaa sen, niin toinen joukko pääsee pakoon".
\par 9 Ja Jaakob sanoi: "Isäni Aabrahamin Jumala ja isäni Iisakin Jumala, Herra, sinä, joka sanoit minulle: 'Palaja maahasi ja sukusi luo, niin minä teen sinulle hyvää!'
\par 10 Minä olen liian halpa kaikkeen siihen armoon ja kaikkeen siihen uskollisuuteen, jota sinä olet palvelijallesi osoittanut; sillä ainoastaan sauva kädessäni minä kuljin tämän Jordanin yli, ja nyt on minulle karttunut kaksi joukkoa.
\par 11 Pelasta minut veljeni Eesaun käsistä, sillä minä pelkään, että hän tulee ja tuhoaa minut ynnä äidit lapsineen.
\par 12 Olethan itse sanonut: 'Minä teen sinulle hyvää ja annan sinun jälkeläistesi luvun tulla paljoksi kuin meren hiekka, jota ei voida lukea sen paljouden tähden'."
\par 13 Ja hän jäi siihen siksi yöksi. Sitten hän erotti omaisuudestaan lahjaksi veljelleen Eesaulle
\par 14 kaksisataa vuohta ja kaksikymmentä vuohipukkia, kaksisataa uuhta ja kaksikymmentä oinasta,
\par 15 kolmekymmentä imettävää kamelia varsoinensa, neljäkymmentä lehmää ja kymmenen härkää, kaksikymmentä aasintammaa ja kymmenen aasia.
\par 16 Ja hän jätti ne palvelijainsa haltuun, kunkin lauman erikseen, ja sanoi palvelijoilleen: "Menkää minun edelläni ja jättäkää välimatka kunkin lauman välille".
\par 17 Ja hän käski ensimmäistä sanoen: "Kun veljeni Eesau kohtaa sinut ja kysyy: 'Kenen sinä olet, ja mihin menet, ja kenen ovat nuo elukat tuolla edelläsi?'
\par 18 niin vastaa: 'Ne ovat palvelijasi Jaakobin, lähetetyt lahjaksi herralleni Eesaulle; ja katso, myös hän itse tulee jäljessämme'."
\par 19 Samoin hän käski toista ja kolmatta ja kaikkia muita, jotka laumoja ajoivat, sanoen: "Juuri näin on teidän sanottava Eesaulle, kun tapaatte hänet.
\par 20 Ja sanokaa myös: 'Katso, sinun palvelijasi Jaakob tulee meidän jäljessämme'." Sillä hän ajatteli: "Minä koetan lepyttää häntä lahjalla, joka kulkee edelläni. Sitten astun itse hänen kasvojensa eteen; ehkä hän ottaa minut ystävällisesti vastaan."
\par 21 Niin lahja kulki hänen edellänsä, mutta itse hän jäi siksi yöksi leiriin.
\par 22 Mutta yöllä hän nousi, otti molemmat vaimonsa ja molemmat orjattarensa ja yksitoista lastansa ja meni kahlauspaikasta Jabbokin yli.
\par 23 Ja hän otti heidät ja vei heidät joen yli ja vei sen yli kaiken, mitä hänellä oli.
\par 24 Ja Jaakob jäi yksinänsä toiselle puolelle. Silloin painiskeli hänen kanssaan muuan mies päivän koittoon saakka.
\par 25 Ja kun mies huomasi, ettei hän häntä voittanut, iski hän häntä lonkkaluuhun, niin että Jaakobin lonkka nyrjähti hänen painiskellessaan hänen kanssaan.
\par 26 Ja mies sanoi: "Päästä minut, sillä päivä koittaa". Mutta hän vastasi: "En päästä sinua, ellet siunaa minua".
\par 27 Ja hän sanoi hänelle: "Mikä sinun nimesi on?" Hän vastasi: "Jaakob".
\par 28 Silloin hän sanoi: "Sinun nimesi älköön enää olko Jaakob, vaan Israel, sillä sinä olet taistellut Jumalan ja ihmisten kanssa ja olet voittanut".
\par 29 Ja Jaakob kysyi ja sanoi: "Ilmoita nimesi". Hän vastasi: "Miksi kysyt minun nimeäni?" Ja hän siunasi hänet siinä.
\par 30 Ja Jaakob antoi sille paikalle nimen Penuel, "sillä", sanoi hän, "minä olen nähnyt Jumalan kasvoista kasvoihin, ja kuitenkin on minun henkeni pelastunut".
\par 31 Ja kun hän oli kulkenut Penuelin ohitse, nousi aurinko; mutta hän ontui lonkkaansa.
\par 32 Sentähden israelilaiset eivät vielä tänäkään päivänä syö reisijännettä, joka kulkee lonkkaluun yli; sillä hän iski Jaakobia lonkkaluuhun, reisijänteen kohdalle.

\chapter{33}

\par 1 Kun Jaakob nosti silmänsä ja katseli, niin katso, Eesau oli tulossa, neljäsataa miestä mukanaan. Silloin hän jakoi lapset Leealle, Raakelille ja molemmille orjattarille.
\par 2 Ja hän asetti orjattaret lapsineen ensimmäisiksi, niiden jälkeen Leean lapsineen ja Raakelin Joosefin kanssa viimeiseksi.
\par 3 Mutta itse hän astui heidän edellänsä ja kumartui maahan seitsemän kertaa, kunnes oli saapunut veljensä eteen.
\par 4 Mutta Eesau riensi häntä vastaan ja sulki hänet syliinsä, halasi häntä kaulasta ja suuteli häntä; ja he itkivät.
\par 5 Ja hän nosti silmänsä ja näki vaimot ja lapset ja kysyi: "Keitä ovat nämä, jotka ovat sinun seurassasi?" Hän vastasi: "Ne ovat minun lapseni, jotka Jumala on palvelijallesi lahjoittanut".
\par 6 Niin orjattaret lähestyivät lapsineen ja kumartuivat maahan.
\par 7 Myöskin Leea lapsineen lähestyi, ja he kumartuivat maahan. Viimein lähestyivät Joosef ja Raakel ja kumartuivat maahan.
\par 8 Sitten hän kysyi: "Mitä tarkoitit kaikella sillä joukolla, jonka minä kohtasin?" Hän vastasi: "Saada armon herrani silmien edessä".
\par 9 Mutta Eesau sanoi: "Minulla on itselläni kyllin; pidä, veljeni, omasi!"
\par 10 Jaakob vastasi: "Ei niin; jos olen saanut armon sinun silmiesi edessä, niin ota minulta vastaan lahjani; sillä olenhan saanut nähdä sinun kasvosi, niinkuin nähdään Jumalan kasvot, ja sinä olet ottanut minut suosiollisesti vastaan.
\par 11 Ota siis tervehdyslahjani, joka sinulle tuotiin, sillä Jumala on ollut minulle armollinen, ja minulla on yllin kyllin kaikkea." Ja hän pyysi häntä pyytämällä, kunnes hän otti sen.
\par 12 Sitten Eesau sanoi: "Lähtekäämme liikkeelle ja vaeltakaamme eteenpäin; minä vaellan sinun edelläsi".
\par 13 Mutta Jaakob sanoi hänelle: "Herrani näkee itse, että lapset ovat pieniä ja että minulla on imettäviä lampaita ja lehmiä mukanani; jos näitä ajaa kiivaasti päivänkään, kuolee koko lauma.
\par 14 Kulkekoon siis herrani palvelijansa edellä; minä seuraan hiljalleen jäljessä, sen mukaan kuin karja, jota kuljetan, ja lapset jaksavat käydä, kunnes saavun herrani luo Seiriin."
\par 15 Eesau vastasi: "Minä jätän luoksesi osan väestäni". Hän sanoi: "Minkätähden niin? Kunhan vain saan armon herrani silmien edessä!"
\par 16 Niin Eesau sinä päivänä kääntyi takaisin ja meni sitä tietään Seiriin.
\par 17 Mutta Jaakob lähti Sukkotiin ja rakensi siellä itsellensä majan. Ja karjallensa hän teki tarhoja. Siitä tuli sen paikan nimeksi Sukkot.
\par 18 Ja Jaakob saapui matkallansa Mesopotamiasta onnellisesti Sikemin kaupunkiin, joka on Kanaanin maassa, ja leiriytyi kaupungin edustalle.
\par 19 Ja hän osti sen maapalstan, johon hän oli pystyttänyt telttansa, Hamorin, Sikemin isän, pojilta sadalla kesitalla.
\par 20 Ja hän pystytti siihen alttarin ja antoi sille nimen Eel, Israelin Jumala.

\chapter{34}

\par 1 Ja Diina, Leean tytär, jonka hän oli synnyttänyt Jaakobille, meni tapaamaan sen maan tyttäriä.
\par 2 Ja Sikem, joka oli hivviläisen Hamorin, sen maan ruhtinaan, poika, näki hänet, otti hänet luokseen, makasi hänen kanssaan ja teki hänelle väkivaltaa.
\par 3 Ja hänen sydämensä kiintyi Diinaan, Jaakobin tyttäreen, ja hän rakasti tyttöä ja viihdytteli häntä.
\par 4 Ja Sikem puhui isällensä Hamorille sanoen: "Hanki minulle tämä tyttö vaimoksi".
\par 5 Ja Jaakob oli saanut kuulla, että Sikem oli raiskannut hänen tyttärensä Diinan; mutta kun hänen poikansa olivat hänen laumansa kanssa kedolla, oli Jaakob vaiti siksi, kunnes he palasivat.
\par 6 Ja Hamor, Sikemin isä, meni Jaakobin luo puhuttelemaan häntä.
\par 7 Ja Jaakobin pojat tulivat kedolta. Kuultuaan, mitä oli tapahtunut, miehet sydäntyivät ja vihastuivat kovin siitä, että hän oli tehnyt häpeällisen teon Israelissa, kun oli maannut Jaakobin tyttären; sillä semmoista ei saa tehdä.
\par 8 Niin Hamor puhui heille sanoen: "Poikani Sikemin sydän on mielistynyt teidän sisareenne; antakaa hänet hänelle vaimoksi.
\par 9 Lankoutukaa meidän kanssamme, antakaa te tyttäriänne meille, ja ottakaa te itsellenne meidän tyttäriämme,
\par 10 ja jääkää asumaan meidän luoksemme. Maa on oleva teille avoinna, asukaa siinä ja kierrelkää siinä ja asettukaa sinne."
\par 11 Ja Sikem sanoi tytön isälle ja veljille: "Suokaa minun saada armo silmienne edessä, ja mitä te vaaditte minulta, sen minä annan.
\par 12 Vaatikaa minulta kuinka suuri morsiamen hinta ja kuinka suuret antimet tahansa: minä annan, mitä te minulta vaaditte. Antakaa minulle vain se tyttö vaimoksi."
\par 13 Mutta Jaakobin pojat vastasivat Sikemille ja hänen isällensä Hamorille puhuen petollisesti, sentähden että hän oli raiskannut heidän sisarensa Diinan;
\par 14 he sanoivat heille: "Emme voi tehdä sitä, että antaisimme sisaremme ympärileikkaamattomalle miehelle, sillä se olisi meistä häpeällistä.
\par 15 Me suostumme ainoastaan sillä ehdolla, että te tulette meidän kaltaisiksemme, niin että kaikki miehenpuolenne ympärileikataan.
\par 16 Silloin me annamme omia tyttäriämme teille ja otamme itsellemme teidän tyttäriänne, ja me asetumme teidän luoksenne ja tulemme yhdeksi kansaksi.
\par 17 Mutta jos te ette kuule meitä ettekä ympärileikkauta itseänne, niin me otamme sisaremme ja menemme tiehemme."
\par 18 Ja heidän puheensa kelpasi Hamorille ja Sikemille, Hamorin pojalle.
\par 19 Eikä nuori mies viivytellyt niin tekemästä, sillä hän oli mieltynyt Jaakobin tyttäreen; ja hän oli suuremmassa arvossa kuin kukaan muu hänen isänsä perheessä.
\par 20 Niin menivät Hamor ja hänen poikansa Sikem kaupunkinsa porttiin ja puhuivat kaupunkinsa miehille sanoen:
\par 21 "Näillä miehillä on rauha mielessä meitä kohtaan; antakaamme heidän siis asettua tähän maahan ja kierrellä siinä, onhan maa tilava heillekin joka suuntaan. Ottakaamme me heidän tyttäriänsä vaimoiksemme, ja antakaamme me tyttäriämme heille.
\par 22 Mutta ainoastaan sillä ehdolla miehet suostuvat asettumaan meidän luoksemme, tullaksensa meidän kanssamme yhdeksi kansaksi, että kaikki miehenpuolemme ympärileikataan, niinkuin he itsekin ovat ympärileikatut.
\par 23 Tulevathan silloin heidän karjansa ja tavaransa ja kaikki heidän juhtansa meidän omiksemme. Suostukaamme siis, niin he asettuvat luoksemme."
\par 24 Ja he kuulivat Hamoria ja hänen poikaansa Sikemiä; kaikki, jotka kulkivat hänen kaupunkinsa portista; ja niin kaikki miehenpuolet, kaikki, jotka kulkivat hänen kaupunkinsa portista, ympärileikkauttivat itsensä.
\par 25 Mutta kolmantena päivänä, kun he olivat kipeimmillään, Jaakobin kaksi poikaa, Simeon ja Leevi, Diinan veljet, ottivat kumpikin miekkansa ja hyökkäsivät, kenenkään aavistamatta, kaupunkiin ja tappoivat jokaisen miehenpuolen.
\par 26 Myöskin Hamorin ja hänen poikansa Sikemin he tappoivat miekan terällä, ottivat Diinan Sikemin talosta ja menivät pois.
\par 27 Ja Jaakobin pojat karkasivat haavoitettujen kimppuun ja ryöstivät kaupungin, sentähden että he olivat raiskanneet heidän sisarensa.
\par 28 Ja he anastivat heidän pikkukarjansa ja raavaskarjansa ja aasinsa ja kaikki, mitä oli sekä kaupungissa että kedolla.
\par 29 Ja kaikki heidän tavaransa, kaikki heidän lapsensa ja vaimonsa he veivät saaliinaan, ja he ryöstivät samoin kaiken muun, mitä taloissa oli.
\par 30 Niin Jaakob sanoi Simeonille ja Leeville: "Te olette syösseet minut onnettomuuteen, saattaneet minut tämän maan asukkaiden, kanaanilaisten ja perissiläisten, vihoihin. Minun joukkoni on vähäinen; jos he kokoontuvat minua vastaan, niin he tuhoavat minut, ja niin minut ja minun sukuni hävitetään."
\par 31 Mutta he vastasivat: "Pitikö hänen kohdella meidän sisartamme niinkuin porttoa!"

\chapter{35}

\par 1 Ja Jumala sanoi Jaakobille: "Nouse, mene Beeteliin, asetu sinne ja rakenna sinne alttari Jumalalle, joka ilmestyi sinulle paetessasi veljeäsi Eesauta".
\par 2 Niin Jaakob sanoi perheellensä ja kaikille, jotka olivat hänen kanssaan: "Poistakaa vieraat jumalat, joita teillä on keskuudessanne, puhdistautukaa ja muuttakaa vaatteenne.
\par 3 Ja nouskaamme ja menkäämme Beeteliin, rakentaakseni sinne alttarin Jumalalle, joka kuuli minua ahdistukseni aikana ja oli minun kanssani tiellä, jota vaelsin."
\par 4 Niin he jättivät Jaakobille kaikki vieraat jumalat, jotka olivat heidän hallussansa, sekä renkaat, jotka olivat heidän korvissaan, ja Jaakob kätki ne maahan tammen alle, joka oli Sikemissä.
\par 5 Ja he lähtivät liikkeelle; ja Jumalan kauhu valtasi heidän ympärillään olevat kaupungit, niin etteivät nämä ajaneet takaa Jaakobin poikia.
\par 6 Ja Jaakob saapui Luusiin, joka on Kanaanin maassa, se on Beeteliin, kaiken väen kanssa, joka oli hänen seurassaan.
\par 7 Ja hän rakensi sinne alttarin ja nimitti paikan Eel-Beeteliksi, koska Jumala oli siellä ilmestynyt hänelle, silloin kun hän pakeni veljeään.
\par 8 Mutta Debora, Rebekan imettäjä, kuoli, ja hänet haudattiin Beetelin alapuolelle tammen alle, ja se sai siitä nimen "Itkutammi".
\par 9 Ja Jumala ilmestyi jälleen Jaakobille hänen palattuaan Mesopotamiasta ja siunasi hänet.
\par 10 Ja Jumala sanoi hänelle: "Sinun nimesi on Jaakob; mutta älköön sinua enää kutsuttako Jaakobiksi, vaan nimesi olkoon Israel". - Niin hän sai nimen Israel.
\par 11 Ja Jumala sanoi hänelle: "Minä olen Jumala, Kaikkivaltias; ole hedelmällinen ja lisäänny. Kansa, suuri kansojen joukko on sinusta tuleva, ja kuninkaita lähtee sinun kupeistasi.
\par 12 Ja maan, jonka minä olen antanut Aabrahamille ja Iisakille, minä annan sinulle; myöskin sinun jälkeläisillesi minä annan sen maan."
\par 13 Ja Jumala kohosi ylös hänen luotaan siitä paikasta, jossa hän oli häntä puhutellut.
\par 14 Ja Jaakob pystytti patsaan siihen paikkaan, jossa hän oli häntä puhutellut, kivipatsaan, ja vuodatti juomauhrin sen päälle ja kaatoi öljyä sen päälle.
\par 15 Ja Jaakob nimitti sen paikan, jossa Jumala oli häntä puhutellut, Beeteliksi.
\par 16 Sitten he lähtivät liikkeelle Beetelistä. Ja kun vielä oli jonkun verran matkaa Efrataan, joutui Raakel synnytystuskiin, ja hänen synnytystuskansa olivat hyvin kovat.
\par 17 Ja kun hänen synnytystuskansa olivat kovimmillaan, sanoi kätilövaimo hänelle: "Älä pelkää, sillä tälläkin kertaa sinä saat pojan".
\par 18 Mutta kun hänen henkensä oli lähtemäisillään, sillä hänen oli kuoltava, antoi hän hänelle nimen Benoni, mutta hänen isänsä antoi hänelle nimen Benjamin.
\par 19 Niin Raakel kuoli siellä, ja hänet haudattiin Efratan tien varteen, se on Beetlehemiin.
\par 20 Ja Jaakob pystytti hänen haudalleen patsaan; tämä Raakelin hautapatsas on olemassa vielä tänäkin päivänä.
\par 21 Ja Israel lähti liikkeelle sieltä ja pystytti telttansa tuolle puolen Karjatornia.
\par 22 Ja tapahtui, kun Israel asui siinä maassa, että Ruuben meni ja makasi Bilhan, isänsä sivuvaimon, kanssa. Ja Israel sai sen kuulla.
\par 23 Jaakobilla oli kaksitoista poikaa. Leean pojat olivat Ruuben, Jaakobin esikoinen, Simeon, Leevi, Juuda, Isaskar ja Sebulon.
\par 24 Raakelin pojat olivat Joosef ja Benjamin.
\par 25 Bilhan, Raakelin orjattaren, pojat olivat Daan ja Naftali.
\par 26 Silpan, Leean orjattaren, pojat olivat Gaad ja Asser. Nämä ovat ne Jaakobin pojat, jotka syntyivät hänelle Mesopotamiassa.
\par 27 Ja Jaakob saapui isänsä Iisakin luo Mamreen, Kirjat-Arbaan, se on Hebroniin, jossa Aabraham ja Iisak olivat asuneet muukalaisina.
\par 28 Ja Iisakin elinaika oli sata kahdeksankymmentä vuotta.
\par 29 Ja Iisak vaipui kuolemaan ja tuli otetuksi heimonsa tykö, vanhana ja elämästä kyllänsä saaneena. Ja hänen poikansa Eesau ja Jaakob hautasivat hänet.

\chapter{36}

\par 1 Tämä on kertomus Eesaun, se on Edomin, suvusta.
\par 2 Eesau otti vaimoikseen Kanaanin tyttäristä Aadan, heettiläisen Eelonin tyttären, ja Oholibaman, joka oli hivviläisen Sibonin pojan Anan tytär,
\par 3 ja Baasematin, Ismaelin tyttären, Nebajotin sisaren.
\par 4 Aada synnytti Eesaulle Elifaan, ja Baasemat synnytti Reguelin.
\par 5 Ja Oholibama synnytti Jeuksen, Jalamin ja Koorahin. Nämä olivat Eesaun pojat, jotka syntyivät hänelle Kanaanin maassa.
\par 6 Ja Eesau otti vaimonsa, poikansa ja tyttärensä ja kaiken talonväkensä, karjansa ja kaikki juhtansa ja kaiken tavaransa, jonka hän oli hankkinut Kanaanin maassa; ja hän lähti toiseen maahan, pois veljensä Jaakobin luota.
\par 7 Sillä heidän omaisuutensa oli niin suuri, etteivät he voineet asua yhdessä, eikä se maa, jossa he asuivat muukalaisina, riittänyt heille heidän karjansa paljouden tähden.
\par 8 Ja Eesau asettui Seirin vuoristoon, Eesau, se on Edom.
\par 9 Ja tämä on kertomus edomilaisten isän Eesaun suvusta, hänen, joka asui Seirin vuoristossa.
\par 10 Nämä ovat Eesaun poikien nimet: Elifas, Eesaun vaimon Aadan poika; Reguel, Eesaun vaimon Baasematin poika.
\par 11 Ja Elifaan pojat olivat Teeman, Oomar, Sefo, Gaetam ja Kenas.
\par 12 Mutta Timna oli Elifaan, Eesaun pojan, sivuvaimo, ja hän synnytti Elifaalle Amalekin. Nämä olivat Eesaun vaimon Aadan pojat.
\par 13 Reguelin pojat olivat nämä: Nahat ja Serah, Samma ja Missa. Ne olivat Eesaun vaimon Baasematin pojat.
\par 14 Mutta nämä olivat Eesaun vaimon Oholibaman, Sibonin pojan Anan tyttären, pojat, jotka hän synnytti Eesaulle: Jeus, Jalam ja Koorah.
\par 15 Nämä olivat Eesaun poikien sukuruhtinaat: Elifaan, Eesaun esikoisen, pojat olivat ruhtinas Teeman, ruhtinas Oomar, ruhtinas Sefo, ruhtinas Kenas,
\par 16 ruhtinas Koorah, ruhtinas Gaetam, ruhtinas Amalek. Nämä olivat ne ruhtinaat, jotka polveutuivat Elifaasta Edomin maassa; ne olivat Aadan pojat.
\par 17 Ja nämä olivat Reguelin, Eesaun pojan, pojat: ruhtinas Nahat, ruhtinas Serah, ruhtinas Samma, ruhtinas Missa. Nämä olivat ne ruhtinaat, jotka polveutuivat Reguelista Edomin maassa; ne olivat Eesaun vaimon Baasematin pojat.
\par 18 Ja nämä olivat Eesaun vaimon Oholibaman pojat: ruhtinas Jeus, ruhtinas Jalam, ruhtinas Koorah. Nämä olivat ne ruhtinaat, jotka polveutuivat Eesaun vaimosta, Anan tyttärestä Oholibamasta.
\par 19 Nämä olivat Eesaun, se on Edomin, pojat, ja nämä heidän sukuruhtinaansa.
\par 20 Mutta nämä olivat hoorilaisen Seirin pojat, sen maan alkuasukkaat: Lootan, Soobal, Sibon, Ana,
\par 21 Diison, Eeser ja Diisan. Nämä olivat hoorilaisten sukuruhtinaat, Seirin pojat, Edomin maassa.
\par 22 Mutta Lootanin pojat olivat Hoori ja Heemam; ja Lootanin sisar oli Timna.
\par 23 Ja nämä olivat Soobalin pojat: Alvan, Maanahat ja Eebal, Sefo ja Oonam.
\par 24 Nämä olivat Sibonin pojat: Aija ja Ana. Tämä oli se Ana, joka löysi ne lämpimät lähteet erämaassa, paimentaessaan isänsä Sibonin aaseja.
\par 25 Ja nämä olivat Anan lapset: Diison ja Oholiba, Anan tytär.
\par 26 Nämä olivat Diisonin pojat: Hemdan, Esban, Jitran ja Keran.
\par 27 Nämä olivat Eeserin pojat: Bilhan, Saavan ja Akan.
\par 28 Nämä olivat Diisanin pojat: Uus ja Aran.
\par 29 Nämä olivat hoorilaisten sukuruhtinaat: ruhtinas Lootan, ruhtinas Soobal, ruhtinas Sibon, ruhtinas Ana,
\par 30 ruhtinas Diison, ruhtinas Eeser, ruhtinas Diisan. Nämä olivat hoorilaisten sukuruhtinaat, ruhtinas ruhtinaalta, Seirin maassa.
\par 31 Ja nämä olivat ne kuninkaat, jotka hallitsivat Edomin maassa, ennenkuin yksikään kuningas oli hallinnut israelilaisia:
\par 32 Bela, Beorin poika, oli kuninkaana Edomissa, ja hänen kaupunkinsa nimi oli Dinhaba.
\par 33 Kun Bela kuoli, tuli Joobab, Serahin poika, Bosrasta, kuninkaaksi hänen sijaansa.
\par 34 Kun Joobab kuoli, tuli Huusam, teemanilaisten maasta, kuninkaaksi hänen sijaansa.
\par 35 Kun Huusam kuoli, tuli Hadad, Bedadin poika, kuninkaaksi hänen sijaansa, hän, joka voitti midianilaiset Mooabin kedolla; ja hänen kaupunkinsa nimi oli Avit.
\par 36 Kun Hadad kuoli, tuli Samla, Masrekasta, kuninkaaksi hänen sijaansa.
\par 37 Kun Samla kuoli, tuli Saul, Rehobotista, virran varrelta, kuninkaaksi hänen sijaansa.
\par 38 Kun Saul kuoli, tuli Baal-Haanan, Akborin poika, kuninkaaksi hänen sijaansa.
\par 39 Kun Baal-Haanan, Akborin poika, kuoli, tuli Hadar kuninkaaksi hänen sijaansa, ja hänen kaupunkinsa nimi oli Paagu; ja hänen vaimonsa nimi oli Mehetabel, joka oli Matredin, Mee-Saahabin tyttären, tytär.
\par 40 Ja nämä ovat Eesaun sukuruhtinasten nimet, heidän sukujensa, asuinpaikkojensa ja nimiensä mukaan: ruhtinas Timna, ruhtinas Alva, ruhtinas Jetet,
\par 41 ruhtinas Oholibama, ruhtinas Eela, ruhtinas Piinon,
\par 42 ruhtinas Kenas, ruhtinas Teeman, ruhtinas Mibsar,
\par 43 ruhtinas Magdiel, ruhtinas Iiram. Nämä olivat Edomin sukuruhtinaat, heidän asuinsijojensa mukaan heidän perintömaassaan - Edomin, se on Eesaun, edomilaisten isän, sukuruhtinaat.

\chapter{37}

\par 1 Mutta Jaakob asui siinä maassa, jossa hänen isänsä oli oleskellut muukalaisena, Kanaanin maassa.
\par 2 Tämä on kertomus Jaakobin suvusta. Kun Joosef oli seitsemäntoista vuoden vanha, oli hän veljiensä kanssa lampaita paimentamassa; hän oli nuorukaisena isänsä vaimojen Bilhan ja Silpan poikien seurassa. Ja hän kertoi isälleen, mitä pahaa kuuli heistä puhuttavan.
\par 3 Ja Israel rakasti Joosefia enemmän kuin kaikkia muita poikiansa, koska hän oli syntynyt hänelle hänen vanhalla iällänsä, ja hän teetti hänelle pitkäliepeisen, hihallisen ihokkaan.
\par 4 Kun hänen veljensä näkivät, että heidän isänsä rakasti häntä enemmän kuin kaikkia hänen veljiänsä, vihasivat he häntä eivätkä voineet puhutella häntä ystävällisesti.
\par 5 Kerran Joosef näki unen ja kertoi sen veljilleen; sen jälkeen he vihasivat häntä vielä enemmän.
\par 6 Hän näet sanoi heille: "Kuulkaa, minkä unen minä olen nähnyt.
\par 7 Katso, me olimme sitovinamme lyhteitä vainiolla, ja katso, minun lyhteeni nousi seisomaan, ja teidän lyhteenne asettuivat ympärille ja kumarsivat minun lyhdettäni."
\par 8 Niin hänen veljensä sanoivat hänelle: "Sinäkö tulisit meidän kuninkaaksemme, sinäkö hallitsisit meitä?" Ja he vihasivat häntä vielä enemmän hänen uniensa ja puheidensa tähden.
\par 9 Ja hän näki vielä toisenkin unen, jonka hän kertoi veljilleen ja sanoi: "Minä näin vielä unen: katso, aurinko ja kuu ja yksitoista tähteä kumarsivat minua".
\par 10 Ja kun hän kertoi sen isälleen ja veljilleen, nuhteli hänen isänsä häntä ja sanoi hänelle: "Mikä uni se on, jonka sinä olet nähnyt? Olisiko minun ja äitisi ja veljiesi tultava kumartumaan sinun eteesi maahan?"
\par 11 Ja hänen veljensä kadehtivat häntä; mutta hänen isänsä pani tämän mieleensä.
\par 12 Kun hänen veljensä olivat menneet kaitsemaan isänsä lampaita Sikemiin,
\par 13 sanoi Israel Joosefille: "Sinun veljesi ovat paimenessa Sikemissä; tule, minä lähetän sinut heidän luokseen". Hän vastasi: "Tässä olen".
\par 14 Ja hän sanoi hänelle: "Mene katsomaan, kuinka veljesi ja karja voivat, ja kerro sitten minulle". Niin hän lähetti hänet matkalle Hebronin laaksosta, ja hän tuli Sikemiin.
\par 15 Ja muuan mies kohtasi hänet, hänen harhaillessaan kedolla; ja mies kysyi häneltä: "Mitä etsit?"
\par 16 Hän vastasi: "Minä etsin veljiäni; sano minulle, missä he ovat paimentamassa".
\par 17 Mies vastasi: "He lähtivät pois täältä, sillä minä kuulin heidän sanovan: 'Menkäämme Dootaniin'." Niin Joosef meni veljiensä jäljissä ja löysi heidät Dootanista.
\par 18 Kun he kaukaa näkivät hänet ja ennenkuin hän saapui heidän luokseen, pitivät he neuvoa tappaaksensa hänet.
\par 19 He sanoivat toisillensa: "Katso, tuolla tulee se unennäkijä!
\par 20 Tulkaa, tappakaamme nyt hänet ja heittäkäämme hänet johonkin kaivoon ja sanokaamme: villipeto on hänet syönyt. Saammepa sitten nähdä, mitä hänen unistaan tulee."
\par 21 Kun Ruuben sen kuuli, tahtoi hän pelastaa hänet heidän käsistään ja sanoi: "Älkäämme lyökö häntä kuoliaaksi".
\par 22 Vielä Ruuben sanoi heille: "Älkää vuodattako verta; heittäkää hänet tähän kaivoon, joka on täällä erämaassa, mutta älkää satuttako kättänne häneen". Hän näet tahtoi pelastaa hänet heidän käsistään, saattaaksensa hänet takaisin isänsä tykö.
\par 23 Kun Joosef sitten tuli veljiensä luo, riisuivat he Joosefilta hänen ihokkaansa, pitkäliepeisen, hihallisen ihokkaan, joka oli hänen yllään,
\par 24 ja ottivat hänet ja heittivät hänet kaivoon; mutta kaivo oli tyhjä, siinä ei ollut vettä.
\par 25 Senjälkeen he istuivat aterioimaan. Ja kun he nostivat silmänsä, näkivät he ismaelilaismatkueen tulevan Gileadista; heidän kamelinsa kuljettivat kumihartsia, balsamia ja hajupihkaa, ja he olivat viemässä niitä Egyptiin.
\par 26 Silloin Juuda sanoi veljillensä: "Mitä hyötyä meillä on siitä, että surmaamme veljemme ja salaamme hänen verensä?
\par 27 Tulkaa, myykäämme hänet ismaelilaisille, mutta älköön kätemme sattuko häneen, sillä hän on meidän veljemme, meidän omaa lihaamme." Ja hänen veljensä kuulivat häntä.
\par 28 Kun nyt midianilaiset kauppiaat menivät siitä ohitse, vetivät veljet Joosefin ylös kaivosta; ja he myivät Joosefin kahdestakymmenestä hopeasekelistä ismaelilaisille. Nämä veivät Joosefin Egyptiin.
\par 29 Kun sitten Ruuben palasi kaivolle, niin Joosefia ei enää ollut kaivossa. Silloin hän repäisi vaatteensa,
\par 30 palasi veljiensä luo ja sanoi: "Poika on kadonnut. Voi minua, minne minä joudun!"
\par 31 Niin he ottivat Joosefin ihokkaan, teurastivat vuohipukin ja kastoivat ihokkaan vereen.
\par 32 Sitten he lähettivät tuon pitkäliepeisen, hihallisen ihokkaan kotiin isälleen ja sanoivat: "Tämän me löysimme; tarkasta, onko se poikasi ihokas vai eikö".
\par 33 Ja tarkastettuaan sen hän sanoi: "Tämä on minun poikani ihokas. Villipeto on hänet syönyt; totisesti, Joosef on raadeltu kuoliaaksi."
\par 34 Ja Jaakob repäisi vaatteensa, pani säkin lanteilleen ja suri poikaansa pitkät ajat.
\par 35 Ja kaikki hänen poikansa ja tyttärensä kävivät häntä lohduttamaan, mutta hän ei huolinut lohdutuksesta, vaan sanoi: "Murehtien minä menen tuonelaan poikani tykö". Ja hänen isänsä itki häntä.
\par 36 Mutta midianilaiset myivät hänet Egyptiin Potifarille, joka oli faraon hoviherra ja henkivartijain päämies.

\chapter{38}

\par 1 Siihen aikaan Juuda lähti pois veljiensä luota ja asettui erään Adullamissa asuvan miehen luo, jonka nimi oli Hiira.
\par 2 Siellä Juuda näki Suua nimisen kanaanilaisen miehen tyttären, ja hän otti tämän luokseen ja yhtyi häneen.
\par 3 Ja hän tuli raskaaksi ja synnytti pojan; ja hän antoi hänelle nimen Eer.
\par 4 Ja taas hän tuli raskaaksi ja synnytti pojan ja antoi hänelle nimen Oonan.
\par 5 Ja hän synnytti vieläkin pojan ja antoi hänelle nimen Seela; ja synnyttäessään hänet hän oli Kesibissä.
\par 6 Ja Juuda otti Eerille, esikoisellensa, vaimon, jonka nimi oli Taamar.
\par 7 Mutta Eer, Juudan esikoinen, ei ollut Herralle otollinen; sentähden Herra antoi hänen kuolla.
\par 8 Niin Juuda sanoi Oonanille: "Yhdy veljesi leskeen ja ota hänet avioksesi ja herätä siemen veljellesi".
\par 9 Mutta kun Oonan tiesi, ettei jälkeläinen olisi oleva hänen, niin hän antoi, aina kun yhtyi veljensä vaimoon, siemenensä mennä maahan, ettei hankkisi jälkeläistä veljelleen.
\par 10 Mutta se, minkä hän teki, oli paha Herran silmissä; sentähden hän antoi hänenkin kuolla.
\par 11 Silloin Juuda sanoi miniällensä Taamarille: "Asu leskenä isäsi talossa, kunnes poikani Seela joutuu täysikasvuiseksi". Hän näet ajatteli: "Kun ei vain tämäkin kuolisi niinkuin hänen veljensä". Niin Taamar meni pois ja jäi asumaan isänsä kotiin.
\par 12 Pitkän aikaa sen jälkeen Suuan tytär, Juudan vaimo, kuoli. Suruajan mentyä Juuda lähti ystävänsä adullamilaisen Hiiran kanssa Timnaan lammastensa keritsiäisiin.
\par 13 Niin tuotiin Taamarille tämä sanoma: "Katso, appesi menee Timnaan keritsemään lampaitaan".
\par 14 Silloin hän riisui pois leskenvaatteensa ja verhoutui huntuun peittäytyen siihen ja istui Eenaimin portille, Timnaan vievän tien varteen. Sillä hän oli nähnyt, että vaikka Seela oli täysikasvuinen, ei häntä annettu hänelle vaimoksi.
\par 15 Kun nyt Juuda näki hänet, luuli hän häntä portoksi; sillä hän oli peittänyt kasvonsa.
\par 16 Ja hän poikkesi hänen luokseen tiepuoleen ja sanoi: "Anna minun yhtyä sinuun". Hän näet ei tiennyt, että nainen oli hänen miniänsä. Tämä vastasi: "Mitä annat minulle saadaksesi yhtyä minuun?"
\par 17 Hän sanoi: "Lähetän sinulle vohlan laumastani". Nainen vastasi: "Annatko minulle pantin, kunnes sen lähetät?"
\par 18 Hän sanoi: "Mitä on minun annettava sinulle pantiksi?" Hän vastasi: "Sinettisi, nauhasi ja sauvasi, joka on kädessäsi". Niin hän antoi ne hänelle ja yhtyi häneen, ja hän tuli hänestä raskaaksi.
\par 19 Ja hän nousi ja meni sieltä ja pani pois huntunsa ja pukeutui leskenvaatteisiinsa.
\par 20 Mutta Juuda lähetti adullamilaisen ystävänsä viemään vohlaa, saadaksensa takaisin pantin naiselta; mutta hän ei löytänyt häntä.
\par 21 Ja hän kyseli sen paikkakunnan miehiltä ja sanoi: "Missä on se pyhäkköportto, joka istui Eenaimissa tien varressa?" He vastasivat: "Ei täällä ole ollut mitään pyhäkköporttoa".
\par 22 Ja hän palasi Juudan luo ja sanoi: "En löytänyt häntä; ja myös sen paikkakunnan miehet sanoivat, ettei siellä ole ollutkaan mitään pyhäkköporttoa".
\par 23 Silloin Juuda sanoi: "Pitäköön sen sitten, ettemme joutuisi häpeään. Katso, minä olen lähettänyt tämän vohlan, mutta sinä et ole löytänyt häntä."
\par 24 Noin kolmen kuukauden kuluttua ilmoitettiin Juudalle: "Sinun miniäsi Taamar on harjoittanut haureutta, ja haureudesta hän on myös tullut raskaaksi". Juuda sanoi: "Viekää hänet poltettavaksi".
\par 25 Mutta kun häntä vietiin, lähetti hän sanan apelleen sanoen: "Minä olen raskaana siitä miehestä, jonka nämä ovat". Ja hän käski sanoa: "Tarkasta, kenen tämä sinetti, nämä nauhat ja tämä sauva ovat".
\par 26 Ja Juuda tunsi ne ja sanoi: "Hän on oikeassa minua vastaan, koska minä en antanut häntä pojalleni Seelalle". Ei hän kuitenkaan enää yhtynyt häneen.
\par 27 Kun hänen synnyttämisensä aika tuli, katso, hänen kohdussaan oli kaksoiset.
\par 28 Ja hänen synnyttäessään pisti toinen kätensä ulos; kätilövaimo otti punaista lankaa ja sitoi sen hänen käteensä ja sanoi: "Tämä tuli ensiksi ulos".
\par 29 Mutta kun hän sitten taas veti kätensä takaisin, katso, silloin tuli hänen veljensä ulos; ja kätilövaimo sanoi: "Minkä repeämän oletkaan reväissyt itsellesi!" Ja hän sai nimen Peres.
\par 30 Sitten tuli hänen veljensä, jonka kädessä oli punainen lanka; ja hän sai nimen Serah.

\chapter{39}

\par 1 Ja Joosef vietiin Egyptiin, ja Potifar, egyptiläinen mies, joka oli faraon hoviherra ja henkivartijain päämies, osti hänet ismaelilaisilta, jotka olivat hänet sinne tuoneet.
\par 2 Mutta Herra oli Joosefin kanssa, niin että hän menestyi kaikessa, ja hän oleskeli isäntänsä, egyptiläisen, talossa.
\par 3 Ja hänen isäntänsä näki, että Herra oli hänen kanssaan ja että Herra antoi kaiken, mitä hän teki, menestyä hänen käsissään.
\par 4 Niin Joosef saavutti hänen suosionsa ja sai palvella häntä. Ja hän asetti hänet talonsa hoitajaksi ja uskoi hänen haltuunsa kaikki, mitä hänellä oli.
\par 5 Ja siitä ajasta lähtien, kun hän oli uskonut hänen hoitoonsa talonsa ja kaiken, mitä hänellä oli, Herra siunasi egyptiläisen taloa Joosefin tähden, ja Herran siunaus oli kaikessa, mitä hänellä oli kotona ja kedolla.
\par 6 Niin hän jätti Joosefin haltuun kaikki, mitä hänellä oli, eikä hän itse välittänyt mistään, paitsi ruuasta, jota hän söi.
\par 7 Mutta Joosefilla oli kaunis vartalo ja kauniit kasvot. Ja tapahtui jonkun ajan kuluttua, että hänen isäntänsä puoliso iski silmänsä Joosefiin ja sanoi: "Makaa minun kanssani".
\par 8 Mutta hän kieltäytyi ja sanoi isäntänsä puolisolle: "Katso, isäntäni ei itse välitä mistään, mitä talossa on, vaan on uskonut minun haltuuni kaikki, mitä hänellä on.
\par 9 Hänellä ei ole tässä talossa enemmän valtaa kuin minulla, eikä hän ole kieltänyt minulta mitään, paitsi sinut, koska olet hänen puolisonsa. Kuinka minä siis tekisin niin suuren pahanteon ja rikkoisin Jumalaa vastaan!"
\par 10 Ja vaikka vaimo joka päivä puhui sellaista Joosefille, ei tämä suostunut makaamaan hänen vieressänsä eikä olemaan hänen kanssaan.
\par 11 Mutta eräänä päivänä, kun Joosef tuli taloon toimittamaan askareitaan eikä ketään talonväestä ollut sisällä saapuvilla,
\par 12 tarttui hän Joosefin vaippaan ja sanoi: "Makaa minun kanssani". Mutta tämä jätti vaippansa hänen käsiinsä, pakeni ja riensi ulos.
\par 13 Kun hän nyt huomasi, että Joosef oli jättänyt vaippansa hänen käsiinsä ja paennut ulos,
\par 14 huusi hän talonväkeään ja sanoi heille näin: "Katsokaa, hän on tuonut meille hebrealaisen miehen pitämään meitä pilkkanaan; tämä tuli luokseni maatakseen minun kanssani, mutta minä huusin kovalla äänellä.
\par 15 Ja kun hän kuuli minun kirkaisevan ja huutavan, jätti hän vaippansa viereeni, pakeni ja riensi ulos."
\par 16 Ja hän pani vaipan viereensä siksi, kunnes Joosefin isäntä tuli kotiin.
\par 17 Ja hän puhui hänelle samalla tavalla, sanoen: "Tuo hebrealainen orja, jonka olet tuonut meille, tuli luokseni pitämään minua pilkkanaan;
\par 18 mutta kun minä kirkaisin ja huusin, jätti hän vaippansa viereeni ja pakeni ulos".
\par 19 Kun hänen isäntänsä kuuli puolisonsa kertovan ja sanovan: "Näin sinun orjasi on tehnyt minulle", syttyi hänen vihansa.
\par 20 Ja Joosefin isäntä otti hänet ja pani hänet vankilaan, paikkaan, jossa kuninkaan vangit säilytettiin; ja hän jäi siihen vankilaan.
\par 21 Mutta Herra oli Joosefin kanssa ja soi hänen saavuttaa suosiota ja päästä vankilan päällikön armoihin.
\par 22 Ja vankilan päällikkö uskoi kaikki vangit, jotka vankilassa olivat, Joosefin haltuun. Ja kaikki, mitä siellä toimitettiin, tehtiin hänen toimestaan.
\par 23 Eikä vankilan päällikkö ollenkaan valvonut sitä, mikä oli Joosefille uskottu, sillä Herra oli Joosefin kanssa. Ja Herra antoi menestyä sen, mitä hän teki.

\chapter{40}

\par 1 Ja tapahtui jonkun ajan kuluttua, että Egyptin kuninkaan juomanlaskija ja leipoja rikkoivat herraansa, Egyptin kuningasta, vastaan.
\par 2 Ja farao vihastui näihin kahteen hoviherraansa, ylimmäiseen juomanlaskijaan ja ylimmäiseen leipojaan,
\par 3 ja panetti heidät vankeuteen henkivartijain päämiehen taloon, samaan vankilaan, jossa Joosef oli vankina.
\par 4 Ja henkivartijain päämies antoi heille Joosefin heitä palvelemaan. Niin he olivat jonkun aikaa vankeudessa.
\par 5 Ollessaan vankilassa vangittuina he molemmat, Egyptin kuninkaan juomanlaskija ja leipoja, näkivät samana yönä unta, kumpikin unensa, ja kummankin unella oli oma selityksensä.
\par 6 Ja kun Joosef aamulla tuli heidän luokseen, huomasi hän heidät alakuloisiksi.
\par 7 Silloin hän kysyi faraon hoviherroilta, jotka olivat hänen kanssansa vankeudessa hänen isäntänsä talossa: "Miksi te olette tänään niin murheellisen näköiset?"
\par 8 He vastasivat hänelle: "Olemme kumpikin nähneet unen, eikä ole niiden selittäjää". Ja Joosef sanoi heille: "Unien selitykset ovat Jumalan; kertokaa kuitenkin minulle".
\par 9 Niin ylimmäinen juomanlaskija kertoi unensa Joosefille ja sanoi hänelle: "Minä näin unta, ja katso, minun edessäni oli viinipuu;
\par 10 viinipuussa oli kolme oksaa, ja samassa kun se alkoi versoa, sen kukat puhkesivat ja marjat sen rypäleissä kypsyivät.
\par 11 Ja minulla oli faraon malja kädessäni, ja minä otin marjat ja pusersin niistä mehun faraon maljaan ja annoin maljan faraon käteen."
\par 12 Ja Joosef sanoi hänelle: "Tämä on sen selitys: kolme oksaa merkitsee kolmea päivää.
\par 13 Kolmen päivän kuluttua farao korottaa sinun pääsi ja asettaa sinut jälleen virkaasi. Ja sinä annat faraon maljan hänen käteensä niinkuin ennenkin, kun olit hänen juomanlaskijansa.
\par 14 Mutta muista minua, kun sinun hyvin käy, ja tee minulle laupeus mainitsemalla minusta faraolle ja toimittamalla minut pois tästä talosta.
\par 15 Sillä minut on varastettu hebrealaisten maasta, enkä minä ole täälläkään tehnyt mitään, mistä minut olisi tullut panna tähän vankikuoppaan."
\par 16 Kun ylimmäinen leipoja näki, että Joosef antoi hyvän selityksen, sanoi hän hänelle: "Myöskin minä näin unen, ja katso, kolme nisuleipäkoria oli minun pääni päällä.
\par 17 Ja ylimmässä korissa oli kaikenlaisia leivoksia faraon syötäväksi, mutta linnut söivät ne korista, joka oli minun pääni päällä."
\par 18 Joosef vastasi ja sanoi: "Tämä on sen selitys: kolme koria merkitsee kolmea päivää.
\par 19 Kolmen päivän kuluttua farao korottaa sinun pääsi ripustamalla sinut hirsipuuhun, ja taivaan linnut syövät sinun lihasi."
\par 20 Kolmantena päivänä sen jälkeen, faraon syntymäpäivänä, tämä laittoi pidot kaikille palvelijoilleen. Silloin hän korotti palvelijainsa joukosta sekä ylimmäisen juomanlaskijan että ylimmäisen leipojan pään.
\par 21 Ylimmäisen juomanlaskijan hän asetti hänen entiseen juomanlaskijan toimeensa, niin että hän sai antaa maljan faraon käteen;
\par 22 mutta ylimmäisen leipojan hän hirtätti, niinkuin Joosef oli heille selityksessään sanonut.
\par 23 Mutta ylimmäinen juomanlaskija ei muistanut Joosefia, vaan unhotti hänet.

\chapter{41}

\par 1 Kahden vuoden kuluttua tapahtui, että farao näki unen; hän oli seisovinaan Niilivirran rannalla.
\par 2 Ja katso, virrasta nousi seitsemän kaunista ja lihavaa lehmää, jotka kävivät laitumella kaislikossa.
\par 3 Ja katso, niiden jälkeen nousi virrasta toiset seitsemän lehmää, rumia ja laihoja; ne asettuivat edellisten lehmien viereen virran rannalle.
\par 4 Ja ne rumat ja laihat lehmät söivät ne seitsemän kaunista ja lihavaa lehmää. Siihen farao heräsi.
\par 5 Mutta hän nukkui uudestaan ja näki toisen kerran unta: seitsemän paksua ja kaunista tähkäpäätä kasvoi samassa oljessa.
\par 6 Ja katso, niiden jälkeen kasvoi vielä seitsemän tähkäpäätä, ohutta ja itätuulen polttamaa.
\par 7 Ja nämä ohuet tähkäpäät nielivät ne seitsemän paksua ja täyteläistä tähkäpäätä. Siihen farao heräsi, ja katso, se oli unta.
\par 8 Mutta aamulla hänen mielensä oli levoton, ja hän kutsutti eteensä kaikki Egyptin tietäjät ja kaikki viisaat; ja farao kertoi heille unensa, mutta ei ollut sitä, joka olisi voinut selittää ne faraolle.
\par 9 Silloin puhui ylimmäinen juomanlaskija faraolle sanoen: "Nyt minä muistan rikokseni.
\par 10 Farao oli vihastunut palvelijoihinsa, ja hän pani minut vankeuteen henkivartijain päämiehen taloon, minut ja ylimmäisen leipojan.
\par 11 Niin me molemmat, minä ja hän, näimme samana yönä unta; me näimme kumpikin unemme, jolla oli oma selityksensä.
\par 12 Ja siellä oli meidän kanssamme hebrealainen nuorukainen, henkivartijain päämiehen palvelija. Hänelle me kerroimme unemme, ja hän selitti ne meille; hän selitti, mitä kummankin uni merkitsi.
\par 13 Ja niinkuin hän meille oli selittänyt, niin kävikin: minut asetettiin entiseen virkaani, toinen hirtettiin."
\par 14 Silloin farao kutsutti Joosefin eteensä. Ja hänet tuotiin kiiruusti vankikuopasta. Ja hän ajatti hiuksensa ja muutti vaatteensa ja tuli faraon eteen.
\par 15 Ja farao sanoi Joosefille: "Minä olen nähnyt unen, eikä ole sen selittäjää, mutta olen kuullut kerrottavan sinusta, että kun kuulet unen, sinä voit sen selittää".
\par 16 Joosef vastasi faraolle sanoen: "En minä; mutta Jumala antaa faraolle suotuisan vastauksen".
\par 17 Ja farao puhui Joosefille: "Minä näin unen; olin seisovinani Niilivirran rannalla.
\par 18 Ja katso, virrasta nousi seitsemän lihavaa ja kaunista lehmää, jotka kävivät laitumella kaislikossa.
\par 19 Ja katso, niiden jälkeen nousi virrasta toiset seitsemän lehmää, kurjia, kovin rumia ja laihoja; en ole koko Egyptin maassa nähnyt niin rumia kuin ne.
\par 20 Ja nämä laihat ja rumat lehmät söivät ne ensimmäiset, ne lihavat lehmät.
\par 21 Mutta vaikka ne olivat nielleet nämä, ei voinut huomata, että ne olivat nielleet ne, vaan ne olivat yhtä rumannäköiset kuin ennenkin. Siihen minä heräsin.
\par 22 Mutta taas minä näin unta: seitsemän täyteläistä ja kaunista tähkäpäätä kasvoi samassa oljessa.
\par 23 Ja katso, niiden jälkeen kasvoi vielä seitsemän tähkäpäätä, kuivunutta, ohutta ja itätuulen polttamaa.
\par 24 Ja nämä ohuet tähkäpäät nielivät ne seitsemän kaunista tähkäpäätä. Minä kerroin tämän tietäjille, mutta ei kukaan kyennyt sanomaan minulle, mitä se merkitsee."
\par 25 Niin Joosef sanoi faraolle: "Faraon unet merkitsevät kumpikin samaa; Jumala on ilmaissut faraolle, mitä hän on tekevä.
\par 26 Seitsemän kaunista lehmää merkitsee seitsemää vuotta, seitsemän kaunista tähkäpäätä merkitsee myös seitsemää vuotta; unilla on sama merkitys.
\par 27 Ja seitsemän laihaa ja rumaa lehmää, jotka nousivat niiden jälkeen, merkitsee seitsemää vuotta, ja seitsemän tyhjää, itätuulen polttamaa tähkäpäätä merkitsee seitsemää nälkävuotta.
\par 28 Tätä minä tarkoitin, kun sanoin faraolle: Jumala on antanut faraon nähdä, mitä hän on tekevä.
\par 29 Katso, tulee seitsemän vuotta, jolloin on suuri viljavuus koko Egyptin maassa.
\par 30 Mutta niitä seuraa seitsemän sellaista nälkävuotta, että Egyptin maan entinen viljavuus kokonaan unhottuu, ja nälänhätä tuottaa maalle häviön.
\par 31 Eikä enää tiedetä mitään maassa vallinneesta viljavuudesta sitä seuraavan nälänhädän vuoksi, sillä se on oleva ylen kova.
\par 32 Mutta että uni toistui faraolle, se tietää, että Jumala on asian varmasti päättänyt ja että Jumala antaa sen pian tapahtua.
\par 33 Nyt valitkoon siis farao ymmärtäväisen ja taitavan miehen ja asettakoon hänet Egyptin hallitusmieheksi.
\par 34 Näin tehköön farao: asettakoon päällysmiehiä maahan ja ottakoon viidennen osan Egyptin maan sadosta seitsemänä viljavuotena.
\par 35 Ja koottakoon näinä hyvinä vuosina, jotka tulevat, kaikki niiden sato ja kasattakoon viljaa faraon haltuun, talletettakoon sato kaupunkeihin ja säilytettäköön,
\par 36 niin että maalla on eloa säästössä seitsemän nälkävuoden varalle, jotka kohtaavat Egyptin maata. Niin ei maa joudu perikatoon nälänhädän aikana."
\par 37 Tämä puhe miellytti faraota ja kaikkia hänen palvelijoitansa.
\par 38 Ja farao sanoi palvelijoilleen: "Voisimmeko löytää ketään, jossa on Jumalan henki niinkuin tässä?"
\par 39 Ja farao sanoi Joosefille: "Koska Jumala on sinulle ilmoittanut kaiken tämän, ei ole ketään niin ymmärtäväistä ja taitavaa, kuin sinä olet.
\par 40 Hoida sinä minun taloani, ja sinun käskyäsi totelkoon kaikki minun kansani; ainoastaan valtaistuimen puolesta minä olen sinua korkeampi."
\par 41 Ja farao sanoi Joosefille: "Katso, minä asetan sinut koko Egyptin maan hallitusmieheksi".
\par 42 Ja farao otti sinettisormuksensa kädestään ja pani sen Joosefin käteen ja puetti hänen ylleen hienot pellavavaatteet ja ripusti kultakäädyt hänen kaulaansa.
\par 43 Ja hän antoi hänen ajaa omissa, lähinnä parhaissa vaunuissaan, ja hänen edellään huudettiin: abrek! Niin asetettiin hänet koko Egyptin maan hallitusmieheksi.
\par 44 Ja farao sanoi Joosefille: "Minä olen farao, mutta sinun tahtomattasi älköön kukaan nostako kättä tai jalkaa koko Egyptin maassa".
\par 45 Ja farao nimitti Joosefin Saafenat-Paneahiksi ja antoi hänelle puolisoksi Aasenatin, Oonin papin Poti-Feran tyttären. Niin Joosef lähti tarkastamaan Egyptin maata.
\par 46 Joosef oli kolmenkymmenen vuoden vanha, kun hän tuli faraon, Egyptin kuninkaan, palvelijaksi. Ja Joosef lähti faraon luota ja kulki läpi koko Egyptin maan.
\par 47 Ja seitsemänä viljavuotena maa kasvoi ylen runsaasti.
\par 48 Ja näinä seitsemänä hyvänä vuotena, jotka tulivat Egyptin maahan, hän kokosi kaikkea eloa ja talletti sen kaupunkeihin. Hän talletti kuhunkin kaupunkiin sen elon, joka tuotiin ympäristön vainioilta.
\par 49 Niin Joosef kasasi viljaa niinkuin meren hiekkaa, ylen suuret määrät, siihen asti että lakattiin sitä mittaamasta, sillä se ei ollut enää mitattavissa.
\par 50 Joosefille syntyi, ennenkuin nälkävuosi tuli, kaksi poikaa, jotka Aasenat, Oonin papin Poti-Feran tytär, synnytti hänelle.
\par 51 Esikoiselle Joosef antoi nimen Manasse, "sillä", sanoi hän, "Jumala on saattanut minut unhottamaan kaikki vaivani ja koko isäni kodin".
\par 52 Toiselle hän antoi nimen Efraim, "sillä", sanoi hän, "Jumala on tehnyt minut hedelmälliseksi kärsimysteni maassa".
\par 53 Ne seitsemän viljavuotta, jotka tulivat Egyptin maahan, kuluivat loppuun.
\par 54 Senjälkeen alkoi seitsemän nälkävuotta, niinkuin Joosef oli sanonut, ja tuli nälänhätä kaikkiin maihin, mutta Egyptin maassa oli leipäviljaa kaikkialla.
\par 55 Mutta koko Egyptin maa näki nälkää, ja kansa huusi faraolta leipää. Silloin farao sanoi kaikille egyptiläisille: "Menkää Joosefin luo ja tehkää, mitä hän käskee teidän tehdä".
\par 56 Kun nälkä ahdisti koko maata, avasi Joosef kaikki varastot ja myi viljaa egyptiläisille. Mutta nälänhätä tuli yhä kovemmaksi Egyptin maassa.
\par 57 Ja kaikista maista tultiin Egyptiin Joosefin luo ostamaan viljaa, sillä kaikissa maissa oli kova nälänhätä.

\chapter{42}

\par 1 Mutta kun Jaakob sai tietää, että Egyptissä oli viljaa, sanoi hän pojillensa: "Mitä epäröitte?"
\par 2 Ja hän sanoi: "Katso, minä olen kuullut, että Egyptissä on viljaa. Menkää sinne ja ostakaa meille sieltä viljaa, että pysyisimme hengissä emmekä kuolisi."
\par 3 Niin kymmenen Joosefin veljeä lähti ostamaan viljaa Egyptistä.
\par 4 Mutta Benjaminia, Joosefin veljeä, Jaakob ei lähettänyt hänen veljiensä mukana, sillä hän pelkäsi, että häntä kohtaisi jokin onnettomuus.
\par 5 Niin Israelin pojat tulivat muiden tulijain mukana ostamaan viljaa, sillä Kanaanin maassa oli nälänhätä.
\par 6 Mutta Joosef oli vallanpitäjänä maassa; hän myi viljaa kaikelle maan kansalle. Niin Joosefin veljet tulivat ja kumartuivat hänen edessään kasvoilleen maahan.
\par 7 Ja Joosef näki veljensä, ja hän tunsi heidät, mutta tekeytyi heille vieraaksi, puhutteli heitä ankarasti ja kysyi heiltä: "Mistä te tulette?" He vastasivat: "Kanaanin maasta tulemme, ostamaan elintarpeita".
\par 8 Ja Joosef tunsi veljensä, mutta he eivät tunteneet häntä.
\par 9 Silloin Joosef muisti unet, jotka hän oli nähnyt heistä, ja sanoi heille: "Te olette vakoojia; olette tulleet katsomaan, mistä maa olisi avoin".
\par 10 He vastasivat hänelle: "Ei, herra; palvelijasi ovat tulleet ostamaan elintarpeita.
\par 11 Me olemme kaikki saman miehen poikia, olemme rehellisiä miehiä; palvelijasi eivät ole vakoojia."
\par 12 Mutta hän sanoi heille: "Ei ole niin, vaan te olette tulleet katsomaan, mistä maa olisi avoin".
\par 13 He vastasivat: "Meitä, sinun palvelijoitasi, on kaksitoista veljestä, saman miehen poikia Kanaanin maasta; nuorin on nyt kotona isämme luona, ja yhtä ei enää ole".
\par 14 Joosef sanoi heille: "Niin on, kuin olen teille puhunut: te olette vakoojia.
\par 15 Näin te tulette koeteltaviksi: niin totta kuin farao elää, te ette pääse täältä lähtemään, ellei nuorin veljenne tule tänne.
\par 16 Lähettäkää yksi joukostanne noutamaan veljenne tänne, mutta teidän muiden on jääminen tänne vangeiksi, että koeteltaisiin, oletteko puhuneet totta; muuten, niin totta kuin farao elää, te olette vakoojia."
\par 17 Ja hän panetti heidät vankeuteen kolmeksi päiväksi.
\par 18 Mutta kolmantena päivänä Joosef sanoi heille: "Jos tahdotte elää, niin tehkää näin, sillä minä olen Jumalaa pelkääväinen:
\par 19 jos olette rehellisiä miehiä, niin jääköön yksi teistä, veljeksistä, vangiksi vankilaan, jossa teitä säilytettiin, mutta te muut menkää viemään kotiin viljaa perheittenne nälänhätään.
\par 20 Ja tuokaa nuorin veljenne minun luokseni. Jos teidän puheenne siten todeksi vahvistuu, niin vältätte kuoleman." Ja heidän täytyi tehdä niin.
\par 21 Mutta he sanoivat toinen toisellensa: "Totisesti, me olemme syylliset sen tähden, mitä teimme veljellemme; sillä me näimme hänen sielunsa tuskan, kun hän anoi meiltä armoa, emmekä kuulleet häntä. Sentähden on meille tullut tämä tuska."
\par 22 Ruuben vastasi heille sanoen: "Enkö minä sanonut teille: 'Älkää tehkö pahoin nuorukaista vastaan!' Mutta te ette kuulleet minua; katsokaa, nyt kostetaan hänen verensä."
\par 23 Mutta he eivät tienneet, että Joosef ymmärsi heitä, sillä hän puhui heille tulkin kautta.
\par 24 Ja hän kääntyi pois heistä ja itki. Sitten hän kääntyi taas heihin päin ja puhui heidän kanssaan. Ja hän otti heidän joukostaan Simeonin ja vangitutti hänet heidän nähtensä.
\par 25 Ja Joosef käski täyttää heidän säkkinsä viljalla ja panna jokaisen rahat takaisin hänen säkkiinsä sekä antaa heille evästä matkalle. Ja heille tehtiin niin.
\par 26 Ja he sälyttivät viljansa aasien selkään ja lähtivät sieltä.
\par 27 Kun sitten eräs heistä yöpaikassa avasi säkkinsä syöttääkseen aasiansa, huomasi hän rahansa säkin suussa.
\par 28 Hän sanoi veljilleen: "Minulle on annettu rahani takaisin; katso, se on minun säkissäni". Silloin heidän sydämensä vavahti, ja he katsoivat säikähtyneinä toisiinsa sanoen: "Mitä Jumala on meille tehnyt?"
\par 29 Tultuansa isänsä Jaakobin luo Kanaanin maahan he ilmoittivat hänelle kaikki, mitä heille oli tapahtunut, ja sanoivat:
\par 30 "Mies, joka on sen maan valtiaana, puhutteli meitä ankarasti ja kohteli meitä, niinkuin olisimme olleet maata vakoilemassa.
\par 31 Mutta me sanoimme hänelle: 'Olemme rehellisiä miehiä emmekä mitään vakoojia;
\par 32 meitä on kaksitoista veljestä, saman isän poikia; yhtä ei enää ole, ja nuorin on nyt kotona isämme luona Kanaanin maassa'.
\par 33 Mutta mies, sen maan valtias, sanoi meille: 'Siitä minä saan tietää, oletteko rehellisiä miehiä: jättäkää yksi veljistänne minun luokseni; ottakaa sitten viljaa perheittenne nälänhätään.
\par 34 Ja menkää ja tuokaa nuorin veljenne luokseni, saadakseni tietää, ettette ole vakoojia, vaan rehellisiä miehiä. Sitten minä annan teille veljenne takaisin, ja te saatte vapaasti liikkua maassa'."
\par 35 Kun he sitten tyhjensivät säkkinsä, niin katso, kunkin rahakukkaro oli hänen säkissään; ja nähdessään rahakukkaronsa he sekä heidän isänsä peljästyivät.
\par 36 Ja heidän isänsä Jaakob sanoi heille: "Te teette minut lapsettomaksi; Joosefia ei enää ole, Simeonia ei enää ole, ja Benjamininkin te tahdotte viedä minulta; kaikki tämä kohtaa minua".
\par 37 Ruuben vastasi isälleen sanoen: "Saat surmata minun molemmat poikani, jos en tuo häntä sinulle takaisin; anna hänet minun huostaani, niin minä tuon hänet sinulle takaisin".
\par 38 Mutta hän sanoi: "Ei minun poikani saa lähteä teidän kanssanne, sillä hänen veljensä on kuollut, ja hän on yksin jäljellä; jos onnettomuus kohtaa häntä matkalla, jolle aiotte lähteä, niin te saatatte minun harmaat hapseni vaipumaan murheella tuonelaan".

\chapter{43}

\par 1 Mutta nälänhätä oli maassa kova.
\par 2 Ja kun he olivat syöneet loppuun sen viljan, jonka olivat tuoneet Egyptistä, sanoi heidän isänsä heille: "Menkää jälleen ostamaan meille vähän elintarpeita".
\par 3 Juuda vastasi hänelle sanoen: "Se mies teroitti meille teroittamalla: 'Ette saa tulla minun kasvojeni eteen, ellei veljenne ole teidän kanssanne'.
\par 4 Jos annat veljemme seurata meidän mukanamme, niin me lähdemme ostamaan sinulle elintarpeita.
\par 5 Mutta jos et anna, niin emme lähde; sillä se mies sanoi meille: 'Ette saa tulla minun kasvojeni eteen, ellei veljenne ole teidän kanssanne'."
\par 6 Israel sanoi: "Minkätähden teitte niin pahasti minua kohtaan, että ilmaisitte tuolle miehelle teillä olevan vielä veljen?"
\par 7 He vastasivat: "Mies kyseli tuiki tarkasti meitä ja meidän sukuamme, sanoen: 'Elääkö isänne vielä? Onko teillä vielä veljeä?' Silloin me ilmoitimme hänelle, niinkuin asia on. Saatoimmeko tietää, että hän sanoisi: 'Tuokaa tänne veljenne'?"
\par 8 Ja Juuda sanoi isällensä Israelille: "Anna nuorukaisen seurata minun mukanani, niin me nousemme ja lähdemme matkalle, että jäisimme eloon, sekä me että sinä ja vaimomme ja lapsemme, emmekä kuolisi.
\par 9 Minä vastaan hänestä; minun kädestäni saat vaatia hänet. Jos en tuo häntä takaisin sinun luoksesi ja aseta häntä eteesi, niin minä olen syyllinen sinun edessäsi kaiken elinaikani.
\par 10 Totisesti, jos emme olisi näin vitkastelleet, niin olisimme jo kaksikin kertaa voineet olla sieltä kotona."
\par 11 Silloin heidän isänsä Israel sanoi heille: "Jos niin on, tehkää ainakin tämä: ottakaa säkkeihinne maan parhaimpia tuotteita ja viekää ne sille miehelle lahjaksi: vähän balsamia ja vähän hunajaa, kumihartsia ja hajupihkaa, pähkinöitä ja manteleita.
\par 12 Ja ottakaa mukaanne kaksinkertainen raha, niin että viette takaisin sen rahan, joka palautettiin säkkienne suussa. Ehkä se oli erehdys.
\par 13 Ottakaa sitten mukaanne myöskin veljenne ja nouskaa ja menkää jälleen sen miehen luo.
\par 14 Jumala, Kaikkivaltias, suokoon, että se mies tekisi teille laupeuden ja antaisi toisen veljenne sekä Benjaminin palata kotiin teidän kanssanne. Mutta jos tulen lapsettomaksi, niin tulen lapsettomaksi."
\par 15 Niin miehet ottivat mukaansa lahjan ja kaksinkertaisen rahan sekä myöskin Benjaminin ja nousivat ja menivät Egyptiin; ja he astuivat Joosefin eteen.
\par 16 Kun Joosef näki Benjaminin heidän seurassaan, sanoi hän huoneenhaltijalleen: "Vie nämä miehet sisään; teurasta teuras ja valmista se, sillä miehet saavat syödä päivällistä minun kanssani".
\par 17 Ja mies teki, niinkuin Joosef oli käskenyt, ja vei miehet Joosefin taloon.
\par 18 Mutta miehet pelkäsivät, kun heitä vietiin Joosefin taloon, arvellen: "Sen rahan tähden, joka viime kerralla tuli takaisin meidän säkeissämme, ne nyt vievät meitä tänne hyökätäkseen ja karatakseen meidän kimppuumme, ottaakseen meidät orjiksi ja anastaakseen aasimme".
\par 19 Niin he menivät Joosefin huoneenhaltijan luo ja puhuttelivat häntä talon ovella
\par 20 ja sanoivat: "Oi kuule, herra, me olemme kerran ennen käyneet täällä ostamassa elintarpeita,
\par 21 ja kun me tulimme yöpaikkaan ja avasimme säkkimme, niin katso, jokaisen raha oli hänen säkkinsä suussa täysipainoisena; olemme nyt tuoneet ne mukanamme takaisin.
\par 22 Ja olemme tuoneet mukanamme toisenkin rahan ostaaksemme viljaa elatukseksemme. Emme tiedä, kuka on pannut meidän rahamme säkkeihimme."
\par 23 Hän vastasi: "Olkaa rauhassa, älkää peljätkö. Teidän Jumalanne ja teidän isänne Jumala on antanut teidän löytää aarteen säkeistänne. Teidän rahanne minä olen saanut." Ja hän toi heidän luokseen Simeonin.
\par 24 Ja hän vei miehet Joosefin taloon ja antoi heille vettä jalkain pesemiseksi ja heidän aaseilleen rehua.
\par 25 He laittoivat lahjansa järjestykseen siksi, kunnes Joosef tulisi päivälliselle; sillä he olivat kuulleet, että saisivat aterioida siellä.
\par 26 Kun Joosef oli tullut kotiin, veivät he hänelle huoneeseen lahjat, jotka heillä oli mukanaan, ja kumartuivat maahan hänen edessänsä.
\par 27 Hän tervehti heitä ja kysyi: "Kuinka voi teidän vanha isänne, josta puhuitte? Vieläkö hän elää?"
\par 28 He vastasivat: "Palvelijasi, meidän isämme, voi hyvin; hän elää vielä". Ja he kumartuivat ja heittäytyivät maahan.
\par 29 Ja hän nosti silmänsä ja näki veljensä Benjaminin, äitinsä pojan, ja kysyi: "Onko tämä teidän nuorin veljenne, josta puhuitte?" Sitten hän sanoi: "Jumala olkoon sinulle, poikani, armollinen".
\par 30 Mutta silloin Joosef keskeytti äkkiä puheensa, sillä nähdessään veljensä hän tuli sydämessään liikutetuksi ja etsi tilaisuutta itkeäkseen; niin hän meni sisähuoneeseen ja itki siellä.
\par 31 Senjälkeen hän pestyään kasvonsa tuli ulos, hillitsi itsensä ja käski: "Tarjotkaa ruokaa".
\par 32 Ja tarjottiin erikseen hänelle ja erikseen heille ja erikseen egyptiläisille, jotka aterioivat hänen kanssaan; egyptiläiset näet eivät saata syödä yhdessä hebrealaisten kanssa, sillä se on egyptiläisille kauhistus.
\par 33 He istuivat vastapäätä häntä iän mukaan, esikoinen ensimmäisenä ja nuorin viimeisenä; ja ihmetellen miehet katselivat toisiaan.
\par 34 Ja hän antoi kantaa omasta pöydästään ruokia heille, ja Benjaminin annos oli viisi kertaa suurempi kuin kaikkien muiden. Ja he joivat hänen kanssaan ja juopuivat.

\chapter{44}

\par 1 Sen jälkeen Joosef käski huoneenhaltijaansa sanoen: "Täytä miesten säkit viljalla, niin paljon kuin he voivat kuljettaa, ja pane itsekunkin raha hänen säkkinsä suuhun.
\par 2 Ja nuorimman säkin suuhun pane minun maljani, tuo hopeamalja, ynnä hänen viljarahansa." Ja hän teki, niinkuin Joosef käski.
\par 3 Aamulla päivän valjetessa miehet saivat aaseinensa lähteä matkalle.
\par 4 Mutta kun he olivat ehtineet vähän matkaa kaupungin ulkopuolelle, sanoi Joosef huoneenhaltijalleen: "Nouse ja aja miehiä takaa, ja kun saavutat heidät, sano heille: 'Minkätähden olette palkinneet hyvän pahalla?
\par 5 Onhan se juuri se, josta isäntäni juo ja josta hän salaisia tiedustelee. Te olette pahoin tehneet menetellessänne näin.'"
\par 6 Kun hän sitten saavutti heidät, puhui hän heille nämä sanat.
\par 7 He vastasivat hänelle: "Minkätähden herramme puhuu näin? Pois se, että palvelijasi tekisivät niin!
\par 8 Katso, rahan, jonka löysimme säkkiemme suusta, me olemme tuoneet takaisin sinulle Kanaanin maasta; kuinka siis olisimme varastaneet hopeata tai kultaa herrasi talosta?
\par 9 Se palvelijoistasi, jolta se löydetään, kuolkoon; ja me muut tulemme herramme orjiksi."
\par 10 Hän vastasi: "Olkoon niin, kuin olette puhuneet; se, jolta se löydetään, olkoon minun orjani. Mutta te muut pääsette vapaiksi."
\par 11 Ja he laskivat nopeasti säkkinsä maahan, ja jokainen avasi säkkinsä.
\par 12 Ja hän etsi, alkaen vanhimmasta ja lopettaen nuorimpaan, ja malja löytyi Benjaminin säkistä.
\par 13 Silloin he repäisivät vaatteensa, kuormasivat kukin tavaransa aasinsa selkään ja palasivat kaupunkiin.
\par 14 Ja Juuda meni veljinensä Joosefin taloon, jossa tämä vielä oli, ja he lankesivat maahan hänen eteensä.
\par 15 Silloin Joosef sanoi heille: "Mitä olettekaan tehneet! Ettekö tienneet, että minun kaltaiseni mies saa salatut ilmi?"
\par 16 Juuda vastasi: "Mitä sanoisimmekaan herralleni, mitä puhuisimme ja millä puolustautuisimme! Jumala on paljastanut palvelijaisi syyllisyyden. Katso, me olemme herrani orjat, niin hyvin me muut kuin se, jolta malja löytyi."
\par 17 Hän sanoi: "Pois se, että minä tekisin niin! Se, jolta malja löytyi, olkoon minun orjani, mutta te muut menkää rauhassa kotiin isänne luo."
\par 18 Silloin Juuda astui hänen eteensä ja sanoi: "Oi herrani, salli palvelijasi puhua sananen herrani kuullen, älköönkä vihasi syttykö palvelijaasi kohtaan, sillä sinä olet niinkuin itse farao!
\par 19 Herrani kysyi palvelijoiltaan sanoen: 'Onko teillä isää tai veljeä?'
\par 20 Me vastasimme herralleni: 'Meillä on kotona vanha isä ja veli, joka on syntynyt hänen vanhoilla päivillänsä ja on vielä nuori; mutta tämän veli on kuollut, ja niin hän on jäänyt yksin äidistänsä, ja hänen isänsä rakastaa häntä'.
\par 21 Niin sinä sanoit palvelijoillesi: 'Tuokaa hänet tänne minun luokseni, että silmäni saisivat katsella häntä'.
\par 22 Me vastasimme herralleni: 'Nuorukainen ei saata jättää isäänsä, sillä jos hän jättäisi isänsä, niin tämä kuolisi'.
\par 23 Mutta sinä sanoit palvelijoillesi: 'Jos nuorin veljenne ei tule tänne teidän kanssanne, niin älkää enää näyttäytykö minun kasvojeni edessä'.
\par 24 Ja me menimme kotiin palvelijasi, minun isäni, luo ja kerroimme hänelle herrani sanat.
\par 25 Niin isämme sanoi: 'Menkää jälleen ostamaan meille vähän elintarpeita'.
\par 26 Me sanoimme: 'Emme voi lähteä sinne; ainoastaan jos nuorin veljemme seuraa mukanamme, me lähdemme, sillä me emme voi näyttäytyä sen miehen kasvojen edessä, jollei nuorin veljemme ole mukanamme'.
\par 27 Niin palvelijasi, minun isäni, sanoi meille: 'Tiedättehän itse, että vaimoni synnytti minulle kaksi poikaa.
\par 28 Toinen lähti pois luotani, ja minä sanoin: Varmaan hänet on raadeltu kuoliaaksi, enkä minä ole häntä siitä päivin nähnyt.
\par 29 Jos te nyt viette minulta tämänkin ja jos onnettomuus kohtaa häntä, niin te saatatte minun harmaat hapseni vaipumaan tuskalla tuonelaan.'
\par 30 Jos minä siis tulisin kotiin palvelijasi, isäni, luo eikä meillä olisi mukanamme nuorukaista, johon hän on kaikesta sielustaan kiintynyt,
\par 31 niin hän nähdessään, ettei nuorukainen ole kanssamme, kuolisi, ja me, sinun palvelijasi, saattaisimme palvelijasi, isämme, harmaat hapset vaipumaan murheella tuonelaan.
\par 32 Sillä palvelijasi on luvannut isälleen vastata nuorukaisesta ja sanonut: 'Jos en tuo häntä takaisin luoksesi, niin minä olen syyllinen isäni edessä kaiken elinaikani'.
\par 33 Ja jääköön siis palvelijasi herralleni orjaksi nuorukaisen sijaan, ja nuorukainen menköön kotiin veljiensä kanssa.
\par 34 Sillä kuinka minä voisin mennä kotiin isäni luo, jollei nuorukainen olisi kanssani? En voisi nähdä sitä surkeutta, joka tulisi isäni osaksi."

\chapter{45}

\par 1 Silloin Joosef ei voinut kauemmin hillitä itseään kaikkien niiden nähden, jotka seisoivat hänen ympärillään. Hän huusi: "Antakaa kaikkien mennä pois minun luotani!" Niin ei ollut ketään saapuvilla, kun Joosef ilmaisi itsensä veljilleen.
\par 2 Ja hän purskahti ääneensä itkemään, niin että egyptiläiset ja faraon hoviväki sen kuulivat.
\par 3 Ja Joosef sanoi veljilleen: "Minä olen Joosef. Vieläkö minun isäni elää?" Mutta hänen veljensä eivät voineet vastata hänelle, niin hämmästyksissään he olivat hänen edessään.
\par 4 Mutta Joosef sanoi veljilleen: "Tulkaa tänne luokseni". Ja he tulivat. Niin hän sanoi: "Minä olen Joosef, teidän veljenne, jonka myitte Egyptiin.
\par 5 Mutta älkää nyt olko murheissanne älkääkä pahoitelko sitä, että olette myyneet minut tänne, sillä Jumala on minut lähettänyt teidän edellänne pitääkseen teidät hengissä.
\par 6 Kaksi vuotta on nyt nälänhätä ollut maassa, ja vielä on jäljellä viisi vuotta, joina ei kynnetä eikä eloa korjata.
\par 7 Niin Jumala lähetti minut teidän edellänne säilyttääkseen teille jälkeläisiä maan päällä ja pitääkseen teidät hengissä, pelastukseksi monille.
\par 8 Ette siis te ole lähettäneet minua tänne, vaan Jumala; hän asetti minut faraon neuvonantajaksi ja koko hänen hovinsa herraksi ja koko Egyptin maan valtiaaksi.
\par 9 Menkää, rientäkää minun isäni tykö ja sanokaa hänelle: 'Näin sanoo poikasi Joosef: Jumala on asettanut minut koko Egyptin herraksi, tule luokseni, älä viivyttele!
\par 10 Sinä saat asettua Goosenin maakuntaan ja olla minun läheisyydessäni, sinä ja sinun lapsesi ja lastesi lapset, pikkukarjasi ja raavaskarjasi, kaikki, mitä sinulla on.
\par 11 Minä elätän sinua siellä - vielä on näet viisi nälkävuotta - niin ettet sinä eikä sinun perheesi eikä kukaan omaisistasi ole sortuva puutteeseen.'
\par 12 Te näette omin silmin, ja myöskin veljeni Benjamin näkee, että minä itse teille puhun.
\par 13 Kertokaa siis isällenne kaikesta siitä kunniasta, joka on tullut minun osakseni Egyptissä, ja kaikesta, mitä olette nähneet, ja rientäkää ja tuokaa isäni tänne."
\par 14 Ja hän lankesi veljensä Benjaminin kaulaan ja itki, ja myöskin Benjamin itki hänen kaulassaan.
\par 15 Ja hän suuteli kaikkia veljiään ja itki heidän rinnoillaan. Senjälkeen hänen veljensä puhelivat hänen kanssaan.
\par 16 Kun sanoma siitä, että Joosefin veljet olivat saapuneet, kuului faraon hoviin, oli se faraon ja kaikkien hänen palvelijainsa mieleen.
\par 17 Ja farao sanoi Joosefille: "Sano veljillesi: 'Tehkää näin: sälyttäkää kuormat juhtainne selkään ja lähtekää kotiin Kanaanin maahan,
\par 18 ottakaa isänne ja perheenne ja tulkaa minun luokseni, niin minä annan teille parasta, mitä Egyptissä on, ja te saatte syödä maan lihavuudesta'.
\par 19 Ja näin sinun on käskettävä heitä: 'Tehkää näin: ottakaa itsellenne vaunuja Egyptin maasta lapsianne ja vaimojanne varten ja tuokaa isänne ja tulkaa.
\par 20 Älkää surko taloustavaroitanne, sillä mitä parasta on koko Egyptin maassa, se on oleva teidän omanne.'"
\par 21 Israelin pojat tekivät niin, ja Joosef antoi heille vaunuja faraon käskyn mukaan sekä evästä matkaa varten.
\par 22 Hän antoi kullekin heistä juhlapuvun, mutta Benjaminille hän antoi kolmesataa hopeasekeliä sekä viisi juhlapukua.
\par 23 Samoin hän lähetti isälleen lahjaksi kymmenen aasia, jotka olivat kuormitetut Egyptin parhaimmilla tavaroilla, ja kymmenen aasintammaa, jotka kantoivat viljaa ja leipää sekä eväitä hänen isälleen matkaa varten.
\par 24 Sitten hän päästi veljensä menemään ja sanoi heille: "Älkää riidelkö matkalla".
\par 25 Niin he lähtivät Egyptistä ja tulivat isänsä Jaakobin luo Kanaanin maahan.
\par 26 Ja he kertoivat hänelle ja sanoivat: "Joosef on vielä elossa ja on koko Egyptin maan valtias". Mutta hänen sydämensä pysyi kylmänä, sillä hän ei uskonut heitä.
\par 27 Niin he kertoivat hänelle kaiken, mitä Joosef oli heille puhunut. Ja kun hän näki vaunut, jotka Joosef oli lähettänyt häntä tuomaan, niin elpyi heidän isänsä Jaakobin henki.
\par 28 Ja Israel sanoi: "Nyt on minulla kyllin; poikani Joosef elää vielä, minä menen häntä katsomaan, ennenkuin kuolen".

\chapter{46}

\par 1 Niin Israel lähti matkalle mukanaan kaikki, mitä hänellä oli. Ja kun hän saapui Beersebaan, uhrasi hän teurasuhreja isänsä Iisakin Jumalalle.
\par 2 Ja Jumala puhui Israelille näyssä yöllä; hän sanoi: "Jaakob, Jaakob!" Tämä vastasi: "Tässä olen".
\par 3 Niin hän sanoi: "Minä olen Jumala, sinun isäsi Jumala; älä pelkää mennä Egyptiin, sillä minä teen sinut siellä suureksi kansaksi.
\par 4 Minä menen sinun kanssasi Egyptiin, ja minä myös johdatan sinut sieltä takaisin. Ja Joosefin käsi on sulkeva sinun silmäsi."
\par 5 Ja Jaakob lähti Beersebasta, ja Israelin pojat nostivat isänsä Jaakobin, lapsensa ja vaimonsa vaunuihin, jotka farao oli lähettänyt häntä noutamaan.
\par 6 Ja he ottivat karjansa ja tavaransa, jotka he olivat hankkineet Kanaanin maassa, ja tulivat niin Egyptiin, Jaakob ynnä kaikki hänen jälkeläisensä.
\par 7 Poikansa ja poikiensa pojat, tyttärensä ja poikiensa tyttäret, kaikki jälkeläisensä, hän vei mukanaan Egyptiin.
\par 8 Nämä ovat Israelin lasten nimet, niiden, jotka tulivat Egyptiin: Jaakob ja hänen poikansa. Jaakobin esikoinen oli Ruuben.
\par 9 Ruubenin pojat olivat Hanok, Pallu, Hesron ja Karmi.
\par 10 Simeonin pojat olivat Jemuel, Jaamin, Oohad, Jaakin, Soohar ja Saul, kanaanilaisen vaimon poika.
\par 11 Leevin pojat olivat Geerson, Kehat ja Merari.
\par 12 Juudan pojat olivat Eer, Oonan, Seela, Peres ja Serah; mutta Eer ja Oonan kuolivat Kanaanin maassa. Pereksen pojat olivat Hesron ja Haamul.
\par 13 Isaskarin pojat olivat Toola, Puvva, Joob ja Simron.
\par 14 Sebulonin pojat olivat Sered, Eelon ja Jahleel.
\par 15 Nämä olivat Leean pojat; ne hän synnytti Jaakobille Mesopotamiassa sekä tyttären Diinan. Näitä Jaakobin poikia ja tyttäriä oli kaikkiaan kolmekymmentä kolme henkeä.
\par 16 Gaadin pojat olivat Sifjon ja Haggi, Suuni ja Esbon, Eeri ja Arodi ja Areli.
\par 17 Asserin pojat olivat Jimna, Jisva, Jisvi ja Beria; heidän sisarensa oli Serah. Berian pojat olivat Heber ja Malkiel.
\par 18 Nämä olivat Silpan lapset, hänen, jonka Laaban antoi tyttärellensä Leealle, ja hän synnytti ne Jaakobille, kuusitoista henkeä.
\par 19 Raakelin, Jaakobin vaimon, pojat olivat Joosef ja Benjamin.
\par 20 Ja pojat, jotka syntyivät Joosefille Egyptin maassa, olivat Manasse ja Efraim; nämä synnytti hänelle Aasenat, Oonin papin Poti-Feran tytär.
\par 21 Benjaminin pojat olivat Bela, Beker ja Asbel, Geera ja Naaman, Eehi ja Roos, Muppim ja Huppim ja Ard.
\par 22 Nämä olivat Raakelin pojat, jotka syntyivät Jaakobille, kaikkiaan neljätoista henkeä.
\par 23 Daanin poika oli Husim.
\par 24 Naftalin pojat olivat Jahseel, Guuni, Jeeser ja Sillem.
\par 25 Nämä olivat Bilhan pojat, hänen, jonka Laaban antoi tyttärellensä Raakelille, ja hän synnytti ne Jaakobille, kaikkiaan seitsemän henkeä.
\par 26 Kaikkiaan oli niitä, jotka Jaakobin kanssa siirtyivät Egyptiin ja olivat lähteneet hänen kupeistansa, paitsi Jaakobin miniöitä, yhteensä kuusikymmentä kuusi henkeä.
\par 27 Ja Joosefin poikia, jotka syntyivät hänelle Egyptissä, oli kaksi. Jaakobin perheen jäseniä, jotka siirtyivät Egyptiin, oli kaikkiaan seitsemänkymmentä henkeä.
\par 28 Ja hän lähetti Juudan edellänsä Joosefin luo ilmoittamaan hänelle tulostaan Gooseniin. Niin he tulivat Goosenin maakuntaan.
\par 29 Ja Joosef valjastutti vaununsa ja meni isäänsä Israelia vastaan Gooseniin. Ja kun hän saapui hänen eteensä, lankesi hän hänen kaulaansa ja itki kauan hänen kaulassaan.
\par 30 Ja Israel sanoi Joosefille: "Nyt minä kuolen mielelläni, kun olen nähnyt sinun kasvosi ja tiedän, että sinä vielä elät".
\par 31 Sen jälkeen Joosef sanoi veljilleen ja isänsä perheelle: "Minä menen ilmoittamaan faraolle ja sanon hänelle: 'Minun veljeni ja minun isäni perhe, jotka ovat olleet Kanaanin maassa, ovat saapuneet luokseni.
\par 32 Ja nämä miehet ovat paimenia, sillä he hoitavat karjaa; ja he ovat tuoneet mukanaan lampaansa, karjansa ja kaiken muun omaisuutensa.'
\par 33 Kun siis farao kutsuu teidät eteensä ja kysyy: 'Mikä teidän ammattinne on?'
\par 34 niin vastatkaa: 'Me, sinun palvelijasi, olemme hoitaneet karjaa nuoruudestamme tähän asti, me niinkuin meidän isämmekin' - että saisitte asettua Goosenin maakuntaan. Sillä kaikki paimenet ovat egyptiläisille kauhistus."

\chapter{47}

\par 1 Ja Joosef meni ja ilmoitti faraolle, sanoen: "Minun isäni ja veljeni ovat pikkukarjoineen ja raavaskarjoineen, kaikkine omaisuuksineen, tulleet Kanaanin maasta, ja katso, he ovat Goosenin maakunnassa".
\par 2 Ja hän oli ottanut mukaansa veljiensä joukosta viisi miestä; ne hän toi faraon eteen.
\par 3 Niin farao kysyi hänen veljiltänsä: "Mikä on teidän ammattinne?" He vastasivat faraolle: "Me, sinun palvelijasi, olemme paimenia, me niinkuin isämmekin".
\par 4 Ja he sanoivat vielä faraolle: "Me olemme tulleet asuaksemme jonkun aikaa tässä maassa; sillä palvelijoillasi ei ollut laidunta karjalleen, koska kova nälänhätä on Kanaanin maassa. Suo siis palvelijaisi asettua Goosenin maakuntaan."
\par 5 Niin farao sanoi Joosefille: "Isäsi ja veljesi ovat tulleet sinun luoksesi.
\par 6 Egyptin maa on avoinna sinun edessäsi; sijoita isäsi ja veljesi maan parhaaseen osaan. Asukoot Goosenin maakunnassa; ja jos tiedät heidän joukossaan olevan kelvollisia miehiä, niin aseta heidät minun karjani päällysmiehiksi."
\par 7 Senjälkeen Joosef toi isänsä Jaakobin sisään ja esitti hänet faraolle. Ja Jaakob toivotti faraolle siunausta.
\par 8 Niin farao kysyi Jaakobilta: "Kuinka monta ikävuotta sinulla on?"
\par 9 Jaakob vastasi faraolle: "Minun vaellusaikani on kestänyt sata kolmekymmentä vuotta. Vähät ja pahat ovat olleet minun elinvuosieni päivät eivätkä ole saavuttaneet sitä elinvuosien määrää, mikä isilläni oli vaelluksensa aikana."
\par 10 Ja Jaakob toivotti faraolle siunausta ja lähti hänen luotaan.
\par 11 Ja Joosef sijoitti isänsä ja veljensä Egyptin maahan ja antoi heille maaomaisuutta maan parhaasta osasta, Ramseksen maakunnasta, niinkuin farao oli hänen käskenyt tehdä.
\par 12 Ja Joosef elätti isäänsä ja veljiänsä ja koko isänsä perhettä antamalla jokaiselle elatusta vaimojen ja lasten luvun mukaan.
\par 13 Mutta ei missään koko maassa ollut leipää; sillä nälänhätä oli hyvin kova, niin että Egyptin maa ja Kanaanin maa olivat nääntymässä nälkään.
\par 14 Ja viljalla, jota ostettiin, Joosef kokosi kaiken rahan, mitä oli Egyptin maassa ja Kanaanin maassa; ja Joosef vei rahat faraon hoviin.
\par 15 Kun raha oli loppunut Egyptin maasta ja Kanaanin maasta, tulivat kaikki egyptiläiset Joosefin luo, sanoen: "Anna meille leipää. Miksi me kuolisimme sinun silmiesi edessä? Sillä raha on loppunut."
\par 16 Joosef vastasi: "Tuokaa tänne karjanne. Minä annan teille leipää karjastanne, jos rahanne on loppunut."
\par 17 Ja he toivat Joosefille karjansa, ja Joosef antoi heille leipää hevosista, lampaista, raavaskarjasta ja aaseista. Niin hän sen vuoden elätti heitä leivällä kaiken heidän karjansa hinnasta.
\par 18 Niin kului se vuosi. Seuraavana vuonna he tulivat taas hänen luoksensa ja sanoivat hänelle: "Emme tahdo salata herraltamme, että raha on lopussa, ja myöskin eläimemme ovat joutuneet herramme omiksi; meillä ei ole muuta jäljellä annettavana herrallemme kuin ruumiimme ja peltomme.
\par 19 Miksi me menehtyisimme sinun silmiesi edessä, sekä me itse että meidän peltomme? Osta meidät ja peltomme leivällä, niin me tulemme peltoinemme faraon orjiksi. Anna meille siementä, että eläisimme emmekä kuolisi eivätkä peltomme joutuisi autioiksi."
\par 20 Niin Joosef osti faraolle kaikki Egyptin pellot; sillä egyptiläiset myivät jokainen vainionsa, koska nälkä ahdisti heitä. Niin joutui maa faraon omaksi.
\par 21 Ja hän siirsi kansan kaupunkeihin, Egyptin toisesta äärestä toiseen saakka.
\par 22 Ainoastaan pappien peltoja hän ei ostanut; sillä papeilla oli määrätyt tulot faraolta ja he elivät niistä määrätyistä tuloistaan, jotka he faraolta saivat. Sentähden heidän ei tarvinnut myydä peltojansa.
\par 23 Ja Joosef sanoi kansalle: "Katso, minä olen nyt ostanut teidät ja teidän peltonne faraolle; katso, tässä on teille siementä, kylväkää peltonne.
\par 24 Mutta sadosta teidän on annettava viides osa faraolle; mutta neljä viidettä osaa jääköön teille pellon siemeneksi sekä ravinnoksi itsellenne ja niille, jotka talossanne ovat, sekä elatukseksi vaimoillenne ja lapsillenne."
\par 25 He vastasivat: "Sinä olet pitänyt meidät hengissä; suo meidän vain saada armo herramme silmien edessä, niin olemme faraon orjia".
\par 26 Niin Joosef teki sen säädökseksi, joka vielä tänäkin päivänä on voimassa Egyptin pelloista, että faraolle on annettava viides osa. Ainoastaan pappien pellot eivät joutuneet faraon omiksi.
\par 27 Niin Israel jäi asumaan Egyptiin, Goosenin maakuntaan; he asettuivat sinne ja olivat hedelmällisiä ja lisääntyivät suuresti.
\par 28 Ja Jaakob eli Egyptin maassa seitsemäntoista vuotta, ja koko hänen elinaikansa oli sata neljäkymmentä seitsemän vuotta.
\par 29 Kun lähestyi aika, jolloin Israelin oli kuoltava, kutsui hän poikansa Joosefin ja sanoi hänelle: "Jos olen saanut armon sinun silmiesi edessä, niin pane nyt kätesi minun kupeeni alle ja osoita minulle laupeus ja uskollisuus: älä hautaa minua Egyptiin,
\par 30 sillä minä tahdon levätä isieni luona; vie siis minut Egyptistä ja hautaa minut heidän hautaansa". Hän vastasi: "Minä teen, niinkuin sanot".
\par 31 Hän sanoi: "Vanno se minulle". Ja hän vannoi hänelle. Silloin Israel rukoili, kumartuneena vuoteensa päänalaista vasten.

\chapter{48}

\par 1 Tämän jälkeen tuotiin Joosefille sana: "Katso, isäsi on sairaana". Ja hän otti mukaansa molemmat poikansa, Manassen ja Efraimin.
\par 2 Ja Jaakobille ilmoitettiin ja sanottiin: "Katso, poikasi Joosef on tullut sinun luoksesi". Niin Israel kokosi voimansa ja nousi istumaan vuoteessaan.
\par 3 Ja Jaakob sanoi Joosefille: "Jumala, Kaikkivaltias, ilmestyi minulle Luusissa Kanaanin maassa ja siunasi minut
\par 4 ja sanoi minulle: 'Katso, minä teen sinut hedelmälliseksi ja annan sinun lisääntyä, annan tulla sinusta suuren kansojen joukon, ja minä annan sinun jälkeläisillesi tämän maan ikuiseksi perintömaaksi'.
\par 5 Kaksi poikaasi, jotka ovat sinulle syntyneet Egyptin maassa, ennenkuin minä tulin luoksesi Egyptiin, olkoot minun omani; Efraim ja Manasse olkoot minun omani niinkuin Ruuben ja Simeon.
\par 6 Mutta ne lapsesi, jotka ovat syntyneet sinulle heidän jälkeensä, olkoot sinun; nimitettäköön heitä veljiensä nimellä heidän perintöosassaan.
\par 7 Palatessani Mesopotamiasta kuoli minulta Raakel matkalla Kanaanissa, kun vielä oli jonkun verran matkaa Efrataan; ja minä hautasin hänet siellä Efratan" - se on Beetlehemin - "tien varteen".
\par 8 Kun nyt Israel huomasi Joosefin pojat, kysyi hän: "Keitä nämä ovat?"
\par 9 Joosef vastasi isälleen: "Ne ovat minun poikani, jotka Jumala on minulle täällä antanut". Hän sanoi: "Tuo heidät minun luokseni siunatakseni heidät".
\par 10 Mutta Israelin silmät olivat vanhuudesta hämärät, niin ettei hän voinut nähdä. Niin Joosef toi heidät hänen luokseen, ja hän suuteli heitä ja syleili heitä.
\par 11 Ja Israel sanoi Joosefille: "En olisi uskonut saavani nähdä sinun kasvojasi; ja katso, Jumala on suonut minun nähdä sinun jälkeläisiäsikin".
\par 12 Ja Joosef otti heidät pois hänen polviltansa ja kumartui maahan kasvoilleen.
\par 13 Sitten Joosef tarttui heihin molempiin, Efraimiin oikealla kädellänsä, vasemmalla Israelista, ja Manasseen vasemmalla kädellänsä, oikealla Israelista, ja toi heidät niin hänen eteensä.
\par 14 Mutta Israel ojensi oikean kätensä ja laski sen Efraimin pään päälle, vaikka tämä oli nuorempi, ja vasemman kätensä Manassen pään päälle; hän pani siis kätensä ristikkäin, sillä Manasse oli esikoinen.
\par 15 Ja hän siunasi Joosefin sanoen: "Jumala, jonka kasvojen edessä minun isäni Aabraham ja Iisak ovat vaeltaneet, Jumala, joka on minua kainnut syntymästäni hamaan tähän päivään asti,
\par 16 enkeli, joka on minut pelastanut kaikesta pahasta, siunatkoon näitä nuorukaisia; heitä mainittaessa mainittakoon minun nimeni ja minun isieni Aabrahamin ja Iisakin nimi, ja he lisääntykööt suuresti keskellä maata".
\par 17 Mutta kun Joosef huomasi, että hänen isänsä laski oikean kätensä Efraimin pään päälle, pani hän sen pahakseen ja tarttui isänsä käteen siirtääkseen sen Efraimin pään päältä Manassen pään päälle.
\par 18 Ja Joosef sanoi isälleen: "Ei niin, isäni, sillä tämä on esikoinen; pane oikea kätesi hänen päänsä päälle".
\par 19 Mutta hänen isänsä epäsi ja sanoi: "Kyllä tiedän, poikani, kyllä tiedän; hänestäkin on tuleva kansa, hänkin on tuleva suureksi, mutta hänen nuorempi veljensä on kuitenkin tuleva häntä suuremmaksi, ja hänen jälkeläisistään on tuleva kansan paljous".
\par 20 Ja niin hän siunasi heidät sinä päivänä, sanoen: "Sinun nimelläsi siunataan Israelissa, sanotaan: Jumala tehköön sinut Efraimin ja Manassen kaltaiseksi". Niin hän asetti Efraimin Manassen edelle.
\par 21 Ja Israel sanoi Joosefille: "Katso, minä kuolen, mutta Jumala on teidän kanssanne ja vie teidät takaisin isienne maahan.
\par 22 Ja lisäksi siihen, minkä veljesi saavat, minä annan sinulle vuorenharjanteen, jonka olen miekallani ja jousellani ottanut amorilaisilta."

\chapter{49}

\par 1 Sitten Jaakob kutsui poikansa ja sanoi: "Kokoontukaa, niin minä ilmoitan teille, mitä teille päivien lopulla tapahtuu.
\par 2 Tulkaa kokoon ja kuulkaa, Jaakobin pojat, kuulkaa isäänne Israelia.
\par 3 Ruuben, sinä olet minun esikoiseni, minun voimani ja minun miehuuteni ensimmäinen, ensi sijalla arvossa, ensi sijalla vallassa.
\par 4 Mutta sinä kuohahdat kuin vesi, et pysy ensi sijalla, sillä sinä nousit isäsi leposijalle; silloin sinä sen saastutit. Niin, hän nousi vuoteeseeni.
\par 5 Simeon ja Leevi, veljekset, heidän aseensa ovat väkivallan aseet.
\par 6 Heidän neuvoonsa ei suostu minun sieluni, heidän seuraansa ei yhdy minun sydämeni; sillä vihassaan he murhasivat miehiä, omavaltaisuudessaan he silpoivat härkiä.
\par 7 Kirottu olkoon heidän vihansa, sillä se on raju, heidän kiukkunsa, sillä se on julma. Minä jakelen heidät Jaakobin sekaan ja hajotan heidät Israelin sekaan.
\par 8 Juuda, sinua sinun veljesi ylistävät; sinun kätesi on vihollistesi niskassa, sinua kumartavat isäsi pojat.
\par 9 Juuda on nuori leijona; saaliilta olet, poikani, noussut. Hän on asettunut makaamaan, hän lepää kuin leijona, kuin naarasleijona - kuka uskaltaa häntä häiritä?
\par 10 Ei siirry valtikka pois Juudalta eikä hallitsijansauva hänen polviensa välistä, kunnes tulee hän, jonka se on ja jota kansat tottelevat.
\par 11 Hän sitoo aasinsa viinipuuhun, viiniköynnökseen aasinsa varsan; hän huuhtoo vaatteensa viinissä; viittansa rypäleen veressä.
\par 12 Hänen silmänsä ovat viinistä sameat, hänen hampaansa valkeat maidosta.
\par 13 Sebulon asuu meren rannalla, laivojen rannikolla, hänen sivunsa on Siidoniin päin.
\par 14 Isaskar on luiseva aasi, joka loikoilee karjatarhojen välissä.
\par 15 Hän huomasi lepopaikkansa suloiseksi ja maan ihanaksi; niin hän taivutti olkansa taakan alle ja joutui työveroa tekemään.
\par 16 Daan hankkii oikeutta kansalleen, hänkin yhtenä Israelin sukukunnista.
\par 17 Daan on käärmeenä tiellä, on polulla kyynä, joka puree hevosta vuohiseen, niin että ratsastaja syöksyy selin maahan.
\par 18 Sinulta minä odotan pelastusta, Herra.
\par 19 Gaadia ahdistavat rosvojoukot, mutta hän itse ahdistaa heitä heidän kintereillään.
\par 20 Asserista tulee lihavuus, hänen leipänsä, hän tarjoaa kuninkaan herkkuja.
\par 21 Naftali on nopea peura; hän antaa kauniita sanoja.
\par 22 Joosef on nuori hedelmäpuu, nuori hedelmäpuu lähteen reunalla; sen oksat ulottuvat yli muurin.
\par 23 Jousimiehet hätyyttävät häntä, ampuvat ja ahdistavat häntä.
\par 24 Mutta hänen jousensa pysyy lujana, ja hänen käsivartensa ovat notkeat Jaakobin Väkevän avulla, kaitsijan, Israelin kallion,
\par 25 isäsi Jumalan, avulla, joka sinua auttakoon, Kaikkivaltiaan avulla, joka sinua siunatkoon, antakoon siunauksia taivaasta ylhäältä, siunauksia syvyydestä alhaalta, siunauksia nisistä ja kohdusta.
\par 26 Sinun isäsi siunaukset kohoavat yli minun vanhempaini siunausten, yli ikuisten kukkulain ihanuuden. Ne laskeutukoot Joosefin pään päälle, veljiensä ruhtinaan päälaelle.
\par 27 Benjamin on raatelevainen susi; aamulla hän syö riistaa, ja illalla hän jakaa saalista."
\par 28 Nämä ovat kaikki Israelin sukukunnat, luvultaan kaksitoista, ja tämän puhui heille heidän isänsä siunatessaan heidät; hän siunasi jokaisen erikseen erityisellä siunauksella.
\par 29 Ja hän käski heitä ja sanoi heille: "Minut otetaan pois heimoni tykö; haudatkaa minut isieni viereen, siihen luolaan, joka on heettiläisen Efronin vainiolla,
\par 30 luolaan, joka on Makpelan vainiolla, itään päin Mamresta Kanaanin maassa, jonka vainion Aabraham osti heettiläiseltä Efronilta perintöhaudakseen.
\par 31 Siihen on haudattu Aabraham ja hänen vaimonsa Saara, siihen on haudattu Iisak ja hänen vaimonsa Rebekka, ja siihen minäkin hautasin Leean,
\par 32 siihen vainioon, joka luolineen on ostettu heettiläisiltä."
\par 33 Kun Jaakob oli antanut määräyksensä pojilleen, veti hän jalkansa vuoteeseen ja kuoli ja tuli otetuksi heimonsa tykö.

\chapter{50}

\par 1 Ja Joosef vaipui isänsä kasvoja vasten, itki siinä kumartuneena hänen ylitsensä ja suuteli häntä.
\par 2 Sitten Joosef käski lääkäreitä, jotka olivat hänen palveluksessaan, balsamoimaan hänen isänsä, ja lääkärit balsamoivat Israelin.
\par 3 Siihen kului neljäkymmentä päivää, sillä niin pitkä aika kuluu balsamoimiseen. Ja egyptiläiset itkivät häntä seitsemänkymmentä päivää.
\par 4 Sittenkuin hänen muistoksensa vietetty suruaika oli päättynyt, puhui Joosef faraon hoviväelle: "Jos olen saanut armon teidän silmienne edessä, niin puhukaa minun puolestani faraolle näin:
\par 5 Isäni vannotti minua sanoen: 'Katso, minä kuolen, hautaa minut omaan hautaani, jonka olen kaivanut itselleni Kanaanin maassa'. Anna minun siis nyt mennä hautaamaan isäni; sitten palaan takaisin."
\par 6 Farao vastasi: "Mene hautaamaan isäsi sen valan mukaan, jonka olet hänelle vannonut".
\par 7 Niin Joosef meni hautaamaan isäänsä, ja hänen kanssaan menivät kaikki faraon palvelijat, hänen hovinsa vanhimmat ja kaikki Egyptin maan vanhimmat
\par 8 sekä koko Joosefin perhe, hänen veljensä ja hänen isänsä perhe; ainoastaan vaimonsa, lapsensa, pikkukarjansa ja raavaskarjansa he jättivät Goosenin maakuntaan;
\par 9 hänen mukanaan meni myös sekä vaunuja että ratsumiehiä. Ja niin heitä oli sangen suuri joukko.
\par 10 Kun he saapuivat Gooren-Aatadiin, joka on Jordanin tuolla puolella, panivat he siellä toimeen ylen suuret ja juhlalliset valittajaiset, ja hän vietti isänsä surujuhlaa seitsemän päivää.
\par 11 Ja kun maan asukkaat, kanaanilaiset, näkivät surujuhlan Gooren-Aatadissa, sanoivat he: "Siellä on egyptiläisillä suuri surujuhla". Siitä sai paikka nimekseen Aabel-Misraim; se on Jordanin tuolla puolella.
\par 12 Ja hänen poikansa tekivät hänelle, niinkuin hän oli määrännyt heille:
\par 13 hänen poikansa veivät hänet Kanaanin maahan ja hautasivat hänet Makpelan vainiolla olevaan luolaan, jonka vainion Aabraham oli ostanut perintöhaudaksi heettiläiseltä Efronilta ja joka oli itään päin Mamresta.
\par 14 Senjälkeen kuin Joosef oli haudannut isänsä, palasi hän Egyptiin, hän ja hänen veljensä sekä kaikki, jotka hänen kanssaan olivat menneet hautaamaan hänen isäänsä.
\par 15 Mutta kun Joosefin veljet näkivät, että heidän isänsä oli kuollut, ajattelivat he: "Ehkä Joosef nyt alkaa vainota meitä ja kostaa meille kaiken sen pahan, mitä me olemme hänelle tehneet".
\par 16 Niin he lähettivät Joosefille tämän sanan: "Isäsi käski ennen kuolemaansa ja sanoi:
\par 17 'Sanokaa Joosefille näin: Oi, anna anteeksi veljiesi rikos ja synti, sillä pahasti he ovat menetelleet sinua kohtaan'. Anna siis isäsi Jumalan palvelijoille anteeksi heidän rikoksensa." Ja Joosef itki kuullessaan nämä heidän sanansa.
\par 18 Sitten tulivat Joosefin veljet itse, lankesivat maahan hänen eteensä ja sanoivat: "Katso, me olemme sinun orjiasi!"
\par 19 Mutta Joosef vastasi heille: "Älkää peljätkö, olenko minä Jumalan sijassa?
\par 20 Te tosin hankitsitte minua vastaan pahaa, mutta Jumala on kääntänyt sen hyväksi, että hän saisi aikaan sen, mikä nyt on tapahtunut, ja pitäisi hengissä paljon kansaa.
\par 21 Älkää siis peljätkö; minä elätän teidät ja teidän vaimonne ja lapsenne." Ja hän lohdutti ja rauhoitti heitä.
\par 22 Ja Joosef sekä hänen isänsä perhe jäivät asumaan Egyptiin. Ja Joosef eli sadan kymmenen vuoden vanhaksi.
\par 23 Ja Joosef sai nähdä Efraimin lapsia kolmanteen polveen; myöskin Maakirista, Manassen pojasta, syntyi lapsia Joosefin polville.
\par 24 Niin Joosef sanoi veljilleen: "Minä kuolen, mutta Jumala pitää huolen teistä ja johdattaa teidät tästä maasta siihen maahan, jonka hän valalla vannoen on luvannut Aabrahamille, Iisakille ja Jaakobille".
\par 25 Ja Joosef vannotti Israelin poikia sanoen: "Kun Jumala pitää huolen teistä, viekää silloin minun luuni täältä".
\par 26 Ja Joosef kuoli sadan kymmenen vuoden vanhana. Ja hänet balsamoitiin ja pantiin arkkuun Egyptissä.


\end{document}