\begin{document}

\title{Neljäs Mooseksen kirja}


\chapter{1}

\par 1 Ja Herra puhui Moosekselle Siinain erämaassa ilmestysmajassa toisen kuukauden ensimmäisenä päivänä, toisena vuotena siitä, kun he olivat lähteneet Egyptin maasta, sanoen:
\par 2 "Laskekaa koko Israelin kansan väkiluku suvuittain ja perhekunnittain, nimien lukumäärän mukaan, kaikki miehenpuolet, pääluvun mukaan.
\par 3 Pitäkää katselmus, sinä ja Aaron, kaikista Israelin sotakelpoisista miehistä, kaksikymmenvuotisista ja sitä vanhemmista, osastoittain.
\par 4 Ja teidän apunanne olkoon yksi mies jokaisesta sukukunnasta, sen perhekuntien päämies.
\par 5 Ja nämä ovat niiden miesten nimet, jotka teitä avustakoot: Ruubenista Elisur, Sedeurin poika;
\par 6 Simeonista Selumiel, Suurisaddain poika;
\par 7 Juudasta Nahson, Amminadabin poika;
\par 8 Isaskarista Netanel, Suuarin poika;
\par 9 Sebulonista Eliab, Heelonin poika;
\par 10 Joosefin pojista: Efraimista Elisama, Ammihudin poika ja Manassesta Gamliel, Pedasurin poika;
\par 11 Benjaminista Abidan, Gideonin poika;
\par 12 Daanista Ahieser, Ammisaddain poika;
\par 13 Asserista Pagiel, Okranin poika;
\par 14 Gaadista Eljasaf, Deguelin poika;
\par 15 Naftalista Ahira, Eenanin poika.
\par 16 Nämä olkoot kansan edusmiehet, heidän isiensä sukukuntien ruhtinaat, Israelin heimojen päämiehet."
\par 17 Ja Mooses ja Aaron ottivat luoksensa nämä nimeltä mainitut miehet
\par 18 ja kokosivat kaiken kansan toisen kuukauden ensimmäisenä päivänä, ja kansa pantiin luetteloihin suvuittain ja perhekunnittain, nimien lukumäärän mukaan, kaksikymmenvuotiset ja sitä vanhemmat, pääluvun mukaan,
\par 19 niinkuin Herra oli Moosekselle käskyn antanut. Niin hän piti heistä katselmuksen Siinain erämaassa.
\par 20 Ruubenin, Israelin esikoisen, jälkeläisiä oli polveutumisensa mukaan, suvuittain ja perhekunnittain, nimien lukumäärän, pääluvun mukaan, kaikkiaan miehenpuolia, kaksikymmenvuotisia ja sitä vanhempia, kaikkia sotakelpoisia,
\par 21 näitä Ruubenin sukukunnasta katselmuksessa olleita oli neljäkymmentäkuusi tuhatta viisisataa.
\par 22 Simeonin jälkeläisiä, katselmuksessa olleita, oli polveutumisensa mukaan, suvuittain ja perhekunnittain, nimien lukumäärän, pääluvun mukaan, kaikkiaan miehenpuolia, kaksikymmenvuotisia ja sitä vanhempia, kaikkia sotakelpoisia,
\par 23 näitä Simeonin sukukunnasta katselmuksessa olleita oli viisikymmentäyhdeksän tuhatta kolmesataa.
\par 24 Gaadin jälkeläisiä oli polveutumisensa mukaan, suvuittain ja perhekunnittain, nimien lukumäärän mukaan, kaksikymmenvuotisia ja sitä vanhempia, kaikkia sotakelpoisia,
\par 25 näitä Gaadin sukukunnasta katselmuksessa olleita oli neljäkymmentäviisi tuhatta kuusisataa viisikymmentä.
\par 26 Juudan jälkeläisiä oli polveutumisensa mukaan, suvuittain ja perhekunnittain, nimien lukumäärän mukaan, kaksikymmenvuotisia ja sitä vanhempia, kaikkia sotakelpoisia,
\par 27 näitä Juudan sukukunnasta katselmuksessa olleita oli seitsemänkymmentäneljä tuhatta kuusisataa.
\par 28 Isaskarin jälkeläisiä oli polveutumisensa mukaan, suvuittain ja perhekunnittain, nimien lukumäärän mukaan, kaksikymmenvuotisia ja sitä vanhempia, kaikkia sotakelpoisia,
\par 29 näitä Isaskarin sukukunnasta katselmuksessa olleita oli viisikymmentäneljä tuhatta neljäsataa.
\par 30 Sebulonin jälkeläisiä oli polveutumisensa mukaan, suvuittain ja perhekunnittain, nimien lukumäärän mukaan, kaksikymmenvuotisia ja sitä vanhempia, kaikkia sotakelpoisia,
\par 31 näitä Sebulonin sukukunnasta katselmuksessa olleita oli viisikymmentäseitsemän tuhatta neljäsataa.
\par 32 Joosefin jälkeläisiä, Efraimin jälkeläisiä, oli polveutumisensa mukaan, suvuittain ja perhekunnittain, nimien lukumäärän mukaan, kaksikymmenvuotisia ja sitä vanhempia, kaikkia sotakelpoisia,
\par 33 näitä Efraimin sukukunnasta katselmuksessa olleita oli neljäkymmentätuhatta viisisataa.
\par 34 Manassen jälkeläisiä oli polveutumisensa mukaan, suvuittain ja perhekunnittain, nimien lukumäärän mukaan, kaksikymmenvuotisia ja sitä vanhempia, kaikkia sotakelpoisia,
\par 35 näitä Manassen sukukunnasta katselmuksessa olleita oli kolmekymmentäkaksi tuhatta kaksisataa.
\par 36 Benjaminin jälkeläisiä oli polveutumisensa mukaan, suvuittain ja perhekunnittain, nimien lukumäärän mukaan, kaksikymmenvuotisia ja sitä vanhempia, kaikkia sotakelpoisia,
\par 37 näitä Benjaminin sukukunnasta katselmuksessa olleita oli kolmekymmentäviisi tuhatta neljäsataa.
\par 38 Daanin jälkeläisiä oli polveutumisensa mukaan, suvuittain ja perhekunnittain, nimien lukumäärän mukaan, kaksikymmenvuotisia ja sitä vanhempia, kaikkia sotakelpoisia,
\par 39 näitä Daanin sukukunnasta katselmuksessa olleita oli kuusikymmentäkaksi tuhatta seitsemänsataa.
\par 40 Asserin jälkeläisiä oli polveutumisensa mukaan, suvuittain ja perhekunnittain, nimien lukumäärän mukaan, kaksikymmenvuotisia ja sitä vanhempia, kaikkia sotakelpoisia,
\par 41 näitä Asserin sukukunnasta katselmuksessa olleita oli neljäkymmentäyksi tuhatta viisisataa.
\par 42 Naftalin jälkeläisiä oli polveutumisensa mukaan, suvuittain ja perhekunnittain, nimien lukumäärän mukaan, kaksikymmenvuotisia ja sitä vanhempia, kaikkia sotakelpoisia,
\par 43 näitä Naftalin sukukunnasta katselmuksessa olleita oli viisikymmentäkolme tuhatta neljäsataa.
\par 44 Nämä ovat ne katselmuksessa olleet, joista Mooses ja Aaron ja Israelin ruhtinaat, nuo kaksitoista miestä, perhekuntiansa edustaen, pitivät katselmuksen.
\par 45 Ja kaikkiaan katselmuksessa olleita israelilaisia oli perhekunnittain, kaksikymmenvuotisia ja sitä vanhempia, kaikkia Israelin sotakelpoisia miehiä,
\par 46 näitä katselmuksessa olleita oli yhteensä kuusisataakolme tuhatta viisisataa viisikymmentä.
\par 47 Mutta leeviläisistä, heidän isiensä sukukunnasta, ei pidetty katselmusta yhdessä muiden kanssa.
\par 48 Sillä Herra puhui Moosekselle sanoen:
\par 49 "Ainoastaan Leevin sukukunnasta älä pidä katselmusta äläkä laske heidän väkilukuansa yhdessä muiden israelilaisten kanssa,
\par 50 vaan pane leeviläiset hoitamaan lain asumusta ja sen kaikkea kalustoa ja kaikkea, mitä siihen kuuluu. He kantakoot asumusta ja sen kaikkea kalustoa, ja he toimittakoot siinä palvelusta sekä leiriytykööt asumuksen ympärille.
\par 51 Kun asumus lähtee liikkeelle, purkakoot leeviläiset sen; ja kun asumus pysähtyy leiripaikkaan, pystyttäkööt leeviläiset sen. Syrjäinen, joka siihen ryhtyy, surmattakoon.
\par 52 Muut israelilaiset asettukoot kukin omaan leiriinsä ja kukin lippunsa luo, osastoittain.
\par 53 Mutta leeviläiset leiriytykööt lain asumuksen ympärille, ettei Herran viha kohtaisi Israelin kansaa, ja leeviläiset hoitakoot tehtäviä lain asumuksessa."
\par 54 Ja israelilaiset tekivät kaiken, niinkuin Herra oli Moosekselle käskyn antanut; niin he tekivät.

\chapter{2}

\par 1 Ja Herra puhui Moosekselle ja Aaronille sanoen:
\par 2 "Israelilaiset leiriytykööt kukin lippunsa luo, perhekuntiensa sotamerkkien kohdalle; he leiriytykööt taammas ilmestysmajan ympärille.
\par 3 Etupuolelle, itään päin, leiriytyköön Juudan leirin lippukunta osastoittain, ja Juudan jälkeläisten päämiehenä olkoon Nahson, Amminadabin poika;
\par 4 hänen osastonsa, katselmuksessa olleet, on seitsemänkymmentäneljä tuhatta kuusisataa miestä.
\par 5 Hänen viereensä leiriytyköön Isaskarin sukukunta, ja Isaskarin jälkeläisten päämiehenä olkoon Netanel, Suuarin poika;
\par 6 hänen osastonsa, katselmuksessa olleet, on viisikymmentäneljä tuhatta neljäsataa miestä.
\par 7 Sitten Sebulonin sukukunta, ja Sebulonin jälkeläisten päämiehenä olkoon Eliab, Heelonin poika;
\par 8 hänen osastonsa, katselmuksessa olleet, on viisikymmentäseitsemän tuhatta neljäsataa miestä.
\par 9 Juudan leirin katselmuksessa olleita on siis kaikkiaan sata kahdeksankymmentäkuusi tuhatta neljäsataa miestä, osastoittain; he lähtekööt liikkeelle ensimmäisenä joukkona.
\par 10 Ruubenin leirin lippukunta leiriytyköön etelään päin osastoittain, ja Ruubenin jälkeläisten päämiehenä olkoon Elisur, Sedeurin poika;
\par 11 hänen osastonsa, katselmuksessa olleet, on neljäkymmentäkuusi tuhatta viisisataa miestä.
\par 12 Hänen viereensä leiriytyköön Simeonin sukukunta, ja Simeonin jälkeläisten päämiehenä olkoon Selumiel, Suurisaddain poika;
\par 13 hänen osastonsa, katselmuksessa olleet, on viisikymmentäyhdeksän tuhatta kolmesataa miestä.
\par 14 Sitten Gaadin sukukunta, ja Gaadin jälkeläisten päämiehenä olkoon Eljasaf, Reguelin poika;
\par 15 hänen osastonsa, katselmuksessa olleet, on neljäkymmentäviisi tuhatta kuusisataa viisikymmentä miestä.
\par 16 Ruubenin leirin katselmuksessa olleita on siis kaikkiaan sata viisikymmentäyksi tuhatta neljäsataa viisikymmentä miestä, osastoittain. He lähtekööt liikkeelle toisena joukkona.
\par 17 Sitten lähtee liikkeelle ilmestysmaja ja leeviläisten leiri toisten leirien keskellä. Niinkuin olivat leiriytyneet, niin lähteköötkin, kukin kohdallansa, lippukunnittain.
\par 18 Efraimin leirin lippukunta leiriytyköön länteen päin osastoittain, ja Efraimin jälkeläisten päämiehenä olkoon Elisama, Ammihudin poika;
\par 19 hänen osastonsa, katselmuksessa olleet, on neljäkymmentä tuhatta viisisataa miestä.
\par 20 Hänen viereensä leiriytyköön Manassen sukukunta, ja Manassen jälkeläisten päämiehenä olkoon Gamliel, Pedasurin poika;
\par 21 hänen osastonsa, katselmuksessa olleet, on kolmekymmentäkaksi tuhatta kaksisataa miestä.
\par 22 Sitten Benjaminin sukukunta, ja Benjaminin jälkeläisten päämiehenä olkoon Abidan, Gideonin poika;
\par 23 hänen osastonsa, katselmuksessa olleet, on kolmekymmentäviisi tuhatta neljäsataa miestä.
\par 24 Efraimin leirin katselmuksessa olleita on siis kaikkiaan satakahdeksan tuhatta ja sata miestä, osastoittain. He lähtekööt liikkeelle kolmantena joukkona.
\par 25 Daanin leirin lippukunta leiriytyköön pohjoiseen päin osastoittain, ja Daanin jälkeläisten päämiehenä olkoon Ahieser, Ammisaddain poika;
\par 26 hänen osastonsa, katselmuksessa olleet, on kuusikymmentäkaksi tuhatta seitsemänsataa miestä.
\par 27 Hänen viereensä leiriytyköön Asserin sukukunta, ja Asserin jälkeläisten päämiehenä olkoon Pagiel, Okranin poika;
\par 28 hänen osastonsa, katselmuksessa olleet, on neljäkymmentäyksi tuhatta viisisataa miestä.
\par 29 Sitten Naftalin sukukunta, ja Naftalin jälkeläisten päämiehenä olkoon Ahira, Eenanin poika;
\par 30 hänen osastonsa, katselmuksessa olleet, on viisikymmentäkolme tuhatta neljäsataa miestä.
\par 31 Daanin leirin katselmuksessa olleita on siis kaikkiaan sata viisikymmentäseitsemän tuhatta kuusisataa miestä. He lähtekööt liikkeelle viimeisenä joukkona, lippukuntinensa."
\par 32 Nämä olivat israelilaisten katselmuksessa olleet, perhekunnittain; eri leirien katselmuksessa olleita oli osastoittain kaikkiaan kuusisataakolme tuhatta viisisataa viisikymmentä miestä.
\par 33 Mutta leeviläisten katselmusta ei pidetty samalla kuin muiden israelilaisten, sillä niin oli Herra Moosekselle käskyn antanut.
\par 34 Ja israelilaiset tekivät kaiken, niinkuin Herra oli Moosekselle käskyn antanut. Niin he leiriytyivät lippukunnittain, ja niin he lähtivät liikkeelle kukin suvuittain ja perhekunnittain.

\chapter{3}

\par 1 Tämä on kertomus Aaronin ja Mooseksen suvusta, siihen aikaan kun Herra puhui Moosekselle Siinain vuorella.
\par 2 Nämä ovat Aaronin poikien nimet: Naadab, esikoinen, Abihu, Eleasar ja Iitamar.
\par 3 Nämä ovat Aaronin poikien, voideltujen pappien, nimet, niiden, jotka olivat vihityt papinvirkaa toimittamaan.
\par 4 Mutta Naadab ja Abihu kuolivat Herran edessä tuodessaan vierasta tulta Herran eteen Siinain erämaassa; ja heillä ei ollut poikia. Niin Eleasar ja Iitamar toimittivat papinvirkaa isänsä Aaronin edessä.
\par 5 Ja Herra puhui Moosekselle sanoen:
\par 6 "Tuo Leevin sukukunta esille ja aseta se pappi Aaronin eteen palvelemaan häntä.
\par 7 Ja he hoitakoot sekä hänen että koko kansan tehtäviä ilmestysmajan edustalla, toimittakoot palvelusta asumuksessa.
\par 8 Ja he hoitakoot kaikkea ilmestysmajan kalustoa ja israelilaisten tehtäviä, toimittakoot palvelusta asumuksessa.
\par 9 Niin on sinun annettava leeviläiset Aaronille ja hänen pojilleen; antakoot israelilaiset heidät hänen haltuunsa, hänelle palvelijoiksi.
\par 10 Mutta Aaronin ja hänen poikiensa käske hoitaa papinvirkansa. Syrjäinen, joka siihen ryhtyy, surmattakoon."
\par 11 Ja Herra puhui Moosekselle sanoen:
\par 12 "Katso, minä itse olen ottanut leeviläiset israelilaisten keskuudesta israelilaisten kaikkien esikoisten sijaan, kaiken sijaan, joka avaa äidinkohdun; leeviläiset ovat siis tulleet minun omikseni.
\par 13 Sillä minun omani on jokainen esikoinen. Sinä päivänä, jona minä surmasin kaikki Egyptin maan esikoiset, minä pyhitin itselleni kaikki Israelin esikoiset, sekä ihmisten että karjan; minun omani ne ovat. Minä olen Herra."
\par 14 Ja Herra puhui Moosekselle Siinain erämaassa sanoen:
\par 15 "Pidä leeviläisten katselmus perhekunnittain ja suvuittain; kaikista miehenpuolista, kuukauden ikäisistä ja sitä vanhemmista, pidä katselmus".
\par 16 Ja Mooses piti heistä katselmuksen Herran käskyn mukaan, niinkuin hänelle oli määrätty.
\par 17 Ja nämä olivat Leevin poikien nimet: Geerson, Kehat ja Merari.
\par 18 Ja nämä olivat Geersonin poikien nimet, suvuittain: Libni ja Siimei.
\par 19 Ja Kehatin pojat olivat, suvuittain, Amram, Jishar, Hebron ja Ussiel.
\par 20 Ja Merarin pojat olivat, suvuittain, Mahli ja Muusi. Nämä olivat Leevin suvut perhekunnittain.
\par 21 Geersonista polveutuivat libniläisten suku ja siimeiläisten suku. Nämä olivat geersonilaisten suvut.
\par 22 Katselmuksessa olleita oli heitä, kaikkien miehenpuolten, kuukauden ikäisten ja sitä vanhempien, lukumäärän mukaan - näitä katselmuksessa olleita oli seitsemäntuhatta viisisataa miestä.
\par 23 Geersonilaisten suvut leiriytyivät asumuksen taakse länteen päin.
\par 24 Ja geersonilaisten perhekunta-päämies oli Eljasaf, Laaelin poika.
\par 25 Geersonin poikien hoidettavana ilmestysmajassa oli asumus ja teltta, sen peite, ilmestysmajan oven uudin
\par 26 ja asumusta ja alttaria ympäröivän esipihan ympärysverhot ja sen portin uudin sekä sen köydet - kaikki ne palvelustehtävineen.
\par 27 Kehatista polveutuivat amramilaisten suku, jisharilaisten suku, hebronilaisten suku ja ussielilaisten suku. Nämä olivat kehatilaisten suvut.
\par 28 Miehenpuolten, kuukauden ikäisten ja sitä vanhempien, lukumäärän mukaan oli heitä kaikkiaan kahdeksantuhatta kuusisataa, jotka hoitivat pyhäkössä suoritettavia tehtäviä.
\par 29 Kehatilaisten suvut leiriytyivät asumuksen sivulle etelään päin.
\par 30 Ja kehatilaisten sukujen perhekunta-päämies oli Elisafan, Ussielin poika.
\par 31 Heidän hoidettavanaan oli arkki, pöytä, seitsenhaarainen lamppu, alttarit ja pyhä kalusto, jota jumalanpalveluksessa käytettiin, ynnä esirippu - kaikkine palvelustehtävineen.
\par 32 Ja leeviläisten päämiesten esimies oli Eleasar, pappi Aaronin poika; hän oli niiden päällysmies, jotka hoitivat tehtäviä pyhäkössä.
\par 33 Merarista polveutuivat mahlilaisten suku ja muusilaisten suku. Nämä olivat Merarin suvut.
\par 34 Katselmuksessa olleita oli heitä kaikkien miehenpuolten, kuukauden ikäisten ja sitä vanhempien, lukumäärän mukaan kuusituhatta kaksisataa.
\par 35 Ja Merarin sukujen perhekunta-päämies oli Suuriel, Abihailin poika. He leiriytyivät asumuksen sivulle pohjoiseen päin.
\par 36 Ja Merarin poikien hoidettavaksi oli uskottu asumuksen laudat, poikkitangot, pylväät ja jalustat sekä kaikki sen kalusto - kaikki ne palvelustehtävineen;
\par 37 niinikään ympärillä olevan esipihan pylväät jalustoinensa, vaarnoinensa ja köysinensä.
\par 38 Mutta asumuksen eteen, itään päin, ilmestysmajan eteen, auringon nousuun päin, leiriytyivät Mooses ja Aaron ynnä hänen poikansa; heidän hoidettavanaan oli pyhäkkö, se, mikä oli israelilaisten hoidettavana. Mutta syrjäinen, joka ryhtyi siihen, oli surmattava.
\par 39 Katselmuksessa olleita leeviläisiä, joista Mooses ja Aaron Herran käskyn mukaan pitivät katselmuksen suvuittain, kaikkia miehenpuolia, kuukauden ikäisiä ja sitä vanhempia, oli kaikkiaan kaksikymmentäkaksi tuhatta.
\par 40 Ja Herra sanoi Moosekselle: "Katsasta kaikki israelilaisten miehenpuoliset esikoiset, kuukauden ikäisistä ja sitä vanhemmista, ja laske heidän nimiensä lukumäärä.
\par 41 Ja ota minulle - minä olen Herra - leeviläiset israelilaisten kaikkien esikoisten sijaan sekä myös leeviläisten karja israelilaisten karjan kaikkien esikoisten sijaan."
\par 42 Ja Mooses katsasti, niinkuin Herra oli häntä käskenyt, kaikki israelilaisten esikoiset.
\par 43 Ja katselmuksessa olleita oli heitä, kaikkia miehenpuolisia esikoisia, nimien lukumäärän mukaan, kuukauden ikäisiä ja sitä vanhempia, kaikkiaan kaksikymmentäkaksi tuhatta kaksisataa seitsemänkymmentä kolme.
\par 44 Ja Herra puhui Moosekselle sanoen:
\par 45 "Ota leeviläiset israelilaisten kaikkien esikoisten sijaan ja leeviläisten karja heidän karjansa sijaan, ja niin leeviläiset tulevat minun omikseni. Minä olen Herra.
\par 46 Mutta lunastukseksi noista kahdestasadasta seitsemästäkymmenestä kolmesta, jonka verran israelilaisten esikoisia on yli leeviläisten määrän,
\par 47 ota miestä kohti viisi sekeliä, ota ne pyhäkkösekelin painon mukaan, kaksikymmentä geeraa laskettuna sekeliin.
\par 48 Ja anna se raha Aaronille ja hänen pojillensa lunastukseksi niistä, joita on yli leeviläisten määrän."
\par 49 Ja Mooses otti lunastushinnan niistä, joita oli yli leeviläisillä lunastettujen määrän.
\par 50 Israelilaisten esikoisista hän otti tämän rahan, tuhat kolmesataa kuusikymmentä viisi sekeliä pyhäkkösekelin painon mukaan.
\par 51 Ja Mooses antoi sen lunastushinnan Aaronille ja hänen pojillensa Herran käskyn mukaan, niinkuin Herra oli Moosesta käskenyt.

\chapter{4}

\par 1 Ja Herra puhui Moosekselle ja Aaronille sanoen:
\par 2 "Laske Leevin jälkeläisten joukosta Kehatin jälkeläisten lukumäärä suvuittain ja perhekunnittain,
\par 3 kolmikymmenvuotiset ja sitä vanhemmat aina viisikymmenvuotisiin saakka, kaikki, joiden on palveltava toimittamalla askareita ilmestysmajassa.
\par 4 Kehatin poikien palvelustehtävänä ilmestysmajassa olkoon tämä: huolenpito korkeasti-pyhistä.
\par 5 Ja leirin liikkeelle lähtiessä menkööt Aaron ja hänen poikansa ja päästäkööt alas esiripun sekä peittäkööt sillä lain arkin;
\par 6 sitten he pankoot sen päälle sireeninnahkapeitteen ja levittäkööt päällimmäiseksi vaatteen, kokonaan punasinisistä langoista tehdyn, ja asettakoot paikoilleen sen korennot.
\par 7 Ja näkyleipäpöydälle he levittäkööt punasinisen vaatteen ja pankoot sen päälle vadit ja kupit, maljat ja juomauhrikannut; ja ainainen leipä olkoon myös sen päällä.
\par 8 Sitten he levittäkööt näiden päälle helakanpunaisen vaatteen ja peittäkööt sen sireeninnahkapeitteellä ja asettakoot paikoilleen sen korennot.
\par 9 Ja he ottakoot punasinisen vaatteen ja peittäkööt seitsenhaaraisen lampun lamppuineen, lamppusaksineen, karstakuppeineen ja kaikkine öljyastioineen, joita sen hoitamisessa käytetään.
\par 10 Ja he käärikööt sen ja kaiken sen kaluston sireeninnahkapeitteeseen ja pankoot kantopaareille.
\par 11 Ja kultaiselle alttarille he levittäkööt punasinisen vaatteen ja peittäkööt sen sireeninnahkapeitteellä ja asettakoot paikoilleen sen korennot.
\par 12 Ja he ottakoot kaiken jumalanpalveluskaluston, jota pyhäkössä käytetään, ja käärikööt sen punasiniseen vaatteeseen ja peittäkööt sen sireeninnahkapeitteellä ja pankoot kantopaareille.
\par 13 Ja he puhdistakoot tuhasta alttarin sekä levittäkööt sen päälle purppuranpunaisen vaatteen
\par 14 ja pankoot sen päälle kaiken kaluston, jota alttarilla käytetään: hiilipannut, haarukat, lapiot ja maljat, kaiken alttarikaluston; ja levittäkööt sen päälle sireeninnahkapeitteen ja asettakoot paikoilleen sen korennot.
\par 15 Ja kun leirin lähtiessä liikkeelle Aaron ja hänen poikansa ovat valmiiksi peittäneet pyhäkön ja pyhäkön kaiken kaluston, niin tulkoot Kehatin pojat kantamaan, mutta älkööt koskeko pyhäkköön, etteivät kuolisi. Tämä on ilmestysmajasta se, mikä on Kehatin poikien kannettava.
\par 16 Ja Eleasar, pappi Aaronin poika, pitäköön huolen seitsenhaaraisen lampun öljystä, hyvänhajuisesta suitsukkeesta, jokapäiväisestä ruokauhrista ja voiteluöljystä; hän pitäköön huolen koko asumuksesta ja kaikesta, mitä siinä on, pyhäköstä ja sen kalustosta."
\par 17 Ja Herra puhui Moosekselle ja Aaronille sanoen:
\par 18 "Älkää päästäkö kehatilaisten sukuhaaraa häviämään leeviläisten joukosta.
\par 19 Näin siis tehkää heille, että he saisivat elää eivätkä kuolisi lähestyessään kaikkeinpyhintä: Aaron ja hänen poikansa menkööt ja asettakoot heidät jokaisen tekemään palvelustehtäväänsä ja kantamaan kannettavaansa;
\par 20 mutta älkööt nämä menkö katsomaan pyhiä esineitä, älkööt silmänräpäykseksikään, etteivät kuolisi."
\par 21 Ja Herra puhui Moosekselle sanoen:
\par 22 "Laske myöskin Geersonin jälkeläisten lukumäärä perhekunnittain ja suvuittain;
\par 23 pidä kolmikymmenvuotisten katselmus ja sitä vanhempien, viisikymmenvuotisiin asti, kaikkien, joiden on palveltava toimittamalla palvelusta ilmestysmajassa.
\par 24 Geersonilaissukujen tehtävänä palveltaessa ja kannettaessa olkoon tämä:
\par 25 he kantakoot asumuksen telttakankaan kaistat ja ilmestysmajan, sen peitteen ja sireeninnahkapeitteen, joka on sen päällä päällimmäisenä, sekä ilmestysmajan oven uutimen
\par 26 ja asumusta ja alttaria ympäröivän esipihan ympärysverhot ja sen portin uutimen sekä niiden köydet ja kaiken niiden hoitamiseen tarvittavan kaluston; he toimittakoot kaiken, mitä siinä on tehtävä.
\par 27 Aaronin ja hänen poikiensa käskyn mukaisesti tapahtukoon kaikki geersonilaisten palvelus, mitä hyvänsä he kantavat ja toimittavatkin, ja antakaa heidän hoitoonsa kaikki heidän kannettavansa.
\par 28 Tämä on geersonilaissukujen palvelustehtävä ilmestysmajassa; ja he hoitakoot toimensa Iitamarin, pappi Aaronin pojan, johdolla.
\par 29 Pidä Merarin jälkeläisten katselmus suvuittain ja perhekunnittain;
\par 30 pidä kolmikymmenvuotisten katselmus ja sitä vanhempien, viisikymmenvuotisiin asti, kaikkien, joiden on palveltava toimittamalla palvelusta ilmestysmajassa.
\par 31 Ja nämä ovat uskotut heille kannettaviksi heidän toimittaessaan palvelusta ilmestysmajassa: asumuksen laudat, sen poikkitangot, pylväät ja jalustat
\par 32 sekä ympärillä olevan esipihan pylväät jalustoineen, vaarnoineen ja köysineen, kaikki niiden kalusto ja kaikki, mitä tarvitaan niiden hoitamiseen. Ja antakaa heidän hoitoonsa nimeltä mainittuina ne kalut, joiden kantaminen on heidän tehtävänään.
\par 33 Tämä on merarilaissukujen palvelustehtävä, mitä hyvänsä he toimittavatkin ilmestysmajassa Iitamarin, pappi Aaronin pojan, johdolla."
\par 34 Ja Mooses ja Aaron sekä kansan päämiehet pitivät Kehatin jälkeläisten katselmuksen suvuittain ja perhekunnittain,
\par 35 kolmikymmenvuotisten ja sitä vanhempien, aina viisikymmenvuotisiin asti, kaikkien, joiden oli palveltava toimittamalla palvelusta ilmestysmajassa.
\par 36 Ja heitä, suvuittain katselmuksessa olleita, oli kaksituhatta seitsemänsataa viisikymmentä.
\par 37 Näin paljon oli kehatilaissuvuista katselmuksessa olleita, kaikkia ilmestysmajassa palvelevia, joiden katselmuksen Mooses ja Aaron pitivät sen käskyn mukaisesti, jonka Herra oli Mooseksen kautta antanut.
\par 38 Ja Geersonin jälkeläisiä, joiden katselmus pidettiin suvuittain ja perhekunnittain,
\par 39 kolmikymmenvuotisia ja sitä vanhempia viisikymmenvuotisiin asti, kaikkia, joiden oli palveltava toimittamalla palvelusta ilmestysmajassa,
\par 40 näitä katselmuksessa olleita oli suvuittain ja perhekunnittain kaksituhatta kuusisataa kolmekymmentä.
\par 41 Näin paljon oli geersonilaisten suvuista katselmuksessa olleita, kaikkia ilmestysmajassa palvelevia, joiden katselmuksen Mooses ja Aaron pitivät Herran käskyn mukaisesti.
\par 42 Ja merarilaisten suvuista oli niitä, joiden katselmus pidettiin suvuittain ja perhekunnittain,
\par 43 kolmikymmenvuotisia ja sitä vanhempia viisikymmenvuotisiin asti, kaikkia, joiden oli palveltava toimittamalla palvelusta ilmestysmajassa,
\par 44 näitä, suvuittain katselmuksessa olleita, oli kolmetuhatta kaksisataa.
\par 45 Näin paljon oli merarilaisten suvuista katselmuksessa olleita, joista Mooses ja Aaron pitivät katselmuksen sen käskyn mukaisesti, jonka Herra oli Mooseksen kautta antanut.
\par 46 Kaikkia katselmuksessa olleita leeviläisiä, joiden katselmuksen Mooses ja Aaron ja Israelin päämiehet pitivät suvuittain ja perhekunnittain,
\par 47 kolmikymmenvuotisia ja sitä vanhempia, viisikymmenvuotisiin asti, kaikkia, joiden oli palveltava toimittamalla palvelusta ja kantamistyötä ilmestysmajassa,
\par 48 näitä katselmuksessa olleita oli kahdeksantuhatta viisisataa kahdeksankymmentä.
\par 49 Sen käskyn mukaisesti, jonka Herra oli Mooseksen kautta antanut, määrättiin jokaiselle heistä oma palvelustehtävänsä ja kantamistyönsä. Tämän katselmuksen Mooses piti, niinkuin Herra oli häntä käskenyt.

\chapter{5}

\par 1 Ja Herra puhui Moosekselle sanoen:
\par 2 "Käske israelilaisten karkoittaa leiristä jokainen pitalinen ja jokainen vuotoa sairastava ja jokainen kuolleesta saastunut.
\par 3 Karkoittakaa sellaiset, sekä miehet että naiset, leirin ulkopuolelle, etteivät he saastuttaisi leiriään, jossa minä asun heidän keskellänsä."
\par 4 Ja israelilaiset tekivät niin ja karkoittivat heidät leirin ulkopuolelle; niinkuin Herra oli puhunut Moosekselle, niin israelilaiset tekivät.
\par 5 Ja Herra puhui Moosekselle sanoen:
\par 6 "Puhu israelilaisille: Jos mies tai nainen tekee rikkomuksen minkä tahansa, jonka ihminen tekee menettelemällä uskottomasti Herraa kohtaan, ja niin joutuu vikapääksi,
\par 7 tunnustakoon hän rikkomuksensa, jonka on tehnyt, ja maksakoon velkansa täyteen määrään sekä pankoon siihen lisäksi viidennen osan sen arvosta ja antakoon sen sille, jolle hän on velkaa.
\par 8 Mutta jos hänellä ei ole sukulunastajaa, jolle velka olisi maksettava, maksettakoon velka Herralle, se on papille, ja sitä paitsi sovitus-oinas, jolla toimitetaan syynalaiselle sovitus.
\par 9 Ja jokainen anti, mikä hyvänsä israelilaisten papille tuomista pyhistä lahjoista, olkoon papin oma.
\par 10 Ja jokaisen pyhät lahjat olkoot papin omat; mitä joku antaa papille, se olkoon papin oma."
\par 11 Ja Herra puhui Moosekselle sanoen:
\par 12 "Puhu israelilaisille ja sano heille: jos jonkun miehen vaimo on pettänyt miehensä ja ollut hänelle uskoton
\par 13 ja joku on maannut hänen kanssaan eikä hänen miehensä ole saanut sitä tietää ja vaimo on saanut sen salatuksi, vaikka hän on saastuttanut itsensä, eikä ole todistajaa häntä vastaan eikä häntä ole teosta tavattu,
\par 14 mutta luulevaisuuden henki on vallannut miehen ja hän luulee vaimoansa, joka onkin saastuttanut itsensä, tahi jos luulevaisuuden henki on vallannut miehen ja hän luulee vaimoansa, vaikka tämä ei ole saastuttanut itseänsä,
\par 15 niin tuokoon mies vaimonsa papin eteen ja uhrilahjana hänen puolestaan kymmenennen osan eefa-mittaa ohrajauhoja, mutta älköön hän sen päälle vuodattako öljyä älköönkä panko suitsuketta, sillä se on luulevaisuusuhri, muistutusuhri, joka johdattaa muistoon pahan teon.
\par 16 Ja pappi tuokoon vaimon ja asettakoon hänet Herran eteen.
\par 17 Ja pappi ottakoon pyhää vettä saviastiaan, ja sitten pappi ottakoon tomua asumuksen lattialta ja pankoon veteen.
\par 18 Ja pappi asettakoon vaimon Herran eteen, päästäköön vaimon tukan hajalle ja asettakoon hänen käsiinsä muistutusuhrin, se on luulevaisuusuhrin, ja papin kädessä olkoon katkera kirovesi.
\par 19 Ja pappi vannottakoon vaimoa ja sanokoon hänelle: 'Jos ei kukaan ole maannut sinun kanssasi etkä ole pettänyt miestäsi saastuttamalla itseäsi, niin älköön tämä katkera kirovesi sinua vahingoittako.
\par 20 Mutta jos olet pettänyt miehesi ja saastuttanut itsesi ja joku muu kuin miehesi on maannut sinun kanssasi,
\par 21 niin' - pappi vannottakoon vaimoa kirovalalla ja pappi sanokoon vaimolle - 'tehköön Herra sinun nimesi kiroukseksi ja sadatukseksi sinun kansasi keskuudessa, kuihduttakoon Herra sinun lanteesi ja paisuttakoon vatsasi;
\par 22 menköön tämä kirovesi sinun sisuksiisi, niin että vatsasi paisuu ja lanteesi kuihtuvat.' Ja vaimo sanokoon: 'Amen, Amen'.
\par 23 Ja pappi kirjoittakoon nämä kiroukset ja huuhtokoon kirjoituksen kiroveteen.
\par 24 Sitten hän juottakoon vaimolla sen katkeran kiroveden, ja menköön kirovesi häneen, tuskaksi hänelle.
\par 25 Ja pappi ottakoon vaimon kädestä luulevaisuusuhrin ja toimittakoon uhrin heilutuksen Herran edessä ja tuokoon sen alttarille.
\par 26 Ja pappi ottakoon uhrista kourallisen alttariuhriosaksi ja polttakoon sen alttarilla ja juottakoon sitten vaimolla veden.
\par 27 Ja kun hän on juottanut vaimolla veden, niin, jos hän on saastuttanut itsensä ja ollut miehelleen uskoton, menee kirovesi häneen, hänelle tuskaksi, ja hänen vatsansa paisuu ja lanteensa kuihtuvat, ja vaimon nimi tulee kiroukseksi hänen kansansa keskuudessa.
\par 28 Mutta jos vaimo ei ole saastuttanut itseänsä, vaan on puhdas, ei hän vahingoitu, vaan pysyy hedelmällisenä.
\par 29 Tämä on laki luulevaisuudesta. Jos vaimo pettää miehensä ja saastuttaa itsensä,
\par 30 tai jos jonkun valtaa luulevaisuuden henki ja hän luulee vaimoansa, niin hän asettakoon vaimonsa Herran eteen, ja pappi tehköön vaimolle kaiken tämän lain mukaan.
\par 31 Niin mies olkoon syyllisyydestä vapaa, mutta vaimo kantakoon syyllisyytensä."

\chapter{6}

\par 1 Ja Herra puhui Moosekselle sanoen:
\par 2 "Puhu israelilaisille ja sano heille: Kun mies tai nainen tekee nasiirilupauksen vihkiytyäksensä Herralle,
\par 3 niin pidättyköön hän viinistä ja väkijuomasta; älköön juoko hapanviiniä tai hapanta juomaa älköönkä mitään viinirypäleen mehua; älköön syökö tuoreita älköönkä kuivia rypäleitä.
\par 4 Niin kauan kuin hänen nasiirilupauksensa kestää, älköön hän syökö mitään, mikä viiniköynnöksestä saadaan, älköön edes raakiloita tai köynnösten versoja.
\par 5 Niin kauan kuin hänen nasiirilupauksensa kestää, älköön partaveitsi koskettako hänen päätänsä. Kunnes kuluu umpeen aika, joksi hän on vihkiytynyt Herralle, hän olkoon pyhä ja kasvattakoon päänsä hiukset pitkiksi.
\par 6 Niin kauan kuin hän on Herralle vihkiytynyt, älköön hän menkö kuolleen luo.
\par 7 Älköön hän saastuko edes isästänsä tai äidistänsä, veljestänsä tai sisarestansa heidän kuoltuaan, sillä hänen päässään on Jumalalle-vihkiytymisen merkki.
\par 8 Niin kauan kuin hänen nasiirilupauksensa kestää, hän on pyhä Herralle.
\par 9 Jos joku odottamatta, äkkiarvaamatta kuolee hänen läheisyydessään ja hän niin saastuttaa vihityn päänsä, niin hän ajattakoon hiuksensa puhdistuspäivänään, seitsemäntenä päivänä hän ajattakoon ne.
\par 10 Ja kahdeksantena päivänä hän tuokoon kaksi metsäkyyhkystä tai kaksi kyyhkysenpoikaa papille ilmestysmajan ovelle.
\par 11 Ja pappi uhratkoon toisen syntiuhriksi ja toisen polttouhriksi ja toimittakoon hänelle sovituksen kuolleen tähden tulleesta rikkomuksesta ja pyhittäköön uudelleen hänen päänsä sinä päivänä.
\par 12 Ja hän vihkiytyköön uudelleen Herralle nasiirilupauksensa ajaksi ja tuokoon vuoden vanhan karitsan vikauhriksi. Mutta kulunut aika jääköön lukuunottamatta, koska hän saastutti vihkimyksensä.
\par 13 Ja tämä on nasiirilaki: Sinä päivänä, jona hänen nasiirilupauksensa aika on kulunut umpeen, tuotakoon hänet ilmestysmajan ovelle,
\par 14 ja hän tuokoon uhrilahjanansa Herralle vuoden vanhan virheettömän karitsan polttouhriksi ja vuoden vanhan virheettömän uuhikaritsan syntiuhriksi sekä virheettömän oinaan yhteysuhriksi
\par 15 ynnä korillisen lestyistä jauhoista leivottuja happamattomia leipiä, öljyyn leivottuja kakkuja ja öljyllä voideltuja happamattomia ohukaisia sekä niihin kuuluvan ruoka- ja juomauhrin.
\par 16 Ja pappi tuokoon ne Herran eteen ja toimittakoon hänen syntiuhrinsa ja polttouhrinsa.
\par 17 Mutta oinaan hän uhratkoon yhteysuhriksi Herralle ynnä korillisen happamattomia leipiä; ja pappi toimittakoon myös hänen ruoka- ja juomauhrinsa.
\par 18 Ja nasiiri ajattakoon hiuksensa, vihkiytymisensä merkin, ilmestysmajan ovella ja ottakoon päänsä hiukset, vihkiytymisensä merkin, ja pankoon ne tuleen, joka palaa yhteysuhrin alla.
\par 19 Ja pappi ottakoon oinaan keitetyn lavan ja korista happamattoman kakun sekä happamattoman ohukaisen ja pankoon ne nasiirin käsiin, senjälkeen kuin tämä on ajattanut pois vihkiytymisensä merkin.
\par 20 Ja pappi toimittakoon niiden heilutuksen Herran edessä. Se olkoon pyhä lahja papille, annettava heilutetun rintalihan ja anniksi annetun reiden lisäksi. Sen jälkeen nasiiri saakoon juoda viiniä.
\par 21 Tämä on laki nasiirista, joka tekee lupauksen, ja hänen uhrilahjastaan, jonka hän uhraa Herralle vihkiytymisensä tähden, sen lisäksi, mitä hän muuten saa hankituksi. Tekemänsä lupauksen mukaan hän menetelköön näin, noudattaen vihkiytymistään koskevaa lakia."
\par 22 Ja Herra puhui Moosekselle sanoen:
\par 23 "Puhu Aaronille ja hänen pojillensa ja sano: Siunatessanne israelilaisia sanokaa heille:
\par 24 Herra siunatkoon sinua ja varjelkoon sinua;
\par 25 Herra valistakoon kasvonsa sinulle ja olkoon sinulle armollinen;
\par 26 Herra kääntäköön kasvonsa sinun puoleesi ja antakoon sinulle rauhan.
\par 27 Näin he laskekoot minun nimeni israelilaisten ylitse, ja minä siunaan heitä."

\chapter{7}

\par 1 Kun nyt Mooses oli saanut pystytetyksi asumuksen ja oli voidellut sen ja pyhittänyt sen kaikkine kalustoineen ynnä alttarin kaikkine kalustoineen sekä voidellut ja pyhittänyt ne,
\par 2 niin Israelin ruhtinaat, perhekuntien päämiehet, nimittäin heimoruhtinaat, katselmuksessa olleiden esimiehet,
\par 3 toivat lahjanansa Herran eteen kuudet katetut vaunut ja kaksitoista raavasta, kaksi ruhtinasta aina yhdet vaunut ja kukin ruhtinas härän; ne he toivat asumuksen eteen.
\par 4 Ja Herra sanoi Moosekselle näin:
\par 5 "Ota ne heiltä ilmestysmajan palveluksessa käytettäviksi ja anna ne leeviläisille, kullekin hänen palvelustehtävänsä mukaisesti".
\par 6 Ja Mooses otti vastaan vaunut ja raavaat ja antoi ne leeviläisille.
\par 7 Kahdet vaunut ja neljä raavasta hän antoi Geersonin pojille heidän palvelustehtävänsä mukaisesti.
\par 8 Ja neljät vaunut sekä kahdeksan raavasta hän antoi Merarin pojille sen palvelustehtävän mukaisesti, joka heidän oli suoritettava Iitamarin, pappi Aaronin pojan, johdolla.
\par 9 Mutta Kehatin pojille hän ei antanut mitään, koska heidän hoidettavinaan oli pyhät esineet, joita heidän oli olallaan kannettava.
\par 10 Ja ruhtinaat toivat vihkiäislahjoja alttaria varten sinä päivänä, jona se voideltiin; ja ruhtinaat toivat lahjansa alttarin eteen.
\par 11 Ja Herra sanoi Moosekselle: "Ruhtinaat tuokoot yksitellen, kukin päivänänsä, uhrilahjansa alttarin vihkiäisiin".
\par 12 Ensimmäisenä päivänä toi uhrilahjansa Nahson, Amminadabin poika, Juudan sukukunnasta.
\par 13 Ja hänen uhrilahjanansa oli hopeavati, sadan kolmenkymmenen sekelin painoinen, ja hopeamalja, joka painoi seitsemänkymmentä sekeliä pyhäkkösekelin painon mukaan, molemmat täynnä öljyllä sekoitettuja lestyjä jauhoja ruokauhriksi,
\par 14 kymmenen sekelin painoinen kultakuppi, täynnä suitsuketta,
\par 15 mullikka, oinas ja vuoden vanha karitsa polttouhriksi,
\par 16 kauris syntiuhriksi
\par 17 ja yhteysuhriksi kaksi raavasta, viisi oinasta, viisi kaurista ja viisi vuoden vanhaa karitsaa. Tämä oli Nahsonin, Amminadabin pojan, uhrilahja.
\par 18 Toisena päivänä Netanel, Suuarin poika, Isaskarin ruhtinas, toi uhrilahjansa.
\par 19 Hän toi uhrilahjanansa hopeavadin, sadan kolmenkymmenen sekelin painoisen, hopeamaljan, joka painoi seitsemänkymmentä sekeliä pyhäkkösekelin painon mukaan, molemmat täynnä öljyllä sekoitettuja lestyjä jauhoja ruokauhriksi,
\par 20 kymmenen sekelin painoisen kultakupin, täynnä suitsuketta,
\par 21 mullikan, oinaan ja vuoden vanhan karitsan polttouhriksi,
\par 22 kauriin syntiuhriksi
\par 23 ja yhteysuhriksi kaksi raavasta, viisi oinasta, viisi kaurista ja viisi vuoden vanhaa karitsaa. Tämä oli Netanelin, Suuarin pojan, uhrilahja.
\par 24 Kolmantena päivänä Sebulonin jälkeläisten ruhtinas Eliab, Heelonin poika:
\par 25 hänen uhrilahjanansa oli hopeavati, sadan kolmenkymmenen sekelin painoinen, hopeamalja, joka painoi seitsemänkymmentä sekeliä pyhäkkösekelin painon mukaan, molemmat täynnä öljyllä sekoitettuja lestyjä jauhoja ruokauhriksi,
\par 26 kymmenen sekelin painoinen kultakuppi, täynnä suitsuketta,
\par 27 mullikka, oinas ja vuoden vanha karitsa polttouhriksi,
\par 28 kauris syntiuhriksi
\par 29 ja yhteysuhriksi kaksi raavasta, viisi oinasta, viisi kaurista ja viisi vuoden vanhaa karitsaa. Tämä oli Eliabin, Heelonin pojan, uhrilahja.
\par 30 Neljäntenä päivänä Ruubenin jälkeläisten ruhtinas Elisur, Sedeurin poika:
\par 31 hänen uhrilahjanansa oli hopeavati, sadan kolmenkymmenen sekelin painoinen, hopeamalja, joka painoi seitsemänkymmentä sekeliä pyhäkkösekelin painon mukaan, molemmat täynnä öljyllä sekoitettuja lestyjä jauhoja ruokauhriksi,
\par 32 kymmenen sekelin painoinen kultakuppi, täynnä suitsuketta,
\par 33 mullikka, oinas ja vuoden vanha karitsa polttouhriksi,
\par 34 kauris syntiuhriksi
\par 35 ja yhteysuhriksi kaksi raavasta, viisi oinasta, viisi kaurista ja viisi vuoden vanhaa karitsaa. Tämä oli Elisurin, Sedeurin pojan, uhrilahja.
\par 36 Viidentenä päivänä Simeonin jälkeläisten ruhtinas Selumiel, Suurisaddain poika:
\par 37 hänen uhrilahjanansa oli hopeavati, sadan kolmenkymmenen sekelin painoinen, hopeamalja, joka painoi seitsemänkymmentä sekeliä pyhäkkösekelin painon mukaan, molemmat täynnä öljyllä sekoitettuja lestyjä jauhoja ruokauhriksi,
\par 38 kymmenen sekelin painoinen kultakuppi, täynnä suitsuketta,
\par 39 mullikka, oinas ja vuoden vanha karitsa polttouhriksi,
\par 40 kauris syntiuhriksi
\par 41 ja yhteysuhriksi kaksi raavasta, viisi oinasta, viisi kaurista ja viisi vuoden vanhaa karitsaa. Tämä oli Selumielin, Suurisaddain pojan, uhrilahja.
\par 42 Kuudentena päivänä Gaadin jälkeläisten ruhtinas Eljasaf, Deguelin poika:
\par 43 hänen uhrilahjanansa oli hopeavati, sadan kolmenkymmenen sekelin painoinen, hopeamalja, joka painoi seitsemänkymmentä sekeliä pyhäkkösekelin painon mukaan, molemmat täynnä öljyllä sekoitettuja lestyjä jauhoja ruokauhriksi,
\par 44 kymmenen sekelin painoinen kultakuppi, täynnä suitsuketta,
\par 45 mullikka, oinas ja vuoden vanha karitsa polttouhriksi,
\par 46 kauris syntiuhriksi
\par 47 ja yhteysuhriksi kaksi raavasta, viisi oinasta, viisi kaurista ja viisi vuoden vanhaa karitsaa. Tämä oli Eljasafin, Deguelin pojan, uhrilahja.
\par 48 Seitsemäntenä päivänä Efraimin jälkeläisten ruhtinas Elisama, Ammihudin poika:
\par 49 hänen uhrilahjanansa oli hopeavati, sadan kolmenkymmenen sekelin painoinen, hopeamalja, joka painoi seitsemänkymmentä sekeliä pyhäkkösekelin painon mukaan, molemmat täynnä öljyllä sekoitettuja lestyjä jauhoja ruokauhriksi,
\par 50 kymmenen sekelin painoinen kultakuppi, täynnä suitsuketta,
\par 51 mullikka, oinas ja vuoden vanha karitsa polttouhriksi,
\par 52 kauris syntiuhriksi
\par 53 ja yhteysuhriksi kaksi raavasta, viisi oinasta, viisi kaurista ja viisi vuoden vanhaa karitsaa. Tämä oli Elisaman, Ammihudin pojan, uhrilahja.
\par 54 Kahdeksantena päivänä Manassen jälkeläisten ruhtinas Gamliel, Pedasurin poika:
\par 55 hänen uhrilahjanansa oli hopeavati, sadan kolmenkymmenen sekelin painoinen, hopeamalja, joka painoi seitsemänkymmentä sekeliä pyhäkkösekelin painon mukaan, molemmat täynnä öljyllä sekoitettuja lestyjä jauhoja ruokauhriksi,
\par 56 kymmenen sekelin painoinen kultakuppi, täynnä suitsuketta,
\par 57 mullikka, oinas ja vuoden vanha karitsa polttouhriksi,
\par 58 kauris syntiuhriksi
\par 59 ja yhteysuhriksi kaksi raavasta, viisi oinasta, viisi kaurista ja viisi vuoden vanhaa karitsaa. Tämä oli Gamlielin, Pedasurin pojan, uhrilahja.
\par 60 Yhdeksäntenä päivänä Benjaminin jälkeläisten ruhtinas Abidan, Gideonin poika:
\par 61 hänen uhrilahjanansa oli hopeavati, sadan kolmenkymmenen sekelin painoinen, hopeamalja, joka painoi seitsemänkymmentä sekeliä pyhäkkösekelin painon mukaan, molemmat täynnä öljyllä sekoitettuja lestyjä jauhoja ruokauhriksi,
\par 62 kymmenen sekelin painoinen kultakuppi, täynnä suitsuketta,
\par 63 mullikka, oinas ja vuoden vanha karitsa polttouhriksi,
\par 64 kauris syntiuhriksi
\par 65 ja yhteysuhriksi kaksi raavasta, viisi oinasta, viisi kaurista ja viisi vuoden vanhaa karitsaa. Tämä oli Abidanin, Gideonin pojan, uhrilahja.
\par 66 Kymmenentenä päivänä Daanin jälkeläisten ruhtinas Ahieser, Ammisaddain poika:
\par 67 hänen uhrilahjanansa oli hopeavati, sadan kolmenkymmenen sekelin painoinen, hopeamalja, joka painoi seitsemänkymmentä sekeliä pyhäkkösekelin painon mukaan, molemmat täynnä öljyllä sekoitettuja lestyjä jauhoja ruokauhriksi,
\par 68 kymmenen sekelin painoinen kultakuppi, täynnä suitsuketta,
\par 69 mullikka, oinas ja vuoden vanha karitsa polttouhriksi,
\par 70 kauris syntiuhriksi
\par 71 ja yhteysuhriksi kaksi raavasta, viisi oinasta, viisi kaurista ja viisi vuoden vanhaa karitsaa. Tämä oli Ahieserin, Ammisaddain pojan, uhrilahja.
\par 72 Yhdentenätoista päivänä Asserin jälkeläisten ruhtinas Pagiel, Okranin poika:
\par 73 hänen uhrilahjanansa oli hopeavati, sadan kolmenkymmenen sekelin painoinen, hopeamalja, joka painoi seitsemänkymmentä sekeliä pyhäkkösekelin painon mukaan, molemmat täynnä öljyllä sekoitettuja lestyjä jauhoja ruokauhriksi,
\par 74 kymmenen sekelin painoinen kultakuppi, täynnä suitsuketta,
\par 75 mullikka, oinas ja vuoden vanha karitsa polttouhriksi,
\par 76 kauris syntiuhriksi
\par 77 ja yhteysuhriksi kaksi raavasta, viisi oinasta, viisi kaurista ja viisi vuoden vanhaa karitsaa. Tämä oli Pagielin, Okranin pojan, uhrilahja.
\par 78 Kahdentenatoista päivänä Naftalin jälkeläisten ruhtinas Ahira, Eenanin poika:
\par 79 hänen uhrilahjanansa oli hopeavati, sadan kolmenkymmenen sekelin painoinen, hopeamalja, joka painoi seitsemänkymmentä sekeliä pyhäkkösekelin painon mukaan, molemmat täynnä öljyllä sekoitettuja lestyjä jauhoja ruokauhriksi,
\par 80 kymmenen sekelin painoinen kultakuppi, täynnä suitsuketta,
\par 81 mullikka, oinas ja vuoden vanha karitsa polttouhriksi,
\par 82 kauris syntiuhriksi
\par 83 ja yhteysuhriksi kaksi raavasta, viisi oinasta, viisi kaurista ja viisi vuoden vanhaa karitsaa. Tämä oli Ahiran, Eenanin pojan, uhrilahja.
\par 84 Nämä olivat ne vihkiäislahjat, jotka Israelin ruhtinaat antoivat alttaria varten sinä päivänä, jona se voideltiin: kaksitoista hopeavatia, kaksitoista maljaa, kaksitoista kultakuppia,
\par 85 kukin hopeavati sadan kolmenkymmenen sekelin ja kukin malja seitsemänkymmenen sekelin painoinen; näissä astioissa oli siis hopeata yhteensä kaksituhatta neljäsataa sekeliä pyhäkkösekelin painon mukaan.
\par 86 Kultakuppeja oli kaksitoista, täynnä suitsuketta, kukin kuppi kymmenen sekelin painoinen pyhäkkösekelin painon mukaan; kupeissa oli siis kultaa yhteensä sata kaksikymmentä sekeliä.
\par 87 Polttouhrieläimiä oli yhteensä: kaksitoista härkää, kaksitoista oinasta, kaksitoista vuoden vanhaa karitsaa ynnä niihin kuuluva ruokauhri; ja syntiuhrikauriita oli kaksitoista.
\par 88 Yhteysuhrieläimiä oli yhteensä: kaksikymmentä neljä härkää, kuusikymmentä oinasta, kuusikymmentä kaurista ja kuusikymmentä vuoden vanhaa karitsaa. Nämä olivat ne vihkiäislahjat, jotka annettiin alttaria varten, sen jälkeen kuin se oli voideltu.
\par 89 Ja kun Mooses meni ilmestysmajaan puhumaan Herran kanssa, kuuli hän äänen, joka puhutteli häntä lain arkin päällä olevalta armoistuimelta, molempien kerubien väliltä; ja Herra puhui hänelle.

\chapter{8}

\par 1 Ja Herra puhui Moosekselle sanoen:
\par 2 "Puhu Aaronille ja sano hänelle: Kun nostat lamput paikoilleen, niin valaiskoot ne seitsemän lamppua seitsenhaaraisen lampun edustaa".
\par 3 Ja Aaron teki niin; hän nosti lamput valaisemaan seitsenhaaraisen lampun edustaa, niinkuin Herra oli Moosekselle käskyn antanut.
\par 4 Ja näin oli seitsenhaarainen lamppu tehty: se oli kullasta, kohotakoista tekoa; sen jalka ja kukkalehdetkin olivat kohotakoista tekoa. Aivan sen kaavan mukaan, jonka Herra oli Moosekselle näyttänyt, hän teki seitsenhaaraisen lampun.
\par 5 Ja Herra puhui Moosekselle sanoen:
\par 6 "Ota leeviläiset israelilaisten keskuudesta ja puhdista heidät.
\par 7 Ja puhdistaaksesi heidät tee näin: Pirskoita heihin synnistä puhdistavaa vettä, ja ajattakoot he koko ruumiinsa partaveitsellä ja peskööt vaatteensa ja puhdistakoot itsensä.
\par 8 Ja sitten he ottakoot mullikan sekä siihen kuuluvana ruokauhrina öljyllä sekoitettuja lestyjä jauhoja; ja ota toinen mullikka syntiuhriksi.
\par 9 Tuo sitten leeviläiset ilmestysmajan eteen sekä kokoa kaikki Israelin kansa.
\par 10 Ja kun olet tuonut leeviläiset Herran eteen, laskekoot israelilaiset kätensä leeviläisten päälle,
\par 11 ja Aaron toimittakoon leeviläisille israelilaisten puolesta heilutuksen Herran edessä, että he kelpaisivat Herran palvelukseen.
\par 12 Mutta leeviläiset laskekoot kätensä mullikkain pään päälle, ja valmista toinen syntiuhriksi ja toinen polttouhriksi Herralle, toimittaaksesi sovituksen leeviläisille.
\par 13 Aseta sitten leeviläiset Aaronin ja hänen poikiensa eteen ja toimita heille heilutus Herran edessä.
\par 14 Erota leeviläiset israelilaisten keskuudesta, että leeviläiset olisivat minun omani.
\par 15 Ja senjälkeen leeviläiset menkööt palvelemaan ilmestysmajaa, kun sinä olet puhdistanut heidät ja toimittanut heille heilutuksen.
\par 16 Sillä he ovat kokonaan annetut minun omikseni israelilaisten keskuudesta; kaiken sen sijaan, joka avaa äidinkohdun, israelilaisten kaikkien esikoisten sijaan, minä olen heidät ottanut omikseni.
\par 17 Sillä jokainen esikoinen israelilaisten keskuudessa, niin ihmisten kuin karjankin, on minun omani; sinä päivänä, jona minä surmasin kaikki Egyptin maan esikoiset, minä pyhitin heidät itselleni.
\par 18 Ja minä otin leeviläiset israelilaisten kaikkien esikoisten sijaan.
\par 19 Ja minä annoin leeviläiset Aaronille ja hänen pojillensa palvelijoiksi, israelilaisten keskuudesta, suorittamaan israelilaisten puolesta palvelusta ilmestysmajassa ja toimittamaan israelilaisille sovitusta; ja niin israelilaisia ei kohtaa rangaistus siitä, että lähestyvät pyhäkköä.
\par 20 Mooses ja Aaron ja koko Israelin kansa tekivät leeviläisille kaiken, niinkuin Herra oli Moosekselle leeviläisistä käskyn antanut; niin israelilaiset heille tekivät.
\par 21 Ja leeviläiset puhdistautuivat synnistä ja pesivät vaatteensa. Ja Aaron toimitti heille heilutuksen Herran edessä, ja Aaron toimitti heille sovituksen heidän puhdistamiseksensa.
\par 22 Ja senjälkeen leeviläiset menivät toimittamaan palvelustansa ilmestysmajassa Aaronin ja hänen poikiensa edessä. Niinkuin Herra oli Moosekselle leeviläisistä käskyn antanut, niin heille tehtiin.
\par 23 Ja Herra puhui Moosekselle sanoen:
\par 24 "Tämä olkoon voimassa leeviläisistä: Kahdenkymmenen viiden vuoden ikäiset ja sitä vanhemmat menkööt palvelemaan ja toimittamaan palvelusta ilmestysmajassa.
\par 25 Mutta viidennestäkymmenennestä ikävuodesta alkaen leeviläinen olkoon palveluksesta vapaa älköönkä enää palvelko,
\par 26 vaan auttakoon veljiään ilmestysmajassa heidän tehtäviensä hoitamisessa, mutta palvelusta hän älköön enää toimittako. Järjestä näin leeviläisille heidän tehtävänsä."

\chapter{9}

\par 1 Ja Herra puhui Moosekselle Siinain erämaassa toisena vuotena siitä, kun he olivat lähteneet Egyptin maasta, sen vuoden ensimmäisessä kuussa, sanoen:
\par 2 "Israelilaiset viettäkööt pääsiäisen määräaikanansa.
\par 3 Tämän kuukauden neljäntenätoista päivänä, iltahämärässä, viettäkää se määräaikanansa; kaikkien sitä koskevien käskyjen ja säädösten mukaan se viettäkää."
\par 4 Ja Mooses käski israelilaisten viettää pääsiäistä.
\par 5 Niin he viettivät pääsiäisen ensimmäisessä kuussa, kuukauden neljäntenätoista päivänä, iltahämärässä, Siinain erämaassa; israelilaiset tekivät kaiken, niinkuin Herra oli Moosekselle käskyn antanut.
\par 6 Mutta siellä oli miehiä, jotka olivat saastuneet kuolleesta eivätkä voineet viettää pääsiäistä sinä päivänä. Niin he astuivat sinä päivänä Mooseksen ja Aaronin eteen.
\par 7 Ja miehet sanoivat hänelle: "Me olemme saastuneet kuolleesta; miksi meidät estetään tuomasta Herralle uhrilahjaa määräaikanansa israelilaisten keskuudessa?"
\par 8 Ja Mooses sanoi heille: "Jääkää tähän, minä menen kuulemaan, mitä Herra teistä säätää".
\par 9 Ja Herra puhui Moosekselle sanoen:
\par 10 "Puhu israelilaisille näin: Jos joku teistä tai teidän jälkeläisistänne on saastunut kuolleesta tai on kaukana matkalla, viettäköön kuitenkin pääsiäistä Herran kunniaksi;
\par 11 viettäkööt sitä toisen kuukauden neljäntenätoista päivänä, iltahämärässä. Happamattoman leivän ja katkerain yrttien kanssa he syökööt pääsiäislampaan.
\par 12 Älkööt he jättäkö siitä mitään seuraavaan aamuun älköötkä siitä luuta rikkoko. Kaikkien pääsiäistä koskevien käskyjen mukaan he viettäkööt sitä.
\par 13 Mutta se mies, joka on puhdas eikä ole matkoilla ja kuitenkin jättää pääsiäisen viettämättä, hävitettäköön kansastansa, koska hän ei tuonut uhrilahjaa Herralle määräaikanansa; se mies joutuu syynalaiseksi.
\par 14 Ja jos muukalainen asuu teidän luonanne ja tahtoo viettää pääsiäistä Herran kunniaksi, viettäköön sen pääsiäistä koskevien käskyjen ja säädösten mukaan; samat käskyt olkoot teillä, niin muukalaisella kuin maassa syntyneelläkin."
\par 15 Mutta sinä päivänä, jona asumus pystytettiin, peitti pilvi asumuksen, lain majan, ja illalla näkyi asumuksen päällä niinkuin tulenhohde, ja aina aamuun asti.
\par 16 Niin oli aina: pilvi peitti sen päivällä ja tulenhohde yöllä.
\par 17 Ja niin usein kuin pilvi kohosi majan päältä, lähtivät israelilaiset liikkeelle; ja mihin pilvi laskeutui, siihen israelilaiset leiriytyivät.
\par 18 Herran käskyn mukaan israelilaiset lähtivät liikkeelle, ja Herran käskyn mukaan israelilaiset leiriytyivät. He olivat leiriytyneinä, niin kauan kuin pilvi pysyi asumuksen päällä.
\par 19 Ja kun pilvi viipyi asumuksen päällä useampia päiviä, noudattivat israelilaiset, mitä Herra oli käskenyt heidän noudattaa, eivätkä lähteneet liikkeelle.
\par 20 Mutta joskus pilvi viipyi asumuksen päällä vain muutamia päiviä; silloinkin he Herran käskyn mukaan olivat leiriytyneinä ja lähtivät Herran käskyn mukaan liikkeelle.
\par 21 Ja joskus pilvi pysyi paikallaan vain illasta aamuun ja kohosi aamulla; silloinkin he lähtivät liikkeelle. Tahi jos pilvi pysyi päivän ja yön ja sitten kohosi, niin he lähtivät liikkeelle.
\par 22 Tahi kun pilvi viipyi pysyen asumuksen päällä pari päivää tai kuukauden tai vielä pitemmän ajan, niin olivat israelilaiset leiriytyneinä eivätkä lähteneet liikkeelle; mutta kun se kohosi, lähtivät he liikkeelle.
\par 23 Herran käskyn mukaan he leiriytyivät ja Herran käskyn mukaan he lähtivät liikkeelle. He noudattivat, mitä Herra Mooseksen kautta oli käskenyt heidän noudattaa.

\chapter{10}

\par 1 Ja Herra puhui Moosekselle sanoen:
\par 2 "Tee itsellesi kaksi hopeatorvea, tee ne kohotakoista tekoa, käyttääksesi niitä kansan kokoonkutsumiseen ja leirien liikkeelle-panemiseen.
\par 3 Kun niihin molempiin puhalletaan, kokoontukoon sinun luoksesi koko kansa ilmestysmajan oven eteen.
\par 4 Mutta kun ainoastaan toiseen puhalletaan, kokoontukoot sinun luoksesi ruhtinaat, Israelin heimojen päämiehet.
\par 5 Mutta kun puhallatte hälytyssoiton, silloin lähtekööt itään päin leiriytyneet leirit liikkeelle.
\par 6 Ja kun toistamiseen puhallatte hälytyssoiton, lähtekööt etelään päin leiriytyneet leirit liikkeelle. Hälytyssoitto puhallettakoon liikkeelle lähdettäessä.
\par 7 Mutta seurakuntaa kokoonkutsuttaessa puhaltakaa torveen, hälytyssoittoa soittamatta.
\par 8 Ja Aaronin pojat, papit, puhaltakoot näitä torvia. Tämä olkoon teille ikuinen säädös sukupolvesta sukupolveen.
\par 9 Ja kun te maassanne lähdette taisteluun vihollista vastaan, joka teitä ahdistaa, niin soittakaa torvilla hälytys; silloin te muistutte Herran, teidän Jumalanne, mieleen ja pelastutte vihollisistanne.
\par 10 Ja ilopäivinänne sekä juhlinanne ynnä uudenkuun päivinä puhaltakaa torviin uhratessanne poltto- ja yhteysuhrejanne, niin ne saattavat teidän Jumalanne muistamaan teitä. Minä olen Herra, teidän Jumalanne."
\par 11 Toisen vuoden toisen kuukauden kahdentenakymmenentenä päivänä pilvi kohosi lain asumuksen päältä.
\par 12 Silloin israelilaiset lähtivät liikkeelle, leiri leirin jälkeen, Siinain erämaasta; ja pilvi pysähtyi Paaranin erämaassa.
\par 13 Niin he ensi kerran lähtivät liikkeelle sillä tavoin, kuin Herra oli Mooseksen kautta käskenyt.
\par 14 Juudan jälkeläisten leirin lippukunta lähti ensimmäisenä joukkona liikkeelle osastoittain, ja tätä osastoa johti Nahson, Amminadabin poika.
\par 15 Ja Isaskarin jälkeläisten sukukuntaosastoa johti Netanel, Suuarin poika.
\par 16 Ja Sebulonin jälkeläisten sukukuntaosastoa johti Eliab, Heelonin poika.
\par 17 Ja kun asumus oli purettu, niin lähtivät liikkeelle Geersonin pojat ja Merarin pojat, jotka kantoivat asumusta.
\par 18 Sitten lähti Ruubenin leirin lippukunta liikkeelle osastoittain, ja tätä osastoa johti Elisur, Sedeurin poika.
\par 19 Ja Simeonin jälkeläisten sukukuntaosastoa johti Selumiel, Suurisaddain poika.
\par 20 Ja Gaadin jälkeläisten sukukuntaosastoa johti Eljasaf, Deguelin poika.
\par 21 Sitten lähtivät kehatilaiset liikkeelle kantaen pyhiä esineitä. Ja asumus pantiin pystyyn, ennenkuin he saapuivat perille.
\par 22 Sitten lähti Efraimin jälkeläisten leirin lippukunta liikkeelle osastoittain, ja tätä osastoa johti Elisama, Ammihudin poika.
\par 23 Manassen jälkeläisten sukukuntaosastoa johti Gamliel, Pedasurin poika.
\par 24 Benjaminin jälkeläisten sukukuntaosastoa johti Abidan, Gideonin poika.
\par 25 Sitten lähti Daanin jälkeläisten leirin lippukunta liikkeelle kaikkien leirien jälkijoukkona, osastoittain, ja tätä osastoa johti Ahieser, Ammisaddain poika.
\par 26 Asserin jälkeläisten sukukuntaosastoa johti Pagiel, Okranin poika.
\par 27 Ja Naftalin jälkeläisten sukukuntaosastoa johti Ahira, Eenanin poika.
\par 28 Tällainen oli israelilaisten liikkeellelähtö osastoittain; niin he lähtivät liikkeelle.
\par 29 Ja Mooses puhui Hoobabille, midianilaisen Reguelin, Mooseksen apen, pojalle: "Me lähdemme siihen paikkaan, josta Herra on sanonut: 'Sen minä annan teille'. Lähde meidän kanssamme, niin me palkitsemme sinut hyvin, sillä Herra on luvannut hyvää Israelille."
\par 30 Mutta hän vastasi hänelle: "En lähde, vaan minä menen omaan maahani ja sukuni luo".
\par 31 Niin Mooses sanoi: "Älä jätä meitä, sillä sinähän tiedät, mihin meidän sopii leiriytyä erämaassa; rupea siis meille oppaaksi.
\par 32 Jos lähdet meidän kanssamme, niin me annamme sinun saada osasi siitä hyvästä, minkä Herra meille suo."
\par 33 Niin he lähtivät liikkeelle Herran vuoren luota ja vaelsivat kolme päivänmatkaa. Ja Herran liitonarkki kulki heidän edellänsä kolme päivänmatkaa etsimässä heille levähdyspaikkaa.
\par 34 Ja Herran pilvi oli heidän päällänsä päivällä, kun he lähtivät leiristä liikkeelle.
\par 35 Ja kun arkki lähti liikkeelle, niin Mooses lausui: "Nouse, Herra, hajaantukoot sinun vihollisesi, ja sinun vihamiehesi paetkoot sinua".
\par 36 Ja kun arkki laskettiin maahan, lausui hän: "Palaja, Herra, Israelin heimojen kymmentuhansien tykö".

\chapter{11}

\par 1 Mutta kansa tuskitteli, ja se oli paha Herran korvissa. Kun Herra sen kuuli, vihastui hän, ja Herran tuli syttyi heidän keskellään, ja se kulutti ulommaisen osan leiriä.
\par 2 Silloin kansa huusi Moosesta, ja Mooses rukoili Herraa; niin tuli alkoi sammua.
\par 3 Ja sen paikan nimeksi pantiin Tabeera, koska Herran tuli oli siellä syttynyt heidän keskellään.
\par 4 Mutta heidän keskuuteensa kerääntyneessä hylkyväessä heräsivät himot, ja niin israelilaisetkin rupesivat jälleen itkemään, sanoen: "Voi jospa meillä olisi lihaa syödäksemme!
\par 5 Me muistelemme kaloja, joita söimme Egyptissä ilmaiseksi, kurkkuja, melooneja, ruoholaukkaa, sipulia ja kynsilaukkaa.
\par 6 Mutta nyt me näännymme, sillä eihän täällä ole mitään; emme saa nähdäkään muuta kuin tuota mannaa."
\par 7 Ja manna oli korianderin siemenen kaltaista ja bedellion-pihkan näköistä.
\par 8 Kansa samoili sitä kokoamassa ja jauhoi sitä käsikivillä tai survoi huhmaressa; sitten he keittivät sitä padassa ja valmistivat siitä kaltiaisia. Ja sen maku oli samanlainen kuin öljyleivoksen.
\par 9 Ja kasteen laskeutuessa yöllä leiriin laskeutui mannakin siihen.
\par 10 Ja Mooses kuuli, kuinka kansa, joka suku, itki, kukin majansa ovella; niin Herra vihastui kovin, ja myös Mooseksen silmissä se oli paha.
\par 11 Ja Mooses sanoi Herralle: "Miksi olet tehnyt niin pahoin palvelijallesi, ja miksi en ole saanut armoa sinun silmiesi edessä, koska olet pannut kaiken tämän kansan minun taakakseni?
\par 12 Olenko minä kaiken tämän kansan äiti tai isä, koska käsket minua kantamaan sitä sylissäni, niinkuin hoitaja kantaa imeväistä lasta, siihen maahan, jonka olet valalla vannoen luvannut heidän isillensä?
\par 13 Mistä minulla on lihaa antaa kaikelle tälle kansalle, kun he ahdistavat minua itkullaan sanoen: 'Anna meille lihaa syödäksemme'?
\par 14 En minä jaksa yksinäni kantaa koko tätä kansaa, sillä se on minulle liian raskas.
\par 15 Ja jos näin aiot kohdella minua, niin surmaa minut mieluummin, jos olen saanut armon sinun silmiesi edessä, ja päästä minut näkemästä tätä kurjuutta."
\par 16 Silloin Herra puhui Moosekselle: "Kutsu kokoon minua varten seitsemänkymmentä miestä Israelin vanhimpia, jotka tiedät kansan vanhimmiksi ja päällysmiehiksi; tuo heidät ilmestysmajalle, ja he asettukoot sinne sinun kanssasi.
\par 17 Minä sitten astun alas ja puhun siellä sinun kanssasi ja otan henkeä, joka sinussa on, ja annan heille; siten he voivat auttaa sinua tuon kansataakan kantamisessa, ettei sinun tarvitse sitä yksinäsi kantaa.
\par 18 Mutta kansalle sano: Pyhittäytykää huomiseksi, niin saatte syödä lihaa, koska itkitte Herran kuullen, sanoen: 'Voi, jospa meillä olisi lihaa syödäksemme! Egyptissä meidän oli hyvä olla.' Herra antaa nyt teille lihaa syödäksenne.
\par 19 Eikä siinä ole teille syömistä ainoastaan päiväksi tai kahdeksi, tai viideksi tai kymmeneksi tai kahdeksikymmeneksi päiväksi,
\par 20 vaan kuukauden päiviksi, kunnes ette enää siedä sen hajua ja se iljettää teitä. Sillä te olette pitäneet halpana Herran, joka on teidän keskellänne, ja itkeneet hänen edessänsä sanoen: 'Miksi lähdimmekään Egyptistä!"
\par 21 Mooses vastasi: "Kuusisataa tuhatta jalkamiestä on sitä kansaa, jonka keskuudessa minä elän, ja kuitenkin sinä sanot: 'Minä annan heille lihaa, niin että heillä on sitä syödä kuukauden päivät'.
\par 22 Onko lampaita ja raavaita teurastettaviksi heille, niin että heille riittää, vai kootaanko heille kaikki meren kalat, niin että heille riittää?"
\par 23 Mutta Herra sanoi Moosekselle: "Onko Herran käsi lyhyt? Nyt saat nähdä, toteutuuko sinulle minun sanani vai eikö."
\par 24 Niin Mooses meni ulos ja kertoi kansalle Herran sanat. Sitten hän kutsui kokoon seitsemänkymmentä miestä kansan vanhimpia ja asetti heidät majan ympärille.
\par 25 Silloin Herra astui alas pilvessä ja puhutteli häntä, otti henkeä, joka hänessä oli, ja antoi sitä niille seitsemällekymmenelle vanhimmalle. Kun henki laskeutui heihin, niin he joutuivat hurmoksiin, mutta sen jälkeen he eivät enää siihen joutuneet.
\par 26 Mutta leiriin oli heitä jäänyt kaksi miestä, toisen nimi oli Eldad ja toisen nimi Meedad. Heihinkin henki laskeutui, sillä hekin olivat luetteloon merkityt, vaikka eivät olleet lähteneet majalle; nämä joutuivat leirissä hurmoksiin.
\par 27 Silloin riensi muuan nuorukainen ja ilmoitti Moosekselle sanoen: "Eldad ja Meedad ovat joutuneet hurmoksiin leirissä".
\par 28 Mutta Joosua, Nuunin poika, joka oli ollut Mooseksen palvelija nuoruudestaan asti, puuttui puheeseen sanoen: "Oi, herrani Mooses, kiellä heitä!"
\par 29 Mutta Mooses vastasi hänelle: "Oletko kateellinen minun puolestani? Oi, jospa koko Herran kansa olisi profeettoja, niin että Herra antaisi henkensä heihin!"
\par 30 Sitten Mooses siirtyi takaisin leiriin ja hänen kanssaan Israelin vanhimmat.
\par 31 Mutta Herran lähettämä tuuli nousi ja ajoi edellään viiriäisiä mereltä päin ja painoi niitä leiriin päivänmatkan laajuudelta joka suunnalle leiristä, lähes kahden kyynärän korkeuteen maasta.
\par 32 Silloin kansa ryhtyi kokoamaan viiriäisiä ja kokosi niitä koko sen päivän ja koko yön ja koko seuraavan päivän. Vähin määrä, jonka joku sai kootuksi, oli kymmenen hoomer-mittaa. Ja he levittivät niitä pitkin leiriä kuivamaan.
\par 33 Mutta lihan vielä ollessa heidän hampaissaan, ennenkuin se oli syöty loppuun, syttyi Herran viha kansaa kohtaan, ja Herra tuotti kansalle hyvin suuren surman.
\par 34 Ja sen paikan nimeksi pantiin Kibrot-Hattaava, koska sinne haudattiin kansasta ne, jotka olivat antautuneet halunsa valtaan.
\par 35 Kibrot-Hattaavasta kansa lähti liikkeelle Haserotia kohti ja pysähtyi Haserotiin.

\chapter{12}

\par 1 Mutta Mirjam ja Aaron parjasivat Moosesta etiopialaisen naisen tähden, jonka hän oli ottanut vaimokseen; sillä hän oli ottanut vaimokseen etiopialaisen naisen.
\par 2 Ja he sanoivat: "Ainoastaan Mooseksen kauttako Herra puhuu? Eikö hän puhu myös meidän kauttamme?" Ja Herra kuuli sen.
\par 3 Mutta Mooses oli sangen nöyrä mies, nöyrempi kuin kukaan muu ihminen maan päällä.
\par 4 Niin Herra sanoi äkisti Moosekselle ja Aaronille ja Mirjamille: "Menkää te kaikki kolme ilmestysmajalle". Ja he menivät kaikki kolme sinne.
\par 5 Silloin Herra astui alas pilvenpatsaassa, asettui majan ovelle ja kutsui Aaronia ja Mirjamia; molemmat menivät sinne.
\par 6 Ja Herra sanoi: "Kuulkaa minun sanani. Jos keskuudessanne on profeetta, niin minä ilmestyn hänelle näyssä, puhun hänen kanssaan unessa.
\par 7 Niin ei ole minun palvelijani Mooses, hän on uskollinen koko minun talossani;
\par 8 hänen kanssaan minä puhun suusta suuhun, avoimesti enkä peitetyin sanoin, ja hän saa katsella Herran muotoa. Miksi ette siis peljänneet puhua minun palvelijaani Moosesta vastaan?"
\par 9 Ja Herran viha syttyi heitä kohtaan, ja hän meni pois.
\par 10 Kun pilvi oli poistunut majan päältä, niin katso, Mirjam oli lumivalkea pitalista; ja Aaron kääntyi Mirjamiin päin, ja katso, tämä oli pitalinen.
\par 11 Silloin Aaron sanoi Moosekselle: "Oi, herrani! Älä pane meidän päällemme syntiä, jonka olemme tyhmyydessä tehneet.
\par 12 Älä anna hänen jäädä kuolleen sikiön kaltaiseksi, jonka ruumis on puoleksi mädännyt, kun se äitinsä kohdusta tulee."
\par 13 Silloin Mooses huusi Herran puoleen sanoen: "Oi, Jumala! Paranna hänet!"
\par 14 Herra vastasi Moosekselle: "Jos hänen isänsä olisi sylkenyt häntä silmille, eikö hänen olisi ollut hävettävä seitsemän päivää? Olkoon hän nyt suljettuna ulos leiristä seitsemän päivää, ja sitten hän pääsköön takaisin."
\par 15 Niin Mirjam oli suljettuna ulos leiristä seitsemän päivää, eikä kansa lähtenyt liikkeelle, ennenkuin Mirjam oli tuotu takaisin.

\chapter{13}

\par 1 Sen jälkeen kansa lähti liikkeelle Haserotista ja leiriytyi Paaranin erämaahan.
\par 2 Ja Herra puhui Moosekselle sanoen:
\par 3 Lähetä miehiä vakoilemaan Kanaanin maata, jonka minä annan israelilaisille; lähettäkää mies kustakin isien sukukunnasta, ainoastaan niitä, jotka ovat ruhtinaita heidän keskuudessaan.
\par 4 Niin Mooses lähetti heidät Paaranin erämaasta Herran käskyn mukaan; kaikki ne miehet olivat israelilaisten päämiehiä.
\par 5 Ja nämä olivat heidän nimensä: Ruubenin sukukunnasta Sammua, Sakkurin poika,
\par 6 Simeonin sukukunnasta Saafat, Hoorin poika,
\par 7 Juudan sukukunnasta Kaaleb, Jefunnen poika,
\par 8 Isaskarin sukukunnasta Jigal, Joosefin poika,
\par 9 Efraimin sukukunnasta Hoosea, Nuunin poika,
\par 10 Benjaminin sukukunnasta Palti, Raafun poika,
\par 11 Sebulonin sukukunnasta Gaddiel, Soodin poika,
\par 12 Manassen sukukunnasta Joosefin sukukuntaa Gaddi, Suusin poika,
\par 13 Daanin sukukunnasta Ammiel, Gemallin poika,
\par 14 Asserin sukukunnasta Setur, Miikaelin poika,
\par 15 Naftalin sukukunnasta Nahbi, Vofsin poika,
\par 16 Gaadin sukukunnasta Genuel, Maakin poika.
\par 17 Nämä olivat niiden miesten nimet, jotka Mooses lähetti maata vakoilemaan. Mutta Mooses kutsui Hoosean, Nuunin pojan, Joosuaksi.
\par 18 Ja lähettäessään heidät vakoilemaan Kanaanin maata Mooses sanoi heille: "Lähtekää nyt Etelämaahan ja nouskaa vuoristoon
\par 19 ja katselkaa, minkälainen maa on ja minkälainen kansa, joka siinä asuu, onko se voimakas vai heikko, onko sitä vähän vai paljon,
\par 20 ja minkälainen maa on, jossa se asuu, onko se hyvä vai huono, ja minkälaiset ne kaupungit ovat, joissa se asuu, avonaisia kyliäkö vai varustettuja kaupunkeja,
\par 21 ja minkälaista on maa, lihavaa vai laihaa, onko siinä puita vai ei. Ja olkaa rohkealla mielellä ja ottakaa mukaanne sen maan hedelmiä." Oli näet se aika, jolloin ensimmäiset rypäleet kypsyivät.
\par 22 Niin he lähtivät sinne ja vakoilivat maata Siinin erämaasta Rehobiin asti, josta mennään Hamatiin.
\par 23 Ja he lähtivät Etelämaahan ja tulivat Hebroniin, jossa asuivat Ahiman, Seesai ja Talmai, Anakin jälkeläiset. Mutta Hebron oli rakennettu seitsemän vuotta ennen Egyptin Sooania.
\par 24 Ja he tulivat Rypälelaaksoon; sieltä he leikkasivat viiniköynnöksen, jossa oli rypäleterttu, ja kahden miehen täytyi kantaa sitä korennolla; samoin he ottivat granaattiomenia ja viikunoita.
\par 25 Se paikka nimitettiin Rypälelaaksoksi, rypäleen tähden, jonka israelilaiset sieltä leikkasivat.
\par 26 Ja he palasivat maata vakoilemasta neljänkymmenen päivän kuluttua.
\par 27 He vaelsivat ja tulivat Mooseksen, Aaronin ja kaiken Israelin kansan luo Paaranin erämaahan Kaadekseen. Ja he tekivät heille ja kaikelle kansalle selkoa matkastaan ja näyttivät heille sen maan hedelmiä.
\par 28 Ja he kertoivat hänelle sanoen: "Me menimme siihen maahan, jonne meidät lähetit. Ja se tosiaankin vuotaa maitoa ja mettä, ja tällaisia ovat sen hedelmät.
\par 29 Mutta kansa, joka siinä maassa asuu, on voimallista, ja kaupungit ovat lujasti varustettuja ja hyvin suuria; näimmepä siellä Anakin jälkeläisiäkin.
\par 30 Amalekilaiset asuvat Etelämaassa ja heettiläiset, jebusilaiset ja amorilaiset asuvat vuoristossa, ja kanaanilaiset asuvat meren rannalla ja Jordanin varsilla."
\par 31 Ja Kaaleb koitti tyynnyttää kansaa napisemasta Moosesta vastaan ja sanoi: "Menkäämme sittenkin sinne ja ottakaamme se haltuumme, sillä varmasti me sen voitamme".
\par 32 Mutta ne miehet, jotka olivat käyneet hänen kanssaan siellä, sanoivat: "Emme me kykene käymään sen kansan kimppuun, sillä se on meitä voimakkaampi".
\par 33 Niin he saattoivat israelilaisten keskuudessa pahaan huutoon sen maan, jota olivat olleet vakoilemassa, sanoessaan: "Se maa, jota olemme kierrelleet ja vakoilleet, on maa, joka syö omat asukkaansa; ja kaikki kansa, jota me siellä näimme, oli kookasta väkeä.
\par 34 Näimmepä siellä jättiläisiäkin, Anakin jättiläisheimon jälkeläisiä, ja me olimme mielestämme kuin heinäsirkkoja; sellaisia me heistäkin olimme."

\chapter{14}

\par 1 Silloin koko kansa alkoi huutaa ja parkua, ja kansa itki sen yön.
\par 2 Ja kaikki israelilaiset napisivat Moosesta ja Aaronia vastaan, ja koko kansa sanoi heille: "Jospa olisimme kuolleet Egyptin maahan tai tähän erämaahan! Jospa olisimme kuolleet!
\par 3 Ja miksi viekään Herra meitä tuohon maahan, jossa me kaadumme miekkaan ja vaimomme ja lapsemme joutuvat vihollisen saaliiksi! Eikö meidän olisi parempi palata Egyptiin?"
\par 4 Ja he puhuivat toinen toiselleen: "Valitkaamme johtaja ja palatkaamme Egyptiin".
\par 5 Silloin Mooses ja Aaron lankesivat kasvoilleen Israelin kansan koko seurakunnan eteen.
\par 6 Mutta Joosua, Nuunin poika, ja Kaaleb, Jefunnen poika, jotka olivat olleet mukana maata vakoilemassa, repäisivät vaatteensa
\par 7 ja puhuivat kaikelle israelilaisten seurakunnalle näin: "Maa, jota kävimme vakoilemassa, on ylen ihana maa.
\par 8 Jos Herra on meille suosiollinen, niin hän vie meidät siihen maahan ja antaa sen meille, maan, joka vuotaa maitoa ja mettä.
\par 9 Älkää vain kapinoiko Herraa vastaan älkääkä peljätkö sen maan kansaa, sillä heitä ei ole meille kuin suupalaksi. Heidän varjelijansa on väistynyt heistä, mutta meidän kanssamme on Herra. Älkää te heitä peljätkö."
\par 10 Silloin koko seurakunta vaati heitä kivitettäviksi, mutta Herran kirkkaus ilmestyi kaikille israelilaisille ilmestysmajassa.
\par 11 Ja Herra sanoi Moosekselle: "Kuinka kauan tämä kansa pitää minua pilkkanaan eikä usko minuun, huolimatta kaikista tunnusteoista, jotka minä olen tehnyt sen keskuudessa?
\par 12 Minä tuhoan sen rutolla ja hävitän sen perinjuurin, mutta sinusta minä teen suuremman ja väkevämmän kansan, kuin se on."
\par 13 Mutta Mooses sanoi Herralle: "Ovathan egyptiläiset kuulleet, että sinä voimallasi olet vienyt tämän kansan heidän keskeltänsä.
\par 14 Ja he ovat puhuneet siitä sen maan asukkaille; hekin siis ovat kuulleet, että sinä, Herra, olet tämän kansan keskellä, sinä Herra, joka olet sille ilmestynyt silmästä silmään, ja että sinun pilvesi pysyy heidän päällänsä ja että sinä kuljet heidän edellänsä pilvenpatsaassa päivin ja tulenpatsaassa öin.
\par 15 Mutta jos sinä nyt surmaat tämän kansan, niinkuin yhden miehen, niin kansat, jotka saavat kuulla sitä kerrottavan sinusta, sanovat näin:
\par 16 'Koska Herra ei kyennyt viemään tätä kansaa siihen maahan, jonka hän oli heille valalla vannoen luvannut, tappoi hän heidät erämaahan'.
\par 17 Osoittautukoon nyt sinun suuri voimasi, Herra, niinkuin olet puhunut sanoen:
\par 18 'Herra on pitkämielinen ja suuri armossa, hän antaa anteeksi pahat teot ja rikokset, mutta ei jätä niitä rankaisematta, vaan kostaa isien pahat teot lapsille kolmanteen ja neljänteen polveen'.
\par 19 Anna siis anteeksi tämän kansan pahat teot suuressa armossasi, niinkuin sinä ennenkin olet anteeksi antanut tälle kansalle, Egyptistä tänne saakka."
\par 20 Ja Herra vastasi: "Minä annan anteeksi, niinkuin sinä olet anonut.
\par 21 Mutta niin totta kuin minä elän, ja niin totta kuin Herran kirkkaus on täyttävä kaiken maan:
\par 22 ei kukaan niistä miehistä, jotka ovat nähneet minun kirkkauteni ja minun tunnustekoni, jotka minä olen tehnyt Egyptissä ja tässä erämaassa, mutta kuitenkin nyt jo kymmenen kertaa ovat minua kiusanneet eivätkä ole kuulleet minun ääntäni,
\par 23 ei yksikään heistä ole näkevä sitä maata, jonka minä valalla vannoen lupasin heidän isillensä; ei yksikään minun pilkkaajistani ole sitä näkevä.
\par 24 Mutta koska minun palvelijassani Kaalebissa on toinen henki, niin että hän on minua uskollisesti seurannut, niin hänet minä vien siihen maahan, jossa hän kävi, ja hänen jälkeläisensä saavat sen omakseen.
\par 25 Mutta koska amalekilaiset ja kanaanilaiset asuvat laaksoissa, niin kääntykää huomenna toisaanne päin ja painukaa erämaahan, Kaislameren tietä."
\par 26 Ja Herra puhui Moosekselle ja Aaronille sanoen:
\par 27 "Kuinka kauan tämä häijy joukko napisee minua vastaan? Minä olen kuullut, kuinka israelilaiset napisevat minua vastaan.
\par 28 Sano heille: Niin totta kuin minä elän, sanoo Herra, aivan niinkuin te olette minulle puhuneet, niin minä teille teen.
\par 29 Tähän erämaahan kaatuvat teidän ruumiinne, teidän kaikkien, jotka olette katselmuksessa olleet, niin monta kuin teitä on kaksikymmenvuotisia ja sitä vanhempia, jotka olette napisseet minua vastaan.
\par 30 Totisesti, te ette pääse siihen maahan, jonka minä olen kättä kohottaen luvannut antaa teille asumasijaksi, ei kukaan teistä, paitsi Kaaleb, Jefunnen poika, ja Joosua, Nuunin poika.
\par 31 Mutta teidän lapsenne, joiden sanoitte joutuvan vihollisen saaliiksi, heidät minä vien sinne, ja he saavat tulla tuntemaan sen maan, jota te halveksitte.
\par 32 Mutta teidän ruumiinne kaatuvat tähän erämaahan,
\par 33 ja teidän lastenne täytyy harhailla paimentolaisina tässä erämaassa neljäkymmentä vuotta ja kärsiä teidän uskottomuutenne tähden, kunnes teidän ruumiinne ovat maatuneet tähän erämaahan.
\par 34 Niinkuin te neljäkymmentä päivää vakoilitte maata, niin saatte nyt, päivä vuodeksi luettuna, neljäkymmentä vuotta kärsiä pahoista teoistanne ja tulla tuntemaan, mitä se on, että minä käännyn teistä pois.
\par 35 Minä, Herra, olen puhunut. Totisesti, niin minä teen tälle häijylle kansalle, joka on käynyt kapinoimaan minua vastaan: he hukkuvat tähän erämaahan, tänne he kuolevat."
\par 36 Mutta ne miehet, jotka Mooses oli lähettänyt maata vakoilemaan ja jotka palattuaan olivat saattaneet koko kansan napisemaan häntä vastaan, saattamalla sen maan pahaan huutoon,
\par 37 ne miehet, jotka olivat saattaneet sen maan pahaan huutoon, kuolivat äkkikuolemalla Herran edessä.
\par 38 Mutta Joosua, Nuunin poika, ja Kaaleb, Jefunnen poika, jäivät eloon niistä miehistä, jotka olivat käyneet maata vakoilemassa.
\par 39 Ja Mooses puhui tämän kaikille israelilaisille. Niin kansa tuli kovin murheelliseksi.
\par 40 Ja he nousivat varhain seuraavana aamuna lähteäkseen ylös vuoristoon ja sanoivat: "Katso, tässä me olemme; me lähdemme siihen paikkaan, josta Herra on puhunut, sillä me olemme syntiä tehneet".
\par 41 Mutta Mooses sanoi: "Minkätähden te nyt käytte rikkomaan Herran käskyä? Ei se onnistu.
\par 42 Älkää lähtekö sinne, sillä Herra ei ole teidän keskellänne; älkää lähtekö, etteivät vihollisenne voittaisi teitä.
\par 43 Sillä amalekilaiset ja kanaanilaiset ovat siellä teitä vastassa, ja niin te kaadutte miekkaan; sillä te olette kääntyneet pois Herrasta, eikä Herra ole teidän kanssanne."
\par 44 Yhtäkaikki he lähtivät uppiniskaisuudessaan kulkemaan ylös vuoristoon; mutta ei Herran liitonarkki eikä Mooseskaan siirtynyt leiristä.
\par 45 Silloin amalekilaiset ja kanaanilaiset, jotka siinä vuoristossa asuivat, syöksyivät alas ja voittivat heidät ja hajottivat heidät, ajaen heitä aina Hormaan asti.

\chapter{15}

\par 1 Ja Herra puhui Moosekselle sanoen:
\par 2 "Puhu israelilaisille ja sano heille: Kun te tulette siihen maahan, jonka minä annan teille asuinsijaksi,
\par 3 ja uhraatte Herralle uhrin, poltto- tai teurasuhrin, joko lupausta täyttääksenne tai vapaaehtoisena lahjana tai juhlissanne, valmistaaksenne Herralle suloisen tuoksun raavaista tai lampaista,
\par 4 niin se, joka tuo uhrin, tuokoon lahjanansa Herralle ruokauhriksi kymmenenneksen lestyjä jauhoja, sekoitettuna neljännekseen hiin-mittaa öljyä;
\par 5 ja juomauhriksi, polttouhrin tai teurasuhrin lisäksi, uhraa neljännes hiin-mittaa viiniä kutakin karitsaa kohti.
\par 6 Mutta oinaan lisäksi uhraa ruokauhrina kaksi kymmenennestä lestyjä jauhoja, sekoitettuna kolmannekseen hiin-mittaa öljyä,
\par 7 ja juomauhriksi tuo kolmannes hiin-mittaa viiniä suloiseksi tuoksuksi Herralle.
\par 8 Mutta kun uhraat mullikan poltto- tai teurasuhriksi, joko lupausta täyttääksesi tai yhteysuhriksi Herralle,
\par 9 niin tuotakoon mullikan lisäksi ruokauhrina kolme kymmenennestä lestyjä jauhoja, sekoitettuna puoleen hiin-mittaa öljyä,
\par 10 ja juomauhriksi tuo puoli hiin-mittaa viiniä suloisesti tuoksuvaksi uhriksi Herralle.
\par 11 Näin tehtäköön kutakin härkää tai oinasta tai karitsaa tai vohlaa uhrattaessa.
\par 12 Niin monta kuin uhrieläimiä on, niin tehkää näin kullekin, niin monta kuin niitä on.
\par 13 Jokainen maassa syntynyt tehköön näin, tuodessaan suloisesti tuoksuvan uhrin Herralle.
\par 14 Ja jos muukalainen, joka asuu teidän luonanne tai on ainiaaksi asettunut teidän keskuuteenne, tahtoo uhrata suloisesti tuoksuvan uhrin Herralle, niin hän tehköön, niinkuin tekin teette.
\par 15 Seurakunnassa olkoon laki sama teillä kuin muukalaisellakin, joka asuu teidän luonanne. Tämä olkoon teille ikuinen säädös sukupolvesta sukupolveen. Se, mikä koskee teitä, koskekoon myös muukalaista Herran edessä.
\par 16 Sama laki ja sama oikeus olkoon sekä teillä että muukalaisella, joka asuu teidän luonanne."
\par 17 Ja Herra puhui Moosekselle sanoen:
\par 18 "Puhu israelilaisille ja sano heille: Kun tulette siihen maahan, johon minä teidät vien,
\par 19 niin syödessänne sen maan leipää antakaa Herralle anti.
\par 20 Ensimmäisistä jyvärouheistanne antakaa kakku anniksi; antakaa se, niinkuin puimatantereelta annatte annin.
\par 21 Ensimmäisistä jyvärouheistanne antakaa Herralle anti sukupolvesta sukupolveen.
\par 22 Mutta jos erehdyksestä jätätte täyttämättä jonkun näistä käskyistä, jotka Herra on Moosekselle puhunut,
\par 23 mitä hyvänsä, mistä Herra Mooseksen kautta on teille käskyn antanut, siitä päivästä lähtien, jolloin Herra antoi käskynsä, ja edelleen sukupolvesta sukupolveen,
\par 24 niin, jos rikkomus tapahtui seurakunnan tietämättä, erehdyksestä, uhratkoon koko seurakunta mullikan polttouhriksi, suloiseksi tuoksuksi Herralle, ynnä siihen kuuluvan ruoka- ja juomauhrin, niinkuin säädetty on, sekä kauriin syntiuhriksi.
\par 25 Kun pappi sitten on toimittanut sovituksen kaikelle israelilaisten seurakunnalle, niin rikkomus annetaan heille anteeksi; sillä se on ollut erehdys ja he ovat tuoneet lahjansa uhriksi Herralle sekä syntiuhrinsa Herran eteen erehdyksensä vuoksi.
\par 26 Se annetaan silloin anteeksi kaikelle israelilaisten seurakunnalle samoinkuin muukalaiselle, joka asuu heidän keskellänsä; sillä koko kansa on vastuunalainen erehdyksestä.
\par 27 Mutta jos joku yksityinen rikkoo erehdyksestä, tuokoon vuoden vanhan vuohen syntiuhriksi.
\par 28 Kun pappi on toimittanut sovituksen sille, joka on erehdyksestä, tahtomattaan, rikkonut Herraa vastaan, kun hän on toimittanut hänelle sovituksen, niin hänelle annetaan anteeksi.
\par 29 Sama laki olkoon sekä maassa syntyneillä israelilaisilla että muukalaisella, joka asuu heidän keskellänsä, kun joku rikkoo erehdyksestä.
\par 30 Mutta se, joka tahallisesti tekee syntiä, olipa hän maassa syntynyt tai muukalainen, pilkkaa Herraa, ja hänet hävitettäköön kansastansa.
\par 31 Sillä hän on pitänyt halpana Herran sanan ja rikkonut hänen käskynsä; hänet tuhottakoon, syntivelka painaa häntä."
\par 32 Israelilaisten oleskellessa erämaassa tavattiin mies kokoamassa puita sapatinpäivänä.
\par 33 Niin ne, jotka hänet tapasivat puita kokoamasta, toivat hänet Mooseksen ja Aaronin ja koko seurakunnan eteen.
\par 34 Ja he panivat hänet vankeuteen, koska ei ollut vielä määrätty, mitä hänelle oli tehtävä.
\par 35 Mutta Herra sanoi Moosekselle: "Se mies rangaistakoon kuolemalla, koko seurakunta kivittäköön hänet leirin ulkopuolella".
\par 36 Silloin koko seurakunta vei hänet leirin ulkopuolelle, ja he kivittivät hänet kuoliaaksi, niinkuin Herra oli Moosekselle käskyn antanut.
\par 37 Ja Herra puhui Moosekselle sanoen:
\par 38 "Puhu israelilaisille ja sano heille, että heidän on sukupolvesta sukupolveen tehtävä itsellensä tupsut viittojensa kulmiin ja sidottava kulmien tupsuihin punasininen lanka.
\par 39 Ne tupsut olkoon teillä, että te, kun ne näette, muistaisitte kaikki Herran käskyt ja ne täyttäisitte ettekä seuraisi sydämenne ettekä silmienne himoja, jotka houkuttelevat teitä haureuteen;
\par 40 niin muistakaa ja täyttäkää kaikki minun käskyni ja olkaa pyhät Jumalallenne.
\par 41 Minä olen Herra, teidän Jumalanne, joka vein teidät pois Egyptin maasta ollakseni teidän Jumalanne. Minä olen Herra, teidän Jumalanne."

\chapter{16}

\par 1 Mutta Koorah, Jisharin poika, joka oli Leevin pojan Kehatin poika, otti Daatanin ja Abiramin, Eliabin pojat, ja Oonin, joka oli Ruubenin pojan Peletin poika,
\par 2 ja he nousivat kapinaan Moosesta vastaan, ja heihin yhtyi israelilaisista kaksisataa viisikymmentä miestä, kansan päämiehiä, kansankokouksen jäseniä, arvokkaita miehiä.
\par 3 Ja he kokoontuivat Moosesta ja Aaronia vastaan ja sanoivat heille: "Jo riittää! Koko seurakunta, he kaikki, ovat pyhät, ja Herra on heidän keskellänsä. Miksi te siis korotatte itsenne Herran seurakunnan yli?"
\par 4 Kun Mooses sen kuuli, lankesi hän kasvoillensa.
\par 5 Sitten hän puhui Koorahille ja kaikelle hänen joukolleen sanoen: "Huomenna Herra ilmoittaa, kuka on hänen omansa ja kuka on pyhä ja kenen hän sallii käydä tykönsä. Kenen hän valitsee, sen hän sallii käydä tykönsä.
\par 6 Tehkää näin: Ottakaa itsellenne hiilipannut, sinä Koorah ja kaikki sinun joukkosi,
\par 7 ja virittäkää niihin tuli ja pankaa niihin huomenna suitsuketta Herran edessä, niin se mies, jonka Herra silloin valitsee, on pyhä. Riittäköön se teille, te leeviläiset!"
\par 8 Ja Mooses puhui Koorahille: "Kuulkaa, te leeviläiset!
\par 9 Eikö teille jo riitä se, että Israelin Jumala on erottanut teidät Israelin seurakunnasta, sallinut teidän käydä hänen tykönsä toimittamaan palvelusta Herran asumuksessa ja seisomaan seurakunnan edessä palvelemassa heitä?
\par 10 Hän salli käydä tykönsä sinun ja kaikkien veljiesi, leeviläisten, sinun kanssasi; ja nyt te tavoittelette pappeuttakin.
\par 11 Niin siis sinä ja koko sinun joukkosi käytte kapinoimaan Herraa vastaan; sillä mikä on Aaron, että te häntä vastaan napisette!"
\par 12 Ja Mooses kutsutti Daatanin ja Abiramin, Eliabin pojat. Mutta he sanoivat: "Emme tule.
\par 13 Eikö jo riitä, että olet tuonut meidät tänne maasta, joka vuotaa maitoa ja mettä, hukuttaaksesi meidät erämaahan? Pyritkö vielä meidän valtiaaksemme?
\par 14 Oletpa totisesti tuonut meidät maahan, joka vuotaa maitoa ja mettä, ja antanut meille perintöosaksi vainioita ja viinitarhoja! Luuletko voivasi sokaista silmät näiltä ihmisiltä? Me emme tule."
\par 15 Silloin Mooses vihastui kovin ja sanoi Herralle: "Älä katso heidän uhrilahjansa puoleen. En ole keneltäkään heistä aasiakaan anastanut enkä kenellekään heistä mitään vääryyttä tehnyt."
\par 16 Silloin Mooses sanoi Koorahille: "Astukaa huomenna Herran eteen, sinä itse ja koko sinun joukkosi, sekä myöskin Aaron,
\par 17 ja ottakoon kukin teistä hiilipannunsa ja pankoon siihen suitsuketta sekä tuokoon Herran eteen hiilipannunsa, kaksisataa viisikymmentä hiilipannua; niin ikään sinä ja Aaron tuokaa kumpikin hiilipannunne".
\par 18 Niin he ottivat kukin hiilipannunsa, virittivät niihin tulen ja panivat niihin suitsuketta ja asettuivat ilmestysmajan ovelle; samoin myöskin Mooses ja Aaron.
\par 19 Mutta Koorah kokosi heitä vastaan koko seurakunnan ilmestysmajan ovelle. Silloin Herran kirkkaus näkyi koko seurakunnalle.
\par 20 Ja Herra puhui Moosekselle ja Aaronille sanoen:
\par 21 "Erottautukaa te tästä joukosta, niin minä tuokiossa heidät tuhoan".
\par 22 Mutta he lankesivat kasvoillensa ja sanoivat: "Oi Jumala, sinä Jumala, jonka vallassa on kaiken lihan henki: jos yksi mies rikkoo, vihastutko silloin koko seurakuntaan?"
\par 23 Niin Herra puhui Moosekselle sanoen:
\par 24 "Puhu seurakunnalle näin: Lähtekää pois Koorahin, Daatanin ja Abiramin asumuksen ympäriltä".
\par 25 Sitten Mooses nousi ja meni Daatanin ja Abiramin luo, ja Israelin vanhimmat seurasivat häntä.
\par 26 Ja hän puhui seurakunnalle sanoen: "Siirtykää pois näiden jumalattomien miesten majoilta älkääkä koskeko mihinkään, mikä on heidän, ettette tuhoutuisi kaikkien heidän syntiensä tähden".
\par 27 Niin he lähtivät pois Koorahin, Daatanin ja Abiramin asumuksen ympäriltä. Mutta Daatan ja Abiram olivat tulleet ulos ja asettuneet majojensa ovelle vaimoinensa ja suurine ja pienine lapsinensa.
\par 28 Ja Mooses puhui: "Tästä te tietäkää, että Herra on minut lähettänyt tekemään kaikki nämä teot ja etteivät ne ole tapahtuneet minun omasta tahdostani:
\par 29 jos nämä kuolevat samalla tavalla, kuin muut ihmiset kuolevat, ja heidän käy, niinkuin kaikkien muiden ihmisten käy, niin ei Herra ole minua lähettänyt.
\par 30 Mutta jos Herra antaa jotakin erikoista tapahtua ja maa avaa kitansa ja nielaisee heidät kaikkinensa, niin että he elävältä suistuvat tuonelaan, niin siitä tietäkää, että nämä miehet ovat pilkanneet Herraa."
\par 31 Tuskin oli hän saanut kaiken tämän sanotuksi, niin maa halkesi heidän altansa
\par 32 ja maa avasi kitansa ja nielaisi heidät, heidän asuntonsa, kaiken Koorahin väen ja kaikki heidän tavaransa.
\par 33 Ja he suistuivat kaikkinensa elävältä tuonelaan, ja maa peitti heidät; ja niin heidät hävitettiin seurakunnan keskeltä.
\par 34 Ja koko Israel, joka oli heidän ympärillänsä, pakeni kuullessaan heidän huutonsa, sillä he pelkäsivät maan nielaisevan heidätkin.
\par 35 Mutta tuli lähti Herran tyköä ja kulutti ne kaksisataa viisikymmentä miestä, jotka olivat tuoneet suitsutusta.
\par 36 Ja Herra puhui Moosekselle sanoen:
\par 37 "Käske Eleasaria, pappi Aaronin poikaa, korjaamaan nuo hiilipannut palosta, mutta pudistamaan tuli niistä kauas,
\par 38 sillä ne ovat tulleet pyhäkön omiksi. Noiden henkensä menettäneiden syntisten hiilipannuista takokaa ohuita levyjä alttarin päällystämiseksi; sillä he toivat ne Herran eteen, ja niin ne tulivat pyhäkön omiksi ja merkiksi israelilaisille."
\par 39 Niin pappi Eleasar otti ne vaskiset hiilipannut, jotka nuo tulen polttamat miehet olivat tuoneet, ja ne taottiin alttarin päällystykseksi,
\par 40 muistuttamaan israelilaisille, ettei kukaan syrjäinen, joka ei ole Aaronin sukua, lähestyisi polttamaan suitsuketta Herran edessä, niin ettei hänen kävisi samoin kuin Koorahin ja hänen joukkonsa, niinkuin Herra oli hänelle Mooseksen kautta puhunut.
\par 41 Mutta seuraavana päivänä kaikki israelilaisten seurakunta napisi Moosesta ja Aaronia vastaan sanoen: "Te olette surmanneet Herran kansaa".
\par 42 Mutta kun kansa kokoontui Moosesta ja Aaronia vastaan, niin he kääntyivät ilmestysmajaan päin, ja katso, pilvi peitti sen, ja Herran kirkkaus näkyi.
\par 43 Silloin Mooses ja Aaron menivät ilmestysmajan edustalle.
\par 44 Ja Herra puhui Moosekselle sanoen:
\par 45 "Poistukaa tämän joukon luota, niin minä tuokiossa heidät tuhoan". Silloin he lankesivat kasvoillensa.
\par 46 Ja Mooses sanoi Aaronille: "Ota hiilipannu ja viritä siihen tuli alttarilta, pane siihen suitsuketta ja vie se nopeasti seurakunnan luo ja toimita heille sovitus, sillä viha on lähtenyt liikkeelle Herran tyköä, vitsaus on jo alkanut".
\par 47 Ja Aaron teki, niinkuin Mooses oli käskenyt, ja riensi seurakunnan keskelle, ja katso, vitsaus oli jo alkanut kansan keskuudessa. Silloin hän suitsutti ja toimitti kansalle sovituksen.
\par 48 Ja hänen siinä seisoessaan kuolleitten ja elävien vaiheella vitsaus taukosi.
\par 49 Mutta niitä, jotka vitsauksesta kuolivat, oli neljätoista tuhatta seitsemänsataa henkeä, paitsi niitä, jotka saivat surmansa Koorahin tähden.
\par 50 Sitten Aaron palasi takaisin Mooseksen luo ilmestysmajan ovelle, ja vitsaus oli tauonnut.

\chapter{17}

\par 1 Ja Herra puhui Moosekselle sanoen:
\par 2 "Puhu israelilaisille ja ota heiltä, kaikilta heidän ruhtinailtansa, sauva kutakin sukukuntaa kohti, heidän sukukuntiensa mukaan, kaksitoista sauvaa. Kirjoita kunkin nimi hänen sauvaansa,
\par 3 mutta Leevin sauvaan kirjoita Aaronin nimi, sillä tämänkin sukukunnan päämiehellä olkoon sauvansa.
\par 4 Pane ne sitten ilmestysmajaan lain arkin eteen, missä minä teille ilmestyn.
\par 5 Ja tapahtuu, että kenen minä valitsen, sen sauva versoo. Siten minä lopetan israelilaisten napinan, niin että pääsen kuulemasta heidän napinaansa teitä vastaan."
\par 6 Ja Mooses puhui tämän israelilaisille, ja kaikki heidän ruhtinaansa antoivat hänelle kukin sauvansa sukukunnittain, yhteensä kaksitoista sauvaa, ja Aaronin sauva oli heidän sauvainsa joukossa.
\par 7 Ja Mooses pani sauvat lain majaan, Herran eteen.
\par 8 Kun Mooses seuraavana päivänä meni lain majaan, niin katso, Aaronin sauva, joka oli siinä Leevin sukukunnan puolesta, oli alkanut versoa, siihen oli ilmestynyt silmuja, puhjennut kukkia ja kasvanut kypsiä manteleita.
\par 9 Ja Mooses vei ulos kaikki sauvat Herran kasvojen edestä kaikkien israelilaisten tykö; ja he katselivat niitä ja ottivat kukin sauvansa.
\par 10 Ja Herra sanoi Moosekselle: "Vie Aaronin sauva takaisin lain arkin eteen säilytettäväksi merkkinä uppiniskaisille, ja lopeta heidän napinansa, niin että minä pääsen sitä kuulemasta, jotta he eivät kuolisi".
\par 11 Ja Mooses teki sen; niinkuin Herra häntä käski, niin hän teki.
\par 12 Mutta israelilaiset sanoivat Moosekselle näin: "Katso, me menehdymme, me hukumme, kaikki me hukumme!
\par 13 Kuka vain lähestyy, kuka ikinä lähestyy Herran asumusta, se kuolee! Onko meidän kaikkien hukkuminen?"

\chapter{18}

\par 1 Ja Herra sanoi Aaronille: "Sinun ja sinun poikiesi, isäsi suvun sinun kanssasi, on kannettava pyhäkköä vastaan tehdyt rikkomukset; samoin sinun ja poikiesi sinun kanssasi on kannettava pappeutta vastaan tehdyt rikkomukset.
\par 2 Mutta salli myös veljiesi, Leevin sukukunnan, isäsi heimon, käydä sinun kanssasi sinne. He liittykööt sinuun ja palvelkoot sinua, kun sinä ja poikasi sinun kanssasi toimitatte palvelusta lain majan edessä.
\par 3 Ja he hoitakoot sekä sinun tehtäviäsi että kaikkia majan tehtäviä, mutta älkööt lähestykö pyhiä esineitä tai alttaria, etteivät kuolisi, niin he kuin tekin.
\par 4 Liittykööt he sinuun ja hoitakoot ilmestysmajan tehtävät, majan kaikki palvelustehtävät; mutta syrjäinen älköön teitä lähestykö.
\par 5 Ja teidän on hoidettava tehtävät pyhäkössä ja alttarilla, ettei Herran viha enää kohtaisi israelilaisia.
\par 6 Ja katso, minä olen ottanut teidän veljenne, leeviläiset, israelilaisten keskuudesta teille lahjaksi, Herralle annettuina toimittamaan palvelusta ilmestysmajassa.
\par 7 Mutta sinä ja poikasi sinun kanssasi hoitakaa papinvirkaanne, pitäen huolta kaikista alttarilla ja esiripun sisäpuolella suoritettavista tehtävistä, ja toimittakaa palvelusta siellä. Lahjana minä annan teille papinvirkanne; mutta syrjäinen, joka siihen ryhtyy, surmattakoon."
\par 8 Ja Herra puhui Aaronille: "Katso, minä annan sinulle sen, mikä saamistani anneista on talteen otettava; kaikista israelilaisten pyhistä lahjoista minä annan sen sinulle osuudeksi ja sinun pojillesi ikuiseksi oikeudeksi.
\par 9 Korkeasti-pyhistä lahjoista olkoon sinun omasi tämä, jota ei tulessa polteta: kaikki ne heidän uhrilahjansa, jotka kuuluvat kaikkiin heidän ruokauhreihinsa, syntiuhreihinsa ja vikauhreihinsa, joita he suorittavat minulle korvauksena; se on korkeasti-pyhää ja olkoon sinun ja sinun poikiesi oma.
\par 10 Syökää se korkeasti-pyhässä paikassa; jokainen miehenpuoli saakoon syödä sitä. Olkoon se sinulle pyhä.
\par 11 Ja antina heidän lahjoistansa olkoon sinun omasi tämä: kaiken sen, minkä israelilaiset uhraavat heilutusmenoin, minä annan sinulle, sinun pojillesi ja tyttärillesi ikuiseksi oikeudeksi; jokainen, joka on puhdas sinun perheessäsi, saakoon syödä sitä.
\par 12 Kaiken parhaan öljyn ja kaiken parhaan viinin ja viljan, minkä he uutisena antavat Herralle, sen minä annan sinulle.
\par 13 Uutiset kaikesta, mitä heidän maassansa kasvaa ja minkä he tuovat Herralle, olkoot sinun omasi; jokainen, joka on puhdas sinun perheessäsi, saakoon syödä niitä.
\par 14 Kaikki, mitä Israelissa vihittyä on, olkoon sinun omaasi.
\par 15 Kaikki, mikä avaa äidinkohdun, mikä elollinen hyvänsä, joka tuodaan Herralle, ihmisistä tai karjasta, olkoon sinun omasi; lunastuta kuitenkin ihmisen esikoinen samoinkuin saastaisen eläimen esikoinen.
\par 16 Ja mitä heidän lunastamiseensa tulee, niin lunastuta heidät kuukauden ikäisistä alkaen sinun asettamastasi viiden sekelin arviohinnasta pyhäkkösekelin painon mukaan, sekeli kaksikymmentä geeraa.
\par 17 Mutta raavaan tai lampaan tai vuohen esikoista älä lunastuta; ne ovat pyhiä. Vihmo niiden veri alttarille ja polta niiden rasva uhriksi, suloiseksi tuoksuksi Herralle.
\par 18 Mutta niiden liha olkoon sinun omasi; niinkuin heilutettu rintalihakin ja oikea reisi olkoon se sinun omasi.
\par 19 Kaikki pyhät annit, joita israelilaiset antavat Herralle, minä annan sinulle, sinun pojillesi ja tyttärillesi ikuiseksi oikeudeksi. Se olkoon ikuinen suolaliitto Herran edessä sinulle ja sinun jälkeläisillesi."
\par 20 Ja Herra puhui Aaronille: "Sinulla älköön olko perintöosaa heidän maassansa älköönkä osuutta heidän keskellänsä; minä itse olen sinun osuutesi ja perintöosasi israelilaisten keskellä.
\par 21 Mutta leeviläisille minä annan kaikki kymmenykset Israelissa perintöosaksi, palkkana siitä palveluksesta, jonka he toimittavat ilmestysmajassa.
\par 22 Älköötkä muut israelilaiset enää lähestykö ilmestysmajaa, että he eivät joutuisi syynalaisiksi ja kuolisi.
\par 23 Ainoastaan leeviläiset toimittakoot palvelusta ilmestysmajassa ja kantakoot tehdyt rikkomukset; se olkoon teille ikuinen säädös sukupolvesta sukupolveen. Mutta heillä älköön olko perintöosaa israelilaisten keskellä.
\par 24 Sillä israelilaisten kymmenykset, jotka he antavat Herralle anniksi, minä annan leeviläisille perintöosaksi; sentähden minä kiellän heiltä perintöosan israelilaisten keskuudessa."
\par 25 Ja Herra puhui Moosekselle sanoen:
\par 26 "Puhu leeviläisille ja sano: Kun te israelilaisilta otatte ne kymmenykset, jotka minä olen määrännyt heidän teille annettavaksi perintöosaksenne, niin antakaa siitä Herralle anti, kymmenykset kymmenyksistä,
\par 27 ja se katsotaan teidän anniksenne, niinkuin puimatantereelta annetut jyvät tai kuurnasta tullut mehu.
\par 28 Niin antakaa tekin Herralle anti kaikista kymmenyksistä, jotka te saatte israelilaisilta, ja tämä Herralle niistä tuleva anti antakaa pappi Aaronille.
\par 29 Kaikista saamistanne lahjoista antakaa Herralle täysi anti; kaikesta, mikä parasta on, antakaa pyhä lahja.
\par 30 Ja sano heille: Kun te siitä annatte parhaimman osan, niin se katsotaan leeviläisten anniksi niinkuin puimatantereen tai kuurnan sato.
\par 31 Ja sen saatte te ja teidän perheenne syödä missä hyvänsä, sillä se on teille palkka palveluksestanne ilmestysmajassa.
\par 32 Kun te näin annatte siitä anniksi parhaimman osan, ette joudu syynalaisiksi ettekä saastuta israelilaisten pyhiä lahjoja ettekä kuole."

\chapter{19}

\par 1 Ja Herra puhui Moosekselle ja Aaronille sanoen:
\par 2 "Tämä on lakimääräys, jonka Herra sääsi sanoen: Käske israelilaisia tuomaan sinulle ruskeanpunainen, virheetön hieho, jossa ei ole mitään vammaa ja jonka päälle iestä ei vielä ole pantu.
\par 3 Antakaa se pappi Eleasarin haltuun, ja vietäköön se leiristä ulos ja teurastettakoon hänen edessänsä.
\par 4 Sitten pappi Eleasar ottakoon sen verta sormeensa ja pirskoittakoon sen verta ilmestysmajan etupuolta kohti seitsemän kertaa.
\par 5 Ja hieho poltettakoon hänen nähtensä; nahkoinensa, lihoinensa ja verinensä ynnä rapoinensa se poltettakoon.
\par 6 Mutta pappi ottakoon setripuuta, isoppikorren ja helakanpunaista lankaa ja viskatkoon ne tuleen, missä hieho palaa.
\par 7 Sitten pappi pesköön vaatteensa ja pesköön ruumiinsa vedessä; sitten menköön leiriin. Mutta pappi olkoon saastainen iltaan asti.
\par 8 Sekin, joka hiehon poltti, pesköön vaatteensa vedessä ja pesköön ruumiinsa vedessä ja olkoon saastainen iltaan asti.
\par 9 Sitten joku, joka on puhdas, kootkoon hiehon tuhan ja pankoon sen talteen leirin ulkopuolelle puhtaaseen paikkaan. Ja säilytettäköön se israelilaisten seurakunnalle puhdistusvettä varten. Tämä on syntiuhri.
\par 10 Ja joka hiehon tuhan kokosi, pesköön vaatteensa ja olkoon saastainen iltaan asti. Tämä olkoon israelilaisille ja muukalaiselle, joka asuu heidän keskellänsä, ikuinen säädös.
\par 11 Joka koskee kuolleen ihmisen ruumiiseen, kenen hyvänsä, olkoon saastainen seitsemän päivää.
\par 12 Hän puhdistakoon itsensä tällä vedellä kolmantena ja seitsemäntenä päivänä, niin hän tulee puhtaaksi. Mutta jos hän ei puhdistaudu kolmantena ja seitsemäntenä päivänä, niin hän ei tule puhtaaksi.
\par 13 Jokainen, joka koskee kuolleeseen, kuolleen ihmisen ruumiiseen, eikä puhdistaudu, hän saastuttaa Herran asumuksen, ja hänet hävitettäköön Israelista, koska puhdistusvettä ei ole vihmottu häneen; hän on saastainen, hänen saastaisuutensa pysyy hänessä yhä.
\par 14 Tämä on laki: Kun joku kuolee telttamajassa, niin jokainen, joka menee siihen majaan, ja jokainen, joka on siinä majassa, olkoon saastainen seitsemän päivää.
\par 15 Ja jokainen avoin astia, johon ei ole peitettä sidottu, olkoon saastainen.
\par 16 Ja jokainen, joka ulkona kedolla koskee miekalla surmattuun tai muulla tavoin kuolleeseen tai ihmisluuhun tai hautaan, olkoon saastainen seitsemän päivää.
\par 17 Mutta näin saastunutta varten otettakoon poltetun syntiuhrin tuhkaa ja kaadettakoon sen päälle astiaan raikasta vettä.
\par 18 Ottakoon sitten joku, joka on puhdas, isoppikorren ja kastakoon sen siihen veteen ja pirskoittakoon sitä tuohon majaan ja kaikkiin sen esineihin ja ihmisiin, jotka siellä olivat, samoinkuin siihen, joka on koskenut luuhun tai surmattuun tai muulla tavoin kuolleeseen tai hautaan.
\par 19 Ja se puhdas pirskoittakoon sitä saastuneeseen kolmantena ja seitsemäntenä päivänä ja puhdistakoon hänet seitsemäntenä päivänä. Sitten hän pesköön vaatteensa ja peseytyköön vedessä, niin hän tulee illalla puhtaaksi.
\par 20 Mutta joka on saastunut eikä puhdistaudu, se hävitettäköön seurakunnasta, koska hän on saastuttanut Herran pyhäkön eikä puhdistusvettä ole vihmottu häneen; hän on saastainen.
\par 21 Ja tämä olkoon heille ikuinen säädös. Mutta puhdistusveden pirskoittaja pesköön vaatteensa, ja se, joka koskee puhdistusveteen, olkoon saastainen iltaan asti.
\par 22 Ja kaikki, mihin saastunut koskee, olkoon saastaista, ja se, joka koskee häneen, olkoon saastainen iltaan asti."

\chapter{20}

\par 1 Senjälkeen israelilaiset, koko seurakunta, tulivat Siinin erämaahan ensimmäisenä kuukautena, ja kansa asettui Kaadekseen. Siellä Mirjam kuoli, ja hänet haudattiin sinne.
\par 2 Mutta kansalla ei ollut vettä; niin he kokoontuivat Moosesta ja Aaronia vastaan.
\par 3 Ja kansa riiteli Moosesta vastaan ja sanoi näin: "Jospa mekin olisimme hukkuneet silloin, kun veljemme hukkuivat Herran edessä!
\par 4 Minkätähden toitte Herran seurakunnan tähän erämaahan, kuollaksemme karjoinemme tänne?
\par 5 Ja minkätähden johdatitte meidät pois Egyptistä tuodaksenne meidät tähän pahaan paikkaan, jossa ei kasva viljaa eikä viikunoita, ei viiniköynnöksiä eikä granaattiomenia, ja jossa ei ole vettä juoda?"
\par 6 Mutta Mooses ja Aaron menivät seurakunnan luota ilmestysmajan ovelle ja lankesivat kasvoilleen. Silloin näkyi Herran kirkkaus heille.
\par 7 Ja Herra puhui Moosekselle sanoen:
\par 8 "Ota sauva ja kokoa seurakunta, sinä ja veljesi Aaron, ja puhukaa heidän silmiensä edessä kalliolle, niin se antaa vettä, ja sinä saat vettä tulemaan heille kalliosta ja juotat joukon ja sen karjan".
\par 9 Niin Mooses otti sauvan Herran kasvojen edestä, niinkuin hän oli häntä käskenyt.
\par 10 Ja Mooses ja Aaron kokosivat seurakunnan kallion eteen, ja hän sanoi heille: "Kuulkaa nyt, te niskurit! Onko meidän saatava teille vettä tästä kalliosta?"
\par 11 Niin Mooses nosti kätensä ja iski kalliota kahdesti sauvallansa, ja siitä lähti runsaasti vettä, niin että kansa ja sen karja saivat juoda.
\par 12 Mutta Herra sanoi Moosekselle ja Aaronille: "Koska ette uskoneet minuun ettekä pitäneet minua pyhänä israelilaisten silmien edessä, niin te ette saa viedä tätä seurakuntaa siihen maahan, jonka minä heille annan".
\par 13 Tämä oli Meriban vesi, jonka luona israelilaiset riitelivät Herraa vastaan ja hän näytti heille pyhyytensä.
\par 14 Mutta Mooses lähetti Kaadeksesta sanansaattajat Edomin kuninkaan tykö sanomaan: "Näin sanoo veljesi Israel: Sinä tunnet kaiken vaivan, joka on meitä kohdannut,
\par 15 kuinka isämme lähtivät Egyptiin, kuinka olemme asuneet Egyptissä kauan aikaa ja kuinka egyptiläiset kohtelivat pahoin meitä ja meidän isiämme.
\par 16 Mutta me huusimme Herran puoleen, ja hän kuuli meidän huutomme ja lähetti enkelin, joka vei meidät pois Egyptistä. Ja katso, me olemme nyt Kaadeksessa, kaupungissa, joka on sinun maasi rajalla.
\par 17 Salli meidän kulkea maasi läpi. Me emme kulje peltojen emmekä viinitarhojen poikki emmekä juo vettä kaivoista. Poikkeamatta oikealle tai vasemmalle me kuljemme valtatietä, kunnes olemme päässeet alueesi läpi."
\par 18 Mutta Edom vastasi hänelle: "Älä kulje minun maani läpi; muutoin minä käyn miekka kädessä sinua vastaan".
\par 19 Mutta israelilaiset sanoivat hänelle: "Maantietä me kuljemme, ja jos juomme vettäsi, me tai meidän karjamme, niin me maksamme sen, kun vain saamme jalkaisin kulkea sinun maasi läpi".
\par 20 Mutta hän vastasi: "Tästä et kulje!" Ja Edom lähti häntä vastaan suurella sotajoukolla ja vahvasti varustettuna.
\par 21 Kun siis Edom kielsi Israelia kulkemasta maansa läpi, niin Israel väistyi sieltä pois.
\par 22 Ja he lähtivät liikkeelle Kaadeksesta. Ja israelilaiset, koko kansa, tulivat Hoorin vuorelle.
\par 23 Mutta Herra sanoi Moosekselle ja Aaronille Hoorin vuorella, Edomin maan rajalla, näin:
\par 24 "Aaron otetaan nyt pois heimonsa tykö, sillä hän ei pääse siihen maahan, jonka minä annan israelilaisille, koska te niskoittelitte minun käskyäni vastaan Meriban veden luona.
\par 25 Ota Aaron ja hänen poikansa Eleasar ja vie heidät Hoorin vuorelle.
\par 26 Ja riisu Aaronilta vaatteet ja pue ne hänen poikansa Eleasarin ylle. Aaron otetaan pois ja kuolee siellä."
\par 27 Ja Mooses teki, niinkuin Herra oli käskenyt; ja he nousivat Hoorin vuorelle koko kansan silmien edessä.
\par 28 Ja Mooses riisui Aaronilta vaatteet ja puki ne hänen poikansa Eleasarin ylle. Niin Aaron kuoli siellä vuoren huipulla, mutta Mooses ja Eleasar astuivat alas vuorelta.
\par 29 Ja kun koko kansa sai tietää, että Aaron oli kuollut, niin he itkivät Aaronia kolmekymmentä päivää, koko Israelin heimo.

\chapter{21}

\par 1 Mutta kun Aradin kuningas, kanaanilainen, joka asui Etelämaassa, kuuli, että Israel oli tulossa Atarimin tietä, ryhtyi hän sotimaan Israelia vastaan ja otti muutamia heistä vangiksi.
\par 2 Silloin Israel teki lupauksen Herralle ja sanoi: "Jos sinä annat tämän kansan minun käsiini, niin minä vihin sen kaupungit tuhon omiksi".
\par 3 Ja Herra kuuli, mitä Israel lausui, ja jätti kanaanilaiset hänen käsiinsä. Niin Israel vihki heidät ja heidän kaupunkinsa tuhon omiksi. Ja sen paikan nimeksi pantiin Horma.
\par 4 Senjälkeen he lähtivät liikkeelle Hoorin vuorelta Kaislameren tietä kiertääksensä Edomin maan. Mutta matkalla kansa kävi kärsimättömäksi.
\par 5 Ja kansa puhui Jumalaa ja Moosesta vastaan: "Minkätähden te johdatitte meidät pois Egyptistä kuolemaan erämaahan? Eihän täällä ole leipää eikä vettä, ja me olemme kyllästyneet tähän huonoon ruokaan."
\par 6 Silloin Herra lähetti kansan sekaan myrkkykäärmeitä, jotka purivat kansaa, niin että paljon kansaa Israelista kuoli.
\par 7 Niin kansa tuli Mooseksen luo, ja he sanoivat: "Me teimme syntiä, kun puhuimme Herraa ja sinua vastaan. Rukoile Herraa, että hän poistaa käärmeet meidän kimpustamme." Ja Mooses rukoili kansan puolesta.
\par 8 Silloin Herra sanoi Moosekselle: "Tee itsellesi käärme ja pane se tangon päähän, niin jokainen purtu, joka siihen katsoo, jää eloon".
\par 9 Niin Mooses teki vaskikäärmeen ja pani sen tangon päähän; jos ketä käärmeet sitten purivat ja tämä katsoi vaskikäärmeeseen, niin hän jäi eloon.
\par 10 Sitten israelilaiset lähtivät liikkeelle ja leiriytyivät Oobotiin.
\par 11 Ja he lähtivät liikkeelle Oobotista ja leiriytyivät Iije-Abarimiin siinä erämaassa, joka on Mooabista itään, auringon nousun puolella.
\par 12 Sieltä he lähtivät liikkeelle ja leiriytyivät Seredin laaksoon.
\par 13 Sieltä he lähtivät liikkeelle ja leiriytyivät erämaahan tuolle puolelle Arnon-jokea, joka lähtee amorilaisten alueelta; sillä Arnon on Mooabin rajana Mooabin ja amorilaisten maan välillä.
\par 14 Sentähden sanotaan "Herran sotien kirjassa": "Vaaheb Suufassa ja Arnonin laaksot
\par 15 ja laaksojen rinteet, jotka ulottuvat Aarin tienoille ja kallistuvat Mooabin rajaan".
\par 16 Ja sieltä he kulkivat Beeriin. Se oli se kaivo, josta Herra oli sanonut Moosekselle: "Kokoa kansa, niin minä annan heille vettä".
\par 17 Silloin lauloi Israel tämän laulun: "Kuohu, kaivo! - laulakaa sille laulu -
\par 18 kaivo, jonka ruhtinaat kaivoivat, kansan parhaat koversivat valtikallaan, sauvoillansa!" Ja siitä erämaasta he kulkivat Mattanaan
\par 19 ja Mattanasta Nahalieliin ja Nahalielista Baamotiin
\par 20 ja Baamotista siihen laaksoon, joka on Mooabin maassa, Pisgan huipun juurella, joka kohoaa yli erämaan.
\par 21 Ja Israel lähetti sanansaattajat Siihonin, amorilaisten kuninkaan, tykö sanomaan:
\par 22 "Salli minun kulkea maasi läpi. Me emme poikkea pelloille emmekä viinitarhoihin emmekä juo vettä kaivoista. Valtatietä me kuljemme, kunnes olemme päässeet sinun alueesi läpi."
\par 23 Mutta Siihon ei sallinut Israelin kulkea alueensa läpi, vaan kokosi kaiken väkensä ja lähti Israelia vastaan erämaahan. Ja kun hän oli tullut Jahaaseen, ryhtyi hän taisteluun Israelia vastaan.
\par 24 Mutta Israel voitti hänet miekan terällä ja valloitti hänen maansa Arnonista Jabbokiin, aina ammonilaisten maahan asti, sillä ammonilaisten raja oli varustettu.
\par 25 Ja Israel valtasi sieltä kaikki kaupungit; ja Israel asettui kaikkiin amorilaisten kaupunkeihin, Hesboniin ja kaikkiin sen tytärkaupunkeihin.
\par 26 Sillä Hesbon oli amorilaisten kuninkaan Siihonin kaupunki; hän oli näet käynyt sotaa entistä Mooabin kuningasta vastaan sekä vallannut häneltä kaiken hänen maansa Arnoniin asti.
\par 27 Sentähden runosepät sanovat: "Tulkaa Hesboniin! Rakennettakoon ja varustettakoon Siihonin kaupunki.
\par 28 Sillä tuli lähti Hesbonista, liekki Siihonin kaupungista; se kulutti Aar-Mooabin, Arnonin kukkulain valtiaat.
\par 29 Voi sinua, Mooab! Sinä hukuit, Kemoksen kansa! Poikiensa hän salli tulla pakolaisiksi, tytärtensä amorilaisten kuninkaan Siihonin vangeiksi.
\par 30 Me ammuimme heidät, Hesbon kukistui Diibonia myöten; me hävitimme Noofahiin asti, aina Meedebaan saakka."
\par 31 Niin Israel asettui amorilaisten maahan.
\par 32 Ja Mooses lähetti vakoilemaan Jaeseria, ja he valloittivat sen tytärkaupungit; ja hän karkoitti amorilaiset, jotka siellä olivat.
\par 33 Sitten he kääntyivät toisaalle ja kulkivat Baasanin tietä. Silloin lähti Oog, Baasanin kuningas, heitä vastaan, hän ja kaikki hänen väkensä, Edreihin taistelemaan.
\par 34 Mutta Herra sanoi Moosekselle: "Älä pelkää häntä, sillä minä annan sinun käteesi hänet ja kaiken hänen kansansa ja maansa. Ja tee hänelle, niinkuin teit Siihonille, amorilaisten kuninkaalle, joka asui Hesbonissa."
\par 35 Ja he voittivat hänet ja hänen poikansa ja kaiken hänen väkensä, päästämättä pakoon ainoatakaan. Niin he valloittivat hänen maansa.

\chapter{22}

\par 1 Sitten israelilaiset lähtivät liikkeelle ja leiriytyivät Mooabin arolle, Jordanin tuolle puolelle, Jerikon kohdalle.
\par 2 Mutta Baalak, Sipporin poika, näki kaiken, mitä Israel oli tehnyt amorilaisille.
\par 3 Ja Mooab pelkäsi tätä kansaa suuresti, kun sitä oli niin paljon; ja Mooab kauhistui israelilaisia.
\par 4 Mooab sanoi Midianin vanhimmille: "Nyt tuo lauma syö puhtaaksi kaiken meidän ympäriltämme, niinkuin härkä syö kedon vihannuuden". Ja Baalak, Sipporin poika, oli Mooabin kuninkaana siihen aikaan.
\par 5 Hän lähetti sanansaattajat Bileamin, Beorin pojan, tykö Petoriin, joka on Eufrat-virran varrella, heimolaistensa maahan, kutsumaan häntä ja käski sanoa hänelle: "Katso, Egyptistä on lähtenyt liikkeelle kansa; katso, se on tulvinut yli maan ja asettunut minua lähelle.
\par 6 Tule siis ja kiroa minun puolestani tämä kansa, sillä se on minua väkevämpi; ehkä minä sitten saan sen voitetuksi ja karkoitetuksi maasta. Sillä minä tiedän, että jonka sinä siunaat, se on siunattu, ja jonka sinä kiroat, se on kirottu."
\par 7 Niin Mooabin vanhimmat ja Midianin vanhimmat lähtivät matkaan, tietäjänpalkka mukanansa. Ja kun he saapuivat Bileamin luo, puhuivat he hänelle Baalakin sanat.
\par 8 Hän vastasi heille: "Viipykää täällä tämä yö, niin minä annan teille vastauksen sen mukaan, kuin Herra minulle puhuu". Silloin jäivät mooabilaisten päämiehet Bileamin luo.
\par 9 Niin Jumala tuli Bileamin tykö ja kysyi: "Keitä ovat ne miehet, jotka ovat sinun luonasi?"
\par 10 Bileam vastasi Jumalalle: "Baalak, Sipporin poika, Mooabin kuningas, on lähettänyt heidät tuomaan minulle tämän sanan:
\par 11 'Katso, kansa on lähtenyt Egyptistä ja on tulvinut yli maan; tule siis, kiroa se minun puolestani, ehkä sitten voin ryhtyä taisteluun sitä vastaan ja karkoittaa sen'".
\par 12 Mutta Jumala sanoi Bileamille: "Älä mene heidän kanssaan äläkä kiroa sitä kansaa, sillä se on siunattu".
\par 13 Niin Bileam nousi aamulla ja sanoi Baalakin päämiehille: "Menkää takaisin maahanne, sillä Herra ei ole sallinut minun lähteä teidän kanssanne".
\par 14 Silloin nousivat Mooabin päämiehet ja tulivat Baalakin luo ja sanoivat: "Bileam ei suostunut lähtemään meidän kanssamme".
\par 15 Mutta Baalak lähetti uudestaan päämiehiä matkaan, vielä useampia ja arvokkaampia kuin edelliset.
\par 16 Ja he saapuivat Bileamin luo ja sanoivat hänelle: "Näin sanoo Baalak, Sipporin poika: 'Älä kieltäydy tulemasta minun luokseni.
\par 17 Sillä minä palkitsen sinut ylenpalttisesti ja teen kaiken, mitä minulta vaadit; tule siis ja kiroa minun puolestani tämä kansa.'"
\par 18 Mutta Bileam vastasi ja sanoi Baalakin palvelijoille: "Vaikka Baalak antaisi minulle talonsa täyden hopeata ja kultaa, en sittenkään voisi, en pienessä enkä suuressa, rikkoa Herran, minun Jumalani, käskyä.
\par 19 Mutta jääkää nyt tänne tekin täksi yöksi, saadakseni tietää, mitä Herra vielä minulle sanoo."
\par 20 Niin Jumala tuli Bileamin tykö yöllä ja sanoi hänelle: "Jos nämä miehet ovat tulleet kutsumaan sinua, niin nouse ja lähde heidän kanssaan, mutta tee vain se, mitä minä sinulle sanon".
\par 21 Ja Bileam nousi aamulla, satuloi aasintammansa ja lähti matkaan Mooabin päämiesten kanssa.
\par 22 Mutta kun hän lähti, syttyi Jumalan viha, ja Herran enkeli asettui tielle estämään häntä, hänen ratsastaessaan aasintammallaan, kaksi palvelijaa mukanansa.
\par 23 Kun aasintamma näki Herran enkelin seisovan tiellä, paljastettu miekka kädessänsä, niin se poikkesi tieltä ja meni peltoon. Mutta Bileam löi aasintammaa palauttaaksensa sen tielle.
\par 24 Sitten Herran enkeli asettui solaan viinitarhojen välille, jossa oli kiviaita kummallakin puolella.
\par 25 Kun aasintamma näki Herran enkelin, niin se painautui aitaan ja likisti Bileamin jalkaa aitaa vasten. Ja hän löi sitä vielä kerran.
\par 26 Silloin Herran enkeli taas meni edemmäs ja asettui ahtaaseen paikkaan, jossa ei ollut tilaa väistyä oikealle eikä vasemmalle.
\par 27 Kun aasintamma näki Herran enkelin, laskeutui se maahan Bileamin alla. Silloin Bileamin viha syttyi, ja hän löi aasintammaa sauvalla.
\par 28 Niin Herra avasi aasintamman suun, ja se sanoi Bileamille: "Mitä minä olen sinulle tehnyt, koska lyöt minua jo kolmannen kerran?"
\par 29 Bileam vastasi aasintammalle: "Sinä olet pitänyt minua pilkkanasi. Olisipa minulla miekka kädessäni, niin nyt minä sinut tappaisin."
\par 30 Mutta aasintamma sanoi Bileamille: "Enkö minä ole sinun aasintammasi, jolla olet ratsastanut kaiken aikasi tähän päivään asti? Onko minun ollut tapana tehdä sinulle näin?" Hän vastasi: "Ei".
\par 31 Niin Herra avasi Bileamin silmät, niin että hän näki Herran enkelin seisovan tiellä, paljastettu miekka kädessänsä. Silloin hän kumartui ja heittäytyi kasvoilleen.
\par 32 Niin Herran enkeli sanoi hänelle: "Minkätähden olet lyönyt aasintammaasi jo kolme kertaa? Katso, minä olen tullut sinua estämään, sillä ajattelemattomasti sinä olet lähtenyt tälle matkalle vastoin minun tahtoani.
\par 33 Mutta aasintamma näki minut ja on väistynyt minun edestäni jo kolme kertaa. Ja jos se ei olisi väistynyt minun edestäni, niin minä olisin surmannut sinut, mutta jättänyt sen elämään."
\par 34 Niin Bileam vastasi Herran enkelille: "Minä olen tehnyt syntiä; sillä minä en tiennyt, että sinä olit asettunut minua vastaan tielle. Mutta jos tämä ei ole sinulle mieleen, niin minä nyt palaan takaisin."
\par 35 Mutta Herran enkeli sanoi Bileamille: "Mene näiden miesten kanssa, mutta puhu ainoastaan se, minkä minä sinulle puhun". Niin Bileam lähti Baalakin päämiesten kanssa.
\par 36 Kun Baalak kuuli, että Bileam oli tulossa, meni hän häntä vastaan Iir-Mooabiin, joka on Arnonia pitkin kulkevalla rajalla, uloimmalla rajalla.
\par 37 Ja Baalak sanoi Bileamille: "Enkö minä vartavasten lähettänyt kutsumaan sinua? Minkätähden et tahtonut tulla minun luokseni? Enkö minä muka voi sinua palkita?"
\par 38 Mutta Bileam vastasi Baalakille: "Olenhan minä nyt tullut sinun luoksesi. Mutta onko minun vallassani puhua mitään? Minkä Jumala panee minun suuhuni, sen minä puhun."
\par 39 Niin Bileam lähti Baalakin kanssa, ja he saapuivat Kirjat-Husotiin.
\par 40 Ja Baalak teurasti raavaita ja lampaita ja lähetti ne Bileamille ja päämiehille, jotka olivat hänen luonansa.
\par 41 Mutta seuraavana aamuna Baalak otti Bileamin mukaansa ja vei hänet Baamot-Baalin kukkulalle, josta hän näki reunimmaisen osan tuota kansaa.

\chapter{23}

\par 1 Ja Bileam sanoi Baalakille: "Rakenna minulle tähän seitsemän alttaria ja hanki minulle tänne seitsemän härkää ja seitsemän oinasta".
\par 2 Niin Baalak teki, niinkuin Bileam käski. Ja Baalak ja Bileam uhrasivat härän ja oinaan kullakin alttarilla.
\par 3 Niin Bileam sanoi Baalakille: "Asetu tähän polttouhrisi ääreen, minä menen tuonne; ehkä minä saan kohdata Herran. Minkä hän minulle näyttää, sen minä ilmoitan sinulle." Ja hän meni eräälle autiolle kukkulalle.
\par 4 Ja Jumala tuli ja kohtasi Bileamin. Silloin tämä sanoi hänelle: "Seitsemän alttaria minä olen pannut kuntoon ja olen uhrannut härän ja oinaan kullakin alttarilla".
\par 5 Ja Herra pani sanat Bileamin suuhun ja sanoi: "Palaja Baalakin luo ja puhu hänelle näin".
\par 6 Niin hän palasi hänen luoksensa, ja hän seisoi siinä polttouhrinsa ääressä, hän ja kaikki Mooabin päämiehet.
\par 7 Niin hän puhkesi lausumaan ja sanoi: "Aramista nouti minut Baalak, idän vuorilta Mooabin kuningas: 'Tule, kiroa minun puolestani Jaakob, tule ja sadattele Israelia'.
\par 8 Kuinka minä kiroaisin sen, jota ei Jumala kiroa, kuinka sadattelisin sitä, jota ei Herra sadattele?
\par 9 Minä näen sen kallioiden huipulta, minä katselen sitä kukkuloilta: katso, se on erillään asuva kansa, se ei lukeudu pakanakansojen joukkoon.
\par 10 Kuka mittaa Jaakobin hiekkajyväset, kuka Israelin tomuhiukkasten luvun? Suotakoon minun kuolla oikeamielisten kuolema, olkoon minun loppuni niinkuin heidän."
\par 11 Silloin Baalak sanoi Bileamille: "Mitä oletkaan minulle tehnyt! Minä toin sinut tänne vihollisiani kiroamaan, ja katso, nyt sinä heidät siunaat!"
\par 12 Mutta hän vastasi ja sanoi: "Eikö minun ole tarkoin puhuttava se, minkä Herra panee minun suuhuni?"
\par 13 Niin Baalak sanoi hänelle: "Tule minun kanssani toiseen paikkaan, josta voit nähdä sen kansan, kuitenkin ainoastaan sen äärimmäisen reunan, et sitä kokonaisuudessaan, ja kiroa se minun puolestani sieltä".
\par 14 Ja hän vei hänet Vartijakedolle Pisga-vuoren huipulle ja rakensi siihen seitsemän alttaria ja uhrasi härän ja oinaan kullakin alttarilla.
\par 15 Sitten Bileam sanoi Baalakille: "Asetu tähän polttouhrisi ääreen, ehkä hän kohtaa minua tuolla".
\par 16 Niin Herra tuli kohtaamaan Bileamia, pani sanat hänen suuhunsa ja sanoi: "Palaja Baalakin luo ja puhu hänelle näin".
\par 17 Niin hän tuli hänen luoksensa, ja hän seisoi siinä polttouhrinsa ääressä, Mooabin päämiehet kanssansa. Baalak kysyi häneltä: "Mitä Herra puhui?"
\par 18 Niin hän puhkesi lausumaan ja sanoi: "Nouse, Baalak, ja kuule! Kuuntele minua, sinä Sipporin poika!
\par 19 Ei Jumala ole ihminen, niin että hän valhettelisi, eikä ihmislapsi, että hän katuisi. Sanoisiko hän jotakin eikä sitä tekisi, puhuisiko jotakin eikä sitä täyttäisi?
\par 20 Katso, minä olen saanut tehtäväkseni siunata: hän on siunannut, enkä minä voi sitä peruuttaa.
\par 21 Ei havaita vaivaa Jaakobissa eikä nähdä onnettomuutta Israelissa; Herra, hänen Jumalansa, on hänen kanssansa, riemuhuuto kuninkaalle kaikuu siellä.
\par 22 Jumala vei sen pois Egyptistä; sen sarvet ovat kuin villihärän.
\par 23 Sillä ei ole noituutta Jaakobissa eikä tavata taikuutta Israelissa. Aikanansa ilmoitetaan Jaakobille ja Israelille, mitä Jumala on tekevä.
\par 24 Katso, se on kansa, joka nousee kuin naarasleijona, joka kohoaa kuin leijona. Ei se paneudu levolle, ennenkuin on saalista syönyt ja juonut surmattujen verta."
\par 25 Mutta Baalak sanoi Bileamille: "Älä kiroa sitä äläkä siunaa sitä".
\par 26 Mutta Bileam vastasi ja sanoi Baalakille: "Enkö minä puhunut sinulle näin: 'Kaikki, mitä Herra sanoo, on minun tehtävä'?"
\par 27 Niin Baalak sanoi Bileamille: "Tule, minä vien sinut toiseen paikkaan; ehkäpä on Jumalan silmissä otollista, että kiroat kansan minun puolestani sieltä".
\par 28 Ja Baalak vei Bileamin Peor-vuoren huipulle, joka kohoaa yli erämaan.
\par 29 Sitten Bileam sanoi Baalakille: "Rakenna minulle tähän seitsemän alttaria ja hanki minulle tänne seitsemän härkää ja seitsemän oinasta".
\par 30 Ja Baalak teki, niinkuin Bileam käski, ja uhrasi härän ja oinaan kullakin alttarilla.

\chapter{24}

\par 1 Kun Bileam näki, että Israelin siunaaminen oli Herralle otollista, ei hän enää mennyt niinkuin aina ennen ennusmerkkejä etsimään, vaan käänsi kasvonsa erämaahan.
\par 2 Ja kun Bileam nosti silmänsä ja näki Israelin leiriytyneenä heimokunnittain, niin Jumalan Henki tuli häneen.
\par 3 Ja hän puhkesi lausumaan ja sanoi: "Näin puhuu Bileam, Beorin poika, näin puhuu mies, jonka silmä on avattu.
\par 4 Näin puhuu hän, joka kuulee Jumalan puheen, joka näkee Kaikkivaltiaan näkyjä, joka lankeaa loveen ja jonka silmät avataan.
\par 5 Kuinka ihanat ovat sinun majasi, Jaakob, sinun asuinsijasi, Israel!
\par 6 Niinkuin laajat purolaaksot, niinkuin puutarhat virran varrella, niinkuin aloepuut, Herran istuttamat, niinkuin setripuut vesien vierillä!
\par 7 Vettä läikkyy sen vesisangoista, ja sen laihot saavat runsaasti vettä. Agagia mahtavampi on sen kuningas, ylhäinen sen kuningasvalta.
\par 8 Jumala vei sen pois Egyptistä; sen sarvet ovat kuin villihärän. Se syö suuhunsa viholliskansat, heidän luunsa se murskaa ja lävistää heidät nuolillansa.
\par 9 Se on kyyristynyt, se on laskeutunut maahan kuin leijona, niinkuin naarasleijona - kuka uskaltaa sitä häiritä? Siunattu olkoon, joka sinua siunaa, kirottu, joka sinua kiroaa!"
\par 10 Silloin Baalak vihastui Bileamiin ja löi kätensä yhteen. Ja Baalak sanoi Bileamille: "Minä kutsuin sinut vihollisiani kiroamaan, ja nyt sinä olet siunannut heidät jo kolme kertaa.
\par 11 Mene tiehesi! Minä aioin sinua runsaasti palkita, mutta katso, Herra on sinulta palkan kieltänyt."
\par 12 Bileam vastasi Baalakille: "Enkö minä jo sanonut sinun sanansaattajillesi, jotka lähetit luokseni:
\par 13 'Vaikka Baalak antaisi minulle talonsa täyden hopeata ja kultaa, en sittenkään voisi rikkoa Herran käskyä, en tehdä mitään oman mieleni mukaan, en hyvää, en pahaa'. Minkä Herra puhuu, sen minäkin puhun.
\par 14 Ja nyt minä lähden kansani tykö, mutta sitä ennen minä ilmoitan sinulle, mitä tämä kansa on tekevä sinun kansallesi aikojen lopulla."
\par 15 Ja hän puhkesi lausumaan ja sanoi: "Näin puhuu Bileam, Beorin poika, näin puhuu mies, jonka silmä on avattu.
\par 16 Näin puhuu hän, joka kuulee Jumalan puheen ja saa tietonsa Korkeimmalta, joka näkee Kaikkivaltiaan näkyjä, joka lankeaa loveen ja jonka silmät avataan:
\par 17 Minä näen hänet, en kuitenkaan nyt, minä katselen häntä, en kuitenkaan läheltä. Tähti nousee Jaakobista, ja valtikka kohoaa Israelista. Se ruhjoo Mooabilta ohimot, päälaen kaikilta Seetin pojilta.
\par 18 Ja Edomista tulee alusmaa, Seir joutuu vihollistensa omaksi; mutta Israel tekee väkeviä tekoja.
\par 19 Ja Jaakobista tulee valtias, hän hävittää kaupungeista niihin pelastuneet."
\par 20 Ja kun hän näki Amalekin, niin hän puhkesi lausumaan ja sanoi: "Kansakunnista ensimmäinen on Amalek, mutta sen loppu on perikato".
\par 21 Ja kun hän näki keeniläiset, niin hän puhkesi lausumaan ja sanoi: "Järkkymätön on sinun asumuksesi, ja kalliolle on pesäsi pantu.
\par 22 Mutta sittenkin Kain hävitetään: ei aikaakaan, niin Assur vie sinut vangiksi."
\par 23 Ja hän puhkesi lausumaan ja sanoi: "Voi! Kuka jää enää elämään, kun Jumala tämän tekee!
\par 24 Laivoja saapuu kittiläisten suunnalta, ja ne kurittavat Assurin ja kurittavat Eeberin. Hänkin on perikadon oma."
\par 25 Senjälkeen Bileam nousi, lähti matkaan ja palasi kotiinsa; ja myöskin Baalak lähti tiehensä.

\chapter{25}

\par 1 Niin Israel asettui Sittimiin. Ja kansa rupesi irstailemaan Mooabin tyttärien kanssa.
\par 2 Nämä kutsuivat kansaa jumaliensa uhreille, ja kansa söi ja kumarsi heidän jumaliansa.
\par 3 Kun Israel näin antautui palvelemaan Baal-Peoria, syttyi Herran viha Israelia kohtaan.
\par 4 Ja Herra sanoi Moosekselle: "Ota kansan kaikki päämiehet ja lävistä heidät paaluihin Herralle, vasten aurinkoa, että Herran vihan hehku kääntyisi pois Israelista".
\par 5 Niin Mooses sanoi Israelin tuomareille: "Surmatkoon jokainen miehistään ne, jotka ovat antautuneet palvelemaan Baal-Peoria".
\par 6 Ja katso, muuan mies israelilaisista tuli ja toi veljiensä luo midianilaisen naisen, Mooseksen ja kaiken israelilaisten seurakunnan nähden, kun nämä olivat itkemässä ilmestysmajan ovella.
\par 7 Kun Piinehas, pappi Aaronin pojan Eleasarin poika, näki sen, nousi hän kansan keskeltä ja otti keihään käteensä
\par 8 ja seurasi tuota Israelin miestä makuusuojaan ja lävisti heidät molemmat, Israelin miehen ja sen naisen, vatsan kohdalta. Silloin taukosi israelilaisilta vitsaus.
\par 9 Ja niitä, jotka kuolivat tässä vitsauksessa, oli kaksikymmentäneljä tuhatta.
\par 10 Ja Herra puhui Moosekselle sanoen:
\par 11 "Piinehas, pappi Aaronin pojan Eleasarin poika, on kääntänyt minun vihani pois israelilaisista, kun hän on kiivaillut minun puolestani heidän keskuudessaan, niin ettei minun ole tarvinnut kiivaudessani tehdä loppua israelilaisista.
\par 12 Sano siis: Katso, minä teen hänen kanssansa liiton hänen menestymiseksensä.
\par 13 Ja se tulee hänelle ja hänen jälkeläisillensä ikuiseksi pappeuden liitoksi, koska hän kiivaili Jumalansa puolesta ja toimitti sovituksen israelilaisille."
\par 14 Ja surmatun Israelin miehen nimi, sen, joka surmattiin midianilaisen naisen kanssa, oli Simri, simeonilaisen perhekunta-päämiehen Saalun poika.
\par 15 Ja surmatun midianilais-naisen nimi oli Kosbi, Suurin, midianilaisen perhekunnan heimopäällikön, tytär.
\par 16 Ja Herra puhui Moosekselle sanoen:
\par 17 "Ahdistakaa midianilaisia ja tuhotkaa heidät.
\par 18 Sillä he ahdistivat teitä viekkailla juonillansa, joilla he houkuttelivat teidät Peorin ja sisarensa Kosbin, midianilaisen päämiehen tyttären, ansaan, hänen, joka surmattiin Peorin vuoksi tapahtuneen vitsauksen päivänä."

\chapter{26}

\par 1 Tämän vitsauksen jälkeen Herra puhui Moosekselle ja Eleasarille, pappi Aaronin pojalle, sanoen:
\par 2 "Laskekaa koko israelilaisten seurakunnan väkiluku, kaksikymmenvuotiset ja sitä vanhemmat, perhekunnittain, kaikki Israelin sotakelpoiset miehet".
\par 3 Ja Mooses ja pappi Eleasar puhuivat heille Mooabin arolla Jordanin luona, Jerikon kohdalla, sanoen:
\par 4 "Kaksikymmenvuotiset ja sitä vanhemmat laskettakoon", niinkuin Herra oli Moosekselle käskyn antanut. Ja israelilaiset, jotka olivat lähteneet Egyptin maasta, olivat:
\par 5 Ruuben, Israelin esikoinen; Ruubenin jälkeläisiä olivat: Hanokista hanokilaisten suku, Pallusta pallulaisten suku,
\par 6 Hesronista hesronilaisten suku, Karmista karmilaisten suku.
\par 7 Nämä olivat ruubenilaisten suvut. Ja katselmuksessa olleita oli heitä neljäkymmentäkolme tuhatta seitsemänsataa kolmekymmentä.
\par 8 Ja Pallun poika oli Eliab.
\par 9 Ja Eliabin pojat olivat Nemuel ja Daatan ja Abiram. Nämä olivat ne kansan edusmiehet Daatan ja Abiram, jotka taistelivat Moosesta ja Aaronia vastaan Koorahin joukossa ja taistelivat Herraa vastaan,
\par 10 jolloin maa avasi kitansa ja nielaisi heidät ja Koorahin, silloin kun hänen joukkonsa kuoli ja tuli kulutti kaksisataa viisikymmentä miestä; ja heistä tuli varoitusmerkki.
\par 11 Mutta Koorahin pojat eivät kuolleet.
\par 12 Simeonin jälkeläisiä, sukujensa mukaan, olivat: Nemuelista nemuelilaisten suku, Jaaminista jaaminilaisten suku, Jaakinista jaakinilaisten suku,
\par 13 Serahista serahilaisten suku, Saulista saulilaisten suku.
\par 14 Nämä olivat simeonilaisten suvut, kaksikymmentäkaksi tuhatta kaksisataa.
\par 15 Gaadin jälkeläisiä, sukujensa mukaan, olivat: Sefonista sefonilaisten suku, Haggista haggilaisten suku, Suunista suunilaisten suku,
\par 16 Osnista osnilaisten suku, Eeristä eeriläisten suku,
\par 17 Arodista arodilaisten suku, Arelista arelilaisten suku.
\par 18 Nämä olivat Gaadin jälkeläisten suvut; katselmuksessa olleita oli heitä neljäkymmentä tuhatta viisisataa.
\par 19 Juudan poikia olivat Eer ja Oonan, mutta Eer ja Oonan kuolivat Kanaanin maassa.
\par 20 Ja Juudan jälkeläisiä, sukujensa mukaan, olivat: Seelasta seelalaisten suku, Pereksestä perekseläisten suku, Serahista serahilaisten suku.
\par 21 Ja Pereksen jälkeläisiä olivat: Hesronista hesronilaisten suku, Haamulista haamulilaisten suku.
\par 22 Nämä olivat Juudan suvut; katselmuksessa olleita oli heitä seitsemänkymmentäkuusi tuhatta viisisataa.
\par 23 Isaskarin jälkeläisiä, sukujensa mukaan, olivat: Toolasta toolalaisten suku, Puvvasta puunilaisten suku,
\par 24 Jaasubista jaasubilaisten suku, Simronista simronilaisten suku.
\par 25 Nämä olivat Isaskarin suvut; katselmuksessa olleita oli heitä kuusikymmentäneljä tuhatta kolmesataa.
\par 26 Sebulonin jälkeläisiä, sukujensa mukaan, olivat: Seredistä serediläisten suku, Eelonista eelonilaisten suku, Jahlelista jahlelilaisten suku.
\par 27 Nämä olivat sebulonilaisten suvut; katselmuksessa olleita oli heitä kuusikymmentä tuhatta viisisataa.
\par 28 Joosefin jälkeläisiä, sukujensa mukaan, olivat Manasse ja Efraim.
\par 29 Manassen jälkeläisiä olivat: Maakirista maakirilaisten suku. Ja Maakirille syntyi Gilead; Gileadista on gileadilaisten suku.
\par 30 Nämä olivat Gileadin jälkeläisiä: Iieseristä iieseriläisten suku, Helekistä helekiläisten suku;
\par 31 Asrielista asrielilaisten suku, Sekemistä sekemiläisten suku;
\par 32 Semidasta semidalaisten suku, Heeferistä heeferiläisten suku.
\par 33 Mutta Selofhadilla, Heeferin pojalla, ei ollut poikia, vaan ainoastaan tyttäriä. Ja Selofhadin tyttärien nimet olivat: Mahla, Nooga, Hogla, Milka ja Tirsa.
\par 34 Nämä olivat Manassen suvut; katselmuksessa olleita oli heitä viisikymmentäkaksi tuhatta seitsemänsataa.
\par 35 Ja nämä olivat Efraimin jälkeläisiä, sukujensa mukaan: Suutelahista suutelahilaisten suku, Bekeristä bekeriläisten suku, Tahanista tahanilaisten suku.
\par 36 Nämä olivat Suutelahin jälkeläisiä: Eeranista eeranilaisten suku.
\par 37 Nämä olivat efraimilaisten suvut; katselmuksessa olleita oli heitä kolmekymmentäkaksi tuhatta viisisataa. Nämä olivat Joosefin jälkeläiset sukujensa mukaan.
\par 38 Benjaminin jälkeläisiä, sukujensa mukaan, olivat: Belasta belalaisten suku, Asbelista asbelilaisten suku, Ahiramista ahiramilaisten suku,
\par 39 Sefufamista suufamilaisten suku, Huufamista huufamilaisten suku.
\par 40 Mutta Belan jälkeläisiä olivat Ard ja Naaman: Ardista ardilaisten suku, Naamanista naamilaisten suku.
\par 41 Nämä olivat Benjaminin jälkeläiset sukujensa mukaan; ja katselmuksessa olleita oli heitä neljäkymmentäviisi tuhatta kuusisataa.
\par 42 Nämä olivat Daanin jälkeläisiä, sukujensa mukaan: Suuhamista suuhamilaisten suku. Nämä ovat Daanin suvut sukujensa mukaan.
\par 43 Kaikista suuhamilaisten suvuista oli katselmuksessa olleita kuusikymmentäneljä tuhatta neljäsataa.
\par 44 Asserin jälkeläisiä, sukujensa mukaan, olivat: Jimnasta jimnalaisten suku, Jisvistä jisviläisten suku, Beriasta berialaisten suku.
\par 45 Berian jälkeläisiä olivat: Heeberistä heeberiläisten suku, Malkielista malkielilaisten suku.
\par 46 Mutta Asserin tyttären nimi oli Serah.
\par 47 Nämä olivat asserilaisten suvut; katselmuksessa olleita oli heitä viisikymmentäkolme tuhatta neljäsataa.
\par 48 Naftalin jälkeläisiä, sukujensa mukaan, olivat: Jahselista jahselilaisten suku, Guunista guunilaisten suku;
\par 49 Jeeseristä jeeseriläisten suku, Sillemistä sillemiläisten suku.
\par 50 Nämä olivat Naftalin suvut sukujensa mukaan; ja katselmuksessa olleita oli heitä neljäkymmentäviisi tuhatta neljäsataa.
\par 51 Nämä olivat katselmuksessa olleet israelilaiset: kuusisataayksi tuhatta seitsemänsataa kolmekymmentä.
\par 52 Ja Herra puhui Moosekselle sanoen:
\par 53 "Näille jaettakoon maa perintöosiksi nimien lukumäärän mukaan.
\par 54 Isommalle suvulle anna isompi perintöosa ja pienemmälle pienempi perintöosa; kullekin annettakoon perintöosa katselmuksessa olleiden määrän mukaan.
\par 55 Mutta maa jaettakoon arvalla. Isiensä heimojen nimien mukaan he saakoot perintöosansa.
\par 56 Arvan määräyksen mukaan jaettakoon perintöosat suurilukuisten ja vähälukuisten heimojen kesken."
\par 57 Ja nämä olivat Leevin katselmuksessa olleet, sukujensa mukaan: Geersonista geersonilaisten suku, Kehatista kehatilaisten suku, Merarista merarilaisten suku.
\par 58 Nämä olivat leeviläisten suvut: libniläisten suku, hebronilaisten suku, mahlilaisten suku, muusilaisten suku, koorahilaisten suku. Ja Kehatille oli syntynyt Amram.
\par 59 Ja Amramin vaimon nimi oli Jookebed, Leevin tytär, joka oli Leeville syntynyt Egyptissä. Ja hän synnytti Amramille Aaronin ja Mooseksen sekä heidän sisarensa Mirjamin.
\par 60 Ja Aaronille syntyi Naadab ja Abihu, Eleasar ja Iitamar.
\par 61 Mutta Naadab ja Abihu kuolivat, kun toivat vierasta tulta Herran eteen.
\par 62 Ja katselmuksessa olleita oli heitä kaksikymmentäkolme tuhatta, kaikki miehenpuolia, kuukauden vanhoja ja sitä vanhempia. Sillä heistä ei pidetty katselmusta yhdessä muiden israelilaisten kanssa, koska heille ei annettu perintöosaa israelilaisten keskuudessa.
\par 63 Nämä olivat siinä israelilaisten katselmuksessa, jonka Mooses ja pappi Eleasar pitivät Mooabin arolla lähellä Jordania, Jerikon kohdalla.
\par 64 Eikä heidän joukossaan ollut ketään siinä israelilaisten katselmuksessa ollutta, jonka Mooses ja pappi Aaron olivat pitäneet Siinain erämaassa.
\par 65 Sillä Herra oli heille sanonut, että heidän oli kuoltava erämaassa. Eikä heistä jäänyt ketään, paitsi Kaaleb, Jefunnen poika, ja Joosua, Nuunin poika.

\chapter{27}

\par 1 Niin astuivat esiin Selofhadin tyttäret - Selofhadin, Heeferin pojan, Gileadin pojan, Maakirin pojan, Manassen pojan, tyttäret - jotka olivat Manassen, Joosefin pojan, suvuista ja joiden nimet olivat Mahla, Nooga, Hogla, Milka ja Tirsa,
\par 2 ja he asettuivat Mooseksen ja pappi Eleasarin ja päämiesten ja kaiken kansan eteen ilmestysmajan ovelle ja sanoivat:
\par 3 "Meidän isämme kuoli erämaassa, mutta hän ei ollut siinä joukossa, joka kävi kapinoimaan Herraa vastaan, Koorahin joukossa, vaan hän kuoli oman syntinsä tähden, eikä hänellä ollut poikia.
\par 4 Miksi häviäisi nyt meidän isämme nimi hänen sukunsa keskuudesta sentähden, että hänellä ei ollut poikaa? Anna meille perintöosuus isämme veljien keskuudessa."
\par 5 Ja Mooses saattoi heidän asiansa Herran eteen.
\par 6 Ja Herra puhui Moosekselle sanoen:
\par 7 "Selofhadin tyttäret puhuvat oikein. Anna heille perintöosuus heidän isänsä veljien keskuudessa ja siirrä heidän isänsä perintöosa heille.
\par 8 Ja puhu israelilaisille sanoen: Jos joku kuolee pojatonna, niin siirtäkää hänen perintöosansa hänen tyttärellensä.
\par 9 Mutta jos hänellä ei ole tytärtä, niin antakaa hänen perintöosansa hänen veljillensä.
\par 10 Mutta jos hänellä ei ole veljiä, niin antakaa hänen perintöosansa hänen isänsä veljille.
\par 11 Mutta jos hänen isällänsä ei ole veljiä, niin antakaa hänen perintöosansa sille hänen veriheimolaisellensa, joka hänen suvussansa on häntä lähin; tämä ottakoon sen haltuunsa. Ja tämä olkoon oikeussäädöksenä israelilaisilla, niinkuin Herra on Moosekselle käskyn antanut."
\par 12 Ja Herra sanoi Moosekselle: "Nouse tälle Abarimin vuorelle ja katsele sitä maata, jonka minä annan israelilaisille.
\par 13 Ja kun olet sitä katsellut, niin sinutkin otetaan pois heimosi tykö, niinkuin veljesi Aaron otettiin,
\par 14 sentähden että te Siinin erämaassa, kansan riidellessä, niskoittelitte minun käskyjäni vastaan ettekä pitäneet minua pyhänä heidän silmiensä edessä, silloin kun vettä hankitte." Se oli Meriban vesi Kaadeksessa Siinin erämaassa.
\par 15 Ja Mooses puhui Herralle sanoen:
\par 16 "Asettakoon Herra, Jumala, jonka vallassa on kaiken lihan henki, tämän seurakunnan johtoon miehen,
\par 17 joka heidän edellänsä lähtee ja joka heidän edellänsä tulee, joka saattaa heidät lähtemään ja tulemaan, ettei Herran seurakunta olisi niinkuin lammaslauma, jolla ei ole paimenta".
\par 18 Ja Herra sanoi Moosekselle: "Ota Joosua, Nuunin poika, mies, jossa on Henki, ja pane kätesi hänen päällensä.
\par 19 Ja aseta hänet pappi Eleasarin ja kaiken kansan eteen ja aseta hänet virkaan heidän nähtensä.
\par 20 Ja siirrä hänelle osa omaa arvoasi, että kaikki israelilaisten seurakunta häntä tottelisi.
\par 21 Ja hän astukoon pappi Eleasarin eteen, ja tämä kysyköön hänen puolestaan uurimin kautta Jumalan vastausta Herran edessä. Hänen käskystänsä he lähtekööt ja hänen käskystänsä he tulkoot, hän ja kaikki israelilaiset hänen kanssansa, koko seurakunta."
\par 22 Ja Mooses teki, niinkuin Herra oli häntä käskenyt, ja otti Joosuan ja asetti hänet pappi Eleasarin ja kaiken seurakunnan eteen.
\par 23 Ja hän pani kätensä hänen päällensä ja asetti hänet virkaan, niinkuin Herra oli puhunut Mooseksen kautta.

\chapter{28}

\par 1 Ja Herra puhui Moosekselle sanoen:
\par 2 "Käske israelilaisia ja sano heille: Pitäkää vaari siitä, että tuotte minulle minun uhrilahjani, ruokani, suloisesti tuoksuvat uhrini, määräaikanaan.
\par 3 Ja sano heille: Tämä on se uhri, joka teidän on tuotava joka päivä Herralle: kaksi vuoden vanhaa virheetöntä karitsaa jokapäiväiseksi polttouhriksi.
\par 4 Uhraa toinen karitsa aamulla, ja toinen karitsa uhraa iltahämärässä
\par 5 ja ruokauhriksi kymmenesosa eefa-mittaa lestyjä jauhoja, sekoitettuna neljännekseen hiin-mittaa survomalla saatua öljyä.
\par 6 Tämä on jokapäiväinen polttouhri, joka toimitettiin Siinain vuorella suloisesti tuoksuvaksi uhriksi Herralle.
\par 7 Ja siihen kuulukoon juomauhrina neljännes hiin-mittaa viiniä kumpaakin karitsaa kohti. Pyhäkössä vuodata Herralle höysteviiniuhri.
\par 8 Ja toinen karitsa uhraa iltahämärässä ja toimita siihen kuuluva ruoka- ja juomauhri samalla tavoin kuin aamullakin, suloisesti tuoksuvaksi uhriksi Herralle.
\par 9 Mutta sapatinpäivänä uhraa kaksi vuoden vanhaa virheetöntä karitsaa ynnä kaksi kymmenennestä lestyjä jauhoja, öljyyn sekoitettuna, ruokauhriksi, sekä siihen kuuluva juomauhri.
\par 10 Tämä on sapatin polttouhri, joka uhrattakoon kunakin sapattina jokapäiväisen polttouhrin ja siihen kuuluvan juomauhrin ohella.
\par 11 Ja kuukausienne ensimmäisenä päivänä tuokaa polttouhriksi Herralle kaksi mullikkaa, oinas ja seitsemän virheetöntä vuoden vanhaa karitsaa
\par 12 sekä kolme kymmenennestä lestyjä jauhoja, öljyyn sekoitettuna, ruokauhriksi kutakin mullikkaa kohti ynnä kaksi kymmenennestä lestyjä jauhoja, öljyyn sekoitettuna, ruokauhriksi oinasta kohti,
\par 13 sekä yksi kymmenennes lestyjä jauhoja, öljyyn sekoitettuna, ruokauhriksi kutakin karitsaa kohti. Tämä on polttouhri, suloisesti tuoksuva uhri Herralle.
\par 14 Ja niihin kuulukoon juomauhrina puoli hiin-mittaa viiniä kutakin mullikkaa kohti ja kolmannes hiin-mittaa oinasta kohti ja neljännes hiin-mittaa karitsaa kohti. Tämä on uudenkuun polttouhri, joka uhrattakoon kunakin uutenakuuna, vuoden kaikkina kuukausina.
\par 15 Ja sitäpaitsi uhratkaa yksi kauris syntiuhriksi Herralle. Se uhrattakoon jokapäiväisen polttouhrin ynnä siihen kuuluvan juomauhrin ohella.
\par 16 Ja ensimmäisessä kuussa, kuukauden neljäntenätoista päivänä, on pääsiäinen Herran kunniaksi.
\par 17 Ja saman kuukauden viidentenätoista päivänä on juhla; seitsemän päivää syötäköön happamatonta leipää.
\par 18 Ensimmäisenä päivänä on pyhä kokous; älkää silloin yhtäkään arkiaskaretta toimittako.
\par 19 Ja tuokaa polttouhriksi Herralle kaksi mullikkaa, oinas ja seitsemän vuodenvanhaa karitsaa; ne olkoot virheettömät.
\par 20 Ja niihin kuuluvana ruokauhrina uhratkaa lestyjä jauhoja, öljyyn sekoitettuna, kolme kymmenennestä mullikkaa kohti ja kaksi kymmenennestä oinasta kohti.
\par 21 Kutakin kohti niistä seitsemästä karitsasta uhratkaa kymmenennes.
\par 22 Ja vielä uhratkaa kauris syntiuhriksi, joka tuottaa teille sovituksen.
\par 23 Paitsi aamupolttouhria, joka on jokapäiväinen polttouhri, uhratkaa nämä.
\par 24 Samoin uhratkaa Herralle joka päivä seitsemän päivän aikana leipää suloisesti tuoksuvaksi uhriksi; uhratkaa tämä jokapäiväisen polttouhrin ynnä siihen kuuluvan juomauhrin ohella.
\par 25 Ja seitsemäntenä päivänä olkoon teillä pyhä kokous; älkää silloin yhtäkään arkiaskaretta toimittako.
\par 26 Mutta uutisen päivänä, kun tuotte uuden ruokauhrin Herralle viikkojuhlananne, olkoon teillä pyhä kokous; älkää silloin yhtäkään arkiaskaretta toimittako.
\par 27 Ja tuokaa suloisesti tuoksuvaksi polttouhriksi Herralle kaksi mullikkaa, oinas ja seitsemän vuodenvanhaa karitsaa
\par 28 ja niihin kuuluvana ruokauhrina lestyjä jauhoja, öljyyn sekoitettuna, kolme kymmenennestä kutakin mullikkaa kohti, kaksi kymmenennestä oinasta kohti
\par 29 ja yksi kymmenennes kutakin kohti niistä seitsemästä karitsasta;
\par 30 ja yksi kauris sovittamiseksenne.
\par 31 Paitsi jokapäiväistä polttouhria ja siihen kuuluvaa ruokauhria uhratkaa nämä ynnä niihin kuuluvat juomauhrit; eläimet olkoot virheettömät."

\chapter{29}

\par 1 "Ja seitsemännessä kuussa, kuukauden ensimmäisenä päivänä, olkoon teillä pyhä kokous; älkää silloin yhtäkään arkiaskaretta toimittako. Olkoon se teille pasunansoiton päivä.
\par 2 Ja uhratkaa suloisesti tuoksuvaksi polttouhriksi Herralle mullikka, oinas ja seitsemän vuodenvanhaa virheetöntä karitsaa
\par 3 ja niihin kuuluvana ruokauhrina lestyjä jauhoja, öljyyn sekoitettuna, kolme kymmenennestä mullikkaa kohti, kaksi kymmenennestä oinasta kohti
\par 4 ja yksi kymmenennes kutakin kohti niistä seitsemästä karitsasta;
\par 5 ja kauris syntiuhriksi, teidän sovittamiseksenne.
\par 6 Paitsi uudenkuun polttouhria ynnä siihen kuuluvaa ruokauhria ja jokapäiväistä polttouhria ynnä siihen kuuluvaa ruokauhria ja niihin kuuluvia juomauhreja, niinkuin niistä on säädetty, uhratkaa kaikki tämä suloisesti tuoksuvaksi uhriksi Herralle.
\par 7 Ja saman seitsemännen kuun kymmenentenä päivänä olkoon teillä pyhä kokous, ja silloin kurittakaa itseänne paastolla, älkääkä yhtäkään askaretta toimittako.
\par 8 Ja tuokaa suloisesti tuoksuvaksi polttouhriksi Herralle mullikka, oinas ja seitsemän vuodenvanhaa karitsaa; ne olkoot virheettömät.
\par 9 Ja niihin kuulukoon ruokauhrina lestyjä jauhoja, öljyyn sekoitettuna, kolme kymmenennestä mullikkaa kohti, kaksi kymmenennestä oinasta kohti
\par 10 ja yksi kymmenennes kutakin kohti niistä seitsemästä karitsasta;
\par 11 ja kauris syntiuhriksi, paitsi sovitusuhria ja jokapäiväistä polttouhria ynnä niihin kuuluvia ruoka- ja juomauhreja.
\par 12 Ja seitsemännen kuukauden viidentenätoista päivänä olkoon teillä pyhä kokous; älkää silloin yhtäkään arkiaskaretta toimittako, vaan viettäkää juhlaa Herran kunniaksi seitsemän päivää.
\par 13 Ja tuokaa polttouhriksi, suloisesti tuoksuvaksi uhriksi Herralle, kolmetoista mullikkaa, kaksi oinasta ja neljätoista vuodenvanhaa karitsaa; olkoot ne virheettömät.
\par 14 Ja kuulukoon niihin ruokauhrina lestyjä jauhoja, öljyyn sekoitettuna, kolme kymmenennestä kutakin kohti niistä kolmestatoista mullikasta, kaksi kymmenennestä kumpaakin kohti niistä kahdesta oinaasta
\par 15 ja yksi kymmenennes kutakin kohti niistä neljästätoista karitsasta;
\par 16 ja kauris syntiuhriksi, paitsi jokapäiväistä polttouhria ynnä siihen kuuluvaa ruoka- ja juomauhria.
\par 17 Ja toisena päivänä: kaksitoista mullikkaa, kaksi oinasta ja neljätoista vuodenvanhaa, virheetöntä karitsaa
\par 18 ja niihin kuuluvat ruoka- ja juomauhrit mullikkain, oinaiden ja karitsain luvun mukaan, niinkuin säädetty on,
\par 19 sekä kauris syntiuhriksi, paitsi jokapäiväistä polttouhria ynnä siihen kuuluvaa ruoka- ja juomauhria.
\par 20 Ja kolmantena päivänä: yksitoista mullikkaa, kaksi oinasta ja neljätoista vuodenvanhaa, virheetöntä karitsaa
\par 21 ja niihin kuuluvat ruoka- ja juomauhrit mullikkain, oinaiden ja karitsain luvun mukaan, niinkuin säädetty on,
\par 22 sekä kauris syntiuhriksi, paitsi jokapäiväistä polttouhria ynnä siihen kuuluvaa ruoka- ja juomauhria.
\par 23 Ja neljäntenä päivänä: kymmenen mullikkaa, kaksi oinasta ja neljätoista vuodenvanhaa, virheetöntä karitsaa
\par 24 ja niihin kuuluvat ruoka- ja juomauhrit mullikkain, oinaiden ja karitsain luvun mukaan, niinkuin säädetty on,
\par 25 sekä kauris syntiuhriksi, paitsi jokapäiväistä polttouhria ynnä siihen kuuluvaa ruoka- ja juomauhria.
\par 26 Ja viidentenä päivänä: yhdeksän mullikkaa, kaksi oinasta ja neljätoista vuodenvanhaa, virheetöntä karitsaa
\par 27 ja niihin kuuluvat ruoka- ja juomauhrit mullikkain, oinaiden ja karitsain luvun mukaan, niinkuin säädetty on,
\par 28 sekä kauris syntiuhriksi, paitsi jokapäiväistä polttouhria ynnä siihen kuuluvaa ruoka- ja juomauhria.
\par 29 Ja kuudentena päivänä: kahdeksan mullikkaa, kaksi oinasta ja neljätoista vuodenvanhaa, virheetöntä karitsaa
\par 30 ja niihin kuuluvat ruoka- ja juomauhrit mullikkain, oinaiden ja karitsain luvun mukaan, niinkuin säädetty on,
\par 31 sekä kauris syntiuhriksi, paitsi jokapäiväistä polttouhria ynnä siihen kuuluvaa ruoka- ja juomauhria.
\par 32 Ja seitsemäntenä päivänä: seitsemän mullikkaa, kaksi oinasta ja neljätoista vuodenvanhaa, virheetöntä karitsaa
\par 33 ja niihin kuuluvat ruoka- ja juomauhrit mullikkain, oinaiden ja karitsain luvun mukaan, niinkuin säädetty on,
\par 34 sekä kauris syntiuhriksi, paitsi jokapäiväistä polttouhria ynnä siihen kuuluvaa ruoka- ja juomauhria.
\par 35 Ja kahdeksantena päivänä olkoon teillä juhlakokous; älkää silloin yhtäkään arkiaskaretta toimittako,
\par 36 vaan tuokaa suloisesti tuoksuvaksi polttouhriksi Herralle mullikka, oinas ja seitsemän vuodenvanhaa, virheetöntä karitsaa
\par 37 ja niihin kuuluvat ruoka- ja juomauhrit mullikan, oinaan ja karitsain luvun mukaan, niinkuin säädetty on,
\par 38 sekä kauris syntiuhriksi, paitsi jokapäiväistä polttouhria ynnä siihen kuuluvaa ruoka- ja juomauhria.
\par 39 Nämä uhratkaa Herralle juhla-aikoinanne, paitsi mitä lupausuhreinanne ja vapaaehtoisina lahjoinanne tuotte polttouhreiksi, ruokauhreiksi, juomauhreiksi ja yhteysuhreiksi."

\chapter{30}

\par 1 Ja Mooses puhui israelilaisille kaiken, mitä Herra oli Mooseksen käskenyt puhua.
\par 2 Ja Mooses puhui Israelin sukukuntien päämiehille sanoen: "Näin on Herra käskenyt:
\par 3 Jos joku tekee lupauksen Herralle tahi valalla vannoen sitoutuu kieltäytymään jostakin, älköön hän rikkoko sanaansa, vaan tehköön kaiken, mitä hänen suunsa on sanonut.
\par 4 Ja jos nainen ollessaan nuorena tyttönä isänsä kodissa tekee lupauksen Herralle tahi sitoutuu kieltäytymään jostakin
\par 5 ja hänen isänsä kuulee hänen lupauksensa tai sitoumuksensa eikä hänen isänsä hänelle siitä mitään puhu, niin kaikki hänen lupauksensa ja kaikki hänen sitoumuksensa, joilla hän kieltäytyy jostakin, olkoot pätevät.
\par 6 Mutta jos hänen isänsä kieltää häntä sinä päivänä, jona hän ne kuulee, niin hänen lupauksensa ja sitoumuksensa, joilla hän kieltäytyy jostakin, älkööt olko pätevät, olivatpa ne mitkä hyvänsä, ja Herra antaa hänelle anteeksi, sentähden että hänen isänsä on häntä kieltänyt.
\par 7 Ja jos hän joutuu miehelle ja hänellä on täytettävänä lupauksia tai ajattelemattomasti lausuttu sana, jolla hän on kieltäytynyt jostakin,
\par 8 ja hänen miehensä saa sen kuulla, mutta ei puhu hänelle siitä mitään sinä päivänä, jona hän sen kuulee, niin hänen lupauksensa ja hänen sitoumuksensa, joilla hän on kieltäytynyt jostakin, olkoot pätevät.
\par 9 Mutta jos hänen miehensä sinä päivänä, jona hän sen kuulee, kieltää häntä, niin hän purkaa hänen lupauksensa, jonka täyttämiseen hän on sitoutunut, ja sen ajattelemattomasti lausutun sanan, jolla hän on kieltäytynyt jostakin, ja Herra antaa hänelle anteeksi.
\par 10 Mutta lesken tai hyljätyn lupaus, mihin hyvänsä hän onkin sitoutunut, olkoon pätevä.
\par 11 Ja jos vaimo on miehensä kodissa tehnyt lupauksen tahi valalla sitoutunut kieltäytymään jostakin
\par 12 ja hänen miehensä saa sen kuulla, mutta ei puhu hänelle siitä mitään eikä kiellä häntä, niin kaikki hänen lupauksensa ja kaikki sitoumukset, joilla hän on kieltäytynyt jostakin, olkoot pätevät.
\par 13 Mutta jos hänen miehensä ne purkaa sinä päivänä, jona hän niistä kuulee, älköön yksikään hänen huuliltaan lähtenyt lupaus tai sitoumus olko pätevä. Hänen miehensä on ne purkanut, ja Herra antaa hänelle anteeksi.
\par 14 Jokaisen lupauksen ja jokaisen valallisen sitoumuksen, jonka vaimo itsensä kurittamiseksi tekee, saakoon hänen miehensä joko vahvistaa tahi purkaa.
\par 15 Mutta jos hänen miehensä ei puhu hänelle niistä mitään sen päivän kuluessa, niin hän on vahvistanut kaikki hänen lupauksensa ja kaikki hänen sitoumuksensa, jotka hänellä on täytettävinä. Hän on ne vahvistanut, koska hän ei niistä hänelle mitään puhunut sinä päivänä, jona hän niistä kuuli.
\par 16 Mutta jos hän ne purkaa vasta jonkun aikaa sen jälkeen, kuin on niistä kuullut, joutuu hän kantamaan vaimonsa syyllisyyden."
\par 17 Nämä ovat ne säädökset, jotka Herra antoi Moosekselle, olemaan voimassa miehen ja hänen vaimonsa välillä sekä isän ja hänen tyttärensä välillä, tämän ollessa nuorena tyttönä isänsä kodissa.

\chapter{31}

\par 1 Ja Herra puhui Moosekselle sanoen:
\par 2 "Kosta israelilaisten puolesta midianilaisille. Senjälkeen sinut otetaan pois heimosi tykö."
\par 3 Niin Mooses puhui kansalle sanoen: "Varustakaa joukostanne miehiä sotaan, ja lähtekööt he Midiania vastaan toimittamaan Herran koston Midianille.
\par 4 Tuhat miestä jokaisesta sukukunnasta, kaikista Israelin sukukunnista, lähettäkää sotaan."
\par 5 Silloin annettiin Israelin heimoista tuhat miestä jokaisesta sukukunnasta, yhteensä kaksitoista tuhatta sotaan varustettua miestä.
\par 6 Ja Mooses lähetti heidät, tuhat miestä jokaisesta sukukunnasta, sotaan sekä heidän kanssaan Piinehaan, pappi Eleasarin pojan; tällä oli mukanaan pyhät esineet ja hälytystorvet.
\par 7 Niin he lähtivät sotimaan Midiania vastaan, niinkuin Herra oli Moosekselle käskyn antanut, ja surmasivat kaikki miehenpuolet.
\par 8 Muiden mukana, jotka kaatuivat, he surmasivat Midianin kuninkaat Evin, Rekemin, Suurin, Huurin ja Reban, Midianin viisi kuningasta. Myös Bileamin, Beorin pojan, he surmasivat miekalla.
\par 9 Mutta israelilaiset ottivat vangiksi midianilaisten vaimot ja lapset ja ryöstivät kaikki heidän juhtansa ja kaiken heidän karjansa ja kaiken heidän tavaransa.
\par 10 Ja kaikki heidän asumansa kaupungit ja kaikki heidän leiripaikkansa he polttivat tulella.
\par 11 Ja he ottivat kaiken ryöstettävän ja kaiken otettavan, sekä ihmiset että karjan.
\par 12 Ja he toivat Moosekselle ja pappi Eleasarille sekä Israelin kansalle vangit ja ottamansa ja ryöstämänsä saaliin leiriin Mooabin arolle, joka on Jordanin luona, Jerikon kohdalla.
\par 13 Silloin Mooses ja pappi Eleasar ja kaikki kansan päämiehet lähtivät heitä vastaan leirin ulkopuolelle.
\par 14 Ja Mooses vihastui sotajoukon johtajiin, tuhannen- ja sadanpäämiehiin, kun he palasivat sotaretkeltä.
\par 15 Ja Mooses sanoi heille: "Oletteko siis jättäneet henkiin kaikki naiset?
\par 16 Katso, nehän ne olivat, jotka Bileamin neuvosta saivat israelilaiset antautumaan uskottomuuteen Herraa vastaan Peorin vuoksi, niin että vitsaus kohtasi Herran seurakuntaa.
\par 17 Niin surmatkaa siis kaikki poikalapset ja surmatkaa myös jokainen vaimo, joka on yhtynyt mieheen.
\par 18 Mutta jokainen tyttö, joka ei ole yhtynyt mieheen, jättäkää itsellenne henkiin.
\par 19 Mutta itse oleskelkaa leirin ulkopuolella seitsemän päivää. Jokainen teistä, joka on jonkun surmannut, ja jokainen teistä, joka on koskenut surmattuun, puhdistautukoon kolmantena ja seitsemäntenä päivänä, sekä te että teidän vankinne.
\par 20 Ja puhdistettakoon jokainen vaate ja jokainen nahkaesine ja kaikki, mikä on tehty vuohen karvoista, ja jokainen puuesine."
\par 21 Ja pappi Eleasar sanoi sotamiehille, jotka olivat menneet sotaan: "Tämä on lakisäädös, jonka Herra antoi Moosekselle:
\par 22 Vain kulta, hopea, vaski, rauta, tina ja lyijy,
\par 23 kaikki, mikä tulta kestää, käyttäkää tulessa, niin se puhdistuu; puhdistettakoon se kuitenkin vielä puhdistusvedellä. Mutta mikä ei tulta kestä, käyttäkää se vedessä.
\par 24 Ja peskää vaatteenne seitsemäntenä päivänä, niin te puhdistutte; ja sitten tulkaa leiriin."
\par 25 Ja Herra puhui Moosekselle sanoen:
\par 26 "Laskekaa, sinä ja pappi Eleasar ja kansan perhekuntien päämiehet, otetun saaliin määrä, ihmiset ja karja.
\par 27 Ja pane saalis kahtia soturien, sotaan lähteneiden, ja kaiken muun kansan kesken.
\par 28 Ja ota Herralle verona sotamiehiltä, sotaan lähteneiltä, yksi viidestäsadasta, ihmisiä, raavaita, aaseja ja lampaita.
\par 29 Ota se heille tulevasta puoliskosta ja anna se pappi Eleasarille antina Herralle.
\par 30 Ja israelilaisille tulevasta puoliskosta ota yksi viidestäkymmenestä, ihmisiä, raavaita, aaseja, lampaita ja kaikkia karjaeläimiä, ja anna ne leeviläisille, joiden on hoidettava tehtävät Herran asumuksessa."
\par 31 Ja Mooses ja pappi Eleasar tekivät, niinkuin Herra oli Moosekselle käskyn antanut.
\par 32 Ja saalis, jäännös siitä, mitä sotajoukko oli ryöstänyt, oli: lampaita kuusisataa seitsemänkymmentäviisi tuhatta
\par 33 ja raavaita seitsemänkymmentäkaksi tuhatta
\par 34 ja aaseja kuusikymmentäyksi tuhatta
\par 35 ja ihmisiä, tyttöjä, jotka eivät olleet yhtyneet mieheen, kaikkiaan kolmekymmentäkaksi tuhatta.
\par 36 Ja puolet siitä eli se osa, joka tuli sotaan lähteneille, oli: lampaita kolmesataa kolmekymmentäseitsemän tuhatta viisisataa,
\par 37 ja Herralle tuleva vero lampaista oli kuusisataa seitsemänkymmentä viisi;
\par 38 ja raavaita kolmekymmentäkuusi tuhatta sekä niistä Herralle tuleva vero seitsemänkymmentä kaksi;
\par 39 ja aaseja kolmekymmentä tuhatta viisisataa sekä niistä Herralle tuleva vero kuusikymmentä yksi;
\par 40 ja ihmisiä kuusitoista tuhatta sekä niistä Herralle tuleva vero kolmekymmentä kaksi.
\par 41 Ja Mooses antoi Herran antiveron pappi Eleasarille, niinkuin Herra oli Moosekselle käskyn antanut.
\par 42 Ja israelilaisille tuleva puolisko, jonka Mooses oli erottanut sotamiesten osasta,
\par 43 tämä seurakunnalle tuleva puolisko oli: lampaita kolmesataa kolmekymmentäseitsemän tuhatta viisisataa
\par 44 ja raavaita kolmekymmentäkuusi tuhatta
\par 45 ja aaseja kolmekymmentä tuhatta viisisataa
\par 46 sekä ihmisiä kuusitoista tuhatta.
\par 47 Tästä israelilaisille tulevasta puoliskosta Mooses otti yhden viidestäkymmenestä, ihmisiä ja karjaa, sekä antoi ne leeviläisille, joiden oli hoidettava tehtävät Herran asumuksessa, niinkuin Herra oli Moosekselle käskyn antanut.
\par 48 Silloin tulivat sotajoukon osastojen johtajat, tuhannen- ja sadanpäämiehet, Mooseksen luo
\par 49 ja sanoivat Moosekselle: "Palvelijasi ovat laskeneet niiden sotamiesten luvun, jotka ovat olleet hallussamme, eikä meistä puutu yhtäkään.
\par 50 Sentähden me tuomme nyt Herralle lahjaksi, mitä kukin on saanut kultakaluja: käätyjä, rannerenkaita, sormuksia, korvarenkaita ja kaulakoristeita, saadaksemme itsellemme sovituksen Herran edessä."
\par 51 Niin Mooses ja pappi Eleasar ottivat heiltä kullan, kaikkinaiset taidokkaasti valmistetut esineet.
\par 52 Ja anniksi annettua kultaa, jonka he antoivat Herralle tuhannen- ja sadanpäämiesten puolesta, oli kaikkiaan kuusitoista tuhatta seitsemänsataa viisikymmentä sekeliä.
\par 53 Sotamiehet olivat ottaneet saalista kukin itselleen.
\par 54 Ja Mooses ja pappi Eleasar ottivat kullan tuhannen- ja sadanpäämiehiltä ja veivät sen ilmestysmajaan, että se johdattaisi israelilaiset Herran muistoon.

\chapter{32}

\par 1 Mutta ruubenilaisilla ja gaadilaisilla oli paljon, ylen runsaasti, karjaa. Kun he nyt katselivat Jaeserin maata ja Gileadin maata, niin he huomasivat, että seutu oli karjanhoitoon sopiva.
\par 2 Niin gaadilaiset ja ruubenilaiset tulivat ja puhuivat Moosekselle ja pappi Eleasarille ja seurakunnan päämiehille sanoen:
\par 3 "Atarot, Diibon, Jaeser, Nimra, Hesbon, Elale, Sebam, Nebo ja Beon,
\par 4 tämä maa, jonka Herra on vallannut Israelin seurakunnalle, on karjanhoitoon sopivaa maata, ja sinun palvelijoillasi on karjaa".
\par 5 Ja he sanoivat vielä: "Jos olemme saaneet armon sinun silmiesi edessä, niin annettakoon tämä maa palvelijoillesi omaksi, äläkä vie meitä Jordanin yli".
\par 6 Mutta Mooses vastasi gaadilaisille ja ruubenilaisille: "Onko teidän veljienne lähdettävä sotaan, ja te jäisitte tänne?
\par 7 Miksi viette israelilaisilta halun mennä siihen maahan, jonka Herra on heille antanut?
\par 8 Niin teidän isännekin tekivät, kun minä lähetin heidät Kaades-Barneasta katselemaan sitä maata.
\par 9 Kun he olivat saapuneet Rypälelaaksoon asti ja katselleet sitä maata, veivät he israelilaisilta halun lähteä siihen maahan, jonka Herra oli heille antanut.
\par 10 Ja sinä päivänä syttyi Herran viha, ja hän vannoi sanoen:
\par 11 'Ne miehet, jotka lähtivät Egyptistä, kaksikymmenvuotiset ja sitä vanhemmat, eivät saa nähdä sitä maata, jonka minä vannoen lupasin Aabrahamille, Iisakille ja Jaakobille, sillä he eivät ole minua uskollisesti seuranneet,
\par 12 paitsi Kaaleb, kenissiläisen Jefunnen poika, ja Joosua, Nuunin poika; sillä nämä ovat uskollisesti seuranneet Herraa'.
\par 13 Ja Herran viha syttyi Israelia kohtaan, ja hän antoi heidän harhailla erämaassa neljäkymmentä vuotta, kunnes koko se sukupolvi hävisi, joka oli tehnyt sitä, mikä oli pahaa Herran silmissä.
\par 14 Mutta katso, te olette nyt astuneet isienne sijaan, te syntisten sikiöt, lisätäksenne vielä Herran vihan kiivautta Israelia kohtaan.
\par 15 Jos te nyt käännytte pois hänestä, niin hän jättää kansan vielä kauemmaksi aikaa tähän erämaahan, ja niin te tuotatte tuhon kaikelle tälle kansalle."
\par 16 Niin he lähestyivät häntä ja sanoivat: "Karjatarhoja me vain rakentaisimme tänne laumoillemme ja kaupunkeja vaimojamme ja lapsiamme varten,
\par 17 mutta itse me varustautuisimme ja rientäisimme israelilaisten etunenässä, kunnes saisimme viedyksi heidät määräpaikkoihinsa, mutta meidän vaimomme ja lapsemme asuisivat sillä aikaa varustetuissa kaupungeissa maan asukkailta rauhassa.
\par 18 Emme me palaisi kotiimme, ennenkuin israelilaiset ovat saaneet haltuunsa kukin perintöosansa,
\par 19 sillä me emme tahdo perintöosaa heidän kanssansa tuolta puolen Jordanin emmekä kauempaa, vaan meidän perintöosamme on joutunut meille tältä puolelta Jordanin, itään päin."
\par 20 Niin Mooses vastasi heille: "Jos näin teette, jos Herran edessä varustaudutte sotaan
\par 21 ja teistä jokainen, sotaan varustettuna, menee Jordanin yli Herran edessä niin pitkäksi aikaa, kunnes on karkoittanut vihollisensa edestään,
\par 22 ja te palaatte vasta senjälkeen, kuin se maa on tehty alamaiseksi Herralle, niin te olette vastuusta vapaat Herran ja Israelin edessä, ja tämä maa tulee teidän omaksenne Herran edessä.
\par 23 Mutta jos ette näin tee, niin katso, te rikotte Herraa vastaan ja saatte tuntea syntinne palkan, joka kohtaa teitä.
\par 24 Rakentakaa siis itsellenne kaupunkeja vaimojanne ja lapsianne varten ja tarhoja karjallenne ja tehkää se, mitä suunne on sanonut."
\par 25 Niin gaadilaiset ja ruubenilaiset vastasivat Moosekselle sanoen: "Sinun palvelijasi tekevät, niinkuin herramme käskee.
\par 26 Lapsemme, vaimomme, laumamme ja kaikki juhtamme jääkööt tänne Gileadin kaupunkeihin,
\par 27 mutta sinun palvelijasi lähtevät, jokainen sotaan varustettuna, sinne taisteluun Herran edessä, niinkuin herramme sanoi."
\par 28 Niin Mooses antoi heistä käskyn pappi Eleasarille ja Joosualle, Nuunin pojalle, ja Israelin sukukuntien perhekuntain päämiehille,
\par 29 ja Mooses sanoi heille: "Jos gaadilaiset ja ruubenilaiset teidän kanssanne lähtevät Jordanin yli, jokainen varustettuna taisteluun Herran edessä, ja se maa tulee teille alamaiseksi, niin antakaa heille Gileadin maa omaksi.
\par 30 Mutta jolleivät he varustaudu ja lähde teidän kanssanne sinne, niin asettukoot teidän keskuuteenne Kanaanin maahan."
\par 31 Silloin gaadilaiset ja ruubenilaiset vastasivat sanoen: "Mitä Herra on puhunut sinun palvelijoillesi, sen me teemme.
\par 32 Me lähdemme varustettuina Herran edessä Kanaanin maahan, että saisimme omaksemme perintöosan tällä puolella Jordanin."
\par 33 Ja Mooses antoi heille, gaadilaisille ja ruubenilaisille sekä toiselle puolelle Manassen, Joosefin pojan, sukukuntaa, amorilaisten kuninkaan Siihonin valtakunnan ja Baasanin kuninkaan Oogin valtakunnan, maan ja sen kaupungit alueinensa, sen maan kaupungit yltympäri.
\par 34 Ja gaadilaiset rakensivat Diibonin, Atarotin, Aroerin,
\par 35 Atrot-Soofanin, Jaeserin, Jogbehan,
\par 36 Beet-Nimran ja Beet-Haaranin varustetuiksi kaupungeiksi sekä karjatarhoja.
\par 37 Ja ruubenilaiset rakensivat Hesbonin, Elalen ja Kirjataimin,
\par 38 Nebon ja Baal-Meonin, joiden nimet muutettiin, sekä Sibman. Ja he panivat nimet niille kaupungeille, jotka he rakensivat.
\par 39 Mutta Maakir, Manassen poika, lähti Gileadiin ja valloitti sen ja karkoitti amorilaiset, jotka asuivat siellä.
\par 40 Ja Mooses antoi Gileadin Maakirille, Manassen pojalle, ja hän asettui sinne.
\par 41 Ja Jaair, Manassen poika, meni ja valloitti heidän leirikylänsä ja kutsui ne Jaairin leirikyliksi.
\par 42 Ja Noobah meni ja valloitti Kenatin ynnä sen alueella olevat kylät ja kutsui sen, nimensä mukaan, Noobahiksi.

\chapter{33}

\par 1 Nämä olivat israelilaisten matkat, jotka he kulkivat Egyptistä osastoittain Mooseksen ja Aaronin johdolla.
\par 2 Ja Mooses kirjoitti Herran käskyn mukaan muistiin ne paikat, joista he lähtivät liikkeelle matkoillansa. Ja nämä ovat heidän matkansa heidän lähtöpaikkojensa mukaan.
\par 3 He lähtivät liikkeelle Ramseksesta ensimmäisessä kuussa, ensimmäisen kuukauden viidentenätoista päivänä; pääsiäisen jälkeisenä päivänä israelilaiset lähtivät matkaan voimallisen käden suojassa, kaikkien egyptiläisten nähden,
\par 4 egyptiläisten haudatessa kaikkia esikoisiansa, jotka Herra heidän keskuudestaan oli surmannut, ja Herran antaessa tuomion kohdata heidän jumaliansa.
\par 5 Niin israelilaiset lähtivät Ramseksesta ja leiriytyivät Sukkotiin.
\par 6 Sitten he lähtivät Sukkotista ja leiriytyivät Eetamiin, joka on erämaan reunassa.
\par 7 Ja he lähtivät Eetamista ja kääntyivät takaisin Pii-Hahirotiin päin, joka on vastapäätä Baal-Sefonia, ja leiriytyivät Migdolin kohdalle.
\par 8 Ja he lähtivät Pii-Hahirotista ja kulkivat meren keskitse erämaahan ja vaelsivat kolmen päivän matkan Eetamin erämaassa ja leiriytyivät Maaraan.
\par 9 Sitten he lähtivät Maarasta ja tulivat Eelimiin. Eelimissä oli kaksitoista vesilähdettä ja seitsemänkymmentä palmupuuta, ja he leiriytyivät sinne.
\par 10 Ja he lähtivät Eelimistä ja leiriytyivät Kaislameren rannalle.
\par 11 Ja he lähtivät Kaislameren rannalta ja leiriytyivät Siinin erämaahan.
\par 12 Ja he lähtivät Siinin erämaasta ja leiriytyivät Dofkaan.
\par 13 Ja he lähtivät Dofkasta ja leiriytyivät Aalukseen.
\par 14 Ja he lähtivät Aaluksesta ja leiriytyivät Refidimiin; siellä ei ollut vettä kansan juoda.
\par 15 Ja he lähtivät Refidimistä ja leiriytyivät Siinain erämaahan.
\par 16 Ja he lähtivät Siinain erämaasta ja leiriytyivät Kibrot-Hattaavaan.
\par 17 Ja he lähtivät Kibrot-Hattaavasta ja leiriytyivät Haserotiin.
\par 18 Ja he lähtivät Haserotista ja leiriytyivät Ritmaan.
\par 19 Ja he lähtivät Ritmasta ja leiriytyivät Rimmon-Perekseen.
\par 20 Ja he lähtivät Rimmon-Pereksestä ja leiriytyivät Libnaan.
\par 21 Ja he lähtivät Libnasta ja leiriytyivät Rissaan.
\par 22 Ja he lähtivät Rissasta ja leiriytyivät Kehelataan.
\par 23 Ja he lähtivät Kehelatasta ja leiriytyivät Seferin vuoren juurelle.
\par 24 Ja he lähtivät Seferin vuoren juurelta ja leiriytyivät Haradaan.
\par 25 Ja he lähtivät Haradasta ja leiriytyivät Makhelotiin.
\par 26 Ja he lähtivät Makhelotista ja leiriytyivät Tahatiin.
\par 27 Ja he lähtivät Tahatista ja leiriytyivät Tarahiin.
\par 28 Ja he lähtivät Tarahista ja leiriytyivät Mitkaan.
\par 29 Ja he lähtivät Mitkasta ja leiriytyivät Hasmonaan.
\par 30 Ja he lähtivät Hasmonasta ja leiriytyivät Mooserotiin.
\par 31 Ja he lähtivät Mooserotista ja leiriytyivät Bene-Jaakaniin.
\par 32 Ja he lähtivät Bene-Jaakanista ja leiriytyivät Hoor-Gidgadiin.
\par 33 Ja he lähtivät Hoor-Gidgadista ja leiriytyivät Jotbataan.
\par 34 Ja he lähtivät Jotbatasta ja leiriytyivät Abronaan.
\par 35 Ja he lähtivät Abronasta ja leiriytyivät Esjon-Geberiin.
\par 36 Ja he lähtivät Esjon-Geberistä ja leiriytyivät Siinin erämaahan, se on Kaadekseen.
\par 37 Ja he lähtivät Kaadeksesta ja leiriytyivät Hoorin vuoren juurelle, Edomin maan rajalle.
\par 38 Ja pappi Aaron nousi Hoorin vuorelle Herran käskyn mukaan ja kuoli siellä neljäntenäkymmenentenä vuotena siitä, kun israelilaiset olivat lähteneet Egyptin maasta, viidennessä kuussa, kuukauden ensimmäisenä päivänä.
\par 39 Ja Aaron oli sadan kahdenkymmenen kolmen vuoden vanha kuollessaan Hoorin vuorella.
\par 40 Mutta Aradin kuningas, kanaanilainen, joka asui Kanaanin maan eteläosassa, sai kuulla israelilaisten tulosta.
\par 41 Ja he lähtivät Hoorin vuorelta ja leiriytyivät Salmonaan.
\par 42 Ja he lähtivät Salmonasta ja leiriytyivät Puunoniin.
\par 43 Ja he lähtivät Puunonista ja leiriytyivät Oobotiin.
\par 44 Ja he lähtivät Oobotista ja leiriytyivät Iije-Abarimiin Mooabin rajalle.
\par 45 Ja he lähtivät Iijimistä ja leiriytyivät Diibon-Gaadiin.
\par 46 Ja he lähtivät Diibon-Gaadista ja leiriytyivät Almon-Diblataimiin.
\par 47 Ja he lähtivät Almon-Diblataimista ja leiriytyivät Abarimin vuoristoon vastapäätä Neboa.
\par 48 Ja he lähtivät Abarimin vuoristosta ja leiriytyivät Mooabin arolle Jordanin rantaan, Jerikon kohdalle.
\par 49 Ja Jordanin rannalla heidän leirinsä ulottui Beet-Jesimotista aina Aabel-Sittimiin asti Mooabin arolle.
\par 50 Ja Herra puhui Moosekselle Mooabin arolla Jordanin rannalla, Jerikon kohdalla, sanoen:
\par 51 "Puhu israelilaisille ja sano heille: Kun olette kulkeneet Jordanin yli Kanaanin maahan,
\par 52 niin karkoittakaa tieltänne kaikki maan asukkaat ja hävittäkää kaikki heidän jumalankuvansa, hävittäkää kaikki heidän valetut kuvansa ja kukistakaa kaikki heidän uhrikukkulansa.
\par 53 Ottakaa sitten maa haltuunne ja asukaa siinä, sillä teille minä annan omaksi sen maan.
\par 54 Ja jakakaa maa keskenänne arvalla sukujenne mukaan; suuremmalle antakaa suurempi perintöosa ja pienemmälle pienempi perintöosa. Kukin saakoon perintöosansa siinä, mihin arpa sen hänelle määrää; isienne sukukuntien mukaan jakakaa maa keskenänne.
\par 55 Mutta jos ette karkoita maan asukkaita tieltänne, niin ne, jotka te heistä jätätte jäljelle, tulevat teille okaiksi silmiinne ja tutkaimiksi kylkiinne, ja he ahdistavat teitä siinä maassa, jossa te asutte.
\par 56 Ja silloin minä teen teille, niinkuin aioin tehdä heille."

\chapter{34}

\par 1 Ja Herra puhui Moosekselle sanoen:
\par 2 "Käske israelilaisia ja sano heille: Kun te tulette Kanaanin maahan - se on se maa, jonka te saatte perintöosaksenne, Kanaanin maa äärestä ääreen -
\par 3 niin teidän eteläinen rajanne kulkekoon Siinin erämaasta Edomia pitkin; eteläinen raja alkakoon idässä Suolameren päästä
\par 4 ja kääntyköön Skorpionisolasta etelään ja kulkekoon Siiniin, ja se päättyköön Kaades-Barneasta etelään. Sieltä raja lähteköön Hasar-Addariin, kulkekoon Asmoniin
\par 5 ja kääntyköön Asmonista Egyptin purolle ja päättyköön mereen.
\par 6 Ja teidän läntisenä rajananne olkoon Suuri meri; tämä olkoon läntisenä rajananne.
\par 7 Ja teidän pohjoinen rajanne olkoon tämä: Suuresta merestä vetäkää raja Hoorin vuoreen;
\par 8 Hoorin vuoresta vetäkää raja siihen, mistä mennään Hamatiin, ja raja päättyköön Sedadiin.
\par 9 Sieltä raja lähteköön Sifroniin ja päättyköön Hasar-Eenaniin. Tämä olkoon pohjoisena rajananne.
\par 10 Ja itäinen rajanne vetäkää Hasar-Eenanista Sefamiin;
\par 11 Sefamista raja painukoon Riblaan, Ainista itään, ja sieltä raja edelleen painukoon, kunnes se sattuu vuoriselänteeseen Kinneretin järven itäpuolella.
\par 12 Sitten raja yhtyköön Jordaniin ja päättyköön Suolamereen. Tämä on oleva teidän maanne rajoineen yltympäri."
\par 13 Ja Mooses käski israelilaisia sanoen: "Tämä on se maa, joka teidän on jaettava arvalla keskenänne ja jonka Herra määräsi annettavaksi yhdeksälle ja puolelle sukukunnalle.
\par 14 Sillä ruubenilaisten sukukunta perhekunnittain ja gaadilaisten sukukunta perhekunnittain ja toinen puoli Manassen sukukuntaa ovat jo saaneet perintöosansa.
\par 15 Nämä kaksi ja puoli sukukuntaa ovat jo saaneet perintöosansa tällä puolella Jordania Jerikon kohdalla, itään päin, auringonnousuun päin."
\par 16 Ja Herra puhui Moosekselle sanoen:
\par 17 "Nämä ovat niiden miesten nimet, joiden on jaettava teille se maa: pappi Eleasar ja Joosua, Nuunin poika;
\par 18 ja näiden lisäksi valitkaa päämies kustakin sukukunnasta maata jakamaan.
\par 19 Nämä ovat niiden miesten nimet: Juudan sukukunnasta Kaaleb, Jefunnen poika;
\par 20 simeonilaisten sukukunnasta Semuel, Ammihudin poika;
\par 21 Benjaminin sukukunnasta Elidad, Kislonin poika;
\par 22 daanilaisten sukukunnasta päämies Bukki, Joglin poika;
\par 23 joosefilaisista, manasselaisten sukukunnasta, päämies Hanniel, Eefodin poika;
\par 24 efraimilaisten sukukunnasta päämies Kemuel, Siftanin poika;
\par 25 sebulonilaisten sukukunnasta päämies Elisafan, Parnakin poika;
\par 26 isaskarilaisten sukukunnasta päämies Paltiel, Assanin poika;
\par 27 asserilaisten sukukunnasta päämies Ahihud, Selomin poika,
\par 28 ja naftalilaisten sukukunnasta Pedahel, Ammihudin poika."
\par 29 Nämä olivat ne, jotka Herra määräsi jakamaan israelilaisille Kanaanin maan.

\chapter{35}

\par 1 Ja Herra puhui Moosekselle Mooabin arolla Jordanin rannalla, Jerikon kohdalla, sanoen:
\par 2 "Käske israelilaisia antamaan omistamistaan perintöosista leeviläisille kaupunkeja heidän asuaksensa ja antamaan leeviläisille myöskin laidunmaata näiden kaupunkien ympäriltä.
\par 3 Nämä kaupungit olkoot heillä asuntoina, ja niiden laidunmaat olkoot heidän juhtiansa, karjaansa ja kaikkia muita elukoitansa varten.
\par 4 Ja näiden kaupunkien laidunmaat, jotka teidän on annettava leeviläisille, ulottukoot kaupungin muurista ulospäin tuhat kyynärää joka taholle.
\par 5 Ja mitatkaa kaupungin ulkopuolella itään päin kaksituhatta kyynärää ja etelään päin kaksituhatta kyynärää ja länteen päin kaksituhatta kyynärää ja pohjoiseen päin kaksituhatta kyynärää, ja kaupunki olkoon siinä keskellä. Tämä olkoon heillä kaupunkien laidunmaa.
\par 6 Ja niistä kaupungeista, jotka teidän on annettava leeviläisille, olkoon kuusi turvakaupunkia, jotka teidän on annettava pakopaikaksi sille, joka on jonkun tappanut; ja niiden lisäksi antakaa neljäkymmentä kaksi kaupunkia.
\par 7 Kaikkiaan olkoon kaupunkeja ja niiden laidunmaita, jotka teidän on annettava leeviläisille, neljäkymmentä kahdeksan.
\par 8 Ja niitä israelilaisten omistamia kaupunkeja, jotka teidän on annettava, otettakoon useampia siltä sukukunnalta, jolla niitä on paljon, ja vähemmän siltä, jolla niitä on vähemmän. Kukin sukukunta antakoon leeviläisille kaupunkejaan saamansa perintöosan mukaan."
\par 9 Ja Herra puhui Moosekselle sanoen:
\par 10 "Puhu israelilaisille ja sano heille: Kun te olette menneet Jordanin yli Kanaanin maahan,
\par 11 niin valitkaa itsellenne kaupunkeja turvakaupungeiksenne. Niihin paetkoon tappaja, joka tapaturmaisesti on jonkun surmannut.
\par 12 Ja nämä kaupungit olkoot teillä turvapaikkoina verenkostajalta, niin ettei tappajan tarvitse kuolla, ennenkuin hän on ollut seurakunnan tuomittavana.
\par 13 Ja kaupunkeja, jotka teidän on määrättävä turvakaupungeiksi, olkoon teillä kuusi.
\par 14 Kolme kaupunkia teidän on määrättävä tältä puolelta Jordanin, ja kolme kaupunkia teidän on määrättävä Kanaanin maasta; ne olkoot turvakaupunkeja.
\par 15 Israelilaisille ja teidän keskuudessanne asuvalle muukalaiselle ja loiselle olkoot ne kuusi kaupunkia turvapaikkoina, joihin voi paeta jokainen, joka tapaturmaisesti jonkun surmaa.
\par 16 Mutta jos joku rauta-aseella lyö toista, niin että tämä kuolee, on hän tahallinen tappaja; sellainen tappaja rangaistakoon kuolemalla.
\par 17 Jos joku ottaa käteensä kiven, jolla voi lyödä kuoliaaksi, ja lyö toista, niin että tämä kuolee, on hän tahallinen tappaja; sellainen tappaja rangaistakoon kuolemalla.
\par 18 Tahi jos joku ottaa käteensä puuaseen, jolla voi lyödä kuoliaaksi, ja lyö toista, niin että tämä kuolee, on hän tahallinen tappaja; sellainen tappaja rangaistakoon kuolemalla.
\par 19 Verenkostaja ottakoon sellaisen tappajan hengiltä; hän ottakoon hänet hengiltä, milloin vain hänet tapaa.
\par 20 Ja jos joku vihassa iskee toista tahi heittää häntä jollakin murhaamisen tarkoituksessa, niin että tämä kuolee,
\par 21 tahi vihaa kantaen lyö häntä kädellään, niin että hän kuolee, rangaistakoon lyöjä kuolemalla, sillä hän on tahallinen tappaja; verenkostaja ottakoon sellaisen tappajan hengiltä, milloin vain hänet tapaa.
\par 22 Mutta jos joku vahingossa, vaan ei vihassa, satuttaa toista tai heittää häntä jollakin esineellä, millä hyvänsä, ilman murhaamisen tarkoitusta,
\par 23 tahi jos hän kivellä, jolla voi lyödä kuoliaaksi, huomaamattansa satuttaa toista, niin että tämä kuolee, eikä hän ollut hänen vihamiehensä eikä tarkoittanut häntä vahingoittaa,
\par 24 niin seurakunta ratkaiskoon tappajan ja verenkostajan välin näiden säädösten mukaan.
\par 25 Ja seurakunta päästäköön tappajan verenkostajan kädestä, ja seurakunta antakoon hänen palata turvakaupunkiin, johon hän oli paennut, ja siellä hän asukoon pyhällä öljyllä voidellun ylimmäisen papin kuolemaan asti.
\par 26 Mutta jos tappaja menee sen turvakaupungin alueen ulkopuolelle, johon hän on paennut,
\par 27 ja verenkostaja tapaa hänet ulkopuolella hänen turvakaupunkinsa aluetta ja verenkostaja surmaa tappajan, niin ei hän joudu verenvikaan;
\par 28 sillä hänen on asuttava turvakaupungissaan ylimmäisen papin kuolemaan asti. Mutta ylimmäisen papin kuoltua tappaja saa palata perintömaallensa.
\par 29 Ja tämä olkoon teillä oikeussäädöksenä sukupolvesta sukupolveen, missä asuttekin.
\par 30 Jos joku lyö toisen kuoliaaksi, surmattakoon tappaja todistajien antaman todistuksen nojalla; mutta yhden todistajan todistuksen nojalla älköön ketään kuolemalla rangaistako.
\par 31 Älkää ottako lunastusmaksua sellaisen tappajan hengestä, joka on tehnyt hengenrikoksen, vaan rangaistakoon hänet kuolemalla.
\par 32 Älkääkä ottako lunastusmaksua siltä, joka on paennut turvakaupunkiin, ettei hän ennen papin kuolemaa saisi palata asumaan omalle maallensa.
\par 33 Älkääkä saastuttako sitä maata, jossa te olette, sillä veri saastuttaa maan. Ja maalle ei voi toimittaa sovitusta verestä, joka siihen on vuodatettu, muulla kuin sen verellä, joka sen on vuodattanut.
\par 34 Älkää siis saastuttako sitä maata, jossa te asutte ja jossa minäkin asun teidän keskellänne, sillä minä, Herra, asun israelilaisten keskellä."

\chapter{36}

\par 1 Silloin Gileadin, Maakirin pojan, Manassen pojanpojan, suvun perhekuntapäämiehet, jotka olivat Joosefin poikien sukua, astuivat esiin ja puhuivat Moosekselle ja ruhtinaille, israelilaisten perhekunta-päämiehille,
\par 2 ja sanoivat: "Herra käski herrani jakaa sen maan arvalla israelilaisille perintöosaksi, ja herrani sai käskyn Herralta antaa Selofhadin, meidän veljemme, perintöosan hänen tyttärillensä.
\par 3 Mutta jos he joutuvat vaimoiksi joillekin israelilaisten muiden heimojen pojista, niin heidän perintöosansa siirtyy pois meidän isiemme perintöosasta ja liitetään sen sukukunnan perintöosaan, johon he tulevat kuulumaan, mutta meidän perintöosamme vähenee.
\par 4 Ja kun israelilaisten riemuvuosi tulee, niin heidän perintöosansa lisätään sen sukukunnan perintöosaan, johon he tulevat kuulumaan, mutta meidän isiemme sukukunnan perintöosasta vähentyy heidän perintöosansa."
\par 5 Silloin Mooses sääti israelilaisille Herran käskyn mukaan, sanoen: "Joosefilaisten sukukunta puhuu oikein.
\par 6 Näin Herra säätää Selofhadin tyttäristä: Menkööt he vaimoiksi kenelle tahtovat, kunhan vain menevät vaimoiksi johonkin isänsä sukukunnan sukuun.
\par 7 Sillä älköön israelilaisilla perintöosa siirtykö sukukunnasta toiseen, vaan israelilaiset säilyttäkööt kukin isiensä sukukunnan perintöosan.
\par 8 Mutta jokainen tytär, joka saa perintöosan israelilaisten sukukunnissa, menköön vaimoksi johonkin isänsä sukukunnan sukuun, että israelilaiset saisivat periä kukin isiensä perintöosan
\par 9 ja ettei perintöosa siirtyisi sukukunnasta toiseen, vaan israelilaisten sukukunnat säilyttäisivät kukin perintöosansa."
\par 10 Niinkuin Herra Moosekselle käskyn antoi, niin Selofhadin tyttäret tekivät.
\par 11 Ja Mahla, Tirsa, Hogla, Milka ja Nooga, Selofhadin tyttäret, menivät vaimoiksi setiensä pojille.
\par 12 Joosefin pojan Manassen poikien sukuihin he menivät vaimoiksi; ja niin heidän perintöosansa jäi heidän isänsä suvun sukukuntaan.
\par 13 Nämä ovat ne käskyt ja säädökset, jotka Herra Mooseksen kautta antoi israelilaisille Mooabin arolla Jordanin rannalla, Jerikon kohdalla.


\end{document}