\begin{document}

\title{Ensimmäinen kuninkaiden kirja}


\chapter{1}

\par 1 Kun kuningas Daavid oli käynyt vanhaksi ja iäkkääksi, ei hän enää voinut pysyä lämpimänä, vaikka häntä peitettiin peitteillä.
\par 2 Silloin hänen palvelijansa sanoivat hänelle: "Etsittäköön herralleni, kuninkaalle, tyttö, neitsyt, palvelemaan kuningasta ja olemaan hänen hoitajattarenaan. Jos hän makaa sinun sylissäsi, niin herrani, kuningas, pysyy lämpimänä."
\par 3 Ja he etsivät koko Israelin alueelta kaunista tyttöä ja löysivät suunemilaisen Abisagin ja veivät hänet kuninkaan tykö.
\par 4 Hän oli hyvin kaunis tyttö ja tuli kuninkaan hoitajattareksi ja palveli häntä. Mutta kuningas ei yhtynyt häneen.
\par 5 Mutta Adonia, Haggitin poika, korotti itsensä ja sanoi: "Minä tahdon tulla kuninkaaksi". Ja hän hankki itselleen sotavaunuja ja ratsumiehiä ja viisikymmentä miestä, jotka juoksivat hänen edellään.
\par 6 Hänen isänsä ei koskaan elämässään ollut pahoittanut hänen mieltään sanomalla: "Kuinka sinä noin teet?" Hän oli myöskin hyvin kaunis mies; ja äiti oli synnyttänyt hänet Absalomin jälkeen.
\par 7 Hän neuvotteli Jooabin, Serujan pojan, ja pappi Ebjatarin kanssa; ja he kannattivat Adonian puoluetta.
\par 8 Mutta pappi Saadok ja Benaja, Joojadan poika, sekä profeetta Naatan, Siimei, Rei ja Daavidin urhot eivät olleet Adonian puolella.
\par 9 Ja Adonia teurasti lampaita, raavaita ja juottovasikoita Soohelet-kiven luona, joka on Roogelin lähteen ääressä. Ja hän kutsui kaikki veljensä, kuninkaan pojat, ja kaikki Juudan miehet, kuninkaan palvelijat;
\par 10 mutta profeetta Naatanin, Benajan, urhot ja veljensä Salomon hän jätti kutsumatta.
\par 11 Silloin Naatan sanoi Batseballe, Salomon äidille, näin: "Etkö ole kuullut, että Adonia, Haggitin poika, on tullut kuninkaaksi, herramme Daavidin siitä mitään tietämättä?
\par 12 Niin tule nyt, minä annan sinulle neuvon, että pelastat oman henkesi ja poikasi Salomon hengen.
\par 13 Mene kuningas Daavidin tykö ja sano hänelle: Herrani, kuningas, etkö sinä itse ole vannonut palvelijattarellesi ja sanonut: Sinun poikasi Salomo on tuleva kuninkaaksi minun jälkeeni; hän on istuva minun valtaistuimellani? Miksi siis Adonia on tullut kuninkaaksi?
\par 14 Ja katso, sinun vielä puhutellessasi kuningasta siellä, tulen minä sinun jälkeesi sisään ja vahvistan sinun sanasi."
\par 15 Niin Batseba meni kuninkaan tykö makuuhuoneeseen. Kuningas oli hyvin vanha; ja suunemilainen Abisag palveli kuningasta.
\par 16 Ja Batseba kumarsi ja osoitti kuninkaalle kunnioitusta. Kuningas sanoi: "Mikä sinun on?"
\par 17 Hän vastasi hänelle: "Herrani, sinä olet itse vannonut palvelijattarellesi Herran, Jumalasi, kautta: 'Sinun poikasi Salomo on tuleva kuninkaaksi minun jälkeeni; hän on istuva minun valtaistuimellani'.
\par 18 Mutta katso, nyt on Adonia tullut kuninkaaksi, ja sinä, herrani, kuningas, et tiedä siitä mitään.
\par 19 Hän on teurastanut paljon härkiä, juottovasikoita ja lampaita ja kutsunut kaikki kuninkaan pojat, pappi Ebjatarin ja sotapäällikkö Jooabin; mutta palvelijasi Salomon hän on jättänyt kutsumatta.
\par 20 Sinua kohti, herrani, kuningas, ovat nyt koko Israelin silmät tähdätyt, että ilmoittaisit heille, kuka on istuva herrani, kuninkaan, valtaistuimella hänen jälkeensä.
\par 21 Muuten käy niin, että kun herrani, kuningas, on mennyt lepoon isiensä tykö, minua ja minun poikaani Salomoa pidetään rikollisina."
\par 22 Ja katso, hänen vielä puhutellessaan kuningasta tuli profeetta Naatan.
\par 23 Ja kuninkaalle ilmoitettiin: "Katso, profeetta Naatan on täällä". Ja tämä tuli kuninkaan eteen ja kumartui kasvoillensa maahan kuninkaan eteen.
\par 24 Ja Naatan sanoi: "Herrani, kuningas, sinäkö olet sanonut: 'Adonia on tuleva kuninkaaksi minun jälkeeni; hän on istuva minun valtaistuimellani'?
\par 25 Sillä hän on tänä päivänä mennyt ja teurastanut paljon härkiä, juottovasikoita ja lampaita ja kutsunut kaikki kuninkaan pojat, sotapäälliköt ja pappi Ebjatarin; ja katso, he syövät ja juovat hänen edessään, ja he huutavat: 'Eläköön kuningas Adonia!'
\par 26 Mutta minut, sinun palvelijasi, pappi Saadokin, Benajan, Joojadan pojan, ja sinun palvelijasi Salomon hän on jättänyt kutsumatta.
\par 27 Onko tämä lähtenyt herrastani, kuninkaasta, sinun antamatta palvelijasi tietää, kuka on istuva herrani, kuninkaan, valtaistuimella hänen jälkeensä?"
\par 28 Kuningas Daavid vastasi ja sanoi: "Kutsukaa minun luokseni Batseba". Kun hän tuli kuninkaan eteen ja seisoi kuninkaan edessä,
\par 29 vannoi kuningas ja sanoi: "Niin totta kuin Herra elää, joka on pelastanut minut kaikesta hädästä:
\par 30 niinkuin minä vannoin sinulle Herran, Israelin Jumalan, kautta ja sanoin: 'Sinun poikasi Salomo on tuleva kuninkaaksi minun jälkeeni, hän on istuva minun valtaistuimellani minun sijassani', niin minä tänä päivänä teen".
\par 31 Silloin Batseba kumartui kasvoillensa maahan ja osoitti kuninkaalle kunnioitusta ja sanoi: "Herrani, kuningas Daavid, eläköön iankaikkisesti!"
\par 32 Ja kuningas Daavid sanoi: "Kutsukaa minun luokseni pappi Saadok, profeetta Naatan ja Benaja, Joojadan poika". Niin he tulivat kuninkaan eteen.
\par 33 Ja kuningas sanoi heille: "Ottakaa mukaanne herranne palvelijat ja pankaa poikani Salomo minun oman muulini selkään ja viekää hänet alas Giihonille.
\par 34 Siellä pappi Saadok ja profeetta Naatan voidelkoot hänet Israelin kuninkaaksi. Ja puhaltakaa pasunaan ja huutakaa: 'Eläköön kuningas Salomo!'
\par 35 Seuratkaa sitten häntä tänne, että hän tulisi ja istuisi minun valtaistuimelleni ja olisi kuninkaana minun sijassani. Sillä hänet minä olen määrännyt Israelin ja Juudan ruhtinaaksi."
\par 36 Silloin Benaja, Joojadan poika, vastasi kuninkaalle ja sanoi: "Amen! Niin sanokoon Herra, minun herrani, kuninkaan, Jumala.
\par 37 Niinkuin Herra on ollut minun herrani, kuninkaan, kanssa, niin olkoon hän Salomon kanssa ja tehköön hänen valtaistuimensa vielä suuremmaksi kuin on herrani, kuningas Daavidin, valtaistuin."
\par 38 Niin pappi Saadok, profeetta Naatan ja Benaja, Joojadan poika, sekä kreetit ja pleetit menivät sinne. He panivat Salomon kuningas Daavidin muulin selkään ja saattoivat hänet Giihonille.
\par 39 Ja pappi Saadok otti öljysarven majasta ja voiteli Salomon. Ja he puhalsivat pasunaan, ja kaikki kansa huusi: "Eläköön kuningas Salomo!"
\par 40 Ja kaikki kansa seurasi häntä, ja kansa soitti huiluilla ja ratkesi niin suureen riemuun, että maa oli haljeta heidän huudostansa.
\par 41 Mutta Adonia ja kaikki kutsuvieraat, jotka olivat hänen kanssaan, kuulivat sen, juuri kun olivat lopettaneet syöntinsä. Ja kun Jooab kuuli pasunan äänen, sanoi hän: "Mitä tuo huuto ja pauhu kaupungissa tietää?"
\par 42 Hänen vielä puhuessaan tuli Joonatan, pappi Ebjatarin poika; ja Adonia sanoi: "Tule tänne, sillä sinä olet kunnon mies ja tuot hyviä sanomia".
\par 43 Joonatan vastasi ja sanoi Adonialle: "Päinvastoin! Herramme, kuningas Daavid, on tehnyt Salomon kuninkaaksi.
\par 44 Kuningas lähetti hänen kanssaan pappi Saadokin, profeetta Naatanin ja Benajan, Joojadan pojan, sekä kreetit ja pleetit, ja he panivat hänet kuninkaan muulin selkään.
\par 45 Ja pappi Saadok ja profeetta Naatan voitelivat hänet Giihonilla kuninkaaksi, ja he tulivat sieltä riemuiten, ja koko kaupunki joutui liikkeelle. Sitä oli se huuto, jonka te kuulitte.
\par 46 Onpa Salomo jo istunut kuninkaan valtaistuimellekin,
\par 47 ja kuninkaan palvelijat ovat tulleet onnittelemaan meidän herraamme, kuningas Daavidia, ja sanoneet: 'Antakoon sinun Jumalasi Salomon nimen tulla vielä mainehikkaammaksi kuin sinun nimesi ja tehköön hänen valtaistuimensa vielä suuremmaksi kuin on sinun valtaistuimesi'. Ja kuningas on rukoillut vuoteessansa.
\par 48 Ja vielä on kuningas puhunut näin: 'Kiitetty olkoon Herra, Israelin Jumala, joka tänä päivänä on asettanut jälkeläisen istumaan minun valtaistuimellani, niin että minä olen sen omin silmin nähnyt'."
\par 49 Silloin kaikki Adonian kutsuvieraat joutuivat kauhun valtaan ja nousivat ja menivät kukin tiehensä.
\par 50 Mutta Adonia pelkäsi Salomoa, nousi, meni ja tarttui alttarin sarviin.
\par 51 Ja Salomolle ilmoitettiin: "Katso, Adonia on peläten kuningas Salomoa tarttunut alttarin sarviin ja sanonut: 'Kuningas Salomo vannokoon minulle tänä päivänä, ettei hän miekalla surmaa palvelijaansa'".
\par 52 Silloin Salomo sanoi: "Jos hän osoittautuu kunnon mieheksi, ei hiuskarvakaan hänen päästään ole putoava maahan; mutta jos hänessä huomataan jotakin pahaa, on hänen kuoltava".
\par 53 Ja kuningas Salomo lähetti tuomaan hänet pois alttarilta; niin hän tuli ja kumarsi kuningas Salomoa. Ja Salomo sanoi hänelle: "Mene kotiisi".

\chapter{2}

\par 1 Kun lähestyi aika, jolloin Daavidin oli kuoltava, käski hän poikaansa Salomoa sanoen:
\par 2 "Minä menen kaiken maailman tietä; ole luja ja ole mies.
\par 3 Ja noudata Herran, Jumalasi, määräyksiä, niin että vaellat hänen teitänsä ja noudatat hänen säädöksiänsä, käskyjänsä, oikeuksiansa ja todistuksiansa, niinkuin on kirjoitettuna Mooseksen laissa, että menestyisit kaikessa, mitä teet, ja kaikkialla, minne käännyt;
\par 4 että Herra täyttäisi tämän sanansa, jonka hän minusta puhui: 'Jos sinun poikasi pitävät vaarin teistänsä, niin että vaeltavat minun edessäni uskollisesti, kaikesta sydämestänsä ja kaikesta sielustansa, niin on mies sinun suvustasi aina oleva Israelin valtaistuimella'.
\par 5 Sinä tiedät myöskin, mitä Jooab, Serujan poika, on tehnyt minulle - tiedät, mitä hän teki kahdelle Israelin sotapäällikölle, Abnerille, Neerin pojalle, ja Amasalle, Jeterin pojalle: kuinka hän tappoi heidät, vuodattaen sotaverta rauhan aikana ja tahraten sotaverellä vyönsä, joka hänellä oli vyötäisillään, ja kenkänsä, jotka hänellä oli jalassaan.
\par 6 Niin tee nyt viisautesi mukaan äläkä anna hänen harmaiden hapsiensa rauhassa mennä alas tuonelaan.
\par 7 Mutta gileadilaisen Barsillain pojille osoita laupeutta, niin että he pääsevät niiden joukkoon, jotka syövät sinun pöydästäsi, sillä hekin tulivat minun avukseni, kun minä pakenin sinun veljeäsi Absalomia.
\par 8 Ja katso, vielä on sinun luonasi Siimei, Geeran poika, benjaminilainen Bahurimista, joka hirveillä kirouksilla kiroili minua, kun minä menin Mahanaimiin. Kun hän tuli minua vastaan alas Jordanille, vannoin minä hänelle Herran kautta ja sanoin: 'Minä en surmaa sinua miekalla'.
\par 9 Mutta älä sinä jätä häntä rankaisematta, sillä sinä olet viisas mies ja tiedät, mitä sinun on hänelle tehtävä, saattaaksesi hänen harmaat hapsensa verisinä alas tuonelaan."
\par 10 Sitten Daavid meni lepoon isiensä tykö, ja hänet haudattiin Daavidin kaupunkiin.
\par 11 Ja aika, jonka Daavid hallitsi Israelia, oli neljäkymmentä vuotta: Hebronissa hän hallitsi seitsemän vuotta, ja Jerusalemissa hän hallitsi kolmekymmentä kolme vuotta.
\par 12 Ja Salomo istui isänsä Daavidin valtaistuimelle, ja hänen kuninkuutensa tuli hyvin vahvaksi.
\par 13 Mutta Adonia, Haggitin poika, tuli Batseban, Salomon äidin, luo. Tämä sanoi: "Rauhaako sinun tulosi tietää?" Hän vastasi: "Rauhaa".
\par 14 Sitten hän sanoi: "Minulla on asiaa sinulle". Batseba sanoi: "Puhu".
\par 15 Silloin hän sanoi: "Sinä tiedät, että kuninkuus oli minun, ja että koko Israel oli kääntänyt katseensa minuun, että minusta tulisi kuningas. Mutta sitten kuninkuus meni minulta pois ja joutui minun veljelleni, sillä Herralta se hänelle tuli.
\par 16 Yhtä minä nyt sinulta pyydän; älä sitä minulta kiellä." Hän sanoi hänelle: "Puhu".
\par 17 Hän sanoi: "Puhu kuningas Salomolle - sillä sinulta hän ei sitä kiellä - että hän antaisi minulle suunemilaisen Abisagin vaimoksi".
\par 18 Batseba vastasi: "Hyvä. Minä puhun kuninkaalle sinun puolestasi."
\par 19 Silloin Batseba meni kuningas Salomon tykö puhumaan hänelle Adonian puolesta. Niin kuningas nousi ja meni häntä vastaan ja kumarsi häntä ja istui valtaistuimelleen. Ja myös kuninkaan äidille asetettiin istuin, ja hän istui hänen oikealle puolellensa.
\par 20 Sitten hän sanoi: "Minulla on sinulle pieni pyyntö, älä sitä minulta kiellä". Kuningas sanoi hänelle: "Pyydä, äiti, minä en sinulta kiellä".
\par 21 Silloin hän sanoi: "Annettakoon suunemilainen Abisag sinun veljellesi Adonialle vaimoksi".
\par 22 Mutta kuningas Salomo vastasi ja sanoi äidillensä: "Miksi pyydät vain suunemilaista Abisagia Adonialle? Pyydä myös kuninkuutta hänelle - hänhän on minun vanhin veljeni - hänelle ja pappi Ebjatarille ja Jooabille, Serujan pojalle."
\par 23 Ja kuningas Salomo vannoi Herran kautta, sanoen: "Jumala rangaiskoon minua nyt ja vasta, jollei Adonia saa hengellään maksaa sitä, että on puhunut näin.
\par 24 Ja nyt, niin totta kuin Herra elää, joka on minut vahvistanut ja on asettanut minut isäni Daavidin valtaistuimelle ja joka lupauksensa mukaan on perustanut minun sukuni: Adonia on tänä päivänä kuolemalla rangaistava."
\par 25 Ja kuningas Salomo lähetti Benajan, Joojadan pojan, lyömään hänet kuoliaaksi; ja hän kuoli.
\par 26 Ja pappi Ebjatarille kuningas sanoi: "Mene maatilallesi Anatotiin, sillä sinä olet kuoleman oma. En minä sinua nyt surmaa, koska sinä olet kantanut Herran Jumalan arkkia minun isäni Daavidin edessä ja koska olet kärsinyt kaikkea, mitä isäni on kärsinyt."
\par 27 Niin Salomo karkoitti Ebjatarin olemasta Herran pappina ja täytti sen sanan, minkä Herra oli Siilossa puhunut Eelin suvusta.
\par 28 Ja kun sanoma tästä tuli Jooabille - sillä Jooab oli liittynyt Adoniaan, vaikka hän ei ollut liittynyt Absalomiin - pakeni Jooab Herran majaan ja tarttui alttarin sarviin.
\par 29 Mutta kun kuningas Salomolle ilmoitettiin: "Jooab on paennut Herran majaan ja on alttarin ääressä", lähetti Salomo Benajan, Joojadan pojan, sanoen: "Mene ja lyö hänet kuoliaaksi".
\par 30 Benaja meni Herran majaan ja sanoi hänelle: "Näin käskee kuningas: 'Tule ulos'". Mutta hän vastasi: "En, tässä minä kuolen". Silloin Benaja palasi kertomaan sitä kuninkaalle ja sanoi: "Näin Jooab puhui, ja näin hän vastasi minulle".
\par 31 Kuningas sanoi hänelle: "Tee, niinkuin hän on puhunut, lyö hänet kuoliaaksi ja hautaa hänet, että poistaisit minusta ja minun isäni perheestä sen viattoman veren, jonka Jooab on vuodattanut.
\par 32 Ja Herra antakoon hänen verensä tulla hänen oman päänsä päälle, koska hän löi kuoliaaksi kaksi miestä, jotka olivat häntä hurskaammat ja paremmat, ja tappoi heidät miekalla, minun isäni Daavidin siitä mitään tietämättä: Abnerin, Neerin pojan, Israelin sotapäällikön, ja Amasan, Jeterin pojan, Juudan sotapäällikön.
\par 33 Heidän verensä tulkoon Jooabin pään päälle ja hänen jälkeläistensä pään päälle, iankaikkisesti. Mutta Daavidilla ja hänen jälkeläisillään, hänen suvullansa ja hänen valtaistuimellansa olkoon rauha Herralta iankaikkisesti."
\par 34 Niin Benaja, Joojadan poika, meni ja löi hänet kuoliaaksi; ja hänet haudattiin omaan taloonsa erämaahan.
\par 35 Ja kuningas asetti Benajan, Joojadan pojan, hänen sijaansa sotaväen päälliköksi; ja pappi Saadokin kuningas asetti Ebjatarin sijalle.
\par 36 Sen jälkeen kuningas lähetti kutsumaan Siimein ja sanoi hänelle: "Rakenna itsellesi talo Jerusalemiin ja asu siellä, menemättä sieltä sinne tai tänne.
\par 37 Sillä jona päivänä sinä lähdet sieltä ja menet Kidronin laakson poikki, sinä päivänä on sinun, tiedä se, kuolemalla kuoltava; sinun veresi on tuleva oman pääsi päälle."
\par 38 Siimei sanoi kuninkaalle: "Hyvä on; niinkuin herrani, kuningas, on puhunut, niin on sinun palvelijasi tekevä". Ja Siimei asui Jerusalemissa kauan aikaa.
\par 39 Mutta kolmen vuoden kuluttua tapahtui, että Siimeiltä karkasi kaksi palvelijaa Aakiin, Maakan pojan, Gatin kuninkaan, luo. Ja Siimeille ilmoitettiin: "Katso, sinun palvelijasi ovat Gatissa".
\par 40 Niin Siimei nousi ja satuloi aasinsa ja lähti Aakiin luo Gatiin etsimään palvelijoitaan; ja Siimei meni ja toi palvelijansa Gatista.
\par 41 Ja kun Salomolle ilmoitettiin, että Siimei oli mennyt Jerusalemista Gatiin ja palannut takaisin,
\par 42 lähetti kuningas kutsumaan Siimein ja sanoi hänelle: "Enkö minä ole vannottanut sinua Herran kautta ja varoittanut sinua sanoen: 'Jona päivänä sinä lähdet ja menet sinne tai tänne, sinä päivänä on sinun, tiedä se, kuolemalla kuoltava'? Ja sinä sanoit minulle: 'Hyvä on; minä olen kuullut'.
\par 43 Miksi et ole pitänyt Herran valaa etkä sitä käskyä, jonka minä sinulle annoin?"
\par 44 Ja kuningas sanoi vielä Siimeille: "Sinä tiedät, tunnossasi tiedät, kaiken pahan, minkä olet tehnyt minun isälleni Daavidille; Herra kääntää tekemäsi pahan omaan päähäsi.
\par 45 Mutta kuningas Salomo on oleva siunattu, ja Daavidin valtaistuin on oleva iäti vahva Herran edessä."
\par 46 Ja kuningas käski Benajaa, Joojadan poikaa; niin tämä meni ja löi hänet kuoliaaksi, ja hän kuoli. Niin kuninkuus vahvistui Salomon käsissä.

\chapter{3}

\par 1 Ja Salomo tuli faraon, Egyptin kuninkaan, vävyksi: hän otti faraon tyttären vaimokseen. Ja hän vei hänet Daavidin kaupunkiin, kunnes oli saanut rakennetuksi oman linnansa, Herran temppelin ja Jerusalemin ympärysmuurin.
\par 2 Mutta kansa uhrasi uhrikukkuloilla, koska niihin aikoihin ei vielä oltu rakennettu temppeliä Herran nimelle.
\par 3 Ja Salomo rakasti Herraa ja vaelsi isänsä Daavidin käskyjen mukaan; kuitenkin hän uhrasi ja suitsutti uhrikukkuloilla.
\par 4 Niin kuningas meni Gibeoniin, uhraamaan siellä, sillä se oli suurin uhrikukkula. Tuhat polttouhria Salomo uhrasi sillä alttarilla.
\par 5 Gibeonissa Herra ilmestyi Salomolle yöllä unessa, ja Jumala sanoi: "Ano, mitä tahdot, että minä sinulle antaisin".
\par 6 Salomo vastasi: "Sinä olet tehnyt suuren laupeuden palvelijallesi, minun isälleni Daavidille, koska hän vaelsi sinun edessäsi totuudessa ja vanhurskaudessa sekä vilpittömällä sydämellä sinua kohtaan. Ja sinä olet säilyttänyt hänelle tämän suuren laupeuden ja antanut hänelle pojan, joka istuu hänen valtaistuimellansa, niinkuin on laita tänä päivänä.
\par 7 Ja nyt, Herra, minun Jumalani, sinä olet tehnyt palvelijasi kuninkaaksi minun isäni Daavidin sijaan; mutta minä olen kuin pieni poikanen: en osaa lähteä enkä tulla.
\par 8 Ja palvelijasi on keskellä sinun kansaasi, jonka olet valinnut, niin monilukuista kansaa, että sitä ei voi laskea eikä lukea sen paljouden tähden.
\par 9 Anna sentähden palvelijallesi kuuliainen sydän tuomitakseni sinun kansaasi ja erottaakseni hyvän pahasta; sillä kuka voi muuten tätä sinun suurta kansaasi tuomita?"
\par 10 Herralle oli otollista, että Salomo tätä anoi.
\par 11 Ja Jumala sanoi hänelle: "Koska sinä anoit tätä etkä anonut itsellesi pitkää ikää, et rikkautta etkä vihamiestesi henkeä, vaan anoit itsellesi ymmärrystä kuullaksesi, mikä oikein on,
\par 12 niin katso, minä teen, niinkuin sanot: katso, minä annan sinulle viisaan ja ymmärtäväisen sydämen, niin ettei sinun vertaistasi ole ollut ennen sinua eikä tule sinun jälkeesi.
\par 13 Ja lisäksi minä annan sinulle, mitä et anonutkaan: sekä rikkautta että kunniaa, niin ettei koko elinaikanasi ole kuningasten joukossa oleva sinun vertaistasi.
\par 14 Ja jos sinä vaellat minun teitäni, noudattaen minun säädöksiäni ja käskyjäni, niinkuin sinun isäsi Daavid vaelsi, niin minä suon sinulle pitkän iän."
\par 15 Siihen Salomo heräsi; ja katso, se oli unta. Ja kun hän tuli Jerusalemiin, astui hän Herran liitonarkin eteen, uhrasi polttouhreja ja toimitti yhteysuhrin ja laittoi pidot kaikille palvelijoillensa.
\par 16 Siihen aikaan tuli kaksi porttoa kuninkaan luo, ja he astuivat hänen eteensä.
\par 17 Ja toinen nainen sanoi: "Oi herrani, minä ja tämä nainen asumme samassa huoneessa. Ja minä synnytin hänen luonansa siinä huoneessa.
\par 18 Ja kolmantena päivänä sen jälkeen, kun minä olin synnyttänyt, synnytti myös tämä nainen. Me olimme yhdessä, eikä ketään vierasta ollut meidän kanssamme huoneessa; ainoastaan me kahden olimme huoneessa.
\par 19 Mutta tämän naisen poika kuoli yöllä, sillä hän oli maannut sen.
\par 20 Niin hän nousi keskellä yötä ja otti minun poikani vierestäni palvelijattaresi nukkuessa ja pani sen povellensa, mutta oman kuolleen poikansa hän pani minun povelleni.
\par 21 Kun minä aamulla nousin imettämään poikaani, niin katso, se oli kuollut. Mutta kun minä tarkastin sitä aamulla, niin katso, se ei ollutkaan minun poikani, jonka minä olin synnyttänyt."
\par 22 Niin toinen nainen sanoi: "Ei ole niin; vaan tuo elossa oleva on minun poikani, ja tämä kuollut on sinun poikasi". Mutta edellinen vastasi: "Ei ole niin; tuo kuollut on sinun poikasi, ja tämä elossa oleva on minun poikani". Näin he riitelivät kuninkaan edessä.
\par 23 Niin kuningas sanoi: "Toinen sanoo: 'Elossa oleva on minun poikani, ja kuollut on sinun poikasi'. Ja toinen sanoo: 'Ei ole niin; vaan kuollut on sinun poikasi, ja elossa oleva on minun poikani'."
\par 24 Sitten kuningas sanoi: "Antakaa minulle miekka". Ja kuninkaan eteen tuotiin miekka.
\par 25 Ja kuningas sanoi: "Viiltäkää elossa oleva lapsi kahtia, ja antakaa toinen puoli toiselle ja toinen puoli toiselle".
\par 26 Mutta silloin sanoi kuninkaalle se nainen, jonka oma elossa oleva poika oli, sillä hänen sydämensä tuli liikutetuksi hänen poikansa tähden - hän sanoi: "Oi herrani, antakaa tuolle elossa oleva lapsi; älkää surmatko sitä". Mutta toinen sanoi: "Ei minulle eikä sinulle; viiltäkää!"
\par 27 Niin kuningas vastasi ja sanoi: "Antakaa tälle elossa oleva lapsi; älkää surmatko sitä. Hän on sen äiti."
\par 28 Ja koko Israel kuuli tästä tuomiosta, jonka kuningas oli antanut, ja he pelkäsivät kuningasta; sillä he näkivät, että hänessä oli Jumalan viisaus oikeuden jakamiseen.

\chapter{4}

\par 1 Kuningas Salomo oli koko Israelin kuningas.
\par 2 Ja nämä olivat hänen ylimmät virkamiehensä: Asarja, Saadokin poika, oli ylipappina;
\par 3 Elihoref ja Ahia, Siisan pojat, olivat kirjureina; Joosafat, Ahiludin poika, oli kanslerina;
\par 4 Benaja, Joojadan poika, oli sotaväen ylipäällikkönä; Saadok ja Ebjatar olivat pappeina.
\par 5 Asarja, Naatanin poika, oli ylimaaherrana; Saabud, Naatanin poika, pappi, oli kuninkaan ystävänä.
\par 6 Ahisar oli linnanpäällikkönä; Adoniram, Abdan poika, oli verotöiden valvojana.
\par 7 Ja Salomolla oli kaksitoista maaherraa, yli koko Israelin, joiden oli hankittava kuninkaan ja hänen hovinsa elintarpeet; yhtenä kuukautena vuodessa oli kunkin hankittava elintarpeet.
\par 8 Nämä olivat heidän nimensä: Huurin poika Efraimin vuoristossa;
\par 9 Dekerin poika Maakaassa, Saalbimissa, Beet-Semeksessä, Eelonissa ja Beet-Haananissa;
\par 10 Hesedin poika Arubbotissa; hänellä oli Sooko ja koko Heeferin maa;
\par 11 Abinadabin poika koko Doorin kukkula-alueella; hänellä oli vaimona Taafat, Salomon tytär;
\par 12 Baana, Ahiludin poika, Taanakissa ja Megiddossa ja koko Beet-Seanissa, joka on Saaretanin sivulla Jisreelin alapuolella, Beet-Seanista aina Aabel-Meholaan, tuolle puolelle Jokmeamin;
\par 13 Geberin poika Gileadin Raamotissa; hänellä oli Jaairin, Manassen pojan, leirikylät, jotka ovat Gileadissa; hänellä oli siis Argobin seutu, joka on Baasanissa, kuusikymmentä suurta, muureilla ja vaskisalvoilla varustettua kaupunkia;
\par 14 Ahinadab, Iddon poika, Mahanaimissa;
\par 15 Ahimaas Naftalissa; hänkin oli ottanut vaimokseen Salomon tyttären, Baasematin;
\par 16 Baana, Huusain poika, Asserissa ja Bealotissa;
\par 17 Joosafat, Paaruahin poika, Isaskarissa;
\par 18 Siimei, Eelan poika, Benjaminissa;
\par 19 Geber, Uurin poika, Gileadin maassa, Siihonin, amorilaisten kuninkaan, ja Oogin, Baasanin kuninkaan, maassa; sillä siinä maassa oli yksi maaherra.
\par 20 Juudaa ja Israelia oli paljon, niin paljon kuin hiekkaa meren rannalla; he söivät ja joivat ja olivat iloisia.
\par 21 Ja Salomo hallitsi kaikkia valtakuntia Eufrat-virrasta aina filistealaisten maahan ja Egyptin rajaan asti. He toivat lahjoja ja palvelivat Salomoa, niin kauan kuin hän eli.
\par 22 Ja Salomon jokapäiväinen muona oli: kolmekymmentä koor-mittaa lestyjä jauhoja ja kuusikymmentä koor-mittaa muita jauhoja,
\par 23 kymmenen syöttöraavasta, kaksikymmentä laitumella käyvää raavasta ja sata lammasta; näitten lisäksi peuroja, gaselleja, metsäkauriita ja syötettyjä hanhia.
\par 24 Sillä hän vallitsi kaikkea Eufrat-virran senpuoleista maata, Tifsahista aina Gassaan saakka, kaikkia Eufrat-virran senpuoleisia kuninkaita; ja hänellä oli rauha joka puolelta yltympäri,
\par 25 niin että Juuda ja Israel asuivat turvallisesti, itsekukin viinipuunsa ja viikunapuunsa alla, Daanista Beersebaan asti, niin kauan kuin Salomo eli.
\par 26 Ja Salomolla oli neljäkymmentä tuhatta hevosvaljakkoa vaunujaan varten ja kaksitoista tuhatta ratsuhevosta.
\par 27 Ja maaherrat hankkivat, kukin kuukautenaan, elintarpeet kuningas Salomolle ja kaikille, joilla oli pääsy kuningas Salomon pöytään, antamatta minkään puuttua.
\par 28 Myöskin ohrat ja oljet hevosille ja juoksijoille he toivat, kukin vuorollaan, siihen paikkaan, missä hän oleskeli.
\par 29 Ja Jumala antoi Salomolle viisautta ja ymmärrystä ylen runsaasti ja älyä niin laajalti, kuin on hiekkaa meren rannalla,
\par 30 niin että Salomon viisaus oli suurempi kuin kaikkien Idän miesten ja kaikkien egyptiläisten viisaus.
\par 31 Hän oli viisaampi kaikkia ihmisiä, viisaampi kuin esrahilainen Eetan ja Heeman, Kalkol ja Darda, Maaholin pojat; ja hänen nimensä tuli kuuluksi kaikkien ympärillä olevien kansojen keskuudessa.
\par 32 Ja Salomo sepitti kolmetuhatta sananlaskua, ja hänen laulujansa oli tuhatviisi.
\par 33 Hän puhui puista, Libanonin setripuusta alkaen isoppiin asti, joka kasvaa seinän vieressä. Hän puhui myös karjaeläimistä, linnuista, matelijoista ja kaloista.
\par 34 Ja Salomon viisautta tultiin kuulemaan kaikista kansoista, kaikkien maan kuninkaiden luota, jotka olivat kuulleet hänen viisaudestansa.

\chapter{5}

\par 1 Hiiram, Tyyron kuningas, lähetti palvelijoitansa Salomon luo, kuultuaan, että hänet oli voideltu kuninkaaksi isänsä sijaan; sillä Hiiram oli aina ollut Daavidin likeinen ystävä.
\par 2 Ja Salomo lähetti Hiiramille tämän sanan:
\par 3 "Sinä tiedät, ettei minun isäni Daavid voinut rakentaa temppeliä Herran, Jumalansa, nimelle niiden sotien tähden, joilla häntä joka puolelta ahdistettiin, kunnes Herra oli laskenut viholliset hänen jalkojensa alle.
\par 4 Mutta nyt Herra, minun Jumalani, on suonut minun päästä rauhaan joka taholla; ei ole vastustajaa eikä vaaran uhkaa.
\par 5 Sentähden minä aion rakentaa temppelin Herran, Jumalani, nimelle, niinkuin Herra puhui minun isälleni Daavidille, sanoen: 'Sinun poikasi, jonka minä asetan valtaistuimellesi sinun sijaasi, on rakentava minun nimelleni temppelin'.
\par 6 Niin käske nyt hakata minulle setripuita Libanonilta. Ja minun palvelijani olkoot sinun palvelijaisi kanssa. Ja minä annan sinulle palvelijaisi palkan, aivan niinkuin sinä määräät. Sillä sinä tiedät, ettei meillä ole ketään, joka osaisi niin hakata puita kuin siidonilaiset."
\par 7 Kun Hiiram kuuli Salomon sanat, tuli hän hyvin iloiseksi ja sanoi: "Kiitetty olkoon tänä päivänä Herra, joka on antanut Daavidille viisaan pojan hallitsemaan tuota monilukuista kansaa".
\par 8 Ja Hiiram lähetti Salomolle tämän sanan: "Minä olen kuullut sanan, jonka sinä minulle lähetit. Minä olen täyttävä kaikessa sinun toivomuksesi, mitä tulee setripuihin ja kypressipuihin.
\par 9 Minun palvelijani vetäkööt ne Libanonilta alas mereen, ja minä panetan ne meressä lauttoihin kuljetettaviksi paikkaan, minkä sinä minulle määräät, ja hajotan ne siellä; sinä saat noutaa ne sieltä. Mutta täytä sinä minun toivomukseni ja anna minun hovilleni muonaa."
\par 10 Niin Hiiram antoi Salomolle setripuita ja kypressipuita niin paljon, kuin tämä toivoi.
\par 11 Mutta Salomo antoi Hiiramille kaksikymmentä tuhatta koor-mittaa nisuja, ravinnoksi hänen hovillensa, ja kaksikymmentä koor-mittaa survomalla saatua öljyä. Tämän Salomo antoi Hiiramille joka vuosi.
\par 12 Ja Herra oli antanut Salomolle viisautta, niinkuin hän oli hänelle luvannut. Ja rauha vallitsi Hiiramin ja Salomon välillä, ja he tekivät liiton keskenänsä.
\par 13 Ja kuningas Salomo otti verotyöläisiä koko Israelista, ja verotyöläisiä oli kolmekymmentä tuhatta miestä.
\par 14 Nämä hän lähetti Libanonille vuorotellen, kymmenentuhatta kunakin kuukautena, niin että he olivat yhden kuukauden Libanonilla ja kaksi kuukautta kotonansa; ja Adoniram oli verotöiden valvojana.
\par 15 Ja Salomolla oli seitsemänkymmentä tuhatta taakankantajaa ja kahdeksankymmentä tuhatta kivenhakkaajaa vuoristossa;
\par 16 sen lisäksi Salomon maaherrojen virkamiehiä, jotka valvoivat töitä, kolmetuhatta kolmesataa miestä, vallitsemassa väkeä, joka teki työtä.
\par 17 Ja kuningas käski louhia suuria kiviä, kallisarvoisia kiviä, laskeakseen temppelin perustuksen hakatuista kivistä.
\par 18 Niin Salomon ja Hiiramin rakentajat ja gebalilaiset hakkasivat niitä ja valmistivat puut ja kivet temppelin rakentamiseksi.

\chapter{6}

\par 1 Neljäntenäsadantena kahdeksantenakymmenentenä vuotena siitä, kun israelilaiset olivat lähteneet Egyptin maasta, neljäntenä vuotena siitä, kun Salomo oli tullut Israelin kuninkaaksi, alkoi hän siiv-kuussa, joka on toinen kuukausi, rakentaa temppeliä Herralle.
\par 2 Temppeli, jonka kuningas Salomo Herralle rakensi, oli kuuttakymmentä kyynärää pitkä, kahtakymmentä kyynärää leveä ja kolmeakymmentä kyynärää korkea.
\par 3 Temppelisalin eteinen, joka oli temppelin itäpäässä, oli kahtakymmentä kyynärää pitkä, ja sen leveys, temppelin itäiseen suuntaan, oli kymmenen kyynärää.
\par 4 Ja hän teki temppeliin sisäänpäin avartuvat ikkuna-aukot.
\par 5 Ja hän rakensi temppelin seinään kiinni kylkirakennuksen, ympäri temppelin seinien, sekä temppelisalin että kaikkeinpyhimmän ympäri, ja teki siihen sivukammioita yltympäri.
\par 6 Kylkirakennuksen alin kerros oli viittä kyynärää leveä, keskimmäinen kuutta kyynärää leveä ja kolmas seitsemää kyynärää leveä; sillä hän teki temppelin ulkomuurin ympärinsä penkerellisen, ettei vuoliaisia tarvinnut upottaa temppelin seiniin.
\par 7 Ja kun temppeli rakennettiin, tehtiin se kivistä, jotka tulivat valmiina louhimosta, niin ettei kuulunut vasaran, ei minkään rauta-aseen kalketta temppeliä rakennettaessa.
\par 8 Alimman kerroksen ovi oli temppelin eteläsivussa; portaita myöten noustiin keskimmäiseen kerrokseen ja keskimmäisestä kolmanteen.
\par 9 Rakennettuaan temppelin valmiiksi hän laudoitti sen sisältä setripuisilla parruilla ja laudoilla.
\par 10 Ja hän rakensi kauttaaltaan kiinni temppeliin kylkirakennuksen, jonka kerrokset olivat viiden kyynärän korkuiset ja joka oli kiinnitetty temppeliin setripalkeilla.
\par 11 Ja Salomolle tuli tämä Herran sana:
\par 12 "Tälle temppelille, jota sinä rakennat, on käyvä näin: jos sinä vaellat minun säädöksieni mukaan ja noudatat minun oikeuksiani, otat vaarin kaikista minun käskyistäni ja vaellat niiden mukaan, niin minä täytän sinulle sanani, jonka olen puhunut sinun isällesi Daavidille:
\par 13 minä asun israelilaisten keskellä enkä hylkää kansaani Israelia".
\par 14 Niin Salomo rakensi temppelin valmiiksi.
\par 15 Hän laudoitti temppelin seinät sisäpuolelta setrilaudoilla; temppelin lattiasta kattopalkkeihin asti hän päällysti sen puulla sisäpuolelta. Temppelin lattian hän päällysti kypressilaudoilla.
\par 16 Ja temppelin peräosan, kaksikymmentä kyynärää, hän laudoitti erilleen setrilaudoilla lattiasta kattopalkkeihin asti; niin hän rakensi siihen sisälle kuorin, kaikkeinpyhimmän.
\par 17 Ja temppeli, se on temppelisali kaikkeinpyhimmän edessä, oli neljääkymmentä kyynärää pitkä.
\par 18 Ja temppeli oli sisäpuolelta setripuuta, koristettu metsäkurpitsi- ja kukkakiehkura-leikkauksilla; kaikki oli setripuuta, kiveä ei näkynyt.
\par 19 Temppelin sisään hän laittoi kaikkeinpyhimmän, pannakseen sinne Herran liitonarkin.
\par 20 Kaikkeinpyhin oli kahtakymmentä kyynärää pitkä, kahtakymmentä kyynärää leveä ja kahtakymmentä kyynärää korkea; ja hän päällysti sen sisältä puhtaalla kullalla, ja hän teki alttarin setripuusta.
\par 21 Ja Salomo päällysti temppelin sisäpuolelta puhtaalla kullalla. Ja hän sulki kultavitjoilla kaikkeinpyhimmän, jonka oli päällystänyt kullalla.
\par 22 Koko temppelin hän päällysti sisältä kullalla, koko temppelin yltyleensä. Myöskin koko sen alttarin, joka oli kaikkeinpyhimmän edessä, hän päällysti kullalla.
\par 23 Ja hän teki kaikkeinpyhimpään kaksi kerubia öljypuusta, kymmenen kyynärän korkuista;
\par 24 ja kerubin toinen siipi oli viisikyynäräinen, ja kerubin toinen siipi oli myöskin viisikyynäräinen, niin että oli kymmenen kyynärää toisen siiven kärjestä toisen kärkeen.
\par 25 Toinen kerubi oli myöskin kymmenkyynäräinen. Molemmat kerubit olivat yhtä suuret ja yhdenmuotoiset:
\par 26 toinen kerubi oli kymmentä kyynärää korkea ja samoin myöskin toinen kerubi.
\par 27 Ja hän pani kerubit temppelin sisimpään osaan, ja kerubit levittivät siipensä niin, että toisen siipi kosketti toista seinää ja toisen kerubin siipi kosketti toista seinää; ja temppelin keskikohdalla koskettivat niiden toiset siivet toisiaan.
\par 28 Ja hän päällysti kerubit kullalla.
\par 29 Ja kaikkiin temppelin seiniin yltympäri hän leikkautti veistoksia, kerubeja, palmuja ja kukkakiehkuroita sekä perä- että etuosaan.
\par 30 Ja temppelin lattian hän päällysti kullalla, sekä perä- että etuosan lattian.
\par 31 Ja kaikkeinpyhimmän oviaukkoon hän teki ovet öljypuusta; kamana ja pielet muodostivat viisikulmion.
\par 32 Ja molempiin öljypuusta tehtyihin oviin hän leikkautti kerubeja, palmuja ja kukkakiehkuroita, ja päällysti ne kullalla; hän levitti kultaa kerubien ja palmujen päälle.
\par 33 Niin hän teki myös temppelisalin oviaukkoon öljypuusta pielet, nelikulmion muotoiset,
\par 34 ja kypressipuusta kaksi ovea; toisen oven molemmat laudat olivat kääntyviä, ja toisen oven molemmat laudat olivat kääntyviä.
\par 35 Ja hän leikkautti niihin kerubeja, palmuja ja kukkakiehkuroita ja päällysti ne kullalla, joka levitettiin veistosten päälle.
\par 36 Sitten hän rakensi sisemmän esipihan muurin, jossa oli aina rinnakkain kolme kivikertaa hakattuja kiviä ja yksi hirsikerta veistettyjä setrihirsiä.
\par 37 Neljäntenä vuotena, siiv-kuussa, laskettiin Herran temppelin perustus.
\par 38 Ja yhdentenätoista vuotena, buul-kuussa, joka on kahdeksas kuukausi, temppeli oli kokonaan ja kaikkineen valmis. Hän rakensi sitä seitsemän vuotta.

\chapter{7}

\par 1 Mutta omaa linnaansa Salomo rakensi kolmetoista vuotta ja sai niin koko linnansa valmiiksi.
\par 2 Hän rakensi Libanoninmetsä-talon, sataa kyynärää pitkän, viittäkymmentä kyynärää leveän ja kolmeakymmentä kyynärää korkean, kolmen setripylväsrivin varaan, ja pylväiden päällä oli veistetyt setriansaat.
\par 3 Ja siinä oli setripuukatto sivukammioiden päällä, jotka olivat pylväiden varassa; pylväitä oli yhteensä neljäkymmentä viisi, viisitoista kussakin rivissä.
\par 4 Ja siinä oli ikkunoita kolmessa rivissä, ja valoaukot olivat vastakkain, aina kolme ja kolme.
\par 5 Kaikki oviaukot ja valoaukot olivat nelikulmaiset; ja oviaukot olivat vastakkain, aina kolme ja kolme.
\par 6 Vielä hän teki pylvässalin, viittäkymmentä kyynärää pitkän ja kolmeakymmentä kyynärää leveän, ja sen eteen eteisen pylväineen sekä pylväiden eteen porraskatoksen.
\par 7 Sitten hän teki valtaistuinsalin, jossa hän jakoi oikeutta, oikeussalin; se oli laudoitettu sisältä setripuulla lattiasta kattoon asti.
\par 8 Ja hänen linnansa, jossa hän itse asui, oli toisella esipihalla, salin takana, ja oli rakennettu samalla tavalla. Salomo teki myöskin faraon tyttärelle, jonka hän oli nainut, palatsin, samanlaisen kuin sali.
\par 9 Kaikki nämä olivat rakennetut kallisarvoisista, mitan mukaan hakatuista ja sisältä ja ulkoa sahalla sahatuista kivistä, perustuksesta räystäisiin asti; samoin ulkopuolella suureen esipihaan saakka.
\par 10 Ja perustus oli laskettu kallisarvoisista kivistä, suurista kivistä, kymmenkyynäräisistä ja kahdeksankyynäräisistä kivistä.
\par 11 Ja sen päällä oli kallisarvoisia, mitan mukaan hakattuja kiviä sekä setripuuta.
\par 12 Ja suuren esipihan muurissa oli yltympäri, aina rinnakkain, kolme kivikertaa hakattuja kiviä ja yksi hirsikerta veistettyjä setrihirsiä. Samoin oli Herran temppelin sisemmän esipihan muuri ja myöskin palatsin eteisen esipihan muuri rakennettu.
\par 13 Ja kuningas Salomo lähetti noutamaan Hiiramin Tyyrosta.
\par 14 Hän oli leskivaimon poika Naftalin sukukunnasta, ja hänen isänsä oli ollut tyyrolainen, vaskiseppä. Hän oli täynnä taidollisuutta, ymmärrystä ja tietoa, niin että hän kykeni valmistamaan kaikkinaisia vaskitöitä. Ja hän tuli kuningas Salomon luo ja valmisti kaikki hänen työnsä.
\par 15 Hän teki vaskesta kaksi pylvästä. Toinen pylväs oli kahdeksantoista kyynärän korkuinen, ja kahdentoista kyynärän pituinen nauha ulottui toisen pylvään ympäri.
\par 16 Hän teki myös kaksi pylväänpäätä, vaskesta valettua, pylväiden päähän pantavaksi. Kummankin pylväänpään korkeus oli viisi kyynärää.
\par 17 Verkon kaltaisia ristikkokoristeita, vitjan kaltaisia riippukoristeita, oli pylväänpäissä, jotka olivat pylväiden päässä, seitsemän kummassakin pylväänpäässä.
\par 18 Ja hän teki granaattiomenia kahteen riviin toisen ristikkokoristeen päälle yltympäri, peittämään pylväänpäät, jotka olivat pylväiden päässä; ja samoin hän teki niitä toiseen pylväänpäähän.
\par 19 Ja pylväänpäät, jotka olivat pylväiden päässä eteisessä, olivat liljan muotoiset, nelikyynäräiset.
\par 20 Pylväänpäissä, ylhäällä kahden pylvään päässä, oli kupevat alaosat, joihin ristikkokoriste ei ulottunut. Ja granaattiomenia oli kaksisataa, rivittäin yltympäri toisen pylväänpään päällä.
\par 21 Ja hän pystytti pylväät temppelin eteisen eteen. Pylväälle, jonka hän pystytti oikealle puolelle, hän antoi nimen Jaakin, ja pylväälle, jonka hän pystytti vasemmalle puolelle, hän antoi nimen Booas.
\par 22 Ylinnä pylväiden päällä oli liljan muotoinen laite. Ja niin päättyi pylväiden valmistus.
\par 23 Hän teki myös meren, valetun, kymmentä kyynärää leveän reunasta reunaan, ympärinsä pyöreän ja viittä kyynärää korkean; ja kolmenkymmenen kyynärän pituinen mittanuora ulottui sen ympäri.
\par 24 Sen reunan alla oli metsäkurpitsikoristeita, jotka kulkivat sen ympäri. Ne ympäröivät merta yltympäri, kymmenen jokaisella kyynärällä. Metsäkurpitsikoristeet olivat kahdessa rivissä, valettuina meren kanssa yhteen.
\par 25 Ja se seisoi kahdentoista raavaan varassa, joista kolme oli käännettynä pohjoiseen, kolme länteen, kolme etelään ja kolme itään päin; meri oli niiden yläpuolella, niiden varassa, ja kaikkien niiden takapuolet olivat sisäänpäin.
\par 26 Se oli kämmenen paksuinen, ja sen reuna oli maljan reunan kaltainen, puhjenneen liljan muotoinen; se veti kaksituhatta bat-mittaa.
\par 27 Hän teki myös kymmenen telinettä vaskesta. Kukin teline oli neljää kyynärää pitkä, neljää kyynärää leveä ja kolmea kyynärää korkea.
\par 28 Ja telineet olivat rakenteeltaan tällaiset: niissä oli kehäpienat, ja myös poikkitankojen välissä oli kehäpienat.
\par 29 Kehäpienain päällä, jotka olivat poikkitankojen välissä, oli leijonia, raavaita ja kerubeja, ja samoin poikkitankojen päällä, sekä ylhäällä että alhaalla. Leijonissa ja raavaissa oli takomalla tehtyjä punonnaiskoristeita.
\par 30 Kussakin telineessä oli neljä vaskipyörää ja vaskiakselit; ja niiden neljässä jalkapylväässä oli olkapää. Olkapäät olivat valetut altaan alle; kunkin ulkopuolella oli punonnaiskoristeita.
\par 31 Telineen aukko oli olkapäiden sisäpuolella, ja sen reuna oli kyynärän korkuinen; aukko oli pyöreä, jalustan tapaan tehty, puolitoistakyynäräinen. Myöskin aukon reunassa oli leikkauksia. Kehäpienat olivat nelikulmaiset eivätkä pyöreät.
\par 32 Ja ne neljä pyörää olivat kehäpienain alla, ja pyöräin pitimet olivat telineessä kiinni. Kukin pyörä oli puoltatoista kyynärää korkea.
\par 33 Pyörät olivat tehdyt niinkuin vaununpyörät; ja niiden pitimet, kehät, puolat ja navat olivat kaikki valetut.
\par 34 Neljä olkapäätä oli kussakin telineessä, sen neljässä kulmassa; olkapäät olivat yhtä telineen kanssa.
\par 35 Ylinnä telineen päällä oli puolen kyynärän korkuinen laite, ympärinsä pyöreä; ja telineen pitimet ja kehäpienat olivat yhtä sen kanssa.
\par 36 Ja sen kehäpienojen pintoihin hän kaiversi kerubeja, leijonia ja palmuja, niin paljon kuin kussakin oli tilaa, sekä punonnaiskoristeita yltympäri.
\par 37 Näin hän teki ne kymmenen telinettä; ne olivat kaikki valetut samalla tavalla, yhtä suuret ja yhdenmuotoiset.
\par 38 Hän teki myös kymmenen vaskiallasta; kukin allas veti neljäkymmentä bat-mittaa, ja kukin allas oli neljää kyynärää läpimitaten. Kullakin kymmenellä telineellä oli altaansa.
\par 39 Ja hän asetti viisi telinettä temppelin oikealle sivulle ja viisi temppelin vasemmalle sivulle; meren hän asetti temppelin oikealle sivulle, kaakkoa kohti.
\par 40 Hiiram teki myös kattilat, lapiot ja maljat. Ja niin Hiiram sai suoritetuksi kaiken työn, mikä hänen oli tehtävä kuningas Salomolle Herran temppeliin:
\par 41 kaksi pylvästä ja ne kaksi pallonmuotoista pylväänpäätä, jotka olivat pylväiden päässä, ja ne kaksi ristikkokoristetta peittämään niitä kahta pallonmuotoista pylväänpäätä, jotka olivat pylväiden päässä;
\par 42 ja ne neljäsataa granaattiomenaa kahteen ristikkokoristeeseen, kaksi riviä granaattiomenia kumpaankin ristikkokoristeeseen, peittämään niitä kahta pallonmuotoista pylväänpäätä, jotka olivat pylväiden päässä;
\par 43 ja ne kymmenen telinettä ja ne kymmenen allasta telineiden päälle;
\par 44 ja sen yhden meren, ja ne kaksitoista raavasta meren alle; ja kattilat, lapiot ja maljat.
\par 45 Kaikki nämä kalut, jotka Hiiram teki kuningas Salomolle Herran temppeliin, olivat kiilloitetusta vaskesta.
\par 46 Jordanin lakeudella kuningas ne valatti savimuotteihin, Sukkotin ja Saaretanin välillä.
\par 47 Ja Salomo jätti kaikki kalut punnitsematta, koska niitä oli ylen paljon; vasken painoa ei määrätty.
\par 48 Salomo teetti myös kaikki muut kalut, mitä Herran temppelissä on: kulta-alttarin, pöydän, jolla näkyleivät ovat, kullasta,
\par 49 lampunjalat, viisi oikealle puolelle ja viisi vasemmalle puolelle kaikkeinpyhimmän eteen, puhtaasta kullasta, kultaisine kukkalehtineen, lamppuineen ja lamppusaksineen,
\par 50 vadit, veitset, maljat, kupit ja hiilipannut, puhtaasta kullasta, sekä kultasaranat niihin temppelin sisäosan oviin, jotka vievät kaikkeinpyhimpään, ja niihin temppelin oviin, jotka vievät temppelisaliin.
\par 51 Kun kaikki työ, minkä kuningas Salomo teetti Herran temppeliin, oli valmis, vei Salomo sinne isänsä Daavidin pyhät lahjat; hopean, kullan ja kalut hän pani Herran temppelin aarrekammioihin.

\chapter{8}

\par 1 Sitten Salomo kokosi Israelin vanhimmat ja kaikki sukukuntien johtomiehet, israelilaisten perhekunta-päämiehet, kuningas Salomon luo Jerusalemiin, tuomaan Herran liitonarkkia Daavidin kaupungista, se on Siionista.
\par 2 Niin kokoontuivat kuningas Salomon luo kaikki Israelin miehet juhlapäivänä eetanim-kuussa, joka on seitsemäs kuukausi.
\par 3 Ja kun kaikki Israelin vanhimmat olivat tulleet saapuville, nostivat papit arkin,
\par 4 ja he veivät Herran arkin ja ilmestysmajan sinne, sekä kaiken pyhän kaluston, joka oli majassa; papit ja leeviläiset veivät ne sinne.
\par 5 Ja kuningas Salomo seisoi arkin edessä ja hänen kanssaan koko Israelin kansa, joka oli kokoontunut hänen luoksensa; ja he uhrasivat lampaita ja raavaita niin paljon, että niitä ei voitu lukea, ei laskea.
\par 6 Ja papit toivat Herran liitonarkin paikoilleen temppelin kuoriin, kaikkeinpyhimpään, kerubien siipien alle.
\par 7 Sillä kerubit levittivät siipensä sen paikan yli, missä arkki oli, ja kerubit suojasivat ylhäältä päin arkkia ja sen korentoja.
\par 8 Ja korennot olivat niin pitkät, että niiden päät voi nähdä kaikkeinpyhimmästä, kuorin edestä; mutta ulkoa niitä ei voinut nähdä. Ja ne jäivät sinne, tähän päivään asti.
\par 9 Arkissa ei ollut muuta kuin ne kaksi kivitaulua, jotka Mooses oli pannut sinne Hoorebilla, kun Herra teki liiton israelilaisten kanssa, heidän lähdettyänsä Egyptin maasta.
\par 10 Ja kun papit lähtivät pyhäköstä, täytti pilvi Herran temppelin,
\par 11 niin että papit eivät voineet astua toimittamaan virkaansa pilven tähden; sillä Herran kirkkaus täytti Herran temppelin.
\par 12 Silloin Salomo sanoi: "Herra on sanonut tahtovansa asua pimeässä.
\par 13 Minä olen rakentanut huoneen sinulle asunnoksi, asuinsijan, asuaksesi siinä iäti."
\par 14 Sitten kuningas käänsi kasvonsa ja siunasi koko Israelin seurakunnan; ja koko Israelin seurakunta seisoi.
\par 15 Hän sanoi: "Kiitetty olkoon Herra, Israelin Jumala, joka kädellänsä on täyttänyt sen, mitä hän suullansa puhui minun isälleni Daavidille, sanoen:
\par 16 'Siitä päivästä saakka, jona minä vein kansani Israelin pois Egyptistä, en ole mistään Israelin sukukunnasta valinnut yhtään kaupunkia, että siihen rakennettaisiin temppeli, missä minun nimeni asuisi; mutta minä olen valinnut Daavidin vallitsemaan kansaani Israelia'.
\par 17 Ja minun isäni Daavid aikoi rakentaa temppelin Herran, Israelin Jumalan, nimelle.
\par 18 Mutta Herra sanoi minun isälleni Daavidille: 'Kun sinä aiot rakentaa temppelin minun nimelleni, niin tosin teet siinä hyvin, että sitä aiot;
\par 19 kuitenkaan et sinä ole sitä temppeliä rakentava, vaan sinun poikasi, joka lähtee sinun kupeistasi, hän on rakentava temppelin minun nimelleni'.
\par 20 Ja Herra on täyttänyt sanansa, jonka hän puhui: minä olen noussut isäni Daavidin sijalle ja istun Israelin valtaistuimella, niinkuin Herra puhui, ja minä olen rakentanut temppelin Herran, Israelin Jumalan, nimelle.
\par 21 Ja minä olen valmistanut siihen sijan arkille, jossa on se Herran liitto, minkä hän teki meidän isiemme kanssa viedessään heidät pois Egyptin maasta."
\par 22 Sitten Salomo astui Herran alttarin eteen koko Israelin seurakunnan nähden, ojensi kätensä taivasta kohti
\par 23 ja sanoi: "Herra, Israelin Jumala, ei ole sinun vertaistasi jumalaa, ei ylhäällä taivaassa eikä alhaalla maan päällä, sinun, joka pidät liiton ja säilytät laupeuden palvelijoitasi kohtaan, jotka vaeltavat sinun edessäsi kaikesta sydämestänsä,
\par 24 sinun, joka olet pitänyt, mitä puhuit palvelijallesi Daavidille, minun isälleni. Mitä suullasi puhuit, sen sinä kädelläsi täytit, niinkuin nyt on tapahtunut.
\par 25 Niin pidä nytkin, Herra, Israelin Jumala, mitä puhuit palvelijallesi Daavidille, minun isälleni, sanoen: 'Aina on mies sinun suvustasi istuva minun edessäni Israelin valtaistuimella, jos vain sinun poikasi pitävät vaarin tiestänsä, niin että he vaeltavat minun edessäni, niinkuin sinä olet minun edessäni vaeltanut'.
\par 26 Niin toteutukoot nyt, Israelin Jumala, sinun sanasi, jotka puhuit palvelijallesi Daavidille, minun isälleni.
\par 27 Mutta asuuko todella Jumala maan päällä? Katso, taivaisiin ja taivasten taivaisiin sinä et mahdu; kuinka sitten tähän temppeliin, jonka minä olen rakentanut!
\par 28 Käänny kuitenkin palvelijasi rukouksen ja anomisen puoleen, Herra, minun Jumalani, niin että kuulet huudon ja rukouksen, jonka palvelijasi tänä päivänä rukoilee sinun edessäsi,
\par 29 ja että silmäsi ovat yöt ja päivät avoinna tätä temppeliä kohti, tätä paikkaa kohti, josta sinä olet sanonut: 'Minun nimeni on asuva siellä', niin että kuulet rukouksen, jonka palvelijasi tähän paikkaan päin kääntyneenä rukoilee.
\par 30 Kuule palvelijasi ja kansasi Israelin rukous, jonka he rukoilevat tähän paikkaan päin kääntyneinä; kuule asuinpaikastasi, taivaasta, ja kun kuulet, niin anna anteeksi.
\par 31 Jos joku rikkoo lähimmäistänsä vastaan ja hänet pannaan valalle ja vannotetaan, ja jos hän tulee ja vannoo sinun alttarisi edessä tässä temppelissä,
\par 32 niin kuule taivaasta ja auta palvelijasi oikeuteensa; tee niin, että tuomitset syyllisen syylliseksi ja annat hänen tekojensa tulla hänen päänsä päälle, mutta julistat syyttömän syyttömäksi ja annat hänelle hänen vanhurskautensa mukaan.
\par 33 Jos vihollinen voittaa sinun kansasi, Israelin, sentähden että he ovat tehneet syntiä sinua vastaan, mutta he palajavat sinun tykösi, kiittävät sinun nimeäsi, rukoilevat sinua ja anovat sinulta armoa tässä temppelissä,
\par 34 niin kuule taivaasta ja anna anteeksi kansasi Israelin synti ja tuo heidät takaisin tähän maahan, jonka olet antanut heidän isillensä.
\par 35 Jos taivas suljetaan, niin ettei tule sadetta, koska he ovat tehneet syntiä sinua vastaan, mutta he rukoilevat kääntyneinä tähän paikkaan päin, ja kiittävät sinun nimeäsi ja kääntyvät synnistään, koska sinä kuulet heitä,
\par 36 niin kuule taivaasta ja anna anteeksi palvelijaisi ja kansasi Israelin synti - sillä sinä osoitat heille hyvän tien, jota heidän on vaeltaminen - ja suo sade maallesi, jonka olet antanut kansallesi perintöosaksi.
\par 37 Jos maahan tulee nälänhätä, rutto, jos tulee nokitähkä ja viljanruoste, jos tulevat heinäsirkat ja tuhosirkat, jos vihollinen ahdistaa sitä maassa, jossa sen portit ovat, jos tulee mikä tahansa vitsaus tai vaiva,
\par 38 ja jos silloin joku ihminen, kuka hyvänsä, tai koko sinun kansasi Israel rukoilee ja anoo armoa, kun he kukin tuntevat omantunnon vaivoja ja ojentavat kätensä tähän temppeliin päin,
\par 39 niin kuule silloin taivaasta, asuinpaikastasi, ja anna anteeksi ja tee niin, että annat jokaiselle aivan hänen tekojensa mukaan, koska sinä tunnet hänen sydämensä - sillä sinä yksin tunnet kaikkien ihmislasten sydämet -
\par 40 jotta he sinua pelkäisivät niin kauan kuin he elävät tässä maassa, jonka sinä olet meidän isillemme antanut.
\par 41 Myös jos joku muukalainen, joka ei ole sinun kansaasi Israelia, tulee kaukaisesta maasta sinun nimesi tähden -
\par 42 sillä sielläkin kuullaan sinun suuresta nimestäsi, väkevästä kädestäsi ja ojennetusta käsivarrestasi - jos hän tulee ja rukoilee kääntyneenä tähän temppeliin päin,
\par 43 niin kuule taivaasta, asuinpaikastasi, häntä ja tee kaikki, mitä muukalainen sinulta rukoilee, että kaikki maan kansat tuntisivat sinun nimesi ja pelkäisivät sinua, samoin kuin sinun kansasi Israel, ja tulisivat tietämään, että sinä olet ottanut nimiisi tämän temppelin, jonka minä olen rakentanut.
\par 44 Jos sinun kansasi lähtee sotaan vihollistansa vastaan sitä tietä, jota sinä heidät lähetät, ja he rukoilevat Herraa kääntyneinä tähän kaupunkiin päin, jonka sinä olet valinnut, ja tähän temppeliin päin, jonka minä olen sinun nimellesi rakentanut,
\par 45 niin kuule taivaasta heidän rukouksensa ja anomisensa ja hanki heille oikeus.
\par 46 Jos he tekevät syntiä sinua vastaan - sillä ei ole ihmistä, joka ei syntiä tee - ja sinä vihastut heihin ja annat heidät vihollisen valtaan, niin että heidän vangitsijansa vievät heidät vangeiksi vihollismaahan, kaukaiseen tai läheiseen,
\par 47 mutta jos he sitten menevät itseensä siinä maassa, johon heidät on vangeiksi viety, kääntyvät ja anovat sinulta armoa vangitsijainsa maassa, sanoen: 'Me olemme tehneet syntiä, tehneet väärin ja olleet jumalattomat',
\par 48 ja palajavat sinun tykösi kaikesta sydämestään ja kaikesta sielustaan vihollistensa maassa, jotka veivät heidät vangeiksi, ja rukoilevat sinua kääntyneinä tähän maahan päin, jonka sinä olet antanut heidän isillensä, tähän kaupunkiin päin, jonka olet valinnut, ja tähän temppeliin päin, jonka minä olen sinun nimellesi rakentanut,
\par 49 niin kuule taivaasta, asuinsijastasi, heidän rukouksensa ja anomisensa, hanki heille oikeus
\par 50 ja anna anteeksi kansallesi, mitä he ovat rikkoneet sinua vastaan, ja kaikki heidän syntinsä, jotka he ovat tehneet sinua vastaan; ja suo, että heidän vangitsijansa olisivat laupiaat heitä kohtaan ja armahtaisivat heitä.
\par 51 Sillä ovathan he sinun kansasi ja sinun perintöosasi, jonka olet vienyt pois Egyptistä, rautapätsistä.
\par 52 Olkoot siis sinun silmäsi avoinna palvelijasi ja sinun kansasi Israelin anomisen puoleen, niin että kuulet heitä kaikessa, mitä he sinulta rukoilevat.
\par 53 Sillä sinä olet erottanut heidät perintöosaksesi kaikista maan kansoista, niinkuin sinä puhuit palvelijasi Mooseksen kautta, viedessäsi isämme pois Egyptistä, Herra, Herra."
\par 54 Ja kun Salomo oli lakannut rukoilemasta ja anomasta Herralta kaikkea tätä, nousi hän Herran alttarin edestä, jossa hän oli ollut polvillaan kädet ojennettuina taivasta kohti,
\par 55 astui esille ja siunasi koko Israelin seurakunnan suurella äänellä, sanoen:
\par 56 "Kiitetty olkoon Herra, joka on antanut levon kansallensa Israelille, aivan niinkuin hän on puhunut. Ei ole jäänyt täyttämättä ainoakaan kaikista lupauksista, jotka Herra on antanut palvelijansa Mooseksen kautta.
\par 57 Olkoon Herra, meidän Jumalamme, meidän kanssamme, niinkuin hän on ollut meidän isiemme kanssa. Älköön hän meitä jättäkö älköönkä hyljätkö,
\par 58 vaan kääntäköön meidän sydämemme puoleensa, niin että aina vaellamme hänen teitänsä ja noudatamme hänen käskyjänsä, säädöksiänsä ja oikeuksiansa, jotka hän on antanut meidän isillemme.
\par 59 Ja olkoot nämä minun sanani, joilla minä olen armoa anonut Herran edessä, päivät ja yöt likellä Herraa, meidän Jumalaamme, että hän hankkisi oikeuden palvelijallensa ja kansallensa Israelille, kunkin päivän tarpeen mukaan,
\par 60 niin että kaikki maan kansat tulisivat tietämään, että Herra on Jumala eikä muuta jumalaa ole.
\par 61 Ja antautukaa te ehyin sydämin Herralle, Jumalallenne, niin että vaellatte hänen säädöksiensä mukaan ja noudatatte hänen käskyjänsä samoin kuin nytkin."
\par 62 Sitten kuningas ja koko Israel hänen kanssansa uhrasivat teurasuhrin Herran edessä.
\par 63 Ja Salomo uhrasi yhteysuhrina, jonka hän Herralle uhrasi, kaksikymmentäkaksi tuhatta raavasta ja satakaksikymmentä tuhatta lammasta. Niin he, kuningas ja kaikki israelilaiset, vihkivät Herran temppelin.
\par 64 Sinä päivänä kuningas pyhitti Herran temppelin edessä olevan esipihan keskiosan, sillä hänen oli siellä uhrattava polttouhri, ruokauhri ja yhteysuhrin rasvat. Vaskialttari, joka oli Herran edessä, oli näet liian pieni, että polttouhri, ruokauhri ja yhteysuhrin rasvat olisivat siihen mahtuneet.
\par 65 Näin Salomo siihen aikaan vietti juhlaa, ja koko Israel hänen kanssansa, Herran, meidän Jumalamme, edessä seitsemän päivää ja vielä toiset seitsemän päivää, yhteensä neljätoista päivää. Se oli suuri kokous, johon kokoonnuttiin aina sieltä, mistä mennään Hamatiin, ja aina Egyptin purolta asti.
\par 66 Kahdeksantena päivänä hän päästi kansan menemään, ja he hyvästelivät kuninkaan. Sitten he menivät majoillensa iloiten ja hyvillä mielin kaikesta hyvästä, mitä Herra oli tehnyt palvelijallensa Daavidille ja kansallensa Israelille.

\chapter{9}

\par 1 Kun Salomo oli saanut rakennetuksi Herran temppelin ja kuninkaan linnan ja kaiken, mitä Salomo oli halunnut tehdä,
\par 2 ilmestyi Herra Salomolle toisen kerran,
\par 3 niinkuin hän oli ilmestynyt hänelle Gibeonissa. Ja Herra sanoi hänelle: "Minä olen kuullut sinun rukouksesi ja anomisesi, kun sinä anoit armoa minun edessäni. Minä olen pyhittänyt tämän temppelin, jonka sinä olet rakentanut sitä varten, että minä sijoittaisin nimeni siihen ainiaaksi; ja minun silmäni ja sydämeni tulevat alati olemaan siellä.
\par 4 Ja jos sinä vaellat minun edessäni, niinkuin sinun isäsi Daavid vaelsi, vilpittömällä sydämellä ja oikeamielisesti, niin että teet kaiken, mitä minä olen käskenyt sinun tehdä, ja noudatat minun käskyjäni ja oikeuksiani,
\par 5 niin minä pidän pystyssä sinun kuninkaallisen valtaistuimesi Israelissa ikuisesti, niinkuin minä puhuin isällesi Daavidille sanoen: 'Aina on mies sinun suvustasi oleva Israelin valtaistuimella'.
\par 6 Mutta jos te käännytte pois minusta, te ja teidän lapsenne, ettekä noudata minun käskyjäni ja säädöksiäni, jotka minä olen teille antanut, vaan menette ja palvelette muita jumalia ja kumarratte niitä,
\par 7 niin minä hävitän Israelin siitä maasta, jonka olen antanut heille, ja temppelin, jonka minä olen pyhittänyt nimelleni, minä heitän pois kasvojeni edestä; ja Israel tulee sananparreksi ja pistopuheeksi kaikille kansoille.
\par 8 Ja tämä temppeli on tosin korkein, mutta jokainen, joka kulkee siitä ohi, on tyrmistyvä ja viheltävä; ja kun kysytään: 'Miksi on Herra tehnyt näin tälle maalle ja tälle temppelille?'
\par 9 niin vastataan: 'Siksi, että he hylkäsivät Herran, Jumalansa, joka oli vienyt heidän isänsä pois Egyptin maasta, ja liittyivät muihin jumaliin ja kumarsivat niitä ja palvelivat niitä; sentähden Herra on antanut kaiken tämän pahan kohdata heitä'."
\par 10 Kun olivat loppuun kuluneet ne kaksikymmentä vuotta, joiden kuluessa Salomo oli rakentanut ne kaksi rakennusta, Herran temppelin ja kuninkaan linnan,
\par 11 Hiiramin, Tyyron kuninkaan, avustaessa Salomoa setripuilla, kypressipuilla ja kullalla, niin paljolla kuin tämä halusi, silloin kuningas Salomo antoi Hiiramille kaksikymmentä kaupunkia Galilean maakunnasta.
\par 12 Niin Hiiram lähti Tyyrosta katsomaan niitä kaupunkeja, jotka Salomo oli antanut hänelle, mutta ne eivät miellyttäneet häntä.
\par 13 Hän sanoi: "Mitä kaupunkeja nämä ovat, jotka sinä olet antanut minulle, veljeni!" Ja niin kutsutaan niitä Kaabulin maaksi vielä tänäkin päivänä.
\par 14 Mutta Hiiram lähetti kuninkaalle sata kaksikymmentä talenttia kultaa.
\par 15 Ja näin on sen työveron laita, jonka kuningas Salomo otatti rakentaaksensa Herran temppeliä, omaa linnaansa, Milloa ja Jerusalemin muuria sekä Haasoria, Megiddoa ja Geseriä.
\par 16 Farao, Egyptin kuningas, oli näet tullut ja valloittanut Geserin, polttanut sen tulella ja surmannut kanaanilaiset, jotka asuivat kaupungissa, ja antanut sen myötäjäisiksi tyttärellensä, Salomon puolisolle.
\par 17 Ja Salomo linnoitti Geserin, alisen Beet-Hooronin,
\par 18 Baalatin ja Taamarin sen maan erämaassa
\par 19 ja kaikki varastokaupungit, jotka hänellä oli, ja kaikki vaunukaupungit ja ratsumiesten kaupungit ja mitä muuta Salomo oli tahtonut linnoittaa Jerusalemissa ja Libanonilla ja koko hallitsemassaan maassa.
\par 20 Kaiken kansan, mitä oli jäänyt jäljelle amorilaisista, heettiläisistä, perissiläisistä, hivviläisistä ja jebusilaisista, kaikki, jotka eivät olleet israelilaisia,
\par 21 ne niiden jälkeläiset, jotka vielä olivat jäljellä maassa ja joita israelilaiset eivät olleet kyenneet vihkimään tuhon omiksi, ne Salomo saattoi työveron alaisiksi, aina tähän päivään asti.
\par 22 Mutta israelilaisista Salomo ei tehnyt ketään orjaksi, vaan heitä oli sotilaina, hänen palvelijoinaan, päällikköinään, vaunusotureinaan ja hänen sotavaunujensa ja ratsumiestensä päällikköinä.
\par 23 Maaherrojen virkamiehiä, jotka valvoivat Salomon töitä, oli viisisataa viisikymmentä miestä, vallitsemassa väkeä, joka teki työtä.
\par 24 Heti kun faraon tytär oli muuttanut Daavidin kaupungista omaan linnaansa, jonka hän oli hänelle rakentanut, rakensi hän Millon.
\par 25 Ja Salomo uhrasi kolme kertaa vuodessa polttouhreja ja yhteysuhreja alttarilla, jonka hän oli rakentanut Herralle, ja suitsutti alttarin ääressä, joka oli Herran edessä. Ja niin hän oli saanut temppelin valmiiksi.
\par 26 Kuningas Salomo rakensi myös laivaston Esjon-Geberissä, joka on lähellä Eelatia Kaislameren rannalla Edomin maassa.
\par 27 Tähän laivastoon Hiiram lähetti palvelijoitaan, meritaitoisia laivamiehiä, Salomon palvelijain mukaan.
\par 28 He menivät Oofiriin ja noutivat sieltä kultaa neljäsataa kaksikymmentä talenttia ja toivat sen kuningas Salomolle.

\chapter{10}

\par 1 Kun Saban kuningatar kuuli, mitä Salomosta kerrottiin Herran nimen kunniaksi, tuli hän koettelemaan häntä arvoituksilla.
\par 2 Hän tuli Jerusalemiin sangen suuren seurueen kanssa, mukanaan kameleja, jotka kantoivat hajuaineita, kultaa ylen paljon ja kalliita kiviä. Ja kun hän tuli Salomon luo, puhui hän tälle kaikki, mitä hänellä oli mielessänsä.
\par 3 Mutta Salomo selitti hänelle kaikki hänen kysymyksensä; kuninkaalle ei mikään jäänyt ongelmaksi, jota hän ei olisi hänelle selittänyt.
\par 4 Kun Saban kuningatar näki kaiken Salomon viisauden, linnan, jonka hän oli rakentanut,
\par 5 ruuat hänen pöydällänsä, kuinka hänen palvelijansa asuivat ja hänen palvelusväkensä palveli ja kuinka he olivat puetut, ja näki hänen juomanlaskijansa ja hänen polttouhrinsa, jotka hän uhrasi Herran temppelissä, meni hän miltei hengettömäksi.
\par 6 Sitten hän sanoi kuninkaalle: "Totta oli se puhe, jonka minä kotimaahani sinusta ja sinun viisaudestasi kuulin.
\par 7 Minä en uskonut, mitä sanottiin, ennenkuin itse tulin ja sain omin silmin nähdä; ja katso, ei puoltakaan oltu minulle kerrottu. Sinulla on paljon enemmän viisautta ja rikkautta, kuin minä olin kuullut huhuttavan.
\par 8 Onnellisia ovat sinun miehesi, onnellisia nämä palvelijasi, jotka aina saavat olla sinun edessäsi ja kuulla sinun viisauttasi.
\par 9 Kiitetty olkoon Herra, sinun Jumalasi, joka sinuun on niin mielistynyt, että on asettanut sinut Israelin valtaistuimelle. Sentähden, että Herra rakastaa Israelia ainiaan, hän on pannut sinut kuninkaaksi, tekemään sitä, mikä oikeus ja vanhurskaus on."
\par 10 Ja hän antoi kuninkaalle sata kaksikymmentä talenttia kultaa, hyvin paljon hajuaineita ja kalliita kiviä. Niin paljon hajuaineita, kuin mitä Saban kuningatar antoi kuningas Salomolle, ei sinne ole sen jälkeen tullut.
\par 11 Myöskin Hiiramin laivat, jotka kuljettivat kultaa Oofirista, toivat Oofirista hyvin paljon santelipuuta ja kalliita kiviä.
\par 12 Ja kuningas teetti santelipuusta kalustoa Herran temppeliin ja kuninkaan linnaan ja kanteleita ja harppuja laulajille. Sellaista santelipuumäärää ei ole tuotu eikä nähty tähän päivään asti.
\par 13 Kuningas Salomo taas antoi Saban kuningattarelle kaiken, mitä tämä halusi ja pyysi, ja sen lisäksi hän antoi muutakin kuninkaallisella anteliaisuudella. Sitten tämä lähti ja meni palvelijoineen omaan maahansa.
\par 14 Sen kullan paino, mikä yhtenä vuotena tuli Salomolle, oli kuusisataa kuusikymmentä kuusi talenttia kultaa,
\par 15 paitsi mitä tuli kauppamiehiltä ja kaupustelijain kaupanteosta ja kaikilta Erebin kuninkailta ja maan käskynhaltijoilta.
\par 16 Ja kuningas Salomo teetti pakotetusta kullasta kaksisataa suurta kilpeä ja käytti jokaiseen kilpeen kuusisataa sekeliä kultaa;
\par 17 ja pakotetusta kullasta kolmesataa pienempää kilpeä, ja käytti jokaiseen kilpeen kolme miinaa kultaa. Ja kuningas asetti ne Libanoninmetsä-taloon.
\par 18 Vielä kuningas teetti suuren norsunluisen valtaistuimen ja päällysti sen puhdistetulla kullalla.
\par 19 Valtaistuimessa oli kuusi porrasta, ja valtaistuimen selusta oli ylhäältä pyöreä. Istuimen kummallakin puolella oli käsinoja, ja kaksi leijonaa seisoi käsinojain vieressä.
\par 20 Ja kaksitoista leijonaa seisoi siinä kuudella portaalla, kummallakin puolella. Senkaltaista ei ole tehty missään muussa valtakunnassa.
\par 21 Ja kaikki kuningas Salomon juoma-astiat olivat kultaa, ja kaikki Libanoninmetsä-talon astiat olivat puhdasta kultaa. Hopeata ei siellä ollut, eikä sitä Salomon päivinä pidetty minkään arvoisena.
\par 22 Kuninkaalla oli Tarsiin-laivasto merellä Hiiramin laivaston kanssa. Kerran kolmessa vuodessa Tarsiin-laivasto tuli ja toi kultaa ja hopeata, norsunluuta, apinoita ja riikinkukkoja.
\par 23 Ja kuningas Salomo oli kaikkia maan kuninkaita suurempi rikkaudessa ja viisaudessa.
\par 24 Ja kaikki maa pyrki näkemään Salomoa kuullakseen hänen viisauttaan, jonka Jumala oli antanut hänen sydämeensä.
\par 25 Ja he toivat kukin lahjansa: hopea- ja kultakaluja, vaatteita, aseita, hajuaineita, hevosia ja muuleja, joka vuosi vuoden tarpeen.
\par 26 Salomo kokosi sotavaunuja ja ratsumiehiä, niin että hänellä oli tuhannet neljätsadat sotavaunut ja kaksitoista tuhatta ratsumiestä; ne hän sijoitti vaunukaupunkeihin ja Jerusalemiin kuninkaan luo.
\par 27 Ja kuningas toimitti niin, että Jerusalemissa oli hopeata kuin kiviä, ja setripuuta niin paljon kuin metsäviikunapuita Alankomaassa.
\par 28 Ja hevoset, mitä Salomolla oli, tuotiin Egyptistä ja Kuvesta. Kuninkaan kauppiaat noutivat niitä Kuvesta maksua vastaan.
\par 29 Egyptistä tuodut vaunut maksoivat kuusisataa hopeasekeliä ja hevonen sata viisikymmentä. Samoin tuotiin niitä heidän välityksellään kaikille heettiläisten ja aramilaisten kuninkaille.

\chapter{11}

\par 1 Mutta kuningas Salomolla oli paitsi faraon tytärtä monta muuta muukalaista vaimoa, joita hän rakasti: mooabilaisia, ammonilaisia, edomilaisia, siidonilaisia ja heettiläisiä,
\par 2 niiden kansain naisia, joista Herra oli sanonut israelilaisille: "Älkää yhtykö heihin, älköötkä hekään yhtykö teihin; he varmasti taivuttavat teidän sydämenne seuraamaan heidän jumaliansa". Näihin Salomo kiintyi rakkaudella.
\par 3 Hänellä oli seitsemänsataa ruhtinaallista puolisoa ja kolmesataa sivuvaimoa; ja hänen vaimonsa taivuttivat hänen sydämensä.
\par 4 Ja kun Salomo vanheni, taivuttivat hänen vaimonsa hänen sydämensä seuraamaan muita jumalia, niin ettei hän antautunut ehyin sydämin Herralle, Jumalallensa, niinkuin hänen isänsä Daavidin sydän oli ollut.
\par 5 Niin Salomo lähti seuraamaan Astartea, siidonilaisten jumalatarta, ja Milkomia, ammonilais-iljetystä.
\par 6 Ja Salomo teki sitä, mikä on pahaa Herran silmissä, eikä uskollisesti seurannut Herraa niinkuin hänen isänsä Daavid.
\par 7 Silloin Salomo rakensi Kemokselle, mooabilais-iljetykselle, uhrikukkulan sille vuorelle, joka on itään päin Jerusalemista, ja samoin Moolokille, ammonilais-iljetykselle.
\par 8 Näin hän teki kaikkien muukalaisten vaimojen mieliksi, jotka suitsuttivat ja uhrasivat jumalilleen.
\par 9 Niin Herra vihastui Salomoon, koska hänen sydämensä oli kääntynyt pois Herrasta, Israelin Jumalasta, joka kahdesti oli ilmestynyt hänelle
\par 10 ja nimenomaan kieltänyt häntä seuraamasta muita jumalia, ja koska hän ei ollut noudattanut Herran kieltoa.
\par 11 Sentähden Herra sanoi Salomolle: "Koska sinun on käynyt näin, ja koska et ole pitänyt minun liittoani etkä noudattanut minun käskyjäni, jotka minä sinulle annoin, niin minä repäisen valtakunnan sinulta ja annan sen sinun palvelijallesi.
\par 12 Mutta isäsi Daavidin tähden minä en tee tätä sinun päivinäsi; sinun poikasi kädestä minä sen repäisen.
\par 13 Kuitenkaan en minä repäise koko valtakuntaa: yhden sukukunnan minä annan sinun pojallesi palvelijani Daavidin tähden ja Jerusalemin tähden, jonka minä olen valinnut."
\par 14 Niin Herra nostatti Salomolle vastustajaksi edomilaisen Hadadin; tämä oli edomilaista kuningassukua.
\par 15 Kun Daavid oli sodassa Edomin kanssa ja sotapäällikkö Jooab meni hautaamaan kaatuneita ja surmasi kaikki miehenpuolet Edomissa -
\par 16 sillä Jooab ja koko Israel viipyi siellä kuusi kuukautta, kunnes olivat hävittäneet kaikki miehenpuolet Edomista -
\par 17 pakeni Hadad ja hänen kanssaan muutamat edomilaiset miehet, hänen isänsä palvelijat, Egyptiin päin; Hadad oli vielä pieni poikanen.
\par 18 He lähtivät liikkeelle Midianista ja tulivat Paaraniin. Ja he ottivat mukaansa miehiä Paaranista ja tulivat Egyptiin faraon, Egyptin kuninkaan, luo. Tämä antoi hänelle talon ja määräsi hänelle elatuksen ja antoi hänelle myös maata.
\par 19 Ja Hadad pääsi faraon suureen suosioon, niin että hän antoi hänelle vaimoksi kälynsä, kuningatar Tahpeneen sisaren.
\par 20 Tahpeneen sisar synnytti hänelle hänen poikansa Genubatin, ja Tahpenes vieroitti hänet faraon palatsissa; ja niin jäi Genubat faraon palatsiin, faraon lasten joukkoon.
\par 21 Kun Hadad Egyptissä kuuli, että Daavid oli mennyt lepoon isiensä tykö ja että sotapäällikkö Jooab oli kuollut, sanoi Hadad faraolle: "Päästä minut menemään omaan maahani".
\par 22 Mutta farao sanoi hänelle: "Mitä sinulta puuttuu minun luonani, koska haluat mennä omaan maahasi?" Hän vastasi: "Ei mitään, mutta päästä minut".
\par 23 Ja Jumala nostatti Salomolle vastustajaksi Resonin, Eljadan pojan, joka oli paennut herransa Hadadeserin, Sooban kuninkaan, luota.
\par 24 Tämä kokosi miehiä ympärilleen ja oli partiojoukon päällikkönä silloin, kun Daavid surmasi heitä. He menivät sitten Damaskoon, asettuivat sinne ja hallitsivat Damaskossa.
\par 25 Ja hän oli Israelin vastustaja, niin kauan kuin Salomo eli, ja teki sille pahaa samoin kuin Hadadkin. Hän inhosi Israelia; ja hänestä tuli Aramin kuningas.
\par 26 Myöskin Salomon palvelija Jerobeam, Nebatin poika, efraimilainen Seredasta, jonka äiti oli nimeltään Serua ja oli leskivaimo, kohotti kätensä kuningasta vastaan.
\par 27 Hän joutui kohottamaan kätensä kuningasta vastaan seuraavalla tavalla. Salomo rakensi Milloa ja sulki siten aukon isänsä Daavidin kaupungissa.
\par 28 Ja Jerobeam oli kelpo mies; ja kun Salomo näki, kuinka tämä nuori mies teki työtä, asetti hän hänet kaiken sen pakkotyön valvojaksi, mikä oli Joosefin heimon osalla.
\par 29 Siihen aikaan tapahtui, kun Jerobeam kerran oli lähtenyt Jerusalemista, että siilolainen Ahia, profeetta, kohtasi hänet tiellä. Tämä oli puettuna uuteen vaippaan, ja he olivat kahdenkesken kedolla.
\par 30 Silloin Ahia tarttui siihen uuteen vaippaan, joka hänellä oli yllään, ja repäisi sen kahdeksitoista kappaleeksi
\par 31 ja sanoi Jerobeamille: "Ota itsellesi kymmenen kappaletta, sillä näin sanoo Herra, Israelin Jumala: Katso, minä repäisen valtakunnan Salomon kädestä ja annan kymmenen sukukuntaa sinulle.
\par 32 Yksi sukukunta jääköön hänelle minun palvelijani Daavidin tähden ja Jerusalemin kaupungin tähden, jonka minä olen valinnut kaikista Israelin sukukunnista.
\par 33 Näin on tapahtuva, koska he ovat hyljänneet minut ja kumartaneet Astartea, siidonilaisten jumalatarta, ja Kemosta, Mooabin jumalaa, ja Milkomia, ammonilaisten jumalaa, eivätkä ole vaeltaneet minun teitäni eivätkä tehneet sitä, mikä on oikein minun silmissäni, eivätkä noudattaneet minun käskyjäni ja oikeuksiani niinkuin Salomon isä Daavid.
\par 34 Kuitenkaan en minä ota hänen kädestään koko valtakuntaa, vaan annan hänen olla ruhtinaana koko elinaikansa palvelijani Daavidin tähden, jonka minä valitsin, koska hän noudatti minun käskyjäni ja säädöksiäni.
\par 35 Mutta hänen poikansa kädestä minä otan kuninkuuden ja annan sen sinulle, nimittäin ne kymmenen sukukuntaa,
\par 36 ja hänen pojallensa minä annan yhden sukukunnan, että minun palvelijallani Daavidilla aina olisi lamppu palamassa minun edessäni Jerusalemissa, siinä kaupungissa, jonka minä olen itselleni valinnut, asettaakseni nimeni siihen.
\par 37 Mutta sinut minä otan, ja sinä saat hallittavaksesi kaikki, joita haluat; sinusta tulee Israelin kuningas.
\par 38 Jos sinä olet kuuliainen kaikessa, mitä minä käsken sinun tehdä, ja vaellat minun tietäni ja teet sitä, mikä on oikein minun silmissäni, ja noudatat minun säädöksiäni ja käskyjäni, niinkuin minun palvelijani Daavid teki, niin minä olen sinun kanssasi ja rakennan sinulle pysyväisen huoneen, niinkuin minä Daavidille rakensin, ja annan Israelin sinulle.
\par 39 Siitä syystä minä nöyryytän Daavidin jälkeläiset, en kuitenkaan ainiaaksi."
\par 40 Salomo koetti saada surmatuksi Jerobeamin; mutta Jerobeam lähti ja pakeni Egyptiin, Suusakin, Egyptin kuninkaan, luo. Ja hän oli Egyptissä Salomon kuolemaan asti.
\par 41 Mitä muuta Salomosta on kerrottavaa, kaikesta, mitä hän teki, ja hänen viisaudestansa, se on kirjoitettuna Salomon historiassa.
\par 42 Ja aika, minkä Salomo hallitsi Jerusalemissa koko Israelia, oli neljäkymmentä vuotta.
\par 43 Sitten Salomo meni lepoon isiensä tykö ja hänet haudattiin isänsä Daavidin kaupunkiin. Ja hänen poikansa Rehabeam tuli kuninkaaksi hänen sijaansa.

\chapter{12}

\par 1 Rehabeam meni Sikemiin, sillä koko Israel oli tullut Sikemiin tekemään häntä kuninkaaksi.
\par 2 Kun Jerobeam, Nebatin poika, kuuli sen - hän oli vielä Egyptissä, jonne hän oli paennut kuningas Salomoa, ja Jerobeam asui Egyptissä,
\par 3 mutta he lähettivät kutsumaan hänet - niin Jerobeam ja koko Israelin seurakunta tuli saapuville, ja he puhuivat Rehabeamille sanoen:
\par 4 "Sinun isäsi teki meidän ikeemme raskaaksi; mutta huojenna sinä nyt se kova työ, jota isäsi teetti, ja se raskas ies, jonka hän pani meidän niskaamme, niin me palvelemme sinua".
\par 5 Hän vastasi heille: "Menkää ja odottakaa kolme päivää ja tulkaa sitten takaisin minun tyköni". Ja kansa meni.
\par 6 Kuningas Rehabeam neuvotteli vanhain kanssa, jotka olivat palvelleet hänen isäänsä Salomoa, kun tämä vielä eli, ja kysyi: "Kuinka te neuvotte vastaamaan tälle kansalle?"
\par 7 He vastasivat hänelle ja sanoivat: "Jos sinä tänä päivänä rupeat tämän kansan palvelijaksi ja palvelet heitä, jos kuulet heitä ja puhut heille hyviä sanoja, niin he ovat sinun palvelijoitasi kaiken elinaikasi".
\par 8 Mutta hän hylkäsi tämän neuvon, jonka vanhat hänelle antoivat, ja neuvotteli nuorten miesten kanssa, jotka olivat kasvaneet hänen kanssaan ja jotka palvelivat häntä.
\par 9 Hän kysyi heiltä: "Kuinka te neuvotte meitä vastaamaan tälle kansalle, joka on puhunut minulle sanoen: 'Huojenna se ies, jonka sinun isäsi on pannut meidän niskaamme'?"
\par 10 Niin nuoret miehet, jotka olivat kasvaneet hänen kanssaan, vastasivat hänelle sanoen: "Sano näin tälle kansalle, joka on puhunut sinulle sanoen: 'Sinun isäsi teki meidän ikeemme raskaaksi, mutta huojenna sinä sitä meiltä' - puhu heille näin: 'Minun pikkusormeni on paksumpi kuin minun isäni lantio.
\par 11 Jos siis isäni on sälyttänyt teidän selkäänne raskaan ikeen, niin minä teen teidän ikeenne vielä raskaammaksi; jos isäni on kurittanut teitä raipoilla, niin minä kuritan teitä piikkiruoskilla.'"
\par 12 Niin Jerobeam ja kaikki kansa tuli Rehabeamin tykö kolmantena päivänä, niinkuin kuningas oli käskenyt sanoen: "Tulkaa takaisin minun tyköni kolmantena päivänä".
\par 13 Ja kuningas antoi kansalle kovan vastauksen, hyljäten sen neuvon, jonka vanhat olivat hänelle antaneet.
\par 14 Ja hän puhui heille nuorten miesten neuvon mukaan, sanoen: "Jos minun isäni on tehnyt teidän ikeenne raskaaksi, niin minä teen ikeenne vielä raskaammaksi; jos minun isäni on kurittanut teitä raipoilla, niin minä kuritan teitä piikkiruoskilla".
\par 15 Kuningas ei siis kuullut kansaa; sillä Herra sen niin salli täyttääkseen sanansa, jonka Herra oli puhunut Jerobeamille, Nebatin pojalle, siilolaisen Ahian kautta.
\par 16 Kun koko Israel huomasi, ettei kuningas heitä kuullut, vastasi kansa kuninkaalle näin: "Mitä osaa meillä on Daavidiin? Ei meillä ole perintöosaa Iisain poikaan. Majoillesi, Israel! Valvo nyt huonettasi, Daavid!" Ja Israel meni majoillensa.
\par 17 Niin Rehabeam tuli ainoastaan niiden israelilaisten kuninkaaksi, jotka asuivat Juudan kaupungeissa.
\par 18 Ja kun kuningas Rehabeam lähetti matkaan verotöiden valvojan Adoramin, kivitti koko Israel hänet kuoliaaksi. Silloin kuningas Rehabeam nousi nopeasti vaunuihinsa ja pakeni Jerusalemiin.
\par 19 Näin Israel luopui Daavidin suvusta, aina tähän päivään asti.
\par 20 Mutta kun koko Israel kuuli, että Jerobeam oli tullut takaisin, lähettivät he kutsumaan hänet kansankokoukseen ja tekivät hänet koko Israelin kuninkaaksi. Daavidin sukua ei seurannut kukaan muu kuin Juudan sukukunta yksin.
\par 21 Ja kun Rehabeam tuli Jerusalemiin, kokosi hän koko Juudan heimon ja Benjaminin sukukunnan, satakahdeksankymmentä tuhatta sotakuntoista valiomiestä, sotimaan Israelin heimoa vastaan ja palauttamaan kuninkuutta Rehabeamille Salomon pojalle.
\par 22 Mutta Jumalan miehelle Semajalle tuli tämä Jumalan sana:
\par 23 "Sano Rehabeamille, Salomon pojalle, Juudan kuninkaalle, ja koko Juudan ja Benjaminin heimoille sekä muulle kansalle näin:
\par 24 'Näin sanoo Herra: Älkää menkö sotimaan veljiänne, israelilaisia, vastaan. Palatkaa kukin kotiinne, sillä minä olen sallinut tämän tapahtua.'" Niin he kuulivat Herran sanaa, kääntyivät takaisin ja menivät pois Herran sanan mukaan.
\par 25 Mutta Jerobeam linnoitti Sikemin Efraimin vuoristossa ja asettui sinne. Sitten hän lähti sieltä ja linnoitti Penuelin.
\par 26 Ja Jerobeam ajatteli sydämessänsä: "Nyt valtakunta joutuu takaisin Daavidin suvulle.
\par 27 Jos tämä kansa menee ja uhraa teurasuhreja Herran temppelissä Jerusalemissa, niin tämän kansan sydän kääntyy jälleen heidän herransa Rehabeamin, Juudan kuninkaan, puolelle; ja he tappavat minut ja palaavat takaisin Rehabeamin, Juudan kuninkaan, luo."
\par 28 Mietittyään asiaa kuningas teetti kaksi kultaista vasikkaa ja sanoi heille: "Te olette jo tarpeeksi kauan kulkeneet Jerusalemissa. Katso, Israel, tässä on sinun Jumalasi, joka on johdattanut sinut Egyptin maasta."
\par 29 Ja hän pystytti toisen Beeteliin, ja toisen hän asetti Daaniin.
\par 30 Ja tämä koitui synniksi. Ja kansa kulki sen toisen kuvan luo Daaniin saakka.
\par 31 Hän rakensi myös uhrikukkulatemppeleitä ja teki kansan keskuudesta papeiksi kaikenkaltaisia miehiä, jotka eivät olleet leeviläisiä.
\par 32 Ja Jerobeam laittoi juhlan kahdeksannessa kuussa, kuukauden viidentenätoista päivänä, sen juhlan kaltaisen, jota vietetään Juudassa, ja nousi silloin itse alttarille; niin hän teki Beetelissä ja uhrasi niille vasikoille, jotka hän oli teettänyt. Ja tekemiänsä uhrikukkulapappeja hän asetti virkaan Beeteliin.
\par 33 Hän nousi sille alttarille, jonka oli teettänyt Beeteliin, viidentenätoista päivänä kahdeksatta kuuta, jonka kuukauden hän oli omasta päästään keksinyt. Hän laittoi silloin juhlan israelilaisille ja nousi alttarille polttamaan uhreja.

\chapter{13}

\par 1 Ja katso, Juudasta Beeteliin tuli Herran käskystä Jumalan mies, juuri kun Jerobeam seisoi alttarin ääressä polttamassa uhreja.
\par 2 Ja hän huusi alttaria kohti Herran käskystä ja sanoi: "Alttari, alttari, näin sanoo Herra: Katso, Daavidin suvusta on syntyvä poika, nimeltä Joosia. Hän on teurastava sinun päälläsi uhrikukkulapapit, jotka polttavat uhreja sinun päälläsi, ja sinun päälläsi tullaan polttamaan ihmisten luita."
\par 3 Ja hän antoi sinä päivänä ennusmerkin, sanoen: "Ennusmerkki siitä, että Herra on puhunut, on tämä: alttari halkeaa, ja tuhka, joka on sen päällä, hajoaa".
\par 4 Kun kuningas Jerobeam kuuli Jumalan miehen sanan, jonka hän huusi alttaria kohti Beetelissä, ojensi hän kätensä alttarilta ja sanoi: "Ottakaa hänet kiinni". Silloin hänen kätensä, jonka hän oli ojentanut häntä vastaan, kuivettui, eikä hän enää voinut vetää sitä takaisin.
\par 5 Ja alttari halkesi ja tuhka hajosi alttarilta, niinkuin Jumalan mies Herran käskystä oli ennusmerkin antanut.
\par 6 Silloin kuningas puhkesi puhumaan ja sanoi Jumalan miehelle: "Lepytä Herraa, Jumalaasi, ja rukoile minun puolestani, että minä voisin vetää käteni takaisin". Ja Jumalan mies lepytti Herraa; niin kuningas voi vetää kätensä takaisin, ja se tuli entiselleen.
\par 7 Ja kuningas puhui Jumalan miehelle: "Tule minun kanssani kotiini virkistämään itseäsi; minä annan sinulle lahjan".
\par 8 Mutta Jumalan mies sanoi kuninkaalle: "Vaikka antaisit minulle puolet linnastasi, en minä tulisi sinun kanssasi. Tässä paikassa en minä syö leipää enkä juo vettä.
\par 9 Sillä niin käski minua Herra sanallansa, sanoen: Älä syö leipää, älä juo vettä äläkä palaa samaa tietä, jota olet tullut."
\par 10 Ja hän meni toista tietä eikä palannut samaa tietä, jota oli tullut Beeteliin.
\par 11 Mutta Beetelissä asui vanha profeetta; ja hänen poikansa tuli ja kertoi hänelle kaiken, mitä Jumalan mies sinä päivänä oli tehnyt Beetelissä ja mitä hän oli puhunut kuninkaalle. Kun he olivat kertoneet sen isällensä,
\par 12 kysyi heidän isänsä heiltä, mitä tietä hän oli mennyt. Ja hänen poikansa olivat nähneet, mitä tietä Juudasta tullut Jumalan mies oli lähtenyt.
\par 13 Silloin hän sanoi pojilleen: "Satuloikaa minulle aasi". Ja kun he olivat satuloineet hänelle aasin, istui hän sen selkään
\par 14 ja lähti Jumalan miehen jälkeen ja tapasi hänet istumassa tammen alla. Ja hän kysyi häneltä: "Sinäkö olet se Juudasta tullut Jumalan mies?" Hän vastasi: "Minä".
\par 15 Hän sanoi hänelle: "Tule kanssani minun kotiini syömään".
\par 16 Hän vastasi: "En voi palata enkä tulla sinun kanssasi, en voi syödä leipää enkä juoda vettä sinun kanssasi tässä paikassa,
\par 17 sillä minulle on tullut sana, Herran sana: 'Älä syö leipää äläkä juo vettä siellä, älä myöskään tule takaisin samaa tietä, jota olet mennyt'".
\par 18 Hän sanoi hänelle: "Minäkin olen profeetta niinkuin sinä, ja enkeli on puhunut minulle Herran käskystä, sanoen: 'Vie hänet kanssasi takaisin kotiisi syömään leipää ja juomaan vettä'". Mutta sen hän valhetteli hänelle.
\par 19 Niin tämä palasi hänen kanssansa ja söi leipää ja joi vettä hänen kodissaan.
\par 20 Mutta heidän istuessaan pöydässä tuli Herran sana profeetalle, joka oli tuonut hänet takaisin.
\par 21 Ja hän huusi Juudasta tulleelle Jumalan miehelle, sanoen: "Näin sanoo Herra: Koska sinä olet niskoitellut Herran käskyä vastaan etkä ole noudattanut sitä määräystä, jonka Herra, sinun Jumalasi, sinulle antoi,
\par 22 vaan olet palannut syömään leipää ja juomaan vettä siinä paikassa, josta sinulle oli sanottu: 'Älä siellä syö leipää äläkä juo vettä', niin älköön sinun ruumiisi tulko isiesi hautaan".
\par 23 Ja kun hän oli syönyt leipää ja juonut, satuloi hän aasin profeetalle, jonka hän oli tuonut takaisin.
\par 24 Kun tämä oli lähtenyt, kohtasi hänet leijona tiellä ja tappoi hänet. Ja hänen ruumiinsa oli pitkänään tiellä, ja aasi seisoi hänen vieressään, ja leijona seisoi ruumiin ääressä.
\par 25 Ja katso, siitä kulki ohitse miehiä, ja he näkivät ruumiin pitkänään tiellä ja leijonan seisomassa ruumiin ääressä. He menivät ja puhuivat siitä kaupungissa, jossa vanha profeetta asui.
\par 26 Kun profeetta, joka oli palauttanut hänet tieltä, kuuli sen, sanoi hän: "Se on se Jumalan mies, joka niskoitteli Herran käskyä vastaan. Sentähden Herra on antanut hänet leijonalle, ja se on ruhjonut ja tappanut hänet sen Herran sanan mukaan, jonka hän oli hänelle puhunut."
\par 27 Ja hän puhui pojilleen, sanoen: "Satuloikaa minulle aasi". Ja he satuloivat.
\par 28 Ja hän lähti ja löysi hänen ruumiinsa, joka oli pitkänään tiellä, ja aasin ja leijonan seisomasta ruumiin ääressä; leijona ei ollut syönyt ruumista eikä myöskään ruhjonut aasia.
\par 29 Niin profeetta otti Jumalan miehen ruumiin, pani sen aasin selkään ja vei sen takaisin; ja vanha profeetta tuli kaupunkiin pitämään valittajaisia ja hautaamaan häntä.
\par 30 Ja hän pani ruumiin omaan hautaansa; ja he pitivät valittajaiset hänelle ja huusivat: "Voi, minun veljeni!"
\par 31 Ja kun hän oli haudannut hänet, sanoi hän pojilleen näin: "Kun minä kuolen, niin haudatkaa minut siihen hautaan, johon Jumalan mies on haudattu, ja pankaa minun luuni hänen luittensa viereen.
\par 32 Sillä toteutuva on sana, jonka hän Herran käskystä huusi Beetelissä olevaa alttaria vastaan ja kaikkia Samarian kaupungeissa olevia uhrikukkulatemppeleitä vastaan."
\par 33 Ei tämänkään jälkeen Jerobeam kääntynyt pahalta tieltään, vaan teki taas uhrikukkulapapeiksi kaikenkaltaisia miehiä kansan keskuudesta. Kuka vain halusi, sen hän vihki papin virkaan, ja niin siitä tuli uhrikukkulapappi.
\par 34 Ja tällä tavalla tämä tuli synniksi Jerobeamin suvulle ja syyksi siihen, että se hävitettiin ja hukutettiin maan päältä.

\chapter{14}

\par 1 Siihen aikaan sairastui Abia, Jerobeamin poika.
\par 2 Niin Jerobeam sanoi vaimollensa: "Nouse ja pukeudu niin, ettei sinua tunneta Jerobeamin vaimoksi, ja mene Siiloon. Katso, siellä on profeetta Ahia, joka ilmoitti, että minusta on tuleva tämän kansan kuningas.
\par 3 Ja ota mukaasi kymmenen leipää sekä pieniä leivoksia ja ruukullinen hunajaa ja mene hänen luoksensa. Hän ilmaisee sinulle, kuinka pojan käy."
\par 4 Jerobeamin vaimo teki niin: hän nousi ja meni Siiloon ja tuli Ahian taloon. Ja Ahia ei voinut nähdä, sillä hänellä oli vanhuuttaan kaihi silmissä.
\par 5 Mutta Herra oli sanonut Ahialle: "Katso, Jerobeamin vaimo tulee sinulta kysymään pojastaan, sillä hän on sairaana. Niin ja niin on sinun puhuttava hänelle." Ja kun hän tuli tekeytyen tuntemattomaksi,
\par 6 ja kun Ahia kuuli hänen askeleensa hänen tullessaan ovessa, sanoi hän: "Tule sisään, Jerobeamin vaimo; miksi sinä tekeydyt tuntemattomaksi? Minä olen saanut ilmoittaakseni sinulle kovan sanoman.
\par 7 Mene ja sano Jerobeamille: 'Näin sanoo Herra, Israelin Jumala: Minä olen korottanut sinut kansan seasta ja asettanut sinut kansani Israelin ruhtinaaksi
\par 8 ja reväissyt valtakunnan Daavidin suvulta ja antanut sen sinulle. Mutta sinä et ole ollut niinkuin minun palvelijani Daavid, joka noudatti minun käskyjäni ja seurasi minua kaikesta sydämestänsä, niin että hän teki ainoastaan sitä, mikä oli oikeata minun silmissäni.
\par 9 Vaan sinä olet tehnyt enemmän pahaa kuin kaikki sinun edeltäjäsi: sinä olet mennyt ja tehnyt itsellesi muita jumalia, valettuja kuvia, ja olet vihoittanut minut ja heittänyt minut selkäsi taakse.
\par 10 Katso, sentähden minä annan onnettomuuden kohdata Jerobeamin sukua ja hävitän Israelista Jerobeamin miespuoliset jälkeläiset, kaikki tyynni, ja minä lakaisen pois Jerobeamin suvun, niinkuin saasta lakaistaan, kunnes siitä on tullut loppu.
\par 11 Joka Jerobeamin jälkeläisistä kuolee kaupungissa, sen koirat syövät, ja joka kuolee kedolle, sen syövät taivaan linnut. Sillä Herra on puhunut.'
\par 12 Niin nouse nyt ja mene kotiisi. Kun sinun jalkasi astuu kaupunkiin, kuolee lapsi.
\par 13 Ja koko Israel on pitävä valittajaiset hänelle, ja hänet haudataan. Sillä Jerobeamin jälkeläisistä on hän yksin tuleva hautaan, koska Jerobeamin suvussa on vain hänessä havaittu jotakin Herralle, Israelin Jumalalle, otollista.
\par 14 Mutta Herra on herättävä itsellensä Israeliin kuninkaan, joka hävittää Jerobeamin suvun. Se on se päivä; ja mitä sitten?
\par 15 Herra on lyövä Israelia, niin että se tulee ruovon kaltaiseksi, joka häilyy vedessä. Ja hän kiskaisee Israelin irti tästä hyvästä maasta, jonka hän on antanut heidän isillensä, ja hajottaa heidät tuolle puolelle Eufrat-virran, koska he ovat tehneet itsellensä asera-karsikkoja ja siten vihoittaneet Herran.
\par 16 Ja hän antaa alttiiksi Israelin niiden syntien tähden, jotka Jerobeam on tehnyt ja joilla hän on saattanut Israelin tekemään syntiä."
\par 17 Niin Jerobeamin vaimo nousi, meni matkaansa ja tuli Tirsaan. Juuri kun hän tuli huoneen kynnykselle, kuoli poika.
\par 18 Ja hänet haudattiin, ja koko Israel piti valittajaiset hänelle, sen sanan mukaan, jonka Herra oli puhunut palvelijansa, profeetta Ahian, kautta.
\par 19 Mitä muuta on kerrottavaa Jerobeamista, kuinka hän soti ja kuinka hän hallitsi, katso, se on kirjoitettuna Israelin kuningasten aikakirjassa.
\par 20 Ja aika, minkä Jerobeam hallitsi, oli kaksikymmentä kaksi vuotta. Sitten hän meni lepoon isiensä tykö. Ja hänen poikansa Naadab tuli kuninkaaksi hänen sijaansa.
\par 21 Mutta Rehabeam, Salomon poika, tuli Juudan kuninkaaksi. Neljänkymmenen yhden vuoden vanha oli Rehabeam tullessaan kuninkaaksi, ja hän hallitsi seitsemäntoista vuotta Jerusalemissa, siinä kaupungissa, jonka Herra oli valinnut kaikista Israelin sukukunnista, asettaakseen nimensä siihen. Hänen äitinsä oli nimeltään Naema, ammonilainen.
\par 22 Ja Juuda teki sitä, mikä oli pahaa Herran silmissä. Synneillänsä, joita he tekivät, he vihoittivat Herraa paljon enemmän, kuin heidän isänsä olivat tehneet.
\par 23 Hekin tekivät itsellensä uhrikukkuloita, patsaita ja asera-karsikkoja kaikille korkeille kukkuloille ja jokaisen viheriän puun alle.
\par 24 Ja maassa oli myöskin haureellisia pyhäkköpoikia. He jäljittelivät niiden kansojen kaikkia kauhistavia tekoja, jotka Herra oli karkoittanut israelilaisten tieltä.
\par 25 Mutta kuningas Rehabeamin viidentenä hallitusvuotena hyökkäsi Suusak, Egyptin kuningas, Jerusalemin kimppuun.
\par 26 Ja hän otti Herran temppelin aarteet ja kuninkaan palatsin aarteet; otti kaikki tyynni. Hän otti myös kaikki ne kultakilvet, jotka Salomo oli teettänyt.
\par 27 Kuningas Rehabeam teetti niiden sijaan vaskikilvet ja jätti ne henkivartijain päälliköiden haltuun, jotka vartioivat kuninkaan linnan ovella.
\par 28 Ja niin usein kuin kuningas meni Herran temppeliin, kantoivat henkivartijat niitä ja veivät ne sitten takaisin henkivartijain huoneeseen.
\par 29 Mitä muuta on kerrottavaa Rehabeamista ja kaikesta, mitä hän teki, se on kirjoitettuna Juudan kuningasten aikakirjassa.
\par 30 Mutta Rehabeam ja Jerobeam olivat kaiken aikaa sodassa keskenään.
\par 31 Sitten Rehabeam meni lepoon isiensä tykö, ja hänet haudattiin isiensä viereen Daavidin kaupunkiin. Hänen äitinsä oli nimeltään Naema, ammonilainen. Ja hänen poikansa Abiam tuli kuninkaaksi hänen sijaansa.

\chapter{15}

\par 1 Ja kuningas Jerobeamin, Nebatin pojan, kahdeksantenatoista hallitusvuotena tuli Abiam Juudan kuninkaaksi.
\par 2 Hän hallitsi kolme vuotta Jerusalemissa. Hänen äitinsä oli nimeltään Maaka, Abisalomin tytär.
\par 3 Ja hän vaelsi kaikissa isänsä synneissä, joita tämä oli tehnyt ennen häntä, eikä hänen sydämensä ollut ehyesti antautunut Herralle, hänen Jumalallensa, niinkuin hänen isänsä Daavidin sydän oli ollut.
\par 4 Mutta Daavidin tähden Herra, hänen Jumalansa, antoi hänen lamppunsa palaa Jerusalemissa, korottamalla hänen jälkeläisekseen hänen poikansa ja pitämällä Jerusalemia pystyssä -
\par 5 koska Daavid oli tehnyt sitä, mikä on oikein Herran silmissä, eikä koko elinaikanaan ollut poikennut mistään, mitä hän on käskenyt, paitsi heettiläisen Uurian asiassa.
\par 6 Ja Abiam oli koko elinaikansa sodassa Jerobeamin kanssa.
\par 7 Mitä muuta on kerrottavaa Abiamista ja kaikesta, mitä hän teki, se on kirjoitettuna Juudan kuningasten aikakirjassa. Mutta Abiam ja Jerobeam olivat sodassa keskenään.
\par 8 Ja Abiam meni lepoon isiensä tykö, ja hänet haudattiin Daavidin kaupunkiin. Ja Aasa, hänen poikansa, tuli kuninkaaksi hänen sijaansa.
\par 9 Israelin kuninkaan Jerobeamin kahdentenakymmenentenä hallitusvuotena tuli Aasa Juudan kuninkaaksi.
\par 10 Hän hallitsi neljäkymmentä yksi vuotta Jerusalemissa. Hänen äitinsä oli nimeltään Maaka, Abisalomin tytär.
\par 11 Ja Aasa teki sitä, mikä oli oikein Herran silmissä, niinkuin hänen isänsä Daavid,
\par 12 ja toimitti haureelliset pyhäkköpojat pois maasta ja hävitti kaikki ne kivijumalat, jotka hänen isänsä olivat tehneet.
\par 13 Jopa hän erotti äitinsä Maakan kuningattaren arvosta, koska tämä oli pystyttänyt inhotuksen Aseralle; ja Aasa kukisti inhotuksen ja poltti sen Kidronin laaksossa.
\par 14 Mutta uhrikukkulat eivät hävinneet. Kuitenkin oli Aasan sydän ehyesti antautunut Herralle, niin kauan kuin hän eli.
\par 15 Ja hän vei Herran temppeliin isänsä pyhät lahjat ja omat pyhät lahjansa: hopeata, kultaa ja kaluja.
\par 16 Mutta Aasa ja Israelin kuningas Baesa olivat kaiken aikansa sodassa keskenään.
\par 17 Baesa, Israelin kuningas, lähti Juudaa vastaan ja linnoitti Raaman, estääkseen ketään pääsemästä Aasan, Juudan kuninkaan, luota tai hänen luokseen.
\par 18 Ja Aasa otti kaiken hopean ja kullan, mikä vielä oli jäljellä Herran temppelin ja kuninkaan linnan aarrekammioissa, ja antoi sen palvelijainsa haltuun; ja kuningas Aasa lähetti heidät Benhadadin, Tabrimmonin pojan, Hesjonin pojanpojan, Aramin kuninkaan, luo, joka asui Damaskossa, ja käski sanoa hänelle:
\par 19 "Onhan liitto meidän välillämme, minun ja sinun, niinkuin oli minun isäni ja sinun isäsi välillä. Katso, minä lähetän sinulle hopeata ja kultaa lahjaksi; mene ja riko Baesan, Israelin kuninkaan, kanssa tekemäsi liitto, että hän lähtisi pois minun kimpustani."
\par 20 Niin Benhadad kuuli kuningas Aasaa ja lähetti sotajoukkojensa päälliköt Israelin kaupunkeja vastaan ja valtasi Iijonin, Daanin, Aabel-Beet-Maakan ja koko Kinnerotin ynnä koko Naftalin maan.
\par 21 Kun Baesa kuuli sen, lakkasi hän linnoittamasta Raamaa ja jäi Tirsaan.
\par 22 Mutta kuningas Aasa kutsui kokoon kaiken Juudan, vapauttamatta ketään. Ja he veivät pois kivet ja puut, joilla Baesa oli linnoittanut Raamaa. Niillä kuningas Aasa linnoitti Geban, joka on Benjaminissa, ja Mispan.
\par 23 Mitä muuta kaikkea on kerrottavaa Aasasta, kaikista hänen urotöistään, kaikesta, mitä hän teki, ja kaupungeista, jotka hän rakensi, se on kirjoitettuna Juudan kuningasten aikakirjassa. Mutta vanhuutensa päivinä hän oli sairas jaloistaan.
\par 24 Sitten Aasa meni lepoon isiensä tykö, ja hänet haudattiin isiensä viereen hänen isänsä Daavidin kaupunkiin. Ja hänen poikansa Joosafat tuli kuninkaaksi hänen sijaansa.
\par 25 Ja Naadab, Jerobeamin poika, tuli Israelin kuninkaaksi Aasan, Juudan kuninkaan, toisena hallitusvuotena, ja hän hallitsi Israelia kaksi vuotta.
\par 26 Hän teki sitä, mikä on pahaa Herran silmissä, ja vaelsi isänsä teitä ja hänen synnissänsä, jolla hän oli saattanut Israelin tekemään syntiä.
\par 27 Mutta Baesa, Ahian poika, Isaskarin sukua, teki salaliiton häntä vastaan, ja Baesa surmasi hänet Gibbetonin luona, joka oli filistealaisilla; sillä Naadab ja koko Israel piirittivät Gibbetonia.
\par 28 Baesa tappoi hänet Juudan kuninkaan Aasan kolmantena hallitusvuotena ja tuli kuninkaaksi hänen sijaansa.
\par 29 Kuninkaaksi tultuaan hän surmasi koko Jerobeamin suvun jättämättä Jerobeamin jälkeläisistä eloon ainoatakaan henkeä: hän hävitti heidät Herran sanan mukaan, jonka hän oli puhunut palvelijansa, siilolaisen Ahian, kautta -
\par 30 niiden syntien tähden, jotka Jerobeam oli tehnyt ja joilla hän oli saattanut Israelin tekemään syntiä, siten vihoittaen Herran, Israelin Jumalan.
\par 31 Mitä muuta on kerrottavaa Naadabista ja kaikesta, mitä hän teki, se on kirjoitettuna Israelin kuningasten aikakirjassa.
\par 32 Mutta Aasa ja Israelin kuningas Baesa olivat kaiken aikansa sodassa keskenään.
\par 33 Aasan, Juudan kuninkaan, kolmantena hallitusvuotena tuli Baesa, Ahian poika, koko Israelin kuninkaaksi Tirsassa, ja hän hallitsi kaksikymmentä neljä vuotta.
\par 34 Hän teki sitä, mikä on pahaa Herran silmissä, ja vaelsi Jerobeamin teitä ja hänen synneissänsä, joilla hän oli saattanut Israelin tekemään syntiä.

\chapter{16}

\par 1 Ja Jeehulle, Hananin pojalle, tuli tämä Herran sana Baesaa vastaan:
\par 2 "Sentähden, että sinä, vaikka minä olen korottanut sinut tomusta ja pannut sinut kansani Israelin ruhtinaaksi, olet vaeltanut Jerobeamin teitä ja saattanut minun kansani Israelin tekemään syntiä, niin että he ovat vihoittaneet minut synneillänsä,
\par 3 katso, minä lakaisen pois Baesan ja hänen sukunsa. Minä teen sinun suvullesi, niinkuin minä tein Jerobeamin, Nebatin pojan, suvulle.
\par 4 Joka Baesan jälkeläisistä kuolee kaupungissa, sen koirat syövät, ja joka kuolee kedolle, sen syövät taivaan linnut."
\par 5 Mitä muuta on kerrottavaa Baesasta, siitä, mitä hän teki, ja hänen urotöistänsä, se on kirjoitettuna Israelin kuningasten aikakirjassa.
\par 6 Ja Baesa meni lepoon isiensä tykö, ja hänet haudattiin Tirsaan. Ja hänen poikansa Eela tuli kuninkaaksi hänen sijaansa.
\par 7 Mutta profeetta Jeehun, Hananin pojan, kautta oli tullut Herran sana Baesaa ja hänen sukuansa vastaan kaiken sen tähden, mitä hän oli tehnyt ja mikä oli pahaa Herran silmissä, hänen kättensä tekojen tähden, joilla hän oli vihoittanut Herran, niin että hänen kävi niinkuin Jerobeamin suvun, kuin myöskin sen tähden, että hän oli surmannut tämän suvun.
\par 8 Aasan, Juudan kuninkaan, kahdentenakymmenentenä kuudentena hallitusvuotena tuli Eela, Baesan poika, Israelin kuninkaaksi Tirsassa, ja hän hallitsi kaksi vuotta.
\par 9 Mutta hänen palvelijansa Simri, jolla oli johdossaan puolet sotavaunuista, teki salaliiton häntä vastaan. Ja kun hän Tirsassa kerran oli juonut itsensä juovuksiin Tirsassa olevan linnan päällikön Arsan talossa,
\par 10 tuli Simri sinne ja löi hänet kuoliaaksi Juudan kuninkaan Aasan kahdentenakymmenentenä seitsemäntenä hallitusvuotena; ja hän tuli kuninkaaksi hänen sijaansa.
\par 11 Ja kun hän oli tullut kuninkaaksi ja istunut valtaistuimelleen, surmasi hän koko Baesan suvun eikä jättänyt hänen jälkeläisistään eloon ainoatakaan miehenpuolta, ei sukulunastajaa eikä ystävää.
\par 12 Ja niin tuhosi Simri koko Baesan suvun, Herran sanan mukaan, jonka hän oli puhunut Baesaa vastaan profeetta Jeehun kautta,
\par 13 kaikkien Baesan ja hänen poikansa Eelan syntien tähden, jotka nämä olivat tehneet ja joilla he olivat saattaneet Israelin tekemään syntiä; he olivat vihoittaneet Herran, Israelin Jumalan, turhilla jumalilla, joita he palvelivat.
\par 14 Mitä muuta on kerrottavaa Eelasta ja kaikesta, mitä hän teki, se on kirjoitettuna Israelin kuningasten aikakirjassa.
\par 15 Juudan kuninkaan Aasan kahdentenakymmenentenä seitsemäntenä hallitusvuotena tuli Simri kuninkaaksi, ja hän hallitsi Tirsassa seitsemän päivää. Väki oli silloin asettuneena leiriin Gibbetonin edustalle, joka oli filistealaisilla.
\par 16 Kun leiriin asettunut väki kuuli sanottavan: "Simri on tehnyt salaliiton ja on myös surmannut kuninkaan", niin koko Israel teki sinä päivänä leirissä Omrin, Israelin sotapäällikön, kuninkaaksi.
\par 17 Ja Omri ja koko Israel hänen kanssaan lähti Gibbetonista, ja he saarsivat Tirsan.
\par 18 Ja kun Simri näki, että kaupunki oli valloitettu, meni hän kuninkaan linnan palatsiin ja poltti kuninkaan linnan päänsä päältä tulella, ja niin hän kuoli,
\par 19 syntiensä tähden, jotka hän oli tehnyt, kun oli tehnyt sitä, mikä on pahaa Herran silmissä, vaeltamalla Jerobeamin teitä ja hänen synnissään, jota tämä oli tehnyt ja jolla hän oli saattanut Israelin tekemään syntiä.
\par 20 Mitä muuta on kerrottavaa Simristä ja salaliitosta, jonka hän teki, se on kirjoitettuna Israelin kuningasten aikakirjassa.
\par 21 Silloin Israelin kansa jakaantui kahtia. Toinen puoli kansaa seurasi Tibniä, Giinatin poikaa, tehdäkseen hänet kuninkaaksi, toinen puoli seurasi Omria.
\par 22 Mutta se osa kansaa, joka seurasi Omria, sai voiton siitä osasta kansaa, joka seurasi Tibniä, Giinatin poikaa. Ja Tibni kuoli, ja Omri tuli kuninkaaksi.
\par 23 Juudan kuninkaan Aasan kolmantenakymmenentenä ensimmäisenä hallitusvuotena tuli Omri Israelin kuninkaaksi, ja hän hallitsi kaksitoista vuotta; Tirsassa hän hallitsi kuusi vuotta.
\par 24 Hän osti Samarian vuoren Semeriltä kahdella talentilla hopeata ja rakensi vuorelle kaupungin ja kutsui rakentamansa kaupungin Samariaksi Semerin, vuoren omistajan, nimen mukaan.
\par 25 Ja Omri teki sitä, mikä on pahaa Herran silmissä, teki enemmän pahaa kuin kaikki hänen edeltäjänsä.
\par 26 Ja hän vaelsi kaikessa Jerobeamin, Nebatin pojan, tietä ja hänen synneissänsä, joilla hän oli saattanut Israelin tekemään syntiä, niin että he vihoittivat Herran, Israelin Jumalan, turhilla jumalillaan.
\par 27 Mitä muuta on kerrottavaa Omrista, siitä, mitä hän teki, ja hänen tekemistään urotöistä, se on kirjoitettuna Israelin kuningasten aikakirjassa.
\par 28 Omri meni lepoon isiensä tykö, ja hänet haudattiin Samariaan. Ja hänen poikansa Ahab tuli kuninkaaksi hänen sijaansa.
\par 29 Ahab, Omrin poika, tuli Israelin kuninkaaksi Aasan, Juudan kuninkaan, kolmantenakymmenentenä kahdeksantena hallitusvuotena; sitten Ahab, Omrin poika, hallitsi Israelia Samariassa kaksikymmentä kaksi vuotta.
\par 30 Mutta Ahab, Omrin poika, teki sitä, mikä on pahaa Herran silmissä, enemmän kuin kaikki hänen edeltäjänsä.
\par 31 Ei ollut siinä kylliksi, että hän vaelsi Jerobeamin, Nebatin pojan, synneissä, vaan hän otti myös vaimokseen Iisebelin, siidonilaisten kuninkaan Etbaalin tyttären, ja rupesi palvelemaan Baalia ja kumartamaan sitä.
\par 32 Ja hän pystytti alttarin Baalille Baalin temppeliin, jonka hän oli rakentanut Samariaan.
\par 33 Ahab teki myös aseran. Ja Ahab teki vielä paljon muuta ja vihoitti Herraa, Israelin Jumalaa, enemmän kuin kukaan niistä Israelin kuninkaista, jotka olivat olleet ennen häntä.
\par 34 Hänen aikanansa beeteliläinen Hiiel rakensi uudelleen Jerikon. Sen perustuksen laskemisesta hän menetti esikoisensa Abiramin, ja sen ovien pystyttämisestä hän menetti nuorimpansa Segubin, Herran sanan mukaan, jonka hän oli puhunut Joosuan, Nuunin pojan, kautta.

\chapter{17}

\par 1 Tisbeläinen Elia, eräs Gileadiin asettuneita siirtolaisia, sanoi Ahabille: "Niin totta kuin Herra, Israelin Jumala, elää, jonka edessä minä seison: näinä vuosina ei tule kastetta eikä sadetta muutoin kuin minun sanani kautta".
\par 2 Ja hänelle tuli tämä Herran sana:
\par 3 "Mene pois täältä ja käänny itään päin ja kätkeydy Keritin purolle, joka on Jordanin itäpuolella.
\par 4 Sinä saat juoda purosta, ja minä olen käskenyt kaarneiden elättää sinua siellä."
\par 5 Niin hän meni ja teki Herran sanan mukaan: hän meni ja asettui Keritin purolle, joka on Jordanin itäpuolella.
\par 6 Ja kaarneet toivat hänelle leipää ja lihaa aamuin sekä leipää ja lihaa illoin, ja hän joi purosta.
\par 7 Mutta jonkun ajan kuluttua puro kuivui, koska siinä maassa ei ollut satanut.
\par 8 Ja hänelle tuli tämä Herran sana:
\par 9 "Nouse ja mene Sarpatiin, joka on Siidonin aluetta, ja asetu sinne. Katso, minä olen käskenyt leskivaimon elättää sinua siellä."
\par 10 Niin hän nousi ja meni Sarpatiin. Ja kun hän tuli kaupungin portille, niin katso, siellä oli leskivaimo keräilemässä puita. Hän huusi tälle ja sanoi: "Tuo minulle vähän vettä astiassa juodakseni".
\par 11 Kun hän meni hakemaan, huusi hän hänelle ja sanoi: "Tuo minulle myös palanen leipää kädessäsi".
\par 12 Mutta hän vastasi: "Niin totta kuin Herra, sinun Jumalasi, elää, minulla ei ole leipäkakkuakaan, vaan ainoastaan kourallinen jauhoja ruukussa ja vähän öljyä astiassa. Ja katso, kerättyäni pari puuta minä menen leipomaan itselleni ja pojalleni, syödäksemme ja sitten kuollaksemme."
\par 13 Niin Elia sanoi hänelle: "Älä pelkää; mene ja tee, niinkuin olet sanonut. Mutta leivo minulle ensin pieni kaltiainen ja tuo se minulle. Leivo sitten itsellesi ja pojallesi.
\par 14 Sillä näin sanoo Herra, Israelin Jumala: Jauhot eivät lopu ruukusta, eikä öljyä ole puuttuva astiasta siihen päivään asti, jona Herra antaa sateen maan päälle."
\par 15 Niin hän meni ja teki, niinkuin Elia oli sanonut. Ja hänellä sekä myös Elialla ja vaimon perheellä oli syötävää pitkäksi aikaa.
\par 16 Jauhot eivät loppuneet ruukusta, eikä öljyä puuttunut astiasta, sen Herran sanan mukaan, jonka hän oli Elian kautta puhunut.
\par 17 Sen jälkeen vaimon, talon emännän, poika sairastui; ja hänen tautinsa kävi hyvin kovaksi, niin ettei hänessä enää ollut henkeä.
\par 18 Silloin vaimo sanoi Elialle: "Mitä minulla on tekemistä sinun kanssasi, Jumalan mies? Sinä olet tullut minun luokseni saattamaan minun pahat tekoni muistoon ja tuottamaan kuoleman minun pojalleni."
\par 19 Mutta hän sanoi hänelle: "Anna poikasi minulle". Ja hän otti tämän hänen sylistään ja vei hänet yliskammioon, jossa asui, ja pani hänet vuoteellensa.
\par 20 Ja hän huusi Herraa ja sanoi: "Herra, minun Jumalani, oletko tehnyt niin pahoin tätä leskeä kohtaan, jonka vieraana minä olen, että olet surmannut hänen poikansa?"
\par 21 Sitten hän ojentautui pojan yli kolme kertaa, huusi Herraa ja sanoi: "Herra, minun Jumalani, anna tämän pojan sielun tulla häneen takaisin".
\par 22 Ja Herra kuuli Eliaa, ja pojan sielu tuli häneen takaisin, ja hän virkosi henkiin.
\par 23 Ja Elia otti pojan ja toi hänet yliskammiosta alas huoneeseen ja antoi hänet hänen äidillensä. Ja Elia sanoi: "Katso, poikasi elää".
\par 24 Niin vaimo sanoi Elialle: "Nyt minä tiedän, että sinä olet Jumalan mies ja että Herran sana sinun suussasi on tosi".

\chapter{18}

\par 1 Pitkän ajan kuluttua, kolmantena vuotena, tuli Elialle tämä Herran sana: "Mene ja näyttäydy Ahabille, niin minä annan sateen maan päälle".
\par 2 Niin Elia meni näyttäytymään Ahabille. Mutta Samariassa oli kova nälänhätä.
\par 3 Ja Ahab kutsui Obadjan, joka oli palatsin päällikkönä. Mutta Obadja oli hyvin Herraa pelkääväinen mies;
\par 4 niinpä Obadja oli silloin, kun Iisebel hävitti Herran profeetat, ottanut sata profeettaa ja piilottanut heidät luolaan, viisikymmentä kerrallaan, ja elättänyt heitä leivällä ja vedellä.
\par 5 Ahab sanoi Obadjalle: "Kulje maa, kaikki vesilähteet ja kaikki purot. Kenties me löydämme ruohoa pitääksemme hevoset ja muulit hengissä, niin ettei meidän tarvitse hävittää elukoita."
\par 6 Ja he jakoivat keskenään kuljettavansa maan. Ahab kulki toista tietä yksinänsä, ja Obadja kulki toista tietä yksinänsä.
\par 7 Kun nyt Obadja oli matkalla, niin katso, Elia tuli häntä vastaan. Tuntiessaan tämän hän heittäytyi kasvoillensa ja sanoi: "Sinäkö se olet, herrani Elia?"
\par 8 Tämä vastasi hänelle: "Minä. Mene ja sano herrallesi: 'Katso, Elia on täällä'."
\par 9 Niin hän sanoi: "Mitä minä olen rikkonut, koska annat palvelijasi Ahabin käsiin, hänen surmattavakseen?
\par 10 Niin totta kuin Herra, sinun Jumalasi, elää, ei ole sitä kansaa eikä sitä valtakuntaa, josta minun herrani ei olisi lähettänyt etsimään sinua, mutta kun on vastattu: 'Ei hän ole täällä', on hän vannottanut sillä valtakunnalla ja sillä kansalla valan, ettei sinua ole löydetty.
\par 11 Ja nyt sinä sanot: 'Mene ja sano herrallesi: Katso, Elia on täällä'.
\par 12 Kun minä lähden pois luotasi, niin Herran Henki kuljettaa sinut, en tiedä minne, ja kun minä menen ilmoittamaan Ahabille eikä hän löydä sinua, niin hän tappaa minut. Ja kuitenkin palvelijasi on peljännyt Herraa nuoruudestaan asti.
\par 13 Eikö herralleni ole kerrottu, mitä minä tein, kun Iisebel tappoi Herran profeetat: kuinka minä piilotin luolaan sata Herran profeettaa, viisikymmentä kerrallaan, ja elätin heitä leivällä ja vedellä?
\par 14 Ja nyt sinä sanot: 'Mene ja sano herrallesi: Katso, Elia on täällä', ja niin hän tappaa minut."
\par 15 Mutta Elia sanoi: "Niin totta kuin Herra Sebaot elää, jonka edessä minä seison, minä näyttäydyn tänä päivänä hänelle".
\par 16 Niin Obadja meni Ahabia vastaan ja ilmoitti hänelle tämän. Ahab meni silloin Eliaa vastaan.
\par 17 Ja nähdessään Elian Ahab sanoi hänelle: "Siinäkö sinä olet, sinä, joka syökset Israelin onnettomuuteen?"
\par 18 Tämä vastasi: "En minä syökse Israelia onnettomuuteen, vaan sinä ja sinun isäsi suku, koska te hylkäätte Herran käskyt ja koska sinä seuraat baaleja.
\par 19 Mutta lähetä nyt kokoamaan kaikki Israel minun luokseni Karmel-vuorelle, sekä neljäsataa viisikymmentä Baalin profeettaa ja neljäsataa Aseran profeettaa, jotka syövät Iisebelin pöydästä."
\par 20 Niin Ahab lähetti sanan kaikille israelilaisille ja kokosi profeetat Karmel-vuorelle.
\par 21 Ja Elia astui kaiken kansan eteen ja sanoi: "Kuinka kauan te onnutte molemmille puolille? Jos Herra on Jumala, seuratkaa häntä; mutta jos Baal on Jumala, seuratkaa häntä." Eikä kansa vastannut hänelle mitään.
\par 22 Niin Elia sanoi kansalle: "Minä olen ainoa jäljelle jäänyt Herran profeetta, mutta Baalin profeettoja on neljäsataa viisikymmentä.
\par 23 Antakaa meille kaksi mullikkaa, ja valitkoot he itselleen toisen mullikan, paloitelkoot sen ja pankoot kappaleet puiden päälle, mutta älkööt panko tulta; ja minä valmistan toisen mullikan ja asetan sen puiden päälle, mutta en pane tulta.
\par 24 Sitten huutakaa te jumalanne nimeä, ja minä huudan Herran nimeä. Se jumala, joka vastaa tulella, on Jumala." Kaikki kansa vastasi ja sanoi: "Niin on hyvä".
\par 25 Ja Elia sanoi Baalin profeetoille: "Valitkaa itsellenne toinen mullikka ja valmistakaa se ensin, sillä teitä on enemmän. Huutakaa sitten jumalanne nimeä, mutta älkää panko tulta".
\par 26 Niin he ottivat sen mullikan, jonka hän antoi heille, ja valmistivat sen. Sitten he huusivat Baalin nimeä aamusta puolipäivään asti, sanoen: "Baal, vastaa meille!" Mutta ei ääntä, ei vastausta! Ja he hyppelivät alttarin ääressä, joka oli tehty.
\par 27 Puolipäivän aikana Elia pilkkasi heitä ja sanoi: "Huutakaa kovemmin; hän on tosin jumala, mutta hänellä voi olla jotakin toimittamista, tahi hän on poistunut johonkin, tahi on matkalla; kenties hän nukkuu, mutta kyllä hän herää".
\par 28 Niin he huusivat vielä kovemmin ja viileksivät itseään tapansa mukaan miekoilla ja keihäillä, niin että heistä vuoti verta.
\par 29 Kun puolipäivä oli kulunut, joutuivat he hurmoksiin, aina siihen hetkeen asti, jolloin ruokauhri uhrataan. Mutta ei ääntä, ei vastausta, ei vaarinottoa!
\par 30 Niin Elia sanoi kaikelle kansalle: "Astukaa minun luokseni". Ja kaikki kansa astui hänen luoksensa. Niin hän korjasi Herran alttarin, joka oli hajotettu.
\par 31 Ja Elia otti kaksitoista kiveä, yhtä monta kuin oli Jaakobin poikien sukukuntia, hänen, jolle oli tullut tämä Herran sana: "Israel on oleva sinun nimesi".
\par 32 Ja hän rakensi kivistä alttarin Herran nimeen ja teki alttarin ympärille ojan, johon olisi mahtunut kaksi sea-mittaa jyviä.
\par 33 Sitten hän latoi puut, paloitteli mullikan ja pani kappaleet puiden päälle.
\par 34 Ja hän sanoi: "Täyttäkää neljä ruukkua vedellä ja vuodattakaa se polttouhrin ja puiden päälle". Ja hän sanoi: "Tehkää se toinen kerta". Ja he tekivät niin toisen kerran. Vielä hän sanoi: "Tehkää se kolmas kerta". Ja he tekivät niin kolmannen kerran.
\par 35 Niin vesi juoksi ympäri alttarin; ojankin hän täytti vedellä.
\par 36 Ja kun oli tullut hetki, jolloin ruokauhri uhrataan, astui profeetta Elia esille ja sanoi: "Herra, Aabrahamin, Iisakin ja Israelin Jumala, tulkoon tänä päivänä tiettäväksi, että sinä olet Jumala Israelissa ja että minä olen sinun palvelijasi ja että minä olen sinun käskystäsi tehnyt kaiken tämän.
\par 37 Vastaa minulle, Herra, vastaa minulle, että tämä kansa tulisi näkemään, että sinä, Herra, olet Jumala ja että sinä käännät heidän sydämensä takaisin."
\par 38 Silloin Herran tuli iski alas ja kulutti polttouhrin, puut, kivet ja mullan sekä nuoli veden, joka oli ojassa.
\par 39 Kun kaikki kansa näki tämän, lankesivat he kasvoillensa ja sanoivat: "Herra on Jumala! Herra on Jumala!"
\par 40 Mutta Elia sanoi heille: "Ottakaa Baalin profeetat kiinni; älköön yksikään heistä pääskö pakoon". Ja he ottivat heidät kiinni. Ja Elia vei heidät Kiisonin purolle ja tappoi heidät siellä.
\par 41 Ja Elia sanoi Ahabille: "Nouse, syö ja juo, sillä sateen kohina kuuluu".
\par 42 Niin Ahab nousi syömään ja juomaan. Mutta Elia nousi Karmelin huipulle, kumartui maahan ja painoi kasvonsa polviensa väliin.
\par 43 Ja hän sanoi palvelijallensa: "Nouse ja katso merelle päin". Tämä nousi ja katsoi, mutta sanoi: "Ei näy mitään". Hän sanoi: "Mene takaisin". Näin seitsemän kertaa.
\par 44 Seitsemännellä kerralla palvelija sanoi: "Katso, pieni pilvi, miehen kämmenen kokoinen, nousee merestä". Niin Elia sanoi: "Nouse ja sano Ahabille: 'Valjasta ja lähde alas, ettei sade sinua pidättäisi'".
\par 45 Ja tuossa tuokiossa taivas kävi mustaksi pilvistä ja myrskytuulesta, ja tuli ankara sade. Mutta Ahab nousi vaunuihinsa ja lähti Jisreeliin.
\par 46 Ja Herran käsi tuli Elian päälle, ja niin hän vyötti kupeensa ja juoksi Ahabin edellä Jisreeliin saakka.

\chapter{19}

\par 1 Mutta kun Ahab kertoi Iisebelille kaiken, mitä Elia oli tehnyt ja kuinka hän oli tappanut miekalla kaikki profeetat,
\par 2 lähetti Iisebel sanansaattajan Elian luo ja käski sanoa: "Jumalat rangaiskoot minua nyt ja vasta, jollen minä huomenna tähän aikaan tee sinulle samaa, mikä jokaiselle näistä on tehty".
\par 3 Kun tämä näki sen, nousi hän ja lähti matkaan pelastaakseen henkensä ja tuli Beersebaan, joka on Juudan aluetta.
\par 4 Sinne hän jätti palvelijansa, mutta meni itse erämaahan päivänmatkan päähän. Hän tuli ja istuutui kinsteripensaan juureen. Ja hän toivotti itsellensä kuolemaa ja sanoi: "Jo riittää, Herra; ota minun henkeni, sillä minä en ole isiäni parempi".
\par 5 Ja hän paneutui maata ja nukkui kinsteripensaan juurelle. Mutta katso, enkeli kosketti häntä ja sanoi hänelle: "Nouse ja syö".
\par 6 Ja kun hän katsahti, niin katso, hänen päänsä pohjissa oli kuumennetuilla kivillä paistettu kaltiainen ja vesiastia. Niin hän söi ja joi ja paneutui jälleen maata.
\par 7 Mutta Herran enkeli kosketti häntä vielä toisen kerran ja sanoi: "Nouse ja syö, sillä muutoin käy matka sinulle liian pitkäksi".
\par 8 Niin hän nousi ja söi ja joi. Ja hän kulki sen ruuan voimalla neljäkymmentä päivää ja neljäkymmentä yötä Jumalan vuorelle, Hoorebille, asti.
\par 9 Siellä hän meni luolaan ja oli siinä yötä. Ja katso, Herran sana tuli hänelle; hän kysyi häneltä: "Mitä sinä täällä teet, Elia?"
\par 10 Hän vastasi: "Minä olen kiivailemalla kiivaillut Herran, Jumalan Sebaotin, puolesta. Sillä israelilaiset ovat hyljänneet sinun liittosi, hajottaneet sinun alttarisi ja tappaneet miekalla sinun profeettasi. Minä yksin olen jäänyt jäljelle, mutta minunkin henkeäni he väijyvät, ottaaksensa sen."
\par 11 Hän sanoi: "Mene ulos ja asetu vuorelle Herran eteen". Ja katso, Herra kulki ohitse, ja suuri ja raju myrsky, joka halkoi vuoret ja särki kalliot, kävi Herran edellä; mutta ei Herra ollut myrskyssä. Myrskyn jälkeen tuli maanjäristys; mutta ei Herra ollut maanjäristyksessä.
\par 12 Maanjäristyksen jälkeen tuli tulta; mutta ei Herra ollut tulessa. Tulen jälkeen tuli hiljainen tuulen hyminä.
\par 13 Kun Elia sen kuuli, peitti hän kasvonsa vaipallansa, meni ulos ja asettui luolan suulle. Ja katso, hänelle puhui ääni ja sanoi: "Mitä sinä täällä teet, Elia?"
\par 14 Hän vastasi: "Minä olen kiivailemalla kiivaillut Herran, Jumalan Sebaotin, puolesta. Sillä israelilaiset ovat hyljänneet sinun liittosi, hajottaneet sinun alttarisi ja tappaneet miekalla sinun profeettasi. Minä yksin olen jäänyt jäljelle, mutta minunkin henkeäni he väijyvät, ottaaksensa sen."
\par 15 Herra sanoi hänelle: "Lähde takaisin samaa tietä, jota tulit, erämaan kautta Damaskoon. Mene ja voitele Hasael Aramin kuninkaaksi.
\par 16 Ja voitele Jeehu, Nimsin poika, Israelin kuninkaaksi. Ja voitele sijaasi profeetaksi Elisa, Saafatin poika, Aabel-Meholasta.
\par 17 Ja on tapahtuva näin: joka välttää Hasaelin miekan, sen surmaa Jeehu, ja joka välttää Jeehun miekan, sen surmaa Elisa.
\par 18 Mutta minä jätän jäljelle Israeliin seitsemäntuhatta: kaikki polvet, jotka eivät ole notkistuneet Baalille, ja kaikki suut, jotka eivät ole hänelle suuta antaneet."
\par 19 Niin hän lähti sieltä ja kohtasi Elisan, Saafatin pojan, kun tämä oli kyntämässä; kaksitoista härkäparia kulki hänen edellänsä, ja itse hän ajoi kahdettatoista. Kulkiessaan hänen ohitsensa Elia heitti vaippansa hänen päällensä.
\par 20 Niin hän jätti härät, riensi Elian jälkeen ja sanoi: "Salli minun ensin antaa suuta isälleni ja äidilleni; sitten minä seuraan sinua". Elia sanoi hänelle: "Mene, mutta tule takaisin; tiedäthän, mitä minä olen sinulle tehnyt".
\par 21 Niin hän meni hänen luotaan takaisin, otti härkäparinsa ja teurasti sen, ja härkäin ikeellä hän keitti lihat; ne hän antoi väelle, ja he söivät. Sitten hän nousi ja seurasi Eliaa ja palveli häntä.

\chapter{20}

\par 1 Ja Benhadad, Aramin kuningas, kokosi kaiken sotajoukkonsa. Hänellä oli mukanaan kolmekymmentä kaksi kuningasta sekä hevosia ja sotavaunuja. Ja hän meni ja piiritti Samariaa ja ryhtyi taisteluun sitä vastaan.
\par 2 Ja hän lähetti sanansaattajat kaupunkiin Ahabin, Israelin kuninkaan, luo
\par 3 ja käski sanoa hänelle: "Näin sanoo Benhadad: 'Sinun hopeasi ja kultasi ovat minun, ja sinun kauneimmat vaimosi ja lapsesi ovat myöskin minun'".
\par 4 Israelin kuningas vastasi ja sanoi: "Niinkuin sinä olet sanonut, herrani, kuningas, minä olen sinun, minä ja kaikki, mitä minulla on".
\par 5 Mutta sanansaattajat tulivat takaisin ja sanoivat: "Näin sanoo Benhadad: 'Minä olen lähettänyt sinulle tämän sanan: Anna minulle hopeasi ja kultasi, vaimosi ja lapsesi.
\par 6 Totisesti minä lähetän huomenna tähän aikaan palvelijani sinun luoksesi tutkimaan sinun taloasi ja sinun palvelijaisi taloja; ja kaiken, mikä on sinun silmiesi ihastus, he ottavat haltuunsa ja vievät.'"
\par 7 Silloin Israelin kuningas kutsui kaikki maan vanhimmat ja sanoi: "Ymmärtäkää ja nähkää, että hän hankkii meille onnettomuutta; sillä kun hän lähetti vaatimaan minulta vaimojani ja lapsiani, hopeatani ja kultaani, en minä niitä häneltä kieltänyt".
\par 8 Kaikki vanhimmat ja kaikki kansa sanoivat hänelle: "Älä kuule häntä äläkä suostu".
\par 9 Niin hän sanoi Benhadadin sanansaattajille: "Sanokaa herralleni, kuninkaalle: 'Kaiken, mitä sinä ensi kerralla käskit palvelijasi tehdä, minä teen; mutta tätä minä en voi tehdä'". Ja sanansaattajat menivät ja veivät tämän vastauksen hänelle.
\par 10 Niin Benhadad lähetti hänelle sanan: "Jumalat rangaiskoot minua nyt ja vasta, jos Samarian tomu riittää täyttämään kaiken sen väen kourat, joka minua seuraa".
\par 11 Mutta Israelin kuningas vastasi ja sanoi: "Sanokaa hänelle: 'Älköön se, joka vyöttäytyy miekkaan, kerskatko niinkuin se, joka sen riisuu'".
\par 12 Kun hän, juodessaan kuninkaitten kanssa lehtimajoissa, kuuli tämän vastauksen, sanoi hän palvelijoilleen: "Piiritys käyntiin!" Ja he panivat piirityksen käyntiin kaupunkia vastaan.
\par 13 Ja katso, eräs profeetta astui Ahabin, Israelin kuninkaan, eteen ja sanoi: "Näin sanoo Herra: Näetkö kaiken tuon suuren lauman? Katso, minä annan sen tänä päivänä sinun käsiisi, että sinä tulisit tietämään, että minä olen Herra."
\par 14 Ahab kysyi: "Kenen avulla?" Hän vastasi: "Näin sanoo Herra: Maaherrain palvelijoitten avulla". Ahab kysyi vielä: "Kuka alottaa taistelun?" Hän vastasi: "Sinä".
\par 15 Niin hän katsasti herrain palvelijat, ja niitä oli kaksisataa kolmekymmentä kaksi. Niitten jälkeen hän katsasti kaiken väen, kaikki israelilaiset, joita oli seitsemäntuhatta.
\par 16 Ja he tekivät hyökkäyksen puolenpäivän aikana, kun Benhadad oli juovuksissa ja joi lehtimajoissa, hän ja ne kolmekymmentä kaksi kuningasta, jotka olivat tulleet hänen avuksensa.
\par 17 Maaherrain palvelijat hyökkäsivät ensimmäisinä. Silloin Benhadad lähetti tiedustelemaan, ja hänelle ilmoitettiin: "Miehiä on lähtenyt liikkeelle Samariasta".
\par 18 Hän sanoi: "Jos he ovat lähteneet liikkeelle rauha mielessä, niin ottakaa heidät kiinni elävinä; ja jos he ovat lähteneet taistellakseen, ottakaa heidät silloinkin kiinni elävinä".
\par 19 Mutta kun maaherrain palvelijat ja sotajoukko, joka seurasi heitä, olivat hyökänneet kaupungista,
\par 20 surmasivat he miehen kukin. Niin aramilaiset pakenivat, ja Israel ajoi heitä takaa. Mutta Benhadad, Aramin kuningas, pääsi muutamien ratsumiesten kanssa pakoon hevosen selässä.
\par 21 Mutta Israelin kuningas lähti liikkeelle ja valtasi hevoset ja sotavaunut; ja niin hän tuotti aramilaisille suuren tappion.
\par 22 Ja profeetta astui Israelin kuninkaan eteen ja sanoi hänelle: "Vahvista itsesi, ymmärrä ja katso, mitä sinun on tehtävä, sillä vuoden vaihteessa hyökkää Aramin kuningas sinun kimppuusi".
\par 23 Aramin kuninkaan palvelijat sanoivat tälle: "Heidän jumalansa on vuorijumala; sentähden he saivat meistä voiton. Mutta taistelkaamme heitä vastaan tasangolla, silloin me varmasti saamme heistä voiton.
\par 24 Ja tee näin: Pane pois kuninkaat, kukin paikaltaan, ja aseta käskynhaltijat heidän sijaansa.
\par 25 Hanki sitten itsellesi sotajoukko, yhtä suuri kuin menettämäsi oli, yhtä monta hevosta ja yhtä monet sotavaunut; ja taistelkaamme heitä vastaan tasangolla, niin me varmasti saamme heistä voiton." Niin hän kuuli heitä ja teki niin.
\par 26 Vuoden vaihteessa Benhadad katsasti aramilaiset ja lähti Afekiin taistelemaan Israelia vastaan.
\par 27 Ja israelilaiset katsastettiin ja muonitettiin, ja he lähtivät heitä vastaan. Israelilaiset leiriytyivät heidän eteensä niinkuin kaksi pientä vuohilaumaa, mutta aramilaiset täyttivät maan.
\par 28 Silloin Jumalan mies astui Israelin kuninkaan eteen ja sanoi: "Näin sanoo Herra: Sentähden, että aramilaiset ovat sanoneet: 'Herra on vuorijumala eikä laaksojumala', annan minä koko tämän suuren lauman sinun käsiisi, tullaksenne tietämään, että minä olen Herra."
\par 29 Ja he olivat leiriytyneinä vastakkain seitsemän päivää. Seitsemäntenä päivänä sukeutui taistelu, ja israelilaiset surmasivat aramilaisia satatuhatta jalkamiestä yhtenä päivänä.
\par 30 Jäljellejääneet pakenivat Afekiin, kaupunkiin. Ja muuri kaatui kahdenkymmenen seitsemän tuhannen jäljellejääneen miehen päälle. Mutta Benhadad pakeni, tuli kaupunkiin ja juoksi huoneesta huoneeseen.
\par 31 Silloin hänen palvelijansa sanoivat hänelle: "Katso, me olemme kuulleet, että Israelin heimon kuninkaat ovat laupiaita kuninkaita. Pankaamme säkit lanteillemme ja nuorat päämme ympärille ja antautukaamme Israelin kuninkaalle; ehkä hän jättää sinut henkiin."
\par 32 Ja he käärivät säkit lanteilleen ja panivat nuorat päänsä ympärille ja tulivat Israelin kuninkaan luo ja sanoivat: "Palvelijasi Benhadad sanoo: 'Salli minun elää'." Hän sanoi: "Vieläkö hän elää? Hän on minun veljeni."
\par 33 Niin miehet katsoivat sen hyväksi enteeksi, tarttuivat kiiruusti hänen sanaansa ja sanoivat: "Benhadad on sinun veljesi". Hän sanoi: "Menkää ja noutakaa hänet". Silloin Benhadad antautui hänelle, ja hän antoi hänen nousta vaunuihinsa.
\par 34 Ja Benhadad sanoi hänelle: "Kaupungit, jotka minun isäni otti sinun isältäsi, minä annan takaisin. Laita itsellesi katumyymälöitä Damaskoon, niinkuin minun isäni laittoi Samariaan." Ahab vastasi: "Minä päästän sinut menemään sillä välipuheella". Ja hän teki liiton hänen kanssaan ja päästi hänet menemään.
\par 35 Silloin eräs profeetanoppilaista sanoi Herran käskystä toverilleen: "Lyö minua". Mutta mies ei tahtonut lyödä häntä.
\par 36 Niin hän sanoi tälle: "Koska et kuullut Herran ääntä, niin katso, leijona surmaa sinut, kun lähdet minun luotani". Ja kun hän lähti hänen luotansa, kohtasi leijona hänet ja surmasi hänet.
\par 37 Sitten hän tapasi toisen miehen ja sanoi: "Lyö minua". Ja mies löi hänet haavoille.
\par 38 Niin profeetta meni ja asettui tielle, jota kuninkaan oli kuljettava, ja teki itsensä tuntemattomaksi panemalla siteen silmilleen.
\par 39 Kun kuningas kulki ohitse, huusi hän kuninkaalle ja sanoi: "Palvelijasi oli lähtenyt keskelle taistelua; ja katso, sieltä tuli mies ja toi toisen miehen minun luokseni ja sanoi: 'Vartioitse tätä miestä. Jos hän katoaa, menee sinun henkesi hänen hengestään, tahi sinä maksat talentin hopeata.'
\par 40 Palvelijallasi oli tehtävää siellä ja täällä, ja sitten ei miestä enää ollut." Israelin kuningas sanoi hänelle: "Tuomiosi on siis se; sinä olet itse julistanut sen".
\par 41 Silloin hän nopeasti poisti siteen silmiltään, ja Israelin kuningas tunsi hänet, että hän oli profeettoja.
\par 42 Ja tämä sanoi hänelle: "Näin sanoo Herra: Koska sinä päästit käsistäsi menemään minulle tuhon omaksi vihityn miehen, menee sinun henkesi hänen hengestänsä ja sinun kansasi hänen kansastaan."
\par 43 Niin Israelin kuningas meni kotiinsa pahastuneena ja alakuloisena ja tuli Samariaan.

\chapter{21}

\par 1 Näiden tapausten jälkeen tapahtui tämä. Jisreeliläisellä Naabotilla oli viinitarha, joka oli Jisreelissä Samarian kuninkaan Ahabin palatsin vieressä.
\par 2 Ja Ahab puhui Naabotille sanoen: "Anna minulle viinitarhasi vihannestarhaksi, koska se on niin likellä minun linnaani; minä annan sinulle sen sijaan paremman viinitarhan, tahi, jos niin tahdot, minä annan sinulle sen hinnan rahana".
\par 3 Mutta Naabot vastasi Ahabille: "Pois se, Herra varjelkoon minua antamasta sinulle isieni perintöosaa".
\par 4 Niin Ahab tuli kotiinsa pahastuneena ja alakuloisena vastauksesta, jonka jisreeliläinen Naabot oli hänelle antanut, kun tämä oli sanonut: "En minä anna sinulle isieni perintöosaa". Ja hän pani maata vuoteellensa, käänsi kasvonsa pois eikä syönyt mitään.
\par 5 Niin hänen vaimonsa Iisebel tuli hänen luokseen ja puhui hänelle: "Miksi olet niin pahoilla mielin ja miksi et syö mitään?"
\par 6 Ahab vastasi hänelle: "Siksi, että puhuttelin jisreeliläistä Naabotia ja sanoin hänelle: 'Anna minulle viinitarhasi rahasta, tahi jos haluat, annan minä sinulle toisen viinitarhan sen sijaan', mutta hän vastasi: 'En minä anna sinulle viinitarhaani'".
\par 7 Niin hänen vaimonsa Iisebel sanoi hänelle: "Sinäkö olet käyttävinäsi kuninkaanvaltaa Israelissa? Nouse ja syö, ja olkoon sydämesi iloinen; minä kyllä toimitan sinulle jisreeliläisen Naabotin viinitarhan."
\par 8 Ja hän kirjoitti kirjeen Ahabin nimessä, sinetöi sen hänen sinetillään ja lähetti kirjeen vanhimmille ja ylimyksille, jotka olivat Naabotin kaupungissa ja asuivat siellä hänen kanssaan.
\par 9 Ja kirjeessä hän kirjoitti näin: "Kuuluttakaa paasto ja asettakaa Naabot istumaan etumaiseksi kansan joukkoon.
\par 10 Ja asettakaa kaksi kelvotonta miestä istumaan häntä vastapäätä, että he todistaisivat hänestä näin: 'Sinä olet kironnut Jumalaa ja kuningasta'. Viekää hänet sitten ulos ja kivittäkää hänet kuoliaaksi."
\par 11 Ja kaupungin miehet, vanhimmat ja ylimykset, jotka asuivat hänen kaupungissansa, tekivät, niinkuin Iisebel oli käskenyt heitä ja niinkuin oli kirjoitettu kirjeessä, jonka hän oli heille lähettänyt.
\par 12 He kuuluttivat paaston ja asettivat Naabotin istumaan etumaiseksi kansan joukkoon.
\par 13 Ja kaksi kelvotonta miestä tuli ja istui häntä vastapäätä. Ja ne kelvottomat miehet todistivat Naabotista kansan edessä sanoen: "Naabot on kironnut Jumalaa ja kuningasta". Silloin he veivät hänet kaupungin ulkopuolelle ja kivittivät hänet kuoliaaksi.
\par 14 Sitten he lähettivät Iisebelille tämän sanan: "Naabot on kivitetty kuoliaaksi".
\par 15 Kun Iisebel kuuli, että Naabot oli kivitetty kuoliaaksi, sanoi Iisebel Ahabille: "Nouse ja ota haltuusi jisreeliläisen Naabotin viinitarha, jota hän ei tahtonut antaa sinulle rahasta; sillä Naabot ei ole enää elossa, vaan on kuollut".
\par 16 Kun Ahab kuuli, että Naabot oli kuollut, nousi hän ja lähti jisreeliläisen Naabotin viinitarhalle ottaakseen sen haltuunsa.
\par 17 Mutta tisbeläiselle Elialle tuli tämä Herran sana:
\par 18 "Nouse ja mene tapaamaan Ahabia, Israelin kuningasta, joka asuu Samariassa. Katso, hän on Naabotin viinitarhassa, jota hän on mennyt ottamaan haltuunsa.
\par 19 Ja puhu hänelle ja sano: 'Näin sanoo Herra: Oletko sinä sekä tappanut että anastanut perinnön?' Ja puhu sitten hänelle ja sano: 'Näin sanoo Herra: Siinä paikassa, missä koirat nuoleskelivat Naabotin veren, tulevat koirat nuoleskelemaan sinunkin veresi'."
\par 20 Ahab sanoi Elialle: "Joko löysit minut, sinä vihamieheni?" Hän vastasi: "Jo löysin. Koska sinä olet myynyt itsesi tekemään sitä, mikä on pahaa Herran silmissä,
\par 21 niin katso, minä annan onnettomuuden kohdata sinua ja lakaisen sinut pois ja hävitän Israelista Ahabin miespuoliset jälkeläiset, kaikki tyynni.
\par 22 Ja minä teen sinun suvullesi niinkuin Jerobeamin, Nebatin pojan, suvulle ja niinkuin Baesan, Ahian pojan, suvulle, koska sinä olet vihoittanut minut ja saattanut Israelin tekemään syntiä.
\par 23 Myöskin Iisebelistä on Herra puhunut sanoen: Koirat syövät Iisebelin Jisreelin muurin luona.
\par 24 Joka Ahabin jälkeläisistä kuolee kaupungissa, sen koirat syövät, ja joka kuolee kedolle, sen syövät taivaan linnut."
\par 25 Totisesti ei ole ollut ketään, joka olisi niin myynyt itsensä tekemään sitä, mikä on pahaa Herran silmissä, kuin Ahab, kun hänen vaimonsa Iisebel vietteli häntä.
\par 26 Hän teki ylen kauhistavasti, kun lähti seuraamaan kivijumalia, aivan niinkuin amorilaiset tekivät, jotka Herra karkoitti israelilaisten tieltä.
\par 27 Mutta kun Ahab kuuli nämä sanat, repäisi hän vaatteensa, pani paljaalle iholleen säkin ja paastosi. Ja hän makasi säkki yllänsä ja liikkui hiljaa.
\par 28 Niin tisbeläiselle Elialle tuli tämä Herran sana:
\par 29 "Oletko nähnyt, kuinka Ahab on nöyrtynyt minun edessäni? Koska hän on nöyrtynyt minun edessäni, en minä anna onnettomuuden tulla hänen aikanansa: hänen poikansa aikana minä annan onnettomuuden kohdata hänen sukuansa."

\chapter{22}

\par 1 He olivat alallaan kolme vuotta: ei ollut sotaa Aramin ja Israelin välillä.
\par 2 Mutta kolmantena vuotena tuli Joosafat, Juudan kuningas, Israelin kuninkaan luo.
\par 3 Ja Israelin kuningas sanoi palvelijoillensa: "Tiedättehän, että Gileadin Raamot on meidän. Ja kuitenkin me olemme toimettomat emmekä ota sitä pois Aramin kuninkaan käsistä."
\par 4 Ja hän sanoi Joosafatille: "Lähdetkö sinä minun kanssani sotaan Gileadin Raamotiin?" Joosafat vastasi Israelin kuninkaalle: "Minä niinkuin sinä, minun kansani niinkuin sinun kansasi, minun hevoseni niinkuin sinun hevosesi!"
\par 5 Mutta Joosafat sanoi Israelin kuninkaalle: "Kysy kuitenkin ensin, mitä Herra sanoo".
\par 6 Niin Israelin kuningas kokosi profeetat, noin neljäsataa miestä, ja sanoi heille: "Onko minun lähdettävä sotaan Gileadin Raamotia vastaan vai oltava lähtemättä?" He vastasivat: "Lähde; Herra antaa sen kuninkaan käsiin".
\par 7 Mutta Joosafat sanoi: "Eikö täällä ole enää ketään muuta Herran profeettaa, jolta voisimme kysyä?"
\par 8 Israelin kuningas vastasi Joosafatille: "On täällä vielä mies, jolta voisimme kysyä Herran mieltä, mutta minä vihaan häntä, sillä hän ei koskaan ennusta minulle hyvää, vaan aina pahaa; se on Miika, Jimlan poika". Joosafat sanoi: "Älköön kuningas niin puhuko".
\par 9 Niin Israelin kuningas kutsui erään hoviherran ja sanoi: "Nouda kiiruusti Miika, Jimlan poika".
\par 10 Mutta Israelin kuningas ja Joosafat, Juudan kuningas, istuivat kumpikin valtaistuimellansa, puettuina kuninkaallisiin pukuihinsa, puimatantereella Samarian portin ovella; ja kaikki profeetat olivat hurmoksissa heidän edessänsä.
\par 11 Ja Sidkia, Kenaanan poika, teki itsellensä rautasarvet ja sanoi: "Näin sanoo Herra: Näillä sinä pusket aramilaisia, kunnes teet heistä lopun".
\par 12 Ja kaikki profeetat ennustivat samalla tavalla, sanoen: "Mene Gileadin Raamotiin, niin sinä saat voiton; Herra antaa sen kuninkaan käsiin".
\par 13 Ja sanansaattaja, joka oli mennyt kutsumaan Miikaa, puhui hänelle sanoen: "Katso, kaikki profeetat ovat yhdestä suusta luvanneet kuninkaalle hyvää; olkoon sinun sanasi heidän sanansa kaltainen, ja lupaa sinäkin hyvää".
\par 14 Mutta Miika vastasi: "Niin totta kuin Herra elää, sen minä puhun, minkä Herra minulle sanoo".
\par 15 Kun hän tuli kuninkaan eteen, sanoi kuningas hänelle: "Miika, onko meidän lähdettävä sotaan Gileadin Raamotiin vai oltava lähtemättä?" Hän vastasi hänelle: "Lähde, niin saat voiton; Herra antaa sen kuninkaan käsiin".
\par 16 Mutta kuningas sanoi hänelle: "Kuinka monta kertaa minun on vannotettava sinua, ettet puhu minulle muuta kuin totuutta Herran nimessä?"
\par 17 Silloin hän sanoi: "Minä näin koko Israelin hajallaan vuorilla niinkuin lampaat, joilla ei ole paimenta. Ja Herra sanoi: 'Näillä ei ole isäntää; palatkoot he kukin rauhassa kotiinsa'."
\par 18 Niin Israelin kuningas sanoi Joosafatille: "Enkö minä sanonut sinulle, ettei tämä koskaan ennusta minulle hyvää, vaan aina pahaa?"
\par 19 Mutta hän sanoi: "Kuule siis Herran sana: Minä näin Herran istuvan istuimellansa ja kaiken taivaan joukon seisovan hänen edessään, hänen oikealla ja vasemmalla puolellansa.
\par 20 Ja Herra sanoi: 'Kuka viekoittelisi Ahabin lähtemään sotaan, että hän kaatuisi Gileadin Raamotissa?' Mikä vastasi niin, mikä näin.
\par 21 Silloin tuli henki ja asettui Herran eteen ja sanoi: 'Minä viekoittelen hänet'. Herra kysyi häneltä: 'Miten?'
\par 22 Hän vastasi: 'Minä menen valheen hengeksi kaikkien hänen profeettainsa suuhun'. Silloin Herra sanoi: 'Saat viekoitella, siihen sinä pystyt; mene ja tee niin'.
\par 23 Katso, nyt Herra on pannut valheen hengen kaikkien näiden sinun profeettaisi suuhun, sillä Herra on päättänyt sinun osaksesi onnettomuuden."
\par 24 Silloin astui esille Sidkia, Kenaanan poika, löi Miikaa poskelle ja sanoi: "Mitä tietä Herran henki on poistunut minusta puhuakseen sinun kanssasi?"
\par 25 Miika vastasi: "Sen saat nähdä sinä päivänä, jona kuljet huoneesta huoneeseen piiloutuaksesi".
\par 26 Mutta Israelin kuningas sanoi: "Ota Miika ja vie hänet takaisin Aamonin, kaupungin päällikön, ja Jooaan, kuninkaan pojan, luo.
\par 27 Ja sano: 'Näin sanoo kuningas: Pankaa tämä vankilaan ja elättäkää häntä vaivaisella vedellä ja leivällä, kunnes minä palaan voittajana takaisin'."
\par 28 Miika vastasi: "Jos sinä palaat voittajana takaisin, niin ei Herra ole puhunut minun kauttani". Ja hän sanoi vielä: "Kuulkaa tämä, kaikki kansat".
\par 29 Niin Israelin kuningas ja Joosafat, Juudan kuningas, menivät Gileadin Raamotiin.
\par 30 Ja Israelin kuningas sanoi Joosafatille: "Täytyypä pukeutua tuntemattomaksi, kun käy taisteluun, mutta ole sinä omissa vaatteissasi". Ja Israelin kuningas pukeutui tuntemattomaksi ja kävi taisteluun.
\par 31 Mutta Aramin kuningas oli käskenyt sotavaunujen päälliköitä, joita hänellä oli kolmekymmentä kaksi, sanoen: "Älkää ryhtykö taisteluun kenenkään muun kanssa, olkoon alempi tai ylempi, kuin ainoastaan Israelin kuninkaan kanssa".
\par 32 Kun sotavaunujen päälliköt näkivät Joosafatin, ajattelivat he: "Tuo on varmaankin Israelin kuningas", ja kääntyivät hyökkäämään häntä vastaan. Silloin Joosafat huusi.
\par 33 Kun sotavaunujen päälliköt näkivät, ettei se ollutkaan Israelin kuningas, vetäytyivät he hänestä pois.
\par 34 Mutta eräs mies, joka oli jännittänyt jousensa ja ampui umpimähkään, satutti Israelin kuningasta vyöpanssarin ja rintahaarniskan väliin. Niin tämä sanoi vaunujensa ohjaajalle: "Käännä vaunut ja vie minut pois sotarinnasta, sillä minä olen haavoittunut".
\par 35 Mutta kun taistelu sinä päivänä yltyi yltymistään, jäi kuningas seisomaan vaunuihinsa, päin aramilaisia. Illalla hän kuoli; ja verta oli vuotanut haavasta vaununpohjaan.
\par 36 Ja auringon laskiessa kaikui kautta sotajoukon huuto: "Joka mies kaupunkiinsa! Joka mies maahansa!"
\par 37 Näin kuoli kuningas, ja hänet vietiin Samariaan; ja kuningas haudattiin Samariaan.
\par 38 Ja kun vaunut huuhdottiin Samarian lammikolla, nuoleskelivat koirat hänen vertansa, ja portot peseytyivät siinä, sen sanan mukaan, jonka Herra oli puhunut.
\par 39 Mitä muuta on kerrottavaa Ahabista ja kaikesta, mitä hän teki, siitä norsunluisesta palatsista, jonka hän rakensi, ja kaikista kaupungeista, jotka hän linnoitti, se on kirjoitettuna Israelin kuningasten aikakirjassa.
\par 40 Ja niin Ahab meni lepoon isiensä tykö. Ja hänen poikansa Ahasja tuli kuninkaaksi hänen sijaansa.
\par 41 Joosafat, Aasan poika, tuli Juudan kuninkaaksi Ahabin, Israelin kuninkaan, neljäntenä hallitusvuotena.
\par 42 Joosafat oli kolmenkymmenen viiden vuoden vanha tullessaan kuninkaaksi, ja hän hallitsi kaksikymmentä viisi vuotta Jerusalemissa. Hänen äitinsä oli nimeltään Asuba, Silhin tytär.
\par 43 Ja hän vaelsi kaikessa isänsä Aasan tietä, siltä poikkeamatta, ja teki sitä, mikä oli oikein Herran silmissä.
\par 44 Mutta uhrikukkulat eivät hävinneet, vaan kansa uhrasi ja suitsutti yhä edelleen uhrikukkuloilla.
\par 45 Ja Joosafat teki rauhan Israelin kuninkaan kanssa.
\par 46 Mitä muuta on kerrottavaa Joosafatista, hänen urotöistään ja hänen sodistaan, se on kirjoitettuna Juudan kuningasten aikakirjassa.
\par 47 Hän hävitti maasta viimeiset niistä haureellisista pyhäkköpojista, jotka olivat jääneet jäljelle hänen isänsä Aasan ajoilta.
\par 48 Edomissa ei siihen aikaan ollut kuningasta; maaherra oli kuninkaana.
\par 49 Ja Joosafat oli teettänyt Tarsiin-laivoja, joiden oli määrä kulkea Oofiriin kultaa noutamaan; mutta laivat eivät päässeet lähtemään, sillä ne särkyivät Esjon-Geberissä.
\par 50 Silloin Ahasja, Ahabin poika, sanoi Joosafatille: "Anna minun palvelijaini kulkea sinun palvelijaisi kanssa laivoissa". Mutta Joosafat ei tahtonut.
\par 51 Ja Joosafat meni lepoon isiensä tykö, ja hänet haudattiin isiensä viereen hänen isänsä Daavidin kaupunkiin. Ja hänen poikansa Jooram tuli kuninkaaksi hänen sijaansa.
\par 52 Ahasja, Ahabin poika, tuli Israelin kuninkaaksi Samariassa Joosafatin, Juudan kuninkaan, seitsemäntenätoista hallitusvuotena, ja hän hallitsi Israelia kaksi vuotta.
\par 53 Hän teki sitä, mikä on pahaa Herran silmissä, ja vaelsi isänsä ja äitinsä tietä ja Jerobeamin, Nebatin pojan, tietä, hänen, joka oli saattanut Israelin tekemään syntiä.
\par 54 Hän palveli Baalia ja kumarsi häntä ja vihoitti Herran, Israelin Jumalan, aivan niinkuin hänen isänsä oli tehnyt.


\end{document}