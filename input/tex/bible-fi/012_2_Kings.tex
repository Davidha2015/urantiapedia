\begin{document}

\title{Toinen kuninkaiden kirja}


\chapter{1}

\par 1 Ahabin kuoleman jälkeen Mooab luopui Israelista.
\par 2 Ahasja putosi Samariassa yläsalinsa ristikon läpi ja makasi sairaana. Ja hän lähetti sanansaattajia ja sanoi heille: "Menkää ja kysykää Baal-Sebubilta, Ekronin jumalalta, toivunko minä tästä taudista".
\par 3 Mutta Herran enkeli oli sanonut tisbeläiselle Elialle: "Nouse ja mene Samarian kuninkaan sanansaattajia vastaan ja sano heille: 'Eikö ole Jumalaa Israelissa, koska menette kysymään neuvoa Baal-Sebubilta, Ekronin jumalalta?'
\par 4 Sentähden sanoo Herra näin: 'Vuoteesta, johon olet noussut, et sinä enää astu alas, sillä sinun on kuoltava'." Ja Elia lähti tiehensä.
\par 5 Kun sanansaattajat sitten palasivat kuninkaan luo, kysyi hän heiltä: "Minkätähden te tulette takaisin?"
\par 6 He vastasivat hänelle: "Meitä vastaan tuli mies, joka sanoi meille: 'Menkää takaisin kuninkaan luo, joka on teidät lähettänyt, ja sanokaa hänelle: Näin sanoo Herra: Eikö ole Jumalaa Israelissa, koska sinä lähetät kysymään neuvoa Baal-Sebubilta, Ekronin jumalalta? Sentähden sinä et enää astu alas vuoteesta, johon olet noussut, sillä sinun on kuoltava.'"
\par 7 Hän kysyi heiltä: "Minkä näköinen oli mies, joka tuli teitä vastaan ja puhui teille näitä?"
\par 8 He vastasivat hänelle: "Hänellä oli yllään karvanahka ja nahkavyö vyötäisillä". Silloin hän sanoi: "Se oli tisbeläinen Elia".
\par 9 Ja hän lähetti hänen luokseen viidenkymmenenpäämiehen ja hänen viisikymmentä miestänsä. Ja kun tämä tuli hänen luoksensa, hänen istuessaan vuoren kukkulalla, sanoi hän hänelle: "Sinä Jumalan mies, kuningas käskee: Tule alas".
\par 10 Mutta Elia vastasi ja sanoi viidenkymmenenpäämiehelle: "Jos minä olen Jumalan mies, niin tulkoon tuli taivaasta ja kuluttakoon sinut ja sinun viisikymmentä miestäsi". Silloin tuli taivaasta tuli ja kulutti hänet ja hänen viisikymmentä miestänsä.
\par 11 Hän lähetti taas hänen luoksensa toisen viidenkymmenenpäämiehen ja hänen viisikymmentä miestänsä. Tämä lausui ja sanoi hänelle: "Sinä Jumalan mies, näin käskee kuningas: Tule kiiruusti alas".
\par 12 Mutta Elia vastasi ja sanoi heille: "Jos minä olen Jumalan mies, niin tulkoon tuli taivaasta ja kuluttakoon sinut ja sinun viisikymmentä miestäsi". Silloin tuli taivaasta Jumalan tuli ja kulutti hänet ja hänen viisikymmentä miestänsä.
\par 13 Hän lähetti vielä kolmannen viidenkymmenenpäämiehen ja hänen viisikymmentä miestänsä. Ja tämä kolmas viidenkymmenenpäämies meni sinne, ja kun hän tuli perille, polvistui hän Elian eteen ja anoi häneltä armoa ja sanoi hänelle: "Sinä Jumalan mies, olkoon minun henkeni ja näiden viidenkymmenen palvelijasi henki kallis sinun silmissäsi.
\par 14 Katso, tuli on tullut taivaasta ja kuluttanut ne kaksi ensimmäistä viidenkymmenenpäämiestä ja heidän viisikymmentä miestänsä, mutta olkoon minun henkeni kallis sinun silmissäsi."
\par 15 Ja Herran enkeli sanoi Elialle: "Mene alas hänen kanssaan, älä pelkää häntä". Niin hän nousi ja meni hänen kanssaan kuninkaan tykö.
\par 16 Ja hän sanoi hänelle: "Näin sanoo Herra: Koska lähetit sanansaattajia kysymään neuvoa Baal-Sebubilta, Ekronin jumalalta, ikäänkuin Israelissa ei olisi Jumalaa, jolta voisi kysyä neuvoa, sentähden sinä et astu alas vuoteesta, johon olet noussut, sillä sinun on kuoltava".
\par 17 Niin hän kuoli, Herran sanan mukaan, jonka Elia oli puhunut. Ja Jooram tuli kuninkaaksi hänen sijaansa Jooramin, Joosafatin pojan, Juudan kuninkaan, toisena hallitusvuotena, sillä Ahasjalla ei ollut poikaa.
\par 18 Mitä muuta on kerrottavaa Ahasjasta, siitä, mitä hän teki, se on kirjoitettuna Israelin kuningasten aikakirjassa.

\chapter{2}

\par 1 Silloin, kun Herra oli vievä Elian tuulispäässä taivaaseen, kulkivat Elia ja Elisa Gilgalista.
\par 2 Ja Elia sanoi Elisalle: "Jää tähän, sillä Herra on lähettänyt minut Beeteliin asti". Mutta Elisa vastasi: "Niin totta kuin Herra elää, ja niin totta kuin sinun sielusi elää: minä en jätä sinua". Ja he menivät Beeteliin.
\par 3 Niin Beetelissä olevat profeetanoppilaat tulivat Elisan tykö ja sanoivat hänelle: "Tiedätkö, että Herra tänä päivänä ottaa sinun herrasi pois sinun pääsi ylitse?" Hän vastasi: "Tiedän kyllä; olkaa vaiti".
\par 4 Ja Elia sanoi hänelle: "Elisa, jää tähän, sillä Herra on lähettänyt minut Jerikoon". Mutta hän vastasi: "Niin totta kuin Herra elää, ja niin totta kuin sinun sielusi elää: minä en jätä sinua". Ja he tulivat Jerikoon.
\par 5 Niin Jerikossa olevat profeetanoppilaat astuivat Elisan tykö ja sanoivat hänelle: "Tiedätkö, että Herra tänä päivänä ottaa herrasi pois sinun pääsi ylitse?" Hän vastasi: "Tiedän kyllä; olkaa vaiti".
\par 6 Ja Elia sanoi hänelle: "Jää tähän, sillä Herra on lähettänyt minut Jordanille". Mutta hän vastasi: "Niin totta kuin Herra elää, ja niin totta kuin sinun sielusi elää: minä en jätä sinua". Niin he kulkivat molemmat yhdessä.
\par 7 Mutta myös profeetanoppilaita lähti matkaan viisikymmentä miestä, ja he asettuivat syrjään jonkun matkan päähän näiden kahden pysähtyessä Jordanille.
\par 8 Niin Elia otti vaippansa, kääri sen kokoon ja löi veteen; ja vesi jakaantui kummallekin puolelle, ja he kävivät molemmat virran yli kuivaa myöten.
\par 9 Kun he olivat tulleet yli, sanoi Elia Elisalle: "Pyydä minua tekemään hyväksesi jotakin, ennenkuin minut otetaan pois sinun tyköäsi". Elisa sanoi: "Tulkoon minuun sinun hengestäsi kaksinkertainen osa".
\par 10 Hän vastasi: "Pyyntösi on vaikea täyttää. Mutta jos näet, kuinka minut otetaan pois sinun tyköäsi, niin se täytetään; jollet, niin se jää täyttämättä."
\par 11 Kun he niin kulkivat ja puhelivat, niin katso, äkkiä ilmestyivät tuliset vaunut ja tuliset hevoset, ja ne erottivat heidät toisistansa, ja Elia nousi tuulispäässä taivaaseen.
\par 12 Kun Elisa sen näki, huusi hän: "Isäni, isäni! Israelin sotavaunut ja ratsumiehet!" Sitten hän ei enää nähnyt häntä. Ja hän tarttui vaatteisiinsa ja repäisi ne kahdeksi kappaleeksi.
\par 13 Sitten hän otti maasta Elian vaipan, joka oli pudonnut hänen päältään, palasi takaisin ja pysähtyi Jordanin rannalle.
\par 14 Ja hän otti Elian vaipan, joka oli pudonnut tämän päältä, löi veteen ja sanoi: "Missä on Herra, Elian Jumala?" Kun hän siis löi veteen, jakaantui se kummallekin puolelle, ja Elisa kävi virran yli.
\par 15 Kun Jerikosta tulleet profeetanoppilaat syrjästä näkivät sen, sanoivat he: "Elian henki on laskeutunut Elisaan". Niin he tulivat häntä vastaan ja kumartuivat hänen edessään maahan.
\par 16 Ja he sanoivat hänelle: "Katso, palvelijaisi joukossa on viisikymmentä urhoollista miestä; menkööt he etsimään sinun isäntääsi. Ehkä Herran Henki on vienyt hänet ja heittänyt jollekin vuorelle tai johonkin laaksoon." Hän vastasi: "Älkää lähettäkö".
\par 17 Mutta he pyysivät häntä pyytämällä, kyllästyksiin asti, kunnes hän sanoi: "Lähettäkää sitten". Niin he lähettivät viisikymmentä miestä; ja nämä etsivät kolme päivää, mutta eivät löytäneet häntä.
\par 18 Kun he tulivat takaisin hänen luoksensa, hänen ollessaan Jerikossa, sanoi hän heille: "Enkö minä kieltänyt teitä menemästä?"
\par 19 Ja kaupungin miehet sanoivat Elisalle: "Katso, kaupungin asema on hyvä, niinkuin herrani näkee, mutta vesi on huonoa, ja maassa syntyy keskoisia".
\par 20 Hän sanoi: "Tuokaa minulle uusi malja ja pankaa siihen suoloja". Ja he toivat sen hänelle.
\par 21 Niin hän meni vesien lähteelle ja heitti siihen suolat, sanoen: "Näin sanoo Herra: Minä olen parantanut tämän veden; siitä ei enää tule kuolemaa eikä keskenmenoa".
\par 22 Niin vesi tuli terveelliseksi, aina tähän päivään asti, Elisan sanan mukaan, jonka hän puhui.
\par 23 Sieltä hän meni Beeteliin, ja hänen käydessään tietä tuli pieniä poikasia kaupungista, ja ne pilkkasivat häntä ja sanoivat hänelle: "Tule ylös, kaljupää, tule ylös, kaljupää!"
\par 24 Ja kun hän kääntyi ja näki heidät, kirosi hän heidät Herran nimeen. Silloin tuli metsästä kaksi karhua ja raateli neljäkymmentä kaksi poikaa kuoliaaksi.
\par 25 Sieltä hän meni Karmel-vuorelle ja palasi sieltä Samariaan.

\chapter{3}

\par 1 Jooram, Ahabin poika, tuli Israelin kuninkaaksi Samariassa Joosafatin, Juudan kuninkaan, kahdeksantenatoista hallitusvuotena, ja hän hallitsi kaksitoista vuotta.
\par 2 Hän teki sitä, mikä on pahaa Herran silmissä, ei kuitenkaan niinkuin hänen isänsä ja äitinsä, sillä hän poisti Baalin patsaan, jonka hänen isänsä oli teettänyt.
\par 3 Kuitenkin hän riippui kiinni Jerobeamin, Nebatin pojan, synnissä, jolla tämä oli saattanut Israelin tekemään syntiä; siitä hän ei luopunut.
\par 4 Meesa, Mooabin kuningas, omisti lammaslaumoja ja maksoi verona Israelin kuninkaalle satatuhatta karitsaa ja sadantuhannen oinaan villat.
\par 5 Mutta Ahabin kuoltua Mooabin kuningas luopui Israelin kuninkaasta.
\par 6 Siihen aikaan lähti kuningas Jooram Samariasta ja katsasti kaiken Israelin.
\par 7 Ja hän meni ja lähetti Joosafatille Juudan kuninkaalle sanan: "Mooabin kuningas on luopunut minusta. Lähdetkö minun kanssani sotaan Mooabia vastaan?" Hän vastasi: "Lähden; minä niinkuin sinä, minun kansani niinkuin sinun kansasi, minun hevoseni niinkuin sinun hevosesi!"
\par 8 Ja Joosafat kysyi: "Mitä tietä meidän on sinne mentävä?" Hän vastasi: "Edomin erämaan tietä".
\par 9 Niin Israelin kuningas, Juudan kuningas ja Edomin kuningas lähtivät liikkeelle. Mutta kun he olivat kierrellen kulkeneet seitsemän päivän matkan eikä ollut vettä sotajoukolle eikä juhdille, jotka seurasivat heitä,
\par 10 sanoi Israelin kuningas: "Voi, Herra on kutsunut nämä kolme kuningasta kokoon antaakseen heidät Mooabin käsiin!"
\par 11 Mutta Joosafat sanoi: "Eikö täällä ole ketään Herran profeettaa, kysyäksemme hänen kauttaan neuvoa Herralta?" Silloin eräs Israelin kuninkaan palvelijoista vastasi ja sanoi: "Täällä on Elisa, Saafatin poika, joka vuodatti vettä Elian käsille".
\par 12 Joosafat sanoi: "Hänellä on Herran sana". Niin Israelin kuningas, Joosafat ja Edomin kuningas menivät hänen tykönsä.
\par 13 Ja Elisa sanoi Israelin kuninkaalle: "Mitä minulla on tekemistä sinun kanssasi? Mene isäsi ja äitisi profeettain tykö." Israelin kuningas sanoi hänelle: "Ei niin! Sillä Herra on kutsunut nämä kolme kuningasta kokoon antaakseen heidät Mooabin käsiin."
\par 14 Elisa sanoi: "Niin totta kuin Herra Sebaot elää, jonka edessä minä seison: jollen tahtoisi tehdä Joosafatille, Juudan kuninkaalle, mieliksi, niin minä tosiaankaan en katsoisi sinuun enkä huomaisi sinua.
\par 15 Mutta tuokaa minulle nyt kanteleensoittaja." Ja kun kanteleensoittaja soitti, niin Herran käsi laskeutui hänen päällensä,
\par 16 ja hän sanoi: "Näin sanoo Herra: Tehkää tämä laakso kuoppia täyteen.
\par 17 Sillä näin sanoo Herra: Te ette tule näkemään tuulta ettekä sadetta, ja kuitenkin tämä laakso tulee vettä täyteen; ja te saatte juoda, sekä te että teidän karjanne ja juhtanne.
\par 18 Mutta tämä on Herran silmissä pieni asia: hän antaa Mooabin teidän käsiinne,
\par 19 ja niin te valloitatte kaikki varustetut kaupungit ja kaikki valitut kaupungit, kaadatte kaikki hedelmäpuut, tukitte kaikki vesilähteet ja turmelette kivillä kaikki hyvät peltopalstat."
\par 20 Ja katso, seuraavana aamuna, sinä hetkenä, jona ruokauhri uhrataan, tuli vettä Edomista päin, niin että maa tuli vettä täyteen.
\par 21 Kun kaikki mooabilaiset kuulivat, että kuninkaat olivat lähteneet sotaan heitä vastaan, kutsuttiin koolle kaikki miekkaan vyöttäytyvät ja myös vanhemmat miehet, ja he asettuivat rajalle.
\par 22 Mutta varhain aamulla, kun aurinko nousi ja paistoi veteen, näkivät mooabilaiset, että vesi heidän edessään oli punaista kuin veri.
\par 23 Ja he sanoivat: "Se on verta; varmaan kuninkaat ovat joutuneet taisteluun keskenään ja surmanneet toisensa. Ja nyt saaliin ryöstöön, Mooab!"
\par 24 Mutta kun he tulivat Israelin leiriin, nousivat israelilaiset ja voittivat mooabilaiset, niin että nämä pakenivat heitä. Ja he hyökkäsivät maahan ja voittivat uudelleen mooabilaiset.
\par 25 Ja he hävittivät kaupungit, kaikille hyville peltopalstoille heitti kukin heistä kiven, täyttäen ne; he tukkivat kaikki vesilähteet ja kaatoivat kaikki hedelmäpuut, niin että ainoastaan Kiir-Haresetissa jäivät kivet paikoilleen. Ja linkomiehet piirittivät kaupungin ja ampuivat sitä.
\par 26 Kun Mooabin kuningas näki joutuvansa tappiolle taistelussa, otti hän mukaansa seitsemänsataa miekkamiestä murtautuakseen Edomin kuninkaan luo; mutta he eivät voineet.
\par 27 Niin hän otti poikansa, esikoisensa, joka oli tuleva kuninkaaksi hänen sijaansa, ja uhrasi hänet polttouhriksi muurilla. Silloin suuri viha kohtasi Israelia, niin että he lähtivät sieltä ja palasivat omaan maahansa.

\chapter{4}

\par 1 Eräs nainen, profeetanoppilaan vaimo, huusi Elisalle sanoen: "Minun mieheni, sinun palvelijasi, kuoli, ja sinä tiedät, että palvelijasi oli herraapelkääväinen mies. Nyt velkoja tulee ottamaan minun molemmat poikani orjikseen."
\par 2 Elisa sanoi hänelle: "Mitä minä voin tehdä sinun hyväksesi? Sano minulle, mitä sinulla on talossasi." Hän vastasi: "Palvelijattarellasi ei ole talossa muuta kuin öljyä sen verran, kuin minä tarvitsen voidellakseni itseäni".
\par 3 Silloin hän sanoi: "Mene ja lainaa itsellesi astioita kylästä, kaikilta naapureiltasi, tyhjiä astioita, tarpeeksi paljon.
\par 4 Mene sitten sisään, sulje ovi jälkeesi ja poikiesi jälkeen ja kaada kaikkiin niihin astioihin. Ja aina kun astia on täysi, nosta se syrjään."
\par 5 Niin hän meni pois hänen tyköänsä ja sulki oven jälkeensä ja poikiensa jälkeen. Nämä toivat sitten hänelle astiat, ja hän kaatoi niihin.
\par 6 Ja kun astiat olivat täynnä, sanoi hän pojallensa: "Tuo minulle vielä yksi astia". Mutta hän vastasi hänelle: "Ei ole enää astiata". Silloin öljy tyrehtyi.
\par 7 Ja hän meni ja kertoi sen Jumalan miehelle. Tämä sanoi: "Mene ja myy öljy ja maksa velkasi. Sinä ja poikasi elätte siitä, mikä jää jäljelle."
\par 8 Eräänä päivänä Elisa meni Suunemiin, ja siellä oli arvossa pidetty vaimo, joka vaati häntä aterioimaan. Ja niin usein kuin hän kulki sen kautta, poikkesi hän sinne aterioimaan.
\par 9 Niin vaimo sanoi miehellensä: "Katso, minä olen huomannut, että hän, joka aina kulkee meidän kauttamme, on pyhä Jumalan mies.
\par 10 Muurauttakaamme pieni yliskammio ja pankaamme hänelle sinne vuode, pöytä, tuoli ja lamppu, niin että hän tullessaan meidän luoksemme voi vetäytyä sinne."
\par 11 Eräänä päivänä hän sitten tuli sinne, poikkesi yliskammioon ja pani sinne maata.
\par 12 Ja hän sanoi palvelijallensa Geehasille: "Kutsu se suunemilainen vaimo". Ja hän kutsui tämän, ja hän astui hänen eteensä.
\par 13 Niin hän sanoi palvelijalle: "Sano hänelle: 'Katso, sinä olet nähnyt kaiken tämän vaivan meidän tähtemme. Voisinko minä tehdä jotakin sinun hyväksesi? Onko sinulla jotakin esitettävää kuninkaalle tai sotapäällikölle?'" Mutta hän vastasi: "Minähän asun heimoni keskellä".
\par 14 Niin hän kysyi: "Eikö siis voi tehdä mitään hänen hyväksensä?" Geehasi vastasi: "Voi kyllä: hänellä ei ole poikaa, ja hänen miehensä on vanha".
\par 15 Hän sanoi: "Kutsu hänet". Niin tämä kutsui hänet, ja hän astui ovelle.
\par 16 Ja Elisa sanoi: "Tähän aikaan tulevana vuonna on sinulla poika sylissäsi". Hän vastasi: "Ei, herrani; sinä Jumalan mies, älä petä palvelijatartasi".
\par 17 Mutta vaimo tuli raskaaksi ja synnytti pojan seuraavana vuonna juuri sinä aikana, jonka Elisa oli hänelle sanonut.
\par 18 Kun poika oli kasvanut isoksi, meni hän eräänä päivänä isänsä tykö, joka oli leikkuuväen luona.
\par 19 Ja hän valitti isällensä: "Voi minun päätäni, voi minun päätäni!" Tämä sanoi palvelijalle: "Kanna hänet äitinsä luo".
\par 20 Niin palvelija otti hänet ja vei hänen äitinsä luo. Ja poika istui hänen polvillansa puolipäivään asti; sitten hän kuoli.
\par 21 Ja äiti meni ylös ja laski hänet Jumalan miehen vuoteeseen, sulki oven, niin että poika jäi yksin, ja lähti pois.
\par 22 Sitten hän kutsui miehensä ja sanoi: "Lähetä minulle palvelija ja aasintamma, niin minä riennän Jumalan miehen tykö ja tulen heti takaisin".
\par 23 Hän sanoi: "Miksi menet hänen tykönsä tänä päivänä? Eihän nyt ole uusikuu eikä sapatti." Hän vastasi: "Ole huoleti".
\par 24 Ja hän satuloi aasintamman ja sanoi palvelijallensa: "Aja yhä eteenpäin äläkä keskeytä minun matkaani, ennenkuin sanon sinulle".
\par 25 Niin hän lähti ja tuli Jumalan miehen tykö Karmel-vuorelle. Kun Jumalan mies näki hänet kaukaa, sanoi hän palvelijalleen Geehasille: "Katso, se on suunemilainen vaimo.
\par 26 Riennä nyt häntä vastaan ja kysy häneltä: 'Voitko sinä hyvin, voivatko miehesi ja poikasi hyvin?'" Hän vastasi: "Hyvin".
\par 27 Tultuaan vuorelle Jumalan miehen tykö hän tarttui hänen jalkoihinsa. Niin Geehasi astui esiin työntääkseen hänet pois. Mutta Jumalan mies sanoi: "Jätä hänet rauhaan, sillä hänen sielunsa on murheellinen, ja Herra on salannut sen minulta eikä ole ilmoittanut sitä minulle".
\par 28 Hän sanoi: "Pyysinkö minä poikaa herraltani? Enkö sanonut: 'Älä uskottele minulle'?"
\par 29 Silloin Elisa sanoi Geehasille: "Vyötä kupeesi ja ota minun sauvani käteesi ja mene. Jos kohtaat jonkun, älä tervehdi häntä, ja jos joku tervehtii sinua, älä vastaa. Ja pane minun sauvani pojan kasvoille."
\par 30 Mutta pojan äiti sanoi: "Niin totta kuin Herra elää, ja niin totta kuin sinun sielusi elää: minä en jätä sinua". Ja Elisa nousi ja seurasi häntä.
\par 31 Mutta Geehasi oli mennyt heidän edellänsä ja pannut sauvan pojan kasvoille, mutta tämä ei ääntänyt eikä kuullut. Niin hän tuli takaisin häntä vastaan ja kertoi hänelle sanoen: "Poika ei herännyt".
\par 32 Ja kun Elisa tuli huoneeseen, niin katso, poika makasi hänen vuoteessaan kuolleena.
\par 33 Tultuaan sisään hän sulki oven, niin että he jäivät kahden, ja rukoili Herraa.
\par 34 Senjälkeen hän nousi vuoteelle ja laskeutui pojan yli, pannen suunsa hänen suunsa päälle, silmänsä hänen silmiensä päälle ja kätensä hänen kättensä päälle. Ja kun hän näin kumartui hänen ylitsensä, lämpeni pojan ruumis.
\par 35 Sitten hän kulki huoneessa kerran edestakaisin ja nousi ja kumartui hänen ylitsensä. Silloin poika aivasti seitsemän kertaa, ja senjälkeen poika avasi silmänsä.
\par 36 Hän kutsui Geehasin ja sanoi: "Kutsu se suunemilainen vaimo". Niin hän kutsui vaimon, ja tämä tuli hänen luokseen. Hän sanoi: "Ota poikasi".
\par 37 Ja vaimo tuli ja lankesi hänen jalkainsa juureen, kumartui maahan ja otti poikansa ja meni ulos.
\par 38 Elisa palasi Gilgaliin, ja maassa oli nälänhätä. Ja kun profeetanoppilaat istuivat hänen edessänsä, sanoi hän palvelijallensa: "Pane suuri pata liedelle ja keitä keitto profeetanoppilaille".
\par 39 Niin muuan heistä meni kedolle poimimaan yrttejä ja löysi villin köynnöskasvin ja poimi siitä villikurkkuja viittansa täyden. Tultuaan hän leikkeli ne keittopataan, sillä he eivät tunteneet niitä.
\par 40 Ja he kaatoivat miesten syödä, mutta kun nämä maistoivat keittoa, huusivat he ja sanoivat: "Jumalan mies, padassa on kuolema!" Eivätkä he voineet syödä.
\par 41 Silloin hän sanoi: "Tuokaa jauhoja". Ja hän heitti ne pataan. Sitten hän sanoi: "Kaatakaa väen syödä". Eikä padassa ollut mitään vahingollista.
\par 42 Ja eräs mies tuli Baal-Saalisasta ja toi Jumalan miehelle uutisleipää, kaksikymmentä ohraleipää ja tuleentumatonta viljaa repussansa. Silloin Elisa sanoi: "Anna väelle syödä".
\par 43 Mutta hänen palvelijansa sanoi: "Kuinka minä voin tarjota siitä sadalle miehelle?" Hän sanoi: "Anna väelle syödä; sillä näin sanoo Herra: He syövät, ja jää tähteeksikin".
\par 44 Ja hän tarjosi heille; ja he söivät, ja jäi tähteeksikin, Herran sanan mukaan.

\chapter{5}

\par 1 Naeman, Aramin kuninkaan sotapäällikkö, oli herransa hyvin arvossa pitämä ja suurta kunnioitusta nauttiva mies, sillä hänen kauttansa Herra oli antanut Aramille voiton; ja hän oli sotaurho, mutta pitalitautinen.
\par 2 Kerran olivat aramilaiset menneet ryöstöretkelle ja tuoneet Israelin maasta vankina pienen tytön, joka joutui palvelukseen Naemanin puolisolle.
\par 3 Tyttö sanoi emännällensä: "Oi, jospa herrani kävisi profeetan luona Samariassa! Hän kyllä päästäisi hänet hänen pitalistansa."
\par 4 Niin Naeman meni ja kertoi tämän herrallensa, sanoen: "Niin ja niin on tyttö, joka on Israelin maasta, puhunut".
\par 5 Aramin kuningas vastasi: "Mene vain sinne; minä lähetän kirjeen Israelin kuninkaalle". Niin Naeman lähti sinne ja otti mukaansa kymmenen talenttia hopeata, kuusituhatta sekeliä kultaa ja kymmenen juhlapukua.
\par 6 Ja hän vei Israelin kuninkaalle kirjeen, joka kuului näin: "Kun tämä kirje tulee sinulle, niin katso, minä olen lähettänyt luoksesi palvelijani Naemanin, että sinä päästäisit hänet hänen pitalistansa".
\par 7 Kun Israelin kuningas oli lukenut kirjeen, repäisi hän vaatteensa ja sanoi: "Olenko minä Jumala, joka ottaa elämän ja antaa elämän, koska hän käskee minua päästämään miehen hänen pitalistansa? Ymmärtäkää ja nähkää, että hän etsii aihetta minua vastaan."
\par 8 Mutta kun Jumalan mies Elisa kuuli, että Israelin kuningas oli reväissyt vaatteensa, lähetti hän kuninkaalle sanan: "Miksi olet reväissyt vaatteesi? Anna hänen tulla minun luokseni, niin hän tulee tietämään, että Israelissa on profeetta."
\par 9 Niin Naeman tuli hevosineen ja vaunuineen ja pysähtyi Elisan talon oven eteen.
\par 10 Ja Elisa lähetti hänen luokseen sanansaattajan ja käski sanoa: "Mene ja peseydy seitsemän kertaa Jordanissa, niin lihasi tulee entisellensä, ja sinä tulet puhtaaksi".
\par 11 Mutta Naeman vihastui ja meni matkaansa sanoen: "Katso, minä luulin hänen edes tulevan ja astuvan esiin ja rukoilevan Herran, Jumalansa, nimeä, heiluttavan kättänsä sen paikan yli ja niin poistavan pitalin.
\par 12 Eivätkö Damaskon virrat, Abana ja Parpar, ole kaikkia Israelin vesiä paremmat? Voisinhan minä yhtä hyvin peseytyä niissä tullakseni puhtaaksi." Ja hän kääntyi ja meni tiehensä kiukustuneena.
\par 13 Mutta hänen palvelijansa astuivat esiin ja puhuttelivat häntä ja sanoivat: "Isäni, jos profeetta olisi määrännyt sinulle jotakin erinomaista, etkö tekisi sitä? Saati sitten, kun hän sanoi sinulle ainoastaan: 'Peseydy, niin tulet puhtaaksi'."
\par 14 Niin hän meni ja sukelsi Jordaniin seitsemän kertaa, niinkuin Jumalan mies oli sanonut; ja hänen lihansa tuli entisellensä, pienen pojan lihan kaltaiseksi, ja hän tuli puhtaaksi.
\par 15 Sitten hän palasi Jumalan miehen luo, hän ja koko hänen joukkonsa, meni sisälle, astui hänen eteensä ja sanoi: "Katso, nyt minä tiedän, ettei Jumalaa ole missään muualla maan päällä kuin Israelissa. Ota siis vastaan jäähyväislahja palvelijaltasi."
\par 16 Mutta hän vastasi: "Niin totta kuin Herra elää, jonka edessä minä seison, en minä sitä ota". Ja Naeman pyytämällä pyysi häntä ottamaan, mutta hän epäsi.
\par 17 Niin Naeman sanoi: "Jos et tätä otakaan, salli kuitenkin palvelijasi saada sen verran maata, kuin muulipari voi kantaa. Sillä palvelijasi ei enää uhraa polttouhria eikä teurasuhria muille jumalille kuin Herralle.
\par 18 Tämän antakoon kuitenkin Herra anteeksi palvelijallesi: kun minun herrani menee Rimmonin temppeliin rukoilemaan, nojaten minun käteeni, ja minäkin kumartaen rukoilen Rimmonin temppelissä, niin antakoon Herra palvelijallesi anteeksi sen, että minä kumartaen rukoilen Rimmonin temppelissä."
\par 19 Elisa sanoi hänelle: "Mene rauhassa". Kun hän oli kulkenut hänen luotansa jonkun matkaa,
\par 20 ajatteli Geehasi, Jumalan miehen Elisan palvelija: "Katso, minun herrani säästi tuota aramilaista Naemania, niin ettei ottanut häneltä, mitä hän oli tuonut. Niin totta kuin Herra elää, minä riennän hänen jälkeensä ja otan häneltä jotakin."
\par 21 Niin Geehasi juoksi Naemanin jälkeen. Kun Naeman näki hänen rientävän jäljessänsä, hyppäsi hän alas vaunuistaan, meni häntä vastaan ja kysyi: "Onko kaikki hyvin?"
\par 22 Hän vastasi: "Hyvin on; mutta herrani on lähettänyt minut ja käskee sanoa: 'Nyt juuri tuli luokseni Efraimin vuoristosta kaksi nuorta miestä, profeetanoppilaita. Anna heille talentti hopeata ja kaksi juhlapukua.'"
\par 23 Naeman vastasi: "Suvaitse ottaa kaksi talenttia". Ja hän pyytämällä pyysi häntä. Niin hän sitoi kaksi talenttia hopeata kahteen kukkaroon ja antoi ne ynnä kaksi juhlapukua kahdelle palvelijallensa, ja he kantoivat niitä hänen edellänsä.
\par 24 Mutta kun hän tuli kummulle, otti hän ne heidän käsistänsä ja kätki ne taloon; sitten hän päästi miehet menemään.
\par 25 Ja hän meni sisälle ja astui herransa luo. Niin Elisa kysyi häneltä: "Mistä tulet, Geehasi?" Hän vastasi: "Palvelijasi ei ole käynyt missään".
\par 26 Elisa sanoi hänelle: "Eikö minun henkeni kulkenut sinun kanssasi, kun eräs mies kääntyi vaunuissansa sinua vastaan? Oliko nyt aika ottaa hopeata ja hankkia vaatteita, öljytarhoja, viinitarhoja, lampaita, raavaita, palvelijoita ja palvelijattaria?
\par 27 Naemanin pitalitauti tarttuu sinuun ja sinun jälkeläisiisi ikiajoiksi." Ja Geehasi lähti hänen luotansa lumivalkeana pitalista.

\chapter{6}

\par 1 Ja profeetanoppilaat sanoivat Elisalle: "Katso, huone, jossa me istumme sinun edessäsi, on meille liian ahdas.
\par 2 Menkäämme siis Jordanille, ja tuokaamme sieltä kukin yksi hirsi ja rakentakaamme sinne huone istuaksemme." Hän sanoi: "Menkää".
\par 3 Eräs heistä sanoi: "Suvaitse tulla palvelijasi kanssa". Hän sanoi: "Minä tulen".
\par 4 Niin hän meni heidän kanssaan. Ja Jordanille tultuaan he hakkasivat puita.
\par 5 Mutta erään heistä kaataessa hirttä kirposi kirves veteen. Niin hän huusi ja sanoi: "Voi, herrani, se oli vielä lainattu!"
\par 6 Jumalan mies kysyi: "Mihin se kirposi?" Ja hän näytti hänelle paikan. Silloin hän veisti puukappaleen ja heitti sen siihen ja sai kirveen nousemaan pinnalle.
\par 7 Sitten hän sanoi: "Nosta se ylös". Ja mies ojensi kätensä ja otti sen.
\par 8 Kun Aramin kuningas oli sodassa Israelia vastaan, neuvotteli hän palvelijainsa kanssa ja sanoi: "Siihen ja siihen paikkaan minä asetun leiriin".
\par 9 Mutta Jumalan mies lähetti Israelin kuninkaalle sanan: "Varo, ettet mene siihen paikkaan, sillä aramilaiset ovat asettuneet sinne".
\par 10 Niin Israelin kuningas lähetti sanan siihen paikkaan, jonka Jumalan mies oli hänelle sanonut. Näin tämä varoitti häntä, ja hän oli siellä varuillaan. Eikä se tapahtunut ainoastaan kerran tai kahdesti.
\par 11 Aramin kuninkaan sydän tuli tästä levottomaksi, ja hän kutsui palvelijansa ja sanoi heille: "Ettekö voi ilmaista minulle, kuka meikäläisistä pitää Israelin kuninkaan puolta?"
\par 12 Silloin eräs hänen palvelijoistaan sanoi: "Ei niin, herrani, kuningas, vaan profeetta Elisa, joka on Israelissa, ilmaisee Israelin kuninkaalle nekin sanat, jotka sinä puhut makuuhuoneessasi".
\par 13 Hän sanoi: "Menkää ja katsokaa, missä hän on, niin minä lähetän ottamaan hänet kiinni". Ja hänelle ilmoitettiin: "Katso, hän on Dootanissa".
\par 14 Niin hän lähetti sinne hevosia ja sotavaunuja ja suuren sotajoukon. He tulivat sinne yöllä ja ympäröivät kaupungin.
\par 15 Kun Jumalan miehen palvelija nousi aamulla varhain ja meni ulos, niin katso, sotajoukko, hevoset ja sotavaunut piirittivät kaupunkia. Ja hänen palvelijansa sanoi hänelle: "Voi, herrani, mitä me nyt teemme?"
\par 16 Hän sanoi: "Älä pelkää, sillä niitä, jotka ovat meidän kanssamme, on enemmän kuin niitä, jotka ovat heidän kanssansa".
\par 17 Ja Elisa rukoili ja sanoi: "Herra, avaa hänen silmänsä, että hän näkisi". Ja Herra avasi palvelijan silmät, ja hän näki, ja katso: vuori oli täynnä tulisia hevosia ja tulisia vaunuja Elisan ympärillä.
\par 18 Kun viholliset sitten tulivat häntä vastaan, rukoili Elisa Herraa ja sanoi: "Sokaise tämä väki". Silloin hän sokaisi heidät Elisan pyynnön mukaan.
\par 19 Ja Elisa sanoi heille: "Ei tämä ole oikea tie, eikä tämä ole oikea kaupunki. Seuratkaa minua, niin minä vien teidät sen miehen luo, jota te etsitte." Ja hän vei heidät Samariaan.
\par 20 Mutta kun he tulivat Samariaan, sanoi Elisa: "Herra, avaa näiden silmät, että he näkisivät". Silloin Herra avasi heidän silmänsä, ja he näkivät; ja katso: he olivat keskellä Samariaa.
\par 21 Ja kun Israelin kuningas näki heidät, sanoi hän Elisalle: "Surmaanko minä heidät, isäni, surmaanko heidät?"
\par 22 Hän sanoi: "Älä surmaa. Surmaatko sinä ne, jotka otat vangiksi miekallasi ja jousellasi? Pane heidän eteensä ruokaa ja juomaa heidän syödä ja juoda; menkööt sitten takaisin herransa luo."
\par 23 Niin tämä valmisti heille suuren aterian, ja kun he olivat syöneet ja juoneet, päästi hän heidät menemään; ja he menivät herransa luo. Eikä aramilaisia partiojoukkoja sitten enää tullut Israelin maahan.
\par 24 Sen jälkeen Benhadad, Aramin kuningas, kokosi kaiken sotajoukkonsa ja tuli ja piiritti Samarian.
\par 25 Silloin syntyi Samariassa, heidän piirittäessään sitä, suuri nälänhätä, niin että aasinpää maksoi kahdeksankymmentä hopeasekeliä ja neljännes kab-mittaa kyyhkysensontaa viisi hopeasekeliä.
\par 26 Ja kun Israelin kuningas käveli muurin päällä, huusi muuan vaimo hänelle ja sanoi: "Auta, herrani, kuningas".
\par 27 Hän vastasi: "Jollei Herra auta sinua, niin mistä minä hankin sinulle apua? Puimatantereeltako vai viinikuurnasta?"
\par 28 Ja kuningas sanoi hänelle: "Mikä sinun on?" Hän vastasi: "Tämä vaimo sanoi minulle: 'Anna tänne poikasi, syödäksemme hänet tänä päivänä, niin syömme huomenna minun poikani'.
\par 29 Ja me keitimme minun poikani ja söimme hänet. Ja minä sanoin toisena päivänä hänelle: 'Anna tänne poikasi, syödäksemme hänet'. Mutta hän piilotti poikansa."
\par 30 Kun kuningas kuuli vaimon sanat, repäisi hän vaatteensa, kävellessään muurin päällä. Niin kansa näki, että hänellä oli vaatteiden alla säkki, paljaalla iholla.
\par 31 Ja hän sanoi: "Jumala rangaiskoon minua nyt ja vasta, jos Elisan, Saafatin pojan, pää tänä päivänä jää hänen hartioilleen".
\par 32 Elisa istui talossaan, ja vanhimmat istuivat hänen tykönänsä. Ja kuningas oli lähettänyt miehen edellänsä. Mutta ennenkuin sanansaattaja tuli hänen luokseen, sanoi hän vanhimmille: "Näettekö, kuinka se murhamiehen poika lähettää hakkaamaan minulta päätä poikki? Mutta kun sanansaattaja tulee, katsokaa, että suljette oven ja pidätte oven kiinni häneltä. Eivätkö jo kuulu hänen herransa askeleet hänen jäljessään?"
\par 33 Hänen vielä puhuessaan heidän kanssansa, tuli sanansaattaja hänen luokseen ja sanoi: "Katso, tämä onnettomuus tulee Herralta; mitä minä enää odottaisin Herraa?"

\chapter{7}

\par 1 Mutta Elisa vastasi: "Kuulkaa Herran sana; näin sanoo Herra: Huomenna tähän aikaan maksaa Samarian portissa sea-mitta lestyjä jauhoja sekelin ja kaksi sea-mittaa ohria sekelin".
\par 2 Niin vaunusoturi, jonka käsivarteen kuningas nojasi, vastasi Jumalan miehelle ja sanoi: "Katso, vaikka Herra tekisi akkunat taivaaseen, kuinka voisi tämä tapahtua?" Hän sanoi: "Sinä olet näkevä sen omin silmin, mutta syödä siitä et saa".
\par 3 Ja kaupungin portin oven edustalla oleskeli neljä pitalitautista miestä. He sanoivat toisillensa: "Mitä me istumme tässä, kunnes kuolemme?
\par 4 Jos päätämme mennä kaupunkiin, jossa on nälänhätä, niin me kuolemme. Jos jäämme tähän, niin me kuolemme. Tulkaa, siirtykäämme nyt aramilaisten leiriin. Jos he jättävät meidät eloon, niin me jäämme eloon; jos he surmaavat meidät, niin me kuolemme."
\par 5 Niin he nousivat hämärissä mennäkseen aramilaisten leiriin. Kun he tulivat aramilaisten leirin laitaan, niin katso: siellä ei ollut ketään.
\par 6 Sillä Herra oli antanut aramilaisten sotajoukon kuulla sotavaunujen, hevosten ja suuren sotajoukon töminää; ja niin he olivat sanoneet toisilleen: "Katso, Israelin kuningas on palkannut meitä vastaan heettiläisten kuninkaat ja egyptiläisten kuninkaat, että nämä hyökkäisivät meidän kimppuumme".
\par 7 Niin he olivat lähteneet liikkeelle ja paenneet hämärissä ja jättäneet telttansa, hevosensa, aasinsa ja leirinsä, niinkuin se oli; he olivat paenneet pelastaakseen henkensä.
\par 8 Tultuaan leirin laitaan pitalitautiset menivät erääseen telttaan, söivät ja joivat, ottivat sieltä hopeata ja kultaa ja vaatteita, menivät pois ja kätkivät ne. Sitten he tulivat takaisin ja menivät toiseen telttaan ja ottivat sieltä saalista ja menivät pois ja kätkivät sen.
\par 9 Mutta sitten he sanoivat toisilleen: "Emme tee oikein. Tämä päivä on hyvän sanoman päivä. Jos olemme vaiti ja odotamme aamun valkenemiseen asti, niin me joudumme syyllisiksi. Tulkaa, menkäämme nyt ilmoittamaan tämä kuninkaan linnaan."
\par 10 Niin he tulivat ja kutsuivat kaupungin portinvartijat ja ilmoittivat heille sanoen: "Me tulimme aramilaisten leiriin, ja katso, siellä ei ollut ketään eikä kuulunut ihmisääntä; siellä oli vain hevosia ja aaseja kytkettyinä kiinni, ja teltat olivat, niinkuin olivat olleet".
\par 11 Portinvartijat huusivat ja ilmoittivat tämän kuninkaan linnaan.
\par 12 Silloin kuningas nousi yöllä ja sanoi palvelijoilleen: "Minä sanon teille, minkä aramilaiset meille tekevät. He tietävät meidän kärsivän nälkää, ja sentähden he ovat poistuneet leiristä ja piiloutuneet kedolle, ajatellen: kun ne lähtevät kaupungista, niin me otamme heidät elävinä kiinni ja menemme kaupunkiin."
\par 13 Mutta eräs hänen palvelijoistaan vastasi ja sanoi: "Otettakoon viisi tähteeksi jääneistä hevosista, jotka ovat vielä täällä jäljellä - niidenhän käy kuitenkin samoin kuin kaiken Israelin joukon, joka on täällä jäljellä, ja samoin kuin kaiken Israelin joukon, joka on hukkunut - ja lähettäkäämme katsomaan".
\par 14 Niin he ottivat kahdet sotavaunut hevosineen, ja kuningas lähetti ne aramilaisten sotajoukon jälkeen, sanoen: "Menkää ja katsokaa".
\par 15 He menivät heidän jälkeensä aina Jordanille asti; ja katso: koko tie oli täynnä vaatteita ja aseita, jotka aramilaiset olivat heittäneet pois rientäessään pakoon. Niin sanansaattajat tulivat takaisin ja ilmoittivat sen kuninkaalle.
\par 16 Silloin kansa lähti ja ryösti aramilaisten leirin. Ja niin sea-mitta lestyjä jauhoja maksoi sekelin ja kaksi sea-mittaa ohria sekelin, Herran sanan mukaan.
\par 17 Kuningas oli asettanut sen vaunusoturin, jonka käsivarteen hän nojasi, porttiin valvomaan järjestystä. Mutta kansa tallasi hänet kuoliaaksi portissa, niinkuin Jumalan mies oli puhunut, silloin kun kuningas tuli hänen luoksensa.
\par 18 Sillä kun Jumalan mies oli puhunut kuninkaalle näin: "Huomenna tähän aikaan maksaa kaksi sea-mittaa ohria Samarian portissa sekelin ja sea-mitta lestyjä jauhoja sekelin",
\par 19 oli vaunusoturi vastannut Jumalan miehelle ja sanonut: "Katso, vaikka Herra tekisi akkunat taivaaseen, kuinka tämä voisi tapahtua?" Silloin hän oli sanonut: "Sinä olet näkevä sen omin silmin, mutta syödä siitä et saa".
\par 20 Ja niin hänen myös kävi: kansa tallasi hänet kuoliaaksi portissa.

\chapter{8}

\par 1 Ja Elisa puhui vaimolle, jonka pojan hän oli palauttanut henkiin, ja sanoi: "Nouse ja lähde, sinä ja sinun perheesi, asumaan muukalaisena, missä vain voit asua, sillä Herra on käskenyt nälänhädän tulla, ja se myös tulee maahan seitsemäksi vuodeksi".
\par 2 Niin vaimo nousi ja teki Jumalan miehen sanan mukaan: hän lähti ja asui perheinensä muukalaisena filistealaisten maassa seitsemän vuotta.
\par 3 Mutta niiden seitsemän vuoden kuluttua vaimo palasi filistealaisten maasta; ja hän tuli anomaan kuninkaalta taloansa ja peltoansa.
\par 4 Ja kuningas puhutteli juuri Geehasia, Jumalan miehen palvelijaa, ja sanoi: "Kerro minulle kaikki ne suuret teot, jotka Elisa on tehnyt".
\par 5 Juuri kun hän kertoi kuninkaalle, kuinka Elisa oli palauttanut kuolleen henkiin, tuli vaimo, jonka pojan hän oli palauttanut henkiin, anomaan kuninkaalta taloansa ja peltoansa. Silloin Geehasi sanoi: "Herrani, kuningas, tämä on se vaimo; ja tämä on se poika, jonka Elisa palautti henkiin".
\par 6 Ja kuningas kyseli vaimolta, ja tämä kertoi sen hänelle. Niin kuningas antoi hänen mukaansa hoviherran ja sanoi: "Toimita tälle takaisin kaikki, mikä on hänen omaansa, ja lisäksi kaikki pellon sadot siitä päivästä alkaen, jona hän jätti maan, aina tähän asti".
\par 7 Ja Elisa tuli Damaskoon. Ja Aramin kuningas Benhadad oli sairaana. Kun hänelle ilmoitettiin: "Jumalan mies on tullut tänne",
\par 8 sanoi kuningas Hasaelille: "Ota mukaasi lahja ja mene Jumalan miestä vastaan ja kysy hänen kauttansa Herralta: 'Toivunko minä tästä taudista?'"
\par 9 Ja Hasael meni häntä vastaan ja otti mukaansa lahjaksi kaikkea, mitä Damaskossa oli parasta, neljäkymmentä kamelinkuormaa. Ja hän tuli ja astui hänen eteensä ja sanoi: "Poikasi Benhadad, Aramin kuningas, lähetti minut sinun luoksesi ja käski kysyä: 'Toivunko minä tästä taudista?'"
\par 10 Elisa sanoi: "Mene ja sano hänelle: 'Toivut'. Mutta Herra on ilmoittanut minulle, että hänen on kuoltava."
\par 11 Ja hän tuijotti eteensä kauan ja liikahtamatta; sitten Jumalan mies alkoi itkeä.
\par 12 Niin Hasael sanoi: "Minkätähden herrani itkee?" Hän vastasi: "Sentähden, että minä tiedän, kuinka paljon pahaa sinä olet tekevä israelilaisille: sinä pistät heidän varustetut kaupunkinsa tuleen, sinä surmaat miekalla heidän valiomiehensä, sinä murskaat heidän pienet lapsensa ja halkaiset heidän raskaat vaimonsa".
\par 13 Hasael sanoi: "Mikä on palvelijasi, tällainen koira, tekemään näitä suuria asioita?" Elisa vastasi: "Herra on ilmoittanut minulle, että sinusta on tuleva Aramin kuningas".
\par 14 Niin hän meni Elisan tyköä, ja kun hän tuli herransa tykö, kysyi tämä häneltä: "Mitä Elisa sanoi sinulle?" Hän vastasi: "Hän sanoi minulle, että sinä toivut".
\par 15 Mutta seuraavana päivänä hän otti peitteen ja kastoi sen veteen ja levitti sen hänen kasvoillensa; niin hän kuoli. Ja Hasael tuli kuninkaaksi hänen sijaansa.
\par 16 Israelin kuninkaan Jooramin, Ahabin pojan, viidentenä hallitusvuotena tuli Jooram, Juudan kuninkaan Joosafatin poika, kuninkaaksi.
\par 17 Hän oli kolmenkymmenen kahden vuoden vanha tullessaan kuninkaaksi, ja hän hallitsi Jerusalemissa kahdeksan vuotta.
\par 18 Mutta hän vaelsi Israelin kuningasten tietä, niinkuin Ahabin suku oli tehnyt, sillä hänellä oli puolisona Ahabin tytär. Ja niin hän teki sitä, mikä on pahaa Herran silmissä.
\par 19 Mutta palvelijansa Daavidin tähden Herra ei tahtonut tuhota Juudaa, koska hän oli luvannut antaa hänelle ja hänen pojilleen lampun ainiaaksi.
\par 20 Hänen aikanaan edomilaiset luopuivat Juudan vallanalaisuudesta ja asettivat itsellensä kuninkaan.
\par 21 Ja Jooram lähti Saairiin kaikkine sotavaunuinensa. Ja hän nousi yöllä ja voitti edomilaiset, jotka olivat saartaneet hänet, sekä sotavaunujen päälliköt. Mutta väki pakeni majoillensa;
\par 22 ja edomilaiset luopuivat Juudan vallanalaisuudesta; niin aina tähän päivään asti. Siihen aikaan luopui myös Libna.
\par 23 Mitä muuta on kerrottavaa Jooramista ja kaikesta, mitä hän teki, se on kirjoitettuna Juudan kuningasten aikakirjassa.
\par 24 Sitten Jooram meni lepoon isiensä tykö, ja hänet haudattiin isiensä viereen Daavidin kaupunkiin. Ja hänen poikansa Ahasja tuli kuninkaaksi hänen sijaansa.
\par 25 Israelin kuninkaan Jooramin, Ahabin pojan, kahdentenatoista hallitusvuotena tuli Ahasja, Juudan kuninkaan Jooramin poika, kuninkaaksi.
\par 26 Ja Ahasja oli kahdenkymmenen kahden vuoden vanha tullessaan kuninkaaksi, ja hän hallitsi Jerusalemissa vuoden. Hänen äitinsä oli nimeltään Atalja, Israelin kuninkaan Omrin tytär.
\par 27 Ahasja vaelsi Ahabin suvun tietä ja teki sitä, mikä on pahaa Herran silmissä, samoin kuin Ahabin suku; sillä hän oli lankoutunut Ahabin suvun kanssa.
\par 28 Ja hän lähti Jooramin, Ahabin pojan, kanssa sotimaan Hasaelia, Aramin kuningasta, vastaan Gileadin Raamotiin; mutta aramilaiset haavoittivat Jooramin.
\par 29 Niin kuningas Jooram tuli takaisin parantuakseen Jisreelissä haavoista, joita aramilaiset olivat iskeneet häneen Raamassa hänen taistellessaan Hasaelia, Aramin kuningasta, vastaan. Ja Ahasja, Jooramin poika, Juudan kuningas, tuli Jisreeliin katsomaan Jooramia, Ahabin poikaa, koska tämä oli sairas.

\chapter{9}

\par 1 Profeetta Elisa kutsui erään profeetanoppilaan ja sanoi hänelle: "Vyötä kupeesi ja ota tämä öljyastia mukaasi ja mene Gileadin Raamotiin.
\par 2 Ja tultuasi sinne hae sieltä käsiisi Jeehu, Joosafatin poika, Nimsin pojanpoika; mene sisään, käske hänen nousta toveriensa keskeltä ja vie hänet sisimpään huoneeseen.
\par 3 Ota sitten öljyastia, vuodata öljyä hänen päähänsä ja sano: 'Näin sanoo Herra: Minä olen voidellut sinut Israelin kuninkaaksi'. Avaa sitten ovi ja pakene viivyttelemättä."
\par 4 Niin nuorukainen, profeetan palvelija, meni Gileadin Raamotiin.
\par 5 Ja kun hän tuli sinne, niin katso: sotaväen päälliköt istuivat siellä. Niin hän sanoi: "Minulla on asiaa sinulle, päällikkö". Jeehu kysyi: "Kenelle meistä kaikista?" Hän vastasi: "Sinulle itsellesi, päällikkö".
\par 6 Silloin tämä nousi ja meni sisälle huoneeseen; ja hän vuodatti öljyä tämän päähän ja sanoi hänelle: "Näin sanoo Herra, Israelin Jumala: Minä olen voidellut sinut Herran kansan, Israelin, kuninkaaksi.
\par 7 Ja sinä olet surmaava herrasi Ahabin suvun; sillä minä kostan Iisebelille palvelijaini, profeettain, veren ja kaikkien Herran palvelijain veren.
\par 8 Ja koko Ahabin suku on hukkuva: minä hävitän Israelista Ahabin miespuoliset jälkeläiset, kaikki tyynni.
\par 9 Ja minä teen Ahabin suvulle saman, minkä Jerobeamin, Nebatin pojan, suvulle ja saman, minkä Baesan, Ahian pojan, suvulle.
\par 10 Ja koirat syövät Iisebelin Jisreelin vainiolla, eikä kukaan ole hautaava häntä." Sitten hän avasi oven ja pakeni.
\par 11 Kun Jeehu tuli ulos herransa palvelijain luo, kysyttiin häneltä: "Onko kaikki hyvin? Miksi tuo hullu kävi sinun luonasi?" Hän vastasi heille: "Tehän tunnette sen miehen ja hänen puheensa".
\par 12 Mutta he sanoivat: "Valhetta! Sano meille totuus." Hän sanoi: "Niin ja niin hän puhui minulle ja sanoi: 'Näin sanoo Herra: Minä olen voidellut sinut Israelin kuninkaaksi'".
\par 13 Silloin he ottivat kiiruusti kukin vaatteensa ja panivat ne hänen allensa paljaille portaille; ja he puhalsivat pasunaan ja huusivat: "Jeehu on tullut kuninkaaksi".
\par 14 Niin Jeehu, Joosafatin poika, Nimsin pojanpoika, teki salaliiton Jooramia vastaan. Jooram oli ollut kaiken Israelin kanssa puolustamassa Gileadin Raamotia Hasaelia, Aramin kuningasta, vastaan.
\par 15 Mutta kuningas Jooram oli tullut takaisin parantuakseen Jisreelissä haavoista, joita aramilaiset olivat iskeneet häneen hänen taistellessaan Hasaelia, Aramin kuningasta, vastaan. Ja Jeehu sanoi: "Jos se teille kelpaa, niin älkää päästäkö pakoon kaupungista ketään, joka menisi ilmoittamaan tätä Jisreeliin".
\par 16 Sitten Jeehu nousi vaunuihinsa ja lähti Jisreeliin, sillä Jooram makasi siellä; ja Ahasja, Juudan kuningas, oli tullut sinne katsomaan Jooramia.
\par 17 Mutta tähystäjä seisoi Jisreelin tornissa, ja kun hän näki Jeehun joukon tulevan, sanoi hän: "Minä näen väkijoukon". Niin Jooram sanoi: "Otettakoon ratsumies ja lähetettäköön heitä vastaan kysymään, onko kaikki hyvin".
\par 18 Ja ratsumies lähti häntä vastaan ja sanoi: "Näin kysyy kuningas: 'Onko kaikki hyvin?'" Jeehu vastasi: "Mitä se sinua liikuttaa, onko hyvin? Käänny ja seuraa minua." Niin tähystäjä ilmoitti sanoen: "Sanansaattaja on tullut heidän luoksensa, mutta ei palaa".
\par 19 Silloin hän lähetti toisen ratsumiehen. Kun tämä oli tullut heidän luoksensa, sanoi hän: "Näin kysyy kuningas: 'Onko kaikki hyvin?' Jeehu vastasi: "Mitä se sinua liikuttaa, onko hyvin? Käänny ja seuraa minua."
\par 20 Niin tähystäjä ilmoitti sanoen: "Hän on tullut heidän luoksensa, mutta ei palaa. Ja ajo on aivan kuin Jeehun, Nimsin pojan, ajoa; sillä hän ajaa niinkuin hurja."
\par 21 Silloin Jooram sanoi: "Valjastettakoon!" Ja he valjastivat hänen sotavaununsa. Niin Jooram, Israelin kuningas, ja Ahasja, Juudan kuningas, lähtivät kumpikin sotavaunuissansa ja menivät Jeehua vastaan. He kohtasivat hänet jisreeliläisen Naabotin maapalstalla.
\par 22 Kun Jooram näki Jeehun, kysyi hän: "Onko kaikki hyvin, Jeehu?" Tämä vastasi: "Kuinka kaikki voisi olla hyvin, kun sinun äitisi Iisebelin haureus ja hänen monet velhoutensa yhä vielä jatkuvat?"
\par 23 Silloin Jooram käänsi vaununsa ja pakeni, huutaen Ahasjalle: "Kavallus, Ahasja!"
\par 24 Mutta Jeehu sieppasi jousen käteensä ja ampui Jooramia hartioiden väliin, niin että nuoli meni sydämen lävitse, ja hän vaipui vaunuihinsa.
\par 25 Sitten hän sanoi Bidkarille, vaunusoturille: "Ota hänet ja heitä jisreeliläisen Naabotin maapalstalle; sillä muistathan, että Herra meidän ratsastaessamme rinnakkain hänen isänsä Ahabin jäljessä lausui hänestä tämän ennustuksen:
\par 26 'Totisesti, minä näin eilen Naabotin ja hänen poikiensa veren, sanoo Herra; ja minä olen juuri tällä maapalstalla kostava sinulle, sanoo Herra'. Ota siis hänet ja heitä tälle maapalstalle, Herran sanan mukaan."
\par 27 Kun Ahasja, Juudan kuningas, sen näki, pakeni hän Beet-Gaaniin päin. Mutta Jeehu ajoi häntä takaa ja sanoi: "Ampukaa hänetkin vaunuihinsa". Ja he ampuivat häntä Guurin solassa, joka on Jibleamin luona; mutta hän pakeni Megiddoon ja kuoli siellä.
\par 28 Ja hänen palvelijansa veivät hänet vaunuissa Jerusalemiin ja hautasivat hänet hänen omaan hautaansa, hänen isiensä viereen, Daavidin kaupunkiin.
\par 29 Ahasja oli tullut Juudan kuninkaaksi Jooramin, Ahabin pojan, yhdentenätoista hallitusvuotena.
\par 30 Sitten Jeehu tuli Jisreeliin. Kun Iisebel kuuli sen, maalasi hän ihomaalilla silmäluomensa, koristeli päänsä ja katseli ulos akkunasta.
\par 31 Ja kun Jeehu tuli portista sisään, kysyi Iisebel: "Kävikö hyvin Simrille, herransa murhaajalle?"
\par 32 Hän käänsi kasvonsa akkunaa kohti ja sanoi: "Kuka on minun puolellani? Kuka?" Niin pari kolme hoviherraa katsoi alas häneen.
\par 33 Jeehu sanoi: "Syöskää hänet alas". Ja he syöksivät hänet alas, niin että hänen vertansa pirskui seinään ja hevosiin; ja nämä tallasivat hänet jalkoihinsa.
\par 34 Sitten Jeehu meni sisään, söi ja joi. Ja hän sanoi: "Korjatkaa se kirottu ja haudatkaa hänet, sillä onhan hän kuninkaan tytär".
\par 35 Mutta kun he menivät hautaamaan häntä, eivät he löytäneet hänestä muuta kuin pääkallon, jalat ja kädet.
\par 36 Ja he tulivat takaisin ja ilmoittivat sen Jeehulle. Niin hän sanoi: "Tässä on toteutunut Herran sana, jonka hän puhui palvelijansa tisbeläisen Elian kautta: 'Jisreelin vainiolla koirat syövät Iisebelin lihan;
\par 37 ja Iisebelin ruumis on oleva niinkuin pellon lanta Jisreelin vainiolla, niin ettei voida sanoa: Tämä on Iisebel'".

\chapter{10}

\par 1 Mutta Ahabilla oli seitsemänkymmentä poikaa Samariassa. Ja Jeehu kirjoitti kirjeet ja lähetti ne Samariaan, Jisreelin päämiehille, vanhimmille ja Ahabin poikien holhoojille; hän kirjoitti näin:
\par 2 "Kun tämä kirje tulee teille, joiden hallussa herranne pojat ovat ja joiden hallussa ovat sotavaunut, hevoset, varustettu kaupunki ja aseet,
\par 3 niin valitkaa herranne pojista paras ja oikeamielisin ja asettakaa hänet isänsä valtaistuimelle ja sotikaa herranne suvun puolesta".
\par 4 Mutta he peljästyivät kovin ja sanoivat: "Katso, ne kaksi kuningasta eivät kestäneet hänen edessään, kuinka me kestäisimme?"
\par 5 Niin linnan päällikkö ja kaupungin päällikkö sekä vanhimmat ja holhoojat lähettivät Jeehulle sanan: "Me olemme sinun palvelijoitasi; me teemme kaiken, mitä sinä meille määräät. Emme me tee ketään kuninkaaksi; tee, mitä tahdot."
\par 6 Silloin hän kirjoitti heille toisen kirjeen, näin kuuluvan: "Jos te olette minun puolellani ja kuulette minua, niin ottakaa herranne poikien päät ja tulkaa huomenna tähän aikaan minun luokseni Jisreeliin". Ne seitsemänkymmentä kuninkaan poikaa asuivat näet kaupungin ylimysten luona, jotka heitä kasvattivat.
\par 7 Kun kirje tuli heille, ottivat he kuninkaan pojat ja tappoivat heidät, seitsemänkymmentä miestä, panivat heidän päänsä koreihin ja lähettivät ne Jeehulle Jisreeliin.
\par 8 Ja sanansaattaja tuli ja ilmoitti hänelle sanoen: "He ovat tuoneet kuninkaan poikien päät". Hän sanoi: "Pankaa ne kahteen roukkioon portin oven edustalle huomiseen saakka".
\par 9 Ja aamulla hän meni ulos, asettui siihen ja sanoi kaikelle kansalle: "Te olette syyttömät. Katso, minä olen tehnyt salaliiton herraani vastaan ja tappanut hänet; mutta kuka on surmannut kaikki nämä?
\par 10 Tietäkää siis, ettei ainoakaan Herran sana, jonka Herra on puhunut Ahabin sukua vastaan, varise maahan. Herra on tehnyt, minkä hän on puhunut palvelijansa Elian kautta."
\par 11 Sitten Jeehu surmasi kaikki, jotka olivat Ahabin suvusta jäljellä Jisreelissä, sekä kaikki hänen ylimyksensä, uskottunsa ja pappinsa, päästämättä pakoon ainoatakaan.
\par 12 Sitten hän nousi ja lähti Samariaan; mutta tullessaan paimenten Beet-Eekediin, joka on tien varrella,
\par 13 Jeehu kohtasi Ahasjan, Juudan kuninkaan, veljet ja kysyi: "Keitä te olette?" He vastasivat: "Me olemme Ahasjan veljiä ja menemme tervehtimään kuninkaan poikia ja kuninkaan äidin poikia".
\par 14 Hän sanoi: "Ottakaa nämä elävinä kiinni". Ja he ottivat heidät elävinä kiinni ja tappoivat heidät ja heittivät Beet-Eekedin vesisäiliöön, neljäkymmentä kaksi miestä. Hän ei jättänyt heistä eloon ainoatakaan.
\par 15 Kun hän sitten lähti sieltä, kohtasi hän Joonadabin, Reekabin pojan, joka tuli häntä vastaan. Ja hän tervehti häntä ja sanoi hänelle: "Onko sinun sydämesi yhtä vilpitön minua kohtaan, kuin minun sydämeni on sinua kohtaan?" Joonadab vastasi: "On". - "Jos niin on, niin lyö kättä minun kanssani." Ja hän löi kättä, ja hän otti hänet vaunuihinsa.
\par 16 Ja hän sanoi: "Tule minun kanssani katsomaan minun kiivailuani Herran puolesta". Niin hän sai ajaa hänen vaunuissansa.
\par 17 Tultuaan Samariaan hän surmasi kaikki, jotka olivat Ahabin jälkeläisistä jäljellä Samariassa, kunnes hän oli hävittänyt hänen sukunsa, Herran sanan mukaan, jonka Herra oli puhunut Elialle.
\par 18 Sitten Jeehu kokosi kaiken kansan ja sanoi heille: "Ahab on palvellut Baalia vähän; Jeehu on palveleva häntä paljon.
\par 19 Kutsukaa nyt minun luokseni kaikki Baalin profeetat, kaikki hänen palvelijansa ja kaikki hänen pappinsa, älköönkä kukaan jääkö pois; sillä minä aion uhrata suuret uhrit Baalille. Eloon ei jää kukaan, joka jää pois." Mutta Jeehu menetteli kavalasti, tuhotakseen Baalin palvelijat.
\par 20 Niin Jeehu sanoi: "Kuuluttakaa pyhä juhlakokous Baalin kunniaksi". Ja se kutsuttiin koolle.
\par 21 Ja Jeehu lähetti sanan kaikkeen Israeliin, ja kaikki Baalin palvelijat tulivat; ei yksikään jäänyt tulematta. Niin he menivät Baalin temppeliin, ja Baalin temppeli täyttyi ääriään myöten.
\par 22 Ja hän sanoi vaatekammion hoitajalle: "Tuo puvut kaikille Baalin palvelijoille". Ja tämä toi heille puvut.
\par 23 Kun Jeehu Joonadabin, Reekabin pojan, kanssa tuli Baalin temppeliin, sanoi hän Baalin palvelijoille: "Tutkikaa ja katsokaa, ettei täällä teidän joukossanne ole ketään Herran palvelijaa, vaan ainoastaan Baalin palvelijoita".
\par 24 Sitten he menivät uhraamaan teurasuhreja ja polttouhreja. Mutta Jeehu oli asettanut ulkopuolelle kahdeksankymmentä miestä ja sanonut: "Jos kuka päästää pakoon yhdenkään niistä miehistä, jotka minä tuon teidän käsiinne, menee henki hengestä".
\par 25 Ja päätettyänsä polttouhrin uhraamisen sanoi Jeehu henkivartijoille ja vaunusotureille: "Menkää sisään ja surmatkaa heidät; älköön yksikään pääskö ulos". Niin he surmasivat heidät miekan terällä, ja henkivartijat ja vaunusoturit heittivät ulos heidän ruumiinsa. Sitten he menivät Baalin temppelilinnaan
\par 26 ja toivat ulos Baalin temppelin patsaat ja polttivat ne,
\par 27 ja he kukistivat Baalin patsaan. He hävittivät Baalin temppelin ja tekivät siitä käymälöitä; niin aina tähän päivään asti.
\par 28 Näin Jeehu hävitti Baalin Israelista.
\par 29 Mutta Jerobeamin, Nebatin pojan, synneistä, joilla hän oli saattanut Israelin tekemään syntiä, Beetelissä ja Daanissa olevista kultavasikoista, Jeehu ei luopunut.
\par 30 Niin Herra sanoi Jeehulle: "Koska olet hyvin toimittanut sen, mikä oli oikein minun silmissäni, ja koska teit Ahabin suvulle aivan minun mieleni mukaan, niin istukoot sinun poikasi Israelin valtaistuimella neljänteen polveen".
\par 31 Mutta Jeehu ei vaeltanut tarkoin, kaikesta sydämestänsä, Herran, Israelin Jumalan, lain mukaan; hän ei luopunut Jerobeamin synneistä, joilla tämä oli saattanut Israelin tekemään syntiä.
\par 32 Niihin aikoihin Herra rupesi lohkomaan Israelia, sillä Hasael voitti heidät kaikkialla Israelin raja-alueella
\par 33 ja valloitti Jordanista auringonnousuun päin koko Gileadin maan, gaadilaiset, ruubenilaiset ja manasselaiset, alkaen Aroerista, joka on Arnon-joen rannalla - sekä Gileadin että Baasanin.
\par 34 Mitä muuta on kerrottavaa Jeehusta ja kaikesta, mitä hän teki, ja kaikista hänen urotöistänsä, se on kirjoitettuna Israelin kuningasten aikakirjassa.
\par 35 Sitten Jeehu meni lepoon isiensä tykö, ja hänet haudattiin Samariaan. Ja hänen poikansa Jooahas tuli kuninkaaksi hänen sijaansa.
\par 36 Ja aika, jonka Jeehu hallitsi Israelia Samariassa, oli kaksikymmentä kahdeksan vuotta.

\chapter{11}

\par 1 Kun Atalja, Ahasjan äiti, näki, että hänen poikansa oli kuollut, nousi hän ja tuhosi koko kuningassuvun.
\par 2 Mutta kuningas Jooramin tytär Jooseba, Ahasjan sisar, otti surmattavien kuninkaan poikien joukosta Ahasjan pojan Jooaan ja vei hänet ja hänen imettäjänsä salaa makuuhuoneeseen. Täällä pidettiin Jooasta kätkettynä Ataljalta, niin ettei hän tullut surmatuksi.
\par 3 Sitten poika oli Jooseban luona Herran temppeliin piilotettuna kuusi vuotta, Ataljan hallitessa maata.
\par 4 Mutta seitsemäntenä vuotena Joojada lähetti hakemaan kaarilaisten ja henkivartijain sadanpäämiehiä ja tuotti heidät luoksensa Herran temppeliin. Ja sittenkuin hän oli tehnyt liiton heidän kanssansa ja vannottanut heidät siellä Herran temppelissä, näytti hän heille kuninkaan pojan.
\par 5 Sitten hän käski heitä sanoen: "Tehkää näin: kolmas osa teistä, joiden on mentävä vartionpitoon sapattina, vartioikoon kuninkaan palatsia,
\par 6 kolmas osa olkoon sivuportilla ja kolmas osa henkivartijain takana olevalla portilla; vartioikaa palatsia, kukin vuorollanne.
\par 7 Mutta kaksi muuta teidän osastoanne, kaikki, jotka pääsevät vartionpidosta sapattina, vartioikoot Herran temppeliä, kuninkaan luona.
\par 8 Asettukaa kuninkaan ympärille, kullakin ase kädessä; ja joka tunkeutuu rivien läpi, se surmattakoon. Näin olkaa kuninkaan luona, menköön hän ulos tai sisään."
\par 9 Sadanpäämiehet tekivät, aivan niinkuin pappi Joojada oli heitä käskenyt: kukin heistä otti miehensä, sekä ne, joiden oli mentävä vartionpitoon sapattina, että ne, jotka pääsivät vartionpidosta sapattina, ja he tulivat pappi Joojadan luo.
\par 10 Ja pappi antoi sadanpäämiehille keihäät ja varustukset, jotka olivat olleet kuningas Daavidin omat ja olivat Herran temppelissä.
\par 11 Ja henkivartijat asettuivat, kullakin ase kädessä, temppelin eteläsivulta aina sen pohjoissivulle saakka, päin alttaria ja temppeliä, kuninkaan ympärille.
\par 12 Sitten hän toi kuninkaan pojan esille, pani hänen päähänsä kruunun ja antoi hänelle lain kirjan, ja he tekivät hänet kuninkaaksi ja voitelivat hänet; ja he paukuttivat käsiänsä ja huusivat: "Eläköön kuningas!"
\par 13 Kun Atalja kuuli henkivartijain ja kansan huudon, meni hän kansan luo Herran temppeliin.
\par 14 Hän näki kuninkaan seisovan pylvään vieressä, niinkuin tapa oli, ja päälliköt ja torvensoittajat kuninkaan luona, ja kaiken maan kansan, joka riemuitsi ja puhalsi torviin. Silloin Atalja repäisi vaatteensa ja huusi: "Kapina! Kapina!"
\par 15 Mutta pappi Joojada käski sadanpäämiehiä, sotajoukon johtajia, ja sanoi heille: "Viekää hänet pois rivien välitse, ja joka yrittää seurata häntä, se surmatkaa miekalla". Sillä pappi oli sanonut: "Älköön häntä surmattako Herran temppelissä".
\par 16 Niin he kävivät häneen käsiksi, ja kun hän oli tullut tielle, jota myöten hevosia kuljetettiin kuninkaan palatsiin, surmattiin hänet siellä.
\par 17 Ja Joojada teki liiton Herran, kuninkaan ja kansan kesken, että he olisivat Herran kansa; niin myös kuninkaan ja kansan kesken.
\par 18 Sitten kaikki maan kansa meni Baalin temppeliin, ja he hävittivät sen. Sen alttarit ja kuvat he löivät aivan murskaksi ja tappoivat alttarien edessä Mattanin, Baalin papin. Sitten pappi asetti vartijat vartioimaan Herran temppeliä.
\par 19 Ja hän otti mukaansa sadanpäämiehet sekä kaarilaiset ja henkivartijat ja kaiken maan kansan, ja he veivät kuninkaan Herran temppelistä ja tulivat henkivartijain portin kautta kuninkaan linnaan, ja hän istui kuninkaalliselle valtaistuimelle.
\par 20 Ja kaikki maan kansa iloitsi, ja kaupunki pysyi rauhallisena. Mutta Ataljan he surmasivat miekalla kuninkaan linnassa.
\par 21 Jooas oli seitsemän vuoden vanha tullessaan kuninkaaksi.

\chapter{12}

\par 1 Jeehun seitsemäntenä hallitusvuotena tuli Jooas kuninkaaksi, ja hän hallitsi Jerusalemissa neljäkymmentä vuotta. Hänen äitinsä oli nimeltään Sibja Beersebasta.
\par 2 Ja Jooas teki sitä, mikä on oikein Herran silmissä, niin kauan kuin hän eli, sillä pappi Joojada oli opettanut häntä.
\par 3 Kuitenkaan eivät uhrikukkulat hävinneet, vaan kansa uhrasi ja suitsutti yhä edelleen uhrikukkuloilla.
\par 4 Ja Jooas sanoi papeille: "Kaiken rahan, joka pyhinä lahjoina tuodaan Herran temppeliin, käyvän rahan, kaiken arvion mukaan suoritettavan henkilörahan ja kaiken rahan, jonka joku sydämensä vaatimuksesta tuo Herran temppeliin,
\par 5 sen ottakoot papit itselleen, kukin tuttavaltaan; mutta heidän on niillä korjattava, mitä on rappeutunutta Herran temppelissä, missä vain jotakin rappeutunutta on".
\par 6 Mutta vielä kuningas Jooaan kahdentenakymmenentenä kolmantena hallitusvuotena eivät papit olleet korjanneet mitään, mikä temppelissä oli rappeutunutta.
\par 7 Silloin kuningas Jooas kutsui pappi Joojadan ja muut papit ja sanoi heille: "Miksi te ette ole korjanneet mitään, mikä temppelissä on rappeutunutta? Nyt te ette enää saa ottaa rahaa tuttaviltanne, vaan teidän on annettava se siihen, mikä temppelissä on rappeutunutta."
\par 8 Ja papit suostuivat siihen, etteivät ottaisi rahaa kansalta, mutta eivät myöskään korjaisi sitä, mikä temppelissä oli rappeutunutta.
\par 9 Mutta pappi Joojada otti arkun ja kaivoi reiän sen kanteen ja asetti sen alttarin ääreen, oikealle puolelle, kun mennään Herran temppeliin. Ja papit, jotka vartioivat ovea, panivat siihen kaiken rahan, mikä Herran temppeliin tuotiin.
\par 10 Kun he näkivät, että arkussa oli paljon rahaa, meni kuninkaan kirjuri sinne ylimmäisen papin kanssa, ja he sitoivat yhteen ja laskivat rahat, jotka olivat Herran temppelissä.
\par 11 Sitten annettiin punnitut rahat työnteettäjille, jotka oli pantu valvomaan töitä Herran temppelissä; ja he maksoivat niillä puusepät ja rakentajat, jotka tekivät työtä Herran temppelissä,
\par 12 ynnä muurarit ja kivenhakkaajat, niin myös puutavarat ja hakatut kivet, jotka oli ostettava sen korjaamiseksi, mikä Herran temppelissä oli rappeutunutta, kaikki temppelin korjausmenot.
\par 13 Ei kuitenkaan teetetty Herran temppeliin hopeavateja, veitsiä, maljoja, torvia, ei mitään kulta- tai hopeakaluja, rahalla, joka tuotiin Herran temppeliin,
\par 14 vaan se annettiin työmiehille, että he sillä korjaisivat Herran temppeliä.
\par 15 Eikä miehiltä, joille rahat luovutettiin työmiehille annettaviksi, vaadittu tilintekoa, vaan he toimivat luottamusmiehinä.
\par 16 Vikauhri- ja syntiuhrirahoja ei tuotu Herran temppeliin; ne tulivat papeille.
\par 17 Siihen aikaan Hasael, Aramin kuningas, tuli ja ryhtyi sotimaan Gatia vastaan, ja hän valloitti sen. Sitten Hasael kävi hyökkäämään Jerusalemia vastaan.
\par 18 Silloin Jooas, Juudan kuningas, otti kaikki pyhät lahjat, jotka hänen isänsä, Juudan kuninkaat Joosafat, Jooram ja Ahasja, olivat pyhittäneet, ja omat pyhät lahjansa sekä kaiken kullan, mitä oli Herran temppelin aarrekammioissa ja kuninkaan linnassa, ja lähetti ne Hasaelille, Aramin kuninkaalle. Niin tämä lähti pois Jerusalemin kimpusta.
\par 19 Mitä muuta on kerrottavaa Jooaasta ja kaikesta, mitä hän teki, se on kirjoitettuna Juudan kuningasten aikakirjassa.
\par 20 Mutta hänen palvelijansa nousivat ja tekivät salaliiton ja surmasivat Jooaan Millo-rakennuksessa, siinä, mistä mennään alas Sillaan.
\par 21 Hänen palvelijansa Joosakar, Simeatin poika, ja Joosabad, Soomerin poika, löivät hänet kuoliaaksi, ja hänet haudattiin isiensä viereen Daavidin kaupunkiin. Ja hänen poikansa Amasja tuli kuninkaaksi hänen sijaansa.

\chapter{13}

\par 1 Juudan kuninkaan Jooaan, Ahasjan pojan, kahdentenakymmenentenä kolmantena hallitusvuotena tuli Jooahas, Jeehun poika, Israelin kuninkaaksi, ja hän hallitsi Samariassa seitsemäntoista vuotta.
\par 2 Ja hän teki sitä, mikä on pahaa Herran silmissä, ja vaelsi Jerobeamin, Nebatin pojan, synneissä, joilla tämä oli saattanut Israelin tekemään syntiä; hän ei luopunut niistä.
\par 3 Niin Herran viha syttyi Israelia kohtaan, ja hän antoi heidät Hasaelin, Aramin kuninkaan, ja Benhadadin, Hasaelin pojan, käsiin koko siksi ajaksi.
\par 4 Mutta Jooahas lepytti Herraa, ja Herra kuuli häntä, sillä hän näki Israelin sorron, kun Aramin kuningas sorti heitä.
\par 5 Ja Herra antoi Israelille vapauttajan, ja he pääsivät aramilaisten käsistä. Ja israelilaiset asuivat majoissansa niinkuin ennenkin.
\par 6 Eivät he kuitenkaan luopuneet Jerobeamin suvun synneistä, joilla hän oli saattanut Israelin tekemään syntiä, vaan he vaelsivat niissä; aserakin jäi paikoilleen Samariaan.
\par 7 Niin ei Herra jättänyt Jooahaalle väkeä enempää kuin viisikymmentä ratsumiestä, kymmenet sotavaunut ja kymmenentuhatta jalkamiestä; sillä Aramin kuningas oli hävittänyt ne ja pannut ne tomuksi, niinkuin puitaessa tulee tomua.
\par 8 Mitä muuta on kerrottavaa Jooahaasta, kaikesta, mitä hän teki, ja hänen urotöistänsä, se on kirjoitettuna Israelin kuningasten aikakirjassa.
\par 9 Ja Jooahas meni lepoon isiensä tykö, ja hänet haudattiin Samariaan. Ja hänen poikansa Jooas tuli kuninkaaksi hänen sijaansa.
\par 10 Juudan kuninkaan Jooaan kolmantenakymmenentenä seitsemäntenä hallitusvuotena tuli Jooas, Jooahaan poika, Israelin kuninkaaksi, ja hän hallitsi Samariassa kuusitoista vuotta.
\par 11 Hän teki sitä, mikä on pahaa Herran silmissä: hän ei luopunut mistään Jerobeamin, Nebatin pojan, synneistä, joilla tämä oli saattanut Israelin tekemään syntiä, vaan vaelsi niissä.
\par 12 Mitä muuta on kerrottavaa Jooaasta, kaikesta, mitä hän teki, ja hänen urotöistään, kuinka hän soti Juudan kuningasta Amasjaa vastaan, se on kirjoitettuna Israelin kuningasten aikakirjassa.
\par 13 Ja Jooas meni lepoon isiensä tykö, ja Jerobeam istui hänen valtaistuimellensa. Ja Jooas haudattiin Samariaan Israelin kuningasten viereen.
\par 14 Mutta kun Elisa sairasti kuolintautiansa, tuli Jooas, Israelin kuningas, hänen tykönsä. Ja kumartuneena hänen kasvojensa yli hän itki ja sanoi: "Isäni, isäni! Israelin sotavaunut ja ratsumiehet!"
\par 15 Niin Elisa sanoi hänelle: "Nouda jousi ja nuolia". Ja hän nouti hänelle jousen ja nuolia.
\par 16 Sitten hän sanoi Israelin kuninkaalle: "Laske kätesi jouselle". Ja kun hän oli laskenut kätensä, pani Elisa kätensä kuninkaan kätten päälle.
\par 17 Ja hän sanoi: "Avaa ikkuna itään päin". Ja kun tämä oli avannut sen, sanoi Elisa: "Ammu". Ja hän ampui. Niin hän sanoi: "Herran voitonnuoli, voitonnuoli Aramia vastaan! Sinä olet voittava aramilaiset Afekissa perinpohjin."
\par 18 Sitten hän sanoi: "Ota nuolet". Ja kun hän oli ne ottanut, sanoi hän Israelin kuninkaalle: "Lyö maahan". Niin hän löi kolme kertaa ja lakkasi sitten.
\par 19 Mutta Jumalan mies vihastui häneen ja sanoi: "Sinun olisi pitänyt lyödä viisi tai kuusi kertaa: silloin olisit voittanut aramilaiset perinpohjin. Mutta nyt olet voittava aramilaiset ainoastaan kolme kertaa."
\par 20 Sitten Elisa kuoli, ja hänet haudattiin. Ja mooabilaisten partiojoukkoja tuli maahan vuosi vuodelta.
\par 21 Ja kerran, kun he haudatessaan erästä miestä näkivät partiojoukon, heittivät he miehen Elisan hautaan ja menivät matkoihinsa. Kun mies kosketti Elisan luita, virkosi hän ja nousi jaloilleen.
\par 22 Hasael, Aramin kuningas, sorti Israelia, niin kauan kuin Jooahas eli.
\par 23 Mutta Herra oli heille laupias, armahti heitä ja kääntyi heidän puoleensa liiton tähden, jonka hän oli tehnyt Aabrahamin, Iisakin ja Jaakobin kanssa. Sillä hän ei tahtonut tuhota heitä, eikä hän ollut vielä heittänyt heitä pois kasvojensa edestä.
\par 24 Ja Hasael, Aramin kuningas, kuoli, ja hänen poikansa Benhadad tuli kuninkaaksi hänen sijaansa.
\par 25 Ja Jooas, Jooahaan poika, otti Benhadadilta, Hasaelin pojalta, takaisin ne kaupungit, jotka tämä oli asevoimalla ottanut hänen isältänsä Jooahaalta. Kolme kertaa Jooas voitti hänet ja otti takaisin Israelin kaupungit.

\chapter{14}

\par 1 Israelin kuninkaan Jooaan, Jooahaan pojan, toisena hallitusvuotena tuli Amasja, Jooaan poika, Juudan kuninkaaksi.
\par 2 Hän oli kahdenkymmenen viiden vuoden vanha tullessaan kuninkaaksi, ja hän hallitsi Jerusalemissa kaksikymmentä yhdeksän vuotta. Hänen äitinsä oli nimeltään Jooaddin, Jerusalemista.
\par 3 Hän teki sitä, mikä on oikein Herran silmissä, ei kuitenkaan niinkuin hänen isänsä Daavid. Kaikessa hän teki, niinkuin hänen isänsä Jooas oli tehnyt.
\par 4 Kuitenkaan eivät uhrikukkulat hävinneet, vaan kansa uhrasi ja suitsutti yhä edelleen uhrikukkuloilla.
\par 5 Ja kun kuninkuus oli vahvistunut hänen käsissään, surmautti hän ne palvelijansa, jotka olivat murhanneet kuninkaan, hänen isänsä.
\par 6 Mutta murhaajain lapset hän jätti surmaamatta, niinkuin on kirjoitettu Mooseksen lain kirjaan, jossa Herra on käskenyt: "Älköön isiä rangaistako kuolemalla lasten tähden älköönkä lapsia isien tähden; kukin rangaistakoon kuolemalla oman syntinsä tähden".
\par 7 Hän voitti edomilaiset Suolalaaksossa, kymmenentuhatta miestä, ja valloitti Seelan asevoimalla ja antoi sille nimen Jokteel, joka sillä tänäkin päivänä on.
\par 8 Silloin Amasja lähetti sanansaattajat Israelin kuninkaan Jooaan, Jooahaan pojan, Jeehun pojanpojan luo ja käski sanoa hänelle: "Tule, otelkaamme keskenämme".
\par 9 Mutta Jooas, Israelin kuningas, lähetti Amasjalle, Juudan kuninkaalle tämän sanan: "Libanonilla kasvava ohdake lähetti setripuulle, joka kasvoi Libanonilla, sanan: 'Anna tyttäresi vaimoksi minun pojalleni'. Mutta metsän eläimet Libanonilla kulkivat ohdakkeen ylitse ja tallasivat sen maahan.
\par 10 Sinä olet voittanut Edomin, ja siitä olet käynyt ylpeäksi. Tyydy siihen kunniaan ja pysy kotonasi. Minkätähden tuotat onnettomuuden? Sinä kukistut itse ynnä Juuda sinun kanssasi."
\par 11 Mutta Amasja ei totellut. Niin Jooas, Israelin kuningas, lähti liikkeelle, ja he, hän ja Amasja, Juudan kuningas, ottelivat keskenään Beet-Semeksessä, joka on Juudan aluetta.
\par 12 Ja israelilaiset voittivat Juudan miehet, ja nämä pakenivat kukin majallensa.
\par 13 Ja Jooas, Israelin kuningas, otti Juudan kuninkaan Amasjan, Jooaan pojan, Ahasjan pojanpojan, Beet-Semeksessä vangiksi. Kun hän tuli Jerusalemiin, revitti hän Jerusalemin muuria Efraimin portista Kulmaporttiin saakka, neljäsataa kyynärää.
\par 14 Ja hän otti kaiken kullan ja hopean sekä kaikki kalut, mitä Herran temppelissä ja kuninkaan linnan aarrekammioissa oli, ynnä panttivankeja, ja palasi Samariaan.
\par 15 Mitä muuta on kerrottavaa Jooaasta, siitä, mitä hän teki, ja hänen urotöistään, kuinka hän soti Amasjaa, Juudan kuningasta, vastaan, se on kirjoitettuna Israelin kuningasten aikakirjassa.
\par 16 Ja Jooas meni lepoon isiensä tykö, ja hänet haudattiin Samariaan Israelin kuningasten viereen. Ja hänen poikansa Jerobeam tuli kuninkaaksi hänen sijaansa.
\par 17 Mutta Juudan kuningas Amasja, Jooaan poika, eli Israelin kuninkaan Jooaan, Jooahaan pojan, kuoleman jälkeen viisitoista vuotta.
\par 18 Mitä muuta on kerrottavaa Amasjasta, se on kirjoitettuna Juudan kuningasten aikakirjassa.
\par 19 Ja häntä vastaan tehtiin Jerusalemissa salaliitto. Hän pakeni Laakiiseen, mutta Laakiiseen lähetettiin miehiä hänen jälkeensä, ja ne surmasivat hänet siellä.
\par 20 Hänet nostettiin hevosten selkään ja haudattiin Jerusalemiin isiensä viereen Daavidin kaupunkiin.
\par 21 Mutta koko Juudan kansa otti Asarjan, joka oli kuudentoista vuoden vanha, ja teki hänet kuninkaaksi hänen isänsä Amasjan sijaan.
\par 22 Hän linnoitti Eelatin ja palautti sen Juudalle, sittenkuin kuningas oli mennyt lepoon isiensä tykö.
\par 23 Juudan kuninkaan Amasjan, Jooaan pojan, viidentenätoista hallitusvuotena tuli Jerobeam, Jooaan poika, Israelin kuninkaaksi, ja hän hallitsi Samariassa neljäkymmentä yksi vuotta.
\par 24 Hän teki sitä, mikä on pahaa Herran silmissä; hän ei luopunut mistään Jerobeamin, Nebatin pojan, synneistä, joilla tämä oli saattanut Israelin tekemään syntiä.
\par 25 Hän palautti entiselleen Israelin alueen, siitä, mistä mennään Hamatiin, aina Aromaan mereen saakka, sen sanan mukaan, jonka Herra, Israelin Jumala, oli puhunut palvelijansa, profeetta Joonan, Amittain pojan, kautta, joka oli kotoisin Gat-Heeferistä.
\par 26 Sillä Herra oli nähnyt Israelin ylen katkeran kurjuuden, kuinka kaikki tyynni menehtyivät eikä Israelilla ollut auttajaa.
\par 27 Herra ei ollut vielä sanonut, että hän oli pyyhkivä Israelin nimen pois taivaan alta, ja niin hän vapautti heidät Jerobeamin, Jooaan pojan, kautta.
\par 28 Mitä muuta on kerrottavaa Jerobeamista, kaikesta, mitä hän teki, ja hänen urotöistänsä, kuinka hän kävi sotaa ja kuinka hän palautti Israelille Damaskosta ja Hamatista sen, mikä oli ollut Juudan, se on kirjoitettuna Israelin kuningasten aikakirjassa.
\par 29 Ja Jerobeam meni lepoon isiensä, Israelin kuningasten, tykö. Ja hänen poikansa Sakarja tuli kuninkaaksi hänen sijaansa.

\chapter{15}

\par 1 Jerobeamin, Israelin kuninkaan, kahdentenakymmenentenä seitsemäntenä hallitusvuotena tuli Asarja, Amasjan poika, Juudan kuninkaaksi.
\par 2 Hän oli kuudentoista vuoden vanha tullessaan kuninkaaksi ja hallitsi Jerusalemissa viisikymmentä kaksi vuotta. Hänen äitinsä oli nimeltään Jekolja, Jerusalemista.
\par 3 Ja hän teki sitä, mikä on oikein Herran silmissä, aivan niinkuin hänen isänsä Amasja oli tehnyt.
\par 4 Kuitenkaan eivät uhrikukkulat hävinneet, vaan kansa uhrasi ja suitsutti yhä edelleen uhrikukkuloilla.
\par 5 Ja Herra löi kuningasta, niin että hän oli pitalitautinen kuolinpäiväänsä saakka. Hän asui eri talossa, mutta Jootam, kuninkaan poika, oli linnan päällikkönä ja tuomitsi maan kansaa.
\par 6 Mitä muuta on kerrottavaa Asarjasta ja kaikesta, mitä hän teki, se on kirjoitettuna Juudan kuningasten aikakirjassa.
\par 7 Ja Asarja meni lepoon isiensä tykö, ja hänet haudattiin isiensä viereen Daavidin kaupunkiin. Ja hänen poikansa Jootam tuli kuninkaaksi hänen sijaansa.
\par 8 Asarjan, Juudan kuninkaan, kolmantenakymmenentenä kahdeksantena hallitusvuotena tuli Sakarja, Jerobeamin poika, Israelin kuninkaaksi, ja hän hallitsi Samariassa kuusi kuukautta.
\par 9 Ja hän teki sitä, mikä on pahaa Herran silmissä, niinkuin hänen isänsäkin olivat tehneet; hän ei luopunut Jerobeamin, Nebatin pojan, synneistä, joilla tämä oli saattanut Israelin tekemään syntiä.
\par 10 Ja Sallum, Jaabeksen poika, teki salaliiton häntä vastaan ja löi hänet kuoliaaksi kansan nähden ja tuli kuninkaaksi hänen sijaansa.
\par 11 Mitä muuta on kerrottavaa Sakarjasta, katso, se on kirjoitettuna Israelin kuningasten aikakirjassa.
\par 12 Näin toteutui Herran sana, jonka hän oli puhunut Jeehulle sanoen: "Sinun poikasi istukoot Israelin valtaistuimella neljänteen polveen": niin tapahtui.
\par 13 Sallum, Jaabeksen poika, tuli kuninkaaksi Ussian, Juudan kuninkaan, kolmantenakymmenentenä yhdeksäntenä hallitusvuotena, ja hän hallitsi Samariassa kuukauden päivät.
\par 14 Mutta silloin Menahem, Gaadin poika, lähti liikkeelle Tirsasta, tuli Samariaan ja löi Sallumin, Jaabeksen pojan, kuoliaaksi Samariassa. Ja hän tuli kuninkaaksi tämän sijaan.
\par 15 Mitä muuta on kerrottavaa Sallumista ja salaliitosta, jonka hän teki, katso, se on kirjoitettuna Israelin kuningasten aikakirjassa.
\par 16 Siihen aikaan Menahem hävitti Tifsahin ja kaiken, mitä siellä oli, niin myös sen alueen Tirsasta lähtien, koska he eivät avanneet hänelle portteja; hän hävitti sen ja halkaisi kaikki raskaat vaimot.
\par 17 Juudan kuninkaan Asarjan kolmantenakymmenentenä yhdeksäntenä hallitusvuotena tuli Menahem, Gaadin poika, Israelin kuninkaaksi, ja hän hallitsi Samariassa kymmenen vuotta.
\par 18 Hän teki sitä, mikä on pahaa Herran edessä; hän ei luopunut koko elinaikanaan mistään Jerobeamin, Nebatin pojan, synneistä, joilla tämä oli saattanut Israelin tekemään syntiä.
\par 19 Puul, Assurin kuningas, hyökkäsi maahan; ja Menahem antoi Puulille tuhat talenttia hopeata, että tämä auttaisi häntä ja vahvistaisi kuninkuuden hänen käsiinsä.
\par 20 Ja Menahem otti rahat verona Israelilta, kaikilta varakkailta miehiltä, niin että jokainen joutui maksamaan Assurin kuninkaalle viisikymmentä sekeliä hopeata.
\par 21 Niin Assurin kuningas palasi takaisin eikä jäänyt siihen maahan. Mitä muuta on kerrottavaa Menahemista ja kaikesta, mitä hän teki, se on kirjoitettuna Israelin kuningasten aikakirjassa.
\par 22 Ja Menahem meni lepoon isiensä tykö. Ja hänen poikansa Pekahja tuli kuninkaaksi hänen sijaansa.
\par 23 Juudan kuninkaan Asarjan viidentenäkymmenentenä hallitusvuotena tuli Pekahja, Menahemin poika, Israelin kuninkaaksi, ja hän hallitsi Samariassa kaksi vuotta.
\par 24 Hän teki sitä, mikä on pahaa Herran silmissä; hän ei luopunut Jerobeamin, Nebatin pojan, synneistä, joilla tämä oli saattanut Israelin tekemään syntiä.
\par 25 Ja Pekah, Remaljan poika, hänen vaunusoturinsa, teki salaliiton häntä vastaan ja tappoi hänet Samariassa kuninkaan linnan palatsissa, sekä hänet että Argobin ja Arjen, jolloin Pekahilla oli mukanaan viisikymmentä Gileadin miestä. Näin tämä surmasi hänet ja tuli kuninkaaksi hänen sijaansa.
\par 26 Mitä muuta on kerrottavaa Pekahjasta ja kaikesta, mitä hän teki, katso, se on kirjoitettuna Israelin kuningasten aikakirjassa.
\par 27 Juudan kuninkaan Asarjan viidentenäkymmenentenä toisena hallitusvuotena tuli Pekah, Remaljan poika, Israelin kuninkaaksi, ja hän hallitsi Samariassa kaksikymmentä vuotta.
\par 28 Hän teki sitä, mikä on pahaa Herran silmissä; hän ei luopunut Jerobeamin, Nebatin pojan, synneistä, joilla tämä oli saattanut Israelin tekemään syntiä.
\par 29 Israelin kuninkaan Pekahin aikana tuli Tiglat-Pileser, Assurin kuningas, ja valloitti Iijonin, Aabel-Beet-Maakan, Jaanoahin, Kedeksen, Haasorin, Gileadin ja Galilean, koko Naftalin maan, ja vei asukkaat pakkosiirtolaisuuteen Assuriin.
\par 30 Mutta Hoosea, Eelan poika, teki salaliiton Pekahia, Remaljan poikaa, vastaan ja löi hänet kuoliaaksi ja tuli kuninkaaksi hänen sijaansa Jootamin, Ussian pojan, kahdentenakymmenentenä hallitusvuotena.
\par 31 Mitä muuta on kerrottavaa Pekahista ja kaikesta, mitä hän teki, katso, se on kirjoitettuna Israelin kuningasten aikakirjassa.
\par 32 Israelin kuninkaan Pekahin, Remaljan pojan, toisena hallitusvuotena tuli Jootam, Juudan kuninkaan Ussian poika, kuninkaaksi.
\par 33 Hän oli kahdenkymmenen viiden vuoden vanha tullessaan kuninkaaksi ja hallitsi Jerusalemissa kuusitoista vuotta. Hänen äitinsä oli nimeltään Jerusa, Saadokin tytär.
\par 34 Hän teki sitä, mikä on oikein Herran silmissä, aivan niinkuin hänen isänsä Ussia oli tehnyt.
\par 35 Kuitenkaan uhrikukkulat eivät hävinneet, vaan kansa uhrasi ja suitsutti yhä edelleen uhrikukkuloilla. Hän rakennutti Yläportin Herran temppeliin.
\par 36 Mitä muuta on kerrottavaa Jootamista ja kaikesta, mitä hän teki, se on kirjoitettuna Juudan kuningasten aikakirjassa.
\par 37 Niihin aikoihin alkoivat Resin, Aramin kuningas, ja Pekah, Remaljan poika, Herran lähettäminä käydä Juudan kimppuun.
\par 38 Ja Jootam meni lepoon isiensä tykö, ja hänet haudattiin isiensä viereen, isänsä Daavidin kaupunkiin. Ja hänen poikansa Aahas tuli kuninkaaksi hänen sijaansa.

\chapter{16}

\par 1 Pekahin, Remaljan pojan, seitsemäntenätoista hallitusvuotena tuli Aahas, Juudan kuninkaan Jootamin poika, kuninkaaksi.
\par 2 Aahas oli kahdenkymmenen vuoden vanha tullessansa kuninkaaksi ja hallitsi Jerusalemissa kuusitoista vuotta. Hän ei tehnyt sitä, mikä oli oikein Herran, hänen Jumalansa, silmissä, niinkuin hänen isänsä Daavid,
\par 3 vaan vaelsi Israelin kuningasten tietä; jopa hän pani poikansakin kulkemaan tulen läpi, niiden kansain kauhistavien tekojen mukaan, jotka Herra oli karkoittanut israelilaisten tieltä.
\par 4 Ja hän uhrasi ja suitsutti uhrikukkuloilla ja kummuilla ja jokaisen viheriän puun alla.
\par 5 Siihen aikaan Resin, Aramin kuningas, ja Pekah, Remaljan poika, Israelin kuningas, lähtivät Jerusalemiin sotimaan. Ja he saartoivat Aahaan, mutta eivät voineet valloittaa kaupunkia.
\par 6 Samaan aikaan Resin, Aramin kuningas, palautti Eelatin Aramille ja karkoitti Juudan miehet Eelatista. Niin tulivat aramilaiset Eelatiin ja asettuivat sinne, ja siellä heitä on tänäkin päivänä.
\par 7 Mutta Aahas lähetti sanansaattajat Assurin kuninkaan Tiglat-Pileserin luo ja käski sanoa hänelle: "Minä olen sinun palvelijasi ja sinun poikasi. Tule ja pelasta minut Aramin kuninkaan ja Israelin kuninkaan käsistä; he ovat hyökänneet minun kimppuuni."
\par 8 Ja Aahas otti hopean ja kullan, mitä Herran temppelissä ja kuninkaan palatsin aarrekammioissa oli, ja lähetti sen lahjaksi Assurin kuninkaalle.
\par 9 Ja Assurin kuningas kuuli häntä. Niin Assurin kuningas lähti Damaskoa vastaan ja valloitti sen, vei asukkaat pakkosiirtolaisuuteen Kiiriin ja surmasi Resinin.
\par 10 Silloin kuningas Aahas lähti Damaskoon kohtaamaan Assurin kuningasta Tiglat-Pileseriä. Ja kun kuningas Aahas näki Damaskossa olevan alttarin, lähetti hän pappi Uurialle kuvan alttarista ja sen mukaan tehdyn tarkan mallin.
\par 11 Ja pappi Uuria rakensi alttarin; aivan sen määräyksen mukaan, jonka kuningas Aahas oli hänelle lähettänyt Damaskosta; aivan sen mukaan pappi Uuria teki, jo ennenkuin kuningas Aahas oli palannut Damaskosta.
\par 12 Ja kun kuningas palasi Damaskosta ja näki alttarin, astui hän alttarin ääreen ja nousi sen päälle,
\par 13 poltti polttouhrinsa ja ruokauhrinsa, vuodatti juomauhrinsa ja vihmoi uhraamansa uhrin veren alttarille.
\par 14 Mutta vaskialttarin, joka oli Herran edessä, hän siirsi pois temppelin edustalta, alttarin ja Herran temppelin väliltä, ja pani sen alttarin pohjoissivulle.
\par 15 Ja kuningas Aahas käski pappi Uuriaa ja sanoi: "Polta suuremmalla alttarilla aamu-polttouhri ja ehtoo-ruokauhri ja kuninkaan polttouhri sekä hänen ruokauhrinsa, niin myös koko maan kansan polttouhri sekä heidän ruoka- ja juomauhrinsa; ja vihmo kaikki sekä polttouhrin että teurasuhrin veri sen päälle. Mutta vaskialttarin käyttämistä minä vielä mietin."
\par 16 Ja pappi Uuria teki kaiken, mitä kuningas Aahas oli käskenyt.
\par 17 Kuningas Aahas leikkautti irti telineiden kehäpienat ja otti pois altaat niiden päältä; ja meren hän nostatti pois vaskiraavasten päältä, jotka olivat sen alla, ja panetti sen kivialustalle.
\par 18 Ja katetun sapattikäytävän, joka oli rakennettu temppeliin, ja kuninkaan ulkopuolisen sisäänkäytävän hän muutti Herran temppelin sisään Assurin kuninkaan vuoksi.
\par 19 Mitä muuta on vielä kerrottavaa Aahaasta, siitä, mitä hän teki, se on kirjoitettuna Juudan kuningasten aikakirjassa.
\par 20 Ja Aahas meni lepoon isiensä tykö, ja hänet haudattiin isiensä viereen Daavidin kaupunkiin. Ja hänen poikansa Hiskia tuli kuninkaaksi hänen sijaansa.

\chapter{17}

\par 1 Juudan kuninkaan Aahaan kahdentenatoista hallitusvuotena tuli Hoosea, Eelan poika, Israelin kuninkaaksi Samariaan, ja hän hallitsi yhdeksän vuotta.
\par 2 Hän teki sitä, mikä on pahaa Herran silmissä, ei kuitenkaan niinkuin ne Israelin kuninkaat, jotka olivat olleet ennen häntä.
\par 3 Hänen kimppuunsa hyökkäsi Salmaneser, Assurin kuningas; ja Hoosea tuli hänen palvelijakseen ja maksoi hänelle veroa.
\par 4 Mutta Assurin kuningas huomasi Hoosean olevan salahankkeissa, koska tämä lähetti sanansaattajat Soon, Egyptin kuninkaan, luo eikä suorittanut enää niinkuin ennen vuotuista veroa Assurin kuninkaalle. Niin Assurin kuningas vangitutti hänet ja piti häntä sidottuna vankilassa.
\par 5 Ja Assurin kuningas hyökkäsi koko maan kimppuun ja meni Samariaan ja piiritti sitä kolme vuotta.
\par 6 Hoosean yhdeksäntenä hallitusvuotena Assurin kuningas valloitti Samarian ja vei Israelin pakkosiirtolaisuuteen Assuriin ja asetti heidät asumaan Halahiin ja Haaborin, Goosanin joen, rannoille sekä Meedian kaupunkeihin.
\par 7 Näin tapahtui, koska israelilaiset olivat tehneet syntiä Herraa, Jumalaansa, vastaan, joka oli johdattanut heidät Egyptin maasta, pois faraon, Egyptin kuninkaan, käsistä, ja koska he olivat peljänneet muita jumalia.
\par 8 He olivat myös vaeltaneet niiden kansain säädösten mukaan, jotka Herra oli karkoittanut israelilaisten tieltä, ja tehneet sitä, mitä Israelin kuninkaatkin.
\par 9 Israelilaiset olivat tehneet Herraa, Jumalaansa, vastaan sellaista, mikä ei ole oikein: he olivat rakentaneet itsellensä uhrikukkuloita kaikkiin kaupunkeihinsa, sekä vartiotornien luo että varustettuihin kaupunkeihin.
\par 10 He olivat pystyttäneet itsellensä patsaita ja asera-karsikkoja jokaiselle korkealle kummulle ja jokaisen viheriän puun alle.
\par 11 He olivat polttaneet uhreja kaikilla uhrikukkuloilla, niinkuin ne kansat, jotka Herra oli karkoittanut heidän tieltänsä, olivat harjoittaneet pahuutta ja vihoittaneet Herran.
\par 12 He olivat palvelleet kivijumalia, vaikka Herra oli heille sanonut: "Älkää tehkö sitä".
\par 13 Ja Herra oli varoittanut sekä Israelia että Juudaa kaikkien profeettainsa, kaikkien näkijäin, kautta ja sanonut: "Palatkaa pahoilta teiltänne ja noudattakaa minun käskyjäni ja säädöksiäni, kaiken sen lain mukaan, jonka minä annoin teidän isillenne ja jonka minä olen teille ilmoittanut palvelijaini, profeettain, kautta".
\par 14 Mutta he eivät kuulleet, vaan olivat niskureita niinkuin heidän isänsäkin, jotka eivät uskoneet Herraan, Jumalaansa.
\par 15 He hylkäsivät hänen käskynsä ja hänen liittonsa, jonka hän oli tehnyt heidän isiensä kanssa, ja todistukset, jotka hän oli heille antanut. He seurasivat turhia jumalia, ja turhanpäiväisiksi he tulivat, kun seurasivat pakanakansoja, jotka asuivat heidän ympärillänsä, vaikka Herra oli kieltänyt heitä tekemästä niinkuin ne.
\par 16 He hylkäsivät Herran, Jumalansa, kaikki käskyt ja tekivät itsellensä valettuja kuvia, kaksi vasikkaa. Ja he tekivät itsellensä myös asera-karsikkoja ja kumarsivat kaikkea taivaan joukkoa ja palvelivat Baalia.
\par 17 Ja he panivat poikansa ja tyttärensä kulkemaan tulen läpi ja tekivät taikoja ja harjoittivat noituutta. He myivät itsensä tekemään sitä, mikä on pahaa Herran silmissä, ja vihoittivat hänet.
\par 18 Niin Herra vihastui suuresti Israeliin ja poisti heidät kasvojensa edestä, niin ettei muuta jäänyt jäljelle kuin Juudan sukukunta yksin.
\par 19 Ei myöskään Juuda pitänyt Herran, Jumalansa, käskyjä, vaan he vaelsivat niiden säädösten mukaan, jotka Israel itse oli tehnyt.
\par 20 Niin Herra hylkäsi koko Israelin heimon, nöyryytti heitä ja antoi heidät ryöstäjien käsiin ja heitti viimein heidät pois kasvojensa edestä.
\par 21 Sillä kun hän oli reväissyt Israelin Daavidin suvulta ja he olivat tehneet kuninkaaksi Jerobeamin, Nebatin pojan, käänsi Jerobeam Israelin pois Herrasta ja saattoi heidät tekemään suuren synnin.
\par 22 Ja israelilaiset vaelsivat kaikissa synneissä, joita Jerobeam oli tehnyt; he eivät luopuneet niistä.
\par 23 Ja niin Herra viimein poisti Israelin kasvojensa edestä, niinkuin hän oli puhunut kaikkien palvelijainsa, profeettain, kautta. Ja Israel vietiin pois maastansa pakkosiirtolaisuuteen Assuriin, jossa he ovat tänäkin päivänä.
\par 24 Sen jälkeen Assurin kuningas antoi tuoda kansaa Baabelista, Kuutasta, Avvasta, Hamatista ja Sefarvaimista ja asetti heidät asumaan Samarian kaupunkeihin israelilaisten sijaan. Ja he ottivat Samarian omaksensa ja asettuivat sen kaupunkeihin.
\par 25 Ja kun he siellä olonsa ensi aikoina eivät peljänneet Herraa, lähetti Herra heidän sekaansa leijonia, jotka tappoivat heitä.
\par 26 Se ilmoitettiin Assurin kuninkaalle näin: "Kansat, jotka sinä veit pakkosiirtolaisuuteen ja asetit asumaan Samarian kaupunkeihin, eivät tiedä, millä tavalla sen maan Jumalaa on palveltava. Sentähden hän on lähettänyt leijonia heidän sekaansa, ja katso, ne surmaavat heitä, koska he eivät tiedä, millä tavalla sen maan Jumalaa on palveltava."
\par 27 Silloin Assurin kuningas käski sanoen: "Viekää sinne joku niistä papeista, jotka toitte sieltä pakkosiirtolaisuuteen; joku heistä menköön sinne ja asukoon siellä ja opettakoon heille, millä tavalla sen maan Jumalaa on palveltava."
\par 28 Ja niin eräs niistä papeista, jotka he olivat vieneet Samariasta pakkosiirtolaisuuteen, tuli ja asettui Beeteliin; ja hän opetti heille, kuinka heidän oli peljättävä Herraa.
\par 29 Mutta kukin kansa teki itselleen omat jumalansa, ja he panivat ne samarialaisten rakentamiin uhrikukkula-temppeleihin, kukin kansa niissä kaupungeissa, joissa se asui.
\par 30 Baabelin miehet tekivät itselleen Sukkot-Benotin, Kuutin miehet tekivät Neergalin, Hamatin miehet Asiman,
\par 31 avvalaiset tekivät Nibhaan ja Tartakin, sefarvaimilaiset polttivat poikiansa tulessa Adrammelekille ja Annammelekille, Sefarvaimin jumalille.
\par 32 Mutta he pelkäsivät myöskin Herraa. Ja he asettivat itsellensä uhrikukkula-pappeja omasta keskuudestaan, ja nämä uhrasivat heidän puolestaan uhrikukkula-temppeleissä.
\par 33 He pelkäsivät tosin Herraa, mutta palvelivat myös omia jumaliansa samalla tavalla kuin ne kansat, joiden keskuudesta heidät oli tuotu.
\par 34 Ja vielä tänäkin päivänä he tekevät vanhan tapansa mukaan: he eivät pelkää Herraa eivätkä tee saamiensa säädösten ja oikeuksien mukaan, eivät sen lain ja niiden käskyjen mukaan, jotka Herra on antanut Jaakobin pojille, hänen, jolle hän antoi nimen Israel.
\par 35 Herra teki liiton näiden kanssa ja käski heitä sanoen: "Älkää peljätkö muita jumalia, älkää kumartako ja palvelko niitä älkääkä uhratko niille,
\par 36 vaan Herraa, joka johdatti teidät Egyptin maasta suurella voimalla ja ojennetulla käsivarrella, häntä te peljätkää, häntä kumartakaa ja hänelle uhratkaa.
\par 37 Ja niitä säädöksiä ja oikeuksia, sitä lakia ja niitä käskyjä, jotka hän on teille kirjoittanut, te alati tarkoin noudattakaa, mutta älkää peljätkö muita jumalia.
\par 38 Liittoa, jonka minä tein teidän kanssanne, älkää unhottako. Älkääkä peljätkö muita jumalia,
\par 39 vaan peljätkää ainoastaan Herraa, teidän Jumalaanne, niin hän pelastaa teidät kaikkien vihollistenne käsistä."
\par 40 Mutta he eivät totelleet, vaan tekivät vanhan tapansa mukaan.
\par 41 Niin nämä kansat pelkäsivät Herraa, mutta palvelivat samalla omia jumalankuviansa. Myöskin heidän lapsensa ja heidän lastensa lapset tekevät vielä tänä päivänä samoin, kuin heidän isänsä ovat tehneet.

\chapter{18}

\par 1 Israelin kuninkaan Hoosean, Eelan pojan, kolmantena hallitusvuotena tuli Hiskia, Juudan kuninkaan Aahaan poika, kuninkaaksi.
\par 2 Hän oli kahdenkymmenen viiden vuoden vanha tullessansa kuninkaaksi, ja hän hallitsi Jerusalemissa kaksikymmentä yhdeksän vuotta. Hänen äitinsä oli nimeltään Abi, Sakarjan tytär.
\par 3 Hän teki sitä, mikä on oikein Herran silmissä, aivan niinkuin hänen isänsä Daavid oli tehnyt.
\par 4 Hän poisti uhrikukkulat, murskasi patsaat, hakkasi maahan asera-karsikon ja löi palasiksi vaskikäärmeen, jonka Mooses oli tehnyt; sillä niihin aikoihin asti israelilaiset olivat polttaneet uhreja sille; sitä kutsuttiin nimellä Nehustan.
\par 5 Hän turvasi Herraan, Israelin Jumalaan, niin ettei kukaan ollut hänen kaltaisensa kaikista hänen jälkeläisistään Juudan kuninkaista eikä niistä, jotka olivat olleet ennen häntä.
\par 6 Hän riippui Herrassa kiinni eikä luopunut hänestä, vaan noudatti hänen käskyjänsä, jotka Herra oli antanut Moosekselle.
\par 7 Niin Herra oli hänen kanssansa; hän menestyi kaikessa, mihin ryhtyi. Hän kapinoi Assurin kuningasta vastaan eikä enää palvellut häntä.
\par 8 Hän voitti filistealaiset ja valtasi heidän alueensa aina Gassaan asti, sekä vartiotornit että varustetut kaupungit.
\par 9 Mutta kuningas Hiskian neljäntenä hallitusvuotena, joka oli Israelin kuninkaan Hoosean, Eelan pojan, seitsemäs hallitusvuosi, hyökkäsi Salmaneser, Assurin kuningas, Samarian kimppuun ja piiritti sitä.
\par 10 Ja hän valloitti sen kolmen vuoden kuluttua. Hiskian kuudentena hallitusvuotena, joka oli Israelin kuninkaan Hoosean yhdeksäs hallitusvuosi, valloitettiin Samaria.
\par 11 Ja Assurin kuningas vei Israelin pakkosiirtolaisuuteen Assuriin ja sijoitti heidät Halahiin ja Haaborin, Goosanin joen, rannoille ja Meedian kaupunkeihin,
\par 12 sentähden että he eivät olleet kuulleet Herran, Jumalansa, ääntä, vaan olivat rikkoneet hänen liittonsa, kaiken, mitä Mooses, Herran palvelija, oli käskenyt; he eivät olleet kuulleet sitä eivätkä tehneet sen mukaan.
\par 13 Kuningas Hiskian neljäntenätoista hallitusvuotena hyökkäsi Sanherib, Assurin kuningas, kaikkien Juudan varustettujen kaupunkien kimppuun ja valloitti ne.
\par 14 Niin Hiskia, Juudan kuningas, lähetti Assurin kuninkaalle Laakiiseen sanan: "Minä olen rikkonut, käänny pois minun kimpustani. Minä kannan, mitä panet kannettavakseni." Niin Assurin kuningas määräsi Hiskian, Juudan kuninkaan, maksettavaksi kolmesataa talenttia hopeata ja kolmekymmentä talenttia kultaa.
\par 15 Ja Hiskia antoi kaikki rahat, mitä Herran temppelissä ja kuninkaan linnan aarrekammioissa oli.
\par 16 Siihen aikaan Hiskia leikkautti irti Herran temppelin ovista ja pihtipielistä sen päällystyksen, jolla Hiskia, Juudan kuningas, oli päällystänyt ne, ja antoi sen Assurin kuninkaalle.
\par 17 Mutta Assurin kuningas lähetti Laakiista Tartanin ja ylimmäisen hoviherran ja Rabsaken suuren sotajoukon kanssa kuningas Hiskiaa vastaan Jerusalemiin. Ja nämä lähtivät liikkeelle ja tulivat Jerusalemiin; ja lähdettyänsä liikkeelle ja tultuansa he pysähtyivät Ylälammikon vesijohdolle, joka on Vanuttajankedon tien varrella.
\par 18 Kun he sitten kutsuivat kuningasta, menivät Eljakim, Hilkian poika, joka oli linnan päällikkönä, ja kirjuri Sebna ja kansleri Jooah, Aasafin poika, heidän luoksensa.
\par 19 Ja Rabsake sanoi heille: "Sanokaa Hiskialle: Näin sanoo suurkuningas, Assurin kuningas: 'Mitä on tuo luottamus, mikä sinulla on?
\par 20 Arvelet kai, että pelkästä huulten puheesta tulee neuvo ja voima sodankäyntiin. Keneen sinä oikein luotat, kun kapinoit minua vastaan?
\par 21 Katso, sinä luotat nyt Egyptiin, tuohon särkyneeseen ruokosauvaan, joka tunkeutuu sen käteen, joka siihen nojaa, ja lävistää sen. Sellainen on farao, Egyptin kuningas, kaikille, jotka häneen luottavat.
\par 22 Vai sanotteko te ehkä minulle: Me luotamme Herraan, meidän Jumalaamme? Mutta eikö hän ole se, jonka uhrikukkulat ja alttarit Hiskia poisti, kun hän sanoi Juudalle ja Jerusalemille: Tämän alttarin edessä on teidän kumartaen rukoiltava, täällä Jerusalemissa?'
\par 23 Mutta lyö nyt vetoa minun herrani, Assurin kuninkaan, kanssa: minä annan sinulle kaksi tuhatta hevosta, jos sinä voit hankkia niille ratsastajat.
\par 24 Kuinka sinä sitten voisit torjua ainoankaan käskynhaltijan, ainoankaan minun herrani vähimmän palvelijan, hyökkäyksen? Ja sinä vain luotat Egyptiin, sen vaunuihin ja ratsumiehiin.
\par 25 Olenko minä siis Herran sallimatta hyökännyt tämän paikan kimppuun hävittämään sitä? Herra itse on sanonut minulle: 'Hyökkää tähän maahan ja hävitä se'."
\par 26 Niin Eljakim, Hilkian poika, ja Sebna ja Jooah sanoivat Rabsakelle: "Puhu palvelijoillesi araminkieltä, sillä me ymmärrämme sitä; älä puhu meidän kanssamme juudankieltä kansan kuullen, jota on muurilla".
\par 27 Mutta Rabsake vastasi heille: "Onko minun herrani lähettänyt minut puhumaan näitä sanoja sinun herrallesi ja sinulle? Eikö juuri niille miehille, jotka istuvat muurilla ja joutuvat teidän kanssanne syömään omaa likaansa ja juomaan omaa vettänsä?"
\par 28 Sitten Rabsake astui esiin, huusi kovalla äänellä juudankielellä, puhui ja sanoi: "Kuulkaa suurkuninkaan, Assurin kuninkaan, sana.
\par 29 Näin sanoo kuningas: 'Älkää antako Hiskian pettää itseänne, sillä hän ei voi pelastaa teitä minun käsistäni.
\par 30 Älköön Hiskia saako teitä luottamaan Herraan, kun hän sanoo: Herra on varmasti pelastava meidät, eikä tätä kaupunkia anneta Assurin kuninkaan käsiin.
\par 31 Älkää kuulko Hiskiaa.' Sillä Assurin kuningas sanoo näin: 'Tehkää sovinto minun kanssani ja antautukaa minulle, niin te saatte syödä kukin omasta viinipuustanne ja viikunapuustanne ja juoda kukin omasta kaivostanne,
\par 32 kunnes minä tulen ja vien teidät maahan, joka on teidän maanne kaltainen, vilja-, viini- ja leipämaahan, viinitarhojen, jalon öljypuun ja hunajan maahan; ja niin te saatte elää ettekä kuole. Mutta älkää kuulko Hiskiaa; sillä hän viettelee teidät, kun sanoo: Herra pelastaa meidät.
\par 33 Onko muidenkaan kansojen jumalista yksikään pelastanut maatansa Assurin kuninkaan käsistä?
\par 34 Missä ovat Hamatin ja Arpadin jumalat? Missä ovat Sefarvaimin, Heenan ja Ivvan jumalat? Ovatko ne pelastaneet Samariaa minun käsistäni?
\par 35 Onko muiden maiden kaikista jumalista ainoakaan pelastanut maatansa minun käsistäni? Kuinka sitten Herra pelastaisi Jerusalemin minun käsistäni?'"
\par 36 Mutta kansa oli vaiti, eivätkä he vastanneet hänelle mitään; sillä kuningas oli käskenyt niin ja sanonut: "Älkää vastatko hänelle".
\par 37 Sitten palatsin päällikkö Eljakim, Hilkian poika, ja kirjuri Sebna ja kansleri Jooah, Aasafin poika, tulivat Hiskian luo vaatteet reväistyinä ja kertoivat hänelle, mitä Rabsake oli sanonut.

\chapter{19}

\par 1 Kun kuningas Hiskia sen kuuli, repäisi hän vaatteensa, pukeutui säkkiin ja meni Herran temppeliin.
\par 2 Ja hän lähetti palatsin päällikön Eljakimin ja kirjuri Sebnan sekä pappien vanhimmat, säkkeihin puettuina, profeetta Jesajan, Aamoksen pojan, tykö.
\par 3 Ja he sanoivat hänelle: "Näin sanoo Hiskia: 'Hädän, kurituksen ja häväistyksen päivä on tämä päivä, sillä lapset ovat tulleet kohdun suulle saakka, mutta ei ole voimaa synnyttää.
\par 4 Ehkä Herra, sinun Jumalasi, kuulee kaikki Rabsaken sanat, joilla hänen herransa, Assurin kuningas, on lähettänyt hänet herjaamaan elävää Jumalaa, ja rankaisee häntä näistä sanoista, jotka Herra, sinun Jumalasi, on kuullut. Niin kohota nyt rukous jäännöksen puolesta, joka vielä on olemassa.'"
\par 5 Kun kuningas Hiskian palvelijat tulivat Jesajan tykö,
\par 6 sanoi Jesaja heille: "Sanokaa näin herrallenne: 'Näin sanoo Herra: Älä pelkää niitä sanoja, jotka olet kuullut ja joilla Assurin kuninkaan poikaset ovat häväisseet minua.
\par 7 Katso, minä annan häneen mennä sellaisen hengen, että hän kuultuaan sanoman palaa omaan maahansa. Ja minä annan hänen kaatua miekkaan omassa maassansa.'"
\par 8 Ja Rabsake kääntyi takaisin ja tapasi Assurin kuninkaan sotimassa Libnaa vastaan; sillä hän oli kuullut, että tämä oli lähtenyt Laakiista pois.
\par 9 Mutta kun Sanherib kuuli Tirhakasta, Etiopian kuninkaasta, sanottavan: "Katso, hän on lähtenyt liikkeelle sotiakseen sinua vastaan", lähetti hän taas sanansaattajat Hiskian tykö ja käski sanoa:
\par 10 "Sanokaa näin Hiskialle, Juudan kuninkaalle: 'Älä anna Jumalasi, johon sinä luotat, pettää itseäsi äläkä ajattele: Jerusalem ei joudu Assurin kuninkaan käsiin.
\par 11 Olethan kuullut, mitä Assurin kuninkaat ovat tehneet kaikille maille, kuinka he ovat vihkineet ne tuhon omiksi. Ja sinäkö pelastuisit!
\par 12 Ovatko kansain jumalat pelastaneet niitä, jotka minun isäni ovat tuhonneet: Goosania, Harrania, Resefiä ja Telassarin edeniläisiä?
\par 13 Missä on Hamatin kuningas ja Arpadin kuningas, Sefarvaimin kaupungin, Heenan ja Ivvan kuningas?'"
\par 14 Kun Hiskia oli ottanut kirjeen sanansaattajilta ja lukenut sen, meni hän Herran temppeliin; ja Hiskia levitti sen Herran eteen.
\par 15 Ja Hiskia rukoili Herran edessä ja sanoi: "Herra, Israelin Jumala, jonka valtaistuinta kerubit kannattavat, sinä yksin olet maan kaikkien valtakuntain Jumala; sinä olet tehnyt taivaan ja maan.
\par 16 Herra, kallista korvasi ja kuule; Herra, avaa silmäsi ja katso. Kuule Sanheribin sanat, kuinka hän lähetti tuon miehen herjaamaan elävää Jumalaa.
\par 17 Se on totta, Herra, että Assurin kuninkaat ovat hävittäneet kansat ja heidän maansa.
\par 18 Ja he ovat heittäneet heidän jumalansa tuleen; sillä ne eivät olleet jumalia, vaan ihmiskätten tekoa, puuta ja kiveä, ja sentähden he voivat hävittää ne.
\par 19 Mutta pelasta nyt meidät, Herra, meidän Jumalamme, hänen käsistänsä, että kaikki maan valtakunnat tulisivat tietämään, että sinä, Herra, yksin olet Jumala."
\par 20 Niin Jesaja, Aamoksen poika, lähetti Hiskialle tämän sanan: "Näin sanoo Herra, Israelin Jumala: Mitä sinä olet minulta Sanheribin, Assurin kuninkaan, tähden rukoillut, sen minä olen kuullut.
\par 21 Ja tämä on sana, jonka Herra on puhunut hänestä: Neitsyt, tytär Siion, halveksii ja pilkkaa sinua; tytär Jerusalem nyökyttää ilkkuen päätänsä sinun jälkeesi.
\par 22 Ketä olet herjannut ja häväissyt, ja ketä vastaan olet korottanut äänesi? Korkealle olet kohottanut silmäsi Israelin Pyhää vastaan.
\par 23 Sanansaattajaisi kautta sinä herjasit Herraa ja sanoit: 'Monilla vaunuillani minä nousin vuorten harjalle, Libanonin ääriin saakka; minä hakkasin maahan sen korkeat setrit, sen parhaat kypressit, ja tunkeuduin sen etäisimpään yöpaikkaan, sen rehevimpään metsään;
\par 24 minä kaivoin kaivoja ja join kuiviin muukalaisten vedet, ja jalkapohjallani minä kuivasin kaikki Egyptin virrat'.
\par 25 Etkö ole kuullut: kauan sitten minä olen tätä valmistanut, muinaisuudesta saakka tätä aivoitellut! Nyt minä olen sen toteuttanut, ja niin sinä sait hävittää varustetut kaupungit autioiksi kiviroukkioiksi,
\par 26 ja niiden asukkaat olivat voimattomat, he kauhistuivat ja joutuivat häpeään; heidän kävi niinkuin kedon ruohon ja niinkuin vihannan heinän, niinkuin katolla kasvavain kortten ja niinkuin laihon, joka kuivettuu ennen oljelle tulemistaan.
\par 27 Istuitpa sinä tai lähdit tai tulit, minä sen tiedän, niinkuin senkin, että sinä raivoat minua vastaan.
\par 28 Koska sinä minua vastaan raivoat ja koska sinun ylpeytesi on tullut minun korviini, niin minä panen koukkuni sinun nenääsi ja suitseni sinun suuhusi ja vien sinut takaisin samaa tietä, jota tulitkin.
\par 29 Ja tämä on oleva sinulle merkkinä: tänä vuonna syödään jälkikasvua ja toisena vuonna kesanto-aaluvaa, mutta kolmantena vuonna te kylväkää ja leikatkaa, istuttakaa viinitarhoja ja syökää niiden hedelmää.
\par 30 Ja Juudan heimon pelastuneet, jotka ovat jäljelle jääneet, tekevät taas juurta alaspäin ja hedelmää ylöspäin.
\par 31 Sillä Jerusalemista lähtee kasvamaan jäännös, pelastunut joukko Siionin vuorelta. Herran kiivaus on sen tekevä.
\par 32 Sentähden, näin sanoo Herra Assurin kuninkaasta: Hän ei tule tähän kaupunkiin eikä siihen nuolta ammu, ei tuo sitä vastaan kilpeä eikä luo sitä vastaan vallia.
\par 33 Samaa tietä, jota hän tuli, hän palajaa, ja tähän kaupunkiin hän ei tule, sanoo Herra.
\par 34 Sillä minä varjelen tämän kaupungin ja pelastan sen itseni tähden ja palvelijani Daavidin tähden."
\par 35 Ja sinä yönä Herran enkeli lähti ja löi Assurin leirissä sata kahdeksankymmentä viisi tuhatta miestä, ja kun noustiin aamulla varhain, niin katso, ne olivat kaikki kuolleina ruumiina.
\par 36 Silloin Sanherib, Assurin kuningas, lähti liikkeelle ja meni pois; hän palasi takaisin ja jäi Niiniveen.
\par 37 Mutta kun hän oli kerran rukoilemassa jumalansa Nisrokin temppelissä, surmasivat Adrammelek ja Sareser hänet miekalla; sitten he pakenivat Araratin maahan. Ja hänen poikansa Eesarhaddon tuli kuninkaaksi hänen sijaansa.

\chapter{20}

\par 1 Niihin aikoihin Hiskia sairastui ja oli kuolemaisillansa; ja profeetta Jesaja, Aamoksen poika, tuli hänen tykönsä ja sanoi hänelle: "Näin sanoo Herra: Toimita talosi; sillä sinä kuolet etkä enää parane".
\par 2 Niin hän käänsi kasvonsa seinään päin ja rukoili Herraa sanoen:
\par 3 "Oi Herra, muista, kuinka minä olen vaeltanut sinun edessäsi uskollisesti ja ehyellä sydämellä ja tehnyt sitä, mikä on hyvää sinun silmissäsi!" Ja Hiskia itki katkerasti.
\par 4 Mutta Jesaja ei ollut vielä lähtenyt keskimmäiseltä esipihalta, kun hänelle tuli tämä Herran sana:
\par 5 "Palaja takaisin ja sano Hiskialle, minun kansani ruhtinaalle: 'Näin sanoo Herra, sinun isäsi Daavidin Jumala: Minä olen kuullut sinun rukouksesi, olen nähnyt sinun kyyneleesi. Katso, minä parannan sinut: jo kolmantena päivänä sinä menet Herran temppeliin.
\par 6 Ja minä lisään sinulle ikää viisitoista vuotta. Ja minä pelastan sinut ja tämän kaupungin Assurin kuninkaan käsistä. Minä varjelen tätä kaupunkia itseni tähden ja palvelijani Daavidin tähden.'"
\par 7 Ja Jesaja sanoi: "Toimittakaa tänne viikunakakkua". Niin he toivat sitä ja panivat paiseen päälle; ja hän parani.
\par 8 Ja Hiskia sanoi Jesajalle: "Mikä on merkkinä siitä, että Herra on parantava minut, niin että minä kolmantena päivänä voin mennä Herran temppeliin?"
\par 9 Jesaja sanoi: "Tämä on oleva sinulle merkkinä Herralta siitä, että Herra tekee, mitä hän on sanonut: varjo kulkee kymmenen astetta eteenpäin tahi siirtyy kymmenen astetta takaisin".
\par 10 Hiskia sanoi: "Varjon on helppo pidetä kymmenen astetta. Ei, vaan siirtyköön varjo takaisin kymmenen astetta."
\par 11 Silloin profeetta Jesaja huusi Herraa, ja hän antoi varjon Aahaan aurinkokellossa siirtyä takaisin kymmenen astetta, jotka se oli jo laskeutunut.
\par 12 Siihen aikaan Merodak-Baladan, Baladanin poika, Baabelin kuningas, lähetti kirjeen ja lahjoja Hiskialle, sillä hän oli kuullut, että Hiskia oli ollut sairaana.
\par 13 Ja kuultuansa heitä Hiskia näytti heille koko varastohuoneensa, hopean ja kullan, hajuaineet ja kalliin öljyn ja koko asehuoneensa ja kaikki, mitä hänen aarrekammioissansa oli. Ei ollut mitään Hiskian talossa eikä koko hänen valtakunnassaan, mitä hän ei olisi heille näyttänyt.
\par 14 Mutta profeetta Jesaja tuli kuningas Hiskian tykö ja sanoi hänelle: "Mitä nämä miehet ovat sanoneet, ja mistä he ovat tulleet sinun tykösi?" Hiskia vastasi: "He ovat tulleet kaukaisesta maasta, Baabelista".
\par 15 Hän sanoi: "Mitä he ovat nähneet sinun talossasi?" Hiskia vastasi: "Kaiken, mitä talossani on, he ovat nähneet; minun aarrekammioissani ei ole mitään, mitä en olisi heille näyttänyt".
\par 16 Niin Jesaja sanoi Hiskialle: "Kuule Herran sana:
\par 17 Katso, päivät tulevat, jolloin kaikki, mitä sinun talossasi on ja mitä sinun isäsi ovat koonneet tähän päivään asti, viedään pois Baabeliin; ei mitään jää jäljelle, sanoo Herra.
\par 18 Ja sinun omia poikiasi, jotka sinusta polveutuvat, jotka sinulle syntyvät, viedään hovipalvelijoiksi Baabelin kuninkaan palatsiin."
\par 19 Hiskia sanoi Jesajalle: "Herran sana, jonka olet puhunut, on hyvä". Sillä hän ajatteli: "Onpahan rauha ja turvallisuus minun päivinäni".
\par 20 Mitä muuta on kerrottavaa Hiskiasta ja kaikista hänen urotöistään, ja kuinka hän rakensi lammikon ja vesijohdon ja johti veden kaupunkiin, se on kirjoitettuna Juudan kuningasten aikakirjassa.
\par 21 Ja Hiskia meni lepoon isiensä tykö. Ja hänen poikansa Manasse tuli kuninkaaksi hänen sijaansa.

\chapter{21}

\par 1 Manasse oli kahdentoista vuoden vanha tullessansa kuninkaaksi, ja hän hallitsi Jerusalemissa viisikymmentä viisi vuotta. Hänen äitinsä oli nimeltään Hefsibah.
\par 2 Hän teki sitä, mikä on pahaa Herran silmissä, niiden kansain kauhistavien tekojen mukaan, jotka Herra oli karkoittanut israelilaisten tieltä.
\par 3 Hän rakensi jälleen uhrikukkulat, jotka hänen isänsä Hiskia oli hävittänyt, ja pystytti alttareja Baalille ja teki aseran, niinkuin Israelin kuningas Ahab oli tehnyt, kumarsi ja palveli kaikkea taivaan joukkoa.
\par 4 Hän rakensi alttareja myös Herran temppeliin, josta paikasta Herra oli sanonut: "Jerusalemiin minä asetan nimeni".
\par 5 Hän rakensi alttareja kaikelle taivaan joukolle Herran temppelin molempiin esipihoihin.
\par 6 Myös pani hän poikansa kulkemaan tulen läpi, ennusteli merkeistä, harjoitti noituutta ja hankki itsellensä vainaja- ja tietäjähenkien manaajia; hän teki paljon sitä, mikä on pahaa Herran silmissä, ja vihoitti hänet.
\par 7 Ja teettämänsä Aseran kuvan hän asetti temppeliin, josta Herra oli sanonut Daavidille ja hänen pojallensa Salomolle: "Tähän temppeliin ja Jerusalemiin, jonka minä olen valinnut kaikista Israelin sukukunnista, minä asetan nimeni ikiajoiksi.
\par 8 Enkä minä anna enää Israelin jalkojen harhailla pois siitä maasta, jonka minä olen antanut heidän isillensä, jos he vain kaikessa noudattavat sitä, mitä minä olen käskenyt heidän noudattaa - koko sitä lakia, jonka minun palvelijani Mooses on heille antanut."
\par 9 Mutta he eivät totelleet, ja Manasse eksytti heidät tekemään pahaa enemmän kuin ne kansat, jotka Herra oli hävittänyt israelilaisten tieltä.
\par 10 Niin Herra puhui palvelijainsa, profeettain, kautta ja sanoi:
\par 11 "Koska Manasse, Juudan kuningas, on tehnyt nämä kauhistavat teot ja siten tehnyt pahaa enemmän kuin amorilaiset, jotka olivat ennen häntä, ja on saattanut kivijumalillansa myös Juudan tekemään syntiä,
\par 12 sentähden sanoo Herra, Israelin Jumala, näin: Katso, minä annan sellaisen onnettomuuden kohdata Jerusalemia ja Juudaa, että joka sen kuulee, sen molemmat korvat soivat.
\par 13 Ja minä mittaan Jerusalemin samalla mittanuoralla kuin Samarian ja punnitsen sen samalla vaa'alla kuin Ahabin suvun. Ja minä huuhdon Jerusalemin, niinkuin vati huuhdotaan: se huuhdotaan ja käännetään kumollensa.
\par 14 Minä hylkään perintöosani jäännöksen ja annan heidät vihollistensa käsiin, niin että he joutuvat kaikkien vihollistensa saaliiksi ja ryöstettäviksi,
\par 15 koska ovat tehneet sitä, mikä on pahaa minun silmissäni, ja vihoittaneet minua siitä päivästä asti, jona heidän isänsä lähtivät Egyptin maasta, tähän päivään saakka."
\par 16 Manasse vuodatti myös sangen paljon viatonta verta, niin että hän sillä täytti Jerusalemin ääriään myöten, sen syntinsä lisäksi, jolla hän saattoi Juudan tekemään syntiä, tekemään sitä, mikä on pahaa Herran silmissä.
\par 17 Mitä muuta on kerrottavaa Manassesta ja kaikesta, mitä hän teki, ja synnistä, minkä hän teki, se on kirjoitettuna Juudan kuningasten aikakirjassa.
\par 18 Ja Manasse meni lepoon isiensä tykö, ja hänet haudattiin linnansa puutarhaan, Ussan puutarhaan. Ja hänen poikansa Aamon tuli kuninkaaksi hänen sijaansa.
\par 19 Aamon oli kahdenkymmenen kahden vuoden vanha tullessaan kuninkaaksi, ja hän hallitsi Jerusalemissa kaksi vuotta. Hänen äitinsä oli nimeltään Mesullemet, Haaruksen tytär, Jotbasta.
\par 20 Hän teki sitä, mikä on pahaa Herran silmissä, niinkuin hänen isänsä Manasse oli tehnyt.
\par 21 Hän vaelsi kaikessa samaa tietä, mitä hänen isänsä oli vaeltanut: hän palveli niitä kivijumalia, joita hänen isänsä oli palvellut, ja kumarsi niitä;
\par 22 hän hylkäsi Herran, isiensä Jumalan, eikä vaeltanut Herran tietä.
\par 23 Ja Aamonin palvelijat tekivät salaliiton häntä vastaan ja tappoivat kuninkaan hänen linnassansa.
\par 24 Mutta maan kansa surmasi kaikki, jotka olivat tehneet salaliiton kuningas Aamonia vastaan. Ja maan kansa teki hänen poikansa Joosian kuninkaaksi hänen sijaansa.
\par 25 Mitä muuta on kerrottavaa Aamonista, siitä, mitä hän teki, se on kirjoitettuna Juudan kuningasten aikakirjassa.
\par 26 Ja hänet haudattiin omaan hautaansa Ussan puutarhaan. Ja hänen poikansa Joosia tuli kuninkaaksi hänen sijaansa.

\chapter{22}

\par 1 Joosia oli kahdeksan vuoden vanha tullessaan kuninkaaksi, ja hän hallitsi Jerusalemissa kolmekymmentä yksi vuotta. Hänen äitinsä oli nimeltään Jedida, Adajan tytär, Boskatista.
\par 2 Hän teki sitä, mikä on oikein Herran silmissä, ja vaelsi kaikessa isänsä Daavidin tietä, poikkeamatta oikealle tai vasemmalle.
\par 3 Kahdeksantenatoista hallitusvuotenaan kuningas Joosia lähetti kirjuri Saafanin, Asaljan pojan, Mesullamin pojanpojan, Herran temppeliin ja sanoi:
\par 4 "Mene ylimmäisen papin Hilkian luo, että hän lukisi valmiiksi Herran temppeliin tuodut rahat, joita ovenvartijat ovat koonneet kansalta.
\par 5 Ja hän antakoon ne työnteettäjille, jotka on pantu valvomaan töitä Herran temppelissä, ja nämä maksakoot niillä työmiehet, jotka ovat Herran temppelissä korjaamassa sitä, mikä temppelissä on rappeutunutta,
\par 6 puusepät, rakentajat ja muurarit, sekä temppelin korjaamista varten ostettavat puutavarat ja hakatut kivet.
\par 7 Älköön heiltä kuitenkaan vaadittako tilintekoa rahoista, jotka heille luovutetaan, vaan toimikoot he luottamusmiehinä."
\par 8 Ja ylimmäinen pappi Hilkia sanoi kirjuri Saafanille: "Minä löysin Herran temppelistä lain kirjan". Ja Hilkia antoi kirjan Saafanille, ja tämä luki sen.
\par 9 Sitten kirjuri Saafan meni kuninkaan tykö ja teki kuninkaalle selon asiasta, sanoen: "Palvelijasi ovat ottaneet esille rahat, mitä temppelissä oli, ja ovat antaneet ne työnteettäjille, jotka on pantu valvomaan töitä Herran temppelissä".
\par 10 Ja kirjuri Saafan kertoi kuninkaalle sanoen: "Pappi Hilkia antoi minulle erään kirjan". Ja Saafan luki sen kuninkaalle.
\par 11 Kun kuningas kuuli lain kirjan sanat, repäisi hän vaatteensa.
\par 12 Ja kuningas käski pappi Hilkiaa, Ahikamia, Saafanin poikaa, Akboria, Miikajan poikaa, kirjuri Saafania ja Asajaa, kuninkaan palvelijaa, sanoen:
\par 13 "Menkää ja kysykää minun ja kansan ja koko Juudan puolesta neuvoa Herralta tästä löydetystä kirjasta. Sillä suuri on Herran viha, joka on syttynyt meitä vastaan, koska meidän isämme eivät ole totelleet tämän kirjan sanoja eivätkä tehneet mitään kaikesta siitä, mitä siinä on meille kirjoitettuna."
\par 14 Niin pappi Hilkia, Ahikam, Akbor, Saafan ja Asaja menivät naisprofeetta Huldan tykö, joka oli vaatevaraston hoitajan Sallumin, Tikvan pojan, Harhaan pojanpojan, vaimo ja asui Jerusalemissa, toisessa kaupunginosassa. He puhuivat hänen kanssaan.
\par 15 Niin hän sanoi heille: "Näin sanoo Herra, Israelin Jumala: Sanokaa sille miehelle, joka lähetti teidät minun tyköni:
\par 16 'Näin sanoo Herra: Katso, minä annan tätä paikkaa ja sen asukkaita kohdata onnettomuuden, kaiken, mitä sanotaan siinä kirjassa, jonka Juudan kuningas on lukenut,
\par 17 sillä he ovat hyljänneet minut ja polttaneet uhreja muille jumalille ja vihoittaneet minut kaikilla kättensä teoilla; ja minussa on syttynyt viha tätä paikkaa kohtaan, eikä se sammu.
\par 18 Mutta Juudan kuninkaalle, joka on lähettänyt teidät kysymään neuvoa Herralta, sanokaa näin: Näin sanoo Herra, Israelin Jumala, niistä sanoista, jotka sinä olet kuullut:
\par 19 Koska sinun sydämesi on pehminnyt ja sinä olet nöyrtynyt Herran edessä, kuullessasi, mitä minä olen puhunut tätä paikkaa ja sen asukkaita vastaan, että he tulevat kauhistukseksi ja kiroussanaksi, ja koska sinä olet reväissyt vaatteesi ja itkenyt minun edessäni, niin minäkin olen kuullut sinua, sanoo Herra.
\par 20 Sentähden minä korjaan sinut isiesi tykö, ja sinä saat rauhassa siirtyä hautaasi, ja sinun silmäsi pääsevät näkemästä kaikkea sitä onnettomuutta, minkä minä annan kohdata tätä paikkaa.'" Ja he toivat kuninkaalle tämän vastauksen.

\chapter{23}

\par 1 Niin kuningas lähetti kokoamaan luoksensa kaikki Juudan ja Jerusalemin vanhimmat.
\par 2 Ja kuningas meni Herran temppeliin, ja hänen kanssaan kaikki Juudan miehet ja kaikki Jerusalemin asukkaat, myöskin papit ja profeetat, koko kansa pienimmästä suurimpaan. Ja hän luki heidän kuultensa kaikki Herran temppelistä löydetyn liitonkirjan sanat.
\par 3 Ja kuningas asettui pylvään viereen ja teki Herran edessä liiton, että heidän tuli seurata Herraa, noudattaa hänen käskyjänsä, todistuksiansa ja säädöksiänsä kaikesta sydämestään ja kaikesta sielustaan ja pitää liiton sanat, jotka olivat kirjoitettuina siinä kirjassa. Ja kaikki kansa yhtyi siihen liittoon.
\par 4 Sitten kuningas käski ylimmäistä pappia Hilkiaa ja häntä lähimpiä pappeja sekä ovenvartijoita viemään Herran temppelistä pois kaikki kalut, mitä oli tehty Baalille, Aseralle ja kaikelle taivaan joukolle. Ja hän poltatti ne Jerusalemin ulkopuolella Kidronin kedoilla, mutta vei niiden tuhan Beeteliin.
\par 5 Hän pani myös viralta epäjumalain papit, jotka Juudan kuninkaat olivat asettaneet polttamaan uhreja uhrikukkuloilla Juudan kaupungeissa ja Jerusalemin ympäristössä ja jotka polttivat uhreja Baalille, auringolle, kuulle, eläinradan tähdille ja kaikelle taivaan joukolle.
\par 6 Hän vei aseran pois Herran temppelistä Jerusalemin ulkopuolelle Kidronin laaksoon ja poltti sen Kidronin laaksossa, rouhensi sen tomuksi ja heitti tomun yhteiselle hautausmaalle.
\par 7 Ja hän kukisti haureellisten pyhäkköpoikien huoneet, jotka olivat Herran temppelissä ja joissa naiset kutoivat verhoja Aseralle.
\par 8 Hän toi pois kaikki papit Juudan kaupungeista ja saastutti uhrikukkulat, joilla papit olivat polttaneet uhreja, Gebasta Beersebaan asti. Ja hän kukisti porteilla olevat uhrikukkulat, joita oli kaupungin päällikön Joosuan portin oven edustalla, vasemmalla puolella mentäessä sisään kaupungin portista.
\par 9 Uhrikukkulapapit eivät kuitenkaan saaneet nousta Herran alttarille Jerusalemissa; he saivat vain syödä happamatonta leipää veljiensä kanssa.
\par 10 Hän saastutti myöskin polttopaikan Ben-Hinnomin laaksossa, ettei kukaan voisi panna poikaansa tai tytärtänsä kulkemaan tulen läpi Molokin kunniaksi.
\par 11 Ja hän poisti ne hevoset, jotka Juudan kuninkaat olivat auringon kunniaksi asettaneet, siitä, mistä mennään Herran temppeliin, hoviherra Netan-Melekin kammion vierestä, joka oli Parvarimissa, ja poltti auringonvaunut tulessa.
\par 12 Ja Aahaan yläsalin katolla olevat alttarit, jotka Juudan kuninkaat olivat teettäneet, ja ne alttarit, jotka Manasse oli teettänyt Herran temppelin molempiin esipihoihin, kuningas kukisti; sitten hän riensi sieltä ja heitti niiden tomun Kidronin laaksoon.
\par 13 Ja ne uhrikukkulat, jotka olivat itään päin Jerusalemista, etelään päin Turmiovuoresta, ja jotka Salomo, Israelin kuningas, oli rakentanut Astartelle, siidonilais-iljetykselle, ja Kemokselle, Mooabin iljetykselle, ja Milkomille, ammonilais-kauhistukselle, ne kuningas saastutti.
\par 14 Hän murskasi patsaat ja hakkasi maahan asera-karsikot ja täytti niiden sijan ihmisten luilla.
\par 15 Myöskin Beetelissä olevan alttarin, sen uhrikukkulan, jonka oli teettänyt Jerobeam, Nebatin poika, joka saattoi Israelin tekemään syntiä, senkin alttarin uhrikukkuloineen hän kukisti; sitten hän poltti uhrikukkulan ja rouhensi sen tomuksi sekä poltti asera-karsikon.
\par 16 Kun Joosia sitten kääntyi ja näki haudat, jotka olivat vuorella, lähetti hän ottamaan luut haudoista, poltti ne alttarilla ja saastutti näin alttarin, Herran sanan mukaan, jonka oli julistanut se Jumalan mies, joka nämä julisti.
\par 17 Ja hän kysyi: "Mikä tuo hautamerkki on, jonka minä näen?" Kaupungin miehet vastasivat hänelle: "Se on sen Jumalan miehen hauta, joka tuli Juudasta ja julisti sen, minkä sinä nyt olet tehnyt Beetelin alttarille".
\par 18 Hän sanoi: "Antakaa hänen olla; älköön kukaan koskeko hänen luihinsa". Niin he jättivät hänen luunsa rauhaan ja samoin sen profeetan luut, joka oli tullut Samariasta.
\par 19 Myöskin Samarian kaupungeista Joosia poisti kaikki uhrikukkulatemppelit, jotka Israelin kuninkaat olivat rakentaneet ja niin vihoittaneet Herran; ja hän teki niille saman, minkä oli tehnyt Beetelissä.
\par 20 Ja kaikki siellä olevat uhrikukkulapapit hän teurasti alttareilla ja poltti ihmisten luita niiden päällä. Sitten hän palasi takaisin Jerusalemiin.
\par 21 Ja kuningas käski kaikkea kansaa sanoen: "Viettäkää pääsiäistä Herran, teidän Jumalanne, kunniaksi, niinkuin on kirjoitettuna tässä liitonkirjassa".
\par 22 Sillä sellaista pääsiäistä ei oltu vietetty sen ajan jälkeen, jona tuomarit tuomitsivat Israelia, ei Israelin kuningasten eikä Juudan kuningasten koko aikana.
\par 23 Vasta kuningas Joosian kahdeksantenatoista hallitusvuotena vietettiin sellainen pääsiäinen Jerusalemissa Herran kunniaksi.
\par 24 Myöskin vainaja- ja tietäjähenkien manaajat, kotijumalat ja kivijumalat ja kaikki iljetykset, joita oli nähty Juudan maassa ja Jerusalemissa, Joosia hävitti, täyttääkseen lain sanat, jotka olivat kirjoitettuina siinä kirjassa, minkä pappi Hilkia oli löytänyt Herran temppelistä.
\par 25 Ei ollut ennen häntä ollut hänen vertaistansa kuningasta, joka niin kaikesta sydämestänsä, kaikesta sielustansa ja kaikesta voimastansa olisi kääntynyt Herran puoleen, kaiken Mooseksen lain mukaan; eikä hänen jälkeensä tullut hänen vertaistansa.
\par 26 Kuitenkaan ei Herra kääntynyt suuren vihansa hehkusta, kun kerran hänen vihansa oli syttynyt Juudaa vastaan kaikesta siitä, millä Manasse oli vihoittanut hänet.
\par 27 Ja Herra sanoi: "Minä toimitan myöskin Juudan pois kasvojeni edestä, niinkuin minä olen toimittanut pois Israelin; ja minä hylkään Jerusalemin, tämän kaupungin, jonka minä olin valinnut, ja temppelin, josta minä olin sanonut: 'Minun nimeni on oleva siinä'".
\par 28 Mitä muuta on kerrottavaa Joosiasta ja kaikesta, mitä hän teki, se on kirjoitettuna Juudan kuningasten aikakirjassa.
\par 29 Hänen aikanansa farao Neko, Egyptin kuningas, lähti Assurin kuningasta vastaan Eufrat-virralle. Niin kuningas Joosia meni häntä vastaan, mutta farao surmasi hänet Megiddossa heti, kun näki hänet.
\par 30 Ja hänen palvelijansa veivät hänet kuolleena vaunuissa Megiddosta, toivat hänet Jerusalemiin ja hautasivat hänet hänen omaan hautaansa. Mutta maan kansa otti Joosian pojan Jooahaan, voiteli hänet ja teki hänet kuninkaaksi hänen isänsä sijaan.
\par 31 Jooahas oli kahdenkymmenen kolmen vuoden vanha tullessaan kuninkaaksi, ja hän hallitsi kolme kuukautta Jerusalemissa. Hänen äitinsä oli nimeltään Hamutal, Jeremian tytär, Libnasta.
\par 32 Ja hän teki sitä, mikä on pahaa Herran silmissä, aivan niinkuin hänen isänsä olivat tehneet.
\par 33 Mutta farao Neko vangitutti hänet Riblassa, Hamatin maassa, ettei hän hallitsisi Jerusalemissa, ja määräsi maan maksettavaksi pakkoveron: sata talenttia hopeata ja kymmenen talenttia kultaa.
\par 34 Ja farao Neko teki Eljakimin, Joosian pojan, kuninkaaksi hänen isänsä Joosian sijaan ja muutti hänen nimensä Joojakimiksi. Mutta Jooahaan hän otti vangiksi, ja tämä joutui Egyptiin; siellä hän kuoli.
\par 35 Hopean ja kullan Joojakim maksoi faraolle; mutta voidakseen maksaa rahat faraon käskyn mukaan hän veroitti maata, ottaen maan kansalta, sen mukaan kuin kukin oli verotettu, hopeata ja kultaa, antaakseen farao Nekolle.
\par 36 Joojakim oli kahdenkymmenen viiden vuoden vanha tullessaan kuninkaaksi, ja hän hallitsi Jerusalemissa yksitoista vuotta. Hänen äitinsä oli nimeltään Sebida, Pedajan tytär, Ruumasta.
\par 37 Hän teki sitä, mikä on pahaa Herran silmissä, aivan niinkuin hänen isänsä olivat tehneet.

\chapter{24}

\par 1 Hänen aikanaan Nebukadnessar, Baabelin kuningas, lähti liikkeelle, ja Joojakim tuli hänen palvelijakseen. Kolmen vuoden kuluttua tämä jälleen kapinoi häntä vastaan.
\par 2 Ja Herra lähetti hänen kimppuunsa kaldealaisten, aramilaisten, mooabilaisten ja ammonilaisten partiojoukkoja; hän lähetti ne Juudan kimppuun sitä tuhoamaan, Herran sanan mukaan, jonka hän oli puhunut palvelijainsa, profeettain, kautta.
\par 3 Mutta tämä tapahtui Juudalle Herran käskystä; ja näin hän toimitti heidät pois kasvojensa edestä Manassen syntien tähden, kaiken tähden, mitä hän oli tehnyt,
\par 4 ja myöskin viattoman veren tähden, minkä hän oli vuodattanut täyttäessään Jerusalemin viattomalla verellä; sitä ei Herra tahtonut antaa anteeksi.
\par 5 Mitä muuta on kerrottavaa Joojakimista ja kaikesta, mitä hän teki, se on kirjoitettuna Juudan kuningasten aikakirjassa.
\par 6 Ja Joojakim meni lepoon isiensä tykö. Ja hänen poikansa Joojakin tuli kuninkaaksi hänen sijaansa.
\par 7 Egyptin kuningas ei enää lähtenyt maastansa, sillä Baabelin kuningas oli vallannut kaiken, mikä oli Egyptin kuninkaan omaa, Egyptin purosta aina Eufrat-virtaan saakka.
\par 8 Joojakin oli kahdeksantoista vuoden vanha tullessansa kuninkaaksi, ja hän hallitsi Jerusalemissa kolme kuukautta. Hänen äitinsä oli nimeltään Nehusta, Elnatanin tytär, Jerusalemista.
\par 9 Hän teki sitä, mikä on pahaa Herran silmissä, aivan niinkuin hänen isänsä oli tehnyt.
\par 10 Siihen aikaan Baabelin kuninkaan Nebukadnessarin palvelijat tulivat Jerusalemiin, ja kaupunki saarrettiin.
\par 11 Ja Nebukadnessar, Baabelin kuningas, ryntäsi kaupunkiin, hänen palvelijainsa saartaessa sitä.
\par 12 Niin Joojakin, Juudan kuningas, antautui Baabelin kuninkaalle äitineen ja palvelijoineen, päällikköineen ja hoviherroineen. Ja Baabelin kuningas otti hänet vangiksi kahdeksantena hallitusvuotenaan.
\par 13 Ja hän vei pois sieltä kaikki Herran temppelin ja kuninkaan linnan aarteet, ja hän rikkoi kaikki kultakalut, jotka Israelin kuningas Salomo oli teettänyt Herran temppeliin. Tapahtui, niinkuin Herra oli puhunut.
\par 14 Ja hän vei pakkosiirtolaisuuteen koko Jerusalemin, kaikki päälliköt ja kaikki sotaurhot, kymmenentuhatta pakkosiirtolaista, ja kaikki sepät ja lukkosepät; ei jäänyt jäljelle muita kuin maakansan köyhät.
\par 15 Hän siirsi Joojakinin vankina Baabeliin; samoin hän vei kuninkaan äidin, kuninkaan puolisot ja hänen hoviherransa sekä muut maan mahtavat pakkosiirtolaisina Jerusalemista Baabeliin,
\par 16 niin myös kaikki sotilaat, luvultaan seitsemäntuhatta, sepät ja lukkosepät, luvultaan tuhat, kaikki sotaan harjaantuneita urhoja. Ne Baabelin kuningas vei pakkosiirtolaisina Baabeliin.
\par 17 Mutta Baabelin kuningas asetti hänen setänsä Mattanjan kuninkaaksi hänen sijaansa ja muutti hänen nimensä Sidkiaksi.
\par 18 Sidkia oli kahdenkymmenen yhden vuoden vanha tullessaan kuninkaaksi, ja hän hallitsi Jerusalemissa yksitoista vuotta. Hänen äitinsä oli nimeltään Hamutal, Jeremian tytär, Libnasta.
\par 19 Hän teki sitä, mikä on pahaa Herran silmissä, aivan samoin kuin Joojakim oli tehnyt.
\par 20 Sillä Herran vihan tähden kävi näin Jerusalemille ja Juudalle, kunnes hän vihdoin heitti heidät pois kasvojensa edestä.

\chapter{25}

\par 1 Sidkia kapinoi Baabelin kuningasta vastaan. Hänen yhdeksäntenä hallitusvuotenaan, sen kymmenennessä kuussa, kuukauden kymmenentenä päivänä, kävi Nebukadnessar, Baabelin kuningas, hän ja koko hänen sotajoukkonsa, Jerusalemin kimppuun ja asettui leiriin sitä vastaan. Ja he rakensivat saartovarusteet sitä vastaan yltympäri.
\par 2 Näin kaupunkia piiritettiin kuningas Sidkian yhdenteentoista hallitusvuoteen asti.
\par 3 Mutta neljännen kuukauden yhdeksäntenä päivänä, kun nälänhätä ahdisti kaupunkia eikä ollut leipää maan kansalle,
\par 4 valloitettiin kaupunki, ja kaikki sotilaat pakenivat yöllä molempien muurien välistä porttitietä, joka oli kuninkaan puutarhan puolella, kaldealaisten ollessa kaupungin ympärillä, ja he kulkivat Aromaahan päin.
\par 5 Mutta kaldealaisten sotajoukko ajoi kuningasta takaa, ja he saavuttivat hänet Jerikon aroilla; ja kaikki hänen sotaväkensä oli jättänyt hänet ja hajaantunut.
\par 6 Ja he ottivat kuninkaan kiinni ja veivät hänet Baabelin kuninkaan eteen Riblaan; ja hänelle julistettiin tuomio:
\par 7 Sidkian pojat teurastettiin hänen silmiensä edessä, ja Sidkialta itseltään hän sokaisutti silmät ja kytketti hänet vaskikahleisiin, ja hänet vietiin Baabeliin.
\par 8 Viidennessä kuussa, kuukauden seitsemäntenä päivänä, Baabelin kuninkaan Nebukadnessarin yhdeksäntenätoista hallitusvuotena, tuli Baabelin kuninkaan palvelija Nebusaradan, henkivartijain päällikkö, Jerusalemiin.
\par 9 Hän poltti Herran temppelin ja kuninkaan linnan ja kaikki Jerusalemin talot; kaikki ylhäisten talot hän poltti tulella.
\par 10 Ja koko kaldealaisten sotajoukko, joka henkivartijain päälliköllä oli mukanaan, repi maahan Jerusalemin muurit, yltympäri.
\par 11 Ja Nebusaradan, henkivartijain päällikkö, vei pakkosiirtolaisuuteen kansan tähteet, mitä oli jäljellä kaupungissa, sekä ne, jotka olivat menneet Baabelin kuninkaan puolelle, ja muun rahvaan.
\par 12 Mutta osan maan köyhiä henkivartijain päällikkö jätti jäljelle viinitarhureiksi ja peltomiehiksi.
\par 13 Vaskipylväät, jotka olivat Herran temppelissä, altaiden telineet ja vaskimeren, jotka olivat Herran temppelissä, kaldealaiset särkivät ja veivät vasken Baabeliin.
\par 14 Ja kattilat, lapiot, veitset, kupit ja kaikki vaskikalut, joita oli käytetty jumalanpalveluksessa, he ottivat pois.
\par 15 Samoin otti henkivartijain päällikkö hiilipannut ja maljat, jotka olivat läpeensä kultaa tai hopeata.
\par 16 Molempien pylväiden, meren ja telineiden, jotka Salomo oli teettänyt Herran temppeliin - näiden kaikkien esineiden vaski ei ollut punnittavissa.
\par 17 Toisen pylvään korkeus oli kahdeksantoista kyynärää, ja sen päässä oli pylväänpää vaskesta; pylväänpään korkeus oli kolme kyynärää, ja pylväänpään päällä oli ristikkokoriste ja granaattiomenia yltympäri, kaikki vaskea; ja samanlaiset oli toisessa pylväässä ristikkokoristeen päällä.
\par 18 Henkivartijain päällikkö otti ylimmäisen papin Serajan ja häntä lähimmän papin Sefanjan sekä kolme ovenvartijaa,
\par 19 ja kaupungista hän otti yhden hoviherran, joka oli sotaväen tarkastaja, ja viisi kuninkaan lähintä miestä, jotka tavattiin kaupungista, sekä sotapäällikön kirjurin, jonka tehtävänä oli ottaa maan kansaa sotapalvelukseen, ja kuusikymmentä miestä maan kansasta, jotka tavattiin kaupungista.
\par 20 Nämä otti Nebusaradan, henkivartijain päällikkö, ja vei heidät Baabelin kuninkaan eteen Riblaan.
\par 21 Ja Baabelin kuningas antoi lyödä heidät kuoliaaksi Riblassa Hamatin maassa. Ja niin Juuda vietiin pois maastansa pakkosiirtolaisuuteen.
\par 22 Mutta sen väen käskynhaltijaksi, joka jäi Juudan maahan, sen, jonka Nebukadnessar, Baabelin kuningas, jätti sinne, hän asetti Gedaljan, Ahikamin pojan, Saafanin pojanpojan.
\par 23 Kun kaikki sotaväen päälliköt, he ja heidän miehensä, kuulivat, että Baabelin kuningas oli asettanut Gedaljan käskynhaltijaksi, tulivat he miehinensä Gedaljan luo Mispaan, nimittäin Ismael, Netanjan poika, Joohanan, Kaareahin poika, netofalainen Seraja, Tanhumetin poika, ja Jaasanja, maakatilaisen poika.
\par 24 Ja Gedalja vannoi heille ja heidän miehillensä ja sanoi heille: "Älkää peljätkö kaldealaisten palvelijoita. Jääkää maahan ja palvelkaa Baabelin kuningasta, niin te menestytte."
\par 25 Mutta seitsemännessä kuussa tuli Ismael, Netanjan poika, Elisaman pojanpoika, joka oli kuninkaallista sukua, mukanaan kymmenen miestä, ja he löivät kuoliaaksi Gedaljan ja ne Juudan miehet ja kaldealaiset, jotka olivat hänen kanssansa Mispassa.
\par 26 Silloin nousi kaikki kansa, pienimmästä suurimpaan, ja samoin sotaväen päälliköt, ja he menivät Egyptiin, sillä he pelkäsivät kaldealaisia.
\par 27 Mutta kolmantenakymmenentenä seitsemäntenä vuotena siitä, kun Joojakin, Juudan kuningas, oli viety pakkosiirtolaisuuteen, kahdennessatoista kuussa, kuukauden kahdentenakymmenentenä seitsemäntenä päivänä, korotti Evil-Merodak, Baabelin kuningas, samana vuonna, jona hän tuli kuninkaaksi, Juudan kuninkaan Joojakinin pään, vapauttaen hänet vankilasta.
\par 28 Ja hän puhutteli häntä ystävällisesti ja asetti hänen istuimensa ylemmäksi muitten kuningasten istuimia, jotka olivat hänen tykönänsä Baabelissa.
\par 29 Hän sai panna pois vangin puvun ja aina aterioida hänen luonaan, niin kauan kuin eli.
\par 30 Ja hän sai kuninkaalta elatuksensa, mitä kunakin päivänä tarvitsi, vakinaisen elatuksen, niin kauan kuin eli.


\end{document}