\begin{document}

\title{Ensimmäinen aikakirja}


\chapter{1}

\par 1 Aadam, Seet, Enos,
\par 2 Keenan, Mahalalel, Jered,
\par 3 Hanok, Metusalah, Lemek,
\par 4 Nooa, Seem, Haam ja Jaafet.
\par 5 Jaafetin pojat olivat Goomer, Maagog, Maadai, Jaavan, Tuubal, Mesek ja Tiiras.
\par 6 Ja Goomerin pojat olivat Askenas, Diifat ja Toogarma.
\par 7 Ja Jaavanin pojat olivat Elisa ja Tarsisa, kittiläiset ja roodanilaiset.
\par 8 Haamin pojat olivat Kuus, Misraim, Puut ja Kanaan.
\par 9 Ja Kuusin pojat olivat Seba, Havila, Sabta, Raema ja Sabteka. Ja Raeman pojat olivat Saba ja Dedan.
\par 10 Ja Kuusille syntyi Nimrod; hän oli ensimmäinen valtias maan päällä.
\par 11 Ja Misraimille syntyivät luudilaiset, anamilaiset, lehabilaiset, naftuhilaiset,
\par 12 patrokselaiset ja kasluhilaiset, joista filistealaiset ovat lähteneet, sekä kaftorilaiset.
\par 13 Ja Kanaanille syntyivät Siidon, hänen esikoisensa, ja Heet,
\par 14 sekä jebusilaiset, amorilaiset, girgasilaiset,
\par 15 hivviläiset, arkilaiset, siiniläiset,
\par 16 arvadilaiset, semarilaiset ja hamatilaiset.
\par 17 Seemin pojat olivat Eelam, Assur, Arpaksad, Luud, Aram, Uus, Huul, Geter ja Mesek.
\par 18 Ja Arpaksadille syntyi Selah, ja Selahille syntyi Eeber.
\par 19 Ja Eeberille syntyi kaksi poikaa; toisen nimi oli Peleg, sillä hänen aikanansa jakaantuivat maan asukkaat, ja hänen veljensä nimi oli Joktan.
\par 20 Ja Joktanille syntyivät Almodad, Selef, Hasarmavet, Jerah,
\par 21 Hadoram, Uusal, Dikla,
\par 22 Eebal, Abimael, Saba,
\par 23 Oofir, Havila ja Joobab. Kaikki nämä olivat Joktanin poikia.
\par 24 Seem, Arpaksad, Selah,
\par 25 Eeber, Peleg, Regu,
\par 26 Serug, Naahor, Terah,
\par 27 Abram, se on Aabraham.
\par 28 Aabrahamin pojat olivat Iisak ja Ismael.
\par 29 Tämä on heidän sukuluettelonsa: Nebajot, Ismaelin esikoinen, Keedar, Adbeel, Mibsan,
\par 30 Misma, Duuma, Massa, Hadad, Teema,
\par 31 Jetur, Naafis ja Keedma. Nämä olivat Ismaelin pojat.
\par 32 Ja Keturan, Aabrahamin sivuvaimon, pojat, jotka tämä synnytti, olivat Simran, Joksan, Medan, Midian, Jisbak ja Suuah. Joksanin pojat olivat Saba ja Dedan.
\par 33 Ja Midianin pojat olivat Eefa, Eefer, Hanok, Abida ja Eldaa. Kaikki nämä olivat Keturan jälkeläisiä.
\par 34 Ja Aabrahamille syntyi Iisak. Iisakin pojat olivat Eesau ja Israel.
\par 35 Eesaun pojat olivat Elifas, Reguel, Jeus, Jaelam ja Koorah.
\par 36 Elifaan pojat olivat Teeman, Oomar, Sefi, Gaetam, Kenas, Timna ja Amalek.
\par 37 Reguelin pojat olivat Nahat, Serah, Samma ja Missa.
\par 38 Seirin pojat olivat Lootan, Soobal, Sibon, Ana, Diison, Eeser ja Diisan.
\par 39 Lootanin pojat olivat Hoori ja Hoomam; ja Lootanin sisar oli Timna.
\par 40 Soobalin pojat olivat Aljan, Maanahat, Eebal, Sefi ja Oonan. Ja Sibonin pojat olivat Aija ja Ana.
\par 41 Anan pojat olivat Diison. Ja Diisonin pojat olivat Hamran, Esban, Jitran ja Keran.
\par 42 Eeserin pojat olivat Bilhan, Saavan ja Jaakan. Diisanin pojat olivat Uus ja Aran.
\par 43 Ja nämä olivat ne kuninkaat, jotka hallitsivat Edomin maassa, ennenkuin mikään kuningas oli hallinnut israelilaisia: Bela, Beorin poika; ja hänen kaupunkinsa nimi oli Dinhaba.
\par 44 Ja kun Bela kuoli, tuli Joobab, Serahin poika, Bosrasta, kuninkaaksi hänen sijaansa.
\par 45 Kun Joobab kuoli, tuli Huusam, teemanilaisten maasta, kuninkaaksi hänen sijaansa.
\par 46 Kun Huusam kuoli, tuli Hadad, Bedadin poika, kuninkaaksi hänen sijaansa, hän, joka voitti midianilaiset Mooabin maassa; ja hänen kaupunkinsa nimi oli Avit.
\par 47 Kun Hadad kuoli, tuli Samla, Masrekasta, kuninkaaksi hänen sijaansa.
\par 48 Kun Samla kuoli, tuli Saul, virran rannalla olevasta Rehobotista, kuninkaaksi hänen sijaansa.
\par 49 Kun Saul kuoli, tuli Baal-Haanan, Akborin poika, kuninkaaksi hänen sijaansa.
\par 50 Kun Baal-Haanan kuoli, tuli Hadad kuninkaaksi hänen sijaansa, ja hänen kaupunkinsa nimi oli Paagi; ja hänen vaimonsa nimi oli Mehetabel, Matredin tytär, joka oli Mee-Saahabin tytär.
\par 51 Ja Hadad kuoli; ja Edomin sukuruhtinaat olivat: ruhtinas Timna, ruhtinas Alva, ruhtinas Jetet,
\par 52 ruhtinas Oholibama, ruhtinas Eela, ruhtinas Piinon,
\par 53 ruhtinas Kenas, ruhtinas Teeman, ruhtinas Mibsar,
\par 54 ruhtinas Magdiel, ruhtinas Iiram. Nämä olivat Edomin sukuruhtinaat.

\chapter{2}

\par 1 Nämä olivat Israelin pojat: Ruuben, Simeon, Leevi ja Juuda, Isaskar ja Sebulon,
\par 2 Daan, Joosef ja Benjamin, Naftali, Gaad ja Asser.
\par 3 Juudan pojat olivat Eer, Oonan ja Seela; nämä kolme synnytti hänelle Suuahin tytär, kanaanilainen. Mutta Eer, Juudan esikoinen, ei ollut otollinen Herran silmissä; sentähden hän antoi hänen kuolla.
\par 4 Ja Taamar, Juudan miniä, synnytti hänelle Pereksen ja Serahin. Juudan poikia oli kaikkiaan viisi.
\par 5 Pereksen pojat olivat Hesron ja Haamul.
\par 6 Ja Serahin pojat olivat Simri, Eetan, Heeman, Kalkol ja Daara; heitä oli kaikkiaan viisi.
\par 7 Ja Karmin pojat olivat Aakar, joka syöksi Israelin onnettomuuteen, kun oli uskoton ja anasti tuhon omaksi vihittyä.
\par 8 Ja Eetanin poika oli Asarja.
\par 9 Ja Hesronin pojat, jotka hänelle syntyivät, olivat Jerahmeel, Raam ja Kelubai.
\par 10 Ja Raamille syntyi Amminadab, ja Amminadabille syntyi Nahson, Juudan jälkeläisten ruhtinas.
\par 11 Ja Nahsonille syntyi Salma, ja Salmalle syntyi Booas.
\par 12 Ja Booaalle syntyi Oobed, ja Oobedille syntyi Iisai.
\par 13 Ja Iisaille syntyi esikoisena Eliab, toisena Abinadab, kolmantena Simea,
\par 14 neljäntenä Netanel, viidentenä Raddai,
\par 15 kuudentena Oosem ja seitsemäntenä Daavid.
\par 16 Ja heidän sisarensa olivat Seruja ja Abigail. Ja Serujan pojat olivat Abisai, Jooab ja Asael, kaikkiaan kolme.
\par 17 Ja Abigail synnytti Amasan; ja Amasan isä oli ismaelilainen Jeter.
\par 18 Ja Kaalebille, Hesronin pojalle, syntyi lapsia Asubasta, hänen vaimostaan, ja Jeriotista; ja nämä olivat Asuban pojat: Jeeser, Soobab ja Ardon.
\par 19 Ja kun Asuba kuoli, nai Kaaleb Efratin, ja tämä synnytti hänelle Huurin.
\par 20 Ja Huurille syntyi Uuri, ja Uurille syntyi Besalel.
\par 21 Senjälkeen meni Hesron Maakirin, Gileadin isän, tyttären tykö; ja hän nai tämän ollessaan kuudenkymmenen vuoden vanha. Ja tämä synnytti hänelle Segubin.
\par 22 Ja Segubille syntyi Jaair; tällä oli kaksikymmentä kolme kaupunkia Gileadin maassa.
\par 23 Mutta gesurilaiset ja aramilaiset ottivat heiltä Jaairin leirikylät ja Kenatin ynnä sen tytärkaupungit, kuusikymmentä kaupunkia. Kaikki nämä olivat Maakirin, Gileadin isän, poikia.
\par 24 Kun Hesron oli kuollut Kaaleb-Efratassa, synnytti Abia - hän oli Hesronin vaimo - hänelle Ashurin, Tekoan isän.
\par 25 Ja Jerahmeelin, Hesronin esikoisen, pojat olivat Raam, esikoinen, ja Buuna, Ooren, Oosem ja Ahia.
\par 26 Ja Jerahmeelilla oli vielä toinen vaimo, jonka nimi oli Atara; hän oli Oonamin äiti.
\par 27 Ja Raamin, Jerahmeelin esikoisen, pojat olivat Maas, Jaamin ja Eeker.
\par 28 Ja Oonamin pojat olivat Sammai ja Jaada; ja Sammain pojat olivat Naadab ja Abisur.
\par 29 Ja Abisurin vaimon nimi oli Abihail, ja tämä synnytti hänelle Ahbanin ja Moolidin.
\par 30 Ja Naadabin pojat olivat Seled ja Appaim. Ja Seled kuoli lapsetonna.
\par 31 Ja Appaimin poika oli Jisi; ja Jisin poika oli Seesan; ja Seesanin poika oli Ahlai.
\par 32 Ja Jaadan, Sammain veljen, pojat olivat Jeter ja Joonatan. Jeter kuoli lapsetonna.
\par 33 Ja Joonatanin pojat olivat Pelet ja Saasa. Nämä olivat Jerahmeelin jälkeläisiä.
\par 34 Mutta Seesanilla ei ollut poikia, vaan ainoastaan tyttäriä. Ja Seesanilla oli egyptiläinen palvelija, jonka nimi oli Jarha.
\par 35 Ja Seesan antoi tyttärensä vaimoksi palvelijallensa Jarhalle, ja hän synnytti tälle Attain.
\par 36 Ja Attaille syntyi Naatan, ja Naatanille syntyi Saabad.
\par 37 Ja Saabadille syntyi Eflal, ja Eflalille syntyi Oobed.
\par 38 Ja Oobedille syntyi Jeehu, ja Jeehulle syntyi Asarja.
\par 39 Ja Asarjalle syntyi Heles, ja Helekselle syntyi Eleasa.
\par 40 Ja Eleasalle syntyi Sismai, ja Sismaille syntyi Sallum.
\par 41 Ja Sallumille syntyi Jekamja, ja Jekamjalle syntyi Elisama.
\par 42 Ja Kaalebin, Jerahmeelin veljen, pojat olivat Meesa, hänen esikoisensa, joka oli Siifin isä, ja Maaresan, Hebronin isän, pojat.
\par 43 Ja Hebronin pojat olivat Koorah, Tappuah, Rekem ja Sema.
\par 44 Ja Semalle syntyi Raham, Jorkeamin isä, ja Rekemille syntyi Sammai.
\par 45 Ja Sammain poika oli Maaon, ja Maaon oli Beet-Suurin isä.
\par 46 Ja Eefa, Kaalebin sivuvaimo, synnytti Haaranin, Moosan ja Gaaseksen; ja Haaranille syntyi Gaases.
\par 47 Ja Jahdain pojat olivat Regem, Jootam, Geesan, Pelet, Eefa ja Saaf.
\par 48 Maaka, Kaalebin sivuvaimo, synnytti Seberin ja Tirhanan.
\par 49 Hän synnytti myös Saafin, Madmannan isän, Sevan, Makbenan isän ja Gibean isän. Ja Kaalebin tytär oli Aksa.
\par 50 Nämä olivat Kaalebin jälkeläisiä. Huurin, Efratan esikoisen, pojat olivat Soobal, Kirjat-Jearimin isä,
\par 51 Salma, Beetlehemin isä, ja Haaref, Beet-Gaaderin isä.
\par 52 Soobalin, Kirjat-Jearimin isän, jälkeläiset olivat Rooe ja toinen puoli Menuhotia.
\par 53 Kirjat-Jearimin suvut olivat jeteriläiset, puutilaiset, suumatilaiset ja misrailaiset. Heistä ovat soratilaiset ja estaolilaiset lähteneet.
\par 54 Ja Salman jälkeläiset olivat Beetlehem ja netofalaiset, Atrot, Beet-Jooab ja toinen puoli manahtilaisia, sorilaiset.
\par 55 Ja kirjanoppineiden suvut, jotka asuivat Jaebeksessa, olivat tiratilaiset, simatilaiset ja sukatilaiset. Nämä olivat ne keeniläiset, jotka polveutuvat Hammatista, Reekabin suvun isästä.

\chapter{3}

\par 1 Nämä ovat ne Daavidin pojat, jotka syntyivät hänelle Hebronissa: esikoinen Amnon, jisreeliläisestä Ahinoamista; toinen Daniel, karmelilaisesta Abigailista;
\par 2 kolmas Absalom, Maakan, Gesurin kuninkaan Talmain tyttären, poika; neljäs Adonia, Haggitin poika;
\par 3 viides Sefatja, Abitalista; kuudes Jitream, hänen vaimostaan Eglasta.
\par 4 Nämä kuusi syntyivät hänelle Hebronissa, jossa hän hallitsi seitsemän vuotta ja kuusi kuukautta. Mutta Jerusalemissa hän hallitsi kolmekymmentä kolme vuotta.
\par 5 Ja nämä syntyivät hänelle Jerusalemissa: Simea, Soobab, Naatan ja Salomo, kaikkiaan neljä, Bat-Suuasta, Ammielin tyttärestä;
\par 6 sitten Jibhar, Elisama, Elifelet,
\par 7 Noogah, Nefeg, Jaafia,
\par 8 Elisama, Eljada ja Elifelet; kaikkiaan yhdeksän.
\par 9 Tässä ovat kaikki Daavidin pojat, lukuunottamatta sivuvaimojen poikia. Ja Taamar oli heidän sisarensa.
\par 10 Ja Salomon poika oli Rehabeam; tämän poika Abia; tämän poika Aasa; tämän poika Joosafat;
\par 11 tämän poika Jooram; tämän poika Ahasja; tämän poika Jooas;
\par 12 tämän poika Amasja; tämän poika Asarja; tämän poika Jootam;
\par 13 tämän poika Aahas; tämän poika Hiskia; tämän poika Manasse;
\par 14 tämän poika Aamon; tämän poika Joosia.
\par 15 Ja Joosian pojat olivat esikoinen Joohanan, toinen Joojakim, kolmas Sidkia, neljäs Sallum.
\par 16 Ja Joojakimin pojat olivat hänen poikansa Jekonja ja tämän poika Sidkia.
\par 17 Ja vangitun Jekonjan pojat olivat: hänen poikansa Sealtiel,
\par 18 Malkiram, Pedaja, Senassar, Jekamja, Hoosama ja Nedabja.
\par 19 Ja Pedajan pojat olivat Serubbaabel ja Siimei. Ja Serubbaabelin pojat olivat Mesullam ja Hananja, ja heidän sisarensa oli Selomit;
\par 20 vielä Hasuba, Oohel, Berekja, Hasadja ja Juusab-Hesed, kaikkiaan viisi.
\par 21 Ja Hananjan pojat olivat Pelatja ja Jesaja, Refajan pojat, Arnanin pojat, Obadjan pojat ja Sekanjan pojat.
\par 22 Ja Sekanjan poika oli Semaja; ja Semajan pojat olivat Hattus, Jigal, Baariah, Nearja ja Saafat, kaikkiaan kuusi.
\par 23 Ja Nearjan pojat olivat Eljoenai, Hiskia, Asrikam, kaikkiaan kolme.
\par 24 Ja Eljoenain pojat olivat Hoodavja, Eljasib, Pelaja, Akkub, Joohanan, Delaja ja Anani, kaikkiaan seitsemän.

\chapter{4}

\par 1 Juudan pojat olivat Peres, Hesron, Karmi, Huur ja Soobal.
\par 2 Ja Reajalle, Soobalin pojalle, syntyi Jahat, ja Jahatille syntyivät Ahumai ja Lahad. Nämä ovat soratilaisten suvut.
\par 3 Ja nämä olivat Eetamin isän pojat: Jisreel, Jisma ja Jidbas; ja heidän sisarensa nimi oli Haslelponi;
\par 4 vielä Penuel, Gedorin isä, ja Eeser, Huusan isä. Nämä olivat Huurin, Efratan esikoisen, Beetlehemin isän, pojat.
\par 5 Ja Ashurilla, Tekoan isällä, oli kaksi vaimoa, Hela ja Naara.
\par 6 Ja Naara synnytti hänelle Ahussamin, Heeferin, Teemenin ja ahastarilaiset. Nämä olivat Naaran pojat.
\par 7 Ja Helan pojat olivat Seret, Soohar ja Etnan.
\par 8 Ja Koosille syntyivät Aanub ja Soobeba sekä Aharhelin, Haarumin pojan, suvut.
\par 9 Mutta Jabes oli suuremmassa arvossa pidetty kuin hänen veljensä; ja hänen äitinsä oli antanut hänelle nimen Jabes sanoen: "Olen synnyttänyt hänet kipuja kärsien".
\par 10 Ja Jabes huusi Israelin Jumalaa, sanoen: "Jospa sinä siunaisit minua ja laajentaisit minun alueeni; jospa sinun kätesi olisi minun kanssani ja sinä varjelisit pahasta, niin että minä pääsisin kipuja kärsimästä!" Ja Jumala antoi hänen pyyntönsä toteutua.
\par 11 Ja Kelubille, Suuhan veljelle, syntyi Mehir; hän oli Estonin isä.
\par 12 Ja Estonille syntyivät Beet-Raafa, Paaseah ja Tehinna, Naahaan kaupungin isä. Nämä olivat Reekan miehet.
\par 13 Ja Kenaan pojat olivat Otniel ja Seraja. Ja Otnielin pojat olivat: Hatat.
\par 14 Ja Meonotaille syntyi Ofra. Ja Serajalle syntyi Jooab, Seppäinlaakson isä; sillä he olivat seppiä.
\par 15 Ja Kaalebin, Jefunnen pojan, pojat olivat Iiru, Eela ja Naam sekä Eelan pojat ja Kenas.
\par 16 Ja Jehallelelin pojat olivat Siif, Siifa, Tiireja ja Asarel.
\par 17 Ja Esran pojat olivat Jeter, Mered, Eefer ja Jaalon. Ja vaimo tuli raskaaksi ja synnytti Mirjamin, Sammain ja Jisbahin, Estemoan isän.
\par 18 Ja hänen juudansukuinen vaimonsa synnytti Jeredin, Gedorin isän, Heberin, Sookon isän, ja Jekutielin, Saanoahin isän. Ja nämä olivat Bitjan, faraon tyttären, poikia, jonka Mered oli ottanut vaimokseen.
\par 19 Ja Hoodian vaimon, Nahamin sisaren, pojat olivat Kegilan isä, garmilainen, ja maakatilainen Estemoa.
\par 20 Ja Siimonin pojat olivat Amnon ja Rinna, Ben-Haanan ja Tiilon. Ja Jisin pojat olivat Soohet ja Soohetin poika.
\par 21 Seelan, Juudan pojan, pojat olivat Eer, Leekan isä, Lada, Maaresan isä, ja pellavakankaiden kutojain suvut Beet-Asbeasta;
\par 22 vielä Jookim ja Kooseban miehet sekä Jooas ja Saaraf, jotka hallitsivat Mooabia, ja Jaasubi-Lehem. Mutta nämä ovat vanhoja asioita.
\par 23 He olivat niitä savenvalajia, jotka asuvat Netaimissa ja Gederassa; he asuivat siellä kuninkaan luona ja olivat hänen työssään.
\par 24 Simeonin pojat olivat Nemuel ja Jaamin, Jaarib, Serah ja Saul;
\par 25 tämän poika oli Sallum, tämän poika Mibsam, tämän poika Misma.
\par 26 Ja Misman pojat olivat: hänen poikansa Hammuel, tämän poika Sakkur ja tämän poika Siimei.
\par 27 Ja Siimeillä oli kuusitoista poikaa ja kuusi tytärtä; mutta hänen veljillään ei ollut monta lasta, eikä koko heidän sukunsa lisääntynyt Juudan lasten määrään.
\par 28 Ja he asuivat Beersebassa, Mooladassa ja Hasar-Suualissa,
\par 29 Bilhassa, Esemissä ja Tooladissa,
\par 30 Betuelissa, Hormassa ja Siklagissa,
\par 31 Beet-Markabotissa, Hasar-Suusimissa, Beet-Birissä ja Saaraimissa. Nämä olivat heidän kaupunkinsa Daavidin hallitukseen asti.
\par 32 Ja heidän kylänsä olivat Eetam, Ain, Rimmon, Tooken ja Aasan - viisi kaupunkia;
\par 33 siihen lisäksi kaikki heidän kylänsä, jotka olivat näiden kaupunkien ympärillä, aina Baaliin asti. Nämä olivat heidän asuinsijansa; ja heillä oli omat sukuluettelonsa.
\par 34 Vielä: Mesobab, Jamlek, Joosa, Amasjan poika,
\par 35 Jooel, Jeehu, Joosibjan poika, joka oli Serajan poika, joka Asielin poika;
\par 36 Eljoenai, Jaakoba, Jesohaja, Asaja, Adiel, Jesimiel, Benaja
\par 37 ja Siisa, Sifin poika, joka oli Allonin poika, joka Jedajan poika, joka Simrin poika, joka Semajan poika.
\par 38 Nämä nimeltä mainitut olivat sukujensa ruhtinaita, ja heidän perhekuntansa olivat levinneet ja lisääntyneet.
\par 39 Ja he kulkivat Gedoriin päin, aina laakson itäiseen osaan saakka, hakeakseen laidunta lampailleen.
\par 40 Ja he löysivät lihavan ja hyvän laitumen, ja maa oli tilava joka suuntaan sekä rauhallinen ja levollinen, sillä ne, jotka olivat asuneet siellä ennen, olivat haamilaisia.
\par 41 Mutta nämä nimeltä luetellut tulivat Hiskian, Juudan kuninkaan, päivinä ja valtasivat heidän telttansa ja voittivat meunilaiset, jotka olivat siellä, ja vihkivät heidät tuhon omiksi, aina tähän päivään asti, ja asettuivat heidän sijaansa, sillä siellä oli laidunta heidän lampaillensa.
\par 42 Ja heistä, simeonilaisista, meni viisisataa miestä Seirin vuoristoon, ja heidän etunenässään olivat Pelatja, Nearja, Refaja ja Ussiel, Jisin pojat.
\par 43 Ja he surmasivat ne, jotka olivat jäljellä pelastuneista amalekilaisista, ja jäivät sinne asumaan aina tähän päivään asti.

\chapter{5}

\par 1 Ruubenin, Israelin esikoisen, pojat olivat - hän oli näet esikoinen, mutta koska hän saastutti isänsä vuoteen, annettiin hänen esikoisuutensa Joosefin, Israelin pojan, pojille, kuitenkin niin, ettei heitä merkitty sukuluetteloon esikoisina,
\par 2 sillä Juuda tuli voimallisimmaksi veljiensä joukossa, ja yhdestä hänen jälkeläisistään tuli ruhtinas, mutta esikoisuus joutui Joosefille -
\par 3 Ruubenin, Israelin esikoisen, pojat olivat Hanok ja Pallu, Hesron ja Karmi.
\par 4 Jooelin pojat olivat: hänen poikansa Semaja, tämän poika Goog, tämän poika Siimei,
\par 5 tämän poika Miika, tämän poika Reaja, tämän poika Baal
\par 6 ja tämän poika Beera, jonka Assurin kuningas Tillegat-Pilneser vei pakkosiirtolaisuuteen; hän oli ruubenilaisten ruhtinas.
\par 7 Ja hänen veljensä olivat sukujensa mukaan, kun heidät merkittiin sukuluetteloon heidän polveutumisensa mukaan: Jegiel, päämies, Sakarja,
\par 8 Bela, Aasaan poika, joka oli Seman poika, joka Jooelin poika; hän asui Aroerissa ja aina Neboon ja Baal-Meoniin asti.
\par 9 Ja idässä hän asui erämaahan saakka, joka alkaa Eufrat-virrasta; sillä heillä oli suuria karjalaumoja Gileadin maassa.
\par 10 Mutta Saulin päivinä he kävivät sotaa hagrilaisia vastaan, ja kun nämä olivat joutuneet heidän käsiinsä, asettuivat he heidän telttoihinsa Gileadin koko itäistä rajaseutua myöten.
\par 11 Gaadilaiset asuivat heidän kanssaan rajakkain Baasanin maassa Salkaan saakka:
\par 12 Jooel päämiehenä ja Saafam toisena, sitten Janai ja Saafat Baasanissa.
\par 13 Ja heidän veljensä olivat, perhekuntiensa mukaan: Miikael, Mesullam, Seba, Joorai, Jakan, Siia ja Eeber, kaikkiaan seitsemän.
\par 14 Nämä olivat Abihailin pojat, joka oli Huurin poika, joka Jaaroahin poika, joka Gileadin poika, joka Miikaelin poika, joka Jesisain poika, joka Jahdon poika, joka Buusin poika.
\par 15 Ahi, Abdielin poika, joka oli Guunin poika, oli heidän perhekuntiensa päämies.
\par 16 Ja he asuivat Gileadissa, Baasanissa ja sen tytärkaupungeissa sekä kaikilla Saaronin laidunmailla, niin kauas kuin ne ulottuivat.
\par 17 Kaikki nämä merkittiin sukuluetteloon Juudan kuninkaan Jootamin ja Israelin kuninkaan Jerobeamin päivinä.
\par 18 Ruubenilaiset, gaadilaiset ja toinen puoli Manassen sukukuntaa - sotakuntoisia miehiä, jotka kantoivat kilpeä ja miekkaa ja jännittivät jousta ja olivat taisteluun harjaantuneita, neljäkymmentäneljä tuhatta seitsemänsataa kuusikymmentä sotakelpoista miestä -
\par 19 kävivät sotaa hagrilaisia, Jeturia, Naafista ja Noodabia vastaan.
\par 20 Ja he saivat apua näitä vastaan, niin että hagrilaiset ja kaikki, jotka olivat heidän kanssaan, joutuivat heidän käsiinsä; sillä he huusivat taistelussa Jumalan puoleen, ja hän kuuli heidän rukouksensa, koska he turvasivat häneen.
\par 21 Ja he veivät saaliinaan heidän karjansa, viisikymmentä tuhatta kamelia, kaksisataa viisikymmentä tuhatta lammasta ja kaksituhatta aasia, sekä satatuhatta ihmistä.
\par 22 Paljon oli surmattuina kaatuneita, sillä sota oli Jumalan sota. Ja he asettuivat asumaan heidän sijaansa aina pakkosiirtolaisuuteen saakka.
\par 23 Toinen puoli Manassen sukukuntaa asui siinä maassa, Baasanista aina Baal-Hermoniin ja Seniriin ja Hermon-vuoreen saakka; heitä oli paljon.
\par 24 Ja nämä olivat heidän perhekuntiensa päämiehet: Eefer, Jisi, Eliel, Asriel, Jeremia, Hoodavja ja Jahdiel, sotaurhoja, kuuluisia miehiä, perhekuntiensa päämiehiä.
\par 25 Mutta he tulivat uskottomiksi isiensä Jumalaa kohtaan ja juoksivat haureudessa maan kansojen jumalain jäljessä, niiden kansojen, jotka Jumala oli hävittänyt heidän edestänsä.
\par 26 Niin Israelin Jumala herätti Puulin, Assurin kuninkaan, hengen ja Tillegat-Pilneserin, Assurin kuninkaan, hengen ja kuljetutti heidät, ruubenilaiset, gaadilaiset ja toisen puolen Manassen sukukuntaa, pakkosiirtolaisuuteen ja antoi viedä heidät Halahiin, Haaboriin, Haaraan ja Goosanin joen rannoille, missä he tänäkin päivänä ovat.

\chapter{6}

\par 1 Leevin pojat olivat Geerson, Kehat ja Merari.
\par 2 Ja Kehatin pojat olivat Amram, Jishar, Hebron ja Ussiel.
\par 3 Ja Amramin lapset olivat Aaron, Mooses ja Mirjam. Ja Aaronin pojat olivat Naadab, Abihu, Eleasar ja Iitamar.
\par 4 Eleasarille syntyi Piinehas; Piinehaalle syntyi Abisua.
\par 5 Ja Abisualle syntyi Bukki, ja Bukille syntyi Ussi.
\par 6 Ja Ussille syntyi Serahja, ja Serahjalle syntyi Merajot.
\par 7 Merajotille syntyi Amarja, ja Amarjalle syntyi Ahitub.
\par 8 Ja Ahitubille syntyi Saadok, ja Saadokille syntyi Ahimaas.
\par 9 Ja Ahimaasille syntyi Asarja, ja Asarjalle syntyi Joohanan.
\par 10 Ja Joohananille syntyi Asarja, hän joka toimitti papinvirkaa temppelissä, jonka Salomo rakennutti Jerusalemiin.
\par 11 Ja Asarjalle syntyi Amarja, ja Amarjalle syntyi Ahitub.
\par 12 Ja Ahitubille syntyi Saadok, ja Saadokille syntyi Sallum.
\par 13 Ja Sallumille syntyi Hilkia, ja Hilkialle syntyi Asarja.
\par 14 Ja Asarjalle syntyi Seraja, ja Serajalle syntyi Jehosadak.
\par 15 Mutta Jehosadakin oli lähdettävä mukaan silloin, kun Herra antoi Nebukadnessarin viedä Juudan ja Jerusalemin pakkosiirtolaisuuteen.
\par 16 Leevin pojat olivat Geersom, Kehat ja Merari.
\par 17 Ja nämä ovat Geersomin poikien nimet: Libni ja Siimei.
\par 18 Ja Kehatin pojat olivat Amram, Jishar, Hebron ja Ussiel.
\par 19 Merarin pojat olivat Mahli ja Muusi. Nämä olivat leeviläisten suvut heidän isiensä mukaan.
\par 20 Geersomista polveutuivat hänen poikansa Libni, tämän poika Jahat, tämän poika Simma,
\par 21 tämän poika Jooah, tämän poika Iddo, tämän Serah ja tämän poika Jeatrai.
\par 22 Kehatin pojat olivat: hänen poikansa Amminadab, tämän poika Koorah, tämän poika Assir,
\par 23 tämän poika Elkana, tämän poika Ebjasaf, tämän poika Assir,
\par 24 tämän poika Tahat, tämän poika Uuriel, tämän poika Ussia ja tämän poika Saul.
\par 25 Ja Elkanan pojat olivat Amasai ja Ahimot.
\par 26 Elkana: Elkanan pojat, Suufai, tämän poika Nahat,
\par 27 tämän poika Eliab, tämän poika Jeroham, tämän poika Elkana.
\par 28 Ja Samuelin pojat olivat esikoinen Vasni ja Abia.
\par 29 Merarin pojat olivat: Mahli, tämän poika Libni, tämän poika Siimei, tämän poika Ussa,
\par 30 tämän poika Simea, tämän poika Haggia ja tämän poika Asaja.
\par 31 Ja nämä ovat ne, jotka Daavid asetti pitämään huolta laulusta Herran temppelissä, senjälkeen kuin arkki oli saanut leposijan.
\par 32 He palvelivat veisaajina ilmestysmajan asumuksen edessä, kunnes Salomo rakensi Herran temppelin Jerusalemiin; he toimittivat virkaansa, niinkuin heille oli säädetty.
\par 33 Ja nämä ovat ne, jotka palvelivat, ja nämä heidän poikansa: Kehatilaisia: Heeman, veisaaja, Jooelin poika, joka oli Samuelin poika,
\par 34 joka Elkanan poika, joka Jerohamin poika, joka Elielin poika, joka Tooahin poika,
\par 35 joka Suufin poika, joka Elkanan poika, joka Mahatin poika, joka Amasain poika,
\par 36 joka Elkanan poika, joka Jooelin poika, joka Asarjan poika, joka Sefanjan poika,
\par 37 joka Tahatin poika, joka Assirin poika, joka Ebjasafin poika, joka Koorahin poika,
\par 38 joka Jisharin poika, joka Kehatin poika, joka Leevin poika, joka Israelin poika.
\par 39 Vielä hänen veljensä Aasaf, joka seisoi hänen oikealla puolellansa, Aasaf, Berekjan poika, joka oli Simean poika,
\par 40 joka Miikaelin poika, joka Baasejan poika, joka Malkian poika,
\par 41 joka Etnin poika, joka Serahin poika, joka Adajan poika,
\par 42 joka Eetanin poika, joka Simman poika, joka Siimein poika,
\par 43 joka Jahatin poika, joka Geersomin poika, joka Leevin poika.
\par 44 Ja heidän veljensä, Merarin pojat, vasemmalla puolella: Eetan, Kiisin poika, joka oli Abdin poika, joka Mallukin poika,
\par 45 joka Hasabjan poika, joka Amasjan poika, joka Hilkian poika,
\par 46 joka Amsin poika, joka Baanin poika, joka Semerin poika,
\par 47 joka Mahlin poika, joka Muusin poika, joka Merarin poika, joka Leevin poika.
\par 48 Ja heidän veljensä, leeviläiset, olivat annetut toimittamaan kaikkinaista palvelusta Jumalan temppeli-asumuksessa.
\par 49 Mutta Aaron ja hänen poikansa polttivat uhreja polttouhrialttarilla ja suitsutusalttarilla, toimittivat kaikki askareet kaikkeinpyhimmässä ja Israelin sovituksen, aivan niinkuin Jumalan palvelija Mooses oli käskenyt.
\par 50 Ja nämä olivat Aaronin pojat: hänen poikansa Eleasar, tämän poika Piinehas, tämän poika Abisua,
\par 51 tämän poika Bukki, tämän poika Ussi, tämän poika Serahja,
\par 52 tämän poika Merajot, tämän poika Amarja, tämän poika Ahitub,
\par 53 tämän poika Saadok, tämän poika Ahimaas.
\par 54 Ja nämä olivat heidän asuinsijansa, heidän leiripaikkojensa mukaan, heidän alueellaan: Niille Aaronin jälkeläisille, jotka olivat kehatilaisten sukua, sillä he saivat arpaosan,
\par 55 annettiin Hebron Juudan maasta, ympärillä olevine laidunmaineen.
\par 56 Mutta kaupungin peltomaat kylineen annettiin Kaalebille, Jefunnen pojalle.
\par 57 Aaronin jälkeläisille annettiin turvakaupungit Hebron, Libna laidunmaineen, Jattir, Estemoa laidunmaineen,
\par 58 Hiilen laidunmaineen, Debir laidunmaineen,
\par 59 Aasan laidunmaineen ja Beet-Semes laidunmaineen.
\par 60 Ja Benjaminin sukukunnasta Geba laidunmaineen, Aalemet laidunmaineen ja Anatot laidunmaineen. Heidän kaupunkejaan oli kaikkiaan kolmetoista kaupunkia, heidän sukujensa mukaan.
\par 61 Ja muut kehatilaiset saivat arvalla sukujensa mukaan Efraimin sukukunnalta, Daanin sukukunnalta ja toiselta puolelta Manassen sukukuntaa kymmenen kaupunkia.
\par 62 Ja geersomilaiset saivat sukujensa mukaan Isaskarin sukukunnalta, Asserin sukukunnalta, Naftalin sukukunnalta ja Manassen sukukunnalta Baasanista kolmetoista kaupunkia.
\par 63 Merarilaiset saivat arvalla sukujensa mukaan Ruubenin sukukunnalta, Gaadin sukukunnalta ja Sebulonin sukukunnalta kaksitoista kaupunkia.
\par 64 Näin israelilaiset antoivat leeviläisille nämä kaupungit laidunmaineen.
\par 65 He antoivat arvalla Juudan lasten sukukunnasta, simeonilaisten sukukunnasta ja benjaminilaisten sukukunnasta nämä nimeltä mainitut kaupungit.
\par 66 Kehatilaisten suvuista muutamat saivat arvalla Efraimin sukukunnalta alueekseen seuraavat kaupungit:
\par 67 heille annettiin turvakaupungit Sikem laidunmaineen Efraimin vuoristosta, Geser laidunmaineen,
\par 68 Jokmeam laidunmaineen, Beet-Hooron laidunmaineen,
\par 69 Aijalon laidunmaineen ja Gat-Rimmon laidunmaineen;
\par 70 ja toisesta puolesta Manassen sukukuntaa Aaner laidunmaineen ja Bileam laidunmaineen. Nämä tulivat muitten kehatilaisten suvuille.
\par 71 Geersomilaiset saivat sukujensa mukaan toiselta puolelta Manassen sukukuntaa Goolanin laidunmaineen Baasanista ja Astarotin laidunmaineen;
\par 72 ja Isaskarin sukukunnalta Kedeksen laidunmaineen, Dobratin laidunmaineen,
\par 73 Raamotin laidunmaineen ja Aanemin laidunmaineen;
\par 74 ja Asserin sukukunnalta Maasalin laidunmaineen, Abdonin laidunmaineen,
\par 75 Huukokin laidunmaineen ja Rehobin laidunmaineen;
\par 76 ja Naftalin sukukunnalta Kedeksen laidunmaineen Galileasta, Hammonin laidunmaineen ja Kirjataimin laidunmaineen.
\par 77 Muut merarilaiset saivat Sebulonin sukukunnalta Rimmonin laidunmaineen ja Taaborin laidunmaineen;
\par 78 ja tuolta puolelta Jerikon Jordanin, Jordanista auringonnousuun päin, Ruubenin sukukunnalta Beserin laidunmaineen erämaasta, Jahaan laidunmaineen,
\par 79 Kedemotin laidunmaineen, Meefaatin laidunmaineen;
\par 80 ja Gaadin sukukunnalta Raamotin laidunmaineen Gileadista, Mahanaimin laidunmaineen,
\par 81 Hesbonin laidunmaineen ja Jaeserin laidunmaineen.

\chapter{7}

\par 1 Isaskarin pojat olivat Toola, Puua, Jaasub ja Simron, kaikkiaan neljä.
\par 2 Ja Toolan pojat olivat Ussi, Refaja, Jeriel, Jahmai, Jibsam ja Samuel, jotka olivat perhekuntiensa päämiehiä, Toolan jälkeläisiä, sotaurhoja, polveutumisensa mukaan. Daavidin aikana oli heidän lukumääränsä kaksikymmentäkaksi tuhatta kuusisataa.
\par 3 Ja Ussin poika oli Jisrahja, ja Jisrahjan pojat olivat Miikael, Obadja, Jooel ja Jissia, kaikkiaan viisi, kaikki päämiehiä.
\par 4 Heitä seurasi polveutumisensa ja perhekuntiensa mukaan taisteluun valmiina sotajoukkoina kolmekymmentäkuusi tuhatta miestä, sillä heillä oli paljon vaimoja ja lapsia.
\par 5 Ja heidän veljensä, kaikissa Isaskarin suvuissa, olivat sotaurhoja. Sukuluetteloihin merkittyjä oli kaikkiaan kahdeksankymmentäseitsemän tuhatta.
\par 6 Benjaminia oli: Bela, Beker ja Jediael, kaikkiaan kolme.
\par 7 Ja Belan pojat olivat Esbon, Ussi, Ussiel, Jerimot ja Iiri, kaikkiaan viisi, perhekunta-päämiehiä, sotaurhoja. Ja sukuluetteloihin merkittyjä oli heitä kaksikymmentäkaksi tuhatta kolmekymmentä neljä.
\par 8 Ja Bekerin pojat olivat Semira, Jooas, Elieser, Eljoenai, Omri, Jeremot, Abia, Anatot ja Aalemet; kaikki nämä olivat Bekerin poikia.
\par 9 Ja sukuluetteloihin merkittyjä oli heitä polveutumisensa mukaan, perhekunta-päämiestensä, sotaurhojen, mukaan, kaksikymmentä tuhatta kaksisataa.
\par 10 Ja Jediaelin poika oli Bilhan, ja Bilhanin pojat olivat Jeus, Benjamin, Eehud, Kenaana, Seetan, Tarsis ja Ahisahar.
\par 11 Kaikki nämä olivat Jediaelin poikia perhekunta-päämiesten, sotaurhojen, mukaan; seitsemäntoista tuhatta kaksisataa sotakelpoista miestä.
\par 12 Ja Suppim ja Huppim olivat Iirin poikia; Huusim oli Aherin poika.
\par 13 Naftalin pojat olivat Jahasiel, Guuni, Jeeser ja Sallum, Bilhan jälkeläisiä.
\par 14 Manassen poika oli Asriel, jonka hänen aramilainen sivuvaimonsa synnytti; tämä synnytti Maakirin, Gileadin isän.
\par 15 Ja Maakir otti vaimon Huppimille ja Suppimille, ja hänen sisarensa nimi oli Maaka. Ja toisen nimi oli Selofhad; ja Selofhadilla oli tyttäriä.
\par 16 Ja Maaka, Maakirin vaimo, synnytti pojan, jolle hän antoi nimen Peres; hänen veljensä nimi oli Seres, ja hänen poikansa olivat Uulam ja Rekem.
\par 17 Uulamin poika oli Bedan. Nämä olivat Gileadin pojat, joka oli Maakirin poika, joka Manassen poika.
\par 18 Ja hänen sisarensa Mooleket synnytti Iishodin, Abieserin ja Mahlan.
\par 19 Ja Semidan pojat olivat Ahjan, Sekem, Likhi ja Aniam.
\par 20 Efraimin pojat olivat: Suutelah, tämän poika Bered, tämän poika Tahat, tämän poika Elada, tämän poika Tahat,
\par 21 tämän poika Saabad, tämän poika Suutelah, sekä Eser ja Elead. Ja Gatin miehet, siinä maassa syntyneet, tappoivat heidät, koska he olivat lähteneet ottamaan heidän karjaansa.
\par 22 Ja heidän isänsä Efraim suri pitkät ajat, ja hänen veljensä tulivat lohduttamaan häntä.
\par 23 Sitten hän yhtyi vaimoonsa, ja tämä tuli raskaaksi ja synnytti pojan; ja hän antoi tälle nimen Beria, koska se oli tapahtunut hänen perheensä onnettomuuden aikana.
\par 24 Ja hänen tyttärensä oli Seera; tämä rakensi Ala- ja Ylä-Beet-Hooronin sekä Ussen-Seeran.
\par 25 Ja Berian poika oli Refah, samoin Resef, tämän poika Telah, tämän poika Tahan,
\par 26 tämän poika Ladan, tämän poika Ammihud, tämän poika Elisama,
\par 27 tämän poika Nuun ja tämän poika Joosua.
\par 28 Ja heidän perintömaansa ja asuinsijansa olivat Beetel ja sen tytärkaupungit, itään päin Naaran ja länteen päin Geser ja sen tytärkaupungit, sekä Sikem ja sen tytärkaupungit, aina Aijaan ja sen tytärkaupunkeihin asti.
\par 29 Ja manasselaisten hallussa olivat Beet-Sean ja sen tytärkaupungit, Taanak ja sen tytärkaupungit, Megiddo ja sen tytärkaupungit, Door ja sen tytärkaupungit. Näissä asuivat Joosefin, Israelin pojan, jälkeläiset.
\par 30 Asserin pojat olivat Jimna, Jisva, Jisvi ja Beria; heidän sisarensa oli Serah.
\par 31 Berian pojat olivat Heber ja Malkiel; tämä oli Birsaitin isä.
\par 32 Ja Heberille syntyi Jaflet, Soomer, Hootam ja heidän sisarensa Suua.
\par 33 Ja Jafletin pojat olivat Paasak, Bimhal ja Asva; nämä ovat Jafletin pojat.
\par 34 Ja Semerin pojat olivat Ahi, Rohga, Hubba ja Aram.
\par 35 Ja hänen veljensä Heelemin pojat olivat Soofah, Jimna, Seeles ja Aamal.
\par 36 Soofahin pojat olivat Suuah, Harnefer, Suual, Beeri, Jimra,
\par 37 Beser, Hood, Samma, Silsa, Jitran ja Beera.
\par 38 Ja Jeterin pojat olivat Jefunne, Fispa ja Ara.
\par 39 Ja Ullan pojat olivat Aarah, Hanniel ja Risja.
\par 40 Kaikki nämä olivat Asserin jälkeläisiä, perhekunta-päämiehiä, valittuja sotaurhoja, ruhtinasten päämiehiä. Ja heidän sukuluetteloihin merkittyjen sotakelpoisten miestensä lukumäärä oli kaksikymmentäkuusi tuhatta.

\chapter{8}

\par 1 Benjaminille syntyi esikoisena Bela, toisena Asbel, kolmantena Ahrah,
\par 2 neljäntenä Nooha ja viidentenä Raafa.
\par 3 Ja Belalla oli pojat: Addar, Geera, Abihud,
\par 4 Abisua, Naaman, Ahoah,
\par 5 Geera, Sefufan ja Huuram.
\par 6 Ja nämä olivat Eehudin pojat - nämä olivat Geban asukkaiden perhekunta-päämiehet. Heidät siirrettiin pois Manahatiin,
\par 7 niin myös Naaman, Ahia ja Geera; hän siirsi heidät pois. Hänelle syntyi Ussa ja Ahihud.
\par 8 Ja Saharaimille syntyi Mooabin maassa, sittenkuin hän oli hyljännyt vaimonsa Huusimin ja Baaran,
\par 9 hänelle syntyi hänen vaimostaan Hoodeksesta Joobab, Sibja, Meesa, Malkam,
\par 10 Jeus, Sokja ja Mirma; nämä olivat hänen poikiansa, perhekunta-päämiehiä.
\par 11 Ja Huusimista oli hänelle syntynyt Abitub ja Elpaal.
\par 12 Ja Elpaalin pojat olivat Eeber, Misam ja Semed. Tämä rakensi Oonon ja Loodin ynnä sen tytärkaupungit.
\par 13 Beria ja Sema olivat Aijalonin asukasten perhekunta-päämiehiä; ja he karkoittivat Gatin asukkaat,
\par 14 he ja Ahjo, Saasak ja Jeremot.
\par 15 Ja Sebadja, Arad, Eder,
\par 16 Miikael, Jispa ja Jooha olivat Berian poikia.
\par 17 Ja Sebadja, Mesullam, Hiski, Heber,
\par 18 Jismerai, Jislia ja Joobab olivat Elpaalin poikia.
\par 19 Ja Jaakim, Sikri, Sabdi,
\par 20 Elienai, Silletai, Eliel,
\par 21 Adaja, Beraja ja Simrat olivat Siimein poikia.
\par 22 Ja Jispan, Eder, Eliel,
\par 23 Abdon, Sikri, Haanan,
\par 24 Hananja, Eelam, Antotia,
\par 25 Jifdeja ja Penuel olivat Saasakin poikia.
\par 26 Ja Samserai, Seharja, Atalja,
\par 27 Jaaresja, Elia ja Sikri olivat Jerohamin poikia.
\par 28 Nämä olivat perhekunta-päämiehiä, päämiehiä polveutumisensa mukaan; he asuivat Jerusalemissa.
\par 29 Gibeonissa asui Gibeonin isä, jonka vaimon nimi oli Maaka.
\par 30 Ja hänen esikoispoikansa oli Abdon, sitten Suur, Kiis, Baal, Naadab,
\par 31 Gedor, Ahjo ja Seker.
\par 32 Ja Miklotille syntyi Simea. Hekin asuivat veljinensä Jerusalemissa, rajakkain veljiensä kanssa.
\par 33 Ja Neerille syntyi Kiis, Kiisille syntyi Saul, ja Saulille syntyi Joonatan, Malkisua, Abinadab ja Esbaal.
\par 34 Ja Joonatanin poika oli Meribbaal, ja Meribbaalille syntyi Miika.
\par 35 Ja Miikan pojat olivat Piiton, Melek, Tarea ja Aahas.
\par 36 Ja Aahaalle syntyi Jooadda, Jooaddalle syntyi Aalemet, Asmavet ja Simri. Ja Simrille syntyi Moosa.
\par 37 Ja Moosalle syntyi Binea; hänen poikansa oli Raafa, tämän poika Elasa ja tämän poika Aasel.
\par 38 Ja Aaselilla oli kuusi poikaa, ja nämä ovat heidän nimensä: Asrikam, Bookeru, Jismael, Searja, Obadja ja Haanan. Nämä kaikki olivat Aaselin poikia.
\par 39 Ja Eesekin, hänen veljensä, pojat olivat: hänen esikoisensa Uulam, toinen Jeus ja kolmas Elifelet.
\par 40 Ja Uulamin pojat olivat sotaurhoja, jousen jännittäjiä, ja heillä oli paljon poikia ja poikien poikia, sataviisikymmentä. Nämä kaikki ovat benjaminilaisia.

\chapter{9}

\par 1 Koko Israel merkittiin sukuluetteloon, ja katso, he ovat kirjoitetut Israelin kuningasten kirjaan. Ja Juuda vietiin Baabeliin pakkosiirtolaisuuteen uskottomuutensa tähden.
\par 2 Entiset asukkaat, jotka elivät perintömaillaan, kaupungeissaan, olivat israelilaisia, pappeja, leeviläisiä ja temppelipalvelijoita.
\par 3 Jerusalemissa asui Juudan lapsia, benjaminilaisia, efraimilaisia ja manasselaisia, nimittäin:
\par 4 Uutai, Ammihudin poika, joka oli Omrin poika, joka Imrin poika, joka Baanin poika, Pereksen, Juudan pojan, jälkeläisiä;
\par 5 ja seelalaisia: Asaja, esikoinen, ja hänen poikansa;
\par 6 serahilaisia: Jeguel ja hänen veljensä, kuusisataa yhdeksänkymmentä;
\par 7 benjaminilaisia: Sallu, Mesullamin poika, joka oli Hoodavjan poika, joka Senuan poika,
\par 8 Jibneja, Jerohamin poika, Eela, Ussin poika, joka oli Mikrin poika, ja Mesullam, Sefatjan poika, joka oli Reguelin poika, joka Jibnean poika,
\par 9 sekä heidän veljensä, heidän polveutumisensa mukaan, yhdeksänsataa viisikymmentä kuusi. Kaikki nämä olivat perhekunta-päämiehiä perhekunnissaan.
\par 10 Ja pappeja: Jedaja, Joojarib, Jaakin,
\par 11 Asarja, Hilkian poika, joka oli Mesullamin poika, joka Saadokin poika, joka Merajotin poika, joka Ahitubin poika, Jumalan temppelin esimies;
\par 12 Adaja, Jerohamin poika, joka oli Pashurin poika, joka Malkian poika, ja Maesai, Adielin poika, joka oli Jahseran poika, joka Mesullamin poika, joka Mesillemitin poika, joka Immerin poika;
\par 13 sekä heidän veljensä, heidän perhekuntiensa päämiehet, tuhat seitsemänsataa kuusikymmentä miestä, kelvollisia toimittamaan palvelusta Jumalan temppelissä.
\par 14 Ja leeviläisiä: Semaja, Hassubin poika, joka oli Asrikamin poika, joka Hasabjan poika, merarilaisia;
\par 15 Bakbakkar, Heres, Gaalal, Mattanja, Miikan poika, joka oli Sikrin poika, joka Aasafin poika;
\par 16 Obadja, Semajan poika, joka oli Gaalalin poika, joka Jedutunin poika; ja Berekja, Aasan poika, joka oli Elkanan poika, hänen, joka asui netofalaisten kylissä.
\par 17 Ja ovenvartijat: Sallum, Akkub, Talmon ja Ahiman veljineen: Sallum oli päämies,
\par 18 ja tähän saakka hän on ollut vartijana Kuninkaan portilla, idän puolella. Nämä olivat ovenvartijoina leeviläisten leirissä.
\par 19 Mutta Sallumilla, Kooren pojalla, joka oli Ebjasafin poika, joka Koorahin poika, ja hänen veljillään, jotka olivat hänen perhekuntaansa - koorahilaisilla, oli palvelustehtävänä majan kynnyksien vartioiminen; heidän isänsä olivat näet vartioineet sisäänkäytävää Herran leirissä.
\par 20 Ja Piinehas, Eleasarin poika, oli muinoin heidän esimiehenään - Herra olkoon hänen kanssaan! -
\par 21 Sakarja, Meselemjan poika, oli ovenvartijana ilmestysmajan ovella.
\par 22 Kaikkiaan oli kynnyksien ovenvartijoiksi valittuja kaksisataa kaksitoista; he olivat kylissään merkityt sukuluetteloon. Daavid ja Samuel, näkijä, olivat asettaneet heidät heidän luottamustoimeensa.
\par 23 Niin he ja heidän poikansa olivat vartijoina Herran temppelin, telttapyhäkön, ovilla.
\par 24 Ovenvartijat olivat asetetut neljälle ilmansuunnalle: idän, lännen, pohjoisen ja etelän puolelle.
\par 25 Ja heidän veljiensä, jotka asuivat kylissään, oli aina määräaikoina seitsemäksi päiväksi tultava toimittamaan palvelusta yhdessä heidän kanssaan;
\par 26 sillä ne neljä ylintä ovenvartijaa olivat pysyväisesti luottamustoimessaan. Nämä olivat leeviläiset. Heidän oli myöskin valvottava kammioita ja Jumalan temppelin aarteita,
\par 27 ja he viettivät yönsä Jumalan temppelin ympärillä, sillä heidän tehtävänään oli vartioiminen ja heidän oli pidettävä huoli ovien avaamisesta joka aamu.
\par 28 Muutamien heistä oli pidettävä huoli jumalanpalveluksessa tarvittavasta kalustosta, sillä heidän oli tuotava se sisään täysilukuisena ja vietävä se ulos täysilukuisena.
\par 29 Ja muutamat heistä olivat määrätyt pitämään huolta kaluista, kaikista pyhistä kaluista, sekä lestyistä jauhoista, viinistä, öljystä, suitsukkeista ja hajuaineista.
\par 30 Ja muutamat pappien pojista valmistivat voiteita hajuaineista.
\par 31 Ja Mattitjalla, joka oli leeviläisiä ja oli koorahilaisen Sallumin esikoinen, oli luottamustoimena leivosten valmistaminen.
\par 32 Ja muutamien kehatilaisista, heidän veljistään, oli pidettävä huoli näkyleivistä, valmistettava ne joka sapatiksi.
\par 33 Mutta veisaajat, leeviläisten perhekuntain päämiehet, oleskelivat kammioissa, muusta palveluksesta vapaina, sillä he olivat toimessa sekä päivällä että yöllä.
\par 34 He olivat leeviläisten perhekuntain päämiehiä, päämiehiä polveutumisensa mukaan; he asuivat Jerusalemissa.
\par 35 Gibeonissa asuivat Gibeonin isä, jonka vaimon nimi oli Maaka,
\par 36 hänen esikoispoikansa Abdon, sitten Suur, Kiis, Baal, Neer, Naadab,
\par 37 Gedor, Ahjo, Sakarja ja Mikleot.
\par 38 Ja Mikleotille syntyi Simeam. Hekin asuivat veljinensä Jerusalemissa, rajakkain veljiensä kanssa.
\par 39 Ja Neerille syntyi Kiis, Kiisille syntyi Saul, ja Saulille syntyi Joonatan, Malkisua, Abinadab ja Esbaal.
\par 40 Ja Joonatanin poika oli Meribbaal, ja Meribbaalille syntyi Miika.
\par 41 Ja Miikan pojat olivat Piiton, Melek ja Tahrea.
\par 42 Ja Aahaalle syntyi Jaera, Jaeralle syntyi Aalemet, Asmavet ja Simri. Ja Simrille syntyi Moosa,
\par 43 ja Moosalle syntyi Binea; hänen poikansa oli Refaja, tämän poika Elasa ja tämän poika Aasel.
\par 44 Ja Aaselilla oli kuusi poikaa, ja nämä ovat heidän nimensä: Asrikam, Bookeru, Jismael, Searja, Obadja ja Haanan. Nämä olivat Aaselin pojat.

\chapter{10}

\par 1 Mutta filistealaiset taistelivat Israelia vastaan; ja Israelin miehet pakenivat filistealaisia, ja heitä kaatui surmattuina Gilboan vuorella.
\par 2 Ja filistealaiset pääsivät Saulin ja hänen poikiensa kintereille, ja filistealaiset surmasivat Joonatanin, Abinadabin ja Malkisuan, Saulin pojat.
\par 3 Ja kun taistelu kiihtyi ankaraksi Saulia vastaan ja jousimiehet keksivät hänet, joutui hän hätään jousimiesten ahdistaessa.
\par 4 Ja Saul sanoi aseenkantajallensa: "Paljasta miekkasi ja lävistä sillä minut, etteivät nuo ympärileikkaamattomat tulisi pitämään minua pilkkanaan". Mutta hänen aseenkantajansa ei tahtonut, sillä hän pelkäsi kovin. Niin Saul itse otti miekan ja heittäytyi siihen.
\par 5 Mutta kun hänen aseenkantajansa näki, että Saul oli kuollut, heittäytyi hänkin miekkaansa ja kuoli.
\par 6 Niin kuolivat Saul ja hänen kolme poikaansa ynnä koko hänen perheensä; he kuolivat yhdessä.
\par 7 Ja kun kaikki Israelin miehet, jotka asuivat tasangolla, huomasivat, että he olivat paenneet ja että Saul poikinensa oli kuollut, jättivät he kaupunkinsa ja pakenivat, ja filistealaiset tulivat ja asettuivat niihin.
\par 8 Seuraavana päivänä filistealaiset tulivat ryöstämään surmattuja ja löysivät Saulin ja hänen poikansa kaatuneina Gilboan vuorelta.
\par 9 Niin he ryöstivät hänet ja ottivat hänen päänsä ja hänen aseensa ja lähettivät ne ympäri filistealaisten maata, julistaaksensa voitonsanomaa epäjumalillensa ja kansalle.
\par 10 Ja he asettivat hänen aseensa jumalansa temppeliin, ja hänen pääkallonsa he kiinnittivät Daagonin temppeliin.
\par 11 Kun koko Gileadin Jaabes kuuli, mitä kaikkea filistealaiset olivat tehneet Saulille,
\par 12 nousivat he, kaikki asekuntoiset miehet, ja ottivat Saulin ja hänen poikainsa ruumiit ja toivat ne Jaabekseen. Sitten he hautasivat heidän luunsa tammen alle Jaabekseen ja paastosivat seitsemän päivää.
\par 13 Niin kuoli Saul, koska hän oli ollut uskoton Herraa kohtaan eikä ollut ottanut vaaria Herran sanasta, ja myös sentähden, että hän oli kysynyt vainajahengeltä neuvoa,
\par 14 mutta ei ollut kysynyt neuvoa Herralta. Sentähden Herra surmasi hänet ja siirsi kuninkuuden Daavidille, Iisain pojalle.

\chapter{11}

\par 1 Sitten koko Israel kokoontui Daavidin tykö Hebroniin, ja he sanoivat: "Katso, me olemme sinun luutasi ja lihaasi.
\par 2 Jo kauan sitten, Saulin vielä ollessa kuninkaana, sinä saatoit Israelin lähtemään ja tulemaan. Ja sinulle on Herra, sinun Jumalasi, sanonut: 'Sinä olet kaitseva minun kansaani Israelia, ja sinä olet oleva minun kansani Israelin ruhtinas'."
\par 3 Ja kaikki Israelin vanhimmat tulivat kuninkaan tykö Hebroniin, ja Daavid teki heidän kanssaan liiton Hebronissa Herran edessä. Ja sitten he voitelivat Daavidin Israelin kuninkaaksi sen sanan mukaan, jonka Herra oli puhunut Samuelin kautta.
\par 4 Ja Daavid ja koko Israel menivät Jerusalemiin, se on Jebukseen, ja siellä olivat jebusilaiset, jotka asuivat siinä maassa.
\par 5 Jebuksen asukkaat sanoivat Daavidille: "Tänne sinä et tule". Mutta Daavid valloitti Siionin vuorilinnan, se on Daavidin kaupungin.
\par 6 Ja Daavid sanoi: "Joka ensimmäisenä surmaa jebusilaisen, hän on oleva päämies ja päällikkö". Jooab, Serujan poika, tuli ensimmäisenä sinne ylös, ja niin hänestä tuli päämies.
\par 7 Sitten Daavid asettui vuorilinnaan; sentähden kutsuttiin sitä Daavidin kaupungiksi.
\par 8 Ja hän rakensi kaupunkia yltympäri, Millosta reunoihin asti; ja Jooab rakensi entiselleen muun osan kaupunkia.
\par 9 Ja Daavid tuli yhä suuremmaksi, ja Herra Sebaot oli hänen kanssansa.
\par 10 Nämä ovat ensimmäiset Daavidin urhoista, jotka voimakkaasti auttoivat häntä kuninkuuteen yhdessä koko Israelin kanssa, tehdäkseen hänet Israelin kuninkaaksi Herran sanan mukaan.
\par 11 Tämä on Daavidin urhojen luettelo: Jaasobeam, hakmonilaisen poika, niiden kolmen päällikkö, hän, joka heilutti keihästään kolmensadan kaatuneen yli yhdellä kertaa.
\par 12 Hänen jälkeensä Eleasar, Doodon poika, ahohilainen, yksi niistä kolmesta urhosta.
\par 13 Hän oli Daavidin kanssa Pas-Dammimissa, kun filistealaiset olivat kokoontuneet sinne sotimaan. Ja siellä oli peltopalsta täynnä ohraa. Ja väki pakeni filistealaisia;
\par 14 mutta he asettuivat keskelle palstaa, saivat sen pelastetuksi ja voittivat filistealaiset; ja niin Herra antoi suuren voiton.
\par 15 Kerran lähti kolme niistä kolmestakymmenestä päälliköstä Daavidin luo kalliolinnaan, Adullamin luolalle. Ja filistealaisten joukko oli leiriytynyt Refaimin tasangolle.
\par 16 Mutta Daavid oli silloin vuorilinnassa, ja filistealaisten vartiosto oli Beetlehemissä.
\par 17 Ja Daavidin rupesi tekemään mieli vettä, ja hän sanoi: "Jospa joku toisi minulle vettä juodakseni Beetlehemin kaivosta, joka on portin edustalla!"
\par 18 Silloin murtautuivat nämä kolme filistealaisten leirin läpi ja ammensivat vettä Beetlehemin kaivosta portin edustalta, kantoivat ja toivat sen Daavidille. Mutta Daavid ei tahtonut sitä juoda, vaan vuodatti sen juomauhriksi Herralle
\par 19 ja sanoi: "Pois se! Jumala varjelkoon minut sitä tekemästä. Joisinko minä näiden miesten verta, heidän henkeänsä, sillä henkensä uhalla he toivat sen." Eikä hän tahtonut juoda sitä. Tämän tekivät ne kolme urhoa.
\par 20 Absai, Jooabin veli, oli niiden kolmen päällikkö; hän heilutti keihästään kolmensadan kaatuneen yli. Ja hän oli kuulu niiden kolmen joukossa.
\par 21 Hän oli kahdenkertaisessa arvossa pidetty niiden kolmen joukossa ja oli heidän päämiehensä, mutta hän ei vetänyt vertoja niille kolmelle.
\par 22 Benaja, Joojadan poika, joka oli urhoollisen ja suurista teoistaan kuuluisan miehen poika, oli kotoisin Kabseelista. Hän surmasi ne kaksi mooabilaista sankaria, ja hän laskeutui alas ja tappoi lumituiskun aikana leijonan kaivoon.
\par 23 Myöskin surmasi hän egyptiläisen miehen, sen suurikasvuisen, viittä kyynärää pitkän miehen. Egyptiläisellä oli kädessä keihäs, joka oli niinkuin kangastukki, mutta hän meni häntä vastaan ainoastaan sauva kädessä. Ja hän tempasi keihään egyptiläisen kädestä ja tappoi hänet hänen omalla keihäällään.
\par 24 Tällaisia teki Benaja, Joojadan poika. Ja hän oli kuulu niiden kolmen urhon joukossa.
\par 25 Hän oli arvossa pidetty niiden kolmenkymmenen joukossa, mutta hän ei vetänyt vertoja niille kolmelle. Ja Daavid asetti hänet henkivartiostonsa päälliköksi.
\par 26 Sotaurhot olivat: Asael, Jooabin veli; Elhanan, Doodon poika, Beetlehemistä;
\par 27 harorilainen Sammot; pelonilainen Heeles;
\par 28 tekoalainen Iira, Ikkeksen poika; anatotilainen Abieser;
\par 29 huusalainen Sibbekai; ahohilainen Iilai;
\par 30 netofalainen Maharai; netofalainen Heeled, Baanan poika;
\par 31 Iitai, Riibain poika, benjaminilaisten Gibeasta, piratonilainen Benaja;
\par 32 Huurai Nahale-Gaasista; arabalainen Abiel;
\par 33 baharumilainen Asmavet; saalbonilainen Eljahba;
\par 34 gisonilainen Bene-Haasem; hararilainen Joonatan, Saagen poika;
\par 35 hararilainen Abiam, Saakarin poika; Elifal, Uurin poika;
\par 36 mekeralainen Heefer; pelonilainen Ahia;
\par 37 karmelilainen Hesro; Naarai, Esbain poika;
\par 38 Jooel, Naatanin veli; Mibhar, Hagrin poika;
\par 39 ammonilainen Selek; beerotilainen Nahrai, Jooabin, Serujan pojan aseenkantaja;
\par 40 jeteriläinen Iira; jeteriläinen Gaareb;
\par 41 heettiläinen Uuria; Saabad, Ahlain poika;
\par 42 ruubenilainen Adina, Siisan poika, ruubenilaisten päämies, ja hänen kanssaan kolmekymmentä muuta;
\par 43 Haanan, Maakan poika, ja mitniläinen Joosafat;
\par 44 astarotilainen Ussia; Saama ja Jegiel, aroerilaisen Hootamin pojat;
\par 45 Jediael, Simrin poika, ja hänen veljensä Jooha, tiisiläinen;
\par 46 Eliel-Mahavim sekä Jeribai ja Joosavja, Elnaamin pojat, ja mooabilainen Jitma;
\par 47 Eliel, Oobed ja Jaasiel-Mesobaja.

\chapter{12}

\par 1 Ja nämä olivat ne, jotka tulivat Daavidin luo Siklagiin, kun hän vielä piileskeli Saulia, Kiisin poikaa; ja he olivat niitä urhoja, jotka auttoivat häntä sodassa.
\par 2 He olivat asestetut jousella ja taitavia sekä oikealla että vasemmalla kädellä linkoamaan kiviä ja ampumaan jousella nuolia. Saulin heimolaisia, benjaminilaisia:
\par 3 Ahieser, päämies, ja Jooas, gibealaisen Hassemaan pojat; Jesiel ja Pelet, Asmavetin pojat; Beraka; anatotilainen Jeehu;
\par 4 gibeonilainen Jismaja, urho niiden kolmenkymmenen joukossa ja niiden kolmenkymmenen päällikkö; Jeremia; Jahasiel; Joohanan; gederalainen Joosabad;
\par 5 Elusai; Jerimot; Bealja; Semarja; harufilainen Sefatja;
\par 6 koorahilaiset Elkana, Jissia, Asarel, Jooeser ja Jaasobeam;
\par 7 Jooela ja Sebadja, Jerohamin pojat, Gedorista.
\par 8 Gaadilaisista meni Daavidin puolelle vuorilinnaan, erämaahan, sotaurhoja, sotakelpoisia miehiä, kilpeä ja keihästä käyttäviä, jotka olivat näöltään kuin leijonat ja nopeat kuin gasellit vuorilla:
\par 9 Eeser, päämies Obadja toinen, Eliab kolmas,
\par 10 Masmanna neljäs, Jeremia viides,
\par 11 Attai kuudes, Eliel seitsemäs,
\par 12 Joohanan kahdeksas, Elsabad yhdeksäs,
\par 13 Jeremia kymmenes, Makbannai yhdestoista.
\par 14 Nämä olivat gaadilaisia, sotajoukon päälliköitä, pienin sadan, suurin tuhannen miehen veroinen.
\par 15 Nämä olivat ne, jotka ensimmäisessä kuussa menivät Jordanin poikki, kun se tulvi yli kaikkien äyräittensä, ja jotka karkoittivat kaikki tasangoilla-asujat itään ja länteen.
\par 16 Myöskin benjaminilaisia ja Juudan miehiä tuli Daavidin vuorilinnaan.
\par 17 Silloin Daavid meni heitä vastaan, lausui ja sanoi heille: "Jos te tulette minun luokseni rauha mielessä, auttaaksenne minua, niin minun sydämeni yhtyy teihin; mutta jos te tulette kavaltamaan minua vihollisilleni, vaikkei väkivalta minun käsiäni tahraa, niin nähköön meidän isiemme Jumala sen ja rangaiskoon".
\par 18 Mutta henki täytti Amasain, niiden kolmenkymmenen päällikön, ja hän sanoi: "Sinun me olemme, Daavid, ja sinun puolellasi olemme, sinä Iisain poika. Rauha, rauha sinulle, ja rauha sinun auttajillesi, sillä sinun Jumalasi auttoi sinua!" Niin Daavid otti heidät vastaan ja asetti heidät partiojoukon päälliköiksi.
\par 19 Manassesta siirtyi muutamia Daavidin puolelle, kun hän filistealaisten kanssa meni taistelemaan Saulia vastaan, vaikkeivät he joutuneetkaan auttamaan näitä, sillä filistealaisten ruhtinaat neuvoteltuaan lähettivät hänet pois sanoen: "Hän voisi meidän päämme menoksi siirtyä herransa Saulin puolelle".
\par 20 Kun hän siis lähti Siklagiin, siirtyivät hänen puolelleen Manassesta: Adna, Joosabad, Jediael, Miikael, Joosabad, Elihu ja Silletai, tuhannenpäämiehet Manassesta.
\par 21 Ja nämä auttoivat Daavidia rosvojoukkoa vastaan, sillä he olivat kaikki sotaurhoja, ja heistä tuli sotajoukon päälliköitä.
\par 22 Sillä joka päivä tuli väkeä Daavidin luo auttamaan häntä, kunnes joukko oli suuri kuin Jumalan joukko.
\par 23 Ja tämä on sotaan varustetun aseväen lukumäärä, niitten, jotka tulivat Daavidin luo Hebroniin siirtääkseen Saulin kuninkuuden hänelle Herran käskyn mukaan:
\par 24 Juudan miehiä, jotka kantoivat kilpeä ja keihästä, kuusituhatta kahdeksansataa, sotaan varustettua;
\par 25 simeonilaisia, sotataitoisia urhoja, seitsemäntuhatta sata;
\par 26 leeviläisiä neljätuhatta kuusisataa,
\par 27 sekä Joojada, Aaronin suvun ruhtinas, ynnä kolmetuhatta seitsemänsataa hänen kanssaan;
\par 28 ja Saadok, nuorukainen, sotaurho, perhekuntinensa, kaksikymmentä kaksi päällikköä;
\par 29 benjaminilaisia, Saulin veljiä, kolmetuhatta, sillä vielä siihen aikaan suurin osa heistä palveli uskollisesti Saulin sukua;
\par 30 efraimilaisia kaksikymmentä tuhatta kahdeksansataa, sotaurhoja, kuuluisia miehiä perhekunnissaan;
\par 31 toisesta puolesta Manassen sukukuntaa kahdeksantoista tuhatta, jotka olivat nimeltä mainitut menemään ja tekemään Daavidin kuninkaaksi;
\par 32 isaskarilaisia, jotka ymmärsivät ajan ja käsittivät, mitä Israelin oli tehtävä, kaksisataa päämiestä, ja kaikki heidän veljensä heidän johdollansa;
\par 33 Sebulonista sotakelpoisia miehiä, kaikkinaisilla sota-aseilla sotaan varustettuja, viisikymmentä tuhatta, yksimielisiä sotarintaan järjestymään;
\par 34 Naftalista tuhat päällikköä, ja heidän kanssansa kolmekymmentäseitsemäntuhatta kilvellä ja keihäällä varustettua miestä;
\par 35 daanilaisia kaksikymmentäkahdeksan tuhatta kuusisataa sotaan varustettua miestä;
\par 36 Asserista sotakelpoisia miehiä, taisteluun valmiita, neljäkymmentä tuhatta;
\par 37 tuolta puolelta Jordanin, ruubenilaisia, gaadilaisia ja manasselaisia sukukunnan toisesta puolesta, satakaksikymmentä tuhatta, kaikkinaisilla sota-aseilla varustettuja.
\par 38 Kaikki nämä sotilaat, sotarintaan järjestyneinä, tulivat ehyellä sydämen halulla Hebroniin tehdäkseen Daavidin koko Israelin kuninkaaksi. Myöskin koko muu Israel oli yksimielinen tehdäkseen Daavidin kuninkaaksi.
\par 39 Ja he olivat siellä Daavidin luona kolme päivää ja söivät ja joivat, sillä heidän heimolaisensa olivat evästäneet heidät.
\par 40 Nekin, jotka asuivat lähellä heitä, aina Isaskariin, Sebuloniin ja Naftaliin saakka, toivat aasien, kamelien, muulien ja härkien selässä ruokatavaraa: jauhoja, viikunakakkuja ja rusinakakkuja, viiniä ja öljyä, raavaita ja pikkukarjaa viljalti; sillä Israelissa oli ilo.

\chapter{13}

\par 1 Ja Daavid neuvotteli tuhannen- ja sadanpäämiesten, kaikkien ruhtinasten, kanssa.
\par 2 Ja Daavid sanoi kaikelle Israelin seurakunnalle: "Jos se on teistä hyvä ja jos se tulee Herralta, meidän Jumalaltamme, niin lähettäkäämme joka taholle sana veljillemme, jotka ovat jääneet kaikkiin Israelin maakuntiin, ja papeille ja leeviläisille, jotka asuvat heidän luonaan laidunmaittensa kaupungeissa, että he kokoontuvat meidän luoksemme,
\par 3 ja siirtäkäämme Jumalamme arkki luoksemme, sillä Saulin päivinä emme siitä välittäneet".
\par 4 Silloin koko seurakunta vastasi, että niin oli tehtävä, sillä se oli oikein koko kansan silmissä.
\par 5 Niin Daavid kokosi kaiken Israelin, aina Egyptin Siihorista asti ja aina sieltä, mistä mennään Hamatiin, tuomaan Jumalan arkkia Kirjat-Jearimista.
\par 6 Ja Daavid ja koko Israel meni Baalatiin, Kirjat-Jearimiin, joka on Juudassa, tuomaan sieltä Jumalan arkkia, jonka Herra oli ottanut nimiinsä, hän, jonka istuinta kerubit kannattavat.
\par 7 Ja he panivat Jumalan arkin uusiin vaunuihin ja veivät sen pois Abinadabin talosta, ja Ussa ja Ahjo ohjasivat vaunuja.
\par 8 Ja Daavid ynnä koko Israel karkeloi kaikin voimin Jumalan edessä laulaen sekä soittaen kanteleita, harppuja, vaskirumpuja, kymbaaleja ja torvia.
\par 9 Mutta kun he tulivat Kiidonin puimatantereen luo, ojensi Ussa kätensä tarttuaksensa arkkiin, sillä härät kompastuivat.
\par 10 Silloin Herran viha syttyi Ussaa kohtaan, ja hän löi hänet sentähden, että hän oli ojentanut kätensä arkkiin, ja niin hän kuoli siihen, Jumalan eteen.
\par 11 Mutta Daavid pahastui siitä, että Herra niin oli murtanut Ussan. Siitä sen paikan nimenä on Peres-Ussa vielä tänäkin päivänä.
\par 12 Ja Daavid pelkäsi sinä päivänä Jumalaa, niin että hän sanoi: "Kuinka minä voin tuoda Jumalan arkin tyköni?"
\par 13 Eikä Daavid siirtänyt arkkia luoksensa Daavidin kaupunkiin, vaan hän toimitti sen syrjään gatilaisen Oobed-Edomin taloon.
\par 14 Ja Jumalan arkki jäi kolmeksi kuukaudeksi Oobed-Edomin talon luo omaan majaansa. Ja Herra siunasi Oobed-Edomin taloa ja kaikkea, mitä hänellä oli.

\chapter{14}

\par 1 Hiiram, Tyyron kuningas, lähetti sanansaattajat Daavidin luo, sekä setripuita ja puuseppiä ja kivenhakkaajia rakentamaan hänelle linnaa.
\par 2 Ja Daavid ymmärsi, että Herra oli vahvistanut hänet Israelin kuninkaaksi, koska hänen kuninkuutensa oli korotettu korkealle Herran kansan Israelin tähden.
\par 3 Jerusalemissa Daavid vielä otti vaimoja, ja Daavidille syntyi vielä poikia ja tyttäriä.
\par 4 Ja nämä ovat niiden poikien nimet, jotka syntyivät hänelle Jerusalemissa: Sammua, Soobab, Naatan, Salomo,
\par 5 Jibhar, Elisua, Elpelet,
\par 6 Noogah, Nefeg, Jaafia,
\par 7 Elisama, Beeljada ja Elifelet.
\par 8 Mutta kun filistealaiset kuulivat, että Daavid oli voideltu koko Israelin kuninkaaksi, lähtivät kaikki filistealaiset etsimään Daavidia. Kun Daavid sen kuuli, lähti hän heitä vastaan.
\par 9 Kun filistealaiset olivat tulleet ja tehneet ryöstöretken Refaimin tasangolle,
\par 10 kysyi Daavid Jumalalta: "Menenkö minä filistealaisia vastaan, ja annatko sinä heidät minun käsiini?" Ja Herra sanoi hänelle: "Mene; minä annan heidät sinun käsiisi".
\par 11 Niin he menivät Baal-Perasimiin, ja siellä Daavid voitti heidät. Ja Daavid sanoi: "Jumala on murtanut viholliseni minun kädelläni, niinkuin vedet murtavat". Siitä sen paikan nimeksi tuli Baal-Perasim.
\par 12 He jättivät siihen jumalansa, ja Daavid käski polttaa ne tulessa.
\par 13 Mutta filistealaiset tekivät vielä kerran ryöstöretken tasangolle.
\par 14 Niin Daavid kysyi taas Jumalalta, ja Jumala vastasi hänelle: "Älä mene heidän jälkeensä, vaan kierrä heidät ja hyökkää heidän kimppuunsa balsamipuiden puolelta.
\par 15 Ja kun kuulet astunnan kahinan balsamipuiden latvoista, niin käy taisteluun, sillä Jumala on käynyt sinun edelläsi tuhotakseen filistealaisten leirin."
\par 16 Daavid teki, niinkuin Jumala oli häntä käskenyt, ja niin he voittivat filistealaisten sotajoukon ja ajoivat heitä takaa Gibeonista aina Geseriin saakka.
\par 17 Niin Daavidin maine levisi kaikkiin maihin, ja Herra nosti kaikissa kansoissa pelon häntä kohtaan.

\chapter{15}

\par 1 Ja hän rakensi itsellensä taloja Daavidin kaupunkiin ja valmisti paikan Jumalan arkille ja pystytti sille majan.
\par 2 Silloin Daavid käski: "Älkööt muut kuin leeviläiset kantako Jumalan arkkia; sillä heidät on Herra valinnut kantamaan Jumalan arkkia ja palvelemaan häntä ikuisesti".
\par 3 Ja Daavid kokosi kaiken Israelin Jerusalemiin, tuomaan Herran arkkia siihen paikkaan, jonka hän oli sille valmistanut.
\par 4 Daavid kokosi Aaronin jälkeläiset ja leeviläiset;
\par 5 Kehatin jälkeläisistä: Uurielin, päämiehen, ja hänen veljensä, sata kaksikymmentä;
\par 6 Merarin jälkeläisistä: Asajan, päämiehen, ja hänen veljensä, kaksisataa kaksikymmentä;
\par 7 Geersomin jälkeläisistä: Jooelin, päämiehen, ja hänen veljensä, sata kolmekymmentä;
\par 8 Elisafanin jälkeläisistä: Semajan, päämiehen, ja hänen veljensä, kaksisataa;
\par 9 Hebronin jälkeläisistä: Elielin, päämiehen, ja hänen veljensä, kahdeksankymmentä;
\par 10 Ussielin jälkeläisistä: Amminadabin, päämiehen, ja hänen veljensä, sata kaksitoista.
\par 11 Ja Daavid kutsui papit Saadokin ja Ebjatarin sekä leeviläiset Uurielin, Asajan, Jooelin, Semajan, Elielin ja Amminadabin
\par 12 ja sanoi heille: "Te olette leeviläisten perhekuntien päämiehet. Pyhittäkää itsenne ja veljenne ja tuokaa Herran, Israelin Jumalan, arkki siihen paikkaan, jonka minä olen sille valmistanut.
\par 13 Sillä sentähden, ettette ensimmäisellä kerralla olleet läsnä, mursi Herra, meidän Jumalamme, meidät; sillä me emme etsineet häntä, niinkuin olisi pitänyt."
\par 14 Silloin papit ja leeviläiset pyhittäytyivät tuomaan Herran, Israelin Jumalan, arkkia.
\par 15 Ja leeviläiset kantoivat Jumalan arkkia korennoilla olallansa, niinkuin Mooses oli Herran sanan mukaan käskenyt.
\par 16 Ja Daavid käski leeviläisten päämiesten asettaa veljensä, veisaajat, soittimilla, harpuilla, kanteleilla ja kymbaaleilla kaiuttamaan riemuvirsiä.
\par 17 Niin leeviläiset asettivat soittamaan Heemanin, Jooelin pojan, ja hänen heimolaisistaan Aasafin, Berekjan pojan, ja heidän heimolaisistaan, Merarin jälkeläisistä, Eetanin, Kuusajan pojan,
\par 18 ja heidän kanssaan heidän arvossa lähinnä olevista heimolaisistaan: Sakarjan, Benin ja Jaasielin, Semiramotin, Jehielin, Unnin, Eliabin, Benajan, Maasejan, Mattitjan, Elifelehun, Miknejan, Oobed-Edomin ja Jegielin, ovenvartijat.
\par 19 Ja veisaajista oli Heemanin, Aasafin ja Eetanin soitettava vaskikymbaaleilla,
\par 20 Sakarjan, Ussielin, Semiramotin, Jehielin, Unnin, Eliabin, Maasejan ja Benajan harpuilla korkeassa äänialassa,
\par 21 Mattitjan, Elifelehun, Miknejan, Oobed-Edomin, Jegielin ja Asasjan kanteleilla matalassa äänialassa, veisuuta johtaaksensa.
\par 22 Kenanja, leeviläisten johtaja kantajatehtävässä, valvoi kantamista, sillä hän oli taitava siihen.
\par 23 Berekja ja Elkana olivat arkin ovenvartijat.
\par 24 Ja papit Sebanja, Joosafat, Netanel, Amasai, Sakarja, Benaja ja Elieser puhalsivat torviin Jumalan arkin edellä, ja Oobed-Edom ja Jehia olivat arkin ovenvartijat.
\par 25 Sitten Daavid, Israelin vanhimmat ja tuhannenpäämiehet menivät ja toivat riemuiten Herran liitonarkin Oobed-Edomin talosta.
\par 26 Ja kun Jumala auttoi leeviläisiä, jotka kantoivat Herran liitonarkkia, uhrattiin seitsemän härkää ja seitsemän oinasta.
\par 27 Daavid oli puettu hienosta pellavakankaasta tehtyyn viittaan ja samoin kaikki leeviläiset, jotka kantoivat arkkia, sekä veisaajat ja kantamisen johtaja Kenanja veisaajain kanssa; ja Daavidilla oli yllään pellavakasukka.
\par 28 Niin koko Israel toi Herran liitonarkin riemun raikuessa ja pasunain pauhatessa, torvien, kymbaalien, harppujen ja kannelten soidessa.
\par 29 Kun Herran liitonarkki tuli Daavidin kaupunkiin, katseli Miikal, Saulin tytär, ikkunasta, ja nähdessään kuningas Daavidin hyppivän ja karkeloivan halveksi hän häntä sydämessään.

\chapter{16}

\par 1 Kun he olivat tuoneet Jumalan arkin ja asettaneet sen majaan, jonka Daavid oli sille pystyttänyt, uhrasivat he polttouhreja ja yhteysuhreja Jumalan edessä.
\par 2 Ja kun Daavid oli uhrannut polttouhrin ja yhteysuhrit, siunasi hän kansan Herran nimeen.
\par 3 Ja hän jakoi kaikille israelilaisille, sekä miehille että naisille, kullekin leipäkakun, kappaleen lihaa ja rypälekakun.
\par 4 Ja hän asetti Herran arkin eteen leeviläisiä palvelemaan ja kunnioittamaan, kiittämään ja ylistämään Herraa, Israelin Jumalaa:
\par 5 Aasafin johtajaksi, toiseksi Sakarjan, sitten Jegielin, Semiramotin, Jehielin, Mattitjan, Eliabin, Benajan, Oobed-Edomin ja Jegielin soittamaan harpuilla ja kanteleilla, Aasafin helistäessä kymbaaleja
\par 6 ja pappien, Benajan ja Jahasielin, soittaessa yhtämittaa torvia Jumalan liitonarkin edessä.
\par 7 Silloin, sinä päivänä, ensi kerran Daavid asetti Aasafin ja hänen veljensä kiittämään Herraa näin:
\par 8 "Kiittäkää Herraa, julistakaa hänen nimeänsä, tehkää hänen suuret tekonsa tiettäviksi kansojen keskuudessa.
\par 9 Laulakaa hänelle, veisatkaa hänelle, puhukaa kaikista hänen ihmeistänsä.
\par 10 Hänen pyhä nimensä olkoon teidän kerskauksenne; iloitkoon niiden sydän, jotka etsivät Herraa.
\par 11 Kysykää Herraa ja hänen voimaansa, etsikää alati hänen kasvojansa.
\par 12 Muistakaa hänen ihmetöitänsä, jotka hän on tehnyt, hänen ihmeitänsä ja hänen suunsa tuomioita,
\par 13 te Israelin, hänen palvelijansa, siemen, Jaakobin lapset, te hänen valittunsa.
\par 14 Hän, Herra, on meidän Jumalamme; hänen tuomionsa käyvät yli kaiken maan.
\par 15 Muistakaa hänen liittonsa iankaikkisesti, hamaan tuhansiin polviin, sana, jonka hän on säätänyt,
\par 16 liitto, jonka hän teki Aabrahamin kanssa, ja hänen lisakille vannomansa vala.
\par 17 Hän vahvisti sen käskyksi Jaakobille, Israelille iankaikkiseksi liitoksi.
\par 18 Hän sanoi: 'Sinulle minä annan Kanaanin maan, se olkoon teidän perintöosanne'.
\par 19 Teitä oli vähäinen joukko, vain harvoja, ja te olitte muukalaisia siellä.
\par 20 Ja he vaelsivat kansasta kansaan ja yhdestä valtakunnasta toiseen kansaan.
\par 21 Hän ei sallinut kenenkään heitä vahingoittaa, ja hän rankaisi kuninkaita heidän tähtensä:
\par 22 'Älkää koskeko minun voideltuihini, älkää tehkö pahaa minun profeetoilleni'.
\par 23 Veisatkaa Herralle, kaikki maa, julistakaa päivästä päivään hänen pelastustekojansa.
\par 24 Ilmoittakaa pakanain seassa hänen kunniaansa, hänen ihmeitänsä kaikkien kansojen seassa.
\par 25 Sillä Herra on suuri ja sangen ylistettävä, hän on peljättävä yli kaikkien jumalain.
\par 26 Sillä kaikki kansojen jumalat ovat epäjumalia, mutta Herra on tehnyt taivaat.
\par 27 Kirkkaus ja kunnia on hänen kasvojensa edessä, väkevyys ja riemu hänen asuinsijassaan.
\par 28 Antakaa Herralle, te kansojen sukukunnat, antakaa Herralle kunnia ja väkevyys.
\par 29 Antakaa Herralle hänen nimensä kunnia, tuokaa lahjoja ja tulkaa hänen kasvojensa eteen, kumartakaa Herraa pyhässä kaunistuksessa.
\par 30 Vaviskaa hänen kasvojensa edessä, kaikki maa. Maan piiri pysyy lujana, se ei horju.
\par 31 Iloitkoot taivaat, ja riemuitkoon maa; ja sanottakoon pakanain seassa: 'Herra on kuningas!'
\par 32 Pauhatkoon meri ja kaikki, mitä siinä on; ihastukoot kedot ja kaikki, mitä niissä on,
\par 33 riemuitkoot silloin metsän puut Herran edessä, sillä hän tulee tuomitsemaan maata.
\par 34 Kiittäkää Herraa, sillä hän on hyvä, sillä hänen armonsa pysyy iankaikkisesti.
\par 35 Ja sanokaa: 'Pelasta meidät, sinä pelastuksemme Jumala, kokoa ja vapahda meidät pakanain seasta, että me kiittäisimme sinun pyhää nimeäsi ja kerskaisimme sinun ylistyksestäsi'.
\par 36 Ylistetty olkoon Herra, Israelin Jumala, iankaikkisesta iankaikkiseen." Ja kaikki kansa sanoi: "Amen", ja ylisti Herraa.
\par 37 Ja hän asetti Aasafin ja hänen veljensä sinne Herran liitonarkin eteen tekemään vakituista palvelusta arkin edessä, kunakin päivänä sen päivän palveluksen,
\par 38 mutta Oobed-Edomin ja heidän veljensä, yhteensä kuusikymmentä kahdeksan, nimittäin Oobed-Edomin, Jeditunin pojan, ja Hoosan, ovenvartijoiksi.
\par 39 Ja pappi Saadokin ja hänen veljensä, papit, hän asetti Herran asumuksen eteen, uhrikukkulalle, joka on Gibeonissa,
\par 40 uhraamaan polttouhreja Herralle polttouhrialttarilla, aina aamuin ja illoin, kaikki niinkuin on kirjoitettuna Herran laissa, jonka hän on antanut Israelille.
\par 41 Ja heidän kanssaan olivat Heeman ja Jedutun ynnä muut valitut, nimeltä mainitut, kiittämässä Herraa siitä, että hänen armonsa pysyy iankaikkisesti.
\par 42 Näiden, Heemanin ja Jedutunin, hallussa oli torvet ja kymbaalit soittajia varten ynnä muut soittimet Jumalan virttä varten. Ja Jedutunin pojat vartioivat ovia.
\par 43 Sitten kaikki kansa lähti kukin kotiinsa, ja Daavid kääntyi takaisin tervehtimään perhettänsä.

\chapter{17}

\par 1 Kerran, kun Daavid istui linnassansa, sanoi Daavid profeetta Naatanille: "Katso, minä asun setripuisessa palatsissa, mutta Herran liitonarkki on telttakankaan alla". Naatan sanoi Daavidille:
\par 2 "Tee vain kaikki, mitä mielessäsi on, sillä Jumala on sinun kanssasi".
\par 3 Mutta sinä yönä tapahtui, että Naatanille tuli tämä Jumalan sana:
\par 4 "Mene ja sano minun palvelijalleni Daavidille: Näin sanoo Herra: Et sinä ole rakentava minulle huonetta asuakseni.
\par 5 Minä en ole asunut huoneessa siitä päivästä asti, jona johdatin Israelin tänne, tähän päivään saakka, vaan minä olen muuttanut teltasta telttaan ja asumuksesta asumukseen.
\par 6 Olenko minä koskaan, missä olenkin vaeltanut kaikessa Israelissa, kenellekään Israelin tuomareista, joita olen asettanut kaitsemaan kansaani, sanonut näin: Miksi ette ole rakentaneet minulle setripuista huonetta?
\par 7 Sano siis minun palvelijalleni Daavidille: Näin sanoo Herra Sebaot: Minä olen ottanut sinut laitumelta, lammasten jäljestä, kansani Israelin ruhtinaaksi.
\par 8 Ja minä olen ollut sinun kanssasi kaikkialla, missä sinä vaelsit, ja olen hävittänyt kaikki vihollisesi sinun tieltäsi. Ja minä teen sinulle nimen, suurimpien nimien vertaisen maan päällä.
\par 9 Ja minä valmistan sijan kansalleni Israelille ja istutan sen niin, että se asuu paikallansa eikä enää ole levoton eivätkä vääryyden tekijät sitä enää vaivaa niinkuin ennen,
\par 10 siitä ajasta saakka, jolloin minä asetin tuomareita kansalleni Israelille. Ja minä nöyryytän kaikki sinun vihollisesi. Ja minä ilmoitan sinulle, että Herra on rakentava sinulle huoneen.
\par 11 Kun sinun päiväsi ovat päättyneet ja sinä lähdet isiesi tykö, korotan minä sinun seuraajaksesi jälkeläisesi, joka on yksi sinun pojistasi; ja minä vahvistan hänen kuninkuutensa.
\par 12 Hän on rakentava minulle huoneen, ja minä vahvistan hänen valtaistuimensa ikuisiksi ajoiksi.
\par 13 Minä olen oleva hänen isänsä, ja hän on oleva minun poikani; ja armoani minä en ota häneltä pois, niinkuin otin siltä, joka oli sinun edelläsi.
\par 14 Ja minä pidän hänet pystyssä huoneessani ja valtakunnassani iäti, ja hänen valtaistuimensa on oleva iäti vahva."
\par 15 Aivan näillä sanoilla ja tämän näyn mukaan Naatan puhui Daavidille.
\par 16 Niin kuningas Daavid meni ja asettui Herran eteen ja sanoi: "Mikä olen minä, Herra Jumala, ja mikä on minun sukuni, että olet saattanut minut tähän asti?
\par 17 Mutta tämäkin on ollut vähän sinun silmissäsi, Jumala, ja niin sinä olet puhunut palvelijasi suvulle myöskin kaukaisista asioista ja olet nähnyt minut nousevina ihmispolvina, Herra Jumala.
\par 18 Mitä Daavid enää sinulle puhuisi kunniasta, jota olet osoittanut palvelijallesi? Sinähän tunnet palvelijasi.
\par 19 Herra, palvelijasi tähden ja oman mielesi mukaan sinä olet tehnyt kaiken tämän suuren ja ilmoittanut kaikki nämä suuret asiat.
\par 20 Herra, kaiken sen mukaan, mitä me olemme korvillamme kuulleet, ei ole sinun vertaistasi, eikä ole muuta jumalaa kuin sinä.
\par 21 Ja missä on maan päällä toista kansaa sinun kansasi Israelin vertaista, jota Jumala itse on käynyt lunastamaan kansaksensa, ja niin sinä olet tehnyt itsellesi suuren ja peljättävän nimen karkoittamalla pakanat kansasi tieltä, jonka sinä Egyptistä lunastit.
\par 22 Ja sinä olet asettanut kansasi Israelin omaksi kansaksesi ikuisiksi ajoiksi, ja sinä, Herra, olet tullut heidän Jumalaksensa.
\par 23 Niin olkoon nyt, Herra, se sana, jonka sinä olet puhunut palvelijastasi ja hänen suvustansa, vahva ikuisiksi ajoiksi. Tee, niinkuin olet puhunut.
\par 24 Silloin sinun nimesi on oleva vahva ja tuleva suureksi ikuisiksi ajoiksi, ja se on oleva: Herra Sebaot, Israelin Jumala, Jumala Israelissa. Ja palvelijasi Daavidin suku on pysyvä sinun edessäsi.
\par 25 Sillä sinä, minun Jumalani, olet ilmoittanut palvelijallesi rakentavasi hänelle huoneen. Siitä palvelijasi on saanut rohkeuden rukoilla sinun edessäsi.
\par 26 Ja nyt, Herra, sinä olet Jumala; ja kun olet luvannut palvelijallesi tämän hyvän,
\par 27 niin suvaitse nyt siunata palvelijasi sukua, että se pysyisi iäti sinun edessäsi. Sillä mitä sinä, Herra, siunaat, se on oleva siunattu iäti."

\chapter{18}

\par 1 Sen jälkeen Daavid voitti filistealaiset ja nöyryytti heidät; ja hän otti Gatin ja sen tytärkaupungit filistealaisten käsistä.
\par 2 Hän voitti myös mooabilaiset, ja niin mooabilaiset tulivat Daavidin veronalaisiksi palvelijoiksi.
\par 3 Samoin Daavid voitti Hadareserin, Sooban kuninkaan, Hamatin suunnalla, kun tämä oli menossa lujittamaan valtaansa Eufrat-virran rannoille.
\par 4 Ja Daavid otti häneltä tuhat vaunuhevosta, seitsemäntuhatta ratsumiestä ja kaksikymmentä tuhatta jalkamiestä, ja Daavid katkoi kaikilta vaunuhevosilta vuohisjänteet; ainoastaan sata vaunuhevosta hän niistä säästi.
\par 5 Ja kun Damaskon aramilaiset tulivat auttamaan Hadareseria, Sooban kuningasta, voitti Daavid kaksikymmentäkaksi tuhatta aramilaista.
\par 6 Ja Daavid asetti maaherroja Damaskon Aramiin; ja aramilaiset tulivat Daavidin veronalaisiksi palvelijoiksi. Näin Herra antoi Daavidille voiton, mihin tahansa tämä meni.
\par 7 Ja Daavid otti ne kultavarustukset, jotka Hadareserin palvelijoilla oli yllään, ja vei ne Jerusalemiin.
\par 8 Mutta Hadareserin kaupungeista, Tibhatista ja Kuunista, Daavid otti sangen paljon vaskea. Siitä Salomo teetti vaskimeren, pylväät ja vaskikalut.
\par 9 Kun Toou, Hamatin kuningas, kuuli, että Daavid oli voittanut Hadareserin, Sooban kuninkaan, koko sotajoukon,
\par 10 lähetti hän poikansa Hadoramin kuningas Daavidin luo tervehtimään häntä ja onnittelemaan häntä siitä, että hän oli taistellut Hadareserin kanssa ja voittanut hänet; Hadareser oli näet ollut Tooun vastustaja. Ja hänellä oli mukanaan kaikkinaisia kulta-, hopea- ja vaskikaluja.
\par 11 Nekin kuningas Daavid pyhitti Herralle samoin kuin hopean ja kullan, minkä hän oli ottanut kaikilta kansoilta: Edomilta, Mooabilta, ammonilaisilta, filistealaisilta ja Amalekilta.
\par 12 Ja kun Abisai, Serujan poika, oli Suolalaaksossa voittanut edomilaiset, kahdeksantoista tuhatta miestä,
\par 13 asetti hän maaherroja Edomiin, ja kaikki edomilaiset tulivat Daavidin palvelijoiksi. Näin Herra antoi Daavidille voiton, mihin tahansa tämä meni.
\par 14 Ja Daavid hallitsi koko Israelia ja teki kaikelle kansallensa sitä, mikä oikeus ja vanhurskaus on.
\par 15 Jooab, Serujan poika, oli sotajoukon ylipäällikkönä, ja Joosafat, Ahiludin poika, oli kanslerina.
\par 16 Saadok, Ahitubin poika, ja Abimelek, Ebjatarin poika, olivat pappeina, ja Savsa oli kirjurina.
\par 17 Benaja, Joojadan poika, oli kreettien ja pleettien päällikkönä; mutta Daavidin pojat olivat ensimmäiset kuninkaan rinnalla.

\chapter{19}

\par 1 Sen jälkeen ammonilaisten kuningas Naahas kuoli, ja hänen poikansa tuli kuninkaaksi hänen sijaansa.
\par 2 Niin Daavid sanoi: "Minä osoitan laupeutta Haanunille, Naahaan pojalle, sillä hänen isänsä osoitti laupeutta minulle". Ja Daavid lähetti sanansaattajat lohduttamaan häntä hänen isänsä kuoleman johdosta. Kun Daavidin palvelijat tulivat ammonilaisten maahan Haanunin luo lohduttamaan häntä,
\par 3 sanoivat ammonilaisten päämiehet Haanunille: "Luuletko sinä, että Daavid tahtoo kunnioittaa sinun isääsi, kun hän lähettää lohduttajia sinun luoksesi? Varmaankin hänen palvelijansa ovat tulleet luoksesi tutkimaan ja hävittämään ja vakoilemaan maata."
\par 4 Niin Haanun otatti kiinni Daavidin palvelijat, ajatti heidän partansa ja leikkautti toisen puolen heidän vaatteistaan, peräpuolia myöten, ja päästi heidät sitten menemään.
\par 5 Kun tultiin kertomaan Daavidille miehistä, lähetti hän sanan heitä vastaan, sillä miehiä oli pahasti häväisty. Ja kuningas käski sanoa: "Jääkää Jerikoon, kunnes partanne on kasvanut, ja tulkaa sitten takaisin".
\par 6 Kun ammonilaiset huomasivat joutuneensa Daavidin vihoihin, lähettivät Haanun ja ammonilaiset tuhat talenttia hopeata palkatakseen itsellensä Mesopotamiasta, Maakan Aramista ja Soobasta sotavaunuja ja ratsumiehiä.
\par 7 Niin he palkkasivat itselleen kolmekymmentäkaksi tuhatta sotavaunua sekä Maakan kuninkaan ja hänen väkensä; nämä tulivat ja leiriytyivät Meedeban edustalle. Ja ammonilaiset kokoontuivat kaupungeistaan ja lähtivät sotaan.
\par 8 Kun Daavid sen kuuli, lähetti hän Jooabin ja koko sotajoukon, urhot.
\par 9 Niin ammonilaiset lähtivät ja asettuivat sotarintaan kaupungin portin edustalle, mutta kuninkaat, jotka olivat tulleet sinne, olivat eri joukkona kedolla.
\par 10 Kun Jooab näki, että häntä uhkasi hyökkäys edestä ja takaa, valitsi hän miehiä kaikista Israelin valiomiehistä ja asettui sotarintaan aramilaisia vastaan.
\par 11 Mutta muun väen hän antoi veljensä Abisain johtoon, ja nämä asettuivat sotarintaan ammonilaisia vastaan.
\par 12 Ja hän sanoi: "Jos aramilaiset tulevat minulle ylivoimaisiksi, niin tule sinä minun avukseni: jos taas ammonilaiset tulevat sinulle ylivoimaisiksi, niin minä autan sinua.
\par 13 Ole luja, ja pysykäämme lujina kansamme puolesta ja meidän Jumalamme kaupunkien puolesta. Tehköön sitten Herra, minkä hyväksi näkee."
\par 14 Sitten Jooab ja väki, joka oli hänen kanssaan, ryhtyi taisteluun aramilaisia vastaan, ja nämä pakenivat häntä.
\par 15 Ja kun ammonilaiset näkivät aramilaisten pakenevan, pakenivat hekin hänen veljeänsä Abisaita ja menivät kaupunkiin. Mutta Jooab lähti Jerusalemiin.
\par 16 Kun aramilaiset näkivät, että Israel oli voittanut heidät, lähettivät he sanansaattajat nostattamaan niitä aramilaisia, jotka asuivat tuolla puolella Eufrat-virran ja joita johti Soofak, Hadareserin sotapäällikkö.
\par 17 Kun se ilmoitettiin Daavidille, kokosi hän kaiken Israelin ja meni Jordanin yli, ja tultuaan heitä lähelle hän asettui sotarintaan heitä vastaan. Ja kun Daavid oli asettunut sotarintaan aramilaisia vastaan, ryhtyivät nämä taisteluun hänen kanssaan.
\par 18 Mutta aramilaiset pakenivat Israelia, ja Daavid tappoi aramilaisilta seitsemäntuhatta vaunuhevosta ja neljäkymmentä tuhatta jalkamiestä ja surmasi heidän sotapäällikkönsä Soofakin.
\par 19 Kun Hadareserin palvelijat näkivät, että Israel oli voittanut heidät, tekivät he Daavidin kanssa rauhan ja palvelivat häntä. Sitten aramilaiset eivät enää tahtoneet auttaa ammonilaisia.

\chapter{20}

\par 1 Vuoden vaihteessa, kuningasten sotaanlähtöaikana, vei Jooab sotajoukon sotaretkelle; ja hän hävitti ammonilaisten maata ja piiritti Rabbaa. Mutta Daavid itse jäi Jerusalemiin. Sitten Jooab valtasi Rabban ja hävitti sen.
\par 2 Ja Daavid otti heidän kuninkaansa kruunun hänen päästänsä ja havaitsi sen painavan talentin kultaa, ja siinä oli kallis kivi; se pantiin Daavidin päähän. Ja hän vei kaupungista hyvin paljon saalista.
\par 3 Ja kansan, joka siellä oli, hän vei pois ja pani sahaamaan kiviä, pani rautahakkujen ja kivisahojen ääreen. Näin Daavid teki kaikille ammonilaisten kaupungeille. Sitten Daavid ja kaikki väki palasi Jerusalemiin.
\par 4 Sen jälkeen syttyi taistelu filistealaisia vastaan Geserissä. Silloin huusalainen Sibbekai surmasi Sippain, joka oli Raafan jälkeläisiä, ja niin heidät nöyryytettiin.
\par 5 Taas oli taistelu filistealaisia vastaan, ja Elhanan, Jaaorin poika, surmasi Lahmin, gatilaisen Goljatin veljen, jonka peitsen varsi oli niinkuin kangastukki.
\par 6 Taas oli taistelu Gatissa. Siellä oli suurikasvuinen mies, jolla oli kuusi sormea ja kuusi varvasta, yhteensä kaksikymmentä neljä; hänkin polveutui Raafasta.
\par 7 Ja kun hän häpäisi Israelia, surmasi hänet Joonatan, Daavidin veljen Simean poika.
\par 8 Nämä polveutuivat gatilaisesta Raafasta; he kaatuivat Daavidin ja hänen palvelijainsa käden kautta.

\chapter{21}

\par 1 Mutta saatana nousi Israelia vastaan ja yllytti Daavidin laskemaan Israelin.
\par 2 Niin Daavid sanoi Jooabille ja kansan päämiehille: "Menkää ja laskekaa Israel Beersebasta Daaniin asti ja ilmoittakaa minulle, että saan tietää heidän lukumääränsä".
\par 3 Jooab vastasi: "Herra lisätköön kansansa, olkoon se kuinka suuri tahansa, satakertaiseksi. Ovathan he, herrani, kuningas, kaikki herrani palvelijoita. Miksi herrani pyytää tätä? Miksi Israel näin joutuisi vikapääksi?"
\par 4 Kuitenkin kuninkaan sana velvoitti Jooabia, ja niin Jooab lähti ja kierteli koko Israelin. Sitten hän tuli Jerusalemiin.
\par 5 Ja Jooab ilmoitti Daavidille kansan lasketun lukumäärän: koko Israelissa oli yksitoistasataa tuhatta sotakuntoista miekkamiestä, ja Juudassa oli neljäsataa seitsemänkymmentä tuhatta miekkamiestä.
\par 6 Mutta Leevistä ja Benjaminista hän ei pitänyt katselmusta yhdessä muiden kanssa, sillä kuninkaan käsky oli Jooabille kauhistus.
\par 7 Mutta tämä asia oli Jumalan silmissä paha, ja hän löi Israelia.
\par 8 Niin Daavid sanoi Jumalalle: "Minä olen tehnyt suuren synnin tehdessäni tämän; mutta anna nyt anteeksi palvelijasi rikos, sillä minä olen menetellyt ylen tyhmästi".
\par 9 Ja Herra puhui Gaadille, Daavidin näkijälle, näin: "Mene ja puhu Daavidille:
\par 10 Näin sanoo Herra: Kolme minä asetan sinun eteesi: valitse itsellesi yksi niistä, niin minä teen sen sinulle".
\par 11 Niin Gaad meni Daavidin tykö ja sanoi hänelle: "Näin sanoo Herra: Valitse itsellesi
\par 12 joko kolme nälkävuotta tahi kolme kuukautta hävitystä ahdistajaisi vainotessa, vihollistesi miekka kintereilläsi, tahi kolmeksi päiväksi Herran miekka ja ruttotauti maahan, Herran enkeli tuottamaan tuhoa koko Israelin alueelle. Katso nyt, mitä minä vastaan hänelle joka minut lähetti."
\par 13 Daavid vastasi Gaadille: "Minä olen suuressa hädässä. Tahdon langeta Herran käsiin, sillä hänen laupeutensa on sangen suuri; ihmisten käsiin minä en tahdo langeta."
\par 14 Niin Herra antoi ruton tulla Israeliin, ja Israelista kaatui seitsemänkymmentä tuhatta miestä.
\par 15 Ja Jumala lähetti enkelin Jerusalemia vastaan tuhoamaan sitä; ja kun se sitä tuhosi, katsoi Herra siihen ja katui sitä pahaa ja sanoi tuhoojaenkelille: "Jo riittää; laske kätesi alas". Ja Herran enkeli seisoi silloin jebusilaisen Ornanin puimatantereen luona.
\par 16 Kun Daavid nosti silmänsä ja näki Herran enkelin seisovan maan ja taivaan välillä, kädessänsä paljastettu miekka ojennettuna Jerusalemin yli, lankesivat Daavid ja vanhimmat, säkkeihin verhottuina, kasvoillensa.
\par 17 Ja Daavid sanoi Jumalalle: "Minähän käskin laskea kansan, ja minä siis olen se, joka olen tehnyt syntiä ja menetellyt pahoin. Mutta nämä minun lampaani, mitä he ovat tehneet? Herra, minun Jumalani, sattukoon sinun kätesi minuun ja minun isäni perheeseen, mutta ei sinun kansaasi, sille vitsaukseksi."
\par 18 Silloin Herran enkeli käski Gaadin sanoa Daavidille, että Daavid menisi pystyttämään Herralle alttarin jebusilaisen Ornanin puimatantereelle.
\par 19 Ja Daavid meni sen sanan mukaan, jonka Gaad oli Herran nimeen puhunut.
\par 20 Ja kun Ornan kääntyi, näki hän enkelin, ja hänen neljä poikaansa piiloutui; sillä Ornan oli puimassa nisuja.
\par 21 Mutta Daavid meni Ornanin luo, ja kun Ornan katsahti ylös ja näki Daavidin, lähti hän puimatantereelta ja kumartui Daavidin eteen kasvoilleen maahan.
\par 22 Ja Daavid sanoi Ornanille: "Anna minulle puimatantereen paikka, rakentaakseni siihen Herralle alttarin; täydestä hinnasta anna se minulle, että vitsaus taukoaisi kansasta".
\par 23 Ornan vastasi Daavidille: "Ota se itsellesi, ja herrani, kuningas, tehköön, mitä hyväksi näkee. Katso, minä annan härät polttouhriksi ja puimaäkeet haloiksi sekä nisut ruokauhriksi; kaikki minä annan."
\par 24 Mutta kuningas Daavid sanoi Ornanille: "Ei niin, vaan minä ostan ne sinulta täydestä hinnasta; sillä minä en ota Herralle sitä, mikä on sinun, enkä uhraa ilmaiseksi saatua polttouhria".
\par 25 Niin Daavid antoi Ornanille siitä paikasta kuudensadan sekelin painon kultaa.
\par 26 Ja Daavid rakensi sinne alttarin Herralle ja uhrasi polttouhreja ja yhteysuhreja, ja kun hän huusi Herraa, vastasi Herra tulella, jonka hän lähetti taivaasta polttouhrialttarille.
\par 27 Ja Herra käski enkelin pistää miekkansa tuppeen.
\par 28 Siihen aikaan kun Daavid näki, että Herra oli kuullut häntä jebusilaisen Ornanin puimatantereella, uhrasi hän siellä.
\par 29 Mutta Herran asumus, jonka Mooses oli teettänyt erämaassa, ja polttouhrialttari olivat siihen aikaan Gibeonin uhrikukkulalla.
\par 30 Eikä Daavid rohjennut mennä Jumalan eteen etsimään häntä, sillä hän oli peljästynyt Herran enkelin miekkaa.

\chapter{22}

\par 1 Ja Daavid sanoi: "Tässä olkoon Herran Jumalan temppeli ja tässä alttari Israelin polttouhria varten".
\par 2 Niin Daavid käski koota muukalaiset, jotka olivat Israelin maassa. Ja hän asetti kivenhakkaajia hakkaamaan kiviä Jumalan temppelin rakentamiseksi.
\par 3 Ja Daavid hankki paljon rautaa portinovien nauloiksi ja sinkilöiksi ja niin paljon vaskea, ettei se ollut punnittavissa,
\par 4 ja suunnattomat määrät setripuita, sillä siidonilaiset ja tyyrolaiset toivat paljon setripuita Daavidille.
\par 5 Sillä Daavid ajatteli: "Minun poikani Salomo on nuori ja hento, ja Herralle rakennettava temppeli on tehtävä ylen suuri, niin että sitä mainitaan ja kiitetään kaikissa maissa. Minä siis hankin hänelle varastot." Niin Daavid hankki ennen kuolemaansa suuret varastot.
\par 6 Ja hän kutsui poikansa Salomon ja käski hänen rakentaa temppelin Herralle, Israelin Jumalalle.
\par 7 Ja Daavid sanoi Salomolle: "Poikani, minä aioin itse rakentaa temppelin Herran, Jumalani, nimelle.
\par 8 Mutta minulle tuli tämä Herran sana: 'Sinä olet vuodattanut paljon verta ja käynyt suuria sotia. Sinä et ole rakentava temppeliä minun nimelleni, koska olet vuodattanut niin paljon verta maahan minun edessäni.
\par 9 Katso, sinulle on syntyvä poika; hänestä on tuleva rauhan mies, ja minä annan hänen päästä rauhaan kaikilta hänen ympärillään asuvilta vihollisilta, sillä Salomo on oleva hänen nimensä ja minä annan rauhan ja levon Israelille hänen päivinänsä.
\par 10 Hän on rakentava temppelin minun nimelleni. Hän on oleva minun poikani, ja minä olen oleva hänen isänsä. Ja minä vahvistan hänen kuninkuutensa valtaistuimen ikuisiksi ajoiksi.'
\par 11 Olkoon siis Herra sinun kanssasi, poikani, että menestyisit ja saisit rakennetuksi temppelin Herralle, Jumalallesi, niinkuin hän on sinusta puhunut.
\par 12 Antakoon vain Herra sinulle älyä ja ymmärrystä ja asettakoon sinut Israelia hallitsemaan ja noudattamaan Herran, Jumalasi, lakia.
\par 13 Silloin sinä menestyt, jos tarkoin noudatat niitä käskyjä ja oikeuksia, jotka Herra antoi Moosekselle Israelia varten. Ole luja ja rohkea, älä pelkää äläkä arkaile.
\par 14 Katso, vaivanalaisenakin minä olen hankkinut Herran temppeliä varten satatuhatta talenttia kultaa ja tuhat kertaa tuhat talenttia hopeata sekä vaskea ja rautaa niin paljon, ettei se ole punnittavissa, sillä sitä on ylen paljon. Minä olen myös hankkinut hirsiä ja kiviä, ja sinä saat hankkia vielä lisää.
\par 15 Paljon on sinulla myöskin työmiehiä, kivenhakkaajia, muurareita, puuseppiä ja kaikkinaisen työn taitajia.
\par 16 Kullalla, hopealla, vaskella ja raudalla ei ole määrää. Nouse, ryhdy työhön, ja Herra olkoon sinun kanssasi!"
\par 17 Ja Daavid käski kaikkia Israelin päämiehiä auttamaan poikaansa Salomoa, sanoen:
\par 18 "Onhan Herra, teidän Jumalanne, ollut teidän kanssanne ja suonut teidän päästä joka taholla rauhaan; sillä hän on antanut maan asukkaat minun käsiini ja maa on tehty alamaiseksi Herralle ja hänen kansallensa.
\par 19 Kääntäkää siis sydämenne ja sielunne etsimään Herraa, Jumalaanne; nouskaa ja rakentakaa Herran, Jumalan, pyhäkkö, että Herran liitonarkki ja Jumalan pyhät kalut voitaisiin viedä temppeliin, joka on rakennettava Herran nimelle."

\chapter{23}

\par 1 Kun Daavid oli tullut vanhaksi ja saanut elämästä kyllänsä, teki hän poikansa Salomon Israelin kuninkaaksi.
\par 2 Ja hän kutsui kokoon kaikki Israelin päämiehet, papit ja leeviläiset.
\par 3 Ja leeviläiset luettiin, kolmikymmenvuotiset ja sitä vanhemmat, ja heidän lukumääränsä, pääluvun mukaan, oli kolmekymmentäkahdeksan tuhatta miestä.
\par 4 "Näistä olkoon kaksikymmentäneljä tuhatta johtamassa töitä Herran temppelissä ja kuusi tuhatta päällysmiehinä ja tuomareina;
\par 5 ja neljä tuhatta olkoon ovenvartijoina, ja neljä tuhatta ylistäköön Herraa soittimilla, jotka minä olen teettänyt ylistämistä varten."
\par 6 Ja Daavid jakoi heidät osastoihin Leevin poikien Geersonin, Kehatin ja Merarin mukaan.
\par 7 Geersonilaisia olivat Ladan ja Siimei.
\par 8 Ladanin pojat olivat Jehiel, päämies, Seetam ja Jooel, kaikkiaan kolme.
\par 9 Siimein pojat olivat Selomit, Hasiel ja Haaran, kaikkiaan kolme. Nämä olivat Ladanin perhekuntien päämiehet.
\par 10 Ja Siimein pojat olivat Jahat, Siina, Jeus ja Beria. Nämä olivat Siimein pojat, kaikkiaan neljä.
\par 11 Jahat oli päämies, Siisa toinen; mutta Jeuksella ja Berialla ei ollut monta poikaa, niin että heistä tuli yksi perhekunta, yksi palvelusvuoro.
\par 12 Kehatin pojat olivat Amram, Jishar, Hebron ja Ussiel, kaikkiaan neljä.
\par 13 Amramin pojat olivat Aaron ja Mooses. Mutta Aaron poikinensa erotettiin olemaan ikuisesti pyhitetty, korkeasti-pyhä, ikuisesti suitsuttamaan Herran edessä, palvelemaan häntä ja siunaamaan hänen nimessään.
\par 14 Jumalan miehen Mooseksen pojat luettiin Leevin sukukuntaan kuuluviksi.
\par 15 Mooseksen pojat olivat Geersom ja Elieser.
\par 16 Geersomin poika oli Sebuel, päämies.
\par 17 Elieserin poika oli Rehabja, päämies; Elieserillä ei ollut muita poikia. Mutta Rehabjan poikia oli ylen paljon.
\par 18 Jisharin poika oli Selomit, päämies.
\par 19 Hebronin pojat olivat Jeria, päämies, Amarja toinen, Jahasiel kolmas ja Jekamam neljäs.
\par 20 Ussielin pojat olivat Miika, päämies, ja Jissia toinen.
\par 21 Merarin pojat olivat Mahli ja Muusi. Mahlin pojat olivat Eleasar ja Kiis.
\par 22 Kun Eleasar kuoli, ei häneltä jäänyt poikia, vaan ainoastaan tyttäriä, jotka heidän serkkunsa, Kiisin pojat, ottivat vaimoiksensa.
\par 23 Muusin pojat olivat Mahli, Eeder ja Jeremot, kaikkiaan kolme.
\par 24 Nämä olivat Leevin pojat, heidän perhekuntiensa mukaan, perhekunta-päämiehet, niin monta kuin heitä oli ollut katselmuksessa, nimien lukumäärän mukaan, pääluvun mukaan, ne, jotka toimittivat palvelustehtäviä Herran temppelissä, kaksikymmenvuotiaat ja sitä vanhemmat.
\par 25 Sillä Daavid sanoi: "Herra, Israelin Jumala, on suonut kansansa päästä lepoon ja on asuva Jerusalemissa ikuisesti.
\par 26 Niinpä ei leeviläistenkään tarvitse kantaa asumusta eikä mitään kaluja, joita tarvitaan siinä tehtävissä töissä."
\par 27 - Sillä Daavidin viimeisten määräysten mukaan laskettiin Leevin poikien lukuun kaksikymmenvuotiaat ja sitä vanhemmat. -
\par 28 Heidän tehtäväkseen tuli: olla Aaronin poikien apuna Herran temppelin töissä; pitää huoli esikartanoista ja kammioista, kaiken pyhän puhtaana pitämisestä ja Jumalan temppelin töistä,
\par 29 näkyleivistä, ruokauhriin tarvittavista lestyistä jauhoista, happamattomista ohukaisista, leivinlevystä ja jauhosekoituksesta sekä astia- ja pituusmitoista;
\par 30 seisoa joka aamu kiittämässä ja ylistämässä Herraa, ja samoin joka ilta;
\par 31 ja uhrata kaikki polttouhrit Herralle sapatteina, uudenkuun päivinä ja juhlina, niin paljon kuin niitä oli säädetty aina uhrattavaksi Herran edessä.
\par 32 Niin heidän oli hoidettava ilmestysmajan ja pyhäkön tehtävät sekä ne tehtävät, jotka heidän veljillään Aaronin pojilla oli palvellessaan Herran temppelissä.

\chapter{24}

\par 1 Nämä olivat Aaronin poikien osastot: Aaronin pojat olivat Naadab, Abihu, Eleasar ja Iitamar.
\par 2 Mutta Naadab ja Abihu kuolivat ennen isäänsä, eikä heillä ollut poikia. Niin palvelivat ainoastaan Eleasar ja Iitamar pappeina.
\par 3 Ja Daavid yhdessä Saadokin kanssa, joka oli Eleasarin poikia, ja Ahimelekin kanssa, joka oli Iitamarin poikia, jakoi heidät osastoihin heidän palvelusvuorojensa mukaan.
\par 4 Kun Eleasarin pojilla havaittiin olevan enemmän päämiehiä kuin Iitamarin pojilla, jaettiin heidät niin, että Eleasarin pojat saivat kuusitoista päämiestä perhekunnilleen ja Iitamarin pojat kahdeksan päämiestä perhekunnilleen.
\par 5 Heidät jaettiin arvalla, toiset niinkuin toisetkin, sillä pyhäkköruhtinaat ja Jumalan ruhtinaat otettiin sekä Eleasarin pojista että Iitamarin pojista.
\par 6 Ja Semaja, Netanelin poika, kirjuri, joka oli Leevin sukua, kirjoitti heidät muistiin kuninkaan, päämiesten, pappi Saadokin ja Ahimelekin, Ebjatarin pojan, sekä pappien ja leeviläisten perhekuntien päämiesten läsnäollessa. Yksi perhekunta otettiin vuorotellen Eleasarin ja Iitamarin suvusta.
\par 7 Ensimmäinen arpa tuli Joojaribille, toinen Jedajalle,
\par 8 kolmas Haarimille, neljäs Seoromille,
\par 9 viides Malkialle, kuudes Miijaminille,
\par 10 seitsemäs Koosille, kahdeksas Abialle,
\par 11 yhdeksäs Jeesualle, kymmenes Sekanjalle,
\par 12 yhdestoista Eljasibille, kahdestoista Jaakimille,
\par 13 kolmastoista Huppalle, neljästoista Jesebabille,
\par 14 viidestoista Bilgalle, kuudestoista Immerille,
\par 15 seitsemästoista Heesirille, kahdeksastoista Pissekselle,
\par 16 yhdeksästoista Petahjalle, kahdeskymmenes Hesekielille,
\par 17 kahdeskymmenes yhdes Jaakinille, kahdeskymmenes kahdes Gaamulille,
\par 18 kahdeskymmenes kolmas Delajalle, kahdeskymmenes neljäs Maasjalle.
\par 19 Nämä ovat heidän palvelusvuoronsa, kun he menevät Herran temppeliin, niinkuin heidän isänsä Aaron oli heille säätänyt, sen mukaan, kuin Herra, Israelin Jumala, oli häntä käskenyt.
\par 20 Mitä tulee muihin Leevin jälkeläisiin, niin oli Amramin jälkeläisiä Suubael, Suubaelin jälkeläisiä Jehdeja,
\par 21 Rehabjan jälkeläisiä päämies Jissia,
\par 22 jisharilaisia Selomot, Selomotin jälkeläisiä Jahat.
\par 23 Ja Jerian jälkeläisiä olivat: Amarja toinen, Jahasiel kolmas, Jekamam neljäs.
\par 24 Ussielin jälkeläisiä oli Miika, Miikan jälkeläisiä Saamir.
\par 25 Miikan veli oli Jissia; Jissian jälkeläisiä oli Sakarja.
\par 26 Merarin jälkeläisiä olivat Mahli ja Muusi sekä hänen poikansa Jaasian jälkeläiset.
\par 27 Merarilla oli jälkeläisiä pojastaan Jaasiasta ynnä Sooham, Sakkur ja Ibri.
\par 28 Mahlista polveutui Eleasar, jolla ei ollut poikia.
\par 29 Kiisistä polveutui Jerahmeel, joka oli Kiisin jälkeläisiä.
\par 30 Ja Muusin jälkeläisiä olivat Mahli, Eeder ja Jerimot. Nämä olivat leeviläisten jälkeläiset heidän perhekuntiensa mukaan.
\par 31 Myöskin nämä, niinhyvin perhekuntapäämiehet kuin heidän nuoremmat veljensä, heittivät arpaa samoin kuin heidän veljensä, Aaronin pojat, kuningas Daavidin, Saadokin ja Ahimelekin sekä pappien ja leeviläisten perhekuntien päämiesten läsnäollessa.

\chapter{25}

\par 1 Ja Daavid ja sotapäälliköt erottivat palvelukseen Aasafin, Heemanin ja Jedutunin pojat, jotka hurmoksissa soittivat kanteleilla, harpuilla ja kymbaaleilla. Ja tämä on luettelo miehistä, jotka tätä palvelustaan toimittivat:
\par 2 Aasafin poikia oli Sakkur, Joosef, Netanja ja Asarela, Aasafin pojat, Aasafin johdolla, joka hurmoksissa soitti kuninkaan johdolla.
\par 3 Jedutunista: Jedutunin pojat Gedalja, Seri, Jesaja, Hasabja ja Mattitja, kaikkiaan kuusi, isänsä Jedutunin johdolla, joka hurmoksissa soitti kanteleilla kiitosta ja ylistystä Herralle.
\par 4 Heemanista: Heemanin pojat Bukkia, Mattanja, Ussiel, Sebuel ja Jerimot, Hananja, Hanani, Eliata, Giddalti ja Roomamti-Eser, Josbekasa, Malloti, Hootir ja Mahasiot;
\par 5 nämä ovat kaikki Heemanin, kuninkaan näkijän, poikia, sen Jumalan sanan mukaan, että hän on korottava korkealle hänen sarvensa: Jumala oli antanut Heemanille neljätoista poikaa ja kolme tytärtä.
\par 6 Nämä kaikki isänsä johdon alaisina veisasivat Herran temppelissä kymbaalien, harppujen ja kanteleitten säestäessä, Jumalan temppelipalveluksessa, kuninkaan, Aasafin, Jedutunin ja Heemanin johdolla.
\par 7 Ja heidän ja heidän veljiensä lukumäärä, jotka oli opetettu veisaamaan Herran kunniaksi, kaikkien taitajien, oli kaksisataa kahdeksankymmentä kahdeksan.
\par 8 Ja he heittivät arpaa palvelusjärjestyksestä, nuoremmat niinkuin vanhemmatkin, taitajat yhdessä oppilasten kanssa.
\par 9 Ensimmäinen arpa, Aasafin arpa, tuli Joosefille. Toinen Gedaljalle, hänelle itselleen sekä hänen veljilleen ja pojilleen, joita oli kaikkiaan kaksitoista.
\par 10 Kolmas Sakkurille, hänen pojilleen ja veljilleen, joita oli kaikkiaan kaksitoista.
\par 11 Neljäs Jisrille, hänen pojilleen ja veljilleen, joita oli kaikkiaan kaksitoista.
\par 12 Viides Netanjalle, hänen pojilleen ja veljilleen, joita oli kaikkiaan kaksitoista.
\par 13 Kuudes Bukkialle, hänen pojilleen ja veljilleen, joita oli kaikkiaan kaksitoista.
\par 14 Seitsemäs Jesarelalle, hänen pojilleen ja veljilleen, joita oli kaikkiaan kaksitoista.
\par 15 Kahdeksas Jesajalle, hänen pojilleen ja veljilleen, joita oli kaikkiaan kaksitoista.
\par 16 Yhdeksäs Mattanjalle, hänen pojilleen ja veljilleen, joita oli kaikkiaan kaksitoista.
\par 17 Kymmenes Siimeille, hänen pojilleen ja veljilleen, joita oli kaikkiaan kaksitoista.
\par 18 Yhdestoista Asarelille, hänen pojilleen ja veljilleen, joita oli kaikkiaan kaksitoista.
\par 19 Kahdestoista Hasabjalle, hänen pojilleen ja veljilleen, joita oli kaikkiaan kaksitoista.
\par 20 Kolmastoista Suubaelille, hänen pojilleen ja veljilleen, joita oli kaikkiaan kaksitoista.
\par 21 Neljästoista Mattitjalle, hänen pojilleen ja veljilleen, joita oli kaikkiaan kaksitoista.
\par 22 Viidestoista Jeremotille, hänen pojilleen ja veljilleen, joita oli kaikkiaan kaksitoista.
\par 23 Kuudestoista Hananjalle, hänen pojilleen ja veljilleen, joita oli kaikkiaan kaksitoista.
\par 24 Seitsemästoista Josbekasalle, hänen pojilleen ja veljilleen, joita oli kaikkiaan kaksitoista.
\par 25 Kahdeksastoista Hananille, hänen pojilleen ja veljilleen, joita oli kaikkiaan kaksitoista.
\par 26 Yhdeksästoista Mallotille, hänen pojilleen ja veljilleen, joita oli kaikkiaan kaksitoista.
\par 27 Kahdeskymmenes Elijjatalle, hänen pojilleen ja veljilleen, joita oli kaikkiaan kaksitoista.
\par 28 Kahdeskymmenes yhdes Hootirille, hänen pojilleen ja veljilleen, joita oli kaikkiaan kaksitoista.
\par 29 Kahdeskymmenes kahdes Giddaltille, hänen pojilleen ja veljilleen, joita oli kaikkiaan kaksitoista.
\par 30 Kahdeskymmenes kolmas Mahasiotille, hänen pojilleen ja veljilleen, joita oli kaikkiaan kaksitoista.
\par 31 Kahdeskymmenes neljäs Roomamti-Eserille, hänen pojilleen ja veljilleen, joita oli kaikkiaan kaksitoista.

\chapter{26}

\par 1 Mitä tulee ovenvartijain osastoihin, niin oli koorahilaisia Meselemja, Kooren poika, Aasafin jälkeläisiä.
\par 2 Meselemjalla oli pojat: esikoinen Sakarja, toinen Jediael, kolmas Sebadja, neljäs Jatniel,
\par 3 viides Eelam, kuudes Joohanan, seitsemäs Eljoenai.
\par 4 Oobed-Edomilla oli pojat: esikoinen Semaja, toinen Joosabad, kolmas Jooah, neljäs Saakar, viides Netanel,
\par 5 kuudes Ammiel, seitsemäs Isaskar, kahdeksas Peulletai; sillä Jumala oli siunannut häntä.
\par 6 Ja hänen pojallensa Semajalle syntyi poikia, jotka hallitsivat isänsä sukua, sillä he olivat kykeneviä miehiä.
\par 7 Semajan pojat olivat Otni, Refael, Oobed ja Elsabad ja hänen veljensä, kykeneviä miehiä, Elihu ja Semakja.
\par 8 Kaikki nämä olivat Oobed-Edomin jälkeläisiä, he sekä heidän poikansa ja veljensä, kykeneviä miehiä, tarmokkaita palveluksessaan, kaikkiaan kuusikymmentä kaksi Oobed-Edomin jälkeläistä.
\par 9 Meselemjalla oli poikia ja veljiä, kykeneviä miehiä, kaikkiaan kahdeksantoista.
\par 10 Ja Hoosalla, joka oli Merarin jälkeläisiä, oli poikia: päämies Simri, jonka hänen isänsä, kun ei ollut esikoista, asetti päämieheksi;
\par 11 toinen Hilkia, kolmas Tebalja, neljäs Sakarja; Hoosan poikia ja veljiä oli kaikkiaan kolmetoista.
\par 12 Näiden ovenvartijain osastojen, päämiesten, samoinkuin heidän veljiensä, tehtävänä oli vartiopalvelus; sillä heidän oli palveltava Herran temppelissä.
\par 13 Ja he heittivät arpaa joka ovesta, nuoremmat niinkuin vanhemmatkin, perhekuntiensa mukaan.
\par 14 Itäpuolen arpa lankesi Selemjalle; myöskin hänen pojallensa Sakarjalle, joka oli ymmärtäväinen neuvonantaja, heitettiin arpa, ja hänelle tuli pohjoispuolen arpa.
\par 15 Oobed-Edomille määräsi arpa etelän ja hänen pojilleen varastohuoneen.
\par 16 Suppimille ja Hoosalle lännen, jossa on Salleket-portti ylöspäin kohoavan tien kohdalla: vartiopaikan toisen vartiopaikan viereen.
\par 17 Idän puolella oli kuusi leeviläistä, pohjoisen puolella neljä joka päivä, etelän puolella neljä joka päivä ja varastohuoneen luona kaksi ja kaksi.
\par 18 Parparin luona, lännen puolella, oli neljä tiellä ja kaksi Parparin luona.
\par 19 Nämä ovat ovenvartijain osastot, koorahilaisten jälkeläisten ja Merarin jälkeläisten.
\par 20 Leeviläisistä hoiti Ahia Jumalan temppelin aarrekammioita ja pyhien lahjojen aarteita.
\par 21 Ladanin pojat, geersonilaisten jälkeläiset, Ladanista polveutuvat geersonilaisen Ladanin perhekuntien päämiehet, jehieliläiset,
\par 22 jehieliläisten jälkeläiset, olivat Seetam ja hänen veljensä Jooel; he hoitivat Herran temppelin aarrekammioita.
\par 23 Mitä tulee amramilaisiin, jisharilaisiin, hebronilaisiin ja ossielilaisiin,
\par 24 niin Sebuel, Geersomin poika, joka oli Mooseksen poika, oli aarrekammioiden esimies.
\par 25 Ja hänen Elieseristä polveutuvat veljensä olivat: tämän poika Rehabja, tämän poika Jesaja, tämän poika Jooram, tämän poika Sikri ja tämän poika Selomot.
\par 26 Tämä Selomot ja hänen veljensä hoitivat kaikkia niiden pyhien lahjojen aarteita, jotka kuningas Daavid, perhekunta-päämiehet, tuhannen- ja sadanpäämiehet ja sotapäälliköt olivat pyhittäneet.
\par 27 Sodista ja saaliista he olivat ne pyhittäneet Herran temppelin voimassapitämiseksi.
\par 28 Samoin kaikki, mitä Samuel, näkijä, Saul, Kiisin poika, Abner, Neerin poika, ja Jooab, Serujan poika, olivat pyhittäneet, kaikki, mikä pyhitettiin, jätettiin Selomitin ja hänen veljiensä hoitoon.
\par 29 Jisharilaisista määrättiin Kenanja ja hänen poikansa maallisiin toimiin Israelissa, päällysmiehiksi ja tuomareiksi.
\par 30 Hebronilaisista määrättiin Hasabja ja hänen veljensä, tuhat seitsemänsataa kykenevää miestä, Israelin hallinnon hoitoon Jordanin länsipuolella, kaikkinaisiin Herran toimiin ja kuninkaan palvelukseen.
\par 31 Hebronilaisia oli Jeria, hebronilaisten päämies, heidän polveutumisensa ja isiensä mukaan - Daavidin neljäntenäkymmenentenä hallitusvuotena heidät tutkittiin ja heidän joukostaan tavattiin kykeneviä miehiä Gileadin Jaeserissa -
\par 32 sekä hänen veljensä, kaksituhatta seitsemänsataa kykenevää miestä, perhekunta-päämiehiä. Heidät kuningas Daavid asetti johtamaan ruubenilaisia, gaadilaisia ja toista puolta manasselaisten sukukuntaa kaikissa Jumalan ja kuninkaan asioissa.

\chapter{27}

\par 1 Israelilaisia, lukumääränsä mukaan, ynnä perhekunta-päämiehiä, tuhannen- ja sadanpäämiehiä ja heidän päällysmiehiänsä, jotka palvelivat kuningasta kaikessa, mikä koski osastoja, jotka tulivat ja lähtivät kuukausi kuukaudelta, vuoden kaikkina kuukausina, oli kussakin osastossa kaksikymmentäneljä tuhatta miestä:
\par 2 Ensimmäisen osaston, ensimmäisen kuukauden osaston, johtajana oli Jaasobeam, Sabdielin poika, ja hänen osastossaan oli kaksikymmentäneljä tuhatta;
\par 3 hän oli Pereksen jälkeläisiä ja oli kaikkien sotapäälliköitten ylipäällikkö, ensimmäisenä kuukautena.
\par 4 Toisen kuukauden osaston johtajana oli ahohilainen Doodai; hänen osastonsa johdossa oli myös ruhtinas Miklot, ja hänen osastossaan oli kaksikymmentäneljä tuhatta.
\par 5 Kolmas sotapäällikkö, kolmantena kuukautena, oli Benaja, pappi Joojadan poika, ylipäällikkö; ja hänen osastossaan oli kaksikymmentaneljätuhatta.
\par 6 Tämä Benaja oli sankari niiden kolmenkymmenen joukossa ja niiden kolmenkymmenen johtaja; ja hänen osastossaan oli hänen poikansa Ammisabad.
\par 7 Neljäs, neljäntenä kuukautena, oli Asahel, Jooabin veli, ja hänen jälkeensä hänen poikansa Sebadja; ja hänen osastossaan oli kaksikymmentäneljä tuhatta.
\par 8 Viides, viidentenä kuukautena, oli päällikkö Samhut, jisrahilainen; ja hänen osastossaan oli kaksikymmentäneljä tuhatta.
\par 9 Kuudes, kuudentena kuukautena, oli tekoalainen Iira, Ikkeksen poika; ja hänen osastossaan oli kaksikymmentäneljä tuhatta.
\par 10 Seitsemäs, seitsemäntenä kuukautena, oli pelonilainen Heeles, efraimilaisia; ja hänen osastossaan oli kaksikymmentäneljä tuhatta.
\par 11 Kahdeksas, kahdeksantena kuukautena, oli huusalainen Sibbekai, serahilaisia; ja hänen osastossaan oli kaksikymmentäneljä tuhatta.
\par 12 Yhdeksäs, yhdeksäntenä kuukautena, oli anatotilainen Abieser, benjaminilaisia; ja hänen osastossaan oli kaksikymmentäneljä tuhatta.
\par 13 Kymmenes, kymmenentenä kuukautena, oli netofalainen Mahrai, serahilaisia; ja hänen osastossaan oli kaksikymmentäneljä tuhatta.
\par 14 Yhdestoista, yhdentenätoista kuukautena, oli piratonilainen Benaja, efraimilaisia; ja hänen osastossaan oli kaksikymmentäneljä tuhatta.
\par 15 Kahdestoista, kahdentenatoista kuukautena, oli netofalainen Heldai, Otnielista polveutuva; ja hänen osastossaan oli kaksikymmentäneljä tuhatta.
\par 16 Israelin sukukuntien johtajat olivat: ruubenilaisten ruhtinas oli Elieser, Sikrin poika; simeonilaisten Sefatja, Maakan poika;
\par 17 Leevin oli Hasabja, Kemuelin poika; Aaronin Saadok;
\par 18 Juudan oli Elihu, Daavidin veljiä; Isaskarin Omri, Miikaelin poika;
\par 19 Sebulonin oli Jismaja, Obadjan poika; Naftalin Jerimot, Asrielin poika;
\par 20 efraimilaisten oli Hoosea, Asasjan poika; toisen puolen Manassen sukukuntaa Jooel, Pedajan poika;
\par 21 toisen puolen Manassea, Gileadissa, oli Jiddo, Sakarjan poika; Benjaminin Jaasiel, Abnerin poika;
\par 22 Daanin Asarel, Jerohamin poika. Nämä olivat Israelin sukukuntien ruhtinaat.
\par 23 Mutta Daavid ei ottanut luetteloon kaksikymmenvuotiaita ja sitä nuorempia, sillä Herra oli luvannut tehdä Israelin monilukuiseksi niinkuin taivaan tähdet.
\par 24 Jooab, Serujan poika, oli alottanut laskemisen, mutta ei sitä lopettanut, sillä siitä kohtasi viha Israelia; eikä se luku tullut kuningas Daavidin aikakirjaan.
\par 25 Kuninkaan varastojen hoitaja oli Asmavet, Adielin poika. Kedolla, kaupungeissa, kylissä ja torneissa olevien varastojen hoitaja oli Joonatan, Ussian poika.
\par 26 Maatöitä tekevien peltotyömiesten kaitsija oli Esri, Kelubin poika.
\par 27 Viinitarhain hoitaja oli raamatilainen Siimei. Viinitarhoista koottujen viinivarastojen hoitaja oli sifmiläinen Sabdi.
\par 28 Öljypuiden ja Alankomaassa kasvavien metsäviikunapuiden hoitaja oli gaderilainen Baal-Haanan. Öljyvarastojen hoitaja oli Jooas.
\par 29 Saaronissa laitumella käyvien raavasten kaitsija oli saaronilainen Sitrai, ja tasangoilla laitumella käyvien raavasten kaitsija Saafat, Adlain poika.
\par 30 Kamelien kaitsija oli ismaelilainen Oobil; aasintammojen meeronotilainen Jehdeja.
\par 31 Pikkukarjan kaitsija oli hagrilainen Jaasis. Nämä kaikki olivat kuningas Daavidin omaisuuden ylihoitajia.
\par 32 Ja Joonatan, Daavidin setä, joka oli ymmärtäväinen mies ja kirjanoppinut, oli neuvonantaja. Jehiel, Hakmonin poika, oli kuninkaan lasten luona.
\par 33 Ahitofel oli kuninkaan neuvonantaja, ja arkilainen Huusai oli kuninkaan ystävä.
\par 34 Ahitofelin jälkeen tulivat Joojada, Benajan poika, ja Ebjatar. Kuninkaan sotapäällikkö oli Jooab.

\chapter{28}

\par 1 Ja Daavid kokosi Jerusalemiin kaikki Israelin päämiehet, sukukuntien päämiehet, osastojen päälliköt, jotka palvelivat kuningasta, tuhannen- ja sadanpäämiehet, kuninkaan ja hänen poikiensa kaiken omaisuuden ja karjan ylihoitajat, hovimiehet, sankarit ja kaikki sotaurhot.
\par 2 Ja kuningas Daavid nousi seisomaan ja sanoi: "Kuulkaa minua, veljeni ja kansani. Minä olin aikonut rakentaa huoneen Herran liitonarkin leposijaksi ja meidän Jumalamme jalkojen astinlaudaksi ja olin valmistautunut rakentamaan.
\par 3 Mutta Jumala sanoi minulle: 'Älä sinä rakenna temppeliä minun nimelleni, sillä sinä olet sotilas ja olet vuodattanut verta'.
\par 4 Kuitenkin valitsi Herra, Israelin Jumala, minut kaikesta isäni suvusta olemaan Israelin kuninkaana ikuisesti. Sillä Juudan hän valitsi ruhtinaaksi ja Juudan heimosta minun isäni suvun, ja isäni poikien joukossa olin minä hänelle otollinen, niin että hän teki minut koko Israelin kuninkaaksi.
\par 5 Ja kaikkien minun poikieni joukosta, sillä Herra on antanut minulle monta poikaa, hän valitsi poikani Salomon istumaan Herran valtakunnan valtaistuimella ja hallitsemaan Israelia.
\par 6 Ja hän sanoi minulle: 'Sinun poikasi Salomo on rakentava minun temppelini ja esikartanoni; sillä minä olen valinnut hänet pojakseni ja minä olen oleva hänen isänsä.
\par 7 Ja minä vahvistan hänen kuninkuutensa iankaikkisesti, jos hän on luja ja pitää minun käskyni ja oikeuteni, niinkuin hän nyt tekee.'
\par 8 Ja nyt minä sanon koko Israelin, Herran seurakunnan, läsnäollessa ja meidän Jumalamme kuullen: Noudattakaa ja tutkikaa kaikkia Herran, Jumalanne, käskyjä, että saisitte pitää omananne tämän hyvän maan ja jättäisitte sen perinnöksi lapsillenne teidän jälkeenne ikuisiksi ajoiksi.
\par 9 Ja sinä, minun poikani Salomo, opi tuntemaan isäsi Jumala ja palvele häntä ehyellä sydämellä ja alttiilla mielellä, sillä Herra tutkii kaikki sydämet ja ymmärtää kaikki ajatukset ja aivoitukset. Jos häntä etsit, niin sinä löydät hänet, mutta jos luovut hänestä, niin hän hylkää sinut iankaikkisesti.
\par 10 Katso siis eteesi, sillä Herra on valinnut sinut rakentamaan temppelipyhäkön. Ole luja ja ryhdy työhön."
\par 11 Senjälkeen Daavid antoi pojallensa Salomolle temppelin eteisen, sen rakennusten, varastohuoneiden, yläsalien, sisähuoneiden ja armoistuimen huoneen mallin
\par 12 sekä määräykset kaikesta, mitä hänellä oli mielessään: Herran temppelin esikartanoista ja kaikista kammioista yltympäri, Jumalan temppelin ja pyhien lahjojen aarrekammioista,
\par 13 kuin myös pappien ja leeviläisten osastoista, kaikesta Herran temppelissä toimitettavasta palveluksesta ja kaikista Herran temppelin palveluksessa tarvittavista kaluista: kullasta,
\par 14 jokaisen palveluksessa tarvittavan kultakalun painosta, kaikista hopeakaluista, jokaisen palveluksessa tarvittavan hopeakalun painosta;
\par 15 kullasta tehtävien lampunjalkojen ja niiden kunkin lampun kullan painosta, jokaisen lampunjalan ja sen lamppujen painosta; samoin hopeasta tehtävien lampunjalkojen painosta, jokaisen lampunjalan ja sen lamppujen painosta, jokaisen lampunjalan käytön mukaan;
\par 16 edelleen näkyleipäpöytien kullan painosta, kunkin pöydän erikseen, samoin hopeasta tehtävien pöytien hopeasta;
\par 17 haarukkain, maljojen ja kannujen puhtaasta kullasta, samoin kultapikareista, kunkin pikarin painosta, ja hopeapikareista, kunkin pikarin painosta;
\par 18 myöskin suitsutusalttarin puhdistetun kullan painosta ja vaunujen mallista, kultakerubeista, joitten tuli levittää siipensä ja peittää Herran liitonarkki.
\par 19 "Tämä kaikki on kirjoituksessa, joka on lähtenyt Herran kädestä; hän on opettanut minulle kaikki tämän mallin mukaan tehtävät työt."
\par 20 Ja Daavid sanoi pojallensa Salomolle: "Ole luja ja rohkea ja ryhdy työhön, älä pelkää äläkä arkaile, sillä Herra Jumala, minun Jumalani, on oleva sinun kanssasi. Hän ei jätä sinua eikä hylkää sinua, kunnes olet saanut valmiiksi kaikki Herran temppelissä tehtävät työt.
\par 21 Ja katso, pappien ja leeviläisten osastot ovat valmiit kaikkinaiseen palvelukseen Jumalan temppelissä, ja sinulla on kaikkiin töihin käytettävinäsi niitä, jotka alttiisti ja taidolla tekevät kaikkinaista palvelusta, niin myös päämiehet ja koko kansa kaikkiin tehtäviisi."

\chapter{29}

\par 1 Sitten kuningas Daavid sanoi koko seurakunnalle: "Minun poikani Salomo, ainoa Jumalan valittu, on nuori ja hento, ja työ on suuri, sillä tämä linna ei ole aiottu ihmiselle, vaan Herralle Jumalalle.
\par 2 Sentähden minä olen kaikin voimin hankkinut Jumalani temppeliä varten kultaa kultaesineihin, hopeata hopeaesineihin, vaskea vaskiesineihin, rautaa rautaesineihin ja puuta puuesineihin, onykskiviä ja muita jalokiviä, mustankiiltävää ja kirjavaa kiveä, kaikkinaisia kalliita kiviä ja marmorikiveä suuret määrät.
\par 3 Koska minun Jumalani temppeli on minulle rakas, minä vielä luovutan omistamani kullan ja hopean Jumalani temppeliä varten, kaiken lisäksi, minkä olen pyhäkköä varten hankkinut:
\par 4 kolmetuhatta talenttia kultaa, Oofirin kultaa, ja seitsemäntuhatta talenttia puhdistettua hopeata huoneitten seinien silaamiseksi,
\par 5 kultaa kultaesineihin ja hopeata hopeaesineihin ja kaikkinaisiin taitomiesten kätten töihin. Tahtooko kukaan muu tänä päivänä tuoda vapaaehtoisesti, täysin käsin, lahjoja Herralle?"
\par 6 Silloin tulivat vapaaehtoisesti perhekunta-päämiehet, Israelin sukukuntien ruhtinaat, tuhannen- ja sadanpäämiehet ja kuninkaan töiden ylivalvojat.
\par 7 Ja he antoivat Jumalan temppelissä tehtäviin töihin viisituhatta talenttia ja kymmenentuhatta dareikkia kultaa, kymmenentuhatta talenttia hopeata, kahdeksantoista tuhatta talenttia vaskea ja satatuhatta talenttia rautaa.
\par 8 Ja jolla oli hallussaan jalokiviä, antoi ne Herran temppelin aarteisiin, geersonilaisen Jehielin huostaan.
\par 9 Ja kansa iloitsi heidän alttiudestaan, sillä he antoivat ehyellä sydämellä vapaaehtoiset lahjansa Herralle; ja kuningas Daavid oli myös suuresti iloissansa.
\par 10 Ja Daavid kiitti Herraa koko seurakunnan läsnäollessa; Daavid sanoi: "Ole kiitetty, Herra, meidän isämme Israelin Jumala, iankaikkisesta iankaikkiseen.
\par 11 Sinun, Herra, on suuruus ja väkevyys ja loisto ja kunnia ja kirkkaus, sillä sinun on kaikki taivaassa ja maan päällä. Sinun, Herra, on valtakunta, ja sinä olet korotettu kaiken pääksi.
\par 12 Rikkaus ja kunnia tulevat sinulta, sinä hallitset kaikki, ja sinun kädessäsi on voima ja väkevyys. Sinun vallassasi on tehdä mikä tahansa suureksi ja väkeväksi.
\par 13 Me siis kiitämme sinua, meidän Jumalaamme; me ylistämme sinun ihanaa nimeäsi.
\par 14 Sillä mitä olen minä, ja mitä on minun kansani, kyetäksemme antamaan tällaisia vapaaehtoisia lahjoja? Vaan kaikki tulee sinulta, ja omasta kädestäsi olemme sen sinulle antaneet.
\par 15 Sillä me olemme muukalaisia ja vieraita sinun edessäsi, niinkuin kaikki meidän isämmekin; niinkuin varjo ovat meidän päivämme maan päällä eikä ole, mihin toivonsa panna.
\par 16 Herra, meidän Jumalamme, kaikki tämä runsaus, jonka me olemme hankkineet rakentaaksemme temppelin sinulle, sinun pyhälle nimellesi, on sinun kädestäsi, ja sinun on kaikki tyynni.
\par 17 Ja minä tiedän, Jumalani, että sinä tutkit sydämen ja mielistyt vilpittömyyteen. Vilpittömällä sydämellä minä olen antanut kaikki nämä vapaaehtoiset lahjat, ja nyt minä olen ilolla nähnyt, kuinka sinun täällä oleva kansasi on antanut sinulle vapaaehtoiset lahjansa.
\par 18 Herra, meidän isäimme Aabrahamin, Iisakin ja Israelin Jumala, säilytä tällaisina ainiaan kansasi sydämen ajatukset ja aivoitukset ja ohjaa heidän sydämensä puoleesi.
\par 19 Ja anna minun pojalleni Salomolle ehyt sydän noudattamaan sinun käskyjäsi, todistuksiasi ja säädöksiäsi, toimittamaan kaikki tämä ja rakentamaan tämä linna, jota varten minä olen valmistuksia tehnyt."
\par 20 Sitten Daavid sanoi kaikelle seurakunnalle: "Kiittäkää Herraa, Jumalaanne". Ja kaikki seurakunta kiitti Herraa, isiensä Jumalaa, ja he kumartuivat maahan ja osoittivat kunniaa Herralle ja kuninkaalle.
\par 21 Ja seuraavana päivänä he teurastivat teuraita Herralle ja uhrasivat polttouhreja Herralle: tuhat härkää, tuhat oinasta ja tuhat karitsaa ja niihin kuuluvat juomauhrit sekä teurasuhreja suuren joukon koko Israelin puolesta.
\par 22 Niin he söivät ja joivat Herran edessä sinä päivänä suuresti iloiten ja tekivät toistamiseen Salomon, Daavidin pojan, kuninkaaksi ja voitelivat hänet Herralle ruhtinaaksi, ja Saadokin papiksi.
\par 23 Ja Salomo istui Herran valtaistuimelle kuninkaaksi isänsä Daavidin sijaan, ja hän menestyi; ja koko Israel totteli häntä.
\par 24 Samoin kaikki päämiehet, urhot ja myös kaikki kuningas Daavidin pojat kättä lyöden lupautuivat kuningas Salomon alamaisiksi.
\par 25 Herra teki Salomon ylen suureksi koko Israelin silmissä ja antoi hänelle kuninkuuden loiston, jonka kaltaista ei ollut yhdelläkään kuninkaalla Israelissa ollut ennen häntä.
\par 26 Daavid, Iisain poika, oli hallinnut koko Israelia.
\par 27 Ja aika, jonka hän hallitsi Israelia, oli neljäkymmentä vuotta; Hebronissa hän hallitsi seitsemän vuotta, ja Jerusalemissa hän hallitsi kolmekymmentä kolme vuotta.
\par 28 Ja hän kuoli päästyänsä korkeaan ikään, elämästä, rikkaudesta ja kunniasta kyllänsä saaneena, ja hänen poikansa Salomo tuli kuninkaaksi hänen sijaansa.
\par 29 Daavidin vaiheet, sekä aikaisemmat että myöhemmät, ovat kirjoitettuina näkijä Samuelin historiassa, profeetta Naatanin historiassa ja tietäjä Gaadin historiassa;
\par 30 samoin koko hänen hallituksensa ja hänen tekemänsä urotyöt sekä hänen, Israelin ja kaikkien maitten valtakuntain kohtalot.


\end{document}