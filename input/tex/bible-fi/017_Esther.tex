\begin{document}

\title{Esterin kirja}


\chapter{1}

\par 1 Ahasveroksen aikana - sen Ahasveroksen, joka hallitsi Intiasta Etiopiaan saakka, sataa kahtakymmentä seitsemää maakuntaa - siihen aikaan,
\par 2 kun kuningas Ahasveros istui kuninkaallisella valtaistuimellansa, joka oli Suusanin linnassa, tapahtui tämä.
\par 3 Kolmantena hallitusvuotenaan hän laittoi pidot kaikille ruhtinaillensa ja palvelijoillensa. Persian ja Meedian voima, ylimykset ja maaherrat olivat hänen luonansa,
\par 4 ja hän näytti heille kuninkaallisen kunniansa rikkautta ja suuruutensa loistavaa komeutta monta päivää, sata kahdeksankymmentä päivää.
\par 5 Ja kun ne päivät olivat kuluneet, laittoi kuningas kaikelle Suusanin linnan väelle, niin pienille kuin suurillekin, kuninkaan palatsin puutarhan esipihaan seitsenpäiväiset pidot.
\par 6 Pellava-, puuvilla- ja punasiniverhoja oli kiinnitetty valkoisilla pellavanauhoilla ja purppuranpunaisilla nauhoilla hopeatankoihin ja valkomarmoripylväisiin. Kultaisia ja hopeaisia lepovuoteita oli pihalla, joka oli laskettu vihreällä ja valkoisella marmorilla, helmiäisellä ja kirjavalla marmorilla.
\par 7 Juotavaa tarjottiin kulta-astioissa, ja astiat olivat erimuotoiset, ja kuninkaan viiniä oli viljalti, kuninkaalliseen tapaan.
\par 8 Ja juomisessa oli lakina, ettei saanut olla mitään pakkoa, sillä niin oli kuningas käskenyt kaikkia hovimestareitansa, että oli tehtävä kunkin oman halun mukaan.
\par 9 Myöskin kuningatar Vasti laittoi naisille pidot kuningas Ahasveroksen kuninkaalliseen palatsiin.
\par 10 Seitsemäntenä päivänä, kun kuninkaan sydän oli viinistä iloinen, käski hän Mehumanin, Bistan, Harbonan, Bigtan, Abagtan, Seetarin ja Karkaan, niiden seitsemän hoviherran, jotka toimittivat palvelusta kuningas Ahasveroksen luona,
\par 11 tuoda kuningatar Vastin, kuninkaallinen kruunu päässä, kuninkaan eteen, että hän saisi näyttää kansoille ja ruhtinaille hänen kauneutensa, sillä hän oli näöltään ihana.
\par 12 Mutta kuningatar Vasti kieltäytyi tulemasta, vaikka kuningas hoviherrojen kautta oli käskenyt. Silloin kuningas suuttui kovin, ja hänessä syttyi viha.
\par 13 Ja kuningas puhui viisaille, ajantietäjille - sillä näin oli tapana esittää kuninkaan sana kaikille lain ja oikeuden tuntijoille,
\par 14 ja hänen lähimpänsä olivat Karsena, Seetar, Admata, Tarsis, Meres, Marsena ja Memukan, seitsemän Persian ja Meedian ruhtinasta, jotka näkivät kuninkaan kasvot ja istuivat valtakunnan ensimmäisinä -:
\par 15 "Mitä on lain mukaan tehtävä kuningatar Vastille, koska hän ei ole noudattanut käskyä, jonka kuningas Ahasveros on antanut hoviherrojen kautta?"
\par 16 Niin Memukan sanoi kuninkaan ja ruhtinasten edessä: "Kuningatar Vasti ei ole rikkonut ainoastaan kuningasta vastaan, vaan myös kaikkia ruhtinaita vastaan ja kaikkia kansoja vastaan kuningas Ahasveroksen kaikissa maakunnissa.
\par 17 Kun kuningattaren teko tulee kaikkien vaimojen tietoon, saattaa se heidät halveksimaan aviomiehiänsä, he kun voivat sanoa: 'Kuningas Ahasveros käski tuoda kuningatar Vastin eteensä, mutta tämä ei tullut'.
\par 18 Jo tänä päivänä voivat Persian ja Meedian ruhtinattaret, jotka kuulevat kuningattaren teon, puhua siitä kaikille kuninkaan ruhtinaille, ja siitä tulee halveksimista ja suuttumusta riittämään asti.
\par 19 Jos kuningas hyväksi näkee, julkaistakoon kuninkaallinen käsky ja kirjoitettakoon muuttumatonna Persian ja Meedian lakeihin, ettei Vasti enää saa tulla kuningas Ahasveroksen eteen ja että kuningas antaa hänen kuninkaallisen arvonsa toiselle, häntä paremmalle.
\par 20 Kuninkaan antama säädös tulee tunnetuksi koko hänen valtakunnassansa, vaikka se onkin suuri, ja kaikki vaimot, niin suuret kuin pienetkin, antavat kunnian aviomiehillensä."
\par 21 Tämä puhe miellytti kuningasta ja ruhtinaita, ja kuningas teki Memukanin sanan mukaan.
\par 22 Hän lähetti kaikkiin kuninkaan maakuntiin - kuhunkin maakuntaan sen omalla kirjoituksella ja kullekin kansalle sen omalla kielellä - kirjeet, että joka mies olkoon herra talossaan ja saakoon puhua oman kansansa kieltä.

\chapter{2}

\par 1 Näiden tapausten jälkeen, kun kuningas Ahasveroksen viha oli asettunut, muisti hän Vastia ja mitä tämä oli tehnyt ja mitä hänestä oli päätetty.
\par 2 Niin kuninkaan palvelijat, jotka toimittivat hänelle palvelusta, sanoivat: "Etsittäköön kuninkaalle nuoria neitsyitä, näöltään ihania.
\par 3 Ja asettakoon kuningas kaikkiin valtakuntansa maakuntiin käskyläiset keräämään kaikki nuoret neitsyet, näöltään ihanat, Suusanin linnaan, vaimolaan, Heegain, kuninkaan hoviherran ja vaimojen vartijan, huostaan, ja annettakoon heille heidän kauneudenhoitonsa.
\par 4 Ja neitsyt, johon kuningas mieltyy, tulkoon kuningattareksi Vastin sijaan." Ja tämä puhe miellytti kuningasta, ja hän teki niin.
\par 5 Suusanin linnassa oli juutalainen mies, nimeltä Mordokai, Jaairin poika, joka oli Siimein poika, joka Kiisin poika, benjaminilainen.
\par 6 Hänet oli viety Jerusalemista pakkosiirtolaisuuteen niiden pakkosiirtolaisten joukossa, jotka vietiin samalla kuin Jekonja, Juudan kuningas, jonka Nebukadnessar, Baabelin kuningas, vei pakkosiirtolaisuuteen.
\par 7 Hän oli setänsä tyttären Hadassan, se on Esterin, kasvatusisä, sillä tämä oli isätön ja äiditön. Tyttö oli vartaloltaan kaunis ja näöltään ihana, ja hänen isänsä ja äitinsä kuoltua oli Mordokai ottanut hänet tyttäreksensä.
\par 8 Ja kun kuninkaan käsky ja laki tuli tunnetuksi ja kun kerättiin paljon tyttöjä Suusanin linnaan Heegain huostaan, otettiin Esterkin kuninkaan palatsiin Heegain, vaimojen vartijan, huostaan.
\par 9 Tyttö miellytti häntä ja sai armon hänen edessään, ja hän antoi hänelle joutuin hänen kauneudenhoitonsa ja määrätyn ravinto-osansa sekä seitsemän valiopalvelijatarta kuninkaan palatsista ja siirsi hänet palvelijattarineen vaimolan parhaaseen paikkaan.
\par 10 Ester ei ilmaissut kansaansa eikä syntyperäänsä, sillä Mordokai oli häntä kieltänyt sitä ilmaisemasta.
\par 11 Mutta Mordokai käyskenteli joka päivä vaimolan esipihan edustalla saadakseen tietää, voiko Ester hyvin ja mitä hänelle tapahtui.
\par 12 Kun jonkun tytön vuoro tuli mennä kuningas Ahasveroksen tykö, sittenkuin hänelle oli kaksitoista kuukautta tehty, niinkuin vaimoista oli määrätty - sillä niin pitkä aika kului heidän kauneudenhoitoonsa: kuusi kuukautta mirhaöljyllä, toiset kuusi kuukautta hajuaineilla sekä muilla naisten kauneudenhoitokeinoilla - meni tyttö kuninkaan tykö.
\par 13 Kaikki, mitä hän pyysi, annettiin hänelle mukaan vaimolasta kuninkaan palatsiin.
\par 14 Illalla hän meni, ja aamulla hän palasi toiseen vaimolaan kuninkaan hoviherran Saasgaan, sivuvaimojen vartijan, huostaan; ei hän enää mennyt kuninkaan tykö, paitsi jos kuningas oli häneen mieltynyt ja hänet nimeltään kutsuttiin.
\par 15 Kun Esterille, Mordokain sedän Abihailin tyttärelle, jonka Mordokai oli ottanut tyttärekseen, tuli vuoro mennä kuninkaan tykö, ei hän halunnut mukaansa muuta, kuin mitä Heegai, kuninkaan hoviherra, vaimojen vartija, neuvoi. Ja Ester sai armon kaikkien niiden edessä, jotka näkivät hänet.
\par 16 Ester vietiin kuningas Ahasveroksen tykö hänen kuninkaalliseen palatsiinsa kymmenennessä kuussa, se on teebet-kuussa, hänen seitsemäntenä hallitusvuotenansa.
\par 17 Ja Ester tuli kuninkaalle kaikkia muita naisia rakkaammaksi ja sai hänen edessään armon ja suosion ennen kaikkia muita neitsyitä, niin että tämä pani kuninkaallisen kruunun hänen päähänsä ja teki hänet kuningattareksi Vastin sijaan.
\par 18 Ja kuningas laittoi suuret pidot kaikille ruhtinailleen ja palvelijoilleen, pidot Esterin kunniaksi, ja myönsi maakunnille veronhuojennusta ja jakelutti lahjoja, kuninkaalliseen tapaan.
\par 19 Kun neitsyitä koottiin toinen kerta ja Mordokai istuskeli kuninkaan portissa
\par 20 - eikä Ester ollut ilmaissut syntyperäänsä eikä kansaansa, sillä Mordokai oli häntä siitä kieltänyt; Ester näet teki, mitä Mordokai käski, samoin kuin ollessaan hänen kasvattinansa -
\par 21 niin siihen aikaan, Mordokain istuskellessa kuninkaan portissa, Bigtan ja Teres, kaksi kuninkaan hoviherraa, ovenvartijoita, suuttuivat ja etsivät tilaisuutta käydäkseen käsiksi kuningas Ahasverokseen.
\par 22 Mutta se tuli Mordokain tietoon, ja hän ilmaisi sen kuningatar Esterille, ja Ester sanoi sen Mordokain nimessä kuninkaalle.
\par 23 Ja asia tutkittiin, ja kun se havaittiin todeksi, ripustettiin ne molemmat hirsipuuhun. Ja se kirjoitettiin aikakirjaan kuningasta varten.

\chapter{3}

\par 1 Näiden tapausten jälkeen kuningas Ahasveros korotti agagilaisen Haamanin, Hammedatan pojan, ylensi hänet ja antoi hänelle ylimmän sijan kaikkien ruhtinasten joukossa, jotka olivat hänen luonansa.
\par 2 Ja kaikki kuninkaan palvelijat, jotka olivat kuninkaan portissa, polvistuivat ja heittäytyivät maahan Haamanin edessä, sillä niin oli kuningas käskenyt häntä kohdella. Mutta Mordokai ei polvistunut eikä heittäytynyt maahan.
\par 3 Niin kuninkaan palvelijat, jotka olivat kuninkaan portissa, sanoivat Mordokaille: "Miksi sinä rikot kuninkaan käskyn?"
\par 4 Ja kun he joka päivä sanoivat hänelle näin, mutta hän ei heitä totellut, ilmoittivat he tämän Haamanille, nähdäksensä, oliko Mordokain sanoma syy pätevä; sillä hän oli ilmoittanut heille olevansa juutalainen.
\par 5 Kun Haaman näki, ettei Mordokai polvistunut eikä heittäytynyt maahan hänen edessään, tuli Haaman kiukkua täyteen.
\par 6 Kun hänelle oli ilmoitettu, mitä kansaa Mordokai oli, vähäksyi hän käydä käsiksi yksin Mordokaihin: Haaman etsi tilaisuutta hävittääkseen kaikki juutalaiset, Mordokain kansan, Ahasveroksen koko valtakunnasta.
\par 7 Ensimmäisessä kuussa, se on niisan-kuussa, kuningas Ahasveroksen kahdentenatoista vuotena, heitettiin Haamanin edessä puur'ia, se on arpaa, jokaisesta päivästä ja jokaisesta kuukaudesta, kahdenteentoista kuukauteen, se on adar-kuuhun, asti.
\par 8 Ja Haaman sanoi kuningas Ahasverokselle: "On yksi kansa hajallaan ja erillään muiden kansojen seassa sinun valtakuntasi kaikissa maakunnissa. Heidän lakinsa ovat toisenlaiset kuin kaikkien muiden kansojen, he eivät noudata kuninkaan lakeja, eikä kuninkaan sovi jättää heitä rauhaan.
\par 9 Jos kuningas hyväksi näkee, kirjoitettakoon määräys, että heidät on tuhottava; ja minä punnitsen kymmenentuhatta talenttia hopeata virkamiehille, vietäväksi kuninkaan aarrekammioihin."
\par 10 Niin kuningas otti kädestään sinettisormuksensa ja antoi sen agagilaiselle Haamanille, Hammedatan pojalle, juutalaisten vastustajalle.
\par 11 Ja kuningas sanoi Haamanille: "Hopea olkoon annettu sinulle, ja samoin se kansa, tehdäksesi sille, mitä hyväksi näet".
\par 12 Niin kutsuttiin kuninkaan kirjurit ensimmäisessä kuussa, sen kolmantenatoista päivänä, ja kirjoitettiin, aivan niinkuin Haaman käski, määräys kuninkaan satraapeille ja jokaisen maakunnan käskynhaltijoille ja jokaisen kansan ruhtinaille, kuhunkin maakuntaan sen omalla kirjoituksella ja kullekin kansalle sen omalla kielellä. Kuningas Ahasveroksen nimessä se kirjoitettiin ja sinetöitiin kuninkaan sinettisormuksella.
\par 13 Ja juoksijain mukana lähetettiin kaikkiin kuninkaan maakuntiin kirjeet, että oli hävitettävä, tapettava ja tuhottava kaikki juutalaiset, nuoret ja vanhat, lapset ja vaimot, samana päivänä, kahdennentoista kuun, se on adar-kuun, kolmantenatoista päivänä, ja että oli ryöstettävä, mitä heiltä oli saatavana saalista.
\par 14 Kirjeen jäljennös oli julkaistava lakina jokaisessa maakunnassa, tiedoksi kaikille kansoille, että olisivat valmiit tuona päivänä.
\par 15 Juoksijat lähtivät kuninkaan käskystä kiiruusti matkaan kohta, kun laki oli annettu Suusanin linnassa. Kuningas ja Haaman istuivat juomaan, mutta Suusanin kaupunki oli hämmästyksissään.

\chapter{4}

\par 1 Kun Mordokai sai tietää kaiken, mitä oli tapahtunut, repäisi Mordokai vaatteensa, pukeutui säkkiin ja tuhkaan ja meni keskelle kaupunkia ja huuteli kovia ja katkeria valitushuutoja.
\par 2 Ja hän meni kuninkaan portin edustalle asti, sillä sisälle kuninkaan porttiin ei saanut mennä säkkiin puettuna.
\par 3 Jokaisessa maakunnassa, joka paikassa, mihin kuninkaan käsky ja hänen lakinsa tuli, syntyi juutalaisten keskuudessa suuri suru: paastottiin, itkettiin ja valitettiin; monet levittivät allensa säkin ja tuhkaa.
\par 4 Kun Esterin palvelijattaret ja hänen hoviherransa tulivat ja kertoivat hänelle tämän, joutui kuningatar suureen tuskaan, ja hän lähetti vaatteita, että Mordokai puettaisiin niihin ja että hän riisuisi säkin yltään, mutta hän ei ottanut niitä vastaan.
\par 5 Niin Ester kutsui Hatakin, joka oli kuninkaan hoviherroja ja jonka tämä oli asettanut häntä palvelemaan, ja käski hänet Mordokain luo, saadakseen tietää, mitä ja mistä syystä tämä kaikki oli.
\par 6 Niin Hatak meni Mordokain luo kaupungin torille, joka on kuninkaan portin edustalla,
\par 7 ja Mordokai ilmoitti hänelle kaikki, mitä hänelle oli tapahtunut, ja myös tarkalleen, kuinka paljon hopeata Haaman oli luvannut punnita kuninkaan aarrekammioihin juutalaisten tuhoamisesta.
\par 8 Myös jäljennöksen sen lain sanamuodosta, joka Suusanissa oli annettu heidän hävittämisekseen, hän antoi hänelle, että hän näyttäisi sen Esterille ja ilmoittaisi tälle asian sekä velvoittaisi häntä menemään kuninkaan luo anomaan armoa ja rukoilemaan häntä kansansa puolesta.
\par 9 Niin Hatak meni ja kertoi Esterille Mordokain sanat.
\par 10 Mutta Ester puhui Hatakille ja käski hänen sanoa Mordokaille:
\par 11 "Kaikki kuninkaan palvelijat ja kuninkaan maakuntien kansa tietävät, että kuka ikinä, mies tai nainen, kutsumatta menee kuninkaan luo sisempään esipihaan, on laki sama jokaiselle: hänet surmataan; ainoastaan se, jota kohti kuningas ojentaa kultavaltikkansa, jää eloon. Mutta minua ei ole kolmeenkymmeneen päivään kutsuttu tulemaan kuninkaan tykö."
\par 12 Kun Mordokaille kerrottiin Esterin sanat,
\par 13 käski Mordokai vastata Esterille: "Älä luulekaan, että sinä, kun olet kuninkaan linnassa, yksin kaikista juutalaisista pelastut.
\par 14 Jos sinä tänä aikana olet vaiti, tulee apu ja pelastus juutalaisille muualta, mutta sinä ja sinun isäsi perhe tuhoudutte. Kuka tietää, etkö sinä juuri tällaista aikaa varten ole päässyt kuninkaalliseen arvoon?"
\par 15 Niin Ester käski vastata Mordokaille:
\par 16 "Mene ja kokoa kaikki Suusanin juutalaiset, ja paastotkaa minun puolestani; olkaa syömättä ja juomatta kolme vuorokautta, yöt ja päivät. Myös minä palvelijattarineni samoin paastoan. Sitten minä menen kuninkaan tykö, vaikka se on vastoin lakia; ja jos tuhoudun, niin tuhoudun."
\par 17 Niin Mordokai meni ja teki, aivan niinkuin Ester oli häntä käskenyt.

\chapter{5}

\par 1 Kolmantena päivänä Ester pukeutui kuninkaallisesti ja astui kuninkaan linnan sisempään esipihaan, vastapäätä kuninkaan linnaa; ja kuningas istui kuninkaallisella valtaistuimellaan kuninkaallisessa linnassa vastapäätä linnan ovea.
\par 2 Kun kuningas näki kuningatar Esterin seisovan esipihassa, sai tämä armon hänen silmiensä edessä, ja kuningas ojensi Esteriä kohti kultavaltikan, joka hänellä oli kädessä. Ja Ester astui esiin ja kosketti valtikan päätä.
\par 3 Kuningas sanoi hänelle: "Mikä sinun on, kuningatar Ester, ja mitä pyydät? Se sinulle annetaan, olkoon vaikka puoli valtakuntaa."
\par 4 Niin Ester vastasi: "Jos kuningas hyväksi näkee, tulkoon kuningas ja myös Haaman tänä päivänä pitoihin, jotka minä olen hänelle laittanut".
\par 5 Kuningas sanoi: "Noutakaa kiiruusti Haaman tehdäksemme Esterin toivomuksen mukaan". Ja kun kuningas ja Haaman olivat tulleet pitoihin, jotka Ester oli laittanut,
\par 6 kysyi kuningas juominkien aikana Esteriltä: "Mitä pyydät? Se sinulle annetaan. Ja mitä haluat? Se täytetään, olkoon vaikka puoli valtakuntaa."
\par 7 Niin Ester vastasi sanoen: "Tätä minä pyydän ja haluan:
\par 8 jos olen saanut armon kuninkaan silmien edessä ja jos kuningas näkee hyväksi antaa minulle, mitä pyydän, ja täyttää, mitä haluan, niin tulkoot kuningas ja Haaman vielä pitoihin, jotka minä heille laitan; huomenna minä teen kuninkaan toivomuksen mukaan".
\par 9 Haaman lähti sieltä sinä päivänä iloisena ja hyvillä mielin; ja kun Haaman näki Mordokain kuninkaan portissa eikä tämä noussut ylös eikä näyttänyt pelkäävän häntä, täytti viha Mordokaita kohtaan Haamanin.
\par 10 Mutta Haaman hillitsi itsensä ja meni kotiinsa. Sitten hän lähetti noutamaan ystävänsä ja vaimonsa Sereksen.
\par 11 Ja Haaman kehuskeli heille rikkautensa loistoa ja poikiensa paljoutta, ja kuinka kuningas kaikessa oli korottanut hänet ja ylentänyt hänet ylemmäksi ruhtinaita ja kuninkaan palvelijoita.
\par 12 Ja Haaman sanoi: "Eipä kuningatar Esterkään antanut laittamiinsa pitoihin kuninkaan kanssa tulla kenenkään muun kuin minun. Ja huomiseksikin minä olen kuninkaan kanssa kutsuttu hänen luoksensa.
\par 13 Mutta tämä kaikki ei minua tyydytä, niin kauan kuin minä näen juutalaisen Mordokain istuvan kuninkaan portissa."
\par 14 Niin hänen vaimonsa Seres ja kaikki hänen ystävänsä sanoivat hänelle: "Tehtäköön hirsipuu, viittäkymmentä kyynärää korkea, ja pyydä huomenaamuna kuninkaalta, että Mordokai ripustetaan siihen, niin voit mennä kuninkaan kanssa pitoihin iloisena". Tämä puhe miellytti Haamania, ja hän teetti hirsipuun.

\chapter{6}

\par 1 Sinä yönä uni pakeni kuningasta. Niin hän käski tuoda muistettavain tapahtumain kirjan, aikakirjan.
\par 2 Ja kun sitä luettiin kuninkaalle, huomattiin kirjoitetun, kuinka Mordokai oli ilmaissut, että Bigtan ja Teres, kaksi kuninkaan hoviherraa, ovenvartijoita, oli etsinyt tilaisuutta käydäkseen käsiksi kuningas Ahasverokseen.
\par 3 Niin kuningas kysyi: "Mitä kunniaa ja korotusta on Mordokai tästä saanut?" Kuninkaan palvelijat, jotka toimittivat hänelle palvelusta, vastasivat: "Ei hän ole saanut mitään".
\par 4 Kuningas kysyi: "Kuka on esipihassa?" Mutta Haaman oli tullut kuninkaan palatsin ulompaan esipihaan pyytämään kuninkaalta, että Mordokai ripustettaisiin hirsipuuhun, jonka hän oli pystyttänyt Mordokain varalle.
\par 5 Niin kuninkaan palvelijat vastasivat hänelle: "Katso, esipihassa seisoo Haaman".
\par 6 Kuningas sanoi: "Tulkoon sisään". Kun Haaman oli tullut, kysyi kuningas häneltä: "Mitä on tehtävä miehelle, jota kuningas tahtoo kunnioittaa?" Niin Haaman ajatteli sydämessään: "Kenellepä muulle kuningas tahtoisi osoittaa kunniaa kuin minulle?"
\par 7 Ja Haaman vastasi kuninkaalle: "Miehelle, jota kuningas tahtoo kunnioittaa,
\par 8 tuotakoon kuninkaallinen puku, johon kuningas on ollut puettuna, ja hevonen, jolla kuningas on ratsastanut ja jonka päähän on pantu kuninkaallinen kruunu.
\par 9 Puku ja hevonen annettakoon jollekin kuninkaan ruhtinaista, ylimyksistä. Puettakoon siihen mies, jota kuningas tahtoo kunnioittaa, ja kuljetettakoon häntä sen hevosen selässä kaupungin torilla ja huudettakoon hänen edellänsä: 'Näin tehdään miehelle, jota kuningas tahtoo kunnioittaa'."
\par 10 Kuningas sanoi Haamanille: "Hae joutuin puku ja hevonen, niinkuin sanoit, ja tee näin juutalaiselle Mordokaille, joka istuu kuninkaan portissa. Älä jätä tekemättä mitään kaikesta, mitä olet puhunut."
\par 11 Niin Haaman otti puvun ja hevosen, puki Mordokain ja kuljetti häntä hevosen selässä kaupungin torilla ja huusi hänen edellänsä: "Näin tehdään miehelle, jota kuningas tahtoo kunnioittaa".
\par 12 Sitten Mordokai palasi takaisin kuninkaan porttiin, mutta Haaman kiiruhti kotiinsa murehtien ja pää peitettynä.
\par 13 Kun Haaman kertoi vaimollensa Serekselle ja kaikille ystävilleen kaiken, mitä hänelle oli tapahtunut, sanoivat hänelle hänen viisaansa ja hänen vaimonsa Seres: "Jos Mordokai, jonka edessä olet alkanut kaatua, on syntyjään juutalainen, et sinä voi hänelle mitään, vaan kaadut hänen eteensä maahan asti".
\par 14 Heidän vielä puhuessaan hänen kanssansa, tulivat kuninkaan hoviherrat ja veivät kiiruusti Haamanin pitoihin, jotka Ester oli laittanut.

\chapter{7}

\par 1 Kun kuningas ja Haaman olivat tulleet kuningatar Esterin luo juomaan,
\par 2 sanoi kuningas juominkien aikana Esterille nytkin, toisena päivänä: "Mitä pyydät, kuningatar Ester? Se sinulle annetaan. Ja mitä haluat? Se täytetään, olkoon vaikka puoli valtakuntaa."
\par 3 Niin kuningatar Ester vastasi ja sanoi: "Jos olen saanut armon sinun silmiesi edessä, kuningas, ja jos kuningas hyväksi näkee, niin annettakoon minulle oma henkeni, kun sitä pyydän, ja kansani, kun sitä haluan.
\par 4 Sillä meidät on myyty, minut ja kansani, hävitettäviksi, tapettaviksi ja tuhottaviksi. Jos meidät olisi myyty vain orjiksi ja orjattariksi, olisin siitä vaiti, sillä sen vaivan takia ei kannattaisi kuningasta vaivata."
\par 5 Niin sanoi kuningas Ahasveros ja kysyi kuningatar Esteriltä: "Kuka se on, ja missä se on, joka on rohjennut tehdä sellaista?"
\par 6 Ester vastasi: "Se vastustaja ja vihamies on tuo paha Haaman". Niin Haaman peljästyi kuningasta ja kuningatarta.
\par 7 Kuningas nousi vihoissaan juomingeista ja meni linnan puutarhaan, mutta Haaman jäi siihen pyytääkseen henkeänsä kuningatar Esteriltä, sillä hän näki, että kuninkaalla oli paha mielessä häntä vastaan.
\par 8 Kun kuningas tuli linnan puutarhasta takaisin linnaan, jossa juomingit olivat, oli Haaman juuri heittäytymässä lepovuoteelle, jolla kuningatar Ester lepäsi. Niin kuningas sanoi: "Vai tekee hän vielä väkivaltaa kuningattarelle linnassa, minun läsnäollessani!" Tuskin oli tämä sana päässyt kuninkaan suusta, kun jo Haamanin kasvot peitettiin.
\par 9 Ja Harbona, yksi kuningasta palvelevista hoviherroista, sanoi: "Katso, seisoohan Haamanin talon edessä hirsipuu, viittäkymmentä kyynärää korkea, se, minkä Haaman on teettänyt Mordokain varalle, jonka puhe kerran koitui kuninkaan hyväksi". Kuningas sanoi: "Ripustakaa hänet siihen".
\par 10 Niin ripustettiin Haaman hirsipuuhun, jonka hän oli pystyttänyt Mordokain varalle. Sitten kuninkaan viha asettui.

\chapter{8}

\par 1 Sinä päivänä kuningas Ahasveros lahjoitti kuningatar Esterille Haamanin, juutalaisten vastustajan, talon. Ja Mordokai pääsi kuninkaan eteen, sillä Ester oli ilmoittanut, mikä Mordokai oli hänelle.
\par 2 Ja kuningas otti kädestään sinettisormuksensa, jonka hän oli otattanut pois Haamanilta, ja antoi sen Mordokaille; ja Ester pani Mordokain Haamanin talon hoitajaksi.
\par 3 Mutta Ester puhui vielä kuninkaan edessä ja lankesi hänen jalkainsa juureen, itki ja rukoili häntä torjumaan agagilaisen Haamanin pahuuden ja sen juonen, jonka tämä oli punonut juutalaisia vastaan.
\par 4 Niin kuningas ojensi Esteriä kohti kultavaltikan, ja Ester nousi ja seisoi kuninkaan edessä.
\par 5 Ja hän sanoi: "Jos kuningas hyväksi näkee ja jos minä olen saanut armon hänen edessänsä ja kuningas sen soveliaaksi katsoo ja minä olen hänen silmissänsä otollinen, niin kirjoitettakoon määräys ja peruutettakoon agagilaisen Haamanin, Hammedatan pojan, juoni, ne kirjeet, jotka hän kirjoitutti tuhotaksensa juutalaiset kaikissa kuninkaan maakunnissa.
\par 6 Sillä kuinka minä jaksaisin nähdä kansaani kohtaavan onnettomuuden, kuinka jaksaisin nähdä sukuni surman!"
\par 7 Niin kuningas Ahasveros sanoi kuningatar Esterille ja juutalaiselle Mordokaille: "Katso, Haamanin talon minä olen lahjoittanut Esterille, ja hän itse on ripustettu hirsipuuhun, koska hän oli käynyt käsiksi juutalaisiin.
\par 8 Ja nyt kirjoittakaa kuninkaan nimessä sellainen juutalaisia koskeva määräys, kuin hyväksi näette, ja sinetöikää se kuninkaan sinettisormuksella; sillä kirjelmä, joka on kirjoitettu kuninkaan nimessä ja sinetöity kuninkaan sinettisormuksella, on peruuttamaton."
\par 9 Niin kutsuttiin kuninkaan kirjurit silloin, kolmannessa kuussa, se on siivan-kuussa, sen kahdentenakymmenentenä kolmantena päivänä, ja kirjoitettiin, aivan niinkuin Mordokai käski, määräys juutalaisille sekä satraapeille, käskynhaltijoille ja maaherroille sataan kahteenkymmeneen seitsemään maakuntaan, Intiasta Etiopiaan saakka, kuhunkin maakuntaan sen omalla kirjoituksella ja kullekin kansalle sen omalla kielellä; myös juutalaisille heidän kirjoituksellaan ja kielellään.
\par 10 Hän kirjoitutti kuningas Ahasveroksen nimessä ja sinetöi kuninkaan sinettisormuksella. Ja hän lähetti ratsulähettien mukana, jotka ratsastivat tammatarhoissa kasvatetuilla hovin hevosilla, kirjeet:
\par 11 että kuningas sallii juutalaisten jokaisessa kaupungissa, missä heitä onkin, kokoontua puolustamaan henkeänsä hävittämällä, tappamalla ja tuhoamalla kansan ja maakunnan kaiken aseväen, joka heitä ahdistaa, sekä myös lapset ja vaimot, ja ryöstämään, mitä heiltä on saatavana saalista,
\par 12 samana päivänä kaikissa kuningas Ahasveroksen maakunnissa, kahdennentoista kuun, se on adar-kuun, kolmantenatoista päivänä.
\par 13 Kirjeen jäljennös oli julkaistava lakina jokaisessa maakunnassa, tiedoksi kaikille kansoille, ja että juutalaiset olisivat valmiit sinä päivänä kostamaan vihollisillensa.
\par 14 Hovin hevosilla ratsastavat lähetit lähtivät kuninkaan käskystä kiiruusti ja nopeasti matkaan kohta, kun laki oli annettu Suusanin linnassa.
\par 15 Mutta Mordokai lähti kuninkaan luota puettuna kuninkaalliseen punasiniseen purppuraan ja pellavapukuun, ja hänellä oli suuri kultakruunu ja viitta, tehty valkoisesta pellavakankaasta ja purppuranpunaisesta kankaasta. Ja Suusanin kaupunki riemuitsi ja oli iloissaan.
\par 16 Juutalaisille oli tullut onni, ilo, riemu ja kunnia,
\par 17 ja jokaisessa maakunnassa ja kaupungissa, joka paikassa, mihin kuninkaan käsky ja hänen lakinsa tuli, oli juutalaisilla ilo ja riemu, pidot ja juhlat. Ja paljon oli maan kansoista niitä, jotka kääntyivät juutalaisiksi, sillä kauhu juutalaisia kohtaan oli vallannut heidät.

\chapter{9}

\par 1 Kahdennessatoista kuussa, se on adar-kuussa, sen kolmantenatoista päivänä, jona kuninkaan käsky ja hänen lakinsa oli toimeenpantava, sinä päivänä, jona juutalaisten viholliset olivat toivoneet saavansa heidät valtaansa, mutta jona päinvastoin juutalaiset saivat valtaansa vihamiehensä,
\par 2 kokoontuivat juutalaiset kaupungeissansa kaikissa kuningas Ahasveroksen maakunnissa käydäkseen käsiksi niihin, jotka hankkivat heille onnettomuutta, eikä kukaan kestänyt heidän edessänsä, sillä kauhu heitä kohtaan oli vallannut kaikki kansat.
\par 3 Ja kaikki maaherrat, satraapit ja käskynhaltijat ja kuninkaan virkamiehet kannattivat juutalaisia, sillä kauhu Mordokaita kohtaan oli vallannut heidät.
\par 4 Sillä Mordokai oli mahtava kuninkaan palatsissa, ja hänen maineensa levisi kaikkiin maakuntiin, sillä se mies, Mordokai, tuli yhä mahtavammaksi.
\par 5 Ja juutalaiset voittivat kaikki vihollisensa lyöden miekoilla, surmaten ja tuhoten, ja tekivät vihamiehillensä, mitä tahtoivat.
\par 6 Suusanin linnassa juutalaiset tappoivat ja tuhosivat viisisataa miestä.
\par 7 Ja Parsandatan, Dalfonin, Aspatan,
\par 8 Pooratan, Adaljan, Aridatan,
\par 9 Parmastan, Arisain, Aridain ja Vaisatan,
\par 10 kymmenen Haamanin, Hammedatan pojan, juutalaisten vastustajan, poikaa, he tappoivat; mutta saaliiseen he eivät käyneet käsiksi.
\par 11 Sinä päivänä tuli Suusanin linnassa tapettujen luku kuninkaan tietoon.
\par 12 Ja kuningas sanoi kuningatar Esterille: "Suusanin linnassa juutalaiset ovat tappaneet ja tuhonneet viisisataa miestä ja Haamanin kymmenen poikaa; muissa kuninkaan maakunnissa mitä lienevätkään tehneet! Mitä nyt pyydät? Se sinulle annetaan. Ja mitä vielä haluat? Se täytetään."
\par 13 Niin Ester sanoi: "Jos kuningas hyväksi näkee, sallittakoon Suusanin juutalaisten huomennakin tehdä saman lain mukaan kuin tänä päivänä. Ja ripustettakoon Haamanin kymmenen poikaa hirsipuuhun."
\par 14 Kuningas käski tehdä niin. Ja siitä annettiin laki Suusanissa. Ja Haamanin kymmenen poikaa ripustettiin.
\par 15 Ja Suusanin juutalaiset kokoontuivat myös adar-kuun neljäntenätoista päivänä ja tappoivat Suusanissa kolmesataa miestä; mutta saaliiseen he eivät käyneet käsiksi.
\par 16 Myös muut juutalaiset, jotka olivat kuninkaan maakunnissa, olivat kokoontuneet puolustamaan henkeänsä ja päässeet rauhaan vihollisistansa, tapettuaan vihamiehiään seitsemänkymmentäviisi tuhatta - käymättä käsiksi saaliiseen -
\par 17 adar-kuun kolmantenatoista päivänä; ja he lepäsivät sen kuun neljännentoista päivän viettäen sen pito- ja ilopäivänä.
\par 18 Mutta Suusanin juutalaiset olivat kokoontuneet sen kuun kolmantenatoista ja neljäntenätoista päivänä, ja he lepäsivät sen kuun viidennentoista päivän viettäen sen pito- ja ilopäivänä.
\par 19 Sentähden maaseudun juutalaiset, jotka asuvat maaseutukaupungeissa, viettävät adar-kuun neljännentoista päivän ilo-, pito- ja juhlapäivänä ja lähettävät toisilleen maistiaisia.
\par 20 Ja Mordokai pani kirjaan nämä tapaukset. Ja hän lähetti kirjeet kaikille juutalaisille kuningas Ahasveroksen kaikkiin maakuntiin, lähellä ja kaukana oleville,
\par 21 säätäen heille, että heidän oli vietettävä adar-kuun neljättätoista päivää ja saman kuun viidettätoista päivää joka vuosi,
\par 22 koska juutalaiset niinä päivinä olivat päässeet rauhaan vihollisistansa ja heille siinä kuussa murhe oli kääntynyt iloksi ja suru juhlaksi - että heidän oli vietettävä ne päivät pito- ja ilopäivinä ja lähetettävä toisilleen maistiaisia ja köyhille lahjoja.
\par 23 Ja juutalaiset ottivat pysyväksi tavaksi, mitä jo olivat alkaneet tehdä ja mitä Mordokai oli heille kirjoittanut.
\par 24 Koska agagilainen Haaman, Hammedatan poika, kaikkien juutalaisten vastustaja, oli punonut juonen juutalaisia vastaan tuhotakseen heidät ja oli heittänyt puur'in, se on arvan, hävittääkseen ja tuhotakseen heidät,
\par 25 ja koska sen tultua kuninkaan tietoon tämä oli kirjeellisesti määrännyt, että Haamanin pahan juonen, jonka hän oli punonut juutalaisia vastaan, tuli kääntyä hänen omaan päähänsä ja että hänet ja hänen poikansa oli ripustettava hirsipuuhun,
\par 26 sentähden antoivat he näille päiville nimeksi puurim, puur-sanan mukaan. Sentähden, tuon käskykirjeen koko sisällyksen johdosta ja sen johdosta, mitä he itse olivat näin nähneet ja mitä heille oli tapahtunut,
\par 27 juutalaiset säätivät ja ottivat itsellensä ja jälkeläisillensä ja kaikille heihin liittyville muuttumattomaksi ja pysyväksi tavaksi, että näitä kahta päivää oli vietettävä määräyksen mukaisesti ja määräaikana joka vuosi
\par 28 ja että jokaisen sukupolven ja suvun oli näitä päiviä muistettava ja vietettävä joka maakunnassa ja kaupungissa sekä että näiden puurim-päivien tuli säilyä muuttumattomina juutalaisten keskuudessa eikä niiden muisto saanut hävitä heidän jälkeläisistänsä.
\par 29 Ja kuningatar Ester, Abihailin tytär, ja juutalainen Mordokai kirjoittivat kaiken valtansa nojalla kirjoituksia saattaakseen säädöksenä voimaan tämän toisen puurim-käskykirjeen.
\par 30 Mordokai lähetti kirjeet, ystävällisin ja vilpittömin sanoin, kaikille juutalaisille Ahasveroksen valtakunnan sataan kahteenkymmeneen seitsemään maakuntaan,
\par 31 saattaakseen säädöksenä voimaan nämä puurim-päivät niiden määräaikoina, niinkuin juutalainen Mordokai ja kuningatar Ester olivat niistä säätäneet ja niinkuin juutalaiset itse olivat säätäneet itselleen ja jälkeläisilleen määräykset niihin kuuluvista paastoista ja valitushuudoista.
\par 32 Näin säädettiin Esterin käskystä nämä puurim-määräykset ja kirjoitettiin kirjaan.

\chapter{10}

\par 1 Kuningas Ahasveros saattoi työveron alaisiksi sekä mannermaan että merensaaret.
\par 2 Ja kaikki hänen valta- ja urotyönsä ja kertomus Mordokain suuruudesta, johon kuningas hänet korotti, ne ovat kirjoitettuina Meedian ja Persian kuningasten aikakirjassa.
\par 3 Sillä juutalainen Mordokai oli kuningas Ahasveroksen lähin mies ja oli suuri juutalaisten keskuudessa ja rakas lukuisille veljillensä, koska hän harrasti kansansa parasta ja puhui koko heimonsa onnen puolesta.


\end{document}