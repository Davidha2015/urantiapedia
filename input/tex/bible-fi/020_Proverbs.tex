\begin{document}

\title{Sananlaskujen kirja}


\chapter{1}

\par 1 Salomon, Daavidin pojan, Israelin kuninkaan, sananlaskut,
\par 2 viisauden ja kurin oppimiseksi, ymmärryksen sanojen ymmärtämiseksi,
\par 3 taitoa tuovan kurin, vanhurskauden, oikeuden ja vilpittömyyden saamiseksi,
\par 4 mielevyyden antamiseksi yksinkertaisille, tiedon ja taidollisuuden nuorille.
\par 5 Viisas kuulkoon ja saakoon oppia lisää, ja ymmärtäväinen hankkikoon elämänohjetta
\par 6 ymmärtääkseen sananlaskuja ja vertauksia, viisasten sanoja ja heidän ongelmiansa.
\par 7 Herran pelko on tiedon alku; hullut pitävät halpana viisauden ja kurin.
\par 8 Kuule, poikani, isäsi kuritusta äläkä hylkää äitisi opetusta,
\par 9 sillä ne ovat ihana seppele sinun päähäsi ja käädyt sinun kaulaasi.
\par 10 Poikani, jos synnintekijät sinua viekoittelevat, älä suostu.
\par 11 Jos he sanovat: "Lähde mukaamme! Väijykäämme verta, vaanikaamme viatonta syyttömästi;
\par 12 nielaiskaamme niinkuin tuonela heidät elävältä, ehyeltään, niinkuin hautaan vaipuvaiset;
\par 13 me saamme kaikenlaista kallista tavaraa, täytämme talomme saaliilla;
\par 14 heitä arpasi meidän kanssamme, yhteinen kukkaro olkoon meillä kaikilla" -
\par 15 älä lähde, poikani, samalle tielle kuin he, pidätä jalkasi heidän poluiltansa.
\par 16 Sillä heidän jalkansa juoksevat pahuuteen, kiiruhtavat vuodattamaan verta.
\par 17 Sillä verkko on viritetty kaikille siivekkäille, niin että ne sen näkevät. - Mutta turhaan:
\par 18 omaa vertansa he väijyvät, vaanivat omaa henkeänsä.
\par 19 Näin käy jokaiselle väärän voiton pyytäjälle: se ottaa haltijaltaan hengen.
\par 20 Viisaus huutaa kadulla, antaa äänensä kuulua toreilla;
\par 21 meluisten katujen kulmissa se kutsuu, porttien ovilta kaupungissa sanansa sanoo:
\par 22 Kuinka kauan te, yksinkertaiset, rakastatte yksinkertaisuutta, kuinka kauan pilkkaajilla on halu pilkkaan ja tyhmät vihaavat tietoa?
\par 23 Kääntykää minun nuhdeltavikseni. Katso, minä vuodatan teille henkeäni, saatan sanani tiedoksenne.
\par 24 Kun minä kutsuin ja te estelitte, kun ojensin kättäni eikä kenkään kuunnellut,
\par 25 vaan te vieroksuitte kaikkia minun neuvojani, ette suostuneet minun nuhteisiini,
\par 26 niin minäkin nauran teidän hädällenne, pilkkaan, kun tulee se, mitä te kauhistutte;
\par 27 kun myrskynä tulee se, mitä te kauhistutte, kun hätänne saapuu tuulispäänä, kun päällenne tulee vaiva ja ahdistus.
\par 28 Silloin he minua kutsuvat, mutta minä en vastaa, etsivät minua, mutta eivät löydä.
\par 29 Koska he vihasivat tietoa, eivät valinneet osaksensa Herran pelkoa
\par 30 eivätkä suostuneet minun neuvooni, vaan katsoivat kaiken minun nuhteluni halvaksi,
\par 31 saavat he syödä oman vaelluksensa hedelmiä ja saavat kyllänsä omista hankkeistaan.
\par 32 Sillä yksinkertaiset tappaa heidän oma luopumuksensa, ja tyhmät tuhoaa heidän oma suruttomuutensa.
\par 33 Mutta joka minua kuulee, saa asua turvassa ja olla rauhassa onnettomuuden kauhuilta.

\chapter{2}

\par 1 Poikani, jos sinä otat minun sanani varteen ja kätket mieleesi minun käskyni,
\par 2 niin että herkistät korvasi viisaudelle ja taivutat sydämesi taitoon -
\par 3 niin, jos kutsut ymmärrystä ja ääneesi huudat taitoa,
\par 4 jos haet sitä kuin hopeata ja etsit sitä kuin aarretta,
\par 5 silloin pääset ymmärtämään Herran pelon ja löydät Jumalan tuntemisen.
\par 6 Sillä Herra antaa viisautta, hänen suustansa lähtee tieto ja taito.
\par 7 Oikeamielisille hänellä on tallella pelastus, kilpi nuhteettomasti vaeltaville,
\par 8 niin että hän suojaa oikeuden polut ja varjelee hurskaittensa tien.
\par 9 Silloin ymmärrät vanhurskauden ja oikeuden ja vilpittömyyden - hyvyyden tien kaiken;
\par 10 sillä viisaus tulee sydämeesi, ja tieto tulee sielullesi suloiseksi,
\par 11 taidollisuus on sinua varjeleva ja ymmärrys suojeleva sinut.
\par 12 Se pelastaa sinut pahojen tiestä, miehestä, joka kavalasti puhuu;
\par 13 niistä, jotka ovat hyljänneet oikeat polut vaeltaaksensa pimeyden teitä;
\par 14 niistä, jotka iloitsevat pahanteosta, riemuitsevat häijystä kavaluudesta,
\par 15 joiden polut ovat mutkaiset ja jotka joutuvat väärään teillänsä. -
\par 16 Se pelastaa sinut irstaasta naisesta, vieraasta vaimosta, joka sanoillansa liehakoitsee,
\par 17 joka on hyljännyt nuoruutensa ystävän ja unhottanut Jumalansa liiton.
\par 18 Sillä hänen huoneensa kallistuu kohti kuolemaa, hänen tiensä haamuja kohden.
\par 19 Ei palaja kenkään, joka hänen luoksensa menee, eikä saavu elämän poluille.
\par 20 Niin sinä vaellat hyvien tietä ja noudatat vanhurskasten polkuja.
\par 21 Sillä oikeamieliset saavat asua maassa, ja nuhteettomat jäävät siihen jäljelle;
\par 22 mutta jumalattomat hävitetään maasta, ja uskottomat siitä reväistään pois.

\chapter{3}

\par 1 Poikani, älä unhota minun opetustani, vaan sinun sydämesi säilyttäköön minun käskyni;
\par 2 sillä pitkää ikää, elinvuosia ja rauhaa ne sinulle kartuttavat.
\par 3 Laupeus ja uskollisuus älkööt hyljätkö sinua. Sido ne kaulaasi, kirjoita ne sydämesi tauluun,
\par 4 niin saat armon ja hyvän ymmärryksen Jumalan ja ihmisten silmien edessä.
\par 5 Turvaa Herraan kaikesta sydämestäsi äläkä nojaudu omaan ymmärrykseesi.
\par 6 Tunne hänet kaikilla teilläsi, niin hän sinun polkusi tasoittaa.
\par 7 Älä ole viisas omissa silmissäsi. Pelkää Herraa ja karta pahaa.
\par 8 Se on terveellistä sinun ruumiillesi ja virkistävää sinun luillesi.
\par 9 Kunnioita Herraa antamalla varoistasi ja kaiken satosi parhaimmasta,
\par 10 niin sinun jyväaittasi täyttyvät runsaudella, ja viini pursuu sinun kuurnistasi.
\par 11 Poikani, älä pidä Herran kuritusta halpana äläkä kyllästy hänen rangaistukseensa;
\par 12 sillä jota Herra rakastaa, sitä hän rankaisee, niinkuin isä poikaa, joka hänelle rakas on.
\par 13 Autuas se ihminen, joka on löytänyt viisauden, ihminen, joka on saanut taidon.
\par 14 Sillä parempi on hankkia sitä kuin hopeata, ja siitä saatu voitto on kultaa jalompi.
\par 15 Se on kalliimpi kuin helmet, eivät mitkään kalleutesi vedä sille vertaa.
\par 16 Pitkä ikä on sen oikeassa kädessä, vasemmassa rikkaus ja kunnia.
\par 17 Sen tiet ovat suloiset tiet, sen polut rauhaisat kaikki tyynni.
\par 18 Elämän puu on se niille, jotka siihen tarttuvat; onnelliset ne, jotka siitä pitävät kiinni.
\par 19 Herra on viisaudella perustanut maan, taivaat taidolla vahvistanut.
\par 20 Hänen toimestansa syvyydet kuohuivat esiin, ja pilvet pisaroivat kastetta.
\par 21 Poikani, nämä älkööt häipykö näkyvistäsi, säilytä neuvokkuus ja taidollisuus,
\par 22 niin ne ovat elämä sinun sielullesi ja kaunistus sinun kaulaasi.
\par 23 Silloin sinä kuljet tiesi turvallisesti etkä loukkaa jalkaasi.
\par 24 Kun menet maata, et pelkää mitään, ja maata mentyäsi on unesi makea.
\par 25 Pääset peljästymästä äkkikauhistuksia ja turmiota, joka jumalattomat yllättää.
\par 26 Sillä sinä saat luottaa Herraan, hän varjelee sinun jalkasi joutumasta kiinni.
\par 27 Älä kiellä tarvitsevalta hyvää, milloin sitä tehdä voit.
\par 28 Älä sano lähimmäisellesi: "Mene nyt ja tule toiste, huomenna minä annan", kun sinulla kuitenkin on.
\par 29 Älä mieti pahaa lähimmäistäsi vastaan, kun hän luottavaisesti luonasi asuu.
\par 30 Älä riitele kenenkään kanssa syyttä, kun ei toinen ole sinulle pahaa tehnyt.
\par 31 Älä kadehdi väkivallan miestä äläkä hänen teitänsä omiksesi valitse;
\par 32 sillä väärämielinen on Herralle kauhistus, mutta oikeamielisille hän on tuttava.
\par 33 Herran kirous on jumalattoman huoneessa, mutta vanhurskasten asuinsijaa hän siunaa.
\par 34 Pilkkaajille hänkin on pilkallinen, mutta nöyrille hän antaa armon.
\par 35 Viisaat perivät kunnian, mutta tyhmäin osa on häpeä.

\chapter{4}

\par 1 Kuulkaa, lapset, isän kuritusta ja kuunnelkaa oppiaksenne ymmärrystä.
\par 2 Sillä minä annan teille hyvän neuvon, älkää hyljätkö minun opetustani.
\par 3 Olinhan minäkin isäni poika, hento ja äitini ainokainen.
\par 4 Ja isä minua opetti ja sanoi minulle: "Pitäköön sydämesi minun sanoistani kiinni, noudata minun käskyjäni, niin sinä saat elää.
\par 5 Hanki viisautta, hanki ymmärrystä, älä sitä unhota, älä väisty pois minun suuni sanoista.
\par 6 Älä sitä hylkää, niin se varjelee sinua; rakasta sitä, niin se sinua suojaa.
\par 7 Viisauden alku on: hanki viisautta, ja kaikella muulla hankkimallasi hanki ymmärrystä.
\par 8 Anna sille korkea arvo, niin se sinut korottaa, se kunnioittaa sinua, jos sen syliisi suljet;
\par 9 se panee päähäsi ihanan seppeleen ja lahjoittaa sinulle kauniin kruunun."
\par 10 Kuule, poikani, ja ota sanani varteen, niin elämäsi vuodet enentyvät.
\par 11 Minä neuvon sinut viisauden tielle, ohjaan sinut oikeille teille.
\par 12 Käydessäsi eivät askeleesi ahtaalle joudu; juostessasi et kompastu.
\par 13 Tartu kiinni kuritukseen äläkä hellitä; säilytä se, sillä se on sinun elämäsi.
\par 14 Älä lähde jumalattomien polulle, älä astu pahojen tielle.
\par 15 Anna sen olla, älä mene sille, poikkea pois ja mene ohitse.
\par 16 Sillä eivät he saa nukkua, elleivät pahaa tee; se riistää heiltä unen, elleivät ole ketään kaataneet.
\par 17 Niin he syövät leipänään jumalattomuutta, juovat viininään väkivallan tekoja.
\par 18 Mutta vanhurskasten polku on kuin aamurusko, joka kirkastuu kirkastumistaan sydänpäivään saakka.
\par 19 Jumalattomain tie on kuin pimeys: eivät he tiedä, mihin kompastuvat.
\par 20 Poikani, kuuntele minun puhettani, kallista korvasi minun sanoilleni.
\par 21 Älkööt ne väistykö silmistäsi, kätke ne sydämesi sisimpään;
\par 22 sillä ne ovat elämä sille, joka ne löytää, ja lääke koko hänen ruumiillensa.
\par 23 Yli kaiken varottavan varjele sydämesi, sillä sieltä elämä lähtee.
\par 24 Poista itsestäsi suun kavaluus, ja karkoita luotasi huulten vääryys.
\par 25 Katsokoot sinun silmäsi suoraan, eteenpäin olkoon katseesi luotu.
\par 26 Tasoita polku jaloillesi, ja kaikki sinun tiesi olkoot vakaat.
\par 27 Älä poikkea oikeaan, älä vasempaan, väistä jalkasi pahasta.

\chapter{5}

\par 1 Poikani, kuuntele minun viisauttani, kallista korvasi minun taidolleni
\par 2 ottaaksesi vaarin taidollisuudesta, ja huulesi säilyttäkööt tiedon.
\par 3 Sillä hunajaa tiukkuvat vieraan vaimon huulet, hänen suunsa on öljyä liukkaampi.
\par 4 Mutta lopulta hän on karvas kuin koiruoho, terävä kuin kaksiteräinen miekka.
\par 5 Hänen jalkansa kulkevat alas kuolemaan, tuonelaan vetävät hänen askeleensa.
\par 6 Ei käy hän elämän tasaista polkua, hänen tiensä horjuvat hänen huomaamattaan.
\par 7 Niinpä, lapset, kuulkaa minua, älkää väistykö minun suuni sanoista.
\par 8 Pidä tiesi kaukana tuollaisesta äläkä lähesty hänen majansa ovea,
\par 9 ettet antaisi muille kunniaasi etkä vuosiasi armottomalle,
\par 10 ettei sinun tavarasi ravitsisi vieraita, sinun vaivannäkösi joutuisi toisen taloon
\par 11 ja ettet lopulta päätyisi huokailemaan ruumiisi ja lihasi riutuessa
\par 12 ja sanomaan: "Miksi minä kuritusta vihasin ja sydämeni halveksui nuhdetta?
\par 13 Miksi en kuullut neuvojaini ääntä, kallistanut korvaani opettajilleni?
\par 14 Olin joutua kokonaan turmion omaksi keskellä seurakunnan ja kansankokouksen."
\par 15 Juo vettä omasta säiliöstäsi, sitä, mikä omasta kaivostasi juoksee.
\par 16 Vuotaisivatko sinun lähteesi kadulle, toreille sinun vesiojasi!
\par 17 Olkoot ne sinun omasi yksin, älkööt vierasten sinun ohessasi.
\par 18 Olkoon sinun lähteesi siunattu, ja iloitse nuoruutesi vaimosta.
\par 19 Armas peura, suloinen vuorikauris - hänen rintansa sinua aina riemulla ravitkoot, hurmautuos alati hänen rakkaudestaan.
\par 20 Miksi, poikani, hurmautuisit irstaaseen naiseen ja syleilisit vieraan vaimon povea?
\par 21 Sillä Herran silmien edessä ovat miehen tiet, ja hän tutkii kaikki hänen polkunsa.
\par 22 Jumalattoman vangitsevat hänen rikoksensa, ja hän tarttuu oman syntinsä pauloihin.
\par 23 Kurittomuuteensa hän kuolee ja suistuu harhaan suuressa hulluudessaan.

\chapter{6}

\par 1 Poikani, jos olet ketä lähimmäisellesi taannut, lyönyt kättä vieraalle;
\par 2 jos olet kietoutunut oman suusi sanoihin, joutunut suusi sanoista kiinni,
\par 3 niin tee toki, poikani, pelastuaksesi tämä, koska olet joutunut lähimmäisesi kouriin: Mene, heittäydy maahan ja ahdista lähimmäistäsi;
\par 4 älä suo silmillesi unta äläkä silmäluomillesi lepoa.
\par 5 Pelastaudu käsistä niinkuin gaselli, niinkuin lintu pyydystäjän käsistä.
\par 6 Mene, laiska, muurahaisen tykö, katso sen menoja ja viisastu.
\par 7 Vaikka sillä ei ole ruhtinasta, ei päällysmiestä eikä hallitsijaa,
\par 8 se kuitenkin hankkii leipänsä kesällä ja kokoaa varastoon ruokansa elonaikana.
\par 9 Kuinka kauan sinä, laiska, makaat, milloinka nouset unestasi?
\par 10 Nuku vielä vähän, torku vähän, makaa vähän ristissä käsin,
\par 11 niin köyhyys käy päällesi niinkuin rosvo ja puute niinkuin asestettu mies.
\par 12 Kelvoton ihminen, väärä mies on se, joka kulkee suu täynnä vilppiä,
\par 13 silmää iskee, jaloillaan merkkiä antaa, sormillansa viittoo,
\par 14 kavaluus mielessä, pahaa aina hankitsee, riitoja rakentaa.
\par 15 Sentähden hänen turmionsa tulee yhtäkkiä, tuokiossa hänet rusennetaan, eikä apua ole.
\par 16 Näitä kuutta Herra vihaa, ja seitsemää hänen sielunsa kauhistuu:
\par 17 ylpeitä silmiä, valheellista kieltä, käsiä, jotka vuodattavat viatonta verta,
\par 18 sydäntä, joka häijyjä juonia miettii, jalkoja, jotka kiiruusti juoksevat pahaan,
\par 19 väärää todistajaa, joka valheita puhuu, ja riidan rakentajaa veljesten kesken.
\par 20 Säilytä, poikani, isäsi käsky äläkä hylkää äitisi opetusta.
\par 21 Pidä ne aina sydämellesi sidottuina, kääri ne kaulasi ympärille.
\par 22 Kulkiessasi ne sinua taluttakoot, maatessasi sinua vartioikoot, herätessäsi sinua puhutelkoot.
\par 23 Sillä käsky on lamppu, opetus on valo, ja kurittava nuhde on elämän tie,
\par 24 että varjeltuisit pahasta naisesta, vieraan vaimon liukkaasta kielestä.
\par 25 Älköön sydämesi himoitko hänen kauneuttaan, älköönkä hän sinua katseillaan vangitko.
\par 26 Sillä porttonaisen tähden menee leipäkakkukin, ja naitu nainen pyydystää kallista sielua.
\par 27 Voiko kukaan kuljettaa tulta helmassaan, puvun häneltä palamatta?
\par 28 Voiko kukaan kävellä hiilloksella, jalkain häneltä kärventymättä?
\par 29 Samoin käy sen, joka menee lähimmäisensä vaimon luo: ei jää rankaisematta kukaan, joka häneen kajoaa.
\par 30 Eikö halveksita varasta, vaikka hän olisi nälissään varastanut hengenpiteikseen?
\par 31 Onhan hänen, jos tavataan, seitsenkertaisesti korvattava, annettava kaikki talonsa varat.
\par 32 Joka vaimon kanssa avion rikkoo, on mieletön; itsensä menettää, joka niin tekee.
\par 33 Hän saa vaivan ja häpeän, eikä hänen häväistystään pyyhitä pois.
\par 34 Sillä luulevaisuus nostaa miehen vihan, ja säälimätön on hän koston päivänä.
\par 35 Ei hän huoli mistään lunastusmaksusta, ei suostu, vaikka kuinka lahjaasi lisäät.

\chapter{7}

\par 1 Poikani, noudata minun sanojani ja kätke mieleesi minun käskyni.
\par 2 Noudata minun käskyjäni, niin saat elää, säilytä opetukseni kuin silmäteräsi.
\par 3 Sido ne kiinni sormiisi, kirjoita ne sydämesi tauluun.
\par 4 Sano viisaudelle: "Sinä olet sisareni", kutsu ymmärrystä sukulaiseksi,
\par 5 että varjeltuisit irstaalta naiselta, vieraalta vaimolta, joka sanoillansa liehakoitsee.
\par 6 Sillä minä katselin taloni ikkunasta ristikon läpi,
\par 7 ja minä näin yksinkertaisten joukossa, havaitsin poikain seassa nuorukaisen, joka oli mieltä vailla.
\par 8 Hän kulki katua erään naisen kulmaukseen ja asteli hänen majaansa päin
\par 9 päivän illaksi hämärtyessä, yön aikana, pimeässä.
\par 10 Ja katso, nainen tulee häntä vastaan, porton puvussa, kavala sydämeltä.
\par 11 Hän on levoton ja hillitön, eivät pysy hänen jalkansa kotona;
\par 12 milloin hän on kadulla, milloin toreilla, ja väijyy joka kulmassa.
\par 13 Hän tarttui nuorukaiseen, suuteli häntä ja julkeasti katsoen sanoi hänelle:
\par 14 "Minun oli uhrattava yhteysuhri, tänä päivänä olen täyttänyt lupaukseni.
\par 15 Sentähden läksin ulos sinua vastaan, etsiäkseni sinua, ja olen sinut löytänyt.
\par 16 Olen leposijalleni peitteitä levittänyt, kirjavaa Egyptin liinavaatetta.
\par 17 Vuoteeseeni olen pirskoitellut mirhaa, aloeta ja kanelia.
\par 18 Tule, nauttikaamme lemmestä aamuun asti, riemuitkaamme rakkaudesta.
\par 19 Sillä mieheni ei ole kotona, hän meni matkalle kauas.
\par 20 Rahakukkaron hän otti mukaansa ja tulee kotiin vasta täydenkuun päiväksi."
\par 21 Hän taivutti hänet paljolla houkuttelullaan, vietteli liukkailla huulillansa:
\par 22 äkkiä hän lähti hänen jälkeensä, niinkuin härkä menee teuraaksi, niinkuin hullu jalkaraudoissa kuritettavaksi,
\par 23 niinkuin lintu kiiruhtaa paulaan; eikä tiennyt, että oli henkeänsä kaupalla, kunnes nuoli lävisti hänen maksansa.
\par 24 Sentähden, poikani, kuulkaa minua, kuunnelkaa minun suuni sanoja.
\par 25 Älköön poiketko sydämesi tuon naisen teille, älä eksy hänen poluillensa.
\par 26 Sillä paljon on surmattuja, hänen kaatamiaan, lukuisasti niitä, jotka hän on kaikki tappanut.
\par 27 Hänen majastaan käyvät tuonelan tiet, jotka vievät alas tuonelan kammioihin.

\chapter{8}

\par 1 Eikö viisaus kutsu, eikö taito anna äänensä kuulua?
\par 2 Ylös kummuille, tien viereen, polkujen risteyksiin hän on asettunut.
\par 3 Porttien pielissä, kaupungin portilla, oviaukoissa hän huutaa:
\par 4 "Teitä minä kutsun, miehet, ja ihmislapsille kaikuu minun ääneni.
\par 5 Tulkaa, yksinkertaiset, mieleviksi; tulkaa järkeviksi, te tyhmät.
\par 6 Kuulkaa, sillä jalosti minä puhun, ja avaan huuleni puhumaan, mikä oikein on;
\par 7 sillä totuutta minun suuni haastaa, ja jumalattomuus on minun huulilleni kauhistus.
\par 8 Vanhurskaat ovat minun suuni sanat kaikki, ei ole niissä mitään petollista, ei väärää.
\par 9 Ne ovat kaikki oikeat ymmärtäväiselle, suorat niille, jotka löysivät tiedon.
\par 10 Ottakaa minun kuritukseni, älkääkä hopeata, ja tieto ennen valituinta kultaa.
\par 11 Sillä parempi on viisaus kuin helmet, eivät mitkään kalleudet vedä sille vertaa.
\par 12 Minä, viisaus, olen perehtynyt mielevyyteen, olen löytänyt tiedon ja taidollisuuden.
\par 13 Herran pelko on pahan vihaamista. Kopeutta ja ylpeyttä, pahaa tietä ja kavalaa suuta minä vihaan.
\par 14 Minulla on neuvo ja neuvokkuus; minä olen ymmärrys, minulla on voima.
\par 15 Minun avullani kuninkaat hallitsevat, ruhtinaat säädöksensä vanhurskaasti säätävät.
\par 16 Minun avullani päämiehet vallitsevat ja ylhäiset, maan tuomarit kaikki.
\par 17 Minä rakastan niitä, jotka minua rakastavat, ja jotka minua varhain etsivät, ne löytävät minut.
\par 18 Minun tykönäni on rikkaus ja kunnia, ikivanha varallisuus ja vanhurskaus.
\par 19 Minun hedelmäni on parempi kuin kulta, kuin puhtain kulta, minun antamani voitto valituinta hopeata parempi.
\par 20 Minä vaellan vanhurskauden polkua, oikeuden teitten keskikohtaa,
\par 21 antaakseni niille, jotka minua rakastavat, pysyvän perinnön ja täyttääkseni heidän aarrekammionsa.
\par 22 Herra loi minut töittensä esikoiseksi, ensimmäiseksi teoistaan, ennen aikojen alkua.
\par 23 Iankaikkisuudesta minä olen asetettu olemaan, alusta asti, hamasta maan ikiajoista.
\par 24 Ennenkuin syvyyksiä oli, synnyin minä, ennenkuin oli lähteitä, vedestä rikkaita.
\par 25 Ennenkuin vuoret upotettiin paikoilleen, ennen kukkuloita, synnyin minä,
\par 26 kun hän ei vielä ollut tehnyt maata, ei mantua, ei maanpiirin tomujen alkuakaan.
\par 27 Kun hän taivaat valmisti, olin minä siinä, kun hän veti piirin syvyyden pinnalle,
\par 28 kun hän teki vahvoiksi pilvet korkeudessa, kun syvyyden lähteet saivat voiman,
\par 29 kun hän merelle asetti sen rajat, että vedet eivät kävisi hänen käskynsä yli, kun hän vahvisti maan perustukset,
\par 30 silloin minä hänen sivullansa hoidokkina olin, ihastuksissani olin päivästä päivään ja leikitsin hänen edessänsä kaikin ajoin;
\par 31 leikitsin hänen maanpiirinsä päällä, ja ihastukseni olivat ihmislapset.
\par 32 Siis te, lapset, kuulkaa minua; autuaat ne, jotka noudattavat minun teitäni!
\par 33 Kuritusta kuulkaa, niin viisastutte; älkää sen antako mennä menojaan.
\par 34 Autuas se ihminen, joka minua kuulee, valvoo minun ovillani päivästä päivään, vartioitsee minun ovieni pieliä!
\par 35 Sillä joka minut löytää, löytää elämän ja saa Herran mielisuosion.
\par 36 Mutta joka menee minusta harhaan, saa vahingon sielullensa; kaikki, jotka minua vihaavat, rakastavat kuolemaa."

\chapter{9}

\par 1 Viisaus on talonsa rakentanut, veistänyt seitsemän pylvästänsä.
\par 2 Hän on teuraansa teurastanut, viininsä sekoittanut ja myöskin pöytänsä kattanut.
\par 3 Hän on palvelijattarensa lähettänyt kutsua kuuluttamaan kaupungin kumpujen rinteiltä:
\par 4 "Joka yksinkertainen on, poiketkoon tänne". Sille, joka on mieltä vailla, hän sanoo:
\par 5 "Tulkaa, syökää minun leipääni ja juokaa viiniä, minun sekoittamaani.
\par 6 Hyljätkää yksinkertaisuus, niin saatte elää, ja astukaa ymmärryksen tielle." -
\par 7 Joka pilkkaajaa ojentaa, saa itsellensä häpeän, ja häpeäpilkun se, joka jumalatonta nuhtelee.
\par 8 Älä nuhtele pilkkaajaa, ettei hän sinua vihaisi; nuhtele viisasta, niin hän sinua rakastaa.
\par 9 Anna viisaalle, niin hän yhä viisastuu; opeta vanhurskasta, niin hän saa oppia lisää.
\par 10 Herran pelko on viisauden alku, ja Pyhimmän tunteminen on ymmärrystä. -
\par 11 "Sillä minun avullani päiväsi enenevät ja jatkuvat elämäsi vuodet.
\par 12 Jos olet viisas, olet omaksi hyväksesi viisas; ja jos olet pilkkaaja, saat sinä sen yksin kestää."
\par 13 Tyhmyys on nainen, levoton ja yksinkertainen, eikä hän mistään mitään tiedä.
\par 14 Hän istuu talonsa ovella, istuimella kaupungin kummuilla,
\par 15 kutsumassa ohikulkijoita, jotka käyvät polkujansa suoraan eteenpäin:
\par 16 "Joka yksinkertainen on, poiketkoon tänne". Ja sille, joka on mieltä vailla, hän sanoo:
\par 17 "Varastettu vesi on makeata, ja salattu leipä on suloista".
\par 18 Eikä toinen tiedä, että haamuja on siellä, että hänen kutsuvieraansa ovat tuonelan laaksoissa.

\chapter{10}

\par 1 Salomon sananlaskut. Viisas poika on isällensä iloksi, mutta tyhmä poika on äidillensä murheeksi.
\par 2 Vääryyden aarteet eivät auta, mutta vanhurskaus vapahtaa kuolemasta.
\par 3 Herra ei salli vanhurskaan nälkää nähdä, mutta jumalattomien himon hän luotansa työntää.
\par 4 Köyhtyy, joka laiskasti kättä käyttää, mutta ahkerain käsi rikastuttaa.
\par 5 Taitava poika kokoaa kesällä, kunnoton poika elonaikana nukkuu.
\par 6 Siunaus on vanhurskaan pään päällä, mutta väkivaltaa kätkee jumalattomien suu.
\par 7 Vanhurskaan muistoa siunataan, mutta jumalattomien nimi lahoaa.
\par 8 Viisassydäminen ottaa käskyt varteen, mutta hulluhuulinen kukistuu.
\par 9 Joka nuhteettomasti vaeltaa, vaeltaa turvassa, jonka tiet ovat väärät, se joutuu ilmi.
\par 10 Joka silmää iskee, saa aikaan tuskaa, ja hulluhuulinen kukistuu.
\par 11 Vanhurskaan suu on elämän lähde, mutta jumalattomien suu kätkee väkivaltaa.
\par 12 Viha virittää riitoja, mutta rakkaus peittää rikkomukset kaikki.
\par 13 Ymmärtäväisen huulilta löytyy viisaus, mutta joka on mieltä vailla, sille vitsa selkään!
\par 14 Viisaat kätkevät, minkä tietävät, mutta hullun suu on läheinen turmio.
\par 15 Rikkaan tavara on hänen vahva kaupunkinsa, mutta vaivaisten köyhyys on heidän turmionsa.
\par 16 Vanhurskaan hankkima on elämäksi, jumalattoman saalis koituu synniksi.
\par 17 Kuritusta noudattava on elämän tiellä, mutta nuhteet hylkäävä eksyy.
\par 18 Joka salavihaa pitää, sen huulilla on valhe, ja joka parjausta levittää, on tyhmä.
\par 19 Missä on paljon sanoja, siinä ei syntiä puutu; mutta joka huulensa hillitsee, se on taitava.
\par 20 Vanhurskaan kieli on valituin hopea, jumalattomien äly on tyhjän veroinen.
\par 21 Vanhurskaan huulet kaitsevat monia, mutta hullut kuolevat mielettömyyteensä.
\par 22 Herran siunaus rikkaaksi tekee, ei oma vaiva siihen mitään lisää.
\par 23 Tyhmälle on iloksi ilkityön teko, mutta ymmärtäväiselle miehelle viisaus.
\par 24 Mitä jumalaton pelkää, se häntä kohtaa; mutta mitä vanhurskaat halajavat, se annetaan.
\par 25 Tuulispään käytyä ei jumalatonta enää ole, mutta vanhurskaan perustus pysyy iäti.
\par 26 Mitä hapan hampaille ja savu silmille, sitä laiska lähettäjillensä.
\par 27 Herran pelko elinpäiviä jatkaa, mutta jumalattomien vuodet lyhenevät.
\par 28 Vanhurskasten odotus koituu iloksi, mutta jumalattomien toivo hukkuu.
\par 29 Herran johdatus on nuhteettoman turva, mutta väärintekijäin turmio.
\par 30 Vanhurskas ei ikinä horju, mutta jumalattomat eivät saa asua maassa.
\par 31 Vanhurskaan suu kasvaa viisauden hedelmän, mutta kavala kieli hävitetään.
\par 32 Vanhurskaan huulet tietävät, mikä otollista on, mutta jumalattomien suu on sulaa kavaluutta.

\chapter{11}

\par 1 Väärä vaaka on Herralle kauhistus, mutta täysi paino on hänelle otollinen.
\par 2 Mihin ylpeys tulee, sinne tulee häpeäkin, mutta nöyräin tykönä on viisaus.
\par 3 Oikeamielisiä ohjaa heidän nuhteettomuutensa, mutta uskottomat hävittää heidän vilppinsä.
\par 4 Ei auta tavara vihan päivänä, mutta vanhurskaus vapahtaa kuolemasta.
\par 5 Nuhteettoman vanhurskaus tasoittaa hänen tiensä, mutta jumalaton sortuu jumalattomuuteensa.
\par 6 Oikeamieliset vapahtaa heidän vanhurskautensa, mutta uskottomat vangitsee heidän oma himonsa.
\par 7 Jumalattoman ihmisen kuollessa hukkuu hänen toivonsa, ja vääräin odotus hukkuu.
\par 8 Vanhurskas pelastetaan hädästä, ja jumalaton joutuu hänen sijaansa.
\par 9 Rietas suullansa turmelee lähimmäisensä, mutta taito on vanhurskaitten pelastus.
\par 10 Vanhurskaitten onnesta kaupunki iloitsee, ja jumalattomain hukkumisesta syntyy riemu.
\par 11 Oikeamielisten siunauksesta kaupunki kohoaa, mutta jumalattomain suu sitä hajottaa.
\par 12 Mieltä vailla on, joka lähimmäistänsä halveksii, mutta ymmärtäväinen mies on vaiti.
\par 13 Joka panettelijana käy, ilmaisee salaisuuden, mutta jolla luotettava henki on, se säilyttää asian.
\par 14 Missä ohjausta ei ole, sortuu kansa, mutta neuvonantajain runsaus tuo menestyksen.
\par 15 Joka vierasta takaa, sen käy pahoin, mutta joka kädenlyöntiä vihaa, se on turvattu.
\par 16 Suloinen nainen saa kunniaa, ja voimalliset saavat rikkautta.
\par 17 Armelias mies tekee hyvää itsellensä, mutta armoton syöksee onnettomuuteen oman lihansa.
\par 18 Jumalaton hankkii pettäväistä voittoa, mutta joka vanhurskautta kylvää, saa pysyvän palkan.
\par 19 Joka on vakaa vanhurskaudessa, saa elämän; mutta joka pahaa tavoittaa, saa kuoleman.
\par 20 Väärämieliset ovat Herralle kauhistus, mutta nuhteettomasti vaeltaviin hän mielistyy.
\par 21 Totisesti: paha ei jää rankaisematta, mutta vanhurskasten jälkeläiset pelastuvat.
\par 22 Kultarengas sian kärsässä on kaunis nainen, älyä vailla.
\par 23 Vanhurskaitten halajaminen vie onneen, jumalattomien toivo vihaan.
\par 24 Toinen on antelias ja saa yhä lisää, toinen säästää yli kohtuuden ja vain köyhtyy.
\par 25 Hyväätekeväinen sielu tulee ravituksi, ja joka muita virvoittaa, se itse kostuu.
\par 26 Joka viljan pitää takanaan, sitä kansa kiroaa, mutta joka viljan kaupaksi antaa, sen pään päälle tulee siunaus.
\par 27 Joka hyvään pyrkii, etsii sitä, mikä otollista on, mutta joka pahaa etsii, sille se tulee.
\par 28 Joka rikkauteensa luottaa, se kukistuu, mutta vanhurskaat viheriöitsevät niinkuin lehvä.
\par 29 Joka talonsa rappiolle saattaa, perii tuulta, ja hullu joutuu viisaan orjaksi.
\par 30 Vanhurskaan hedelmä on elämän puu, ja viisas voittaa sieluja.
\par 31 Katso, vanhurskas saa palkkansa maan päällä, saati sitten jumalaton ja syntinen.

\chapter{12}

\par 1 Tietoa rakastaa, joka kuritusta rakastaa, mutta järjetön se, joka nuhdetta vihaa.
\par 2 Hyvä saa Herran mielisuosion, mutta juonittelijan hän tuomitsee syylliseksi.
\par 3 Ei ihminen kestä jumalattomuuden varassa, mutta vanhurskasten juuri on horjumaton.
\par 4 Kelpo vaimo on puolisonsa kruunu, mutta kunnoton on kuin mätä hänen luissansa.
\par 5 Vanhurskasten aivoitukset ovat oikeat, jumalattomien hankkeet petolliset.
\par 6 Jumalattomien puheet väijyvät verta, mutta oikeamieliset pelastaa heidän suunsa.
\par 7 Jumalattomat kukistuvat olemattomiin, mutta vanhurskasten huone pysyy.
\par 8 Ymmärryksensä mukaan miestä kiitetään, mutta nurjasydämistä halveksitaan.
\par 9 Parempi halpa-arvoinen, jolla on palvelija, kuin rehentelijä, joka on vailla leipää.
\par 10 Vanhurskas tuntee, mitä hänen karjansa kaipaa, mutta jumalattomain sydän on armoton.
\par 11 Joka peltonsa viljelee, saa leipää kyllin, mutta tyhjän tavoittelija on mieltä vailla.
\par 12 Jumalaton himoitsee pahojen saalista, mutta vanhurskasten juuri on antoisa.
\par 13 Huulten rikkomus on paha ansa, mutta vanhurskas pääsee hädästä.
\par 14 Suunsa hedelmästä saa kyllälti hyvää, ja ihmisen eteen kiertyvät hänen kättensä työt.
\par 15 Hullun tie on hänen omissa silmissään oikea, mutta joka neuvoa kuulee, on viisas.
\par 16 Hullun suuttumus tulee kohta ilmi, mutta mielevä peittää kärsimänsä häpeän.
\par 17 Toden puhuja lausuu oikeuden, mutta väärä todistaja petoksen.
\par 18 Moni viskoo sanoja kuin miekanpistoja, mutta viisasten kieli on lääke.
\par 19 Totuuden huulet pysyvät iäti, mutta valheen kieli vain tuokion.
\par 20 Jotka pahaa miettivät, niillä on mielessä petos, mutta jotka rauhaan neuvovat, niille tulee ilo.
\par 21 Ei tule turmiota vanhurskaalle, mutta jumalattomat ovat onnettomuutta täynnä.
\par 22 Herralle kauhistus ovat valheelliset huulet, mutta teoissaan uskolliset ovat hänelle otolliset.
\par 23 Mielevä ihminen peittää tietonsa, mutta tyhmäin sydän huutaa julki hulluutensa.
\par 24 Ahkerain käsi saa hallita, mutta laiska joutuu työveron alaiseksi.
\par 25 Huoli painaa alas miehen mielen, mutta hyvä sana sen ilahuttaa.
\par 26 Vanhurskas opastaa lähimmäistänsä, mutta jumalattomat eksyttää heidän oma tiensä.
\par 27 Laiska ei saa ajetuksi itselleen riistaa, mutta ahkeruus on ihmiselle kallis tavara.
\par 28 Vanhurskauden polulla on elämä, ja sen tien kulku ei ole kuolemaksi.

\chapter{13}

\par 1 Viisas poika kuulee isän kuritusta, mutta pilkkaaja ei ota nuhdetta kuullaksensa.
\par 2 Suunsa hedelmästä saa nauttia hyvää, mutta uskottomilla on halu väkivaltaan.
\par 3 Joka suistaa suunsa, se säilyttää henkensä, mutta avosuinen joutuu turmioon.
\par 4 Laiskan sielu haluaa, saamatta mitään, mutta ahkerain sielu tulee ravituksi.
\par 5 Vanhurskas vihaa valhepuhetta, mutta jumalattoman meno on iljettävä ja häpeällinen.
\par 6 Vanhurskaus varjelee nuhteettomasti vaeltavan, mutta jumalattomuus syöksee syntisen kumoon.
\par 7 Toinen on olevinaan rikas, omistamatta mitään, toinen olevinaan köyhä, vaikka on tavaraa paljon.
\par 8 Rikkautensa saa mies antaa henkensä lunnaiksi, mutta köyhän ei tarvitse uhkauksia kuunnella.
\par 9 Vanhurskasten valo loistaa iloisesti, mutta jumalattomien lamppu sammuu.
\par 10 Ylpeys tuottaa pelkkää toraa, mutta jotka ottavat neuvon varteen, niillä on viisaus.
\par 11 Tyhjällä saatu tavara vähenee, mutta joka vähin erin kokoaa, se saa karttumaan.
\par 12 Pitkä odotus tekee sydämen sairaaksi, mutta täyttynyt halu on elämän puu.
\par 13 Joka sanaa halveksii, joutuu sanan pantiksi; mutta joka käskyä pelkää, saa palkan.
\par 14 Viisaan opetus on elämän lähde kuoleman paulain välttämiseksi.
\par 15 Hyvä ymmärrys tuottaa suosiota, mutta uskottomien tie on koleikkoa.
\par 16 Jokainen mielevä toimii taitavasti, mutta tyhmä levittää hulluutta.
\par 17 Jumalaton sanansaattaja suistuu turmioon, mutta uskollinen lähetti on kuin lääke.
\par 18 Köyhyys ja häpeä kuritusta vierovalle, kunnia nuhdetta noudattavalle!
\par 19 Tyydytetty halu on sielulle suloinen, pahan karttaminen tyhmille kauhistus.
\par 20 Vaella viisasten kanssa, niin viisastut; tyhmäin seuratoverin käy pahoin.
\par 21 Syntisiä vainoaa onnettomuus, mutta vanhurskaat saavat onnen palkakseen.
\par 22 Hyvä jättää perinnön lastensa lapsillekin, mutta syntisen tavara talletetaan vanhurskaalle.
\par 23 Köyhien uudiskyntö antaa runsaan ruuan; mutta moni tuhoutuu, joka ei oikeudessa pysy.
\par 24 Joka vitsaa säästää, se vihaa lastaan; mutta joka häntä rakastaa, se häntä ajoissa kurittaa.
\par 25 Vanhurskas saa syödä kylläksensä, mutta jumalattomain vatsa jää vajaaksi.

\chapter{14}

\par 1 Vaimojen viisaus talon rakentaa, mutta hulluus sen omin käsin purkaa.
\par 2 Joka vaeltaa oikein, se pelkää Herraa, mutta jonka tiet ovat väärät, se hänet katsoo ylen.
\par 3 Hullun suussa on ylpeydelle vitsa, mutta viisaita vartioivat heidän huulensa.
\par 4 Missä raavaita puuttuu, on seimi tyhjä, mutta runsas sato saadaan härkien voimasta.
\par 5 Uskollinen todistaja ei valhettele, mutta väärä todistaja puhuu valheita.
\par 6 Pilkkaaja etsii viisautta turhaan, mutta ymmärtäväisen on tietoa helppo saada.
\par 7 Menet pois tyhmän miehen luota: et tullut tuntemaan tiedon huulia.
\par 8 Mielevän viisaus on, että hän vaelluksestaan vaarin pitää; tyhmien hulluus on petos.
\par 9 Hulluja pilkkaa vikauhri, mutta oikeamielisten kesken on mielisuosio.
\par 10 Sydän tuntee oman surunsa, eikä sen iloon saa vieras sekaantua.
\par 11 Jumalattomain huone hävitetään, mutta oikeamielisten maja kukoistaa.
\par 12 Miehen mielestä on oikea monikin tie, joka lopulta on kuoleman tie.
\par 13 Nauraessakin voi sydän kärsiä, ja ilon lopuksi tulee murhe.
\par 14 Omista teistään saa kyllänsä se, jolla on luopunut sydän, mutta itsestään löytää tyydytyksen hyvä mies.
\par 15 Yksinkertainen uskoo joka sanan, mutta mielevä ottaa askeleistansa vaarin.
\par 16 Viisas pelkää ja karttaa pahaa, mutta tyhmä on huoleton ja suruton.
\par 17 Pikavihainen tekee hullun töitä, ja juonittelija joutuu vihatuksi.
\par 18 Yksinkertaiset saavat perinnökseen hulluuden, mutta mielevät tiedon kruunuksensa.
\par 19 Pahojen täytyy kumartua hyvien edessä ja jumalattomien seisoa vanhurskaan porteilla.
\par 20 Köyhää vihaa hänen ystävänsäkin, mutta rikasta rakastavat monet.
\par 21 Syntiä tekee, joka lähimmäistään halveksii, mutta autuas se, joka kurjia armahtaa!
\par 22 Eivätkö eksy ne, jotka hankitsevat pahaa? Mutta armo ja totuus niille, jotka hankitsevat hyvää!
\par 23 Kaikesta vaivannäöstä tulee hyötyä, mutta tyhjästä puheesta vain vahinkoa.
\par 24 Viisasten kruunu on heidän rikkautensa, mutta tyhmäin hulluus hulluudeksi jää.
\par 25 Uskollinen todistaja on hengen pelastaja, mutta joka valheita puhuu, on petosta täynnä.
\par 26 Herran pelossa on vahva varmuus ja turva vielä lapsillekin.
\par 27 Herran pelko on elämän lähde kuoleman paulain välttämiseksi.
\par 28 Kansan paljous on kuninkaan kunnia, väen vähyys ruhtinaan turmio.
\par 29 Pitkämielisellä on paljon taitoa, mutta pikavihaisen osa on hulluus.
\par 30 Sävyisä sydän on ruumiin elämä, mutta luulevaisuus on mätä luissa.
\par 31 Joka vaivaista sortaa, se herjaa hänen Luojaansa, mutta se häntä kunnioittaa, joka köyhää armahtaa.
\par 32 Jumalaton sortuu omaan pahuuteensa, mutta vanhurskas on turvattu kuollessaan.
\par 33 Ymmärtäväisen sydämeen ottaa majansa viisaus, ja tyhmien keskellä se itsensä tiettäväksi tekee.
\par 34 Vanhurskaus kansan korottaa, mutta synti on kansakuntien häpeä.
\par 35 Taitava palvelija saa kuninkaan suosion, mutta kunnoton hänen vihansa.

\chapter{15}

\par 1 Leppeä vastaus taltuttaa kiukun, mutta loukkaava sana nostaa vihan.
\par 2 Viisasten kieli puhuu tietoa taitavasti, mutta tyhmäin suu purkaa hulluutta.
\par 3 Herran silmät ovat joka paikassa; ne vartioitsevat hyviä ja pahoja.
\par 4 Sävyisä kieli on elämän puu, mutta vilpillinen kieli haavoittaa mielen.
\par 5 Hullu pitää halpana isänsä kurituksen, mutta joka nuhdetta noudattaa, tulee mieleväksi.
\par 6 Vanhurskaan huoneessa on suuret aarteet, mutta jumalattoman saalis on turmion oma.
\par 7 Viisasten huulet kylvävät tietoa, mutta tyhmäin sydän ei ole vakaa.
\par 8 Jumalattomien uhri on Herralle kauhistus, mutta oikeamielisten rukous on hänelle otollinen.
\par 9 Jumalattoman tie on Herralle kauhistus, mutta joka vanhurskauteen pyrkii, sitä hän rakastaa.
\par 10 Kova tulee kuritus sille, joka tien hylkää; joka nuhdetta vihaa, saa kuoleman.
\par 11 Tuonelan ja manalan Herra näkee, saati sitten ihmislasten sydämet.
\par 12 Pilkkaaja ei pidä siitä, että häntä nuhdellaan; viisasten luo hän ei mene.
\par 13 Iloinen sydän kaunistaa kasvot, mutta sydämen tuskassa on mieli murtunut.
\par 14 Ymmärtäväisen sydän etsii tietoa, mutta tyhmien suu hulluutta suosii.
\par 15 Kurjalle ovat pahoja kaikki päivät, mutta hyvä mieli on kuin alituiset pidot.
\par 16 Parempi vähä Herran pelossa kuin paljot varat levottomuudessa.
\par 17 Parempi vihannesruoka rakkaudessa kuin syöttöhärkä vihassa.
\par 18 Kiukkuinen mies nostaa riidan, mutta pitkämielinen asettaa toran.
\par 19 Laiskan tie on kuin orjantappurapehko, mutta oikeamielisten polku on raivattu.
\par 20 Viisas poika on isällensä iloksi, mutta tyhmä ihminen halveksii äitiänsä.
\par 21 Hulluus on ilo sille, joka on mieltä vailla, mutta ymmärtäväinen mies kulkee suoraan.
\par 22 Hankkeet sortuvat, missä neuvonpito puuttuu; mutta ne toteutuvat, missä on runsaasti neuvonantajia.
\par 23 Miehellä on ilo suunsa vastauksesta; ja kuinka hyvä onkaan sana aikanansa!
\par 24 Taitava käy elämän tietä ylöspäin, välttääkseen tuonelan, joka alhaalla on.
\par 25 Ylpeitten huoneen Herra hajottaa, mutta lesken rajan hän vahvistaa.
\par 26 Häijyt juonet ovat Herralle kauhistus, mutta lempeät sanat ovat puhtaat.
\par 27 Väärän voiton pyytäjä hävittää huoneensa, mutta joka lahjuksia vihaa, saa elää.
\par 28 Vanhurskaan sydän miettii, mitä vastata, mutta jumalattomien suu purkaa pahuutta.
\par 29 Jumalattomista on Herra kaukana, mutta vanhurskasten rukouksen hän kuulee.
\par 30 Valoisa silmänluonti ilahuttaa sydämen; hyvä sanoma tuo ydintä luihin.
\par 31 Korva, joka kuuntelee elämän nuhdetta, saa majailla viisaitten keskellä.
\par 32 Joka kuritusta vieroo, pitää sielunsa halpana; mutta joka nuhdetta kuuntelee, se saa mieltä.
\par 33 Herran pelko on kuri viisauteen, ja kunnian edellä käy nöyryys.

\chapter{16}

\par 1 Ihmisen ovat mielen aivoittelut, mutta Herralta tulee kielen vastaus.
\par 2 Kaikki miehen tiet ovat hänen omissa silmissään puhtaat, mutta Herra tutkii henget.
\par 3 Heitä työsi Herran haltuun, niin sinun hankkeesi menestyvät.
\par 4 Kaiken on Herra tehnyt määrätarkoitukseen, niinpä jumalattomankin onnettomuuden päivän varalle.
\par 5 Jokainen ylpeämielinen on Herralle kauhistus: totisesti, ei sellainen jää rankaisematta.
\par 6 Laupeudella ja uskollisuudella rikos sovitetaan, ja Herran pelolla paha vältetään.
\par 7 Jos miehen tiet ovat Herralle otolliset, saattaa hän vihamiehetkin sovintoon hänen kanssansa.
\par 8 Parempi vähä vanhurskaudessa kuin suuret voitot vääryydessä.
\par 9 Ihmisen sydän aivoittelee hänen tiensä, mutta Herra ohjaa hänen askeleensa.
\par 10 Kuninkaan huulilla on jumalallinen ratkaisu; hänen suunsa ei petä tuomitessaan.
\par 11 Puntari ja oikea vaaka ovat Herran, hänen tekoaan ovat kaikki painot kukkarossa.
\par 12 Jumalattomuuden teko on kuninkaille kauhistus, sillä vanhurskaudesta valtaistuin vahvistuu.
\par 13 Vanhurskaat huulet ovat kuninkaille mieleen, ja oikein puhuvaa he rakastavat.
\par 14 Kuninkaan viha on kuoleman sanansaattaja, mutta sen lepyttää viisas mies.
\par 15 Kuninkaan kasvojen valo on elämäksi, ja hänen suosionsa on kuin keväinen sadepilvi.
\par 16 Parempi kultaa on hankkia viisautta, kalliimpi hopeata hankkia ymmärrystä.
\par 17 Oikeamielisten tie välttää onnettomuuden; henkensä saa pitää, joka pitää vaelluksestansa vaarin.
\par 18 Kopeus käy kukistumisen edellä, ylpeys lankeemuksen edellä.
\par 19 Parempi alavana nöyrien parissa kuin jakamassa saalista ylpeitten kanssa.
\par 20 Joka painaa mieleensä sanan, se löytää onnen; ja autuas se, joka Herraan turvaa!
\par 21 Jolla on viisas sydän, sitä ymmärtäväiseksi sanotaan, ja huulten suloisuus antaa opetukselle tehoa.
\par 22 Ymmärrys on omistajalleen elämän lähde, mutta hulluus on hulluille kuritus.
\par 23 Viisaan sydän tekee taitavaksi hänen suunsa ja antaa tehoa hänen huultensa opetukselle.
\par 24 Lempeät sanat ovat mesileipää; ne ovat makeat sielulle ja lääkitys luille.
\par 25 Miehen mielestä on oikea monikin tie, joka lopulta on kuoleman tie.
\par 26 Työmiehen nälkä tekee työtä hänen hyväkseen, sillä oma suu panee hänelle pakon.
\par 27 Kelvoton mies kaivaa toiselle onnettomuutta; hänen huulillaan on kuin polttava tuli.
\par 28 Kavala mies rakentaa riitaa, ja panettelija erottaa ystävykset.
\par 29 Väkivallan mies viekoittelee lähimmäisensä ja vie hänet tielle, joka ei ole hyvä.
\par 30 Joka silmiänsä luimistelee, sillä on kavaluus mielessä; joka huulensa yhteen puristaa, sillä on paha valmiina.
\par 31 Harmaat hapset ovat kunnian kruunu; se saadaan vanhurskauden tiellä.
\par 32 Pitkämielinen on parempi kuin sankari, ja mielensä hillitseväinen parempi kuin kaupungin valloittaja.
\par 33 Helmassa pudistellen arpa heitetään, mutta Herralta tulee aina sen ratkaisu.

\chapter{17}

\par 1 Parempi kuiva kannikka rauhassa kuin talon täysi uhripaistia riidassa.
\par 2 Taitava palvelija hallitsee kunnotonta poikaa ja pääsee perinnönjaolle veljesten rinnalla.
\par 3 Hopealle sulatin, kullalle uuni, mutta sydämet koettelee Herra.
\par 4 Paha kuuntelee häijyjä huulia, petollisuus kuulee pahoja kieliä.
\par 5 Joka köyhää pilkkaa, se herjaa hänen luojaansa; joka toisen onnettomuudesta iloitsee, ei jää rankaisematta.
\par 6 Vanhusten kruunu ovat lastenlapset, ja isät ovat lasten kunnia.
\par 7 Ei sovi houkalle ylevä puhe, saati sitten ruhtinaalle valhe.
\par 8 Lahjus on käyttäjänsä silmissä kallis kivi: mihin vain hän kääntyy, hän menestyy.
\par 9 Joka rikkeen peittää, se rakkautta harrastaa; mutta joka asioita kaivelee, se erottaa ystävykset.
\par 10 Nuhde pystyy paremmin ymmärtäväiseen kuin sata lyöntiä tyhmään.
\par 11 Pelkkää onnettomuutta hankkii kapinoitsija, mutta häntä vastaan lähetetään armoton sanansaattaja.
\par 12 Kohdatkoon miestä karhu, jolta on riistetty poikaset, mutta älköön tyhmä hulluudessansa.
\par 13 Joka hyvän pahalla palkitsee, sen kodista ei onnettomuus väisty.
\par 14 Alottaa tora on päästää vedet valloilleen; herkeä, ennenkuin riita syttyy.
\par 15 Syyllisen syyttömäksi ja syyttömän syylliseksi tekijä ovat kumpikin Herralle kauhistus.
\par 16 Mitä hyötyä on rahasta tyhmän käsissä? Viisauden hankkimiseen ei ole ymmärrystä.
\par 17 Ystävä rakastaa ainiaan ja veli syntyy varaksi hädässä.
\par 18 Mieltä vailla on mies, joka kättä lyöpi, joka menee toista takaamaan.
\par 19 Joka toraa rakastaa, se rikkomusta rakastaa; joka ovensa korottaa, se hankkii kukistumistaan.
\par 20 Väärämielinen ei onnea löydä, ja kavalakielinen suistuu onnettomuuteen.
\par 21 Tyhmä on murheeksi siittäjällensä, ja houkan isä on iloa vailla.
\par 22 Terveydeksi on iloinen sydän, mutta murtunut mieli kuivuttaa luut.
\par 23 Jumalaton ottaa lahjuksen vastaan toisen povelta vääristääksensä oikeuden tiet.
\par 24 Ymmärtäväisellä on viisaus kasvojensa edessä, mutta tyhmän silmät kiertävät maailman rantaa.
\par 25 Tyhmä poika on isällensä suruksi ja synnyttäjällensä mielihaikeaksi.
\par 26 Paha jo sekin, jos syytöntä sakotetaan; kovin kohtuutonta, jos jaloja lyödään.
\par 27 Joka hillitsee sanansa, on taitava, ja mielensä malttava on ymmärtäväinen mies.
\par 28 Hullukin käy viisaasta, jos vaiti on; joka huulensa sulkee, on ymmärtäväinen.

\chapter{18}

\par 1 Eriseurainen noudattaa omia pyyteitään; kaikin neuvoin hän riitaa haastaa.
\par 2 Tyhmän halu ei ole ymmärrykseen, vaan tuomaan julki oma mielensä.
\par 3 Kunne jumalaton tulee, tulee ylenkatsekin, ja häpeällisen menon mukana häväistys.
\par 4 Syviä vesiä ovat sanat miehen suusta, ovat virtaava puro ja viisauden lähde.
\par 5 Ei ole hyvä pitää syyllisen puolta ja vääräksi vääntää syyttömän asiaa oikeudessa.
\par 6 Tyhmän huulet tuovat mukanaan riidan, ja hänen suunsa kutsuu lyöntejä.
\par 7 Oma suu on tyhmälle turmioksi ja omat huulet ansaksi hänelle itselleen.
\par 8 Panettelijan puheet ovat kuin herkkupalat ja painuvat sisusten kammioihin asti.
\par 9 Joka on veltto toimessansa, se on jo tuhontekijän veli.
\par 10 Herran nimi on vahva torni; hurskas juoksee sinne ja saa turvan.
\par 11 Rikkaan tavara on hänen vahva kaupunkinsa, ja korkean muurin kaltainen hänen kuvitteluissaan.
\par 12 Kukistumisen edellä miehen sydän ylpistyy, mutta kunnian edellä käy nöyryys.
\par 13 Jos kuka vastaa, ennenkuin on kuullut, on se hulluutta ja koituu hänelle häpeäksi.
\par 14 Miehekäs mieli pitää sairaankin pystyssä, mutta kuka voi kantaa murtunutta mieltä?
\par 15 Tietoa hankkii ymmärtäväisen sydän, tietoa etsii viisasten korva.
\par 16 Lahja avartaa alat ihmiselle ja vie hänet isoisten pariin.
\par 17 Käräjissä on kukin ensiksi oikeassa, mutta sitten tulee hänen riitapuolensa ja ottaa hänestä selvän.
\par 18 Arpa riidat asettaa ja ratkaisee väkevien välit.
\par 19 Petetty veli on vaikeampi voittaa kuin vahva kaupunki, ja riidat ovat kuin linnan salvat.
\par 20 Suunsa hedelmästä saa mies vatsansa kylläiseksi, saa kyllikseen huultensa satoa.
\par 21 Kielellä on vallassansa kuolema ja elämä; jotka sitä rakastavat, saavat syödä sen hedelmää.
\par 22 Joka vaimon löysi, se onnen löysi, sai Herralta mielisuosion.
\par 23 Köyhä puhuu pyydellen, mutta rikas vastaa tylysti.
\par 24 Häviökseen mies on monien ystävä, mutta on myös ystäviä, veljiäkin uskollisempia.

\chapter{19}

\par 1 Parempi on köyhä, joka nuhteettomasti vaeltaa, kuin huuliltansa nurja, joka on vielä tyhmäkin.
\par 2 Ilman taitoa ei ole intokaan hyväksi, ja kiirehtivän jalka astuu harhaan.
\par 3 Ihmisen oma hulluus turmelee hänen tiensä, mutta Herralle hän sydämessään vihoittelee.
\par 4 Tavara tuo ystäviä paljon, mutta vaivainen joutuu ystävästänsä eroon.
\par 5 Väärä todistaja ei jää rankaisematta, ja joka valheita puhuu, se ei pelastu.
\par 6 Monet etsivät ylhäisen suosiota, ja kaikki ovat anteliaan ystäviä.
\par 7 Köyhä on kaikkien veljiensä vihattu, vielä vierotumpi ystävilleen. Tyhjiä sanoja hän saa tavoitella.
\par 8 Joka mieltä hankkii, se sieluansa rakastaa; joka ymmärryksen säilyttää, se onnen löytää.
\par 9 Väärä todistaja ei jää rankaisematta, ja joka valheita puhuu, se hukkuu.
\par 10 Ei sovi tyhmälle hyvät päivät, saati sitten palvelijalle hallita ruhtinaita.
\par 11 Ymmärrys tekee ihmisen pitkämieliseksi, ja hänen kunniansa on antaa rikos anteeksi.
\par 12 Kuninkaan viha on kuin nuoren leijonan kiljunta, mutta hänen suosionsa on kuin kaste ruoholle.
\par 13 Tyhmä poika on isänsä turmio, ja vaimon tora on kuin räystäästä tippuva vesi.
\par 14 Talo ja tavara peritään isiltä, mutta toimellinen vaimo tulee Herralta.
\par 15 Laiskuus vaivuttaa sikeään uneen, ja veltto joutuu näkemään nälkää.
\par 16 Joka käskyt pitää, saa henkensä pitää; joka ei teistänsä välitä, on kuoleman oma.
\par 17 Joka vaivaista armahtaa, se lainaa Herralle, ja hän maksaa jälleen hänen hyvän tekonsa.
\par 18 Kurita poikaasi, kun vielä toivoa on; ethän halunne hänen kuolemaansa.
\par 19 Rajuluontoinen joutuu sakkoihin: vain yllytät, jos yrität apuun.
\par 20 Kuule neuvoa, ota kuritus varteen, että olisit vasta viisaampi.
\par 21 Monet ovat miehen mielessä aivoitukset, mutta Herran neuvo on pysyväinen.
\par 22 Ihaninta ihmisessä on hänen laupeutensa, ja köyhä on parempi kuin valhettelija.
\par 23 Herran pelko on elämäksi: saa levätä yönsä ravittuna, eikä mikään paha kohtaa.
\par 24 Laiska pistää kätensä vatiin, mutta ei saa sitä viedyksi suuhunsa jälleen.
\par 25 Lyö pilkkaajaa, niin yksinkertainen saa mieltä, ja jos ymmärtäväistä nuhdellaan, niin hän käsittää tiedon.
\par 26 Joka isäänsä pahoin pitelee ja ajaa äitinsä pois, se on kunnoton ja rietas poika.
\par 27 Jos herkeät, poikani, kuulemasta kuritusta, niin eksyt pois tiedon sanoista.
\par 28 Kelvoton todistaja pitää oikeuden pilkkanaan, ja jumalattomien suu nielee vääryyttä.
\par 29 Tuomiot ovat valmiina pilkkaajille ja lyönnit tyhmien selkään.

\chapter{20}

\par 1 Viini on pilkkaaja, väkijuoma remunpitäjä; eikä ole viisas kenkään, joka siitä hoipertelee.
\par 2 Kuninkaan peljättäväisyys on kuin nuoren leijonan kiljunta; joka hänet vihoittaa, se henkensä rikkoo.
\par 3 On kunniaksi miehelle riitaa karttaa, mutta kaikki hullut riitaa haastavat.
\par 4 Syksyllä ei laiska kynnä; elonaikana hän tyhjää tapailee.
\par 5 Kuin syvät vedet ovat miehen sydämen aivoitukset, mutta ymmärtäväinen mies ne ammentaa esiin.
\par 6 Monet huutavat hyvyyttänsä kukin, mutta kuka löytää luotettavan miehen?
\par 7 Vanhurskas vaeltaa nuhteettomasti, onnelliset ovat lapset hänen jälkeensä.
\par 8 Kuningas istuu tuomioistuimella, hän silmillänsä perkaa kaiken pahan pois.
\par 9 Kuka voi sanoa: "Olen puhdistanut sydämeni, olen puhdas synnistäni"?
\par 10 Kahtalainen paino ja kahtalainen mitta - molemmat ovat Herralle kauhistus.
\par 11 Teoistansa tuntee jo poikasenkin, onko hänen menonsa puhdas ja oikea.
\par 12 Kuulevan korvan ja näkevän silmän - molemmat on Herra luonut.
\par 13 Älä unta rakasta, ettet köyhtyisi; pidä silmäsi auki, niin saat leipää kyllin.
\par 14 "Huonoa, huonoa", sanoo ostaja, mutta mentyänsä pois hän kehuskelee.
\par 15 Olkoon kultaa, olkoon helmiä paljon, kallein kalu ovat taidolliset huulet.
\par 16 Ota siltä vaatteet, joka toista takasi, ja ota häneltä pantti vieraitten puolesta.
\par 17 Makea on miehelle petoksella saatu leipä, mutta perästäpäin hän saa suunsa täyteen soraa.
\par 18 Neuvotellen suunnitelmat vahvistuvat, ja sotaa on sinun käytävä neuvokkuudella.
\par 19 Salaisuuden ilmaisee, joka panettelijana käy; älä siis antaudu avosuisen pariin.
\par 20 Joka isäänsä ja äitiänsä kiroaa, sen lamppu sammuu pilkkopimeään.
\par 21 Tavara, jota aluksi kiivaasti tavoitellaan, ei lopulta tuo siunausta.
\par 22 Älä sano: "Minä kostan pahan"; odota Herraa, hän auttaa sinua.
\par 23 Kahtalainen paino on Herralle kauhistus, ja väärä vaaka ei ole hyvä.
\par 24 Herralta tulevat miehen askeleet; mitäpä ymmärtäisi tiestänsä ihminen itse?
\par 25 Ihmiselle on ansaksi luvata hätiköiden pyhä lahja ja vasta perästäpäin harkita lupauksiansa.
\par 26 Viisas kuningas perkaa jumalattomat pois viskimellään ja antaa puimajyrän käydä heidän ylitsensä.
\par 27 Ihmisen henki on Herran lamppu: se tutkistelee sydämen kammiot kaikki.
\par 28 Laupeus ja uskollisuus on kuninkaan turva, ja laupeudella hän valtaistuimensa tukee.
\par 29 Voima on nuorukaisen kunnia, ja harmaat hapset ovat vanhusten kaunistus.
\par 30 Mustelmat ja haavat puhdistavat pahantekijän, lyönnit puhdistavat sydämen kammiot.

\chapter{21}

\par 1 Kuninkaan sydän on Herran kädessä kuin vesiojat: hän taivuttaa sen, kunne tahtoo.
\par 2 Kaikki miehen tiet ovat hänen omissa silmissään oikeat, mutta Herra tutkii sydämet.
\par 3 Vanhurskauden ja oikeuden harjoittaminen on Herralle otollisempi kuin uhri.
\par 4 Ylpeät silmät ja pöyhkeä sydän - jumalattomien lamppu - ovat syntiä.
\par 5 Vain hyödyksi ovat ahkeran ajatukset, mutta kaikki touhuilijat saavat vain vahinkoa.
\par 6 Jotka hankkivat aarteita petollisin kielin, ovat haihtuva tuulahdus, hakevat kuolemaa.
\par 7 Jumalattomat tempaa pois heidän väkivaltansa, sillä eivät he tahdo oikeutta tehdä.
\par 8 Rikollisen tie on mutkainen, mutta puhtaan teot ovat oikeat.
\par 9 Parempi asua katon kulmalla kuin toraisan vaimon huonetoverina.
\par 10 Jumalattoman sielu himoitsee pahaa, lähimmäinen ei saa armoa hänen silmiensä edessä.
\par 11 Kun pilkkaajaa rangaistaan, viisastuu yksinkertainen; ja kun viisasta neuvotaan, ottaa hän sen opiksensa.
\par 12 Vanhurskas Jumala tarkkaa jumalattoman taloa, hän syöksee jumalattomat onnettomuuteen.
\par 13 Joka tukkii korvansa vaivaisen huudolta, se joutuu itse huutamaan, vastausta saamatta.
\par 14 Salainen lahja lepyttää vihan ja poveen kätketty lahjus kiivaan kiukun.
\par 15 Vanhurskaalle on ilo, kun oikeutta tehdään, mutta väärintekijöille kauhu.
\par 16 Ihminen, joka eksyy taidon tieltä, joutuu lepäämään haamujen seuraan.
\par 17 Puutteen mieheksi päätyy riemujen rakastaja, eikä rikastu se, joka viiniä ja öljyä rakastaa.
\par 18 Jumalaton joutuu lunastusmaksuksi vanhurskaan puolesta ja uskoton oikeamielisten sijaan.
\par 19 Parempi asua autiossa maassa kuin toraisan vaimon vaivattavana.
\par 20 Kalliita aarteita ja öljyä on viisaan majassa, mutta tyhmä ihminen syö suuhunsa sellaiset.
\par 21 Joka vanhurskauteen ja laupeuteen pyrkii, se löytää elämän, vanhurskauden ja kunnian.
\par 22 Viisas ryntää sankarien kaupunkiin ja kukistaa varustuksen, joka oli sen turva.
\par 23 Joka suunsa ja kielensä varoo, se henkensä ahdistuksilta varoo.
\par 24 Pilkkaajan nimen saa julkea röyhkeilijä, jonka meno on määrätöntä julkeutta.
\par 25 Oma halu laiskan tappaa, sillä hänen kätensä eivät tahdo työtä tehdä.
\par 26 Aina on hartaasti haluavia, mutta vanhurskas antaa säästelemättä.
\par 27 Jumalattomien uhri on kauhistus; saati sitten, jos se tuodaan ilkityön edestä!
\par 28 Valheellinen todistaja hukkuu, mutta kuunteleva mies saa aina puhua.
\par 29 Julkeiksi tekee jumalaton kasvonsa, vakaiksi tekee oikeamielinen tiensä.
\par 30 Ei auta viisaus, ei ymmärrys, ei mikään neuvo Herraa vastaan.
\par 31 Hevonen on varustettu taistelun päiväksi, mutta voitto on Herran hallussa.

\chapter{22}

\par 1 Nimi on kalliimpi suurta rikkautta, suosio hopeata ja kultaa parempi.
\par 2 Rikas ja köyhä kohtaavat toisensa; Herra on luonut kumpaisenkin.
\par 3 Mielevä näkee vaaran ja kätkeytyy, mutta yksinkertaiset käyvät kohti ja saavat vahingon.
\par 4 Nöyryyden ja Herran pelon palkka on rikkaus, kunnia ja elämä.
\par 5 Orjantappuroita ja pauloja on väärän tiellä; henkensä varjelee, joka niistä kaukana pysyy.
\par 6 Totuta poikanen tiensä suuntaan, niin hän ei vanhanakaan siitä poikkea.
\par 7 Rikas hallitsee köyhiä, ja velallinen joutuu velkojan orjaksi.
\par 8 Joka vääryyttä kylvää, se turmiota niittää, ja hänen vihansa vitsa häviää.
\par 9 Hyvänsuopa saa siunauksen, sillä hän antaa leivästään vaivaiselle.
\par 10 Aja pois pilkkaaja, niin poistuu tora ja loppuu riita ja häväistys.
\par 11 Joka sydämen puhtautta rakastaa, jolla on suloiset huulet, sen ystävä on kuningas.
\par 12 Herran silmät suojelevat taitoa, mutta uskottoman sanat hän kääntää väärään.
\par 13 Laiska sanoo: "Ulkona on leijona; tappavat vielä minut keskellä toria".
\par 14 Irstaitten vaimojen suu on syvä kuoppa; Herran vihan alainen kaatuu siihen.
\par 15 Hulluus on kiertynyt kiinni poikasen sydämeen, mutta kurituksen vitsa sen hänestä kauas karkoittaa.
\par 16 Vaivaiselle on voitoksi, jos häntä sorretaan, rikkaalle tappioksi, jos hänelle annetaan.
\par 17 Kallista korvasi ja kuuntele viisaitten sanoja ja tarkkaa minun taitoani.
\par 18 Sillä suloista on, jos kätket ne sisimpääsi; olkoot ne kaikki huulillasi valmiina.
\par 19 Että Herra olisi sinun turvanasi, siksi olen minä nyt neuvonut juuri sinua.
\par 20 Olenhan ennenkin sinulle kirjoittanut, antanut neuvoja ja tietoa,
\par 21 opettaakseni sinulle totuutta, vakaita sanoja, että voisit vakain sanoin vastata lähettäjällesi.
\par 22 Älä raasta vaivaista, siksi että hän on vaivainen, äläkä polje kurjaa portissa,
\par 23 sillä Herra ajaa hänen asiansa ja riistää hänen riistäjiltään hengen.
\par 24 Älä rupea pikavihaisen ystäväksi äläkä seurustele kiukkuisen kanssa,
\par 25 että et tottuisi hänen teihinsä ja saattaisi sieluasi ansaan.
\par 26 Älä ole niitä, jotka kättä lyövät, jotka menevät takuuseen veloista.
\par 27 Jollei sinulla ole, millä maksaa, mitäs muuta, kuin viedään vuode altasi!
\par 28 Älä siirrä ikivanhaa rajaa, jonka esi-isäsi ovat asettaneet.
\par 29 Jos näet miehen, kerkeän toimissaan, hänen paikkansa on kuningasten, ei alhaisten, palveluksessa.

\chapter{23}

\par 1 Kun istut aterialle hallitsijan seurassa, niin pidä tarkoin mielessä, kuka edessäsi on,
\par 2 ja pane veitsi kurkullesi, jos olet kovin nälkäinen.
\par 3 Älä himoitse hänen herkkujansa, sillä ne ovat petollisia ruokia.
\par 4 Älä näe vaivaa rikastuaksesi, lakkaa käyttämästä ymmärrystäsi siihen.
\par 5 Kun silmäsi siihen lentävät, on rikkaus mennyttä; sillä se saa siivet kuin kotka, joka lentää taivaalle.
\par 6 Älä syö pahansuovan leipää äläkä himoitse hänen herkkujansa.
\par 7 Sillä niinkuin hän mielessään laskee, niin hän menettelee: hän sanoo sinulle: "Syö ja juo", mutta hänen sydämensä ei ole sinun puolellasi.
\par 8 Syömäsi palan sinä olet oksentava, ja suloiset sanasi sinä tuhlasit turhaan.
\par 9 Älä puhu tyhmän kuullen, sillä hän katsoo ymmärtäväiset sanasi ylen.
\par 10 Älä siirrä ikivanhaa rajaa äläkä mene orpojen pelloille.
\par 11 Sillä heidän sukulunastajansa on väkevä, ja hän ajaa heidän asiansa sinua vastaan.
\par 12 Tuo sydämesi kuritettavaksi ja korvasi taidon sanojen ääreen.
\par 13 Älä kiellä poikaselta kuritusta, sillä jos lyöt häntä vitsalla, säästyy hän kuolemasta.
\par 14 Vitsalla sinä häntä lyöt, tuonelasta hänen sielunsa pelastat.
\par 15 Poikani, jos sinun sydämesi viisastuu, niin minunkin sydämeni iloitsee;
\par 16 ja sisimpäni riemuitsee, jos sinun huulesi puhuvat sitä, mikä oikein on.
\par 17 Älköön sydämesi kadehtiko jumalattomia, mutta kiivaile aina Jumalan pelon puolesta,
\par 18 niin sinulla totisesti on tulevaisuus, ja toivosi ei mene turhaan.
\par 19 Kuule, poikani, ja viisastu, ja ohjaa sydämesi oikealle tielle.
\par 20 Älä oleskele viininjuomarien parissa äläkä lihansyömärien.
\par 21 Sillä juomari ja syömäri köyhtyy, ja unteluus puettaa ryysyihin.
\par 22 Kuule isääsi, joka on sinut siittänyt, äläkä äitiäsi halveksi, kun hän on vanhennut.
\par 23 Osta totuutta, älä myy, osta viisautta, kuria ja ymmärrystä.
\par 24 Ääneen saa riemuita vanhurskaan isä; joka viisaan on siittänyt, sillä on ilo hänestä.
\par 25 Olkoon sinun isälläsi ja äidilläsi ilo, sinun synnyttäjäsi riemuitkoon.
\par 26 Anna sydämesi, poikani, minulle, ja olkoot minun tieni sinun silmissäsi mieluiset.
\par 27 Sillä portto on syvä kuoppa, ja vieras vaimo on ahdas kaivo.
\par 28 Vieläpä hän väijyy kuin rosvo, ja hän kartuttaa uskottomia ihmisten seassa.
\par 29 Kenellä on voivotus, kenellä vaikerrus? Kenellä torat, kenellä valitus? Kenellä haavat ilman syytä? Kenellä sameat silmät?
\par 30 Niillä, jotka viinin ääressä viipyvät, jotka tulevat makujuomaa maistelemaan.
\par 31 Älä katsele viiniä, kuinka se punoittaa, kuinka se maljassa hohtaa ja helposti valahtaa alas.
\par 32 Lopulta se puree kuin käärme ja pistää kuin myrkkylisko.
\par 33 Silmäsi outoja näkevät, ja sydämesi haastelee sekavia.
\par 34 Sinusta on kuin makaisit keskellä merta, on kuin maston huipussa makaisit.
\par 35 "Löivät minua, mutta ei koskenut minuun; pieksivät minua, mutta en tiennyt mitään. Milloinkahan herännen? Tahdonpa taas hakea tätä samaa."

\chapter{24}

\par 1 Älä kadehdi pahoja ihmisiä äläkä halua heidän seuraansa.
\par 2 Sillä heidän mielensä miettii väkivaltaa, ja turmiota haastavat heidän huulensa.
\par 3 Viisaudella talo rakennetaan ja ymmärryksellä vahvaksi varustetaan.
\par 4 Taidolla täytetään kammiot, kaikkea kallista ja ihanaa tavaraa täyteen.
\par 5 Viisas mies on väkevä, ja taidon mies on voipa voimaltansa.
\par 6 Neuvokkuudella näet on sinun käytävä sotaa, ja neuvonantajain runsaus tuo menestyksen.
\par 7 Kovin on korkea hullulle viisaus, ei hän suutansa avaa portissa.
\par 8 Jolla on pahanteko mielessä, sitä juonittelijaksi sanotaan.
\par 9 Synti on hulluuden työ, ja pilkkaaja on ihmisille kauhistus.
\par 10 Jos olet ollut veltto, joutuu ahtaana aikana voimasi ahtaalle.
\par 11 Pelasta ne, joita kuolemaan viedään, pysäytä ne, jotka surmapaikalle hoippuvat.
\par 12 Jos sanot: "Katso, emme tienneet siitä", niin ymmärtäähän asian sydänten tutkija; sinun sielusi vartioitsija sen tietää, ja hän kostaa ihmiselle hänen tekojensa mukaan.
\par 13 Syö hunajaa, poikani, sillä se on hyvää, ja mesi on makeaa suussasi.
\par 14 Samankaltaiseksi tunne viisaus sielullesi; jos sen löydät, on sinulla tulevaisuus, ja toivosi ei mene turhaan.
\par 15 Älä väijy, jumalaton, vanhurskaan majaa, älä hävitä hänen leposijaansa.
\par 16 Sillä seitsemästi vanhurskas lankeaa ja nousee jälleen, mutta jumalattomat suistuvat onnettomuuteen.
\par 17 Älä iloitse vihamiehesi langetessa, älköön sydämesi riemuitko hänen suistuessaan kumoon,
\par 18 ettei Herra, kun sen näkee, sitä pahana pitäisi, ja kääntäisi vihaansa pois hänestä.
\par 19 Älä vihastu pahantekijäin tähden, älä kadehdi jumalattomia.
\par 20 Sillä ei ole pahalla tulevaisuutta; jumalattomien lamppu sammuu.
\par 21 Pelkää, poikani, Herraa ja kuningasta, älä sekaannu kapinallisten seuraan.
\par 22 Sillä yhtäkkiä tulee heille onnettomuus, tuomio - kuka tietää milloin - toisille niinkuin toisillekin.
\par 23 Nämäkin ovat viisaitten sanoja. Ei ole hyvä tuomitessa henkilöön katsoa.
\par 24 Joka sanoo syylliselle: "Sinä olet syytön", sitä kansat kiroavat, kansakunnat sadattelevat.
\par 25 Mutta jotka oikein tuomitsevat, niiden käy hyvin, ja heille tulee onnen siunaus.
\par 26 Se huulille suutelee, joka oikean vastauksen antaa.
\par 27 Toimita tehtäväsi ulkona ja tee valmista pellollasi; sitten perusta itsellesi perhe.
\par 28 Älä ole syyttä todistajana lähimmäistäsi vastaan, vai petätkö sinä huulillasi?
\par 29 Älä sano: "Niinkuin hän teki minulle, niin teen minä hänelle, minä kostan miehelle hänen tekojensa mukaan".
\par 30 Minä kuljin laiskurin pellon ohitse, mielettömän miehen viinitarhan vieritse.
\par 31 Ja katso: se kasvoi yltänsä polttiaisia; sen pinta oli nokkosten peitossa ja sen kiviaita luhistunut.
\par 32 Minä katselin ja painoin mieleeni, havaitsin ja otin opikseni:
\par 33 Nuku vielä vähän, torku vähän, makaa vähän ristissä käsin,
\par 34 niin köyhyys käy päällesi niinkuin rosvo ja puute niinkuin asestettu mies.

\chapter{25}

\par 1 Nämäkin ovat Salomon sananlaskuja, Hiskian, Juudan kuninkaan, miesten kokoamia.
\par 2 Jumalan kunnia on salata asia, ja kuningasten kunnia on tutkia asia.
\par 3 Taivaan korkeus ja maan syvyys ja kuningasten sydän on tutkimaton.
\par 4 Kun hopeasta poistetaan kuona, kuontuu kultasepältä astia.
\par 5 Kun jumalaton poistetaan kuninkaan luota, vahvistuu hänen valtaistuimensa vanhurskaudessa.
\par 6 Älä tavoittele kunniaa kuninkaan edessä äläkä asetu isoisten sijalle.
\par 7 Sillä parempi on, jos sinulle sanotaan: "Käy tänne ylös", kuin että sinut alennetaan ylhäisen edessä, jonka silmäsi olivat nähneet.
\par 8 Älä ole kärkäs käräjöimään; muutoin sinulla ei lopulta ole, mitä tehdä, kun vastapuolesi on saattanut sinut häpeään.
\par 9 Riitele oma riitasi vastapuolesi kanssa, mutta toisen salaisuutta älä ilmaise.
\par 10 Muutoin sinua häpäisee, kuka sen kuuleekin, eikä huono huuto sinusta lakkaa.
\par 11 Kultaomenia hopeamaljoissa ovat sanat, sanotut aikanansa.
\par 12 Kultainen korvarengas ja hienokultainen kaulakoru ovat viisas neuvoja ynnä kuuleva korva.
\par 13 Kuin lumen viileys elonaikana on luotettava lähetti lähettäjälleen: herransa sielun hän virvoittaa.
\par 14 Kuin pilvet ja tuuli, jotka eivät sadetta tuo, on mies, joka kerskuu lahjoilla, joita ei anna.
\par 15 Kärsivällisyydellä taivutetaan ruhtinas, ja leppeä kieli murskaa luut.
\par 16 Jos hunajata löydät, syö kohtuudella, ettet kyllästyisi siihen ja sitä oksentaisi.
\par 17 Astu jalallasi harvoin lähimmäisesi kotiin, ettei hän sinuun kyllästyisi ja alkaisi sinua vihata.
\par 18 Nuija ja miekka ja terävä nuoli on mies, joka väärin todistaa lähimmäistänsä vastaan.
\par 19 Kuin mureneva hammas ja horjuva jalka on uskottoman turva ahdingon päivänä.
\par 20 Kuin se, joka riisuu vaatteet pakkaspäivänä, kuin etikka lipeän sekaan, on se, joka laulaa lauluja murheelliselle sydämelle.
\par 21 Jos vihamiehelläsi on nälkä, anna hänelle leipää syödä, ja jos hänellä on jano, anna hänelle vettä juoda.
\par 22 Sillä niin sinä kokoat tulisia hiiliä hänen päänsä päälle, ja Herra sen sinulle palkitsee.
\par 23 Pohjatuuli saa aikaan sateen ja salainen kielittely vihaiset kasvot.
\par 24 Parempi on asua katon kulmalla kuin toraisan vaimon huonetoverina.
\par 25 Kuin nääntyväiselle raikas vesi on hyvä sanoma kaukaisesta maasta.
\par 26 Kuin sekoitettu lähde ja turmeltu kaivo on vanhurskas, joka horjuu jumalattoman edessä.
\par 27 Liika hunajan syönti ei ole hyväksi, ja raskaitten asiain tutkiminen on raskasta.
\par 28 Kuin kaupunki, varustukset hajalla, muuria vailla, on mies, joka ei mieltänsä hillitse.

\chapter{26}

\par 1 Yhtä vähän kuin lumi kesällä ja sade elonaikana soveltuu tyhmälle kunnia.
\par 2 Kuin liitävä lintu, kuin lentävä pääskynen on aiheeton kirous: ei se toteen käy.
\par 3 Hevoselle ruoska, aasille suitset, tyhmille vitsa selkään!
\par 4 Älä vastaa tyhmälle hänen hulluutensa mukaan, ettet olisi hänen kaltaisensa sinäkin.
\par 5 Vastaa tyhmälle hänen hulluutensa mukaan, ettei hän itseänsä viisaana pitäisi.
\par 6 Jalat altaan katkaisee ja vääryyttä saa juoda, joka sanan lähettää tyhmän mukana.
\par 7 Velttoina riippuvat halvatun sääret; samoin sananlasku tyhmäin suussa.
\par 8 Yhtä kuin sitoisi kiven linkoon kiinni, on antaa kunniaa tyhmälle.
\par 9 Kuin ohdake, joka on osunut juopuneen käteen, on sananlasku tyhmäin suussa.
\par 10 Kuin jousimies, joka kaikkia haavoittaa, on se, joka tyhmän pestaa, kulkureita pestaa.
\par 11 Kuin koira, joka palajaa oksennuksilleen, on tyhmä, joka yhä uusii hulluuksiansa.
\par 12 Näet miehen, viisaan omissa silmissään - enemmän on toivoa tyhmästä kuin hänestä.
\par 13 Laiska sanoo: "Tuolla tiellä on leijona, jalopeura torien vaiheilla".
\par 14 Ovi saranoillaan kääntyilee, laiska vuoteellansa.
\par 15 Laiska pistää kätensä vatiin; ei viitsi sitä viedä suuhunsa jälleen.
\par 16 Laiska on omissa silmissään viisaampi kuin seitsemän, jotka vastaavat taitavasti.
\par 17 Kulkukoiraa korviin tarttuu se, joka syrjäisten riidasta suuttuu.
\par 18 Kuin mieletön, joka ammuskelee tulisia surmannuolia,
\par 19 on mies, joka pettää lähimmäisensä ja sanoo: "Leikillähän minä sen tein".
\par 20 Halkojen loppuessa sammuu tuli, ja panettelijan poistuessa taukoaa tora.
\par 21 Hehkuksi hiilet, tuleksi halot, riidan lietsomiseksi toraisa mies.
\par 22 Panettelijan puheet ovat kuin herkkupalat ja painuvat sisusten kammioihin asti.
\par 23 Kuin hopeasilaus saviastian pinnalla ovat hehkuvat huulet ja paha sydän.
\par 24 Vihamies teeskentelee huulillaan, mutta hautoo petosta sydämessänsä.
\par 25 Jos hän muuttaa suloiseksi äänensä, älä häntä usko, sillä seitsemän kauhistusta hänellä on sydämessä.
\par 26 Vihamielisyys kätkeytyy kavalasti, mutta seurakunnan kokouksessa sen pahuus paljastuu.
\par 27 Joka kuopan kaivaa, se itse siihen lankeaa; ja joka kiveä vierittää, sen päälle se takaisin vyörähtää.
\par 28 Valheellinen kieli vihaa omia ruhjomiansa, ja liukas suu saa turmiota aikaan.

\chapter{27}

\par 1 Älä huomispäivästä kersku, sillä et tiedä, mitä mikin päivä synnyttää.
\par 2 Kehukoon sinua toinen, ei oma suusi; vieras, eikä omat huulesi.
\par 3 Raskas on kivi ja painava hiekka, mutta molempia raskaampi hullun suuttumus.
\par 4 Kiukku on julma, viha on niinkuin tulva; mutta kuka voi kestää luulevaisuutta?
\par 5 Parempi julkinen nuhde kuin salattu rakkaus.
\par 6 Ystävän lyönnit ovat luotettavat, mutta vihamiehen suutelot ylenpalttiset.
\par 7 Kylläinen polkee hunajaakin, nälkäiselle on kaikki karvaskin makeata.
\par 8 Kuin pesästään paennut lintu, on mies paossa kotipaikoiltaan.
\par 9 Öljy ja suitsuke ilahuttavat sydämen; samoin ystävän hellyys, alttiisti neuvoja antavainen.
\par 10 Ystävääsi ja isäsi ystävää älä hylkää, äläkä hätäpäivänäsi mene veljesi taloon: parempi läheinen naapuri kuin kaukainen veli.
\par 11 Viisastu, poikani, ja ilahuta minun sydämeni, niin minä voin antaa herjaajalleni vastauksen.
\par 12 Mielevä näkee vaaran ja kätkeytyy, mutta yksinkertaiset käyvät kohti ja saavat vahingon.
\par 13 Ota siltä vaatteet, joka toista takasi, ja ota häneltä pantti vieraan naisen tähden.
\par 14 Joka siunaa ystäväänsä isoäänisesti aamulla varhain, sille se luetaan kiroukseksi.
\par 15 Räystäästä tippuva vesi sadepäivänä ja toraisa vaimo ovat yhdenveroiset.
\par 16 Joka tahtoo hänet salassa pitää, se tuulta salassa pitää, se tavoittaa öljyä oikeaan käteensä.
\par 17 Rauta rautaa hioo, ja ihminen toistansa hioo.
\par 18 Joka viikunapuuta hoitaa, saa syödä sen hedelmää; ja joka isännästänsä vaarin pitää, se tulee kunniaan.
\par 19 Niinkuin kasvot kuvastuvat vedessä, niin ihmisen sydän toisessa ihmisessä.
\par 20 Tuonela ja horna eivät kylläänsä saa; eivät myös saa kylläänsä ihmisen silmät.
\par 21 Hopealle sulatin, kullalle uuni; mies maineensa mukainen.
\par 22 Survo hullua huhmaressa, petkelellä surveitten seassa: ei erkane hänestä hänen hulluutensa.
\par 23 Tiedä tarkoin, miltä pikkukarjasi näyttää; pidä huoli laumoista.
\par 24 Sillä eivät aarteet säily iäti; ja pysyykö kruunukaan polvesta polveen?
\par 25 Kun heinä on mennyt ja tuore äpäre tulee näkyviin ja ruoho on koottu vuorilta,
\par 26 on sinulla karitsoita puvuksesi ja vuohipukkeja pellon ostohinnaksi
\par 27 ja vuohenmaitoa kyllin ravinnoksesi, perheesi ravinnoksi ja palvelijatartesi elatukseksi.

\chapter{28}

\par 1 Jumalattomat pakenevat, vaikka ei kenkään aja takaa, mutta vanhurskaat ovat turvassa kuin nuori jalopeura.
\par 2 Rikkomisensa tähden maa saa ruhtinaita paljon, mutta yhden ymmärtäväisen miehen taidolla järjestys kauan pysyy.
\par 3 Köyhä mies, joka vaivaisia sortaa, on kuin sade, joka lyö lakoon eikä anna leipää.
\par 4 Lain hylkijät kehuvat jumalattomia, mutta lain noudattajat kauhistuvat heitä.
\par 5 Pahat ihmiset eivät ymmärrä, mikä oikein on, mutta Herraa etsiväiset ymmärtävät kaiken.
\par 6 Parempi on köyhä, joka nuhteettomasti vaeltaa, kuin kahdella tiellä mutkitteleva rikas.
\par 7 Joka laista ottaa vaarin, on ymmärtäväinen poika; mutta irstailijain seuratoveri saattaa isänsä häpeään.
\par 8 Joka kartuttaa varojaan korolla ja voitolla, kokoaa niitä sille, joka vaivaisia armahtaa.
\par 9 Joka korvansa kääntää kuulemasta lakia, sen rukouskin on kauhistus.
\par 10 Joka eksyttää oikeamielisiä pahalle tielle, se lankeaa omaan kuoppaansa; mutta nuhteettomat perivät onnen.
\par 11 Rikas mies on omissa silmissään viisas, mutta ymmärtäväinen köyhä ottaa hänestä selvän.
\par 12 Kun vanhurskaat riemuitsevat, on ihanuus suuri; mutta kun jumalattomat nousevat, saa ihmisiä hakea.
\par 13 Joka rikkomuksensa salaa, se ei menesty; mutta joka ne tunnustaa ja hylkää, se saa armon.
\par 14 Onnellinen se ihminen, joka aina on aralla tunnolla; mutta joka sydämensä paaduttaa, se onnettomuuteen lankeaa.
\par 15 Kuin muriseva leijona ja ahnas karhu on kurjan kansan jumalaton hallitsija.
\par 16 Vähätaitoinen ruhtinas runsaasti kiskoo, mutta väärän voiton vihaaja saa elää kauan.
\par 17 Ihmisen, jota verivelka painaa, on pakoiltava hamaan hautaan asti; älköön häntä suojeltako.
\par 18 Nuhteettomasti vaeltavainen saa avun, mutta kahdella tiellä mutkittelija kerralla kaatuu.
\par 19 Joka peltonsa viljelee, saa leipää kyllin, mutta tyhjän tavoittelija saa köyhyyttä kyllin.
\par 20 Luotettava mies saa runsaan siunauksen, mutta jolla on kiihko rikastua, se ei rankaisematta jää.
\par 21 Ei ole hyvä henkilöön katsoa, mutta leipäpalankin tähden rikkomus tehdään.
\par 22 Pahansuova haluaa kiihkeästi varallisuutta eikä tiedä, että hänet tapaa puute.
\par 23 Joka toista nuhtelee, niinkuin minä neuvon, saa suosiota enemmän kuin se, joka kielellänsä liehakoitsee.
\par 24 Joka isältään ja äidiltään riistää ja sanoo: "Ei tämä ole rikos", se on tuhontekijän toveri.
\par 25 Tavaranahne nostaa riidan, mutta Herraan luottavainen tulee ravituksi.
\par 26 Omaan sydämeensä luottavainen on tyhmä, mutta viisaudessa vaeltava pelastuu.
\par 27 Joka köyhälle antaa, se ei puutteeseen joudu; mutta joka silmänsä häneltä sulkee, saa kirouksia paljon.
\par 28 Kun jumalattomat nousevat, piileksivät ihmiset; mutta kun he hukkuvat, niin hurskaat enentyvät.

\chapter{29}

\par 1 Kuritusta saanut mies, joka niskurina pysyy, rusennetaan äkisti, eikä apua ole.
\par 2 Hurskaitten enentyessä kansa iloitsee, mutta jumalattoman hallitessa kansa huokaa.
\par 3 Viisautta rakastavainen on isällensä iloksi, mutta porttojen seuratoveri hävittää varansa.
\par 4 Oikeudella kuningas pitää maan pystyssä, mutta verojen kiskoja sen hävittää.
\par 5 Mies, joka lähimmäistään liehakoitsee, virittää verkon hänen askeleilleen.
\par 6 Pahalle miehelle on oma rikos paulaksi, mutta vanhurskas saa riemuita ja iloita.
\par 7 Vanhurskas tuntee vaivaisten asian, mutta jumalaton ei siitä mitään ymmärrä.
\par 8 Pilkkaajat kaupungin villitsevät, mutta viisaat hillitsevät vihan.
\par 9 Viisas mies kun käräjöi hullun miehen kanssa, niin tämä reutoo ja nauraa eikä asetu.
\par 10 Murhamiehet vihaavat nuhteetonta, oikeamielisten henkeä he väijyvät.
\par 11 Tyhmä purkaa kaiken sisunsa, mutta viisas sen viimein tyynnyttää.
\par 12 Hallitsija, joka kuuntelee valhepuheita, saa palvelijoikseen pelkkiä jumalattomia.
\par 13 Köyhä ja sortaja kohtaavat toisensa; kumpaisenkin silmille Herra antaa valon.
\par 14 Kuninkaalla, joka tuomitsee vaivaisia oikein, on valtaistuin iäti vahva.
\par 15 Vitsa ja nuhde antavat viisautta, mutta kuriton poika on äitinsä häpeä.
\par 16 Kun jumalattomat lisääntyvät, lisääntyy rikos, mutta vanhurskaat saavat nähdä, kuinka he kukistuvat.
\par 17 Kurita poikaasi, niin hän sinua virvoittaa ja sielullesi herkkuja tarjoaa.
\par 18 Missä ilmoitus puuttuu, siinä kansa käy kurittomaksi; autuas se, joka noudattaa lakia.
\par 19 Ei ota palvelija sanoista ojentuakseen: hän kyllä ymmärtää, mutta ei tottele.
\par 20 Näet miehen, kärkkään puhumaan - enemmän on toivoa tyhmästä kuin hänestä.
\par 21 Jos palvelijaansa nuoresta pitäen hemmottelee, tulee hänestä lopulta kiittämätön.
\par 22 Pikavihainen mies nostaa riidan, ja kiukkuinen tulee rikkoneeksi paljon.
\par 23 Ihmisen alentaa hänen oma ylpeytensä, mutta alavamielinen saa kunnian.
\par 24 Joka käy osille varkaan kanssa, se sieluansa vihaa; hän kuulee vannotuksen, mutta ei ilmaise mitään.
\par 25 Ihmispelko panee paulan, mutta Herraan luottavainen on turvattu.
\par 26 Hallitsijan suosiota etsivät monet, mutta Herralta tulee miehelle oikeus.
\par 27 Vääryyden mies on vanhurskaille kauhistus, ja oikean tien kulkija on kauhistus jumalattomalle.

\chapter{30}

\par 1 Aagurin, Jaaken pojan, sanat; lauselma. Näin puhuu se mies: Minä olen väsyttänyt itseni, Jumala; olen väsyttänyt itseni, Jumala, ja menehdyn.
\par 2 Sillä järjetön olen minä mieheksi, ei ole minulla ihmisymmärrystä;
\par 3 enkä ole oppinut viisautta, tullakseni tuntemaan Pyhintä.
\par 4 Kuka on noussut taivaaseen ja astunut sieltä alas? Kuka on koonnut kouriinsa tuulen? Kuka on sitonut vedet vaipan sisään? Kuka on kohdalleen asettanut maan ääret kaikki? Mikä on hänen nimensä ja mikä hänen poikansa nimi, jos sen tiedät?
\par 5 Jokainen Jumalan sana on taattu; hän on niiden kilpi, jotka häneen turvaavat.
\par 6 Älä lisää hänen sanoihinsa mitään, ettei hän vaatisi sinua tilille ja ettet valhettelijaksi joutuisi.
\par 7 Kahta minä sinulta pyydän, älä niitä minulta koskaan kiellä, kuolemaani saakka:
\par 8 Vilppi ja valhepuhe pidä minusta kaukana. Älä köyhyyttä, älä rikkautta minulle anna; anna minulle ravinnoksi määräosani leipää,
\par 9 etten kylläisenä tulisi kieltäjäksi ja sanoisi: "Kuka on Herra?" ja etten köyhtyneenä varastaisi ja rikkoisi Jumalani nimeä vastaan.
\par 10 Älä kieli palvelijasta hänen herrallensa; muutoin hän sinut kiroaa, ja sinä saat siitä kärsiä.
\par 11 Voi sukua, joka isäänsä kiroaa eikä siunaa äitiänsä;
\par 12 sukua, joka on omissa silmissään puhdas, vaikka ei ole pesty liastansa!
\par 13 Voi sukua - kuinka ylpeät ovatkaan sen silmät ja kuinka korskea silmänluonti -
\par 14 sukua, jonka hampaat ovat miekkoja ja leukaluut veitsiä syödäksensä kurjat maasta pois ja köyhät ihmisten joukosta!
\par 15 Verenimijällä on kaksi tytärtä: Anna vielä! Anna vielä! Kolme on, jotka eivät kylläänsä saa, neljä, jotka eivät sano: "Jo riittää":
\par 16 tuonela, hedelmätön kohtu, maa, joka ei saa kylläänsä vedestä, ja tuli, joka ei sano: "Jo riittää".
\par 17 Joka isäänsä pilkkaa ja pitää halpana totella äitiänsä, häneltä korpit puron luona hakkaavat silmän, ja kotkan poikaset syövät sen.
\par 18 Kolme on minusta ylen ihmeellistä, ja neljä on, joita en käsitä:
\par 19 kotkan jäljet taivaalla, käärmeen jäljet kalliolla, laivan jäljet keskellä merta ja miehen jäljet nuoren naisen tykönä.
\par 20 Samoin ovat avionrikkoja-vaimon jäljet: hän syö, pyyhkii suunsa ja sanoo: "En ole pahaa tehnyt".
\par 21 Kolmen alla järkkyy maa, ja neljän alla ei se jaksa kestää:
\par 22 orjan alla, kun hän kuninkaaksi pääsee, houkan, kun hän saa kyllälti leipää,
\par 23 hyljityn alla, kun hän miehen saa, ja palvelijattaren, kun hän emäntänsä syrjäyttää.
\par 24 Neljä on maassa vähäisintä, mutta viisaan viisasta silti:
\par 25 Muurahaiset ovat voimaton kansa, mutta he hankkivat leipänsä kesällä;
\par 26 tamaanit ovat heikko kansa, mutta he laittavat majansa kallioihin;
\par 27 heinäsirkoilla ei ole kuningasta, mutta koko lauma lähtee järjestyksessä liikkeelle;
\par 28 sisiliskoon voi tarttua käsin, mutta kuitenkin se oleskelee kuninkaan linnoissa.
\par 29 Kolmella on komea astunta, ja neljä komeasti kulkee:
\par 30 leijona, sankari eläinten joukossa, joka ei vääjää ketään,
\par 31 hevonen, solakkakylki, tai kauris, ja kuningas joukkonsa johdossa.
\par 32 Jos ylpeilit - olipa se houkkamaisuutta tai harkittua - niin laske käsi suullesi.
\par 33 Sillä maitoa pusertamalla saa voin, nenää pusertamalla saa veren, ja vihoja pusertamalla saa riidan.

\chapter{31}

\par 1 Lemuelin, Massan kuninkaan, sanat, joilla hänen äitinsä kasvatti häntä.
\par 2 Mitä, poikani; mitä, kohtuni poika; mitä, lupausteni poika?
\par 3 Älä anna voimaasi naisille, vaellustasi kuningasten turmelijatarten valtaan.
\par 4 Ei sovi kuningasten, Lemuel, ei sovi kuningasten viiniä juoda eikä ruhtinasten kysellä: "Missä väkijuomaa?"
\par 5 Muutoin hän juodessaan unhottaa, mitä saädetty on, ja vääntelee kaikkien kurjuuden lasten oikeuden.
\par 6 Antakaa väkevää juomaa menehtyvälle ja viiniä murhemielisille.
\par 7 Sellainen juokoon ja unhottakoon köyhyytensä älköönkä enää vaivaansa muistelko.
\par 8 Avaa suusi mykän hyväksi, oikeuden hankkimiseksi kaikille sortuville.
\par 9 Avaa suusi, tuomitse oikein, hanki kurjalle ja köyhälle oikeus.
\par 10 Kelpo vaimon kuka löytää? Sellaisen arvo on helmiä paljon kalliimpi.
\par 11 Hänen miehensä sydän häneen luottaa, eikä siltä mieheltä riistaa puutu.
\par 12 Hän tekee miehellensä hyvää, ei pahaa, kaikkina elinpäivinänsä.
\par 13 Hän puuhaa villat ja pellavat ja halullisin käsin askartelee.
\par 14 Hän on kauppiaan laivojen kaltainen: leipänsä hän noutaa kaukaa.
\par 15 Kun yö vielä on, hän nousee ja antaa ravinnon perheellensä, piioilleen heidän osansa.
\par 16 Hän haluaa peltoa ja hankkii sen, istuttaa viinitarhan kättensä hedelmällä.
\par 17 Hän voimalla vyöttää kupeensa ja käsivartensa vahvistaa.
\par 18 Hankkeensa hän huomaa käyvän hyvin, ei sammu hänen lamppunsa yöllä.
\par 19 Hän ojentaa kätensä kehrävarteen ja käyttelee värttinää kämmenissään.
\par 20 Hän avaa kätensä kurjalle, ojentaa köyhälle molemmat kätensä.
\par 21 Ei pelkää hän perheensä puolesta lunta, sillä koko hänen perheensä on puettu purppuravillaan.
\par 22 Hän valmistaa itsellensä peitteitä; hienoa pellavaa ja punapurppuraa on hänen pukunsa.
\par 23 Hänen miehensä on tunnettu porteissa, maanvanhinten seassa istuessansa.
\par 24 Hän aivinapaitoja tekee ja myy, vöitä hän kauppiaalle toimittaa.
\par 25 Vallalla ja kunnialla hän on vaatetettu, ja hän nauraa tulevalle päivälle.
\par 26 Suunsa hän avaa viisauden sanoihin, hänen kielellään on lempeä opetus.
\par 27 Hän tarkkaa talonsa menoa, eikä hän laiskan leipää syö.
\par 28 Hänen poikansa nousevat ja kiittävät hänen onneansa; hänen miehensä nousee ja ylistää häntä:
\par 29 "Paljon on naisia, toimellisia menoissaan, mutta yli niitten kaikkien kohoat sinä".
\par 30 Pettävä on sulous, kauneus katoavainen; ylistetty se vaimo, joka Herraa pelkää!
\par 31 Suokaa hänen nauttia kättensä hedelmiä, hänen tekonsa häntä porteissa ylistäkööt.


\end{document}