\begin{document}

\title{Jesajan kirja}


\chapter{1}

\par 1 Jesajan, Aamoksen pojan, näky, jonka hän näki Juudasta ja Jerusalemista Ussian, Jootamin, Aahaan ja Hiskian, Juudan kuningasten, päivinä.
\par 2 Kuulkaa, taivaat, ota korviisi, maa, sillä Herra puhuu: Minä kasvatin lapsia, sain heidät suuriksi, mutta he luopuivat minusta.
\par 3 Härkä tuntee omistajansa ja aasi isäntänsä seimen; mutta Israel ei tunne, minun kansani ei ymmärrä.
\par 4 Voi syntistä sukua, raskaasti rikkonutta kansaa, pahantekijäin siementä, kelvottomia lapsia! He ovat hyljänneet Herran, pitäneet Israelin Pyhää pilkkanansa, he ovat kääntyneet pois.
\par 5 Mihin pitäisi teitä vielä lyödä, kun yhä jatkatte luopumustanne? Koko pää on kipeä, koko sydän sairas.
\par 6 Kantapäästä kiireeseen asti ei ole tervettä paikkaa, ainoastaan haavoja, mustelmia ja vereksiä lyömiä, joita ei ole puserrettu, ei sidottu eikä öljyllä pehmitetty.
\par 7 Teidän maanne on autiona, teidän kaupunkinne ovat tulella poltetut; teidän peltojanne syövät muukalaiset silmäinne edessä; ne ovat autioina niinkuin ainakin muukalaisten hävittämät.
\par 8 Jäljellä on tytär Siion yksinänsä niinkuin maja viinitarhassa, niinkuin lehvämaja kurkkumaassa, niinkuin saarrettu kaupunki.
\par 9 Ellei Herra Sebaot olisi jättänyt meistä jäännöstä, olisi meidän käynyt pian kuin Sodoman, olisimme tulleet Gomorran kaltaisiksi.
\par 10 Kuulkaa Herran sana, te Sodoman päämiehet, ota korviisi meidän Jumalamme opetus, sinä Gomorran kansa.
\par 11 Mitä ovat minulle teidän paljot teurasuhrinne? sanoo Herra. Minä olen kyllästynyt oinas-polttouhreihin ja juottovasikkain rasvaan. Mullikkain, karitsain ja kauristen vereen minä en mielisty.
\par 12 Kun te tulette minun kasvojeni eteen, kuka sitä teiltä vaatii - minun esikartanoitteni tallaamista?
\par 13 Älkää enää tuoko minulle turhaa ruokauhria; suitsutus on minulle kauhistus. En kärsi uuttakuuta enkä sapattia, en kokouksen kuuluttamista, en vääryyttä ynnä juhlakokousta.
\par 14 Minun sieluni vihaa teidän uusiakuitanne ja juhla-aikojanne; ne ovat käyneet minulle kuormaksi, jota kantamaan olen väsynyt.
\par 15 Kun te ojennatte käsiänne, minä peitän silmäni teiltä; vaikka kuinka paljon rukoilisitte, minä en kuule: teidän kätenne ovat verta täynnä.
\par 16 Peseytykää, puhdistautukaa; pankaa pois pahat tekonne minun silmäini edestä, lakatkaa pahaa tekemästä.
\par 17 Oppikaa tekemään hyvää; harrastakaa oikeutta, ojentakaa väkivaltaista, hankkikaa orvolle oikeus, ajakaa lesken asiaa.
\par 18 Niin tulkaa, käykäämme oikeutta keskenämme, sanoo Herra. Vaikka teidän syntinne ovat veriruskeat, tulevat ne lumivalkeiksi; vaikka ne ovat purppuranpunaiset, tulevat ne villanvalkoisiksi.
\par 19 Jos suostutte ja olette kuuliaiset, niin te saatte syödä maan hyvyyttä;
\par 20 mutta jos vastustatte ja niskoittelette, niin miekka syö teidät. Sillä Herran suu on puhunut.
\par 21 Voi, kuinka onkaan portoksi tullut uskollinen kaupunki! Se oli täynnä oikeutta, siellä asui vanhurskaus, mutta nyt murhamiehet.
\par 22 Sinun hopeasi on kuonaksi käynyt, jaloviinisi vedellä laimennettu.
\par 23 Sinun päämiehesi ovat niskureita ja varkaiden tovereita; kaikki he lahjuksia rakastavat ja palkkoja tavoittelevat; eivät he hanki orvolle oikeutta, lesken asia ei pääse heidän eteensä.
\par 24 Sentähden sanoo Herra, Herra Sebaot, Israelin Väkevä: Voi! Minä viihdytän vihani vastustajissani, minä kostan vihollisilleni;
\par 25 minä käännän käteni sinua vastaan ja puhdistan sinusta kuonan niinkuin lipeällä ja poistan sinusta kaiken lyijyn.
\par 26 Ja minä palautan sinun tuomarisi muinaiselleen ja sinun neuvonantajasi sellaisiksi, kuin alkuaan olivat. Sitten sinua kutsutaan "vanhurskauden linnaksi", "uskolliseksi kaupungiksi".
\par 27 Siion lunastetaan oikeudella ja sen kääntyneet vanhurskaudella.
\par 28 Mutta luopiot ja syntiset saavat kaikki turmion, ja ne, jotka hylkäävät Herran, hukkuvat.
\par 29 Sillä he saavat häpeän niistä tammista, joita te himoitsitte, ja pettymyksen puutarhoista, jotka te valitsitte.
\par 30 Sillä teidän käy kuin tammen, jonka lehdet lakastuvat, ja kuin puutarhan, joka on vailla vettä.
\par 31 Ja mahtaja tulee rohtimeksi ja hänen tekonsa kipinäksi, ja molemmat palavat yhdessä, eikä ole sammuttajaa.

\chapter{2}

\par 1 Sana, jonka Jesaja, Aamoksen poika, näki Juudasta ja Jerusalemista.
\par 2 Aikojen lopussa on Herran temppelin vuori seisova vahvana, ylimmäisenä vuorista, kukkuloista korkeimpana, ja kaikki pakanakansat virtaavat sinne.
\par 3 Monet kansat lähtevät liikkeelle sanoen: "Tulkaa, nouskaamme Herran vuorelle, Jaakobin Jumalan temppeliin, että hän opettaisi meille teitänsä ja me vaeltaisimme hänen polkujansa; sillä Siionista lähtee laki, Jerusalemista Herran sana".
\par 4 Ja hän tuomitsee pakanakansojen kesken, säätää oikeutta monille kansoille. Niin he takovat miekkansa vantaiksi ja keihäänsä vesureiksi; kansa ei nosta miekkaa kansaa vastaan, eivätkä he enää opettele sotimaan.
\par 5 Jaakobin heimo, tulkaa, vaeltakaamme Herran valkeudessa.
\par 6 Sillä sinä olet hyljännyt kansasi, Jaakobin heimon, koska he ovat täynnä Idän menoa, täynnä ennustelijoita niinkuin filistealaiset, ja lyövät kättä muukalaisten kanssa.
\par 7 Heidän maansa tuli täyteen hopeata ja kultaa, eikä heidän aarteillaan ole määrää; heidän maansa tuli täyteen hevosia, eikä heidän vaunuillaan ole määrää.
\par 8 Heidän maansa tuli täyteen epäjumalia, he kumartavat kättensä tekoa, sitä, minkä heidän sormensa ovat tehneet.
\par 9 Silloin ihminen masentuu ja mies painuu maahan, mutta älä anna heille anteeksi.
\par 10 Mene kallion kätköön, piiloudu maan peittoon Herran kauhua ja hänen valtansa kirkkautta pakoon.
\par 11 Ihmisten ylpeät silmät painuvat maahan, miesten korskeus masentuu, ja sinä päivänä Herra yksinänsä on korkea.
\par 12 Sillä Herran Sebaotin päivä kohtaa kaikkea ylpeää ja korskeata ja kaikkea ylhäistä, niin että se maahan painuu,
\par 13 kaikkia Libanonin setripuita, noita korkeita ja ylhäisiä, kaikkia Baasanin tammia,
\par 14 kaikkia korkeita vuoria ja kaikkia ylhäisiä kukkuloita,
\par 15 kaikkia korkeita torneja ja kaikkia vahvoja muureja,
\par 16 kaikkia Tarsiin-laivoja ja kaikkea kallista ja ihanaa.
\par 17 Silloin masentuu ihmisten ylpeys, ja miesten korskeus painuu maahan, ja sinä päivänä Herra yksinänsä on korkea.
\par 18 Mutta epäjumalat katoavat kaikki tyynni.
\par 19 Silloin mennään kallioluoliin ja maakuoppiin Herran kauhua ja hänen valtansa kirkkautta pakoon, kun hän nousee maata kauhistuttamaan.
\par 20 Sinä päivänä ihmiset viskaavat pois myyrille ja yököille hopea- ja kultajumalansa, jotka he ovat tehneet kumarrettaviksensa.
\par 21 Ja he menevät vuorenrotkoihin ja kallionkoloihin Herran kauhua ja hänen valtansa kirkkautta pakoon, kun hän nousee maata kauhistuttamaan.
\par 22 Luopukaa siis ihmisistä, joilla on vain henkäys sieraimissa, sillä minkä arvoiset he ovat!

\chapter{3}

\par 1 Sillä katso, Herra, Herra Sebaot ottaa pois Jerusalemilta ja Juudalta varan ja turvan, kaiken leivänvaran ja kaiken vedenvaran,
\par 2 sankarin ja soturin, tuomarin ja profeetan, tietäjän ja vanhimman,
\par 3 viidenkymmenenpäämiehen ja arvomiehen, neuvonantajan, taitoniekan ja taikurin.
\par 4 Ja minä panen nuorukaiset heille päämiehiksi, ja pahankuriset hallitsevat heitä.
\par 5 Kansassa sortaa kukin toistaan, jokainen lähimmäistään; nuorukainen rehentelee vanhusta vastaan, halpa-arvoinen arvollista vastaan.
\par 6 Kun veli tarttuu veljeensä isänsä talossa sanoen: "Sinulla on vielä vaatteet, rupea ruhtinaaksemme ja ota huostaasi tämä raunioläjä",
\par 7 niin tämä sinä päivänä korottaa äänensä sanoen: "Ei minusta ole haavoja sitomaan; ei ole minun talossani leipää, ei vaatetta, älkää panko minua kansan ruhtinaaksi".
\par 8 Sillä Jerusalem kaatuu ja Juuda kukistuu, koska heidän kielensä ja heidän tekonsa ovat Herraa vastaan ja he uhittelevat hänen kunniansa kasvoja.
\par 9 Heidän kasvojensa hahmo todistaa heitä vastaan; syntinsä he tuovat julki niinkuin sodomalaiset, he eivät niitä salaa. Voi heitä! Itsellensä he pahaa tekevät.
\par 10 Sanokaa hurskaasta, että hänen käy hyvin, sillä he saavat nauttia tekojensa hedelmiä.
\par 11 Voi jumalatonta! Hänen käy pahoin, sillä hänen kättensä teot maksetaan hänelle.
\par 12 Minun kansani käskijät ovat lapsia, ja naiset sitä hallitsevat. Kansani, sinun johtajasi ovat eksyttäjiä, he ovat hämmentäneet sinun polkujesi suunnan.
\par 13 Herra asettuu käymään oikeutta, nousee kansoja tuomitsemaan.
\par 14 Herra käy tuomiolle kansansa vanhinten ja sen päämiesten kanssa: Te olette raiskanneet viinitarhan; teidän taloissanne on kurjilta ryöstettyä tavaraa.
\par 15 Kuinka saatatte runnella minun kansaani ja ruhjoa kurjien kasvot? sanoo Herra, Herra Sebaot.
\par 16 Ja Herra sanoi: Koska Siionin tyttäret korskeilevat, kulkevat kaula kenossa ja silmillään vilkuillen, astua sipsuttelevat ja nilkkarenkaitaan kilistelevät,
\par 17 tekee Herra Siionin tytärten päälaen rupiseksi, ja Herra paljastaa heidän häpynsä.
\par 18 Sinä päivänä Herra poistaa koreat nilkkarenkaat, otsanauhat, puolikuukorut,
\par 19 korvarenkaat, rannerenkaat, hunnut,
\par 20 juhlapäähineet, jalkakäädyt, koruvyöt, hajupullot, taikahelyt,
\par 21 sormukset, nenärenkaat,
\par 22 juhlavaatteet, kaavut, vaipat, kukkarot,
\par 23 kuvastimet, aivinapaidat, käärelakit ja päällysharsot.
\par 24 Silloin tulee tuoksun sijaan löyhkä, vyön sijaan nuora, käherretyn tukan sijaan kaljuus, korupuvun sijaan säkkiverho, kauneuden sijaan polttomerkki.
\par 25 Sinun miehesi kaatuvat miekkaan ja sinun sankarisi sotaan.
\par 26 Ja Siionin portit valittavat ja vaikeroivat, ja typötyhjänä hän istuu maassa.

\chapter{4}

\par 1 Sinä päivänä seitsemän naista tarttuu yhteen mieheen sanoen: "Me syömme omaa leipäämme ja vaatetamme itsemme omilla vaatteilla, kunhan vain meidät otetaan sinun nimiisi; poista meidän häpeämme".
\par 2 Sinä päivänä on Herran vesa oleva Israelin pelastuneille kaunistus ja kunnia ja maan hedelmä heille korkeus ja loisto.
\par 3 Ja ne, jotka Siionissa ovat säilyneet ja Jerusalemissa jäljelle jääneet, kutsutaan pyhiksi, kaikki, jotka ovat kirjoitetut Jerusalemissa elävien lukuun -
\par 4 silloin kun Herra on pessyt pois Siionin tytärten saastan ja huuhtonut Jerusalemista sen verenviat tuomion ja puhdistuksen hengellä.
\par 5 Ja Herra luo Siionin vuorelle koko asuinsijansa ylle ja sen juhlakokousten ylle pilven päivän ajaksi ja savun ynnä tulenliekin hohteen yöksi, sillä kaiken kirkkauden yllä on oleva peite
\par 6 ja verho varjoamassa päivän helteeltä sekä turvaamassa ja suojaamassa rajuilmalta ja sateelta.

\chapter{5}

\par 1 Minä laulan ystävästäni, laulan rakkaani laulun hänen viinitarhastaan. Ystävälläni oli viinitarha lihavalla vuorenrinteellä.
\par 2 Hän kuokki ja kivesi sen ja istutti siihen jaloja köynnöksiä, rakensi sen keskelle tornin ja hakkasi sinne myös kuurnan. Ja hän odotti sen kasvavan rypäleitä, mutta se kasvoi villimarjoja.
\par 3 Ja nyt, te Jerusalemin asukkaat ja Juudan miehet, tuomitkaa minun ja minun viinitarhani välillä.
\par 4 Mitä olisi viinitarhalleni vielä ollut tehtävä, jota en olisi sille tehnyt? Miksi se kasvoi villimarjoja, kun minä odotin sen kasvavan rypäleitä?
\par 5 Mutta nyt minä ilmoitan teille, mitä teen viinitarhalleni: minä poistan siitä aidan, niin että se jää hävitettäväksi, särjen siitä muurin, niin että se jää tallattavaksi.
\par 6 Minä hävitän sen: ei sitä enää vesota eikä kuokita, vaan se on kasvava ohdaketta ja orjantappuraa, ja minä kiellän pilvet sille vettä satamasta.
\par 7 Sillä Israelin heimo on Herran Sebaotin viinitarha, ja Juudan miehet ovat hänen ilo-istutuksensa. Ja hän odotti oikeutta, mutta katso, tuli oikeuttomuus, ja vanhurskautta, mutta katso, tuli vaikerrus.
\par 8 Voi niitä, jotka liittävät talon taloon, yhdistävät pellon peltoon, kunnes ei jää enää tilaa ja te yksin asutte maassa!
\par 9 Minun korviini kuului Herran Sebaotin sana: Totisesti, ne monet talot tulevat autioiksi, ne suuret ja kauniit asujattomiksi.
\par 10 Sillä kymmenen auranalaa viinitarhaa on antava yhden bat-mitan, ja hoomerin kylvö on antava eefan.
\par 11 Voi niitä, jotka aamuvarhaisesta väkijuoman jäljessä juoksevat ja iltamyöhään viipyvät viinistä hehkuvina!
\par 12 Kanteleet, harput, vaskirummut, huilut ja viinit on heillä pidoissansa, mutta Herran tekoja he eivät tarkkaa, eivät näe hänen kättensä töitä.
\par 13 Sentähden minun kansani siirretään maastansa äkkiarvaamatta, sen ylhäiset kärsivät nälkää, ja sen remuava joukko nääntyy janoon.
\par 14 Sentähden tuonela levittää kitansa ammolleen, avaa suunsa suunnattomaksi; ja sinne menee sen loisto, sen remuava ja pauhaava joukko, kaikki sen ilonpitäjät.
\par 15 Silloin ihminen masentuu ja mies painuu maahan, maahan painuvat ylpeitten silmät;
\par 16 mutta Herra Sebaot on oleva korkea tuomiossa, pyhä Jumala on oleva pyhä vanhurskaudessa.
\par 17 Ja karitsat käyvät siellä niinkuin laitumillansa, ja vieraat syöttävät rikkaitten rauniosijoja.
\par 18 Voi niitä, jotka vetävät perässänsä rangaistusta turhuuden köysillä, synnin palkkaa kuin vaunujen valjailla;
\par 19 niitä, jotka sanovat: "Joutukoon pian hänen tekonsa, että me sen näemme; täyttyköön ja tapahtukoon Israelin Pyhän aivoitus, että me sen tiedämme!"
\par 20 Voi niitä, jotka sanovat pahan hyväksi ja hyvän pahaksi, jotka tekevät pimeyden valkeudeksi ja valkeuden pimeydeksi, jotka tekevät karvaan makeaksi ja makean karvaaksi!
\par 21 Voi niitä, jotka ovat viisaita omissa silmissään ja ymmärtäväisiä omasta mielestänsä!
\par 22 Voi niitä, jotka ovat urhoja viinin juonnissa ja aimomiehiä väkijuoman sekoittamisessa,
\par 23 jotka lahjuksesta julistavat syyllisen syyttömäksi ja ottavat oikeuden siltä, joka oikeassa on!
\par 24 Sentähden, niinkuin tulen kieli syö oljet ja kuloheinä raukeaa liekkiin, niin heidän juurensa mätänee, ja heidän kukkansa hajoaa tomuna ilmaan, sillä he ovat ylenkatsoneet Herran Sebaotin lain ja pitäneet pilkkanansa Israelin Pyhän sanaa.
\par 25 Sentähden syttyi Herran viha hänen kansaansa kohtaan: hän ojensi kätensä sitä vastaan ja löi sitä, niin että vuoret vapisivat ja heidän raatonsa olivat tunkiona keskellä katuja. Tästä kaikesta ei hänen vihansa asettunut, vaan hänen kätensä on vielä ojennettuna.
\par 26 Ja hän pystyttää viirin kaukaiselle kansalle ja viheltää sen luokseen maan äärestä. Ja katso, se tulee viipymättä, nopeasti.
\par 27 Ei kukaan heistä väsy eikä kompastu, ei kukaan torku eikä nuku, ei vyö kupeilta aukene, eikä kengänpaula katkea.
\par 28 Sen nuolet ovat teroitetut, sen jouset kaikki jännitetyt, sen orhien kaviot ovat kuin pii, ja sen rattaat kuin rajuilma.
\par 29 Sen kiljunta on kuin naarasleijonan, se kiljuu kuin nuoret jalopeurat, se ärjyy ja tempaa saaliin ja vie sen pois, eikä pelastajaa ole.
\par 30 Se ärjyy sille sinä päivänä, niinkuin meri ärjyy. Silloin tähyillään maata, mutta katso, on pimeys, on ahdistus, ja valo on pimennyt sen synkeyteen.

\chapter{6}

\par 1 Kuningas Ussian kuolinvuotena minä näin Herran istuvan korkealla ja ylhäisellä istuimella, ja hänen vaatteensa liepeet täyttivät temppelin.
\par 2 Serafit seisoivat hänen ympärillään; kullakin oli kuusi siipeä: kahdella he peittivät kasvonsa, kahdella he peittivät jalkansa, ja kahdella he lensivät.
\par 3 Ja he huusivat toinen toisellensa ja sanoivat: "Pyhä, pyhä, pyhä Herra Sebaot; kaikki maa on täynnä hänen kunniaansa".
\par 4 Ja kynnysten perustukset vapisivat heidän huutonsa äänestä, ja huone täyttyi savulla.
\par 5 Niin minä sanoin: "Voi minua! Minä hukun, sillä minulla on saastaiset huulet, ja minä asun kansan keskellä, jolla on saastaiset huulet; sillä minun silmäni ovat nähneet kuninkaan, Herran Sebaotin."
\par 6 Silloin lensi minun luokseni yksi serafeista, kädessään hehkuva kivi, jonka hän oli pihdeillä ottanut alttarilta,
\par 7 ja kosketti sillä minun suutani sanoen: "Katso, tämä on koskettanut sinun huuliasi; niin on sinun velkasi poistettu ja syntisi sovitettu".
\par 8 Ja minä kuulin Herran äänen sanovan: "Kenenkä minä lähetän? Kuka menee meidän puolestamme?" Minä sanoin: "Katso, tässä minä olen, lähetä minut".
\par 9 Niin hän sanoi: "Mene ja sano tälle kansalle: 'Kuulemalla kuulkaa, älkääkä ymmärtäkö, näkemällä nähkää, älkääkä käsittäkö'.
\par 10 Paaduta tämän kansan sydän, koveta sen korvat, sokaise sen silmät, ettei se näkisi silmillään, ei kuulisi korvillaan, ei ymmärtäisi sydämellään eikä kääntyisi ja parannetuksi tulisi."
\par 11 Mutta minä sanoin: "Kuinka kauaksi aikaa, Herra?" Hän vastasi: "Siihen asti, kunnes kaupungit tulevat autioiksi, asumattomiksi, ja talot tyhjiksi ihmisistä ja pellot on hävitetty erämaaksi;
\par 12 kunnes Herra on karkoittanut ihmiset kauas ja suuri autius tullut keskelle maata.
\par 13 Ja jos siellä on jäljellä kymmenes osa, niin hävitetään vielä sekin. Mutta niinkuin tammesta ja rautatammesta jää kaadettaessa kanto, niin siitäkin: se kanto on pyhä siemen."

\chapter{7}

\par 1 Aahaan, Jootamin pojan, Ussian pojanpojan, Juudan kuninkaan, aikana lähtivät Resin, Aramin kuningas, ja Pekah, Remaljan poika, Israelin kuningas, sotimaan Jerusalemia vastaan, mutta he eivät voineet valloittaa sitä.
\par 2 Ja kun Daavidin huoneelle ilmoitettiin sanoma: "Aram on leiriytynyt Efraimiin", niin kuninkaan ja hänen kansansa sydän vapisi, niinkuin metsän puut vapisevat tuulessa.
\par 3 Silloin Herra sanoi Jesajalle: "Mene Aahasta vastaan, sinä ja sinun poikasi Sear-Jaasub, Ylälammikon vesijohdon päähän, Vanuttajankedon tielle,
\par 4 ja sano hänelle: Ole varuillasi ja pysy rauhallisena, älä pelkää, älköönkä sydämesi säikkykö noita kahta savuavaa kekäleenpätkää: Resinin ja Aramin ynnä Remaljan pojan vihanvimmaa.
\par 5 Koska Aram, Efraim ja Remaljan poika ovat hankkineet pahaa sinua vastaan sanoen:
\par 6 'Lähtekäämme Juudaa vastaan, täyttäkäämme se kauhulla, vallatkaamme se itsellemme ja asettakaamme sinne kuninkaaksi Taabalin poika',
\par 7 niin sanoo Herra, Herra näin: Ei se onnistu, eikä se tapahdu.
\par 8 Sillä Aramin pää on Damasko, ja Damaskon pää on Resin - mutta kuusikymmentä viisi vuotta vielä, ja Efraim on muserrettu eikä ole enää kansa - ja Efraimin pää on Samaria,
\par 9 ja Samarian pää on Remaljan poika. Ellette usko, niin ette kestä."
\par 10 Ja Herra puhui jälleen Aahaalle sanoen:
\par 11 "Pyydä itsellesi merkki Herralta, Jumalaltasi, pyydä alhaalta syvyydestä taikka ylhäältä korkeudesta".
\par 12 Mutta Aahas vastasi: "En pyydä enkä kiusaa Herraa".
\par 13 Siihen hän lausui: "Kuulkaa, te Daavidin suku, eikö se riitä, että te ihmiset väsytätte, kun vielä minun Jumalanikin itseenne väsytätte?
\par 14 Sentähden Herra itse antaa teille merkin: Katso, neitsyt tulee raskaaksi ja synnyttää pojan ja antaa hänelle nimen Immanuel.
\par 15 Voita ja hunajaa hän syö siinä iässä, jolloin hän oppii hylkäämään pahan ja valitsemaan hyvän;
\par 16 sillä ennenkuin poika on oppinut hylkäämään pahan ja valitsemaan hyvän, tulee autioksi maa, jonka kahta kuningasta sinä kauhistut.
\par 17 Herra antaa tulla sinulle, sinun kansallesi ja isäsi suvulle ne päivät, joita ei ole ollut siitä asti, kun Efraim luopui Juudasta - antaa tulla Assurin kuninkaan.
\par 18 Ja sinä päivänä Herra viheltää kärpäset Egyptin virtain suulta ja mehiläiset Assurin maasta,
\par 19 ja ne tulevat ja laskeutuvat kaikki vuorenrotkoihin ja kallionkoloihin, kaikkiin orjantappurapensaisiin ja kaikille juottopaikoille.
\par 20 Sinä päivänä Herra ajattaa virran tuolta puolen palkatulla partaveitsellä, Assurin kuninkaalla, pään ja häpykarvat; viepä se vielä parrankin.
\par 21 Siihen aikaan pitää mies lehmäsen ja pari lammasta,
\par 22 mutta runsaan maidon tulon tähden hän syö voita; sillä voita ja hunajaa syövät kaikki maahan jäljelle jääneet.
\par 23 Ja siihen aikaan jokainen paikka, jossa kasvaa tuhat viinipuuta, arvoltaan tuhat hopeasekeliä, joutuu orjantappurain ja ohdakkeiden valtaan.
\par 24 Sinne on mentävä nuolet ja jousi mukana, sillä koko maa on pelkkää orjantappuraa ja ohdaketta.
\par 25 Ei yhdellekään niistä vuorista, joita nyt kuokalla kuokitaan, enää mennä orjantappurain ja ohdakkeiden pelosta; ne jäävät härkien laitumeksi ja lammasten tallattavaksi."

\chapter{8}

\par 1 Ja Herra sanoi minulle: "Ota iso taulu ja kirjoita siihen selkeällä kirjoituksella: Maher-Saalal Haas-Bas.
\par 2 Ja minä otan itselleni luotettavat todistajat, pappi Uurian ja Sakarjan, Jeberekjan pojan."
\par 3 Sitten minä lähestyin profeetta-vaimoani, ja hän tuli raskaaksi ja synnytti pojan. Niin Herra sanoi minulle: "Pane hänelle nimeksi Maher-Saalal Haas-Bas.
\par 4 Sillä ennenkuin poika oppii sanomaan: 'isä' ja 'äiti', viedään Damaskoon tavarat ja Samarian saalis Assurin kuninkaan eteen."
\par 5 Ja Herra puhui jälleen minulle ja sanoi:
\par 6 "Koska tämä kansa halveksii Siiloan hiljaa virtaavia vesiä ja iloitsee Resinin ja Remaljan pojan kanssa,
\par 7 niin katso, sentähden Herra antaa tulla heidän ylitsensä virran vedet, valtavat ja suuret: Assurin kuninkaan ja kaiken hänen kunniansa; se nousee kaikkien uomiensa yli ja menee kaikkien äyräittensä yli,
\par 8 ja se tunkeutuu Juudaan, paisuu ja tulvii ja ulottuu kaulaan saakka. Ja levittäen siipensä se täyttää yliyltään sinun maasi, Immanuel."
\par 9 Riehukaa, kansat, kuitenkin kukistutte; kuulkaa, kaikki kaukaiset maat: sonnustautukaa, kuitenkin kukistutte; sonnustautukaa, kuitenkin kukistutte.
\par 10 Pitäkää neuvoa: se raukeaa; sopikaa sopimus: ei se pysy. Sillä Jumala on meidän kanssamme.
\par 11 Sillä näin sanoi minulle Herra, kun hänen kätensä valtasi minut ja hän varoitti minua vaeltamasta tämän kansan tietä:
\par 12 "Älkää sanoko salaliitoksi kaikkea, mitä tämä kansa salaliitoksi sanoo; älkää peljätkö, mitä se pelkää, älkääkä kauhistuko.
\par 13 Herra Sebaot pitäkää pyhänä, häntä te peljätkää ja kauhistukaa.
\par 14 Ja hän on oleva pyhäkkö, hän loukkauskivi ja kompastuksen kallio molemmille Israelin huonekunnille, paula ja ansa Jerusalemin asukkaille.
\par 15 Monet heistä kompastuvat ja kaatuvat ja ruhjoutuvat, monet kiedotaan ja vangitaan.
\par 16 Sido todistus talteen, lukitse laki sinetillä minun opetuslapsiini."
\par 17 Niin minä odotan Herraa, joka kätkee kasvonsa Jaakobin heimolta, ja panen toivoni häneen.
\par 18 Katso, minä ja lapset, jotka Herra on minulle antanut, me olemme Herran Sebaotin merkkeinä ja ihmeinä Israelissa, hänen, joka asuu Siionin vuorella.
\par 19 Ja kun he sanovat teille: "Kysykää vainaja- ja tietäjähengiltä, jotka supisevat ja mumisevat", niin eikö kansa kysyisi Jumalaltansa? Kuolleiltako elävien puolesta?
\par 20 "Pysykää laissa ja todistuksessa!" Elleivät he näin sano, ei heillä aamunkoittoa ole.
\par 21 Ja siellä he kuljeskelevat vaivattuina ja nälkäisinä, ja nälissään he vimmastuvat ja kiroavat kuninkaansa ja Jumalansa. He luovat silmänsä korkeuteen,
\par 22 he katsahtavat maan puoleen, mutta katso: ainoastaan ahdistusta ja pimeyttä; he ovat syöstyt tuskan synkeyteen ja pimeyteen.
\par 23 Mutta ei jää pimeään se, mikä nyt on vaivan alla. Entiseen aikaan hän saattoi halveksituksi Sebulonin maan ja Naftalin maan, mutta tulevaisuudessa hän saattaa kunniaan merentien, Jordanin tuonpuoleisen maan, pakanain alueen.

\chapter{9}

\par 1 Kansa, joka pimeydessä vaeltaa, näkee suuren valkeuden; jotka asuvat kuoleman varjon maassa, niille loistaa valkeus.
\par 2 Sinä lisäät kansan, annat sille suuren ilon; he iloitsevat sinun edessäsi, niinkuin elonaikana iloitaan, niinkuin saaliinjaossa riemuitaan.
\par 3 Sillä heidän kuormansa ikeen, heidän olkainsa vitsan ja heidän käskijänsä sauvan sinä särjet niinkuin Midianin päivänä;
\par 4 ja kaikki taistelun pauhussa tallatut sotakengät ja verellä tahratut vaipat poltetaan ja tulella kulutetaan.
\par 5 Sillä lapsi on meille syntynyt, poika on meille annettu, jonka hartioilla on herraus, ja hänen nimensä on: Ihmeellinen neuvonantaja, Väkevä Jumala, Iankaikkinen isä, Rauhanruhtinas.
\par 6 Herraus on oleva suuri ja rauha loppumaton Daavidin valtaistuimella ja hänen valtakunnallansa; se perustetaan ja vahvistetaan tuomiolla ja vanhurskaudella nyt ja iankaikkisesti. Herran Sebaotin kiivaus on sen tekevä.
\par 7 Herra lähetti sanan Jaakobia vastaan, ja se iski Israeliin,
\par 8 ja sen sai tuntea koko kansa, Efraim ja Samarian asukkaat, mutta he sanoivat ylpeydessään ja sydämensä kopeudessa:
\par 9 "Tiilikivet sortuivat maahan, mutta me rakennamme hakatuista kivistä; metsäviikunapuut lyötiin poikki, mutta me panemme setripuita sijaan".
\par 10 Silloin Herra nostatti sitä vastaan Resinin ahdistajat ja kiihoitti sen viholliset,
\par 11 aramilaiset edestä ja filistealaiset takaa, ja he söivät Israelia suun täydeltä. Kaikesta tästä ei hänen vihansa ole asettunut, ja vielä on hänen kätensä ojennettu.
\par 12 Mutta kansa ei palannut kurittajansa tykö, eivätkä he etsineet Herraa Sebaotia.
\par 13 Sentähden Herra katkaisi Israelilta pään ja hännän, palmunlehvän ja kaislan, yhtenä päivänä.
\par 14 Vanhin ja arvomies on pää, mutta profeetta, joka valhetta opettaa, on häntä.
\par 15 Ja tämän kansan johtajat tulivat eksyttäjiksi, ja johdettavat joutuivat hämminkiin.
\par 16 Sentähden Herra ei iloitse sen nuorista miehistä eikä armahda sen orpoja ja leskiä, sillä he ovat kaikki jumalattomia ja pahantekijöitä, ja jokainen suu puhuu hulluutta. Kaikesta tästä ei hänen vihansa ole asettunut, ja vielä on hänen kätensä ojennettu.
\par 17 Sillä jumalattomuus palaa tulena, kuluttaa orjantappurat ja ohdakkeet ja sytyttää sankan metsän, niin että se savuna tupruaa ilmaan.
\par 18 Herran Sebaotin vihasta maa syttyy palamaan, ja kansa on kuin tulen syötävänä; toinen ei sääli toistansa.
\par 19 Ahmivat oikealta, mutta on yhä nälkä, syövät vasemmalta, mutta eivät saa kylläänsä; jokainen syö oman käsivartensa lihaa:
\par 20 Manasse Efraimia ja Efraim Manassea, molemmat yhdessä käyvät Juudan kimppuun. Kaikesta tästä ei hänen vihansa ole asettunut, ja vielä on hänen kätensä ojennettu.

\chapter{10}

\par 1 Voi niitä, jotka vääriä säädöksiä säätävät, jotka turmiollisia tuomioita kirjoittelevat,
\par 2 vääntääksensä vaivaisten asian ja riistääksensä minun kansani kurjilta oikeuden, että lesket joutuisivat heidän saaliiksensa ja orvot heidän ryöstettäviksensä!
\par 3 Mutta mitä te teette koston päivänä, rajumyrskyssä, joka tulee kaukaa? Kenen turviin pakenette apua saamaan ja minne talletatte tavaranne?
\par 4 Ei muuta kuin vaipua vangittujen joukkoon tai kaatua surmattujen sekaan. Kaikesta tästä ei hänen vihansa ole asettunut, ja vielä on hänen kätensä ojennettu.
\par 5 Voi Assuria, joka on minun vihani vitsa ja jolla on kädessään minun suuttumukseni sauva!
\par 6 Minä lähetän hänet jumalattoman kansakunnan kimppuun, käsken hänet vihastukseni kansaa vastaan saalista saamaan ja ryöstöä ryöstämään ja tallaamaan sitä kuin katujen lokaa.
\par 7 Mutta hän ei ajattele niin, se ei ole hänen sydämensä aivoitus, vaan hänen sydämensä halu on hävittää ja tuhota kansoja paljon.
\par 8 Sillä hän sanoo: "Eivätkö minun päämieheni ole kaikki kuninkaita?
\par 9 Eikö käynyt Kalnon niinkuin Karkemiin, eikö Hamatin niinkuin Arpadin ja Samarian niinkuin Damaskon?
\par 10 Niinkuin minun käteni on saavuttanut epäjumalien valtakunnat, joiden jumalankuvat olivat paremmat kuin Jerusalemin ja Samarian -
\par 11 niin enkö minä tekisi Jerusalemille ja sen epäjumalankuville, samoin kuin minä tein Samarialle ja sen epäjumalille?"
\par 12 Mutta kun Herra on päättänyt kaiken työnsä Siionin vuorella ja Jerusalemissa, niin minä kostan Assurin kuninkaalle hänen sydämensä ylpeyden hedelmän ja hänen kopeitten silmiensä korskan.
\par 13 Sillä hän sanoo: "Oman käteni voimalla minä sen tein ja viisaudellani, sillä minä olen ymmärtäväinen. Minä siirsin kansojen rajat ja ryöstin heidän aarteensa ja puskin kumoon kuin härkä valtaistuimella-istujat.
\par 14 Minun käteni tavoitti kansojen rikkaudet niinkuin linnun pesän; ja niinkuin hyljätyt munat kootaan, niin minä olen koonnut kaikki maat, eikä ketään ollut, joka olisi siipeä räpyttänyt tai nokkaa avannut ja piipittänyt."
\par 15 Korskeileeko kirves hakkaajaansa vastaan, tai suurenteleeko saha heiluttajaansa vastaan? Ikäänkuin vitsa heiluttaisi kohottajaansa ja sauva saisi koholle sen, joka ei ole puuta!
\par 16 Sentähden Herra, Herra Sebaot, lähettää hänen lihavuuteensa näivetystaudin, ja hänen kunniansa alle syttyy tuli niinkuin tulipalo.
\par 17 Israelin valkeus on tuleva tuleksi ja hänen Pyhänsä liekiksi, joka polttaa ja kuluttaa hänen ohdakkeensa ja orjantappuransa yhtenä päivänä
\par 18 ja hänen metsänsä ja puutarhansa ihanuuden, hamaan luihin ja ytimiin, ja hän tulee riutuvan sairaan kaltaiseksi.
\par 19 On helppo lukea ne puut, jotka jäävät hänen metsästänsä jäljelle; poikanen voi ne kirjoittaa muistiin.
\par 20 Sinä päivänä ei Israelin jäännös eivätkä Jaakobin heimon pelastuneet enää turvaudu lyöjäänsä, vaan totuudessa he turvautuvat Herraan, Israelin Pyhään.
\par 21 Jäännös palajaa, Jaakobin jäännös, väkevän Jumalan tykö.
\par 22 Sillä vaikka sinun kansasi, Israel, olisi kuin meren hiekka, on siitä palajava ainoastaan jäännös. Päätetty on hävitys, joka tulee, vanhurskautta tulvillaan.
\par 23 Sillä hävityksen ja tuomiopäätöksen panee Herra, Herra Sebaot, toimeen kaikessa maassa.
\par 24 Sentähden sanoo Herra, Herra Sebaot, näin: "Älä pelkää, minun kansani, joka Siionissa asut, Assuria, kun hän sinua vitsalla lyö ja kohottaa sauvansa sinua vastaan Egyptin tavalla.
\par 25 Sillä lyhyt hetki vielä, niin suuttumus täyttyy, ja minun vihani kääntyy hävittämään heidät."
\par 26 Ja Herra Sebaot heiluttaa ruoskaa häntä vastaan, niinkuin silloin, kun Midian lyötiin Oorebin kalliolla, ja hänen sauvansa on ojennettuna meren yli, ja hän kohottaa sen niinkuin muinoin Egyptiä vastaan.
\par 27 Sinä päivänä heltiää hänen kuormansa sinun hartioiltasi ja hänen ikeensä sinun niskaltasi, sillä ies särkyy lihavuuden pakosta.
\par 28 Hän tulee Aijatiin, kulkee Migronin kautta, jättää kuormastonsa Mikmaaseen;
\par 29 he kulkevat solatien poikki: "Geba on yöpaikkamme". Raama vapisee, Saulin Gibea pakenee.
\par 30 Huuda kimakasti, tytär Gallim! Kuuntele, Laisa! Poloinen Anatot!
\par 31 Madmena menee pakoon, Geebimin asukkaat saattavat tavaransa turviin.
\par 32 Vielä samana päivänä hän pysähtyy Noobiin, hän kohottaa kätensä tytär Siionin vuorta, Jerusalemin kukkulaa, vastaan -
\par 33 katso, silloin Herra, Herra Sebaot, katkaisee hänen latvansa kauhistavalla voimalla, vartevat rungot kaadetaan, ja korkeat kukistuvat.
\par 34 Metsän tiheikkö hakataan kirveellä maahan, ja Libanon kaatuu Voimallisen edessä.

\chapter{11}

\par 1 Mutta Iisain kannosta puhkeaa virpi, ja vesa versoo hänen juuristansa.
\par 2 Ja hänen päällänsä lepää Herran Henki, viisauden ja ymmärryksen henki, neuvon ja voiman henki, tiedon ja Herran pelon henki.
\par 3 Hän halajaa Herran pelkoa; ei hän tuomitse silmän näöltä eikä jaa oikeutta korvan kuulolta,
\par 4 vaan tuomitsee vaivaiset vanhurskaasti ja jakaa oikein oikeutta maan nöyrille; suunsa sauvalla hän lyö maata, surmaa jumalattomat huultensa henkäyksellä.
\par 5 Vanhurskaus on hänen kupeittensa vyö ja totuus hänen lanteittensa side.
\par 6 Silloin susi asuu karitsan kanssa, ja pantteri makaa vohlan vieressä; vasikka ja nuori leijona ja syöttöhärkä ovat yhdessä, ja pieni poikanen niitä paimentaa.
\par 7 Lehmä ja karhu käyvät laitumella, niiden vasikat ja pennut yhdessä makaavat, ja jalopeura syö rehua kuin raavas.
\par 8 Imeväinen leikittelee kyykäärmeen kololla, ja vieroitettu kurottaa kätensä myrkkyliskon luolaan.
\par 9 Ei missään minun pyhällä vuorellani tehdä pahaa eikä vahinkoa, sillä maa on täynnä Herran tuntemusta, niinkuin vedet peittävät meren.
\par 10 Sinä päivänä pakanat etsivät Iisain juurta, joka on kansojen lippuna, ja hänen asumuksensa on oleva kunniata täynnä.
\par 11 Ja sinä päivänä Herra vielä toisen kerran ojentaa kätensä hankkiakseen itselleen kansansa jäännöksen, joka on jäljellä Assurissa, Egyptissä, Patroksessa, Etiopiassa, Eelamissa, Sinearissa, Hamatissa ja merensaarilla.
\par 12 Hän nostaa viirin pakanakansoille ja kokoaa Israelin karkoitetut miehet; ja Juudan hajoitetut naiset hän kerää maan neljästä äärestä.
\par 13 Silloin katoaa Efraimin kateus, ja Juudan vihat häviävät. Efraim ei kadehdi Juudaa, eikä Juuda vihaa Efraimia.
\par 14 Ja he lentävät länteen päin filistealaisten niskaan, yhdessä he ryöstävät Idän miehiä. Edom ja Mooab joutuvat heidän käsiinsä, ammonilaiset heidän alamaisikseen.
\par 15 Ja Herra vihkii tuhon omaksi Egyptin merenlahden ja vihansa hehkussa kohottaa kätensä Eufrat-virtaa vastaan, lyö sen hajalle seitsemäksi puroksi ja tekee sen kengin kuljettavaksi.
\par 16 Siitä tulee valtatie hänen kansansa jäännökselle, joka on jäljellä Assurissa, niinkuin tuli Israelille silloin, kun se Egyptin maasta lähti.

\chapter{12}

\par 1 Sinä päivänä sinä sanot: "Minä kiitän sinua, Herra, sillä sinä olit minuun vihastunut, mutta sinun vihasi asettui, ja sinä lohdutit minua.
\par 2 Katso, Jumala on minun pelastukseni; minä olen turvassa enkä pelkää, sillä Herra, Herra on minun väkevyyteni ja ylistysvirteni, hän tuli minulle pelastukseksi."
\par 3 Te saatte ilolla ammentaa vettä pelastuksen lähteistä.
\par 4 Ja sinä päivänä te sanotte: "Kiittäkää Herraa, julistakaa hänen nimeänsä, tehkää hänen suuret tekonsa tiettäviksi kansain keskuudessa, tunnustakaa, että hänen nimensä on korkea.
\par 5 Veisatkaa ylistystä Herralle, sillä jaloja töitä hän on tehnyt; tulkoot ne tunnetuiksi kaikessa maassa.
\par 6 Huutakaa ja riemuitkaa, Siionin asukkaat, sillä suuri on teidän keskellänne Israelin Pyhä."

\chapter{13}

\par 1 Ennustus Baabelista, Jesajan, Aamoksen pojan, näkemä.
\par 2 Pystyttäkää viiri paljaalle vuoren laelle, korottakaa äänenne heille, viittokaa kädellä heitä menemään sisälle ruhtinasten porteista.
\par 3 Minä olen antanut käskyn vihkiytyneilleni ja kutsunut urhoni vihani työhön, ylvääni, riemuitsevaiseni.
\par 4 Kuule, vuorilla käy kuin paljon väen pauhina; kuule valtakuntain, kokoontuneitten kansojen kohinaa: Herra Sebaot katsastaa sotajoukkoansa.
\par 5 He tulevat kaukaisesta maasta, taivaan ääristä, Herra ja hänen vihansa aseet, hävittämään kaiken maan.
\par 6 Valittakaa, sillä Herran päivä on lähellä, se tulee kuin hävitys Kaikkivaltiaalta.
\par 7 Sentähden herpoavat kaikki kädet, ja kaikki ihmissydämet raukeavat.
\par 8 He peljästyvät, kivut ja tuskat valtaavat heidät, he vääntelehtivät kuin synnyttäjä, tuijottavat toisiinsa tyrmistyneinä, kasvot tulenkarvaisina.
\par 9 Katso, Herran päivä tulee, tulee armottomana, tulee kiivaus ja vihan hehku, tekemään autioksi maan ja hävittämään siitä sen syntiset.
\par 10 Sillä taivaan tähdet ja Kalevanmiekat eivät loista valollansa, aurinko on pimeä noustessansa, kuu ei kirkkaana kumota.
\par 11 Minä kostan maanpiirille sen pahuuden ja jumalattomille heidän pahat tekonsa; minä lopetan julkeitten kopeuden ja painan maahan väkivaltaisten ylpeyden.
\par 12 Minä teen kuolevaiset harvinaisemmiksi kuin puhdas kulta, ihmiset harvinaisemmiksi kuin Oofirin kulta.
\par 13 Sentähden minä järisytän taivaat, ja maa järkkyy paikaltansa Herran Sebaotin kiivaudesta, hänen vihansa hehkun päivänä.
\par 14 Ja niinkuin säikytetyt gasellit ja niinkuin lampaat, joilla ei ole kokoajaa, he kääntyvät kukin kansansa luo, pakenevat kukin omalle maallensa.
\par 15 Kuka ikinä tavataan, se lävistetään, kuka kiinni joutuu, se miekkaan kaatuu.
\par 16 Heidän pienet lapsensa murskataan heidän silmäinsä edessä, heidän talonsa ryöstetään, ja heidän vaimonsa raiskataan.
\par 17 Katso, minä herätän heitä vastaan meedialaiset, jotka eivät hopeasta huoli eivätkä kullasta välitä.
\par 18 Heidän jousensa kaatavat nuoret miehet; he eivät armahda kohdun hedelmää eivätkä lapsia sääli.
\par 19 Ja Baabelin, valtakuntain kaunistuksen, kaldealaisten ylpeyden ja loiston, käy niinkuin Sodoman ja Gomorran, jotka Jumala hävitti.
\par 20 Ei sitä ikinä enää asuta, autioksi jää se polvesta polveen; ei arabialainen sinne telttaansa tee, eivätkä paimenet siellä laumaansa lepuuta.
\par 21 Erämaan eläimet lepäävät siellä, sen huoneet ovat täynnä huuhkajia, kamelikurjet asuvat siellä, ja metsänpeikot siellä hyppelevät.
\par 22 Sakaalit ulvovat sen palatseissa, aavikkosudet huvilinnoissa. Sen aika on lähellä, tulemaisillaan, ei sen päiviä pidennetä.

\chapter{14}

\par 1 Sillä Herra armahtaa Jaakobia ja valitsee vielä Israelin ja sijoittaa heidät heidän omaan maahansa; muukalaiset liittyvät heihin ja yhtyvät Jaakobin heimoon.
\par 2 Kansat ottavat heidät ja tuovat heidät heidän kotiinsa, ja Israelin heimo saa heidät Herran maassa omiksensa, orjiksi ja orjattariksi, ja he vangitsevat vangitsijansa ja vallitsevat käskijöitänsä.
\par 3 Ja sinä päivänä, jona Herra päästää sinut rauhaan vaivastasi, tuskastasi ja siitä kovasta työstä, jota sinulla teetettiin,
\par 4 sinä virität tämän pilkkalaulun Baabelin kuninkaasta ja sanot: "Kuinka on käskijästä tullut loppu, tullut loppu ahdistuksesta!
\par 5 Herra on murtanut jumalattomain sauvan, valtiaitten vitsan,
\par 6 joka kiukussa löi kansoja, löi lakkaamatta, joka vihassa vallitsi kansakuntia, vainosi säälimättä.
\par 7 Kaikki maa on saanut levon ja rauhan, he puhkeavat riemuun.
\par 8 Kypressitkin sinusta iloa pitävät sekä Libanonin setrit: 'Sinun maata mentyäsi ei nouse kukaan meitä hakkaamaan'.
\par 9 Tuonela tuolla alhaalla liikkuu sinun tähtesi, ottaaksensa sinut vastaan, kun tulet. Se herättää haamut sinun tähtesi, kaikki maan johtomiehet, nostattaa valtaistuimiltansa kaikki kansojen kuninkaat.
\par 10 Kaikki he lausuvat ja sanovat sinulle: 'Myös sinä olet riutunut niinkuin mekin, olet tullut meidän kaltaiseksemme'.
\par 11 Alas tuonelaan on vaipunut sinun komeutesi, sinun harppujesi helinä; sinun allasi ovat vuoteena madot, toukat ovat peittonasi.
\par 12 Kuinka olet taivaalta pudonnut, sinä kointähti, aamuruskon poika! Kuinka olet maahan syösty, sinä kansojen kukistaja!
\par 13 Sinä sanoit sydämessäsi: 'Minä nousen taivaaseen, korkeammalle Jumalan tähtiä minä istuimeni korotan ja istun ilmestysvuorelle, pohjimmaiseen Pohjolaan.
\par 14 Minä nousen pilvien kukkuloille, teen itseni Korkeimman vertaiseksi.'
\par 15 Mutta sinut heitettiin alas tuonelaan, pohjimmaiseen hautaan.
\par 16 Jotka sinut näkevät, ne katsovat pitkään, tarkastavat sinua: 'Onko tämä se mies, joka järisytti maan, järkytti valtakunnat,
\par 17 joka teki maanpiirin erämaaksi ja hävitti sen kaupungit, joka ei päästänyt vankejansa kotiin?'
\par 18 Kansojen kuninkaat kaikki lepäävät kunniassa, kukin kammiossansa.
\par 19 Mutta sinä olet kaukana haudastasi, poisviskattuna niinkuin hylkyvesa, olet peittynyt surmattujen, miekalla lävistettyjen, kiviseen kuoppaan suistuneitten alle, olet kuin tallattu raato.
\par 20 Et sinä saa yhtyä heidän kanssaan haudassa, sillä sinä olet maasi hävittänyt, kansasi tappanut. Ei ikinä enää mainita pahantekijäin sukua.
\par 21 Laittakaa hänen lapsillensa verilöyly heidän isiensä pahain tekojen tähden, etteivät he nousisi ottamaan omaksensa maata ja täyttäisi maanpiiriä kaupungeilla."
\par 22 Minä nousen heitä vastaan, sanoo Herra Sebaot, ja hävitän Baabelilta nimen ja jäännöksen, suvun ja jälkeläiset, sanoo Herra.
\par 23 Minä teen sen tuonenkurkien perinnöksi ja vesirämeeksi, minä lakaisen sen pois hävityksen luudalla, sanoo Herra Sebaot.
\par 24 Herra Sebaot on vannonut sanoen: Totisesti, mitä minä olen ajatellut, se tapahtuu, mitä minä olen päättänyt, se toteutuu:
\par 25 minä muserran Assurin maassani ja tallaan hänet vuorillani, niin että hänen ikeensä heltiää heidän päältään ja hänen kuormansa heltiää heidän hartioiltansa.
\par 26 Tämä on päätös, päätetty kaikesta maasta, tämä on käsi, ojennettu kaikkien kansojen yli.
\par 27 Sillä Herra Sebaot on päättänyt, - kuka sen tyhjäksi tekee? Hänen kätensä on ojennettu, - kuka sen torjuu?
\par 28 Kuningas Aahaan kuolinvuotena tuli tämä ennustus:
\par 29 Älä iloitse, Filistea kaikkinesi, siitä että sauva, joka sinua löi, on murtunut. Sillä käärmeen juuresta kasvaa myrkkylisko, ja sen hedelmä on lentävä käärme.
\par 30 Vaivaisista kurjimmat löytävät laitumen, ja köyhät saavat turvassa levätä, mutta sinun juuresi minä tapan nälkään, ja sinun jäännöksesi surmataan.
\par 31 Valita, portti, huuda, kaupunki; menehdy pelkoon, Filistea kaikkinesi, sillä pohjoisesta tulee savu; eikä yksikään erkane vihollisen riveistä.
\par 32 Ja mitä vastataan pakanakansan sanansaattajille? "Herra on perustanut Siionin, ja hänen kansansa kurjat saavat siinä turvan".

\chapter{15}

\par 1 Ennustus Mooabista. Hävityksen yönä tuhoutuu Aar-Mooab; hävityksen yönä tuhoutuu Kiir-Mooab.
\par 2 Bait ja Diibon nousevat uhrikukkuloille itkemään, Nebolla ja Meedebassa Mooab valittaa; kaikki päät ovat paljaiksi ajellut, kaikki parrat ovat leikatut.
\par 3 Kaduilla he kääriytyvät säkkeihin, katoilla ja toreilla kaikki valittavat, menehtyvät itkuun.
\par 4 Hesbon ja Elale huutavat; Jahaaseen asti kuuluu niiden parku. Sentähden Mooabin varustetut miehet vaikeroivat, sen sielu vapisee.
\par 5 Minun sydämeni huutaa Mooabin tähden; sen pakolaisia on aina Sooariin, aina Eglat-Selisijjaan asti. Luuhitin solatietä he nousevat itkien, Hooronaimin tiellä he puhkeavat parkuun hävityksen tähden.
\par 6 Nimrimin vedet ehtyvät erämaaksi, heinä kuivuu, ruoho lakastuu, vihantaa ei ole.
\par 7 Sentähden he kantavat hankkimansa säästön ja tallettamansa tavaran Pajupuron yli.
\par 8 Parku kiertää Mooabin rajoja, valitus kuuluu Eglaimiin asti, valitus Beer-Eelimiin asti.
\par 9 Sillä Diimonin vedet ovat verta täynnä, ja vielä muutakin minä tuotan Diimonille: leijonan Mooabin pelastettujen ja maahan jääneitten kimppuun.

\chapter{16}

\par 1 "Lähettäkää maanhallitsijalle tulevat lampaat Selasta erämaan kautta tytär Siionin vuorelle."
\par 2 Ja niinkuin pakenevat linnut, säikytetty pesue, ovat Mooabin tyttäret Arnonin kahlauspaikoilla.
\par 3 "Anna neuvo, käy välittämään, tee varjosi yön kaltaiseksi keskellä päivää, kätke karkoitetut, älä ilmaise pakenevia.
\par 4 Salli minun karkoitettujeni asua tykönäsi, ole Mooabille suojana hävittäjää vastaan. Sillä sortajalle tulee loppu, hävitys lakkaa, ja polkija on maasta poissa;
\par 5 ja valtaistuin vahvistetaan laupeudella, ja sillä istuu vakaasti Daavidin majassa tuomari, joka harrastaa oikeutta ja vanhurskautta ahkeroitsee."
\par 6 "Me olemme kuulleet Mooabin ylpeilyn, tuon ylen korskean, hänen kopeilunsa, ylpeilynsä ja vihansa, hänen väärät puheensa."
\par 7 Sentähden Mooab Mooabille valittaa, kaikki he valittavat; Kiir-Haresetin rypälekakkuja te huokailette, ratki murtuneina.
\par 8 Sillä kuihtuneet ovat Hesbonin viinitarhat, Sibman viiniköynnökset; niiden jalot rypäleet kaatoivat kansojen valtiaita, ne ulottuivat Jaeseriin asti, harhailivat halki erämaan, niiden vesat levisivät, menivät yli meren.
\par 9 Sentähden minä itken Jaeserin kanssa, itken Sibman viiniköynnöksiä, kastelen kyyneleilläni sinua, Hesbon, ja sinua, Elale, sillä hedelmän- ja viininkorjuun keskeen on kajahtanut sotahuuto.
\par 10 Ilo ja riemu on tauonnut puutarhasta, viinitarhoissa ei riemuita, ei remahdella; ei poljeta viiniä viinikuurnissa. "Minä olen vaientanut ilohuudon."
\par 11 Siksi väräjää minun sisimpäni Mooabin tähden niinkuin kannel, minun sydämeni Kiir-Hereksen tähden.
\par 12 Ja vaikka Mooab kuinka astuisi uhrikukkulalle ja sillä itsensä uuvuttaisi ja vaikka kuinka menisi pyhäkköönsä rukoilemaan, ei se häntä auta.
\par 13 Tämä on sana, jonka Herra ennen puhui Mooabille.
\par 14 Mutta nyt Herra puhuu sanoen: Ennenkuin kolme vuotta on kulunut - kolme palkkalaisen vuotta - painuu halvaksi Mooabin kunnia, sen suuret laumat, ja jäännös on oleva pieni, vähäpätöinen, mitätön.

\chapter{17}

\par 1 Ennustus Damaskosta. Katso, Damasko poistetaan kaupunkien luvusta ja luhistuu raunioiksi.
\par 2 Autioiksi jäävät Aroerin kaupungit, karjalaumojen haltuun; ne lepäävät siellä, kenenkään peloittelematta.
\par 3 Efraimilta menee suojavarustus ja Damaskolta kuninkuus ja Aramista jäännöskin: niiden käy niinkuin israelilaisten kunnian, sanoo Herra Sebaot.
\par 4 Ja sinä päivänä Jaakobin kunnia köyhtyy, ja hänen ruumiinsa lihavuus laihtuu.
\par 5 Ja käy niinkuin leikkaajan kouraistessa viljaa ja hänen käsivartensa leikatessa tähkäpäitä, käy niinkuin tähkäpäitä poimittaessa Refaimin tasangolla.
\par 6 Se siinä jää jälkikorjuuta, mikä öljypuuta karistettaessa: pari, kolme marjaa korkealle latvaan, neljä, viisi hedelmäpuun oksiin, sanoo Herra, Israelin Jumala.
\par 7 Sinä päivänä ihminen luo katseen Luojaansa, ja hänen silmänsä katsovat Israelin Pyhään,
\par 8 eikä hän luo katsettaan alttareihin, omain kättensä tekoon, eikä katso niihin, mitä hänen sormensa ovat tehneet, ei asera-karsikkoihin, ei auringonpatsaisiin.
\par 9 Sinä päivänä ovat hänen linnakaupunkinsa niinkuin autiopaikat metsissä ja kukkuloilla, jotka jätettiin autioiksi israelilaisten tullessa, ja maa muuttuu erämaaksi.
\par 10 Sillä sinä unhotit pelastuksesi Jumalan etkä muistanut suojakalliotasi. Sentähden sinä istutat ihania istutuksia ja kylvät vieraita viiniköynnöksiä;
\par 11 istuttamispäivänä sinä saat ne isonemaan, ja aamulla saat kylvösi kukoistamaan - mutta poissa on sato sairauden päivänä, ja parantumaton on kipu.
\par 12 Voi paljojen kansojen pauhua - ne pauhaavat, niinkuin meri pauhaa, ja kansakuntien kohinaa - ne kohisevat, niinkuin valtavat vedet kohisevat!
\par 13 Kansakunnat kohisevat, niinkuin suuret vedet kohisevat, mutta hän nuhtelee niitä, ja ne pakenevat kauas; ne karkoitetaan niinkuin akanat tuulessa vuorilla, niinkuin lentävät lehdet rajuilmassa.
\par 14 Katso, ehtoolla on oleva kauhu, aamun tullen ei heitä enää ole. Tämä on riistäjäimme osa, ryöstäjäimme arpa.

\chapter{18}

\par 1 Voi siipien surinan maata, joka on tuolla puolen Etiopian jokien,
\par 2 joka lähettää sanansaattajat virralle, ruokovenheissä vesiä myöten! Lähtekää, te nopeat sanansaattajat, vartevan ja kiiltävä-ihoisen kansan luo, lähellä ja kaukana peljätyn kansan luo, väkevän sortajakansan luo, jonka maata joet halkovat.
\par 3 Te kaikki maanpiirin asukkaat, te maata asuvaiset, katsokaa, milloin viiri nostetaan vuorille, ja kuunnelkaa, kun pasunaan puhalletaan.
\par 4 Sillä näin on Herra minulle sanonut: "Minä pysyn hiljaa ja katselen asunnostani, niinkuin hehkuva helle päivän paistaessa, niinkuin kastepilvi elonajan helteessä".
\par 5 Sillä ennen viininkorjuuta, kun kukinta on lopussa ja kukasta on tullut kypsyvä marja, hän hakkaa vesurilla köynnökset ja hävittää, katkoo varret.
\par 6 Ne jätetään kaikki vuorten petolintujen ja maan eläinten haltuun. Petolinnut siellä kesää pitävät, ja kaikkinaiset maan eläimet siellä talvehtivat.
\par 7 Siihen aikaan varteva ja kiiltävä-ihoinen kansa, lähellä ja kaukana peljätty kansa, väkevä sortajakansa, jonka maata joet halkovat, tuodaan lahjaksi Herralle Sebaotille siihen paikkaan, missä asuu Herran Sebaotin nimi: Siionin vuorelle.

\chapter{19}

\par 1 Ennustus Egyptistä. Katso, Herra ajaa nopealla pilvellä ja tulee Egyptiin. Silloin vapisevat Egyptin epäjumalat hänen edessään, ja sydän raukeaa Egyptin rinnassa.
\par 2 Minä kiihoitan Egyptin Egyptiä vastaan, niin että he keskenänsä sotivat, ystävä ystävää vastaan, kaupunki kaupunkia vastaan, valtakunta valtakuntaa vastaan.
\par 3 Egyptiltä katoaa rohkeus rinnasta, ja minä hämmennän hänen neuvonsa; ja he kysyvät epäjumalilta, henkienmanaajilta, vainaja- ja tietäjähengiltä.
\par 4 Minä annan Egyptin tylyn herran käsiin, ja kova kuningas on hallitseva heitä, sanoo Herra, Herra Sebaot.
\par 5 Vedet virrasta loppuvat, joki ehtyy ja kuivuu.
\par 6 Joet levittävät löyhkää, Egyptin kanavat mataloituvat ja ehtyvät, ruoko ja kaisla lakastuvat.
\par 7 Ruohostot Niilivirran varsilla, virran suussa, ja kaikki Niilivirran kylvömaat kuivuvat, kuihtuvat ja häviävät.
\par 8 Kalastajat valittavat ja murehtivat; kaikki, jotka heittävät onkea virtaan ja laskevat verkkoa veteen, nääntyvät.
\par 9 Häpeään joutuvat, jotka häkilöityjä pellavia pitelevät ja jotka kutovat pellavakankaita.
\par 10 Maan peruspylväät murskataan, ja kaikki palkkalaiset ovat murheellisella mielellä.
\par 11 Hulluina ovat Sooanin päämiehet kaikki, viisaimmat faraon neuvonantajista ovat neuvossaan tyhmistyneet. Kuinka voitte sanoa faraolle: "Olen viisaitten poika, muinaisajan kuninkaitten jälkeläinen"?
\par 12 Missä ovat viisaasi? Ilmoittakoot sinulle - hehän sen tietävät - mitä Herra Sebaot on Egyptin osalle päättänyt.
\par 13 Sooanin päämiehet ovat typertyneet, Noofin päämiehet ovat pettyneet. Egyptiä horjuttavat sen sukukuntain kulmakivet;
\par 14 Herra on vuodattanut sen keskuuteen sekasorron hengen, ja niin he saattavat Egyptin horjumaan kaikissa menoissansa, niinkuin juopunut horjahtaa oksennukseensa.
\par 15 Eikä menesty Egyptille mikään teko, tekipä sen pää tai häntä, palmunlehvä tai kaisla.
\par 16 Sinä päivänä Egypti on oleva naisten kaltainen; se vapisee ja pelkää Herran Sebaotin käden heilutusta, kun hän heiluttaa kättään sitä vastaan.
\par 17 Ja Juudan maa on oleva Egyptille kauhuksi; niin usein kuin sitä sille mainitaan, pelkää se Herran Sebaotin päätöstä, jonka hän siitä on päättänyt.
\par 18 Sinä päivänä on viisi kaupunkia Egyptin maassa puhuva Kanaanin kieltä ja vannova valan Herralle Sebaotille. Yhtä mainitaan nimellä Iir-Heres.
\par 19 Sinä päivänä on oleva Herran alttari keskellä Egyptin maata ja sen rajalla Herran patsas.
\par 20 Ja se on oleva merkkinä ja todistuksena Herralle Sebaotille Egyptin maassa; sillä he huutavat Herraa sortajain tähden, ja hän lähettää heille vapauttajan ja puolustajan, joka pelastaa heidät.
\par 21 Ja Herra tekee itsensä tunnetuksi egyptiläisille, ja egyptiläiset tuntevat Herran sinä päivänä ja palvelevat häntä teuras- ja ruokauhreilla, ja he tekevät Herralle lupauksia ja täyttävät ne.
\par 22 Herra lyö egyptiläisiä, lyö ja parantaa; he palajavat Herran tykö, ja hän kuulee heidän rukouksensa ja parantaa heidät.
\par 23 Sinä päivänä on oleva valtatie Egyptistä Assuriin, ja Assur yhtyy Egyptiin ja Egypti Assuriin, ja he, Egypti ynnä Assur, palvelevat Herraa.
\par 24 Sinä päivänä on Israel oleva kolmantena Egyptin ja Assurin rinnalla, siunauksena keskellä maata,
\par 25 sillä Herra Sebaot siunaa sitä sanoen: "Siunattu olkoon Egypti, minun kansani, ja Assur, minun kätteni teko, ja Israel, minun perintöosani".

\chapter{20}

\par 1 Sinä vuonna, jona Tartan, Sargonin, Assurin kuninkaan, lähettämänä, tuli Asdodiin, ryhtyi taisteluun Asdodia vastaan ja valloitti sen -
\par 2 siihen aikaan Herra puhui Jesajan, Aamoksen pojan, kautta sanoen: "Mene, riisu säkkipuku lanteiltasi ja vedä kengät jalastasi". Hän teki niin: kulki vaipatta ja avojaloin.
\par 3 Sitten Herra sanoi: "Niinkuin palvelijani Jesaja on kolme vuotta kulkenut vaipatta ja avojaloin, merkiksi ja enteeksi Egyptille ja Etiopialle,
\par 4 niin on Assurin kuningas kuljettava Egyptin sotavankeja ja Etiopian pakkosiirtolaisia, sekä nuoria että vanhoja, vaipattomina ja avojaloin, takapuolet paljaina - Egyptin häpeä!
\par 5 Silloin he kauhistuvat ja saavat häpeän Etiopiasta, joka oli heidän toivonsa, ja Egyptistä, joka oli heidän ylpeytensä.
\par 6 Ja tämän rantamaan asukkaat sanovat sinä päivänä: 'Katso, näin on käynyt meidän toivomme, jonka turviin me pakenimme apua saamaan, vapautuaksemme Assurin kuninkaasta. Kuinka me sitten itse pelastuisimme?'"

\chapter{21}

\par 1 Ennustus meren viereisestä erämaasta. Niinkuin myrskytuulet kiitävät Etelämaassa, niin tulee se erämaasta, peloittavasta maasta.
\par 2 Julma näky ilmoitettiin minulle: "Ryöstäjä ryöstää, hävittäjä hävittää. Hyökkää, Eelam! Piiritä, Meedia! Kaiken huokailun minä lopetan."
\par 3 Sentähden minun lanteeni ovat täynnä vavistusta, minut valtaavat tuskat, niinkuin synnyttäjän tuskat; minua huimaa, niin etten kuule, kauhistuttaa, niin etten näe.
\par 4 Minun sydämeni värisee, kauhu peljästyttää minut. Ikävöimäni iltahämärän se muuttaa minulle vavistukseksi.
\par 5 Pöytä katetaan, matto levitetään, syödään, juodaan. Nouskaa, te ruhtinaat, voidelkaa kilvet.
\par 6 Sillä näin on Herra minulle sanonut: "Mene, aseta tähystäjä; hän ilmoittakoon, mitä näkee.
\par 7 Ja kun hän näkee jonon ratsastajia parittain, jonon aaseja, jonon kameleja, niin kuunnelkoon tarkasti, hyvin tarkasti."
\par 8 Ja hän huusi kuin leijona: "Tähystyspaikalla, Herra, minä seison alati, päivät pitkät, pysyn vartiopaikallani kaiket yöt.
\par 9 Ja katso, tuolla tulee jonossa miehiä, parivaljakoita." Ja hän lausui sanoen: "Kukistunut, kukistunut on Baabel, ja kaikki sen jumalain kuvat hän on murskannut maahan".
\par 10 Maahan puitu kansani, puimatantereella poljettuni! Mitä olen kuullut Herralta Sebaotilta, Israelin Jumalalta, sen minä teille ilmoitan.
\par 11 Ennustus Duumasta. Minulle huudetaan Seiristä: "Vartija, mikä hetki yöstä on? Vartija, mikä hetki yöstä on?"
\par 12 Vartija vastaa: "Aamu on tullut, mutta silti on yö. Jos vielä kyselette, niin kyselkää; tulkaa uudelleen."
\par 13 Ennustus Arabiaa vastaan. Yöpykää Arabian viidakoissa, dedanilaiset matkueet.
\par 14 Menkää vastaan, viekää vettä janoaville. Teeman maan asukkaat ottavat pakolaiset vastaan, leipää tariten.
\par 15 Sillä he ovat miekkoja paossa, paljastettua miekkaa, jännitettyä jousta ja sodan tuimuutta paossa.
\par 16 Sillä näin on Herra minulle sanonut: Vielä vuosi - sellainen kuin on palkkalaisen vuosi - niin kaikki Keedarin kunnia on kadonnut,
\par 17 ja Keedarin urhojen jousiluvun jäännös on oleva vähäinen. Sillä Herra, Israelin Jumala, on puhunut.

\chapter{22}

\par 1 Ennustus Näkylaaksosta. Mikä sinun on, kun sinä kaikkinesi katoille nouset,
\par 2 sinä humuavainen, pauhaava kaupunki, sinä remuava kylä? Surmattusi eivät ole miekan surmaamia, eivät sotaan kuolleita.
\par 3 Kaikki sinun päämiehesi yhdessä pakenivat, joutuivat vangiksi, jousta vailla; keitä ikinä sinun omiasi tavattiin, ne vangittiin kaikki tyynni, kuinka kauas pakenivatkin.
\par 4 Sentähden minä sanon: "Kääntäkää katseenne minusta pois: minä itken katkerasti; älkää tunkeilko minua lohduttamaan, kun tytär, minun kansani, tuhoutuu".
\par 5 Sillä hämmingin, hävityksen ja häiriön päivän on Herra, Herra Sebaot, pitävä Näkylaaksossa. Muurit murrettiin, vuorille kohosi huuto.
\par 6 Eelam nosti viinen, tullen vaunuineen, väkineen ja ratsumiehineen, ja Kiir paljasti kilven.
\par 7 Ihanimmat laaksosi täyttyivät vaunuista, ratsumiehet asettuivat asemiin portin eteen.
\par 8 Hän riisui Juudalta verhon, ja sinä päivänä sinä loit katseesi Metsätalon asevarastoon.
\par 9 Te huomasitte monta repeämää Daavidin kaupungin muurissa ja kokositte vedet Alalammikkoon;
\par 10 te luitte Jerusalemin talot ja hajotitte taloja vahvistaaksenne muurin,
\par 11 ja te teitte molempien muurien välille paikan, johon Vanhan lammikon vedet koottiin. Mutta hänen puoleensa te ette katsoneet, joka tämän tuotti, häntä te ette nähneet, joka tämän kauan sitten valmisti.
\par 12 Ja sinä päivänä Herra, Herra Sebaot, kutsui itkuun ja valitukseen, pään paljaaksi ajamaan ja säkkiin vyöttäytymään.
\par 13 Mutta katso: on ilo ja riemu, raavaitten tappaminen ja lammasten teurastus, lihan syönti ja viinin juonti! "Syökäämme ja juokaamme, sillä huomenna me kuolemme."
\par 14 Niin kuului minun korviini Herran Sebaotin ilmoitus: "Totisesti, ei tämä teidän syntinne tule sovitetuksi, ei kuolemaanne saakka, sanoo Herra, Herra Sebaot".
\par 15 Näin sanoo Herra, Herra Sebaot: Mene tuon huoneenhaltijan tykö, Sebnan tykö, joka on linnan päällikkönä.
\par 16 "Mitä sinulla täällä on asiaa, ja keitä sinulla täällä on, kun tänne itsellesi haudan hakkautat, hakkautat hautasi korkeuteen, kallioon itsellesi asunnon koverrat?"
\par 17 Katso, Herra heittää sinua, mies, rajusti: hän kouraisee sinut kokoon,
\par 18 kääräisee sinut keräksi ja paiskaa pallona menemään maahan, jossa on laajuutta joka suuntaan. Sinne sinä kuolet, ja sinne jäävät sinun kunniavaunusi, sinä herrasi huoneen häpeä.
\par 19 Minä syöksen sinut pois paikastasi; hän kukistaa sinut asemastasi.
\par 20 Mutta sinä päivänä minä kutsun palvelijani Eljakimin, Hilkian pojan,
\par 21 ja puetan hänen yllensä sinun ihokkaasi, vyötän hänet sinun vyölläsi ja panen sinun valtasi hänen käteensä, niin että hän on oleva isänä Jerusalemin asukkaille ja Juudan suvulle.
\par 22 Ja minä panen Daavidin huoneen avaimen hänen olallensa; ja hän avaa, eikä kukaan sulje, ja hän sulkee, eikä kukaan avaa.
\par 23 Minä lyön hänet vaarnaksi vahvaan paikkaan, ja hän tulee kunniaistuimeksi isänsä suvulle.
\par 24 Hänen varaansa ripustautuu hänen isänsä suvun kaikki paljous, vesat ja versot, kaikki pienet astiat, maljat ja ruukut kaikki.
\par 25 Sinä päivänä, sanoo Herra Sebaot, pettää vahvaan paikkaan lyöty vaarna, se katkeaa ja putoaa, ja kuorma, joka oli sen varassa, särkyy. Sillä Herra on puhunut.

\chapter{23}

\par 1 Ennustus Tyyrosta. Valittakaa, te Tarsiin-laivat, sillä se on hävitetty talottomaksi, käymättömäksi; kittiläisten maasta he saivat tämän tiedon.
\par 2 Mykistykää, te rantamaan asukkaat! Siidonin kauppamiehet, merenkulkijat, täyttivät sinut.
\par 3 Ja suuria vesiä kulki Siihorin siemen, Niilivirran vilja; se oli Tyyron sato, ja siitä tuli kansojen kauppatavara.
\par 4 Häpeä, Siidon, sillä näin sanoo meri, meren linnoitus: "En ole kivuissa ollut, en ole synnyttänyt, en nuorukaisia kasvattanut, en neitoja vartuttanut".
\par 5 Kun tämä kuullaan Egyptissä, niin vavistaan Tyyron kuulumisia.
\par 6 Menkää Tarsiiseen; valittakaa, te rantamaan asukkaat.
\par 7 Onko tämä teidän remuava kaupunkinne, jonka alku on hamasta muinaisajasta, jonka jalat kuljettivat sen kauas muukalaisena asumaan?
\par 8 Kuka on tämän päättänyt Tyyron osalle, kruunujen jakelijan, jonka kauppamiehet olivat ruhtinaita, kauppiaat maanmainioita?
\par 9 Herra Sebaot on sen päättänyt, häväistäkseen kaiken koreuden korskan, saattaakseen kaikki maanmainiot halveksituiksi.
\par 10 Tulvi yli maasi, tytär Tarsis, niinkuin Niilivirta; ei ole patoa enää.
\par 11 Hän on ojentanut kätensä meren yli, järkyttänyt valtakunnat. Herra on antanut käskyn Kanaania vastaan, että sen linnoitukset hävitettäköön.
\par 12 Hän sanoo: "Älä enää ilakoitse, sinä häväisty neitsyt, tytär Siidon. Nouse, mene kittiläisten maahan; et sinä sielläkään saa levätä.
\par 13 Katso, kaldealaisten maa, kansa, jota ei ollut enää olemassa, jonka Assur oli valmistanut erämaan eläimille, - ne pystyttävät vartiotorninsa, ne kukistavat sen palatsit, tekevät sen raunioiksi.
\par 14 Valittakaa, te Tarsiin-laivat, sillä hävitetty on teidän linnoituksenne."
\par 15 Ja siihen aikaan Tyyro unhotetaan seitsemäksikymmeneksi vuodeksi, jotka ovat kuin saman kuninkaan aikaa. Seitsemänkymmenen vuoden kuluttua käy Tyyron, niinkuin porton laulussa sanotaan:
\par 16 "Ota kantele, kierrä kaupunkia, sinä unhotettu portto; soita kauniisti, laula vireästi, että sinut muistettaisiin!"
\par 17 Ja seitsemänkymmenen vuoden kuluttua Herra katsoo Tyyron puoleen, ja se pääsee jälleen portonpalkoilleen ja tekee huorin kaikkien maan valtakuntain kanssa, mitä maan päällä on.
\par 18 Mutta sen voitto ja palkka on oleva Herralle pyhitetty; ei sitä koota eikä talleteta, vaan sen voitto on tuleva niille, jotka Herran edessä asuvat, runsaaksi ravinnoksi ja jaloksi vaatetukseksi.

\chapter{24}

\par 1 Katso, Herra tekee maan tyhjäksi ja autioksi, mullistaa sen muodon ja hajottaa sen asukkaat.
\par 2 Ja niinkuin kansan käy, niin papinkin, niinkuin orjan, niin hänen herransa, niinkuin orjattaren, niin hänen emäntänsä, niinkuin ostajan, niin myyjän, niinkuin lainanottajan, niin lainanantajan, niinkuin velallisen, niin velkojankin.
\par 3 Maa tyhjentämällä tyhjennetään ja ryöstämällä ryöstetään. Sillä Herra on tämän sanan puhunut.
\par 4 Maa murehtii ja lakastuu, maanpiiri nääntyy ja lakastuu; kansan ylhäiset maassa nääntyvät.
\par 5 Maa on saastunut asukkaittensa alla, sillä he ovat rikkoneet lait, muuttaneet käskyt, hyljänneet iankaikkisen liiton.
\par 6 Sentähden kirous kalvaa maata, ja sen asukkaat syystänsä kärsivät; sentähden maan asukkaat kuumuudesta korventuvat, ja vähän jää ihmisiä jäljelle.
\par 7 Viini murehtii, viiniköynnös kuihtuu, kaikki ilomieliset huokaavat.
\par 8 Loppunut on vaskirumpujen riemu, lakannut remuavaisten melu, loppunut kanteleitten riemu.
\par 9 Ei laulaen viiniä juoda, väkijuoma käy karvaaksi juojillensa.
\par 10 Kukistettu on autio kaupunki, joka talo teljetty, sisään pääsemätön.
\par 11 Viinistä on kaduilla valitus, kaikki ilo on mennyt mailleen, riemu maasta paennut.
\par 12 Jäljellä on kaupungissa hävitys, portti on pirstaleiksi lyöty.
\par 13 Sillä niin on käyvä maan päällä, kansojen keskuudessa, kuin öljypuuta karistettaessa, kuin jälkikorjuussa, viininkorjuun päätyttyä.
\par 14 Ne korottavat äänensä ja riemuitsevat, ne huutavat mereltä päin Herran valtasuuruutta:
\par 15 "Sentähden kunnioittakaa Herraa valon mailla, ja meren saarilla Herran, Israelin Jumalan, nimeä".
\par 16 Maan äärestä kuulemme ylistysvirret: "Ihana on vanhurskaan osa!" Mutta minä sanon: "Riutumus, riutumus on minun osani, voi minua: ryöstäjät ryöstävät, raastaen ryöstäjät ryöstävät!"
\par 17 Kauhu ja kuoppa ja paula on edessäsi, sinä maan asukas.
\par 18 Joka pakenee kauhun ääntä, se putoaa kuoppaan, ja joka kuopasta nousee, se puuttuu paulaan. Sillä korkeuden akkunat aukenevat ja maan perustukset järkkyvät.
\par 19 Maa murskaksi musertuu, maa halkee ja hajoaa, maa horjuu ja huojuu.
\par 20 Maa hoippuu ja hoipertelee niinkuin juopunut, huojuu niinkuin lehvämaja. Raskaana painaa sitä sen rikkomus, se kaatuu eikä enää nouse.
\par 21 Sinä päivänä Herra kostaa korkeuden sotajoukolle korkeudessa ja maan kuninkaille maan päällä.
\par 22 Heidät kootaan sidottuina vankikuoppaan ja suljetaan vankeuteen; pitkän ajan kuluttua heitä etsiskellään.
\par 23 Ja kuu punastuu, ja aurinko häpeää, sillä Herra Sebaot on kuningas Siionin vuorella ja Jerusalemissa, ja hänen vanhintensa edessä loistaa kirkkaus.

\chapter{25}

\par 1 Herra, sinä olet minun Jumalani; minä kunnioitan sinua, kiitän sinun nimeäsi, sillä sinä olet tehnyt ihmeitä, sinun aivoituksesi kaukaisilta päiviltä ovat todet ja vakaat.
\par 2 Sillä sinä olet tehnyt kaupungin kiviroukkioksi, varustetun kaupungin raunioiksi; muukalaisten linna on kadonnut kaupunkien luvusta, ei sitä ikinä enää rakenneta.
\par 3 Sentähden sinua kunnioittaa väkevä kansa, väkivaltaisten pakanain kaupunki sinua pelkää.
\par 4 Sillä sinä olit turvana vaivaiselle, turvana köyhälle hänen ahdingossansa, suojana rankkasateelta, varjona helteeltä; sillä väkivaltaisten kiukku on kuin rankkasade seinää vastaan.
\par 5 Niinkuin helteen kuivassa maassa, niin sinä vaimensit muukalaisten melun. Niinkuin helle pilven varjossa vaipuu väkivaltaisten voittolaulu.
\par 6 Ja Herra Sebaot laittaa kaikille kansoille tällä vuorella pidot rasvasta, pidot voimaviinistä, ydinrasvasta, puhtaasta voimaviinistä.
\par 7 Ja hän hävittää tällä vuorella verhon, joka verhoaa kaikki kansat, ja peiton, joka peittää kaikki kansakunnat.
\par 8 Hän hävittää kuoleman ainiaaksi, ja Herra, Herra pyyhkii kyyneleet kaikkien kasvoilta ja ottaa pois kansansa häväistyksen kaikesta maasta. Sillä Herra on puhunut.
\par 9 Ja sinä päivänä sanotaan: "Katso, tämä on meidän Jumalamme, jota me odotimme meitä pelastamaan; tämä on Herra, jota me odotimme: iloitkaamme ja riemuitkaamme pelastuksesta, jonka hän toi".
\par 10 Sillä Herran käsi lepää tällä vuorella. Mutta Mooab tallataan siihen paikkaansa, niinkuin oljet tallautuvat lantaveteen.
\par 11 Ja hän haroo siinä käsiään, niinkuin uija haroo uidessansa, mutta Herra painaa alas hänen ylpeytensä ynnä hänen kättensä rimpuilut.
\par 12 Sinun korkeitten muuriesi varustukset hän kukistaa, painaa alas, syöksee maahan, tomuun asti.

\chapter{26}

\par 1 Sinä päivänä lauletaan Juudan maassa tämä laulu: "Meillä on vahva kaupunki, pelastuksen hän asettaa muuriksi ja varustukseksi.
\par 2 Avatkaa portit vanhurskaan kansan käydä sisälle, joka uskollisena pysyy.
\par 3 Vakaamieliselle sinä talletat rauhan, rauhan, sillä hän turvaa sinuun.
\par 4 Turvatkaa Herraan ainiaan, sillä Herra, Herra on iankaikkinen kallio.
\par 5 Sillä hän on kukistanut korkealla asuvaiset, ylhäisen kaupungin, hän painoi sen alas, painoi maan tasalle, syöksi sen tomuun asti.
\par 6 Sitä tallaa jalka, kurjan jalat, vaivaisten askeleet.
\par 7 Vanhurskaan polku on suora, sinä teet vanhurskaan tien tasaiseksi.
\par 8 Niin, sinun tuomioittesi tiellä me odotamme sinua, Herra; sinun nimeäsi ja sinun muistoasi sielu ikävöitsee.
\par 9 Minun sieluni ikävöitsee sinua yöllä, minun henkeni sisimmässäni etsii sinua varhain; sillä kun sinun tuomiosi kohtaavat maata, oppivat maanpiirin asukkaat vanhurskautta.
\par 10 Jos jumalaton saa armon, ei hän opi vanhurskautta; oikeuden maassa hän tekee vääryyttä eikä näe Herran korkeutta.
\par 11 Herra, sinun kätesi on kohotettu, mutta he eivät sitä näe. He saakoot häpeäksensä nähdä sinun kiivautesi kansan puolesta; kuluttakoon heidät tuli, joka sinun vihollisesi kuluttaa.
\par 12 Herra, sinä saatat meille rauhan, sillä myös kaikki meidän tekomme olet sinä tehnyt.
\par 13 Herra, meidän Jumalamme, meitä ovat vallinneet muut herrat, et sinä; sinua yksin me ylistämme, sinun nimeäsi.
\par 14 Kuolleet eivät virkoa eloon, vainajat eivät nouse: niin sinä olet heille kostanut, tuhonnut heidät ja hävittänyt kaiken heidän muistonsa.
\par 15 Sinä olet lisännyt kansan, Herra, olet lisännyt kansan, olet kunniasi näyttänyt, olet laajentanut kaikki maan rajat.
\par 16 Herra, ahdistuksessa he etsivät sinua, vuodattivat hiljaisia rukouksia, kun sinä heitä kuritit.
\par 17 Niinkuin raskas vaimo, joka on synnyttämäisillään, vääntelehtii ja huutaa kivuissansa, niin me olimme sinun edessäsi, Herra.
\par 18 Me olimme raskaina, vääntelehdimme, mutta oli niinkuin olisimme synnyttäneet tuulta: emme saaneet aikaan pelastusta maalle, maanpiirin asukkaat eivät ilmoille päässeet.
\par 19 Mutta sinun kuolleesi virkoavat eloon, minun ruumiini nousevat ylös. Herätkää ja riemuitkaa, te jotka tomussa lepäätte, sillä sinun kasteesi on valkeuksien kaste, ja maa tuo vainajat ilmoille.
\par 20 Mene, kansani, kammioihisi ja sulje ovet jälkeesi, lymyä hetkinen, kunnes viha on ohitse mennyt.
\par 21 Sillä katso, Herra lähtee asuinsijastaan kostamaan maan asukkaille heidän pahat tekonsa, ja maa paljastaa verivelkansa eikä surmattujansa enää peitä."

\chapter{27}

\par 1 Sinä päivänä Herra kostaa kovalla, suurella ja väkevällä miekallansa Leviatanille, kiitävälle käärmeelle, ja Leviatanille, kiemurtelevalle käärmeelle, ja tappaa lohikäärmeen, joka on meressä.
\par 2 Sinä päivänä sanotaan: "On viinitarha, tulisen viinin tarha; laulakaa siitä:
\par 3 'Minä, Herra, olen sen vartija, minä kastelen sitä hetkestä hetkeen; minä vartioitsen sitä öin ja päivin, ettei sitä mikään vahingoita.
\par 4 Vihaa minulla ei ole; olisipa vain orjantappuroita ja ohdakkeita, niiden kimppuun minä kävisin sodalla ja polttaisin ne kaikki tyynni -
\par 5 kaikki, jotka eivät antaudu minun turviini, eivät tee rauhaa minun kanssani, tee rauhaa minun kanssani.'"
\par 6 Tulevina aikoina juurtuu Jaakob, Israel kukkii ja kukoistaa ja täyttää maanpiirin hedelmällänsä.
\par 7 Löikö hän sitä, niinkuin sen lyöjät lyötiin, tapettiinko se, niinkuin sen tappajat tapettiin?
\par 8 Karkoittamalla sen, lähettämällä sen pois sinä sitä rankaisit. Hän pyyhkäisi sen pois kovalla myrskyllänsä itätuulen päivänä.
\par 9 Sentähden sovitetaan Jaakobin pahat teot sillä, ja se on hänen syntiensä poistamisen täysi hedelmä, että hän tekee kaikki alttarikivet rikottujen kalkkikivien kaltaisiksi: eivät kohoa enää asera-karsikot eivätkä auringonpatsaat.
\par 10 Sillä varustettu kaupunki on autio, se on hyljätty maja, jätetty tyhjäksi kuin erämaa. Siellä vasikat käyvät laitumella, siellä makailevat ja kaluavat sieltä vesat kaikki.
\par 11 Ja kun oksat kuivuvat, niin ne taitetaan; vaimot tulevat ja tekevät niillä tulta. Sillä se ei ole ymmärtäväistä kansaa; sentähden sen tekijä ei sitä armahda, sen Luoja ei sitä sääli.
\par 12 Sinä päivänä Herra karistaa hedelmät maahan, Eufrat-virrasta aina Egyptin puroon asti, ja teidät, te israelilaiset, poimitaan talteen yksitellen.
\par 13 Sinä päivänä puhalletaan suureen pasunaan, ja Assurin maahan hävinneet ja Egyptin maahan karkoitetut tulevat ja kumartavat Herraa Pyhällä vuorella Jerusalemissa.

\chapter{28}

\par 1 Voi Efraimin juopuneitten ylvästä kruunua ja sen kunnian loisteen kuihtuvata kukkaa, joka on kukkulan laella, viinistä päihtyneitten lihavan laakson keskellä!
\par 2 Katso, Herralta tulee hän, joka on väkevä ja voimallinen, niinkuin raesade, rajumyrsky. Niinkuin rankkasade, väkeväin tulvavetten kuohu, hän voimalla maahan kaataa.
\par 3 Jalkoihin tallataan Efraimin juopuneitten ylväs kruunu.
\par 4 Ja sen kunnian loisteen kuihtuvan kukan, joka on kukkulan laella, lihavan laakson keskellä, käy niinkuin varhaisviikunan ennen kesää: kuka vain sen näkee, tuskin se on hänen kourassaan, niin hän sen jo nielaisee.
\par 5 Sinä päivänä Herra Sebaot on oleva loistava kruunu ja kunnian seppele kansansa jäännökselle
\par 6 ja oikeuden henki sille, joka oikeutta istuu, ja väkevyys niille, jotka torjuvat hyökkäyksen takaisin porttia kohden.
\par 7 Ja nämäkin horjuvat viinistä ja hoipertelevat väkijuomasta. Väkijuomasta horjuu pappi ja profeetta; he ovat sekaisin viinistä, hoipertelevat väkijuomasta. He horjuvat näyissä, huojuvat tuomioissa.
\par 8 Sillä täynnä oksennusta ja saastaa ovat kaikki pöydät - ei puhdasta paikkaa!
\par 9 "Keitähän tuokin luulee taitoon neuvovansa, keitä saarnalla opettavansa? Olemmeko me vasta maidolta vieroitettuja, äidin rinnoilta otettuja?
\par 10 Käsky käskyn päälle, käsky käskyn päälle, läksy läksyn päälle, läksy läksyn päälle, milloin siellä, milloin täällä!" -
\par 11 Niin, sopertavin huulin ja vieraalla kielellä hän on puhuva tälle kansalle,
\par 12 hän, joka on sanonut heille: "Tässä on lepo; antakaa väsyneen levätä, tässä on levähdyspaikka". Mutta he eivät ole tahtoneet kuulla.
\par 13 Niinpä on Herran sana oleva heille: "Käsky käskyn päälle, käsky käskyn päälle, läksy läksyn päälle, läksy läksyn päälle, milloin siellä, milloin täällä", niin että he kulkiessaan kaatuvat selälleen ja ruhjoutuvat, että heidät kiedotaan ja vangitaan.
\par 14 Sentähden kuulkaa Herran sana, te pilkkaajat, te jotka hallitsette tätä kansaa Jerusalemissa.
\par 15 Koska te sanotte: "Me olemme tehneet liiton kuoleman kanssa ja tuonelan kanssa sopimuksen; tulkoon vitsaus kuin tulva, ei se meitä saavuta, sillä me olemme tehneet valheen turvaksemme ja piiloutuneet petokseen" -
\par 16 sentähden, näin sanoo Herra, Herra: Katso, minä lasken Siioniin peruskiven, koetellun kiven, kalliin kulmakiven, lujasti perustetun; joka uskoo, se ei pakene.
\par 17 Ja minä panen oikeuden mittanuoraksi ja vanhurskauden vaa'aksi, ja rakeet hävittävät valheturvan, ja vedet huuhtovat pois piilopaikan.
\par 18 Teidän liittonne kuoleman kanssa pyyhkäistään pois, ja teidän sopimuksenne tuonelan kanssa ei kestä; kun vitsaus tulee niinkuin tulva, niin se teidät maahan tallaa.
\par 19 Niin usein kuin se tulee, tempaa se teidät valtaansa, sillä aamu aamulta se tulee, tulee päivällä ja yöllä: totisesti, kauhuksi on tämän saarnan opetus.
\par 20 Sillä vuode on oleva liian lyhyt ojentautua ja peitto liian kaita kääriytyä.
\par 21 Sillä Herra nousee niinkuin Perasimin vuorella, hän kiivastuu niinkuin Gibeonin laaksossa tehdäkseen työnsä, oudon työnsä, ja toimittaakseen tekonsa, kumman tekonsa.
\par 22 Älkää siis pilkatko, etteivät teidän siteenne vielä kiristyisi; sillä Herralta Sebaotilta minä olen kuullut hävitys- ja tuomiopäätöksen, joka on kohtaava kaikkea maata.
\par 23 Kuunnelkaa ja kuulkaa minun ääntäni, tarkatkaa ja kuulkaa minun sanojani.
\par 24 Ainako kyntäjä vain kyntää, kun olisi kylväminen, ainako vakoaa ja äestää maatansa?
\par 25 Eikö niin: kun hän on tasoittanut sen pinnan, hän kylvää mustaa kuminaa, sirottelee höystekuminaa, panee nisunjyvät riviin, ohran omaan paikkaansa ja kolmitahkoista vehnää vierelle?
\par 26 Hänen Jumalansa on neuvonut hänelle oikean tavan ja opettaa häntä.
\par 27 Sillä ei mustaa kuminaa puida puimaäkeellä eikä puimajyrän tela pyöri höystekuminan yli, vaan musta kumina lyödään irti sauvalla ja höystekumina vitsalla.
\par 28 Puidaankos leipävilja murskaksi? Ei sitä kukaan iankaiken pui eikä aina aja sen päällitse puimajyrällään ja hevosillaan; ei sitä murskaksi puida.
\par 29 Tämäkin on tullut Herralta Sebaotilta; hänen neuvonsa on ihmeellinen ja ymmärryksensä ylen suuri.

\chapter{29}

\par 1 Voi Arielia, Arielia, kaupunkia, jonne Daavid asetti leirinsä! Liittäkää vuosi vuoteen, kiertäkööt juhlat kiertonsa,
\par 2 niin minä ahdistan Arielia, ja se on oleva täynnä valitusta ja vaikerrusta, oleva minulle kuin Jumalan uhriliesi.
\par 3 Minä asetan leirini yltympäri sinua vastaan ja saarran sinut vartiostoilla ja luon vallit sinua vastaan.
\par 4 Silloin sinä puhut maasta matalalta, sanasi tulevat vaimeina tomusta, sinä uikutat kuin vainajahenki maasta, puheesi tulee supisten tomusta.
\par 5 Mutta sinun vainolaistesi lauma on oleva niinkuin hieno pöly, ja väkivaltaisten lauma niinkuin lentävät akanat, ja tämä tapahtuu äkkiä, silmänräpäyksessä:
\par 6 Herra Sebaot etsiskelee sinua ukkosenjylinässä ja maanjäristyksessä, kovassa pauhinassa, myrskyssä ja rajuilmassa, kuluttavan tulen liekissä.
\par 7 Niinkuin yöllinen uninäky on oleva kansain lauma, joka sotii Arielia vastaan, kaikki, jotka sotivat sitä ja sen varustuksia vastaan ja sitä ahdistavat.
\par 8 Niinkuin nälkäinen on unissaan syövinänsä, mutta herää hiuka sydämessä, ja niinkuin janoinen on unissaan juovinansa, mutta herää, ja katso, hän on nääntynyt ja himoitsee juoda, niin on oleva kaikki kansain lauma, joka sotii Siionin vuorta vastaan.
\par 9 Hämmästykää ja ihmetelkää, tuijottakaa sokeiksi itsenne ja sokeiksi jääkää! He ovat juovuksissa, vaikkeivät viinistä, hoipertelevat, vaikkeivät väkijuomasta.
\par 10 Sillä Herra on vuodattanut teidän päällenne raskaan unen hengen ja sulkenut teidän silmänne - profeettanne, ja peittänyt teidän päänne - näkijänne.
\par 11 Niin on kaikki ilmoitus teille niinkuin lukitun kirjan sanat; jos se annetaan kirjantaitavalle ja sanotaan: "Lue tämä", niin hän vastaa: "Ei voi, sillä se on lukittu",
\par 12 ja jos kirja annetaan kirjantaitamattomalle ja sanotaan: "Lue tämä", niin hän vastaa: "En osaa lukea".
\par 13 Ja Herra sanoi: Koska tämä kansa lähestyy minua suullaan ja kunnioittaa minua huulillaan, mutta pitää sydämensä minusta kaukana, ja koska heidän jumalanpelkonsa on vain opittuja ihmiskäskyjä,
\par 14 sentähden, katso, minä vielä teen tälle kansalle kummia tekoja - kummia ja ihmeitä, ja sen viisaitten viisaus häviää, ja sen ymmärtäväisten ymmärrys katoaa.
\par 15 Voi niitä, jotka syvälle kätkevät hankkeensa Herralta, joiden teot tapahtuvat pimeässä ja jotka sanovat: "Kuka meitä näkee, kuka meistä tietää?"
\par 16 Voi mielettömyyttänne! Onko savi savenvalajan veroinen? Ja sanooko työ tekijästään: "Ei hän ole minua tehnyt", tai sanooko kuva kuvaajastaan: "Ei hän mitään ymmärrä"?
\par 17 Vain lyhyt hetki enää, niin Libanon muuttuu puutarhaksi, ja puutarha on metsän veroinen.
\par 18 Sinä päivänä kuurot kuulevat kirjan sanat, ja sokeiden silmät näkevät vapaina synkeästä pimeydestä.
\par 19 Nöyrät saavat yhä uutta iloa Herrassa, ja ihmisistä köyhimmätkin riemuitsevat Israelin Pyhästä.
\par 20 Sillä väkivaltaisista on tullut loppu, pilkkaajat ovat hävinneet, ja kaikki vääryyteen valppaat ovat tuhotut,
\par 21 ne, jotka sanallansa langettavat ihmisiä syyhyn ja virittävät pauloja sille, joka oikeutta puolustaa portissa, ja verukkeilla syyttömän asian vääräksi vääntävät.
\par 22 Sentähden Herra, joka Aabrahamin vapahti, sanoo Jaakobin heimolle näin: Ei Jaakob enää joudu häpeään, eivätkä hänen kasvonsa enää kalpene.
\par 23 Sillä kun hän näkee, kun hänen lapsensa näkevät keskellänsä minun kätteni työn, niin he pyhittävät minun nimeni, pitävät pyhänä Jaakobin Pyhän ja peljästyvät Israelin Jumalaa.
\par 24 Ja hengessään eksyväiset käsittävät ymmärryksen, ja napisevaiset ottavat oppia.

\chapter{30}

\par 1 Voi uppiniskaisia lapsia, sanoo Herra, jotka pitävät neuvoa, mikä ei ole minusta, ja hierovat liittoa ilman minun henkeäni, kooten syntiä synnin päälle;
\par 2 jotka menevät Egyptiin, kysymättä minulta, turvautuakseen faraon turviin ja etsiäkseen suojaa Egyptin varjossa!
\par 3 Faraon turva koituu teille häpeäksi ja suojan etsiminen Egyptin varjosta häväistykseksi.
\par 4 Sillä vaikka hänen ruhtinaansa ovat Sooanissa ja hänen sanansaattajansa saapuneet Haanekseen asti,
\par 5 joutuvat kaikki häpeään kansan tähden, josta heillä ei ole hyötyä - ei apua, ei hyötyä, vaan häpeätä ja pilkkaa.
\par 6 Ennustus Eteläisen maan Behemotia vastaan. Halki ahdingon ja ahdistuksen maan, halki naarasleijonan ja jalopeuran maan, kyykäärmeen ja lentävän käärmeen maan he kuljettavat aasinvarsojen selässä rikkautensa ja kamelien kyttyrällä aarteensa kansan tykö, josta ei hyötyä ole.
\par 7 Egyptin apu on turha ja tyhjä; sentähden minä annan sille nimen: "Rahab, joka ei pääse paikaltansa".
\par 8 Mene siis nyt ja kirjoita se tauluun heidän läsnäollessaan ja piirrä se kirjaan, että se säilyisi tuleviin aikoihin, ainiaan, iankaikkisesti.
\par 9 Sillä he ovat niskoitteleva kansa, ovat vilpillisiä lapsia, lapsia, jotka eivät tahdo kuulla Herran lakia,
\par 10 jotka sanovat näkijöille: "Älkää nähkö", ja ennustajille: "Älkää ennustako meille tosia, puhukaa meille mieluisia, ennustakaa silmänlumeita.
\par 11 Poiketkaa tieltä, väistykää polulta; viekää pois silmistämme Israelin Pyhä."
\par 12 Sentähden, näin sanoo Israelin Pyhä: Koska te halveksitte tätä sanaa ja luotatte väkivaltaan ja vääryyteen ja siihen turvaudutte,
\par 13 niin tämä synti on oleva kuin repeämä korkeassa muurissa, joka uhkaa sortua ja pullistuu; se särkyy äkkiä, yhtäkkiä.
\par 14 Se särkyy, niinkuin särkyy savenvalajan astia, joka lyödään rikki säälimättä, niin ettei sen sirpaleista löydy palasta, millä ottaa tulta liedestä tai ammentaa vettä altaasta.
\par 15 Sillä näin sanoo Herra, Herra, Israelin Pyhä: Kääntymällä ja pysymällä hiljaa te pelastutte, rauhallisuus ja luottamus on teidän väkevyytenne; mutta te ette tahtoneet,
\par 16 vaan sanoitte: "Ei! Hevosilla me tahdomme kiitää" - niinpä saatte kiitää pakoon. "Nopean selässä me tahdomme ratsastaa" - niinpä ovat vainoojanne nopeat.
\par 17 Tuhat pakenee yhden uhkaa; viiden uhkaa te pakenette, kunnes se, mikä teistä jää, on kuin merkkipuu vuoren huipulla, kuin lipputanko kukkulalla.
\par 18 Sentähden Herra odottaa, että voisi olla teille armollinen, sentähden hän nousee armahtaaksensa teitä; sillä Herra on oikeuden Jumala. Autuaita kaikki, jotka häntä odottavat!
\par 19 Sinä kansa, joka asut Siionissa, Jerusalemissa, älä itke! Hän on sinulle totisesti armollinen, kun apua huudat; sen kuullessaan hän vastaa sinulle kohta.
\par 20 Vaikka Herra antaa teille hädän leipää ja ahdistuksen vettä, niin ei sinun opettajasi sitten enää kätkeydy, vaan sinun silmäsi saavat nähdä sinun opettajasi.
\par 21 Ja sinun korvasi kuulevat takaasi tämän sanan, milloin poikkeatte oikealle tai vasemmalle: "Tässä on tie, sitä käykää".
\par 22 Silloin sinä havaitset saastaiseksi veistettyjen jumalankuviesi hopeapäällystyksen ja valettujen jumalankuviesi kultakuoren; sinä viskaat ne pois kuin inhotuksen, sinä sanot niille: "Ulos!"
\par 23 Niin hän antaa sateen sinun siemenellesi, jonka maahan kylvät, ja maan sadosta leivän, joka on oleva mehevä ja lihava, ja sinun karjasi on sinä päivänä käyvä laajalla laitumella.
\par 24 Härät ja aasinvarsat, jotka peltotyötä tekevät, syövät suolaista rehuviljaa, joka on viskattu hangolla ja viskimellä.
\par 25 Ja kaikilla korkeilla vuorilla ja kaikilla ylhäisillä kukkuloilla on oleva puroja, vesivirtoja, suuren surmaamisen päivänä, tornien sortuessa.
\par 26 Ja kuun valo on oleva kuin auringon valo, ja auringon valo on oleva seitsenkertainen, oleva niinkuin seitsemän päivän valo, sinä päivänä, jona Herra sitoo kansansa vammat ja parantaa siihen isketyt haavat.
\par 27 Katso, Herran nimi tulee kaukaa, hänen vihansa leimuaa, ja sankka savu tupruaa; hänen huulensa ovat täynnä hirmuisuutta, ja hänen kielensä on kuin kuluttava tuli.
\par 28 Hänen hengityksensä on kuin virta, joka tulvii ja ulottuu kaulaan asti; se seuloo kansakuntia turmion seulassa ja panee eksyttäväiset suitset kansojen suupieliin.
\par 29 Silloin te veisaatte niinkuin yöllä, kun pyhä juhla alkaa, ja sydämenne riemuitsee niinkuin sen, joka huilujen soidessa astuu vaeltaen Herran vuorelle, Israelin kallion tykö.
\par 30 Herra antaa kuulla äänensä voiman ja nähdä käsivartensa laskeutuvan alas vihan tuimuudessa ja kuluttavan tulen liekissä, pilvenpurkuna, rankkasateena ja raekivinä.
\par 31 Sillä Herran äänestä peljästyy Assur. Hän lyö vitsalla.
\par 32 Ja jokaisella sallimuksen sauvan iskulla, jonka Herra häneen satuttaa, soivat vaskirummut ja kanteleet, ja hän sotii häntä vastaan, sotii kättä heiluttaen.
\par 33 Sillä aikoja sitten on polttopaikka valmistettu; kuninkaallekin se on varattu. Syvä ja leveä on sen liesi, tulta ja polttopuita paljon. Herran henkäys kuin tulikivi-virta sytyttää sen.

\chapter{31}

\par 1 Voi niitä, jotka menevät alas Egyptiin apua etsimään ja turvautuvat hevosiin, luottavat sotavaunuihin, koska niitä on paljon, ja ratsumiehiin, koska niitten luku on ylen suuri, mutta eivät katso Israelin Pyhään, eivät kysy neuvoa Herralta.
\par 2 Mutta myös hän on viisas, ja hän tuottaa onnettomuuden; hän ei peruuta sanojansa, vaan nousee pahantekijäin sukua vastaan ja väärintekijäin apuuntuloa vastaan.
\par 3 Egypti on ihminen eikä Jumala, ja heidän hevosensa ovat lihaa eivätkä henkeä. Kun Herra ojentaa kätensä, suistuu auttaja, ja autettava kaatuu, ja yhdessä he kumpikin hukkuvat.
\par 4 Sillä näin on Herra minulle sanonut: Niinkuin leijona, nuori leijona, murisee saaliinsa ääressä, kun sitä vastaan hälytetään paimenten parvi, eikä peljästy heidän huutoansa, ei huoli heidän hälinästään, niin Herra Sebaot on astuva alas sotimaan Siionin vuorella ja sen kukkulalla.
\par 5 Niinkuin liitelevät linnut, niin varjelee Herra Sebaot Jerusalemia - varjelee ja pelastaa, säästää ja vapahtaa.
\par 6 Palatkaa, te israelilaiset, hänen tykönsä, josta olette niin kauas luopuneet.
\par 7 Sillä sinä päivänä jokainen hylkää hopeaiset ja kultaiset epäjumalansa, jotka kätenne ovat tehneet teille synniksi.
\par 8 Ja Assur kaatuu miekkaan, joka ei ole miehen, ja hänet syö miekka, joka ei ole ihmisen; hän pakenee miekkaa, ja hänen nuorukaisensa joutuvat työorjuuteen.
\par 9 Hänen kallionsa kukistuu kauhusta, ja hänen ruhtinaansa säikkyvät lipun luota pakoon, sanoo Herra, jonka tuli on Siionissa ja pätsi Jerusalemissa.

\chapter{32}

\par 1 Katso, kuningas on hallitseva vanhurskaudessa, ja valtiaat vallitsevat oikeuden mukaan.
\par 2 Silloin on jokainen heistä oleva turvana tuulelta ja suojana rankkasateelta, oleva kuin vesipurot kuivassa maassa, kuin korkean kallion varjo nääntyvässä maassa.
\par 3 Silloin eivät näkevien silmät ole soaistut, ja kuulevien korvat kuulevat tarkkaan.
\par 4 Ajattelemattomien sydän käsittää taidon, ja änkyttäväin kieli puhuu sujuvasti ja selkeästi.
\par 5 Ei houkkaa enää kutsuta jaloksi, eikä petollista enää sanota yleväksi.
\par 6 Sillä houkka puhuu houkan lailla, ja hänen sydämensä hankkii turmiota, ja niin hän harjoittaa riettautta ja puhuu eksyttäväisesti Herrasta, jättää tyhjäksi nälkäisen sielun ja janoavaisen juomaa vaille.
\par 7 Ja pahat ovat petollisen aseet, hän miettii ilkitöitä, tuhotakseen kurjat valheen sanoilla, vaikka köyhä kuinka oikeata asiaa puhuisi.
\par 8 Mutta jalo jaloja miettii ja jaloudessa lujana pysyy.
\par 9 Nouskaa, te suruttomat naiset, kuulkaa minun ääntäni; te huolettomat tyttäret, tarkatkaa minun sanojani.
\par 10 Vielä vuosi, ja päiviä päälle, niin vapisette, te huolettomat; sillä silloin on viininkorjuusta tullut loppu, hedelmänkorjuuta ei tule.
\par 11 Kauhistukaa, te suruttomat, vaviskaa, te huolettomat, riisuutukaa, paljastukaa, vyöttäkää säkki lanteillenne.
\par 12 Silloin valitetaan ja lyödään rintoihin ihanien peltojen tähden ja hedelmällisten viiniköynnösten tähden,
\par 13 minun kansani maan tähden, joka kasvaa orjantappuraa ja ohdaketta, remuavan kaupungin kaikkien iloisten talojen tähden.
\par 14 Sillä palatsi on hyljätty, kaupungin kohina lakannut, Oofel ja vartiotorni ovat jääneet luoliksi iankaikkisesti, villiaasien iloksi ja laumojen laitumeksi.
\par 15 Näin on hamaan siihen asti, kunnes meidän päällemme vuodatetaan Henki korkeudesta. Silloin erämaa muuttuu puutarhaksi,
\par 16 ja puutarha on metsän veroinen. Ja erämaassa asuu oikeus, ja puutarhassa majailee vanhurskaus.
\par 17 Silloin vanhurskauden hedelmä on rauha, vanhurskauden vaikutus lepo ja turvallisuus iankaikkisesti.
\par 18 Ja minun kansani asuu rauhan majoissa, turvallisissa asunnoissa, huolettomissa lepopaikoissa.
\par 19 Mutta raesade tulee, metsä kaatuu, ja kaupunki alennetaan alhaiseksi.
\par 20 Onnelliset te, jotka kylvätte kaikkien vetten vierille ja laskette härän ja aasin jalat valtoimina kulkemaan!

\chapter{33}

\par 1 Voi sinua hävittäjää, joka itse olet hävittämättä, sinua ryöstäjää, jota ei kenkään ole ryöstänyt! Kun olet loppuun asti hävittänyt, hävitetään sinut, kun olet ryöstösi ryöstänyt, ryöstetään sinut.
\par 2 Herra, armahda meitä, sinua me odotamme. Ole näitten käsivarsi joka aamu, ole meidän apumme hädän aikana.
\par 3 Kansat pakenevat sinun jylinäsi ääntä; kun sinä nouset, hajoavat kansakunnat.
\par 4 Teidän saaliinne viedään, niinkuin tuhosirkat vievät; niinkuin hyppysirkat hyökkäävät, niin sen kimppuun hyökätään.
\par 5 Herra on korkea, sillä hän asuu korkeudessa. Hän täyttää Siionin oikeudella ja vanhurskaudella.
\par 6 Ja hän on sinun aikojesi vakuus, avun runsaus, viisaus ja ymmärrys; Herran pelko on oleva Siionin aarre.
\par 7 Katso, sankarit huutavat ulkona, rauhan sanansaattajat itkevät katkerasti.
\par 8 Tiet ovat autioina, polulta on kulkija poissa: hän on rikkonut liiton, ylenkatsonut kaupungit, ei ole ihmistä minäkään pitänyt.
\par 9 Maa murehtii ja nääntyy, Libanon kuihtuu häpeästä, Saaron on aromaaksi tullut, Baasan ja Karmel varistavat lehtensä.
\par 10 Nyt minä nousen, sanoo Herra, nyt minä itseni korotan, nyt minä kohoan korkealle.
\par 11 Olkia te kannatte kohdussanne, akanoita synnytätte; teidän kiukkunne on tuli, joka kuluttaa teidät.
\par 12 Ja kansat poltetaan kuin kalkki, kuin katkotut orjantappurat, jotka tulessa palavat.
\par 13 Kuulkaa, te kaukaiset, mitä minä olen tehnyt; te lähelläolevat, tuntekaa minun voimani.
\par 14 Syntiset Siionissa peljästyvät, vavistus valtaa jumalattomat: "Kuka meistä voi asua kuluttavassa tulessa, kuka asua iankaikkisessa hehkussa?"
\par 15 Joka vanhurskaudessa vaeltaa ja puhuu sitä, mikä oikein on, joka halveksii väärää voittoa, jonka käsi torjuu lahjukset luotaan, joka tukkii korvansa kuulemasta veritöitä ja sulkee silmänsä näkemästä pahaa,
\par 16 hän on asuva kukkuloilla, kalliolinnat ovat hänen turvansa; hänelle annetaan hänen leipänsä, eikä vesi häneltä ehdy.
\par 17 Sinun silmäsi saavat katsoa kuningasta hänen ihanuudessaan, saavat nähdä avaran maan.
\par 18 Sinun sydämesi muistelee kauhuja: missä on nyt veronlaskija, missä punnitsija, missä tornien lukija?
\par 19 Et näe enää sitä röyhkeätä kansaa, kansaa, jolla on outo, käsittämätön puhe, jonka sopertavaa kieltä ei kukaan ymmärrä.
\par 20 Katso Siionia, juhliemme kaupunkia. Sinun silmäsi näkevät Jerusalemin, rauhaisan asuinsijan, telttamajan, jota ei muuteta, jonka vaarnoja ei ikinä reväistä irti, jonka köysistä ei yhtäkään katkaista.
\par 21 Sillä voimallinen on meillä siellä Herra, siellä on joet, on virrat, leveät rannasta toiseen, joita ei kulje soutualus, joiden poikki ei pääse uljas laiva.
\par 22 Sillä Herra on meidän tuomarimme, Herra on johdattajamme, Herra on meidän kuninkaamme; hän pelastaa meidät.
\par 23 Nyt ovat köytesi höltyneet, eivät pidä mastoa kannassaan kiinni, eivät vedä lippua liehumaan. Mutta silloin jaetaan riistosaalista runsaasti, rammatkin ryöstävät ja raastavat.
\par 24 Eikä yksikään asukas sano: "Minä olen vaivanalainen". Kansa, joka siellä asuu, on saanut syntinsä anteeksi.

\chapter{34}

\par 1 Astukaa esiin, te kansat, ja kuulkaa; te kansakunnat, tarkatkaa. Kuulkoon maa ja kaikki, mitä siinä on, maanpiiri ja kaikki, mikä siitä kasvaa.
\par 2 Sillä Herra on vihastunut kaikkiin kansoihin ja kiivastunut kaikkiin heidän joukkoihinsa; hän on vihkinyt heidät tuhon omiksi, jättänyt heidät teurastettaviksi.
\par 3 Heidän surmattunsa viskataan pois, ja niitten raadoista nousee löyhkä, ja vuoret valuvat heidän vertansa.
\par 4 Kaikki taivaan joukot menehtyvät, taivas kääritään kokoon niinkuin kirja, ja kaikki sen joukot varisevat alas, niinkuin lehdet varisevat viinipuusta, niinkuin viikunapuusta raakaleet.
\par 5 Sillä minun miekkani taivaassa on juopunut vimmaan; katso, se iskee alas Edomiin, tuomioksi kansalle, jonka minä olen vihkinyt tuhoon.
\par 6 Herran miekka on verta täynnä, rasvaa tiukkuva, karitsain ja kauristen verta, oinasten munuaisrasvaa. Sillä Herralla on uhri Bosrassa, suuri teurastus Edomin maassa.
\par 7 Villihärkiä kaatuu yhteen joukkoon, mullikoita härkien mukana. Heidän maansa juopuu verestä, ja heidän multansa tiukkuu rasvaa.
\par 8 Sillä Herralla on koston päivä, maksun vuosi Siionin asian puolesta.
\par 9 Edomin purot muuttuvat pieksi ja sen multa tulikiveksi; sen maa tulee palavaksi pieksi.
\par 10 Ei sammu se yöllä eikä päivällä, iäti nousee siitä savu; se on oleva raunioina polvesta polveen, ei kulje siellä kukaan, iankaikkisesta iankaikkiseen.
\par 11 Sen perivät pelikaanit ja tuonenkurjet, kissapöllöt ja kaarneet asuvat siellä; ja hän vetää sen ylitse autiuden mittanuoran ja tyhjyyden luotilangan.
\par 12 Ei ole siellä enää ylimyksiä huutamassa ketään kuninkaaksi, kaikki sen ruhtinaat ovat poissa.
\par 13 Ja sen palatsit kasvavat orjantappuroita, sen linnat polttiaisia ja ohdakkeita; siitä tulee aavikkosutten asunto, kamelikurkien tyyssija.
\par 14 Siellä erämaan ulvojat ja ulisijat yhtyvät, metsänpeikot toisiansa tapaavat. Siellä yksin öinen syöjätär saa rauhan ja löytää lepopaikan.
\par 15 Siellä nuolikäärme pesii ja laskee munansa, kuorii ne ja kiertyy kerälle pimentoonsa. Sinne haarahaukatkin kokoontuvat yhteen.
\par 16 Etsikää Herran kirjasta ja lukekaa: ei yhtäkään näistä ole puuttuva, ei yksikään toistansa kaipaava. - "Sillä minun suuni on niin käskenyt." - Hänen henkensä on ne yhteen koonnut.
\par 17 Hän on heittänyt arpaa niitten kesken, ja hänen kätensä on sen niille mittanuoralla jakanut; ne perivät sen ikiajoiksi, asuvat siellä polvesta polveen.

\chapter{35}

\par 1 Erämaa ja hietikko iloitsee, aromaa riemuitsee ja kukoistaa kuin lilja.
\par 2 Se kauniisti kukoistaa ja iloitsee ilolla ja riemulla. Sille annetaan Libanonin kunnia, Karmelin ja Saaronin ihanuus. He saavat nähdä Herran kunnian, meidän Jumalamme ihanuuden.
\par 3 Vahvistakaa hervonneet kädet, voimistakaa horjuvat polvet.
\par 4 Sanokaa hätääntyneille sydämille: "Olkaa lujat, älkää peljätkö. Katso, teidän Jumalanne! Kosto tulee, Jumalan rangaistus. Hän tulee ja pelastaa teidät."
\par 5 Silloin avautuvat sokeain silmät ja kuurojen korvat aukenevat.
\par 6 Silloin rampa hyppii niinkuin peura ja mykän kieli riemuun ratkeaa; sillä vedet puhkeavat erämaahan ja aromaahan purot.
\par 7 Hehkuva hiekka tulee lammikoiksi ja kuiva maa vesilähteiksi. Aavikkosutten asunnossa, missä ne makasivat, kasvaa ruoho ynnä ruoko ja kaisla.
\par 8 Ja siellä on oleva valtatie, ja sen nimi on "pyhä tie": sitä ei kulje saastainen; se on heitä itseänsä varten. Joka sitä tietä kulkee, ei eksy - eivät hullutkaan.
\par 9 Ei ole siellä leijonaa, ei nouse sinne raateleva peto; ei sellaista siellä tavata: lunastetut sitä kulkevat.
\par 10 Niin Herran vapahdetut palajavat ja tulevat Siioniin riemuiten, päänsä päällä iankaikkinen ilo. Riemu ja ilo saavuttavat heidät, mutta murhe ja huokaus pakenevat.

\chapter{36}

\par 1 Kuningas Hiskian neljäntenätoista hallitusvuotena hyökkäsi Sanherib, Assurin kuningas, kaikkien Juudan varustettujen kaupunkien kimppuun ja valloitti ne.
\par 2 Ja Assurin kuningas lähetti Laakiista Rabsaken suuren sotajoukon kanssa kuningas Hiskiaa vastaan Jerusalemiin, ja hän pysähtyi Ylälammikon vesijohdolle, joka on Vanuttajankedon tien varrella.
\par 3 Ja Eljakim, Hilkian poika, joka oli palatsin päällikkönä, ja kirjuri Sebna ja kansleri Jooah, Aasafin poika, menivät hänen luoksensa.
\par 4 Ja Rabsake sanoi heille: "Sanokaa Hiskialle: Näin sanoo suurkuningas, Assurin kuningas: 'Mitä on tuo luottamus, mikä sinulla on?
\par 5 Minä sanon: pelkkää huulten puhetta on moinen neuvo ja voima sodankäyntiin. Keneen sinä oikein luotat, kun kapinoit minua vastaan?
\par 6 Katso, sinä luotat Egyptiin, tuohon särkyneeseen ruokosauvaan, joka tunkeutuu sen käteen, joka siihen nojaa, ja lävistää sen. Sellainen on farao, Egyptin kuningas, kaikille, jotka häneen luottavat.
\par 7 Vai sanotko ehkä minulle: Me luotamme Herraan, meidän Jumalaamme? Mutta eikö hän ole se, jonka uhrikukkulat ja alttarit Hiskia poisti, kun hän sanoi Juudalle ja Jerusalemille: Tämän alttarin edessä on teidän kumartaen rukoiltava?'
\par 8 Mutta lyö nyt vetoa minun herrani, Assurin kuninkaan, kanssa: minä annan sinulle kaksi tuhatta hevosta, jos sinä voit hankkia niille ratsastajat.
\par 9 Kuinka sinä sitten voisit torjua ainoankaan käskynhaltijan, ainoankaan minun herrani vähimmän palvelijan, hyökkäyksen? Ja sinä vain luotat Egyptiin, sen vaunuihin ja ratsumiehiin.
\par 10 Olenko minä siis Herran sallimatta hyökännyt tähän maahan hävittämään sitä? Herra itse on sanonut minulle: 'Hyökkää tähän maahan ja hävitä se'."
\par 11 Niin Eljakim, Sebna ja Jooah sanoivat Rabsakelle: "Puhu palvelijoillesi araminkieltä, sillä me ymmärrämme sitä; älä puhu meille juudankieltä kansan kuullen, jota on muurilla".
\par 12 Mutta Rabsake vastasi: "Onko minun herrani lähettänyt minut puhumaan näitä sanoja sinun herrallesi ja sinulle? Eikö juuri niille miehille, jotka istuvat muurilla ja joutuvat teidän kanssanne syömään omaa likaansa ja juomaan omaa vettänsä?"
\par 13 Sitten Rabsake astui esiin, huusi kovalla äänellä juudankielellä ja sanoi: "Kuulkaa suurkuninkaan, Assurin kuninkaan, sanoja.
\par 14 Näin sanoo kuningas: 'Älkää antako Hiskian pettää itseänne, sillä hän ei voi teitä pelastaa.
\par 15 Älköön Hiskia saako teitä luottamaan Herraan, kun hän sanoo: Herra on varmasti pelastava meidät; ei tätä kaupunkia anneta Assurin kuninkaan käsiin.
\par 16 Älkää kuulko Hiskiaa.' Sillä Assurin kuningas sanoo näin: 'Tehkää sovinto minun kanssani ja antautukaa minulle, niin saatte syödä kukin viinipuustanne ja viikunapuustanne ja juoda kukin kaivostanne,
\par 17 kunnes minä tulen ja vien teidät maahan, joka on teidän maanne kaltainen, vilja- ja viinimaahan, leivän ja viinitarhojen maahan.
\par 18 Älköön vain Hiskia saako vietellä teitä, sanoessaan: Herra pelastaa meidät. Onko muidenkaan kansojen jumalista kukaan pelastanut maatansa Assurin kuninkaan käsistä?
\par 19 Missä ovat Hamatin ja Arpadin jumalat? Missä ovat Sefarvaimin jumalat? Ovatko ne pelastaneet Samariaa minun käsistäni?
\par 20 Kuka näiden maiden kaikista jumalista on pelastanut maansa minun käsistäni? Kuinka sitten Herra pelastaisi Jerusalemin minun käsistäni?'"
\par 21 Mutta he olivat vaiti eivätkä vastanneet hänelle mitään, sillä kuningas oli käskenyt niin ja sanonut: "Älkää vastatko hänelle".
\par 22 Sitten palatsin päällikkö Eljakim, Hilkian poika, ja kirjuri Sebna ja kansleri Jooah, Aasafin poika, tulivat Hiskian luo vaatteet reväistyinä ja kertoivat hänelle, mitä Rabsake oli sanonut.

\chapter{37}

\par 1 Kun kuningas Hiskia sen kuuli, repäisi hän vaatteensa, pukeutui säkkiin ja meni Herran temppeliin.
\par 2 Ja hän lähetti palatsin päällikön Eljakimin ja kirjuri Sebnan sekä pappein vanhimmat, säkkeihin puettuina, profeetta Jesajan, Aamoksen pojan, tykö.
\par 3 Ja he sanoivat hänelle: "Näin sanoo Hiskia: 'Hädän, kurituksen ja häväistyksen päivä on tämä päivä, sillä lapset ovat tulleet kohdun suulle saakka, mutta ei ole voimaa synnyttää.
\par 4 Ehkä Herra, sinun Jumalasi, kuulee Rabsaken sanat, joilla hänen Herransa, Assurin kuningas, on lähettänyt hänet herjaamaan elävää Jumalaa, ja rankaisee häntä näistä sanoista, jotka Herra, sinun Jumalasi, on kuullut. Niin kohota nyt rukous jäännöksen puolesta, joka vielä on olemassa.'"
\par 5 Kun kuningas Hiskian palvelijat tulivat Jesajan tykö,
\par 6 sanoi Jesaja heille: "Sanokaa näin herrallenne: 'Näin sanoo Herra: Älä pelkää niitä sanoja, jotka olet kuullut ja joilla Assurin kuninkaan poikaset ovat häväisseet minua.
\par 7 Katso, minä annan häneen mennä sellaisen hengen, että hän kuultuaan sanoman palajaa omaan maahansa; ja minä annan hänen kaatua miekkaan omassa maassansa.'"
\par 8 Ja Rabsake kääntyi takaisin ja tapasi Assurin kuninkaan sotimassa Libnaa vastaan; sillä hän oli kuullut, että tämä oli lähtenyt Laakiista pois.
\par 9 Mutta kun Sanherib kuuli Tirhakasta, Etiopian kuninkaasta, sanottavan: "Hän on lähtenyt liikkeelle sotiakseen sinua vastaan", niin hän sen kuultuaan lähetti sanansaattajat Hiskian tykö ja käski sanoa:
\par 10 "Sanokaa näin Hiskialle, Juudan kuninkaalle: 'Älä anna Jumalasi, johon sinä luotat, pettää itseäsi äläkä ajattele: Jerusalem ei joudu Assurin kuninkaan käsiin.
\par 11 Olethan kuullut, mitä Assurin kuninkaat ovat tehneet kaikille maille, kuinka he ovat vihkineet ne tuhon omiksi. Ja sinäkö pelastuisit!
\par 12 Ovatko kansain jumalat pelastaneet niitä, jotka minun isäni ovat tuhonneet: Goosania, Harrania, Resefiä ja Telassarin edeniläisiä?
\par 13 Missä on Hamatin kuningas ja Arpadin kuningas, Sefarvaimin kaupungin, Heenan ja Ivvan kuningas?'"
\par 14 Kun Hiskia oli ottanut kirjeen sanansaattajilta ja lukenut sen, meni hän Herran temppeliin; ja Hiskia levitti sen Herran eteen.
\par 15 Ja Hiskia rukoili Herraa ja sanoi:
\par 16 "Herra Sebaot, Israelin Jumala, jonka valtaistuinta kerubit kannattavat, sinä yksin olet maan kaikkien valtakuntain Jumala; sinä olet tehnyt taivaan ja maan.
\par 17 Herra, kallista korvasi ja kuule; Herra, avaa silmäsi ja katso. Kuule kaikki Sanheribin sanat, jotka hän lähetti herjatakseen elävää Jumalaa.
\par 18 Se on totta, Herra, että Assurin kuninkaat ovat hävittäneet kaikki maat ja omankin maansa.
\par 19 Ja he ovat heittäneet niiden jumalat tuleen; sillä ne eivät olleet jumalia, vaan ihmiskätten tekoa, puuta ja kiveä, ja sentähden he voivat hävittää ne.
\par 20 Mutta pelasta nyt meidät, Herra, meidän Jumalamme, hänen käsistänsä, että kaikki maan valtakunnat tulisivat tietämään, että sinä, Herra, olet ainoa."
\par 21 Niin Jesaja, Aamoksen poika, lähetti Hiskialle tämän sanan: "Näin sanoo Herra, Israelin Jumala: Koska sinä olet rukoillut minua avuksi Sanheribia, Assurin kuningasta, vastaan,
\par 22 niin tämä on se sana, jonka Herra on puhunut hänestä: Neitsyt, tytär Siion, halveksii ja pilkkaa sinua; tytär Jerusalem nyökyttää ilkkuen päätänsä sinun jälkeesi.
\par 23 Ketä olet herjannut ja häväissyt, ja ketä vastaan olet korottanut äänesi? Korkealle olet kohottanut silmäsi Israelin Pyhää vastaan.
\par 24 Palvelijaisi kautta sinä herjasit Herraa ja sanoit: 'Monilla vaunuillani minä nousin vuorten harjalle, Libanonin ääriin saakka; minä hakkasin maahan sen korkeat setrit, sen parhaat kypressit ja tunkeuduin sen etäisimmälle harjalle, sen rehevimpään metsään;
\par 25 minä kaivoin kaivoja ja join kuiviin vedet, ja jalkapohjallani minä kuivasin kaikki Egyptin virrat'.
\par 26 Etkö ole kuullut: kauan sitten minä olen tätä valmistanut, muinaisuudesta saakka tätä aivoitellut! Nyt minä olen sen toteuttanut, ja niin sinä sait hävittää varustetut kaupungit autioiksi kiviroukkioiksi,
\par 27 ja niiden asukkaat olivat voimattomat, he kauhistuivat ja joutuivat häpeään; heidän kävi niinkuin kedon ruohon ja niinkuin vihannan heinän, niinkuin katolla kasvavain kortten ja niinkuin laihon ennen oljelle tulemistaan.
\par 28 Istuitpa sinä tai lähdit tai tulit, minä sen tiedän, niinkuin senkin, että sinä raivoat minua vastaan.
\par 29 Koska sinä minua vastaan raivoat ja koska sinun ylpeytesi on tullut minun korviini, niin minä panen koukkuni sinun nenääsi ja suitseni sinun suuhusi ja vien sinut takaisin samaa tietä, jota tulitkin.
\par 30 Ja tämä on oleva sinulle merkkinä: tänä vuonna syödään jälkikasvua ja toisena vuonna kesanto-aaluvaa, mutta kolmantena vuonna te kylväkää ja leikatkaa, istuttakaa viinitarhoja ja syökää niiden hedelmää.
\par 31 Ja Juudan heimon pelastuneet, jotka ovat jäljelle jääneet, tekevät taas juurta alaspäin ja hedelmää ylöspäin.
\par 32 Sillä Jerusalemista lähtee kasvamaan jäännös, pelastunut joukko Siionin vuorelta. Herran Sebaotin kiivaus on sen tekevä.
\par 33 Sentähden, näin sanoo Herra Assurin kuninkaasta: Hän ei tule tähän kaupunkiin eikä siihen nuolta ammu, ei tuo sitä vastaan kilpeä eikä luo sitä vastaan vallia.
\par 34 Samaa tietä, jota hän tuli, hän palajaa, ja tähän kaupunkiin hän ei tule, sanoo Herra.
\par 35 Sillä minä varjelen tämän kaupungin ja pelastan sen itseni tähden ja palvelijani Daavidin tähden."
\par 36 Niin Herran enkeli lähti ja löi Assurin leirissä sata kahdeksankymmentä viisi tuhatta miestä, ja kun noustiin aamulla varhain, niin katso, he olivat kaikki kuolleina ruumiina.
\par 37 Silloin Sanherib, Assurin kuningas, lähti liikkeelle ja meni pois; hän palasi maahansa ja jäi Niiniveen.
\par 38 Mutta kun hän oli kerran rukoilemassa jumalansa Nisrokin temppelissä, surmasivat hänen poikansa Adrammelek ja Sareser hänet miekalla; sitten he pakenivat Araratin maahan. Ja hänen poikansa Eesarhaddon tuli kuninkaaksi hänen sijaansa.

\chapter{38}

\par 1 Niihin aikoihin sairastui Hiskia ja oli kuolemaisillaan; ja profeetta Jesaja, Aamoksen poika, tuli hänen tykönsä ja sanoi hänelle: "Näin sanoo Herra: Toimita talosi; sillä sinä kuolet etkä enää parane".
\par 2 Niin Hiskia käänsi kasvonsa seinään päin ja rukoili Herraa
\par 3 ja sanoi: "Oi Herra, muista, kuinka minä olen vaeltanut sinun edessäsi uskollisesti ja ehyellä sydämellä ja tehnyt sitä, mikä on hyvää sinun silmissäsi!" Ja Hiskia itki katkerasti.
\par 4 Mutta Jesajalle tuli tämä Herran sana:
\par 5 "Mene ja sano Hiskialle: 'Näin sanoo Herra, sinun isäsi Daavidin Jumala: Minä olen kuullut sinun rukouksesi, olen nähnyt sinun kyyneleesi. Katso, minä lisään sinulle ikää viisitoista vuotta;
\par 6 ja minä pelastan sinut ja tämän kaupungin Assurin kuninkaan käsistä ja varjelen tätä kaupunkia.
\par 7 Ja tämä on oleva sinulle merkkinä Herralta siitä, että Herra tekee, mitä on sanonut:
\par 8 katso, minä annan aurinkokellon varjon siirtyä takaisin kymmenen astetta, jotka se jo on auringon mukana laskeutunut Aahaan aurinkokellossa.'" Ja aurinko siirtyi aurinkokellossa takaisin kymmenen astetta, jotka se jo oli laskeutunut.
\par 9 Hiskian, Juudan kuninkaan, kirjoittama laulu, kun hän oli ollut sairaana ja toipunut taudistansa:
\par 10 "Minä sanoin: Kesken rauhallisten päivieni minun on mentävä tuonelan porteista; jäljellä olevat vuoteni on minulta riistetty pois.
\par 11 Minä sanoin: En saa minä enää nähdä Herraa, Herraa elävien maassa, en enää ihmisiä katsella manalan asukasten joukossa.
\par 12 Minun majani puretaan ja viedään minulta pois niinkuin paimenen teltta; olen kutonut loppuun elämäni, niinkuin kankuri kankaansa, minut leikataan irti loimentutkaimista. Ennenkuin päivä yöksi muuttuu, sinä teet minusta lopun.
\par 13 Minä viihdyttelin itseäni aamuun asti - niinkuin leijona hän murskaa kaikki minun luuni; ennenkuin päivä yöksi muuttuu, sinä teet minusta lopun.
\par 14 Niinkuin pääskynen, niinkuin kurki minä kuikutan, minä kujerran kuin kyyhkynen; hiueten katsovat minun silmäni korkeuteen: Herra, minulla on ahdistus, puolusta minua.
\par 15 Mitä nyt sanonkaan? Hän lupasi minulle ja täytti myös: minä vaellan hiljaisesti kaikki elämäni vuodet sieluni murheen tähden.
\par 16 Herra, tämänkaltaiset ovat elämäksi, ja niissä on koko minun henkeni elämä. Sinä teet minut terveeksi; anna minun elää.
\par 17 Katso, onneksi muuttui minulle katkera murhe: sinä rakastit minun sieluani, nostit sen kadotuksen kuopasta, sillä sinä heitit kaikki minun syntini selkäsi taa.
\par 18 Sillä ei tuonela sinua kiitä, ei kuolema sinua ylistä; eivät hautaan vaipuneet pane sinun totuuteesi toivoansa.
\par 19 Elävät, elävät sinua kiittävät, niinkuin minä tänä päivänä; isä ilmoittaa lapsillensa sinun totuutesi.
\par 20 Herra on minun auttajani. Minun kanteleeni soittoja soittakaamme kaikkina elinpäivinämme Herran temppelissä."
\par 21 Ja Jesaja käski tuoda viikunakakkua ja hautoa paisetta, että hän tulisi terveeksi.
\par 22 Niin Hiskia sanoi: "Mikä on merkkinä siitä, että minä voin mennä Herran temppeliin?"

\chapter{39}

\par 1 Siihen aikaan Merodak-Baladan, Baladanin poika, Baabelin kuningas, lähetti kirjeen ja lahjoja Hiskialle, kun oli kuullut hänen olleen sairaana ja parantuneen.
\par 2 Ja Hiskia iloitsi lähettiläistä ja näytti heille varastohuoneensa, hopean ja kullan, hajuaineet ja kalliin öljyn ja koko asehuoneensa ja kaikki, mitä hänen aarrekammioissansa oli. Ei ollut mitään Hiskian talossa eikä koko hänen valtakunnassaan, mitä hän ei olisi heille näyttänyt.
\par 3 Mutta profeetta Jesaja tuli kuningas Hiskian tykö ja sanoi hänelle: "Mitä nämä miehet ovat sanoneet, ja mistä he ovat tulleet sinun tykösi?" Hiskia vastasi: "He ovat tulleet minun tyköni kaukaisesta maasta, Baabelista".
\par 4 Hän sanoi: "Mitä he ovat nähneet sinun talossasi?" Hiskia vastasi: "Kaiken, mitä talossani on, he ovat nähneet; aarrekammiossani ei ole mitään, mitä en olisi heille näyttänyt".
\par 5 Niin Jesaja sanoi Hiskialle: "Kuule Herran Sebaotin sana:
\par 6 Katso, päivät tulevat, jolloin kaikki, mitä sinun talossasi on ja mitä sinun isäsi ovat koonneet tähän päivään asti, viedään pois Baabeliin; ei mitään jää jäljelle, sanoo Herra.
\par 7 Ja sinun omia poikiasi, jotka sinusta polveutuvat, jotka sinulle syntyvät, viedään hovipalvelijoiksi Baabelin kuninkaan palatsiin."
\par 8 Hiskia sanoi Jesajalle: "Herran sana, jonka olet puhunut, on hyvä". Sillä hän ajatteli: "Onpahan rauha ja turvallisuus minun päivinäni".

\chapter{40}

\par 1 "Lohduttakaa, lohduttakaa minun kansaani", sanoo teidän Jumalanne.
\par 2 "Puhukaa suloisesti Jerusalemille ja julistakaa sille, että sen vaivanaika on päättynyt, että sen velka on sovitettu, sillä se on saanut Herran kädestä kaksinkertaisesti kaikista synneistänsä."
\par 3 Huutavan ääni kuuluu: "Valmistakaa Herralle tie erämaahan, tehkää arolle tasaiset polut meidän Jumalallemme.
\par 4 Kaikki laaksot korotettakoon, kaikki vuoret ja kukkulat alennettakoon; koleikot tulkoot tasangoksi ja kalliolouhut lakeaksi maaksi.
\par 5 Herran kunnia ilmestyy: kaikki liha saa sen nähdä. Sillä Herran suu on puhunut."
\par 6 Ääni sanoo: "Julista!" Toinen vastaa: "Mitä minun pitää julistaman?" Kaikki liha on kuin ruoho, ja kaikki sen kauneus kuin kedon kukkanen:
\par 7 ruoho kuivuu, kukkanen lakastuu, kun Herran henkäys puhaltaa siihen. Totisesti, ruohoa on kansa.
\par 8 Ruoho kuivuu, kukkanen lakastuu, mutta meidän Jumalamme sana pysyy iankaikkisesti.
\par 9 Nouse korkealle vuorelle, Siion, sinä ilosanoman tuoja; korota voimakkaasti äänesi, Jerusalem, sinä ilosanoman tuoja. Korota, älä pelkää, sano Juudan kaupungeille: "Katso, teidän Jumalanne!"
\par 10 Katso, Herra, Herra tulee voimallisena, hänen käsivartensa vallitsee. Katso, hänen palkkansa on hänen mukanansa, hänen työnsä ansio käy hänen edellänsä.
\par 11 Niinkuin paimen hän kaitsee laumaansa, kokoaa karitsat käsivarrellensa ja kantaa niitä sylissään, johdattelee imettäviä lampaita.
\par 12 Kuka on kourallaan mitannut vedet ja vaaksalla määrännyt taivaitten mitat? Kuka on kolmannesmittaan mahduttanut maan tomun, puntarilla punninnut vuoret, vaa'alla kukkulat?
\par 13 Kuka on Herran Henkeä ohjannut, ollut hänen neuvonantajansa ja opettajansa?
\par 14 Kenen kanssa hän on neuvotellut, joka olisi hänelle ymmärrystä antanut ja opettanut oikean polun, opettanut hänelle tiedon ja osoittanut hänelle ymmärryksen tien?
\par 15 Katso, kansakunnat ovat kuin pisara vesisangon uurteessa, ovat kuin tomuhiukkanen vaa'assa. Katso, merensaaret hän nostaa kuin hiekkajyvän.
\par 16 Ei Libanon riittäisi polttopuiksi eikä sen riista polttouhriksi.
\par 17 Kaikki kansakunnat ovat niinkuin ei mitään hänen edessään, ne ovat hänelle kuin olematon ja tyhjä.
\par 18 Keneenkä siis te vertaatte Jumalan, ja minkä muotoiseksi te hänet teette?
\par 19 Jumalankuvanko? - Sen valaa valaja, ja kultaseppä kullalla päällystää, sepittää sille hopeaketjut.
\par 20 Kenellä ei ole varaa sellaiseen antimeen, se valitsee puun, joka ei lahoa, hakee taitavan tekijän pystyttämään jumalankuvan, joka ei horju.
\par 21 Ettekö te tiedä, ettekö kuule, eikö teille ole alusta asti ilmoitettu, ettekö ole maan perustuksista vaaria ottaneet?
\par 22 Hän istuu korkealla maanpiirin päällä, kuin heinäsirkkoja ovat sen asukkaat; hän levittää taivaan niinkuin harson, pingoittaa sen niinkuin teltan asuttavaksi.
\par 23 Hän tekee ruhtinaat olemattomiksi, saattaa maan tuomarit tyhjän veroisiksi.
\par 24 Tuskin he ovat istutetut, tuskin kylvetyt, tuskin on heidän vartensa juurtunut maahan, niin hän puhaltaa heihin, ja he kuivettuvat; myrsky vie heidät kuin oljenkorret.
\par 25 "Keneenkä siis te vertaatte minut, jonka kaltainen minä olisin", sanoo Pyhä.
\par 26 Nostakaa silmänne korkeuteen ja katsokaa: kuka on nämä luonut? Hän, joka johdattaa esiin niitten joukot täysilukuisina, joka nimeltä kutsuu ne kaikki; suuri on hänen voimansa ja valtainen hänen väkensä: ei yksikään jää häneltä pois.
\par 27 Miksi sinä, Jaakob, sanot ja sinä, Israel, puhut: "Minun tieni on Herralta salassa, minun oikeuteni on joutunut pois minun Jumalani huomasta"?
\par 28 Etkö tiedä, etkö ole kuullut: Herra on iankaikkinen Jumala, joka on luonut maan ääret? Ei hän väsy eikä näänny, hänen ymmärryksensä on tutkimaton.
\par 29 Hän antaa väsyneelle väkeä ja voimattomalle voimaa yltäkyllin.
\par 30 Nuorukaiset väsyvät ja nääntyvät, nuoret miehet kompastuvat ja kaatuvat;
\par 31 mutta ne, jotka Herraa odottavat, saavat uuden voiman, he kohottavat siipensä kuin kotkat. He juoksevat eivätkä näänny, he vaeltavat eivätkä väsy.

\chapter{41}

\par 1 Vaietkaa minun edessäni, te merensaaret. Kansat verestäkööt voimansa, astukoot esiin ja puhukoot sitten; käykäämme oikeutta keskenämme.
\par 2 Kuka herätti päivänkoiton maasta hänet, jota vanhurskaus seuraa joka askeleella? Kuka antaa kansat hänen valtaansa, kukistaa kuninkaat hänen jalkoihinsa? Kuka muuttaa heidän miekkansa tomuksi, heidän jousensa lentäviksi oljenkorsiksi?
\par 3 Hän ajaa heitä takaa, samoaa vammatonna polkua, hänen jalkainsa ennen kulkematonta.
\par 4 Kuka on tämän tehnyt ja toimittanut? Hän, joka alusta asti kutsuu sukupolvet esiin: minä, Herra, joka olen ensimmäinen ja viimeisten luona vielä sama.
\par 5 Merensaaret näkivät sen ja peljästyivät, maan ääret vapisivat. He lähestyivät, he tulivat,
\par 6 he auttoivat toinen toistaan ja sanoivat toisillensa: "Ole luja!"
\par 7 Valaja rohkaisee kultaseppää, levyn vasaroitsija alasimen iskijää; hän sanoo juotoksesta: "Se on hyvä", ja vahvistaa sen nauloilla, niin ettei se horju.
\par 8 Mutta sinä Israel, minun palvelijani, sinä Jaakob, jonka minä olen valinnut, Aabrahamin, minun ystäväni, siemen,
\par 9 jonka minä olen ottanut maan ääristä ja kutsunut maan kaukaisimmilta periltä, jolle minä sanoin: "Sinä olet minun palvelijani, sinut minä olen valinnut enkä sinua halpana pitänyt",
\par 10 älä pelkää, sillä minä olen sinun kanssasi; älä arkana pälyile, sillä minä olen sinun Jumalasi; minä vahvistan sinua, minä autan sinua, minä tuen sinua vanhurskauteni oikealla kädellä.
\par 11 Katso, häpeän ja pilkan saavat kaikki, jotka palavat vihasta sinua vastaan; tyhjiin raukeavat ja hukkuvat, jotka sinun kanssasi riitelevät.
\par 12 Hakemallakaan et löydä niitä, jotka sinua vastaan taistelivat; tyhjiin raukeavat ja lopun saavat, jotka sinun kanssasi sotivat.
\par 13 Sillä minä, Herra, sinun Jumalasi, tartun sinun oikeaan käteesi, minä sanon sinulle: "Älä pelkää, minä autan sinua".
\par 14 Älä pelkää, Jaakob, sinä mato, sinä Israelin vähäinen väki: minä autan sinua, sanoo Herra, ja sinun lunastajasi on Israelin Pyhä.
\par 15 Katso, minä panen sinut raastavaksi puimaäkeeksi, uudeksi, monihampaiseksi. Sinä puit ja rouhennat vuoret, muutat kukkulat akanoiksi;
\par 16 sinä ne viskaat, ja tuuli ne vie ja myrsky ne hajottaa. Mutta sinä iloitset Herrassa, Israelin Pyhä on sinun kerskauksesi.
\par 17 Kurjat ja köyhät etsivät vettä, eikä sitä ole; heidän kielensä kuivuu janosta. Mutta minä, Herra, kuulen heitä, minä, Israelin Jumala, en heitä hylkää.
\par 18 Minä puhkaisen purot kalliokukkuloihin, lähteet laaksojen pohjiin; minä muutan erämaan vesilammikoiksi ja hietikon hetteiköksi.
\par 19 Minä kasvatan erämaahan setripuita, akasioita, myrttejä ja öljypuita; minä istutan arolle kypressejä, jalavia ynnä hopeakuusia,
\par 20 jotta he näkisivät ja tietäisivät, huomaisivat ja myös ymmärtäisivät, että Herran käsi on tämän tehnyt, Israelin Pyhä sen luonut.
\par 21 Tuokaa esiin riita-asianne, sanoo Herra, esittäkää todisteenne, sanoo Jaakobin kuningas.
\par 22 Esittäkööt ja ilmoittakoot meille, mitä tapahtuva on; ilmoittakaa entiset, mitä ne olivat, tarkataksemme ja tietääksemme, mitä niistä on tullut, tahi antakaa meidän kuulla tulevaisia.
\par 23 Ilmoittakaa, mitä vastedes tapahtuu, tietääksemme, oletteko te jumalia. Tehkää hyvää tai tehkää pahaa, niin me katsomme ja ihmettelemme.
\par 24 Katso, te olette pelkkää tyhjää, ja teidän tekonne ovat turhat. Kauhistus se, joka teidät valitsee!
\par 25 Minä herätin pohjoisesta hänet, ja hän tuli, päivänkoitosta hänet, joka rukoilee minun nimeäni, ja hän tallaa käskynhaltijoita kuin lokaa, niinkuin savenvalaja savea sotkee.
\par 26 Kuka on sen alunpitäen ilmoittanut, että olisimme sen tienneet, ja edeltäpäin, niin että voisimme sanoa: "Hän oli oikeassa"? Ei kukaan sitä ilmoittanut, ei kukaan sitä kuuluttanut, ei kukaan kuullut teiltä sanaakaan.
\par 27 Minä ensimmäisenä sanon Siionille: "Katso, katso, siinä ne ovat!" ja annan Jerusalemille ilosanomantuojan.
\par 28 Minä katselen ympärilleni, mutta ei ole ketään; ei kenkään näistä voi antaa neuvoa, että kysyisin heiltä ja he vastaisivat.
\par 29 Katso, kaikki he ovat pelkkää petosta, turhat ovat heidän työnsä; tuulta ja tyhjää ovat heidän valetut kuvansa.

\chapter{42}

\par 1 Katso, minun palvelijani, jota minä tuen, minun valittuni, johon minun sieluni mielistyi. Minä olen pannut Henkeni häneen, hän levittää kansakuntiin oikeuden.
\par 2 Ei hän huuda eikä korota ääntään, ei anna sen kuulua kaduilla.
\par 3 Särjettyä ruokoa hän ei muserra, ja suitsevaista kynttilänsydäntä hän ei sammuta. Hän levittää oikeutta uskollisesti.
\par 4 Hän itse ei sammu eikä murru, kunnes on saattanut oikeuden maan päälle, ja merensaaret odottavat hänen opetustansa.
\par 5 Näin sanoo Jumala, Herra, joka on luonut taivaan ja levittänyt sen, joka on tehnyt maan laveuden ja mitä siinä kasvaa, antanut henkäyksensä kansalle, joka siinä on, ja hengen niille, jotka siellä vaeltavat:
\par 6 Minä, Herra, olen vanhurskaudessa kutsunut sinut, olen tarttunut sinun käteesi, varjellut sinut ja pannut sinut kansoille liitoksi, pakanoille valkeudeksi,
\par 7 avaamaan sokeat silmät, päästämään sidotut vankeudesta, pimeydessä istuvat vankihuoneesta.
\par 8 Minä, Herra, se on minun nimeni, minä en anna kunniaani toiselle enkä ylistystäni epäjumalille.
\par 9 Katso, entiset ovat toteen käyneet, ja uusia minä ilmoitan; ennenkuin ne puhkeavat taimelle, annan minä teidän ne kuulla.
\par 10 Veisatkaa Herralle uusi virsi, veisatkaa hänen ylistystänsä hamasta maan äärestä, te merenkulkijat ja meri täysinensä, te merensaaret ja niissä asuvaiset.
\par 11 Korottakoot äänensä erämaa ja sen kaupungit, kylät, joissa Keedar asuu. Riemuitkoot kallioilla asuvaiset, vuorten huipuilta huutakoot ilosta.
\par 12 Antakoot Herralle kunnian ja julistakoot hänen ylistystään merensaarissa.
\par 13 Herra lähtee sotaan niinkuin sankari, niinkuin soturi hän kiihoittaa kiivautensa; hän nostaa sotahuudon ja karjuu, uhittelee vihollisiansa.
\par 14 Minä olen ollut vaiti ikiajoista asti, olen ollut hiljaa ja pidättänyt itseni. Mutta nyt minä huudan kuin lapsensynnyttäjä, puhallan ja puuskun.
\par 15 Minä teen autioiksi vuoret ja kukkulat, ja kuihdutan niiltä kaiken ruohon; minä muutan virrat saariksi ja kuivaan vesilammikot.
\par 16 Minä johdatan sokeat tietä, jota he eivät tunne; polkuja, joita he eivät tunne, minä kuljetan heidät. Minä muutan pimeyden heidän edellään valkeudeksi ja koleikot tasangoksi. Nämä minä teen enkä niitä tekemättä jätä.
\par 17 Mutta ne peräytyvät ja joutuvat häpeään, jotka turvaavat veistettyyn kuvaan, jotka sanovat valetuille kuville: "Te olette meidän jumalamme".
\par 18 Te kuurot, kuulkaa, ja te sokeat, katsokaa ja nähkää.
\par 19 Kuka on sokea, ellei minun palvelijani, ja kuka niin kuuro kuin minun sanansaattajani, jonka minä lähetän? Kuka on niin sokea kuin minun palkkalaiseni, niin sokea kuin Herran palvelija?
\par 20 Paljon sinä olet nähnyt, mutta et ole ottanut varteen; korvat avattiin, mutta ei kuulla.
\par 21 Herra on nähnyt hyväksi vanhurskautensa tähden tehdä lain suureksi ja ihanaksi.
\par 22 Mutta tämä on raastettu ja ryöstetty kansa kaikki nuoret miehet ovat sidotut ja vankihuoneisiin kätketyt; he ovat joutuneet saaliiksi, eikä ole auttajaa, ryöstetyiksi, eikä ole, kuka sanoisi: "Anna takaisin!"
\par 23 Kuka teistä ottaa tämän korviinsa, tarkkaa ja kuulee vastaisen varalta?
\par 24 Kuka on antanut Jaakobin ryöstettäväksi ja Israelin raastajain valtaan? Eikö Herra, jota vastaan me olemme syntiä tehneet, jonka teitä he eivät tahtoneet vaeltaa ja jonka lakia he eivät totelleet?
\par 25 Niin hän vuodatti Israelin päälle vihansa hehkun ja sodan tuimuuden. Se liekehti hänen ympärillään, mutta hän ei ollut tietääksensä; se poltti häntä, mutta hän ei siitä huolinut.

\chapter{43}

\par 1 Mutta nyt, näin sanoo Herra, joka loi sinut, Jaakob, joka valmisti sinut, Israel: Älä pelkää, sillä minä olen lunastanut sinut, minä olen sinut nimeltä kutsunut; sinä olet minun.
\par 2 Jos vetten läpi kuljet, olen minä sinun kanssasi, jos virtojen läpi, eivät ne sinua upota; jos tulen läpi käyt, et sinä kärvenny, eikä liekki sinua polta.
\par 3 Sillä minä olen Herra, sinun Jumalasi, Israelin Pyhä, sinun vapahtajasi: minä annan sinun lunnaiksesi Egyptin, sinun sijastasi Etiopian ja Seban.
\par 4 Koska sinä olet minun silmissäni kallis ja suuriarvoinen ja koska minä sinua rakastan, annan minä ihmisiä sinun sijastasi ja kansakuntia sinun hengestäsi.
\par 5 Älä pelkää, sillä minä olen sinun kanssasi; minä tuon sinun siemenesi päivänkoiton ääriltä, päivän laskemilta minä sinut kokoan.
\par 6 Minä sanon pohjoiselle: Anna tänne! ja etelälle: Älä pidätä! Tuo minun poikani kaukaa ja minun tyttäreni hamasta maan äärestä -
\par 7 kaikki, jotka ovat otetut minun nimiini ja jotka minä olen kunniakseni luonut, jotka minä olen valmistanut ja tehnyt.
\par 8 Tuo esiin sokea kansa, jolla kuitenkin on silmät, ja kuurot, joilla kuitenkin on korvat.
\par 9 Kaikki kansat ovat kokoontuneet yhteen ja kansakunnat tulleet kokoon. Kuka heistä voi ilmoittaa tämänkaltaista? Tai antakoot meidän kuulla entisiä; asettakoot todistajansa ja näyttäkööt, että ovat oikeassa: sittenpä kuullaan ja sanotaan: "Se on totta".
\par 10 Te olette minun todistajani, sanoo Herra, minun palvelijani, jonka minä olen valinnut, jotta te tuntisitte minut ja uskoisitte minuun ja ymmärtäisitte, että minä se olen. Ennen minua ei ole luotu yhtäkään jumalaa, eikä minun jälkeeni toista tule.
\par 11 Minä, minä olen Herra, eikä ole muuta pelastajaa, kuin minä.
\par 12 Minä olen ilmoittanut, olen pelastanut ja julistanut, eikä ollut vierasta jumalaa teidän keskuudessanne. Te olette minun todistajani, sanoo Herra, ja minä olen Jumala.
\par 13 Tästedeskin minä olen sama. Ei kukaan voi vapauttaa minun kädestäni; minkä minä teen, kuka sen peruuttaa?
\par 14 Näin sanoo Herra, teidän lunastajanne, Israelin Pyhä: Teidän tähtenne minä lähetän sanan Baabeliin, minä syöksen heidät kaikki pakoon, syöksen kaldealaiset laivoihin, jotka olivat heidän ilonsa.
\par 15 Minä, Herra, olen teidän Pyhänne, minä, Israelin Luoja, olen teidän kuninkaanne.
\par 16 Näin sanoo Herra, joka teki tien mereen ja polun valtaviin vesiin,
\par 17 joka vei sotaan vaunut ja hevoset, sotaväen ja sankarit kaikki; he vaipuivat eivätkä nousseet, he raukenivat, sammuivat niinkuin lampunsydän:
\par 18 Älkää entisiä muistelko, älkää menneistä välittäkö.
\par 19 Katso, minä teen uutta; nyt se puhkeaa taimelle, ettekö sitä huomaa? Niin, minä teen tien korpeen, virrat erämaahan.
\par 20 Minua kunnioittavat metsän eläimet, aavikkosudet ja kamelikurjet, koska minä johdan vedet korpeen, virrat erämaahan, antaakseni kansani, minun valittuni, juoda.
\par 21 Kansa, jonka minä olen itselleni valmistanut, on julistava minun kiitostani.
\par 22 Mutta et ole sinä, Jaakob, minua kutsunut, et ole sinä, Israel, itseäsi minun tähteni vaivannut.
\par 23 Et ole tuonut minulle lampaitasi polttouhriksi, et teurasuhreillasi minua kunnioittanut; en ole minä vaivannut sinua ruokauhrilla enkä suitsutusuhrilla sinua rasittanut.
\par 24 Et ole minulle kalmoruokoa hopealla ostanut etkä minua teurasuhriesi rasvalla ravinnut - et, vaan sinä olet minua synneilläsi vaivannut, rasittanut minua pahoilla töilläsi.
\par 25 Minä, minä pyyhin pois sinun rikkomuksesi itseni tähden, enkä sinun syntejäsi muista.
\par 26 Muistuta sinä minua, käykäämme oikeutta keskenämme; puhu sinä ja näytä, että olet oikeassa.
\par 27 Jo sinun esi-isäsi teki syntiä, sinun puolusmiehesi luopuivat minusta.
\par 28 Niin minä annoin häväistä pyhät ruhtinaat, jätin Jaakobin tuhon omaksi, Israelin alttiiksi pilkalle.

\chapter{44}

\par 1 Mutta nyt kuule, Jaakob, minun palvelijani, ja Israel, jonka minä olen valinnut.
\par 2 Näin sanoo Herra, sinun Luojasi, joka on valmistanut sinut hamasta äidin kohdusta, joka sinua auttaa: Älä pelkää, minun palvelijani Jaakob, sinä Jesurun, jonka minä olen valinnut.
\par 3 Sillä minä vuodatan vedet janoavaisen päälle ja virrat kuivan maan päälle. Minä vuodatan Henkeni sinun siemenesi päälle ja siunaukseni sinun vesojesi päälle,
\par 4 niin että ne kasvavat nurmikossa kuin pajut vesipurojen partaalla.
\par 5 Mikä sanoo: "Minä olen Herran oma", mikä nimittää itsensä Jaakobin nimellä, mikä piirtää käteensä: "Herran oma", ja ottaa Israelin kunnianimeksensä.
\par 6 Näin sanoo Herra, Israelin kuningas, ja sen lunastaja, Herra Sebaot: Minä olen ensimmäinen, ja minä olen viimeinen, ja paitsi minua ei ole yhtäkään Jumalaa.
\par 7 Kuka on minun kaltaiseni? Hän julistakoon ja ilmoittakoon ja osoittakoon sen minulle, siitä alkaen kuin minä perustin ikiaikojen kansan. He ilmoittakoot tulevaiset, ja mitä tapahtuva on.
\par 8 Älkää vavisko älkääkä peljätkö. Enkö minä aikoja sitten antanut sinun kuulla ja sinulle ilmoittanut, ja te olette minun todistajani: Onko muuta Jumalaa kuin minä? Ei ole muuta pelastuskalliota, minä en ketään tunne.
\par 9 Jumalankuvien tekijät ovat turhia kaikki tyynni, eivätkä nuo heidän rakkaansa mitään auta; niiden todistajat eivät näe eivätkä tiedä mitään, ja niin he joutuvat häpeään.
\par 10 Kuka hyvänsä muovatkoon jumalan ja valakoon kuvan, ei se mitään auta.
\par 11 Katso, kaikki sen seuraajat joutuvat häpeään, ja sen sepittäjät ovat vain ihmisiä. Tulkoot kokoon kaikki ja astukoot esiin: vaviskoot ja hävetkööt!
\par 12 Rautaseppä ottaa työaseen ja työskentelee hiilten hehkussa, muodostelee kuvaa vasaralla ja takoo sitä käsivartensa väellä; hänen tulee nälkä, ja voima menee, hän ei saa vettä juodaksensa, ja hän nääntyy.
\par 13 Puuseppä jännittää mittanuoran, kaavailee piirtimellä, vuolee kovertimella, mittailee harpilla ja tekee miehen kuvan, inhimillisen kauneuden mukaan, huoneeseen asumaan.
\par 14 Hän hakkaa itselleen setripuita, hän ottaa rautatammen ja tammen ja kasvattaa ne itselleen vahvoiksi metsän puitten seassa, istuttaa lehtikuusen, ja sade kasvattaa sen suureksi.
\par 15 Se on ihmisillä polttopuuna; hän ottaa sitä lämmitelläkseen, sytyttää uunin ja paistaa leipää, vieläpä veistää siitä jumalan ja kumartaa sitä, tekee siitä jumalankuvan ja lankeaa maahan sen eteen.
\par 16 Osan siitä hän polttaa tulessa, toisen osan ääressä hän syö lihaa, paistaa paistin ja tulee ravituksi; hän myöskin lämmittelee itseänsä ja sanoo: "Hyvä, minun on lämmin, minä näen valkean".
\par 17 Ja lopusta hän tekee jumalan, jumalankuvan, jonka eteen hän lankeaa maahan, jota hän kumartaa ja rukoilee sanoen: "Pelasta minut, sillä sinä olet minun Jumalani".
\par 18 Eivät he tajua, eivät ymmärrä mitään, sillä suljetut ovat heidän silmänsä, niin etteivät he näe, ja heidän sydämensä, niin etteivät he käsitä.
\par 19 Ei tule heidän mieleensä, ei ole heillä järkeä eikä ymmärrystä, että sanoisivat: "Osan siitä olen polttanut tulessa, olen paistanut sen hiilillä leipää, paistanut lihaa ja syönyt; tekisinkö tähteestä kauhistuksen, lankeaisinko maahan puupölkyn eteen!"
\par 20 Joka tuhassa kiinni riippuu, sen on petetty sydän harhaan vienyt, ei hän pelasta sieluansa eikä sano: "Eikö ole petosta se, mikä on oikeassa kädessäni?"
\par 21 Muista tämä, Jaakob, ja sinä, Israel, sillä sinä olet minun palvelijani. Minä olen sinut valmistanut, sinä olet minun palvelijani: en unhota minä sinua, Israel.
\par 22 Minä pyyhin pois sinun rikkomuksesi niinkuin pilven ja sinun syntisi niinkuin sumun. Palaja minun tyköni, sillä minä lunastan sinut.
\par 23 Iloitkaa, te taivaat, sillä Herra sen tekee; riemuitkaa, te maan syvyydet, puhjetkaa riemuun, te vuoret, ynnä metsä ja kaikki sen puut; sillä Herra lunastaa Jaakobin, kirkastaa itsensä Israelissa.
\par 24 Näin sanoo Herra, sinun lunastajasi, joka on valmistanut sinut hamasta äidin kohdusta: Minä olen Herra, joka teen kaiken, joka yksinäni jännitin taivaan, joka levitin maan - kuka oli minun kanssani? -
\par 25 joka teen tyhjäksi valhettelijain merkit, teen taikurit tyhmiksi, joka panen tietäjät peräytymään ja muutan heidän tietonsa typeryydeksi.
\par 26 Mutta palvelijani sanan minä toteutan ja saatan täyttymään sanansaattajaini neuvon. Minä olen se, joka Jerusalemille sanon: "Sinussa asuttakoon!" ja Juudan kaupungeille: "Teidät rakennettakoon!" Sen rauniot minä kohotan.
\par 27 Minä olen se, joka sanon syvyydelle: "Kuivu; minä kuivutan sinun virtasi!"
\par 28 joka sanon Koorekselle: "Minun paimeneni!" Hän täyttää kaiken minun tahtoni, hän sanoo Jerusalemille: "Sinut rakennettakoon!" ja temppelille: "Sinut perustettakoon!"

\chapter{45}

\par 1 Näin sanoo Herra voidellulleen Koorekselle, jonka oikeaan käteen minä olen tarttunut kukistaakseni kansat hänen edestään ja riisuakseni kuninkaitten kupeilta vyöt, että ovet hänen edessään avautuisivat eivätkä portit sulkeutuisi:
\par 2 Minä käyn sinun edelläsi ja tasoitan kukkulat, minä murran vaskiovet ja rikon rautasalvat.
\par 3 Minä annan sinulle aarteet pimeän peitosta, kalleudet kätköistänsä, tietääksesi, että minä, Herra, olen se, joka sinut nimeltä kutsuin, minä, Israelin Jumala.
\par 4 Palvelijani Jaakobin ja valittuni Israelin tähden minä kutsuin sinut nimeltä ja annoin sinulle kunnianimen, vaikka sinä et minua tuntenut.
\par 5 Minä olen Herra, eikä toista ole, paitsi minua ei ole yhtään jumalaa. Minä vyötän sinut, vaikka sinä et minua tunne,
\par 6 jotta tiedettäisiin auringon noususta sen laskemille asti, että paitsi minua ei ole yhtäkään: minä olen Herra, eikä toista ole,
\par 7 minä, joka teen valkeuden ja luon pimeyden, joka tuotan onnen ja luon onnettomuuden; minä, Herra, teen kaiken tämän.
\par 8 Tiukkukaa, te taivaat, ylhäältä, vuotakoot pilvet vanhurskautta. Avautukoon maa ja antakoon hedelmänänsä pelastuksen, versokoon se myös vanhurskautta. Minä, Herra, olen sen luonut.
\par 9 Voi sitä, joka riitelee tekijänsä kanssa, saviastia saviastiain joukossa - maasta tehtyjä kaikki! Sanooko savi valajallensa: "Mitä sinä kelpaat tekemään? Sinun työsi on kädettömän työtä!"
\par 10 Voi sitä, joka sanoo isälleen: "Mitä sinä kelpaat siittämään?" ja äidilleen: "Mitä sinä kelpaat synnyttämään?"
\par 11 Näin sanoo Herra, Israelin Pyhä, joka on hänet tehnyt: Kysykää tulevaisia minulta ja jättäkää minun haltuuni minun lapseni, minun kätteni teot.
\par 12 Minä olen tehnyt maan ja luonut ihmisen maan päälle; minun käteni ovat levittäneet taivaan, minä olen kutsunut koolle kaikki sen joukot.
\par 13 Minä herätin hänet vanhurskaudessa, ja minä tasoitin kaikki hänen tiensä. Hän rakentaa minun kaupunkini ja päästää vapaiksi minun pakkosiirtolaiseni ilman maksua ja ilman lahjusta, sanoo Herra Sebaot.
\par 14 Näin sanoo Herra: Egyptin työansio ja Etiopian kauppavoitto ja sebalaiset, suurikasvuiset miehet, tulevat sinun tykösi, tulevat sinun omiksesi. Sinun perässäsi he käyvät, kulkevat kahleissa, sinua kumartavat, sinua rukoilevat: "Ainoastaan sinun tykönäsi on Jumala, ei ole toista, ei yhtään muuta jumalaa".
\par 15 Totisesti, sinä olet salattu Jumala, sinä Israelin Jumala, sinä Vapahtaja.
\par 16 Häpeän ja pilkan he saavat kaikki, pilkan alaisina he kulkevat kaikki, nuo kuvien tekijät.
\par 17 Mutta Israelin pelastaa Herra iankaikkisella pelastuksella, te ette joudu häpeään ettekä pilkan alaisiksi, ette ikinä, hamaan iankaikkisuuteen saakka.
\par 18 Sillä näin sanoo Herra, joka on luonut taivaan - hän on Jumala - joka on valmistanut maan ja tehnyt sen; hän on sen vahvistanut, ei hän sitä autioksi luonut, asuttavaksi hän sen valmisti: Minä olen Herra, eikä toista ole.
\par 19 En ole minä puhunut salassa, en pimeässä maan paikassa; en ole sanonut Jaakobin jälkeläisille: etsikää minua tyhjyydestä. Minä Herra puhun vanhurskautta, ilmoitan, mikä oikein on.
\par 20 Kokoontukaa ja tulkaa, lähestykää kaikki, te henkiinjääneet kansakunnista. Eivät ne mitään ymmärrä, jotka kantavat puukuviansa ja rukoilevat jumalaa, joka ei voi auttaa.
\par 21 Ilmoittakaa ja esiin tuokaa - neuvotelkoot keskenänsä - kuka on tämän julistanut hamasta muinaisuudesta, aikoja sitten ilmoittanut? Enkö minä, Herra! Paitsi minua ei ole yhtään jumalaa; ei ole vanhurskasta ja auttavaa jumalaa muuta kuin minä.
\par 22 Kääntykää minun tyköni ja antakaa pelastaa itsenne, te maan ääret kaikki, sillä minä olen Jumala, eikä toista ole.
\par 23 Minä olen vannonut itse kauttani, minun suustani on lähtenyt totuus, peruuttamaton sana: Minun edessäni pitää kaikkien polvien notkistuman, minulle jokaisen kielen valansa vannoman.
\par 24 Ainoastaan Herrassa - niin pitää minusta sanottaman - on vanhurskaus ja voima. Hänen tykönsä tulevat häveten kaikki, jotka ovat palaneet vihasta häntä vastaan.
\par 25 Herrassa tulee vanhurskaaksi kaikki Israelin siemen, ja hän on heidän kerskauksensa.

\chapter{46}

\par 1 Beel vaipuu, Nebo taipuu; heidän kuvansa joutuvat elukkain ja juhtain selkään; mitä te kulkueessa kannoitte, se sälytetään kuormaksi uupuville.
\par 2 He taipuvat, he vaipuvat molemmat; he eivät voi pelastaa kuormaa, ja itse he vaeltavat vankeuteen.
\par 3 Kuulkaa minua, te Jaakobin heimo, te kaikki Israelin heimon tähteet, te, joita on pitänyt kantaa äidinkohdusta asti, nostaa hamasta äidinhelmasta.
\par 4 Teidän vanhuuteenne asti minä olen sama, hamaan harmaantumiseenne saakka minä kannan; niin minä olen tehnyt, ja vastedeskin minä nostan, minä kannan ja pelastan.
\par 5 Keneenkä te vertaatte minut, kenenkä rinnalle minut asetatte, kenenkä kaltaiseksi te minut katsotte, että olisimme toistemme vertaiset?
\par 6 He kaatavat kultaa kukkarosta ja punnitsevat hopeata vaa'alla; he palkkaavat kultasepän, ja hän tekee siitä jumalan, jonka eteen he lankeavat maahan ja jota he kumartavat.
\par 7 He nostavat sen olallensa, kantavat ja asettavat sen paikoilleen, ja se seisoo eikä liikahda paikaltansa. Sitä huudetaan avuksi, mutta se ei vastaa, hädästä se ei pelasta.
\par 8 Muistakaa tämä ja olkaa vahvat. Menkää itseenne, te luopiot.
\par 9 Muistakaa entisiä ikiajoista asti, sillä minä olen Jumala, eikä toista ole; minä olen Jumala, eikä ole minun vertaistani.
\par 10 Minä ilmoitan alusta asti, mitä tuleva on, ammoisia aikoja ennen, mitä ei vielä ole tapahtunut; minä sanon: minun neuvoni pysyy, kaiken, mitä tahdon, minä teen.
\par 11 Minä olen kutsunut kotkan päivänkoitosta, kaukaisesta maasta neuvopäätökseni miehen. Minkä olen puhunut, sen minä myös toteutan; mitä olen aivoitellut, sen minä myös teen.
\par 12 Kuulkaa minua, te kovasydämiset, jotka olette kaukana vanhurskaudesta.
\par 13 Minä olen antanut vanhurskauteni lähestyä, se ei ole kaukana; ei viivy pelastus, jonka minä tuon. Minä annan Siionissa pelastuksen, annan kirkkauteni Israelille.

\chapter{47}

\par 1 Astu alas ja istu tomuun, sinä neitsyt, tytär Baabel, istu maahan, valtaistuinta vailla, sinä Kaldean tytär; sillä ei sinua enää kutsuta hempeäksi ja hekumalliseksi.
\par 2 Tartu käsikiviin ja jauha jauhoja, riisu huntusi, nosta helmuksesi, paljasta sääresi, kahlaa jokien poikki.
\par 3 Häpysi paljastuu, häpeäsi näkyy; minä kostan enkä ainoatakaan armahda.
\par 4 Meidän lunastajamme nimi on Herra Sebaot, Israelin Pyhä.
\par 5 Istu äänetönnä ja väisty pimeään, sinä Kaldean tytär; sillä ei sinua enää kutsuta valtakuntien valtiattareksi.
\par 6 Minä vihastuin kansaani, annoin häväistä perintöni, minä annoin heidät sinun käsiisi; et osoittanut sinä heille sääliä, vanhuksellekin sinä teit ikeesi ylen raskaaksi.
\par 7 Ja sinä sanoit: "Iäti minä olen valtiatar", niin ettet näitä mieleesi pannut, et loppua ajatellut.
\par 8 Mutta nyt kuule tämä, sinä hekumassa-eläjä, joka istut turvallisena, joka sanot sydämessäsi: "Minä, eikä ketään muuta! Minä en ole leskenä istuva enkä lapsettomuudesta tietävä."
\par 9 Mutta nämä molemmat tulevat sinun osaksesi äkisti, yhtenä päivänä: lapsettomuus ja leskeys; ne kohtaavat sinua täydeltänsä, huolimatta velhouksiesi paljoudesta, loitsujesi suuresta voimasta.
\par 10 Sinä luotit pahuuteesi, sinä sanoit: "Ei kukaan minua näe". Sinun viisautesi ja tietosi, ne sinut eksyttivät, ja niin sinä sanoit sydämessäsi: "Minä, eikä ketään muuta!"
\par 11 Sentähden kohtaa sinua onnettomuus, jota et osaa manata pois; sinut yllättää tuho, josta et lunnailla pääse, äkkiä kohtaa sinua perikato, aavistamattasi.
\par 12 Astu esiin loitsuinesi ja paljoine velhouksinesi, joilla olet vaivannut itseäsi nuoruudestasi asti: ehkä hyvinkin saat avun, ehkä herätät pelkoa.
\par 13 Sinä olet väsyttänyt itsesi paljolla neuvottelullasi. Astukoot esiin ja auttakoot sinua taivaan mittaajat, tähtien tähystäjät, jotka kuu kuulta ilmoittavat, mitä sinulle tapahtuva on.
\par 14 Katso, he ovat kuin kuivat korret: tuli polttaa heidät; he eivät pelasta henkeään liekin vallasta. Se ei ole hiillos heidän lämmitelläkseen eikä valkea, jonka ääressä istutaan.
\par 15 Tämän saat sinä niistä, joista olet vaivaa nähnyt: ne, joiden kanssa olet kauppaa käynyt hamasta nuoruudestasi asti, harhailevat kukin haarallensa, ei kukaan sinua pelasta.

\chapter{48}

\par 1 Kuulkaa tämä, Jaakobin heimo, te, joita kutsutaan Israelin nimellä ja jotka olette lähteneet Juudan lähteestä, jotka vannotte Herran nimeen ja tunnustatte Israelin Jumalan, mutta ette totuudessa ettekä vanhurskaudessa -
\par 2 sillä heitä kutsutaan pyhän kaupungin mukaan ja he pitävät tukenansa Israelin Jumalaa, jonka nimi on Herra Sebaot:
\par 3 Entiset minä olen aikoja ennen ilmoittanut, minun suustani ne ovat lähteneet, ja minä olen ne kuuluttanut. Äkkiä minä panin ne täytäntöön, ja ne tapahtuivat.
\par 4 Koska minä tiesin, että sinä olet paatunut, että sinun niskajänteesi on rautaa ja otsasi vaskea,
\par 5 niin minä ilmoitin ne sinulle aikoja ennen, kuulutin ne sinulle, ennenkuin ne tapahtuivat, ettet sanoisi: "Epäjumalankuvani on ne tehnyt, veistetty kuvani ja valettu kuvani on niin säätänyt".
\par 6 Sinä olet ne kuullut, katso nyt kaikkea: ettekö sitä tunnusta? Tästä lähtien minä kuulutan sinulle uusia, salatuita, joita et ole tietänyt.
\par 7 Ne ovat luodut nyt, eikä aikoja sitten, ennen tätä päivää et ole niistä kuullut, ettet saattaisi sanoa: "Katso, jo minä ne tiesin!"
\par 8 Et ole sinä niistä kuullut etkä tietänyt, ei ole sinun korvasi niille aikaisemmin auennut; sillä minä tiesin, että sinä olet aivan uskoton ja luopioksi kutsuttu hamasta äidin kohdusta.
\par 9 Oman nimeni tähden minä olen pitkämielinen, ylistykseni tähden minä hillitsen vihani, etten sinua tuhoaisi.
\par 10 Katso, minä olen sinua sulattanut, hopeata saamatta, minä olen sinua koetellut kärsimyksen pätsissä.
\par 11 Itseni, itseni tähden minä sen teen; sillä kuinka onkaan minun nimeäni häväisty! Kunniaani en minä toiselle anna.
\par 12 Kuule minua, Jaakob, ja sinä, Israel, jonka minä olen kutsunut: Minä olen aina sama, minä olen ensimmäinen, minä olen myös viimeinen.
\par 13 Minun käteni on perustanut maan, minun oikea käteni on levittänyt taivaan; minä kutsun ne, ja siinä ne ovat.
\par 14 Kokoontukaa kaikki ja kuulkaa - kuka niistä muista on ilmoittanut tämän: että hän, jota Herra rakastaa, tekee hänen tahtonsa Baabelia vastaan, on hänen käsivartenaan kaldealaisia vastaan.
\par 15 Minä, minä olen puhunut, minä olen hänet kutsunut, olen hänet tuonut, ja hän on menestynyt teillänsä.
\par 16 Lähestykää minua, kuulkaa tämä: Alkujaankaan minä en ole puhunut salassa; kun nämä tapahtuivat, olin minä jo siellä. Ja nyt Herra, Herra on lähettänyt minut ynnä oman Henkensä.
\par 17 Näin sanoo Herra, sinun lunastajasi, Israelin Pyhä: Minä olen Herra, sinun Jumalasi, joka opetan sinulle, mikä hyödyllistä on, johdatan sinua sitä tietä, jota sinun tulee käydä.
\par 18 Jospa ottaisit minun käskyistäni vaarin, niin olisi sinun rauhasi niinkuin virta ja sinun vanhurskautesi niinkuin meren aallot;
\par 19 sinun lastesi paljous olisi niinkuin hiekka, sinun kohtusi hedelmä niinkuin hiekkajyväset, sen nimi ei häviäisi, ei katoaisi minun kasvojeni edestä.
\par 20 Lähtekää Baabelista, paetkaa Kaldeasta; riemuhuudoin ilmoittakaa, kuuluttakaa tämä, viekää tieto siitä maan ääriin asti, sanokaa: "Herra on lunastanut palvelijansa Jaakobin".
\par 21 Eivät he janoa kärsineet, kun hän heitä kuljetti erämaitten halki; hän vuodatti heille vettä kalliosta, hän halkaisi kallion, ja vettä virtasi.
\par 22 Jumalattomilla ei ole rauhaa, sanoo Herra.

\chapter{49}

\par 1 Kuulkaa minua, te merensaaret, ja tarkatkaa, kaukaiset kansat. Herra on minut kutsunut hamasta äidinkohdusta saakka, hamasta äitini helmasta minun nimeni maininnut.
\par 2 Hän teki minun suuni terävän miekan kaltaiseksi, kätki minut kätensä varjoon; hän teki minut hiotuksi nuoleksi, talletti minut viineensä.
\par 3 Ja hän sanoi minulle: "Sinä olet minun palvelijani, sinä Israel, jossa minä osoitan kirkkauteni".
\par 4 Mutta minä sanoin: "Hukkaan minä olen itseäni vaivannut, kuluttanut voimani turhaan ja tyhjään; kuitenkin on minun oikeuteni Herran huomassa, minun palkkani on Jumalan tykönä".
\par 5 Ja nyt sanoo Herra, joka on minut palvelijakseen valmistanut hamasta äitini kohdusta, palauttamaan Jaakobin hänen tykönsä, niin että Israel koottaisiin hänen omaksensa - ja minä olen kallis Herran silmissä, minun Jumalani on tullut minun voimakseni -
\par 6 hän sanoo: Liian vähäistä on sinulle, joka olet minun palvelijani, kohottaa ennalleen Jaakobin sukukunnat ja tuoda takaisin Israelin säilyneet: minä panen sinut pakanain valkeudeksi, että minulta tulisi pelastus maan ääriin asti.
\par 7 Näin sanoo Herra, Israelin lunastaja, hänen Pyhänsä, syvästi halveksitulle, kansan inhoamalle, valtiaitten orjalle: Kuninkaat näkevät sen ja nousevat seisomaan, ruhtinaat näkevät ja kumartuvat maahan Herran tähden, joka on uskollinen, Israelin Pyhän tähden, joka on sinut valinnut.
\par 8 Näin sanoo Herra: Otollisella ajalla minä olen sinua kuullut ja pelastuksen päivänä sinua auttanut; minä olen valmistanut sinut ja pannut sinut kansoille liitoksi, kohottamaan ennalleen maan, jakamaan hävitetyt perintöosat,
\par 9 sanomaan vangituille: "Käykää ulos!" ja pimeässä oleville: "Tulkaa esiin!" Teiden varsilta he löytävät laitumen, kaikki kalliokukkulat ovat heillä laidunpaikkoina.
\par 10 Ei heidän tule nälkä eikä jano, ei hietikon helle eikä aurinko satu heihin, sillä heidän armahtajansa johdattaa heitä ja vie heidät vesilähteille.
\par 11 Minä teen kaikki vuoreni teiksi, ja minun valtatieni kulkevat korkealla.
\par 12 Katso heitä, he tulevat kaukaa! Katso, nuo pohjoisesta, nuo lännestä, nuo Siinimin maalta!
\par 13 Riemuitkaa, te taivaat, iloitse, sinä maa, puhjetkaa riemuun, te vuoret, sillä Herra lohduttaa kansaansa ja armahtaa kurjiansa.
\par 14 Mutta Siion sanoo: "Herra on minut hyljännyt, Herra on minut unhottanut".
\par 15 Unhottaako vaimo rintalapsensa, niin ettei hän armahda kohtunsa poikaa? Ja vaikka he unhottaisivatkin, minä en sinua unhota.
\par 16 Katso, kätteni hipiään olen minä sinut piirtänyt, sinun muurisi ovat aina minun edessäni.
\par 17 Sinun lapsesi tulevat rientäen; sinun hävittäjäsi ja raunioiksi-raastajasi menevät sinun luotasi pois.
\par 18 Nosta silmäsi ja katso ympärillesi: he kokoontuvat, he tulevat sinun tykösi kaikki. Niin totta kuin minä elän, sanoo Herra, sinä puet heidät kaikki yllesi niinkuin koristeen ja sidot heidät vyöllesi niinkuin morsian vyönsä.
\par 19 Sillä sinä - sinun rauniosi ja autiot paikkasi, sinun hävitetty maasi - sinä käyt silloin ahtaaksi asukkaille, ja kaukana ovat ne, jotka sinua söivät.
\par 20 Vielä saavat sinun lapsettomuutesi lapset sanoa korviesi kuullen: "Paikka on minulle ahdas, tee tilaa, että voin asua".
\par 21 Silloin sinä sanot sydämessäsi: "Kuka on nämä minulle synnyttänyt? Minä olin lapseton ja hedelmätön, karkoitettu ja hyljätty. Kuka on heidät kasvattanut? Katso, minä olin jätetty yksin. Missä nämä silloin olivat?
\par 22 Näin sanoo Herra, Herra: Katso, minä nostan käteni kansakuntien puoleen, kohotan lippuni kansoja kohti, niin he tuovat sinun poikasi sylissänsä ja kantavat sinun tyttäresi olkapäillään.
\par 23 Kuninkaista tulee sinulle lastenhoitajat, heidän ruhtinattaristaan sinulle imettäjät. Sinun edessäsi he kumartuvat maahan kasvoillensa ja nuolevat tomun sinun jaloistasi. Silloin sinä tiedät, että minä olen Herra ja että ne, jotka minua odottavat, eivät häpeään joudu.
\par 24 Otetaanko sankarilta saalis, tai riistetäänkö vangit vanhurskaalta?
\par 25 Sillä näin sanoo Herra: Vaikka vangit otettaisiinkin sankarilta ja saalis riistettäisiin väkevältä, niin minä kuitenkin taistelen sitä vastaan, joka sinua vastaan taistelee, ja minä pelastan sinun lapsesi.
\par 26 Minä panen sinun sortajasi syömään omaa lihaansa, ja he juopuvat omasta verestään niinkuin rypälemehusta; ja kaikki liha on tietävä, että minä, Herra, olen sinun pelastajasi, että Jaakobin Väkevä on sinun lunastajasi.

\chapter{50}

\par 1 Näin sanoo Herra: Missä on teidän äitinne erokirja, jolla olisin lähettänyt hänet pois? Tai kuka on minun velkojani, jolle minä olisin teidät myynyt? Katso, pahojen tekojenne tähden te olette myydyt, teidän rikkomustenne tähden on äitinne lähetetty pois.
\par 2 Miksi ei ollut ketään, kun minä tulin, miksi ei kukaan vastannut, kun minä huusin? Onko minun käteni liian lyhyt vapahtamaan, olenko minä voimaton auttamaan? Katso, nuhtelullani minä kuivaan meren, minä teen virrat erämaaksi, niin että niitten kalat mätänevät, kun ei ole vettä; ne kuolevat janoon.
\par 3 Minä puetan taivaat mustiin ja panen murhepuvun niitten verhoksi.
\par 4 Herra, Herra on minulle antanut opetuslasten kielen, niin että minä taidan sanalla virvoittaa väsynyttä; hän herättää aamu aamulta, herättää minun korvani kuulemaan opetuslasten tavalla.
\par 5 Herra, Herra on avannut minun korvani; minä en ole niskoitellut, en vetäytynyt pois.
\par 6 Selkäni minä annoin lyötäväksi, poskieni parran revittäväksi, en peittänyt kasvojani pilkalta ja syljeltä.
\par 7 Herra, Herra auttaa minua; sentähden ei minuun pilkka koskenut, sentähden tein kasvoni koviksi kuin piikivi, sillä minä tiedän, etten häpeään joudu.
\par 8 Lähellä on hän, joka minut vanhurskaaksi tuomitsee. Kuka voi minun kanssani riidellä? Astukaamme yhdessä esiin! Kuka on minun vastapuoleni? Tulkoon tänne lähelleni!
\par 9 Katso, Herra, Herra auttaa minua; kuka minut syylliseksi tuomitsee? Katso, kaikki he hajoavat kuin vaate; koi heidät syö.
\par 10 Kuka teistä pelkää Herraa ja kuulee hänen palvelijansa ääntä? Joka vaeltaa pimeydessä ja valoa vailla, se luottakoon Herran nimeen ja turvautukoon Jumalaansa.
\par 11 Mutta katso, te kaikki, jotka palon sytytätte, te palavilla nuolilla varustetut, suistukaa oman tulenne liekkeihin, niihin palaviin nuoliin, jotka olette sytyttäneet. Minun kädestäni tämä teille tulee; vaivassa täytyy teidän asua.

\chapter{51}

\par 1 Kuulkaa minua, te jotka vanhurskauteen pyritte, te jotka Herraa etsitte. Katsokaa kalliota, josta olette lohkaistut, ja kaivos-onkaloa, josta olette kaivetut.
\par 2 Katsokaa Aabrahamia, isäänne, ja Saaraa, joka teidät synnytti. Sillä hän oli vain yksi, kun minä hänet kutsuin; mutta minä siunasin hänet ja enensin hänet.
\par 3 Niin Herra lohduttaa Siionin, lohduttaa kaikki sen rauniot, hän tekee sen erämaasta kuin Eedenin ja sen arosta kuin Herran puutarhan; siellä on oleva riemu ja ilo, kiitos ja ylistysvirren ääni.
\par 4 Kuuntele minua, kansani, kuule minua, kansakuntani, sillä minusta lähtee laki, ja minä panen oikeuteni valkeudeksi kansoille.
\par 5 Lähellä on minun vanhurskauteni, minun autuuteni ilmestyy, minun käsivarteni tuomitsevat kansat; minua odottavat merensaaret ja panevat toivonsa minun käsivarteeni.
\par 6 Nostakaa silmänne taivasta kohti ja katsokaa maata, joka alhaalla on, sillä taivaat katoavat kuin savu ja maa hajoaa kuin vaate ja sen asukkaat kuolevat kuin sääsket, mutta minun autuuteni pysyy iankaikkisesti, ja minun vanhurskauteni ei kukistu.
\par 7 Kuulkaa minua, te jotka vanhurskauden tunnette, kansa, jonka sydämessä on minun lakini: älkää peljätkö ihmisten pilkkaa älkääkä kauhistuko heidän herjauksiansa.
\par 8 Sillä koi syö heidät niinkuin vaatteen, koiperhonen syö heidät niinkuin villan, mutta minun vanhurskauteni pysyy iankaikkisesti, minun autuuteni polvesta polveen.
\par 9 Heräjä, heräjä, pukeudu voimaan, sinä Herran käsivarsi; heräjä niinkuin muinaisina päivinä, ammoisten sukupolvien aikoina. Etkö sinä ole se, joka löit Rahabin kuoliaaksi, joka lävistit lohikäärmeen?
\par 10 Etkö sinä ole se, joka kuivasit meren, suuren syvyyden vedet, joka teit meren syvänteet tieksi lunastettujen kulkea?
\par 11 Niin Herran vapahdetut palajavat ja tulevat Siioniin riemuiten, päänsä päällä iankaikkinen ilo. Riemu ja ilo saavuttavat heidät, mutta murhe ja huokaus pakenevat.
\par 12 Minä, minä olen teidän lohduttajanne; mikä olet sinä, että pelkäät ihmistä, joka on kuolevainen, ihmislasta, jonka käy niinkuin ruohon,
\par 13 ja unhotat Herran, joka on sinut tehnyt, joka on levittänyt taivaan ja perustanut maan, ja vapiset alati, kaiket päivät, sortajan vihaa, kun hän tähtää tuhotaksensa? Mutta missä on sortajan viha?
\par 14 Pian päästetään kumaraan koukistunut vapaaksi kahleistaan: ei hän kuole, ei kuoppaan jää, eikä häneltä leipä puutu.
\par 15 Minä olen Herra, sinun Jumalasi, joka liikutan meren, niin että sen aallot pauhaavat, jonka nimi on Herra Sebaot.
\par 16 Ja minä olen pannut sanani sinun suuhusi, minä olen kätkenyt sinut käteni varjoon, pystyttääkseni taivaan ja perustaakseni maan ja sanoakseni Siionille: "Sinä olet minun kansani".
\par 17 Heräjä, heräjä, nouse, Jerusalem, sinä joka olet juonut Herran kädestä hänen vihansa maljan, joka olet päihdyttävän pikarin juonut, tyhjäksi särpinyt.
\par 18 Ei kukaan kaikista lapsista, jotka hän oli synnyttänyt, ollut häntä taluttamassa; ei kukaan kaikista lapsista, jotka hän oli kasvattanut, hänen käteensä tarttunut.
\par 19 Nämä kohtasivat sinua kaksittain - kuka sinua surkuttelee - tuho ja turmio, nälkä ja miekka; millä voin sinua lohduttaa?
\par 20 Tajuttomina makasivat sinun poikasi joka kadun kulmassa, niinkuin antiloopit pyydyshaudassa, täynnä Herran vihaa, sinun Jumalasi nuhtelua.
\par 21 Sentähden kuule tätä, sinä poloinen, joka olet juopunut, vaikka et viinistä:
\par 22 Näin sanoo sinun Herrasi, Herra sinun Jumalasi, joka ajaa kansansa asian: Katso, minä otan sinun kädestäsi päihdyttävän maljan, vihani pikarin; ei tarvitse sinun siitä enää juoda.
\par 23 Ja minä panen sen sinun vaivaajaisi käteen, jotka sinulle sanoivat: "Lankea maahan, kulkeaksemme sinun päällitsesi"; ja sinä panit selkäsi maaksi ja kaduksi kulkijoille.

\chapter{52}

\par 1 Heräjä, heräjä, pukeudu voimaasi, Siion; pukeudu juhlapukuusi, Jerusalem, sinä pyhä kaupunki. Sillä ei koskaan enää astu sinun sisällesi ympärileikkaamaton eikä saastainen.
\par 2 Pudista päältäsi tomu, nouse istuimellesi, Jerusalem; irroita kahleet kaulastasi, sinä vangittu tytär Siion.
\par 3 Sillä näin sanoo Herra: Ilmaiseksi teidät myytiin, rahatta teidät lunastetaan.
\par 4 Sillä näin sanoo Herra, Herra: Minun kansani meni ensin alas Egyptiin asumaan siellä muukalaisena, ja sitten Assur sorti sitä ilman syytä.
\par 5 Ja nyt, mitä minulla on tekemistä täällä, sanoo Herra, kun minun kansani on viety pois ilmaiseksi? Sen valtiaat elämöivät, sanoo Herra, ja minun nimeäni pilkataan alati, kaiket päivät.
\par 6 Sentähden minun kansani on tunteva minun nimeni, sentähden se on tunteva sinä päivänä, että minä olen se, joka sanon: "Katso, tässä minä olen".
\par 7 Kuinka suloiset ovat vuorilla ilosanoman tuojan jalat, hänen, joka julistaa rauhaa, ilmoittaa hyvän sanoman, joka julistaa pelastusta, sanoo Siionille: "Sinun Jumalasi on kuningas!"
\par 8 Kuule! Vartijasi korottavat äänensä, kaikki he riemuitsevat, sillä he näkevät silmästä silmään, kuinka Herra palajaa Siioniin.
\par 9 Huutakaa ilosta, riemuitkaa, kaikki te Jerusalemin rauniot, sillä Herra lohduttaa kansansa, lunastaa Jerusalemin.
\par 10 Herra paljastaa pyhän käsivartensa kaikkien kansojen nähden, ja kaikki maan ääret saavat nähdä meidän Jumalamme autuuden.
\par 11 Pois, pois! Lähtekää sieltä, älkää koskeko saastaiseen; lähtekää sen keskeltä, puhdistautukaa, te Herran aseenkantajat.
\par 12 Sillä ei teidän tarvitse kiiruusti lähteä, ei paeten kulkea; sillä Herra käy teidän edellänne, Israelin Jumala seuraa suojananne.
\par 13 Katso, minun palvelijani menestyy, hän on nouseva, kohoava ja sangen korkea oleva.
\par 14 Niinkuin monet kauhistuivat häntä - sillä niin runneltu, ei enää ihmisenkaltainen, oli hänen muotonsa, hänen hahmonsa ei ollut ihmislasten hahmo -
\par 15 niin hän on saattava ihmetyksiin monet kansat, hänen tähtensä kuninkaat sulkevat suunsa. Sillä mitä heille ei ikinä ole kerrottu, sen he saavat nähdä, mitä he eivät ole kuulleet, sen he saavat havaita.

\chapter{53}

\par 1 Kuka uskoo meidän saarnamme, kenelle Herran käsivarsi ilmoitetaan?
\par 2 Hän kasvoi Herran edessä niinkuin vesa, niinkuin juuri kuivasta maasta. Ei ollut hänellä vartta eikä kauneutta; me näimme hänet, mutta ei ollut hänellä muotoa, johon me olisimme mielistyneet.
\par 3 Hän oli ylenkatsottu, ihmisten hylkäämä, kipujen mies ja sairauden tuttava, jota näkemästä kaikki kasvonsa peittivät, halveksittu, jota emme minäkään pitäneet.
\par 4 Mutta totisesti, meidän sairautemme hän kantoi, meidän kipumme hän sälytti päällensä. Me pidimme häntä rangaistuna, Jumalan lyömänä ja vaivaamana,
\par 5 mutta hän on haavoitettu meidän rikkomustemme tähden, runneltu meidän pahain tekojemme tähden. Rangaistus oli hänen päällänsä, että meillä rauha olisi, ja hänen haavainsa kautta me olemme paratut.
\par 6 Me vaelsimme kaikki eksyksissä niinkuin lampaat, kukin meistä poikkesi omalle tielleen. Mutta Herra heitti hänen päällensä kaikkien meidän syntivelkamme.
\par 7 Häntä piinattiin, ja hän alistui siihen eikä suutansa avannut; niinkuin karitsa, joka teuraaksi viedään, niinkuin lammas, joka on ääneti keritsijäinsä edessä, niin ei hän suutansa avannut.
\par 8 Ahdistettuna ja tuomittuna hänet otettiin pois, mutta kuka hänen polvikunnastaan sitä ajatteli? Sillä hänet temmattiin pois elävien maasta; minun kansani rikkomuksen tähden kohtasi rangaistus häntä.
\par 9 Hänelle annettiin hauta jumalattomain joukossa; mutta rikkaan tykö hän tuli kuoltuansa, sillä hän ei ollut vääryyttä tehnyt eikä petosta ollut hänen suussansa.
\par 10 Mutta Herra näki hyväksi runnella häntä, lyödä hänet sairaudella. Jos sinä panet hänen sielunsa vikauhriksi, saa hän nähdä jälkeläisiä ja elää kauan, ja Herran tahto toteutuu hänen kauttansa.
\par 11 Sielunsa vaivan tähden hän saa nähdä sen ja tulee ravituksi. Tuntemuksensa kautta hän, minun vanhurskas palvelijani, vanhurskauttaa monet, sälyttäen päällensä heidän pahat tekonsa.
\par 12 Sentähden minä jaan hänelle osan suurten joukossa, ja väkevien kanssa hän saalista jakaa; sillä hän antoi sielunsa alttiiksi kuolemaan, ja hänet luettiin pahantekijäin joukkoon, hän kantoi monien synnit, ja hän rukoili pahantekijäin puolesta.

\chapter{54}

\par 1 Riemuitse, sinä hedelmätön, joka et ole synnyttänyt, huuda ilosta ja riemahda, sinä, joka et ole synnytyskivuissa ollut. Sillä hyljätyllä on lapsia enemmän kuin aviovaimolla, sanoo Herra.
\par 2 Tee avaraksi telttasi sija, levennettäköön sinun majojesi seinien kangas. Älä säästele! Pidennä telttaköytesi ja vahvista vaarnasi.
\par 3 Sillä sinä olet leviävä oikealle ja vasemmalle, sinun jälkeläisesi ottavat omiksensa kansat ja tekevät autiot kaupungit asutuiksi.
\par 4 Älä pelkää, sillä et sinä häpeään joudu; älä ole häpeissäsi, sillä et sinä ole pettyvä. Nuoruutesi häpeän sinä olet unhottava, leskeytesi pilkkaa et ole enää muistava.
\par 5 Sillä hän, joka sinut teki, on sinun aviomiehesi, Herra Sebaot on hänen nimensä, sinun lunastajasi on Israelin Pyhä, hän joka kaiken maan Jumalaksi kutsutaan.
\par 6 Sillä niinkuin hyljätyn, syvästi murheellisen vaimon on Herra sinut kutsunut - nuoruuden vaimon, joka on ollut halveksittu, sanoo sinun Jumalasi.
\par 7 Vähäksi silmänräpäykseksi minä hylkäsin sinut, mutta minä kokoan sinut jälleen suurella laupeudella.
\par 8 Ylitsevuotavassa vihassani minä peitin sinulta kasvoni silmänräpäykseksi, mutta minä armahdan sinut iankaikkisella armolla, sanoo Herra, sinun lunastajasi.
\par 9 Sillä tämä on minulle, niinkuin olivat Nooan vedet: niinkuin minä vannoin, etteivät Nooan vedet enää tulvi maan ylitse, niin minä vannon, etten enää vihastu sinuun enkä sinua nuhtele.
\par 10 Sillä vuoret väistykööt ja kukkulat horjukoot, mutta minun armoni ei sinusta väisty, eikä minun rauhanliittoni horju, sanoo Herra, sinun armahtajasi.
\par 11 Sinä kurja, myrskyn raastama, sinä lohduton! Katso, minä muuraan sinun kivesi kiiltokivellä, panen sinun perustuksesi safiireista,
\par 12 minä teen sinun harjasi rubiineista ja sinun porttisi kristalleista ja koko sinun ympärysmuurisi jalokivistä.
\par 13 Sinun lapsesi ovat kaikki Herran opetuslapsia, ja suuri rauha on sinun lapsillasi oleva.
\par 14 Sinut vahvistetaan vanhurskaudella; sinä olet oleva kaukana väkivallasta, sillä ei sinulla ole pelkäämistä, ja kaukana hävityksestä, sillä ei se sinua lähesty.
\par 15 Jos sinun kimppuusi karataan, ei se ole minusta; joka kimppuusi karkaa, se eteesi kaatuu.
\par 16 Katso, minä olen luonut sepän, joka lietsoo hiilivalkeata ja kuonnuttaa aseen käytäntöönsä; mutta minä olen myös luonut tuhontuottajan hävittämään sen.
\par 17 Jokainen ase, joka valmistetaan sinun varallesi, on oleva tehoton; ja jokaisen kielen, joka nousee käymään sinun kanssasi oikeutta, sinä osoitat vääräksi. Tämä on Herran palvelijain perintöosa, tämä heidän vanhurskautensa, minulta saatu, sanoo Herra.

\chapter{55}

\par 1 Kuulkaa, kaikki janoavaiset, tulkaa veden ääreen. Tekin, joilla ei ole rahaa, tulkaa, ostakaa ja syökää; tulkaa, ostakaa ilman rahatta, ilman hinnatta viiniä ja maitoa.
\par 2 Miksi annatte rahan siitä, mikä ei ole leipää, ja työnne ansion siitä, mikä ei ravitse? Kuulkaa minua, niin saatte syödä hyvää, ja teidän sielunne virvoittuu lihavuuden ääressä.
\par 3 Kallistakaa korvanne ja tulkaa minun tyköni; kuulkaa, niin teidän sielunne saa elää. Ja minä teen teidän kanssanne iankaikkisen liiton, annan lujat Daavidin armot.
\par 4 Katso, hänet minä asetin kansoille todistajaksi, kansojen ruhtinaaksi ja käskijäksi.
\par 5 Katso, sinä olet kutsuva pakanoita, joita sinä et tunne, ja pakanat, jotka eivät sinua tunne, rientävät sinun tykösi Herran, sinun Jumalasi, tähden, Israelin Pyhän tähden, sillä hän kirkastaa sinut.
\par 6 Etsikää Herraa silloin, kun hänet löytää voidaan; huutakaa häntä avuksi, kun hän läsnä on.
\par 7 Jumalaton hyljätköön tiensä ja väärintekijä ajatuksensa ja palatkoon Herran tykö, niin hän armahtaa häntä, ja meidän Jumalamme tykö, sillä hänellä on paljon anteeksiantamusta.
\par 8 Sillä minun ajatukseni eivät ole teidän ajatuksianne, eivätkä teidän tienne ole minun teitäni, sanoo Herra.
\par 9 Vaan niin paljon korkeampi kuin taivas on maata, ovat minun tieni korkeammat teidän teitänne ja minun ajatukseni teidän ajatuksianne.
\par 10 Sillä niinkuin sade ja lumi, joka taivaasta tulee, ei sinne palaja, vaan kostuttaa maan, tekee sen hedelmälliseksi ja kasvavaksi, antaa kylväjälle siemenen ja syöjälle leivän,
\par 11 niin on myös minun sanani, joka minun suustani lähtee: ei se minun tyköni tyhjänä palaja, vaan tekee sen, mikä minulle otollista on, ja saa menestymään sen, mitä varten minä sen lähetin.
\par 12 Sillä iloiten te lähdette, ja rauhassa teitä saatetaan; vuoret ja kukkulat puhkeavat riemuun teidän edessänne, ja kaikki kedon puut paukuttavat käsiänsä.
\par 13 Orjantappurain sijaan on kasvava kypressejä, nokkosten sijaan on kasvava myrttipuita; ja se tulee Herran kunniaksi, iankaikkiseksi merkiksi, joka ei häviä.

\chapter{56}

\par 1 Näin sanoo Herra: Noudattakaa oikeutta ja tehkää vanhurskaus, sillä minun autuuteni on lähellä ja minun vanhurskauteni ilmestyy.
\par 2 Autuas se ihminen, joka tämän tekee, se ihmislapsi, joka tässä pysyy, joka pitää sapatin eikä sitä riko, joka varoo kätensä tekemästä mitään pahaa!
\par 3 Älköön sanoko muukalainen, joka on liittynyt Herraan: "Herra erottaa minut peräti kansastansa", älköönkä kuohittu sanoko: "Minä olen kuiva puu".
\par 4 Sillä näin sanoo Herra: Kuohituille, jotka pitävät minun sapattini ja valitsevat sen, mikä minulle otollista on, ja pysyvät minun liitossani,
\par 5 heille minä annan huoneessani ja muurieni sisällä muistomerkin ja nimen, joka on poikia ja tyttäriä parempi; minä annan heille iankaikkisen nimen, joka ei häviä.
\par 6 Ja muukalaiset, jotka ovat liittyneet Herraan, palvellakseen häntä ja rakastaakseen Herran nimeä, ollakseen hänen palvelijoitansa, kaikki, jotka pitävät sapatin eivätkä sitä riko ja pysyvät minun liitossani,
\par 7 ne minä tuon pyhälle vuorelleni ja ilahutan heitä rukoushuoneessani, ja heidän polttouhrinsa ja teurasuhrinsa ovat otolliset minun alttarillani, sillä minun huoneeni on kutsuttava kaikkien kansojen rukoushuoneeksi.
\par 8 Herra, Herra sanoo, hän, joka kokoaa Israelin karkoitetut: Minä kokoan vielä muitakin sen koottujen lisäksi.
\par 9 Kaikki kedon eläimet, tulkaa syömään, te metsän eläimet kaikki.
\par 10 Israelin vartijat ovat kaikki sokeita, eivät he mitään käsitä; he ovat kaikki mykkiä koiria, jotka eivät osaa haukkua. He näkevät unta, makailevat ja nukkuvat mielellään.
\par 11 Ja näillä koirilla on vimmainen nälkä, ei niitä mikään täytä. Ja tällaisia ovat paimenet! Eivät pysty mitään huomaamaan, ovat kaikki kääntyneet omille teilleen, etsivät kukin omaa voittoansa, kaikki tyynni.
\par 12 "Tulkaa, minä hankin viiniä, ryypätkäämme väkevätä; olkoon huomispäivä niinkuin tämäkin ylenpalttisen ihana."

\chapter{57}

\par 1 Vanhurskas hukkuu, eikä kukaan pane sitä sydämelleen, hurskaat miehet otetaan pois kenenkään siitä välittämättä, sillä vanhurskas otetaan pois pahuutta näkemästä.
\par 2 Hän menee rauhaan: jotka vakaasti vaeltavat, he saavat levätä kammioissansa.
\par 3 Mutta te tulkaa tänne, te velhottaren lapset, te avionrikkojamiehen ja porttovaimon sikiöt.
\par 4 Kenellä te iloanne pidätte? Kenelle avaatte suunne ammolleen ja kieltä pitkälle pistätte? Ettekö te ole rikoksen lapsia, valheen sikiöitä?
\par 5 Te, jotka hehkutte himosta tammien varjossa, jokaisen vihreän puun alla, te, jotka teurastatte lapsia laaksoissa, kallionrotkoissa!
\par 6 Laakson sileät paadet ovat sinun osasi, siinä se on, sinun arpasi; niille sinä olet juomauhrit vuodattanut, ruokauhrit uhrannut. Siihenkö minä tyytyisin!
\par 7 Korkealle ja valtavalle vuorelle sinä valmistit vuoteesi; sinne sinä myös nousit teurasuhria uhraamaan.
\par 8 Oven ja pihtipielen taakse sinä panit omat merkkisi. Sillä minusta luopuen sinä paljastit itsesi ja nousit vuoteellesi, teit sen tilavaksi; sinä sovit kaupoista heidän kanssaan, makasit heidän kanssaan mielelläsi, näit heidän häpynsä.
\par 9 Sinä kuljit kuninkaan tykö öljyinesi, runsaine voiteinesi; sinä lähetit sanansaattajasi kauas, laskeuduit alas tuonelaan asti.
\par 10 Sinä väsyit matkasi pituudesta, et kuitenkaan sanonut: "Turha vaiva!" Sinä sait elpynyttä voimaa, sentähden et heikoksi käynyt.
\par 11 Ketä sinä arkailit ja pelkäsit, koska vilpistelit etkä minua muistanut, et minusta välittänyt? Eikö niin: minä olen ollut vaiti aina ikiajoista asti, ja niin et sinä minua pelkää?
\par 12 Mutta minä ilmoitan, mitä on sinun vanhurskautesi ja sinun tekosi; ne eivät sinua auta.
\par 13 Kun sinä huudat, auttakoon sinua jumaliesi joukko! Mutta tuuli vie ne kaikki, henkäys ottaa ne pois. Mutta joka minuun turvaa, se perii maan ja ottaa omaksensa minun pyhän vuoreni.
\par 14 Hän sanoo: Tehkää, tehkää tie, tasoittakaa tie, poistakaa kompastuskivet minun kansani tieltä.
\par 15 Sillä näin sanoo Korkea ja Ylhäinen, jonka asumus on iankaikkinen ja jonka nimi on Pyhä: Minä asun korkeudessa ja pyhyydessä ja niitten tykönä, joilla on särjetty ja nöyrä henki, että minä virvoittaisin nöyrien hengen ja saattaisin särjettyjen sydämet eläviksi.
\par 16 Sillä en minä iankaiken riitele enkä vihastu ainiaaksi; muutoin henki nääntyisi minun kasvojeni edessä, ne sielut, jotka minä tehnyt olen.
\par 17 Hänen ahneutensa synnin tähden minä vihastuin; minä löin häntä, kätkin itseni ja olin vihastunut. Mutta hän luopui minusta ja kulki oman sydämensä tietä.
\par 18 Minä olen nähnyt hänen tiensä, mutta minä parannan hänet ja johdatan häntä ja annan jälleen lohdutuksen hänelle ja hänen surevillensa.
\par 19 Minä luon huulten hedelmän, rauhan, rauhan kaukaisille ja läheisille, sanoo Herra, ja minä parannan hänet.
\par 20 Mutta jumalattomat ovat kuin kuohuva meri, joka ei voi tyyntyä ja jonka aallot kuohuttavat muraa ja mutaa.
\par 21 Jumalattomilla ei ole rauhaa, sanoo minun Jumalani.

\chapter{58}

\par 1 Huuda täyttä kurkkua, älä säästä, korota äänesi niinkuin pasuna, ilmoita minun kansalleni heidän rikoksensa ja Jaakobin huoneelle heidän syntinsä.
\par 2 Minua he muka etsivät joka päivä ja haluavat tietoa minun teistäni niinkuin kansa, joka tekee vanhurskautta eikä hylkää Jumalansa oikeutta. He vaativat minulta vanhurskaita tuomioita, haluavat, että Jumala heitä lähestyisi:
\par 3 "Miksi me paastoamme, kun et sinä sitä näe, kuritamme itseämme, kun et sinä sitä huomaa?" Katso, paastopäivänänne te ajatte omia asioitanne ja ahdistatte työhön kaiken työväkenne.
\par 4 Katso, riidaksi ja toraksi te paastoatte, lyödäksenne jumalattomalla nyrkillä. Te ette nyt paastoa niin, että teidän äänenne kuultaisiin korkeudessa.
\par 5 Tällainenko on se paasto, johon minä mielistyn, se päivä, jona ihminen kurittaa itseänsä? Jos kallistaa päänsä kuin kaisla ja makaa säkissä ja tuhassa, sitäkö sinä sanot paastoksi ja päiväksi, joka on Herralle otollinen?
\par 6 Eikö tämä ole paasto, johon minä mielistyn: että avaatte vääryyden siteet, irroitatte ikeen nuorat, ja päästätte sorretut vapaiksi, että särjette kaikki ikeet?
\par 7 Eikö tämä: että taitat leipäsi isoavalle ja viet kurjat kulkijat huoneeseesi, kun näet alastoman, vaatetat hänet etkä kätkeydy siltä, joka on omaa lihaasi?
\par 8 Silloin sinun valkeutesi puhkeaa esiin niinkuin aamurusko, ja haavasi kasvavat nopeasti umpeen; sinun vanhurskautesi käy sinun edelläsi, ja Jumalan kunnia seuraa suojanasi.
\par 9 Silloin sinä rukoilet, ja Herra vastaa, sinä huudat, ja hän sanoo: "Katso, tässä minä olen". Jos sinä keskuudestasi poistat ikeen, sormella-osoittelun ja vääryyden puhumisen,
\par 10 jos taritset elannostasi isoavalle ja ravitset vaivatun sielun, niin valkeus koittaa sinulle pimeydessä, ja sinun pilkkopimeäsi on oleva niinkuin keskipäivä.
\par 11 Ja Herra johdattaa sinua alati ja ravitsee sinun sielusi kuivissa erämaissa; hän vahvistaa sinun luusi, ja sinä olet oleva niinkuin runsaasti kasteltu puutarha, niinkuin lähde, josta vesi ei koskaan puutu.
\par 12 Sinun jälkeläisesi rakentavat jälleen ikivanhat rauniot, sinä kohotat perusmuurit, muinaisten polvien laskemat; ja sinun nimesi on oleva: "halkeamain umpeenmuuraaja" ja "teitten korjaaja maan asuttamiseksi".
\par 13 Jos sinä pidätät jalkasi sapattia rikkomasta, niin ettet toimita omia asioitasi minun pyhäpäivänäni, vaan kutsut sapatin ilopäiväksi, Herran pyhäpäivän kunnioitettavaksi ja kunnioitat sitä, niin ettet toimita omia toimiasi, et aja omia asioitasi etkä puhu joutavia,
\par 14 silloin on ilosi oleva Herrassa, ja minä kuljetan sinut maan kukkuloitten ylitse, ja minä annan sinun nauttia isäsi Jaakobin perintöosaa. Sillä Herran suu on puhunut.

\chapter{59}

\par 1 Katso, ei Herran käsi ole liian lyhyt auttamaan, eikä hänen korvansa kuuro kuulemaan:
\par 2 vaan teidän pahat tekonne erottavat teidät Jumalastanne, ja teidän syntinne peittävät teiltä hänen kasvonsa, niin ettei hän kuule.
\par 3 Sillä teidän kätenne ovat tahratut verellä ja sormenne vääryydellä; teidän huulenne puhuvat valhetta, teidän kielenne latelee petosta.
\par 4 Ei kukaan vaadi oikeuteen vanhurskaasti, eikä kukaan käräjöi rehellisesti. He turvautuvat tyhjään ja puhuvat vilppiä, kantavat kohdussaan tuhoa ja synnyttävät turmion.
\par 5 Myrkkyliskon munia he hautovat, hämähäkin verkkoja he kutovat; joka niitä munia syö, se kuolee, rikkipoljetusta puhkeaa kyykäärme.
\par 6 Heidän verkkonsa eivät kelpaa vaatteeksi, heidän tekemäänsä ei voi verhoutua; heidän työnsä ovat vääryyden töitä, ja heidän kätensä ovat täynnä väkivallan tekoa.
\par 7 Heidän jalkansa juoksevat pahuuteen, kiiruhtavat vuodattamaan viatonta verta; heidän ajatuksensa ovat vääryyden ajatuksia, tuho ja turmio on heidän teillänsä.
\par 8 Rauhan tietä he eivät tunne, oikeutta ei ole heidän askeleissansa; polkunsa he tekevät mutkaisiksi, ei kukaan, joka niitä käy, tunne rauhaa.
\par 9 Sentähden on oikeus meistä kaukana, eikä vanhurskaus saavuta meitä; me odotamme valoa, mutta katso, on pimeä, aamunkoittoa, mutta vaellamme yön synkeydessä.
\par 10 Me haparoimme seinää pitkin niinkuin sokeat, haparoimme niinkuin silmiä vailla; me kompastelemme sydänpäivällä niinkuin hämärässä, me olemme terveitten keskellä niinkuin kuolleet.
\par 11 Me murisemme kaikki kuin karhut ja kujerramme kuin kyyhkyset; me odotamme oikeutta, mutta sitä ei tule, pelastusta, mutta se on kaukana meistä.
\par 12 Sillä meidän rikoksemme ovat monilukuiset sinun edessäsi, ja meidän syntimme todistavat meitä vastaan; sillä meidän rikoksemme seuraavat meitä, ja pahat tekomme me tunnemme:
\par 13 me olemme luopuneet Herrasta ja kieltäneet hänet, vetäytyneet pois Jumalaamme seuraamasta, puhuneet sortoa ja kapinaa, kantaneet kohdussamme ja purkaneet sisimmästämme valheen sanoja.
\par 14 Oikeus työnnetään takaperin, ja vanhurskaus seisoo kaukana, sillä totuus kompastelee torilla, suoruus ei voi sisälle tulla.
\par 15 Niin oli totuus kadonnut, ja joka pahasta luopui, se ryöstettiin paljaaksi. Herra näki sen, ja se oli hänen silmissänsä paha, ettei ollut oikeutta.
\par 16 Ja hän näki, ettei ollut yhtäkään miestä, ja hän ihmetteli, ettei kukaan astunut väliin. Silloin hänen oma käsivartensa auttoi häntä, ja hänen vanhurskautensa häntä tuki.
\par 17 Ja hän puki yllensä vanhurskauden kuin rintahaarniskan ja pani pelastuksen kypärin päähänsä, hän puki koston vaatteet puvuksensa ja verhoutui kiivauteen niinkuin viittaan.
\par 18 Tekojen mukaan hän maksaa palkan: vihan vastustajillensa, koston vihollisillensa; merensaarille hän kostaa.
\par 19 Ja päivän laskun äärillä he pelkäävät Herran nimeä ja päivän koittamilla hänen kunniaansa. Sillä se tulee kuin padottu virta, jota Herran henki ajaa eteenpäin.
\par 20 Mutta Siionille se tulee lunastajana, niille Jaakobissa, jotka synnistä kääntyvät, sanoo Herra.
\par 21 Ja tämä on minun liittoni heidän kanssansa, sanoo Herra: minun Henkeni, joka on sinun päälläsi, ja minun sanani, jonka minä suuhusi panen, eivät väisty sinun suustasi, eivät lastesi suusta eivätkä lastesi lasten suusta, sanoo Herra, nyt ja iankaikkisesti.

\chapter{60}

\par 1 Nouse, ole kirkas, sillä sinun valkeutesi tulee, ja Herran kunnia koittaa sinun ylitsesi.
\par 2 Sillä katso, pimeys peittää maan ja synkeys kansat, mutta sinun ylitsesi koittaa Herra, ja sinun ylläsi näkyy hänen kunniansa.
\par 3 Kansat vaeltavat sinun valkeuttasi kohti, kuninkaat sinun koitteesi kirkkautta kohti.
\par 4 Nosta silmäsi, katso ympärillesi: kaikki nämä ovat kokoontuneet, tulevat sinun tykösi; sinun poikasi tulevat kaukaa, sinun tyttäriäsi kainalossa kannetaan.
\par 5 Silloin sinä saat sen nähdä, ja sinä loistat ilosta, sinun sydämesi sykkii ja avartuu, kun meren aarteet kääntyvät sinun tykösi, kansojen rikkaudet tulevat sinulle.
\par 6 Kamelien paljous peittää sinut, Midianin ja Eefan varsat; kaikki tulevat Sabasta, kantavat kultaa ja suitsutusta ja Herran ylistystä ilmoittavat.
\par 7 Kaikki Keedarin laumat kokoontuvat sinun tykösi, Nebajotin oinaat palvelevat sinua; minulle otollisina ne nousevat minun alttarilleni, ja minä kirkastan kirkkauteni huoneen.
\par 8 Keitä ovat nuo, jotka lentävät niinkuin pilvet ja niinkuin kyyhkyset lakkoihinsa?
\par 9 Merensaaret odottavat minua, ja etumaisina tulevat Tarsiin-laivat tuodakseen sinun lapsesi kaukaa; hopeansa ja kultansa heillä on mukanansa Herran, sinun Jumalasi, nimelle, Israelin Pyhälle, sillä hän kirkastaa sinut.
\par 10 Ja muukalaiset rakentavat sinun muurisi, ja heidän kuninkaansa palvelevat sinua; sillä vihassani minä löin sinua, mutta mielisuosiossani minä sinua armahdan.
\par 11 Sinun porttisi pidetään aina auki, ei niitä suljeta päivällä eikä yöllä, että kansojen rikkaudet tuotaisiin ja heidän kuninkaansa saatettaisiin sinun tykösi.
\par 12 Sillä se kansa tai valtakunta, joka ei sinua palvele, hukkuu, ja ne kansat hävitetään perinjuurin.
\par 13 Libanonin kunnia tulee sinun tykösi, kypressit, jalavat ynnä hopeakuuset, kaunistamaan minun pyhäkköni paikkaa, ja minä saatan jalkaini sijan kunniaan.
\par 14 Ja kumarassa käyden tulevat sinun tykösi sinun sortajaisi pojat, ja kaikki sinun pilkkaajasi heittäytyvät sinun jalkaisi juureen. He nimittävät sinut "Herran kaupungiksi", "Israelin Pyhän Siioniksi".
\par 15 Sen sijaan, että sinä olet ollut hyljätty ja vihattu, niin ettei ollut kauttasi kulkijaa, teen minä sinut korkeaksi iankaikkisesti, iloksi polvesta polveen.
\par 16 Ja sinä saat imeä kansojen maidon, imeä kuningasten rintoja, saat tuntea, että minä, Herra, olen sinun auttajasi, että Jaakobin Väkevä on sinun lunastajasi.
\par 17 Minä tuon vasken sijaan kultaa, ja raudan sijaan minä tuon hopeata, puun sijaan vaskea ja kiven sijaan rautaa. Ja minä panen sinulle esivallaksi rauhan ja käskijäksi vanhurskauden.
\par 18 Ei kuulu enää väkivaltaa sinun maassasi, ei tuhoa, ei turmiota sinun rajaisi sisällä, ja sinä kutsut pelastuksen muuriksesi ja kiitoksen portiksesi.
\par 19 Ei ole enää aurinko sinun valonasi päivällä, eikä valaise sinua kuun kumotus, vaan Herra on sinun iankaikkinen valkeutesi, ja sinun Jumalasi sinun kirkkautesi.
\par 20 Ei sinun aurinkosi enää laske, eikä sinun kuusi vajene, sillä Herra on sinun iankaikkinen valkeutesi, ja päättyneet ovat sinun murheesi päivät.
\par 21 Sinun kansassasi ovat kaikki vanhurskaita, he saavat periä maan iankaikkisesti, he, minun istutukseni vesa, minun kätteni teko, minun kirkkauteni ilmoitukseksi.
\par 22 Pienimmästä kasvaa heimo, vähäisimmästä väkevä kansa. Minä, Herra, sen aikanansa nopeasti täytän.

\chapter{61}

\par 1 Herran, Herran Henki on minun päälläni, sillä hän on voidellut minut julistamaan ilosanomaa nöyrille, lähettänyt minut sitomaan särjettyjä sydämiä, julistamaan vangituille vapautusta ja kahlituille kirvoitusta,
\par 2 julistamaan Herran otollista vuotta ja meidän Jumalamme kostonpäivää, lohduttamaan kaikkia murheellisia,
\par 3 panemaan Siionin murheellisten päähän - antamaan heille - juhlapäähineen tuhkan sijaan, iloöljyä murheen sijaan, ylistyksen vaipan masentuneen hengen sijaan; ja heidän nimensä on oleva "vanhurskauden tammet", "Herran istutus", hänen kirkkautensa ilmoitukseksi.
\par 4 Ja he rakentavat jälleen ikivanhat rauniot, kohottavat ennalleen esi-isien autiot paikat; ja he uudistavat rauniokaupungit, jotka ovat olleet autiot polvesta polveen.
\par 5 Vieraat ovat teidän laumojenne paimenina, muukalaiset teidän peltomiehinänne ja viinitarhureinanne.
\par 6 Mutta teitä kutsutaan Herran papeiksi, sanotaan meidän Jumalamme palvelijoiksi; te saatte nauttia kansain rikkaudet ja periä heidän kunniansa.
\par 7 Häpeänne hyvitetään teille kaksin kerroin, ja pilkatut saavat riemuita osastansa. Niin he saavat kaksinkertaisen perinnön maassansa; heillä on oleva iankaikkinen ilo.
\par 8 Sillä minä, Herra, rakastan oikeutta, vihaan vääryyttä ja ryöstöä; ja minä annan heille palkan uskollisesti ja teen heidän kanssansa iankaikkisen liiton.
\par 9 Heidän siemenensä tulee tunnetuksi kansain keskuudessa ja heidän jälkeläisensä kansakuntien keskellä; kaikki, jotka näkevät heitä, tuntevat heidät Herran siunaamaksi siemeneksi.
\par 10 Minä iloitsen suuresti Herrassa, minun sieluni riemuitsee minun Jumalassani, sillä hän pukee minun ylleni autuuden vaatteet ja verhoaa minut vanhurskauden viittaan, yljän kaltaiseksi, joka kantaa juhlapäähinettä niinkuin pappi, ja morsiamen kaltaiseksi, joka on koruillansa kaunistettu.
\par 11 Sillä niinkuin maa tuottaa kasvunsa ja niinkuin kasvitarha saa siemenkylvönsä versomaan, niin saattaa Herra, Herra versomaan vanhurskauden ja kiitoksen kaikkien kansojen nähden.

\chapter{62}

\par 1 Siionin tähden minä en voi vaieta, ja Jerusalemin tähden en lepoa saa, ennenkuin sen vanhurskaus nousee kuin aamunkoi ja sen autuus kuin palava tulisoihtu.
\par 2 Ja kansat näkevät sinun vanhurskautesi, kaikki kuninkaat sinun kunniasi; ja sinulle annetaan uusi nimi, jonka Herran suu säätää.
\par 3 Sinä tulet kauniiksi kruunuksi Herran kädessä ja kuninkaalliseksi päähineeksi Jumalasi kädessä.
\par 4 Ei sinua enää sanota "hyljätyksi", eikä sinun maatasi enää sanota "autioksi"; vaan sinua kutsutaan "minun rakkaakseni" ja sinun maatasi "aviovaimoksi", sillä Herra rakastaa sinua, ja sinun maasi otetaan avioksi.
\par 5 Sillä niinkuin nuori mies ottaa neitsyen avioksi, niin sinun lapsesi ottavat sinut omaksensa; ja niinkuin ylkä iloitsee morsiamesta, niin sinun Jumalasi iloitsee sinusta.
\par 6 Sinun muureillesi, Jerusalem, minä asetan vartijat; älkööt he milloinkaan vaietko, ei päivällä eikä yöllä. Te, jotka ylistätte Herraa, älkää itsellenne lepoa suoko.
\par 7 Älkää antako hänelle lepoa, ennenkuin hän on asettanut ennallensa Jerusalemin, tehnyt sen ylistykseksi maassa.
\par 8 Herra on vannonut oikean kätensä ja väkevyytensä käsivarren kautta: En totisesti minä enää anna jyviäsi sinun vihollistesi ruuaksi, eivätkä juo enää muukalaiset sinun viiniäsi, josta sinä olet vaivan nähnyt.
\par 9 Vaan jotka viljan kokoavat, ne sen syövät ja ylistävät Herraa; ja jotka viinin korjaavat, ne sen juovat minun pyhissä esikartanoissani.
\par 10 Käykää, käykää ulos porteista, tasoittakaa kansalle tie, tehkää, tehkää valtatie, raivatkaa kivet pois, kohottakaa lippu kansoille.
\par 11 Katso, Herra on kuuluttanut maan ääriin asti: Sanokaa tytär Siionille: Katso, sinun pelastuksesi tulee. Katso, hänen palkkansa on hänen mukanansa, hänen työnsä ansio käy hänen edellänsä.
\par 12 Ja heitä kutsutaan "pyhäksi kansaksi", "Herran lunastetuiksi"; ja sinua kutsutaan "halutuksi", "kaupungiksi, joka ei ole hyljätty".

\chapter{63}

\par 1 Kuka tuolla tulee Edomista, tulipunaisissa vaatteissa Bosrasta, tuo puvultansa komea, joka uljaana astelee suuressa voimassansa? "Minä, joka puhun vanhurskautta, joka olen voimallinen auttamaan."
\par 2 Miksi on punaa sinun puvussasi, miksi ovat vaatteesi kuin viinikuurnan polkijan?
\par 3 "Kuurnan minä poljin, minä yksinäni, ei ketään kansojen joukosta ollut minun kanssani; minä poljin heidät vihassani, tallasin heidät kiivaudessani, ja niin pirskui heidän vertansa vaatteilleni, ja minä tahrasin koko pukuni.
\par 4 Sillä koston päivä oli minun mielessäni ja minun lunastettujeni vuosi oli tullut.
\par 5 Ja minä katselin, mutta ei ollut auttajaa, ja minä ihmettelin, kun ei kukaan tueksi tullut. Silloin minun oma käsivarteni minua auttoi, ja minun vihani minua tuki.
\par 6 Ja minä tallasin kansat vihassani ja juovutin heidät kiivaudellani; minä vuodatin maahan heidän verensä."
\par 7 Herran armotöitä minä julistan, Herran ylistettäviä tekoja, kaikkea, mitä Herra on meille tehnyt, suurta hyvyyttä Israelin heimoa kohtaan, mitä hän on heille osoittanut laupeudessansa ja suuressa armossansa.
\par 8 Ja hän sanoi: "Ovathan he minun kansani, he ovat lapsia, joissa ei ole vilppiä"; ja niin hän tuli heille vapahtajaksi.
\par 9 Kaikissa heidän ahdistuksissansa oli hänelläkin ahdistus, ja hänen kasvojensa enkeli vapahti heidät. Rakkaudessaan ja sääliväisyydessään hän lunasti heidät, nosti heitä ja kantoi heitä kaikkina muinaisina päivinä.
\par 10 Mutta he niskoittelivat ja saattoivat murheelliseksi hänen Pyhän Henkensä; ja niin hän muuttui heidän viholliseksensa, hän itse soti heitä vastaan.
\par 11 Silloin hänen kansansa muisti muinaisia päiviä, muisti Moosesta: Missä on hän, joka toi heidät ylös merestä, heidät ynnä hänen laumansa paimenen? Missä on hän, joka pani tämän sydämeen Pyhän Henkensä;
\par 12 hän, joka antoi kunniansa käsivarren kulkea Mooseksen oikealla puolella, joka halkaisi vedet heidän edestänsä tehdäkseen itsellensä iankaikkisen nimen;
\par 13 hän, joka kuljetti heitä syvyyksissä niinkuin hevosia erämaassa, eivätkä he kompastelleet?
\par 14 Niinkuin karja astuu alas laaksoon, niin vei Herran Henki heidät lepoon. Niin sinä olet kansaasi johdattanut, tehdäksesi itsellesi kunniallisen nimen.
\par 15 Katsele taivaasta, katso pyhyytesi ja kirkkautesi asunnosta. Missä on sinun kiivautesi ja voimalliset tekosi? Sinun sydämesi sääli ja sinun armahtavaisuutesi ovat minulta sulkeutuneet.
\par 16 Sinähän olet meidän isämme, sillä Aabraham ei meistä tiedä eikä Israel meitä tunne: sinä, Herra, olet meidän isämme; "meidän Lunastajamme" on ikiajoista sinun nimesi.
\par 17 Miksi sallit meidän eksyä pois sinun teiltäsi, Herra, ja annoit meidän sydämemme paatua, niin ettemme sinua pelkää? Palaja takaisin palvelijaisi tähden, perintöosasi sukukuntain tähden.
\par 18 Vähän aikaa vain sinun pyhä kansasi sai pitää perintönsä; meidän ahdistajamme ovat tallanneet sinun pyhäkkösi.
\par 19 Me olemme niinkuin ne, joita sinä et ole ikinä hallinnut, olemme, niinkuin ei meitä olisi otettu sinun nimiisi.

\chapter{64}

\par 1 Oi, jospa sinä halkaisisit taivaat ja astuisit alas, niin että vuoret järkkyisivät sinun edessäsi, niinkuin risut leimahtavat tuleen, niinkuin vesi tulella kiehuu,
\par 2 että tekisit nimesi tunnetuksi vastustajillesi ja kansat vapisisivat sinun kasvojesi edessä,
\par 3 kun sinä teet peljättäviä tekoja, joita emme odottaa voineet! Oi, jospa astuisit alas, niin että vuoret järkkyisivät sinun edessäsi!
\par 4 Ei ole ikiajoista kuultu, ei ole korviin tullut, ei ole silmä nähnyt muuta Jumalaa, paitsi sinut, joka senkaltaisia tekisi häntä odottavalle.
\par 5 Sinä käyt niitä kohden, jotka ilolla vanhurskautta tekevät ja jotka sinun teilläsi sinua muistavat. Katso, sinä vihastuit, ja me jouduimme synnin alaisiksi; niin on ollut ikiajoista asti - saammeko avun?
\par 6 Kaikki me olimme kuin saastaiset, ja niinkuin tahrattu vaate oli kaikki meidän vanhurskautemme. Ja kaikki me olemme lakastuneet kuin lehdet, ja pahat tekomme heittelevät meitä niinkuin tuuli.
\par 7 Ei ole ketään, joka avukseen huutaa sinun nimeäsi, joka herää pitämään sinusta kiinni; sillä sinä olet peittänyt kasvosi meiltä, jättänyt meidät menehtymään syntiemme valtaan.
\par 8 Mutta olethan sinä, Herra, meidän isämme; me olemme savi, ja sinä olet meidän valajamme, kaikki me olemme sinun kättesi tekoa.
\par 9 Älä, Herra, vihastu ylenmäärin, äläkä ainiaan muistele pahoja tekoja. Katso ja huomaa, että me kaikki olemme sinun kansasi.
\par 10 Sinun pyhät kaupunkisi ovat tulleet erämaaksi, Siion on erämaaksi tullut, Jerusalem autioksi.
\par 11 Meidän pyhä ja ihana temppelimme, jossa isämme sinua ylistivät, on tulella poltettu, ja rauniona on kaikki meille kallisarvoinen.
\par 12 Vaikka nämä näin ovat, voitko, Herra, pidättää itsesi, voitko olla vaiti ja vaivata meitä ylenmäärin?

\chapter{65}

\par 1 Minä olen suostunut niiden etsittäväksi, jotka eivät minua kysyneet, niiden löydettäväksi, jotka eivät minua etsineet; minä olen sanonut kansalle, joka ei ole otettu minun nimiini: Tässä minä olen, tässä minä olen!
\par 2 Koko päivän minä olen ojentanut käsiäni uppiniskaista kansaa kohden, joka vaeltaa tietä, mikä ei ole hyvä, omain ajatustensa mukaan;
\par 3 kansaa kohden, joka vihoittaa minua alinomaa, vasten kasvojani, uhraa puutarhoissa ja suitsuttaa tiilikivialttareilla.
\par 4 He asustavat haudoissa ja yöpyvät salaisiin paikkoihin; he syövät sianlihaa, ja saastaiset liemet on heillä astioissansa.
\par 5 He sanovat: "Pysy erilläsi, älä tule minua lähelle, sillä minä olen sinulle pyhä". Nämä ovat savu minun sieramissani, tuli, joka palaa kaiken päivää.
\par 6 Katso, se on kirjoitettuna minun edessäni. En ole minä vaiti, vaan minä maksan, maksan heille helmaan
\par 7 heidän pahat tekonsa ynnä teidän isienne pahat teot, sanoo Herra, niiden, jotka ovat vuorilla suitsuttaneet ja kukkuloilla minua häväisseet; ja ensiksi minä mittaan heille palkan heidän helmaansa.
\par 8 Näin sanoo Herra: Niinkuin sanotaan rypäleestä, jos siinä on mehua: "Älä hävitä sitä, sillä siinä on siunaus", niin teen minä palvelijaini tähden, etten kaikkea hävittäisi.
\par 9 Minä tuotan Jaakobista siemenen ja Juudasta vuorteni perillisen; ja minun valittuni saavat periä maan, ja minun palvelijani saavat siinä asua.
\par 10 Ja Saaron on oleva lammasten laitumena ja Aakorin laakso karjan lepopaikkana minun kansallani, joka minua etsii.
\par 11 Mutta te, jotka hylkäätte Herran ja unhotatte minun pyhän vuoreni, jotka valmistatte Gadille pöydän ja vuodatatte uhrijuomaa Menille -
\par 12 teidät minä määrään miekan omiksi, ja kaikki te kumarrutte teurastettaviksi, sentähden ettette vastanneet, kun minä kutsuin, ettekä kuulleet, kun minä puhuin, vaan teitte sitä, mikä on pahaa minun silmissäni, ja valitsitte sen, mikä ei ole minulle otollista.
\par 13 Sentähden, näin sanoo Herra, Herra: Katso, minun palvelijani syövät, mutta te näette nälkää; katso, minun palvelijani juovat, mutta te kärsitte janoa; katso, minun palvelijani iloitsevat, mutta te joudutte häpeään.
\par 14 Katso, minun palvelijani riemuitsevat sydämen onnesta, mutta te huudatte sydämen tuskasta ja voivotatte mielimurteissanne.
\par 15 Ja te jätätte nimenne kirouslauseeksi minun valituilleni: "Niin sinut surmatkoon Herra, Herra!" Mutta palvelijoilleen hän on antava toisen nimen.
\par 16 Joka maassa itsensä siunaa, siunaa itsensä totisen Jumalan nimeen, ja joka maassa vannoo, vannoo totisen Jumalan nimeen; sillä entiset ahdistukset ovat unhotetut ja peitossa minun silmiltäni.
\par 17 Sillä katso, minä luon uudet taivaat ja uuden maan. Entisiä ei enää muisteta, eivätkä ne enää ajatukseen astu;
\par 18 vaan te saatte iloita ja riemuita iankaikkisesti siitä, mitä minä luon. Sillä katso, iloksi luon minä Jerusalemin, riemuksi sen kansan.
\par 19 Minä iloitsen Jerusalemista ja riemuitsen kansastani; eikä siellä enää kuulla itkun ääntä eikä valituksen ääntä.
\par 20 Ei siellä ole enää lasta, joka eläisi vain muutaman päivän, ei vanhusta, joka ei täyttäisi päiviensä määrää; sillä nuorin kuolee satavuotiaana, ja vasta satavuotiaana synnintekijä joutuu kiroukseen.
\par 21 He rakentavat taloja ja asuvat niissä, he istuttavat viinitarhoja ja syövät niiden hedelmät;
\par 22 he eivät rakenna muitten asua, eivät istuta muitten syödä; sillä niinkuin puitten päivät ovat, niin ovat elinpäivät minun kansassani. Minun valittuni kuluttavat itse kättensä työn.
\par 23 He eivät tee työtä turhaan eivätkä lapsia synnytä äkkikuoleman omiksi, sillä he ovat Herran siunattujen siemen, ja heidän vesansa ovat heidän tykönänsä.
\par 24 Ennenkuin he huutavat, minä vastaan, heidän vielä puhuessaan minä kuulen.
\par 25 Susi ja lammas käyvät yhdessä laitumella, ja leijona syö rehua niinkuin raavas, ja käärmeen ruokana on maan tomu; ei missään minun pyhällä vuorellani tehdä pahaa eikä vahinkoa, sanoo Herra.

\chapter{66}

\par 1 Näin sanoo Herra: Taivas on minun valtaistuimeni, ja maa on minun jalkojeni astinlauta. Mikä olisi huone, jonka te minulle rakentaisitte, mikä paikka olisi minun leposijani?
\par 2 Minun käteni on kaikki nämä tehnyt, ja niin ovat kaikki nämä syntyneet, sanoo Herra. Mutta minä katson sen puoleen, joka on nöyrä, jolla on särjetty henki ja arka tunto minun sanani edessä.
\par 3 Joka teurastaa härän, mutta myös tappaa miehen, joka uhraa lampaan, mutta myös taittaa koiralta niskan, joka uhraa ruokauhrin, mutta myös sian verta, joka polttaa suitsuketta, mutta myös ylistää epäjumalaa: nämä ovat valinneet omat tiensä. Ja niinkuin heidän sielunsa on mielistynyt heidän iljetyksiinsä,
\par 4 niin minäkin valitsen heidän vaivaamisensa ja annan heille tulla sen, mitä he pelkäävät, koska ei kukaan vastannut, kun minä kutsuin, koska he eivät kuulleet, kun minä puhuin, vaan tekivät sitä, mikä on pahaa minun silmissäni, ja valitsivat sen, mikä ei ole minulle otollista.
\par 5 Kuulkaa Herran sana, te, jotka olette aralla tunnolla hänen sanansa edessä: Teidän veljenne, jotka vihaavat teitä ja työntävät teidät luotaan minun nimeni tähden, sanovat: "Osoittakoon Herra kunniansa, että me näemme teidän ilonne!" Mutta he joutuvat häpeään.
\par 6 Metelin ääni kaupungista! Ääni temppelistä! Herran ääni: hän maksaa palkan vihollisillensa!
\par 7 Ennenkuin Siion kipuja tuntee, hän synnyttää; ennenkuin hänelle tuskat tulevat, hän saa poikalapsen.
\par 8 Kuka on sellaista kuullut, kuka senkaltaista nähnyt? Syntyykö maa yhden päivän kivulla, tahi synnytetäänkö kansa yhdellä haavaa? Siionhan tunsi kipuja ja samalla jo synnytti lapsensa.
\par 9 Minäkö avaisin kohdun sallimatta synnyttää, sanoo Herra, tahi minäkö, joka saatan synnyttämään, sulkisin kohdun?
\par 10 Iloitkaa Jerusalemin kanssa ja riemuitkaa hänestä kaikki, jotka häntä rakastatte; iloitsemalla iloitkaa hänen kanssaan, te, jotka olette hänen tähtensä surreet,
\par 11 että imisitte ja tulisitte ravituiksi hänen lohdutuksensa rinnoista, että joisitte ja virkistyisitte hänen kunniansa runsaudesta.
\par 12 Sillä näin sanoo Herra: Minä ohjaan hänen tykönsä rauhan niinkuin virran ja kansojen kunnian niinkuin tulvajoen. Ja te saatte imeä, kainalossa teitä kannetaan, ja polvilla pitäen teitä hyväillään.
\par 13 Niinkuin äiti lohduttaa lastansa, niin minä lohdutan teitä, ja Jerusalemissa te saatte lohdutuksen.
\par 14 Te näette sen, ja teidän sydämenne iloitsee, ja teidän luunne virkistyvät kuin vihanta ruoho; ja Herran käsi tulee tunnetuksi hänen palvelijoissansa, mutta vihollistensa hän antaa tuntea vihansa.
\par 15 Sillä katso, Herra tulee tulessa, ja hänen vaununsa ovat kuin myrskytuuli; ja hän antaa vihansa purkautua hehkussa ja nuhtelunsa tulenliekeissä.
\par 16 Herra käy tuomiolle kaiken lihan kanssa tulella ja miekallaan; ja Herran surmaamia on oleva paljon.
\par 17 Jotka pyhittäytyvät ja puhdistautuvat puutarha-menoja varten, seuraten miestä, joka on heidän keskellänsä, jotka syövät sianlihaa ja muuta inhottavaa sekä hiiriä, niistä kaikista tulee loppu, sanoo Herra.
\par 18 Minä tunnen heidän tekonsa ja ajatuksensa. Aika tulee, että minä kokoan kaikki kansat ja kielet, ja he tulevat ja näkevät minun kunniani.
\par 19 Ja minä teen tunnusteon heidän keskellänsä ja lähetän pakoonpäässeitä heidän joukostansa pakanain tykö Tarsiiseen, Puuliin ja Luudiin, jousenjännittäjäin tykö, Tuubaliin ja Jaavaniin, kaukaisiin merensaariin, jotka eivät ole kuulleet minusta kerrottavan eivätkä nähneet minun kunniaani, ja he ilmoittavat minun kunniani pakanain keskuudessa.
\par 20 Ja he tuovat kaikki teidän veljenne kaikista kansoista uhrilahjana Herralle hevosilla, vaunuilla, kantotuoleilla, muuleilla ja ratsukameleilla minun pyhälle vuorelleni Jerusalemiin, sanoo Herra, niinkuin israelilaiset tuovat ruokauhrin puhtaassa astiassa Herran temppeliin.
\par 21 Ja heitäkin minä otan leeviläisiksi papeiksi, sanoo Herra.
\par 22 Niinkuin uudet taivaat ja uusi maa, jotka minä teen, pysyvät minun kasvojeni edessä, sanoo Herra, niin pysyy teidän siemenenne ja teidän nimenne.
\par 23 Joka kuukausi uudenkuun päivänä ja joka viikko sapattina tulee kaikki liha kumartaen rukoilemaan minua, sanoo Herra.
\par 24 Ja he käyvät ulos katselemaan niiden miesten ruumiita, jotka ovat luopuneet minusta; sillä heidän matonsa ei kuole, eikä heidän tulensa sammu, ja he ovat kauhistukseksi kaikelle lihalle.


\end{document}