\begin{document}

\title{Hesekielin kirja}


\chapter{1}

\par 1 Kolmantenakymmenentenä vuotena, neljännessä kuussa, kuukauden viidentenä päivänä, kun minä olin pakkosiirtolaisten joukossa Kebar-joen varrella, aukenivat taivaat, ja minä näin Jumalan näkyjä.
\par 2 Kuukauden viidentenä päivänä, kuningas Joojakinin pakkosiirtolaisuuden viidentenä vuotena,
\par 3 tuli Jumalan sana pappi Hesekielille, Buusin pojalle, Kaldean maassa Kebar-joen varrella, ja hänen päällensä tuli siellä Herran käsi.
\par 4 Ja minä näin, ja katso: myrskytuuli tuli pohjoisesta, suuri pilvi ja leimahteleva tuli, ja pilveä ympäröitsi hohde, ja tulen keskeltä näkyi ikäänkuin hehkuvaa malmia, keskeltä tulta.
\par 5 Ja sen keskeltä näkyivät neljän olennon hahmot. Ja näöltänsä ne olivat tällaiset: niillä oli ihmisen hahmo.
\par 6 Ja niillä oli neljät kasvot kullakin ja neljä siipeä kullakin.
\par 7 Ja säärivarret niillä oli suorat ja jalkaterät kuin vasikansorkat, ja ne välkkyivät kuin kiiltävä vaski.
\par 8 Ja siipiensä alla niillä oli, neljällä eri puolellansa, ihmiskädet. Kasvot ja siivet niillä neljällä olivat näin:
\par 9 Niiden siivet koskettivat toisiansa. Kulkiessaan ne eivät kääntyneet: ne kulkivat kukin suoraan eteenpäin.
\par 10 Ja niiden kasvot olivat ihmiskasvojen kaltaiset; mutta oikealla puolen oli niillä neljällä leijonankasvot, vasemmalla puolen oli niillä neljällä häränkasvot, myös oli niillä neljällä kotkankasvot.
\par 11 Niin niiden kasvot. Mutta siivet niillä oli levällään ylöspäin. Kullakin oli kaksi, jotka koskettivat toisen siipiä, ja kaksi, joilla ne peittivät ruumistansa.
\par 12 Ja ne kulkivat suoraan eteenpäin. Minne henki vaati kulkemaan, sinne ne kulkivat. Kulkiessaan ne eivät kääntyneet.
\par 13 Ja olentojen hahmo oli näöltänsä kuin tuliset hiilet, jotka paloivat tulisoihtujen näköisinä. Tulta liekehti olentojen välissä, ja tuli hohti, ja tulesta lähti salamoita.
\par 14 Ja olennot kiitivät edestakaisin ja olivat nähdä kuin salamanleimaus.
\par 15 Mutta kun minä katselin olentoja, niin katso: yksi pyörä oli maassa olentojen kohdalla, kunkin niiden neljän etupuolella.
\par 16 Pyörät olivat näöltään ja teoltaan niinkuin krysoliitti, ja niillä neljällä oli sama muoto, ja ne olivat näöltään ja teoltaan, kuin olisi ollut sisäkkäin pyörä pyörässä.
\par 17 Ne kulkivat neljään eri suuntaansa, kun kulkivat.
\par 18 Kulkiessaan ne eivät kääntyneet. Ja niiden kehät olivat korkeat ja peljättävät; ja niiden kehät olivat täynnä silmiä, yltympäri, niissä neljässä.
\par 19 Ja kun olennot kulkivat, kulkivat pyörät niiden ohella; ja kun olennot kohosivat ylös maasta, kohosivat myös pyörät.
\par 20 Minne henki vaati kulkemaan, sinne ne kulkivat - minne vain henki kulkemaan vaati. Ja pyörät kohosivat samalla kuin nekin, sillä olentojen henki oli pyörissä.
\par 21 Kun olennot kulkivat, kulkivat nämäkin; kun ne seisoivat, seisoivat nämäkin; kun ne kohosivat ylös maasta, kohosivat pyörät samalla kuin nekin, sillä olentojen henki oli pyörissä.
\par 22 Ja olentojen päitten ylle hahmottui taivaanvahvuus, niinkuin peljättävä kristalli, kaartuen ylös niiden päitten ylitse.
\par 23 Ja taivaanvahvuuden alla oli niillä siivet suorina, toisen siipi toisen siipeä kohti. Kullakin oli kaksi, jotka sitä peittivät - kaksi, jotka peittivät sen ruumista.
\par 24 Ja minä kuulin niiden siipien kohinan niinkuin paljojen vetten kohinan, niinkuin Kaikkivaltiaan jylinän, kun ne kulkivat; pauhinan ääni oli niinkuin sotaleirin pauhu. Kun ne seisoivat, laskivat ne siipensä alas.
\par 25 Ja kuului ääni taivaanvahvuuden yläpuolelta, joka oli niitten pään päällä. - Kun ne seisoivat, laskivat ne siipensä alas.
\par 26 Ja taivaanvahvuuden yläpuolella, joka oli niitten pään päällä, oli valtaistuimen muotoinen, näöltään kuin safiirikiveä. Ja valtaistuimen muotoisella istui hahmo, ihmisen näköinen, kohoten korkealle.
\par 27 Ja minä näin ikäänkuin hehkuvaa malmia, tulen näköistä, jota ympäröi kehä ylöspäin siitä, mikä näytti hänen lanteiltansa; ja alaspäin siitä, mikä näytti hänen lanteiltansa, minä näin kuin tulen näköistä, ja sitä ympäröitsi hohde.
\par 28 Kuin kaari, joka on pilvessä sadepäivänä, niin oli näöltään sitä ympäröivä hohde. Senkaltainen oli katsoa Herran kirkkauden hahmo. Ja kun minä sen näin, lankesin minä kasvoilleni. Ja minä kuulin äänen, kun joku puhui.

\chapter{2}

\par 1 Ja hän sanoi minulle: "Ihmislapsi, nouse jaloillesi, niin minä puhuttelen sinua".
\par 2 Niin minuun tuli henki, kun hän puhui minulle, ja se nosti minut jaloilleni, ja minä kuulin hänen puhuvan minulle.
\par 3 Ja hän sanoi minulle: "Ihmislapsi, minä lähetän sinut israelilaisten tykö, kapinallisten pakanain tykö, jotka ovat kapinoineet minua vastaan; he ja heidän isänsä ovat luopuneet minusta, hamaan tähän päivään asti.
\par 4 Nuo lapset, joilla on kovat kasvot ja paatuneet sydämet - niitten luokse minä sinut lähetän, ja sinun on sanottava heille: Näin sanoo Herra, Herra.
\par 5 Kuulkoot tai olkoot kuulematta - sillä uppiniskainen suku he ovat - he tulevat kuitenkin tietämään, että profeetta on ollut heidän keskellänsä.
\par 6 Mutta sinä, ihmislapsi, älä pelkää heitä, äläkä pelkää heidän sanojansa, vaikka edessäsi on ohdakkeita ja orjantappuroita ja sinä asut skorpionien seassa; älä pelkää heidän sanojansa äläkä arkaile heidän kasvojansa, sillä he ovat uppiniskainen suku.
\par 7 Vaan puhu heille minun sanani, kuulkoot tai olkoot kuulematta; sillä uppiniskaisia he ovat.
\par 8 Mutta sinä, ihmislapsi, kuule, mitä minä sinulle sanon. Älä ole uppiniskainen, niinkuin uppiniskainen suku on. Avaa suusi ja syö, mitä minä sinulle annan."
\par 9 Niin minä näin, ja katso: käsi ojennettiin minua kohti, ja katso: siinä oli kirjakäärö.
\par 10 Ja hän levitti sen minun eteeni, ja katso: se oli kirjoitettu täyteen sisältä ja päältä. Ja siihen oli kirjoitettu itkuvirret ja huokaukset ja voi-huudot.

\chapter{3}

\par 1 Ja hän sanoi minulle: "Ihmislapsi, syö, minkä tässä saat; syö tämä kirjakäärö, mene ja puhu Israelin heimolle".
\par 2 Niin minä avasin suuni, ja hän antoi tämän kirjakäärön minun syödäkseni.
\par 3 Ja hän sanoi minulle: "Ihmislapsi, ravitse vatsasi ja täytä sisälmyksesi tällä kirjakääröllä, jonka minä sinulle annan". Niin minä söin, ja se oli minun suussani makea kuin hunaja.
\par 4 Ja hän sanoi minulle: "Ihmislapsi, mene nyt Israelin heimon tykö ja puhu heille minun sanani.
\par 5 Sillä ei sinua lähetetä outokielisen ja kankeapuheisen kansan tykö, vaan Israelin heimon tykö;
\par 6 ei monien kansojen tykö, outokielisten ja kankeapuheisten, joiden puhetta et ymmärrä. Jos niiden tykö lähettäisin sinut, he sinua kuulisivat.
\par 7 Mutta Israelin heimo ei tahdo sinua kuulla, koska he eivät tahdo minua kuulla. Sillä koko Israelin heimolla on kova otsa ja paatunut sydän.
\par 8 Katso, minä teen sinun kasvosi yhtä koviksi kuin heidän kasvonsa ja sinun otsasi yhtä kovaksi kuin heidän otsansa.
\par 9 Timantin kaltaiseksi, kiveä kovemmaksi minä teen sinun otsasi. Älä pelkää heitä äläkä arkaile heidän kasvojansa, sillä he ovat uppiniskainen suku."
\par 10 Ja hän sanoi minulle: "Ihmislapsi, kaikki minun sanani, jotka minä sinulle puhun, ota sydämeesi ja kuule korvillasi.
\par 11 Ja nyt mene pakkosiirtolaisten tykö, kansasi lasten tykö; puhu heille ja sano heille: Näin sanoo Herra, Herra - kuulkoot he tai olkoot kuulematta."
\par 12 Niin henki nosti minut, ja minä kuulin takaani suuren, jylisevän äänen: "Kiitetty olkoon Herran kirkkaus, siinä kussa se on!"
\par 13 ja olentojen siipien kohinan, kun ne löivät toisiinsa, samalla pyöräin jyrinän ja suuren, jylisevän äänen.
\par 14 Ja henki nosti minut ja otti minut, ja minä kuljin murheellisena henkeni kiivaudessa, ja Herran käsi oli väkevänä minun päälläni.
\par 15 Ja minä tulin Tell-Aabibiin pakkosiirtolaisten tykö, jotka asuivat Kebar-joen varrella; ja siellä, missä he asuivat, minä istuin jäykistyneenä heidän keskuudessaan seitsemän päivää.
\par 16 Mutta seitsemän päivän kuluttua tuli minulle tämä Herran sana:
\par 17 "Ihmislapsi, minä olen asettanut sinut Israelin heimolle vartijaksi. Kun kuulet sanan minun suustani, on sinun varoitettava heitä minun puolestani.
\par 18 Jos minä sanon jumalattomalle: sinun on kuolemalla kuoltava, mutta sinä et häntä varoita etkä puhu varoittaaksesi jumalatonta hänen jumalattomasta tiestänsä, että pelastaisit hänen henkensä, niin jumalaton kuolee synnissänsä, mutta hänen verensä minä vaadin sinun kädestäsi.
\par 19 Mutta jos sinä varoitat jumalatonta ja hän ei käänny jumalattomuudestansa eikä jumalattomalta tieltänsä, niin hän kuolee synnissänsä, mutta sinä olet sielusi pelastanut.
\par 20 Ja jos vanhurskas kääntyy pois vanhurskaudestansa ja tekee vääryyttä ja minä panen kompastuksen hänen eteensä, niin hän kuolee - kun et sinä häntä varoittanut, niin hän synnissänsä kuolee, ja vanhurskautta, jota hän oli harjoittanut, ei muisteta - mutta hänen verensä minä vaadin sinun kädestäsi.
\par 21 Mutta jos sinä vanhurskasta varoitat, ettei vanhurskas tekisi syntiä, ja hän ei tee syntiä, niin hän totisesti saa elää, koska otti varoituksesta vaarin, ja sinä olet sielusi pelastanut."
\par 22 Herran käsi tuli siellä minun päälleni, ja hän sanoi minulle: "Nouse ja mene laaksoon, niin minä siellä puhuttelen sinua".
\par 23 Niin minä nousin ja menin laaksoon, ja katso, siellä seisoi Herran kirkkaus, samankaltainen kuin kirkkaus, jonka minä olin nähnyt Kebar-joen varrella. Ja minä lankesin kasvoilleni.
\par 24 Mutta minuun tuli henki ja nosti minut jaloilleni, ja hän puhutteli minua ja sanoi minulle: "Mene, sulkeudu sisälle huoneeseesi.
\par 25 Ja sinä, ihmislapsi! Katso, sinut pannaan köysiin ja sidotaan niillä, niin ettet voi mennä ulos heidän keskuuteensa.
\par 26 Ja minä tartutan sinun kielesi suusi lakeen, niin että mykistyt etkä voi olla heille nuhtelijana, sillä he ovat uppiniskainen suku.
\par 27 Mutta kun minä puhuttelen sinua, avaan minä sinun suusi, ja sinun on sanottava heille: Näin sanoo Herra, Herra. Kuulkoon, joka kuulee, ja olkoon kuulematta, joka ei kuule; sillä he ovat uppiniskainen suku."

\chapter{4}

\par 1 "Ja sinä, ihmislapsi, ota itsellesi tiilikivi ja aseta se eteesi ja piirrä siihen kaupunki, Jerusalem.
\par 2 Ja pane se piirityksiin, rakenna sitä vastaan saartovarusteet, luo sitä vastaan valli, aseta sitä vastaan sotaleirejä ja pane sitä vastaan yltympäri muurinmurtajia.
\par 3 Ota sitten itsellesi rautainen leivinlevy ja aseta se rautamuuriksi itsesi ja kaupungin väliin ja suuntaa kasvosi kaupunkia kohti, niin että se tulee piirityksiin, ja piiritä sitä. Tämä on oleva merkki Israelin heimolle.
\par 4 Ja sinä pane maata vasemmalle kyljellesi ja aseta sitä painamaan Israelin heimon syntivelka. Yhtä monta päivää, kuin sinä makaat sillä, on sinun kannettava heidän syntivelkaansa.
\par 5 Ja minä panen sinulle saman luvun päiviä, kuin on heidän syntivelkansa vuosia: kolmesataa yhdeksänkymmentä päivää; niin sinun on kannettava Israelin heimon syntivelkaa.
\par 6 Ja kun olet saanut ne päättymään, on sinun pantava maata toistamiseen, oikealle kyljellesi, ja kannettava Juudan heimon syntivelkaa neljäkymmentä päivää; päivän kutakin vuotta kohti minä olen sinulle pannut.
\par 7 Ja suuntaa kasvosi ja paljastettu käsivartesi piiritettyä Jerusalemia kohti ja ennusta sitä vastaan.
\par 8 Ja katso, minä panen sinut köysiin, niin ettet voi kääntyä kyljeltä toiselle, kunnes olet saanut päättymään piirityspäiväsi.
\par 9 Ota sinä myös itsellesi nisuja, ohria, papuja, herneitä, hirssiä ja kolmitahkoista vehnää, pane ne samaan astiaan ja tee niistä itsellesi leipää. Yhtä monta päivää, kuin makaat kyljelläsi, kolmesataa yhdeksänkymmentä päivää, on sinun syötävä sitä.
\par 10 Ja ruoka, jota syöt, on sinun syötävä painon mukaan: kaksikymmentä sekeliä päivässä; sinun on syötävä se määräaikana.
\par 11 Ja vettä sinun on juotava mitan mukaan: kuudennes hiin-mittaa; sinun on juotava määräaikana.
\par 12 Ja ohrakaltiaisena on sinun se syötävä, ja ihmisulostuksella on sinun se paistettava heidän silmäinsä edessä."
\par 13 Ja Herra sanoi: "Näin tulevat israelilaiset syömään leipänsä saastaisena pakanain joukossa, jonne minä heidät karkoitan".
\par 14 Mutta minä sanoin: "Voi Herra, Herra! Katso, ei milloinkaan ole minun sieluni ollut saastutettu, minä en ole syönyt itsestään kuollutta enkä kuoliaaksi raadeltua nuoruudestani tähän asti, eikä ole mitään saastaista lihaa minun suuhuni tullut."
\par 15 Niin hän sanoi minulle: "Katso, minä annan sinulle eläimen lannan ihmisen jäljen sijaan; valmista leipäsi sen päällä".
\par 16 Ja hän sanoi minulle: "Ihmislapsi, katso, minä murran leivän tuen Jerusalemilta, ja he saavat syödä leipää painon mukaan ja huolissansa sekä juoda vettä mitan mukaan ja kauhuissansa,
\par 17 niin että he joutuvat leivän ja veden puutteeseen ja yhdessä nääntyvät ja riutuvat syntivelkansa tähden".

\chapter{5}

\par 1 "Ja sinä, ihmislapsi, ota itsellesi terävä miekka; ota se partaveitseksesi, ja ajele sillä hiuksesi ja partasi. Ota sitten vaaka ja jaa ne.
\par 2 Polta kolmannes tulessa keskellä kaupunkia, sittenkuin piirityspäivät ovat lopussa. Kolmannes ota ja lyö miekalla sen ympärillä. Kolmannes hajota tuuleen, ja minä ajan niitä takaa paljastetulla miekalla;
\par 3 ota niistä sitten vähäinen määrä ja sido ne helmukseesi,
\par 4 ja ota niistä vielä osa, viskaa ne tuleen ja polta ne tulessa. Siitä lähtee tuli koko Israelin heimoon.
\par 5 Näin sanoo Herra, Herra: Tämä on Jerusalem; pakanain keskelle minä olen sen asettanut, ja niiden maat ovat sen ympärillä.
\par 6 Mutta se on niskoitellut minun oikeuksiani vastaan jumalattomammin kuin pakanat ja minun käskyjäni vastaan jumalattomammin kuin maat, jotka ovat sen ympärillä; sillä minun oikeuteni he ovat pitäneet halpoina ja minun käskyjeni mukaan he eivät ole vaeltaneet.
\par 7 Sentähden, näin sanoo Herra, Herra: Koska teidän metelinne on ollut pahempi kuin pakanain, jotka ympärillänne ovat, ja koska te ette ole vaeltaneet minun käskyjeni mukaan, ette ole noudattaneet minun oikeuksiani ettekä ole tehneet niiden pakanainkaan oikeuksien mukaan, jotka ympärillänne ovat,
\par 8 sentähden, näin sanoo Herra, Herra: Katso, myös minä käyn sinun kimppuusi ja teen tuomiot sinun keskelläsi pakanain silmien edessä
\par 9 ja teen sinulle kaikkien kauhistustesi tähden, mitä en ole ennen tehnyt ja minkäkaltaista en vasta ole tekevä.
\par 10 Sentähden tulevat sinun keskelläsi isät syömään lapsiansa, ja lapset tulevat syömään isiänsä, ja minä teen sinussa tuomiot ja hajotan sinun jäännöksesi kaikkiin tuuliin.
\par 11 Sentähden, niin totta kuin minä elän, sanoo Herra, Herra: Totisesti, koska olet saastuttanut minun pyhäkköni kaikilla iljetyksilläsi ja kaikilla kauhistuksillasi, käännyn minäkin pois, en sääli enkä armahda.
\par 12 Kolmannes sinusta kuolee ruttoon ja nääntyy nälkään sinun keskelläsi. Kolmannes kaatuu miekkaan sinun ympärilläsi. Kolmanneksen minä hajotan kaikkiin tuuliin, ja minä ajan niitä takaa paljastetulla miekalla.
\par 13 Ja minun vihani täyttyy, ja minä tyydytän kiivauteni heissä ja kostan. Ja he tulevat tietämään, että minä, Herra, olen puhunut kiivaudessani, kun minä panen vihani täytäntöön heissä.
\par 14 Ja minä teen sinut raunioksi ja häväistyksi pakanain kesken, jotka sinun ympärilläsi ovat, jokaisen ohikulkijan silmäin edessä.
\par 15 Ja se on oleva häväistys ja herjaus, varoitus ja peljätys pakanain seassa, jotka ympärilläsi ovat, kun minä teen sinussa tuomiot vihassa ja kiivaudessa ja kiivailla rangaistuksilla - minä, Herra, olen puhunut -
\par 16 kun minä lähetän heidän kimppuunsa nälän pahat nuolet, joista tulee tuhooja ja jotka minä lähetän teitä tuhoamaan; ja minä vielä lisään teille nälkää ja murran teiltä leivän tuen.
\par 17 Ja minä lähetän teidän kimppuunne nälän ja pahat petoeläimet, jotka riistävät sinulta lapset; rutto ja veri käyvät sinun ylitsesi, ja minä annan miekan tulla sinun ylitsesi. Minä, Herra, olen puhunut."

\chapter{6}

\par 1 Minulle tuli tämä Herran sana:
\par 2 "Ihmislapsi, käännä kasvosi Israelin vuoria kohti ja ennusta niitä vastaan
\par 3 ja sano: Israelin vuoret, kuulkaa Herran, Herran sana. Näin sanoo Herra, Herra vuorille ja kukkuloille, puronotkoille ja laaksoille: Katso, minä annan miekan tulla teidän ylitsenne ja hävitän teidän uhrikukkulanne.
\par 4 Teidän alttarinne jäävät autioiksi, auringonpatsaanne murskataan, ja teidän surmattujenne minä annan kaatua teidän kivijumalainne eteen.
\par 5 Minä panen israelilaisten ruumiit heidän kivijumalainsa eteen ja hajotan heidän luunsa heidän alttariensa ympärille.
\par 6 Kaikilla teidän asuinpaikoillanne tulevat kaupungit raunioiksi ja uhrikukkulat autioiksi; niin jäävät teidän alttarinne raunioiksi ja autioiksi, teidän kivijumalanne murskataan ja hävitetään, teidän auringonpatsaanne särjetään, ja teidän tekeleenne pyyhkäistään pois,
\par 7 surmattuja kaatuu teidän keskellänne, ja te tulette tietämään, että minä olen Herra.
\par 8 Mutta minä jätän jäännöksen: teillä on oleva miekasta pelastuneita pakanain seassa, kun teidät on hajotettu niiden maihin;
\par 9 ja ne teistä, jotka pelastuvat, muistavat minut pakanain seassa, mihin ovat vangeiksi viedyt, sillä minä särjen heidän haureellisen sydämensä, joka poikkesi minusta pois, ja heidän silmänsä, jotka haureellisina kulkivat heidän kivijumalainsa jäljessä. Ja heitä kyllästyttää oma itsensä pahain töitten tähden, joita he ovat tehneet, kaikkien heidän kauhistustensa tähden.
\par 10 Ja he tulevat tietämään, että minä olen Herra; en ole turhaan puhunut, että minä tuotan teille tämän onnettomuuden.
\par 11 Näin sanoo Herra, Herra: Paukuta käsiäsi ja polje jalkaasi ja sano: 'voi!' kaikista Israelin heimon häijyistä kauhistuksista, heidän, jotka kaatuvat miekkaan, nälkään ja ruttoon.
\par 12 Kaukainen kuolee ruttoon, läheinen kaatuu miekkaan, jäljellejäänyt ja säilynyt kuolee nälkään, ja minä panen vihani täytäntöön heissä.
\par 13 Ja te tulette tietämään, että minä olen Herra, kun heitä on surmattuina heidän kivijumalainsa keskellä, heidän alttariensa ympärillä, jokaisella korkealla kukkulalla, kaikilla vuorten huipuilla, jokaisen viheriän puun alla ja jokaisen tuuhean tammen alla, missä paikassa vain he ovat tarinneet suloista tuoksua kaikille kivijumalillensa.
\par 14 Ja minä ojennan käteni heitä vastaan ja teen maan autiommaksi ja hävitetymmäksi, kuin on Diblatan erämaa, kaikilla heidän asuinpaikoillansa; ja he tulevat tietämään, että minä olen Herra."

\chapter{7}

\par 1 Minulle tuli tämä Herran sana:
\par 2 "Sinä, ihmislapsi, näin sanoo Herra, Herra Israelin maalle: Loppu! Maan neljälle äärelle tulee loppu.
\par 3 Nyt tulee sinulle loppu, ja minä lähetän vihani sinua vastaan, tuomitsen sinut vaelluksesi mukaan ja annan kaikkien kauhistustesi kohdata sinua.
\par 4 En sääli sinua enkä armahda, vaan annan vaelluksesi kohdata sinua, ja kauhistuksesi tulevat sinun keskellesi, ja te tulette tietämään, että minä olen Herra.
\par 5 Näin sanoo Herra, Herra: Onnettomuus! Yksi ja ainoa onnettomuus! Katso, se tulee!
\par 6 Loppu tulee, tulee loppu! Se heräjää sinua vastaan! Katso, se tulee!
\par 7 Vuoro tulee sinulle, maan asuja. Aika tulee, päivä on lähellä: hämminki, ei ilohuuto vuorilta.
\par 8 Nyt minä kohta vuodatan kiivauteni sinun ylitsesi, panen vihani täytäntöön sinussa, tuomitsen sinut vaelluksesi mukaan ja annan kaikkien kauhistustesi kohdata sinua.
\par 9 En sääli enkä armahda; minä annan vaelluksesi kohdata sinua, ja kauhistuksesi tulevat sinun keskellesi, ja te tulette tietämään, että minä, Herra, olen se, joka lyön.
\par 10 Katso, päivä! Katso, se tulee! Vuoro on tullut, vitsa kukkii, julkeus versoo.
\par 11 Väkivalta on noussut vitsaksi jumalattomuudelle, ei mitään jää heistä, ei heidän pauhaavasta joukostansa, ei heidän melustansa; poissa on heidän komeutensa.
\par 12 Tullut on aika, joutunut päivä, älköön ostaja iloitko, älköön myyjä murehtiko, sillä viha tulee kaiken siellä pauhaavan joukon ylitse.
\par 13 Sillä myyjä ei enää palaja myynnöksillensä, vaikka he vielä jäisivätkin henkiin, sillä näky kaikkea sen pauhaavaa joukkoa vastaan ei peräydy, eikä kukaan synnillänsä vahvista elämäänsä.
\par 14 Puhalletaan torviin, varustetaan kaikki, mutta ei ole lähtijää sotaan, sillä minun vihani tulee kaiken siellä pauhaavan joukon ylitse.
\par 15 Miekka on ulkona, sisällä rutto ja nälkä. Kedolla oleva kuolee miekkaan, kaupungissa olevan syö nälkä ja rutto.
\par 16 Ja jos heitä joitakin pelastuu ja pääsee pakoon, ovat he vuorilla kuin laaksojen kyyhkyset, ja he vaikertelevat kaikki, syntiänsä kukin.
\par 17 Kaikki kädet herpoavat, kaikki polvet käyvät veltoiksi kuin vesi.
\par 18 He kääriytyvät säkkeihin, heidät peittää vavistus, kaikilla kasvoilla on häpeä, ja kaikki päät ovat paljaiksi ajellut.
\par 19 Hopeansa he viskaavat kaduille, ja heidän kultansa tulee heille saastaksi. Heidän kultansa ja hopeansa eivät voi heitä pelastaa Herran vihan päivänä, eivät he sillä nälkäänsä tyydytä eivätkä vatsaansa täytä; sillä siitä tuli heille kompastus syntiin.
\par 20 Siitä tehtyjen korujen loistolla he ylpeilivät, siitä he tekivät kauhistavat kuvansa, iljetyksensä; sentähden minä teen sen heille saastaksi.
\par 21 Ja minä annan sen vieraitten käsiin ryöstettäväksi ja maan jumalattomien saaliiksi, ja ne häpäisevät sen.
\par 22 Minä käännän kasvoni heistä pois, niin että minun aarteeni häväistään: väkivaltaiset menevät sinne ja häpäisevät sen.
\par 23 Valmista kahle, sillä maa on täynnä verivelkain tuomiota, ja kaupunki on täynnä väkivaltaa.
\par 24 Ja minä tuon pakanoista pahimmat, ja ne ottavat omiksensa teidän talonne. Minä teen lopun voimallisten ylpeydestä, ja heidän pyhäkkönsä häväistään.
\par 25 Tulee hätä, ja he etsivät pelastusta, mutta sitä ei ole.
\par 26 Häviö tulee häviön jälkeen, sanoma tulee sanoman jälkeen. He havittelevat näkyä profeetalta; papilta katoaa opetus ja vanhimmilta neuvo.
\par 27 Kuningas murehtii, päämies pukeutuu kauhuun, ja maan kansan kädet pelosta vapisevat. Minä teen heille heidän teittensä mukaan, ja tuomitsen heidät heidän tuomioittensa mukaan, ja he tulevat tietämään, että minä olen Herra."

\chapter{8}

\par 1 Kuudentena vuotena, kuudennessa kuussa, kuukauden viidentenä päivänä minä istuin huoneessani, ja Juudan vanhimmat istuivat minun edessäni; silloin Herran, Herran käsi laskeutui siellä minun päälleni.
\par 2 Ja minä näin, ja katso: tulennäköinen hahmo! Ja alaspäin siitä, mikä näytti sen lanteilta, oli tulta, ja ylöspäin sen lanteista näkyi loiste, joka oli nähdä niinkuin hehkuva malmi.
\par 3 Hän ojensi ikäänkuin käden ja otti minua pääni hiussuortuvasta, ja henki nosti minut maan ja taivaan välille ja vei minut Jerusalemiin Jumalan näyissä, sisemmän portin ovelle, joka on pohjoista kohden, sinne missä oli kiivauspatsas, joka oli herättänyt Herran kiivauden.
\par 4 Ja katso, siellä oli Israelin Jumalan kirkkaus, samankaltainen nähdä kuin se, minkä minä olin nähnyt laaksossa.
\par 5 Hän sanoi minulle: "Ihmislapsi, nosta silmäsi pohjoista kohti". Niin minä nostin silmäni pohjoista kohti, ja katso: pohjoiseen päin alttariportista oli tuo kiivauspatsas sisäänkäytävässä.
\par 6 Ja hän sanoi minulle: "Ihmislapsi, näetkö sinä, mitä he tekevät - suuria kauhistuksia, joita Israelin heimo tekee täällä, että minä menisin kauas pois pyhäköstäni? Mutta sinä saat nähdä vielä suurempia kauhistuksia."
\par 7 Ja hän vei minut esipihan ovelle, ja minä näin, ja katso: seinässä oli jokin aukko.
\par 8 Ja hän sanoi minulle: "Ihmislapsi, murtaudu läpi seinän".
\par 9 Niin minä murtauduin läpi seinän, ja katso: jokin ovi! Ja hän sanoi minulle: "Mene ja katso häijyjä kauhistuksia, joita he täällä tekevät".
\par 10 Niin minä menin ja näin, ja katso: kaikenkaltaisia inhottavia matelijain ja karjaeläinten kuvia ja kaikenkaltaisia Israelin heimon kivijumalia oli piirretty seinään yltympäri.
\par 11 Ja niiden edessä seisoi seitsemänkymmentä miestä, Israelin heimon vanhimpia, ja näiden keskellä seisoi Jaasanja, Saafanin poika; ja kullakin heillä oli suitsutusastiansa kädessään, ja suitsutuspilvestä nousi tuoksu.
\par 12 Niin hän sanoi minulle: "Näetkö, ihmislapsi, mitä Israelin heimon vanhimmat tekevät pimeässä, itsekukin kuvakammiossansa? Sillä he sanovat:
\par 13 'Ei Herra meitä näe; Herra on hyljännyt tämän maan'." Ja hän sanoi minulle: "Sinä saat nähdä vieläkin suurempia kauhistuksia, joita he tekevät".
\par 14 Ja hän vei minut Herran huoneen portin ovelle, joka on pohjoista kohden, ja katso: siellä istui naisia itkemässä Tammusta.
\par 15 Ja hän sanoi minulle: "Näetkö, ihmislapsi? Vieläkin sinä saat nähdä kauhistuksia, vielä suurempia kuin nämä."
\par 16 Ja hän vei minut Herran huoneen sisemmälle esipihalle, ja katso: Herran temppelin ovella, eteisen ja alttarin välillä, oli noin kaksikymmentä viisi miestä. Heillä oli selät Herran temppeliin päin ja kasvot itään päin, ja päin itää he kumarsivat aurinkoa.
\par 17 Ja hän sanoi minulle: "Näitkö, ihmislapsi? Eikö ole Juudan heimolle kylliksi, että he tekevät kauhistuksia, joita täällä on tehty, koska he täyttävät maan väkivallalla ja niin aina uudelleen vihoittavat minut? Katso, kuinka he vievät viiniköynnöksen lehvää nenänsä eteen!
\par 18 Mutta minäkin teen, minkä teen, vihassani: en sääli enkä armahda; ja vaikka he huutavat minun korviini suurella äänellä, minä en heitä kuule."

\chapter{9}

\par 1 Sitten hän huusi suurella äänellä minun korviini ja sanoi: "Tulkaa, kaupungin rankaisijat, ja kullakin olkoon tuhoaseensa kädessään!"
\par 2 Ja katso, kuusi miestä tuli Yläportilta päin, joka on pohjoista kohden, ja kullakin oli kädessä hävitysaseensa. Mutta yksi mies oli heidän keskellään, puettu pellavavaatteisiin ja vyöllänsä kirjoitusneuvot. Ja he tulivat ja asettuivat seisomaan vaskialttarin ääreen.
\par 3 Mutta Israelin Jumalan kirkkaus oli kohonnut yläpuolelta kerubin, jonka yllä se oli ollut, huoneen kynnykselle ja huusi pellavavaatteisiin puetulle miehelle, jolla oli kirjoitusneuvot vyöllänsä;
\par 4 ja Herra sanoi hänelle: "Kierrä kaupungin, Jerusalemin, läpi ja tee merkki niitten miesten otsiin, jotka huokaavat ja valittavat kaikkia kauhistuksia, mitä sen keskuudessa tehdään".
\par 5 Ja niille toisille hän sanoi minun kuulteni: "Kiertäkää kaupungin läpi hänen jäljessään ja surmatkaa. Älkää säälikö, älkää armahtako,
\par 6 tappakaa tyyten vanhukset, nuorukaiset, neitsyet, lapset ja vaimot, mutta älkää koskeko keneenkään, jolla on otsassaan merkki; ja alottakaa minun pyhäköstäni." Niin he alottivat niistä miehistä, vanhimmista, jotka olivat temppelin edessä.
\par 7 Ja hän sanoi heille: "Saastuttakaa temppeli ja täyttäkää esipihat surmatuilla. Menkää!" Niin he menivät ja surmasivat kaupungissa.
\par 8 Mutta kun minä heidän surmatessaan olin jäänyt jäljelle, lankesin minä kasvoilleni, huusin ja sanoin: "Voi Herra, Herra! Hävitätkö sinä Israelin koko jäännöksen, kun vuodatat vihasi Jerusalemin ylitse?"
\par 9 Niin hän sanoi minulle: "Israelin ja Juudan heimon syntivelka on ylen suuri; maa on täynnä verivelkoja, ja kaupunki on täynnä oikeuden vääristelyä, sillä he sanovat: 'Herra on hyljännyt tämän maan, ei Herra näe'.
\par 10 Niinpä minäkään en sääli enkä armahda. Minä annan heidän vaelluksensa tulla heidän oman päänsä päälle."
\par 11 Ja katso, pellavavaatteisiin puettu mies, jolla oli kirjoitusneuvot vyöllänsä, toi tämän sanan: "Minä olen tehnyt, niinkuin sinä käskit minun tehdä".

\chapter{10}

\par 1 Sitten minä näin, ja katso: taivaanvahvuuden yläpuolella, joka oli kerubien pään päällä, oli ikäänkuin safiirikivi, näöltään valtaistuimen kaltainen; se näkyi heidän yläpuolellansa.
\par 2 Ja hän sanoi pellavavaatteisiin puetulle miehelle näin: "Mene rattaiden väliin, kerubin alle ja täytä kahmalosi tulisilla hiilillä kerubien välistä ja sirottele ne yli kaupungin". Niin hän meni minun silmäini edessä.
\par 3 Ja kerubit seisoivat temppelin oikealla puolella, kun mies meni; ja pilvi peitti sisemmän esipihan.
\par 4 Ja Herran kirkkaus kohosi kerubin yltä temppelin kynnykselle, ja pilvi täytti temppelin, ja esipiha tuli täyteen Herran kirkkauden hohdetta.
\par 5 Ja kerubien siipien kohina kuului ulompaan esipihaan asti kuin kaikkivaltiaan Jumalan ääni, kun hän puhuu.
\par 6 Ja kun hän käski pellavavaatteisiin puettua miestä sanoen: "Ota tulta rattaiden välistä, kerubien välistä", niin tämä meni ja seisahtui pyörän ääreen.
\par 7 Ja kerubi ojensi kätensä kerubien välitse tuleen, joka oli kerubien välissä, otti sitä ja antoi pellavavaatteisiin puetun kahmaloon. Tämä otti sen ja meni ulos.
\par 8 Silloin näkyi kerubeilla ikäänkuin ihmiskäsi heidän siipiensä alta.
\par 9 Ja minä näin, ja katso: kerubien kohdalla oli neljä pyörää, yksi pyörä aina yhden kerubin ohella, ja pyörät olivat näöltään niinkuin krysoliittikivi.
\par 10 Ja niillä neljällä näytti olevan sama muoto, niinkuin olisi ollut sisäkkäin pyörä pyörässä.
\par 11 Ne kulkivat neljään eri suuntaansa, kun kulkivat. Kulkiessaan ne eivät kääntyneet, sillä minnepäin etumainen meni, sinne ne kulkivat sen jäljessä; ne eivät kääntyneet kulkiessaan.
\par 12 Ja niillä oli koko ruumis ja selät ja kädet ja siivet ja pyörät täynnä silmiä, yltympäri.
\par 13 Niillä neljällä oli pyöränsä; ja pyöriä nimitettiin minun kuulteni "rattaiksi".
\par 14 Ja kullakin oli neljät kasvot: yhdet kerubinkasvot, toiset ihmiskasvot, kolmannet leijonankasvot ja neljännet kotkankasvot.
\par 15 Ja kerubit kohosivat - ne olivat samat olennot, jotka minä olin nähnyt Kebar-joen varrella. -
\par 16 Kun kerubit kulkivat, kulkivat pyörät niiden ohella, ja kun kerubit nostivat siipensä kohotaksensa ylös maasta, eivät myöskään pyörät kääntyneet pois niiden ohelta.
\par 17 Kun ne seisahtuivat, seisahtuivat nämäkin, ja kun ne kohosivat, kohosivat nämä niiden kanssa, sillä olentojen henki oli näissä.
\par 18 Sitten Herran kirkkaus lähti temppelin kynnykseltä ja asettui kerubien ylle.
\par 19 Ja lähtiessään kerubit nostivat siipensä ja kohosivat maasta minun silmäini edessä, ja pyörät samalla kuin ne; ja ne seisahtuivat Herran huoneen itäportin ovelle, ja niiden yläpuolella oli Israelin Jumalan kirkkaus.
\par 20 Nämä olivat samat olennot, jotka minä olin nähnyt Israelin Jumalan alla Kebar-joen varrella; ja minä tulin tietämään, että ne olivat kerubeja.
\par 21 Neljät kasvot oli kullakin ja neljä siipeä kullakin, ja niiden siipien alla oli ikäänkuin ihmiskädet.
\par 22 Ja niiden kasvot olivat muodoltaan samat, jotka minä olin nähnyt - sekä niiden näön että ne itsensä - Kebar-joen varrella. Ne kulkivat kukin suoraan eteenpäin.

\chapter{11}

\par 1 Sitten Henki nosti minut ja vei minut Herran huoneen itäportille, joka on itään päin. Ja katso: portin ovella oli kaksikymmentä viisi miestä, ja minä näin niitten keskellä Jaasanjan, Assurin pojan, ja Pelatjan, Benajan pojan, kansan päämiehet.
\par 2 Ja hän sanoi minulle: "Ihmislapsi, nämä ovat ne miehet, jotka miettivät turmiota ja pitävät pahaa neuvoa tätä kaupunkia vastaan,
\par 3 jotka sanovat: 'Ei ole nyt kohta rakennettava taloja: se on pata, me olemme liha'.
\par 4 Sentähden ennusta heitä vastaan, ennusta, ihmislapsi."
\par 5 Niin Herran Henki laskeutui minun päälleni ja sanoi minulle: "Sano: Näin sanoo Herra: Näin te, Israelin heimo, sanotte, ja minä tunnen, mitä teidän hengestänne nousee.
\par 6 Paljon on teidän surmaamianne tässä kaupungissa, ja sen kadut te olette surmatuilla täyttäneet.
\par 7 Sentähden, näin sanoo Herra, Herra: Surmattunne, jotka te olette heittäneet sen keskelle, ne ovat liha, ja tämä kaupunki on pata, mutta teidät minä sen keskeltä vien pois.
\par 8 Miekkaa te pelkäätte, mutta miekan minä annan tulla teidän päällenne, sanoo Herra, Herra.
\par 9 Pois minä vien teidät sen keskeltä ja annan teidät muukalaisten käsiin ja teen teidän seassanne tuomiot.
\par 10 Te kaadutte miekkaan, Israelin rajalla minä teidät tuomitsen; ja te tulette tietämään, että minä olen Herra.
\par 11 Ei tämä tule teille padaksi ettekä te siihen lihaksi: Israelin rajalla minä teidät tuomitsen.
\par 12 Ja te tulette tietämään, että minä olen Herra - te, jotka ette ole vaeltaneet minun käskyjeni mukaan ettekä ole tehneet minun oikeuksieni mukaan, vaan olette tehneet niiden pakanain oikeuksien mukaan, jotka ovat teidän ympärillänne."
\par 13 Kun minä ennustin, niin Pelatja, Benajan poika, kuoli. Niin minä lankesin kasvoilleni, huusin kovalla äänellä ja sanoin: "Voi Herra, Herra! Aivanko sinä teet lopun Israelin jäännöksestä?"
\par 14 Sitten tuli minulle tämä Herran sana:
\par 15 "Ihmislapsi, sinun veljesi, sinun veljesi, sukulaisesi ja koko Israelin heimo, kaikki, joille Jerusalemin asukkaat sanovat: 'Olkaa te vain kaukana Herrasta, meille on tämä maa annettu perinnöksi!'
\par 16 Sano sentähden: Näin sanoo Herra, Herra: Koska minä olen vienyt heidät kauas pakanain sekaan ja hajottanut heidät muihin maihin, olen minä ollut heille pyhäkkönä vähän aikaa niissä maissa, joihin he ovat joutuneet.
\par 17 Sano sentähden: Näin sanoo Herra, Herra: Minä kokoan teidät kansojen seasta ja kerään teidät maista, joihin teidät on hajotettu, ja annan teille Israelin maan.
\par 18 Ja he tulevat sinne ja poistavat siitä kaikki sen iljetykset ja kaikki sen kauhistukset.
\par 19 Ja minä annan heille yhden sydämen, ja uuden hengen minä annan teidän sisimpäänne, ja minä poistan kivisydämen heidän ruumiistansa ja annan heille lihasydämen,
\par 20 niin että he vaeltavat minun käskyjeni mukaan ja noudattavat minun oikeuksiani ja pitävät ne; ja he ovat minun kansani, ja minä olen heidän Jumalansa.
\par 21 Mutta joitten sydän vaeltaa heidän iljetystensä ja kauhistustensa mieltä myöten, niitten vaelluksen minä annan tulla heidän oman päänsä päälle, sanoo Herra, Herra."
\par 22 Sitten kerubit nostivat siipensä ja pyörät samalla kuin ne, ja Israelin Jumalan kirkkaus oli heidän yläpuolellansa;
\par 23 ja Herran kirkkaus kohosi ylös, pois kaupungin keskeltä, ja asettui vuorelle, joka on itään päin kaupungista.
\par 24 Mutta Henki nosti ja vei minut Kaldeaan pakkosiirtolaisten tykö, näyssä, Jumalan Hengen voimalla; ja näky, jonka olin nähnyt, katosi minulta.
\par 25 Ja minä puhuin pakkosiirtolaisille kaikki Herran sanat, jotka hän oli minulle näyttänyt.

\chapter{12}

\par 1 Ja minulle tuli tämä Herran sana:
\par 2 "Ihmislapsi, sinä asut uppiniskaisen suvun keskellä, niiden, joilla on silmät nähdä, mutta eivät näe, ja korvat kuulla, mutta eivät kuule; sillä he ovat uppiniskainen suku.
\par 3 Sinä, ihmislapsi, tee itsellesi matkavarusteet ja mene päiväiseen aikaan heidän silmäinsä edessä pakkosiirtolaisuuteen; sinun on mentävä pakkosiirtolaisuuteen omalta paikkakunnaltasi toiseen paikkakuntaan heidän silmäinsä edessä. Ehkäpä he näkevät sen, sillä he ovat uppiniskainen suku.
\par 4 Ja vie ulos varusteesi, niinkuin pakkosiirtolaiset varusteensa, päiväiseen aikaan heidän silmäinsä edessä; mutta lähde itse illalla heidän silmäinsä edessä, niinkuin pakkosiirtolaiset lähtevät.
\par 5 Murtaudu läpi seinän heidän silmäinsä edessä ja vie ne siitä ulos.
\par 6 Nosta ne hartioillesi heidän silmäinsä edessä ja vie ulos pilkkopimeässä; peitä kasvosi, niin ettet näe maata. Sillä ennusmerkiksi minä olen sinut pannut Israelin heimolle."
\par 7 Ja minä tein, niinkuin minua oli käsketty: minä vein ulos päiväiseen aikaan varusteeni, niinkuin pakkosiirtolaiset varusteensa, mutta illalla minä murtauduin käsin läpi seinän; pilkkopimeässä minä ne vein ulos, minä nostin ne hartioilleni heidän silmäinsä edessä.
\par 8 Mutta aamulla minulle tuli tämä Herran sana:
\par 9 "Ihmislapsi, eikö ole Israelin heimo, uppiniskainen suku, sinulle sanonut: 'Mitä sinä teet?'
\par 10 Sano heille: Näin sanoo Herra, Herra: Ruhtinasta, joka on Jerusalemissa, koskee tämä ennustus ja kaikkea Israelin heimoa, mikä on heidän keskuudessaan.
\par 11 Sano: Minä olen teille ennusmerkki: niinkuin minä tein, niin heille tehdään, pakkosiirtolaisuuteen, vankeuteen he menevät.
\par 12 Ruhtinas, joka on heidän keskellänsä, nostaa nostettavansa hartioilleen ja lähtee ulos pilkkopimeässä; seinästä he murtautuvat läpi, viedäkseen siitä ulos vietävänsä. Kasvonsa hän peittää, koska hän ei ole silmillänsä näkevä maata.
\par 13 Minä levitän verkkoni hänen ylitsensä, ja hän takertuu minun pyydykseeni; minä vien hänet Baabeliin, kaldealaisten maahan, mutta sitä hän ei saa nähdä, ja sinne hän kuolee.
\par 14 Ja kaikki, keitä hänellä on ympärillään apunansa, ja kaikki hänen sotajoukkonsa minä hajotan kaikkiin tuuliin, ja minä ajan niitä takaa paljastetulla miekalla.
\par 15 Ja he tulevat tietämään, että minä olen Herra, kun minä hajotan heidät pakanain sekaan ja sirotan heidät muihin maihin.
\par 16 Mutta vähäisen määrän heitä minä jätän miekasta, nälästä ja rutosta jäljelle kertomaan pakanain seassa, siellä, minne he joutuvat, kaikista kauhistuksistansa; ja ne tulevat tietämään, että minä olen Herra."
\par 17 Ja minulle tuli tämä Herran sana:
\par 18 "Ihmislapsi, syö leipäsi vavisten, juo vetesi väristen ja huolenalaisena
\par 19 ja sano maan kansalle: Näin sanoo Herra, Herra Jerusalemin asukkaista Israelin maassa: He tulevat syömään leipänsä huolenalaisina ja juomaan vetensä kauhun vallassa, ja sen maa tulee autioksi kaikesta, mitä siinä on, kaikkien siinä asuvaisten väkivaltaisuuden tähden.
\par 20 Asutut kaupungit joutuvat raunioiksi, ja maa tulee autioksi; ja te tulette tietämään, että minä olen Herra."
\par 21 Ja minulle tuli tämä Herran sana:
\par 22 "Ihmislapsi, miksi teillä on tämä pilkkalause Israelin maasta, kun sanotte: 'Aika venyy pitkälle, ja tyhjään raukeavat kaikki näyt?'
\par 23 Sentähden sano heille: Näin sanoo Herra, Herra: Minä teen lopun tästä pilkkalauseesta, niin ettei sitä enää lausuta Israelissa. Sano heille päinvastoin: Lähellä on aika ja kaikkien näkyjen täyttymys.
\par 24 Sillä ei ole oleva enää petollista näkyä, ei liukasta ennustelua Israelin heimon keskuudessa.
\par 25 Sillä minä, Herra, minä puhun; ja minkä sanan minä puhun, se tapahtuu eikä enää viivy. Sillä teidän aikananne, uppiniskainen suku, minä sanan puhun ja sen myös täytän, sanoo Herra, Herra."
\par 26 Ja minulle tuli tämä Herran sana:
\par 27 "Ihmislapsi, katso, Israelin heimo sanoo: 'Niihin näkyihin, joita tämä näkee, on päiviä paljon, ja hän ennustaa kaukaisista ajoista'.
\par 28 Sentähden sano heille: Näin sanoo Herra, Herra: Ei viivy enää yksikään minun sanani: Minkä sanan minä puhun, se tapahtuu, sanoo Herra, Herra."

\chapter{13}

\par 1 Ja minulle tuli tämä Herran sana:
\par 2 "Ihmislapsi, ennusta Israelin ennustavista profeetoista ja sano noille, jotka ovat profeettoja oman sydämensä voimasta: Kuulkaa Herran sana.
\par 3 Näin sanoo Herra, Herra: Voi profeettahoukkia, jotka seuraavat omaa henkeänsä ja sitä, mitä eivät ole nähneet!
\par 4 Niinkuin ketut raunioissa ovat sinun profeettasi, Israel.
\par 5 Te ette ole nousseet muurin aukkoihin ettekä korjanneet muuria Israelin heimon ympärillä, että se kestäisi sodassa, Herran päivänä.
\par 6 Mitä he ovat nähneet, on petosta, ja heidän ennustelunsa on valhetta, kun he sanovat: 'Näin sanoo Herra', vaikka Herra ei ole heitä lähettänyt; ja he muka odottavat, että hän toteuttaisi heidän sanansa.
\par 7 Ettekö ole petollisia näkyjä nähneet ja valhe-ennusteluja puhuneet, kun olette sanoneet: 'Näin sanoo Herra', vaikka minä en ole puhunut?
\par 8 Sentähden, näin sanoo Herra, Herra: Koska te olette puhuneet petosta ja nähneet valhenäkyjä, niin sentähden, katso, minä käyn teidän kimppuunne, sanoo Herra, Herra.
\par 9 Minun käteni on profeettoja vastaan, jotka petosnäkyjä näkevät ja valhetta ennustelevat. Ei pidä heidän oleman minun kansani yhteydessä, ei heitä kirjoiteta Israelin heimon kirjaan, eivätkä he tule Israelin maahan; ja te tulette tietämään, että minä olen Herra, Herra.
\par 10 Koska he, koska he vievät minun kansani harhaan, sanoen: 'rauha!', vaikka ei rauhaa ole, ja katso, koska he, kun kansa rakentaa seinän, valkaisevat sen kalkilla,
\par 11 niin sano noille kalkilla-valkaisijoille, että se kaatuu. Tulee kaatosade, te syöksytte alas, raekivet, ja sinä pusket puhki, myrskynpuuska,
\par 12 ja katso, seinä kaatuu! Eikö silloin teiltä kysytä: 'Missä on valkaisu, jonka olette sivelleet?'
\par 13 Sentähden, näin sanoo Herra, Herra: Minä kiivaudessani annan myrskynpuuskan puhjeta, kaatosade tulee minun vihastani ja raekivet minun kiivaudestani, niin että tulee loppu.
\par 14 Minä hajotan seinän, jonka te olitte kalkilla valkaisseet, ja syöksen sen maahan, niin että sen perustus paljastuu; niin se kaatuu, ja te saatte siinä lopun. Ja te tulette tietämään, että minä olen Herra.
\par 15 Niin minä panen kiivauteni täytäntöön seinässä ja niissä, jotka sen kalkilla valkaisivat, ja sanon teille: Ei ole enää seinää eikä sen valkaisijoita,
\par 16 Israelin profeettoja, jotka ennustivat Jerusalemista ja näkivät sille rauhan näkyjä, vaikka ei rauhaa ollut; sanoo Herra, Herra.
\par 17 Ja sinä, ihmislapsi, käännä kasvosi kansasi tyttäriä vastaan, jotka joutuvat hurmoksiin oman sydämensä voimasta, ja ennusta heitä vastaan
\par 18 ja sano: Näin sanoo Herra, Herra: Voi niitä naisia, jotka sitovat taikasiteet kaikkiin ranteisiin ja tekevät hunnut kaikenkorkuisiin päihin pyydystääksensä sieluja! Toisia sieluja te pyydystätte minun kansaltani pois, toisten sielujen annatte elää omaksi eduksenne.
\par 19 Ja te häpäisette minut kansani edessä muutamista ohrakourallisista ja leipäpalasista, kun kuoletatte sieluja, joiden ei olisi kuoltava, ja annatte elää sielujen, jotka eivät saisi elää; kun valhettelette minun kansalleni, joka kuuntelee valhetta.
\par 20 Sentähden, näin sanoo Herra, Herra: Minä käyn käsiksi teidän taikasiteisiinne, joilla olette pyydystäneet sieluja niinkuin lintuja, ja revin ne teidän käsivarsistanne ja lasken ne sielut irti - sielut, jotka te olette pyydystäneet - niinkuin linnut.
\par 21 Ja minä revin teidän huntunne ja pelastan kansani teidän käsistänne, niin etteivät he enää joudu teidän kätenne saaliiksi; ja te tulette tietämään, että minä olen Herra.
\par 22 Koska olette valheella murehduttaneet vanhurskaan sydämen, vaikka minä en tahtonut häntä murehduttaa, ja olette vahvistaneet jumalattoman käsiä, ettei hän kääntyisi pahalta tieltänsä ja saisi elää,
\par 23 sentähden ette enää saa petosnäkyjä nähdä ettekä ennustella ennustelujanne, vaan minä pelastan kansani teidän käsistänne; ja te tulette tietämään, että minä olen Herra."

\chapter{14}

\par 1 Minun luokseni tuli miehiä Israelin vanhinten joukosta, ja he istuivat minun eteeni.
\par 2 Niin minulle tuli tämä Herran sana:
\par 3 "Ihmislapsi, nämä miehet ovat sulkeneet kivijumalansa sydämeensä ja asettaneet kasvojensa eteen sen, mikä heille oli kompastukseksi syntiin. Antaisinko minä heidän kysyä minulta neuvoa?
\par 4 Sentähden puhuttele heitä ja sano heille: Näin sanoo Herra, Herra: Kuka ikinä israelilainen sulkee kivijumalansa sydämeensä ja asettaa kasvojensa eteen sen, mikä hänelle oli kompastukseksi syntiin, ja sitten menee profeetan tykö, hänelle minä, Herra, kyllä annan vastauksen - hänelle ja hänen monille kivijumalilleen,
\par 5 ja tartun Israelin heimoa sydämeen, koska he ovat kääntyneet minusta pois kaikki tyynni seuraamalla kivijumaliansa.
\par 6 Sentähden sano Israelin heimolle: Näin sanoo Herra, Herra: Kääntykää, ja kääntykää pois kivijumalistanne ja kääntäkää kasvonne pois kaikista kauhistuksistanne.
\par 7 Sillä kuka ikinä israelilainen tai muukalainen, joka asuu Israelissa, luopuu minusta ja sulkee kivijumalansa sydämeensä ja asettaa kasvojensa eteen sen, mikä hänelle oli kompastukseksi syntiin, ja sitten menee profeetan tykö kysymään itsellensä neuvoa minulta, hänelle minä, Herra, itse annan vastauksen.
\par 8 Minä käännän kasvoni sitä miestä vastaan, teen hänet merkiksi ja sananlaskuksi ja hävitän hänet kansastani; ja te tulette tietämään, että minä olen Herra.
\par 9 Mutta jos profeetta antaa viekoitella itsensä ja sanoo sanan, niin minä, Herra, olen sen profeetan viekoitellut; ja minä ojennan käteni häntä vastaan ja hävitän hänet kansastani Israelista.
\par 10 Heidän on kannettava syntinsä - niinkuin kysyjän synti, niin on profeetankin synti -
\par 11 etteivät he, Israelin heimo, enää eksyisi minusta pois eivätkä enää saastuttaisi itseänsä kaikilla rikkomuksillaan, vaan olisivat minun kansani ja minä olisin heidän Jumalansa; sanoo Herra, Herra."
\par 12 Ja minulle tuli tämä Herran sana:
\par 13 "Ihmislapsi! Jos maa tekisi syntiä minua vastaan olemalla uskoton ja minä ojentaisin sitä vastaan käteni ja murtaisin siltä leivän tuen ja lähettäisin siihen nälän ja hävittäisin siitä ihmiset ja eläimet,
\par 14 ja sen keskellä olisivat nämä kolme miestä: Nooa, Daniel ja Job, niin oman henkensä he vanhurskaudellaan pelastaisivat; sanoo Herra, Herra.
\par 15 Jos minä antaisin pahain petoeläinten käydä halki maan ja ne riistäisivät siltä lapset ja se tulisi autioksi, niin ettei kukaan siellä kulkisi petoeläinten tähden,
\par 16 niin nämä kolme miestä, jos olisivat sen keskellä, eivät voisi - niin totta kuin minä elän, sanoo Herra, Herra - pelastaa poikiansa eikä tyttäriänsä; ainoastaan he itse pelastuisivat, mutta maa tulisi autioksi.
\par 17 Taikka jos minä antaisin miekan tulla tämän maan yli ja sanoisin: 'Miekka, käy halki maan!' ja hävittäisin siitä ihmiset ja eläimet,
\par 18 ja sen keskellä olisivat nämä kolme miestä, eivät he voisi - niin totta kuin minä elän, sanoo Herra, Herra - pelastaa poikiansa eikä tyttäriänsä, vaan ainoastaan he itse pelastuisivat.
\par 19 Taikka, jos minä lähettäisin ruton tähän maahan ja vuodattaisin vihani sen ylitse verenä, niin että hävittäisin siitä ihmiset ja eläimet,
\par 20 ja Nooa ja Daniel ja Job olisivat sen keskellä, eivät he voisi - niin totta kuin minä elän, sanoo Herra, Herra - pelastaa poikaansa eikä tytärtänsä; vain itse he vanhurskaudellaan pelastaisivat henkensä.
\par 21 Mutta näin sanoo Herra, Herra: Mitä sitten, kun minä lähetän Jerusalemin yli neljä kovaa tuomiotani: miekan, nälän, pahat petoeläimet ja ruton hävittämään siitä ihmiset ja eläimet!
\par 22 Mutta katso, sinne on jäävä jäljelle pelastuneita, poikia ja tyttäriä, jotka viedään pois. Katso, ne tulevat teidän luoksenne, ja te näette heidän vaelluksensa ja tekonsa ja tulette lohdutetuiksi siitä onnettomuudesta, jonka minä olen tuottanut Jerusalemille, kaikesta, minkä olen sille tuottanut.
\par 23 He lohduttavat teitä, kun näette heidän vaelluksensa ja tekonsa. Ja te tulette tietämään, etten minä syyttä ole tehnyt sitä kaikkea, minkä olen sille tehnyt; sanoo Herra, Herra."

\chapter{15}

\par 1 Ja minulle tuli tämä Herran sana:
\par 2 "Ihmislapsi, mikä parempi on viinipuu kuin kaikki muut puut, tuo köynnös, joka on metsän puitten seassa?
\par 3 Otetaanko siitä puuta, josta tehdään tarviskalu? Otetaanko siitä vaarnaakaan, johon ripustetaan kaikkea tavaraa?
\par 4 Katso, se annetaan tulen kulutettavaksi. Sen molemmat päät on tuli kuluttanut, ja sen keskus on kärventynyt - kelpaako se tarviskaluksi?
\par 5 Katso, silloinkaan, kun se eheä oli, ei siitä voinut tarviskalua tehdä, saati sitten, kun tuli on sen kuluttanut ja se on kärventynyt - silloinko siitä tehtäisiin tarviskalu?
\par 6 Sentähden, näin sanoo Herra, Herra: Niinkuin metsän puitten seassa käy viinipuulle, jonka minä olen antanut tulen kulutettavaksi, niin minä annan käydä myös Jerusalemin asukkaille.
\par 7 Minä käännän kasvoni heitä vastaan: tulesta he ovat päässeet, mutta tuli heidät kuluttaa; ja te tulette tietämään, että minä olen Herra, kun minä käännän kasvoni heitä vastaan.
\par 8 Ja minä teen maan autioksi, koska he ovat olleet uskottomat; sanoo Herra, Herra."

\chapter{16}

\par 1 Ja minulle tuli tämä Herran sana:
\par 2 "Ihmislapsi, ilmoita Jerusalemille sen kauhistukset
\par 3 ja sano: Näin sanoo Jerusalemille Herra, Herra: Sinun sukusi ja syntysi on kanaanilaisten maasta; isäsi oli amorilainen ja äitisi heettiläinen.
\par 4 Ja näin oli sinun syntymäsi: sinä päivänä, jona synnyit, sinulta ei leikattu napanuoraa, sinua ei pesty vedellä, että olisit puhdistunut, sinua ei hierottu suolalla eikä sinua kääritty kapaloihin.
\par 5 Ei kenkään sinua säälinyt, niin että olisi tehnyt sinulle mitään tällaista ja armahtanut sinua, vaan sinut pantiin heitteille kedolle: niin halpana pidettiin sinun henkesi sinä päivänä, jona synnyit.
\par 6 Mutta minä kuljin ohitsesi ja näin sinut, kun sätkyttelit verissäsi. Ja minä sanoin sinulle, kun olit siinä verissäsi: 'Sinun pitää elämän' - niin sanoin minä sinulle, kun olit siinä verissäsi: 'Sinun pitää elämän,
\par 7 minä teen sinut kymmentuhansiksi kuin pellon laihon'. Sitten sinä vartuit, tulit isoksi ja ehdit kauneimpaan kukoistukseesi, rintasi paisuivat, ja hiuksesi kasvoivat; mutta vielä sinä olit alaston ja paljas.
\par 8 Niin minä kuljin ohitsesi ja näin sinut, ja katso, sinun aikasi oli lemmen aika. Ja minä levitin liepeeni sinun ylitsesi ja peitin häpysi. Ja minä vannoin sinulle ja menin liittoon sinun kanssasi, sanoo Herra, Herra; ja sinä tulit minun omakseni.
\par 9 Minä pesin sinut vedellä, huuhtelin sinut verestäsi ja voitelin sinut öljyllä.
\par 10 Minä puetin sinut kirjaeltuihin vaatteisiin, kengitsin sinut sireeninnahkakenkiin, sidoin päähäsi hienopellavaisen siteen ja hunnutin sinut silkillä.
\par 11 Minä koristin sinut koruilla, panin rannerenkaat käsiisi ja käädyt kaulaasi,
\par 12 panin nenärenkaan nenääsi, korvarenkaat korviisi ja päähäsi kauniin kruunun.
\par 13 Niin koristettiin sinut kullalla ja hopealla, sinun pukusi oli hienoa pellavaa, silkkiä ja kirjaeltua vaatetta, ja sinä sait syödä lestyjä jauhoja, hunajata ja öljyä. Sinusta tuli ylenmäärin kaunis, ja sinä kelpasit kuninkaalliseen arvoon.
\par 14 Ja sinun maineesi kulki pakanakansoihin kauneutesi tähden, sillä se oli täydellinen niiden kaunistusten takia, jotka minä sinun yllesi panin; sanoo Herra, Herra.
\par 15 Mutta sinä luotit kauneuteesi ja harjoitit haureutta maineesi nojalla ja vuodatit haureuttasi jokaiselle ohikulkijalle: 'Saakoon tuokin!'
\par 16 Sinä otit vaatteitasi ja teit itsellesi kirjavia uhrikukkuloita ja harjoitit haureutta niiden päällä - moista ei ole tapahtunut eikä ole tapahtuva -
\par 17 Sinä otit korukalujasi, minun kultaani ja hopeatani, jota minä olin sinulle antanut, ja teit itsellesi miehenkuvia ja harjoitit haureutta niiden kanssa.
\par 18 Sinä otit kirjaeltuja vaatteitasi ja verhosit ne niillä; ja minun öljyni ja suitsukkeeni sinä panit niiden eteen.
\par 19 Ruokani, jota minä sinulle annoin - minähän syötin sinua lestyillä jauhoilla, öljyllä ja hunajalla - sinä panit niiden eteen suloiseksi tuoksuksi. Näin tapahtui, sanoo Herra, Herra.
\par 20 Sinä otit poikasi ja tyttäresi, jotka olit minulle synnyttänyt, ja teurastit heidät niiden syödä. Eikö riittänyt sinulle haureutesi,
\par 21 kun vielä teurastit minun poikani ja annoit heidät poltettaviksi uhrina niille?
\par 22 Ja kaikkien kauhistustesi ja haureutesi ohessa sinä et muistanut nuoruutesi päiviä, jolloin olit alaston ja paljas ja sätkyttelit verissäsi.
\par 23 Ja kaiken muun pahuutesi lisäksi - voi, voi sinua! sanoo Herra, Herra -
\par 24 sinä rakensit itsellesi korokkeita ja teit itsellesi kumpuja kaikille toreille;
\par 25 kaikkiin kadunkulmiin sinä rakensit kumpujasi ja häpäisit kauneutesi, levitit sääresi jokaiselle ohikulkijalle ja yhä lisäsit haureuttasi.
\par 26 Sinä harjoitit haureutta naapuriesi, Egyptin suurijäsenisten poikain, kanssa ja yhä lisäsit haureuttasi ja niin vihoitit minut.
\par 27 Ja katso, minä ojensin käteni sinua vastaan ja vähensin sinun määräosasi ja jätin sinut vihollistesi, filistealaisten tyttärien, raivon valtaan, jotka häpesivät sinun iljettävää vaellustasi.
\par 28 Sitten sinä harjoitit haureutta Assurin poikain kanssa, koska et voinut saada kyllääsi; ja vaikka harjoitit haureutta heidän kanssansa, et sittenkään kyllääsi saanut.
\par 29 Sitten sinä yhä enensit haureuttasi kauppiasten maahan päin, Kaldeaan, mutta et siitäkään saanut kyllääsi.
\par 30 Kuinka himosta hiukeava olikaan sinun sydämesi, sanoo Herra, Herra, kun teit tämän kaiken, niinkuin tekee itse pääportto;
\par 31 kun rakensit korokkeesi kaikkiin kadunkulmiin ja teit kumpusi kaikille toreille! Mutta siinä sinä olit erilainen kuin muut portot, että halveksuit portonpalkkaa -
\par 32 avionrikkoja-vaimo, joka miehensä sijaan ottaa vieraita!
\par 33 Kaikille muille portoille annetaan lahja, mutta sinä annoit portonlahjojasi kaikille rakastajillesi ja haureudessasi lahjoit heitä tulemaan luoksesi joka taholta.
\par 34 Sinun oli haureudessasi laita päinvastoin kuin muitten naisten: sinun perässäsi ei juostu haureuteen, ja sinä maksoit portonpalkkaa, mutta sinulle portonpalkkaa ei maksettu; näin se oli päinvastoin.
\par 35 Sentähden, sinä portto, kuule Herran sana:
\par 36 Näin sanoo Herra, Herra: Koska olet antanut riettautesi vuotaa ja olet paljastanut häpysi haureudessasi kaikille rakastajillesi ja kaikille kauhistaville kivijumalillesi, sekä lastesi veren tähden, jonka olet niille antanut,
\par 37 sentähden, katso, minä kokoan kaikki sinun rakastajasi, jotka olivat sinulle mieleen, kaikki, joita sinä rakastit, ja kaikki, joihin kyllästyit, ne minä kokoan sinua vastaan joka taholta ja paljastan heille sinun häpysi, niin että he näkevät koko häpysi.
\par 38 Minä tuomitsen sinut sen mukaan, mitä on säädetty avionrikkoja- ja verenvuodattaja-naisista, ja annan vihassa ja kiivaudessa sinun veresi vuotaa.
\par 39 Ja minä annan sinut heidän käsiinsä, ja he hajottavat sinun korokkeesi, kukistavat sinun kumpusi, raastavat sinulta vaatteesi, ottavat korukalusi ja jättävät sinut alastomaksi ja paljaaksi.
\par 40 He nostattavat kansanjoukon sinua vastaan, ja ne kivittävät sinut, hakkaavat sinut maahan miekoillansa,
\par 41 polttavat tulella sinun talosi ja panevat sinussa toimeen tuomiot paljojen naisten silmäin edessä. Minä teen lopun sinun porttona-olostasi, etkä sinä enää sitten portonpalkkoja maksa.
\par 42 Niin minä tyydytän vihani sinussa, niin että minun kiivauteni sinusta väistyy, ja minä tyynnyn enkä enää ole vihastunut.
\par 43 Koska et muistanut nuoruutesi päiviä, vaan ärsytit minut kaikilla näillä, niin katso: minäkin annan sinun vaelluksesi tulla oman pääsi päälle, sanoo Herra, Herra. Etkö sinä tehnyt näitä iljetyksiä kaikkien kauhistustesi lisäksi?
\par 44 Katso, jokainen, joka sananlaskuja lausuu, on lausuva sinusta: 'Tytär tulee äitiinsä'.
\par 45 Sinä olet äitisi tytär, hänen, joka vieroi miestään ja lapsiansa, ja olet sisartesi sisar, heidän, jotka vieroivat miehiään ja lapsiansa: teidän äitinne oli heettiläinen ja isänne amorilainen.
\par 46 Isompi sisaresi oli Samaria tyttärineen, joka asui vasemmalla puolellasi, ja pienempi sisaresi, joka asui oikealla puolellasi, oli Sodoma tyttärineen.
\par 47 Mutta sinä et vaeltanut heidän teitänsä etkä tehnyt kauhistuksia samalla tavoin kuin he: vähän aikaa vain, niin sinä jo teit kelvottomammin kuin he kaikilla teilläsi.
\par 48 Niin totta kuin minä elän, sanoo Herra, Herra: ei totisesti sisaresi Sodoma tyttärineen tehnyt sitä, mitä sinä tyttärinesi olet tehnyt.
\par 49 Katso, tämä oli sisaresi Sodoman synti: ylpeys, leivän yltäkylläisyys ja huoleton lepo hänellä ja hänen tyttärillään; mutta kurjaa ja köyhää hän ei kädestä ottanut.
\par 50 He korskeilivat ja tekivät kauhistuksia minun edessäni, ja minä, kun sen näin, toimitin heidät pois.
\par 51 Ja Samaria ei tehnyt puoltakaan sinun syntiesi vertaa; mutta sinä olet tehnyt kauhistuksia paljon enemmän kuin he, niin että olet kaikilla kauhistuksillasi, joita olet tehnyt, saanut sisaresi näyttämään vanhurskailta.
\par 52 Niinpä kanna myös sinä häpeäsi, kun olet hankkinut moisen hyvityksen sisarillesi: sinun syntiesi takia, kun olet menetellyt vielä kauhistavammin kuin he, ovat he vanhurskaampia kuin sinä. Häpeä sinäkin ja kanna häpeäsi, kun olet saanut sisaresi näyttämään vanhurskailta.
\par 53 Ja minä tahdon kääntää heidän kohtalonsa: Sodoman ja hänen tyttäriensä kohtalon sekä Samarian ja hänen tyttäriensä kohtalon; ja minä tahdon kääntää sinun kohtalosi, sinun, joka olet heidän keskellänsä,
\par 54 että kantaisit häpeäsi ja olisit häpeissäsi kaikesta, mitä teit, kun lohdutit heidät.
\par 55 Sinun sisaresi - Sodoma tyttärineen saa palata entisellensä, ja Samaria tyttärineen saa palata entisellensä. Ja sinä tyttärinesi saat palata entisellesi.
\par 56 Eikö ollut sisaresi Sodoma huhupuheena sinun suussasi sinun ylpeytesi aikana,
\par 57 ennenkuin sinun oma pahuutesi paljastui, silloin kun jouduit Aramin tyttärien ja kaikkien heidän ympärillään asuvaisten, filistealaisten tyttärien, häväistäväksi, jotka joka taholta pilkkaavat sinua?
\par 58 Sinä saat kantaa iljetyksesi ja kauhistuksesi, sanoo Herra.
\par 59 Sillä näin sanoo Herra, Herra: Minä olen tehnyt sinulle sen mukaan, kuin sinä olet tehnyt, kun olet pitänyt valan halpana ja rikkonut liiton.
\par 60 Mutta minä muistan liittoni, jonka tein sinun kanssasi sinun nuoruutesi päivinä, ja minä teen sinun kanssasi iankaikkisen liiton.
\par 61 Ja sinä muistat vaelluksesi ja häpeät, kun otat vastaan sisaresi, ne, jotka ovat sinua isommat, ynnä ne, jotka ovat sinua pienemmät, ja minä annan heidät sinulle tyttäriksi; mutta en sinun liittosi voimasta.
\par 62 Ja minä teen liittoni sinun kanssasi, ja sinä tulet tietämään, että minä olen Herra.
\par 63 Niin sinä muistat ja häpeät etkä voi häpeäsi tähden suutasi avata, kun minä annan sinulle anteeksi kaikki, mitä sinä tehnyt olet; sanoo Herra, Herra."

\chapter{17}

\par 1 Ja minulle tuli tämä Herran sana:
\par 2 "Ihmislapsi, esitä arvoitus ja lausu vertaus Israelin heimolle
\par 3 ja sano: Näin sanoo Herra, Herra: Suuri kotka, suurisiipinen, pitkäsulkainen, täysihöyheninen, kirjava, tuli Libanonille ja otti latvuksen setripuusta.
\par 4 Hän taittoi siitä latvalehvän ja vei sen kauppiasten maahan, asetti sen kauppurien kaupunkiin.
\par 5 Sitten hän otti siitä maasta taimen ja pani sen kylvöpeltoon. Hän otti sen ja pani pajuksi runsaan veden ääreen.
\par 6 Se versoi, ja siitä tuli rehevä viinipuu, matalakasvuinen; sen oksien tuli kääntyä kotkaan päin ja sen juurten olla hänen allansa. Siitä tuli viinipuu, se teki haaroja, työnsi oksia.
\par 7 Mutta oli toinen suuri kotka, suurisiipinen, runsashöyheninen; ja katso, tämä viinipuu ojensi juurensa sitä kotkaa kohti ja työnsi oksansa siihen päin, sen kasteltaviksi, siitä penkereestä, johon se oli istutettu.
\par 8 Se oli istutettu hyvään peltoon, runsaan veden ääreen, että se tekisi lehviä, kantaisi hedelmää ja tulisi ihanaksi viinipuuksi.
\par 9 Sano: Näin sanoo Herra, Herra: Menestyyköhän se? Eiköhän kotka kisko ylös sen juuria ja raasta sen hedelmiä; niin että kaikki siitä versoneet lehdet kuivuvat ja puu kuivuu? Eikä tarvita suurta voimaa, ei paljoa väkeä sen nostamiseksi juuriltansa.
\par 10 Katso, istutettu se on - menestyyköhän se? Eiköhän se kuivu, kun siihen käy itätuuli, kuivu penkereessä, jossa se versoi?"
\par 11 Ja minulle tuli tämä Herran sana:
\par 12 "Sano uppiniskaiselle suvulle: Ettekö te tiedä, mitä tämä tarkoittaa? Sano: Katso, Baabelin kuningas tuli Jerusalemiin, otti sen kuninkaan ja päämiehet ja vei heidät luoksensa Baabeliin.
\par 13 Ja hän otti yhden kuninkaallisesta suvusta, teki hänen kanssansa liiton ja otti häneltä valan, mutta maan mahtavat hän vei mukanaan,
\par 14 että valtakunta tulisi vähäpätöiseksi eikä kohoaisi, että se pitäisi liittonsa ja liitto pysyisi.
\par 15 Mutta hän kapinoi häntä vastaan ja laittoi lähettiläänsä Egyptiin, että hänelle annettaisiin hevosia ja paljon väkeä. Menestyyköhän se, pelastuukohan se, joka tällaista tekee? Pelastuuko se, joka liiton rikkoo?
\par 16 Niin totta kuin minä elän, sanoo Herra, Herra: sen kuninkaan asuinpaikalla, joka hänet kuninkaaksi teki, jonka valan hän on halpana pitänyt ja jonka liiton hän on rikkonut, sen luona, keskellä Baabelia, hänen totisesti on kuoltava.
\par 17 Ja farao suurella sotajoukollaan ja paljolla väellään ei tee mitään hänen hyväkseen sodassa, kun luodaan valli ja rakennetaan saartovarusteet paljojen ihmisten hävittämiseksi.
\par 18 Hän piti halpana valan ja rikkoi liiton; katso, vaikka oli kättä lyönyt, hän teki kaiken tämän - ei hän pelastu.
\par 19 Sentähden, näin sanoo Herra, Herra: Niin totta kuin minä elän, niin valani, jonka hän halpana piti, ja liittoni, jonka hän rikkoi, minä annan totisesti tulla hänen päänsä päälle.
\par 20 Minä levitän verkkoni hänen ylitsensä, ja hän takertuu minun pyydykseeni. Minä vien hänet Baabeliin ja käyn siellä oikeutta hänen kanssansa hänen uskottomuudestaan, jota hän on osoittanut minua kohtaan.
\par 21 Ja kaikki hänen pakolaisensa kaikista hänen sotajoukoistaan kaatuvat miekkaan, ja jäljellejääneet hajotetaan kaikkiin tuuliin. Ja te tulette tietämään, että minä, Herra, olen puhunut.
\par 22 Näin sanoo Herra, Herra: Mutta minä otan yhden latvuksen siitä korkeasta setripuusta ja istutan sen; hennon latvalehvän minä siitä taitan ja istutan korkealle ja jyrkälle vuorelle.
\par 23 Israelin vuoren korkeuteen minä sen istutan: ja se kantaa lehviä ja tekee hedelmää, ja siitä tulee mahtava setri. Ja sen alla asuvat kaikki linnut, kaikki, mitä siivekästä on; ne asuvat sen oksain varjossa.
\par 24 Ja kaikki metsän puut tulevat tietämään, että minä olen Herra, joka teen korkean puun matalaksi ja matalan puun korkeaksi, tuoreen puun kuivaksi ja kuivan puun kukoistavaksi. Minä, Herra, Herra, olen puhunut, ja minä sen teen."

\chapter{18}

\par 1 Ja minulle tuli tämä Herran sana:
\par 2 "Mikä teillä on, kun te lausutte tätä pilkkalausetta Israelin maasta: 'Isät söivät raakoja rypäleitä, lasten hampaat heltyivät'?
\par 3 Niin totta kuin minä elän, sanoo Herra, Herra, ei tule teidän enää lausua tätä pilkkalausetta Israelissa.
\par 4 Katso, kaikki sielut ovat minun: niinkuin isän sielu, niin pojankin sielu - ne ovat minun. Se sielu, joka syntiä tekee - sen on kuoltava.
\par 5 Jos mies on vanhurskas ja tekee oikeuden ja vanhurskauden:
\par 6 ei syö uhrivuorilla, ei luo silmiänsä Israelin heimon kivijumaliin, ei saastuta lähimmäisensä vaimoa, ei ryhdy naiseen, joka on kuukautistilassa,
\par 7 ei sorra toista, vaan antaa takaisin velanpantin, ei riistä eikä raasta, vaan antaa leipäänsä nälkäiselle, verhoaa vaatteella alastonta,
\par 8 ei anna rahaansa korolle, ei ota voittoa, vaan pidättää kätensä vääryydestä, tekee oikean tuomion miesten välillä,
\par 9 vaeltaa minun käskyjeni mukaan ja noudattaa minun oikeuksiani, niin että tekee sitä, mikä oikein on - hän on vanhurskas, hän totisesti saa elää, sanoo Herra, Herra.
\par 10 Mutta jos hänelle on syntynyt poika, väkivaltainen, verenvuodattaja, joka tekee yhtäkin näistä,
\par 11 mitä hän itse ei ole tehnyt: syö uhrivuorilla, saastuttaa lähimmäisensä vaimon,
\par 12 sortaa kurjaa ja köyhää, riistää ja raastaa, ei anna takaisin panttia, luo silmänsä kivijumaliin, tekee kauhistuksia,
\par 13 antaa rahansa korolle ja ottaa voittoa - saisiko tämä elää? Ei hän saa elää: kaikkia näitä kauhistuksia hän on tehnyt, hänet on kuolemalla rangaistava; hän on verivelan alainen.
\par 14 Mutta katso, jos tälle on syntynyt poika ja hän näkee kaikki isänsä synnit, jotka tämä on tehnyt - näkee ne eikä tee sellaisia:
\par 15 ei syö uhrivuorilla, ei luo silmiänsä Israelin heimon kivijumaliin, ei saastuta lähimmäisensä vaimoa,
\par 16 ei sorra toista, ei ota panttia, ei riistä eikä raasta, vaan antaa leipäänsä nälkäiselle, verhoaa vaatteella alastonta,
\par 17 pidättää kätensä kurjasta, ei ota korkoa eikä voittoa, vaan tekee minun oikeuksieni mukaan, vaeltaa käskyjeni mukaan - hänen ei ole kuoltava isänsä syntivelan tähden, hän totisesti saa elää.
\par 18 Hänen isänsä taas kun teki väkivallan töitä, riisti ja raastoi veljeltänsä ja teki kansansa keskuudessa sitä, mikä ei ole hyvää, niin katso, hänen oli kuoltava syntivelkansa tähden.
\par 19 Ja vielä te kysytte: 'Minkätähden ei poika kanna isän syntivelkaa?' Kun poika on tehnyt oikeuden ja vanhurskauden, noudattanut kaikkia minun käskyjäni ja tehnyt niitten mukaan, hän totisesti saa elää. Se sielu, joka syntiä tekee - sen on kuoltava.
\par 20 Poika ei kanna isän syntivelkaa, eikä isä kanna pojan syntivelkaa. Vanhurskaan ylitse on tuleva hänen vanhurskautensa, ja jumalattoman ylitse on tuleva hänen jumalattomuutensa.
\par 21 Ja jos jumalaton kääntyy pois kaikesta synnistänsä, mitä hän on tehnyt, ja noudattaa kaikkia minun käskyjäni ja tekee oikeuden ja vanhurskauden, hän totisesti saa elää; ei hänen ole kuoltava.
\par 22 Ei yhtäkään hänen synneistänsä, jotka hän on tehnyt, muisteta; vanhurskautensa tähden, jota hän on noudattanut, hän saa elää.
\par 23 Olisiko minulle mieleen jumalattoman kuolema, sanoo Herra, Herra; eikö se, että hän kääntyy pois teiltänsä ja elää?
\par 24 Ja jos vanhurskas kääntyy pois vanhurskaudestansa ja tekee vääryyttä, tekee kaikkien niiden kauhistusten kaltaisia, joita jumalaton tekee - saisiko hän tehdä niin ja elää? Ei yhtäkään hänen vanhurskasta tekoansa, jonka hän on tehnyt, muisteta. Uskottomuutensa tähden, johon hän on langennut, ja syntinsä tähden, jota on tehnyt, niiden tähden hänen on kuoltava.
\par 25 Ja vielä te sanotte: 'Herran tie ei ole oikea'. Kuulkaa siis, te Israelin heimo! Minunko tieni ei olisi oikea? Eikö niin: teidän omat tienne eivät ole oikeat!
\par 26 Jos vanhurskas kääntyy pois vanhurskaudestansa ja tekee vääryyttä, sentähden hänen on kuoltava; vääryytensä tähden, jota on tehnyt, hänen on kuoltava.
\par 27 Ja jos jumalaton kääntyy pois jumalattomuudestaan, jota on harjoittanut, ja tekee oikeuden ja vanhurskauden, hän säilyttää sielunsa elossa.
\par 28 Koska hän näki ja kääntyi pois kaikista synneistänsä, joita oli tehnyt, hän totisesti saa elää; ei hänen ole kuoltava.
\par 29 Mutta Israelin heimo sanoo: 'Herran tie ei ole oikea'. Minunko tieni eivät olisi oikeat, te Israelin heimo? Eikö niin: teidän omat tienne eivät ole oikeat!
\par 30 Niinpä minä tuomitsen teidät, te Israelin heimo, itsekunkin hänen teittensä mukaan, sanoo Herra, Herra. Kääntykää, palatkaa pois kaikista synneistänne, ja älköön syntivelka tulko teille lankeemukseksi.
\par 31 Heittäkää pois päältänne kaikki syntinne, joilla te olette rikkoneet, ja tehkää itsellenne uusi sydän ja uusi henki. Ja minkätähden te kuolisitte, Israelin heimo?
\par 32 Sillä ei ole minulle mieleen kuolevan kuolema, sanoo Herra, Herra. Siis kääntykää, niin te saatte elää."

\chapter{19}

\par 1 "Mutta sinä, viritä itkuvirsi Israelin ruhtinaista
\par 2 ja sano: Mikä naarasleijona olikaan sinun äitisi leijonain joukossa! Se makasi nuorten jalopeurain keskellä, kasvatti poikasiansa
\par 3 ja sai ylenemään yhden poikasistaan: siitä tuli nuori jalopeura, se oppi saalista raatelemaan, se söi ihmisiä.
\par 4 Mutta kansat kuulivat siitä: se pyydystettiin heidän kuoppaansa ja vietiin turpakoukussa Egyptin maahan.
\par 5 Kun emo näki, että viipyi, että hukkui hänen toivonsa, otti se toisen poikasistaan, sai sen nuoreksi jalopeuraksi.
\par 6 Se käyskenteli leijonain keskellä, siitä tuli nuori jalopeura, se oppi saalista raatelemaan, se söi ihmisiä.
\par 7 Se ryhtyi heidän leskiinsä ja teki autioiksi heidän kaupunkinsa, ja maa ja kaikki, mitä siinä on, kauhistui sen ärjynnän äänestä.
\par 8 Silloin kansat maakunnista yltympäri asettivat ja virittivät sille verkkonsa; se pyydystettiin heidän kuoppaansa.
\par 9 Ja se pantiin häkkiin, turpakoukkuun ja vietiin Baabelin kuninkaan eteen. Se vietiin vuorilinnoihin, ettei sen ääni enää kuuluisi Israelin vuorille.
\par 10 Sinun äitisi oli sinulle kuin verevä viinipuu, veden ääreen istutettu. Se tuli runsaasta vedestä hedelmöitseväksi ja tuuhealehväiseksi.
\par 11 Siihen tuli ylväitä oksia hallitsijain valtikoiksi, ja sen runko kohosi korkealle tiheän lehvistön keskellä ja näkyi kauas korkeana ja runsas-oksaisena.
\par 12 Mutta se temmattiin vihaisesti irti, viskattiin maahan, ja itätuuli kuivasi sen hedelmät, ne revittiin hajalleen, ja sen ylväät oksat kuivuivat, ne kulutti tuli.
\par 13 Nyt se on istutettuna erämaahan, kuivaan ja janoiseen maahan.
\par 14 Ja sen valtaoksasta on lähtenyt tuli, se on kuluttanut sen hedelmät, ja ylvästä oksaa hallitusvaltikaksi ei siinä enää ole." Itkuvirsi tämä oli oleva, ja itkuvirsi siitä on tullut.

\chapter{20}

\par 1 Seitsemäntenä vuotena, viidennessä kuussa, kuukauden kymmenentenä päivänä tuli miehiä Juudan vanhinten joukosta kysymään neuvoa Herralta, ja he istuivat minun eteeni.
\par 2 Niin minulle tuli tämä Herran sana:
\par 3 "Ihmislapsi, puhu Israelin vanhimmille ja sano heille: Näin sanoo Herra, Herra: Tekö tulette minulta neuvoa kysymään? Niin totta kuin minä elän, en anna minä teidän kysyä minulta neuvoa, sanoo Herra, Herra.
\par 4 Etkö tuomitse heitä, etkö tuomitse, ihmislapsi? Tee heille tiettäviksi heidän isiensä kauhistukset
\par 5 ja sano heille: Näin sanoo Herra, Herra: Sinä päivänä, jona minä Israelin valitsin, minä kättä kohottaen lupasin Jaakobin heimon jälkeläisille, minä tein itseni heille tunnetuksi Egyptin maassa ja kättä kohottaen lupasin heille sanoen: 'Minä olen Herra, teidän Jumalanne'.
\par 6 Sinä päivänä minä kättä kohottaen lupasin heille, että vien heidät pois Egyptin maasta siihen maahan, jonka olin heille katsonut ja joka vuotaa maitoa ja mettä - se on kaunistus kaikkien maitten joukossa. -
\par 7 Ja minä sanoin heille: 'Heittäkää, itsekukin, pois silmienne iljetykset älkääkä saastuttako itseänne Egyptin kivijumalilla: minä olen Herra, teidän Jumalanne'.
\par 8 Mutta he niskoittelivat minua vastaan eivätkä tahtoneet minua kuulla; eivät heittäneet pois itsekukin silmiensä iljetyksiä eivätkä hyljänneet Egyptin kivijumalia. Niin minä ajattelin vuodattaa kiivauteni heidän ylitsensä ja panna vihani täytäntöön heissä keskellä Egyptin maata.
\par 9 Mutta minä tein, minkä tein, oman nimeni tähden, ettei se tulisi häväistyksi pakanain silmissä, joitten keskellä he olivat ja joitten silmäin edessä minä olin tehnyt itseni heille tunnetuksi viemällä heidät pois Egyptin maasta.
\par 10 Ja kun olin vienyt heidät pois Egyptin maasta ja tuonut heidät erämaahan,
\par 11 niin minä annoin heille käskyni ja tein heille tiettäviksi oikeuteni: se ihminen, joka ne pitää, on niistä elävä.
\par 12 Myöskin sapattini minä annoin heille, olemaan merkkinä minun ja heidän välillään, että he tulisivat tietämään, että minä olen Herra, joka pyhitän heidät.
\par 13 Mutta Israelin heimo niskoitteli minua vastaan erämaassa: he eivät vaeltaneet minun käskyjeni mukaan, ylenkatsoivat minun oikeuteni, jotka ihmisen on pidettävä, että hän niistä eläisi, ja minun sapattini he kovin rikkoivat. Niin minä ajattelin vuodattaa kiivauteni heidän ylitsensä erämaassa ja lopettaa heidät.
\par 14 Mutta minä tein, minkä tein, oman nimeni tähden, ettei se tulisi häväistyksi pakanain silmissä, joitten silmäin edessä minä olin vienyt heidät pois.
\par 15 Kuitenkin minä kättä kohottaen vannoin heille erämaassa, etten heitä tuo siihen maahan, jonka olin antanut heille, joka vuotaa maitoa ja mettä - se on kaunistus kaikkien maitten joukossa -
\par 16 koska he pitivät minun oikeuteni halpoina, eivät vaeltaneet minun käskyjeni mukaan, vaan rikkoivat minun sapattini; sillä heidän sydämensä vaelsi heidän kivijumalainsa jäljessä.
\par 17 Mutta minä säälin heitä, niin etten heitä hävittänyt enkä tehnyt heistä loppua erämaassa.
\par 18 Sitten minä sanoin heidän lapsillensa erämaassa: Älkää vaeltako isäinne käskyjen mukaan, heidän oikeuksiansa älkää noudattako älkääkä saastuttako itseänne heidän kivijumalillansa.
\par 19 Minä olen Herra, teidän Jumalanne; minun käskyjeni mukaan vaeltakaa, minun oikeuksiani noudattakaa, ne pitäkää
\par 20 ja pyhittäkää minun sapattini; ne olkoot merkkinä välillämme, minun ja teidän, että tulisitte tietämään, että minä olen Herra, teidän Jumalanne.
\par 21 Mutta lapset niskoittelivat minua vastaan: he eivät vaeltaneet minun käskyjeni mukaan, eivät noudattaneet minun oikeuksiani, niin että olisivat ne pitäneet, - jotka ihmisen on pidettävä, että hän niistä eläisi - ja rikkoivat minun sapattini. Niin minä ajattelin vuodattaa kiivauteni heidän ylitsensä ja panna vihani heissä täytäntöön erämaassa.
\par 22 Mutta minä pidätin käteni ja tein, minkä tein, oman nimeni tähden, ettei se tulisi häväistyksi pakanain silmissä, joitten silmäin edessä minä olin vienyt heidät pois.
\par 23 Kuitenkin minä kättä kohottaen vannoin heille erämaassa, että minä hajotan heidät pakanain sekaan ja sirotan heidät muihin maihin,
\par 24 koska he eivät pitäneet minun käskyjäni, vaan ylenkatsoivat minun käskyni, rikkoivat minun sapattini ja heidän silmänsä pälyivät heidän isiensä kivijumalain perään.
\par 25 Niinpä minäkin annoin heille käskyjä, jotka eivät olleet hyviä, ja oikeuksia, joista he eivät voineet elää,
\par 26 ja annoin heidän saastua lahjoistansa, siitä, että polttivat uhrina kaiken, mikä avasi äidinkohdun, jotta saattaisin heidät kauhun valtaan ja he tulisivat tietämään, että minä olen Herra.
\par 27 Sentähden puhu Israelin heimolle, ihmislapsi, ja sano heille: Näin sanoo Herra, Herra: Vielä niinkin ovat isänne minua herjanneet, että ovat olleet uskottomat minua kohtaan.
\par 28 Kun minä toin heidät maahan, jonka olin kättä kohottaen luvannut heille antaa, niin missä vain he näkivät korkean kukkulan tai tuuhean puun, siinä he uhrasivat teurasuhrinsa ja antoivat vihastuttavat uhrilahjansa, siinä panivat esiin suloisesti tuoksuvat uhrinsa ja siinä vuodattivat juomauhrinsa.
\par 29 Niin minä sanoin heille: 'Mikä tämä uhrikukkula on, jolle te menette?' ja niin sai sellainen nimen uhrikukkula aina tähän päivään asti.
\par 30 Sentähden sano Israelin heimolle: Näin sanoo Herra, Herra: Ettekö te saastuta itseänne isienne tiellä? Ettekö kulje uskottomina heidän iljetystensä jäljessä?
\par 31 Ettekö ole saastuttaneet itseänne kaikilla kivijumalillanne aina tähän päivään asti, kun tuotte lahjojanne ja panette lapsenne käymään tulen läpi? Ja minäkö antaisin teidän kysyä minulta neuvoa, te Israelin heimo? Niin totta kuin minä elän, sanoo Herra, Herra, en anna minä teidän kysyä minulta neuvoa.
\par 32 Se, mikä on tullut teidän mieleenne, ei totisesti ole tapahtuva - se, mitä sanotte: 'Me tahdomme olla pakanain kaltaisia, muitten maitten sukukuntain kaltaisia, niin että palvelemme puuta ja kiveä'.
\par 33 Niin totta kuin minä elän, sanoo Herra, Herra: totisesti minä olen hallitseva teitä väkevällä kädellä, ojennetulla käsivarrella ja vuodatetulla vihalla.
\par 34 Ja minä vien teidät pois kansojen seasta ja kokoan teidät maista, joihin olitte hajotetut, väkevällä kädellä, ojennetulla käsivarrella ja vuodatetulla vihalla.
\par 35 Ja minä tuon teidät kansojen erämaahan, ja siellä minä käyn oikeutta teidän kanssanne kasvoista kasvoihin.
\par 36 Niinkuin minä kävin oikeutta isienne kanssa Egyptinmaan erämaassa, niin minä käyn oikeutta teidän kanssanne, sanoo Herra, Herra.
\par 37 Minä panen teidät kulkemaan sauvan alitse ja saatan teidät liiton siteeseen.
\par 38 Ja minä erotan teistä ne, jotka kapinoivat minua vastaan ja luopuvat minusta: muukalaisuutensa maasta minä vien heidät pois, mutta Israelin maahan he eivät tule; ja te tulette tietämään, että minä olen Herra.
\par 39 Mutta te, Israelin heimo! Näin sanoo Herra, Herra: Menkää vain ja palvelkaa itsekukin omia kivijumalianne. Mutta vastedes te totisesti kuulette minua ettekä enää häpäise minun pyhää nimeäni lahjoillanne ynnä kivijumalillanne.
\par 40 Sillä minun pyhällä vuorellani, Israelin korkealla vuorella, sanoo Herra, Herra, siellä he palvelevat minua, koko Israelin heimo, kaikki tyynni, mitä maassa on. Siellä minä heihin mielistyn, siellä minä halajan teidän antimianne, uutisverojanne, kaikkinaisia teidän pyhiä lahjojanne.
\par 41 Niinkuin suloisesti tuoksuvaan uhriin minä teihin mielistyn, kun minä vien teidät pois kansojen seasta ja kokoan teidät maista, joihin olette hajotetut, ja osoitan teissä pyhyyteni pakanain silmien edessä.
\par 42 Ja te tulette tietämään, että minä olen Herra, kun minä tuon teidät Israelin maahan, siihen maahan, jonka minä kättä kohottaen olin luvannut antaa teidän isillenne.
\par 43 Ja te muistatte siellä vaelluksenne ja kaikki tekonne, joilla olette itsenne saastuttaneet, ja teitä kyllästyttää oma itsenne kaikkien pahain töittenne tähden, mitä olette tehneet.
\par 44 Ja te tulette tietämään, että minä olen Herra, kun minä teen teille, minkä teen, oman nimeni tähden, en teidän pahan vaelluksenne enkä riettaiden tekojenne ansion mukaan, te Israelin heimo; sanoo Herra, Herra."
\par 45 Ja minulle tuli tämä Herran sana:
\par 46 "Ihmislapsi, käännä kasvosi etelää kohti, vuodata sanasi etelää vastaan ja ennusta kedon metsikköä vastaan, joka on Etelämaassa.
\par 47 Ja sano Etelämaan metsikölle: Kuule Herran sana. Näin sanoo Herra, Herra: Katso, minä sytytän sinut tuleen, ja se kuluttaa kaikki sinun tuoreet puusi ja kuivat puusi. Ei sammu leimuava liekki, ja siitä kärventyvät kaikki kasvot, etelästä pohjoiseen asti;
\par 48 ja kaikki liha on näkevä, että minä, Herra, olen sen sytyttänyt: se ei sammu."
\par 49 Niin minä sanoin: "Voi Herra, Herra! Ne sanovat minusta: 'Tämähän puhuu pelkkiä vertauksia'."

\chapter{21}

\par 1 Ja minulle tuli tämä Herran sana:
\par 2 "Ihmislapsi, käännä kasvosi Jerusalemia kohti, vuodata sanasi pyhäköitä vastaan ja ennusta Israelin maata vastaan ja sano Israelin maalle:
\par 3 Näin sanoo Herra: Katso, minä käyn sinun kimppuusi, vedän miekkani tupestaan ja hävitän sinusta vanhurskaan ja jumalattoman.
\par 4 Koska minä tahdon hävittää sinusta vanhurskaan ja jumalattoman, sentähden lähtee tupestaan minun miekkani kaiken lihan kimppuun, etelästä pohjoiseen asti.
\par 5 Ja kaikki liha tulee tietämään, että minä, Herra, olen vetänyt miekkani tupestaan; eikä se siihen enää palaja.
\par 6 Mutta sinä, ihmislapsi, huokaile! Lanteet murtuneina, katkerassa tuskassa huokaile heidän silmiensä edessä.
\par 7 Ja kun he sinulta kysyvät: 'Minkätähden sinä huokailet?' niin sano: Sanoman tähden, sillä se saapuu, ja kaikki sydämet raukeavat, kaikki kädet herpoavat, kaikkien henki tyrmistyy, kaikki polvet käyvät veltoiksi kuin vesi. Katso, se tulee, se tapahtuu; sanoo Herra, Herra."
\par 8 Ja minulle tuli tämä Herran sana:
\par 9 "Ihmislapsi, ennusta ja sano: Näin sanoo Herra. Sano: Miekka, miekka on teroitettu, on myös kirkkaaksi hiottu.
\par 10 Teurasta teurastamaan se on teroitettu, salamoitsevaksi se on hiottu. Vai iloitsisimmeko? Vitsa, joka lyö minun poikaani, pitää halpana kaiken, mikä puuta on.
\par 11 Ja hän on antanut sen hioa otettavaksi käteen; se miekka on teroitettu, on hiottu annettavaksi surmaajan käteen.
\par 12 Huuda ja valita, ihmislapsi, sillä tämä tapahtuu minun kansaani vastaan, kaikkia Israelin ruhtinaita vastaan: he ovat annetut alttiiksi miekalle, he ynnä minun kansani; lyö sentähden lanteeseesi.
\par 13 Sillä vitsaa on jo koeteltu. Mutta entäpä, jos vitsa pitääkin halpana tämän kaiken? Se ei tapahdu, sanoo Herra, Herra.
\par 14 Mutta sinä, ihmislapsi, ennusta ja lyö käsiäsi yhteen; tulkoon miekasta kaksi miekkaa, tulkoon kolme; se on surmattujen surmamiekka, miekka, joka surmaa suuretkin, joka kiertää heitä yltympäri,
\par 15 että sydän pelkoon menehtyisi, että kompastuksia olisi paljon. Kaikille heidän porteillensa minä olen asettanut miekan välkkymään. Voi, se on tehty salamaksi, on hiottu teurastukseen!
\par 16 Lyö terävästi oikealle, ojennu vasemmalle, mihin vain teräsi suunnataan.
\par 17 Minäkin lyön käsiäni yhteen ja tyydytän kiivauteni. Minä, Herra, olen puhunut."
\par 18 Ja minulle tuli tämä Herran sana:
\par 19 "Sinä, ihmislapsi, laita kaksi tietä, Baabelin kuninkaan miekan tulla. Samasta maasta lähtekööt molemmat. Ja veistä viitta; veistä se kaupunkiin vievän tien suuhun.
\par 20 Laita tie miekan tulla ammonilaisten Rabbaan sekä Juudaan, varustettuun Jerusalemiin.
\par 21 Sillä Baabelin kuningas seisahtuu tien haaraan, molempain teitten suuhun, taikojansa taikomaan: hän pudistaa nuolia, kysyy kotijumalilta, tarkkaa maksaa.
\par 22 Hänen oikeaan käteensä tulee Jerusalemin arpa: panna muurinmurtajia, avata suu ärjyntään, kohottaa sotahuuto, panna muurinmurtajia portteja vastaan, luoda valli, rakentaa saartovarusteet.
\par 23 Mutta Jerusalemin asukkaiden silmissä tämä on pettävää taikomista: heillähän on valojen valat; mutta hän saattaa muistoon syntivelan, että he joutuisivat kiinni.
\par 24 Sentähden, näin sanoo Herra, Herra: Koska te olette saattaneet muistoon syntivelkanne, kun rikkomuksenne ovat tulleet julki, niin että teidän syntinne on näkyvissä kaikissa teidän teoissanne, niin - koska te olette saatetut muistoon - teidät otetaan käsin kiinni.
\par 25 Ja sinä saastutettu, jumalaton, sinä Israelin ruhtinas, jonka päivä on silloin tullut, kun syntivelka on loppumäärässään!
\par 26 Näin sanoo Herra, Herra: Ota pois käärelakki, nosta pois kruunu.
\par 27 Tämä ei jää tällensä: alhainen korotetaan, korkea alennetaan. Raunioiksi, raunioiksi, raunioiksi minä panen tämän; eikä tästä ole jäävä mitään, kunnes tulee hän, jolla on oikeus, ja minä annan sen hänelle.
\par 28 Ja sinä, ihmislapsi, ennusta ja sano: Näin sanoo Herra, Herra ammonilaisia ja heidän herjauksiansa vastaan: Sano näin: Miekka, miekka on paljastettu, teurastukseen hiottu, saamaan kyllänsä, salamoitsemaan,
\par 29 vaikka sinulle petollisia näkyjä nähdään, valheita ennustellaan, että sinut muka pannaan tallaamaan saastaisten, jumalattomain niskaa, joitten päivä on silloin tullut, kun syntivelka on loppumäärässään.
\par 30 Pane se tuppeensa takaisin. Paikassa, jossa olet luotu, synnyinmaassasi, minä tahdon sinut tuomita.
\par 31 Minä vuodatan sinun ylitsesi kiivastukseni, puhallan sinuun vihani tulen ja annan sinut raakain miesten käsiin, turmiontekijäin.
\par 32 Tulen kulutettavaksi sinut annetaan, sinun veresi on oleva keskellä maata, ei sinua enää muisteta. Sillä minä, Herra, olen puhunut."

\chapter{22}

\par 1 Ja minulle tuli tämä Herran sana:
\par 2 "Sinä, ihmislapsi, etkö tuomitse, etkö tuomitse verivelkojen kaupunkia? Tee sille tiettäviksi kaikki sen kauhistukset
\par 3 ja sano: Näin sanoo Herra, Herra: Voi kaupunkia, joka on vuodattanut keskuudessaan verta, että tulisi sen aika, ja tehnyt itselleen kivijumalia, että se siitä saastuisi!
\par 4 Verestäsi, jonka olet vuodattanut, sinä olet tullut vikapääksi, ja kivijumalistasi, jotka olet tehnyt, sinä olet saastunut; sinä olet jouduttanut päiviäsi ja olet vuottesi määrään päässyt. Sentähden minä annan sinut pakanain häväistäväksi ja kaikkien maitten pilkaksi.
\par 5 Läheiset ja kaukaiset pilkkaavat sinua, jonka nimi on saastutettu ja jossa on hämminkiä paljon.
\par 6 Katso, Israelin ruhtinaat sinussa luottavat kukin omaan käsivarteensa vuodattaaksensa verta.
\par 7 Isää ja äitiä sinussa ylenkatsotaan, muukalaiselle sinun keskelläsi tehdään väkivaltaa, orpoa ja leskeä sinussa sorretaan.
\par 8 Mikä on minulle pyhitettyä, sitä sinä halveksit, minun sapattini sinä rikot.
\par 9 Sinussa on niitä, jotka panettelevat vuodattaaksensa verta. Uhrivuorilla sinussa syödään. Ilkitöitä sinun keskelläsi tehdään.
\par 10 Isän häpy sinussa paljastetaan. Naiselle, joka on saastainen kuukautistilansa tähden, tehdään sinussa väkivaltaa.
\par 11 Toinen harjoittaa kauhistusta toisensa vaimon kanssa, mies saastuttaa sukurutsauksessa miniänsä, mies tekee sinussa väkivaltaa sisarelleen, isänsä tyttärelle.
\par 12 Sinussa otetaan lahjuksia veren vuodattamiseksi, sinä otat korkoa ja voittoa, kiskot väkivaltaisesti lähimmäistäsi, mutta minut sinä unhotat, sanoo Herra, Herra.
\par 13 Mutta katso, minä lyön käsiäni yhteen sinun väärän voittosi tähden, jota olet hankkinut, ja sinun verivelkojesi tähden, joita keskuudessasi on.
\par 14 Kestääköhän rohkeutesi, pysyvätköhän kätesi lujina niinä päivinä, joina minä sinulle teen, minkä teen? Minä, Herra, olen puhunut, ja minä teen sen.
\par 15 Minä hajotan sinut pakanain sekaan ja sirotan sinut muihin maihin ja poistan sinusta saastaisuutesi.
\par 16 Sinä tulit saastaiseksi oman itsesi tähden kansojen silmissä, mutta sinä tulet tietämään, että minä olen Herra."
\par 17 Ja minulle tuli tämä Herran sana:
\par 18 "Ihmislapsi, Israelin heimo on kuonaksi tullut. He ovat kaikki tyynni vaskea, tinaa, rautaa ja lyijyä ahjossa; he ovat tulleet hopean kuonaksi.
\par 19 Sentähden, näin sanoo Herra, Herra: Koska te kaikki tyynni olette tulleet kuonaksi, sentähden, katso, minä kokoan teidät keskelle Jerusalemia.
\par 20 Niinkuin hopea, vaski, rauta, lyijy ja tina kootaan keskelle ahjoa, että niihin lietsottaisiin tulta ja ne sulatettaisiin, niin minä vihassani ja kiivaudessani teidät kokoan ja asetan ahjoon ja sulatan.
\par 21 Minä kerään teidät ja lietson teihin vihani tulta, ja te sulatte keskellä Jerusalemia.
\par 22 Niinkuin hopea sulatetaan keskellä ahjoa, niin sulatetaan teidät sen keskellä. Ja te tulette tietämään, että minä, Herra, olen vuodattanut kiivauteni teidän ylitsenne."
\par 23 Ja minulle tuli tämä Herran sana:
\par 24 "Ihmislapsi, sano sille: Sinä olet maa, jota ei ole puhdistettu, joka ei ole saanut sadetta vihan päivänä.
\par 25 Profeettain salaliitto sen keskellä on niinkuin ärjyvä, saalista raateleva leijona: he syövät sieluja, ottavat aarteet ja kalleudet ja lisäävät sen keskuudessa sen leskien lukua.
\par 26 Sen papit tekevät väkivaltaa minun lailleni ja häpäisevät sitä, mikä on minulle pyhitetty, eivät tee erotusta pyhän ja epäpyhän välillä, eivät tee tiettäväksi, mikä on saastaista, mikä puhdasta, ja sulkevat silmänsä minun sapateiltani, niin että minä tulen häväistyksi heidän keskellänsä.
\par 27 Sen päämiehet siellä ovat niinkuin saalista raatelevaiset sudet: he vuodattavat verta, hukuttavat sieluja kiskoaksensa väärää voittoa.
\par 28 Sen profeetat valkaisevat heille kaiken kalkilla, kun näkevät petollisia näkyjä ja ennustelevat heille valheita sanoen: 'Näin sanoo Herra, Herra', vaikka Herra ei ole puhunut.
\par 29 Maan kansa harjoittaa väkivaltaa, riistää ja raastaa: kurjaa ja köyhää he sortavat, muukalaiselle tekevät väkivaltaa oikeudesta välittämättä.
\par 30 Minä etsin heidän joukostansa miestä, joka korjaisi muurin ja seisoisi muurinaukossa minun edessäni maan puolesta, etten minä sitä hävittäisi, mutta en löytänyt.
\par 31 Sentähden minä vuodatan heidän ylitsensä kiivauteni, hukutan heidät vihani tulella ja annan heidän vaelluksensa tulla heidän päänsä päälle, sanoo Herra, Herra."

\chapter{23}

\par 1 Ja minulle tuli tämä Herran sana:
\par 2 "Ihmislapsi, oli kaksi naista, saman äidin tyttäriä.
\par 3 He harjoittivat haureutta Egyptissä; nuoruudessaan he haureutta harjoittivat. Siellä heidän rintojansa likisteltiin ja heidän neitsyellisiä nisiänsä puristeltiin.
\par 4 Heidän nimensä olivat: vanhemman Ohola ja hänen sisarensa Oholiba. Sitten he tulivat minun omikseni ja synnyttivät poikia ja tyttäriä. - Heidän nimensä: Samaria on Ohola ja Jerusalem Oholiba.
\par 5 Mutta Ohola harjoitti haureutta, vaikka oli minun, ja himoitsi rakastajiansa, assurilaisia, naapureita,
\par 6 jotka olivat punasiniseen purppuraan puettuja ja olivat käskynhaltijoita ja päämiehiä, komeita nuorukaisia kaikki, ratsumiehiä, hevosen selässä ajajia.
\par 7 Ja hän antautui haureuteen näiden kanssa, jotka olivat Assurin valiopoikia kaikki; ja keitä vain hän himoitsi, niiden kaikkien kanssa hän saastutti itsensä kaikilla heidän kivijumalillaan.
\par 8 Mutta Egyptin-aikaista haureuttansa hän ei jättänyt, sillä he olivat maanneet hänen kanssansa hänen nuoruudessaan, olivat puristelleet hänen neitsyellisiä nisiänsä ja vuodattaneet hänen ylitsensä haureuttaan.
\par 9 Sentähden minä annoin hänet rakastajainsa käsiin, Assurin poikain käsiin, joita hän oli himoinnut.
\par 10 He paljastivat hänen häpynsä ja ottivat hänen poikansa ja tyttärensä ja tappoivat hänet itsensä miekalla, niin että hänestä tuli varoittava esimerkki muille naisille, ja panivat hänessä toimeen tuomiot.
\par 11 Mutta hänen sisarensa Oholiba, vaikka näki tämän, oli himossaan vielä häntäkin riettaampi ja oli haureudessaan vielä riettaampi, kuin hänen sisarensa oli ollut haureudessaan.
\par 12 Hän himoitsi Assurin poikia, jotka olivat käskynhaltijoita, päämiehiä, naapureita, pulskasti puettuja, ratsumiehiä, hevosen selässä ajajia, komeita nuorukaisia kaikki.
\par 13 Silloin minä näin, että hän saastutti itsensä: sama oli tie kumpaisellakin.
\par 14 Mutta hän meni vielä pitemmälle haureudessaan; kun hän näki seinään piirrettyjä miehiä, kaldealaisten kuvia, punavärillä piirrettyjä,
\par 15 vyö vyötettynä kupeille, päässä muhkea päähine, sankarien näköisiä kaikki, Baabelin poikain muotoisia, joiden synnyinmaa on Kaldea,
\par 16 niin hän silmän näkemältä sai himon heihin ja lähetti lähettiläitä heidän luoksensa Kaldeaan.
\par 17 Niin Baabelin pojat tulivat ja makasivat hänen kanssaan hekumassa ja saastuttivat hänet haureudellaan, niin että hän saastui heistä. Sitten hänen sielunsa vieraantui heistä.
\par 18 Mutta kun hän paljasti haureutensa ja paljasti häpynsä, niin minun sieluni vieraantui hänestä, niinkuin minun sieluni oli vieraantunut hänen sisarestansa.
\par 19 Mutta hän yhä enensi haureuttaan, kun muisti nuoruutensa päivät, jolloin oli haureutta harjoittanut Egyptin maassa.
\par 20 Ja hän sai himon heidän hekumoitsijoihinsa, joilla on jäsen kuin aaseilla ja vuoto kuin orheilla.
\par 21 Ja sinä etsit nuoruutesi iljettävyyttä, jolloin egyptiläiset puristelivat nisiäsi nuorekkaiden rintojesi tähden.
\par 22 Sentähden, Oholiba, näin sanoo Herra, Herra: Katso, minä nostatan sinun kimppuusi rakastajasi, joista sielusi on vieraantunut, ja tuon heidät sinun kimppuusi joka taholta:
\par 23 Baabelin pojat ja kaikki kaldealaiset, Pekodin, Sooan ja Kooan ja heidän kanssaan kaikki Assurin pojat - komeita nuorukaisia, käskynhaltijoita ja päämiehiä kaikki, sankareita ja mainioita miehiä, ratsumiehiä, hevosten selässä ajajia kaikki.
\par 24 He tulevat sinun kimppuusi, mukanaan paljon vaunuja ja rattaita ja suuret joukot kansoja; suurine ja pienine kilpineen ja kypäreineen he käyvät sinua vastaan joka taholta. Minä annan heille tuomiovallan, ja he tuomitsevat sinut oikeuksiensa mukaan.
\par 25 Minä osoitan sinussa kiivauteni, niin että he kohtelevat sinua tuimasti: he leikkaavat sinulta pois nenän ja korvat, ja sinun jälkeläisesi kaatuvat miekkaan. He ottavat sinun poikasi ja tyttäresi, ja sinun jälkeläisesi kuluttaa tuli.
\par 26 He raastavat sinulta vaatteesi ja ottavat korukalusi.
\par 27 Niin minä teen lopun sinun iljettävyydestäsi, jo Egyptin-aikaisesta haureudestasi; etkä sinä sitten luo silmiäsi heihin etkä enää muistele Egyptiä.
\par 28 Sillä näin sanoo Herra, Herra: Katso, minä annan sinut niitten käsiin, joita vihaat, niitten käsiin, joista sielusi on vieraantunut.
\par 29 Ja he kohtelevat sinua vihamielisesti, ottavat kaiken vaivannäkösi ja jättävät sinut alastomaksi ja paljaaksi; ja sinun haureellinen häpysi ja iljettävyytesi ja haureutesi paljastuu.
\par 30 Näin sinulle tehdään sentähden, että uskottomana juoksit pakanain perässä, että saastutit itsesi heidän kivijumalillansa.
\par 31 Sinä vaelsit sisaresi tietä, ja minä annan hänen maljansa sinun käteesi.
\par 32 Näin sanoo Herra, Herra: Sisaresi maljan sinä juot, syvän ja laajan ja paljon vetävän. Sinä tulet nauruksi ja pilkaksi.
\par 33 Tulet juopumusta ja murhetta täyteen: kauhun ja häviön malja on sinun sisaresi Samarian malja.
\par 34 Sen sinä juot ja särvit, sen sirpaleita sinä kaluat ja niillä rintasi revit. Sillä minä olen puhunut, sanoo Herra, Herra.
\par 35 Sentähden, näin sanoo Herra, Herra: Koska sinä olet minut unhottanut, minut selkäsi taakse heittänyt, niin kanna myös iljettävyytesi ja haureutesi."
\par 36 Ja Herra sanoi minulle: "Ihmislapsi, etkö tuomitse Oholaa ja Oholibaa? Ilmoita heille heidän kauhistuksensa.
\par 37 Sillä he ovat rikkoneet avion, ja heidän käsissään on verta; kivijumalainsa kanssa he ovat avion rikkoneet, ja myös lapsensa, jotka he olivat minulle synnyttäneet, he ovat polttaneet, ruuaksi niille.
\par 38 Vielä tämänkin he ovat minulle tehneet: ovat samana päivänä saastuttaneet minun pyhäkköni ja rikkoneet minun sapattini.
\par 39 Teurastettuaan lapsiansa kivijumalilleen he samana päivänä tulivat minun pyhäkkööni ja niin häpäisivät sen. Katso, näin he ovat tehneet keskellä minun huonettani.
\par 40 Vieläpä he lähettivät sanan miehille, jotka tulivat kaukaa: katso, ne tulivat siinä samassa, kun sana oli lähetetty, nuo, joita varten sinä peseydyit ja maalasit silmäluomesi ja panit päällesi korut.
\par 41 Ja sinä istuit komealle vuoteelle, sen ääressä oli katettu pöytä, ja sille sinä panit minun suitsukkeeni ja öljyni.
\par 42 Sitten sieltä kuului suruton melu. Ihmisjoukosta otettujen miesten lisäksi tuotiin erämaasta juomareita, ja ne panivat noiden naisten käsiin rannerenkaat ja heidän päähänsä kauniit kruunut.
\par 43 Mutta minä sanoin: 'Tuollekin kuihtuneelle kelpaa aviorikos! Hänen kanssaan nyt harjoitetaan haureutta - ja vielä hänkin!'
\par 44 Hänen luoksensa tultiin, aivan kuin tullaan porttonaisen luo: niin tultiin Oholan ja Oholiban luokse - noiden iljettävien naisten.
\par 45 Mutta vanhurskaat miehet tulevat tuomitsemaan heidät sen mukaan, mitä on säädetty avionrikkoja-naisista ja mitä on säädetty verenvuodattaja-naisista; sillä he ovat rikkoneet avion, ja heidän käsissään on verta.
\par 46 Sillä näin sanoo Herra, Herra: Nostatettakoon kansanjoukko heitä vastaan, ja heidät pantakoon kauhistukseksi ja ryöstettäköön.
\par 47 Ja kansanjoukko kivittäköön heidät ja hakatkoon kappaleiksi miekoillansa; tappakoot ne heidän poikansa ja tyttärensä ja polttakoot heidän talonsa tulella.
\par 48 Niin minä lopetan iljettävän menon maasta. Ja kaikki naiset ottavat siitä ojentuakseen, niin etteivät harjoita iljettävää menoa, niinkuin te.
\par 49 Teidän iljettävyytenne pannaan teidän päällenne, ja te saatte kantaa kivijumalainne kanssa tehdyt syntinne. Ja te tulette tietämään, että minä olen Herra, Herra."

\chapter{24}

\par 1 Tämä Herran sana tuli minulle yhdeksäntenä vuotena, kymmenennessä kuussa, kuukauden kymmenentenä päivänä:
\par 2 "Ihmislapsi, kirjoita muistiisi päivän nimi - juuri tämän päivän nimi: Baabelin kuningas rynnistää Jerusalemia vastaan juuri tänä päivänä.
\par 3 Lausu uppiniskaiselle suvulle vertaus ja sano heille: Näin sanoo Herra, Herra: Pane pata liedelle, pane. Kaada myös siihen vettä.
\par 4 Kokoile siihen lihakappaleet - kaikki hyvät kappaleet, reittä ja lapaa - ja täytä se valituilla luilla.
\par 5 Ota valiolampaita, lado myös halkoja sen alle. Anna sen kiehumistaan kiehua, niin että siinä luutkin tulevat keitetyiksi.
\par 6 Sentähden, näin sanoo Herra, Herra: Voi verivelkojen kaupunkia, pataa, joka on ruostunut ja josta ei lähde sen ruoste! Tyhjennä se kappale kappaleelta: sille ei ole langennut arpaosaa.
\par 7 Sillä sen vuodattama veri on sen keskellä, paljaalle kalliolle se on sen koonnut, ei ole vuodattanut sitä maahan, tomun peitettäväksi.
\par 8 Nostattaakseni kiivauden, tuottaakseni koston minä olen pannut sen veren paljaalle kalliolle, ettei se peittyisi.
\par 9 Sentähden, näin sanoo Herra, Herra: Voi verivelkojen kaupunkia! Minäkin suurennan liettä.
\par 10 Lisää halkoja, viritä tulta, keitä liha loppuun, kiehuta liemi kuiviin, polta luut karreksi.
\par 11 Anna sen sitten olla tyhjänä hiiltensä päällä, niin että se kuumenee, sen vaski hehkuu ja sen saastat siitä sulavat, sen ruosteesta tulee loppu.
\par 12 Siitä on ollut vaivaa väsymykseen asti, mutta ei ole lähtenyt siitä sen ruosteen paljous. Tuleen sen ruoste!
\par 13 Sinun iljettävän saastaisuutesi takia - koska minä tahdoin puhdistaa sinut, mutta sinä et puhdistunut saastaisuudestasi - sinä et enää puhdistu, ennenkuin minä olen tyydyttänyt kiivauteni sinussa.
\par 14 Minä, Herra, olen puhunut. Se tapahtuu, ja minä teen sen. Minä en hellitä, en säästä enkä kadu: teittesi ja tekojesi mukaan sinut tuomitaan, sanoo Herra, Herra."
\par 15 Ja minulle tuli tämä Herran sana:
\par 16 "Ihmislapsi, katso, minä otan sinulta äkkikuolemalla pois silmiesi ihastuksen, mutta älä valita, älä itke, älköönkä tulko sinulta kyyneltä.
\par 17 Huokaile hiljaa, mutta älä pane toimeen kuolleenvalittajaisia. Sido juhlapäähine päähäsi, pane kengät jalkaasi, älä peitä partaasi äläkä syö suruleipää."
\par 18 Ja minä puhuin kansalle aamulla, mutta illalla kuoli minun vaimoni; sitten aamulla minä tein, niinkuin minun oli käsketty tehdä.
\par 19 Niin kansa sanoi minulle: "Etkö meille ilmoita, mitä se tietää meille, kun sinä noin teet?"
\par 20 Minä sanoin heille: "Minulle tuli tämä Herran sana: Sano Israelin heimolle:
\par 21 Näin sanoo Herra, Herra: Katso, minä häpäisen pyhäkköni, joka on teidän varustuksenne ja ylpeytenne, teidän silmienne ihastus ja sielujenne ikävä; ja teidän poikanne ja tyttärenne, jotka teidän on ollut jätettävä, kaatuvat miekkaan.
\par 22 Sitten te teette, niinkuin minä olen tehnyt: ette peitä partaanne ettekä syö suruleipää,
\par 23 juhlapäähineen pidätte päässänne ja kengät jalassanne, ette valita ettekä itke, mutta te riudutte syntivelkanne tähden ja huokailette toinen toisellenne.
\par 24 Hesekiel on oleva teille ennusmerkki: aivan niin, kuin hän on tehnyt, niin tekin teette. Kun tämä tapahtuu, te tulette tietämään, että minä olen Herra, Herra.
\par 25 Ja sinä, ihmislapsi! Sinä päivänä, jona minä otan heiltä heidän varustuksensa, ihanan ilonsa, silmiensä ihastuksen, sielujensa halajamisen, heidän poikansa ja tyttärensä,
\par 26 sinä päivänä tulee pakolainen sinun luoksesi ilmoittamaan tätä korvaisi kuullen.
\par 27 Sinä päivänä avautuu sinun suusi yhtaikaa kuin pakolaisen, ja sinä puhut etkä enää ole mykkänä. Sinä olet oleva heille ennusmerkki, ja he tulevat tietämään, että minä olen Herra."

\chapter{25}

\par 1 Minulle tuli tämä Herran sana:
\par 2 "Ihmislapsi, käännä kasvosi ammonilaisia kohti ja ennusta heitä vastaan
\par 3 ja sano ammonilaisille: Kuulkaa Herran, Herran sana: Näin sanoo Herra, Herra: Koska sinä sanoit: 'Kas niin!' minun pyhäköstäni, kun se häväistiin, ja Israelin maasta, kun se hävitettiin, ja Juudan heimosta, kun he menivät pakkosiirtolaisuuteen,
\par 4 sentähden, katso, minä annan sinut Idän miehille omaksi, ja he laittavat sinuun leiripaikkansa ja asettavat sinuun asuntonsa. He syövät sinun hedelmäsi ja juovat maitosi.
\par 5 Ja Rabban minä panen kamelien laitumeksi ja ammonilaisten maan lammaslaumojen leposijaksi. Ja te tulette tietämään, että minä olen Herra.
\par 6 Sillä näin sanoo Herra, Herra: Koska sinä käsiä taputit ja jalkaa poljit ja, sielu täynnä ylenkatsetta, pidit iloa Israelin maasta,
\par 7 sentähden, katso, minä ojennan käteni sinua vastaan ja annan sinut pakanain ryöstettäväksi ja hävitän sinut kansojen joukosta ja hukutan sinut maitten luvusta ja tuhoan sinut. Ja sinä tulet tietämään, että minä olen Herra.
\par 8 Näin sanoo Herra, Herra: Koska Mooab ja Seir ovat sanoneet: 'Katso, Juudan heimon käy samoin kuin kaikkien muitten kansain',
\par 9 sentähden, katso, minä teen avonaiseksi Mooabin vuoriselänteen, paljaaksi kaupungeista - kaupungeista sen rajasta toiseen - jotka ovat maan kaunistus: Beet-Jesimot, Baal-Meon ja Kirjataim.
\par 10 Idän miehille omaksi minä annan sen ynnä ammonilaiset, niin ettei ammonilaisia enää muisteta kansojen seassa,
\par 11 ja Mooabissa minä panen toimeen tuomiot. Ja he tulevat tietämään, että minä olen Herra.
\par 12 Näin sanoo Herra, Herra: Koska Edom menetteli kovin kostonhimoisesti Juudan heimoa kohtaan ja tuli suuresti syynalaiseksi, kostaessansa heille,
\par 13 sentähden, näin sanoo Herra, Herra: Minä ojennan käteni Edomia vastaan ja hävitän siitä ihmiset ja eläimet. Teemanista lähtien minä panen sen raunioiksi, ja Dedania myöten he kaatuvat miekkaan.
\par 14 Ja kostoni, joka Edomia kohtaa, minä annan kansani Israelin käteen, ja he tekevät Edomia kohtaan minun vihani ja kiivauteni mukaan. Ja Edom saa tuntea minun kostoni, sanoo Herra, Herra.
\par 15 Näin sanoo Herra, Herra: Koska filistealaiset menettelivät kostonhimoisesti ja, ylenkatse sielussaan, kostivat kovasti, ikivihassa tuottaaksensa tuhon,
\par 16 sentähden, näin sanoo Herra, Herra: Katso, minä ojennan käteni filistealaisia vastaan, hävitän kreetit, hukutan, mitä on jäljellä meren rannikolla,
\par 17 ja teen heille suuret kostoteot, rangaisten kiivaudessa. Ja he tulevat tietämään, että minä olen Herra, kun minä annan kostoni heitä kohdata."

\chapter{26}

\par 1 Yhdentenätoista vuotena, kuukauden ensimmäisenä päivänä, tuli minulle tämä Herran sana:
\par 2 "Ihmislapsi, koska Tyyro sanoi Jerusalemista: 'Kas niin! Murrettu on kansojen ovi, minuun päin se on kääntynyt, minä tulen täyteen - se on rauniona!'
\par 3 sentähden, näin sanoo Herra, Herra: Katso, minä käyn sinun kimppuusi, Tyyro, ja nostatan monet kansat sinua vastaan, niinkuin meri nostaa aaltonsa.
\par 4 Ne hävittävät Tyyron muurit, repivät maahan sen tornit, ja minä lakaisen siitä pois sen tomutkin ja panen sen paljaaksi kallioksi.
\par 5 Siitä tulee verkkojen kuivauspaikka keskelle merta. Sillä minä olen puhunut, sanoo Herra, Herra; ja se joutuu kansojen ryöstettäväksi.
\par 6 Sen tytärkaupungit, jotka ovat mantereella, surmataan miekalla, ja he tulevat tietämään, että minä olen Herra.
\par 7 Sillä näin sanoo Herra, Herra: Katso, minä tuon Tyyron kimppuun Nebukadressarin, Baabelin kuninkaan pohjoisesta, kuninkaitten kuninkaan, hevosten ja vaunujen ja ratsumiesten ja suuren väenpaljouden kanssa.
\par 8 Sinun tytärkaupunkisi, jotka ovat mantereella, hän surmaa miekalla ja asettaa sinua vastaan saartovarusteet, luo sinua vastaan vallin, nostaa sinua vastaan kilpikatoksen,
\par 9 suuntaa sinun muureihisi murtajansa iskut ja kukistaa tornisi rauta-aseillansa.
\par 10 Hänen hevostensa paljous peittää sinut pölyyn. Ratsumiesten, pyöräin ja vaunujen ryskeestä sinun muurisi vapisevat, kun hän hyökkää sisälle sinun porteistasi, niinkuin valloitettuun kaupunkiin hyökätään.
\par 11 Hevostensa kavioilla hän tallaa rikki sinun katusi kaikki, surmaa miekalla sinun kansasi, ja sinun mahtavat patsaasi kaatuvat maahan.
\par 12 He riistävät sinun rikkautesi ja ryöstävät kauppatavarasi, repivät muurisi maahan ja kukistavat kauniit talosi, ja kivesi, puusi, tomusi he heittävät meren syvyyteen.
\par 13 Minä lakkautan laulujesi helinän, eikä kuulu enää kanneltesi soitto.
\par 14 Minä panen sinut paljaaksi kallioksi, sinusta tulee verkkojen kuivauspaikka, eikä sinua enää rakenneta. Sillä minä, Herra, olen puhunut, sanoo Herra, Herra.
\par 15 Näin sanoo Herra, Herra Tyyrolle: Eivätkö sinun kukistumisesi pauhusta, kun haavoitetut voihkivat, kun surman omat surmataan sinun keskelläsi, saaret vapise?
\par 16 Valtaistuimiltaan astuvat alas kaikki meren ruhtinaat. He heittävät pois viittansa ja riisuvat kirjaellut vaatteensa, he pukeutuvat kauhuun ja istuvat maahan, värisevät joka hetki, tyrmistyneinä sinun tähtesi.
\par 17 He virittävät sinusta itkuvirren ja sanovat sinulle: 'Kuinka olet sinä, joka olit asuttu, hävinnyt meriltä, sinä ylistetty kaupunki, väkevä merellä, sinä ja sinun asukkaasi, jotka levittivät kauhuansa kaikkiin siellä asuvaisiin.
\par 18 Nyt värisevät saaret sinun kukistumisesi päivänä, merensaaret kauhistuvat sinun loppuasi.'
\par 19 Sillä näin sanoo Herra, Herra: Kun minä teen sinusta aution kaupungin, niinkuin asumattomat kaupungit ovat, kun minä annan syvyyden käydä sinun ylitsesi ja paljot vedet peittävät sinut,
\par 20 silloin minä syöksen sinut alas hautaanvaipuneitten pariin, ikiaikojen kansan tykö, ja annan sinulle asunnon maan syvyyksissä, niinkuin siellä ovat ikiaikojen rauniot, hautaanvaipuneitten parissa, ettei sinussa asuttaisi; mutta ihanuuden minä annan elävien maahan.
\par 21 Sinut minä panen kauhuksi, sinua ei sitten enää ole; ja jos sinua etsitään, ei sinua enää löydy, hamaan ikiaikoihin asti, sanoo Herra, Herra."

\chapter{27}

\par 1 Ja minulle tuli tämä Herran sana:
\par 2 "Sinä, ihmislapsi! Viritä itkuvirsi Tyyrosta
\par 3 ja sano Tyyrolle, joka asuu meren porteilla ja käy kauppaa kansojen kanssa, monilukuisilla saarilla: Näin sanoo Herra, Herra: Tyyro, sinä sanot: 'Minä olen kauneuden täydellisyys!'
\par 4 Meren sydämessä on sinun alueesi, rakentajasi tekivät sinun kauneutesi täydelliseksi,
\par 5 Senirin kypresseistä he rakensivat sinuun kaiken, mikä laudoista on, ottivat Libanonin setrin tehdäkseen sinuun maston,
\par 6 tekivät Baasanin tammista airosi, kantesi tekivät norsunluusta ja Kittiläis-saarten hopeakuusesta.
\par 7 Kirjaeltua Egyptin pellavaa oli purjeesi, ja se oli sinulla lippuna. Katoksesi oli punasinistä ja purppuranpunaista Elisa-saarten kangasta.
\par 8 Siidonin ja Arvadin asukkaat olivat sinulla soutajina; omat viisaasi, Tyyro, olivat sinussa - he olivat laivureinasi.
\par 9 Gebalin vanhimmat ja viisaat olivat sinussa vuotokohtain korjaajina. Meren kaikki laivat merimiehineen olivat sinussa tavaroitasi vaihtamassa.
\par 10 Persia ja Luud ja Puut olivat sotajoukossasi, sinun sotureinasi. He ripustivat sinuun kypäreitä ja kilpiä; ne sinua koristivat.
\par 11 Arvadilaiset ja muu sotajoukkosi olivat muureillasi yltympäri ja gammadilaiset torneissasi. He ripustivat varustuksensa sinun muureillesi yltympäri; ne tekivät sinun kauneutesi täydelliseksi.
\par 12 Tarsis oli kauppatuttusi sinun kaikkinaisten rikkauksiesi runsauden takia: hopealla, raudalla, tinalla ja lyijyllä he maksoivat sinun tavarasi.
\par 13 Jaavan, Tuubal ja Mesek kävivät kauppaa sinun kanssasi: orjilla ja vaskikaluilla he maksoivat sinun vaihtotavarasi.
\par 14 Toogarman heimo maksoi sinun tavarasi hevosilla, sotaratsuilla ja muuleilla.
\par 15 Dedanilaiset kävivät kauppaa sinun kanssasi. Monilukuiset saaret olivat kauppatuttujasi, sinun käskysi alaisia; norsunhampaita ja mustaa puuta ne toivat sinulle verona.
\par 16 Aram oli kauppatuttusi sinun tuotteittesi runsauden takia: rubiineilla, punapurppuralla, kirjaellulla kankaalla, valkopellavalla, koralleilla ja jaspiksella he maksoivat sinun tavarasi.
\par 17 Juuda ja Israelin maa kävivät kauppaa sinun kanssasi: Minnitin nisulla, leivoksilla, hunajalla, öljyllä ja balsamilla he maksoivat sinun vaihtotavarasi.
\par 18 Damasko oli kauppatuttusi sinun tuotteittesi runsauden ja kaikkinaisten rikkauksiesi runsauden takia: Helbonin viiniä ja Saharin villoja.
\par 19 Vedan ja Jaavan maksoivat sinun tavarasi kehruuteoksilla; taottua rautaa, kassiaa ja kalmoruokoa tuli sinulle vaihtotavarana.
\par 20 Dedan kävi kauppaa sinun kanssasi ratsastussatulain loimilla.
\par 21 Arabia ja kaikki Keedarin ruhtinaat olivat kauppatuttaviasi, sinun käskysi alaisia: lampaita, oinaita ja vuohipukkeja he sinulle kauppasivat.
\par 22 Saban ja Raman kauppiaat kävivät kauppaa sinun kanssasi: kaikkinaisilla parhailla balsameilla, kaikilla kalliilla kivillä ja kullalla he maksoivat sinun tavarasi.
\par 23 Haaran ja Kanne ja Eden, Saban kauppiaat, Assur ja Kilmad kävivät kauppaa sinun kanssasi;
\par 24 he kävivät kanssasi kauppaa sinun markkinoillasi kauniilla vaatteilla, punasinisillä ja kirjaelluilla vaipoilla, kirjavakuteisilla matoilla ja punotuilla, kestävillä köysillä.
\par 25 Tarsiin-laivat kuljettivat sinun vaihtotavaroitasi. Niin sinä tulit täpötäyteen ja ylen raskaaseen lastiin merten sydämessä.
\par 26 Soutajasi veivät sinut suurille vesille. - Itätuuli särkee sinut merten sydämessä.
\par 27 Sinun rikkautesi, tavarasi, vaihtotavarasi, merimiehesi ja laivurisi, vuotokohtaisi korjaajat, tavaranvaihtajasi ja kaikki sotamiehesi, jotka ovat sinussa ja koko suuressa joukossasi, joka sinun keskelläsi on, vaipuvat merten sydämeen sinun kukistumisesi päivänä.
\par 28 Laivuriesi parkunan äänestä vapisevat tantereet.
\par 29 Alas haaksistansa astuvat kaikki aironpitelijät, merimiehet ja kaikki meren laivurit astuvat maihin,
\par 30 huutavat sinun tähtesi suurella äänellä ja parkuvat haikeasti, heittävät tomua päänsä päälle ja vieriskelevät tuhassa.
\par 31 He ajelevat sinun tähtesi päänsä paljaaksi, kääriytyvät säkkeihin ja itkevät sinun tähtesi, mieli murheellisena, valittaen haikeasti.
\par 32 He valittaessaan virittävät sinusta itkuvirren ja laulavat sinusta: 'Kuka oli Tyyron vertainen, hänen, joka nyt on niin hiljaa keskellä merta?
\par 33 Kun sinun tavarasi saapuivat meriltä, ravitsit sinä monet kansat. Rikkauksiesi ja vaihtotavaraisi runsaudella sinä teit rikkaiksi maan kuninkaat.
\par 34 Nyt, kun jouduit haaksirikkoon, pois meriltä, vetten syvyyksiin, vaipuivat sinun mukanasi sinne vaihtotavarasi ja koko sinun suuri joukkosi.'
\par 35 Sinun tähtesi tyrmistyvät kaikki saarten asukkaat, ja heidän kuninkaansa ovat kauhun vallassa, vavistus kasvoillansa.
\par 36 Sinulle viheltävät kauppatutut kansojen joukossa. Kauhuksi olet sinä tullut, eikä sinua enää ole, hamaan ikiaikoihin asti."

\chapter{28}

\par 1 Ja minulle tuli tämä Herran sana:
\par 2 "Ihmislapsi, sano Tyyron ruhtinaalle: Näin sanoo Herra, Herra: Koska sinun sydämesi on ylpistynyt ja sinä sanot: 'Minä olen jumala, jumalain istuimella minä istun merten sydämessä', ja olet kuitenkin ihminen, et jumala, vaikka omasta mielestäsi olet jumalan vertainen -
\par 3 ja katso, viisaampi sinä oletkin kuin Daniel, ei mikään salaisuus ole sinulta pimitetty;
\par 4 viisaudessasi ja ymmärtäväisyydessäsi olet hankkinut itsellesi rikkautta, hankkinut kultaa ja hopeata aarreaittoihisi,
\par 5 ja suuressa viisaudessasi olet kaupankäynnillä kartuttanut rikkautesi, ja niin sinun sydämesi on ylpistynyt rikkaudestasi -
\par 6 sentähden, näin sanoo Herra, Herra: Koska omasta mielestäsi olet jumalan vertainen,
\par 7 sentähden, katso, minä tuon sinun kimppuusi muukalaiset, julmimmat pakanoista; he paljastavat miekkansa sinun viisautesi kauneutta vastaan ja häpäisevät sinun ihanuutesi.
\par 8 He syöksevät sinut alas kuoppaan, ja sinä kuolet, niinkuin kaadetut kuolevat, merten sydämessä.
\par 9 Vieläköhän sanot surmaajasi edessä: 'Minä olen jumala', kun kuitenkin olet ihminen, et jumala, kaatajaisi käsissä?
\par 10 Niinkuin kuolevat ympärileikkaamattomat, niin sinä kuolet muukalaisten käden kautta. Sillä minä olen puhunut, sanoo Herra, Herra."
\par 11 Ja minulle tuli tämä Herran sana:
\par 12 "Ihmislapsi, viritä itkuvirsi Tyyron kuninkaasta ja sano hänelle: Näin sanoo Herra, Herra: Sinä olet sopusuhtaisuuden sinetti, täynnä viisautta, täydellinen kauneudessa.
\par 13 Eedenissä, Jumalan puutarhassa, sinä olit. Peitteenäsi olivat kaikkinaiset kalliit kivet; karneolia, topaasia ja jaspista, krysoliittia, onyksia ja berylliä, safiiria, rubiinia ja smaragdia sekä kultaa olivat upotus- ja syvennystyöt sinussa, valmistetut sinä päivänä, jona sinut luotiin.
\par 14 Sinä olit kerubi, laajalti suojaavainen, ja minä asetin sinut pyhälle vuorelle; sinä olit jumal'olento ja käyskentelit säihkyväin kivien keskellä.
\par 15 Nuhteeton sinä olit vaellukseltasi siitä päivästä, jona sinut luotiin, siihen saakka, kunnes sinussa löydettiin vääryys.
\par 16 Suuressa kaupankäynnissäsi tuli sydämesi täyteen väkivaltaa, ja sinä teit syntiä. Niin minä karkoitin sinut häväistynä Jumalan vuorelta ja hävitin sinut, suojaava kerubi, pois säihkyväin kivien keskeltä.
\par 17 Sinun sydämesi ylpistyi sinun kauneudestasi, ihanuutesi tähden sinä kadotit viisautesi. Minä viskasin sinut maahan, annoin sinut alttiiksi kuninkaille, heidän silmänherkukseen.
\par 18 Paljoilla synneilläsi, tekemällä vääryyttä kaupoissasi, sinä olet häväissyt pyhäkkösi. Niin minä annoin sinun keskeltäsi puhjeta tulen; se kulutti sinut. Ja minä panin sinut tuhaksi maahan kaikkien silmäin edessä, jotka sinut näkivät.
\par 19 Kaikki tuttavasi kansojen seassa ovat tyrmistyneet sinun tähtesi. Kauhuksi olet sinä tullut, eikä sinua enää ole, hamaan ikiaikoihin asti."
\par 20 Ja minulle tuli tämä Herran sana:
\par 21 "Ihmislapsi, käännä kasvosi Siidonia kohti ja ennusta sitä vastaan
\par 22 ja sano: Näin sanoo Herra, Herra: Katso, minä käyn sinun kimppuusi, Siidon, ja näytän kunniani sinun keskuudessasi. Ja he tulevat tietämään, että minä olen Herra, kun minä siinä panen toimeen tuomiot ja näytän siinä pyhyyteni.
\par 23 Minä lähetän sinne ruton ja verta sen kaduille, ja kaatuneita on viruva sen keskellä, kaatuneita miekkaan, joka käy sen kimppuun joka taholta. Ja he tulevat tietämään, että minä olen Herra.
\par 24 Sitten ei Israelin heimolle enää ole pistävänä orjantappurana eikä vaivaavana ohdakkeena yksikään sen naapureista, jotka ovat heitä halveksineet. Ja he tulevat tietämään, että minä olen Herra, Herra.
\par 25 Näin sanoo Herra, Herra: Kun minä kokoan Israelin heimon kansoista, joitten sekaan he ovat hajotetut, silloin minä näytän pyhyyteni heissä pakanakansain silmien edessä, ja he saavat asua maassansa, jonka minä annoin palvelijalleni Jaakobille.
\par 26 He asuvat siinä turvallisesti ja rakentavat taloja ja istuttavat viinitarhoja. Turvallisesti he asuvat, kun minä panen toimeen tuomiot kaikille heidän naapureillensa, jotka ovat heitä halveksineet. Ja he tulevat tietämään, että minä olen Herra, heidän Jumalansa."

\chapter{29}

\par 1 Kymmenentenä vuotena, kymmenennessä kuussa, kuukauden kahdentenatoista päivänä tuli minulle tämä Herran sana:
\par 2 "Ihmislapsi, käännä kasvosi faraota, Egyptin kuningasta, kohti ja ennusta häntä ja koko Egyptiä vastaan.
\par 3 Puhu ja sano: Näin sanoo Herra, Herra: Katso, minä käyn sinun kimppuusi, farao, Egyptin kuningas - suuri krokodiili, joka makaat virtojesi keskellä, joka sanot: 'Niilivirtani on minun, ja minä olen sen itselleni tehnyt'.
\par 4 Minä panen koukut sinun leukoihisi, tartutan virtojesi kalat sinun suomuihisi ja nostan sinut ylös virtojesi keskeltä sekä kaikki virtojesi kalat, jotka ovat suomuihisi tarttuneet.
\par 5 Minä viskaan erämaahan sinut ja kaikki virtojesi kalat. Sinä putoat kedolle, ei sinua korjata, ei koota. Metsän pedoille ja taivaan linnuille minä annan sinut ruuaksi.
\par 6 Ja kaikki Egyptin asukkaat tulevat tietämään, että minä olen Herra. Sillä he ovat ruokosauva Israelin heimolle:
\par 7 kun he kädellä tarttuvat sinuun, niin sinä säryt ja lävistät heiltä olkapäät kaikki; ja kun he nojaavat sinuun, niin sinä murrut ja jäykistät heiltä lanteet kaikki.
\par 8 Sentähden, näin sanoo Herra, Herra: Katso, minä tuon sinun kimppuusi miekan ja hävitän sinusta ihmiset ja eläimet,
\par 9 ja Egyptin maa tulee autioksi ja raunioksi. Ja he tulevat tietämään, että minä olen Herra. Sillä hän on sanonut: 'Niilivirta on minun, ja minä olen sen tehnyt'.
\par 10 Sentähden, katso, minä käyn sinun ja sinun virtojesi kimppuun ja teen Egyptin maan raunioiksi, kuivaksi ja autioksi Migdolista Seveneen ja hamaan Etiopian rajaan saakka.
\par 11 Ei käy siellä ihmisjalka, ei käy siellä eläimen jalka, eikä asuta siellä neljäänkymmeneen vuoteen.
\par 12 Autioksi minä teen Egyptin maan autioiksi tehtyjen maitten joukossa, ja sen kaupungit tulevat olemaan autioina raunioiksi pantujen kaupunkien joukossa neljäkymmentä vuotta, ja egyptiläiset minä hajotan kansojen sekaan ja sirotan heidät muihin maihin.
\par 13 Sillä näin sanoo Herra, Herra: Neljänkymmenen vuoden kuluttua minä kokoan egyptiläiset kansoista, joiden sekaan he olivat hajotetut,
\par 14 käännän Egyptin kohtalon ja tuon heidät takaisin Patroksen maahan, synnyinmaahansa. Siellä he tulevat olemaan vähäpätöisenä valtakuntana.
\par 15 He tulevat olemaan vähäpätöisin valtakunnista, eivätkä enää kohoa kansakuntien ylitse. Minä vähennän heidät niin, etteivät he enää kansoja vallitse.
\par 16 Eivätkä he enää kelpaa turvaksi Israelin heimolle: he saattavat muistoon Israelin syntivelan, jos se kääntyy heitä seuraamaan. Ja he tulevat tietämään, että minä olen Herra, Herra."
\par 17 Kahdentenakymmenentenä seitsemäntenä vuotena, ensimmäisessä kuussa, kuukauden ensimmäisenä päivänä tuli minulle tämä Herran sana:
\par 18 "Ihmislapsi, Baabelin kuningas Nebukadressar on teettänyt sotajoukoillansa raskasta työtä Tyyroa vastaan: kaikki päät ovat käyneet kaljuiksi ja kaikki olkapäät lyöpyneet; mutta palkkaa ei Tyyrosta ole tullut hänelle eikä hänen sotajoukoilleen työstä, jota ovat tehneet sitä vastaan.
\par 19 Sentähden, näin sanoo Herra, Herra: Katso, minä annan Baabelin kuninkaalle Nebukadressarille Egyptin maan. Hän vie sen tavaran paljouden, saa sen saaliit, ryöstää sen ryöstökset: se on oleva hänen sotajoukkojensa palkka.
\par 20 Palkaksi, jonka toivossa hän on työtä tehnyt, minä annan hänelle Egyptin maan; sillä minun hyväkseni he ovat sitä tehneet, sanoo Herra, Herra.
\par 21 Sinä päivänä minä annan puhjeta Israelin heimolle sarven, ja sinulle minä annan voiman avata suusi heidän keskellänsä. Ja he tulevat tietämään, että minä olen Herra."

\chapter{30}

\par 1 Ja minulle tuli tämä Herran sana:
\par 2 "Ihmislapsi, ennusta ja sano: Näin sanoo Herra, Herra: Valittakaa: 'Voi sitä päivää!'
\par 3 Sillä lähellä on päivä, lähellä Herran päivä; se on pilvinen päivä, pakanakansojen aika.
\par 4 Miekka yllättää Egyptin, ja Etiopialla on tuska, kun surmattuja kaatuu Egyptissä, kun siltä otetaan pois tavarain paljous ja revitään perustukset.
\par 5 Etiopia, Puut ja Luud, kaikki sekakansa, Kuub ja liittoutuneen maan miehet kaatuvat heidän kanssansa miekkaan.
\par 6 Näin sanoo Herra: Egyptin tuet kaatuvat, ja sen ylpeä uhma alenee. Migdolista Seveneen asti he siellä kaatuvat miekkaan, sanoo Herra, Herra.
\par 7 Autioiksi he joutuvat autioiksi tehtyjen maitten joukossa, ja heidän kaupunkinsa tulevat olemaan raunioiksi pantujen kaupunkien joukossa.
\par 8 Ja he tulevat tietämään, että minä olen Herra, kun minä sytytän Egyptin tuleen ja kaikki sen auttajat murtuvat.
\par 9 Sinä päivänä lähtee minun tyköäni laivoilla lähettiläitä peljästyttämään levossa elävää Etiopiaa, ja heille tulee tuska niinkuin Egyptin päivänä; sillä katso, se tulee.
\par 10 Näin sanoo Herra, Herra: Minä teen lopun Egyptin meluavista joukoista Baabelin kuninkaan Nebukadressarin käden kautta.
\par 11 Hän ja hänen väkensä hänen kanssaan, julmimmat pakanoista, tuodaan hävittämään maata, ja he paljastavat miekkansa Egyptiä vastaan ja täyttävät maan surmatuilla.
\par 12 Minä panen virrat kuiville ja myyn maan pahojen käsiin ja teen autioksi maan ja kaiken, mitä siinä on, muukalaisten käden kautta. Minä, Herra, olen puhunut.
\par 13 Näin sanoo Herra, Herra: Minä tuhoan kivijumalat ja lopetan epäjumalat Noofista ja ruhtinaat Egyptistä: niitä ei ole enää oleva; ja minä saatan Egyptin maan pelon valtaan.
\par 14 Minä teen Patroksen autioksi, sytytän Sooanin tuleen ja panen toimeen tuomiot Noossa.
\par 15 Minä vuodatan kiivauteni Siiniin, Egyptin linnoitukseen, hävitän Noon meluavat joukot
\par 16 ja sytytän Egyptin tuleen. Siin kierii tuskassa, Noo valloitetaan, ja Noof - ahdistajia selvällä päivällä!
\par 17 Aavenin ja Piibesetin nuorukaiset kaatuvat miekkaan, ja ne itse menevät vankeuteen.
\par 18 Tehafneheessa päivä pimenee, kun minä siellä särjen Egyptin ikeen, ja siinä saa lopun sen ylpeä uhma. Sen itsensä peittää pilvi, ja sen tyttäret menevät vankeuteen.
\par 19 Niin minä panen tuomiot toimeen Egyptissä, ja he tulevat tietämään, että minä olen Herra."
\par 20 Yhdentenätoista vuotena, ensimmäisessä kuussa, kuukauden seitsemäntenä päivänä tuli minulle tämä Herran sana:
\par 21 "Ihmislapsi, minä olen murskannut faraon, Egyptin kuninkaan, käsivarren, ja katso, sitä ei sidota, ei parannella, ei panna kääreeseen, että se sidottuna vahvistuisi tarttuaksensa miekkaan.
\par 22 Sentähden, näin sanoo Herra, Herra: Katso, minä käyn faraon, Egyptin kuninkaan, kimppuun ja murskaan häneltä käsivarret, sekä voimakkaan että murskatun käden, ja pudotan miekan hänen kädestänsä.
\par 23 Minä hajotan egyptiläiset kansojen sekaan ja sirotan heidät muihin maihin.
\par 24 Baabelin kuninkaan käsivarret minä vahvistan ja annan miekkani hänen käteensä; mutta faraolta minä murskaan käsivarret, ja hän voihkii hänen edessänsä, niinkuin kaadettu voihkii.
\par 25 Niin, minä vahvistan Baabelin kuninkaan käsivarret, mutta faraon käsivarret vaipuvat alas. Ja he tulevat tietämään, että minä olen Herra, kun minä annan miekkani Baabelin kuninkaan käteen, että hän ojentaa sen Egyptin maata vastaan.
\par 26 Ja minä hajotan egyptiläiset kansojen sekaan ja sirotan heidät muihin maihin. Ja he tulevat tietämään, että minä olen Herra."

\chapter{31}

\par 1 Yhdentenätoista vuotena, kolmannessa kuussa, kuukauden ensimmäisenä päivänä tuli minulle tämä Herran sana:
\par 2 "Ihmislapsi, sano faraolle, Egyptin kuninkaalle, ja hänen meluaville joukoillensa: Kenen kaltainen sinä olet suuruudessasi?
\par 3 Katso, Assur oli setri Libanonilla, kaunislehväinen, taajavarjoinen, korkeakasvuinen, ja sen latva oli tiheän lehvistön keskellä.
\par 4 Vedet olivat sen suureksi kasvattaneet, korkeaksi oli sen saanut syvyys, joka virtoinensa kiersi sen istutusmaata ja lähetti ojiansa kaikille metsän puille.
\par 5 Sentähden siitä tuli korkeakasvuisin kaikista metsän puista, se sai haaroja paljon, ja oksat, joita se levitti, venyivät pitkiksi runsaista vesistä.
\par 6 Sen lehvillä pesivät kaikki taivaan linnut, sen oksien alla synnyttivät kaikki metsän eläimet, ja sen varjossa asuivat kaikki suuret kansat.
\par 7 Kaunis se oli suuruudessaan, oksiensa pituudessa, sillä sen juuri oli runsasten vetten ääressä.
\par 8 Eivät olleet setrit sen vertaiset Jumalan puutarhassa, eivät kypressit sen oksien veroiset, eivät plataanit niinkuin sen haarat. Ei yksikään puu Jumalan puutarhassa ollut sen veroinen kauneudessa.
\par 9 Minä tein sen niin kauniiksi oksarunsaudessaan, että kaikki Eedenin puut Jumalan puutarhassa sitä kadehtivat.
\par 10 Sentähden, näin sanoo Herra, Herra: Koska se tuli korkeakasvuiseksi ja sen latva oli tiheän lehvistön keskellä ja sen sydän ylpistyi sen korkeudesta,
\par 11 annan minä sen pakanoista mahtavimman käsiin, ja hän tekee sille, minkä tekee; jumalattomuutensa tähden minä sen olen karkoittanut.
\par 12 Ja sen hakkasivat muukalaiset, julmimmat pakanoista, ja heittivät sen maahan. Vuorille ja kaikkiin laaksoihin kaatuivat sen haarat, ja sen oksat murskaantuivat kaikkiin maan puronotkoihin. Kaikki maan kansat laskeutuivat alas sen varjosta ja jättivät sen.
\par 13 Sen kaatuneella rungolla asuvat kaikki taivaan linnut, ja sen oksien ääressä oleskelevat kaikki metsän eläimet -
\par 14 ettei yksikään puu vetten vierellä kasvaisi niin korkeaksi eikä ojentaisi latvaansa pilvien väliin ja etteivät mahtavimmatkaan niistä, ei mitkään, jotka vettä juovat, olisi pysyväisiä korkeudessaan, sillä ne ovat kaikki annetut alttiiksi kuolemalle, menemään maan syvyyteen, hautaanvaipuneitten pariin yhdessä ihmislasten kanssa.
\par 15 Näin sanoo Herra, Herra: Sinä päivänä, jona hän astui alas tuonelaan, minä verhosin syvyyden suruun hänen tähtensä, pidätin virrat, niin että paljot vedet pysähtyivät juoksussaan, ja puin Libanonin hänen tähtensä mustiin, ja kaikki metsän puut nääntyivät hänen tähtensä.
\par 16 Hänen kaatumisensa ryskeestä minä panin kansakunnat vapisemaan, kun minä syöksin hänet alas tuonelaan, hautaanvaipuneitten pariin. Mutta lohdutetuiksi tulivat maan syvyydessä kaikki Eedenin puut, Libanonin valiot ja parhaat, kaikki, jotka vettä juovat.
\par 17 Myöskin ne olivat hänen kanssansa astuneet alas tuonelaan, miekalla kaadettujen pariin; samoin hänen auttajansa, ne, jotka istuivat hänen varjossaan kansojen joukossa.
\par 18 Oliko yhtään Eedenin puista sinun vertaistasi kunniassa ja suuruudessa? Mutta sinut on Eedenin puitten kanssa syösty alas maan syvyyteen; ympärileikkaamattomien keskellä sinä makaat miekallakaadettujen parissa. Näin käy faraon ja koko hänen meluavan joukkonsa, sanoo Herra, Herra."

\chapter{32}

\par 1 Kahdentenatoista vuotena, kahdennessatoista kuussa, kuukauden ensimmäisenä päivänä tuli minulle tämä Herran sana:
\par 2 "Ihmislapsi, viritä itkuvirsi faraosta, Egyptin kuninkaasta, ja sano hänelle: Nuori leijona kansojen seassa, sinä olet hukassa! Sinä olit kuin krokodiili virroissa: sinä kuohutit virtojasi, sotkit vettä jaloillasi ja hämmensit sen virtoja.
\par 3 Näin sanoo Herra, Herra: Minä levitytän verkkoni sinun ylitsesi monien kansojen suurella joukolla, ja ne vetävät sinut ylös minun pyydykselläni.
\par 4 Minä viskaan sinut maalle, heitän sinut kedolle, panen asustamaan sinun ylläsi kaikki taivaan linnut ja syötän sinusta kylläisiksi kaiken maan eläimet.
\par 5 Minä jätän sinun lihasi vuorten hyviksi ja täytän raatokasallasi laaksot,
\par 6 juotan maan sillä, mikä sinusta valuu, sinun verelläsi aina vuorille asti, ja purojen uomat tulevat sinusta täyteen.
\par 7 Minä peitän taivaan, kun sinut sammutan, ja puen mustiin sen tähdet; auringon minä peitän pilviin, eikä kuu anna valonsa loistaa.
\par 8 Kaikki taivaan loistavat valot minä puen mustiin sinun tähtesi ja peitän sinun maasi pimeyteen, sanoo Herra, Herra.
\par 9 Minä murehdutan monien kansain sydämet, kun saatan tiedoksi sinun perikatosi kansakunnille, maihin, joita sinä et tuntenut;
\par 10 ja minä tyrmistytän sinun tähtesi monet kansat, ja heidän kuninkaansa kovin värisevät sinun tähtesi, kun minä heilutan miekkaani heidän nähtensä, ja he vapisevat joka hetki kukin omaa henkeänsä sinun kukistumisesi päivänä.
\par 11 Sillä näin sanoo Herra, Herra: Baabelin kuninkaan miekka yllättää sinut.
\par 12 Minä kaadan sinun meluisan joukkosi sankarien miekoilla; ne ovat julmimpia pakanoista kaikki tyynni. He kukistavat Egyptin komeuden, kaikki sen meluisa joukko tuhotaan.
\par 13 Ja minä hävitän siitä kaiken karjan runsasten vetten ääriltä. Eikä niitä enää sotke ihmisen jalka, eivätkä sotke niitä karjan sorkat.
\par 14 Silloin minä annan sen vetten laskeutua kirkkaiksi, ja minä panen sen virrat juoksemaan kuin öljyn, sanoo Herra, Herra.
\par 15 Kun minä teen Egyptin maan autioksi, kun maa tulee autioksi kaikesta, mitä siinä on, ja kun minä surmaan kaikki sen asukkaat, tulevat he tietämään, että minä olen Herra.
\par 16 Itkuvirsi tämä on, ja sitä kyllä viritetään. Pakanakansain tyttäret sitä virittävät - virittävät sitä Egyptistä ja kaikesta sen meluisasta joukosta, sanoo Herra, Herra."
\par 17 Kahdentenatoista vuotena, kuukauden viidentenätoista päivänä tuli minulle tämä Herran sana:
\par 18 "Ihmislapsi, veisaa kuolinvalitus Egyptin meluisasta joukosta, saata se ja mahtavain pakanakansojen tyttäret alas maan syvyyksiin, hautaanvaipuneitten pariin.
\par 19 Oletko vielä kaikkia muita ihanampi? Astu alas ja anna laittaa makuusijasi ympärileikkaamattomien pariin.
\par 20 He kaatuvat miekallasurmattujen joukkoon. Miekka on jo annettu! Temmatkaa pois Egypti kaikkine meluisine joukkoineen.
\par 21 Mahtavimmat sankareista puhuvat tuonelan keskeltä sille ynnä sen auttajille: 'Alas ovat astuneet, siellä makaavat ympärileikkaamattomat, miekalla surmatut'.
\par 22 Siellä on Assur kaikkine joukkoinensa. Niitten haudat ovat hänen ympärillänsä. Kaikki tyynni ovat surmattuja, miekkaan kaatuneita.
\par 23 Hänen hautansa ovat pohjimmaisessa kuopassa, ja hänen joukkonsa on hänen hautansa ympärillä; ne ovat kaikki tyynni surmattuja, miekkaan kaatuneita, nuo, jotka levittivät kauhua elävien maassa.
\par 24 Siellä on Eelam, ja koko hänen meluisa joukkonsa on hänen hautansa ympärillä; kaikki tyynni surmattuja, miekkaan kaatuneita, jotka ympärileikkaamattomina astuivat alas maan syvyyksiin, nuo, jotka levittivät kauhuansa elävien maassa, mutta saavat kantaa häpeänsä hautaanvaipuneitten parissa.
\par 25 Surmattujen keskellä on makuusija annettu hänelle ynnä kaikelle hänen meluisalle joukolleen. Niitten haudat ovat hänen ympärillänsä. Kaikki tyynni ne ovat ympärileikkaamattomia, miekalla surmattuja, sillä he olivat kauhuna elävien maassa, mutta saavat kantaa häpeänsä hautaanvaipuneitten parissa. Surmattujen keskelle on hänet pantu.
\par 26 Siellä on Mesek-Tuubal kaikkine meluisine joukkoineen. Niitten haudat ovat hänen ympärillänsä; ne ovat kaikki tyynni ympärileikkaamattomia, miekalla surmattuja, sillä ne levittivät kauhuansa elävien maassa.
\par 27 He eivät makaa sankarien parissa, jotka ovat kaatuneet ympärileikkaamattomien joukosta, jotka ovat astuneet alas tuonelaan sota-aseinensa, joitten pään alle on pantu heidän miekkansa ja joitten luiden yllä on heidän syntivelkansa, sillä sankarien kauhu oli elävien maassa.
\par 28 Sinutkin muserretaan ympärileikkaamattomien joukossa, ja sinä saat maata miekallasurmattujen parissa.
\par 29 Siellä on Edom, sen kuninkaat ja kaikki sen ruhtinaat, jotka sankaruudessaan pantiin miekallasurmattujen pariin. He makaavat ympärileikkaamattomien ja hautaanvaipuneitten parissa.
\par 30 Siellä ovat kaikki pohjoisen maan ruhtinaat ja kaikki siidonilaiset, jotka ovat astuneet surmattujen pariin kauhistavaisuudessaan, ovat tulleet häpeään sankaruudessaan ja makaavat ympärileikkaamattomina miekallasurmattujen parissa ja saavat kantaa häpeänsä hautaanvaipuneitten parissa. -
\par 31 Ne on farao näkevä ja tuleva lohdutetuksi kaikesta meluisasta joukostansa. Miekalla on surmattu farao kaikkine sotaväkineen, sanoo Herra, Herra.
\par 32 Sillä minä levitin hänen kauhuansa elävien maassa, mutta faraolle ynnä kaikelle hänen meluisalle joukolleen laitetaan makuusija ympärileikkaamattomien sekaan, miekallasurmattujen pariin, sanoo Herra, Herra."

\chapter{33}

\par 1 Minulle tuli tämä Herran sana:
\par 2 "Ihmislapsi, puhu kansasi lapsille ja sano heille: Jos minä annan miekan tulla maan kimppuun ja maan kansa on ottanut yhden miehen joukostansa ja asettanut hänet vartijakseen
\par 3 ja jos hän näkee miekan tulevan maan kimppuun ja puhaltaa pasunaan ja varoittaa kansaa,
\par 4 ja joku kuulee pasunan äänen, mutta ei ota varoituksesta vaaria, ja miekka tulee ja ottaa hänet pois, niin hänen verensä tulee hänen oman päänsä päälle:
\par 5 hän kuuli pasunan äänen, mutta ei ottanut varoituksesta vaaria; hänen verensä tulee hänen päällensä. Jos olisi ottanut varoituksesta vaarin, olisi hän pelastanut sielunsa.
\par 6 Jos taas vartija näkee miekan tulevan, mutta ei puhalla pasunaan eikä kansa saa varoitusta, ja miekka tulee ja ottaa pois jonkun sielun heistä, on hän otettu pois synnissänsä, mutta hänen verensä minä vaadin vartijan kädestä.
\par 7 Ja sinä, ihmislapsi! Minä olen asettanut sinut Israelin heimolle vartijaksi. Kun kuulet sanan minun suustani, on sinun varoitettava heitä minun puolestani.
\par 8 Jos minä sanon jumalattomalle: jumalaton, sinun on kuolemalla kuoltava, mutta sinä et puhu varoittaaksesi jumalatonta hänen tiestänsä, niin se jumalaton kuolee synnissänsä, mutta hänen verensä minä vaadin sinun kädestäsi.
\par 9 Mutta jos sinä varoitat jumalatonta hänen tiestänsä, että hän kääntyisi siltä pois, eikä hän tieltänsä käänny, niin hän kuolee synnissänsä, mutta sinä olet sielusi pelastanut.
\par 10 Ja sinä, ihmislapsi! Sano Israelin heimolle: Näin te sanotte: 'Niin, rikoksemme ja syntimme ovat meidän päällämme, ja me riudumme niiden tähden. Kuinka me voisimme pysyä elossa?'
\par 11 Sano heille: Niin totta kuin minä elän, sanoo Herra, Herra, ei ole minulle mieleen jumalattoman kuolema, vaan se, että jumalaton kääntyy tieltänsä ja elää. Kääntykää, kääntykää pois pahoilta teiltänne; ja minkätähden te kuolisitte, Israelin heimo!
\par 12 Ja sinä, ihmislapsi! Sano kansasi lapsille: Vanhurskasta ei pelasta hänen vanhurskautensa sinä päivänä, jona hän rikkoo, ja jumalaton pääsee suistumasta turmioon jumalattomuutensa tähden sinä päivänä, jona hän kääntyy pois jumalattomuudestaan, ja vanhurskas ei voi elää vanhurskautensa turvin sinä päivänä, jona hän syntiä tekee.
\par 13 Jos minä sanon vanhurskaalle, että hän totisesti saa elää, mutta hän sitten luottaa vanhurskauteensa ja tekee vääryyttä, niin hänen vanhurskauttansa ei ensinkään muisteta, vaan hän kuolee vääryydessään, jota on tehnyt.
\par 14 Ja jos minä sanon jumalattomalle, että hänen totisesti on kuoltava, mutta hän kääntyy pois synnistänsä ja tekee oikeuden ja vanhurskauden
\par 15 - antaa takaisin, tuo jumalaton, pantin, korvaa riistämänsä ja vaeltaa elämän käskyjen mukaan, niin ettei vääryyttä tee - niin totisesti hän saa elää; ei hänen ole kuoltava.
\par 16 Hänen syntejänsä, jotka hän on tehnyt, ei ensinkään muisteta: hän on tehnyt oikeuden ja vanhurskauden, hän totisesti saa elää.
\par 17 Ja vielä sinun kansasi lapset sanovat: 'Herran tie ei ole oikea'. Juuri heidän omat tiensä eivät ole oikeat.
\par 18 Jos vanhurskas kääntyy pois vanhurskaudestansa ja vääryyttä tekee, on hänen sentähden kuoltava.
\par 19 Ja jos jumalaton kääntyy pois jumalattomuudestansa ja tekee oikeuden ja vanhurskauden, saa hän sentähden elää.
\par 20 Ja vielä te sanotte: 'Herran tie ei ole oikea': minä tuomitsen teidät, te Israelin heimo, itsekunkin hänen teittensä mukaan."
\par 21 Kahdentenatoista meidän pakkosiirtolaisuutemme vuotena, kymmenennessä kuussa, kuukauden viidentenä päivänä tuli minun luokseni pakolainen Jerusalemista ja sanoi: "Kaupunki on valloitettu".
\par 22 Herran käsi oli tullut minun päälleni jo illalla, ennen pakolaisen tuloa, ja hän avasi minun suuni ennen tämän tuloa aamulla. Niin avautui minun suuni, enkä minä enää ollut mykkänä.
\par 23 Ja minulle tuli tämä Herran sana:
\par 24 "Ihmislapsi, ne, jotka asuvat noilla raunioilla Israelin maassa, sanovat näin: 'Aabraham oli vain yksi, ja hän peri maan; meitä on paljon, meille on maa perinnöksi annettu'.
\par 25 Sano sentähden heille: Näin sanoo Herra, Herra: Te syötte lihaa verinensä, luotte silmänne kivijumaliinne ja vuodatatte verta; ja tekö perisitte maan?
\par 26 Te seisotte miekkanne varassa, harjoitatte kauhistuksia ja saastutatte toistenne vaimoja; ja tekö perisitte maan?
\par 27 Näin on sinun heille sanottava: Näin sanoo Herra, Herra: Niin totta kuin minä elän, niin ne, jotka ovat raunioilla, kaatuvat miekkaan, ja ne, jotka ovat kedolla, minä annan petoeläimille syötäväksi, ja ne, jotka ovat vuorenhuipuilla ja luolissa, kuolevat ruttoon.
\par 28 Ja minä teen maan autioksi ja hävitetyksi, ja heidän ylpeästä uhmastaan tulee loppu, ja Israelin vuoret joutuvat autioiksi, niin ettei siellä kenkään kulje.
\par 29 Ja he tulevat tietämään, että minä olen Herra, kun minä teen maan autioksi ja hävitetyksi kaikkien kauhistusten tähden, joita he ovat harjoittaneet.
\par 30 Ja sinä, ihmislapsi! Sinun kansasi lapset puhuvat sinusta seinänvierustoilla ja talojen ovilla ja sanovat keskenään, toinen toisellensa, näin: 'Lähtekää kuulemaan, millainen sana nyt on tullut Herralta'.
\par 31 He tulevat sinun luoksesi joukoittain, istuvat edessäsi minun kansanani ja kuuntelevat sinun sanojasi, mutta he eivät tee niitten mukaan, sillä he osoittavat rakkautta suullansa, mutta heidän sydämensä kulkee väärän voiton perässä.
\par 32 Ja katso, sinä olet heille kuin rakkauslaulu, kauniisti laulettu ja hyvin soitettu: he sanojasi kyllä kuuntelevat, mutta eivät tee niitten mukaan.
\par 33 Mutta kun se toteutuu - ja katso, se toteutuu - silloin he tulevat tietämään, että heidän keskuudessansa on ollut profeetta."

\chapter{34}

\par 1 Minulle tuli tämä Herran sana:
\par 2 "Ihmislapsi, ennusta Israelin paimenia vastaan, ennusta ja sano heille - paimenille: Näin sanoo Herra, Herra: Voi Israelin paimenia, jotka ovat itseänsä kainneet! Eikö paimenten ole kaittava lampaita?
\par 3 Te olette syöneet rasvat, pukeneet päällenne villat, teurastaneet lihavat; mutta ette ole kainneet laumaa,
\par 4 ette ole vahvistaneet heikkoja, ette ole parantaneet sairaita, sitoneet haavoittuneita, tuoneet takaisin eksyneitä, etsineet kadonneita, vaan te olette vallinneet niitä tylysti ja väkivaltaisesti.
\par 5 Ja niin ne ovat hajaantuneet paimenta vailla ja joutuneet kaikkien metsän petojen syötäviksi - hajaantuneet ne ovat.
\par 6 Minun lampaani harhailevat kaikilla vuorilla ja kaikilla korkeilla kukkuloilla; pitkin koko maata ovat minun lampaani hajallaan, eikä kenkään niistä välitä eikä niitä etsi.
\par 7 Sentähden, paimenet, kuulkaa Herran sana:
\par 8 Niin totta kuin minä elän, sanoo Herra, Herra, totisesti, koska minun lampaani ovat ryöstettävinä ja koska minun lampaani ovat kaikkien metsän petojen syötävinä, kun paimenta ei ole ja kun minun paimeneni eivät välitä minun lampaistani, vaan minun paimeneni kaitsevat itseänsä, eivätkä kaitse minun lampaitani,
\par 9 sentähden, paimenet, kuulkaa Herran sana:
\par 10 Näin sanoo Herra, Herra: Katso, minä käyn paimenten kimppuun, vaadin lampaani heidän kädestänsä ja teen lopun heidän lammasten-kaitsennastaan, eivätkä paimenet saa enää kaita itseänsä. Minä pelastan lampaani heidän kidastansa, eivätkä ne sitten enää ole heidän syötävinänsä.
\par 11 Sillä näin sanoo Herra, Herra: Katso, minä itse etsin lampaani ja pidän niistä huolen.
\par 12 Niinkuin paimen pitää huolen laumastaan, kun hän on lampaittensa keskellä ja ne ovat hajallaan, niin minä pidän huolen lampaistani, ja minä pelastan ne joka paikasta, minne ne ovat hajaantuneet pilvisenä ja pimeänä päivänä.
\par 13 Minä vien ne pois kansojen seasta ja kokoan ne muista maista, tuon ne omaan maahansa ja kaitsen niitä Israelin vuorilla, puronotkoissa ja kaikissa maan asuttavissa paikoissa.
\par 14 Hyvillä ruokamailla minä niitä kaitsen, ja Israelin korkeilla vuorilla on niillä oleva laitumensa. Siellä ne saavat levätä hyvällä laitumella, ja lihava ruokamaa niillä on oleva Israelin vuorilla.
\par 15 Minä itse kaitsen lampaani ja vien itse ne lepäämään, sanoo Herra, Herra.
\par 16 Kadonneet minä tahdon etsiä, eksyneet tuoda takaisin, haavoittuneet sitoa, heikkoja vahvistaa; mutta lihavat ja väkevät minä hävitän. Minä kaitsen niitä niin, kuin oikein on.
\par 17 Mutta te, minun lampaani! Näin sanoo Herra, Herra: Katso, minä tahdon tuomita lampaan ja lampaan, oinasten ja vuohipukkien välillä.
\par 18 Eikö teille riitä, että olette hyvällä laitumella, kun vielä tallaatte jaloillanne loput laitumestanne, ja että saatte juoda kirkasta vettä, kun vielä hämmennätte jaloillanne loput?
\par 19 Ja onko minun lampaitteni oltava laitumella siinä, mitä jalkanne ovat tallanneet, ja juotava sitä, mitä jalkanne ovat hämmentäneet?
\par 20 Sentähden sanoo Herra, Herra niille näin: Katso, minä, minä tuomitsen lihavan lampaan ja laihan lampaan välillä.
\par 21 Koska te olette kylki- ja niskavoimalla sysineet ja sarvillanne puskeneet kaikkia heikkoja, kunnes olette saaneet ne ajetuiksi ulos ja hajallensa,
\par 22 niin minä tahdon vapauttaa lampaani, etteivät ne enää jää ryöstettäviksi; ja minä tahdon tuomita lampaan ja lampaan välillä.
\par 23 Ja minä herätän heille yhden paimenen heitä kaitsemaan, palvelijani Daavidin; hän on kaitseva heitä ja oleva heidän paimenensa.
\par 24 Ja minä, Herra, olen heidän Jumalansa, ja minun palvelijani Daavid on ruhtinas heidän keskellänsä. Minä, Herra, olen puhunut.
\par 25 Ja minä teen heidän kanssansa rauhan liiton; minä lopetan maasta pahat eläimet, niin että he asuvat turvallisesti erämaassa ja nukkuvat metsiköissä;
\par 26 ja minä teen siunatuiksi heidät ja kaiken, mitä minun kukkulani ympärillä on, ja vuodatan sateen ajallansa - ne ovat siunauksen sateita.
\par 27 Ja kedon puut kantavat hedelmänsä, ja maa antaa satonsa, ja he saavat olla turvassa maassansa. Ja he tulevat tietämään, että minä olen Herra, kun minä särjen heidän ikeensä puut ja pelastan heidät heidän orjuuttajainsa käsistä.
\par 28 Eivätkä he enää ole pakanain ryöstettävinä, eivätkä metsän pedot heitä syö, vaan he asuvat turvassa, kenenkään peloittelematta.
\par 29 Ja minä annan nousta heille istutuksen, joka on oleva kunniaksi, niin ettei heidän tarvitse siinä maassa menehtyä nälkään eikä enää kärsiä pakanain pilkkaa.
\par 30 Ja he tulevat tietämään, että minä, Herra, heidän Jumalansa, olen heidän kanssansa ja että he, Israelin heimo, ovat minun kansani, sanoo Herra, Herra.
\par 31 Niin, te olette minun lampaani, minun laitumeni lampaat, te ihmiset; minä olen teidän Jumalanne, sanoo Herra, Herra."

\chapter{35}

\par 1 Minulle tuli tämä Herran sana:
\par 2 "Ihmislapsi, käännä kasvosi Seirin vuorta kohti ja ennusta sitä vastaan
\par 3 ja sano sille: Näin sanoo Herra, Herra: Katso, minä käyn sinun kimppuusi, Seirin vuori, ojennan käteni sinua vastaan ja teen sinut autioksi ja hävitetyksi.
\par 4 Sinun kaupunkisi minä teen raunioiksi ja sinä itse tulet autioksi. Ja sinä tulet tietämään, että minä olen Herra.
\par 5 Koska sinä pidit ikuista vihaa ja annoit israelilaiset alttiiksi miekalle heidän onnettomuutensa aikana, aikana, jolloin syntivelka tuli loppumääräänsä,
\par 6 sentähden, niin totta kuin minä elän, sanoo Herra, Herra, minä teen sinut verivirraksi, ja veri on sinua vainoava: totisesti, vereen saakka sinä olet vihannut, ja veri on sinua vainoava.
\par 7 Minä teen Seirin vuoren tyhjäksi ja autioksi ja hävitän sieltä menijän ja tulijan.
\par 8 Sen vuoret minä täytän sen surmatuilla. Miekalla surmattuja kaatuu sinun kukkuloillasi ja laaksoissasi ja kaikissa puronotkoissasi.
\par 9 Minä teen sinut ikiautioksi, ja sinun kaupunkisi jäävät asumattomiksi. Ja te tulette tietämään, että minä olen Herra.
\par 10 Koska sinä olet sanonut: 'Ne kaksi kansaa ja kaksi maata ovat minun, me otamme ne omiksemme', vaikka siellä on Herra,
\par 11 sentähden, niin totta kuin minä elän, sanoo Herra, Herra, minä teen sinun kiukkusi ja kiivautesi mukaan, teen sen, minkä sinä vihassasi olet heille tehnyt. Ja minä teen itseni heille tunnetuksi, kun niin tuomitsen sinut,
\par 12 ja sinä tulet tietämään, että minä olen Herra. Minä olen kuullut kaikki sinun pilkkasi, jotka olet puhunut Israelin vuoria vastaan, kun olet sanonut: 'Ne ovat autioina, ne ovat annetut meidän syötäviksemme'.
\par 13 Te suurentelitte minua vastaan suullanne, ja ylenpalttiset olivat teidän puheenne minua vastaan: minä olen ne kuullut.
\par 14 Näin sanoo Herra, Herra: Iloksi kaikelle maalle minä teen sinut autioksi.
\par 15 Niinkuin sinä iloitsit Israelin heimon perintöosasta, kun se tuli autioksi, niin teen minäkin sinulle: autioksi tulet sinä, Seirin vuori, samoin koko Edom kaikkinensa. Ja niin he tulevat tietämään, että minä olen Herra."

\chapter{36}

\par 1 "Ja sinä, ihmislapsi, ennusta Israelin vuorista ja sano:
\par 2 Israelin vuoret, kuulkaa Herran sana: Näin sanoo Herra, Herra: Koska vihollinen on sanonut teistä: 'Kas niin!' ja: 'Ikuiset kukkulat ovat tulleet meille perinnöksi',
\par 3 sentähden ennusta ja sano: Näin sanoo Herra, Herra: Koska - niin, koska teitä on hävitetty ja poljettu joka taholta, että joutuisitte muiden kansain omaisuudeksi, ja koska te olette joutuneet pahain kielten ja kansan panettelun alaisiksi,
\par 4 sentähden, Israelin vuoret, kuulkaa Herran, Herran sana: Näin sanoo Herra, Herra vuorille ja kukkuloille, puronotkoille ja laaksoille, autioille raunioille ja hyljätyille kaupungeille, jotka ovat joutuneet saaliiksi ja pilkaksi muille kansoille, mitä ympärillä on.
\par 5 Sentähden sanoo Herra, Herra näin: Totisesti, kiivauteni tulessa minä puhun muita kansoja vastaan ja koko Edomia vastaan, jotka ovat kaikesta sydämestään iloiten ja sielu täynnä ylenkatsetta ottaneet omakseen minun maani karkoittaaksensa siitä ihmiset ja ryöstääksensä sen.
\par 6 Sentähden ennusta Israelin maasta ja sano vuorille ja kukkuloille, puronotkoille ja laaksoille: Näin sanoo Herra, Herra: Katso, kiivaudessani ja vihassani minä puhun, koska teidän täytyy kärsiä pakanakansain pilkkaa.
\par 7 Sentähden, näin sanoo Herra, Herra: Minä kohotan käteni: totisesti saavat pakanakansat, jotka teidän ympärillänne ovat, itse kärsiä oman pilkkansa.
\par 8 Mutta te, Israelin vuoret, teette lehvänne ja kannatte hedelmänne minun kansalleni Israelille, sillä he ovat aivan lähellä, ovat tulossa.
\par 9 Sillä katso, minä tulen teidän tykönne ja käännyn teidän puoleenne, ja teidät viljellään ja teihin kylvetään.
\par 10 Ja minä lisään teille ihmisiä runsaasti - kaiken Israelin heimon kokonansa - ja kaupungit asutetaan ja rauniot rakennetaan.
\par 11 Ja minä lisään teille ihmisiä ja karjaa runsaasti, ja ne lisääntyvät ja ovat hedelmälliset. Minä teen teidät asutuiksi, niinkuin olitte muinaisina päivinänne, ja teen hyvää vielä enemmän kuin alkuaikoinanne. Ja te tulette tietämään, että minä olen Herra.
\par 12 Minä annan teidän kukkuloillanne kulkea ihmisten, kansani Israelin; ja he ottavat sinut omaksensa, ja sinä tulet heille perintöosaksi, ja sinä et enää tästälähin riistä heidän lapsiansa.
\par 13 Näin sanoo Herra, Herra: Koska teille on sanottu: 'Sinä olet ihmissyöjätär, oman kansasi lasten riistäjätär',
\par 14 sentähden et sinä ole enää syövä ihmisiä etkä enää riistävä lapsia omalta kansaltasi, sanoo Herra, Herra.
\par 15 Enkä minä enää anna sinun joutua kuulemaan pakanakansojen pilkkaa, eikä tarvitse sinun enää kärsiä kansojen herjauksia, et myöskään sinä enää omaa kansaasi kaada, sanoo Herra, Herra."
\par 16 Ja minulle tuli tämä Herran sana:
\par 17 "Ihmislapsi, kun Israelin heimo asui maassansa, he saastuttivat sen vaelluksellaan ja teoillaan. Niinkuin kuukautistilan saastaisuus oli minun edessäni heidän vaelluksensa.
\par 18 Niin minä vuodatin kiivauteni heidän ylitsensä veren tähden, jota he olivat vuodattaneet maassa, ja heidän kivijumalainsa tähden, joilla he olivat sen saastuttaneet.
\par 19 Ja minä hajotin heidät pakanakansojen sekaan, ja heidät siroteltiin muihin maihin. Vaelluksensa ja tekojensa mukaan minä heidät tuomitsin.
\par 20 Niin he tulivat pakanakansain sekaan; minne tulivatkin, he häpäisivät minun pyhän nimeni, kun heistä sanottiin: 'Nämä ovat Herran kansa, mutta heidän on täytynyt lähteä pois hänen maastansa'.
\par 21 Niin minun tuli sääli pyhää nimeäni, jonka Israelin heimo saastutti pakanakansain seassa, minne tulivatkin.
\par 22 Sentähden sano Israelin heimolle: Näin sanoo Herra, Herra: En tee minä teidän tähtenne, Israelin heimo, sitä minkä teen, vaan oman pyhän nimeni tähden, jonka te olette häväisseet pakanakansain seassa, minne tulittekin.
\par 23 Minä pyhitän suuren nimeni, joka on häväisty pakanakansain seassa, jonka te olette häväisseet heidän keskellänsä. Ja pakanakansat tulevat tietämään, että minä olen Herra, sanoo Herra, Herra, kun minä osoitan teissä pyhyyteni heidän silmäinsä edessä.
\par 24 Minä otan teidät pois pakanakansoista ja kokoan teidät kaikista maista ja tuon teidät omaan maahanne.
\par 25 Ja minä vihmon teidän päällenne puhdasta vettä, niin että te puhdistutte; kaikista saastaisuuksistanne ja kaikista kivijumalistanne minä teidät puhdistan.
\par 26 Ja minä annan teille uuden sydämen, ja uuden hengen minä annan teidän sisimpäänne. Minä poistan teidän ruumiistanne kivisydämen ja annan teille lihasydämen.
\par 27 Henkeni minä annan teidän sisimpäänne ja vaikutan sen, että te vaellatte minun käskyjeni mukaan, noudatatte minun oikeuksiani ja pidätte ne.
\par 28 Niin te saatte asua maassa, jonka minä annoin teidän isillenne; ja te olette minun kansani, ja minä olen teidän Jumalanne.
\par 29 Minä vapahdan teidät kaikista saastaisuuksistanne. Ja minä kutsun esiin viljan ja teen sen runsaaksi enkä anna teille nälänhätää.
\par 30 Minä teen runsaaksi puitten hedelmän ja pellon sadon, niin ettette enää joudu kärsimään kansojen seassa herjausta nälän takia.
\par 31 Niin te muistatte huonon vaelluksenne ja tekonne, jotka eivät olleet hyvät, ja teitä kyllästyttää oma itsenne rikostenne ja kauhistustenne tähden.
\par 32 En minä teidän tähtenne sitä tee, sanoo Herra, Herra: se olkoon teille tiettävä. Hävetkää ja tuntekaa häpeätä vaelluksenne tähden, te Israelin heimo.
\par 33 Näin sanoo Herra, Herra: Sinä päivänä, jona minä teidät puhdistan kaikista pahoista teoistanne, minä asutan kaupungit, ja rauniot rakennetaan jälleen,
\par 34 ja autiomaa viljellään, sen sijaan että se on ollut autiona jokaisen ohitsekulkijan silmäin edessä.
\par 35 Silloin sanotaan: 'Tästä maasta, joka oli autiona, on tullut kuin Eedenin puutarha; ja raunioina olleet, autiot ja maahan revityt kaupungit ovat varustettuja ja asuttuja'.
\par 36 Niin tulevat pakanakansat, joita on jäljellä teidän ympärillänne, tietämään, että minä, Herra, rakennan jälleen alasrevityn ja istutan aution: minä, Herra, olen puhunut, ja minä teen sen.
\par 37 Näin sanoo Herra, Herra: Vielä tätäkin annan Israelin heimon minulta anoa, että tekisin heille sen: minä lisään heille ihmisiä runsaasti kuin lammaslaumaa.
\par 38 Niinkuin on pyhitettyjen uhrilammasten laumaa, niinkuin Jerusalemin lammaslaumaa sen juhlissa, niin tulevat raunioina olleet kaupungit täyteen ihmislaumaa. Ja niin he tulevat tietämään, että minä olen Herra."

\chapter{37}

\par 1 Herran käsi tuli minun päälleni ja vei minut pois Herran hengessä ja laski minut keskelle laaksoa. Ja se oli täynnä luita.
\par 2 Ja hän kuljetti minua ympäri niitten ohitse; ja katso, niitä oli hyvin paljon laakson kamaralla, ja katso, ne olivat hyvin kuivia.
\par 3 Niin hän sanoi minulle: "Ihmislapsi, voivatkohan nämä luut tulla eläviksi?" Mutta minä sanoin: "Herra, Herra, sinä sen tiedät".
\par 4 Niin hän sanoi minulle: "Ennusta näistä luista ja sano niille: Kuivat luut, kuulkaa Herran sana.
\par 5 Näin sanoo Herra, Herra näille luille: Katso, minä annan tulla teihin hengen, ja te tulette eläviksi.
\par 6 Minä panen teihin jänteet, kasvatan teihin lihan, vedän yllenne nahan ja annan teihin hengen, ja te tulette eläviksi. Ja te tulette tietämään, että minä olen Herra."
\par 7 Minä ennustin, niinkuin minua oli käsketty. Ja kävi humahdus, kun minä ennustin; ja katso, kuului kolina, ja luut lähenivät toisiaan, luu luutansa.
\par 8 Minä näin, ja katso: niihin tulivat jänteet ja kasvoi liha, ja päälle vetäytyi niihin nahka; mutta henkeä niissä ei ollut.
\par 9 Niin hän sanoi minulle: "Ennusta hengestä, ennusta, ihmislapsi, ja sano hengelle: Näin sanoo Herra, Herra: Tule, henki, neljästä tuulesta ja puhalla näihin surmattuihin, että ne tulisivat eläviksi."
\par 10 Minä ennustin, niinkuin hän oli minua käskenyt, ja niihin tuli henki, ja ne tulivat eläviksi ja nousivat ylös jaloillensa: ylenmäärin suuri joukko.
\par 11 Ja hän sanoi minulle: "Ihmislapsi, nämä luut ovat koko Israelin heimo. Katso, he sanovat: 'Meidän luumme ovat kuivettuneet, toivomme on mennyttä, me olemme hukassa'.
\par 12 Sentähden ennusta ja sano heille: Näin sanoo Herra, Herra: Katso, minä avaan teidän hautanne ja nostan teidät, minun kansani, ylös haudoistanne ja vien teidät Israelin maahan.
\par 13 Ja siitä te tulette tietämään, että minä olen Herra, kun minä avaan teidän hautanne ja nostan teidät, minun kansani, ylös haudoistanne.
\par 14 Ja minä annan teihin henkeni, niin että te tulette eläviksi, ja sijoitan teidät omaan maahanne. Ja te tulette tietämään, että minä olen Herra: minä olen puhunut, ja minä sen teen, sanoo Herra."
\par 15 Ja minulle tuli tämä Herran sana:
\par 16 "Sinä, ihmislapsi, ota puusauva ja kirjoita siihen: 'Juudalle ja häneen liittyneille israelilaisille'. Ota sitten toinen puusauva ja kirjoita siihen: 'Joosefille; Efraimin ja kaiken häneen liittyneen Israelin heimon sauva'.
\par 17 Ja pane ne lähekkäin, pääksytysten, niin että ne tulevat yhdeksi sinun kädessäsi.
\par 18 Kun sitten kansasi lapset sanovat sinulle näin: 'Etkö selitä meille, mitä sinä tällä tarkoitat?'
\par 19 niin puhu heille: Näin sanoo Herra, Herra: Katso, minä otan Joosefin sauvan, joka on Efraimin kädessä, ja häneen liittyneet Israelin sukukunnat, ja minä asetan ne yhteen Juudan sauvan kanssa ja teen niistä yhden sauvan, niin että ne tulevat yhdeksi minun kädessäni.
\par 20 Ja kun sauvat, joihin olet kirjoittanut, ovat sinun kädessäsi, heidän silmäinsä edessä,
\par 21 niin puhu heille: Näin sanoo Herra, Herra: Katso, minä otan israelilaiset pois pakanakansojen keskuudesta, minne vain he ovat kulkeutuneet, kokoan heidät joka taholta ja tuon heidät omaan maahansa.
\par 22 Minä teen heidät yhdeksi kansaksi siinä maassa, Israelin vuorilla, ja yksi kuningas on oleva kuninkaana heillä kaikilla. Eivätkä he enää ole kahtena kansana eivätkä enää jakaantuneina kahdeksi valtakunnaksi.
\par 23 Eivät myöskään he enää saastuta itseänsä kivijumalillaan, iljetyksillään eivätkä millään rikkomuksillansa, vaan minä vapautan heidät kaikista asuinpaikoistaan, joissa ovat syntiä tehneet, ja puhdistan heidät. Ja he ovat minun kansani, ja minä olen heidän Jumalansa.
\par 24 Minun palvelijani Daavid on oleva heidän kuninkaansa, ja heillä kaikilla on oleva yksi paimen. Ja he vaeltavat minun oikeuksieni mukaan ja noudattavat minun käskyjäni ja pitävät ne.
\par 25 He saavat asua maassa, jonka minä annoin palvelijalleni Jaakobille ja jossa teidän isänne ovat asuneet. Siinä saavat asua he, heidän lapsensa ja lastensa lapset iankaikkisesti, ja minun palvelijani Daavid on oleva heidän ruhtinaansa iankaikkisesti.
\par 26 Minä teen heidän kanssansa rauhan liiton - se on oleva iankaikkinen liitto heidän kanssansa - istutan ja runsaasti kartutan heidät ja asetan pyhäkköni olemaan heidän keskellänsä iankaikkisesti.
\par 27 Minun asumukseni on oleva heidän yllänsä, ja minä olen heidän Jumalansa, ja he ovat minun kansani.
\par 28 Ja pakanakansat tulevat tietämään, että minä olen Herra, joka pyhitän Israelin, kun minun pyhäkköni on heidän keskellänsä iankaikkisesti."

\chapter{38}

\par 1 Minulle tuli tämä Herran sana:
\par 2 "Ihmislapsi, käännä kasvosi kohti Googia Maagogin maassa, Roosin, Mesekin ja Tuubalin ruhtinasta, ja ennusta häntä vastaan
\par 3 ja sano: Näin sanoo Herra, Herra: Katso, minä käyn sinun kimppuusi, Goog, sinä Roosin, Mesekin ja Tuubalin ruhtinas.
\par 4 Minä kuljetan sinua, panen koukut sinun leukoihisi ja nostatan sotaan sinut ja kaiken sinun sotaväkesi: hevoset ja ratsumiehet, kaikki pulskasti puettuja, suuren joukon suurine ja pienine kilpineen, miekankantajia kaikki.
\par 5 Persia, Etiopia ja Puut ovat heidän kanssansa, kilvet ja kypärit on heillä kaikilla.
\par 6 Goomer ja kaikki sen sotalaumat, Toogarman heimo pohjan periltä ja kaikki sen sotalaumat - lukuisat kansat ovat sinun kanssasi.
\par 7 Ole valmis, varustaudu, sinä ja kaikki joukkosi, jotka ovat kokoontuneet sinun luoksesi, ja ole sinä varalla heitä varten.
\par 8 Pitkien aikojen perästä sinä saat määräyksen, vuotten lopulla sinun on karattava maahan, joka on tointunut miekan jäljiltä, koottu monien kansain seasta, - mentävä Israelin vuorille, jotka kauan aikaa olivat olleet rauniomaana; se on tuotu pois kansojen seasta, ja he asuvat turvallisina kaikki tyynni.
\par 9 Sinä hyökkäät kuin rajuilma, tulet kuin pilvi, pettääksesi maan, sinä ja kaikki sotalaumasi ja lukuisat kansat, jotka ovat sinun kanssasi.
\par 10 Näin sanoo Herra, Herra: Mutta sinä päivänä tulee mieleesi jotakin, ja sinä mietit pahan juonen
\par 11 ja sanot: Minä hyökkään suojattomaan maahan, karkaan rauhallisten ihmisten kimppuun, jotka asuvat turvallisina - asuvat muuria vailla kaikki tyynni, ja joilla ei ole salpoja, ei ovia.
\par 12 Sinä aiot saalista saada, ryöstettävää ryöstää, ojentaa kätesi raunioita kohti, jotka on saatu asutuiksi, ja kansaa kohti, joka on koottu pakanakansain seasta, joka on hankkinut karjaa ja omaisuutta ja asuu maan navassa.
\par 13 Saba ja Dedan ja Tarsiin kauppiaat ja kaikki heidän nuoret jalopeuransa kyselevät sinulta: 'Oletko sinä menossa saalista saamaan, ryöstettävää ryöstämään? Oletko koonnut joukkosi kantamaan hopeata ja kultaa, ottamaan karjaa ja omaisuutta, suurta saalista saamaan?'
\par 14 Sentähden ennusta, ihmislapsi, ja sano Googille: Näin sanoo Herra, Herra: Sinä päivänä sinä kyllä huomaat, että minun kansani Israel asuu turvallisena,
\par 15 ja lähdet asuinpaikastasi pohjan periltä, sinä ja sinun kanssasi lukuisat kansat, jotka kaikki ratsastavat hevosilla: suuri joukko, lukuisa sotaväki.
\par 16 Sinä hyökkäät minun kansani Israelin kimppuun kuin pilvi, peittääksesi maan. Päivien lopulla on tämä tapahtuva. Ja minä annan sinun karata maahani, että pakanakansat tulisivat tuntemaan minut, kun minä osoitan pyhyyteni sinussa, Goog, heidän silmäinsä edessä.
\par 17 Näin sanoo Herra, Herra: Etkö se ole sinä, josta minä muinaisina päivinä puhuin palvelijaini, Israelin profeettain, kautta, jotka niinä päivinä ennustivat, vuodesta vuoteen, että minä annan sinun karata heidän kimppuunsa?
\par 18 Mutta sinä päivänä, jona Goog karkaa Israelin maahan, sanoo Herra, Herra, nousee minun vihani hehku.
\par 19 Kiivaudessani, tuimuuteni tulessa minä sanon: Totisesti tulee sinä päivänä suuri maanjäristys Israelin maahan.
\par 20 Ja minun edessäni vapisevat meren kalat ja taivaan linnut ja metsän eläimet ja kaikki maassa liikkuvat matelijat ja kaikki ihmiset, jotka maan pinnalla ovat; ja vuoret luhistuvat, ja vuorenpengermät sortuvat, ja kaikki muurit sortuvat maahan.
\par 21 Ja minä kutsun häntä vastaan kaikille vuorilleni miekan, sanoo Herra, Herra: toisen miekka on oleva toista vastaan.
\par 22 Minä käyn oikeutta hänen kanssansa rutolla ja verellä. Ja minä annan sataa kaatosadetta, raekiviä, tulta ja tulikiveä hänen päällensä, hänen sotalaumojensa päälle ja lukuisain kansojen päälle, jotka hänen kanssansa ovat.
\par 23 Niin minä osoitan suuruuteni ja pyhyyteni sekä teen itseni tunnetuksi lukuisain pakanakansain silmien edessä. Ja he tulevat tietämään, että minä olen Herra."

\chapter{39}

\par 1 "Ja sinä, ihmislapsi, ennusta Googia vastaan ja sano: Näin sanoo Herra, Herra: Katso, minä tulen sinun tykösi, Goog, sinä Roosin, Mesekin ja Tuubalin ruhtinas.
\par 2 Minä kuljetan sinua, talutan sinua, nostatan sinut pohjan periltä ja annan sinun hyökätä Israelin vuorille.
\par 3 Mutta minä lyön pois jousesi sinun vasemmasta kädestäsi ja pudotan nuolesi sinun oikeasta kädestäsi.
\par 4 Israelin vuorille sinä olet kaatuva, samoin kaikki sinun sotalaumasi ja kansat, jotka sinun kanssasi ovat: minä annan sinut petolinnuille, kaikille siivekkäille ja metsän eläimille ruuaksi.
\par 5 Kentälle sinä kaadut. Sillä minä olen puhunut, sanoo Herra, Herra.
\par 6 Ja minä lähetän tulen Maagogiin ja rantamaalla turvassa asuvien keskeen. Ja he tulevat tietämään, että minä olen Herra.
\par 7 Ja pyhän nimeni minä teen tunnetuksi kansani Israelin keskuudessa enkä enää salli häväistävän pyhää nimeäni. Ja pakanakansat tulevat tietämään, että minä olen Herra, Israelin Pyhä.
\par 8 Katso, se tulee, se tapahtuu, sanoo Herra, Herra: Tämä on se päivä, josta minä olen puhunut.
\par 9 Israelin kaupunkien asukkaat menevät ulos ja sytyttävät ja polttavat aseita, pieniä ja suuria kilpiä, jousia, nuolia, käsikarttuja ja keihäitä. He pitävät niitä polttoaineina seitsemän vuotta;
\par 10 eivät he kanna puita kedolta eivätkä hakkaa metsistä, vaan pitävät polttoaineina aseita. Näin he saalistavat saalistajiaan ja ryöstävät ryöstäjiään, sanoo Herra, Herra.
\par 11 Sinä päivänä minä annan Googille siellä, Israelissa, hautasijan, 'Kulkijain laakson', itää kohden merestä; se sulkee tien noilta kulkijoilta. Sinne haudataan Goog ja koko hänen meluisa joukkonsa, ja sille pannaan nimeksi: 'Googin meluisan joukon laakso'.
\par 12 Niitä hautaa Israelin heimo seitsemän kuukautta puhdistaaksensa maan.
\par 13 Koko maan kansa hautaa heitä, ja se on koituva heille kiitokseksi sinä päivänä, jona minä näytän kunniani, sanoo Herra, Herra.
\par 14 Valitaan vakinaiset miehet, jotka kulkevat maata ja hautaavat noita kulkijoita, mitä niitä vielä on jäljellä maan pinnalla, puhdistaaksensa sen; seitsemän kuukauden kuluttua he sen tutkivat läpikotaisin.
\par 15 Kun nämä kulkijat kulkevat maata ja joku heistä näkee ihmisen luut, laittaa hän niitten ääreen kivimerkin, siihen asti että haudankaivajat saavat ne haudatuiksi 'Googin meluisan joukon laaksoon'.
\par 16 Onpa eräällä kaupungillakin nimenä Hamona. - Niin he puhdistavat maan.
\par 17 Ja sinä, ihmislapsi! Näin sanoo Herra, Herra: Käske lintuja, kaikkia siivekkäitä, ja kaikkia metsän eläimiä: Kokoontukaa, tulkaa, yhtykää joka taholta minun teurasuhrilleni jonka minä teitä varten uhraan, suurelle teurasuhrille Israelin vuorille; syökää lihaa ja juokaa verta.
\par 18 Syökää sankarien lihaa ja juokaa maan ruhtinaitten verta - oinaita, lampaita, kauriita ja härkiä, kaikki tyynni Baasanissa syötettyjä.
\par 19 Syökää itsenne kylläisiksi rasvasta ja juokaa itsenne juovuksiin verestä, teurasuhrista, jonka minä teitä varten uhraan.
\par 20 Tulkaa kylläisiksi minun pöydässäni ratsuista ja vaunuhevosista, sankareista ja kaikenkaltaisista sotamiehistä, sanoo Herra, Herra.
\par 21 Minä asetan kunniani pakanakansojen keskeen, ja kaikki pakanakansat saavat nähdä minun tuomioni, jonka minä toimitan, ja käteni, jonka minä lasken heidän päällensä.
\par 22 Mutta Israelin heimo tulee tietämään, että minä, Herra, olen heidän Jumalansa siitä päivästä lähtien ja aina eteenpäin.
\par 23 Ja pakanakansat tulevat tietämään, että Israelin heimo joutui pakkosiirtolaisuuteen syntivelkansa tähden, koska olivat olleet minulle uskottomat, ja minä kätkin heiltä kasvoni, ja niin minä annoin heidät heidän vihollistensa käsiin, ja he kaatuivat miekkaan kaikki tyynni.
\par 24 Minä tein heille heidän saastaisuutensa ja rikkomustensa mukaan ja kätkin heiltä kasvoni.
\par 25 Sentähden, näin sanoo Herra, Herra: Nyt minä käännän Jaakobin kohtalon ja armahdan kaikkea Israelin heimoa ja kiivailen pyhän nimeni puolesta.
\par 26 Ja häpeänsä ja kaiken uskottomuutensa, jota he ovat minulle osoittaneet, he unhottavat, asuessaan nyt maassansa turvallisina, kenenkään peloittelematta.
\par 27 Kun minä tuon heidät takaisin kansojen seasta ja kokoan heidät heidän vihamiestensä maista, niin minä osoitan heissä pyhyyteni lukuisain pakanakansain silmien edessä.
\par 28 Ja he tulevat tietämään, että minä, Herra, olen heidän Jumalansa, kun minä, vietyäni heidät pakkosiirtolaisuuteen pakanakansojen luokse, kokoan heidät omaan maahansa enkä jätä sinne jäljelle ainoatakaan heistä.
\par 29 Enkä minä enää kätke heiltä kasvojani, sillä minä vuodatan Henkeni Israelin heimon päälle, sanoo Herra, Herra."

\chapter{40}

\par 1 Meidän pakkosiirtolaisuutemme kahdentenakymmenentenä viidentenä vuotena, vuoden alussa, kuukauden kymmenentenä päivänä, neljäntenätoista vuotena kaupungin valloituksen jälkeen, juuri sinä päivänä tuli minun päälleni Herran käsi, ja hän vei minut sinne.
\par 2 Jumalan näyissä hän vei minut Israelin maahan ja laski minut hyvin korkealle vuorelle, jolla oli ikäänkuin kaupunki rakennettuna, etelään päin.
\par 3 Hän vei minut sinne. Ja katso, oli mies, näöltänsä vasken kaltainen, ja pellavanuora oli hänellä kädessänsä sekä mittaruoko, ja hän seisoi portilla.
\par 4 Ja mies puhui minulle: "Ihmislapsi, katso silmilläsi, kuule korvillasi ja ota huomioosi kaikki, mitä minä sinulle näytän; sillä sinut on tuotu tänne sitä varten, että nämä sinulle näytettäisiin. Ilmoita kaikki, mitä näet, Israelin heimolle.
\par 5 Ja katso, oli muuri temppelin ulkopuolella, yltympäri sen. Ja miehen kädessä oli mittaruoko, kuusikyynäräinen, kyynärään luettuna kyynärä ja kämmenenleveys. Ja hän mittasi muurirakennuksen leveyden: yksi ruoko, ja korkeuden: yksi ruoko.
\par 6 Sitten hän meni portille, jonka etupuoli oli itää kohden, nousi ylös sen portaita ja mittasi portin kynnyksen: se oli yhden ruovon levyinen - ensimmäinen kynnys: yhden ruovon levyinen.
\par 7 Sivuhuone oli aina yhden ruovon pituinen ja yhden ruovon levyinen, ja sivuhuoneitten väliä oli viisi kyynärää. Portin kynnys porttieteisen laidassa, sisäpuolella: yksi ruoko.
\par 8 Ja hän mittasi porttieteisen, joka oli sisäpuolella: yksi ruoko.
\par 9 Ja hän mittasi porttieteisen: kahdeksan kyynärää, ja sen seinäpatsaat: kaksi kyynärää. Ja porttieteinen oli sisemmällä puolen.
\par 10 Itää kohden antavan portin sivuhuoneita oli kolme kummallakin puolella. Ne kolme olivat yhtä suuria, ja niiden seinäpatsaat kummallakin puolella olivat yhtä suuria.
\par 11 Ja hän mittasi portin oviaukon leveyden: kymmenen kyynärää, ja portin pituuden: kolmetoista kyynärää.
\par 12 Mutta sivuhuoneitten edessä oli aitaus: yksi kyynärä, samoin toisella puolella oleva aitaus: yksi kyynärä; ja kukin sivuhuone kummallakin puolella: kuusi kyynärää.
\par 13 Ja hän mittasi portin, yhden sivuhuoneen katosta toisen kattoon: leveys oli kaksikymmentä viisi kyynärää, ovet vastatusten.
\par 14 Sitten hän otti mitan seinäpatsaista: kaksikymmentä kyynärää. Seinäpatsaisiin liittyi esipiha portin ympärillä.
\par 15 Ja sisäänkäyntiportin edestä sisäportin eteisen eteen oli viisikymmentä kyynärää.
\par 16 Portissa yltympäri oli sisäänpäin, sivuhuoneisiin ja niitten seinäpatsaisiin päin, avartuvia ikkuna-aukkoja; samoin eteisissä. Ikkuna-aukot ympärinsä laajenivat sisäänpäin. Ja seinäpatsaissa oli palmukoristeita.
\par 17 Sitten hän vei minut ulompaan esipihaan. Ja katso: ympäri esipihan oli kammioita ja kivillä laskettu permanto. Kivillä lasketun permannon laidassa oli kolmekymmentä kammiota.
\par 18 Kivillä laskettu permanto oli porttien sivuseinämillä, porttien pituuden mukaisesti; tämä oli alempi kivillä laskettu permanto.
\par 19 Ja hän mittasi leveyden alemman portin edestä sisemmän esipihan edustalle, ulkopuolelle: sata kyynärää sekä itä- että pohjoispuolella.
\par 20 Sitten hän mittasi sen portin pituuden ja leveyden, jonka etupuoli oli pohjoista kohden ja joka vei ulompaan esipihaan.
\par 21 Siinä oli kolme sivuhuonetta kummallakin puolella, ja sen seinäpatsaat ja eteinen olivat yhtä suuret kuin ensimmäisessä portissa. Sen pituus oli viisikymmentä kyynärää ja leveys kaksikymmentä viisi kyynärää.
\par 22 Ja sen ikkuna-aukot, eteinen ja palmukoristeet olivat yhtä suuret kuin portin, jonka etupuoli oli itää kohden, ja sinne noustiin seitsemää porrasta, ja niistä eteenpäin oli sen eteinen.
\par 23 Sisemmällä esipihalla oli portti porttia vastassa pohjoista ja itää kohden. Ja hän mittasi portista porttiin: sata kyynärää.
\par 24 Sitten hän kuljetti minua etelää kohden, ja katso, tuli portti, joka antoi etelää kohden. Ja hän mittasi sen seinäpatsaat ja eteisen: ne olivat yhtä suuret kuin nuo toiset.
\par 25 Ja siinä ja sen eteisessä oli ikkuna-aukkoja ympärinsä, samanlaisia kuin nuo toiset ikkuna-aukot. Pituus oli viisikymmentä kyynärää ja leveys kaksikymmentä viisi kyynärää.
\par 26 Sen ylöskäytävässä oli seitsemän porrasta, ja niistä eteenpäin oli sen eteinen. Ja siinä oli palmukoristeet, yksi kummallakin puolella, sen seinäpatsaissa.
\par 27 Sisemmässä esipihassa oli portti etelää kohden. Ja hän mittasi portista porttiin etelää kohden: sata kyynärää.
\par 28 Sitten hän vei minut sisempään esipihaan eteläportin kautta. Ja hän mittasi eteläportin: se oli yhtä suuri kuin nuo toiset.
\par 29 Ja sen sivuhuoneet, seinäpatsaat ja eteinen olivat yhtä suuret kuin nuo toiset. Ja siinä ja sen eteisessä oli ikkuna-aukkoja ympärinsä. Pituus oli viisikymmentä kyynärää ja leveys kaksikymmentä viisi kyynärää.
\par 30 Ja eteisiä oli ympärinsä; pituus kaksikymmentä viisi kyynärää, leveys viisi kyynärää.
\par 31 Sen eteinen oli ulompaan esipihaan päin, ja sen seinäpatsaissa oli palmukoristeita. Sen portaissa oli kahdeksan porrasta.
\par 32 Sitten hän vei minut sisempään esipihaan itää kohden. Ja hän mittasi portin: se oli yhtä suuri kuin nuo toiset.
\par 33 Ja sen sivuhuoneet, seinäpatsaat ja eteinen olivat yhtä suuret kuin nuo toiset. Ja siinä ja sen eteisessä oli ikkuna-aukkoja ympärinsä. Pituus oli viisikymmentä kyynärää ja leveys kaksikymmentä viisi kyynärää.
\par 34 Sen eteinen oli ulompaan esipihaan päin, ja sen seinäpatsaissa oli palmukoristeita kummallakin puolella. Sen portaissa oli kahdeksan porrasta.
\par 35 Sitten hän vei minut pohjoisportille ja mittasi: yhtä suuret kuin nuo toiset
\par 36 olivat sen sivuhuoneet, seinäpatsaat ja eteinen. Ja ikkuna-aukkoja oli siinä ympärinsä. Pituus oli viisikymmentä kyynärää ja leveys kaksikymmentä viisi kyynärää.
\par 37 Sen eteinen oli ulompaan esipihaan päin, ja sen seinäpatsaissa oli palmukoristeita kummallakin puolella. Sen portaissa oli kahdeksan porrasta.
\par 38 Ja oli kammio, jonka ovi oli seinäpatsaissa, porteissa. Siellä oli polttouhri huuhdottava.
\par 39 Ja porttieteisessä oli kaksi pöytää kummallakin puolella; niillä oli polttouhri, syntiuhri ja vikauhri teurastettava.
\par 40 Ja ulkopuolisella sivuseinämällä, portin ovelle noustessa, pohjoispuolella, oli kaksi pöytää; samoin porttieteisen toisella sivuseinämällä kaksi pöytää:
\par 41 neljä pöytää portin sivuseinämillä kummallakin puolella, siis kahdeksan pöytää. Niillä oli teurastettava.
\par 42 Ja ylöskäytävän ääressä oli neljä pöytää hakatuista kivistä. Ne olivat puolentoista kyynärän pituisia, puolentoista kyynärän levyisiä ja yhden kyynärän korkuisia. Niille oli laskettava ne aseet, joilla teurastettiin polttouhri ja teurasuhri.
\par 43 Ja haarukkakoukkuja, kämmenenleveyttä pitkiä, oli kiinnitetty ympäri huonetta, mutta uhriliha pantiin pöydille.
\par 44 Sisemmän portin ulkopuolella oli kaksi kammiota sisemmässä esipihassa, toinen pohjoisen portin sivuseinämällä, etupuoli etelää kohden, ja toinen eteläisen portin sivuseinämällä, etupuoli pohjoista kohden.
\par 45 Ja hän puhui minulle: "Tämä kammio, jonka etupuoli on etelää kohden, on pappeja varten, jotka hoitavat temppelitehtäviä.
\par 46 Ja kammio, jonka etupuoli on pohjoista kohden, on pappeja varten, jotka hoitavat alttaritehtäviä. Ne ovat Saadokin jälkeläiset, ne leeviläisistä, jotka saavat lähestyä Herraa häntä palvellakseen."
\par 47 Ja hän mittasi esipihan: pituus oli sata kyynärää ja leveys sata kyynärää; se oli neliön muotoinen. Ja alttari oli temppelin edustalla.
\par 48 Sitten hän vei minut temppelin eteiseen ja mittasi eteisen seinäpatsaan: viisi kyynärää kummallakin puolella, ja portin leveyden: kolme kyynärää kummallakin puolella.
\par 49 Eteisen pituus oli kaksikymmentä kyynärää ja leveys yksitoista kyynärää. Sinne noustiin kymmentä porrasta. Ja seinäpatsaitten ääressä oli pylväät, yksi kummallakin puolella.

\chapter{41}

\par 1 Sitten hän vei minut temppelisaliin ja mittasi seinäpatsaat; ne olivat kuutta kyynärää leveitä kummallakin puolella: ilmestysmajan leveys.
\par 2 Ja oviaukko oli kymmentä kyynärää leveä, ja oviaukon sivuseinät kummallakin puolella olivat viisikyynäräiset. Hän mittasi temppelisalin pituuden: neljäkymmentä kyynärää, ja leveyden: kaksikymmentä kyynärää.
\par 3 Sitten hän meni sisimpään ja mittasi oviaukon seinäpatsaan: kaksi kyynärää, oviaukon: kuusi kyynärää, ja oviaukon leveyden: seitsemän kyynärää.
\par 4 Ja hän mittasi sen pituuden: kaksikymmentä kyynärää, ja leveyden: kaksikymmentä kyynärää temppelisalin puolelta. Ja hän sanoi minulle: "Tämä on kaikkeinpyhin".
\par 5 Sitten hän mittasi temppelin seinän: kuusi kyynärää, ja sivukammion leveyden: neljä kyynärää, yltympäri temppelin.
\par 6 Ja sivukammioita oli: kolme sivukammiota päällekkäin, kolmeenkymmeneen kertaan; ja ne kulkivat ympärinsä sen temppeliseinän varassa, joka oli sivukammioihin päin, niin että pysyivät siinä kiinni, mutta ne eivät olleet kiinnitetyt temppelinseinän sisään.
\par 7 Ja sivukammiot olivat aina sitä leveämpiä kiertäessään, mitä ylempänä olivat, sillä temppelin ympäryskammiot ulottuivat ylemmäksi ja ylemmäksi ympäri temppelin, ollen siten ylhäällä leveämmät sisäänpäin. Ja alakerrasta noustiin yläkertaan keskikerroksen kautta.
\par 8 Ja minä näin, että temppelillä oli koroke ympärinsä: sivukammioitten perustukset, koko ruovon korkuiset, kuusi kyynärää, reunustaan asti.
\par 9 Sivukammioitten ulkoseinän paksuus oli viisi kyynärää; yhtä leveä oli avoin tila sivukammiorakennuksen ääressä, joka temppeliin kuului.
\par 10 Välimatkaa pihakammiorakennuksiin oli kahtakymmentä kyynärää leveälti ympäri temppelin - yltympäri.
\par 11 Ja sivukammioista vei ovi avoimeen tilaan: yksi ovi pohjoista kohden ja toinen ovi etelää kohden. Ja avoimen tilan leveys oli viisi kyynärää yltympäri.
\par 12 Ja rakennus, joka oli eristetyn alueen laidassa lännen puoleisella sivulla, oli seitsemääkymmentä kyynärää leveä. Rakennuksen seinän paksuus oli viisi kyynärää, ympärinsä, ja sen pituus oli yhdeksänkymmentä kyynärää.
\par 13 Sitten hän mittasi temppelin: pituus oli sata kyynärää, sekä eristetyn alueen ja rakennuksen seininensä: pituus sata kyynärää.
\par 14 Temppelin etupuolen ja eristetyn alueen leveys itään päin oli sata kyynärää.
\par 15 Sitten hän mittasi sen rakennuksen pituuden, joka oli eristetyn alueen laidassa, sen takana, ynnä sen käytävät, kummallakin puolella: sata kyynärää; ja sisemmän temppelisalin, esipihan eteiset,
\par 16 kynnykset, sisäänpäin avartuvat ikkuna-aukot, käytävät niitten kolmen ympärillä - kynnyksen edessä oli puulaudoitus yltympäri - ja välin maasta ikkuna-aukkoihin. Ja ikkuna-aukot olivat peitetyt.
\par 17 Oviaukkojen yläpuolelta lähtien sekä sisempään temppeliin asti että ulos asti, koko seinällä yltympäri, sisä- ja ulkopuolella, oli mitatut alat,
\par 18 ja niihin oli tehty kerubeja ja palmuja, palmu aina kahden kerubin väliin. Ja kullakin kerubilla oli kahdet kasvot:
\par 19 ihmisen kasvot palmuun päin toisaalle ja leijonan kasvot palmuun päin toisaalle. Niin oli tehty koko temppelissä yltympäri:
\par 20 maasta alkaen ja aina oviaukkojen yläpuolelle oli tehty kerubeja ja palmuja, samoin temppelisalin seinälle.
\par 21 Temppelisalissa oli nelikulmaiset ovenpielet. Mutta pyhimmän edessä näkyi
\par 22 ikäänkuin alttari, puinen, kolmea kyynärää korkea, ja sen pituus oli kaksi kyynärää. Siinä oli nurkkaukset, ja sen jalusta ja seinät olivat puuta. Ja hän sanoi minulle: "Tämä on pöytä, joka on Herran edessä".
\par 23 Sekä temppelisalissa että pyhimmässä oli kaksoisovet.
\par 24 Ja ovissa oli kaksi ovipuoliskoa, kaksi kääntyvää ovipuoliskoa: kumpaisessakin ovessa oli kaksi ovipuoliskoa.
\par 25 Ja niihin, temppelisalin oviin, oli tehty kerubeja ja palmuja, samoin kuin oli tehty seiniin. Ulkona, eteisen edessä, oli puinen porraskatos.
\par 26 Ja eteisen sivuseinissä, kummallakin puolella, oli sisäänpäin avartuvia ikkuna-aukkoja ja palmuja; samoin temppelin sivukammioissa ja porraskatoksissa.

\chapter{42}

\par 1 Sitten hän vei minut ulompaan esipihaan pohjoista kohden viepää tietä myöten, vei minut pihakammio-rakennuksen luo, joka oli eristettyä aluetta vastassa ja pohjoispuolista muurirakennusta vastassa,
\par 2 satakyynäräisen pitkänsivun edustalle, jossa oli oviaukko pohjoiseen päin; leveys taas oli viisikymmentä kyynärää.
\par 3 Sen kaksikymmenkyynäräisen paikan edustalla, joka kuului sisempään esipihaan, ja vastapäätä sitä kivillä laskettua permantoa, joka oli ulommassa esipihassa, oli käytävä käytävää vastassa, kolmessa kerroksessa.
\par 4 Kammioiden editse meni kulkutie, kymmentä kyynärää leveä, sisempään esipihaan, sataa kyynärää pitkä, ja oviaukot olivat pohjoiseen päin.
\par 5 Mutta ylimmät kammiot olivat kapeimmat, sillä käytävät ottivat niiltä tilaa enemmän kuin alimmilta ja keskimmäisiltä kammioilta rakennuksessa.
\par 6 Sillä nämä olivat kolmikertaiset, mutta niissä ei ollut pylväitä, niinkuin esipihoissa oli pylväät; sentähden ne tulivat olemaan kaitaisemmat kuin alimmat ja keskimmäiset, maasta lukien.
\par 7 Oli myös muuri, joka ulkopuolella kulki yhtä suuntaa kammioiden kanssa ulompaa esipihaa kohden kammioiden editse; sen pituus oli viisikymmentä kyynärää.
\par 8 Sillä niiden kammioiden pituus, jotka olivat ulompaan esipihaan päin, oli yhteensä viisikymmentä kyynärää; mutta, katso, temppelisalin puolella sata kyynärää.
\par 9 Ja näiden kammioiden alapuolella oli sisäänkäynti idästäpäin, kun tultiin niitten luokse ulommasta esipihasta,
\par 10 esipihaa ympäröivän muurin leveyspuolella. Etelää kohden, eristetyn alan edustalla ja rakennuksen edustalla, oli kammioita,
\par 11 ja niiden editse kävi tie. Ne olivat samanmuotoisia kuin ne kammiot, jotka olivat pohjoista kohden: yhtä pitkiä ja leveitä. Samanlaisia olivat kaikki näitten uloskäytävät, laitteet ja oviaukot.
\par 12 Ja samoin kuin olivat niiden kammioiden oviaukot, jotka olivat etelää kohden, oli se oviaukko, joka oli tien päässä - tien, joka kulki senpuolisen muurin viertä itää kohden, sinne tultaessa.
\par 13 Ja hän sanoi minulle: "Pohjoiskammiot ja eteläkammiot, jotka ovat eristetyn alan edustalla, ne ovat pyhiä kammioita, joissa papit, jotka lähestyvät Herraa, syövät sitä, mikä on korkeasti-pyhää. Sinne he pankoot korkeasti-pyhän: ruokauhrin, syntiuhrin ja vikauhrin; sillä se paikka on pyhä.
\par 14 Kun papit ovat tulleet, älkööt he menkö pyhäköstä ulos esipihaan muutoin, kuin että tänne panevat vaatteensa, joissa toimittavat virkaansa, sillä ne ovat pyhät, ja pukeutuvat toisiin vaatteisiin; sitten lähestykööt paikkaa, mikä on kansaa varten."
\par 15 Kun hän oli saanut loppuun sisemmän temppelin mittaamisen, vei hän minut ulos sille portille, jonka etupuoli oli itää kohden. Ja hän mittasi sen ympärinsä.
\par 16 Hän mittasi itäilmalta mittaruovolla: viisi sataa ruokoa mittaruovon mukaan. Hän kääntyi
\par 17 ja mittasi pohjoisilmalta: viisisataa ruokoa mittaruovon mukaan. Hän kääntyi
\par 18 ja mittasi eteläilmalta: viisisataa ruokoa mittaruovon mukaan.
\par 19 Hän kääntyi ja mittasi länsi-ilmalta: viisisataa ruokoa mittaruovon mukaan.
\par 20 Neljältä ilmansuunnalta hän sen mittasi. Siinä oli muuri yltympäri; pituutta viisisataa ja leveyttä viisisataa. Sen tuli erottaa pyhä epäpyhästä.

\chapter{43}

\par 1 Sitten hän kuljetti minut portille - sille portille, joka antoi itää kohden.
\par 2 Ja katso: Israelin Jumalan kunnia tuli idästä päin. Sen kohina oli niinkuin paljojen vetten kohina, ja maa kirkastui hänen kunniastansa.
\par 3 Ja näky, jonka minä näin, oli samanlainen kuin se näky, jonka olin nähnyt tullessani hävittämään kaupunkia; samanlaiset olivat näyt kuin se näky, jonka olin nähnyt Kebar-joen varrella. Niin minä lankesin kasvoilleni.
\par 4 Ja Herran kunnia kulki temppeliin sen portin kautta, jonka etupuoli oli itää kohden.
\par 5 Ja henki nosti minut ja vei minut sisempään esipihaan. Ja katso: Herran kunnia täytti temppelin.
\par 6 Ja minä kuulin jonkun puhuttelevan minua temppelistä; mutta mies seisoi minun vieressäni.
\par 7 Se ääni sanoi minulle: "Ihmislapsi, tämä on minun valtaistuimeni sija ja minun jalkapohjaini sija, jossa minä tahdon asua israelilaisten keskellä iankaikkisesti. Ja Israelin heimo ei ole enää saastuttava minun pyhää nimeäni - eivät he eivätkä heidän kuninkaansa - haureudellansa, kuninkaittensa ruumiilla ja uhrikukkuloillansa.
\par 8 Kun he asettivat kynnyksensä minun kynnykseni ääreen, ovenpielensä minun ovenpielteni ääreen, niin että muuri vain oli minun ja heidän välillään, niin he saastuttivat minun pyhän nimeni kauhistuksillansa, joita harjoittivat, ja minä lopetin heidät vihassani.
\par 9 Nyt he vievät haureutensa ja kuninkaittensa ruumiit kauas minusta, ja minä asun heidän keskellänsä iankaikkisesti.
\par 10 Sinä, ihmislapsi, saata Israelin heimolle sanoma tästä temppelistä, että he häpeäisivät rikkomuksiansa; ja he mitatkoot sen sopusuhtaisuuden.
\par 11 Ja jos he häpeävät kaikkea, mitä ovat tehneet, niin tee heille tiettäväksi temppelin muoto sekä sen sisustus, uloskäytävät, sisäänkäytävät, kaikki muodot, kaikki säädökset, kaikki muodot ja kaikki lait; kirjoita ne heidän silmiensä eteen, että he ottaisivat vaarin sen koko muodosta ja kaikista säädöksistä ja tekisivät niiden mukaan.
\par 12 Tämä on temppelin laki: sen koko alue vuoren laella yltympäri on korkeastipyhä. Katso, tämä on temppelin laki."
\par 13 Ja nämä olivat alttarin mitat kyynärissä, kyynärään luettuna kyynärä ja kämmenenleveys: sen alusta oli kyynärän korkuinen ja kyynärän levyinen, ja sen reunusta ympäri sen laidan oli yhden vaaksan mittainen. Tämä oli alttarin koroke.
\par 14 Alustasta, joka oli maassa, alimmalle välireunalle: kaksi kyynärää sekä leveyttä yksi kyynärä; ja pienemmältä välireunalta suuremmalle välireunalle: neljä kyynärää sekä leveyttä kyynärä.
\par 15 Jumalan lieteen oli neljä kyynärää, ja Jumalan liedestä ylöspäin olivat ne neljä sarvea.
\par 16 Jumalan lieden pituus oli kaksitoista kyynärää ja leveys kaksitoista, niin että sen neljästä sivusta tuli neliö.
\par 17 Välireunan pituus oli neljätoista kyynärää ja leveys neljätoista, sen neljältä sivulta. Sitä ympäröivä reunusta oli puolikyynäräinen, ja sen alusta oli kyynäräinen ympärinsä. Ja sen portaat olivat itäisellä puolella.
\par 18 Ja hän sanoi minulle: "Ihmislapsi, näin sanoo Herra, Herra: Nämä ovat alttaria koskevat säädökset: Sinä päivänä, jona se on saatu tehdyksi, niin että voidaan sillä uhrata polttouhria ja vihmoa sille verta,
\par 19 anna papeille, leeviläisille, jotka ovat Saadokin jälkeläisiä ja saavat minua lähestyä palvellaksensa minua, sanoo Herra, Herra, mullikka syntiuhriksi.
\par 20 Ota sen verta ja sivele alttarin neljään sarveen ja välireunan neljään kulmaan ja reunustaan yltympäri; niin on sinun puhdistettava se ja toimitettava sen sovitus.
\par 21 Ota sitten syntiuhri-mullikka ja polta se säädetyssä, temppelille kuuluvassa paikassa, pyhäkön ulkopuolella.
\par 22 Tuo toisena päivänä virheetön kauris syntiuhriksi; alttari on puhdistettava, samoin kuin se puhdistettiin mullikalla.
\par 23 Suoritettuasi loppuun puhdistuksen tuo virheetön mullikka ja virheetön oinas pikkukarjasta.
\par 24 Tuo ne Herran eteen, ja papit heittäkööt niiden päälle suolaa ja uhratkoot ne polttouhriksi Herralle.
\par 25 Uhraa joka päivä, seitsemänä päivänä, syntiuhri-kauris; myös on uhrattava mullikka ja oinas pikkukarjasta, virheettömät.
\par 26 Seitsemänä päivänä toimitettakoon alttarin sovitus ja puhdistettakoon ja vihittäköön se.
\par 27 Niiden päivien kuluttua papit kahdeksantena päivänä ja aina edelleen uhratkoot alttarilla polttouhrejansa ja yhteysuhrejansa. Ja niin minä teihin mielistyn, sanoo Herra, Herra."

\chapter{44}

\par 1 Sitten hän vei minut takaisin pyhäkön ulommalle portille, sille, joka antoi itään päin; mutta se oli suljettu.
\par 2 Niin Herra sanoi minulle: "Tämä portti on oleva suljettuna: sitä ei saa avata, eikä kenkään saa käydä siitä sisälle, sillä Herra, Israelin Jumala, on käynyt siitä sisälle. Se on oleva suljettu.
\par 3 Ruhtinas ainoastaan saakoon ruhtinaana istua siellä ja aterioida Herran edessä; käyköön hän sisään porttieteisen kautta ja tulkoon ulos samaa tietä."
\par 4 Sitten hän vei minut pohjoisportin kautta temppelin edustalle, ja minä näin, ja katso: Herran kunnia täytti Herran temppelin. Niin minä lankesin kasvoilleni.
\par 5 Ja Herra sanoi minulle: "Ihmislapsi, ota huomioosi, katso silmilläsi ja kuule korvillasi kaikki, mitä minä sinulle puhun kaikista Herran temppelin säädöksistä ja kaikista sen laeista, ja ota huomioosi, kuinka temppeliin on mentävä kustakin pyhäkön uloskäytävästä.
\par 6 Ja sano niille uppiniskaisille, sano Israelin heimolle: Näin sanoo Herra, Herra: Jo riittävät kaikki teidän kauhistuksenne, te Israelin heimo,
\par 7 te kun olette tuoneet muukalaisia, sydämeltään ja ruumiiltaan ympärileikkaamattomia, minun pyhäkkööni olemaan siellä ja häpäisemään minun temppeliäni, samalla kuin olette uhranneet minun leipääni ja rasvaa ja verta; ja niin te olette rikkoneet minun liittoni - kaikkien muitten kauhistustenne lisäksi.
\par 8 Te ette ole itse hoitaneet minulle suoritettavia pyhiä tehtäviä, vaan olette panneet heitä puolestanne hoitamaan minulle pyhäkössäni suoritettavia tehtäviä.
\par 9 Näin sanoo Herra, Herra: ei yksikään muukalainen, sydämeltään ja ruumiiltaan ympärileikkaamaton, saa tulla minun pyhäkkööni, olkoon se kuka tahansa muukalainen, joka israelilaisten keskuudessa on.
\par 10 Vaan ne leeviläiset, jotka menivät minun luotani kauas, silloin kun Israel oli joutunut eksyksiin, nuo, jotka eksyivät pois minusta kivijumalainsa jälkeen, kantakoot syntinsä;
\par 11 he toimittakoot minun pyhäkössäni vartioinnin temppelin porteilla ja toimittakoot temppelipalveluksen: he teurastakoot polttouhrit ja teurasuhrit kansalle ja seisokoot heidän edessään heitä palvelemassa.
\par 12 Koska he ovat heitä palvelleet heidän kivijumalainsa edessä ja olleet Israelin heimolle kompastuksena syntiin, sentähden minä kättä kohottaen vakuutan heistä, sanoo Herra, Herra, että heidän on kannettava syntinsä.
\par 13 He eivät saa lähestyä minua, niin että toimittaisivat minun edessäni papinvirkaa ja lähestyisivät mitään minulle pyhitettyä, mitään korkeasti-pyhää, vaan heidän on kannettava häpeänsä ja kauhistuksensa, joita ovat tehneet.
\par 14 Ja minä panen heidät hoitamaan temppelissä suoritettavia tehtäviä, mitä siellä vain on palvelijantyötä ja toimitettavaa.
\par 15 Mutta ne leeviläiset papit, Saadokin jälkeläiset, jotka hoitivat minun pyhäkössäni suoritettavat tehtävät silloin, kun israelilaiset olivat eksyneet minusta pois, he saavat lähestyä minua, palvella minua ja seisoa minun edessäni uhraamassa minulle rasvaa ja verta, sanoo Herra, Herra.
\par 16 He saavat tulla minun pyhäkkööni ja saavat lähestyä minun pöytääni, palvellen minua ja hoitaen minulle suoritettavat tehtävät.
\par 17 Ja tullessaan sisemmän esipihan porteille he pukeutukoot pellavavaatteisiin älköötkä pitäkö yllänsä villaista toimittaessaan virkaansa sisemmän esipihan porteissa ja sisällä temppelissä.
\par 18 Pellavaiset juhlapäähineet olkoon heillä päässä ja pellavakaatiot lanteilla. Älkööt he panko vyötäisillensä mitään hiostuttavaa.
\par 19 Kun he sitten menevät pois ulompaan esipihaan, kansan tykö ulompaan esipihaan, riisukoot vaatteensa, joissa ovat virkaa toimittaneet, jättäkööt ne pyhäkön kammioihin ja pukekoot yllensä toiset vaatteet, etteivät pyhittäisi kansaa vaatteillansa.
\par 20 Älkööt he ajelko päätänsä, mutta älkööt myöskään antako tukkansa valtoinaan kasvaa: heidän on leikattava hiuksensa.
\par 21 Viiniä älköön kukaan pappi juoko tultuansa sisempään esipihaan.
\par 22 Leskeä ja hyljättyä älkööt he ottako vaimoksensa, vaan ottakoot neitsyen Israelin heimon jälkeläisistä tai lesken, joka on leskenä papin jälkeen.
\par 23 Ja opettakoot he minun kansalleni erotuksen pyhän ja epäpyhän välillä sekä tehkööt heille tiettäväksi, mikä on saastaista, mikä puhdasta.
\par 24 Riita-asioissa he seisokoot tuomitsemassa ja tuomitkoot niissä minun oikeuksieni mukaan. Minun lakejani ja käskyjäni he noudattakoot kaikkien minun juhlieni vietossa ja pyhittäkööt minun sapattini.
\par 25 Älköön hän menkö kuolleen ihmisen luo ja siten tulko saastaiseksi; ainoastaan isästä ja äidistä, pojasta ja tyttärestä, veljestä ja sisaresta, jos tämä ei ollut naitu, hän saa näin saattaa itsensä saastaiseksi.
\par 26 Mutta puhdistuksensa jälkeen luettakoon hänelle vielä seitsemän päivää;
\par 27 ja sinä päivänä, jona hän tulee pyhäkköön, sisempään esipihaan, toimittamaan virkaansa pyhäkössä, hän uhratkoon syntiuhrinsa, sanoo Herra, Herra.
\par 28 Ja tämä on oleva heidän perintöosansa: minä olen heidän perintöosansa. Teidän ei tule antaa heille perintömaata Israelissa: minä olen heidän perintömaansa.
\par 29 Ruokauhrin, syntiuhrin ja vikauhrin he saavat syödä, ja kaikki, mikä on tuhon omaksi vihittyä Israelissa, on tuleva heille.
\par 30 Paras kaikesta uutisesta, kaikenlaatuisesta, ja kaikki annit, mitä laatua hyvänsä antinne ovatkin, tulevat papeille; ja parhaat jyvärouheistanne on teidän annettava papille, että sinä tuottaisit siunauksen huoneellesi.
\par 31 Papit älkööt syökö mitään itsestään kuollutta tai kuoliaaksi raadeltua, ei lintua eikä raavasta."

\chapter{45}

\par 1 "Kun te arvotte maata perintöosiksi, niin antakaa Herralle anti, pyhä osa maasta, kahtakymmentäviittä tuhatta pitkä ja kahtakymmentä tuhatta leveä. Se olkoon pyhä koko alueeltaan, yltympäri.
\par 2 Siitä tulkoon pyhäkölle neliö, viisisataa ja viisisataa joka taholta, ja sen lisäksi avointa tilaa viisikymmentä joka taholta.
\par 3 Ja mittaa tästä mitatusta alueesta kahtakymmentäviittä tuhatta pitkälti ja kymmentätuhatta leveälti, ja siinä olkoon pyhäkkö, korkeasti-pyhä.
\par 4 Se olkoon pyhä osa maasta papeille, jotka palvelevat pyhäkössä, jotka lähestyvät ja palvelevat Herraa; se olkoon talojen sijana heitä varten ja pyhänä paikkana pyhäkköä varten.
\par 5 Kahtakymmentäviittä tuhatta pitkälti ja kymmentätuhatta leveälti on tuleva leeviläisille, jotka toimittavat palvelusta pyhäkössä, heille perintömaaksi - kaksikymmentä kammiota -.
\par 6 Kaupungille antakaa perintömaata viittätuhatta leveälti ja kahtakymmentäviittä tuhatta pitkälti, pyhän antimaan viereltä; se olkoon kaiken Israelin heimon oma.
\par 7 Ja ruhtinaalle: molemmilta puolin pyhää antimaata ja kaupungin perintömaata, pyhän antimaan sivusta ja kaupungin perintömaan sivusta, lännen puolelta länteen päin ja idän puolelta itään päin, maata yhtä pitkälti kuin on yhden sukukuntaosan pituus länsirajasta itärajaan.
\par 8 Se olkoon hänellä maana, perintömaana Israelissa; ja älkööt minun ruhtinaani enää sortako minun kansaani, vaan jättäkööt muun maan Israelin heimolle sukukunnittain.
\par 9 Näin sanoo Herra, Herra: Jo riittää, te Israelin ruhtinaat! Heittäkää pois väkivalta ja sorto, noudattakaa oikeutta ja vanhurskautta, heretkää häätämästä mailtansa minun kansaani, sanoo Herra, Herra.
\par 10 Olkoon teillä oikea paino, oikea eefa-mitta ja oikea bat-mitta.
\par 11 Eefa- ja bat-mitta olkoot tarkoin yhtä suuret: bat vetäköön kymmenenneksen hoomer-mittaa ja eefa kymmenenneksen hoomer-mittaa; hoomerin mukaan määrättäköön kumpaisenkin suuruus.
\par 12 Sekelissä olkoon kaksikymmentä geeraa; miina olkoon teillä kaksikymmentä sekeliä, kaksikymmentäviisi sekeliä, viisitoista sekeliä.
\par 13 Tämä on anti, joka teidän on annettava: kuudennes eefaa nisuhoomerista ja kuudennes eefaa ohrahoomerista;
\par 14 öljyn määrä bat-mitasta öljyä: kymmenennes batia koor-mitasta - kymmenestä batista, hoomerista - sillä hoomer on kymmenen batia;
\par 15 pikkukarjasta: kahdestasadasta yksi lammas Israelin runsasvetisiltä laitumilta ruokauhriksi, polttouhriksi ja yhteysuhriksi, että voitaisiin toimittaa heille sovitus, sanoo Herra, Herra.
\par 16 Kaikki maan kansa on velvoitettu tähän antiin Israelin ruhtinaalle.
\par 17 Mutta ruhtinas toimittakoon polttouhrit, ruokauhrin ja juomauhrin juhlina, uusinakuina ja sapatteina, kaikkina Israelin heimon juhla-aikoina; hän uhratkoon syntiuhrin, ruokauhrin, polttouhrin ja yhteysuhrin toimittaakseen sovituksen Israelin heimolle.
\par 18 Näin sanoo Herra, Herra: Ota ensimmäisessä kuussa, kuukauden ensimmäisenä päivänä virheetön mullikka ja puhdista pyhäkkö.
\par 19 Ja pappi ottakoon syntiuhrin verta ja sivelköön sitä temppelin ovenpieliin ja alttarin välireunan neljään kulmaan ja sisemmän esipihan portin ovenpieliin.
\par 20 Samoin tee kuukauden seitsemäntenä päivänä sen varalta, että joku olisi erehdyksestä tai tietämättömyydestä rikkonut. Ja niin te toimitatte temppelin sovituksen.
\par 21 Ensimmäisessä kuussa, kuukauden neljäntenätoista päivänä, on teillä pääsiäinen. Se on seitsenpäiväinen juhla. Syötäköön silloin happamatonta leipää.
\par 22 Sinä päivänä uhratkoon ruhtinas omasta puolestansa ja maan kaiken kansan puolesta mullikan syntiuhriksi.
\par 23 Ja seitsemänä juhlapäivänä hän uhratkoon polttouhriksi Herralle seitsemän mullikkaa ja seitsemän oinasta, virheettömiä, joka päivä seitsemänä päivänä, ja syntiuhriksi kauriin joka päivä.
\par 24 Ja ruokauhriksi hän uhratkoon eefa-mitan mullikkaa ja eefa-mitan oinasta kohti sekä öljyä hiin-mitan eefaa kohti.
\par 25 Seitsemännessä kuussa, kuukauden viidentenätoista päivänä, juhlana, hän uhratkoon samalla tavoin - niin seitsemänä päivänä - samankaltaisen syntiuhrin, polttouhrin ja ruokauhrin sekä saman määrän öljyä."

\chapter{46}

\par 1 "Näin sanoo Herra, Herra: Sisemmän esipihan portti, se, joka antaa itään päin, olkoon suljettuna kuusi työpäivää. Mutta sapatinpäivänä se avattakoon; myös avattakoon se uudenkuun päivänä.
\par 2 Silloin tulkoon ruhtinas porttieteisen kautta ulkoa ja asettukoon portin ovenpieleen; ja kun papit uhraavat hänen polttouhriansa ja yhteysuhriansa, niin hän kumartaen rukoilkoon portin kynnyksellä ja menköön sitten ulos. Mutta porttia älköön suljettako ennen iltaa.
\par 3 Ja maan kansa kumartaen rukoilkoon sen portin ovella sapatteina ja uusinakuina Herran edessä.
\par 4 Ja polttouhrina, joka ruhtinaan on uhrattava Herralle sapatinpäivänä, olkoon: kuusi virheetöntä karitsaa ja virheetön oinas;
\par 5 ja ruokauhrina: eefa-mitta oinasta kohti, mutta ruokauhrina karitsoita kohti se, mitä hän voi ja tahtoo antaa, ynnä hiin-mitta öljyä eefaa kohti.
\par 6 Uudenkuun päivänä olkoon polttouhrina: virheetön mullikka, kuusi karitsaa ja oinas, virheettömiä;
\par 7 ja ruokauhrina hän uhratkoon eefan mullikkaa kohti ja eefan oinasta kohti sekä karitsoita kohti sen, mitä hän saa hankituksi, ynnä öljyä hiin-mitan eefaa kohti.
\par 8 Ja kun ruhtinas tulee, niin tulkoon porttieteisen kautta ja menköön ulos samaa tietä.
\par 9 Mutta kun maan kansa tulee juhlina Herran eteen, niin se, joka pohjoisportin kautta tuli kumartaen rukoilemaan, menköön ulos eteläportin kautta, ja joka tuli eteläportin kautta, se menköön ulos pohjoisportin kautta; älköön kenkään palatko sen portin kautta, josta tuli, vaan menköön ulos vastakkaisesta.
\par 10 Ja ruhtinas tulkoon heidän joukossansa, kun he tulevat, ja menköön ulos, kun he menevät.
\par 11 Juhlina ja juhla-aikoina olkoon ruokauhri: eefa mullikkaa kohti ja eefa oinasta kohti sekä karitsoita kohti se, mitä mikin voi ja tahtoo antaa, ynnä öljyä hiin-mitta eefaa kohti.
\par 12 Milloin ruhtinas uhraa vapaaehtoisia lahjoja, polttouhrin tai yhteysuhrin vapaaehtoisena lahjana Herralle, avattakoon hänelle portti, joka antaa itään päin, ja hän uhratkoon polttouhrinsa sekä yhteysuhrinsa samoin, kuin hän uhraa sapatinpäivänä, ja menköön ulos; ja hänen mentyänsä suljettakoon portti.
\par 13 Uhraa vuoden vanha virheetön karitsa joka päivä polttouhriksi Herralle: uhraa se joka aamu.
\par 14 Ja sen lisäksi uhraa ruokauhriksi joka aamu kuudennes eefaa ynnä öljyä kolmannes hiin-mittaa lestyjen jauhojen kostuttamiseksi. Tämä on ruokauhri Herralle - ikuinen, pysyvä säädös.
\par 15 Niin uhratkaa joka aamu karitsa, ruokauhri ja öljy jokapäiväiseksi polttouhriksi.
\par 16 Näin sanoo Herra, Herra: Jos ruhtinas antaa jollekin pojistansa lahjan, on se tämän perintöosa. Se on tuleva hänen pojillensa: se on perintöosana heidän omaisuuttansa.
\par 17 Mutta jos hän antaa lahjan perintöosastaan jollekin palvelijoistansa, olkoon se tämän omana vapautusvuoteen saakka, mutta sitten tulkoon takaisin ruhtinaalle: sehän on hänen perintöosaansa ja on tuleva hänen pojilleen.
\par 18 Älköönkä ruhtinas ottako kansan perintöosia, niin että sortaisi heitä pois heidän perintömaaltansa. Omasta perintömaastaan hän antakoon perintöosia pojillensa, ettei kukaan minun kansastani tulisi häädetyksi pois omalta perintömaaltansa."
\par 19 Sitten hän vei minut siitä sisäänkäytävästä, joka oli portin sivuseinämällä, niiden kammioiden luo, jotka olivat pyhitettyjä papeille ja jotka antoivat pohjoiseen päin. Ja katso, siellä oli eräs paikka kauimpana länttä kohti.
\par 20 Ja hän sanoi minulle: "Tämä on paikka, missä pappien on keitettävä vikauhri ja syntiuhri sekä leivottava ruokauhri, etteivät veisi ulos sitä ulompaan esipihaan ja siten tulisi pyhittäneeksi kansaa".
\par 21 Sitten hän vei minut ulompaan esipihaan ja johdatti minut esipihan neljän nurkkauksen ohitse, ja katso, esipihan joka nurkkauksessa oli piha.
\par 22 Esipihan neljässä nurkkauksessa oli suljetut pihat, neljänkymmenen pituiset ja kolmenkymmenen levyiset; nämä neljä nurkka-alaa olivat yhtä suuret.
\par 23 Ja niissä neljässä oli ympärinsä kivikehä, ja alas kivikehään oli tehty keittoliesiä ympärinsä.
\par 24 Ja hän sanoi minulle: "Nämä ovat keittäjäin suojat, joissa temppelipalvelijat keittävät kansan teurasuhrit".

\chapter{47}

\par 1 Sitten hän vei minut takaisin temppelin ovelle. Ja katso, vettä kumpusi temppelin kynnyksen alta itään päin, sillä temppelin etusivu oli itää kohti. Ja vesi juoksi alas temppelin oikeanpuolisen sivuseinämän alitse, alttarin eteläpuolitse.
\par 2 Sitten hän toi minut ulos pohjoisportin kautta ja kierrätti minut ulkopuolitse ulkoportille, joka antoi itää kohden; ja katso, vesi virtasi oikeanpuoliselta sivuseinämältä päin.
\par 3 Mennessänsä itää kohti mies, mittanuora kädessään, mittasi tuhat kyynärää ja antoi minun käydä veden poikki: vettä oli nilkkoihin asti.
\par 4 Sitten hän mittasi tuhat ja antoi minun käydä veden poikki: vettä oli polviin asti. Sitten hän mittasi tuhat ja antoi minun käydä poikki: vettä oli lanteisiin asti.
\par 5 Sitten hän mittasi tuhat: tuli virta, jonka poikki minä en voinut käydä, sillä vesi nousi uimavedeksi, virraksi, josta ei voinut käydä poikki.
\par 6 Niin hän kysyi minulta: "Oletko nähnyt, ihmislapsi?" ja kuljetti minua ja toi takaisin pitkin virran rantaa.
\par 7 Mutta kun minä tulin takaisin, niin katso: virran rannalla kasvoi hyvin paljon puita molemmilla puolin.
\par 8 Ja hän sanoi minulle: "Nämä vedet juoksevat itäiselle alueelle, virtaavat alas Aromaahan ja tulevat mereen; niiden jouduttua mereen vesi siinä paranee.
\par 9 Ja kaikki elolliset, kaikki, jotka liikkuvat, virkoavat elämään kaikkialla, mihin tämä kaksoisvirta tulee. Ja kaloja on oleva hyvin paljon; sillä kun nämä vedet sinne tulevat ja vesi paranee, niin kaikki virkoaa elämään, minne vain virta tulee.
\par 10 Ja kalastajia seisoo sen rannalla. Een-Gedistä Een-Eglaimiin asti se on oleva yhtä verkkoapajaa. Siinä on kaikenlaisia kaloja, aivan kuin suuren meren kaloja, hyvin paljon.
\par 11 Sen rämeet ja lätäköt eivät parane: ne jätetään suolan valtaan.
\par 12 Mutta virran varrella, sen molemmilla rannoilla, kasvaa kaikkinaisia hedelmäpuita. Niistä eivät lakastu lehdet eivätkä lopu hedelmät: joka kuukausi ne kantavat tuoreet hedelmät, sillä niitten vedet juoksevat pyhäköstä, ja niitten hedelmät ovat ravitsevaiset ja niitten lehdet parantavaiset.
\par 13 Näin sanoo Herra, Herra: Tämä on raja, jonka mukaan teidän on jaettava maa perintöosiksi kahdelletoista Israelin sukukunnalle - Joosef saakoon kaksi osaa -.
\par 14 Ja te saatte siitä perintöosan jokainen kohdaltansa; sillä minä olen kättä kohottaen luvannut antaa sen teidän isillenne, ja niin tämä maa tulee teille perintöosaksi.
\par 15 Tämä on maan pohjoispuolinen raja: Suuresta merestä Hetlonin tietä siihen asti, mistä mennään Sedadiin.
\par 16 Hamat, Beerota, Sibraim, joka on Damaskon alueen ja Hamatin alueen välissä, keskimmäinen Haaser, joka on Hauranin rajalla;
\par 17 ja näin menee raja merestä Hasar-Eenoniin - Damaskon alue jää pohjoiseen ja pohjoiseen myös Hamatin alue. Tämä on pohjoispuoli.
\par 18 Sitten itäpuoli: Hauranin ja Damaskon välistä sekä Gileadin ja Israelin maan välistä, Jordania pitkin. Mitatkaa se rajasta Idänmereen. Tämä on itäpuoli.
\par 19 Sitten eteläpuoli, päivään päin: Taamarista Meriban veteen, joka on Kaadeksessa, Puroon ja Suureen mereen. Tämä on päivänpuoli, etelään päin.
\par 20 Sitten länsipuoli: Suuri meri rajasta sen paikan kohdalle, mistä mennään Hamatiin. Tämä on länsipuoli.
\par 21 Jakakaa tämä maa keskenänne Israelin sukukuntien mukaan.
\par 22 Ja arpokaa se perintöosiksi itsellenne ja muukalaisille, jotka asuvat teidän keskuudessanne ja ovat synnyttäneet lapsia teidän keskuudessanne. Olkoot he teille saman arvoisia kuin maassa syntyneet israelilaiset: he saakoot arvalla perintöosan Israelin sukukuntain keskuudessa teidän kanssanne.
\par 23 Missä sukukunnassa muukalainen asuu, siinä antakaa hänelle perintöosa, sanoo Herra, Herra."

\chapter{48}

\par 1 "Ja nämä ovat sukukuntien nimet: Pohjois-äärellä, pitkin Hetlonin tien vartta siihen asti, mistä mennään Hamatiin, ja siitä Hasar-Eenaniin - Damaskon alue jää pohjoiseen Hamatin sivulle -; tämä tulee hänelle idän puolelta lännen puolelle asti: Daan, yksi osa.
\par 2 Daanin alueen sivussa, idän puolelta lännen puolelle asti: Asser, yksi osa.
\par 3 Asserin alueen sivussa, idän puolelta lännen puolelle asti: Naftali, yksi osa.
\par 4 Naftalin alueen sivussa, idän puolelta lännen puolelle asti: Manasse, yksi osa.
\par 5 Manassen alueen sivussa, idän puolelta lännen puolelle asti: Efraim, yksi osa.
\par 6 Efraimin alueen sivussa, idän puolelta lännen puolelle asti: Ruuben, yksi osa.
\par 7 Ruubenin alueen sivussa, idän puolelta lännen puolelle asti: Juuda, yksi osa.
\par 8 Juudan alueen sivussa, idän puolelta lännen puolelle asti, on oleva se antimaa, joka teidän on annettava, kahtakymmentäviittä tuhatta leveä ja niin pitkä kuin yksi sukukuntaosa idän puolelta lännen puolelle; ja pyhäkkö olkoon sen keskellä.
\par 9 Antimaa, joka teidän on annettava Herralle, olkoon kahtakymmentäviittä tuhatta pitkä ja kahtakymmentä tuhatta leveä.
\par 10 Ja pyhää antimaata tulkoon seuraaville: papeille pohjoisesta kahtakymmentäviittä tuhatta, lännestä kymmentätuhatta leveälti, idästä kymmentätuhatta leveälti ja etelästä kahtakymmentäviittä tuhatta pitkälti; ja Herran pyhäkkö olkoon siinä keskellä.
\par 11 Papeille, niille Saadokin jälkeläisistä, jotka ovat pyhitettyjä, jotka ovat hoitaneet minulle suoritettavat tehtävät ja jotka eivät eksyneet, niinkuin leeviläiset eksyivät, silloin kun israelilaiset joutuivat eksyksiin,
\par 12 heille se tulkoon verona antimaasta, korkeasti-pyhänä; tulkoon leeviläisten alueen vierestä.
\par 13 Leeviläiset saakoot samanlaisen alueen kuin papit, kahtakymmentäviittä tuhatta pitkän ja kymmentätuhatta leveän. Koko pituus olkoon kaksikymmentäviisi tuhatta ja leveys kaksikymmentä tuhatta.
\par 14 Älkööt he myykö siitä mitään, älköönkä parasta maata vaihdettako tai luovutettako, sillä se on pyhitetty Herralle.
\par 15 Mutta viittätuhatta leveä maa, mikä jää yli kahdenkymmenenviiden tuhannen sivusta, on kaupungin yhteisomaisuutta asumuksia ja avointa tilaa varten; ja kaupunki olkoon sen keskellä.
\par 16 Ja nämä olkoot sen mitat: pohjoispuoli neljätuhatta viisisataa, eteläpuoli neljätuhatta viisisataa, itäpuoli neljätuhatta viisisataa ja länsipuoli neljätuhatta viisisataa.
\par 17 Ja kaupungilla olkoon avoin tila: pohjoiseen päin kaksisataa viisikymmentä, etelään päin kaksisataa viisikymmentä, itään päin kaksisataa viisikymmentä ja länteen päin kaksisataa viisikymmentä.
\par 18 Mutta niin pitkälti kuin jää yli pyhän antimaan viereltä, kymmenentuhatta itään päin ja kymmenentuhatta länteen päin, se olkoon pyhän antimaan vierellä, ja sen sato tulkoon leiväksi kaupungin työmiehille;
\par 19 ja sitä viljelkööt kaupungin työmiehet kaikista Israelin sukukunnista.
\par 20 Koko antimaa on kahtakymmentäviittä tuhatta ja kahtakymmentäviittä tuhatta. Neliönmuotoisena on teidän annettava pyhä antimaa ynnä kaupungin perintömaa.
\par 21 Mikä jää yli, se tulkoon ruhtinaalle: molemmilta puolilta pyhää antimaata ja kaupungin perintömaata, niitten kahdenkymmenenviiden tuhannen - antimaan - sivusta itärajaan asti, ynnä länteen päin niitten kahdenkymmenen viiden tuhannen sivusta länsirajaan asti, on ruhtinaalle tuleva sukukuntaosia vastaava maa; ja pyhä antimaa sekä temppelipyhäkkö olkoot sen keskellä.
\par 22 Leeviläisten perintömaasta ja kaupungin perintömaasta asti, jotka ovat ruhtinaalle tulevan maan keskellä, tulee ruhtinaalle se, mikä on Juudan alueen ja Benjaminin alueen välissä.
\par 23 Sitten muut sukukunnat: Idän puolelta lännen puolelle asti: Benjamin, yksi osa.
\par 24 Benjaminin alueen sivussa, idän puolelta lännen puolelle asti: Simeon, yksi osa.
\par 25 Simeonin alueen sivussa, idän puolelta lännen puolelle asti: Isaskar, yksi osa.
\par 26 Isaskarin alueen sivussa, idän puolelta lännen puolelle asti: Sebulon, yksi osa.
\par 27 Sebulonin alueen sivussa, idän puolelta lännen puolelle asti: Gaad, yksi osa.
\par 28 Ja Gaadin alueen sivua, etelän puolella, päivään päin, menee raja Taamarista Meriban veteen, Kaadekseen, Puroon ja Suureen mereen.
\par 29 Tämä on maa, joka teidän on arvottava perintöosiksi Israelin sukukunnille, ja nämä ovat heidän osuutensa, sanoo Herra, Herra.
\par 30 Ja nämä ovat kaupungin uloskäytävät: Pohjoispuolella on mitta neljätuhatta viisisataa.
\par 31 Ja kaupungin portteja, jotka ovat nimitetyt Israelin sukukuntain mukaan, on pohjoisessa kolme: Ruubenin portti yksi, Juudan portti toinen ja Leevin portti kolmas.
\par 32 Itäpuolella: neljätuhatta viisisataa, ja portteja kolme: Joosefin portti yksi, Benjaminin portti toinen ja Daanin portti kolmas.
\par 33 Eteläpuolella on mitta neljätuhatta viisisataa, ja portteja kolme: Simeonin portti yksi, Isaskarin portti toinen ja Sebulonin portti kolmas.
\par 34 Länsipuolella: neljätuhatta viisisataa, ja portteja kolme: Gaadin portti yksi, Asserin portti toinen ja Naftalin portti kolmas.
\par 35 Ympärinsä: kahdeksantoista tuhatta. Ja kaupungin nimi on tästedes oleva: Herra on täällä."


\end{document}