\begin{document}

\title{Evankeliumi Johanneksen mukaan}


\chapter{1}

\par 1 Alussa oli Sana, ja Sana oli Jumalan tykönä, ja Sana oli Jumala.
\par 2 Hän oli alussa Jumalan tykönä.
\par 3 Kaikki on saanut syntynsä hänen kauttaan, ja ilman häntä ei ole syntynyt mitään, mikä syntynyt on.
\par 4 Hänessä oli elämä, ja elämä oli ihmisten valkeus.
\par 5 Ja valkeus loistaa pimeydessä, ja pimeys ei sitä käsittänyt.
\par 6 Oli mies, Jumalan lähettämä; hänen nimensä oli Johannes.
\par 7 Hän tuli todistamaan, todistaaksensa valkeudesta, että kaikki uskoisivat hänen kauttansa.
\par 8 Ei hän ollut se valkeus, mutta hän tuli valkeudesta todistamaan.
\par 9 Totinen valkeus, joka valistaa jokaisen ihmisen, oli tulossa maailmaan.
\par 10 Maailmassa hän oli, ja maailma on hänen kauttaan saanut syntynsä, ja maailma ei häntä tuntenut.
\par 11 Hän tuli omiensa tykö, ja hänen omansa eivät ottaneet häntä vastaan.
\par 12 Mutta kaikille, jotka ottivat hänet vastaan, hän antoi voiman tulla Jumalan lapsiksi, niille, jotka uskovat hänen nimeensä,
\par 13 jotka eivät ole syntyneet verestä eikä lihan tahdosta eikä miehen tahdosta, vaan Jumalasta.
\par 14 Ja Sana tuli lihaksi ja asui meidän keskellämme, ja me katselimme hänen kirkkauttansa, senkaltaista kirkkautta, kuin ainokaisella Pojalla on Isältä; ja hän oli täynnä armoa ja totuutta.
\par 15 Johannes todisti hänestä ja huusi sanoen: "Tämä on se, josta minä sanoin: se, joka minun jälkeeni tulee, on ollut minun edelläni, sillä hän on ollut ennen kuin minä."
\par 16 Ja hänen täyteydestään me kaikki olemme saaneet, ja armoa armon päälle.
\par 17 Sillä laki on annettu Mooseksen kautta; armo ja totuus on tullut Jeesuksen Kristuksen kautta.
\par 18 Ei kukaan ole Jumalaa milloinkaan nähnyt; ainokainen Poika, joka on Isän helmassa, on hänet ilmoittanut.
\par 19 Ja tämä on Johanneksen todistus, kun juutalaiset lähettivät hänen luoksensa Jerusalemista pappeja ja leeviläisiä kysymään häneltä: "Kuka sinä olet?"
\par 20 Ja hän tunnusti eikä kieltänyt; ja hän tunnusti: "Minä en ole Kristus".
\par 21 Ja he kysyivät häneltä: "Mikä sitten? Oletko sinä Elias?" Hän sanoi: "En ole". "Se profeettako olet?" Hän vastasi: "En".
\par 22 Niin he sanoivat hänelle: "Kuka olet, että voisimme antaa vastauksen niille, jotka meidät lähettivät? Mitä sanot itsestäsi?"
\par 23 Hän sanoi: "Minä olen huutavan ääni erämaassa: 'Tehkää tie tasaiseksi Herralle', niinkuin profeetta Esaias on sanonut."
\par 24 Ja lähetetyt olivat fariseuksia;
\par 25 ja he kysyivät häneltä ja sanoivat hänelle: "Miksi sitten kastat, jos et ole Kristus etkä Elias etkä se profeetta?"
\par 26 Johannes vastasi heille sanoen: "Minä kastan vedellä; mutta teidän keskellänne seisoo hän, jota te ette tunne.
\par 27 Hän on se, joka tulee minun jälkeeni ja jonka kengänpaulaa minä en ole arvollinen päästämään."
\par 28 Tämä tapahtui Betaniassa, Jordanin tuolla puolella, jossa Johannes oli kastamassa.
\par 29 Seuraavana päivänä hän näki Jeesuksen tulevan tykönsä ja sanoi: "Katso, Jumalan Karitsa, joka ottaa pois maailman synnin!
\par 30 Tämä on se, josta minä sanoin: 'Minun jälkeeni tulee mies, joka on ollut minun edelläni, sillä hän on ollut ennen kuin minä'.
\par 31 Ja minä en tuntenut häntä; mutta sitä varten, että hän tulisi julki Israelille, minä olen tullut vedellä kastamaan."
\par 32 Ja Johannes todisti sanoen: "Minä näin Hengen laskeutuvan taivaasta alas niinkuin kyyhkysen, ja se jäi hänen päällensä.
\par 33 Ja minä en tuntenut häntä; mutta hän, joka lähetti minut vedellä kastamaan, sanoi minulle: 'Se, jonka päälle sinä näet Hengen laskeutuvan ja jäävän, hän on se, joka kastaa Pyhällä Hengellä'.
\par 34 Ja minä olen sen nähnyt ja olen todistanut, että tämä on Jumalan Poika."
\par 35 Seuraavana päivänä Johannes taas seisoi siellä ja kaksi hänen opetuslapsistansa.
\par 36 Ja kiinnittäen katseensa Jeesukseen, joka siellä käveli, hän sanoi: "Katso, Jumalan Karitsa!"
\par 37 Ja ne kaksi opetuslasta kuulivat hänen näin puhuvan ja seurasivat Jeesusta.
\par 38 Niin Jeesus kääntyi ja nähdessään heidän seuraavan sanoi heille: "Mitä te etsitte?" He vastasivat hänelle: "Rabbi" - se on käännettynä: opettaja - "missä sinä majailet?"
\par 39 Hän sanoi heille: "Tulkaa ja katsokaa". Niin he menivät ja näkivät, missä hän majaili, ja viipyivät hänen tykönään sen päivän. Silloin oli noin kymmenes hetki.
\par 40 Andreas, Simon Pietarin veli, oli toinen niistä kahdesta, jotka olivat kuulleet, mitä Johannes sanoi, ja seuranneet Jeesusta.
\par 41 Hän tapasi ensin veljensä Simonin ja sanoi hänelle: "Me olemme löytäneet Messiaan", se on käännettynä: Kristus.
\par 42 Ja hän vei hänet Jeesuksen tykö. Jeesus kiinnitti katseensa häneen ja sanoi: "Sinä olet Simon, Johanneksen poika; sinun nimesi on oleva Keefas", joka käännettynä on Pietari.
\par 43 Seuraavana päivänä Jeesus tahtoi lähteä Galileaan; ja hän tapasi Filippuksen ja sanoi hänelle: "Seuraa minua".
\par 44 Ja Filippus oli Beetsaidasta, Andreaan ja Pietarin kaupungista.
\par 45 Filippus tapasi Natanaelin ja sanoi hänelle: "Me olemme löytäneet sen, josta Mooses laissa ja profeetat ovat kirjoittaneet, Jeesuksen, Joosefin pojan, Nasaretista".
\par 46 Natanael sanoi hänelle: "Voiko Nasaretista tulla mitään hyvää?" Filippus sanoi hänelle: "Tule ja katso".
\par 47 Jeesus näki Natanaelin tulevan tykönsä ja sanoi hänestä: "Katso, oikea israelilainen, jossa ei vilppiä ole!"
\par 48 Natanael sanoi hänelle: "Mistä minut tunnet?" Jeesus vastasi ja sanoi hänelle: "Ennenkuin Filippus sinua kutsui, kun olit viikunapuun alla, näin minä sinut".
\par 49 Natanael vastasi ja sanoi hänelle: "Rabbi, sinä olet Jumalan Poika, sinä olet Israelin kuningas".
\par 50 Jeesus vastasi ja sanoi hänelle: "Sentähden, että minä sanoin sinulle: 'minä näin sinut viikunapuun alla', sinä uskot. Sinä saat nähdä suurempia, kuin nämä ovat."
\par 51 Ja hän sanoi hänelle: "Totisesti, totisesti minä sanon teille: te saatte nähdä taivaan avoinna ja Jumalan enkelien nousevan ylös ja laskeutuvan alas Ihmisen Pojan päälle."

\chapter{2}

\par 1 Ja kolmantena päivänä oli häät Galilean Kaanassa, ja Jeesuksen äiti oli siellä.
\par 2 Ja myös Jeesus ja hänen opetuslapsensa olivat kutsutut häihin.
\par 3 Ja kun viini loppui, sanoi Jeesuksen äiti hänelle: "Heillä ei ole viiniä".
\par 4 Jeesus sanoi hänelle: "Mitä sinä tahdot minusta, vaimo? Minun aikani ei ole vielä tullut."
\par 5 Hänen äitinsä sanoi palvelijoille: "Mitä hän teille sanoo, se tehkää".
\par 6 Niin oli siinä juutalaisten puhdistamistavan mukaan kuusi kivistä vesiastiaa, kukin kahden tai kolmen mitan vetoinen.
\par 7 Jeesus sanoi heille: "Täyttäkää astiat vedellä". Ja he täyttivät ne reunoja myöten.
\par 8 Ja hän sanoi heille: "Ammentakaa nyt ja viekää edeskäyvälle". Ja he veivät.
\par 9 Mutta kun edeskäypä maistoi vettä, joka oli muuttunut viiniksi, eikä tiennyt, mistä se oli tullut - mutta palvelijat, jotka veden olivat ammentaneet, tiesivät sen - kutsui edeskäypä yljän
\par 10 ja sanoi hänelle: "Jokainen panee ensin esille hyvän viinin ja sitten, kun juopuvat, huonomman. Sinä olet säästänyt hyvän viinin tähän asti."
\par 11 Tämän ensimmäisen tunnustekonsa Jeesus teki Galilean Kaanassa ja ilmoitti kirkkautensa; ja hänen opetuslapsensa uskoivat häneen.
\par 12 Sen jälkeen hän meni alas Kapernaumiin, hän ja hänen äitinsä ja veljensä ja opetuslapsensa; ja siellä he eivät viipyneet monta päivää.
\par 13 Ja juutalaisten pääsiäinen oli lähellä, ja Jeesus meni ylös Jerusalemiin.
\par 14 Niin hän tapasi pyhäkössä ne, jotka myivät härkiä ja lampaita ja kyyhkysiä, ja rahanvaihtajat istumassa.
\par 15 Ja hän teki nuorista ruoskan ja ajoi ulos pyhäköstä heidät kaikki lampaineen ja härkineen ja kaasi vaihtajain rahat maahan ja työnsi heidän pöytänsä kumoon.
\par 16 Ja hän sanoi kyyhkysten myyjille: "Viekää pois nämä täältä. Älkää tehkö minun Isäni huonetta markkinahuoneeksi."
\par 17 Silloin hänen opetuslapsensa muistivat, että on kirjoitettu: "Kiivaus sinun huoneesi puolesta kuluttaa minut".
\par 18 Niin juutalaiset vastasivat ja sanoivat hänelle: "Minkä merkin sinä näytät meille, koska näitä teet?"
\par 19 Jeesus vastasi ja sanoi heille: "Hajottakaa maahan tämä temppeli, niin minä pystytän sen kolmessa päivässä".
\par 20 Niin juutalaiset sanoivat: "Neljäkymmentä kuusi vuotta on tätä temppeliä rakennettu, ja sinäkö pystytät sen kolmessa päivässä?"
\par 21 Mutta hän puhui ruumiinsa temppelistä.
\par 22 Kun hän sitten oli noussut kuolleista, muistivat hänen opetuslapsensa, että hän oli tämän sanonut; ja he uskoivat Raamatun ja sen sanan, jonka Jeesus oli sanonut.
\par 23 Mutta kun hän oli Jerusalemissa pääsiäisenä, juhlan aikana, uskoivat monet hänen nimeensä, nähdessään hänen tunnustekonsa, jotka hän teki.
\par 24 Mutta Jeesus itse ei uskonut itseänsä heille, sentähden että hän tunsi kaikki
\par 25 eikä tarvinnut kenenkään todistusta ihmisestä, sillä hän tiesi itse, mitä ihmisessä on.

\chapter{3}

\par 1 Mutta oli mies, fariseusten joukosta, nimeltä Nikodeemus, juutalaisten hallitusmiehiä.
\par 2 Hän tuli Jeesuksen tykö yöllä ja sanoi hänelle: "Rabbi, me tiedämme, että sinun opettajaksi tulemisesi on Jumalasta, sillä ei kukaan voi tehdä niitä tunnustekoja, joita sinä teet, ellei Jumala ole hänen kanssansa."
\par 3 Jeesus vastasi ja sanoi hänelle: "Totisesti, totisesti minä sanon sinulle: joka ei synny uudesti, ylhäältä, se ei voi nähdä Jumalan valtakuntaa".
\par 4 Nikodeemus sanoi hänelle: "Kuinka voi ihminen vanhana syntyä? Eihän hän voi jälleen mennä äitinsä kohtuun ja syntyä?"
\par 5 Jeesus vastasi: "Totisesti, totisesti minä sanon sinulle: jos joku ei synny vedestä ja Hengestä, ei hän voi päästä sisälle Jumalan valtakuntaan.
\par 6 Mikä lihasta on syntynyt, on liha; ja mikä Hengestä on syntynyt, on henki.
\par 7 Älä ihmettele, että minä sanoin sinulle: teidän täytyy syntyä uudesti, ylhäältä.
\par 8 Tuuli puhaltaa, missä tahtoo, ja sinä kuulet sen huminan, mutta et tiedä, mistä se tulee ja minne se menee; niin on jokaisen, joka on Hengestä syntynyt."
\par 9 Nikodeemus vastasi ja sanoi hänelle: "Kuinka tämä voi tapahtua?"
\par 10 Jeesus vastasi ja sanoi hänelle: "Sinä olet Israelin opettaja etkä tätä tiedä!
\par 11 Totisesti, totisesti minä sanon sinulle: me puhumme, mitä tiedämme, ja todistamme, mitä olemme nähneet, ettekä te ota vastaan meidän todistustamme.
\par 12 Jos ette usko, kun minä puhun teille maallisista, kuinka te uskoisitte, jos minä puhun teille taivaallisista?
\par 13 Ei kukaan ole noussut ylös taivaaseen, paitsi hän, joka taivaasta tuli alas, Ihmisen Poika, joka on taivaassa.
\par 14 Ja niinkuin Mooses ylensi käärmeen erämaassa, niin pitää Ihmisen Poika ylennettämän,
\par 15 että jokaisella, joka häneen uskoo, olisi iankaikkinen elämä.
\par 16 Sillä niin on Jumala maailmaa rakastanut, että hän antoi ainokaisen Poikansa, ettei yksikään, joka häneen uskoo, hukkuisi, vaan hänellä olisi iankaikkinen elämä.
\par 17 Sillä ei Jumala lähettänyt Poikaansa maailmaan tuomitsemaan maailmaa, vaan sitä varten, että maailma hänen kauttansa pelastuisi.
\par 18 Joka uskoo häneen, sitä ei tuomita; mutta joka ei usko, se on jo tuomittu, koska hän ei ole uskonut Jumalan ainokaisen Pojan nimeen.
\par 19 Mutta tämä on tuomio, että valkeus on tullut maailmaan, ja ihmiset rakastivat pimeyttä enemmän kuin valkeutta; sillä heidän tekonsa olivat pahat.
\par 20 Sillä jokainen, joka pahaa tekee, vihaa valkeutta eikä tule valkeuteen, ettei hänen tekojansa nuhdeltaisi.
\par 21 Mutta joka totuuden tekee, se tulee valkeuteen, että hänen tekonsa tulisivat julki, sillä ne ovat Jumalassa tehdyt."
\par 22 Sen jälkeen Jeesus meni opetuslapsineen Juudean maaseudulle ja oleskeli siellä heidän kanssaan ja kastoi.
\par 23 Mutta Johanneskin kastoi Ainonissa lähellä Salimia, koska siellä oli paljon vettä; ja ihmiset tulivat ja antoivat kastaa itsensä.
\par 24 Sillä Johannesta ei vielä oltu heitetty vankeuteen.
\par 25 Niin Johanneksen opetuslapset rupesivat väittelemään erään juutalaisen kanssa puhdistuksesta.
\par 26 Ja he tulivat Johanneksen luo ja sanoivat hänelle: "Rabbi, se, joka oli sinun kanssasi Jordanin tuolla puolella ja josta sinä olet todistanut, katso, hän kastaa, ja kaikki menevät hänen tykönsä".
\par 27 Johannes vastasi ja sanoi: "Ei ihminen voi ottaa mitään, ellei hänelle anneta taivaasta.
\par 28 Te olette itse minun todistajani, että minä sanoin: en minä ole Kristus, vaan minä olen hänen edellänsä lähetetty.
\par 29 Jolla on morsian, se on ylkä; mutta yljän ystävä, joka seisoo ja kuuntelee häntä, iloitsee suuresti yljän äänestä. Tämä minun iloni on nyt tullut täydelliseksi.
\par 30 Hänen tulee kasvaa, mutta minun vähetä.
\par 31 Hän, joka ylhäältä tulee, on yli kaikkien. Joka on syntyisin maasta, se on maasta, ja maasta on, mitä hän puhuu; hän, joka taivaasta tulee, on yli kaikkien.
\par 32 Ja mitä hän on nähnyt ja kuullut, sitä hän todistaa; ja hänen todistustansa ei kukaan ota vastaan.
\par 33 Joka ottaa vastaan hänen todistuksensa, se sinetillä vahvistaa, että Jumala on totinen.
\par 34 Sillä hän, jonka Jumala on lähettänyt, puhuu Jumalan sanoja; sillä ei Jumala anna Henkeä mitalla.
\par 35 Isä rakastaa Poikaa ja on antanut kaikki hänen käteensä.
\par 36 Joka uskoo Poikaan, sillä on iankaikkinen elämä; mutta joka ei ole kuuliainen Pojalle, se ei ole elämää näkevä, vaan Jumalan viha pysyy hänen päällänsä."

\chapter{4}

\par 1 Kun nyt Herra sai tietää fariseusten kuulleen, että Jeesus teki opetuslapsiksi ja kastoi useampia kuin Johannes
\par 2 - vaikka Jeesus ei itse kastanut, vaan hänen opetuslapsensa -
\par 3 jätti hän Juudean ja meni taas Galileaan.
\par 4 Mutta hänen oli kuljettava Samarian kautta.
\par 5 Niin hän tuli Sykar nimiseen Samarian kaupunkiin, joka on lähellä sitä maa-aluetta, minkä Jaakob oli antanut pojalleen Joosefille.
\par 6 Ja siellä oli Jaakobin lähde. Kun nyt Jeesus oli matkasta väsynyt, istui hän lähteen reunalle. Ja oli noin kuudes hetki.
\par 7 Niin tuli eräs Samarian nainen ammentamaan vettä. Jeesus sanoi hänelle: "Anna minulle juoda".
\par 8 Sillä hänen opetuslapsensa olivat lähteneet kaupunkiin ruokaa ostamaan.
\par 9 Niin Samarian nainen sanoi hänelle: "Kuinka sinä, joka olet juutalainen, pyydät juotavaa minulta, samarialaiselta naiselta?" Sillä juutalaiset eivät seurustele samarialaisten kanssa.
\par 10 Jeesus vastasi ja sanoi hänelle: "Jos sinä tietäisit Jumalan lahjan, ja kuka se on, joka sinulle sanoo: 'Anna minulle juoda', niin sinä pyytäisit häneltä, ja hän antaisi sinulle elävää vettä."
\par 11 Nainen sanoi hänelle: "Herra, eipä sinulla ole ammennusastiaa, ja kaivo on syvä; mistä sinulla sitten on se elävä vesi?
\par 12 Et kai sinä ole suurempi kuin meidän isämme Jaakob, joka antoi meille tämän kaivon ja joi siitä, hän itse sekä hänen poikansa ja karjansa?"
\par 13 Jeesus vastasi ja sanoi hänelle: "Jokainen, joka juo tätä vettä, janoaa jälleen,
\par 14 mutta joka juo sitä vettä, jota minä hänelle annan, se ei ikinä janoa; vaan se vesi, jonka minä hänelle annan, tulee hänessä sen veden lähteeksi, joka kumpuaa iankaikkiseen elämään".
\par 15 Nainen sanoi hänelle: "Herra, anna minulle sitä vettä, ettei minun tulisi jano eikä minun tarvitsisi käydä täällä ammentamassa".
\par 16 Jeesus sanoi hänelle: "Mene, kutsu miehesi ja tule tänne".
\par 17 Nainen vastasi ja sanoi: "Ei minulla ole miestä". Jeesus sanoi hänelle: "Oikein sinä sanoit: 'Ei minulla ole miestä',
\par 18 sillä viisi miestä sinulla on ollut, ja se, joka sinulla nyt on, ei ole sinun miehesi; siinä sanoit totuuden."
\par 19 Nainen sanoi hänelle: "Herra, minä näen, että sinä olet profeetta.
\par 20 Meidän isämme ovat kumartaen rukoilleet tällä vuorella; ja te sanotte, että Jerusalemissa on se paikka, jossa tulee kumartaen rukoilla."
\par 21 Jeesus sanoi hänelle: "Vaimo, usko minua! Tulee aika, jolloin ette rukoile Isää tällä vuorella ettekä Jerusalemissa.
\par 22 Te kumarratte sitä, mitä ette tunne; me kumarramme sitä, minkä me tunnemme. Sillä pelastus on juutalaisista.
\par 23 Mutta tulee aika ja on jo, jolloin totiset rukoilijat rukoilevat Isää hengessä ja totuudessa; sillä senkaltaisia rukoilijoita myös Isä tahtoo.
\par 24 Jumala on Henki; ja jotka häntä rukoilevat, niiden tulee rukoilla hengessä ja totuudessa."
\par 25 Nainen sanoi hänelle: "Minä tiedän, että Messias on tuleva, hän, jota sanotaan Kristukseksi; kun hän tulee, ilmoittaa hän meille kaikki".
\par 26 Jeesus sanoi hänelle: "Minä olen se, minä, joka puhun sinun kanssasi".
\par 27 Samassa hänen opetuslapsensa tulivat; ja he ihmettelivät, että hän puhui naisen kanssa. Kuitenkaan ei kukaan sanonut: "Mitä pyydät?" tai: "Mitä puhelet hänen kanssaan?"
\par 28 Niin nainen jätti vesiastiansa ja meni kaupunkiin ja sanoi ihmisille:
\par 29 "Tulkaa katsomaan miestä, joka on sanonut minulle kaikki, mitä minä olen tehnyt. Eihän se vain liene Kristus?"
\par 30 Niin he lähtivät kaupungista ja menivät hänen luoksensa.
\par 31 Sillävälin opetuslapset pyysivät häntä sanoen: "Rabbi, syö!"
\par 32 Mutta hän sanoi heille: "Minulla on syötävänä ruokaa, josta te ette tiedä".
\par 33 Niin opetuslapset sanoivat keskenään: "Lieneekö joku tuonut hänelle syötävää?"
\par 34 Jeesus sanoi heille: "Minun ruokani on se, että minä teen lähettäjäni tahdon ja täytän hänen tekonsa.
\par 35 Ettekö sano: 'Vielä on neljä kuukautta, niin elonleikkuu joutuu'? Katso, minä sanon teille: nostakaa silmänne ja katselkaa vainioita, kuinka ne ovat valjenneet leikattaviksi.
\par 36 Jo nyt saa leikkaaja palkan ja kokoaa hedelmää iankaikkiseen elämään, että kylväjä ja leikkaaja saisivat yhdessä iloita.
\par 37 Sillä tässä on se sana tosi, että toinen on kylväjä, ja leikkaaja toinen.
\par 38 Minä olen lähettänyt teidät leikkaamaan sitä, josta te ette ole vaivaa nähneet; toiset ovat vaivan nähneet, ja te olette päässeet heidän vaivansa hedelmille."
\par 39 Ja monet samarialaiset siitä kaupungista uskoivat häneen naisen puheen tähden, kun tämä todisti: "Hän on sanonut minulle kaikki, mitä minä olen tehnyt".
\par 40 Kun nyt samarialaiset tulivat hänen luoksensa, pyysivät he häntä viipymään heidän luonaan; ja hän viipyi siellä kaksi päivää.
\par 41 Ja vielä paljoa useammat uskoivat hänen sanansa tähden,
\par 42 ja he sanoivat naiselle: "Emme enää usko sinun puheesi tähden, sillä me itse olemme kuulleet ja tiedämme, että tämä totisesti on maailman Vapahtaja".
\par 43 Mutta niiden kahden päivän kuluttua hän lähti sieltä Galileaan.
\par 44 Sillä Jeesus itse todisti, ettei profeetalla ole arvoa omassa isiensä maassa.
\par 45 Kun hän siis tuli Galileaan, ottivat galilealaiset hänet vastaan, koska olivat nähneet kaikki, mitä hän oli tehnyt Jerusalemissa juhlan aikana; sillä hekin olivat tulleet juhlille.
\par 46 Niin hän tuli taas Galilean Kaanaan, jossa hän oli tehnyt veden viiniksi. Ja Kapernaumissa oli eräs kuninkaan virkamies, jonka poika sairasti.
\par 47 Kun hän kuuli Jeesuksen tulleen Juudeasta Galileaan, meni hän hänen luoksensa ja pyysi häntä tulemaan ja parantamaan hänen poikansa; sillä tämä oli kuolemaisillaan.
\par 48 Niin Jeesus sanoi hänelle: "Ellette näe merkkejä ja ihmeitä, te ette usko".
\par 49 Kuninkaan virkamies sanoi hänelle: "Herra, tule, ennenkuin minun lapseni kuolee".
\par 50 Jeesus sanoi hänelle: "Mene, sinun poikasi elää". Ja mies uskoi sanan, jonka Jeesus sanoi hänelle, ja meni.
\par 51 Ja jo hänen ollessaan paluumatkalla hänen palvelijansa kohtasivat hänet ja sanoivat, että hänen poikansa eli.
\par 52 Niin hän tiedusteli heiltä, millä hetkellä hän oli alkanut toipua. Ja he sanoivat hänelle: "Eilen seitsemännellä hetkellä kuume lähti hänestä".
\par 53 Niin isä ymmärsi, että se oli tapahtunut sillä hetkellä, jolloin Jeesus oli sanonut hänelle: "Sinun poikasi elää". Ja hän uskoi, hän ja koko hänen huonekuntansa.
\par 54 Tämä oli taas tunnusteko, toinen, jonka Jeesus teki, tultuaan Juudeasta Galileaan.

\chapter{5}

\par 1 Sen jälkeen oli juutalaisten juhla, ja Jeesus meni ylös Jerusalemiin.
\par 2 Ja Jerusalemissa on Lammasportin luona lammikko, jonka nimi hebreankielellä on Betesda, ja sen reunalla on viisi pylväskäytävää.
\par 3 Niissä makasi suuri joukko sairaita, sokeita, rampoja ja näivetystautisia, jotka odottivat veden liikuttamista.
\par 5 Ja siellä oli mies, joka oli sairastanut kolmekymmentä kahdeksan vuotta.
\par 6 Kun Jeesus näki hänen siinä makaavan ja tiesi hänen jo kauan aikaa sairastaneen, sanoi hän hänelle: "Tahdotko tulla terveeksi?"
\par 7 Sairas vastasi hänelle: "Herra, minulla ei ole ketään, joka veisi minut lammikkoon, kun vesi on kuohutettu; ja kun minä olen menemässä, astuu toinen sinne ennen minua".
\par 8 Jeesus sanoi hänelle: "Nouse, ota vuoteesi ja käy".
\par 9 Ja mies tuli kohta terveeksi ja otti vuoteensa ja kävi. Mutta se päivä oli sapatti.
\par 10 Sentähden juutalaiset sanoivat parannetulle: "Nyt on sapatti, eikä sinun ole lupa kantaa vuodetta".
\par 11 Hän vastasi heille: "Se, joka teki minut terveeksi, sanoi minulle: 'Ota vuoteesi ja käy'."
\par 12 He kysyivät häneltä: "Kuka on se mies, joka sanoi sinulle: 'Ota vuoteesi ja käy'?"
\par 13 Mutta parannettu ei tiennyt, kuka se oli; sillä Jeesus oli poistunut, kun siinä paikassa oli paljon kansaa.
\par 14 Sen jälkeen Jeesus tapasi hänet pyhäkössä ja sanoi hänelle: "Katso, sinä olet tullut terveeksi; älä enää syntiä tee, ettei sinulle jotakin pahempaa tapahtuisi".
\par 15 Niin mies meni ja ilmoitti juutalaisille, että Jeesus oli hänet terveeksi tehnyt.
\par 16 Ja sentähden juutalaiset vainosivat Jeesusta, koska hän semmoista teki sapattina.
\par 17 Mutta Jeesus vastasi heille: "Minun Isäni tekee yhäti työtä, ja minä myös teen työtä".
\par 18 Sentähden juutalaiset vielä enemmän tavoittelivat häntä tappaaksensa, kun hän ei ainoastaan kumonnut sapattia, vaan myös sanoi Jumalaa Isäksensä, tehden itsensä Jumalan vertaiseksi.
\par 19 Niin Jeesus vastasi ja sanoi heille: "Totisesti, totisesti minä sanon teille: Poika ei voi itsestänsä mitään tehdä, vaan ainoastaan sen, minkä hän näkee Isän tekevän; sillä mitä Isä tekee, sitä myös Poika samoin tekee.
\par 20 Sillä Isä rakastaa Poikaa ja näyttää hänelle kaikki, mitä hän itse tekee; ja hän on näyttävä hänelle suurempia tekoja kuin nämä, niin että te ihmettelette.
\par 21 Sillä niinkuin Isä herättää kuolleita ja tekee eläviksi, niin myös Poika tekee eläviksi, ketkä hän tahtoo.
\par 22 Sillä Isä ei myöskään tuomitse ketään, vaan hän on antanut kaiken tuomion Pojalle,
\par 23 että kaikki kunnioittaisivat Poikaa, niinkuin he kunnioittavat Isää. Joka ei kunnioita Poikaa, se ei kunnioita Isää, joka on hänet lähettänyt.
\par 24 Totisesti, totisesti minä sanon teille: joka kuulee minun sanani ja uskoo häneen, joka on minut lähettänyt, sillä on iankaikkinen elämä, eikä hän joudu tuomittavaksi, vaan on siirtynyt kuolemasta elämään.
\par 25 Totisesti, totisesti minä sanon teille: aika tulee ja on jo, jolloin kuolleet kuulevat Jumalan Pojan äänen, ja jotka sen kuulevat ne saavat elää.
\par 26 Sillä niinkuin Isällä on elämä itsessänsä, niin hän on antanut elämän myös Pojalle, niin että myös hänellä on elämä itsessänsä.
\par 27 Ja hän on antanut hänelle vallan tuomita, koska hän on Ihmisen Poika.
\par 28 Älkää ihmetelkö tätä, sillä hetki tulee, jolloin kaikki, jotka haudoissa ovat, kuulevat hänen äänensä
\par 29 ja tulevat esiin, ne, jotka ovat hyvää tehneet, elämän ylösnousemukseen, mutta ne, jotka ovat pahaa tehneet, tuomion ylösnousemukseen.
\par 30 En minä itsestäni voi mitään tehdä. Niinkuin minä kuulen, niin minä tuomitsen; ja minun tuomioni on oikea, sillä minä en kysy omaa tahtoani, vaan hänen tahtoaan, joka on minut lähettänyt.
\par 31 Jos minä itsestäni todistan, ei minun todistukseni ole pätevä.
\par 32 On toinen, joka todistaa minusta, ja minä tiedän, että se todistus, jonka hän minusta todistaa, on pätevä.
\par 33 Te lähetitte lähettiläät Johanneksen luo, ja hän todisti sen, mikä totta on.
\par 34 Mutta minä en ota ihmiseltä todistusta, vaan puhun tämän, että te pelastuisitte.
\par 35 Hän oli palava ja loistava lamppu, mutta te tahdoitte ainoastaan hetken iloitella hänen valossansa.
\par 36 Mutta minulla on todistus, joka on suurempi kuin Johanneksen; sillä ne teot, jotka Isä on antanut minun täytettävikseni, ne teot, jotka minä teen, todistavat minusta, että Isä on minut lähettänyt.
\par 37 Ja Isä, joka on minut lähettänyt, hän on todistanut minusta. Te ette ole koskaan kuulleet hänen ääntänsä ettekä nähneet hänen muotoansa,
\par 38 eikä teillä ole hänen sanaansa teissä pysyväisenä; sillä te ette usko sitä, jonka hän on lähettänyt.
\par 39 Te tutkitte kirjoituksia, sillä teillä on mielestänne niissä iankaikkinen elämä, ja ne juuri todistavat minusta;
\par 40 ja te ette tahdo tulla minun tyköni, että saisitte elämän.
\par 41 En minä ota vastaan kunniaa ihmisiltä;
\par 42 mutta minä tunnen teidät, ettei teillä ole Jumalan rakkautta itsessänne.
\par 43 Minä olen tullut Isäni nimessä, ja te ette ota minua vastaan; jos toinen tulee omassa nimessään, niin hänet te otatte vastaan.
\par 44 Kuinka te voisitte uskoa, te, jotka otatte vastaan kunniaa toinen toiseltanne, ettekä etsi sitä kunniaa, mikä tulee häneltä, joka yksin on Jumala?
\par 45 Älkää luulko, että minä olen syyttävä teitä Isän tykönä; teillä on syyttäjänne, Mooses, johon te panette toivonne.
\par 46 Sillä jos te Moosesta uskoisitte, niin te uskoisitte minua; sillä minusta hän on kirjoittanut.
\par 47 Mutta jos te ette usko hänen kirjoituksiaan, kuinka te uskoisitte minun sanojani?"

\chapter{6}

\par 1 Sen jälkeen Jeesus meni Galileaan, se on Tiberiaan, järven tuolle puolelle.
\par 2 Ja häntä seurasi paljon kansaa, koska he näkivät ne tunnusteot, joita hän teki sairaille.
\par 3 Ja Jeesus nousi vuorelle ja istui sinne opetuslapsinensa.
\par 4 Ja pääsiäinen, juutalaisten juhla, oli lähellä.
\par 5 Kun Jeesus nosti silmänsä ja näki paljon kansaa tulevan tykönsä, sanoi hän Filippukselle: "Mistä ostamme leipää näiden syödä?"
\par 6 Mutta sen hän sanoi koetellakseen häntä, sillä itse hän tiesi, mitä aikoi tehdä.
\par 7 Filippus vastasi hänelle: "Eivät kahdensadan denarin leivät heille riittäisi, niin että kukin saisi edes vähän".
\par 8 Niin toinen hänen opetuslapsistansa, Andreas, Simon Pietarin veli, sanoi hänelle:
\par 9 "Täällä on poikanen, jolla on viisi ohraleipää ja kaksi kalaa, mutta mitä ne ovat näin monelle?"
\par 10 Jeesus sanoi: "Asettakaa kansa aterioimaan". Ja siinä paikassa oli paljon ruohoa. Niin miehet, luvultaan noin viisituhatta, laskeutuivat maahan.
\par 11 Ja Jeesus otti leivät ja kiitti ja jakeli istuville; samoin kaloistakin, niin paljon kuin he tahtoivat.
\par 12 Mutta kun he olivat ravitut, sanoi hän opetuslapsillensa: "Kootkaa tähteeksi jääneet palaset, ettei mitään joutuisi hukkaan".
\par 13 Niin he kokosivat ne ja täyttivät kaksitoista vakkaa palasilla, mitkä olivat viidestä ohraleivästä jääneet tähteeksi niiltä, jotka olivat aterioineet.
\par 14 Kun nyt ihmiset näkivät sen tunnusteon, jonka Jeesus oli tehnyt, sanoivat he: "Tämä on totisesti se profeetta, joka oli maailmaan tuleva".
\par 15 Kun nyt Jeesus huomasi, että he aikoivat tulla ja väkisin ottaa hänet, tehdäkseen hänet kuninkaaksi, väistyi hän taas pois vuorelle, hän yksinänsä.
\par 16 Mutta kun ilta tuli, menivät hänen opetuslapsensa alas järven rantaan,
\par 17 astuivat venheeseen ja lähtivät menemään järven toiselle puolelle, Kapernaumiin. Ja oli jo tullut pimeä, eikä Jeesus ollut vielä saapunut heidän luokseen;
\par 18 ja järvi aaltoili ankarasti kovan tuulen puhaltaessa.
\par 19 Kun he olivat soutaneet noin viisikolmatta tai kolmekymmentä vakomittaa, näkivät he Jeesuksen kävelevän järven päällä ja tulevan lähelle venhettä; ja he peljästyivät.
\par 20 Mutta hän sanoi heille: "Minä se olen; älkää peljätkö".
\par 21 Niin he tahtoivat ottaa hänet venheeseen, ja kohta venhe saapui sen maan rantaan, jonne he olivat matkalla.
\par 22 Seuraavana päivänä kansa yhä vielä oli järven toisella puolella, sillä he olivat nähneet, ettei siellä ollut muuta venhettä kuin se yksi ja ettei Jeesus mennyt opetuslastensa kanssa venheeseen, vaan että hänen opetuslapsensa lähtivät yksinään pois.
\par 23 Kuitenkin oli muita venheitä tullut Tiberiaasta lähelle sitä paikkaa, jossa he olivat syöneet leipää, sittenkuin Herra oli lausunut kiitoksen.
\par 24 Kun siis kansa näki, ettei Jeesus ollut siellä eivätkä hänen opetuslapsensa, astuivat hekin venheisiin ja menivät Kapernaumiin ja etsivät Jeesusta.
\par 25 Ja kun he löysivät hänet järven toiselta puolelta, sanoivat he hänelle: "Rabbi, milloin tulit tänne?"
\par 26 Jeesus vastasi heille ja sanoi: "Totisesti, totisesti minä sanon teille: ette te minua sentähden etsi, että olette nähneet tunnustekoja, vaan sentähden, että saitte syödä niitä leipiä ja tulitte ravituiksi.
\par 27 Älkää hankkiko sitä ruokaa, joka katoaa, vaan sitä ruokaa, joka pysyy hamaan iankaikkiseen elämään ja jonka Ihmisen Poika on teille antava; sillä häneen on Isä, Jumala itse, sinettinsä painanut."
\par 28 Niin he sanoivat hänelle: "Mitä meidän pitää tekemän, että me Jumalan tekoja tekisimme?"
\par 29 Jeesus vastasi ja sanoi heille: "Se on Jumalan teko, että te uskotte häneen, jonka Jumala on lähettänyt".
\par 30 He sanoivat hänelle: "Minkä tunnusteon sinä sitten teet, että me näkisimme sen ja uskoisimme sinua? Minkä teon sinä teet?
\par 31 Meidän isämme söivät mannaa erämaassa, niinkuin kirjoitettu on: 'Hän antoi leipää taivaasta heille syötäväksi'."
\par 32 Niin Jeesus sanoi heille: "Totisesti, totisesti minä sanon teille: ei Mooses antanut teille sitä leipää taivaasta, vaan minun Isäni antaa teille taivaasta totisen leivän.
\par 33 Sillä Jumalan leipä on se, joka tulee alas taivaasta ja antaa maailmalle elämän."
\par 34 Niin he sanoivat hänelle: "Herra, anna meille aina sitä leipää".
\par 35 Jeesus sanoi heille: "Minä olen elämän leipä; joka tulee minun tyköni, se ei koskaan isoa, ja joka uskoo minuun, se ei koskaan janoa.
\par 36 Mutta minä olen sanonut teille, että te olette nähneet minut, ettekä kuitenkaan usko.
\par 37 Kaikki, minkä Isä antaa minulle, tulee minun tyköni; ja sitä, joka minun tyköni tulee, minä en heitä ulos.
\par 38 Sillä minä olen tullut taivaasta, en tekemään omaa tahtoani, vaan hänen tahtonsa, joka on minut lähettänyt.
\par 39 Ja minun lähettäjäni tahto on se, että minä kaikista niistä, jotka hän on minulle antanut, en kadota yhtäkään, vaan herätän heidät viimeisenä päivänä.
\par 40 Sillä minun Isäni tahto on se, että jokaisella, joka näkee Pojan ja uskoo häneen, on iankaikkinen elämä; ja minä herätän hänet viimeisenä päivänä."
\par 41 Niin juutalaiset nurisivat häntä vastaan, koska hän sanoi: "Minä olen se leipä, joka on tullut alas taivaasta";
\par 42 ja he sanoivat: "Eikö tämä ole Jeesus, Joosefin poika, jonka isän ja äidin me tunnemme? Kuinka hän sitten sanoo: 'Minä olen tullut alas taivaasta'?"
\par 43 Jeesus vastasi ja sanoi heille: "Älkää nurisko keskenänne.
\par 44 Ei kukaan voi tulla minun tyköni, ellei Isä, joka on minut lähettänyt, häntä vedä; ja minä herätän hänet viimeisenä päivänä.
\par 45 Profeetoissa on kirjoitettuna: 'Ja he tulevat kaikki Jumalan opettamiksi'. Jokainen, joka on Isältä kuullut ja oppinut, tulee minun tyköni.
\par 46 Ei niin, että kukaan olisi Isää nähnyt; ainoastaan hän, joka on Jumalasta, on nähnyt Isän.
\par 47 Totisesti, totisesti minä sanon teille: joka uskoo, sillä on iankaikkinen elämä.
\par 48 Minä olen elämän leipä.
\par 49 Teidän isänne söivät mannaa erämaassa, ja he kuolivat.
\par 50 Mutta tämä on se leipä, joka tulee alas taivaasta, että se, joka sitä syö, ei kuolisi.
\par 51 Minä olen se elävä leipä, joka on tullut alas taivaasta. Jos joku syö tätä leipää, hän elää iankaikkisesti. Ja se leipä, jonka minä annan, on minun lihani, maailman elämän puolesta."
\par 52 Silloin juutalaiset riitelivät keskenään sanoen: "Kuinka tämä voi antaa lihansa meille syötäväksi?"
\par 53 Niin Jeesus sanoi heille: "Totisesti, totisesti minä sanon teille: ellette syö Ihmisen Pojan lihaa ja juo hänen vertansa, ei teillä ole elämää itsessänne.
\par 54 Joka syö minun lihani ja juo minun vereni, sillä on iankaikkinen elämä, ja minä herätän hänet viimeisenä päivänä.
\par 55 Sillä minun lihani on totinen ruoka, ja minun vereni on totinen juoma.
\par 56 Joka syö minun lihani ja juo minun vereni, se pysyy minussa, ja minä hänessä.
\par 57 Niinkuin Isä, joka elää, on minut lähettänyt, ja minä elän Isän kautta, niin myös se, joka minua syö, elää minun kauttani.
\par 58 Tämä on se leipä, joka tuli alas taivaasta. Ei ole, niinkuin oli teidän isienne: he söivät ja kuolivat; joka tätä leipää syö, se elää iankaikkisesti."
\par 59 Tämän hän puhui synagoogassa opettaessaan Kapernaumissa.
\par 60 Niin monet hänen opetuslapsistansa, sen kuultuaan, sanoivat: "Tämä on kova puhe, kuka voi sitä kuulla?"
\par 61 Mutta kun Jeesus sydämessään tiesi, että hänen opetuslapsensa siitä nurisivat, sanoi hän heille: "Loukkaako tämä teitä?
\par 62 Mitä sitten, jos saatte nähdä Ihmisen Pojan nousevan sinne, missä hän oli ennen!
\par 63 Henki on se, joka eläväksi tekee; ei liha mitään hyödytä. Ne sanat, jotka minä olen teille puhunut, ovat henki ja ovat elämä.
\par 64 Mutta teissä on muutamia, jotka eivät usko." Sillä Jeesus tiesi alusta asti, ketkä ne olivat, jotka eivät uskoneet, ja kuka se oli, joka oli kavaltava hänet.
\par 65 Ja hän sanoi: "Sentähden minä olen sanonut teille, ettei kukaan voi tulla minun tyköni, ellei minun Isäni sitä hänelle anna".
\par 66 Tämän tähden monet hänen opetuslapsistaan vetäytyivät pois eivätkä enää vaeltaneet hänen kanssansa.
\par 67 Niin Jeesus sanoi niille kahdelletoista: "Tahdotteko tekin mennä pois?"
\par 68 Simon Pietari vastasi hänelle: "Herra, kenen tykö me menisimme? Sinulla on iankaikkisen elämän sanat;
\par 69 ja me uskomme ja ymmärrämme, että sinä olet Jumalan Pyhä."
\par 70 Jeesus vastasi heille: "Enkö minä ole valinnut teitä, te kaksitoista? Ja yksi teistä on perkele."
\par 71 Mutta sen hän sanoi Juudaasta, Simon Iskariotin pojasta; sillä tämä oli hänet kavaltava ja oli yksi niistä kahdestatoista.

\chapter{7}

\par 1 Ja sen jälkeen Jeesus vaelsi ympäri Galileassa; sillä hän ei tahtonut vaeltaa Juudeassa, koska juutalaiset tavoittelivat häntä tappaaksensa.
\par 2 Ja juutalaisten juhla, lehtimajanjuhla, oli lähellä.
\par 3 Niin hänen veljensä sanoivat hänelle: "Lähde täältä ja mene Juudeaan, että myös sinun opetuslapsesi näkisivät sinun tekosi, joita sinä teet;
\par 4 sillä ei kukaan, joka itse tahtoo tulla julki, tee mitään salassa. Koska sinä näitä tekoja teet, niin ilmoita itsesi maailmalle."
\par 5 Sillä hänen veljensäkään eivät häneen uskoneet.
\par 6 Niin Jeesus sanoi heille: "Minun aikani ei ole vielä tullut; mutta teille aika on aina sovelias.
\par 7 Teitä ei maailma voi vihata, mutta minua se vihaa, sillä minä todistan siitä, että sen teot ovat pahat.
\par 8 Menkää te ylös juhlille; minä en vielä mene näille juhlille, sillä minun aikani ei ole vielä täyttynyt.
\par 9 Tämän hän sanoi heille ja jäi Galileaan.
\par 10 Mutta kun hänen veljensä olivat menneet juhlille, silloin hänkin meni sinne, ei julki, vaan ikäänkuin salaa.
\par 11 Niin juutalaiset etsivät häntä juhlan aikana ja sanoivat: "Missä hän on?"
\par 12 Ja hänestä oli paljon kiistelyä kansassa; muutamat sanoivat: "Hän on hyvä", mutta toiset sanoivat: "Ei ole, vaan hän villitsee kansan".
\par 13 Ei kuitenkaan kukaan puhunut hänestä julkisesti, koska he pelkäsivät juutalaisia.
\par 14 Mutta kun jo puoli juhlaa oli kulunut, meni Jeesus ylös pyhäkköön ja opetti.
\par 15 Niin juutalaiset ihmettelivät ja sanoivat: "Kuinka tämä osaa kirjoituksia, vaikkei ole oppia saanut?"
\par 16 Jeesus vastasi heille ja sanoi: "Minun oppini ei ole minun, vaan hänen, joka on minut lähettänyt.
\par 17 Jos joku tahtoo tehdä hänen tahtonsa, tulee hän tuntemaan, onko tämä oppi Jumalasta, vai puhunko minä omiani.
\par 18 Joka omiaan puhuu, se pyytää omaa kunniaansa, mutta joka pyytää lähettäjänsä kunniaa, se on totinen, eikä hänessä ole vääryyttä.
\par 19 Eikö Mooses ole antanut teille lakia? Ja kukaan teistä ei lakia täytä. Miksi tavoittelette minua tappaaksenne?"
\par 20 Kansa vastasi: "Sinussa on riivaaja; kuka sinua tavoittelee tappaaksensa?"
\par 21 Jeesus vastasi ja sanoi heille: "Yhden teon minä tein, ja te kaikki kummastelette.
\par 22 Mooses antoi teille ympärileikkauksen - ei niin, että se olisi Moosekselta, vaan se on isiltä - ja sapattinakin te ympärileikkaatte ihmisen.
\par 23 Sentähden: jos ihminen saa ympärileikkauksen sapattina, ettei Mooseksen lakia rikottaisi, miksi te olette vihoissanne minulle siitä, että minä tein koko ihmisen terveeksi sapattina?
\par 24 Älkää tuomitko näön mukaan, vaan tuomitkaa oikea tuomio."
\par 25 Niin muutamat jerusalemilaisista sanoivat: "Eikö tämä ole se, jota he tavoittelevat tappaaksensa?
\par 26 Ja katso, hän puhuu vapaasti, eivätkä he sano hänelle mitään. Olisivatko hallitusmiehet tosiaan saaneet tietoonsa, että tämä on Kristus?
\par 27 Kuitenkin, me tiedämme, mistä tämä on; mutta kun Kristus tulee, niin ei kukaan tiedä, mistä hän on."
\par 28 Niin Jeesus puhui pyhäkössä suurella äänellä, opetti ja sanoi: "Te tunnette minut ja tiedätte, mistä minä olen; ja itsestäni minä en ole tullut, vaan hän, joka minut on lähettänyt, on oikea lähettäjä, ja häntä te ette tunne.
\par 29 Minä tunnen hänet, sillä hänestä minä olen, ja hän on minut lähettänyt."
\par 30 Niin heillä oli halu ottaa hänet kiinni; mutta ei kukaan käynyt häneen käsiksi, sillä hänen hetkensä ei ollut vielä tullut.
\par 31 Mutta monet kansasta uskoivat häneen ja sanoivat: "Kun Kristus tulee, tehneekö hän enemmän tunnustekoja, kuin tämä on tehnyt?"
\par 32 Fariseukset kuulivat kansan näin kiistelevän hänestä; niin ylipapit ja fariseukset lähettivät palvelijoita ottamaan häntä kiinni.
\par 33 Mutta Jeesus sanoi: "Minä olen vielä vähän aikaa teidän kanssanne, ja sitten minä menen pois hänen tykönsä, joka on minut lähettänyt.
\par 34 Silloin te etsitte minua, mutta ette löydä; ja missä minä olen, sinne te ette voi tulla."
\par 35 Niin juutalaiset sanoivat keskenään: "Minne tämä aikoo mennä, koska emme voi löytää häntä? Eihän vain aikone mennä niiden luo, jotka asuvat hajallaan kreikkalaisten keskellä, ja opettaa kreikkalaisia?
\par 36 Mitä tämä sana on, jonka hän sanoi: 'Te etsitte minua, mutta ette löydä', ja: 'Missä minä olen, sinne te ette voi tulla'?"
\par 37 Mutta juhlan viimeisenä, suurena päivänä Jeesus seisoi ja huusi ja sanoi: "Jos joku janoaa, niin tulkoon minun tyköni ja juokoon.
\par 38 Joka uskoo minuun, hänen sisimmästään on, niinkuin Raamattu sanoo, juokseva elävän veden virrat."
\par 39 Mutta sen hän sanoi Hengestä, joka niiden piti saaman, jotka uskoivat häneen; sillä Henki ei ollut vielä tullut, koska Jeesus ei vielä ollut kirkastettu.
\par 40 Niin muutamat kansasta, kuultuaan nämä sanat, sanoivat: "Tämä on totisesti se profeetta".
\par 41 Toiset sanoivat: "Tämä on Kristus". Mutta toiset sanoivat: "Ei suinkaan Kristus tule Galileasta?
\par 42 Eikö Raamattu sano, että Kristus on oleva Daavidin jälkeläisiä ja tuleva pienestä Beetlehemin kaupungista, jossa Daavid oli?"
\par 43 Niin syntyi kansassa eripuraisuutta hänen tähtensä.
\par 44 Ja muutamat heistä tahtoivat ottaa hänet kiinni. Mutta ei kukaan käynyt häneen käsiksi.
\par 45 Niin palvelijat palasivat ylipappien ja fariseusten luo, ja nämä sanoivat heille: "Miksi ette tuoneet häntä tänne?"
\par 46 Palvelijat vastasivat: "Ei ole koskaan ihminen puhunut niin, kuin se mies puhuu".
\par 47 Niin fariseukset vastasivat heille: "Oletteko tekin eksytetyt?
\par 48 Onko kukaan hallitusmiehistä uskonut häneen tai kukaan fariseuksista?
\par 49 Mutta tuo kansa, joka ei lakia tunne, on kirottu."
\par 50 Niin Nikodeemus, joka ennen oli käynyt Jeesuksen luona ja joka oli yksi heistä, sanoi heille:
\par 51 "Tuomitseeko lakimme ketään, ennenkuin häntä on kuulusteltu ja saatu tietää, mitä hän on tehnyt?"
\par 52 He vastasivat ja sanoivat hänelle: "Oletko sinäkin Galileasta? Tutki ja näe, ettei Galileasta nouse profeettaa."
\par 53 Ja he menivät kukin kotiinsa.

\chapter{8}

\par 1 Mutta Jeesus meni Öljymäelle.
\par 2 Ja varhain aamulla hän taas saapui pyhäkköön, ja kaikki kansa tuli hänen luoksensa; ja hän istuutui ja opetti heitä.
\par 3 Silloin kirjanoppineet ja fariseukset toivat hänen luoksensa aviorikoksesta kiinniotetun naisen, asettivat hänet keskelle
\par 4 ja sanoivat Jeesukselle: "Opettaja, tämä nainen on tavattu itse teosta, aviorikosta tekemästä.
\par 5 Ja Mooses on laissa antanut meille käskyn, että tuommoiset on kivitettävä. Mitäs sinä sanot?"
\par 6 Mutta sen he sanoivat kiusaten häntä, päästäkseen häntä syyttämään. Silloin Jeesus kumartui alas ja kirjoitti sormellaan maahan.
\par 7 Mutta kun he yhä edelleen kysyivät häneltä, ojensi hän itsensä ja sanoi heille: "Joka teistä on synnitön, se heittäköön häntä ensimmäisenä kivellä".
\par 8 Ja taas hän kumartui alas ja kirjoitti maahan.
\par 9 Kun he tämän kuulivat ja heidän omatuntonsa todisti heidät syyllisiksi, menivät he pois, toinen toisensa perästä, vanhimmista alkaen viimeisiin asti; ja siihen jäi ainoastaan Jeesus sekä nainen, joka seisoi hänen edessään.
\par 10 Ja kun Jeesus ojensi itsensä eikä nähnyt ketään muuta kuin naisen, sanoi hän hänelle: "Nainen, missä ne ovat, sinun syyttäjäsi? Eikö kukaan ole sinua tuominnut?"
\par 11 Hän vastasi: "Herra, ei kukaan". Niin Jeesus sanoi hänelle: "En minäkään sinua tuomitse; mene, äläkä tästedes enää syntiä tee".
\par 12 Niin Jeesus taas puhui heille sanoen: "Minä olen maailman valkeus; joka minua seuraa, se ei pimeydessä vaella, vaan hänellä on oleva elämän valkeus".
\par 13 Niin fariseukset sanoivat hänelle: "Sinä todistat itsestäsi; sinun todistuksesi ei ole pätevä".
\par 14 Jeesus vastasi ja sanoi heille: "Vaikka minä todistankin itsestäni, on todistukseni pätevä, sillä minä tiedän, mistä minä olen tullut ja mihin minä menen; mutta te ette tiedä, mistä minä tulen, ettekä, mihin minä menen.
\par 15 Te tuomitsette lihan mukaan; minä en tuomitse ketään.
\par 16 Ja vaikka minä tuomitsisinkin, niin minun tuomioni olisi oikea, sillä minä en ole yksinäni, vaan minä ja hän, joka on minut lähettänyt.
\par 17 Onhan teidän laissannekin kirjoitettuna, että kahden ihmisen todistus on pätevä.
\par 18 Minä olen se, joka todistan itsestäni, ja minusta todistaa myös Isä, joka on minut lähettänyt."
\par 19 Niin he sanoivat hänelle: "Missä sinun isäsi on?" Jeesus vastasi: "Te ette tunne minua ettekä minun Isääni; jos te tuntisitte minut, niin te tuntisitte myös minun Isäni".
\par 20 Nämä sanat Jeesus puhui uhriarkun ääressä, opettaessaan pyhäkössä; eikä kukaan ottanut häntä kiinni, sillä hänen hetkensä ei ollut vielä tullut.
\par 21 Niin Jeesus taas sanoi heille: "Minä menen pois, ja te etsitte minua, ja te kuolette syntiinne. Mihin minä menen, sinne te ette voi tulla."
\par 22 Niin juutalaiset sanoivat: "Ei kai hän aikone tappaa itseänsä, koska sanoo: 'Mihin minä menen, sinne te ette voi tulla'?"
\par 23 Ja hän sanoi heille: "Te olette alhaalta, minä olen ylhäältä; te olette tästä maailmasta, minä en ole tästä maailmasta.
\par 24 Sentähden minä sanoin teille, että te kuolette synteihinne; sillä ellette usko minua siksi, joka minä olen, niin te kuolette synteihinne."
\par 25 Niin he sanoivat hänelle: "Kuka sinä olet?" Jeesus sanoi heille: "Juuri se, mitä minä puhunkin teille.
\par 26 Paljon on minulla teistä puhuttavaa ja teissä tuomittavaa; mutta hän, joka on minut lähettänyt, on totinen, ja minkä minä olen kuullut häneltä, sen minä puhun maailman kuulla."
\par 27 Mutta he eivät ymmärtäneet, että hän puhui heille Isästä.
\par 28 Niin Jeesus sanoi heille: "Kun olette ylentäneet Ihmisen Pojan, silloin te ymmärrätte, että minä olen se, joka minä olen, ja etten minä itsestäni tee mitään, vaan puhun tätä sen mukaan, kuin minun Isäni on minulle opettanut.
\par 29 Ja hän, joka on minut lähettänyt, on minun kanssani; hän ei ole jättänyt minua yksinäni, koska minä aina teen sitä, mikä hänelle on otollista."
\par 30 Kun hän näin puhui, uskoivat monet häneen.
\par 31 Niin Jeesus sanoi niille juutalaisille, jotka uskoivat häneen: "Jos te pysytte minun sanassani, niin te totisesti olette minun opetuslapsiani;
\par 32 ja te tulette tuntemaan totuuden, ja totuus on tekevä teidät vapaiksi".
\par 33 He vastasivat hänelle: "Me olemme Aabrahamin jälkeläisiä emmekä ole koskaan olleet kenenkään orjia. Kuinka sinä sitten sanot: 'Te tulette vapaiksi'?"
\par 34 Jeesus vastasi heille: "Totisesti, totisesti minä sanon teille: jokainen, joka tekee syntiä, on synnin orja.
\par 35 Mutta orja ei pysy talossa iäti; Poika pysyy iäti.
\par 36 Jos siis Poika tekee teidät vapaiksi, niin te tulette todellisesti vapaiksi.
\par 37 Minä tiedän, että te olette Aabrahamin jälkeläisiä; mutta te tavoittelette minua tappaaksenne, koska minun sanani ei saa tilaa teissä.
\par 38 Minä puhun, mitä minä olen nähnyt Isäni tykönä; niin tekin teette, mitä olette kuulleet omalta isältänne."
\par 39 He vastasivat ja sanoivat hänelle: "Aabraham on meidän isämme". Jeesus sanoi heille: "Jos olisitte Aabrahamin lapsia, niin te tekisitte Aabrahamin tekoja.
\par 40 Mutta nyt te tavoittelette minua tappaaksenne, miestä, joka on puhunut teille totuuden, jonka hän on kuullut Jumalalta. Niin ei Aabraham tehnyt.
\par 41 Te teette isänne tekoja." He sanoivat hänelle: "Me emme ole aviorikoksesta syntyneitä; meillä on yksi Isä, Jumala".
\par 42 Jeesus sanoi heille: "Jos Jumala olisi teidän Isänne, niin te rakastaisitte minua, sillä minä olen Jumalasta lähtenyt ja tullut; en minä ole itsestäni tullut, vaan hän on minut lähettänyt.
\par 43 Minkätähden te ette ymmärrä minun puhettani? Sentähden, että te ette kärsi kuulla minun sanaani.
\par 44 Te olette isästä perkeleestä, ja isänne himoja te tahdotte noudattaa. Hän on ollut murhaaja alusta asti, ja totuudessa hän ei pysy, koska hänessä ei totuutta ole. Kun hän puhuu valhetta, niin hän puhuu omaansa, sillä hän on valhettelija ja sen isä.
\par 45 Mutta minua te ette usko, sentähden että minä sanon totuuden.
\par 46 Kuka teistä voi näyttää minut syypääksi syntiin? Jos minä totuutta puhun, miksi ette minua usko?
\par 47 Joka on Jumalasta, se kuulee Jumalan sanat. Sentähden te ette kuule, koska ette ole Jumalasta."
\par 48 Niin juutalaiset vastasivat ja sanoivat hänelle: "Emmekö ole oikeassa, kun sanomme, että sinä olet samarialainen ja että sinussa on riivaaja?"
\par 49 Jeesus vastasi: "Minussa ei ole riivaajaa, vaan minä kunnioitan Isääni, ja te häpäisette minua.
\par 50 Mutta minä en etsi omaa kunniaani; yksi on, joka etsii ja tuomitsee.
\par 51 Totisesti, totisesti minä sanon teille: jos joku pitää minun sanani, hän ei ikinä näe kuolemaa."
\par 52 Juutalaiset sanoivat hänelle: "Nyt me ymmärrämme, että sinussa on riivaaja. Aabraham on kuollut ja profeetat, ja sinä sanot: 'Jos joku pitää minun sanani, hän ei ikinä maista kuolemaa'.
\par 53 Oletko sinä suurempi kuin isämme Aabraham, joka on kuollut? Ja profeetat ovat kuolleet; keneksi sinä itsesi teet?"
\par 54 Jeesus vastasi: "Jos minä itse itselleni otan kunnian, niin minun kunniani ei ole mitään. Minun Isäni on se, joka minulle kunnian antaa, hän, josta te sanotte: 'Hän on meidän Jumalamme',
\par 55 ettekä tunne häntä; mutta minä tunnen hänet. Ja jos sanoisin, etten tunne häntä, niin minä olisin valhettelija niinkuin tekin; mutta minä tunnen hänet ja pidän hänen sanansa.
\par 56 Aabraham, teidän isänne, riemuitsi siitä, että hän oli näkevä minun päiväni; ja hän näki sen ja iloitsi."
\par 57 Niin juutalaiset sanoivat hänelle: "Et ole vielä viidenkymmenen vuoden vanha, ja olet nähnyt Aabrahamin!"
\par 58 Jeesus sanoi heille: "Totisesti, totisesti minä sanon teille: ennenkuin Aabraham syntyi, olen minä ollut".
\par 59 Silloin he poimivat kiviä heittääksensä häntä niillä; mutta Jeesus lymysi ja lähti pyhäköstä.

\chapter{9}

\par 1 Ja ohi kulkiessaan hän näki miehen, joka syntymästään saakka oli ollut sokea.
\par 2 Ja hänen opetuslapsensa kysyivät häneltä sanoen: "Rabbi, kuka teki syntiä, tämäkö vai hänen vanhempansa, että hänen piti sokeana syntymän?"
\par 3 Jeesus vastasi: "Ei tämä tehnyt syntiä eivätkä hänen vanhempansa, vaan Jumalan tekojen piti tuleman hänessä julki.
\par 4 Niin kauan kuin päivä on, tulee meidän tehdä hänen tekojansa, joka on minut lähettänyt; tulee yö, jolloin ei kukaan voi työtä tehdä.
\par 5 Niin kauan kuin minä maailmassa olen, olen minä maailman valkeus."
\par 6 Tämän sanottuaan hän sylki maahan ja teki syljestä tahtaan ja siveli tahtaan hänen silmilleen
\par 7 ja sanoi hänelle: "Mene ja peseydy Siiloan lammikossa" - se on käännettynä: lähetetty. - Niin hän meni ja peseytyi ja palasi näkevänä.
\par 8 Silloin naapurit ja ne, jotka ennen olivat nähneet hänet kerjääjänä, sanoivat: "Eikö tämä ole se, joka istui ja kerjäsi?"
\par 9 Toiset sanoivat: "Hän se on", toiset sanoivat: "Ei ole, vaan hän on hänen näköisensä". Hän itse sanoi: "Minä se olen".
\par 10 Niin he sanoivat hänelle: "Miten sinun silmäsi ovat auenneet?"
\par 11 Hän vastasi: "Se mies, jota kutsutaan Jeesukseksi, teki tahtaan ja voiteli minun silmäni ja sanoi minulle: 'Mene ja peseydy Siiloan lammikossa'; niin minä menin ja peseydyin ja sain näköni".
\par 12 He sanoivat hänelle: "Missä hän on?" Hän vastasi: "En tiedä".
\par 13 Niin he veivät hänet, joka ennen oli ollut sokea, fariseusten luo.
\par 14 Ja se päivä, jona Jeesus teki tahtaan ja avasi hänen silmänsä, oli sapatti.
\par 15 Niin myöskin fariseukset kysyivät häneltä, miten hän oli saanut näkönsä. Ja hän sanoi heille: "Hän siveli tahtaan minun silmilleni, ja minä peseydyin, ja nyt minä näen".
\par 16 Niin muutamat fariseuksista sanoivat: "Se mies ei ole Jumalasta, koska hän ei pidä sapattia". Toiset sanoivat: "Kuinka voi syntinen ihminen tehdä senkaltaisia tunnustekoja?" Ja he olivat keskenänsä eri mieltä.
\par 17 Niin he taas sanoivat sokealle: "Mitä sinä sanot hänestä, koskapa hän avasi sinun silmäsi?" Ja hän sanoi: "Hän on profeetta".
\par 18 Mutta juutalaiset eivät uskoneet hänestä, että hän oli ollut sokea ja saanut näkönsä, ennenkuin kutsuivat sen näkönsä saaneen vanhemmat
\par 19 ja kysyivät heiltä sanoen: "Onko tämä teidän poikanne, jonka sanotte sokeana syntyneen? Kuinka hän sitten nyt näkee?"
\par 20 Hänen vanhempansa vastasivat ja sanoivat: "Me tiedämme, että tämä on meidän poikamme ja että hän on sokeana syntynyt;
\par 21 mutta kuinka hän nyt näkee, emme tiedä; emme myöskään tiedä, kuka on avannut hänen silmänsä. Kysykää häneltä; hänellä on kyllin ikää, puhukoon itse puolestansa."
\par 22 Näin hänen vanhempansa sanoivat, koska pelkäsivät juutalaisia. Sillä juutalaiset olivat jo sopineet keskenään, että se, joka tunnusti hänet Kristukseksi, oli erotettava synagoogasta.
\par 23 Sentähden hänen vanhempansa sanoivat: "Hänellä on kyllin ikää, kysykää häneltä".
\par 24 Niin he kutsuivat toistamiseen miehen, joka oli ollut sokea, ja sanoivat hänelle: "Anna kunnia Jumalalle; me tiedämme, että se mies on syntinen".
\par 25 Hän vastasi: "Onko hän syntinen, sitä en tiedä; sen vain tiedän, että minä, joka olin sokea, nyt näen".
\par 26 Niin he sanoivat hänelle: "Mitä hän sinulle teki? Miten hän avasi sinun silmäsi?"
\par 27 Hän vastasi heille: "Johan minä teille sanoin, ettekä te kuulleet. Miksi taas tahdotte sitä kuulla? Tahdotteko tekin ruveta hänen opetuslapsiksensa?"
\par 28 Niin he herjasivat häntä ja sanoivat: "Sinä olet hänen opetuslapsensa, mutta me olemme Mooseksen opetuslapsia.
\par 29 Me tiedämme Jumalan puhuneen Moosekselle, mutta mistä tämä on, sitä emme tiedä."
\par 30 Mies vastasi ja sanoi heille: "Sehän tässä on ihmeellistä, että te ette tiedä, mistä hän on, ja kuitenkin hän on avannut minun silmäni.
\par 31 Me tiedämme, ettei Jumala kuule syntisiä; vaan joka on jumalaapelkääväinen ja tekee hänen tahtonsa, sitä hän kuulee.
\par 32 Ei ole maailman alusta kuultu, että kukaan on avannut sokeana syntyneen silmät.
\par 33 Jos hän ei olisi Jumalasta, ei hän voisi mitään tehdä."
\par 34 He vastasivat ja sanoivat hänelle: "Sinä olet kokonaan synneissä syntynyt, ja sinä tahdot opettaa meitä!" Ja he ajoivat hänet ulos.
\par 35 Ja Jeesus sai kuulla heidän ajaneen hänet ulos; ja hänet tavatessaan hän sanoi hänelle: "Uskotko sinä Jumalan Poikaan?"
\par 36 Hän vastasi ja sanoi: "Herra, kuka hän on, että minä häneen uskoisin?"
\par 37 Jeesus sanoi hänelle: "Sinä olet hänet nähnyt, ja hän on se, joka sinun kanssasi puhuu".
\par 38 Niin hän sanoi: "Herra, minä uskon"; ja hän kumartaen rukoili häntä.
\par 39 Ja Jeesus sanoi: "Tuomioksi minä olen tullut tähän maailmaan, että ne, jotka eivät näe, näkisivät, ja ne, jotka näkevät, tulisivat sokeiksi".
\par 40 Ja muutamat fariseukset, jotka olivat siinä häntä lähellä, kuulivat tämän ja sanoivat hänelle: "Olemmeko mekin sokeat?"
\par 41 Jeesus sanoi heille: "Jos te olisitte sokeat, ei teillä olisi syntiä; mutta nyt te sanotte: 'Me näemme'; sentähden teidän syntinne pysyy".

\chapter{10}

\par 1 "Totisesti, totisesti minä sanon teille: joka ei mene ovesta lammastarhaan, vaan nousee sinne muualta, se on varas ja ryöväri.
\par 2 Mutta joka menee ovesta sisälle, se on lammasten paimen.
\par 3 Hänelle ovenvartija avaa, ja lampaat kuulevat hänen ääntänsä; ja hän kutsuu omat lampaansa nimeltä ja vie heidät ulos.
\par 4 Ja laskettuaan kaikki omansa ulos hän kulkee niiden edellä, ja lampaat seuraavat häntä, sillä ne tuntevat hänen äänensä.
\par 5 Mutta vierasta ne eivät seuraa, vaan pakenevat häntä, koska eivät tunne vierasten ääntä."
\par 6 Tämän kuvauksen Jeesus puhui heille; mutta he eivät ymmärtäneet, mitä hänen puheensa tarkoitti.
\par 7 Niin Jeesus vielä sanoi heille: "Totisesti, totisesti minä sanon teille: minä olen lammasten ovi.
\par 8 Kaikki, jotka ovat tulleet ennen minua, ovat varkaita ja ryöväreitä; mutta lampaat eivät ole heitä kuulleet.
\par 9 Minä olen ovi; jos joku minun kauttani menee sisälle, niin hän pelastuu, ja hän on käyvä sisälle ja käyvä ulos ja löytävä laitumen.
\par 10 Varas ei tule muuta kuin varastamaan ja tappamaan ja tuhoamaan. Minä olen tullut, että heillä olisi elämä ja olisi yltäkylläisyys.
\par 11 Minä olen se hyvä paimen. Hyvä paimen antaa henkensä lammasten edestä.
\par 12 Mutta palkkalainen, joka ei ole paimen ja jonka omia lampaat eivät ole, kun hän näkee suden tulevan, niin hän jättää lampaat ja pakenee; ja susi ryöstää ja hajottaa ne.
\par 13 Hän pakenee, sillä hän on palkattu eikä välitä lampaista.
\par 14 Minä olen se hyvä paimen, ja minä tunnen omani, ja minun omani tuntevat minut,
\par 15 niinkuin Isä tuntee minut ja minä tunnen Isän; ja minä annan henkeni lammasten edestä.
\par 16 Minulla on myös muita lampaita, jotka eivät ole tästä lammastarhasta; myös niitä tulee minun johdattaa, ja ne saavat kuulla minun ääneni, ja on oleva yksi lauma ja yksi paimen.
\par 17 Sentähden Isä minua rakastaa, koska minä annan henkeni, että minä sen jälleen ottaisin.
\par 18 Ei kukaan sitä minulta ota, vaan minä annan sen itsestäni. Minulla on valta antaa se, ja minulla on valta ottaa se jälleen; sen käskyn minä olen saanut Isältäni."
\par 19 Niin syntyi taas erimielisyys juutalaisten kesken näiden sanain tähden.
\par 20 Ja useat heistä sanoivat: "Hänessä on riivaaja, ja hän on järjiltään; mitä te häntä kuuntelette?"
\par 21 Toiset sanoivat: "Nämä eivät ole riivatun sanoja; eihän riivaaja voi avata sokeain silmiä?"
\par 22 Sitten oli temppelin vihkimisen muistojuhla Jerusalemissa, ja oli talvi.
\par 23 Ja Jeesus käyskeli pyhäkössä, Salomon pylväskäytävässä.
\par 24 Niin juutalaiset ympäröivät hänet ja sanoivat hänelle: "Kuinka kauan sinä pidät meidän mieltämme kiihdyksissä? Jos sinä olet Kristus, niin sano se meille suoraan."
\par 25 Jeesus vastasi heille: "Minä olen sanonut sen teille, ja te ette usko. Ne teot, joita minä teen Isäni nimessä, ne todistavat minusta.
\par 26 Mutta te ette usko, sillä te ette ole minun lampaitani.
\par 27 Minun lampaani kuulevat minun ääntäni, ja minä tunnen ne, ja ne seuraavat minua.
\par 28 Ja minä annan heille iankaikkisen elämän, ja he eivät ikinä huku, eikä kukaan ryöstä heitä minun kädestäni.
\par 29 Minun Isäni, joka on heidät minulle antanut, on suurempi kaikkia, eikä kukaan voi ryöstää heitä minun Isäni kädestä.
\par 30 Minä ja Isä olemme yhtä."
\par 31 Niin juutalaiset ottivat taas kiviä maasta kivittääksensä hänet.
\par 32 Jeesus vastasi heille: "Minä olen näyttänyt teille monta hyvää tekoa, jotka ovat Isästä; mikä niistä on se, jonka tähden te tahdotte minut kivittää?"
\par 33 Juutalaiset vastasivat hänelle: "Hyvän teon tähden me emme sinua kivitä, vaan jumalanpilkan tähden, ja koska sinä, joka olet ihminen, teet itsesi Jumalaksi".
\par 34 Jeesus vastasi heille: "Eikö teidän laissanne ole kirjoitettuna: 'Minä sanoin: te olette jumalia'?
\par 35 Jos hän sanoo jumaliksi niitä, joille Jumalan sana tuli - ja Raamattu ei voi raueta tyhjiin -
\par 36 niin kuinka te sanotte sille, jonka Isä on pyhittänyt ja lähettänyt maailmaan: 'Sinä pilkkaat Jumalaa', sentähden että minä sanoin: 'Minä olen Jumalan Poika'?
\par 37 Jos minä en tee Isäni tekoja, älkää uskoko minua.
\par 38 Mutta jos minä niitä teen, niin, vaikka ette uskoisikaan minua, uskokaa minun tekojani, että tulisitte tuntemaan ja ymmärtäisitte Isän olevan minussa ja minun olevan Isässä."
\par 39 Niin he taas tahtoivat ottaa hänet kiinni, mutta hän lähti pois heidän käsistänsä.
\par 40 Ja hän meni taas Jordanin tuolle puolelle siihen paikkaan, missä Johannes ensin kastoi, ja viipyi siellä.
\par 41 Ja monet tulivat hänen tykönsä ja sanoivat: "Johannes ei tehnyt yhtäkään tunnustekoa; mutta kaikki, mitä Johannes sanoi tästä, on totta".
\par 42 Ja monet siellä uskoivat häneen.

\chapter{11}

\par 1 Ja eräs mies, Lasarus, Betaniasta, Marian ja hänen sisarensa Martan kylästä, oli sairaana.
\par 2 Ja tämä Maria oli se, joka hajuvoiteella voiteli Herran ja pyyhki hiuksillaan hänen jalkansa; ja Lasarus, joka sairasti, oli hänen veljensä.
\par 3 Niin sisaret lähettivät Jeesukselle tämän sanan: "Herra, katso, se, joka on sinulle rakas, sairastaa".
\par 4 Mutta sen kuultuaan Jeesus sanoi: "Ei tämä tauti ole kuolemaksi, vaan Jumalan kunniaksi, että Jumalan Poika sen kautta kirkastuisi".
\par 5 Ja Jeesus rakasti Marttaa ja hänen sisartaan ja Lasarusta.
\par 6 Kun hän siis kuuli hänen sairastavan, viipyi hän siinä paikassa, missä hän oli, vielä kaksi päivää;
\par 7 mutta niiden kuluttua hän sanoi opetuslapsilleen: "Menkäämme taas Juudeaan".
\par 8 Opetuslapset sanoivat hänelle: "Rabbi, äsken juutalaiset yrittivät kivittää sinut, ja taas sinä menet sinne!"
\par 9 Jeesus vastasi: "Eikö päivässä ole kaksitoista hetkeä? Joka vaeltaa päivällä, se ei loukkaa itseänsä, sillä hän näkee tämän maailman valon.
\par 10 Mutta joka vaeltaa yöllä, se loukkaa itsensä, sillä ei hänessä ole valoa."
\par 11 Näin hän puhui, ja sitten hän sanoi heille: "Ystävämme Lasarus nukkuu, mutta minä menen herättämään hänet unesta".
\par 12 Niin opetuslapset sanoivat hänelle: "Herra, jos hän nukkuu, niin hän tulee terveeksi".
\par 13 Mutta Jeesus puhui hänen kuolemastaan; he taas luulivat hänen puhuneen unessa-nukkumisesta.
\par 14 Silloin Jeesus sanoi heille suoraan: "Lasarus on kuollut,
\par 15 ja minä iloitsen teidän tähtenne siitä, etten ollut siellä, jotta te uskoisitte; mutta menkäämme hänen tykönsä".
\par 16 Niin Tuomas, jota sanottiin Didymukseksi, sanoi toisille opetuslapsille: "Menkäämme mekin sinne, kuollaksemme hänen kanssansa".
\par 17 Niin Jeesus tuli ja sai tietää, että hän jo neljä päivää oli ollut haudassa.
\par 18 Ja Betania oli lähellä Jerusalemia, noin viidentoista vakomitan päässä.
\par 19 Ja useita juutalaisia oli tullut Martan ja Marian luokse lohduttamaan heitä heidän veljensä kuolemasta.
\par 20 Kun Martta kuuli, että Jeesus oli tulossa, meni hän häntä vastaan; mutta Maria istui kotona.
\par 21 Ja Martta sanoi Jeesukselle: "Herra, jos sinä olisit ollut täällä, niin minun veljeni ei olisi kuollut.
\par 22 Mutta nytkin minä tiedän, että Jumala antaa sinulle kaiken, mitä sinä Jumalalta anot."
\par 23 Jeesus sanoi hänelle: "Sinun veljesi on nouseva ylös".
\par 24 Martta sanoi hänelle: "Minä tiedän hänen nousevan ylösnousemuksessa, viimeisenä päivänä".
\par 25 Jeesus sanoi hänelle: "Minä olen ylösnousemus ja elämä; joka uskoo minuun, se elää, vaikka olisi kuollut.
\par 26 Eikä yksikään, joka elää ja uskoo minuun, ikinä kuole. Uskotko sen?"
\par 27 Hän sanoi hänelle: "Uskon, Herra; minä uskon, että sinä olet Kristus, Jumalan Poika, se, joka oli tuleva maailmaan".
\par 28 Ja tämän sanottuaan hän meni ja kutsui salaa sisarensa Marian sanoen: "Opettaja on täällä ja kutsuu sinua".
\par 29 Kun Maria sen kuuli, nousi hän nopeasti ja meni hänen luoksensa.
\par 30 Mutta Jeesus ei ollut vielä saapunut kylään, vaan oli yhä siinä paikassa, missä Martta oli hänet kohdannut.
\par 31 Kun nyt juutalaiset, jotka olivat Marian kanssa huoneessa häntä lohduttamassa, näkivät hänen nopeasti nousevan ja lähtevän ulos, seurasivat he häntä, luullen hänen menevän haudalle, itkemään siellä.
\par 32 Kun siis Maria saapui sinne, missä Jeesus oli, ja näki hänet, lankesi hän hänen jalkojensa eteen ja sanoi hänelle: "Herra, jos sinä olisit ollut täällä, ei minun veljeni olisi kuollut".
\par 33 Kun Jeesus näki hänen itkevän ja hänen kanssaan tulleiden juutalaisten itkevän, joutui hän hengessään syvän liikutuksen valtaan ja vapisi;
\par 34 ja hän sanoi: "Mihin te panitte hänet?" He sanoivat hänelle: "Herra, tule ja katso".
\par 35 Ja Jeesus itki.
\par 36 Niin juutalaiset sanoivat: "Katso, kuinka rakas hän oli hänelle!"
\par 37 Mutta muutamat heistä sanoivat: "Eikö hän, joka avasi sokean silmät, olisi voinut tehdä sitäkin, ettei tämä olisi kuollut?"
\par 38 Niin Jeesus joutui taas liikutuksen valtaan ja meni haudalle; ja se oli luola, ja sen suulla oli kivi.
\par 39 Jeesus sanoi: "Ottakaa kivi pois". Martta, kuolleen sisar, sanoi hänelle: "Herra, hän haisee jo, sillä hän on ollut haudassa neljättä päivää".
\par 40 Jeesus sanoi hänelle: "Enkö minä sanonut sinulle, että jos uskoisit, niin sinä näkisit Jumalan kirkkauden?"
\par 41 Niin he ottivat kiven pois. Ja Jeesus loi silmänsä ylös ja sanoi: "Isä, minä kiitän sinua, että olet minua kuullut.
\par 42 Minä kyllä tiesin, että sinä minua aina kuulet; mutta kansan tähden, joka seisoo tässä ympärillä, minä tämän sanon, että he uskoisivat sinun lähettäneen minut."
\par 43 Ja sen sanottuansa hän huusi suurella äänellä: "Lasarus, tule ulos!"
\par 44 Ja kuollut tuli ulos, jalat ja kädet siteisiin käärittyinä, ja hänen kasvojensa ympärille oli kääritty hikiliina. Jeesus sanoi heille: "Päästäkää hänet ja antakaa hänen mennä".
\par 45 Niin useat juutalaisista, jotka olivat tulleet Marian luokse ja nähneet, mitä Jeesus teki, uskoivat häneen.
\par 46 Mutta muutamat heistä menivät fariseusten luo ja puhuivat heille, mitä Jeesus oli tehnyt.
\par 47 Niin ylipapit ja fariseukset kokosivat neuvoston ja sanoivat: "Mitä me teemme, sillä tuo mies tekee paljon tunnustekoja?
\par 48 Jos annamme hänen näin olla, niin kaikki uskovat häneen, ja roomalaiset tulevat ja ottavat meiltä sekä maan että kansan."
\par 49 Mutta eräs heistä, Kaifas, joka sinä vuonna oli ylimmäinen pappi, sanoi heille: "Te ette tiedä mitään
\par 50 ettekä ajattele, että teille on parempi, että yksi ihminen kuolee kansan edestä, kuin että koko kansa hukkuu".
\par 51 Mutta sitä hän ei sanonut itsestään, vaan koska hän oli sinä vuonna ylimmäinen pappi, hän ennusti, että Jeesus oli kuoleva kansan edestä,
\par 52 eikä ainoastaan tämän kansan edestä, vaan myös kootakseen yhdeksi hajalla olevat Jumalan lapset.
\par 53 Siitä päivästä lähtien oli heillä siis tehtynä päätös tappaa hänet.
\par 54 Sentähden Jeesus ei enää vaeltanut julkisesti juutalaisten keskellä, vaan lähti sieltä lähellä erämaata olevaan paikkaan, Efraim nimiseen kaupunkiin; ja siellä hän oleskeli opetuslapsineen.
\par 55 Mutta juutalaisten pääsiäinen oli lähellä, ja monet menivät maaseudulta ylös Jerusalemiin ennen pääsiäisjuhlaa, puhdistamaan itsensä.
\par 56 Ja he etsivät Jeesusta ja sanoivat toisilleen seisoessaan pyhäkössä: "Mitä arvelette? Eikö hän tullekaan juhlille?"
\par 57 Mutta ylipapit ja fariseukset olivat antaneet käskyjä, että jos joku tietäisi, missä hän oli, hänen oli annettava se ilmi, jotta he ottaisivat hänet kiinni.

\chapter{12}

\par 1 Kuusi päivää ennen pääsiäistä Jeesus saapui Betaniaan, jossa Lasarus asui, hän, jonka Jeesus oli herättänyt kuolleista.
\par 2 Siellä valmistettiin hänelle ateria, ja Martta palveli, mutta Lasarus oli yksi niistä, jotka olivat aterialla hänen kanssaan.
\par 3 Niin Maria otti naulan oikeata, kallisarvoista nardusvoidetta ja voiteli Jeesuksen jalat ja pyyhki ne hiuksillaan; ja huone tuli täyteen voiteen tuoksua.
\par 4 Silloin sanoi yksi hänen opetuslapsistaan, Juudas Iskariot, joka oli hänet kavaltava:
\par 5 "Miksi ei tätä voidetta myyty kolmeensataan denariin ja niitä annettu köyhille?"
\par 6 Mutta tätä hän ei sanonut sentähden, että olisi pitänyt huolta köyhistä, vaan sentähden, että hän oli varas ja että hän rahakukkaron hoitajana otti itselleen, mitä siihen oli pantu.
\par 7 Niin Jeesus sanoi: "Anna hänen olla, että hän saisi toimittaa tämän minun hautaamispäiväni varalle.
\par 8 Sillä köyhät teillä aina on keskuudessanne, mutta minua teillä ei ole aina."
\par 9 Silloin suuri joukko juutalaisia sai tietää, että hän oli siellä; ja he menivät sinne, ei ainoastaan Jeesuksen tähden, vaan myöskin nähdäkseen Lasaruksen, jonka hän oli herättänyt kuolleista.
\par 10 Mutta ylipapit päättivät tappaa Lasaruksenkin,
\par 11 koska monet juutalaiset hänen tähtensä menivät sinne ja uskoivat Jeesukseen.
\par 12 Seuraavana päivänä, kun suuri kansanjoukko, joka oli saapunut juhlille, kuuli, että Jeesus oli tulossa Jerusalemiin,
\par 13 ottivat he palmupuiden oksia ja menivät häntä vastaan ja huusivat: "Hoosianna, siunattu olkoon hän, joka tulee Herran nimeen, Israelin kuningas!"
\par 14 Ja saatuansa nuoren aasin Jeesus istui sen selkään, niinkuin kirjoitettu on:
\par 15 "Älä pelkää, tytär Siion; katso, sinun kuninkaasi tulee istuen aasin varsan selässä".
\par 16 Tätä hänen opetuslapsensa eivät aluksi ymmärtäneet; mutta kun Jeesus oli kirkastettu, silloin he muistivat, että tämä oli hänestä kirjoitettu ja että he olivat tämän hänelle tehneet.
\par 17 Niin kansa, joka oli ollut hänen kanssansa, kun hän kutsui Lasaruksen haudasta ja herätti hänet kuolleista, todisti hänestä.
\par 18 Sentähden kansa menikin häntä vastaan, koska he kuulivat, että hän oli tehnyt sen tunnusteon.
\par 19 Niin fariseukset sanoivat keskenään: "Te näette, ettette saa mitään aikaan; katso, koko maailma juoksee hänen perässään".
\par 20 Ja oli muutamia kreikkalaisia niiden joukosta, jotka tulivat ylös juhlaan rukoilemaan.
\par 21 Nämä menivät Filippuksen luo, joka oli Galilean Beetsaidasta, ja pyysivät häntä sanoen: "Herra, me haluamme nähdä Jeesuksen".
\par 22 Filippus meni ja sanoi sen Andreaalle; Andreas ja Filippus menivät ja sanoivat Jeesukselle.
\par 23 Mutta Jeesus vastasi heille sanoen: "Hetki on tullut, että Ihmisen Poika kirkastetaan.
\par 24 Totisesti, totisesti minä sanon teille: jos ei nisun jyvä putoa maahan ja kuole, niin se jää yksin; mutta jos se kuolee, niin se tuottaa paljon hedelmää.
\par 25 Joka elämäänsä rakastaa, kadottaa sen; mutta joka vihaa elämäänsä tässä maailmassa, hän on säilyttävä sen iankaikkiseen elämään.
\par 26 Jos joku minua palvelee, seuratkoon hän minua; ja missä minä olen, siellä on myös minun palvelijani oleva. Ja jos joku minua palvelee, niin Isä on kunnioittava häntä.
\par 27 Nyt minun sieluni on järkytetty; ja mitä pitäisi minun sanoman? Isä, pelasta minut tästä hetkestä. Kuitenkin: sitä varten minä olen tähän hetkeen tullut.
\par 28 Isä, kirkasta nimesi!" Niin taivaasta tuli ääni: "Minä olen sen kirkastanut, ja olen sen vielä kirkastava".
\par 29 Niin kansa, joka seisoi ja kuuli sen, sanoi ukkosen jylisseen. Toiset sanoivat: "Häntä puhutteli enkeli".
\par 30 Jeesus vastasi ja sanoi: "Ei tämä ääni tullut minun tähteni, vaan teidän tähtenne.
\par 31 Nyt käy tuomio tämän maailman ylitse; nyt tämän maailman ruhtinas pitää heitettämän ulos.
\par 32 Ja kun minut ylennetään maasta, niin minä vedän kaikki tyköni."
\par 33 Mutta sen hän sanoi antaen tietää, minkäkaltaisella kuolemalla hän oli kuoleva.
\par 34 Kansa vastasi hänelle: "Me olemme laista kuulleet, että Kristus pysyy iankaikkisesti; kuinka sinä sitten sanot, että Ihmisen Poika pitää ylennettämän? Kuka on se Ihmisen Poika?"
\par 35 Niin Jeesus sanoi heille: "Vielä vähän aikaa valkeus on teidän keskuudessanne. Vaeltakaa, niin kauan kuin teillä valkeus on, ettei pimeys saisi teitä valtaansa. Joka pimeässä vaeltaa, se ei tiedä, mihin hän menee.
\par 36 Niin kauan kuin teillä valkeus on, uskokaa valkeuteen, että te valkeuden lapsiksi tulisitte." Tämän Jeesus puhui ja meni pois ja kätkeytyi heiltä.
\par 37 Ja vaikka hän oli tehnyt niin monta tunnustekoa heidän nähtensä, eivät he uskoneet häneen,
\par 38 että kävisi toteen profeetta Esaiaan sana, jonka hän on sanonut: "Herra, kuka uskoo meidän saarnamme, ja kenelle Herran käsivarsi ilmoitetaan?"
\par 39 Sentähden he eivät voineet uskoa, koska Esaias on vielä sanonut:
\par 40 "Hän on sokaissut heidän silmänsä ja paaduttanut heidän sydämensä, että he eivät näkisi silmillään eivätkä ymmärtäisi sydämellään eivätkä kääntyisi ja etten minä heitä parantaisi".
\par 41 Tämän Esaias sanoi, kun hän näki hänen kirkkautensa ja puhui hänestä.
\par 42 Kuitenkin useat hallitusmiehistäkin uskoivat häneen, mutta fariseusten tähden he eivät sitä tunnustaneet, etteivät joutuisi synagoogasta erotetuiksi.
\par 43 Sillä he rakastivat ihmiskunniaa enemmän kuin Jumalan kunniaa.
\par 44 Mutta Jeesus huusi ja sanoi: "Joka uskoo minuun, se ei usko minuun, vaan häneen, joka on minut lähettänyt.
\par 45 Ja joka näkee minut, näkee hänet, joka on minut lähettänyt.
\par 46 Minä olen tullut valkeudeksi maailmaan, ettei yksikään, joka minuun uskoo, jäisi pimeyteen.
\par 47 Ja jos joku kuulee minun sanani eikä niitä noudata, niin häntä en minä tuomitse; sillä en minä ole tullut maailmaa tuomitsemaan, vaan pelastamaan maailman.
\par 48 Joka katsoo minut ylen eikä ota vastaan minun sanojani, hänellä on tuomitsijansa: se sana, jonka minä olen puhunut, se on tuomitseva hänet viimeisenä päivänä.
\par 49 Sillä en minä itsestäni ole puhunut, vaan Isä, joka on minut lähettänyt, on itse antanut minulle käskyn, mitä minun pitää sanoman ja mitä puhuman.
\par 50 Ja minä tiedän, että hänen käskynsä on iankaikkinen elämä. Sentähden, minkä minä puhun, sen minä puhun niin, kuin Isä on minulle sanonut."

\chapter{13}

\par 1 Mutta ennen pääsiäisjuhlaa, kun Jeesus tiesi hetkensä tulleen, että hän oli siirtyvä tästä maailmasta Isän tykö, niin hän, joka oli rakastanut omiansa, jotka maailmassa olivat, osoitti heille rakkautta loppuun asti.
\par 2 Ja ehtoollisella oltaessa, kun perkele jo oli pannut Juudas Iskariotin, Simonin pojan, sydämeen, että hän kavaltaisi Jeesuksen,
\par 3 niin Jeesus, tietäen, että Isä oli antanut kaikki hänen käsiinsä ja että hän oli lähtenyt Jumalan tyköä ja oli menevä Jumalan tykö,
\par 4 nousi ehtoolliselta ja riisui vaippansa, otti liinavaatteen ja vyötti sillä itsensä.
\par 5 Sitten hän kaatoi vettä pesumaljaan ja rupesi pesemään opetuslastensa jalkoja ja pyyhkimään niitä liinavaatteella, jolla oli vyöttäytynyt.
\par 6 Niin hän tuli Simon Pietarin kohdalle, ja tämä sanoi hänelle: "Herra, sinäkö peset minun jalkani?"
\par 7 Jeesus vastasi ja sanoi hänelle: "Mitä minä teen, sitä et nyt käsitä, mutta vastedes sinä sen ymmärrät".
\par 8 Pietari sanoi hänelle: "Et ikinä sinä saa pestä minun jalkojani". Jeesus vastasi hänelle: "Ellen minä sinua pese, ei sinulla ole osuutta minun kanssani".
\par 9 Simon Pietari sanoi hänelle: "Herra, älä pese ainoastaan minun jalkojani, vaan myös kädet ja pää".
\par 10 Jeesus sanoi hänelle: "Joka on kylpenyt, ei tarvitse muuta, kuin että jalat pestään, ja niin hän on kokonaan puhdas; ja te olette puhtaat, ette kuitenkaan kaikki".
\par 11 Sillä hän tiesi kavaltajansa; sentähden hän sanoi: "Ette kaikki ole puhtaat".
\par 12 Kun hän siis oli pessyt heidän jalkansa ja ottanut vaippansa ja taas asettunut aterialle, sanoi hän heille: "Ymmärrättekö, mitä minä olen teille tehnyt?
\par 13 Te puhuttelette minua opettajaksi ja Herraksi, ja oikein te sanotte, sillä se minä olen.
\par 14 Jos siis minä, teidän Herranne ja opettajanne, olen pessyt teidän jalkanne, olette tekin velvolliset pesemään toistenne jalat.
\par 15 Sillä minä annoin teille esikuvan, että myös te niin tekisitte, kuin minä olen teille tehnyt.
\par 16 Totisesti, totisesti minä sanon teille: ei ole palvelija herraansa suurempi eikä lähettiläs lähettäjäänsä suurempi.
\par 17 Jos te tämän tiedätte, niin olette autuaat, jos te sen teette.
\par 18 En minä puhu teistä kaikista: minä tiedän, ketkä olen valinnut; mutta tämän kirjoituksen piti käymän toteen: 'Joka minun leipääni syö, on nostanut kantapäänsä minua vastaan'.
\par 19 Jo nyt minä sanon sen teille, ennenkuin se tapahtuu, että te, kun se tapahtuu, uskoisitte, että minä olen se.
\par 20 Totisesti, totisesti minä sanon teille: joka ottaa vastaan sen, jonka minä lähetän, se ottaa vastaan minut; mutta joka ottaa vastaan minut, se ottaa vastaan hänet, joka on minut lähettänyt.
\par 21 Tämän sanottuaan Jeesus tuli järkytetyksi hengessään ja todisti ja sanoi: "Totisesti, totisesti minä sanon teille: yksi teistä on minut kavaltava".
\par 22 Niin opetuslapset katsoivat toisiinsa epätietoisina, kenestä hän puhui.
\par 23 Ja eräs hänen opetuslapsistaan, se, jota Jeesus rakasti, lepäsi aterioitaessa Jeesuksen syliä vasten.
\par 24 Simon Pietari nyökäytti hänelle päätään ja sanoi hänelle: "Sano, kuka se on, josta hän puhuu".
\par 25 Niin tämä, nojautuen Jeesuksen rintaa vasten, sanoi hänelle: "Herra, kuka se on?"
\par 26 Jeesus vastasi: "Se on se, jolle minä kastan ja annan tämän palan". Niin hän otti palan, kastoi sen ja antoi Juudaalle, Simon Iskariotin pojalle.
\par 27 Ja silloin, sen palan jälkeen, meni häneen saatana. Niin Jeesus sanoi hänelle: "Minkä teet, se tee pian".
\par 28 Mutta ei kukaan aterioivista ymmärtänyt, mitä varten hän sen hänelle sanoi.
\par 29 Sillä muutamat luulivat, koska rahakukkaro oli Juudaalla, Jeesuksen sanoneen hänelle: "Osta, mitä tarvitsemme juhlaksi", tai että hän antaisi jotakin köyhille.
\par 30 Niin hän, otettuaan sen palan, meni kohta ulos; ja oli yö.
\par 31 Kun hän oli mennyt ulos, sanoi Jeesus: "Nyt Ihmisen Poika on kirkastettu, ja Jumala on kirkastettu hänessä.
\par 32 Jos Jumala on kirkastettu hänessä, niin kirkastaa myös Jumala hänet itsessään ja kirkastaa hänet pian.
\par 33 Lapsukaiset, vielä vähän aikaa minä olen teidän kanssanne. Te tulette minua etsimään, ja niinkuin sanoin juutalaisille: 'Mihin minä menen, sinne te ette voi tulla', niin minä sanon nyt myös teille.
\par 34 Uuden käskyn minä annan teille, että rakastatte toisianne, niinkuin minä olen teitä rakastanut - että tekin niin rakastatte toisianne.
\par 35 Siitä kaikki tuntevat teidät minun opetuslapsikseni, jos teillä on keskinäinen rakkaus."
\par 36 Simon Pietari sanoi hänelle: "Herra, mihin sinä menet?" Jeesus vastasi hänelle: "Mihin minä menen, sinne sinä et voi nyt minua seurata, mutta vastedes olet minua seuraava".
\par 37 Pietari sanoi hänelle: "Herra, miksi en nyt voi seurata sinua? Henkeni minä annan sinun edestäsi."
\par 38 Jeesus vastasi: "Sinäkö annat henkesi minun edestäni? Totisesti, totisesti minä sanon sinulle: ei laula kukko, ennenkuin sinä minut kolmesti kiellät."

\chapter{14}

\par 1 "Älköön teidän sydämenne olko murheellinen. Uskokaa Jumalaan, ja uskokaa minuun.
\par 2 Minun Isäni kodissa on monta asuinsijaa. Jos ei niin olisi, sanoisinko minä teille, että minä menen valmistamaan teille sijaa?
\par 3 Ja vaikka minä menen valmistamaan teille sijaa, tulen minä takaisin ja otan teidät tyköni, että tekin olisitte siellä, missä minä olen.
\par 4 Ja mihin minä menen - tien sinne te tiedätte."
\par 5 Tuomas sanoi hänelle: "Herra, me emme tiedä, mihin sinä menet; kuinka sitten tietäisimme tien?"
\par 6 Jeesus sanoi hänelle: "Minä olen tie ja totuus ja elämä; ei kukaan tule Isän tykö muutoin kuin minun kauttani.
\par 7 Jos te olisitte tunteneet minut, niin te tuntisitte myös minun Isäni; tästälähin te tunnette hänet, ja te olette nähneet hänet."
\par 8 Filippus sanoi hänelle: "Herra, näytä meille Isä, niin me tyydymme".
\par 9 Jeesus sanoi hänelle: "Niin kauan aikaa minä olen ollut teidän kanssanne, etkä sinä tunne minua, Filippus! Joka on nähnyt minut, on nähnyt Isän; kuinka sinä sitten sanot: 'Näytä meille Isä'?
\par 10 Etkö usko, että minä olen Isässä, ja että Isä on minussa? Niitä sanoja, jotka minä teille puhun, minä en puhu itsestäni; ja Isä, joka minussa asuu, tekee teot, jotka ovat hänen.
\par 11 Uskokaa minua, että minä olen Isässä, ja että Isä on minussa; mutta jos ette, niin uskokaa itse tekojen tähden.
\par 12 Totisesti, totisesti minä sanon teille: joka uskoo minuun, myös hän on tekevä niitä tekoja, joita minä teen, ja suurempiakin, kuin ne ovat, hän on tekevä; sillä minä menen Isän tykö,
\par 13 ja mitä hyvänsä te anotte minun nimessäni, sen minä teen, että Isä kirkastettaisiin Pojassa.
\par 14 Jos te anotte minulta jotakin minun nimessäni, niin minä sen teen.
\par 15 Jos te minua rakastatte, niin te pidätte minun käskyni.
\par 16 Ja minä olen rukoileva Isää, ja hän antaa teille toisen Puolustajan olemaan teidän kanssanne iankaikkisesti,
\par 17 totuuden Hengen, jota maailma ei voi ottaa vastaan, koska se ei näe häntä eikä tunne häntä; mutta te tunnette hänet, sillä hän pysyy teidän tykönänne ja on teissä oleva.
\par 18 En minä jätä teitä orvoiksi; minä tulen teidän tykönne.
\par 19 Vielä vähän aikaa, niin maailma ei enää minua näe, mutta te näette minut; koska minä elän, niin tekin saatte elää.
\par 20 Sinä päivänä te ymmärrätte, että minä olen Isässäni, ja että te olette minussa ja minä teissä.
\par 21 Jolla on minun käskyni ja joka ne pitää, hän on se, joka minua rakastaa; mutta joka minua rakastaa, häntä minun Isäni rakastaa, ja minä rakastan häntä ja ilmoitan itseni hänelle."
\par 22 Juudas, ei se Iskariot, sanoi hänelle: "Herra, mistä syystä sinä aiot ilmoittaa itsesi meille etkä maailmalle?"
\par 23 Jeesus vastasi ja sanoi hänelle: "Jos joku rakastaa minua, niin hän pitää minun sanani, ja minun Isäni rakastaa häntä, ja me tulemme hänen tykönsä ja jäämme hänen tykönsä asumaan.
\par 24 Joka ei minua rakasta, se ei pidä minun sanojani; ja se sana, jonka te kuulette, ei ole minun, vaan Isän, joka on minut lähettänyt.
\par 25 Tämän minä olen teille puhunut ollessani teidän tykönänne.
\par 26 Mutta Puolustaja, Pyhä Henki, jonka Isä on lähettävä minun nimessäni, hän opettaa teille kaikki ja muistuttaa teitä kaikesta, minkä minä olen teille sanonut.
\par 27 Rauhan minä jätän teille: minun rauhani - sen minä annan teille. En minä anna teille, niinkuin maailma antaa. Älköön teidän sydämenne olko murheellinen älköönkä peljätkö.
\par 28 Te kuulitte minun sanovan teille: 'Minä menen pois ja palajan jälleen teidän tykönne'. Jos te minua rakastaisitte, niin te iloitsisitte siitä, että minä menen Isän tykö, sillä Isä on minua suurempi.
\par 29 Ja nyt minä olen sanonut sen teille, ennenkuin se on tapahtunut, että te uskoisitte, kun se tapahtuu.
\par 30 En minä enää puhu paljoa teidän kanssanne, sillä maailman ruhtinas tulee, ja minussa hänellä ei ole mitään.
\par 31 Mutta että maailma ymmärtäisi minun rakastavan Isää ja tekevän, niinkuin Isä on minua käskenyt: nouskaa, lähtekäämme täältä."

\chapter{15}

\par 1 "Minä olen totinen viinipuu, ja minun Isäni on viinitarhuri.
\par 2 Jokaisen oksan minussa, joka ei kanna hedelmää, hän karsii pois; ja jokaisen, joka kantaa hedelmää, hän puhdistaa, että se kantaisi runsaamman hedelmän.
\par 3 Te olette jo puhtaat sen sanan tähden, jonka minä olen teille puhunut.
\par 4 Pysykää minussa, niin minä pysyn teissä. Niinkuin oksa ei voi kantaa hedelmää itsestään, ellei se pysy viinipuussa, niin ette tekään, ellette pysy minussa.
\par 5 Minä olen viinipuu, te olette oksat. Joka pysyy minussa ja jossa minä pysyn, se kantaa paljon hedelmää; sillä ilman minua te ette voi mitään tehdä.
\par 6 Jos joku ei pysy minussa, niin hänet heitetään pois niinkuin oksa, ja hän kuivettuu; ja ne kootaan yhteen ja heitetään tuleen, ja ne palavat.
\par 7 Jos te pysytte minussa ja minun sanani pysyvät teissä, niin anokaa, mitä ikinä tahdotte, ja te saatte sen.
\par 8 Siinä minun Isäni kirkastetaan, että te kannatte paljon hedelmää ja tulette minun opetuslapsikseni.
\par 9 Niinkuin Isä on minua rakastanut, niin minäkin olen rakastanut teitä; pysykää minun rakkaudessani.
\par 10 Jos te pidätte minun käskyni, niin te pysytte minun rakkaudessani, niinkuin minä olen pitänyt Isäni käskyt ja pysyn hänen rakkaudessaan.
\par 11 Tämän minä olen teille puhunut, että minun iloni olisi teissä ja teidän ilonne tulisi täydelliseksi.
\par 12 Tämä on minun käskyni, että te rakastatte toisianne, niinkuin minä olen teitä rakastanut.
\par 13 Sen suurempaa rakkautta ei ole kenelläkään, kuin että hän antaa henkensä ystäväinsä edestä.
\par 14 Te olette minun ystäväni, jos teette, mitä minä käsken teidän tehdä.
\par 15 En minä enää sano teitä palvelijoiksi, sillä palvelija ei tiedä, mitä hänen herransa tekee; vaan ystäviksi minä sanon teitä, sillä minä olen ilmoittanut teille kaikki, mitä minä olen kuullut Isältäni.
\par 16 Te ette valinneet minua, vaan minä valitsin teidät ja asetin teidät, että te menisitte ja kantaisitte hedelmää ja että teidän hedelmänne pysyisi: että mitä ikinä te anotte Isältä minun nimessäni, hän sen teille antaisi.
\par 17 Sen käskyn minä teille annan, että rakastatte toisianne.
\par 18 Jos maailma teitä vihaa, niin tietäkää, että se on vihannut minua ennen kuin teitä.
\par 19 Jos te maailmasta olisitte, niin maailma omaansa rakastaisi; mutta koska te ette ole maailmasta, vaan minä olen teidät maailmasta valinnut, sentähden maailma teitä vihaa.
\par 20 Muistakaa se sana, jonka minä teille sanoin: 'Ei ole palvelija herraansa suurempi'. Jos he ovat minua vainonneet, niin he teitäkin vainoavat; jos he ovat ottaneet vaarin minun sanastani, niin he ottavat vaarin teidänkin sanastanne.
\par 21 Mutta kaiken tämän he tekevät teille minun nimeni tähden, koska he eivät tunne häntä, joka on minut lähettänyt.
\par 22 Jos minä en olisi tullut ja puhunut heille, ei heillä olisi syntiä; mutta nyt heillä ei ole, millä syntiänsä puolustaisivat.
\par 23 Joka vihaa minua, se vihaa myös minun Isääni.
\par 24 Jos minä en olisi tehnyt heidän keskuudessaan niitä tekoja, joita ei kukaan muu ole tehnyt, ei heillä olisi syntiä; mutta nyt he ovat nähneet ja ovat vihanneet sekä minua että minun Isääni.
\par 25 Mutta se sana oli käyvä toteen, joka on kirjoitettuna heidän laissaan: 'He ovat vihanneet minua syyttä'.
\par 26 Mutta kun Puolustaja tulee, jonka minä lähetän teille Isän tyköä, totuuden Henki, joka lähtee Isän tyköä, niin hän on todistava minusta.
\par 27 Ja te myös todistatte, sillä te olette alusta asti olleet minun kanssani."

\chapter{16}

\par 1 "Tämän minä olen teille puhunut, ettette loukkaantuisi.
\par 2 He erottavat teidät synagoogasta; ja tulee aika, jolloin jokainen, joka tappaa teitä, luulee tekevänsä uhripalveluksen Jumalalle.
\par 3 Ja sen he tekevät teille, koska he eivät tunne Isää eivätkä minua.
\par 4 Mutta tämän minä olen puhunut teille, että, kun se aika tulee, te muistaisitte minun sen teille sanoneen. Tätä minä en ole sanonut teille alusta, koska minä olin teidän kansanne.
\par 5 Mutta nyt minä menen hänen tykönsä, joka on minut lähettänyt, eikä kukaan teistä kysy minulta: 'Mihin sinä menet?'
\par 6 Mutta koska minä olen tämän teille puhunut, täyttää murhe teidän sydämenne.
\par 7 Kuitenkin minä sanon teille totuuden: teille on hyväksi, että minä menen pois. Sillä ellen minä mene pois, ei Puolustaja tule teidän tykönne; mutta jos minä menen, niin minä hänet teille lähetän.
\par 8 Ja kun hän tulee, niin hän näyttää maailmalle todeksi synnin ja vanhurskauden ja tuomion:
\par 9 synnin, koska he eivät usko minuun;
\par 10 vanhurskauden, koska minä menen Isän tykö, ettekä te enää minua näe;
\par 11 ja tuomion, koska tämän maailman ruhtinas on tuomittu.
\par 12 Minulla on vielä paljon sanottavaa teille, mutta te ette voi nyt sitä kantaa.
\par 13 Mutta kun hän tulee, totuuden Henki, johdattaa hän teidät kaikkeen totuuteen. Sillä se, mitä hän puhuu, ei ole hänestä itsestään; vaan minkä hän kuulee, sen hän puhuu, ja tulevaiset hän teille julistaa.
\par 14 Hän on minut kirkastava, sillä hän ottaa minun omastani ja julistaa teille.
\par 15 Kaikki, mitä Isällä on, on minun; sentähden minä sanoin, että hän ottaa minun omastani ja julistaa teille.
\par 16 Vähän aikaa, niin te ette enää minua näe, ja taas vähän aikaa, niin te näette minut."
\par 17 Silloin muutamat hänen opetuslapsistansa sanoivat toisilleen: "Mitä se tarkoittaa, kun hän sanoo meille: 'Vähän aikaa, niin te ette minua näe, ja taas vähän aikaa, niin te näette minut', ja: 'Minä menen Isän tykö'?"
\par 18 Niin he sanoivat: "Mitä se tarkoittaa, kun hän sanoo: 'Vähän aikaa'? Emme ymmärrä, mitä hän puhuu."
\par 19 Jeesus huomasi heidän tahtovan kysyä häneltä ja sanoi heille: "Sitäkö te kyselette keskenänne, että minä sanoin: 'Vähän aikaa, niin te ette minua näe, ja taas vähän aikaa, niin te näette minut'?
\par 20 Totisesti, totisesti minä sanon teille: te joudutte itkemään ja valittamaan, mutta maailma on iloitseva; te tulette murheellisiksi, mutta teidän murheenne on muuttuva iloksi.
\par 21 Kun vaimo synnyttää, on hänellä murhe, koska hänen hetkensä on tullut; mutta kun hän on synnyttänyt lapsen, ei hän enää muista ahdistustaan sen ilon tähden, että ihminen on syntynyt maailmaan.
\par 22 Niin on myös teillä nyt murhe; mutta minä olen taas näkevä teidät, ja teidän sydämenne on iloitseva, eikä kukaan ota teiltä pois teidän iloanne.
\par 23 Ja sinä päivänä te ette minulta mitään kysy. Totisesti, totisesti minä sanon teille: jos te anotte jotakin Isältä, on hän sen teille antava minun nimessäni.
\par 24 Tähän asti te ette ole anoneet mitään minun nimessäni; anokaa, niin te saatte, että teidän ilonne olisi täydellinen.
\par 25 Tämän minä olen puhunut teille kuvauksilla; mutta tulee aika, jolloin minä en puhu teille enää kuvauksilla, vaan avonaisesti julistan teille sanomaa Isästä.
\par 26 Sinä päivänä te anotte minun nimessäni; enkä minä sano teille, että minä olen rukoileva Isää teidän edestänne;
\par 27 sillä Isä itse rakastaa teitä, sentähden että te olette minua rakastaneet ja uskoneet minun lähteneen Jumalan tyköä.
\par 28 Minä olen lähtenyt Isästä ja tullut maailmaan; jälleen minä jätän maailman ja menen Isän tykö."
\par 29 Hänen opetuslapsensa sanoivat: "Katso, nyt sinä puhut avonaisesti etkä käytä mitään kuvausta.
\par 30 Nyt me tiedämme, että sinä tiedät kaikki, etkä tarvitse, että kukaan sinulta kysyy; sentähden me uskomme sinun Jumalan tyköä lähteneen."
\par 31 Jeesus vastasi heille: "Nyt te uskotte.
\par 32 Katso, tulee hetki ja on jo tullut, jona teidät hajotetaan kukin tahollensa ja te jätätte minut yksin; en minä kuitenkaan yksin ole, sillä Isä on minun kanssani.
\par 33 Tämän minä olen teille puhunut, että teillä olisi minussa rauha. Maailmassa teillä on ahdistus; mutta olkaa turvallisella mielellä: minä olen voittanut maailman."

\chapter{17}

\par 1 Tämän Jeesus puhui ja nosti silmänsä taivasta kohti ja sanoi: "Isä, hetki on tullut, kirkasta Poikasi, että Poikasi kirkastaisi sinut;
\par 2 koska sinä olet antanut hänen valtaansa kaiken lihan, että hän antaisi iankaikkisen elämän kaikille, jotka sinä olet hänelle antanut.
\par 3 Mutta tämä on iankaikkinen elämä, että he tuntevat sinut, joka yksin olet totinen Jumala, ja hänet, jonka sinä olet lähettänyt, Jeesuksen Kristuksen.
\par 4 Minä olen kirkastanut sinut maan päällä: minä olen täyttänyt sen työn, jonka sinä annoit minun tehtäväkseni.
\par 5 Ja nyt, Isä, kirkasta sinä minut tykönäsi sillä kirkkaudella, joka minulla oli sinun tykönäsi, ennenkuin maailma olikaan.
\par 6 Minä olen ilmoittanut sinun nimesi ihmisille, jotka sinä annoit minulle maailmasta. He olivat sinun, ja sinä annoit heidät minulle, ja he ovat ottaneet sinun sanastasi vaarin.
\par 7 Nyt he tietävät, että kaikki, minkä olet minulle antanut, on sinulta.
\par 8 Sillä ne sanat, jotka sinä minulle annoit, minä olen antanut heille; ja he ovat ottaneet ne vastaan ja tietävät totisesti minun lähteneen sinun tyköäsi ja uskovat, että sinä olet minut lähettänyt.
\par 9 Minä rukoilen heidän edestänsä; en minä maailman edestä rukoile, vaan niiden edestä, jotka sinä olet minulle antanut, koska he ovat sinun -
\par 10 ja kaikki minun omani ovat sinun, ja sinun omasi ovat minun - ja minä olen kirkastettu heissä.
\par 11 Ja minä en enää ole maailmassa, mutta he ovat maailmassa, ja minä tulen sinun tykösi. Pyhä Isä, varjele heidät nimessäsi, jonka sinä olet minulle antanut, että he olisivat yhtä niinkuin mekin.
\par 12 Kun minä olin heidän kanssansa, varjelin minä heidät sinun nimessäsi, jonka sinä olet minulle antanut, ja suojelin heitä, eikä heistä joutunut kadotetuksi yksikään muu kuin se kadotuksen lapsi, että kirjoitus kävisi toteen.
\par 13 Mutta nyt minä tulen sinun tykösi ja puhun tätä maailmassa, että heillä olisi minun iloni täydellisenä heissä itsessään.
\par 14 Minä olen antanut heille sinun sanasi, ja maailma vihaa heitä, koska he eivät ole maailmasta, niinkuin en minäkään maailmasta ole.
\par 15 En minä rukoile, että ottaisit heidät pois maailmasta, vaan että sinä varjelisit heidät pahasta.
\par 16 He eivät ole maailmasta, niinkuin en minäkään maailmasta ole.
\par 17 Pyhitä heidät totuudessa; sinun sanasi on totuus.
\par 18 Niinkuin sinä olet lähettänyt minut maailmaan, niin olen minäkin lähettänyt heidät maailmaan;
\par 19 ja minä pyhitän itseni heidän tähtensä, että myös he olisivat pyhitetyt totuudessa.
\par 20 Mutta en minä rukoile ainoastaan näiden edestä, vaan myös niiden edestä, jotka heidän sanansa kautta uskovat minuun,
\par 21 että he kaikki olisivat yhtä, niinkuin sinä, Isä, olet minussa ja minä sinussa, että hekin meissä olisivat, niin että maailma uskoisi, että sinä olet minut lähettänyt.
\par 22 Ja sen kirkkauden, jonka sinä minulle annoit, minä olen antanut heille, että he olisivat yhtä, niinkuin me olemme yhtä -
\par 23 minä heissä, ja sinä minussa - että he olisivat täydellisesti yhtä, niin että maailma ymmärtäisi, että sinä olet minut lähettänyt ja rakastanut heitä, niinkuin sinä olet minua rakastanut.
\par 24 Isä, minä tahdon, että missä minä olen, siellä nekin, jotka sinä olet minulle antanut, olisivat minun kanssani, että he näkisivät minun kirkkauteni, jonka sinä olet minulle antanut, koska olet rakastanut minua jo ennen maailman perustamista.
\par 25 Vanhurskas Isä, maailma ei ole sinua tuntenut, mutta minä tunnen sinut, ja nämä ovat tulleet tuntemaan, että sinä olet minut lähettänyt.
\par 26 Ja minä olen tehnyt sinun nimesi heille tunnetuksi ja teen vastakin, että se rakkaus, jolla sinä olet minua rakastanut, olisi heissä ja minä olisin heissä."

\chapter{18}

\par 1 Kun Jeesus oli tämän puhunut, lähti hän pois opetuslastensa kanssa Kedronin puron tuolle puolelle; siellä oli puutarha, johon hän meni opetuslapsinensa.
\par 2 Mutta myös Juudas, joka hänet kavalsi, tiesi sen paikan, koska Jeesus ja hänen opetuslapsensa usein olivat kokoontuneet sinne.
\par 3 Niin Juudas otti sotilasjoukon sekä ylipappien ja fariseusten palvelijoita ja tuli sinne soihdut ja lamput ja aseet mukanaan.
\par 4 Silloin Jeesus, joka tiesi kaiken, mikä oli häntä kohtaava, astui esiin ja sanoi heille: "Ketä te etsitte?"
\par 5 He vastasivat hänelle: "Jeesusta, Nasaretilaista". Jeesus sanoi heille: "Minä se olen". Ja Juudas, joka hänet kavalsi, seisoi myös heidän kanssaan.
\par 6 Kun hän siis sanoi heille: "Minä se olen", peräytyivät he ja kaatuivat maahan.
\par 7 Niin hän taas kysyi heiltä: "Ketä te etsitte?" He sanoivat: "Jeesusta, Nasaretilaista".
\par 8 Jeesus vastasi: "Minähän sanoin teille, että minä se olen. Jos te siis minua etsitte, niin antakaa näiden mennä";
\par 9 että se sana kävisi toteen, jonka hän oli sanonut: "En minä ole kadottanut ketään niistä, jotka sinä olet minulle antanut".
\par 10 Niin Simon Pietari, jolla oli miekka, veti sen ja iski ylimmäisen papin palvelijaa ja sivalsi häneltä pois oikean korvan; ja palvelijan nimi oli Malkus.
\par 11 Niin Jeesus sanoi Pietarille: "Pistä miekkasi tuppeen. Enkö minä joisi sitä maljaa, jonka Isä on minulle antanut?"
\par 12 Niin sotilasjoukko ja päällikkö ja juutalaisten palvelijat ottivat Jeesuksen kiinni ja sitoivat hänet
\par 13 ja veivät ensin Hannaan luo, sillä hän oli Kaifaan appi, ja Kaifas oli sinä vuonna ylimmäisenä pappina.
\par 14 Ja Kaifas oli se, joka neuvottelussa oli juutalaisille sanonut: "On hyödyllistä, että yksi ihminen kuolee kansan edestä."
\par 15 Ja Jeesusta seurasi Simon Pietari ja eräs toinen opetuslapsi. Tämä opetuslapsi oli ylimmäisen papin tuttava ja meni Jeesuksen kanssa sisälle ylimmäisen papin kartanoon.
\par 16 Mutta Pietari seisoi portilla ulkona. Niin se toinen opetuslapsi, joka oli ylimmäisen papin tuttava, meni ulos ja puhutteli portinvartijatarta ja toi Pietarin sisälle.
\par 17 Niin palvelijatar, joka vartioi porttia, sanoi Pietarille: "Etkö sinäkin ole tuon miehen opetuslapsia?" Hän sanoi: "En ole".
\par 18 Mutta palvelijat ja käskyläiset olivat tehneet hiilivalkean, koska oli kylmä, ja seisoivat ja lämmittelivät. Ja myös Pietari seisoi heidän kanssaan lämmittelemässä.
\par 19 Niin ylimmäinen pappi kysyi Jeesukselta hänen opetuslapsistaan ja opistaan.
\par 20 Jeesus vastasi hänelle: "Minä olen julkisesti puhunut maailmalle; minä olen aina opettanut synagoogissa ja pyhäkössä, joihin kaikki juutalaiset kokoontuvat, enkä ole salassa puhunut mitään.
\par 21 Miksi minulta kysyt? Kysy niiltä, jotka ovat kuulleet, mitä minä olen heille puhunut; katso, he tietävät, mitä minä olen sanonut."
\par 22 Mutta kun Jeesus oli tämän sanonut, antoi eräs palvelija, joka seisoi vieressä, hänelle korvapuustin sanoen: "Niinkö sinä vastaat ylimmäiselle papille?"
\par 23 Jeesus vastasi hänelle: "Jos minä pahasti puhuin, niin näytä toteen, että se on pahaa; mutta jos minä puhuin oikein, miksi minua lyöt?"
\par 24 Niin Hannas lähetti hänet sidottuna ylimmäisen papin Kaifaan luo.
\par 25 Mutta Simon Pietari seisoi lämmittelemässä. Niin he sanoivat hänelle: "Etkö sinäkin ole hänen opetuslapsiaan?" Hän kielsi ja sanoi: "En ole".
\par 26 Silloin eräs ylimmäisen papin palvelijoista, sen sukulainen, jolta Pietari oli sivaltanut korvan pois, sanoi: "Enkö minä nähnyt sinua puutarhassa hänen kanssaan?"
\par 27 Niin Pietari taas kielsi, ja samassa lauloi kukko.
\par 28 Niin he veivät Jeesuksen Kaifaan luota palatsiin; ja oli varhainen aamu. Ja itse he eivät menneet sisälle palatsiin, etteivät saastuisi, vaan saattaisivat syödä pääsiäislammasta.
\par 29 Niin Pilatus meni ulos heidän luokseen ja sanoi: "Mikä syytös ja kanne teillä on tätä miestä vastaan?"
\par 30 He vastasivat ja sanoivat hänelle: "Jos tämä ei olisi pahantekijä, emme olisi antaneet häntä sinun käsiisi".
\par 31 Niin Pilatus sanoi heille: "Ottakaa te hänet ja tuomitkaa hänet lakinne mukaan". Juutalaiset sanoivat hänelle: "Meidän ei ole lupa ketään tappaa";
\par 32 että Jeesuksen sana kävisi toteen, jonka hän oli sanonut, antaen tietää, minkälaisella kuolemalla hän oli kuoleva.
\par 33 Niin Pilatus meni taas sisälle palatsiin ja kutsui Jeesuksen ja sanoi hänelle: "Oletko sinä juutalaisten kuningas?"
\par 34 Jeesus vastasi hänelle: "Itsestäsikö sen sanot, vai ovatko muut sen sinulle minusta sanoneet?"
\par 35 Pilatus vastasi: "Olenko minä juutalainen? Oma kansasi ja ylipapit ovat antaneet sinut minun käsiini. Mitä olet tehnyt?"
\par 36 Jeesus vastasi: "Minun kuninkuuteni ei ole tästä maailmasta; jos minun kuninkuuteni olisi tästä maailmasta, niin minun palvelijani olisivat taistelleet, ettei minua olisi annettu juutalaisten käsiin; mutta nyt minun kuninkuuteni ei ole täältä".
\par 37 Niin Pilatus sanoi hänelle: "Sinä siis kuitenkin olet kuningas?" Jeesus vastasi: "Sinäpä sen sanot, että minä olen kuningas. Sitä varten minä olen syntynyt ja sitä varten maailmaan tullut, että minä todistaisin totuuden puolesta. Jokainen, joka on totuudesta, kuulee minun ääneni."
\par 38 Pilatus sanoi hänelle: "Mikä on totuus?" Ja sen sanottuaan hän taas meni ulos juutalaisten luo ja sanoi heille: "Minä en löydä hänessä yhtäkään syytä.
\par 39 Mutta te olette tottuneet siihen, että minä päästän teille yhden vangin irti pääsiäisenä; tahdotteko siis, että päästän teille juutalaisten kuninkaan?"
\par 40 Niin he taas huusivat sanoen: "Älä häntä, vaan Barabbas!" Mutta Barabbas oli ryöväri.

\chapter{19}

\par 1 Silloin Pilatus otti Jeesuksen ja ruoskitti hänet.
\par 2 Ja sotamiehet väänsivät kruunun orjantappuroista, panivat sen hänen päähänsä ja pukivat hänen ylleen purppuraisen vaipan
\par 3 ja tulivat hänen luoksensa ja sanoivat: "Terve, juutalaisten kuningas"; ja he antoivat hänelle korvapuusteja.
\par 4 Pilatus meni taas ulos ja sanoi heille: "Katso, minä tuon hänet ulos teille, tietääksenne, etten minä löydä hänessä yhtäkään syytä".
\par 5 Niin Jeesus tuli ulos, orjantappurakruunu päässään ja purppurainen vaippa yllään. Ja Pilatus sanoi heille: "Katso ihmistä!"
\par 6 Kun siis ylipapit ja palvelijat näkivät hänet, huusivat he sanoen: "Ristiinnaulitse, ristiinnaulitse!" Pilatus sanoi heille: "Ottakaa te hänet ja ristiinnaulitkaa, sillä minä en löydä hänessä mitään syytä".
\par 7 Juutalaiset vastasivat hänelle: "Meillä on laki, ja lain mukaan hänen pitää kuoleman, koska hän on tehnyt itsensä Jumalan Pojaksi".
\par 8 Kun nyt Pilatus kuuli tämän sanan, pelkäsi hän vielä enemmän
\par 9 ja meni taas sisälle palatsiin ja sanoi Jeesukselle: "Mistä sinä olet?" Mutta Jeesus ei hänelle vastannut.
\par 10 Niin Pilatus sanoi hänelle: "Etkö puhu minulle? Etkö tiedä, että minulla on valta sinut päästää ja minulla on valta sinut ristiinnaulita?"
\par 11 Jeesus vastasi: "Sinulla ei olisi mitään valtaa minuun, ellei sitä olisi annettu sinulle ylhäältä. Sentähden on sen synti suurempi, joka jätti minut sinun käsiisi."
\par 12 Tämän tähden Pilatus koetti päästää hänet irti. Mutta juutalaiset huusivat sanoen: "Jos päästät hänet, et ole keisarin ystävä; jokainen, joka tekee itsensä kuninkaaksi, asettuu keisaria vastaan".
\par 13 Kun Pilatus kuuli nämä sanat, antoi hän viedä Jeesuksen ulos ja istui tuomarinistuimelle, paikalle, jonka nimi on Litostroton, hebreaksi Gabbata.
\par 14 Ja oli pääsiäisen valmistuspäivä, noin kuudes hetki. Ja hän sanoi juutalaisille: "Katso, teidän kuninkaanne!"
\par 15 Niin he huusivat: "Vie pois, vie pois, ristiinnaulitse hänet!" Pilatus sanoi heille: "Onko minun ristiinnaulittava teidän kuninkaanne?" Ylipapit vastasivat: "Ei meillä ole kuningasta, vaan keisari".
\par 16 Silloin hän luovutti hänet heille ja antoi ristiinnaulittavaksi. Ja he ottivat Jeesuksen.
\par 17 Ja kantaen itse omaa ristiänsä hän meni ulos niin sanotulle Pääkallonpaikalle, jota kutsutaan hebreankielellä Golgataksi.
\par 18 Siellä he hänet ristiinnaulitsivat ja hänen kanssaan kaksi muuta, yhden kummallekin puolelle, ja Jeesuksen keskelle.
\par 19 Ja Pilatus kirjoitti myös päällekirjoituksen ja kiinnitti sen ristiin; ja se oli näin kirjoitettu: "Jeesus Nasaretilainen, juutalaisten kuningas".
\par 20 Tämän päällekirjoituksen lukivat monet juutalaiset, sillä paikka, jossa Jeesus ristiinnaulittiin, oli lähellä kaupunkia; ja se oli kirjoitettu hebreaksi, latinaksi ja kreikaksi.
\par 21 Niin juutalaisten ylipapit sanoivat Pilatukselle: "Älä kirjoita: 'Juutalaisten kuningas', vaan että hän on sanonut: 'Minä olen juutalaisten kuningas'."
\par 22 Pilatus vastasi: "Minkä minä kirjoitin, sen minä kirjoitin".
\par 23 Kun sotamiehet olivat ristiinnaulinneet Jeesuksen, ottivat he hänen vaatteensa ja jakoivat ne neljään osaan, kullekin sotamiehelle osansa, sekä ihokkaan. Mutta ihokas oli saumaton, kauttaaltaan ylhäältä asti kudottu.
\par 24 Sentähden he sanoivat toisillensa: "Älkäämme leikatko sitä rikki, vaan heittäkäämme siitä arpaa, kenen se on oleva"; että tämä kirjoitus kävisi toteen: "He jakoivat keskenänsä minun vaatteeni ja heittivät minun puvustani arpaa". Ja sotamiehet tekivät niin.
\par 25 Mutta Jeesuksen ristin ääressä seisoivat hänen äitinsä ja hänen äitinsä sisar ja Maria, Kloopaan vaimo, ja Maria Magdaleena.
\par 26 Kun Jeesus näki äitinsä ja sen opetuslapsen, jota hän rakasti, seisovan siinä vieressä, sanoi hän äidillensä: "Vaimo, katso, poikasi!"
\par 27 Sitten hän sanoi opetuslapselle: "Katso, äitisi!" Ja siitä hetkestä opetuslapsi otti hänet kotiinsa.
\par 28 Sen jälkeen, kun Jeesus tiesi, että kaikki jo oli täytetty, sanoi hän, että kirjoitus kävisi toteen: "Minun on jano".
\par 29 Siinä oli astia, hapanviiniä täynnä; niin he täyttivät sillä hapanviinillä sienen ja panivat sen isoppikorren päähän ja ojensivat sen hänen suunsa eteen.
\par 30 Kun nyt Jeesus oli ottanut hapanviinin, sanoi hän: "Se on täytetty", ja kallisti päänsä ja antoi henkensä.
\par 31 Koska silloin oli valmistuspäivä, niin - etteivät ruumiit jäisi ristille sapatiksi, sillä se sapatinpäivä oli suuri - juutalaiset pyysivät Pilatukselta, että ristiinnaulittujen sääriluut rikottaisiin ja ruumiit otettaisiin alas.
\par 32 Niin sotamiehet tulivat ja rikkoivat sääriluut ensin toiselta ja sitten toiselta hänen kanssaan ristiinnaulitulta.
\par 33 Mutta kun he tulivat Jeesuksen luo ja näkivät hänet jo kuolleeksi, eivät he rikkoneet hänen luitaan,
\par 34 vaan yksi sotamiehistä puhkaisi keihäällä hänen kylkensä, ja heti vuoti siitä verta ja vettä.
\par 35 Ja joka sen näki, on sen todistanut, ja hänen todistuksensa on tosi, ja hän tietää totta puhuvansa, että tekin uskoisitte.
\par 36 Sillä tämä tapahtui, että kirjoitus kävisi toteen: "Älköön häneltä luuta rikottako".
\par 37 Ja vielä sanoo toinen kirjoitus: "He luovat katseensa häneen, jonka he ovat lävistäneet".
\par 38 Mutta sen jälkeen Joosef, arimatialainen, joka oli Jeesuksen opetuslapsi, vaikka salaa, juutalaisten pelosta, pyysi Pilatukselta saada ottaa Jeesuksen ruumiin; ja Pilatus myöntyi siihen. Niin hän tuli ja otti Jeesuksen ruumiin.
\par 39 Tuli myös Nikodeemus, joka ensi kerran oli yöllä tullut Jeesuksen tykö, ja toi mirhan ja aloen seosta noin sata naulaa.
\par 40 Niin he ottivat Jeesuksen ruumiin ja käärivät sen hyvänhajuisten yrttien kanssa käärinliinoihin, niinkuin juutalaisilla on tapana haudata.
\par 41 Ja sillä paikalla, missä hänet ristiinnaulittiin, oli puutarha, ja puutarhassa uusi hauta, johon ei vielä oltu ketään pantu.
\par 42 Siihen he nyt panivat Jeesuksen, koska oli juutalaisten valmistuspäivä ja se hauta oli lähellä.

\chapter{20}

\par 1 Mutta viikon ensimmäisenä päivänä Maria Magdaleena meni varhain, kun vielä oli pimeä, haudalle ja näki kiven otetuksi pois haudan suulta.
\par 2 Niin hän riensi pois ja tuli Simon Pietarin luo ja sen toisen opetuslapsen luo, joka oli Jeesukselle rakas, ja sanoi heille: "Ovat ottaneet Herran pois haudasta, emmekä tiedä, mihin ovat hänet panneet".
\par 3 Niin Pietari ja se toinen opetuslapsi lähtivät ja menivät haudalle.
\par 4 Ja he juoksivat molemmat yhdessä; mutta se toinen opetuslapsi juoksi edellä, nopeammin kuin Pietari, ja saapui ensin haudalle.
\par 5 Ja kun hän kurkisti sisään, näki hän käärinliinat siellä; kuitenkaan hän ei mennyt sisälle.
\par 6 Niin Simon Pietarikin tuli hänen perässään ja meni sisälle hautaan ja näki käärinliinat siellä
\par 7 ja hikiliinan, joka oli ollut hänen päässään, ei pantuna yhteen käärinliinojen kanssa, vaan toiseen paikkaan erikseen kokoonkäärittynä.
\par 8 Silloin toinenkin opetuslapsi, joka ensimmäisenä oli tullut haudalle, meni sisään ja näki ja uskoi.
\par 9 Sillä he eivät vielä ymmärtäneet Raamattua, että hän oli kuolleista nouseva.
\par 10 Niin opetuslapset menivät takaisin kotiinsa.
\par 11 Mutta Maria seisoi haudan edessä ulkopuolella ja itki. Kun hän näin itki, kurkisti hän hautaan
\par 12 ja näki kaksi enkeliä valkeissa vaatteissa istuvan, toisen pääpuolessa ja toisen jalkapuolessa, siinä, missä Jeesuksen ruumis oli maannut.
\par 13 Nämä sanoivat hänelle: "Vaimo, mitä itket?" Hän sanoi heille: "Ovat ottaneet pois minun Herrani, enkä tiedä, mihin ovat hänet panneet".
\par 14 Tämän sanottuaan hän kääntyi taaksepäin ja näki Jeesuksen siinä seisovan, eikä tiennyt, että se oli Jeesus.
\par 15 Jeesus sanoi hänelle: "Vaimo, mitä itket? Ketä etsit?" Tämä luuli häntä puutarhuriksi ja sanoi hänelle: "Herra, jos sinä olet kantanut hänet pois, sano minulle, mihin olet hänet pannut, niin minä otan hänet".
\par 16 Jeesus sanoi hänelle: "Maria!" Tämä kääntyi ja sanoi hänelle hebreankielellä: "Rabbuuni!" se on: opettaja.
\par 17 Jeesus sanoi hänelle: "Älä minuun koske, sillä en minä ole vielä mennyt ylös Isäni tykö; mutta mene minun veljieni tykö ja sano heille, että minä menen ylös, minun Isäni tykö ja teidän Isänne tykö, ja minun Jumalani tykö ja teidän Jumalanne tykö".
\par 18 Maria Magdaleena meni ja ilmoitti opetuslapsille, että hän oli nähnyt Herran ja että Herra oli hänelle näin sanonut.
\par 19 Samana päivänä, viikon ensimmäisenä, myöhään illalla, kun opetuslapset olivat koolla lukittujen ovien takana, juutalaisten pelosta, tuli Jeesus ja seisoi heidän keskellään ja sanoi heille: "Rauha teille!"
\par 20 Ja sen sanottuaan hän näytti heille kätensä ja kylkensä. Niin opetuslapset iloitsivat nähdessään Herran.
\par 21 Niin Jeesus sanoi heille jälleen: "Rauha teille! Niinkuin Isä on lähettänyt minut, niin lähetän minäkin teidät."
\par 22 Ja tämän sanottuaan hän puhalsi heidän päällensä ja sanoi heille: "Ottakaa Pyhä Henki.
\par 23 Joiden synnit te anteeksi annatte, niille ne ovat anteeksi annetut; joiden synnit te pidätätte, niille ne ovat pidätetyt."
\par 24 Mutta Tuomas, jota sanottiin Didymukseksi, yksi niistä kahdestatoista, ei ollut heidän kanssansa, kun Jeesus tuli.
\par 25 Niin muut opetuslapset sanoivat hänelle: "Me näimme Herran". Mutta hän sanoi heille: "Ellen näe hänen käsissään naulojen jälkiä ja pistä sormeani naulojen sijoihin ja pistä kättäni hänen kylkeensä, en minä usko".
\par 26 Ja kahdeksan päivän perästä hänen opetuslapsensa taas olivat huoneessa, ja Tuomas oli heidän kanssansa. Niin Jeesus tuli, ovien ollessa lukittuina, ja seisoi heidän keskellään ja sanoi: "Rauha teille!"
\par 27 Sitten hän sanoi Tuomaalle: "Ojenna sormesi tänne ja katso minun käsiäni, ja ojenna kätesi ja pistä se minun kylkeeni, äläkä ole epäuskoinen, vaan uskovainen".
\par 28 Tuomas vastasi ja sanoi hänelle: "Minun Herrani ja minun Jumalani!"
\par 29 Jeesus sanoi hänelle: "Sentähden, että minut näit, sinä uskot. Autuaat ne, jotka eivät näe ja kuitenkin uskovat!"
\par 30 Paljon muitakin tunnustekoja, joita ei ole kirjoitettu tähän kirjaan, Jeesus teki opetuslastensa nähden;
\par 31 mutta nämä ovat kirjoitetut, että te uskoisitte, että Jeesus on Kristus, Jumalan Poika, ja että teillä uskon kautta olisi elämä hänen nimessänsä.

\chapter{21}

\par 1 Sen jälkeen Jeesus taas ilmestyi opetuslapsilleen Tiberiaan järven rannalla; ja hän ilmestyi näin:
\par 2 Simon Pietari ja Tuomas, jota sanottiin Didymukseksi, ja Natanael, joka oli Galilean Kaanasta, ja Sebedeuksen pojat sekä kaksi muuta hänen opetuslapsistaan olivat yhdessä.
\par 3 Simon Pietari sanoi heille: "Minä menen kalaan". He sanoivat hänelle: "Me lähdemme myös sinun kanssasi". Niin he lähtivät ja astuivat venheeseen; mutta eivät sinä yönä saaneet mitään.
\par 4 Ja kun jo oli aamu, seisoi Jeesus rannalla. Opetuslapset eivät kuitenkaan tienneet, että se oli Jeesus.
\par 5 Niin Jeesus sanoi heille: "Lapset, onko teillä mitään syötävää?" He vastasivat hänelle: "Ei ole".
\par 6 Hän sanoi heille: "Heittäkää verkko oikealle puolelle venhettä, niin saatte". He heittivät verkon, mutta eivät jaksaneet vetää sitä ylös kalojen paljouden tähden.
\par 7 Silloin se opetuslapsi, jota Jeesus rakasti, sanoi Pietarille: "Se on Herra". Kun Simon Pietari kuuli, että se oli Herra, vyötti hän vaippansa ympärilleen, sillä hän oli ilman vaatteita, ja heittäytyi järveen.
\par 8 Mutta muut opetuslapset tulivat venheellä ja vetivät perässään verkkoa kaloineen, sillä he eivät olleet maasta kauempana kuin noin kahdensadan kyynärän päässä.
\par 9 Kun he astuivat maalle, näkivät he siellä hiilloksen ja kalan pantuna sen päälle, sekä leipää.
\par 10 Jeesus sanoi heille: "Tuokaa tänne niitä kaloja, joita nyt saitte".
\par 11 Niin Simon Pietari astui venheeseen ja veti maalle verkon, täynnä suuria kaloja, sata viisikymmentä kolme. Ja vaikka niitä oli niin paljon, ei verkko revennyt.
\par 12 Jeesus sanoi heille: "Tulkaa einehtimään". Mutta ei kukaan opetuslapsista uskaltanut kysyä häneltä: "Kuka sinä olet?", koska he tiesivät, että se oli Herra.
\par 13 Niin Jeesus meni ja otti leivän ja antoi heille, ja samoin kalan.
\par 14 Tämä oli jo kolmas kerta, jolloin Jeesus noustuaan kuolleista ilmestyi opetuslapsillensa.
\par 15 Kun he olivat einehtineet, sanoi Jeesus Simon Pietarille: "Simon, Johanneksen poika, rakastatko sinä minua enemmän kuin nämä?" Hän vastasi hänelle: "Rakastan, Herra; sinä tiedät, että olet minulle rakas". Hän sanoi hänelle: "Ruoki minun karitsoitani".
\par 16 Hän sanoi hänelle taas toistamiseen: "Simon, Johanneksen poika, rakastatko minua?" Hän vastasi hänelle: "Rakastan, Herra; sinä tiedät, että olet minulle rakas". Hän sanoi hänelle: "Kaitse minun lampaitani".
\par 17 Hän sanoi hänelle kolmannen kerran: "Simon, Johanneksen poika, olenko minä sinulle rakas?" Pietari tuli murheelliseksi siitä, että hän kolmannen kerran sanoi hänelle: "Olenko minä sinulle rakas?" ja vastasi hänelle: "Herra, sinä tiedät kaikki; sinä tiedät, että olet minulle rakas". Jeesus sanoi hänelle: "Ruoki minun lampaitani.
\par 18 Totisesti, totisesti minä sanon sinulle: kun olit nuori, niin sinä vyötit itsesi ja kuljit, minne tahdoit; mutta kun vanhenet, niin sinä ojennat kätesi, ja sinut vyöttää toinen ja vie sinut, minne et tahdo."
\par 19 Mutta sen hän sanoi antaakseen tietää, minkäkaltaisella kuolemalla Pietari oli kirkastava Jumalaa. Ja tämän sanottuaan hän lausui hänelle: "Seuraa minua".
\par 20 Niin Pietari kääntyi ja näki sen opetuslapsen seuraavan, jota Jeesus rakasti ja joka myös oli aterioitaessa nojannut hänen rintaansa vasten ja sanonut: "Herra, kuka on sinun kavaltajasi?"
\par 21 Kun Pietari hänet näki, sanoi hän Jeesukselle: "Herra, kuinka sitten tämän käy?"
\par 22 Jeesus sanoi hänelle: "Jos minä tahtoisin, että hän jää tänne siihen asti, kunnes minä tulen, mitä se sinuun koskee? Seuraa sinä minua."
\par 23 Niin semmoinen puhe levisi veljien keskuuteen, ettei se opetuslapsi kuole; mutta ei Jeesus sanonut hänelle, ettei hän kuole, vaan: "Jos minä tahtoisin, että hän jää tänne siihen asti, kunnes minä tulen, mitä se sinuun koskee?"
\par 24 Tämä on se opetuslapsi, joka todistaa näistä ja on nämä kirjoittanut; ja me tiedämme, että hänen todistuksensa on tosi.
\par 25 On paljon muutakin, mitä Jeesus teki; ja jos se kohta kohdalta kirjoitettaisiin. luulen, etteivät koko maailmaan mahtuisi ne kirjat, jotka pitäisi kirjoittaa.


\end{document}