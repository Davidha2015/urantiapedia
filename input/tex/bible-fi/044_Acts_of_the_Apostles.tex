\begin{document}

\title{Apostolien teot}


\chapter{1}

\par 1 Edellisessä kertomuksessani kirjoitin, oi Teofilus, kaikesta, mitä Jeesus alkoi tehdä ja opettaa,
\par 2 hamaan siihen päivään asti, jona hänet otettiin ylös, sittenkun hän Pyhän Hengen kautta oli antanut käskynsä apostoleille, jotka hän oli valinnut,
\par 3 ja joille hän myös kärsimisensä jälkeen moninaisten epäämättömien todistusten kautta osoitti elävänsä, ilmestyen heille neljänkymmenen päivän aikana ja puhuen Jumalan valtakunnasta.
\par 4 Ja kun hän oli yhdessä heidän kanssansa, käski hän heitä ja sanoi: "Älkää lähtekö Jerusalemista, vaan odottakaa Isältä sen lupauksen täyttymistä, jonka te olette minulta kuulleet.
\par 5 Sillä Johannes kastoi vedellä, mutta teidät kastetaan Pyhällä Hengellä, ei kauan näitten päivien jälkeen."
\par 6 Niin he ollessansa koolla kysyivät häneltä sanoen: "Herra, tälläkö ajalla sinä jälleen rakennat Israelille valtakunnan?"
\par 7 Hän sanoi heille: "Ei ole teidän asianne tietää aikoja eikä hetkiä, jotka Isä oman valtansa voimalla on asettanut,
\par 8 vaan, kun Pyhä Henki tulee teihin, niin te saatte voiman, ja te tulette olemaan minun todistajani sekä Jerusalemissa että koko Juudeassa ja Samariassa ja aina maan ääriin saakka".
\par 9 Kun hän oli tämän sanonut, kohotettiin hänet ylös heidän nähtensä, ja pilvi vei hänet pois heidän näkyvistään.
\par 10 Ja kun he katselivat taivaalle hänen mennessään, niin katso, heidän tykönänsä seisoi kaksi miestä valkeissa vaatteissa;
\par 11 ja nämä sanoivat: "Galilean miehet, mitä te seisotte ja katsotte taivaalle? Tämä Jeesus, joka otettiin teiltä ylös taivaaseen, on tuleva samalla tavalla, kuin te näitte hänen taivaaseen menevän."
\par 12 Silloin he palasivat Jerusalemiin vuorelta, jota kutsutaan Öljymäeksi ja joka on lähellä Jerusalemia, sapatinmatkan päässä.
\par 13 Ja kun he olivat tulleet kaupunkiin, menivät he siihen yläsaliin, jossa he tavallisesti oleskelivat: Pietari ja Johannes ja Jaakob ja Andreas, Filippus ja Tuomas, Bartolomeus ja Matteus, Jaakob Alfeuksen poika ja Simon, kiivailija, ja Juudas Jaakobin poika.
\par 14 Nämä kaikki pysyivät yksimielisesti rukouksessa vaimojen kanssa ja Marian, Jeesuksen äidin, kanssa ja Jeesuksen veljien kanssa.
\par 15 Ja niinä päivinä Pietari nousi veljien keskellä, kun oli väkeä koolla noin sata kaksikymmentä henkeä, ja sanoi:
\par 16 "Miehet, veljet, sen raamatunsanan piti käymän toteen, jonka Pyhä Henki on Daavidin suun kautta edeltä puhunut Juudaasta, joka rupesi niiden oppaaksi, jotka ottivat kiinni Jeesuksen.
\par 17 Sillä hän oli meidän joukkoomme luettu ja oli saanut osalleen tämän viran.
\par 18 Hän hankki itsellensä pellon väärintekonsa palkalla, ja hän suistui alas ja pakahtui keskeltä, niin että kaikki hänen sisälmyksensä valuivat ulos.
\par 19 Ja se tuli kaikkien Jerusalemin asukasten tietoon; ja niin sitä peltoa kutsutaan heidän kielellään Akeldamaksi, se on: Veripelloksi.
\par 20 Sillä psalmien kirjassa on kirjoitettuna: 'Tulkoon hänen talonsa autioksi, älköönkä siinä asukasta olko', ja: Ottakoon toinen hänen kaitsijatoimensa'.
\par 21 Niin pitää siis yhden niistä miehistä, jotka ovat vaeltaneet meidän kanssamme kaiken sen ajan, jona Herra Jeesus kävi sisälle ja ulos meidän tykönämme,
\par 22 Johanneksen kasteesta alkaen hamaan siihen päivään, jona hänet meiltä otettiin ylös, tuleman hänen ylösnousemisensa todistajaksi meidän kanssamme."
\par 23 Ja he asettivat ehdolle kaksi, Joosefin, jota kutsuttiin Barsabbaaksi, lisänimeltä Justukseksi, ja Mattiaan.
\par 24 Ja he rukoilivat ja sanoivat: "Herra, sinä, joka kaikkien sydämet tunnet, osoita, kummanko näistä kahdesta sinä olet valinnut
\par 25 ottamaan sen paikan tässä palveluksessa ja apostolinvirassa, josta Juudas vilpistyi pois, mennäkseen omaan paikkaansa".
\par 26 Ja he heittivät heistä arpaa, ja arpa lankesi Mattiaalle, ja hänet luettiin niiden yhdentoista kanssa apostolien joukkoon.

\chapter{2}

\par 1 Ja kun helluntaipäivä oli tullut, olivat he kaikki yhdessä koolla.
\par 2 Ja tuli yhtäkkiä humaus taivaasta, niinkuin olisi käynyt väkevä tuulispää, ja täytti koko huoneen, jossa he istuivat.
\par 3 Ja he näkivät ikäänkuin tulisia kieliä, jotka jakaantuivat ja asettuivat heidän itsekunkin päälle.
\par 4 Ja he tulivat kaikki Pyhällä Hengellä täytetyiksi ja alkoivat puhua muilla kielillä, sen mukaan mitä Henki heille puhuttavaksi antoi.
\par 5 Ja Jerusalemissa asui juutalaisia, jumalaapelkääväisiä miehiä, kaikkinaisista kansoista, mitä taivaan alla on.
\par 6 Ja kun tämä ääni kuului, niin kokoontui paljon kansaa; ja he tulivat ymmälle, sillä kukin kuuli heidän puhuvan hänen omaa kieltänsä.
\par 7 Ja he hämmästyivät ja ihmettelivät sanoen: "Katso, eivätkö nämä kaikki, jotka puhuvat, ole galilealaisia?
\par 8 Kuinka me sitten kuulemme kukin sen maan kieltä, jossa olemme syntyneet?
\par 9 Me parttilaiset ja meedialaiset ja eelamilaiset ja me, jotka asumme Mesopotamiassa, Juudeassa ja Kappadokiassa, Pontossa ja Aasiassa,
\par 10 Frygiassa ja Pamfyliassa, Egyptissä ja Kyrenen puoleisen Liibyan alueilla, ja täällä oleskelevat roomalaiset, juutalaiset ja käännynnäiset,
\par 11 kreetalaiset ja arabialaiset, me kuulemme kukin heidän puhuvan omalla kielellämme Jumalan suuria tekoja."
\par 12 Ja he olivat kaikki hämmästyksissään eivätkä tienneet, mitä ajatella, ja sanoivat toinen toisellensa: "Mitä tämä mahtaakaan olla?"
\par 13 Mutta toiset pilkkasivat heitä ja sanoivat: "He ovat täynnä makeata viiniä".
\par 14 Niin Pietari astui esiin niiden yhdentoista kanssa, korotti äänensä ja puhui heille: "Miehet, juutalaiset ja kaikki Jerusalemissa asuvaiset, olkoon tämä teille tiettävä, ja ottakaa minun sanani korviinne.
\par 15 Eivät nämä ole juovuksissa, niinkuin te luulette; sillä nyt on vasta kolmas hetki päivästä.
\par 16 Vaan tämä on se, mikä on sanottu profeetta Jooelin kautta:
\par 17 'Ja on tapahtuva viimeisinä päivinä, sanoo Jumala, että minä vuodatan Henkeni kaiken lihan päälle, ja teidän poikanne ja tyttärenne ennustavat, ja nuorukaisenne näkyjä näkevät, ja vanhuksenne unia uneksuvat.
\par 18 Ja myös palvelijaini ja palvelijattarieni päälle minä niinä päivinä vuodatan Henkeni, ja he ennustavat.
\par 19 Ja minä annan näkyä ihmeitä ylhäällä taivaalla ja merkkejä alhaalla maan päällä, verta ja tulta ja savupatsaita.
\par 20 Aurinko muuttuu pimeydeksi ja kuu vereksi, ennenkuin Herran päivä tulee, se suuri ja julkinen.
\par 21 Ja on tapahtuva, että jokainen, joka huutaa avuksi Herran nimeä, pelastuu.'
\par 22 Te Israelin miehet, kuulkaa nämä sanat: Jeesuksen, Nasaretilaisen, sen miehen, josta Jumala todisti teille voimallisilla teoilla ja ihmeillä ja merkeillä, joita Jumala hänen kauttansa teki teidän keskellänne, niinkuin te itse tiedätte,
\par 23 hänet, joka teille luovutettiin, Jumalan ennaltamäärätyn päätöksen ja edeltätietämyksen mukaan, te laista tietämättömien miesten kätten kautta naulitsitte ristille ja tapoitte.
\par 24 Hänet Jumala herätti ja päästi kuoleman kivuista, niinkuin ei ollutkaan mahdollista, että kuolema olisi voinut hänet pitää.
\par 25 Sillä Daavid sanoo hänestä: 'Minä näen alati edessäni Herran, sillä hän on minun oikealla puolellani, etten horjahtaisi.
\par 26 Sentähden minun sydämeni iloitsee ja kieleni riemuitsee, ja myös minun ruumiini on lepäävä toivossa;
\par 27 sillä sinä et hylkää minun sieluani tuonelaan etkä salli Pyhäsi nähdä katoavaisuutta.
\par 28 Sinä teet minulle tiettäviksi elämän tiet, sinä täytät minut ilolla kasvojesi edessä.'
\par 29 Te miehet, veljet, on lupa teille rohkeasti sanoa, mitä kantaisäämme Daavidiin tulee, että hän on sekä kuollut että haudattu; onhan hänen hautansa meidän keskellämme vielä tänäkin päivänä.
\par 30 Koska hän nyt oli profeetta ja tiesi, että Jumala oli valalla vannoen hänelle luvannut asettavansa hänen kupeittensa hedelmän hänen valtaistuimelleen,
\par 31 niin hän edeltä nähden puhui Kristuksen ylösnousemuksesta, sanoen, ettei Kristus ollut jäävä hyljätyksi tuonelaan eikä hänen ruumiinsa näkevä katoavaisuutta.
\par 32 Tämän Jeesuksen on Jumala herättänyt, minkä todistajia me kaikki olemme.
\par 33 Koska hän siis on Jumalan oikean käden voimalla korotettu ja on Isältä saanut Pyhän Hengen lupauksen, on hän vuodattanut sen, minkä te nyt näette ja kuulette.
\par 34 Sillä ei Daavid ole astunut ylös taivaisiin, vaan hän sanoo itse: 'Herra sanoi minun Herralleni: Istu minun oikealle puolelleni,
\par 35 kunnes minä panen sinun vihollisesi sinun jalkojesi astinlaudaksi.'
\par 36 Varmasti tietäköön siis koko Israelin huone, että Jumala on hänet Herraksi ja Kristukseksi tehnyt, tämän Jeesuksen, jonka te ristiinnaulitsitte."
\par 37 Kun he tämän kuulivat, saivat he piston sydämeensä ja sanoivat Pietarille ja muille apostoleille: "Miehet, veljet, mitä meidän pitää tekemän?"
\par 38 Niin Pietari sanoi heille: "Tehkää parannus ja ottakoon kukin teistä kasteen Jeesuksen Kristuksen nimeen syntienne anteeksisaamiseksi, niin te saatte Pyhän Hengen lahjan.
\par 39 Sillä teille ja teidän lapsillenne tämä lupaus on annettu ja kaikille, jotka kaukana ovat, ketkä ikinä Herra, meidän Jumalamme, kutsuu."
\par 40 Ja monilla muillakin sanoilla hän vakaasti todisti; ja hän kehoitti heitä sanoen: "Antakaa pelastaa itsenne tästä nurjasta sukupolvesta".
\par 41 Jotka nyt ottivat hänen sanansa vastaan, ne kastettiin, ja niin heitä lisääntyi sinä päivänä noin kolmetuhatta sielua.
\par 42 Ja he pysyivät apostolien opetuksessa ja keskinäisessä yhteydessä ja leivän murtamisessa ja rukouksissa.
\par 43 Ja jokaiselle sielulle tuli pelko; ja monta ihmettä ja tunnustekoa tapahtui apostolien kautta.
\par 44 Ja kaikki, jotka uskoivat, olivat yhdessä ja pitivät kaikkea yhteisenä,
\par 45 ja he myivät maansa ja tavaransa ja jakelivat kaikille, sen mukaan kuin kukin tarvitsi.
\par 46 Ja he olivat alati, joka päivä, yksimielisesti pyhäkössä ja mursivat kodeissa leipää ja nauttivat ruokansa riemulla ja sydämen yksinkertaisuudella,
\par 47 kiittäen Jumalaa ja ollen kaiken kansan suosiossa. Ja Herra lisäsi heidän yhteyteensä joka päivä niitä, jotka saivat pelastuksen.

\chapter{3}

\par 1 Ja Pietari ja Johannes menivät pyhäkköön yhdeksännellä hetkellä, rukoushetkellä.
\par 2 Silloin kannettiin esille miestä, joka oli ollut rampa hamasta äitinsä kohdusta ja jonka he joka päivä panivat pyhäkön niin kutsutun Kauniin portin pieleen anomaan almua pyhäkköön meneviltä.
\par 3 Nähdessään Pietarin ja Johanneksen, kun he olivat menossa pyhäkköön, hän pyysi heiltä almua.
\par 4 Niin Pietari ja Johannes katsoivat häneen kiinteästi, ja Pietari sanoi: "Katso meihin".
\par 5 Ja hän tarkkasi heitä odottaen heiltä jotakin saavansa.
\par 6 Niin Pietari sanoi: "Hopeaa ja kultaa ei minulla ole, mutta mitä minulla on, sitä minä sinulle annan: Jeesuksen Kristuksen, Nasaretilaisen, nimessä, nouse ja käy."
\par 7 Ja hän tarttui hänen oikeaan käteensä ja nosti hänet ylös; ja heti hänen jalkansa ja nilkkansa vahvistuivat,
\par 8 ja hän hypähti pystyyn, seisoi ja käveli; ja hän meni heidän kanssansa pyhäkköön, käyden ja hypellen ja ylistäen Jumalaa.
\par 9 Ja kaikki kansa näki hänen kävelevän ja ylistävän Jumalaa;
\par 10 ja he tunsivat hänet siksi, joka almuja saadakseen oli istunut pyhäkön Kauniin portin pielessä, ja he olivat täynnä hämmästystä ja ihmettelyä siitä, mikä hänelle oli tapahtunut.
\par 11 Ja kun hän yhä pysyttäytyi Pietarin ja Johanneksen seurassa, riensi kaikki kansa hämmästyksen vallassa heidän luoksensa niin sanottuun Salomon pylväskäytävään.
\par 12 Sen nähdessään Pietari rupesi puhumaan kansalle ja sanoi: "Israelin miehet, mitä te tätä ihmettelette, tai mitä te meitä noin katselette, ikäänkuin me omalla voimallamme tai hurskaudellamme olisimme saaneet hänet kävelemään.
\par 13 Aabrahamin ja Iisakin ja Jaakobin Jumala, meidän isiemme Jumala, on kirkastanut Poikansa Jeesuksen, jonka te annoitte alttiiksi ja kielsitte Pilatuksen edessä, kun tämä oli päättänyt hänet päästää.
\par 14 Te kielsitte Pyhän ja Vanhurskaan ja anoitte, että teille annettaisiin murhamies,
\par 15 mutta elämän ruhtinaan te tapoitte; hänet Jumala on herättänyt kuolleista, ja me olemme sen todistajat.
\par 16 Ja uskon kautta hänen nimeensä on hänen nimensä vahvistanut tämän miehen, jonka te näette ja tunnette, ja usko, jonka Jeesus vaikuttaa, on hänelle antanut hänen jäsentensä terveyden kaikkien teidän nähtenne.
\par 17 Ja nyt, veljet, minä tiedän, että te olette tietämättömyydestä sen tehneet, te niinkuin teidän hallitusmiehennekin.
\par 18 Mutta näin on Jumala täyttänyt sen, minkä hän oli edeltä ilmoittanut kaikkien profeettain suun kautta, että nimittäin hänen Voideltunsa piti kärsimän.
\par 19 Tehkää siis parannus ja kääntykää, että teidän syntinne pyyhittäisiin pois,
\par 20 että virvoituksen ajat tulisivat Herran kasvoista ja hän lähettäisi hänet, joka on teille edeltämäärätty, Kristuksen Jeesuksen.
\par 21 Taivaan piti omistaman hänet niihin aikoihin asti, jolloin kaikki jälleen kohdallensa asetetaan, mistä Jumala on ikiajoista saakka puhunut pyhäin profeettainsa suun kautta.
\par 22 Sillä Mooses on sanonut: 'Profeetan, minun kaltaiseni, Herra Jumala on teille herättävä veljienne joukosta; häntä kuulkaa kaikessa, mitä hän teille puhuu.
\par 23 Ja on tapahtuva, että jokainen, joka ei sitä profeettaa kuule, hävitetään kansasta.'
\par 24 Ja kaikki profeetat Samuelista alkaen ja kaikki järjestään, jotka puhuneet ovat, ovat myös ennustaneet näitä päiviä.
\par 25 Te olette profeettain ja sen liiton lapsia, jonka Jumala teki meidän isiemme kanssa sanoen Aabrahamille: 'Ja sinun siemenessäsi tulevat siunatuiksi kaikki maan sukukunnat'.
\par 26 Teille ensiksi Jumala on herättänyt Poikansa ja lähettänyt hänet siunaamaan teitä, kun käännytte itsekukin pois pahuudestanne."

\chapter{4}

\par 1 Mutta kun he puhuivat kansalle, astuivat papit ja pyhäkön vartioston päällikkö ja saddukeukset heidän eteensä,
\par 2 närkästyneinä siitä, että he opettivat kansaa ja julistivat Jeesuksessa ylösnousemusta kuolleista.
\par 3 Ja he kävivät heihin käsiksi ja panivat heidät vankeuteen seuraavaan päivään asti, sillä oli jo ehtoo.
\par 4 Mutta monet niistä, jotka olivat kuulleet sanan, uskoivat, ja miesten luku nousi noin viiteentuhanteen.
\par 5 Seuraavana päivänä heidän hallitusmiehensä ja vanhimpansa ja kirjanoppineensa kokoontuivat Jerusalemissa,
\par 6 niin myös ylimmäinen pappi Hannas ja Kaifas ja Johannes ja Aleksander sekä kaikki, jotka olivat ylimmäispapillista sukua.
\par 7 Ja he asettivat heidät eteensä ja kysyivät: "Millä voimalla tai kenen nimeen te tämän teitte?"
\par 8 Silloin Pietari, Pyhää Henkeä täynnä, sanoi heille: "Kansan hallitusmiehet ja vanhimmat!
\par 9 Jos meitä tänään kuulustellaan sairaalle miehelle tehdystä hyvästä työstä ja siitä, kenen kautta hän on parantunut,
\par 10 niin olkoon teille kaikille ja koko Israelin kansalle tiettävä, että Jeesuksen Kristuksen, Nasaretilaisen, nimen kautta, hänen, jonka te ristiinnaulitsitte, mutta jonka Jumala kuolleista herätti, hänen nimensä kautta tämä seisoo terveenä teidän edessänne.
\par 11 Hän on 'se kivi, jonka te, rakentajat, hylkäsitte, mutta joka on kulmakiveksi tullut'.
\par 12 Eikä ole pelastusta yhdessäkään toisessa; sillä ei ole taivaan alla muuta nimeä ihmisille annettu, jossa meidän pitäisi pelastuman."
\par 13 Mutta kun he näkivät Pietarin ja Johanneksen rohkeuden ja havaitsivat heidän olevan koulunkäymättömiä ja oppimattomia miehiä, he ihmettelivät; ja he tunsivat heidät niiksi, jotka olivat olleet Jeesuksen kanssa.
\par 14 Ja nähdessään parannetun miehen seisovan heidän kanssansa he eivät voineet mitään vastaansanoa,
\par 15 vaan käskettyään heidän astua ulos neuvostosta he pitivät keskenänsä neuvoa
\par 16 ja sanoivat: "Mitä me teemme näille miehille? Sillä että heidän kauttansa on tapahtunut ilmeinen ihme, sen kaikki Jerusalemin asukkaat tietävät, emmekä me voi sitä kieltää.
\par 17 Mutta ettei se leviäisi laajemmalle kansaan, niin kieltäkäämme ankarasti heitä enää tähän nimeen puhumasta yhdellekään ihmiselle."
\par 18 Niin he kutsuivat heidät ja kielsivät heitä mitään puhumasta ja opettamasta Jeesuksen nimeen.
\par 19 Mutta Pietari ja Johannes vastasivat heille ja sanoivat: "Päättäkää itse, onko oikein Jumalan edessä kuulla teitä enemmän kuin Jumalaa;
\par 20 mutta me emme voi olla puhumatta siitä, mitä olemme nähneet ja kuulleet".
\par 21 Niin he uhkasivat heitä vielä enemmän ja päästivät heidät, koska eivät kansan tähden keksineet, miten rangaista heitä, sillä kaikki ylistivät Jumalaa siitä, mitä tapahtunut oli.
\par 22 Sillä sivu neljänkymmenen oli jo vuosiltaan se mies, jossa tämä parantumisen ihme oli tapahtunut.
\par 23 Ja päästyään vapaiksi he menivät omiensa tykö ja kertoivat kaiken, mitä ylipapit ja vanhimmat olivat heille sanoneet.
\par 24 Sen kuultuansa he yksimielisesti korottivat äänensä Jumalan puoleen ja sanoivat: "Herra, sinä, joka olet tehnyt taivaan ja maan ja meren ja kaikki, mitä niissä on!
\par 25 Sinä, joka Pyhän Hengen kautta, isämme Daavidin, sinun palvelijasi, suun kautta, olet puhunut: 'Miksi pakanat pauhaavat ja kansat turhia ajattelevat?
\par 26 Maan kuninkaat nousevat, ja ruhtinaat kokoontuvat yhteen Herraa ja hänen Voideltuansa vastaan.'
\par 27 Sillä totisesti, tässä kaupungissa kokoontuivat sinun pyhää Poikaasi Jeesusta vastaan, jonka sinä olet voidellut, sekä Herodes että Pontius Pilatus pakanain ja Israelin sukukuntain kanssa,
\par 28 tekemään kaiken, minkä sinun kätesi ja päätöksesi oli edeltämäärännyt tapahtuvaksi.
\par 29 Ja nyt, Herra, katso heidän uhkauksiansa ja anna palvelijaisi kaikella rohkeudella puhua sinun sanaasi;
\par 30 ja ojenna kätesi, niin että sairaat parantuvat ja tunnustekoja ja ihmeitä tapahtuu sinun pyhän Poikasi Jeesuksen nimen kautta."
\par 31 Ja kun he olivat rukoilleet, vapisi se paikka, jossa he olivat koolla, ja he tulivat kaikki Pyhällä Hengellä täytetyiksi ja puhuivat Jumalan sanaa rohkeasti.
\par 32 Ja uskovaisten suuressa joukossa oli yksi sydän ja yksi sielu; eikä kenkään heistä sanonut omaksensa mitään siitä, mitä hänellä oli, vaan kaikki oli heillä yhteistä.
\par 33 Ja apostolit todistivat suurella voimalla Herran Jeesuksen ylösnousemuksesta, ja suuri armo oli heillä kaikilla.
\par 34 Ei myöskään ollut heidän seassaan ketään puutteenalaista; sillä kaikki, joilla oli maatiloja tai taloja, myivät ne ja toivat myytyjen hinnan
\par 35 ja panivat apostolien jalkojen juureen; ja jokaiselle jaettiin sen mukaan, kuin hän tarvitsi.
\par 36 Niinpä Joosef, jota apostolit kutsuivat Barnabaaksi - se on käännettynä: kehoittaja - leeviläinen, syntyisin Kyprosta,
\par 37 myi omistamansa pellon, toi rahat ja pani ne apostolien jalkojen juureen.

\chapter{5}

\par 1 Mutta eräs mies, nimeltä Ananias, ja hänen vaimonsa Safiira myivät maatilan,
\par 2 ja mies kätki vaimonsa tieten osan hinnasta, ja osan hän toi ja pani apostolien jalkojen eteen.
\par 3 Mutta Pietari sanoi: "Ananias, miksi on saatana täyttänyt sinun sydämesi, niin että koetit pettää Pyhää Henkeä ja kätkit osan maatilan hinnasta?
\par 4 Eikö se myymätönnä ollut sinun omasi, ja eikö myynnin jälkeenkin sen hinta ollut sinun? Miksi päätit sydämessäsi tämän tehdä? Et sinä ole valhetellut ihmisille, vaan Jumalalle."
\par 5 Kun Ananias kuuli nämä sanat, kaatui hän maahan ja heitti henkensä. Ja suuri pelko valtasi kaikki, jotka sen kuulivat.
\par 6 Ja nuoret miehet nousivat ja korjasivat hänet ja kantoivat hänet pois ja hautasivat.
\par 7 Noin kolmen hetken kuluttua hänen vaimonsa tuli sisään eikä tiennyt, mitä oli tapahtunut.
\par 8 Niin Pietari kysyi häneltä: "Sano minulle: siihenkö hintaan te myitte maatilan?" Hän vastasi: "Kyllä, juuri siihen hintaan".
\par 9 Mutta Pietari sanoi hänelle: "Miksi olette yksissä neuvoin käyneet kiusaamaan Herran Henkeä? Katso, niiden jalat, jotka hautasivat sinun miehesi, ovat oven takana, ja he kantavat sinutkin pois."
\par 10 Niin hän heti kaatui hänen jalkojensa eteen ja heitti henkensä; ja kun nuorukaiset tulivat sisään, tapasivat he hänet kuolleena, kantoivat pois ja hautasivat hänet hänen miehensä viereen.
\par 11 Ja suuri pelko valtasi koko seurakunnan ja kaikki ne, jotka tämän kuulivat.
\par 12 Ja apostolien kätten kautta tapahtui kansassa monta tunnustekoa ja ihmettä; ja he olivat kaikki yksimielisesti koolla Salomon pylväskäytävässä.
\par 13 Eikä muista kukaan uskaltanut heihin liittyä, mutta kansa piti heitä suuressa kunniassa.
\par 14 Ja yhä enemmän karttui niitä, jotka uskoivat Herraan, sekä miehiä että naisia suuret joukot.
\par 15 Kannettiinpa sairaita kaduillekin ja pantiin vuoteille ja paareille, että Pietarin kulkiessa edes hänen varjonsa sattuisi johonkuhun heistä.
\par 16 Myöskin kaupungeista Jerusalemin ympäriltä tuli paljon kansaa, ja he toivat sairaita ja saastaisten henkien vaivaamia, ja ne kaikki tulivat parannetuiksi.
\par 17 Silloin nousi ylimmäinen pappi ja kaikki, jotka olivat hänen puolellansa, saddukeusten lahko, ja he tulivat kiihkoa täyteen
\par 18 ja kävivät käsiksi apostoleihin ja panivat heidät yleiseen vankihuoneeseen.
\par 19 Mutta yöllä avasi Herran enkeli vankilan ovet ja vei heidät ulos ja sanoi:
\par 20 "Menkää ja astukaa esiin ja puhukaa pyhäkössä kansalle kaikki tämän elämän sanat".
\par 21 Sen kuultuansa he menivät päivän koittaessa pyhäkköön ja opettivat. Niin saapui ylimmäinen pappi ja ne, jotka olivat hänen puolellansa, ja he kutsuivat koolle neuvoston ja israelilaisten vanhinten kokouksen; ja he lähettivät noutamaan heitä vankilasta.
\par 22 Mutta kun oikeudenpalvelijat tulivat vankilaan, eivät he löytäneet heitä sieltä, vaan palasivat takaisin ja kertoivat,
\par 23 sanoen: "Vankilan me kyllä huomasimme hyvin tarkasti suljetuksi ja vartijat seisomassa ovien edessä; mutta kun avasimme, emme sisältä ketään löytäneet".
\par 24 Kun pyhäkön vartioston päällikkö ja ylipapit kuulivat nämä sanat, eivät he tienneet, mitä heistä ajatella ja mitä tästä tulisi.
\par 25 Niin tuli joku ja kertoi heille: "Katso, ne miehet, jotka te panitte vankilaan, seisovat pyhäkössä ja opettavat kansaa".
\par 26 Silloin päällikkö meni oikeudenpalvelijain kanssa ja nouti heidät; ei kuitenkaan väkisin, sillä he pelkäsivät, että kansa heidät kivittäisi.
\par 27 Ja he toivat heidät ja asettivat neuvoston eteen. Ja ylimmäinen pappi kuulusteli heitä
\par 28 ja sanoi: "Me olemme kieltämällä kieltäneet teitä opettamasta tähän nimeen; ja katso, te olette täyttäneet Jerusalemin opetuksellanne ja tahdotte saattaa meidän päällemme tuon miehen veren".
\par 29 Mutta Pietari ja muut apostolit vastasivat ja sanoivat: "Enemmän tulee totella Jumalaa kuin ihmisiä.
\par 30 Meidän isiemme Jumala on herättänyt Jeesuksen, jonka te ripustitte puuhun ja surmasitte.
\par 31 Hänet on Jumala oikealla kädellänsä korottanut Päämieheksi ja Vapahtajaksi, antamaan Israelille parannusta ja syntien anteeksiantamusta.
\par 32 Ja me olemme kaiken tämän todistajat, niin myös Pyhä Henki, jonka Jumala on antanut niille, jotka häntä tottelevat."
\par 33 Kun he sen kuulivat, viilsi se heidän sydäntänsä, ja he tahtoivat tappaa heidät.
\par 34 Mutta neuvostossa nousi eräs fariseus, nimeltä Gamaliel, lainopettaja, jota koko kansa piti arvossa, ja hän käski viedä miehet vähäksi aikaa ulos.
\par 35 Sitten hän sanoi neuvostolle: "Israelin miehet, kavahtakaa, mitä aiotte tehdä näille miehille.
\par 36 Sillä ennen näitä päiviä nousi Teudas, sanoen jokin olevansa, ja häneen liittyi noin neljäsataa miestä; hänet tapettiin, ja kaikki, jotka olivat häneen suostuneet, hajotettiin, ja he joutuivat häviöön.
\par 37 Hänen jälkeensä nousi Juudas, galilealainen, verollepanon päivinä ja vietteli kansaa luopumaan puolellensa; hänkin hukkui, ja kaikki, jotka olivat suostuneet häneen, hajotettiin.
\par 38 Ja nyt minä sanon teille: pysykää erillänne näistä miehistä ja antakaa heidän olla; sillä jos tämä hanke eli tämä teko on ihmisistä, niin se tyhjään raukeaa;
\par 39 mutta jos se on Jumalasta, niin te ette voi heitä kukistaa. Varokaa, ettei teitä ehkä havaittaisi sotiviksi itse Jumalaa vastaan."
\par 40 Niin he noudattivat hänen neuvoansa. Ja he kutsuivat apostolit sisään ja pieksättivät heitä ja kielsivät heitä puhumasta Jeesuksen nimeen ja päästivät heidät menemään.
\par 41 Niin he lähtivät pois neuvostosta iloissaan siitä, että olivat katsotut arvollisiksi kärsimään häväistystä Jeesuksen nimen tähden.
\par 42 Eivätkä he lakanneet, vaan opettivat joka päivä pyhäkössä ja kodeissa ja julistivat evankeliumia Kristuksesta Jeesuksesta.

\chapter{6}

\par 1 Niinä päivinä, kun opetuslasten luku lisääntyi, syntyi hellenisteissä nurinaa hebrealaisia vastaan siitä, että heidän leskiänsä syrjäytettiin jokapäiväisessä avunannossa.
\par 2 Niin ne kaksitoista kutsuivat kokoon opetuslasten joukon ja sanoivat: "Ei ole soveliasta, että me laiminlyömme Jumalan sanan toimittaaksemme pöytäpalvelusta.
\par 3 Valitkaa sentähden, veljet, keskuudestanne seitsemän miestä, joista on hyvä todistus ja jotka ovat Henkeä ja viisautta täynnä, niin me asetamme heidät tähän toimeen.
\par 4 Mutta me tahdomme pysyä rukouksessa ja sanan palveluksessa."
\par 5 Ja se puhe kelpasi kaikelle joukolle; ja he valitsivat Stefanuksen, miehen, joka oli täynnä uskoa ja Pyhää Henkeä, ja Filippuksen ja Prokoruksen ja Nikanorin ja Timonin ja Parmenaan ja Nikolauksen, antiokialaisen käännynnäisen,
\par 6 ja asettivat heidät apostolien eteen, ja nämä rukoilivat ja panivat kätensä heidän päällensä.
\par 7 Ja Jumalan sana menestyi, ja opetuslasten luku lisääntyi suuresti Jerusalemissa. Ja lukuisa joukko pappeja tuli uskolle kuuliaisiksi.
\par 8 Ja Stefanus, täynnä armoa ja voimaa, teki suuria ihmeitä ja tunnustekoja kansassa.
\par 9 Niin nousi muutamia niin kutsutusta libertiinien ja kyreniläisten ja aleksandrialaisten synagoogasta sekä niiden joukosta, jotka olivat Kilikiasta ja Aasiasta, väittelemään Stefanuksen kanssa,
\par 10 mutta he eivät kyenneet pitämään puoliaan sitä viisautta ja henkeä vastaan, jolla hän puhui.
\par 11 Silloin he salaa hankkivat miehiä sanomaan: "Me olemme kuulleet hänen puhuvan pilkkasanoja Moosesta ja Jumalaa vastaan".
\par 12 Ja he yllyttivät kansan ja vanhimmat ja kirjanoppineet ja astuivat esiin, raastoivat hänet mukaansa ja veivät neuvoston eteen.
\par 13 Ja he toivat esiin vääriä todistajia, jotka sanoivat: "Tämä mies ei lakkaa puhumasta tätä pyhää paikkaa vastaan ja lakia vastaan;
\par 14 sillä me olemme kuulleet hänen sanovan, että Jeesus, tuo Nasaretilainen, on hajottava maahan tämän paikan ja muuttava ne säädökset, jotka Mooses on meille antanut".
\par 15 Ja kaikki, jotka neuvostossa istuivat, loivat katseensa häneen, ja hänen kasvonsa olivat heistä niinkuin enkelin kasvot.

\chapter{7}

\par 1 Niin ylimmäinen pappi sanoi: "Onko niin?"
\par 2 Stefanus sanoi: "Miehet, veljet ja isät, kuulkaa! Kirkkauden Jumala ilmestyi meidän isällemme Aabrahamille hänen ollessaan Mesopotamiassa, ennenkuin hän oli asettunut asumaan Harraniin,
\par 3 ja sanoi hänelle: 'Lähde maastasi ja suvustasi ja mene siihen maahan, jonka minä sinulle osoitan'.
\par 4 Silloin hän lähti kaldealaisten maasta ja asettui asumaan Harraniin. Ja kun hänen isänsä oli kuollut, siirsi Jumala hänet tähän maahan, jossa te nyt asutte.
\par 5 Eikä hän antanut hänelle siinä perintöosaa, ei jalan leveyttäkään, vaan lupasi, Aabrahamin vielä lapsetonna ollessa, antaa sen hänelle ja hänen siemenelleen hänen jälkeensä.
\par 6 Ja Jumala puhui näin: 'Hänen jälkeläisensä tulevat olemaan muukalaisina vieraalla maalla, ja siellä ne tekevät heidät orjiksensa ja sortavat heitä neljäsataa vuotta;
\par 7 ja sen kansan, jonka orjiksi he tulevat, minä olen tuomitseva', sanoi Jumala, 'ja sen jälkeen he lähtevät sieltä ja palvelevat minua tässä paikassa'.
\par 8 Ja hän antoi hänelle ympärileikkauksen liiton; ja niin Aabrahamille syntyi Iisak, ja hän ympärileikkasi hänet kahdeksantena päivänä, ja Iisakille syntyi Jaakob, ja Jaakobille ne kaksitoista kantaisää.
\par 9 Ja kantaisät kadehtivat Joosefia ja myivät hänet Egyptiin. Mutta Jumala oli hänen kanssansa
\par 10 ja pelasti hänet kaikista hänen ahdistuksistansa. Ja hän antoi hänelle armon ja viisauden faraon, Egyptin kuninkaan, edessä; ja tämä pani hänet Egyptin ja kaiken huoneensa haltijaksi.
\par 11 Ja tuli nälänhätä koko Egyptiin ja Kanaaniin ja suuri vaiva, eivätkä meidän isämme saaneet mistään ravintoa.
\par 12 Mutta kun Jaakob kuuli Egyptissä olevan viljaa, lähetti hän meidän isämme sinne ensimmäisen kerran.
\par 13 Ja toisella kerralla veljet tunsivat Joosefin, ja farao sai tietää Joosefin sukuperän.
\par 14 Niin Joosef lähetti kutsumaan luokseen isänsä Jaakobin ja koko sukunsa, seitsemänkymmentä viisi henkeä.
\par 15 Ja Jaakob meni Egyptiin ja kuoli siellä, niin myös kuolivat isämme,
\par 16 ja heidät siirrettiin Sikemiin ja pantiin siihen hautaan, jonka Aabraham oli rahalla ostanut Emmorin lapsilta Sikemissä.
\par 17 Mutta sitä mukaa kuin lähestyi sen lupauksen aika, jonka Jumala oli Aabrahamille antanut, kasvoi kansa ja lisääntyi Egyptissä,
\par 18 kunnes Egyptiä hallitsemaan nousi toinen kuningas, joka ei Joosefista mitään tiennyt.
\par 19 Tämä kohteli kavalasti meidän kansaamme ja sorti meidän isiämme ja pakotti heidät panemaan heitteille pienet lapsensa, etteivät ne jäisi eloon.
\par 20 Siihen aikaan syntyi Mooses, ja hän oli Jumalalle otollinen. Häntä elätettiin kolme kuukautta isänsä kodissa.
\par 21 Mutta kun hänet oli pantu heitteille, otti faraon tytär hänet ja kasvatti hänet pojaksensa.
\par 22 Ja Mooses kasvatettiin kaikkeen egyptiläisten viisauteen, ja hän oli voimallinen sanoissa ja teoissa.
\par 23 Mutta kun hän oli täyttänyt neljäkymmentä vuotta, heräsi hänen sydämessään ajatus mennä katsomaan veljiänsä, israelilaisia.
\par 24 Ja nähdessään eräälle heistä vääryyttä tehtävän hän puolusti häntä ja kosti pahoinpidellyn puolesta ja löi egyptiläisen kuoliaaksi.
\par 25 Ja hän luuli veljiensä ymmärtävän, että Jumala hänen kätensä kautta oli antava heille pelastuksen; mutta he eivät sitä ymmärtäneet.
\par 26 Seuraavana päivänä hän ilmestyi heidän luoksensa, heidän riidellessään, ja koetti saada heitä sopimaan sanoen: 'Tehän olette veljiä, miehet; miksi teette vääryyttä toisillenne?'
\par 27 Mutta se, joka teki lähimmäisellensä vääryyttä, työnsi hänet pois ja sanoi: 'Kuka on asettanut sinut meidän päämieheksemme ja tuomariksemme?
\par 28 Aiotko tappaa minutkin, niinkuin eilen tapoit egyptiläisen?'
\par 29 Tämän puheen tähden Mooses pakeni ja oli muukalaisena Midianin maassa, ja siellä hänelle syntyi kaksi poikaa.
\par 30 Ja kun neljäkymmentä vuotta oli kulunut, ilmestyi hänelle Siinain vuoren erämaassa enkeli palavan orjantappurapensaan liekissä.
\par 31 Kun Mooses sen näki, ihmetteli hän tätä näkyä; ja kun hän meni tarkemmin katsomaan, kuului Herran ääni, joka sanoi:
\par 32 'Minä olen sinun isiesi Jumala, Aabrahamin ja Iisakin ja Jaakobin Jumala'. Niin Mooses alkoi vapista eikä tohtinut katsoa sinne.
\par 33 Mutta Herra sanoi hänelle: 'Riisu kengät jalastasi; sillä paikka, jossa seisot, on pyhä maa.
\par 34 Minä olen nähnyt kansani kurjuuden Egyptissä ja kuullut heidän huokauksensa, ja minä olen astunut alas vapauttamaan heidät. Ja nyt, tule tänne, minä lähetän sinut Egyptiin.'
\par 35 Tämän Mooseksen, jonka he kielsivät sanoen: 'Kuka sinut on asettanut päämieheksi ja tuomariksi?', hänet Jumala lähetti päämieheksi ja lunastajaksi sen enkelin kautta, joka hänelle orjantappurapensaassa oli ilmestynyt.
\par 36 Hän johdatti heidät sieltä pois, tehden ihmeitä ja tunnustekoja Egyptin maassa ja Punaisessa meressä ja erämaassa neljänäkymmenenä vuotena.
\par 37 Tämä on se Mooses, joka sanoi israelilaisille: 'Profeetan, minun kaltaiseni, Jumala on teille herättävä teidän veljienne joukosta'.
\par 38 Hän on se, joka seurakunnassa, erämaassa, oli enkelin kanssa, joka puhui hänelle Siinain vuorella, ja oli myös isiemme kanssa; ja hän sai eläviä sanoja meille annettaviksi.
\par 39 Mutta häntä meidän isämme eivät tahtoneet totella, vaan työnsivät hänet pois ja kääntyivät sydämessänsä jälleen Egyptiin,
\par 40 sanoen Aaronille: 'Tee meille jumalia, jotka käyvät meidän edellämme, sillä me emme tiedä, mitä on tapahtunut Moosekselle, hänelle, joka johdatti meidät Egyptin maasta'.
\par 41 Ja he tekivät niinä päivinä vasikan ja toivat uhreja epäjumalalleen ja riemuitsivat kättensä töistä.
\par 42 Mutta Jumala kääntyi heistä pois ja hylkäsi heidät palvelemaan taivaan sotajoukkoa, niinkuin on kirjoitettu profeettain kirjassa: 'Toitteko te teurasuhreja ja muita uhreja minulle erämaassa neljänäkymmenenä vuotena, te, Israelin heimo?
\par 43 Ette; vaan te kannoitte Molokin majaa ja Romfa jumalan tähteä, niitä kuvia, jotka te olitte tehneet kumarrettaviksenne. Sentähden minä siirrän teidät toiselle puolelle Babylonin.'
\par 44 Todistuksen maja oli meidän isillämme erämaassa, niinkuin hän, joka puhui Moosekselle, oli määrännyt sen tehtäväksi, sen kaavan mukaan, minkä Mooses oli nähnyt.
\par 45 Ja meidän isämme ottivat sen perintönä vastaan ja toivat sen Joosuan johdolla maahan, minkä he valtasivat pakanoilta, jotka Jumala karkoitti meidän isiemme tieltä. Näin oli Daavidin päiviin saakka.
\par 46 Hän sai armon Jumalan edessä ja anoi, että hän saisi valmistaa majan Jaakobin Jumalalle.
\par 47 Mutta Salomo hänelle huoneen rakensi.
\par 48 Korkein ei kuitenkaan asu käsillä tehdyissä huoneissa; sillä näin sanoo profeetta:
\par 49 'Taivas on minun valtaistuimeni ja maa minun jalkojeni astinlauta; minkäkaltaisen huoneen te minulle rakentaisitte, sanoo Herra, tai mikä paikka olisi minun leposijani?
\par 50 Eikö minun käteni ole tätä kaikkea tehnyt?'
\par 51 Te niskurit ja ympärileikkaamattomat sydämeltä ja korvilta, aina te vastustatte Pyhää Henkeä - niinkuin teidän isänne, niin tekin.
\par 52 Ketä profeetoista eivät teidän isänne vainonneet? He tappoivat ne, jotka ennustivat sen Vanhurskaan tulemista, jonka kavaltajiksi ja murhaajiksi te nyt olette tulleet,
\par 53 te, jotka enkelien toimen kautta saitte lain, mutta ette sitä pitäneet."
\par 54 Mutta kun he tämän kuulivat, viilsi se heidän sydäntänsä, ja he kiristelivät hänelle hampaitansa.
\par 55 Mutta täynnä Pyhää Henkeä hän loi katseensa taivaaseen päin ja näki Jumalan kirkkauden ja Jeesuksen seisovan Jumalan oikealla puolella
\par 56 ja sanoi: "Katso, minä näen taivaat auenneina ja Ihmisen Pojan seisovan Jumalan oikealla puolella".
\par 57 Niin he huusivat suurella äänellä ja tukkivat korvansa ja karkasivat kaikki yhdessä hänen päällensä
\par 58 ja ajoivat hänet ulos kaupungista ja kivittivät. Ja todistajat riisuivat vaippansa Saulus nimisen nuorukaisen jalkojen juureen.
\par 59 Ja niin he kivittivät Stefanuksen, joka rukoili ja sanoi: "Herra Jeesus, ota minun henkeni!"
\par 60 Ja hän laskeutui polvilleen ja huusi suurella äänellä: "Herra, älä lue heille syyksi tätä syntiä!" Ja sen sanottuaan hän nukkui pois.

\chapter{8}

\par 1 Myös Saulus hyväksyi Stefanuksen surmaamisen. Ja sinä päivänä nousi suuri vaino Jerusalemin seurakuntaa vastaan; ja kaikki hajaantuivat ympäri Juudean ja Samarian paikkakuntia, paitsi apostolit.
\par 2 Ja muutamat jumalaapelkääväiset miehet hautasivat Stefanuksen ja pitivät hänelle suuret valittajaiset.
\par 3 Mutta Saulus raateli seurakuntaa, kulki talosta taloon ja raastoi ulos miehiä ja naisia ja panetti heidät vankeuteen.
\par 4 Ne, jotka näin olivat hajaantuneet, vaelsivat paikasta toiseen ja julistivat evankeliumin sanaa.
\par 5 Ja Filippus meni Samarian kaupunkiin ja saarnasi heille Kristusta.
\par 6 Ja kansa otti yksimielisesti vaarin siitä, mitä Filippus puhui, kun he kuulivat hänen sanansa ja näkivät ne tunnusteot, jotka hän teki.
\par 7 Sillä monista, joissa oli saastaisia henkiä, ne lähtivät pois huutaen suurella äänellä; ja moni halvattu ja rampa parani.
\par 8 Ja syntyi suuri ilo siinä kaupungissa.
\par 9 Mutta ennestään oli kaupungissa muuan mies, nimeltä Simon, joka harjoitti noituutta ja hämmästytti Samarian kansaa sanoen olevansa jokin suuri;
\par 10 ja häntä kuuntelivat kaikki, pienet ja suuret, ja sanoivat: "Tämä mies on se Jumalan voima, jota kutsutaan 'suureksi'".
\par 11 Ja he kuuntelivat häntä sentähden, että hän kauan aikaa oli noituuksillaan heitä hämmästyttänyt.
\par 12 Mutta kun he nyt uskoivat Filippusta, joka julisti evankeliumia Jumalan valtakunnasta ja Jeesuksen Kristuksen nimestä, niin he ottivat kasteen, sekä miehet että naiset.
\par 13 Ja Simon itsekin uskoi, ja kasteen saatuansa hän pysytteli Filippuksen seurassa; ja nähdessään ihmeitä ja suuria, voimallisia tekoja hän hämmästyi.
\par 14 Mutta kun apostolit, jotka olivat Jerusalemissa, kuulivat, että Samaria oli ottanut vastaan Jumalan sanan, lähettivät he heidän tykönsä Pietarin ja Johanneksen.
\par 15 Ja tultuaan sinne nämä rukoilivat heidän edestänsä, että he saisivat Pyhän Hengen;
\par 16 sillä hän ei ollut vielä tullut yhteenkään heistä, vaan he olivat ainoastaan kastetut Herran Jeesuksen nimeen.
\par 17 Silloin he panivat kätensä heidän päällensä, ja he saivat Pyhän Hengen.
\par 18 Mutta kun Simon näki, että Henki annettiin sille, jonka päälle apostolit panivat kätensä, toi hän heille rahaa
\par 19 ja sanoi: "Antakaa minullekin se valta, että kenen päälle minä käteni panen, se saa Pyhän Hengen".
\par 20 Mutta Pietari sanoi hänelle: "Menkööt rahasi sinun kanssasi kadotukseen, koska luulet Jumalan lahjan olevan rahalla saatavissa.
\par 21 Ei sinulla ole osaa eikä arpaa tähän sanaan, sillä sinun sydämesi ei ole oikea Jumalan edessä.
\par 22 Tee siis parannus ja käänny tästä pahuudestasi ja rukoile Herraa - jos ehkä vielä sinun sydämesi ajatus sinulle anteeksi annetaan.
\par 23 Sillä minä näen sinun olevan täynnä katkeruuden sappea ja kiinni vääryyden siteissä."
\par 24 Niin Simon vastasi ja sanoi: "Rukoilkaa te minun edestäni Herraa, ettei minulle tapahtuisi mitään siitä, mitä te olette sanoneet".
\par 25 Ja kun he olivat todistaneet ja Herran sanaa puhuneet, palasivat he Jerusalemiin ja julistivat evankeliumia monessa Samarian kylässä.
\par 26 Mutta Filippukselle puhui Herran enkeli sanoen: "Nouse ja mene puolipäivään päin sille tielle, joka vie Jerusalemista alas Gassaan ja on autio".
\par 27 Ja hän nousi ja lähti. Ja katso, siellä kulki etiopialainen mies, Etiopian kuningattaren Kandaken hoviherra, mahtava mies ja koko hänen aarteistonsa hoitaja; hän oli tullut Jerusalemiin rukoilemaan
\par 28 ja oli nyt paluumatkalla ja istui vaunuissaan ja luki profeetta Esaiasta.
\par 29 Niin Henki sanoi Filippukselle: "Käy luo ja pysyttele lähellä noita vaunuja".
\par 30 Niin Filippus juoksi luo ja kuuli hänen lukevan profeetta Esaiasta ja sanoi: "Ymmärrätkö myös, mitä luet?"
\par 31 Niin hän sanoi: "Kuinka minä voisin ymmärtää, ellei kukaan minua opasta?" Ja hän pyysi Filippusta nousemaan ja istumaan viereensä.
\par 32 Ja se kirjoitus, jota hän luki, oli tämä: "Niinkuin lammas hänet viedään teuraaksi; ja niinkuin karitsa on ääneti keritsijänsä edessä, niin ei hänkään suutansa avaa.
\par 33 Hänen alentumisensa kautta hänen tuomionsa otetaan pois. Kuka kertoo hänen syntyperänsä? Sillä hänen elämänsä otetaan pois maan päältä."
\par 34 Ja hoviherra kysyi Filippukselta sanoen: "Minä pyydän sinua: sano, kenestä profeetta puhuu näin, itsestäänkö vai jostakin toisesta?"
\par 35 Niin Filippus avasi suunsa ja lähtien tästä kirjoituksesta julisti hänelle evankeliumia Jeesuksesta.
\par 36 Ja kulkiessaan tietä he tulivat veden ääreen; ja hoviherra sanoi: "Katso, tässä on vettä. Mikä estää kastamasta minua?"
\par 37 []
\par 38 Ja hän käski pysäyttää vaunut, ja he astuivat kumpikin veteen, sekä Filippus että hoviherra, ja Filippus kastoi hänet.
\par 39 Ja kun he olivat astuneet ylös vedestä, tempasi Herran Henki Filippuksen pois, eikä hoviherra häntä enää nähnyt. Ja hän jatkoi matkaansa iloiten.
\par 40 Mutta Filippus tavattiin Asdodissa; ja hän vaelsi ympäri ja julisti evankeliumia kaikissa kaupungeissa, kunnes tuli Kesareaan.

\chapter{9}

\par 1 Mutta Saulus puuskui yhä uhkaa ja murhaa Herran opetuslapsia vastaan ja meni ylimmäisen papin luo
\par 2 ja pyysi häneltä kirjeitä Damaskon synagoogille, että keitä hän vain löytäisi sen tien vaeltajia, miehiä tai naisia, ne hän saisi tuoda sidottuina Jerusalemiin.
\par 3 Ja kun hän oli matkalla, tapahtui hänen lähestyessään Damaskoa, että yhtäkkiä valo taivaasta leimahti hänen ympärillänsä;
\par 4 ja hän kaatui maahan ja kuuli äänen, joka sanoi hänelle: "Saul, Saul, miksi vainoat minua?"
\par 5 Hän sanoi: "Kuka olet, herra?" Hän vastasi: "Minä olen Jeesus, jota sinä vainoat.
\par 6 Mutta nouse ja mene kaupunkiin, niin sinulle sanotaan, mitä sinun pitää tekemän."
\par 7 Ja miehet, jotka matkustivat hänen kanssansa, seisoivat mykistyneinä: he kuulivat kyllä äänen, mutta eivät ketään nähneet.
\par 8 Niin Saulus nousi maasta; mutta kun hän avasi silmänsä, ei hän nähnyt mitään, vaan he taluttivat häntä kädestä ja veivät hänet Damaskoon.
\par 9 Ja hän oli kolme päivää näkemätönnä, ei syönyt eikä juonut.
\par 10 Ja Damaskossa oli eräs opetuslapsi, nimeltä Ananias. Hänelle Herra sanoi näyssä: "Ananias!" Hän vastasi: "Katso, tässä olen, Herra".
\par 11 Niin Herra sanoi hänelle: "Nouse ja mene sille kadulle, jota sanotaan Suoraksi kaduksi, ja kysy Juudaan talosta Saulus nimistä tarsolaista miestä. Sillä katso, hän rukoilee;
\par 12 ja hän on nähnyt näyssä miehen, Ananias nimisen, tulevan sisälle ja panevan kätensä hänen päällensä, että hän saisi näkönsä jälleen."
\par 13 Mutta Ananias vastasi: "Herra, minä olen monelta kuullut siitä miehestä, kuinka paljon pahaa hän on tehnyt sinun pyhillesi Jerusalemissa;
\par 14 ja täälläkin hänellä on ylipapeilta valtuus vangita kaikki, jotka sinun nimeäsi avuksi huutavat".
\par 15 Mutta Herra sanoi hänelle: "Mene; sillä hän on minulle valittu ase, kantamaan minun nimeäni pakanain ja kuningasten ja Israelin lasten eteen.
\par 16 Sillä minä tahdon näyttää hänelle, kuinka paljon hänen pitää kärsimän minun nimeni tähden."
\par 17 Niin Ananias meni ja astui huoneeseen, pani molemmat kätensä hänen päälleen ja sanoi: "Veljeni Saul, Herra lähetti minut - Jeesus, joka ilmestyi sinulle tiellä, jota tulit - että saisit näkösi jälleen ja tulisit täytetyksi Pyhällä Hengellä".
\par 18 Ja heti putosivat hänen silmistään ikäänkuin suomukset, ja hän sai näkönsä ja nousi ja otti kasteen.
\par 19 Ja kun hän nautti ruokaa, niin hän vahvistui. Ja hän oli opetuslasten seurassa Damaskossa jonkun aikaa.
\par 20 Ja kohta hän saarnasi synagoogissa Jeesusta, julistaen, että hän on Jumalan Poika.
\par 21 Ja kaikki, jotka kuulivat, hämmästyivät ja sanoivat: "Eikö tämä ole se, joka Jerusalemissa tuhosi ne, jotka tätä nimeä avuksi huutavat? Ja eikö hän ole tullut tänne viedäksensä ne vangittuina ylipappien käsiin?"
\par 22 Mutta Saulus sai yhä enemmän voimaa ja saattoi Damaskossa asuvat juutalaiset ymmälle näyttäen toteen, että Jeesus on Kristus.
\par 23 Ja pitkän ajan kuluttua juutalaiset pitivät keskenään neuvoa tappaaksensa hänet.
\par 24 Mutta heidän salahankkeensa tuli Sauluksen tietoon. Ja he vartioivat porttejakin yöt päivät, saadakseen hänet tapetuksi.
\par 25 Mutta hänen opetuslapsensa ottivat hänet yöllä, päästivät hänet muurin aukosta ja laskivat alas vasussa.
\par 26 Ja kun hän oli tullut Jerusalemiin, yritti hän liittyä opetuslapsiin; mutta he kaikki pelkäsivät häntä, koska eivät uskoneet, että hän oli opetuslapsi.
\par 27 Mutta Barnabas otti hänet turviinsa ja vei hänet apostolien tykö ja kertoi heille, kuinka Saulus tiellä oli nähnyt Herran, ja että Herra oli puhunut hänelle, ja kuinka hän Damaskossa oli rohkeasti julistanut Jeesuksen nimeä.
\par 28 Ja niin hän kävi heidän keskuudessaan sisälle ja ulos Jerusalemissa ja julisti rohkeasti Herran nimeä.
\par 29 Ja hän puhui ja väitteli hellenistien kanssa; mutta he koettivat tappaa hänet.
\par 30 Kun veljet sen huomasivat, veivät he hänet Kesareaan ja lähettivät hänet sieltä Tarsoon.
\par 31 Niin oli nyt seurakunnalla koko Juudeassa ja Galileassa ja Samariassa rauha; ja se rakentui ja vaelsi Herran pelossa ja lisääntyi Pyhän Hengen virvoittavasta vaikutuksesta.
\par 32 Ja tapahtui, että Pietari, kiertäessään kaikkien luona, tuli myöskin niiden pyhien tykö, jotka asuivat Lyddassa.
\par 33 Siellä hän tapasi Aineas nimisen miehen, joka kahdeksan vuotta oli maannut vuoteessaan ja oli halvattu.
\par 34 Ja Pietari sanoi hänelle: "Aineas, Jeesus Kristus parantaa sinut; nouse ja korjaa vuoteesi". Ja kohta hän nousi.
\par 35 Ja kaikki Lyddan ja Saaronin asukkaat näkivät hänet; ja he kääntyivät Herran tykö.
\par 36 Mutta Joppessa oli naisopetuslapsi, nimeltä Tabita, mikä kreikaksi käännettynä on: Dorkas; hän teki paljon hyviä töitä ja antoi runsaasti almuja.
\par 37 Ja tapahtui niinä päivinä, että hän sairastui ja kuoli; ja he pesivät hänet ja panivat yläsaliin.
\par 38 Ja koska Lydda oli lähellä Joppea, niin opetuslapset, kun kuulivat Pietarin olevan siellä, lähettivät kaksi miestä hänen luoksensa pyytämään: "Tule viipymättä meidän tykömme".
\par 39 Niin Pietari nousi ja meni heidän kanssansa. Ja hänen sinne saavuttuaan he veivät hänet yläsaliin, ja kaikki lesket tulivat hänen luoksensa itkien ja näytellen hänelle ihokkaita ja vaippoja, joita Dorkas oli tehnyt, ollessaan heidän kanssansa.
\par 40 Mutta Pietari toimitti kaikki ulos ja laskeutui polvilleen ja rukoili; ja hän kääntyi ruumiin puoleen ja sanoi: "Tabita, nouse ylös!" Niin tämä avasi silmänsä, ja nähdessään Pietarin hän nousi istumaan.
\par 41 Ja Pietari ojensi hänelle kätensä ja nosti hänet seisomaan ja kutsui sisään pyhät ja lesket ja asetti hänet elävänä heidän eteensä.
\par 42 Ja se tuli tiedoksi koko Joppessa, ja monet uskoivat Herraan.
\par 43 Ja Pietari viipyi Joppessa jonkun aikaa erään nahkuri Simonin luona.

\chapter{10}

\par 1 Ja Kesareassa oli mies, nimeltä Kornelius, sadanpäämies niin kutsutussa italialaisessa sotaväenosastossa.
\par 2 Hän oli hurskas ja Jumalaa pelkääväinen, niinkuin koko hänen perhekuntansakin, ja antoi paljon almuja kansalle ja rukoili alati Jumalaa.
\par 3 Hän näki selvästi näyssä, noin yhdeksännellä hetkellä päivästä, Jumalan enkelin, joka tuli sisään hänen tykönsä ja sanoi hänelle: "Kornelius!"
\par 4 Tämä loi katseensa häneen ja sanoi peljästyneenä: "Mikä on, Herra?" Enkeli sanoi hänelle: "Sinun rukouksesi ja almusi ovat tulleet muistoon Jumalan edessä.
\par 5 Niin lähetä nyt miehiä Joppeen noutamaan eräs Simon, jota myös Pietariksi kutsutaan;
\par 6 hän majailee nahkuri Simonin luona, jonka talo on meren rannalla."
\par 7 Ja kun enkeli, joka Korneliusta puhutteli, oli mennyt pois, kutsui tämä kaksi palvelijaansa ja hurskaan sotamiehen uskollisimpiensa joukosta
\par 8 ja kertoi heille kaikki ja lähetti heidät Joppeen.
\par 9 Ja seuraavana päivänä, kun he olivat matkalla ja lähestyivät kaupunkia, nousi Pietari noin kuudennen hetken vaiheilla katolle rukoilemaan.
\par 10 Ja hänen tuli nälkä, ja hän halusi ruokaa. Mutta sitä valmistettaessa hän joutui hurmoksiin.
\par 11 Ja hän näki taivaan avoinna ja tulevan alas astian, ikäänkuin suuren liinavaatteen, joka neljästä kulmastaan laskettiin maahan.
\par 12 Ja siinä oli kaikkinaisia maan nelijalkaisia ja matelijoita ja taivaan lintuja.
\par 13 Ja tuli ääni, joka sanoi hänelle: "Nouse, Pietari, teurasta ja syö".
\par 14 Mutta Pietari sanoi: "En suinkaan, Herra; sillä en minä ole ikinä syönyt mitään epäpyhää enkä saastaista".
\par 15 Ja taas ääni sanoi hänelle toistamiseen: "Minkä Jumala on puhdistanut, sitä älä sinä sano epäpyhäksi".
\par 16 Tämä tapahtui kolme kertaa; sitten astia otettiin kohta ylös taivaaseen.
\par 17 Ja kun Pietari oli epätietoinen siitä, mitä hänen näkemänsä näky mahtoi merkitä, niin katso, ne miehet, jotka Kornelius oli lähettänyt ja jotka kyselemällä olivat löytäneet Simonin talon, seisoivat portilla
\par 18 ja tiedustelivat kuuluvalla äänellä, majailiko siellä Simon, jota myös Pietariksi kutsuttiin.
\par 19 Kun Pietari yhä mietti tuota näkyä, sanoi Henki hänelle: "Katso, kaksi miestä etsii sinua;
\par 20 niin nouse nyt, astu alas ja mene arvelematta heidän kanssaan, sillä minä olen heidät lähettänyt".
\par 21 Niin Pietari meni alas miesten tykö ja sanoi: "Katso, minä olen se, jota te etsitte; mitä varten te olette tulleet?"
\par 22 He sanoivat: "Sadanpäämies Kornelius, hurskas ja Jumalaa pelkääväinen mies, josta koko Juudan kansa todistaa hyvää, on pyhältä enkeliltä ilmestyksessä saanut käskyn haettaa sinut kotiinsa ja kuulla, mitä sinulla on sanottavaa".
\par 23 Niin hän kutsui heidät sisään ja piti heitä vierainansa. Seuraavana päivänä Pietari nousi ja lähti heidän kanssaan, ja muutamat veljet Joppesta seurasivat hänen mukanaan.
\par 24 Ja sen jälkeisenä päivänä he saapuivat Kesareaan; ja Kornelius odotti heitä ja oli kutsunut koolle sukulaisensa ja lähimmät ystävänsä.
\par 25 Ja kun Pietari oli astumassa sisään, meni Kornelius häntä vastaan, lankesi hänen jalkojensa juureen ja kumartui maahan.
\par 26 Mutta Pietari nosti hänet ylös sanoen: "Nouse; minäkin olen ihminen".
\par 27 Ja puhellen hänen kanssaan hän meni sisään ja tapasi monta koolla.
\par 28 Ja hän sanoi heille: "Te tiedätte, että on luvatonta juutalaisen miehen seurustella vierasheimoisen kanssa tai mennä hänen tykönsä; mutta minulle Jumala on osoittanut, etten saa sanoa ketään ihmistä epäpyhäksi enkä saastaiseksi.
\par 29 Sentähden minä vastaansanomatta tulinkin, kun minua noudettiin. Ja nyt minä kysyn: mitä varten te olette minut noutaneet?"
\par 30 Ja Kornelius sanoi: "Neljä päivää sitten, juuri tähän aikaan päivästä, minä kotonani rukoilin tällä yhdeksännellä hetkellä, ja katso, edessäni seisoi mies loistavissa vaatteissa
\par 31 ja sanoi: 'Kornelius, sinun rukouksesi on kuultu, ja sinun almusi ovat tulleet muistoon Jumalan edessä.
\par 32 Niin lähetä nyt Joppeen ja kutsu tykösi Simon, jota myös Pietariksi kutsutaan; hän majailee nahkuri Simonin talossa meren rannalla.'
\par 33 Sentähden minä lähetin heti sinulle sanan, ja sinä teit hyvin, kun tulit. Nyt olemme siis tässä kaikki Jumalan edessä, kuullaksemme kaiken, mitä Herra on käskenyt sinun puhua."
\par 34 Niin Pietari avasi suunsa ja sanoi: "Nyt minä totisesti käsitän, ettei Jumala katso henkilöön,
\par 35 vaan että jokaisessa kansassa se, joka häntä pelkää ja tekee vanhurskautta, on hänelle otollinen.
\par 36 Sen sanan, jonka hän lähetti Israelin lapsille, julistaen evankeliumia rauhasta Jeesuksessa Kristuksessa, joka on kaikkien Herra,
\par 37 sen sanan, joka lähtien Galileasta on levinnyt koko Juudeaan, sen kasteen jälkeen, jota Johannes saarnasi, sen te tiedätte;
\par 38 te tiedätte, kuinka Jumala Pyhällä Hengellä ja voimalla oli voidellut Jeesuksen Nasaretilaisen, hänet, joka vaelsi ympäri ja teki hyvää ja paransi kaikki perkeleen valtaan joutuneet; sillä Jumala oli hänen kanssansa.
\par 39 Ja me olemme kaiken sen todistajat, mitä hän teki juutalaisten maassa ja Jerusalemissa; ja hänet he ripustivat puuhun ja tappoivat.
\par 40 Hänet Jumala herätti kolmantena päivänä ja antoi hänen ilmestyä,
\par 41 ei kaikelle kansalle, vaan Jumalan ennen valitsemille todistajille, meille, jotka söimme ja joimme hänen kanssaan sen jälkeen, kuin hän oli kuolleista noussut.
\par 42 Ja hän käski meidän saarnata kansalle ja todistaa, että hän on se, jonka Jumala on asettanut elävien ja kuolleitten tuomariksi.
\par 43 Hänestä kaikki profeetat todistavat, että jokainen, joka uskoo häneen, saa synnit anteeksi hänen nimensä kautta."
\par 44 Kun Pietari vielä näitä puhui, tuli Pyhä Henki kaikkien päälle, jotka puheen kuulivat.
\par 45 Ja kaikki ne uskovaiset, jotka olivat ympärileikatut ja olivat tulleet Pietarin mukana, hämmästyivät sitä, että Pyhän Hengen lahja vuodatettiin pakanoihinkin,
\par 46 sillä he kuulivat heidän puhuvan kielillä ja ylistävän Jumalaa. Silloin Pietari vastasi:
\par 47 "Ei kaiketi kukaan voi kieltää kastamasta vedellä näitä, jotka ovat saaneet Pyhän Hengen niinkuin mekin?"
\par 48 Ja hän käski kastaa heidät Jeesuksen Kristuksen nimeen. Silloin he pyysivät häntä viipymään siellä muutamia päiviä.

\chapter{11}

\par 1 Ja apostolit ja veljet ympäri Juudeaa kuulivat, että pakanatkin olivat ottaneet vastaan Jumalan sanan.
\par 2 Ja kun Pietari tuli Jerusalemiin, ahdistelivat ympärileikatut häntä
\par 3 sanoen: "Sinä olet käynyt ympärileikkaamattomien miesten luona ja syönyt heidän kanssansa".
\par 4 Niin Pietari selitti heille alusta alkaen asiat järjestänsä ja sanoi:
\par 5 "Minä olin Joppen kaupungissa ja rukoilin; silloin minä näin hurmoksissa näyn: tuli alas astia, ikäänkuin suuri liinavaate, joka neljästä kulmastaan laskettiin taivaasta, ja se tuli aivan minun eteeni.
\par 6 Ja kun minä katsoin sitä tarkasti, näin minä siinä maan nelijalkaisia ja petoja ja matelijoita ja taivaan lintuja.
\par 7 Ja minä kuulin myös äänen, joka sanoi minulle: 'Nouse, Pietari, teurasta ja syö'.
\par 8 Mutta minä sanoin: 'En suinkaan, Herra; sillä ei mitään epäpyhää eikä saastaista ole koskaan minun suuhuni tullut'.
\par 9 Niin vastasi ääni taivaasta toistamiseen ja sanoi: 'Minkä Jumala on puhdistanut, sitä älä sinä sano epäpyhäksi'.
\par 10 Ja tämä tapahtui kolme kertaa; sitten vedettiin kaikki taas ylös taivaaseen.
\par 11 Ja katso, samassa seisoi sen talon edessä, jossa me olimme, kolme miestä, jotka Kesareasta oli lähetetty minun luokseni.
\par 12 Ja Henki käski minun mennä arvelematta heidän kanssansa; ja myös nämä kuusi veljeä lähtivät minun kanssani. Me menimme sen miehen taloon,
\par 13 ja hän kertoi meille, kuinka hän oli nähnyt enkelin seisovan hänen huoneessaan ja sanovan: 'Lähetä Joppeen noutamaan Simon, jota myös Pietariksi kutsutaan;
\par 14 hän on puhuva sinulle sanoja, joiden kautta sinä pelastut, ja koko sinun perhekuntasi'.
\par 15 Ja kun minä rupesin puhumaan, tuli Pyhä Henki heidän päällensä, niinkuin alussa meidänkin päällemme.
\par 16 Silloin minä muistin Herran sanan, jonka hän sanoi: 'Johannes kastoi vedellä, mutta teidät kastetaan Pyhällä Hengellä'.
\par 17 Koska siis Jumala antoi yhtäläisen lahjan heille kuin meillekin, kun olimme uskoneet Herraan Jeesukseen Kristukseen, niin mikä olin minä voidakseni estää Jumalaa?"
\par 18 Tämän kuultuansa he rauhoittuivat ja ylistivät Jumalaa sanoen: "Niin on siis Jumala pakanoillekin antanut parannuksen elämäksi".
\par 19 Ne, jotka olivat hajaantuneet sen ahdingon vuoksi, joka oli syntynyt Stefanuksen tähden, vaelsivat ympäri hamaan Foinikiaan ja Kyproon ja Antiokiaan saakka, mutta eivät puhuneet sanaa muille kuin ainoastaan juutalaisille.
\par 20 Heidän joukossaan oli kuitenkin muutamia kyprolaisia ja kyreneläisiä miehiä, jotka, tultuaan Antiokiaan, puhuivat kreikkalaisillekin ja julistivat evankeliumia Herrasta Jeesuksesta.
\par 21 Ja Herran käsi oli heidän kanssansa, ja suuri oli se joukko, joka uskoi ja kääntyi Herran puoleen.
\par 22 Ja sanoma heistä tuli Jerusalemin seurakunnan korviin, ja he lähettivät Barnabaan Antiokiaan.
\par 23 Ja kun hän saapui sinne ja näki Jumalan armon, niin hän iloitsi ja kehoitti kaikkia vakaalla sydämellä pysymään Herrassa.
\par 24 Sillä hän oli hyvä mies ja täynnä Pyhää Henkeä ja uskoa. Ja Herralle lisääntyi paljon kansaa.
\par 25 Niin hän lähti Tarsoon etsimään Saulusta, ja kun hän oli hänet löytänyt, toi hän hänet Antiokiaan.
\par 26 Ja he vaikuttivat yhdessä kokonaisen vuoden seurakunnassa, ja niitä oli paljon, jotka saivat heiltä opetusta; ja Antiokiassa ruvettiin opetuslapsia ensiksi nimittämään kristityiksi.
\par 27 Siihen aikaan tuli profeettoja Jerusalemista Antiokiaan.
\par 28 Ja eräs heistä, nimeltä Agabus, nousi ja antoi Hengen vaikutuksesta tiedoksi, että oli tuleva suuri nälkä kaikkeen maailmaan; ja se tulikin Klaudiuksen aikana.
\par 29 Niin opetuslapset päättivät kukin varojensa mukaan lähettää avustusta Juudeassa asuville veljille.
\par 30 Ja niin he tekivätkin ja lähettivät sen vanhimmille Barnabaan ja Sauluksen kätten kautta.

\chapter{12}

\par 1 Siihen aikaan kuningas Herodes otatti muutamia seurakunnan jäseniä kiinni kiduttaaksensa heitä.
\par 2 Ja hän mestautti miekalla Jaakobin, Johanneksen veljen.
\par 3 Ja kun hän näki sen olevan juutalaisille mieleen, niin hän sen lisäksi vangitutti Pietarinkin. Silloin olivat happamattoman leivän päivät.
\par 4 Ja otettuaan hänet kiinni hän pani hänet vankeuteen ja jätti neljän nelimiehisen sotilasvartioston vartioitavaksi, aikoen pääsiäisen jälkeen asettaa hänet kansan eteen.
\par 5 Niin pidettiin siis Pietaria vankeudessa; mutta seurakunta rukoili lakkaamatta Jumalaa hänen edestänsä.
\par 6 Ja yöllä sitä päivää vasten, jona Herodeksella oli aikomus viedä hänet oikeuden eteen, Pietari nukkui kahden sotamiehen välissä, sidottuna kaksilla kahleilla; ja vartijat vartioitsivat oven edessä vankilaa.
\par 7 Ja katso, hänen edessään seisoi Herran enkeli, ja huoneessa loisti valo, ja enkeli sysäsi Pietaria kylkeen ja herätti hänet sanoen: "Nouse nopeasti!" Ja kahleet putosivat hänen käsistään.
\par 8 Ja enkeli sanoi hänelle: "Vyötä itsesi ja sido paula-anturat jalkaasi". Ja hän teki niin. Vielä enkeli sanoi hänelle: "Heitä vaippa yllesi ja seuraa minua".
\par 9 Ja Pietari lähti ja seurasi häntä, mutta ei tiennyt, että se, mikä enkelin vaikutuksesta tapahtui, oli totta, vaan luuli näkevänsä näyn.
\par 10 Ja he kulkivat läpi ensimmäisen vartion ja toisen ja tulivat rautaportille, joka vei kaupunkiin. Se aukeni heille itsestään, ja he menivät ulos ja kulkivat eteenpäin muutamaa katua; ja yhtäkkiä enkeli erkani hänestä.
\par 11 Kun Pietari tointui, sanoi hän: "Nyt minä totisesti tiedän, että Herra on lähettänyt enkelinsä ja pelastanut minut Herodeksen käsistä ja kaikesta, mitä Juudan kansa odotti".
\par 12 Ja päästyään siitä selville hän kulki kohti Marian, Johanneksen äidin, taloa, sen Johanneksen, jota myös Markukseksi kutsuttiin. Siellä oli monta koolla rukoilemassa.
\par 13 Ja kun Pietari kolkutti eteisen ovea, tuli siihen palvelijatar, nimeltä Rode, kuulostamaan;
\par 14 ja tunnettuaan Pietarin äänen hän ilossansa ei avannut eteistä vaan juoksi sisään ja kertoi Pietarin seisovan portin takana.
\par 15 He sanoivat hänelle: "Sinä hourit". Mutta hän vakuutti puheensa todeksi. Niin he sanoivat: "Se on hänen enkelinsä".
\par 16 Mutta Pietari kolkutti yhä; ja kun he avasivat, näkivät he hänet ja hämmästyivät.
\par 17 Niin hän viittasi kädellään heitä vaikenemaan ja kertoi heille, kuinka Herra oli vienyt hänet ulos vankeudesta, ja sanoi: "Ilmoittakaa tämä Jaakobille ja veljille". Ja hän lähti pois ja meni toiseen paikkaan.
\par 18 Mutta kun päivä koitti, tuli sotamiehille kova hätä siitä, mihin Pietari oli joutunut.
\par 19 Ja kun Herodes oli haettanut häntä eikä löytänyt, tutki hän vartijoita ja käski viedä heidät rangaistaviksi. Sitten hän meni Juudeasta Kesareaan ja oleskeli siellä.
\par 20 Ja Herodes oli vihoissansa tyyrolaisille ja siidonilaisille. Mutta nämä tulivat yksissä neuvoin hänen luoksensa, ja suostutettuaan puolelleen Blastuksen, kuninkaan kamaripalvelijan, he anoivat rauhaa; sillä heidän maakuntansa sai elatuksensa kuninkaan maasta.
\par 21 Niin Herodes määrättynä päivänä pukeutui kuninkaalliseen pukuun, istui istuimelleen ja piti heille puheen;
\par 22 siihen kansa huusi: "Jumalan ääni, eikä ihmisen!"
\par 23 Mutta heti löi häntä Herran enkeli, sentähden ettei hän antanut kunniaa Jumalalle; ja madot söivät hänet, ja hän heitti henkensä.
\par 24 Mutta Jumalan sana menestyi ja levisi.
\par 25 Ja Barnabas ja Saulus palasivat Jerusalemista toimitettuansa avustustehtävän ja toivat sieltä mukanaan Johanneksen, jota myös Markukseksi kutsuttiin.

\chapter{13}

\par 1 Ja Antiokian seurakunnassa oli profeettoja ja opettajia: Barnabas ja Simeon, jota kutsuttiin Nigeriksi, ja Lukius, kyreneläinen, ja Manaen, neljännysruhtinas Herodeksen kasvinkumppani, ja Saulus.
\par 2 Ja heidän toimittaessaan palvelusta Herralle ja paastotessaan Pyhä Henki sanoi: "Erottakaa minulle Barnabas ja Saulus siihen työhön, johon minä olen heidät kutsunut".
\par 3 Silloin he paastosivat ja rukoilivat ja panivat kätensä heidän päällensä ja laskivat heidät menemään.
\par 4 Niin he Pyhän Hengen lähettäminä menivät Seleukiaan ja purjehtivat sieltä Kyproon.
\par 5 Ja tultuaan Salamiiseen he julistivat Jumalan sanaa juutalaisten synagoogissa, ja heillä oli mukanaan myös Johannes, palvelijana.
\par 6 Ja kun he olivat vaeltaneet kautta koko saaren Pafoon asti, tapasivat he erään juutalaisen miehen, noidan ja väärän profeetan, jonka nimi oli Barjeesus.
\par 7 Hän oleskeli käskynhaltijan, Sergius Pauluksen, luona, joka oli ymmärtäväinen mies. Tämä kutsui luoksensa Barnabaan ja Sauluksen ja halusi kuulla Jumalan sanaa.
\par 8 Mutta Elymas, noita - sillä niin tulkitaan hänen nimensä - vastusti heitä, koettaen kääntää käskynhaltijaa pois uskosta.
\par 9 Niin Saulus, myös Paavaliksi kutsuttu, täynnä Pyhää Henkeä loi katseensa häneen
\par 10 ja sanoi: "Voi sinua, joka olet kaikkea vilppiä ja kavaluutta täynnä, sinä perkeleen sikiö, kaiken vanhurskauden vihollinen, etkö lakkaa vääristelemästä Herran suoria teitä?
\par 11 Ja nyt, katso, Herran käsi on sinun päälläsi, ja sinä tulet sokeaksi etkä aurinkoa näe säädettyyn aikaan asti." Ja heti lankesi hänen päällensä synkeys ja pimeys, ja hän kävi ympäri ja etsi taluttajaa.
\par 12 Kun nyt käskynhaltija näki, mitä oli tapahtunut, niin hän uskoi, ihmetellen Herran oppia.
\par 13 Ja kun Paavali seuralaisineen oli purjehtinut Pafosta, tulivat he Pamfylian Pergeen; siellä Johannes erosi heistä ja palasi Jerusalemiin.
\par 14 Mutta he vaelsivat Pergestä eteenpäin ja saapuivat Pisidian Antiokiaan; ja he menivät synagoogaan sapatinpäivänä ja istuutuivat.
\par 15 Ja sittenkuin lakia ja profeettoja oli luettu, lähettivät synagoogan esimiehet sanomaan heille: "Miehet, veljet, jos teillä on jokin kehoituksen sana kansalle, niin puhukaa".
\par 16 Niin Paavali nousi ja viittasi kädellään ja sanoi: "Te Israelin miehet ja te, jotka Jumalaa pelkäätte, kuulkaa!
\par 17 Tämän Israelin kansan Jumala valitsi meidän isämme ja korotti tämän kansan, heidän muukalaisina ollessaan Egyptin maassa, ja vei heidät sieltä ulos kohotetulla käsivarrella,
\par 18 ja hän kärsi heidän tapojansa noin neljäkymmentä vuotta erämaassa
\par 19 ja hävitti seitsemän kansaa Kanaanin maasta ja jakoi niiden maan heille perinnöksi.
\par 20 Näin kului noin neljäsataa viisikymmentä vuotta. Sen jälkeen hän antoi heille tuomareita profeetta Samueliin saakka.
\par 21 Ja sitten he anoivat kuningasta, ja Jumala antoi heille Saulin, Kiisin pojan, miehen Benjaminin sukukunnasta, neljäksikymmeneksi vuodeksi.
\par 22 Mutta hän pani hänet viralta ja herätti heille kuninkaaksi Daavidin, josta hän myös todisti ja sanoi: 'Minä olen löytänyt Daavidin, Iisain pojan, sydämeni mukaisen miehen, joka on tekevä kaikessa minun tahtoni'.
\par 23 Tämän jälkeläisistä on Jumala lupauksensa mukaan antanut tulla Jeesuksen Israelille Vapahtajaksi,
\par 24 sittenkuin Johannes ennen hänen tuloansa oli saarnannut parannuksen kastetta kaikelle Israelin kansalle.
\par 25 Mutta kun Johannes oli juoksunsa päättävä, sanoi hän: 'En minä ole se, joksi minua luulette; mutta katso, minun jälkeeni tulee se, jonka kenkiä minä en ole arvollinen jaloista riisumaan'.
\par 26 Miehet, veljet, te Aabrahamin suvun lapset, ja te, jotka Jumalaa pelkäätte, meille on tämän pelastuksen sana lähetetty.
\par 27 Sillä koska Jerusalemin asukkaat ja heidän hallitusmiehensä eivät Jeesusta tunteneet, niin he tuomitessaan hänet myös toteuttivat profeettain sanat, joita kunakin sapattina luetaan;
\par 28 ja vaikka he eivät löytäneet mitään, mistä hän olisi kuoleman ansainnut, anoivat he Pilatukselta, että hänet surmattaisiin.
\par 29 Ja kun he olivat täyttäneet kaiken, mikä hänestä on kirjoitettu, ottivat he hänet alas puusta ja panivat hautaan.
\par 30 Mutta Jumala herätti hänet kuolleista.
\par 31 Ja hän ilmestyi useina päivinä niille, jotka olivat tulleet hänen kanssansa Galileasta Jerusalemiin ja jotka nyt ovat hänen todistajansa kansan edessä.
\par 32 Ja me julistamme teille sen hyvän sanoman, että Jumala on isille annetun lupauksen täyttänyt meidän lapsillemme, herättäen Jeesuksen,
\par 33 niinkuin myös toisessa psalmissa on kirjoitettu: 'Sinä olet minun Poikani, tänä päivänä minä olen sinut synnyttänyt'.
\par 34 Ja että hän herätti hänet kuolleista, niin ettei hän enää palaja katoavaisuuteen, siitä hän on sanonut näin: 'Minä annan teille pyhät ja lujat Daavidin armot'.
\par 35 Sentähden hän myös toisessa paikassa sanoo: 'Sinä et salli Pyhäsi nähdä katoavaisuutta'.
\par 36 Sillä kun Daavid oli aikanansa Jumalan tahtoa palvellut, nukkui hän ja tuli otetuksi isiensä tykö ja näki katoavaisuuden.
\par 37 Mutta hän, jonka Jumala herätti, ei nähnyt katoavaisuutta.
\par 38 Olkoon siis teille tiettävä, miehet ja veljet, että hänen kauttansa julistetaan teille syntien anteeksiantamus
\par 39 ja että jokainen, joka uskoo, tulee hänessä vanhurskaaksi, vapaaksi kaikesta, mistä te ette voineet Mooseksen lain kautta vanhurskaiksi tulla.
\par 40 Kavahtakaa siis, ettei teitä kohtaa se, mikä on puhuttu profeetoissa:
\par 41 'Katsokaa, te halveksijat, ja ihmetelkää ja hukkukaa; sillä minä teen teidän päivinänne teon, teon, jota ette uskoisi, jos joku sen kertoisi teille'."
\par 42 Kun he lähtivät ulos, pyydettiin heitä puhumaan näitä asioita tulevanakin sapattina.
\par 43 Kun synagoogasta hajaannuttiin, seurasivat monet juutalaiset ja jumalaapelkääväiset käännynnäiset Paavalia ja Barnabasta, jotka puhuivat heille ja kehoittivat heitä pysymään Jumalan armossa.
\par 44 Seuraavana sapattina kokoontui lähes koko kaupunki kuulemaan Jumalan sanaa.
\par 45 Mutta nähdessään kansanjoukot juutalaiset tulivat kiihkoa täyteen ja väittelivät Paavalin puheita vastaan ja herjasivat.
\par 46 Silloin Paavali ja Barnabas puhuivat rohkeasti ja sanoivat: "Teille oli Jumalan sana ensiksi puhuttava; mutta koska te työnnätte sen luotanne ettekä katso itseänne mahdollisiksi iankaikkiseen elämään, niin katso, me käännymme pakanain puoleen.
\par 47 Sillä näin on Herra meitä käskenyt: 'Minä olen pannut sinut pakanain valkeudeksi, että sinä olisit pelastukseksi maan ääriin asti'."
\par 48 Sen kuullessansa pakanat iloitsivat ja ylistivät Herran sanaa ja uskoivat, kaikki, jotka olivat säädetyt iankaikkiseen elämään.
\par 49 Ja Herran sanaa levitettiin kaikkeen siihen maakuntaan.
\par 50 Mutta juutalaiset yllyttivät jumalaapelkääväisiä ylhäisiä naisia ja kaupungin ensimmäisiä miehiä ja nostivat vainon Paavalia ja Barnabasta vastaan, ja ne ajoivat heidät pois alueiltansa.
\par 51 Niin he pudistivat tomun jaloistansa heitä vastaan ja menivät Ikonioniin.
\par 52 Ja opetuslapset tulivat täytetyiksi ilolla ja Pyhällä Hengellä.

\chapter{14}

\par 1 Ikonionissa he samoin menivät juutalaisten synagoogaan ja puhuivat niin, että suuri joukko sekä juutalaisia että kreikkalaisia uskoi.
\par 2 Mutta ne juutalaiset, jotka eivät uskoneet, yllyttivät ja kiihdyttivät pakanain mieltä veljiä vastaan.
\par 3 Niin he oleskelivat siellä kauan aikaa ja puhuivat rohkeasti, luottaen Herraan, joka armonsa sanan todistukseksi antoi tapahtua tunnustekoja ja ihmeitä heidän kättensä kautta.
\par 4 Ja kaupungin väestö jakaantui: toiset olivat juutalaisten puolella, toiset taas apostolien puolella.
\par 5 Mutta kun pakanat ja juutalaiset ynnä heidän hallitusmiehensä mielivät ryhtyä pahoinpitelemään ja kivittämään heitä,
\par 6 ja he sen huomasivat, pakenivat he Lykaonian kaupunkeihin, Lystraan ja Derbeen, ja niiden ympäristöön.
\par 7 Ja siellä he julistivat evankeliumia.
\par 8 Ja Lystrassa oli mies, joka istui siellä, hervoton jaloistaan ja rampa äitinsä kohdusta saakka, eikä ollut koskaan kävellyt.
\par 9 Hän kuunteli Paavalin puhetta. Ja kun Paavali loi katseensa häneen ja näki hänellä olevan uskon, että hän voi tulla terveeksi,
\par 10 sanoi hän suurella äänellä: "Nouse pystyyn jaloillesi". Ja hän kavahti ylös ja käveli.
\par 11 Kun kansa näki, mitä Paavali oli tehnyt, korottivat he äänensä ja sanoivat lykaoniankielellä: "Jumalat ovat ihmishahmossa astuneet alas meidän luoksemme".
\par 12 Ja he sanoivat Barnabasta Zeukseksi ja Paavalia Hermeeksi, koska hän oli se, joka puhui.
\par 13 Ja kaupungin edustalla olevan Zeuksen temppelin pappi toi härkiä ja seppeleitä porttien eteen ja tahtoi väkijoukon kanssa uhrata.
\par 14 Mutta kun apostolit Barnabas ja Paavali sen kuulivat, repäisivät he vaatteensa ja juoksivat ulos kansanjoukkoon, huusivat
\par 15 ja sanoivat: "Miehet, miksi te näin teette? Mekin olemme ihmisiä, yhtä vajavaisia kuin te, ja julistamme teille evankeliumia, että te kääntyisitte noista turhista jumalista elävän Jumalan puoleen, joka on tehnyt taivaan ja maan ja meren ja kaikki, mitä niissä on.
\par 16 Menneitten sukupolvien aikoina hän on sallinut kaikkien pakanain vaeltaa omia teitänsä;
\par 17 ja kuitenkaan hän ei ole ollut antamatta todistusta itsestään, sillä hän on tehnyt teille hyvää, antaen teille taivaasta sateita ja hedelmällisiä aikoja ja raviten teidän sydämenne ruualla ja ilolla."
\par 18 Näin puhuen he vaivoin saivat kansan hillityksi uhraamasta heille.
\par 19 Mutta sinne tuli Antiokiasta ja Ikonionista juutalaisia, ja he suostuttivat kansan puolellensa ja kivittivät Paavalia ja raastoivat hänet kaupungin ulkopuolelle, luullen hänet kuolleeksi.
\par 20 Mutta kun opetuslapset olivat kokoontuneet hänen ympärilleen, nousi hän ja meni kaupunkiin. Ja seuraavana päivänä hän lähti Barnabaan kanssa Derbeen.
\par 21 Ja julistettuaan evankeliumia siinä kaupungissa ja tehtyään monta opetuslapsiksi he palasivat Lystraan ja Ikonioniin ja Antiokiaan
\par 22 ja vahvistivat opetuslasten sieluja ja kehoittivat heitä pysymään uskossa ja sanoivat: "Monen ahdistuksen kautta meidän pitää menemän sisälle Jumalan valtakuntaan".
\par 23 Ja kun he olivat valinneet heille vanhimmat jokaisessa seurakunnassa, niin he rukoillen ja paastoten jättivät heidät Herran haltuun, johon he nyt uskoivat.
\par 24 Ja he kulkivat läpi Pisidian ja tulivat Pamfyliaan;
\par 25 ja julistettuaan sanaa Pergessä he menivät Attaliaan.
\par 26 Ja sieltä he purjehtivat Antiokiaan, josta he olivat lähteneet, annettuina Jumalan armon haltuun sitä työtä varten, jonka he nyt olivat suorittaneet.
\par 27 Ja sinne saavuttuaan he kutsuivat seurakunnan koolle ja kertoivat, kuinka Jumala oli ollut heidän kanssansa ja tehnyt suuria ja kuinka hän oli avannut pakanoille uskon oven.
\par 28 Ja he viipyivät pitkän aikaa opetuslasten tykönä.

\chapter{15}

\par 1 Ja Juudeasta tuli sinne muutamia, jotka opettivat veljiä: "Ellette ympärileikkauta itseänne, niinkuin Mooses on säätänyt, ette voi pelastua".
\par 2 Kun siitä syntyi riita ja kun Paavali ja Barnabas kiivaasti väittelivät heitä vastaan, niin päätettiin, että Paavalin ja Barnabaan ja muutamien muiden heistä tuli mennä tämän riitakysymyksen tähden apostolien ja vanhinten tykö Jerusalemiin.
\par 3 Niin seurakunta varusti heidät matkalle, ja he kulkivat Foinikian ja Samarian kautta ja kertoivat pakanain kääntymyksestä ja ilahuttivat sillä suuresti kaikkia veljiä.
\par 4 Ja kun he saapuivat Jerusalemiin, niin seurakunta ja apostolit ja vanhimmat ottivat heidät vastaan; ja he kertoivat, kuinka Jumala oli ollut heidän kanssansa ja tehnyt suuria.
\par 5 Mutta fariseusten lahkosta nousivat muutamat, jotka olivat tulleet uskoon, ja sanoivat: "Heidät on ympärileikattava ja heitä on käskettävä noudattamaan Mooseksen lakia".
\par 6 Niin apostolit ja vanhimmat kokoontuivat tutkimaan tätä asiaa.
\par 7 Ja kun oli paljon väitelty, nousi Pietari ja sanoi heille: "Miehet, veljet, te tiedätte, että Jumala jo kauan aikaa sitten teki teidän keskuudessanne sen valinnan, että pakanat minun suustani saisivat kuulla evankeliumin sanan ja tulisivat uskoon.
\par 8 Ja Jumala, sydänten tuntija, todisti heidän puolestansa, antaen heille Pyhän Hengen samoinkuin meillekin.
\par 9 eikä tehnyt mitään erotusta meidän ja heidän kesken, sillä hän puhdisti heidän sydämensä uskolla.
\par 10 Miksi te siis nyt kiusaatte Jumalaa ja tahdotte panna opetuslasten niskaan ikeen, jota eivät meidän isämme emmekä mekään ole jaksaneet kantaa?
\par 11 Mutta me uskomme Herran Jeesuksen armon kautta pelastuvamme, samalla tapaa kuin hekin."
\par 12 Niin koko joukko vaikeni, ja he kuuntelivat Barnabasta ja Paavalia, jotka kertoivat, kuinka suuria tunnustekoja ja ihmeitä Jumala oli tehnyt pakanain keskuudessa heidän kauttansa.
\par 13 Kun he olivat lakanneet puhumasta, lausui Jaakob: "Miehet, veljet, kuulkaa minua!
\par 14 Simeon on kertonut, kuinka Jumala ensi kerran katsoi pakanain puoleen ottaakseen heistä kansan omalle nimellensä.
\par 15 Ja tämän kanssa pitävät yhtä profeettain sanat, sillä näin on kirjoitettu:
\par 16 'Sen jälkeen minä palajan ja pystytän jälleen Daavidin sortuneen majan; minä korjaan sen repeämät ja nostan sen jälleen pystyyn,
\par 17 että jäljelle jääneet ihmiset etsisivät Herraa, ynnä kaikki pakanat, jotka ovat minun nimiini otetut, sanoo Herra, joka tämän tekee,
\par 18 mikä on ollut tunnettua hamasta ikiajoista'.
\par 19 Sentähden minä olen sitä mieltä, ettei tule rasittaa niitä, jotka pakanuudesta kääntyvät Jumalan puoleen,
\par 20 vaan heille kirjoitettakoon, että heidän pitää karttaman epäjumalien saastuttamaa ja haureutta ja lihaa, josta ei veri ole laskettu, sekä verta.
\par 21 Sillä Mooseksella on ammoisista ajoista asti joka kaupungissa julistajansa; luetaanhan häntä synagoogissa jokaisena sapattina."
\par 22 Silloin apostolit ja vanhimmat ja koko seurakunta näkivät hyväksi valita keskuudestaan muutamia miehiä ja lähettää heidät Antiokiaan Paavalin ja Barnabaan mukana, nimittäin Juudaan, jota kutsuttiin Barsabbaaksi, ja Silaan, jotka olivat johtavia miehiä veljien joukossa;
\par 23 ja kirjoittivat heidän mukaansa näin kuuluvan kirjeen: "Me apostolit ja vanhimmat, teidän veljenne, lähetämme teille, pakanuudesta kääntyneille veljille Antiokiassa ja Syyriassa ja Kilikiassa, tervehdyksen.
\par 24 Koska olemme kuulleet, että muutamat meistä lähteneet, joille emme ole mitään käskyä antaneet, ovat puheillaan tehneet teidät levottomiksi ja saattaneet teidän sielunne hämmennyksiin,
\par 25 niin me olemme yksimielisesti nähneet hyväksi valita miehiä ja lähettää heidät teidän tykönne rakkaiden veljiemme Barnabaan ja Paavalin kanssa,
\par 26 jotka ovat panneet henkensä alttiiksi meidän Herramme Jeesuksen Kristuksen nimen tähden.
\par 27 Me lähetämme siis Juudaan ja Silaan, jotka myös suusanalla ilmoittavat teille saman.
\par 28 Sillä Pyhä Henki ja me olemme nähneet hyväksi, ettei teidän päällenne ole pantava enempää kuormaa kuin nämä välttämättömät:
\par 29 että kartatte epäjumalille uhrattua ja verta ja lihaa, josta ei veri ole laskettu, ja haureutta. Jos te näitä vältätte, niin teidän käy hyvin. Jääkää hyvästi!"
\par 30 Niin heidät lähetettiin matkalle, ja he tulivat Antiokiaan; siellä he kutsuivat koolle seurakunnan ja antoivat heille kirjeen.
\par 31 Ja kun he olivat sen lukeneet, iloitsivat he tästä lohdutuksesta.
\par 32 Ja Juudas ja Silas, jotka itsekin olivat profeettoja, kehoittivat veljiä monin sanoin ja vahvistivat heitä.
\par 33 Ja heidän viivyttyään siellä jonkin aikaa veljet laskivat heidät rauhassa palaamaan niiden tykö, jotka olivat heidät lähettäneet.
\par 34 []
\par 35 Ja Paavali ja Barnabas viipyivät Antiokiassa opettaen ja julistaen useiden muidenkin kanssa Herran sanaa.
\par 36 Mutta muutamien päivien kuluttua Paavali sanoi Barnabaalle: "Lähtekäämme takaisin kaikkiin niihin kaupunkeihin, joissa olemme julistaneet Herran sanaa, katsomaan veljiä, miten heidän on".
\par 37 Niin Barnabas tahtoi ottaa mukaan Johanneksenkin, jota kutsuttiin Markukseksi.
\par 38 Mutta Paavali katsoi oikeaksi olla ottamatta häntä mukaan, koska hän oli luopunut heistä Pamfyliassa eikä ollut heidän kanssaan lähtenyt työhön.
\par 39 Ja he kiivastuivat niin, että erkanivat toisistaan, ja Barnabas otti mukaansa Markuksen ja purjehti Kyproon.
\par 40 Mutta Paavali valitsi Silaan, ja veljet jättivät hänet Herran armon haltuun, ja hän lähti matkalle.
\par 41 Ja hän vaelsi läpi Syyrian ja Kilikian vahvistaen seurakuntia.

\chapter{16}

\par 1 Niin hän saapui myös Derbeen ja Lystraan. Ja katso, siellä oli eräs opetuslapsi, nimeltä Timoteus, joka oli uskovaisen juutalaisvaimon poika, mutta isä oli kreikkalainen.
\par 2 Hänestä veljet, jotka olivat Lystrassa ja Ikonionissa, todistivat hyvää.
\par 3 Paavali tahtoi häntä mukaansa matkalle ja otti hänet ja ympärileikkasi hänet juutalaisten tähden, joita oli niillä paikkakunnilla; sillä kaikki tiesivät, että hänen isänsä oli kreikkalainen.
\par 4 Ja sitä mukaa kuin he vaelsivat kaupungista kaupunkiin, antoivat he heille noudatettaviksi ne säädökset, jotka apostolit ja Jerusalemin vanhimmat olivat hyväksyneet.
\par 5 Niin seurakunnat vahvistuivat uskossa ja saivat päivä päivältä yhä enemmän jäseniä.
\par 6 Ja he kulkivat Frygian ja Galatian maan kautta, sillä Pyhä Henki esti heitä julistamasta sanaa Aasiassa.
\par 7 Ja tultuaan Mysian kohdalle he yrittivät lähteä Bityniaan, mutta Jeesuksen Henki ei sallinut heidän sitä tehdä.
\par 8 Niin he vaelsivat ohi Mysian ja menivät Trooaaseen.
\par 9 Ja Paavali näki yöllä näyn: makedonialainen mies seisoi ja pyysi häntä sanoen: "Tule yli Makedoniaan ja auta meitä".
\par 10 Ja kun hän oli nähnyt sen näyn, niin me kohta tahdoimme päästä lähtemään Makedoniaan, sillä me käsitimme, että Jumala oli kutsunut meitä julistamaan heille evankeliumia.
\par 11 Kun nyt olimme purjehtineet Trooaasta, kuljimme suoraan Samotrakeen, ja seuraavana päivänä Neapoliin,
\par 12 ja sieltä Filippiin, joka on ensimmäinen kaupunki siinä osassa Makedoniaa, siirtokunta. Siinä kaupungissa me viivyimme muutamia päiviä.
\par 13 Ja sapatinpäivänä me menimme kaupungin portin ulkopuolelle, joen rannalle, jossa arvelimme olevan rukouspaikan, ja istuimme sinne ja puhuimme kokoontuneille naisille.
\par 14 Ja eräs Lyydia niminen purppuranmyyjä Tyatiran kaupungista, jumalaapelkääväinen nainen, oli kuulemassa; ja Herra avasi hänen sydämensä ottamaan vaarin siitä, mitä Paavali puhui.
\par 15 Ja kun hänet ja hänen perhekuntansa oli kastettu, pyysi hän meitä sanoen: "Jos te pidätte minua Herraan uskovaisena, niin tulkaa minun kotiini ja majailkaa siellä". Ja hän vaati meitä.
\par 16 Ja tapahtui meidän mennessämme rukouspaikkaan, että meitä vastaan tuli eräs palvelijatar, jossa oli tietäjähenki ja joka tuotti paljon tuloja isännilleen ennustamisellaan.
\par 17 Hän seurasi Paavalia ja meitä ja huusi sanoen: "Nämä miehet ovat korkeimman Jumalan palvelijoita, jotka julistavat teille pelastuksen tien".
\par 18 Ja tätä hän teki monta päivää. Mutta se vaivasi Paavalia, ja hän kääntyi ja sanoi hengelle: "Jeesuksen Kristuksen nimessä minä käsken sinun lähteä hänestä". Ja se lähti sillä hetkellä.
\par 19 Mutta kun hänen isäntänsä näkivät, että tulojen toivo oli heiltä kadonnut, ottivat he Paavalin ja Silaan kiinni ja vetivät heidät torille hallitusmiesten eteen.
\par 20 Ja vietyänsä heidät päällikköjen eteen he sanoivat: "Nämä miehet häiritsevät meidän kaupunkimme rauhaa; he ovat juutalaisia
\par 21 ja opettavat tapoja, joita meidän ei ole lupa omaksua eikä noudattaa, koska me olemme roomalaisia".
\par 22 Ja kansakin nousi heitä vastaan, ja päälliköt revittivät heiltä vaatteet ja käskivät lyödä heitä raipoilla.
\par 23 Ja kun olivat heitä paljon pieksättäneet, heittivät he heidät vankeuteen ja käskivät vanginvartijan tarkasti vartioida heitä.
\par 24 Sellaisen käskyn saatuaan tämä heitti heidät sisimpään vankihuoneeseen ja pani heidät jalkapuuhun.
\par 25 Mutta keskiyön aikaan Paavali ja Silas olivat rukouksissa ja veisasivat ylistystä Jumalalle; ja vangit kuuntelivat heitä.
\par 26 Silloin tapahtui yhtäkkiä suuri maanjäristys, niin että vankilan perustukset järkkyivät, ja samassa kaikki ovet aukenivat, ja kaikkien kahleet irtautuivat.
\par 27 Kun vanginvartija heräsi ja näki vankilan ovien olevan auki, veti hän miekkansa ja aikoi surmata itsensä, luullen vankien karanneen.
\par 28 Mutta Paavali huusi suurella äänellä sanoen: "Älä tee itsellesi mitään pahaa, sillä me kaikki olemme täällä".
\par 29 Niin hän pyysi valoa, juoksi sisälle ja lankesi vavisten Paavalin ja Silaan eteen.
\par 30 Ja hän vei heidät ulos ja sanoi: "Herrat, mitä minun pitää tekemän, että minä pelastuisin?"
\par 31 Niin he sanoivat: "Usko Herraan Jeesukseen, niin sinä pelastut, niin myös sinun perhekuntasi".
\par 32 Ja he puhuivat Jumalan sanaa hänelle ynnä kaikille, jotka hänen kodissansa olivat.
\par 33 Ja hän otti heidät mukaansa samalla yön hetkellä ja pesi heidän haavansa, ja hänet ja kaikki hänen omaisensa kastettiin kohta.
\par 34 Ja hän vei heidät ylös asuntoonsa, laittoi heille aterian ja riemuitsi siitä, että hän ja koko hänen perheensä oli tullut Jumalaan uskovaksi.
\par 35 Päivän tultua päälliköt lähettivät oikeudenpalvelijat sanomaan: "Päästä irti ne miehet".
\par 36 Niin vanginvartija ilmoitti tämän Paavalille sanoen: "Päälliköt ovat lähettäneet sanan, että teidät on päästettävä irti; lähtekää siis nyt ulos ja menkää rauhassa".
\par 37 Mutta Paavali sanoi heille: "He ovat julkisesti, vieläpä ilman tuomiota, ruoskineet meitä, jotka olemme Rooman kansalaisia, ja ovat heittäneet meidät vankeuteen; ja nytkö he salaa ajaisivat meidät tiehemme! Ei niin, vaan tulkoot itse ja viekööt meidät ulos."
\par 38 Ja oikeudenpalvelijat kertoivat ne sanat päälliköille; niin nämä peljästyivät kuullessaan heidän olevan roomalaisia,
\par 39 ja he tulivat ja suostuttelivat heitä ja veivät heidät ulos ja pyysivät heitä lähtemään pois kaupungista.
\par 40 Niin he lähtivät vankilasta ja menivät Lyydian tykö; ja nähtyään veljet ja rohkaistuaan heitä he lähtivät pois.

\chapter{17}

\par 1 Ja he matkustivat Amfipolin ja Apollonian kautta ja tulivat Tessalonikaan, jossa oli juutalaisten synagooga.
\par 2 Ja tapansa mukaan Paavali meni sisälle heidän luoksensa ja keskusteli kolmena sapattina heidän kanssansa, lähtien kirjoituksista,
\par 3 selitti ne ja osoitti, että Kristuksen piti kärsimän ja nouseman kuolleista, ja sanoi: "Tämä Jeesus, jota minä teille julistan, on Kristus".
\par 4 Ja muutamat heistä tulivat uskoon ja liittyivät Paavaliin ja Silaaseen, niin myös suuri joukko jumalaapelkääväisiä kreikkalaisia sekä useat ylhäiset naiset.
\par 5 Mutta juutalaiset joutuivat kiihkoon ja ottivat avukseen muutamia pahanilkisiä miehiä joutoväestä, haalivat kansaa kokoon ja nostivat kaupungissa metelin. He asettuivat Jaasonin talon edustalle ja hakivat Paavalia ja Silasta viedäkseen heidät kansan eteen.
\par 6 Mutta kun he eivät heitä löytäneet, raastoivat he Jaasonin ja muutamia veljiä kaupungin hallitusmiesten eteen ja huusivat: "Nuo koko maailman villitsijät ovat tännekin tulleet,
\par 7 ja heidät Jaason on ottanut vastaan; ja nämä kaikki tekevät vastoin keisarin asetuksia, sanoen erään toisen, Jeesuksen, olevan kuninkaan".
\par 8 Kun kansa ja hallitusmiehet tämän kuulivat, tulivat he levottomiksi.
\par 9 Ja nämä ottivat takauksen Jaasonilta ja muilta ja päästivät heidät.
\par 10 Mutta veljet lähettivät heti yötä myöten Paavalin ja Silaan Bereaan. Ja kun he olivat saapuneet sinne, menivät he juutalaisten synagoogaan.
\par 11 Nämä olivat jalompia kuin Tessalonikan juutalaiset; he ottivat sanan vastaan hyvin halukkaasti ja tutkivat joka päivä kirjoituksia, oliko asia niin.
\par 12 Ja monet heistä uskoivat, niin myös useat ylhäiset kreikkalaiset naiset ja miehet.
\par 13 Mutta kun Tessalonikan juutalaiset saivat tietää, että Paavali Bereassakin julisti Jumalan sanaa, tulivat he sinnekin yllyttämään ja kiihoittamaan kansaa.
\par 14 Silloin veljet heti lähettivät Paavalin menemään meren rantaan; mutta sekä Silas että Timoteus jäivät Bereaan.
\par 15 Ja Paavalin saattajat veivät hänet Ateenaan saakka, ja saatuaan vietäväksi Silaalle ja Timoteukselle käskyn mitä pikimmin tulla hänen luoksensa, he lähtivät sieltä pois.
\par 16 Mutta Paavalin odottaessa heitä Ateenassa hänen henkensä hänessä kiivastui, kun hän näki, että kaupunki oli täynnä epäjumalankuvia.
\par 17 Niin hän keskusteli synagoogassa juutalaisten ja jumalaapelkääväisten kanssa ja torilla joka päivä niiden kanssa, joita hän siellä tapasi.
\par 18 Ja muutamat epikurolaiset ja stoalaiset filosofit väittelivät hänen kanssansa; ja toiset sanoivat: "Mitähän tuo lavertelija oikein tahtoo sanoa?" Toiset taas sanoivat: "Näkyy olevan vieraiden jumalien julistaja", koska hän julisti heille evankeliumia Jeesuksesta ja ylösnousemuksesta.
\par 19 Ja he ottivat hänet ja veivät Areiopagille ja sanoivat: "Voimmeko saada tietää, mikä se uusi oppi on, jota sinä ilmoitat?
\par 20 Sillä outoja asioita sinä tuot meidän korvaimme kuulla. Me siis tahdomme tietää, mitä ne oikein ovat."
\par 21 Sillä ateenalaisilla ja siellä oleskelevilla muukalaisilla ei kenelläkään ollut aikaa muuhun kuin uutta puhumaan ja uutta kuulemaan.
\par 22 Niin Paavali astui keskelle Areiopagia ja sanoi: "Ateenan miehet, minä näen kaikesta, että te suuresti kunnioitatte jumalia.
\par 23 Sillä kävellessäni ympäri ja katsellessani teidän pyhiä paikkojanne minä löysin myös alttarin, johon oli kirjoitettu: 'Tuntemattomalle jumalalle'. Mitä te siis tuntemattanne palvelette, sen minä teille ilmoitan.
\par 24 Jumala, joka on tehnyt maailman ja kaikki, mitä siinä on, hän, joka on taivaan ja maan Herra, ei asu käsillä tehdyissä temppeleissä,
\par 25 eikä häntä voida ihmisten käsillä palvella, ikäänkuin hän jotakin tarvitsisi, hän, joka itse antaa kaikille elämän ja hengen ja kaiken.
\par 26 Ja hän on tehnyt koko ihmissuvun yhdestä ainoasta asumaan kaikkea maanpiiriä ja on säätänyt heille määrätyt ajat ja heidän asumisensa rajat,
\par 27 että he etsisivät Jumalaa, jos ehkä voisivat hapuilemalla hänet löytää - hänet, joka kuitenkaan ei ole kaukana yhdestäkään meistä;
\par 28 sillä hänessä me elämme ja liikumme ja olemme, niinkuin myös muutamat teidän runoilijoistanne ovat sanoneet: 'Sillä me olemme myös hänen sukuansa'.
\par 29 Koska me siis olemme Jumalan sukua, emme saa luulla, että jumaluus on samankaltainen kuin kulta tai hopea tai kivi, sellainen kuin inhimillisen taiteen ja ajatuksen kuvailema.
\par 30 Noita tietämättömyyden aikoja Jumala on kärsinyt, mutta nyt hän tekee tiettäväksi, että kaikkien ihmisten kaikkialla on tehtävä parannus.
\par 31 Sillä hän on säätänyt päivän, jona hän on tuomitseva maanpiirin vanhurskaudessa sen miehen kautta, jonka hän siihen on määrännyt; ja hän on antanut kaikille siitä vakuuden, herättämällä hänet kuolleista."
\par 32 Kuullessaan kuolleitten ylösnousemuksesta toiset ivasivat, toiset taas sanoivat: "Me tahdomme kuulla sinulta tästä vielä toistekin".
\par 33 Ja niin Paavali lähti heidän keskeltänsä.
\par 34 Mutta muutamat liittyivät häneen ja uskoivat; niiden joukossa oli Dionysius, Areiopagin jäsen, ja eräs nainen, nimeltä Damaris, sekä muita heidän kanssansa.

\chapter{18}

\par 1 Sen jälkeen Paavali lähti Ateenasta ja meni Korinttoon.
\par 2 Siellä hän tapasi erään Akylas nimisen juutalaisen, joka oli Pontosta syntyisin ja äsken tullut Italiasta, ja hänen vaimonsa Priskillan. Klaudius oli näet käskenyt kaikkien juutalaisten poistua Roomasta. Ja Paavali meni heidän luoksensa.
\par 3 Ja kun hänellä oli sama ammatti kuin heillä, jäi hän heidän luoksensa, ja he tekivät työtä yhdessä; sillä he olivat ammatiltaan teltantekijöitä.
\par 4 Ja hän keskusteli synagoogassa jokaisena sapattina ja sai sekä juutalaisia että kreikkalaisia uskomaan.
\par 5 Ja kun Silas ja Timoteus tulivat Makedoniasta, oli Paavali kokonaan antautunut sanan julistamiseen ja todisti juutalaisille, että Jeesus on Kristus.
\par 6 Mutta kun he vastustivat ja herjasivat, pudisti hän vaatteitansa ja sanoi heille: "Tulkoon teidän verenne oman päänne päälle! Viaton olen minä; tästedes minä menen pakanain tykö."
\par 7 Ja hän lähti sieltä ja meni erään Titius Justus nimisen jumalaapelkääväisen miehen tykö, jonka talo oli aivan synagoogan vieressä.
\par 8 Mutta synagoogan esimies Krispus ja koko hänen perhekuntansa uskoivat Herraan; ja myöskin monet korinttolaiset, jotka olivat kuulemassa, uskoivat, ja heidät kastettiin.
\par 9 Ja Herra sanoi yöllä näyssä Paavalille: "Älä pelkää, vaan puhu, äläkä vaikene,
\par 10 sillä minä olen sinun kanssasi, eikä kukaan ole ryhtyvä sinuun tehdäkseen sinulle pahaa, sillä minulla on paljon kansaa tässä kaupungissa".
\par 11 Niin hän viipyi siellä vuoden ja kuusi kuukautta opettaen heidän keskuudessaan Jumalan sanaa.
\par 12 Mutta Gallionin ollessa Akaian käskynhaltijana juutalaiset yksimielisesti nousivat Paavalia vastaan ja veivät hänet tuomioistuimen eteen
\par 13 ja sanoivat: "Tämä viettelee ihmisiä palvelemaan Jumalaa lainvastaisella tavalla".
\par 14 Ja kun Paavali aikoi avata suunsa, sanoi Gallion juutalaisille: "Jos olisi tehty rikos tai häijy ilkityö, olisi kohtuullista, että minä kärsivällisesti kuuntelisin teitä, juutalaiset.
\par 15 Mutta jos teillä on riitakysymyksiä opista ja nimistä ja teidän laistanne, olkoot ne teidän huolenanne; niiden tuomari minä en tahdo olla."
\par 16 Ja hän ajoi heidät pois tuomioistuimen edestä.
\par 17 Niin he kaikki ottivat kiinni Soosteneen, synagoogan esimiehen, ja löivät häntä tuomioistuimen edessä, eikä Gallion välittänyt siitä mitään.
\par 18 Mutta Paavali viipyi siellä vielä jonkun aikaa; sitten hän sanoi veljille jäähyväiset ja purjehti Syyriaan, mukanansa Priskilla ja Akylas; hän oli leikkauttanut tukkansa Kenkreassa, sillä hän oli tehnyt lupauksen.
\par 19 Ja he saapuivat Efesoon; sinne hän jätti heidät. Ja hän meni synagoogaan ja keskusteli juutalaisten kanssa.
\par 20 Ja he pyysivät häntä viipymään kauemmin, mutta hän ei suostunut,
\par 21 vaan sanoi heille jäähyväiset ja lausui: "Minä palaan jälleen teidän tykönne, jos Jumala suo". Ja hän lähti purjehtimaan Efesosta.
\par 22 Ja noustuaan maihin Kesareassa hän vaelsi ylös Jerusalemiin ja tervehti seurakuntaa ja meni sitten alas Antiokiaan.
\par 23 Kun hän oli viettänyt siellä jonkun aikaa, lähti hän matkalle ja kulki järjestään kautta Galatian maakunnan ja Frygian, vahvistaen kaikkia opetuslapsia.
\par 24 Ja Efesoon saapui eräs juutalainen, nimeltä Apollos, syntyisin Aleksandriasta, puhetaitoinen mies ja väkevä raamatuissa.
\par 25 Tälle oli opetettu Herran tie, ja hän puhui palavana hengessä ja opetti tarkoin Jeesuksesta, mutta tunsi ainoastaan Johanneksen kasteen.
\par 26 Hän rupesi rohkeasti puhumaan synagoogassa. Mutta kun Priskilla ja Akylas olivat häntä kuunnelleet, ottivat he hänet luokseen ja selvittivät hänelle tarkemmin Jumalan tien.
\par 27 Ja kun hän tahtoi mennä Akaiaan, niin veljet kehoittivat häntä siihen ja kirjoittivat opetuslapsille, että nämä ottaisivat hänet vastaan. Ja sinne saavuttuaan hän armon kautta oli suureksi hyödyksi uskoon tulleille.
\par 28 Sillä hän kumosi suurella voimalla julkisesti juutalaisten väitteet ja näytti kirjoituksista toteen, että Jeesus on Kristus.

\chapter{19}

\par 1 Kun Apollos oli Korintossa, tuli Paavali, kuljettuaan läpi ylämaakuntien, Efesoon ja tapasi siellä muutamia opetuslapsia.
\par 2 Ja hän sanoi heille: "Saitteko Pyhän Hengen silloin, kun te tulitte uskoon?" Niin he sanoivat hänelle: "Emme ole edes kuulleet, että Pyhää Henkeä on olemassakaan".
\par 3 Ja hän sanoi: "Millä kasteella te sitten olette kastetut?" He vastasivat: "Johanneksen kasteella".
\par 4 Niin Paavali sanoi: "Johannes kastoi parannuksen kasteella, kehoittaen kansaa uskomaan häneen, joka oli tuleva hänen jälkeensä, se on, Jeesukseen".
\par 5 Sen kuultuaan he ottivat kasteen Herran Jeesuksen nimeen.
\par 6 Ja kun Paavali pani kätensä heidän päälleen, tuli heidän päällensä Pyhä Henki, ja he puhuivat kielillä ja ennustivat.
\par 7 Heitä oli kaikkiaan noin kaksitoista miestä.
\par 8 Ja hän meni synagoogaan, ja kolmen kuukauden ajan hän puhui heidän kanssansa rohkeasti ja vakuuttavasti Jumalan valtakunnasta.
\par 9 Mutta kun muutamat paaduttivat itsensä eivätkä uskoneet, vaan puhuivat pahaa Herran tiestä kansan edessä, niin hän meni pois heidän luotaan ja erotti opetuslapset heistä ja piti joka päivä keskusteluja Tyrannuksen koulussa.
\par 10 Ja sitä kesti kaksi vuotta, niin että kaikki Aasian asukkaat, sekä juutalaiset että kreikkalaiset, saivat kuulla Herran sanan.
\par 11 Ja Jumala teki ylen voimallisia tekoja Paavalin kätten kautta,
\par 12 niin että vieläpä hikiliinoja ja esivaatteita hänen iholtansa vietiin sairasten päälle, ja taudit lähtivät heistä ja pahat henget pakenivat pois.
\par 13 Myöskin muutamat kuljeksivat juutalaiset loitsijat rupesivat lausumaan Herran Jeesuksen nimeä niiden ylitse, joissa oli pahoja henkiä, sanoen: "Minä vannotan teitä sen Jeesuksen kautta, jota Paavali julistaa".
\par 14 Ja niiden joukossa, jotka näin tekivät, oli myös erään juutalaisen ylipapin, Skeuaan, seitsemän poikaa;
\par 15 mutta paha henki vastasi heille sanoen: "Jeesuksen minä tunnen, ja Paavalin minä tiedän, mutta keitä te olette?"
\par 16 Ja se mies, jossa paha henki oli, karkasi heidän kimppuunsa, voitti heidät toisen toisensa perästä ja runteli heitä, niin että he alastomina ja haavoitettuina pakenivat siitä huoneesta.
\par 17 Ja tämän saivat tietää kaikki Efeson asukkaat, sekä juutalaiset että kreikkalaiset; ja heidät kaikki valtasi pelko, ja Herran Jeesuksen nimeä ylistettiin suuresti.
\par 18 Ja monet niistä, jotka olivat tulleet uskoon, menivät ja tunnustivat ja ilmoittivat tekonsa.
\par 19 Ja useat niistä, jotka olivat taikuutta harjoittaneet, kantoivat kirjansa kokoon ja polttivat ne kaikkien nähden; ja kun niiden arvo laskettiin yhteen, huomattiin sen olevan viisikymmentä tuhatta hopearahaa.
\par 20 Näin Herran sana voimallisesti kasvoi ja vahvistui.
\par 21 Kun kaikki tämä oli tapahtunut, niin Paavali hengessä päätti kulkea Makedonian ja Akaian kautta ja matkustaa Jerusalemiin ja sanoi: "Käytyäni siellä minun pitää nähdä myös Rooma".
\par 22 Ja hän lähetti Makedoniaan kaksi apumiehistänsä, Timoteuksen ja Erastuksen, mutta jäi itse joksikin aikaa Aasiaan.
\par 23 Tähän aikaan syntyi sangen suuri melu siitä tiestä.
\par 24 Sillä eräs hopeaseppä, nimeltä Demetrius, joka valmisti hopeaisia Artemiin temppeleitä, hankki sillä ammattilaisille melkoisia tuloja.
\par 25 Hän kutsui kokoon nämä sekä muut, jotka sellaista työtä tekivät, ja sanoi: "Miehet, te tiedätte, että meillä on hyvä toimeentulomme tästä työstä;
\par 26 mutta nyt te näette ja kuulette, että tuo Paavali on, ei ainoastaan Efesossa, vaan melkein koko Aasiassa, uskotellut ja vietellyt paljon kansaa, sanoen, etteivät ne ole jumalia, jotka käsillä tehdään.
\par 27 Ja nyt uhkaa se vaara, että ei ainoastaan tämä meidän elinkeinomme joudu halveksituksi, vaan myöskin, että suuren Artemis jumalattaren temppeliä ei pidetä minäkään ja että hän menettää mahtavuutensa, hän, jota koko Aasia ja koko maanpiiri palvelee."
\par 28 Kun he sen kuulivat, tulivat he vihaa täyteen ja huusivat sanoen: "Suuri on efesolaisten Artemis!"
\par 29 Ja koko kaupunki joutui sekasortoon, ja he ryntäsivät kaikki yhdessä näytelmäpaikkaan ja tempasivat mukaansa Gaiuksen ja Aristarkuksen, kaksi makedonialaista, jotka olivat Paavalin matkatovereita.
\par 30 Ja kun Paavali tahtoi mennä kansanjoukkoon, eivät opetuslapset sitä sallineet.
\par 31 Ja myös muutamat Aasian hallitusmiehistä, jotka olivat hänen ystäviään, lähettivät hänelle sanan ja pyysivät, ettei hän menisi näytelmäpaikkaan.
\par 32 Ja he huusivat, mikä mitäkin; sillä kokous oli sekasortoinen, ja useimmat eivät tienneet, minkätähden he olivat tulleet kokoon.
\par 33 Silloin vedettiin joukosta esille Aleksander, jota juutalaiset työnsivät esiin; niin Aleksander viittasi kädellään merkiksi, että hän tahtoi pitää puolustuspuheen kansan edessä.
\par 34 Mutta kun he huomasivat, että hän oli juutalainen, rupesivat he kaikki yhteen ääneen huutamaan ja kirkuivat noin kaksi hetkeä: "Suuri on efesolaisten Artemis!"
\par 35 Mutta kun kaupungin kansleri oli saanut kansan rauhoittumaan, sanoi hän: "Efeson miehet, onko ketään, joka ei tiedä, että efesolaisten kaupunki on suuren Artemiin temppelin ja hänen taivaasta pudonneen kuvansa vaalija?
\par 36 Koska ei kukaan voi tätä kieltää, tulee teidän siis pysyä rauhallisina eikä tehdä mitään harkitsematonta.
\par 37 Te olette kuitenkin tuoneet tänne nämä miehet, jotka eivät ole temppelin ryöstäjiä eivätkä ole meidän jumalatartamme pilkanneet.
\par 38 Jos siis Demetriuksella ja hänen ammattiveljillänsä on riita-asiaa jotakuta vastaan, niin pidetäänhän oikeudenistuntoja ja onhan käskynhaltijoita; vetäkööt toisensa oikeuteen.
\par 39 Ja jos teillä on vielä jotakin muuta vaadittavaa, niin ratkaistakoon asia laillisessa kansankokouksessa.
\par 40 Sillä tämänpäiväisen tapahtuman tähden me olemme vaarassa joutua syytteeseen jopa kapinasta, vaikkei mitään aihetta olekaan; ja silloin me emme voi vastata tästä mellakasta."

\chapter{20}

\par 1 Kun meteli oli asettunut, kutsui Paavali opetuslapset luoksensa; ja rohkaistuaan heitä hän jätti heidät hyvästi ja lähti matkustamaan Makedoniaan.
\par 2 Ja kuljettuaan niiden paikkakuntien läpi ja puhuttuaan siellä monta kehoituksen sanaa hän tuli Kreikkaan.
\par 3 Siellä hän oleskeli kolme kuukautta. Ja kun juutalaiset olivat tehneet häntä vastaan salahankkeen hänen aikoessaan lähteä meritse Syyriaan, päätti hän tehdä paluumatkansa Makedonian kautta.
\par 4 Ja häntä seurasivat berealainen Soopater, Pyrruksen poika, ja tessalonikalaisista Aristarkus ja Sekundus, derbeläinen Gaius, Timoteus sekä aasialaiset Tykikus ja Trofimus.
\par 5 Nämä menivät edeltä ja odottivat meitä Trooaassa;
\par 6 mutta me purjehdimme happamattoman leivän juhlan jälkeen Filippistä ja tulimme viidentenä päivänä heidän luoksensa Trooaaseen ja viivyimme siellä seitsemän päivää.
\par 7 Ja kun viikon ensimmäisenä päivänä olimme kokoontuneet murtamaan leipää, niin Paavali, joka seuraavana päivänä aikoi matkustaa pois, keskusteli heidän kanssansa ja pitkitti puhettaan puoliyöhön saakka.
\par 8 Ja monta lamppua oli palamassa yläsalissa, jossa me olimme koolla.
\par 9 Niin eräs nuorukainen, nimeltä Eutykus, istui ikkunalla, ja kun Paavalin puhe kesti niin kauan, vaipui hän sikeään uneen ja putosi unen vallassa kolmannesta kerroksesta maahan; ja hänet nostettiin ylös kuolleena.
\par 10 Mutta Paavali meni alas, heittäytyi hänen ylitsensä, kiersi kätensä hänen ympärilleen ja sanoi: "Älkää hätäilkö, sillä hänessä on vielä henki".
\par 11 Niin hän meni jälleen ylös, mursi leipää ja söi; ja hän puhui kauan heidän kanssansa, päivän koittoon asti, ja lähti sitten matkalle.
\par 12 Ja he veivät pojan sieltä elävänä ja tulivat suuresti lohdutetuiksi.
\par 13 Mutta me menimme edeltäpäin ja astuimme laivaan ja purjehdimme Assoon. Sieltä aioimme ottaa Paavalin laivaan, sillä hän oli niin määrännyt, aikoen itse kulkea maitse.
\par 14 Ja kun hän yhtyi meihin Assossa, otimme hänet laivaan ja kuljimme Mityleneen.
\par 15 Sieltä me purjehdimme ja saavuimme toisena päivänä Kion kohdalle; seuraavana päivänä laskimme Samoon ja tulimme sen jälkeisenä päivänä Miletoon.
\par 16 Sillä Paavali oli päättänyt purjehtia Efeson ohitse, ettei häneltä kuluisi aikaa Aasiassa; sillä hän kiiruhti joutuakseen, jos suinkin mahdollista, helluntaiksi Jerusalemiin.
\par 17 Mutta Miletosta hän lähetti sanan Efesoon ja kutsui tykönsä seurakunnan vanhimmat.
\par 18 Ja kun he saapuivat hänen tykönsä, sanoi hän heille: "Te tiedätte ensimmäisestä päivästä asti, kun minä Aasiaan tulin, miten minä kaiken aikaa olen ollut teidän kanssanne;
\par 19 kuinka minä olen palvellut Herraa kaikella nöyryydellä ja kyynelillä, koettelemuksissa, jotka ovat kohdanneet minua juutalaisten salahankkeiden tähden;
\par 20 kuinka minä en ole vetäytynyt pois julistamasta teille sitä, mikä hyödyllistä on, ja opettamasta teitä sekä julkisesti että huone huoneelta,
\par 21 vaan olen todistanut sekä juutalaisille että kreikkalaisille parannusta kääntymyksessä Jumalan puoleen ja uskoa meidän Herraamme Jeesukseen Kristukseen.
\par 22 Ja nyt, katso, minä matkustan, sidottuna hengessä, Jerusalemiin, enkä tiedä, mikä minua siellä kohtaa.
\par 23 Sen vain tiedän, että Pyhä Henki jokaisessa kaupungissa todistaa minulle ja sanoo, että kahleet ja ahdistukset minua odottavat.
\par 24 En minä kuitenkaan pidä henkeäni itselleni minkään arvoisena, kunhan vain täytän juoksuni ja sen viran, jonka minä Herralta Jeesukselta olen saanut: Jumalan armon evankeliumin todistamisen.
\par 25 Ja nyt, katso, minä tiedän, ettette enää saa nähdä minun kasvojani, ei kukaan teistä, joiden keskuudessa minä olen vaeltanut ja saarnannut valtakuntaa.
\par 26 Sentähden minä todistan teille tänä päivänä, että minä olen viaton kaikkien vereen.
\par 27 Sillä minä en ole vetäytynyt pois julistamasta teille kaikkea Jumalan tahtoa.
\par 28 Ottakaa siis itsestänne vaari ja kaikesta laumasta, johon Pyhä Henki on teidät pannut kaitsijoiksi, paimentamaan Herran seurakuntaa, jonka hän omalla verellänsä on itselleen ansainnut.
\par 29 Minä tiedän, että minun lähtöni jälkeen teidän keskuuteenne tulee julmia susia, jotka eivät laumaa säästä,
\par 30 ja teidän omasta joukostanne nousee miehiä, jotka väärää puhetta puhuvat, vetääkseen opetuslapset mukaansa.
\par 31 Valvokaa sentähden ja muistakaa, että minä olen kolme vuotta lakkaamatta yötä ja päivää kyynelin neuvonut teitä itsekutakin.
\par 32 Ja nyt minä uskon teidät Jumalan ja hänen armonsa sanan haltuun, hänen, joka on voimallinen rakentamaan teitä ja antamaan teille perintöosan kaikkien pyhitettyjen joukossa.
\par 33 En minä ole halunnut kenenkään hopeata tai kultaa tai vaatteita;
\par 34 te tiedätte itse, että nämä minun käteni ovat työllänsä hankkineet, mitä minä ja seuralaiseni olemme tarvinneet.
\par 35 Kaikessa minä olen osoittanut teille, että näin työtä tehden tulee huolehtia heikoista ja muistaa nämä Herran Jeesuksen sanat, jotka hän itse sanoi: 'Autuaampi on antaa kuin ottaa'."
\par 36 Ja tämän sanottuaan hän polvistui ja rukoili kaikkien heidän kanssansa.
\par 37 Ja he ratkesivat kaikki haikeasti itkemään ja lankesivat Paavalin kaulaan ja suutelivat häntä,
\par 38 ja enimmän suretti heitä se sana, jonka hän oli sanonut, etteivät he enää saisi nähdä hänen kasvojansa. Ja he saattoivat hänet laivaan.

\chapter{21}

\par 1 Kun olimme eronneet heistä ja lähteneet purjehtimaan, laskimme suoraan Koossaareen ja seuraavana päivänä Rodoon ja sieltä Pataraan.
\par 2 Siellä tapasimme Foinikiaan menevän laivan, astuimme siihen ja lähdimme purjehtimaan.
\par 3 Ja kun Kypro rupesi näkymään ja oli jäänyt meistä vasemmalle, purjehdimme Syyriaan ja nousimme maihin Tyyrossa; siellä näet laivan oli määrä purkaa lastinsa.
\par 4 Ja tavattuamme opetuslapset me viivyimme siellä seitsemän päivää. Ja Hengen vaikutuksesta he varoittivat Paavalia menemästä Jerusalemiin.
\par 5 Mutta kun olimme viettäneet loppuun ne päivät, lähdimme matkalle, ja kaikki saattoivat vaimoineen ja lapsineen meitä kaupungin ulkopuolelle saakka. Ja me laskeuduimme rannalla polvillemme ja rukoilimme;
\par 6 ja sanottuamme jäähyväiset toisillemme astuimme laivaan, ja he palasivat kotiinsa.
\par 7 Tyyrosta me saavuimme Ptolemaikseen, ja siihen päättyi purjehduksemme. Ja me tervehdimme veljiä siellä ja viivyimme päivän heidän luonansa.
\par 8 Mutta seuraavana päivänä me lähdimme sieltä ja tulimme Kesareaan, jossa menimme evankelista Filippuksen tykö, joka oli yksi niistä seitsemästä, ja jäimme hänen tykönsä.
\par 9 Ja hänellä oli neljä tytärtä, neitsyttä, joilla oli profetoimisen lahja.
\par 10 Siellä me viivyimme useita päiviä. Niin tuli sinne Juudeasta eräs profeetta, nimeltä Agabus.
\par 11 Ja tultuaan meidän luoksemme hän otti Paavalin vyön, sitoi sillä jalkansa ja kätensä ja lausui: "Näin sanoo Pyhä Henki: 'Sen miehen, jonka vyö tämä on, juutalaiset näin sitovat Jerusalemissa ja antavat pakanain käsiin'."
\par 12 Kun sen kuulimme, pyysimme, sekä me että ne, jotka siellä asuivat, ettei hän menisi Jerusalemiin.
\par 13 Silloin Paavali vastasi ja sanoi: "Mitä te teette, kun itkette ja särjette minun sydäntäni. Sillä minä olen valmis, en ainoastaan käymään sidottavaksi vaan myöskin kuolemaan Jerusalemissa Herran Jeesuksen nimen tähden."
\par 14 Ja kun hän ei taipunut, niin me rauhoituimme ja sanoimme: "Tapahtukoon Herran tahto".
\par 15 Niiden päivien kuluttua me hankkiuduimme ja menimme ylös Jerusalemiin.
\par 16 Ja meidän kanssamme tuli myös opetuslapsia Kesareasta, jotka veivät meidät majapaikkaamme, erään vanhan opetuslapsen, kyprolaisen Mnasonin, tykö.
\par 17 Ja saavuttuamme Jerusalemiin veljet ottivat meidät iloiten vastaan.
\par 18 Seuraavana päivänä Paavali meni meidän kanssamme Jaakobin tykö, ja kaikki vanhimmat tulivat sinne saapuville.
\par 19 Ja kun hän oli heitä tervehtinyt, kertoi hän kohta kohdalta kaikki, mitä Jumala hänen palveluksensa kautta oli tehnyt pakanain keskuudessa.
\par 20 Sen kuultuaan he ylistivät Jumalaa ja sanoivat Paavalille: "Sinä näet, veli, kuinka monta tuhatta juutalaista on tullut uskoon, ja he ovat kaikki lainkiivailijoita.
\par 21 Mutta heille on kerrottu sinusta, että sinä opetat kaikkia pakanain seassa asuvia juutalaisia luopumaan Mooseksesta ja kiellät heitä ympärileikkaamasta lapsiaan ja vaeltamasta säädettyjen tapojen mukaan.
\par 22 Mitä siis on tehtävä? Varmaankin on suuri joukko kokoontuva, sillä he saavat kuulla sinun tulleen.
\par 23 Tee siis tämä, minkä me nyt sinulle sanomme. Meillä on täällä neljä miestä, joilla on lupaus täytettävänä.
\par 24 Ota ne luoksesi ja puhdista itsesi heidän kanssansa ja maksa kulut heidän puolestaan, että he saisivat leikkauttaa tukkansa; siitä kaikki huomaavat, ettei ole mitään perää siinä, mitä heille on kerrottu sinusta, vaan että sinäkin vaellat lain mukaan ja noudatat sitä.
\par 25 Mutta uskoon tulleista pakanoista me olemme päättäneet ja kirjoittaneet, että heidän on välttäminen epäjumalille uhrattua ja verta ja lihaa, josta ei veri ole laskettu, ja haureutta."
\par 26 Silloin Paavali otti ne miehet luokseen, ja kun hän seuraavana päivänä oli puhdistanut itsensä heidän kanssaan, meni hän pyhäkköön ja ilmoitti, milloin heidän puhdistumispäivänsä tulisivat päättymään, jota ennen heidän kunkin edestä oli tuotava uhri.
\par 27 Mutta kun ne seitsemän päivää olivat päättymässä, näkivät Aasiasta tulleet juutalaiset hänet pyhäkössä, kiihoittivat kaiken kansan ja kävivät häneen käsiksi
\par 28 ja huusivat: "Israelin miehet, auttakaa! Tämä on se mies, joka kaikkialla opettaa kaikkia ihmisiä meidän kansaamme ja lakiamme ja tätä paikkaa vastaan, onpa vielä tuonut kreikkalaisia pyhäkköönkin ja saastuttanut tämän pyhän paikan."
\par 29 Sillä he olivat ennen nähneet efesolaisen Trofimuksen kaupungilla hänen kanssaan ja luulivat, että Paavali oli tuonut hänet pyhäkköön.
\par 30 Ja koko kaupunki tuli liikkeelle, ja väkeä juoksi kokoon; ja he ottivat Paavalin kiinni, raastoivat hänet ulos pyhäköstä, ja heti portit suljettiin.
\par 31 Ja kun he tahtoivat hänet tappaa, sai sotaväenosaston päällikkö sanan, että koko Jerusalem oli kuohuksissa.
\par 32 Tämä otti heti paikalla mukaansa sotilaita ja sadanpäämiehiä ja riensi alas heidän luoksensa. Kun he näkivät päällikön ja sotilaat, lakkasivat he lyömästä Paavalia.
\par 33 Silloin päällikkö astui esiin, otatti hänet kiinni ja käski sitoa hänet kaksilla kahleilla ja kysyi, kuka hän oli ja mitä hän oli tehnyt.
\par 34 Mutta kansanjoukosta huusivat toiset sitä, toiset tätä. Ja koska hän melun tähden ei voinut saada varmaa selkoa, käski hän viedä hänet kasarmiin.
\par 35 Ja kun Paavali tuli portaille, täytyi sotamiesten kantaa häntä kansan väkivallan tähden;
\par 36 sillä suuri kansanpaljous seurasi perässä ja huusi: "Vie pois hänet!"
\par 37 Ja kun oltiin kuljettamassa Paavalia sisälle kasarmiin, sanoi hän päällikölle: "Onko minun lupa sanoa sinulle jotakin?" Niin tämä sanoi: "Osaatko sinä siis kreikkaa?
\par 38 Etkö sitten olekaan se egyptiläinen, joka hiljakkoin villitsi ne neljätuhatta murhamiestä ja vei heidät erämaahan?"
\par 39 Niin Paavali sanoi: "Minä olen juutalainen mies, Tarson, tunnetun Kilikian kaupungin, kansalainen; pyydän sinua, salli minun puhua kansalle".
\par 40 Ja kun hän sen salli, niin Paavali, seisoen portailla, viittasi kädellään kansalle; ja kun oli syntynyt syvä hiljaisuus, puhui hän heille hebreankielellä ja sanoi:

\chapter{22}

\par 1 "Miehet, veljet ja isät, kuulkaa, mitä minä nyt teille puolustuksekseni puhun".
\par 2 Kun he kuulivat hänen puhuvan heille hebreankielellä, syntyi vielä suurempi hiljaisuus. Ja hän sanoi:
\par 3 "Minä olen juutalainen, syntynyt Kilikian Tarsossa, mutta kasvatettu tässä kaupungissa ja Gamalielin jalkojen juuressa opetettu tarkkaan noudattamaan isien lakia; ja minä kiivailin Jumalan puolesta, niinkuin te kaikki tänä päivänä kiivailette.
\par 4 Ja minä vainosin tätä tietä aina kuolemaan asti, sitoen ja heittäen vankeuteen sekä miehiä että naisia,
\par 5 niinkuin myös ylimmäinen pappi voi minusta todistaa, ja kaikki vanhimmat. Minä sain heiltä myös kirjeitä veljille Damaskoon, ja minä matkustin sinne tuodakseni nekin, jotka siellä olivat, sidottuina Jerusalemiin rangaistaviksi.
\par 6 Niin tapahtui, kun minä matkalla ollessani lähestyin Damaskoa, että keskipäivän aikaan yhtäkkiä taivaasta leimahti suuri valo minun ympärilläni;
\par 7 ja minä kaaduin maahan ja kuulin äänen sanovan minulle: 'Saul, Saul, miksi vainoat minua?'
\par 8 Niin minä vastasin: 'Kuka olet, herra?' Ja hän sanoi minulle: 'Minä olen Jeesus Nasaretilainen, jota sinä vainoat'.
\par 9 Ja minun seuralaiseni näkivät kyllä valon, mutta eivät kuulleet sen ääntä, joka minulle puhui.
\par 10 Ja minä sanoin: 'Herra, mitä minun pitää tekemän?' Herra sanoi minulle: 'Nouse ja mene Damaskoon, niin siellä sinulle sanotaan kaikki, mikä sinulle on tehtäväksi asetettu'.
\par 11 Ja kun minä sen valon kirkkaudesta tulin näkemättömäksi, taluttivat seuralaiseni minua kädestä, ja niin minä tulin Damaskoon.
\par 12 Ja eräs mies, hurskas lain mukaan, nimeltä Ananias, josta kaikki siellä asuvat juutalaiset todistivat hyvää,
\par 13 tuli minun tyköni, astui eteeni ja sanoi minulle: 'Saul, veljeni, saa näkösi jälleen'. Ja sillä hetkellä minä sain näköni ja katsoin häneen.
\par 14 Niin hän sanoi: 'Meidän isiemme Jumala on valinnut sinut tuntemaan hänen tahtonsa ja näkemään Vanhurskaan ja kuulemaan hänen suunsa äänen;
\par 15 sillä sinä olet oleva hänen todistajansa kaikkien ihmisten edessä, sen todistaja, mitä olet nähnyt ja kuullut.
\par 16 Ja nyt, mitä viivyttelet? Nouse, huuda avuksi hänen nimeänsä ja anna kastaa itsesi ja pestä pois syntisi.'
\par 17 Kun olin palannut Jerusalemiin, tapahtui minun rukoillessani pyhäkössä, että minä jouduin hurmoksiin
\par 18 ja näin hänet. Ja hän sanoi minulle: 'Riennä ja lähde pian pois Jerusalemista, sillä he eivät ota vastaan sinun todistustasi minusta'.
\par 19 Ja minä sanoin: 'Herra, he tietävät itse, että minä panin vankeuteen ja ruoskitin jokaisessa synagoogassa niitä, jotka uskoivat sinuun.
\par 20 Ja kun Stefanuksen, sinun todistajasi, veri vuodatettiin, olin minäkin läsnä, hyväksyin sen ja vartioin hänen surmaajainsa vaatteita.'
\par 21 Ja hän sanoi minulle: 'Mene, sillä minä lähetän sinut kauas pakanain tykö'."
\par 22 Tähän sanaan asti he kuuntelivat häntä; mutta silloin he korottivat äänensä ja sanoivat: "Pois maan päältä tuommoinen! Sillä ei hän saa elää."
\par 23 Ja kun he huusivat ja heittelivät vaatteitaan ja viskoivat tomua ilmaan,
\par 24 käski päällikkö viedä hänet kasarmiin, ja saadakseen tietää, mistä syystä he niin hänelle huusivat, hän määräsi hänet ruoskimalla tutkittavaksi.
\par 25 Mutta kun he olivat oikaisseet hänet ruoskittavaksi, sanoi Paavali siinä seisovalle sadanpäämiehelle: "Onko teidän lupa ruoskia Rooman kansalaista, vieläpä ilman tuomiota?"
\par 26 Kun sadanpäämies sen kuuli, meni hän päällikölle ilmoittamaan ja sanoi: "Mitä aiot tehdä? Tämä mies on Rooman kansalainen."
\par 27 Niin päällikkö meni Paavalin luo ja sanoi hänelle: "Sano minulle: oletko sinä Rooman kansalainen?" Hän vastasi: "Olen".
\par 28 Niin päällikkö sanoi: "Minä olen paljolla rahalla hankkinut itselleni tämän kansalaisoikeuden". Paavali sanoi: "Mutta minulla se on syntymästäni asti".
\par 29 Silloin ne, joiden piti häntä tutkia, lähtivät heti hänen luotaan. Ja myös päällikkö peljästyi saatuaan tietää, että Paavali oli Rooman kansalainen, kun oli sidottanut hänet.
\par 30 Mutta seuraavana päivänä, koska hän tahtoi saada varman tiedon, mistä juutalaiset häntä syyttivät, päästi hän hänet siteistä ja käski ylipappien ja koko neuvoston kokoontua, vei Paavalin alas ja asetti hänet heidän eteensä.

\chapter{23}

\par 1 Niin Paavali loi katseensa neuvostoon ja sanoi: "Miehet, veljet, minä olen kaikessa hyvällä omallatunnolla vaeltanut Jumalan edessä tähän päivään asti".
\par 2 Mutta ylimmäinen pappi Ananias käski lähellä seisovia lyömään häntä vasten suuta.
\par 3 Silloin Paavali sanoi hänelle: "Jumala on lyövä sinua, sinä valkeaksi kalkittu seinä; istutko sinä tuomitsemassa minua lain mukaan ja käsket vastoin lakia lyödä minua?"
\par 4 Niin ne, jotka seisoivat lähellä, sanoivat: "Herjaatko sinä Jumalan ylimmäistä pappia?"
\par 5 Ja Paavali sanoi: "En tiennyt, veljet, että hän on ylimmäinen pappi; sillä kirjoitettu on: 'Kansasi ruhtinasta älä kiroa'".
\par 6 Mutta koska Paavali tiesi osan heistä olevan saddukeuksia ja toisen osan fariseuksia, huusi hän neuvoston edessä: "Miehet, veljet, minä olen fariseus, fariseusten jälkeläinen; toivon ja kuolleitten ylösnousemuksen tähden minä olen tuomittavana".
\par 7 Tuskin hän oli tämän sanonut, niin nousi riita fariseusten ja saddukeusten kesken, ja kokous jakautui.
\par 8 Sillä saddukeukset sanovat, ettei ylösnousemusta ole, ei enkeliä eikä henkeä, mutta fariseukset tunnustavat kumpaisetkin.
\par 9 Ja syntyi suuri huuto, ja muutamat kirjanoppineet fariseusten puolueesta nousivat ja väittelivät kiivaasti ja sanoivat: "Emme löydä mitään pahaa tässä miehessä; entäpä jos henki tai enkeli on hänelle puhunut?"
\par 10 Ja kun riita yhä kiihtyi, pelkäsi päällikkö, että he repisivät Paavalin kappaleiksi, ja käski sotaväen tulla alas ja temmata hänet heidän keskeltään ja viedä hänet kasarmiin.
\par 11 Mutta seuraavana yönä Herra seisoi Paavalin tykönä ja sanoi: "Ole turvallisella mielellä, sillä niinkuin sinä olet todistanut minusta Jerusalemissa, niin sinun pitää todistaman minusta myös Roomassa".
\par 12 Mutta päivän tultua juutalaiset tekivät salaliiton ja vannoivat valan, etteivät söisi eivätkä joisi, ennenkuin olivat tappaneet Paavalin.
\par 13 Ja niitä oli viidettäkymmentä miestä, jotka yhtyivät tähän valaan.
\par 14 He menivät ylipappien ja vanhinten luo ja sanoivat: "Me olemme kirouksen uhalla vannoneet, ettemme mitään maista, ennenkuin olemme tappaneet Paavalin.
\par 15 Pyytäkää te siis nyt yhdessä neuvoston kanssa päälliköltä, että hän toisi hänet alas teidän luoksenne, ikäänkuin aikoisitte tarkemmin tutkia hänen asiaansa. Mutta me olemme valmiit tappamaan hänet, ennenkuin hän pääsee perille."
\par 16 Mutta Paavalin sisarenpoika, joka oli saanut kuulla väijytyksestä, saapui kasarmille, meni sisälle ja ilmoitti sen Paavalille.
\par 17 Niin Paavali kutsui luoksensa erään sadanpäämiehen ja sanoi: "Vie tämä nuorukainen päällikön luo, sillä hänellä on jotakin hänelle ilmoitettavaa".
\par 18 Niin hän otti hänet mukaansa, vei hänet päällikön luo ja sanoi: "Vanki Paavali kutsui minut luokseen ja pyysi tuomaan sinun luoksesi tämän nuorukaisen, jolla on jotakin puhuttavaa sinulle".
\par 19 Niin päällikkö tarttui hänen käteensä, vei hänet erikseen ja kysyi: "Mitä sinulla on minulle ilmoitettavaa?"
\par 20 Hän sanoi: "Juutalaiset ovat päättäneet anoa sinulta, että huomenna veisit Paavalin alas neuvostoon, ikäänkuin aikoisit vielä tarkemmin tutkia hänen asiaansa.
\par 21 Mutta älä sinä siihen suostu, sillä viidettäkymmentä miestä heidän joukostaan on häntä väijymässä, ja he ovat vannoneet valan, etteivät syö eivätkä juo, ennenkuin ovat tappaneet hänet. Ja nyt he ovat valmiina ja odottavat sinun suostumustasi."
\par 22 Niin päällikkö päästi nuorukaisen menemään ja sanoi hänelle: "Älä virka kenellekään, että olet ilmaissut tämän minulle".
\par 23 Sitten hän kutsui luoksensa kaksi sadanpäämiestä ja sanoi heille: "Pitäkää yön kolmannesta hetkestä lähtien kaksisataa sotamiestä valmiina lähtemään Kesareaan ja seitsemänkymmentä ratsumiestä ja kaksisataa keihäsmiestä,
\par 24 ja varatkaa ratsuja pannaksenne Paavalin ratsaille ja viedäksenne hänet vahingoittumatonna maaherra Feeliksin luo".
\par 25 Ja hän kirjoitti kirjeen, joka kuului näin:
\par 26 "Klaudius Lysias lausuu tervehdyksen korkea-arvoiselle maaherralle Feeliksille.
\par 27 Tämän miehen ottivat juutalaiset kiinni ja olivat vähällä hänet tappaa; silloin minä tulin saapuville sotaväen kanssa ja pelastin hänet, saatuani tietää, että hän on Rooman kansalainen.
\par 28 Ja koska tahdoin tietää, mistä asiasta he häntä syyttivät, vein hänet heidän neuvostoonsa
\par 29 ja havaitsin, että häntä syytettiin heidän lakiaan koskevista riitakysymyksistä, mutta ettei ollut kannetta mistään, mikä ansaitsisi kuoleman tai kahleet.
\par 30 Mutta kun minulle on annettu ilmi, että miestä vastaan on tekeillä salahanke, lähetän hänet nyt heti sinun luoksesi; olen myös kehoittanut hänen syyttäjiään sanomaan sanottavansa häntä vastaan sinun edessäsi."
\par 31 Niin sotamiehet, saamansa käskyn mukaan, ottivat Paavalin ja veivät hänet yötä myöten Antipatrikseen.
\par 32 Seuraavana päivänä he antoivat ratsumiesten jatkaa hänen kanssaan matkaa, mutta itse he palasivat kasarmiin.
\par 33 Kun ratsumiehet tulivat Kesareaan, antoivat he kirjeen maaherralle ja veivät myös Paavalin hänen eteensä.
\par 34 Luettuaan kirjeen hän kysyi, mistä maakunnasta Paavali oli; ja saatuaan tietää, että hän oli Kilikiasta,
\par 35 hän sanoi: "Minä kuulustelen sinua, kun syyttäjäsikin saapuvat". Ja hän käski vartioida häntä Herodeksen linnassa.

\chapter{24}

\par 1 Viiden päivän kuluttua ylimmäinen pappi Ananias meni sinne alas muutamien vanhinten ja erään asianajajan, Tertulluksen, kanssa, ja he ilmoittivat maaherralle syyttävänsä Paavalia.
\par 2 Ja kun Paavali oli kutsuttu esille, rupesi Tertullus syyttämään ja sanoi:
\par 3 "Runsasta rauhaa me olemme sinun kauttasi, korkea-arvoinen Feeliks, saaneet nauttia, ja sinun huolenpidostasi on parannuksia aikaansaatu tämän kansan hyväksi, sen me kaikin puolin ja kaikkialla ja kaikella kiitollisuudella tunnustamme.
\par 4 Mutta etten aivan kauan sinua viivyttäisi, pyydän sinua hetkisen meitä suosiollisesti kuulemaan.
\par 5 Me olemme havainneet, että tämä mies on ruttotauti ja metelinnostaja kaikkien koko maailman juutalaisten keskuudessa ja nasaretilaisten lahkon päämies,
\par 6 ja hän on koettanut pyhäkönkin saastuttaa. Sentähden me otimme hänet kiinni.
\par 7 []
\par 8 Voit itse häntä tutkimalla saada tietää kaiken, mistä me häntä syytämme."
\par 9 Ja myös juutalaiset yhtyivät syyttämään häntä ja väittivät asian niin olevan.
\par 10 Paavali vastasi, kun maaherra oli viitannut, että hän sai puhua: "Koska tiedän sinun monta vuotta olleen tämän kansan tuomarina, puhun luottamuksella asiani puolesta.
\par 11 Niinkuin voit saada tietää, ei ole kuin kaksitoista päivää siitä, kun menin Jerusalemiin rukoilemaan.
\par 12 Eivät he ole tavanneet minua kenenkään kanssa väittelemästä eikä väentungoksia aikaansaamasta, ei pyhäkössä, ei synagoogissa eikä kaupungilla,
\par 13 eivätkä myöskään voi näyttää sinulle toteen sitä, mistä he nyt minua syyttävät.
\par 14 Mutta sen minä sinulle tunnustan, että minä sitä tietä vaeltaen, jota he lahkoksi sanovat, niin palvelen isieni Jumalaa, että minä uskon kaiken, mitä on kirjoitettuna laissa ja profeetoissa,
\par 15 ja pidän sen toivon Jumalaan, että on oleva ylösnousemus, jota nämä itsekin odottavat, sekä vanhurskasten että vääräin.
\par 16 Sentähden minä myös ahkeroitsen, että minulla aina olisi loukkaamaton omatunto Jumalan ja ihmisten edessä.
\par 17 Niin minä nyt useampien vuosien kuluttua tulin tuomaan almuja kansalleni ja toimittamaan uhreja.
\par 18 Näitä toimittaessani muutamat Aasiasta tulleet juutalaiset tapasivat minut puhdistautuneena pyhäkössä, eikä ollut mitään väentungosta tai meteliä;
\par 19 heidän tulisi nyt olla saapuvilla sinun edessäsi ja syyttää, jos heillä olisi jotakin minua vastaan.
\par 20 Tai sanokoot nämä läsnäolevat, mitä rikollista he minussa huomasivat, kun minä seisoin neuvoston edessä;
\par 21 jollei siksi luettane tätä ainoata lausetta, jonka huusin seisoessani heidän keskellään: 'Kuolleitten ylösnousemuksen tähden minä tänään olen teidän tuomittavananne'."
\par 22 Mutta Feeliks, jolla oli hyvin tarkka tieto tästä tiestä, lykkäsi heidän asiansa toistaiseksi, sanoen: "Kun päällikkö Lysias tulee tänne, tutkin minä teidän asianne".
\par 23 Ja hän käski sadanpäämiehen pitää Paavalia vartioituna, mutta lievässä vankeudessa, estämättä ketään hänen omaisistaan tekemästä hänelle palvelusta.
\par 24 Muutamien päivien kuluttua Feeliks tuli vaimonsa Drusillan kanssa, joka oli juutalainen, haetti Paavalin ja kuunteli hänen puhettaan uskosta Kristukseen Jeesukseen.
\par 25 Mutta kun Paavali puhui vanhurskaudesta ja itsensähillitsemisestä ja tulevasta tuomiosta, peljästyi Feeliks ja sanoi: "Mene tällä haavaa pois, mutta kun minulle sopii, kutsutan sinut taas".
\par 26 Sen ohessa hän myös toivoi saavansa Paavalilta rahaa, jonka tähden hän useita kertoja haetti hänet luokseen ja puheli hänen kanssansa.
\par 27 Mutta kahden vuoden kuluttua Porkius Festus tuli Feeliksin sijaan; ja kun Feeliks tavoitteli juutalaisten suosiota, jätti hän Paavalin kahleisiin.

\chapter{25}

\par 1 Kun nyt Festus oli astunut maaherranvirkaan, meni hän kolmen päivän kuluttua Kesareasta ylös Jerusalemiin.
\par 2 Niin ylipapit ja juutalaisten ensimmäiset miehet ilmoittivat hänelle syyttävänsä Paavalia ja pyysivät häneltä
\par 3 ja anoivat sitä suosionosoitusta itsellensä, Paavalia vastaan, että hän haettaisi hänet Jerusalemiin; sillä he valmistivat väijytystä tappaakseen hänet tiellä.
\par 4 Mutta Festus vastasi, että Paavalia pidettiin vartioituna Kesareassa ja että hän itse aikoi piakkoin lähteä sinne.
\par 5 Ja hän lisäsi: "Tulkoot siis teidän johtomiehenne minun mukanani sinne alas, ja jos siinä miehessä on jotakin väärää, syyttäkööt häntä".
\par 6 Ja viivyttyään heidän luonansa ainoastaan kahdeksan tai kymmenen päivää hän meni alas Kesareaan. Seuraavana päivänä hän istui tuomarinistuimelle ja käski tuoda Paavalin eteensä.
\par 7 Ja kun hän oli saapunut, asettuivat ne juutalaiset, jotka olivat tulleet Jerusalemista, hänen ympärilleen ja tekivät useita ja raskaita syytöksiä, joita he eivät kuitenkaan kyenneet näyttämään toteen;
\par 8 sillä Paavali torjui syytökset ja sanoi: "Minä en ole mitään rikkonut juutalaisten lakia enkä pyhäkköä enkä keisaria vastaan".
\par 9 Niin Festus, joka tavoitteli juutalaisten suosiota, vastasi Paavalille ja sanoi: "Tahdotko lähteä Jerusalemiin ja siellä vastata näihin syytöksiin minun edessäni?"
\par 10 Mutta Paavali sanoi: "Minä seison keisarin tuomioistuimen edessä, ja sen edessä minut tuomittakoon. Juutalaisia vastaan en ole mitään rikkonut, niinkuin sinäkin aivan hyvin tiedät.
\par 11 Vaan jos muuten olen rikkonut ja tehnyt jotakin, mikä ansaitsee kuoleman, en pyri pääsemään kuolemasta; mutta jos se, mistä nämä minua syyttävät, on perätöntä, niin ei kukaan voi luovuttaa minua heille. Minä vetoan keisariin."
\par 12 Silloin Festus, neuvoteltuaan neuvoskuntansa kanssa, vastasi: "Keisariin sinä olet vedonnut, niinpä mene keisarin eteen".
\par 13 Muutamien päivien kuluttua kuningas Agrippa ja Bernike saapuivat Kesareaan tervehtimään Festusta.
\par 14 Ja kun he viipyivät siellä useampia päiviä, kertoi Festus Paavalin asian kuninkaalle ja sanoi: "Täällä on eräs mies, jonka Feeliks on jättänyt vankeuteen;
\par 15 käydessäni Jerusalemissa juutalaisten ylipapit ja vanhimmat ilmoittivat syyttävänsä häntä ja pyysivät, että hänet tuomittaisiin.
\par 16 Mutta minä vastasin heille: 'Ei ole roomalaisten tapa antaa ketään alttiiksi, ennenkuin syytetty on asetettu vastakkain syyttäjäinsä kanssa ja on saanut puolustautua syytöstä vastaan'.
\par 17 Kun he olivat kokoontuneet tänne, niin minä viivyttelemättä seuraavana päivänä istuin tuomarinistuimelle ja käskin tuoda miehen eteeni.
\par 18 Mutta kun hänen syyttäjänsä seisoivat hänen ympärillään, eivät he syyttäneet häntä mistään sellaisesta rikoksesta, kuin minä olin odottanut,
\par 19 vaan heillä oli häntä vastaan riitaa joistakin heidän uskonasioistaan ja jostakin Jeesuksesta, joka oli kuollut, mutta jonka Paavali väitti elävän.
\par 20 Ja kun olin epätietoinen, miten tällainen asia oli tutkittava, kysyin, tahtoiko hän mennä Jerusalemiin ja siellä vastata näihin syytöksiin.
\par 21 Mutta kun Paavali vetosi ja vaati, että hänet oli säilytettävä majesteetin tutkittavaksi, käskin minä vartioida häntä, kunnes lähetän hänet keisarin eteen."
\par 22 Niin Agrippa sanoi Festukselle: "Minäkin tahtoisin kuulla sitä miestä". Tämä sanoi: "Huomenna saat kuulla häntä".
\par 23 Seuraavana päivänä Agrippa ja Bernike tulivat suurella komeudella ja menivät oikeussaliin päällikköjen ja kaupungin ylhäisten miesten kanssa; ja Paavali tuotiin Festuksen käskystä sinne.
\par 24 Ja Festus sanoi: "Kuningas Agrippa ja kaikki muut, jotka meidän kanssamme olette läsnä, tässä näette sen miehen, jonka tähden koko juutalaisten joukko sekä Jerusalemissa että täällä on ahdistanut minua huutaen, ettei hänen pidä enää saaman elää.
\par 25 Minä kuitenkin huomasin, ettei hän ole tehnyt mitään, mikä ansaitsisi kuoleman; mutta kun hän itse vetosi majesteettiin, niin minä päätin lähettää hänet sinne.
\par 26 Mitään varmaa minulla ei kuitenkaan ole, mitä hänestä herralleni kirjoittaisin. Sen vuoksi tuotin hänet teidän eteenne ja varsinkin sinun eteesi, kuningas Agrippa, että minulla tutkinnon tapahduttua olisi, mitä kirjoittaa.
\par 27 Sillä mielettömältä näyttää minusta lähettää vanki, antamatta samalla tietää häntä vastaan tehtyjä syytöksiä."

\chapter{26}

\par 1 Niin Agrippa sanoi Paavalille: "Sinun on lupa puhua puolestasi". Silloin Paavali ojensi kätensä ja lausui puolustuksekseen:
\par 2 "Pidän itseäni onnellisena, kuningas Agrippa, kun sinun edessäsi tänä päivänä saan puolustautua kaikesta siitä, mistä juutalaiset minua syyttävät,
\par 3 olletikin, koska sinä tarkkaan tunnet kaikki juutalaisten tavat ja riitakysymykset. Sentähden pyydän sinua kärsivällisesti minua kuulemaan.
\par 4 Kaikki juutalaiset tuntevat minun elämäni nuoruudestani asti, koska alusta alkaen olen elänyt kansani keskuudessa ja Jerusalemissa.
\par 5 He tuntevat minut jo entuudestaan, jos tahtovat sen todistaa, että minä meidän uskontomme ankarimman lahkon mukaan olen elänyt fariseuksena.
\par 6 Ja nyt minä seison oikeuden edessä sentähden, että panen toivoni siihen lupaukseen, jonka Jumala on meidän isillemme antanut
\par 7 ja jonka meidän kaksitoista sukukuntaamme, yötä ja päivää herkeämättä palvellen Jumalaa, toivovat heille toteutuvan; tämän toivon tähden, kuningas, juutalaiset minua syyttävät.
\par 8 Miksi on teistä uskomatonta, että Jumala herättää kuolleet?
\par 9 Luulin minäkin, että minun tuli paljon taistella Jeesuksen, Nasaretilaisen, nimeä vastaan,
\par 10 ja niin minä teinkin Jerusalemissa. Paljon pyhiä minä suljin vankiloihin, saatuani ylipapeilta siihen valtuuden, ja kun heitä tapettiin, annoin minä ääneni sen puolesta.
\par 11 Ja kaikkialla synagoogissa minä usein koetin rankaisemalla pakottaa heitä herjaamaan Jeesusta, ja menin niin pitkälle vimmassani heitä vastaan, että vainosin heitä aina ulkomaan kaupunkeihin saakka.
\par 12 Kun näissä asioissa matkustin Damaskoon ylipappien valtuudella ja suostumuksella,
\par 13 näin minä, oi kuningas, tiellä keskellä päivää taivaasta valon, auringon paistetta kirkkaamman, leimahtavan minun ja matkatoverieni ympärillä,
\par 14 ja me kaaduimme kaikki maahan, ja minä kuulin äänen sanovan minulle hebreankielellä: 'Saul, Saul, miksi vainoat minua? Työläs on sinun potkia tutkainta vastaan.'
\par 15 Niin minä sanoin: 'Kuka olet, herra?' Ja Herra sanoi: 'Minä olen Jeesus, jota sinä vainoat.
\par 16 Mutta nouse ja seiso jaloillasi; sillä sitä varten minä olen sinulle ilmestynyt, että asettaisin sinut palvelijakseni ja sen todistajaksi, mitä varten sinä olet minut nähnyt, niin myös sen, mitä varten minä sinulle vastedes ilmestyn.
\par 17 Ja minä pelastan sinut sekä oman kansasi että pakanain käsistä, joitten tykö minä sinut lähetän
\par 18 avaamaan heidän silmänsä, että he kääntyisivät pimeydestä valkeuteen ja saatanan vallasta Jumalan tykö ja saisivat uskomalla minuun synnit anteeksi ja perintöosan pyhitettyjen joukossa.'
\par 19 Sentähden, kuningas Agrippa, minä en voinut olla tottelematta taivaallista näkyä,
\par 20 vaan saarnasin ensin sekä Damaskon että Jerusalemin asukkaille, ja sitten koko Juudean maalle ja pakanoille parannusta ja kääntymystä Jumalan puoleen, ja että he tekisivät parannuksen soveliaita tekoja.
\par 21 Tämän tähden juutalaiset ottivat minut kiinni pyhäkössä ja yrittivät surmata minut.
\par 22 Mutta Jumalan avulla, jota olen saanut tähän päivään asti, minä seison ja todistan sekä pienille että suurille, enkä puhu mitään muuta, kuin minkä profeetat ja Mooses ovat sanoneet tulevan tapahtumaan,
\par 23 että nimittäin Kristuksen piti kärsimän ja kuolleitten ylösnousemuksen esikoisena julistaman valkeutta sekä tälle kansalle että pakanoille."
\par 24 Mutta kun hän näin puolustautui, sanoi Festus suurella äänellä: "Sinä olet hullu, Paavali, suuri oppi hulluttaa sinut".
\par 25 Mutta Paavali sanoi: "En ole hullu, korkea-arvoinen Festus, vaan puhun totuuden ja toimen sanoja.
\par 26 Kuningas kyllä nämä tietää, jonka tähden minä puhunkin hänelle rohkeasti. Sillä minä en usko minkään näistä asioista olevan häneltä salassa; eiväthän nämä ole missään syrjäsopessa tapahtuneet.
\par 27 Uskotko, kuningas Agrippa, profeettoja? Minä tiedän, että uskot."
\par 28 Niin Agrippa sanoi Paavalille: "Vähälläpä luulet taivuttavasi minut kristityksi".
\par 29 Mutta Paavali sanoi: "Toivoisin Jumalalta, että, olipa vähällä tai paljolla, et ainoastaan sinä, vaan myös kaikki te, jotka minua tänään kuulette, tulisitte semmoisiksi, kuin minä olen, näitä kahleita lukuunottamatta".
\par 30 Niin kuningas nousi ja maaherra ja Bernike sekä ne, jotka istuivat heidän kanssansa.
\par 31 Ja mennessään he puhuivat keskenänsä sanoen: "Tämä mies ei ole tehnyt mitään, mikä ansaitsisi kuoleman tai kahleet".
\par 32 Ja Agrippa sanoi Festukselle: "Tämän miehen olisi voinut päästää irti, jos hän ei olisi vedonnut keisariin".

\chapter{27}

\par 1 Kun oli päätetty, että meidän oli purjehtiminen Italiaan, annettiin Paavali ja muutamat muut vangit erään Julius nimisen, keisarilliseen sotaväenosastoon kuuluvan sadanpäämiehen haltuun.
\par 2 Ja me astuimme adramyttiläiseen laivaan, jonka oli määrä purjehtia Aasian rannikkopaikkoihin, ja lähdimme merelle, ja seurassamme oli Aristarkus, makedonialainen Tessalonikasta.
\par 3 Seuraavana päivänä laskimme Siidoniin. Ja Julius kohteli Paavalia ystävällisesti ja salli hänen mennä ystäviensä luo hoitoa saamaan.
\par 4 Ja sieltä laskettuamme merelle purjehdimme Kypron suojaan, koska tuulet olivat vastaiset.
\par 5 Ja kun olimme merta purjehtien sivuuttaneet Kilikian ja Pamfylian, tulimme Myrraan, joka on Lykiassa.
\par 6 Siellä sadanpäämies tapasi aleksandrialaisen laivan, jonka oli määrä purjehtia Italiaan, ja siirsi meidät siihen.
\par 7 Ja monta päivää me purjehdimme hitaasti ja pääsimme vaivoin Knidon kohdalle. Ja kun tuulelta emme päässeet sinne, purjehdimme Salmonen nenitse Kreetan suojaan.
\par 8 Ja vaivoin kuljettuamme liki sen rantaa saavuimme erääseen paikkaan, jonka nimi oli Kauniit Satamat ja jonka lähellä Lasaian kaupunki oli.
\par 9 Mutta kun paljon aikaa oli kulunut ja purjehtiminen jo oli vaarallista, sillä paastonaikakin oli jo ohi, varoitti Paavali heitä
\par 10 ja sanoi: "Miehet, minä näen, että purjehtiminen käy vaivalloiseksi ja vaaralliseksi, ei ainoastaan lastille ja laivalle, vaan myös meidän hengellemme".
\par 11 Mutta sadanpäämies uskoi enemmän perämiestä ja laivanisäntää kuin Paavalin sanoja.
\par 12 Ja koska satama oli sopimaton talvehtimiseen, olivat useimmat sitä mieltä, että heidän oli sieltä lähdettävä, voidakseen ehkä päästä talvehtimaan Foiniksiin, erääseen Kreetan satamaan, joka antaa lounaaseen ja luoteeseen päin.
\par 13 Ja kun etelätuuli alkoi puhaltaa, luulivat he pääsevänsä tarkoituksensa perille, nostivat ankkurin ja kulkivat aivan likitse Kreetaa.
\par 14 Mutta ennen pitkää syöksyi saaren päällitse raju tuuli, niin sanottu koillismyrsky.
\par 15 Kun laiva ryöstäytyi sen mukaan eikä voinut nousta tuuleen, jätimme sen valtoihinsa ja jouduimme tuuliajolle.
\par 16 Ja päästyämme erään pienen, Klauda nimisen saaren suojaan me töintuskin saimme venheen korjuuseen.
\par 17 Vedettyään sen ylös he ryhtyivät varokeinoihin ja sitoivat laivan ympäri köysiä, ja kun pelkäsivät ajautuvansa Syrtteihin, laskivat he purjeet alas, ja niin he ajelehtivat.
\par 18 Mutta kun rajuilma ankarasti ahdisti meitä, heittivät he seuraavana päivänä lastia mereen,
\par 19 ja kolmantena päivänä he omin käsin viskasivat mereen laivan kaluston.
\par 20 Mutta kun ei aurinkoa eikä tähtiä näkynyt moneen päivään ja kova myrsky painoi, katosi meiltä viimein kaikki pelastumisen toivo.
\par 21 Kun oli oltu kauan syömättä, niin Paavali nousi heidän keskellään ja sanoi: "Miehet, teidän olisi pitänyt noudattaa minun neuvoani eikä lähteä Kreetasta; siten olisitte säästyneet tästä vaivasta ja vahingosta.
\par 22 Mutta nyt minä kehoitan teitä olemaan rohkealla mielellä, sillä ei yksikään teistä huku, ainoastaan laiva hukkuu.
\par 23 Sillä tänä yönä seisoi minun tykönäni sen Jumalan enkeli, jonka oma minä olen ja jota minä myös palvelen,
\par 24 ja sanoi: 'Älä pelkää, Paavali, keisarin eteen sinun pitää menemän; ja katso, Jumala on lahjoittanut sinulle kaikki, jotka sinun kanssasi purjehtivat'.
\par 25 Olkaa sentähden rohkealla mielellä, miehet; sillä minulla on se usko Jumalaan, että niin käy, kuin minulle on puhuttu.
\par 26 Mutta jollekin saarelle meidän täytyy viskautua."
\par 27 Ja kun tuli neljästoista yö meidän ajelehtiessamme Adrianmerellä, tuntui merimiehistä keskiyön aikaan, että lähestyttiin jotakin maata.
\par 28 Ja luodattuaan he huomasivat syvyyden olevan kaksikymmentä syltä, ja vähän matkaa kuljettuaan he taas luotasivat ja huomasivat syvyyden viideksitoista syleksi.
\par 29 Ja kun he pelkäsivät meidän viskautuvan karille, laskivat he laivan perästä neljä ankkuria ja odottivat ikävöiden päivän tuloa.
\par 30 Mutta merimiehet yrittivät paeta laivasta ja laskivat venheen mereen sillä tekosyyllä, että muka aikoivat keulapuolesta viedä ulos ankkureita.
\par 31 Silloin Paavali sanoi sadanpäämiehelle ja sotilaille: "Jos nuo eivät pysy laivassa, niin te ette voi pelastua".
\par 32 Silloin sotamiehet hakkasivat poikki venheen köydet ja päästivät sen menemään.
\par 33 Vähää ennen päivän tuloa Paavali kehoitti kaikkia nauttimaan ruokaa, sanoen: "Tänään olette jo neljättätoista päivää odottaneet ja olleet syömättä ettekä ole mitään ravintoa ottaneet.
\par 34 Sentähden minä kehoitan teitä nauttimaan ruokaa, sillä se on tarpeen meidän pelastuaksemme; sillä ei yhdeltäkään teistä ole hiuskarvaakaan päästä katoava."
\par 35 Tämän sanottuaan hän otti leivän ja kiitti Jumalaa kaikkien nähden, mursi ja rupesi syömään.
\par 36 Silloin kaikki tulivat rohkealle mielelle ja ottivat hekin ruokaa.
\par 37 Ja meitä oli laivassa kaikkiaan kaksisataa seitsemänkymmentä kuusi henkeä.
\par 38 Ja kun he olivat tulleet ravituiksi, kevensivät he laivaa heittämällä viljan mereen.
\par 39 Päivän tultua he eivät tunteneet maata, mutta huomasivat lahden, jossa oli sopiva ranta; siihen he päättivät, jos mahdollista, laskea laivan.
\par 40 Ja he hakkasivat ankkuriköydet poikki ja jättivät ankkurit mereen; samalla he päästivät peräsinten nuorat, nostivat keulapurjeen tuuleen ja ohjasivat rantaa kohti.
\par 41 Mutta he joutuivat riutalle ja antoivat laivan törmätä siihen; keulapuoli tarttui kiinni ja jäi liikkumattomaksi, mutta peräpuoli hajosi aaltojen voimasta.
\par 42 Niin sotamiehillä oli aikomus tappaa vangit, ettei kukaan pääsisi uimalla karkuun.
\par 43 Mutta sadanpäämies, joka tahtoi pelastaa Paavalin, esti heidät siitä aikeesta ja käski uimataitoisten ensiksi heittäytyä veteen ja lähteä maihin
\par 44 ja sitten muiden, minkä laudoilla, minkä laivankappaleilla. Ja näin kaikki pelastuivat maalle.

\chapter{28}

\par 1 Kun olimme pelastuneet, niin me sitten saimme tietää, että saaren nimi oli Melite.
\par 2 Ja sen asukkaat osoittivat meille suurta ystävällisyyttä: he sytyttivät nuotion ja ottivat meidät kaikki sen ääreen, kun oli ruvennut satamaan ja oli kylmä.
\par 3 Mutta Paavali kokosi kasan risuja, ja kun hän pani ne nuotioon, tuli kyykäärme kuumuuden tähden esiin ja kävi kiinni hänen käteensä.
\par 4 Kun asukkaat näkivät tuon elukan riippuvan kiinni hänen kädessään, sanoivat he toisilleen: "Varmaan tuo mies on murhaaja, koska kostotar ei sallinut hänen elää, vaikka hän pelastuikin merestä".
\par 5 Mutta hän pudisti elukan tuleen, eikä hänelle tullut mitään vahinkoa.
\par 6 Ja he odottivat hänen ajettuvan tai äkisti kaatuvan kuolleena maahan. Mutta kun he olivat kauan odottaneet ja näkivät, ettei hänelle mitään pahaa tapahtunut, muuttivat he mielensä ja sanoivat hänen olevan jumalan.
\par 7 Lähellä sitä paikkaa oli saaren ensimmäisellä miehellä, jonka nimi oli Publius, maatiloja. Hän otti meidät vastaan ja piti meitä ystävällisesti kolme päivää vierainansa.
\par 8 Ja Publiuksen isä makasi sairaana kuumeessa ja punataudissa; ja Paavali meni hänen luoksensa, rukoili ja pani kätensä hänen päälleen ja paransi hänet.
\par 9 Kun tämä oli tapahtunut, tulivat muutkin sairaat, mitä saarella oli, ja heidät parannettiin.
\par 10 He osoittivat meille myös monin tavoin kunniaa, ja lähtiessämme merelle he panivat mukaan, mitä tarvitsimme.
\par 11 Kolmen kuukauden kuluttua me purjehdimme sieltä aleksandrialaisessa laivassa, joka oli talvehtinut saarella ja jolla oli merkkinä Kastorin ja Polluksin kuva.
\par 12 Ja me laskimme maihin Syrakuusassa ja viivyimme siellä kolme päivää,
\par 13 ja sieltä me kierrettyämme saavuimme Reegioniin, ja kun yhden päivän perästä nousi etelätuuli, tulimme seuraavana päivänä Puteoliin.
\par 14 Siellä tapasimme veljiä, jotka pyysivät meitä viipymään heidän tykönänsä seitsemän päivää. Ja sitten me lähdimme Roomaan.
\par 15 Ja kun veljet siellä saivat kuulla meistä, tulivat he meitä vastaan Appii Forumiin ja Tres Tabernaen kohdalle saakka; ja heidät nähdessään Paavali kiitti Jumalaa ja sai rohkeutta.
\par 16 Ja kun tulimme Roomaan, sallittiin Paavalin asua erikseen häntä vartioivan sotamiehen kanssa.
\par 17 Kolmen päivän kuluttua Paavali kutsui kokoon juutalaisten ensimmäiset. Ja kun he olivat kokoontuneet, sanoi hän heille: "Miehet, veljet! Vaikka en ole mitään tehnyt kansaamme tai isiemme tapoja vastaan, annettiin minut kuitenkin Jerusalemista vankina roomalaisten käsiin.
\par 18 Ja kun he olivat minua tutkineet, tahtoivat he päästää minut irti, koska en ollut tehnyt mitään kuoleman rikosta.
\par 19 Mutta kun juutalaiset sitä vastustivat, oli minun pakko vedota keisariin; ei kuitenkaan niin, että minulla olisi mitään kannetta kansaani vastaan.
\par 20 Tästä syystä minä nyt olen kutsunut teidät, saadakseni nähdä ja puhutella teitä; sillä Israelin toivon tähden minä kannan tätä kahletta."
\par 21 Niin he sanoivat hänelle: "Emme ole saaneet kirjeitä sinusta Juudean maalta, eikä kukaan tänne saapunut veli ole ilmoittanut eikä puhunut sinusta mitään pahaa.
\par 22 Katsomme kuitenkin syytä olevan kuulla sinulta, mitä sinun mielessäsi on; sillä tästä lahkosta on meillä tiedossamme, että sitä vastaan kaikkialla kiistetään."
\par 23 Ja he määräsivät hänelle päivän, ja silloin tuli heitä vielä useampia hänen luoksensa majapaikkaan. Ja näille hän aamuvarhaisesta iltaan saakka selitti ja todisti Jumalan valtakunnasta, lähtien Mooseksen laista ja profeetoista, saadakseen heidät vakuutetuiksi Jeesuksesta.
\par 24 Niin se, mitä sanottiin, sai toiset vakuutetuiksi, mutta toiset eivät uskoneet.
\par 25 Ja kun he olivat keskenään erimielisiä, erosivat he toisistaan, Paavalin sanoessa ainoastaan nämä sanat: "Oikein on Pyhä Henki puhunut profeetta Esaiaan kautta teidän isillenne,
\par 26 sanoen: 'Mene tämän kansan luo ja sano: Kuulemalla kuulkaa älkääkä ymmärtäkö, näkemällä nähkää älkääkä käsittäkö.
\par 27 Sillä paatunut on tämän kansan sydän, ja korvillaan he työläästi kuulevat, ja silmänsä he ovat ummistaneet, että he eivät näkisi silmillään, eivät kuulisi korvillaan, eivät ymmärtäisi sydämellään eivätkä kääntyisi ja etten minä heitä parantaisi.'
\par 28 Olkoon siis teille tiettävä, että tämä Jumalan pelastussanoma on lähetetty pakanoille; ja he kuulevat sen."
\par 29 []
\par 30 Ja Paavali asui omassa vuokra-asunnossaan kaksi täyttä vuotta ja otti vastaan kaikki, jotka hänen tykönsä tulivat;
\par 31 ja hän julisti Jumalan valtakuntaa ja opetti Herran Jeesuksen Kristuksen tuntemista kaikella rohkeudella, kenenkään estämättä.


\end{document}