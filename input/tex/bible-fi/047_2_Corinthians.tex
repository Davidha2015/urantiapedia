\begin{document}

\title{Toinen kirje korinttilaisille}


\chapter{1}

\par 1 Paavali, Jumalan tahdosta Kristuksen Jeesuksen apostoli, ja veli Timoteus Korintossa olevalle Jumalan seurakunnalle ynnä kaikille pyhille koko Akaiassa.
\par 2 Armo teille ja rauha Jumalalta, meidän Isältämme, ja Herralta Jeesukselta Kristukselta!
\par 3 Kiitetty olkoon meidän Herramme Jeesuksen Kristuksen Jumala ja Isä, laupeuden Isä ja kaiken lohdutuksen Jumala,
\par 4 joka lohduttaa meitä kaikessa ahdistuksessamme, että me sillä lohdutuksella, jolla Jumala meitä itseämme lohduttaa, voisimme lohduttaa niitä, jotka kaikkinaisessa ahdistuksessa ovat.
\par 5 Sillä samoin kuin Kristuksen kärsimykset runsaina tulevat meidän osaksemme, samoin tulee meidän osaksemme myöskin lohdutus runsaana Kristuksen kautta.
\par 6 Mutta jos olemme ahdistuksessa, niin tapahtuu se teille lohdutukseksi ja pelastukseksi; jos taas saamme lohdutusta, niin tapahtuu sekin teille lohdutukseksi, ja se vaikuttaa, että te kestätte samat kärsimykset, joita mekin kärsimme; ja toivomme teistä on vahva,
\par 7 koska me tiedämme, että samoin kuin olette osalliset kärsimyksistä, samoin olette osalliset myöskin lohdutuksesta.
\par 8 Sillä me emme tahdo, veljet, pitää teitä tietämättöminä siitä ahdistuksesta, jossa me olimme Aasiassa, kuinka ylenpalttiset, yli voimiemme käyvät, meidän rasituksemme olivat, niin että jo olimme epätoivossa hengestämmekin,
\par 9 ja itse me jo luulimme olevamme kuolemaan tuomitut, ettemme luottaisi itseemme, vaan Jumalaan, joka kuolleet herättää.
\par 10 Ja hän pelasti meidät niin suuresta kuolemanvaarasta, ja yhä pelastaa, ja häneen me olemme panneet toivomme, että hän vielä vastakin pelastaa,
\par 11 kun tekin autatte meitä rukouksillanne, että monesta suusta meidän tähtemme kohoaisi runsas kiitos siitä armosta, joka on osaksemme tullut.
\par 12 Sillä meidän kerskauksemme on tämä: meidän omantuntomme todistus siitä, että me maailmassa ja varsinkin teidän luonanne olemme vaeltaneet Jumalan pyhyydessä ja puhtaudessa, emme lihallisessa viisaudessa, vaan Jumalan armossa.
\par 13 Sillä eihän siinä, mitä teille kirjoitamme, ole muuta, kuin mikä siinä on luettavana ja minkä te myös ymmärrätte; ja minä toivon teidän loppuun asti ymmärtävän
\par 14 - niinkuin meitä osaksi myös olette oppineet ymmärtämään - että me olemme teidän kerskauksenne, samoin kuin tekin meidän, Herramme Jeesuksen päivänä.
\par 15 Ja tässä luottamuksessa minä aioin ensin tulla teidän tykönne, että saisitte vielä toisenkin todistuksen minun suosiostani,
\par 16 ja sitten teidän kauttanne matkustaa Makedoniaan, ja taas Makedoniasta palata teidän tykönne ja teidän varustamananne matkata Juudeaan.
\par 17 Kun minulla siis oli tämä aikomus, en kaiketi ole menetellyt kevytmielisesti? Vai päätänkö lihan mukaan, minkä päätän, niin että puheeni on sekä "on, on", että "ei, ei"?
\par 18 Mutta Jumala sen takaa, että puheemme teille ei ole "on" ja "ei".
\par 19 Sillä Jumalan Poika, Kristus Jeesus, jota me, minä ja Silvanus ja Timoteus, olemme teidän keskellänne saarnanneet, ei tullut ollakseen "on" ja "ei", vaan hänessä tuli "on".
\par 20 Sillä niin monta kuin Jumalan lupausta on, kaikki ne ovat hänessä "on"; sentähden tulee hänen kauttaan myös niiden "amen", Jumalalle kunniaksi meidän kauttamme.
\par 21 Mutta se, joka lujittaa meidät yhdessä teidän kanssanne Kristukseen ja joka on voidellut meidät, on Jumala,
\par 22 joka myös on painanut meihin sinettinsä ja antanut Hengen vakuudeksi meidän sydämiimme.
\par 23 Mutta minä kutsun Jumalan sieluni todistajaksi, että minä teitä säästääkseni en vielä tullut Korinttoon;
\par 24 ei niin, että tahdomme vallita teidän uskoanne, vaan me yhdessä autamme teitä teidän iloonne; sillä uskossa te olette lujat.

\chapter{2}

\par 1 Olin nimittäin mielessäni päättänyt, etten tullessani teidän tykönne taas toisi murhetta mukanani.
\par 2 Sillä jos minä saatan teidät murheellisiksi, niin eihän minua voi saada iloiseksi kukaan muu kuin se, jonka minä olen murheelliseksi saattanut.
\par 3 Ja juuri sen minä kirjoitin sitä varten, etten tullessani saisi murhetta niistä, joista minun piti saada iloa, koska minulla on teihin kaikkiin se luottamus, että minun iloni on kaikkien teidän ilonne.
\par 4 Sillä suuressa sydämen ahdistuksessa ja hädässä minä kirjoitin teille monin kyynelin, en sitä varten, että te murheellisiksi tulisitte, vaan että tuntisitte sen erinomaisen rakkauden, joka minulla on teihin.
\par 5 Mutta jos eräs on tuottanut murhetta, ei hän ole tuottanut murhetta minulle, vaan teille kaikille, jossakin määrin, etten liikaa sanoisi.
\par 6 Semmoiselle riittää se rangaistus, minkä hän useimmilta on saanut;
\par 7 niin että teidän päinvastoin ennemmin tulee antaa anteeksi ja lohduttaa, ettei hän ehkä menehtyisi liian suureen murheeseen.
\par 8 Sentähden minä kehoitan teitä, että päätätte ruveta osoittamaan rakkautta häntä kohtaan;
\par 9 sillä sitä varten minä kirjoitinkin, että saisin nähdä, kuinka te kestätte koetuksen, oletteko kaikessa kuuliaiset.
\par 10 Mutta kenelle te jotakin anteeksi annatte, sille minäkin; sillä mitä minä olen anteeksi antanut - jos minulla on ollut jotakin anteeksiannettavaa - sen olen anteeksi antanut teidän tähtenne Kristuksen kasvojen edessä,
\par 11 ettei saatana pääsisi meistä voitolle; sillä hänen juonensa eivät ole meille tuntemattomat.
\par 12 Tultuani Trooaaseen julistamaan Kristuksen evankeliumia avautui minulle ovi työhön Herrassa,
\par 13 mutta minä en saanut lepoa hengessäni, kun en tavannut Tiitusta, veljeäni. Sentähden sanoin heille jäähyväiset ja lähdin Makedoniaan.
\par 14 Mutta kiitos olkoon Jumalan, joka aina kuljettaa meitä voittosaatossa Kristuksessa ja meidän kauttamme joka paikassa tuo ilmi hänen tuntemisensa tuoksun!
\par 15 Sillä me olemme Kristuksen tuoksu Jumalalle sekä pelastuvien että kadotukseen joutuvien joukossa:
\par 16 näille tosin kuoleman haju kuolemaksi, mutta noille elämän tuoksu elämäksi. Ja kuka on tällaiseen kelvollinen?
\par 17 Sillä me emme ole niinkuin nuo monet, jotka myyskentelevät Jumalan sanaa; vaan puhtaasta mielestä, niinkuin Jumalan vaikutuksesta, Jumalan edessä, me Kristuksessa puhumme.

\chapter{3}

\par 1 Alammeko taas suositella itseämme? Vai tarvinnemmeko, niinkuin muutamat, suosituskirjeitä teille tai teiltä?
\par 2 Te itse olette meidän kirjeemme, joka on sydämeemme kirjoitettu ja jonka kaikki ihmiset tuntevat ja lukevat.
\par 3 Sillä ilmeistä on, että te olette Kristuksen kirje, meidän palvelustyöllämme kirjoitettu, ei musteella, vaan elävän Jumalan Hengellä, ei kivitauluihin, vaan sydämen lihatauluihin.
\par 4 Tämmöinen luottamus meillä on Kristuksen kautta Jumalaan;
\par 5 ei niin, että meillä itsellämme olisi kykyä ajatella jotakin, ikäänkuin se tulisi meistä itsestämme, vaan se kyky, mikä meillä on, on Jumalasta,
\par 6 joka myös on tehnyt meidät kykeneviksi olemaan uuden liiton palvelijoita, ei kirjaimen, vaan Hengen; sillä kirjain kuolettaa, mutta Henki tekee eläväksi.
\par 7 Mutta jos jo kuoleman virka, joka oli kirjaimin kaiverrettu kiviin, ilmestyi kirkkaudessa, niin etteivät Israelin lapset kärsineet katsella Mooseksen kasvoja hänen kasvojensa kirkkauden tähden, joka kuitenkin oli katoavaista,
\par 8 kuinka paljoa enemmän onkaan Hengen virka oleva kirkkaudessa!
\par 9 Sillä jos kadotustuomion virka jo oli kirkkautta, niin on vanhurskauden virka vielä paljoa runsaammassa määrin kirkkautta.
\par 10 Sillä se, millä ennen oli kirkkaus, on tämän rinnalla kirkkautta vailla, tämän ylenpalttisen kirkkauden tähden.
\par 11 Jos sillä, mikä on katoavaista, oli kirkkaus, niin vielä paljoa enemmän on sillä, mikä on pysyväistä, oleva kirkkautta.
\par 12 Koska meillä siis on tämmöinen toivo, niin me olemme aivan rohkeat
\par 13 emmekä tee niinkuin Mooses, joka pani peitteen kasvoillensa, etteivät Israelin lapset näkisi sen loppua, mikä on katoavaista.
\par 14 Mutta heidän mielensä paatuivat, sillä vielä tänäkin päivänä sama peite, vanhan liiton kirjoituksia luettaessa, pysyy poisottamatta, sillä vasta Kristuksessa se katoaa.
\par 15 Vielä tänäkin päivänä, kun Moosesta luetaan, on peite heidän sydämensä päällä;
\par 16 mutta kun heidän sydämensä kääntyy Herran tykö, otetaan peite pois.
\par 17 Sillä Herra on Henki, ja missä Herran Henki on, siinä on vapaus.
\par 18 Mutta me kaikki, jotka peittämättömin kasvoin katselemme Herran kirkkautta kuin kuvastimesta, muutumme saman kuvan kaltaisiksi kirkkaudesta kirkkauteen, niinkuin muuttaa Herra, joka on Henki.

\chapter{4}

\par 1 Sentähden, kun meillä on tämä virka sen laupeuden mukaan, joka on osaksemme tullut, me emme lannistu,
\par 2 vaan olemme hyljänneet kaikki häpeälliset salatiet, niin ettemme vaella kavaluudessa emmekä väärennä Jumalan sanaa, vaan julkituomalla totuuden me suositamme itseämme jokaisen ihmisen omalletunnolle Jumalan edessä.
\par 3 Mutta jos meidän evankeliumimme on peitossa, niin se peite on niissä, jotka kadotukseen joutuvat,
\par 4 niissä uskottomissa, joiden mielet tämän maailman jumala on niin sokaissut, ettei heille loista valkeus, joka lähtee Kristuksen kirkkauden evankeliumista, hänen, joka on Jumalan kuva.
\par 5 Sillä me emme julista itseämme, vaan Kristusta Jeesusta, että hän on Herra ja me teidän palvelijanne Jeesuksen tähden.
\par 6 Sillä Jumala, joka sanoi: "Loistakoon valkeus pimeydestä", on se, joka loisti sydämiimme, että Jumalan kirkkauden tunteminen, sen kirkkauden, joka loistaa Kristuksen kasvoissa, levittäisi valoansa.
\par 7 Mutta tämä aarre on meillä saviastioissa, että tuo suunnattoman suuri voima olisi Jumalan eikä näyttäisi tulevan meistä.
\par 8 Me olemme kaikin tavoin ahdingossa, mutta emme umpikujassa, neuvottomat, mutta emme toivottomat,
\par 9 vainotut, mutta emme hyljätyt, maahan kukistetut, mutta emme tuhotut.
\par 10 Me kuljemme, aina kantaen Jeesuksen kuolemaa ruumiissamme, että Jeesuksen elämäkin tulisi meidän ruumiissamme näkyviin.
\par 11 Sillä me, jotka elämme, olemme alati annetut kuolemaan Jeesuksen tähden, että Jeesuksen elämäkin tulisi kuolevaisessa lihassamme näkyviin.
\par 12 Niinpä siis kuolema tekee työtään meissä, mutta elämä teissä.
\par 13 Mutta koska meillä on sama uskon Henki, niinkuin kirjoitettu on: "Minä uskon, sentähden minä puhun", niin mekin uskomme, ja sentähden me myös puhumme,
\par 14 tietäen, että hän, joka herätti Herran Jeesuksen, on herättävä meidätkin Jeesuksen kanssa ja asettava esiin yhdessä teidän kanssanne.
\par 15 Sillä kaikki tapahtuu teidän tähtenne, että aina enenevä armo yhä useampien kautta saisi aikaan yhä runsaampaa kiitosta Jumalan kunniaksi.
\par 16 Sentähden me emme lannistu; vaan vaikka ulkonainen ihmisemme menehtyykin, niin sisällinen kuitenkin päivä päivältä uudistuu.
\par 17 Sillä tämä hetkisen kestävä ja kevyt ahdistuksemme tuottaa meille iankaikkisen ja määrättömän kirkkauden, ylenpalttisesti,
\par 18 meille, jotka emme katso näkyväisiä, vaan näkymättömiä; sillä näkyväiset ovat ajallisia, mutta näkymättömät iankaikkisia.

\chapter{5}

\par 1 Sillä me tiedämme, että vaikka tämä meidän maallinen majamme hajotetaankin maahan, meillä on asumus Jumalalta, iankaikkinen maja taivaissa, joka ei ole käsin tehty.
\par 2 Sentähden me huokaammekin ikävöiden, että saisimme pukeutua taivaalliseen majaamme,
\par 3 sillä kun me kerran olemme siihen pukeutuneet, ei meitä enää havaita alastomiksi.
\par 4 Sillä me, jotka olemme tässä majassa, huokaamme raskautettuina, koska emme tahdo riisuutua, vaan pukeutua, että elämä nielisi sen, mikä on kuolevaista.
\par 5 Mutta se, joka on valmistanut meidät juuri tähän, on Jumala, joka on antanut meille Hengen vakuudeksi.
\par 6 Sentähden me aina olemme turvallisella mielellä ja tiedämme, että, niin kauan kuin olemme kotona tässä ruumiissamme, me olemme poissa Herrasta;
\par 7 sillä me vaellamme uskossa emmekä näkemisessä.
\par 8 Mutta me olemme turvallisella mielellä ja haluaisimme mieluummin muuttaa pois ruumiista ja päästä kotiin Herran tykö.
\par 9 Sentähden me, olimmepa kotona tai olimmepa poissa, ahkeroitsemme olla hänelle mieliksi.
\par 10 Sillä kaikkien meidän pitää ilmestymän Kristuksen tuomioistuimen eteen, että kukin saisi sen mukaan, kuin hän ruumiissa ollessaan on tehnyt, joko hyvää tai pahaa.
\par 11 Kun siis tiedämme, mitä Herran pelko on, niin me koetamme saada ihmisiä uskomaan, mutta Jumala kyllä meidät tuntee; ja minä toivon, että tekin omissatunnoissanne meidät tunnette.
\par 12 Emme nyt taas suosittele itseämme teille, vaan tahdomme antaa teille aihetta kerskata meistä, että teillä olisi mitä vastata niille, jotka kerskaavat siitä, mikä silmään näkyy, eikä siitä, mikä sydämessä on.
\par 13 Sillä jos me olemme olleet suunniltamme, niin olemme olleet Jumalan tähden; jos taas maltamme mielemme, teemme sen teidän tähtenne.
\par 14 Sillä Kristuksen rakkaus vaatii meitä, jotka olemme tulleet tähän päätökseen: yksi on kuollut kaikkien edestä, siis myös kaikki ovat kuolleet;
\par 15 ja hän on kuollut kaikkien edestä, että ne, jotka elävät, eivät enää eläisi itselleen, vaan hänelle, joka heidän edestään on kuollut ja ylösnoussut.
\par 16 Sentähden me emme tästä lähtien tunne ketään lihan mukaan; jos olemmekin tunteneet Kristuksen lihan mukaan, emme kuitenkaan nyt enää tunne.
\par 17 Siis, jos joku on Kristuksessa, niin hän on uusi luomus; se, mikä on vanhaa, on kadonnut, katso, uusi on sijaan tullut.
\par 18 Mutta kaikki on Jumalasta, joka on sovittanut meidät itsensä kanssa Kristuksen kautta ja antanut meille sovituksen viran.
\par 19 Sillä Jumala oli Kristuksessa ja sovitti maailman itsensä kanssa eikä lukenut heille heidän rikkomuksiaan, ja hän uskoi meille sovituksen sanan.
\par 20 Kristuksen puolesta me siis olemme lähettiläinä, ja Jumala kehoittaa meidän kauttamme. Me pyydämme Kristuksen puolesta: antakaa sovittaa itsenne Jumalan kanssa.
\par 21 Sen, joka ei synnistä tiennyt, hän meidän tähtemme teki synniksi, että me hänessä tulisimme Jumalan vanhurskaudeksi.

\chapter{6}

\par 1 Hänen työtovereinaan me myös kehoitamme teitä vastaanottamaan Jumalan armon niin, ettei se jää turhaksi.
\par 2 Sillä hän sanoo: "Otollisella ajalla minä olen sinua kuullut ja pelastuksen päivänä sinua auttanut". Katso, nyt on otollinen aika, katso, nyt on pelastuksen päivä.
\par 3 Me emme missään kohden anna aihetta pahennukseen, ettei virkaamme moitittaisi,
\par 4 vaan kaikessa me osoittaudumme Jumalan palvelijoiksi: suuressa kärsivällisyydessä, vaivoissa, hädissä, ahdistuksissa,
\par 5 ruoskittaessa, vankeudessa, meteleissä, vaivannäöissä, valvomisissa, paastoissa;
\par 6 puhtaudessa, tiedossa, pitkämielisyydessä, ystävällisyydessä, Pyhässä Hengessä, vilpittömässä rakkaudessa,
\par 7 totuuden sanassa, Jumalan voimassa, vanhurskauden sota-aseet oikeassa kädessä ja vasemmassa;
\par 8 kunniassa ja häpeässä, pahassa maineessa ja hyvässä, villitsijöinä ja kuitenkin totta puhuvina,
\par 9 tuntemattomina ja kuitenkin hyvin tunnettuina; kuolemaisillamme, ja katso, me elämme, kuritettuina emmekä kuitenkaan tapettuina,
\par 10 murheellisina, mutta aina iloisina, köyhinä, mutta kuitenkin monia rikkaiksi tekevinä, mitään omistamatta, mutta kuitenkin omistaen kaiken.
\par 11 Suumme on auennut puhumaan teille, korinttolaiset, sydämemme on avartunut.
\par 12 Ei ole teillä ahdasta meidän sydämessämme, mutta ahdas on teidän oma sydämenne.
\par 13 Antakaa verta verrasta - puhun kuin lapsilleni - avartukaa tekin.
\par 14 Älkää antautuko kantamaan vierasta iestä yhdessä uskottomien kanssa; sillä mitä yhteistä on vanhurskaudella ja vääryydellä? Tai mitä yhteyttä on valkeudella ja pimeydellä?
\par 15 Ja miten sopivat yhteen Kristus ja Beliar? Tai mitä yhteistä osaa uskovaisella on uskottoman kanssa?
\par 16 Ja miten soveltuvat yhteen Jumalan temppeli ja epäjumalat? Sillä me olemme elävän Jumalan temppeli, niinkuin Jumala on sanonut: "Minä olen heissä asuva ja vaeltava heidän keskellään ja oleva heidän Jumalansa, ja he ovat minun kansani".
\par 17 Sentähden: "Lähtekää pois heidän keskeltänsä ja erotkaa heistä, sanoo Herra, älkääkä saastaiseen koskeko; niin minä otan teidät huostaani
\par 18 ja olen teidän Isänne, ja te tulette minun pojikseni ja tyttärikseni, sanoo Herra, Kaikkivaltias".

\chapter{7}

\par 1 Koska meillä siis on nämä lupaukset, rakkaani, niin puhdistautukaamme kaikesta lihan ja hengen saastutuksesta, saattaen pyhityksemme täydelliseksi Jumalan pelossa.
\par 2 Antakaa meille tilaa sydämessänne. Emme ole tehneet kenellekään vääryyttä, emme ole olleet kenellekään turmioksi, emme kenellekään vahinkoa tuottaneet.
\par 3 En sano tätä tuomitakseni teitä, sillä olenhan jo sanonut, että te olette meidän sydämessämme, yhdessä kuollaksemme ja yhdessä elääksemme.
\par 4 Paljon minulla on luottamusta teihin, paljon minulla on kerskaamista teistä; olen täynnä lohdutusta, minulla on ylenpalttinen ilo kaikessa ahdingossamme.
\par 5 Sillä ei Makedoniaan tultuammekaan lihamme saanut mitään rauhaa, vaan me olimme kaikin tavoin ahdistetut: ulkoapäin taisteluja, sisältäpäin pelkoa.
\par 6 Mutta Jumala, joka masentuneita lohduttaa, lohdutti meitä Tiituksen tulolla,
\par 7 eikä ainoastaan hänen tulollansa, vaan myöskin sillä lohdutuksella, jonka hän oli teistä saanut, sillä hän on kertonut meille teidän ikävöimisestänne, valittelustanne ja innostanne minun hyväkseni, niin että minä iloitsin vielä enemmän.
\par 8 Sillä vaikka murehutinkin teitä kirjeelläni, en sitä kadu, ja jos kaduinkin, niin minä - kun näen, että tuo kirje on, vaikkapa vain vähäksi aikaa, murehuttanut teitä -
\par 9 nyt iloitsen, en siitä, että tulitte murheellisiksi, vaan siitä, että murheenne oli teille parannukseksi; sillä te tulitte murheellisiksi Jumalan mielen mukaan, ettei teillä olisi mitään vahinkoa meistä.
\par 10 Sillä Jumalan mielen mukainen murhe saa aikaan parannuksen, joka koituu pelastukseksi ja jota ei kukaan kadu; mutta maailman murhe tuottaa kuoleman.
\par 11 Sillä katsokaa, kuinka suurta intoa juuri tuo Jumalan mielen mukainen murehtimisenne on saanut teissä aikaan, mitä puolustautumista, mitä paheksumista, mitä pelkoa, ikävöimistä, kiivautta, mitä kurittamista! Olette kaikin tavoin osoittaneet olevanne puhtaat tässä asiassa.
\par 12 Vaikka minä siis kirjoitinkin teille, en kirjoittanut vääryyttä tehneen enkä vääryyttä kärsineen vuoksi, vaan sentähden, että teidän intonne meidän hyväksemme tulisi julki teidän keskuudessanne Jumalan edessä.
\par 13 Sentähden me olemme nyt lohdutetut. Mutta tämän lohdutuksemme ohessa on meitä vielä paljoa enemmän ilahuttanut Tiituksen ilo, sillä hänen henkensä on saanut virvoitusta teiltä kaikilta.
\par 14 Sillä jos olenkin jossakin kohden teitä hänelle kehunut, en ole joutunut häpeään, vaan niinkuin kaikki, mitä olemme teille puhuneet, on totta, niin on myös se, mistä olemme teitä Tiitukselle kehuneet, osoittautunut todeksi.
\par 15 Ja hänen sydämensä heltyy yhä enemmän teitä kohtaan, kun hän muistelee kaikkien teidän kuuliaisuuttanne, kuinka te pelolla ja vavistuksella otitte hänet vastaan.
\par 16 Minä iloitsen, että kaikessa voin olla teistä turvallisella mielellä.

\chapter{8}

\par 1 Mutta me saatamme teidän tietoonne, veljet, mitä Jumalan armo on vaikuttanut Makedonian seurakunnissa:
\par 2 että, vaikka he olivatkin monessa ahdistuksen koetuksessa, niin oli heidän ilonsa heidän suuressa köyhyydessäänkin niin ylenpalttinen, että he alttiisti antoivat runsaita lahjoja.
\par 3 Sillä voimiensa mukaan, sen minä todistan, jopa yli voimiensakin he antoivat omasta halustansa,
\par 4 paljolla pyytämisellä anoen meiltä sitä suosiota, että pääsisivät osallisiksi pyhien avustamiseen;
\par 5 eivätkä he vain tehneet, niinkuin me olimme toivoneet, vaan antoivat itsensäkin, ennen kaikkea Herralle ja sitten meille, Jumalan tahdosta,
\par 6 niin että me kehoitimme Tiitusta, niinkuin hän jo oli alkanut, saattamaan teidän keskuudessanne päätökseen tämänkin rakkaudentyön.
\par 7 Mutta niinkuin teillä on ylenpalttisesti kaikkea: uskoa, sanaa, tietoa, kaikkinaista intoa ja meistä teihin tullutta rakkautta, niin olkaa ylenpalttiset tässäkin rakkaudentyössä.
\par 8 En sano tätä käskien, vaan viittaamalla muiden intoon minä tahdon koetella teidänkin rakkautenne vilpittömyyttä.
\par 9 Sillä te tunnette meidän Herramme Jeesuksen Kristuksen armon, että hän, vaikka oli rikas, tuli teidän tähtenne köyhäksi, että te hänen köyhyydestään rikastuisitte.
\par 10 Minä annan vain neuvon tässä asiassa; sillä se on hyödyksi teille, jotka jo viime vuonna olitte ensimmäiset, ette ainoastaan tekemässä, vaan myös tahtomassa.
\par 11 Täyttäkää nyt siis tekonne, niin että, yhtä alttiisti kuin olitte sen päättäneet, sen myös täyttäisitte, varojenne mukaan.
\par 12 Sillä jos on alttiutta, niin se on otollinen sen mukaan, kuin on varoja, eikä sen mukaan, kuin niitä ei ole.
\par 13 Sillä ei ole tarkoitus, että muilla olisi huojennus, teillä rasitus, vaan tasauksen vuoksi tulkoon tätä nykyä teidän yltäkylläisyytenne heidän puutteensa hyväksi,
\par 14 että heidänkin yltäkylläisyytensä tulisi teidän puutteenne hyväksi, niin että syntyisi tasaus,
\par 15 niinkuin kirjoitettu on: "Joka oli paljon koonnut, sille ei jäänyt liikaa, ja joka oli koonnut vähän, siltä ei mitään puuttunut".
\par 16 Mutta kiitos Jumalalle, joka antaa Tiituksen sydämeen saman innon teidän hyväksenne!
\par 17 Sillä hän otti varteen minun kehoitukseni, innostuipa niinkin, että lähtee omasta halustaan teidän tykönne.
\par 18 Ja me lähetämme hänen kanssaan veljen, jota evankeliumin julistamisesta kiitetään kaikissa seurakunnissa
\par 19 ja jonka seurakunnat vielä sen lisäksi myös ovat valinneet matkatoveriksemme viemään tätä rakkauden lahjaa, joka on meidän toimitettavanamme itse Herran kunniaksi ja oman alttiutemme osoitukseksi.
\par 20 Näin me teemme, ettei kukaan pääsisi moittimaan meitä mistään, mikä koskee tätä runsasta avustusta, joka on meidän toimitettavanamme.
\par 21 Sillä me ahkeroitsemme sitä, mikä on hyvää ei ainoastaan Herran, vaan myös ihmisten edessä.
\par 22 Ja näiden kanssa me lähetämme vielä erään veljemme, jonka intoa usein ja monessa asiassa olemme koetelleet ja joka nyt on entistä paljon innokkaampi, koska hänellä on niin suuri luottamus teihin.
\par 23 Jos siis on puhe Tiituksesta, niin hän on minun toverini ja työkumppanini teidän hyväksenne; meidän veljemme taas ovat seurakuntien lähettiläitä, ovat Kristuksen kunnia.
\par 24 Kun te siis osoitatte heille rakkauttanne ja näytätte todeksi sen, mistä me olemme teitä kehuneet, niin teette sen seurakuntien edessä.

\chapter{9}

\par 1 Pyhien avustamisesta minun tosin on tarpeetonta kirjoittaa teille;
\par 2 sillä minä tunnen teidän alttiutenne, ja siitä minä kehun teitä makedonialaisille, että näet Akaia on ollut valmiina menneestä vuodesta alkaen, ja niin on teidän intonne saanut sangen monta innostumaan.
\par 3 Lähetän nyt kuitenkin nämä veljet, ettei kerskaamisemme teistä tässä kohden näyttäytyisi tyhjäksi, vaan että olisitte valmiit, niinkuin olen sanonut teidän olevan;
\par 4 muutoin, jos makedonialaisia tulee minun kanssani ja he tapaavat teidät valmistumattomina, me - ettemme sanoisi te - ehkä joutuisimme häpeään tässä luottamuksessamme.
\par 5 Olen siis katsonut tarpeelliseksi kehoittaa veljiä edeltäpäin lähtemään teidän tykönne ja toimittamaan valmiiksi ennen lupaamanne runsaan lahjan, niin että se olisi valmis runsaana eikä kitsaana.
\par 6 Huomatkaa tämä: joka niukasti kylvää, se myös niukasti niittää, ja joka runsaasti kylvää, se myös runsaasti niittää.
\par 7 Antakoon kukin, niinkuin hänen sydämensä vaatii, ei surkeillen eikä pakosta; sillä iloista antajaa Jumala rakastaa.
\par 8 Ja Jumala on voimallinen antamaan teille ylenpalttisesti kaikkea armoa, että teillä kaikessa aina olisi kaikkea riittävästi, voidaksenne ylenpalttisesti tehdä kaikkinaista hyvää;
\par 9 niinkuin kirjoitettu on: "Hän sirottelee, hän antaa köyhille, hänen vanhurskautensa pysyy iankaikkisesti".
\par 10 Ja hän, joka antaa siemenen kylväjälle ja leivän ruuaksi, on antava teillekin ja enentävä kylvönne ja kasvattava teidän vanhurskautenne hedelmät,
\par 11 niin että te kaikessa vaurastuen voitte vilpittömästi harjoittaa kaikkinaista anteliaisuutta, joka meidän kauttamme saa aikaan kiitosta Jumalalle.
\par 12 Sillä tämä avustamispalvelus ei ainoastaan poista pyhien puutteita, vaan käy vieläkin hedelmällisemmäksi Jumalalle annettujen monien kiitosten kautta,
\par 13 kun he, tästä teidän palveluksestanne huomattuaan, kuinka taattu teidän mielenne on, ylistävät Jumalaa siitä, että te näin alistuvaisesti tunnustaudutte Kristuksen evankeliumiin ja näin vilpittömästi olette ruvenneet yhteyteen heidän kanssaan ja kaikkien kanssa.
\par 14 Ja hekin rukoilevat teidän edestänne ja ikävöivät teitä sen ylen runsaan Jumalan armon tähden, joka on teidän osaksenne tullut.
\par 15 Kiitos Jumalalle hänen sanomattomasta lahjastaan!

\chapter{10}

\par 1 Minä, Paavali, itse kehoitan teitä Kristuksen laupeuden ja lempeyden kautta, minä, joka olen muka nöyrä kasvotusten teidän kanssanne, mutta poissa ollessani rohkea teitä kohtaan,
\par 2 ja pyydän, ettei minun, kun tulen teidän tykönne, tarvitsisi käyttää sitä rohkeutta, millä aion luottavaisesti uskaltaa käydä eräiden kimppuun, jotka ajattelevat meistä, aivan kuin vaeltaisimme lihan mukaan.
\par 3 Vaikka me vaellammekin lihassa, emme kuitenkaan lihan mukaan sodi;
\par 4 sillä meidän sota-aseemme eivät ole lihalliset, vaan ne ovat voimalliset Jumalan edessä hajottamaan maahan linnoituksia.
\par 5 Me hajotamme maahan järjen päätelmät ja jokaisen varustuksen, joka nostetaan Jumalan tuntemista vastaan, ja vangitsemme jokaisen ajatuksen kuuliaiseksi Kristukselle
\par 6 ja olemme valmiit rankaisemaan kaikkea tottelemattomuutta, kunhan te ensin olette täysin kuuliaisiksi tulleet.
\par 7 Nähkää, mitä silmäin edessä on. Jos joku on mielessään varma siitä, että hän on Kristuksen oma, ajatelkoon hän edelleen mielessään, että samoin kuin hän on Kristuksen, samoin olemme mekin.
\par 8 Ja vaikka minä jonkun verran enemmänkin kerskaisin siitä vallastamme, jonka Herra on antanut teitä rakentaaksemme eikä kukistaaksemme, en ole häpeään joutuva.
\par 9 Tämän minä sanon, ettei näyttäisi siltä, kuin peloittelisin teitä kirjeilläni.
\par 10 Sillä hänen kirjeensä ovat, sanotaan, kyllä mahtavat ja pontevat, mutta ruumiillisesti läsnäollessaan hän on heikko, eikä hänen puheensa ole minkään arvoista.
\par 11 Joka niin sanoo, ajatelkoon, että samaa, mitä me poissaolevina olemme kirjeissämme sanoissa, samaa me myös olemme läsnäolevina teoissa.
\par 12 Sillä me emme rohkene lukeutua emmekä verrata itseämme eräisiin, jotka itseänsä suosittelevat; mutta he ovat ymmärtämättömiä, kun mittaavat itsensä omalla itsellään ja vertailevat itseään omaan itseensä.
\par 13 Me taas emme rupea kerskaamaan yli määrän, vaan ainoastaan sen määrätyn vaikutusalan mukaan, minkä Jumala on asettanut meille määräksi, ulottuaksemme teihinkin asti.
\par 14 Sillä me emme kurota itseämme liiaksi, ikäänkuin emme teihin ulottuisikaan, sillä olemmehan ehtineet Kristuksen evankeliumin julistamisessa teihinkin asti.
\par 15 Emme kerskaa yli määrän, emme muiden vaivannäöistä, mutta meillä on se toivo, että teidän uskonne lisääntyessä me oman vaikutusalamme mukaan kasvamme teidän keskuudessanne niin suuriksi,
\par 16 että saamme viedä evankeliumin myöskin tuolla puolen teitä oleviin maihin - tahtomatta kerskata siitä, mikä jo on valmista toisten vaikutusalalla.
\par 17 Mutta joka kerskaa, hänen kerskauksenaan olkoon Herra.
\par 18 Sillä ei se ole koetuksen kestävä, joka itse itseään suosittelee, vaan se, jota Herra suosittelee.

\chapter{11}

\par 1 Jospa kärsisitte minulta hiukan mielettömyyttäkin! Ja kyllähän te minua kärsittekin.
\par 2 Sillä minä kiivailen teidän puolestanne Jumalan kiivaudella; minähän olen kihlannut teidät miehelle, yhdelle ainoalle, asettaakseni Kristuksen eteen puhtaan neitsyen.
\par 3 Mutta minä pelkään, että niinkuin käärme kavaluudellaan petti Eevan, niin teidän mielenne ehkä turmeltuu pois vilpittömyydestä ja puhtaudesta, joka teissä on Kristusta kohtaan.
\par 4 Sillä jos joku tulee ja saarnaa jotakin toista Jeesusta kuin sitä, jota me olemme saarnanneet, tai jos te saatte toisen hengen, kuin minkä olette saaneet, tai toisen evankeliumin, kuin minkä olette vastaanottaneet, niin sen te hyvin kärsitte.
\par 5 Mutta minä en katso itseäni missään suhteessa noita isoisia apostoleja huonommaksi.
\par 6 Jos olenkin oppimaton puheessa, en kuitenkaan tiedossa; olemmehan tuoneet sen teille kaikin tavoin ilmi kaikissa asioissa.
\par 7 Vai olenko tehnyt syntiä siinä, että - alentaessani itseni, jotta te ylenisitte - olen ilmaiseksi julistanut teille Jumalan evankeliumia?
\par 8 Muita seurakuntia minä riistin, ottaessani heiltä palkkaa palvellakseni teitä. Kun olin teidän luonanne ja kärsin puutetta, en rasittanut ketään.
\par 9 Sillä mitä minulta puuttui, sen täyttivät veljet, jotka tulivat Makedoniasta; ja kaikessa minä varoin olemasta teille rasitukseksi, ja olen vastakin varova.
\par 10 Niin totta kuin Kristuksen totuus on minussa, ei tätä kerskausta minulta riistetä Akaian maanäärissä.
\par 11 Minkätähden? Senkötähden, etten muka rakasta teitä? Jumala tietää sen.
\par 12 Mutta mitä minä teen, sen olen vastakin tekevä, riistääkseni aiheen niiltä, jotka aihetta etsivät, että heidät siinä, missä kerskaavat, havaittaisiin samankaltaisiksi kuin mekin.
\par 13 Sillä semmoiset ovat valheapostoleja, petollisia työntekijöitä, jotka tekeytyvät Kristuksen apostoleiksi.
\par 14 Eikä ihme; sillä itse saatana tekeytyy valkeuden enkeliksi.
\par 15 Ei ole siis paljon, jos hänen palvelijansakin tekeytyvät vanhurskauden palvelijoiksi, mutta heidän loppunsa on oleva heidän tekojensa mukainen.
\par 16 Vielä minä sanon: älköön kukaan luulko minua mielettömäksi; mutta vaikka olisinkin, ottakaa minut mieletönnäkin vastaan, että minäkin saisin hiukan kerskata.
\par 17 Mitä nyt puhun, kun näin suurella luottamuksella kerskaan, sitä en puhu Herran mielen mukaan, vaan niinkuin mieletön.
\par 18 Koska niin monet kerskaavat lihan mukaan, niin kerskaan minäkin.
\par 19 Tehän hyvin suvaitsette mielettömiä, kun itse olette niin mieleviä.
\par 20 Tehän suvaitsette, että joku teidät orjuuttaa, että joku teidät syö puhtaaksi, että joku teidät saa saaliiksensa, että joku itsensä korottaa, että joku lyö teitä kasvoihin.
\par 21 Häpeäkseni sanon: tähän me kyllä olemme olleet liian heikkoja. Mutta minkä joku toinen uskaltaa - puhun kuin mieletön - sen uskallan minäkin.
\par 22 He ovat hebrealaisia; minä myös. He ovat israelilaisia; minä myös. He ovat Aabrahamin siementä; minä myös.
\par 23 He ovat Kristuksen palvelijoita - puhun kuin mieltä vailla - minä vielä enemmän. Olen nähnyt vaivaa enemmän, olen ollut useammin vankeudessa, minua on ruoskittu ylen paljon, olen monta kertaa ollut kuoleman vaarassa.
\par 24 Juutalaisilta olen viidesti saanut neljäkymmentä lyöntiä, yhtä vaille;
\par 25 kolmesti olen saanut raippoja, kerran minua kivitettiin, kolmesti olen joutunut haaksirikkoon, vuorokauden olen meressä ajelehtinut;
\par 26 olen usein ollut matkoilla, vaaroissa virtojen vesillä, vaaroissa rosvojen keskellä, vaaroissa heimoni puolelta, vaaroissa pakanain puolelta, vaaroissa kaupungeissa, vaaroissa erämaassa, vaaroissa merellä, vaaroissa valheveljien keskellä;
\par 27 ollut työssä ja vaivassa; paljon valvonut, kärsinyt nälkää ja janoa, paljon paastonnut, kärsinyt vilua ja alastomuutta.
\par 28 Ja kaiken muun lisäksi jokapäiväistä tunkeilua luonani, huolta kaikista seurakunnista.
\par 29 Kuka on heikko, etten minäkin olisi heikko? Kuka lankeaa, ettei se minua polttaisi?
\par 30 Jos minun kerskata täytyy, niin kerskaan heikkoudestani.
\par 31 Herran Jeesuksen Jumala ja Isä, joka on ylistetty iankaikkisesti, tietää, etten valhettele.
\par 32 Damaskossa kuningas Aretaan käskynhaltija vartioi damaskolaisten kaupunkia ottaaksensa minut kiinni,
\par 33 ja muurin ikkuna-aukosta minut laskettiin korissa maahan, ja niin minä pääsin hänen käsistänsä.

\chapter{12}

\par 1 Minun täytyy kerskata; se tosin ei ole hyödyllistä, mutta minä siirryn nyt näkyihin ja Herran ilmestyksiin.
\par 2 Tunnen miehen, joka on Kristuksessa: neljätoista vuotta sitten hänet temmattiin kolmanteen taivaaseen - oliko hän ruumiissaan, en tiedä, vai poissa ruumiista, en tiedä, Jumala sen tietää.
\par 3 Ja minä tiedän, että tämä mies - oliko hän ruumiissaan vai poissa ruumiista, en tiedä, Jumala sen tietää -
\par 4 temmattiin paratiisiin ja kuuli sanomattomia sanoja, joita ihmisen ei ole lupa puhua.
\par 5 Tästä miehestä minä kerskaan, mutta itsestäni en kerskaa, paitsi heikkoudestani.
\par 6 Sillä jos tahtoisinkin kerskata, en olisi mieletön, sillä minä puhuisin totta; mutta minä pidättäydyn siitä, ettei kukaan ajattelisi minusta enempää, kuin mitä näkee minun olevan tai mitä hän minusta kuulee.
\par 7 Ja etten niin erinomaisten ilmestysten tähden ylpeilisi, on minulle annettu lihaani pistin, saatanan enkeli, rusikoimaan minua, etten ylpeilisi.
\par 8 Tämän tähden olen kolmesti rukoillut Herraa, että se erkanisi minusta.
\par 9 Ja hän sanoi minulle: "Minun armossani on sinulle kyllin; sillä minun voimani tulee täydelliseksi heikkoudessa". Sentähden minä mieluimmin kerskaan heikkoudestani, että Kristuksen voima asettuisi minuun asumaan.
\par 10 Sentähden minä olen mielistynyt heikkouteen, pahoinpitelyihin, hätään, vainoihin, ahdistuksiin, Kristuksen tähden; sillä kun olen heikko, silloin minä olen väkevä.
\par 11 Olen joutunut mielettömyyksiin; te olette minut siihen pakottaneet. Minunhan olisi pitänyt saada suositusta teiltä, sillä en ole missään ollut noita isoisia apostoleja huonompi, vaikka en olekaan mitään.
\par 12 Onhan apostolin tunnusteot teidän keskuudessanne tehty kaikella kestävyydellä, tunnusmerkeillä ja ihmeillä ja voimateoilla.
\par 13 Sillä missä muussa te olette jääneet muita seurakuntia vähemmälle kuin siinä, etten minä puolestani ole rasittanut teitä? Antakaa minulle anteeksi tämä vääryys.
\par 14 Katso, kolmannen kerran minä nyt olen valmis tulemaan teidän tykönne, enkä ole oleva teille rasitukseksi; sillä minä en etsi teidän omaanne, vaan teitä itseänne. Eiväthän lapset ole velvolliset kokoamaan tavaraa vanhemmilleen, vaan vanhemmat lapsilleen.
\par 15 Ja minä olen mielelläni uhraava kaikki, uhraava itsenikin, teidän sielujenne hyväksi. Senkötähden, että teitä näin suuresti rakastan, minä saan teiltä vähemmän vastarakkautta?
\par 16 Olkoonpa niin, etten minä ole teitä rasittanut; mutta entä jos olen viekas ja olen kavaluudella kietonut teidät pauloihini!
\par 17 Olenkohan minä kenenkään kautta, joita olen luoksenne lähettänyt, pyrkinyt teistä hyötymään?
\par 18 Kehoitin Tiitusta menemään ja lähetin veljen hänen kanssaan; ei kai Tiitus ole pyrkinyt teistä hyötymään? Emmekö ole vaeltaneet samassa hengessä? Emmekö samoja jälkiä?
\par 19 Olette kai jo kauan sitten luulleet, että me puolustaudumme teidän edessänne. Jumalan edessä, Kristuksessa me puhumme; mutta kaikki teille rakennukseksi, rakkaani.
\par 20 Sillä minä pelkään, että tullessani ehkä en tapaa teitä semmoisina, kuin tahdon, ja että te tapaatte minut semmoisena, kuin te ette tahdo. Minä pelkään, että teidän keskuudessanne ehkä on riitaa, kateutta, vihaa, juonia, panetteluja, juoruja, pöyhkeilyä, epäjärjestyksiä;
\par 21 ja että, kun tulen, Jumalani on taas nöyryyttävä minua teidän tykönänne, ja että joudun suremaan monen tähden, jotka ennen ovat synnissä eläneet eivätkä ole katuneet sitä saastaisuutta ja haureutta ja irstautta, jota ovat harjoittaneet.

\chapter{13}

\par 1 Kolmannen kerran minä nyt tulen teidän tykönne. Kahden tai kolmen todistajan sanalla on jokainen asia vahvistettava.
\par 2 Olen edeltäpäin sanonut ja sanon edeltäpäin niille, jotka ennen ovat syntiä tehneet, ja kaikille muille - niinkuin silloin sanoin, kun olin toista kertaa tykönänne, samoin nytkin, kun olen poissa - etten, kun taas tulen, ole teitä säästävä,
\par 3 koska te etsitte todistetta siitä, että minussa puhuu Kristus, joka ei ole heikko teitä kohtaan, vaan on teissä voimallinen.
\par 4 Sillä vaikka hänet ristiinnaulittiin, kun hän oli heikko, elää hän kuitenkin Jumalan voimasta; olemmehan mekin hänessä heikot, mutta me elämme hänen kanssaan Jumalan voimasta teitä kohtaan.
\par 5 Koetelkaa itseänne, oletteko uskossa; tutkikaa itseänne. Vai ettekö tunne itseänne, että Jeesus Kristus on teissä? Ellei, niin ette kestä koetusta.
\par 6 Minä toivon teidän tulevan tuntemaan, että me emme ole niitä, jotka eivät koetusta kestä.
\par 7 Mutta me rukoilemme Jumalaa, ettette tekisi mitään pahaa, emme sitä varten, että näyttäisi siltä, kuin me olisimme koetuksen kestäviä, vaan että te tekisitte hyvää ja me olisimme ikäänkuin ne, jotka eivät koetusta kestä.
\par 8 Sillä me emme voi mitään totuutta vastaan, vaan totuuden puolesta.
\par 9 Sillä me iloitsemme, kun me olemme heikot, mutta te olette voimalliset; sitä me rukoilemmekin, että te täydellisiksi tulisitte.
\par 10 Sentähden minä kirjoitan tämän poissa ollessani, ettei minun teidän luonanne ollessani tarvitsisi käyttää ankaruutta sen vallan nojalla, minkä Herra on minulle antanut rakentamiseksi, ei kukistamiseksi.
\par 11 Lopuksi, veljet, iloitkaa, tulkaa täydellisiksi, ottakaa vastaan kehoituksia, olkaa yhtä mieltä, eläkää sovussa, niin rakkauden ja rauhan Jumala on oleva teidän kanssanne.
\par 12 Tervehtikää toisianne pyhällä suunannolla.
\par 13 Kaikki pyhät lähettävät teille tervehdyksen.
\par 14 Herran Jeesuksen Kristuksen armo ja Jumalan rakkaus ja Pyhän Hengen osallisuus olkoon kaikkien teidän kanssanne.


\end{document}