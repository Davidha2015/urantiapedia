\begin{document}

\title{Kirje galatalaisille}


\chapter{1}

\par 1 Paavali, apostoli, virkansa saanut, ei ihmisiltä eikä ihmisen kautta, vaan Jeesuksen Kristuksen kautta ja Isän Jumalan, joka on hänet kuolleista herättänyt,
\par 2 ja kaikki veljet, jotka ovat minun kanssani, Galatian seurakunnille.
\par 3 Armo teille ja rauha Jumalalta, meidän Isältämme, ja Herralta Jeesukselta Kristukselta,
\par 4 joka antoi itsensä alttiiksi meidän syntiemme tähden, pelastaaksensa meidät nykyisestä pahasta maailmanajasta meidän Jumalamme ja Isämme tahdon mukaan!
\par 5 Hänen olkoon kunnia aina ja iankaikkisesti! Amen.
\par 6 Minua kummastuttaa, että te niin äkkiä käännytte hänestä, joka on kutsunut teidät Kristuksen armossa, pois toisenlaiseen evankeliumiin,
\par 7 joka kuitenkaan ei ole mikään toinen; on vain eräitä, jotka hämmentävät teitä ja tahtovat vääristellä Kristuksen evankeliumin.
\par 8 Mutta vaikka me, tai vaikka enkeli taivaasta julistaisi teille evankeliumia, joka on vastoin sitä, minkä me olemme teille julistaneet, hän olkoon kirottu.
\par 9 Niinkuin ennenkin olemme sanoneet, niin sanon nytkin taas: jos joku julistaa teille evankeliumia, joka on vastoin sitä, minkä te olette saaneet, hän olkoon kirottu.
\par 10 Ihmistenkö suosiota minä nyt etsin vai Jumalan? Tai ihmisillekö pyydän olla mieliksi? Jos minä vielä tahtoisin olla ihmisille mieliksi, en olisi Kristuksen palvelija.
\par 11 Sillä minä teen teille tiettäväksi, veljet, että minun julistamani evankeliumi ei ole ihmisten mukaista;
\par 12 enkä minä olekaan sitä ihmisiltä saanut, eikä sitä ole minulle opetettu, vaan Jeesus Kristus on sen minulle ilmoittanut.
\par 13 Olettehan kuulleet minun entisestä vaelluksestani juutalaisuudessa, että minä ylenmäärin vainosin Jumalan seurakuntaa ja sitä hävitin
\par 14 ja että edistyin juutalaisuudessa pitemmälle kuin monet samanikäiset heimossani ja ylen innokkaasti kiivailin isieni perinnäissääntöjen puolesta.
\par 15 Mutta kun hän, joka äitini kohdusta saakka on minut erottanut ja kutsunut armonsa kautta, näki hyväksi
\par 16 ilmaista minussa Poikansa, että minä julistaisin evankeliumia hänestä pakanain seassa, niin minä heti alunpitäenkään en kysynyt neuvoa lihalta ja vereltä,
\par 17 enkä lähtenyt ylös Jerusalemiin niiden luo, jotka ennen minua olivat apostoleja, vaan menin pois Arabiaan ja palasin taas takaisin Damaskoon.
\par 18 Sitten, kolmen vuoden kuluttua, minä menin ylös Jerusalemiin tutustuakseni Keefaaseen ja jäin hänen tykönsä viideksitoista päiväksi.
\par 19 Mutta muita apostoleja minä en nähnyt; näin ainoastaan Jaakobin, Herran veljen.
\par 20 Ja mitä kirjoitan teille, katso, Jumalan kasvojen edessä minä sanon, etten valhettele.
\par 21 Sitten menin Syyrian ja Kilikian paikkakuntiin.
\par 22 Mutta olin kasvoiltani tuntematon Juudean seurakunnille, jotka ovat Kristuksessa.
\par 23 Heidän kuuloonsa oli vain tullut: "Meidän entinen vainoojamme julistaa nyt sitä uskoa, jota hän ennen hävitti";
\par 24 ja he ylistivät Jumalaa minun tähteni.

\chapter{2}

\par 1 Sitten, neljäntoista vuoden kuluttua, minä taas menin ylös Jerusalemiin Barnabaan kanssa ja otin Tiituksenkin mukaani.
\par 2 Mutta minä menin sinne ilmestyksen johdosta ja esitin heille sen evankeliumin, jota julistan pakanain keskuudessa; esitin sen yksityisesti arvokkaimmille heistä, etten ehkä juoksisi tai olisi juossut turhaan.
\par 3 Mutta ei edes seuralaistani Tiitusta, joka oli kreikkalainen, pakotettu ympärileikkauttamaan itseänsä.
\par 4 Noiden pariimme luikertaneiden valheveljien tähden, jotka orjuuttaakseen meitä olivat hiipineet vakoilemaan vapauttamme, mikä meillä on Kristuksessa Jeesuksessa,
\par 5 me emme hetkeksikään alistuneet antamaan heille myöten, että evankeliumin totuus säilyisi teidän keskuudessanne.
\par 6 Ja nuo, joita jonakin pidettiin - millaisia lienevät olleet, ei kuulu minuun; Jumala ei katso henkilöön - nuo arvossapidetyt eivät velvoittaneet minua mihinkään enempään,
\par 7 vaan päinvastoin, kun näkivät, että minulle oli uskottu evankeliumin julistaminen ympärileikkaamattomille, samoin kuin Pietarille sen julistaminen ympärileikatuille -
\par 8 sillä hän, joka antoi Pietarille voimaa hänen apostolintoimeensa ympärileikattujen keskuudessa, antoi minullekin siihen voimaa pakanain keskuudessa -
\par 9 ja kun olivat tulleet tuntemaan sen armon, mikä oli minulle annettu, niin Jaakob ja Keefas ja Johannes, joita pidettiin pylväinä, antoivat minulle ja Barnabaalle yhteisen työn merkiksi kättä, mennäksemme, me pakanain keskuuteen ja he ympärileikattujen.
\par 10 Meidän oli vain muistaminen köyhiä, ja juuri sitä minä olenkin ahkeroinut tehdä.
\par 11 Mutta kun Keefas tuli Antiokiaan, vastustin minä häntä vasten kasvoja, koska hän oli herättänyt suurta paheksumista.
\par 12 Sillä ennenkuin Jaakobin luota oli tullut muutamia miehiä, oli hän syönyt yhdessä pakanain kanssa; mutta heidän tultuaan hän vetäytyi pois ja pysytteli erillään peläten ympärileikattuja,
\par 13 ja hänen kanssaan lankesivat ulkokultaisuuteen muutkin juutalaiset, niin että heidän ulkokultaisuutensa tempasi mukaansa Barnabaankin.
\par 14 Mutta kun minä näin, etteivät he vaeltaneet suoraan evankeliumin totuuden mukaan, sanoin minä Keefaalle kaikkien kuullen: "Jos sinä, joka olet juutalainen, noudatat pakanain tapoja etkä juutalaisten, miksi sinä pakotat pakanoita noudattamaan juutalaisten tapoja?"
\par 15 Me olemme luonnostamme juutalaisia, emmekä pakanasyntisiä;
\par 16 mutta koska tiedämme, ettei ihminen tule vanhurskaaksi lain teoista, vaan uskon kautta Jeesukseen Kristukseen, niin olemme mekin uskoneet Kristukseen Jeesukseen tullaksemme vanhurskaiksi uskosta Kristukseen eikä lain teoista, koska ei mikään liha tule vanhurskaaksi lain teoista.
\par 17 Mutta jos meidät itsemmekin, pyrkiessämme vanhurskautumaan Kristuksessa, on havaittu syntisiksi, onko sitten Kristus synnin palvelija? Pois se!
\par 18 Sillä jos minä uudestaan rakennan sen, minkä olen hajottanut maahan, osoitan minä olevani lain rikkoja.
\par 19 Sillä minä olen lain kautta kuollut pois laista, elääkseni Jumalalle. Minä olen Kristuksen kanssa ristiinnaulittu,
\par 20 ja minä elän, en enää minä, vaan Kristus elää minussa; ja minkä nyt elän lihassa, sen minä elän Jumalan Pojan uskossa, hänen, joka on rakastanut minua ja antanut itsensä minun edestäni.
\par 21 En minä tee mitättömäksi Jumalan armoa, sillä jos vanhurskaus on saatavissa lain kautta, silloinhan Kristus on turhaan kuollut.

\chapter{3}

\par 1 Oi te älyttömät galatalaiset! Kuka on lumonnut teidät, joiden silmäin eteen Jeesus Kristus oli kuvattu ristiinnaulittuna?
\par 2 Tämän vain tahdon saada teiltä tietää: lain teoistako saitte Hengen vai uskossa kuulemisesta?
\par 3 Niinkö älyttömiä olette? Te alotitte Hengessä, lihassako nyt lopetatte?
\par 4 Niin paljonko olette turhaan kärsineet? - jos se on turhaa ollut.
\par 5 Joka siis antaa teille Hengen ja tekee voimallisia tekoja teidän keskuudessanne, saako hän sen aikaan lain tekojen vai uskossa kuulemisen kautta,
\par 6 samalla tavalla kuin "Aabraham uskoi Jumalaa, ja se luettiin hänelle vanhurskaudeksi"?
\par 7 Tietäkää siis, että ne, jotka uskoon perustautuvat, ovat Aabrahamin lapsia.
\par 8 Ja koska Raamattu edeltäpäin näki, että Jumala vanhurskauttaa pakanat uskosta, julisti se Aabrahamille edeltäpäin tämän hyvän sanoman: "Sinussa kaikki kansat tulevat siunatuiksi".
\par 9 Niinmuodoin ne, jotka perustautuvat uskoon, siunataan uskovan Aabrahamin kanssa.
\par 10 Sillä kaikki, jotka perustautuvat lain tekoihin, ovat kirouksen alaisia; sillä kirjoitettu on: "Kirottu olkoon jokainen, joka ei pysy kaikessa, mikä on kirjoitettuna lain kirjassa, niin että hän sen tekee".
\par 11 Ja selvää on, ettei kukaan tule vanhurskaaksi Jumalan edessä lain kautta, koska "vanhurskas on elävä uskosta".
\par 12 Mutta laki ei perustaudu uskoon, vaan: "Joka ne täyttää, on niistä elävä".
\par 13 Kristus on lunastanut meidät lain kirouksesta, kun hän tuli kiroukseksi meidän edestämme - sillä kirjoitettu on: "Kirottu on jokainen, joka on puuhun ripustettu" -
\par 14 että Aabrahamin siunaus tulisi Jeesuksessa Kristuksessa pakanain osaksi ja me niin uskon kautta saisimme luvatun Hengen.
\par 15 Veljet, minä puhun ihmisten tavalla. Eihän kukaan voi kumota ihmisenkään vahvistettua testamenttia eikä siihen mitään lisätä.
\par 16 Mutta nyt lausuttiin lupaukset Aabrahamille ja hänen siemenelleen. Hän ei sano: "Ja siemenille", ikäänkuin monesta, vaan ikäänkuin yhdestä: "Ja sinun siemenellesi", joka on Kristus.
\par 17 Minä tarkoitan tätä: Jumalan ennen vahvistamaa testamenttia ei neljänsadan kolmenkymmenen vuoden perästä tullut laki voi kumota, niin että se tekisi lupauksen mitättömäksi.
\par 18 Sillä jos perintö tulisi laista, niin se ei enää tulisikaan lupauksesta. Mutta Aabrahamille Jumala on sen lahjoittanut lupauksen kautta.
\par 19 Mitä varten sitten on laki? Se on rikkomusten tähden jäljestäpäin lisätty olemaan siihen asti, kunnes oli tuleva se siemen, jolle lupaus oli annettu; ja se säädettiin enkelien kautta, välimiehen kädellä.
\par 20 Välimies taas ei ole yhtä varten; mutta Jumala on yksi.
\par 21 Onko sitten laki vastoin Jumalan lupauksia? Pois se! Sillä jos olisi annettu laki, joka voisi eläväksi tehdä, niin vanhurskaus todella tulisi laista.
\par 22 Mutta Raamattu on sulkenut kaikki synnin alle, että se, mikä luvattu oli, annettaisiin uskosta Jeesukseen Kristukseen niille, jotka uskovat.
\par 23 Mutta ennenkuin usko tuli, vartioitiin meitä lain alle suljettuina uskoa varten, joka oli vastedes ilmestyvä.
\par 24 Niinmuodoin on laista tullut meille kasvattaja Kristukseen, että me uskosta vanhurskaiksi tulisimme.
\par 25 Mutta uskon tultua me emme enää ole kasvattajan alaisia.
\par 26 Sillä te olette kaikki uskon kautta Jumalan lapsia Kristuksessa Jeesuksessa.
\par 27 Sillä kaikki te, jotka olette Kristukseen kastetut, olette Kristuksen päällenne pukeneet.
\par 28 Ei ole tässä juutalaista eikä kreikkalaista, ei ole orjaa eikä vapaata, ei ole miestä eikä naista; sillä kaikki te olette yhtä Kristuksessa Jeesuksessa.
\par 29 Mutta jos te olette Kristuksen omat, niin te siis olette Aabrahamin siementä, perillisiä lupauksen mukaan.

\chapter{4}

\par 1 Mutta minä sanon: niin kauan kuin perillinen on alaikäinen, ei hän missään kohden eroa orjasta, vaikka hän onkin kaiken herra;
\par 2 vaan hän on holhoojain ja huoneenhaltijain alainen isän määräämään aikaan asti.
\par 3 Samoin mekin; kun olimme alaikäisiä, olimme orjuutetut maailman alkeisvoimien alle.
\par 4 Mutta kun aika oli täytetty, lähetti Jumala Poikansa, vaimosta syntyneen, lain alaiseksi syntyneen,
\par 5 lunastamaan lain alaiset, että me pääsisimme lapsen asemaan.
\par 6 Ja koska te olette lapsia, on Jumala lähettänyt meidän sydämeemme Poikansa Hengen, joka huutaa: "Abba! Isä!"
\par 7 Niinpä sinä et siis enää ole orja, vaan lapsi; mutta jos olet lapsi, olet myös perillinen Jumalan kautta.
\par 8 Mutta silloin, kun ette tunteneet Jumalaa, te palvelitte jumalia, jotka luonnostaan eivät jumalia ole.
\par 9 Nyt sitävastoin, kun olette tulleet tuntemaan Jumalan ja, mikä enemmän on, kun Jumala tuntee teidät, kuinka te jälleen käännytte noiden heikkojen ja köyhien alkeisvoimien puoleen, joiden orjiksi taas uudestaan tahdotte tulla?
\par 10 Te otatte vaarin päivistä ja kuukausista ja juhla-ajoista ja vuosista.
\par 11 Minä pelkään teidän tähtenne, että olen ehkä turhaan teistä vaivaa nähnyt.
\par 12 Tulkaa minun kaltaisikseni, koska minäkin olen tullut teidän kaltaiseksenne, veljet, minä pyydän sitä teiltä. Ette ole minua mitenkään loukanneet.
\par 13 Tiedättehän, että ruumiillinen heikkous oli syynä siihen, että minä ensi kerralla julistin teille evankeliumia,
\par 14 ja tiedätte, mikä kiusaus teillä oli minun ruumiillisesta tilastani; ette minua halveksineet ettekä vieroneet, vaan otitte minut vastaan niinkuin Jumalan enkelin, jopa niinkuin Kristuksen Jeesuksen.
\par 15 Missä on nyt teiltä kerskaaminen onnestanne? Sillä minä annan teistä sen todistuksen, että te, jos se olisi ollut mahdollista, olisitte kaivaneet silmät päästänne ja antaneet minulle.
\par 16 Onko minusta siis tullut teidän vihamiehenne sentähden, että minä puhun teille totuuden?
\par 17 Heillä on intoa teidän hyväksenne, mutta ei oikeata; vaan he tahtovat eristää teidät, että teillä olisi intoa heidän hyväksensä.
\par 18 On hyvä, jos osoitetaan intoa hyvässä asiassa aina, eikä ainoastaan silloin, kun minä olen teidän tykönänne.
\par 19 Lapsukaiseni, jotka minun jälleen täytyy kivulla synnyttää, kunnes Kristus saa muodon teissä,
\par 20 tahtoisinpa nyt olla teidän tykönänne ja äänenikin muuttaa! Sillä minä olen aivan ymmällä teistä.
\par 21 Sanokaa minulle te, jotka tahdotte lain alaisia olla, ettekö kuule, mitä laki sanoo?
\par 22 Onhan kirjoitettu, että Aabrahamilla oli kaksi poikaa, toinen orjattaresta, toinen vapaasta.
\par 23 Mutta orjattaren poika oli syntynyt lihan mukaan, vapaan taas lupauksen voimasta.
\par 24 Tämä on kuvannollista puhetta; nämä naiset ovat kaksi liittoa: toinen on Siinain vuorelta, joka synnyttää orjuuteen, ja se on Haagar;
\par 25 sillä Haagar on Siinain vuori Arabiassa ja vastaa nykyistä Jerusalemia, joka elää orjuudessa lapsineen.
\par 26 Mutta se Jerusalem, joka ylhäällä on, on vapaa, ja se on meidän äitimme.
\par 27 Sillä kirjoitettu on: "Iloitse, sinä hedelmätön, joka et synnytä, riemahda ja huuda sinä, jolla ei ole synnytyskipuja. Sillä yksinäisellä on paljon lapsia, enemmän kuin sillä, jolla on mies."
\par 28 Ja te, veljet, olette lupauksen lapsia, niinkuin Iisak oli.
\par 29 Mutta niinkuin lihan mukaan syntynyt silloin vainosi Hengen mukaan syntynyttä, niin nytkin.
\par 30 Mutta mitä sanoo Raamattu? "Aja pois orjatar poikinensa; sillä orjattaren poika ei saa periä vapaan vaimon pojan kanssa."
\par 31 Niin me siis, veljet, emme ole orjattaren lapsia, vaan vapaan.

\chapter{5}

\par 1 Vapauteen Kristus vapautti meidät. Pysykää siis lujina, älkääkä antako uudestaan sitoa itseänne orjuuden ikeeseen.
\par 2 Katso, minä, Paavali, sanon teille, että jos ympärileikkautatte itsenne, niin Kristus ei ole oleva teille miksikään hyödyksi.
\par 3 Ja minä todistan taas jokaiselle ihmiselle, joka ympärileikkauttaa itsensä, että hän on velvollinen täyttämään kaiken lain.
\par 4 Te olette joutuneet pois Kristuksesta, te, jotka tahdotte lain kautta tulla vanhurskaiksi; te olette langenneet pois armosta.
\par 5 Sillä me odotamme uskosta vanhurskauden toivoa Hengen kautta.
\par 6 Sillä Kristuksessa Jeesuksessa ei auta ympärileikkaus eikä ympärileikkaamattomuus, vaan rakkauden kautta vaikuttava usko.
\par 7 Te juoksitte hyvin; kuka esti teitä olemasta totuudelle kuuliaisia?
\par 8 Houkutus siihen ei ole hänestä, joka teitä kutsuu.
\par 9 Vähäinen hapatus hapattaa koko taikinan.
\par 10 Minulla on teihin se luottamus Herrassa, että te ette missään kohden tule ajattelemaan toisin; mutta teidän häiritsijänne saa kantaa tuomionsa, olkoon kuka hyvänsä.
\par 11 Mutta jos minä, veljet, vielä saarnaan ympärileikkausta, miksi minua vielä vainotaan? Silloinhan olisi ristin pahennus poistettu.
\par 12 Kunpa aivan silpoisivat itsensä, nuo teidän kiihoittajanne!
\par 13 Te olette näet kutsutut vapauteen, veljet; älkää vain salliko vapauden olla yllykkeeksi lihalle, vaan palvelkaa toisianne rakkaudessa.
\par 14 Sillä kaikki laki on täytetty yhdessä käskysanassa, tässä: "Rakasta lähimmäistäsi niinkuin itseäsi".
\par 15 Mutta jos te purette ja syötte toisianne, katsokaa, ettette toinen toistanne perin hävitä.
\par 16 Minä sanon: vaeltakaa Hengessä, niin ette lihan himoa täytä.
\par 17 Sillä liha himoitsee Henkeä vastaan, ja Henki lihaa vastaan; nämä ovat nimittäin toisiansa vastaan, niin että te ette tee sitä, mitä tahdotte.
\par 18 Mutta jos te olette Hengen kuljetettavina, niin ette ole lain alla.
\par 19 Mutta lihan teot ovat ilmeiset, ja ne ovat: haureus, saastaisuus, irstaus,
\par 20 epäjumalanpalvelus, noituus, vihamielisyys, riita, kateellisuus, vihat, juonet, eriseurat, lahkot,
\par 21 kateus, juomingit, mässäykset ja muut senkaltaiset, joista teille edeltäpäin sanon, niinkuin jo ennenkin olen sanonut, että ne, jotka semmoista harjoittavat, eivät peri Jumalan valtakuntaa.
\par 22 Mutta Hengen hedelmä on rakkaus, ilo, rauha, pitkämielisyys, ystävällisyys, hyvyys, uskollisuus, sävyisyys, itsensähillitseminen.
\par 23 Sellaista vastaan ei ole laki.
\par 24 Ja ne, jotka ovat Kristuksen Jeesuksen omat, ovat ristiinnaulinneet lihansa himoineen ja haluineen.
\par 25 Jos me Hengessä elämme, niin myös Hengessä vaeltakaamme.
\par 26 Älkäämme olko turhan kunnian pyytäjiä, niin että toisiamme ärsyttelemme, toisiamme kadehdimme.

\chapter{6}

\par 1 Veljet, jos joku tavataan jostakin rikkomuksesta, niin ojentakaa te, hengelliset, häntä sävyisyyden hengessä; ja ole varuillasi, ettet sinäkin joutuisi kiusaukseen.
\par 2 Kantakaa toistenne kuormia, ja niin te täytätte Kristuksen lain.
\par 3 Sillä jos joku luulee jotakin olevansa, vaikka ei ole mitään, niin hän pettää itsensä.
\par 4 Mutta tutkikoon kukin omat tekonsa, ja silloin hänen kerskaamisensa on vain siinä, mitä hän itse on, ei siinä, mitä toinen on;
\par 5 sillä kunkin on kannettava oma taakkansa.
\par 6 Jolle sanaa opetetaan, se jakakoon kaikkea hyvää opettajallensa.
\par 7 Älkää eksykö, Jumala ei salli itseänsä pilkata; sillä mitä ihminen kylvää, sitä hän myös niittää.
\par 8 Joka lihaansa kylvää, se lihasta turmeluksen niittää; mutta joka Henkeen kylvää, se Hengestä iankaikkisen elämän niittää.
\par 9 Ja kun hyvää teemme, älkäämme lannistuko, sillä me saamme ajan tullen niittää, jos emme väsy.
\par 10 Sentähden, kun meillä vielä aikaa on, tehkäämme hyvää kaikille, mutta varsinkin uskonveljille.
\par 11 Katsokaa, kuinka suurilla kirjaimilla minä omakätisesti teille kirjoitan!
\par 12 Kaikki, jotka pyrkivät lihassa olemaan mieliksi, ne pakottavat teitä ympärileikkauttamaan itsenne vain siksi, ettei heitä Kristuksen ristin tähden vainottaisi.
\par 13 Eiväthän nekään, jotka ympärileikkauttavat itsensä, itse noudata lakia, vaan he tahtovat teitä ympärileikkauttamaan itsenne saadakseen kerskata teidän lihastanne.
\par 14 Mutta pois se minusta, että minä muusta kerskaisin kuin meidän Herramme Jeesuksen Kristuksen rististä, jonka kautta maailma on ristiinnaulittu minulle, ja minä maailmalle!
\par 15 Sillä ei ympärileikkaus ole mitään eikä ympärileikkaamattomuus, vaan uusi luomus.
\par 16 Ja kaikille, jotka tämän säännön mukaan vaeltavat, kaikille heille rauha ja laupeus, ja Jumalan Israelille!
\par 17 Älköön tästedes kukaan minulle vaivoja tuottako; sillä minä kannan Jeesuksen arvet ruumiissani.
\par 18 Meidän Herramme Jeesuksen Kristuksen armo olkoon teidän henkenne kanssa, veljet. Amen.


\end{document}