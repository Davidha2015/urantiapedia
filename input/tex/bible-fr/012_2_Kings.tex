\begin{document}

\title{2 Kings}


\chapter{1}

\par 1 Moab se révolta contre Israël, après la mort d'Achab.
\par 2 Or Achazia tomba par le treillis de sa chambre haute à Samarie, et il en fut malade. Il fit partir des messagers, et leur dit: Allez, consultez Baal Zebub, dieu d'Ékron, pour savoir si je guérirai de cette maladie.
\par 3 Mais l'ange de l'Éternel dit à Élie, le Thischbite: Lève-toi, monte à la rencontre des messagers du roi de Samarie, et dis-leur: Est-ce parce qu'il n'y a point de Dieu en Israël que vous allez consulter Baal Zebub, dieu d'Ékron?
\par 4 C'est pourquoi ainsi parle l'Éternel: Tu ne descendras pas du lit sur lequel tu es monté, car tu mourras. Et Élie s'en alla.
\par 5 Les messagers retournèrent auprès d'Achazia. Et il leur dit: Pourquoi revenez-vous?
\par 6 Ils lui répondirent: Un homme est monté à notre rencontre, et nous a dit: Allez, retournez vers le roi qui vous a envoyés, et dites-lui: Ainsi parle l'Éternel: Est-ce parce qu'il n'y a point de Dieu en Israël que tu envoies consulter Baal Zebub, dieu d'Ékron? C'est pourquoi tu ne descendras pas du lit sur lequel tu es monté, car tu mourras.
\par 7 Achazia leur dit: Quel air avait l'homme qui est monté à votre rencontre et qui vous a dit ces paroles?
\par 8 Ils lui répondirent: C'était un homme vêtu de poil et ayant une ceinture de cuir autour des reins. Et Achazia dit: C'est Élie, le Thischbite.
\par 9 Il envoya vers lui un chef de cinquante avec ses cinquante hommes. Ce chef monta auprès d'Élie, qui était assis sur le sommet de la montagne, et il lui dit: Homme de Dieu, le roi a dit: Descends!
\par 10 Élie répondit au chef de cinquante: Si je suis un homme de Dieu, que le feu descende du ciel et te consume, toi et tes cinquante hommes! Et le feu descendit du ciel et le consuma, lui et ses cinquante hommes.
\par 11 Achazia envoya de nouveau vers lui un autre chef de cinquante avec ses cinquante hommes. Ce chef prit la parole et dit à Élie: Homme de Dieu, ainsi a dit le roi: Hâte-toi de descendre!
\par 12 Élie leur répondit: Si je suis un homme de Dieu, que le feu descende du ciel et te consume, toi et tes cinquante hommes! Et le feu de Dieu descendit du ciel et le consuma, lui et ses cinquante hommes.
\par 13 Achazia envoya de nouveau un troisième chef de cinquante avec ses cinquante hommes. Ce troisième chef de cinquante monta; et à son arrivée, il fléchit les genoux devant Élie, et lui dit en suppliant: Homme de Dieu, que ma vie, je te prie, et que la vie de ces cinquante hommes tes serviteurs soit précieuse à tes yeux!
\par 14 Voici, le feu est descendu du ciel et a consumé les deux premiers chefs de cinquante et leurs cinquante hommes: mais maintenant, que ma vie soit précieuse à tes yeux!
\par 15 L'ange de l'Éternel dit à Élie: Descends avec lui, n'aie aucune crainte de lui. Élie se leva et descendit avec lui vers le roi.
\par 16 Il lui dit: Ainsi parle l'Éternel: Parce que tu as envoyé des messagers pour consulter Baal Zebub, dieu d'Ékron, comme s'il n'y avait en Israël point de Dieu dont on puisse consulter la parole, tu ne descendras pas du lit sur lequel tu es monté, car tu mourras.
\par 17 Achazia mourut, selon la parole de l'Éternel prononcée par Élie. Et Joram régna à sa place, la seconde année de Joram, fils de Josaphat, roi de Juda; car il n'avait point de fils.
\par 18 Le reste des actions d'Achazia, et ce qu'il a fait, cela n'est-il pas écrit dans le livre des Chroniques des rois d'Israël?

\chapter{2}

\par 1 Lorsque l'Éternel fit monter Élie au ciel dans un tourbillon, Élie partait de Guilgal avec Élisée.
\par 2 Élie dit à Élisée: Reste ici, je te prie, car l'Éternel m'envoie jusqu'à Béthel. Élisée répondit: L'Éternel est vivant et ton âme est vivante! je ne te quitterai point. Et ils descendirent à Béthel.
\par 3 Les fils des prophètes qui étaient à Béthel sortirent vers Élisée, et lui dirent: Sais-tu que l'Éternel enlève aujourd'hui ton maître au-dessus de ta tête? Et il répondit: Je le sais aussi; taisez-vous.
\par 4 Élie lui dit: Élisée, reste ici, je te prie, car l'Éternel m'envoie à Jéricho. Il répondit: L'Éternel est vivant et ton âme est vivante! je ne te quitterai point. Et ils arrivèrent à Jéricho.
\par 5 Les fils des prophètes qui étaient à Jéricho s'approchèrent d'Élisée, et lui dirent: Sais-tu que l'Éternel enlève aujourd'hui ton maître au-dessus de ta tête? Et il répondit: Je le sais aussi; taisez-vous.
\par 6 Élie lui dit: Reste ici, je te prie, car l'Éternel m'envoie au Jourdain. Il répondit: L'Éternel est vivant et ton âme est vivante! je ne te quitterai point. Et ils poursuivirent tous deux leur chemin.
\par 7 Cinquante hommes d'entre les fils des prophètes arrivèrent et s'arrêtèrent à distance vis-à-vis, et eux deux s'arrêtèrent au bord du Jourdain.
\par 8 Alors Élie prit son manteau, le roula, et en frappa les eaux, qui se partagèrent çà et là, et ils passèrent tous deux à sec.
\par 9 Lorsqu'ils eurent passé, Élie dit à Élisée: Demande ce que tu veux que je fasse pour toi, avant que je sois enlevé d'avec toi. Élisée répondit: Qu'il y ait sur moi, je te prie, une double portion de ton esprit!
\par 10 Élie dit: Tu demandes une chose difficile. Mais si tu me vois pendant que je serai enlevé d'avec toi, cela t'arrivera ainsi; sinon, cela n'arrivera pas.
\par 11 Comme ils continuaient à marcher en parlant, voici, un char de feu et des chevaux de feu les séparèrent l'un de l'autre, et Élie monta au ciel dans un tourbillon.
\par 12 Élisée regardait et criait: Mon père! mon père! Char d'Israël et sa cavalerie! Et il ne le vit plus. Saisissant alors ses vêtements, il les déchira en deux morceaux,
\par 13 et il releva le manteau qu'Élie avait laissé tomber. Puis il retourna, et s'arrêta au bord du Jourdain;
\par 14 il prit le manteau qu'Élie avait laissé tomber, et il en frappa les eaux, et dit: Où est l'Éternel, le Dieu d'Élie? Lui aussi, il frappa les eaux, qui se partagèrent çà et là, et Élisée passa.
\par 15 Les fils des prophètes qui étaient à Jéricho, vis-à-vis, l'ayant vu, dirent: L'esprit d'Élie repose sur Élisée! Et ils allèrent à sa rencontre, et se prosternèrent contre terre devant lui.
\par 16 Ils lui dirent: Voici, il y a parmi tes serviteurs cinquante hommes vaillants; veux-tu qu'ils aillent chercher ton maître? Peut-être que l'esprit de l'Éternel l'a emporté et l'a jeté sur quelque montagne ou dans quelque vallée. Il répondit: Ne les envoyez pas.
\par 17 Mais ils le pressèrent longtemps; et il dit: Envoyez-les. Ils envoyèrent les cinquante hommes, qui cherchèrent Élie pendant trois jours et ne le trouvèrent point.
\par 18 Lorsqu'ils furent de retour auprès d'Élisée, qui était à Jéricho, il leur dit: Ne vous avais-je pas dit: N'allez pas?
\par 19 Les gens de la ville dirent à Élisée: Voici, le séjour de la ville est bon, comme le voit mon seigneur; mais les eaux sont mauvaises, et le pays est stérile.
\par 20 Il dit: Apportez-moi un plat neuf, et mettez-y du sel. Et ils le lui apportèrent.
\par 21 Il alla vers la source des eaux, et il y jeta du sel, et dit: Ainsi parle l'Éternel: J'assainis ces eaux; il n'en proviendra plus ni mort, ni stérilité.
\par 22 Et les eaux furent assainies, jusqu'à ce jour, selon la parole qu'Élisée avait prononcée.
\par 23 Il monta de là à Béthel; et comme il cheminait à la montée, des petits garçons sortirent de la ville, et se moquèrent de lui. Ils lui disaient: Monte, chauve! monte, chauve!
\par 24 Il se retourna pour les regarder, et il les maudit au nom de l'Éternel. Alors deux ours sortirent de la forêt, et déchirèrent quarante-deux de ces enfants.
\par 25 De là il alla sur la montagne du Carmel, d'où il retourna à Samarie.

\chapter{3}

\par 1 Joram, fils d'Achab, régna sur Israël à Samarie, la dix-huitième année de Josaphat, roi de Juda. Il régna douze ans.
\par 2 Il fit ce qui est mal aux yeux de l'Éternel, non pas toutefois comme son père et sa mère. Il renversa les statues de Baal que son père avait faites;
\par 3 mais il se livra aux péchés de Jéroboam, fils de Nebath, qui avait fait pécher Israël, et il ne s'en détourna point.
\par 4 Méscha, roi de Moab, possédait des troupeaux, et il payait au roi d'Israël un tribut de cent mille agneaux et de cent mille béliers avec leur laine.
\par 5 A la mort d'Achab, le roi de Moab se révolta contre le roi d'Israël.
\par 6 Le roi Joram sortit alors de Samarie, et passa en revue tout Israël.
\par 7 Il se mit en marche, et il fit dire à Josaphat, roi de Juda: Le roi de Moab s'est révolté contre moi; veux-tu venir avec moi attaquer Moab? Josaphat répondit: J'irai, moi comme toi, mon peuple comme ton peuple, mes chevaux comme tes chevaux.
\par 8 Et il dit: Par quel chemin monterons-nous? Joram dit: Par le chemin du désert d'Édom.
\par 9 Le roi d'Israël, le roi de Juda et le roi d'Édom, partirent; et après une marche de sept jours, ils manquèrent d'eau pour l'armée et pour les bêtes qui la suivaient.
\par 10 Alors le roi d'Israël dit: Hélas! l'Éternel a appelé ces trois rois pour les livrer entre les mains de Moab.
\par 11 Mais Josaphat dit: N'y a-t-il ici aucun prophète de l'Éternel, par qui nous puissions consulter l'Éternel? L'un des serviteurs du roi d'Israël répondit: Il y a ici Élisée, fils de Schaphath, qui versait l'eau sur les mains d'Élie.
\par 12 Et Josaphat dit: La parole de l'Éternel est avec lui. Le roi d'Israël, Josaphat et le roi d'Édom, descendirent auprès de lui.
\par 13 Élisée dit au roi d'Israël: Qu'y a-t-il entre moi et toi? Va vers les prophètes de ton père et vers les prophètes de ta mère. Et le roi d'Israël lui dit: Non! car l'Éternel a appelé ces trois rois pour les livrer entre les mains de Moab.
\par 14 Élisée dit: L'Éternel des armées, dont je suis le serviteur, est vivant! si je n'avais égard à Josaphat, roi de Juda, je ne ferais aucune attention à toi et je ne te regarderais pas.
\par 15 Maintenant, amenez-moi un joueur de harpe. Et comme le joueur de harpe jouait, la main de l'Éternel fut sur Élisée.
\par 16 Et il dit: Ainsi parle l'Éternel: Faites dans cette vallée des fosses, des fosses!
\par 17 Car ainsi parle l'Éternel: Vous n'apercevrez point de vent et vous ne verrez point de pluie, et cette vallée se remplira d'eau, et vous boirez, vous, vos troupeaux et votre bétail.
\par 18 Mais cela est peu de chose aux yeux de l'Éternel. Il livrera Moab entre vos mains;
\par 19 vous frapperez toutes les villes fortes et toutes les villes d'élite, vous abattrez tous les bons arbres, vous boucherez toutes les sources d'eau, et vous ruinerez avec des pierres tous les meilleurs champs.
\par 20 Or le matin, au moment de la présentation de l'offrande, voici, l'eau arriva du chemin d'Édom, et le pays fut rempli d'eau.
\par 21 Cependant, tous les Moabites ayant appris que les rois montaient pour les attaquer, on convoqua tous ceux en âge de porter les armes et même au-dessus, et ils se tinrent sur la frontière.
\par 22 Ils se levèrent de bon matin, et quand le soleil brilla sur les eaux, les Moabites virent en face d'eux les eaux rouges comme du sang.
\par 23 Ils dirent: C'est du sang! les rois ont tiré l'épée entre eux, ils se sont frappés les uns les autres; maintenant, Moabites, au pillage!
\par 24 Et ils marchèrent contre le camp d'Israël. Mais Israël se leva, et frappa Moab, qui prit la fuite devant eux. Ils pénétrèrent dans le pays, et frappèrent Moab.
\par 25 Ils renversèrent les villes, ils jetèrent chacun des pierres dans tous les meilleurs champs et les en remplirent, ils bouchèrent toutes les sources d'eau, et ils abattirent tous les bons arbres; et les frondeurs enveloppèrent et battirent Kir Haréseth, dont on ne laissa que les pierres.
\par 26 Le roi de Moab, voyant qu'il avait le dessous dans le combat, prit avec lui sept cents hommes tirant l'épée pour se frayer un passage jusqu'au roi d'Édom; mais ils ne purent pas.
\par 27 Il prit alors son fils premier-né, qui devait régner à sa place, et il l'offrit en holocauste sur la muraille. Et une grande indignation s'empara d'Israël, qui s'éloigna du roi de Moab et retourna dans son pays.

\chapter{4}

\par 1 Une femme d'entre les femmes des fils des prophètes cria à Élisée, en disant: Ton serviteur mon mari est mort, et tu sais que ton serviteur craignait l'Éternel; or le créancier est venu pour prendre mes deux enfants et en faire ses esclaves.
\par 2 Élisée lui dit: Que puis-je faire pour toi? Dis-moi, qu'as-tu à la maison? Elle répondit: Ta servante n'a rien du tout à la maison qu'un vase d'huile.
\par 3 Et il dit: Va demander au dehors des vases chez tous tes voisins, des vases vides, et n'en demande pas un petit nombre.
\par 4 Quand tu seras rentrée, tu fermeras la porte sur toi et sur tes enfants; tu verseras dans tous ces vases, et tu mettras de côté ceux qui seront pleins.
\par 5 Alors elle le quitta. Elle ferma la porte sur elle et sur ses enfants; ils lui présentaient les vases, et elle versait.
\par 6 Lorsque les vases furent pleins, elle dit à son fils: Présente-moi encore un vase. Mais il lui répondit: Il n'y a plus de vase. Et l'huile s'arrêta.
\par 7 Elle alla le rapporter à l'homme de Dieu, et il dit: Va vendre l'huile, et paie ta dette; et tu vivras, toi et tes fils, de ce qui restera.
\par 8 Un jour Élisée passait par Sunem. Il y avait là une femme de distinction, qui le pressa d'accepter à manger. Et toutes les fois qu'il passait, il se rendait chez elle pour manger.
\par 9 Elle dit à son mari: Voici, je sais que cet homme qui passe toujours chez nous est un saint homme de Dieu.
\par 10 Faisons une petite chambre haute avec des murs, et mettons-y pour lui un lit, une table, un siège et un chandelier, afin qu'il s'y retire quand il viendra chez nous.
\par 11 Élisée, étant revenu à Sunem, se retira dans la chambre haute et y coucha.
\par 12 Il dit à Guéhazi, son serviteur: Appelle cette Sunamite. Guéhazi l'appela, et elle se présenta devant lui.
\par 13 Et Élisée dit à Guéhazi: Dis-lui: Voici, tu nous as montré tout cet empressement; que peut-on faire pour toi? Faut-il parler pour toi au roi ou au chef de l'armée? Elle répondit: J'habite au milieu de mon peuple.
\par 14 Et il dit: Que faire pour elle? Guéhazi répondit: Mais, elle n'a point de fils, et son mari est vieux.
\par 15 Et il dit: Appelle-la. Guéhazi l'appela, et elle se présenta à la porte.
\par 16 Élisée lui dit: A cette même époque, l'année prochaine, tu embrasseras un fils. Et elle dit: Non! mon seigneur, homme de Dieu, ne trompe pas ta servante!
\par 17 Cette femme devint enceinte, et elle enfanta un fils à la même époque, l'année suivante, comme Élisée lui avait dit.
\par 18 L'enfant grandit. Et un jour qu'il était allé trouver son père vers les moissonneurs,
\par 19 il dit à son père: Ma tête! ma tête! Le père dit à son serviteur: Porte-le à sa mère.
\par 20 Le serviteur l'emporta et l'amena à sa mère. Et l'enfant resta sur les genoux de sa mère jusqu'à midi, puis il mourut.
\par 21 Elle monta, le coucha sur le lit de l'homme de Dieu, ferma la porte sur lui, et sortit.
\par 22 Elle appela son mari, et dit: Envoie-moi, je te prie, un des serviteurs et une des ânesses; je veux aller en hâte vers l'homme de Dieu, et je reviendrai.
\par 23 Et il dit: Pourquoi veux-tu aller aujourd'hui vers lui? Ce n'est ni nouvelle lune ni sabbat. Elle répondit: Tout va bien.
\par 24 Puis elle fit seller l'ânesse, et dit à son serviteur: Mène et pars; ne m'arrête pas en route sans que je te le dise.
\par 25 Elle partit donc et se rendit vers l'homme de Dieu sur la montagne du Carmel. L'homme de Dieu, l'ayant aperçue de loin, dit à Guéhazi, son serviteur: Voici cette Sunamite!
\par 26 Maintenant, cours donc à sa rencontre, et dis-lui: Te portes-tu bien? Ton mari et ton enfant se portent-ils bien? Elle répondit: Bien.
\par 27 Et dès qu'elle fut arrivée auprès de l'homme de Dieu sur la montagne, elle embrassa ses pieds. Guéhazi s'approcha pour la repousser. Mais l'homme de Dieu dit: Laisse-la, car son âme est dans l'amertume, et l'Éternel me l'a caché et ne me l'a point fait connaître.
\par 28 Alors elle dit: Ai-je demandé un fils à mon seigneur? N'ai-je pas dit: Ne me trompe pas?
\par 29 Et Élisée dit à Guéhazi: Ceins tes reins, prends mon bâton dans ta main, et pars. Si tu rencontres quelqu'un, ne le salue pas; et si quelqu'un te salue, ne lui réponds pas. Tu mettras mon bâton sur le visage de l'enfant.
\par 30 La mère de l'enfant dit: L'Éternel est vivant et ton âme est vivante! je ne te quitterai point. Et il se leva et la suivit.
\par 31 Guéhazi les avait devancés, et il avait mis le bâton sur le visage de l'enfant; mais il n'y eut ni voix ni signe d'attention. Il s'en retourna à la rencontre d'Élisée, et lui rapporta la chose, en disant: L'enfant ne s'est pas réveillé.
\par 32 Lorsque Élisée arriva dans la maison, voici, l'enfant était mort, couché sur son lit.
\par 33 Élisée entra et ferma la porte sur eux deux, et il pria l'Éternel.
\par 34 Il monta, et se coucha sur l'enfant; il mit sa bouche sur sa bouche, ses yeux sur ses yeux, ses mains sur ses mains, et il s'étendit sur lui. Et la chair de l'enfant se réchauffa.
\par 35 Élisée s'éloigna, alla çà et là par la maison, puis remonta et s'étendit sur l'enfant. Et l'enfant éternua sept fois, et il ouvrit les yeux.
\par 36 Élisée appela Guéhazi, et dit: Appelle cette Sunamite. Guéhazi l'appela, et elle vint vers Élisée, qui dit: Prends ton fils!
\par 37 Elle alla se jeter à ses pieds, et se prosterna contre terre. Et elle prit son fils, et sortit.
\par 38 Élisée revint à Guilgal, et il y avait une famine dans le pays. Comme les fils des prophètes étaient assis devant lui, il dit à son serviteur: Mets le grand pot, et fais cuire un potage pour les fils des prophètes.
\par 39 L'un d'eux sortit dans les champs pour cueillir des herbes; il trouva de la vigne sauvage et il y cueillit des coloquintes sauvages, plein son vêtement. Quand il rentra, il les coupa en morceaux dans le pot où était le potage, car on ne les connaissait pas.
\par 40 On servit à manger à ces hommes; mais dès qu'ils eurent mangé du potage, ils s'écrièrent: La mort est dans le pot, homme de Dieu! Et ils ne purent manger.
\par 41 Élisée dit: Prenez de la farine. Il en jeta dans le pot, et dit: Sers à ces gens, et qu'ils mangent. Et il n'y avait plus rien de mauvais dans le pot.
\par 42 Un homme arriva de Baal Schalischa. Il apporta du pain des prémices à l'homme de Dieu, vingt pains d'orge, et des épis nouveaux dans son sac. Élisée dit: Donne à ces gens, et qu'ils mangent.
\par 43 Son serviteur répondit: Comment pourrais-je en donner à cent personnes? Mais Élisée dit: Donne à ces gens, et qu'ils mangent; car ainsi parle l'Éternel: On mangera, et on en aura de reste.
\par 44 Il mit alors les pains devant eux; et ils mangèrent et en eurent de reste, selon la parole de l'Éternel.

\chapter{5}

\par 1 Naaman, chef de l'armée du roi de Syrie, jouissait de la faveur de son maître et d'une grande considération; car c'était par lui que l'Éternel avait délivré les Syriens. Mais cet homme fort et vaillant était lépreux.
\par 2 Or les Syriens étaient sortis par troupes, et ils avaient emmené captive une petite fille du pays d'Israël, qui était au service de la femme de Naaman.
\par 3 Et elle dit à sa maîtresse: Oh! si mon seigneur était auprès du prophète qui est à Samarie, le prophète le guérirait de sa lèpre!
\par 4 Naaman alla dire à son maître: La jeune fille du pays d'Israël a parlé de telle et telle manière.
\par 5 Et le roi de Syrie dit: Va, rends-toi à Samarie, et j'enverrai une lettre au roi d'Israël. Il partit, prenant avec lui dix talents d'argent, six mille sicles d'or, et dix vêtements de rechange.
\par 6 Il porta au roi d'Israël la lettre, où il était dit: Maintenant, quand cette lettre te sera parvenue, tu sauras que je t'envoie Naaman, mon serviteur, afin que tu le guérisses de sa lèpre.
\par 7 Après avoir lu la lettre, le roi d'Israël déchira ses vêtements, et dit: Suis-je un dieu, pour faire mourir et pour faire vivre, qu'il s'adresse à moi afin que je guérisse un homme de sa lèpre? Sachez donc et comprenez qu'il cherche une occasion de dispute avec moi.
\par 8 Lorsqu'Élisée, homme de Dieu, apprit que le roi d'Israël avait déchiré ses vêtements, il envoya dire au roi: Pourquoi as-tu déchiré tes vêtements? Laisse-le venir à moi, et il saura qu'il y a un prophète en Israël.
\par 9 Naaman vint avec ses chevaux et son char, et il s'arrêta à la porte de la maison d'Élisée.
\par 10 Élisée lui fit dire par un messager: Va, et lave-toi sept fois dans le Jourdain; ta chair deviendra saine, et tu seras pur.
\par 11 Naaman fut irrité, et il s'en alla, en disant: Voici, je me disais: Il sortira vers moi, il se présentera lui-même, il invoquera le nom de l'Éternel, son Dieu, il agitera sa main sur la place et guérira le lépreux.
\par 12 Les fleuves de Damas, l'Abana et le Parpar, ne valent-ils pas mieux que toutes les eaux d'Israël? Ne pourrais-je pas m'y laver et devenir pur? Et il s'en retournait et partait avec fureur.
\par 13 Mais ses serviteurs s'approchèrent pour lui parler, et ils dirent: Mon père, si le prophète t'eût demandé quelque chose de difficile, ne l'aurais-tu pas fait? Combien plus dois-tu faire ce qu'il t'a dit: Lave-toi, et tu seras pur!
\par 14 Il descendit alors et se plongea sept fois dans le Jourdain, selon la parole de l'homme de Dieu; et sa chair redevint comme la chair d'un jeune enfant, et il fut pur.
\par 15 Naaman retourna vers l'homme de Dieu, avec toute sa suite. Lorsqu'il fut arrivé, il se présenta devant lui, et dit: Voici, je reconnais qu'il n'y a point de Dieu sur toute la terre, si ce n'est en Israël. Et maintenant, accepte, je te prie, un présent de la part de ton serviteur.
\par 16 Élisée répondit: L'Éternel, dont je suis le serviteur, est vivant! je n'accepterai pas. Naaman le pressa d'accepter, mais il refusa.
\par 17 Alors Naaman dit: Puisque tu refuses, permets que l'on donne de la terre à ton serviteur, une charge de deux mulets; car ton serviteur ne veut plus offrir à d'autres dieux ni holocauste ni sacrifice, il n'en offrira qu'à l'Éternel.
\par 18 Voici toutefois ce que je prie l'Éternel de pardonner à ton serviteur. Quand mon maître entre dans la maison de Rimmon pour s'y prosterner et qu'il s'appuie sur ma main, je me prosterne aussi dans la maison de Rimmon: veuille l'Éternel pardonner à ton serviteur, lorsque je me prosternerai dans la maison de Rimmon!
\par 19 Élisée lui dit: Va en paix. Lorsque Naaman eut quitté Élisée et qu'il fut à une certaine distance,
\par 20 Guéhazi, serviteur d'Élisée, homme de Dieu, se dit en lui-même: Voici, mon maître a ménagé Naaman, ce Syrien, en n'acceptant pas de sa main ce qu'il avait apporté; l'Éternel est vivant! je vais courir après lui, et j'en obtiendrai quelque chose.
\par 21 Et Guéhazi courut après Naaman. Naaman, le voyant courir après lui, descendit de son char pour aller à sa rencontre, et dit: Tout va-t-il bien?
\par 22 Il répondit: Tout va bien. Mon maître m'envoie te dire: Voici, il vient d'arriver chez moi deux jeunes gens de la montagne d'Éphraïm, d'entre les fils des prophètes; donne pour eux, je te prie, un talent d'argent et deux vêtements de rechange.
\par 23 Naaman dit: Consens à prendre deux talents. Il le pressa, et il serra deux talents d'argent dans deux sacs, donna deux habits de rechange, et les fit porter devant Guéhazi par deux de ses serviteurs.
\par 24 Arrivé à la colline, Guéhazi les prit de leurs mains et les déposa dans la maison, et il renvoya ces gens qui partirent.
\par 25 Puis il alla se présenter à son maître. Élisée lui dit: D'où viens-tu, Guéhazi? Il répondit: Ton serviteur n'est allé ni d'un côté ni d'un autre.
\par 26 Mais Élisée lui dit: Mon esprit n'était pas absent, lorsque cet homme a quitté son char pour venir à ta rencontre. Est-ce le temps de prendre de l'argent et de prendre des vêtements, puis des oliviers, des vignes, des brebis, des boeufs, des serviteurs et des servantes?
\par 27 La lèpre de Naaman s'attachera à toi et à ta postérité pour toujours. Et Guéhazi sortit de la présence d'Élisée avec une lèpre comme la neige.

\chapter{6}

\par 1 Les fils des prophètes dirent à Élisée: Voici, le lieu où nous sommes assis devant toi est trop étroit pour nous.
\par 2 Allons jusqu'au Jourdain; nous prendrons là chacun une poutre, et nous nous y ferons un lieu d'habitation. Élisée répondit: Allez.
\par 3 Et l'un d'eux dit: Consens à venir avec tes serviteurs. Il répondit: J'irai.
\par 4 Il partit donc avec eux. Arrivés au Jourdain, ils coupèrent du bois.
\par 5 Et comme l'un d'eux abattait une poutre, le fer tomba dans l'eau. Il s'écria: Ah! mon seigneur, il était emprunté!
\par 6 L'homme de Dieu dit: Où est-il tombé? Et il lui montra la place. Alors Élisée coupa un morceau de bois, le jeta à la même place, et fit surnager le fer.
\par 7 Puis il dit: Enlève-le! Et il avança la main, et le prit.
\par 8 Le roi de Syrie était en guerre avec Israël, et, dans un conseil qu'il tint avec ses serviteurs, il dit: Mon camp sera dans un tel lieu.
\par 9 Mais l'homme de Dieu fit dire au roi d'Israël: Garde-toi de passer dans ce lieu, car les Syriens y descendent.
\par 10 Et le roi d'Israël envoya des gens, pour s'y tenir en observation, vers le lieu que lui avait mentionné et signalé l'homme de Dieu. Cela arriva non pas une fois ni deux fois.
\par 11 Le roi de Syrie en eut le coeur agité; il appela ses serviteurs, et leur dit: Ne voulez-vous pas me déclarer lequel de nous est pour le roi d'Israël?
\par 12 L'un de ses serviteurs répondit: Personne! ô roi mon seigneur; mais Élisée, le prophète, qui est en Israël, rapporte au roi d'Israël les paroles que tu prononces dans ta chambre à coucher.
\par 13 Et le roi dit: Allez et voyez où il est, et je le ferai prendre. On vint lui dire: Voici, il est à Dothan.
\par 14 Il y envoya des chevaux, des chars et une forte troupe, qui arrivèrent de nuit et qui enveloppèrent la ville.
\par 15 Le serviteur de l'homme de Dieu se leva de bon matin et sortit; et voici, une troupe entourait la ville, avec des chevaux et des chars. Et le serviteur dit à l'homme de Dieu: Ah! mon seigneur, comment ferons-nous?
\par 16 Il répondit: Ne crains point, car ceux qui sont avec nous sont en plus grand nombre que ceux qui sont avec eux.
\par 17 Élisée pria, et dit: Éternel, ouvre ses yeux, pour qu'il voie. Et l'Éternel ouvrit les yeux du serviteur, qui vit la montagne pleine de chevaux et de chars de feu autour d'Élisée.
\par 18 Les Syriens descendirent vers Élisée. Il adressa alors cette prière à l'Éternel: Daigne frapper d'aveuglement cette nation! Et l'Éternel les frappa d'aveuglement, selon la parole d'Élisée.
\par 19 Élisée leur dit: Ce n'est pas ici le chemin, et ce n'est pas ici la ville; suivez-moi, et je vous conduirai vers l'homme que vous cherchez. Et il les conduisit à Samarie.
\par 20 Lorsqu'ils furent entrés dans Samarie, Élisée dit: Éternel, ouvre les yeux de ces gens, pour qu'ils voient! Et l'Éternel ouvrit leurs yeux, et ils virent qu'ils étaient au milieu de Samarie.
\par 21 Le roi d'Israël, en les voyant, dit à Élisée: Frapperai-je, frapperai-je, mon père?
\par 22 Tu ne frapperas point, répondit Élisée; est-ce que tu frappes ceux que tu fais prisonniers avec ton épée et avec ton arc? Donne-leur du pain et de l'eau, afin qu'ils mangent et boivent; et qu'ils s'en aillent ensuite vers leur maître.
\par 23 Le roi d'Israël leur fit servir un grand repas, et ils mangèrent et burent; puis il les renvoya, et ils s'en allèrent vers leur maître. Et les troupes des Syriens ne revinrent plus sur le territoire d'Israël.
\par 24 Après cela, Ben Hadad, roi de Syrie, ayant rassemblé toute son armée, monta et assiégea Samarie.
\par 25 Il y eut une grande famine dans Samarie; et ils la serrèrent tellement qu'une tête d'âne valait quatre-vingts sicles d'argent, et le quart d'un kab de fiente de pigeon cinq sicles d'argent.
\par 26 Et comme le roi passait sur la muraille, une femme lui cria: Sauve-moi, ô roi, mon seigneur!
\par 27 Il répondit: Si l'Éternel ne te sauve pas, avec quoi te sauverais-je? avec le produit de l'aire ou du pressoir?
\par 28 Et le roi lui dit: Qu'as-tu? Elle répondit: Cette femme-là m'a dit: Donne ton fils! nous le mangerons aujourd'hui, et demain nous mangerons mon fils.
\par 29 Nous avons fait cuire mon fils, et nous l'avons mangé. Et le jour suivant, je lui ai dit: Donne ton fils, et nous le mangerons. Mais elle a caché son fils.
\par 30 Lorsque le roi entendit les paroles de cette femme, il déchira ses vêtements, en passant sur la muraille; et le peuple vit qu'il avait en dedans un sac sur son corps.
\par 31 Le roi dit: Que Dieu me punisse dans toute sa rigueur, si la tête d'Élisée, fils de Schaphath, reste aujourd'hui sur lui!
\par 32 Or Élisée était dans sa maison, et les anciens étaient assis auprès de lui. Le roi envoya quelqu'un devant lui. Mais avant que le messager soit arrivé, Élisée dit aux anciens: Voyez-vous que ce fils d'assassin envoie quelqu'un pour m'ôter la tête? Écoutez! quand le messager viendra, fermez la porte, et repoussez-le avec la porte: le bruit des pas de son maître ne se fait-il pas entendre derrière lui?
\par 33 Il leur parlait encore, et déjà le messager était descendu vers lui, et disait: Voici, ce mal vient de l'Éternel; qu'ai-je à espérer encore de l'Éternel?

\chapter{7}

\par 1 Élisée dit: Écoutez la parole de l'Éternel! Ainsi parle l'Éternel: Demain, à cette heure, on aura une mesure de fleur de farine pour un sicle et deux mesures d'orge pour un sicle, à la porte de Samarie.
\par 2 L'officier sur la main duquel s'appuyait le roi répondit à l'homme de Dieu: Quand l'Éternel ferait des fenêtres au ciel, pareille chose arriverait-elle? Et Élisée dit: Tu le verras de tes yeux; mais tu n'en mangeras point.
\par 3 Il y avait à l'entrée de la porte quatre lépreux, qui se dirent l'un à l'autre: Quoi! resterons-nous ici jusqu'à ce que nous mourions?
\par 4 Si nous songeons à entrer dans la ville, la famine est dans la ville, et nous y mourrons; et si nous restons ici, nous mourrons également. Allons nous jeter dans le camp des Syriens; s'ils nous laissent vivre, nous vivrons et s'ils nous font mourir, nous mourrons.
\par 5 Ils partirent donc au crépuscule, pour se rendre au camp des Syriens; et lorsqu'ils furent arrivés à l'entrée du camp des Syriens, voici, il n'y avait personne.
\par 6 Le Seigneur avait fait entendre dans le camp des Syriens un bruit de chars et un bruit de chevaux, le bruit d'une grande armée, et ils s'étaient dit l'un à l'autre: Voici, le roi d'Israël a pris à sa solde contre nous les rois des Héthiens et les rois des Égyptiens pour venir nous attaquer.
\par 7 Et ils se levèrent et prirent la fuite au crépuscule, abandonnant leurs tentes, leurs chevaux et leurs ânes, le camp tel qu'il était, et ils s'enfuirent pour sauver leur vie.
\par 8 Les lépreux, étant arrivés à l'entrée du camp, pénétrèrent dans une tente, mangèrent et burent, et en emportèrent de l'argent, de l'or, et des vêtements, qu'ils allèrent cacher. Ils revinrent, pénétrèrent dans une autre tente, et en emportèrent des objets qu'ils allèrent cacher.
\par 9 Puis ils se dirent l'un à l'autre: Nous n'agissons pas bien! Cette journée est une journée de bonne nouvelle; si nous gardons le silence et si nous attendons jusqu'à la lumière du matin, le châtiment nous atteindra. Venez maintenant, et allons informer la maison du roi.
\par 10 Ils partirent, et ils appelèrent les gardes de la porte de la ville, auxquels ils firent ce rapport: Nous sommes entrés dans le camp des Syriens, et voici, il n'y a personne, on n'y entend aucune voix d'homme; il n'y a que des chevaux attachés et des ânes attachés, et les tentes comme elles étaient.
\par 11 Les gardes de la porte crièrent, et ils transmirent ce rapport à l'intérieur de la maison du roi.
\par 12 Le roi se leva de nuit, et il dit à ses serviteurs: Je veux vous communiquer ce que nous font les Syriens. Comme ils savent que nous sommes affamés, ils ont quitté le camp pour se cacher dans les champs, et ils se sont dit: Quand ils sortiront de la ville, nous les saisirons vivants, et nous entrerons dans la ville.
\par 13 L'un des serviteurs du roi répondit: Que l'on prenne cinq des chevaux qui restent encore dans la ville, -ils sont comme toute la multitude d'Israël qui y est restée, ils sont comme toute la multitude d'Israël qui dépérit, -et envoyons voir ce qui se passe.
\par 14 On prit deux chars avec les chevaux, et le roi envoya des messagers sur les traces de l'armée des Syriens, en disant: Allez et voyez.
\par 15 Ils allèrent après eux jusqu'au Jourdain; et voici, toute la route était pleine de vêtements et d'objets que les Syriens avaient jetés dans leur précipitation. Les messagers revinrent, et le rapportèrent au roi.
\par 16 Le peuple sortit, et pilla le camp des Syriens. Et l'on eut une mesure de fleur de farine pour un sicle et deux mesures d'orge pour un sicle, selon la parole de l'Éternel.
\par 17 Le roi avait remis la garde de la porte à l'officier sur la main duquel il s'appuyait; mais cet officier fut écrasé à la porte par le peuple et il mourut, selon la parole qu'avait prononcée l'homme de Dieu quand le roi était descendu vers lui.
\par 18 L'homme de Dieu avait dit alors au roi: On aura deux mesures d'orge pour un sicle et une mesure de fleur de farine pour un sicle, demain, à cette heure, à la porte de Samarie.
\par 19 Et l'officier avait répondu à l'homme de Dieu: Quand l'Éternel ferait des fenêtres au ciel, pareille chose arriverait-elle? Et Élisée avait dit: Tu le verras de tes yeux; mais tu n'en mangeras point.
\par 20 C'est en effet ce qui lui arriva: il fut écrasé à la porte par le peuple, et il mourut.

\chapter{8}

\par 1 Élisée dit à la femme dont il avait fait revivre le fils: Lève-toi, va t'en, toi et ta maison, et séjourne où tu pourras; car l'Éternel appelle la famine, et même elle vient sur le pays pour sept années.
\par 2 La femme se leva, et elle fit selon la parole de l'homme de Dieu: elle s'en alla, elle et sa maison, et séjourna sept ans au pays des Philistins.
\par 3 Au bout des sept ans, la femme revint du pays des Philistins, et elle alla implorer le roi au sujet de sa maison et de son champ.
\par 4 Le roi s'entretenait avec Guéhazi, serviteur de l'homme de Dieu, et il disait: Raconte-moi, je te prie, toutes les grandes choses qu'Élisée a faites.
\par 5 Et pendant qu'il racontait au roi comment Élisée avait rendu la vie à un mort, la femme dont Élisée avait fait revivre le fils vint implorer le roi au sujet de sa maison et de son champ. Guéhazi dit: O roi, mon seigneur, voici la femme, et voici son fils qu'Élisée a fait revivre.
\par 6 Le roi interrogea la femme, et elle lui fit le récit. Puis le roi lui donna un eunuque, auquel il dit: Fais restituer tout ce qui appartient à cette femme, avec tous les revenus du champ, depuis le jour où elle a quitté le pays jusqu'à maintenant.
\par 7 Élisée se rendit à Damas. Ben Hadad, roi de Syrie, était malade; et on l'avertit, en disant: L'homme de Dieu est arrivé ici.
\par 8 Le roi dit à Hazaël: Prends avec toi un présent, et va au-devant de l'homme de Dieu; consulte par lui l'Éternel, en disant: Guérirai-je de cette maladie?
\par 9 Hazaël alla au-devant d'Élisée, prenant avec lui un présent, tout ce qu'il y avait de meilleur à Damas, la charge de quarante chameaux. Lorsqu'il fut arrivé, il se présenta à lui, et dit: Ton fils Ben Hadad, roi de Syrie, m'envoie vers toi pour dire: Guérirai-je de cette maladie?
\par 10 Élisée lui répondit: Va, dis-lui: Tu guériras! Mais l'Éternel m'a révélé qu'il mourra.
\par 11 L'homme de Dieu arrêta son regard sur Hazaël, et le fixa longtemps, puis il pleura.
\par 12 Hazaël dit: Pourquoi mon seigneur pleure-t-il? Et Élisée répondit: Parce que je sais le mal que tu feras aux enfants d'Israël; tu mettras le feu à leurs villes fortes, tu tueras avec l'épée leurs jeunes gens, tu écraseras leurs petits enfants, et tu fendras le ventre de leurs femmes enceintes.
\par 13 Hazaël dit: Mais qu'est-ce que ton serviteur, ce chien, pour faire de si grandes choses? Et Élisée dit: L'Éternel m'a révélé que tu seras roi de Syrie.
\par 14 Hazaël quitta Élisée, et revint auprès de son maître, qui lui dit: Que t'a dit Élisée? Et il répondit: Il m'a dit: Tu guériras!
\par 15 Le lendemain, Hazaël prit une couverture, qu'il plongea dans l'eau, et il l'étendit sur le visage du roi, qui mourut. Et Hazaël régna à sa place.
\par 16 La cinquième année de Joram, fils d'Achab, roi d'Israël, Joram, fils de Josaphat, roi de Juda, régna.
\par 17 Il avait trente-deux ans lorsqu'il devint roi, et il régna huit ans à Jérusalem.
\par 18 Il marcha dans la voie des rois d'Israël, comme avait fait la maison d'Achab, car il avait pour femme une fille d'Achab, et il fit ce qui est mal aux yeux de l'Éternel.
\par 19 Mais l'Éternel ne voulut point détruire Juda, à cause de David, son serviteur, selon la promesse qu'il lui avait faite de lui donner toujours une lampe parmi ses fils.
\par 20 De son temps, Édom se révolta contre l'autorité de Juda, et se donna un roi.
\par 21 Joram passa à Tsaïr, avec tous ses chars; s'étant levé de nuit, il battit les Édomites, qui l'entouraient et les chefs des chars, et le peuple s'enfuit dans ses tentes.
\par 22 La rébellion d'Édom contre l'autorité de Juda a duré jusqu'à ce jour. Libna se révolta aussi dans le même temps.
\par 23 Le reste des actions de Joram, et tout ce qu'il a fait, cela n'est-il pas écrit dans le livre des Chroniques des rois de Juda?
\par 24 Joram se coucha avec ses pères, et il fut enterré avec ses pères dans la ville de David. Et Achazia, son fils, régna à sa place.
\par 25 La douzième année de Joram, fils d'Achab, roi d'Israël, Achazia, fils de Joram, roi de Juda, régna.
\par 26 Achazia avait vingt-deux ans lorsqu'il devint roi, et il régna un an à Jérusalem. Sa mère s'appelait Athalie, fille d'Omri, roi d'Israël.
\par 27 Il marcha dans la voie de la maison d'Achab, et il fit ce qui est mal aux yeux de l'Éternel, comme la maison d'Achab, car il était allié par mariage à la maison d'Achab.
\par 28 Il alla avec Joram, fils d'Achab, à la guerre contre Hazaël, roi de Syrie, à Ramoth en Galaad. Et les Syriens blessèrent Joram.
\par 29 Le roi Joram s'en retourna pour se faire guérir à Jizreel des blessures que les Syriens lui avaient faites à Rama, lorsqu'il se battait contre Hazaël, roi de Syrie. Achazia, fils de Joram, roi de Juda, descendit pour voir Joram, fils d'Achab, à Jizreel, parce qu'il était malade.

\chapter{9}

\par 1 Élisée, le prophète, appela l'un des fils des prophètes, et lui dit: Ceins tes reins, prends avec toi cette fiole d'huile, et va à Ramoth en Galaad.
\par 2 Quand tu y seras arrivé, vois Jéhu, fils de Josaphat, fils de Nimschi. Tu iras le faire lever du milieu de ses frères, et tu le conduiras dans une chambre retirée.
\par 3 Tu prendras la fiole d'huile, que tu répandras sur sa tête, et tu diras: Ainsi parle l'Éternel: Je t'oins roi d'Israël! Puis tu ouvriras la porte, et tu t'enfuiras sans t'arrêter.
\par 4 Le jeune homme, serviteur du prophète, partit pour Ramoth en Galaad.
\par 5 Quand il arriva, voici, les chefs de l'armée étaient assis. Il dit: Chef, j'ai un mot à te dire. Et Jéhu dit: Auquel de nous tous? Il répondit: A toi, chef.
\par 6 Jéhu se leva et entra dans la maison, et le jeune homme répandit l'huile sur sa tête, en lui disant: Ainsi parle l'Éternel, le Dieu d'Israël: Je t'oins roi d'Israël, du peuple de l'Éternel.
\par 7 Tu frapperas la maison d'Achab, ton maître, et je vengerai sur Jézabel le sang de mes serviteurs les prophètes et le sang de tous les serviteurs de l'Éternel.
\par 8 Toute la maison d'Achab périra; j'exterminerai quiconque appartient à Achab, celui qui est esclave et celui qui est libre en Israël,
\par 9 et je rendrai la maison d'Achab semblable à la maison de Jéroboam, fils de Nebath, et à la maison de Baescha, fils d'Achija.
\par 10 Les chiens mangeront Jézabel dans le champ de Jizreel, et il n'y aura personne pour l'enterrer. Puis le jeune homme ouvrit la porte, et s'enfuit.
\par 11 Lorsque Jéhu sortit pour rejoindre les serviteurs de son maître, on lui dit: Tout va-t-il bien? Pourquoi ce fou est-il venu vers toi? Jéhu leur répondit: Vous connaissez bien l'homme et ce qu'il peut dire.
\par 12 Mais ils répliquèrent: Mensonge! Réponds-nous donc! Et il dit: Il m'a parlé de telle et telle manière, disant: Ainsi parle l'Éternel: Je t'oins roi d'Israël.
\par 13 Aussitôt ils prirent chacun leurs vêtements, qu'ils mirent sous Jéhu au haut des degrés; ils sonnèrent de la trompette, et dirent: Jéhu est roi!
\par 14 Ainsi Jéhu, fils de Josaphat, fils de Nimschi, forma une conspiration contre Joram. -Or Joram et tout Israël défendaient Ramoth en Galaad contre Hazaël, roi de Syrie;
\par 15 mais le roi Joram s'en était retourné pour se faire guérir à Jizreel des blessures que les Syriens lui avaient faites, lorsqu'il se battait contre Hazaël, roi de Syrie. -Jéhu dit: Si c'est votre volonté, personne ne s'échappera de la ville pour aller porter la nouvelle à Jizreel.
\par 16 Et Jéhu monta sur son char et partit pour Jizreel, car Joram y était alité, et Achazia, roi de Juda, était descendu pour le visiter.
\par 17 La sentinelle placée sur la tour de Jizreel vit venir la troupe de Jéhu, et dit: Je vois une troupe. Joram dit: Prends un cavalier, et envoie-le au-devant d'eux pour demander si c'est la paix.
\par 18 Le cavalier alla au-devant de Jéhu, et dit: Ainsi parle le roi: Est-ce la paix? Et Jéhu répondit: Que t'importe la paix? Passe derrière moi. La sentinelle en donna avis, et dit: Le messager est allé jusqu'à eux, et il ne revient pas.
\par 19 Joram envoya un second cavalier, qui arriva vers eux et dit: Ainsi parle le roi: Est-ce la paix? Et Jéhu répondit: Que t'importe la paix? Passe derrière moi.
\par 20 La sentinelle en donna avis, et dit: Il est allé jusqu'à eux, et il ne revient pas. Et le train est comme celui de Jéhu, fils de Nimschi, car il conduit d'une manière insensée.
\par 21 Alors Joram dit: Attelle! Et on attela son char. Joram, roi d'Israël, et Achazia, roi de Juda, sortirent chacun dans son char pour aller au-devant de Jéhu, et ils le rencontrèrent dans le champ de Naboth de Jizreel.
\par 22 Dès que Joram vit Jéhu, il dit: Est-ce la paix, Jéhu? Jéhu répondit: Quoi, la paix! tant que durent les prostitutions de Jézabel, ta mère, et la multitude de ses sortilèges!
\par 23 Joram tourna bride et s'enfuit, et il dit à Achazia: Trahison, Achazia!
\par 24 Mais Jéhu saisit son arc, et il frappa Joram entre les épaules: la flèche sortit par le coeur, et Joram s'affaissa dans son char.
\par 25 Jéhu dit à son officier Bidkar: Prends-le, et jette-le dans le champ de Naboth de Jizreel; car souviens-t'en, lorsque moi et toi, nous étions ensemble à cheval derrière Achab, son père, l'Éternel prononça contre lui cette sentence:
\par 26 J'ai vu hier le sang de Naboth et le sang de ses fils, dit l'Éternel, et je te rendrai la pareille dans ce champ même, dit l'Éternel! Prends-le donc, et jette-le dans le champ, selon la parole de l'Éternel.
\par 27 Achazia, roi de Juda, ayant vu cela, s'enfuit par le chemin de la maison du jardin. Jéhu le poursuivit, et dit: Lui aussi, frappez-le sur le char! Et on le frappa à la montée de Gur, près de Jibleam. Il se réfugia à Meguiddo, et il y mourut.
\par 28 Ses serviteurs le transportèrent sur un char à Jérusalem, et ils l'enterrèrent dans son sépulcre avec ses pères, dans la ville de David.
\par 29 Achazia était devenu roi de Juda la onzième année de Joram, fils d'Achab.
\par 30 Jéhu entra dans Jizreel. Jézabel, l'ayant appris, mit du fard à ses yeux, se para la tête, et regarda par la fenêtre.
\par 31 Comme Jéhu franchissait la porte, elle dit: Est-ce la paix, nouveau Zimri, assassin de son maître?
\par 32 Il leva le visage vers la fenêtre, et dit: Qui est pour moi? qui? Et deux ou trois eunuques le regardèrent en s'approchant de la fenêtre.
\par 33 Il dit: Jetez-la en bas! Ils la jetèrent, et il rejaillit de son sang sur la muraille et sur les chevaux. Jéhu la foula aux pieds;
\par 34 puis il entra, mangea et but, et il dit: Allez voir cette maudite, et enterrez-la, car elle est fille de roi.
\par 35 Ils allèrent pour l'enterrer; mais ils ne trouvèrent d'elle que le crâne, les pieds et les paumes des mains.
\par 36 Ils retournèrent l'annoncer à Jéhu, qui dit: C'est ce qu'avait déclaré l'Éternel par son serviteur Élie, le Thischbite, en disant: Les chiens mangeront la chair de Jézabel dans le camp de Jizreel;
\par 37 et le cadavre de Jézabel sera comme du fumier sur la face des champs, dans le champ de Jizreel, de sorte qu'on ne pourra dire: C'est Jézabel.

\chapter{10}

\par 1 Il y avait dans Samarie soixante-dix fils d'Achab. Jéhu écrivit des lettres qu'il envoya à Samarie aux chefs de Jizreel, aux anciens, et aux gouverneurs des enfants d'Achab. Il y était dit:
\par 2 Maintenant, quand cette lettre vous sera parvenue, -puisque vous avez avec vous les fils de votre maître, avec vous les chars et les chevaux, une ville forte et les armes, -
\par 3 voyez lequel des fils de votre maître est le meilleur et convient le mieux, mettez-le sur le trône de son père, et combattez pour la maison de votre maître!
\par 4 Ils eurent une très grande peur, et ils dirent: Voici, deux rois n'ont pu lui résister; comment résisterions-nous?
\par 5 Et le chef de la maison, le chef de la ville, les anciens, et les gouverneurs des enfants, envoyèrent dire à Jéhu: Nous sommes tes serviteurs, et nous ferons tout ce que tu nous diras; nous n'établirons personne roi, fais ce qui te semble bon.
\par 6 Jéhu leur écrivit une seconde lettre où il était dit: Si vous êtes à moi et si vous obéissez à ma voix, prenez les têtes de ces hommes, fils de votre maître, et venez auprès de moi demain à cette heure, à Jizreel. Or les soixante-dix fils du roi étaient chez les grands de la ville, qui les élevaient.
\par 7 Quand la lettre leur fut parvenue, ils prirent les fils du roi, et ils égorgèrent ces soixante-dix hommes; puis ils mirent leurs têtes dans des corbeilles, et les envoyèrent à Jéhu, à Jizreel.
\par 8 Le messager vint l'en informer, en disant: Ils ont apporté les têtes des fils du roi. Et il dit: Mettez-les en deux tas à l'entrée de la porte, jusqu'au matin.
\par 9 Le matin, il sortit; et se présentant à tout le peuple, il dit: Vous êtes justes! voici, moi, j'ai conspiré contre mon maître et je l'ai tué; mais qui a frappé tous ceux-ci?
\par 10 Sachez donc qu'il ne tombera rien à terre de la parole de l'Éternel, de la parole que l'Éternel a prononcée contre la maison d'Achab; l'Éternel accomplit ce qu'il a déclaré par son serviteur Élie.
\par 11 Et Jéhu frappa tous ceux qui restaient de la maison d'Achab à Jizreel, tous ses grands, ses familiers et ses ministres, sans en laisser échapper un seul.
\par 12 Puis il se leva, et partit pour aller à Samarie. Arrivé à une maison de réunion des bergers, sur le chemin,
\par 13 Jéhu trouva les frères d'Achazia, roi de Juda, et il dit: Qui êtes-vous? Ils répondirent: Nous sommes les frères d'Achazia, et nous descendons pour saluer les fils du roi et les fils de la reine.
\par 14 Jéhu dit: Saisissez-les vivants. Et ils les saisirent vivants, et les égorgèrent au nombre de quarante-deux, à la citerne de la maison de réunion; Jéhu n'en laissa échapper aucun.
\par 15 Étant parti de là, il rencontra Jonadab, fils de Récab, qui venait au-devant de lui. Il le salua, et lui dit: Ton coeur est-il sincère, comme mon coeur l'est envers le tien? Et Jonadab répondit: Il l'est. S'il l'est, répliqua Jéhu, donne-moi ta main. Jonadab lui donna la main. Et Jéhu le fit monter auprès de lui dans son char,
\par 16 et dit: Viens avec moi, et tu verras mon zèle pour l'Éternel. Il l'emmena ainsi dans son char.
\par 17 Lorsque Jéhu fut arrivé à Samarie, il frappa tous ceux qui restaient d'Achab à Samarie, et il les détruisit entièrement, selon la parole que l'Éternel avait dite à Élie.
\par 18 Puis il assembla tout le peuple, et leur dit: Achab a peu servi Baal, Jéhu le servira beaucoup.
\par 19 Maintenant convoquez auprès de moi tous les prophètes de Baal, tous ses serviteurs et tous ses prêtres, sans qu'il en manque un seul, car je veux offrir un grand sacrifice à Baal: quiconque manquera ne vivra pas. Jéhu agissait avec ruse, pour faire périr les serviteurs de Baal.
\par 20 Il dit: Publiez une fête en l'honneur de Baal. Et ils la publièrent.
\par 21 Il envoya des messagers dans tout Israël; et tous les serviteurs de Baal arrivèrent, il n'y en eut pas un qui ne vînt; ils entrèrent dans la maison de Baal, et la maison de Baal fut remplie d'un bout à l'autre.
\par 22 Jéhu dit à celui qui avait la garde du vestiaire: Sors des vêtements pour tous les serviteurs de Baal. Et cet homme sortit des vêtements pour eux.
\par 23 Alors Jéhu vint à la maison de Baal avec Jonadab, fils de Récab, et il dit aux serviteurs de Baal: Cherchez et regardez, afin qu'il n'y ait pas ici des serviteurs de l'Éternel, mais qu'il y ait seulement des serviteurs de Baal.
\par 24 Et ils entrèrent pour offrir des sacrifices et des holocaustes. Jéhu avait placé dehors quatre-vingts hommes, en leur disant: Celui qui laissera échapper quelqu'un des hommes que je remets entre vos mains, sa vie répondra de la sienne.
\par 25 Lorsqu'on eut achevé d'offrir les holocaustes, Jéhu dit aux coureurs et aux officiers: Entrez, frappez-les, que pas un ne sorte. Et ils les frappèrent du tranchant de l'épée. Les coureurs et les officiers les jetèrent là, et ils allèrent jusqu'à la ville de la maison de Baal.
\par 26 Ils tirèrent dehors les statues de la maison de Baal, et les brûlèrent.
\par 27 Ils renversèrent la statue de Baal, ils renversèrent aussi la maison de Baal, et ils en firent un cloaque, qui a subsisté jusqu'à ce jour.
\par 28 Jéhu extermina Baal du milieu d'Israël;
\par 29 mais il ne se détourna point des péchés de Jéroboam, fils de Nebath, qui avait fait pécher Israël, il n'abandonna point les veaux d'or qui étaient à Béthel et à Dan.
\par 30 L'Éternel dit à Jéhu: Parce que tu as bien exécuté ce qui était droit à mes yeux, et que tu as fait à la maison d'Achab tout ce qui était conforme à ma volonté, tes fils jusqu'à la quatrième génération seront assis sur le trône d'Israël.
\par 31 Toutefois Jéhu ne prit point garde à marcher de tout son coeur dans la loi de l'Éternel, le Dieu d'Israël; il ne se détourna point des péchés que Jéroboam avait fait commettre à Israël.
\par 32 Dans ce temps-là, l'Éternel commença à entamer le territoire d'Israël; et Hazaël les battit sur toute la frontière d'Israël.
\par 33 Depuis le Jourdain, vers le soleil levant, il battit tout le pays de Galaad, les Gadites, les Rubénites et les Manassites, depuis Aroër sur le torrent de l'Arnon jusqu'à Galaad et à Basan.
\par 34 Le reste des actions de Jéhu, tout ce qu'il a fait, et tous ses exploits, cela n'est-il pas écrit dans le livre des Chroniques des rois d'Israël?
\par 35 Jéhu se coucha avec ses pères, et on l'enterra à Samarie. Et Joachaz, son fils, régna à sa place.
\par 36 Jéhu avait régné vingt-huit ans sur Israël à Samarie.

\chapter{11}

\par 1 Athalie, mère d'Achazia, voyant que son fils était mort, se leva et fit périr toute la race royale.
\par 2 Mais Joschéba, fille du roi Joram, soeur d'Achazia, prit Joas, fils d'Achazia, et l'enleva du milieu des fils du roi, quand on les fit mourir: elle le mit avec sa nourrice dans la chambre des lits. Il fut ainsi dérobé aux regards d'Athalie, et ne fut point mis à mort.
\par 3 Il resta six ans caché avec Joschéba dans la maison de l'Éternel. Et c'était Athalie qui régnait dans le pays.
\par 4 La septième année, Jehojada envoya chercher les chefs de centaines des Kéréthiens et des coureurs, et il les fit venir auprès de lui dans la maison de l'Éternel. Il traita alliance avec eux et les fit jurer dans la maison de l'Éternel, et il leur montra le fils du roi.
\par 5 Puis il leur donna ses ordres, en disant: Voici ce que vous ferez. Parmi ceux de vous qui entrent en service le jour du sabbat, un tiers doit monter la garde à la maison du roi,
\par 6 un tiers à la porte de Sur, et un tiers à la porte derrière les coureurs: vous veillerez à la garde de la maison, de manière à en empêcher l'entrée.
\par 7 Vos deux autres divisions, tous ceux qui sortent de service le jour du sabbat feront la garde de la maison de l'Éternel auprès du roi:
\par 8 vous entourerez le roi de toutes parts, chacun les armes à la main, et l'on donnera la mort à quiconque s'avancera dans les rangs; vous serez près du roi quand il sortira et quand il entrera.
\par 9 Les chefs de centaines exécutèrent tous les ordres qu'avait donnés le sacrificateur Jehojada. Ils prirent chacun leurs gens, ceux qui entraient en service et ceux qui sortaient de service le jour du sabbat, et ils se rendirent vers le sacrificateur Jehojada.
\par 10 Le sacrificateur remit aux chefs de centaines les lances et les boucliers qui provenaient du roi David, et qui se trouvaient dans la maison de l'Éternel.
\par 11 Les coureurs, chacun les armes à la main, entourèrent le roi, en se plaçant depuis le côté droit jusqu'au côté gauche de la maison, près de l'autel et près de la maison.
\par 12 Le sacrificateur fit avancer le fils du roi, et il mit sur lui le diadème et le témoignage. Ils l'établirent roi et l'oignirent, et frappant des mains, ils dirent: Vive le roi!
\par 13 Athalie entendit le bruit des coureurs et du peuple, et elle vint vers le peuple à la maison de l'Éternel.
\par 14 Elle regarda. Et voici, le roi se tenait sur l'estrade, selon l'usage; les chefs et les trompettes étaient près du roi: tout le peuple du pays était dans la joie, et l'on sonnait des trompettes. Athalie déchira ses vêtements, et cria: Conspiration! conspiration!
\par 15 Alors le sacrificateur Jehojada donna cet ordre aux chefs de centaines, qui étaient à la tête de l'armée: Faites-la sortir en dehors des rangs, et tuez par l'épée quiconque la suivra. Car le sacrificateur avait dit: Qu'elle ne soit pas mise à mort dans la maison de l'Éternel!
\par 16 On lui fit place, et elle se rendit à la maison du roi par le chemin de l'entrée des chevaux: c'est là qu'elle fut tuée.
\par 17 Jehojada traita entre l'Éternel, le roi et le peuple, l'alliance par laquelle ils devaient être le peuple de l'Éternel; il établit aussi l'alliance entre le roi et le peuple.
\par 18 Tout le peuple du pays entra dans la maison de Baal, et ils la démolirent; ils brisèrent entièrement ses autels et ses images, et ils tuèrent devant les autels Matthan, prêtre de Baal. Le sacrificateur Jehojada mit des surveillants dans la maison de l'Éternel.
\par 19 Il prit les chefs de centaines, les Kéréthiens et les coureurs, et tout le peuple du pays; et ils firent descendre le roi de la maison de l'Éternel, et ils entrèrent dans la maison du roi par le chemin de la porte des coureurs. Et Joas s'assit sur le trône des rois.
\par 20 Tout le peuple du pays se réjouissait, et la ville était tranquille. On avait fait mourir Athalie par l'épée dans la maison du roi.
\par 21 Joas avait sept ans lorsqu'il devint roi.

\chapter{12}

\par 1 La septième année de Jéhu, Joas devint roi, et il régna quarante ans à Jérusalem. Sa mère s'appelait Tsibja, de Beer Schéba.
\par 2 Joas fit ce qui est droit aux yeux de l'Éternel tout le temps qu'il suivit les directions du sacrificateur Jehojada.
\par 3 Seulement, les hauts lieux ne disparurent point; le peuple offrait encore des sacrifices et des parfums sur les hauts lieux.
\par 4 Joas dit aux sacrificateurs: Tout l'argent consacré qu'on apporte dans la maison de l'Éternel, l'argent ayant cours, savoir l'argent pour le rachat des personnes d'après l'estimation qui en est faite, et tout l'argent qu'il vient au coeur de quelqu'un d'apporter à la maison de l'Éternel,
\par 5 que les sacrificateurs le prennent chacun de la part des gens de sa connaissance, et qu'ils l'emploient à réparer la maison partout où il se trouvera quelque chose à réparer.
\par 6 Mais il arriva que, la vingt-troisième année du roi Joas, les sacrificateurs n'avaient point réparé ce qui était à réparer à la maison.
\par 7 Le roi Joas appela le sacrificateur Jehojada et les autres sacrificateurs, et leur dit: Pourquoi n'avez-vous pas réparé ce qui est à réparer à la maison? Maintenant, vous ne prendrez plus l'argent de vos connaissances, mais vous le livrerez pour les réparations de la maison.
\par 8 Les sacrificateurs convinrent de ne pas prendre l'argent du peuple, et de n'être chargés des réparations de la maison.
\par 9 Alors le sacrificateur Jehojada prit un coffre, perça un trou dans son couvercle, et le plaça à côté de l'autel, à droite, sur le passage par lequel on entrait à la maison de l'Éternel. Les sacrificateurs qui avaient la garde du seuil y mettaient tout l'argent qu'on apportait dans la maison de l'Éternel.
\par 10 Quand ils voyaient qu'il y avait beaucoup d'argent dans le coffre, le secrétaire du roi montait avec le souverain sacrificateur, et ils serraient et comptaient l'argent qui se trouvait dans la maison de l'Éternel.
\par 11 Ils remettaient l'argent pesé entre les mains de ceux qui étaient chargés de faire exécuter l'ouvrage dans la maison de l'Éternel. Et l'on employait cet argent pour les charpentiers et pour les ouvriers qui travaillaient à la maison de l'Éternel,
\par 12 pour les maçons et les tailleurs de pierres, pour les achats de bois et de pierres de taille nécessaires aux réparations de la maison de l'Éternel, et pour toutes les dépenses concernant les réparations de la maison.
\par 13 Mais, avec l'argent qu'on apportait dans la maison de l'Éternel, on ne fit pour la maison de l'Éternel ni bassins d'argent, ni couteaux, ni coupes, ni trompettes, ni aucun ustensile d'or ou d'argent:
\par 14 on le donnait à ceux qui faisaient l'ouvrage, afin qu'ils l'employassent à réparer la maison de l'Éternel.
\par 15 On ne demandait pas de compte aux hommes entre les mains desquels on remettait l'argent pour qu'ils le donnassent à ceux qui faisaient l'ouvrage, car ils agissaient avec probité.
\par 16 L'argent des sacrifices de culpabilité et des sacrifices d'expiation n'était point apporté dans la maison de l'Éternel: il était pour les sacrificateurs.
\par 17 Alors Hazaël, roi de Syrie, monta et se battit contre Gath, dont il s'empara. Hazaël avait l'intention de monter contre Jérusalem.
\par 18 Joas, roi de Juda, prit toutes les choses consacrées, ce qui avait été consacré par Josaphat, par Joram et par Achazia, ses pères, rois de Juda, ce qu'il avait consacré lui-même, et tout l'or qui se trouvait dans les trésors de la maison de l'Éternel et de la maison du roi, et il envoya le tout à Hazaël, roi de Syrie, qui ne monta pas contre Jérusalem.
\par 19 Le reste des actions de Joas, et tout ce qu'il a fait, cela n'est-il pas écrit dans le livre des Chroniques des rois de Juda?
\par 20 Ses serviteurs se soulevèrent et formèrent une conspiration; ils frappèrent Joas dans la maison de Millo, qui est à la descente de Silla.
\par 21 Jozacar, fils de Schimeath, et Jozabad, fils de Schomer, ses serviteurs, le frappèrent, et il mourut. On l'enterra avec ses pères, dans la ville de David. Et Amatsia, son fils, régna à sa place.

\chapter{13}

\par 1 La vingt-troisième année de Joas, fils d'Achazia, roi de Juda, Joachaz, fils de Jéhu, régna sur Israël à Samarie. Il régna dix-sept ans.
\par 2 Il fit ce qui est mal aux yeux de l'Éternel; il commit les mêmes péchés que Jéroboam, fils de Nebath, qui avait fait pécher Israël, et il ne s'en détourna point.
\par 3 La colère de l'Éternel s'enflamma contre Israël, et il les livra entre les mains de Hazaël, roi de Syrie, et entre les mains de Ben Hadad, fils de Hazaël, tout le temps que ces rois vécurent.
\par 4 Joachaz implora l'Éternel. L'Éternel l'exauça, car il vit l'oppression sous laquelle le roi de Syrie tenait Israël,
\par 5 et l'Éternel donna un libérateur à Israël. Les enfants d'Israël échappèrent aux mains des Syriens, et ils habitèrent dans leurs tentes comme auparavant.
\par 6 Mais ils ne se détournèrent point des péchés de la maison de Jéroboam, qui avait fait pécher Israël; ils s'y livrèrent aussi, et même l'idole d'Astarté était debout à Samarie.
\par 7 De tout le peuple de Joachaz l'Éternel ne lui avait laissé que cinquante cavaliers, dix chars, et dix mille hommes de pied; car le roi de Syrie les avait fait périr et les avait rendus semblables à la poussière qu'on foule aux pieds.
\par 8 Le reste des actions de Joachaz, tout ce qu'il a fait, et ses exploits, cela n'est-il pas écrit dans le livre des Chroniques des rois d'Israël?
\par 9 Joachaz se coucha avec ses pères, et on l'enterra à Samarie. Et Joas, son fils, régna à sa place.
\par 10 La trente-septième année de Joas, roi de Juda, Joas, fils de Joachaz, régna sur Israël à Samarie. Il régna seize ans.
\par 11 Il fit ce qui est mal aux yeux de l'Éternel; il ne se détourna d'aucun des péchés de Jéroboam, fils de Nebath, qui avait fait pécher Israël, et il s'y livra comme lui.
\par 12 Le reste des actions de Joas, tout ce qu'il a fait, ses exploits, et la guerre qu'il eut avec Amatsia, roi de Juda, cela n'est-il pas écrit dans le livre des Chroniques des rois d'Israël?
\par 13 Joas se coucha avec ses pères. Et Jéroboam s'assit sur son trône. Joas fut enterré à Samarie avec les rois d'Israël.
\par 14 Élisée était atteint de la maladie dont il mourut; et Joas, roi d'Israël, descendit vers lui, pleura sur son visage, et dit: Mon père! mon père! Char d'Israël et sa cavalerie!
\par 15 Élisée lui dit: Prends un arc et des flèches. Et il prit un arc et des flèches.
\par 16 Puis Élisée dit au roi d'Israël: Bande l'arc avec ta main. Et quand il l'eut bandé de sa main, Élisée mit ses mains sur les mains du roi,
\par 17 et il dit: Ouvre la fenêtre à l'orient. Et il l'ouvrit. Élisée dit: Tire. Et il tira. Élisée dit: C'est une flèche de délivrance de la part de l'Éternel, une flèche de délivrance contre les Syriens; tu battras les Syriens à Aphek jusqu'à leur extermination.
\par 18 Élisée dit encore: Prends les flèches. Et il les prit. Élisée dit au roi d'Israël: Frappe contre terre. Et il frappa trois fois, et s'arrêta.
\par 19 L'homme de Dieu s'irrita contre lui, et dit: Il fallait frapper cinq ou six fois; alors tu aurais battu les Syriens jusqu'à leur extermination; maintenant tu les battras trois fois.
\par 20 Élisée mourut, et on l'enterra. L'année suivante, des troupes de Moabites pénétrèrent dans le pays.
\par 21 Et comme on enterrait un homme, voici, on aperçut une de ces troupes, et l'on jeta l'homme dans le sépulcre d'Élisée. L'homme alla toucher les os d'Élisée, et il reprit vie et se leva sur ses pieds.
\par 22 Hazaël, roi de Syrie, avait opprimé Israël pendant toute la vie de Joachaz.
\par 23 Mais l'Éternel leur fit miséricorde et eut compassion d'eux, il tourna sa face vers eux à cause de son alliance avec Abraham, Isaac et Jacob, il ne voulut pas les détruire, et jusqu'à présent il ne les a pas rejetés de sa face.
\par 24 Hazaël, roi de Syrie, mourut, et Ben Hadad, son fils, régna à sa place.
\par 25 Joas, fils de Joachaz, reprit des mains de Ben Hadad, fils de Hazaël, les villes enlevées par Hazaël à Joachaz, son père, pendant la guerre. Joas le battit trois fois, et il recouvra les villes d'Israël.

\chapter{14}

\par 1 La seconde année de Joas, fils de Joachaz, roi d'Israël, Amatsia, fils de Joas, roi de Juda, régna.
\par 2 Il avait vingt-cinq ans lorsqu'il devint roi, et il régna vingt-neuf ans à Jérusalem. Sa mère s'appelait Joaddan, de Jérusalem.
\par 3 Il fit ce qui est droit aux yeux de l'Éternel, non pas toutefois comme David, son père; il agit entièrement comme avait agi Joas, son père.
\par 4 Seulement, les hauts-lieux ne disparurent point; le peuple offrait encore des sacrifices et des parfums sur les hauts-lieux.
\par 5 Lorsque la royauté fut affermie entre ses mains, il frappa ses serviteurs qui avaient tué le roi, son père.
\par 6 Mais il ne fit pas mourir les fils des meurtriers, selon ce qui est écrit dans le livre de la loi de Moïse, où l'Éternel donne ce commandement: On ne fera point mourir les pères pour les enfants, et l'on ne fera point mourir les enfants pour les pères; mais on fera mourir chacun pour son péché.
\par 7 Il battit dix mille Édomites dans la vallée du sel; et durant la guerre, il prit Séla, et l'appela Joktheel, nom qu'elle a conservé jusqu'à ce jour.
\par 8 Alors Amatsia envoya des messagers à Joas, fils de Joachaz, fils de Jéhu, roi d'Israël, pour lui dire: Viens, voyons-nous en face!
\par 9 Et Joas, roi d'Israël, fit dire à Amatsia, roi de Juda: L'épine du Liban envoya dire au cèdre du Liban: Donne ta fille pour femme à mon fils! Et les bêtes sauvages qui sont au Liban passèrent et foulèrent l'épine.
\par 10 Tu as battu les Édomites, et ton coeur s'élève. Jouis de ta gloire, et reste chez toi. Pourquoi t'engager dans une malheureuse entreprise, qui amènerait ta ruine et celle de Juda?
\par 11 Mais Amatsia ne l'écouta pas. Et Joas, roi d'Israël, monta; et ils se virent en face, lui et Amatsia, roi de Juda, à Beth Schémesch, qui est à Juda.
\par 12 Juda fut battu par Israël, et chacun s'enfuit dans sa tente.
\par 13 Joas, roi d'Israël, prit à Beth Schémesch Amatsia, roi de Juda, fils de Joas, fils d'Achazia. Il vint à Jérusalem, et fit une brèche de quatre cents coudées dans la muraille de Jérusalem, depuis la porte d'Éphraïm jusqu'à la porte de l'angle.
\par 14 Il prit tout l'or et l'argent et tous les vases qui se trouvaient dans la maison de l'Éternel et dans les trésors de la maison du roi; il prit aussi des otages, et il retourna à Samarie.
\par 15 Le reste des actions de Joas, ce qu'il a fait, ses exploits, et la guerre qu'il eut avec Amatsia, roi de Juda, cela n'est-il pas écrit dans le livre des Chroniques des rois d'Israël?
\par 16 Joas se coucha avec ses pères, et il fut enterré à Samarie avec les rois d'Israël. Et Jéroboam, son fils, régna à sa place.
\par 17 Amatsia, fils de Joas, roi de Juda, vécut quinze ans après la mort de Joas, fils de Joachaz, roi d'Israël.
\par 18 Le reste des actions d'Amatsia, cela n'est-il pas écrit dans le livre des Chroniques des rois de Juda?
\par 19 On forma contre lui une conspiration à Jérusalem, et il s'enfuit à Lakis; mais on le poursuivit à Lakis, où on le fit mourir.
\par 20 On le transporta sur des chevaux, et il fut enterré à Jérusalem avec ses pères, dans la ville de David.
\par 21 Et tout le peuple de Juda prit Azaria, âgé de seize ans, et l'établit roi à la place de son père Amatsia.
\par 22 Azaria rebâtit Élath et la fit rentrer sous la puissance de Juda, après que le roi fut couché avec ses pères.
\par 23 La quinzième année d'Amatsia, fils de Joas, roi de Juda, Jéroboam, fils de Joas, roi d'Israël, régna à Samarie. Il régna quarante et un ans.
\par 24 Il fit ce qui est mal aux yeux de l'Éternel; il ne se détourna d'aucun des péchés de Jéroboam, fils de Nebath, qui avait fait pécher Israël.
\par 25 Il rétablit les limites d'Israël depuis l'entrée de Hamath jusqu'à la mer de la plaine, selon la parole que l'Éternel, le Dieu d'Israël, avait prononcée par son serviteur Jonas, le prophète, fils d'Amitthaï, de Gath Hépher.
\par 26 Car l'Éternel vit l'affliction d'Israël à son comble et l'extrémité à laquelle se trouvaient réduits esclaves et hommes libres, sans qu'il y eût personne pour venir au secours d'Israël.
\par 27 Or l'Éternel n'avait point résolu d'effacer le nom d'Israël de dessous les cieux, et il les délivra par Jéroboam, fils de Joas.
\par 28 Le reste des actions de Jéroboam, tout ce qu'il a fait, ses exploits à la guerre, et comment il fit rentrer sous la puissance d'Israël Damas et Hamath qui avaient appartenu à Juda, cela n'est-il pas écrit dans le livre des Chroniques des rois d'Israël?
\par 29 Jéroboam se coucha avec ses pères, avec les rois d'Israël. Et Zacharie, son fils, régna à sa place.

\chapter{15}

\par 1 La vingt-septième année de Jéroboam, roi d'Israël, Azaria, fils d'Amatsia, roi de Juda, régna.
\par 2 Il avait seize ans lorsqu'il devint roi, et il régna cinquante-deux ans à Jérusalem. Sa mère s'appelait Jecolia, de Jérusalem.
\par 3 Il fit ce qui est droit aux yeux de l'Éternel, entièrement comme avait fait Amatsia, son père.
\par 4 Seulement, les hauts lieux ne disparurent point; le peuple offrait encore des sacrifices et des parfums sur les hauts lieux.
\par 5 L'Éternel frappa le roi, qui fut lépreux jusqu'au jour de sa mort et demeura dans une maison écartée. Et Jotham, fils du roi, était à la tête de la maison et jugeait le peuple du pays.
\par 6 Le reste des actions d'Azaria, et tout ce qu'il a fait, cela n'est-il pas écrit dans le livre des Chroniques des rois de Juda?
\par 7 Azaria se coucha avec ses pères, et on l'enterra avec ses pères dans la ville de David. Et Jotham, son fils, régna à sa place.
\par 8 La trente-huitième année d'Azaria, roi de Juda, Zacharie, fils de Jéroboam, régna sur Israël à Samarie. Il régna six mois.
\par 9 Il fit ce qui est mal aux yeux de l'Éternel, comme avaient fait ses pères; il ne se détourna point des péchés de Jéroboam, fils de Nebath, qui avait fait pécher Israël.
\par 10 Schallum, fils de Jabesch, conspira contre lui, le frappa devant le peuple, et le fit mourir; et il régna à sa place.
\par 11 Le reste des actions de Zacharie, cela est écrit dans le livre des Chroniques des rois d'Israël.
\par 12 Ainsi s'accomplit ce que l'Éternel avait déclaré à Jéhu, en disant: Tes fils jusqu'à la quatrième génération seront assis sur le trône d'Israël.
\par 13 Schallum, fils de Jabesch, régna la trente-neuvième année d'Ozias, roi de Juda. Il régna pendant un mois à Samarie.
\par 14 Menahem, fils de Gadi, monta de Thirtsa et vint à Samarie, frappa dans Samarie Schallum, fils de Jabesch, et le fit mourir; et il régna à sa place.
\par 15 Le reste des actions de Schallum, et la conspiration qu'il forma, cela est écrit dans le livre des Chroniques des rois d'Israël.
\par 16 Alors Menahem frappa Thiphsach et tous ceux qui y étaient, avec son territoire depuis Thirtsa; il la frappa parce qu'elle n'avait pas ouvert ses portes, et il fendit le ventre de toutes les femmes enceintes.
\par 17 La trente-neuvième année d'Azaria, roi de Juda, Menahem, fils de Gadi, régna sur Israël. Il régna dix ans à Samarie.
\par 18 Il fit ce qui est mal aux yeux de l'Éternel; il ne se détourna point, tant qu'il vécut, des péchés de Jéroboam, fils de Nebath, qui avait fait pécher Israël.
\par 19 Pul, roi d'Assyrie, vint dans le pays; et Menahem donna à Pul mille talents d'argent, pour qu'il aidât à affermir la royauté entre ses mains.
\par 20 Menahem leva cet argent sur tous ceux d'Israël qui avaient de la richesse, afin de le donner au roi d'Assyrie; il les taxa chacun à cinquante sicles d'argent. Le roi d'Assyrie s'en retourna, et ne s'arrêta pas alors dans le pays.
\par 21 Le reste des actions de Menahem, et tout ce qu'il a fait, cela n'est-il pas écrit dans le livre des Chroniques des rois d'Israël?
\par 22 Menahem se coucha avec ses pères. Et Pekachia, son fils, régna à sa place.
\par 23 La cinquantième année d'Azaria, roi de Juda, Pekachia, fils de Menahem, régna sur Israël à Samarie. Il régna deux ans.
\par 24 Il fit ce qui est mal aux yeux de l'Éternel; il ne se détourna point des péchés de Jéroboam, fils de Nebath, qui avait fait pécher Israël.
\par 25 Pékach, fils de Remalia, son officier, conspira contre lui; il le frappa à Samarie, dans le palais de la maison du roi, de même qu'Argob et Arié; il avait avec lui cinquante hommes d'entre les fils des Galaadites. Il fit ainsi mourir Pekachia, et il régna à sa place.
\par 26 Le reste des actions de Pekachia, et tout ce qu'il a fait, cela est écrit dans le livre des Chroniques des rois d'Israël.
\par 27 La cinquante-deuxième année d'Azaria, roi de Juda, Pékach, fils de Remalia, régna sur Israël à Samarie. Il régna vingt ans.
\par 28 Il fit ce qui est mal aux yeux de l'Éternel; il ne se détourna point des péchés de Jéroboam, fils de Nebath, qui avait fait pécher Israël.
\par 29 Du temps de Pékach, roi d'Israël, Tiglath Piléser, roi d'Assyrie, vint et prit Ijjon, Abel Beth Maaca, Janoach, Kédesch, Hatsor, Galaad et la Galilée, tout le pays de Nephthali, et il emmena captifs les habitants en Assyrie.
\par 30 Osée, fils d'Éla, forma une conspiration contre Pékach, fils de Remalia, le frappa et le fit mourir; et il régna à sa place, la vingtième année de Jotham, fils d'Ozias.
\par 31 Le reste des actions de Pékach, et tout ce qu'il a fait, cela est écrit dans le livre des Chroniques des rois d'Israël.
\par 32 La seconde année de Pékach, fils de Remalia, roi d'Israël, Jotham, fils d'Ozias, roi de Juda, régna.
\par 33 Il avait vingt-cinq ans lorsqu'il devint roi, et il régna seize ans à Jérusalem. Sa mère s'appelait Jeruscha, fille de Tsadok.
\par 34 Il fit ce qui est droit aux yeux de l'Éternel; il agit entièrement comme avait agi Ozias, son père.
\par 35 Seulement, les hauts lieux ne disparurent point; le peuple offrait encore des sacrifices et des parfums sur les hauts lieux. Jotham bâtit la porte supérieure de la maison de l'Éternel.
\par 36 Le reste des actions de Jotham, et tout ce qu'il a fait, cela n'est-il pas écrit dans le livre des Chroniques des rois de Juda?
\par 37 Dans ce temps-là, l'Éternel commença à envoyer contre Juda Retsin, roi de Syrie, et Pékach, fils de Remalia.
\par 38 Jotham se coucha avec ses pères, et il fut enterré avec ses pères dans la ville de David, son père. Et Achaz, son fils, régna à sa place.

\chapter{16}

\par 1 La dix-septième année de Pékach, fils de Remalia, Achaz, fils de Jotham, roi de Juda, régna.
\par 2 Achaz avait vingt ans lorsqu'il devint roi, et il régna seize ans à Jérusalem. Il ne fit point ce qui est droit aux yeux de l'Éternel, son Dieu, comme avait fait David, son père.
\par 3 Il marcha dans la voie des rois d'Israël; et même il fit passer son fils par le feu, suivant les abominations des nations que l'Éternel avait chassées devant les enfants d'Israël.
\par 4 Il offrait des sacrifices et des parfums sur les hauts lieux, sur les collines et sous tout arbre vert.
\par 5 Alors Retsin, roi de Syrie, et Pékach, fils de Remalia, roi d'Israël, montèrent contre Jérusalem pour l'attaquer. Ils assiégèrent Achaz; mais ils ne purent pas le vaincre.
\par 6 Dans ce même temps, Retsin, roi de Syrie, fit rentrer Élath au pouvoir des Syriens; il expulsa d'Élath les Juifs, et les Syriens vinrent à Élath, où ils ont habité jusqu'à ce jour.
\par 7 Achaz envoya des messagers à Tiglath Piléser, roi d'Assyrie, pour lui dire: Je suis ton serviteur et ton fils; monte, et délivre-moi de la main du roi de Syrie et de la main du roi d'Israël, qui s'élèvent contre moi.
\par 8 Et Achaz prit l'argent et l'or qui se trouvaient dans la maison de l'Éternel et dans les trésors de la maison du roi, et il l'envoya en présent au roi d'Assyrie.
\par 9 Le roi d'Assyrie l'écouta; il monta contre Damas, la prit, emmena les habitants en captivité à Kir, et fit mourir Retsin.
\par 10 Le roi Achaz se rendit à Damas au-devant de Tiglath Piléser, roi d'Assyrie. Et ayant vu l'autel qui était à Damas, le roi Achaz envoya au sacrificateur Urie le modèle et la forme exacte de cet autel.
\par 11 Le sacrificateur Urie construisit un autel entièrement d'après le modèle envoyé de Damas par le roi Achaz, et le sacrificateur Urie le fit avant que le roi Achaz fût de retour de Damas.
\par 12 A son arrivée de Damas, le roi vit l'autel, s'en approcha et y monta:
\par 13 il fit brûler son holocauste et son offrande, versa ses libations, et répandit sur l'autel le sang de ses sacrifices d'actions de grâces.
\par 14 Il éloigna de la face de la maison l'autel d'airain qui était devant l'Éternel, afin qu'il ne fût pas entre le nouvel autel et la maison de l'Éternel; et il le plaça à côté du nouvel autel, vers le nord.
\par 15 Et le roi Achaz donna cet ordre au sacrificateur Urie: Fais brûler sur le grand autel l'holocauste du matin et l'offrande du soir, l'holocauste du roi et son offrande, les holocaustes de tout le peuple du pays et leurs offrandes, verses-y leurs libations, et répands-y tout le sang des holocaustes et tout le sang des sacrifices; pour ce qui concerne l'autel d'airain, je m'en occuperai.
\par 16 Le sacrificateur Urie se conforma à tout ce que le roi Achaz avait ordonné.
\par 17 Et le roi Achaz brisa les panneaux des bases, et en ôta les bassins qui étaient dessus. Il descendit la mer de dessus les boeufs d'airain qui étaient sous elle, et il la posa sur un pavé de pierres.
\par 18 Il changea dans la maison de l'Éternel, à cause du roi d'Assyrie, le portique du sabbat qu'on y avait bâti et l'entrée extérieure du roi.
\par 19 Le reste des actions d'Achaz, et tout ce qu'il a fait, cela n'est-il pas écrit dans le livre des Chroniques des rois de Juda?
\par 20 Achaz se coucha avec ses pères, et il fut enterré avec ses pères dans la ville de David. Et Ézéchias, son fils, régna à sa place.

\chapter{17}

\par 1 La douzième année d'Achaz, roi de Juda, Osée, fils d'Éla, régna sur Israël à Samarie. Il régna neuf ans.
\par 2 Il fit ce qui est mal aux yeux de l'Éternel, non pas toutefois comme les rois d'Israël qui avaient été avant lui.
\par 3 Salmanasar, roi d'Assyrie, monta contre lui; et Osée lui fut assujetti, et lui paya un tribut.
\par 4 Mais le roi d'Assyrie découvrit une conspiration chez Osée, qui avait envoyé des messagers à So, roi d'Égypte, et qui ne payait plus annuellement le tribut au roi d'Assyrie. Le roi d'Assyrie le fit enfermer et enchaîner dans une prison.
\par 5 Et le roi d'Assyrie parcourut tout le pays, et monta contre Samarie, qu'il assiégea pendant trois ans.
\par 6 La neuvième année d'Osée, le roi d'Assyrie prit Samarie, et emmena Israël captif en Assyrie. Il les fit habiter à Chalach, et sur le Chabor, fleuve de Gozan, et dans les villes des Mèdes.
\par 7 Cela arriva parce que les enfants d'Israël péchèrent contre l'Éternel, leur Dieu, qui les avait fait monter du pays d'Égypte, de dessous la main de Pharaon, roi d'Égypte, et parce qu'ils craignirent d'autres dieux.
\par 8 Ils suivirent les coutumes des nations que l'Éternel avait chassés devant les enfants d'Israël, et celles que les rois d'Israël avaient établies.
\par 9 Les enfants d'Israël firent en secret contre l'Éternel, leur Dieu, des choses qui ne sont pas bien. Ils se bâtirent des hauts lieux dans toutes leurs villes, depuis les tours des gardes jusqu'aux villes fortes.
\par 10 Ils se dressèrent des statues et des idoles sur toute colline élevée et sous tout arbre vert.
\par 11 Et là ils brûlèrent des parfums sur tous les hauts lieux, comme les nations que l'Éternel avait chassées devant eux, et ils firent des choses mauvaises, par lesquelles ils irritèrent l'Éternel.
\par 12 Ils servirent les idoles dont l'Éternel leur avait dit: Vous ne ferez pas cela.
\par 13 L'Éternel fit avertir Israël et Juda par tous ses prophètes, par tous les voyants, et leur dit: Revenez de vos mauvaises voies, et observez mes commandements et mes ordonnances, en suivant entièrement la loi que j'ai prescrite à vos pères et que je vous ai envoyée par mes serviteurs les prophètes.
\par 14 Mais ils n'écoutèrent point, et ils roidirent leur cou, comme leurs pères, qui n'avaient pas cru en l'Éternel, leur Dieu.
\par 15 Ils rejetèrent ses lois, l'alliance qu'il avait faite avec leurs pères, et les avertissements qu'il leur avait adressés. Ils allèrent après des choses de néant et ne furent eux-mêmes que néant, et après les nations qui les entouraient et que l'Éternel leur avait défendu d'imiter.
\par 16 Ils abandonnèrent tous les commandements de l'Éternel, leur Dieu, ils se firent deux veaux en fonte, ils fabriquèrent des idoles d'Astarté, ils se prosternèrent devant toute l'armée des cieux, et ils servirent Baal.
\par 17 Ils firent passer par le feu leurs fils et leurs filles, ils se livrèrent à la divination et aux enchantements, et ils se vendirent pour faire ce qui est mal aux yeux de l'Éternel, afin de l'irriter.
\par 18 Aussi l'Éternel s'est-il fortement irrité contre Israël, et les a-t-il éloignés de sa face. -Il n'est resté que la seule tribu de Juda.
\par 19 Juda même n'avait pas gardé les commandements de l'Éternel, son Dieu, et ils avaient suivi les coutumes établies par Israël. -
\par 20 L'Éternel a rejeté toute la race d'Israël; il les a humiliés, il les a livrés entre les mains des pillards, et il a fini par les chasser loin de sa face.
\par 21 Car Israël s'était détaché de la maison de David, et ils avaient fait roi Jéroboam, fils de Nebath, qui les avait détournés de l'Éternel, et avait fait commettre à Israël un grand péché.
\par 22 Les enfants d'Israël s'étaient livrés à tous les péchés que Jéroboam avait commis; ils ne s'en détournèrent point,
\par 23 jusqu'à ce que l'Éternel eût chassé Israël loin de sa face, comme il l'avait annoncé par tous ses serviteurs les prophètes. Et Israël a été emmené captif loin de son pays en Assyrie, où il est resté jusqu'à ce jour.
\par 24 Le roi d'Assyrie fit venir des gens de Babylone, de Cutha, d'Avva, de Hamath et de Sepharvaïm, et les établit dans les villes de Samarie à la place des enfants d'Israël. Ils prirent possession de Samarie, et ils habitèrent dans ses villes.
\par 25 Lorsqu'ils commencèrent à y habiter, ils ne craignaient pas l'Éternel, et l'Éternel envoya contre eux des lions qui les tuaient.
\par 26 On dit au roi d'Assyrie: Les nations que tu as transportées et établies dans les villes de Samarie ne connaissent pas la manière de servir le dieu du pays, et il a envoyé contre elles des lions qui les font mourir, parce qu'elles ne connaissent pas la manière de servir le dieu du pays.
\par 27 Le roi d'Assyrie donna cet ordre: Faites-y aller l'un des prêtres que vous avez emmenés de là en captivité; qu'il parte pour s'y établir, et qu'il leur enseigne la manière de servir le dieu du pays.
\par 28 Un des prêtres qui avaient été emmenés captifs de Samarie vint s'établir à Béthel, et leur enseigna comment ils devaient craindre l'Éternel.
\par 29 Mais les nations firent chacune leurs dieux dans les villes qu'elles habitaient, et les placèrent dans les maisons des hauts lieux bâties par les Samaritains.
\par 30 Les gens de Babylone firent Succoth Benoth, les gens de Cuth firent Nergal, les gens de Hamath firent Aschima,
\par 31 ceux d'Avva firent Nibchaz et Tharthak; ceux de Sepharvaïm brûlaient leurs enfants par le feu en l'honneur d'Adrammélec et d'Anammélec, dieux de Sepharvaïm.
\par 32 Ils craignaient aussi l'Éternel, et ils se créèrent des prêtres des hauts lieux pris parmi tout le peuple: ces prêtres offraient pour eux des sacrifices dans les maisons des hauts lieux.
\par 33 Ainsi ils craignaient l'Éternel, et ils servaient en même temps leurs dieux d'après la coutume des nations d'où on les avait transportés.
\par 34 Ils suivent encore aujourd'hui leurs premiers usages: ils ne craignent point l'Éternel, et ils ne se conforment ni à leurs lois et à leurs ordonnances, ni à la loi et aux commandements prescrits par l'Éternel aux enfants de Jacob qu'il appela du nom d'Israël.
\par 35 L'Éternel avait fait alliance avec eux, et leur avait donné cet ordre: Vous ne craindrez point d'autres dieux; vous ne vous prosternerez point devant eux, vous ne les servirez point, et vous ne leur offrirez point de sacrifices.
\par 36 Mais vous craindrez l'Éternel, qui vous a fait monter du pays d'Égypte avec une grande puissance et à bras étendu; c'est devant lui que vous vous prosternerez, et c'est à lui que vous offrirez des sacrifices.
\par 37 Vous observerez et mettrez toujours en pratique les préceptes, les ordonnances, la loi et les commandements, qu'il a écrits pour vous, et vous ne craindrez point d'autres dieux.
\par 38 Vous n'oublierez pas l'alliance que j'ai faite avec vous, et vous ne craindrez point d'autres dieux.
\par 39 Mais vous craindrez l'Éternel, votre Dieu; et il vous délivrera de la main de tous vos ennemis.
\par 40 Et ils n'ont point obéi, et ils ont suivi leurs premiers usages.
\par 41 Ces nations craignaient l'Éternel et servaient leurs images; et leurs enfants et les enfants de leurs enfants font jusqu'à ce jour ce que leurs pères ont fait.

\chapter{18}

\par 1 La troisième année d'Osée, fils d'Éla, roi d'Israël, Ézéchias, fils d'Achaz, roi de Juda, régna.
\par 2 Il avait vingt-cinq ans lorsqu'il devint roi, et il régna vingt-neuf ans à Jérusalem. Sa mère s'appelait Abi, fille de Zacharie.
\par 3 Il fit ce qui est droit aux yeux de l'Éternel, entièrement comme avait fait David, son père.
\par 4 Il fit disparaître les hauts lieux, brisa les statues, abattit les idoles, et mit en pièces le serpent d'airain que Moïse avait fait, car les enfants d'Israël avaient jusqu'alors brûlé des parfums devant lui: on l'appelait Nehuschtan.
\par 5 Il mit sa confiance en l'Éternel, le Dieu d'Israël; et parmi tous les rois de Juda qui vinrent après lui ou qui le précédèrent, il n'y en eut point de semblable à lui.
\par 6 Il fut attaché à l'Éternel, il ne se détourna point de lui, et il observa les commandements que l'Éternel avait prescrits à Moïse.
\par 7 Et l'Éternel fut avec Ézéchias, qui réussit dans toutes ses entreprises. Il se révolta contre le roi d'Assyrie, et ne lui fut plus assujetti.
\par 8 Il battit les Philistins jusqu'à Gaza, et ravagea leur territoire depuis les tours des gardes jusqu'aux villes fortes.
\par 9 La quatrième année du roi Ézéchias, qui était la septième année d'Osée, fils d'Éla, roi d'Israël, Salmanasar, roi d'Assyrie, monta contre Samarie et l'assiégea.
\par 10 Il la prit au bout de trois ans, la sixième année d'Ézéchias, qui était la neuvième année d'Osée, roi d'Israël: alors Samarie fut prise.
\par 11 Le roi d'Assyrie emmena Israël captif en Assyrie, et il les établit à Chalach, et sur le Chabor, fleuve de Gozan, et dans les villes des Mèdes,
\par 12 parce qu'ils n'avaient point écouté la voix de l'Éternel, leur Dieu, et qu'ils avaient transgressé son alliance, parce qu'ils n'avaient ni écouté ni mis en pratique tout ce qu'avait ordonné Moïse, serviteur de l'Éternel.
\par 13 La quatorzième année du roi Ézéchias, Sanchérib, roi d'Assyrie, monta contre toutes les villes fortes de Juda, et s'en empara.
\par 14 Ézéchias, roi de Juda, envoya dire au roi d'Assyrie à Lakis: J'ai commis une faute! Éloigne-toi de moi. Ce que tu m'imposeras, je le supporterai. Et le roi d'Assyrie imposa à Ézéchias, roi de Juda, trois cents talents d'argent et trente talents d'or.
\par 15 Ézéchias donna tout l'argent qui se trouvait dans la maison de l'Éternel et dans les trésors de la maison du roi.
\par 16 Ce fut alors qu'Ézéchias, roi de Juda, enleva, pour les livrer au roi d'Assyrie, les lames d'or dont il avait couvert les portes et les linteaux du temple de l'Éternel.
\par 17 Le roi d'Assyrie envoya de Lakis à Jérusalem, vers le roi Ézéchias, Tharthan, Rab Saris et Rabschaké avec une puissante armée. Ils montèrent, et ils arrivèrent à Jérusalem. Lorsqu'ils furent montés et arrivés, ils s'arrêtèrent à l'aqueduc de l'étang supérieur, sur le chemin du champ du foulon.
\par 18 Ils appelèrent le roi; et Éliakim, fils de Hilkija, chef de la maison du roi, se rendit auprès d'eux, avec Schebna, le secrétaire, et Joach, fils d'Asaph, l'archiviste.
\par 19 Rabschaké leur dit: Dites à Ézéchias: Ainsi parle le grand roi, le roi d'Assyrie: Quelle est cette confiance, sur laquelle tu t'appuies?
\par 20 Tu as dit: Il faut pour la guerre de la prudence et de la force. Mais ce ne sont que des paroles en l'air. En qui donc as-tu placé ta confiance, pour t'être révolté contre moi?
\par 21 Voici, tu l'as placée dans l'Égypte, tu as pris pour soutien ce roseau cassé, qui pénètre et perce la main de quiconque s'appuie dessus: tel est Pharaon, roi d'Égypte, pour tous ceux qui se confient en lui.
\par 22 Peut-être me direz-vous: C'est en l'Éternel, notre Dieu, que nous nous confions. Mais n'est-ce pas lui dont Ézéchias a fait disparaître les hauts lieux et les autels, en disant à Juda et à Jérusalem: Vous vous prosternerez devant cet autel à Jérusalem?
\par 23 Maintenant, fais une convention avec mon maître, le roi d'Assyrie, et je te donnerai deux mille chevaux, si tu peux fournir des cavaliers pour les monter.
\par 24 Comment repousserais-tu un seul chef d'entre les moindres serviteurs de mon maître? Tu mets ta confiance dans l'Égypte pour les chars et pour les cavaliers.
\par 25 D'ailleurs, est-ce sans la volonté de l'Éternel que je suis monté contre ce lieu, pour le détruire? L'Éternel m'a dit: Monte contre ce pays, et détruis-le.
\par 26 Éliakim, fils de Hilkija, Schebna et Joach, dirent à Rabschaké: Parle à tes serviteurs en araméen, car nous le comprenons; et ne nous parle pas en langue judaïque, aux oreilles du peuple qui est sur la muraille.
\par 27 Rabschaké leur répondit: Est-ce à ton maître et à toi que mon maître m'a envoyé dire ces paroles? N'est-ce pas à ces hommes assis sur la muraille pour manger leurs excréments et pour boire leur urine avec vous?
\par 28 Alors Rabschaké, s'étant avancé, cria à haute voix en langue judaïque, et dit: Écoutez la parole du grand roi, du roi d'Assyrie!
\par 29 Ainsi parle le roi: Qu'Ézéchias ne vous abuse point, car il ne pourra vous délivrer de ma main.
\par 30 Qu'Ézéchias ne vous amène point à vous confier en l'Éternel, en disant: L'Éternel nous délivrera, et cette ville ne sera pas livrée entre les mains du roi d'Assyrie.
\par 31 N'écoutez point Ézéchias; car ainsi parle le roi d'Assyrie: Faites la paix avec moi, rendez-vous à moi, et chacun de vous mangera de sa vigne et de son figuier, et chacun boira de l'eau de sa citerne,
\par 32 jusqu'à ce que je vienne, et que je vous emmène dans un pays comme le vôtre, dans un pays de blé et de vin, un pays de pain et de vignes, un pays d'oliviers à huile et de miel, et vous vivrez et vous ne mourrez point. N'écoutez donc point Ézéchias; car il pourrait vous séduire en disant: L'Éternel nous délivrera.
\par 33 Les dieux des nations ont-ils délivré chacun son pays de la main du roi d'Assyrie?
\par 34 Où sont les dieux de Hamath et d'Arpad? Où sont les dieux de Sepharvaïm, d'Héna et d'Ivva? Ont-ils délivré Samarie de ma main?
\par 35 Parmi tous les dieux de ces pays, quels sont ceux qui ont délivré leur pays de ma main, pour que l'Éternel délivre Jérusalem de ma main?
\par 36 Le peuple se tut, et ne lui répondit pas un mot; car le roi avait donné cet ordre: Vous ne lui répondrez pas.
\par 37 Et Éliakim, fils de Hilkija, chef de la maison du roi, Schebna, le secrétaire, et Joach, fils d'Asaph, l'archiviste, vinrent auprès d'Ézéchias, les vêtements déchirés, et lui rapportèrent les paroles de Rabschaké.

\chapter{19}

\par 1 Lorsque le roi Ézéchias eut entendu cela, il déchira ses vêtements, se couvrit d'un sac, et alla dans la maison de l'Éternel.
\par 2 Il envoya Éliakim, chef de la maison du roi, Schebna, le secrétaire, et les plus anciens des sacrificateurs, couverts de sacs, vers Ésaïe, le prophète, fils d'Amots.
\par 3 Et ils lui dirent: Ainsi parle Ézéchias: Ce jour est un jour d'angoisse, de châtiment et d'opprobre; car les enfants sont près de sortir du sein maternel, et il n'y a point de force pour l'enfantement.
\par 4 Peut-être l'Éternel, ton Dieu, a-t-il entendu toutes les paroles de Rabschaké, que le roi d'Assyrie, son maître, a envoyé pour insulter au Dieu vivant, et peut-être l'Éternel, ton Dieu, exercera-t-il ses châtiments à cause des paroles qu'il a entendues. Fais donc monter une prière pour le reste qui subsiste encore.
\par 5 Les serviteurs du roi Ézéchias allèrent donc auprès d'Ésaïe.
\par 6 Et Ésaïe leur dit: Voici ce que vous direz à votre maître: Ainsi parle l'Éternel: Ne t'effraie point des paroles que tu as entendues et par lesquelles m'ont outragé les serviteurs du roi d'Assyrie.
\par 7 Je vais mettre en lui un esprit tel que, sur une nouvelle qu'il recevra, il retournera dans son pays; et je le ferai tomber par l'épée dans son pays.
\par 8 Rabschaké, s'étant retiré, trouva le roi d'Assyrie qui attaquait Libna, car il avait appris son départ de Lakis.
\par 9 Alors le roi d'Assyrie reçut une nouvelle au sujet de Tirhaka, roi d'Éthiopie; on lui dit: Voici, il s'est mis en marche pour te faire la guerre. Et le roi d'Assyrie envoya de nouveau des messagers à Ézéchias, en disant:
\par 10 Vous parlerez ainsi à Ézéchias, roi de Juda: Que ton Dieu, auquel tu te confies, ne t'abuse point en disant: Jérusalem ne sera pas livrée entre les mains du roi d'Assyrie.
\par 11 Voici, tu as appris ce qu'ont fait les rois d'Assyrie à tous les pays, et comment ils les ont détruits; et toi, tu serais délivré!
\par 12 Les dieux des nations que mes pères ont détruites les ont-ils délivrées, Gozan, Charan, Retseph, et les fils d'Éden qui sont à Telassar?
\par 13 Où sont le roi de Hamath, le roi d'Arpad, et le roi de la ville de Sepharvaïm, d'Héna et d'Ivva?
\par 14 Ézéchias prit la lettre de la main des messagers, et la lut. Puis il monta à la maison de l'Éternel, et la déploya devant l'Éternel,
\par 15 à qui il adressa cette prière: Éternel, Dieu d'Israël, assis sur les chérubins! C'est toi qui es le seul Dieu de tous les royaumes de la terre, c'est toi qui as fait les cieux et la terre.
\par 16 Éternel! incline ton oreille, et écoute. Éternel! ouvre tes yeux, et regarde. Entends les paroles de Sanchérib, qui a envoyé Rabschaké pour insulter au Dieu vivant.
\par 17 Il est vrai, ô Éternel! que les rois d'Assyrie ont détruit les nations et ravagé leurs pays,
\par 18 et qu'ils ont jeté leurs dieux dans le feu; mais ce n'étaient point des dieux, c'étaient des ouvrages de mains d'homme, du bois et de la pierre; et ils les ont anéantis.
\par 19 Maintenant, Éternel, notre Dieu! délivre-nous de la main de Sanchérib, et que tous les royaumes de la terre sachent que toi seul es Dieu, ô Éternel!
\par 20 Alors Ésaïe, fils d'Amots, envoya dire à Ézéchias: Ainsi parle l'Éternel, le Dieu d'Israël: J'ai entendu la prière que tu m'as adressée au sujet de Sanchérib, roi d'Assyrie.
\par 21 Voici la parole que l'Éternel a prononcée contre lui: Elle te méprise, elle se moque de toi, La vierge, fille de Sion; Elle hoche la tête après toi, La fille de Jérusalem.
\par 22 Qui as-tu insulté et outragé? Contre qui as-tu élevé la voix? Tu as porté tes yeux en haut Sur le Saint d'Israël!
\par 23 Par tes messagers tu as insulté le Seigneur, Et tu as dit: Avec la multitude de mes chars, J'ai gravi le sommet des montagnes, Les extrémités du Liban; Je couperai les plus élevés de ses cèdres, Les plus beaux de ses cyprès, Et j'atteindrai sa dernière cime, Sa forêt semblable à un verger;
\par 24 J'ai creusé, et j'ai bu des eaux étrangères, Et je tarirai avec la plante de mes pieds Tous les fleuves de l'Égypte.
\par 25 N'as-tu pas appris que j'ai préparé ces choses de loin, Et que je les ai résolues dès les temps anciens? Maintenant j'ai permis qu'elles s'accomplissent, Et que tu réduisisses des villes fortes en monceaux de ruines.
\par 26 Leurs habitants sont impuissants, Épouvantés et confus; Ils sont comme l'herbe des champs et la tendre verdure, Comme le gazon des toits Et le blé qui sèche avant la formation de sa tige.
\par 27 Mais je sais quand tu t'assieds, quand tu sors et quand tu entres, Et quand tu es furieux contre moi.
\par 28 Parce que tu es furieux contre moi, Et que ton arrogance est montée à mes oreilles, Je mettrai ma boucle à tes narines et mon mors entre tes lèvres, Et je te ferai retourner par le chemin par lequel tu es venu.
\par 29 Que ceci soit un signe pour toi: On a mangé une année le produit du grain tombé, et une seconde année ce qui croît de soi-même; mais la troisième année, vous sèmerez, vous moissonnerez, vous planterez des vignes, et vous en mangerez le fruit.
\par 30 Ce qui aura été sauvé de la maison de Juda, ce qui sera resté poussera encore des racines par-dessous, et portera du fruit par-dessus.
\par 31 Car de Jérusalem il sortira un reste, et de la montagne de Sion des réchappés. Voilà ce que fera le zèle de l'Éternel des armées.
\par 32 C'est pourquoi ainsi parle l'Éternel sur le roi d'Assyrie: Il n'entrera point dans cette ville, Il n'y lancera point de traits, Il ne lui présentera point de boucliers, Et il n'élèvera point de retranchements contre elle.
\par 33 Il s'en retournera par le chemin par lequel il est venu, Et il n'entrera point dans cette ville, dit l'Éternel.
\par 34 Je protégerai cette ville pour la sauver, A cause de moi, et à cause de David, mon serviteur.
\par 35 Cette nuit-là, l'ange de l'Éternel sortit, et frappa dans le camp des Assyriens cent quatre-vingt-cinq mille hommes. Et quand on se leva le matin, voici, c'étaient tous des corps morts.
\par 36 Alors Sanchérib, roi d'Assyrie, leva son camp, partit et s'en retourna; et il resta à Ninive.
\par 37 Or, comme il était prosterné dans la maison de Nisroc, son dieu, Adrammélec et Scharetser, ses fils, le frappèrent avec l'épée, et s'enfuirent au pays d'Ararat. Et Ésar Haddon, son fils, régna à sa place.

\chapter{20}

\par 1 En ce temps-là, Ézéchias fut malade à la mort. Le prophète Ésaïe, fils d'Amots, vint auprès de lui, et lui dit: Ainsi parle l'Éternel: Donne tes ordres à ta maison, car tu vas mourir, et tu ne vivras plus.
\par 2 Ézéchias tourna son visage contre le mur, et fit cette prière à l'Éternel:
\par 3 O Éternel! souviens-toi que j'ai marché devant ta face avec fidélité et intégrité de coeur, et que j'ai fait ce qui est bien à tes yeux! Et Ézéchias répandit d'abondantes larmes.
\par 4 Ésaïe, qui était sorti, n'était pas encore dans la cour du milieu, lorsque la parole de l'Éternel lui fut adressée en ces termes:
\par 5 Retourne, et dis à Ézéchias, chef de mon peuple: Ainsi parle l'Éternel, le Dieu de David, ton père: J'ai entendu ta prière, j'ai vu tes larmes. Voici, je te guérirai; le troisième jour, tu monteras à la maison de l'Éternel.
\par 6 J'ajouterai à tes jours quinze années. Je te délivrerai, toi et cette ville, de la main du roi d'Assyrie; je protégerai cette ville, à cause de moi, et à cause de David, mon serviteur.
\par 7 Ésaïe dit: Prenez une masse de figues. On la prit, et on l'appliqua sur l'ulcère. Et Ézéchias guérit.
\par 8 Ézéchias avait dit à Ésaïe: A quel signe connaîtrai-je que l'Éternel me guérira, et que je monterai le troisième jour à la maison de l'Éternel?
\par 9 Et Ésaïe dit: Voici, de la part de l'Éternel, le signe auquel tu connaîtras que l'Éternel accomplira la parole qu'il a prononcée: L'ombre avancera-t-elle de dix degrés, ou reculera-t-elle de dix degrés?
\par 10 Ézéchias répondit: C'est peu de chose que l'ombre avance de dix degrés; mais plutôt qu'elle recule de dix degrés.
\par 11 Alors Ésaïe, le prophète, invoqua l'Éternel, qui fit reculer l'ombre de dix degrés sur les degrés d'Achaz, où elle était descendue.
\par 12 En ce même temps, Berodac Baladan, fils de Baladan, roi de Babylone, envoya une lettre et un présent à Ézéchias, car il avait appris la maladie d'Ézéchias.
\par 13 Ézéchias donna audience aux envoyés, et il leur montra le lieu où étaient ses choses de prix, l'argent et l'or, les aromates et l'huile précieuse, son arsenal, et tout ce qui se trouvait dans ses trésors: il n'y eut rien qu'Ézéchias ne leur fît voir dans sa maison et dans tous ses domaines.
\par 14 Ésaïe, le prophète, vint ensuite auprès du roi Ézéchias, et lui dit: Qu'ont dit ces gens-là, et d'où sont-ils venus vers toi? Ézéchias répondit: Ils sont venus d'un pays éloigné, de Babylone.
\par 15 Ésaïe dit encore: Qu'ont-ils vu dans ta maison? Ézéchias répondit: Ils ont vu tout ce qui est dans ma maison: il n'y a rien dans mes trésors que je ne leur aie fait voir.
\par 16 Alors Ésaïe dit à Ézéchias: Écoute la parole de l'Éternel!
\par 17 Voici, les temps viendront où l'on emportera à Babylone tout ce qui est dans ta maison et ce que tes pères ont amassé jusqu'à ce jour; il n'en restera rien, dit l'Éternel.
\par 18 Et l'on prendra de tes fils, qui seront sortis de toi, que tu auras engendrés, pour en faire des eunuques dans le palais du roi de Babylone.
\par 19 Ézéchias répondit à Ésaïe: La parole de l'Éternel, que tu as prononcée, est bonne. Et il ajouta: N'y aura-t-il pas paix et sécurité pendant ma vie?
\par 20 Le reste des actions d'Ézéchias, tous ses exploits, et comment il fit l'étang et l'aqueduc, et amena les eaux dans la ville, cela n'est-il pas écrit dans le livre des Chroniques des rois de Juda?
\par 21 Ézéchias se coucha avec ses pères. Et Manassé, son fils, régna à sa place.

\chapter{21}

\par 1 Manassé avait douze ans lorsqu'il devint roi, et il régna cinquante-cinq ans à Jérusalem. Sa mère s'appelait Hephtsiba.
\par 2 Il fit ce qui est mal aux yeux de l'Éternel, selon les abominations des nations que l'Éternel avait chassées devant les enfants d'Israël.
\par 3 Il rebâtit les hauts lieux qu'Ézéchias, son père, avait détruits, il éleva des autels à Baal, il fit une idole d'Astarté, comme avait fait Achab, roi d'Israël, et il se prosterna devant toute l'armée des cieux et la servit.
\par 4 Il bâtit des autels dans la maison de l'Éternel, quoique l'Éternel eût dit: C'est dans Jérusalem que je placerai mon nom.
\par 5 Il bâtit des autels à toute l'armée des cieux dans les deux parvis de la maison de l'Éternel.
\par 6 Il fit passer son fils par le feu; il observait les nuages et les serpents pour en tirer des pronostics, et il établit des gens qui évoquaient les esprits et qui prédisaient l'avenir. Il fit de plus en plus ce qui est mal aux yeux de l'Éternel, afin de l'irriter.
\par 7 Il mit l'idole d'Astarté, qu'il avait faite, dans la maison de laquelle l'Éternel avait dit à David et à Salomon, son fils: C'est dans cette maison, et c'est dans Jérusalem, que j'ai choisie parmi toutes les tribus d'Israël, que je veux à toujours placer mon nom.
\par 8 Je ne ferai plus errer le pied d'Israël hors du pays que j'ai donné à ses pères, pourvu seulement qu'ils aient soin de mettre en pratique tout ce que je leur ai commandé et toute la loi que leur a prescrite mon serviteur Moïse.
\par 9 Mais ils n'obéirent point; et Manassé fut cause qu'ils s'égarèrent et firent le mal plus que les nations que l'Éternel avait détruites devant les enfants d'Israël.
\par 10 Alors l'Éternel parla en ces termes par ses serviteurs les prophètes:
\par 11 Parce que Manassé, roi de Juda, a commis ces abominations, parce qu'il a fait pis que tout ce qu'avaient fait avant lui les Amoréens, et parce qu'il a aussi fait pécher Juda par ses idoles,
\par 12 voici ce que dit l'Éternel, le Dieu d'Israël: Je vais faire venir sur Jérusalem et sur Juda des malheurs qui étourdiront les oreilles de quiconque en entendra parler.
\par 13 J'étendrai sur Jérusalem le cordeau de Samarie et le niveau de la maison d'Achab; et je nettoierai Jérusalem comme un plat qu'on nettoie, et qu'on renverse sens dessus dessous après l'avoir nettoyé.
\par 14 J'abandonnerai le reste de mon héritage, et je les livrerai entre les mains de leurs ennemis; et ils deviendront le butin et la proie de tous leurs ennemis,
\par 15 parce qu'ils ont fait ce qui est mal à mes yeux et qu'ils m'ont irrité depuis le jour où leurs pères sont sortis d'Égypte jusqu'à ce jour.
\par 16 Manassé répandit aussi beaucoup de sang innocent, jusqu'à en remplir Jérusalem d'un bout à l'autre, outre les péchés qu'il commit et qu'il fit commettre à Juda en faisant ce qui est mal aux yeux de l'Éternel.
\par 17 Le reste des actions de Manassé, tout ce qu'il a fait, et les péchés auxquels il se livra, cela n'est-il pas écrit dans le livre des Chroniques des rois de Juda?
\par 18 Manassé se coucha avec ses pères, et il fut enterré dans le jardin de sa maison, dans le jardin d'Uzza. Et Amon, son fils, régna à sa place.
\par 19 Amon avait vingt-deux ans lorsqu'il devint roi, et il régna deux ans à Jérusalem. Sa mère s'appelait Meschullémeth, fille de Haruts, de Jotba.
\par 20 Il fit ce qui est mal aux yeux de l'Éternel, comme avait fait Manassé, son père;
\par 21 il marcha dans toute la voie où avait marché son père, il servit les idoles qu'avait servies son père, et il se prosterna devant elles;
\par 22 il abandonna l'Éternel, le Dieu de ses pères, et il ne marcha point dans la voie de l'Éternel.
\par 23 Les serviteurs d'Amon conspirèrent contre lui, et firent mourir le roi dans sa maison.
\par 24 Mais le peuple du pays frappa tous ceux qui avaient conspiré contre le roi Amon; et le peuple du pays établit roi Josias, son fils, à sa place.
\par 25 Le reste des actions d'Amon, et ce qu'il a fait, cela n'est-il pas écrit dans le livre des Chroniques des rois de Juda?
\par 26 On l'enterra dans son sépulcre, dans le jardin d'Uzza. Et Josias, son fils, régna à sa place.

\chapter{22}

\par 1 Josias avait huit ans lorsqu'il devint roi, et il régna trente et un ans à Jérusalem. Sa mère s'appelait Jedida, fille d'Adaja, de Botskath.
\par 2 Il fit ce qui est droit aux yeux de l'Éternel, et il marcha dans toute la voie de David, son père; il ne s'en détourna ni à droite ni à gauche.
\par 3 La dix-huitième année du roi Josias, le roi envoya dans la maison de l'Éternel Schaphan, le secrétaire, fils d'Atsalia, fils de Meschullam.
\par 4 Il lui dit: Monte vers Hilkija, le souverain sacrificateur, et qu'il amasse l'argent qui a été apporté dans la maison de l'Éternel et que ceux qui ont la garde du seuil ont recueilli du peuple.
\par 5 On remettra cet argent entre les mains de ceux qui sont chargés de faire exécuter l'ouvrage dans la maison de l'Éternel. Et ils l'emploieront pour ceux qui travaillent aux réparations de la maison de l'Éternel,
\par 6 pour les charpentiers, les manoeuvres et les maçons, pour les achats de bois et de pierres de taille nécessaires aux réparations de la maison.
\par 7 Mais on ne leur demandera pas de compte pour l'argent remis entre leurs mains, car ils agissent avec probité.
\par 8 Alors Hilkija, le souverain sacrificateur, dit à Schaphan, le secrétaire: J'ai trouvé le livre de la loi dans la maison de l'Éternel. Et Hilkija donna le livre à Schaphan, et Schaphan le lut.
\par 9 Puis Schaphan, le secrétaire, alla rendre compte au roi, et dit: Tes serviteurs ont amassé l'argent qui se trouvait dans la maison, et l'ont remis entre les mains de ceux qui sont chargés de faire exécuter l'ouvrage dans la maison de l'Éternel.
\par 10 Schaphan, le secrétaire, dit encore au roi: Le sacrificateur Hilkija m'a donné un livre. Et Schaphan le lut devant le roi.
\par 11 Lorsque le roi entendit les paroles du livre de la loi, il déchira ses vêtements.
\par 12 Et le roi donna cet ordre au sacrificateur Hilkija, à Achikam, fils de Schaphan, à Acbor, fils de Michée, à Schaphan, le secrétaire, et à Asaja, serviteur du roi:
\par 13 Allez, consultez l'Éternel pour moi, pour le peuple, et pour tout Juda, au sujet des paroles de ce livre qu'on a trouvé; car grande est la colère de l'Éternel, qui s'est enflammée contre nous, parce que nos pères n'ont point obéi aux paroles de ce livre et n'ont point mis en pratique tout ce qui nous y est prescrit.
\par 14 Le sacrificateur Hilkija, Achikam, Acbor, Schaphan et Asaja, allèrent auprès de la prophétesse Hulda, femme de Schallum, fils de Thikva, fils de Harhas, gardien des vêtements. Elle habitait à Jérusalem, dans l'autre quartier de la ville.
\par 15 Après qu'ils eurent parlé, elle leur dit: Ainsi parle l'Éternel, le Dieu d'Israël: Dites à l'homme qui vous a envoyés vers moi:
\par 16 Ainsi parle l'Éternel: Voici, je vais faire venir des malheurs sur ce lieu et sur ses habitants, selon toutes les paroles du livre qu'a lu le roi de Juda.
\par 17 Parce qu'ils m'ont abandonné et qu'ils ont offert des parfums à d'autres dieux, afin de m'irriter par tous les ouvrages de leurs mains, ma colère s'est enflammée contre ce lieu, et elle ne s'éteindra point.
\par 18 Mais vous direz au roi de Juda, qui vous a envoyés pour consulter l'Éternel: Ainsi parle l'Éternel, le Dieu d'Israël, au sujet des paroles que tu as entendues:
\par 19 Parce que ton coeur a été touché, parce que tu t'es humilié devant l'Éternel en entendant ce que j'ai prononcé contre ce lieu et contre ses habitants, qui seront un objet d'épouvante et de malédiction, et parce que tu as déchiré tes vêtements et que tu as pleuré devant moi, moi aussi, j'ai entendu, dit l'Éternel.
\par 20 C'est pourquoi, voici, je te recueillerai auprès de tes pères, tu seras recueilli en paix dans ton sépulcre, et tes yeux ne verront pas tous les malheurs que je ferai venir sur ce lieu. Ils rapportèrent au roi cette réponse.

\chapter{23}

\par 1 Le roi Josias fit assembler auprès de lui tous les anciens de Juda et de Jérusalem.
\par 2 Puis il monta à la maison de l'Éternel, avec tous les hommes de Juda et tous les habitants de Jérusalem, les sacrificateurs, les prophètes, et tout le peuple, depuis le plus petit jusqu'au plus grand. Il lut devant eux toutes les paroles du livre de l'alliance, qu'on avait trouvé dans la maison de l'Éternel.
\par 3 Le roi se tenait sur l'estrade, et il traita alliance devant l'Éternel, s'engageant à suivre l'Éternel, et à observer ses ordonnances, ses préceptes et ses lois, de tout son coeur et de toute son âme, afin de mettre en pratique les paroles de cette alliance, écrites dans ce livre. Et tout le peuple entra dans l'alliance.
\par 4 Le roi ordonna à Hilkija, le souverain sacrificateur, aux sacrificateurs du second ordre, et à ceux qui gardaient le seuil, de sortir du temple de l'Éternel tous les ustensiles qui avaient été faits pour Baal, pour Astarté, et pour toute l'armée des cieux; et il les brûla hors de Jérusalem, dans les champs du Cédron, et en fit porter la poussière à Béthel.
\par 5 Il chassa les prêtres des idoles, établis par les rois de Juda pour brûler des parfums sur les hauts lieux dans les villes de Juda et aux environs de Jérusalem, et ceux qui offraient des parfums à Baal, au soleil, à la lune, au zodiaque et à toute l'armée des cieux.
\par 6 Il sortit de la maison de l'Éternel l'idole d'Astarté, qu'il transporta hors de Jérusalem vers le torrent de Cédron; il la brûla au torrent de Cédron et la réduisit en poussière, et il en jeta la poussière sur les sépulcres des enfants du peuple.
\par 7 Il abattit les maisons des prostitués qui étaient dans la maison de l'Éternel, et où les femmes tissaient des tentes pour Astarté.
\par 8 Il fit venir tous les prêtres des villes de Juda; il souilla les hauts lieux où les prêtres brûlaient des parfums, depuis Guéba jusqu'à Beer Schéba; et il renversa les hauts lieux des portes, celui qui était à l'entrée de la porte de Josué, chef de la ville, et celui qui était à gauche de la porte de la ville.
\par 9 Toutefois les prêtres des hauts lieux ne montaient pas à l'autel de l'Éternel à Jérusalem, mais ils mangeaient des pains sans levain au milieu de leurs frères.
\par 10 Le roi souilla Topheth dans la vallée des fils de Hinnom, afin que personne ne fît plus passer son fils ou sa fille par le feu en l'honneur de Moloc.
\par 11 Il fit disparaître de l'entrée de la maison de l'Éternel les chevaux que les rois de Juda avaient consacrés au soleil, près de la chambre de l'eunuque Nethan Mélec, qui demeurait dans le faubourg; et il brûla au feu les chars du soleil.
\par 12 Le roi démolit les autels qui étaient sur le toit de la chambre haute d'Achaz et que les rois de Juda avaient faits, et les autels qu'avait faits Manassé dans les deux parvis de la maison de l'Éternel; après les avoir brisés et enlevés de là, il en jeta la poussière dans le torrent de Cédron.
\par 13 Le roi souilla les hauts lieux qui étaient en face de Jérusalem, sur la droite de la montagne de perdition, et que Salomon, roi d'Israël, avait bâtis à Astarté, l'abomination des Sidoniens, à Kemosch, l'abomination de Moab, et à Milcom, l'abomination des fils d'Ammon.
\par 14 Il brisa les statues et abattit les idoles, et il remplit d'ossements d'hommes la place qu'elles occupaient.
\par 15 Il renversa aussi l'autel qui était à Béthel, et le haut lieu qu'avait fait Jéroboam, fils de Nebath, qui avait fait pécher Israël; il brûla le haut lieu et le réduisit en poussière, et il brûla l'idole.
\par 16 Josias, s'étant tourné et ayant vu les sépulcres qui étaient là dans la montagne, envoya prendre les ossements des sépulcres, et il les brûla sur l'autel et le souilla, selon la parole de l'Éternel prononcée par l'homme de Dieu qui avait annoncé ces choses.
\par 17 Il dit: Quel est ce monument que je vois? Les gens de la ville lui répondirent: C'est le sépulcre de l'homme de Dieu, qui est venu de Juda, et qui a crié contre l'autel de Béthel ces choses que tu as accomplies.
\par 18 Et il dit: Laissez-le; que personne ne remue ses os! On conserva ainsi ses os avec les os du prophète qui était venu de Samarie.
\par 19 Josias fit encore disparaître toutes les maisons des hauts lieux, qui étaient dans les villes de Samarie, et qu'avaient faites les rois d'Israël pour irriter l'Éternel; il fit à leur égard entièrement comme il avait fait à Béthel.
\par 20 Il immola sur les autels tous les prêtres des hauts lieux, qui étaient là, et il y brûla des ossements d'hommes. Puis il retourna à Jérusalem.
\par 21 Le roi donna cet ordre à tout le peuple: Célébrez la Pâque en l'honneur de l'Éternel, votre Dieu, comme il est écrit dans ce livre de l'alliance.
\par 22 Aucune Pâque pareille à celle-ci n'avait été célébrée depuis le temps où les juges jugeaient Israël et pendant tous les jours des rois d'Israël et des rois de Juda.
\par 23 Ce fut la dix-huitième année du roi Josias qu'on célébra cette Pâque en l'honneur de l'Éternel à Jérusalem.
\par 24 De plus, Josias fit disparaître ceux qui évoquaient les esprits et ceux qui prédisaient l'avenir, et les théraphim, et les idoles, et toutes les abominations qui se voyaient dans le pays de Juda et à Jérusalem, afin de mettre en pratique les paroles de la loi, écrites dans le livre que le sacrificateur Hilkija avait trouvé dans la maison de l'Éternel.
\par 25 Avant Josias, il n'y eut point de roi qui, comme lui, revînt à l'Éternel de tout son coeur, de toute son âme et de toute sa force, selon toute la loi de Moïse; et après lui, il n'en a point paru de semblable.
\par 26 Toutefois l'Éternel ne se désista point de l'ardeur de sa grande colère dont il était enflammé contre Juda, à cause de tout ce qu'avait fait Manassé pour l'irriter.
\par 27 Et l'Éternel dit: J'ôterai aussi Juda de devant ma face comme j'ai ôté Israël, et je rejetterai cette ville de Jérusalem que j'avais choisie, et la maison de laquelle j'avais dit: Là sera mon nom.
\par 28 Le reste des actions de Josias, et tout ce qu'il a fait, cela n'est-il pas écrit dans le livre des Chroniques des rois de Juda?
\par 29 De son temps, Pharaon Néco, roi d'Égypte, monta contre le roi d'Assyrie, vers le fleuve de l'Euphrate. Le roi Josias marcha à sa rencontre; et Pharaon le tua à Meguiddo, dès qu'il le vit.
\par 30 Ses serviteurs l'emportèrent mort sur un char; ils l'amenèrent de Meguiddo à Jérusalem, et ils l'enterrèrent dans son sépulcre. Et le peuple du pays prit Joachaz, fils de Josias; ils l'oignirent, et le firent roi à la place de son père.
\par 31 Joachaz avait vingt-trois ans lorsqu'il devint roi, et il régna trois mois à Jérusalem. Sa mère s'appelait Hamuthal, fille de Jérémie, de Libna.
\par 32 Il fit ce qui est mal aux yeux de l'Éternel, entièrement comme avaient fait ses pères.
\par 33 Pharaon Néco l'enchaîna à Ribla, dans le pays de Hamath, pour qu'il ne régnât plus à Jérusalem; et il mit sur le pays une contribution de cent talents d'argent et d'un talent d'or.
\par 34 Et Pharaon Néco établit roi Éliakim, fils de Josias, à la place de Josias, son père, et il changea son nom en celui de Jojakim. Il prit Joachaz, qui alla en Égypte et y mourut.
\par 35 Jojakim donna à Pharaon l'argent et l'or; mais il taxa le pays pour fournir cet argent, d'après l'ordre de Pharaon; il détermina la part de chacun et exigea du peuple du pays l'argent et l'or qu'il devait livrer à Pharaon Néco.
\par 36 Jojakim avait vingt-cinq ans lorsqu'il devint roi, et il régna onze ans à Jérusalem. Sa mère s'appelait Zebudda, fille de Pedaja, de Ruma.
\par 37 Il fit ce qui est mal aux yeux de l'Éternel, entièrement comme avaient fait ses pères.

\chapter{24}

\par 1 De son temps, Nebucadnetsar, roi de Babylone, se mit en campagne. Jojakim lui fut assujetti pendant trois ans; mais il se révolta de nouveau contre lui.
\par 2 Alors l'Éternel envoya contre Jojakim des troupes de Chaldéens, des troupes de Syriens, des troupes de Moabites et des troupes d'Ammonites; il les envoya contre Juda pour le détruire, selon la parole que l'Éternel avait prononcée par ses serviteurs les prophètes.
\par 3 Cela arriva uniquement sur l'ordre de l'Éternel, qui voulait ôter Juda de devant sa face, à cause de tous les péchés commis par Manassé,
\par 4 et à cause du sang innocent qu'avait répandu Manassé et dont il avait rempli Jérusalem. Aussi l'Éternel ne voulut-il point pardonner.
\par 5 Le reste des actions de Jojakim, et tout ce qu'il a fait, cela n'est-il pas écrit dans le livre des Chroniques des rois de Juda?
\par 6 Jojakin se coucha avec ses pères. Et Jojakin, son fils, régna à sa place.
\par 7 Le roi d'Égypte ne sortit plus de son pays, car le roi de Babylone avait pris tout ce qui était au roi d'Égypte depuis le torrent d'Égypte jusqu'au fleuve de l'Euphrate.
\par 8 Jojakin avait dix-huit ans lorsqu'il devint roi, et il régna trois mois à Jérusalem. Sa mère s'appelait Nehuschtha, fille d'Elnathan, de Jérusalem.
\par 9 Il fit ce qui est mal aux yeux de l'Éternel, entièrement comme avait fait son père.
\par 10 En ce temps-là, les serviteurs de Nebucadnetsar, roi de Babylone, montèrent contre Jérusalem, et la ville fut assiégée.
\par 11 Nebucadnetsar, roi de Babylone, arriva devant la ville pendant que ses serviteurs l'assiégeaient.
\par 12 Alors Jojakin, roi de Juda, se rendit auprès du roi de Babylone, avec sa mère, ses serviteurs, ses chefs et ses eunuques. Et le roi de Babylone le fit prisonnier, la huitième année de son règne.
\par 13 Il tira de là tous les trésors de la maison de l'Éternel et les trésors de la maison du roi; et il brisa tous les ustensiles d'or que Salomon, roi d'Israël, avait faits dans le temple de l'Éternel, comme l'Éternel l'avait prononcé.
\par 14 Il emmena en captivité tout Jérusalem, tous les chefs et tous les hommes vaillants, au nombre de dix mille exilés, avec tous les charpentiers et les serruriers: il ne resta que le peuple pauvre du pays.
\par 15 Il transporta Jojakin à Babylone; et il emmena captifs de Jérusalem à Babylone la mère du roi, les femmes du roi et ses eunuques, et les grands du pays,
\par 16 tous les guerriers au nombre de sept mille, et les charpentiers et les serruriers au nombre de mille, tous hommes vaillants et propres à la guerre. Le roi de Babylone les emmena captifs à Babylone.
\par 17 Et le roi de Babylone établit roi, à la place de Jojakin, Matthania, son oncle, dont il changea le nom en celui de Sédécias.
\par 18 Sédécias avait vingt et un ans lorsqu'il devint roi, et il régna onze ans à Jérusalem. Sa mère s'appelait Hamuthal, fille de Jérémie, de Libna.
\par 19 Il fit ce qui est mal aux yeux de l'Éternel, entièrement comme avait fait Jojakim.
\par 20 Et cela arriva à cause de la colère de l'Éternel contre Jérusalem et contre Juda, qu'il voulait rejeter de devant sa face. Et Sédécias se révolta contre le roi de Babylone.

\chapter{25}

\par 1 La neuvième année du règne de Sédécias, le dixième jour du dixième mois, Nebucadnetsar, roi de Babylone, vint avec toute son armée contre Jérusalem; il campa devant elle, et éleva des retranchements tout autour.
\par 2 La ville fut assiégée jusqu'à la onzième année du roi Sédécias.
\par 3 Le neuvième jour du mois, la famine était forte dans la ville, et il n'y avait pas de pain pour le peuple du pays.
\par 4 Alors la brèche fut faite à la ville; et tous les gens de guerre s'enfuirent de nuit par le chemin de la porte entre les deux murs près du jardin du roi, pendant que les Chaldéens environnaient la ville. Les fuyards prirent le chemin de la plaine.
\par 5 Mais l'armée des Chaldéens poursuivit le roi et l'atteignit dans les plaines de Jéricho, et toute son armée se dispersa loin de lui.
\par 6 Ils saisirent le roi, et le firent monter vers le roi de Babylone à Ribla; et l'on prononça contre lui une sentence.
\par 7 Les fils de Sédécias furent égorgés en sa présence; puis on creva les yeux à Sédécias, on le lia avec des chaînes d'airain, et on le mena à Babylone.
\par 8 Le septième jour du cinquième mois, -c'était la dix-neuvième année du règne de Nebucadnetsar, roi de Babylone, -Nebuzaradan, chef des gardes, serviteur du roi de Babylone, entra dans Jérusalem.
\par 9 Il brûla la maison de l'Éternel, la maison du roi, et toutes les maisons de Jérusalem; il livra au feu toutes les maisons de quelque importance.
\par 10 Toute l'armée des Chaldéens, qui était avec le chef des gardes, démolit les murailles formant l'enceinte de Jérusalem.
\par 11 Nebuzaradan, chef des gardes, emmena captifs ceux du peuple qui étaient demeurés dans la ville, ceux qui s'étaient rendus au roi de Babylone, et le reste de la multitude.
\par 12 Cependant le chef des gardes laissa comme vignerons et comme laboureurs quelques-uns des plus pauvres du pays.
\par 13 Les Chaldéens brisèrent les colonnes d'airain qui étaient dans la maison de l'Éternel, les bases, la mer d'airain qui était dans la maison de l'Éternel, et ils en emportèrent l'airain à Babylone.
\par 14 Ils prirent les cendriers, les pelles, les couteaux, les tasses, et tous les ustensiles d'airain avec lesquels on faisait le service.
\par 15 Le chef des gardes prit encore les brasiers et les coupes, ce qui était d'or et ce qui était d'argent.
\par 16 Les deux colonnes, la mer, et les bases, que Salomon avait faites pour la maison de l'Éternel, tous ces ustensiles d'airain avaient un poids inconnu.
\par 17 La hauteur d'une colonne était de dix-huit coudées, et il y avait au-dessus un chapiteau d'airain dont la hauteur était de trois coudées; autour du chapiteau il y avait un treillis et des grenades, le tout d'airain; il en était de même pour la seconde colonne avec le treillis.
\par 18 Le chef des gardes prit Seraja, le souverain sacrificateur, Sophonie, le second sacrificateur, et les trois gardiens du seuil.
\par 19 Et dans la ville il prit un eunuque qui avait sous son commandement les gens de guerre, cinq hommes qui faisaient partie des conseillers du roi et qui furent trouvés dans la ville, le secrétaire du chef de l'armée qui était chargé d'enrôler le peuple du pays, et soixante hommes du peuple du pays qui se trouvèrent dans la ville.
\par 20 Nebuzaradan, chef des gardes, les prit, et les conduisit vers le roi de Babylone à Ribla.
\par 21 Le roi de Babylone les frappa et les fit mourir à Ribla, dans le pays de Hamath.
\par 22 Ainsi Juda fut emmené captif loin de son pays. Et Nebucadnetsar, roi de Babylone, plaça le reste du peuple, qu'il laissa dans le pays de Juda, sous le commandement de Guedalia, fils d'Achikam, fils de Schaphan.
\par 23 Lorsque tous les chefs des troupes eurent appris, eux et leurs hommes, que le roi de Babylone avait établi Guedalia pour gouverneur, ils se rendirent auprès de Guedalia à Mitspa, savoir Ismaël, fils de Nethania, Jochanan, fils de Karéach, Seraja, fils de Thanhumeth, de Nethopha, et Jaazania, fils du Maacathien, eux et leurs hommes.
\par 24 Guedalia leur jura, à eux et à leurs hommes, et leur dit: Ne craignez rien de la part des serviteurs des Chaldéens; demeurez dans le pays, servez le roi de Babylone, et vous vous en trouverez bien.
\par 25 Mais au septième mois, Ismaël, fils de Nethania, fils d'Élischama, de la race royale, vint, accompagné de dix hommes, et ils frappèrent mortellement Guedalia, ainsi que les Juifs et les Chaldéens qui étaient avec lui à Mitspa.
\par 26 Alors tout le peuple, depuis le plus petit jusqu'au plus grand, et les chefs des troupes, se levèrent et s'en allèrent en Égypte, parce qu'ils avaient peur des Chaldéens.
\par 27 La trente-septième année de la captivité de Jojakin, roi de Juda, le vingt-septième jour du douzième mois, Évil Merodac, roi de Babylone, dans la première année de son règne, releva la tête de Jojakin, roi de Juda, et le tira de prison.
\par 28 Il lui parla avec bonté, et il mit son trône au-dessus du trône des rois qui étaient avec lui à Babylone.
\par 29 Il lui fit changer ses vêtements de prison, et Jojakin mangea toujours à sa table tout le temps de sa vie.
\par 30 Le roi pourvut constamment à son entretien journalier tout le temps de sa vie.


\end{document}