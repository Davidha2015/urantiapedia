\begin{document}

\title{Néhémie}


\chapter{1}

\par 1 Paroles de Néhémie, fils de Hacalia. Au mois de Kisleu, la vingtième année, comme j'étais à Suse, dans la capitale,
\par 2 Hanani, l'un de mes frères, et quelques hommes arrivèrent de Juda. Je les questionnai au sujet des Juifs réchappés qui étaient restés de la captivité, et au sujet de Jérusalem.
\par 3 Ils me répondirent: Ceux qui sont restés de la captivité sont là dans la province, au comble du malheur et de l'opprobre; les murailles de Jérusalem sont en ruines, et ses portes sont consumées par le feu.
\par 4 Lorsque j'entendis ces choses, je m'assis, je pleurai, et je fus plusieurs jours dans la désolation. Je jeûnai et je priai devant le Dieu des cieux,
\par 5 et je dis: O Éternel, Dieu des cieux, Dieu grand et redoutable, toi qui gardes ton alliance et qui fais miséricorde à ceux qui t'aiment et qui observent tes commandements!
\par 6 Que ton oreille soit attentive et que tes yeux soient ouverts: écoute la prière que ton serviteur t'adresse en ce moment, jour et nuit, pour tes serviteurs les enfants d'Israël, en confessant les péchés des enfants d'Israël, nos péchés contre toi; car moi et la maison de mon père, nous avons péché.
\par 7 Nous t'avons offensé, et nous n'avons point observé les commandements, les lois et les ordonnances que tu prescrivis à Moïse, ton serviteur.
\par 8 Souviens-toi de cette parole que tu donnas ordre à Moïse, ton serviteur, de prononcer. Lorsque vous pécherez, je vous disperserai parmi les peuples;
\par 9 mais si vous revenez à moi, et si vous observez mes commandements et les mettez en pratique, alors, quand vous seriez exilés à l'extrémité du ciel, de là je vous rassemblerai et je vous ramènerai dans le lieu que j'ai choisi pour y faire résider mon nom.
\par 10 Ils sont tes serviteurs et ton peuple, que tu as rachetés par ta grande puissance et par ta main forte.
\par 11 Ah! Seigneur, que ton oreille soit attentive à la prière de ton serviteur, et à la prière de tes serviteurs qui veulent craindre ton nom! Donne aujourd'hui du succès à ton serviteur, et fais-lui trouver grâce devant cet homme! J'étais alors échanson du roi.

\chapter{2}

\par 1 Au mois de Nisan, la vingtième année du roi Artaxerxès, comme le vin était devant lui, je pris le vin et je l'offris au roi. Jamais je n'avais paru triste en sa présence.
\par 2 Le roi me dit: Pourquoi as-tu mauvais visage? Tu n'es pourtant pas malade; ce ne peut être qu'un chagrin de coeur. Je fus saisi d'une grande crainte,
\par 3 et je répondis au roi: Que le roi vive éternellement! Comment n'aurais-je pas mauvais visage, lorsque la ville où sont les sépulcres de mes pères est détruite et que ses portes sont consumées par le feu?
\par 4 Et le roi me dit: Que demandes-tu? Je priai le Dieu des cieux,
\par 5 et je répondis au roi: Si le roi le trouve bon, et si ton serviteur lui est agréable, envoie-moi en Juda, vers la ville des sépulcres de mes pères, pour que je la rebâtisse.
\par 6 Le roi, auprès duquel la reine était assise, me dit alors: Combien ton voyage durera-t-il, et quand seras-tu de retour? Il plut au roi de me laisser partir, et je lui fixai un temps.
\par 7 Puis je dis au roi: Si le roi le trouve bon, qu'on me donne des lettres pour les gouverneurs de l'autre côté du fleuve, afin qu'ils me laissent passer et entrer en Juda,
\par 8 et une lettre pour Asaph, garde forestier du roi, afin qu'il me fournisse du bois de charpente pour les portes de la citadelle près de la maison, pour la muraille de la ville, et pour la maison que j'occuperai. Le roi me donna ces lettres, car la bonne main de mon Dieu était sur moi.
\par 9 Je me rendis auprès des gouverneurs de l'autre côté du fleuve, et je leur remis les lettres du roi, qui m'avait fait accompagner par des chefs de l'armée et par des cavaliers.
\par 10 Sanballat, le Horonite, et Tobija, le serviteur ammonite, l'ayant appris, eurent un grand déplaisir de ce qu'il venait un homme pour chercher le bien des enfants d'Israël.
\par 11 J'arrivai à Jérusalem, et j'y passai trois jours.
\par 12 Après quoi, je me levai pendant la nuit avec quelques hommes, sans avoir dit à personne ce que mon Dieu m'avait mis au coeur de faire pour Jérusalem. Il n'y avait avec moi d'autre bête de somme que ma propre monture.
\par 13 Je sortis de nuit par la porte de la vallée, et je me dirigeai contre la source du dragon et vers la porte du fumier, considérant les murailles en ruines de Jérusalem et réfléchissant à ses portes consumées par le feu.
\par 14 Je passai près de la porte de la source et de l'étang du roi, et il n'y avait point de place par où pût passer la bête qui était sous moi.
\par 15 Je montai de nuit par le torrent, et je considérai encore la muraille. Puis je rentrai par la porte de la vallée, et je fus ainsi de retour.
\par 16 Les magistrats ignoraient où j'étais allé, et ce que je faisais. Jusqu'à ce moment, je n'avais rien dit aux Juifs, ni aux sacrificateurs, ni aux grands, ni aux magistrats, ni à aucun de ceux qui s'occupaient des affaires.
\par 17 Je leur dis alors: Vous voyez le malheureux état où nous sommes! Jérusalem est détruite, et ses portes sont consumées par le feu! Venez, rebâtissons la muraille de Jérusalem, et nous ne serons plus dans l'opprobre.
\par 18 Et je leur racontai comment la bonne main de mon Dieu avait été sur moi, et quelles paroles le roi m'avait adressées. Ils dirent: Levons-nous, et bâtissons! Et ils se fortifièrent dans cette bonne résolution.
\par 19 Sanballat, le Horonite, Tobija, le serviteur ammonite, et Guéschem, l'Arabe, en ayant été informés, se moquèrent de nous et nous méprisèrent. Ils dirent: Que faites-vous là? Vous révoltez-vous contre le roi?
\par 20 Et je leur fis cette réponse: Le Dieu des cieux nous donnera le succès. Nous, ses serviteurs, nous nous lèverons et nous bâtirons; mais vous, vous n'avez ni part, ni droit, ni souvenir dans Jérusalem.

\chapter{3}

\par 1 Éliaschib, le souverain sacrificateur, se leva avec ses frères, les sacrificateurs, et ils bâtirent la porte des brebis. Ils la consacrèrent et en posèrent les battants; ils la consacrèrent, depuis la tour de Méa jusqu'à la tour de Hananeel.
\par 2 A côté d'Éliaschib bâtirent les hommes de Jéricho; à côté de lui bâtit aussi Zaccur, fils d'Imri.
\par 3 Les fils de Senaa bâtirent la porte des poissons. Ils la couvrirent, et en posèrent les battants, les verrous et les barres.
\par 4 A côté d'eux travailla aux réparations Merémoth, fils d'Urie, fils d'Hakkots; à côté d'eux travailla Meschullam, fils de Bérékia, fils de Meschézabeel; à côté d'eux travailla Tsadok, fils de Baana;
\par 5 à côté d'eux travaillèrent les Tekoïtes, dont les principaux ne se soumirent pas au service de leur seigneur.
\par 6 Jojada, fils de Paséach, et Meschullam, fils de Besodia, réparèrent la vieille porte. Ils la couvrirent, et en posèrent les battants, les verrous et les barres.
\par 7 A côté d'eux travaillèrent Melatia, le Gabaonite, Jadon, le Méronothite, et les hommes de Gabaon et de Mitspa, ressortissant au siège du gouverneur de ce côté du fleuve;
\par 8 à côté d'eux travailla Uzziel, fils de Harhaja, d'entre les orfèvres, et à côté de lui travailla Hanania, d'entre les parfumeurs. Ils laissèrent Jérusalem jusqu'à la muraille large.
\par 9 A côté d'eux travailla aux réparations Rephaja, fils de Hur, chef de la moitié du district de Jérusalem.
\par 10 A côté d'eux travailla vis-à-vis de sa maison Jedaja, fils de Harumaph, et à côté de lui travailla Hattusch, fils de Haschabnia.
\par 11 Une autre portion de la muraille et la tour des fours furent réparées par Malkija, fils de Harim, et par Haschub, fils de Pachath Moab.
\par 12 A côté d'eux travailla, avec ses filles, Schallum, fils d'Hallochesch, chef de la moitié du district de Jérusalem.
\par 13 Hanun et les habitants de Zanoach réparèrent la porte de la vallée. Ils la bâtirent, et en posèrent les battants, les verrous et les barres. Ils firent de plus mille coudées de mur jusqu'à la porte du fumier.
\par 14 Malkija, fils de Récab, chef du district de Beth Hakkérem, répara la porte du fumier. Il la bâtit, et en posa les battants, les verrous et les barres.
\par 15 Schallum, fils de Col Hozé, chef du district de Mitspa, répara la porte de la source. Il la bâtit, la couvrit, et en posa les battants, les verrous et les barres. Il fit de plus le mur de l'étang de Siloé, près du jardin du roi, jusqu'aux degrés qui descendent de la cité de David.
\par 16 Après lui Néhémie, fils d'Azbuk, chef de la moitié du district de Beth Tsur, travailla aux réparations jusque vis-à-vis des sépulcres de David, jusqu'à l'étang qui avait été construit, et jusqu'à la maison des héros.
\par 17 Après lui travaillèrent les Lévites, Rehum, fils de Bani, et à côté de lui travailla pour son district Haschabia, chef de la moitié du district de Keïla.
\par 18 Après lui travaillèrent leurs frères, Bavvaï, fils de Hénadad, chef de la moitié du district de Keïla;
\par 19 et à côté de lui Ézer, fils de Josué, chef de Mitspa, répara une autre portion de la muraille, vis-à-vis de la montée de l'arsenal, à l'angle.
\par 20 Après lui Baruc, fils de Zabbaï, répara avec ardeur une autre portion, depuis l'angle jusqu'à la porte de la maison d'Éliaschib, le souverain sacrificateur.
\par 21 Après lui Merémoth, fils d'Urie, fils d'Hakkots, répara une autre portion depuis la porte de la maison d'Éliaschib jusqu'à l'extrémité de la maison d'Éliaschib.
\par 22 Après lui travaillèrent les sacrificateurs des environs de Jérusalem.
\par 23 Après eux Benjamin et Haschub travaillèrent vis-à-vis de leur maison. Après eux Azaria, fils de Maaséja, fils d'Anania, travailla à côté de sa maison.
\par 24 Après lui Binnuï, fils de Hénadad, répara une autre portion, depuis la maison d'Azaria jusqu'à l'angle et jusqu'au coin.
\par 25 Palal, fils d'Uzaï, travailla vis-à-vis de l'angle et de la tour supérieure qui fait saillie en avant de la maison du roi près de la cour de la prison. Après lui travailla Pedaja, fils de Pareosch.
\par 26 Les Néthiniens demeurant sur la colline travaillèrent jusque vis-à-vis de la porte des eaux, à l'orient, et de la tour en saillie.
\par 27 Après eux les Tekoïtes réparèrent une autre portion, vis-à-vis de la grande tour en saillie jusqu'au mur de la colline.
\par 28 Au-dessus de la porte des chevaux, les sacrificateurs travaillèrent chacun devant sa maison.
\par 29 Après eux Tsadok, fils d'Immer, travailla devant sa maison. Après lui travailla Schemaeja, fils de Schecania, gardien de la porte de l'orient.
\par 30 Après eux Hanania, fils de Schélémia, et Hanun, le sixième fils de Tsalaph, réparèrent une autre portion de la muraille. Après eux Meschullam, fils de Bérékia, travailla vis-à-vis de sa chambre.
\par 31 Après lui Malkija, d'entre les orfèvres, travailla jusqu'aux maisons des Néthiniens et des marchands, vis-à-vis de la porte de Miphkad, et jusqu'à la chambre haute du coin.
\par 32 Les orfèvres et les marchands travaillèrent entre la chambre haute du coin et la porte des brebis.

\chapter{4}

\par 1 Lorsque Sanballat apprit que nous rebâtissions la muraille, il fut en colère et très irrité.
\par 2 Il se moqua des Juifs, et dit devant ses frères et devant les soldats de Samarie: A quoi travaillent ces Juifs impuissants? Les laissera-t-on faire? Sacrifieront-ils? Vont-ils achever? Redonneront-ils vie à des pierres ensevelies sous des monceaux de poussière et consumées par le feu?
\par 3 Tobija, l'Ammonite, était à côté de lui, et il dit: Qu'ils bâtissent seulement! Si un renard s'élance, il renversera leur muraille de pierres!
\par 4 Écoute, ô notre Dieu, comme nous sommes méprisés! Fais retomber leurs insultes sur leur tête, et livre-les au pillage sur une terre où ils soient captifs.
\par 5 Ne pardonne pas leur iniquité, et que leur péché ne soit pas effacé de devant toi; car ils ont offensé ceux qui bâtissent.
\par 6 Nous rebâtîmes la muraille, qui fut partout achevée jusqu'à la moitié de sa hauteur. Et le peuple prit à coeur ce travail.
\par 7 Mais Sanballat, Tobija, les Arabes, les Ammonites et les Asdodiens, furent très irrités en apprenant que la réparation des murs avançait et que les brèches commençaient à se fermer.
\par 8 Ils se liguèrent tous ensemble pour venir attaquer Jérusalem et lui causer du dommage.
\par 9 Nous priâmes notre Dieu, et nous établîmes une garde jour et nuit pour nous défendre contre leurs attaques.
\par 10 Cependant Juda disait: Les forces manquent à ceux qui portent les fardeaux, et les décombres sont considérables; nous ne pourrons pas bâtir la muraille.
\par 11 Et nos ennemis disaient: Ils ne sauront et ne verront rien jusqu'à ce que nous arrivions au milieu d'eux; nous les tuerons, et nous ferons ainsi cesser l'ouvrage.
\par 12 Or les Juifs qui habitaient près d'eux vinrent dix fois nous avertir, de tous les lieux d'où ils se rendaient vers nous.
\par 13 C'est pourquoi je plaçai, dans les enfoncements derrière la muraille et sur des terrains secs, le peuple par familles, tous avec leurs épées, leurs lances et leurs arcs.
\par 14 Je regardai, et m'étant levé, je dis aux grands, aux magistrats, et au reste du peuple: Ne les craignez pas! Souvenez-vous du Seigneur, grand et redoutable, et combattez pour vos frères, pour vos fils et vos filles, pour vos femmes et pour vos maisons!
\par 15 Lorsque nos ennemis apprirent que nous étions avertis, Dieu anéantit leur projet, et nous retournâmes tous à la muraille, chacun à son ouvrage.
\par 16 Depuis ce jour, la moitié de mes serviteurs travaillait, et l'autre moitié était armée de lances, de boucliers, d'arcs et de cuirasses. Les chefs étaient derrière toute la maison de Juda.
\par 17 Ceux qui bâtissaient la muraille, et ceux qui portaient ou chargeaient les fardeaux, travaillaient d'une main et tenaient une arme de l'autre;
\par 18 chacun d'eux, en travaillant, avait son épée ceinte autour des reins. Celui qui sonnait de la trompette se tenait près de moi.
\par 19 Je dis aux grands, aux magistrats, et au reste du peuple: L'ouvrage est considérable et étendu, et nous sommes dispersés sur la muraille, éloignés les uns des autres.
\par 20 Au son de la trompette, rassemblez-vous auprès de nous, vers le lieu d'où vous l'entendrez; notre Dieu combattra pour nous.
\par 21 C'est ainsi que nous poursuivions l'ouvrage, la moitié d'entre nous la lance à la main depuis le lever de l'aurore jusqu'à l'apparition des étoiles.
\par 22 Dans ce même temps, je dis encore au peuple: Que chacun passe la nuit dans Jérusalem avec son serviteur; faisons la garde pendant la nuit, et travaillons pendant le jour.
\par 23 Et nous ne quittions point nos vêtements, ni moi, ni mes frères, ni mes serviteurs, ni les hommes de garde qui me suivaient; chacun n'avait que ses armes et de l'eau.

\chapter{5}

\par 1 Il s'éleva de la part des gens du peuple et de leurs femmes de grandes plaintes contre leurs frères les Juifs.
\par 2 Les uns disaient: Nous, nos fils et nos filles, nous sommes nombreux; qu'on nous donne du blé, afin que nous mangions et que nous vivions.
\par 3 D'autres disaient: Nous engageons nos champs, nos vignes, et nos maisons, pour avoir du blé pendant la famine.
\par 4 D'autres disaient: Nous avons emprunté de l'argent sur nos champs et nos vignes pour le tribut du roi.
\par 5 Et pourtant notre chair est comme la chair de nos frères, nos enfants sont comme leurs enfants; et voici, nous soumettons à la servitude nos fils et nos filles, et plusieurs de nos filles y sont déjà réduites; nous sommes sans force, et nos champs et nos vignes sont à d'autres.
\par 6 Je fus très irrité lorsque j'entendis leurs plaintes et ces paroles-là.
\par 7 Je résolus de faire des réprimandes aux grands et aux magistrats, et je leur dis: Quoi! vous prêtez à intérêt à vos frères! Et je rassemblai autour d'eux une grande foule,
\par 8 et je leur dis: Nous avons racheté selon notre pouvoir nos frères les Juifs vendus aux nations; et vous vendriez vous-mêmes vos frères, et c'est à nous qu'ils seraient vendus! Ils se turent, ne trouvant rien à répondre.
\par 9 Puis je dis: Ce que vous faites n'est pas bien. Ne devriez-vous pas marcher dans la crainte de notre Dieu, pour n'être pas insultés par les nations nos ennemies?
\par 10 Moi aussi, et mes frères et mes serviteurs, nous leur avons prêté de l'argent et du blé. Abandonnons ce qu'ils nous doivent!
\par 11 Rendez-leur donc aujourd'hui leurs champs, leurs vignes, leurs oliviers et leurs maisons, et le centième de l'argent, du blé, du moût et de l'huile que vous avez exigé d'eux comme intérêt.
\par 12 Ils répondirent: Nous les rendrons, et nous ne leur demanderons rien, nous ferons ce que tu dis. Alors j'appelai les sacrificateurs, devant lesquels je les fis jurer de tenir parole.
\par 13 Et je secouai mon manteau, en disant: Que Dieu secoue de la même manière hors de sa maison et de ses biens tout homme qui n'aura point tenu parole, et qu'ainsi cet homme soit secoué et laissé à vide! Toute l'assemblée dit: Amen! On célébra l'Éternel. Et le peuple tint parole.
\par 14 Dès le jour où le roi m'établit leur gouverneur dans le pays de Juda, depuis la vingtième année jusqu'à la trente-deuxième année du roi Artaxerxès, pendant douze ans, ni moi ni mes frères n'avons vécu des revenus du gouverneur.
\par 15 Avant moi, les premiers gouverneurs accablaient le peuple, et recevaient de lui du pain et du vin, outre quarante sicles d'argent; leurs serviteurs mêmes opprimaient le peuple. Je n'ai point agi de la sorte, par crainte de Dieu.
\par 16 Bien plus, j'ai travaillé à la réparation de cette muraille, et nous n'avons acheté aucun champ, et mes serviteurs tous ensemble étaient à l'ouvrage.
\par 17 J'avais à ma table cent cinquante hommes, Juifs et magistrats, outre ceux qui venaient à nous des nations d'alentour.
\par 18 On m'apprêtait chaque jour un boeuf, six moutons choisis, et des oiseaux; et tous les dix jours on préparait en abondance tout le vin nécessaire. Malgré cela, je n'ai point réclamé les revenus du gouverneur, parce que les travaux étaient à la charge de ce peuple.
\par 19 Souviens-toi favorablement de moi, ô mon Dieu, à cause de tout ce que j'ai fait pour ce peuple!

\chapter{6}

\par 1 Je n'avais pas encore posé les battants des portes, lorsque Sanballat, Tobija, Guéschem, l'Arabe, et nos autres ennemis apprirent que j'avais rebâti la muraille et qu'il n'y restait plus de brèche.
\par 2 Alors Sanballat et Guéschem m'envoyèrent dire: Viens, et ayons ensemble une entrevue dans les villages de la vallée d'Ono. Ils avaient médité de me faire du mal.
\par 3 Je leur envoyai des messagers avec cette réponse: J'ai un grand ouvrage à exécuter, et je ne puis descendre; le travail serait interrompu pendant que je quitterais pour aller vers vous.
\par 4 Ils m'adressèrent quatre fois la même demande, et je leur fis la même réponse.
\par 5 Sanballat m'envoya ce message une cinquième fois par son serviteur, qui tenait à la main une lettre ouverte.
\par 6 Il y était écrit: Le bruit se répand parmi les nations et Gaschmu affirme que toi et les Juifs vous pensez à vous révolter, et que c'est dans ce but que tu rebâtis la muraille. Tu vas, dit-on, devenir leur roi,
\par 7 tu as même établi des prophètes pour te proclamer à Jérusalem roi de Juda. Et maintenant ces choses arriveront à la connaissance du roi. Viens donc, et consultons-nous ensemble.
\par 8 Je fis répondre à Sanballat: Ce que tu dis là n'est pas; c'est toi qui l'inventes!
\par 9 Tous ces gens voulaient nous effrayer, et ils se disaient: Ils perdront courage, et l'oeuvre ne se fera pas. Maintenant, ô Dieu, fortifie-moi!
\par 10 Je me rendis chez Schemaeja, fils de Delaja, fils de Mehétabeel. Il s'était enfermé, et il dit: Allons ensemble dans la maison de Dieu, au milieu du temple, et fermons les portes du temple; car ils viennent pour te tuer, et c'est pendant la nuit qu'ils viendront pour te tuer.
\par 11 Je répondis: Un homme comme moi prendre la fuite! Et quel homme tel que moi pourrait entrer dans le temple et vivre? Je n'entrerai point.
\par 12 Et je reconnus que ce n'était pas Dieu qui l'envoyait. Mais il prophétisa ainsi sur moi parce que Sanballat et Tobija lui avaient donné de l'argent.
\par 13 En le gagnant ainsi, ils espéraient que j'aurais peur, et que je suivrais ses avis et commettrais un péché; et ils auraient profité de cette atteinte à ma réputation pour me couvrir d'opprobre.
\par 14 Souviens-toi, ô mon Dieu, de Tobija et de Sanballat, et de leurs oeuvres! Souviens-toi aussi de Noadia, la prophétesse, et des autres prophètes qui cherchaient à m'effrayer!
\par 15 La muraille fut achevée le vingt-cinquième jour du mois d'Élul, en cinquante-deux jours.
\par 16 Lorsque tous nos ennemis l'apprirent, toutes les nations qui étaient autour de nous furent dans la crainte; elles éprouvèrent une grande humiliation, et reconnurent que l'oeuvre s'était accomplie par la volonté de notre Dieu.
\par 17 Dans ce temps-là, il y avait aussi des grands de Juda qui adressaient fréquemment des lettres à Tobija et qui en recevaient de lui.
\par 18 Car plusieurs en Juda étaient liés à lui par serment, parce qu'il était gendre de Schecania, fils d'Arach, et que son fils Jochanan avait pris la fille de Meschullam, fils de Bérékia.
\par 19 Ils disaient même du bien de lui en ma présence, et ils lui rapportaient mes paroles. Tobija envoyait des lettres pour m'effrayer.

\chapter{7}

\par 1 Lorsque la muraille fut rebâtie et que j'eus posé les battants des portes, on établit dans leurs fonctions les portiers, les chantres et les Lévites.
\par 2 Je donnai mes ordres à Hanani, mon frère, et à Hanania, chef de la citadelle de Jérusalem, homme supérieur au grand nombre par sa fidélité et par sa crainte de Dieu.
\par 3 Je leur dis: Les portes de Jérusalem ne s'ouvriront pas avant que la chaleur du soleil soit venue, et l'on fermera les battants aux verrous en votre présence; les habitants de Jérusalem feront la garde, chacun à son poste devant sa maison.
\par 4 La ville était spacieuse et grande, mais peu peuplée, et les maisons n'étaient pas bâties.
\par 5 Mon Dieu me mit au coeur d'assembler les grands, les magistrats et le peuple, pour en faire le dénombrement. Je trouvai un registre généalogique de ceux qui étaient montés les premiers, et j'y vis écrit ce qui suit.
\par 6 Voici ceux de la province qui revinrent de l'exil, ceux que Nebucadnetsar, roi de Babylone, avait emmenés captifs, et qui retournèrent à Jérusalem et en Juda, chacun dans sa ville.
\par 7 Ils partirent avec Zorobabel, Josué, Néhémie, Azaria, Raamia, Nachamani, Mardochée, Bilschan, Mispéreth, Bigvaï, Nehum, Baana. Nombre des hommes du peuple d'Israël:
\par 8 les fils de Pareosch, deux mille cent soixante-douze;
\par 9 les fils de Schephathia, trois cent soixante-douze;
\par 10 les fils d'Arach, six cent cinquante-deux;
\par 11 les fils de Pachath Moab, des fils de Josué et de Joab, deux mille huit cent dix-huit;
\par 12 les fils d'Élam, mille deux cent cinquante-quatre;
\par 13 les fils de Zatthu, huit cent quarante-cinq;
\par 14 les fils de Zaccaï, sept cent soixante;
\par 15 les fils de Binnuï, six cent quarante-huit;
\par 16 les fils de Bébaï, six cent vingt-huit;
\par 17 les fils d'Azgad, deux mille trois cent vingt-deux;
\par 18 les fils d'Adonikam, six cent soixante-sept;
\par 19 les fils de Bigvaï, deux mille soixante-sept;
\par 20 les fils d'Adin, six cent cinquante-cinq;
\par 21 les fils d'Ather, de la famille d'Ézéchias, quatre-vingt-dix-huit;
\par 22 les fils de Haschum, trois cent vingt-huit;
\par 23 les fils de Betsaï, trois cent vingt-quatre;
\par 24 les fils de Hariph, cent douze;
\par 25 les fils de Gabaon, quatre-vingt-quinze;
\par 26 les gens de Bethléhem et de Netopha, cent quatre-vingt-huit;
\par 27 les gens d'Anathoth, cent vingt-huit;
\par 28 les gens de Beth Azmaveth, quarante-deux;
\par 29 les gens de Kirjath Jearim, de Kephira et de Beéroth, sept cent quarante-trois;
\par 30 les gens de Rama et de Guéba, six cent vingt et un;
\par 31 les gens de Micmas, cent vingt-deux;
\par 32 les gens de Béthel et d'Aï, cent vingt-trois;
\par 33 les gens de l'autre Nebo, cinquante-deux;
\par 34 les fils de l'autre Élam, mille deux cent cinquante-quatre;
\par 35 les fils de Harim, trois cent vingt;
\par 36 les fils de Jéricho, trois cent quarante-cinq;
\par 37 les fils de Lod, de Hadid et d'Ono, sept cent vingt et un;
\par 38 les fils de Senaa, trois mille neuf cent trente.
\par 39 Sacrificateurs: les fils de Jedaeja, de la maison de Josué, neuf cent soixante-treize;
\par 40 les fils d'Immer, mille cinquante-deux;
\par 41 les fils de Paschhur, mille deux cent quarante-sept;
\par 42 les fils de Harim, mille dix-sept.
\par 43 Lévites: les fils de Josué et de Kadmiel, des fils d'Hodva, soixante-quatorze.
\par 44 Chantres: les fils d'Asaph, cent quarante-huit.
\par 45 Portiers: les fils de Schallum, les fils d'Ather, les fils de Thalmon, les fils d'Akkub, les fils de Hathitha, les fils de Schobaï, cent trente-huit.
\par 46 Néthiniens: les fils de Tsicha, les fils de Hasupha, les fils de Thabbaoth,
\par 47 les fils de Kéros, les fils de Sia, les fils de Padon,
\par 48 les fils de Lebana, les fils de Hagaba, les fils de Salmaï,
\par 49 les fils de Hanan, les fils de Guiddel, les fils de Gachar,
\par 50 les fils de Reaja, les fils de Retsin, les fils de Nekoda,
\par 51 les fils de Gazzam, les fils d'Uzza, les fils de Paséach,
\par 52 les fils de Bésaï, les fils de Mehunim, les fils de Nephischsim,
\par 53 les fils de Bakbuk, les fils de Hakupha, les fils de Harhur,
\par 54 les fils de Batslith, les fils de Mehida, les fils de Harscha,
\par 55 les fils de Barkos, les fils de Sisera, les fils de Thamach,
\par 56 les fils de Netsiach, les fils de Hathipha.
\par 57 Fils des serviteurs de Salomon: les fils de Sothaï, les fils de Sophéreth, les fils de Perida,
\par 58 les fils de Jaala, les fils de Darkon, les fils de Guiddel,
\par 59 les fils de Schephathia, les fils de Hatthil, les fils de Pokéreth Hatsebaïm, les fils d'Amon.
\par 60 Total des Néthiniens et des fils des serviteurs de Salomon: trois cent quatre-vingt-douze.
\par 61 Voici ceux qui partirent de Thel Mélach, de Thel Harscha, de Kerub Addon, et d'Immer, et qui ne purent pas faire connaître leur maison paternelle et leur race, pour prouver qu'ils étaient d'Israël.
\par 62 Les fils de Delaja, les fils de Tobija, les fils de Nekoda, six cent quarante-deux.
\par 63 Et parmi les sacrificateurs: les fils de Hobaja, les fils d'Hakkots, les fils de Barzillaï, qui avait pris pour femme une des filles de Barzillaï, le Galaadite, et fut appelé de leur nom.
\par 64 Ils cherchèrent leurs titres généalogiques, mais ils ne les trouvèrent point. On les exclut du sacerdoce,
\par 65 et le gouverneur leur dit de ne pas manger des choses très saintes jusqu'à ce qu'un sacrificateur eût consulté l'urim et le thummim.
\par 66 L'assemblée tout entière était de quarante-deux mille trois cent soixante personnes,
\par 67 sans compter leurs serviteurs et leurs servantes, au nombre de sept mille trois cent trente-sept. Parmi eux se trouvaient deux cent quarante-cinq chantres et chanteuses.
\par 68 Ils avaient sept cent trente-six chevaux, deux cent quarante-cinq mulets,
\par 69 quatre cent trente-cinq chameaux, et six mille sept cent vingt ânes.
\par 70 Plusieurs des chefs de famille firent des dons pour l'oeuvre. Le gouverneur donna au trésor mille dariques d'or, cinquante coupes, cinq cent trente tuniques sacerdotales.
\par 71 Les chefs de familles donnèrent au trésor de l'oeuvre vingt mille dariques d'or et deux mille deux cents mines d'argent.
\par 72 Le reste du peuple donna vingt mille dariques d'or, deux mille mines d'argent, et soixante-sept tuniques sacerdotales.
\par 73 Les sacrificateurs et les Lévites, les portiers, les chantres, les gens du peuple, les Néthiniens et tout Israël s'établirent dans leurs villes. Le septième mois arriva, et les enfants d'Israël étaient dans leurs villes.

\chapter{8}

\par 1 Alors tout le peuple s'assembla comme un seul homme sur la place qui est devant la porte des eaux. Ils dirent à Esdras, le scribe, d'apporter le livre de la loi de Moïse, prescrite par l'Éternel à Israël.
\par 2 Et le sacrificateur Esdras apporta la loi devant l'assemblée, composée d'hommes et de femmes et de tous ceux qui étaient capables de l'entendre. C'était le premier jour du septième mois.
\par 3 Esdras lut dans le livre depuis le matin jusqu'au milieu du jour, sur la place qui est devant la porte des eaux, en présence des hommes et des femmes et de ceux qui étaient capables de l'entendre. Tout le peuple fut attentif à la lecture du livre de la loi.
\par 4 Esdras, le scribe, était placé sur une estrade de bois, dressée à cette occasion. Auprès de lui, à sa droite, se tenaient Matthithia, Schéma, Anaja, Urie, Hilkija et Maaséja, et à sa gauche, Pedaja, Mischaël, Malkija, Haschum, Haschbaddana, Zacharie et Meschullam.
\par 5 Esdras ouvrit le livre à la vue de tout le peuple, car il était élevé au-dessus de tout le peuple; et lorsqu'il l'eut ouvert, tout le peuple se tint en place.
\par 6 Esdras bénit l'Éternel, le grand Dieu, et tout le peuple répondit, en levant les mains: Amen! amen! Et ils s'inclinèrent et se prosternèrent devant l'Éternel, le visage contre terre.
\par 7 Josué, Bani, Schérébia, Jamin, Akkub, Schabbethaï, Hodija, Maaséja, Kelitha, Azaria, Jozabad, Hanan, Pelaja, et les Lévites, expliquaient la loi au peuple, et chacun restait à sa place.
\par 8 Ils lisaient distinctement dans le livre de la loi de Dieu, et ils en donnaient le sens pour faire comprendre ce qu'ils avaient lu.
\par 9 Néhémie, le gouverneur, Esdras, le sacrificateur et le scribe, et les Lévites qui enseignaient le peuple, dirent à tout le peuple: Ce jour est consacré à l'Éternel, votre Dieu; ne soyez pas dans la désolation et dans les larmes! Car tout le peuple pleurait en entendant les paroles de la loi.
\par 10 Ils leur dirent: Allez, mangez des viandes grasses et buvez des liqueurs douces, et envoyez des portions à ceux qui n'ont rien de préparé, car ce jour est consacré à notre Seigneur; ne vous affligez pas, car la joie de l'Éternel sera votre force.
\par 11 Les Lévites calmaient tout le peuple, en disant: Taisez-vous, car ce jour est saint; ne vous affligez pas!
\par 12 Et tout le peuple s'en alla pour manger et boire, pour envoyer des portions, et pour se livrer à de grandes réjouissances. Car ils avaient compris les paroles qu'on leur avait expliquées.
\par 13 Le second jour, les chefs de famille de tout le peuple, les sacrificateurs et les Lévites, s'assemblèrent auprès d'Esdras, le scribe, pour entendre l'explication des paroles de la loi.
\par 14 Et ils trouvèrent écrit dans la loi que l'Éternel avait prescrite par Moïse, que les enfants d'Israël devaient habiter sous des tentes pendant la fête du septième mois,
\par 15 et proclamer cette publication dans toutes leurs villes et à Jérusalem: Allez chercher à la montagne des rameaux d'olivier, des rameaux d'olivier sauvage, des rameaux de myrte, des rameaux de palmier, et des rameaux d'arbres touffus, pour faire des tentes, comme il est écrit.
\par 16 Alors le peuple alla chercher des rameaux, et ils se firent des tentes sur le toit de leurs maisons, dans leurs cours, dans les parvis de la maison de Dieu, sur la place de la porte des eaux et sur la place de la porte d'Éphraïm.
\par 17 Toute l'assemblée de ceux qui étaient revenus de la captivité fit des tentes, et ils habitèrent sous ces tentes. Depuis le temps de Josué, fils de Nun, jusqu'à ce jour, les enfants d'Israël n'avaient rien fait de pareil. Et il y eut de très grandes réjouissances.
\par 18 On lut dans le livre de la loi de Dieu chaque jour, depuis le premier jour jusqu'au dernier. On célébra la fête pendant sept jours, et il y eut une assemblée solennelle le huitième jour, comme cela est ordonné.

\chapter{9}

\par 1 Le vingt-quatrième jour du même mois, les enfants d'Israël s'assemblèrent, revêtus de sacs et couverts de poussière, pour la célébration d'un jeûne.
\par 2 Ceux qui étaient de la race d'Israël, s'étant séparés de tous les étrangers, se présentèrent et confessèrent leurs péchés et les iniquités de leurs pères.
\par 3 Lorsqu'ils furent placés, on lut dans le livre de la loi de l'Éternel, leur Dieu, pendant un quart de la journée; et pendant un autre quart ils confessèrent leurs péchés et se prosternèrent devant l'Éternel, leur Dieu.
\par 4 Josué, Bani, Kadmiel, Schebania, Bunni, Schérébia, Bani et Kenani montèrent sur l'estrade des Lévites et crièrent à haute voix vers l'Éternel, leur Dieu.
\par 5 Et les Lévites Josué, Kadmiel, Bani, Haschabnia, Schérébia, Hodija, Schebania et Pethachja, dirent: Levez-vous, bénissez l'Éternel, votre Dieu, d'éternité en éternité! Que l'on bénisse ton nom glorieux, qui est au-dessus de toute bénédiction et de toute louange!
\par 6 C'est toi, Éternel, toi seul, qui as fait les cieux, les cieux des cieux et toute leur armée, la terre et tout ce qui est sur elle, les mers et tout ce qu'elles renferment. Tu donnes la vie à toutes ces choses, et l'armée des cieux se prosterne devant toi.
\par 7 C'est toi, Éternel Dieu, qui as choisi Abram, qui l'as fait sortir d'Ur en Chaldée, et qui lui as donné le nom d'Abraham.
\par 8 Tu trouvas son coeur fidèle devant toi, tu fis alliance avec lui, et tu promis de donner à sa postérité le pays des Cananéens, des Héthiens, des Amoréens, des Phéréziens, des Jébusiens et des Guirgasiens. Et tu as tenu ta parole, car tu es juste.
\par 9 Tu vis l'affliction de nos pères en Égypte, et tu entendis leurs cris vers la mer Rouge.
\par 10 Tu opéras des miracles et des prodiges contre Pharaon, contre tous ses serviteurs et contre tout le peuple de son pays, parce que tu savais avec quelle méchanceté ils avaient traité nos pères, et tu fis paraître ta gloire comme elle paraît aujourd'hui.
\par 11 Tu fendis la mer devant eux, et ils passèrent à sec au milieu de la mer; mais tu précipitas dans l'abîme, comme une pierre au fond des eaux, ceux qui marchaient à leur poursuite.
\par 12 Tu les guidas le jour par une colonne de nuée, et la nuit par une colonne de feu qui les éclairait dans le chemin qu'ils avaient à suivre.
\par 13 Tu descendis sur la montagne de Sinaï, tu leur parlas du haut des cieux, et tu leur donnas des ordonnances justes, des lois de vérité, des préceptes et des commandements excellents.
\par 14 Tu leur fis connaître ton saint sabbat, et tu leur prescrivis par Moïse, ton serviteur, des commandements, des préceptes et une loi.
\par 15 Tu leur donnas, du haut des cieux, du pain quand ils avaient faim, et tu fis sortir de l'eau du rocher quand ils avaient soif. Et tu leur dis d'entrer en possession du pays que tu avais juré de leur donner.
\par 16 Mais nos pères se livrèrent à l'orgueil et raidirent leur cou. Ils n'écoutèrent point tes commandements,
\par 17 ils refusèrent d'obéir, et ils mirent en oubli les merveilles que tu avais faites en leur faveur. Ils raidirent leur cou; et, dans leur rébellion, ils se donnèrent un chef pour retourner à leur servitude. Mais toi, tu es un Dieu prêt à pardonner, compatissant et miséricordieux, lent à la colère et riche en bonté, et tu ne les abandonnas pas,
\par 18 même quand ils se firent un veau en fonte et dirent: Voici ton Dieu qui t'a fait sortir d'Égypte, et qu'ils se livrèrent envers toi à de grands outrages.
\par 19 Dans ton immense miséricorde, tu ne les abandonnas pas au désert, et la colonne de nuée ne cessa point de les guider le jour dans leur chemin, ni la colonne de feu de les éclairer la nuit dans le chemin qu'ils avaient à suivre.
\par 20 Tu leur donnas ton bon esprit pour les rendre sages, tu ne refusas point ta manne à leur bouche, et tu leur fournis de l'eau pour leur soif.
\par 21 Pendant quarante ans, tu pourvus à leur entretien dans le désert, et ils ne manquèrent de rien, leurs vêtements ne s'usèrent point, et leurs pieds ne s'enflèrent point.
\par 22 Tu leur livras des royaumes et des peuples, dont tu partageas entre eux les contrées, et ils possédèrent le pays de Sihon, roi de Hesbon, et le pays d'Og, roi de Basan.
\par 23 Tu multiplias leurs fils comme les étoiles des cieux, et tu les fis entrer dans le pays dont tu avais dit à leurs pères qu'ils prendraient possession.
\par 24 Et leurs fils entrèrent et prirent possession du pays; tu humilias devant eux les habitants du pays, les Cananéens, et tu les livras entre leurs mains, avec leurs rois et les peuples du pays, pour qu'ils les traitassent à leur gré.
\par 25 Ils devinrent maîtres de villes fortifiées et de terres fertiles; ils possédèrent des maisons remplies de toutes sortes de biens, des citernes creusées, des vignes, des oliviers, et des arbres fruitiers en abondance; ils mangèrent, ils se rassasièrent, ils s'engraissèrent, et ils vécurent dans les délices par ta grande bonté.
\par 26 Néanmoins, ils se soulevèrent et se révoltèrent contre toi. Ils jetèrent ta loi derrière leur dos, ils tuèrent tes prophètes qui les conjuraient de revenir à toi, et ils se livrèrent envers toi à de grands outrages.
\par 27 Alors tu les abandonnas entre les mains de leurs ennemis, qui les opprimèrent. Mais, au temps de leur détresse, ils crièrent à toi; et toi, tu les entendis du haut des cieux, et, dans ta grande miséricorde, tu leur donnas des libérateurs qui les sauvèrent de la main de leurs ennemis.
\par 28 Quand ils eurent du repos, ils recommencèrent à faire le mal devant toi. Alors tu les abandonnas entre les mains de leurs ennemis, qui les dominèrent. Mais, de nouveau, ils crièrent à toi; et toi, tu les entendis du haut des cieux, et, dans ta grande miséricorde, tu les délivras maintes fois.
\par 29 Tu les conjuras de revenir à ta loi; et ils persévérèrent dans l'orgueil, ils n'écoutèrent point tes commandements, ils péchèrent contre tes ordonnances, qui font vivre celui qui les met en pratique, ils eurent une épaule rebelle, ils raidirent leur cou, et ils n'obéirent point.
\par 30 Tu les supportas de nombreuses années, tu leur donnas des avertissements par ton esprit, par tes prophètes; et ils ne prêtèrent point l'oreille. Alors tu les livras entre les mains des peuples étrangers.
\par 31 Mais, dans ta grande miséricorde, tu ne les anéantis pas, et tu ne les abandonnas pas, car tu es un Dieu compatissant et miséricordieux.
\par 32 Et maintenant, ô notre Dieu, Dieu grand, puissant et redoutable, toi qui gardes ton alliance et qui exerces la miséricorde, ne regarde pas comme peu de chose toutes les souffrances que nous avons éprouvées, nous, nos rois, nos chefs, nos sacrificateurs, nos prophètes, nos pères et tout ton peuple, depuis le temps des rois d'Assyrie jusqu'à ce jour.
\par 33 Tu as été juste dans tout ce qui nous est arrivé, car tu t'es montré fidèle, et nous avons fait le mal.
\par 34 Nos rois, nos chefs, nos sacrificateurs et nos pères n'ont point observé ta loi, et ils n'ont été attentifs ni à tes commandements ni aux avertissements que tu leur adressais.
\par 35 Pendant qu'ils étaient les maîtres, au milieu des bienfaits nombreux que tu leur accordais, dans le pays vaste et fertile que tu leur avais livré, ils ne t'ont point servi et ils ne se sont point détournés de leurs oeuvres mauvaises.
\par 36 Et aujourd'hui, nous voici esclaves! Nous voici esclaves sur la terre que tu as donnée à nos pères, pour qu'ils jouissent de ses fruits et de ses biens!
\par 37 Elle multiplie ses produits pour les rois auxquels tu nous as assujettis, à cause de nos péchés; ils dominent à leur gré sur nos corps et sur notre bétail, et nous sommes dans une grande angoisse!
\par 38 Pour tout cela, nous contractâmes une alliance, que nous mîmes par écrit; et nos chefs, nos Lévites et nos sacrificateurs y apposèrent leur sceau.

\chapter{10}

\par 1 Voici ceux qui apposèrent leur sceau. Néhémie, le gouverneur, fils de Hacalia.
\par 2 Sédécias, Seraja, Azaria, Jérémie,
\par 3 Paschhur, Amaria, Malkija,
\par 4 Hattusch, Schebania, Malluc,
\par 5 Harim, Merémoth, Abdias,
\par 6 Daniel, Guinnethon, Baruc,
\par 7 Meschullam, Abija, Mijamin,
\par 8 Maazia, Bilgaï, Schemaeja, sacrificateurs.
\par 9 Lévites: Josué, fils d'Azania, Binnuï, des fils de Hénadad, Kadmiel,
\par 10 et leurs frères, Schebania, Hodija, Kelitha, Pelaja, Hanan,
\par 11 Michée, Rehob, Haschabia,
\par 12 Zaccur, Schérébia, Schebania,
\par 13 Hodija, Bani, Beninu.
\par 14 Chefs du peuple: Pareosch, Pachath Moab, Élam, Zatthu, Bani,
\par 15 Bunni, Azgad, Bébaï,
\par 16 Adonija, Bigvaï, Adin,
\par 17 Ather, Ézéchias, Azzur,
\par 18 Hodija, Haschum, Betsaï,
\par 19 Hariph, Anathoth, Nébaï,
\par 20 Magpiasch, Meschullam, Hézir,
\par 21 Meschézabeel, Tsadok, Jaddua,
\par 22 Pelathia, Hanan, Anaja,
\par 23 Hosée, Hanania, Haschub,
\par 24 Hallochesch, Pilcha, Schobek,
\par 25 Rehum, Haschabna, Maaséja,
\par 26 Achija, Hanan, Anan,
\par 27 Malluc, Harim, Baana.
\par 28 Le reste du peuple, les sacrificateurs, les Lévites, les portiers, les chantres, les Néthiniens, et tous ceux qui s'étaient séparés des peuples étrangers pour suivre la loi de Dieu, leurs femmes, leurs fils et leurs filles, tous ceux qui étaient capables de connaissance et d'intelligence,
\par 29 se joignirent à leurs frères les plus considérables d'entre eux. Ils promirent avec serment et jurèrent de marcher dans la loi de Dieu donnée par Moïse, serviteur de Dieu, d'observer et de mettre en pratique tous les commandements de l'Éternel, notre Seigneur, ses ordonnances et ses lois.
\par 30 Nous promîmes de ne pas donner nos filles aux peuples du pays et de ne pas prendre leurs filles pour nos fils;
\par 31 de ne rien acheter, le jour du sabbat et les jours de fête, des peuples du pays qui apporteraient à vendre, le jour du sabbat, des marchandises ou denrées quelconques; et de faire relâche la septième année, en n'exigeant le paiement d'aucune dette.
\par 32 Nous nous imposâmes aussi des ordonnances qui nous obligeaient à donner un tiers de sicle par année pour le service de la maison de notre Dieu,
\par 33 pour les pains de proposition, pour l'offrande perpétuelle, pour l'holocauste perpétuel des sabbats, des nouvelles lunes et des fêtes, pour les choses consacrées, pour les sacrifices d'expiation en faveur d'Israël, et pour tout ce qui se fait dans la maison de notre Dieu.
\par 34 Nous tirâmes au sort, sacrificateurs, Lévites et peuple, au sujet du bois qu'on devait chaque année apporter en offrande à la maison de notre Dieu, selon nos maisons paternelles, à des époques fixes, pour qu'il fût brûlé sur l'autel de l'Éternel, notre Dieu, comme il est écrit dans la loi.
\par 35 Nous résolûmes d'apporter chaque année à la maison de l'Éternel les prémices de notre sol et les prémices de tous les fruits de tous les arbres;
\par 36 d'amener à la maison de notre Dieu, aux sacrificateurs qui font le service dans la maison de notre Dieu, les premiers-nés de nos fils et de notre bétail, comme il est écrit dans la loi, les premiers-nés de nos boeufs et de nos brebis;
\par 37 d'apporter aux sacrificateurs, dans les chambres de la maison de notre Dieu, les prémices de notre pâte et nos offrandes, des fruits de tous les arbres, du moût et de l'huile; et de livrer la dîme de notre sol aux Lévites qui doivent la prendre eux-mêmes dans toutes les villes situées sur les terres que nous cultivons.
\par 38 Le sacrificateur, fils d'Aaron, sera avec les Lévites quand ils lèveront la dîme; et les Lévites apporteront la dîme de la dîme à la maison de notre Dieu, dans les chambres de la maison du trésor.
\par 39 Car les enfants d'Israël et les fils de Lévi apporteront dans ces chambres les offrandes de blé, du moût et d'huile; là sont les ustensiles du sanctuaire, et se tiennent les sacrificateurs qui font le service, les portiers et les chantres. C'est ainsi que nous résolûmes de ne pas abandonner la maison de notre Dieu.

\chapter{11}

\par 1 Les chefs du peuple s'établirent à Jérusalem. Le reste du peuple tira au sort, pour qu'un sur dix vînt habiter Jérusalem, la ville sainte, et que les autres demeurassent dans les villes.
\par 2 Le peuple bénit tous ceux qui consentirent volontairement à résider à Jérusalem.
\par 3 Voici les chefs de la province qui s'établirent à Jérusalem. Dans les villes de Juda, chacun s'établit dans sa propriété, dans sa ville, Israël, les sacrificateurs et les Lévites, les Néthiniens, et les fils des serviteurs de Salomon.
\par 4 A Jérusalem s'établirent des fils de Juda et des fils de Benjamin. -Des fils de Juda: Athaja, fils d'Ozias, fils de Zacharie, fils d'Amaria, fils de Schephathia, fils de Mahalaleel, des fils de Pérets,
\par 5 et Maaséja, fils de Baruc, fils de Col Hozé, fils de Hazaja, fils d'Adaja, fils de Jojarib, fils de Zacharie, fils de Schiloni.
\par 6 Total des fils de Pérets qui s'établirent à Jérusalem: quatre cent soixante-huit hommes vaillants. -
\par 7 Voici les fils de Benjamin: Sallu, fils de Meschullam, fils de Joëd, fils de Pedaja, fils de Kolaja, fils de Maaséja, fils d'Ithiel, fils d'Ésaïe,
\par 8 et, après lui, Gabbaï et Sallaï, neuf cent vingt-huit.
\par 9 Joël, fils de Zicri, était leur chef; et Juda, fils de Senua, était le second chef de la ville.
\par 10 Des sacrificateurs: Jedaeja, fils de Jojarib, Jakin,
\par 11 Seraja, fils de Hilkija, fils de Meschullam, fils de Tsadok, fils de Merajoth, fils d'Achithub, prince de la maison de Dieu,
\par 12 et leurs frères occupés au service de la maison, huit cent vingt-deux; Adaja, fils de Jerocham, fils de Pelalia, fils d'Amtsi, fils de Zacharie, fils de Paschhur, fils de Malkija,
\par 13 et ses frères, chefs des maisons paternelles, deux cent quarante-deux; et Amaschsaï, fils d'Azareel, fils d'Achzaï, fils de Meschillémoth, fils d'Immer,
\par 14 et leurs frères, vaillants hommes, cent vingt-huit. Zabdiel, fils de Guedolim, était leur chef.
\par 15 Des Lévites: Schemaeja, fils de Haschub, fils d'Azrikam, fils de Haschabia, fils de Bunni,
\par 16 Schabbethaï et Jozabad, chargés des affaires extérieures de la maison de Dieu, et faisant partie des chefs des Lévites;
\par 17 Matthania, fils de Michée, fils de Zabdi, fils d'Asaph, le chef qui entonnait la louange à la prière, et Bakbukia, le second parmi ses frères, et Abda, fils de Schammua, fils de Galal, fils de Jeduthun.
\par 18 Total des Lévites dans la ville sainte: deux cent quatre-vingt-quatre.
\par 19 Et les portiers: Akkub, Thalmon, et leurs frères, gardiens des portes, cent soixante-douze.
\par 20 Le reste d'Israël, les sacrificateurs, les Lévites, s'établirent dans toutes les villes de Juda, chacun dans sa propriété.
\par 21 Les Néthiniens s'établirent sur la colline, et ils avaient pour chefs Tsicha et Guischpa.
\par 22 Le chef des Lévites à Jérusalem était Uzzi, fils de Bani, fils de Haschabia, fils de Matthania, fils de Michée, d'entre les fils d'Asaph, les chantres chargés des offices de la maison de Dieu;
\par 23 car il y avait un ordre du roi concernant les chantres, et un salaire fixe leur était accordé pour chaque jour.
\par 24 Pethachja, fils de Meschézabeel, des fils de Zérach, fils de Juda, était commissaire du roi pour toutes les affaires du peuple.
\par 25 Dans les villages et leurs territoires, des fils de Juda s'établirent à Kirjath Arba et dans les lieux de son ressort, à Dibon et dans les lieux de son ressort, à Jekabtseel et dans les villages de son ressort,
\par 26 à Jéschua, à Molada, à Beth Paleth,
\par 27 à Hatsar Schual, à Beer Schéba, et dans les lieux de son ressort,
\par 28 à Tsiklag, à Mecona et dans les lieux de son ressort,
\par 29 à En Rimmon, à Tsorea, à Jarmuth,
\par 30 à Zanoach, à Adullam, et dans les villages de leur ressort, à Lakis et dans son territoire, à Azéka et dans les lieux de son ressort. Ils s'établirent depuis Beer Schéba jusqu'à la vallée de Hinnom.
\par 31 Les fils de Benjamin s'établirent, depuis Guéba, à Micmasch, à Ajja, à Béthel et dans les lieux de son ressort,
\par 32 à Anathoth, à Nob, à Hanania,
\par 33 à Hatsor, à Rama, à Guitthaïm,
\par 34 à Hadid, à Tseboïm, à Neballath,
\par 35 à Lod et à Ono, la vallée des ouvriers.
\par 36 Il y eut des Lévites qui se joignirent à Benjamin, quoique appartenant aux divisions de Juda.

\chapter{12}

\par 1 Voici les sacrificateurs et les Lévites qui revinrent avec Zorobabel, fils de Schealthiel, et avec Josué: Seraja, Jérémie, Esdras,
\par 2 Amaria, Malluc, Hattusch,
\par 3 Schecania, Rehum, Merémoth,
\par 4 Iddo, Guinnethoï, Abija,
\par 5 Mijamin, Maadia, Bilga,
\par 6 Schemaeja, Jojarib, Jedaeja,
\par 7 Sallu, Amok, Hilkija, Jedaeja. Ce furent là les chefs des sacrificateurs et de leurs frères, au temps de Josué. -
\par 8 Lévites: Josué, Binnuï, Kadmiel, Schérébia, Juda, Matthania, qui dirigeait avec ses frères le chant des louanges;
\par 9 Bakbukia et Unni, qui remplissaient leurs fonctions auprès de leurs frères.
\par 10 Josué engendra Jojakim, Jojakim engendra Éliaschib, Éliaschib engendra Jojada,
\par 11 Jojada engendra Jonathan, et Jonathan engendra Jaddua.
\par 12 Voici, au temps de Jojakim, quels étaient les sacrificateurs, chefs de famille: pour Seraja, Meraja; pour Jérémie, Hanania;
\par 13 pour Esdras, Meschullam; pour Amaria, Jochanan;
\par 14 pour Meluki, Jonathan; pour Schebania, Joseph;
\par 15 pour Harim, Adna; pour Merajoth, Helkaï;
\par 16 pour Iddo, Zacharie; pour Guinnethon, Meschullam;
\par 17 pour Abija, Zicri; pour Minjamin et Moadia, Pilthaï;
\par 18 pour Bilga, Schammua; pour Schemaeja, Jonathan;
\par 19 pour Jojarib, Matthnaï; pour Jedaeja, Uzzi;
\par 20 pour Sallaï, Kallaï; pour Amok, Éber;
\par 21 pour Hilkija, Haschabia; pour Jedaeja, Nethaneel.
\par 22 Au temps d'Éliaschib, de Jojada, de Jochanan et de Jaddua, les Lévites, chefs de familles, et les sacrificateurs, furent inscrits, sous le règne de Darius, le Perse.
\par 23 Les fils de Lévi, chefs de familles, furent inscrits dans le livre des Chroniques jusqu'au temps de Jochanan, fils d'Éliaschib.
\par 24 Les chefs des Lévites, Haschabia, Schérébia, et Josué, fils de Kadmiel, et leurs frères avec eux, les uns vis-à-vis des autres, étaient chargés de célébrer et de louer l'Éternel, selon l'ordre de David, homme de Dieu.
\par 25 Matthania, Bakbukia, Abdias, Meschullam, Thalmon et Akkub, portiers, faisaient la garde aux seuils des portes.
\par 26 Ils vivaient au temps de Jojakim, fils de Josué, fils de Jotsadak, et au temps de Néhémie, le gouverneur, et d'Esdras, le sacrificateur et le scribe.
\par 27 Lors de la dédicace des murailles de Jérusalem, on appela les Lévites de tous les lieux qu'ils habitaient et on les fit venir à Jérusalem, afin de célébrer la dédicace et la fête par des louanges et par des chants, au son des cymbales, des luths et des harpes.
\par 28 Les fils des chantres se rassemblèrent des environs de Jérusalem, des villages des Nethophatiens,
\par 29 de Beth Guilgal, et du territoire de Guéba et d'Azmaveth; car les chantres s'étaient bâti des villages aux alentours de Jérusalem.
\par 30 Les sacrificateurs et les Lévites se purifièrent, et ils purifièrent le peuple, les portes et la muraille.
\par 31 Je fis monter sur la muraille les chefs de Juda, et je formai deux grands choeurs. Le premier se mit en marche du côté droit sur la muraille, vers la porte du fumier.
\par 32 Derrière ce choeur marchaient Hosée et la moitié des chefs de Juda,
\par 33 Azaria, Esdras, Meschullam,
\par 34 Juda, Benjamin, Schemaeja et Jérémie,
\par 35 des fils de sacrificateurs avec des trompettes, Zacharie, fils de Jonathan, fils de Schemaeja, fils de Matthania, fils de Michée, fils de Zaccur, fils d'Asaph,
\par 36 et ses frères, Schemaeja, Azareel, Milalaï, Guilalaï, Maaï, Nethaneel, Juda et Hanani, avec les instruments de musique de David, homme de Dieu. Esdras, le scribe, était à leur tête.
\par 37 A la porte de la source, ils montèrent vis-à-vis d'eux les degrés de la cité de David par la montée de la muraille, au-dessus de la maison de David, jusqu'à la porte des eaux, vers l'orient.
\par 38 Le second choeur se mit en marche à l'opposite. J'étais derrière lui avec l'autre moitié du peuple, sur la muraille. Passant au-dessus de la tour des fours, on alla jusqu'à la muraille large;
\par 39 puis au-dessus de la porte d'Éphraïm, de la vieille porte, de la porte des poissons, de la tour de Hananeel et de la tour de Méa, jusqu'à la porte des brebis. Et l'on s'arrêta à la porte de la prison.
\par 40 Les deux choeurs s'arrêtèrent dans la maison de Dieu; et nous fîmes de même, moi et les magistrats qui étaient avec moi,
\par 41 et les sacrificateurs Éliakim, Maaséja, Minjamin, Michée, Eljoénaï, Zacharie, Hanania, avec des trompettes,
\par 42 et Maaséja, Schemaeja, Éléazar, Uzzi, Jochanan, Malkija, Élam et Ézer. Les chantres se firent entendre, dirigés par Jizrachja.
\par 43 On offrit ce jour-là de nombreux sacrifices, et on se livra aux réjouissances, car Dieu avait donné au peuple un grand sujet de joie. Les femmes et les enfants se réjouirent aussi, et les cris de joie de Jérusalem furent entendus au loin.
\par 44 En ce jour, on établit des hommes ayant la surveillance des chambres qui servaient de magasins pour les offrandes, les prémices et les dîmes, et on les chargea d'y recueillir du territoire des villes les portions assignées par la loi aux sacrificateurs et aux Lévites. Car Juda se réjouissait de ce que les sacrificateurs et les Lévites étaient à leur poste,
\par 45 observant tout ce qui concernait le service de Dieu et des purifications. Les chantres et les portiers remplissaient aussi leurs fonctions, selon l'ordre de David et de Salomon, son fils;
\par 46 car autrefois, du temps de David et d'Asaph, il y avait des chefs de chantres et des chants de louanges et d'actions de grâces en l'honneur de Dieu.
\par 47 Tout Israël, au temps de Zorobabel et de Néhémie, donna les portions des chantres et des portiers, jour par jour; on donna aux Lévites les choses consacrées, et les Lévites donnèrent aux fils d'Aaron les choses consacrées.

\chapter{13}

\par 1 Dans ce temps, on lut en présence du peuple dans le livre de Moïse, et l'on y trouva écrit que l'Ammonite et le Moabite ne devraient jamais entrer dans l'assemblée de Dieu,
\par 2 parce qu'ils n'étaient pas venus au-devant des enfants d'Israël avec du pain et de l'eau, et parce qu'ils avaient appelé contre eux à prix d'argent Balaam pour qu'il les maudît; mais notre Dieu changea la malédiction en bénédiction.
\par 3 Lorsqu'on eut entendu la loi, on sépara d'Israël tous les étrangers.
\par 4 Avant cela, le sacrificateur Éliaschib, établi dans les chambres de la maison de notre Dieu, et parent de Tobija,
\par 5 avait disposé pour lui une grande chambre où l'on mettait auparavant les offrandes, l'encens, les ustensiles, la dîme du blé, du moût et de l'huile, ce qui était ordonné pour les Lévites, les chantres et les portiers, et ce qui était prélevé pour les sacrificateurs.
\par 6 Je n'étais point à Jérusalem quand tout cela eut lieu, car j'étais retourné auprès du roi la trente-deuxième année d'Artaxerxès, roi de Babylone.
\par 7 A la fin de l'année, j'obtins du roi la permission de revenir à Jérusalem, et je m'aperçus du mal qu'avait fait Éliaschib, en disposant une chambre pour Tobija dans les parvis de la maison de Dieu.
\par 8 J'en éprouvai un vif déplaisir, et je jetai hors de la chambre tous les objets qui appartenaient à Tobija;
\par 9 j'ordonnai qu'on purifiât les chambres, et j'y replaçai les ustensiles de la maison de Dieu, les offrandes et l'encens.
\par 10 J'appris aussi que les portions des Lévites n'avaient point été livrées, et que les Lévites et les chantres chargés du service s'étaient enfuis chacun dans son territoire.
\par 11 Je fils des réprimandes aux magistrats, et je dis: Pourquoi la maison de Dieu a-t-elle été abandonnée? Et je rassemblai les Lévites et les chantres, et je les remis à leur poste.
\par 12 Alors tout Juda apporta dans les magasins la dîme du blé, du moût et de l'huile.
\par 13 Je confiai la surveillance des magasins à Schélémia, le sacrificateur, à Tsadok, le scribe, et à Pedaja, l'un des Lévites, et je leur adjoignis Hanan, fils de Zaccur, fils de Matthania, car ils avaient la réputation d'être fidèles. Ils furent chargés de faire les distributions à leurs frères.
\par 14 Souviens-toi de moi, ô mon Dieu, à cause de cela, et n'oublie pas mes actes de piété à l'égard de la maison de mon Dieu et des choses qui doivent être observées!
\par 15 A cette époque, je vis en Juda des hommes fouler au pressoir pendant le sabbat, rentrer des gerbes, charger sur des ânes même du vin, des raisins et des figues, et toutes sortes de choses, et les amener à Jérusalem le jour du sabbat; et je leur donnai des avertissements le jour où ils vendaient leurs denrées.
\par 16 Il y avait aussi des Tyriens, établis à Jérusalem, qui apportaient du poisson et toutes sortes de marchandises, et qui les vendaient aux fils de Juda le jour du sabbat et dans Jérusalem.
\par 17 Je fis des réprimandes aux grands de Juda, et je leur dis: Que signifie cette mauvaise action que vous faites, en profanant le jour du sabbat?
\par 18 N'est-ce pas ainsi qu'ont agi vos père, et n'est-ce pas à cause de cela que notre Dieu a fait venir tous ces malheurs sur nous et sur cette ville? Et vous, vous attirez de nouveau sa colère contre Israël, en profanant le sabbat!
\par 19 Puis j'ordonnai qu'on fermât les portes de Jérusalem avant le sabbat, dès qu'elles seraient dans l'ombre, et qu'on ne les ouvrît qu'après le sabbat. Et je plaçai quelques-uns de mes serviteurs aux portes, pour empêcher l'entrée des fardeaux le jour du sabbat.
\par 20 Alors les marchands et les vendeurs de toutes sortes de choses passèrent une ou deux fois la nuit hors de Jérusalem.
\par 21 Je les avertis, en leur disant: Pourquoi passez-vous la nuit devant la muraille? Si vous le faites encore, je mettrai la main sur vous. Dès ce moment, ils ne vinrent plus pendant le sabbat.
\par 22 J'ordonnai aussi aux Lévites de se purifier et de venir garder les portes pour sanctifier le jour du sabbat. Souviens-toi de moi, ô mon Dieu, à cause de cela, et protège-moi selon ta grande miséricorde!
\par 23 A cette même époque, je vis des Juifs qui avaient pris des femmes asdodiennes, ammonites, moabites.
\par 24 La moitié de leurs fils parlaient l'asdodien, et ne savaient pas parler le juif; ils ne connaissaient que la langue de tel ou tel peuple.
\par 25 Je leur fis des réprimandes, et je les maudis; j'en frappai quelques-uns, je leur arrachai les cheveux, et je les fis jurer au nom de Dieu, en disant: Vous ne donnerez pas vos filles à leurs fils, et vous ne prendrez leurs filles ni pour vos fils ni pour vous.
\par 26 N'est-ce pas en cela qu'a péché Salomon, roi d'Israël? Il n'y avait point de roi semblable à lui parmi la multitude des nations, il était aimé de son Dieu, et Dieu l'avait établi roi sur tout Israël; néanmoins, les femmes étrangères l'entraînèrent aussi dans le péché.
\par 27 Faut-il donc apprendre à votre sujet que vous commettez un aussi grand crime et que vous péchez contre notre Dieu en prenant des femmes étrangères?
\par 28 Un des fils de Jojada, fils d'Éliaschib, le souverain sacrificateur, était gendre de Sanballat, le Horonite. Je le chassai loin de moi.
\par 29 Souviens-toi d'eux, ô mon Dieu, car ils ont souillé le sacerdoce et l'alliance contractée par les sacrificateurs et les Lévites.
\par 30 Je les purifiai de tout étranger, et je remis en vigueur ce que devaient observer les sacrificateurs et les Lévites, chacun dans sa fonction,
\par 31 et ce qui concernait l'offrande du bois aux époques fixées, de même que les prémices. Souviens-toi favorablement de moi, ô mon Dieu!


\end{document}