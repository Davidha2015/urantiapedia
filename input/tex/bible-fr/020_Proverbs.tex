\begin{document}

\title{Proverbs}


\chapter{1}

\par 1 Proverbes de Salomon, fils de David, roi d'Israël,
\par 2 Pour connaître la sagesse et l'instruction, Pour comprendre les paroles de l'intelligence;
\par 3 Pour recevoir des leçons de bon sens, De justice, d'équité et de droiture;
\par 4 Pour donner aux simples du discernement, Au jeune homme de la connaissance et de la réflexion.
\par 5 Que le sage écoute, et il augmentera son savoir, Et celui qui est intelligent acquerra de l'habileté,
\par 6 Pour saisir le sens d'un proverbe ou d'une énigme, Des paroles des sages et de leurs sentences.
\par 7 La crainte de l'Éternel est le commencement de la science; Les insensés méprisent la sagesse et l'instruction.
\par 8 Écoute, mon fils, l'instruction de ton père, Et ne rejette pas l'enseignement de ta mère;
\par 9 Car c'est une couronne de grâce pour ta tête, Et une parure pour ton cou.
\par 10 Mon fils, si des pécheurs veulent te séduire, Ne te laisse pas gagner.
\par 11 S'ils disent: Viens avec nous! dressons des embûches, versons du sang, Tendons des pièges à celui qui se repose en vain sur son innocence,
\par 12 Engloutissons-les tout vifs, comme le séjour des morts, Et tout entiers, comme ceux qui descendent dans la fosse;
\par 13 Nous trouverons toute sorte de biens précieux, Nous remplirons de butin nos maisons;
\par 14 Tu auras ta part avec nous, Il n'y aura qu'une bourse pour nous tous!
\par 15 Mon fils, ne te mets pas en chemin avec eux, Détourne ton pied de leur sentier;
\par 16 Car leurs pieds courent au mal, Et ils ont hâte de répandre le sang.
\par 17 Mais en vain jette-t-on le filet Devant les yeux de tout ce qui a des ailes;
\par 18 Et eux, c'est contre leur propre sang qu'ils dressent des embûches, C'est à leur âme qu'ils tendent des pièges.
\par 19 Ainsi arrive-t-il à tout homme avide de gain; La cupidité cause la perte de ceux qui s'y livrent.
\par 20 La sagesse crie dans les rues, Elle élève sa voix dans les places:
\par 21 Elle crie à l'entrée des lieux bruyants; Aux portes, dans la ville, elle fait entendre ses paroles:
\par 22 Jusqu'à quand, stupides, aimerez-vous la stupidité? Jusqu'à quand les moqueurs se plairont-ils à la moquerie, Et les insensés haïront-ils la science?
\par 23 Tournez-vous pour écouter mes réprimandes! Voici, je répandrai sur vous mon esprit, Je vous ferez connaître mes paroles...
\par 24 Puisque j'appelle et que vous résistez, Puisque j'étends ma main et que personne n'y prend garde,
\par 25 Puisque vous rejetez tous mes conseils, Et que vous n'aimez pas mes réprimandes,
\par 26 Moi aussi, je rirai quand vous serez dans le malheur, Je me moquerai quand la terreur vous saisira,
\par 27 Quand la terreur vous saisira comme une tempête, Et que le malheur vous enveloppera comme un tourbillon, Quand la détresse et l'angoisse fondront sur vous.
\par 28 Alors ils m'appelleront, et je ne répondrai pas; Ils me chercheront, et ils ne me trouveront pas.
\par 29 Parce qu'ils ont haï la science, Et qu'ils n'ont pas choisi la crainte de l'Éternel,
\par 30 Parce qu'ils n'ont point aimé mes conseils, Et qu'ils ont dédaigné toutes mes réprimandes,
\par 31 Ils se nourriront du fruit de leur voie, Et ils se rassasieront de leurs propres conseils,
\par 32 Car la résistance des stupides les tue, Et la sécurité des insensés les perd;
\par 33 Mais celui qui m'écoute reposera avec assurance, Il vivra tranquille et sans craindre aucun mal.

\chapter{2}

\par 1 Mon fils, si tu reçois mes paroles, Et si tu gardes avec toi mes préceptes,
\par 2 Si tu rends ton oreille attentive à la sagesse, Et si tu inclines ton coeur à l'intelligence;
\par 3 Oui, si tu appelles la sagesse, Et si tu élèves ta voix vers l'intelligence,
\par 4 Si tu la cherches comme l'argent, Si tu la poursuis comme un trésor,
\par 5 Alors tu comprendras la crainte de l'Éternel, Et tu trouveras la connaissance de Dieu.
\par 6 Car l'Éternel donne la sagesse; De sa bouche sortent la connaissance et l'intelligence;
\par 7 Il tient en réserve le salut pour les hommes droits, Un bouclier pour ceux qui marchent dans l'intégrité,
\par 8 En protégeant les sentiers de la justice Et en gardant la voie de ses fidèles.
\par 9 Alors tu comprendras la justice, l'équité, La droiture, toutes les routes qui mènent au bien.
\par 10 Car la sagesse viendra dans ton coeur, Et la connaissance fera les délices de ton âme;
\par 11 La réflexion veillera sur toi, L'intelligence te gardera,
\par 12 Pour te délivrer de la voie du mal, De l'homme qui tient des discours pervers,
\par 13 De ceux qui abandonnent les sentiers de la droiture Afin de marcher dans des chemins ténébreux,
\par 14 Qui trouvent de la jouissance à faire le mal, Qui mettent leur plaisir dans la perversité,
\par 15 Qui suivent des sentiers détournés, Et qui prennent des routes tortueuses;
\par 16 Pour te délivrer de la femme étrangère, De l'étrangère qui emploie des paroles doucereuses,
\par 17 Qui abandonne l'ami de sa jeunesse, Et qui oublie l'alliance de son Dieu;
\par 18 Car sa maison penche vers la mort, Et sa route mène chez les morts:
\par 19 Aucun de ceux qui vont à elle ne revient, Et ne retrouve les sentiers de la vie.
\par 20 Tu marcheras ainsi dans la voie des gens de bien, Tu garderas les sentiers des justes.
\par 21 Car les hommes droits habiteront le pays, Les hommes intègres y resteront;
\par 22 Mais les méchants seront retranchés du pays, Les infidèles en seront arrachés.

\chapter{3}

\par 1 Mon fils, n'oublie pas mes enseignements, Et que ton coeur garde mes préceptes;
\par 2 Car ils prolongeront les jours et les années de ta vie, Et ils augmenteront ta paix.
\par 3 Que la bonté et la fidélité ne t'abandonnent pas; Lie-les à ton cou, écris-les sur la table de ton coeur.
\par 4 Tu acquerras ainsi de la grâce et une raison saine, Aux yeux de Dieu et des hommes.
\par 5 Confie-toi en l'Éternel de tout ton coeur, Et ne t'appuie pas sur ta sagesse;
\par 6 Reconnais-le dans toutes tes voies, Et il aplanira tes sentiers.
\par 7 Ne sois point sage à tes propres yeux, Crains l'Éternel, et détourne-toi du mal:
\par 8 Ce sera la santé pour tes muscles, Et un rafraîchissement pour tes os.
\par 9 Honore l'Éternel avec tes biens, Et avec les prémices de tout ton revenu:
\par 10 Alors tes greniers seront remplis d'abondance, Et tes cuves regorgeront de moût.
\par 11 Mon fils, ne méprise pas la correction de l'Éternel, Et ne t'effraie point de ses châtiments;
\par 12 Car l'Éternel châtie celui qu'il aime, Comme un père l'enfant qu'il chérit.
\par 13 Heureux l'homme qui a trouvé la sagesse, Et l'homme qui possède l'intelligence!
\par 14 Car le gain qu'elle procure est préférable à celui de l'argent, Et le profit qu'on en tire vaut mieux que l'or;
\par 15 Elle est plus précieuse que les perles, Elle a plus de valeur que tous les objets de prix.
\par 16 Dans sa droite est une longue vie; Dans sa gauche, la richesse et la gloire.
\par 17 Ses voies sont des voies agréables, Et tous ses sentiers sont paisibles.
\par 18 Elle est un arbre de vie pour ceux qui la saisissent, Et ceux qui la possèdent sont heureux.
\par 19 C'est par la sagesse que l'Éternel a fondé la terre, C'est par l'intelligence qu'il a affermi les cieux;
\par 20 C'est par sa science que les abîmes se sont ouverts, Et que les nuages distillent la rosée.
\par 21 Mon fils, que ces enseignements ne s'éloignent pas de tes yeux, Garde la sagesse et la réflexion:
\par 22 Elles seront la vie de ton âme, Et l'ornement de ton cou.
\par 23 Alors tu marcheras avec assurance dans ton chemin, Et ton pied ne heurtera pas.
\par 24 Si tu te couches, tu seras sans crainte; Et quand tu seras couché, ton sommeil sera doux.
\par 25 Ne redoute ni une terreur soudaine, Ni une attaque de la part des méchants;
\par 26 Car l'Éternel sera ton assurance, Et il préservera ton pied de toute embûche.
\par 27 Ne refuse pas un bienfait à celui qui y a droit, Quand tu as le pouvoir de l'accorder.
\par 28 Ne dis pas à ton prochain: Va et reviens, Demain je donnerai! quand tu as de quoi donner.
\par 29 Ne médite pas le mal contre ton prochain, Lorsqu'il demeure tranquillement près de toi.
\par 30 Ne conteste pas sans motif avec quelqu'un, Lorsqu'il ne t'a point fait de mal.
\par 31 Ne porte pas envie à l'homme violent, Et ne choisis aucune de ses voies.
\par 32 Car l'Éternel a en horreur les hommes pervers, Mais il est un ami pour les hommes droits;
\par 33 La malédiction de l'Éternel est dans la maison du méchant, Mais il bénit la demeure des justes;
\par 34 Il se moque des moqueurs, Mais il fait grâce aux humbles;
\par 35 Les sages hériteront la gloire, Mais les insensés ont la honte en partage.

\chapter{4}

\par 1 Écoutez, mes fils, l'instruction d'un père, Et soyez attentifs, pour connaître la sagesse;
\par 2 Car je vous donne de bons conseils: Ne rejetez pas mon enseignement.
\par 3 J'étais un fils pour mon père, Un fils tendre et unique auprès de ma mère.
\par 4 Il m'instruisait alors, et il me disait: Que ton coeur retienne mes paroles; Observe mes préceptes, et tu vivras.
\par 5 Acquiers la sagesse, acquiers l'intelligence; N'oublie pas les paroles de ma bouche, et ne t'en détourne pas.
\par 6 Ne l'abandonne pas, et elle te gardera; Aime-la, et elle te protégera.
\par 7 Voici le commencement de la sagesse: Acquiers la sagesse, Et avec tout ce que tu possèdes acquiers l'intelligence.
\par 8 Exalte-la, et elle t'élèvera; Elle fera ta gloire, si tu l'embrasses;
\par 9 Elle mettra sur ta tête une couronne de grâce, Elle t'ornera d'un magnifique diadème.
\par 10 Écoute, mon fils, et reçois mes paroles; Et les années de ta vie se multiplieront.
\par 11 Je te montre la voie de la sagesse, Je te conduis dans les sentiers de la droiture.
\par 12 Si tu marches, ton pas ne sera point gêné; Et si tu cours, tu ne chancelleras point.
\par 13 Retiens l'instruction, ne t'en dessaisis pas; Garde-la, car elle est ta vie.
\par 14 N'entre pas dans le sentier des méchants, Et ne marche pas dans la voie des hommes mauvais.
\par 15 Évite-la, n'y passe point; Détourne-t'en, et passe outre.
\par 16 Car ils ne dormiraient pas s'ils n'avaient fait le mal, Le sommeil leur serait ravi s'ils n'avaient fait tomber personne;
\par 17 Car c'est le pain de la méchanceté qu'ils mangent, C'est le vin de la violence qu'ils boivent.
\par 18 Le sentier des justes est comme la lumière resplendissante, Dont l'éclat va croissant jusqu'au milieu du jour.
\par 19 La voie des méchants est comme les ténèbres; Ils n'aperçoivent pas ce qui les fera tomber.
\par 20 Mon fils, sois attentif à mes paroles, Prête l'oreille à mes discours.
\par 21 Qu'ils ne s'éloignent pas de tes yeux; Garde-les dans le fond de ton coeur;
\par 22 Car c'est la vie pour ceux qui les trouvent, C'est la santé pour tout leur corps.
\par 23 Garde ton coeur plus que toute autre chose, Car de lui viennent les sources de la vie.
\par 24 Écarte de ta bouche la fausseté, Éloigne de tes lèvres les détours.
\par 25 Que tes yeux regardent en face, Et que tes paupières se dirigent devant toi.
\par 26 Considère le chemin par où tu passes, Et que toutes tes voies soient bien réglées;
\par 27 N'incline ni à droite ni à gauche, Et détourne ton pied du mal.

\chapter{5}

\par 1 Mon fils, sois attentif à ma sagesse, Prête l'oreille à mon intelligence,
\par 2 Afin que tu conserves la réflexion, Et que tes lèvres gardent la connaissance.
\par 3 Car les lèvres de l'étrangère distillent le miel, Et son palais est plus doux que l'huile;
\par 4 Mais à la fin elle est amère comme l'absinthe, Aiguë comme un glaive à deux tranchants.
\par 5 Ses pieds descendent vers la mort, Ses pas atteignent le séjour des morts.
\par 6 Afin de ne pas considérer le chemin de la vie, Elle est errante dans ses voies, elle ne sait où elle va.
\par 7 Et maintenant, mes fils, écoutez-moi, Et ne vous écartez pas des paroles de ma bouche.
\par 8 Éloigne-toi du chemin qui conduit chez elle, Et ne t'approche pas de la porte de sa maison,
\par 9 De peur que tu ne livres ta vigueur à d'autres, Et tes années à un homme cruel;
\par 10 De peur que des étrangers ne se rassasient de ton bien, Et du produit de ton travail dans la maison d'autrui;
\par 11 De peur que tu ne gémisses, près de ta fin, Quand ta chair et ton corps se consumeront,
\par 12 Et que tu ne dises: Comment donc ai-je pu haïr la correction, Et comment mon coeur a-t-il dédaigné la réprimande?
\par 13 Comment ai-je pu ne pas écouter la voix de mes maîtres, Ne pas prêter l'oreille à ceux qui m'instruisaient?
\par 14 Peu s'en est fallu que je n'aie éprouvé tous les malheurs Au milieu du peuple et de l'assemblée.
\par 15 Bois les eaux de ta citerne, Les eaux qui sortent de ton puits.
\par 16 Tes sources doivent-elles se répandre au dehors? Tes ruisseaux doivent ils couler sur les places publiques?
\par 17 Qu'ils soient pour toi seul, Et non pour des étrangers avec toi.
\par 18 Que ta source soit bénie, Et fais ta joie de la femme de ta jeunesse,
\par 19 Biche des amours, gazelle pleine de grâce: Sois en tout temps enivré de ses charmes, Sans cesse épris de son amour.
\par 20 Et pourquoi, mon fils, serais-tu épris d'une étrangère, Et embrasserais-tu le sein d'une inconnue?
\par 21 Car les voies de l'homme sont devant les yeux de l'Éternel, Qui observe tous ses sentiers.
\par 22 Le méchant est pris dans ses propres iniquités, Il est saisi par les liens de son péché.
\par 23 Il mourra faute d'instruction, Il chancellera par l'excès de sa folie.

\chapter{6}

\par 1 Mon fils, si tu as cautionné ton prochain, Si tu t'es engagé pour autrui,
\par 2 Si tu es enlacé par les paroles de ta bouche, Si tu es pris par les paroles de ta bouche,
\par 3 Fais donc ceci, mon fils, dégage-toi, Puisque tu es tombé au pouvoir de ton prochain; Va, prosterne-toi, et fais des instances auprès de lui;
\par 4 Ne donne ni sommeil à tes yeux, Ni assoupissement à tes paupières;
\par 5 Dégage-toi comme la gazelle de la main du chasseur, Comme l'oiseau de la main de l'oiseleur.
\par 6 Va vers la fourmi, paresseux; Considère ses voies, et deviens sage.
\par 7 Elle n'a ni chef, Ni inspecteur, ni maître;
\par 8 Elle prépare en été sa nourriture, Elle amasse pendant la moisson de quoi manger.
\par 9 Paresseux, jusqu'à quand seras-tu couché? Quand te lèveras-tu de ton sommeil?
\par 10 Un peu de sommeil, un peu d'assoupissement, Un peu croiser les mains pour dormir!...
\par 11 Et la pauvreté te surprendra, comme un rôdeur, Et la disette, comme un homme en armes.
\par 12 L'homme pervers, l'homme inique, Marche la fausseté dans la bouche;
\par 13 Il cligne des yeux, parle du pied, Fait des signes avec les doigts;
\par 14 La perversité est dans son coeur, Il médite le mal en tout temps, Il excite des querelles.
\par 15 Aussi sa ruine arrivera-t-elle subitement; Il sera brisé tout d'un coup, et sans remède.
\par 16 Il y a six choses que hait l'Éternel, Et même sept qu'il a en horreur;
\par 17 Les yeux hautains, la langue menteuse, Les mains qui répandent le sang innocent,
\par 18 Le coeur qui médite des projets iniques, Les pieds qui se hâtent de courir au mal,
\par 19 Le faux témoin qui dit des mensonges, Et celui qui excite des querelles entre frères.
\par 20 Mon fils, garde les préceptes de ton père, Et ne rejette pas l'enseignement de ta mère.
\par 21 Lie-les constamment sur ton coeur, Attache-les à ton cou.
\par 22 Ils te dirigeront dans ta marche, Ils te garderont sur ta couche, Ils te parleront à ton réveil.
\par 23 Car le précepte est une lampe, et l'enseignement une lumière, Et les avertissements de la correction sont le chemin de la vie:
\par 24 Ils te préserveront de la femme corrompue, De la langue doucereuse de l'étrangère.
\par 25 Ne la convoite pas dans ton coeur pour sa beauté, Et ne te laisse pas séduire par ses paupières.
\par 26 Car pour la femme prostituée on se réduit à un morceau de pain, Et la femme mariée tend un piège à la vie précieuse.
\par 27 Quelqu'un mettra-t-il du feu dans son sein, Sans que ses vêtements s'enflamment?
\par 28 Quelqu'un marchera-t-il sur des charbons ardents, Sans que ses pieds soient brûlés?
\par 29 Il en est de même pour celui qui va vers la femme de son prochain: Quiconque la touche ne restera pas impuni.
\par 30 On ne tient pas pour innocent le voleur qui dérobe Pour satisfaire son appétit, quand il a faim;
\par 31 Si on le trouve, il fera une restitution au septuple, Il donnera tout ce qu'il a dans sa maison.
\par 32 Mais celui qui commet un adultère avec une femme est dépourvu de sens, Celui qui veut se perdre agit de la sorte;
\par 33 Il n'aura que plaie et ignominie, Et son opprobre ne s'effacera point.
\par 34 Car la jalousie met un homme en fureur, Et il est sans pitié au jour de la vengeance;
\par 35 Il n'a égard à aucune rançon, Et il est inflexible, quand même tu multiplierais les dons.

\chapter{7}

\par 1 Mon fils, retiens mes paroles, Et garde avec toi mes préceptes.
\par 2 Observe mes préceptes, et tu vivras; Garde mes enseignements comme la prunelle de tes yeux.
\par 3 Lie-les sur tes doigts, Écris-les sur la table de ton coeur.
\par 4 Dis à la sagesse: Tu es ma soeur! Et appelle l'intelligence ton amie,
\par 5 Pour qu'elles te préservent de la femme étrangère, De l'étrangère qui emploie des paroles doucereuses.
\par 6 J'étais à la fenêtre de ma maison, Et je regardais à travers mon treillis.
\par 7 J'aperçus parmi les stupides, Je remarquai parmi les jeunes gens un garçon dépourvu de sens.
\par 8 Il passait dans la rue, près de l'angle où se tenait une de ces étrangères, Et il se dirigeait lentement du côté de sa demeure:
\par 9 C'était au crépuscule, pendant la soirée, Au milieu de la nuit et de l'obscurité.
\par 10 Et voici, il fut abordé par une femme Ayant la mise d'une prostituée et la ruse dans le coeur.
\par 11 Elle était bruyante et rétive; Ses pieds ne restaient point dans sa maison;
\par 12 Tantôt dans la rue, tantôt sur les places, Et près de tous les angles, elle était aux aguets.
\par 13 Elle le saisit et l'embrassa, Et d'un air effronté lui dit:
\par 14 Je devais un sacrifice d'actions de grâces, Aujourd'hui j'ai accompli mes voeux.
\par 15 C'est pourquoi je suis sortie au-devant de toi Pour te chercher, et je t'ai trouvé.
\par 16 J'ai orné mon lit de couvertures, De tapis de fil d'Égypte;
\par 17 J'ai parfumé ma couche De myrrhe, d'aloès et de cinnamome.
\par 18 Viens, enivrons-nous d'amour jusqu'au matin, Livrons-nous joyeusement à la volupté.
\par 19 Car mon mari n'est pas à la maison, Il est parti pour un voyage lointain;
\par 20 Il a pris avec lui le sac de l'argent, Il ne reviendra à la maison qu'à la nouvelle lune.
\par 21 Elle le séduisit à force de paroles, Elle l'entraîna par ses lèvres doucereuses.
\par 22 Il se mit tout à coup à la suivre, Comme le boeuf qui va à la boucherie, Comme un fou qu'on lie pour le châtier,
\par 23 Jusqu'à ce qu'une flèche lui perce le foie, Comme l'oiseau qui se précipite dans le filet, Sans savoir que c'est au prix de sa vie.
\par 24 Et maintenant, mes fils, écoutez-moi, Et soyez attentifs aux paroles de ma bouche.
\par 25 Que ton coeur ne se détourne pas vers les voies d'une telle femme, Ne t'égare pas dans ses sentiers.
\par 26 Car elle a fait tomber beaucoup de victimes, Et ils sont nombreux, tous ceux qu'elle a tués.
\par 27 Sa maison, c'est le chemin du séjour des morts; Il descend vers les demeures de la mort.

\chapter{8}

\par 1 La sagesse ne crie-t-elle pas? L'intelligence n'élève-t-elle pas sa voix?
\par 2 C'est au sommet des hauteurs près de la route, C'est à la croisée des chemins qu'elle se place;
\par 3 A côté des portes, à l'entrée de la ville, A l'intérieur des portes, elle fait entendre ses cris:
\par 4 Hommes, c'est à vous que je crie, Et ma voix s'adresse aux fils de l'homme.
\par 5 Stupides, apprenez le discernement; Insensés, apprenez l'intelligence.
\par 6 Écoutez, car j'ai de grandes choses à dire, Et mes lèvres s'ouvrent pour enseigner ce qui est droit.
\par 7 Car ma bouche proclame la vérité, Et mes lèvres ont en horreur le mensonge;
\par 8 Toutes les paroles de ma bouche sont justes, Elles n'ont rien de faux ni de détourné;
\par 9 Toutes sont claires pour celui qui est intelligent, Et droites pour ceux qui ont trouvé la science.
\par 10 Préférez mes instructions à l'argent, Et la science à l'or le plus précieux;
\par 11 Car la sagesse vaut mieux que les perles, Elle a plus de valeur que tous les objets de prix.
\par 12 Moi, la sagesse, j'ai pour demeure le discernement, Et je possède la science de la réflexion.
\par 13 La crainte de l'Éternel, c'est la haine du mal; L'arrogance et l'orgueil, la voie du mal, Et la bouche perverse, voilà ce que je hais.
\par 14 Le conseil et le succès m'appartiennent; Je suis l'intelligence, la force est à moi.
\par 15 Par moi les rois règnent, Et les princes ordonnent ce qui est juste;
\par 16 Par moi gouvernent les chefs, Les grands, tous les juges de la terre.
\par 17 J'aime ceux qui m'aiment, Et ceux qui me cherchent me trouvent.
\par 18 Avec moi sont la richesse et la gloire, Les biens durables et la justice.
\par 19 Mon fruit est meilleur que l'or, que l'or pur, Et mon produit est préférable à l'argent.
\par 20 Je marche dans le chemin de la justice, Au milieu des sentiers de la droiture,
\par 21 Pour donner des biens à ceux qui m'aiment, Et pour remplir leurs trésors.
\par 22 L'Éternel m'a créée la première de ses oeuvres, Avant ses oeuvres les plus anciennes.
\par 23 J'ai été établie depuis l'éternité, Dès le commencement, avant l'origine de la terre.
\par 24 Je fus enfantée quand il n'y avait point d'abîmes, Point de sources chargées d'eaux;
\par 25 Avant que les montagnes soient affermies, Avant que les collines existent, je fus enfantée;
\par 26 Il n'avait encore fait ni la terre, ni les campagnes, Ni le premier atome de la poussière du monde.
\par 27 Lorsqu'il disposa les cieux, j'étais là; Lorsqu'il traça un cercle à la surface de l'abîme,
\par 28 Lorsqu'il fixa les nuages en haut, Et que les sources de l'abîme jaillirent avec force,
\par 29 Lorsqu'il donna une limite à la mer, Pour que les eaux n'en franchissent pas les bords, Lorsqu'il posa les fondements de la terre,
\par 30 J'étais à l'oeuvre auprès de lui, Et je faisais tous les jours ses délices, Jouant sans cesse en sa présence,
\par 31 Jouant sur le globe de sa terre, Et trouvant mon bonheur parmi les fils de l'homme.
\par 32 Et maintenant, mes fils, écoutez-moi, Et heureux ceux qui observent mes voies!
\par 33 Écoutez l'instruction, pour devenir sages, Ne la rejetez pas.
\par 34 Heureux l'homme qui m'écoute, Qui veille chaque jour à mes portes, Et qui en garde les poteaux!
\par 35 Car celui qui me trouve a trouvé la vie, Et il obtient la faveur de l'Éternel.
\par 36 Mais celui qui pèche contre moi nuit à son âme; Tous ceux qui me haïssent aiment la mort.

\chapter{9}

\par 1 La sagesse a bâti sa maison, Elle a taillé ses sept colonnes.
\par 2 Elle a égorgé ses victimes, mêlé son vin, Et dressé sa table.
\par 3 Elle a envoyé ses servantes, elle crie Sur le sommet des hauteurs de la ville:
\par 4 Que celui qui est stupide entre ici! Elle dit à ceux qui sont dépourvus de sens:
\par 5 Venez, mangez de mon pain, Et buvez du vin que j'ai mêlé;
\par 6 Quittez la stupidité, et vous vivrez, Et marchez dans la voie de l'intelligence!
\par 7 Celui qui reprend le moqueur s'attire le dédain, Et celui qui corrige le méchant reçoit un outrage.
\par 8 Ne reprends pas le moqueur, de crainte qu'il ne te haïsse; Reprends le sage, et il t'aimera.
\par 9 Donne au sage, et il deviendra plus sage; Instruis le juste, et il augmentera son savoir.
\par 10 Le commencement de la sagesse, c'est la crainte de l'Éternel; Et la science des saints, c'est l'intelligence.
\par 11 C'est par moi que tes jours se multiplieront, Et que les années de ta vie augmenteront.
\par 12 Si tu es sage, tu es sage pour toi; Si tu es moqueur, tu en porteras seul la peine.
\par 13 La folie est une femme bruyante, Stupide et ne sachant rien.
\par 14 Elle s'assied à l'entrée de sa maison, Sur un siège, dans les hauteurs de la ville,
\par 15 Pour crier aux passants, Qui vont droit leur chemin:
\par 16 Que celui qui est stupide entre ici! Elle dit à celui qui est dépourvu de sens:
\par 17 Les eaux dérobées sont douces, Et le pain du mystère est agréable!
\par 18 Et il ne sait pas que là sont les morts, Et que ses invités sont dans les vallées du séjour des morts.

\chapter{10}

\par 1 Proverbes de Salomon. Un fils sage fait la joie d'un père, Et un fils insensé le chagrin de sa mère.
\par 2 Les trésors de la méchanceté ne profitent pas, Mais la justice délivre de la mort.
\par 3 L'Éternel ne laisse pas le juste souffrir de la faim, Mais il repousse l'avidité des méchants.
\par 4 Celui qui agit d'une main lâche s'appauvrit, Mais la main des diligents enrichit.
\par 5 Celui qui amasse pendant l'été est un fils prudent, Celui qui dort pendant la moisson est un fils qui fait honte.
\par 6 Il y a des bénédictions sur la tête du juste, Mais la violence couvre la bouche des méchants.
\par 7 La mémoire du juste est en bénédiction, Mais le nom des méchants tombe en pourriture.
\par 8 Celui qui est sage de coeur reçoit les préceptes, Mais celui qui est insensé des lèvres court à sa perte.
\par 9 Celui qui marche dans l'intégrité marche avec assurance, Mais celui qui prend des voies tortueuses sera découvert.
\par 10 Celui qui cligne des yeux est une cause de chagrin, Et celui qui est insensé des lèvres court à sa perte.
\par 11 La bouche du juste est une source de vie, Mais la violence couvre la bouche des méchants.
\par 12 La haine excite des querelles, Mais l'amour couvre toutes les fautes.
\par 13 Sur les lèvres de l'homme intelligent se trouve la sagesse, Mais la verge est pour le dos de celui qui est dépourvu de sens.
\par 14 Les sages tiennent la science en réserve, Mais la bouche de l'insensé est une ruine prochaine.
\par 15 La fortune est pour le riche une ville forte; La ruine des misérables, c'est leur pauvreté.
\par 16 L'oeuvre du juste est pour la vie, Le gain du méchant est pour le péché.
\par 17 Celui qui se souvient de la correction prend le chemin de la vie, Mais celui qui oublie la réprimande s'égare.
\par 18 Celui qui dissimule la haine a des lèvres menteuses, Et celui qui répand la calomnie est un insensé.
\par 19 Celui qui parle beaucoup ne manque pas de pécher, Mais celui qui retient ses lèvres est un homme prudent.
\par 20 La langue du juste est un argent de choix; Le coeur des méchants est peu de chose.
\par 21 Les lèvres du juste dirigent beaucoup d'hommes, Et les insensés meurent par défaut de raison.
\par 22 C'est la bénédiction de l'Éternel qui enrichit, Et il ne la fait suivre d'aucun chagrin.
\par 23 Commettre le crime paraît un jeu à l'insensé, Mais la sagesse appartient à l'homme intelligent.
\par 24 Ce que redoute le méchant, c'est ce qui lui arrive; Et ce que désirent les justes leur est accordé.
\par 25 Comme passe le tourbillon, ainsi disparaît le méchant; Mais le juste a des fondements éternels.
\par 26 Ce que le vinaigre est aux dents et la fumée aux yeux, Tel est le paresseux pour celui qui l'envoie.
\par 27 La crainte de l'Éternel augmente les jours, Mais les années des méchants sont abrégées.
\par 28 L'attente des justes n'est que joie, Mais l'espérance des méchants périra.
\par 29 La voie de l'Éternel est un rempart pour l'intégrité, Mais elle est une ruine pour ceux qui font le mal.
\par 30 Le juste ne chancellera jamais, Mais les méchants n'habiteront pas le pays.
\par 31 La bouche du juste produit la sagesse, Mais la langue perverse sera retranchée.
\par 32 Les lèvres du juste connaissent la grâce, Et la bouche des méchants la perversité.

\chapter{11}

\par 1 La balance fausse est en horreur à l'Éternel, Mais le poids juste lui est agréable.
\par 2 Quand vient l'orgueil, vient aussi l'ignominie; Mais la sagesse est avec les humbles.
\par 3 L'intégrité des hommes droits les dirige, Mais les détours des perfides causent leur ruine.
\par 4 Au jour de la colère, la richesse ne sert à rien; Mais la justice délivre de la mort.
\par 5 La justice de l'homme intègre aplanit sa voie, Mais le méchant tombe par sa méchanceté.
\par 6 La justice des hommes droits les délivre, Mais les méchants sont pris par leur malice.
\par 7 A la mort du méchant, son espoir périt, Et l'attente des hommes iniques est anéantie.
\par 8 Le juste est délivré de la détresse, Et le méchant prend sa place.
\par 9 Par sa bouche l'impie perd son prochain, Mais les justes sont délivrés par la science.
\par 10 Quand les justes sont heureux, la ville est dans la joie; Et quand les méchants périssent, on pousse des cris d'allégresse.
\par 11 La ville s'élève par la bénédiction des hommes droits, Mais elle est renversée par la bouche des méchants.
\par 12 Celui qui méprise son prochain est dépourvu de sens, Mais l'homme qui a de l'intelligence se tait.
\par 13 Celui qui répand la calomnie dévoile les secrets, Mais celui qui a l'esprit fidèle les garde.
\par 14 Quand la prudence fait défaut, le peuple tombe; Et le salut est dans le grand nombre des conseillers.
\par 15 Celui qui cautionne autrui s'en trouve mal, Mais celui qui craint de s'engager est en sécurité.
\par 16 Une femme qui a de la grâce obtient la gloire, Et ceux qui ont de la force obtiennent la richesse.
\par 17 L'homme bon fait du bien à son âme, Mais l'homme cruel trouble sa propre chair.
\par 18 Le méchant fait un gain trompeur, Mais celui qui sème la justice a un salaire véritable.
\par 19 Ainsi la justice conduit à la vie, Mais celui qui poursuit le mal trouve la mort.
\par 20 Ceux qui ont le coeur pervers sont en abomination à l'Éternel, Mais ceux dont la voie est intègre lui sont agréables.
\par 21 Certes, le méchant ne restera pas impuni, Mais la postérité des justes sera sauvée.
\par 22 Un anneau d'or au nez d'un pourceau, C'est une femme belle et dépourvue de sens.
\par 23 Le désir des justes, c'est seulement le bien; L'attente des méchants, c'est la fureur.
\par 24 Tel, qui donne libéralement, devient plus riche; Et tel, qui épargne à l'excès, ne fait que s'appauvrir.
\par 25 L'âme bienfaisante sera rassasiée, Et celui qui arrose sera lui-même arrosé.
\par 26 Celui qui retient le blé est maudit du peuple, Mais la bénédiction est sur la tête de celui qui le vend.
\par 27 Celui qui recherche le bien s'attire de la faveur, Mais celui qui poursuit le mal en est atteint.
\par 28 Celui qui se confie dans ses richesses tombera, Mais les justes verdiront comme le feuillage.
\par 29 Celui qui trouble sa maison héritera du vent, Et l'insensé sera l'esclave de l'homme sage.
\par 30 Le fruit du juste est un arbre de vie, Et le sage s'empare des âmes.
\par 31 Voici, le juste reçoit sur la terre une rétribution; Combien plus le méchant et le pécheur!

\chapter{12}

\par 1 Celui qui aime la correction aime la science; Celui qui hait la réprimande est stupide.
\par 2 L'homme de bien obtient la faveur de l'Éternel, Mais l'Éternel condamne celui qui est plein de malice.
\par 3 L'homme ne s'affermit pas par la méchanceté, Mais la racine des justes ne sera point ébranlée.
\par 4 Une femme vertueuse est la couronne de son mari, Mais celle qui fait honte est comme la carie dans ses os.
\par 5 Les pensées des justes ne sont qu'équité; Les desseins des méchants ne sont que fraude.
\par 6 Les paroles des méchants sont des embûches pour verser le sang, Mais la bouche des hommes droits est une délivrance.
\par 7 Renversés, les méchants ne sont plus; Et la maison des justes reste debout.
\par 8 Un homme est estimé en raison de son intelligence, Et celui qui a le coeur pervers est l'objet du mépris.
\par 9 Mieux vaut être d'une condition humble et avoir un serviteur Que de faire le glorieux et de manquer de pain.
\par 10 Le juste prend soin de son bétail, Mais les entrailles des méchants sont cruelles.
\par 11 Celui qui cultive son champ est rassasié de pain, Mais celui qui poursuit des choses vaines est dépourvu de sens.
\par 12 Le méchant convoite ce que prennent les méchants, Mais la racine des justes donne du fruit.
\par 13 Il y a dans le péché des lèvres un piège pernicieux, Mais le juste se tire de la détresse.
\par 14 Par le fruit de la bouche on est rassasié de biens, Et chacun reçoit selon l'oeuvre de ses mains.
\par 15 La voie de l'insensé est droite à ses yeux, Mais celui qui écoute les conseils est sage.
\par 16 L'insensé laisse voir à l'instant sa colère, Mais celui qui cache un outrage est un homme prudent.
\par 17 Celui qui dit la vérité proclame la justice, Et le faux témoin la tromperie.
\par 18 Tel, qui parle légèrement, blesse comme un glaive; Mais la langue des sages apporte la guérison.
\par 19 La lèvre véridique est affermie pour toujours, Mais la langue fausse ne subsiste qu'un instant.
\par 20 La tromperie est dans le coeur de ceux qui méditent le mal, Mais la joie est pour ceux qui conseillent la paix.
\par 21 Aucun malheur n'arrive au juste, Mais les méchants sont accablés de maux.
\par 22 Les lèvres fausses sont en horreur à l'Éternel, Mais ceux qui agissent avec vérité lui sont agréables.
\par 23 L'homme prudent cache sa science, Mais le coeur des insensés proclame la folie.
\par 24 La main des diligents dominera, Mais la main lâche sera tributaire.
\par 25 L'inquiétude dans le coeur de l'homme l'abat, Mais une bonne parole le réjouit.
\par 26 Le juste montre à son ami la bonne voie, Mais la voie des méchants les égare.
\par 27 Le paresseux ne rôtit pas son gibier; Mais le précieux trésor d'un homme, c'est l'activité.
\par 28 La vie est dans le sentier de la justice, La mort n'est pas dans le chemin qu'elle trace.

\chapter{13}

\par 1 Un fils sage écoute l'instruction de son père, Mais le moqueur n'écoute pas la réprimande.
\par 2 Par le fruit de la bouche on jouit du bien; Mais ce que désirent les perfides, c'est la violence.
\par 3 Celui qui veille sur sa bouche garde son âme; Celui qui ouvre de grandes lèvres court à sa perte.
\par 4 L'âme du paresseux a des désirs qu'il ne peut satisfaire; Mais l'âme des hommes diligents sera rassasiée.
\par 5 Le juste hait les paroles mensongères; Le méchant se rend odieux et se couvre de honte.
\par 6 La justice garde celui dont la voie est intègre, Mais la méchanceté cause la ruine du pécheur.
\par 7 Tel fait le riche et n'a rien du tout, Tel fait le pauvre et a de grands biens.
\par 8 La richesse d'un homme sert de rançon pour sa vie, Mais le pauvre n'écoute pas la réprimande.
\par 9 La lumière des justes est joyeuse, Mais la lampe des méchants s'éteint.
\par 10 C'est seulement par orgueil qu'on excite des querelles, Mais la sagesse est avec ceux qui écoutent les conseils.
\par 11 La richesse mal acquise diminue, Mais celui qui amasse peu à peu l'augmente.
\par 12 Un espoir différé rend le coeur malade, Mais un désir accompli est un arbre de vie.
\par 13 Celui qui méprise la parole se perd, Mais celui qui craint le précepte est récompensé.
\par 14 L'enseignement du sage est une source de vie, Pour détourner des pièges de la mort.
\par 15 Une raison saine a pour fruit la grâce, Mais la voie des perfides est rude.
\par 16 Tout homme prudent agit avec connaissance, Mais l'insensé fait étalage de folie.
\par 17 Un envoyé méchant tombe dans le malheur, Mais un messager fidèle apporte la guérison.
\par 18 La pauvreté et la honte sont le partage de celui qui rejette la correction, Mais celui qui a égard à la réprimande est honoré.
\par 19 Un désir accompli est doux à l'âme, Mais s'éloigner du mal fait horreur aux insensés.
\par 20 Celui qui fréquente les sages devient sage, Mais celui qui se plaît avec les insensés s'en trouve mal.
\par 21 Le malheur poursuit ceux qui pèchent, Mais le bonheur récompense les justes.
\par 22 L'homme de bien a pour héritiers les enfants de ses enfants, Mais les richesses du pécheur sont réservées pour le juste.
\par 23 Le champ que défriche le pauvre donne une nourriture abondante, Mais tel périt par défaut de justice.
\par 24 Celui qui ménage sa verge hait son fils, Mais celui qui l'aime cherche à le corriger.
\par 25 Le juste mange et satisfait son appétit, Mais le ventre des méchants éprouve la disette.

\chapter{14}

\par 1 La femme sage bâtit sa maison, Et la femme insensée la renverse de ses propres mains.
\par 2 Celui qui marche dans la droiture craint l'Éternel, Mais celui qui prend des voies tortueuses le méprise.
\par 3 Dans la bouche de l'insensé est une verge pour son orgueil, Mais les lèvres des sages les gardent.
\par 4 S'il n'y a pas de boeufs, la crèche est vide; C'est à la vigueur des boeufs qu'on doit l'abondance des revenus.
\par 5 Un témoin fidèle ne ment pas, Mais un faux témoin dit des mensonges.
\par 6 Le moqueur cherche la sagesse et ne la trouve pas, Mais pour l'homme intelligent la science est chose facile.
\par 7 Éloigne-toi de l'insensé; Ce n'est pas sur ses lèvres que tu aperçois la science.
\par 8 La sagesse de l'homme prudent, c'est l'intelligence de sa voie; La folie des insensés, c'est la tromperie.
\par 9 Les insensés se font un jeu du péché, Mais parmi les hommes droits se trouve la bienveillance.
\par 10 Le coeur connaît ses propres chagrins, Et un étranger ne saurait partager sa joie.
\par 11 La maison des méchants sera détruite, Mais la tente des hommes droits fleurira.
\par 12 Telle voie paraît droite à un homme, Mais son issue, c'est la voie de la mort.
\par 13 Au milieu même du rire le coeur peut être affligé, Et la joie peut finir par la détresse.
\par 14 Celui dont le coeur s'égare se rassasie de ses voies, Et l'homme de bien se rassasie de ce qui est en lui.
\par 15 L'homme simple croit tout ce qu'on dit, Mais l'homme prudent est attentif à ses pas.
\par 16 Le sage a de la retenue et se détourne du mal, Mais l'insensé est arrogant et plein de sécurité.
\par 17 Celui qui est prompt à la colère fait des sottises, Et l'homme plein de malice s'attire la haine.
\par 18 Les simples ont en partage la folie, Et les hommes prudents se font de la science une couronne.
\par 19 Les mauvais s'inclinent devant les bons, Et les méchants aux portes du juste.
\par 20 Le pauvre est odieux même à son ami, Mais les amis du riche sont nombreux.
\par 21 Celui qui méprise son prochain commet un péché, Mais heureux celui qui a pitié des misérables!
\par 22 Ceux qui méditent le mal ne s'égarent-ils pas? Mais ceux qui méditent le bien agissent avec bonté et fidélité.
\par 23 Tout travail procure l'abondance, Mais les paroles en l'air ne mènent qu'à la disette.
\par 24 La richesse est une couronne pour les sages; La folie des insensés est toujours de la folie.
\par 25 Le témoin véridique délivre des âmes, Mais le trompeur dit des mensonges.
\par 26 Celui qui craint l'Éternel possède un appui ferme, Et ses enfants ont un refuge auprès de lui.
\par 27 La crainte de l'Éternel est une source de vie, Pour détourner des pièges de la mort.
\par 28 Quand le peuple est nombreux, c'est la gloire d'un roi; Quand le peuple manque, c'est la ruine du prince.
\par 29 Celui qui est lent à la colère a une grande intelligence, Mais celui qui est prompt à s'emporter proclame sa folie.
\par 30 Un coeur calme est la vie du corps, Mais l'envie est la carie des os.
\par 31 Opprimer le pauvre, c'est outrager celui qui l'a fait; Mais avoir pitié de l'indigent, c'est l'honorer.
\par 32 Le méchant est renversé par sa méchanceté, Mais le juste trouve un refuge même en sa mort.
\par 33 Dans un coeur intelligent repose la sagesse, Mais au milieu des insensés elle se montre à découvert.
\par 34 La justice élève une nation, Mais le péché est la honte des peuples.
\par 35 La faveur du roi est pour le serviteur prudent, Et sa colère pour celui qui fait honte.

\chapter{15}

\par 1 Une réponse douce calme la fureur, Mais une parole dure excite la colère.
\par 2 La langue des sages rend la science aimable, Et la bouche des insensés répand la folie.
\par 3 Les yeux de l'Éternel sont en tout lieu, Observant les méchants et les bons.
\par 4 La langue douce est un arbre de vie, Mais la langue perverse brise l'âme.
\par 5 L'insensé dédaigne l'instruction de son père, Mais celui qui a égard à la réprimande agit avec prudence.
\par 6 Il y a grande abondance dans la maison du juste, Mais il y a du trouble dans les profits du méchant.
\par 7 Les lèvres des sages répandent la science, Mais le coeur des insensés n'est pas droit.
\par 8 Le sacrifice des méchants est en horreur à l'Éternel, Mais la prière des hommes droits lui est agréable.
\par 9 La voie du méchant est en horreur à l'Éternel, Mais il aime celui qui poursuit la justice.
\par 10 Une correction sévère menace celui qui abandonne le sentier; Celui qui hait la réprimande mourra.
\par 11 Le séjour des morts et l'abîme sont devant l'Éternel; Combien plus les coeurs des fils de l'homme!
\par 12 Le moqueur n'aime pas qu'on le reprenne, Il ne va point vers les sages.
\par 13 Un coeur joyeux rend le visage serein; Mais quand le coeur est triste, l'esprit est abattu.
\par 14 Un coeur intelligent cherche la science, Mais la bouche des insensés se plaît à la folie.
\par 15 Tous les jours du malheureux sont mauvais, Mais le coeur content est un festin perpétuel.
\par 16 Mieux vaut peu, avec la crainte de l'Éternel, Qu'un grand trésor, avec le trouble.
\par 17 Mieux vaut de l'herbe pour nourriture, là où règne l'amour, Qu'un boeuf engraissé, si la haine est là.
\par 18 Un homme violent excite des querelles, Mais celui qui est lent à la colère apaise les disputes.
\par 19 Le chemin du paresseux est comme une haie d'épines, Mais le sentier des hommes droits est aplani.
\par 20 Un fils sage fait la joie de son père, Et un homme insensé méprise sa mère.
\par 21 La folie est une joie pour celui qui est dépourvu de sens, Mais un homme intelligent va le droit chemin.
\par 22 Les projets échouent, faute d'une assemblée qui délibère; Mais ils réussissent quand il y a de nombreux conseillers.
\par 23 On éprouve de la joie à donner une réponse de sa bouche; Et combien est agréable une parole dite à propos!
\par 24 Pour le sage, le sentier de la vie mène en haut, Afin qu'il se détourne du séjour des morts qui est en bas.
\par 25 L'Éternel renverse la maison des orgueilleux, Mais il affermit les bornes de la veuve.
\par 26 Les pensées mauvaises sont en horreur à l'Éternel, Mais les paroles agréables sont pures à ses yeux.
\par 27 Celui qui est avide de gain trouble sa maison, Mais celui qui hait les présents vivra.
\par 28 Le coeur du juste médite pour répondre, Mais la bouche des méchants répand des méchancetés.
\par 29 L'Éternel s'éloigne des méchants, Mais il écoute la prière des justes.
\par 30 Ce qui plaît aux yeux réjouit le coeur; Une bonne nouvelle fortifie les membres.
\par 31 L'oreille attentive aux réprimandes qui mènent à la vie Fait son séjour au milieu des sages.
\par 32 Celui qui rejette la correction méprise son âme, Mais celui qui écoute la réprimande acquiert l'intelligence.
\par 33 La crainte de l'Éternel enseigne la sagesse, Et l'humilité précède la gloire.

\chapter{16}

\par 1 Les projets que forme le coeur dépendent de l'homme, Mais la réponse que donne la bouche vient de l'Éternel.
\par 2 Toutes les voies de l'homme sont pures à ses yeux; Mais celui qui pèse les esprits, c'est l'Éternel.
\par 3 Recommande à l'Éternel tes oeuvres, Et tes projets réussiront.
\par 4 L'Éternel a tout fait pour un but, Même le méchant pour le jour du malheur.
\par 5 Tout coeur hautain est en abomination à l'Éternel; Certes, il ne restera pas impuni.
\par 6 Par la bonté et la fidélité on expie l'iniquité, Et par la crainte de l'Éternel on se détourne du mal.
\par 7 Quand l'Éternel approuve les voies d'un homme, Il dispose favorablement à son égard même ses ennemis.
\par 8 Mieux vaut peu, avec la justice, Que de grands revenus, avec l'injustice.
\par 9 Le coeur de l'homme médite sa voie, Mais c'est l'Éternel qui dirige ses pas.
\par 10 Des oracles sont sur les lèvres du roi: Sa bouche ne doit pas être infidèle quand il juge.
\par 11 Le poids et la balance justes sont à l'Éternel; Tous les poids du sac sont son ouvrage.
\par 12 Les rois ont horreur de faire le mal, Car c'est par la justice que le trône s'affermit.
\par 13 Les lèvres justes gagnent la faveur des rois, Et ils aiment celui qui parle avec droiture.
\par 14 La fureur du roi est un messager de mort, Et un homme sage doit l'apaiser.
\par 15 La sérénité du visage du roi donne la vie, Et sa faveur est comme une pluie du printemps.
\par 16 Combien acquérir la sagesse vaut mieux que l'or! Combien acquérir l'intelligence est préférable à l'argent!
\par 17 Le chemin des hommes droits, c'est d'éviter le mal; Celui qui garde son âme veille sur sa voie.
\par 18 L'arrogance précède la ruine, Et l'orgueil précède la chute.
\par 19 Mieux vaut être humble avec les humbles Que de partager le butin avec les orgueilleux.
\par 20 Celui qui réfléchit sur les choses trouve le bonheur, Et celui qui se confie en l'Éternel est heureux.
\par 21 Celui qui est sage de coeur est appelé intelligent, Et la douceur des lèvres augmente le savoir.
\par 22 La sagesse est une source de vie pour celui qui la possède; Et le châtiment des insensés, c'est leur folie.
\par 23 Celui qui est sage de coeur manifeste la sagesse par sa bouche, Et l'accroissement de son savoir paraît sur ses lèvres.
\par 24 Les paroles agréables sont un rayon de miel, Douces pour l'âme et salutaires pour le corps.
\par 25 Telle voie paraît droite à un homme, Mais son issue, c'est la voie de la mort.
\par 26 Celui qui travaille, travaille pour lui, Car sa bouche l'y excite.
\par 27 L'homme pervers prépare le malheur, Et il y a sur ses lèvres comme un feu ardent.
\par 28 L'homme pervers excite des querelles, Et le rapporteur divise les amis.
\par 29 L'homme violent séduit son prochain, Et le fait marcher dans une voie qui n'est pas bonne.
\par 30 Celui qui ferme les yeux pour se livrer à des pensées perverses, Celui qui se mord les lèvres, a déjà consommé le mal.
\par 31 Les cheveux blancs sont une couronne d'honneur; C'est dans le chemin de la justice qu'on la trouve.
\par 32 Celui qui est lent à la colère vaut mieux qu'un héros, Et celui qui est maître de lui-même, que celui qui prend des villes.
\par 33 On jette le sort dans le pan de la robe, Mais toute décision vient de l'Éternel.

\chapter{17}

\par 1 Mieux vaut un morceau de pain sec, avec la paix, Qu'une maison pleine de viandes, avec des querelles.
\par 2 Un serviteur prudent domine sur le fils qui fait honte, Et il aura part à l'héritage au milieu des frères.
\par 3 Le creuset est pour l'argent, et le fourneau pour l'or; Mais celui qui éprouve les coeurs, c'est l'Éternel.
\par 4 Le méchant est attentif à la lèvre inique, Le menteur prête l'oreille à la langue pernicieuse.
\par 5 Celui qui se moque du pauvre outrage celui qui l'a fait; Celui qui se réjouit d'un malheur ne restera pas impuni.
\par 6 Les enfants des enfants sont la couronne des vieillards, Et les pères sont la gloire de leurs enfants.
\par 7 Les paroles distinguées ne conviennent pas à un insensé; Combien moins à un noble les paroles mensongères!
\par 8 Les présents sont une pierre précieuse aux yeux de qui en reçoit; De quelque côté qu'ils se tournes, ils ont du succès.
\par 9 Celui qui couvre une faute cherche l'amour, Et celui qui la rappelle dans ses discours divise les amis.
\par 10 Une réprimande fait plus d'impression sur l'homme intelligent Que cent coups sur l'insensé.
\par 11 Le méchant ne cherche que révolte, Mais un messager cruel sera envoyé contre lui.
\par 12 Rencontre une ourse privée de ses petits, Plutôt qu'un insensé pendant sa folie.
\par 13 De celui qui rend le mal pour le bien Le mal ne quittera point la maison.
\par 14 Commencer une querelle, c'est ouvrir une digue; Avant que la dispute s'anime, retire-toi.
\par 15 Celui qui absout le coupable et celui qui condamne le juste Sont tous deux en abomination à l'Éternel.
\par 16 A quoi sert l'argent dans la main de l'insensé? A acheter la sagesse?... Mais il n'a point de sens.
\par 17 L'ami aime en tout temps, Et dans le malheur il se montre un frère.
\par 18 L'homme dépourvu de sens prend des engagements, Il cautionne son prochain.
\par 19 Celui qui aime les querelles aime le péché; Celui qui élève sa porte cherche la ruine.
\par 20 Un coeur faux ne trouve pas le bonheur, Et celui dont la langue est perverse tombe dans le malheur.
\par 21 Celui qui donne naissance à un insensé aura du chagrin; Le père d'un fou ne peut pas se réjouir.
\par 22 Un coeur joyeux est un bon remède, Mais un esprit abattu dessèche les os.
\par 23 Le méchant accepte en secret des présents, Pour pervertir les voies de la justice.
\par 24 La sagesse est en face de l'homme intelligent, Mais les yeux de l'insensé sont à l'extrémité de la terre.
\par 25 Un fils insensé fait le chagrin de son père, Et l'amertume de celle qui l'a enfanté.
\par 26 Il n'est pas bon de condamner le juste à une amende, Ni de frapper les nobles à cause de leur droiture.
\par 27 Celui qui retient ses paroles connaît la science, Et celui qui a l'esprit calme est un homme intelligent.
\par 28 L'insensé même, quand il se tait, passe pour sage; Celui qui ferme ses lèvres est un homme intelligent.

\chapter{18}

\par 1 Celui qui se tient à l'écart cherche ce qui lui plaît, Il s'irrite contre tout ce qui est sage.
\par 2 Ce n'est pas à l'intelligence que l'insensé prend plaisir, C'est à la manifestation de ses pensées.
\par 3 Quand vient le méchant, vient aussi le mépris; Et avec la honte, vient l'opprobre.
\par 4 Les paroles de la bouche d'un homme sont des eaux profondes; La source de la sagesse est un torrent qui jaillit.
\par 5 Il n'est pas bon d'avoir égard à la personne du méchant, Pour faire tort au juste dans le jugement.
\par 6 Les lèvres de l'insensé se mêlent aux querelles, Et sa bouche provoque les coups.
\par 7 La bouche de l'insensé cause sa ruine, Et ses lèvres sont un piège pour son âme.
\par 8 Les paroles du rapporteur sont comme des friandises, Elles descendent jusqu'au fond des entrailles.
\par 9 Celui qui se relâche dans son travail Est frère de celui qui détruit.
\par 10 Le nom de l'Éternel est une tour forte; Le juste s'y réfugie, et se trouve en sûreté.
\par 11 La fortune est pour le riche une ville forte; Dans son imagination, c'est une haute muraille.
\par 12 Avant la ruine, le coeur de l'homme s'élève; Mais l'humilité précède la gloire.
\par 13 Celui qui répond avant d'avoir écouté Fait un acte de folie et s'attire la confusion.
\par 14 L'esprit de l'homme le soutient dans la maladie; Mais l'esprit abattu, qui le relèvera?
\par 15 Un coeur intelligent acquiert la science, Et l'oreille des sages cherche la science.
\par 16 Les présents d'un homme lui élargissent la voie, Et lui donnent accès auprès des grands.
\par 17 Le premier qui parle dans sa cause paraît juste; Vient sa partie adverse, et on l'examine.
\par 18 Le sort fait cesser les contestations, Et décide entre les puissants.
\par 19 Des frères sont plus intraitables qu'une ville forte, Et leurs querelles sont comme les verrous d'un palais.
\par 20 C'est du fruit de sa bouche que l'homme rassasie son corps, C'est du produit de ses lèvres qu'il se rassasie.
\par 21 La mort et la vie sont au pouvoir de la langue; Quiconque l'aime en mangera les fruits.
\par 22 Celui qui trouve une femme trouve le bonheur; C'est une grâce qu'il obtient de l'Éternel.
\par 23 Le pauvre parle en suppliant, Et le riche répond avec dureté.
\par 24 Celui qui a beaucoup d'amis les a pour son malheur, Mais il est tel ami plus attaché qu'un frère.

\chapter{19}

\par 1 Mieux vaut le pauvre qui marche dans son intégrité, Que l'homme qui a des lèvres perverses et qui est un insensé.
\par 2 Le manque de science n'est bon pour personne, Et celui qui précipite ses pas tombe dans le péché.
\par 3 La folie de l'homme pervertit sa voie, Et c'est contre l'Éternel que son coeur s'irrite.
\par 4 La richesse procure un grand nombre d'amis, Mais le pauvre est séparé de son ami.
\par 5 Le faux témoin ne restera pas impuni, Et celui qui dit des mensonges n'échappera pas.
\par 6 Beaucoup de gens flattent l'homme généreux, Et tous sont les amis de celui qui fait des présents.
\par 7 Tous les frères du pauvre le haïssent; Combien plus ses amis s'éloignent-ils de lui! Il leur adresse des paroles suppliantes, mais ils disparaissent.
\par 8 Celui qui acquiert du sens aime son âme; Celui qui garde l'intelligence trouve le bonheur.
\par 9 Le faux témoin ne restera pas impuni, Et celui qui dit des mensonges périra.
\par 10 Il ne sied pas à un insensé de vivre dans les délices; Combien moins à un esclave de dominer sur des princes!
\par 11 L'homme qui a de la sagesse est lent à la colère, Et il met sa gloire à oublier les offenses.
\par 12 La colère du roi est comme le rugissement d'un lion, Et sa faveur est comme la rosée sur l'herbe.
\par 13 Un fils insensé est une calamité pour son père, Et les querelles d'une femme sont une gouttière sans fin.
\par 14 On peut hériter de ses pères une maison et des richesses, Mais une femme intelligente est un don de l'Éternel.
\par 15 La paresse fait tomber dans l'assoupissement, Et l'âme nonchalante éprouve la faim.
\par 16 Celui qui garde ce qui est commandé garde son âme; Celui qui ne veille pas sur sa voie mourra.
\par 17 Celui qui a pitié du pauvre prête à l'Éternel, Qui lui rendra selon son oeuvre.
\par 18 Châtie ton fils, car il y a encore de l'espérance; Mais ne désire point le faire mourir.
\par 19 Celui que la colère emporte doit en subir la peine; Car si tu le libères, tu devras y revenir.
\par 20 Écoute les conseils, et reçois l'instruction, Afin que tu sois sage dans la suite de ta vie.
\par 21 Il y a dans le coeur de l'homme beaucoup de projets, Mais c'est le dessein de l'Éternel qui s'accomplit.
\par 22 Ce qui fait le charme d'un homme, c'est sa bonté; Et mieux vaut un pauvre qu'un menteur.
\par 23 La crainte de l'Éternel mène à la vie, Et l'on passe la nuit rassasié, sans être visité par le malheur.
\par 24 Le paresseux plonge sa main dans le plat, Et il ne la ramène pas à sa bouche.
\par 25 Frappe le moqueur, et le sot deviendra sage; Reprends l'homme intelligent, et il comprendra la science.
\par 26 Celui qui ruine son père et qui met en fuite sa mère Est un fils qui fait honte et qui fait rougir.
\par 27 Cesse, mon fils, d'écouter l'instruction, Si c'est pour t'éloigner des paroles de la science.
\par 28 Un témoin pervers se moque de la justice, Et la bouche des méchants dévore l'iniquité.
\par 29 Les châtiments sont prêts pour les moqueurs, Et les coups pour le dos des insensés.

\chapter{20}

\par 1 Le vin est moqueur, les boissons fortes sont tumultueuses; Quiconque en fait excès n'est pas sage.
\par 2 La terreur qu'inspire le roi est comme le rugissement d'un lion; Celui qui l'irrite pèche contre lui-même.
\par 3 C'est une gloire pour l'homme de s'abstenir des querelles, Mais tout insensé se livre à l'emportement.
\par 4 A cause du froid, le paresseux ne laboure pas; A la moisson, il voudrait récolter, mais il n'y a rien.
\par 5 Les desseins dans le coeur de l'homme sont des eaux profondes, Mais l'homme intelligent sait y puiser.
\par 6 Beaucoup de gens proclament leur bonté; Mais un homme fidèle, qui le trouvera?
\par 7 Le juste marche dans son intégrité; Heureux ses enfants après lui!
\par 8 Le roi assis sur le trône de la justice Dissipe tout mal par son regard.
\par 9 Qui dira: J'ai purifié mon coeur, Je suis net de mon péché?
\par 10 Deux sortes de poids, deux sortes d'épha, Sont l'un et l'autre en abomination à l'Éternel.
\par 11 L'enfant laisse déjà voir par ses actions Si sa conduite sera pure et droite.
\par 12 L'oreille qui entend, et l'oeil qui voit, C'est l'Éternel qui les a faits l'un et l'autre.
\par 13 N'aime pas le sommeil, de peur que tu ne deviennes pauvre; Ouvre les yeux, tu seras rassasié de pain.
\par 14 Mauvais! mauvais! dit l'acheteur; Et en s'en allant, il se félicite.
\par 15 Il y a de l'or et beaucoup de perles; Mais les lèvres savantes sont un objet précieux.
\par 16 Prends son vêtement, car il a cautionné autrui; Exige de lui des gages, à cause des étrangers.
\par 17 Le pain du mensonge est doux à l'homme, Et plus tard sa bouche est remplie de gravier.
\par 18 Les projets s'affermissent par le conseil; Fais la guerre avec prudence.
\par 19 Celui qui répand la calomnie dévoile les secrets; Ne te mêle pas avec celui qui ouvre ses lèvres.
\par 20 Si quelqu'un maudit son père et sa mère, Sa lampe s'éteindra au milieu des ténèbres.
\par 21 Un héritage promptement acquis dès l'origine Ne sera pas béni quand viendra la fin.
\par 22 Ne dis pas: Je rendrai le mal. Espère en l'Éternel, et il te délivrera.
\par 23 L'Éternel a en horreur deux sortes de poids, Et la balance fausse n'est pas une chose bonne.
\par 24 C'est l'Éternel qui dirige les pas de l'homme, Mais l'homme peut-il comprendre sa voie?
\par 25 C'est un piège pour l'homme que de prendre à la légère un engagement sacré, Et de ne réfléchir qu'après avoir fait un voeu.
\par 26 Un roi sage dissipe les méchants, Et fait passer sur eux la roue.
\par 27 Le souffle de l'homme est une lampe de l'Éternel; Il pénètre jusqu'au fond des entrailles.
\par 28 La bonté et la fidélité gardent le roi, Et il soutient son trône par la bonté.
\par 29 La force est la gloire des jeunes gens, Et les cheveux blancs sont l'ornement des vieillards.
\par 30 Les plaies d'une blessure sont un remède pour le méchant; De même les coups qui pénètrent jusqu'au fond des entrailles.

\chapter{21}

\par 1 Le coeur du roi est un courant d'eau dans la main de l'Éternel; Il l'incline partout où il veut.
\par 2 Toutes les voies de l'homme sont droites à ses yeux; Mais celui qui pèse les coeurs, c'est l'Éternel.
\par 3 La pratique de la justice et de l'équité, Voilà ce que l'Éternel préfère aux sacrifices.
\par 4 Des regards hautains et un coeur qui s'enfle, Cette lampe des méchants, ce n'est que péché.
\par 5 Les projets de l'homme diligent ne mènent qu'à l'abondance, Mais celui qui agit avec précipitation n'arrive qu'à la disette.
\par 6 Des trésors acquis par une langue mensongère Sont une vanité fugitive et l'avant-coureur de la mort.
\par 7 La violence des méchants les emporte, Parce qu'ils refusent de faire ce qui est juste.
\par 8 Le coupable suit des voies détournées, Mais l'innocent agit avec droiture.
\par 9 Mieux vaut habiter à l'angle d'un toit, Que de partager la demeure d'une femme querelleuse.
\par 10 L'âme du méchant désire le mal; Son ami ne trouve pas grâce à ses yeux.
\par 11 Quand on châtie le moqueur, le sot devient sage; Et quand on instruit le sage, il accueille la science.
\par 12 Le juste considère la maison du méchant; L'Éternel précipite les méchants dans le malheur.
\par 13 Celui qui ferme son oreille au cri du pauvre Criera lui-même et n'aura point de réponse.
\par 14 Un don fait en secret apaise la colère, Et un présent fait en cachette calme une fureur violente.
\par 15 C'est une joie pour le juste de pratiquer la justice, Mais la ruine est pour ceux qui font le mal.
\par 16 L'homme qui s'écarte du chemin de la sagesse Reposera dans l'assemblée des morts.
\par 17 Celui qui aime la joie reste dans l'indigence; Celui qui aime le vin et l'huile ne s'enrichit pas.
\par 18 Le méchant sert de rançon pour le juste, Et le perfide pour les hommes droits.
\par 19 Mieux vaut habiter dans une terre déserte, Qu'avec une femme querelleuse et irritable.
\par 20 De précieux trésors et de l'huile sont dans la demeure du sage; Mais l'homme insensé les engloutit.
\par 21 Celui qui poursuit la justice et la bonté Trouve la vie, la justice et la gloire.
\par 22 Le sage monte dans la ville des héros, Et il abat la force qui lui donnait de l'assurance.
\par 23 Celui qui veille sur sa bouche et sur sa langue Préserve son âme des angoisses.
\par 24 L'orgueilleux, le hautain, s'appelle un moqueur; Il agit avec la fureur de l'arrogance.
\par 25 Les désirs du paresseux le tuent, Parce que ses mains refusent de travailler;
\par 26 Tout le jour il éprouve des désirs; Mais le juste donne sans parcimonie.
\par 27 Le sacrifice des méchants est quelque chose d'abominable; Combien plus quand ils l'offrent avec des pensées criminelles!
\par 28 Le témoin menteur périra, Mais l'homme qui écoute parlera toujours.
\par 29 Le méchant prend un air effronté, Mais l'homme droit affermit sa voie.
\par 30 Il n'y a ni sagesse, ni intelligence, Ni conseil, en face de l'Éternel.
\par 31 Le cheval est équipé pour le jour de la bataille, Mais la délivrance appartient à l'Éternel.

\chapter{22}

\par 1 La réputation est préférable à de grandes richesses, Et la grâce vaut mieux que l'argent et que l'or.
\par 2 Le riche et le pauvre se rencontrent; C'est l'Éternel qui les a faits l'un et l'autre.
\par 3 L'homme prudent voit le mal et se cache, Mais les simples avancent et sont punis.
\par 4 Le fruit de l'humilité, de la crainte de l'Éternel, C'est la richesse, la gloire et la vie.
\par 5 Des épines, des pièges sont sur la voie de l'homme pervers; Celui qui garde son âme s'en éloigne.
\par 6 Instruis l'enfant selon la voie qu'il doit suivre; Et quand il sera vieux, il ne s'en détournera pas.
\par 7 Le riche domine sur les pauvres, Et celui qui emprunte est l'esclave de celui qui prête.
\par 8 Celui qui sème l'iniquité moissonne l'iniquité, Et la verge de sa fureur disparaît.
\par 9 L'homme dont le regard est bienveillant sera béni, Parce qu'il donne de son pain au pauvre.
\par 10 Chasse le moqueur, et la querelle prendra fin; Les disputes et les outrages cesseront.
\par 11 Celui qui aime la pureté du coeur, Et qui a la grâce sur les lèvres, a le roi pour ami.
\par 12 Les yeux de l'Éternel gardent la science, Mais il confond les paroles du perfide.
\par 13 Le paresseux dit: Il y a un lion dehors! Je serai tué dans les rues!
\par 14 La bouche des étrangères est une fosse profonde; Celui contre qui l'Éternel est irrité y tombera.
\par 15 La folie est attachée au coeur de l'enfant; La verge de la correction l'éloignera de lui.
\par 16 Opprimer le pauvre pour augmenter son bien, C'est donner au riche pour n'arriver qu'à la disette.
\par 17 Prête l'oreille, et écoute les paroles des sages; Applique ton coeur à ma science.
\par 18 Car il est bon que tu les gardes au dedans de toi, Et qu'elles soient toutes présentes sur tes lèvres.
\par 19 Afin que ta confiance repose sur l'Éternel, Je veux t'instruire aujourd'hui, oui, toi.
\par 20 N'ai-je pas déjà pour toi mis par écrit Des conseils et des réflexions,
\par 21 Pour t'enseigner des choses sûres, des paroles vraies, Afin que tu répondes par des paroles vraies à celui qui t'envoie?
\par 22 Ne dépouille pas le pauvre, parce qu'il est pauvre, Et n'opprime pas le malheureux à la porte;
\par 23 Car l'Éternel défendra leur cause, Et il ôtera la vie à ceux qui les auront dépouillés.
\par 24 Ne fréquente pas l'homme colère, Ne va pas avec l'homme violent,
\par 25 De peur que tu ne t'habitues à ses sentiers, Et qu'ils ne deviennent un piège pour ton âme.
\par 26 Ne sois pas parmi ceux qui prennent des engagements, Parmi ceux qui cautionnent pour des dettes;
\par 27 Si tu n'as pas de quoi payer, Pourquoi voudrais-tu qu'on enlève ton lit de dessous toi?
\par 28 Ne déplace pas la borne ancienne, Que tes pères ont posée.
\par 29 Si tu vois un homme habile dans son ouvrage, Il se tient auprès des rois; Il ne se tient pas auprès des gens obscurs.

\chapter{23}

\par 1 Si tu es à table avec un grand, Fais attention à ce qui est devant toi;
\par 2 Mets un couteau à ta gorge, Si tu as trop d'avidité.
\par 3 Ne convoite pas ses friandises: C'est un aliment trompeur.
\par 4 Ne te tourmente pas pour t'enrichir, N'y applique pas ton intelligence.
\par 5 Veux-tu poursuivre du regard ce qui va disparaître? Car la richesse se fait des ailes, Et comme l'aigle, elle prend son vol vers les cieux.
\par 6 Ne mange pas le pain de celui dont le regard est malveillant, Et ne convoite pas ses friandises;
\par 7 Car il est comme les pensées de son âme. Mange et bois, te dira-t-il; Mais son coeur n'est point avec toi.
\par 8 Tu vomiras le morceau que tu as mangé, Et tu auras perdu tes propos agréables.
\par 9 Ne parle pas aux oreilles de l'insensé, Car il méprise la sagesse de tes discours.
\par 10 Ne déplace pas la borne ancienne, Et n'entre pas dans le champ des orphelins;
\par 11 Car leur vengeur est puissant: Il défendra leur cause contre toi.
\par 12 Ouvre ton coeur à l'instruction, Et tes oreilles aux paroles de la science.
\par 13 N'épargne pas la correction à l'enfant; Si tu le frappes de la verge, il ne mourra point.
\par 14 En le frappant de la verge, Tu délivres son âme du séjour des morts.
\par 15 Mon fils, si ton coeur est sage, Mon coeur à moi sera dans la joie;
\par 16 Mes entrailles seront émues d'allégresse, Quand tes lèvres diront ce qui est droit.
\par 17 Que ton coeur n'envie point les pécheurs, Mais qu'il ait toujours la crainte de l'Éternel;
\par 18 Car il est un avenir, Et ton espérance ne sera pas anéantie.
\par 19 Écoute, mon fils, et sois sage; Dirige ton coeur dans la voie droite.
\par 20 Ne sois pas parmi les buveurs de vin, Parmi ceux qui font excès des viandes:
\par 21 Car l'ivrogne et celui qui se livre à des excès s'appauvrissent, Et l'assoupissement fait porter des haillons.
\par 22 Écoute ton père, lui qui t'a engendré, Et ne méprise pas ta mère, quand elle est devenue vieille.
\par 23 Acquiers la vérité, et ne la vends pas, La sagesse, l'instruction et l'intelligence.
\par 24 Le père du juste est dans l'allégresse, Celui qui donne naissance à un sage aura de la joie.
\par 25 Que ton père et ta mère se réjouissent, Que celle qui t'a enfanté soit dans l'allégresse!
\par 26 Mon fils, donne-moi ton coeur, Et que tes yeux se plaisent dans mes voies.
\par 27 Car la prostituée est une fosse profonde, Et l'étrangère un puits étroit.
\par 28 Elle dresse des embûches comme un brigand, Et elle augmente parmi les hommes le nombre des perfides.
\par 29 Pour qui les ah? pour qui les hélas? Pour qui les disputes? pour qui les plaintes? Pour qui les blessures sans raison? pour qui les yeux rouges?
\par 30 Pour ceux qui s'attardent auprès du vin, Pour ceux qui vont déguster du vin mêlé.
\par 31 Ne regarde pas le vin qui paraît d'un beau rouge, Qui fait des perles dans la coupe, Et qui coule aisément.
\par 32 Il finit par mordre comme un serpent, Et par piquer comme un basilic.
\par 33 Tes yeux se porteront sur des étrangères, Et ton coeur parlera d'une manière perverse.
\par 34 Tu seras comme un homme couché au milieu de la mer, Comme un homme couché sur le sommet d'un mât:
\par 35 On m'a frappé,... je n'ai point de mal!... On m'a battu,... je ne sens rien!... Quand me réveillerai-je?... J'en veux encore!

\chapter{24}

\par 1 Ne porte pas envie aux hommes méchants, Et ne désire pas être avec eux;
\par 2 Car leur coeur médite la ruine, Et leurs lèvres parlent d'iniquité.
\par 3 C'est par la sagesse qu'une maison s'élève, Et par l'intelligence qu'elle s'affermit;
\par 4 C'est par la science que les chambres se remplissent De tous les biens précieux et agréables.
\par 5 Un homme sage est plein de force, Et celui qui a de la science affermit sa vigueur;
\par 6 Car tu feras la guerre avec prudence, Et le salut est dans le grand nombre des conseillers.
\par 7 La sagesse est trop élevée pour l'insensé; Il n'ouvrira pas la bouche à la porte.
\par 8 Celui qui médite de faire le mal S'appelle un homme plein de malice.
\par 9 La pensée de la folie n'est que péché, Et le moqueur est en abomination parmi les hommes.
\par 10 Si tu faiblis au jour de la détresse, Ta force n'est que détresse.
\par 11 Délivre ceux qu'on traîne à la mort, Ceux qu'on va égorger, sauve-les!
\par 12 Si tu dis: Ah! nous ne savions pas!... Celui qui pèse les coeurs ne le voit-il pas? Celui qui veille sur ton âme ne le connaît-il pas? Et ne rendra-t-il pas à chacun selon ses oeuvres?
\par 13 Mon fils, mange du miel, car il est bon; Un rayon de miel sera doux à ton palais.
\par 14 De même, connais la sagesse pour ton âme; Si tu la trouves, il est un avenir, Et ton espérance ne sera pas anéantie.
\par 15 Ne tends pas méchamment des embûches à la demeure du juste, Et ne dévaste pas le lieu où il repose;
\par 16 Car sept fois le juste tombe, et il se relève, Mais les méchants sont précipités dans le malheur.
\par 17 Ne te réjouis pas de la chute de ton ennemi, Et que ton coeur ne soit pas dans l'allégresse quand il chancelle,
\par 18 De peur que l'Éternel ne le voie, que cela ne lui déplaise, Et qu'il ne détourne de lui sa colère.
\par 19 Ne t'irrite pas à cause de ceux qui font le mal, Ne porte pas envie aux méchants;
\par 20 Car il n'y a point d'avenir pour celui qui fait le mal, La lampe des méchants s'éteint.
\par 21 Mon fils, crains l'Éternel et le roi; Ne te mêle pas avec les hommes remuants;
\par 22 Car soudain leur ruine surgira, Et qui connaît les châtiments des uns et des autres?
\par 23 Voici encore ce qui vient des sages: Il n'est pas bon, dans les jugements, d'avoir égard aux personnes.
\par 24 Celui qui dit au méchant: Tu es juste! Les peuples le maudissent, les nations le maudissent.
\par 25 Mais ceux qui le châtient s'en trouvent bien, Et le bonheur vient sur eux comme une bénédiction.
\par 26 Il baise les lèvres, Celui qui répond des paroles justes.
\par 27 Soigne tes affaires au dehors, Mets ton champ en état, Puis tu bâtiras ta maison.
\par 28 Ne témoigne pas à la légère contre ton prochain; Voudrais-tu tromper par tes lèvres?
\par 29 Ne dis pas: Je lui ferai comme il m'a fait, Je rendrai à chacun selon ses oeuvres.
\par 30 J'ai passé près du champ d'un paresseux, Et près de la vigne d'un homme dépourvu de sens.
\par 31 Et voici, les épines y croissaient partout, Les ronces en couvraient la face, Et le mur de pierres était écroulé.
\par 32 J'ai regardé attentivement, Et j'ai tiré instruction de ce que j'ai vu.
\par 33 Un peu de sommeil, un peu d'assoupissement, Un peu croiser les mains pour dormir!...
\par 34 Et la pauvreté te surprendra, comme un rôdeur, Et la disette, comme un homme en armes.

\chapter{25}

\par 1 Voici encore des Proverbes de Salomon, recueillis par les gens d'Ézéchias, roi de Juda.
\par 2 La gloire de Dieu, c'est de cacher les choses; La gloire des rois, c'est de sonder les choses.
\par 3 Les cieux dans leur hauteur, la terre dans sa profondeur, Et le coeur des rois, sont impénétrables.
\par 4 Ote de l'argent les scories, Et il en sortira un vase pour le fondeur.
\par 5 Ote le méchant de devant le roi, Et son trône s'affermira par la justice.
\par 6 Ne t'élève pas devant le roi, Et ne prends pas la place des grands;
\par 7 Car il vaut mieux qu'on te dise: Monte-ici! Que si l'on t'abaisse devant le prince que tes yeux voient.
\par 8 Ne te hâte pas d'entrer en contestation, De peur qu'à la fin tu ne saches que faire, Lorsque ton prochain t'aura outragé.
\par 9 Défends ta cause contre ton prochain, Mais ne révèle pas le secret d'un autre,
\par 10 De peur qu'en l'apprenant il ne te couvre de honte, Et que ta mauvaise renommée ne s'efface pas.
\par 11 Comme des pommes d'or sur des ciselures d'argent, Ainsi est une parole dite à propos.
\par 12 Comme un anneau d'or et une parure d'or fin, Ainsi pour une oreille docile est le sage qui réprimande.
\par 13 Comme la fraîcheur de la neige au temps de la moisson, Ainsi est un messager fidèle pour celui qui l'envoie; Il restaure l'âme de son maître.
\par 14 Comme des nuages et du vent sans pluie, Ainsi est un homme se glorifiant à tort de ses libéralités.
\par 15 Par la lenteur à la colère on fléchit un prince, Et une langue douce peut briser des os.
\par 16 Si tu trouves du miel, n'en mange que ce qui te suffit, De peur que tu n'en sois rassasié et que tu ne le vomisses.
\par 17 Mets rarement le pied dans la maison de ton prochain, De peur qu'il ne soit rassasié de toi et qu'il ne te haïsse.
\par 18 Comme une massue, une épée et une flèche aiguë, Ainsi est un homme qui porte un faux témoignage contre son prochain.
\par 19 Comme une dent cassée et un pied qui chancelle, Ainsi est la confiance en un perfide au jour de la détresse.
\par 20 Oter son vêtement dans un jour froid, Répandre du vinaigre sur du nitre, C'est dire des chansons à un coeur attristé.
\par 21 Si ton ennemi a faim, donne-lui du pain à manger; S'il a soif, donne-lui de l'eau à boire.
\par 22 Car ce sont des charbons ardents que tu amasses sur sa tête, Et l'Éternel te récompensera.
\par 23 Le vent du nord enfante la pluie, Et la langue mystérieuse un visage irrité.
\par 24 Mieux vaut habiter à l'angle d'un toit, Que de partager la demeure d'une femme querelleuse.
\par 25 Comme de l'eau fraîche pour une personne fatiguée, Ainsi est une bonne nouvelle venant d'une terre lointaine.
\par 26 Comme une fontaine troublée et une source corrompue, Ainsi est le juste qui chancelle devant le méchant.
\par 27 Il n'est pas bon de manger beaucoup de miel, Mais rechercher la gloire des autres est un honneur.
\par 28 Comme une ville forcée et sans murailles, Ainsi est l'homme qui n'est pas maître de lui-même.

\chapter{26}

\par 1 Comme la neige en été, et la pluie pendant la moisson, Ainsi la gloire ne convient pas à un insensé.
\par 2 Comme l'oiseau s'échappe, comme l'hirondelle s'envole, Ainsi la malédiction sans cause n'a point d'effet.
\par 3 Le fouet est pour le cheval, le mors pour l'âne, Et la verge pour le dos des insensés.
\par 4 Ne réponds pas à l'insensé selon sa folie, De peur que tu ne lui ressembles toi-même.
\par 5 Réponds à l'insensé selon sa folie, Afin qu'il ne se regarde pas comme sage.
\par 6 Il se coupe les pieds, il boit l'injustice, Celui qui donne des messages à un insensé.
\par 7 Comme les jambes du boiteux sont faibles, Ainsi est une sentence dans la bouche des insensés.
\par 8 C'est attacher une pierre à la fronde, Que d'accorder des honneurs à un insensé.
\par 9 Comme une épine qui se dresse dans la main d'un homme ivre, Ainsi est une sentence dans la bouche des insensés.
\par 10 Comme un archer qui blesse tout le monde, Ainsi est celui qui prend à gage les insensés et les premiers venus.
\par 11 Comme un chien qui retourne à ce qu'il a vomi, Ainsi est un insensé qui revient à sa folie.
\par 12 Si tu vois un homme qui se croit sage, Il y a plus à espérer d'un insensé que de lui.
\par 13 Le paresseux dit: Il y a un lion sur le chemin, Il y a un lion dans les rues!
\par 14 La porte tourne sur ses gonds, Et le paresseux sur son lit.
\par 15 Le paresseux plonge sa main dans le plat, Et il trouve pénible de la ramener à sa bouche.
\par 16 Le paresseux se croit plus sage Que sept hommes qui répondent avec bon sens.
\par 17 Comme celui qui saisit un chien par les oreilles, Ainsi est un passant qui s'irrite pour une querelle où il n'a que faire.
\par 18 Comme un furieux qui lance des flammes, Des flèches et la mort,
\par 19 Ainsi est un homme qui trompe son prochain, Et qui dit: N'était-ce pas pour plaisanter?
\par 20 Faute de bois, le feu s'éteint; Et quand il n'y a point de rapporteur, la querelle s'apaise.
\par 21 Le charbon produit un brasier, et le bois du feu; Ainsi un homme querelleur échauffe une dispute.
\par 22 Les paroles du rapporteur sont comme des friandises, Elles descendent jusqu'au fond des entrailles.
\par 23 Comme des scories d'argent appliquées sur un vase de terre, Ainsi sont des lèvres brûlantes et un coeur mauvais.
\par 24 Par ses lèvres celui qui hait se déguise, Et il met au dedans de lui la tromperie.
\par 25 Lorsqu'il prend une voix douce, ne le crois pas, Car il y a sept abominations dans son coeur.
\par 26 S'il cache sa haine sous la dissimulation, Sa méchanceté se révélera dans l'assemblée.
\par 27 Celui qui creuse une fosse y tombe, Et la pierre revient sur celui qui la roule.
\par 28 La langue fausse hait ceux qu'elle écrase, Et la bouche flatteuse prépare la ruine.

\chapter{27}

\par 1 Ne te vante pas du lendemain, Car tu ne sais pas ce qu'un jour peut enfanter.
\par 2 Qu'un autre te loue, et non ta bouche, Un étranger, et non tes lèvres.
\par 3 La pierre est pesante et le sable est lourd, Mais l'humeur de l'insensé pèse plus que l'un et l'autre.
\par 4 La fureur est cruelle et la colère impétueuse, Mais qui résistera devant la jalousie?
\par 5 Mieux vaut une réprimande ouverte Qu'une amitié cachée.
\par 6 Les blessures d'un ami prouvent sa fidélité, Mais les baisers d'un ennemi sont trompeurs.
\par 7 Celui qui est rassasié foule aux pieds le rayon de miel, Mais celui qui a faim trouve doux tout ce qui est amer.
\par 8 Comme l'oiseau qui erre loin de son nid, Ainsi est l'homme qui erre loin de son lieu.
\par 9 L'huile et les parfums réjouissent le coeur, Et les conseils affectueux d'un ami sont doux.
\par 10 N'abandonne pas ton ami et l'ami de ton père, Et n'entre pas dans la maison de ton frère au jour de ta détresse; Mieux vaut un voisin proche qu'un frère éloigné.
\par 11 Mon fils, sois sage, et réjouis mon coeur, Et je pourrai répondre à celui qui m'outrage.
\par 12 L'homme prudent voit le mal et se cache; Les simples avancent et sont punis.
\par 13 Prends son vêtement, car il a cautionné autrui; Exige de lui des gages, à cause des étrangers.
\par 14 Si l'on bénit son prochain à haute voix et de grand matin, Cela est envisagé comme une malédiction.
\par 15 Une gouttière continue dans un jour de pluie Et une femme querelleuse sont choses semblables.
\par 16 Celui qui la retient retient le vent, Et sa main saisit de l'huile.
\par 17 Comme le fer aiguise le fer, Ainsi un homme excite la colère d'un homme.
\par 18 Celui qui soigne un figuier en mangera le fruit, Et celui qui garde son maître sera honoré.
\par 19 Comme dans l'eau le visage répond au visage, Ainsi le coeur de l'homme répond au coeur de l'homme.
\par 20 Le séjour des morts et l'abîme sont insatiables; De même les yeux de l'homme sont insatiables.
\par 21 Le creuset est pour l'argent, et le fourneau pour l'or; Mais un homme est jugé d'après sa renommée.
\par 22 Quand tu pilerais l'insensé dans un mortier, Au milieu des grains avec le pilon, Sa folie ne se séparerait pas de lui.
\par 23 Connais bien chacune de tes brebis, Donne tes soins à tes troupeaux;
\par 24 Car la richesse ne dure pas toujours, Ni une couronne éternellement.
\par 25 Le foin s'enlève, la verdure paraît, Et les herbes des montagnes sont recueillies.
\par 26 Les agneaux sont pour te vêtir, Et les boucs pour payer le champ;
\par 27 Le lait des chèvres suffit à ta nourriture, à celle de ta maison, Et à l'entretien de tes servantes.

\chapter{28}

\par 1 Le méchant prend la fuite sans qu'on le poursuive, Le juste a de l'assurance comme un jeune lion.
\par 2 Quand un pays est en révolte, les chefs sont nombreux; Mais avec un homme qui a de l'intelligence et de la science, Le règne se prolonge.
\par 3 Un homme pauvre qui opprime les misérables Est une pluie violente qui fait manquer le pain.
\par 4 Ceux qui abandonnent la loi louent le méchant, Mais ceux qui observent la loi s'irritent contre lui.
\par 5 Les hommes livrés au mal ne comprennent pas ce qui est juste, Mais ceux qui cherchent l'Éternel comprennent tout.
\par 6 Mieux vaut le pauvre qui marche dans son intégrité, Que celui qui a des voies tortueuses et qui est riche.
\par 7 Celui qui observe la loi est un fils intelligent, Mais celui qui fréquente les débauchés fait honte à son père.
\par 8 Celui qui augmente ses biens par l'intérêt et l'usure Les amasse pour celui qui a pitié des pauvres.
\par 9 Si quelqu'un détourne l'oreille pour ne pas écouter la loi, Sa prière même est une abomination.
\par 10 Celui qui égare les hommes droits dans la mauvaise voie Tombe dans la fosse qu'il a creusée; Mais les hommes intègres héritent le bonheur.
\par 11 L'homme riche se croit sage; Mais le pauvre qui est intelligent le sonde.
\par 12 Quand les justes triomphent, c'est une grande gloire; Quand les méchants s'élèvent, chacun se cache.
\par 13 Celui qui cache ses transgressions ne prospère point, Mais celui qui les avoue et les délaisse obtient miséricorde.
\par 14 Heureux l'homme qui est continuellement dans la crainte! Mais celui qui endurcit son coeur tombe dans le malheur.
\par 15 Comme un lion rugissant et un ours affamé, Ainsi est le méchant qui domine sur un peuple pauvre.
\par 16 Un prince sans intelligence multiplie les actes d'oppression, Mais celui qui est ennemi de la cupidité prolonge ses jours.
\par 17 Un homme chargé du sang d'un autre Fuit jusqu'à la fosse: qu'on ne l'arrête pas!
\par 18 Celui qui marche dans l'intégrité trouve le salut, Mais celui qui suit deux voies tortueuses tombe dans l'une d'elles.
\par 19 Celui qui cultive son champ est rassasié de pain, Mais celui qui poursuit des choses vaines est rassasié de pauvreté.
\par 20 Un homme fidèle est comblé de bénédictions, Mais celui qui a hâte de s'enrichir ne reste pas impuni.
\par 21 Il n'est pas bon d'avoir égard aux personnes, Et pour un morceau de pain un homme se livre au péché.
\par 22 Un homme envieux a hâte de s'enrichir, Et il ne sait pas que la disette viendra sur lui.
\par 23 Celui qui reprend les autres trouve ensuite plus de faveur Que celui dont la langue est flatteuse.
\par 24 Celui qui vole son père et sa mère, Et qui dit: Ce n'est pas un péché! Est le compagnon du destructeur.
\par 25 L'orgueilleux excite les querelles, Mais celui qui se confie en l'Éternel est rassasié.
\par 26 Celui qui a confiance dans son propre coeur est un insensé, Mais celui qui marche dans la sagesse sera sauvé.
\par 27 Celui qui donne au pauvre n'éprouve pas la disette, Mais celui qui ferme les yeux est chargé de malédictions.
\par 28 Quand les méchants s'élèvent, chacun se cache; Et quand ils périssent, les justes se multiplient.

\chapter{29}

\par 1 Un homme qui mérite d'être repris, et qui raidit le cou, Sera brisé subitement et sans remède.
\par 2 Quand les justes se multiplient, le peuple est dans la joie; Quand le méchant domine, le peuple gémit.
\par 3 Un homme qui aime la sagesse réjouit son père, Mais celui qui fréquente des prostituées dissipe son bien.
\par 4 Un roi affermit le pays par la justice, Mais celui qui reçoit des présents le ruine.
\par 5 Un homme qui flatte son prochain Tend un filet sous ses pas.
\par 6 Il y a un piège dans le péché de l'homme méchant, Mais le juste triomphe et se réjouit.
\par 7 Le juste connaît la cause des pauvres, Mais le méchant ne comprend pas la science.
\par 8 Les moqueurs soufflent le feu dans la ville, Mais les sages calment la colère.
\par 9 Si un homme sage conteste avec un insensé, Il aura beau se fâcher ou rire, la paix n'aura pas lieu.
\par 10 Les hommes de sang haïssent l'homme intègre, Mais les hommes droits protègent sa vie.
\par 11 L'insensé met en dehors toute sa passion, Mais le sage la contient.
\par 12 Quand celui qui domine a égard aux paroles mensongères, Tous ses serviteurs sont des méchants.
\par 13 Le pauvre et l'oppresseur se rencontrent; C'est l'Éternel qui éclaire les yeux de l'un et de l'autre.
\par 14 Un roi qui juge fidèlement les pauvres Aura son trône affermi pour toujours.
\par 15 La verge et la correction donnent la sagesse, Mais l'enfant livré à lui-même fait honte à sa mère.
\par 16 Quand les méchants se multiplient, le péché s'accroît; Mais les justes contempleront leur chute.
\par 17 Châtie ton fils, et il te donnera du repos, Et il procurera des délices à ton âme.
\par 18 Quand il n'y a pas de révélation, le peuple est sans frein; Heureux s'il observe la loi!
\par 19 Ce n'est pas par des paroles qu'on châtie un esclave; Même s'il comprend, il n'obéit pas.
\par 20 Si tu vois un homme irréfléchi dans ses paroles, Il y a plus à espérer d'un insensé que de lui.
\par 21 Le serviteur qu'on traite mollement dès l'enfance Finit par se croire un fils.
\par 22 Un homme colère excite des querelles, Et un furieux commet beaucoup de péchés.
\par 23 L'orgueil d'un homme l'abaisse, Mais celui qui est humble d'esprit obtient la gloire.
\par 24 Celui qui partage avec un voleur est ennemi de son âme; Il entend la malédiction, et il ne déclare rien.
\par 25 La crainte des hommes tend un piège, Mais celui qui se confie en l'Éternel est protégé.
\par 26 Beaucoup de gens recherchent la faveur de celui qui domine, Mais c'est l'Éternel qui fait droit à chacun.
\par 27 L'homme inique est en abomination aux justes, Et celui dont la voie est droite est en abomination aux méchants.

\chapter{30}

\par 1 Paroles d'Agur, fils de Jaké. Sentences prononcées par cet homme pour Ithiel, pour Ithiel et pour Ucal.
\par 2 Certes, je suis plus stupide que personne, Et je n'ai pas l'intelligence d'un homme;
\par 3 Je n'ai pas appris la sagesse, Et je ne connais pas la science des saints.
\par 4 Qui est monté aux cieux, et qui en est descendu? Qui a recueilli le vent dans ses mains? Qui a serré les eaux dans son vêtement? Qui a fait paraître les extrémités de la terre? Quel est son nom, et quel est le nom de son fils? Le sais-tu?
\par 5 Toute parole de Dieu est éprouvée. Il est un bouclier pour ceux qui cherchent en lui un refuge.
\par 6 N'ajoute rien à ses paroles, De peur qu'il ne te reprenne et que tu ne sois trouvé menteur.
\par 7 Je te demande deux choses: Ne me les refuse pas, avant que je meure!
\par 8 Éloigne de moi la fausseté et la parole mensongère; Ne me donne ni pauvreté, ni richesse, Accorde-moi le pain qui m'est nécessaire.
\par 9 De peur que, dans l'abondance, je ne te renie Et ne dise: Qui est l'Éternel? Ou que, dans la pauvreté, je ne dérobe, Et ne m'attaque au nom de mon Dieu.
\par 10 Ne calomnie pas un serviteur auprès de son maître, De peur qu'il ne te maudisse et que tu ne te rendes coupable.
\par 11 Il est une race qui maudit son père, Et qui ne bénit point sa mère.
\par 12 Il est une race qui se croit pure, Et qui n'est pas lavée de sa souillure.
\par 13 Il est une race dont les yeux sont hautains, Et les paupières élevées.
\par 14 Il est une race dont les dents sont des glaives Et les mâchoires des couteaux, Pour dévorer le malheureux sur la terre Et les indigents parmi les hommes.
\par 15 La sangsue a deux filles: Donne! donne! Trois choses sont insatiables, Quatre ne disent jamais: Assez!
\par 16 Le séjour des morts, la femme stérile, La terre, qui n'est pas rassasiée d'eau, Et le feu, qui ne dit jamais: Assez!
\par 17 L'oeil qui se moque d'un père Et qui dédaigne l'obéissance envers une mère, Les corbeaux du torrent le perceront, Et les petits de l'aigle le mangeront.
\par 18 Il y a trois choses qui sont au-dessus de ma portée, Même quatre que je ne puis comprendre:
\par 19 La trace de l'aigle dans les cieux, La trace du serpent sur le rocher, La trace du navire au milieu de la mer, Et la trace de l'homme chez la jeune femme.
\par 20 Telle est la voie de la femme adultère: Elle mange, et s'essuie la bouche, Puis elle dit: Je n'ai point fait de mal.
\par 21 Trois choses font trembler la terre, Et il en est quatre qu'elle ne peut supporter:
\par 22 Un esclave qui vient à régner, Un insensé qui est rassasié de pain,
\par 23 Une femme dédaignée qui se marie, Et une servante qui hérite de sa maîtresse.
\par 24 Il y a sur la terre quatre animaux petits, Et cependant des plus sages;
\par 25 Les fourmis, peuple sans force, Préparent en été leur nourriture;
\par 26 Les damans, peuple sans puissance, Placent leur demeure dans les rochers;
\par 27 Les sauterelles n'ont point de roi, Et elles sortent toutes par divisions;
\par 28 Le lézard saisit avec les mains, Et se trouve dans les palais des rois.
\par 29 Il y en a trois qui ont une belle allure, Et quatre qui ont une belle démarche:
\par 30 Le lion, le héros des animaux, Ne reculant devant qui que ce soit;
\par 31 Le cheval tout équipé; ou le bouc; Et le roi à qui personne ne résiste.
\par 32 Si l'orgueil te pousse à des actes de folie, Et si tu as de mauvaises pensées, mets la main sur la bouche:
\par 33 Car la pression du lait produit de la crème, La pression du nez produit du sang, Et la pression de la colère produit des querelles.

\chapter{31}

\par 1 Paroles du roi Lemuel. Sentences par lesquelles sa mère l'instruisit.
\par 2 Que te dirai-je, mon fils? que te dirai-je, fils de mes entrailles? Que te dirai-je, mon fils, objet de mes voeux?
\par 3 Ne livre pas ta vigueur aux femmes, Et tes voies à celles qui perdent les rois.
\par 4 Ce n'est point aux rois, Lemuel, Ce n'est point aux rois de boire du vin, Ni aux princes de rechercher des liqueurs fortes,
\par 5 De peur qu'en buvant ils n'oublient la loi, Et ne méconnaissent les droits de tous les malheureux.
\par 6 Donnez des liqueurs fortes à celui qui périt, Et du vin à celui qui a l'amertume dans l'âme;
\par 7 Qu'il boive et oublie sa pauvreté, Et qu'il ne se souvienne plus de ses peines.
\par 8 Ouvre ta bouche pour le muet, Pour la cause de tous les délaissés.
\par 9 Ouvre ta bouche, juge avec justice, Et défends le malheureux et l'indigent.
\par 10 Qui peut trouver une femme vertueuse? Elle a bien plus de valeur que les perles.
\par 11 Le coeur de son mari a confiance en elle, Et les produits ne lui feront pas défaut.
\par 12 Elle lui fait du bien, et non du mal, Tous les jours de sa vie.
\par 13 Elle se procure de la laine et du lin, Et travaille d'une main joyeuse.
\par 14 Elle est comme un navire marchand, Elle amène son pain de loin.
\par 15 Elle se lève lorsqu'il est encore nuit, Et elle donne la nourriture à sa maison Et la tâche à ses servantes.
\par 16 Elle pense à un champ, et elle l'acquiert; Du fruit de son travail elle plante une vigne.
\par 17 Elle ceint de force ses reins, Et elle affermit ses bras.
\par 18 Elle sent que ce qu'elle gagne est bon; Sa lampe ne s'éteint point pendant la nuit.
\par 19 Elle met la main à la quenouille, Et ses doigts tiennent le fuseau.
\par 20 Elle tend la main au malheureux, Elle tend la main à l'indigent.
\par 21 Elle ne craint pas la neige pour sa maison, Car toute sa maison est vêtue de cramoisi.
\par 22 Elle se fait des couvertures, Elle a des vêtements de fin lin et de pourpre.
\par 23 Son mari est considéré aux portes, Lorsqu'il siège avec les anciens du pays.
\par 24 Elle fait des chemises, et les vend, Et elle livre des ceintures au marchand.
\par 25 Elle est revêtue de force et de gloire, Et elle se rit de l'avenir.
\par 26 Elle ouvre la bouche avec sagesse, Et des instructions aimables sont sur sa langue.
\par 27 Elle veille sur ce qui se passe dans sa maison, Et elle ne mange pas le pain de paresse.
\par 28 Ses fils se lèvent, et la disent heureuse; Son mari se lève, et lui donne des louanges:
\par 29 Plusieurs filles ont une conduite vertueuse; Mais toi, tu les surpasses toutes.
\par 30 La grâce est trompeuse, et la beauté est vaine; La femme qui craint l'Éternel est celle qui sera louée.
\par 31 Récompensez-la du fruit de son travail, Et qu'aux portes ses oeuvres la louent.


\end{document}