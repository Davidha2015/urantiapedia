\begin{document}

\title{Song of Solomon}


\chapter{1}

\par 1 Cantique des cantiques, de Salomon.
\par 2 Qu'il me baise des baisers de sa bouche! Car ton amour vaut mieux que le vin,
\par 3 Tes parfums ont une odeur suave; Ton nom est un parfum qui se répand; C'est pourquoi les jeunes filles t'aiment.
\par 4 Entraîne-moi après toi! Nous courrons! Le roi m'introduit dans ses appartements... Nous nous égaierons, nous nous réjouirons à cause de toi; Nous célébrerons ton amour plus que le vin. C'est avec raison que l'on t'aime.
\par 5 Je suis noire, mais je suis belle, filles de Jérusalem, Comme les tentes de Kédar, comme les pavillons de Salomon.
\par 6 Ne prenez pas garde à mon teint noir: C'est le soleil qui m'a brûlée. Les fils de ma mère se sont irrités contre moi, Ils m'ont faite gardienne des vignes. Ma vigne, à moi, je ne l'ai pas gardée.
\par 7 Dis-moi, ô toi que mon coeur aime, Où tu fais paître tes brebis, Où tu les fais reposer à midi; Car pourquoi serais-je comme une égarée Près des troupeaux de tes compagnons? -
\par 8 Si tu ne le sais pas, ô la plus belle des femmes, Sors sur les traces des brebis, Et fais paître tes chevreaux Près des demeures des bergers. -
\par 9 A ma jument qu'on attelle aux chars de Pharaon Je te compare, ô mon amie.
\par 10 Tes joues sont belles au milieu des colliers, Ton cou est beau au milieu des rangées de perles.
\par 11 Nous te ferons des colliers d'or, Avec des points d'argent. -
\par 12 Tandis que le roi est dans son entourage, Mon nard exhale son parfum.
\par 13 Mon bien-aimé est pour moi un bouquet de myrrhe, Qui repose entre mes seins.
\par 14 Mon bien-aimé est pour moi une grappe de troëne Des vignes d'En Guédi. -
\par 15 Que tu es belle, mon amie, que tu es belle! Tes yeux sont des colombes. -
\par 16 Que tu es beau, mon bien-aimé, que tu es aimable! Notre lit, c'est la verdure. -
\par 17 Les solives de nos maisons sont des cèdres, Nos lambris sont des cyprès. -

\chapter{2}

\par 1 Je suis un narcisse de Saron, Un lis des vallées. -
\par 2 Comme un lis au milieu des épines, Telle est mon amie parmi les jeunes filles. -
\par 3 Comme un pommier au milieu des arbres de la forêt, Tel est mon bien-aimé parmi les jeunes hommes. J'ai désiré m'asseoir à son ombre, Et son fruit est doux à mon palais.
\par 4 Il m'a fait entrer dans la maison du vin; Et la bannière qu'il déploie sur moi, c'est l'amour.
\par 5 Soutenez-moi avec des gâteaux de raisins, Fortifiez-moi avec des pommes; Car je suis malade d'amour.
\par 6 Que sa main gauche soit sous ma tête, Et que sa droite m'embrasse! -
\par 7 Je vous en conjure, filles de Jérusalem, Par les gazelles et les biches des champs, Ne réveillez pas, ne réveillez pas l'amour, Avant qu'elle le veuille. -
\par 8 C'est la voix de mon bien-aimé! Le voici, il vient, Sautant sur les montagnes, Bondissant sur les collines.
\par 9 Mon bien-aimé est semblable à la gazelle Ou au faon des biches. Le voici, il est derrière notre mur, Il regarde par la fenêtre, Il regarde par le treillis.
\par 10 Mon bien-aimé parle et me dit: Lève-toi, mon amie, ma belle, et viens!
\par 11 Car voici, l'hiver est passé; La pluie a cessé, elle s'en est allée.
\par 12 Les fleurs paraissent sur la terre, Le temps de chanter est arrivé, Et la voix de la tourterelle se fait entendre dans nos campagnes.
\par 13 Le figuier embaume ses fruits, Et les vignes en fleur exhalent leur parfum. Lève-toi, mon amie, ma belle, et viens!
\par 14 Ma colombe, qui te tiens dans les fentes du rocher, Qui te caches dans les parois escarpées, Fais-moi voir ta figure, Fais-moi entendre ta voix; Car ta voix est douce, et ta figure est agréable.
\par 15 Prenez-nous les renards, Les petits renards qui ravagent les vignes; Car nos vignes sont en fleur.
\par 16 Mon bien-aimé est à moi, et je suis à lui; Il fait paître son troupeau parmi les lis.
\par 17 Avant que le jour se rafraîchisse, Et que les ombres fuient, Reviens!... sois semblable, mon bien-aimé, A la gazelle ou au faon des biches, Sur les montagnes qui nous séparent.

\chapter{3}

\par 1 Sur ma couche, pendant les nuits, J'ai cherché celui que mon coeur aime; Je l'ai cherché, et je ne l'ai point trouvé...
\par 2 Je me lèverai, et je ferai le tour de la ville, Dans les rues et sur les places; Je chercherai celui que mon coeur aime... Je l'ai cherché, et je ne l'ai point trouvé.
\par 3 Les gardes qui font la ronde dans la ville m'ont rencontrée: Avez-vous vu celui que mon coeur aime?
\par 4 A peine les avais-je passés, Que j'ai trouvé celui que mon coeur aime; Je l'ai saisi, et je ne l'ai point lâché Jusqu'à ce que je l'aie amené dans la maison de ma mère, Dans la chambre de celle qui m'a conçue. -
\par 5 Je vous en conjure, filles de Jérusalem, Par les gazelles et les biches des champs, Ne réveillez pas, ne réveillez pas l'amour, Avant qu'elle le veuille. -
\par 6 Qui est celle qui monte du désert, Comme des colonnes de fumée, Au milieu des vapeurs de myrrhe et d'encens Et de tous les aromates des marchands? -
\par 7 Voici la litière de Salomon, Et autour d'elle soixante vaillants hommes, Des plus vaillants d'Israël.
\par 8 Tous sont armés de l'épée, Sont exercés au combat; Chacun porte l'épée sur sa hanche, En vue des alarmes nocturnes.
\par 9 Le roi Salomon s'est fait une litière De bois du Liban.
\par 10 Il en a fait les colonnes d'argent, Le dossier d'or, Le siège de pourpre; Au milieu est une broderie, oeuvre d'amour Des filles de Jérusalem.
\par 11 Sortez, filles de Sion, regardez Le roi Salomon, Avec la couronne dont sa mère l'a couronné Le jour de ses fiançailles, Le jour de la joie de son coeur. -

\chapter{4}

\par 1 Que tu es belle, mon amie, que tu es belle! Tes yeux sont des colombes, Derrière ton voile. Tes cheveux sont comme un troupeau de chèvres, Suspendues aux flancs de la montagne de Galaad.
\par 2 Tes dents sont comme un troupeau de brebis tondues, Qui remontent de l'abreuvoir; Toutes portent des jumeaux, Aucune d'elles n'est stérile.
\par 3 Tes lèvres sont comme un fil cramoisi, Et ta bouche est charmante; Ta joue est comme une moitié de grenade, Derrière ton voile.
\par 4 Ton cou est comme la tour de David, Bâtie pour être un arsenal; Mille boucliers y sont suspendus, Tous les boucliers des héros.
\par 5 Tes deux seins sont comme deux faons, Comme les jumeaux d'une gazelle, Qui paissent au milieu des lis.
\par 6 Avant que le jour se rafraîchisse, Et que les ombres fuient, J'irai à la montagne de la myrrhe Et à la colline de l'encens.
\par 7 Tu es toute belle, mon amie, Et il n'y a point en toi de défaut.
\par 8 Viens avec moi du Liban, ma fiancée, Viens avec moi du Liban! Regarde du sommet de l'Amana, Du sommet du Senir et de l'Hermon, Des tanières des lions, Des montagnes des léopards.
\par 9 Tu me ravis le coeur, ma soeur, ma fiancée, Tu me ravis le coeur par l'un de tes regards, Par l'un des colliers de ton cou.
\par 10 Que de charmes dans ton amour, ma soeur, ma fiancée! Comme ton amour vaut mieux que le vin, Et combien tes parfums sont plus suaves que tous les aromates!
\par 11 Tes lèvres distillent le miel, ma fiancée; Il y a sous ta langue du miel et du lait, Et l'odeur de tes vêtements est comme l'odeur du Liban.
\par 12 Tu es un jardin fermé, ma soeur, ma fiancée, Une source fermée, une fontaine scellée.
\par 13 Tes jets forment un jardin, où sont des grenadiers, Avec les fruits les plus excellents, Les troënes avec le nard;
\par 14 Le nard et le safran, le roseau aromatique et le cinnamome, Avec tous les arbres qui donnent l'encens; La myrrhe et l'aloès, Avec tous les principaux aromates;
\par 15 Une fontaine des jardins, Une source d'eaux vives, Des ruisseaux du Liban.
\par 16 Lève-toi, aquilon! viens, autan! Soufflez sur mon jardin, et que les parfums s'en exhalent! -Que mon bien-aimé entre dans son jardin, Et qu'il mange de ses fruits excellents! -

\chapter{5}

\par 1 J'entre dans mon jardin, ma soeur, ma fiancée; Je cueille ma myrrhe avec mes aromates, Je mange mon rayon de miel avec mon miel, Je bois mon vin avec mon lait... -Mangez, amis, buvez, enivrez-vous d'amour! -
\par 2 J'étais endormie, mais mon coeur veillait... C'est la voix de mon bien-aimé, qui frappe: -Ouvre-moi, ma soeur, mon amie, Ma colombe, ma parfaite! Car ma tête est couverte de rosée, Mes boucles sont pleines des gouttes de la nuit. -
\par 3 J'ai ôté ma tunique; comment la remettrais-je? J'ai lavé mes pieds; comment les salirais-je?
\par 4 Mon bien-aimé a passé la main par la fenêtre, Et mes entrailles se sont émues pour lui.
\par 5 Je me suis levée pour ouvrir à mon bien-aimé; Et de mes mains a dégoutté la myrrhe, De mes doigts, la myrrhe répandue Sur la poignée du verrou.
\par 6 J'ai ouvert à mon bien-aimé; Mais mon bien-aimé s'en était allé, il avait disparu. J'étais hors de moi, quand il me parlait. Je l'ai cherché, et je ne l'ai point trouvé; Je l'ai appelé, et il ne m'a point répondu.
\par 7 Les gardes qui font la ronde dans la ville m'ont rencontrée; Ils m'ont frappée, ils m'ont blessée; Ils m'ont enlevé mon voile, les gardes des murs.
\par 8 Je vous en conjure, filles de Jérusalem, Si vous trouvez mon bien-aimé, Que lui direz-vous?... Que je suis malade d'amour. -
\par 9 Qu'a ton bien-aimé de plus qu'un autre, O la plus belle des femmes? Qu'a ton bien-aimé de plus qu'un autre, Pour que tu nous conjures ainsi? -
\par 10 Mon bien-aimé est blanc et vermeil; Il se distingue entre dix mille.
\par 11 Sa tête est de l'or pur; Ses boucles sont flottantes, Noires comme le corbeau.
\par 12 Ses yeux sont comme des colombes au bord des ruisseaux, Se baignant dans le lait, Reposant au sein de l'abondance.
\par 13 Ses joues sont comme un parterre d'aromates, Une couche de plantes odorantes; Ses lèvres sont des lis, D'où découle la myrrhe.
\par 14 Ses mains sont des anneaux d'or, Garnis de chrysolithes; Son corps est de l'ivoire poli, Couvert de saphirs;
\par 15 Ses jambes sont des colonnes de marbre blanc, Posées sur des bases d'or pur. Son aspect est comme le Liban, Distingué comme les cèdres.
\par 16 Son palais n'est que douceur, Et toute sa personne est pleine de charme. Tel est mon bien-aimé, tel est mon ami, Filles de Jérusalem! -

\chapter{6}

\par 1 Où est allé ton bien-aimé, O la plus belle des femmes? De quel côté ton bien-aimé s'est-il dirigé? Nous le chercherons avec toi.
\par 2 Mon bien-aimé est descendu à son jardin, Au parterre d'aromates, Pour faire paître son troupeau dans les jardins, Et pour cueillir des lis.
\par 3 Je suis à mon bien-aimé, et mon bien-aimé est à moi; Il fait paître son troupeau parmi les lis. -
\par 4 Tu es belle, mon amie, comme Thirtsa, Agréable comme Jérusalem, Mais terrible comme des troupes sous leurs bannières.
\par 5 Détourne de moi tes yeux, car ils me troublent. Tes cheveux sont comme un troupeau de chèvres, Suspendues aux flancs de Galaad.
\par 6 Tes dents sont comme un troupeau de brebis, Qui remontent de l'abreuvoir; Toutes portent des jumeaux, Aucune d'elles n'est stérile.
\par 7 Ta joue est comme une moitié de grenade, Derrière ton voile...
\par 8 Il y a soixante reines, quatre-vingts concubines, Et des jeunes filles sans nombre.
\par 9 Une seule est ma colombe, ma parfaite; Elle est l'unique de sa mère, La préférée de celle qui lui donna le jour. Les jeunes filles la voient, et la disent heureuse; Les reines et les concubines aussi, et elles la louent. -
\par 10 Qui est celle qui apparaît comme l'aurore, Belle comme la lune, pure comme le soleil, Mais terrible comme des troupes sous leurs bannières? -
\par 11 Je suis descendue au jardin des noyers, Pour voir la verdure de la vallée, Pour voir si la vigne pousse, Si les grenadiers fleurissent.
\par 12 Je ne sais, mais mon désir m'a rendue semblable Aux chars de mon noble peuple. -

\chapter{7}

\par 1 Reviens, reviens, Sulamithe! Reviens, reviens, afin que nous te regardions. -Qu'avez-vous à regarder la Sulamithe Comme une danse de deux choeurs?
\par 2 Que tes pieds sont beaux dans ta chaussure, fille de prince! Les contours de ta hanche sont comme des colliers, Oeuvre des mains d'un artiste.
\par 3 Ton sein est une coupe arrondie, Où le vin parfumé ne manque pas; Ton corps est un tas de froment, Entouré de lis.
\par 4 Tes deux seins sont comme deux faons, Comme les jumeaux d'une gazelle.
\par 5 Ton cou est comme une tour d'ivoire; Tes yeux sont comme les étangs de Hesbon, Près de la porte de Bath Rabbim; Ton nez est comme la tour du Liban, Qui regarde du côté de Damas.
\par 6 Ta tête est élevée comme le Carmel, Et les cheveux de ta tête sont comme la pourpre; Un roi est enchaîné par des boucles!...
\par 7 Que tu es belle, que tu es agréable, O mon amour, au milieu des délices!
\par 8 Ta taille ressemble au palmier, Et tes seins à des grappes.
\par 9 Je me dis: Je monterai sur le palmier, J'en saisirai les rameaux! Que tes seins soient comme les grappes de la vigne, Le parfum de ton souffle comme celui des pommes,
\par 10 Et ta bouche comme un vin excellent,... -Qui coule aisément pour mon bien-aimé, Et glisse sur les lèvres de ceux qui s'endorment!
\par 11 Je suis à mon bien-aimé, Et ses désirs se portent vers moi.
\par 12 Viens, mon bien-aimé, sortons dans les champs, Demeurons dans les villages!
\par 13 Dès le matin nous irons aux vignes, Nous verrons si la vigne pousse, si la fleur s'ouvre, Si les grenadiers fleurissent. Là je te donnerai mon amour.
\par 14 Les mandragores répandent leur parfum, Et nous avons à nos portes tous les meilleurs fruits, Nouveaux et anciens: Mon bien-aimé, je les ai gardés pour toi.

\chapter{8}

\par 1 Oh! Que n'es-tu mon frère, Allaité des mamelles de ma mère! Je te rencontrerais dehors, je t'embrasserais, Et l'on ne me mépriserait pas.
\par 2 Je veux te conduire, t'amener à la maison de ma mère; Tu me donneras tes instructions, Et je te ferai boire du vin parfumé, Du moût de mes grenades.
\par 3 Que sa main gauche soit sous ma tête, Et que sa droite m'embrasse! -
\par 4 Je vous en conjure, filles de Jérusalem, Ne réveillez pas, ne réveillez pas l'amour, Avant qu'elle le veuille. -
\par 5 Qui est celle qui monte du désert, Appuyée sur son bien-aimé? -Je t'ai réveillée sous le pommier; Là ta mère t'a enfantée, C'est là qu'elle t'a enfantée, qu'elle t'a donné le jour. -
\par 6 Mets-moi comme un sceau sur ton coeur, Comme un sceau sur ton bras; Car l'amour est fort comme la mort, La jalousie est inflexible comme le séjour des morts; Ses ardeurs sont des ardeurs de feu, Une flamme de l'Éternel.
\par 7 Les grandes eaux ne peuvent éteindre l'amour, Et les fleuves ne le submergeraient pas; Quand un homme offrirait tous les biens de sa maison contre l'amour, Il ne s'attirerait que le mépris.
\par 8 Nous avons une petite soeur, Qui n'a point encore de mamelles; Que ferons-nous de notre soeur, Le jour où on la recherchera?
\par 9 Si elle est un mur, Nous bâtirons sur elle des créneaux d'argent; Si elle est une porte, Nous la fermerons avec une planche de cèdre. -
\par 10 Je suis un mur, Et mes seins sont comme des tours; J'ai été à ses yeux comme celle qui trouve la paix.
\par 11 Salomon avait une vigne à Baal Hamon; Il remit la vigne à des gardiens; Chacun apportait pour son fruit mille sicles d'argent.
\par 12 Ma vigne, qui est à moi, je la garde. A toi, Salomon, les mille sicles, Et deux cents à ceux qui gardent le fruit! -
\par 13 Habitante des jardins! Des amis prêtent l'oreille à ta voix. Daigne me la faire entendre! -
\par 14 Fuis, mon bien-aimé! Sois semblable à la gazelle ou au faon des biches, Sur les montagnes des aromates!


\end{document}