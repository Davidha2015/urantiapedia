\begin{document}

\title{Jonah}


\chapter{1}

\par 1 La parole de l'Éternel fut adressée à Jonas, fils d'Amitthaï, en ces mots:
\par 2 Lève-toi, va à Ninive, la grande ville, et crie contre elle! car sa méchanceté est montée jusqu'à moi.
\par 3 Et Jonas se leva pour s'enfuir à Tarsis, loin de la face de l'Éternel. Il descendit à Japho, et il trouva un navire qui allait à Tarsis; il paya le prix du transport, et s'embarqua pour aller avec les passagers à Tarsis, loin de la face de l'Éternel.
\par 4 Mais l'Éternel fit souffler sur la mer un vent impétueux, et il s'éleva sur la mer une grande tempête. Le navire menaçait de faire naufrage.
\par 5 Les mariniers eurent peur, ils implorèrent chacun leur dieu, et ils jetèrent dans la mer les objets qui étaient sur le navire, afin de le rendre plus léger. Jonas descendit au fond du navire, se coucha, et s'endormit profondément.
\par 6 Le pilote s'approcha de lui, et lui dit: Pourquoi dors-tu? Lève-toi, invoque ton Dieu! peut-être voudra-t-il penser à nous, et nous ne périrons pas.
\par 7 Et il se rendirent l'un à l'autre: Venez, et tirons au sort, pour savoir qui nous attire ce malheur. Ils tirèrent au sort, et le sort tomba sur Jonas.
\par 8 Alors ils lui dirent: Dis-nous qui nous attire ce malheur. Quelles sont tes affaires, et d'où viens-tu? Quel est ton pays, et de quel peuple es-tu?
\par 9 Il leur répondit: Je suis Hébreu, et je crains l'Éternel, le Dieu des cieux, qui a fait la mer et la terre.
\par 10 Ces hommes eurent une grande frayeur, et ils lui dirent: Pourquoi as-tu fait cela? Car ces hommes savaient qu'il fuyait loin de la face de l'Éternel, parce qu'il le leur avait déclaré.
\par 11 Ils lui dirent: Que te ferons-nous, pour que la mer se calme envers nous? Car la mer était de plus en plus orageuse.
\par 12 Il leur répondit: Prenez-moi, et jetez-moi dans la mer, et la mer se calmera envers vous; car je sais que c'est moi qui attire sur vous cette grande tempête.
\par 13 Ces hommes ramaient pour gagner la terre, mais ils ne le purent, parce que la mer s'agitait toujours plus contre eux.
\par 14 Alors ils invoquèrent l'Éternel, et dirent: O Éternel, ne nous fais pas périr à cause de la vie de cet homme, et ne nous charge pas du sang innocent! Car toi, Éternel, tu fais ce que tu veux.
\par 15 Puis ils prirent Jonas, et le jetèrent dans la mer. Et la fureur de la mer s'apaisa.
\par 16 Ces hommes furent saisis d'une grande crainte de l'Éternel, et ils offrirent un sacrifice à l'Éternel, et firent des voeux.

\chapter{2}

\par 1 L'Éternel fit venir un grand poisson pour engloutir Jonas, et Jonas fut dans le ventre du poisson trois jours et trois nuits.
\par 2 Jonas, dans le ventre du poisson, pria l'Éternel, son Dieu.
\par 3 Il dit: Dans ma détresse, j'ai invoqué l'Éternel, Et il m'a exaucé; Du sein du séjour des morts j'ai crié, Et tu as entendu ma voix.
\par 4 Tu m'as jeté dans l'abîme, dans le coeur de la mer, Et les courants d'eau m'ont environné; Toutes tes vagues et tous tes flots ont passé sur moi.
\par 5 Je disais: Je suis chassé loin de ton regard! Mais je verrai encore ton saint temple.
\par 6 Les eaux m'ont couvert jusqu'à m'ôter la vie, L'abîme m'a enveloppé, Les roseaux ont entouré ma tête.
\par 7 Je suis descendu jusqu'aux racines des montagnes, Les barres de la terre m'enfermaient pour toujours; Mais tu m'as fait remonter vivant de la fosse, Éternel, mon Dieu!
\par 8 Quand mon âme était abattue au dedans de moi, Je me suis souvenu de l'Éternel, Et ma prière est parvenue jusqu'à toi, Dans ton saint temple.
\par 9 Ceux qui s'attachent à de vaines idoles Éloignent d'eux la miséricorde.
\par 10 Pour moi, je t'offrirai des sacrifices avec un cri d'actions de grâces, J'accomplirai les voeux que j'ai faits: Le salut vient de l'Éternel.
\par 11 L'Éternel parla au poisson, et le poisson vomit Jonas sur la terre.

\chapter{3}

\par 1 La parole de l'Éternel fut adressée à Jonas une seconde fois, en ces mots:
\par 2 Lève-toi, va à Ninive, la grande ville, et proclames-y la publication que je t'ordonne!
\par 3 Et Jonas se leva, et alla à Ninive, selon la parole de l'Éternel. Or Ninive était une très grande ville, de trois jours de marche.
\par 4 Jonas fit d'abord dans la ville une journée de marche; il criait et disait: Encore quarante jours, et Ninive est détruite!
\par 5 Les gens de Ninive crurent à Dieu, ils publièrent un jeûne, et se revêtirent de sacs, depuis les plus grands jusqu'aux plus petits.
\par 6 La chose parvint au roi de Ninive; il se leva de son trône, ôta son manteau, se couvrit d'un sac, et s'assit sur la cendre.
\par 7 Et il fit faire dans Ninive cette publication, par ordre du roi et de ses grands; Que les hommes et les bête, les boeufs et les brebis, ne goûtent de rien, ne paissent point, et ne boivent point d'eau!
\par 8 Que les hommes et les bêtes soient couverts de sacs, qu'ils crient à Dieu avec force, et qu'ils reviennent tous de leur mauvaise voie et des actes de violence dont leurs mains sont coupables!
\par 9 Qui sait si Dieu ne reviendra pas et ne se repentira pas, et s'il ne renoncera pas à son ardente colère, en sorte que nous ne périssions point?
\par 10 Dieu vit qu'ils agissaient ainsi et qu'ils revenaient de leur mauvaise voie. Alors Dieu se repentit du mal qu'il avait résolu de leur faire, et il ne le fit pas.

\chapter{4}

\par 1 Cela déplut fort à Jonas, et il fut irrité.
\par 2 Il implora l'Éternel, et il dit: Ah! Éternel, n'est-ce pas ce que je disais quand j'étais encore dans mon pays? C'est ce que je voulais prévenir en fuyant à Tarsis. Car je savais que tu es un Dieu compatissant et miséricordieux, lent à la colère et riche en bonté, et qui te repens du mal.
\par 3 Maintenant, Éternel, prends-moi donc la vie, car la mort m'est préférable à la vie.
\par 4 L'Éternel répondit: Fais-tu bien de t'irriter?
\par 5 Et Jonas sortit de la ville, et s'assit à l'orient de la ville, Là il se fit une cabane, et s'y tint à l'ombre, jusqu'à ce qu'il vît ce qui arriverait dans la ville.
\par 6 L'Éternel Dieu fit croître un ricin, qui s'éleva au-dessus de Jonas, pour donner de l'ombre sur sa tête et pour lui ôter son irritation. Jonas éprouva une grande joie à cause de ce ricin.
\par 7 Mais le lendemain, à l'aurore, Dieu fit venir un ver qui piqua le ricin, et le ricin sécha.
\par 8 Au lever du soleil, Dieu fit souffler un vent chaud d'orient, et le soleil frappa la tête de Jonas, au point qu'il tomba en défaillance. Il demanda la mort, et dit: La mort m'est préférable à la vie.
\par 9 Dieu dit à Jonas: Fais-tu bien de t'irriter à cause du ricin? Il répondit: Je fais bien de m'irriter jusqu'à la mort.
\par 10 Et l'Éternel dit: Tu as pitié du ricin qui ne t'a coûté aucune peine et que tu n'as pas fait croître, qui est né dans une nuit et qui a péri dans une nuit.
\par 11 Et moi, je n'aurais pas pitié de Ninive, la grande ville, dans laquelle se trouvent plus de cent vingt mille hommes qui ne savent pas distinguer leur droite de leur gauche, et des animaux en grand nombre!


\end{document}