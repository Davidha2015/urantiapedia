\begin{document}

\title{Luke}


\chapter{1}

\par 1 Plusieurs ayant entrepris de composer un récit des événements qui se sont accomplis parmi nous,
\par 2 suivant ce que nous ont transmis ceux qui ont été des témoins oculaires dès le commencement et sont devenus des ministres de la parole,
\par 3 il m'a aussi semblé bon, après avoir fait des recherches exactes sur toutes ces choses depuis leur origine, de te les exposer par écrit d'une manière suivie, excellent Théophile,
\par 4 afin que tu reconnaisses la certitude des enseignements que tu as reçus.
\par 5 Du temps d'Hérode, roi de Judée, il y avait un sacrificateur, nommé Zacharie, de la classe d'Abia; sa femme était d'entre les filles d'Aaron, et s'appelait Élisabeth.
\par 6 Tous deux étaient justes devant Dieu, observant d'une manière irréprochable tous les commandements et toutes les ordonnances du Seigneur.
\par 7 Ils n'avaient point d'enfants, parce qu'Élisabeth était stérile; et ils étaient l'un et l'autre avancés en âge.
\par 8 Or, pendant qu'il s'acquittait de ses fonctions devant Dieu, selon le tour de sa classe, il fut appelé par le sort,
\par 9 d'après la règle du sacerdoce, à entrer dans le temple du Seigneur pour offrir le parfum.
\par 10 Toute la multitude du peuple était dehors en prière, à l'heure du parfum.
\par 11 Alors un ange du Seigneur apparut à Zacharie, et se tint debout à droite de l'autel des parfums.
\par 12 Zacharie fut troublé en le voyant, et la frayeur s'empara de lui.
\par 13 Mais l'ange lui dit: Ne crains point, Zacharie; car ta prière a été exaucée. Ta femme Élisabeth t'enfantera un fils, et tu lui donneras le nom de Jean.
\par 14 Il sera pour toi un sujet de joie et d'allégresse, et plusieurs se réjouiront de sa naissance.
\par 15 Car il sera grand devant le Seigneur. Il ne boira ni vin, ni liqueur enivrante, et il sera rempli de l'Esprit Saint dès le sein de sa mère;
\par 16 il ramènera plusieurs des fils d'Israël au Seigneur, leur Dieu;
\par 17 il marchera devant Dieu avec l'esprit et la puissance d'Élie, pour ramener les coeurs des pères vers les enfants, et les rebelles à la sagesse des justes, afin de préparer au Seigneur un peuple bien disposé.
\par 18 Zacharie dit à l'ange: A quoi reconnaîtrai-je cela? Car je suis vieux, et ma femme est avancée en âge.
\par 19 L'ange lui répondit: Je suis Gabriel, je me tiens devant Dieu; j'ai été envoyé pour te parler, et pour t'annoncer cette bonne nouvelle.
\par 20 Et voici, tu seras muet, et tu ne pourras parler jusqu'au jour où ces choses arriveront, parce que tu n'as pas cru à mes paroles, qui s'accompliront en leur temps.
\par 21 Cependant, le peuple attendait Zacharie, s'étonnant de ce qu'il restait si longtemps dans le temple.
\par 22 Quand il sortit, il ne put leur parler, et ils comprirent qu'il avait eu une vision dans le temple; il leur faisait des signes, et il resta muet.
\par 23 Lorsque ses jours de service furent écoulés, il s'en alla chez lui.
\par 24 Quelque temps après, Élisabeth, sa femme, devint enceinte. Elle se cacha pendant cinq mois, disant:
\par 25 C'est la grâce que le Seigneur m'a faite, quand il a jeté les yeux sur moi pour ôter mon opprobre parmi les hommes.
\par 26 Au sixième mois, l'ange Gabriel fut envoyé par Dieu dans une ville de Galilée, appelée Nazareth,
\par 27 auprès d'une vierge fiancée à un homme de la maison de David, nommé Joseph. Le nom de la vierge était Marie.
\par 28 L'ange entra chez elle, et dit: Je te salue, toi à qui une grâce a été faite; le Seigneur est avec toi.
\par 29 Troublée par cette parole, Marie se demandait ce que pouvait signifier une telle salutation.
\par 30 L'ange lui dit: Ne crains point, Marie; car tu as trouvé grâce devant Dieu.
\par 31 Et voici, tu deviendras enceinte, et tu enfanteras un fils, et tu lui donneras le nom de Jésus.
\par 32 Il sera grand et sera appelé Fils du Très Haut, et le Seigneur Dieu lui donnera le trône de David, son père.
\par 33 Il règnera sur la maison de Jacob éternellement, et son règne n'aura point de fin.
\par 34 Marie dit à l'ange: Comment cela se fera-t-il, puisque je ne connais point d'homme?
\par 35 L'ange lui répondit: Le Saint Esprit viendra sur toi, et la puissance du Très Haut te couvrira de son ombre. C'est pourquoi le saint enfant qui naîtra de toi sera appelé Fils de Dieu.
\par 36 Voici, Élisabeth, ta parente, a conçu, elle aussi, un fils en sa vieillesse, et celle qui était appelée stérile est dans son sixième mois.
\par 37 Car rien n'est impossible à Dieu.
\par 38 Marie dit: Je suis la servante du Seigneur; qu'il me soit fait selon ta parole! Et l'ange la quitta.
\par 39 Dans ce même temps, Marie se leva, et s'en alla en hâte vers les montagnes, dans une ville de Juda.
\par 40 Elle entra dans la maison de Zacharie, et salua Élisabeth.
\par 41 Dès qu'Élisabeth entendit la salutation de Marie, son enfant tressaillit dans son sein, et elle fut remplie du Saint Esprit.
\par 42 Elle s'écria d'une voix forte: Tu es bénie entre les femmes, et le fruit de ton sein est béni.
\par 43 Comment m'est-il accordé que la mère de mon Seigneur vienne auprès de moi?
\par 44 Car voici, aussitôt que la voix de ta salutation a frappé mon oreille, l'enfant a tressailli d'allégresse dans mon sein.
\par 45 Heureuse celle qui a cru, parce que les choses qui lui ont été dites de la part du Seigneur auront leur accomplissement.
\par 46 Et Marie dit: Mon âme exalte le Seigneur,
\par 47 Et mon esprit se réjouit en Dieu, mon Sauveur,
\par 48 Parce qu'il a jeté les yeux sur la bassesse de sa servante. Car voici, désormais toutes les générations me diront bienheureuse,
\par 49 Parce que le Tout Puissant a fait pour moi de grandes choses. Son nom est saint,
\par 50 Et sa miséricorde s'étend d'âge en âge Sur ceux qui le craignent.
\par 51 Il a déployé la force de son bras; Il a dispersé ceux qui avaient dans le coeur des pensées orgueilleuses.
\par 52 Il a renversé les puissants de leurs trônes, Et il a élevé les humbles.
\par 53 Il a rassasié de biens les affamés, Et il a renvoyé les riches à vide.
\par 54 Il a secouru Israël, son serviteur, Et il s'est souvenu de sa miséricorde, -
\par 55 Comme il l'avait dit à nos pères, -Envers Abraham et sa postérité pour toujours.
\par 56 Marie demeura avec Élisabeth environ trois mois. Puis elle retourna chez elle.
\par 57 Le temps où Élisabeth devait accoucher arriva, et elle enfanta un fils.
\par 58 Ses voisins et ses parents apprirent que le Seigneur avait fait éclater envers elle sa miséricorde, et ils se réjouirent avec elle.
\par 59 Le huitième jour, ils vinrent pour circoncire l'enfant, et ils l'appelaient Zacharie, du nom de son père.
\par 60 Mais sa mère prit la parole, et dit: Non, il sera appelé Jean.
\par 61 Ils lui dirent: Il n'y a dans ta parenté personne qui soit appelé de ce nom.
\par 62 Et ils firent des signes à son père pour savoir comment il voulait qu'on l'appelle.
\par 63 Zacharie demanda des tablettes, et il écrivit: Jean est son nom. Et tous furent dans l'étonnement.
\par 64 Au même instant, sa bouche s'ouvrit, sa langue se délia, et il parlait, bénissant Dieu.
\par 65 La crainte s'empara de tous les habitants d'alentour, et, dans toutes les montagnes de la Judée, on s'entretenait de toutes ces choses.
\par 66 Tous ceux qui les apprirent les gardèrent dans leur coeur, en disant: Que sera donc cet enfant? Et la main du Seigneur était avec lui.
\par 67 Zacharie, son père, fut rempli du Saint Esprit, et il prophétisa, en ces mots:
\par 68 Béni soit le Seigneur, le Dieu d'Israël, De ce qu'il a visité et racheté son peuple,
\par 69 Et nous a suscité un puissant Sauveur Dans la maison de David, son serviteur,
\par 70 Comme il l'avait annoncé par la bouche de ses saints prophètes des temps anciens, -
\par 71 Un Sauveur qui nous délivre de nos ennemis et de la main de tous ceux qui nous haïssent!
\par 72 C'est ainsi qu'il manifeste sa miséricorde envers nos pères, Et se souvient de sa sainte alliance,
\par 73 Selon le serment par lequel il avait juré à Abraham, notre père,
\par 74 De nous permettre, après que nous serions délivrés de la main de nos ennemis, De le servir sans crainte,
\par 75 En marchant devant lui dans la sainteté et dans la justice tous les jours de notre vie.
\par 76 Et toi, petit enfant, tu seras appelé prophète du Très Haut; Car tu marcheras devant la face du Seigneur, pour préparer ses voies,
\par 77 Afin de donner à son peuple la connaissance du salut Par le pardon de ses péchés,
\par 78 Grâce aux entrailles de la miséricorde de notre Dieu, En vertu de laquelle le soleil levant nous a visités d'en haut,
\par 79 Pour éclairer ceux qui sont assis dans les ténèbres et dans l'ombre de la mort, Pour diriger nos pas dans le chemin de la paix.
\par 80 Or, l'enfant croissait, et se fortifiait en esprit. Et il demeura dans les déserts, jusqu'au jour où il se présenta devant Israël.

\chapter{2}

\par 1 En ce temps-là parut un édit de César Auguste, ordonnant un recensement de toute la terre.
\par 2 Ce premier recensement eut lieu pendant que Quirinius était gouverneur de Syrie.
\par 3 Tous allaient se faire inscrire, chacun dans sa ville.
\par 4 Joseph aussi monta de la Galilée, de la ville de Nazareth, pour se rendre en Judée, dans la ville de David, appelée Bethléhem, parce qu'il était de la maison et de la famille de David,
\par 5 afin de se faire inscrire avec Marie, sa fiancée, qui était enceinte.
\par 6 Pendant qu'ils étaient là, le temps où Marie devait accoucher arriva,
\par 7 et elle enfanta son fils premier-né. Elle l'emmaillota, et le coucha dans une crèche, parce qu'il n'y avait pas de place pour eux dans l'hôtellerie.
\par 8 Il y avait, dans cette même contrée, des bergers qui passaient dans les champs les veilles de la nuit pour garder leurs troupeaux.
\par 9 Et voici, un ange du Seigneur leur apparut, et la gloire du Seigneur resplendit autour d'eux. Ils furent saisis d'une grande frayeur.
\par 10 Mais l'ange leur dit: Ne craignez point; car je vous annonce une bonne nouvelle, qui sera pour tout le peuple le sujet d'une grande joie:
\par 11 c'est qu'aujourd'hui, dans la ville de David, il vous est né un Sauveur, qui est le Christ, le Seigneur.
\par 12 Et voici à quel signe vous le reconnaîtrez: vous trouverez un enfant emmailloté et couché dans une crèche.
\par 13 Et soudain il se joignit à l'ange une multitude de l'armée céleste, louant Dieu et disant:
\par 14 Gloire à Dieu dans les lieux très hauts, Et paix sur la terre parmi les hommes qu'il agrée!
\par 15 Lorsque les anges les eurent quittés pour retourner au ciel, les bergers se dirent les uns aux autres: Allons jusqu'à Bethléhem, et voyons ce qui est arrivé, ce que le Seigneur nous a fait connaître.
\par 16 Ils y allèrent en hâte, et ils trouvèrent Marie et Joseph, et le petit enfant couché dans la crèche.
\par 17 Après l'avoir vu, ils racontèrent ce qui leur avait été dit au sujet de ce petit enfant.
\par 18 Tous ceux qui les entendirent furent dans l'étonnement de ce que leur disaient les bergers.
\par 19 Marie gardait toutes ces choses, et les repassait dans son coeur.
\par 20 Et les bergers s'en retournèrent, glorifiant et louant Dieu pour tout ce qu'ils avaient entendu et vu, et qui était conforme à ce qui leur avait été annoncé.
\par 21 Le huitième jour, auquel l'enfant devait être circoncis, étant arrivé, on lui donna le nom de Jésus, nom qu'avait indiqué l'ange avant qu'il fût conçu dans le sein de sa mère.
\par 22 Et, quand les jours de leur purification furent accomplis, selon la loi de Moïse, Joseph et Marie le portèrent à Jérusalem, pour le présenter au Seigneur, -
\par 23 suivant ce qui est écrit dans la loi du Seigneur: Tout mâle premier-né sera consacré au Seigneur, -
\par 24 et pour offrir en sacrifice deux tourterelles ou deux jeunes pigeons, comme cela est prescrit dans la loi du Seigneur.
\par 25 Et voici, il y avait à Jérusalem un homme appelé Siméon. Cet homme était juste et pieux, il attendait la consolation d'Israël, et l'Esprit Saint était sur lui.
\par 26 Il avait été divinement averti par le Saint Esprit qu'il ne mourrait point avant d'avoir vu le Christ du Seigneur.
\par 27 Il vint au temple, poussé par l'Esprit. Et, comme les parents apportaient le petit enfant Jésus pour accomplir à son égard ce qu'ordonnait la loi,
\par 28 il le reçut dans ses bras, bénit Dieu, et dit:
\par 29 Maintenant, Seigneur, tu laisses ton serviteur S'en aller en paix, selon ta parole.
\par 30 Car mes yeux ont vu ton salut,
\par 31 Salut que tu as préparé devant tous les peuples,
\par 32 Lumière pour éclairer les nations, Et gloire d'Israël, ton peuple.
\par 33 Son père et sa mère étaient dans l'admiration des choses qu'on disait de lui.
\par 34 Siméon les bénit, et dit à Marie, sa mère: Voici, cet enfant est destiné à amener la chute et le relèvement de plusieurs en Israël, et à devenir un signe qui provoquera la contradiction,
\par 35 et à toi-même une épée te transpercera l'âme, afin que les pensées de beaucoup de coeurs soient dévoilées.
\par 36 Il y avait aussi une prophétesse, Anne, fille de Phanuel, de la tribu d'Aser. Elle était fort avancée en âge, et elle avait vécu sept ans avec son mari depuis sa virginité.
\par 37 Restée veuve, et âgée de quatre vingt-quatre ans, elle ne quittait pas le temple, et elle servait Dieu nuit et jour dans le jeûne et dans la prière.
\par 38 Étant survenue, elle aussi, à cette même heure, elle louait Dieu, et elle parlait de Jésus à tous ceux qui attendaient la délivrance de Jérusalem.
\par 39 Lorsqu'ils eurent accompli tout ce qu'ordonnait la loi du Seigneur, Joseph et Marie retournèrent en Galilée, à Nazareth, leur ville.
\par 40 Or, l'enfant croissait et se fortifiait. Il était rempli de sagesse, et la grâce de Dieu était sur lui.
\par 41 Les parents de Jésus allaient chaque année à Jérusalem, à la fête de Pâque.
\par 42 Lorsqu'il fut âgé de douze ans, ils y montèrent, selon la coutume de la fête.
\par 43 Puis, quand les jours furent écoulés, et qu'ils s'en retournèrent, l'enfant Jésus resta à Jérusalem. Son père et sa mère ne s'en aperçurent pas.
\par 44 Croyant qu'il était avec leurs compagnons de voyage, ils firent une journée de chemin, et le cherchèrent parmi leurs parents et leurs connaissances.
\par 45 Mais, ne l'ayant pas trouvé, ils retournèrent à Jérusalem pour le chercher.
\par 46 Au bout de trois jours, ils le trouvèrent dans le temple, assis au milieu des docteurs, les écoutant et les interrogeant.
\par 47 Tous ceux qui l'entendaient étaient frappés de son intelligence et de ses réponses.
\par 48 Quand ses parents le virent, ils furent saisis d'étonnement, et sa mère lui dit: Mon enfant, pourquoi as-tu agi de la sorte avec nous? Voici, ton père et moi, nous te cherchions avec angoisse.
\par 49 Il leur dit: Pourquoi me cherchiez-vous? Ne saviez-vous pas qu'il faut que je m'occupe des affaires de mon Père?
\par 50 Mais ils ne comprirent pas ce qu'il leur disait.
\par 51 Puis il descendit avec eux pour aller à Nazareth, et il leur était soumis. Sa mère gardait toutes ces choses dans son coeur.
\par 52 Et Jésus croissait en sagesse, en stature, et en grâce, devant Dieu et devant les hommes.

\chapter{3}

\par 1 La quinzième année du règne de Tibère César, -lorsque Ponce Pilate était gouverneur de la Judée, Hérode tétrarque de la Galilée, son frère Philippe tétrarque de l'Iturée et du territoire de la Trachonite, Lysanias tétrarque de l'Abilène,
\par 2 et du temps des souverains sacrificateurs Anne et Caïphe, -la parole de Dieu fut adressée à Jean, fils de Zacharie, dans le désert.
\par 3 Et il alla dans tout le pays des environs de Jourdain, prêchant le baptême de repentance, pour la rémission des péchés,
\par 4 selon ce qui est écrit dans le livre des paroles d'Ésaïe, le prophète: C'est la voix de celui qui crie dans le désert: Préparez le chemin du Seigneur, Aplanissez ses sentiers.
\par 5 Toute vallée sera comblée, Toute montagne et toute colline seront abaissées; Ce qui est tortueux sera redressé, Et les chemins raboteux seront aplanis.
\par 6 Et toute chair verra le salut de Dieu.
\par 7 Il disait donc à ceux qui venaient en foule pour être baptisés par lui: Races de vipères, qui vous a appris à fuir la colère à venir?
\par 8 Produisez donc des fruits dignes de la repentance, et ne vous mettez pas à dire en vous-mêmes: Nous avons Abraham pour père! Car je vous déclare que de ces pierres Dieu peut susciter des enfants à Abraham.
\par 9 Déjà même la cognée est mise à la racine des arbres: tout arbre donc qui ne produit pas de bons fruits sera coupé et jeté au feu.
\par 10 La foule l'interrogeait, disant: Que devons-nous donc faire?
\par 11 Il leur répondit: Que celui qui a deux tuniques partage avec celui qui n'en a point, et que celui qui a de quoi manger agisse de même.
\par 12 Il vint aussi des publicains pour être baptisés, et ils lui dirent: Maître, que devons-nous faire?
\par 13 Il leur répondit: N'exigez rien au delà de ce qui vous a été ordonné.
\par 14 Des soldats aussi lui demandèrent: Et nous, que devons-nous faire? Il leur répondit: Ne commettez ni extorsion ni fraude envers personne, et contentez-vous de votre solde.
\par 15 Comme le peuple était dans l'attente, et que tous se demandaient en eux-même si Jean n'était pas le Christ,
\par 16 il leur dit à tous: Moi, je vous baptise d'eau; mais il vient, celui qui est plus puissant que moi, et je ne suis pas digne de délier la courroie de ses souliers. Lui, il vous baptisera du Saint Esprit et de feu.
\par 17 Il a son van à la main; il nettoiera son aire, et il amassera le blé dans son grenier, mais il brûlera la paille dans un feu qui ne s'éteint point.
\par 18 C'est ainsi que Jean annonçait la bonne nouvelle au peuple, en lui adressant encore beaucoup d'autres exhortations.
\par 19 Mais Hérode le tétrarque, étant repris par Jean au sujet d'Hérodias, femme de son frère, et pour toutes les mauvaises actions qu'il avait commises,
\par 20 ajouta encore à toutes les autres celle d'enfermer Jean dans la prison.
\par 21 Tout le peuple se faisant baptiser, Jésus fut aussi baptisé; et, pendant qu'il priait, le ciel s'ouvrit,
\par 22 et le Saint Esprit descendit sur lui sous une forme corporelle, comme une colombe. Et une voix fit entendre du ciel ces paroles: Tu es mon Fils bien-aimé; en toi j'ai mis toute mon affection.
\par 23 Jésus avait environ trente ans lorsqu'il commença son ministère, étant, comme on le croyait, fils de Joseph, fils d'Héli,
\par 24 fils de Matthat, fils de Lévi, fils de Melchi, fils de Jannaï, fils de Joseph,
\par 25 fils de Mattathias, fils d'Amos, fils de Nahum, fils d'Esli, fils de Naggaï,
\par 26 fils de Maath, fils de Mattathias, fils de Sémeï, fils de Josech, fils de Joda,
\par 27 fils de Joanan, fils de Rhésa, fils de Zorobabel, fils de Salathiel, fils de Néri,
\par 28 fils de Melchi, fils d'Addi, fils de Kosam, fils d'Elmadam, fils D'Er,
\par 29 fils de Jésus, fils d'Éliézer, fils de Jorim, fils de Matthat, fils de Lévi,
\par 30 fils de Siméon, fils de Juda, fils de Joseph, fils de Jonam, fils d'Éliakim,
\par 31 fils de Méléa, fils de Menna, fils de Mattatha, fils de Nathan, fils de David,
\par 32 fils d'Isaï, fils de Jobed, fils de Booz, fils de Salmon, fils de Naasson,
\par 33 fils d'Aminadab, fils d'Admin, fils d'Arni, fils d'Esrom, fils de Pharès, fils de Juda,
\par 34 fils de Jacob, fils d'Isaac, fils d'Abraham, fis de Thara, fils de Nachor,
\par 35 fils de Seruch, fils de Ragau, fils de Phalek, fils d'Éber, fils de Sala,
\par 36 fils de Kaïnam, fils d'Arphaxad, fils de Sem, fils de Noé, fils de Lamech,
\par 37 fils de Mathusala, fils d'Énoch, fils de Jared, fils de Maléléel, fils de Kaïnan,
\par 38 fils d'Énos, fils de Seth, fils d'Adam, fils de Dieu.

\chapter{4}

\par 1 Jésus, rempli du Saint Esprit, revint du Jourdain, et il fut conduit par l'Esprit dans le désert,
\par 2 où il fut tenté par le diable pendant quarante jours. Il ne mangea rien durant ces jours-là, et, après qu'ils furent écoulés, il eut faim.
\par 3 Le diable lui dit: Si tu es Fils de Dieu, ordonne à cette pierre qu'elle devienne du pain.
\par 4 Jésus lui répondit: Il est écrit: L'Homme ne vivra pas de pain seulement.
\par 5 Le diable, l'ayant élevé, lui montra en un instant tous les royaumes de la terre,
\par 6 et lui dit: Je te donnerai toute cette puissance, et la gloire de ces royaumes; car elle m'a été donnée, et je la donne à qui je veux.
\par 7 Si donc tu te prosternes devant moi, elle sera toute à toi.
\par 8 Jésus lui répondit: Il est écrit: Tu adoreras le Seigneur, ton Dieu, et tu le serviras lui seul.
\par 9 Le diable le conduisit encore à Jérusalem, le plaça sur le haut du temple, et lui dit: Si tu es Fils de Dieu, jette-toi d'ici en bas; car il est écrit:
\par 10 Il donnera des ordres à ses anges à ton sujet, Afin qu'ils te gardent;
\par 11 et: Ils te porteront sur les mains, De peur que ton pied ne heurte contre une pierre.
\par 12 Jésus lui répondit: Il es dit: Tu ne tenteras point le Seigneur, ton Dieu.
\par 13 Après l'avoir tenté de toutes ces manières, le diable s'éloigna de lui jusqu'à un moment favorable.
\par 14 Jésus, revêtu de la puissance de l'Esprit, retourna en Galilée, et sa renommée se répandit dans tout le pays d'alentour.
\par 15 Il enseignait dans les synagogues, et il était glorifié par tous.
\par 16 Il se rendit à Nazareth, où il avait été élevé, et, selon sa coutume, il entra dans la synagogue le jour du sabbat. Il se leva pour faire la lecture,
\par 17 et on lui remit le livre du prophète Ésaïe. L'ayant déroulé, il trouva l'endroit où il était écrit:
\par 18 L'Esprit du Seigneur est sur moi, Parce qu'il m'a oint pour annoncer une bonne nouvelle aux pauvres; Il m'a envoyé pour guérir ceux qui ont le coeur brisé,
\par 19 Pour proclamer aux captifs la délivrance, Et aux aveugles le recouvrement de la vue, Pour renvoyer libres les opprimés, Pour publier une année de grâce du Seigneur.
\par 20 Ensuite, il roula le livre, le remit au serviteur, et s'assit. Tous ceux qui se trouvaient dans la synagogue avaient les regards fixés sur lui.
\par 21 Alors il commença à leur dire: Aujourd'hui cette parole de l'Écriture, que vous venez d'entendre, est accomplie.
\par 22 Et tous lui rendaient témoignage; ils étaient étonnés des paroles de grâce qui sortaient de sa bouche, et ils disaient: N'est-ce pas le fils de Joseph?
\par 23 Jésus leur dit: Sans doute vous m'appliquerez ce proverbe: Médecin, guéris-toi toi-même; et vous me direz: Fais ici, dans ta patrie, tout ce que nous avons appris que tu as fait à Capernaüm.
\par 24 Mais, ajouta-t-il, je vous le dis en vérité, aucun prophète n'est bien reçu dans sa patrie.
\par 25 Je vous le dis en vérité: il y avait plusieurs veuves en Israël du temps d'Élie, lorsque le ciel fut fermé trois ans et six mois et qu'il y eut une grande famine sur toute la terre;
\par 26 et cependant Élie ne fut envoyé vers aucune d'elles, si ce n'est vers une femme veuve, à Sarepta, dans le pays de Sidon.
\par 27 Il y avait aussi plusieurs lépreux en Israël du temps d'Élisée, le prophète; et cependant aucun d'eux ne fut purifié, si ce n'est Naaman le Syrien.
\par 28 Ils furent tous remplis de colère dans la synagogue, lorsqu'ils entendirent ces choses.
\par 29 Et s'étant levés, ils le chassèrent de la ville, et le menèrent jusqu'au sommet de la montagne sur laquelle leur ville était bâtie, afin de le précipiter en bas.
\par 30 Mais Jésus, passant au milieu d'eux, s'en alla.
\par 31 Il descendit à Capernaüm, ville de la Galilée; et il enseignait, le jour du sabbat.
\par 32 On était frappé de sa doctrine; car il parlait avec autorité.
\par 33 Il se trouva dans la synagogue un homme qui avait un esprit de démon impur, et qui s'écria d'une voix forte:
\par 34 Ah! qu'y a-t-il entre nous et toi, Jésus de Nazareth? Tu es venu pour nous perdre. Je sais qui tu es: le Saint de Dieu.
\par 35 Jésus le menaça, disant: Tais-toi, et sors de cet homme. Et le démon le jeta au milieu de l'assemblée, et sortit de lui, sans lui faire aucun mal.
\par 36 Tous furent saisis de stupeur, et ils se disaient les uns aux autres: Quelle est cette parole? il commande avec autorité et puissance aux esprits impurs, et ils sortent!
\par 37 Et sa renommée se répandit dans tous les lieux d'alentour.
\par 38 En sortant de la synagogue, il se rendit à la maison de Simon. La belle-mère de Simon avait une violente fièvre, et ils le prièrent en sa faveur.
\par 39 S'étant penché sur elle, il menaça la fièvre, et la fièvre la quitta. A l'instant elle se leva, et les servit.
\par 40 Après le couché du soleil, tous ceux qui avaient des malades atteints de diverses maladies les lui amenèrent. Il imposa les mains à chacun d'eux, et il les guérit.
\par 41 Des démons aussi sortirent de beaucoup de personnes, en criant et en disant: Tu es le Fils de Dieu. Mais il les menaçait et ne leur permettait pas de parler, parce qu'ils savaient qu'il était le Christ.
\par 42 Dès que le jour parut, il sortit et alla dans un lieu désert. Une foule de gens se mirent à sa recherche, et arrivèrent jusqu'à lui; ils voulaient le retenir, afin qu'il ne les quittât point.
\par 43 Mais il leur dit: Il faut aussi que j'annonce aux autres villes la bonne nouvelle du royaume de Dieu; car c'est pour cela que j'ai été envoyé.
\par 44 Et il prêchait dans les synagogues de la Galilée.

\chapter{5}

\par 1 Comme Jésus se trouvait auprès du lac de Génésareth, et que la foule se pressait autour de lui pour entendre la parole de Dieu,
\par 2 il vit au bord du lac deux barques, d'où les pêcheurs étaient descendus pour laver leurs filets.
\par 3 Il monta dans l'une de ces barques, qui était à Simon, et il le pria de s'éloigner un peu de terre. Puis il s'assit, et de la barque il enseignait la foule.
\par 4 Lorsqu'il eut cessé de parler, il dit à Simon: Avance en pleine eau, et jetez vos filets pour pêcher.
\par 5 Simon lui répondit: Maître, nous avons travaillé toute la nuit sans rien prendre; mais, sur ta parole, je jetterai le filet.
\par 6 L'ayant jeté, ils prirent une grande quantité de poissons, et leur filet se rompait.
\par 7 Ils firent signe à leurs compagnons qui étaient dans l'autre barque de venir les aider. Ils vinrent et ils remplirent les deux barques, au point qu'elles enfonçaient.
\par 8 Quand il vit cela, Simon Pierre tomba aux genoux de Jésus, et dit: Seigneur, retire-toi de moi, parce que je suis un homme pécheur.
\par 9 Car l'épouvante l'avait saisi, lui et tous ceux qui étaient avec lui, à cause de la pêche qu'ils avaient faite.
\par 10 Il en était de même de Jacques et de Jean, fils de Zébédée, les associés de Simon. Alors Jésus dit à Simon: Ne crains point; désormais tu seras pêcheur d'hommes.
\par 11 Et, ayant ramené les barques à terre, ils laissèrent tout, et le suivirent.
\par 12 Jésus était dans une des villes; et voici, un homme couvert de lèpre, l'ayant vu, tomba sur sa face, et lui fit cette prière: Seigneur, si tu le veux, tu peux me rendre pur.
\par 13 Jésus étendit la main, le toucha, et dit: Je le veux, sois pur. Aussitôt la lèpre le quitta.
\par 14 Puis il lui ordonna de n'en parler à personne. Mais, dit-il, va te montrer au sacrificateur, et offre pour ta purification ce que Moïse a prescrit, afin que cela leur serve de témoignage.
\par 15 Sa renommée se répandait de plus en plus, et les gens venaient en foule pour l'entendre et pour être guéris de leurs maladies.
\par 16 Et lui, il se retirait dans les déserts, et priait.
\par 17 Un jour Jésus enseignait. Des pharisiens et des docteurs de la loi étaient là assis, venus de tous les villages de la Galilée, de la Judée et de Jérusalem; et la puissance du Seigneur se manifestait par des guérisons.
\par 18 Et voici, des gens, portant sur un lit un homme qui était paralytique, cherchaient à le faire entrer et à le placer sous ses regards.
\par 19 Comme ils ne savaient par où l'introduire, à cause de la foule, ils montèrent sur le toit, et ils le descendirent par une ouverture, avec son lit, au milieu de l'assemblée, devant Jésus.
\par 20 Voyant leur foi, Jésus dit: Homme, tes péchés te sont pardonnés.
\par 21 Les scribes et les pharisiens se mirent à raisonner et à dire: Qui est celui-ci, qui profère des blasphèmes? Qui peut pardonner les péchés, si ce n'est Dieu seul?
\par 22 Jésus, connaissant leurs pensées, prit la parole et leur dit: Quelles pensées avez-vous dans vos coeurs?
\par 23 Lequel est le plus aisé, de dire: Tes péchés te sont pardonnés, ou de dire: Lève-toi, et marche?
\par 24 Or, afin que vous sachiez que le Fils de l'homme a sur la terre le pouvoir de pardonner les péchés: Je te l'ordonne, dit-il au paralytique, lève-toi, prends ton lit, et va dans ta maison.
\par 25 Et, à l'instant, il se leva en leur présence, prit le lit sur lequel il était couché, et s'en alla dans sa maison, glorifiant Dieu.
\par 26 Tous étaient dans l'étonnement, et glorifiaient Dieu; remplis de crainte, ils disaient: Nous avons vu aujourd'hui des choses étranges.
\par 27 Après cela, Jésus sortit, et il vit un publicain, nommé Lévi, assis au lieu des péages. Il lui dit: Suis-moi.
\par 28 Et, laissant tout, il se leva, et le suivit.
\par 29 Lévi lui donna un grand festin dans sa maison, et beaucoup de publicains et d'autres personnes étaient à table avec eux.
\par 30 Les pharisiens et les scribes murmurèrent, et dirent à ses disciples: Pourquoi mangez-vous et buvez-vous avec les publicains et les gens de mauvaise vie?
\par 31 Jésus, prenant la parole, leur dit: Ce ne sont pas ceux qui se portent bien qui ont besoin de médecin, mais les malades.
\par 32 Je ne suis pas venu appeler à la repentance des justes, mais des pécheurs.
\par 33 Ils lui dirent: Les disciples de Jean, comme ceux des pharisiens, jeûnent fréquemment et font des prières, tandis que les tiens mangent et boivent.
\par 34 Il leur répondit: Pouvez-vous faire jeûner les amis de l'époux pendant que l'époux est avec eux?
\par 35 Les jours viendront où l'époux leur sera enlevé, alors ils jeûneront en ces jours-là.
\par 36 Il leur dit aussi une parabole: Personne ne déchire d'un habit neuf un morceau pour le mettre à un vieil habit; car, il déchire l'habit neuf, et le morceau qu'il en a pris n'est pas assorti au vieux.
\par 37 Et personne ne met du vin nouveau dans de vieilles outres; autrement, le vin nouveau fait rompre les outres, il se répand, et les outres sont perdues;
\par 38 mais il faut mettre le vin nouveau dans des outres neuves.
\par 39 Et personne, après avoir bu du vin vieux, ne veut du nouveau, car il dit: Le vieux est bon.

\chapter{6}

\par 1 Il arriva, un jour de sabbat appelé second-premier, que Jésus traversait des champs de blé. Ses disciples arrachaient des épis et les mangeaient, après les avoir froissés dans leurs mains.
\par 2 Quelques pharisiens leur dirent: Pourquoi faites-vous ce qu'il n'est pas permis de faire pendant le sabbat?
\par 3 Jésus leur répondit: N'avez-vous pas lu ce que fit David, lorsqu'il eut faim, lui et ceux qui étaient avec lui;
\par 4 comment il entra dans la maison de Dieu, prit les pains de proposition, en mangea, et en donna à ceux qui étaient avec lui, bien qu'il ne soit permis qu'aux sacrificateurs de les manger?
\par 5 Et il leur dit: Le Fils de l'homme est maître même du sabbat.
\par 6 Il arriva, un autre jour de sabbat, que Jésus entra dans la synagogue, et qu'il enseignait. Il s'y trouvait un homme dont la main droite était sèche.
\par 7 Les scribes et les pharisiens observaient Jésus, pour voir s'il ferait une guérison le jour du sabbat: c'était afin d'avoir sujet de l'accuser.
\par 8 Mais il connaissait leurs pensées, et il dit à l'homme qui avait la main sèche: Lève-toi, et tiens-toi là au milieu. Il se leva, et se tint debout.
\par 9 Et Jésus leur dit: Je vous demande s'il est permis, le jour du sabbat, de faire du bien ou de faire du mal, de sauver une personne ou de la tuer.
\par 10 Alors, promenant ses regards sur eux tous, il dit à l'homme: Étends ta main. Il le fit, et sa main fut guérie.
\par 11 Ils furent remplis de fureur, et ils se consultèrent pour savoir ce qu'ils feraient à Jésus.
\par 12 En ce temps-là, Jésus se rendit sur la montagne pour prier, et il passa toute la nuit à prier Dieu.
\par 13 Quand le jour parut, il appela ses disciples, et il en choisit douze, auxquels il donna le nom d'apôtres:
\par 14 Simon, qu'il nomma Pierre; André, son frère; Jacques; Jean; Philippe; Barthélemy;
\par 15 Matthieu; Thomas; Jacques, fils d'Alphée; Simon, appelé le zélote;
\par 16 Jude, fils de Jacques; et Judas Iscariot, qui devint traître.
\par 17 Il descendit avec eux, et s'arrêta sur un plateau, où se trouvaient une foule de ses disciples et une multitude de peuple de toute la Judée, de Jérusalem, et de la contrée maritime de Tyr et de Sidon. Ils étaient venus pour l'entendre, et pour être guéris de leurs maladies.
\par 18 Ceux qui étaient tourmentés par des esprits impurs étaient guéris.
\par 19 Et toute la foule cherchait à le toucher, parce qu'une force sortait de lui et les guérissait tous.
\par 20 Alors Jésus, levant les yeux sur ses disciples, dit: Heureux vous qui êtes pauvres, car le royaume de Dieu est à vous!
\par 21 Heureux vous qui avez faim maintenant, car vous serez rassasiés! Heureux vous qui pleurez maintenant, car vous serez dans la joie!
\par 22 Heureux serez-vous, lorsque les hommes vous haïront, lorsqu'on vous chassera, vous outragera, et qu'on rejettera votre nom comme infâme, à cause du Fils de l'homme!
\par 23 Réjouissez-vous en ce jour-là et tressaillez d'allégresse, parce que votre récompense sera grande dans le ciel; car c'est ainsi que leurs pères traitaient les prophètes.
\par 24 Mais, malheur à vous, riches, car vous avez votre consolation!
\par 25 Malheur à vous qui êtes rassasiés, car vous aurez faim! Malheur à vous qui riez maintenant, car vous serez dans le deuil et dans les larmes!
\par 26 Malheur, lorsque tous les hommes diront du bien de vous, car c'est ainsi qu'agissaient leurs pères à l'égard des faux prophètes!
\par 27 Mais je vous dis, à vous qui m'écoutez: Aimez vos ennemis, faites du bien à ceux qui vous haïssent,
\par 28 bénissez ceux qui vous maudissent, priez pour ceux qui vous maltraitent.
\par 29 Si quelqu'un te frappe sur une joue, présente-lui aussi l'autre. Si quelqu'un prend ton manteau, ne l'empêche pas de prendre encore ta tunique.
\par 30 Donne à quiconque te demande, et ne réclame pas ton bien à celui qui s'en empare.
\par 31 Ce que vous voulez que les hommes fassent pour vous, faites-le de même pour eux.
\par 32 Si vous aimez ceux qui vous aiment, quel gré vous en saura-t-on? Les pécheurs aussi aiment ceux qui les aiment.
\par 33 Si vous faites du bien à ceux qui vous font du bien, quel gré vous en saura-t-on? Les pécheurs aussi agissent de même.
\par 34 Et si vous prêtez à ceux de qui vous espérez recevoir, quel gré vous en saura-t-on? Les pécheurs aussi prêtent aux pécheurs, afin de recevoir la pareille.
\par 35 Mais aimez vos ennemis, faites du bien, et prêtez sans rien espérer. Et votre récompense sera grande, et vous serez fils du Très Haut, car il est bon pour les ingrats et pour les méchants.
\par 36 Soyez donc miséricordieux, comme votre Père est miséricordieux.
\par 37 Ne jugez point, et vous ne serez point jugés; ne condamnez point, et vous ne serez point condamnés; absolvez, et vous serez absous.
\par 38 Donnez, et il vous sera donné: on versera dans votre sein une bonne mesure, serrée, secouée et qui déborde; car on vous mesurera avec la mesure dont vous vous serez servis.
\par 39 Il leur dit aussi cette parabole: Un aveugle peut-il conduire un aveugle? Ne tomberont-ils pas tous deux dans une fosse?
\par 40 Le disciple n'est pas plus que le maître; mais tout disciple accompli sera comme son maître.
\par 41 Pourquoi vois-tu la paille qui est dans l'oeil de ton frère, et n'aperçois-tu pas la poutre qui est dans ton oeil?
\par 42 Ou comment peux-tu dire à ton frère: Frère, laisse-moi ôter la paille qui est dans ton oeil, toi qui ne vois pas la poutre qui est dans le tien? Hypocrite, ôte premièrement la poutre de ton oeil, et alors tu verras comment ôter la paille qui est dans l'oeil de ton frère.
\par 43 Ce n'est pas un bon arbre qui porte du mauvais fruit, ni un mauvais arbre qui porte du bon fruit.
\par 44 Car chaque arbre se connaît à son fruit. On ne cueille pas des figues sur des épines, et l'on ne vendange pas des raisins sur des ronces.
\par 45 L'homme bon tire de bonnes choses du bon trésor de son coeur, et le méchant tire de mauvaises choses de son mauvais trésor; car c'est de l'abondance du coeur que la bouche parle.
\par 46 Pourquoi m'appelez-vous Seigneur, Seigneur! et ne faites-vous pas ce que je dis?
\par 47 Je vous montrerai à qui est semblable tout homme qui vient à moi, entend mes paroles, et les met en pratique.
\par 48 Il est semblable à un homme qui, bâtissant une maison, a creusé, creusé profondément, et a posé le fondement sur le roc. Une inondation est venue, et le torrent s'est jeté contre cette maison, sans pouvoir l'ébranler, parce qu'elle était bien bâtie.
\par 49 Mais celui qui entend, et ne met pas en pratique, est semblable à un homme qui a bâti une maison sur la terre, sans fondement. Le torrent s'est jeté contre elle: aussitôt elle est tombée, et la ruine de cette maison a été grande.

\chapter{7}

\par 1 Après avoir achevé tous ces discours devant le peuple qui l'écoutait, Jésus entra dans Capernaüm.
\par 2 Un centenier avait un serviteur auquel il était très attaché, et qui se trouvait malade, sur le point de mourir.
\par 3 Ayant entendu parler de Jésus, il lui envoya quelques anciens des Juifs, pour le prier de venir guérir son serviteur.
\par 4 Ils arrivèrent auprès de Jésus, et lui adressèrent d'instantes supplications, disant: Il mérite que tu lui accordes cela;
\par 5 car il aime notre nation, et c'est lui qui a bâti notre synagogue.
\par 6 Jésus, étant allé avec eux, n'était guère éloigné de la maison, quand le centenier envoya des amis pour lui dire: Seigneur, ne prends pas tant de peine; car je ne suis pas digne que tu entres sous mon toit.
\par 7 C'est aussi pour cela que je ne me suis pas cru digne d'aller en personne vers toi. Mais dis un mot, et mon serviteur sera guéri.
\par 8 Car, moi qui suis soumis à des supérieurs, j'ai des soldats sous mes ordres; et je dis à l'un: Va! et il va; à l'autre: Viens! et il vient; et à mon serviteur: Fais cela! et il le fait.
\par 9 Lorsque Jésus entendit ces paroles, il admira le centenier, et, se tournant vers la foule qui le suivait, il dit: Je vous le dis, même en Israël je n'ai pas trouvé une aussi grande foi.
\par 10 De retour à la maison, les gens envoyés par le centenier trouvèrent guéri le serviteur qui avait été malade.
\par 11 Le jour suivant, Jésus alla dans une ville appelée Naïn; ses disciples et une grande foule faisaient route avec lui.
\par 12 Lorsqu'il fut près de la porte de la ville, voici, on portait en terre un mort, fils unique de sa mère, qui était veuve; et il y avait avec elle beaucoup de gens de la ville.
\par 13 Le Seigneur, l'ayant vue, fut ému de compassion pour elle, et lui dit: Ne pleure pas!
\par 14 Il s'approcha, et toucha le cercueil. Ceux qui le portaient s'arrêtèrent. Il dit: Jeune homme, je te le dis, lève-toi!
\par 15 Et le mort s'assit, et se mit à parler. Jésus le rendit à sa mère.
\par 16 Tous furent saisis de crainte, et ils glorifiaient Dieu, disant: Un grand prophète a paru parmi nous, et Dieu a visité son peuple.
\par 17 Cette parole sur Jésus se répandit dans toute la Judée et dans tout le pays d'alentour.
\par 18 Jean fut informé de toutes ces choses par ses disciples.
\par 19 Il en appela deux, et les envoya vers Jésus, pour lui dire: Es-tu celui qui doit venir, ou devons-nous en attendre un autre?
\par 20 Arrivés auprès de Jésus, ils dirent: Jean Baptiste nous a envoyés vers toi, pour dire: Es-tu celui qui doit venir, ou devons-nous en attendre un autre?
\par 21 A l'heure même, Jésus guérit plusieurs personnes de maladies, d'infirmités, et d'esprits malins, et il rendit la vue à plusieurs aveugles.
\par 22 Et il leur répondit: Allez rapporter à Jean ce que vous avez vu et entendu: les aveugles voient, les boiteux marchent, les lépreux sont purifiés, les sourds entendent, les morts ressuscitent, la bonne nouvelle est annoncée aux pauvres.
\par 23 Heureux celui pour qui je ne serai pas une occasion de chute!
\par 24 Lorsque les envoyés de Jean furent partis, Jésus se mit à dire à la foule, au sujet de Jean: Qu'êtes-vous allés voir au désert? un roseau agité par le vent?
\par 25 Mais, qu'êtes-vous allés voir? un homme vêtu d'habits précieux? Voici, ceux qui portent des habits magnifiques, et qui vivent dans les délices, sont dans les maisons des rois.
\par 26 Qu'êtes-vous donc allés voir? un prophète? Oui, vous dis-je, et plus qu'un prophète.
\par 27 C'est celui dont il est écrit: Voici, j'envoie mon messager devant ta face, Pour préparer ton chemin devant toi.
\par 28 Je vous le dis, parmi ceux qui sont nés de femmes, il n'y en a point de plus grand que Jean. Cependant, le plus petit dans le royaume de Dieu est plus grand que lui.
\par 29 Et tout le peuple qui l'a entendu et même les publicains ont justifié Dieu, en se faisant baptiser du baptême de Jean;
\par 30 mais les pharisiens et les docteurs de la loi, en ne se faisant pas baptiser par lui, ont rendu nul à leur égard le dessein de Dieu.
\par 31 A qui donc comparerai-je les hommes de cette génération, et à qui ressemblent-ils?
\par 32 Ils ressemblent aux enfants assis dans la place publique, et qui, se parlant les uns aux autres, disent: Nous vous avons joué de la flûte, et vous n'avez pas dansé; nous vous avons chanté des complaintes, et vous n'avez pas pleuré.
\par 33 Car Jean Baptiste est venu, ne mangeant pas de pain et ne buvant pas de vin, et vous dites: Il a un démon.
\par 34 Le Fils de l'homme est venu, mangeant et buvant, et vous dites: C'est un mangeur et un buveur, un ami des publicains et des gens de mauvaise vie.
\par 35 Mais la sagesse a été justifiée par tous ses enfants.
\par 36 Un pharisien pria Jésus de manger avec lui. Jésus entra dans la maison du pharisien, et se mit à table.
\par 37 Et voici, une femme pécheresse qui se trouvait dans la ville, ayant su qu'il était à table dans la maison du pharisien, apporta un vase d'albâtre plein de parfum,
\par 38 et se tint derrière, aux pieds de Jésus. Elle pleurait; et bientôt elle lui mouilla les pieds de ses larmes, puis les essuya avec ses cheveux, les baisa, et les oignit de parfum.
\par 39 Le pharisien qui l'avait invité, voyant cela, dit en lui-même: Si cet homme était prophète, il connaîtrait qui et de quelle espèce est la femme qui le touche, il connaîtrait que c'est une pécheresse.
\par 40 Jésus prit la parole, et lui dit: Simon, j'ai quelque chose à te dire. -Maître, parle, répondit-il. -
\par 41 Un créancier avait deux débiteurs: l'un devait cinq cents deniers, et l'autre cinquante.
\par 42 Comme ils n'avaient pas de quoi payer, il leur remit à tous deux leur dette. Lequel l'aimera le plus?
\par 43 Simon répondit: Celui, je pense, auquel il a le plus remis. Jésus lui dit: Tu as bien jugé.
\par 44 Puis, se tournant vers la femme, il dit à Simon: Vois-tu cette femme? Je suis entré dans ta maison, et tu ne m'as point donné d'eau pour laver mes pieds; mais elle, elle les a mouillés de ses larmes, et les a essuyés avec ses cheveux.
\par 45 Tu ne m'as point donné de baiser; mais elle, depuis que je suis entré, elle n'a point cessé de me baiser les pieds.
\par 46 Tu n'as point versé d'huile sur ma tête; mais elle, elle a versé du parfum sur mes pieds.
\par 47 C'est pourquoi, je te le dis, ses nombreux péchés ont été pardonnés: car elle a beaucoup aimé. Mais celui à qui on pardonne peu aime peu.
\par 48 Et il dit à la femme: Tes péchés sont pardonnés.
\par 49 Ceux qui étaient à table avec lui se mirent à dire en eux-mêmes: Qui est celui-ci, qui pardonne même les péchés?
\par 50 Mais Jésus dit à la femme: Ta foi t'a sauvée, va en paix.

\chapter{8}

\par 1 Ensuite, Jésus allait de ville en ville et de village en village, prêchant et annonçant la bonne nouvelle du royaume de Dieu.
\par 2 Les douze étaient avec de lui et quelques femmes qui avaient été guéries d'esprits malins et de maladies: Marie, dite de Magdala, de laquelle étaient sortis sept démons,
\par 3 Jeanne, femme de Chuza, intendant d'Hérode, Susanne, et plusieurs autres, qui l'assistaient de leurs biens.
\par 4 Une grande foule s'étant assemblée, et des gens étant venus de diverses villes auprès de lui, il dit cette parabole:
\par 5 Un semeur sortit pour semer sa semence. Comme il semait, une partie de la semence tomba le long du chemin: elle fut foulée aux pieds, et les oiseaux du ciel la mangèrent.
\par 6 Une autre partie tomba sur le roc: quand elle fut levée, elle sécha, parce qu'elle n'avait point d'humidité.
\par 7 Une autre partie tomba au milieu des épines: les épines crûrent avec elle, et l'étouffèrent.
\par 8 Une autre partie tomba dans la bonne terre: quand elle fut levée, elle donna du fruit au centuple. Après avoir ainsi parlé, Jésus dit à haute voix: Que celui qui a des oreilles pour entendre entende!
\par 9 Ses disciples lui demandèrent ce que signifiait cette parabole.
\par 10 Il répondit: Il vous a été donné de connaître les mystères du royaume de Dieu; mais pour les autres, cela leur est dit en paraboles, afin qu'en voyant ils ne voient point, et qu'en entendant ils ne comprennent point.
\par 11 Voici ce que signifie cette parabole: La semence, c'est la parole de Dieu.
\par 12 Ceux qui sont le long du chemin, ce sont ceux qui entendent; puis le diable vient, et enlève de leur coeur la parole, de peur qu'ils ne croient et soient sauvés.
\par 13 Ceux qui sont sur le roc, ce sont ceux qui, lorsqu'ils entendent la parole, la reçoivent avec joie; mais ils n'ont point de racine, ils croient pour un temps, et ils succombent au moment de la tentation.
\par 14 Ce qui est tombé parmi les épines, ce sont ceux qui, ayant entendu la parole, s'en vont, et la laissent étouffer par les soucis, les richesses et les plaisirs de la vie, et ils ne portent point de fruit qui vienne à maturité.
\par 15 Ce qui est tombé dans la bonne terre, ce sont ceux qui, ayant entendu la parole avec un coeur honnête et bon, la retiennent, et portent du fruit avec persévérance.
\par 16 Personne, après avoir allumé une lampe, ne la couvre d'un vase, ou ne la met sous un lit; mais il la met sur un chandelier, afin que ceux qui entrent voient la lumière.
\par 17 Car il n'est rien de caché qui ne doive être découvert, rien de secret qui ne doive être connu et mis au jour.
\par 18 Prenez donc garde à la manière dont vous écoutez; car on donnera à celui qui a, mais à celui qui n'a pas on ôtera même ce qu'il croit avoir.
\par 19 La mère et les frères de Jésus vinrent le trouver; mais ils ne purent l'aborder, à cause de la foule.
\par 20 On lui dit: Ta mère et tes frères sont dehors, et ils désirent te voir.
\par 21 Mais il répondit: Ma mère et mes frères, ce sont ceux qui écoutent la parole de Dieu, et qui la mettent en pratique.
\par 22 Un jour, Jésus monta dans une barque avec ses disciples. Il leur dit: Passons de l'autre côté du lac. Et ils partirent.
\par 23 Pendant qu'ils naviguaient, Jésus s'endormit. Un tourbillon fondit sur le lac, la barque se remplissait d'eau, et ils étaient en péril.
\par 24 Ils s'approchèrent et le réveillèrent, en disant: Maître, maître, nous périssons! S'étant réveillé, il menaça le vent et les flots, qui s'apaisèrent, et le calme revint.
\par 25 Puis il leur dit: Où est votre foi? Saisis de frayeur et d'étonnement, ils se dirent les uns aux autres: Quel est donc celui-ci, qui commande même au vent et à l'eau, et à qui ils obéissent?
\par 26 Ils abordèrent dans le pays des Géraséniens, qui est vis-à-vis de la Galilée.
\par 27 Lorsque Jésus fut descendu à terre, il vint au-devant de lui un homme de la ville, qui était possédé de plusieurs démons. Depuis longtemps il ne portait point de vêtement, et avait sa demeure non dans une maison, mais dans les sépulcres.
\par 28 Ayant vu Jésus, il poussa un cri, se jeta à ses pieds, et dit d'une voix forte: Qu'y a-t-il entre moi et toi, Jésus, Fils du Dieu Très Haut? Je t'en supplie, ne me tourmente pas.
\par 29 Car Jésus commandait à l'esprit impur de sortir de cet homme, dont il s'était emparé depuis longtemps; on le gardait lié de chaînes et les fers aux pieds, mais il rompait les liens, et il était entraîné par le démon dans les déserts.
\par 30 Jésus lui demanda: Quel est ton nom? Légion, répondit-il. Car plusieurs démons étaient entrés en lui.
\par 31 Et ils priaient instamment Jésus de ne pas leur ordonner d'aller dans l'abîme.
\par 32 Il y avait là, dans la montagne, un grand troupeau de pourceaux qui paissaient. Et les démons supplièrent Jésus de leur permettre d'entrer dans ces pourceaux. Il le leur permit.
\par 33 Les démons sortirent de cet homme, entrèrent dans les pourceaux, et le troupeau se précipita des pentes escarpées dans le lac, et se noya.
\par 34 Ceux qui les faisaient paître, voyant ce qui était arrivé, s'enfuirent, et répandirent la nouvelle dans la ville et dans les campagnes.
\par 35 Les gens allèrent voir ce qui était arrivé. Ils vinrent auprès de Jésus, et ils trouvèrent l'homme de qui étaient sortis les démons, assis à ses pieds, vêtu, et dans son bon sens; et ils furent saisis de frayeur.
\par 36 Ceux qui avaient vu ce qui s'était passé leur racontèrent comment le démoniaque avait été guéri.
\par 37 Tous les habitants du pays des Géraséniens prièrent Jésus de s'éloigner d'eux, car ils étaient saisis d'une grande crainte. Jésus monta dans la barque, et s'en retourna.
\par 38 L'homme de qui étaient sortis les démons lui demandait la permission de rester avec lui. Mais Jésus le renvoya, en disant:
\par 39 Retourne dans ta maison, et raconte tout ce que Dieu t'a fait. Il s'en alla, et publia par toute la ville tout ce que Jésus avait fait pour lui.
\par 40 A son retour, Jésus fut reçu par la foule, car tous l'attendaient.
\par 41 Et voici, il vint un homme, nommé Jaïrus, qui était chef de la synagogue. Il se jeta à ses pieds, et le supplia d'entrer dans sa maison,
\par 42 parce qu'il avait une fille unique d'environ douze ans qui se mourait. Pendant que Jésus y allait, il était pressé par la foule.
\par 43 Or, il y avait une femme atteinte d'une perte de sang depuis douze ans, et qui avait dépensé tout son bien pour les médecins, sans qu'aucun ait pu la guérir.
\par 44 Elle s'approcha par derrière, et toucha le bord du vêtement de Jésus. Au même instant la perte de sang s'arrêta.
\par 45 Et Jésus dit: Qui m'a touché? Comme tous s'en défendaient, Pierre et ceux qui étaient avec lui dirent: Maître, la foule t'entoure et te presse, et tu dis: Qui m'a touché?
\par 46 Mais Jésus répondit: Quelqu'un m'a touché, car j'ai connu qu'une force était sortie de moi.
\par 47 La femme, se voyant découverte, vint toute tremblante se jeter à ses pieds, et déclara devant tout le peuple pourquoi elle l'avait touché, et comment elle avait été guérie à l'instant.
\par 48 Jésus lui dit: Ma fille, ta foi t'a sauvée; va en paix.
\par 49 Comme il parlait encore, survint de chez le chef de la synagogue quelqu'un disant: Ta fille est morte; n'importune pas le maître.
\par 50 Mais Jésus, ayant entendu cela, dit au chef de la synagogue: Ne crains pas, crois seulement, et elle sera sauvée.
\par 51 Lorsqu'il fut arrivé à la maison, il ne permit à personne d'entrer avec lui, si ce n'est à Pierre, à Jean et à Jacques, et au père et à la mère de l'enfant.
\par 52 Tous pleuraient et se lamentaient sur elle. Alors Jésus dit: Ne pleurez pas; elle n'est pas morte, mais elle dort.
\par 53 Et ils se moquaient de lui, sachant qu'elle était morte.
\par 54 Mais il la saisit par la main, et dit d'une voix forte: Enfant, lève-toi.
\par 55 Et son esprit revint en elle, et à l'instant elle se leva; et Jésus ordonna qu'on lui donnât à manger.
\par 56 Les parents de la jeune fille furent dans l'étonnement, et il leur recommanda de ne dire à personne ce qui était arrivé.

\chapter{9}

\par 1 Jésus, ayant assemblé les douze, leur donna force et pouvoir sur tous les démons, avec la puissance de guérir les maladies.
\par 2 Il les envoya prêcher le royaume de Dieu, et guérir les malades.
\par 3 Ne prenez rien pour le voyage, leur dit-il, ni bâton, ni sac, ni pain, ni argent, et n'ayez pas deux tuniques.
\par 4 Dans quelque maison que vous entriez, restez-y; et c'est de là que vous partirez.
\par 5 Et, si les gens ne vous reçoivent pas, sortez de cette ville, et secouez la poussière de vos pieds, en témoignage contre eux.
\par 6 Ils partirent, et ils allèrent de village en village, annonçant la bonne nouvelle et opérant partout des guérisons.
\par 7 Hérode le tétrarque entendit parler de tout ce qui se passait, et il ne savait que penser. Car les uns disaient que Jean était ressuscité des morts;
\par 8 d'autres, qu'Élie était apparu; et d'autres, qu'un des anciens prophètes était ressuscité.
\par 9 Mais Hérode disait: J'ai fait décapiter Jean; qui donc est celui-ci, dont j'entends dire de telles choses? Et il cherchait à le voir.
\par 10 Les apôtres, étant de retour, racontèrent à Jésus tout ce qu'ils avaient fait. Il les prit avec lui, et se retira à l'écart, du côté d'une ville appelée Bethsaïda.
\par 11 Les foules, l'ayant su, le suivirent. Jésus les accueillit, et il leur parlait du royaume de Dieu; il guérit aussi ceux qui avaient besoin d'être guéris.
\par 12 Comme le jour commençait à baisser, les douze s'approchèrent, et lui dirent: Renvoie la foule, afin qu'elle aille dans les villages et dans les campagnes des environs, pour se loger et pour trouver des vivres; car nous sommes ici dans un lieu désert.
\par 13 Jésus leur dit: Donnez-leur vous-mêmes à manger. Mais ils répondirent: Nous n'avons que cinq pains et deux poissons, à moins que nous n'allions nous-mêmes acheter des vivres pour tout ce peuple.
\par 14 Or, il y avait environ cinq mille hommes. Jésus dit à ses disciples: Faites-les asseoir par rangées de cinquante.
\par 15 Ils firent ainsi, ils les firent tous asseoir.
\par 16 Jésus prit les cinq pains et les deux poissons, et, levant les yeux vers le ciel, il les bénit. Puis, il les rompit, et les donna aux disciples, afin qu'ils les distribuassent à la foule.
\par 17 Tous mangèrent et furent rassasiés, et l'on emporta douze paniers pleins des morceaux qui restaient.
\par 18 Un jour que Jésus priait à l'écart, ayant avec lui ses disciples, il leur posa cette question: Qui dit-on que je suis?
\par 19 Ils répondirent: Jean Baptiste; les autres, Élie; les autres, qu'un des anciens prophètes est ressuscité.
\par 20 Et vous, leur demanda-t-il, qui dites-vous que je suis? Pierre répondit: Le Christ de Dieu.
\par 21 Jésus leur recommanda sévèrement de ne le dire à personne.
\par 22 Il ajouta qu'il fallait que le Fils de l'homme souffrît beaucoup, qu'il fût rejeté par les anciens, par les principaux sacrificateurs et par les scribes, qu'il fût mis à mort, et qu'il ressuscitât le troisième jour.
\par 23 Puis il dit à tous: Si quelqu'un veut venir après moi, qu'il renonce à lui-même, qu'il se charge chaque jour de sa croix, et qu'il me suive.
\par 24 Car celui qui voudra sauver sa vie la perdra, mais celui qui la perdra à cause de moi la sauvera.
\par 25 Et que servirait-il à un homme de gagner tout le monde, s'il se détruisait ou se perdait lui-même?
\par 26 Car quiconque aura honte de moi et de mes paroles, le Fils de l'homme aura honte de lui, quand il viendra dans sa gloire, et dans celle du Père et des saints anges.
\par 27 Je vous le dis en vérité, quelques-uns de ceux qui sont ici ne mourront point qu'ils n'aient vu le royaume de Dieu.
\par 28 Environ huit jours après qu'il eut dit ces paroles, Jésus prit avec lui Pierre, Jean et Jacques, et il monta sur la montagne pour prier.
\par 29 Pendant qu'il priait, l'aspect de son visage changea, et son vêtement devint d'une éclatante blancheur.
\par 30 Et voici, deux hommes s'entretenaient avec lui: c'étaient Moïse et Élie,
\par 31 qui, apparaissant dans la gloire, parlaient de son départ qu'il allait accomplir à Jérusalem.
\par 32 Pierre et ses compagnons étaient appesantis par le sommeil; mais, s'étant tenus éveillés, ils virent la gloire de Jésus et les deux hommes qui étaient avec lui.
\par 33 Au moment où ces hommes se séparaient de Jésus, Pierre lui dit: Maître, il est bon que nous soyons ici; dressons trois tentes, une pour toi, une pour Moïse, et une pour Élie. Il ne savait ce qu'il disait.
\par 34 Comme il parlait ainsi, une nuée vint les couvrir; et les disciples furent saisis de frayeur en les voyant entrer dans la nuée.
\par 35 Et de la nuée sortit une voix, qui dit: Celui-ci est mon Fils élu: écoutez-le!
\par 36 Quand la voix se fit entendre, Jésus se trouva seul. Les disciples gardèrent le silence, et ils ne racontèrent à personne, en ce temps-là, rien de ce qu'ils avaient vu.
\par 37 Le lendemain, lorsqu'ils furent descendus de la montagne, une grande foule vint au-devant de Jésus.
\par 38 Et voici, du milieu de la foule un homme s'écria: Maître, je t'en prie, porte les regards sur mon fils, car c'est mon fils unique.
\par 39 Un esprit le saisit, et aussitôt il pousse des cris; et l'esprit l'agite avec violence, le fait écumer, et a de la peine à se retirer de lui, après l'avoir tout brisé.
\par 40 J'ai prié tes disciples de le chasser, et ils n'ont pas pu.
\par 41 Race incrédule et perverse, répondit Jésus, jusqu'à quand serai-je avec vous, et vous supporterai-je? Amène ici ton fils.
\par 42 Comme il approchait, le démon le jeta par terre, et l'agita avec violence. Mais Jésus menaça l'esprit impur, guérit l'enfant, et le rendit à son père.
\par 43 Et tous furent frappés de la grandeur de Dieu. Tandis que chacun était dans l'admiration de tout ce que faisait Jésus, il dit à ses disciples:
\par 44 Pour vous, écoutez bien ceci: Le Fils de l'homme doit être livré entre les mains des hommes.
\par 45 Mais les disciples ne comprenaient pas cette parole; elle était voilée pour eux, afin qu'ils n'en eussent pas le sens; et ils craignaient de l'interroger à ce sujet.
\par 46 Or, une pensée leur vint à l'esprit, savoir lequel d'entre eux était le plus grand.
\par 47 Jésus, voyant la pensée de leur coeur, prit un petit enfant, le plaça près de lui,
\par 48 et leur dit: Quiconque reçoit en mon nom ce petit enfant me reçoit moi-même; et quiconque me reçoit reçoit celui qui m'a envoyé. Car celui qui est le plus petit parmi vous tous, c'est celui-là qui est grand.
\par 49 Jean prit la parole, et dit: Maître, nous avons vu un homme qui chasse des démons en ton nom; et nous l'en avons empêché, parce qu'il ne nous suit pas.
\par 50 Ne l'en empêchez pas, lui répondit Jésus; car qui n'est pas contre vous est pour vous.
\par 51 Lorsque le temps où il devait être enlevé du monde approcha, Jésus prit la résolution de se rendre à Jérusalem.
\par 52 Il envoya devant lui des messagers, qui se mirent en route et entrèrent dans un bourg des Samaritains, pour lui préparer un logement.
\par 53 Mais on ne le reçut pas, parce qu'il se dirigeait sur Jérusalem.
\par 54 Les disciples Jacques et Jean, voyant cela, dirent: Seigneur, veux-tu que nous commandions que le feu descende du ciel et les consume?
\par 55 Jésus se tourna vers eux, et les réprimanda, disant: Vous ne savez de quel esprit vous êtes animés.
\par 56 Car le Fils de l'homme est venu, non pour perdre les âmes des hommes, mais pour les sauver. Et ils allèrent dans un autre bourg.
\par 57 Pendant qu'ils étaient en chemin, un homme lui dit: Seigneur, je te suivrai partout où tu iras.
\par 58 Jésus lui répondit: Les renards ont des tanières, et les oiseaux du ciel ont des nids: mais le Fils de l'homme n'a pas un lieu où il puisse reposer sa tête.
\par 59 Il dit à un autre: Suis-moi. Et il répondit: Seigneur, permets-moi d'aller d'abord ensevelir mon père.
\par 60 Mais Jésus lui dit: Laisse les morts ensevelir leurs morts; et toi, va annoncer le royaume de Dieu.
\par 61 Un autre dit: Je te suivrai, Seigneur, mais permets-moi d'aller d'abord prendre congé de ceux de ma maison.
\par 62 Jésus lui répondit: Quiconque met la main à la charrue, et regarde en arrière, n'est pas propre au royaume de Dieu.

\chapter{10}

\par 1 Après cela, le Seigneur désigna encore soixante-dix autres disciples, et il les envoya deux à deux devant lui dans toutes les villes et dans tous les lieux où lui-même devait aller.
\par 2 Il leur dit: La moisson est grande, mais il y a peu d'ouvriers. Priez donc le maître de la moisson d'envoyer des ouvriers dans sa moisson.
\par 3 Partez; voici, je vous envoie comme des agneaux au milieu des loups.
\par 4 Ne portez ni bourse, ni sac, ni souliers, et ne saluez personne en chemin.
\par 5 Dans quelque maison que vous entriez, dites d'abord: Que la paix soit sur cette maison!
\par 6 Et s'il se trouve là un enfant de paix, votre paix reposera sur lui; sinon, elle reviendra à vous.
\par 7 Demeurez dans cette maison-là, mangeant et buvant ce qu'on vous donnera; car l'ouvrier mérite son salaire. N'allez pas de maison en maison.
\par 8 Dans quelque ville que vous entriez, et où l'on vous recevra, mangez ce qui vous sera présenté,
\par 9 guérissez les malades qui s'y trouveront, et dites-leur: Le royaume de Dieu s'est approché de vous.
\par 10 Mais dans quelque ville que vous entriez, et où l'on ne vous recevra pas, allez dans ses rues, et dites:
\par 11 Nous secouons contre vous la poussière même de votre ville qui s'est attachée à nos pieds; sachez cependant que le royaume de Dieu s'est approché.
\par 12 Je vous dis qu'en ce jour Sodome sera traitée moins rigoureusement que cette ville-là.
\par 13 Malheur à toi, Chorazin! malheur à toi, Bethsaïda! car, si les miracles qui ont été faits au milieu de vous avaient été faits dans Tyr et dans Sidon, il y a longtemps qu'elles se seraient repenties, en prenant le sac et la cendre.
\par 14 C'est pourquoi, au jour du jugement, Tyr et Sidon seront traitées moins rigoureusement que vous.
\par 15 Et toi, Capernaüm, qui as été élevée jusqu'au ciel, tu seras abaissée jusqu'au séjour des morts.
\par 16 Celui qui vous écoute m'écoute, et celui qui vous rejette me rejette; et celui qui me rejette rejette celui qui m'a envoyé.
\par 17 Les soixante-dix revinrent avec joie, disant: Seigneur, les démons mêmes nous sont soumis en ton nom.
\par 18 Jésus leur dit: Je voyais Satan tomber du ciel comme un éclair.
\par 19 Voici, je vous ai donné le pouvoir de marcher sur les serpents et les scorpions, et sur toute la puissance de l'ennemi; et rien ne pourra vous nuire.
\par 20 Cependant, ne vous réjouissez pas de ce que les esprits vous sont soumis; mais réjouissez-vous de ce que vos noms sont écrits dans les cieux.
\par 21 En ce moment même, Jésus tressaillit de joie par le Saint Esprit, et il dit: Je te loue, Père, Seigneur du ciel et de la terre, de ce que tu as caché ces choses aux sages et aux intelligents, et de ce que tu les as révélées aux enfants. Oui, Père, je te loue de ce que tu l'as voulu ainsi.
\par 22 Toutes choses m'ont été données par mon Père, et personne ne connaît qui est le Fils, si ce n'est le Père, ni qui est le Père, si ce n'est le Fils et celui à qui le Fils veut le révéler.
\par 23 Et, se tournant vers les disciples, il leur dit en particulier: Heureux les yeux qui voient ce que vous voyez!
\par 24 Car je vous dis que beaucoup de prophètes et de rois ont désiré voir ce que vous voyez, et ne l'ont pas vu, entendre ce que vous entendez, et ne l'ont pas entendu.
\par 25 Un docteur de la loi se leva, et dit à Jésus, pour l'éprouver: Maître, que dois-je faire pour hériter la vie éternelle?
\par 26 Jésus lui dit: Qu'est-il écrit dans la loi? Qu'y lis-tu?
\par 27 Il répondit: Tu aimeras le Seigneur, ton Dieu, de tout ton coeur, de toute ton âme, de toute ta force, et de toute ta pensée; et ton prochain comme toi-même.
\par 28 Tu as bien répondu, lui dit Jésus; fais cela, et tu vivras.
\par 29 Mais lui, voulant se justifier, dit à Jésus: Et qui est mon prochain?
\par 30 Jésus reprit la parole, et dit: Un homme descendait de Jérusalem à Jéricho. Il tomba au milieu des brigands, qui le dépouillèrent, le chargèrent de coups, et s'en allèrent, le laissant à demi mort.
\par 31 Un sacrificateur, qui par hasard descendait par le même chemin, ayant vu cet homme, passa outre.
\par 32 Un Lévite, qui arriva aussi dans ce lieu, l'ayant vu, passa outre.
\par 33 Mais un Samaritain, qui voyageait, étant venu là, fut ému de compassion lorsqu'il le vit.
\par 34 Il s'approcha, et banda ses plaies, en y versant de l'huile et du vin; puis il le mit sur sa propre monture, le conduisit à une hôtellerie, et prit soin de lui.
\par 35 Le lendemain, il tira deux deniers, les donna à l'hôte, et dit: Aie soin de lui, et ce que tu dépenseras de plus, je te le rendrai à mon retour.
\par 36 Lequel de ces trois te semble avoir été le prochain de celui qui était tombé au milieu des brigands?
\par 37 C'est celui qui a exercé la miséricorde envers lui, répondit le docteur de la loi. Et Jésus lui dit: Va, et toi, fais de même.
\par 38 Comme Jésus était en chemin avec ses disciples, il entra dans un village, et une femme, nommée Marthe, le reçut dans sa maison.
\par 39 Elle avait une soeur, nommée Marie, qui, s'étant assise aux pieds du Seigneur, écoutait sa parole.
\par 40 Marthe, occupée à divers soins domestiques, survint et dit: Seigneur, cela ne te fait-il rien que ma soeur me laisse seule pour servir? Dis-lui donc de m'aider.
\par 41 Le Seigneur lui répondit: Marthe, Marthe, tu t'inquiètes et tu t'agites pour beaucoup de choses.
\par 42 Une seule chose est nécessaire. Marie a choisi la bonne part, qui ne lui sera point ôtée.

\chapter{11}

\par 1 Jésus priait un jour en un certain lieu. Lorsqu'il eut achevé, un de ses disciples lui dit: Seigneur, enseigne-nous à prier, comme Jean l'a enseigné à ses disciples.
\par 2 Il leur dit: Quand vous priez, dites: Père! Que ton nom soit sanctifié; que ton règne vienne.
\par 3 Donne-nous chaque jour notre pain quotidien;
\par 4 pardonne-nous nos péchés, car nous aussi nous pardonnons à quiconque nous offense; et ne nous induis pas en tentation.
\par 5 Il leur dit encore: Si l'un de vous a un ami, et qu'il aille le trouver au milieu de la nuit pour lui dire: Ami, prête-moi trois pains,
\par 6 car un de mes amis est arrivé de voyage chez moi, et je n'ai rien à lui offrir,
\par 7 et si, de l'intérieur de sa maison, cet ami lui répond: Ne m'importune pas, la porte est déjà fermée, mes enfants et moi sommes au lit, je ne puis me lever pour te donner des pains, -
\par 8 je vous le dis, même s'il ne se levait pas pour les lui donner parce que c'est son ami, il se lèverait à cause de son importunité et lui donnerait tout ce dont il a besoin.
\par 9 Et moi, je vous dis: Demandez, et l'on vous donnera; cherchez, et vous trouverez; frappez, et l'on vous ouvrira.
\par 10 Car quiconque demande reçoit, celui qui cherche trouve, et l'on ouvre à celui qui frappe.
\par 11 Quel est parmi vous le père qui donnera une pierre à son fils, s'il lui demande du pain? Ou, s'il demande un poisson, lui donnera-t-il un serpent au lieu d'un poisson?
\par 12 Ou, s'il demande un oeuf, lui donnera-t-il un scorpion?
\par 13 Si donc, méchants comme vous l'êtes, vous savez donner de bonnes choses à vos enfants, à combien plus forte raison le Père céleste donnera-t-il le Saint Esprit à ceux qui le lui demandent.
\par 14 Jésus chassa un démon qui était muet. Lorsque le démon fut sorti, le muet parla, et la foule fut dans l'admiration.
\par 15 Mais quelques-uns dirent: c'est par Béelzébul, le prince des démons, qu'il chasse les démons.
\par 16 Et d'autres, pour l'éprouver, lui demandèrent un signe venant du ciel.
\par 17 Comme Jésus connaissait leurs pensées, il leur dit: Tout royaume divisé contre lui-même est dévasté, et une maison s'écroule sur une autre.
\par 18 Si donc Satan est divisé contre lui-même, comment son royaume subsistera-t-il, puisque vous dites que je chasse les démons par Béelzébul?
\par 19 Et si moi, je chasse les démons par Béelzébul, vos fils, par qui les chassent-ils? C'est pourquoi ils seront eux-mêmes vos juges.
\par 20 Mais, si c'est par le doigt de Dieu que je chasse les démons, le royaume de Dieu est donc venu vers vous.
\par 21 Lorsqu'un homme fort et bien armé garde sa maison, ce qu'il possède est en sûreté.
\par 22 Mais, si un plus fort que lui survient et le dompte, il lui enlève toutes les armes dans lesquelles il se confiait, et il distribue ses dépouilles.
\par 23 Celui qui n'est pas avec moi est contre moi, et celui qui n'assemble pas avec moi disperse.
\par 24 Lorsque l'esprit impur est sorti d'un homme, il va dans des lieux arides, pour chercher du repos. N'en trouvant point, il dit: Je retournerai dans ma maison d'où je suis sorti;
\par 25 et, quand il arrive, il la trouve balayée et ornée.
\par 26 Alors il s'en va, et il prend sept autres esprits plus méchants que lui; ils entrent dans la maison, s'y établissent, et la dernière condition de cet homme est pire que la première.
\par 27 Tandis que Jésus parlait ainsi, une femme, élevant la voix du milieu de la foule, lui dit: Heureux le sein qui t'a porté! heureuses les mamelles qui t'ont allaité!
\par 28 Et il répondit: Heureux plutôt ceux qui écoutent la parole de Dieu, et qui la gardent!
\par 29 Comme le peuple s'amassait en foule, il se mit à dire: Cette génération est une génération méchante; elle demande un miracle; il ne lui sera donné d'autre miracle que celui de Jonas.
\par 30 Car, de même que Jonas fut un signe pour les Ninivites, de même le Fils de l'homme en sera un pour cette génération.
\par 31 La reine du Midi se lèvera, au jour du jugement, avec les hommes de cette génération et les condamnera, parce qu'elle vint des extrémités de la terre pour entendre la sagesse de Salomon; et voici, il y a ici plus que Salomon.
\par 32 Les hommes de Ninive se lèveront, au jour du jugement, avec cette génération et la condamneront, parce qu'ils se repentirent à la prédication de Jonas; et voici, il y a ici plus que Jonas.
\par 33 Personne n'allume une lampe pour la mettre dans un lieu caché ou sous le boisseau, mais on la met sur le chandelier, afin que ceux qui entrent voient la lumière.
\par 34 Ton oeil est la lampe de ton corps. Lorsque ton oeil est en bon état, tout ton corps est éclairé; mais lorsque ton oeil est en mauvais état, ton corps est dans les ténèbres.
\par 35 Prends donc garde que la lumière qui est en toi ne soit ténèbres.
\par 36 Si donc tout ton corps est éclairé, n'ayant aucune partie dans les ténèbres, il sera entièrement éclairé, comme lorsque la lampe t'éclaire de sa lumière.
\par 37 Pendant que Jésus parlait, un pharisien le pria de dîner chez lui. Il entra, et se mit à table.
\par 38 Le pharisien vit avec étonnement qu'il ne s'était pas lavé avant le repas.
\par 39 Mais le Seigneur lui dit: Vous, pharisiens, vous nettoyez le dehors de la coupe et du plat, et à l'intérieur vous êtes pleins de rapine et de méchanceté.
\par 40 Insensés! celui qui a fait le dehors n'a-t-il pas fait aussi le dedans?
\par 41 Donnez plutôt en aumônes ce qui est dedans, et voici, toutes choses seront pures pour vous.
\par 42 Mais malheur à vous, pharisiens! parce que vous payez la dîme de la menthe, de la rue, et de toutes les herbes, et que vous négligez la justice et l'amour de Dieu: c'est là ce qu'il fallait pratiquer, sans omettre les autres choses.
\par 43 Malheur à vous, pharisiens! parce que vous aimez les premiers sièges dans les synagogues, et les salutations dans les places publiques.
\par 44 Malheur à vous! parce que vous êtes comme les sépulcres qui ne paraissent pas, et sur lesquels on marche sans le savoir.
\par 45 Un des docteurs de la loi prit la parole, et lui dit: Maître, en parlant de la sorte, c'est aussi nous que tu outrages.
\par 46 Et Jésus répondit: Malheur à vous aussi, docteurs de la loi! parce que vous chargez les hommes de fardeaux difficiles à porter, et que vous ne touchez pas vous-mêmes de l'un de vos doigts.
\par 47 Malheur à vous! parce que vous bâtissez les tombeaux des prophètes, que vos pères ont tués.
\par 48 Vous rendez donc témoignage aux oeuvres de vos pères, et vous les approuvez; car eux, ils ont tué les prophètes, et vous, vous bâtissez leurs tombeaux.
\par 49 C'est pourquoi la sagesse de Dieu a dit: Je leur enverrai des prophètes et des apôtres; ils tueront les uns et persécuteront les autres,
\par 50 afin qu'il soit demandé compte à cette génération du sang de tous les prophètes qui a été répandu depuis la création du monde,
\par 51 depuis le sang d'Abel jusqu'au sang de Zacharie, tué entre l'autel et le temple; oui, je vous le dis, il en sera demandé compte à cette génération.
\par 52 Malheur à vous, docteurs de la loi! parce que vous avez enlevé la clef de la science; vous n'êtes pas entrés vous-mêmes, et vous avez empêché d'entrer ceux qui le voulaient.
\par 53 Quand il fut sorti de là, les scribes et les pharisiens commencèrent à le presser violemment, et à le faire parler sur beaucoup de choses,
\par 54 lui tendant des pièges, pour surprendre quelque parole sortie de sa bouche.

\chapter{12}

\par 1 Sur ces entrefaites, les gens s'étant rassemblés par milliers, au point de se fouler les uns les autres, Jésus se mit à dire à ses disciples: Avant tout, gardez-vous du levain des pharisiens, qui est l'hypocrisie.
\par 2 Il n'y a rien de caché qui ne doive être découvert, ni de secret qui ne doive être connu.
\par 3 C'est pourquoi tout ce que vous aurez dit dans les ténèbres sera entendu dans la lumière, et ce que vous aurez dit à l'oreille dans les chambres sera prêché sur les toits.
\par 4 Je vous dis, à vous qui êtes mes amis: Ne craignez pas ceux qui tuent le corps et qui, après cela, ne peuvent rien faire de plus.
\par 5 Je vous montrerai qui vous devez craindre. Craignez celui qui, après avoir tué, a le pouvoir de jeter dans la géhenne; oui, je vous le dis, c'est lui que vous devez craindre.
\par 6 Ne vend-on pas cinq passereaux pour deux sous? Cependant, aucun d'eux n'est oublié devant Dieu.
\par 7 Et même les cheveux de votre tête sont tous comptés. Ne craignez donc point: vous valez plus que beaucoup de passereaux.
\par 8 Je vous le dis, quiconque me confessera devant les hommes, le Fils de l'homme le confessera aussi devant les anges de Dieu;
\par 9 mais celui qui me reniera devant les hommes sera renié devant les anges de Dieu.
\par 10 Et quiconque parlera contre le Fils de l'homme, il lui sera pardonné; mais à celui qui blasphémera contre le Saint Esprit il ne sera point pardonné.
\par 11 Quand on vous mènera devant les synagogues, les magistrats et les autorités, ne vous inquiétez pas de la manière dont vous vous défendrez ni de ce que vous direz;
\par 12 car le Saint Esprit vous enseignera à l'heure même ce qu'il faudra dire.
\par 13 Quelqu'un dit à Jésus, du milieu de la foule: Maître, dis à mon frère de partager avec moi notre héritage.
\par 14 Jésus lui répondit: O homme, qui m'a établi pour être votre juge, ou pour faire vos partages?
\par 15 Puis il leur dit: Gardez-vous avec soin de toute avarice; car la vie d'un homme ne dépend pas de ses biens, fût-il dans l'abondance.
\par 16 Et il leur dit cette parabole: Les terres d'un homme riche avaient beaucoup rapporté.
\par 17 Et il raisonnait en lui-même, disant: Que ferai-je? car je n'ai pas de place pour serrer ma récolte.
\par 18 Voici, dit-il, ce que je ferai: j'abattrai mes greniers, j'en bâtirai de plus grands, j'y amasserai toute ma récolte et tous mes biens;
\par 19 et je dirai à mon âme: Mon âme, tu as beaucoup de biens en réserve pour plusieurs années; repose-toi, mange, bois, et réjouis-toi.
\par 20 Mais Dieu lui dit: Insensé! cette nuit même ton âme te sera redemandée; et ce que tu as préparé, pour qui cela sera-t-il?
\par 21 Il en est ainsi de celui qui amasse des trésors pour lui-même, et qui n'est pas riche pour Dieu.
\par 22 Jésus dit ensuite à ses disciples: C'est pourquoi je vous dis: Ne vous inquiétez pas pour votre vie de ce que vous mangerez, ni pour votre corps de quoi vous serez vêtus.
\par 23 La vie est plus que la nourriture, et le corps plus que le vêtement.
\par 24 Considérez les corbeaux: ils ne sèment ni ne moissonnent, ils n'ont ni cellier ni grenier; et Dieu les nourrit. Combien ne valez-vous pas plus que les oiseaux!
\par 25 Qui de vous, par ses inquiétudes, peut ajouter une coudée à la durée de sa vie?
\par 26 Si donc vous ne pouvez pas même la moindre chose, pourquoi vous inquiétez-vous du reste?
\par 27 Considérez comment croissent les lis: ils ne travaillent ni ne filent; cependant je vous dis que Salomon même, dans toute sa gloire, n'a pas été vêtu comme l'un d'eux.
\par 28 Si Dieu revêt ainsi l'herbe qui est aujourd'hui dans les champs et qui demain sera jetée au four, à combien plus forte raison ne vous vêtira-t-il pas, gens de peu de foi?
\par 29 Et vous, ne cherchez pas ce que vous mangerez et ce que vous boirez, et ne soyez pas inquiets.
\par 30 Car toutes ces choses, ce sont les païens du monde qui les recherchent. Votre Père sait que vous en avez besoin.
\par 31 Cherchez plutôt le royaume de Dieu; et toutes ces choses vous seront données par-dessus.
\par 32 Ne crains point, petit troupeau; car votre Père a trouvé bon de vous donner le royaume.
\par 33 Vendez ce que vous possédez, et donnez-le en aumônes. Faites-vous des bourses qui ne s'usent point, un trésor inépuisable dans les cieux, où le voleur n'approche point, et où la teigne ne détruit point.
\par 34 Car là où est votre trésor, là aussi sera votre coeur.
\par 35 Que vos reins soient ceints, et vos lampes allumées.
\par 36 Et vous, soyez semblables à des hommes qui attendent que leur maître revienne des noces, afin de lui ouvrir dès qu'il arrivera et frappera.
\par 37 Heureux ces serviteurs que le maître, à son arrivée, trouvera veillant! Je vous le dis en vérité, il se ceindra, les fera mettre à table, et s'approchera pour les servir.
\par 38 Qu'il arrive à la deuxième ou à la troisième veille, heureux ces serviteurs, s'il les trouve veillant!
\par 39 Sachez-le bien, si le maître de la maison savait à quelle heure le voleur doit venir, il veillerait et ne laisserait pas percer sa maison.
\par 40 Vous aussi, tenez-vous prêts, car le Fils de l'homme viendra à l'heure où vous n'y penserez pas.
\par 41 Pierre lui dit: Seigneur, est-ce à nous, ou à tous, que tu adresses cette parabole?
\par 42 Et le Seigneur dit: Quel est donc l'économe fidèle et prudent que le maître établira sur ses gens, pour leur donner la nourriture au temps convenable?
\par 43 Heureux ce serviteur, que son maître, à son arrivée, trouvera faisant ainsi!
\par 44 Je vous le dis en vérité, il l'établira sur tous ses biens.
\par 45 Mais, si ce serviteur dit en lui-même: Mon maître tarde à venir; s'il se met à battre les serviteurs et les servantes, à manger, à boire et à s'enivrer,
\par 46 le maître de ce serviteur viendra le jour où il ne s'y attend pas et à l'heure qu'il ne connaît pas, il le mettra en pièces, et lui donnera sa part avec les infidèles.
\par 47 Le serviteur qui, ayant connu la volonté de son maître, n'a rien préparé et n'a pas agi selon sa volonté, sera battu d'un grand nombre de coups.
\par 48 Mais celui qui, ne l'ayant pas connue, a fait des choses dignes de châtiment, sera battu de peu de coups. On demandera beaucoup à qui l'on a beaucoup donné, et on exigera davantage de celui à qui l'on a beaucoup confié.
\par 49 Je suis venu jeter un feu sur la terre, et qu'ai-je à désirer, s'il est déjà allumé?
\par 50 Il est un baptême dont je dois être baptisé, et combien il me tarde qu'il soit accompli!
\par 51 Pensez-vous que je sois venu apporter la paix sur la terre? Non, vous dis-je, mais la division.
\par 52 Car désormais cinq dans une maison seront divisés, trois contre deux, et deux contre trois;
\par 53 le père contre le fils et le fils contre le père, la mère contre la fille et la fille contre la mère, la belle-mère contre la belle-fille et la belle-fille contre la belle-mère.
\par 54 Il dit encore aux foules: Quand vous voyez un nuage se lever à l'occident, vous dites aussitôt: La pluie vient. Et il arrive ainsi.
\par 55 Et quand vous voyez souffler le vent du midi, vous dites: Il fera chaud. Et cela arrive.
\par 56 Hypocrites! vous savez discerner l'aspect de la terre et du ciel; comment ne discernez-vous pas ce temps-ci?
\par 57 Et pourquoi ne discernez-vous pas de vous-mêmes ce qui est juste?
\par 58 Lorsque tu vas avec ton adversaire devant le magistrat, tâche en chemin de te dégager de lui, de peur qu'il ne te traîne devant le juge, que le juge ne te livre à l'officier de justice, et que celui-ci ne te mette en prison.
\par 59 Je te le dis, tu ne sortiras pas de là que tu n'aies payé jusqu'à dernière pite.

\chapter{13}

\par 1 En ce même temps, quelques personnes qui se trouvaient là racontaient à Jésus ce qui était arrivé à des Galiléens dont Pilate avait mêlé le sang avec celui de leurs sacrifices.
\par 2 Il leur répondit: Croyez-vous que ces Galiléens fussent de plus grands pécheurs que tous les autres Galiléens, parce qu'ils ont souffert de la sorte?
\par 3 Non, je vous le dis. Mais si vous ne vous repentez, vous périrez tous également.
\par 4 Ou bien, ces dix-huit personnes sur qui est tombée la tour de Siloé et qu'elle a tuées, croyez-vous qu'elles fussent plus coupables que tous les autres habitants de Jérusalem?
\par 5 Non, je vous le dis. Mais si vous ne vous repentez, vous périrez tous également.
\par 6 Il dit aussi cette parabole: Un homme avait un figuier planté dans sa vigne. Il vint pour y chercher du fruit, et il n'en trouva point.
\par 7 Alors il dit au vigneron: Voilà trois ans que je viens chercher du fruit à ce figuier, et je n'en trouve point. Coupe-le: pourquoi occupe-t-il la terre inutilement?
\par 8 Le vigneron lui répondit: Seigneur, laisse-le encore cette année; je creuserai tout autour, et j'y mettrai du fumier.
\par 9 Peut-être à l'avenir donnera-t-il du fruit; sinon, tu le couperas.
\par 10 Jésus enseignait dans une des synagogues, le jour du sabbat.
\par 11 Et voici, il y avait là une femme possédée d'un esprit qui la rendait infirme depuis dix-huit ans; elle était courbée, et ne pouvait pas du tout se redresser.
\par 12 Lorsqu'il la vit, Jésus lui adressa la parole, et lui dit: Femme, tu es délivrée de ton infirmité.
\par 13 Et il lui imposa les mains. A l'instant elle se redressa, et glorifia Dieu.
\par 14 Mais le chef de la synagogue, indigné de ce que Jésus avait opéré cette guérison un jour de sabbat, dit à la foule: Il y a six jours pour travailler; venez donc vous faire guérir ces jours-là, et non pas le jour du sabbat.
\par 15 Hypocrites! lui répondit le Seigneur, est-ce que chacun de vous, le jour du sabbat, ne détache pas de la crèche son boeuf ou son âne, pour le mener boire?
\par 16 Et cette femme, qui est une fille d'Abraham, et que Satan tenait liée depuis dix-huit ans, ne fallait-il pas la délivrer de cette chaîne le jour du sabbat?
\par 17 Tandis qu'il parlait ainsi, tous ses adversaires étaient confus, et la foule se réjouissait de toutes les choses glorieuses qu'il faisait.
\par 18 Il dit encore: A quoi le royaume de Dieu est-il semblable, et à quoi le comparerai-je?
\par 19 Il est semblable à un grain de sénevé qu'un homme a pris et jeté dans son jardin; il pousse, devient un arbre, et les oiseaux du ciel habitent dans ses branches.
\par 20 Il dit encore: A quoi comparerai-je le royaume de Dieu?
\par 21 Il est semblable à du levain qu'une femme a pris et mis dans trois mesures de farine, pour faire lever toute la pâte.
\par 22 Jésus traversait les villes et les villages, enseignant, et faisant route vers Jérusalem.
\par 23 Quelqu'un lui dit: Seigneur, n'y a-t-il que peu de gens qui soient sauvés? Il leur répondit:
\par 24 Efforcez-vous d'entrer par la porte étroite. Car, je vous le dis, beaucoup chercheront à entrer, et ne le pourront pas.
\par 25 Quand le maître de la maison se sera levé et aura fermé la porte, et que vous, étant dehors, vous commencerez à frapper à la porte, en disant: Seigneur, Seigneur, ouvre-nous! il vous répondra: Je ne sais d'où vous êtes.
\par 26 Alors vous vous mettrez à dire: Nous avons mangé et bu devant toi, et tu as enseigné dans nos rues.
\par 27 Et il répondra: Je vous le dis, je ne sais d'où vous êtes; retirez-vous de moi, vous tous, ouvriers d'iniquité.
\par 28 C'est là qu'il y aura des pleurs et des grincements de dents, quand vous verrez Abraham, Isaac et Jacob, et tous les prophètes, dans le royaume de Dieu, et que vous serez jetés dehors.
\par 29 Il en viendra de l'orient et de l'occident, du nord et du midi; et ils se mettront à table dans le royaume de Dieu.
\par 30 Et voici, il y en a des derniers qui seront les premiers, et des premiers qui seront les derniers.
\par 31 Ce même jour, quelques pharisiens vinrent lui dire: Va-t'en, pars d'ici, car Hérode veut te tuer.
\par 32 Il leur répondit: Allez, et dites à ce renard: Voici, je chasse les démons et je fais des guérisons aujourd'hui et demain, et le troisième jour j'aurai fini.
\par 33 Mais il faut que je marche aujourd'hui, demain, et le jour suivant; car il ne convient pas qu'un prophète périsse hors de Jérusalem.
\par 34 Jérusalem, Jérusalem, qui tues les prophètes et qui lapides ceux qui te sont envoyés, combien de fois ai-je voulu rassembler tes enfants, comme une poule rassemble sa couvée sous ses ailes, et vous ne l'avez pas voulu!
\par 35 Voici, votre maison vous sera laissée; mais, je vous le dis, vous ne me verrez plus, jusqu'à ce que vous disiez: Béni soit celui qui vient au nom du Seigneur!

\chapter{14}

\par 1 Jésus étant entré, un jour de sabbat, dans la maison de l'un des chefs des pharisiens, pour prendre un repas, les pharisiens l'observaient.
\par 2 Et voici, un homme hydropique était devant lui.
\par 3 Jésus prit la parole, et dit aux docteurs de la loi et aux pharisiens: Est-il permis, ou non, de faire une guérison le jour du sabbat?
\par 4 Ils gardèrent le silence. Alors Jésus avança la main sur cet homme, le guérit, et le renvoya.
\par 5 Puis il leur dit: Lequel de vous, si son fils ou son boeuf tombe dans un puits, ne l'en retirera pas aussitôt, le jour du sabbat?
\par 6 Et ils ne purent rien répondre à cela.
\par 7 Il adressa ensuite une parabole aux conviés, en voyant qu'ils choisissaient les premières places; et il leur dit:
\par 8 Lorsque tu seras invité par quelqu'un à des noces, ne te mets pas à la première place, de peur qu'il n'y ait parmi les invités une personne plus considérable que toi,
\par 9 et que celui qui vous a invités l'un et l'autre ne vienne te dire: Cède la place à cette personne-là. Tu aurais alors la honte d'aller occuper la dernière place.
\par 10 Mais, lorsque tu seras invité, va te mettre à la dernière place, afin que, quand celui qui t'a invité viendra, il te dise: Mon ami, monte plus haut. Alors cela te fera honneur devant tous ceux qui seront à table avec toi.
\par 11 Car quiconque s'élève sera abaissé, et quiconque s'abaisse sera élevé.
\par 12 Il dit aussi à celui qui l'avait invité: Lorsque tu donnes à dîner ou à souper, n'invite pas tes amis, ni tes frères, ni tes parents, ni des voisins riches, de peur qu'ils ne t'invitent à leur tour et qu'on ne te rende la pareille.
\par 13 Mais, lorsque tu donnes un festin, invite des pauvres, des estropiés, des boiteux, des aveugles.
\par 14 Et tu seras heureux de ce qu'ils ne peuvent pas te rendre la pareille; car elle te sera rendue à la résurrection des justes.
\par 15 Un de ceux qui étaient à table, après avoir entendu ces paroles, dit à Jésus: Heureux celui qui prendra son repas dans le royaume de Dieu!
\par 16 Et Jésus lui répondit: Un homme donna un grand souper, et il invita beaucoup de gens.
\par 17 A l'heure du souper, il envoya son serviteur dire aux conviés: Venez, car tout est déjà prêt.
\par 18 Mais tous unanimement se mirent à s'excuser. Le premier lui dit: J'ai acheté un champ, et je suis obligé d'aller le voir; excuse-moi, je te prie.
\par 19 Un autre dit: J'ai acheté cinq paires de boeufs, et je vais les essayer; excuse-moi, je te prie.
\par 20 Un autre dit: Je viens de me marier, et c'est pourquoi je ne puis aller.
\par 21 Le serviteur, de retour, rapporta ces choses à son maître. Alors le maître de la maison irrité dit à son serviteur: Va promptement dans les places et dans les rues de la ville, et amène ici les pauvres, les estropiés, les aveugles et les boiteux.
\par 22 Le serviteur dit: Maître, ce que tu as ordonné a été fait, et il y a encore de la place.
\par 23 Et le maître dit au serviteur: Va dans les chemins et le long des haies, et ceux que tu trouveras, contrains-les d'entrer, afin que ma maison soit remplie.
\par 24 Car, je vous le dis, aucun de ces hommes qui avaient été invités ne goûtera de mon souper.
\par 25 De grandes foules faisaient route avec Jésus. Il se retourna, et leur dit:
\par 26 Si quelqu'un vient à moi, et s'il ne hait pas son père, sa mère, sa femme, ses enfants, ses frères, et ses soeurs, et même à sa propre vie, il ne peut être mon disciple.
\par 27 Et quiconque ne porte pas sa croix, et ne me suis pas, ne peut être mon disciple.
\par 28 Car, lequel de vous, s'il veut bâtir une tour, ne s'assied d'abord pour calculer la dépense et voir s'il a de quoi la terminer,
\par 29 de peur qu'après avoir posé les fondements, il ne puisse l'achever, et que tous ceux qui le verront ne se mettent à le railler,
\par 30 en disant: Cet homme a commencé à bâtir, et il n'a pu achever?
\par 31 Ou quel roi, s'il va faire la guerre à un autre roi, ne s'assied d'abord pour examiner s'il peut, avec dix mille hommes, marcher à la rencontre de celui qui vient l'attaquer avec vingt mille?
\par 32 S'il ne le peut, tandis que cet autre roi est encore loin, il lui envoie une ambassade pour demander la paix.
\par 33 Ainsi donc, quiconque d'entre vous ne renonce pas à tout ce qu'il possède ne peut être mon disciple.
\par 34 Le sel est une bonne chose; mais si le sel perd sa saveur, avec quoi l'assaisonnera-t-on?
\par 35 Il n'est bon ni pour la terre, ni pour le fumier; on le jette dehors. Que celui qui a des oreilles pour entendre entende.

\chapter{15}

\par 1 Tous les publicains et les gens de mauvaise vie s'approchaient de Jésus pour l'entendre.
\par 2 Et les pharisiens et les scribes murmuraient, disant: Cet homme accueille des gens de mauvaise vie, et mange avec eux.
\par 3 Mais il leur dit cette parabole:
\par 4 Quel homme d'entre vous, s'il a cent brebis, et qu'il en perde une, ne laisse les quatre-vingt-dix-neuf autres dans le désert pour aller après celle qui est perdue, jusqu'à ce qu'il la retrouve?
\par 5 Lorsqu'il l'a retrouvée, il la met avec joie sur ses épaules,
\par 6 et, de retour à la maison, il appelle ses amis et ses voisins, et leur dit: Réjouissez-vous avec moi, car j'ai retrouvé ma brebis qui était perdue.
\par 7 De même, je vous le dis, il y aura plus de joie dans le ciel pour un seul pécheur qui se repent, que pour quatre-vingt-dix-neuf justes qui n'ont pas besoin de repentance.
\par 8 Ou quelle femme, si elle a dix drachmes, et qu'elle en perde une, n'allume une lampe, ne balaie la maison, et ne cherche avec soin, jusqu'à ce qu'elle la retrouve?
\par 9 Lorsqu'elle l'a retrouvée, elle appelle ses amies et ses voisines, et dit: Réjouissez-vous avec moi, car j'ai retrouvé la drachme que j'avais perdue.
\par 10 De même, je vous le dis, il y a de la joie devant les anges de Dieu pour un seul pécheur qui se repent.
\par 11 Il dit encore: Un homme avait deux fils.
\par 12 Le plus jeune dit à son père: Mon père, donne-moi la part de bien qui doit me revenir. Et le père leur partagea son bien.
\par 13 Peu de jours après, le plus jeune fils, ayant tout ramassé, partit pour un pays éloigné, où il dissipa son bien en vivant dans la débauche.
\par 14 Lorsqu'il eut tout dépensé, une grande famine survint dans ce pays, et il commença à se trouver dans le besoin.
\par 15 Il alla se mettre au service d'un des habitants du pays, qui l'envoya dans ses champs garder les pourceaux.
\par 16 Il aurait bien voulu se rassasier des carouges que mangeaient les pourceaux, mais personne ne lui en donnait.
\par 17 Étant rentré en lui-même, il se dit: Combien de mercenaires chez mon père ont du pain en abondance, et moi, ici, je meurs de faim!
\par 18 Je me lèverai, j'irai vers mon père, et je lui dirai: Mon père, j'ai péché contre le ciel et contre toi,
\par 19 je ne suis plus digne d'être appelé ton fils; traite-moi comme l'un de tes mercenaires.
\par 20 Et il se leva, et alla vers son père. Comme il était encore loin, son père le vit et fut ému de compassion, il courut se jeter à son cou et le baisa.
\par 21 Le fils lui dit: Mon père, j'ai péché contre le ciel et contre toi, je ne suis plus digne d'être appelé ton fils.
\par 22 Mais le père dit à ses serviteurs: Apportez vite la plus belle robe, et l'en revêtez; mettez-lui un anneau au doigt, et des souliers aux pieds.
\par 23 Amenez le veau gras, et tuez-le. Mangeons et réjouissons-nous;
\par 24 car mon fils que voici était mort, et il est revenu à la vie; il était perdu, et il est retrouvé. Et ils commencèrent à se réjouir.
\par 25 Or, le fils aîné était dans les champs. Lorsqu'il revint et approcha de la maison, il entendit la musique et les danses.
\par 26 Il appela un des serviteurs, et lui demanda ce que c'était.
\par 27 Ce serviteur lui dit: Ton frère est de retour, et, parce qu'il l'a retrouvé en bonne santé, ton père a tué le veau gras.
\par 28 Il se mit en colère, et ne voulut pas entrer. Son père sortit, et le pria d'entrer.
\par 29 Mais il répondit à son père: Voici, il y a tant d'années que je te sers, sans avoir jamais transgressé tes ordres, et jamais tu ne m'as donné un chevreau pour que je me réjouisse avec mes amis.
\par 30 Et quand ton fils est arrivé, celui qui a mangé ton bien avec des prostituées, c'est pour lui que tu as tué le veau gras!
\par 31 Mon enfant, lui dit le père, tu es toujours avec moi, et tout ce que j'ai est à toi;
\par 32 mais il fallait bien s'égayer et se réjouir, parce que ton frère que voici était mort et qu'il est revenu à la vie, parce qu'il était perdu et qu'il est retrouvé.

\chapter{16}

\par 1 Jésus dit aussi à ses disciples: Un homme riche avait un économe, qui lui fut dénoncé comme dissipant ses biens.
\par 2 Il l'appela, et lui dit: Qu'est-ce que j'entends dire de toi? Rends compte de ton administration, car tu ne pourras plus administrer mes biens.
\par 3 L'économe dit en lui-même: Que ferai-je, puisque mon maître m'ôte l'administration de ses biens? Travailler à la terre? je ne le puis. Mendier? j'en ai honte.
\par 4 Je sais ce que je ferai, pour qu'il y ait des gens qui me reçoivent dans leurs maisons quand je serai destitué de mon emploi.
\par 5 Et, faisant venir chacun des débiteurs de son maître, il dit au premier: Combien dois-tu à mon maître?
\par 6 Cent mesures d'huile, répondit-il. Et il lui dit: Prends ton billet, assieds-toi vite, et écris cinquante.
\par 7 Il dit ensuite à un autre: Et toi, combien dois-tu? Cent mesures de blé, répondit-il. Et il lui dit: Prends ton billet, et écris quatre-vingts.
\par 8 Le maître loua l'économe infidèle de ce qu'il avait agi prudemment. Car les enfants de ce siècle sont plus prudents à l'égard de leurs semblables que ne le sont les enfants de lumière.
\par 9 Et moi, je vous dis: Faites-vous des amis avec les richesses injustes, pour qu'ils vous reçoivent dans les tabernacles éternels, quand elles viendront à vous manquer.
\par 10 Celui qui est fidèle dans les moindres choses l'est aussi dans les grandes, et celui qui est injuste dans les moindres choses l'est aussi dans les grandes.
\par 11 Si donc vous n'avez pas été fidèle dans les richesses injustes, qui vous confiera les véritables?
\par 12 Et si vous n'avez pas été fidèles dans ce qui est à autrui, qui vous donnera ce qui est à vous?
\par 13 Nul serviteur ne peut servir deux maîtres. Car, ou il haïra l'un et aimera l'autre; ou il s'attachera à l'un et méprisera l'autre. Vous ne pouvez servir Dieu et Mamon.
\par 14 Les pharisiens, qui étaient avares, écoutaient aussi tout cela, et ils se moquaient de lui.
\par 15 Jésus leur dit: Vous, vous cherchez à paraître justes devant les hommes, mais Dieu connaît vos coeurs; car ce qui est élevé parmi les hommes est une abomination devant Dieu.
\par 16 La loi et les prophètes ont subsisté jusqu'à Jean; depuis lors, le royaume de Dieu est annoncé, et chacun use de violence pour y entrer.
\par 17 Il est plus facile que le ciel et la terre passent, qu'il ne l'est qu'un seul trait de lettre de la loi vienne à tomber.
\par 18 Quiconque répudie sa femme et en épouse une autre commet un adultère, et quiconque épouse une femme répudiée par son mari commet un adultère.
\par 19 Il y avait un homme riche, qui était vêtu de pourpre et de fin lin, et qui chaque jour menait joyeuse et brillante vie.
\par 20 Un pauvre, nommé Lazare, était couché à sa porte, couvert d'ulcères,
\par 21 et désireux de se rassasier des miettes qui tombaient de la table du riche; et même les chiens venaient encore lécher ses ulcères.
\par 22 Le pauvre mourut, et il fut porté par les anges dans le sein d'Abraham. Le riche mourut aussi, et il fut enseveli.
\par 23 Dans le séjour des morts, il leva les yeux; et, tandis qu'il était en proie aux tourments, il vit de loin Abraham, et Lazare dans son sein.
\par 24 Il s'écria: Père Abraham, aie pitié de moi, et envoie Lazare, pour qu'il trempe le bout de son doigt dans l'eau et me rafraîchisse la langue; car je souffre cruellement dans cette flamme.
\par 25 Abraham répondit: Mon enfant, souviens-toi que tu as reçu tes biens pendant ta vie, et que Lazare a eu les maux pendant la sienne; maintenant il est ici consolé, et toi, tu souffres.
\par 26 D'ailleurs, il y a entre nous et vous un grand abîme, afin que ceux qui voudraient passer d'ici vers vous, ou de là vers nous, ne puissent le faire.
\par 27 Le riche dit: Je te prie donc, père Abraham, d'envoyer Lazare dans la maison de mon père; car j'ai cinq frères.
\par 28 C'est pour qu'il leur atteste ces choses, afin qu'ils ne viennent pas aussi dans ce lieu de tourments.
\par 29 Abraham répondit: Ils ont Moïse et les prophètes; qu'ils les écoutent.
\par 30 Et il dit: Non, père Abraham, mais si quelqu'un des morts va vers eux, ils se repentiront.
\par 31 Et Abraham lui dit: S'ils n'écoutent pas Moïse et les prophètes, ils ne se laisseront pas persuader quand même quelqu'un des morts ressusciterait.

\chapter{17}

\par 1 Jésus dit à ses disciples: Il est impossible qu'il n'arrive pas des scandales; mais malheur à celui par qui ils arrivent!
\par 2 Il vaudrait mieux pour lui qu'on mît à son cou une pierre de moulin et qu'on le jetât dans la mer, que s'il scandalisait un de ces petits.
\par 3 Prenez garde à vous-mêmes. Si ton frère a péché, reprends-le; et, s'il se repent, pardonne-lui.
\par 4 Et s'il a péché contre toi sept fois dans un jour et que sept fois il revienne à toi, disant: Je me repens, -tu lui pardonneras.
\par 5 Les apôtres dirent au Seigneur: Augmente-nous la foi.
\par 6 Et le Seigneur dit: Si vous aviez de la foi comme un grain de sénevé, vous diriez à ce sycomore: Déracine-toi, et plante-toi dans la mer; et il vous obéirait.
\par 7 Qui de vous, ayant un serviteur qui laboure ou paît les troupeaux, lui dira, quand il revient des champs: Approche vite, et mets-toi à table?
\par 8 Ne lui dira-t-il pas au contraire: Prépare-moi à souper, ceins-toi, et sers-moi, jusqu'à ce que j'aie mangé et bu; après cela, toi, tu mangeras et boiras?
\par 9 Doit-il de la reconnaissance à ce serviteur parce qu'il a fait ce qui lui était ordonné?
\par 10 Vous de même, quand vous avez fait tout ce qui vous a été ordonné, dites: Nous sommes des serviteurs inutiles, nous avons fait ce que nous devions faire.
\par 11 Jésus, se rendant à Jérusalem, passait entre la Samarie et la Galilée.
\par 12 Comme il entrait dans un village, dix lépreux vinrent à sa rencontre. Se tenant à distance, ils élevèrent la voix, et dirent:
\par 13 Jésus, maître, aie pitié de nous!
\par 14 Dès qu'il les eut vus, il leur dit: Allez vous montrer aux sacrificateurs. Et, pendant qu'ils y allaient, il arriva qu'ils furent guéris.
\par 15 L'un deux, se voyant guéri, revint sur ses pas, glorifiant Dieu à haute voix.
\par 16 Il tomba sur sa face aux pieds de Jésus, et lui rendit grâces. C'était un Samaritain.
\par 17 Jésus, prenant la parole, dit: Les dix n'ont-ils pas été guéris? Et les neuf autres, où sont-ils?
\par 18 Ne s'est-il trouvé que cet étranger pour revenir et donner gloire à Dieu?
\par 19 Puis il lui dit: Lève-toi, va; ta foi t'a sauvé.
\par 20 Les pharisiens demandèrent à Jésus quand viendrait le royaume de Dieu. Il leur répondit: Le royaume de Dieu ne vient pas de manière à frapper les regards.
\par 21 On ne dira point: Il est ici, ou: Il est là. Car voici, le royaume de Dieu est au milieu de vous.
\par 22 Et il dit aux disciples: Des jours viendront où vous désirerez voir l'un des jours du Fils de l'homme, et vous ne le verrez point.
\par 23 On vous dira: Il est ici, il est là. N'y allez pas, ne courez pas après.
\par 24 Car, comme l'éclair resplendit et brille d'une extrémité du ciel à l'autre, ainsi sera le Fils de l'homme en son jour.
\par 25 Mais il faut auparavant qu'il souffre beaucoup, et qu'il soit rejeté par cette génération.
\par 26 Ce qui arriva du temps de Noé arrivera de même aux jours du Fils de l'homme.
\par 27 Les hommes mangeaient, buvaient, se mariaient et mariaient leurs enfants, jusqu'au jour où Noé entra dans l'arche; le déluge vint, et les fit tous périr.
\par 28 Ce qui arriva du temps de Lot arrivera pareillement. Les hommes mangeaient, buvaient, achetaient, vendaient, plantaient, bâtissaient;
\par 29 mais le jour où Lot sortit de Sodome, une pluie de feu et de souffre tomba du ciel, et les fit tous périr.
\par 30 Il en sera de même le jour où le Fils de l'homme paraîtra.
\par 31 En ce jour-là, que celui qui sera sur le toit, et qui aura ses effets dans la maison, ne descende pas pour les prendre; et que celui qui sera dans les champs ne retourne pas non plus en arrière.
\par 32 Souvenez-vous de la femme de Lot.
\par 33 Celui qui cherchera à sauver sa vie la perdra, et celui qui la perdra la retrouvera.
\par 34 Je vous le dis, en cette nuit-là, de deux personnes qui seront dans un même lit, l'une sera prise et l'autre laissée;
\par 35 de deux femmes qui moudront ensemble, l'une sera prise et l'autre laissée.
\par 36 De deux hommes qui seront dans un champ, l'un sera pris et l'autre laissé.
\par 37 Les disciples lui dirent: Où sera-ce, Seigneur? Et il répondit: Où sera le corps, là s'assembleront les aigles.

\chapter{18}

\par 1 Jésus leur adressa une parabole, pour montrer qu'il faut toujours prier, et ne point se relâcher.
\par 2 Il dit: Il y avait dans une ville un juge qui ne craignait point Dieu et qui n'avait d'égard pour personne.
\par 3 Il y avait aussi dans cette ville une veuve qui venait lui dire: Fais-moi justice de ma partie adverse.
\par 4 Pendant longtemps il refusa. Mais ensuite il dit en lui-même: Quoique je ne craigne point Dieu et que je n'aie d'égard pour personne,
\par 5 néanmoins, parce que cette veuve m'importune, je lui ferai justice, afin qu'elle ne vienne pas sans cesse me rompre la tête.
\par 6 Le Seigneur ajouta: Entendez ce que dit le juge inique.
\par 7 Et Dieu ne fera-t-il pas justice à ses élus, qui crient à lui jour et nuit, et tardera-t-il à leur égard?
\par 8 Je vous le dis, il leur fera promptement justice. Mais, quand le Fils de l'homme viendra, trouvera-t-il la foi sur la terre?
\par 9 Il dit encore cette parabole, en vue de certaines personnes se persuadant qu'elles étaient justes, et ne faisant aucun cas des autres:
\par 10 Deux hommes montèrent au temple pour prier; l'un était pharisien, et l'autre publicain.
\par 11 Le pharisien, debout, priait ainsi en lui-même: O Dieu, je te rends grâces de ce que je ne suis pas comme le reste des hommes, qui sont ravisseurs, injustes, adultères, ou même comme ce publicain;
\par 12 je jeûne deux fois la semaine, je donne la dîme de tous mes revenus.
\par 13 Le publicain, se tenant à distance, n'osait même pas lever les yeux au ciel; mais il se frappait la poitrine, en disant: O Dieu, sois apaisé envers moi, qui suis un pécheur.
\par 14 Je vous le dis, celui-ci descendit dans sa maison justifié, plutôt que l'autre. Car quiconque s'élève sera abaissé, et celui qui s'abaisse sera élevé.
\par 15 On lui amena aussi les petits enfants, afin qu'il les touchât. Mais les disciples, voyant cela, reprenaient ceux qui les amenaient.
\par 16 Et Jésus les appela, et dit: Laissez venir à moi les petits enfants, et ne les en empêchez pas; car le royaume de Dieu est pour ceux qui leur ressemblent.
\par 17 Je vous le dis en vérité, quiconque ne recevra pas le royaume de Dieu comme un petit enfant n'y entrera point.
\par 18 Un chef interrogea Jésus, et dit: Bon maître, que dois-je faire pour hériter la vie éternelle?
\par 19 Jésus lui répondit: Pourquoi m'appelles-tu bon? Il n'y a de bon que Dieu seul.
\par 20 Tu connais les commandements: Tu ne commettras point d'adultère; tu ne tueras point; tu ne déroberas point; tu ne diras point de faux témoignage; honore ton père et ta mère.
\par 21 J'ai, dit-il, observé toutes ces choses dès ma jeunesse.
\par 22 Jésus, ayant entendu cela, lui dit: Il te manque encore une chose: vends tout ce que tu as, distribue-le aux pauvres, et tu auras un trésor dans les cieux. Puis, viens, et suis-moi.
\par 23 Lorsqu'il entendit ces paroles, il devint tout triste; car il était très riche.
\par 24 Jésus, voyant qu'il était devenu tout triste, dit: Qu'il est difficile à ceux qui ont des richesses d'entrer dans le royaume de Dieu!
\par 25 Car il est plus facile à un chameau de passer par le trou d'une aiguille qu'à un riche d'entrer dans le royaume de Dieu.
\par 26 Ceux qui l'écoutaient dirent: Et qui peut être sauvé?
\par 27 Jésus répondit: Ce qui est impossible aux hommes est possible à Dieu.
\par 28 Pierre dit alors: Voici, nous avons tout quitté, et nous t'avons suivi.
\par 29 Et Jésus leur dit: Je vous le dis en vérité, il n'est personne qui, ayant quitté, à cause du royaume de Dieu, sa maison, ou sa femme, ou ses frères, ou ses parents, ou ses enfants,
\par 30 ne reçoive beaucoup plus dans ce siècle-ci, et, dans le siècle à venir, la vie éternelle.
\par 31 Jésus prit les douze auprès de lui, et leur dit: Voici, nous montons à Jérusalem, et tout ce qui a été écrit par les prophètes au sujet du Fils de l'homme s'accomplira.
\par 32 Car il sera livré aux païens; on se moquera de lui, on l'outragera, on crachera sur lui,
\par 33 et, après l'avoir battu de verges, on le fera mourir; et le troisième jour il ressuscitera.
\par 34 Mais ils ne comprirent rien à cela; c'était pour eux un langage caché, des paroles dont ils ne saisissaient pas le sens.
\par 35 Comme Jésus approchait de Jéricho, un aveugle était assis au bord du chemin, et mendiait.
\par 36 Entendant la foule passer, il demanda ce que c'était.
\par 37 On lui dit: C'est Jésus de Nazareth qui passe.
\par 38 Et il cria: Jésus, Fils de David, aie pitié de moi!
\par 39 Ceux qui marchaient devant le reprenaient, pour le faire taire; mais il criait beaucoup plus fort: Fils de David, aie pitié de moi!
\par 40 Jésus, s'étant arrêté, ordonna qu'on le lui amène; et, quand il se fut approché,
\par 41 il lui demanda: Que veux-tu que je te fasse? Il répondit: Seigneur, que je recouvre la vue.
\par 42 Et Jésus lui dit: Recouvre la vue; ta foi t'a sauvé.
\par 43 A l'instant il recouvra la vue, et suivit Jésus, en glorifiant Dieu. Tout le peuple, voyant cela, loua Dieu.

\chapter{19}

\par 1 Jésus, étant entré dans Jéricho, traversait la ville.
\par 2 Et voici, un homme riche, appelé Zachée, chef des publicains, cherchait à voir qui était Jésus;
\par 3 mais il ne pouvait y parvenir, à cause de la foule, car il était de petite taille.
\par 4 Il courut en avant, et monta sur un sycomore pour le voir, parce qu'il devait passer par là.
\par 5 Lorsque Jésus fut arrivé à cet endroit, il leva les yeux et lui dit: Zachée, hâte-toi de descendre; car il faut que je demeure aujourd'hui dans ta maison.
\par 6 Zachée se hâta de descendre, et le reçut avec joie.
\par 7 Voyant cela, tous murmuraient, et disaient: Il est allé loger chez un homme pécheur.
\par 8 Mais Zachée, se tenant devant le Seigneur, lui dit: Voici, Seigneur, je donne aux pauvres la moitié de mes biens, et, si j'ai fait tort de quelque chose à quelqu'un, je lui rends le quadruple.
\par 9 Jésus lui dit: Le salut est entré aujourd'hui dans cette maison, parce que celui-ci est aussi un fils d'Abraham.
\par 10 Car le Fils de l'homme est venu chercher et sauver ce qui était perdu.
\par 11 Ils écoutaient ces choses, et Jésus ajouta une parabole, parce qu'il était près de Jérusalem, et qu'on croyait qu'à l'instant le royaume de Dieu allait paraître.
\par 12 Il dit donc: Un homme de haute naissance s'en alla dans un pays lointain, pour se faire investir de l'autorité royale, et revenir ensuite.
\par 13 Il appela dix de ses serviteurs, leur donna dix mines, et leur dit: Faites-les valoir jusqu'à ce que je revienne.
\par 14 Mais ses concitoyens le haïssaient, et ils envoyèrent une ambassade après lui, pour dire: Nous ne voulons pas que cet homme règne sur nous.
\par 15 Lorsqu'il fut de retour, après avoir été investi de l'autorité royale, il fit appeler auprès de lui les serviteurs auxquels il avait donné l'argent, afin de connaître comment chacun l'avait fait valoir.
\par 16 Le premier vint, et dit: Seigneur, ta mine a rapporté dix mines.
\par 17 Il lui dit: C'est bien, bon serviteur; parce que tu as été fidèle en peu de chose, reçois le gouvernement de dix villes.
\par 18 Le second vint, et dit: Seigneur, ta mine a produit cinq mines.
\par 19 Il lui dit: Toi aussi, sois établi sur cinq villes.
\par 20 Un autre vint, et dit: Seigneur, voici ta mine, que j'ai gardée dans un linge;
\par 21 car j'avais peur de toi, parce que tu es un homme sévère; tu prends ce que tu n'as pas déposé, et tu moissonnes ce que tu n'as pas semé.
\par 22 Il lui dit: Je te juge sur tes paroles, méchant serviteur; tu savais que je suis un homme sévère, prenant ce que je n'ai pas déposé, et moissonnant ce que je n'ai pas semé;
\par 23 pourquoi donc n'as-tu pas mis mon argent dans une banque, afin qu'à mon retour je le retirasse avec un intérêt?
\par 24 Puis il dit à ceux qui étaient là: Otez-lui la mine, et donnez-la à celui qui a les dix mines.
\par 25 Ils lui dirent: Seigneur, il a dix mines. -
\par 26 Je vous le dis, on donnera à celui qui a, mais à celui qui n'a pas on ôtera même ce qu'il a.
\par 27 Au reste, amenez ici mes ennemis, qui n'ont pas voulu que je régnasse sur eux, et tuez-les en ma présence.
\par 28 Après avoir ainsi parlé, Jésus marcha devant la foule, pour monter à Jérusalem.
\par 29 Lorsqu'il approcha de Bethphagé et de Béthanie, vers la montagne appelée montagne des Oliviers, Jésus envoya deux de ses disciples,
\par 30 en disant: Allez au village qui est en face; quand vous y serez entrés, vous trouverez un ânon attaché, sur lequel aucun homme ne s'est jamais assis; détachez-le, et amenez-le.
\par 31 Si quelqu'un vous demande: Pourquoi le détachez-vous? vous lui répondrez: Le Seigneur en a besoin.
\par 32 Ceux qui étaient envoyés allèrent, et trouvèrent les choses comme Jésus leur avait dit.
\par 33 Comme ils détachaient l'ânon, ses maîtres leur dirent: Pourquoi détachez-vous l'ânon?
\par 34 Ils répondirent: Le Seigneur en a besoin.
\par 35 Et ils amenèrent à Jésus l'ânon, sur lequel ils jetèrent leurs vêtements, et firent monter Jésus.
\par 36 Quand il fut en marche, les gens étendirent leurs vêtements sur le chemin.
\par 37 Et lorsque déjà il approchait de Jérusalem, vers la descente de la montagne des Oliviers, toute la multitude des disciples, saisie de joie, se mit à louer Dieu à haute voix pour tous les miracles qu'ils avaient vus.
\par 38 Ils disaient: Béni soit le roi qui vient au nom du Seigneur! Paix dans le ciel, et gloire dans les lieux très hauts!
\par 39 Quelques pharisiens, du milieu de la foule, dirent à Jésus: Maître, reprends tes disciples.
\par 40 Et il répondit: Je vous le dis, s'ils se taisent, les pierres crieront!
\par 41 Comme il approchait de la ville, Jésus, en la voyant, pleura sur elle, et dit:
\par 42 Si toi aussi, au moins en ce jour qui t'est donné, tu connaissais les choses qui appartiennent à ta paix! Mais maintenant elles sont cachées à tes yeux.
\par 43 Il viendra sur toi des jours où tes ennemis t'environneront de tranchées, t'enfermeront, et te serreront de toutes parts;
\par 44 ils te détruiront, toi et tes enfants au milieu de toi, et ils ne laisseront pas en toi pierre sur pierre, parce que tu n'as pas connu le temps où tu as été visitée.
\par 45 Il entra dans le temple, et il se mit à chasser ceux qui vendaient,
\par 46 leur disant: Il est écrit: Ma maison sera une maison de prière. Mais vous, vous en avez fait une caverne de voleurs.
\par 47 Il enseignait tous les jours dans le temple. Et les principaux sacrificateurs, les scribes, et les principaux du peuple cherchaient à le faire périr;
\par 48 mais ils ne savaient comment s'y prendre, car tout le peuple l'écoutait avec admiration.

\chapter{20}

\par 1 Un de ces jours-là, comme Jésus enseignait le peuple dans le temple et qu'il annonçait la bonne nouvelle, les principaux sacrificateurs et les scribes, avec les anciens, survinrent,
\par 2 et lui dirent: Dis-nous, par quelle autorité fais-tu ces choses, ou qui est celui qui t'a donné cette autorité?
\par 3 Il leur répondit: Je vous adresserai aussi une question.
\par 4 Dites-moi, le baptême de Jean venait-il du ciel, ou des hommes?
\par 5 Mais ils raisonnèrent ainsi entre eux: Si nous répondons: Du ciel, il dira: Pourquoi n'avez-vous pas cru en lui?
\par 6 Et si nous répondons: Des hommes, tout le peuple nous lapidera, car il est persuadé que Jean était un prophète.
\par 7 Alors ils répondirent qu'ils ne savaient d'où il venait.
\par 8 Et Jésus leur dit: Moi non plus, je ne vous dirai pas par quelle autorité je fais ces choses.
\par 9 Il se mit ensuite à dire au peuple cette parabole: Un homme planta une vigne, l'afferma à des vignerons, et quitta pour longtemps le pays.
\par 10 Au temps de la récolte, il envoya un serviteur vers les vignerons, pour qu'ils lui donnent une part du produit de la vigne. Les vignerons le battirent, et le renvoyèrent à vide.
\par 11 Il envoya encore un autre serviteur; ils le battirent, l'outragèrent, et le renvoyèrent à vide.
\par 12 Il en envoya encore un troisième; ils le blessèrent, et le chassèrent.
\par 13 Le maître de la vigne dit: Que ferai-je? J'enverrai mon fils bien-aimé; peut-être auront-ils pour lui du respect.
\par 14 Mais, quand les vignerons le virent, ils raisonnèrent entre eux, et dirent: Voici l'héritier; tuons-le, afin que l'héritage soit à nous.
\par 15 Et ils le jetèrent hors de la vigne, et le tuèrent. Maintenant, que leur fera le maître de la vigne?
\par 16 Il viendra, fera périr ces vignerons, et il donnera la vigne à d'autres. Lorsqu'ils eurent entendu cela, ils dirent: A Dieu ne plaise!
\par 17 Mais, jetant les regards sur eux, Jésus dit: Que signifie donc ce qui est écrit: La pierre qu'ont rejetée ceux qui bâtissaient Est devenue la principale de l'angle?
\par 18 Quiconque tombera sur cette pierre s'y brisera, et celui sur qui elle tombera sera écrasé.
\par 19 Les principaux sacrificateurs et les scribes cherchèrent à mettre la main sur lui à l'heure même, mais ils craignirent le peuple. Ils avaient compris que c'était pour eux que Jésus avait dit cette parabole.
\par 20 Ils se mirent à observer Jésus; et ils envoyèrent des gens qui feignaient d'être justes, pour lui tendre des pièges et saisir de lui quelque parole, afin de le livrer au magistrat et à l'autorité du gouverneur.
\par 21 Ces gens lui posèrent cette question: Maître, nous savons que tu parles et enseignes droitement, et que tu ne regardes pas à l'apparence, mais que tu enseignes la voie de Dieu selon la vérité.
\par 22 Nous est-il permis, ou non, de payer le tribut à César?
\par 23 Jésus, apercevant leur ruse, leur répondit: Montrez-moi un denier.
\par 24 De qui porte-t-il l'effigie et l'inscription? De César, répondirent-ils.
\par 25 Alors il leur dit: Rendez donc à César ce qui est à César, et à Dieu ce qui est à Dieu.
\par 26 Ils ne purent rien reprendre dans ses paroles devant le peuple; mais, étonnés de sa réponse, ils gardèrent le silence.
\par 27 Quelques-uns des sadducéens, qui disent qu'il n'y a point de résurrection, s'approchèrent, et posèrent à Jésus cette question:
\par 28 Maître, voici ce que Moïse nous a prescrit: Si le frère de quelqu'un meurt, ayant une femme sans avoir d'enfants, son frère épousera la femme, et suscitera une postérité à son frère.
\par 29 Or, il y avait sept frères. Le premier se maria, et mourut sans enfants.
\par 30 Le second et le troisième épousèrent la veuve;
\par 31 il en fut de même des sept, qui moururent sans laisser d'enfants.
\par 32 Enfin, la femme mourut aussi.
\par 33 A la résurrection, duquel d'entre eux sera-t-elle donc la femme? Car les sept l'ont eue pour femme.
\par 34 Jésus leur répondit: Les enfants de ce siècle prennent des femmes et des maris;
\par 35 mais ceux qui seront trouvés dignes d'avoir part au siècle à venir et à la résurrection des morts ne prendront ni femmes ni maris.
\par 36 Car ils ne pourront plus mourir, parce qu'ils seront semblables aux anges, et qu'ils seront fils de Dieu, étant fils de la résurrection.
\par 37 Que les morts ressuscitent, c'est ce que Moïse a fait connaître quand, à propos du buisson, il appelle le Seigneur le Dieu d'Abraham, le Dieu d'Isaac, et le Dieu de Jacob.
\par 38 Or, Dieu n'est pas Dieu des morts, mais des vivants; car pour lui tous sont vivants.
\par 39 Quelques-uns des scribes, prenant la parole, dirent: Maître, tu as bien parlé.
\par 40 Et ils n'osaient plus lui faire aucune question.
\par 41 Jésus leur dit: Comment dit-on que le Christ est fils de David?
\par 42 David lui-même dit dans le livre des Psaumes: Le Seigneur a dit à mon Seigneur: Assieds-toi à ma droite,
\par 43 Jusqu'à ce que je fasse de tes ennemis ton marchepied.
\par 44 David donc l'appelle Seigneur; comment est-il son fils?
\par 45 Tandis que tout le peuple l'écoutait, il dit à ses disciples:
\par 46 Gardez-vous des scribes, qui aiment à se promener en robes longues, et à être salués dans les places publiques; qui recherchent les premiers sièges dans les synagogues, et les premières places dans les festins;
\par 47 qui dévorent les maisons des veuves, et qui font pour l'apparence de longues prières. Ils seront jugés plus sévèrement.

\chapter{21}

\par 1 Jésus, ayant levé les yeux, vit les riches qui mettaient leurs offrandes dans le tronc.
\par 2 Il vit aussi une pauvre veuve, qui y mettait deux petites pièces.
\par 3 Et il dit: Je vous le dis en vérité, cette pauvre veuve a mis plus que tous les autres;
\par 4 car c'est de leur superflu que tous ceux-là ont mis des offrandes dans le tronc, mais elle a mis de son nécessaire, tout ce qu'elle avait pour vivre.
\par 5 Comme quelques-uns parlaient des belles pierres et des offrandes qui faisaient l'ornement du temple, Jésus dit:
\par 6 Les jours viendront où, de ce que vous voyez, il ne restera pas pierre sur pierre qui ne soit renversée.
\par 7 Ils lui demandèrent: Maître, quand donc cela arrivera-t-il, et à quel signe connaîtra-t-on que ces choses vont arriver?
\par 8 Jésus répondit: Prenez garde que vous ne soyez séduits. Car plusieurs viendront en mon nom, disant: C'est moi, et le temps approche. Ne les suivez pas.
\par 9 Quand vous entendrez parler de guerres et de soulèvements, ne soyez pas effrayés, car il faut que ces choses arrivent premièrement. Mais ce ne sera pas encore la fin.
\par 10 Alors il leur dit: Une nation s'élèvera contre une nation, et un royaume contre un royaume;
\par 11 il y aura de grands tremblements de terre, et, en divers lieux, des pestes et des famines; il y aura des phénomènes terribles, et de grands signes dans le ciel.
\par 12 Mais, avant tout cela, on mettra la main sur vous, et l'on vous persécutera; on vous livrera aux synagogues, on vous jettera en prison, on vous mènera devant des rois et devant des gouverneurs, à cause de mon nom.
\par 13 Cela vous arrivera pour que vous serviez de témoignage.
\par 14 Mettez-vous donc dans l'esprit de ne pas préméditer votre défense;
\par 15 car je vous donnerai une bouche et une sagesse à laquelle tous vos adversaires ne pourront résister ou contredire.
\par 16 Vous serez livrés même par vos parents, par vos frères, par vos proches et par vos amis, et ils feront mourir plusieurs d'entre vous.
\par 17 Vous serez haïs de tous, à cause de mon nom.
\par 18 Mais il ne se perdra pas un cheveu de votre tête;
\par 19 par votre persévérance vous sauverez vos âmes.
\par 20 Lorsque vous verrez Jérusalem investie par des armées, sachez alors que sa désolation est proche.
\par 21 Alors, que ceux qui seront en Judée fuient dans les montagnes, que ceux qui seront au milieu de Jérusalem en sortent, et que ceux qui seront dans les champs n'entrent pas dans la ville.
\par 22 Car ce seront des jours de vengeance, pour l'accomplissement de tout ce qui est écrit.
\par 23 Malheur aux femmes qui seront enceintes et à celles qui allaiteront en ces jours-là! Car il y aura une grande détresse dans le pays, et de la colère contre ce peuple.
\par 24 Ils tomberont sous le tranchant de l'épée, ils seront emmenés captifs parmi toutes les nations, et Jérusalem sera foulée aux pieds par les nations, jusqu'à ce que les temps des nations soient accomplies.
\par 25 Il y aura des signes dans le soleil, dans la lune et dans les étoiles. Et sur la terre, il y aura de l'angoisse chez les nations qui ne sauront que faire, au bruit de la mer et des flots,
\par 26 les hommes rendant l'âme de terreur dans l'attente de ce qui surviendra pour la terre; car les puissances des cieux seront ébranlées.
\par 27 Alors on verra le Fils de l'homme venant sur une nuée avec puissance et une grande gloire.
\par 28 Quand ces choses commenceront à arriver, redressez-vous et levez vos têtes, parce que votre délivrance approche.
\par 29 Et il leur dit une comparaison: Voyez le figuier, et tous les arbres.
\par 30 Dès qu'ils ont poussé, vous connaissez de vous-mêmes, en regardant, que déjà l'été est proche.
\par 31 De même, quand vous verrez ces choses arriver, sachez que le royaume de Dieu est proche.
\par 32 Je vous le dis en vérité, cette génération ne passera point, que tout cela n'arrive.
\par 33 Le ciel et la terre passeront, mais mes paroles ne passeront point.
\par 34 Prenez garde à vous-mêmes, de crainte que vos coeurs ne s'appesantissent par les excès du manger et du boire, et par les soucis de la vie, et que ce jour ne vienne sur vous à l'improviste;
\par 35 car il viendra comme un filet sur tous ceux qui habitent sur la face de toute la terre.
\par 36 Veillez donc et priez en tout temps, afin que vous ayez la force d'échapper à toutes ces choses qui arriveront, et de paraître debout devant le Fils de l'homme.
\par 37 Pendant le jour, Jésus enseignait dans le temple, et il allait passer la nuit à la montagne appelée montagne des Oliviers.
\par 38 Et tout le peuple, dès le matin, se rendait vers lui dans le temple pour l'écouter.

\chapter{22}

\par 1 La fête des pains sans levain, appelée la Pâque, approchait.
\par 2 Les principaux sacrificateurs et les scribes cherchaient les moyens de faire mourir Jésus; car ils craignaient le peuple.
\par 3 Or, Satan entra dans Judas, surnommé Iscariot, qui était du nombre des douze.
\par 4 Et Judas alla s'entendre avec les principaux sacrificateurs et les chefs des gardes, sur la manière de le leur livrer.
\par 5 Ils furent dans la joie, et ils convinrent de lui donner de l'argent.
\par 6 Après s'être engagé, il cherchait une occasion favorable pour leur livrer Jésus à l'insu de la foule.
\par 7 Le jour des pains sans levain, où l'on devait immoler la Pâque, arriva,
\par 8 et Jésus envoya Pierre et Jean, en disant: Allez nous préparer la Pâque, afin que nous la mangions.
\par 9 Ils lui dirent: Où veux-tu que nous la préparions?
\par 10 Il leur répondit: Voici, quand vous serez entrés dans la ville, vous rencontrerez un homme portant une cruche d'eau; suivez-le dans la maison où il entrera,
\par 11 et vous direz au maître de la maison: Le maître te dit: Où est le lieu où je mangerai la Pâque avec mes disciples?
\par 12 Et il vous montrera une grande chambre haute, meublée: c'est là que vous préparerez la Pâque.
\par 13 Ils partirent, et trouvèrent les choses comme il le leur avait dit; et ils préparèrent la Pâque.
\par 14 L'heure étant venue, il se mit à table, et les apôtres avec lui.
\par 15 Il leur dit: J'ai désiré vivement manger cette Pâque avec vous, avant de souffrir;
\par 16 car, je vous le dis, je ne la mangerai plus, jusqu'à ce qu'elle soit accomplie dans le royaume de Dieu.
\par 17 Et, ayant pris une coupe et rendu grâces, il dit: Prenez cette coupe, et distribuez-la entre vous;
\par 18 car, je vous le dis, je ne boirai plus désormais du fruit de la vigne, jusqu'à ce que le royaume de Dieu soit venu.
\par 19 Ensuite il prit du pain; et, après avoir rendu grâces, il le rompit, et le leur donna, en disant: Ceci est mon corps, qui est donné pour vous; faites ceci en mémoire de moi.
\par 20 Il prit de même la coupe, après le souper, et la leur donna, en disant: Cette coupe est la nouvelle alliance en mon sang, qui est répandu pour vous.
\par 21 Cependant voici, la main de celui qui me livre est avec moi à cette table.
\par 22 Le Fils de l'homme s'en va selon ce qui est déterminé. Mais malheur à l'homme par qui il est livré!
\par 23 Et ils commencèrent à se demander les uns aux autres qui était celui d'entre eux qui ferait cela.
\par 24 Il s'éleva aussi parmi les apôtres une contestation: lequel d'entre eux devait être estimé le plus grand?
\par 25 Jésus leur dit: Les rois des nations les maîtrisent, et ceux qui les dominent sont appelés bienfaiteurs.
\par 26 Qu'il n'en soit pas de même pour vous. Mais que le plus grand parmi vous soit comme le plus petit, et celui qui gouverne comme celui qui sert.
\par 27 Car quel est le plus grand, celui qui est à table, ou celui qui sert? N'est-ce pas celui qui est à table? Et moi, cependant, je suis au milieu de vous comme celui qui sert.
\par 28 Vous, vous êtes ceux qui avez persévéré avec moi dans mes épreuves;
\par 29 c'est pourquoi je dispose du royaume en votre faveur, comme mon Père en a disposé en ma faveur,
\par 30 afin que vous mangiez et buviez à ma table dans mon royaume, et que vous soyez assis sur des trônes, pour juger les douze tribus d'Israël.
\par 31 Le Seigneur dit: Simon, Simon, Satan vous a réclamés, pour vous cribler comme le froment.
\par 32 Mais j'ai prié pour toi, afin que ta foi ne défaille point; et toi, quand tu seras converti, affermis tes frères.
\par 33 Seigneur, lui dit Pierre, je suis prêt à aller avec toi et en prison et à la mort.
\par 34 Et Jésus dit: Pierre, je te le dis, le coq ne chantera pas aujourd'hui que tu n'aies nié trois fois de me connaître.
\par 35 Il leur dit encore: Quand je vous ai envoyés sans bourse, sans sac, et sans souliers, avez-vous manqué de quelque chose? Ils répondirent: De rien.
\par 36 Et il leur dit: Maintenant, au contraire, que celui qui a une bourse la prenne et que celui qui a un sac le prenne également, que celui qui n'a point d'épée vende son vêtement et achète une épée.
\par 37 Car, je vous le dis, il faut que cette parole qui est écrite s'accomplisse en moi: Il a été mis au nombre des malfaiteurs. Et ce qui me concerne est sur le point d'arriver.
\par 38 Ils dirent: Seigneur, voici deux épées. Et il leur dit: Cela suffit.
\par 39 Après être sorti, il alla, selon sa coutume, à la montagne des Oliviers. Ses disciples le suivirent.
\par 40 Lorsqu'il fut arrivé dans ce lieu, il leur dit: Priez, afin que vous ne tombiez pas en tentation.
\par 41 Puis il s'éloigna d'eux à la distance d'environ un jet de pierre, et, s'étant mis à genoux, il pria,
\par 42 disant: Père, si tu voulais éloigner de moi cette coupe! Toutefois, que ma volonté ne se fasse pas, mais la tienne.
\par 43 Alors un ange lui apparut du ciel, pour le fortifier.
\par 44 Étant en agonie, il priait plus instamment, et sa sueur devint comme des grumeaux de sang, qui tombaient à terre.
\par 45 Après avoir prié, il se leva, et vint vers les disciples, qu'il trouva endormis de tristesse,
\par 46 et il leur dit: Pourquoi dormez-vous? Levez-vous et priez, afin que vous ne tombiez pas en tentation.
\par 47 Comme il parlait encore, voici, une foule arriva; et celui qui s'appelait Judas, l'un des douze, marchait devant elle. Il s'approcha de Jésus, pour le baiser.
\par 48 Et Jésus lui dit: Judas, c'est par un baiser que tu livres le Fils de l'homme!
\par 49 Ceux qui étaient avec Jésus, voyant ce qui allait arriver, dirent: Seigneur, frapperons-nous de l'épée?
\par 50 Et l'un d'eux frappa le serviteur du souverain sacrificateur, et lui emporta l'oreille droite.
\par 51 Mais Jésus, prenant la parole, dit: Laissez, arrêtez! Et, ayant touché l'oreille de cet homme, il le guérit.
\par 52 Jésus dit ensuite aux principaux sacrificateurs, aux chefs des gardes du temple, et aux anciens, qui étaient venus contre lui: Vous êtes venus, comme après un brigand, avec des épées et des bâtons.
\par 53 J'étais tous les jours avec vous dans le temple, et vous n'avez pas mis la main sur moi. Mais c'est ici votre heure, et la puissance des ténèbres.
\par 54 Après avoir saisi Jésus, ils l'emmenèrent, et le conduisirent dans la maison du souverain sacrificateur. Pierre suivait de loin.
\par 55 Ils allumèrent du feu au milieu de la cour, et ils s'assirent. Pierre s'assit parmi eux.
\par 56 Une servante, qui le vit assis devant le feu, fixa sur lui les regards, et dit: Cet homme était aussi avec lui.
\par 57 Mais il le nia disant: Femme, je ne le connais pas.
\par 58 Peu après, un autre, l'ayant vu, dit: Tu es aussi de ces gens-là. Et Pierre dit: Homme, je n'en suis pas.
\par 59 Environ une heure plus tard, un autre insistait, disant: Certainement cet homme était aussi avec lui, car il est Galiléen.
\par 60 Pierre répondit: Homme, je ne sais ce que tu dis. Au même instant, comme il parlait encore, le coq chanta.
\par 61 Le Seigneur, s'étant retourné, regarda Pierre. Et Pierre se souvint de la parole que le Seigneur lui avait dite: Avant que le coq chante aujourd'hui, tu me renieras trois fois.
\par 62 Et étant sorti, il pleura amèrement.
\par 63 Les hommes qui tenaient Jésus se moquaient de lui, et le frappaient.
\par 64 Ils lui voilèrent le visage, et ils l'interrogeaient, en disant: Devine qui t'a frappé.
\par 65 Et ils proféraient contre lui beaucoup d'autres injures.
\par 66 Quand le jour fut venu, le collège des anciens du peuple, les principaux sacrificateurs et les scribes, s'assemblèrent, et firent amener Jésus dans leur sanhédrin.
\par 67 Ils dirent: Si tu es le Christ, dis-le nous. Jésus leur répondit: Si je vous le dis, vous ne le croirez pas;
\par 68 et, si je vous interroge, vous ne répondrez pas.
\par 69 Désormais le Fils de l'homme sera assis à la droite de la puissance de Dieu.
\par 70 Tous dirent: Tu es donc le Fils de Dieu? Et il leur répondit: Vous le dites, je le suis.
\par 71 Alors ils dirent: Qu'avons-nous encore besoin de témoignage? Nous l'avons entendu nous-mêmes de sa bouche.

\chapter{23}

\par 1 Ils se levèrent tous, et ils conduisirent Jésus devant Pilate.
\par 2 Ils se mirent à l'accuser, disant: Nous avons trouvé cet homme excitant notre nation à la révolte, empêchant de payer le tribut à César, et se disant lui-même Christ, roi.
\par 3 Pilate l'interrogea, en ces termes: Es-tu le roi des Juifs? Jésus lui répondit: Tu le dis.
\par 4 Pilate dit aux principaux sacrificateurs et à la foule: Je ne trouve rien de coupable en cet homme.
\par 5 Mais ils insistèrent, et dirent: Il soulève le peuple, en enseignant par toute la Judée, depuis la Galilée, où il a commencé, jusqu'ici.
\par 6 Quand Pilate entendit parler de la Galilée, il demanda si cet homme était Galiléen;
\par 7 et, ayant appris qu'il était de la juridiction d'Hérode, il le renvoya à Hérode, qui se trouvait aussi à Jérusalem en ces jours-là.
\par 8 Lorsque Hérode vit Jésus, il en eut une grande joie; car depuis longtemps, il désirait le voir, à cause de ce qu'il avait entendu dire de lui, et il espérait qu'il le verrait faire quelque miracle.
\par 9 Il lui adressa beaucoup de questions; mais Jésus ne lui répondit rien.
\par 10 Les principaux sacrificateurs et les scribes étaient là, et l'accusaient avec violence.
\par 11 Hérode, avec ses gardes, le traita avec mépris; et, après s'être moqué de lui et l'avoir revêtu d'un habit éclatant, il le renvoya à Pilate.
\par 12 Ce jour même, Pilate et Hérode devinrent amis, d'ennemis qu'ils étaient auparavant.
\par 13 Pilate, ayant assemblé les principaux sacrificateurs, les magistrats, et le peuple, leur dit:
\par 14 Vous m'avez amené cet homme comme excitant le peuple à la révolte. Et voici, je l'ai interrogé devant vous, et je ne l'ai trouvé coupable d'aucune des choses dont vous l'accusez;
\par 15 Hérode non plus, car il nous l'a renvoyé, et voici, cet homme n'a rien fait qui soit digne de mort.
\par 16 Je le relâcherai donc, après l'avoir fait battre de verges.
\par 17 A chaque fête, il était obligé de leur relâcher un prisonnier.
\par 18 Ils s'écrièrent tous ensemble: Fais mourir celui-ci, et relâche-nous Barabbas.
\par 19 Cet homme avait été mis en prison pour une sédition qui avait eu lieu dans la ville, et pour un meurtre.
\par 20 Pilate leur parla de nouveau, dans l'intention de relâcher Jésus.
\par 21 Et ils crièrent: Crucifie, crucifie-le!
\par 22 Pilate leur dit pour la troisième fois: Quel mal a-t-il fait? Je n'ai rien trouvé en lui qui mérite la mort. Je le relâcherai donc, après l'avoir fait battre de verges.
\par 23 Mais ils insistèrent à grands cris, demandant qu'il fût crucifié. Et leurs cris l'emportèrent:
\par 24 Pilate prononça que ce qu'ils demandaient serait fait.
\par 25 Il relâcha celui qui avait été mis en prison pour sédition et pour meurtre, et qu'ils réclamaient; et il livra Jésus à leur volonté.
\par 26 Comme ils l'emmenaient, ils prirent un certain Simon de Cyrène, qui revenait des champs, et ils le chargèrent de la croix, pour qu'il la porte derrière Jésus.
\par 27 Il était suivi d'une grande multitude des gens du peuple, et de femmes qui se frappaient la poitrine et se lamentaient sur lui.
\par 28 Jésus se tourna vers elles, et dit: Filles de Jérusalem, ne pleurez pas sur moi; mais pleurez sur vous et sur vos enfants.
\par 29 Car voici, des jours viendront où l'on dira: Heureuses les stériles, heureuses les entrailles qui n'ont point enfanté, et les mamelles qui n'ont point allaité!
\par 30 Alors ils se mettront à dire aux montagnes: Tombez sur nous! Et aux collines: Couvrez-nous!
\par 31 Car, si l'on fait ces choses au bois vert, qu'arrivera-t-il au bois sec?
\par 32 On conduisait en même temps deux malfaiteurs, qui devaient être mis à mort avec Jésus.
\par 33 Lorsqu'ils furent arrivés au lieu appelé Crâne, ils le crucifièrent là, ainsi que les deux malfaiteurs, l'un à droite, l'autre à gauche.
\par 34 Jésus dit: Père, pardonne-leur, car ils ne savent ce qu'ils font. Ils se partagèrent ses vêtements, en tirant au sort.
\par 35 Le peuple se tenait là, et regardait. Les magistrats se moquaient de Jésus, disant: Il a sauvé les autres; qu'il se sauve lui-même, s'il est le Christ, l'élu de Dieu!
\par 36 Les soldats aussi se moquaient de lui; s'approchant et lui présentant du vinaigre,
\par 37 ils disaient: Si tu es le roi des Juifs, sauve-toi toi-même!
\par 38 Il y avait au-dessus de lui cette inscription: Celui-ci est le roi des Juifs.
\par 39 L'un des malfaiteurs crucifiés l'injuriait, disant: N'es-tu pas le Christ? Sauve-toi toi-même, et sauve-nous!
\par 40 Mais l'autre le reprenait, et disait: Ne crains-tu pas Dieu, toi qui subis la même condamnation?
\par 41 Pour nous, c'est justice, car nous recevons ce qu'ont mérité nos crimes; mais celui-ci n'a rien fait de mal.
\par 42 Et il dit à Jésus: Souviens-toi de moi, quand tu viendras dans ton règne.
\par 43 Jésus lui répondit: Je te le dis en vérité, aujourd'hui tu seras avec moi dans le paradis.
\par 44 Il était déjà environ la sixième heure, et il y eut des ténèbres sur toute la terre, jusqu'à la neuvième heure.
\par 45 Le soleil s'obscurcit, et le voile du temple se déchira par le milieu.
\par 46 Jésus s'écria d'une voix forte: Père, je remets mon esprit entre tes mains. Et, en disant ces paroles, il expira.
\par 47 Le centenier, voyant ce qui était arrivé, glorifia Dieu, et dit: Certainement, cet homme était juste.
\par 48 Et tous ceux qui assistaient en foule à ce spectacle, après avoir vu ce qui était arrivé, s'en retournèrent, se frappant la poitrine.
\par 49 Tous ceux de la connaissance de Jésus, et les femmes qui l'avaient accompagné depuis la Galilée, se tenaient dans l'éloignement et regardaient ce qui se passait.
\par 50 Il y avait un conseiller, nommé Joseph, homme bon et juste,
\par 51 qui n'avait point participé à la décision et aux actes des autres; il était d'Arimathée, ville des Juifs, et il attendait le royaume de Dieu.
\par 52 Cet homme se rendit vers Pilate, et demanda le corps de Jésus.
\par 53 Il le descendit de la croix, l'enveloppa d'un linceul, et le déposa dans un sépulcre taillé dans le roc, où personne n'avait encore été mis.
\par 54 C'était le jour de la préparation, et le sabbat allait commencer.
\par 55 Les femmes qui étaient venues de la Galilée avec Jésus accompagnèrent Joseph, virent le sépulcre et la manière dont le corps de Jésus y fut déposé,
\par 56 et, s'en étant retournées, elles préparèrent des aromates et des parfums. Puis elles se reposèrent le jour du sabbat, selon la loi.

\chapter{24}

\par 1 Le premier jour de la semaine, elles se rendirent au sépulcre de grand matin, portant les aromates qu'elles avaient préparés.
\par 2 Elles trouvèrent que la pierre avait été roulée de devant le sépulcre;
\par 3 et, étant entrées, elles ne trouvèrent pas le corps du Seigneur Jésus.
\par 4 Comme elles ne savaient que penser de cela, voici, deux hommes leur apparurent, en habits resplendissants.
\par 5 Saisies de frayeur, elles baissèrent le visage contre terre; mais ils leur dirent: Pourquoi cherchez-vous parmi les morts celui qui est vivant?
\par 6 Il n'est point ici, mais il est ressuscité. Souvenez-vous de quelle manière il vous a parlé, lorsqu'il était encore en Galilée,
\par 7 et qu'il disait: Il faut que le Fils de l'homme soit livré entre les mains des pécheurs, qu'il soit crucifié, et qu'il ressuscite le troisième jour.
\par 8 Et elles se ressouvinrent des paroles de Jésus.
\par 9 A leur retour du sépulcre, elles annoncèrent toutes ces choses aux onze, et à tous les autres.
\par 10 Celles qui dirent ces choses aux apôtres étaient Marie de Magdala, Jeanne, Marie, mère de Jacques, et les autres qui étaient avec elles.
\par 11 Ils tinrent ces discours pour des rêveries, et ils ne crurent pas ces femmes.
\par 12 Mais Pierre se leva, et courut au sépulcre. S'étant baissé, il ne vit que les linges qui étaient à terre; puis il s'en alla chez lui, dans l'étonnement de ce qui était arrivé.
\par 13 Et voici, ce même jour, deux disciples allaient à un village nommé Emmaüs, éloigné de Jérusalem de soixante stades;
\par 14 et ils s'entretenaient de tout ce qui s'était passé.
\par 15 Pendant qu'ils parlaient et discutaient, Jésus s'approcha, et fit route avec eux.
\par 16 Mais leurs yeux étaient empêchés de le reconnaître.
\par 17 Il leur dit: De quoi vous entretenez-vous en marchant, pour que vous soyez tout tristes?
\par 18 L'un d'eux, nommé Cléopas, lui répondit: Es-tu le seul qui, séjournant à Jérusalem ne sache pas ce qui y est arrivé ces jours-ci? -
\par 19 Quoi? leur dit-il. -Et ils lui répondirent: Ce qui est arrivé au sujet de Jésus de Nazareth, qui était un prophète puissant en oeuvres et en paroles devant Dieu et devant tout le peuple,
\par 20 et comment les principaux sacrificateurs et nos magistrats l'on livré pour le faire condamner à mort et l'ont crucifié.
\par 21 Nous espérions que ce serait lui qui délivrerait Israël; mais avec tout cela, voici le troisième jour que ces choses se sont passées.
\par 22 Il est vrai que quelques femmes d'entre nous nous ont fort étonnés; s'étant rendues de grand matin au sépulcre
\par 23 et n'ayant pas trouvé son corps, elles sont venues dire que des anges leurs sont apparus et ont annoncé qu'il est vivant.
\par 24 Quelques-uns de ceux qui étaient avec nous sont allés au sépulcre, et ils ont trouvé les choses comme les femmes l'avaient dit; mais lui, ils ne l'ont point vu.
\par 25 Alors Jésus leur dit: O hommes sans intelligence, et dont le coeur est lent à croire tout ce qu'ont dit les prophètes!
\par 26 Ne fallait-il pas que le Christ souffrît ces choses, et qu'il entrât dans sa gloire?
\par 27 Et, commençant par Moïse et par tous les prophètes, il leur expliqua dans toutes les Écritures ce qui le concernait.
\par 28 Lorsqu'ils furent près du village où ils allaient, il parut vouloir aller plus loin.
\par 29 Mais ils le pressèrent, en disant: Reste avec nous, car le soir approche, le jour est sur son déclin. Et il entra, pour rester avec eux.
\par 30 Pendant qu'il était à table avec eux, il prit le pain; et, après avoir rendu grâces, il le rompit, et le leur donna.
\par 31 Alors leurs yeux s'ouvrirent, et ils le reconnurent; mais il disparut de devant eux.
\par 32 Et ils se dirent l'un à l'autre: Notre coeur ne brûlait-il pas au dedans de nous, lorsqu'il nous parlait en chemin et nous expliquait les Écritures?
\par 33 Se levant à l'heure même, ils retournèrent à Jérusalem, et ils trouvèrent les onze, et ceux qui étaient avec eux, assemblés
\par 34 et disant: Le Seigneur est réellement ressuscité, et il est apparu à Simon.
\par 35 Et ils racontèrent ce qui leur était arrivé en chemin, et comment ils l'avaient reconnu au moment où il rompit le pain.
\par 36 Tandis qu'ils parlaient de la sorte, lui-même se présenta au milieu d'eux, et leur dit: La paix soit avec vous!
\par 37 Saisis de frayeur et d'épouvante, ils croyaient voir un esprit.
\par 38 Mais il leur dit: Pourquoi êtes-vous troublés, et pourquoi pareilles pensées s'élèvent-elles dans vos coeurs?
\par 39 Voyez mes mains et mes pieds, c'est bien moi; touchez-moi et voyez: un esprit n'a ni chair ni os, comme vous voyez que j'ai.
\par 40 Et en disant cela, il leur montra ses mains et ses pieds.
\par 41 Comme, dans leur joie, ils ne croyaient point encore, et qu'ils étaient dans l'étonnement, il leur dit: Avez-vous ici quelque chose à manger?
\par 42 Ils lui présentèrent du poisson rôti et un rayon de miel.
\par 43 Il en prit, et il mangea devant eux.
\par 44 Puis il leur dit: C'est là ce que je vous disais lorsque j'étais encore avec vous, qu'il fallait que s'accomplît tout ce qui est écrit de moi dans la loi de Moïse, dans les prophètes, et dans les psaumes.
\par 45 Alors il leur ouvrit l'esprit, afin qu'ils comprissent les Écritures.
\par 46 Et il leur dit: Ainsi il est écrit que le Christ souffrirait, et qu'il ressusciterait des morts le troisième jour,
\par 47 et que la repentance et le pardon des péchés seraient prêchés en son nom à toutes les nations, à commencer par Jérusalem.
\par 48 Vous êtes témoins de ces choses.
\par 49 Et voici, j'enverrai sur vous ce que mon Père a promis; mais vous, restez dans la ville jusqu'à ce que vous soyez revêtus de la puissance d'en haut.
\par 50 Il les conduisit jusque vers Béthanie, et, ayant levé les mains, il les bénit.
\par 51 Pendant qu'il les bénissait, il se sépara d'eux, et fut enlevé au ciel.
\par 52 Pour eux, après l'avoir adoré, ils retournèrent à Jérusalem avec une grande joie;
\par 53 et ils étaient continuellement dans le temple, louant et bénissant Dieu.


\end{document}