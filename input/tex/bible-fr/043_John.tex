\begin{document}

\title{John}


\chapter{1}

\par 1 Au commencement était la Parole, et la Parole était avec Dieu, et la Parole était Dieu.
\par 2 Elle était au commencement avec Dieu.
\par 3 Toutes choses ont été faites par elle, et rien de ce qui a été fait n'a été fait sans elle.
\par 4 En elle était la vie, et la vie était la lumière des hommes.
\par 5 La lumière luit dans les ténèbres, et les ténèbres ne l'ont point reçue.
\par 6 Il y eut un homme envoyé de Dieu: son nom était Jean.
\par 7 Il vint pour servir de témoin, pour rendre témoignage à la lumière, afin que tous crussent par lui.
\par 8 Il n'était pas la lumière, mais il parut pour rendre témoignage à la lumière.
\par 9 Cette lumière était la véritable lumière, qui, en venant dans le monde, éclaire tout homme.
\par 10 Elle était dans le monde, et le monde a été fait par elle, et le monde ne l'a point connue.
\par 11 Elle est venue chez les siens, et les siens ne l'ont point reçue.
\par 12 Mais à tous ceux qui l'ont reçue, à ceux qui croient en son nom, elle a donné le pouvoir de devenir enfants de Dieu, lesquels sont nés,
\par 13 non du sang, ni de la volonté de la chair, ni de la volonté de l'homme, mais de Dieu.
\par 14 Et la parole a été faite chair, et elle a habité parmi nous, pleine de grâce et de vérité; et nous avons contemplé sa gloire, une gloire comme la gloire du Fils unique venu du Père.
\par 15 Jean lui a rendu témoignage, et s'est écrié: C'est celui dont j'ai dit: Celui qui vient après moi m'a précédé, car il était avant moi.
\par 16 Et nous avons tous reçu de sa plénitude, et grâce pour grâce;
\par 17 car la loi a été donnée par Moïse, la grâce et la vérité sont venues par Jésus Christ.
\par 18 Personne n'a jamais vu Dieu; le Fils unique, qui est dans le sein du Père, est celui qui l'a fait connaître.
\par 19 Voici le témoignage de Jean, lorsque les Juifs envoyèrent de Jérusalem des sacrificateurs et des Lévites, pour lui demander: Toi, qui es-tu?
\par 20 Il déclara, et ne le nia point, il déclara qu'il n'était pas le Christ.
\par 21 Et ils lui demandèrent: Quoi donc? es-tu Élie? Et il dit: Je ne le suis point. Es-tu le prophète? Et il répondit: Non.
\par 22 Ils lui dirent alors: Qui es-tu? afin que nous donnions une réponse à ceux qui nous ont envoyés. Que dis-tu de toi-même?
\par 23 Moi, dit-il, je suis la voix de celui qui crie dans le désert: Aplanissez le chemin du Seigneur, comme a dit Ésaïe, le prophète.
\par 24 Ceux qui avaient été envoyés étaient des pharisiens.
\par 25 Ils lui firent encore cette question: Pourquoi donc baptises-tu, si tu n'es pas le Christ, ni Élie, ni le prophète?
\par 26 Jean leur répondit: Moi, je baptise d'eau, mais au milieu de vous il y a quelqu'un que vous ne connaissez pas, qui vient après moi;
\par 27 je ne suis pas digne de délier la courroie de ses souliers.
\par 28 Ces choses se passèrent à Béthanie, au delà du Jourdain, où Jean baptisait.
\par 29 Le lendemain, il vit Jésus venant à lui, et il dit: Voici l'Agneau de Dieu, qui ôte le péché du monde.
\par 30 C'est celui dont j'ai dit: Après moi vient un homme qui m'a précédé, car il était avant moi.
\par 31 Je ne le connaissais pas, mais c'est afin qu'il fût manifesté à Israël que je suis venu baptiser d'eau.
\par 32 Jean rendit ce témoignage: J'ai vu l'Esprit descendre du ciel comme une colombe et s'arrêter sur lui.
\par 33 Je ne le connaissais pas, mais celui qui m'a envoyé baptiser d'eau, celui-là m'a dit: Celui sur qui tu verras l'Esprit descendre et s'arrêter, c'est celui qui baptise du Saint Esprit.
\par 34 Et j'ai vu, et j'ai rendu témoignage qu'il est le Fils de Dieu.
\par 35 Le lendemain, Jean était encore là, avec deux de ses disciples;
\par 36 et, ayant regardé Jésus qui passait, il dit: Voilà l'Agneau de Dieu.
\par 37 Les deux disciples l'entendirent prononcer ces paroles, et ils suivirent Jésus.
\par 38 Jésus se retourna, et voyant qu'ils le suivaient, il leur dit: Que cherchez-vous? Ils lui répondirent: Rabbi (ce qui signifie Maître), où demeures-tu?
\par 39 Venez, leur dit-il, et voyez. Ils allèrent, et ils virent où il demeurait; et ils restèrent auprès de lui ce jour-là. C'était environ la dixième heure.
\par 40 André, frère de Simon Pierre, était l'un des deux qui avaient entendu les paroles de Jean, et qui avaient suivi Jésus.
\par 41 Ce fut lui qui rencontra le premier son frère Simon, et il lui dit: Nous avons trouvé le Messie (ce qui signifie Christ).
\par 42 Et il le conduisit vers Jésus. Jésus, l'ayant regardé, dit: Tu es Simon, fils de Jonas; tu seras appelé Céphas (ce qui signifie Pierre).
\par 43 Le lendemain, Jésus voulut se rendre en Galilée, et il rencontra Philippe. Il lui dit: Suis-moi.
\par 44 Philippe était de Bethsaïda, de la ville d'André et de Pierre.
\par 45 Philippe rencontra Nathanaël, et lui dit: Nous avons trouvé celui de qui Moïse a écrit dans la loi et dont les prophètes ont parlé, Jésus de Nazareth, fils de Joseph.
\par 46 Nathanaël lui dit: Peut-il venir de Nazareth quelque chose de bon? Philippe lui répondit: Viens, et vois.
\par 47 Jésus, voyant venir à lui Nathanaël, dit de lui: Voici vraiment un Israélite, dans lequel il n'y a point de fraude.
\par 48 D'où me connais-tu? lui dit Nathanaël. Jésus lui répondit: Avant que Philippe t'appelât, quand tu étais sous le figuier, je t'ai vu.
\par 49 Nathanaël répondit et lui dit: Rabbi, tu es le Fils de Dieu, tu es le roi d'Israël.
\par 50 Jésus lui répondit: Parce que je t'ai dit que je t'ai vu sous le figuier, tu crois; tu verras de plus grandes choses que celles-ci.
\par 51 Et il lui dit: En vérité, en vérité, vous verrez désormais le ciel ouvert et les anges de Dieu monter et descendre sur le Fils de l'homme.

\chapter{2}

\par 1 Trois jours après, il y eut des noces à Cana en Galilée. La mère de Jésus était là,
\par 2 et Jésus fut aussi invité aux noces avec ses disciples.
\par 3 Le vin ayant manqué, la mère de Jésus lui dit: Ils n'ont plus de vin.
\par 4 Jésus lui répondit: Femme, qu'y a-t-il entre moi et toi? Mon heure n'est pas encore venue.
\par 5 Sa mère dit aux serviteurs: Faites ce qu'il vous dira.
\par 6 Or, il y avait là six vases de pierre, destinés aux purifications des Juifs, et contenant chacun deux ou trois mesures.
\par 7 Jésus leur dit: Remplissez d'eau ces vases. Et ils les remplirent jusqu'au bord.
\par 8 Puisez maintenant, leur dit-il, et portez-en à l'ordonnateur du repas. Et ils en portèrent.
\par 9 Quand l'ordonnateur du repas eut goûté l'eau changée en vin, -ne sachant d'où venait ce vin, tandis que les serviteurs, qui avaient puisé l'eau, le savaient bien, -il appela l'époux,
\par 10 et lui dit: Tout homme sert d'abord le bon vin, puis le moins bon après qu'on s'est enivré; toi, tu as gardé le bon vin jusqu'à présent.
\par 11 Tel fut, à Cana en Galilée, le premier des miracles que fit Jésus. Il manifesta sa gloire, et ses disciples crurent en lui.
\par 12 Après cela, il descendit à Capernaüm, avec sa mère, ses frères et ses disciples, et ils n'y demeurèrent que peu de jours.
\par 13 La Pâque des Juifs était proche, et Jésus monta à Jérusalem.
\par 14 Il trouva dans le temple les vendeurs de boeufs, de brebis et de pigeons, et les changeurs assis.
\par 15 Ayant fait un fouet avec des cordes, il les chassa tous du temple, ainsi que les brebis et les boeufs; il dispersa la monnaie des changeurs, et renversa les tables;
\par 16 et il dit aux vendeurs de pigeons: Otez cela d'ici, ne faites pas de la maison de mon Père une maison de trafic.
\par 17 Ses disciples se souvinrent qu'il est écrit: Le zèle de ta maison me dévore.
\par 18 Les Juifs, prenant la parole, lui dirent: Quel miracle nous montres-tu, pour agir de la sorte?
\par 19 Jésus leur répondit: Détruisez ce temple, et en trois jours je le relèverai.
\par 20 Les Juifs dirent: Il a fallu quarante-six ans pour bâtir ce temple, et toi, en trois jours tu le relèveras!
\par 21 Mais il parlait du temple de son corps.
\par 22 C'est pourquoi, lorsqu'il fut ressuscité des morts, ses disciples se souvinrent qu'il avait dit cela, et ils crurent à l'Écriture et à la parole que Jésus avait dite.
\par 23 Pendant que Jésus était à Jérusalem, à la fête de Pâque, plusieurs crurent en son nom, voyant les miracles qu'il faisait.
\par 24 Mais Jésus ne se fiait point à eux, parce qu'il les connaissait tous,
\par 25 et parce qu'il n'avait pas besoin qu'on lui rendît témoignage d'aucun homme; car il savait lui-même ce qui était dans l'homme.

\chapter{3}

\par 1 Mais il y eut un homme d'entre les pharisiens, nommé Nicodème, un chef des Juifs,
\par 2 qui vint, lui, auprès de Jésus, de nuit, et lui dit: Rabbi, nous savons que tu es un docteur venu de Dieu; car personne ne peut faire ces miracles que tu fais, si Dieu n'est avec lui.
\par 3 Jésus lui répondit: En vérité, en vérité, je te le dis, si un homme ne naît de nouveau, il ne peut voir le royaume de Dieu.
\par 4 Nicodème lui dit: Comment un homme peut-il naître quand il est vieux? Peut-il rentrer dans le sein de sa mère et naître?
\par 5 Jésus répondit: En vérité, en vérité, je te le dis, si un homme ne naît d'eau et d'Esprit, il ne peut entrer dans le royaume de Dieu.
\par 6 Ce qui est né de la chair est chair, et ce qui est né de l'Esprit est Esprit.
\par 7 Ne t'étonne pas que je t'aie dit: Il faut que vous naissiez de nouveau.
\par 8 Le vent souffle où il veut, et tu en entends le bruit; mais tu ne sais d'où il vient, ni où il va. Il en est ainsi de tout homme qui est né de l'Esprit.
\par 9 Nicodème lui dit: Comment cela peut-il se faire?
\par 10 Jésus lui répondit: Tu es le docteur d'Israël, et tu ne sais pas ces choses!
\par 11 En vérité, en vérité, je te le dis, nous disons ce que nous savons, et nous rendons témoignage de ce que nous avons vu; et vous ne recevez pas notre témoignage.
\par 12 Si vous ne croyez pas quand je vous ai parlé des choses terrestres, comment croirez-vous quand je vous parlerai des choses célestes?
\par 13 Personne n'est monté au ciel, si ce n'est celui qui est descendu du ciel, le Fils de l'homme qui est dans le ciel.
\par 14 Et comme Moïse éleva le serpent dans le désert, il faut de même que le Fils de l'homme soit élevé,
\par 15 afin que quiconque croit en lui ait la vie éternelle.
\par 16 Car Dieu a tant aimé le monde qu'il a donné son Fils unique, afin que quiconque croit en lui ne périsse point, mais qu'il ait la vie éternelle.
\par 17 Dieu, en effet, n'a pas envoyé son Fils dans le monde pour qu'il juge le monde, mais pour que le monde soit sauvé par lui.
\par 18 Celui qui croit en lui n'est point jugé; mais celui qui ne croit pas est déjà jugé, parce qu'il n'a pas cru au nom du Fils unique de Dieu.
\par 19 Et ce jugement c'est que, la lumière étant venue dans le monde, les hommes ont préféré les ténèbres à la lumière, parce que leurs oeuvres étaient mauvaises.
\par 20 Car quiconque fait le mal hait la lumière, et ne vient point à la lumière, de peur que ses oeuvres ne soient dévoilées;
\par 21 mais celui qui agit selon la vérité vient à la lumière, afin que ses oeuvres soient manifestées, parce qu'elles sont faites en Dieu.
\par 22 Après cela, Jésus, accompagné de ses disciples, se rendit dans la terre de Judée; et là il demeurait avec eux, et il baptisait.
\par 23 Jean aussi baptisait à Énon, près de Salim, parce qu'il y avait là beaucoup d'eau; et on y venait pour être baptisé.
\par 24 Car Jean n'avait pas encore été mis en prison.
\par 25 Or, il s'éleva de la part des disciples de Jean une dispute avec un Juif touchant la purification.
\par 26 Ils vinrent trouver Jean, et lui dirent: Rabbi, celui qui était avec toi au delà du Jourdain, et à qui tu as rendu témoignage, voici, il baptise, et tous vont à lui.
\par 27 Jean répondit: Un homme ne peut recevoir que ce qui lui a été donné du ciel.
\par 28 Vous-mêmes m'êtes témoins que j'ai dit: Je ne suis pas le Christ, mais j'ai été envoyé devant lui.
\par 29 Celui à qui appartient l'épouse, c'est l'époux; mais l'ami de l'époux, qui se tient là et qui l'entend, éprouve une grande joie à cause de la voix de l'époux: aussi cette joie, qui est la mienne, est parfaite.
\par 30 Il faut qu'il croisse, et que je diminue.
\par 31 Celui qui vient d'en haut est au-dessus de tous; celui qui est de la terre est de la terre, et il parle comme étant de la terre. Celui qui vient du ciel est au-dessus de tous,
\par 32 il rend témoignage de ce qu'il a vu et entendu, et personne ne reçoit son témoignage.
\par 33 Celui qui a reçu son témoignage a certifié que Dieu est vrai;
\par 34 car celui que Dieu a envoyé dit les paroles de Dieu, parce que Dieu ne lui donne pas l'Esprit avec mesure.
\par 35 Le Père aime le Fils, et il a remis toutes choses entre ses mains.
\par 36 Celui qui croit au Fils a la vie éternelle; celui qui ne croit pas au Fils ne verra point la vie, mais la colère de Dieu demeure sur lui.

\chapter{4}

\par 1 Le Seigneur sut que les pharisiens avaient appris qu'il faisait et baptisait plus de disciples que Jean.
\par 2 Toutefois Jésus ne baptisait pas lui-même, mais c'étaient ses disciples.
\par 3 Alors il quitta la Judée, et retourna en Galilée.
\par 4 Comme il fallait qu'il passât par la Samarie,
\par 5 il arriva dans une ville de Samarie, nommée Sychar, près du champ que Jacob avait donné à Joseph, son fils.
\par 6 Là se trouvait le puits de Jacob. Jésus, fatigué du voyage, était assis au bord du puits. C'était environ la sixième heure.
\par 7 Une femme de Samarie vint puiser de l'eau. Jésus lui dit: Donne-moi à boire.
\par 8 Car ses disciples étaient allés à la ville pour acheter des vivres.
\par 9 La femme samaritaine lui dit: Comment toi, qui es Juif, me demandes-tu à boire, à moi qui suis une femme samaritaine? -Les Juifs, en effet, n'ont pas de relations avec les Samaritains. -
\par 10 Jésus lui répondit: Si tu connaissais le don de Dieu et qui est celui qui te dit: Donne-moi à boire! tu lui aurais toi-même demandé à boire, et il t'aurait donné de l'eau vive.
\par 11 Seigneur, lui dit la femme, tu n'as rien pour puiser, et le puits est profond; d'où aurais-tu donc cette eau vive?
\par 12 Es-tu plus grand que notre père Jacob, qui nous a donné ce puits, et qui en a bu lui-même, ainsi que ses fils et ses troupeaux?
\par 13 Jésus lui répondit: Quiconque boit de cette eau aura encore soif;
\par 14 mais celui qui boira de l'eau que je lui donnerai n'aura jamais soif, et l'eau que je lui donnerai deviendra en lui une source d'eau qui jaillira jusque dans la vie éternelle.
\par 15 La femme lui dit: Seigneur, donne-moi cette eau, afin que je n'aie plus soif, et que je ne vienne plus puiser ici.
\par 16 Va, lui dit Jésus, appelle ton mari, et viens ici.
\par 17 La femme répondit: Je n'ai point de mari. Jésus lui dit: Tu as eu raison de dire: Je n'ai point de mari.
\par 18 Car tu as eu cinq maris, et celui que tu as maintenant n'est pas ton mari. En cela tu as dit vrai.
\par 19 Seigneur, lui dit la femme, je vois que tu es prophète.
\par 20 Nos pères ont adoré sur cette montagne; et vous dites, vous, que le lieu où il faut adorer est à Jérusalem.
\par 21 Femme, lui dit Jésus, crois-moi, l'heure vient où ce ne sera ni sur cette montagne ni à Jérusalem que vous adorerez le Père.
\par 22 Vous adorez ce que vous ne connaissez pas; nous, nous adorons ce que nous connaissons, car le salut vient des Juifs.
\par 23 Mais l'heure vient, et elle est déjà venue, où les vrais adorateurs adoreront le Père en esprit et en vérité; car ce sont là les adorateurs que le Père demande.
\par 24 Dieu est Esprit, et il faut que ceux qui l'adorent l'adorent en esprit et en vérité.
\par 25 La femme lui dit: Je sais que le Messie doit venir (celui qu'on appelle Christ); quand il sera venu, il nous annoncera toutes choses.
\par 26 Jésus lui dit: Je le suis, moi qui te parle.
\par 27 Là-dessus arrivèrent ses disciples, qui furent étonnés de ce qu'il parlait avec une femme. Toutefois aucun ne dit: Que demandes-tu? ou: De quoi parles-tu avec elle?
\par 28 Alors la femme, ayant laissé sa cruche, s'en alla dans la ville, et dit aux gens:
\par 29 Venez voir un homme qui m'a dit tout ce que j'ai fait; ne serait-ce point le Christ?
\par 30 Ils sortirent de la ville, et ils vinrent vers lui.
\par 31 Pendant ce temps, les disciples le pressaient de manger, disant: Rabbi, mange.
\par 32 Mais il leur dit: J'ai à manger une nourriture que vous ne connaissez pas.
\par 33 Les disciples se disaient donc les uns aux autres: Quelqu'un lui aurait-il apporté à manger?
\par 34 Jésus leur dit: Ma nourriture est de faire la volonté de celui qui m'a envoyé, et d'accomplir son oeuvre.
\par 35 Ne dites-vous pas qu'il y a encore quatre mois jusqu'à la moisson? Voici, je vous le dis, levez les yeux, et regardez les champs qui déjà blanchissent pour la moisson.
\par 36 Celui qui moissonne reçoit un salaire, et amasse des fruits pour la vie éternelle, afin que celui qui sème et celui qui moissonne se réjouissent ensemble.
\par 37 Car en ceci ce qu'on dit est vrai: Autre est celui qui sème, et autre celui qui moissonne.
\par 38 Je vous ai envoyés moissonner ce que vous n'avez pas travaillé; d'autres ont travaillé, et vous êtes entrés dans leur travail.
\par 39 Plusieurs Samaritains de cette ville crurent en Jésus à cause de cette déclaration formelle de la femme: Il m'a dit tout ce que j'ai fait.
\par 40 Aussi, quand les Samaritains vinrent le trouver, ils le prièrent de rester auprès d'eux. Et il resta là deux jours.
\par 41 Un beaucoup plus grand nombre crurent à cause de sa parole;
\par 42 et ils disaient à la femme: Ce n'est plus à cause de ce que tu as dit que nous croyons; car nous l'avons entendu nous-mêmes, et nous savons qu'il est vraiment le Sauveur du monde.
\par 43 Après ces deux jours, Jésus partit de là, pour se rendre en Galilée;
\par 44 car il avait déclaré lui-même qu'un prophète n'est pas honoré dans sa propre patrie.
\par 45 Lorsqu'il arriva en Galilée, il fut bien reçu des Galiléens, qui avaient vu tout ce qu'il avait fait à Jérusalem pendant la fête; car eux aussi étaient allés à la fête.
\par 46 Il retourna donc à Cana en Galilée, où il avait changé l'eau en vin. Il y avait à Capernaüm un officier du roi, dont le fils était malade.
\par 47 Ayant appris que Jésus était venu de Judée en Galilée, il alla vers lui, et le pria de descendre et de guérir son fils, qui était près de mourir.
\par 48 Jésus lui dit: Si vous ne voyez des miracles et des prodiges, vous ne croyez point.
\par 49 L'officier du roi lui dit: Seigneur, descends avant que mon enfant meure.
\par 50 Va, lui dit Jésus, ton fils vit. Et cet homme crut à la parole que Jésus lui avait dite, et il s'en alla.
\par 51 Comme déjà il descendait, ses serviteurs venant à sa rencontre, lui apportèrent cette nouvelle: Ton enfant vit.
\par 52 Il leur demanda à quelle heure il s'était trouvé mieux; et ils lui dirent: Hier, à la septième heure, la fièvre l'a quitté.
\par 53 Le père reconnut que c'était à cette heure-là que Jésus lui avait dit: Ton fils vit. Et il crut, lui et toute sa maison.
\par 54 Jésus fit encore ce second miracle lorsqu'il fut venu de Judée en Galilée.

\chapter{5}

\par 1 Après cela, il y eut une fête des Juifs, et Jésus monta à Jérusalem.
\par 2 Or, à Jérusalem, près de la porte des brebis, il y a une piscine qui s'appelle en hébreu Béthesda, et qui a cinq portiques.
\par 3 Sous ces portiques étaient couchés en grand nombre des malades, des aveugles, des boiteux, des paralytiques, qui attendaient le mouvement de l'eau;
\par 4 car un ange descendait de temps en temps dans la piscine, et agitait l'eau; et celui qui y descendait le premier après que l'eau avait été agitée était guéri, quelle que fût sa maladie.
\par 5 Là se trouvait un homme malade depuis trente-huit ans.
\par 6 Jésus, l'ayant vu couché, et sachant qu'il était malade depuis longtemps, lui dit: Veux-tu être guéri?
\par 7 Le malade lui répondit: Seigneur, je n'ai personne pour me jeter dans la piscine quand l'eau est agitée, et, pendant que j'y vais, un autre descend avant moi.
\par 8 Lève-toi, lui dit Jésus, prends ton lit, et marche.
\par 9 Aussitôt cet homme fut guéri; il prit son lit, et marcha.
\par 10 C'était un jour de sabbat. Les Juifs dirent donc à celui qui avait été guéri: C'est le sabbat; il ne t'est pas permis d'emporter ton lit.
\par 11 Il leur répondit: Celui qui m'a guéri m'a dit: Prends ton lit, et marche.
\par 12 Ils lui demandèrent: Qui est l'homme qui t'a dit: Prends ton lit, et marche?
\par 13 Mais celui qui avait été guéri ne savait pas qui c'était; car Jésus avait disparu de la foule qui était en ce lieu.
\par 14 Depuis, Jésus le trouva dans le temple, et lui dit: Voici, tu as été guéri; ne pèche plus, de peur qu'il ne t'arrive quelque chose de pire.
\par 15 Cet homme s'en alla, et annonça aux Juifs que c'était Jésus qui l'avait guéri.
\par 16 C'est pourquoi les Juifs poursuivaient Jésus, parce qu'il faisait ces choses le jour du sabbat.
\par 17 Mais Jésus leur répondit: Mon Père agit jusqu'à présent; moi aussi, j'agis.
\par 18 A cause de cela, les Juifs cherchaient encore plus à le faire mourir, non seulement parce qu'il violait le sabbat, mais parce qu'il appelait Dieu son propre Père, se faisant lui-même égal à Dieu.
\par 19 Jésus reprit donc la parole, et leur dit: En vérité, en vérité, je vous le dis, le Fils ne peut rien faire de lui-même, il ne fait que ce qu'il voit faire au Père; et tout ce que le Père fait, le Fils aussi le fait pareillement.
\par 20 Car le Père aime le Fils, et lui montre tout ce qu'il fait; et il lui montrera des oeuvres plus grandes que celles-ci, afin que vous soyez dans l'étonnement.
\par 21 Car, comme le Père ressuscite les morts et donne la vie, ainsi le Fils donne la vie à qui il veut.
\par 22 Le Père ne juge personne, mais il a remis tout jugement au Fils,
\par 23 afin que tous honorent le Fils comme ils honorent le Père. Celui qui n'honore pas le Fils n'honore pas le Père qui l'a envoyé.
\par 24 En vérité, en vérité, je vous le dis, celui qui écoute ma parole, et qui croit à celui qui m'a envoyé, a la vie éternelle et ne vient point en jugement, mais il est passé de la mort à la vie.
\par 25 En vérité, en vérité, je vous le dis, l'heure vient, et elle est déjà venue, où les morts entendront la voix du Fils de Dieu; et ceux qui l'auront entendue vivront.
\par 26 Car, comme le Père a la vie en lui-même, ainsi il a donné au Fils d'avoir la vie en lui-même.
\par 27 Et il lui a donné le pouvoir de juger, parce qu'il est Fils de l'homme.
\par 28 Ne vous étonnez pas de cela; car l'heure vient où tous ceux qui sont dans les sépulcres entendront sa voix, et en sortiront.
\par 29 Ceux qui auront fait le bien ressusciteront pour la vie, mais ceux qui auront fait le mal ressusciteront pour le jugement.
\par 30 Je ne puis rien faire de moi-même: selon que j'entends, je juge; et mon jugement est juste, parce que je ne cherche pas ma volonté, mais la volonté de celui qui m'a envoyé.
\par 31 Si c'est moi qui rends témoignage de moi-même, mon témoignage n'est pas vrai.
\par 32 Il y en a un autre qui rend témoignage de moi, et je sais que le témoignage qu'il rend de moi est vrai.
\par 33 Vous avez envoyé vers Jean, et il a rendu témoignage à la vérité.
\par 34 Pour moi ce n'est pas d'un homme que je reçois le témoignage; mais je dis ceci, afin que vous soyez sauvés.
\par 35 Jean était la lampe qui brûle et qui luit, et vous avez voulu vous réjouir une heure à sa lumière.
\par 36 Moi, j'ai un témoignage plus grand que celui de Jean; car les oeuvres que le Père m'a donné d'accomplir, ces oeuvres mêmes que je fais, témoignent de moi que c'est le Père qui m'a envoyé.
\par 37 Et le Père qui m'a envoyé a rendu lui-même témoignage de moi. Vous n'avez jamais entendu sa voix, vous n'avez point vu sa face,
\par 38 et sa parole ne demeure point en vous, parce que vous ne croyez pas à celui qu'il a envoyé.
\par 39 Vous sondez les Écritures, parce que vous pensez avoir en elles la vie éternelle: ce sont elles qui rendent témoignage de moi.
\par 40 Et vous ne voulez pas venir à moi pour avoir la vie!
\par 41 Je ne tire pas ma gloire des hommes.
\par 42 Mais je sais que vous n'avez point en vous l'amour de Dieu.
\par 43 Je suis venu au nom de mon Père, et vous ne me recevez pas; si un autre vient en son propre nom, vous le recevrez.
\par 44 Comment pouvez-vous croire, vous qui tirez votre gloire les uns des autres, et qui ne cherchez point la gloire qui vient de Dieu seul?
\par 45 Ne pensez pas que moi je vous accuserai devant le Père; celui qui vous accuse, c'est Moïse, en qui vous avez mis votre espérance.
\par 46 Car si vous croyiez Moïse, vous me croiriez aussi, parce qu'il a écrit de moi.
\par 47 Mais si vous ne croyez pas à ses écrits, comment croirez-vous à mes paroles?

\chapter{6}

\par 1 Après cela, Jésus s'en alla de l'autre côté de la mer de Galilée, de Tibériade.
\par 2 Une grande foule le suivait, parce qu'elle voyait les miracles qu'il opérait sur les malades.
\par 3 Jésus monta sur la montagne, et là il s'assit avec ses disciples.
\par 4 Or, la Pâque était proche, la fête des Juifs.
\par 5 Ayant levé les yeux, et voyant qu'une grande foule venait à lui, Jésus dit à Philippe: Où achèterons-nous des pains, pour que ces gens aient à manger?
\par 6 Il disait cela pour l'éprouver, car il savait ce qu'il allait faire.
\par 7 Philippe lui répondit: Les pains qu'on aurait pour deux cents deniers ne suffiraient pas pour que chacun en reçût un peu.
\par 8 Un de ses disciples, André, frère de Simon Pierre, lui dit:
\par 9 Il y a ici un jeune garçon qui a cinq pains d'orge et deux poissons; mais qu'est-ce que cela pour tant de gens?
\par 10 Jésus dit: Faites-les asseoir. Il y avait dans ce lieu beaucoup d'herbe. Ils s'assirent donc, au nombre d'environ cinq mille hommes.
\par 11 Jésus prit les pains, rendit grâces, et les distribua à ceux qui étaient assis; il leur donna de même des poissons, autant qu'ils en voulurent.
\par 12 Lorsqu'ils furent rassasiés, il dit à ses disciples: Ramassez les morceaux qui restent, afin que rien ne se perde.
\par 13 Ils les ramassèrent donc, et ils remplirent douze paniers avec les morceaux qui restèrent des cinq pains d'orge, après que tous eurent mangé.
\par 14 Ces gens, ayant vu le miracle que Jésus avait fait, disaient: Celui-ci est vraiment le prophète qui doit venir dans le monde.
\par 15 Et Jésus, sachant qu'ils allaient venir l'enlever pour le faire roi, se retira de nouveau sur la montagne, lui seul.
\par 16 Quand le soir fut venu, ses disciples descendirent au bord de la mer.
\par 17 Étant montés dans une barque, ils traversaient la mer pour se rendre à Capernaüm. Il faisait déjà nuit, et Jésus ne les avait pas encore rejoints.
\par 18 Il soufflait un grand vent, et la mer était agitée.
\par 19 Après avoir ramé environ vingt-cinq ou trente stades, ils virent Jésus marchant sur la mer et s'approchant de la barque. Et ils eurent peur.
\par 20 Mais Jésus leur dit: C'est moi; n'ayez pas peur!
\par 21 Ils voulaient donc le prendre dans la barque, et aussitôt la barque aborda au lieu où ils allaient.
\par 22 La foule qui était restée de l'autre côté de la mer avait remarqué qu'il ne se trouvait là qu'une seule barque, et que Jésus n'était pas monté dans cette barque avec ses disciples, mais qu'ils étaient partis seuls.
\par 23 Le lendemain, comme d'autres barques étaient arrivées de Tibériade près du lieu où ils avaient mangé le pain après que le Seigneur eut rendu grâces,
\par 24 les gens de la foule, ayant vu que ni Jésus ni ses disciples n'étaient là, montèrent eux-mêmes dans ces barques et allèrent à Capernaüm à la recherche de Jésus.
\par 25 Et l'ayant trouvé au delà de la mer, ils lui dirent: Rabbi, quand es-tu venu ici?
\par 26 Jésus leur répondit: En vérité, en vérité, je vous le dis, vous me cherchez, non parce que vous avez vu des miracles, mais parce que vous avez mangé des pains et que vous avez été rassasiés.
\par 27 Travaillez, non pour la nourriture qui périt, mais pour celle qui subsiste pour la vie éternelle, et que le Fils de l'homme vous donnera; car c'est lui que le Père, que Dieu a marqué de son sceau.
\par 28 Ils lui dirent: Que devons-nous faire, pour faire les oeuvres de Dieu?
\par 29 Jésus leur répondit: L'oeuvre de Dieu, c'est que vous croyiez en celui qu'il a envoyé.
\par 30 Quel miracle fais-tu donc, lui dirent-ils, afin que nous le voyions, et que nous croyions en toi? Que fais-tu?
\par 31 Nos pères ont mangé la manne dans le désert, selon ce qui est écrit: Il leur donna le pain du ciel à manger.
\par 32 Jésus leur dit: En vérité, en vérité, je vous le dis, Moïse ne vous a pas donné le pain du ciel, mais mon Père vous donne le vrai pain du ciel;
\par 33 car le pain de Dieu, c'est celui qui descend du ciel et qui donne la vie au monde.
\par 34 Ils lui dirent: Seigneur, donne-nous toujours ce pain.
\par 35 Jésus leur dit: Je suis le pain de vie. Celui qui vient à moi n'aura jamais faim, et celui qui croit en moi n'aura jamais soif.
\par 36 Mais, je vous l'ai dit, vous m'avez vu, et vous ne croyez point.
\par 37 Tous ceux que le Père me donne viendront à moi, et je ne mettrai pas dehors celui qui vient à moi;
\par 38 car je suis descendu du ciel pour faire, non ma volonté, mais la volonté de celui qui m'a envoyé.
\par 39 Or, la volonté de celui qui m'a envoyé, c'est que je ne perde rien de tout ce qu'il m'a donné, mais que je le ressuscite au dernier jour.
\par 40 La volonté de mon Père, c'est que quiconque voit le Fils et croit en lui ait la vie éternelle; et je le ressusciterai au dernier jour.
\par 41 Les Juifs murmuraient à son sujet, parce qu'il avait dit: Je suis le pain qui est descendu du ciel.
\par 42 Et ils disaient: N'est-ce pas là Jésus, le fils de Joseph, celui dont nous connaissons le père et la mère? Comment donc dit-il: Je suis descendu du ciel?
\par 43 Jésus leur répondit: Ne murmurez pas entre vous.
\par 44 Nul ne peut venir à moi, si le Père qui m'a envoyé ne l'attire; et je le ressusciterai au dernier jour.
\par 45 Il est écrit dans les prophètes: Ils seront tous enseignés de Dieu. Ainsi quiconque a entendu le Père et a reçu son enseignement vient à moi.
\par 46 C'est que nul n'a vu le Père, sinon celui qui vient de Dieu; celui-là a vu le Père.
\par 47 En vérité, en vérité, je vous le dis, celui qui croit en moi a la vie éternelle.
\par 48 Je suis le pain de vie.
\par 49 Vos pères ont mangé la manne dans le désert, et ils sont morts.
\par 50 C'est ici le pain qui descend du ciel, afin que celui qui en mange ne meure point.
\par 51 Je suis le pain vivant qui est descendu du ciel. Si quelqu'un mange de ce pain, il vivra éternellement; et le pain que je donnerai, c'est ma chair, que je donnerai pour la vie du monde.
\par 52 Là-dessus, les Juifs disputaient entre eux, disant: Comment peut-il nous donner sa chair à manger?
\par 53 Jésus leur dit: En vérité, en vérité, je vous le dis, si vous ne mangez la chair du Fils de l'homme, et si vous ne buvez son sang, vous n'avez point la vie en vous-mêmes.
\par 54 Celui qui mange ma chair et qui boit mon sang a la vie éternelle; et je le ressusciterai au dernier jour.
\par 55 Car ma chair est vraiment une nourriture, et mon sang est vraiment un breuvage.
\par 56 Celui qui mange ma chair et qui boit mon sang demeure en moi, et je demeure en lui.
\par 57 Comme le Père qui est vivant m'a envoyé, et que je vis par le Père, ainsi celui qui me mange vivra par moi.
\par 58 C'est ici le pain qui est descendu du ciel. Il n'en est pas comme de vos pères qui ont mangé la manne et qui sont morts: celui qui mange ce pain vivra éternellement.
\par 59 Jésus dit ces choses dans la synagogue, enseignant à Capernaüm.
\par 60 Plusieurs de ses disciples, après l'avoir entendu, dirent: Cette parole est dure; qui peut l'écouter?
\par 61 Jésus, sachant en lui-même que ses disciples murmuraient à ce sujet, leur dit: Cela vous scandalise-t-il?
\par 62 Et si vous voyez le Fils de l'homme monter où il était auparavant?...
\par 63 C'est l'esprit qui vivifie; la chair ne sert de rien. Les paroles que je vous ai dites sont esprit et vie.
\par 64 Mais il en est parmi vous quelques-uns qui ne croient point. Car Jésus savait dès le commencement qui étaient ceux qui ne croyaient point, et qui était celui qui le livrerait.
\par 65 Et il ajouta: C'est pourquoi je vous ai dit que nul ne peut venir à moi, si cela ne lui a été donné par le Père.
\par 66 Dès ce moment, plusieurs de ses disciples se retirèrent, et ils n'allaient plus avec lui.
\par 67 Jésus donc dit aux douze: Et vous, ne voulez-vous pas aussi vous en aller?
\par 68 Simon Pierre lui répondit: Seigneur, à qui irions-nous? Tu as les paroles de la vie éternelle.
\par 69 Et nous avons cru et nous avons connu que tu es le Christ, le Saint de Dieu.
\par 70 Jésus leur répondit: N'est-ce pas moi qui vous ai choisis, vous les douze? Et l'un de vous est un démon!
\par 71 Il parlait de Judas Iscariot, fils de Simon; car c'était lui qui devait le livrer, lui, l'un des douze.

\chapter{7}

\par 1 Après cela, Jésus parcourait la Galilée, car il ne voulait pas séjourner en Judée, parce que les Juifs cherchaient à le faire mourir.
\par 2 Or, la fête des Juifs, la fête des Tabernacles, était proche.
\par 3 Et ses frères lui dirent: Pars d'ici, et va en Judée, afin que tes disciples voient aussi les oeuvres que tu fais.
\par 4 Personne n'agit en secret, lorsqu'il désire paraître: si tu fais ces choses, montre-toi toi-même au monde.
\par 5 Car ses frères non plus ne croyaient pas en lui.
\par 6 Jésus leur dit: Mon temps n'est pas encore venu, mais votre temps est toujours prêt.
\par 7 Le monde ne peut vous haïr; moi, il me hait, parce que je rends de lui le témoignage que ses oeuvres sont mauvaises.
\par 8 Montez, vous, à cette fête; pour moi, je n'y monte point, parce que mon temps n'est pas encore accompli.
\par 9 Après leur avoir dit cela, il resta en Galilée.
\par 10 Lorsque ses frères furent montés à la fête, il y monta aussi lui-même, non publiquement, mais comme en secret.
\par 11 Les Juifs le cherchaient pendant la fête, et disaient: Où est-il?
\par 12 Il y avait dans la foule grande rumeur à son sujet. Les uns disaient: C'est un homme de bien. D'autres disaient: Non, il égare la multitude.
\par 13 Personne, toutefois, ne parlait librement de lui, par crainte des Juifs.
\par 14 Vers le milieu de la fête, Jésus monta au temple. Et il enseignait.
\par 15 Les Juifs s'étonnaient, disant: Comment connaît-il les Écritures, lui qui n'a point étudié?
\par 16 Jésus leur répondit: Ma doctrine n'est pas de moi, mais de celui qui m'a envoyé.
\par 17 Si quelqu'un veut faire sa volonté, il connaîtra si ma doctrine est de Dieu, ou si je parle de mon chef.
\par 18 Celui qui parle de son chef cherche sa propre gloire; mais celui qui cherche la gloire de celui qui l'a envoyé, celui-là est vrai, et il n'y a point d'injustice en lui.
\par 19 Moïse ne vous a-t-il pas donné la loi? Et nul de vous n'observe la loi. Pourquoi cherchez-vous à me faire mourir?
\par 20 La foule répondit: Tu as un démon. Qui est-ce qui cherche à te faire mourir?
\par 21 Jésus leur répondit: J'ai fait une oeuvre, et vous en êtes tous étonnés.
\par 22 Moïse vous a donné la circoncision, -non qu'elle vienne de Moïse, car elle vient des patriarches, -et vous circoncisez un homme le jour du sabbat.
\par 23 Si un homme reçoit la circoncision le jour du sabbat, afin que la loi de Moïse ne soit pas violée, pourquoi vous irritez-vous contre moi de ce que j'ai guéri un homme tout entier le jour du sabbat?
\par 24 Ne jugez pas selon l'apparence, mais jugez selon la justice.
\par 25 Quelques habitants de Jérusalem disaient: N'est-ce pas là celui qu'ils cherchent à faire mourir?
\par 26 Et voici, il parle librement, et ils ne lui disent rien! Est-ce que vraiment les chefs auraient reconnu qu'il est le Christ?
\par 27 Cependant celui-ci, nous savons d'où il est; mais le Christ, quand il viendra, personne ne saura d'où il est.
\par 28 Et Jésus, enseignant dans le temple, s'écria: Vous me connaissez, et vous savez d'où je suis! Je ne suis pas venu de moi-même: mais celui qui m'a envoyé est vrai, et vous ne le connaissez pas.
\par 29 Moi, je le connais; car je viens de lui, et c'est lui qui m'a envoyé.
\par 30 Ils cherchaient donc à se saisir de lui, et personne ne mit la main sur lui, parce que son heure n'était pas encore venue.
\par 31 Plusieurs parmi la foule crurent en lui, et ils disaient: Le Christ, quand il viendra, fera-t-il plus de miracles que n'en a fait celui-ci?
\par 32 Les pharisiens entendirent la foule murmurant de lui ces choses. Alors les principaux sacrificateurs et les pharisiens envoyèrent des huissiers pour le saisir.
\par 33 Jésus dit: Je suis encore avec vous pour un peu de temps, puis je m'en vais vers celui qui m'a envoyé.
\par 34 Vous me chercherez et vous ne me trouverez pas, et vous ne pouvez venir où je serai.
\par 35 Sur quoi les Juifs dirent entre eux: Où ira-t-il, que nous ne le trouvions pas? Ira-t-il parmi ceux qui sont dispersés chez les Grecs, et enseignera-t-il les Grecs?
\par 36 Que signifie cette parole qu'il a dite: Vous me chercherez et vous ne me trouverez pas, et vous ne pouvez venir où je serai?
\par 37 Le dernier jour, le grand jour de la fête, Jésus, se tenant debout, s'écria: Si quelqu'un a soif, qu'il vienne à moi, et qu'il boive.
\par 38 Celui qui croit en moi, des fleuves d'eau vive couleront de son sein, comme dit l'Écriture.
\par 39 Il dit cela de l'Esprit que devaient recevoir ceux qui croiraient en lui; car l'Esprit n'était pas encore, parce que Jésus n'avait pas encore été glorifié.
\par 40 Des gens de la foule, ayant entendu ces paroles, disaient: Celui-ci est vraiment le prophète.
\par 41 D'autres disaient: C'est le Christ. Et d'autres disaient: Est-ce bien de la Galilée que doit venir le Christ?
\par 42 L'Écriture ne dit-elle pas que c'est de la postérité de David, et du village de Bethléhem, où était David, que le Christ doit venir?
\par 43 Il y eut donc, à cause de lui, division parmi la foule.
\par 44 Quelques-uns d'entre eux voulaient le saisir, mais personne ne mit la main sur lui.
\par 45 Ainsi les huissiers retournèrent vers les principaux sacrificateurs et les pharisiens. Et ceux-ci leur dirent: Pourquoi ne l'avez-vous pas amené?
\par 46 Les huissiers répondirent: Jamais homme n'a parlé comme cet homme.
\par 47 Les pharisiens leur répliquèrent: Est-ce que vous aussi, vous avez été séduits?
\par 48 Y a-t-il quelqu'un des chefs ou des pharisiens qui ait cru en lui?
\par 49 Mais cette foule qui ne connaît pas la loi, ce sont des maudits!
\par 50 Nicodème, qui était venu de nuit vers Jésus, et qui était l'un d'entre eux, leur dit:
\par 51 Notre loi condamne-t-elle un homme avant qu'on l'entende et qu'on sache ce qu'il a fait?
\par 52 Ils lui répondirent: Es-tu aussi Galiléen? Examine, et tu verras que de la Galilée il ne sort point de prophète.
\par 53 Et chacun s'en retourna dans sa maison.

\chapter{8}

\par 1 Jésus se rendit à la montagne des oliviers.
\par 2 Mais, dès le matin, il alla de nouveau dans le temple, et tout le peuple vint à lui. S'étant assis, il les enseignait.
\par 3 Alors les scribes et les pharisiens amenèrent une femme surprise en adultère;
\par 4 et, la plaçant au milieu du peuple, ils dirent à Jésus: Maître, cette femme a été surprise en flagrant délit d'adultère.
\par 5 Moïse, dans la loi, nous a ordonné de lapider de telles femmes: toi donc, que dis-tu?
\par 6 Ils disaient cela pour l'éprouver, afin de pouvoir l'accuser. Mais Jésus, s'étant baissé, écrivait avec le doigt sur la terre.
\par 7 Comme ils continuaient à l'interroger, il se releva et leur dit: Que celui de vous qui est sans péché jette le premier la pierre contre elle.
\par 8 Et s'étant de nouveau baissé, il écrivait sur la terre.
\par 9 Quand ils entendirent cela, accusés par leur conscience, ils se retirèrent un à un, depuis les plus âgés jusqu'aux derniers; et Jésus resta seul avec la femme qui était là au milieu.
\par 10 Alors s'étant relevé, et ne voyant plus que la femme, Jésus lui dit: Femme, où sont ceux qui t'accusaient? Personne ne t'a-t-il condamnée?
\par 11 Elle répondit: Non, Seigneur. Et Jésus lui dit: Je ne te condamne pas non plus: va, et ne pèche plus.
\par 12 Jésus leur parla de nouveau, et dit: Je suis la lumière du monde; celui qui me suit ne marchera pas dans les ténèbres, mais il aura la lumière de la vie.
\par 13 Là-dessus, les pharisiens lui dirent: Tu rends témoignage de toi-même; ton témoignage n'est pas vrai.
\par 14 Jésus leur répondit: Quoique je rende témoignage de moi-même, mon témoignage est vrai, car je sais d'où je suis venu et où je vais; mais vous, vous ne savez d'où je viens ni où je vais.
\par 15 Vous jugez selon la chair; moi, je ne juge personne.
\par 16 Et si je juge, mon jugement est vrai, car je ne suis pas seul; mais le Père qui m'a envoyé est avec moi.
\par 17 Il est écrit dans votre loi que le témoignage de deux hommes est vrai;
\par 18 je rends témoignage de moi-même, et le Père qui m'a envoyé rend témoignage de moi.
\par 19 Ils lui dirent donc: Où est ton Père? Jésus répondit: Vous ne connaissez ni moi, ni mon Père. Si vous me connaissiez, vous connaîtriez aussi mon Père.
\par 20 Jésus dit ces paroles, enseignant dans le temple, au lieu où était le trésor; et personne ne le saisit, parce que son heure n'était pas encore venue.
\par 21 Jésus leur dit encore: Je m'en vais, et vous me chercherez, et vous mourrez dans votre péché; vous ne pouvez venir où je vais.
\par 22 Sur quoi les Juifs dirent: Se tuera-t-il lui-même, puisqu'il dit: Vous ne pouvez venir où je vais?
\par 23 Et il leur dit: Vous êtes d'en bas; moi, je suis d'en haut. Vous êtes de ce monde; moi, je ne suis pas de ce monde.
\par 24 C'est pourquoi je vous ai dit que vous mourrez dans vos péchés; car si vous ne croyez pas ce que je suis, vous mourrez dans vos péchés.
\par 25 Qui es-tu? lui dirent-ils. Jésus leur répondit: Ce que je vous dis dès le commencement.
\par 26 J'ai beaucoup de choses à dire de vous et à juger en vous; mais celui qui m'a envoyé est vrai, et ce que j'ai entendu de lui, je le dis au monde.
\par 27 Ils ne comprirent point qu'il leur parlait du Père.
\par 28 Jésus donc leur dit: Quand vous aurez élevé le Fils de l'homme, alors vous connaîtrez ce que je suis, et que je ne fais rien de moi-même, mais que je parle selon ce que le Père m'a enseigné.
\par 29 Celui qui m'a envoyé est avec moi; il ne m'a pas laissé seul, parce que je fais toujours ce qui lui est agréable.
\par 30 Comme Jésus parlait ainsi, plusieurs crurent en lui.
\par 31 Et il dit aux Juifs qui avaient cru en lui: Si vous demeurez dans ma parole, vous êtes vraiment mes disciples;
\par 32 vous connaîtrez la vérité, et la vérité vous affranchira.
\par 33 Ils lui répondirent: Nous sommes la postérité d'Abraham, et nous ne fûmes jamais esclaves de personne; comment dis-tu: Vous deviendrez libres?
\par 34 En vérité, en vérité, je vous le dis, leur répliqua Jésus, quiconque se livre au péché est esclave du péché.
\par 35 Or, l'esclave ne demeure pas toujours dans la maison; le fils y demeure toujours.
\par 36 Si donc le Fils vous affranchit, vous serez réellement libres.
\par 37 Je sais que vous êtes la postérité d'Abraham; mais vous cherchez à me faire mourir, parce que ma parole ne pénètre pas en vous.
\par 38 Je dis ce que j'ai vu chez mon Père; et vous, vous faites ce que vous avez entendu de la part de votre père.
\par 39 Ils lui répondirent: Notre père, c'est Abraham. Jésus leur dit: Si vous étiez enfants d'Abraham, vous feriez les oeuvres d'Abraham.
\par 40 Mais maintenant vous cherchez à me faire mourir, moi qui vous ai dit la vérité que j'ai entendue de Dieu. Cela, Abraham ne l'a point fait.
\par 41 Vous faites les oeuvres de votre père. Ils lui dirent: Nous ne sommes pas des enfants illégitimes; nous avons un seul Père, Dieu.
\par 42 Jésus leur dit: Si Dieu était votre Père, vous m'aimeriez, car c'est de Dieu que je suis sorti et que je viens; je ne suis pas venu de moi-même, mais c'est lui qui m'a envoyé.
\par 43 Pourquoi ne comprenez-vous pas mon langage? Parce que vous ne pouvez écouter ma parole.
\par 44 Vous avez pour père le diable, et vous voulez accomplir les désirs de votre père. Il a été meurtrier dès le commencement, et il ne se tient pas dans la vérité, parce qu'il n'y a pas de vérité en lui. Lorsqu'il profère le mensonge, il parle de son propre fonds; car il est menteur et le père du mensonge.
\par 45 Et moi, parce que je dis la vérité, vous ne me croyez pas.
\par 46 Qui de vous me convaincra de péché? Si je dis la vérité, pourquoi ne me croyez-vous pas?
\par 47 Celui qui est de Dieu, écoute les paroles de Dieu; vous n'écoutez pas, parce que vous n'êtes pas de Dieu.
\par 48 Les Juifs lui répondirent: N'avons-nous pas raison de dire que tu es un Samaritain, et que tu as un démon?
\par 49 Jésus répliqua: Je n'ai point de démon; mais j'honore mon Père, et vous m'outragez.
\par 50 Je ne cherche point ma gloire; il en est un qui la cherche et qui juge.
\par 51 En vérité, en vérité, je vous le dis, si quelqu'un garde ma parole, il ne verra jamais la mort.
\par 52 Maintenant, lui dirent les Juifs, nous connaissons que tu as un démon. Abraham est mort, les prophètes aussi, et tu dis: Si quelqu'un garde ma parole, il ne verra jamais la mort.
\par 53 Es-tu plus grand que notre père Abraham, qui est mort? Les prophètes aussi sont morts. Qui prétends-tu être?
\par 54 Jésus répondit: Si je me glorifie moi-même, ma gloire n'est rien. C'est mon père qui me glorifie, lui que vous dites être votre Dieu,
\par 55 et que vous ne connaissez pas. Pour moi, je le connais; et, si je disais que je ne le connais pas, je serais semblable à vous, un menteur. Mais je le connais, et je garde sa parole.
\par 56 Abraham, votre père, a tressailli de joie de ce qu'il verrait mon jour: il l'a vu, et il s'est réjoui.
\par 57 Les Juifs lui dirent: Tu n'as pas encore cinquante ans, et tu as vu Abraham!
\par 58 Jésus leur dit: En vérité, en vérité, je vous le dis, avant qu'Abraham fût, je suis.
\par 59 Là-dessus, ils prirent des pierres pour les jeter contre lui; mais Jésus se cacha, et il sortit du temple.

\chapter{9}

\par 1 Jésus vit, en passant, un homme aveugle de naissance.
\par 2 Ses disciples lui firent cette question: Rabbi, qui a péché, cet homme ou ses parents, pour qu'il soit né aveugle?
\par 3 Jésus répondit: Ce n'est pas que lui ou ses parents aient péché; mais c'est afin que les oeuvres de Dieu soient manifestées en lui.
\par 4 Il faut que je fasse, tandis qu'il est jour, les oeuvres de celui qui m'a envoyé; la nuit vient, où personne ne peut travailler.
\par 5 Pendant que je suis dans le monde, je suis la lumière du monde.
\par 6 Après avoir dit cela, il cracha à terre, et fit de la boue avec sa salive. Puis il appliqua cette boue sur les yeux de l'aveugle,
\par 7 et lui dit: Va, et lave-toi au réservoir de Siloé (nom qui signifie envoyé). Il y alla, se lava, et s'en retourna voyant clair.
\par 8 Ses voisins et ceux qui auparavant l'avaient connu comme un mendiant disaient: N'est-ce pas là celui qui se tenait assis et qui mendiait?
\par 9 Les uns disaient: C'est lui. D'autres disaient: Non, mais il lui ressemble. Et lui-même disait: C'est moi.
\par 10 Ils lui dirent donc: Comment tes yeux ont-ils été ouverts?
\par 11 Il répondit: L'Homme qu'on appelle Jésus a fait de la boue, a oint mes yeux, et m'a dit: Va au réservoir de Siloé, et lave-toi. J'y suis allé, je me suis lavé, et j'ai recouvré la vue.
\par 12 Ils lui dirent: Où est cet homme? Il répondit: Je ne sais.
\par 13 Ils menèrent vers les pharisiens celui qui avait été aveugle.
\par 14 Or, c'était un jour de sabbat que Jésus avait fait de la boue, et lui avait ouvert les yeux.
\par 15 De nouveau, les pharisiens aussi lui demandèrent comment il avait recouvré la vue. Et il leur dit: Il a appliqué de la boue sur mes yeux, je me suis lavé, et je vois.
\par 16 Sur quoi quelques-uns des pharisiens dirent: Cet homme ne vient pas de Dieu, car il n'observe pas le sabbat. D'autres dirent: Comment un homme pécheur peut-il faire de tels miracles?
\par 17 Et il y eut division parmi eux. Ils dirent encore à l'aveugle: Toi, que dis-tu de lui, sur ce qu'il t'a ouvert les yeux? Il répondit: C'est un prophète.
\par 18 Les Juifs ne crurent point qu'il eût été aveugle et qu'il eût recouvré la vue jusqu'à ce qu'ils eussent fait venir ses parents.
\par 19 Et ils les interrogèrent, disant: Est-ce là votre fils, que vous dites être né aveugle? Comment donc voit-il maintenant?
\par 20 Ses parents répondirent: Nous savons que c'est notre fils, et qu'il est né aveugle;
\par 21 mais comment il voit maintenant, ou qui lui a ouvert les yeux, c'est ce que nous ne savons. Interrogez-le lui-même, il a de l'âge, il parlera de ce qui le concerne.
\par 22 Ses parents dirent cela parce qu'ils craignaient les Juifs; car les Juifs étaient déjà convenus que, si quelqu'un reconnaissait Jésus pour le Christ, il serait exclu de la synagogue.
\par 23 C'est pourquoi ses parents dirent: Il a de l'âge, interrogez-le lui-même.
\par 24 Les pharisiens appelèrent une seconde fois l'homme qui avait été aveugle, et ils lui dirent: Donne gloire à Dieu; nous savons que cet homme est un pécheur.
\par 25 Il répondit: S'il est un pécheur, je ne sais; je sais une chose, c'est que j'étais aveugle et que maintenant je vois.
\par 26 Ils lui dirent: Que t'a-t-il fait? Comment t'a-t-il ouvert les yeux?
\par 27 Il leur répondit: Je vous l'ai déjà dit, et vous n'avez pas écouté; pourquoi voulez-vous l'entendre encore? Voulez-vous aussi devenir ses disciples?
\par 28 Ils l'injurièrent et dirent: C'est toi qui es son disciple; nous, nous sommes disciples de Moïse.
\par 29 Nous savons que Dieu a parlé à Moïse; mais celui-ci, nous ne savons d'où il est.
\par 30 Cet homme leur répondit: Il est étonnant que vous ne sachiez d'où il est; et cependant il m'a ouvert les yeux.
\par 31 Nous savons que Dieu n'exauce point les pécheurs; mais, si quelqu'un l'honore et fait sa volonté, c'est celui là qu'il l'exauce.
\par 32 Jamais on n'a entendu dire que quelqu'un ait ouvert les yeux d'un aveugle-né.
\par 33 Si cet homme ne venait pas de Dieu, il ne pourrait rien faire.
\par 34 Ils lui répondirent: Tu es né tout entier dans le péché, et tu nous enseignes! Et ils le chassèrent.
\par 35 Jésus apprit qu'ils l'avaient chassé; et, l'ayant rencontré, il lui dit: Crois-tu au Fils de Dieu?
\par 36 Il répondit: Et qui est-il, Seigneur, afin que je croie en lui?
\par 37 Tu l'as vu, lui dit Jésus, et celui qui te parle, c'est lui.
\par 38 Et il dit: Je crois, Seigneur. Et il se prosterna devant lui.
\par 39 Puis Jésus dit: Je suis venu dans ce monde pour un jugement, pour que ceux qui ne voient point voient, et que ceux qui voient deviennent aveugles.
\par 40 Quelques pharisiens qui étaient avec lui, ayant entendu ces paroles, lui dirent: Nous aussi, sommes-nous aveugles?
\par 41 Jésus leur répondit: Si vous étiez aveugles, vous n'auriez pas de péché. Mais maintenant vous dites: Nous voyons. C'est pour cela que votre péché subsiste.

\chapter{10}

\par 1 En vérité, en vérité, je vous le dis, celui qui n'entre pas par la porte dans la bergerie, mais qui y monte par ailleurs, est un voleur et un brigand.
\par 2 Mais celui qui entre par la porte est le berger des brebis.
\par 3 Le portier lui ouvre, et les brebis entendent sa voix; il appelle par leur nom les brebis qui lui appartiennent, et il les conduit dehors.
\par 4 Lorsqu'il a fait sortir toutes ses propres brebis, il marche devant elles; et les brebis le suivent, parce qu'elles connaissent sa voix.
\par 5 Elles ne suivront point un étranger; mais elles fuiront loin de lui, parce qu'elles ne connaissent pas la voix des étrangers.
\par 6 Jésus leur dit cette parabole, mais ils ne comprirent pas de quoi il leur parlait.
\par 7 Jésus leur dit encore: En vérité, en vérité, je vous le dis, je suis la porte des brebis.
\par 8 Tous ceux qui sont venus avant moi sont des voleurs et des brigands; mais les brebis ne les ont point écoutés.
\par 9 Je suis la porte. Si quelqu'un entre par moi, il sera sauvé; il entrera et il sortira, et il trouvera des pâturages.
\par 10 Le voleur ne vient que pour dérober, égorger et détruire; moi, je suis venu afin que les brebis aient la vie, et qu'elles soient dans l'abondance.
\par 11 Je suis le bon berger. Le bon berger donne sa vie pour ses brebis.
\par 12 Mais le mercenaire, qui n'est pas le berger, et à qui n'appartiennent pas les brebis, voit venir le loup, abandonne les brebis, et prend la fuite; et le loup les ravit et les disperse.
\par 13 Le mercenaire s'enfuit, parce qu'il est mercenaire, et qu'il ne se met point en peine des brebis. Je suis le bon berger.
\par 14 Je connais mes brebis, et elles me connaissent,
\par 15 comme le Père me connaît et comme je connais le Père; et je donne ma vie pour mes brebis.
\par 16 J'ai encore d'autres brebis, qui ne sont pas de cette bergerie; celles-là, il faut que je les amène; elles entendront ma voix, et il y aura un seul troupeau, un seul berger.
\par 17 Le Père m'aime, parce que je donne ma vie, afin de la reprendre.
\par 18 Personne ne me l'ôte, mais je la donne de moi-même; j'ai le pouvoir de la donner, et j'ai le pouvoir de la reprendre: tel est l'ordre que j'ai reçu de mon Père.
\par 19 Il y eut de nouveau, à cause de ces paroles, division parmi les Juifs.
\par 20 Plusieurs d'entre eux disaient: Il a un démon, il est fou; pourquoi l'écoutez-vous?
\par 21 D'autres disaient: Ce ne sont pas les paroles d'un démoniaque; un démon peut-il ouvrir les yeux des aveugles?
\par 22 On célébrait à Jérusalem la fête de la Dédicace. C'était l'hiver.
\par 23 Et Jésus se promenait dans le temple, sous le portique de Salomon.
\par 24 Les Juifs l'entourèrent, et lui dirent: Jusques à quand tiendras-tu notre esprit en suspens? Si tu es le Christ, dis-le nous franchement.
\par 25 Jésus leur répondit: Je vous l'ai dit, et vous ne croyez pas. Les oeuvres que je fais au nom de mon Père rendent témoignage de moi.
\par 26 Mais vous ne croyez pas, parce que vous n'êtes pas de mes brebis.
\par 27 Mes brebis entendent ma voix; je les connais, et elles me suivent.
\par 28 Je leur donne la vie éternelle; et elles ne périront jamais, et personne ne les ravira de ma main.
\par 29 Mon Père, qui me les a données, est plus grand que tous; et personne ne peut les ravir de la main de mon Père.
\par 30 Moi et le Père nous sommes un.
\par 31 Alors les Juifs prirent de nouveau des pierres pour le lapider.
\par 32 Jésus leur dit: Je vous ai fait voir plusieurs bonnes oeuvres venant de mon Père: pour laquelle me lapidez-vous?
\par 33 Les Juifs lui répondirent: Ce n'est point pour une bonne oeuvre que nous te lapidons, mais pour un blasphème, et parce que toi, qui es un homme, tu te fais Dieu.
\par 34 Jésus leur répondit: N'est-il pas écrit dans votre loi: J'ai dit: Vous êtes des dieux?
\par 35 Si elle a appelé dieux ceux à qui la parole de Dieu a été adressée, et si l'Écriture ne peut être anéantie,
\par 36 celui que le Père a sanctifié et envoyé dans le monde, vous lui dites: Tu blasphèmes! Et cela parce que j'ai dit: Je suis le Fils de Dieu.
\par 37 Si je ne fais pas les oeuvres de mon Père, ne me croyez pas.
\par 38 Mais si je les fais, quand même vous ne me croyez point, croyez à ces oeuvres, afin que vous sachiez et reconnaissiez que le Père est en moi et que je suis dans le Père.
\par 39 Là-dessus, ils cherchèrent encore à le saisir, mais il s'échappa de leurs mains.
\par 40 Jésus s'en alla de nouveau au delà du Jourdain, dans le lieu où Jean avait d'abord baptisé. Et il y demeura.
\par 41 Beaucoup de gens vinrent à lui, et ils disaient: Jean n'a fait aucun miracle; mais tout ce que Jean a dit de cet homme était vrai.
\par 42 Et, dans ce lieu-là, plusieurs crurent en lui.

\chapter{11}

\par 1 Il y avait un homme malade, Lazare, de Béthanie, village de Marie et de Marthe, sa soeur.
\par 2 C'était cette Marie qui oignit de parfum le Seigneur et qui lui essuya les pieds avec ses cheveux, et c'était son frère Lazare qui était malade.
\par 3 Les soeurs envoyèrent dire à Jésus: Seigneur, voici, celui que tu aimes est malade.
\par 4 Après avoir entendu cela, Jésus dit: Cette maladie n'est point à la mort; mais elle est pour la gloire de Dieu, afin que le Fils de Dieu soit glorifié par elle.
\par 5 Or, Jésus aimait Marthe, et sa soeur, et Lazare.
\par 6 Lors donc qu'il eut appris que Lazare était malade, il resta deux jours encore dans le lieu où il était,
\par 7 et il dit ensuite aux disciples: Retournons en Judée.
\par 8 Les disciples lui dirent: Rabbi, les Juifs tout récemment cherchaient à te lapider, et tu retournes en Judée!
\par 9 Jésus répondit: N'y a-t-il pas douze heures au jour? Si quelqu'un marche pendant le jour, il ne bronche point, parce qu'il voit la lumière de ce monde;
\par 10 mais, si quelqu'un marche pendant la nuit, il bronche, parce que la lumière n'est pas en lui.
\par 11 Après ces paroles, il leur dit: Lazare, notre ami, dort; mais je vais le réveiller.
\par 12 Les disciples lui dirent: Seigneur, s'il dort, il sera guéri.
\par 13 Jésus avait parlé de sa mort, mais ils crurent qu'il parlait de l'assoupissement du sommeil.
\par 14 Alors Jésus leur dit ouvertement: Lazare est mort.
\par 15 Et, à cause de vous, afin que vous croyiez, je me réjouis de ce que je n'étais pas là. Mais allons vers lui.
\par 16 Sur quoi Thomas, appelé Didyme, dit aux autres disciples: Allons aussi, afin de mourir avec lui.
\par 17 Jésus, étant arrivé, trouva que Lazare était déjà depuis quatre jours dans le sépulcre.
\par 18 Et, comme Béthanie était près de Jérusalem, à quinze stades environ,
\par 19 beaucoup de Juifs étaient venus vers Marthe et Marie, pour les consoler de la mort de leur frère.
\par 20 Lorsque Marthe apprit que Jésus arrivait, elle alla au-devant de lui, tandis que Marie se tenait assise à la maison.
\par 21 Marthe dit à Jésus: Seigneur, si tu eusses été ici, mon frère ne serait pas mort.
\par 22 Mais, maintenant même, je sais que tout ce que tu demanderas à Dieu, Dieu te l'accordera.
\par 23 Jésus lui dit: Ton frère ressuscitera.
\par 24 Je sais, lui répondit Marthe, qu'il ressuscitera à la résurrection, au dernier jour.
\par 25 Jésus lui dit: Je suis la résurrection et la vie. Celui qui croit en moi vivra, quand même il serait mort;
\par 26 et quiconque vit et croit en moi ne mourra jamais. Crois-tu cela?
\par 27 Elle lui dit: Oui, Seigneur, je crois que tu es le Christ, le Fils de Dieu, qui devait venir dans le monde.
\par 28 Ayant ainsi parlé, elle s'en alla. Puis elle appela secrètement Marie, sa soeur, et lui dit: Le maître est ici, et il te demande.
\par 29 Dès que Marie eut entendu, elle se leva promptement, et alla vers lui.
\par 30 Car Jésus n'était pas encore entré dans le village, mais il était dans le lieu où Marthe l'avait rencontré.
\par 31 Les Juifs qui étaient avec Marie dans la maison et qui la consolaient, l'ayant vue se lever promptement et sortir, la suivirent, disant: Elle va au sépulcre, pour y pleurer.
\par 32 Lorsque Marie fut arrivée là où était Jésus, et qu'elle le vit, elle tomba à ses pieds, et lui dit: Seigneur, si tu eusses été ici, mon frère ne serait pas mort.
\par 33 Jésus, la voyant pleurer, elle et les Juifs qui étaient venus avec elle, frémit en son esprit, et fut tout ému.
\par 34 Et il dit: Où l'avez-vous mis? Seigneur, lui répondirent-ils, viens et vois.
\par 35 Jésus pleura.
\par 36 Sur quoi les Juifs dirent: Voyez comme il l'aimait.
\par 37 Et quelques-uns d'entre eux dirent: Lui qui a ouvert les yeux de l'aveugle, ne pouvait-il pas faire aussi que cet homme ne mourût point?
\par 38 Jésus frémissant de nouveau en lui-même, se rendit au sépulcre. C'était une grotte, et une pierre était placée devant.
\par 39 Jésus dit: Otez la pierre. Marthe, la soeur du mort, lui dit: Seigneur, il sent déjà, car il y a quatre jours qu'il est là.
\par 40 Jésus lui dit: Ne t'ai-je pas dit que, si tu crois, tu verras la gloire de Dieu?
\par 41 Ils ôtèrent donc la pierre. Et Jésus leva les yeux en haut, et dit: Père, je te rends grâces de ce que tu m'as exaucé.
\par 42 Pour moi, je savais que tu m'exauces toujours; mais j'ai parlé à cause de la foule qui m'entoure, afin qu'ils croient que c'est toi qui m'as envoyé.
\par 43 Ayant dit cela, il cria d'une voix forte: Lazare, sors!
\par 44 Et le mort sortit, les pieds et les mains liés de bandes, et le visage enveloppé d'un linge. Jésus leur dit: Déliez-le, et laissez-le aller.
\par 45 Plusieurs des Juifs qui étaient venus vers Marie, et qui virent ce que fit Jésus, crurent en lui.
\par 46 Mais quelques-uns d'entre eux allèrent trouver les pharisiens, et leur dirent ce que Jésus avait fait.
\par 47 Alors les principaux sacrificateurs et les pharisiens assemblèrent le sanhédrin, et dirent: Que ferons-nous? Car cet homme fait beaucoup de miracles.
\par 48 Si nous le laissons faire, tous croiront en lui, et les Romains viendront détruire et notre ville et notre nation.
\par 49 L'un d'eux, Caïphe, qui était souverain sacrificateur cette année-là, leur dit: Vous n'y entendez rien;
\par 50 vous ne réfléchissez pas qu'il est dans votre intérêt qu'un seul homme meure pour le peuple, et que la nation entière ne périsse pas.
\par 51 Or, il ne dit pas cela de lui-même; mais étant souverain sacrificateur cette année-là, il prophétisa que Jésus devait mourir pour la nation.
\par 52 Et ce n'était pas pour la nation seulement; c'était aussi afin de réunir en un seul corps les enfants de Dieu dispersés.
\par 53 Dès ce jour, ils résolurent de le faire mourir.
\par 54 C'est pourquoi Jésus ne se montra plus ouvertement parmi les Juifs; mais il se retira dans la contrée voisine du désert, dans une ville appelée Éphraïm; et là il demeurait avec ses disciples.
\par 55 La Pâque des Juifs était proche. Et beaucoup de gens du pays montèrent à Jérusalem avant la Pâque, pour se purifier.
\par 56 Ils cherchaient Jésus, et ils se disaient les uns aux autres dans le temple: Que vous en semble? Ne viendra-t-il pas à la fête?
\par 57 Or, les principaux sacrificateurs et les pharisiens avaient donné l'ordre que, si quelqu'un savait où il était, il le déclarât, afin qu'on se saisît de lui.

\chapter{12}

\par 1 Six jours avant la Pâque, Jésus arriva à Béthanie, où était Lazare, qu'il avait ressuscité des morts.
\par 2 Là, on lui fit un souper; Marthe servait, et Lazare était un de ceux qui se trouvaient à table avec lui.
\par 3 Marie, ayant pris une livre d'un parfum de nard pur de grand prix, oignit les pieds de Jésus, et elle lui essuya les pieds avec ses cheveux; et la maison fut remplie de l'odeur du parfum.
\par 4 Un de ses disciples, Judas Iscariot, fils de Simon, celui qui devait le livrer, dit:
\par 5 Pourquoi n'a-t-on pas vendu ce parfum trois cent deniers, pour les donner aux pauvres?
\par 6 Il disait cela, non qu'il se mît en peine des pauvres, mais parce qu'il était voleur, et que, tenant la bourse, il prenait ce qu'on y mettait.
\par 7 Mais Jésus dit: Laisse-la garder ce parfum pour le jour de ma sépulture.
\par 8 Vous avez toujours les pauvres avec vous, mais vous ne m'avez pas toujours.
\par 9 Une grande multitude de Juifs apprirent que Jésus était à Béthanie; et ils y vinrent, non pas seulement à cause de lui, mais aussi pour voir Lazare, qu'il avait ressuscité des morts.
\par 10 Les principaux sacrificateurs délibérèrent de faire mourir aussi Lazare,
\par 11 parce que beaucoup de Juifs se retiraient d'eux à cause de lui, et croyaient en Jésus.
\par 12 Le lendemain, une foule nombreuse de gens venus à la fête ayant entendu dire que Jésus se rendait à Jérusalem,
\par 13 prirent des branches de palmiers, et allèrent au-devant de lui, en criant: Hosanna! Béni soit celui qui vient au nom du Seigneur, le roi d'Israël!
\par 14 Jésus trouva un ânon, et s'assit dessus, selon ce qui est écrit:
\par 15 Ne crains point, fille de Sion; Voici, ton roi vient, Assis sur le petit d'une ânesse.
\par 16 Ses disciples ne comprirent pas d'abord ces choses; mais, lorsque Jésus eut été glorifié, ils se souvinrent qu'elles étaient écrites de lui, et qu'il les avaient été accomplies à son égard.
\par 17 Tous ceux qui étaient avec Jésus, quand il appela Lazare du sépulcre et le ressuscita des morts, lui rendaient témoignage;
\par 18 et la foule vint au-devant de lui, parce qu'elle avait appris qu'il avait fait ce miracle.
\par 19 Les pharisiens se dirent donc les uns aux autres: Vous voyez que vous ne gagnez rien; voici, le monde est allé après lui.
\par 20 Quelques Grecs, du nombre de ceux qui étaient montés pour adorer pendant la fête,
\par 21 s'adressèrent à Philippe, de Bethsaïda en Galilée, et lui dirent avec instance: Seigneur, nous voudrions voir Jésus.
\par 22 Philippe alla le dire à André, puis André et Philippe le dirent à Jésus.
\par 23 Jésus leur répondit: L'heure est venue où le Fils de l'homme doit être glorifié.
\par 24 En vérité, en vérité, je vous le dis, si le grain de blé qui est tombé en terre ne meurt, il reste seul; mais, s'il meurt, il porte beaucoup de fruit.
\par 25 Celui qui aime sa vie la perdra, et celui qui hait sa vie dans ce monde la conservera pour la vie éternelle.
\par 26 Si quelqu'un me sert, qu'il me suive; et là où je suis, là aussi sera mon serviteur. Si quelqu'un me sert, le Père l'honorera.
\par 27 Maintenant mon âme est troublée. Et que dirais-je?... Père, délivre-moi de cette heure?... Mais c'est pour cela que je suis venu jusqu'à cette heure.
\par 28 Père, glorifie ton nom! Et une voix vint du ciel: Je l'ai glorifié, et je le glorifierai encore.
\par 29 La foule qui était là, et qui avait entendu, disait que c'était un tonnerre. D'autres disaient: Un ange lui a parlé.
\par 30 Jésus dit: Ce n'est pas à cause de moi que cette voix s'est fait entendre; c'est à cause de vous.
\par 31 Maintenant a lieu le jugement de ce monde; maintenant le prince de ce monde sera jeté dehors.
\par 32 Et moi, quand j'aurai été élevé de la terre, j'attirerai tous les hommes à moi.
\par 33 En parlant ainsi, il indiquait de quelle mort il devait mourir. -
\par 34 La foule lui répondit: Nous avons appris par la loi que le Christ demeure éternellement; comment donc dis-tu: Il faut que le Fils de l'homme soit élevé? Qui est ce Fils de l'homme?
\par 35 Jésus leur dit: La lumière est encore pour un peu de temps au milieu de vous. Marchez, pendant que vous avez la lumière, afin que les ténèbres ne vous surprennent point: celui qui marche dans les ténèbres ne sait où il va.
\par 36 Pendant que vous avez la lumière, croyez en la lumière, afin que vous soyez des enfants de lumière. Jésus dit ces choses, puis il s'en alla, et se cacha loin d'eux.
\par 37 Malgré tant de miracles qu'il avait faits en leur présence, ils ne croyaient pas en lui,
\par 38 afin que s'accomplît la parole qu'Ésaïe, le prophète, a prononcée: Seigneur, Qui a cru à notre prédication? Et à qui le bras du Seigneur a-t-il été révélé?
\par 39 Aussi ne pouvaient-ils croire, parce qu'Ésaïe a dit encore:
\par 40 Il a aveuglé leurs yeux; et il a endurci leur coeur, De peur qu'ils ne voient des yeux, Qu'ils ne comprennent du coeur, Qu'ils ne se convertissent, et que je ne les guérisse.
\par 41 Ésaïe dit ces choses, lorsqu'il vit sa gloire, et qu'il parla de lui.
\par 42 Cependant, même parmi les chefs, plusieurs crurent en lui; mais, à cause des pharisiens, ils n'en faisaient pas l'aveu, dans la crainte d'être exclus de la synagogue.
\par 43 Car ils aimèrent la gloire des hommes plus que la gloire de Dieu.
\par 44 Or, Jésus s'était écrié: Celui qui croit en moi croit, non pas en moi, mais en celui qui m'a envoyé;
\par 45 et celui qui me voit voit celui qui m'a envoyé.
\par 46 Je suis venu comme une lumière dans le monde, afin que quiconque croit en moi ne demeure pas dans les ténèbres.
\par 47 Si quelqu'un entend mes paroles et ne les garde point, ce n'est pas moi qui le juge; car je suis venu non pour juger le monde, mais pour sauver le monde.
\par 48 Celui qui me rejette et qui ne reçoit pas mes paroles a son juge; la parole que j'ai annoncée, c'est elle qui le jugera au dernier jour.
\par 49 Car je n'ai point parlé de moi-même; mais le Père, qui m'a envoyé, m'a prescrit lui-même ce que je dois dire et annoncer.
\par 50 Et je sais que son commandement est la vie éternelle. C'est pourquoi les choses que je dis, je les dis comme le Père me les a dites.

\chapter{13}

\par 1 Avant la fête de Pâque, Jésus, sachant que son heure était venue de passer de ce monde au Père, et ayant aimé les siens qui étaient dans le monde, mit le comble à son amour pour eux.
\par 2 Pendant le souper, lorsque le diable avait déjà inspiré au coeur de Judas Iscariot, fils de Simon, le dessein de le livrer,
\par 3 Jésus, qui savait que le Père avait remis toutes choses entre ses mains, qu'il était venu de Dieu, et qu'il s'en allait à Dieu,
\par 4 se leva de table, ôta ses vêtements, et prit un linge, dont il se ceignit.
\par 5 Ensuite il versa de l'eau dans un bassin, et il se mit à laver les pieds des disciples, et à les essuyer avec le linge dont il était ceint.
\par 6 Il vint donc à Simon Pierre; et Pierre lui dit: Toi, Seigneur, tu me laves les pieds!
\par 7 Jésus lui répondit: Ce que je fais, tu ne le comprends pas maintenant, mais tu le comprendras bientôt.
\par 8 Pierre lui dit: Non, jamais tu ne me laveras les pieds. Jésus lui répondit: Si je ne te lave, tu n'auras point de part avec moi.
\par 9 Simon Pierre lui dit: Seigneur, non seulement les pieds, mais encore les mains et la tête.
\par 10 Jésus lui dit: Celui qui est lavé n'a besoin que de se laver les pieds pour être entièrement pur; et vous êtes purs, mais non pas tous.
\par 11 Car il connaissait celui qui le livrait; c'est pourquoi il dit: Vous n'êtes pas tous purs.
\par 12 Après qu'il leur eut lavé les pieds, et qu'il eut pris ses vêtements, il se remit à table, et leur dit: Comprenez-vous ce que je vous ai fait?
\par 13 Vous m'appelez Maître et Seigneur; et vous dites bien, car je le suis.
\par 14 Si donc je vous ai lavé les pieds, moi, le Seigneur et le Maître, vous devez aussi vous laver les pieds les uns aux autres;
\par 15 car je vous ai donné un exemple, afin que vous fassiez comme je vous ai fait.
\par 16 En vérité, en vérité, je vous le dis, le serviteur n'est pas plus grand que son seigneur, ni l'apôtre plus grand que celui qui l'a envoyé.
\par 17 Si vous savez ces choses, vous êtes heureux, pourvu que vous les pratiquiez.
\par 18 Ce n'est pas de vous tous que je parle; je connais ceux que j'ai choisis. Mais il faut que l'Écriture s'accomplisse: Celui qui mange avec moi le pain A levé son talon contre moi.
\par 19 Dès à présent je vous le dis, avant que la chose arrive, afin que, lorsqu'elle arrivera, vous croyiez à ce que je suis.
\par 20 En vérité, en vérité, je vous le dis, celui qui reçoit celui que j'aurai envoyé me reçoit, et celui qui me reçoit, reçoit celui qui m'a envoyé.
\par 21 Ayant ainsi parlé, Jésus fut troublé en son esprit, et il dit expressément: En vérité, en vérité, je vous le dis, l'un de vous me livrera.
\par 22 Les disciples se regardaient les uns les autres, ne sachant de qui il parlait.
\par 23 Un des disciples, celui que Jésus aimait, était couché sur le sein de Jésus.
\par 24 Simon Pierre lui fit signe de demander qui était celui dont parlait Jésus.
\par 25 Et ce disciple, s'étant penché sur la poitrine de Jésus, lui dit: Seigneur, qui est-ce?
\par 26 Jésus répondit: C'est celui à qui je donnerai le morceau trempé. Et, ayant trempé le morceau, il le donna à Judas, fils de Simon, l'Iscariot.
\par 27 Dès que le morceau fut donné, Satan entra dans Judas. Jésus lui dit: Ce que tu fais, fais-le promptement.
\par 28 Mais aucun de ceux qui étaient à table ne comprit pourquoi il lui disait cela;
\par 29 car quelques-uns pensaient que, comme Judas avait la bourse, Jésus voulait lui dire: Achète ce dont nous avons besoin pour la fête, ou qu'il lui commandait de donner quelque chose aux pauvres.
\par 30 Judas, ayant pris le morceau, se hâta de sortir. Il était nuit.
\par 31 Lorsque Judas fut sorti, Jésus dit: Maintenant, le Fils de l'homme a été glorifié, et Dieu a été glorifié en lui.
\par 32 Si Dieu a été glorifié en lui, Dieu aussi le glorifiera en lui-même, et il le glorifiera bientôt.
\par 33 Mes petits enfants, je suis pour peu de temps encore avec vous. Vous me chercherez; et, comme j'ai dit aux Juifs: Vous ne pouvez venir où je vais, je vous le dis aussi maintenant.
\par 34 Je vous donne un commandement nouveau: Aimez-vous les uns les autres; comme je vous ai aimés, vous aussi, aimez-vous les uns les autres.
\par 35 A ceci tous connaîtront que vous êtes mes disciples, si vous avez de l'amour les uns pour les autres.
\par 36 Simon Pierre lui dit: Seigneur, où vas-tu? Jésus répondit: Tu ne peux pas maintenant me suivre où je vais, mais tu me suivras plus tard.
\par 37 Seigneur, lui dit Pierre, pourquoi ne puis-je pas te suivre maintenant? Je donnerai ma vie pour toi.
\par 38 Jésus répondit: Tu donneras ta vie pour moi! En vérité, en vérité, je te le dis, le coq ne chantera pas que tu ne m'aies renié trois fois.

\chapter{14}

\par 1 Que votre coeur ne se trouble point. Croyez en Dieu, et croyez en moi.
\par 2 Il y a plusieurs demeures dans la maison de mon Père. Si cela n'était pas, je vous l'aurais dit. Je vais vous préparer une place.
\par 3 Et, lorsque je m'en serai allé, et que je vous aurai préparé une place, je reviendrai, et je vous prendrai avec moi, afin que là où je suis vous y soyez aussi.
\par 4 Vous savez où je vais, et vous en savez le chemin.
\par 5 Thomas lui dit: Seigneur, nous ne savons où tu vas; comment pouvons-nous en savoir le chemin?
\par 6 Jésus lui dit: Je suis le chemin, la vérité, et la vie. Nul ne vient au Père que par moi.
\par 7 Si vous me connaissiez, vous connaîtriez aussi mon Père. Et dès maintenant vous le connaissez, et vous l'avez vu.
\par 8 Philippe lui dit: Seigneur, montre-nous le Père, et cela nous suffit.
\par 9 Jésus lui dit: Il y a si longtemps que je suis avec vous, et tu ne m'as pas connu, Philippe! Celui qui m'a vu a vu le Père; comment dis-tu: Montre-nous le Père?
\par 10 Ne crois-tu pas que je suis dans le Père, et que le Père est en moi? Les paroles que je vous dis, je ne les dis pas de moi-même; et le Père qui demeure en moi, c'est lui qui fait les oeuvres.
\par 11 Croyez-moi, je suis dans le Père, et le Père est en moi; croyez du moins à cause de ces oeuvres.
\par 12 En vérité, en vérité, je vous le dis, celui qui croit en moi fera aussi les oeuvres que je fais, et il en fera de plus grandes, parce que je m'en vais au Père;
\par 13 et tout ce que vous demanderez en mon nom, je le ferai, afin que le Père soit glorifié dans le Fils.
\par 14 Si vous demandez quelque chose en mon nom, je le ferai.
\par 15 Si vous m'aimez, gardez mes commandements.
\par 16 Et moi, je prierai le Père, et il vous donnera un autre consolateur, afin qu'il demeure éternellement avec vous,
\par 17 l'Esprit de vérité, que le monde ne peut recevoir, parce qu'il ne le voit point et ne le connaît point; mais vous, vous le connaissez, car il demeure avec vous, et il sera en vous.
\par 18 Je ne vous laisserai pas orphelins, je viendrai à vous.
\par 19 Encore un peu de temps, et le monde ne me verra plus; mais vous, vous me verrez, car je vis, et vous vivrez aussi.
\par 20 En ce jour-là, vous connaîtrez que je suis en mon Père, que vous êtes en moi, et que je suis en vous.
\par 21 Celui qui a mes commandements et qui les garde, c'est celui qui m'aime; et celui qui m'aime sera aimé de mon Père, je l'aimerai, et je me ferai connaître à lui.
\par 22 Jude, non pas l'Iscariot, lui dit: Seigneur, d'où vient que tu te feras connaître à nous, et non au monde?
\par 23 Jésus lui répondit: Si quelqu'un m'aime, il gardera ma parole, et mon Père l'aimera; nous viendrons à lui, et nous ferons notre demeure chez lui.
\par 24 Celui qui ne m'aime pas ne garde point mes paroles. Et la parole que vous entendez n'est pas de moi, mais du Père qui m'a envoyé.
\par 25 Je vous ai dit ces choses pendant que je demeure avec vous.
\par 26 Mais le consolateur, l'Esprit Saint, que le Père enverra en mon nom, vous enseignera toutes choses, et vous rappellera tout ce que je vous ai dit.
\par 27 Je vous laisse la paix, je vous donne ma paix. Je ne vous donne pas comme le monde donne. Que votre coeur ne se trouble point, et ne s'alarme point.
\par 28 Vous avez entendu que je vous ai dit: Je m'en vais, et je reviens vers vous. Si vous m'aimiez, vous vous réjouiriez de ce que je vais au Père; car le Père est plus grand que moi.
\par 29 Et maintenant je vous ai dit ces choses avant qu'elles arrivent, afin que, lorsqu'elles arriveront, vous croyiez.
\par 30 Je ne parlerai plus guère avec vous; car le prince du monde vient. Il n'a rien en moi;
\par 31 mais afin que le monde sache que j'aime le Père, et que j'agis selon l'ordre que le Père m'a donné, levez-vous, partons d'ici.

\chapter{15}

\par 1 Je suis le vrai cep, et mon Père est le vigneron.
\par 2 Tout sarment qui est en moi et qui ne porte pas de fruit, il le retranche; et tout sarment qui porte du fruit, il l'émonde, afin qu'il porte encore plus de fruit.
\par 3 Déjà vous êtes purs, à cause de la parole que je vous ai annoncée.
\par 4 Demeurez en moi, et je demeurerai en vous. Comme le sarment ne peut de lui-même porter du fruit, s'il ne demeure attaché au cep, ainsi vous ne le pouvez non plus, si vous ne demeurez en moi.
\par 5 Je suis le cep, vous êtes les sarments. Celui qui demeure en moi et en qui je demeure porte beaucoup de fruit, car sans moi vous ne pouvez rien faire.
\par 6 Si quelqu'un ne demeure pas en moi, il est jeté dehors, comme le sarment, et il sèche; puis on ramasse les sarments, on les jette au feu, et ils brûlent.
\par 7 Si vous demeurez en moi, et que mes paroles demeurent en vous, demandez ce que vous voudrez, et cela vous sera accordé.
\par 8 Si vous portez beaucoup de fruit, c'est ainsi que mon Père sera glorifié, et que vous serez mes disciples.
\par 9 Comme le Père m'a aimé, je vous ai aussi aimés. Demeurez dans mon amour.
\par 10 Si vous gardez mes commandements, vous demeurerez dans mon amour, de même que j'ai gardé les commandements de mon Père, et que je demeure dans son amour.
\par 11 Je vous ai dit ces choses, afin que ma joie soit en vous, et que votre joie soit parfaite.
\par 12 C'est ici mon commandement: Aimez-vous les uns les autres, comme je vous ai aimés.
\par 13 Il n'y a pas de plus grand amour que de donner sa vie pour ses amis.
\par 14 Vous êtes mes amis, si vous faites ce que je vous commande.
\par 15 Je ne vous appelle plus serviteurs, parce que le serviteur ne sait pas ce que fait son maître; mais je vous ai appelés amis, parce que je vous ai fait connaître tout ce que j'ai appris de mon Père.
\par 16 Ce n'est pas vous qui m'avez choisi; mais moi, je vous ai choisis, et je vous ai établis, afin que vous alliez, et que vous portiez du fruit, et que votre fruit demeure, afin que ce que vous demanderez au Père en mon nom, il vous le donne.
\par 17 Ce que je vous commande, c'est de vous aimer les uns les autres.
\par 18 Si le monde vous hait, sachez qu'il m'a haï avant vous.
\par 19 Si vous étiez du monde, le monde aimerait ce qui est à lui; mais parce que vous n'êtes pas du monde, et que je vous ai choisis du milieu du monde, à cause de cela le monde vous hait.
\par 20 Souvenez-vous de la parole que je vous ai dite: Le serviteur n'est pas plus grand que son maître. S'ils m'ont persécuté, ils vous persécuteront aussi; s'ils ont gardé ma parole, ils garderont aussi la vôtre.
\par 21 Mais ils vous feront toutes ces choses à cause de mon nom, parce qu'ils ne connaissent pas celui qui m'a envoyé.
\par 22 Si je n'étais pas venu et que je ne leur eusses point parlé, ils n'auraient pas de péché; mais maintenant ils n'ont aucune excuse de leur péché.
\par 23 Celui qui me hait, hait aussi mon Père.
\par 24 Si je n'avais pas fait parmi eux des oeuvres que nul autre n'a faites, ils n'auraient pas de péché; mais maintenant ils les ont vues, et ils ont haï et moi et mon Père.
\par 25 Mais cela est arrivé afin que s'accomplît la parole qui est écrite dans leur loi: Ils m'ont haï sans cause.
\par 26 Quand sera venu le consolateur, que je vous enverrai de la part du Père, l'Esprit de vérité, qui vient du Père, il rendra témoignage de moi;
\par 27 et vous aussi, vous rendrez témoignage, parce que vous êtes avec moi dès le commencement.

\chapter{16}

\par 1 Je vous ai dit ces choses, afin qu'elles ne soient pas pour vous une occasion de chute.
\par 2 Ils vous excluront des synagogues; et même l'heure vient où quiconque vous fera mourir croira rendre un culte à Dieu.
\par 3 Et ils agiront ainsi, parce qu'ils n'ont connu ni le Père ni moi.
\par 4 Je vous ai dit ces choses, afin que, lorsque l'heure sera venue, vous vous souveniez que je vous les ai dites. Je ne vous en ai pas parlé dès le commencement, parce que j'étais avec vous.
\par 5 Maintenant je m'en vais vers celui qui m'a envoyé, et aucun de vous ne me demande: Où vas-tu?
\par 6 Mais, parce que je vous ai dit ces choses, la tristesse a rempli votre coeur.
\par 7 Cependant je vous dis la vérité: il vous est avantageux que je m'en aille, car si je ne m'en vais pas, le consolateur ne viendra pas vers vous; mais, si je m'en vais, je vous l'enverrai.
\par 8 Et quand il sera venu, il convaincra le monde en ce qui concerne le péché, la justice, et le jugement:
\par 9 en ce qui concerne le péché, parce qu'ils ne croient pas en moi;
\par 10 la justice, parce que je vais au Père, et que vous ne me verrez plus;
\par 11 le jugement, parce que le prince de ce monde est jugé.
\par 12 J'ai encore beaucoup de choses à vous dire, mais vous ne pouvez pas les porter maintenant.
\par 13 Quand le consolateur sera venu, l'Esprit de vérité, il vous conduira dans toute la vérité; car il ne parlera pas de lui-même, mais il dira tout ce qu'il aura entendu, et il vous annoncera les choses à venir.
\par 14 Il me glorifiera, parce qu'il prendra de ce qui est à moi, et vous l'annoncera.
\par 15 Tout ce que le Père a est à moi; c'est pourquoi j'ai dit qu'il prend de ce qui est à moi, et qu'il vous l'annoncera.
\par 16 Encore un peu de temps, et vous ne me verrez plus; et puis encore un peu de temps, et vous me verrez, parce que je vais au Père.
\par 17 Là-dessus, quelques-uns de ses disciples dirent entre eux: Que signifie ce qu'il nous dit: Encore un peu de temps, et vous ne me verrez plus; et puis encore un peu de temps, et vous me verrez? et: Parce que je vais au Père?
\par 18 Ils disaient donc: Que signifie ce qu'il dit: Encore un peu de temps? Nous ne savons de quoi il parle.
\par 19 Jésus, connut qu'ils voulaient l'interroger, leur dit: Vous vous questionnez les uns les autres sur ce que j'ai dit: Encore un peu de temps, et vous ne me verrez plus; et puis encore un peu de temps, et vous me verrez.
\par 20 En vérité, en vérité, je vous le dis, vous pleurerez et vous vous lamenterez, et le monde se réjouira: vous serez dans la tristesse, mais votre tristesse se changera en joie.
\par 21 La femme, lorsqu'elle enfante, éprouve de la tristesse, parce que son heure est venue; mais, lorsqu'elle a donné le jour à l'enfant, elle ne se souvient plus de la souffrance, à cause de la joie qu'elle a de ce qu'un homme est né dans le monde.
\par 22 Vous donc aussi, vous êtes maintenant dans la tristesse; mais je vous reverrai, et votre coeur se réjouira, et nul ne vous ravira votre joie.
\par 23 En ce jour-là, vous ne m'interrogerez plus sur rien. En vérité, en vérité, je vous le dis, ce que vous demanderez au Père, il vous le donnera en mon nom.
\par 24 Jusqu'à présent vous n'avez rien demandé en mon nom. Demandez, et vous recevrez, afin que votre joie soit parfaite.
\par 25 Je vous ai dit ces choses en paraboles. L'heure vient où je ne vous parlerai plus en paraboles, mais où je vous parlerai ouvertement du Père.
\par 26 En ce jour, vous demanderez en mon nom, et je ne vous dis pas que je prierai le Père pour vous;
\par 27 car le Père lui-même vous aime, parce que vous m'avez aimé, et que vous avez cru que je suis sorti de Dieu.
\par 28 Je suis sorti du Père, et je suis venu dans le monde; maintenant je quitte le monde, et je vais au Père.
\par 29 Ses disciples lui dirent: Voici, maintenant tu parles ouvertement, et tu n'emploies aucune parabole.
\par 30 Maintenant nous savons que tu sais toutes choses, et que tu n'as pas besoin que personne t'interroge; c'est pourquoi nous croyons que tu es sorti de Dieu.
\par 31 Jésus leur répondit: Vous croyez maintenant.
\par 32 Voici, l'heure vient, et elle est déjà venue, où vous serez dispersés chacun de son côté, et où vous me laisserez seul; mais je ne suis pas seul, car le Père est avec moi.
\par 33 Je vous ai dit ces choses, afin que vous ayez la paix en moi. Vous aurez des tribulations dans le monde; mais prenez courage, j'ai vaincu le monde.

\chapter{17}

\par 1 Après avoir ainsi parlé, Jésus leva les yeux au ciel, et dit: Père, l'heure est venue! Glorifie ton Fils, afin que ton Fils te glorifie,
\par 2 selon que tu lui as donné pouvoir sur toute chair, afin qu'il accorde la vie éternelle à tous ceux que tu lui as donnés.
\par 3 Or, la vie éternelle, c'est qu'ils te connaissent, toi, le seul vrai Dieu, et celui que tu as envoyé, Jésus Christ.
\par 4 Je t'ai glorifié sur la terre, j'ai achevé l'oeuvre que tu m'as donnée à faire.
\par 5 Et maintenant toi, Père, glorifie-moi auprès de toi-même de la gloire que j'avais auprès de toi avant que le monde fût.
\par 6 J'ai fait connaître ton nom aux hommes que tu m'as donnés du milieu du monde. Ils étaient à toi, et tu me les as donnés; et ils ont gardé ta parole.
\par 7 Maintenant ils ont connu que tout ce que tu m'as donné vient de toi.
\par 8 Car je leur ai donné les paroles que tu m'as données; et ils les ont reçues, et ils ont vraiment connu que je suis sorti de toi, et ils ont cru que tu m'as envoyé.
\par 9 C'est pour eux que je prie. Je ne prie pas pour le monde, mais pour ceux que tu m'as donnés, parce qu'ils sont à toi; -
\par 10 et tout ce qui est à moi est à toi, et ce qui est à toi est à moi; -et je suis glorifié en eux.
\par 11 Je ne suis plus dans le monde, et ils sont dans le monde, et je vais à toi. Père saint, garde en ton nom ceux que tu m'as donnés, afin qu'ils soient un comme nous.
\par 12 Lorsque j'étais avec eux dans le monde, je les gardais en ton nom. J'ai gardé ceux que tu m'as donnés, et aucun d'eux ne s'est perdu, sinon le fils de perdition, afin que l'Écriture fût accomplie.
\par 13 Et maintenant je vais à toi, et je dis ces choses dans le monde, afin qu'ils aient en eux ma joie parfaite.
\par 14 Je leur ai donné ta parole; et le monde les a haïs, parce qu'ils ne sont pas du monde, comme moi je ne suis pas du monde.
\par 15 Je ne te prie pas de les ôter du monde, mais de les préserver du mal.
\par 16 Ils ne sont pas du monde, comme moi je ne suis pas du monde.
\par 17 Sanctifie-les par ta vérité: ta parole est la vérité.
\par 18 Comme tu m'as envoyé dans le monde, je les ai aussi envoyés dans le monde.
\par 19 Et je me sanctifie moi-même pour eux, afin qu'eux aussi soient sanctifiés par la vérité.
\par 20 Ce n'est pas pour eux seulement que je prie, mais encore pour ceux qui croiront en moi par leur parole,
\par 21 afin que tous soient un, comme toi, Père, tu es en moi, et comme je suis en toi, afin qu'eux aussi soient un en nous, pour que le monde croie que tu m'as envoyé.
\par 22 Je leur ai donné la gloire que tu m'as donnée, afin qu'ils soient un comme nous sommes un, -
\par 23 moi en eux, et toi en moi, -afin qu'ils soient parfaitement un, et que le monde connaisse que tu m'as envoyé et que tu les as aimés comme tu m'as aimé.
\par 24 Père, je veux que là où je suis ceux que tu m'as donnés soient aussi avec moi, afin qu'ils voient ma gloire, la gloire que tu m'as donnée, parce que tu m'as aimé avant la fondation du monde.
\par 25 Père juste, le monde ne t'a point connu; mais moi je t'ai connu, et ceux-ci ont connu que tu m'as envoyé.
\par 26 Je leur ai fait connaître ton nom, et je le leur ferai connaître, afin que l'amour dont tu m'as aimé soit en eux, et que je sois en eux.

\chapter{18}

\par 1 Lorsqu'il eut dit ces choses, Jésus alla avec ses disciples de l'autre côté du torrent du Cédron, où se trouvait un jardin, dans lequel il entra, lui et ses disciples.
\par 2 Judas, qui le livrait, connaissait ce lieu, parce que Jésus et ses disciples s'y étaient souvent réunis.
\par 3 Judas donc, ayant pris la cohorte, et des huissiers qu'envoyèrent les principaux sacrificateurs et les pharisiens, vint là avec des lanternes, des flambeaux et des armes.
\par 4 Jésus, sachant tout ce qui devait lui arriver, s'avança, et leur dit: Qui cherchez-vous?
\par 5 Ils lui répondirent: Jésus de Nazareth. Jésus leur dit: C'est moi. Et Judas, qui le livrait, était avec eux.
\par 6 Lorsque Jésus leur eut dit: C'est moi, ils reculèrent et tombèrent par terre.
\par 7 Il leur demanda de nouveau: Qui cherchez-vous? Et ils dirent: Jésus de Nazareth.
\par 8 Jésus répondit: Je vous ai dit que c'est moi. Si donc c'est moi que vous cherchez, laissez aller ceux-ci.
\par 9 Il dit cela, afin que s'accomplît la parole qu'il avait dite: Je n'ai perdu aucun de ceux que tu m'as donnés.
\par 10 Simon Pierre, qui avait une épée, la tira, frappa le serviteur du souverain sacrificateur, et lui coupa l'oreille droite. Ce serviteur s'appelait Malchus.
\par 11 Jésus dit à Pierre: Remets ton épée dans le fourreau. Ne boirai-je pas la coupe que le Père m'a donnée à boire?
\par 12 La cohorte, le tribun, et les huissiers des Juifs, se saisirent alors de Jésus, et le lièrent.
\par 13 Ils l'emmenèrent d'abord chez Anne; car il était le beau-père de Caïphe, qui était souverain sacrificateur cette année-là.
\par 14 Et Caïphe était celui qui avait donné ce conseil aux Juifs: Il est avantageux qu'un seul homme meure pour le peuple.
\par 15 Simon Pierre, avec un autre disciple, suivait Jésus. Ce disciple était connu du souverain sacrificateur, et il entra avec Jésus dans la cour du souverain sacrificateur;
\par 16 mais Pierre resta dehors près de la porte. L'autre disciple, qui était connu du souverain sacrificateur, sortit, parla à la portière, et fit entrer Pierre.
\par 17 Alors la servante, la portière, dit à Pierre: Toi aussi, n'es-tu pas des disciples de cet homme? Il dit: Je n'en suis point.
\par 18 Les serviteurs et les huissiers, qui étaient là, avaient allumé un brasier, car il faisait froid, et ils se chauffaient. Pierre se tenait avec eux, et se chauffait.
\par 19 Le souverain sacrificateur interrogea Jésus sur ses disciples et sur sa doctrine.
\par 20 Jésus lui répondit: J'ai parlé ouvertement au monde; j'ai toujours enseigné dans la synagogue et dans le temple, où tous les Juifs s'assemblent, et je n'ai rien dit en secret.
\par 21 Pourquoi m'interroges-tu? Interroge sur ce que je leur ai dit ceux qui m'ont entendu; voici, ceux-là savent ce que j'ai dit.
\par 22 A ces mots, un des huissiers, qui se trouvait là, donna un soufflet à Jésus, en disant: Est-ce ainsi que tu réponds au souverain sacrificateur?
\par 23 Jésus lui dit: Si j'ai mal parlé, fais voir ce que j'ai dit de mal; et si j'ai bien parlé, pourquoi me frappes-tu?
\par 24 Anne l'envoya lié à Caïphe, le souverain sacrificateur.
\par 25 Simon Pierre était là, et se chauffait. On lui dit: Toi aussi, n'es-tu pas de ses disciples? Il le nia, et dit: Je n'en suis point.
\par 26 Un des serviteurs du souverain sacrificateur, parent de celui à qui Pierre avait coupé l'oreille, dit: Ne t'ai-je pas vu avec lui dans le jardin?
\par 27 Pierre le nia de nouveau. Et aussitôt le coq chanta.
\par 28 Ils conduisirent Jésus de chez Caïphe au prétoire: c'était le matin. Ils n'entrèrent point eux-mêmes dans le prétoire, afin de ne pas se souiller, et de pouvoir manger la Pâque.
\par 29 Pilate sortit donc pour aller à eux, et il dit: Quelle accusation portez-vous contre cet homme?
\par 30 Ils lui répondirent: Si ce n'était pas un malfaiteur, nous ne te l'aurions pas livré.
\par 31 Sur quoi Pilate leur dit: Prenez-le vous-mêmes, et jugez-le selon votre loi. Les Juifs lui dirent: Il ne nous est pas permis de mettre personne à mort.
\par 32 C'était afin que s'accomplît la parole que Jésus avait dite, lorsqu'il indiqua de quelle mort il devait mourir.
\par 33 Pilate rentra dans le prétoire, appela Jésus, et lui dit: Es-tu le roi des Juifs?
\par 34 Jésus répondit: Est-ce de toi-même que tu dis cela, ou d'autres te l'ont-ils dit de moi?
\par 35 Pilate répondit: Moi, suis-je Juif? Ta nation et les principaux sacrificateurs t'ont livré à moi: qu'as-tu fait?
\par 36 Mon royaume n'est pas de ce monde, répondit Jésus. Si mon royaume était de ce monde, mes serviteurs auraient combattu pour moi afin que je ne fusse pas livré aux Juifs; mais maintenant mon royaume n'est point d'ici-bas.
\par 37 Pilate lui dit: Tu es donc roi? Jésus répondit: Tu le dis, je suis roi. Je suis né et je suis venu dans le monde pour rendre témoignage à la vérité. Quiconque est de la vérité écoute ma voix.
\par 38 Pilate lui dit: Qu'est-ce que la vérité? Après avoir dit cela, il sortit de nouveau pour aller vers les Juifs, et il leur dit: Je ne trouve aucun crime en lui.
\par 39 Mais, comme c'est parmi vous une coutume que je vous relâche quelqu'un à la fête de Pâque, voulez-vous que je vous relâche le roi des Juifs?
\par 40 Alors de nouveau tous s'écrièrent: Non pas lui, mais Barabbas. Or, Barabbas était un brigand.

\chapter{19}

\par 1 Alors Pilate prit Jésus, et le fit battre de verges.
\par 2 Les soldats tressèrent une couronne d'épines qu'ils posèrent sur sa tête, et ils le revêtirent d'un manteau de pourpre; puis, s'approchant de lui,
\par 3 ils disaient: Salut, roi des Juifs! Et ils lui donnaient des soufflets.
\par 4 Pilate sortit de nouveau, et dit aux Juifs: Voici, je vous l'amène dehors, afin que vous sachiez que je ne trouve en lui aucun crime.
\par 5 Jésus sortit donc, portant la couronne d'épines et le manteau de pourpre. Et Pilate leur dit: Voici l'homme.
\par 6 Lorsque les principaux sacrificateurs et les huissiers le virent, ils s'écrièrent: Crucifie! crucifie! Pilate leur dit: Prenez-le vous-mêmes, et crucifiez-le; car moi, je ne trouve point de crime en lui.
\par 7 Les Juifs lui répondirent: Nous avons une loi; et, selon notre loi, il doit mourir, parce qu'il s'est fait Fils de Dieu.
\par 8 Quand Pilate entendit cette parole, sa frayeur augmenta.
\par 9 Il rentra dans le prétoire, et il dit à Jésus: D'où es-tu? Mais Jésus ne lui donna point de réponse.
\par 10 Pilate lui dit: Est-ce à moi que tu ne parles pas? Ne sais-tu pas que j'ai le pouvoir de te crucifier, et que j'ai le pouvoir de te relâcher?
\par 11 Jésus répondit: Tu n'aurais sur moi aucun pouvoir, s'il ne t'avait été donné d'en haut. C'est pourquoi celui qui me livre à toi commet un plus grand péché.
\par 12 Dès ce moment, Pilate cherchait à le relâcher. Mais les Juifs criaient: Si tu le relâches, tu n'es pas ami de César. Quiconque se fait roi se déclare contre César.
\par 13 Pilate, ayant entendu ces paroles, amena Jésus dehors; et il s'assit sur le tribunal, au lieu appelé le Pavé, et en hébreu Gabbatha.
\par 14 C'était la préparation de la Pâque, et environ la sixième heure. Pilate dit aux Juifs: Voici votre roi.
\par 15 Mais ils s'écrièrent: Ote, ôte, crucifie-le! Pilate leur dit: Crucifierai-je votre roi? Les principaux sacrificateurs répondirent: Nous n'avons de roi que César.
\par 16 Alors il le leur livra pour être crucifié. Ils prirent donc Jésus, et l'emmenèrent.
\par 17 Jésus, portant sa croix, arriva au lieu du crâne, qui se nomme en hébreu Golgotha.
\par 18 C'est là qu'il fut crucifié, et deux autres avec lui, un de chaque côté, et Jésus au milieu.
\par 19 Pilate fit une inscription, qu'il plaça sur la croix, et qui était ainsi conçue: Jésus de Nazareth, roi des Juifs.
\par 20 Beaucoup de Juifs lurent cette inscription, parce que le lieu où Jésus fut crucifié était près de la ville: elle était en hébreu, en grec et en latin.
\par 21 Les principaux sacrificateurs des Juifs dirent à Pilate: N'écris pas: Roi des Juifs. Mais écris qu'il a dit: Je suis roi des Juifs.
\par 22 Pilate répondit: Ce que j'ai écrit, je l'ai écrit.
\par 23 Les soldats, après avoir crucifié Jésus, prirent ses vêtements, et ils en firent quatre parts, une part pour chaque soldat. Ils prirent aussi sa tunique, qui était sans couture, d'un seul tissu depuis le haut jusqu'en bas. Et ils dirent entre eux:
\par 24 Ne la déchirons pas, mais tirons au sort à qui elle sera. Cela arriva afin que s'accomplît cette parole de l'Écriture: Ils se sont partagé mes vêtements, Et ils ont tiré au sort ma tunique. Voilà ce que firent les soldats.
\par 25 Près de la croix de Jésus se tenaient sa mère et la soeur de sa mère, Marie, femme de Clopas, et Marie de Magdala.
\par 26 Jésus, voyant sa mère, et auprès d'elle le disciple qu'il aimait, dit à sa mère: Femme, voilà ton fils.
\par 27 Puis il dit au disciple: Voilà ta mère. Et, dès ce moment, le disciple la prit chez lui.
\par 28 Après cela, Jésus, qui savait que tout était déjà consommé, dit, afin que l'Écriture fût accomplie: J'ai soif.
\par 29 Il y avait là un vase plein de vinaigre. Les soldats en remplirent une éponge, et, l'ayant fixée à une branche d'hysope, ils l'approchèrent de sa bouche.
\par 30 Quand Jésus eut pris le vinaigre, il dit: Tout est accompli. Et, baissant la tête, il rendit l'esprit.
\par 31 Dans la crainte que les corps ne restassent sur la croix pendant le sabbat, -car c'était la préparation, et ce jour de sabbat était un grand jour, -les Juifs demandèrent à Pilate qu'on rompît les jambes aux crucifiés, et qu'on les enlevât.
\par 32 Les soldats vinrent donc, et ils rompirent les jambes au premier, puis à l'autre qui avait été crucifié avec lui.
\par 33 S'étant approchés de Jésus, et le voyant déjà mort, ils ne lui rompirent pas les jambes;
\par 34 mais un des soldats lui perça le côté avec une lance, et aussitôt il sortit du sang et de l'eau.
\par 35 Celui qui l'a vu en a rendu témoignage, et son témoignage est vrai; et il sait qu'il dit vrai, afin que vous croyiez aussi.
\par 36 Ces choses sont arrivées, afin que l'Écriture fût accomplie: Aucun de ses os ne sera brisé.
\par 37 Et ailleurs l'Écriture dit encore: Ils verront celui qu'ils ont percé.
\par 38 Après cela, Joseph d'Arimathée, qui était disciple de Jésus, mais en secret par crainte des Juifs, demanda à Pilate la permission de prendre le corps de Jésus. Et Pilate le permit. Il vint donc, et prit le corps de Jésus.
\par 39 Nicodème, qui auparavant était allé de nuit vers Jésus, vint aussi, apportant un mélange d'environ cent livres de myrrhe et d'aloès.
\par 40 Ils prirent donc le corps de Jésus, et l'enveloppèrent de bandes, avec les aromates, comme c'est la coutume d'ensevelir chez les Juifs.
\par 41 Or, il y avait un jardin dans le lieu où Jésus avait été crucifié, et dans le jardin un sépulcre neuf, où personne encore n'avait été mis.
\par 42 Ce fut là qu'ils déposèrent Jésus, à cause de la préparation des Juifs, parce que le sépulcre était proche.

\chapter{20}

\par 1 Le premier jour de la semaine, Marie de Magdala se rendit au sépulcre dès le matin, comme il faisait encore obscur; et elle vit que la pierre était ôtée du sépulcre.
\par 2 Elle courut vers Simon Pierre et vers l'autre disciple que Jésus aimait, et leur dit: Ils ont enlevé du sépulcre le Seigneur, et nous ne savons où ils l'ont mis.
\par 3 Pierre et l'autre disciple sortirent, et allèrent au sépulcre.
\par 4 Ils couraient tous deux ensemble. Mais l'autre disciple courut plus vite que Pierre, et arriva le premier au sépulcre;
\par 5 s'étant baissé, il vit les bandes qui étaient à terre, cependant il n'entra pas.
\par 6 Simon Pierre, qui le suivait, arriva et entra dans le sépulcre; il vit les bandes qui étaient à terre,
\par 7 et le linge qu'on avait mis sur la tête de Jésus, non pas avec les bandes, mais plié dans un lieu à part.
\par 8 Alors l'autre disciple, qui était arrivé le premier au sépulcre, entra aussi; et il vit, et il crut.
\par 9 Car ils ne comprenaient pas encore que, selon l'Écriture, Jésus devait ressusciter des morts.
\par 10 Et les disciples s'en retournèrent chez eux.
\par 11 Cependant Marie se tenait dehors près du sépulcre, et pleurait. Comme elle pleurait, elle se baissa pour regarder dans le sépulcre;
\par 12 et elle vit deux anges vêtus de blanc, assis à la place où avait été couché le corps de Jésus, l'un à la tête, l'autre aux pieds.
\par 13 Ils lui dirent: Femme, pourquoi pleures-tu? Elle leur répondit: Parce qu'ils ont enlevé mon Seigneur, et je ne sais où ils l'ont mis.
\par 14 En disant cela, elle se retourna, et elle vit Jésus debout; mais elle ne savait pas que c'était Jésus.
\par 15 Jésus lui dit: Femme, pourquoi pleures-tu? Qui cherches-tu? Elle, pensant que c'était le jardinier, lui dit: Seigneur, si c'est toi qui l'as emporté, dis-moi où tu l'as mis, et je le prendrai.
\par 16 Jésus lui dit: Marie! Elle se retourna, et lui dit en hébreu: Rabbouni! c'est-à-dire, Maître!
\par 17 Jésus lui dit: Ne me touche pas; car je ne suis pas encore monté vers mon Père. Mais va trouver mes frères, et dis-leur que je monte vers mon Père et votre Père, vers mon Dieu et votre Dieu.
\par 18 Marie de Magdala alla annoncer aux disciples qu'elle avait vu le Seigneur, et qu'il lui avait dit ces choses.
\par 19 Le soir de ce jour, qui était le premier de la semaine, les portes du lieu où se trouvaient les disciples étant fermées, à cause de la crainte qu'ils avaient des Juifs, Jésus vint, se présenta au milieu d'eux, et leur dit: La paix soit avec vous!
\par 20 Et quand il eut dit cela, il leur montra ses mains et son côté. Les disciples furent dans la joie en voyant le Seigneur.
\par 21 Jésus leur dit de nouveau: La paix soit avec vous! Comme le Père m'a envoyé, moi aussi je vous envoie.
\par 22 Après ces paroles, il souffla sur eux, et leur dit: Recevez le Saint Esprit.
\par 23 Ceux à qui vous pardonnerez les péchés, ils leur seront pardonnés; et ceux à qui vous les retiendrez, ils leur seront retenus.
\par 24 Thomas, appelé Didyme, l'un des douze, n'était pas avec eux lorsque Jésus vint.
\par 25 Les autres disciples lui dirent donc: Nous avons vu le Seigneur. Mais il leur dit: Si je ne vois dans ses mains la marque des clous, et si je ne mets mon doigt dans la marque des clous, et si je ne mets ma main dans son côté, je ne croirai point.
\par 26 Huit jours après, les disciples de Jésus étaient de nouveau dans la maison, et Thomas se trouvait avec eux. Jésus vint, les portes étant fermées, se présenta au milieu d'eux, et dit: La paix soit avec vous!
\par 27 Puis il dit à Thomas: Avance ici ton doigt, et regarde mes mains; avance aussi ta main, et mets-la dans mon côté; et ne sois pas incrédule, mais crois.
\par 28 Thomas lui répondit: Mon Seigneur et mon Dieu! Jésus lui dit:
\par 29 Parce que tu m'as vu, tu as cru. Heureux ceux qui n'ont pas vu, et qui ont cru!
\par 30 Jésus a fait encore, en présence de ses disciples, beaucoup d'autres miracles, qui ne sont pas écrits dans ce livre.
\par 31 Mais ces choses ont été écrites afin que vous croyiez que Jésus est le Christ, le Fils de Dieu, et qu'en croyant vous ayez la vie en son nom.

\chapter{21}

\par 1 Après cela, Jésus se montra encore aux disciples, sur les bords de la mer de Tibériade. Et voici de quelle manière il se montra.
\par 2 Simon Pierre, Thomas, appelé Didyme, Nathanaël, de Cana en Galilée, les fils de Zébédée, et deux autres disciples de Jésus, étaient ensemble.
\par 3 Simon Pierre leur dit: Je vais pêcher. Ils lui dirent: Nous allons aussi avec toi. Ils sortirent et montèrent dans une barque, et cette nuit-là ils ne prirent rien.
\par 4 Le matin étant venu, Jésus se trouva sur le rivage; mais les disciples ne savaient pas que c'était Jésus.
\par 5 Jésus leur dit: Enfants, n'avez-vous rien à manger? Ils lui répondirent: Non.
\par 6 Il leur dit: Jetez le filet du côté droit de la barque, et vous trouverez. Ils le jetèrent donc, et ils ne pouvaient plus le retirer, à cause de la grande quantité de poissons.
\par 7 Alors le disciple que Jésus aimait dit à Pierre: C'est le Seigneur! Et Simon Pierre, dès qu'il eut entendu que c'était le Seigneur, mit son vêtement et sa ceinture, car il était nu, et se jeta dans la mer.
\par 8 Les autres disciples vinrent avec la barque, tirant le filet plein de poissons, car ils n'étaient éloignés de terre que d'environ deux cents coudées.
\par 9 Lorsqu'ils furent descendus à terre, ils virent là des charbons allumés, du poisson dessus, et du pain.
\par 10 Jésus leur dit: Apportez des poissons que vous venez de prendre.
\par 11 Simon Pierre monta dans la barque, et tira à terre le filet plein de cent cinquante-trois grands poissons; et quoiqu'il y en eût tant, le filet ne se rompit point.
\par 12 Jésus leur dit: Venez, mangez. Et aucun des disciples n'osait lui demander: Qui es-tu? sachant que c'était le Seigneur.
\par 13 Jésus s'approcha, prit le pain, et leur en donna; il fit de même du poisson.
\par 14 C'était déjà la troisième fois que Jésus se montrait à ses disciples depuis qu'il était ressuscité des morts.
\par 15 Après qu'ils eurent mangé, Jésus dit à Simon Pierre: Simon, fils de Jonas, m'aimes-tu plus que ne m'aiment ceux-ci? Il lui répondit: Oui, Seigneur, tu sais que je t'aime. Jésus lui dit: Pais mes agneaux.
\par 16 Il lui dit une seconde fois: Simon, fils de Jonas, m'aimes-tu? Pierre lui répondit: Oui, Seigneur, tu sais que je t'aime. Jésus lui dit: Pais mes brebis.
\par 17 Il lui dit pour la troisième fois: Simon, fils de Jonas, m'aimes-tu? Pierre fut attristé de ce qu'il lui avait dit pour la troisième fois: M'aimes-tu? Et il lui répondit: Seigneur, tu sais toutes choses, tu sais que je t'aime. Jésus lui dit: Pais mes brebis.
\par 18 En vérité, en vérité, je te le dis, quand tu étais plus jeune, tu te ceignais toi-même, et tu allais où tu voulais; mais quand tu seras vieux, tu étendras tes mains, et un autre te ceindra, et te mènera où tu ne voudras pas.
\par 19 Il dit cela pour indiquer par quelle mort Pierre glorifierait Dieu. Et ayant ainsi parlé, il lui dit: Suis-moi.
\par 20 Pierre, s'étant retourné, vit venir après eux le disciple que Jésus aimait, celui qui, pendant le souper, s'était penché sur la poitrine de Jésus, et avait dit: Seigneur, qui est celui qui te livre?
\par 21 En le voyant, Pierre dit à Jésus: Et celui-ci, Seigneur, que lui arrivera-t-il?
\par 22 Jésus lui dit: Si je veux qu'il demeure jusqu'à ce que je vienne, que t'importe? Toi, suis-moi.
\par 23 Là-dessus, le bruit courut parmi les frères que ce disciple ne mourrait point. Cependant Jésus n'avait pas dit à Pierre qu'il ne mourrait point; mais: Si je veux qu'il demeure jusqu'à ce que je vienne, que t'importe?
\par 24 C'est ce disciple qui rend témoignage de ces choses, et qui les a écrites. Et nous savons que son témoignage est vrai.
\par 25 Jésus a fait encore beaucoup d'autres choses; si on les écrivait en détail, je ne pense pas que le monde même pût contenir les livres qu'on écrirait.


\end{document}