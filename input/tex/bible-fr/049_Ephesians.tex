\begin{document}

\title{Épître de Paul aux Éphésiens}


\chapter{1}

\par 1 Paul, apôtre de Jésus Christ par la volonté de Dieu, aux saints qui sont à Éphèse et aux fidèles en Jésus Christ:
\par 2 Que la grâce et la paix vous soient données de la part de Dieu notre Père et du Seigneur Jésus Christ!
\par 3 Béni soit Dieu, le Père de notre Seigneur Jésus Christ, qui nous a bénis de toute sortes de bénédictions spirituelles dans les lieux célestes en Christ!
\par 4 En lui Dieu nous a élus avant la fondation du monde, pour que nous soyons saints et irrépréhensibles devant lui,
\par 5 nous ayant prédestinés dans son amour à être ses enfants d'adoption par Jésus Christ, selon le bon plaisir de sa volonté,
\par 6 à la louange de la gloire de sa grâce qu'il nous a accordée en son bien-aimé.
\par 7 En lui nous avons la rédemption par son sang, la rémission des péchés, selon la richesse de sa grâce,
\par 8 que Dieu a répandue abondamment sur nous par toute espèce de sagesse et d'intelligence,
\par 9 nous faisant connaître le mystère de sa volonté, selon le bienveillant dessein qu'il avait formé en lui-même,
\par 10 pour le mettre à exécution lorsque les temps seraient accomplis, de réunir toutes choses en Christ, celles qui sont dans les cieux et celles qui sont sur la terre.
\par 11 En lui nous sommes aussi devenus héritiers, ayant été prédestinés suivant la résolution de celui qui opère toutes choses d'après le conseil de sa volonté,
\par 12 afin que nous servions à la louange de sa gloire, nous qui d'avance avons espéré en Christ.
\par 13 En lui vous aussi, après avoir entendu la parole de la vérité, l'Évangile de votre salut, en lui vous avez cru et vous avez été scellés du Saint Esprit qui avait été promis,
\par 14 lequel est un gage de notre héritage, pour la rédemption de ceux que Dieu s'est acquis, à la louange de sa gloire.
\par 15 C'est pourquoi moi aussi, ayant entendu parler de votre foi au Seigneur Jésus et de votre charité pour tous les saints,
\par 16 je ne cesse de rendre grâces pour vous, faisant mention de vous dans mes prières,
\par 17 afin que le Dieu de notre Seigneur Jésus Christ, le Père de gloire, vous donne un esprit de sagesse et de révélation, dans sa connaissance,
\par 18 et qu'il illumine les yeux de votre coeur, pour que vous sachiez quelle est l'espérance qui s'attache à son appel, quelle est la richesse de la gloire de son héritage qu'il réserve aux saints,
\par 19 et quelle est envers nous qui croyons l'infinie grandeur de sa puissance, se manifestant avec efficacité par la vertu de sa force.
\par 20 Il l'a déployée en Christ, en le ressuscitant des morts, et en le faisant asseoir à sa droite dans les lieux célestes,
\par 21 au-dessus de toute domination, de toute autorité, de toute puissance, de toute dignité, et de tout nom qui se peut nommer, non seulement dans le siècle présent, mais encore dans le siècle à venir.
\par 22 Il a tout mis sous ses pieds, et il l'a donné pour chef suprême à l'Église,
\par 23 qui est son corps, la plénitude de celui qui remplit tout en tous.

\chapter{2}

\par 1 Vous étiez morts par vos offenses et par vos péchés,
\par 2 dans lesquels vous marchiez autrefois, selon le train de ce monde, selon le prince de la puissance de l'air, de l'esprit qui agit maintenant dans les fils de la rébellion.
\par 3 Nous tous aussi, nous étions de leur nombre, et nous vivions autrefois selon les convoitises de notre chair, accomplissant les volontés de la chair et de nos pensées, et nous étions par nature des enfants de colère, comme les autres...
\par 4 Mais Dieu, qui est riche en miséricorde, à cause du grand amour dont il nous a aimés,
\par 5 nous qui étions morts par nos offenses, nous a rendus à la vie avec Christ (c'est par grâce que vous êtes sauvés);
\par 6 il nous a ressuscités ensemble, et nous a fait asseoir ensemble dans les lieux célestes, en Jésus Christ,
\par 7 afin de montrer dans les siècles à venir l'infinie richesse de sa grâce par sa bonté envers nous en Jésus Christ.
\par 8 Car c'est par la grâce que vous êtes sauvés, par le moyen de la foi. Et cela ne vient pas de vous, c'est le don de Dieu.
\par 9 Ce n'est point par les oeuvres, afin que personne ne se glorifie.
\par 10 Car nous sommes son ouvrage, ayant été créés en Jésus Christ pour de bonnes oeuvres, que Dieu a préparées d'avance, afin que nous les pratiquions.
\par 11 C'est pourquoi, vous autrefois païens dans la chair, appelés incirconcis par ceux qu'on appelle circoncis et qui le sont en la chair par la main de l'homme,
\par 12 souvenez-vous que vous étiez en ce temps-là sans Christ, privés du droit de cité en Israël, étrangers aux alliances de la promesse, sans espérance et sans Dieu dans le monde.
\par 13 Mais maintenant, en Jésus Christ, vous qui étiez jadis éloignés, vous avez été rapprochés par le sang de Christ.
\par 14 Car il est notre paix, lui qui des deux n'en a fait qu'un, et qui a renversé le mur de séparation, l'inimitié,
\par 15 ayant anéanti par sa chair la loi des ordonnances dans ses prescriptions, afin de créer en lui-même avec les deux un seul homme nouveau, en établissant la paix,
\par 16 et de les réconcilier, l'un et l'autre en un seul corps, avec Dieu par la croix, en détruisant par elle l'inimitié.
\par 17 Il est venu annoncer la paix à vous qui étiez loin, et la paix à ceux qui étaient près;
\par 18 car par lui nous avons les uns et les autres accès auprès du Père, dans un même Esprit.
\par 19 Ainsi donc, vous n'êtes plus des étrangers, ni des gens du dehors; mais vous êtes concitoyens des saints, gens de la maison de Dieu.
\par 20 Vous avez été édifiés sur le fondement des apôtres et des prophètes, Jésus Christ lui-même étant la pierre angulaire.
\par 21 En lui tout l'édifice, bien coordonné, s'élève pour être un temple saint dans le Seigneur.
\par 22 En lui vous êtes aussi édifiés pour être une habitation de Dieu en Esprit.

\chapter{3}

\par 1 A cause de cela, moi Paul, le prisonnier de Christ pour vous païens...
\par 2 si du moins vous avez appris quelle est la dispensation de la grâce de Dieu, qui m'a été donnée pour vous.
\par 3 C'est par révélation que j'ai eu connaissance du mystère sur lequel je viens d'écrire en peu de mots.
\par 4 En les lisant, vous pouvez vous représenter l'intelligence que j'ai du mystère de Christ.
\par 5 Il n'a pas été manifesté aux fils des hommes dans les autres générations, comme il a été révélé maintenant par l'Esprit aux saints apôtres et prophètes de Christ.
\par 6 Ce mystère, c'est que les païens sont cohéritiers, forment un même corps, et participent à la même promesse en Jésus Christ par l'Évangile,
\par 7 dont j'ai été fait ministre selon le don de la grâce de Dieu, qui m'a été accordée par l'efficacité de sa puissance.
\par 8 A moi, qui suis le moindre de tous les saints, cette grâce a été accordée d'annoncer aux païens les richesses incompréhensibles de Christ,
\par 9 et de mettre en lumière quelle est la dispensation du mystère caché de tout temps en Dieu qui a créé toutes choses,
\par 10 afin que les dominations et les autorités dans les lieux célestes connaissent aujourd'hui par l'Église la sagesse infiniment variée de Dieu,
\par 11 selon le dessein éternel qu'il a mis à exécution par Jésus Christ notre Seigneur,
\par 12 en qui nous avons, par la foi en lui, la liberté de nous approcher de Dieu avec confiance.
\par 13 Aussi je vous demande de ne pas perdre courage à cause de mes tribulations pour vous: elles sont votre gloire.
\par 14 A cause de cela, je fléchis les genoux devant le Père,
\par 15 duquel tire son nom toute famille dans les cieux et sur la terre,
\par 16 afin qu'il vous donne, selon la richesse de sa gloire, d'être puissamment fortifiés par son Esprit dans l'homme intérieur,
\par 17 en sorte que Christ habite dans vos coeurs par la foi; afin qu'étant enracinés et fondés dans l'amour,
\par 18 vous puissiez comprendre avec tous les saints quelle est la largeur, la longueur, la profondeur et la hauteur,
\par 19 et connaître l'amour de Christ, qui surpasse toute connaissance, en sorte que vous soyez remplis jusqu'à toute la plénitude de Dieu.
\par 20 Or, à celui qui peut faire, par la puissance qui agit en nous, infiniment au delà de tout ce que nous demandons ou pensons,
\par 21 à lui soit la gloire dans l'Église et en Jésus Christ, dans toutes les générations, aux siècles des siècles! Amen!

\chapter{4}

\par 1 Je vous exhorte donc, moi, le prisonnier dans le Seigneur, à marcher d'une manière digne de la vocation qui vous a été adressée,
\par 2 en toute humilité et douceur, avec patience, vous supportant les uns les autres avec charité,
\par 3 vous efforçant de conserver l'unité de l'esprit par le lien de la paix.
\par 4 Il y a un seul corps et un seul Esprit, comme aussi vous avez été appelés à une seule espérance par votre vocation;
\par 5 il y a un seul Seigneur, une seule foi, un seul baptême,
\par 6 un seul Dieu et Père de tous, qui est au-dessus de tous, et parmi tous, et en tous.
\par 7 Mais à chacun de nous la grâce a été donnée selon la mesure du don de Christ.
\par 8 C'est pourquoi il est dit: Étant monté en haut, il a emmené des captifs, Et il a fait des dons aux hommes.
\par 9 Or, que signifie: Il est monté, sinon qu'il est aussi descendu dans les régions inférieures de la terre?
\par 10 Celui qui est descendu, c'est le même qui est monté au-dessus de tous les cieux, afin de remplir toutes choses.
\par 11 Et il a donné les uns comme apôtres, les autres comme prophètes, les autres comme évangélistes, les autres comme pasteurs et docteurs,
\par 12 pour le perfectionnement des saints en vue de l'oeuvre du ministère et de l'édification du corps de Christ,
\par 13 jusqu'à ce que nous soyons tous parvenus à l'unité de la foi et de la connaissance du Fils de Dieu, à l'état d'homme fait, à la mesure de la stature parfaite de Christ,
\par 14 afin que nous ne soyons plus des enfants, flottants et emportés à tout vent de doctrine, par la tromperie des hommes, par leur ruse dans les moyens de séduction,
\par 15 mais que, professant la vérité dans la charité, nous croissions à tous égards en celui qui est le chef, Christ.
\par 16 C'est de lui, et grâce à tous les liens de son assistance, que tout le corps, bien coordonné et formant un solide assemblage, tire son accroissement selon la force qui convient à chacune de ses parties, et s'édifie lui-même dans la charité.
\par 17 Voici donc ce que je dis et ce que je déclare dans le Seigneur, c'est que vous ne devez plus marcher comme les païens, qui marchent selon la vanité de leurs pensées.
\par 18 Ils ont l'intelligence obscurcie, ils sont étrangers à la vie de Dieu, à cause de l'ignorance qui est en eux, à cause de l'endurcissement de leur coeur.
\par 19 Ayant perdu tout sentiment, ils se sont livrés à la dissolution, pour commettre toute espèce d'impureté jointe à la cupidité.
\par 20 Mais vous, ce n'est pas ainsi que vous avez appris Christ,
\par 21 si du moins vous l'avez entendu, et si, conformément à la vérité qui est en Jésus, c'est en lui que vous avez été instruits à vous dépouiller,
\par 22 eu égard à votre vie passée, du vieil homme qui se corrompt par les convoitises trompeuses,
\par 23 à être renouvelés dans l'esprit de votre intelligence,
\par 24 et à revêtir l'homme nouveau, créé selon Dieu dans une justice et une sainteté que produit la vérité.
\par 25 C'est pourquoi, renoncez au mensonge, et que chacun de vous parle selon la vérité à son prochain; car nous sommes membres les uns des autres.
\par 26 Si vous vous mettez en colère, ne péchez point; que le soleil ne se couche pas sur votre colère,
\par 27 et ne donnez pas accès au diable.
\par 28 Que celui qui dérobait ne dérobe plus; mais plutôt qu'il travaille, en faisant de ses mains ce qui est bien, pour avoir de quoi donner à celui qui est dans le besoin.
\par 29 Qu'il ne sorte de votre bouche aucune parole mauvaise, mais, s'il y a lieu, quelque bonne parole, qui serve à l'édification et communique une grâce à ceux qui l'entendent.
\par 30 N'attristez pas le Saint Esprit de Dieu, par lequel vous avez été scellés pour le jour de la rédemption.
\par 31 Que toute amertume, toute animosité, toute colère, toute clameur, toute calomnie, et toute espèce de méchanceté, disparaissent du milieu de vous.
\par 32 Soyez bons les uns envers les autres, compatissants, vous pardonnant réciproquement, comme Dieu vous a pardonné en Christ.

\chapter{5}

\par 1 Devenez donc les imitateurs de Dieu, comme des enfants bien-aimés;
\par 2 et marchez dans la charité, à l'exemple de Christ, qui nous a aimés, et qui s'est livré lui-même à Dieu pour nous comme une offrande et un sacrifice de bonne odeur.
\par 3 Que l'impudicité, qu'aucune espèce d'impureté, et que la cupidité, ne soient pas même nommées parmi vous, ainsi qu'il convient à des saints.
\par 4 Qu'on n'entende ni paroles déshonnêtes, ni propos insensés, ni plaisanteries, choses qui sont contraires à la bienséance; qu'on entende plutôt des actions de grâces.
\par 5 Car, sachez-le bien, aucun impudique, ou impur, ou cupide, c'est-à-dire, idolâtre, n'a d'héritage dans le royaume de Christ et de Dieu.
\par 6 Que personne ne vous séduise par de vains discours; car c'est à cause de ces choses que la colère de Dieu vient sur les fils de la rébellion.
\par 7 N'ayez donc aucune part avec eux.
\par 8 Autrefois vous étiez ténèbres, et maintenant vous êtes lumière dans le Seigneur. Marchez comme des enfants de lumière!
\par 9 Car le fruit de la lumière consiste en toute sorte de bonté, de justice et de vérité.
\par 10 Examinez ce qui est agréable au Seigneur;
\par 11 et ne prenez point part aux oeuvres infructueuses des ténèbres, mais plutôt condamnez-les.
\par 12 Car il est honteux de dire ce qu'ils font en secret;
\par 13 mais tout ce qui est condamné est manifesté par la lumière, car tout ce qui est manifesté est lumière.
\par 14 C'est pour cela qu'il est dit: Réveille-toi, toi qui dors, Relève-toi d'entre les morts, Et Christ t'éclairera.
\par 15 Prenez donc garde de vous conduire avec circonspection, non comme des insensés, mais comme des sages;
\par 16 rachetez le temps, car les jours sont mauvais.
\par 17 C'est pourquoi ne soyez pas inconsidérés, mais comprenez quelle est la volonté du Seigneur.
\par 18 Ne vous enivrez pas de vin: c'est de la débauche. Soyez, au contraire, remplis de l'Esprit;
\par 19 entretenez-vous par des psaumes, par des hymnes, et par des cantiques spirituels, chantant et célébrant de tout votre coeur les louanges du Seigneur;
\par 20 rendez continuellement grâces pour toutes choses à Dieu le Père, au nom de notre Seigneur Jésus Christ,
\par 21 vous soumettant les uns aux autres dans la crainte de Christ.
\par 22 Femmes, soyez soumises à vos maris, comme au Seigneur;
\par 23 car le mari est le chef de la femme, comme Christ est le chef de l'Église, qui est son corps, et dont il est le Sauveur.
\par 24 Or, de même que l'Église est soumise à Christ, les femmes aussi doivent l'être à leurs maris en toutes choses.
\par 25 Maris, aimez vos femmes, comme Christ a aimé l'Église, et s'est livré lui-même pour elle,
\par 26 afin de la sanctifier par la parole, après l'avoir purifiée par le baptême d'eau,
\par 27 afin de faire paraître devant lui cette Église glorieuse, sans tache, ni ride, ni rien de semblable, mais sainte et irrépréhensible.
\par 28 C'est ainsi que les maris doivent aimer leurs femmes comme leurs propres corps. Celui qui aime sa femme s'aime lui-même.
\par 29 Car jamais personne n'a haï sa propre chair; mais il la nourrit et en prend soin, comme Christ le fait pour l'Église,
\par 30 parce que nous sommes membres de son corps.
\par 31 C'est pourquoi l'homme quittera son père et sa mère, et s'attachera à sa femme, et les deux deviendront une seule chair.
\par 32 Ce mystère est grand; je dis cela par rapport à Christ et à l'Église.
\par 33 Du reste, que chacun de vous aime sa femme comme lui-même, et que la femme respecte son mari.

\chapter{6}

\par 1 Enfants, obéissez à vos parents, selon le Seigneur, car cela est juste.
\par 2 Honore ton père et ta mère (c'est le premier commandement avec une promesse),
\par 3 afin que tu sois heureux et que tu vives longtemps sur la terre.
\par 4 Et vous, pères, n'irritez pas vos enfants, mais élevez-les en les corrigeant et en les instruisant selon le Seigneur.
\par 5 Serviteurs, obéissez à vos maîtres selon la chair, avec crainte et tremblement, dans la simplicité de votre coeur, comme à Christ,
\par 6 non pas seulement sous leurs yeux, comme pour plaire aux hommes, mais comme des serviteurs de Christ, qui font de bon coeur la volonté de Dieu.
\par 7 Servez-les avec empressement, comme servant le Seigneur et non des hommes,
\par 8 sachant que chacun, soit esclave, soit libre, recevra du Seigneur selon ce qu'il aura fait de bien.
\par 9 Et vous, maîtres, agissez de même à leur égard, et abstenez-vous de menaces, sachant que leur maître et le vôtre est dans les cieux, et que devant lui il n'y a point d'acception de personnes.
\par 10 Au reste, fortifiez-vous dans le Seigneur, et par sa force toute-puissante.
\par 11 Revêtez-vous de toutes les armes de Dieu, afin de pouvoir tenir ferme contre les ruses du diable.
\par 12 Car nous n'avons pas à lutter contre la chair et le sang, mais contre les dominations, contre les autorités, contre les princes de ce monde de ténèbres, contre les esprits méchants dans les lieux célestes.
\par 13 C'est pourquoi, prenez toutes les armes de Dieu, afin de pouvoir résister dans le mauvais jour, et tenir ferme après avoir tout surmonté.
\par 14 Tenez donc ferme: ayez à vos reins la vérité pour ceinture; revêtez la cuirasse de la justice;
\par 15 mettez pour chaussure à vos pieds le zèle que donne l'Évangile de paix;
\par 16 prenez par-dessus tout cela le bouclier de la foi, avec lequel vous pourrez éteindre tous les traits enflammés du malin;
\par 17 prenez aussi le casque du salut, et l'épée de l'Esprit, qui est la parole de Dieu.
\par 18 Faites en tout temps par l'Esprit toutes sortes de prières et de supplications. Veillez à cela avec une entière persévérance, et priez pour tous les saints.
\par 19 Priez pour moi, afin qu'il me soit donné, quand j'ouvre la bouche, de faire connaître hardiment et librement le mystère de l'Évangile,
\par 20 pour lequel je suis ambassadeur dans les chaînes, et que j'en parle avec assurance comme je dois en parler.
\par 21 Afin que vous aussi, vous sachiez ce qui me concerne, ce que je fais, Tychique, le bien-aimé frère et fidèle ministre dans le Seigneur, vous informera de tout.
\par 22 Je l'envoie exprès vers vous, pour que vous connaissiez notre situation, et pour qu'il console vos coeurs.
\par 23 Que la paix et la charité avec la foi soient donnés aux frères de la part de Dieu le Père et du Seigneur Jésus Christ!
\par 24 Que la grâce soit avec tous ceux qui aiment notre Seigneur Jésus Christ d'un amour inaltérable!


\end{document}