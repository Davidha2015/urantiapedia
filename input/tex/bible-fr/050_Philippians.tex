\begin{document}

\title{Philippians}


\chapter{1}

\par 1 Paul et Timothée, serviteurs de Jésus Christ, à tous les saints en Jésus Christ qui sont à Philippes, aux évêques et aux diacres:
\par 2 que la grâce et la paix vous soient données de la part de Dieu notre Père et du Seigneur Jésus Christ!
\par 3 Je rends grâces à mon Dieu de tout le souvenir que je garde de vous,
\par 4 ne cessant, dans toutes mes prières pour vous tous,
\par 5 de manifester ma joie au sujet de la part que vous prenez à l'Évangile, depuis le premier jour jusqu'à maintenant.
\par 6 Je suis persuadé que celui qui a commencé en vous cette bonne oeuvre la rendra parfaite pour le jour de Jésus Christ.
\par 7 Il est juste que je pense ainsi de vous tous, parce que je vous porte dans mon coeur, soit dans mes liens, soit dans la défense et la confirmation de l'Évangile, vous qui tous participez à la même grâce que moi.
\par 8 Car Dieu m'est témoin que je vous chéris tous avec la tendresse de Jésus Christ.
\par 9 Et ce que je demande dans mes prières, c'est que votre amour augmente de plus en plus en connaissance et en pleine intelligence
\par 10 pour le discernement des choses les meilleures, afin que vous soyez purs et irréprochables pour le jour de Christ,
\par 11 remplis du fruit de justice qui est par Jésus Christ, à la gloire et à la louange de Dieu.
\par 12 Je veux que vous sachiez, frères, que ce qui m'est arrivé a plutôt contribué aux progrès de l'Évangile.
\par 13 En effet, dans tout le prétoire et partout ailleurs, nul n'ignore que c'est pour Christ que je suis dans les liens,
\par 14 et la plupart des frères dans le Seigneur, encouragés par mes liens, ont plus d'assurance pour annoncer sans crainte la parole.
\par 15 Quelques-uns, il est vrai, prêchent Christ par envie et par esprit de dispute; mais d'autres le prêchent avec des dispositions bienveillantes.
\par 16 Ceux-ci agissent par amour, sachant que je suis établi pour la défense de l'Évangile,
\par 17 tandis que ceux-là, animés d'un esprit de dispute, annoncent Christ par des motifs qui ne sont pas purs et avec la pensée de me susciter quelque tribulation dans mes liens.
\par 18 Qu'importe? De toute manière, que ce soit pour l'apparence, que ce soit sincèrement, Christ n'est pas moins annoncé: je m'en réjouis, et je m'en réjouirai encore.
\par 19 Car je sais que cela tournera à mon salut, grâce à vos prières et à l'assistance de l'Esprit de Jésus Christ,
\par 20 selon ma ferme attente et mon espérance que je n'aurai honte de rien, mais que, maintenant comme toujours, Christ sera glorifié dans mon corps avec une pleine assurance, soit par ma vie, soit par ma mort;
\par 21 car Christ est ma vie, et la mort m'est un gain.
\par 22 Mais s'il est utile pour mon oeuvre que je vive dans la chair, je ne saurais dire ce que je dois préférer.
\par 23 Je suis pressé des deux côtés: j'ai le désir de m'en aller et d'être avec Christ, ce qui de beaucoup est le meilleur;
\par 24 mais à cause de vous il est plus nécessaire que je demeure dans la chair.
\par 25 Et je suis persuadé, je sais que je demeurerai et que je resterai avec vous tous, pour votre avancement et pour votre joie dans la foi,
\par 26 afin que, par mon retour auprès de vous, vous ayez en moi un abondant sujet de vous glorifier en Jésus Christ.
\par 27 Seulement, conduisez-vous d'une manière digne de l'Évangile de Christ, afin que, soit que je vienne vous voir, soit que je reste absent, j'entende dire de vous que vous demeurez fermes dans un même esprit, combattant d'une même âme pour la foi de l'Évangile,
\par 28 sans vous laisser aucunement effrayer par les adversaires, ce qui est pour eux une preuve de perdition, mais pour vous de salut;
\par 29 et cela de la part de Dieu, car il vous a été fait la grâce, par rapport à Christ, non seulement de croire en lui, mais encore de souffrir pour lui,
\par 30 en soutenant le même combat que vous m'avez vu soutenir, et que vous apprenez maintenant que je soutiens.

\chapter{2}

\par 1 Si donc il y a quelque consolation en Christ, s'il y a quelque soulagement dans la charité, s'il y a quelque union d'esprit, s'il y a quelque compassion et quelque miséricorde,
\par 2 rendez ma joie parfaite, ayant un même sentiment, un même amour, une même âme, une même pensée.
\par 3 Ne faites rien par esprit de parti ou par vaine gloire, mais que l'humilité vous fasse regarder les autres comme étant au-dessus de vous-mêmes.
\par 4 Que chacun de vous, au lieu de considérer ses propres intérêts, considère aussi ceux des autres.
\par 5 Ayez en vous les sentiments qui étaient en Jésus Christ,
\par 6 lequel, existant en forme de Dieu, n'a point regardé comme une proie à arracher d'être égal avec Dieu,
\par 7 mais s'est dépouillé lui-même, en prenant une forme de serviteur, en devenant semblable aux hommes; et ayant paru comme un simple homme,
\par 8 il s'est humilié lui-même, se rendant obéissant jusqu'à la mort, même jusqu'à la mort de la croix.
\par 9 C'est pourquoi aussi Dieu l'a souverainement élevé, et lui a donné le nom qui est au-dessus de tout nom,
\par 10 afin qu'au nom de Jésus tout genou fléchisse dans les cieux, sur la terre et sous la terre,
\par 11 et que toute langue confesse que Jésus Christ est Seigneur, à la gloire de Dieu le Père.
\par 12 Ainsi, mes bien-aimés, comme vous avez toujours obéi, travaillez à votre salut avec crainte et tremblement, non seulement comme en ma présence, mais bien plus encore maintenant que je suis absent;
\par 13 car c'est Dieu qui produit en vous le vouloir et le faire, selon son bon plaisir.
\par 14 Faites toutes choses sans murmures ni hésitations,
\par 15 afin que vous soyez irréprochables et purs, des enfants de Dieu irrépréhensibles au milieu d'une génération perverse et corrompue, parmi laquelle vous brillez comme des flambeaux dans le monde,
\par 16 portant la parole de vie; et je pourrai me glorifier, au jour de Christ, de n'avoir pas couru en vain ni travaillé en vain.
\par 17 Et même si je sers de libation pour le sacrifice et pour le service de votre foi, je m'en réjouis, et je me réjouis avec vous tous.
\par 18 Vous aussi, réjouissez-vous de même, et réjouissez-vous avec moi.
\par 19 J'espère dans le Seigneur Jésus vous envoyer bientôt Timothée, afin d'être encouragé moi-même en apprenant ce qui vous concerne.
\par 20 Car je n'ai personne ici qui partage mes sentiments, pour prendre sincèrement à coeur votre situation;
\par 21 tous, en effet, cherchent leurs propres intérêts, et non ceux de Jésus Christ.
\par 22 Vous savez qu'il a été mis à l'épreuve, en se consacrant au service de l'Évangile avec moi, comme un enfant avec son père.
\par 23 J'espère donc vous l'envoyer dès que j'apercevrai l'issue de l'état où je suis;
\par 24 et j'ai cette confiance dans le Seigneur que moi-même aussi j'irai bientôt.
\par 25 J'ai estimé nécessaire de vous envoyer mon frère Épaphrodite, mon compagnon d'oeuvre et de combat, par qui vous m'avez fait parvenir de quoi pourvoir à mes besoins.
\par 26 Car il désirait vous voir tous, et il était fort en peine de ce que vous aviez appris sa maladie.
\par 27 Il a été malade, en effet, et tout près de la mort; mais Dieu a eu pitié de lui, et non seulement de lui, mais aussi de moi, afin que je n'eusse pas tristesse sur tristesse.
\par 28 Je l'ai donc envoyé avec d'autant plus d'empressement, afin que vous vous réjouissiez de le revoir, et que je sois moi-même moins triste.
\par 29 Recevez-le donc dans le Seigneur avec une joie entière, et honorez de tels hommes.
\par 30 Car c'est pour l'oeuvre de Christ qu'il a été près de la mort, ayant exposé sa vie afin de suppléer à votre absence dans le service que vous me rendiez.

\chapter{3}

\par 1 Au reste, mes frères, réjouissez-vous dans le Seigneur. Je ne me lasse point de vous écrire les mêmes choses, et pour vous cela est salutaire.
\par 2 Prenez garde aux chiens, prenez garde aux mauvais ouvriers, prenez garde aux faux circoncis.
\par 3 Car les circoncis, c'est nous, qui rendons à Dieu notre culte par l'Esprit de Dieu, qui nous glorifions en Jésus Christ, et qui ne mettons point notre confiance en la chair.
\par 4 Moi aussi, cependant, j'aurais sujet de mettre ma confiance en la chair. Si quelque autre croit pouvoir se confier en la chair, je le puis bien davantage,
\par 5 moi, circoncis le huitième jour, de la race d'Israël, de la tribu de Benjamin, Hébreu né d'Hébreux; quant à la loi, pharisien;
\par 6 quant au zèle, persécuteur de l'Église; irréprochable, à l'égard de la justice de la loi.
\par 7 Mais ces choses qui étaient pour moi des gains, je les ai regardées comme une perte, à cause de Christ.
\par 8 Et même je regarde toutes choses comme une perte, à cause de l'excellence de la connaissance de Jésus Christ mon Seigneur, pour lequel j'ai renoncé à tout, et je les regarde comme de la boue, afin de gagner Christ,
\par 9 et d'être trouvé en lui, non avec ma justice, celle qui vient de la loi, mais avec celle qui s'obtient par la foi en Christ, la justice qui vient de Dieu par la foi,
\par 10 Afin de connaître Christ, et la puissance de sa résurrection, et la communion de ses souffrances, en devenant conforme à lui dans sa mort, pour parvenir,
\par 11 si je puis, à la résurrection d'entre les morts.
\par 12 Ce n'est pas que j'aie déjà remporté le prix, ou que j'aie déjà atteint la perfection; mais je cours, pour tâcher de le saisir, puisque moi aussi j'ai été saisi par Jésus Christ.
\par 13 Frères, je ne pense pas l'avoir saisi; mais je fais une chose: oubliant ce qui est en arrière et me portant vers ce qui est en avant,
\par 14 je cours vers le but, pour remporter le prix de la vocation céleste de Dieu en Jésus Christ.
\par 15 Nous tous donc qui sommes parfaits, ayons cette même pensée; et si vous êtes en quelque point d'un autre avis, Dieu vous éclairera aussi là-dessus.
\par 16 Seulement, au point où nous sommes parvenus, marchons d'un même pas.
\par 17 Soyez tous mes imitateurs, frères, et portez les regards sur ceux qui marchent selon le modèle que vous avez en nous.
\par 18 Car il en est plusieurs qui marchent en ennemis de la croix de Christ, je vous en ai souvent parlé, et j'en parle maintenant encore en pleurant.
\par 19 Leur fin sera la perdition; ils ont pour dieu leur ventre, ils mettent leur gloire dans ce qui fait leur honte, ils ne pensent qu'aux choses de la terre.
\par 20 Mais notre cité à nous est dans les cieux, d'où nous attendons aussi comme Sauveur le Seigneur Jésus Christ,
\par 21 qui transformera le corps de notre humiliation, en le rendant semblable au corps de sa gloire, par le pouvoir qu'il a de s'assujettir toutes choses.

\chapter{4}

\par 1 C'est pourquoi, mes bien-aimés, et très chers frères, vous qui êtes ma joie et ma couronne, demeurez ainsi fermes dans le Seigneur, mes bien-aimés!
\par 2 J'exhorte Évodie et j'exhorte Syntyche à être d'un même sentiment dans le Seigneur.
\par 3 Et toi aussi, fidèle collègue, oui, je te prie de les aider, elles qui ont combattu pour l'Évangile avec moi, et avec Clément et mes autres compagnons d'oeuvre, dont les noms sont dans le livre de vie.
\par 4 Réjouissez-vous toujours dans le Seigneur; je le répète, réjouissez-vous.
\par 5 Que votre douceur soit connue de tous les hommes. Le Seigneur est proche.
\par 6 Ne vous inquiétez de rien; mais en toute chose faites connaître vos besoins à Dieu par des prières et des supplications, avec des actions de grâces.
\par 7 Et la paix de Dieu, qui surpasse toute intelligence, gardera vos coeurs et vos pensées en Jésus Christ.
\par 8 Au reste, frères, que tout ce qui est vrai, tout ce qui est honorable, tout ce qui est juste, tout ce qui est pur, tout ce qui est aimable, tout ce qui mérite l'approbation, ce qui est vertueux et digne de louange, soit l'objet de vos pensées.
\par 9 Ce que vous avez appris, reçu et entendu de moi, et ce que vous avez vu en moi, pratiquez-le. Et le Dieu de paix sera avec vous.
\par 10 J'ai éprouvé une grande joie dans le Seigneur de ce que vous avez pu enfin renouveler l'expression de vos sentiments pour moi; vous y pensiez bien, mais l'occasion vous manquait.
\par 11 Ce n'est pas en vue de mes besoins que je dis cela, car j'ai appris à être content de l'état où je me trouve.
\par 12 Je sais vivre dans l'humiliation, et je sais vivre dans l'abondance. En tout et partout j'ai appris à être rassasié et à avoir faim, à être dans l'abondance et à être dans la disette.
\par 13 Je puis tout par celui qui me fortifie.
\par 14 Cependant vous avez bien fait de prendre part à ma détresse.
\par 15 Vous le savez vous-mêmes, Philippiens, au commencement de la prédication de l'Évangile, lorsque je partis de la Macédoine, aucune Église n'entra en compte avec moi pour ce qu'elle donnait et recevait;
\par 16 vous fûtes les seuls à le faire, car vous m'envoyâtes déjà à Thessalonique, et à deux reprises, de quoi pourvoir à mes besoins.
\par 17 Ce n'est pas que je recherche les dons; mais je recherche le fruit qui abonde pour votre compte.
\par 18 J'ai tout reçu, et je suis dans l'abondance; j'ai été comblé de biens, en recevant par Épaphrodite ce qui vient de vous comme un parfum de bonne odeur, un sacrifice que Dieu accepte, et qui lui est agréable.
\par 19 Et mon Dieu pourvoira à tous vos besoins selon sa richesse, avec gloire, en Jésus Christ.
\par 20 A notre Dieu et Père soit la gloire aux siècles des siècles! Amen!
\par 21 Saluez tous les saints en Jésus Christ. Les frères qui sont avec moi vous saluent.
\par 22 Tous les saints vous saluent, et principalement ceux de la maison de César.
\par 23 Que la grâce du Seigneur Jésus Christ soit avec votre esprit!


\end{document}