\begin{document}

\title{1 Macchabées}

\chapter{1}

\par 1 Or, après qu'Alexandre, fils de Philippe, le Macédonien, sorti du pays de Chettiim, eut frappé Darius, roi des Perses et des Mèdes, il régna à sa place, le premier sur la Grèce,
\par 2 Et il fit de nombreuses guerres, et conquit de nombreuses forteresses, et tua les rois de la terre,
\par 3 Et il traversa jusqu'aux extrémités de la terre, et prit le butin de nombreuses nations, de sorte que la terre était tranquille devant lui ; sur quoi il fut exalté et son cœur s'est élevé.
\par 4 Et il rassembla une armée puissante et forte et régna sur des pays, des nations et des rois qui devinrent ses tributaires.
\par 5 Et après ces choses, il tomba malade, et comprit qu'il allait mourir.
\par 6 C'est pourquoi il appela ses serviteurs, ceux qui étaient honorables et qui avaient été élevés avec lui dès sa jeunesse, et il partagea son royaume entre eux, tant qu'il était encore en vie.
\par 7 Alexandre régna donc douze ans, puis mourut.
\par 8 Et ses serviteurs devaient gouverner chacun à sa place.
\par 9 Et après sa mort, ils se mirent tous des couronnes ; leurs fils après eux firent de même pendant de nombreuses années ; et les maux se multiplièrent sur la terre.
\par 10 Et d'eux sortit une méchante racine, Antiochus, surnommé Épiphane, fils du roi Antiochus, qui avait été otage à Rome, et il régna la cent trente-septième année du royaume des Grecs.
\par 11 En ces jours-là, des hommes méchants sortaient d'Israël, qui persuadaient beaucoup de gens, disant : Allons faire alliance avec les païens qui nous entourent ; car depuis que nous les avons quittés, nous avons eu beaucoup de tristesse.
\par 12 Cet appareil leur a donc bien plu.
\par 13 Alors certains du peuple furent si enthousiastes à cet égard qu'ils allèrent trouver le roi, qui leur donna l'autorisation de suivre les ordonnances des païens :
\par 14 Sur quoi ils bâtirent un lieu d'exercice à Jérusalem, selon les coutumes des païens :
\par 15 Et ils se sont rendus incirconcis, et ont abandonné la sainte alliance, et se sont joints aux païens, et ont été vendus pour faire du mal.
\par 16 Or, lorsque le royaume fut établi avant Antiochus, il pensait régner sur l'Égypte afin d'avoir la domination de deux royaumes.
\par 17 C'est pourquoi il entra en Égypte avec une grande multitude, avec des chars, des éléphants, des cavaliers et une grande flotte,
\par 18 Et il fit la guerre à Ptolémée, roi d'Égypte ; mais Ptolémée eut peur de lui et s'enfuit ; et beaucoup furent blessés à mort.
\par 19 Ainsi ils s'emparèrent des villes fortes du pays d'Égypte et il en prit le butin.
\par 20 Et après qu'Antiochus eut frappé l'Égypte, il revint de nouveau l'année cent quarante-troisième, et monta contre Israël et Jérusalem avec une grande multitude,
\par 21 Et il entra fièrement dans le sanctuaire, et enleva l'autel d'or, et le chandelier de lumière, et tous ses ustensiles,
\par 22 Et la table des pains de proposition, et les vases et les coupes et les encensoirs d'or, et le voile, et la couronne, et les ornements d'or qui étaient devant le temple, et il les ôta tous.
\par 23 Il prit aussi l'argent et l'or, et les objets précieux ; il prit aussi les trésors cachés qu'il trouva.
\par 24 Et après avoir tout emporté, il s'en alla dans son propre pays, après avoir commis un grand massacre, et il parla très fièrement.
\par 25 C'est pourquoi il y eut un grand deuil en Israël, partout où ils étaient ;
\par 26 De sorte que les princes et les anciens étaient en deuil, les vierges et les jeunes hommes étaient affaiblis, et la beauté des femmes était changée.
\par 27 Chaque époux se lamentait, et celle qui était assise dans la chambre nuptiale était dans le chagrin,
\par 28 Le pays aussi fut déplacé à cause de ses habitants, et toute la maison de Jacob fut couverte de confusion.
\par 29 Et après deux années pleinement expirées, le roi envoya son principal collecteur de tribut dans les villes de Juda, qui vint à Jérusalem avec une grande multitude,
\par 30 Et il leur dit des paroles pacifiques, mais tout n'était que tromperie ; car, après qu'ils lui eurent cru, il tomba subitement sur la ville, la frappa très durement, et détruisit une grande partie du peuple d'Israël.
\par 31 Et après avoir pris le butin de la ville, il y mit le feu, et démolit les maisons et les murs de tous côtés.
\par 32 Mais les femmes et les enfants les emmenèrent captifs et s'emparèrent du bétail.
\par 33 Alors ils bâtirent la ville de David avec une muraille grande et solide, et avec de puissantes tours, et ils en firent une forteresse pour eux.
\par 34 Et ils y mirent une nation pécheresse, des hommes méchants, et s'y fortifièrent.
\par 35 Ils l'entreposèrent aussi avec des armures et des vivres, et après avoir rassemblé les dépouilles de Jérusalem, ils les y déposèrent, et ainsi ils devinrent un piège épineux.
\par 36 Car c'était un lieu d'embûche contre le sanctuaire, et un mauvais adversaire pour Israël.
\par 37 Ainsi ils répandirent le sang innocent de tous les côtés du sanctuaire, et le profanèrent :
\par 38 De sorte que les habitants de Jérusalem s'enfuirent à cause d'eux ; après quoi la ville devint une habitation d'étrangers et devint étrangère à ceux qui y étaient nés ; et ses propres enfants l'ont quittée.
\par 39 Son sanctuaire fut dévasté comme un désert, ses fêtes furent changées en deuil, ses sabbats en opprobre, son honneur en mépris.
\par 40 Comme sa gloire avait été, son déshonneur s'est accru, et sa grandeur s'est transformée en deuil.
\par 41 Le roi Antiochus écrivit à tout son royaume que tous seraient un seul peuple,
\par 42 Et chacun devait abandonner ses lois; ainsi tous les païens étaient d'accord selon le commandement du roi.
\par 43 Oui, beaucoup d'Israélites aussi consentirent à sa religion, sacrifièrent aux idoles et profanèrent le sabbat.
\par 44 Car le roi avait envoyé des lettres par messagers à Jérusalem et aux villes de Juda pour qu'elles suivent les lois étrangères du pays,
\par 45 Et interdisez les holocaustes, les sacrifices et les libations dans le temple ; et qu'ils devraient profaner les sabbats et les jours de fête :
\par 46 Et tu souilleras le sanctuaire et le peuple saint :
\par 47 Élevez des autels, des bosquets et des chapelles d'idoles, et sacrifiez de la chair de porc et des bêtes impures.
\par 48 Qu'ils laissent aussi leurs enfants incirconcis et rendent leurs âmes abominables par toutes sortes d'impuretés et de profanations :
\par 49 Jusqu'à la fin, ils pourraient oublier la loi et changer toutes les ordonnances.
\par 50 Et quiconque ne ferait pas selon le commandement du roi, il dit qu'il mourrait.
\par 51 De la même manière, il écrivit à tout son royaume, et établit des surveillants sur tout le peuple, ordonnant aux villes de Juda de sacrifier, ville par ville.
\par 52 Alors une grande partie du peuple se rassembla auprès d'eux, à savoir tous ceux qui avaient abandonné la loi ; et c'est ainsi qu'ils commettèrent des maux dans le pays ;
\par 53 Et ils chassèrent les Israélites dans des lieux secrets, partout où ils pouvaient chercher du secours.
\par 54 Le quinzième jour du mois de Casleu, l'année cent quarante-cinquième, ils dressèrent sur l'autel l'abomination de la désolation, et bâtirent des autels à idoles dans toutes les villes de Juda, de tous côtés ;
\par 55 Et ils brûlaient de l'encens aux portes de leurs maisons et dans les rues.
\par 56 Et après avoir déchiré les livres de la loi qu'ils avaient trouvés, ils les brûlèrent au feu.
\par 57 Et quiconque serait trouvé avec quelqu'un du livre du testament, ou si quelqu'un était soumis à la loi, le commandement du roi était de le faire mourir.
\par 58 Ainsi, selon leur autorité, ils s'adressaient chaque mois aux Israélites, à tous ceux qui se trouvaient dans les villes.
\par 59 Le vingt-cinquième jour du mois, ils sacrifièrent sur l'autel des idoles, qui était sur l'autel de Dieu.
\par 60 Et alors, selon le commandement, ils mirent à mort certaines femmes qui avaient fait circoncire leurs enfants.
\par 61 Et ils pendirent les enfants au cou, pillèrent leurs maisons et tuèrent ceux qui les avaient circoncis.
\par 62 Cependant, beaucoup en Israël étaient pleinement résolus et confirmés en eux-mêmes à ne manger aucune chose impure.
\par 63 C'est pourquoi il vaut mieux mourir, afin qu'ils ne se souillent pas par les viandes, et qu'ils ne profanent pas la sainte alliance. Alors ils moururent.
\par 64 Et il y eut une très grande colère contre Israël.

\chapter{2}

\par 1 En ces jours-là, Mattathias, fils de Jean, fils de Siméon, prêtre des fils de Joarib, naquit de Jérusalem, et habita à Modin.
\par 2 Et il eut cinq fils, Joannan, appelé Caddis :
\par 3 Simon; appelé Thassi :
\par 4 Judas, appelé Maccabée :
\par 5 Éléazar, appelé Avaran, et Jonathan, dont le nom était Apphus.
\par 6 Et quand il vit les blasphèmes qui étaient commis en Juda et à Jérusalem,
\par 7 Il dit : Malheur à moi ! Pourquoi suis-je né pour voir cette misère de mon peuple et de la ville sainte, et pour y habiter, quand elle fut livrée entre les mains de l'ennemi et le sanctuaire entre les mains des étrangers ?
\par 8 Son temple est devenu comme un homme sans gloire.
\par 9 Ses vases glorieux sont emmenés en captivité, ses enfants sont tués dans les rues, ses jeunes gens sont tués par l'épée de l'ennemi.
\par 10 Quelle nation n'a pas eu part à son royaume et n'a pas obtenu de son butin ?
\par 11 Tous ses ornements lui sont enlevés ; de femme libre, elle est devenue esclave.
\par 12 Et voici, notre sanctuaire, même notre beauté et notre gloire, est dévasté, et les païens l'ont profané.
\par 13 Dans quel but donc vivrons-nous plus longtemps ?
\par 14 Alors Mattathias et ses fils déchirèrent leurs vêtements, et revêtirent des sacs, et ils pleurèrent très fort.
\par 15 Pendant ce temps, les officiers du roi, ceux qui poussaient le peuple à la révolte, entraient dans la ville de Modin pour lui faire des sacrifices.
\par 16 Et comme beaucoup d'Israël arrivaient vers eux, Mattathias aussi et ses fils se rassemblèrent.
\par 17 Alors les officiers du roi répondirent et dirent à Mattathias : Tu es un chef, un homme honorable et grand dans cette ville, et tu es fortifié par des fils et des frères.
\par 18 Maintenant donc, viens premièrement, et accomplis le commandement du roi, comme l'ont fait tous les païens, oui, et les hommes de Juda aussi, et ceux qui restent à Jérusalem : ainsi toi et ta maison serez-vous au nombre des les amis du roi, et toi et tes enfants serez honorés d'argent et d'or, et de nombreuses récompenses.
\par 19 Alors Mattathias répondit et parla à haute voix, bien que toutes les nations qui sont sous la domination du roi lui obéissent, et qu'elles s'éloignent chacune de la religion de leurs pères et consentent à ses commandements :
\par 20 Pourtant, moi, mes fils et mes frères, nous marcherons dans l'alliance de nos pères.
\par 21 À Dieu ne plaise que nous abandonnions la loi et les ordonnances.
\par 22 Nous n'écouterons pas les paroles du roi, pour nous éloigner de notre religion, ni à droite, ni à gauche.
\par 23 Alors qu'il avait fini de prononcer ces paroles, un des Juifs vint, à la vue de tous, pour offrir un sacrifice sur l'autel qui était à Modin, selon l'ordre du roi.
\par 24 Ce que Mattathias vit, s'enflamma de zèle, et ses reins tremblèrent, et il ne put s'empêcher de montrer sa colère selon le jugement ; c'est pourquoi il courut et le tua sur l'autel.
\par 25 Il tua également à cette époque-là le commissaire du roi, qui obligeait les hommes à sacrifier, et il démolit l'autel.
\par 26 Il agit ainsi avec zèle pour la loi de Dieu, comme Phinées envers Zambri, fils de Salom.
\par 27 Et Mattathias criait à haute voix dans toute la ville, disant : Quiconque est zélé pour la loi et respecte l'alliance, qu'il me suive.
\par 28 Alors lui et ses fils s'enfuirent dans les montagnes, et laissèrent tout ce qu'ils possédaient dans la ville.
\par 29 Alors beaucoup de ceux qui recherchaient la justice et le jugement descendirent dans le désert pour y habiter :
\par 30 Eux, leurs enfants et leurs femmes ; et leur bétail; parce que les afflictions se sont multipliées pour eux.
\par 31 Or, lorsqu'on apprit aux serviteurs du roi et à l'armée qui était à Jérusalem, dans la ville de David, que certains hommes, qui avaient violé l'ordre du roi, étaient descendus dans les lieux secrets du désert,
\par 32 Ils les poursuivirent en grand nombre, et les ayant rattrapés, ils campèrent contre eux et leur firent la guerre le jour du sabbat.
\par 33 Et ils leur dirent : Que ce que vous avez fait jusqu'à présent suffise ; sortez et faites selon le commandement du roi, et vous vivrez.
\par 34 Mais ils dirent : Nous ne sortirons pas, et nous n'observerons pas l'ordre du roi, pour profaner le jour du sabbat.
\par 35 Alors ils leur livrèrent le combat en toute hâte.
\par 36 Mais ils ne leur répondirent pas, ne leur jetèrent pas de pierre et ne fermèrent pas les lieux où ils se cachaient ;
\par 37 Mais il dit : Mourons tous dans notre innocence ; le ciel et la terre témoigneront pour nous que vous nous avez fait mourir injustement.
\par 38 Alors ils se levèrent contre eux dans une bataille le jour du sabbat, et ils les tuèrent, avec leurs femmes, leurs enfants et leur bétail, au nombre de mille personnes.
\par 39 Or, lorsque Mattathias et ses amis comprirent cela, ils les pleurèrent profondément.
\par 40 Et l'un d'eux dit à l'autre : Si nous faisons tous comme nos frères, et si nous ne luttons pas pour notre vie et nos lois contre les païens, ils nous déracineront bientôt de la terre.
\par 41 En ce temps-là donc, ils décrétèrent, disant : Quiconque viendra nous combattre le jour du sabbat, nous le combattrons ; nous ne mourrons pas non plus, comme nos frères qui ont été assassinés dans les lieux secrets.
\par 42 Alors vint vers lui un groupe d'Assidéens qui étaient des hommes puissants d'Israël, tous ceux qui étaient volontairement dévoués à la loi.
\par 43 Et tous ceux qui fuyaient pour être persécutés se joignirent à eux et leur furent un refuge.
\par 44 Alors ils joignirent leurs forces, et frappèrent les pécheurs dans leur colère, et les méchants dans leur colère ; mais les autres s'enfuirent vers les païens pour chercher du secours.
\par 45 Alors Mattathias et ses amis firent le tour et démolirent les autels.
\par 46 Et quels que soient les enfants qu'ils trouvèrent incirconcis sur le territoire d'Israël, ils les circoncis vaillamment.
\par 47 Ils poursuivirent aussi les hommes orgueilleux, et l'ouvrage prospéra entre leurs mains.
\par 48 Ainsi ils retirèrent la loi de la main des païens et de la main des rois, et ils ne laissèrent pas non plus que le pécheur triomphât.
\par 49 Alors que le moment de la mort de Mattathias approchait, il dit à ses fils : Maintenant l'orgueil et la réprimande ont acquis de la force, et le temps de la destruction et la colère de l'indignation.
\par 50 Maintenant donc, mes fils, soyez zélés pour la loi, et donnez votre vie pour l'alliance de vos pères.
\par 51 Rappelez-vous les actes que nos pères ont faits en leur temps ; ainsi vous recevrez un grand honneur et un nom éternel.
\par 52 Abraham n'a-t-il pas été trouvé fidèle dans la tentation, et cela lui a été imputé à justice ?
\par 53 Joseph, au temps de sa détresse, garda le commandement et fut nommé seigneur de l'Égypte.
\par 54 Phinées, notre père, en étant zélé et fervent, a obtenu l'alliance d'un sacerdoce éternel.
\par 55 Jésus, pour avoir accompli la parole, a été fait juge en Israël.
\par 56 Caleb pour avoir rendu témoignage avant que l'assemblée ne reçoive l'héritage du pays.
\par 57 David, pour sa miséricorde, possédait le trône d'un royaume éternel.
\par 58 Elie, pour son zèle et son fervent pour la loi, fut enlevé au ciel.
\par 59 Ananias, Azarias et Misaël, en croyant, furent sauvés de la flamme.
\par 60 Daniel, à cause de son innocence, fut délivré de la gueule des lions.
\par 61 Et ainsi, pensez à travers tous les âges, qu'aucun de ceux qui mettent leur confiance en lui ne sera vaincu.
\par 62 Ne craignez donc pas les paroles d'un homme pécheur, car sa gloire sera du fumier et des vers.
\par 63 Aujourd'hui, il sera élevé et demain il ne sera plus retrouvé, parce qu'il est retourné dans sa poussière, et sa pensée est vaine.
\par 64 C'est pourquoi, vous, mes fils, soyez vaillants et montrez-vous des hommes pour la loi ; car c'est par elle que vous obtiendrez la gloire.
\par 65 Et voici, je sais que ton frère Simon est un homme de conseil, prête-lui toujours l'oreille : il sera pour toi ton père.
\par 66 Quant à Judas Maccabée, il a été puissant et fort dès sa jeunesse : qu'il soit votre capitaine et qu'il combatte le combat du peuple.
\par 67 Prenez aussi avec vous tous ceux qui observent la loi, et vengez-vous du tort de votre peuple.
\par 68 Récompensez pleinement les païens, et soyez attentifs aux commandements de la loi.
\par 69 Alors il les bénit et fut recueilli auprès de ses pères.
\par 70 Et il mourut la cent quarante-sixième année, et ses fils l'enterrèrent dans les sépulcres de ses pères à Modin, et tout Israël fit de grandes lamentations sur lui.

\chapter{3}

\par 1 Alors son fils Judas, appelé Maccabée, se leva à sa place.
\par 2 Et tous ses frères l'aidèrent, ainsi que tous ceux qui étaient aux côtés de son père, et ils combattirent joyeusement la bataille d'Israël.
\par 3 Il fit donc grand honneur à son peuple, et il revêtit une cuirasse comme un géant, et ceint ses harnais de guerre, et il combattit, protégeant l'armée avec son épée.
\par 4 Dans ses actes, il était comme un lion et comme un jeune lion rugissant après sa proie.
\par 5 Car il poursuivait les méchants, il les recherchait, et il brûlait ceux qui indisposaient son peuple.
\par 6 C'est pourquoi les méchants reculèrent par crainte de lui, et tous les ouvriers d'iniquité furent troublés, parce que le salut prospérait entre ses mains.
\par 7 Il a aussi attristé beaucoup de rois, et a réjoui Jacob par ses actes, et son mémorial est béni pour toujours.
\par 8 Il parcourut également les villes de Juda, exterminant parmi elles les impies et détournant la colère d'Israël.
\par 9 De sorte qu'il était renommé jusqu'à l'extrémité de la terre, et il reçut auprès de lui ceux qui étaient prêts à périr.
\par 10 Alors Apollonius rassembla les païens et une grande armée de Samarie pour combattre contre Israël.
\par 11 Ce que Judas s'aperçut, il sortit à sa rencontre, le frappa et le tua : beaucoup tombèrent aussi tués, mais les autres s'enfuirent.
\par 12 C'est pourquoi Judas prit leur butin, ainsi que l'épée d'Apollonios, et avec cela il combattit toute sa vie.
\par 13 Or, lorsque Seron, prince de l'armée de Syrie, entendit dire que Judas avait rassemblé auprès de lui une multitude et un groupe de fidèles pour sortir avec lui à la guerre ;
\par 14 Il dit : Je me procurerai un nom et un honneur dans le royaume ; car j'irai combattre Judas et ceux qui sont avec lui, qui méprisent le commandement du roi.
\par 15 Alors il le prépara à monter, et une armée nombreuse d'impies alla avec lui pour l'aider et se venger des enfants d'Israël.
\par 16 Et comme il approchait de la montée de Bethhoron, Judas sortit à sa rencontre avec un petit groupe :
\par 17 Qui, voyant l'armée venir à leur rencontre, dit à Judas : Comment pourrons-nous, étant si peu nombreux, lutter contre une si grande multitude et une si forte force, puisque nous sommes prêts à nous évanouir à cause du jeûne de tout cela ? jour?
\par 18 A quoi Judas répondit : Il n'est pas difficile pour beaucoup d'être enfermés entre les mains de quelques-uns ; et avec le Dieu du ciel, tout est un, livrer avec une grande multitude ou une petite compagnie :
\par 19 Car la victoire dans la bataille ne dépend pas de la multitude d'une armée ; mais la force vient du ciel.
\par 20 Ils viennent contre nous avec beaucoup d'orgueil et d'iniquité pour nous détruire, ainsi que nos femmes et nos enfants, et pour nous gâter :
\par 21 Mais nous luttons pour nos vies et nos lois.
\par 22 C'est pourquoi l'Eternel lui-même les renversera devant nous ; et quant à vous, n'ayez pas peur d'eux.
\par 23 Dès qu'il eut cessé de parler, il se jeta brusquement sur eux, et ainsi Seron et son armée furent renversés devant lui.
\par 24 Et ils les poursuivirent depuis la descente de Bethhoron jusqu'à la plaine, où environ huit cents hommes furent tués ; et le reste s'enfuit au pays des Philistins.
\par 25 Alors la crainte de Judas et de ses frères, et une très grande frayeur, commencèrent à s'abattre sur les nations qui les entouraient.
\par 26 De sorte que sa renommée parvenait jusqu'au roi, et que toutes les nations parlaient des batailles de Judas.
\par 27 Or, lorsque le roi Antiochus apprit ces choses, il fut rempli d'indignation ; c'est pourquoi il envoya et rassembla toutes les forces de son royaume, même une armée très forte.
\par 28 Il ouvrit aussi son trésor et paya ses soldats pendant un an, leur ordonnant d'être prêts chaque fois qu'il en aurait besoin.
\par 29 Néanmoins, lorsqu'il vit que l'argent de ses trésors manquait et que les tributs dans le pays étaient petits, à cause des dissensions et de la peste qu'il avait provoquées dans le pays en supprimant les lois qui étaient d'autrefois. ;
\par 30 Il craignait de ne plus pouvoir supporter les charges, ni d'avoir des présents à donner aussi généreusement qu'auparavant, car il avait abondé au-dessus des rois qui étaient avant lui.
\par 31 C'est pourquoi, étant très perplexe dans son esprit, il résolut d'aller en Perse, d'y prendre les tributs des pays et de rassembler beaucoup d'argent.
\par 32 Il laissa donc Lysias, noble et de sang royal, pour surveiller les affaires du roi depuis le fleuve Euphrate jusqu'aux frontières de l'Égypte.
\par 33 Et pour élever son fils Antiochus, jusqu'à ce qu'il revienne.
\par 34 Il lui remit la moitié de son armée et les éléphants, et lui confia la responsabilité de tout ce qu'il aurait fait, ainsi que de ceux qui habitaient en Juda et à Jérusalem.
\par 35 C'est-à-dire qu'il enverrait une armée contre eux, pour détruire et déraciner la force d'Israël et le reste de Jérusalem, et pour enlever leur mémorial de ce lieu ;
\par 36 Et qu'il placerait des étrangers dans tous leurs quartiers, et qu'il diviserait leur pays par le sort.
\par 37 Le roi prit donc la moitié des forces qui restaient et quitta Antioche, sa ville royale, la cent quarante-septième année ; et après avoir passé le fleuve Euphrate, il traversa les hauts pays.
\par 38 Alors Lysias choisit Ptolémée, fils de Doryménès, Nicanor et Gorgias, hommes vaillants parmi les amis du roi :
\par 39 Et il envoya avec eux quarante mille fantassins et sept mille cavaliers pour entrer dans le pays de Juda et le détruire, comme le roi l'avait ordonné.
\par 40 Ils partirent donc de toutes leurs forces, et vinrent camper près d'Emmaüs dans la plaine.
\par 41 Et les marchands du pays, entendant parler d'eux, prirent beaucoup d'argent et d'or, avec des serviteurs, et vinrent au camp pour acheter les enfants d'Israël comme esclaves ; une puissance aussi de la Syrie et du pays d'Israël les Philistins se joignirent à eux.
\par 42 Or, lorsque Judas et ses frères virent que les misères se multipliaient et que les forces campaient dans leurs frontières, car ils savaient comment le roi avait donné l'ordre de détruire le peuple et de l'abolir complètement ;
\par 43 Ils se dirent les uns aux autres : restaurons la fortune déchue de notre peuple, et combattons pour notre peuple et pour le sanctuaire.
\par 44 Alors l'assemblée se rassembla pour se préparer au combat, prier et demander miséricorde et compassion.
\par 45 Or Jérusalem était vide comme un désert, aucun de ses enfants n'entrait ni ne sortait ; le sanctuaire aussi était foulé aux pieds, et des étrangers gardaient la forteresse ; les païens avaient leur habitation à cet endroit ; et la joie fut ôtée à Jacob, et le flûte et la harpe cessèrent.
\par 46 C'est pourquoi les Israélites se rassemblèrent et arrivèrent à Maspha, en face de Jérusalem ; car c'était à Maspha qu'on priait autrefois en Israël.
\par 47 Alors ils jeûnèrent ce jour-là, et revêtirent des sacs, et jetèrent de la cendre sur leur tête, et déchirèrent leurs vêtements,
\par 48 Et il ouvrit le livre de la loi, dans lequel les païens avaient cherché à peindre l'image de leurs images.
\par 49 Ils apportèrent aussi les vêtements des prêtres, les prémices et les dîmes, et ils réveillèrent les Nazaréens, qui avaient accompli leurs jours.
\par 50 Alors ils crièrent d'une voix forte vers le ciel, disant : Que ferons-nous de ceux-ci, et où les emporterons-nous ?
\par 51 Car ton sanctuaire est foulé et profané, et tes prêtres sont accablés et abaissés.
\par 52 Et voici, les païens se sont rassemblés contre nous pour nous détruire : tu sais ce qu'ils imaginent contre nous.
\par 53 Comment pourrons-nous leur tenir tête, si toi, ô Dieu, tu ne sois notre secours ?
\par 54 Alors ils sonnèrent des trompettes et crièrent à haute voix.
\par 55 Et après cela, Judas ordonna chefs du peuple, chefs de milliers, et de centaines, et de cinquantaines, et de dix.
\par 56 Mais quant à ceux qui bâtissaient des maisons, ou qui avaient des fiancées, ou qui plantaient des vignes, ou qui avaient peur, il ordonna à ceux-là de rentrer chacun dans sa maison, selon la loi.
\par 57 Le camp se retira donc et s'établit du côté sud d'Emmaüs.
\par 58 Et Judas dit : Armez-vous, et soyez des hommes vaillants, et veillez à être prêts contre le matin, afin de combattre contre ces nations qui se sont rassemblées contre nous pour nous détruire, nous et notre sanctuaire.
\par 59 Car il vaut mieux pour nous mourir au combat que de voir les calamités de notre peuple et de notre sanctuaire.
\par 60 Néanmoins, comme la volonté de Dieu est dans le ciel, qu'il fasse ainsi.

\chapter{4}

\par 1 Gorgias prit alors cinq mille fantassins et mille cavaliers parmi les meilleurs, et il sortit du camp pendant la nuit ;
\par 2 Jusqu'au bout, il pouvait se précipiter sur le camp des Juifs et les frapper subitement. Et les hommes de la forteresse étaient ses guides.
\par 3 Or, lorsque Judas apprit cela, il se retira lui-même, avec les hommes vaillants qui étaient avec lui, afin de frapper l'armée du roi qui était à Emmaüs.
\par 4 Tandis que les forces étaient encore dispersées du camp.
\par 5 Entre-temps, Gorgias entra de nuit dans le camp de Judas ; et n'y trouvant personne, il les chercha dans les montagnes. Car il dit : Ces hommes fuient loin de nous.
\par 6 Mais dès que le jour parut, Judas se montra dans la plaine avec trois mille hommes, qui pourtant n'avaient ni armures ni épées en tête.
\par 7 Et ils virent que le camp des païens était fort et bien harnaché, et entouré de cavaliers tout autour ; et ceux-là étaient des experts en guerre.
\par 8 Alors Judas dit aux hommes qui étaient avec lui : Ne craignez pas leur multitude, et n'ayez pas peur de leur assaut.
\par 9 Rappelez-vous comment nos pères furent délivrés dans la mer Rouge, lorsque Pharaon les poursuivit avec une armée.
\par 10 Maintenant donc, crions vers le ciel, si par hasard le Seigneur veut avoir pitié de nous, et se souviennent de l'alliance de nos pères, et détruisons aujourd'hui cette armée devant notre face.
\par 11 Afin que tous les païens sachent qu'il y a quelqu'un qui délivre et sauve Israël.
\par 12 Alors les étrangers levèrent les yeux et les virent venir à leur rencontre.
\par 13 C'est pourquoi ils sortirent du camp pour combattre ; mais ceux qui étaient avec Judas sonnèrent de la trompette.
\par 14 Ils entrèrent donc en bataille, et les païens, déconcertés, s'enfuirent dans la plaine.
\par 15 Mais tous ceux d'entre eux furent tués par l'épée ; car ils les poursuivirent jusqu'à Gazera, et jusqu'aux plaines d'Idumée, d'Azotus et de Jamnia, de sorte qu'ils furent tués sur trois mille hommes.
\par 16 Ceci fait, Judas revint avec son armée après les avoir poursuivis,
\par 17 Et il dit au peuple : Ne soyez pas avides de butin, car il y a une bataille devant nous.
\par 18 Et Gorgias et son armée sont ici près de nous sur la montagne ; mais tenez-vous maintenant contre nos ennemis, et vaincrez-les, et après cela vous pourrez hardiment prendre le butin.
\par 19 Comme Judas prononçait encore ces paroles, une partie d'eux apparut regardant hors de la montagne :
\par 20 Qui, lorsqu'ils s'aperçurent que les Juifs avaient mis leur armée en fuite et brûlaient les tentes ; car la fumée qu'on voyait annonçait ce qui s'était passé :
\par 21 Lorsqu'ils aperçurent donc ces choses, ils furent très effrayés, et voyant aussi l'armée de Judas dans la plaine prête à combattre,
\par 22 Ils s'enfuirent tous dans le pays des étrangers.
\par 23 Alors Judas revint pour piller les tentes, où ils obtinrent beaucoup d'or, et d'argent, et de la soie bleue, et de la pourpre de la mer, et de grandes richesses.
\par 24 Après cela, ils rentrèrent chez eux, chantèrent un chant de remerciement et louèrent le Seigneur dans les cieux, parce que c'est bon, parce que sa miséricorde dure à toujours.
\par 25 Ainsi Israël eut une grande délivrance ce jour-là.
\par 26 Or tous les étrangers qui s'étaient échappés vinrent et racontèrent à Lysias ce qui était arrivé :
\par 27 Celui-ci, lorsqu'il en entendit parler, fut confondu et découragé, parce que ni les choses qu'il voulait n'avaient été faites à Israël, ni les choses que le roi lui avait commandées n'étaient arrivées.
\par 28 L'année suivante, Lysias rassembla donc soixante mille fantassins d'élite et cinq mille cavaliers, pour les soumettre.
\par 29 Ils arrivèrent donc en Idumée, et dressèrent leurs tentes à Bethsura, et Judas les rencontra avec dix mille hommes.
\par 30 Et quand il vit cette puissante armée, il pria et dit : Béni sois-tu, ô Sauveur d'Israël, qui as réprimé la violence de l'homme fort par la main de ton serviteur David, et tu as livré l'armée des étrangers entre les mains de Jonathan, fils de Saül, et de son écuyer ;
\par 31 Enferme cette armée entre les mains de ton peuple Israël, et qu'elle soit confondue par sa puissance et sa cavalerie.
\par 32 Rendez-les sans courage, et faites tomber la hardiesse de leur force, et qu'ils tremblent devant leur destruction.
\par 33 Fais-les tomber avec l'épée de ceux qui t'aiment, et que tous ceux qui connaissent ton nom te louent avec actions de grâces.
\par 34 Alors ils joignirent le combat ; Et l'armée de Lysias tua environ cinq mille hommes, avant même qu'ils soient tués.
\par 35 Or, lorsque Lysias vit son armée mise en fuite, et la virilité des soldats de Judas, et combien ils étaient prêts à vivre ou à mourir vaillamment, il se rendit à Antiochia, et rassembla un groupe d'étrangers, et après avoir fait son armée plus grande qu'elle ne l'était, il se proposa de revenir en Judée.
\par 36 Alors Judas et ses frères dirent : Voici, nos ennemis sont déconcertés : montons pour purifier et consacrer le sanctuaire.
\par 37 Là-dessus, toute l'armée se rassembla et monta sur la montagne de Sion.
\par 38 Et quand ils virent le sanctuaire désolé, et l'autel profané, et les portes incendiées, et les buissons poussant dans les parvis comme dans une forêt, ou dans l'une des montagnes, oui, et les chambres des prêtres démolies ;
\par 39 Ils déchirèrent leurs vêtements, poussèrent de grandes lamentations et jetèrent de la cendre sur leur tête,
\par 40 Et ils tombèrent à plat ventre, la face contre terre, et sonnèrent l'alarme avec les trompettes, et crièrent vers le ciel.
\par 41 Alors Judas désigna des hommes pour combattre ceux qui étaient dans la forteresse, jusqu'à ce qu'il ait purifié le sanctuaire.
\par 42 Il choisit donc des prêtres au conversation irréprochable, tels que ceux qui prenaient plaisir à la loi :
\par 43 Qui a purifié le sanctuaire, et qui a mis à nu les pierres souillées dans un lieu impur.
\par 44 Et comme ils consultaient ce qu'il fallait faire de l'autel des holocaustes, qui avait été profané ;
\par 45 Ils pensèrent qu'il valait mieux le démolir, de peur que cela ne leur soit un reproche, parce que les païens l'avaient souillé. C'est pourquoi ils le démolirent,
\par 46 Et il posa les pierres sur la montagne du temple, dans un endroit convenable, jusqu'à ce qu'un prophète vienne pour montrer ce qu'il faudrait en faire.
\par 47 Alors ils prirent des pierres entières, selon la loi, et bâtirent un nouvel autel selon le premier ;
\par 48 Et ils composèrent le sanctuaire et les choses qui étaient à l'intérieur du temple, et ils sanctifièrent les parvis.
\par 49 Ils fabriquèrent aussi de nouveaux vases sacrés, et ils apportèrent dans le temple le chandelier, l'autel des holocaustes et des parfums, et la table.
\par 50 Et sur l'autel ils brûlèrent de l'encens, et ils allumèrent les lampes qui étaient sur le chandelier, pour éclairer le temple.
\par 51 Ils posèrent ensuite les pains sur la table, étendirent les voiles et achevèrent tous les ouvrages qu'ils avaient commencés à faire.
\par 52 Or, le vingt-cinquième jour du neuvième mois, appelé mois de Casleu, de la cent quarante-huitième année, ils se levèrent de bon matin,
\par 53 Et ils offrirent des sacrifices selon la loi sur le nouvel autel des holocaustes qu'ils avaient fait.
\par 54 Regardez, à quelle heure et quel jour les païens l'avaient profané, même en cela, il était consacré avec des chants, des cithers, des harpes et des cymbales.
\par 55 Alors tout le peuple tomba la face contre terre, adorant et louant le Dieu du ciel, qui leur avait donné un bon succès.
\par 56 Et ainsi ils observaient la dédicace de l'autel pendant huit jours, et offraient des holocaustes avec allégresse, et sacrifiaient le sacrifice de délivrance et de louange.
\par 57 Ils décorèrent aussi le devant du temple de couronnes d'or et de boucliers ; Ils renouvelèrent les portes et les chambres, et y accrochèrent des portes.
\par 58 Il y eut donc une très grande joie parmi le peuple, car l'opprobre des païens était écarté.
\par 59 De plus, Judas et ses frères et toute la congrégation d'Israël ordonnèrent que les jours de la dédicace de l'autel soient observés en leur saison d'année en année pendant huit jours, à partir du vingt-cinquième jour du mois Casleu, avec gaieté et allégresse.
\par 60 À cette époque aussi, ils bâtirent la montagne de Sion avec de hautes murailles et de fortes tours tout autour, de peur que les païens ne viennent la fouler comme ils l'avaient fait auparavant.
\par 61 Et ils établirent là une garnison pour la garder, et fortifièrent Bethsura pour la conserver ; afin que le peuple puisse avoir une défense contre l'Idumée.

\chapter{5}

\par 1 Or, lorsque les nations d'alentour apprirent que l'autel avait été rebâti et le sanctuaire rénové comme auparavant, cela leur déplut beaucoup.
\par 2 C'est pourquoi ils pensèrent détruire la génération de Jacob qui était parmi eux, et alors ils commencèrent à tuer et à détruire le peuple.
\par 3 Alors Judas combattit contre les enfants d'Ésaü en Idumée, près d'Arabattine, parce qu'ils assiégeaient Gaël ;
\par 4 Il se souvint aussi de l'injure des enfants de Bean, qui étaient un piège et une offense pour le peuple, en ce sens qu'ils les guettaient dans les chemins.
\par 5 Il les enferma donc dans les tours, et campa contre elles, et les détruisit entièrement, et brûla au feu les tours de ce lieu et tout ce qui s'y trouvait.
\par 6 Ensuite il passa chez les enfants d'Ammon, où il trouva une grande puissance et un peuple nombreux, avec Timothée leur chef.
\par 7 Il livra donc contre eux de nombreuses batailles, jusqu'à ce qu'à la fin ils furent vaincus devant lui ; et il les a frappés.
\par 8 Et après avoir pris Jazar et les villes qui lui appartenaient, il retourna en Judée.
\par 9 Alors les païens qui étaient à Galaad se rassemblèrent contre les Israélites qui étaient dans leurs quartiers, pour les détruire ; mais ils s'enfuirent vers la forteresse de Dathema.
\par 10 Et il envoya des lettres à Judas et à ses frères : Les païens qui nous entourent se sont rassemblés contre nous pour nous détruire :
\par 11 Et ils se préparent à venir prendre la forteresse où nous nous sommes enfuis, Timothée étant le capitaine de leur armée.
\par 12 Venez donc maintenant et délivrez-nous de leurs mains, car beaucoup d'entre nous ont été tués.
\par 13 Oui, tous nos frères qui étaient dans les lieux de Tobie sont mis à mort : leurs femmes et leurs enfants aussi ont été emmenés captifs et emportés leurs affaires ; et ils y ont détruit environ mille hommes.
\par 14 Pendant que ces lettres étaient encore lues, voici, d'autres messagers arrivèrent de Galilée, les vêtements déchirés, qui rapportèrent ceci :
\par 15 Et il dit : Ceux de Ptolémaïs, de Tyrus, de Sidon et de toute la Galilée des Gentils se sont rassemblés contre nous pour nous consumer.
\par 16 Or, lorsque Judas et le peuple entendirent ces paroles, une grande assemblée se rassembla pour consulter ce qu'ils devaient faire pour leurs frères, qui étaient en difficulté et qui étaient assaillis par eux.
\par 17 Alors Judas dit à Simon, son frère : Choisis des hommes, et va délivrer tes frères qui sont en Galilée, car moi et Jonathan mon frère irons dans le pays de Galaad.
\par 18 Il laissa donc Joseph, fils de Zacharie, et Azarias, chefs du peuple, avec le reste de l'armée en Judée pour la garder.
\par 19 À qui il donna ce commandement, disant : Prenez la responsabilité de ce peuple, et veillez à ne pas faire la guerre aux païens jusqu'à notre retour.
\par 20 Or, on donna à Simon trois mille hommes pour aller en Galilée, et à Judas huit mille hommes pour le pays de Galaad.
\par 21 Alors Simon partit en Galilée, où il livra de nombreuses batailles contre les païens, de sorte que les païens furent déconcertés par lui.
\par 22 Et il les poursuivit jusqu'à la porte de Ptolémaïs ; et environ trois mille hommes païens furent tués, dont il prit le butin.
\par 23 Et ceux qui étaient en Galilée et à Arbattis, avec leurs femmes et leurs enfants, et tout ce qu'ils possédaient, il les emmena avec lui, et les emmena en Judée avec une grande joie.
\par 24 Judas Maccabée et son frère Jonathan passèrent également le Jourdain et firent trois jours de marche dans le désert.
\par 25 Où ils rencontrèrent les Nabathites, qui vinrent vers eux en paix et leur racontèrent tout ce qui était arrivé à leurs frères au pays de Galaad :
\par 26 Et combien d'entre eux furent enfermés à Bosora, et Bosor, et Alema, Casphor, Maked et Carnaim ; toutes ces villes sont fortes et grandes :
\par 27 Et qu'ils étaient enfermés dans le reste des villes du pays de Galaad, et que demain ils avaient décidé d'amener leur armée contre les forts, de les prendre et de les détruire tous en un seul jour. .
\par 28 Alors Judas et son armée se dirigèrent brusquement par le chemin du désert vers Bosora ; Et après avoir conquis la ville, il tua tous les mâles au tranchant de l'épée, et prit tout leur butin, et brûla la ville par le feu.
\par 29 D'où il partit de nuit et alla jusqu'à arriver à la forteresse.
\par 30 Et tôt le matin, ils levèrent les yeux, et voici, il y avait un peuple innombrable portant des échelles et d'autres engins de guerre, pour prendre la forteresse ; car ils les attaquèrent.
\par 31 Lorsque Judas vit donc que la bataille était commencée, et que le cri de la ville montait jusqu'au ciel, avec des trompettes et un grand bruit,
\par 32 Il dit à son armée : Combattez aujourd'hui pour vos frères.
\par 33 Alors il sortit derrière eux en trois groupes, qui sonnaient de la trompette et criaient avec prière.
\par 34 Alors l'armée de Timothée, sachant que c'était Maccabée, s'enfuit loin de lui ; c'est pourquoi il les frappa d'un grand massacre ; de sorte qu'ils furent tués ce jour-là environ huit mille hommes.
\par 35 Ceci fait, Judas se détourna vers Maspha ; et après l'avoir attaqué, il prit et tua tous les mâles qui s'y trouvaient, et en reçut le butin et le brûla au feu.
\par 36 De là il partit et prit Casphon, Maged, Bosor et les autres villes du pays de Galaad.
\par 37 Après cela, Timothée rassembla une autre armée et campa contre Raphon au-delà du ruisseau.
\par 38 Alors Judas envoya des hommes pour observer l'armée, qui lui rapportèrent ces mots : Tous les païens qui nous entourent sont rassemblés auprès d'eux, même une très grande armée.
\par 39 Il a aussi engagé les Arabes pour les aider, et ils ont dressé leurs tentes au-delà du ruisseau, prêts à venir te combattre. Là-dessus, Judas alla à leur rencontre.
\par 40 Alors Timothée dit aux chefs de son armée : Quand Judas et son armée approcheront du ruisseau, s'il passe le premier vers nous, nous ne pourrons pas lui résister ; car il prévaudra puissamment contre nous :
\par 41 Mais s'il a peur et campe au-delà du fleuve, nous passerons vers lui et l'emporterons sur lui.
\par 42 Or, lorsque Judas s'approcha du ruisseau, il fit rester près du ruisseau les scribes du peuple, à qui il donna cet ordre : Ne laissez personne rester dans le camp, mais que tous viennent au combat.
\par 43 Alors il se dirigea d'abord vers eux, et tout le peuple après lui ; puis tous les païens, déconcertés devant lui, jetèrent leurs armes et s'enfuirent vers le temple qui était à Carnaïm.
\par 44 Mais ils prirent la ville et incendièrent le temple avec tout ce qui s'y trouvait. Ainsi Carnaïm fut soumis, et ils ne purent plus tenir plus longtemps devant Judas.
\par 45 Alors Judas rassembla tous les Israélites qui étaient dans le pays de Galaad, depuis le plus petit jusqu'au plus grand, même leurs femmes, et leurs enfants, et leurs affaires, une très grande armée, afin qu'ils puissent entrer dans le pays de Judée.
\par 46 Or, lorsqu'ils arrivèrent à Ephron (c'était une grande ville sur le chemin qu'ils devaient parcourir, très bien fortifiée), ils ne pouvaient s'en détourner, ni à droite ni à gauche, mais il fallait nécessairement passer par le au milieu de cela.
\par 47 Alors les habitants de la ville les enfermèrent et bouchèrent les portes avec des pierres.
\par 48 Sur quoi Judas leur envoya pacifiquement leur dire : Passons par votre pays pour entrer dans notre propre pays, et personne ne vous fera de mal ; nous ne passerons qu'à pied, mais ils ne lui ouvriront pas.
\par 49 C'est pourquoi Judas ordonna qu'une proclamation soit faite dans toute l'armée, que chacun dresse sa tente à l'endroit où il se trouve.
\par 50 Et les soldats campèrent et attaquèrent la ville tout ce jour-là et toute cette nuit, jusqu'à ce qu'à la fin la ville fut livrée entre ses mains.
\par 51 qui tua alors tous les mâles au tranchant de l'épée, qui rasa la ville, et en prit le butin, et qui traversa la ville sur ceux qui furent tués.
\par 52 Après cela, ils traversèrent le Jourdain et se dirigèrent vers la grande plaine devant Bethsan.
\par 53 Et Judas rassembla ceux qui étaient en arrière, et exhorta le peuple jusqu'à son arrivée dans le pays de Judée.
\par 54 Ils montèrent donc avec joie et allégresse sur la montagne de Sion, où ils offraient des holocaustes, car aucun d'eux n'était tué avant qu'ils ne soient revenus en paix.
\par 55 Or, à quelle époque Judas et Jonathan étaient au pays de Galaad, et Simon son frère en Galilée devant Ptolémaïs,
\par 56 Joseph, fils de Zacharie, et Azarias, chefs des garnisons, apprirent les actes de vaillance et les hauts faits de guerre qu'ils avaient accomplis.
\par 57 C'est pourquoi ils dirent : Donnons-nous aussi un nom, et allons combattre les païens qui nous entourent.
\par 58 Après avoir chargé la garnison qui était avec eux, ils se dirigèrent vers Jamnia.
\par 59 Alors Gorgias et ses hommes sortirent de la ville pour les combattre.
\par 60 Et ainsi Joseph et Azaras furent mis en fuite et poursuivis jusqu'aux frontières de la Judée ; et ce jour-là, environ deux mille hommes des enfants d'Israël furent tués.
\par 61 Il y eut donc un grand bouleversement parmi les enfants d'Israël, parce qu'ils n'obéissaient pas à Judas et à ses frères, mais qu'ils pensaient accomplir quelque acte de vaillance.
\par 62 Et ces hommes ne sont pas issus de la postérité de ceux par qui la délivrance a été donnée à Israël.
\par 63 Cependant l'homme Judas et ses frères étaient très renommés aux yeux de tout Israël et de tous les païens, partout où leur nom était entendu ;
\par 64 D'autant que le peuple s'assemblait vers eux avec de joyeuses acclamations.
\par 65 Ensuite Judas sortit avec ses frères et combattit contre les enfants d'Ésaü dans le pays vers le sud, où il frappa Hébron et ses villes, démolit sa forteresse et brûla ses tours tout autour. .
\par 66 De là il partit pour aller au pays des Philistins et traversa Samarie.
\par 67 À cette époque-là, certains prêtres, désireux de montrer leur vaillance, furent tués au combat, car ils sortirent pour combattre à l'improviste.
\par 68 Alors Judas se tourna vers Azotus, au pays des Philistins, et après avoir démoli leurs autels, brûlé au feu leurs images taillées et pillé leurs villes, il retourna au pays de Judée.

\chapter{6}

\par 1 Vers cette époque, le roi Antiochus, voyageant à travers les hauts pays, entendit dire qu'Elymais, dans le pays de Perse, était une ville très renommée pour ses richesses, son argent et son or ;
\par 2 Et il y avait là un temple très riche, dans lequel se trouvaient des couvertures d'or, des cuirasses et des boucliers qu'Alexandre, fils de Philippe, roi de Macédoine, qui régna le premier parmi les Grecs, y avait laissé.
\par 3 C'est pourquoi il vint et chercha à prendre la ville et à la piller ; mais il ne le pouvait pas, parce que les habitants de la ville, en ayant été avertis,
\par 4 Se leva contre lui dans la bataille ; alors il s'enfuit, et partit de là avec beaucoup de lourdeur, et retourna à Babylone.
\par 5 Et quelqu'un vint lui rapporter en Perse la nouvelle que les armées qui marchaient contre le pays de Judée étaient mises en fuite.
\par 6 Et que Lysias, qui sortit le premier avec une grande puissance, fut chassé des Juifs ; et qu'ils étaient rendus forts par l'armure, la puissance et la réserve de butin qu'ils avaient obtenus des armées qu'ils avaient détruites :
\par 7 Ils avaient également démoli l'abomination qu'il avait dressée sur l'autel de Jérusalem, et avaient entouré le sanctuaire de hautes murailles, comme auparavant, ainsi que sa ville de Bethsura.
\par 8 Or, lorsque le roi entendit ces paroles, il fut étonné et très ému ; sur quoi il le coucha sur son lit, et tomba malade de chagrin, parce que cela ne lui était pas arrivé comme il l'attendait.
\par 9 Et il resta là plusieurs jours ; car sa douleur était de plus en plus grande, et il tenait compte qu'il mourrait.
\par 10 C'est pourquoi il appela tous ses amis et leur dit : Mes yeux ne dorment plus, et mon cœur manque de soins.
\par 11 Et je pensais en moi-même : Dans quelle tribulation suis-je venu, et quel grand flot de misère est celui dans lequel je me trouve maintenant ! car j'étais généreux et aimé dans ma puissance.
\par 12 Mais maintenant je me souviens des maux que j'ai commis à Jérusalem, et du fait que j'ai pris tous les ustensiles d'or et d'argent qui s'y trouvaient, et que j'ai envoyé détruire les habitants de la Judée sans motif.
\par 13 Je comprends donc que c'est à cause de cela que ces troubles m'arrivent, et voici, je péris de grand chagrin dans un pays étranger.
\par 14 Puis il appela Philippe, un de ses amis, qu'il avait établi chef sur tout son royaume,
\par 15 Et il lui donna la couronne, et sa robe, et son sceau, afin qu'il élève son fils Antiochus et le nourrisse pour le royaume.
\par 16 Et le roi Antiochus mourut là la cent quarante-neuvième année.
\par 17 Lorsque Lysias apprit que le roi était mort, il fit régner à sa place Antiochus, son fils, qu'il avait élevé lorsqu'il était jeune, et il lui donna le nom d'Eupator.
\par 18 Vers cette époque, ceux qui étaient dans la tour enfermèrent les Israélites autour du sanctuaire, et cherchèrent toujours leur malheur et le renforcement des païens.
\par 19 C'est pourquoi Judas, voulant les détruire, assembla tout le peuple pour les assiéger.
\par 20 Et ils se rassemblèrent et les assiégèrent la cent cinquantième année, et il fit contre eux des montures à canon et d'autres engins.
\par 21 Mais certains des assiégés sortirent, auxquels se joignirent quelques hommes impies d'Israël.
\par 22 Et ils allèrent vers le roi, et lui dirent : Combien de temps faudra-t-il avant que tu exécutes le jugement et que tu vengees nos frères ?
\par 23 Nous avons voulu servir ton père, faire ce qu'il voulait de nous et obéir à ses commandements ;
\par 24 C'est pourquoi les membres de notre nation assiègent la tour et se sont éloignés de nous. De plus, tous ceux d'entre nous qu'ils ont pu apercevoir, ils ont tué et pillé notre héritage.
\par 25 Ils n'ont pas étendu la main seulement contre nous, mais aussi contre leurs frontières.
\par 26 Et voici, aujourd'hui ils assiègent la tour de Jérusalem pour s'en emparer ; ils ont aussi fortifié le sanctuaire et Bethsura.
\par 27 C'est pourquoi, si tu ne les empêches pas promptement, ils feront des choses plus grandes que celles-ci, et tu ne pourras pas non plus les gouverner.
\par 28 Lorsque le roi entendit cela, il se mit en colère et rassembla tous ses amis, les chefs de son armée et ceux qui avaient la garde des chevaux.
\par 29 Des bandes de mercenaires lui vinrent aussi d'autres royaumes et des îles de la mer.
\par 30 De sorte que le nombre de son armée était de cent mille fantassins, et vingt mille cavaliers, et trente-deux éléphants exercés au combat.
\par 31 Ceux-ci traversèrent l'Idumée et campèrent contre Bethsura, qu'ils attaquèrent plusieurs jours, fabriquant des machines de guerre ; mais ceux de Bethsura sortirent, les brûlèrent au feu et combattirent vaillamment.
\par 32 Alors Judas quitta la tour et campa à Bathzacharias, en face du camp du roi.
\par 33 Alors le roi, se levant de bon matin, marcha avec acharnement avec son armée vers Bathzacharias, où ses armées les préparèrent au combat et sonnèrent des trompettes.
\par 34 Et pour provoquer le combat des éléphants, ils leur montrèrent le sang des raisins et des mûres.
\par 35 Et ils répartirent les bêtes entre les armées, et pour chaque éléphant ils nommèrent mille hommes, armés de cottes de mailles et de casques d'airain sur la tête ; et en outre, pour chaque bête, cinq cents cavaliers parmi les meilleurs étaient ordonnés.
\par 36 Ceux-ci étaient prêts à toute occasion : partout où se trouvait la bête et partout où elle allait, ils allaient aussi, et ne s'éloignaient pas de lui.
\par 37 Et sur les bêtes il y avait de fortes tours de bois qui les couvraient chacune d'elles et étaient ceintes de dispositifs. Il y avait aussi sur chacune d'elles trente-deux hommes forts qui combattaient contre elles, à côté de l'Indien qui les dirigeait.
\par 38 Quant au reste des cavaliers, ils les placèrent de part et d'autre dans les deux parties de l'armée, leur donnant des signes sur ce qu'ils devaient faire, et les attelèrent partout au milieu des rangs.
\par 39 Or, lorsque le soleil brillait sur les boucliers d'or et d'airain, les montagnes en brillaient et brillaient comme des lampes de feu.
\par 40 Ainsi, une partie de l'armée du roi étant répartie sur les hautes montagnes, et une partie sur les vallées en contrebas, ils marchèrent en toute sécurité et en ordre.
\par 41 C'est pourquoi tous ceux qui entendaient le bruit de leur multitude, le pas de la troupe et le bruit des harnais, furent émus ; car l'armée était très grande et très puissante.
\par 42 Alors Judas et son armée s'approchèrent et entrèrent en bataille, et six cents hommes de l'armée du roi furent tués.
\par 43 Éléazar aussi, surnommé Savaran, s'aperçut que l'une des bêtes, armée d'un harnais royal, était plus haute que toutes les autres, et supposa que le roi était sur lui,
\par 44 Se mettre en péril, afin de délivrer son peuple et de lui obtenir un nom perpétuel :
\par 45 C'est pourquoi il courut contre lui avec courage au milieu de la bataille, tuant à droite et à gauche, de sorte qu'ils furent séparés de lui des deux côtés.
\par 46 Ce qui fait, il se glissa sous l'éléphant, le poussa dessous et le tua ; sur quoi l'éléphant tomba sur lui, et là il mourut.
\par 47 Mais le reste des Juifs, voyant la force du roi et la violence de ses troupes, se détournèrent d'eux.
\par 48 Alors l'armée du roi monta à Jérusalem pour les rencontrer, et le roi dressa ses tentes contre la Judée et contre la montagne de Sion.
\par 49 Mais il fit la paix avec ceux qui étaient à Bethsura; car ils sortirent de la ville, parce qu'ils n'y avaient pas de vivres pour supporter le siège, car c'était une année de repos pour le pays.
\par 50 Le roi prit donc Bethsura et y établit une garnison pour la garder.
\par 51 Quant au sanctuaire, il l'assiégea plusieurs jours ; et il y installa de l'artillerie avec des engins et des instruments pour lancer du feu et des pierres, et des pièces pour lancer des dards et des frondes.
\par 52 Sur quoi ils fabriquèrent aussi des moteurs contre leurs moteurs, et les tinrent en bataille pendant une longue saison.
\par 53 Mais à la fin, leurs ustensiles étant sans provisions (car c'était la septième année, et ceux de Judée qui avaient été délivrés des païens avaient mangé le reste du magasin ;)
\par 54 Il n'en restait qu'un petit nombre dans le sanctuaire, parce que la famine prévalait tellement contre eux, qu'ils furent obligés de se disperser, chacun chez soi.
\par 55 A cette époque-là, Lysias entendit dire que Philippe, que le roi Antiochus, de son vivant, avait chargé d'élever son fils Antiochus, afin qu'il soit roi,
\par 56 Il fut revenu de Perse et de Médie, ainsi que l'armée du roi qui l'accompagnait, et il cherchait à lui confier la direction des affaires.
\par 57 C'est pourquoi il partit en toute hâte et dit au roi et aux capitaines de l'armée et de la compagnie : Nous dépérissons chaque jour, et nos vivres sont peu nombreux, et le lieu que nous assiégeons est fort, et les affaires de le royaume repose sur nous :
\par 58 Maintenant donc soyons amis avec ces hommes, et faisons la paix avec eux et avec toute leur nation ;
\par 59 Et faites alliance avec eux, qu'ils vivront selon leurs lois, comme ils le faisaient auparavant ; car c'est pourquoi ils sont mécontents, et ont fait toutes ces choses, parce que nous avons aboli leurs lois.
\par 60 Le roi et les princes furent donc contents ; c'est pourquoi il les envoya faire la paix ; et ils l'acceptèrent.
\par 61 Le roi et les princes leur prêtèrent serment ; après quoi ils sortirent de la forteresse.
\par 62 Alors le roi entra dans la montagne de Sion ; mais voyant la force du lieu, il rompit le serment qu'il avait fait et ordonna d'abattre le mur tout autour.
\par 63 Puis il partit en toute hâte et retourna à Antiochia, où il trouva Philippe maître de la ville ; il combattit donc contre lui et s'empara de la ville de force.

\chapter{7}

\par 1 La cent cinquantième année, Démétrius, fils de Séleucus, quitta Rome, et monta avec quelques hommes dans une ville au bord de la mer, où il régna.
\par 2 Et comme il entrait dans le palais de ses ancêtres, ses troupes s'emparèrent d'Antiochus et de Lysias, pour les lui amener.
\par 3 C'est pourquoi, l'ayant su, il dit : Que je ne voie pas leurs visages.
\par 4 Alors son hôte les tua. Lorsque Démétrius fut placé sur le trône de son royaume,
\par 5 Tous les hommes méchants et impies d'Israël vinrent vers lui, ayant pour chef Alcimus, qui désirait devenir grand-prêtre.
\par 6 Et ils accusèrent le peuple auprès du roi, disant : Judas et ses frères ont tué tous tes amis et nous ont chassés de notre pays.
\par 7 Maintenant donc, envoie quelqu'un en qui tu as confiance, et qu'il aille voir quel mal il a fait parmi nous et dans le pays du roi, et qu'il les punisse avec tous ceux qui les aident.
\par 8 Alors le roi choisit Bacchidès, ami du roi, qui régnait au-delà du déluge, et qui était un grand homme du royaume et fidèle au roi,
\par 9 Et il l'envoya avec le méchant Alcimus, qu'il établit grand prêtre, et lui ordonna de se venger des enfants d'Israël.
\par 10 Ils partirent donc et arrivèrent avec une grande puissance au pays de Judée, où ils envoyèrent des messagers à Judas et à ses frères avec des paroles pacifiques et trompeuses.
\par 11 Mais ils ne prêtèrent aucune attention à leurs paroles ; car ils virent qu'ils étaient venus avec une grande puissance.
\par 12 Alors une troupe de scribes s'assembla auprès d'Alcimus et de Bacchidès pour demander justice.
\par 13 Les Assidéens furent les premiers parmi les enfants d'Israël à chercher la paix avec eux.
\par 14 Car ils disaient : Un prêtre de la postérité d'Aaron est venu avec cette armée, et il ne nous fera aucun mal.
\par 15 Alors il leur parla paisiblement et leur jura, disant : Nous ne ferons du mal ni à vous ni à vos amis.
\par 16 Sur quoi ils le crurent ; mais il prit d'eux soixante hommes et les tua en un seul jour, selon les paroles qu'il avait écrites :
\par 17 Ils ont chassé la chair de tes saints, et ils ont répandu leur sang tout autour de Jérusalem, et il n'y avait personne pour les enterrer.
\par 18 C'est pourquoi leur crainte et leur frayeur s'emparèrent de tout le peuple, qui disait : Il n'y a en eux ni vérité ni justice ; car ils ont rompu l'alliance et le serment qu'ils avaient fait.
\par 19 Après cela, Bacchidès éloigna de Jérusalem et dressa ses tentes à Bezeth, où il envoya et prit beaucoup d'hommes qui l'avaient abandonné, ainsi que certains du peuple, et après les avoir tués, il les jeta dans la grande fosse.
\par 20 Puis il confia le pays à Alcimus, et lui laissa un pouvoir pour l'aider. Bacchidès alla donc trouver le roi.
\par 21 Mais Alcimus se disputa pour le grand sacerdoce.
\par 22 Et c'est à lui que recoururent tous ceux qui troublaient le peuple, qui, après avoir pris le pays de Juda en leur pouvoir, faisaient beaucoup de mal à Israël.
\par 23 Or, Judas voyant tout le mal qu'Alcimus et sa troupe avaient commis parmi les Israélites, même parmi les païens,
\par 24 Il parcourut toutes les côtes de la Judée alentour, et se vengea de ceux qui s'étaient révoltés contre lui, de sorte qu'ils n'osèrent plus sortir dans le pays.
\par 25 D'un autre côté, quand Alcimus vit que Judas et sa compagnie avaient pris le dessus, et comprit qu'il n'était pas capable de supporter leur force, il retourna vers le roi et dit à tous les pires d'entre eux qu'il pourrait.
\par 26 Alors le roi envoya Nicanor, l'un de ses princes honorables, un homme qui nourrissait une haine mortelle envers Israël, avec l'ordre de détruire le peuple.
\par 27 Nicanor arriva donc à Jérusalem avec une grande force ; et il envoya trompeusement à Judas et à ses frères des paroles amicales, disant :
\par 28 Qu'il n'y ait pas de bataille entre moi et toi ; Je viendrai avec quelques hommes pour vous voir en paix.
\par 29 Il vint donc vers Judas, et ils se saluèrent paisiblement. Mais les ennemis étaient prêts à emmener Judas par la violence.
\par 30 Et Judas apprit ensuite qu'il était venu vers lui par tromperie, qu'il avait très peur de lui et qu'il ne voulait plus voir son visage.
\par 31 Nicanor aussi, voyant que son conseil était découvert, sortit pour combattre Judas près de Capharsalama :
\par 32 Environ cinq mille hommes du côté de Nikanor furent tués, et le reste s'enfuit dans la ville de David.
\par 33 Après cela, Nicanor monta sur la montagne de Sion, et certains des prêtres et certains des anciens du peuple sortirent du sanctuaire pour le saluer paisiblement et lui montrer l'holocauste offert pour le roi. .
\par 34 Mais il se moquait d'eux, se moquait d'eux, les injuriait honteusement et parlait avec fierté :
\par 35 Et il jura dans sa colère, disant : Si jamais Judas et son armée ne sont pas livrés entre mes mains, si jamais je reviens sain et sauf, j'incendierai cette maison. Et il sortit avec une grande colère.
\par 36 Alors les prêtres entrèrent et se tinrent devant l'autel et le temple, pleurant et disant :
\par 37 Toi, Seigneur, tu as choisi cette maison pour qu'elle soit appelée par ton nom et pour qu'elle soit une maison de prière et de supplication pour ton peuple :
\par 38 Vengez-vous de cet homme et de son armée, et laissez-les tomber par l'épée ; souvenez-vous de leurs blasphèmes, et ne permettez pas qu'ils continuent.
\par 39 Nicanor sortit donc de Jérusalem et dressa ses tentes à Bethhoron, où une armée venue de Syrie le rencontra.
\par 40 Mais Judas campa à Adasa avec trois mille hommes, et là il pria, disant :
\par 41 O Seigneur, quand ceux qui étaient envoyés par le roi des Assyriens blasphémèrent, ton ange sortit et en frappa cent quatre-vingt-cinq mille.
\par 42 De même, détruis aujourd'hui cette armée devant nous, afin que les autres sachent qu'il a blasphémé contre ton sanctuaire, et que tu le juges selon sa méchanceté.
\par 43 Le treizième jour du mois d'Adar, les armées joignirent le combat ; mais l'armée de Nicanor fut déconcertée, et lui-même fut le premier tué dans la bataille.
\par 44 Or, lorsque l'armée de Nicanor vit qu'il avait été tué, ils jetèrent leurs armes et s'enfuirent.
\par 45 Puis ils les poursuivirent pendant une journée de marche, depuis Adasa jusqu'à Gazera, sonnant l'alarme après eux avec leurs trompettes.
\par 46 Alors ils sortirent de toutes les villes de Judée alentour, et les encerclèrent ; de sorte qu'ils se retournèrent contre ceux qui les poursuivaient, et furent tous tués par l'épée, et il ne resta plus aucun d'eux.
\par 47 Ensuite, ils prirent le butin et la proie, et arrachèrent la tête et la main droite de Nicanor, qu'il étendait si fièrement, et les emmenèrent et les pendirent vers Jérusalem.
\par 48 C'est pour cette raison que le peuple se réjouit grandement, et ils célébrèrent ce jour comme un jour de grande allégresse.
\par 49 Et ils ordonnèrent de célébrer chaque année ce jour, qui était le treizième jour d'Adar.
\par 50 Ainsi le pays de Juda fut en repos pendant un peu de temps.

\chapter{8}

\par 1 Or Judas avait entendu parler des Romains, qu'ils étaient des hommes forts et vaillants, et capables d'accepter avec amour tout ce qui s'unissait à eux et de former une ligue d'amitié avec tout ce qui leur venait ;
\par 2 Et qu'ils étaient des hommes d'une grande vaillance. On lui parla aussi de leurs guerres et des actes nobles qu'ils avaient accomplis parmi les Galates, et comment ils les avaient vaincus et soumis à tribut ;
\par 3 Et ce qu'ils avaient fait dans le pays d'Espagne, pour l'exploitation des mines d'argent et d'or qui s'y trouvent ;
\par 4 Et que par leur politique et leur patience, ils avaient conquis tout le lieu, bien qu'il fût très loin d'eux ; et les rois aussi qui vinrent contre eux depuis les extrémités de la terre, jusqu'à ce qu'ils les aient déconcertés et les aient renversés, de sorte que les autres leur donnaient un tribut chaque année.
\par 5 En outre, comment ils avaient vaincu dans la bataille Philippe et Persée, roi des Citims, et d'autres qui s'étaient dressés contre eux et les avaient vaincus :
\par 6 Comment aussi Antiochus, le grand roi d'Asie, qui leur attaquait dans la bataille, ayant cent vingt éléphants, des cavaliers et des chars, et une très grande armée, fut décontenancé par eux ;
\par 7 Et comment ils l'ont pris vivant, et ont convenu que lui et ceux qui régneraient après lui paieraient un grand tribut et donneraient des otages, et ce qui avait été convenu,
\par 8 Et le pays de l'Inde, de la Médie et de la Lydie, et des pays les plus beaux, qu'ils lui prirent et donnèrent au roi Eumène.
\par 9 Et comment les Grecs avaient résolu de venir les détruire ;
\par 10 Et qu'eux, en ayant connaissance, envoyèrent contre eux un certain capitaine, et combattirent avec eux, tuèrent beaucoup d'entre eux, et emmenèrent captifs leurs femmes et leurs enfants, les pillèrent, prirent possession de leurs terres et démolirent leurs forteresses. , et les a amenés à être leurs serviteurs jusqu'à ce jour :
\par 11 On lui raconta en outre comment ils détruisirent et mirent sous leur domination tous les autres royaumes et îles qui leur résistèrent à un moment donné ;
\par 12 Mais ils gardaient l'amitié avec leurs amis et ceux qui comptaient sur eux, et ils avaient conquis des royaumes lointains et proches, de sorte que tous ceux qui entendaient parler de leur nom avaient peur d'eux.
\par 13 Et ceux qui veulent aider à un royaume, ceux-là règnent ; et qui encore ils voudraient, ils déplacent : enfin, qu'ils étaient grandement exaltés :
\par 14 Pourtant, malgré tout cela, aucun d'eux ne portait de couronne ou n'était vêtu de pourpre, pour en être magnifié :
\par 15 Et comment ils s'étaient fait une maison de sénat, où trois cent vingt hommes siégeaient quotidiennement en conseil, consultant toujours pour le peuple, afin qu'il soit bien ordonné.
\par 16 Et qu'ils confiaient chaque année leur gouvernement à un seul homme, qui régnait sur tout leur pays, et que tous obéissaient à celui-là, et qu'il n'y avait ni envie ni émulation parmi eux.
\par 17 En considération de ces choses, Judas choisit Eupolème, fils de Jean, fils d'Accos, et Jason, fils d'Éléazar, et les envoya à Rome pour conclure avec eux une alliance d'amitié et de confédération.
\par 18 Et pour les supplier qu'ils leur ôtent le joug ; car ils voyaient que le royaume des Grecs opprimait Israël par la servitude.
\par 19 Ils se rendirent donc à Rome, ce qui était un très grand voyage, et arrivèrent au Sénat, où ils parlèrent et dirent.
\par 20 Judas Maccabée, ses frères et le peuple des Juifs nous ont envoyés vers vous pour conclure une confédération et la paix avec vous, et afin que nous soyons enregistrés parmi vos confédérés et amis.
\par 21 Cette affaire plut donc bien aux Romains.
\par 22 Et voici la copie de l'épître que le sénat a réécrite sur des tables d'airain, et qu'il a envoyée à Jérusalem, afin qu'il y ait là un mémorial de paix et de confédération :
\par 23 Bon succès soit aux Romains et au peuple des Juifs, sur mer et sur terre pour toujours ; l'épée et l'ennemi soient loin d'eux,
\par 24 S'il y avait premièrement une guerre contre les Romains ou contre l'un de leurs confédérés dans toute leur domination,
\par 25 Le peuple des Juifs les aidera, selon le temps fixé, de tout son cœur.
\par 26 Ils ne donneront rien à ceux qui leur font la guerre, ni ne leur fourniront des vivres, des armes, de l'argent ou des navires, comme cela a semblé bon aux Romains ; mais ils respecteront leurs alliances sans rien prendre pour cela.
\par 27 De même, si la guerre éclate d'abord contre la nation des Juifs, les Romains les secourront de tout leur cœur, selon le moment qui leur sera fixé.
\par 28 Il ne sera pas non plus donné de vivres à ceux qui prennent parti contre eux, ni d'armes, ni d'argent, ni de navires, comme cela a semblé bon aux Romains ; mais ils garderont leurs alliances, et cela sans tromperie.
\par 29 C'est selon ces articles que les Romains ont conclu une alliance avec le peuple juif.
\par 30 Cependant, si par la suite l'une ou l'autre partie songe à se réunir pour ajouter ou diminuer quelque chose, ils pourront le faire à leur guise, et tout ce qu'ils ajouteront ou retrancheront sera ratifié.
\par 31 Et quant aux maux que Démétrius fait aux Juifs, nous lui avons écrit, disant : Pourquoi as-tu imposé ton joug sur nos amis et confédérés les Juifs ?
\par 32 Si donc ils se plaignent encore de toi, nous leur rendrons justice et nous combattrons contre toi sur mer et sur terre.

\chapter{9}

\par 1 De plus, lorsque Démétrius apprit que Nicanor et son armée avaient été tués au combat, il envoya une seconde fois Bacchidès et Alcimus dans le pays de Judée, et avec eux la principale force de son armée.
\par 2 Ils sortirent par le chemin qui mène à Galgala, et dressèrent leurs tentes devant Masaloth, qui est à Arbela, et après l'avoir conquise, ils tuèrent beaucoup de monde.
\par 3 Le premier mois de la cent cinquante-deuxième année, ils campèrent devant Jérusalem.
\par 4 D'où ils partirent et se rendirent à Bérée avec vingt mille fantassins et deux mille cavaliers.
\par 5 Or Judas avait dressé ses tentes à Eleasa, et trois mille hommes d'élite avec lui.
\par 6 Qui, voyant la multitude de l'autre armée si nombreuse, furent très effrayés ; sur quoi beaucoup sortirent de l'armée, de sorte qu'ils ne demeurèrent que huit cents hommes.
\par 7 Lorsque Judas vit donc que son armée s'éloignait et que la bataille se pressait sur lui, il fut très troublé dans son esprit et très affligé, car il n'avait pas le temps de les rassembler.
\par 8 Néanmoins il dit à ceux qui restaient : Levons-nous et marchons contre nos ennemis, si par hasard nous pouvons combattre avec eux.
\par 9 Mais ils le réprimandèrent, disant : Nous ne le pourrons jamais ; sauvons plutôt nos vies, et désormais nous reviendrons avec nos frères et combattrons contre eux ; car nous sommes peu nombreux.
\par 10 Alors Judas dit : À Dieu ne plaise que je fasse cela et que je m'enfuie loin d'eux ; si notre heure est venue, mourons vaillamment pour nos frères, et ne souillons pas notre honneur.
\par 11 Sur ce, l'armée des Bacchides sortit de leurs tentes et se plaça en face d'eux, leurs cavaliers étant divisés en deux troupes, et leurs frondeurs et leurs archers marchant devant l'armée et ceux qui marchaient en avant étaient tous des hommes vaillants.
\par 12 Quant à Bacchidès, il était dans l'aile droite ; alors l'armée s'approcha des deux côtés et sonna des trompettes.
\par 13 Eux aussi du côté de Judas sonnèrent aussi leurs trompettes, de sorte que la terre trembla au bruit des armées, et la bataille dura du matin au soir.
\par 14 Or, lorsque Judas s'aperçut que Bacchidès et la force de son armée étaient du côté droit, il emmena avec lui tous les hommes vaillants,
\par 15 qui déconcerta l'aile droite et la poursuivit jusqu'à la montagne d'Azotus.
\par 16 Mais quand ceux de l'aile gauche virent que ceux de l'aile droite étaient déconcertés, ils suivirent Judas et ceux qui étaient avec lui, les talonnant par-derrière :
\par 17 Sur quoi il y eut une bataille acharnée, à tel point qu'il y eut beaucoup de morts des deux côtés.
\par 18 Judas fut également tué, et le reste s'enfuit.
\par 19 Alors Jonathan et Simon prirent Judas, leur frère, et l'enterrèrent dans le sépulcre de ses pères à Modin.
\par 20 Et ils le pleurèrent, et tout Israël fit de grandes lamentations sur lui, et pleura plusieurs jours, disant :
\par 21 Comment est tombé le vaillant homme qui délivrait Israël !
\par 22 Quant aux autres choses concernant Judas et ses guerres, et les nobles actes qu'il a accomplis, et sa grandeur, elles ne sont pas écrites, car elles étaient très nombreuses.
\par 23 Après la mort de Judas, les méchants commencèrent à sortir la tête dans toutes les régions d'Israël, et tous ceux qui commettaient l'iniquité se levèrent.
\par 24 En ce temps-là aussi, il y eut une très grande famine, à cause de laquelle le pays se révolta et partit avec eux.
\par 25 Alors Bacchidès choisit les méchants et les établit seigneurs du pays.
\par 26 Et ils firent des recherches et recherchèrent les amis de Judas, et les conduisirent à Bacchidès, qui se vengea d'eux et les maltraita.
\par 27 Il y eut donc une grande affliction en Israël, telle qu'il n'y en avait pas eu de pareille depuis le temps où l'on ne voyait pas de prophète parmi eux.
\par 28 C'est pourquoi tous les amis de Judas se rassemblèrent et dirent à Jonathan :
\par 29 Depuis que ton frère Judas est mort, nous n'avons plus d'homme comme lui pour marcher contre nos ennemis, contre Bacchidès, et contre ceux de notre nation qui nous sont adversaires.
\par 30 Maintenant donc, nous t'avons choisi aujourd'hui pour être notre prince et capitaine à sa place, afin que tu puisses combattre nos batailles.
\par 31 Sur ce, Jonathan prit alors le gouvernement et se leva à la place de son frère Judas.
\par 32 Mais lorsque Bacchidès en eut connaissance, il chercha à le tuer.
\par 33 Alors Jonathan, et Simon son frère, et tous ceux qui étaient avec lui, s'en aperçurent, s'enfuirent dans le désert de Thécoé, et dressèrent leurs tentes près de l'eau de l'étang d'Asphar.
\par 34 Ce que Bacchidès comprit, et s'approcha du Jourdain avec toute son armée le jour du sabbat.
\par 35 Or Jonathan avait envoyé son frère Jean, chef du peuple, pour prier ses amis les Nabathites, afin qu'ils puissent laisser avec eux leur voiture, qui était longue.
\par 36 Mais les enfants de Jambri sortirent de Medaba, prirent Jean et tout ce qui lui appartenait, et partirent avec eux.
\par 37 Après cela, Jonathan et Simon, son frère, furent informés que les enfants de Jambri avaient fait un grand mariage et qu'ils amenaient la mariée de Nadabatha avec une grande traîne, comme étant la fille de l'un des grands princes de Chanaan.
\par 38 C'est pourquoi ils se souvinrent de Jean, leur frère, et montèrent et se cachèrent sous le couvert de la montagne.
\par 39 Là où ils levèrent les yeux et regardèrent, et voici, il y eut beaucoup de bruit et de grandes voitures ; et l'époux sortit, ainsi que ses amis et ses frères, à leur rencontre avec des tambours et des instruments de musique, et de nombreux armes.
\par 40 Alors Jonathan et ceux qui étaient avec lui se levèrent contre eux du lieu où ils étaient en embuscade, et les massacrèrent de telle sorte que beaucoup tombèrent morts, et le reste s'enfuit dans la montagne, et ils pris tout leur butin.
\par 41 Ainsi le mariage fut changé en deuil, et le bruit de leur mélodie en lamentation.
\par 42 Après avoir pleinement vengé le sang de leur frère, ils se tournèrent de nouveau vers le marais du Jourdain.
\par 43 Or, lorsque Bacchidès apprit cela, il arriva un jour de sabbat sur les rives du Jourdain avec une grande puissance.
\par 44 Alors Jonathan dit à ses compagnons : Montons maintenant et combattons pour notre vie, car cela ne nous tient plus aujourd'hui comme autrefois.
\par 45 Car voici, la bataille est devant nous et derrière nous, et les eaux du Jourdain d'un côté et de l'autre, le marais et la forêt aussi, et il n'y a pas d'endroit où nous puissions nous détourner.
\par 46 C'est pourquoi criez maintenant vers le ciel, afin que vous puissiez être délivrés de la main de vos ennemis.
\par 47 Là-dessus ils joignirent le combat, et Jonathan étendit la main pour frapper Bacchidès, mais celui-ci se détourna de lui.
\par 48 Alors Jonathan et ceux qui étaient avec lui sautèrent dans le Jourdain et nagèrent jusqu'à l'autre rive ; mais l'autre ne passa pas le Jourdain pour leur rejoindre.
\par 49 Ce jour-là, du côté de Bacchidès, environ mille hommes furent tués.
\par 50 Ensuite Bacchidès revint à Jérusalem et répara les villes fortes de Judée ; Il fortifia les forts de Jéricho, d'Emmaüs, de Bethhoron, de Béthel, de Thamnatha, de Pharathoni et de Taphon avec de hautes murailles, des portes et des barres.
\par 51 Et il y établit une garnison, afin qu'ils fassent du mal à Israël.
\par 52 Il fortifia aussi la ville de Bethsura, Gazera et la tour, et y mit des forces et des vivres.
\par 53 En outre, il prit en otages les fils des principaux du pays, et les mit dans la tour de Jérusalem pour y être gardés.
\par 54 La cent cinquante-troisième année, le deuxième mois, Alcimus ordonna de démolir le mur du parvis intérieur du sanctuaire ; il a également démoli les œuvres des prophètes
\par 55 Et comme il commençait à s'abaisser, Alcimus était déjà en proie à la tourmente, et ses entreprises étaient entravées; car sa bouche était fermée, et il était atteint de paralysie, de sorte qu'il ne pouvait plus rien dire, ni donner ordre concernant sa maison.
\par 56 Alcimus mourut donc à cette époque dans de grands tourments.
\par 57 Bacchidès, voyant qu'Alcimus était mort, revint auprès du roi. Le pays de Judée fut alors en repos pendant deux ans.
\par 58 Alors tous les hommes impies tinrent conseil, disant : Voici, Jonathan et sa troupe sont à l'aise et demeurent sans souci. C'est pourquoi nous allons maintenant amener ici Bacchidès, qui les prendra tous en une nuit.
\par 59 Alors ils allèrent le consulter.
\par 60 Puis il partit, et vint avec une grande armée, et envoya en secret des lettres à ses partisans en Judée, pour qu'ils prennent Jonathan et ceux qui étaient avec lui ; mais ils ne le purent pas, parce que leur conseil leur était connu.
\par 61 C'est pourquoi ils prirent environ cinquante personnes parmi les hommes du pays qui étaient les auteurs de ce mal, et les tuèrent.
\par 62 Ensuite Jonathan, Simon et ceux qui étaient avec lui se rendirent à Bethbasi, qui est dans le désert, et ils en réparèrent les ruines et la renforcèrent.
\par 63 Ce que Bacchidès apprit, il rassembla toute son armée et envoya parler à ceux qui étaient de Judée.
\par 64 Alors il partit et assiégea Bethbasi ; et ils luttèrent contre elle pendant une longue saison et fabriquèrent des machines de guerre.
\par 65 Mais Jonathan laissa son frère Simon dans la ville, et partit lui-même à la campagne, et il partit avec un certain nombre de personnes.
\par 66 Et il frappa Odonarkès et ses frères, et les enfants de Phasiron dans leur tente.
\par 67 Et quand il commença à les frapper, et qu'il approcha avec ses troupes, Simon et sa troupe sortirent de la ville et brûlèrent les machines de guerre,
\par 68 Et il combattit Bacchidès, qui fut déconcerté par eux, et ils le tourmentèrent cruellement, car ses conseils et ses travaux étaient vains.
\par 69 C'est pourquoi il fut très irrité contre les hommes méchants qui lui conseillaient de venir dans le pays, car il en tua beaucoup et se proposa de retourner dans son propre pays.
\par 70 Et lorsque Jonathan en eut connaissance, il lui envoya des ambassadeurs, afin qu'il fasse la paix avec lui et leur délivre les prisonniers.
\par 71 Ce qu'il accepta et fit selon ses exigences, et lui jura qu'il ne lui ferait jamais de mal tous les jours de sa vie.
\par 72 Après donc qu'il lui eut restitué les prisonniers qu'il avait fait autrefois sortir du pays de Judée, il revint et s'en alla dans son propre pays, et il ne revint plus dans leurs frontières.
\par 73 Ainsi l'épée cessa d'Israël ; mais Jonathan habita à Machmas et commença à gouverner le peuple ; et il fit périr les impies d'Israël.

\chapter{10}

\par 1 La cent soixantième année, Alexandre, fils d'Antiochus surnommé Épiphane, monta et prit Ptolémaïs; car le peuple l'avait reçu, grâce auquel il régnait là-bas,
\par 2 Or, lorsque le roi Démétrius l'apprit, il rassembla une armée extrêmement nombreuse et partit contre lui pour combattre.
\par 3 De plus, Démétrius envoya des lettres à Jonathan avec des paroles affectueuses, à mesure qu'il le magnifiait.
\par 4 Car il dit : Faisons d'abord la paix avec lui, avant qu'il ne se joigne à Alexandre contre nous.
\par 5 Autrement, il se souviendra de tous les maux que nous avons commis contre lui, contre ses frères et contre son peuple.
\par 6 C'est pourquoi il lui donna le pouvoir de rassembler une armée et de lui fournir des armes, afin qu'il puisse l'aider dans la bataille ; il ordonna également que les otages qui étaient dans la tour lui soient délivrés.
\par 7 Alors Jonathan vint à Jérusalem, et lut les lettres devant l'assistance de tout le peuple et de ceux qui étaient dans la tour :
\par 8 qui furent très effrayés lorsqu'ils apprirent que le roi lui avait donné le pouvoir de rassembler une armée.
\par 9 Sur quoi les gens de la tour livrèrent leurs otages à Jonathan, et il les livra à leurs parents.
\par 10 Cela fait, Jonathan s'établit à Jérusalem, et commença à bâtir et à réparer la ville.
\par 11 Et il ordonna aux ouvriers de bâtir les murs et la montagne de Sion et tout autour avec des pierres carrées pour fortification ; et ils l'ont fait.
\par 12 Alors les étrangers qui se trouvaient dans les forteresses que Bacchidès avait bâties s'enfuirent ;
\par 13 De sorte que chacun quitta sa place et s'en alla dans son propre pays.
\par 14 Seulement à Bethsura, certains de ceux qui avaient abandonné la loi et les commandements restèrent tranquilles : car c'était leur lieu de refuge.
\par 15 Or, lorsque le roi Alexandre eut entendu les promesses que Démétrius avait adressées à Jonathan, lorsqu'on lui parla des batailles et des actes nobles que lui et ses frères avaient accomplis, et des souffrances qu'ils avaient endurées,
\par 16 Il dit : Trouverons-nous un tel autre homme ? maintenant nous allons donc en faire notre ami et notre complice.
\par 17 Là-dessus il écrivit une lettre et la lui envoya, selon ces paroles, disant :
\par 18 Le roi Alexandre salue son frère Jonathan :
\par 19 Nous avons entendu parler de toi, que tu es un homme de grande puissance et que tu es digne d'être notre ami.
\par 20 C'est pourquoi nous t'ordonnons aujourd'hui pour être le grand prêtre de ta nation, et pour être appelé l'ami du roi ; (et avec cela il lui envoya une robe de pourpre et une couronne d'or :) et te demande de prendre notre parti et de garder amitié avec nous.
\par 21 Ainsi, le septième mois de la cent soixantième année, à la fête des tabernacles, Jonathan revêtit la robe sainte, rassembla ses forces et se munit de nombreuses armes.
\par 22 Quand Démétrius l'apprit, il fut très désolé et dit :
\par 23 Qu'avons-nous fait, pour qu'Alexandre nous empêche de nous lier d'amitié avec les Juifs pour se fortifier ?
\par 24 Je leur écrirai aussi des paroles d'encouragement, et je leur promets des dignités et des dons, afin d'avoir leur aide.
\par 25 Il leur envoya donc ceci : Le roi Démétrius salue le peuple juif :
\par 26 Puisque vous avez respecté vos alliances avec nous et que vous avez persévéré dans notre amitié, sans vous lier à nos ennemis, nous en avons entendu parler et nous nous en réjouissons.
\par 27 C'est pourquoi maintenant, continuez à nous être fidèles, et nous vous récompenserons bien pour les choses que vous faites en notre faveur,
\par 28 Et il vous accordera de nombreuses immunités et vous donnera des récompenses.
\par 29 Et maintenant je vous affranchis, et à cause de vous je libère tous les Juifs des tributs, de la coutume du sel et des impôts de la couronne,
\par 30 Et de ce qui m'appartient de recevoir le tiers de la semence et la moitié des fruits des arbres, je les libère à partir de ce jour, afin qu'ils ne soient pas enlevés du pays de Judée. , ni des trois gouvernements qui y sont ajoutés hors du pays de Samarie et de Galilée, à partir de ce jour et pour toujours.
\par 31 Que Jérusalem aussi soit sainte et libre, avec ses limites, tant en dîmes qu'en tributs.
\par 32 Et quant à la tour qui est à Jérusalem, je cède le pouvoir sur elle, et je donne au souverain sacrificateur qu'il y installe tels hommes qu'il choisira pour la garder.
\par 33 De plus, j'ai librement libéré tous les Juifs qui ont été emmenés captifs du pays de Judée dans n'importe quelle partie de mon royaume, et je veux que tous mes officiers remettent les tributs même de leur bétail.
\par 34 De plus, je veux que toutes les fêtes, et les sabbats, et les nouvelles lunes, et les jours solennels, et les trois jours avant la fête, et les trois jours après la fête, soient tous d'immunité et de liberté pour tous les Juifs de mon royaume.
\par 35 De plus, personne n'aura le pouvoir de se mêler ou de molester l'un d'entre eux dans quelque affaire que ce soit.
\par 36 Je veux en outre qu'il y ait parmi les forces du roi environ trente mille hommes juifs, à qui sera donnée une solde, comme il convient à toutes les forces du roi.
\par 37 Et parmi eux, certains seront placés dans les forteresses du roi, parmi lesquels aussi certains seront chargés des affaires du royaume, qui sont de confiance ; et je veux que leurs surveillants et gouverneurs soient eux-mêmes, et qu'ils vivez selon leurs propres lois, comme le roi l'a ordonné au pays de Judée.
\par 38 Et concernant les trois gouvernements qui seront ajoutés à la Judée à partir du pays de Samarie, qu'ils soient joints à la Judée, afin qu'ils soient considérés comme étant sous un seul, et qu'ils ne soient pas tenus d'obéir à une autre autorité que celle du grand prêtre.
\par 39 Quant à Ptolémaïs et au pays qui lui appartient, je le donne en don gratuit au sanctuaire de Jérusalem, pour les dépenses nécessaires du sanctuaire.
\par 40 Et je donne chaque année quinze mille sicles d'argent sur les comptes du roi, pour les lieux qui en dépendent.
\par 41 Et tout l'excédent que les officiers ne payaient pas comme autrefois, sera désormais consacré aux travaux du temple.
\par 42 Et en plus de cela, les cinq mille sicles d'argent qu'ils prenaient chaque année sur les comptes du temple, ces choses-là seront restituées, car elles appartiennent aux prêtres qui font le service.
\par 43 Et quiconque fuit vers le temple de Jérusalem, ou jouit de la liberté de celui-ci, étant redevable au roi, ou pour toute autre raison, qu'il soit libre, ainsi que tout ce qu'il possède dans mon royaume.
\par 44 Les dépenses pour la construction et la réparation des travaux du sanctuaire seront également inscrites dans les comptes du roi.
\par 45 Oui, et pour la construction des murs de Jérusalem et ses fortifications tout autour, les dépenses seront imputées sur les comptes du roi, comme aussi pour la construction des murs en Judée.
\par 46 Or, lorsque Jonathan et le peuple entendirent ces paroles, ils ne leur accordèrent aucun crédit et ne les reçurent pas, parce qu'ils se souvenaient du grand mal qu'il avait fait en Israël ; car il les avait affligés très douloureusement.
\par 47 Mais ils étaient très satisfaits d'Alexandre, parce qu'il était le premier à implorer avec eux une vraie paix, et ils étaient toujours de mèche avec lui.
\par 48 Alors le roi Alexandre rassembla de grandes forces et campa contre Démétrius.
\par 49 Et après que les deux rois eurent engagé le combat, l'armée de Démétrius s'enfuit ; mais Alexandre le suivit et l'emporta contre eux.
\par 50 Et il continua le combat avec acharnement jusqu'au coucher du soleil ; et ce jour-là Démétrius fut tué.
\par 51 Ensuite Alexandre envoya des ambassadeurs à Ptolémée, roi d'Egypte, avec un message à cet effet :
\par 52 Puisque je suis revenu dans mon royaume, que je suis assis sur le trône de mes ancêtres, que j'ai acquis la domination, que j'ai renversé Démétrius et que j'ai récupéré notre pays ;
\par 53 Car après que j'ai engagé le combat contre lui, lui et son armée ont été déconcertés par nous, de sorte que nous sommes assis sur le trône de son royaume.
\par 54 Maintenant donc, faisons ensemble une alliance d'amitié, et donnons-moi maintenant ta fille pour femme ; et je serai ton gendre, et je te donnerai, toi et elle, selon ta dignité.
\par 55 Alors le roi Ptolémée répondit, disant : Heureux soit le jour où tu seras retourné au pays de tes pères et où tu seras assis sur le trône de leur royaume.
\par 56 Et maintenant je te ferai comme tu l'as écrit : retrouve-moi donc à Ptolémaïs, afin que nous nous voyions ; car je te marierai ma fille selon ton désir.
\par 57 Ptolémée sortit donc d'Égypte avec sa fille Cléopâtre, et ils arrivèrent à Ptolémaïs en la cent soixante-deuxième année.
\par 58 Là où le roi Alexandre le rencontra, il lui donna sa fille Cléopâtre, et célébra son mariage à Ptolémaïs avec une grande gloire, comme c'est la coutume des rois.
\par 59 Le roi Alexandre avait écrit à Jonathan pour qu'il vienne à sa rencontre.
\par 60 Lequel se rendit alors honorablement à Ptolémaïs, où il rencontra les deux rois, et leur donna, à eux et à leurs amis, de l'argent et de l'or, ainsi que de nombreux présents, et trouva grâce à leurs yeux.
\par 61 En ce temps-là, certains hommes pestilentiels d'Israël, hommes d'une vie méchante, se rassemblèrent contre lui pour l'accuser; mais le roi ne voulut pas les écouter.
\par 62 Bien plus, le roi ordonna d'ôter ses vêtements et de le vêtir de pourpre : et ils le firent.
\par 63 Et il le fit asseoir seul, et dit à ses princes : Allez avec lui au milieu de la ville, et proclamez que personne ne se plaindra contre lui de quelque manière que ce soit, et que personne ne le dérangera pour quelque cause que ce soit.
\par 64 Or, quand ses accusateurs virent qu'il était honoré selon la proclamation et vêtu de pourpre, ils s'enfuirent tous.
\par 65 Le roi l'honora donc, et l'inscrivit parmi ses principaux amis, et le fit duc et participant de sa domination.
\par 66 Ensuite Jonathan revint à Jérusalem avec paix et joie.
\par 67 De plus dans le; Cent soixante-cinquième année, Démétrius, fils de Démétrius, sortit de Crète dans le pays de ses pères.
\par 68 C'est pourquoi, lorsque le roi Alexandre entendit parler, il fut vraiment désolé et retourna à Antioche.
\par 69 Alors Démétrius fit d'Apollonios, gouverneur de Célosyrie, son général. Il rassembla une grande armée et campa à Jamnia, et il envoya dire à Jonathan, le grand prêtre :
\par 70 C'est toi seul qui te dresses contre nous, et à cause de toi je suis ridiculisé et insulté. Et pourquoi vantes-tu ta puissance contre nous dans les montagnes ?
\par 71 Maintenant donc, si tu as confiance en ta propre force, descends vers nous dans la plaine, et là essayons ensemble l'affaire; car avec moi est la puissance des villes.
\par 72 Demande et apprends qui je suis, et ceux qui prennent notre parti, et ils te diront que ton pied n'est pas capable de fuir dans leur propre pays.
\par 73 C'est pourquoi tu ne pourras plus supporter les cavaliers et une si grande puissance dans la plaine, où il n'y a ni pierre ni silex, ni lieu où fuir.
\par 74 Ainsi, lorsque Jonathan entendit ces paroles d'Apollonios, il fut ému dans son esprit, et choisissant dix mille hommes, il sortit de Jérusalem, où Simon son frère le rencontra pour l'aider.
\par 75 Et il dressa ses tentes contre Joppé : mais ; ceux de Joppé l'enfermèrent hors de la ville, parce qu'Apollonios y avait une garnison.
\par 76 Alors Jonathan l'assiégea ; sur quoi les habitants de la ville le laissèrent entrer par crainte ; et ainsi Jonathan gagna Joppé.
\par 77 Quand Apollonius l'apprit, il prit trois mille cavaliers et une grande armée de fantassins, et se rendit à Azotus comme un voyageur, et l'entraîna alors dans la plaine parce qu'il avait un grand nombre de cavaliers en qui il mettait sa confiance. .
\par 78 Alors Jonathan le suivit jusqu'à Azotus, où les armées entrèrent en bataille.
\par 79 Or Apollonius avait laissé en embuscade mille cavaliers.
\par 80 Et Jonathan savait qu'il y avait une embuscade derrière lui ; car ils avaient encerclé son armée et lancé des traits sur le peuple, du matin au soir.
\par 81 Mais le peuple s'arrêta, comme Jonathan le lui avait ordonné, et les chevaux des ennemis étaient fatigués.
\par 82 Alors Simon fit sortir son armée, et la dressa contre les fantassins (car les cavaliers étaient épuisés) qui furent déconcertés par lui et s'enfuirent.
\par 83 Les cavaliers, eux aussi, dispersés dans les champs, s'enfuirent à Azot et se rendirent à Bethdagon, le temple de leur idole, pour se mettre en sécurité.
\par 84 Mais Jonathan mit le feu à Azotus et aux villes alentour, et s'empara de leur butin ; et il brûla au feu le temple de Dagon et ceux qui s'y étaient réfugiés.
\par 85 Ainsi furent brûlés et tués par l'épée près de huit mille hommes.
\par 86 Et de là Jonathan quitta son armée et campa contre Ascalon, d'où les hommes de la ville sortirent et le rencontrèrent en grande pompe.
\par 87 Après cela, Jonathan et son armée retournèrent à Jérusalem, avec du butin.
\par 88 Or, lorsque le roi Alexandre entendit ces choses, il honora encore davantage Jonathan.
\par 89 Et il lui envoya une boucle d'or, destinée à être utilisée par ceux qui sont du sang du roi ; il lui donna aussi Accaron, avec ses bords, en possession.

\chapter{11}

\par 1 Et le roi d'Égypte rassembla une armée nombreuse, comme le sable qui repose sur le bord de la mer, et de nombreux navires, et parcourut par tromperie pour s'emparer du royaume d'Alexandre et le joindre au sien.
\par 2 Alors il entreprit paisiblement son voyage en Espagne, tandis que les habitants des villes s'ouvraient à lui et le rencontraient ; car le roi Alexandre le leur avait ordonné de faire ainsi, parce qu'il était son beau-frère.
\par 3 Alors que Ptolémée entrait dans les villes, il établit dans chacune d'elles une garnison de soldats pour la garder.
\par 4 Et lorsqu'il approcha d'Azotus, ils lui montrèrent le temple de Dagon qui avait été incendié, et Azotus et ses faubourgs qui avaient été détruits, et les corps qui avaient été jetés à l'étranger et ceux qu'il avait brûlés dans la bataille ; car ils en avaient fait des tas sur le chemin où il devait passer.
\par 5 Ils racontèrent également au roi tout ce que Jonathan avait fait, afin qu'il puisse le blâmer; mais le roi se tut.
\par 6 Alors Jonathan rencontra le roi en grande pompe à Joppé, où ils se saluèrent et passèrent la nuit.
\par 7 Ensuite Jonathan, après avoir suivi le roi jusqu'au fleuve appelé Éleuthéros, revint à Jérusalem.
\par 8 Le roi Ptolémée, ayant acquis la domination des villes au bord de la mer jusqu'à Séleucie sur la côte de la mer, imagina de mauvais desseins contre Alexandre.
\par 9 Sur quoi il envoya des ambassadeurs auprès du roi Démétrius, pour lui dire : Viens, faisons une alliance entre nous, et je te donnerai ma fille qu'Alexandre a, et tu régneras dans le royaume de ton père.
\par 10 Car je me repens de lui avoir donné ma fille, car il cherchait à me tuer.
\par 11 Il le calomniait ainsi, parce qu'il désirait son royaume.
\par 12 C'est pourquoi il lui prit sa fille, la donna à Démétrius, et abandonna Alexandre, de sorte que leur haine était ouvertement connue.
\par 13 Alors Ptolémée entra à Antioche, où il mit sur sa tête deux couronnes, la couronne d'Asie et celle d'Egypte.
\par 14 Entre-temps, le roi Alexandre était en Cilicie, parce que les habitants de ces régions s'étaient révoltés contre lui.
\par 15 Mais Alexandre, ayant appris cela, partit en guerre contre lui. Sur quoi le roi Ptolémée fit sortir son armée, le rencontra avec une puissance puissante et le mit en fuite.
\par 16 Alors Alexandre s'enfuit en Arabie pour y être défendu ; mais le roi Ptolémée fut exalté :
\par 17 Car Zabdiel l'Arabe ôta la tête d'Alexandre et l'envoya à Ptolémée.
\par 18 Le roi Ptolémée mourut également le troisième jour après, et ceux qui étaient dans les forteresses furent tués les uns les autres.
\par 19 C'est ainsi que Démétrius régna la cent soixante-septième année.
\par 20 Au même moment, Jonathan rassembla ceux qui étaient en Judée pour s'emparer de la tour qui était à Jérusalem ; et il fit contre elle de nombreuses machines de guerre.
\par 21 Alors des impies, qui haïssaient leur propre peuple, vinrent trouver le roi et lui dirent que Jonathan assiégeait la tour.
\par 22 Ayant entendu cela, il se mit en colère, et partant aussitôt, il vint à Ptolémaïs et écrivit à Jonathan qu'il ne devait pas assiéger la tour, mais venir lui parler en toute hâte à Ptolémaïs.
\par 23 Néanmoins Jonathan, ayant entendu cela, ordonna de l'assiéger encore ; et il choisit certains des anciens d'Israël et des prêtres, et se mit en péril ;
\par 24 Et il prit de l'argent et de l'or, des vêtements et divers présents en outre, et il se rendit à Ptolémaïs chez le roi, où il trouva grâce à ses yeux.
\par 25 Et bien que certains hommes impies du peuple eussent porté plainte contre lui,
\par 26 Mais le roi le supplia comme ses prédécesseurs l'avaient fait auparavant, et le promouva aux yeux de tous ses amis,
\par 27 Et il le confirma dans le grand sacerdoce et dans tous les honneurs qu'il avait auparavant, et lui donna la prééminence parmi ses principaux amis.
\par 28 Alors Jonathan pria le roi de libérer la Judée de tout tribut, ainsi que les trois gouvernements, avec le pays de Samarie ; et il lui promit trois cents talents.
\par 29 Le roi y consentit, et écrivit des lettres à Jonathan concernant toutes ces choses de la manière suivante :
\par 30 Le roi Démétrius salue son frère Jonathan et la nation des Juifs :
\par 31 Nous vous envoyons ici une copie de la lettre que nous avons écrite à votre sujet à notre cousin Lasthenes, afin que vous puissiez la voir.
\par 32 Le roi Démétrius salue son père Lasthène :
\par 33 Nous sommes déterminés à faire du bien au peuple juif, qui est nos amis, et à respecter nos alliances avec nous, à cause de sa bonne volonté à notre égard.
\par 34 C'est pourquoi nous leur avons ratifié les frontières de la Judée, avec les trois gouvernements d'Apherema, Lydda et Ramathem, qui ont été ajoutés à la Judée à partir du pays de Samarie, et tout ce qui leur appartient, pour tous ceux qui font des sacrifices en Jérusalem, au lieu des paiements que le roi recevait d'eux chaque année auparavant, sur les fruits de la terre et des arbres.
\par 35 Et quant aux autres choses qui nous appartiennent, les dîmes et les douanes qui nous appartiennent, ainsi que les salines et les impôts de la couronne, qui nous sont dus, nous les déchargeons de tout cela pour leur soulagement.
\par 36 Et rien de ce qui précède ne sera révoqué à partir de maintenant pour toujours.
\par 37 Maintenant, veille à faire une copie de ces choses, et qu'elle soit remise à Jonathan, et placée sur la montagne sainte, dans un lieu bien en vue.
\par 38 Après cela, lorsque le roi Démétrius vit que le pays était tranquille devant lui et qu'il n'y avait aucune résistance contre lui, il renvoya toutes ses forces, chacun chez lui, à l'exception de certaines bandes d'étrangers qu'il avait rassemblés des îles des païens : c'est pourquoi toutes les forces de ses pères le haïssaient.
\par 39 Il y avait aussi un certain Tryphon, qui avait déjà été du côté d'Alexandre, et qui, voyant que toute l'armée murmurait contre Démétrius, se rendit chez Simalcue, l'Arabe, qui avait élevé Antiochus, le jeune fils d'Alexandre.
\par 40 And lay sore upon him to deliver him this young Antiochus, that he might reign in his father's stead: he told him therefore all that Demetrius had done, and how his men of war were at enmity with him, and there he remained a long season.
\par 41 Pendant ce temps, Jonathan envoya au roi Démétrius pour qu'il chasse ceux de la tour hors de Jérusalem, ainsi que ceux des forteresses, car ils combattaient contre Israël.
\par 42 Alors Démétrius envoya dire à Jonathan : Non seulement je ferai cela pour toi et ton peuple, mais je t'honorerai grandement, toi et ta nation, si l'occasion s'en présente.
\par 43 Maintenant donc tu feras bien si tu m'envoies des hommes pour m'aider ; car toutes mes forces sont parties de moi.
\par 44 Sur ce, Jonathan envoya trois mille hommes forts à Antioche ; et lorsqu'ils arrivèrent auprès du roi, le roi fut très heureux de leur venue.
\par 45 Mais ceux qui étaient de la ville se rassemblèrent au milieu de la ville, au nombre de cent vingt mille hommes, et voulurent tuer le roi.
\par 46 C'est pourquoi le roi s'enfuit dans la cour, mais les habitants de la ville gardèrent les passages de la ville et commencèrent à se battre.
\par 47 Alors le roi appela les Juifs à l'aide, qui vinrent à lui tout à coup, et se dispersant dans la ville, ils tuèrent ce jour-là dans la ville cent mille personnes.
\par 48 Ils mirent également le feu à la ville, et emportèrent ce jour-là beaucoup de butin, et délivrèrent le roi.
\par 49 Ainsi, quand les habitants de la ville virent que les Juifs avaient pris la ville comme ils l'avaient voulu, leur courage fut diminué. C'est pourquoi ils supplièrent le roi et crièrent, disant :
\par 50 Accordez-nous la paix et que les Juifs cessent de nous attaquer ainsi que la ville.
\par 51 Sur ce, ils jetèrent leurs armes et firent la paix ; et les Juifs furent honorés aux yeux du roi et aux yeux de tous ceux qui étaient dans son royaume ; et ils retournèrent à Jérusalem, avec un grand butin.
\par 52 Le roi Démétrius était donc assis sur le trône de son royaume, et le pays était tranquille devant lui.
\par 53 Néanmoins il dissimula tout ce qu'il disait, et s'éloigna de Jonathan, et ne le récompensa pas selon les bienfaits qu'il avait reçus de lui, mais le troubla beaucoup.
\par 54 Après cela Tryphon revint, et avec lui le jeune enfant Antiochus, qui régnait et fut couronné.
\par 55 Alors se rassemblèrent autour de lui tous les hommes de guerre que Démétrius avait renvoyés, et ils combattirent contre Démétrius, qui lui tourna le dos et s'enfuit.
\par 56 De plus, Tryphon prit les éléphants et conquit Antioche.
\par 57 En ce temps-là, le jeune Antiochus écrivit à Jonathan, disant : Je te confirme dans le grand sacerdoce, et je t'établis comme chef des quatre gouvernements, et pour être l'un des amis du roi.
\par 58 Là-dessus, il lui envoya des vases d'or pour qu'il y soit servi, et lui donna la permission de boire de l'or, de se vêtir de pourpre et de porter une boucle d'or.
\par 59 Il établit également son frère Simon capitaine depuis le lieu appelé l'échelle de Tyrus jusqu'aux frontières de l'Égypte.
\par 60 Alors Jonathan sortit et traversa les villes au-delà de l'eau, et toutes les forces de la Syrie se rassemblèrent autour de lui pour l'aider. Et quand il arriva à Ascalon, les habitants de la ville le rencontrèrent honorablement.
\par 61 D'où il est parti pour Gaza, mais ceux de Gaza l'ont exclu ; c'est pourquoi il l'assiégea, et brûla au feu ses faubourgs, et les ravagea.
\par 62 Ensuite, lorsque les habitants de Gaza firent des supplications à Jonathan, celui-ci fit la paix avec eux, et prit en otages les fils de leurs chefs, et les envoya à Jérusalem, et traversa le pays jusqu'à Damas.
\par 63 Or, lorsque Jonathan apprit que les princes de Démétrius étaient venus à Cadès, qui est en Galilée, avec une grande puissance, dans le but de l'expulser du pays,
\par 64 Il alla à leur rencontre et laissa Simon son frère à la campagne.
\par 65 Alors Simon campa contre Bethsura et combattit contre elle pendant une longue saison, et la ferma.
\par 66 Mais ils désirèrent avoir la paix avec lui, ce qu'il leur accorda, puis les chassa de là, s'emparèrent de la ville et y établirent une garnison.
\par 67 Quant à Jonathan et son armée, ils campèrent près des eaux de Gennésar, d'où ils les conduisirent de bonne heure le matin dans la plaine de Nasor.
\par 68 Et voici, une armée d'étrangers les rencontra dans la plaine, et, lui ayant tendu des embuscades dans les montagnes, ils vinrent à sa rencontre.
\par 69 Alors ceux qui étaient en embuscade quittèrent leurs places et s'engagèrent dans le combat, tous ceux qui étaient du côté de Jonathan s'enfuirent ;
\par 70 De sorte qu'il n'en restait plus un seul, à l'exception de Mattathias, fils d'Absalom, et de Judas, fils de Calphi, les chefs de l'armée.
\par 71 Alors Jonathan déchira ses vêtements, jeta de la terre sur sa tête et pria.
\par 72 Puis il se remit au combat, et il les mit en fuite, et ils s'enfuirent.
\par 73 Les siens qui étaient en fuite, voyant cela, se tournèrent de nouveau vers lui et les poursuivirent avec lui jusqu'à Cadès jusqu'à leurs tentes, et là ils campèrent.
\par 74 Ce jour-là, environ trois mille hommes furent tués parmi les païens ; mais Jonathan revint à Jérusalem.

\chapter{12}

\par 1 Or, lorsque Jonathan vit que le temps lui était favorable, il choisit certains hommes et les envoya à Rome, pour confirmer et renouveler l'amitié qu'ils avaient avec eux.
\par 2 Il envoya aussi des lettres aux Lacédémoniens et en d'autres lieux, dans le même but.
\par 3 Ils allèrent donc à Rome, et entrèrent dans le sénat, et dirent : Jonathan, le souverain sacrificateur, et le peuple des Juifs nous ont envoyés vers vous, afin que vous renouveliez l'amitié que vous aviez avec eux, et ligue, comme autrefois.
\par 4 Sur ce, les Romains leur donnèrent des lettres aux gouverneurs de chaque lieu pour qu'ils les amènent pacifiquement dans le pays de Judée.
\par 5 Et voici la copie des lettres que Jonathan écrivit aux Lacédémoniens :
\par 6 Jonathan, le grand prêtre, et les anciens de la nation, et les prêtres, et les autres Juifs, et leurs frères Lacédémoniens, envoient leurs salutations :
\par 7 Des lettres ont été envoyées autrefois à Onias, le grand prêtre, de la part de Darius, qui régnait alors parmi vous, pour signifier que vous êtes nos frères, comme le précise la copie ici souscrite.
\par 8 A ce moment-là, Onias supplia honorablement l'ambassadeur qui avait été envoyé, et reçut les lettres dans lesquelles il était fait déclaration d'alliance et d'amitié.
\par 9 C'est pourquoi nous aussi, bien que nous n'ayons besoin de rien de tout cela, nous avons entre nos mains les livres saints de l'Écriture pour nous consoler,
\par 10 J'ai néanmoins essayé de vous envoyer pour renouveler la fraternité et l'amitié, de peur que nous ne vous devenions complètement étrangers : car il y a longtemps que vous nous avez envoyé.
\par 11 Nous donc, à tout moment et sans cesse, tant dans nos fêtes que dans d'autres jours opportuns, nous souvenons de vous dans les sacrifices que nous offrons et dans nos prières, selon la raison et comme il nous convient de penser à nos frères. :
\par 12 Et nous sommes très heureux de votre honneur.
\par 13 Quant à nous, nous avons eu de grandes difficultés et des guerres de toutes parts, à cause que les rois qui nous entourent ont combattu contre nous.
\par 14 Cependant, nous ne voudrions pas vous gêner, ni à nos autres alliés et amis, dans ces guerres :
\par 15 Car nous avons l'aide du ciel qui nous secourt, de même que nous sommes délivrés de nos ennemis, et que nos ennemis sont mis sous les pieds.
\par 16 C'est pour cette raison que nous avons choisi Numénius, fils d'Antiochus, et Antipater, fils de Jason, et nous les avons envoyés vers les Romains, pour renouveler l'amitié que nous avions avec eux et l'ancienne ligue.
\par 17 Nous leur avons commandé aussi d'aller vers vous, de vous saluer et de vous remettre nos lettres concernant le renouvellement de notre fraternité.
\par 18 C'est pourquoi vous feriez bien maintenant de nous donner une réponse à ce sujet.
\par 19 Et voici la copie des lettres envoyées par Oniares.
\par 20 Areus, roi des Lacédémoniens, au grand prêtre Onias, salue :
\par 21 Il est écrit que les Lacédémoniens et les Juifs sont frères, et qu'ils sont de la souche d'Abraham :
\par 22 Maintenant donc, puisque cela est parvenu à notre connaissance, vous feriez bien de nous écrire sur votre prospérité.
\par 23 Nous vous écrivons encore que votre bétail et vos biens sont à nous, et que les nôtres sont à vous. Nous ordonnons donc à nos ambassadeurs de vous faire rapport à ce sujet.
\par 24 Or, lorsque Jonathan apprit que les princes de Démébius étaient venus pour lui faire la guerre avec une armée plus nombreuse qu'auparavant,
\par 25 Il quitta Jérusalem et les rencontra au pays d'Amathis, car il ne leur laissa aucun répit pour entrer dans son pays.
\par 26 Il envoya aussi des espions dans leurs tentes, qui revinrent et lui dirent qu'ils avaient été désignés pour les apercevoir pendant la nuit.
\par 27 C'est pourquoi dès que le soleil fut couché, Jonathan ordonna à ses hommes de veiller et d'être en armes, afin d'être prêts à combattre toute la nuit ; et il envoya aussi des centinelles autour de l'armée.
\par 28 Mais lorsque les adversaires apprirent que Jonathan et ses hommes étaient prêts au combat, ils furent saisis de crainte et frémirent dans leur cœur, et ils allumèrent des feux dans leur camp.
\par 29 Mais Jonathan et ses compagnons ne s'en aperçurent que le matin, car ils virent les lumières allumées.
\par 30 Alors Jonathan les poursuivit, mais ne les atteignit pas, car ils avaient traversé le fleuve Éleuthère.
\par 31 C'est pourquoi Jonathan se tourna vers les Arabes, appelés Zabadéens, et les frappa et s'empara de leur butin.
\par 32 Et partant de là, il arriva à Damas, et parcourut ainsi tout le pays,
\par 33 Simon partit aussi et traversa le pays jusqu'à Ascalon et les forteresses qui y sont voisines, d'où il se détourna vers Joppé et conquit la conquête.
\par 34 Car il avait entendu dire qu'ils livreraient la forteresse à ceux qui prenaient le parti de Démétrius ; c'est pourquoi il y établit une garnison pour la garder.
\par 35 Après cela, Jonathan revint chez lui, et, convoquant les anciens du peuple, il les consulta au sujet de la construction de forteresses en Judée.
\par 36 Et il élevait les murs de Jérusalem, et élevait une grande montagne entre la tour et la ville, pour la séparer de la ville, afin qu'elle soit seule, et qu'on ne puisse y vendre ni acheter.
\par 37 Là-dessus, ils se rassemblèrent pour reconstruire la ville, car une partie du mur du côté du ruisseau, du côté de l'orient, était tombée, et ils réparèrent ce qu'on appelait Caphenatha.
\par 38 Simon établit aussi Adida à Séphéla, et la fortifia avec des portes et des barres.
\par 39 Tryphon partit pour s'emparer du royaume d'Asie et pour tuer le roi Antiochus, afin qu'il puisse mettre la couronne sur sa tête.
\par 40 Mais il craignait que Jonathan ne le souffrît pas et qu'il ne combatte contre lui ; c'est pourquoi il cherchait un moyen de prendre Jonathan, afin qu'il le tue. Il partit donc et vint à Bethsan.
\par 41 Alors Jonathan sortit à sa rencontre avec quarante mille hommes choisis pour le combat, et arriva à Bethsan.
\par 42 Or, lorsque Tryphon vit que Jonathan arrivait avec une si grande force, il n'osa pas tendre la main contre lui ;
\par 43 Mais il le reçut honorablement, le recommanda à tous ses amis, lui fit des présents et ordonna à ses hommes de guerre de lui être aussi obéissants qu'à lui-même.
\par 44 Il dit aussi à Jonathan : Pourquoi as-tu causé de si grands ennuis à tout ce peuple, puisqu'il n'y a pas de guerre entre nous ?
\par 45 Renvoie-les donc maintenant chez eux, et choisis quelques hommes pour te servir, et viens avec moi à Ptolémaïs, car je te le donnerai, ainsi que le reste des forteresses et des forces, et tous ceux qui ont quelque charge, comme pour moi, je reviendrai et je m'en irai : car c'est la cause de ma venue.
\par 46 Alors Jonathan, le croyant, fit ce qu'il lui avait ordonné, et renvoya son armée, qui partit pour le pays de Judée.
\par 47 Et il ne garda avec lui que trois mille hommes, dont il envoya deux mille en Galilée, et mille partit avec lui.
\par 48 Dès que Jonathan entra dans Ptolémaïs, les habitants de Ptolémaïs fermèrent les portes et le prirent, et tous ceux qui étaient venus avec lui furent tués par l'épée.
\par 49 Alors Tryphon envoya une armée de fantassins et de cavaliers en Galilée et dans la grande plaine, pour détruire toute la troupe de Jonathan.
\par 50 Mais lorsqu'ils savaient que Jonathan et ceux qui étaient avec lui avaient été pris et tués, ils se sont encouragés les uns les autres ; et se rapprochèrent, prêts à se battre.
\par 51 Ceux qui les suivaient donc, s'apercevant qu'ils étaient prêts à se battre pour leur vie, rebroussèrent chemin.
\par 52 Alors ils arrivèrent tous en paix au pays de Judée, et là ils pleurèrent Jonathan et ceux qui étaient avec lui, et ils eurent très peur ; c'est pourquoi tout Israël poussa de grandes lamentations.
\par 53 Alors tous les païens qui étaient aux alentours cherchèrent alors à les détruire ; car ils disaient : Ils n'ont ni capitaine, ni personne pour les aider ; maintenant donc, faisons-leur la guerre, et ôtons du milieu des hommes leur mémorial.

\chapter{13}

\par 1 Or, lorsque Simon apprit que Tryphon avait rassemblé une grande armée pour envahir le pays de Judée et le détruire,
\par 2 Et voyant que le peuple était très tremblant et effrayé, il monta à Jérusalem et rassembla le peuple,
\par 3 Et il leur donna une exhortation, disant : Vous savez vous-mêmes quelles grandes choses moi, mes frères et la maison de mon père avons faites pour les lois et le sanctuaire, ainsi que les batailles et les troubles que nous avons vus.
\par 4 C'est pourquoi tous mes frères sont tués à cause d'Israël, et je reste seul.
\par 5 Maintenant donc, qu'il ne me soit pas permis d'épargner ma propre vie dans un moment de détresse, car je ne vaux pas mieux que mes frères.
\par 6 Sans aucun doute, je vengerai ma nation, et le sanctuaire, et nos femmes et nos enfants : car tous les païens sont rassemblés pour nous détruire avec une grande méchanceté.
\par 7 Dès que le peuple entendit ces paroles, son esprit revint.
\par 8 Et ils répondirent à haute voix, disant : Tu seras notre chef à la place de Judas et de Jonathan, ton frère.
\par 9 Combattez nos batailles, et tout ce que vous nous commandez, nous le ferons.
\par 10 Alors il rassembla tous les hommes de guerre, et se hâta d'achever les murs de Jérusalem, et il la fortifia tout autour.
\par 11 Il envoya aussi Jonathan, fils d'Absolom, et avec lui une grande puissance, à Joppé, qui chassant ceux qui s'y trouvaient, y resta.
\par 12 Ainsi Tryphon quitta Ptolémée avec une grande puissance pour envahir le pays de Judée, et Jonathan était avec lui en garde.
\par 13 Mais Simon dressa ses tentes à Adida, du côté de la plaine.
\par 14 Or, lorsque Tryphon apprit que Simon s'était levé à la place de son frère Jonathan, et qu'il avait l'intention de lui livrer bataille, il lui envoya des messagers pour lui dire :
\par 15 Tandis que nous tenons Jonathan, ton frère, en notre possession, c'est pour de l'argent qu'il doit au trésor du roi, au sujet de l'affaire qui lui a été confiée.
\par 16 C'est pourquoi maintenant, envoie cent talents d'argent et deux de ses fils en otages, afin qu'il ne se révolte pas contre nous lorsqu'il sera en liberté, et nous le laisserons partir.
\par 17 Alors Simon, bien qu'il s'aperçut qu'ils lui parlaient de manière trompeuse, envoya néanmoins l'argent et les enfants, de peur qu'il ne se suscite une grande haine envers le peuple.
\par 18 Qui aurait pu dire : Parce que je ne lui ai pas envoyé l'argent et les enfants, Jonathan est donc mort.
\par 19 Il leur envoya donc les enfants et les cent talents. Cependant Tryphon se dissimula et ne laissa pas partir Jonathan.
\par 20 Après cela, Tryphon vint envahir le pays et le détruire, en contournant le chemin qui mène à Adora. Mais Simon et son armée marchèrent contre lui partout où il allait.
\par 21 Ceux qui étaient dans la tour envoyèrent des messagers à Tryphon, afin qu'il hâte sa venue vers eux par le désert et leur envoyât des vivres.
\par 22 C'est pourquoi Tryphon prépara tous ses cavaliers à venir cette nuit-là ; mais il tomba une très grande neige, à cause de laquelle il ne vint pas. Il partit donc et vint au pays de Galaad.
\par 23 Et lorsqu'il approcha de Bascama, il tua Jonathan, qui y était enterré.
\par 24 Ensuite Tryphon revint et s'en alla dans son pays.
\par 25 Alors Simon envoya prendre les ossements de Jonathan, son frère, et les enterra à Modin, la ville de ses pères.
\par 26 Et tout Israël fit de grandes lamentations sur lui et le pleura plusieurs jours.
\par 27 Simon bâtit aussi un monument sur le sépulcre de son père et de ses frères, et il l'éleva bien haut, avec des pierres de taille derrière et devant.
\par 28 Et il dressa sept pyramides les unes contre les autres, pour son père, et sa mère, et ses quatre frères.
\par 29 Et dans ceux-ci, il fabriqua des dispositifs astucieux, autour desquels il plaça de grands piliers, et sur les piliers il fit toutes leurs armures pour un souvenir perpétuel, et par les cuirassés il sculpta des navires, afin qu'ils puissent être vus de tous ceux qui naviguent sur la mer.
\par 30 C'est ici le sépulcre qu'il a fait à Modin, et il existe encore aujourd'hui.
\par 31 Or Tryphon trahit le jeune roi Antiochus et le tua.
\par 32 Et il régna à sa place, et se couronna roi d'Asie, et il fit venir un grand malheur sur le pays.
\par 33 Alors Simon bâtit les forteresses en Judée, et les entoura de hautes tours, de grandes murailles, de portes et de barres, et il y déposa des vivres.
\par 34 Simon choisit des hommes et les envoya au roi Démétrius, afin qu'il accorde l'immunité au pays, car tout ce que faisait Tryphon était de le piller.
\par 35 À quoi le roi Démétrius répondit et écrivit de cette manière :
\par 36 Le roi Démétrius envoie son salut à Simon, le grand prêtre et ami des rois, ainsi qu'aux anciens et à la nation des Juifs :
\par 37 La couronne d'or et la robe écarlate que vous nous avez envoyées, nous les avons reçues ; et nous sommes prêts à conclure une paix ferme avec vous, oui, et à écrire à nos officiers pour confirmer les immunités que nous avons accordées.
\par 38 Et toutes les alliances que nous avons conclues avec vous resteront valables ; et les forteresses que vous avez bâties vous appartiendront.
\par 39 Quant à tout oubli ou faute commis jusqu'à ce jour, nous le pardonnons, ainsi que l'impôt de la couronne que vous nous devez ; et s'il y avait un autre tribut payé à Jérusalem, il ne sera plus payé.
\par 40 Et voyez qui d'entre vous est digne d'être dans notre parvis, qu'ils soient alors inscrits, et qu'il y ait la paix entre nous.
\par 41 Ainsi, le joug des païens fut ôté d'Israël la cent soixante-dixième année.
\par 42 Alors les enfants d'Israël commencèrent à écrire dans leurs instruments et leurs contrats : La première année de Simon, le grand prêtre, gouverneur et chef des Juifs.
\par 43 En ce temps-là, Simon campait contre Gaza et l'assiégeait tout autour ; il fabriqua aussi une machine de guerre, la plaça près de la ville, battit une certaine tour et la prit.
\par 44 Et ceux qui étaient dans la machine sautèrent dans la ville ; sur quoi il y eut un grand tumulte dans la ville :
\par 45 De sorte que les habitants de la ville déchiraient leurs vêtements, grimpaient sur les murs avec leurs femmes et leurs enfants, et criaient à haute voix, suppliant Simon de leur accorder la paix.
\par 46 Et ils dirent : Ne nous traite pas selon notre méchanceté, mais selon ta miséricorde.
\par 47 Alors Simon fut apaisé envers eux, et ne combattit plus contre eux, mais il les chassa de la ville, et nettoya les maisons où étaient les idoles, et y entra avec des chants et des actions de grâces.
\par 48 Oui, il en élimina toute impureté, et y plaça des hommes capables d'observer la loi, et la rendit plus forte qu'elle n'était auparavant, et s'y bâtit une demeure pour lui-même.
\par 49 Ceux aussi de la tour de Jérusalem étaient tenus si à l'étroit qu'ils ne pouvaient ni sortir, ni entrer dans la campagne, ni acheter ni vendre. C'est pourquoi ils étaient dans une grande détresse, faute de vivres, et d'un grand nombre de ils périrent de famine.
\par 50 Alors ils crièrent à Simon, le suppliant d'être un avec eux : ce qu'il leur accorda ; et après les avoir fait sortir de là, il purifia la tour des souillures :
\par 51 Et il y entra le vingt-troisième jour du deuxième mois de la cent soixante et onzième année, avec des actions de grâces, et des branches de palmiers, et avec des harpes, et des cymbales, et avec des violes, et des hymnes et des chants, car il y avait détruit un grand ennemi hors d’Israël.
\par 52 Il a aussi ordonné que ce jour soit célébré chaque année avec joie. Il rendit également la colline du temple près de la tour plus forte qu'elle ne l'était, et il y demeura avec sa troupe.
\par 53 Et Simon, voyant que Jean, son fils, était un homme vaillant, le nomma capitaine de toutes les armées ; et il habita à Gazera.

\chapter{14}

\par 1 Or, en la cent soixante-douzième année, le roi Démétrius rassembla ses forces et se rendit en Médie pour lui demander de l'aide pour lutter contre Tryphone.
\par 2 Mais Arsaces, roi de Perse et de Médie, apprit que Démétrius était entré dans son territoire, et envoya un de ses princes pour le prendre vivant.
\par 3 Qui alla frapper l'armée de Démétrius, le prit et l'amena à Arsaces, par qui il fut mis en garde.
\par 4 Quant au pays de Judée, il fut tranquille tous les jours de Simon ; car il recherchait le bien de sa nation de telle manière que son autorité et son honneur leur plaisaient toujours.
\par 5 Et comme il était honorable dans tous ses actes, de même en ceci qu'il prit Joppé pour refuge et fit une entrée vers les îles de la mer,
\par 6 Et il élargit les limites de sa nation et recouvra le pays,
\par 7 Et il rassembla un grand nombre de captifs, et eut la domination sur Gazera, et Bethsura, et la tour, d'où il sortit toutes les impuretés, et personne ne lui résista.
\par 8 Alors ils labourèrent leur terre en paix, et la terre donna ses produits, et les arbres des champs leurs fruits.
\par 9 Les hommes anciens étaient tous assis dans les rues, communiant ensemble de bonnes choses, et les jeunes hommes portaient des vêtements glorieux et guerriers.
\par 10 Il approvisionna les villes en vivres et y installa toutes sortes de munitions, de sorte que son honorable nom fut renommé jusqu'à la fin du monde.
\par 11 Il fit la paix dans le pays, et Israël se réjouit d'une grande joie :
\par 12 Car chacun était assis sous sa vigne et son figuier, et il n'y avait personne pour les effilocher.
\par 13 Il ne restait plus personne dans le pays pour les combattre ; oui, les rois eux-mêmes furent renversés à cette époque-là.
\par 14 Et il fortifia tous ceux de son peuple qui étaient humiliés ; il rechercha la loi ; et il enleva tout méprisant la loi et tout méchant.
\par 15 Il embellit le sanctuaire et multiplia les ustensiles du temple.
\par 16 Or, quand on apprit à Rome et jusqu'à Sparte que Jonathan était mort, ils en furent très désolés.
\par 17 Mais dès qu'ils apprirent que son frère Simon avait été fait grand prêtre à sa place et qu'il dirigeait le pays et les villes qui y sont :
\par 18 Ils lui écrivirent sur des tables d'airain, pour renouveler l'amitié et l'alliance qu'ils avaient nouées avec Judas et Jonathan, ses frères :
\par 19 Quels écrits furent lus devant l'assemblée de Jérusalem.
\par 20 Et voici la copie des lettres envoyées par les Lacédémoniens ; Les chefs des Lacédémoniens et la ville, à Simon le grand prêtre, aux anciens, aux prêtres et au reste du peuple juif, nos frères, saluent :
\par 21 Les ambassadeurs qui ont été envoyés auprès de notre peuple nous ont certifié votre gloire et votre honneur : c'est pourquoi nous avons été heureux de leur venue,
\par 22 Et ils enregistrèrent de cette manière les choses qu'ils disaient au conseil du peuple ; Numénius, fils d'Antiochus, et Antipater, fils de Jason, ambassadeurs des Juifs, vinrent vers nous pour renouveler l'amitié qu'ils avaient avec nous.
\par 23 Et il a plu au peuple de recevoir les hommes honorablement et de mettre la copie de leur ambassade dans les registres publics, afin que le peuple des Lacédémoniens en ait un mémorial. Nous en avons également écrit une copie à Simon le grand prêtre.
\par 24 Après cela, Simon envoya Numénius à Rome avec un grand bouclier d'or pesant mille livres, pour confirmer l'alliance avec eux.
\par 25 Et quand le peuple l'entendit, ils dirent : Quelles reconnaissances rendrons-nous à Simon et à ses fils ?
\par 26 Car lui, ses frères et la maison de son père ont établi Israël, ils ont chassé d'eux leurs ennemis par la guerre et ils ont confirmé leur liberté.
\par 27 Ils l'écrivirent donc sur des tables d'airain qu'ils placèrent sur des colonnes sur la montagne de Sion. Et voici la copie de l'écriture : Le dix-huitième jour du mois d'Elul, la cent soixante-douzième année, étant la troisième année de Simon, le grand prêtre,
\par 28 A Saramel, dans la grande congrégation des prêtres, du peuple, des chefs de la nation et des anciens du pays, ces choses nous furent notifiées.
\par 29 Comme il y a souvent eu des guerres dans le pays, dans lesquelles, pour le maintien de leur sanctuaire et de la loi, Simon, fils de Mattathias, de la postérité de Jarib, et ses frères, se sont mis en péril et ont résisté aux ennemis de leur nation a fait à sa nation un grand honneur :
\par 30 (Car après cela Jonathan, ayant rassemblé sa nation et étant leur souverain sacrificateur, fut ajouté à son peuple,
\par 31 Leurs ennemis se préparaient à envahir leur pays, pour le détruire, et s'emparer du sanctuaire :
\par 32 A cette époque-là, Simon se leva, combattit pour sa nation, dépensa une grande partie de ses biens, arma les vaillants hommes de sa nation et leur donna un salaire.
\par 33 Et il fortifia les villes de Judée, ainsi que Bethsura, qui est située sur les frontières de la Judée, là où se trouvaient auparavant les armes des ennemis ; mais il y installa une garnison de Juifs :
\par 34 Il fortifia Joppé, qui est au bord de la mer, et Gazera, qui borde Azotus, où les ennemis avaient habité auparavant ; mais il y plaça les Juifs et leur fournit tout ce qui était convenable pour leur réparation.)
\par 35 Le peuple chanta donc les actes de Simon, et à quelle gloire il pensait amener sa nation, il le nomma gouverneur et grand prêtre, parce qu'il avait fait toutes ces choses, et à cause de la justice et de la foi qu'il avait gardées envers sa nation, et pour cela il cherchait par tous les moyens à exalter son peuple.
\par 36 Car en son temps les choses prospérèrent entre ses mains, de sorte que les païens furent retirés de leur pays, ainsi que ceux qui étaient dans la ville de David à Jérusalem, qui s'étaient fait une tour d'où ils sortaient, et il souilla tout autour du sanctuaire, et fit beaucoup de mal dans le lieu saint :
\par 37 Mais il y plaça les Juifs, et le fortifia pour assurer la sécurité du pays et de la ville, et il érigea les murs de Jérusalem.
\par 38 Le roi Démétrius l'a aussi confirmé dans le grand sacerdoce selon ces choses,
\par 39 Et il en fit un de ses amis, et il l'honora d'un grand honneur.
\par 40 Car il avait entendu dire que les Romains appelaient les Juifs leurs amis, leurs alliés et leurs frères ; et qu'ils avaient reçu honorablement les ambassadeurs de Simon ;
\par 41 De plus, les Juifs et les prêtres étaient très contents que Simon soit leur gouverneur et leur souverain sacrificateur pour toujours, jusqu'à ce qu'un prophète fidèle apparaisse ;
\par 42 De plus, qu'il serait leur capitaine et qu'il aurait la charge du sanctuaire, qu'il les établirait sur leurs ouvrages, et sur le pays, et sur les armures, et sur les forteresses, que, dis-je, il devrait avoir la charge du sanctuaire. ;
\par 43 De plus, afin qu'il soit obéi de tous, et que toutes les écritures du pays soient faites en son nom, et qu'il soit vêtu de pourpre et portant de l'or :
\par 44 Aussi qu'il ne soit permis à aucun du peuple ni des prêtres de violer aucune de ces choses, ou de contredire ses paroles, ou de rassembler une assemblée dans le pays sans lui, ou de se vêtir de pourpre, ou de porter un boucle d'or;
\par 45 Et quiconque agirait autrement, ou briserait l'une de ces choses, devrait être puni.
\par 46 Ainsi, tout le peuple aimait à agir envers Simon et à faire ce qui avait été dit.
\par 47 Alors Simon accepta cela, et fut bien content d'être grand prêtre, capitaine et gouverneur des Juifs et des prêtres, et de les défendre tous.
\par 48 Ils ordonnèrent donc que cette écriture soit placée sur des tables d'airain, et qu'elles soient placées dans l'enceinte du sanctuaire, dans un endroit bien en vue ;
\par 49 Et que les copies en soient déposées dans le trésor, afin que Simon et ses fils les aient.

\chapter{15}

\par 1 Le roi Antiochus, fils du roi Démétrius, envoya des îles de la mer des lettres à Simon, prêtre et prince des Juifs, et à tout le peuple ;
\par 2 Voici le contenu de celui-ci : Le roi Antiochus à Simon, grand prêtre et prince de sa nation, et au peuple des Juifs, salutation :
\par 3 Puisque certains hommes pestilentiels ont usurpé le royaume de nos pères, et que mon dessein est de le contester à nouveau, afin de le restaurer à l'ancien domaine, et à cette fin, j'ai rassemblé une multitude de soldats étrangers et préparé navires de guerre;
\par 4 Mon intention est aussi de parcourir le pays, afin de me venger de ceux qui l'ont détruit et qui ont dévasté beaucoup de villes du royaume.
\par 5 Maintenant donc je te confirme toutes les offrandes que les rois avant moi t'ont accordées, et tous les présents qu'ils ont accordés en plus.
\par 6 Je t'autorise aussi à frapper de la monnaie pour ton pays avec ton propre timbre.
\par 7 Et quant à Jérusalem et au sanctuaire, qu'ils soient libres ; et toutes les armures que tu as faites, et les forteresses que tu as bâties et que tu gardes entre tes mains, qu'elles te restent.
\par 8 Et si quelque chose est ou doit être dû au roi, qu'il te soit pardonné désormais et pour toujours.
\par 9 De plus, lorsque nous aurons obtenu notre royaume, nous t'honorerons, ainsi que ta nation et ton temple, avec un grand honneur, afin que ton honneur soit connu dans le monde entier.
\par 10 La cent soixante-quatorzième année, Antiochus partit pour le pays de ses pères ; alors toutes les forces se rassemblèrent vers lui, de sorte qu'il ne restait que peu de gens avec Tryphon.
\par 11 C'est pourquoi, poursuivi par le roi Antiochus, il s'enfuit à Dora, qui est située au bord de la mer.
\par 12 Car il voyait que les troubles s'abattaient sur lui tout à coup, et que ses forces l'avaient abandonné.
\par 13 Alors Antiochus campa contre Dora, ayant avec lui cent vingt mille hommes de guerre et huit mille cavaliers.
\par 14 Et après avoir fait le tour de la ville tout autour, et joint les navires près de la ville du côté de la mer, il tourmenta la ville sur terre et sur mer, et ne permit à personne d'en sortir ou d'y entrer.
\par 15 Dans le même temps, Numénius et sa compagnie arrivèrent de Rome, ayant des lettres pour les rois et les pays ; où furent écrites ces choses :
\par 16 Lucius, consul des Romains auprès du roi Ptolémée, salut :
\par 17 Les ambassadeurs des Juifs, nos amis et nos confédérés, sont venus vers nous pour renouveler l'ancienne amitié et l'alliance, envoyés par Simon le grand prêtre et par le peuple des Juifs :
\par 18 Et ils apportèrent un bouclier d'or de mille livres.
\par 19 Nous avons donc pensé qu'il était bon d'écrire aux rois et aux pays, qu'ils ne devaient pas leur faire de mal, ni combattre contre eux, leurs villes ou leurs pays, ni encore aider leurs ennemis contre eux.
\par 20 Il nous a semblé bon aussi de recevoir leur bouclier.
\par 21 S'il y a donc des pestiférés qui ont fui leur pays vers vous, livrez-les à Simon le grand prêtre, afin qu'il les punisse selon leur propre loi.
\par 22 Il écrivit les mêmes choses au roi Démétrius, à Attale, à Ariarathe et à Arsace,
\par 23 Et dans tous les pays, et à Sampsames, et aux Lacédémoniens, et à Délus, et Myndus, et Sicyone, et Carie, et Samos, et Pamphylie, et Lycie, et Halicarnasse, et Rhodus, et Aradus, et Cos, et Side, et Aradus, et Gortyna, et Cnide, et Chypre, et Cyrène.
\par 24 Et ils en écrivirent une copie à Simon, le souverain sacrificateur.
\par 25 Le roi Antiochus campa donc contre Dora le deuxième jour, attaquant continuellement Dora et construisant des machines, ce qui lui permit d'enfermer Tryphon, de sorte qu'il ne pouvait ni sortir ni entrer.
\par 26 En ce temps-là, Simon lui envoya deux mille hommes choisis pour l'aider ; de l'argent aussi, de l'or et de nombreuses armes.
\par 27 Néanmoins il ne voulut pas les recevoir, mais il rompit toutes les alliances qu'il avait faites avec lui auparavant, et lui devint étranger.
\par 28 Et il lui envoya Athénobe, l'un de ses amis, pour communier avec lui et lui dire : Vous refusez Joppé et Gazera ; avec la tour qui est à Jérusalem, qui sont les villes de mon royaume.
\par 29 Vous en avez dévasté les frontières, et vous avez fait de grands dégâts dans le pays, et vous avez acquis la domination sur de nombreux lieux de mon royaume.
\par 30 Maintenant donc, délivrez les villes que vous avez prises et les tributs des lieux dont vous avez acquis la domination hors des frontières de la Judée :
\par 31 Ou bien, donne-moi pour eux cinq cents talents d'argent ; et pour le mal que vous avez fait, et les tributs des villes, cinq cents talents supplémentaires ; sinon, nous viendrons vous combattre.
\par 32 Alors Athénobe, ami du roi, vint à Jérusalem ; et voyant la gloire de Simon, l'armoire d'or et d'argent, et sa grande présence, il fut étonné et lui rapporta le message du roi.
\par 33 Alors Simon répondit et lui dit : Nous n'avons pris ni la terre d'autrui, ni détenu ce qui appartient à autrui, mais l'héritage de nos pères, que nos ennemis avaient injustement possédé pendant un certain temps.
\par 34 C'est pourquoi, lorsque nous en avons l'occasion, nous détenons l'héritage de nos pères.
\par 35 Et puisque tu demandes Joppé et Gazera, bien qu'ils aient fait un grand mal aux gens de notre pays, nous te donnerons cependant cent talents pour eux. Athénobe ne lui répondit pas un mot ;
\par 36 Mais il revint en colère vers le roi et lui fit rapport de ces discours, de la gloire de Simon et de tout ce qu'il avait vu. Sur quoi le roi fut extrêmement en colère.
\par 37 Pendant ce temps, Tryphon s'enfuit par bateau vers Orthosias.
\par 38 Alors le roi nomma Cendébée capitaine de la côte de la mer, et lui donna une armée de fantassins et de cavaliers.
\par 39 Et il lui ordonna de déplacer son armée vers la Judée ; il lui ordonna aussi de rebâtir Cédron, d'en fortifier les portes, et de faire la guerre au peuple ; mais le roi lui-même poursuivit Tryphon.
\par 40 Cendébéus arriva donc à Jamnia et se mit à provoquer le peuple, à envahir la Judée, à faire prisonnier le peuple et à le tuer.
\par 41 Et après avoir bâti Cérou, il y installa des cavaliers et une armée de fantassins, afin qu'en sortant ils puissent faire des routes sur les chemins de Judée, comme le roi lui avait ordonné.

\chapter{16}

\par 1 Alors Jean monta de Gazera et raconta à Simon, son père, ce qu'avait fait Cendébée.
\par 2 C'est pourquoi Simon appela ses deux fils aînés, Judas et Jean, et leur dit : Moi, mes frères et la maison de mon père, j'ai toujours combattu depuis ma jeunesse jusqu'à ce jour contre les ennemis d'Israël ; et les choses ont si bien prospéré entre nos mains, que nous avons souvent délivré Israël.
\par 3 Mais maintenant je suis vieux, et vous, par la miséricorde de Dieu, avez un âge suffisant : soyez à ma place et à ma place, et allez combattre pour notre nation, et le secours du ciel soit avec vous.
\par 4 Il choisit donc du pays vingt mille hommes de guerre avec des cavaliers, qui sortirent contre Cendébée et passèrent la nuit à Modin.
\par 5 Et comme ils se levaient le matin et s'en allaient dans la plaine, voici, une armée nombreuse et nombreuse de fantassins et de cavaliers vint contre eux ; mais il y avait un ruisseau entre eux.
\par 6 Et quand il vit que les gens avaient peur de passer le ruisseau, il se précipita d'abord sur lui-même, puis les hommes qui le voyaient passèrent après lui.
\par 7 Cela fait, il divisa ses hommes et plaça les cavaliers au milieu des fantassins, car les cavaliers ennemis étaient très nombreux.
\par 8 Alors ils sonnèrent des trompettes sacrées ; alors Cendébée et son armée furent mis en fuite, de sorte que beaucoup d'entre eux furent tués, et le reste les conduisit à la forteresse.
\par 9 À cette époque-là, le frère de Judas Jean était blessé ; mais Jean les suivit encore jusqu'à ce qu'il arrivât au Cédron, que Cendébée avait bâti.
\par 10 Et ils s'enfuirent jusqu'aux tours, dans les champs d'Azotus ; c'est pourquoi il le brûla au feu, de sorte qu'environ deux mille hommes furent tués. Ensuite, il revint en paix au pays de Judée.
\par 11 Ptolémée, fils d'Aboubus, fut nommé capitaine dans la plaine de Jéricho, et il avait beaucoup d'argent et d'or.
\par 12 Car il était le gendre du grand prêtre.
\par 13 C'est pourquoi, son cœur s'étant élevé, il pensa s'emparer du pays, et alors il consulta trompeusement contre Simon et ses fils pour les détruire.
\par 14 Or Simon visitait les villes qui étaient dans le pays et veillait à leur bon ordre ; C'est alors qu'il descendit lui-même à Jéricho avec ses fils Mattathias et Judas, la cent soixante-dix-septième année, le onzième mois appelé Sabat.
\par 15 Alors le fils d'Aboubus les reçut trompeusement dans une petite forteresse, appelée Docus, qu'il avait bâtie, et leur fit un grand festin ; mais il y avait caché des hommes.
\par 16 Et lorsque Simon et ses fils eurent beaucoup bu, Ptolémée et ses hommes se levèrent, prirent leurs armes, et tombèrent sur Simon dans le lieu du festin, et le tuèrent, ainsi que ses deux fils et certains de ses serviteurs.
\par 17 Ce faisant, il a commis une grande trahison et a rendu le mal pour le bien.
\par 18 Alors Ptolémée écrivit ces choses, et envoya au roi qu'il lui enverrait une armée pour l'aider, et qu'il lui délivrerait le pays et les villes.
\par 19 Il envoya aussi d'autres personnes à Gazera pour tuer Jean ; et il envoya aux tribuns des lettres pour qu'il leur donne de l'argent, de l'or et des récompenses.
\par 20 Et il en envoya d'autres pour prendre Jérusalem et la montagne du Temple.
\par 21 Or, quelqu'un avait couru à Gazera et avait dit à Jean que son père et ses frères avaient été tués, et, dit-il, Ptolémée a envoyé pour te tuer aussi.
\par 22 Lorsqu'il entendit cela, il fut très étonné ; alors il mit la main sur ceux qui étaient venus pour le détruire, et les tua ; car il savait qu'ils cherchaient à l'éloigner.
\par 23 Quant au reste des actions de Jean, et à ses guerres, et aux bonnes actions qu'il a faites, et à la construction des murs qu'il a faits, et à ses actions,
\par 24 Voici, ceci est écrit dans les chroniques de son sacerdoce, depuis le temps où il fut fait grand prêtre après son père.

\end{document}