\begin{document}

\title{2 Macchabées}


\chapter{1}

\par 1 Les frères Juifs qui sont à Jérusalem et dans le pays de Judée souhaitent aux frères Juifs qui sont dans toute l'Égypte santé et paix :
\par 2 Que Dieu vous fasse grâce, et souvenez-vous de son alliance qu'il a conclue avec Abraham, Isaac et Jacob, ses fidèles serviteurs ;
\par 3 Et donnez-vous à tous un cœur pour le servir et pour faire sa volonté, avec un bon courage et un esprit bien disposé ;
\par 4 Et ouvrez vos cœurs à sa loi et à ses commandements, et envoyez-vous la paix,
\par 5 Et écoutez vos prières, soyez en harmonie avec vous, et ne vous abandonnerez jamais dans les moments de détresse.
\par 6 Et maintenant nous sommes ici pour prier pour vous.
\par 7 À quelle époque régnait Démétrius, la cent soixante-neuvième année, nous, les Juifs, vous l'avons écrit dans l'extrême détresse qui nous survint au cours de ces années-là, depuis le moment où Jason et sa troupe se révoltèrent hors de la terre sainte et Royaume,
\par 8 Et nous avons brûlé le porche et versé le sang innocent ; alors nous avons prié l'Éternel, et nous avons été exaucés ; nous avons également offert des sacrifices et de la fine farine, allumé les lampes et présenté les pains.
\par 9 Et maintenant, veillez à célébrer la fête des tabernacles au mois de Casleu.
\par 10 La cent quatre-vingt-huitième année, le peuple qui était à Jérusalem et en Judée, le sanhédrin et Judas envoyèrent salutations et santé à Aristobule, maître du roi Ptolémée, qui était de la race des prêtres oints. et aux Juifs qui étaient en Égypte :
\par 11 Dans la mesure où Dieu nous a délivrés de grands périls, nous le remercions hautement, comme ayant combattu contre un roi.
\par 12 Car il chassa ceux qui combattaient dans la ville sainte.
\par 13 Car lorsque le chef fut entré en Perse, et avec lui l'armée qui semblait invincible, ils furent tués dans le temple de Nanea par la tromperie des prêtres de Nanea.
\par 14 Car Antiochus, comme s'il voulait l'épouser, vint dans ce lieu, avec ses amis qui étaient avec lui, pour recevoir de l'argent en guise de dot.
\par 15 Et lorsque les prêtres de Nanea furent partis, et qu'il fut entré avec un petit groupe dans l'enceinte du temple, ils fermèrent le temple dès qu'Antiochus entra.
\par 16 Et, ouvrant une porte privée du toit, ils jetèrent des pierres comme la foudre, et frappèrent le capitaine, les mirent en pièces, leur coupèrent la tête et les jetèrent à ceux qui étaient dehors.
\par 17 Béni soit notre Dieu en toutes choses, qui a livré les impies.
\par 18 C'est pourquoi, puisque nous avons maintenant l'intention de célébrer la purification du temple le vingt-cinquième jour du mois de Casleu, nous avons jugé nécessaire de vous en certifier, afin que vous puissiez aussi la célébrer, comme la fête des tabernacles, et du feu qui nous fut donné lorsque Néémie offrit le sacrifice, après qu'il eut bâti le temple et l'autel.
\par 19 Car lorsque nos pères furent conduits en Perse, les prêtres qui étaient alors pieux prirent en secret le feu de l'autel et le cachèrent dans le creux d'une fosse sans eau, où ils le gardèrent en sécurité, de sorte que l'endroit était inconnu de tous les hommes.
\par 20 Après de nombreuses années, quand cela plut à Dieu, Néémie, envoyé du roi de Perse, envoya au feu la postérité des prêtres qui l'avaient caché. Mais quand ils nous dirent qu'ils ne trouvèrent pas de feu, mais eau épaisse;
\par 21 Alors il leur ordonna de le rédiger et de l'apporter ; Et quand les sacrifices furent déposés, Néémie ordonna aux prêtres d'asperger d'eau le bois et les objets posés dessus.
\par 22 Lorsque cela fut fait, et que le temps arriva où le soleil, qui auparavant était caché dans la nuée, brillait, un grand feu s'alluma, de sorte que chacun fut dans l'étonnement.
\par 23 Et les sacrificateurs firent une prière pendant que le sacrifice se consumait, dis-je, les prêtres et tous les autres, Jonathan commençant, et les autres y répondant, comme le fit Néémie.
\par 24 Et la prière était de cette manière ; O Seigneur, Seigneur Dieu, Créateur de toutes choses, toi qui es craintif et fort, juste et miséricordieux, et le Roi unique et miséricordieux,
\par 25 Toi, le seul donneur de toutes choses, le seul juste, tout-puissant et éternel, toi qui as délivré Israël de toute détresse, qui as choisi les pères et les sanctifiés.
\par 26 Reçois le sacrifice pour tout ton peuple Israël, conserve ta part et sanctifie-la.
\par 27 Rassemble ceux qui sont dispersés loin de nous, délivre ceux qui servent parmi les païens, regarde ceux qui sont méprisés et abhorrés, et fais savoir aux païens que tu es notre Dieu.
\par 28 Châtiz ceux qui nous oppriment, et qui nous font du mal avec orgueil.
\par 29 Plante à nouveau ton peuple dans ton lieu saint, comme Moïse l'a dit.
\par 30 Et les prêtres chantaient des psaumes d'action de grâce.
\par 31 Lorsque le sacrifice fut consommé, Néémie ordonna de verser l'eau qui restait sur les grandes pierres.
\par 32 Lorsque cela fut fait, une flamme s'alluma, mais elle fut consumée par la lumière qui brillait de l'autel.
\par 33 Ainsi, lorsque cette affaire fut connue, on rapporta au roi de Perse que, dans le lieu où les prêtres qui étaient emmenés avaient caché le feu, de l'eau apparut, et que Néémie en avait purifié les sacrifices.
\par 34 Alors le roi, fermant le lieu, le rendit saint, après avoir examiné la question.
\par 35 Et le roi prit de nombreux cadeaux et les distribua à ceux qu'il voulait satisfaire.
\par 36 Et Néémie appela cette chose Naphthar, ce qui revient à dire une purification : mais beaucoup d'hommes l'appellent Néphi.

\chapter{2}

\par 1 On trouve aussi dans les annales que Jérémie le prophète a ordonné à ceux qui étaient emmenés de s'éloigner du feu, comme cela a été signifié :
\par 2 Et comment le prophète, leur ayant donné la loi, leur a ordonné de ne pas oublier les commandements du Seigneur, et de ne pas se tromper d'esprit lorsqu'ils voient des images d'argent et d'or avec leurs ornements.
\par 3 Et par d'autres discours semblables, il les exhorta à ce que la loi ne s'éloigne pas de leur cœur.
\par 4 Il était également contenu dans le même écrit, que le prophète, étant averti de Dieu, ordonna au tabernacle et à l'arche de l'accompagner, alors qu'il sortait dans la montagne où Moïse montait, et vit l'héritage de Dieu.
\par 5 Et quand Jérémie arriva là, il trouva une grotte creuse, dans laquelle il posa le tabernacle, et l'arche, et l'autel des parfums, et ainsi ferma la porte.
\par 6 Et quelques-uns de ceux qui le suivaient vinrent marquer le chemin, mais ils ne le trouvèrent pas.
\par 7 Ce que Jérémie s'en aperçut, les blâma, disant : Quant à cet endroit, il sera inconnu jusqu'au moment où Dieu rassemblera de nouveau son peuple et le recevra en miséricorde.
\par 8 Alors l'Éternel leur montrera ces choses, et la gloire de l'Éternel apparaîtra, ainsi que la nuée aussi, comme elle fut montrée sous Moïse, et comme lorsque Salomon désira que le lieu soit honorablement sanctifié.
\par 9 Il a également été déclaré que, étant sage, il offrit le sacrifice de la dédicace et de l'achèvement du temple.
\par 10 Et comme lorsque Moïse pria l'Éternel, le feu descendit du ciel et consuma les sacrifices : ainsi Salomon pria aussi, et le feu descendit du ciel et consuma les holocaustes.
\par 11 Et Moïse dit : Parce que le sacrifice pour le péché ne devait pas être mangé, il a été consommé.
\par 12 Salomon observa donc ces huit jours.
\par 13 Les mêmes choses ont également été rapportées dans les écrits et les commentaires de Néémie ; et comment il fonda une bibliothèque, rassemblant les actes des rois, des prophètes et de David, et les épîtres des rois concernant les saints dons.
\par 14 De la même manière, Judas a rassemblé toutes les choses qui ont été perdues à cause de la guerre que nous avons eue, et elles restent avec nous,
\par 15 C'est pourquoi, si vous en avez besoin, envoyez quelqu'un vous les chercher.
\par 16 Alors que nous allons donc célébrer la purification, nous vous avons écrit, et vous ferez bien, si vous observez les mêmes jours.
\par 17 Nous espérons aussi que le Dieu qui a délivré tout son peuple et lui a donné à tous un héritage, et le royaume, et la prêtrise, et le sanctuaire,
\par 18 Comme il l'a promis dans la loi, il aura bientôt pitié de nous et nous rassemblera de tous les pays sous les cieux dans le lieu saint ; car il nous a délivrés de grandes détresses et a purifié le lieu.
\par 19 Quant à Judas Maccabée et à ses frères, à la purification du grand temple et à la dédicace de l'autel,
\par 20 Et les guerres contre Antiochus Epiphane et Eupator son fils,
\par 21 Et les signes manifestes qui sont venus du ciel à ceux qui se sont comportés vaillamment pour leur honneur pour le judaïsme : de sorte que, n'étant qu'un petit nombre, ils ont vaincu tout le pays et ont pourchassé des multitudes barbares,
\par 22 Et il recouvra le temple célèbre dans le monde entier, et libéra la ville, et fit respecter les lois qui étaient en train de tomber, l'Éternel leur accordant toute sa faveur.
\par 23 Toutes ces choses, dis-je, étant déclarées par Jason de Cyrène en cinq livres, nous essaierons de les abréger en un seul volume.
\par 24 Car, considérant le nombre infini et la difficulté qu'ils rencontrent, ce désir d'examiner les récits de l'histoire, pour la variété du sujet,
\par 25 Nous avons veillé à ce que ceux qui veulent lire aient du plaisir, à ce que ceux qui désirent se souvenir soient à l'aise, et à ce que tous ceux entre les mains desquels cela tombe puissent en tirer profit.
\par 26 C'est pourquoi pour nous, qui avons entrepris ce pénible travail d'abréger, ce n'était pas facile, mais une question de sueur et de vigilance ;
\par 27 Même si ce n'est pas une facilité pour celui qui prépare un banquet et cherche le bénéfice des autres, cependant, pour le plaisir de beaucoup, nous entreprendrons volontiers ces grandes peines ;
\par 28 Laissant à l'auteur le traitement exact de chaque particulier, et s'efforçant de suivre les règles d'un abrégé.
\par 29 Car, comme le maître d'œuvre d'une maison neuve doit prendre soin de l'ensemble du bâtiment ; mais celui qui entreprend de l'exposer et de le peindre doit chercher des choses convenables pour l'orner : même ainsi, je pense que c'est chez nous.
\par 30 S'arrêter sur tous les points, examiner les choses dans leur ensemble, et être curieux des détails, appartient au premier auteur de l'histoire :
\par 31 Mais il doit être accordé à celui qui fera un abrégé de faire preuve de brièveté et d'éviter de trop travailler.
\par 32 Ici donc nous commencerons l'histoire : en ajoutant seulement ceci à ce qui a été dit, que c'est une chose insensée de faire un long prologue et d'être court dans l'histoire elle-même.

\chapter{3}

\par 1 Or, lorsque la ville sainte était habitée en toute paix et que les lois étaient très bien observées, à cause de la piété du grand prêtre Onias et de sa haine de la méchanceté,
\par 2 Il arriva que même les rois eux-mêmes honorèrent le lieu et magnifièrent le temple avec leurs plus beaux cadeaux ;
\par 3 De sorte que Séleucus d'Asie supporta de ses propres revenus tous les frais appartenant au service des sacrifices.
\par 4 Mais un certain Simon, de la tribu de Benjamin, qui avait été nommé gouverneur du temple, se brouilla avec le souverain sacrificateur au sujet du désordre qui régnait dans la ville.
\par 5 Et comme il ne pouvait pas vaincre Onias, il le confia à Apollonius, fils de Thraseas, qui était alors gouverneur de la Célosyrie et de la Phénice,
\par 6 Et il lui dit que le trésor de Jérusalem était rempli de sommes d'argent infinies, de sorte que la multitude de leurs richesses, qui n'avaient rien à voir avec le compte des sacrifices, était innombrable, et qu'il était possible de tout amener dans la main du roi.
\par 7 Or, Apollonius étant venu trouver le roi, et lui ayant montré l'argent dont on lui avait dit, le roi choisit Héliodore, son trésorier, et lui envoya avec ordre de lui apporter ledit argent.
\par 8 Aussitôt Héliodore partit en voyage ; sous le couvert de visiter les villes de Célosyrie et de Phénice, mais bien pour accomplir le dessein du roi.
\par 9 Et lorsqu'il fut arrivé à Jérusalem, et qu'il fut reçu avec courtoisie par le grand prêtre de la ville, il lui rapporta quelles nouvelles on lui avait données concernant l'argent, et lui expliqua pourquoi il était venu, et il lui demanda si ces choses étaient effectivement ainsi.
\par 10 Alors le grand prêtre lui dit qu'il y avait un tel argent réservé pour le secours des veuves et des enfants orphelins :
\par 11 Et qu'une partie appartenait à Hircanus, fils de Tobie, homme d'une grande dignité, et non comme ce méchant Simon l'avait mal informé : la somme en tout était de quatre cents talents d'argent et deux cents d'or.
\par 12 Et qu'il était tout à fait impossible que de tels torts soient causés à ceux qui l'avaient confié à la sainteté du lieu, à la majesté et à la sainteté inviolable du temple, honoré dans le monde entier.
\par 13 Mais Héliodore, à cause de l'ordre que le roi lui avait donné, dit : De quelque manière que ce soit, il faudra l'introduire dans le trésor du roi.
\par 14 Et au jour qu'il avait fixé, il entra pour régler cette affaire ; c'est pourquoi il y eut une grande agonie dans toute la ville.
\par 15 Mais les prêtres, se prosternant devant l'autel dans leurs vêtements sacerdotaux, invoquèrent au ciel celui qui avait fait une loi concernant les choses données à observer, afin qu'elles soient conservées en toute sécurité pour ceux qui les avaient confiées à l'observance.
\par 16 Alors quiconque aurait regardé le souverain sacrificateur en face aurait blessé son cœur : car son visage et le changement de sa couleur révélaient l'agonie intérieure de son esprit.
\par 17 Car cet homme était si rempli de crainte et d'horreur à l'égard de son corps, qu'il était évident à ceux qui le regardaient quelle tristesse il avait maintenant dans son cœur.
\par 18 D'autres accouraient en masse hors de leurs maisons pour répondre à la supplication générale, parce que l'endroit semblait être méprisable.
\par 19 Et les femmes, ceintes d'un sac sous la poitrine, abondaient dans les rues, et les vierges qui y étaient gardées couraient, les unes vers les portes, les autres vers les murs, et d'autres regardaient par les fenêtres.
\par 20 Et tous, levant les mains vers le ciel, imploraient.
\par 21 Alors il aurait eu pitié d'un homme de voir la chute de la multitude de toutes sortes, et la crainte que le souverain sacrificateur soit dans une telle agonie.
\par 22 Ils invoquèrent alors le Seigneur Tout-Puissant pour qu'il garde les choses confiées en toute confiance pour ceux qui les avaient commises.
\par 23 Néanmoins Héliodore exécuta ce qui avait été décrété.
\par 24 Or, pendant qu'il était là, se présentant avec sa garde autour du trésor, le Seigneur des esprits et le Prince de toute puissance provoqua une grande apparition, de sorte que tous ceux qui prétendaient entrer avec lui furent étonnés de la puissance de Dieu, et je me suis évanoui, et j'ai eu très peur.
\par 25 Car leur apparut un cheval avec un cavalier terrible sur lui et paré d'une très belle couverture, et il courut avec férocité et frappa Héliodore avec ses pieds antérieurs, et il semblait que celui qui était assis sur le cheval avait complètement harnais d'or.
\par 26 De plus, deux autres jeunes hommes apparurent devant lui, remarquables par leur force, excellents par leur beauté et agréables par leur habillement, qui se tenaient à ses côtés ; et il le flagellait continuellement, et lui infligeait de nombreux coups douloureux.
\par 27 Et Héliodore tomba subitement à terre, et fut entouré d'une grande obscurité ; mais ceux qui étaient avec lui le relevèrent et le mirent dans une litière.
\par 28 Ainsi, celui qui était récemment venu avec un grand cortège et avec toute sa garde dans ledit trésor, ils l'emportèrent, ne pouvant s'aider de ses armes : et manifestement ils reconnurent la puissance de Dieu.
\par 29 Car il fut abattu par la main de Dieu et resta sans voix, sans aucun espoir de vie.
\par 30 Mais ils louèrent le Seigneur, qui avait miraculeusement honoré son propre lieu : pour le temple ; qui, peu auparavant, était pleine de peur et de trouble, lorsque le Seigneur Tout-Puissant apparut, fut remplie de joie et d'allégresse.
\par 31 Alors aussitôt certains amis d'Héliodore prièrent Onias pour qu'il invoque le Très-Haut pour lui accorder la vie, qui était prêt à rendre l'âme.
\par 32 Le grand prêtre, craignant que le roi ne se méprenne sur une trahison commise par les Juifs envers Héliodore, offrit un sacrifice pour la santé de cet homme.
\par 33 Alors que le souverain sacrificateur faisait l'expiation, les mêmes jeunes gens, vêtus de la même manière, apparurent et se tinrent à côté d'Héliodore, disant : Rendons de grandes grâces à Onias, le souverain sacrificateur, de ce que l'Éternel t'a accordé la vie à cause de lui.
\par 34 Et puisque tu as été flagellé du ciel, déclare à tous la toute-puissance de Dieu. Et après avoir prononcé ces paroles, ils ne reparurent plus.
\par 35 Ainsi Héliodore, après avoir offert un sacrifice à l'Éternel, et fait de grands vœux à celui qui lui avait sauvé la vie, et salué Onias, revint avec son armée vers le roi.
\par 36 Puis il rendit témoignage à tous les hommes des œuvres du grand Dieu, qu'il avait vues de ses yeux.
\par 37 Et lorsque le roi Héliodore, qui pouvait être un homme apte à être envoyé une fois de plus à Jérusalem, dit :
\par 38 Si tu as un ennemi ou un traître, envoie-le là-bas, et tu le recevras bien flagellé, s'il échappe à la vie : car dans ce lieu-là, sans doute ; il y a une puissance particulière de Dieu.
\par 39 Car celui qui habite dans les cieux a les yeux fixés sur ce lieu et le défend ; et il bat et détruit ceux qui viennent lui faire du mal.
\par 40 C'est ainsi que se déroulèrent les affaires concernant Héliodore et la tenue du trésor.

\chapter{4}

\par 1 Ce Simon, dont nous avons parlé plus haut, ayant trahi l'argent et la patrie, calomnia Onias, comme s'il avait effrayé Héliodore et avait été l'auteur de ces maux.
\par 2 Ainsi, il était audacieux de le traiter de traître, qui avait bien mérité de la ville, et avait offert sa propre nation, et était si zélé pour les lois.
\par 3 Mais leur haine étant allée si loin, que des meurtres furent commis par l'un des partisans de Simon,
\par 4 Onias, voyant le danger de cette dispute, et qu'Apollonius, en tant que gouverneur de la Célosyrie et de la Phénice, était en colère et augmentait la méchanceté de Simon,
\par 5 Il alla vers le roi, non pour accuser ses compatriotes, mais pour rechercher le bien de tous, tant public que privé :
\par 6 Car il voyait qu'il était impossible que l'État restât tranquille et que Simon abandonne sa folie, à moins que le roi n'y prête attention.
\par 7 Mais après la mort de Séleucus, quand Antiochus, appelé Épiphane, prit le royaume, Jason, frère d'Onias, travailla sournoisement pour devenir grand prêtre,
\par 8 Promettant au roi par intercession trois cent soixante talents d'argent, et un autre revenu de quatre-vingts talents :
\par 9 En outre, il promit d'en assigner cent cinquante autres, s'il pouvait avoir l'autorisation de lui établir un lieu pour l'exercice et pour l'entraînement de la jeunesse aux modes des païens, et pour les écrire de Jérusalem. du nom d'Antiochiens.
\par 10 Ce que le roi ayant accordé, et ayant mis entre ses mains le pouvoir, il amena aussitôt sa propre nation à la mode grecque.
\par 11 Et les privilèges royaux accordés en faveur spéciale aux Juifs par l'intermédiaire de Jean, père d'Eupolème, qui était envoyé ambassadeur à Rome pour l'amitié et l'aide, il les ôta ; et renversant les gouvernements qui étaient selon la loi, il introduisit de nouvelles coutumes contre la loi :
\par 12 Car il construisit volontiers un lieu d'exercice sous la tour elle-même, et il soumettait les principaux jeunes gens à sa soumission, et leur faisait porter un chapeau.
\par 13 Or, telle était la hauteur des modes grecques et l'augmentation des manières païennes, à cause de l'extrême profanation de Jason, ce misérable impie et sans grand prêtre ;
\par 14 Que les prêtres n'avaient plus le courage de servir à l'autel, mais méprisant le temple et négligeant les sacrifices, se hâtaient de participer à l'allocation illégale dans le lieu d'exercice, après que le jeu de disque les avait appelés ;
\par 15 Ils ne se fient pas aux honneurs de leurs pères, mais ils aiment avant tout la gloire des Grecs.
\par 16 C'est pourquoi un grand malheur les frappa, car ils avaient pour ennemis et vengeurs ceux dont ils suivaient si sincèrement la coutume, et auxquels ils désiraient être semblables en toutes choses.
\par 17 Car ce n'est pas une chose légère de faire le mal contre les lois de Dieu ; mais le temps qui suivra déclarera ces choses.
\par 18 Or, lorsque le jeu dont on jouait chaque année de foi était célébré à Tyrus, le roi étant présent,
\par 19 Ce méchant Jason envoya de Jérusalem des messagers spéciaux, qui étaient des Antiochiens, pour apporter trois cents drachmes d'argent au sacrifice d'Hercule, que même ceux qui les portaient jugeaient à propos de ne pas accorder au sacrifice, parce que cela ne convenait pas, mais à réserver pour d'autres charges.
\par 20 Cet argent donc, en ce qui concerne l'expéditeur, était destiné au sacrifice d'Hercule ; mais à cause de ses porteurs, on l'employait à faire des galères.
\par 21 Or, quand Apollonius, fils de Ménesthée, fut envoyé en Égypte pour le couronnement du roi Ptolémée Philométor, Antiochus, comprenant qu'il ne se souciait pas de ses affaires, pourvu à sa propre sécurité. Sur quoi il vint à Joppé, et de là à Jérusalem :
\par 22 Où il fut honorablement reçu par Jason et par la ville, et fut amené avec des torches allumées et avec de grands cris ; puis il partit avec son armée vers Phénice.
\par 23 Trois ans après, Jason envoya Ménélans, frère de Simon susmentionné, pour porter l'argent au roi et lui rappeler certaines choses nécessaires.
\par 24 Mais étant amené devant le roi, après l'avoir magnifié pour l'apparence glorieuse de sa puissance, il obtint le sacerdoce pour lui-même, offrant plus que Jason pour trois cents talents d'argent.
\par 25 Il vint donc avec le mandat du roi, n'apportant rien qui soit digne du grand sacerdoce, mais ayant la fureur d'un tyran cruel et la rage d'une bête sauvage.
\par 26 Alors Jason, qui avait miné son propre frère, étant miné par un autre, fut contraint de fuir au pays des Ammonites.
\par 27 Ainsi Ménélans obtint la principauté ; mais quant à l'argent qu'il avait promis au roi, il n'en prit pas bonne commande, bien que Sostratis, le chef du château, l'exigeât.
\par 28 Car c'était à lui qu'appartenait le rassemblement des coutumes. C'est pourquoi ils furent tous deux convoqués devant le roi.
\par 29 Or Ménélans laissa son frère Lysimaque à sa place dans le sacerdoce ; et Sostratus quitta Cratès, qui était gouverneur des Cypriens.
\par 30 Pendant que ces choses se produisaient, ceux de Tarse et de Mallos se révoltèrent, parce qu'ils avaient été donnés à la concubine du roi, appelée Antiochus.
\par 31 Alors le roi vint en toute hâte pour apaiser les choses, laissant Andronicus, un homme d'autorité, pour son adjoint.
\par 32 Or Ménélans, croyant avoir trouvé le moment opportun, déroba du temple certains vases en or, en donna une partie à Andronicus, et en vendit une autre à Tyrus et dans les villes avoisinantes.
\par 33 Ce qui, ayant connu une certitude, Onias le réprimanda et se retira dans un sanctuaire à Daphné, qui se trouve près d'Antiochia.
\par 34 C'est pourquoi Ménélans, prenant Andronicus à part, le pria de mettre Onias entre ses mains ; Celui-ci, persuadé de cela, et venant à Onias par tromperie, lui donna la main droite avec serment ; et bien qu'il ait été soupçonné par lui, il l'a néanmoins persuadé de sortir du sanctuaire : il l'a immédiatement enfermé sans égard à la justice.
\par 35 Pour cette cause, non seulement les Juifs, mais aussi beaucoup d'autres nations furent très indignés et furent très attristés à cause du meurtre injuste de cet homme.
\par 36 Et lorsque le roi revint des environs de Cilicie, les Juifs qui étaient dans la ville et certains Grecs qui détestaient aussi ce fait, se plaignirent parce qu'Onias avait été tué sans motif.
\par 37 C'est pourquoi Antiochus fut profondément désolé, ému de pitié et pleurant à cause de la conduite sobre et modeste de celui qui était mort.
\par 38 Et, enflammé de colère, il ôta aussitôt à Andronicus sa pourpre, déchira ses vêtements, et le conduisit à travers toute la ville jusqu'au lieu même où il avait commis l'impiété contre Onias, où il tua le meurtrier maudit. Ainsi le Seigneur lui récompensa son châtiment, comme il l'avait mérité.
\par 39 Or, comme Lysimaque avait commis dans la ville de nombreux sacrilèges, avec le consentement de Ménélans, et que leurs fruits se répandaient, la multitude se rassembla contre Lysimaque, de nombreux vases d'or étant déjà emportés.
\par 40 Alors le peuple se souleva et fut rempli de colère. Lysimaque arma environ trois mille hommes, et commença d'abord à offrir la violence ; un certain Auranus étant le chef, un homme disparu en années, et non moins en folie.
\par 41 Alors, voyant l'attaque de Lysimaque, les uns attrapèrent des pierres, d'autres des bâtons, d'autres prenant des poignées de poussière qui se trouvaient ensuite à portée de main, et les jetèrent toutes ensemble sur Lysimaque, et ceux qui les frappèrent.
\par 42 Ils en blessèrent beaucoup, en frappèrent quelques-uns à terre, et tous ils les forcèrent à fuir ; mais quant au voleur d'église lui-même, ils le tuèrent près du trésor.
\par 43 C'est pourquoi une accusation fut portée contre Ménélans.
\par 44 Lorsque le roi arriva à Tyrus, trois hommes envoyés du Sénat plaidèrent devant lui :
\par 45 Mais Ménélans, convaincu, promit à Ptolémée, fils de Doryménès, de lui donner beaucoup d'argent, s'il voulait apaiser le roi à son égard.
\par 46 Alors Ptolémée, emmenant le roi à l'écart dans une certaine galerie, comme pour prendre l'air, l'amena à changer d'avis :
\par 47 De sorte qu'il délivra Ménélans des accusations, qui pourtant était la cause de tous les malheurs ; et ces pauvres hommes qui, s'ils avaient exposé leur cause, oui, devant les Scythes, auraient dû être jugés innocents, il les condamna. à mort.
\par 48 Ainsi, ceux qui s'occupaient de l'affaire pour la ville, et pour le peuple, et pour les objets sacrés, subirent bientôt un châtiment injuste.
\par 49 C'est pourquoi même ceux de Tyrus, émus par la haine de cette mauvaise action, les firent enterrer honorablement.
\par 50 Et ainsi, grâce à la convoitise de ceux qui détenaient le pouvoir, Ménélans resta toujours au pouvoir, augmentant sa méchanceté et étant un grand traître envers les citoyens.

\chapter{5}

\par 1 Vers la même époque, Antiochus préparait son deuxième voyage en Égypte :
\par 2 Et alors il arriva que dans toute la ville, pendant presque quarante jours, on vit des cavaliers courir dans les airs, vêtus de draps d'or et armés de lances, comme une troupe de soldats,
\par 3 Et des troupes de cavaliers en rangée, se rencontrant et courant les uns contre les autres, avec des boucliers agités, et une multitude de piques, et tirant des épées, et lançant des dards, et des ornements d'or scintillants, et des harnais de toutes sortes.
\par 4 C'est pourquoi chacun a prié pour que cette apparition se transforme en bien.
\par 5 Or, comme une fausse rumeur courait selon laquelle Antiochus était mort, Jason prit au moins mille hommes et lança tout à coup un assaut sur la ville ; Et ceux qui étaient sur les murs étant repoussés, et la ville enfin prise, Ménélans s'enfuit dans le château.
\par 6 Mais Jason tua ses propres citoyens sans pitié, ne considérant pas que gagner le jour des membres de sa propre nation serait un jour des plus malheureux pour lui ; mais pensant qu'ils avaient été ses ennemis, et non ses compatriotes, qu'il avait vaincus.
\par 7 Cependant, malgré tout cela, il n'obtint pas la principauté, mais il reçut finalement la honte pour la récompense de sa trahison, et s'enfuit de nouveau dans le pays des Ammonites.
\par 8 À la fin, il eut un retour malheureux, étant accusé devant Aretas, roi des Arabes, fuyant de ville en ville, poursuivi de tous les hommes, haï comme un transgresseur des lois, et étant en abomination comme un traître ouvert. ennemi de son pays et de ses compatriotes, il fut chassé en Égypte.
\par 9 Ainsi, celui qui en avait chassé beaucoup de leur pays périt dans un pays étranger, se retirant chez les Lacédémoniens et pensant y trouver du secours en raison de sa parenté.
\par 10 Et celui qui avait chassé beaucoup de gens sans sépulture n'avait aucun lieu de deuil pour lui, ni de funérailles solennelles, ni de sépulcre avec ses pères.
\par 11 Or, lorsque ce qui avait été fait arriva au char du roi, il crut que la Judée s'était révoltée ; sur quoi, sortant d'Egypte avec un esprit furieux, il prit la ville par la force des armes,
\par 12 Et il ordonna à ses hommes de guerre de ne pas épargner ceux qu'ils rencontraient, et de tuer ceux qui montaient sur les maisons.
\par 13 Ainsi, on tua des jeunes et des vieux, on tua des hommes, des femmes et des enfants, on tua des vierges et des enfants.
\par 14 Et en l'espace de trois jours entiers, quatre-soixante mille furent détruits, dont quarante mille furent tués dans le combat ; et pas moins vendu que tué.
\par 15 Mais il ne se contentait pas de cela, mais il prétendait entrer dans le temple le plus saint du monde ; Ménélans, ce traître aux lois et à sa patrie, étant son guide :
\par 16 Et prenant les vases sacrés avec des mains impures, et avec des mains profanes détruisant les choses qui étaient consacrées par d'autres rois à l'augmentation, à la gloire et à l'honneur du lieu, il les donna.
\par 17 Et Antiochus était si hautain qu'il ne considéra pas que l'Éternel était pendant un moment en colère à cause des péchés des habitants de la ville, et c'est pourquoi ses yeux ne furent pas fixés sur cet endroit.
\par 18 Car s'ils n'avaient pas été autrefois enveloppés de beaucoup de péchés, cet homme, dès son arrivée, aurait été aussitôt flagellé et détourné de sa présomption, comme Héliodore, que le roi Séleucus envoya visiter le trésor.
\par 19 Néanmoins Dieu n'a pas choisi le peuple à cause du lieu, mais le lieu à cause du peuple.
\par 20 Et c'est pourquoi le lieu lui-même, qui avait part avec eux à l'adversité qui arrivait à la nation, communiqua ensuite aux bienfaits envoyés par le Seigneur ; et comme il fut abandonné dans la colère du Tout-Puissant, de même encore, le le grand Seigneur étant réconcilié, il fut érigé en toute gloire.
\par 21 Ainsi, après qu'Antiochus eut emporté du temple mille huit cents talents, il partit en toute hâte pour Antiochia, animé par son orgueil de rendre la terre navigable et la mer praticable à pied : telle était la hauteur de son esprit.
\par 22 Et il laissa des gouverneurs pour contrarier la nation : à Jérusalem, Philippe, Phrygien pour son pays, et pour ses manières plus barbares que celui qui l'y avait placé ;
\par 23 Et à Garizim, Andronicus ; et en outre, Ménélans, qui, plus que tous les autres, exerça une main lourde sur les citoyens, ayant un esprit malveillant contre ses compatriotes les Juifs.
\par 24 Il envoya aussi le détestable chef Apollonius avec une armée de vingt-deux mille hommes, lui ordonnant de tuer tous ceux qui étaient dans leur plus bel âge, et de vendre les femmes et les plus jeunes.
\par 25 Celui qui, étant venu à Jérusalem et prétendant la paix, s'abstint jusqu'au jour saint du sabbat, lorsqu'il emmena les Juifs célébrer le jour saint, il ordonna à ses hommes de s'armer.
\par 26 Et ainsi il tua tous ceux qui allaient célébrer le sabbat, et courant à travers la ville avec des armes, il tua une grande foule.
\par 27 Mais Judas Maccabée et neuf autres personnes, ou à peu près, se retirèrent dans le désert et vécurent dans les montagnes à la manière des bêtes, avec sa compagnie, qui se nourrissaient continuellement d'herbes, de peur qu'ils ne participent à la pollution.

\chapter{6}

\par 1 Peu de temps après, le roi envoya un vieil homme d'Athènes pour contraindre les Juifs à s'écarter des lois de leurs pères et à ne pas vivre selon les lois de Dieu :
\par 2 Et de polluer aussi le temple de Jérusalem, et de l'appeler le temple de Jupiter Olympius ; et celui de Garizim, de Jupiter, le défenseur des étrangers, comme ils le souhaitaient qui habitaient dans ce lieu.
\par 3 L'arrivée de ce malheur fut douloureuse et pénible pour le peuple :
\par 4 Car le temple était rempli d'émeutes et de réjouissances de la part des Gentils, qui fréquentaient les prostituées et avaient affaire aux femmes dans le périmètre des lieux saints, et qui en outre apportaient des choses qui n'étaient pas licites.
\par 5 L'autel aussi était rempli de choses profanes, que la loi interdit.
\par 6 Il n'était pas non plus permis à un homme d'observer les jours de sabbat ou les jeûnes anciens, ni de se déclarer juif.
\par 7 Et chaque mois, le jour de la naissance du roi, ils étaient amenés par une amère contrainte à manger des sacrifices ; et lorsque le jeûne de Bacchus était observé, les Juifs étaient obligés de se rendre en procession à Bacchus, portant du lierre.
\par 8 De plus, un décret fut publié dans les villes voisines des païens, sur la suggestion de Ptolémée, contre les Juifs, afin qu'ils observent les mêmes modes et participent à leurs sacrifices :
\par 9 Et quiconque ne se conformerait pas aux manières des païens serait mis à mort. Alors un homme aurait pu voir la misère actuelle.
\par 10 Car on avait amené deux femmes qui avaient circoncis leurs enfants ; Après avoir ouvertement fait faire le tour de la ville, les enfants leur tendant les seins, ils les jetèrent tête baissée du haut du mur.
\par 11 Et d'autres, qui s'étaient rassemblés dans des grottes voisines pour observer secrètement le jour du sabbat, furent découverts par Philippe et furent tous brûlés ensemble, parce qu'ils avaient pris conscience de se servir eux-mêmes pour l'honneur du jour le plus sacré.
\par 12 Maintenant, je supplie ceux qui lisent ce livre, de ne pas se décourager à cause de ces calamités, mais de considérer que ces châtiments ne sont pas destinés à la destruction, mais à un châtiment de notre nation.
\par 13 Car c'est un signe de sa grande bonté que de ne pas souffrir longtemps les méchants, mais de les punir immédiatement.
\par 14 Car ce n'est pas comme les autres nations, que le Seigneur s'abstient patiemment de punir, jusqu'à ce qu'elles soient parvenues à la plénitude de leurs péchés, ainsi il nous traite,
\par 15 De peur que, étant arrivé au comble du péché, il ne se venge ensuite de nous.
\par 16 Et c'est pourquoi il ne nous retire jamais sa miséricorde ; et bien qu'il punisse par l'adversité, il n'abandonne jamais son peuple.
\par 17 Mais que ce que nous avons dit soit pour nous un avertissement. Et maintenant, venons-en à l’exposé de la question en quelques mots.
\par 18 Éléazar, l'un des principaux scribes, un homme âgé et de belle apparence, fut contraint d'ouvrir la bouche et de manger de la chair de porc.
\par 19 Mais lui, préférant mourir glorieusement plutôt que de vivre entaché d'une telle abomination, le cracha et vint de lui-même au tourment,
\par 20 Comme il convenait à ceux qui sont résolus de s'opposer à des choses telles qu'il n'est pas permis de goûter à l'amour de la vie, ils doivent venir.
\par 21 Mais ceux qui étaient chargés de ce méchant festin, à cause de leur ancienne relation avec cet homme, le prièrent à part, et le prièrent d'apporter de la viande de sa propre provision, telle qu'il lui était permis d'en consommer, et de la préparer comme il se doit. s'il a mangé de la chair prélevée sur le sacrifice commandé par le roi ;
\par 22 Afin qu'en agissant ainsi, il soit délivré de la mort, et que sa vieille amitié avec eux trouve grâce.
\par 23 Mais il commença à considérer discrètement, et ce qui convenait à son âge, et à l'excellence de ses vieilles années, et à l'honneur de sa tête grise, sur laquelle était venu, et à son éducation la plus honnête d'un enfant, ou plutôt à la sainte loi. fait et donné par Dieu : c'est pourquoi il répondit en conséquence, et voulut qu'on l'envoie immédiatement au tombeau.
\par 24 Car il ne convient pas à notre âge, dit-il, de dissimuler en quoi que ce soit, par quoi beaucoup de jeunes gens pourraient penser qu'Éléazar, âgé de quatre-vingt-dix ans, était maintenant allé à une religion étrangère ;
\par 25 Et ainsi, à cause de mon hypocrisie et de mon désir de vivre un peu de temps et un instant de plus, ils devraient être trompés par moi, et je tacherai ma vieillesse et la rendrai abominable.
\par 26 Car, même si pour le moment je serais délivré du châtiment des hommes, je n'échapperais pas à la main du Tout-Puissant, ni vivant ni mort.
\par 27 C'est pourquoi maintenant, changeant vaillamment cette vie, je me montrerai tel que mon âge l'exige,
\par 28 Et laissez un exemple notable à ceux qui sont jeunes de mourir volontairement et courageusement pour les lois honorables et saintes. Et après avoir dit ces paroles, aussitôt il se tourna vers le tourment :
\par 29 Ceux qui l'ont amené à changer la bonne volonté l'ont livré un peu auparavant à la haine, parce que les discours susmentionnés provenaient, comme ils le pensaient, d'un esprit désespéré.
\par 30 Mais alors qu'il était sur le point de mourir frappé de coups, il gémit et dit : Il est manifeste pour le Seigneur, qui a la sainte connaissance, que, alors que j'aurais pu être délivré de la mort, j'endure maintenant des douleurs douloureuses dans le corps par être battu : mais dans mon âme, je suis bien content de souffrir ces choses, parce que je le crains.
\par 31 Et ainsi cet homme mourut, laissant sa mort comme un exemple de noble courage et un mémorial de vertu, non seulement pour les jeunes gens, mais pour toute sa nation.

\chapter{7}

\par 1 Il arriva aussi que sept frères et leur mère furent pris, et contraints par le roi, contre la loi, de goûter de la chair de pourceau, et furent tourmentés avec des fouets et des fouets.
\par 2 Mais l'un de ceux qui parlèrent le premier dit ainsi : Que veux-tu nous demander ou apprendre de nous ? nous sommes prêts à mourir plutôt que de transgresser les lois de nos pères.
\par 3 Alors le roi, étant en colère, ordonna de faire chauffer les casseroles et les chaudrons :
\par 4 Lequel étant aussitôt chauffé, il ordonna de couper la langue à celui qui parlait le premier, et de lui couper les parties les plus extrêmes du corps, sous le regard du reste de ses frères et de sa mère.
\par 5 Or, lorsqu'il fut ainsi mutilé de tous ses membres, il ordonna qu'il soit encore vivant, qu'il soit amené au feu et qu'il soit frit dans la poêle ; et comme la vapeur de la poêle était dispersée pendant un bon espace, ils s'exhortèrent mutuellement avec la mère à mourir vaillamment, disant ainsi :
\par 6 Le Seigneur Dieu nous regarde, et en vérité il a du réconfort en nous, comme Moïse dans son chant, qui témoignait à leurs visages, l'a déclaré, en disant : Et il sera consolé par ses serviteurs.
\par 7 Ainsi, quand le premier fut mort après ce nombre, ils amenèrent le second pour en faire une dérision ; et après qu'ils lui eurent arraché la peau de la tête avec les cheveux, ils lui demandèrent : Veux-tu manger avant de devenir puni dans tous les membres de ton corps ?
\par 8 Mais il répondit dans sa propre langue et dit : Non. C'est pourquoi il reçut également le tourment suivant dans l'ordre, comme le premier.
\par 9 Et quand il fut au dernier soupir, il dit : Comme une fureur, tu nous fais sortir de cette vie présente, mais le roi du monde nous ressuscitera, nous qui sommes morts pour ses lois, pour la vie éternelle.
\par 10 Après lui, le troisième fut fait moqueur ; et quand on le demandait, il tirait la langue, et cela aussitôt, en tendant vaillamment ses mains.
\par 11 Et il dit courageusement : Ceux-là, je les ai reçus du ciel ; et à cause de ses lois, je les méprise ; et de lui j'espère les recevoir à nouveau.
\par 12 De sorte que le roi et ceux qui étaient avec lui s'émerveillèrent du courage du jeune homme, pour cela il ne fit aucun cas des douleurs.
\par 13 Or, lorsque cet homme était mort aussi, ils tourmentèrent et mutilèrent le quatrième de la même manière.
\par 14 Alors, alors qu'il était sur le point de mourir, il dit ainsi : Il est bon, après avoir été mis à mort par les hommes, d'attendre de Dieu l'espérance d'être ressuscité par lui ; quant à toi, tu n'auras pas de résurrection à la vie. .
\par 15 Ensuite, ils amenèrent aussi le cinquième et le mutilèrent.
\par 16 Alors il regarda le roi et dit : Tu as pouvoir sur les hommes, tu es corruptible, tu fais ce que tu veux ; mais ne pensez pas que notre nation soit abandonnée de Dieu ;
\par 17 Mais attends un peu, et vois sa grande puissance, comment il te tourmentera, toi et ta postérité.
\par 18 Après lui aussi, ils amenèrent le sixième, qui, étant prêt à mourir, dit : Ne vous laissez pas tromper sans raison ; car nous souffrons ces choses pour nous-mêmes, après avoir péché contre notre Dieu ; c'est pourquoi des choses merveilleuses nous sont faites.
\par 19 Mais ne pense pas que toi, qui entreprends de lutter contre Dieu, tu échapperas impuni.
\par 20 Mais la mère était surtout merveilleuse et digne d'un souvenir honorable : car lorsqu'elle vit ses sept fils tués en l'espace d'un jour, elle le supporta avec bon courage, à cause de l'espérance qu'elle avait dans le Seigneur. .
\par 21 Oui, elle les exhortait chacun dans sa langue, pleine d'esprit courageux ; et remuant ses pensées féminines avec un ventre viril, elle leur dit :
\par 22 Je ne puis dire comment vous êtes entrés dans mon sein ; car je ne vous ai donné ni souffle ni vie, et ce n'est pas non plus moi qui ai formé les membres de chacun de vous ;
\par 23 Mais sans aucun doute, le Créateur du monde, qui a formé la génération de l'homme et découvert le commencement de toutes choses, vous redonnera aussi, par sa propre miséricorde, le souffle et la vie, puisque vous ne vous souciez plus de lui-même. pour l'amour des lois.
\par 24 Or Antiochus, se croyant méprisé et soupçonnant que c'était un discours de reproche, alors que le plus jeune était encore en vie, non seulement l'exhorta par des paroles, mais encore l'assura par des serments qu'il ferait de lui à la fois un riche et un riche. un homme heureux, s'il se détournait des lois de ses pères ; et qu'il le prendrait aussi pour son ami et lui confierait les affaires.
\par 25 Mais comme le jeune homme ne voulait en aucun cas l'écouter, le roi appela sa mère et l'exhorta à conseiller au jeune homme de lui sauver la vie.
\par 26 Et après qu'il l'eut exhortée par de nombreuses paroles, elle lui promit qu'elle conseillerait son fils.
\par 27 Mais elle s'inclinait vers lui, se moquant du cruel tyran, et parlait de cette manière dans la langue de son pays ; Ô mon fils, aie pitié de moi qui t'ai enfanté neuf mois dans mon ventre, et qui t'ai donné trois années, qui t'ai nourri, qui t'a élevé jusqu'à cet âge et qui a enduré les ennuis de l'éducation.
\par 28 Je te prie, mon fils, de regarder le ciel et la terre, et tout ce qui s'y trouve, et de considérer que Dieu les a faits de choses qui n'existaient pas ; et c’est ainsi que l’humanité a été créée.
\par 29 Ne crains pas ce bourreau, mais, étant digne de tes frères, accepte ta mort, afin que je te reçoive de nouveau en miséricorde avec tes frères.
\par 30 Tandis qu'elle prononçait encore ces paroles, le jeune homme dit : Qui attendez-vous ? Je n'obéirai pas au commandement du roi, mais j'obéirai au commandement de la loi qui a été donnée à nos pères par Moïse.
\par 31 Et toi, qui es l'auteur de tous les méfaits contre les Hébreux, tu n'échapperas pas aux mains de Dieu.
\par 32 Car nous souffrons à cause de nos péchés.
\par 33 Et même si le Seigneur vivant se met en colère contre nous pour un petit moment à cause de notre châtiment et de notre correction, il sera de nouveau d'accord avec ses serviteurs.
\par 34 Mais toi, homme impie, et entre tous les plus méchants, ne t'élève pas sans cause, ni enflé d'espérances incertaines, en levant ta main contre les serviteurs de Dieu :
\par 35 Car tu n'as pas encore échappé au jugement du Dieu Tout-Puissant, qui voit toutes choses.
\par 36 Car nos frères, qui ont maintenant souffert une courte souffrance, sont morts sous l'alliance de Dieu de la vie éternelle ; mais toi, par le jugement de Dieu, tu recevras un juste châtiment pour ton orgueil.
\par 37 Mais moi, en tant que frères, j'offre mon corps et ma vie pour les lois de nos pères, implorant Dieu d'être promptement miséricordieux envers notre nation ; et que tu puisses confesser, par les tourments et les plaies, que lui seul est Dieu ;
\par 38 Et qu'en moi et en mes frères cesse la colère du Tout-Puissant, qui est justement attirée sur notre nation.
\par 39 Le roi, étant en colère, le traita plus mal que tous les autres, et prit au sérieux qu'on se moquât de lui.
\par 40 Cet homme mourut donc sans souillure, et il mit toute sa confiance dans le Seigneur.
\par 41 Enfin, après les fils, la mère mourut.
\par 42 Qu'il suffise maintenant d'avoir parlé des fêtes idolâtres et des tortures extrêmes.

\chapitre{8}

\par 1 Alors Judas Maccabée et ceux qui étaient avec lui se rendirent en secret dans les villes, et rassemblèrent leurs parents, et prirent avec eux tous ceux qui persévéraient dans la religion des Juifs, et rassemblèrent environ six mille hommes.
\par 2 Et ils invoquèrent l'Éternel, pour qu'il regarde le peuple qui a été foulé aux pieds; et aussi plaindre le temple profané par des hommes impies ;
\par 3 Et qu'il aurait compassion de la ville, très dégradée et prête à être rasée avec le sol ; et entends le sang qui criait vers lui,
\par 4 Et souvenez-vous du massacre cruel d'enfants inoffensifs et des blasphèmes commis contre son nom ; et qu'il montrerait sa haine contre les méchants.
\par 5 Or, lorsque Maccabeis était entouré de sa compagnie, les païens ne pouvaient lui résister, car la colère de l'Éternel s'est changée en miséricorde.
\par 6 C'est pourquoi il est venu à l'improviste, et a incendié des villes et des villages, et s'est emparé des lieux les plus commodes, et a vaincu et mis en fuite un grand nombre de ses ennemis.
\par 7 Mais il profita spécialement de la nuit pour de telles tentatives secrètes, de sorte que le fruit de sa sainteté se répandit partout.
\par 8 Ainsi, quand Philippe vit que cet homme grandissait peu à peu, et que ses choses prospéraient de plus en plus, il écrivit à Ptolémée, gouverneur de la Célosyrie et de la Phénicie, pour qu'il apporte davantage d'aide aux affaires du roi.
\par 9 Et aussitôt il choisit Nicanor, fils de Patrocle, un de ses amis privilégiés, et il l'envoya avec au moins vingt mille hommes de toutes nations sous sa direction, pour exterminer toute la génération des Juifs ; Il rejoignit avec lui le capitaine Gorgias, qui avait une grande expérience en matière de guerre.
\par 10 Nicanor se chargea donc de gagner autant d'argent avec les Juifs captifs que pour payer le tribut de deux mille talents que le roi devait payer aux Romains.
\par 11 C'est pourquoi il envoya aussitôt dans les villes du bord de la mer proclamer la vente des Juifs captifs et promettre qu'ils auraient quatre-vingt-dix corps pour un talent, sans s'attendre à la vengeance qui devait s'ensuivre sur lui de la part du Tout-Puissant. Dieu.
\par 12 Or, lorsque Judas fut informé de l'arrivée de Nicanor, et qu'il annonça à ceux qui étaient avec lui que l'armée était proche,
\par 13 Ceux qui avaient peur et se méfiaient de la justice de Dieu s'enfuirent et s'enfuirent.
\par 14 D'autres vendirent tout ce qui leur restait, et supplièrent l'Éternel de les délivrer, vendus par le méchant Nicanor avant qu'ils ne se réunissent :
\par 15 Et sinon pour eux-mêmes, du moins pour les alliances qu'il avait faites avec leurs pères, et à cause de son nom saint et glorieux, par lequel ils étaient appelés.
\par 16 Alors Maccabée rassembla ses hommes au nombre de six mille, et les exhorta à ne pas être frappés par la terreur de l'ennemi, ni à craindre la grande multitude des païens, qui marchaient injustement contre eux ; mais pour lutter vaillamment,
\par 17 Et de mettre sous leurs yeux le tort qu'ils avaient injustement fait au lieu saint, et la cruelle manipulation de la ville, dont ils se sont moqués, et aussi la suppression du gouvernement de leurs ancêtres :
\par 18 Car eux, dit-il, se confient en leurs armes et en leur audace ; mais notre confiance est dans le Tout-Puissant qui, d’un seul coup, peut renverser ceux qui viennent contre nous, ainsi que le monde entier.
\par 19 Et il leur raconta les secours que leurs ancêtres avaient trouvés, et comment ils furent délivrés, lorsque sous Sennachérib cent quatre-vingt-cinq mille périrent.
\par 20 Et il leur raconta la bataille qu'ils avaient eue à Babylone avec les Galates, comment ils n'étaient venus que huit mille en tout, avec quatre mille Macédoniens, et que les Macédoniens étant perplexes, les huit mille en détruisirent cent et vingt mille grâce à l'aide qu'ils reçurent du ciel, et ils reçurent ainsi un grand butin.
\par 21 Ainsi, après les avoir rendus audacieux avec ces paroles, et prêts à mourir pour la loi et pour le pays, il divisa son armée en quatre parties ;
\par 22 Et il joignit à lui ses propres frères, chefs de chaque bande, à savoir Simon, Joseph et Jonathan, donnant chacun quinze cents hommes.
\par 23 Et il chargea Éléazar de lire le livre saint ; et après leur avoir donné ce mot d'ordre : Le secours de Dieu ; lui-même dirigeant le premier groupe,
\par 24 Et avec l'aide du Tout-Puissant, ils tuèrent plus de neuf mille de leurs ennemis, et blessèrent et mutilèrent la plus grande partie de l'armée de Nicanor, et ainsi tous mirent en fuite ;
\par 25 Et ils prirent l'argent qu'ils venaient pour les acheter, et les poursuivirent au loin ; mais faute de temps, ils revinrent :
\par 26 Car c'était la veille du sabbat, et c'est pourquoi ils ne voulaient plus les poursuivre.
\par 27 Ainsi, après avoir rassemblé leurs armures et gâté leurs ennemis, ils s'occupèrent du sabbat, rendant des louanges et des remerciements extrêmes au Seigneur, qui les avait préservés jusqu'à ce jour, ce qui fut le début de la miséricorde distillant sur eux. .
\par 28 Et après le sabbat, après avoir donné une partie du butin aux estropiés, aux veuves et aux orphelins, ils partageaient le reste entre eux et leurs serviteurs.
\par 29 Lorsque cela fut fait, et qu'ils eurent fait une supplication commune, ils supplièrent le Seigneur miséricordieux de se réconcilier pour toujours avec ses serviteurs.
\par 30 De plus, parmi ceux qui étaient avec Timothée et Bacchidès qui combattaient contre eux, ils tuèrent plus de vingt mille hommes, et s'emparèrent très facilement de places élevées et fortes, et se partagèrent entre eux beaucoup plus de butin, et rendirent les mutilés, les orphelins, les veuves. oui, et les vieillards aussi, égaux en butin avec eux-mêmes.
\par 31 Et après avoir rassemblé leurs armures, ils les déposèrent toutes soigneusement dans des endroits convenables, et ils apportèrent le reste du butin à Jérusalem.
\par 32 Ils tuèrent aussi Philarque, ce méchant homme qui était avec Timothée et qui avait irrité les Juifs de bien des façons.
\par 33 Et pendant qu'ils célébraient dans leur pays la fête de la victoire, ils brûlèrent Callisthène, qui avait mis le feu aux portes saintes, et qui s'était enfui dans une petite maison ; et ainsi il reçut une récompense digne de sa méchanceté.
\par 34 Quant à ce très disgracieux Nicanor, qui avait amené mille marchands pour acheter les Juifs,
\par 35 Il a été abattu par l'aide du Seigneur, dont il tenait le moins compte ; et, ôtant ses vêtements glorieux, et congédiant sa compagnie, il vint comme un serviteur fugitif à travers le centre du pays jusqu'à Antioche, ayant un très grand déshonneur, car son armée fut détruite.
\par 36 Ainsi, celui qui s'était chargé de rendre aux Romains leur tribut au moyen de captifs à Jérusalem, disait à l'étranger que les Juifs avaient Dieu pour combattre pour eux, et que par conséquent ils ne pouvaient pas être blessés, parce qu'ils suivaient les ordres. les lois qu'il leur a données.

\chapitre{9}

\par 1 Vers cette époque, Antiochus sortit du pays de Perse avec le déshonneur.
\par 2 Car il était entré dans la ville appelée Persépolis, et il était allé de partout pour piller le temple et tenir la ville ; sur quoi la multitude courant pour se défendre avec ses armes les mit en fuite ; Et c'est ainsi qu'Antiochus, mis en fuite par les habitants, revint honteux.
\par 3 Or, lorsqu'il arriva à Ecbatane, on lui apporta la nouvelle de ce qui était arrivé à Nicanor et à Timothée.
\par 4 Puis enflé de colère. il crut venger les Juifs du déshonneur que lui avaient fait ceux qui l'avaient fait fuir. C'est pourquoi il ordonna à son conducteur de conduire sans cesse et d'expédier le voyage, le jugement de Dieu le suivant maintenant. Car il avait parlé fièrement de cette manière, qu'il viendrait à Jérusalem et en ferait un lieu de sépulture commun aux Juifs.
\par 5 Mais le Seigneur Tout-Puissant, le Dieu d'Israël, le frappa d'une plaie incurable et invisible ; ou dès qu'il eut prononcé ces paroles, une douleur d'entrailles sans remède l'atteignit, et de douloureux tourments de l'intérieur. les pièces;
\par 6 Et cela à juste titre, car il avait tourmenté les entrailles d'autres hommes par des tourments nombreux et étranges.
\par 7 Cependant il ne cessait de se vanter, mais il était toujours rempli d'orgueil, crachant du feu dans sa rage contre les Juifs, et ordonnant de hâter le voyage. Mais il arriva qu'il tomba de son char, porté violemment; de sorte qu'ayant une chute douloureuse, tous les membres de son corps furent très peinés.
\par 8 Et ainsi celui qui, un peu auparavant, pensait pouvoir dominer les vagues de la mer (tel il était fier au-delà de la condition de l'homme) et peser les hautes montagnes dans une balance, fut maintenant jeté à terre et transporté. une litière pour chevaux, montrant à tous la puissance manifeste de Dieu.
\par 9 De sorte que les vers s'élevaient du corps de ce méchant homme, et tandis qu'il vivait dans le chagrin et la douleur, sa chair tombait, et la saleté de son odeur était nuisible à toute son armée.
\par 10 Et l'homme qui avait réfléchi un peu avant de pouvoir atteindre les étoiles du ciel, aucun homme ne pouvait supporter de le porter à cause de sa puanteur intolérable.
\par 11 Ici donc, étant tourmenté, il commença à abandonner son grand orgueil et à se connaître lui-même par le fléau de Dieu, sa douleur augmentant à chaque instant.
\par 12 Et comme lui-même ne pouvait pas supporter sa propre odeur, il dit ces paroles : Il convient d'être soumis à Dieu, et qu'un homme mortel ne devrait pas penser avec fierté à lui-même s'il était Dieu.
\par 13 Ce méchant fit aussi un vœu au Seigneur, qui maintenant ne voulait plus avoir pitié de lui, disant ainsi :
\par 14 Que la ville sainte (vers laquelle il se rendait en toute hâte pour la raser de terre et en faire un lieu de sépulture commun), il la rendrait libre :
\par 15 Et quant aux Juifs, qu'il avait jugés dignes non pas tant d'être enterrés, mais d'être jetés dehors avec leurs enfants pour être dévorés par les oiseaux et les bêtes sauvages, il les rendrait tous égaux aux citoyens de Athènes:
\par 16 Et le saint temple, qu'il avait auparavant gâté, il le garnirait de beaux présents, et il restaurerait tous les objets sacrés avec beaucoup d'autres, et il paierait sur ses propres revenus les charges liées aux sacrifices.
\par 17 Oui, et qu'il deviendrait lui-même Juif, qu'il parcourrait tout le monde habité et qu'il déclarerait la puissance de Dieu.
\par 18 Mais malgré tout cela, ses souffrances ne cessèrent pas, car le juste jugement de Dieu était venu sur lui. C'est pourquoi, désespérant de sa santé, il écrivit aux Juifs la lettre souscrite, contenant la forme d'une supplication, de cette manière :
\par 19 Antiochus, roi et gouverneur, aux bons Juifs, ses citoyens souhaitent beaucoup de joie, de santé et de prospérité :
\par 20 Si vous et vos enfants vous portez bien, et si vos affaires vous conviennent, je rends de très grandes grâces à Dieu, ayant mon espérance dans le ciel.
\par 21 Quant à moi, j'étais faible, sinon je me serais souvenu avec bonté de votre honneur et de votre bonne volonté, revenant de Perse, et étant atteint d'une grave maladie, j'ai jugé nécessaire de veiller au salut commun de tous :
\par 22 Je ne me méfie pas de ma santé, mais j'ai un grand espoir d'échapper à cette maladie.
\par 23 Mais considérant que même mon père, à quelle époque il conduisait une armée dans les hauts pays. nommé un successeur,
\par 24 Afin que, si quelque chose arrivait contre toute attente, ou si quelque nouvelle malheureuse arrivait, les habitants du pays, sachant à qui l'État était laissé, ne seraient pas troublés.
\par 25 Encore une fois, considérant comment les princes qui sont frontaliers et voisins de mon royaume attendent des opportunités et attendent ce qui sera l'événement. J'ai établi roi mon fils Antiochus, que j'ai souvent confié et recommandé à beaucoup d'entre vous, lorsque je montais dans les hautes provinces ; à qui j'ai écrit ce qui suit :
\par 26 C'est pourquoi je vous prie et vous demande de vous souvenir des bienfaits que je vous ai faits en général et en particulier, et que chacun sera toujours fidèle à moi et à mon fils.
\par 27 Car je suis persuadé que celui qui comprend mon esprit se pliera favorablement et gracieusement à vos désirs.
\par 28 Ainsi, le meurtrier et le blasphémateur ayant souffert le plus cruellement, alors qu'il suppliait d'autres hommes, il mourut d'une mort misérable dans un pays étranger dans les montagnes.
\par 29 Et Philippe, qui avait été élevé avec lui, emporta son corps, qui, lui aussi, craignant le fils d'Antiochus, se rendit en Égypte chez Ptolémée Philométor.

\chapitre{10}

\par 1 Maccabée et ses compagnons, guidés par le Seigneur, récupérèrent le temple et la ville.
\par 2 Mais les autels que les païens avaient bâtis en pleine rue, ainsi que les chapelles, furent démolis.
\par 3 Et après avoir purifié le temple, ils firent un autre autel, et frappant des pierres, ils en sortirent du feu, et après deux ans ils offrirent un sacrifice, et ils présentèrent de l'encens, des luminaires et des pains de proposition.
\par 4 Lorsque cela fut fait, ils tombèrent à plat ventre et supplièrent le Seigneur de ne plus se retrouver dans de tels ennuis ; mais s'ils péchaient encore contre lui, il les châtierait lui-même avec miséricorde, et afin qu'ils ne soient pas livrés aux nations blasphématoires et barbares.
\par 5 Le jour même où les étrangers profanèrent le temple, le même jour il fut de nouveau purifié, le vingt-cinquième jour du même mois, qui est Casleu.
\par 6 Et ils célébrèrent les huit jours avec joie, comme à la fête des tabernacles, se souvenant que peu de temps auparavant ils avaient célébré la fête des tabernacles, lorsqu'ils erraient dans les montagnes et dans les tanières comme des bêtes.
\par 7 C'est pourquoi ils portèrent des branches, de belles branches, et aussi des palmiers, et ils chantèrent des psaumes à celui qui leur avait donné du succès dans la purification de son lieu.
\par 8 Ils ordonnèrent aussi par une loi et un décret communs, que chaque année ces jours soient observés par toute la nation des Juifs.
\par 9 Et ce fut la fin d'Antiochus, appelé Épiphane.
\par 10 Maintenant, nous allons raconter les actes d'Antiochus Eupator, qui était le fils de ce méchant homme, en rassemblant brièvement les calamités des guerres.
\par 11 Ainsi, lorsqu'il fut parvenu à la couronne, il chargea un certain Lysias des affaires de son royaume, et le nomma son principal gouverneur de la Célosyrie et de la Phénicie.
\par 12 Car Ptolémée, qu'on appelait Macron, choisissant plutôt de rendre justice aux Juifs pour le mal qui leur avait été fait, s'efforça de continuer la paix avec eux.
\par 13 Après quoi, accusé par les amis du roi devant Eupator, et traité de traître à chaque mot parce qu'il avait quitté Chypre, que Philométor lui avait confié, et était parti pour Antiochus Épiphane, et voyant qu'il n'était pas dans une place honorable, il fut tellement découragé qu'il s'empoisonna et mourut.
\par 14 Mais lorsque Gorgias était gouverneur des forteresses, il engageait des soldats et entretenait continuellement la guerre avec les Juifs.
\par 15 Et pendant ce temps les Iduméens, s'étant emparés des forteresses les plus commodes, occupèrent les Juifs, et recevant ceux qui étaient bannis de Jérusalem, ils allèrent alimenter la guerre.
\par 16 Alors ceux qui étaient avec Maccabée supplièrent Dieu et prièrent Dieu de les aider. Et ils coururent avec violence contre les forteresses des Iduméens,
\par 17 Et les attaquant avec force, ils gagnèrent les places fortes, et retinrent tous ceux qui combattaient contre la muraille, et tuèrent tous ceux qui tombaient entre leurs mains, et tuèrent pas moins de vingt mille.
\par 18 Et parce que certains, qui n'étaient pas moins de neuf mille, s'enfuirent ensemble dans deux châteaux très forts, ayant toutes sortes de choses convenables pour soutenir le siège,
\par 19 Maccabée quitta Simon et Joseph, ainsi que Zachée et ceux qui étaient avec lui, qui étaient en nombre suffisant pour les assiéger, et partit vers les lieux qui avaient le plus besoin de son secours.
\par 20 Or, ceux qui étaient avec Simon, poussés par la convoitise, se laissèrent persuader par l'intermédiaire de certains de ceux qui étaient dans le château, et prirent soixante-dix mille drachmes, et laissèrent s'enfuir quelques-uns d'entre eux.
\par 21 Mais Maccabée ayant été informé de ce qui s'était passé, il convoqua les chefs du peuple et accusa ces hommes d'avoir vendu leurs frères pour de l'argent et d'avoir libéré leurs ennemis pour combattre contre eux.
\par 22 Il tua donc ceux qui étaient trouvés traîtres, et s'empara aussitôt des deux châteaux.
\par 23 Et ayant eu du succès avec ses armes dans tout ce qu'il entreprenait, il en tua plus de vingt mille dans les deux forteresses.
\par 24 Or Timothée, que les Juifs avaient vaincu auparavant, après avoir rassemblé une grande multitude de forces étrangères et de nombreux chevaux venus d'Asie, vint comme s'il voulait s'emparer des Juifs par la force des armes.
\par 25 Mais quand il s'approcha, ceux qui étaient avec Maccabée se tournèrent pour prier Dieu, et aspergèrent leurs têtes de terre, et ceignirent leurs reins d'un sac,
\par 26 Et il se prosterna au pied de l'autel, et le supplia d'avoir pitié d'eux, d'être l'ennemi de leurs ennemis, et l'adversaire de leurs adversaires, comme le dit la loi.
\par 27 Après la prière, ils prirent leurs armes et s'éloignèrent de la ville ; et lorsqu'ils s'approchèrent de leurs ennemis, ils restèrent seuls.
\par 28 Or, le soleil étant nouvellement levé, ils joignirent tous deux ; l'une des parties ayant avec leur vertu leur refuge également auprès du Seigneur pour un gage de leur succès et de leur victoire : l'autre camp fait de sa rage le chef de sa bataille.
\par 29 Mais lorsque la bataille devint intense, apparurent du ciel aux ennemis cinq hommes jolis sur des chevaux, avec des brides d'or, et deux d'entre eux conduisaient les Juifs,
\par 30 Et il prit Maccabée entre eux, et le couvrit de toutes parts d'armes, et le garda en sécurité, mais il lança des flèches et des éclairs contre les ennemis ; de sorte qu'étant confondus d'aveuglement et pleins de trouble, ils furent tués.
\par 31 Et vingt mille cinq cents fantassins et six cents cavaliers furent tués.
\par 32 Quant à Timothée lui-même, il s'enfuit dans une forteresse très forte, appelée Gawra, où Chereas était gouverneur.
\par 33 Mais ceux qui étaient avec Maccabée assiégèrent courageusement la forteresse pendant quatre jours.
\par 34 Et ceux qui étaient à l'intérieur, confiants dans la force du lieu, blasphémaient excessivement et prononçaient des paroles méchantes.
\par 35 Cependant, le cinquième jour, vingt jeunes hommes de la compagnie de Maccabée, enflammés de colère à cause des blasphèmes, attaquèrent courageusement la muraille et, avec un courage féroce, tuèrent tout ce qu'ils rencontrèrent.
\par 36 D'autres montaient également après eux, pendant qu'ils s'occupaient de ceux qui étaient à l'intérieur, brûlèrent les tours, et allumèrent des feux brûlèrent vifs les blasphémateurs ; et d'autres brisèrent les portes, et, ayant reçu le reste de l'armée, prirent la ville.
\par 37 Et il tua Timothée, qui était caché dans une certaine fosse, et Chéréas, son frère, avec Apollophane.
\par 38 Après cela, ils louèrent par des psaumes et des actions de grâces l'Éternel, qui avait fait de si grandes choses pour Israël, et leur avait donné la victoire.

\chapitre{11}

\par 1 Peu de temps après, Lysias, le protecteur et cousin du roi, qui dirigeait également les affaires, fut très mécontent des choses qui se faisaient.
\par 2 Et après avoir rassemblé environ quatre-vingt mille hommes avec tous les cavaliers, il marcha contre les Juifs, pensant faire de la ville une habitation pour les païens,
\par 3 Et pour tirer profit du temple, comme des autres chapelles des païens, et pour mettre en vente le souverain sacerdoce chaque année :
\par 4 Sans considérer du tout la puissance de Dieu, mais enflé d'orgueil avec ses dix mille fantassins, ses milliers de cavaliers et ses quatre-vingts éléphants.
\par 5 Il arriva donc en Judée, et s'approcha de Bethsura, qui était une ville forte, mais éloignée d'environ cinq stades de Jérusalem, et il l'assiégea durement.
\par 6 Or, quand ceux qui étaient avec Maccabée apprirent qu'il assiégeait les forteresses, eux et tout le peuple, avec des lamentations et des larmes, supplièrent l'Éternel d'envoyer un bon ange pour délivrer Israël.
\par 7 Alors Maccabée lui-même prit le premier les armes, exhortant les autres à se risquer avec lui pour aider leurs frères. Ils partirent donc ensemble, de bonne humeur.
\par 8 Et comme ils étaient à Jérusalem, apparut devant eux un cavalier vêtu de blanc, secouant son armure d'or.
\par 9 Alors tous ensemble ils louèrent le Dieu miséricordieux et prirent courage, de sorte qu'ils étaient prêts non seulement à combattre contre les hommes, mais contre les bêtes les plus cruelles, et à percer les murs de fer.
\par 10 Ainsi ils marchèrent en armure, ayant un secours venu du ciel, car l'Éternel leur fut miséricordieux.
\par 11 Et lançant une charge sur leurs ennemis comme des lions, ils tuèrent onze mille fantassins et seize cents cavaliers, et mirent tous les autres en fuite.
\par 12 Beaucoup d'entre eux, blessés aussi, s'enfuirent nus ; Lysias lui-même s'enfuit honteusement et s'échappa ainsi.
\par 13 Lequel, comme il était un homme intelligent, rejetant en lui-même la perte qu'il avait subie, et considérant que les Hébreux ne pouvaient pas être vaincus, parce que le Dieu Tout-Puissant les aidait, il leur envoya :
\par 14 Et il les persuada d'accepter toutes les conditions raisonnables, et promit qu'il persuaderait le roi qu'il devait nécessairement être leur ami.
\par 15 Alors Maccabée consentit à tout ce que Lysias désirait, soucieux du bien commun ; et tout ce que Maccabée écrivit à Lysias concernant les Juifs, le roi l'accorda.
\par 16 Car il y avait des lettres écrites aux Juifs de Lysias à cet effet : Lysias au peuple des Juifs envoie son salut :
\par 17 Jean et Absolom, qui ont été envoyés par vous, m'ont remis la pétition souscrite et ont demandé l'exécution de son contenu.
\par 18 C'est pourquoi tout ce qui devait être rapporté au roi, je l'ai déclaré, et il l'a accordé autant qu'il était possible.
\par 19 Et si donc vous restez fidèles à l'État, désormais aussi je m'efforcerai d'être un moyen pour votre bien.
\par 20 Mais parmi les détails, j'ai donné ordre à ceux-ci et à ceux qui sont venus de moi de communier avec vous.
\par 21 Portez-vous bien. La cent huit quarantième année, le vingt-quatrième jour du mois Dioscorinthe.
\par 22 La lettre du roi contenait ces mots : Le roi Antiochus salue son frère Lysias :
\par 23 Puisque notre père a été transféré aux dieux, notre volonté est que ceux qui sont dans notre royaume vivent tranquillement, afin que chacun puisse s'occuper de ses propres affaires.
\par 24 Nous comprenons aussi que les Juifs ne consentiraient pas à ce que notre père soit amené à la coutume des païens, mais qu'ils préféraient garder leur propre manière de vivre : c'est pourquoi ils exigent de nous que nous souffrions qu'ils vivent selon leurs propres lois.
\par 25 C'est pourquoi notre pensée est que cette nation soit en repos, et nous avons décidé de lui restituer son temple, afin qu'elle vive selon les coutumes de ses ancêtres.
\par 26 Tu feras donc bien de leur envoyer des messages et de leur accorder la paix, afin qu'une fois certifiés dans notre pensée, ils puissent être bien réconfortés et vaquer toujours joyeusement à leurs propres affaires.
\par 27 Et la lettre du roi à la nation des Juifs était de cette manière : Le roi Antiochus salue le conseil et le reste des Juifs :
\par 28 Si vous vous portez bien, nous avons ce que nous désirons ; nous sommes également en bonne santé.
\par 29 Ménélans nous a déclaré que votre désir était de rentrer chez vous et de vaquer à vos affaires :
\par 30 C'est pourquoi ceux qui partiront auront un sauf-conduit jusqu'au trentième jour de Xanthicus, en toute sécurité.
\par 31 Et les Juifs utiliseront leurs propres sortes de viandes et leurs lois, comme auparavant ; et aucun d'entre eux, de quelque manière que ce soit, ne sera inquiété pour des choses faites par ignorance.
\par 32 J'ai aussi envoyé Ménélans pour qu'il vous console.
\par 33 Adieu. La cent quarante-huitième année, le quinzième jour du mois de Xanthicus.
\par 34 Les Romains leur envoyèrent aussi une lettre contenant ces mots : Quintus Memmius et Titus Manlius, ambassadeurs des Romains, saluent le peuple des Juifs.
\par 35 Tout ce que Lysias, cousin du roi, nous a accordé, nous en sommes également très satisfaits.
\par 36 Mais concernant les choses qu'il a jugé devoir être soumises au roi, après que vous en ayez avisé, envoyez-en une immédiatement, afin que nous puissions déclarer ce qui vous convient; car nous allons maintenant à Antioche.
\par 37 Envoie-en donc promptement, afin que nous sachions ce que tu penses.
\par 38 Adieu. Cette cent huit quarantième année, le quinzième jour du mois Xanthicus.

\chapitre{12}

\par 1 Lorsque ces alliances furent conclues, Lysias alla vers le roi, et les Juifs s'occupaient de leurs travaux agricoles.
\par 2 Mais parmi les gouverneurs de plusieurs localités, Timothée et Apollonius, fils de Genneus, ainsi que Hiéronymus et Démophon, et à côté d'eux Nicanor, gouverneur de Chypre, ne voulurent pas qu'ils se taisent et vivent en paix.
\par 3 Les hommes de Joppé commettèrent aussi un tel acte impie : ils prièrent les Juifs qui habitaient parmi eux de monter avec leurs femmes et leurs enfants dans les bateaux qu'ils avaient préparés, comme s'ils ne leur voulaient aucun mal.
\par 4 Ils l'acceptèrent selon le décret commun de la ville, comme désireux de vivre en paix et ne se doutant de rien ; mais étant sortis dans l'abîme, ils n'en noyèrent pas moins de deux cents.
\par 5 Lorsque Judas apprit cette cruauté commise envers ses compatriotes, il ordonna à ceux qui étaient avec lui de les préparer.
\par 6 Et invoquant Dieu le juste juge, il se lança contre les meurtriers de ses frères, et incendia le port pendant la nuit, et incendia les bateaux, et il tua ceux qui s'y enfuyaient.
\par 7 Et quand la ville fut fermée, il recula, comme s'il voulait revenir pour exterminer tous les habitants de la ville de Joppé.
\par 8 Mais quand il apprit que les Jamnites voulaient faire de même envers les Juifs qui habitaient parmi eux,
\par 9 Il rencontra aussi les Jamnites de nuit, et mit le feu au port et à la marine, de sorte que la lumière du feu fut visible à Jérusalem à deux cent quarante stades.
\par 10 Lorsqu'ils furent partis de là neuf stades dans leur voyage vers Timothée, pas moins de cinq mille hommes à pied et cinq cents cavaliers arabes se jetèrent sur lui.
\par 11 Sur quoi il y eut une bataille très douloureuse ; mais le côté de Judas, avec l'aide de Dieu, remporta la victoire ; de sorte que les Nomades d'Arabie, vaincus, demandèrent à Judas la paix, promettant à la fois de lui donner du bétail et de lui faire plaisir autrement.
\par 12 Alors Judas, pensant en effet qu'ils seraient utiles dans beaucoup de choses, leur accorda la paix ; sur quoi ils se serrèrent la main, et ils repartirent vers leurs tentes.
\par 13 Il fit aussi un pont vers une certaine ville forte, entourée de murs et habitée par des gens de divers pays ; et son nom était Caspis.
\par 14 Mais ceux qui étaient à l'intérieur avaient une telle confiance dans la solidité des murs et dans l'approvisionnement en vivres, qu'ils se comportèrent grossièrement envers ceux qui étaient avec Judas, injuriant et blasphémant, et prononçant des paroles qui ne devaient pas être prononcées.
\par 15 C'est pourquoi Judas et sa troupe, invoquant le grand Seigneur du monde, qui, sans béliers ni engins de guerre, avait renversé Jéricho au temps de Josué, lancèrent un violent assaut contre les murs,
\par 16 Et il s'empara de la ville par la volonté de Dieu, et fit des massacres indescriptibles, de telle sorte qu'un lac de deux stades de large près d'elle, étant rempli, fut vu ruisselant de sang.
\par 17 Puis ils partirent de là à sept cent cinquante stades, et arrivèrent à Characa chez les Juifs appelés Tubieni.
\par 18 Mais quant à Timothée, ils ne le trouvèrent pas sur place ; car, avant d'avoir envoyé quoi que ce soit, il partit de là, après avoir laissé dans une forteresse une garnison très forte.
\par 19 Mais Dositheus et Sosipater, qui étaient les capitaines de Maccabée, sortirent et tuèrent ceux que Timothée avait laissés dans la forteresse, soit plus de dix mille hommes.
\par 20 Et Maccabée rangea son armée par bandes, et les plaça à la tête des bandes, et partit contre Timothée, qui avait autour de lui cent vingt mille hommes d'infanterie et deux mille cinq cents cavaliers.
\par 21 Or, lorsque Timothée apprit l'arrivée de Judas, il envoya les femmes, les enfants et les autres bagages dans une forteresse appelée Carnion ; car la ville était difficile à assiéger et difficile à atteindre, à cause de la situation difficile de tous. les places.
\par 22 Mais lorsque Judas, sa première bande, apparut, les ennemis, frappés de peur et de terreur à cause de l'apparition de celui qui voit toutes choses, s'enfuirent de nouveau, l'un courant par ici, l'autre par là, de sorte qu'ils furent souvent blessés par leurs propres hommes et blessés par la pointe de leurs propres épées.
\par 23 Judas aussi les poursuivait avec beaucoup d'ardeur, tuant ces méchants misérables, dont il tua environ trente mille hommes.
\par 24 Timothée lui-même tomba entre les mains de Dosithée et de Sosipater, qu'il suppliait avec beaucoup d'astuce de le laisser partir avec sa vie, parce qu'il avait beaucoup de parents de Juifs et les frères de quelques-uns d'entre eux, qui, s'ils le voulaient, ils l'ont mis à mort, ne devraient pas être considérés.
\par 25 Alors, après leur avoir assuré par de nombreuses paroles qu'il les rendrait sains et saufs, conformément au contrat, ils le laissèrent partir pour le salut de leurs frères.
\par 26 Alors Maccabée se dirigea vers Carnion et vers le temple d'Atargatis, et là il tua vingt-cinq mille personnes.
\par 27 Et après les avoir mis en fuite et les avoir détruits, Judas emmena l'armée vers Ephron, une ville forte, où demeuraient Lysias et une grande multitude de nations diverses, et les jeunes hommes forts gardaient les murs et les défendaient puissamment. : où se trouvait également une grande provision de moteurs et de fléchettes.
\par 28 Mais lorsque Judas et ses compagnons eurent invoqué le Dieu Tout-Puissant, qui par sa puissance brise la force de ses ennemis, ils prirent la ville et tuèrent vingt-cinq mille de ceux qui étaient à l'intérieur,
\par 29 De là ils partirent pour Scythopolis, située à six cents stades de Jérusalem,
\par 30 Mais lorsque les Juifs qui habitaient là eurent témoigné que les Scythopolites les traitaient avec amour et les suppliaient avec bonté au temps de leur adversité ;
\par 31 Ils leur rendirent grâce, les priant de se montrer encore amicaux envers eux. Et ils arrivèrent à Jérusalem, la fête des semaines approchant.
\par 32 Et après la fête appelée Pentecôte, ils sortirent contre Gorgias, gouverneur de l'Idumée,
\par 33 qui sortit avec trois mille hommes d'infanterie et quatre cents cavaliers.
\par 34 Et il arriva que, dans leur combat ensemble, quelques Juifs furent tués.
\par 35 A ce moment-là, Dositheus, un homme de la compagnie de Bacenor, qui était à cheval et un homme fort, était encore sur Gorgias, et, saisissant son manteau, l'entraîna de force ; Et alors qu'il voulait prendre vivant ce maudit homme, un cavalier de Thracie venant sur lui lui frappa l'épaule, de sorte que Gorgias s'enfuit vers Marisa.
\par 36 Or, comme ceux qui étaient avec Gorgias avaient longtemps combattu et étaient fatigués, Judas invoqua l'Éternel, pour qu'il se montre pour être leur aide et leur chef dans la bataille.
\par 37 Et il commença par cela dans sa propre langue, et chanta des psaumes à haute voix, et se précipitant à l'improviste sur les hommes de Gorgias, il les mit en fuite.
\par 38 Judas rassembla donc son armée et entra dans la ville d'Odollam. Et le septième jour venu, ils se purifièrent, comme c'était la coutume, et gardèrent le sabbat au même endroit.
\par 39 Et le lendemain, comme l'usage en avait été, Judas et sa troupe vinrent ramasser les corps de ceux qui avaient été tués et les enterrer avec leurs parents dans les tombeaux de leurs pères.
\par 40 On trouva sous les vêtements de tous ceux qui furent tués des objets consacrés aux idoles des Jamnites, ce qui est interdit aux Juifs par la loi. Alors chacun comprit que c'était la raison pour laquelle ils avaient été tués.
\par 41 Tous donc louant le Seigneur, le juste juge, qui avait ouvert les choses cachées,
\par 42 Ils se mirent à la prière et le supplièrent que le péché commis soit complètement effacé du souvenir. En outre, le noble Judas exhortait le peuple à se garder du péché, car il voyait sous ses yeux les choses qui arrivaient à cause des péchés de ceux qui avaient été tués.
\par 43 Et après avoir rassemblé dans toute la troupe une somme de deux mille drachmes d'argent, il l'envoya à Jérusalem pour offrir un sacrifice d'expiation, faisant cela très bien et honnêtement, en ce sens qu'il se souvenait de la résurrection.
\par 44 Car s'il n'avait pas espéré que ceux qui avaient été tués ressusciteraient, il aurait été superflu et vain de prier pour les morts.
\par 45 Et aussi dans le fait qu'il percevait qu'il y avait une grande faveur réservée à ceux qui mouraient pieusement, c'était une pensée sainte et bonne. Sur quoi il fit une réconciliation pour les morts, afin qu'ils soient délivrés du péché.

\chapitre{13}

\par 1 La cent quarante-neuvième année, on annonça à Judas qu'Antiochus Eupator venait avec une grande puissance en Judée,
\par 2 Et avec lui Lysias, son protecteur et chef de ses affaires, ayant chacun d'eux une puissance grecque de fantassins, cent dix mille, et cinq mille trois cents cavaliers, et vingt-deux éléphants, et trois cents chars. armé de crochets.
\par 3 Ménélans se joignit également à eux, et, avec une grande dissimulation, encouragea Antiochus, non pour sauvegarder le pays, mais parce qu'il croyait avoir été nommé gouverneur.
\par 4 Mais le roi des rois excita l'esprit d'Antiochus contre ce méchant misérable, et Lysias informa le roi que cet homme était la cause de tous les malheurs, de sorte que le roi ordonna de l'amener à Bérée et de le mettre à mort. comme c'est la coutume à cet endroit.
\par 5 Or, il y avait à cet endroit une tour de cinquante coudées de haut, pleine de cendres, et elle avait un instrument rond qui pendait de tous côtés dans la cendre.
\par 6 Et quiconque était condamné pour sacrilège, ou avait commis tout autre crime grave, là tous les hommes le jetaient à mort.
\par 7 C'est par une telle mort que ce méchant homme mourut, sans avoir même été enterré dans la terre ; et cela très justement :
\par 8 Car comme il avait commis beaucoup de péchés autour de l'autel, dont le feu et les cendres étaient saintes, il reçut sa mort dans les cendres.
\par 9 Or le roi vint avec un esprit barbare et hautain pour faire aux Juifs bien pire que ce qui avait été fait du temps de son père.
\par 10 Ce que Judas comprit, il ordonna à la multitude d'invoquer le Seigneur nuit et jour, afin que si jamais à un autre moment il les secourait aussi, étant sur le point d'être éloignés de leur loi, de leur pays, et du saint temple :
\par 11 Et qu'il ne permettrait pas que le peuple, qui n'était encore qu'un peu rafraîchi, soit soumis aux nations blasphématoires.
\par 12 Ainsi, après qu'ils eurent tous fait cela ensemble, et qu'ils implorèrent le Seigneur miséricordieux en pleurant et en jeûnant, et en restant couchés à plat ventre pendant trois jours, Judas, les ayant exhortés, ordonna qu'ils soient prêts.
\par 13 Et Judas, étant à l'écart avec les anciens, résolut, avant que l'armée du roi n'entre en Judée et ne prenne la ville, de sortir et de juger l'affaire par le combat avec l'aide de l'Éternel.
\par 14 Ainsi, après avoir tout confié au Créateur du monde et exhorté ses soldats à combattre vaillamment, même jusqu'à la mort, pour les lois, le temple, la ville, le pays et la république, il campa près de Modin :
\par 15 Et ayant donné le mot d'ordre à ceux qui l'entouraient : La victoire appartient à Dieu ; Avec les jeunes hommes les plus vaillants et les plus choisis, il entra de nuit dans la tente du roi et tua dans le camp environ quatre mille hommes, ainsi que le chef des éléphants, avec tout ce qui était sur lui.
\par 16 Et enfin ils remplirent le camp de peur et de tumulte, et repartirent avec succès.
\par 17 Cela se faisait au point du jour, parce que la protection de l'Éternel l'aidait.
\par 18 Or, lorsque le roi eut goûté à la virilité des Juifs, il entreprit de prendre les possessions par politique,
\par 19 Et il marcha vers Bethsura, qui était une place forte des Juifs ; mais il fut mis en fuite, échoua et perdit ses hommes :
\par 20 Car Judas avait transmis à ceux qui s'y trouvaient ce qui était nécessaire.
\par 21 Mais Rhodocus, qui était dans l'armée des Juifs, révéla les secrets aux ennemis ; c'est pourquoi on le chercha, et après l'avoir attrapé, ils le mirent en prison.
\par 22 Le roi traita avec eux une seconde fois à Bethsum, leur donna la main, prit la leur, s'en alla, combattit Judas, et fut vaincu ;
\par 23 J'ai entendu dire que Philippe, qui restait responsable des affaires d'Antioche, était désespérément courbé, confondu, suppliait les Juifs, se soumettait et jurait à toutes les conditions égales, était d'accord avec eux, offrait des sacrifices, honorait le temple et traitait. gentiment avec l'endroit,
\par 24 Et il accepta bien Maccabée et le nomma gouverneur principal depuis Ptolémaïs jusqu'aux Gerrhéniens ;
\par 25 Nous sommes arrivés à Ptolémaïs : le peuple était attristé à cause des alliances ; car ils ont pris d'assaut, parce qu'ils voulaient annuler leurs alliances :
\par 26 Lysias monta au tribunal, dit tout ce qu'il pouvait pour défendre la cause, persuada, apaisa, les fit bien affecter, revint à Antioche. Ainsi en furent-il de l'arrivée et du départ du roi.

\chapitre{14}

\par 1 Au bout de trois ans, Judas fut informé que Démétrius, fils de Séleucus, étant entré par le port de Tripolis avec une grande puissance et une grande flotte,
\par 2 Il avait pris le pays et tué Antiochus et Lysias son protecteur.
\par 3 Or, un certain Alcimus, qui avait été grand prêtre, et qui s'était souillé volontairement au temps de leur mélange avec les païens, voyant qu'il ne pouvait en aucun cas se sauver lui-même et n'avoir plus accès au saint autel,
\par 4 La cent cinquantième année, il vint trouver le roi Démétrius, et lui présenta une couronne d'or, une palme et des branches qui étaient utilisées solennellement dans le temple. Et ce jour-là, il se tut.
\par 5 Cependant, ayant eu l'occasion de poursuivre sa folle entreprise, et étant appelé en conseil par Démétrius, et lui demandant comment les Juifs étaient affectés et quelles étaient leurs intentions, il répondit :
\par 6 Ceux des Juifs qu'il appelle Assidéens, dont le capitaine est Judas Maccabée, nourrissent la guerre et sont séditieux, et ne veulent pas laisser les autres être en paix.
\par 7 C'est pourquoi moi, étant privé de l'honneur de mes ancêtres, je veux dire du grand sacerdoce, je suis maintenant venu ici :
\par 8 Premièrement, en vérité, à cause du souci sincère que j'ai des choses qui concernent le roi ; et deuxièmement, même pour cela, j'entends le bien de mes propres compatriotes : car toute notre nation est dans une misère non négligeable à cause des agissements inconsidérés de ceux-ci mentionnés ci-dessus.
\par 9 C'est pourquoi, ô roi, puisque tu sais toutes ces choses, prends soin du pays et de notre nation, qui est pressée de toutes parts, selon la clémence que tu manifestes volontiers envers tous.
\par 10 Tant que Judas vivra, il n'est pas possible que l'État soit tranquille.
\par 11 A peine on parla de lui, que d'autres amis du roi, s'opposant malicieusement à Judas, encensèrent encore Démétrius.
\par 12 Et aussitôt appelant Nicanor, qui avait été le maître des éléphants, et l'établissant gouverneur de la Judée, il l'envoya,
\par 13 Lui ordonnant de tuer Judas, de disperser ceux qui étaient avec lui, et de faire d'Alcimus le grand prêtre du grand temple.
\par 14 Alors les païens, qui s'étaient enfuis de Judée devant Judas, arrivèrent à Nicanor en troupeaux, pensant que le mal et les calamités des Juifs étaient pour leur bien.
\par 15 Or, lorsque les Juifs apprirent l'arrivée de Nicanor et que les païens leur étaient opposés, ils jetèrent de la terre sur leurs têtes et implorèrent celui qui avait établi son peuple pour toujours et qui aide toujours sa part par la manifestation de sa présence.
\par 16 Ainsi, sur l'ordre du capitaine, ils partirent aussitôt de là et s'approchèrent d'eux à la ville de Dessau.
\par 17 Or Simon, le frère de Judas, s'était engagé dans la bataille avec Nicanor, mais il était quelque peu déconcerté par le silence soudain de ses ennemis.
\par 18 Néanmoins Nicanor, entendant parler de la virilité de ceux qui étaient avec Judas et du courage qu'ils avaient pour combattre pour leur pays, n'osa pas tenter l'affaire par l'épée.
\par 19 C'est pourquoi il envoya Posidonius, Théodote et Mattathias pour faire la paix.
\par 20 Ainsi, après avoir longuement réfléchi à ce sujet, et que le capitaine en eut informé la multitude, et il apparut qu'ils étaient tous d'accord, ils consentirent aux alliances,
\par 21 Et ils fixèrent un jour pour se réunir ensemble; et quand le jour arriva, et que des tabourets furent dressés pour l'un d'eux,
\par 22 Ludas plaça des hommes armés dans des endroits convenables, de peur que quelque trahison ne soit soudainement commise par les ennemis : ils tinrent donc une conférence pacifique.
\par 23 Or Nicanor demeura à Jérusalem et ne fit aucun mal, mais il renvoya le peuple qui venait en masse vers lui.
\par 24 Et il ne voulait pas perdre Judas de sa vue, car il aime cet homme de tout son cœur.
\par 25 Il le pria aussi de prendre une femme et d'avoir des enfants : il se maria donc, resta tranquille et participa à cette vie.
\par 26 Mais Alcimus, voyant l'amour qui les unissait et considérant les alliances qu'ils avaient faites, vint trouver Démétrius et lui dit que Nicanor n'était pas bien affectueux envers l'État ; pour cela, il avait ordonné Judas, un traître à son royaume, pour succéder au roi.
\par 27 Alors le roi, furieux et irrité par les accusations de l'homme le plus méchant, écrivit à Nicanor, lui signifiant qu'il était très mécontent des alliances, et lui ordonna d'envoyer en toute hâte Maccabée prisonnier à Antioche. .
\par 28 Lorsque Nicanor entendit cela, il fut très confus en lui-même et prit au sérieux l'idée d'annuler les articles convenus, l'homme n'étant en rien coupable.
\par 29 Mais comme il n'y avait aucune affaire contre le roi, il a pris soin de son temps pour accomplir cette chose par une politique.
\par 30 Cependant, lorsque Maccabée vit que Nicanor commençait à se montrer grossier à son égard, et qu'il le suppliait plus durement qu'à son habitude, comprenant qu'une conduite aussi amère n'était pas une bonne chose, il rassembla un grand nombre de ses hommes, et se retira de Nicanor.
\par 31 Mais l'autre, sachant qu'il était particulièrement empêché par la politique de Judas, entra dans le grand et saint temple, et ordonna aux prêtres, qui offraient leurs sacrifices habituels, de lui délivrer l'homme.
\par 32 Et quand ils jurèrent qu'ils ne sauraient dire où était l'homme qu'il cherchait,
\par 33 Il étendit sa main droite vers le temple et fit ce serment de cette manière : Si vous ne me livrez pas Judas comme prisonnier, je raserai ce temple de Dieu et je démolirai le temple. autel et ériger un temple remarquable à Bacchus.
\par 34 Après ces paroles, il s'en alla. Alors les prêtres levèrent les mains vers le ciel et supplièrent celui qui fut toujours défenseur de leur nation, en disant de cette manière :
\par 35 Toi, Seigneur de toutes choses, qui n'as besoin de rien, tu as voulu que le temple de ton habitation soit parmi nous :
\par 36 C'est pourquoi maintenant, ô saint Seigneur de toute sainteté, garde toujours intacte cette maison, qui a été récemment purifiée, et ferme toute bouche injuste.
\par 37 Or, Nicanor accusait Razis, l'un des anciens de Jérusalem, ami de ses compatriotes et homme de très bonne réputation, qui, à cause de sa bonté, était appelé père des Juifs.
\par 38 Car autrefois, alors qu'ils ne se mêlaient pas aux Gentils, il avait été accusé de judaïsme, et il avait hardiment risqué son corps et sa vie avec toute la véhémence pour la religion des Juifs.
\par 39 Alors Nicanor, voulant déclarer la haine qu'il portait aux Juifs, envoya plus de cinq cents hommes de guerre pour le prendre :
\par 40 Car il pensait qu'en le prenant, il ferait beaucoup de mal aux Juifs.
\par 41 Or, comme la foule voulait prendre la tour, et violemment enfoncer la porte extérieure, et ordonner qu'on apporte du feu pour la brûler, il était prêt à être pris de tous côtés, tomba sur son épée ;
\par 42 Préférant mourir vaillamment, plutôt que de tomber entre les mains des méchants, d'être maltraité autrement que ne convenait sa noble naissance :
\par 43 Mais manquant son coup à cause de la hâte, la multitude se précipitant aussi vers les portes, il courut hardiment jusqu'au mur et se jeta vaillamment parmi les plus épais d'entre eux.
\par 44 Mais ils cédèrent promptement, et un espace étant fait, il tomba au milieu du lieu vide.
\par 45 Néanmoins, alors qu'il avait encore un souffle en lui, enflammé de colère, il se leva ; et bien que son sang jaillisse comme des jets d'eau et que ses blessures fussent graves, il courut néanmoins au milieu de la foule ; et debout sur un rocher escarpé,
\par 46 Comme son sang était maintenant complètement disparu, il lui arracha les entrailles, les prit dans ses deux mains, les jeta sur la foule, et appelant le Seigneur de la vie et de l'esprit de lui rendre celles-ci, il ainsi décédé.

\chapitre{15}

\par 1 Mais Nicanor, apprenant que Judas et sa compagnie étaient dans les places fortes autour de Samarie, résolut sans aucun danger de les attaquer le jour du sabbat.
\par 2 Néanmoins les Juifs qui furent obligés de l'accompagner dirent : Ne détruisez pas de manière si cruelle et barbare, mais honorez ce jour que celui qui voit toutes choses a honoré de sainteté plus que tous les autres jours.
\par 3 Alors le misérable le plus ingrat demanda s'il y avait un puissant dans le ciel qui aurait ordonné que le jour du sabbat soit observé.
\par 4 Et quand ils dirent : Il y a dans les cieux un Seigneur vivant et puissant, qui a ordonné que le septième jour soit observé :
\par 5 Alors l'autre dit : Et moi aussi, je suis puissant sur terre, et j'ordonne de prendre les armes et de faire les affaires du roi. Pourtant, il a obtenu que sa mauvaise volonté ne soit pas accomplie.
\par 6 Alors Nicanor, dans un orgueil et une hauteur excessifs, résolut d'ériger un monument public de sa victoire sur Judas et sur ceux qui étaient avec lui.
\par 7 Mais Maccabée avait toujours la certitude que le Seigneur l'aiderait :
\par 8 C'est pourquoi il exhorta son peuple à ne pas craindre l'arrivée des païens contre lui, mais à se souvenir du secours qu'il avait reçu autrefois du ciel, et à attendre maintenant la victoire et le secours qui lui viendraient du ciel. Tout-Puissant.
\par 9 Et ainsi, les réconfortant par la loi et les prophètes, et en leur rappelant les batailles qu'ils avaient gagnées auparavant, il les rendit plus joyeux.
\par 10 Et après avoir éveillé leurs esprits, il leur donna leur commandement, leur montrant par là tout le mensonge des païens et la violation des serments.
\par 11 Ainsi il arma chacun d'eux, non pas tant de boucliers et de lances, que de paroles confortables et bonnes ; et en outre, il leur raconta un rêve digne d'être cru, comme s'il avait été vrai, ce qui ne les réjouit pas du tout.
\par 12 Et voici sa vision : Cet Onias, qui avait été grand prêtre, homme vertueux et bon, révérend en conversation, doux de condition, bien parlé aussi, et exercé dès l'enfance dans tous les points de vertu, soutenant ses mains priaient pour tout le corps des Juifs.
\par 13 Ceci fait, apparut de la même manière un homme aux cheveux gris et extrêmement glorieux, qui était d'une majesté merveilleuse et excellente.
\par 14 Alors Onias répondit, disant : Celui-ci est un ami des frères, qui prie beaucoup pour le peuple et pour la ville sainte, à savoir Jérémie, le prophète de Dieu.
\par 15 Sur quoi Jérémie, tendant sa main droite, donna à Judas une épée d'or, et en la donnant, il parla ainsi :
\par 16 Prends cette épée sainte, don de Dieu, avec laquelle tu blesseras les adversaires.
\par 17 Ainsi étant bien réconfortés par les paroles de Judas, qui étaient très bonnes, et capables de les inciter à la vaillance et d'encourager le cœur des jeunes gens, ils décidèrent de ne pas établir le camp, mais de partir courageusement contre eux. , et vaillamment tenter l'affaire par le conflit, parce que la ville, le sanctuaire et le temple étaient en danger.
\par 18 Car le soin qu'ils prenaient pour leurs femmes, et leurs enfants, leurs frères et leurs gens, leur importait le moins; mais la plus grande et principale crainte était pour le saint temple.
\par 19 Et ceux qui étaient dans la ville n'y prêtèrent aucune attention, étant troublés par le conflit qui se déroulait à l'extérieur.
\par 20 Et maintenant, comme tous regardaient ce que devait être l'épreuve, et que les ennemis étaient déjà proches, et que l'armée était rangée, et les bêtes convenablement placées, et les cavaliers disposés en ailes,
\par 21 Maccabée, voyant venir la multitude, les diverses préparations d'armures et la férocité des bêtes, étendit les mains vers le ciel et invoqua le Seigneur qui fait des prodiges, sachant que la victoire ne vient pas par les armes, mais même si cela lui semble bon, il le donne à ceux qui en sont dignes :
\par 22 C'est pourquoi dans sa prière il dit de cette manière : O Seigneur, tu as envoyé ton ange au temps d'Ézéchias, roi de Judée, et tu as tué dans l'armée de Sennachérib cent quatre-vingt-cinq mille :
\par 23 C'est pourquoi maintenant aussi, Seigneur du ciel, envoie un bon ange devant nous pour qu'ils soient effrayés et effrayés ;
\par 24 Et par la puissance de ton bras, que soient frappés de terreur ceux qui viennent contre ton peuple saint pour blasphémer. Et il finit ainsi.
\par 25 Alors Nicanor et ceux qui étaient avec lui s'avancèrent avec des trompettes et des chants.
\par 26 Mais Judas et sa compagnie rencontrèrent les ennemis par l'invocation et la prière.
\par 27 De sorte qu'en combattant de leurs mains, et en priant Dieu de leur cœur, ils tuèrent pas moins de trente-cinq mille hommes ; car grâce à l'apparition de Dieu, ils furent grandement réconfortés.
\par 28 Or, lorsque la bataille fut terminée, et revenant avec joie, ils savaient que Nicanor gisait mort dans son harnais.
\par 29 Alors ils poussèrent un grand cri et un grand bruit, louant le Tout-Puissant dans leur propre langue.
\par 30 Et Judas, qui fut toujours le principal défenseur des citoyens, tant dans le corps que dans l'esprit, et qui continua toute sa vie à aimer ses compatriotes, ordonna de trancher la tête de Nicanor et sa main avec son épaule, et de les amener à Jérusalem.
\par 31 Et quand il fut là, et qu'il convoqua les membres de sa nation, et qu'il plaça les prêtres devant l'autel, il fit appeler ceux qui étaient de la tour,
\par 32 Et il leur montra la tête de l'infâme Nicanor et la main de ce blasphémateur qu'il avait étendue avec une vantardise orgueilleuse contre le saint temple du Tout-Puissant.
\par 33 Et après avoir coupé la langue de cet impie Nicanor, il ordonna qu'on la donne par morceaux aux oiseaux, et qu'on suspende devant le temple la récompense de sa folie.
\par 34 Ainsi chacun loua vers le ciel le glorieux Seigneur, en disant : Béni soit celui qui a gardé sa place intacte.
\par 35 Il pendit également la tête de Nicanor à la tour, signe évident et manifeste pour tous de l'aide du Seigneur.
\par 36 Et ils ordonnèrent à tous, par un décret commun, de ne pas laisser ce jour passer sans solennité, mais de célébrer le trentième jour du douzième mois, qui en langue syrienne est appelé Adar, la veille du jour de Mardochée.
\par 37 Ainsi en fut-il avec Nicanor : et à partir de ce moment les Hébreux eurent la ville en leur pouvoir. Et c'est ici que je mettrai fin.
\par 38 Et si j'ai bien fait, et comme cela convient à l'histoire, c'est ce que je désirais : mais si j'ai bien fait, c'est ce que j'ai pu atteindre.
\par 39 Car, comme il est nuisible de boire seul du vin ou de l'eau ; et comme le vin mêlé à l'eau est agréable et ravit le goût, de même une parole finement formulée ravit les oreilles de ceux qui lisent l'histoire. Et ici ce sera la fin.

\end{document}