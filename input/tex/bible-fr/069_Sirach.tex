\begin{document}

\title{Sirach}


\chapter{1}

\par 1 Toute sagesse vient du Seigneur et est avec lui pour toujours.
\par 2 Qui peut compter le sable de la mer, les gouttes de pluie et les jours de l'éternité ?
\par 3 Qui peut connaître la hauteur du ciel, et la largeur de la terre, et l'abîme, et la sagesse ?
\par 4 La sagesse a été créée avant toutes choses, et l'intelligence de la prudence est éternelle.
\par 5 La parole de Dieu très-haut est la source de la sagesse ; et ses voies sont des commandements éternels.
\par 6 À qui la racine de la sagesse a-t-elle été révélée ? ou qui a connu ses sages conseils ?
\par 7 [À qui la connaissance de la sagesse a-t-elle été manifestée ? et qui a compris sa grande expérience ?]
\par 8 Il y a un sage et un grand redoutable, le Seigneur assis sur son trône.
\par 9 Il l'a créée, et l'a vue, et l'a dénombrée, et l'a répandue sur toutes ses œuvres.
\par 10 Elle est avec toute chair selon son don, et il l'a donnée à ceux qui l'aiment.
\par 11 La crainte du Seigneur est honneur, gloire, joie et couronne de réjouissance.
\par 12 La crainte du Seigneur réjouit le cœur, donne de la joie, de l'allégresse et une longue vie.
\par 13 Celui qui craint l'Éternel finira par se sentir bien, et il trouvera grâce au jour de sa mort.
\par 14 Craigner le Seigneur est le commencement de la sagesse : et elle a été créée avec les fidèles dans le sein maternel.
\par 15 Elle a bâti un fondement éternel avec les hommes, et elle perdurera avec leur postérité.
\par 16 Craigner l'Éternel est une plénitude de sagesse et remplit les hommes de ses fruits.
\par 17 Elle remplit toute leur maison de choses désirables, et les greniers de son revenu.
\par 18 La crainte du Seigneur est une couronne de sagesse, qui fait fleurir la paix et la santé parfaite ; qui sont tous deux des dons de Dieu : et cela élargit la joie de ceux qui l'aiment.
\par 19 La sagesse fait pleuvoir l'habileté et la connaissance de l'intelligence, et elle élève à l'honneur ceux qui la retiennent.
\par 20 La racine de la sagesse est la crainte du Seigneur, et ses branches durent une longue vie.
\par 21 La crainte du Seigneur chasse les péchés, et là où elle est présente, elle détourne la colère.
\par 22 Un homme furieux ne peut être justifié ; car l'emprise de sa fureur sera sa destruction.
\par 23 Un homme patient déchirera pendant un certain temps, et ensuite la joie lui jaillira.
\par 24 Il cachera ses paroles pour un temps, et les lèvres de plusieurs proclameront sa sagesse.
\par 25 Les paraboles de la connaissance sont dans les trésors de la sagesse ; mais la piété est une abomination au pécheur.
\par 26 Si tu désires la sagesse, garde les commandements, et le Seigneur te la donnera.
\par 27 Car la crainte de l'Éternel est sagesse et instruction, et la foi et la douceur sont ses délices.
\par 28 Ne te méfie pas de la crainte du Seigneur quand tu es pauvre, et ne viens pas à lui avec un cœur double.
\par 29 Ne sois pas hypocrite aux yeux des hommes, et prends garde à ce que tu dis.
\par 30 Ne t'exalte pas, de peur que tu ne tombes et que tu ne déshonores ton âme, et qu'ainsi Dieu ne découvre tes secrets et ne te jette au milieu de l'assemblée, parce que tu n'es pas vraiment venu à la crainte du Seigneur, mais ton cœur est plein de tromperie.

\chapter{2}

\par 1 Mon fils, si tu viens servir le Seigneur, prépare ton âme à la tentation.
\par 2 Ajuste ton cœur, et endure constamment, et ne te hâte pas dans les temps de détresse.
\par 3 Attache-toi à lui et ne t'éloigne pas, afin que tu puisses croître à ta dernière fin.
\par 4 Tout ce qui t'arrive, prends-le avec joie, et sois patient lorsque tu es transformé en un état inférieur.
\par 5 Car l'or est éprouvé dans le feu, et les hommes agréables dans la fournaise de l'adversité.
\par 6 Crois en lui, et il t'aidera ; Ordonne ton chemin et fais-lui confiance.
\par 7 Vous qui craignez le Seigneur, attendez sa miséricorde ; et ne vous écartez pas, de peur que vous ne tombiez.
\par 8 Vous qui craignez le Seigneur, croyez-le ; et ta récompense ne manquera pas.
\par 9 Vous qui craignez le Seigneur, espérez le bien, la joie et la miséricorde éternelles.
\par 10 Regardez les générations anciennes, et voyez : avez-vous déjà fait confiance au Seigneur et avez-vous été confondu ? ou est-ce que quelqu'un est resté dans sa peur et a été abandonné ? ou qui a-t-il jamais méprisé, qui l'a invoqué ?
\par 11 Car le Seigneur est plein de compassion et de miséricorde, longanimité et très pitoyable, il pardonne les péchés et sauve dans les temps d'affliction.
\par 12 Malheur aux cœurs craintifs, aux mains faibles, et au pécheur qui va dans deux directions !
\par 13 Malheur à celui qui a le cœur timide ! car il ne croit pas; c'est pourquoi il ne sera pas défendu.
\par 14 Malheur à vous qui avez perdu patience ! et que ferez-vous lorsque le Seigneur vous visitera ?
\par 15 Ceux qui craignent le Seigneur ne désobéiront pas à sa Parole ; et ceux qui l'aiment garderont ses voies.
\par 16 Ceux qui craignent l'Éternel rechercheront ce qui est bien et qui lui plaît ; et ceux qui l'aiment seront remplis de la loi.
\par 17 Ceux qui craignent l'Éternel prépareront leur cœur et humilieront leur âme devant lui,
\par 18 En disant : Nous tomberons entre les mains du Seigneur, et non entre les mains des hommes ; car telle est sa majesté, telle est sa miséricorde.

\chapter{3}

\par 1 Écoutez-moi, votre père, ô enfants, et agissez ensuite afin que vous soyez en sécurité.
\par 2 Car l'Éternel a donné honneur au père sur les enfants, et a confirmé l'autorité de la mère sur les fils.
\par 3 Celui qui honore son père fait l'expiation pour ses péchés :
\par 4 Et celui qui honore sa mère est comme celui qui amasse un trésor.
\par 5 Celui qui honore son père aura de la joie pour ses propres enfants ; et quand il fera sa prière, il sera exaucé.
\par 6 Celui qui honore son père aura une longue vie ; et celui qui obéit au Seigneur sera une consolation pour sa mère.
\par 7 Celui qui craint l'Éternel honorera son père et rendra service à ses parents, comme à ses maîtres.
\par 8 Honore ton père et ta mère en paroles et en actes, afin qu'une bénédiction vienne d'eux sur toi.
\par 9 Car la bénédiction du père établit les maisons des enfants ; mais la malédiction de la mère déracine les fondements.
\par 10 Ne te glorifie pas du déshonneur de ton père ; car le déshonneur de ton père n'est pas une gloire pour toi.
\par 11 Car la gloire d'un homme vient de l'honneur de son père ; et une mère déshonorée est un opprobre pour ses enfants.
\par 12 Mon fils, aide ton père dans son âge, et ne l'afflige pas tant qu'il vit.
\par 13 Et si son intelligence fait défaut, ayez patience envers lui ; et ne le méprise pas quand tu es dans toute ta force.
\par 14 Car le soulagement de ton père ne sera pas oublié, et au lieu des péchés, il sera ajouté pour te bâtir.
\par 15 Au jour de ton affliction, on s'en souviendra; tes péchés aussi fondront, comme la glace par beau temps chaud.
\par 16 Celui qui abandonne son père est comme un blasphémateur ; et celui qui met sa mère en colère est maudit : de Dieu.
\par 17 Mon fils, vaque à tes affaires avec douceur ; ainsi tu seras aimé de celui qui est approuvé.
\par 18 Plus tu es grand, plus humble-toi, et tu trouveras grâce devant l'Éternel.
\par 19 Beaucoup occupent des positions élevées et renommées, mais les mystères sont révélés aux humbles.
\par 20 Car la puissance du Seigneur est grande, et il est honoré parmi les humbles.
\par 21 Ne cherche pas les choses qui sont trop difficiles pour toi, et ne cherche pas ce qui est au-dessus de tes forces.
\par 22 Mais ce qui t'est commandé, réfléchis-y avec révérence, car il n'est pas nécessaire que tu voies de tes yeux les choses qui sont dans le secret.
\par 23 Ne sois pas curieux de choses inutiles ; car on te montre plus de choses que les hommes n'en comprennent.
\par 24 Car beaucoup sont trompés par leur propre vaine opinion ; et un mauvais soupçon a renversé leur jugement.
\par 25 Sans yeux, tu manqueras de lumière : ne professes donc pas la connaissance que tu ne possèdes pas.
\par 26 Un cœur obstiné finira par connaître le malheur ; et celui qui aime le danger y périra.
\par 27 Un cœur obstiné sera chargé de chagrins ; et le méchant accumulera péché sur péché.
\par 28 Il n'y a pas de remède au châtiment des orgueilleux ; car la plante du mal a pris racine en lui.
\par 29 Le cœur des prudents comprendra une parabole ; et une oreille attentive est le désir d'un homme sage.
\par 30 L'eau éteint un feu flamboyant; et l'aumône fait l'expiation des péchés.
\par 31 Et celui qui rend de bonnes actions se souvient de ce qui peut arriver dans la suite ; et quand il tombera, il trouvera un appui.

\chapter{4}

\par 1 Mon fils, ne prive pas le pauvre de sa vie, et ne fais pas attendre longtemps les yeux des nécessiteux.
\par 2 Ne rendez pas triste une âme affamée ; ni provoquer un homme dans sa détresse.
\par 3 N'ajoutez pas plus de trouble à un cœur contrarié ; et ne tarde pas à donner à celui qui est dans le besoin.
\par 4 Ne rejetez pas la supplication des affligés ; et ne détourne pas ton visage du pauvre.
\par 5 Ne détourne pas ton regard du nécessiteux, et ne lui donne aucune occasion de te maudire :
\par 6 Car s'il te maudit dans l'amertume de son âme, sa prière sera exaucée de celui qui l'a créé.
\par 7 Obtenez-vous l'amour de l'assemblée, et inclinez votre tête devant un grand homme.
\par 8 Que cela ne te chagrine pas de prêter l'oreille au pauvre, et de lui répondre amicalement avec douceur.
\par 9 Délivrez celui qui souffre de l'injustice de la main de l'oppresseur ; et ne te décourage pas lorsque tu es assis en jugement.
\par 10 Sois comme un père pour l'orphelin, et au lieu d'un mari pour leur mère ; ainsi tu seras comme le fils du Très-Haut, et il t'aimera plus que ta mère.
\par 11 La sagesse élève ses enfants et saisit ceux qui la cherchent.
\par 12 Celui qui l'aime aime la vie ; et ceux qui la chercheront de bonne heure seront remplis de joie.
\par 13 Celui qui la retient héritera de la gloire ; et partout où elle entrera, le Seigneur la bénira.
\par 14 Ceux qui la servent serviront le Saint, et ceux qui l'aiment, le Seigneur les aime.
\par 15 Celui qui l'écoute jugera les nations, et celui qui la suit habitera en sécurité.
\par 16 Si un homme se confie à elle, il en héritera ; et sa génération la possédera.
\par 17 Car au début, elle marchera avec lui par des chemins tortueux, et lui fera craindre et effroi, et le tourmentera par sa discipline, jusqu'à ce qu'elle puisse avoir confiance en son âme et l'éprouver par ses lois.
\par 18 Alors elle reviendra vers lui par le droit chemin, le consolera et lui montrera ses secrets.
\par 19 Mais s'il se trompe, elle l'abandonnera et le livrera à sa propre ruine.
\par 20 Observez l'occasion et méfiez-vous du mal ; et n'aie pas honte quand il s'agit de ton âme.
\par 21 Car il y a une honte qui amène le péché ; et il y a une honte qui est gloire et grâce.
\par 22 N'accepte personne contre ton âme, et que la révérence de personne ne te fasse tomber.
\par 23 Et ne parle pas quand il y a lieu de faire le bien, et ne cache pas ta sagesse dans sa beauté.
\par 24 Car c'est par la parole que l'on connaît la sagesse, et l'instruction par la parole de la langue.
\par 25 Ne parlez en aucune manière contre la vérité ; mais sois honteux de l'erreur de ton ignorance.
\par 26 N'aie pas honte de confesser tes péchés ; et ne forcez pas le cours du fleuve.
\par 27 Ne te rends pas l'esclave d'un homme insensé ; ni accepter la personne du puissant.
\par 28 Luttez jusqu'à la mort pour la vérité, et l'Éternel combattra pour vous.
\par 29 Ne sois pas pressé dans ta langue, et ne sois pas paresseux et négligent dans tes actions.
\par 30 Ne sois pas comme un lion dans ta maison, ni comme un frénétique parmi tes serviteurs.
\par 31 Que ta main ne soit pas étendue pour recevoir, et fermée quand tu dois rendre.

\chapter{5}

\par 1 Place ton cœur sur tes biens ; et dis non, j'en ai assez pour ma vie.
\par 2 Ne suis pas ton propre esprit et ta force, pour marcher dans les voies de ton cœur :
\par 3 Et ne dis pas : Qui me contrôlera à cause de mes œuvres ? car le Seigneur vengera sûrement ton orgueil.
\par 4 Ne dis pas : J'ai péché, et quel mal m'est-il arrivé ? car le Seigneur est patient, il ne te laissera en aucun cas partir.
\par 5 Concernant la propitiation, ne soyez pas sans crainte d'ajouter péché sur péché :
\par 6 Et ne dites pas que sa miséricorde est grande ; il sera apaisé pour la multitude de mes péchés : car la miséricorde et la colère viennent de lui, et son indignation repose sur les pécheurs.
\par 7 Ne tarde pas à te tourner vers l'Éternel, et ne tarde pas de jour en jour ; car soudain la colère de l'Éternel éclatera, et dans ta sécurité tu seras détruit et tu périras au jour de la vengeance.
\par 8 Ne concentrez pas votre cœur sur des biens injustement acquis, car ils ne vous serviront à rien au jour du malheur.
\par 9 Ne vannez pas à tout vent, et n'allez pas dans tous les chemins ; car ainsi fait le pécheur qui a une double langue.
\par 10 Sois ferme dans ton intelligence ; et que ta parole soit la même.
\par 11 Soyez prompt à entendre ; et que ta vie soit sincère ; et avec patience, répondez.
\par 12 Si tu as de l'intelligence, réponds à ton prochain ; sinon, mets ta main sur ta bouche.
\par 13 L'honneur et la honte sont dans les paroles, et la langue de l'homme est sa chute.
\par 14 Ne sois pas appelé chuchoteur, et ne te dresse pas en embuscade avec ta langue; car une honte infâme est sur le voleur, et une mauvaise condamnation pour la double langue.
\par 15 N'ignorez rien, ni dans les grandes ni dans les petites affaires.

\chapter{6}

\par 1 Au lieu d'un ami, ne devenez pas un ennemi ; car [ainsi] tu hériteras d'une mauvaise réputation, d'une honte et d'un opprobre ; de même le pécheur qui a une double langue.
\par 2 Ne t'exalte pas selon le conseil de ton propre cœur ; afin que ton âme ne soit pas déchirée comme un taureau [égaré seul.]
\par 3 Tu mangeras tes feuilles, tu perdras ton fruit, et tu te laisseras comme un arbre sec.
\par 4 Une âme méchante détruira celui qui le possède, et le fera ridiculiser ses ennemis.
\par 5 Une langue douce multipliera les amis, et une langue juste multipliera les salutations aimables.
\par 6 Soyez en paix avec plusieurs; n'ayez cependant qu'un seul conseiller sur mille.
\par 7 Si tu veux avoir un ami, éprouve-le d'abord et ne te hâte pas de lui faire confiance.
\par 8 Car quelqu'un est ami pour son propre cas, et ne demeurera pas au jour de ta détresse.
\par 9 Et il y a un ami qui, devenu inimitié et en conflit, découvrira ton opprobre.
\par 10 Encore une fois, un ami est un compagnon de table, et il ne demeurera pas au jour de ton affliction.
\par 11 Mais dans ta prospérité, il sera comme toi, et il sera audacieux envers tes serviteurs.
\par 12 Si tu es humilié, il sera contre toi et se cachera devant ta face.
\par 13 Sépare-toi de tes ennemis, et prends garde à tes amis.
\par 14 Un ami fidèle est une défense solide, et celui qui a trouvé un tel a trouvé un trésor.
\par 15 Rien ne vaut un ami fidèle, et son excellence est inestimable.
\par 16 Un ami fidèle est le remède de la vie ; et ceux qui craignent l'Éternel le trouveront.
\par 17 Celui qui craint l'Éternel dirigera correctement son amitié; car tel qu'il est, tel sera aussi son prochain.
\par 18 Mon fils, rassemble l'instruction dès ta jeunesse ; ainsi tu trouveras la sagesse jusqu'à ta vieillesse.
\par 19 Viens à elle comme celui qui laboure et sème, et attends ses bons fruits ; car tu ne te fatigueras pas beaucoup en travaillant pour elle, mais tu mangeras bientôt de ses fruits.
\par 20 Elle est très désagréable pour celui qui est ignorant : celui qui est sans intelligence ne restera pas avec elle.
\par 21 Elle reposera sur lui comme une puissante pierre d'épreuve ; et il la chassera loin de lui avant que cela ne soit long.
\par 22 Car la sagesse est selon son nom, et elle n'est pas manifeste à beaucoup.
\par 23 Prête l'oreille, mon fils, reçois mes conseils, et ne refuse pas mes conseils,
\par 24 Et mets tes pieds dans ses chaînes, et ton cou dans ses chaînes.
\par 25 Incline ton épaule, et porte-la, et ne sois pas attristé par ses liens.
\par 26 Viens à elle de tout ton cœur, et observe ses voies de toute ta puissance.
\par 27 Cherchez et cherche, et elle te sera révélée; et quand tu l'auras saisie, ne la laisse pas partir.
\par 28 Car à la fin tu trouveras son repos, et cela sera tourné vers ta joie.
\par 29 Alors ses chaînes seront pour toi une défense solide, et ses chaînes un vêtement de gloire.
\par 30 Car elle a un ornement d'or, et ses bandeaux sont en dentelle pourpre.
\par 31 Tu la revêtiras comme une robe d'honneur, et tu la mettras autour de toi comme une couronne de joie.
\par 32 Mon fils, si tu le veux, tu seras instruit ; et si tu appliques ton esprit, tu seras prudent.
\par 33 Si tu aimes entendre, tu deviendras intelligent ; et si tu tends l'oreille, tu deviendras sage,
\par 34 Tenez-vous dans la multitude des anciens ; et attache-toi à celui qui est sage.
\par 35 Soyez prêt à entendre tout discours pieux ; et que les paraboles de l'intelligence ne t'échappent pas.
\par 36 Et si tu vois un homme intelligent, approche-toi de bonne heure, et que ton pied porte les marches de sa porte.
\par 37 Que ton esprit soit tourné vers les ordonnances du Seigneur et médite continuellement sur ses commandements : il affermira ton cœur et te donnera la sagesse selon ton propre désir.

\chapter{7}

\par 1 Ne fais pas de mal, et il ne t'arrivera aucun mal.
\par 2 Éloigne-toi de l'injuste, et l'iniquité se détournera de toi.
\par 3 Mon fils, ne sème pas dans les sillons de l'injustice, et tu ne les récolteras pas sept fois.
\par 4 Ne cherchez pas la prééminence du Seigneur, ni le roi le siège d'honneur.
\par 5 ne te justifie pas devant le Seigneur ; et ne te vante pas de ta sagesse devant le roi.
\par 6 Ne cherchez pas à être juge, car vous ne pouvez pas ôter l'iniquité ; de peur que tu ne crains jamais la personne du puissant, une pierre d'achoppement sur le chemin de ta droiture.
\par 7 Ne offense pas la multitude d'une ville, et tu ne te jetteras pas au milieu du peuple.
\par 8 Ne lie pas un péché sur un autre ; car en un seul tu ne seras pas impuni.
\par 9 Ne dites pas : Dieu regardera la multitude de mes offrandes, et quand je les offrirai au Dieu Très-Haut, il l'acceptera.
\par 10 Ne sois pas découragé quand tu fais ta prière, et ne néglige pas de faire l'aumône.
\par 11 Ne vous moquez pas de l'amertume de son âme, car il y en a un qui humilie et qui élève.
\par 12 N'invente pas de mensonge contre ton frère ; et ne fais pas pareil avec ton ami.
\par 13 Ne faites aucune sorte de mensonge, car cette coutume n'est pas bonne.
\par 14 Ne parle pas beaucoup devant une multitude d'anciens, et ne bavarde pas beaucoup quand tu pries.
\par 15 Ne haïssez pas les travaux pénibles, ni les travaux agricoles que le Très-Haut a ordonnés.
\par 16 Ne te compte pas parmi la multitude des pécheurs, mais souviens-toi que la colère ne tardera pas.
\par 17 Humiliez-vous grandement, car la vengeance des impies, c'est le feu et les vers.
\par 18 Ne changez en aucun cas un ami pour un bien ; ni un frère fidèle pour l'or d'Ophir.
\par 19 Ne renoncez pas à une femme sage et bonne : car sa grâce est au-dessus de l'or.
\par 20 Tandis que ton serviteur travaille véritablement, ne lui implore pas le mal ni le mercenaire qui se donne entièrement pour toi.
\par 21 Que ton âme aime un bon serviteur, et ne lui prive pas de sa liberté.
\par 22 As-tu du bétail ? surveille-les; et s'ils sont pour ton profit, garde-les auprès de toi.
\par 23 As-tu des enfants ? instruisez-les et inclinez leur cou dès leur jeunesse.
\par 24 As-tu des filles ? prends soin de leur corps et ne te montre pas joyeux à leur égard.
\par 25 Épouse ta fille, et tu accompliras ainsi une affaire importante; mais donne-la à un homme intelligent.
\par 26 As-tu une femme selon tes pensées ? ne l'abandonne pas, mais ne t'abandonne pas à une femme légère.
\par 27 Honore ton père de tout ton cœur, et n'oublie pas les chagrins de ta mère.
\par 28 Souviens-toi que tu es né d'eux ; et comment peux-tu leur récompenser les choses qu'ils ont faites pour toi ?
\par 29 Craignez l'Éternel de toute votre âme, et révérez ses prêtres.
\par 30 Aime celui qui t'a créé de toutes tes forces, et n'abandonne pas ses ministres.
\par 31 Craignez l'Éternel et honorez le prêtre ; et donne-lui sa part, comme cela te l'est ordonné ; les prémices, et le sacrifice de culpabilité, et le don des épaules, et le sacrifice de sanctification, et les prémices des choses saintes.
\par 32 Et étends ta main vers les pauvres, afin que ta bénédiction soit parfaite.
\par 33 Un don a de la grâce aux yeux de tout homme vivant ; et ne le retiens pas pour les morts.
\par 34 Ne manquez pas d'être avec ceux qui pleurent, et ne pleurez pas avec ceux qui pleurent.
\par 35 Ne tarde pas à visiter les malades : cela te fera être aimé.
\par 36 Quoi que tu prennes en main, souviens-toi de la fin, et tu ne feras jamais de mal.

\chapter{8}

\par 1 Ne lutte pas avec un homme puissant, de peur que tu ne tombes entre ses mains.
\par 2 Ne soyez pas en désaccord avec un homme riche, de peur qu'il ne vous pèse; car l'or en a détruit beaucoup et a perverti le cœur des rois.
\par 3 Ne contestez pas avec un homme plein de langue, et n'entassez pas de bois sur son feu.
\par 4 Ne plaisante pas avec un homme grossier, de peur que tes ancêtres ne soient déshonorés.
\par 5 Ne faites pas de reproches à celui qui se détourne du péché, mais rappelez-vous que nous sommes tous dignes d'être punis.
\par 6 Ne déshonorez pas un homme dans sa vieillesse, car même certains d'entre nous vieillissent.
\par 7 Ne vous réjouissez pas de la mort de votre plus grand ennemi, mais rappelez-vous que nous mourons tous.
\par 8 Ne méprise pas les discours des sages, mais connais-toi leurs proverbes : car d'eux tu apprendras l'instruction et comment servir facilement les grands hommes.
\par 9 Ne néglige pas le discours des anciens ; car eux aussi ont appris de leurs pères, et d'eux tu apprendras l'intelligence, et tu sauras répondre selon les besoins.
\par 10 N'allume pas les braises du pécheur, de peur que tu ne sois brûlé par la flamme de son feu.
\par 11 Ne te lève pas [avec colère] en présence d'une personne injurieuse, de peur qu'elle ne t'attende pour te piéger dans tes paroles.
\par 12 Ne prête pas à celui qui est plus puissant que toi ; car si tu le prêtes, compte-le comme perdu.
\par 13 Ne te porte pas caution au-dessus de tes forces ; car si tu te porte garant, prends soin de la payer.
\par 14 N'allez pas en justice avec un juge ; car ils jugeront pour lui selon son honneur.
\par 15 Ne voyage pas en chemin avec un homme audacieux, de peur qu'il ne te soit désagréable; car il agira selon sa propre volonté, et tu périras avec lui par sa folie.
\par 16 Ne lutte pas avec un homme colérique, et n'entre pas avec lui dans un lieu solitaire ; car le sang n'est rien à ses yeux, et là où il n'y a pas de secours, il te renversera.
\par 17 Ne consultez pas un insensé ; car il ne peut pas tenir conseil.
\par 18 Ne faites rien de secret devant un étranger ; car tu ne sais pas ce qu'il enfantera.
\par 19 N'ouvre pas ton cœur à chacun, de peur qu'il ne te récompense par une ruse.

\chapter{9}

\par 1 Ne sois pas jaloux de la femme qui est dans ton sein, et ne lui donne pas de mauvaises leçons contre toi-même.
\par 2 Ne donne pas ton âme à une femme pour qu'elle pose le pied sur tes biens.
\par 3 Ne rencontre pas une prostituée, de peur que tu ne tombes dans ses pièges.
\par 4 N'utilise pas beaucoup la compagnie d'une femme chanteuse, de peur d'être séduit par ses tentatives.
\par 5 Ne regarde pas une jeune fille, afin de ne pas tomber à cause de ce qui est précieux en elle.
\par 6 Ne livre pas ton âme aux prostituées, afin de ne pas perdre ton héritage.
\par 7 Ne regarde pas autour de toi dans les rues de la ville, et ne t'erre pas dans son lieu solitaire.
\par 8 Détourne ton regard d'une belle femme, et ne regarde pas la beauté d'une autre ; car beaucoup ont été trompés par la beauté d’une femme ; car ici l'amour s'allume comme un feu.
\par 9 Ne t'assieds pas du tout avec la femme d'un autre homme, ne t'assieds pas avec elle dans tes bras, et ne dépense pas ton argent avec elle au vin ; de peur que ton cœur ne s'incline vers elle, et qu'ainsi, par ton désir, tu ne tombes dans la destruction.
\par 10 N'abandonne pas un vieil ami ; car le nouveau ne lui est pas comparable : un nouvel ami est comme du vin nouveau ; quand il sera vieux, tu le boiras avec plaisir.
\par 11 N'envie pas la gloire du pécheur, car tu ne sais pas quelle sera sa fin.
\par 12 Ne vous réjouissez pas des choses qui plaisent aux impies ; mais rappelez-vous qu'ils ne resteront pas impunis dans leur tombe.
\par 13 Garde-toi loin de l'homme qui a le pouvoir de tuer ; et si tu viens vers lui, ne fais aucune faute, de peur qu'il ne t'ôte la vie tout à l'heure : souviens-toi que tu marches au milieu des pièges et que tu marches sur les créneaux de la ville.
\par 14 Autant que tu le peux, devine ton prochain et consulte les sages.
\par 15 Que ta conversation se fasse avec les sages, et que toute ta communication soit dans la loi du Très-Haut.
\par 16 Et que les justes mangent et boivent avec toi ; et que ta gloire soit dans la crainte du Seigneur.
\par 17 L'œuvre de l'artisan sera louée, et le sage chef du peuple sera loué pour son discours.
\par 18 Un homme de mauvaise langue est dangereux dans sa ville ; et celui qui parle témérairement sera haï.

\chapter{10}

\par 1 Un juge sage instruira son peuple ; et le gouvernement d'un homme prudent est bien ordonné.
\par 2 Comme le juge du peuple est lui-même, ainsi sont ses officiers ; et quel genre d'homme est le chef de la ville, tels sont tous ceux qui l'habitent.
\par 3 Un roi imprudent détruit son peuple ; mais grâce à la prudence de ceux qui détiennent l'autorité, la ville sera habitée.
\par 4 La puissance de la terre est entre les mains de l'Éternel, et au temps voulu, il y établira celle qui sera profitable.
\par 5 Dans la main de Dieu est la prospérité de l'homme ; et il reposera son honneur sur la personne du scribe.
\par 6 Ne porte pas de haine envers ton prochain pour tout tort ; et ne faites rien du tout par des pratiques préjudiciables.
\par 7 L'orgueil est odieux devant Dieu et devant les hommes, et c'est par l'un et l'autre que l'on commet l'iniquité.
\par 8 À cause des transactions injustes, des injures et des richesses obtenues par tromperie, le royaume est transféré d'un peuple à un autre.
\par 9 Pourquoi la terre et les cendres sont-elles fières ? Il n’y a rien de plus méchant qu’un homme cupide : car un tel homme met son âme en vente ; parce que pendant qu'il vit, il jette ses entrailles.
\par 10 Le médecin coupe une longue maladie; et celui qui est aujourd'hui roi mourra demain.
\par 11 Car quand un homme est mort, il héritera des reptiles, des bêtes et des vers.
\par 12 Le commencement de l'orgueil, c'est quand quelqu'un s'éloigne de Dieu et que son cœur se détourne de son Créateur.
\par 13 Car l'orgueil est le commencement du péché, et celui qui l'a répandra des abominations. C'est pourquoi l'Éternel fit venir sur eux d'étranges calamités et les renversa complètement.
\par 14 L'Éternel a renversé les trônes des princes orgueilleux, et il a établi à leur place les humbles.
\par 15 L'Éternel a arraché les racines des nations orgueilleuses, et a planté les humbles à leur place.
\par 16 L'Éternel a renversé les pays des païens et les a détruits jusqu'aux fondations de la terre.
\par 17 Il en a enlevé quelques-uns, il les a détruits, et il a fait disparaître leur mémorial de la terre.
\par 18 L'orgueil n'a pas été fait pour les hommes, ni la colère furieuse pour ceux qui sont nés d'une femme.
\par 19 Ceux qui craignent l'Éternel sont une semence sûre, et ceux qui l'aiment sont une plante honorable ; ceux qui ne respectent pas la loi sont une semence déshonorante ; ceux qui transgressent les commandements sont une semence trompeuse.
\par 20 Parmi les frères, celui qui est le premier est honoré ; ainsi sont ceux qui craignent le Seigneur à ses yeux.
\par 21 La crainte de l'Éternel précède l'obtention de l'autorité, mais la rudesse et l'orgueil en sont la perte.
\par 22 Qu'il soit riche, noble ou pauvre, leur gloire est la crainte du Seigneur.
\par 23 Il n'est pas convenable de mépriser le pauvre qui a de l'intelligence ; il n’est pas non plus commode de magnifier un homme pécheur.
\par 24 Les grands hommes, les juges et les puissants seront honorés ; pourtant il n’y en a aucun plus grand que celui qui craint le Seigneur.
\par 25 Ceux qui sont libres rendent service au serviteur sage, et celui qui a la connaissance n'aura pas de rancune lorsqu'il sera réformé.
\par 26 Ne sois pas exagéré dans tes affaires ; et ne te vante pas au moment de ta détresse.
\par 27 Mieux vaut celui qui travaille et qui abonde en toutes choses, que celui qui se vante et manque de pain.
\par 28 Mon fils, glorifie ton âme dans la douceur, et honore-la selon sa dignité.
\par 29 Qui justifiera celui qui pèche contre sa propre âme ? et qui honorera celui qui déshonore sa propre vie ?
\par 30 Le pauvre est honoré pour son talent, et le riche est honoré pour sa richesse.
\par 31 Celui qui est honoré dans la pauvreté, combien plus dans la richesse ? et celui qui est déshonorant dans la richesse, combien plus dans la pauvreté ?

\chapter{11}

\par 1 La sagesse élève la tête de celui qui est petit et le fait asseoir parmi les grands.
\par 2 Ne félicitez pas un homme pour sa beauté ; ni détester un homme pour son apparence extérieure.
\par 3 L'abeille est petite parmi les mouches ; mais son fruit est le principal des douceurs.
\par 4 Ne te vante pas de tes vêtements et de tes vêtements, et ne t'exalte pas au jour d'honneur ; car les œuvres de l'Éternel sont merveilleuses, et ses œuvres parmi les hommes sont cachées.
\par 5 Beaucoup de rois se sont assis par terre ; et celui auquel on n'avait jamais pensé a porté la couronne.
\par 6 Beaucoup d'hommes puissants ont été grandement déshonorés ; et les honorables livrés entre les mains d'autres hommes.
\par 7 Ne blâme pas avant d'avoir examiné la vérité : comprends d'abord, puis reprends.
\par 8 Ne réponds pas avant d'avoir entendu la cause ; et n'interromps pas les hommes au milieu de leur conversation.
\par 9 Ne lutte pas dans une affaire qui ne te regarde pas ; et ne siège pas en jugement avec les pécheurs.
\par 10 Mon fils, ne te mêle pas de beaucoup de choses ; car si tu te mêles de beaucoup, tu ne seras pas innocent ; et si tu poursuis, tu n'obtiendras rien, et tu n'échapperas pas en fuyant.
\par 11 Il y en a un qui travaille, qui se donne de la peine, et qui se hâte, et qui est d'autant plus en retard.
\par 12 Il y en a encore un autre qui est lent, qui a besoin d'aide, qui manque de capacité et qui est plein de pauvreté ; pourtant l'œil du Seigneur le regarda pour le bien et le releva de son humble état,
\par 13 Et il releva la tête de misère ; afin que beaucoup de ceux qui ont vu de lui la paix sur tout le monde
\par 14 La prospérité et l'adversité, la vie et la mort, la pauvreté et la richesse, viennent du Seigneur.
\par 15 La sagesse, la connaissance et l'intelligence de la loi viennent du Seigneur : l'amour et la voie des bonnes œuvres viennent de lui.
\par 16 L'erreur et les ténèbres ont commencé avec les pécheurs, et le mal vieillira chez ceux qui s'en glorifient.
\par 17 Le don du Seigneur demeure chez les impies, et sa faveur apporte la prospérité pour toujours.
\par 18 Il y a celui qui s'enrichit par sa méfiance et ses pincements, et c'est là sa part de sa récompense :
\par 19 Attendu qu'il dit : J'ai trouvé le repos, et maintenant je mangerai continuellement de mes biens ; et pourtant il ne sait pas quel moment lui arrivera, et qu'il devra laisser ces choses à d'autres et mourir.
\par 20 Sois ferme dans ton alliance, et y sois habitué, et vieillis dans ton œuvre.
\par 21 Ne vous étonnez pas des œuvres des pécheurs ; mais confie-toi au Seigneur et demeure dans ton travail : car il est chose facile aux yeux du Seigneur d'enrichir tout d'un coup un pauvre.
\par 22 La bénédiction du Seigneur est dans la récompense de celui qui est pieux, et soudain il fait fleurir sa bénédiction.
\par 23 Ne dis pas : Quel profit y a-t-il à tirer de mon service ? et quelles bonnes choses aurai-je désormais ?
\par 24 Encore une fois, ne dites pas : J'ai assez, et je possède beaucoup de choses, et quel mal aurai-je désormais ?
\par 25 Au jour de la prospérité, on oublie l'affliction, et au jour de l'affliction, on ne se souvient plus de la prospérité.
\par 26 Car il est facile à l'Éternel, au jour de la mort, de récompenser un homme selon ses voies.
\par 27 L'affliction d'une heure fait oublier à l'homme le plaisir, et à la fin ses actions seront découvertes.
\par 28 Ne jugez personne bienheureux avant sa mort, car l'homme se fera reconnaître dans ses enfants.
\par 29 Ne fais pas entrer tout le monde dans ta maison, car l'homme trompeur a beaucoup de suites.
\par 30 Comme une perdrix prise [et gardée] dans une cage, tel est le cœur de l'orgueilleux ; et comme un espion, il surveille ta chute :
\par 31 Car il guette et change le bien en mal, et dans les choses qui en sont dignes, la louange te blâmera.
\par 32 D'une étincelle de feu un tas de charbons s'allume, et l'homme pécheur attend le sang.
\par 33 Prenez garde à l'homme méchant, car il commet le mal ; de peur qu'il ne t'attire une souillure perpétuelle.
\par 34 Reçois un étranger dans ta maison, et il te dérangera et te chassera de chez toi.

\chapter{12}

\par 1 Quand tu veux faire du bien, sache à qui tu le fais ; ainsi tu seras remercié pour tes bienfaits.
\par 2 Fais du bien à l'homme pieux, et tu trouveras une récompense ; et sinon de lui, du moins du Très-Haut.
\par 3 Il ne peut arriver aucun bien à celui qui est toujours occupé au mal, ni à celui qui ne fait aucune aumône.
\par 4 Donnez à l'homme pieux, et ne secourez pas le pécheur.
\par 5 Fais du bien à celui qui est humble, mais ne donne pas à l'impie ; retiens ton pain et ne le lui donne pas, de peur qu'il ne te domine ainsi ; car [sinon] tu recevras deux fois plus de mal pour tout le bien que tu recevras lui ai fait.
\par 6 Car le Très-Haut hait les pécheurs, et il rendra vengeance aux impies, et il les gardera jusqu'au jour puissant de leur châtiment.
\par 7 Donnez aux bons, et n'aidez pas le pécheur.
\par 8 Un ami ne peut être connu dans la prospérité, et un ennemi ne peut pas être caché dans l'adversité.
\par 9 Dans la prospérité d'un homme, les ennemis seront attristés, mais dans son adversité même un ami s'en ira.
\par 10 Ne te fie jamais à ton ennemi, car sa méchanceté est comme le fer rouille.
\par 11 Même s'il s'humilie et s'accroupit, prends cependant bien garde et prends garde à lui, et tu seras envers lui comme si tu avais essuyé un miroir, et tu sauras que sa rouille n'a pas été entièrement effacée.
\par 12 Ne le laisse pas près de toi, de peur qu'après t'avoir renversé, il ne se lève à ta place ; et qu'il ne s'assoie pas à ta droite, de peur qu'il ne cherche à s'asseoir sur toi, et qu'à la fin tu te souviennes de mes paroles et que tu en sois piqué.
\par 13 Qui aura pitié d'un charmeur mordu par un serpent, ou de quelqu'un qui s'approche des bêtes sauvages ?
\par 14 Ainsi, celui qui va vers un pécheur et qui se souille avec lui par ses péchés, qui aura pitié ?
\par 15 Il restera avec toi pendant un certain temps, mais si tu commences à tomber, il ne tardera pas.
\par 16 Un ennemi parle doucement avec ses lèvres, mais dans son cœur il imagine comment te jeter dans une fosse : il pleurera des yeux, mais s'il en trouve l'occasion, il ne se contentera pas de sang.
\par 17 Si l'adversité t'arrive, c'est là que tu le trouveras le premier ; et bien qu'il prétende t'aider, il te sapera.
\par 18 Il secouera la tête, frappera dans ses mains, murmurera beaucoup et changera de visage.

\chapter{13}

\par 1 Celui qui touche la poix en sera souillé ; et celui qui est en communion avec un homme orgueilleux lui sera semblable.
\par 2 Ne te charge pas au-dessus de tes forces pendant que tu vis ; et n'aie aucune communion avec quelqu'un qui est plus puissant et plus riche que toi : car comment la bouilloire et le pot de terre s'accordent-ils ensemble ? car si l'un est frappé l'un contre l'autre, il sera brisé.
\par 3 Le riche a fait du mal, et pourtant il menace ; le pauvre est lésé, et il doit aussi implorer.
\par 4 Si tu es à son profit, il t'utilisera ; mais si tu n'as rien, il t'abandonnera.
\par 5 Si tu as quelque chose, il vivra avec toi ; oui, il te mettra à nu et ne le regrettera pas.
\par 6 S'il a besoin de toi, il te trompera, te sourira et te mettra en espérance ; il te parlera équitablement et dira : Que veux-tu ?
\par 7 Et il te fera honte par ses viandes, jusqu'à ce qu'il t'ait séché deux ou trois fois, et à la fin il se moquera de toi par la suite, quand il te verra, il t'abandonnera et secouera la tête vers toi. .
\par 8 Prends garde à ne pas te laisser tromper et abattre dans ta gaieté.
\par 9 Si tu es invité par un homme puissant, retire-toi, et il t'invitera d'autant plus.
\par 10 Ne presse pas sur lui, de peur d'être repoussé ; ne t'éloigne pas, de peur d'être oublié.
\par 11 Ne sois pas égal à lui en paroles, et ne crois pas à ses nombreuses paroles ; car avec beaucoup de communication il te tentera, et en te souriant il révélera tes secrets.
\par 12 Mais il mettra cruellement en réserve tes paroles, et n'épargnera pas de te faire du mal et de te mettre en prison.
\par 13 Observe et prends bien garde, car tu marches en péril d'être renversé ; quand tu entends ces choses, réveille-toi dans ton sommeil.
\par 14 Aime le Seigneur toute ta vie et invoque-le pour ton salut.
\par 15 Chaque bête aime son semblable, et chacun aime son prochain.
\par 16 Toute chair se marie selon son espèce, et l'homme s'attachera à son semblable.
\par 17 Quelle communion le loup a-t-il avec l'agneau ? ainsi le pécheur avec le pieux.
\par 18 Quel accord y a-t-il entre la hyène et un chien ? et quelle paix entre les riches et les pauvres ?
\par 19 Comme l'âne sauvage est la proie du lion dans le désert, ainsi les riches dévorent les pauvres.
\par 20 Comme les orgueilleux détestent l'humilité, ainsi les riches détestent les pauvres.
\par 21 Un riche qui commence à tomber est soutenu par ses amis, mais un pauvre qui est abattu est rejeté par ses amis.
\par 22 Quand un homme riche est tombé, il a de nombreux secours : il dit des choses qui ne doivent pas être dites, et pourtant les hommes le justifient : le pauvre a glissé, et pourtant ils l'ont réprimandé aussi ; il parlait avec sagesse et ne pouvait avoir aucune place.
\par 23 Quand un homme riche parle, chacun tient sa langue, et regardez ce qu'il dit, ils l'exaltent jusqu'aux nuages ​​; mais si le pauvre parle, ils disent : Quel est cet homme ? et s'il trébuche, ils aideront à le renverser.
\par 24 La richesse est bonne à celui qui n'a pas de péché, et la pauvreté est mauvaise dans la bouche des impies.
\par 25 Le cœur d'un homme change de visage, que ce soit pour le bien ou pour le mal, et un cœur joyeux donne un visage joyeux.
\par 26 Un visage joyeux est le signe d'un cœur prospère ; et découvrir des paraboles est un travail fastidieux de l'esprit.

\chapter{14}

\par 1 Bienheureux l'homme qui n'a pas glissé avec sa bouche, et qui n'est pas piqué par la multitude des péchés.
\par 2 Bienheureux celui dont la conscience ne l'a pas condamné, et qui n'a pas perdu son espérance dans le Seigneur.
\par 3 Les richesses ne sont pas agréables à un nègre : et que devrait faire de l'argent un homme envieux ?
\par 4 Celui qui rassemble en escroquant son âme, rassemble pour autrui, qui dépensera ses biens à outrance.
\par 5 Celui qui est méchant envers lui-même, à qui sera-t-il bon ? il ne prendra pas plaisir à ses biens.
\par 6 Il n'y a rien de pire que celui qui s'envie lui-même ; et c'est une récompense de sa méchanceté.
\par 7 Et s'il fait le bien, il le fait à contrecœur ; et à la fin il déclarera sa méchanceté.
\par 8 L'homme envieux a un mauvais œil ; il détourne son visage et méprise les hommes.
\par 9 L'oeil de l'homme cupide n'est pas satisfait de sa part ; et l'iniquité du méchant dessèche son âme.
\par 10 Un mauvais œil envie son pain, et il est avare à sa table.
\par 11 Mon fils, selon tes capacités, fais-toi du bien et donne à l'Éternel son offrande qui lui est due.
\par 12 Souviens-toi que la mort ne tardera pas à venir, et que l'alliance du sépulcre ne t'est pas montrée.
\par 13 Fais du bien à ton ami avant de mourir, et selon tes capacités, étends ta main et donne-lui.
\par 14 Ne te prive pas du bon jour, et ne laisse pas la part d'un bon désir t'emporter.
\par 15 Ne confieras-tu pas tes travaux à un autre ? et que tes travaux soient partagés par le sort ?
\par 16 Donne et prends, et sanctifie ton âme ; car on ne cherche pas de friandises dans la tombe.
\par 17 Toute chair vieillit comme un vêtement ; car l'alliance depuis le commencement est : Tu mourras de mort.
\par 18 Comme parmi les feuilles vertes d'un arbre touffu, les unes tombent et les autres poussent ; ainsi en est-il de la génération de chair et de sang : l’une prend fin et une autre naît.
\par 19 Tout ouvrage pourrit et consume, et son ouvrier s'en va.
\par 20 Bienheureux l'homme qui médite les bonnes choses avec sagesse, et qui raisonne sur les choses saintes par son intelligence.
\par 21 Celui qui considère ses voies dans son cœur aura aussi de l'intelligence dans ses secrets.
\par 22 Poursuivez-la comme quelqu'un qui trace, et guettez ses voies.
\par 23 Celui qui regarde à ses fenêtres écoutera aussi à ses portes.
\par 24 Celui qui loge près de sa maison attachera aussi une épingle dans ses murs.
\par 25 Il dressera sa tente près d'elle, et logera dans un gîte où il y a de bonnes choses.
\par 26 Il placera ses enfants sous son abri et passera la nuit sous ses branches.
\par 27 Par elle il sera couvert de la chaleur, et il habitera dans sa gloire.

\chapter{15}

\par 1 Celui qui craint l'Éternel fera le bien, et celui qui a la connaissance de la loi l'obtiendra.
\par 2 Et comme une mère, elle le rencontrera, et le recevra comme une épouse mariée à une vierge.
\par 3 Elle le nourrira du pain de l'intelligence et lui donnera à boire l'eau de la sagesse.
\par 4 Il sera retenu sur elle et ne sera pas ébranlé ; et je compterai sur elle, et je ne serai pas confondu.
\par 5 Elle l'exaltera au-dessus de ses voisins, et au milieu de l'assemblée elle ouvrira sa bouche.
\par 6 Il trouvera de la joie et une couronne d'allégresse, et elle lui fera hériter d'un nom éternel.
\par 7 Mais les hommes insensés ne l'atteindront pas, et les pécheurs ne la verront pas.
\par 8 Car elle est loin de l'orgueil, et les hommes qui mentent ne peuvent se souvenir d'elle.
\par 9 La louange n'est pas convenable dans la bouche du pécheur, car elle ne lui a pas été envoyée par le Seigneur.
\par 10 Car la louange sera prononcée avec sagesse, et l'Éternel la fera prospérer.
\par 11 Ne dis pas : C'est à cause du Seigneur que je suis tombé ; car tu ne dois pas faire les choses qu'il déteste.
\par 12 Ne dis pas : Il m'a fait égarer, car il n'a pas besoin de l'homme pécheur.
\par 13 L'Éternel hait toute abomination ; et ceux qui craignent Dieu ne l'aiment pas.
\par 14 Lui-même a fait l'homme dès le commencement, et l'a laissé entre les mains de son conseil ;
\par 15 Si tu le veux, garde les commandements et pratique une fidélité acceptable.
\par 16 Il a mis du feu et de l'eau devant toi : étends ta main pour savoir si tu le veux.
\par 17 Devant l'homme il y a la vie et la mort ; et si cela lui plaît, il lui sera donné.
\par 18 Car la sagesse de l'Éternel est grande, et il est puissant en puissance, et il voit toutes choses.
\par 19 Et ses yeux sont sur ceux qui le craignent, et il connaît chaque œuvre de l'homme.
\par 20 Il n'a ordonné à personne de faire le mal, et il n'a donné à personne le droit de pécher.

\chapter{16}

\par 1 Ne désirez pas une multitude d'enfants inutiles, et ne vous réjouissez pas des fils impies.
\par 2 Même s'ils se multiplient, ne vous réjouissez pas en eux, à moins que la crainte de l'Éternel ne soit avec eux.
\par 3 Ne te fie pas à leur vie, et ne respecte pas leur multitude ; car un juste vaut mieux que mille ; et il vaut mieux mourir sans enfants que d'en avoir des impies.
\par 4 Car par quelqu'un qui a de l'intelligence la ville sera reconstituée, mais la famille des méchants sera bientôt désolée.
\par 5 J'ai vu beaucoup de choses semblables de mes yeux, et mon oreille a entendu des choses plus grandes que celles-ci.
\par 6 Dans la congrégation des impies, un feu s'allumera ; et dans une nation rebelle, la colère s'enflamme.
\par 7 Il n'était pas apaisé envers les vieux géants, qui tombaient dans la force de leur folie.
\par 8 Il n'a pas non plus épargné le lieu où séjournait Lot, mais il les a abhorrés à cause de leur orgueil.
\par 9 Il n'a pas eu pitié des gens de perdition, qui ont été emportés dans leurs péchés :
\par 10 Ni les six cent mille fantassins, qui étaient rassemblés dans l'endurcissement de leur cœur.
\par 11 Et s'il y a parmi le peuple quelqu'un au cou raide, il est étonnant qu'il échappe à la punition ; car la miséricorde et la colère sont avec lui ; il est puissant pour pardonner et pour exprimer son mécontentement.
\par 12 Autant sa miséricorde est grande, autant sa correction l'est aussi : il juge un homme selon ses œuvres.
\par 13 Le pécheur n'échappera pas avec son butin, et la patience des hommes pieux ne sera pas frustrée.
\par 14 Faites place à toute œuvre de miséricorde, car chacun trouvera selon ses œuvres.
\par 15 L'Éternel a endurci Pharaon, afin qu'il ne le connaisse pas, afin que ses œuvres puissantes soient connues du monde.
\par 16 Sa miséricorde est manifeste envers toute créature ; et il a séparé sa lumière des ténèbres avec une fermeté.
\par 17 Ne dis pas : Je me cacherai loin du Seigneur ; quelqu'un se souviendra-t-il de moi d'en haut ? On ne se souviendra pas de moi parmi tant de gens : car qu'est mon âme parmi un nombre si infini de créatures ?
\par 18 Voici, les cieux et les cieux des cieux, l'abîme et la terre, et tout ce qu'ils contiennent, seront ébranlés lorsqu'il les visitera.
\par 19 Les montagnes aussi et les fondements de la terre seront ébranlés de tremblement, quand l'Éternel les regardera.
\par 20 Aucun cœur ne peut penser dignement à ces choses ; et qui est capable de concevoir ses voies ?
\par 21 C'est une tempête que personne ne peut voir : car la plupart de ses œuvres sont cachées.
\par 22 Qui peut déclarer les œuvres de sa justice ? ou qui peut les supporter ? car son alliance est lointaine, et l'épreuve de toutes choses est à la fin.
\par 23 Celui qui manque d'intelligence pensera à des choses vaines, et l'homme insensé qui s'égare imagine des folies.
\par 24 Mon fils, écoute-moi, apprends la connaissance, et marque mes paroles dans ton cœur.
\par 25 Je présenterai la doctrine en termes de poids, et je déclarerai sa connaissance avec exactitude.
\par 26 Les œuvres de l'Éternel sont faites en jugement dès le commencement ; et dès le moment où il les a faites, il en a disposé les parties.
\par 27 Il a garni ses œuvres pour toujours, et entre ses mains sont les principaux d'entre eux pour toutes les générations : ils ne travaillent ni ne se lassent, et ne cessent pas de leurs œuvres.
\par 28 Aucun d'eux ne gêne l'autre, et ils ne désobéiront jamais à sa parole.
\par 29 Après cela, le Seigneur regarda la terre et la remplit de ses bénédictions.
\par 30 Il en a couvert la face de toutes sortes d'êtres vivants ; et ils y retourneront.

\chapter{17}

\par 1 Le Seigneur a créé l'homme de la terre et l'a transformé à nouveau en elle.
\par 2 Il leur a donné peu de jours et un peu de temps, et aussi le pouvoir sur les choses qui s'y trouvent.
\par 3 Il les a dotés de force par eux-mêmes, et les a faits selon son image,
\par 4 Et il mit la crainte de l'homme sur toute chair, et lui donna la domination sur les bêtes et les oiseaux.
\par 5 Ils reçurent l'usage des cinq opérations du Seigneur, et en sixième lieu il leur donna la compréhension, et dans le septième discours, un interprète de leurs cogitations.
\par 6 Un conseil, et une langue, et des yeux, des oreilles et un cœur, leur donnèrent à comprendre.
\par 7 En outre, il les remplit de la connaissance de l'intelligence, et leur montra le bien et le mal.
\par 8 Il posa les yeux sur leurs cœurs, afin de leur montrer la grandeur de ses œuvres.
\par 9 Il leur a donné pour toujours la gloire de ses actes merveilleux, afin qu'ils puissent raconter ses œuvres avec intelligence.
\par 10 Et les élus loueront son saint nom.
\par 11 En outre, il leur a donné la connaissance et la loi de la vie en héritage.
\par 12 Il a conclu avec eux une alliance éternelle et leur a montré ses jugements.
\par 13 Leurs yeux virent la majesté de sa gloire, et leurs oreilles entendirent sa voix glorieuse.
\par 14 Et il leur dit : Gardez-vous de toute injustice ; et il donna à chacun des commandements concernant son prochain.
\par 15 Leurs voies sont toujours devant lui et ne seront pas cachées à ses yeux.
\par 16 Tout homme dès sa jeunesse est livré au mal ; ils ne pouvaient pas non plus se faire des cœurs de chair pour des cœurs de pierre.
\par 17 Car lors de la division des nations de toute la terre, il a établi un chef sur chaque peuple ; mais Israël est la part de l'Éternel :
\par 18 Lui, étant son premier-né, il le nourrit avec discipline, et en lui donnant la lumière de son amour, il ne l'abandonne pas.
\par 19 C'est pourquoi toutes leurs œuvres sont comme le soleil devant lui, et ses yeux sont continuellement sur leurs voies.
\par 20 Aucune de leurs actions injustes ne lui est cachée, mais tous leurs péchés sont devant l'Éternel.
\par 21 Mais l'Eternel, étant miséricordieux et connaissant son ouvrage, ne les abandonna ni ne les abandonna, mais il les épargna.
\par 22 L'aumône d'un homme est comme un sceau pour lui, et il gardera les bonnes actions de l'homme comme la prunelle de ses yeux, et donnera la repentance à ses fils et à ses filles.
\par 23 Ensuite, il se lèvera et les récompensera, et il rendra leur récompense sur leurs têtes.
\par 24 Mais à ceux qui se repentent, il leur a accordé le retour, et il a consolé ceux qui manquaient de patience.
\par 25 Retourne au Seigneur, et abandonne tes péchés, fais ta prière devant sa face, et offense moins.
\par 26 Tourne-toi encore vers le Très-Haut, et détourne-toi de l'iniquité ; car il te fera sortir des ténèbres vers la lumière de la santé, et il te détestera avec véhémence.
\par 27 Qui louera le Très-Haut dans le tombeau, à la place de ceux qui vivent et rendent grâces ?
\par 28 L'action de grâce périt d'entre les morts, comme de celui qui n'est pas : les vivants et les sains de cœur loueront l'Éternel.
\par 29 Quelle est la bonté du Seigneur notre Dieu et sa compassion envers ceux qui se tournent vers lui dans la sainteté !
\par 30 Car tout ne peut être dans les hommes, parce que le fils de l'homme n'est pas immortel.
\par 31 Qu'est-ce qui est plus brillant que le soleil ? pourtant sa lumière faiblit ; et la chair et le sang imagineront le mal.
\par 32 Il voit la puissance de la hauteur du ciel ; et tous les hommes ne sont que terre et cendres.

\chapter{18}

\par 1 Celui qui vit éternellement a créé toutes choses en général.
\par 2 Le Seigneur seul est juste, et il n'y a d'autre que lui,
\par 3 Qui gouverne le monde avec la paume de sa main, et toutes choses obéissent à sa volonté ; car il est le Roi de tous, par sa puissance séparant entre eux les choses saintes des profanes.
\par 4 À qui a-t-il donné le pouvoir de déclarer ses œuvres ? et qui découvrira ses nobles actes ?
\par 5 Qui comptera la force de sa majesté ? et qui déclarera aussi ses miséricordes ?
\par 6 Quant aux œuvres merveilleuses du Seigneur, rien ne peut leur être enlevé, rien ne peut leur être apporté, et leur fondement ne peut pas non plus être découvert.
\par 7 Quand un homme a fait, alors il commence ; et quand il s'arrêtera, alors il doutera.
\par 8 Qu'est-ce que l'homme, et à quoi sert-il ? quel est son bien et quel est son mal ?
\par 9 Le nombre des jours d'un homme est au maximum de cent ans.
\par 10 Comme une goutte d'eau dans la mer, et comme un gravier en comparaison du sable ; ainsi sont mille ans jusqu'aux jours de l'éternité.
\par 11 C'est pourquoi Dieu est patient avec eux et répand sur eux sa miséricorde.
\par 12 Il vit et comprit que leur fin était mauvaise ; c'est pourquoi il multiplia sa compassion.
\par 13 La miséricorde de l'homme est envers son prochain ; mais la miséricorde du Seigneur est sur toute chair : il reprend, nourrit, enseigne et ramène, comme un berger son troupeau.
\par 14 Il a pitié de ceux qui reçoivent la discipline et qui recherchent diligemment ses jugements.
\par 15 Mon fils, ne ternis pas tes bonnes actions, et ne prononce pas de paroles inconfortables quand tu donnes quelque chose.
\par 16 La rosée ne remplacera-t-elle pas la chaleur ? un mot vaut donc mieux qu'un cadeau.
\par 17 Voici, une parole ne vaut-elle pas mieux qu'un don ? mais tous deux sont avec un homme aimable.
\par 18 L'insensé fait des reproches grossiers, et le don de l'envieux consume les yeux.
\par 19 Apprenez avant de parler, et utilisez des médicaments, sinon vous serez malade.
\par 20 Avant le jugement, examine-toi, et au jour de la visite tu trouveras miséricorde.
\par 21 Humiliez-vous avant de tomber malade, et au temps des péchés, montrez-vous repentant.
\par 22 Que rien ne t'empêche d'accomplir ton vœu en temps voulu, et n'attends pas la mort pour être justifié.
\par 23 Avant de prier, prépare-toi ; et ne soyez pas comme celui qui tente le Seigneur.
\par 24 Pensez à la colère qui sera à la fin, et au temps de la vengeance, quand il détournera son visage.
\par 25 Quand tu en as assez, souviens-toi du temps de la faim ; et quand tu es riche, pense à la pauvreté et au besoin.
\par 26 Depuis le matin jusqu'au soir, l'heure change, et tout s'accomplit bientôt devant le Seigneur.
\par 27 Le sage aura peur en toute chose, et au jour du péché il se gardera de l'offense ; mais l'insensé n'observera pas le temps.
\par 28 Tout homme intelligent connaît la sagesse et louera celui qui l'a trouvée.
\par 29 Ceux qui avaient de l'intelligence dans les paroles devinrent eux-mêmes sages et déclamèrent des paraboles exquises.
\par 30 Ne poursuis pas tes convoitises, mais abstiens-toi de tes appétits.
\par 31 Si tu donnes à ton âme les désirs qui lui plaisent, elle fera de toi la risée de tes ennemis qui te calomnient.
\par 32 Ne prenez pas plaisir à beaucoup de bonne chère, et ne soyez pas lié aux dépenses qui en découlent.
\par 33 Ne deviens pas mendiant en faisant un festin après avoir emprunté, alors que tu n'as rien dans ta bourse; car tu guetteras ta propre vie et tu attireras l'attention.

\chapter{19}

\par 1 Un homme qui travaille et qui s'adonne à l'ivresse ne sera pas riche, et celui qui méprise les petites choses tombera peu à peu.
\par 2 Le vin et les femmes feront tomber les hommes intelligents, et celui qui s'attache aux prostituées deviendra impudent.
\par 3 Les mites et les vers l'auront en héritage, et l'homme audacieux sera enlevé.
\par 4 Celui qui s'empresse d'accorder du crédit est léger d'esprit ; et celui qui pèche offensera sa propre âme.
\par 5 Celui qui prend plaisir au mal sera condamné, mais celui qui résiste aux plaisirs couronne sa vie.
\par 6 Celui qui sait gouverner sa langue vivra sans querelle ; et celui qui déteste le bavardage aura moins de mal.
\par 7 Ne répète pas à un autre ce qu'on te dit, et tu ne t'en sortiras jamais plus mal.
\par 8 Que ce soit à un ami ou à un ennemi, ne parlez pas de la vie des autres hommes ; et si tu le peux sans offense, ne les révèle pas.
\par 9 Car il t'a entendu et observé, et le moment venu, il te haïra.
\par 10 Si tu as entendu une parole, qu'elle meure avec toi ; et sois audacieux, il ne t'éclatera pas.
\par 11 L'insensé travaille avec une parole, comme une femme qui enfante.
\par 12 Comme une flèche qui s'enfonce dans la cuisse d'un homme, ainsi est une parole dans le ventre d'un insensé.
\par 13 Réprimandez un ami, il se peut qu'il ne l'ait pas fait, et s'il l'a fait, qu'il ne le fasse plus.
\par 14 Réprimande ton ami, il se peut qu'il ne l'ait pas dit ; et s'il l'a fait, qu'il ne le répète pas.
\par 15 Réprimandez un ami : car bien souvent c'est une calomnie, et ne croyez pas toutes les histoires.
\par 16 Il y en a un qui glisse dans ses paroles, mais non dans son cœur ; et qui est celui qui n'a pas offensé par sa langue ?
\par 17 Réprimande ton prochain avant de le menacer ; et sans vous mettre en colère, abandonnez-vous à la loi du Très-Haut.
\par 18 La crainte du Seigneur est le premier pas pour être accepté [de lui], et la sagesse obtient son amour.
\par 19 La connaissance des commandements du Seigneur est la doctrine de la vie : et ceux qui font les choses qui lui plaisent recevront le fruit de l'arbre de l'immortalité.
\par 20 La crainte du Seigneur est toute sagesse ; et en toute sagesse réside l'accomplissement de la loi et la connaissance de sa toute-puissance.
\par 21 Si un serviteur dit à son maître : Je ne ferai pas ce qu'il te plaira ; même s'il le fait ensuite, il met en colère celui qui le nourrit.
\par 22 La connaissance de la méchanceté n'est pas la sagesse, ni à aucun moment le conseil des pécheurs la prudence.
\par 23 Il y a une méchanceté, et celle-là est une abomination ; et il y a un insensé qui manque de sagesse.
\par 24 Celui qui a peu d'intelligence et qui craint Dieu vaut mieux que celui qui a beaucoup de sagesse et qui transgresse la loi du Très-Haut.
\par 25 Il y a une subtilité exquise, et celle-là est injuste ; et il y en a un qui se détourne pour faire apparaître le jugement ; et il y a un homme sage qui justifie dans son jugement.
\par 26 Il y a un méchant homme qui penche tristement la tête; mais intérieurement il est plein de tromperie,
\par 27 Baissant son visage et faisant comme s'il n'avait pas entendu : là où il n'est pas connu, il te fera du mal avant que tu t'en aperçoives.
\par 28 Et si, faute de pouvoir, il est empêché de pécher, quand il en trouve l'occasion, il fera le mal.
\par 29 Un homme peut être reconnu par son regard, et celui qui a de l'intelligence par son visage, quand tu le rencontres.
\par 30 L'habillement d'un homme, ses rires excessifs et sa démarche montrent ce qu'il est.

\chapter{20}

\par 1 Il y a une réprimande qui n'est pas convenable ; encore une fois, quelqu'un tient sa langue, et il est sage.
\par 2 Il vaut bien mieux reprendre que se mettre en colère en secret ; et celui qui confesse sa faute sera préservé du mal.
\par 3 Qu'il est bon, quand tu es repris, de faire preuve de repentance ! car ainsi tu échapperas au péché volontaire.
\par 4 Comme le désir d'un eunuque de déflorer une vierge ; ainsi en est-il de celui qui exécute le jugement avec violence.
\par 5 Il y en a un qui garde le silence et est trouvé sage, et un autre devient odieux à force de bavardages.
\par 6 Certains se taisent parce qu'ils n'ont pas à répondre, et d'autres se taisent, connaissant leur heure.
\par 7 Le sage tiendra sa langue jusqu'à ce qu'il voie l'occasion, mais le bavard et l'insensé ne s'intéresseront pas au temps.
\par 8 Celui qui parle beaucoup sera abhorré ; et celui qui s'en attribue l'autorité sera haï.
\par 9 Il y a un pécheur qui réussit bien dans les mauvaises choses ; et il y a un gain qui se transforme en perte.
\par 10 Il y a un don qui ne te profitera pas ; et il y a un don dont la récompense est double.
\par 11 Il y a un abaissement à cause de la gloire ; et il y en a qui relève la tête d'un état bas.
\par 12 Il y a celui qui achète beaucoup pour peu et qui rend sept fois plus.
\par 13 Le sage se fait aimer par ses paroles, mais les grâces des insensés se répandent.
\par 14 Le don d'un insensé ne te servira à rien quand tu l'auras ; ni encore de l'envieux pour sa nécessité : car il espère recevoir beaucoup de choses pour une seule.
\par 15 Il donne peu et fait beaucoup de reproches ; il ouvre la bouche comme un crieur ; aujourd'hui il prête, et demain il le demandera encore : un tel homme doit être haï de Dieu et des hommes.
\par 16 L'insensé dit : Je n'ai pas d'amis, je n'ai aucune reconnaissance pour toutes mes bonnes actions, et ceux qui mangent mon pain disent du mal de moi.
\par 17 Combien de fois et combien de personnes se moqueront-ils de lui ! car il ne sait pas bien ce que c'est qu'avoir ; et tout est un pour lui, comme s'il ne l'avait pas.
\par 18 Mieux vaut glisser sur un pavé que glisser avec la langue : ainsi la chute des méchants viendra bientôt.
\par 19 Une histoire inopportune sera toujours dans la bouche des imprudents.
\par 20 Une sentence sage sera rejetée quand elle sort de la bouche d'un insensé ; car il ne le dira pas au temps convenable.
\par 21 Il y a celui qui est empêché de pécher à cause du besoin ; et quand il prend du repos, il ne sera pas troublé.
\par 22 Il y en a qui détruit sa propre âme par pudeur, et qui se renverse en acceptant des personnes.
\par 23 Il y a celui qui, par pudeur, promet à son ami et en fait son ennemi pour rien.
\par 24 Le mensonge est une tache répugnante chez l'homme, mais il est continuellement dans la bouche de celui qui n'est pas instruit.
\par 25 Mieux vaut un voleur qu'un homme habitué à mentir; mais tous deux auront la destruction en héritage.
\par 26 Le tempérament du menteur est déshonorant, et sa honte est toujours avec lui.
\par 27 L'homme sage se fera honneur par ses paroles, et celui qui a de l'intelligence plaira aux grands.
\par 28 Celui qui cultive sa terre augmentera son monceau, et celui qui plaît aux grands obtiendra le pardon de son iniquité.
\par 29 Les cadeaux et les offrandes aveuglent les yeux du sage, et bouchent sa bouche pour qu'il ne puisse pas reprendre.
\par 30 La sagesse cachée et le trésor thésaurisé, quel profit y a-t-il à l'une et à l'autre ?
\par 31 Mieux vaut celui qui cache sa folie qu'un homme qui cache sa sagesse.
\par 32 La patience nécessaire à la recherche du Seigneur vaut mieux que celui qui mène sa vie sans guide.

\chapter{21}

\par 1 Mon fils, as-tu péché ? ne le fais plus, mais demande pardon pour tes péchés antérieurs.
\par 2 Fuis le péché comme devant la face d'un serpent ; car si tu t'en approches trop, il te mordra ; ses dents sont comme les dents d'un lion, tuant les âmes des hommes.
\par 3 Toute iniquité est comme une épée à deux tranchants dont les blessures ne peuvent être guéries.
\par 4 Terrifier et faire le mal gaspillera les richesses ; ainsi la maison des hommes orgueilleux sera désolée.
\par 5 Une prière de la bouche d'un pauvre parvient aux oreilles de Dieu, et son jugement arrive promptement.
\par 6 Celui qui déteste être repris est dans la voie des pécheurs, mais celui qui craint l'Éternel se repentira de tout son cœur.
\par 7 Un homme éloquent est connu de loin et de près ; mais l'homme intelligent sait quand il glisse.
\par 8 Celui qui bâtit sa maison avec l'argent d'autrui est comme celui qui rassemble lui-même des pierres pour son tombeau.
\par 9 L'assemblée des méchants est comme de l'étoupe enveloppée; et leur extrémité est une flamme de feu pour les détruire.
\par 10 Le chemin des pécheurs est tracé par des pierres, mais au bout se trouve la fosse de l'enfer.
\par 11 Celui qui garde la loi du Seigneur en acquiert la compréhension, et la perfection de la crainte du Seigneur, c'est la sagesse.
\par 12 Celui qui n'est pas sage ne sera pas instruit ; mais il y a une sagesse qui multiplie l'amertume.
\par 13 La connaissance du sage abonde comme un flot, et ses conseils sont comme une pure source de vie.
\par 14 Les entrailles de l'insensé sont comme un vase brisé, et il ne détiendra aucune connaissance tant qu'il vivra.
\par 15 Si un homme habile entend une parole sage, il la recommandera et y ajoutera ; mais dès que quelqu'un qui n'a pas d'intelligence l'entend, cela lui déplaît, et il la jette derrière son dos.
\par 16 Les paroles de l'insensé sont comme un fardeau sur le chemin, mais la grâce se trouvera sur les lèvres du sage.
\par 17 Ils s'enquièrent de la bouche du sage dans l'assemblée, et ils méditeront ses paroles dans leur cœur.
\par 18 Telle est une maison détruite, telle est la sagesse pour un insensé, et la connaissance de l'insensé est comme un discours insensé.
\par 19 La doctrine est pour les insensés comme des chaînes aux pieds, et comme des menottes à la main droite.
\par 20 L'insensé élève la voix en riant ; mais un homme sage sourit à peine un peu.
\par 21 La science est pour le sage comme un ornement d'or et comme un bracelet à son bras droit.
\par 22 Le pied de l'homme insensé est bientôt dans la maison de son voisin, mais l'homme d'expérience a honte de lui.
\par 23 L'insensé jette un coup d'oeil à la porte de la maison, mais celui qui est bien nourri reste dehors.
\par 24 C'est une impolitesse pour un homme que d'écouter à la porte, mais un homme sage sera attristé par la honte.
\par 25 Les lèvres de ceux qui parlent disent des choses qui ne les regardent pas, mais les paroles de ceux qui ont de l'intelligence sont pesées dans la balance.
\par 26 Le coeur des insensés est dans leur bouche, mais la bouche des sages est dans leur coeur.
\par 27 Quand l'impie maudit Satan, il maudit sa propre âme.
\par 28 Celui qui murmure souille son âme et est haï partout où il habite.

\chapter{22}

\par 1 L'homme paresseux est comparé à une pierre sale, et chacun le sifflera à sa disgrâce.
\par 2 Un homme paresseux est comparé à la saleté d'un fumier : quiconque la ramasse lui serrera la main.
\par 3 Un homme mal nourri est le déshonneur de son père qui l'a engendré, et une fille [insensée] naît pour sa perte.
\par 4 Une fille sage apportera un héritage à son mari; mais celle qui vit dans l'inhonnêteté est la lourdeur de son père.
\par 5 Celle qui est audacieuse déshonore son père et son mari, mais tous deux la mépriseront.
\par 6 Un conte hors de propos [est comme] une musique dans un deuil ; mais les châtiments et la correction de la sagesse ne sont jamais hors de propos.
\par 7 Celui qui enseigne un insensé est comme celui qui colle un tesson, et comme celui qui réveille quelqu'un d'un profond sommeil.
\par 8 Celui qui raconte une histoire à un insensé parle à celui qui est endormi : quand il aura raconté son histoire, il dira : Qu'y a-t-il ?
\par 9 Si les enfants vivent honnêtement et ont de quoi, ils couvriront la bassesse de leurs parents.
\par 10 Mais les enfants, étant hautains, par dédain et manque d'éducation, souillent la noblesse de leurs parents.
\par 11 Pleurez sur le mort, car il a perdu la lumière ; et pleurez sur l'insensé, car il manque de compréhension ; pleurez peu sur le mort, car il est en repos ; mais la vie de l'insensé est pire que la mort.
\par 12 Sept jours on pleure le mort ; mais pour un homme insensé et impie, tous les jours de sa vie.
\par 13 Ne parle pas beaucoup avec un insensé, et n'allez pas vers celui qui n'a pas d'intelligence. Prends garde à lui, de peur que tu n'aies des ennuis, et tu ne seras jamais souillé par ses folies. Éloigne-toi de lui, et tu trouveras du repos, et ne vous inquiétez jamais de la folie.
\par 14 Qu'est-ce qui est plus lourd que le plomb ? et quel est son nom, sinon un insensé ?
\par 15 Le sable, le sel et une masse de fer sont plus faciles à supporter qu'un homme sans intelligence.
\par 16 De même que les bois ceints et liés dans un bâtiment ne peuvent être détachés en étant secoués : ainsi le cœur qui est affermi par un conseil avisé ne doit jamais craindre.
\par 17 Un cœur fixé sur une pensée de compréhension est comme un beau plâtre sur le mur d'une galerie.
\par 18 Les pâles placés sur un lieu élevé ne résisteront jamais au vent ; ainsi un cœur craintif dans l'imagination d'un insensé ne peut résister à aucune peur.
\par 19 Celui qui pique l'œil fera couler des larmes, et celui qui pique le cœur le fera montrer sa connaissance.
\par 20 Celui qui jette une pierre aux oiseaux les effraie, et celui qui reproche à son ami rompt son amitié.
\par 21 Même si tu tires l'épée contre ton ami, ne désespère pas, car il peut y avoir un retour [en faveur .]
\par 22 Si tu as ouvert la bouche contre ton ami, ne crains rien ; car il peut y avoir une réconciliation : sauf pour les reproches, ou l'orgueil, ou la révélation de secrets, ou une blessure perfide : car pour ces choses tout ami partira.
\par 23 Sois fidèle à ton prochain dans sa pauvreté, afin que tu puisses te réjouir dans sa prospérité ; demeure ferme envers lui au temps de sa détresse, afin que tu sois héritier avec lui dans son héritage ; car un domaine médiocre n'est pas toujours à faire. être méprisé, ni les riches qu'il est insensé de susciter en admiration.
\par 24 Comme la vapeur et la fumée d'une fournaise marchent devant le feu ; si injurieux avant le sang.
\par 25 Je n'aurai pas honte de défendre un ami ; je ne me cacherai pas non plus de lui.
\par 26 Et s'il m'arrive quelque malheur de sa part, quiconque l'entendra se méfiera de lui.
\par 27 Qui mettra une garde devant ma bouche, et un sceau de sagesse sur mes lèvres, pour que je ne tombe pas subitement par elles, et que ma langue ne me détruise pas ?

\chapter{23}

\par 1 O Seigneur, Père et Gouverneur de toute ma vie, ne me laisse pas à leurs conseils, et ne me laisse pas tomber par eux.
\par 2 Qui mettra des fléaux sur mes pensées, et la discipline de la sagesse sur mon cœur ? qu'ils ne m'épargnent pas mes ignorances, et que cela ne passe pas par mes péchés :
\par 3 De peur que mes ignorances ne s'accroissent, et que mes péchés n'abondent jusqu'à ma destruction, et que je ne tombe devant mes adversaires, et que mon ennemi ne se réjouisse à propos de moi, dont l'espérance est loin de ta miséricorde.
\par 4 O Seigneur, Père et Dieu de ma vie, ne me regarde pas avec orgueil, mais détourne-toi de tes serviteurs d'un esprit toujours hautain.
\par 5 Détourne de moi les vaines espérances et la concupiscence, et tu soutiendras celui qui veut toujours te servir.
\par 6 Que l'avidité du ventre ni la convoitise de la chair ne s'emparent pas de moi ; et ne me livre pas ton serviteur à un esprit impudent.
\par 7 Écoutez, ô vous les enfants, la discipline de la bouche : celui qui la garde ne sera jamais pris dans ses lèvres.
\par 8 Le pécheur sera laissé dans sa folie ; le méchant orateur et l'orgueilleux tomberont par là.
\par 9 N'habitue pas ta bouche à jurer ; et ne t'utilise pas non plus à nommer le Saint.
\par 10 Car, de même qu'un serviteur qui est continuellement battu ne sera pas sans marque bleue, de même celui qui jure et nomme Dieu continuellement ne sera pas irréprochable.
\par 11 Un homme qui jure beaucoup sera rempli d'iniquité, et la plaie ne quittera jamais sa maison ; s'il commet une offense, son péché retombera sur lui ; et s'il ne reconnaît pas son péché, il commet une double offense. : et s'il jure en vain, il ne sera pas innocent, mais sa maison sera pleine de calamités.
\par 12 Il y a une parole qui est revêtue de mort : Dieu veuille qu'elle ne se retrouve pas dans l'héritage de Jacob ; car toutes ces choses seront loin des pieux, et ils ne se vautreront pas dans leurs péchés.
\par 13 N'utilise pas ta bouche pour jurer avec intempérance, car c'est là que se trouve la parole du péché.
\par 14 Souviens-toi de ton père et de ta mère, quand tu es assis parmi les grands. N'oublie pas devant eux, et ainsi, par ta coutume, tu deviendras un insensé, et tu souhaiteras ne pas être né, et tu maudiras le jour de ta nativité.
\par 15 L'homme habitué aux paroles injurieuses ne se réformera jamais tous les jours de sa vie.
\par 16 Deux sortes d'hommes multiplient le péché, et la troisième amènera la colère : un esprit brûlant est comme un feu brûlant, il ne s'éteindra jamais jusqu'à ce qu'il soit consumé : un fornicateur dans le corps de sa chair ne cessera jamais jusqu'à ce qu'il ait allumé un feu.
\par 17 Tout pain est doux pour un impudique, il ne s'arrêtera pas jusqu'à sa mort.
\par 18 Un homme qui rompt le mariage, disant ainsi dans son cœur : Qui me voit ? Je suis entouré de ténèbres, les murs me couvrent et personne ne me voit ; de quoi ai-je besoin de craindre ? le Très-Haut ne se souviendra pas de mes péchés :
\par 19 Un tel homme ne craint que les yeux des hommes, et ne sait pas que les yeux du Seigneur sont dix mille fois plus brillants que le soleil, voyant toutes les voies des hommes et considérant les parties les plus secrètes.
\par 20 Il connaissait toutes choses avant même qu'elles soient créées ; de même, après qu'ils furent perfectionnés, il les regarda tous.
\par 21 Cet homme sera puni dans les rues de la ville, et là où il ne se doute pas, il sera emmené.
\par 22 Ainsi en sera-t-il aussi de la femme qui quitte son mari et amène un héritier par un autre.
\par 23 Car premièrement, elle a désobéi à la loi du Très-Haut ; et deuxièmement, elle a péché contre son propre mari ; et troisièmement, elle s'est prostituée en adultère et a amené des enfants par un autre homme.
\par 24 Elle sera amenée dans l'assemblée, et l'enquête sera faite sur ses enfants.
\par 25 Ses enfants ne prendront pas de racines, et ses branches ne produiront aucun fruit.
\par 26 Elle laissera sa mémoire être maudite, et son opprobre ne sera pas effacé.
\par 27 Et ceux qui resteront sauront qu'il n'y a rien de meilleur que la crainte du Seigneur, et qu'il n'y a rien de plus doux que de prêter attention aux commandements du Seigneur.
\par 28 C'est une grande gloire de suivre le Seigneur, et être reçu de lui, c'est une longue vie.

\chapter{24}

\par 1 La sagesse se louera elle-même et se glorifiera au milieu de son peuple.
\par 2 Dans la congrégation du Très-Haut, elle ouvrira la bouche et triomphera devant sa puissance.
\par 3 Je suis sorti de la bouche du Très-Haut, et j'ai couvert la terre comme une nuée.
\par 4 J'ai habité dans les hauts lieux, et mon trône est dans une colonne de nuée.
\par 5 Moi seul j'ai parcouru le tour du ciel et j'ai marché au fond de l'abîme.
\par 6 Dans les vagues de la mer et dans toute la terre, et dans chaque peuple et nation, j'ai acquis une possession.
\par 7 Avec tout cela j'ai cherché le repos ; et dans l'héritage de qui demeurerai-je ?
\par 8 Alors le Créateur de toutes choses m'a donné un commandement, et celui qui m'a fait a fait reposer mon tabernacle, et a dit : Que ta demeure soit à Jacob, et ton héritage en Israël.
\par 9 Il m'a créé dès le commencement avant le monde, et je ne faillirai jamais.
\par 10 Dans le saint tabernacle, j'ai servi devant lui ; et c'est ainsi que je fus établi à Sion.
\par 11 De même, dans la ville bien-aimée, il m'a donné du repos, et à Jérusalem était ma puissance.
\par 12 Et j'ai pris racine dans un peuple honorable, même dans la part de l'héritage du Seigneur.
\par 13 J'étais élevé comme un cèdre au Liban, et comme un cyprès sur les montagnes de l'Hermon.
\par 14 J'ai été élevé comme un palmier à En-Gaddi, et comme un rosier à Jéricho, comme un bel olivier dans un champ agréable, et j'ai grandi comme un platane au bord de l'eau.
\par 15 Je dégageais une odeur douce comme celle du cinnamome et de l'aspalathus, et je produisais une odeur agréable comme la meilleure myrrhe, comme le galbanum, l'onyx et le storax odoriférant, et comme la fumée de l'encens dans le tabernacle.
\par 16 Comme l'arbre à térébenthine, j'ai étendu mes branches, et mes branches sont les branches de l'honneur et de la grâce.
\par 17 Comme la vigne a produit une odeur agréable, et mes fleurs sont le fruit de l'honneur et de la richesse.
\par 18 Je suis la mère du bel amour, de la crainte, de la connaissance et de la sainte espérance ; c'est pourquoi, étant éternelle, je suis donnée à tous mes enfants qui portent son nom.
\par 19 Venez à moi, vous tous qui me désirez, et remplissez-vous de mes fruits.
\par 20 Car mon souvenir est plus doux que le miel, et mon héritage que le rayon de miel.
\par 21 Ceux qui me mangent auront encore faim, et ceux qui me boivent auront encore soif.
\par 22 Celui qui m'obéit ne sera jamais confondu, et ceux qui travaillent pour moi ne feront pas de mal.
\par 23 Toutes ces choses sont le livre de l'alliance du Dieu Très-Haut, la loi que Moïse a prescrite en héritage aux congrégations de Jacob.
\par 24 Ne vous évanouissez pas pour être fort dans le Seigneur ; afin qu'il vous confirme, attachez-vous à lui : car le Seigneur Tout-Puissant est Dieu seul, et à côté de lui il n'y a pas d'autre Sauveur.
\par 25 Il remplit toutes choses de sa sagesse, comme Phison et comme Tigre au temps des nouveaux fruits.
\par 26 Il fait en sorte que l'intelligence abonde comme l'Euphrate et comme le Jourdain au temps de la moisson.
\par 27 Il fait apparaître la doctrine de la connaissance comme la lumière et comme Geon au temps des vendanges.
\par 28 Le premier homme ne l'a pas parfaitement connue : le dernier ne la découvrira pas non plus.
\par 29 Car ses pensées sont plus grandes que la mer, et ses conseils plus profonds que les grands abîmes.
\par 30 Moi aussi, je suis sorti comme un ruisseau d'une rivière, et comme un conduit dans un jardin.
\par 31 J'ai dit : J'arroserai mon plus beau jardin, et j'arroserai abondamment mon lit de jardin ; et voici, mon ruisseau est devenu un fleuve, et mon fleuve est devenu une mer.
\par 32 Je ferai encore briller la doctrine comme le matin, et j'enverrai sa lumière au loin.
\par 33 Je répandrai encore la doctrine comme une prophétie, et je la laisserai à tous les âges pour toujours.
\par 34 Voici que je n'ai pas travaillé pour moi seulement, mais pour tous ceux qui recherchent la sagesse.

\chapter{25}

\par 1 En trois choses, j'ai été embelli et je me suis tenu beau devant Dieu et devant les hommes : l'unité des frères, l'amour du prochain, un homme et une femme qui s'accordent.
\par 2 Trois sortes d'hommes que mon âme déteste, et je suis grandement offensé par leur vie : un pauvre qui est orgueilleux, un riche qui est menteur, et un vieil adultère qui s'adonne.
\par 3 Si tu n'as rien amassé dans ta jeunesse, comment peux-tu trouver quelque chose dans ton âge ?
\par 4 O comme c'est une chose belle de juger des cheveux gris et de connaître les conseils des hommes anciens !
\par 5 Oh, comme la sagesse des vieillards est belle, et l'intelligence et le conseil pour les hommes d'honneur.
\par 6 Une grande expérience est la couronne des vieillards, et la crainte de Dieu est leur gloire.
\par 7 Il y a neuf choses que j'ai jugées heureuses dans mon cœur, et la dixième je prononcerai avec ma langue : Un homme qui a la joie de ses enfants ; et celui qui vit pour voir la chute de son ennemi :
\par 8 Bien est celui qui habite avec une femme intelligente, qui n'a pas glissé avec sa langue, et qui n'a pas servi un homme plus indigne que lui.
\par 9 Bien est celui qui a trouvé la prudence, et celui qui parle aux oreilles de ceux qui veulent entendre :
\par 10 Ô combien grand est celui qui trouve la sagesse ! pourtant il n'y a personne au-dessus de lui qui craint le Seigneur.
\par 11 Mais l'amour du Seigneur surpasse toutes choses en matière d'illumination : celui qui le détient, à quoi sera-t-il comparé ?
\par 12 La crainte du Seigneur est le commencement de son amour, et la foi est le commencement de l'attachement à lui.
\par 13 [Donnez-moi] n'importe quelle plaie, sauf la plaie du cœur ; et toute méchanceté, sauf la méchanceté d'une femme :
\par 14 Et toute affliction, sauf l'affliction de ceux qui me haïssent ; et toute vengeance, sauf la vengeance des ennemis.
\par 15 Il n'y a pas de tête au-dessus de la tête d'un serpent ; et il n’y a pas de colère supérieure à la colère d’un ennemi.
\par 16 J'aime mieux demeurer avec un lion et un dragon que d'avoir une maison avec une méchante femme.
\par 17 La méchanceté d'une femme change son visage et obscurcit son visage comme un sac.
\par 18 Son mari s'assiéra parmi ses voisins ; et quand il entendra, il soupirera amèrement.
\par 19 Toute méchanceté est peu de chose comparée à la méchanceté d'une femme : que la part du pécheur lui revienne.
\par 20 Comme la montée d'un chemin sablonneux est aux pieds des vieillards, ainsi une femme pleine de paroles est pour un homme tranquille.
\par 21 Ne trébuche pas devant la beauté d'une femme, et ne la désire pas pour le plaisir.
\par 22 Une femme, si elle entretient son mari, est pleine de colère, d'impudence et de beaucoup de reproches.
\par 23 Une femme méchante perd le courage, elle a un visage lourd et un cœur blessé ; une femme qui ne veut pas consoler son mari dans la détresse rend les mains faibles et les genoux faibles.
\par 24 C'est de la femme qu'est venu le commencement du péché, et par elle nous mourons tous.
\par 25 Ne laissez pas passer l'eau ; ni une méchante femme n'a la liberté de partir à l'étranger.
\par 26 Si elle ne part pas comme tu le souhaites, retranche-la de ta chair, donne-lui un acte de divorce, et laisse-la partir.

\chapter{26}

\par 1 Bienheureux l'homme qui a une femme vertueuse, car le nombre de ses jours sera double.
\par 2 Une femme vertueuse réjouit son mari, et il accomplira les années de sa vie en paix.
\par 3 Une bonne épouse est une bonne part, qui sera donnée à la part de ceux qui craignent l'Éternel.
\par 4 Qu'un homme soit riche ou pauvre, s'il a un bon cœur envers le Seigneur, il se réjouira à tout moment avec un visage joyeux.
\par 5 Il y a trois choses que mon cœur craint ; et pour le quatrième, j'avais très peur : la calomnie d'une ville, le rassemblement d'une multitude indisciplinée et une fausse accusation : tout cela est pire que la mort.
\par 6 Mais la tristesse du cœur et la tristesse sont pour une femme jalouse d'une autre femme, et un fléau de la langue qui communique avec tous.
\par 7 Une mauvaise épouse est un joug secoué: celui qui la tient est comme s'il tenait un scorpion.
\par 8 Une femme ivre et un gadder dehors provoquent une grande colère, et elle ne cache pas sa propre honte.
\par 9 La prostitution d'une femme se reconnaît à ses regards hautains et à ses paupières.
\par 10 Si ta fille est impudique, retiens-la strictement, de peur qu'elle ne se maltraite par trop de liberté.
\par 11 Garde un œil impudent, et ne t'étonne pas si elle pèche contre toi.
\par 12 Elle ouvrira la bouche, comme un voyageur assoiffé lorsqu'il aura trouvé une fontaine, et boira de toutes les eaux près d'elle ; elle s'assiéra près de chaque haie, et ouvrira son carquois contre toute flèche.
\par 13 La grâce d'une femme ravit son mari, et sa discrétion engraisse ses os.
\par 14 Une femme silencieuse et aimante est un don du Seigneur ; et il n’y a rien de plus précieux qu’un esprit bien instruit.
\par 15 Une femme honteuse et fidèle est une double grâce, et son esprit continent ne peut être valorisé.
\par 16 Comme le soleil lorsqu'il se lève dans les cieux élevés ; telle est la beauté d'une bonne épouse dans l'ordre de sa maison.
\par 17 Comme la claire lumière est sur le chandelier sacré ; ainsi est la beauté du visage à l’âge mûr.
\par 18 Comme les colonnes d'or sont sur des bases d'argent ; ainsi sont les pieds blonds au cœur constant.
\par 19 Mon fils, garde saine la fleur de ton âge ; et ne donne pas ta force à des étrangers.
\par 20 Quand tu auras acquis une possession fructueuse dans tout le champ, sème-le avec ta propre semence, confiant dans la qualité de ton bétail.
\par 21 Ainsi ta race que tu quittes sera magnifiée, ayant la confiance de leur bonne descendance.
\par 22 Une prostituée sera considérée comme un crachat ; mais une femme mariée est pour son mari une tour contre la mort.
\par 23 Une femme méchante est donnée en part à un homme méchant, mais une femme pieuse est donnée à celui qui craint l'Éternel.
\par 24 Une femme malhonnête méprise la honte, mais une femme honnête respectera son mari.
\par 25 Une femme impudique sera considérée comme un chien ; mais celle qui a honte craindra l'Éternel.
\par 26 Une femme qui honore son mari sera jugée sage par tous ; mais celle qui le déshonore dans son orgueil sera considérée comme impie parmi tous.
\par 27 On cherchera une femme qui crie fort et un grondeur pour chasser les ennemis.
\par 28 Il y a deux choses qui attristent mon cœur ; et le troisième me met en colère : un homme de guerre qui souffre de pauvreté ; et des hommes intelligents qui ne sont pas fixés ; et celui qui revient de la justice au péché ; le Seigneur prépare un tel homme pour l'épée.
\par 29 Un commerçant ne s'empêchera guère de faire le mal ; et un colporteur ne sera pas libéré du péché.

\chapter{27}

\par 1 Beaucoup ont péché pour une petite affaire ; et celui qui cherche l'abondance détournera les yeux.
\par 2 Comme un clou s'attache entre les joints des pierres ; ainsi le péché reste-t-il étroit entre l’achat et la vente.
\par 3 Si un homme ne se maintient diligemment dans la crainte du Seigneur, sa maison sera bientôt renversée.
\par 4 Comme lorsqu'on tamise avec un tamis, les ordures restent ; donc la saleté de l'homme dans son discours.
\par 5 Le fourneau éprouve les ustensiles du potier ; ainsi l'épreuve de l'homme est dans son raisonnement.
\par 6 Le fruit déclare si l'arbre a été taillé ; il en est de même pour l’expression d’une vanité dans le cœur de l’homme.
\par 7 Ne louez personne avant de l'entendre parler ; car c'est l'épreuve des hommes.
\par 8 Si tu suis la justice, tu l'obtiendras et tu la revêtiras comme une longue robe glorieuse.
\par 9 Les oiseaux auront recours à leurs pareils ; ainsi la vérité reviendra à ceux qui pratiquent en elle.
\par 10 Comme le lion guette sa proie ; péchez donc pour ceux qui commettent l'iniquité.
\par 11 Le discours d'un homme pieux est toujours avec sagesse ; mais l'insensé change comme la lune.
\par 12 Si tu es du nombre des indiscrets, observe le temps ; mais soyez continuellement parmi des hommes intelligents.
\par 13 Le discours des insensés est ennuyeux, et leur jeu est la folie du péché.
\par 14 Les paroles de celui qui jure beaucoup font dresser les cheveux ; et leurs bagarres font boucher les oreilles.
\par 15 La querelle des orgueilleux est une effusion de sang, et leurs injures sont douloureuses à l'oreille.
\par 16 Celui qui découvre les secrets perd son crédit ; et ne trouvera jamais d'ami à son esprit.
\par 17 Aime ton ami et sois-lui fidèle ; mais si tu trahis ses secrets, ne le poursuis plus.
\par 18 Car comme un homme a détruit son ennemi ; ainsi as-tu perdu l'amour de ton prochain.
\par 19 Comme celui qui laisse un oiseau sortir de sa main, ainsi tu as laissé partir ton prochain, et tu ne le reprendras plus.
\par 20 Ne le suivez plus, car il est trop loin ; il est comme un chevreuil échappé du piège.
\par 21 Quant à une blessure, elle peut être pansée ; et après l'injure, il peut y avoir une réconciliation ; mais celui qui trahit les secrets est sans espoir.
\par 22 Celui qui cligne des yeux fait le mal, et celui qui le connaît s'éloignera de lui.
\par 23 Quand tu seras présent, il parlera avec douceur et admirera tes paroles ; mais à la fin il se tordra la bouche et calomniera tes paroles.
\par 24 J'ai haï beaucoup de choses, mais rien de semblable à lui ; car le Seigneur le haïra.
\par 25 Celui qui jette une pierre en haut la jette sur sa tête ; et un coup trompeur fera des blessures.
\par 26 Celui qui creuse une fosse y tombera, et celui qui tend un piège y sera pris.
\par 27 Celui qui fait le mal, cela tombera sur lui, et il ne saura pas d'où cela vient.
\par 28 La moquerie et l'opprobre viennent des orgueilleux ; mais la vengeance, comme un lion, les guettera.
\par 29 Ceux qui se réjouissent de la chute des justes seront pris dans le piège ; et l'angoisse les consumera avant qu'ils ne meurent.
\par 30 La méchanceté et la colère, même celles-là sont des abominations ; et le pécheur les aura tous deux.

\chapter{28}

\par 1 Celui qui se venge trouvera vengeance auprès du Seigneur, et il gardera sûrement ses péchés [en souvenir.]
\par 2 Pardonne à ton prochain le mal qu'il t'a fait, ainsi tes péchés seront aussi pardonnés quand tu pries.
\par 3 L'un a de la haine contre l'autre, et demande-t-il pardon à l'Éternel ?
\par 4 Il ne fait preuve d'aucune miséricorde envers un homme qui lui ressemble ; et demande-t-il pardon de ses propres péchés ?
\par 5 Si celui qui n'est que chair nourrit la haine, qui implorera le pardon de ses péchés ?
\par 6 Souviens-toi de ta fin, et que cesse l'inimitié ; [rappelle-toi] de la corruption et de la mort, et demeure dans les commandements.
\par 7 Souviens-toi des commandements, et ne porte pas de méchanceté envers ton prochain : [souviens-toi] de l'alliance du Très-Haut, et cligne de l'œil à l'ignorance.
\par 8 Abstiens-toi de la querelle, et tu diminueras tes péchés ; car un homme furieux attisera la querelle,
\par 9 L'homme pécheur inquiète ses amis et suscite des débats entre ceux qui sont en paix.
\par 10 Telle est la matière du feu, ainsi il brûle ; et telle est la force d'un homme, telle est sa colère ; et selon ses richesses sa colère monte ; et plus ceux qui luttent sont forts, plus ils s'enflammeront.
\par 11 Une dispute précipitée allume le feu, et un combat précipité fait couler le sang.
\par 12 Si tu souffles l'étincelle, elle brûlera ; si tu craches dessus, elle s'éteindra ; et les deux sortiront de ta bouche.
\par 13 Maudits soient ceux qui chuchotent et ceux qui parlent deux fois : car ceux-là ont détruit beaucoup de ceux qui étaient en paix.
\par 14 Une langue médisante a inquiété beaucoup de gens et les a chassés de nation en nation ; elle a démoli des villes fortes et renversé les maisons des grands hommes.
\par 15 Une langue médisante a chassé les femmes vertueuses et les a privées de leurs travaux.
\par 16 Celui qui l'écoute ne trouvera jamais de repos et ne demeurera jamais tranquille.
\par 17 Le coup du fouet fait des marques dans la chair, mais le coup de la langue brise les os.
\par 18 Beaucoup sont tombés sous le tranchant de l'épée, mais il n'y en a pas autant qui sont tombés par la langue.
\par 19 Bien est celui qui se défend à cause de son venin ; qui n'en a pas tiré le joug, et qui n'a pas été liée par ses liens.
\par 20 Car son joug est un joug de fer, et ses attaches sont des attaches d'airain.
\par 21 Sa mort est une mort mauvaise, le tombeau serait meilleur qu'elle.
\par 22 Il ne dominera pas sur ceux qui craignent Dieu, et ils ne seront pas non plus brûlés par sa flamme.
\par 23 Ceux qui abandonnent l'Éternel y tomberont ; et cela brûlera en eux et ne s'éteindra pas ; il sera envoyé sur eux comme un lion, et il les dévorera comme un léopard.
\par 24 Veille à ce que tu clôtures ta propriété avec des épines, et à lier ton argent et ton or,
\par 25 Et pèse tes paroles dans une balance, et fais une porte et un barreau pour ta bouche.
\par 26 Prenez garde de ne pas y glisser, de peur de tomber devant celui qui vous attend.

\chapter{29}

\par 1 Celui qui est miséricordieux prêtera à son prochain ; et celui qui fortifie sa main garde les commandements.
\par 2 Prête à ton prochain au moment où il en a besoin, et rembourse ton prochain au temps convenable.
\par 3 Garde ta parole et agis fidèlement envers lui, et tu trouveras toujours ce qui t'est nécessaire.
\par 4 Beaucoup, lorsqu'on leur prêtait une chose, pensaient qu'elle était trouvée, et leur causaient des ennuis qui les aidaient.
\par 5 Jusqu'à ce qu'il ait reçu, il baisera la main d'un homme ; et pour l'argent de son voisin, il parlera avec soumission ; mais quand il devra rembourser, il prolongera le temps, et rendra des paroles de douleur et se plaindra du temps.
\par 6 S'il l'emporte, il recevra à peine la moitié, et il comptera comme s'il l'avait trouvée ; sinon, il l'a privé de son argent, et il s'est fait un ennemi sans raison : il le paie avec injures et railleries ; et pour honneur il lui rendra la disgrâce.
\par 7 Beaucoup ont donc refusé de prêter pour les mauvais traitements d'autrui, craignant d'être fraudés.
\par 8 Mais aie patience envers un homme pauvre, et ne tarde pas à lui faire miséricorde.
\par 9 Aidez le pauvre à cause du commandement, et ne le repoussez pas à cause de sa pauvreté.
\par 10 Perds ton argent pour ton frère et ton ami, et qu'il ne rouille pas sous une pierre pour se perdre.
\par 11 Amasse ton trésor selon les commandements du Très-Haut, et il te rapportera plus que l'or.
\par 12 Enferme l'aumône dans tes greniers, et elle te délivrera de toute affliction.
\par 13 Il combattra pour toi contre tes ennemis mieux qu'un puissant bouclier et une lance puissante.
\par 14 L'honnête homme se porte garant de son prochain, mais l'impudent l'abandonnera.
\par 15 N'oublie pas l'amitié de ton garant, car il a donné sa vie pour toi.
\par 16 Le pécheur renversera le bon patrimoine de sa caution :
\par 17 Et celui qui est d'esprit ingrat laissera celui qui l'a délivré.
\par 18 Le cautionnement a détruit beaucoup de biens et les a secoués comme une vague de la mer ; il a chassé les hommes forts de leurs maisons, et les a fait errer parmi des nations étrangères.
\par 19 Le méchant qui transgresse les commandements de l'Éternel tombera dans le cautionnement, et celui qui entreprend et suit les affaires d'autrui dans un but lucratif tombera dans des poursuites.
\par 20 Aide ton prochain selon ta puissance, et prends garde que toi-même ne tombe dans le même sort.
\par 21 L'essentiel pour la vie, c'est l'eau, le pain, les vêtements et une maison pour couvrir la honte.
\par 22 Mieux vaut la vie d'un pauvre homme dans une chaumière mesquine, qu'un repas délicat dans la maison d'un autre.
\par 23 Que ce soit peu ou beaucoup, sois content de ne pas entendre l'opprobre de ta maison.
\par 24 Car c'est une vie misérable que d'aller de maison en maison ; car là où tu es étranger, tu n'oses pas ouvrir la bouche.
\par 25 Tu recevras et tu feras un festin, et tu n'auras aucune reconnaissance ; et tu entendras des paroles amères :
\par 26 Viens, étranger, et dresse une table, et nourris-moi de ce que tu as préparé.
\par 27 Cède la place, étranger, à un homme honorable ; mon frère vient pour être hébergé, et j'ai besoin de ma maison.
\par 28 Ces choses sont pénibles pour un homme intelligent ; la réprimande de la maison et les reproches du prêteur.

\chapter{30}

\par 1 Celui qui aime son fils lui fait souvent sentir le bâton, afin qu'il finisse par en avoir de la joie.
\par 2 Celui qui châtie son fils se réjouira de lui et se réjouira de lui parmi ses connaissances.
\par 3 Celui qui enseigne à son fils attriste l'ennemi, et il se réjouit de lui devant ses amis.
\par 4 Même si son père meurt, il est comme s'il n'était pas mort, car il a laissé derrière lui quelqu'un qui lui ressemble.
\par 5 Pendant qu'il vivait, il le voyait et se réjouissait en lui ; et quand il mourait, il n'était pas triste.
\par 6 Il a laissé derrière lui un vengeur contre ses ennemis, et un homme qui rendra de la bonté à ses amis.
\par 7 Celui qui fait trop de cas de son fils pansera ses blessures ; et ses entrailles seront troublées à chaque cri.
\par 8 Un cheval qui n'est pas dressé devient entêté, et un enfant livré à lui-même sera volontaire.
\par 9 Accroche ton enfant, et il te fera peur ; joue avec lui, et il t'alourdira.
\par 10 Ne riez pas avec lui, de peur que vous n'ayez du chagrin avec lui et que vous ne finissiez par grincer des dents.
\par 11 Ne lui donne aucune liberté dans sa jeunesse, et ne cligne pas de l'œil devant ses folies.
\par 12 Inclinez-lui le cou lorsqu'il est jeune, et frappez-le sur les côtés pendant qu'il est enfant, de peur qu'il ne s'entête et ne vous désobéisse, et qu'il ne provoque ainsi de la tristesse dans votre cœur.
\par 13 Châtie ton fils et fais-le travailler, de peur que sa conduite impudique ne te scandalise.
\par 14 Mieux vaut un pauvre sain et fort de constitution qu'un riche qui est affligé dans son corps.
\par 15 La santé et la bonne santé du corps sont au-dessus de tout l'or, et un corps fort au-dessus de la richesse infinie.
\par 16 Il n'y a pas de richesse au-dessus d'un corps sain, et pas de joie au-dessus de la joie du cœur.
\par 17 La mort vaut mieux qu'une vie amère ou une maladie continuelle.
\par 18 Les mets délicats versés sur une bouche fermée sont comme des morceaux de viande déposés sur un tombeau.
\par 19 A quoi sert l'offrande à une idole ? car il ne peut ni manger ni sentir. Ainsi en est-il de celui qui est persécuté par l'Éternel.
\par 20 Il voit de ses yeux et gémit, comme un eunuque qui embrasse une vierge et soupire.
\par 21 Ne livre pas ton esprit à la lourdeur, et ne t'afflige pas dans ton propre conseil.
\par 22 La joie du cœur est la vie de l'homme, et la joie de l'homme prolonge ses jours.
\par 23 Aime ton âme et console ton cœur, éloigne de toi le chagrin, car le chagrin en a tué beaucoup, et il n'y a aucun profit à en tirer.
\par 24 L'envie et la colère raccourcissent la vie, et la prudence amène la vieillesse avant le temps.
\par 25 Un cœur joyeux et bon prendra soin de sa viande et de son alimentation.

\chapter{31}

\par 1 La recherche des richesses consume la chair, et son souci chasse le sommeil.
\par 2 Les soins qui veillent ne laissent pas l'homme dormir, comme une maladie douloureuse interrompt le sommeil,
\par 3 Les riches ont un grand travail pour rassembler les richesses ; et quand il se repose, il est rempli de ses délices.
\par 4 Le pauvre travaille dans son pauvre domaine ; et quand il s'arrête, il est encore dans le besoin.
\par 5 Celui qui aime l'or ne sera pas justifié, et celui qui suit la corruption en aura assez.
\par 6 L'or a été la ruine de beaucoup, et leur destruction était présente.
\par 7 C'est une pierre d'achoppement pour ceux qui y sacrifient, et tout insensé en sera pris.
\par 8 Heureux le riche qui se trouve sans défaut et qui n'a pas recherché l'or.
\par 9 Qui est-il ? et nous le dirons bienheureux, car il a fait des merveilles parmi son peuple.
\par 10 Qui a été ainsi éprouvé et trouvé parfait ? alors laissez-le se glorifier. Qui pourrait offenser et n’a pas offensé ? ou a fait le mal, et ne l'a-t-il pas fait ?
\par 11 Ses biens seront établis, et l'assemblée déclarera ses aumônes.
\par 12 Si tu es assis à une table abondante, n'en sois pas avare et ne dis pas : Il y a beaucoup de viande dessus.
\par 13 Rappelez-vous qu'un mauvais œil est une chose mauvaise : et qu'est-ce qui est créé plus méchant qu'un œil ? c'est pourquoi il pleure à chaque occasion.
\par 14 N'étends pas ta main partout où elle regarde, et ne la mets pas avec lui dans le plat.
\par 15 Ne juge pas ton prochain par toi-même, et sois prudent en tout.
\par 16 Mangez comme il convient à un homme, de ce qui est servi devant toi ; et dévore-le, de peur d'être haï.
\par 17 Pars d'abord pour l'amour des bonnes manières ; et ne sois pas insatiable, de peur d'offenser.
\par 18 Quand tu es assis parmi plusieurs, ne tends pas premièrement la main.
\par 19 Très peu suffit à un homme bien nourri, et il ne manque pas de souffle sur son lit.
\par 20 Un bon sommeil vient d'une alimentation modérée : il se lève tôt, et son esprit est avec lui ; mais la douleur de veiller, la colère et les douleurs de ventre sont chez un homme insatiable.
\par 21 Et si tu as été forcé de manger, lève-toi, sors, vomis, et tu te reposeras.
\par 22 Mon fils, écoute-moi et ne me méprise pas, et à la fin tu trouveras ce que je t'ai dit : sois prompt dans toutes tes œuvres, afin qu'aucune maladie ne t'arrive.
\par 23 Quiconque est généreux en matière de nourriture, on dira du bien de lui ; et le bruit de sa bonne gestion sera cru.
\par 24 Mais contre celui qui est avare de sa nourriture, toute la ville murmurera ; et les témoignages de son avarice ne seront pas mis en doute.
\par 25 Ne montre pas ta vaillance dans le vin ; car le vin en a détruit beaucoup.
\par 26 La fournaise éprouve le bord en l'abîmant, ainsi le vin rend le cœur des orgueilleux par l'ivresse.
\par 27 Le vin est aussi bon que la vie pour un homme, s'il est bu avec modération : qu'est donc la vie pour un homme qui n'a pas de vin ? car il a été fait pour rendre les hommes heureux.
\par 28 Le vin bu dans la mesure et à sa saison apporte la joie du cœur et la gaieté de l'esprit :
\par 29 Mais le vin bu avec excès rend l'amertume des esprits, avec des bagarres et des querelles.
\par 30 L'ivresse augmente la rage de l'insensé jusqu'à le rendre offensant : elle diminue la force et fait des blessures.
\par 31 Ne réprimande pas ton prochain à propos du vin, et ne le méprise pas dans sa joie ; ne lui donne pas de paroles méchantes, et ne l'insiste pas en l'incitant à boire.

\chapter{32}

\par 1 Si tu es nommé maître [d'un festin], ne t'élève pas, mais sois parmi eux comme l'un des autres ; prenez-en grand soin et asseyez-vous.
\par 2 Et quand tu auras accompli tout ton office, prends ta place, afin de te réjouir avec eux, et reçois une couronne pour ta bonne organisation de la fête.
\par 3 Parle, toi qui es l'aîné, car cela te convient, mais avec un bon jugement ; et ne gêne pas la musique.
\par 4 Ne répandez pas de paroles là où il y a un musicien, et ne faites pas preuve de sagesse hors du temps.
\par 5 Un concert de musique dans un banquet de vin est comme un sceau d'escarboucle serti d'or.
\par 6 Comme le sceau d'une émeraude serti dans un ouvrage d'or, telle est la mélodie de la musique avec un vin agréable.
\par 7 Parle, jeune homme, si tu as besoin de toi ; et pourtant, à peine lorsqu'on te le demande deux fois.
\par 8 Que ton discours soit court, et que tu comprennes beaucoup en peu de mots ; soyez comme quelqu'un qui sait et qui tient sa langue.
\par 9 Si tu es parmi les grands, ne te rends pas égal à eux ; et quand les hommes anciens sont en place, n'utilisez pas beaucoup de mots.
\par 10 Avant que le tonnerre éclate, les éclairs ; et devant un homme honteux, il ira en grâce.
\par 11 Levez-vous tôt, et ne soyez pas le dernier ; mais rentre chez toi sans tarder.
\par 12 Là, prends ton passe-temps et fais ce que tu veux ; mais ne péche pas par des paroles orgueilleuses.
\par 13 Et pour ces choses, bénis celui qui t'a créé et qui t'a comblé de ses bonnes choses.
\par 14 Celui qui craint l'Éternel recevra sa discipline ; et ceux qui le cherchent tôt trouveront grâce.
\par 15 Celui qui recherche la loi en sera rempli, mais l'hypocrite en sera scandalisé.
\par 16 Ceux qui craignent l'Éternel trouveront le jugement, et allumeront la justice comme une lumière.
\par 17 Un homme pécheur ne sera pas repris, mais il trouve une excuse selon sa volonté.
\par 18 Un homme de conseil sera prévenant; mais un homme étranger et fier n'est pas intimidé par la peur, même lorsqu'il a agi sans conseil.
\par 19 Ne faites rien sans conseil ; et quand tu l'as fait une fois, ne te repens pas.
\par 20 Ne marche pas dans un chemin où tu pourrais tomber, et ne trébuche pas parmi les pierres.
\par 21 Ne soyez pas confiant d'une manière simple.
\par 22 Et prends garde à tes propres enfants.
\par 23 Dans toute bonne œuvre, confie-toi à ton âme ; car c'est là l'observance des commandements.
\par 24 Celui qui croit au Seigneur prend garde au commandement ; et celui qui se confie en lui ne s'en portera jamais plus mal.

\chapter{33}

\par 1 Il n'arrivera aucun mal à celui qui craint l'Éternel ; mais, encore une fois, dans la tentation, il le délivrera.
\par 2 Le sage ne hait pas la loi ; mais celui qui y est hypocrite est comme un navire dans la tempête.
\par 3 L'homme intelligent se confie dans la loi ; et la loi lui est fidèle, comme un oracle.
\par 4 Préparez ce que vous devez dire, et ainsi vous serez entendu ; et reliez les instructions, puis répondez.
\par 5 Le cœur de l'insensé est comme une roue de charrette ; et ses pensées sont comme un essieu qui roule.
\par 6 Un cheval étalon est comme un ami moqueur, il hennit sous quiconque est assis sur lui.
\par 7 Pourquoi un jour surpasse-t-il un autre, alors que toute la lumière de chaque jour de l'année est celle du soleil ?
\par 8 Par la connaissance du Seigneur, ils se distinguaient : et il modifiait les saisons et les fêtes.
\par 9 Il a fait certains d'entre eux des jours élevés et les a sanctifiés, et d'autres il a fait des jours ordinaires.
\par 10 Et tous les hommes sont issus de la terre, et Adam a été créé de la terre :
\par 11 Avec une grande connaissance, l'Éternel les a divisés et a diversifié leurs voies.
\par 12 Il a béni et exalté certains d'entre eux, il a sanctifié certains d'entre eux et les a placés près de lui ; mais il a maudit et humilié certains d'entre eux et les a chassés de leur place.
\par 13 Comme l'argile est dans la main du potier, pour la façonner à son gré, ainsi l'homme est entre les mains de celui qui l'a fait, pour lui rendre ce qui lui convient le mieux.
\par 14 Le bien s'oppose au mal, et la vie à la mort : ainsi le pieux s'oppose au pécheur, et le pécheur au pieux.
\par 15 Considérez donc toutes les œuvres du Très-Haut ; et il y en a deux et deux, l'un contre l'autre.
\par 16 Je me suis réveillé le dernier, comme quelqu'un qui vendange après les vendangeurs : par la bénédiction du Seigneur, j'ai profité, et j'ai foulé mon pressoir comme un vendangeur.
\par 17 Considérez que j'ai travaillé non seulement pour moi-même, mais pour tous ceux qui cherchent à apprendre.
\par 18 Écoutez-moi, ô grands hommes du peuple, et écoutez de vos oreilles, chefs de l'assemblée.
\par 19 Ne donne pas à ton fils et à ta femme, à ton frère et à ton ami, pouvoir sur toi pendant que tu es vivant, et ne donne pas tes biens à un autre, de peur qu'il ne te repente et que tu ne demandes à nouveau pour cela.
\par 20 Tant que tu vis et que tu respires, ne te livre à personne.
\par 21 Car il vaut mieux que tes enfants te recherchent, plutôt que tu restes fidèle à leur courtoisie.
\par 22 Dans toutes tes œuvres, garde pour toi la prééminence ; ne laisse pas une tache en ton honneur.
\par 23 Au moment où tu finiras tes jours et finiras ta vie, distribue ton héritage.
\par 24 Le fourrage, la baguette et les fardeaux sont pour l'âne ; et du pain, de la correction et du travail pour un serviteur. .
\par 25 Si tu fais travailler ton serviteur, tu trouveras du repos ; mais si tu le laisses oisif, il cherchera la liberté.
\par 26 Le joug et le collier courbent le cou : ainsi sont les tortures et les tourments pour un mauvais serviteur.
\par 27 Envoyez-le travailler, afin qu'il ne reste pas oisif ; car l'oisiveté enseigne beaucoup de mal.
\par 28 Mettez-le au travail comme il lui convient ; s'il n'est pas obéissant, mettez-lui des chaînes plus lourdes.
\par 29 Mais ne soyez excessif envers personne ; et sans discrétion, ne faites rien.
\par 30 Si tu as un serviteur, qu'il soit pour toi comme toi-même, car tu l'as acheté à prix.
\par 31 Si tu as un serviteur, supplie-le comme un frère ; car tu as besoin de lui comme de ton âme. Si tu le supplies mal, et qu'il s'enfuie loin de toi, par quel chemin iras-tu pour le chercher ?

\chapter{34}

\par 1 Les espérances d'un homme dépourvu d'intelligence sont vaines et fausses, et les rêves élèvent les insensés.
\par 2 Celui qui regarde les songes est comme celui qui saisit une ombre et suit le vent.
\par 3 La vision des rêves est la ressemblance d'une chose à une autre, comme la ressemblance d'un visage à un autre.
\par 4 D'une chose impure, qu'est-ce qui peut être purifié ? et de ce qui est faux, quelle vérité peut sortir ?
\par 5 Les divinations, les devins et les rêves sont vains, et le cœur a des imaginations comme le cœur d'une femme en travail.
\par 6 S'ils ne sont pas envoyés du Très-Haut lors de ta visite, ne mets pas ton cœur sur eux.
\par 7 Car les rêves ont trompé beaucoup de gens, et ceux qui avaient mis leur confiance en eux ont échoué.
\par 8 La loi sera trouvée parfaite sans mensonge, et la sagesse est une perfection pour une bouche fidèle.
\par 9 Un homme qui a voyagé sait beaucoup de choses ; et celui qui a beaucoup d'expérience déclarera la sagesse.
\par 10 Celui qui n'a aucune expérience sait peu de choses, mais celui qui a voyagé est plein de prudence.
\par 11 Quand je voyageais, j'ai vu beaucoup de choses ; et je comprends plus que je ne peux exprimer.
\par 12 J'ai souvent été en danger de mort, mais j'ai été délivré à cause de ces choses.
\par 13 L'esprit de ceux qui craignent l'Éternel vivra ; car leur espérance est en celui qui les sauve.
\par 14 Celui qui craint l'Éternel ne craindra ni n'aura peur ; car il est son espoir.
\par 15 Bienheureuse est l'âme de celui qui craint l'Éternel : à qui se tourne-t-il ? et quelle est sa force ?
\par 16 Car les yeux du Seigneur sont sur ceux qui l'aiment, il est leur puissante protection et leur solide appui, une défense contre la chaleur et une couverture contre le soleil de midi, une préservation de trébucher et un secours contre la chute.
\par 17 Il élève l'âme et éclaircit les yeux : il donne la santé, la vie et la bénédiction.
\par 18 Celui qui sacrifie une chose injustement acquise, son offrande est ridicule ; et les dons des hommes injustes ne sont pas acceptés.
\par 19 Le Très-Haut n'apprécie pas les offrandes des méchants ; il n'est pas non plus apaisé au péché par la multitude des sacrifices.
\par 20 Celui qui apporte une offrande des biens des pauvres fait comme celui qui tue son fils sous les yeux de son père.
\par 21 Le pain du nécessiteux, c'est sa vie ; celui qui le lui escroque est un homme de sang.
\par 22 Celui qui enlève la vie de son prochain le tue ; et celui qui prive l'ouvrier de son salaire est un verseur de sang.
\par 23 Quand l'un bâtit, et l'autre démolit, quel profit ont-ils alors sinon le travail ?
\par 24 Quand l'un prie et que l'autre maudit, quelle voix le Seigneur entendra-t-il ?
\par 25 Celui qui se lave après avoir touché un cadavre, s'il le touche encore, à quoi sert son lavage ?
\par 26 Ainsi en est-il de l'homme qui jeûne pour ses péchés, puis s'en va et fait de même : qui entendra sa prière ? ou à quoi lui profite son humiliation ?

\chapter{35}

\par 1 Celui qui observe la loi apporte des offrandes suffisantes ; celui qui prend garde au commandement offre une offrande de paix.
\par 2 Celui qui rend un bon retour offre de la fine farine ; et celui qui fait l'aumône sacrifie la louange.
\par 3 Se détourner de la méchanceté est une chose agréable au Seigneur ; et abandonner l'injustice est une propitiation.
\par 4 Tu ne paraîtras pas vide devant le Seigneur.
\par 5 Car toutes ces choses [doivent être faites] à cause du commandement.
\par 6 L'offrande des justes engraisse l'autel, et sa douce odeur est devant le Très-Haut.
\par 7 Le sacrifice d'un homme juste est acceptable et son mémorial ne sera jamais oublié.
\par 8 Rendez son honneur au Seigneur d'un bon œil, et ne diminuez pas les prémices de vos mains.
\par 9 Dans tous tes dons, montre un visage joyeux, et consacre ta dîme avec allégresse.
\par 10 Donne au Très-Haut selon qu'il t'a enrichi ; et comme tu l'as obtenu, donne-le d'un œil joyeux.
\par 11 Car l'Éternel te récompense, et il te donnera sept fois plus.
\par 12 Ne pensez pas à corrompre avec des cadeaux ; car il ne recevra pas de tels sacrifices : et ne se fie pas à des sacrifices injustes ; car le Seigneur est juge, et avec lui il n'y a aucun respect pour les personnes.
\par 13 Il n'acceptera personne contre un pauvre, mais il entendra la prière de l'opprimé.
\par 14 Il ne méprisera pas la supplication de l'orphelin ; ni la veuve, quand elle déverse sa plainte.
\par 15 Les larmes ne coulent-elles pas sur les joues de la veuve ? et n'est-ce pas son cri contre lui qui les fait tomber ?
\par 16 Celui qui sert le Seigneur sera accepté avec faveur, et sa prière atteindra les nuages.
\par 17 La prière des humbles perce les nuages ​​; et jusqu'à ce qu'elle soit proche, il ne sera pas consolé ; et il ne s'éloignera pas jusqu'à ce que le Très-Haut le voie pour juger avec justice et exécuter le jugement.
\par 18 Car l'Éternel ne se relâchera pas, et les puissants ne seront pas patients envers eux, jusqu'à ce qu'il ait brisé les reins des impitoyables et rendu vengeance aux païens ; jusqu'à ce qu'il ait enlevé la multitude des orgueilleux et brisé le sceptre des injustes ;
\par 19 Jusqu'à ce qu'il ait rendu à chacun selon ses actions, et aux œuvres des hommes selon leurs desseins ; jusqu'à ce qu'il ait jugé la cause de son peuple et l'ait fait se réjouir de sa miséricorde.
\par 20 La miséricorde est de saison au temps de l'affliction, comme les nuages ​​de pluie au temps de la sécheresse.

\chapter{36}

\par 1 Aie pitié de nous, Seigneur Dieu de tous, et regarde-nous :
\par 2 Et envoie ta crainte sur toutes les nations qui ne te cherchent pas.
\par 3 Lève ta main contre les nations étrangères, et qu'elles voient ta puissance.
\par 4 Comme tu as été sanctifié en nous devant eux, ainsi sois-tu magnifié parmi eux devant nous.
\par 5 Et qu'ils te connaissent, comme nous t'avons connu, qu'il n'y a de Dieu que toi seul, ô Dieu.
\par 6 Montre de nouveaux signes et fais d'autres prodiges étranges : glorifie ta main et ton bras droit, afin qu'ils exposent tes merveilles.
\par 7 Soulevez l'indignation et déversez la colère : éloignez l'adversaire et détruisez l'ennemi.
\par 8 Ne prends pas de temps, souviens-toi de l'alliance, et qu'ils racontent tes merveilles.
\par 9 Que celui qui s'échappe soit consumé par la rage du feu ; et que périssent ceux qui oppriment le peuple.
\par 10 Frappez la tête des chefs des païens, qui disent : Il n'y a personne d'autre que nous.
\par 11 Rassemble toutes les tribus de Jacob, et hérite-en comme dès le commencement.
\par 12 O Seigneur, aie pitié du peuple sur lequel ton nom est invoqué, et d'Israël, que tu as nommé ton premier-né.
\par 13 O sois miséricordieux envers Jérusalem, ta ville sainte, le lieu de ton repos.
\par 14 Remplis Sion de tes oracles indicibles, et ton peuple de ta gloire :
\par 15 Rends témoignage à ceux que tu possèdes depuis le commencement, et suscite des prophètes qui ont été en ton nom.
\par 16 Récompense ceux qui t'attendent, et que tes prophètes soient trouvés fidèles.
\par 17 O Seigneur, écoute la prière de tes serviteurs, selon la bénédiction d'Aaron sur ton peuple, afin que tous ceux qui habitent sur la terre sachent que tu es le Seigneur, le Dieu éternel.
\par 18 Le ventre dévore toutes les viandes, mais une viande est meilleure qu'une autre.
\par 19 Comme le palais goûte diverses espèces de gibier, ainsi un cœur qui comprend les fausses paroles.
\par 20 Un cœur rebelle cause de la lourdeur, mais un homme d'expérience le récompensera.
\par 21 Une femme recevra tout homme, mais une fille vaut-elle mieux qu'une autre.
\par 22 La beauté d'une femme réjouit le visage, et un homme n'aime rien de mieux.
\par 23 S'il y a de la bonté, de la douceur et du réconfort dans sa langue, alors son mari n'est pas comme les autres hommes.
\par 24 Celui qui prend une femme commence une possession, une aide semblable à lui-même et une colonne de repos.
\par 25 Là où il n'y a pas de haie, là la propriété est gâtée ; et celui qui n'a pas de femme erre de long en large en pleurant.
\par 26 Qui fera confiance à un voleur bien nommé, qui va de ville en ville ? alors [qui croira] un homme qui n'a pas de maison et qui passe la nuit partout où la nuit l'emmène ?

\chapter{37}

\par 1 Tout ami dit : Je suis aussi son ami ; mais il y a un ami qui n'est ami que de nom.
\par 2 N'est-ce pas un chagrin jusqu'à la mort, lorsqu'un compagnon et un ami se transforme en ennemi ?
\par 3 Ô méchante imagination, d'où es-tu venue pour couvrir la terre de tromperie ?
\par 4 Il y a un compagnon qui se réjouit de la prospérité d'un ami, mais qui, dans les moments de détresse, sera contre lui.
\par 5 Il y a un compagnon qui aide son ami pour le ventre, et qui prend le bouclier contre l'ennemi.
\par 6 N'oublie pas ton ami dans ton esprit, et ne l'oublie pas dans tes richesses.
\par 7 Tout conseiller prône le conseil ; mais il y en a qui se conseillent pour lui-même.
\par 8 Méfiez-vous d'un conseiller, et sachez avant quel besoin il a ; car il conseillera pour lui-même; de peur qu'il ne jette le sort sur toi,
\par 9 Et te dira : Ta voie est bonne ; et ensuite il se tiendra de l'autre côté, pour voir ce qui t'arrivera.
\par 10 Ne consulte pas celui qui te soupçonne, et cache tes conseils à ceux qui t'envient.
\par 11 Ne consultez pas non plus une femme qui touche celle dont elle est jalouse ; ni avec un lâche en matière de guerre ; ni avec un commerçant concernant l'échange ; ni avec un acheteur de vente ; ni avec un homme envieux et reconnaissant ; ni avec un homme impitoyable qui touche à la bonté ; ni avec les paresseux pour aucun travail ; ni avec un mercenaire pour un an de travaux de finition ; ni avec un serviteur oisif qui a beaucoup d'affaires : ne les écoutez pas en matière de conseil.
\par 12 Mais sois continuellement avec un homme pieux, que tu sais garder les commandements du Seigneur, dont la pensée est selon ta pensée, et qui sera triste avec toi si tu fais une fausse couche.
\par 13 Et que le conseil de ton cœur demeure ferme, car il n'y a personne qui te soit plus fidèle que lui.
\par 14 Car l'esprit d'un homme a parfois l'habitude de lui dire qu'il y a plus de sept sentinelles assises au sommet d'une haute tour.
\par 15 Et surtout priez le Très-Haut, afin qu'il dirige votre chemin dans la vérité.
\par 16 Que la raison prévienne toute entreprise, et le conseil avant toute action.
\par 17 Le visage est un signe de changement du cœur.
\par 18 Quatre sortes de choses apparaissent : le bien et le mal, la vie et la mort ; mais la langue règne sur elles continuellement.
\par 19 Il y a quelqu'un qui est sage et qui enseigne à beaucoup, et pourtant cela ne lui profite pas.
\par 20 Il y en a un qui fait preuve de sagesse en paroles et qui est haï : il sera privé de toute nourriture.
\par 21 Car la grâce ne lui est pas donnée de la part du Seigneur, parce qu'il est privé de toute sagesse.
\par 22 Un autre est sage envers lui-même ; et les fruits de l'intelligence sont louables dans sa bouche.
\par 23 Un homme sage instruit son peuple ; et les fruits de sa compréhension ne manquent pas.
\par 24 Un homme sage sera rempli de bénédiction ; et tous ceux qui le verront le trouveront heureux.
\par 25 Les jours de la vie de l'homme sont comptés, mais les jours d'Israël sont innombrables.
\par 26 Le sage héritera de la gloire parmi son peuple, et son nom sera perpétuel.
\par 27 Mon fils, éprouve ton âme dans ta vie, et vois ce qui est mauvais pour elle, et ne lui donne pas cela.
\par 28 Car tout ne sert pas à tous les hommes, et chaque âme ne prend pas plaisir à tout.
\par 29 Ne sois pas insatiable de tout ce qui est délicat, ni trop gourmand de viandes :
\par 30 Car l'excès de viande amène la maladie, et la suralimentation se transforme en colère.
\par 31 Beaucoup ont péri à cause de la suralimentation ; mais celui qui prend garde prolonge sa vie.

\chapter{38}

\par 1 Honorez un médecin avec l'honneur qui lui est dû pour les usages que vous pourrez en avoir : car l'Éternel l'a créé.
\par 2 Car du Très-Haut vient la guérison, et il recevra l'honneur du roi.
\par 3 L'habileté du médecin relèvera sa tête, et aux yeux des grands hommes il sera en admiration.
\par 4 L'Éternel a créé des médicaments à partir de la terre ; et celui qui est sage ne les détestera pas.
\par 5 L'eau n'a-t-elle pas été adoucie avec du bois, afin que sa vertu soit connue ?
\par 6 Et il a donné aux hommes de l'habileté, afin qu'ils soient honorés dans leurs œuvres merveilleuses.
\par 7 Avec de tels hommes, il guérit [les hommes] et enlève leurs souffrances.
\par 8 L'apothicaire en fait une confection ; et ses œuvres n'ont pas de fin ; et de lui vient la paix sur toute la terre,
\par 9 Mon fils, dans ta maladie, ne sois pas négligent ; mais prie le Seigneur, et il te guérira.
\par 10 Renonce au péché, ordonne tes mains, et purifie ton cœur de toute méchanceté.
\par 11 Donnez une odeur douce et un souvenir de fleur de farine ; et faites une offrande grasse, comme n'étant pas.
\par 12 Alors cède la place au médecin, car l'Éternel l'a créé : qu'il ne s'éloigne pas de toi, car tu as besoin de lui.
\par 13 Il y a un temps où entre leurs mains il y a un bon succès.
\par 14 Car ils prieront aussi le Seigneur, afin qu'il fasse prospérer ce qu'ils donnent comme facilité et remède pour prolonger la vie.
\par 15 Celui qui pèche devant son Créateur, qu'il tombe entre les mains du médecin.
\par 16 Mon fils, que des larmes coulent sur les morts, et commence à te lamenter, comme si tu avais toi-même subi un grand mal ; puis couvrez son corps selon la coutume, et ne négligez pas son enterrement.
\par 17 Pleure amèrement, et pousse de grands gémissements, et use de lamentations, autant qu'il en est digne, et cela pendant un jour ou deux, de peur qu'on ne parle mal de toi ; puis console-toi de ta lourdeur.
\par 18 Car c'est de la lourdeur que vient la mort, et la lourdeur du cœur brise la force.
\par 19 Dans l'affliction aussi la tristesse demeure, et la vie des pauvres est la malédiction du cœur.
\par 20 Ne prenez pas à cœur la lourdeur : chassez-la, et participez à la dernière fin.
\par 21 Ne l'oublie pas, car il n'y a plus de retour possible : tu ne lui feras pas de bien, mais tu te feras du mal.
\par 22 Souviens-toi de mon jugement ; car le tien aussi sera pareil ; hier pour moi, et aujourd'hui pour toi.
\par 23 Quand le mort repose, que son souvenir repose ; et sois consolé pour lui, lorsque son Esprit s'éloigne de lui.
\par 24 La sagesse d'un homme instruit vient du loisir ; et celui qui a peu d'affaires deviendra sage.
\par 25 Comment peut-il acquérir la sagesse celui qui tient la charrue, qui se glorifie de l'aiguillon, qui conduit les bœufs et s'occupe de leurs travaux, et qui parle des bœufs ?
\par 26 Il s'efforce de creuser des sillons ; et il s'applique à donner du fourrage aux vaches.
\par 27 Ainsi, tout charpentier et ouvrier qui travaille nuit et jour, et ceux qui taillent et gravent des sceaux, et qui s'appliquent à faire une grande variété, et se livrent à des images contrefaites, et veillent à achever un ouvrage :
\par 28 Le forgeron aussi, assis près de l'enclume, et considérant le travail du fer, la vapeur du feu consume sa chair, et il combat avec la chaleur du fourneau : le bruit du marteau et de l'enclume est toujours dans ses oreilles, et ses yeux regardent toujours le modèle de la chose qu'il fait ; il décide de terminer son travail et veille à le polir parfaitement :
\par 29 Ainsi en est-il du potier assis à son ouvrage et faisant tourner le tour avec ses pieds, qui est toujours soigneusement mis à son ouvrage et qui fait tout son ouvrage par numéro ;
\par 30 Il façonne l'argile avec son bras, et il étend sa force devant ses pieds ; il s'applique à le conduire ; et il s'applique à faire nettoyer la fournaise :
\par 31 Tous s'en remettent à leurs mains, et chacun est sage dans son œuvre.
\par 32 Sans cela, aucune ville ne pourrait être habitée ; et ils n'habiteraient pas où ils voudraient, ni ne monteraient et ne descendraient.
\par 33 Ils ne seront pas recherchés dans le conseil public, ni ne siégeront en haut de l'assemblée ; ils ne siégeront pas sur le siège des juges, et ne comprendront pas la sentence du jugement ; ils ne peuvent pas déclarer la justice et le jugement ; et on ne les trouvera pas là où l'on dit des paraboles.
\par 34 Mais ils maintiendront l'état du monde, et [tout] leur désir est dans l'ouvrage de leur métier.

\chapter{39}

\par 1 Mais celui qui consacre son esprit à la loi du Très-Haut et s'occupe de sa méditation recherchera la sagesse de tous les anciens et s'occupera des prophéties.
\par 2 Il gardera les paroles des hommes renommés ; et là où se trouvent les paraboles subtiles, il sera là aussi.
\par 3 Il recherchera les secrets des phrases graves et se familiarisera avec de sombres paraboles.
\par 4 Il servira parmi les grands et comparaîtra devant les princes ; il voyagera à travers des pays étrangers ; car il a éprouvé le bien et le mal parmi les hommes.
\par 5 Il donnera son cœur pour recourir de bonne heure au Seigneur qui l'a créé, et priera devant le Très-Haut, et ouvrira la bouche en prière et fera des supplications pour ses péchés.
\par 6 Quand le grand Seigneur le voudra, il sera rempli de l'esprit de compréhension : il prononcera des phrases sages et rendra grâce au Seigneur dans sa prière.
\par 7 Il dirigera ses conseils et sa connaissance, et méditera ses secrets.
\par 8 Il montrera ce qu'il a appris, et se glorifiera de la loi de l'alliance du Seigneur.
\par 9 Beaucoup loueront son intelligence ; et aussi longtemps que le monde durera, il ne sera pas effacé ; son mémorial ne disparaîtra pas, et son nom vivra de génération en génération.
\par 10 Les nations feront connaître sa sagesse, et l'assemblée proclamera ses louanges.
\par 11 S'il meurt, il laissera un nom plus grand que mille ; et s'il vit, il l'augmentera.
\par 12 J'ai encore quelque chose à dire sur lequel j'ai réfléchi ; car je suis rempli comme la lune au maximum.
\par 13 Écoutez-moi, vous, saints enfants, et bourgeonnez comme une rose qui pousse près du ruisseau des champs :
\par 14 Et donnez une douce odeur comme l'encens, et fleurissez comme un lis, répandez une odeur, et chantez un cantique de louange, bénissez le Seigneur dans toutes ses œuvres.
\par 15 Magnifiez son nom, et manifestez sa louange avec les chants de vos lèvres et avec les harpes, et en le louant, vous direz de cette manière :
\par 16 Toutes les œuvres de l'Éternel sont extrêmement bonnes, et tout ce qu'il commande sera accompli au temps convenable.
\par 17 Et personne ne peut dire : Qu'est-ce que c'est ? pourquoi est-ce ? car au moment opportun, ils seront tous recherchés : à son commandement les eaux se dressèrent comme un tas, et aux paroles de sa bouche des récipients d'eaux.
\par 18 Tout ce qui lui plaît est fait sur son commandement ; et personne ne peut empêcher, quand il veut sauver.
\par 19 Les œuvres de toute chair sont devant lui, et rien ne peut être caché à ses yeux.
\par 20 Il voit d'éternité en éternité ; et il n'y a rien de merveilleux devant lui.
\par 21 Un homme n'a pas besoin de dire : Qu'est-ce que c'est ? pourquoi est-ce ? car il a fait toutes choses pour leur usage.
\par 22 Sa bénédiction couvrait la terre ferme comme un fleuve, et l'arrosait comme un déluge.
\par 23 Comme il a transformé les eaux en sel, ainsi les païens hériteront de sa colère.
\par 24 Comme ses voies sont claires pour les saints ; ainsi sont-ils des pierres d'achoppement pour les méchants.
\par 25 Car les bons sont les bonnes choses créées dès le commencement : les mauvaises sont pour les pécheurs.
\par 26 Les choses principales pour toute l'usage de la vie de l'homme sont l'eau, le feu, le fer et le sel, la farine de blé, le miel, le lait et le sang du raisin, l'huile et les vêtements.
\par 27 Toutes ces choses sont un bien pour les pieux ; ainsi, pour les pécheurs, elles deviennent un mal.
\par 28 Il y a des esprits créés pour la vengeance, qui dans leur fureur s'étendent sur des coups douloureux ; au moment de la destruction, ils déploient leur force et apaisent la colère de celui qui les a créés.
\par 29 Le feu, la grêle, la famine et la mort, tout cela a été créé pour la vengeance ;
\par 30 Dents de bêtes sauvages, de scorpions, de serpents et d'épée qui punissent les méchants jusqu'à la perdition.
\par 31 Ils se réjouiront de son commandement, et ils seront prêts sur terre, quand il le faudra ; et quand leur heure sera venue, ils ne transgresseront pas sa parole.
\par 32 C'est pourquoi, dès le début, j'ai été résolu, j'ai pensé à ces choses, et je les ai écrites.
\par 33 Toutes les œuvres de l'Éternel sont bonnes, et il donnera tout ce qui est nécessaire au temps convenable.
\par 34 De sorte qu'un homme ne peut pas dire : Ceci est pire que cela ; car avec le temps, ils seront tous bien approuvés.
\par 35 C'est pourquoi louez le Seigneur de tout votre cœur et de toute votre bouche, et bénissez le nom du Seigneur.

\chapter{40}

\par 1 Un grand travail est créé pour chaque homme, et un joug pesant est sur les fils d'Adam, depuis le jour où ils sortent du ventre de leur mère, jusqu'au jour où ils retournent à la mère de toutes choses.
\par 2 Leur imagination des choses à venir et du jour de la mort, [trouble] leurs pensées et [cause] la peur du cœur ;
\par 3 Depuis celui qui est assis sur un trône de gloire, jusqu'à celui qui est humilié dans la terre et la cendre ;
\par 4 Depuis celui qui porte de la pourpre et une couronne, jusqu'à celui qui est vêtu d'une robe de lin.
\par 5 La colère, et l'envie, le trouble et l'inquiétude, la peur de la mort, et la colère, et les querelles, et pendant le temps de repos sur son lit, sa nuit de sommeil, changent sa connaissance.
\par 6 Son repos est peu ou rien, et ensuite il dort, comme pendant un jour de veille, troublé dans la vision de son cœur, comme s'il avait échappé à une bataille.
\par 7 Quand tout est en sécurité, il se réveille et s'étonne que la peur n'était rien.
\par 8 [De telles choses arrivent] à toute chair, tant aux hommes qu'aux bêtes, et cela est sept fois plus grave pour les pécheurs.
\par 9 La mort et le sang versé, les conflits et l'épée, les calamités, la famine, la tribulation et le fléau ;
\par 10 Ces choses ont été créées pour les méchants, et c'est à cause d'eux qu'est venu le déluge.
\par 11 Tout ce qui est de la terre retournera à la terre, et ce qui est des eaux retournera à la mer.
\par 12 Tous les pots-de-vin et toutes les injustices seront effacés, mais la vérité durera à jamais.
\par 13 Les biens des injustes seront à sec comme une rivière, et disparaîtront avec bruit, comme un grand tonnerre sous la pluie.
\par 14 Tandis qu'il ouvre la main, il se réjouira; ainsi les transgresseurs seront réduits à néant.
\par 15 Les enfants des impies ne produiront pas beaucoup de branches, mais ils seront comme des racines impures sur un rocher dur.
\par 16 La mauvaise herbe qui pousse sur chaque eau et sur chaque rive d'une rivière sera arrachée avant toute herbe.
\par 17 La bonté est comme un jardin très fertile, et la miséricorde dure à jamais.
\par 18 Travailler et se contenter de ce qu'un homme possède est une vie douce; mais celui qui trouve un trésor est au-dessus de tous deux.
\par 19 Les enfants et la construction d'une ville perpétuent le nom d'un homme; mais une femme irréprochable est au-dessus de tous deux.
\par 20 Le vin et la musique réjouissent le cœur, mais l'amour de la sagesse est au-dessus d'eux deux.
\par 21 La flûte et le psaltérion font une douce mélodie, mais une langue agréable est au-dessus d'eux deux.
\par 22 Ton œil désire la faveur et la beauté, mais plus que le blé tant qu'il est vert.
\par 23 L'ami et le compagnon ne font jamais de mal : mais au-dessus de l'un et de l'autre il y a une femme et son mari.
\par 24 Les frères et le secours sont contre le temps de détresse ; mais l'aumône rapportera plus qu'eux deux.
\par 25 L'or et l'argent assurent la stabilité du pied, mais le conseil est au-dessus de l'un et de l'autre.
\par 26 La richesse et la force élèvent le cœur ; mais la crainte du Seigneur est au-dessus d'eux deux : la crainte du Seigneur ne manque pas, et il n'est pas nécessaire de chercher de l'aide.
\par 27 La crainte du Seigneur est un jardin fécond et le couvre au-dessus de toute gloire.
\par 28 Mon fils, ne mène pas une vie de mendiant ; car il vaut mieux mourir que mendier.
\par 29 La vie de celui qui dépend de la table d'autrui ne doit pas être comptée pour une vie ; car il se souille avec la nourriture des autres ; mais un homme sage et bien nourri s'en méfiera.
\par 30 La mendicité est douce dans la bouche de l'impudent, mais dans son ventre il y aura du feu.

\chapter{41}

\par 1 Ô mort, combien le souvenir de toi est amer pour l'homme qui vit en repos dans ses possessions, pour l'homme qui n'a rien qui le contrarie et qui a la prospérité en toutes choses ; oui, pour celui qui est encore capable recevoir de la viande!
\par 2 Ô mort, ta sentence est agréable pour celui qui est dans le besoin, et pour celui dont la force faiblit, qui est maintenant dans le dernier siècle et qui est irrité par toutes choses, et pour celui qui désespère et a perdu patience !
\par 3 Ne crains pas la sentence de mort, souviens-toi de ceux qui ont été avant toi et de ceux qui sont venus après ; car telle est la sentence du Seigneur sur toute chair.
\par 4 Et pourquoi es-tu contre le plaisir du Très-Haut ? il n'y a pas d'inquisition dans la tombe, si tu as vécu dix, cent ou mille ans.
\par 5 Les enfants des pécheurs sont des enfants abominables, et ceux qui fréquentent la demeure des impies.
\par 6 L'héritage des enfants des pécheurs périra, et leur postérité aura un opprobre perpétuel.
\par 7 Les enfants se plaindront d'un père impie, parce qu'ils seront insultés à cause de lui.
\par 8 Malheur à vous, hommes impies, qui avez abandonné la loi du Dieu Très-Haut ! car si vous augmentez, ce sera votre destruction :
\par 9 Et si vous naissez, vous naîtrez sous une malédiction ; et si vous mourez, une malédiction sera votre part.
\par 10 Tous les habitants de la terre retourneront sur la terre ; ainsi les impies passeront de la malédiction à la perdition.
\par 11 Le deuil des hommes concerne leurs corps, mais la mauvaise réputation des pécheurs sera effacée.
\par 12 Fais attention à ton nom ; car cela continuera avec toi au-dessus de mille grands trésors d'or.
\par 13 Une bonne vie n'a que peu de jours, mais une bonne réputation dure à toujours.
\par 14 Mes enfants, gardez la discipline en paix : pour une sagesse cachée et un trésor qu'on ne voit pas, quel profit y a-t-il à l'un et à l'autre ?
\par 15 Mieux vaut un homme qui cache sa folie qu'un homme qui cache sa sagesse.
\par 16 Ayez donc honte selon ma parole ; car il n'est pas bon de conserver toute honte ; il n’est pas non plus entièrement approuvé en tout.
\par 17 Ayez honte de la prostitution devant votre père et votre mère, et du mensonge devant un prince et un homme puissant ;
\par 18 D'une offense devant un juge et un dirigeant; de l'iniquité devant une congrégation et des gens; de traitement injuste devant votre partenaire et ami ;
\par 19 Et du vol en ce qui concerne le lieu où tu séjournes, et en ce qui concerne la vérité de Dieu et son alliance ; et t'appuyer du coude sur la viande ; et du mépris de donner et de prendre ;
\par 20 Et de silence devant ceux qui te saluent ; et regarder une prostituée;
\par 21 Et pour détourner ta face de ton parent ; ou pour emporter une portion ou un cadeau ; ou contempler la femme d'un autre homme.
\par 22 Ou être trop occupé avec sa servante, et ne pas s'approcher de son lit ; ou de discours de reproche devant des amis ; et après avoir donné, ne fais pas de reproches ;
\par 23 Ou de répéter et de répéter ce que tu as entendu ; et de révélation de secrets.
\par 24 Ainsi tu seras vraiment honteux et tu trouveras grâce devant tous les hommes.

\chapter{42}

\par 1 N'aie pas honte de ces choses, et n'accepte que personne ne péche par là :
\par 2 De la loi du Très-Haut et de son alliance ; et de jugement pour justifier les impies ;
\par 3 De rendre compte à tes partenaires et à tes voyageurs ; ou du don de l'héritage d'amis ;
\par 4 De l'exactitude de l'équilibre et des poids ; ou d'obtenir beaucoup ou peu;
\par 5 Et des ventes indifférentes des marchands ; de beaucoup de correction des enfants; et faire saigner le côté d'un mauvais serviteur.
\par 6 Il est bon de garder la sécurité là où se trouve une femme méchante ; et tais-toi, là où se trouvent beaucoup de mains.
\par 7 Livrez toutes choses en nombre et en poids ; et mets par écrit tout ce que tu donnes ou que tu reçois.
\par 8 N'aie pas honte d'informer les insensés et les insensés, et les personnes âgées qui luttent avec les jeunes : ainsi tu seras vraiment instruit et approuvé de tous les hommes vivants.
\par 9 Le père veille pour la fille, quand personne ne le sait ; et les soins qu'on lui porte lui enlèvent le sommeil : quand elle est jeune, de peur qu'elle ne passe la fleur de son âge ; et étant mariée, de peur qu'elle ne soit haïe :
\par 10 Dans sa virginité, de peur qu'elle ne soit souillée et qu'elle ne devienne enceinte dans la maison de son père ; et avoir un mari, de peur qu'elle ne se conduise mal ; et quand elle sera mariée, de peur qu'elle ne devienne stérile.
\par 11 Prends garde à une fille impudente, de peur qu'elle ne fasse de toi la risée de tes ennemis, et un sujet de moquerie dans la ville, et un objet d'opprobre parmi le peuple, et qu'elle ne te fasse honte devant la multitude.
\par 12 Ne contemplez pas la beauté de chacun, et ne vous asseyez pas au milieu des femmes.
\par 13 Car des vêtements sort la teigne, et des femmes la méchanceté.
\par 14 Mieux vaut la grossièreté d'un homme qu'une femme courtoise, une femme, dis-je, qui apporte la honte et l'opprobre.
\par 15 Je me souviendrai maintenant des œuvres du Seigneur, et je déclarerai les choses que j'ai vues : Dans les paroles du Seigneur sont ses œuvres.
\par 16 Le soleil qui éclaire regarde toutes choses, et son œuvre est pleine de la gloire du Seigneur.
\par 17 Le Seigneur n'a pas donné le pouvoir aux saints de déclarer toutes ses merveilles, que le Seigneur Tout-Puissant a fermement établies, afin que tout ce qui existe puisse être établi pour sa gloire.
\par 18 Il cherche l'abîme et le cœur, et examine leurs ruses; car l'Éternel connaît tout ce qu'on peut connaître, et il voit les signes du monde.
\par 19 Il déclare les choses passées et à venir, et révèle les étapes des choses cachées.
\par 20 Aucune pensée ne lui échappe, aucune parole ne lui est cachée.
\par 21 Il a garni les œuvres excellentes de sa sagesse, et il est d'éternité en éternité : rien ne peut lui être ajouté, ni diminuer, et il n'a besoin d'aucun conseiller.
\par 22 Oh, comme toutes ses œuvres sont désirables ! et qu'un homme puisse voir même une étincelle.
\par 23 Toutes ces choses vivent et demeurent pour toujours pour tous les usages, et elles sont toutes obéissantes.
\par 24 Toutes choses sont doubles les unes par rapport aux autres, et il n'a rien rendu imparfait.
\par 25 Une chose établit le bien ou une autre : et qui sera rempli de contempler sa gloire ?

\chapter{43}

\par 1 L'orgueil des hauteurs, le clair firmament, la beauté du ciel, avec son spectacle glorieux ;
\par 2 Le soleil quand il apparaît, déclarant à son lever un instrument merveilleux, œuvre du Très-Haut :
\par 3 A midi, le pays dessèche le pays, et qui peut en supporter la chaleur brûlante ?
\par 4 Un homme qui souffle dans une fournaise travaille avec chaleur, mais le soleil brûle les montagnes trois fois plus ; haletant des vapeurs ardentes et projetant des rayons lumineux, il obscurcit les yeux.
\par 5 Grand est le Seigneur qui l'a fait ; et sur son commandement il court en toute hâte.
\par 6 Il fit aussi de la lune pour servir, en son temps, de déclaration des temps et de signe du monde.
\par 7 De la lune vient le signe des fêtes, une lumière qui décroît dans sa perfection.
\par 8 Le mois porte son nom, augmentant merveilleusement dans son changement, étant un instrument des armées d'en haut, brillant dans le firmament du ciel ;
\par 9 La beauté du ciel, la gloire des étoiles, un ornement qui éclaire les plus hauts lieux du Seigneur.
\par 10 Au commandement du Saint, ils se tiendront en ordre et ne se lasseront jamais pendant leurs veilles.
\par 11 Regarde l'arc-en-ciel et loue celui qui l'a fait ; c'est très beau dans sa luminosité.
\par 12 Il entoure le ciel d'un cercle glorieux, et les mains du Très-Haut l'ont courbé.
\par 13 Par son commandement, il fait tomber la neige et envoie rapidement les éclairs de son jugement.
\par 14 Par là les trésors s'ouvrent, et les nuages ​​s'envolent comme des oiseaux.
\par 15 Par sa grande puissance, il rend les nuages ​​fermes, et les grêlons sont brisés petits.
\par 16 A sa vue les montagnes sont ébranlées, et à sa volonté souffle le vent du sud.
\par 17 Le bruit du tonnerre fait trembler la terre, ainsi que la tempête du nord et le tourbillon. Comme des oiseaux en vol, il disperse la neige, et sa chute est comme l'éclairage des sauterelles.
\par 18 L'œil s'émerveille de la beauté de sa blancheur, et le cœur s'étonne de sa pluie.
\par 19 Il verse aussi la gelée blanche sur la terre comme du sel, et, une fois gelée, elle repose sur des pieux pointus.
\par 20 Lorsque le vent froid du nord souffle et que l'eau se fige en glace, elle demeure sur chaque rassemblement d'eau et revêt l'eau comme d'un pectoral.
\par 21 Il dévore les montagnes, brûle le désert, et consume l'herbe comme un feu.
\par 22 L'un des remèdes actuels est une brume qui arrive rapidement, une rosée qui vient après que la chaleur ait rafraîchi.
\par 23 Par son conseil, il apaise l'abîme et y plante des îles.
\par 24 Ceux qui naviguent sur la mer parlent du danger qu'elle représente ; et quand nous l'entendons avec nos oreilles, nous nous en émerveillons.
\par 25 Car là se trouvent des œuvres étranges et merveilleuses, une variété de toutes sortes d'animaux et de baleines créées.
\par 26 C'est par lui que leur fin est couronnée de succès, et c'est par sa parole que toutes choses subsistent.
\par 27 Nous pouvons parler beaucoup, mais ne pas être à la hauteur : c'est pourquoi en somme, il est tout.
\par 28 Comment pourrons-nous le magnifier ? car il est grand au-dessus de toutes ses œuvres.
\par 29 Le Seigneur est terrible et très grand, et sa puissance est merveilleuse.
\par 30 Quand vous glorifiez le Seigneur, exaltez-le autant que vous le pouvez ; car même encore, il dépassera de loin : et lorsque vous l'exalterez, déploiez toutes vos forces et ne vous lassez pas ; car vous ne pouvez jamais aller assez loin.
\par 31 Qui l'a vu, pour qu'il nous le dise ? et qui peut le magnifier tel qu'il est ?
\par 32 Il y a encore des choses cachées plus grandes que celles-ci, car nous n'avons vu que quelques-unes de ses œuvres.
\par 33 Car l'Éternel a fait toutes choses ; et aux pieux il a donné la sagesse.

\chapter{44}

\par 1 Louons maintenant les hommes célèbres et nos pères qui nous ont engendrés.
\par 2 Le Seigneur a opéré par eux une grande gloire par sa grande puissance dès le commencement.
\par 3 Ceux qui régnaient sur leurs royaumes, des hommes réputés pour leur puissance, donnant des conseils par leur intelligence et annonçant des prophéties :
\par 4 Dirigeants du peuple par leurs conseils et par leur science, dignes du peuple, leurs instructions sont sages et éloquentes :
\par 5 Ceux qui ont découvert des airs musicaux et récité des vers écrits :
\par 6 Hommes riches dotés de capacités, vivant paisiblement dans leurs habitations :
\par 7 Tous ceux-ci furent honorés dans leurs générations et furent la gloire de leur temps.
\par 8 Il y en a parmi eux qui ont laissé un nom derrière eux, afin que leurs louanges soient connues.
\par 9 Et il y en a qui n'ont pas de mémorial ; qui ont péri, comme s'ils ne l'avaient jamais été ; et sont devenus comme s'ils n'étaient jamais nés ; et leurs enfants après eux.
\par 10 Mais c'étaient des hommes miséricordieux, dont la justice n'a pas été oubliée.
\par 11 À leur postérité restera continuellement un bon héritage, et leurs enfants seront dans le cadre de l'alliance.
\par 12 C'est à cause d'eux que leur postérité demeure ferme, et leurs enfants.
\par 13 Leur postérité demeurera à jamais, et leur gloire ne sera pas effacée.
\par 14 Leurs corps sont enterrés en paix ; mais leur nom vit éternellement.
\par 15 Le peuple parlera de sa sagesse, et l'assemblée publiera ses louanges.
\par 16 Hénoc plut au Seigneur et fut transporté, étant un exemple de repentir pour toutes les générations.
\par 17 Noé fut trouvé parfait et juste ; au temps de la colère, il fut pris en échange [du monde] ; c'est pourquoi il fut laissé comme un reste sur la terre, lorsque le déluge vint.
\par 18 Une alliance éternelle a été conclue avec lui, afin que toute chair ne périsse plus par le déluge.
\par 19 Abraham était le grand père d'un grand nombre de peuples : nul n'était semblable à lui en gloire ;
\par 20 Qui a observé la loi du Très-Haut et a conclu une alliance avec lui : il a établi l'alliance dans sa chair ; et quand il fut éprouvé, il fut trouvé fidèle.
\par 21 C'est pourquoi il lui assura par serment qu'il bénirait les nations dans sa postérité, qu'il la multiplierait comme la poussière de la terre, qu'il exalterait sa postérité comme les étoiles, et qu'il leur ferait hériter d'une mer à l'autre, et depuis le fleuve jusqu'à l'extrémité du pays.
\par 22 Il établit également avec Isaac [à cause d'Abraham son père] la bénédiction de tous les hommes et l'alliance, et il la fit reposer sur la tête de Jacob. Il le reconnut dans sa bénédiction, lui donna un héritage et partagea ses parts ; Il les répartit entre les douze tribus.

\chapter{45}

\par 1 Et il fit sortir de lui un homme miséricordieux, qui trouva grâce aux yeux de toute chair, Moïse, bien-aimé de Dieu et des hommes, dont le mémorial est béni.
\par 2 Il l'a rendu semblable aux saints glorieux et l'a magnifié, de sorte que ses ennemis le craignaient.
\par 3 Par ses paroles, il fit cesser les prodiges, et il le rendit glorieux aux yeux des rois, et lui donna un commandement pour son peuple, et lui montra une partie de sa gloire.
\par 4 Il l'a sanctifié dans sa fidélité et sa douceur, et l'a choisi entre tous les hommes.
\par 5 Il lui fit entendre sa voix, et le fit entrer dans la nuée sombre, et lui donna des commandements devant sa face, la loi de la vie et de la connaissance, afin qu'il puisse enseigner à Jacob ses alliances et à Israël ses jugements.
\par 6 Il exalta Aaron, un saint homme comme lui, son frère, de la tribu de Lévi.
\par 7 Il fit avec lui une alliance éternelle et lui donna le sacerdoce parmi le peuple ; il l'embellit de jolis ornements et le revêtit d'une robe de gloire.
\par 8 Il lui a revêtu une gloire parfaite ; et il le fortifia avec de riches vêtements, des culottes, une longue robe et l'éphod.
\par 9 Et il l'entoura de grenades et de nombreuses cloches d'or tout autour, afin qu'à son passage il y ait un bruit et un bruit qui pût être entendu dans le temple, en souvenir des enfants de son peuple ;
\par 10 Avec un vêtement sacré, avec de l'or, de la soie bleue et de la pourpre, en ouvrage de broderie, avec un pectoral de jugement, et avec l'urim et le thummim;
\par 11 Avec de l'écarlate torsadé, ouvrage d'artisan habile, avec des pierres précieuses taillées comme des sceaux et serties d'or, ouvrage de bijoutier, avec une écriture gravée pour un mémorial, d'après le nombre des tribus d'Israël.
\par 12 Il plaça sur la mitre une couronne d'or, où était gravée la sainteté, un ornement d'honneur, un ouvrage coûteux, les désirs des yeux, bons et beaux.
\par 13 Avant lui, il n'y en avait pas, et aucun étranger n'en a jamais porté, mais seulement ses enfants et les enfants de ses enfants à perpétuité.
\par 14 Leurs sacrifices seront entièrement consommés chaque jour, deux fois continuellement.
\par 15 Moïse le consacra et l'oignit d'huile sainte : cela lui fut réservé par une alliance éternelle, ainsi qu'à sa postérité, aussi longtemps que les cieux subsisteraient, afin qu'ils le servent et exécutent l'office du sacerdoce et bénisse le peuple en son nom.
\par 16 Il l'a choisi parmi tous les hommes vivants pour offrir à l'Éternel des sacrifices, de l'encens et une odeur agréable, en souvenir, pour faire la réconciliation de son peuple.
\par 17 Il lui donna ses commandements et son autorité dans les statuts des jugements, afin qu'il enseigne à Jacob les témoignages et qu'il instruise Israël dans ses lois.
\par 18 Des étrangers conspirèrent contre lui et le calomnièrent dans le désert, même les hommes qui étaient du côté de Dathan et d'Abiron, et l'assemblée de Coré, avec fureur et colère.
\par 19 L'Éternel vit cela, et cela lui déplut, et dans son indignation furieuse ils furent consumés; il fit sur eux des prodiges, pour les consumer par la flamme ardente.
\par 20 Mais il rendit Aaron plus honorable, et lui donna un héritage, et lui partagea les prémices du revenu ; surtout il préparait du pain en abondance :
\par 21 Car ils mangent des sacrifices de l'Éternel, qu'il a donnés à lui et à sa postérité.
\par 22 Mais il n'avait pas d'héritage dans le pays du peuple, et il n'avait pas non plus de part parmi le peuple, car l'Éternel lui-même est sa part et son héritage.
\par 23 Le troisième en gloire est Phinées, fils d'Éléazar, parce qu'il a eu du zèle dans la crainte de l'Éternel et s'est levé avec un bon courage de cœur lorsque le peuple a fait demi-tour et a fait la réconciliation pour Israël.
\par 24 C'est pourquoi une alliance de paix fut conclue avec lui, afin qu'il soit le chef du sanctuaire et de son peuple, et que lui et sa postérité aient pour toujours la dignité du sacerdoce.
\par 25 Conformément à l'alliance conclue avec David, fils de Jessé, de la tribu de Juda, selon laquelle l'héritage du roi reviendrait à sa postérité seule, de même l'héritage d'Aaron reviendrait aussi à sa postérité.
\par 26 Dieu vous donne la sagesse dans votre cœur pour juger son peuple avec justice, afin que ses bonnes choses ne soient pas abolies et que sa gloire dure à jamais.

\chapter{46}

\par 1 Jésus, le fils de Nave, fut vaillant dans les guerres et fut le successeur de Moïse dans les prophéties, qui, selon son nom, fut rendu grand pour le salut des élus de Dieu et pour se venger des ennemis qui s'élevaient contre eux, afin que il pourrait mettre Israël dans son héritage.
\par 2 Quelle grande gloire il a eu lorsqu'il a levé les mains et étendu son épée contre les villes !
\par 3 Qui avant lui s'y est tenu ainsi ? car le Seigneur lui-même lui a amené ses ennemis.
\par 4 Le soleil n'est-il pas revenu par son intermédiaire ? et un jour n'était-il pas aussi long que deux ?
\par 5 Il invoqua le Seigneur Très-Haut, lorsque les ennemis le pressaient de toutes parts ; et le grand Seigneur l'entendit.
\par 6 Et avec des grêlons d'une puissance puissante, il fit tomber violemment la bataille sur les nations, et lors de la descente [de Beth-horon] il détruisit ceux qui résistaient, afin que les nations connaissent toute leur force, parce qu'il combattit dans le vue du Seigneur, et il suivit le Tout-Puissant.
\par 7 Au temps de Moïse aussi, il fit une œuvre de miséricorde, lui et Caleb, fils de Jephunné, en s'opposant à l'assemblée, en retenant le peuple du péché et en apaisant les méchants qui murmuraient.
\par 8 Et sur six cent mille personnes à pied, ils furent tous deux préservés pour les amener en héritage, jusqu'au pays qui coule du lait et du miel.
\par 9 L'Éternel donna aussi de la force à Caleb, qui resta avec lui jusqu'à sa vieillesse ; de sorte qu'il entra dans les hauts lieux du pays, et que sa postérité l'obtint en héritage.
\par 10 Afin que tous les enfants d'Israël voient qu'il est bon de suivre le Seigneur.
\par 11 Et concernant les juges, tous nommément, dont le cœur ne s'est pas livré à la prostitution ni ne s'est éloigné du Seigneur, que leur mémoire soit bénie.
\par 12 Que leurs os s'épanouissent hors de leur place, et que le nom de ceux qui ont été honorés se perpétue sur leurs enfants.
\par 13 Samuel, le prophète du Seigneur, bien-aimé de son Seigneur, a établi un royaume et a oint des princes sur son peuple.
\par 14 Par la loi de l'Éternel, il jugea l'assemblée, et l'Éternel eut du respect pour Jacob.
\par 15 Par sa fidélité, il s'est révélé être un vrai prophète, et par sa parole, il s'est révélé fidèle en vision.
\par 16 Il invoquait le Seigneur puissant, lorsque ses ennemis le pressaient de toutes parts, lorsqu'il offrait l'agneau de lait.
\par 17 Et l'Éternel tonna du ciel, et avec un grand bruit fit entendre sa voix.
\par 18 Et il détruisit les chefs des Tyriens et tous les princes des Philistins.
\par 19 Et avant son long sommeil, il fit une protestation devant l'Éternel et son oint : Je n'ai pris les biens de personne, pas même une chaussure, et personne ne l'a accusé.
\par 20 Et après sa mort, il prophétisa, et annonça au roi sa fin, et il éleva du haut de la terre sa voix en prophétie, pour effacer la méchanceté du peuple.

\chapter{47}

\par 1 Et après lui se leva Nathan pour prophétiser au temps de David.
\par 2 Comme on retranche la graisse du sacrifice de prospérités, ainsi David fut choisi parmi les enfants d'Israël.
\par 3 Il jouait avec les lions comme avec les chevreaux, et avec les ours comme avec les agneaux.
\par 4 N'a-t-il pas tué un géant, alors qu'il était encore jeune ? et n'a-t-il pas enlevé l'opprobre du peuple, lorsqu'il a levé la main avec la pierre dans la fronde et a réprimé la vantardise de Goliath ?
\par 5 Car il a invoqué le Seigneur Très-Haut ; et il lui donna la force dans sa main droite pour tuer ce vaillant guerrier et pour redresser la force de son peuple.
\par 6 Alors le peuple l'honora par dix mille, et le loua dans les bénédictions du Seigneur, en ce qu'il lui donna une couronne de gloire.
\par 7 Car il a détruit les ennemis de tous côtés, et a réduit à néant les Philistins, ses adversaires, et a brisé leur corne jusqu'à ce jour.
\par 8 Dans toutes ses œuvres, il louait le Saint Très-Haut avec des paroles de gloire ; de tout son cœur il chantait des chansons et il aimait celui qui l'avait créé.
\par 9 Il plaça aussi des chantres devant l'autel, afin qu'ils produisent par leurs voix une douce mélodie, et qu'ils chantent chaque jour des louanges dans leurs chants.
\par 10 Il embellit leurs fêtes et arrangea les moments solennels jusqu'à la fin, afin qu'ils puissent louer son saint nom et que le temple sonne dès le matin.
\par 11 L'Éternel a ôté ses péchés, et il a exalté sa corne pour toujours : il lui a donné une alliance de rois et un trône de gloire en Israël.
\par 12 Après lui surgit un fils sage, et à cause de lui il habita en liberté.
\par 13 Salomon régna dans un temps paisible et fut honoré ; car Dieu fit tout tranquille autour de lui, afin qu'il puisse bâtir une maison en son nom et préparer son sanctuaire pour toujours.
\par 14 Comme tu étais sage dans ta jeunesse et, comme un déluge, rempli d'intelligence !
\par 15 Ton âme couvrait toute la terre, et tu la remplissais de sombres paraboles.
\par 16 Ton nom s'étendit jusqu'aux îles ; et c'est pour ta paix que tu étais aimé.
\par 17 Les pays étaient émerveillés devant toi à cause de tes chants, et de tes proverbes, et de tes paraboles, et de tes interprétations.
\par 18 Au nom du Seigneur Dieu, qui est appelé Seigneur Dieu d'Israël, tu as amassé de l'or comme de l'étain et tu as multiplié de l'argent comme du plomb.
\par 19 Tu as courbé tes reins devant les femmes, et par ton corps tu as été soumis.
\par 20 Tu as souillé ton honneur et tu as pollué ta postérité, de sorte que tu as attiré la colère sur tes enfants et que tu as été attristé à cause de ta folie.
\par 21 Ainsi le royaume fut divisé, et d'Éphraïm régna un royaume rebelle.
\par 22 Mais l'Éternel ne cessera jamais sa miséricorde, et aucune de ses œuvres ne périra, et il n'abolira pas non plus la postérité de ses élus, et il n'enlèvera pas la postérité de celui qui l'aime. C'est pourquoi il a donné un reste. à Jacob, et de lui une racine à David.
\par 23 Ainsi reposa Salomon avec ses pères, et de sa postérité il laissa derrière lui Roboam, même la folie du peuple, et celui qui n'avait pas d'intelligence, qui détournait le peuple par son conseil. Il y avait aussi Jéroboam, fils de Nebath, qui fit pécher Israël et montra à Éphraïm la voie du péché :
\par 24 Et leurs péchés se multiplièrent à l'extrême, au point qu'ils furent chassés du pays.
\par 25 Car ils recherchaient toute méchanceté, jusqu'à ce que la vengeance s'abatte sur eux.

\chapter{48}

\par 1 Alors Elie, le prophète, se leva comme un feu, et sa parole brûlait comme une lampe.
\par 2 Il leur causa une grande famine, et par son zèle il diminua leur nombre.
\par 3 Par la parole de l'Éternel, il ferma le ciel et fit aussi tomber le feu trois fois.
\par 4 Ô Élie, comme tu as été honoré dans tes actes merveilleux ! et qui peut se glorifier comme toi !
\par 5 Qui as ressuscité un mort de la mort, et son âme du lieu des morts, par la parole du Très-Haut ?
\par 6 Qui a fait périr les rois et les hommes honorables de leur lit :
\par 7 Qui a entendu la réprimande de l'Éternel au Sinaï, et en Horeb le jugement de la vengeance ?
\par 8 Qui a consacré des rois pour se venger, et des prophètes pour lui succéder :
\par 9 Qui fut enlevé dans un tourbillon de feu et sur un char aux chevaux de feu :
\par 10 qui ont été destinés à être réprimandés en leur temps, pour apaiser la colère du jugement du Seigneur, avant qu'elle n'éclate en fureur, et pour ramener le cœur du père vers le fils, et pour restaurer les tribus de Jacob.
\par 11 Bienheureux ceux qui t'ont vu et qui ont dormi dans l'amour ; car nous vivrons sûrement.
\par 12 C'était Élie, qui était couvert d'un tourbillon ; et Élisée était rempli de son esprit ; tant qu'il vivait, il n'était ému par la présence d'aucun prince, et personne ne pouvait le soumettre.
\par 13 Aucune parole ne pouvait le vaincre ; et après sa mort, son corps a prophétisé.
\par 14 Il a fait des merveilles dans sa vie, et à sa mort ses œuvres furent merveilleuses.
\par 15 Car tout cela, le peuple ne se repentit pas et ne se détourna pas de ses péchés, jusqu'à ce qu'il soit dépouillé et expulsé de son pays, et dispersé par toute la terre. Pourtant, il restait un petit peuple et un chef dans la maison de David.
\par 16 Parmi eux, les uns faisaient ce qui était agréable à Dieu, et les autres multipliaient les péchés.
\par 17 Ézéchias fortifia sa ville et fit entrer de l'eau au milieu d'elle ; il creusa le rocher dur avec du fer et fit des puits pour les eaux.
\par 18 En son temps, Sennachérib monta, envoya Rabsace, leva la main contre Sion et se vanta fièrement.
\par 19 Alors leurs cœurs et leurs mains tremblaient, et ils souffraient comme des femmes en travail.
\par 20 Mais ils invoquèrent le Seigneur qui est miséricordieux et étendirent leurs mains vers lui ; et aussitôt le Saint les exauça du ciel et les délivra par le ministère d'Esay.
\par 21 Il frappa l'armée des Assyriens, et son ange les détruisit.
\par 22 Car Ezéchias avait fait ce qui plaisait à l'Éternel, et il était fort dans les voies de David, son père, comme le lui avait ordonné Esay, le prophète, qui était grand et fidèle dans sa vision.
\par 23 De son temps, le soleil reculait, et il prolongeait la vie du roi.
\par 24 Il vit par un excellent esprit ce qui devait arriver à la fin, et il consola ceux qui pleuraient à Sion.
\par 25 Il a montré ce qui devait arriver pour toujours, et les choses secrètes ou jamais elles arrivaient.

\chapter{49}

\par 1 Le souvenir de Josias est comme la composition du parfum que fait l'art de l'apothicaire : il est doux comme le miel dans toutes les bouches, et comme la musique dans un festin de vin.
\par 2 Il s'est comporté honnêtement dans la conversion du peuple et a ôté les abominations de l'iniquité.
\par 3 Il a dirigé son cœur vers le Seigneur, et au temps des impies, il a établi le culte de Dieu.
\par 4 Tous, sauf David, Ezéchias et Josias, étaient défectueux : car ils ont abandonné la loi du Très-Haut, même les rois de Juda ont failli.
\par 5 C'est pourquoi il a donné leur puissance à d'autres, et leur gloire à une nation étrangère.
\par 6 Ils incendièrent la ville choisie du sanctuaire, et dévastèrent les rues, selon la prophétie de Jérémie.
\par 7 Car ils lui implorèrent le mal, qui pourtant était un prophète sanctifié dans le sein de sa mère, afin qu'il puisse extirper, affliger et détruire ; et afin qu'il puisse aussi bâtir et planter.
\par 8 C'est Ézéchiel qui eut la glorieuse vision qui lui fut montrée sur le char des chérubins.
\par 9 Car il faisait mention des ennemis sous la figure de la pluie, et il leur dirigeait ceux qui allaient à droite.
\par 10 Et des douze prophètes, que le mémorial soit béni, et que leurs os refleurissent de leur place; car ils ont consolé Jacob et l'ont délivré par une espérance assurée.
\par 11 Comment magnifier Zorobabel ? même lui était comme un sceau à la main droite :
\par 12 Ainsi en était-il de Jésus, fils de Josedec : qui en leur temps bâtit la maison et dressa un temple saint au Seigneur, préparé pour la gloire éternelle.
\par 13 Et parmi les élus se trouvait Néémie, dont la renommée est grande, qui releva pour nous les murs qui étaient tombés, et releva les portes et les barres, et releva nos ruines.
\par 14 Mais sur la terre aucun homme n'a été créé comme Enoch ; car il a été retiré de la terre.
\par 15 Il n'y avait pas non plus de jeune homme né comme Joseph, gouverneur de ses frères, refuge du peuple, dont les os étaient considérés par l'Éternel.
\par 16 Sem et Seth étaient très honorés parmi les hommes, et Adam était donc au-dessus de tout être vivant dans la création.

\chapter{50}

\par 1 Simon, le grand prêtre, fils d'Onias, qui, de son vivant, répara la maison et qui, de son vivant, fortifia le temple :
\par 2 Et par lui fut bâtie, à partir des fondations, la double hauteur, la haute forteresse de la muraille qui entoure le temple :
\par 3 De son temps, la citerne pour recevoir l'eau, ayant un périmètre comme la mer, était recouverte de plaques d'airain :
\par 4 Il prit soin du temple pour qu'il ne tombe pas, et fortifia la ville contre un siège :
\par 5 Comme il fut honoré au milieu du peuple à sa sortie du sanctuaire !
\par 6 Il était comme l'étoile du matin au milieu d'un nuage, et comme la lune en pleine pleine :
\par 7 Comme le soleil brille sur le temple du Très-Haut, et comme l'arc-en-ciel éclairant les nuages ​​clairs :
\par 8 Et comme la fleur des roses au printemps de l'année, comme les lis près des fleuves d'eaux, et comme les branches de l'arbre à encens au temps de l'été :
\par 9 Comme le feu et l'encens dans l'encensoir, et comme un vase d'or battu serti de toutes sortes de pierres précieuses :
\par 10 Et comme un bel olivier qui bourgeonne, et comme un cyprès qui pousse jusqu'aux nuages.
\par 11 Lorsqu'il revêtit la robe d'honneur et fut revêtu de la perfection de la gloire, lorsqu'il monta au saint autel, il rendit honorable le vêtement de sainteté.
\par 12 Lorsqu'il prit les portions des mains des prêtres, il se tenait lui-même près du foyer de l'autel, entouré, comme un jeune cèdre du Liban ; et comme les palmiers l'entouraient, ils l'entouraient.
\par 13 Ainsi étaient tous les fils d'Aaron dans leur gloire, et les offrandes de l'Éternel entre leurs mains, devant toute l'assemblée d'Israël.
\par 14 Et achevant le service à l'autel, afin de parer l'offrande du Très-Haut Tout-Puissant,
\par 15 Il étendit la main vers la coupe, et versa du sang du raisin, il répandit au pied de l'autel une odeur agréable au plus haut roi de tous.
\par 16 Alors les fils d'Aaron poussèrent des cris, et sonnèrent des trompettes d'argent, et firent un grand bruit pour se faire entendre, en souvenir devant le Très-Haut.
\par 17 Alors tout le peuple se hâta et se jeta à terre, la face contre terre, pour adorer leur Seigneur Dieu Tout-Puissant, le Très-Haut.
\par 18 Les chanteurs chantaient aussi des louanges avec leurs voix, avec une grande variété de sons une douce mélodie était là.
\par 19 Et le peuple implorait le Seigneur, le Très-Haut, par la prière devant celui qui est miséricordieux, jusqu'à ce que la solennité du Seigneur soit terminée, et qu'ils aient achevé son service.
\par 20 Puis il descendit et leva les mains sur toute l'assemblée des enfants d'Israël, pour donner de ses lèvres la bénédiction de l'Éternel et pour se réjouir en son nom.
\par 21 Et ils se prosternèrent pour adorer une seconde fois, afin de recevoir une bénédiction du Très-Haut.
\par 22 Maintenant donc, bénissez le Dieu de tous, qui ne fait que des merveilles partout, qui exalte nos jours dès le sein maternel et qui nous traite selon sa miséricorde.
\par 23 Il nous accorde la joie de cœur, et que la paix soit pour toujours dans nos jours en Israël :
\par 24 Qu'il confirmerait sa miséricorde envers nous, et nous délivrerait en son temps !
\par 25 Il y a deux sortes de nations que mon cœur déteste, et la troisième n'est pas une nation :
\par 26 Ceux qui sont assis sur la montagne de Samarie, et ceux qui habitent parmi les Philistins, et ce peuple insensé qui habite à Sichem.
\par 27 Jésus, fils de Sirach de Jérusalem, a écrit dans ce livre l'instruction de l'intelligence et de la connaissance, qui a répandu de son cœur la sagesse.
\par 28 Bienheureux celui qui s'exercera à ces choses ; et celui qui les met dans son cœur deviendra sage.
\par 29 Car s'il les fait, il sera fort en toutes choses ; car la lumière de l'Éternel le conduit, qui donne la sagesse aux pieux. Béni soit le nom du Seigneur pour toujours. Amen, Amen.

\chapter{51}

\par Une prière de Jésus, fils de Sirach.

\par 1 Je te remercierai, Seigneur et Roi, et je te louerai, ô Dieu mon Sauveur : je loue ton nom :
\par 2 Car tu es mon défenseur et mon secours, et tu as préservé mon corps de la destruction, et du piège de la langue calomnieuse, et des lèvres qui forgent des mensonges, et tu as été mon secours contre mes adversaires :
\par 3 Et tu m'as délivré, selon la multitude de tes miséricordes et la grandeur de ton nom, des dents de ceux qui étaient prêts à me dévorer, et des mains de ceux qui cherchaient ma vie, et des multiples les afflictions que j'avais;
\par 4 De l'étouffement du feu de tous côtés, et du milieu du feu que je n'ai pas allumé ;
\par 5 Du fond du ventre de l'enfer, d'une langue impure et de paroles mensongères.
\par 6 Par une accusation adressée au roi par une langue injuste, mon âme s'est approchée jusqu'à la mort, ma vie a été proche de l'enfer en bas.
\par 7 Ils m'entouraient de toutes parts, et il n'y avait personne pour m'aider : j'attendais le secours des hommes, mais il n'y en avait pas.
\par 8 Alors j'ai pensé à ta miséricorde, ô Seigneur, et à tes actes d'autrefois, comment tu délivres ceux qui t'attendent et les sauves des mains des ennemis.
\par 9 Alors j'ai élevé mes supplications de la terre, et j'ai prié pour être délivré de la mort.
\par 10 J'ai invoqué le Seigneur, le Père de mon Seigneur, pour qu'il ne me quitte pas aux jours de ma détresse, et au temps des orgueilleux, quand il n'y avait pas de secours.
\par 11 Je louerai continuellement ton nom, et je chanterai des louanges avec actions de grâces ; et ainsi ma prière a été entendue :
\par 12 Car tu m'as sauvé de la destruction et tu m'as délivré du mauvais temps : c'est pourquoi je te rendrai grâce, je te louerai et je bénirai leur nom, ô Seigneur.
\par 13 Quand j'étais encore jeune, ou lorsque j'allais à l'étranger, je désirais ouvertement la sagesse dans ma prière.
\par 14 J'ai prié pour elle devant le temple, et je la chercherai jusqu'au bout.
\par 15 Depuis la fleur jusqu'à ce que le raisin soit mûr, mon cœur s'est réjoui d'elle; mon pied a marché dans le bon sens, dès ma jeunesse je l'ai cherchée.
\par 16 J'ai courbé un peu l'oreille, je l'ai reçue et j'ai beaucoup appris.
\par 17 J'en ai profité, c'est pourquoi je rendrai gloire à celui qui me donne la sagesse.
\par 18 Car j'ai décidé de faire après elle, et j'ai suivi avec ardeur ce qui est bon ; ainsi je ne serai pas confus.
\par 19 Mon âme a lutté avec elle, et dans mes actes j'ai été exact : j'ai étendu mes mains vers le ciel d'en haut, et j'ai déploré mon ignorance d'elle.
\par 20 J'ai dirigé mon âme vers elle, et je l'ai trouvée dans la pureté : j'ai eu mon cœur uni à elle dès le commencement, c'est pourquoi je ne serai pas abandonné.
\par 21 Mon cœur était troublé en la cherchant : c'est pourquoi j'ai acquis une bonne possession.
\par 22 L'Éternel m'a donné une langue pour ma récompense, et je le louerai avec elle.
\par 23 Approchez-vous de moi, hommes du peuple, et demeurez dans la maison du savoir.
\par 24 Pourquoi êtes-vous lents, et que dites-vous de ces choses, puisque vos âmes ont très soif ?
\par 25 J'ouvris la bouche et dis : Achetez-la pour vous sans argent.
\par 26 Mettez votre cou sous le joug, et que votre âme reçoive l'instruction : elle est difficile à trouver.
\par 27 Voyez de vos yeux, comme je n'ai que peu de travail et que je me suis procuré beaucoup de repos.
\par 28 Apprenez avec une grande somme d'argent, et obtenez beaucoup d'or grâce à elle.
\par 29 Que ton âme se réjouisse de sa miséricorde, et n'aie pas honte de ses louanges.
\par 30 Travaillez tôt à votre travail, et en son temps il vous donnera votre récompense.

\end{document}