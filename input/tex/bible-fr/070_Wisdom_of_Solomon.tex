\begin{document}

\title{Sagesse}


\chapter{1}

\par 1 Aimez la justice, vous qui êtes juges de la terre : pensez au Seigneur d'un bon cœur et cherchez-le dans la simplicité du cœur.
\par 2 Car il sera trouvé par ceux qui ne le tentent pas ; et se montre à ceux qui ne se méfient pas de lui.
\par 3 Car les pensées rebelles s'éloignent de Dieu, et sa puissance, lorsqu'elle est éprouvée, reprend les insensés.
\par 4 Car la sagesse n'entrera pas dans une âme méchante ; ni demeurer dans un corps soumis au péché.
\par 5 Car le Saint-Esprit de discipline fuira la tromperie et s'éloignera des pensées qui sont sans intelligence, et ne demeurera pas lorsque l'injustice surviendra.
\par 6 Car la sagesse est un esprit aimant ; et il n'acquittera pas un blasphémateur de ses paroles : car Dieu est témoin de ses rênes, et véritable spectateur de son cœur, et auditeur de sa langue.
\par 7 Car l'Esprit du Seigneur remplit le monde, et ce qui contient toutes choses a la connaissance de la voix.
\par 8 C'est pourquoi celui qui dit des choses injustes ne peut être caché ; et la vengeance, lorsqu'elle punit, ne passe pas par lui.
\par 9 Car l'inquisition sera transformée en conseils des impies, et le son de ses paroles parviendra à l'Éternel pour la manifestation de ses mauvaises actions.
\par 10 Car l'oreille de la jalousie entend tout, et le bruit des murmures n'est pas caché.
\par 11 Gardez-vous donc des murmures, qui ne servent à rien ; et retiens ta langue de la médisance, car il n'y a pas de parole si secrète qui soit vaine, et la bouche qui ment tue l'âme.
\par 12 Ne cherchez pas la mort dans l'erreur de votre vie, et ne vous attirez pas la destruction par les œuvres de vos mains.
\par 13 Car Dieu n'a pas créé la mort, et il n'a pas non plus pris plaisir à la destruction des vivants.
\par 14 Car il a créé toutes choses afin qu'elles aient leur existence ; et les générations du monde étaient en bonne santé ; et il n'y a pas en eux de poison de destruction, ni de royaume de mort sur la terre :
\par 15 (Car la justice est immortelle :)
\par 16 Mais des hommes impies, par leurs œuvres et par leurs paroles, le leur appelèrent ; car, lorsqu'ils pensaient l'avoir pour ami, ils dévorèrent pour rien, et conclurent une alliance avec lui, parce qu'ils sont dignes d'y prendre part.

\chapter{2}

\par 1 Car les impies disaient, raisonnant avec eux-mêmes, mais sans raison : Notre vie est courte et fastidieuse, et il n'y a pas de remède à la mort d'un homme ; et on ne connaît pas non plus un homme qui soit revenu du tombeau.
\par 2 Car nous sommes nés à toute aventure : et nous serons désormais comme si nous n'avions jamais été : car le souffle dans nos narines est comme une fumée, et une petite étincelle dans le mouvement de notre cœur :
\par 3 Qui étant éteint, notre corps sera transformé en cendres, et notre esprit disparaîtra comme l'air doux,
\par 4 Et notre nom sera oublié avec le temps, et personne ne se souviendra de nos œuvres, et notre vie passera comme la trace d'un nuage et sera dispersée comme une brume chassée par les rayons du soleil. , et vaincu par sa chaleur.
\par 5 Car notre temps est une ombre qui passe ; et après notre fin, il n'y a pas de retour, car c'est scellé, de sorte que personne ne reviendra.
\par 6 Allons donc, jouissons des bonnes choses qui sont présentes : et utilisons promptement les créatures comme dans la jeunesse.
\par 7 Remplis-nous de vin et d'onguents précieux, et qu'aucune fleur du printemps ne passe près de nous :
\par 8 Couronnons-nous de boutons de roses, avant qu'ils ne soient flétris :
\par 9 Qu'aucun de nous ne parte sans sa part de notre volupté : laissons partout des témoignages de notre joie : car telle est notre part, et tel est notre lot.
\par 10 Opprimons le pauvre juste, n'épargnons pas la veuve, et ne révérons pas les vieux cheveux gris des vieillards.
\par 11 Que notre force soit la loi de la justice ; car ce qui est faible ne vaut rien.
\par 12 C'est pourquoi tenons des embûches aux justes ; parce qu'il n'est pas pour notre tour, et qu'il est tout à fait contraire à nos actes : il nous reproche nos offenses à la loi, et s'oppose à notre infamie les transgressions de notre éducation.
\par 13 Il prétend avoir la connaissance de Dieu, et il se dit enfant du Seigneur.
\par 14 Il a été fait pour reprendre nos pensées.
\par 15 Il nous est pénible même à voir; car sa vie n'est pas comme celle des autres hommes, ses voies sont d'une autre manière.
\par 16 Nous sommes considérés par lui comme des contrefaçons : il s'abstient de nos voies comme de la souillure ; il prononce la fin du juste pour être béni, et se vante que Dieu est son père.
\par 17 Voyons si ses paroles sont vraies, et prouvons ce qui arrivera à sa fin.
\par 18 Car si le juste est fils de Dieu, il l'aidera et le délivrera de la main de ses ennemis.
\par 19 Examinons-le avec méchanceté et torture, afin de connaître sa douceur et de prouver sa patience.
\par 20 Condamnons-le d'une mort honteuse, car c'est par sa propre parole qu'il sera respecté.
\par 21 Ils ont imaginé de telles choses, et ils ont été trompés, car leur propre méchanceté les a aveuglés.
\par 22 Quant aux mystères de Dieu, ils ne les connaissaient pas ; ils n'espéraient pas le salaire de la justice, ni ne discernaient une récompense pour les âmes irréprochables.
\par 23 Car Dieu a créé l'homme pour qu'il soit immortel, et il l'a fait être l'image de sa propre éternité.
\par 24 Mais c'est à cause de l'envie du diable que la mort est entrée dans le monde, et ceux qui tiennent à son côté la trouvent.

\chapter{3}

\par 1 Mais les âmes des justes sont entre les mains de Dieu, et aucun tourment ne les touchera.
\par 2 Aux yeux des insensés, ils semblaient mourir ; et leur départ est pris pour de la misère,
\par 3 Et leur départ de nous sera une destruction totale; mais ils sont en paix.
\par 4 Car, bien qu'ils soient punis aux yeux des hommes, leur espérance est pourtant pleine d'immortalité.
\par 5 Et ayant été un peu châtiés, ils seront grandement récompensés : car Dieu les a éprouvés et les a trouvés dignes de lui-même.
\par 6 Il les éprouva comme de l'or dans la fournaise, et il les reçut en holocauste.
\par 7 Et au temps de leur visite, ils brilleront et courront çà et là comme des étincelles parmi le chaume.
\par 8 Ils jugeront les nations et domineront sur les peuples, et leur Seigneur régnera pour toujours.
\par 9 Ceux qui mettent leur confiance en lui comprendront la vérité ; et ceux qui sont fidèles dans l'amour demeureront avec lui ; car la grâce et la miséricorde sont pour ses saints, et il a soin de ses élus.
\par 10 Mais les impies seront punis selon leurs propres imaginations, qui ont négligé les justes et abandonné le Seigneur.
\par 11 Car celui qui méprise la sagesse et la culture est malheureux, et son espérance est vaine, ses travaux infructueux et ses œuvres inutiles.
\par 12 Leurs femmes sont insensées et leurs enfants méchants :
\par 13 Leur postérité est maudite. C'est pourquoi bienheureuse la stérile qui n'est pas souillée, qui n'a pas connu le lit du péché : elle portera du fruit dans la visite des âmes.
\par 14 Et bienheureux est l'eunuque, qui de ses mains n'a commis aucune iniquité, ni imaginé de mauvaises choses contre Dieu ; car il lui sera donné le don spécial de la foi, et un héritage dans le temple du Seigneur plus agréable à son esprit.
\par 15 Car le fruit des bons travaux est glorieux, et la racine de la sagesse ne tombera jamais.
\par 16 Quant aux enfants des adultères, ils ne parviendront pas à leur perfection, et la semence d'un lit injuste sera déracinée.
\par 17 Car, même s'ils vivent longtemps, ils ne seront rien considéré, et leur dernier âge sera sans honneur.
\par 18 Ou, s'ils meurent rapidement, ils n'ont ni espoir, ni réconfort au jour de l'épreuve.
\par 19 Car horrible est la fin de la génération injuste.

\chapter{4}

\par 1 Mieux vaut ne pas avoir d'enfants et avoir de la vertu, car son mémorial est immortel, parce qu'il est connu de Dieu et des hommes.
\par 2 Lorsqu'elle est présente, les hommes en prennent exemple ; et quand il disparaît, ils le désirent : il porte une couronne et triomphe pour toujours, ayant obtenu la victoire, luttant pour des récompenses sans souillure.
\par 3 Mais la couvée multipliée des impies ne prospérera pas, ne s'enracinera pas profondément sur des boutures bâtardes, ni ne posera de fondations solides.
\par 4 Car bien qu'ils fleurissent dans les branches pour un temps ; mais ils ne resteront pas les derniers, ils seront secoués par le vent, et par la force des vents ils seront déracinés.
\par 5 Les branches imparfaites seront cassées, leurs fruits seront inutiles, non mûrs pour être mangés, oui, satisfaisants pour rien.
\par 6 Car les enfants nés de lits illégaux sont témoins de la méchanceté contre leurs parents lors de leur épreuve.
\par 7 Mais même si le juste est empêché par la mort, il sera néanmoins en repos.
\par 8 Car un âge honorable n'est pas celui qui dure dans le temps, ni celui qui se mesure en nombre d'années.
\par 9 Mais la sagesse est pour les hommes les cheveux gris, et une vie sans tache est la vieillesse.
\par 10 Il plut à Dieu et fut aimé de lui ; de sorte que vivant parmi les pécheurs, il fut transporté.
\par 11 Oui, il fut rapidement emmené, de peur que la méchanceté n'altère son intelligence, ou que la tromperie ne séduise son âme.
\par 12 Car l'envoûtement de la méchanceté obscurcit les choses honnêtes ; et l'errance de la concupiscence mine l'esprit simple.
\par 13 Lui, étant rendu parfait en peu de temps, a accompli longtemps :
\par 14 Car son âme plut à l'Éternel; c'est pourquoi il se hâta de l'éloigner du milieu des méchants.
\par 15 Cela, le peuple l'a vu, et ne l'a pas compris, et ils n'ont pas non plus mis cela dans leur esprit, que sa grâce et sa miséricorde sont avec ses saints, et qu'il a du respect pour ses élus.
\par 16 Ainsi, les justes morts condamneront les impies qui sont vivants ; et la jeunesse qui est bientôt perfectionnée, les nombreuses années et la vieillesse des injustes.
\par 17 Car ils verront la fin du sage, et ils ne comprendront pas ce que Dieu dans son conseil a décrété de lui, et dans quel but l'Éternel l'a mis en sécurité.
\par 18 Ils le verront et le mépriseront ; mais Dieu se moquera d'eux : et ils seront désormais une vile carcasse et un opprobre parmi les morts pour toujours.
\par 19 Car il les déchirera et les jettera tête baissée, et ils resteront muets ; et il les ébranlera depuis le fondement ; et ils seront complètement dévastés et seront dans le chagrin ; et leur mémorial périra.
\par 20 Et lorsqu'ils rendront compte de leurs péchés, ils viendront avec crainte, et leurs propres iniquités les convaincront en face.

\chapter{5}

\par 1 Alors le juste se tiendra avec une grande audace devant ceux qui l'ont affligé et qui ne tiennent aucun compte de ses travaux.
\par 2 Quand ils le verront, ils seront troublés par une peur terrible, et seront étonnés de l'étrangeté de son salut, si bien au-delà de tout ce qu'ils attendaient.
\par 3 Et ceux qui se repentent et gémissent d'angoisse spirituelle diront en eux-mêmes : C'était lui, dont nous avions quelquefois en dérision, et un proverbe d'opprobre :
\par 4 Nous, les insensés, considérions sa vie comme une folie et sa fin comme sans honneur :
\par 5 Comment est-il compté parmi les enfants de Dieu, et son sort est-il parmi les saints !
\par 6 C'est pourquoi nous nous sommes égarés du chemin de la vérité, et la lumière de la justice ne nous a pas brillé, et le soleil de la justice ne s'est pas levé sur nous.
\par 7 Nous nous sommes fatigués dans la voie de la méchanceté et de la destruction ; oui, nous avons traversé des déserts, où il n'y avait aucun chemin ; mais quant à la voie du Seigneur, nous ne l'avons pas connue.
\par 8 À quoi nous a servi l'orgueil ? ou à quoi nous sert la richesse grâce à notre vantardise ?
\par 9 Toutes ces choses sont passées comme une ombre et comme un poteau qui passe en toute hâte ;
\par 10 Et comme un navire qui passe sur les vagues de l'eau, dont quand il passe, on ne peut en trouver la trace, ni le chemin de la quille dans les vagues ;
\par 11 Ou comme lorsqu'un oiseau a volé dans les airs, il n'y a aucun signe de son chemin à trouver, mais l'air léger, battu par le battement de ses ailes et séparé par le bruit et le mouvement violents de celles-ci, est passé. à travers, et là-dedans par la suite aucun signe où elle est allée n'est trouvé ;
\par 12 Ou comme lorsqu'une flèche est tirée sur un but, elle divise l'air, qui se rassemble immédiatement, de sorte qu'un homme ne peut pas savoir par où elle est passée :
\par 13 De la même manière, dès notre naissance, nous avons commencé à toucher à notre fin, et nous n'avions aucun signe de vertu à montrer ; mais nous avons été consumés par notre propre méchanceté.
\par 14 Car l'espérance des fidèles est comme la poussière emportée par le vent ; comme une fine écume chassée par la tempête ; comme la fumée qui se disperse ici et là avec une tempête, et disparaît comme le souvenir d'un hôte qui ne tarde qu'un jour.
\par 15 Mais les justes vivent éternellement ; leur récompense aussi est auprès du Seigneur, et leur soin appartient au Très-Haut.
\par 16 C'est pourquoi ils recevront de la main de l'Éternel un royaume glorieux et une belle couronne; car de sa main droite il les couvrira, et de son bras il les protégera.
\par 17 Il prendra sa jalousie pour une armure complète, et fera de la créature son arme pour la vengeance de ses ennemis.
\par 18 Il revêtira la justice comme une cuirasse, et le vrai jugement au lieu d'un casque.
\par 19 Il prendra la sainteté pour un bouclier invincible.
\par 20 Sa colère sévère sera aiguisée comme une épée, et le monde combattra avec lui contre les insensés.
\par 21 Alors les foudres dirigées vers la droite sortiront ; et des nuages, comme d'un arc bien tendu, ils voleront vers le but.
\par 22 Et des grêlons pleins de colère seront jetés comme d'un arc de pierre, et l'eau de la mer se déchaînera contre eux, et les flots les noieront cruellement.
\par 23 Oui, un vent puissant s'élèvera contre eux, et comme une tempête les emportera ; ainsi l'iniquité dévastera toute la terre, et la maltraitance renversera les trônes des puissants.

\chapter{6}

\par 1 Écoutez donc, ô vous, rois, et comprenez ; apprenez, vous qui jugez les extrémités de la terre.
\par 2 Prêtez l'oreille, vous qui gouvernez les peuples, et glorifiez-vous dans la multitude des nations.
\par 3 Car le pouvoir vous est donné par l'Éternel, et la souveraineté du Très-Haut, qui testera vos œuvres et étudiera vos conseils.
\par 4 Parce que, étant ministres de son royaume, vous n'avez pas bien jugé, ni observé la loi, ni suivi le conseil de Dieu ;
\par 5 Il viendra sur vous avec horreur et rapidité, car un jugement sévère sera porté sur ceux qui sont dans les hauts lieux.
\par 6 Car la miséricorde pardonnera bientôt au plus méchant, mais les hommes forts seront puissamment tourmentés.
\par 7 Car celui qui est le Seigneur de tout ne craindra personne, et il ne craindra pas la grandeur d'aucun homme ; car il a fait les petits et les grands, et il se soucie de tous également.
\par 8 Mais une épreuve douloureuse viendra sur les puissants.
\par 9 C'est donc à vous, ô rois, que je vous parle, afin que vous appreniez la sagesse et que vous ne tombiez pas.
\par 10 Car ceux qui gardent saintement la sainteté seront jugés saints, et ceux qui ont appris de telles choses trouveront quoi répondre.
\par 11 C'est pourquoi, attachez votre affection à mes paroles ; désirez-les, et vous serez instruits.
\par 12 La sagesse est glorieuse et ne se fane jamais : oui, elle est facilement vue par ceux qui l'aiment, et trouvée par ceux qui la recherchent.
\par 13 Elle empêche ceux qui la désirent de se faire connaître à eux en premier.
\par 14 Celui qui la cherche de bonne heure n'aura pas de grandes difficultés, car il la trouvera assise à sa porte.
\par 15 Penser à elle est donc la perfection de la sagesse ; et celui qui veille sur elle sera vite insouciant.
\par 16 Car elle va de lieu en lieu à la recherche de ceux qui sont dignes d'elle, leur montre favorablement dans les chemins et les rencontre dans toutes ses pensées.
\par 17 Car son véritable commencement est le désir de discipline ; et le souci de la discipline est l'amour ;
\par 18 Et l'amour est le respect de ses lois ; et prêter attention à ses lois est l'assurance de l'incorruption ;
\par 19 Et l'incorruption nous rapproche de Dieu.
\par 20 C'est pourquoi le désir de la sagesse amène à un royaume.
\par 21 Si donc vous aimez les trônes et les sceptres, ô vous, rois des peuples, honorez la sagesse, afin que vous régniez pour toujours.
\par 22 Quant à la sagesse, ce qu'elle est et comment elle est née, je vous le dirai, et je ne vous cacherai pas de mystères ; mais je la chercherai dès le début de sa naissance et je mettrai en lumière sa connaissance. , et ne passera pas sous silence la vérité.
\par 23 Je n'irai pas non plus avec une envie dévorante ; car un tel homme n’aura aucune communion avec la sagesse.
\par 24 Mais la multitude des sages est le bien du monde, et un roi sage est le soutien du peuple.
\par 25 Recevez donc l'instruction par mes paroles, et cela vous fera du bien.

\chapter{7}

\par 1 Moi aussi, je suis un homme mortel, comme tous, et la postérité de celui qui a été le premier fait de la terre,
\par 2 Et dans le ventre de ma mère, en l'espace de dix mois, il a été façonné pour être chair, étant compacté dans le sang, de la semence de l'homme et dans le plaisir qui accompagne le sommeil.
\par 3 Et quand je suis né, j'ai aspiré l'air commun et je suis tombé sur la terre, qui est de même nature, et la première voix que j'ai émise était un cri, comme le font toutes les autres.
\par 4 J'ai été allaitée dans des langes, et cela avec soins.
\par 5 Car il n'y a pas de roi qui ait eu un autre commencement de naissance.
\par 6 Car tous les hommes ont une seule entrée dans la vie, et une même sortie.
\par 7 C'est pourquoi j'ai prié, et la compréhension m'a été donnée : j'ai invoqué Dieu, et l'esprit de sagesse m'est venu.
\par 8 Je la préférais aux sceptres et aux trônes, et je n'estimais pas les richesses en comparaison d'elle.
\par 9 Je ne lui ai comparé aucune pierre précieuse, car tout or pour elle est comme un peu de sable, et l'argent sera compté comme de l'argile devant elle.
\par 10 Je l'ai aimée plus que la santé et la beauté, et j'ai choisi de l'avoir au lieu de la lumière : car la lumière qui vient d'elle ne s'éteint jamais.
\par 11 Toutes les bonnes choses ensemble sont venues à moi avec elle, et d'innombrables richesses entre ses mains.
\par 12 Et je me suis réjoui en eux tous, parce que la sagesse les précède; et je ne savais pas qu'elle était leur mère.
\par 13 J'ai appris avec diligence, et je la communique généreusement : je ne cache pas ses richesses.
\par 14 Car elle est pour les hommes un trésor inépuisable : ceux qui l'utilisent deviennent les amis de Dieu, étant loués pour les dons qui viennent de l'étude.
\par 15 Dieu m'a donné de parler comme je veux et de concevoir comme il convient aux choses qui me sont données, car c'est lui qui conduit à la sagesse et qui dirige les sages.
\par 16 Car nous sommes entre ses mains, nous et nos paroles ; aussi toute sagesse et connaissance de l'ouvrage.
\par 17 Car il m'a donné une certaine connaissance des choses qui sont, à savoir comment le monde a été créé et le fonctionnement des éléments.
\par 18 Le commencement, la fin et le milieu des temps : les changements de rotation du soleil et le changement des saisons :
\par 19 Les circuits des années et les positions des étoiles :
\par 20 La nature des êtres vivants, et les fureurs des bêtes sauvages : la violence des vents, et les raisonnements des hommes : la diversité des plantes et les vertus des racines :
\par 21 Et toutes les choses qui sont secrètes ou manifestes, je les connais.
\par 22 Car la sagesse, qui est l'artisane de toutes choses, m'a enseigné ; car en elle est un esprit intelligent, saint, un seul, multiple, subtil, vif, clair, sans souillure, clair, non sujet au mal, aimant ce qui est bon, rapide, qui ne peut être laissé, prêt à faire le bien,
\par 23 Bon envers l'homme, ferme, sûr, insouciant, ayant tout pouvoir, surveillant toutes choses et passant par toute intelligence, des esprits purs et très subtils.
\par 24 Car la sagesse est plus émouvante que tout mouvement : elle traverse et traverse toutes choses à cause de sa pureté.
\par 25 Car elle est le souffle de la puissance de Dieu et une pure influence qui découle de la gloire du Tout-Puissant : c'est pourquoi rien de souillé ne peut tomber en elle.
\par 26 Car elle est l'éclat de la lumière éternelle, le miroir sans tache de la puissance de Dieu et l'image de sa bonté.
\par 27 Et n'étant qu'une, elle peut tout faire ; et restant en elle-même, elle rend toutes choses nouvelles ; et, de tous âges, entrant dans les âmes saintes, elle en fait des amis de Dieu et des prophètes.
\par 28 Car Dieu n'aime que celui qui habite avec sagesse.
\par 29 Car elle est plus belle que le soleil, et avant tout l'ordre des étoiles : comparée à la lumière, elle se trouve devant elle.
\par 30 Car après cela vient la nuit ; mais le vice ne prévaudra pas contre la sagesse.

\chapter{8}

\par 1 La sagesse s'étend puissamment d'un bout à l'autre, et elle ordonne toutes choses avec douceur.
\par 2 Je l'aimais et je la recherchais dès ma jeunesse, je désirais en faire mon épouse et j'étais amoureux de sa beauté.
\par 3 En ce qu'elle connaît Dieu, elle magnifie sa noblesse : oui, le Seigneur de toutes choses lui-même l'a aimée.
\par 4 Car elle connaît les mystères de la connaissance de Dieu et aime ses œuvres.
\par 5 Si les richesses sont un bien à désirer dans cette vie ; Qu'y a-t-il de plus riche que la sagesse, qui opère toutes choses ?
\par 6 Et si la prudence agit ; lequel de tous est un ouvrier plus habile qu’elle ?
\par 7 Et si un homme aime la justice, ses travaux sont des vertus : car elle enseigne la tempérance et la prudence, la justice et le courage : ce sont des choses telles qu'il ne peut rien avoir de plus profitable dans sa vie.
\par 8 Si un homme désire beaucoup d'expérience, elle connaît les choses anciennes et conjecture correctement ce qui est à venir ; elle connaît les subtilités des discours et peut énoncer des phrases sombres ; elle prévoit les signes et les prodiges, et les événements des saisons et des temps. .
\par 9 C'est pourquoi j'ai décidé de l'emmener chez moi pour vivre avec moi, sachant qu'elle serait une conseillère en bonnes choses et une consolation dans les soucis et les chagrins.
\par 10 C'est à cause d'elle que j'aurai de l'estime parmi la multitude et de l'honneur auprès des anciens, même si je suis jeune.
\par 11 Je serai trouvé d'une vanité rapide dans le jugement, et je serai admiré aux yeux des grands hommes.
\par 12 Quand je tiendrai ma langue, ils attendront mon loisir, et quand je parlerai, ils me prêteront une bonne oreille ; si je parle beaucoup, ils poseront les mains sur leur bouche.
\par 13 De plus, grâce à elle, j'obtiendrai l'immortalité, et je laisserai derrière moi un mémorial éternel pour ceux qui viendront après moi.
\par 14 Je mettrai l'ordre dans le peuple, et les nations me seront soumises.
\par 15 Les tyrans horribles auront peur, quand ils entendront parler de moi ; Je serai trouvé bon parmi la multitude et vaillant dans la guerre.
\par 16 Après être entré dans ma maison, je me reposerai avec elle ; car sa conversation n'a aucune amertume ; et vivre avec elle n'a pas de chagrin, mais de la joie et de la joie.
\par 17 Or, lorsque j'ai considéré ces choses en moi-même et que j'y ai réfléchi dans mon cœur, comment être allié à la sagesse est l'immortalité ;
\par 18 Et c'est un grand plaisir d'avoir son amitié ; et dans les œuvres de ses mains se trouvent des richesses infinies ; et dans l'exercice de la conférence avec elle, prudence ; et en causant avec elle, un bon rapport ; J'ai cherché comment l'amener à moi.
\par 19 Car j'étais un enfant plein d'esprit et j'avais un bon esprit.
\par 20 Oui plutôt, étant bon, je suis entré dans un corps sans souillure.
\par 21 Néanmoins, quand j'ai compris que je ne pourrais pas l'obtenir autrement, si Dieu ne me la donnait ; et c'était aussi une question de sagesse que de savoir à qui elle appartenait ; J'ai prié le Seigneur et je l'ai supplié, et de tout mon cœur j'ai dit :

\chapter{9}

\par 1 Ô Dieu de mes pères et Seigneur de miséricorde, qui as fait toutes choses par ta parole,
\par 2 Et tu as ordonné l'homme, par ta sagesse, pour qu'il domine sur les créatures que tu as faites,
\par 3 Et commandez le monde selon l'équité et la justice, et exécutez le jugement avec un cœur droit :
\par 4 Donne-moi la sagesse, qui est assise près de ton trône ; et ne me rejette pas du milieu de tes enfants :
\par 5 Car moi, ton serviteur et fils de ta servante, je suis une personne faible, de petite taille, et trop jeune pour comprendre le jugement et les lois.
\par 6 Car, même si un homme n'est jamais aussi parfait parmi les enfants des hommes, si ta sagesse n'est pas avec lui, il ne sera rien considéré.
\par 7 Tu m'as choisi pour être roi de ton peuple et juge de tes fils et de tes filles.
\par 8 Tu m'as ordonné de bâtir un temple sur ta sainte montagne, et un autel dans la ville où tu habites, semblable au saint tabernacle que tu as préparé dès le commencement.
\par 9 Et la sagesse était avec toi, elle connaît tes œuvres, et elle était présente lorsque tu as créé le monde, et elle savait ce qui était agréable à tes yeux et ce qui était juste dans tes commandements.
\par 10 Ô, envoie-la hors de tes cieux saints et du trône de ta gloire, afin qu'elle soit présente et qu'elle travaille avec moi, afin que je sache ce qui te plaît.
\par 11 Car elle sait et comprend toutes choses, et elle me conduira sobrement dans mes actions, et me gardera en son pouvoir.
\par 12 Ainsi mes œuvres seront agréables, et alors je jugerai ton peuple avec justice, et je serai digne de m'asseoir à la place de mon père.
\par 13 Car quel est l'homme qui peut connaître le conseil de Dieu ? ou qui peut penser quelle est la volonté du Seigneur ?
\par 14 Car les pensées des hommes mortels sont misérables, et nos projets ne sont qu'incertains.
\par 15 Car le corps corruptible écrase l'âme, et le tabernacle terrestre alourdit l'esprit qui réfléchit sur beaucoup de choses.
\par 16 Et à peine devinons-nous correctement les choses qui sont sur la terre, et avec du travail nous trouvons les choses qui sont devant nous ; mais les choses qui sont dans les cieux, qui les a fouillé ?
\par 17 Et ton conseil, qui a connu, si tu ne donnes la sagesse et n'envoies pas ton Saint-Esprit d'en haut ?
\par 18 Car ainsi les voies de ceux qui vivaient sur la terre ont été réformées, et les hommes ont appris ce qui te plaît, et ont été sauvés par la sagesse.

\chapter{10}

\par 1 Elle a préservé le premier père formé du monde, qui a été créé seul, et l'a tiré de sa chute,
\par 2 Et lui a donné le pouvoir de gouverner toutes choses.
\par 3 Mais lorsque l'injuste s'éloigna d'elle dans sa colère, il périt aussi dans la fureur avec laquelle il assassina son frère.
\par 4 Pour la cause de qui la terre étant noyée par le déluge, la sagesse la préserva encore, et dirigea la marche des justes dans un morceau de bois de peu de valeur.
\par 5 De plus, les nations dans leur méchante conspiration étant confondues, elle trouva le juste, et le garda irréprochable devant Dieu, et le garda fort contre sa tendre compassion envers son fils.
\par 6 Lorsque les impies périrent, elle délivra le juste, qui fuyait le feu qui s'abattit sur les cinq villes.
\par 7 De la méchanceté de laquelle, jusqu'à ce jour, le désert qui fume est un témoignage, et les plantes qui portent des fruits qui ne mûrissent jamais, et une statue de sel debout est un monument d'une âme incrédule.
\par 8 Car, en ce qui concerne la sagesse, ils n'ont pas seulement eu ce mal, en ce qu'ils ne connaissaient pas les choses qui étaient bonnes ; mais il a aussi laissé derrière eux au monde un mémorial de leur folie : de sorte que dans les choses qu'ils ont offensées, ils ne pouvaient même pas se cacher.
\par 9 Mais la sagesse délivra de la douleur ceux qui la servaient.
\par 10 Quand le juste fuyait la colère de son frère, elle le guidait dans les bons sentiers, lui montrait le royaume de Dieu, lui donnait la connaissance des choses saintes, l'enrichissait dans ses voyages et multipliait le fruit de son travail.
\par 11 Dans la convoitise de ceux qui l'opprimaient, elle se tenait à ses côtés et l'enrichissait.
\par 12 Elle le défendit contre ses ennemis, et le garda à l'abri de ceux qui l'embusquaient, et dans un combat acharné, elle lui donna la victoire ; afin qu'il sache que la bonté est plus forte que tout.
\par 13 Quand le juste fut vendu, elle ne l'abandonna pas, mais le délivra du péché : elle descendit avec lui dans la fosse,
\par 14 Et elle ne le laissa pas enchaîné jusqu'à ce qu'elle lui apporte le sceptre du royaume et le pouvoir contre ceux qui l'opprimaient. Quant à ceux qui l'avaient accusé, elle les montra menteurs et lui donna une gloire perpétuelle.
\par 15 Elle a délivré le peuple juste et la postérité irréprochable de la nation qui les opprimait.
\par 16 Elle entra dans l'âme du serviteur du Seigneur, et résista aux rois terribles par des prodiges et des signes ;
\par 17 Il rendait aux justes la récompense de leurs travaux, les guidait d'une manière merveilleuse, et leur était un abri le jour et une lumière d'étoiles la nuit ;
\par 18 Il les fit traverser la mer Rouge et les fit traverser de grandes eaux.
\par 19 Mais elle noya leurs ennemis et les fit remonter du fond de l'abîme.
\par 20 C'est pourquoi les justes ont gâté les impies, et ont loué ton saint nom, ô Seigneur, et ont magnifié d'un commun accord ta main qui combattait pour eux.
\par 21 Car la sagesse a ouvert la bouche des muets et a rendu éloquente la langue de ceux qui ne savent pas parler.

\chapter{11}

\par 1 Elle a fait prospérer leurs œuvres entre les mains du saint prophète.
\par 2 Ils traversèrent un désert inhabité, et dressèrent leurs tentes dans des endroits où il n'y avait aucun chemin.
\par 3 Ils s'opposèrent à leurs ennemis et se vengèrent de leurs adversaires.
\par 4 Quand ils avaient soif, ils t'invoquèrent, et de l'eau leur fut donnée du rocher de silex, et leur soif fut étanche de la pierre dure.
\par 5 Car ce par quoi leurs ennemis ont été punis, ils en ont tiré profit dans leurs besoins.
\par 6 Car au lieu d'un fleuve perpétuel, troublé de sang impur,
\par 7 En guise de reproche manifeste à l'égard du commandement par lequel les enfants étaient tués, tu leur as donné de l'eau en abondance par un moyen qu'ils n'espéraient pas.
\par 8 Déclarant donc par cette soif comment tu as puni leurs adversaires.
\par 9 Car lorsqu'ils étaient éprouvés, quoiqu'ils fussent châtiés dans la miséricorde, ils savaient que les impies étaient jugés avec colère et tourmentés, ayant soif d'une autre manière que les justes.
\par 10 C'est pour ceux-là que tu les as réprimandés et éprouvés, comme un père ; mais pour les autres, comme un roi sévère, tu les as condamnés et punis.
\par 11 Qu'ils fussent absents ou présents, ils étaient également contrariés.
\par 12 Car une double douleur les saisit, et un gémissement à cause du souvenir des choses passées.
\par 13 Car lorsqu'ils entendirent que leurs propres châtiments seraient bénéfiques pour les autres, ils eurent quelque sentiment du Seigneur.
\par 14 Pour celui qu'ils respectaient avec mépris, alors qu'il avait été jeté dehors longtemps auparavant lors de la mise au monde des enfants, ils l'admirèrent à la fin, quand ils virent ce qui arrivait, qu'ils l'admirèrent.
\par 15 Mais à cause des intrigues insensées de leur méchanceté, par lesquelles, trompés, ils adoraient des serpents dépourvus de raison et des bêtes viles, tu as envoyé contre eux une multitude de bêtes déraisonnables pour se venger ;
\par 16 Afin qu'ils sachent de quoi un homme pèche, c'est par là aussi qu'il sera puni.
\par 17 Car ta main toute-puissante, qui a créé le monde de la matière informe, n'a pas manqué de moyens pour envoyer parmi eux une multitude d'ours ou de lions féroces,
\par 18 Ou des bêtes sauvages inconnues, pleines de rage, nouvellement créées, exhalant soit une vapeur ardente, soit des odeurs immondes de fumée éparse, ou tirant de leurs yeux d'horribles étincelles :
\par 19 De sorte que non seulement le mal pourrait les expédier immédiatement, mais aussi le spectacle terrible les détruirait complètement.
\par 20 Oui, et sans cela, ils auraient pu tomber d'un seul coup, étant persécutés par vengeance et dispersés par le souffle de ta puissance ; mais tu as ordonné toutes choses en mesure, en nombre et en poids.
\par 21 Car tu peux montrer ta grande force à tout moment quand tu le veux ; et qui peut résister à la puissance de ton bras ?
\par 22 Car le monde entier devant toi est comme un petit grain de la balance, oui, comme une goutte de rosée du matin qui tombe sur la terre.
\par 23 Mais tu as pitié de tous ; car tu peux tout faire, et faire un clin d'œil aux péchés des hommes, parce qu'ils doivent s'amender.
\par 24 Car tu aimes tout ce qui est, et tu n'as en horreur rien de ce que tu as fait ; car tu n'aurais jamais fait quoi que ce soit, si tu l'avais haï.
\par 25 Et comment quelque chose aurait-il pu durer, si cela n'avait pas été ta volonté ? ou a-t-il été préservé, sinon appelé par toi ?
\par 26 Mais tu épargnes tout, car ils sont à toi, Seigneur, toi qui aime les âmes.

\chapter{12}

\par 1 Car ton Esprit incorruptible est en toutes choses.
\par 2 C'est pourquoi tu châties peu à peu ceux qui ont offensé, et tu les avertis en leur rappelant ce qu'ils ont offensé, afin qu'en abandonnant leur méchanceté, ils croient en toi, ô Seigneur.
\par 3 Car tu voulais détruire par les mains de nos pères ces deux anciens habitants de ta terre sainte,
\par 4 Que tu as haï pour avoir commis les œuvres de sorcellerie les plus odieuses et les sacrifices les plus méchants ;
\par 5 Et aussi ces impitoyables meurtriers d'enfants, et dévoreurs de chair humaine, et les festins de sang,
\par 6 Avec leurs prêtres du milieu de leur équipage idolâtre, et les parents, qui tuaient de leurs propres mains les âmes sans secours :
\par 7 Afin que le pays, que tu estimes au-dessus de tout autre, puisse recevoir une digne colonie d'enfants de Dieu.
\par 8 Néanmoins, même ceux que tu as épargnés en tant qu'hommes, tu as envoyé des guêpes, précurseurs de ton armée, pour les détruire peu à peu.
\par 9 Non pas que tu n'aies pas pu mettre les impies sous la main des justes dans la bataille, ou les détruire immédiatement avec des bêtes cruelles, ou avec une seule parole dure :
\par 10 Mais en exécutant peu à peu tes jugements sur eux, tu leur as donné un lieu de repentir, n'ignorant pas qu'ils étaient une génération méchante, et que leur méchanceté était engendrée en eux, et que leur pensée ne changerait jamais.
\par 11 Car c'était une semence maudite dès le commencement ; et tu ne leur as pas non plus, par crainte de qui que ce soit, donné leur pardon pour les choses dans lesquelles ils ont péché.
\par 12 Car qui dira : Qu'as-tu fait ? ou qui résistera à ton jugement ? ou qui t'accusera à cause des nations périssables que tu as créées ? ou qui viendra se dresser contre toi, pour se venger des hommes injustes ?
\par 13 Car il n'y a d'autre Dieu que toi, qui se soucie de tous, à qui tu puisses montrer que ton jugement n'est pas injuste.
\par 14 Ni le roi ni le tyran ne pourront se retourner contre toi à cause de celui que tu as puni.
\par 15 Puisque donc tu es juste toi-même, tu ordonnes toutes choses avec justice, pensant qu'il ne convient pas à ton pouvoir de condamner celui qui n'a pas mérité d'être puni.
\par 16 Car ta puissance est le commencement de la justice, et parce que tu es le Seigneur de tous, elle te rend miséricordieux envers tous.
\par 17 Car quand les hommes ne croient pas que tu es plein de puissance, tu montres ta force, et parmi ceux qui le savent tu manifestes leur audace.
\par 18 Mais toi, maîtrisant ta puissance, juge avec équité et commande-nous avec une grande faveur, car tu peux user de ta puissance quand tu veux.
\par 19 Mais par de telles œuvres tu as enseigné à ton peuple que le juste doit être miséricordieux, et tu as donné à tes enfants une bonne espérance pour que tu leur donnes la repentance de leurs péchés.
\par 20 Car si tu punis les ennemis de tes enfants et les condamnés à mort avec une telle délibération, en leur donnant le temps et le lieu par lesquels ils pourraient être délivrés de leur méchanceté :
\par 21 Avec quelle circonspection as-tu jugé tes propres fils, aux pères desquels tu as juré et fait des alliances de bonnes promesses ?
\par 22 C'est pourquoi, tandis que tu nous châties, tu fouettes mille fois plus nos ennemis, afin que, lorsque nous jugeons, nous réfléchissions soigneusement à ta bonté, et que lorsque nous sommes nous-mêmes jugés, nous attendions la miséricorde.
\par 23 C'est pourquoi, tandis que les hommes ont vécu dans la dissolution et l'injustice, tu les as tourmentés par leurs propres abominations.
\par 24 Car ils se sont égarés très loin dans les voies de l'erreur, et les ont pris pour des dieux, qui même parmi les bêtes de leurs ennemis ont été méprisés, étant séduits, comme des enfants sans intelligence.
\par 25 C'est pourquoi tu leur as envoyé, comme à des enfants sans raison, un jugement pour se moquer d'eux.
\par 26 Mais ceux qui ne veulent pas être réformés par cette correction, dans laquelle il les a traînés, ressentiront un jugement digne de Dieu.
\par 27 Car, voyez, de quelles choses ils ont eu de la rancune, lorsqu'ils ont été punis, c'est-à-dire de ceux qu'ils pensaient être des dieux ; [maintenant] étant punis en eux, quand ils le virent, ils reconnurent qu'il était le vrai Dieu, qu'ils niaient auparavant connaître : et c'est pourquoi une damnation extrême s'abattit sur eux.

\chapter{13}

\par 1 Sûrement vains sont tous les hommes par nature, qui ignorent Dieu, et qui ne peuvent, d'après les bonnes choses visibles, connaître celui qui est : ni en considérant les œuvres, ils n'ont reconnu l'artisan ;
\par 2 Mais on considère que soit le feu, soit le vent, soit l'air rapide, soit le cercle des étoiles, soit l'eau violente, soit les lumières du ciel, sont les dieux qui gouvernent le monde.
\par 3 De la beauté de qui, s'ils étaient ravis, ils les prenaient pour des dieux ; faites-leur savoir combien leur Seigneur est meilleur : car le premier auteur de la beauté les a créés.
\par 4 Mais s'ils ont été étonnés de leur puissance et de leur vertu, qu'ils comprennent par eux combien plus puissant est celui qui les a créés.
\par 5 Car c'est à la grandeur et à la beauté des créatures que l'on voit proportionnellement celui qui les a créées.
\par 6 Mais c'est pour cela qu'ils sont d'autant moins coupables : car ils se trompent peut-être, cherchant Dieu et désireux de le trouver.
\par 7 Car, connaissant ses œuvres, ils le sondent soigneusement et croient en leur vue, parce que les choses qu'on voit sont belles.
\par 8 Mais ils ne seront pas non plus pardonnés.
\par 9 Car s'ils pouvaient en savoir autant, ils pourraient viser le monde ; comment n'ont-ils pas découvert plus tôt quel était leur Seigneur ?
\par 10 Mais ils sont misérables, et leur espoir est dans les choses mortes, qui les appellent dieux, qui sont des œuvres de mains d'homme, de l'or et de l'argent, pour montrer l'art, et les ressemblances de bêtes, ou une pierre qui ne vaut rien, le travail d'une main ancienne.
\par 11 Or, un charpentier qui abattit du bois, après avoir scié un arbre adapté à cet effet, et enlevé habilement toute l'écorce tout autour, et l'avoir magnifiquement travaillé, et en avoir fait un récipient propre au service de la vie de l'homme. ;
\par 12 Et après avoir dépensé les déchets de son travail pour préparer sa viande, il s'est rassasié ;
\par 13 Et prenant parmi ceux qui ne servaient à rien, un morceau de bois tordu et plein de nœuds, il l'a sculpté avec diligence, alors qu'il n'avait rien d'autre à faire, et l'a façonné par l'habileté de son intelligence. , et l'a façonné à l'image d'un homme;
\par 14 Ou il l'a fait comme une vile bête, en le recouvrant de vermillon et de peinture le colorant en rouge, et en couvrant chaque tache qui y est ;
\par 15 Et quand il lui eut fait un espace convenable, il le plaça dans un mur et le fixa avec du fer :
\par 16 Car il a pourvu à ce qu'il ne tombe pas, sachant qu'il ne pouvait pas s'aider lui-même ; car c'est une image et elle a besoin d'aide :
\par 17 Alors il prie pour ses biens, pour sa femme et ses enfants, et il n'a pas honte de parler à celui qui n'a pas de vie.
\par 18 Pour la santé, il invoque ce qui est faible ; car la vie prie ce qui est mort ; car il demande humblement de l'aide à celui qui a le moins de moyens d'aider ; et pour un bon voyage, il demande à celui qui ne peut pas mettre le pied en avant :
\par 19 Et pour gagner et obtenir, et pour le bon succès de ses mains, on demande à celui qui est le plus incapable de faire quoi que ce soit, la capacité de faire.

\chapter{14}

\par 1 Encore, celui qui se prépare à naviguer, et sur le point de traverser les vagues déchaînées, invoque un morceau de bois plus pourri que le vaisseau qui le porte.
\par 2 C'est en vérité le désir du gain qui a conçu cela, et l'ouvrier l'a construit par son habileté.
\par 3 Mais ta providence, ô Père, le gouverne ; car tu as ouvert un chemin dans la mer et un chemin sûr dans les vagues ;
\par 4 Montrer que tu peux sauver de tout danger : oui, même si un homme est allé en mer sans art.
\par 5 Cependant tu ne voudrais pas que les œuvres de ta sagesse restent vaines, et c'est pourquoi les hommes consacrent leur vie à un petit morceau de bois, et traversent la mer agitée sur un faible navire sont sauvés.
\par 6 Car autrefois aussi, lorsque périrent les géants orgueilleux, l'espérance du monde gouverné par ta main s'échappa dans un vase faible, et laissa à tous les âges une semence de génération.
\par 7 Car béni est le bois par lequel vient la justice.
\par 8 Mais ce qui est fait de main d'homme est maudit, autant lui que celui qui l'a fait : lui, parce qu'il l'a fait ; et cela parce que, étant corruptible, on l'appelait dieu.
\par 9 Car l'impie et son impiété sont également odieux à Dieu.
\par 10 Car ce qui est fait sera puni avec celui qui l'a fait.
\par 11 C'est pourquoi même les idoles des païens seront frappées, car dans la créature de Dieu elles sont devenues une abomination, une pierre d'achoppement pour les âmes des hommes, et un piège aux pieds des insensés.
\par 12 Car la création d'idoles fut le commencement de la fornication spirituelle, et leur invention la corruption de la vie.
\par 13 Car ils n'étaient ni dès le commencement, ni pour toujours.
\par 14 Car c'est par la vaine gloire des hommes qu'ils sont entrés dans le monde, et c'est pourquoi ils connaîtront bientôt leur fin.
\par 15 Car un père affligé d'un deuil prématuré, après avoir fait une image de son enfant bientôt enlevé, l'honorait maintenant comme un dieu, qui était alors un homme mort, et offrait à ceux qui étaient sous lui des cérémonies et des sacrifices.
\par 16 Ainsi, au fil du temps, une coutume impie devenue forte fut observée comme une loi, et les images taillées furent adorées par les commandements des rois.
\par 17 Celui que les hommes ne pouvaient honorer en présence, parce qu'ils habitaient au loin, ils prirent de loin la contrefaçon de son visage, et firent une image expresse d'un roi qu'ils honorèrent, afin que par cette audace ils puissent flatter celui qui était absent, comme s'il était présent.
\par 18 Aussi la diligence singulière de l'artisan a contribué à orienter les ignorants vers plus de superstition.
\par 19 Car lui, voulant peut-être plaire à quelqu'un en autorité, a forcé toute son habileté à faire la ressemblance de la meilleure mode.
\par 20 Et ainsi la multitude, séduite par la grâce de l'ouvrage, le prit maintenant pour un dieu, qui, peu auparavant, n'était qu'honoré.
\par 21 Et c'était là une occasion de tromper le monde : car les hommes, servant soit le malheur, soit la tyrannie, attribuaient aux pierres et aux ceps le nom incommunicable.
\par 22 Et cela ne leur suffisait pas, qu'ils se soient égarés dans la connaissance de Dieu ; mais alors qu'ils vivaient dans la grande guerre de l'ignorance, ces si grandes plaies les appelaient paix.
\par 23 Car tandis qu'ils tuaient leurs enfants en sacrifices, ou utilisaient des cérémonies secrètes, ou se livraient à des rites étranges ;
\par 24 Ils ne gardèrent plus intactes ni les vies ni les mariages ; mais soit l'un tuait l'autre par méchanceté, soit l'affligeait par l'adultère.
\par 25 De sorte qu'il régnait chez tous les hommes sans exception le sang, les homicides, le vol et la dissimulation, la corruption, l'infidélité, les tumultes, le parjure,
\par 26 Inquiétude des hommes de bien, oubli des bonnes actions, souillure des âmes, changement de genre, désordre dans les mariages, adultère et impureté éhontée.
\par 27 Car l'adoration d'idoles sans nom est le commencement, la cause et la fin de tout mal.
\par 28 Car soit ils sont fous quand ils sont joyeux, soit ils prophétisent des mensonges, soit ils vivent injustement, soit ils se renoncent à la légère.
\par 29 Car dans la mesure où leur confiance est dans les idoles, qui n'ont pas de vie ; bien qu'ils jurent faussement, ils ne semblent pourtant pas être blessés.
\par 30 Cependant, pour les deux causes, ils seront justement punis : à la fois parce qu'ils n'ont pas eu une bonne opinion de Dieu, en s'intéressant aux idoles, et aussi parce qu'ils ont juré injustement dans la tromperie, en méprisant la sainteté.
\par 31 Car ce n'est pas la puissance de ceux par qui ils jurent ; mais c'est la juste vengeance des pécheurs, qui punit toujours l'offense des impies.

\chapter{15}

\par 1 Mais toi, ô Dieu, tu es miséricordieux et fidèle, long en colère, et tu ordonnes toutes choses avec miséricorde,
\par 2 Car si nous péchons, nous sommes à toi, connaissant ta puissance ; mais nous ne pécherons pas, sachant que nous sommes à toi.
\par 3 Car te connaître est une justice parfaite ; oui, connaître ta puissance est la racine de l'immortalité.
\par 4 Car ni l'invention malicieuse des hommes ne nous a trompés, ni une image tachetée de diverses couleurs, le travail infructueux du peintre ;
\par 5 Dont la vue incite les insensés à la convoiter, et ainsi ils désirent la forme d'une image morte, qui n'a pas de souffle.
\par 6 Ceux qui les fabriquent, ceux qui les désirent et ceux qui les adorent, aiment les mauvaises choses et sont dignes d'avoir de telles choses sur lesquelles se fier.
\par 7 Car le potier, en trempant la terre molle, façonne avec beaucoup de travail tous les vases pour notre service ; oui, avec la même argile, il fait à la fois les vases qui servent à des usages purs, et de même aussi tous ceux qui servent au contraire ; mais à quoi servent les deux sortes, le potier lui-même en est le juge.
\par 8 Et employant ses travaux lubriquement, il fait un dieu vain de la même argile, même celui qui peu auparavant était lui-même fait de terre, et peu de temps après y retourne, lorsque sa vie qui lui a été prêtée sera exigé.
\par 9 Malgré son souci, ce n'est pas qu'il ait beaucoup de travail, ni que sa vie soit courte ; mais il s'efforce de surpasser les orfèvres et les orfèvres, et s'efforce de faire comme les ouvriers de l'airain, et considère comme sa gloire de fabriquer des choses contrefaites. .
\par 10 Son cœur est cendre, son espérance est plus vile que la terre, et sa vie a moins de valeur que l'argile :
\par 11 Parce qu'il ne connaissait pas son Créateur, ni celui qui lui insufflait une âme active et qui respirait un esprit vivant.
\par 12 Mais ils considéraient notre vie comme un passe-temps, et notre séjour ici comme un marché de gain : car, disent-ils, nous devons réussir dans tous les sens, même si ce fut par de mauvais moyens.
\par 13 Car cet homme, qui fait de la matière terrestre des vases fragiles et des images taillées, sait qu'il offense plus que tous les autres.
\par 14 Et tous les ennemis de ton peuple, qui le soumettent, sont des plus insensés et plus misérables que des bébés.
\par 15 Car ils considéraient comme des dieux toutes les idoles des païens, qui n'ont ni yeux pour voir, ni nez pour respirer, ni oreilles pour entendre, ni doigts de mains pour manipuler ; et quant à leurs pieds, ils avancent lentement.
\par 16 Car l'homme les a créés, et celui qui a emprunté son propre esprit les a façonnés; mais personne ne peut créer un dieu semblable à lui-même.
\par 17 Car étant mortel, il travaille une chose morte avec des mains méchantes ; car lui-même vaut mieux que les choses qu'il adore ; alors qu'il a vécu une fois, mais elles n'ont jamais vécu.
\par 18 Oui, ils adoraient aussi les bêtes les plus odieuses ; car, comparées entre elles, les unes sont pires que les autres.
\par 19 Ils ne sont pas non plus beaux au point d'être désirables par rapport aux bêtes; mais ils sont allés sans la louange de Dieu et sa bénédiction.

\chapter{16}

\par 1 C'est pourquoi ils furent dignement châtiés par des choses semblables, et tourmentés par une multitude de bêtes.
\par 2 Au lieu de ce châtiment, ayant traité avec grâce envers ton propre peuple, tu leur as préparé de la viande d'un goût étrange, même des cailles, pour attiser leur appétit :
\par 3 Afin qu'eux, désirant de la nourriture, puissent, à cause de la laideur des bêtes envoyées parmi eux, détester même ce qu'ils doivent nécessairement désirer ; mais ceux-ci, souffrant de pénurie pendant un court espace de temps, pourraient devenir participants d'un goût étrange.
\par 4 Car il fallait que ceux qui exerçaient la tyrannie entraînaient la pénurie, qu'ils ne pouvaient éviter ; mais à ceux-là, il fallait seulement montrer comment leurs ennemis étaient tourmentés.
\par 5 Car lorsque l'horrible férocité des bêtes s'est abattue sur eux, et qu'ils ont péri sous les mors de serpents tortueux, ta colère n'a pas duré éternellement :
\par 6 Mais ils furent troublés pendant un court moment, afin d'être exhortés, ayant un signe de salut, à leur rappeler le commandement de ta loi.
\par 7 Car celui qui s'est tourné vers elle n'a pas été sauvé par ce qu'il a vu, mais par toi, qui es le Sauveur de tous.
\par 8 Et en cela tu as fait avouer à tes ennemis que c'est toi qui délivres de tout mal :
\par 9 Les piqûres de sauterelles et de mouches les tuèrent, et on ne trouva aucun remède à leur vie, car ils méritaient d'être punis par de telles mouches.
\par 10 Mais tes fils n'ont pas pu vaincre les dents des dragons venimeux, car ta miséricorde était toujours envers eux et les guérissait.
\par 11 Car ils ont été piqués, pour se souvenir de tes paroles ; et ont été rapidement sauvés, afin que, sans tomber dans un profond oubli, ils puissent se souvenir continuellement de ta bonté.
\par 12 Car ce n'est ni l'herbe ni l'enduit apaisant qui leur ont rendu la santé, mais ta parole, ô Seigneur, qui guérit toutes choses.
\par 13 Car tu as le pouvoir de vie et de mort : tu conduis aux portes de l'enfer et tu en fais remonter.
\par 14 En effet, un homme tue par sa méchanceté ; et l'esprit, lorsqu'il est sorti, ne revient pas ; ni l'âme reçue ne revient.
\par 15 Mais il n'est pas possible d'échapper à ta main.
\par 16 Car les impies, qui niaient te connaître, furent flagellés par la force de ton bras ; avec d'étranges pluies, grêles et averses, ils furent persécutés, qu'ils ne pouvaient éviter, et furent consumés par le feu.
\par 17 Car, ce qui est le plus étonnant, c'est que le feu avait plus de force dans l'eau, qui éteint toutes choses ; car le monde combat pour les justes.
\par 18 Car quelque temps la flamme fut atténuée, afin qu'elle ne brûle pas les bêtes envoyées contre les impies ; mais eux-mêmes pouvaient voir et percevoir qu'ils étaient persécutés par le jugement de Dieu.
\par 19 Et une autre fois, elle brûle même au milieu de l'eau, au-dessus de la puissance du feu, afin de détruire les fruits d'un pays injuste.
\par 20 Au lieu de cela, tu as nourri ton peuple avec la nourriture des anges, et tu lui as envoyé du ciel du pain préparé sans leur travail, capable de satisfaire les délices de chacun et de convenir à tous les goûts.
\par 21 Car ta nourriture a déclaré ta douceur à tes enfants, et, servant à l'appétit de celui qui mange, s'est tempérée au goût de chacun.
\par 22 Mais la neige et la glace supportèrent le feu et ne fondirent pas, afin qu'ils sachent que le feu brûlant dans la grêle et étincelant sous la pluie détruisait les fruits des ennemis.
\par 23 Mais encore une fois, il oublia sa propre force, afin que les justes puissent être nourris.
\par 24 Car la créature qui te sert, qui est le Créateur, augmente sa force contre les injustes pour leur châtiment, et diminue sa force pour le bénéfice de ceux qui mettent leur confiance en toi.
\par 25 C'est pourquoi déjà alors elle s'est transformée à toutes les modes, et elle a obéi à ta grâce, qui nourrit toutes choses, selon le désir de ceux qui en avaient besoin :
\par 26 Afin que tes enfants, Seigneur, que tu aimes, sachent que ce n'est pas la croissance des fruits qui nourrit l'homme, mais que c'est ta parole qui préserve ceux qui se confient en toi.
\par 27 Car ce qui n'a pas été détruit par le feu, chauffé par un petit rayon de soleil, a vite fondu :
\par 28 Afin qu'on sache que nous devons empêcher le soleil de te rendre grâce, et qu'au lever du jour, je te prie.
\par 29 Car l'espérance des ingrats fondra comme le givre de l'hiver, et s'enfuira comme une eau inutile.

\chapter{17}

\par 1 Car tes jugements sont grands et ne peuvent être exprimés : c'est pourquoi les âmes incultes se sont égarées.
\par 2 Car lorsque des hommes injustes pensaient opprimer la nation sainte ; ils étaient enfermés dans leurs maisons, prisonniers des ténèbres et enchaînés par les liens d'une longue nuit, gisaient [là] exilés de la providence éternelle.
\par 3 Car alors qu'ils pensaient être cachés dans leurs péchés secrets, ils étaient dispersés sous un voile sombre d'oubli, étant horriblement étonnés et troublés par des apparitions [étranges].
\par 4 Car le coin qui les retenait ne pouvait pas non plus les empêcher de craindre ; mais des bruits [comme des eaux qui tombaient] résonnaient autour d'eux, et des visions tristes leur apparurent avec des visages lourds.
\par 5 Aucune puissance du feu ne pourrait leur éclairer ; les flammes brillantes des étoiles ne pourraient pas non plus supporter d'éclairer cette horrible nuit.
\par 6 Seulement il leur apparut un feu allumé tout seul, très terrible ; car étant très effrayés, ils pensèrent que les choses qu'ils voyaient étaient pires que ce qu'ils ne voyaient pas.
\par 7 Quant aux illusions de l'art magique, elles furent réprimées, et leur vantardise en sagesse fut réprouvée avec honte.
\par 8 Car ceux qui promettaient de chasser les terreurs et les troubles d'une âme malade, étaient eux-mêmes malades de peur, dignes d'être ridiculisés.
\par 9 Car bien qu'aucune chose terrible ne les craignait ; mais ayant peur des bêtes qui passaient et des sifflements des serpents,
\par 10 Ils sont morts de peur, niant avoir vu l'air, qui ne pouvait être évité d'aucun côté.
\par 11 Car la méchanceté, condamnée par son propre témoignage, est très craintive, et, pressée par la conscience, elle prévoit toujours des choses graves.
\par 12 Car la peur n'est rien d'autre qu'une trahison des secours que la raison offre.
\par 13 Et l'attente du dedans, étant moindre, compte plus l'ignorance que la cause qui apporte le tourment.
\par 14 Mais ils dormirent du même sommeil cette nuit-là, qui était vraiment intolérable, et qui leur arriva du fond de l'enfer inévitable,
\par 15 Étaient en partie tourmentés par des apparitions monstrueuses, et en partie évanouis, leur cœur leur défaillant : car une peur soudaine, et non attendue, les surprit.
\par 16 Ainsi donc, quiconque tombait là était étroitement gardé, enfermé dans une prison sans barreaux de fer,
\par 17 Car, qu'il fût laboureur, ou berger, ou ouvrier des champs, il fut rattrapé et endura cette nécessité qui ne pouvait être évitée, car ils étaient tous liés par une seule chaîne de ténèbres.
\par 18 Que ce soit un vent sifflant, ou un bruit mélodieux d'oiseaux parmi les branches étendues, ou une agréable chute d'eau qui coule avec violence,
\par 19 Ou un bruit terrible de pierres jetées, ou une course invisible de bêtes sautillantes, ou une voix rugissante des bêtes sauvages les plus sauvages, ou un écho rebondissant des montagnes creuses ; ces choses les faisaient s’évanouir de peur.
\par 20 Car le monde entier brillait d'une claire lumière, et personne n'était gêné dans son travail :
\par 21 Sur eux seulement s'étendait une nuit lourde, image de ces ténèbres qui devaient ensuite les recevoir ; mais pourtant ils étaient pour eux-mêmes plus pénibles que les ténèbres.

\chapter{18}

\par 1 Néanmoins tes saints avaient une très grande lumière, dont ils entendaient la voix, et ne voyant pas sa forme, parce qu'eux aussi n'avaient pas souffert les mêmes choses, ils les considéraient heureux.
\par 2 Mais pour cela ils ne faisaient plus de mal à ceux dont ils avaient été lésés auparavant, ils les remercièrent et leur demandèrent pardon d'avoir été ennemis.
\par 3 Au lieu de quoi tu leur as donné une colonne de feu ardente, à la fois pour être un guide du voyage inconnu, et un soleil inoffensif pour les divertir honorablement.
\par 4 Car ils méritaient d'être privés de lumière et emprisonnés dans les ténèbres, eux qui avaient gardé enfermés tes fils, par qui la lumière non corrompue de la loi devait être donnée au monde.
\par 5 Et quand ils eurent décidé de tuer les enfants des saints, un enfant ayant été jeté et sauvé, pour les réprimander, tu as enlevé la multitude de leurs enfants, et tu les as détruits entièrement dans une eau puissante.
\par 6 De cette nuit-là, nos pères furent certifiés d'avance, afin que, sachant assurément à quels serments ils avaient prêté foi, ils pourraient ensuite avoir bonne humeur.
\par 7 Ainsi, de ton peuple a été accepté à la fois le salut des justes et la destruction des ennemis.
\par 8 Car par quoi tu as châtié nos adversaires, tu nous as glorifiés par là, nous que tu avais appelés.
\par 9 Car les justes enfants des hommes de bien sacrifiaient en secret, et d'un commun accord édictaient une loi sainte, selon laquelle les saints devaient être comme participants du même bien et du même mal, les pères chantant maintenant des chants de louange.
\par 10 Mais de l'autre côté, les cris des ennemis retentissaient, et un bruit lamentable se répandait au sujet des enfants qui pleuraient.
\par 11 Le maître et le serviteur furent punis d'une même manière ; et comme le roi, les gens ordinaires souffraient ainsi.
\par 12 Ainsi, ils eurent tous ensemble d'innombrables morts avec une seule sorte de mort ; les vivants n'étaient pas non plus suffisants pour les enterrer : car en un instant, la progéniture la plus noble d'entre eux fut détruite.
\par 13 Car attendu qu'ils ne croiraient rien à cause des enchantements ; après la destruction des premiers-nés, ils reconnurent ce peuple comme fils de Dieu.
\par 14 Car tandis que tout était dans un silence tranquille, et que la nuit était au milieu de sa course rapide,
\par 15 Ta parole toute-puissante a sauté du ciel, de ton trône royal, comme un homme de guerre féroce au milieu d'un pays de destruction,
\par 16 Et il a apporté ton commandement non feint comme une épée tranchante, et le fait de se lever a rempli toutes choses de mort ; et il touchait le ciel, mais il se tenait sur la terre.
\par 17 Alors soudain, des visions de rêves horribles les troublèrent profondément, et des terreurs les surprirent à l'improviste.
\par 18 Et l'un jeté ici, et l'autre là, à moitié mort, montrèrent la cause de sa mort.
\par 19 Car les songes qui les troublaient prédisaient cela, de peur qu'ils ne périssent et ne sachent pourquoi ils étaient affligés.
\par 20 Oui, le goût de la mort toucha aussi les justes, et il y eut une destruction de la foule dans le désert ; mais la colère ne dura pas longtemps.
\par 21 Car alors l'homme irréprochable se hâta et se leva pour les défendre ; et apportant le bouclier de son propre ministère, la prière et la propitiation de l'encens, s'opposa à la colère, et mit ainsi fin à la calamité, déclarant qu'il était ton serviteur.
\par 22 Ainsi il vainquit le destructeur, non par la force du corps ni par la force des armes, mais par une parole soumettant celui qui le punissait, invoquant les serments et les alliances faites avec les pères.
\par 23 Car alors que les morts étaient maintenant entassés les uns sur les autres, se tenant entre eux, il arrêta la colère et ouvrit le chemin vers les vivants.
\par 24 Car dans le long vêtement était le monde entier, et dans les quatre rangées de pierres était gravée la gloire des pères, et ta majesté sur le daidem de sa tête.
\par 25 C'est à eux que le destructeur céda la place et eut peur d'eux ; car il suffisait qu'ils goûtent seulement à la colère.

\chapter{19}

\par 1 Quant aux impies, la colère s'abattit sur eux sans pitié jusqu'à la fin ; car il savait d'avance ce qu'ils feraient ;
\par 2 Comment, après leur avoir donné l'autorisation de partir et les avoir renvoyés en toute hâte, ils se repentiraient et les poursuivraient.
\par 3 Car pendant qu'ils pleuraient et se lamentaient encore sur les tombes des morts, ils ajoutèrent un autre stratagème insensé et les poursuivirent comme des fugitifs, qu'ils avaient suppliés de partir.
\par 4 Car la destinée dont ils étaient dignes les entraînait à ce but, et leur faisait oublier les choses qui étaient déjà arrivées, afin qu'ils puissent accomplir le châtiment qui manquait à leurs tourments :
\par 5 Et afin que ton peuple puisse parcourir un chemin merveilleux, mais qu'il puisse trouver une mort étrange.
\par 6 Car toute la créature, dans son espèce propre, a été façonnée de nouveau, servant les commandements particuliers qui leur ont été donnés, afin que tes enfants soient gardés sans dommage :
\par 7 À savoir, un nuage ombrageant le camp ; et là où se trouvait auparavant l'eau, la terre ferme apparut ; et hors de la mer Rouge un chemin sans obstacle ; et du torrent violent un champ vert :
\par 8 Par où sont passés tous les peuples qui étaient défendus par ta main, voyant tes merveilleuses et étranges merveilles.
\par 9 Car ils allaient en liberté comme des chevaux, et sautaient comme des agneaux, te louant, ô Seigneur, qui les avais délivrés.
\par 10 Car ils se souvenaient encore de ce qui s'était passé pendant leur séjour dans le pays étranger, comment la terre produisait des mouches au lieu du bétail, et comment la rivière rejetait une multitude de grenouilles au lieu de poissons.
\par 11 Mais ensuite ils virent une nouvelle génération de volailles, quand, poussés par leur appétit, ils demandèrent des viandes délicates.
\par 12 Car des cailles montaient de la mer vers eux pour leur contentement.
\par 13 Et les châtiments tombaient sur les pécheurs, non sans signes antérieurs, par la force des tonnerres ; car ils souffraient justement selon leur propre méchanceté, dans la mesure où ils avaient une conduite plus dure et plus haineuse envers les étrangers.
\par 14 Car les Sodomites n'ont pas reçu ceux qu'ils ne connaissaient pas lorsqu'ils sont arrivés ; mais ceux-ci ont réduit en esclavage des amis qui les avaient bien mérités.
\par 15 Et non seulement ainsi, mais peut-être qu'on aura quelque respect envers eux, parce qu'ils ont utilisé des étrangers peu amicaux :
\par 16 Mais ceux-ci affligèrent très gravement ceux qu'ils avaient reçus avec des festins, et qui étaient déjà participants aux mêmes lois avec eux.
\par 17 C'est pourquoi même ceux-là furent frappés de cécité, comme ceux qui étaient aux portes du juste : quand, étant environnés d'horribles et grandes ténèbres, chacun cherchait le passage de sa propre porte.
\par 18 Car les éléments ont été modifiés en eux-mêmes par une sorte d'harmonie, comme dans un psaltérion les notes changent le nom de la mélodie, et pourtant sont toujours des sons ; ce qui peut bien être perçu à la vue des choses qui ont été faites.
\par 19 Car les choses terrestres ont été transformées en eaux, et les choses qui auparavant nageaient dans l'eau sont maintenant tombées sur la terre.
\par 20 Le feu avait du pouvoir sur l'eau, oubliant sa propre vertu : et l'eau oubliait sa propre nature désaltérante.
\par 21 De l'autre côté, les flammes n'ont pas dévasté la chair des êtres vivants corruptibles, même s'ils y marchaient ; ils ne fondirent pas non plus la sorte de viande glacée et céleste qui était naturellement susceptible de fondre.
\par 22 Car en toutes choses, ô Seigneur, tu as magnifié ton peuple et tu l'as glorifié, et tu ne l'as pas pris à la légère, mais tu l'as secouru en tout temps et en tout lieu.


\end{document}