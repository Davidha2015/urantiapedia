\begin{document}

\title{Tobie}


\chapter{1}

\par 1 Le livre des paroles de Tobit, fils de Tobiel, fils d'Ananiel, fils d'Aduel, fils de Gabael, de la postérité d'Asaël, de la tribu de Nephthali ;
\par 2 Qui, du temps d'Enemessar, roi des Assyriens, fut emmené captif hors de Thisbé, qui est à la droite de cette ville, qui est appelée proprement Nephthali en Galilée au-dessus d'Aser.
\par 3 Moi, Tobit, j'ai marché tous les jours de ma vie dans les voies de la vérité et de la justice, et j'ai fait de nombreuses aumônes à mes frères et à ma nation, qui sont venus avec moi à Ninive, au pays des Assyriens.
\par 4 Et lorsque j'étais dans mon propre pays, dans le pays d'Israël, alors que j'étais jeune, toute la tribu de Nephtali, mon père, tomba de la maison de Jérusalem, qui avait été choisie entre toutes les tribus d'Israël, afin que toutes les tribus il fallait y sacrifier, où le temple de l'habitation du Très-Haut était consacré et construit pour tous les âges.
\par 5 Or, toutes les tribus qui se sont révoltées ensemble, ainsi que la maison de mon père Nephthali, ont été sacrifiées à la génisse Baal.
\par 6 Mais moi seul, j'allais souvent à Jérusalem lors des fêtes, comme cela avait été ordonné à tout le peuple d'Israël par un décret éternel, d'avoir les prémices et les dixièmes du revenu, avec ce qui était d'abord tondu ; et je les remis sur l'autel aux prêtres, enfants d'Aaron.
\par 7 Je donnai le premier dixième de tout revenu aux fils d'Aaron, qui servaient à Jérusalem ; un autre dixième, je le vendis, et j'allais le dépenser chaque année à Jérusalem.
\par 8 Et je donnai le troisième à ceux qui le convenaient, comme Débora, la mère de mon père, me l'avait ordonné, parce que mon père m'avait laissé orphelin.
\par 9 De plus, lorsque j'ai atteint l'âge d'un homme, j'ai épousé Anna, une de mes parentes, et d'elle j'ai engendré Tobias.
\par 10 Et lorsque nous fûmes emmenés captifs à Ninive, tous mes frères et ceux de ma parenté mangèrent du pain des païens.
\par 11 Mais je me suis retenu de manger ;
\par 12 Parce que je me suis souvenu de Dieu de tout mon cœur.
\par 13 Et le Très-Haut m'a accordé grâce et faveur devant Enemessar, de sorte que j'étais son pourvoyeur.
\par 14 Et je suis allé en Médie, et j'ai laissé en dépôt à Gabaël, frère de Gabrias, à Rages, ville de Médie, dix talents d'argent.
\par 15 Or, après la mort d'Enemessar, Sennachérib, son fils, régna à sa place ; dont la succession était troublée, que je ne pouvais pas entrer en Médie.
\par 16 Et au temps d'Enemessar, j'ai fait de nombreuses aumônes à mes frères, et j'ai donné mon pain à ceux qui avaient faim,
\par 17 Et mes vêtements à ceux qui étaient nus ; et si je voyais quelqu'un de ma nation mort, ou jeté autour des murs de Ninive, je l'enterrais.
\par 18 Et si le roi Sennachérib avait tué quelqu'un à son arrivée et s'il s'était enfui de la Judée, je les ai enterrés en secret ; car dans sa colère il en tua beaucoup ; mais les corps ne furent pas retrouvés lorsqu'ils furent recherchés auprès du roi.
\par 19 Et quand l'un des Ninivites alla se plaindre de moi auprès du roi, que je les ai enterrés et que je me suis caché ; Comprenant qu'on me cherchait pour être mis à mort, je me retirai de peur.
\par 20 Alors tous mes biens furent emportés de force, et il ne me resta plus rien, à part ma femme Anna et mon fils Tobias.
\par 21 Et il ne s'écoula pas cinquante jours, avant que deux de ses fils le tuèrent, et qu'ils s'enfuirent dans les montagnes d'Ararath ; et Sarchédonus, son fils, régna à sa place ; qui chargea les comptes de son père et toutes ses affaires d'Achiacharus, fils d'Anaël, mon frère.
\par 22 Et Achiacharus me suppliant, je retournai à Ninive. Achiacharus était échanson, gardien du sceau, intendant et chargé des comptes. Sarchédonus le plaça ensuite à sa suite, et il était le fils de mon frère.

\chapter{2}

\par 1 Or, quand je revins à la maison, et que ma femme Anna me fut rendue avec mon fils Tobias, à la fête de la Pentecôte, qui est la sainte fête des sept semaines, un bon dîner me fut préparé, à celui que je me suis assis pour manger.
\par 2 Et quand j'ai vu de la nourriture en abondance, j'ai dit à mon fils : Va et amène le pauvre que tu trouveras parmi nos frères, qui se souvient du Seigneur ; et voici, je m'attarde pour toi.
\par 3 Mais il revint et dit : Père, un des nôtres a été étranglé et jeté dehors sur la place du marché.
\par 4 Alors, avant d'avoir goûté de la viande, je me levai et je l'emmenai dans une chambre jusqu'au coucher du soleil.
\par 5 Alors je reviens, je me lave, et je mange lourdement ma viande,
\par 6 Se souvenant de cette prophétie d'Amos, comme il disait : Vos fêtes seront changées en deuil, et toute votre gaieté en lamentation.
\par 7 C'est pourquoi j'ai pleuré ; et après le coucher du soleil, je suis allé faire un tombeau et je l'ai enterré.
\par 8 Mais mes voisins se moquaient de moi et disaient : Cet homme ne craint pas encore d'être mis à mort à cause de cette affaire : qui s'est enfui ; et pourtant voici, il enterre à nouveau les morts.
\par 9 La même nuit, je revins de l'enterrement, et je dormis près du mur de ma cour, étant pollué et mon visage découvert :
\par 10 Et je ne savais pas qu'il y avait des moineaux dans le mur, et mes yeux étant ouverts, les moineaux jetèrent de la bouse chaude dans mes yeux, et une blancheur apparut dans mes yeux. Et j'allai chez les médecins, mais ils ne m'aidèrent pas. : en outre Achiacharus m'a nourri, jusqu'à ce que j'entre en Elymais.
\par 11 Et ma femme Anna s'occupait des travaux des femmes.
\par 12 Et lorsqu'elle les eut renvoyés chez les propriétaires, ils lui payèrent un salaire et lui donnèrent aussi un chevreau.
\par 13 Et quand il fut dans ma maison, et que je me mis à pleurer, je lui dis : D'où vient ce chevreau ? il n'est pas volé ? le rendre aux propriétaires ; car il n'est pas permis de manger de ce qui a été volé.
\par 14 Mais elle m'a répondu : C'est un cadeau qui dépasse le salaire. Cependant, je ne la croyais pas, mais je lui ordonnais de le rendre aux propriétaires : et j'étais confus à son égard. Mais elle me répondit : Où sont tes aumônes et tes bonnes actions ? voici, toi et toutes tes œuvres êtes connus.

\chapter{3}

\par 1 Alors, affligé, j'ai pleuré, et dans ma tristesse, j'ai prié, disant :
\par 2 O Seigneur, tu es juste, et toutes tes œuvres et toutes tes voies sont miséricorde et vérité, et tu juges véritablement et justement pour toujours.
\par 3 Souviens-toi de moi, et regarde-moi, ne me punis pas pour mes péchés et mes ignorances, ni pour les péchés de mes pères, qui ont péché avant toi :
\par 4 Car ils n'ont pas obéi à tes commandements ; c'est pourquoi tu nous as livrés pour le butin, la captivité et la mort, et pour un proverbe d'opprobre envers toutes les nations parmi lesquelles nous sommes dispersés.
\par 5 Et maintenant tes jugements sont nombreux et vrais : traite-moi selon mes péchés et ceux de mes pères : parce que nous n'avons pas gardé tes commandements, et que nous n'avons pas marché dans la vérité devant toi.
\par 6 Maintenant donc, traite-moi comme bon te semble, et ordonne que mon esprit soit ôté de moi, afin que je sois dissous et devienne terre ; car il est plus avantageux pour moi de mourir que de vivre, parce que j'ai J'ai entendu de faux reproches et j'ai eu beaucoup de tristesse : commande donc que je puisse maintenant être délivré de cette détresse et aller dans le lieu éternel : ne détourne pas ta face de moi.
\par 7 Il arriva le même jour qu'à Ecbatane, ville de Médie Sara, la fille de Raguel fut aussi injuriée par les servantes de son père ;
\par 8 Parce qu'elle avait eu sept maris, qu'Asmodée, le mauvais esprit, avait tués avant qu'ils n'aient couché avec elle. Ne sais-tu pas, disaient-ils, que tu as étranglé tes maris ? tu as déjà eu sept maris, et tu n'as jamais porté le nom d'aucun d'eux.
\par 9 Pourquoi nous bats-tu à cause d'eux ? s'ils sont morts, va après eux, que nous ne te voyons jamais, ni fils ni fille.
\par 10 Lorsqu'elle entendit ces choses, elle fut très triste, au point qu'elle crut s'être étranglée ; et elle dit : Je suis la fille unique de mon père, et si je fais cela, ce sera un opprobre pour lui, et j'emmènerai sa vieillesse avec tristesse jusqu'au tombeau.
\par 11 Alors elle pria vers la fenêtre, et dit : Tu es béni, Seigneur mon Dieu, et ton nom saint et glorieux est béni et honorable pour toujours : que toutes tes œuvres te louent pour toujours.
\par 12 Et maintenant, Seigneur, je tourne mes yeux et ma face vers toi,
\par 13 Et dis : Sortez-moi de la terre, afin que je n'entende plus l'opprobre.
\par 14 Tu sais, Seigneur, que je suis pur de tout péché avec l'homme,
\par 15 Et que je n'ai jamais profané mon nom, ni le nom de mon père, dans le pays de ma captivité : je suis la fille unique de mon père, et il n'a pas non plus d'enfant pour être son héritier, ni aucun proche parent, ni tout fils de lui vivant, pour qui je me garderai pour épouse : mes sept maris sont déjà morts ; et pourquoi devrais-je vivre ? mais s'il ne te plaît pas que je meure, ordonne qu'on me respecte et qu'on me prenne en pitié, afin que je n'entende plus de reproches.
\par 16 Ainsi leurs prières tous deux furent exaucées devant la majesté du grand Dieu.
\par 17 Et Raphaël fut envoyé pour les guérir tous deux, c'est-à-dire pour ôter la blancheur des yeux de Tobit, et pour donner Sara, fille de Raguel, pour femme à Tobias, fils de Tobit ; et pour lier Asmodée le mauvais esprit ; parce qu'elle appartenait à Tobias par droit d'héritage. Au même moment Tobit rentra chez lui et entra dans sa maison, et Sara, fille de Raguel, descendit de sa chambre haute.

\chapter{4}

\par 1 Ce jour-là, Tobit se souvint de l'argent qu'il avait confié à Gabael dans Rages of Media,
\par 2 Et il dit en lui-même : J'ai souhaité la mort ; pourquoi n'appelle-je pas mon fils Tobias pour lui remettre l'argent avant de mourir ?
\par 3 Et après l'avoir appelé, il dit : Mon fils, quand je serai mort, enterre-moi ; et ne méprise pas ta mère, mais honore-la tous les jours de ta vie, et fais ce qui lui plaît, et ne l'afflige pas.
\par 4 Souviens-toi, mon fils, qu'elle a vu beaucoup de dangers pour toi, quand tu étais dans son ventre ; et quand elle sera morte, enterre-la près de moi dans un seul tombeau.
\par 5 Mon fils, souviens-toi du Seigneur notre Dieu tous tes jours, et que ta volonté ne soit pas amenée à pécher ou à transgresser ses commandements : fais la droiture toute ta vie et ne suis pas les voies de l'injustice.
\par 6 Car si tu agis avec vérité, tes actions te réussiront, ainsi qu'à tous ceux qui vivent dans la justice.
\par 7 Fais l'aumône de tes biens ; et quand tu fais l'aumône, que ton regard ne soit pas envieux, et que ton visage ne se détourne pas des pauvres, et la face de Dieu ne se détournera pas de toi.
\par 8 Si tu as de l'abondance, fais l'aumône en conséquence ; si tu n'as que peu, ne crains pas de donner selon ce peu :
\par 9 Car tu t'amasses un bon trésor pour le jour de la nécessité.
\par 10 Parce que cette aumône délivre de la mort et ne permet pas d'entrer dans les ténèbres.
\par 11 Car l'aumône est un bon présent pour quiconque la donne en présence du Très-Haut.
\par 12 Garde-toi de toute prostitution, mon fils, et surtout prends une femme de la postérité de tes pères, et ne prends pas pour femme une femme étrangère qui n'est pas de la tribu de ton père ; car nous sommes les enfants des prophètes, Non. , Abraham, Isaac et Jacob : souviens-toi, mon fils, que nos pères depuis le commencement, même qu'ils ont tous épousé des femmes de leur propre parenté, et ont été bénis dans leurs enfants, et que leur postérité héritera du pays.
\par 13 Maintenant donc, mon fils, aime tes frères, et ne méprise pas dans ton cœur tes frères, les fils et les filles de ton peuple, en ne prenant pas femme d'eux; car dans l'orgueil il y a la destruction et beaucoup de malheur, et dans la débauche. c'est la décadence et le grand besoin ; car l'impudicité est la mère de la famine.
\par 14 Que le salaire de celui qui a travaillé pour toi ne reste pas avec toi, mais donne-le-lui d'emblée ; car si tu sers Dieu, il te le rendra aussi. Sois prudent, mon fils, en toutes choses tu fais, et sois sage dans toutes tes conversations.
\par 15 Ne fais à personne ce que tu hais : ne bois pas de vin pour t'enivrer ; et ne laisse pas l'ivresse t'accompagner dans ton voyage.
\par 16 Donne de ton pain à celui qui a faim, et de tes vêtements à ceux qui sont nus ; et selon ton abondance, fais l'aumône ; et que ton œil ne soit pas envieux quand tu fais l'aumône.
\par 17 Déverse ton pain sur la sépulture du juste, mais ne donne rien aux méchants.
\par 18 Demandez conseil à tous les sages, et ne méprisez aucun conseil utile.
\par 19 Bénis toujours l'Éternel, ton Dieu, et désire qu'il fasse que tes voies soient dirigées, et que tous tes sentiers et tous tes conseils prospèrent ; car aucune nation n'a de conseil ; mais le Seigneur lui-même donne toutes les bonnes choses, et il humilie qui il veut, comme il veut ; Maintenant donc, mon fils, souviens-toi de mes commandements, et ne les laisse pas sortir de ton esprit.
\par 20 Et maintenant, je leur déclare que j'ai confié dix talents à Gabael, fils de Gabrias, à Rages en Médie.
\par 21 Et ne crains pas, mon fils, que nous soyons appauvris; car tu as beaucoup de richesses, si tu crains Dieu, si tu t'éloignes de tout péché et si tu fais ce qui lui plaît.

\chapter{5}

\par 1 Tobias répondit alors et dit : Père, je ferai tout ce que tu m'as commandé.
\par 2 Mais comment puis-je recevoir l'argent, puisque je ne le connais pas ?
\par 3 Alors il lui remit l'écriture et lui dit : Cherche un homme qui puisse marcher avec toi, tant que je suis encore en vie, et je lui donnerai un salaire ; et va recevoir l'argent.
\par 4 C'est pourquoi, quand il alla chercher un homme, il trouva Raphaël qui était un ange.
\par 5 Mais il ne le savait pas ; et il lui dit : Peux-tu m'accompagner à Rages ? et connais-tu bien ces lieux ?
\par 6 À qui l'ange dit : J'irai avec toi, et je connais bien le chemin, car j'ai logé chez notre frère Gabael.
\par 7 Alors Tobias lui dit : Attends pour moi jusqu'à ce que j'en parle à mon père.
\par 8 Alors il lui dit : Va et ne tarde pas. Il entra donc et dit à son père : Voici, j'en ai trouvé un qui m'accompagnera. Alors il dit : Appelle-le-moi, afin que je sache de quelle tribu il est et s'il est un homme de confiance pour t'accompagner.
\par 9 Alors il l'appela, et il entra, et ils se saluèrent.
\par 10 Alors Tobit lui dit : Frère, montre-moi de quelle tribu et de quelle famille tu es.
\par 11 À qui il dit : Cherches-tu une tribu, ou une famille, ou un mercenaire pour aller avec ton fils ? Alors Tobit lui dit : Je voudrais connaître, frère, ta parenté et ton nom.
\par 12 Alors il dit : Je suis Azarias, fils d'Ananias le grand et de tes frères.
\par 13 Alors Tobit dit : Tu es le bienvenu, frère ; ne sois pas maintenant en colère contre moi, parce que j'ai demandé à connaître ta tribu et ta famille ; car tu es mon frère, d'une honnête et bonne souche : car je connais Ananias et Jonathas, fils de ce grand Samaias, alors que nous allions ensemble à Jérusalem pour adorer, et que nous offrions le premier-né et les dixièmes des fruits ; et ils n'ont pas été séduits par l'erreur de nos frères : mon frère, tu es de bonne souche.
\par 14 Mais dis-moi, quel salaire te donnerai-je ? Veux-tu une drachme par jour et les choses nécessaires pour mon propre fils ?
\par 15 Oui, et si vous revenez sains et saufs, j'ajouterai quelque chose à votre salaire.
\par 16 Ils furent donc très contents. Alors il dit à Tobias : Préparez-vous pour le voyage, et Dieu vous envoie un bon voyage. Et quand son fils eut tout préparé pour le voyage, son père dit : Va avec cet homme, et Dieu qui habite dans les cieux, fais réussir ton voyage, et l'ange de Dieu te tient compagnie. Ils partirent donc tous deux, accompagnés du chien du jeune homme.
\par 17 Mais Anne, sa mère, pleura et dit à Tobit : Pourquoi as-tu renvoyé notre fils ? n'est-il pas le bâton de notre main, pour entrer et sortir devant nous ?
\par 18 Ne soyez pas avide d'ajouter de l'argent sur l'argent : mais que ce soit comme un déchet à l'égard de notre enfant.
\par 19 Car ce avec quoi le Seigneur nous a donné pour vivre nous suffit.
\par 20 Alors Tobit lui dit : Ne prends pas garde, ma sœur ; il reviendra sain et sauf, et tes yeux le verront.
\par 21 Car le bon ange lui tiendra compagnie, et son voyage sera prospère, et il reviendra sain et sauf.
\par 22 Alors elle cessa de pleurer.

\chapter{6}

\par 1 Et pendant qu'ils étaient en route, ils arrivèrent le soir au bord du Tigre, et ils y passèrent la nuit.
\par 2 Et lorsque le jeune homme descendit pour se laver, un poisson sauta hors de la rivière et voulut le dévorer.
\par 3 Alors l'ange lui dit : Prends le poisson. Et le jeune homme saisit le poisson et le tira à terre.
\par 4 À qui l'ange dit : Ouvrez le poisson, prenez le cœur, le foie et le fiel, et rangez-les en toute sécurité.
\par 5 Le jeune homme fit donc ce que l'ange lui avait ordonné ; et quand ils eurent rôti le poisson, ils le mangèrent ; puis ils continuèrent tous deux leur chemin, jusqu'à ce qu'ils approchèrent d'Ecbatane.
\par 6 Alors le jeune homme dit à l'ange : Frère Azarias, à quoi servent le cœur, le foie et la ficelle du poisson ?
\par 7 Et il lui dit : Touchant le cœur et le foie, si un diable ou un mauvais esprit trouble quelqu'un, nous devons en faire une fumée devant l'homme ou la femme, et la fête ne sera plus irritée.
\par 8 Quant au fiel, il est bon d'oindre celui qui a les yeux blancs, et il sera guéri.
\par 9 Et lorsqu'ils furent approchés de Rages,
\par 10 L'ange dit au jeune homme : Frère, aujourd'hui nous logerons chez Raguel, qui est ton cousin ; il a aussi une fille unique, nommée Sara ; Je parlerai pour elle, afin qu'elle te soit donnée pour femme.
\par 11 Car à toi appartient le droit d'elle, puisque tu es seul de sa parenté.
\par 12 Et la jeune fille est belle et sage : maintenant donc écoute-moi, et je parlerai à son père ; et quand nous reviendrons de Rages, nous célébrerons le mariage : car je sais que Raguel ne peut pas la marier à un autre selon la loi de Moïse, mais il sera coupable de mort, parce que le droit d'héritage t'appartient plutôt qu'à n'importe qui d'autre. autre.
\par 13 Alors le jeune homme répondit à l'ange : J'ai entendu dire, frère Azarias, que cette jeune fille a été donnée à sept hommes, qui sont tous morts dans la chambre nuptiale.
\par 14 Et maintenant, je suis le fils unique de mon père, et je crains que, si j'entre chez elle, je ne meure comme l'autre auparavant ; car un mauvais esprit l'aime, et il ne fait de mal à personne, mais à ceux qui viens à elle; c'est pourquoi je crains aussi de mourir, et de mettre avec tristesse à cause de moi la vie de mon père et de ma mère au tombeau, car ils n'ont pas d'autre fils pour les enterrer.
\par 15 Alors l'ange lui dit : Ne te souviens-tu pas des préceptes que ton père t'a donnés, selon lesquels tu dois épouser une femme de ta propre parenté ? c’est pourquoi écoute-moi, ô mon frère ; car elle te sera donnée pour femme ; et ne fais pas de compte au mauvais esprit ; car cette même nuit, elle te sera donnée en mariage.
\par 16 Et quand tu entreras dans la chambre nuptiale, tu prendras les cendres du parfum, et tu mettras dessus du cœur et du foie du poisson, et tu en feras une fumée.
\par 17 Et le diable le sentira, s'enfuira et ne reviendra plus; mais quand vous viendrez vers elle, levez-vous tous deux et priez Dieu qui est miséricordieux, qui aura pitié de vous, et sauve-toi : ne crains rien, car elle t'est destinée dès le commencement ; et tu la préserveras, et elle ira avec toi. De plus, je suppose qu'elle te donnera des enfants. Lorsque Tobias eut entendu ces choses, il l'aimait et son cœur s'unissait effectivement à elle.

\chapter{7}

\par 1 Et quand ils furent arrivés à Ecbatane, ils arrivèrent à la maison de Raguel, et Sara les rencontra ; et après qu'ils se furent salués, elle les fit entrer dans la maison.
\par 2 Alors Raguel dit à Edna, sa femme : Comme ce jeune homme ressemble à Tobit, mon cousin !
\par 3 Et Raguel leur demanda : D'où êtes-vous, frères ? A qui ils dirent : Nous sommes des fils de Nephtalim, qui sont captifs à Ninive.
\par 4 Alors il leur dit : Connaissez-vous Tobit, notre parent ? Et ils ont dit : Nous le connaissons. Alors il dit : Est-il en bonne santé ?
\par 5 Et ils dirent : Il est vivant et en bonne santé. Et Tobias dit : C'est mon père.
\par 6 Alors Raguel bondit, le baisa et pleura :
\par 7 Et il le bénit et lui dit : Tu es le fils d'un homme honnête et bon. Mais quand il apprit que Tobie était aveugle, il fut triste et pleura.
\par 8 Et de même Edna, sa femme, et Sara, sa fille, pleurèrent. De plus, ils les divertissaient joyeusement ; et après avoir tué un bélier du troupeau, ils mirent de la viande en réserve sur la table. Alors Tobias dit à Raphaël : Frère Azarias, parle de ce dont tu as parlé en chemin, et que cette affaire soit expédiée.
\par 9 Il en fit donc part à Raguel. Et Raguel dit à Tobias : Mangez et buvez, et amusez-vous.
\par 10 Car il est juste que tu épouses ma fille ; néanmoins je te déclarerai la vérité.
\par 11 J'ai donné ma fille en mariage à sept hommes, qui sont morts la nuit où ils sont venus vers elle ; mais pour le moment, soyez joyeux. Mais Tobias dit : Je ne mangerai rien ici, jusqu'à ce que nous soyons d'accord et que nous nous jurions l'un envers l'autre.
\par 12 Raguel dit : Alors prends-la désormais selon la manière, car tu es son cousin, et elle est à toi, et le Dieu miséricordieux te donne du succès en toutes choses.
\par 13 Alors il appela sa fille Sara, et elle vint vers son père, et il la prit par la main et la donna pour femme à Tobias, en disant : Voici, prends-la selon la loi de Moïse, et emmène-la. à ton père. Et il les bénit ;
\par 14 Et il appela Edna sa femme, et prit du papier, et rédigea un instrument d'alliances, et le scella.
\par 15 Alors ils commencèrent à manger.
\par 16 Après que Raguel appela sa femme Edna et lui dit : Ma sœur, prépare une autre chambre et amène-la-y.
\par 17 Et après avoir fait ce qu'il lui avait ordonné, elle l'y amena ; et elle pleura, et elle reçut les larmes de sa fille, et lui dit :
\par 18 Sois rassurée, ma fille ; le Seigneur du ciel et de la terre te donne de la joie pour ton chagrin : console-toi, ma fille.

\chapitre{8}

\par 1 Et après avoir soupé, ils lui amenèrent Tobias.
\par 2 Et tandis qu'il marchait, il se souvint des paroles de Raphaël, et prit les cendres des parfums, et y mit le cœur et le foie du poisson, et en fit une fumée.
\par 3 L'odeur ayant été sentie par le mauvais esprit, il s'enfuit jusqu'aux extrémités de l'Égypte, et l'ange le lia.
\par 4 Et après qu'ils furent tous deux enfermés ensemble, Tobias se leva du lit et dit : Ma sœur, levez-vous, et prions pour que Dieu ait pitié de nous.
\par 5 Alors Tobias commença à dire : Tu es béni, ô Dieu de nos pères, et ton nom saint et glorieux est béni pour toujours ; que les cieux te bénissent, ainsi que toutes tes créatures.
\par 6 Tu as créé Adam, et tu lui as donné Ève, sa femme, pour aide et soutien. C'est d'eux que sont issus les hommes. Tu as dit : Il n'est pas bon que l'homme soit seul ; faisons-lui une aide semblable à lui.
\par 7 Et maintenant, Seigneur, je ne prends pas ma sœur pour une luxuriante, mais pour une honnêteté : ordonne donc avec miséricorde que nous puissions vieillir ensemble.
\par 8 Et elle dit avec lui : Amen.
\par 9 Ils dormirent donc tous les deux cette nuit-là. Et Raguel se leva et alla faire un tombeau,
\par 10 Je dis : Je crains qu'il ne soit mort aussi.
\par 11 Mais lorsque Raguel fut entré dans sa maison,
\par 12 Il dit à sa femme Edna. Envoyez une des servantes et qu'elle voie s'il est vivant. S'il ne l'est pas, nous l'enterrerons, sans que personne ne le sache.
\par 13 Alors la servante ouvrit la porte, entra et les trouva tous les deux endormis.
\par 14 Et il sortit et leur dit qu'il était vivant.
\par 15 Alors Raguel loua Dieu et dit : Ô Dieu, tu es digne d'être loué de toute louange pure et sainte ; c'est pourquoi que tes saints te louent avec toutes tes créatures ; et que tous tes anges et tes élus te louent pour toujours.
\par 16 Tu mérites d'être loué, car tu m'as rendu joyeux ; et ce que je soupçonnais ne m'est pas venu ; mais tu nous as traités selon ta grande miséricorde.
\par 17 Tu es loué parce que tu as eu pitié de deux enfants uniques de leurs pères : accorde-leur miséricorde, ô Seigneur, et termine leur vie en bonne santé avec joie et miséricorde.
\par 18 Alors Raguel ordonna à ses serviteurs de remplir le tombeau.
\par 19 Et il célébra les noces pendant quatorze jours.
\par 20 Car avant que les jours du mariage fussent accomplis, Raguel lui avait dit par serment qu'il ne partirait pas avant que les quatorze jours du mariage ne soient expirés ;
\par 21 Et alors il prendra la moitié de ses biens et s'en ira en sécurité vers son père ; et je devrais avoir le reste quand ma femme et moi serons morts.

\chapitre{9}

\par 1 Alors Tobias appela Raphaël et lui dit :
\par 2 Frère Azarias, prends avec toi un serviteur et deux chameaux, et va à Rages de Médie vers Gabael, et apporte-moi l'argent, et amène-le aux noces.
\par 3 Car Raguel a juré que je ne partirai pas.
\par 4 Mais mon père compte les jours ; et si je tarde longtemps, il en sera très désolé.
\par 5 Alors Raphaël sortit, et logea chez Gabael, et lui remit l'écriture. Celui-ci sortit des sacs scellés et les lui remit.
\par 6 Et de bon matin, ils sortirent tous deux ensemble et arrivèrent aux noces ; et Tobias bénit sa femme.

\chapitre{10}

\par 1 Or Tobit, son père, comptait chaque jour ; et lorsque les jours du voyage étaient expirés et qu'ils ne revenaient pas,
\par 2 Alors Tobit dit : Sont-ils détenus ? ou bien Gabaël est-il mort, et il n'y a personne pour lui donner l'argent ?
\par 3 C'est pourquoi il était vraiment désolé.
\par 4 Alors sa femme lui dit : Mon fils est mort, puisqu'il reste longtemps ; et elle commença à le gémir et dit :
\par 5 Maintenant, je ne me soucie de rien, mon fils, puisque je t'ai laissé partir, la lumière de mes yeux.
\par 6 À qui Tobit dit : Tais-toi, ne prends pas garde, car il est en sécurité.
\par 7 Mais elle dit : Tais-toi, et ne me trompe pas ; mon fils est mort. Et elle sortait chaque jour par le chemin qu'ils allaient, et ne mangeait pas de viande pendant la journée, et ne cessait de pleurer des nuits entières sur son fils Tobias, jusqu'à ce que soient expirés les quatorze jours du mariage, que Raguel avait juré de lui faire. y passer. Alors Tobias dit à Raguel : Laisse-moi partir, car mon père et ma mère ne cherchent plus à me voir.
\par 8 Mais son beau-père lui dit : Reste avec moi, et j'enverrai vers ton père, et ils lui raconteront comment tes choses se passent.
\par 9 Mais Tobias dit : Non ; mais laisse-moi aller chez mon père.
\par 10 Alors Raguel se leva et lui donna à Sara sa femme et la moitié de ses biens, des serviteurs, du bétail et de l'argent.
\par 11 Et il les bénit et les renvoya, en disant : Le Dieu du ciel vous accorde un bon voyage, mes enfants.
\par 12 Et il dit à sa fille : Honore ton père et ta belle-mère, qui sont maintenant tes parents, afin que j'entende de bonnes nouvelles de toi. Et il l'embrassa. Edna dit aussi à Tobias : Que le Seigneur des cieux te rétablisse, mon cher frère, et accorde-moi de voir tes enfants de ma fille Sara avant de mourir, afin que je puisse me réjouir devant le Seigneur. Voici, je te confie ma fille de confiance spéciale; où sont, ne lui implore pas le mal.

\chapitre{11}

\par 1 Après ces choses, Tobie partit, louant Dieu de lui avoir donné un voyage prospère, et bénit Raguel et Edna sa femme, et continua son chemin jusqu'à ce qu'ils approchèrent de Ninive.
\par 2 Alors Raphaël dit à Tobias : Tu sais, frère, comment tu as quitté ton père :
\par 3 Hâtons-nous devant ta femme, et préparons la maison.
\par 4 Et prends dans ta main le fiel du poisson. Ils partirent donc et le chien les poursuivit.
\par 5 Maintenant Anna était assise, regardant autour de lui le chemin de son fils.
\par 6 Et l'ayant vu venir, elle dit à son père : Voici, ton fils vient, ainsi que l'homme qui allait avec lui.
\par 7 Alors Raphaël dit : Je sais, Tobias, que ton père lui ouvrira les yeux.
\par 8 C'est pourquoi oins ses yeux avec du fiel, et étant piqué avec, il se frottera, et la blancheur tombera, et il te verra.
\par 9 Alors Anna courut et se jeta au cou de son fils, et lui dit : Puisque je t'ai vu, mon fils, désormais je suis contente de mourir. Et ils pleurèrent tous les deux.
\par 10 Tobit sortit aussi vers la porte et trébucha ; mais son fils courut vers lui,
\par 11 Et il saisit son père, et il frappa le fiel des yeux de son père, en disant : Aie bonne espérance, mon père.
\par 12 Et quand ses yeux commencèrent à le brûler, il les frotta ;
\par 13 Et la blancheur s'effaça au coin de ses yeux ; et lorsqu'il aperçut son fils, il tomba sur son cou.
\par 14 Et il pleura et dit : Tu es béni, ô Dieu, et ton nom est béni pour toujours ; et bénis soient tous tes saints anges :
\par 15 Car tu m'as flagellé et tu as eu pitié de moi; car voici, je vois mon fils Tobias. Son fils s'en alla tout joyeux et raconta à son père les grandes choses qui lui étaient arrivées en Médie.
\par 16 Alors Tobit sortit à la rencontre de sa belle-fille à la porte de Ninive, se réjouissant et louant Dieu ; et ceux qui le voyaient partir étaient étonnés de ce qu'il avait recouvré la vue.
\par 17 Mais Tobias rendit grâces devant eux, parce que Dieu avait eu pitié de lui. Et lorsqu'il s'approcha de Sara, sa belle-fille, il la bénit, en disant : Tu es la bienvenue, ma fille. Que Dieu soit béni, qui t'a amenée vers nous, et que ton père et ta mère soient bénis. Et il y eut de la joie parmi tous ses frères qui étaient à Ninive.
\par 18 Et Achiacharus et Nasbas, fils de son frère, arrivèrent.
\par 19 Et les noces de Tobias furent célébrées sept jours avec une grande joie.

\chapitre{12}

\par 1 Alors Tobit appela son fils Tobias et lui dit : Mon fils, veille à ce que cet homme ait son salaire qui va avec toi, et tu devras lui en donner davantage.
\par 2 Et Tobias lui dit : Ô père, cela ne me fera pas de mal de lui donner la moitié de ce que j'ai apporté :
\par 3 Car il m'a ramené vers toi en toute sécurité, il a guéri ma femme, il m'a apporté l'argent et t'a également guéri.
\par 4 Alors le vieil homme dit : Cela lui est dû.
\par 5 Alors il appela l'ange, et il lui dit : Prends la moitié de tout ce que tu as apporté et pars en toute sécurité.
\par 6 Alors il les sépara tous deux et leur dit : Bénissez Dieu, louez-le, magnifiez-le, et louez-le pour les choses qu'il vous a faites aux yeux de tous les vivants. Il est bon de louer Dieu, d'exalter son nom et de montrer honorablement les œuvres de Dieu ; ne tardez donc pas à le louer.
\par 7 Il est bon de garder secret le secret d'un roi, mais il est honorable de révéler les œuvres de Dieu. Faites ce qui est bien et aucun mal ne vous touchera.
\par 8 La prière est bonne avec le jeûne, l'aumône et la justice. Mieux vaut un peu de justice que beaucoup d’injustice. Il vaut mieux faire l'aumône que d'accumuler de l'or :
\par 9 Car l'aumône délivre de la mort et purifie tout péché. Ceux qui pratiquent l’aumône et la justice seront remplis de vie :
\par 10 Mais ceux qui pèchent sont ennemis de leur propre vie.
\par 11 Assurément, je ne te cacherai rien. Car j'ai dit : Il était bon de garder secret le secret d'un roi, mais il était honorable de révéler les œuvres de Dieu.
\par 12 Or, quand tu priais, toi et Sara, ta belle-fille, j'apportais le souvenir de tes prières devant le Saint ; et quand tu enterrais les morts, j'étais également avec toi.
\par 13 Et quand tu n'as pas tardé à te lever et à quitter ton dîner pour aller couvrir les morts, ta bonne action ne m'a pas été cachée : mais j'étais avec toi.
\par 14 Et maintenant, Dieu m'a envoyé pour te guérir, toi et Sara, ta belle-fille.
\par 15 Je suis Raphaël, l'un des sept saints anges, qui présentent les prières des saints, et qui entrent et sortent devant la gloire du Saint.
\par 16 Alors ils furent tous deux troublés et tombèrent la face contre terre, car ils avaient peur.
\par 17 Mais il leur dit : N'ayez crainte, car tout ira bien pour vous ; Louez donc Dieu.
\par 18 Car ce n'est pas grâce à ma faveur, mais c'est par la volonté de notre Dieu que je suis venu ; c'est pourquoi louez-le pour toujours.
\par 19 Tous ces jours, je vous suis apparu ; mais je n'ai ni mangé ni bu, mais vous avez eu une vision.
\par 20 Maintenant donc, rendez grâce à Dieu : car je monte vers celui qui m'a envoyé ; mais écrivez tout ce qui se fait dans un livre.
\par 21 Et quand ils se levèrent, ils ne le virent plus.
\par 22 Alors ils confessèrent les grandes et merveilleuses œuvres de Dieu, et comment l'ange du Seigneur leur était apparu.

\chapitre{13}

\par 1 Alors Tobit écrivit une prière de réjouissance et dit : Béni soit Dieu qui vit éternellement, et béni soit son royaume.
\par 2 Car il fouette et a pitié ; il descend en enfer et en fait remonter : et personne ne peut échapper à sa main.
\par 3 Confessez-le devant les païens, vous, enfants d'Israël, car il nous a dispersés parmi eux.
\par 4 Là, déclarez sa grandeur, et exaltez-le devant tous les vivants : car il est notre Seigneur, et il est le Dieu notre Père pour toujours.
\par 5 Et il nous fouettera à cause de nos iniquités, et aura encore pitié, et nous rassemblera de toutes les nations parmi lesquelles il nous a dispersés.
\par 6 Si vous vous tournez vers lui de tout votre cœur et de tout votre esprit, et que vous agissez honnêtement devant lui, alors il se tournera vers vous et ne vous cachera pas sa face. Voyez donc ce qu'il fera de vous, confessez-le de toute votre bouche, louez le Seigneur de la puissance et exaltez le Roi éternel. Dans le pays de ma captivité, je le loue et je déclare sa puissance et sa majesté à une nation pécheresse. Ô vous, pécheurs, retournez-vous et rendez justice devant lui : qui peut dire s'il vous acceptera et aura pitié de vous ?
\par 7 J'exalterai mon Dieu, et mon âme louera le Roi des cieux, et se réjouira de sa grandeur.
\par 8 Que tous parlent, et que tous le louent pour sa justice.
\par 9 Ô Jérusalem, ville sainte, il te fouettera à cause des œuvres de tes enfants, et il aura encore pitié des fils des justes.
\par 10 Loue l'Éternel, car il est bon, et loue le roi éternel, afin que son tabernacle soit reconstruit en toi avec joie, et qu'il rende là en toi la joie ceux qui sont captifs et l'amour en toi pour toujours ceux qui sont misérables.
\par 11 De nombreuses nations viendront de loin au nom du Seigneur Dieu avec des dons en main, même des dons au Roi des cieux ; toutes les générations te loueront avec une grande joie.
\par 12 Maudits soient tous ceux qui te haïssent, et bénis seront tous ceux qui t'aiment pour toujours.
\par 13 Réjouissez-vous et soyez dans l'allégresse pour les enfants des justes, car ils se rassembleront et béniront le Seigneur des justes.
\par 14 Bienheureux ceux qui t'aiment, car ils se réjouiront de ta paix ; bienheureux ceux qui ont été attristés à cause de tous tes fléaux ; car ils se réjouiront à cause de toi, quand ils auront vu toute ta gloire, et seront dans l'allégresse pour toujours.
\par 15 Que mon âme bénisse Dieu le grand Roi.
\par 16 Car Jérusalem sera bâtie avec des saphirs, des émeraudes et des pierres précieuses, et tes murs, tes tours et tes créneaux d'or pur.
\par 17 Et les rues de Jérusalem seront pavées de béryl, d'escarboucle et de pierres d'Ophir.
\par 18 Et toutes ses rues diront : Alléluia ! et ils le loueront, en disant : Béni soit Dieu, qui l'a exalté à toujours.

\chapitre{14}

\par 1 Tobie finit donc de louer Dieu.
\par 2 Et il avait cinquante-huit ans lorsqu'il perdit la vue, qui lui fut rétablie au bout de huit ans ; et il fit l'aumône, et il croissa dans la crainte du Seigneur Dieu, et le loua.
\par 3 Et quand il fut très âgé, il appela son fils et les fils de son fils, et lui dit : Mon fils, prends tes enfants ; car voici, je suis vieux et je suis prêt à quitter cette vie.
\par 4 Va en Médie, mon fils, car je crois sûrement à ce que Jonas, le prophète, a dit à Ninive, qu'elle sera renversée ; et que pendant un certain temps, la paix sera plutôt en Médie ; et que nos frères resteront dispersés sur la terre depuis ce bon pays : et Jérusalem sera désolée, et la maison de Dieu qui s'y trouve sera brûlée, et sera désolée pour un temps ;
\par 5 Et qu'à nouveau Dieu aura pitié d'eux, et les ramènera dans le pays, où ils bâtiront un temple, mais pas comme le premier, jusqu'à ce que les temps de cet âge soient accomplis ; et ensuite ils reviendront de tous les lieux de leur captivité, et bâtiront glorieusement Jérusalem, et la maison de Dieu y sera bâtie pour toujours avec un édifice glorieux, comme l'ont dit les prophètes.
\par 6 Et toutes les nations se retourneront et craindront véritablement le Seigneur Dieu, et enterreront leurs idoles.
\par 7 Ainsi toutes les nations loueront l'Éternel, et son peuple confessera Dieu, et l'Éternel exaltera son peuple ; et tous ceux qui aiment le Seigneur Dieu en vérité et en justice se réjouiront, faisant preuve de miséricorde envers nos frères.
\par 8 Et maintenant, mon fils, pars de Ninive, car ce que le prophète Jonas a dit arrivera sûrement.
\par 9 Mais garde la loi et les commandements, et montre-toi miséricordieux et juste, afin que tout se passe bien pour toi.
\par 10 Et enterre-moi décemment, et ta mère avec moi ; mais ne restez plus à Ninive. Souviens-toi, mon fils, comment Aman a traité Achiacharus qui l'a élevé, comment de la lumière il l'a fait entrer dans les ténèbres et comment il l'a récompensé de nouveau : cependant Achiacharus a été sauvé, mais l'autre a eu sa récompense, car il est descendu dans les ténèbres. Manassé fit l'aumône et échappa aux pièges de la mort qu'on lui avait tendus ; mais Aman tomba dans le piège et périt.
\par 11 C'est pourquoi maintenant, mon fils, considère ce que fait l'aumône, et comment la justice délivre. Après avoir dit ces choses, il rendit l'âme dans son lit, âgé de cent cinquante-huit ans ; et il l'enterra honorablement.
\par 12 Et quand Anna, sa mère, fut morte, il l'enterra avec son père. Mais Tobias partit avec sa femme et ses enfants à Ecbatane chez Raguel son beau-père,
\par 13 Où il devint vieux avec honneur, et il enterra honorablement son père et sa belle-mère, et il hérita de leurs biens et de ceux de son père Tobit.
\par 14 Et il mourut à Ecbatane en Médie, âgé de cent vingt-sept ans.
\par 15 Mais avant de mourir, il apprit la destruction de Ninive, prise par Nabuchodonosor et Assuérus ; et avant de mourir, il se réjouit de Ninive.

\end{document}