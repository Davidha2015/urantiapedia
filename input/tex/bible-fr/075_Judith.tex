\begin{document}

\title{Judith}


\chapter{1}

\par 1 La douzième année du règne de Nabuchodonosor, qui régnait à Ninive, la grande ville ; du temps d'Arphaxad, qui régnait sur les Mèdes à Ecbatane,
\par 2 Et bâti dans Ecbatane des murs tout autour avec des pierres taillées de trois coudées de large et six coudées de long, et la hauteur du mur était de soixante-dix coudées, et sa largeur de cinquante coudées.
\par 3 Et tu placeras ses tours sur ses portes, hautes de cent coudées, et ses fondations larges de soixante coudées.
\par 4 Et il fit ses portes, des portes qui s'élevaient à soixante-dix coudées de hauteur, et dont la largeur était de quarante coudées, pour la sortie de ses vaillantes armées et pour le déploiement de ses fantassins.
\par 5 En ce temps-là déjà, le roi Nabuchodonosor fit la guerre au roi Arphaxad dans la grande plaine, qui est la plaine limitrophe de Ragau.
\par 6 Et tous ceux qui habitaient dans la région montagneuse, et tous ceux qui habitaient près de l'Euphrate, du Tigre et de l'Hydaspe, et de la plaine d'Arioch, roi des Élyméens, et de très nombreuses nations des fils de Chelod, vinrent vers lui. se sont rassemblés pour la bataille.
\par 7 Alors Nabuchodonosor, roi des Assyriens, envoya vers tous ceux qui habitaient en Perse, et vers tous ceux qui habitaient à l'ouest, et vers ceux qui habitaient en Cilicie, et Damas, et Libanus, et Antiliban, et vers tous ceux qui habitaient sur la côte de la mer. ,
\par 8 Et à ceux des nations qui étaient du Carmel, et de Galaad, et de la haute Galilée, et de la grande plaine d'Esdrelom,
\par 9 Et à tous ceux qui étaient à Samarie et dans ses villes, et au-delà du Jourdain jusqu'à Jérusalem, et Bétane, et Chelus, et Kadès, et le fleuve d'Égypte, et Taphnès, et Ramesse, et tout le pays de Gesem,
\par 10 Jusqu'à ce que vous ayez dépassé Tanis et Memphis, et vers tous les habitants de l'Égypte, jusqu'à ce que vous arriviez aux frontières de l'Éthiopie.
\par 11 Mais tous les habitants du pays se moquèrent du commandement de Nabuchodonosor, roi des Assyriens, et ne l'accompagnèrent pas au combat ; car ils n'avaient pas peur de lui : oui, il était devant eux comme un seul homme, et ils renvoyèrent d'eux ses ambassadeurs sans effet et avec disgrâce.
\par 12 C'est pourquoi Nabuchodonosor fut très irrité contre tout ce pays et jura par son trône et son royaume qu'il se vengerait sûrement sur toutes ces côtes de Cilicie, de Damas et de Syrie, et qu'il tuerait par l'épée tous les habitants du pays de Moab, et les enfants d'Ammon, et toute la Judée, et tous ceux qui étaient en Égypte, jusqu'à ce que vous arriviez aux frontières des deux mers.
\par 13 Puis il marcha en bataille avec sa puissance contre le roi Arphaxad la dix-septième année, et il fut vainqueur dans sa bataille; car il renversa toute la puissance d'Arphaxad, et tous ses cavaliers, et tous ses chars,
\par 14 Et il devint seigneur de ses villes, et vint à Ecbatane, et prit les tours, et pilla les rues, et en changea la beauté en honte.
\par 15 Il prit aussi Arphaxad dans les montagnes de Ragau, le frappa de part en part avec ses dards et le détruisit entièrement ce jour-là.
\par 16 Alors il revint ensuite à Ninive, lui et toute sa troupe de diverses nations constituant une très grande multitude d'hommes de guerre, et là il s'installa et donna un banquet, lui et son armée, pendant cent vingt jours. .

\chapter{2}

\par 1 Et la dix-huitième année, le vingt-deuxième jour du premier mois, on parla dans la maison de Nabuchodonosor, roi des Assyriens, qu'il devait, comme il le disait, se venger sur toute la terre.
\par 2 Alors il appela tous ses officiers et tous ses nobles, et leur communiqua son conseil secret, et conclut de sa propre bouche l'affliction de la terre entière.
\par 3 Alors ils décrétèrent de détruire toute chair qui n'obéirait pas au commandement de sa bouche.
\par 4 Et quand il eut terminé son conseil, Nabuchodonosor, roi des Assyriens, appela Holopherne, chef de son armée, qui était après lui, et lui dit :
\par 5 Ainsi parle le grand roi, le seigneur de toute la terre : Voici, tu sortiras de devant moi, et tu prendras avec toi des hommes qui ont confiance en leurs propres forces, cent vingt mille fantassins ; et le nombre de chevaux et de leurs cavaliers était de douze mille.
\par 6 Et tu iras contre tout le pays de l'Occident, parce qu'ils ont désobéi à mon commandement.
\par 7 Et tu déclareras qu'ils me préparent de la terre et de l'eau ; car j'irai contre eux dans ma colère et je couvrirai toute la face de la terre avec les pieds de mon armée, et je leur donnerai pour un gâtez-leur :
\par 8 De sorte que leurs tués rempliront leurs vallées et leurs ruisseaux et que le fleuve sera rempli de leurs morts, jusqu'à ce qu'il déborde :
\par 9 Et je les conduirai captifs jusqu'aux extrémités de toute la terre.
\par 10 Tu sortiras donc. et prends-moi d'avance toutes leurs côtes ; et s'ils veulent se livrer à toi, tu me les réserveras jusqu'au jour de leur châtiment.
\par 11 Mais concernant ceux qui se rebellent, que ton œil ne les épargne pas ; mais mets-les à l'abattoir, et pille-les partout où tu vas.
\par 12 Car tant que je suis vivant et par la puissance de mon royaume, tout ce que j'ai dit, je le ferai par ma main.
\par 13 Et prends garde à ne transgresser aucun des commandements de ton seigneur, mais accomplis-les pleinement, comme je te l'ai commandé, et ne diffère pas de les faire.
\par 14 Alors Holopherne sortit de chez son seigneur et appela tous les gouverneurs et capitaines et les officiers de l'armée d'Assur ;
\par 15 Et il rassembla les hommes choisis pour le combat, comme son seigneur le lui avait ordonné, au nombre de cent vingt mille, et douze mille archers à cheval ;
\par 16 Et il les rangea, comme une grande armée est ordonnée pour la guerre.
\par 17 Et il prit pour leurs voitures des chameaux et des ânes, en très grand nombre ; et des moutons, des bœufs et des chèvres, en nombre incalculable, pour leur subsistance :
\par 18 Et des vivres en abondance pour chaque homme de l'armée, et une grande quantité d'or et d'argent provenant de la maison du roi.
\par 19 Puis il sortit avec toute sa force pour aller devant le roi Nabuchodonosor dans le voyage, et pour couvrir toute la face de la terre vers l'occident avec leurs chars, leurs cavaliers et leurs fantassins d'élite.
\par 20 Un grand nombre de pays aussi venaient avec eux comme des sauterelles et comme le sable de la terre, car la multitude était innombrable.
\par 21 Et ils partirent de Ninive, faisant trois jours de marche vers la plaine de Bectileth, et campèrent de Bectileth près de la montagne qui est à gauche de la haute Cilicie.
\par 22 Puis il prit toute son armée, ses fantassins, ses cavaliers et ses chars, et partit de là vers la région montagneuse ;
\par 23 Et il détruisit Phud et Lud, et pilla tous les enfants de Rasses et les enfants d'Israël, qui étaient vers le désert au sud du pays des Chelliens.
\par 24 Puis il traversa l'Euphrate, traversa la Mésopotamie et détruisit toutes les hautes villes qui étaient sur le fleuve Arbonaï, jusqu'à ce que vous arriviez à la mer.
\par 25 Et il prit les frontières de la Cilicie, tua tous ceux qui lui résistaient, et arriva aux frontières de Japhet, qui étaient vers le sud, vis-à-vis de l'Arabie.
\par 26 Il entoura aussi tous les enfants de Madian, et incendia leurs tentes, et pilla leurs bergeries.
\par 27 Puis il descendit dans la plaine de Damas au moment de la moisson du blé, et brûla tous leurs champs, et détruisit leurs troupeaux, et il pilla aussi leurs villes, et ravagea complètement leurs pays, et frappa tous leurs petits. des hommes au fil de l'épée.
\par 28 C'est pourquoi sa crainte et sa frayeur s'emparèrent de tous les habitants des côtes de la mer qui étaient à Sidon et à Tyrus, et de ceux qui habitaient à Sour et à Ocina, et de tous ceux qui habitaient à Jemnaan ; et ceux qui habitaient à Azotus et à Ascalon le craignaient grandement.

\chapter{3}

\par 1 Ils lui envoyèrent donc des ambassadeurs pour traiter de paix, disant :
\par 2 Voici, nous, les serviteurs de Nabuchodonosor, le grand roi, sommes couchés devant toi ; utilise-nous comme cela te paraîtra bon.
\par 3 Voici, nos maisons, et tous nos lieux, et tous nos champs de blé, et nos troupeaux, et toutes les huttes de nos tentes sont couchés devant ta face ; utilisez-les comme bon vous semble.
\par 4 Voici, même nos villes et leurs habitants sont tes serviteurs ; viens et traite-les comme bon te semble.
\par 5 Les hommes arrivèrent donc à Holopherne et lui parlèrent de cette manière.
\par 6 Puis il descendit vers la côte de la mer, lui et son armée, et établit des garnisons dans les hautes villes, et en tira des hommes choisis pour secourir.
\par 7 Et eux et tout le pays alentour les reçurent avec des guirlandes, des danses et des tambourins.
\par 8 Pourtant il a abattu leurs frontières et abattu leurs bosquets ; car il avait décrété de détruire tous les dieux du pays, que toutes les nations n'adoreraient que Nabuchodonosor, et que toutes les langues et tribus l'invoqueraient comme dieu. .
\par 9 Il arriva également contre Esdraelon, près de la Judée, du côté du grand détroit de Judée.
\par 10 Et il campa entre Guéba et Scythopolis, et y resta un mois entier, afin de rassembler tous les chariots de son armée.

\chapter{4}

\par 1 Les enfants d'Israël qui habitaient en Judée apprirent tout ce qu'Holoferne, chef de Nabuchodonosor, roi des Assyriens, avait fait aux nations, et de quelle manière il avait pillé tous leurs temples et les avait réduits à néant.
\par 2 C'est pourquoi ils eurent extrêmement peur de lui, et furent inquiets pour Jérusalem et pour le temple de l'Éternel, leur Dieu.
\par 3 Car ils étaient récemment revenus de captivité, et tout le peuple de Judée était récemment rassemblé ; et les ustensiles, et l'autel, et la maison, furent sanctifiés après la profanation.
\par 4 C'est pourquoi ils envoyèrent dans toutes les côtes de Samarie, et dans les villages, et à Bethoron, et Belmen, et Jéricho, et à Choba, et Esora, et dans la vallée de Salem :
\par 5 Et ils s'emparèrent d'avance de tous les sommets des hautes montagnes, et fortifièrent les villages qui s'y trouvaient, et préparèrent des vivres pour l'approvisionnement de la guerre, car leurs champs étaient récemment moissonnés.
\par 6 Joacim, le grand prêtre, qui était alors à Jérusalem, écrivit aux habitants de Béthulie et de Bétomestham, qui est en face d'Esdraelon, vers la campagne, près de Dothaïm :
\par 7 Ils leur furent chargés de garder les passages de la région montagneuse ; car près d'eux se trouvait l'entrée en Judée, et il était facile d'arrêter ceux qui monteraient, car le passage était droit, pour deux hommes au maximum.
\par 8 Et les enfants d'Israël firent ce que Joacim, le grand prêtre, leur avait ordonné, avec les anciens de tout le peuple d'Israël qui habitait à Jérusalem.
\par 9 Alors tous les hommes d'Israël crièrent vers Dieu avec une grande ferveur, et ils humilièrent leur âme avec une grande véhémence :
\par 10 Eux, leurs femmes et leurs enfants, et leur bétail, et tous les étrangers et mercenaires, et leurs serviteurs achetés à prix d'argent, mirent un sac sur leurs reins.
\par 11 Ainsi tous, hommes et femmes, et les petits enfants, et les habitants de Jérusalem, tombèrent devant le temple, jetèrent de la cendre sur leur tête, et étendirent leur sac devant la face de l'Éternel ; ils mirent aussi un sac autour du autel,
\par 12 Et ils crièrent tous d'un commun accord au Dieu d'Israël, qu'il ne livrerait pas leurs enfants pour proie, et leurs femmes pour butin, et les villes de leur héritage à la destruction, et le sanctuaire à la profanation et à l'opprobre. , et pour que les nations se réjouissent.
\par 13 Et Dieu entendit leurs prières et considéra leurs afflictions ; car le peuple jeûna plusieurs jours dans toute la Judée et à Jérusalem devant le sanctuaire du Seigneur tout-puissant.
\par 14 Et Joacim, le grand prêtre, et tous les prêtres qui se tenaient devant l'Éternel, et ceux qui servaient l'Éternel, avaient leurs reins ceints d'un sac, et offraient les holocaustes quotidiens, avec les vœux et les dons gratuits du peuple. ,
\par 15 Et ils avaient de la cendre sur leurs mitres, et ils criaient à l'Éternel de toute leur puissance, afin qu'il regarde avec grâce toute la maison d'Israël.

\chapter{5}

\par 1 Alors on annonça à Holopherne, capitaine en chef de l'armée d'Assur, que les enfants d'Israël s'étaient préparés à la guerre, qu'ils avaient fermé les passages de la région montagneuse et qu'ils avaient fortifié tous les sommets des hautes collines. et avait posé des obstacles dans les pays de campagne :
\par 2 Et il fut très irrité, et il appela tous les princes de Moab, et les capitaines d'Ammon, et tous les gouverneurs de la côte de la mer,
\par 3 Et il leur dit : Dites-moi maintenant, vous, fils de Chanaan, qui est ce peuple qui habite dans la région montagneuse, et quelles sont les villes qu'ils habitent, et quelle est la multitude de leur armée, et dans laquelle est leur pouvoir et leur force, et quel roi est placé sur eux, ou capitaine de leur armée ;
\par 4 Et pourquoi ont-ils décidé de ne pas venir à ma rencontre, plus que tous les habitants de l'Occident.
\par 5 Alors Achior, le chef de tous les fils d'Ammon, dit : Que mon seigneur entende maintenant une parole de la bouche de ton serviteur, et je te déclarerai la vérité concernant ce peuple qui habite près de toi et qui habite le pays. pays montagneux, et aucun mensonge ne sortira de la bouche de ton serviteur.
\par 6 Ce peuple descend des Chaldéens :
\par 7 Et ils résidèrent autrefois en Mésopotamie, parce qu'ils ne voulaient pas suivre les dieux de leurs pères, qui étaient au pays de Chaldée.
\par 8 Car ils quittèrent le chemin de leurs ancêtres et adorèrent le Dieu du ciel, le Dieu qu'ils connaissaient ; ils les chassèrent donc loin de leurs dieux, et ils s'enfuirent en Mésopotamie et y séjournèrent plusieurs jours.
\par 9 Alors leur Dieu leur ordonna de quitter le lieu où ils séjournaient et d'aller au pays de Chanaan, où ils habitèrent et où ils furent enrichis d'or et d'argent et d'un très grand bétail.
\par 10 Mais comme une famine couvrit tout le pays de Chanaan, ils descendirent en Egypte et y séjournèrent, pendant qu'ils se nourrissaient, et ils devinrent là une grande multitude, de sorte qu'on ne pouvait pas dénombrer leur nation.
\par 11 C'est pourquoi le roi d'Égypte s'est soulevé contre eux, et a agi avec eux avec subtilité, et les a rabaissés en travaillant dans la brique, et en a fait des esclaves.
\par 12 Alors ils invoquèrent leur Dieu, et il frappa tout le pays d'Égypte de plaies incurables : alors les Égyptiens les chassèrent hors de leur vue.
\par 13 Et Dieu sécha devant eux la mer Rouge,
\par 14 Et ils les amenèrent sur la montagne de Sina et à Cadès-Barne, et chassèrent tous ceux qui habitaient dans le désert.
\par 15 Ils habitèrent donc au pays des Amoréens, et ils détruisirent par leur force tous ceux d'Esébon, et, passant le Jourdain, ils possédèrent toute la région montagneuse.
\par 16 Et ils chassèrent devant eux les Chananéens, les Phéréziens, les Jébusiens, les Sychémites et tous les Gergésiens, et ils demeurèrent dans ce pays plusieurs jours.
\par 17 Et bien qu'ils n'aient pas péché devant leur Dieu, ils ont prospéré, parce que le Dieu qui hait l'iniquité était avec eux.
\par 18 Mais lorsqu'ils s'écartèrent du chemin qu'il leur avait prescrit, ils furent détruits dans de nombreux combats très douloureux, et furent emmenés captifs dans un pays qui n'était pas le leur, et le temple de leur Dieu fut détruit, et leur les villes furent prises par les ennemis.
\par 19 Mais maintenant ils sont retournés à leur Dieu, et sont revenus des lieux où ils étaient dispersés, et ont possédé Jérusalem, où est leur sanctuaire, et sont assis dans la montagne ; car c'était désolé.
\par 20 Maintenant donc, mon seigneur et gouverneur, s'il y a quelque erreur contre ce peuple, et s'il pèche contre son Dieu, considérons que cela sera sa ruine, et montons, et nous les vaincrons.
\par 21 Mais s'il n'y a pas d'iniquité dans leur nation, que mon seigneur passe maintenant, de peur que leur Seigneur ne les défende et que leur Dieu ne soit pour eux, et que nous ne devenions un opprobre devant le monde entier.
\par 22 Et quand Achior eut fini de dire ces paroles, tout le peuple qui se tenait autour de la tente murmura, et les chefs d'Holoferne et tous les habitants du bord de la mer et de Moab dirent qu'il le tuerait.
\par 23 Car, disent-ils, nous n'aurons pas peur de la face des enfants d'Israël ; car voici, c'est un peuple qui n'a ni force ni puissance pour une bataille acharnée.
\par 24 Maintenant donc, seigneur Holopherne, nous monterons, et ils seront une proie qui sera dévorée par toute ton armée.

\chapter{6}

\par 1 Et lorsque le tumulte des hommes qui étaient autour du conseil eut cessé, Holopherne, chef de l'armée d'Assur, dit à Achior et à tous les Moabites, devant toute la troupe des autres nations :
\par 2 Et qui es-tu, Achior et les mercenaires d'Éphraïm, pour avoir prophétisé contre nous sur ce qui se passe aujourd'hui, et avoir dit que nous ne devions pas faire la guerre aux enfants d'Israël, parce que leur Dieu les défendrait ? et qui est Dieu sinon Nabuchodonosor ?
\par 3 Il enverra sa puissance et les détruira de la surface de la terre, et leur Dieu ne les délivrera pas ; mais nous, ses serviteurs, les détruirons comme un seul homme ; car ils ne sont pas capables de soutenir la puissance de nos chevaux.
\par 4 Car avec eux nous les foulerons aux pieds, et leurs montagnes seront enivrées de leur sang, et leurs champs seront remplis de leurs cadavres, et leurs pas ne pourront pas tenir devant nous, car ils seront complètement périssez, dit le roi Nabuchodonosor, seigneur de toute la terre ; car il dit : Aucune de mes paroles ne sera vaine.
\par 5 Et toi, Achior, mercenaire d'Ammon, qui as prononcé ces paroles au jour de ton iniquité, tu ne verras plus ma face à partir de ce jour, jusqu'à ce que je me venge de cette nation sortie d'Egypte.
\par 6 Et alors l'épée de mon armée et la multitude de ceux qui me servent passeront à travers tes côtés, et tu tomberas parmi leurs tués à mon retour.
\par 7 Maintenant donc mes serviteurs te ramèneront dans la région montagneuse, et te placeront dans l'une des villes de passage.
\par 8 Et tu ne périras pas, jusqu'à ce que tu sois détruit avec eux.
\par 9 Et si tu te persuades dans ton esprit qu'ils seront pris, que ton visage ne tombe pas : je l'ai dit, et aucune de mes paroles ne sera vaine.
\par 10 Alors Holopherne ordonna à ses serviteurs, qui attendaient dans sa tente, de prendre Achior, de l'amener à Béthulie, et de le livrer entre les mains des enfants d'Israël.
\par 11 Alors ses serviteurs le prirent et le firent sortir du camp dans la plaine, et ils allèrent du milieu de la plaine vers la montagne, et arrivèrent aux sources qui étaient sous Béthulie.
\par 12 Et quand les hommes de la ville les virent, ils prirent leurs armes et sortirent de la ville jusqu'au sommet de la colline. Et tous ceux qui utilisaient une fronde les empêchaient de monter en leur jetant des pierres. .
\par 13 Néanmoins, s'étant cachés sous la colline, ils ligotèrent Achior, le jetèrent à terre, le laissèrent au pied de la colline, et retournèrent vers leur seigneur.
\par 14 Mais les Israélites descendirent de leur ville, vinrent vers lui, le détachèrent et l'amenèrent à Béthulie, et le présentèrent aux gouverneurs de la ville.
\par 15 C'étaient alors Ozias, fils de Micha, de la tribu de Siméon, et Chabris, fils de Gothoniel, et Charmis, fils de Melchiel.
\par 16 Et ils convoquèrent tous les anciens de la ville, et tous leurs jeunes gens accoururent avec leurs femmes, à l'assemblée, et ils placèrent Achior au milieu de tout leur peuple. Alors Ozias lui demanda ce qui avait été fait.
\par 17 Et il répondit et leur rapporta les paroles du concile d'Holoferne, et toutes les paroles qu'il avait prononcées au milieu des princes d'Assur, et tout ce qu'Holoferne avait dit fièrement contre la maison d'Israël.
\par 18 Alors le peuple se prosterna, adora Dieu et cria vers Dieu. en disant,
\par 19 Ô Seigneur Dieu du ciel, contemple leur orgueil, et plains la misère de notre nation, et regarde le visage de ceux qui sont sanctifiés pour toi aujourd'hui.
\par 20 Alors ils réconfortèrent Achior et le louèrent grandement.
\par 21 Et Ozias le fit sortir de l'assemblée dans sa maison, et fit un festin aux anciens ; et ils invoquèrent toute la nuit le secours du Dieu d'Israël.

\chapter{7}

\par 1 Le lendemain, Holopherne ordonna à toute son armée et à tout son peuple venu pour prendre son parti, de retirer leur camp contre Béthulie, de prendre d'avance les montées de la région montagneuse et de faire la guerre aux enfants. d'Israël.
\par 2 Alors leurs hommes forts quittèrent leurs camps ce jour-là, et l'armée des hommes de guerre était de cent soixante-dix mille fantassins et douze mille cavaliers, sans compter les bagages, et d'autres hommes qui étaient à pied parmi eux, un très grand nombre. grande multitude.
\par 3 Et ils campèrent dans la vallée près de Béthulie, près de la fontaine, et ils s'étendirent en largeur depuis Dothaïm jusqu'à Belmaim, et en longueur depuis Béthulie jusqu'à Cynamon, qui est vis-à-vis d'Esdraelon.
\par 4 Or les enfants d'Israël, voyant leur multitude, furent très troublés, et dirent chacun à son voisin : Maintenant, ces hommes vont lécher la surface de la terre ; car ni les hautes montagnes, ni les vallées, ni les collines ne peuvent supporter leur poids.
\par 5 Alors chacun prit ses armes de guerre, et après avoir allumé du feu sur leurs tours, ils restèrent et veillèrent toute la nuit.
\par 6 Mais le deuxième jour, Holopherne fit sortir tous ses cavaliers sous les yeux des enfants d'Israël qui étaient à Béthulie,
\par 7 Et il regarda les passages qui montaient à la ville, et arriva aux sources de leurs eaux, et les prit, et établit sur elles des garnisons d'hommes de guerre, et il se dirigea lui-même vers son peuple.
\par 8 Alors tous les chefs des enfants d'Esaü, tous les gouverneurs du peuple de Moab et les chefs des côtes de la mer vinrent vers lui et lui dirent :
\par 9 Que notre seigneur entende maintenant une parole, afin qu'il n'y ait pas de renversement dans ton armée.
\par 10 Car ce peuple des enfants d'Israël ne se fie pas à ses lances, mais à la hauteur des montagnes où il habite, car il n'est pas facile de monter au sommet de ses montagnes.
\par 11 Maintenant donc, mon seigneur, ne combats pas contre eux en bataille, et pas un seul homme de ton peuple ne périra.
\par 12 Reste dans ton camp, et garde tous les hommes de ton armée, et que tes serviteurs mettent entre leurs mains la source d'eau qui sort du pied de la montagne.
\par 13 Car tous les habitants de Béthulie tirent de là leur eau ; la soif les tuera, et ils abandonneront leur ville, et nous et notre peuple monterons au sommet des montagnes qui sont proches, et camperons sur elles, pour veiller à ce que personne ne sorte de la ville.
\par 14 Ainsi, eux, leurs femmes et leurs enfants seront consumés par le feu, et avant que l'épée ne vienne contre eux, ils seront renversés dans les rues où ils habitent.
\par 15 Ainsi tu leur rendras une mauvaise récompense ; parce qu'ils se sont rebellés et n'ont pas rencontré ta personne en paix.
\par 16 Et ces paroles plurent à Holopherne et à tous ses serviteurs, et il décida de faire ce qu'ils avaient dit.
\par 17 Alors le camp des enfants d'Ammon partit, et avec eux cinq mille Assyriens, et ils campèrent dans la vallée, et prirent les eaux et les sources des eaux des enfants d'Israël.
\par 18 Alors les enfants d'Ésaü montèrent avec les enfants d'Ammon, et campèrent dans la montagne en face de Dothaïm ; et ils en envoyèrent quelques-uns vers le sud et vers l'est, vers Ekrebel, qui est près de Chusi, c'est sur le ruisseau Mochmur ; et le reste de l'armée des Assyriens campait dans la plaine et couvrait tout le pays ; et leurs tentes et leurs voitures étaient dressées devant une très grande multitude.
\par 19 Alors les enfants d'Israël invoquèrent l'Éternel, leur Dieu, parce que leur cœur défaillait, car tous leurs ennemis les entouraient, et il n'y avait aucun moyen de s'échapper du milieu d'eux.
\par 20 Ainsi toute la compagnie d'Assur resta autour d'eux, tant leurs fantassins, leurs chars et leurs cavaliers, pendant trente-quatre jours, de sorte que tous leurs vaisseaux d'eau manquèrent à tous les habitants de Béthulie.
\par 21 Et les citernes furent vidées, et ils n'eurent pas d'eau pour boire à leur faim pendant un jour ; car ils leur donnaient à boire avec mesure.
\par 22 C'est pourquoi leurs jeunes enfants étaient découragés, et leurs femmes et leurs jeunes hommes défaillirent de soif et tombèrent dans les rues de la ville et près des passages des portes, et ils n'avaient plus aucune force en eux.
\par 23 Alors tout le peuple se rassembla auprès d'Ozias et auprès du chef de la ville, jeunes hommes, femmes et enfants, et crièrent d'une voix forte, et dirent devant tous les anciens :
\par 24 Que Dieu soit juge entre nous et vous ; car vous nous avez fait un grand tort, en ce que vous n'avez pas exigé la paix des enfants d'Assur.
\par 25 Car maintenant nous n'avons aucun secours ; mais Dieu nous a vendus entre leurs mains, afin que nous soyons jetés devant eux avec soif et une grande destruction.
\par 26 Maintenant donc, appelle-les chez toi, et livre toute la ville en butin aux habitants d'Holoferne et à toute son armée.
\par 27 Car il vaut mieux pour nous être leur butin que de mourir de soif ; car nous serons ses serviteurs, afin que nos âmes vivent et ne voient pas sous nos yeux la mort de nos enfants, ni celle de nos enfants. nos femmes ni nos enfants ne meurent.
\par 28 Nous prenons à témoin contre vous le ciel et la terre, et notre Dieu et Seigneur de nos pères, qui nous punit selon nos péchés et les péchés de nos pères, pour ne pas faire ce que nous avons dit aujourd'hui.
\par 29 Alors il y eut de grands pleurs d'un commun accord au milieu de l'assemblée ; et ils crièrent à haute voix au Seigneur Dieu.
\par 30 Alors Ozias leur dit : Frères, prenez courage, supportons encore cinq jours, pendant lesquels l'Éternel notre Dieu tournera sa miséricorde vers nous ; car il ne nous abandonnera pas complètement.
\par 31 Et si ces jours passent et que rien ne nous vient en aide, j'agirai selon ta parole.
\par 32 Et il dispersa le peuple, chacun à sa charge ; Et ils se dirigèrent vers les murs et les tours de leur ville, et envoyèrent les femmes et les enfants dans leurs maisons ; et ils furent très mal amenés dans la ville.

\chapitre{8}

\par 1 Or, à ce moment-là, Judith l'apprit, qui était la fille de Merari, le fils de Ox, le fils de Joseph, le fils d'Ozel, le fils d'Elcia, le fils d'Ananias, le fils de Gédéon, le fils de Raphaim, fils d'Acitho, fils d'Eliu, fils d'Eliab, fils de Nathanaël, fils de Samaël, fils de Salasadal, fils d'Israël.
\par 2 Et Manassé était son mari, de sa tribu et de sa famille, qui mourut pendant la moisson de l'orge.
\par 3 Car pendant qu'il surveillait ceux qui liaient les gerbes dans les champs, la chaleur lui frappa la tête, et il tomba sur son lit, et mourut dans la ville de Béthulie. Et ils l'enterrèrent avec ses pères dans le champ entre Dothaïm. et Balamo.
\par 4 Judith resta donc veuve dans sa maison pendant trois ans et quatre mois.
\par 5 Et elle lui fit une tente sur le toit de sa maison, et mit un sac sur ses reins et revêtit les vêtements de sa veuve.
\par 6 Et elle jeûna tous les jours de son veuvage, sauf les veilles de sabbats, et les sabbats, et les veilles des nouvelles lunes, et les nouvelles lunes, et les fêtes et jours solennels de la maison d'Israël.
\par 7 Elle était aussi d'un beau visage et très belle à voir ; et son mari Manassé lui avait laissé de l'or et de l'argent, des serviteurs et des servantes, du bétail et des terres ; et elle resta sur eux.
\par 8 Et personne ne lui a dit un mauvais mot ; ar elle craignait Dieu beaucoup.
\par 9 Or, lorsqu'elle entendit les mauvaises paroles du peuple contre le gouverneur, qu'il s'évanouissait à cause du manque d'eau ; car Judith avait entendu toutes les paroles qu'Ozias leur avait dites, et qu'il avait juré de livrer la ville aux Assyriens au bout de cinq jours ;
\par 10 Puis elle envoya sa servante, qui avait le gouvernement de tout ce qu'elle possédait, appeler Ozias, Chabris et Charmis, les anciens de la ville.
\par 11 Et ils s'approchèrent d'elle, et elle leur dit : Écoutez-moi maintenant, ô vous, gouverneurs des habitants de Béthulie ! Car les paroles que vous avez prononcées aujourd'hui devant le peuple ne sont pas vraies, concernant ce serment que vous avez fait. et prononcé entre Dieu et vous, et nous avons promis de livrer la ville à nos ennemis, à moins que d'ici ces jours le Seigneur ne se vienne à votre secours.
\par 12 Et maintenant, qui êtes-vous, vous qui avez tenté Dieu aujourd'hui, et qui vous tenez à la place de Dieu parmi les enfants des hommes ?
\par 13 Et maintenant, essayez le Seigneur Tout-Puissant, mais vous ne saurez jamais rien.
\par 14 Car vous ne pouvez pas trouver la profondeur du cœur de l'homme, ni percevoir les choses qu'il pense : alors comment pouvez-vous sonder Dieu, qui a fait toutes ces choses, et connaître sa pensée, ou comprendre son dessein ? Non, mes frères, n'irritez pas le Seigneur notre Dieu.
\par 15 Car s'il ne nous aide pas dans ces cinq jours, il a le pouvoir de nous défendre quand il le veut, même chaque jour, ou de nous détruire devant nos ennemis.
\par 16 Ne liez pas les conseils du Seigneur notre Dieu ; car Dieu n'est pas comme un homme pour être menacé ; il n'est pas non plus comme le fils de l'homme, pour qu'il hésite.
\par 17 Attendons donc son salut, et invoquons-le pour nous secourir, et il entendra notre voix, s'il lui plaît.
\par 18 Car il n'est apparu personne de notre époque, et il n'y a aujourd'hui personne, ni tribu, ni famille, ni peuple, ni ville parmi nous, qui adorent des dieux faits de main d'homme, comme cela a été autrefois.
\par 19 C'est pourquoi nos pères ont été livrés à l'épée et au butin, et ont connu une grande chute devant nos ennemis.
\par 20 Mais nous ne connaissons aucun autre dieu, c'est pourquoi nous espérons qu'il ne nous méprisera pas, ni aucun membre de notre nation.
\par 21 Car si nous sommes pris ainsi, toute la Judée sera dévastée, et notre sanctuaire sera détruit ; et il en exigera la profanation de notre bouche.
\par 22 Et le massacre de nos frères, et la captivité du pays, et la désolation de notre héritage, tomberont sur nos têtes parmi les païens, partout où nous serons en esclavage ; et nous serons une offense et un opprobre pour tous ceux qui nous possèdent.
\par 23 Car notre servitude ne sera pas dirigée vers la faveur, mais l'Éternel notre Dieu la tournera en déshonneur.
\par 24 Maintenant donc, ô frères, donnons l'exemple à nos frères, parce que leur cœur dépend de nous, et que le sanctuaire, la maison et l'autel reposent sur nous.
\par 25 Rendons également grâce au Seigneur notre Dieu, qui nous éprouve, comme il a éprouvé nos pères.
\par 26 Rappelez-vous ce qu'il a fait à Abraham, et comment il a éprouvé Isaac, et ce qui est arrivé à Jacob en Mésopotamie de Syrie, lorsqu'il gardait les brebis de Laban, le frère de sa mère.
\par 27 Car il ne nous a pas éprouvés par le feu, comme il l'a fait eux, pour examiner leurs coeurs, et il ne s'est pas vengé de nous; mais l'Éternel fouette ceux qui s'approchent de lui, pour les avertir.
\par 28 Alors Ozias lui dit : Tout ce que tu as dit, tu l'as dit avec bon cœur, et personne ne peut contredire tes paroles.
\par 29 Car ce n'est pas le premier jour où ta sagesse se manifeste ; mais dès le commencement de tes jours, tout le peuple connaît ton intelligence, parce que la disposition de ton cœur est bonne.
\par 30 Mais le peuple avait très soif, et nous a forcés de lui faire ce que nous lui avons dit, et de nous faire un serment que nous ne romprons pas.
\par 31 C'est pourquoi maintenant, prie pour nous, car tu es une femme pieuse, et le Seigneur nous enverra de la pluie pour remplir nos citernes, et nous ne nous évanouirons plus.
\par 32 Alors Judith leur dit : Écoutez-moi, et je ferai une chose qui passera de génération en génération en faveur des enfants de notre nation.
\par 33 Vous resterez cette nuit à la porte, et je sortirai avec ma servante ; et dans les jours que vous avez promis de livrer la ville à nos ennemis, l'Éternel visitera Israël par ma main.
\par 34 Mais ne vous interrogez pas sur mon acte, car je ne vous le dirai pas jusqu'à ce que les choses que je fais soient achevées.
\par 35 Alors Ozias et les princes lui dirent : Va en paix, et le Seigneur Dieu soit devant toi, pour te venger de nos ennemis.
\par 36 Ils revinrent donc de la tente et se rendirent dans leurs quartiers.

\chapitre{9}

\par 1 Judith tomba la face contre terre, et mit de la cendre sur sa tête, et découvrit le sac dont elle était vêtue ; Et vers le moment où l'encens de ce soir-là était offert à Jérusalem, dans la maison du Seigneur Judith, elle s'écria d'une voix forte et dit :
\par 2 O Seigneur Dieu de mon père Siméon, à qui tu as donné une épée pour te venger des étrangers, qui ont dénoué la ceinture d'une jeune fille pour la souiller, et ont découvert la cuisse à sa honte, et ont pollué sa virginité à son reproche. ; car tu as dit : Il n'en sera pas ainsi ; et pourtant ils l'ont fait :
\par 3 C'est pourquoi tu as fait tuer leurs chefs, de sorte qu'ils sont morts dans le sang sur leur lit, trompés, et ont frappé les serviteurs avec leurs seigneurs, et les seigneurs sur leurs trônes ;
\par 4 Et tu as livré leurs femmes en proie, et leurs filles en captivité, et tout leur butin pour être partagé entre tes chers enfants ; qui étaient touchés par ton zèle, et qui abhorraient la souillure de leur sang, et qui t'appelaient à l'aide : ô Dieu, ô mon Dieu, écoute-moi aussi, veuve.
\par 5 Car tu as fait non seulement ces choses, mais aussi les choses qui sont arrivées avant et qui ont suivi après ; tu as pensé aux choses qui sont maintenant et à celles qui sont à venir.
\par 6 Oui, ce que tu as déterminé était à portée de main, et tu as dit : Voici, nous sommes ici ; car toutes tes voies sont préparées, et tes jugements sont dans ta prescience.
\par 7 Car voici, les Assyriens sont multipliés dans leur puissance ; ils sont exaltés avec le cheval et l'homme ; ils se glorifient de la force de leurs fantassins ; ils font confiance au bouclier, à la lance, à l'arc et à la fronde ; et ne sais pas que tu es l'Éternel qui brise les batailles : l'Éternel est ton nom.
\par 8 Jetez leur force dans ta puissance, et abaisse leur force dans ta colère ; car ils ont résolu de profaner ton sanctuaire, de profaner le tabernacle où repose ton glorieux nom et d'abattre par l'épée la corne de ton autel. .
\par 9 Regarde leur orgueil, et envoie ta colère sur leurs têtes : remets entre mes mains, qui suis veuve, le pouvoir que j'ai conçu.
\par 10 Frappez par la tromperie de mes lèvres le serviteur avec le prince, et le prince avec le serviteur ; détruisez leur majesté par la main d'une femme.
\par 11 Car ta puissance ne réside pas dans la multitude, ni ta puissance dans les hommes forts; car tu es le Dieu des affligés, le secours des opprimés, le soutien des faibles, le protecteur des abandonnés, le sauveur de ceux qui sont sans espoir.
\par 12 Je te prie, je te prie, ô Dieu de mon père et Dieu de l'héritage d'Israël, Seigneur des cieux et de la terre, Créateur des eaux, roi de toute créature, écoute ma prière :
\par 13 Et fais de ma parole et de ma tromperie leur blessure et leur meurtrissure, qui ont conçu des choses cruelles contre ton alliance et contre ta maison sainte, contre les sommets de Sion et contre la maison de possession de tes enfants.
\par 14 Et fais reconnaître à chaque nation et tribu que tu es le Dieu de toute puissance et de toute puissance, et qu'il n'y a personne d'autre que toi qui protège le peuple d'Israël.

\chapitre{10}

\par 1 Après cela, elle avait cessé de crier au Dieu d'Israël, et avait mis fin à toutes ces paroles.
\par 2 Elle se leva là où elle était tombée, appela sa servante, et descendit dans la maison où elle demeurait les jours de sabbat et de fête,
\par 3 Et elle ôta le sac qu'elle portait, et ôta les vêtements de son veuvage, et lava tout son corps avec de l'eau, et s'oignit d'un onguent précieux, et tressa les cheveux de sa tête, et mit un fatiguez-vous et revêtez les vêtements de joie dont elle était vêtue pendant la vie de Manassé, son mari.
\par 4 Et elle prit des sandales à ses pieds, et mit sur elle ses bracelets, et ses chaînes, et ses bagues, et ses boucles d'oreilles, et tous ses ornements, et se para courageusement, pour attirer les yeux de tous les hommes qui verraient. son.
\par 5 Puis elle donna à sa servante une bouteille de vin et une cruche d'huile, et elle remplit un sac de blé rôti, de morceaux de figues et de pain fin ; alors elle plia toutes ces choses ensemble et les posa sur elle.
\par 6 Ils sortirent donc vers la porte de la ville de Béthulie, et trouvèrent là Ozias et les anciens de la ville, Chabris et Charmis.
\par 7 Et quand ils la virent que son visage était changé et que ses vêtements étaient changés, ils furent très étonnés de sa beauté, et lui dirent.
\par 8 Le Dieu, le Dieu de nos pères te fait grâce et accomplit tes entreprises à la gloire des enfants d'Israël et à l'exaltation de Jérusalem. Ensuite, ils ont adoré Dieu.
\par 9 Et elle leur dit : Commandez qu'on m'ouvre les portes de la ville, afin que je puisse sortir pour accomplir les choses dont vous m'avez parlé. Ils ordonnèrent donc aux jeunes gens de s'ouvrir à elle, comme elle l'avait dit.
\par 10 Et quand ils eurent fait cela, Judith sortit, elle et sa servante avec elle ; Et les hommes de la ville la surveillèrent jusqu'à ce qu'elle descendit la montagne et qu'elle eut dépassé la vallée et ne put plus la voir.
\par 11 Ils s'avancèrent ainsi directement dans la vallée ; et la première garde des Assyriens la rencontra,
\par 12 Et il la prit et lui demanda : De quel peuple es-tu ? et d'où viens-tu ? et où vas-tu ? Et elle dit : Je suis une femme des Hébreux, et j'ai fui loin d'eux, car ils vous seront donnés pour être consumés.
\par 13 Et je me présente devant Holopherne, le chef de votre armée, pour déclarer des paroles de vérité ; et je lui montrerai le chemin par lequel il pourra avancer et conquérir toute la région montagneuse, sans perdre le corps ni la vie d'aucun de ses hommes.
\par 14 Or, quand les hommes entendirent ses paroles et virent son visage, ils furent très étonnés de sa beauté, et lui dirent :
\par 15 Tu as sauvé ta vie en ce que tu es descendu en toute hâte devant notre seigneur ; viens donc maintenant à sa tente, et quelques-uns d'entre nous te conduiront jusqu'à ce qu'ils t'aient livré entre ses mains.
\par 16 Et quand tu te tiendras devant lui, n'aie pas peur dans ton cœur, mais montre-lui selon ta parole ; et il te suppliera bien.
\par 17 Alors ils choisirent parmi eux cent hommes pour l'accompagner, ainsi que sa servante ; et ils l'amenèrent à la tente d'Holoferne.
\par 18 Alors il y eut un attroupement dans tout le camp; car on entendit du bruit parmi les tentes, et ils l'entourèrent, alors qu'elle se tenait hors de la tente d'Holoferne, jusqu'à ce qu'on lui parlât d'elle.
\par 19 Et ils s'étonnaient de sa beauté, et admiraient les enfants d'Israël à cause d'elle, et chacun disait à son prochain : Qui mépriserait ce peuple, qui a parmi lui de telles femmes ? Il n'est certainement pas bon qu'il reste un seul d'entre eux qui, s'il est laissé partir, pourrait tromper la terre entière.
\par 20 Et ceux qui étaient couchés près d'Holoferne sortirent, ainsi que tous ses serviteurs, et ils l'amenèrent dans la tente.
\par 21 Holopherne se reposait sur son lit, sous un dais tissé de pourpre, d'or, d'émeraudes et de pierres précieuses.
\par 22 Ils lui montrèrent donc cette femme ; et il sortit devant sa tente, avec des lampes d'argent devant lui.
\par 23 Et quand Judith fut arrivée devant lui et ses serviteurs, ils furent tous émerveillés par la beauté de son visage ; et elle tomba sur sa face et lui rendit hommage ; et ses serviteurs la relevèrent.

\chapitre{11}

\par 1 Alors Holopherne lui dit : Femme, console-toi, ne crains rien dans ton cœur ; car je n'ai jamais fait de mal à quiconque voulait servir Nabuchodonosor, le roi de toute la terre.
\par 2 Maintenant donc, si ton peuple qui habite dans les montagnes n'avait pas allumé la lumière près de moi, je n'aurais pas levé ma lance contre eux; mais ils se sont fait ces choses à eux-mêmes.
\par 3 Mais maintenant, dis-moi pourquoi tu t'es enfui loin d'eux et pourquoi tu es venu vers nous : car tu es venu pour nous protéger ; console-toi, tu vivras cette nuit, et par la suite :
\par 4 Car personne ne te fera de mal, mais il te suppliera bien, comme ils le font les serviteurs du roi Nabuchodonosor, mon seigneur.
\par 5 Alors Judith lui dit : Reçois les paroles de ton serviteur, et permets que ta servante parle en ta présence, et je ne déclarerai aucun mensonge à mon seigneur cette nuit.
\par 6 Et si tu suis les paroles de ta servante, Dieu fera en sorte que la chose se réalise parfaitement par toi ; et mon seigneur ne faillira pas à ses desseins.
\par 7 Aussi vivant que Nabuchodonosor, roi de toute la terre, et aussi vivant que sa puissance, lui qui t'a envoyé pour défendre tout être vivant, car non seulement les hommes le serviront, mais aussi les bêtes des champs et les animaux. le bétail et les oiseaux du ciel vivront par ta puissance sous Nabuchodonosor et toute sa maison.
\par 8 Car nous avons entendu parler de ta sagesse et de ta politique, et on rapporte dans toute la terre que tu es le seul excellent dans tout le royaume, puissant en connaissance et merveilleux en faits de guerre.
\par 9 Quant à ce dont Achior a parlé dans ton conseil, nous avons entendu ses paroles ; car les hommes de Béthulie l'ont sauvé, et il leur a raconté tout ce qu'il t'avait dit.
\par 10 C'est pourquoi, ô seigneur et gouverneur, ne respecte pas sa parole ; mais garde-le dans ton cœur, car c'est vrai : car notre nation ne sera pas punie, et l'épée ne peut pas non plus prévaloir contre elle, à moins qu'elle ne pèche contre son Dieu.
\par 11 Et maintenant, afin que mon seigneur ne soit pas vaincu et contrarié dans son dessein, même la mort est maintenant tombée sur eux, et leur péché les a rattrapés, avec lesquels ils provoqueront la colère de leur Dieu chaque fois qu'ils feront ce qui ne convient pas. être fait:
\par 12 Car leurs vivres leur manquent, et toute leur eau est rare, et ils ont résolu de s'emparer de leur bétail, et ont résolu de consommer toutes ces choses que Dieu leur a défendues de manger par ses lois.
\par 13 Et ils sont résolus à dépenser les prémices des dixièmes du vin et de l'huile qu'ils avaient sanctifiés et réservés aux prêtres qui servent à Jérusalem devant la face de notre Dieu ; ces choses qu'il n'est permis à aucun peuple de toucher avec ses mains.
\par 14 Car ils en ont envoyé quelques-uns à Jérusalem, parce que ceux qui y habitent ont fait de même, pour leur apporter une licence du sénat.
\par 15 Maintenant, quand ils leur feront part de leur message, ils le feront aussitôt, et ils te seront livrés pour être détruits le même jour.
\par 16 C'est pourquoi, moi, ta servante, sachant tout cela, je me suis enfui loin d'eux ; et Dieu m'a envoyé pour accomplir avec toi des choses dont toute la terre et tous ceux qui l'entendront seront étonnés.
\par 17 Car ton serviteur est religieux, et il sert le Dieu du ciel jour et nuit. Maintenant donc, mon seigneur, je resterai avec toi, et ton serviteur sortira de nuit dans la vallée, et je prierai Dieu, et il me dira quand ils auront commis leurs péchés :
\par 18 Et je viendrai et te le montrerai; alors tu sortiras avec toute ton armée, et il n'y en aura aucun qui te résistera.
\par 19 Et je te conduirai au milieu de la Judée, jusqu'à ce que tu viennes devant Jérusalem ; et je placerai ton trône au milieu d'elle ; et tu les conduiras comme des brebis qui n'ont pas de berger, et un chien n'ouvrira pas même la gueule contre toi. Car ces choses m'ont été dites selon ma prescience, et elles m'ont été déclarées, et je suis envoyé pour le dire. te.
\par 20 Alors ses paroles plurent à Holopherne et à tous ses serviteurs ; et ils furent étonnés de sa sagesse, et dirent :
\par 21 Il n'y a pas de femme pareille d'un bout à l'autre de la terre, tant pour la beauté de son visage que pour la sagesse de ses paroles.
\par 22 Holopherne lui dit de même. Dieu a bien fait de t'envoyer devant le peuple, afin que la force soit entre nos mains et la destruction sur ceux qui méprisent mon seigneur.
\par 23 Et maintenant tu es à la fois beau de visage et spirituel dans tes paroles. Si tu fais ce que tu as dit, ton Dieu sera mon Dieu, et tu habiteras dans la maison du roi Nabuchodonosor, et tu seras célèbre par la terre entière.

\chapitre{12}

\par 1 Puis il ordonna de l'amener là où était son assiette ; et ordonna qu'ils lui préparent de ses propres viandes, et qu'elle boive de son propre vin.
\par 2 Et Judith dit : Je n'en mangerai pas, de peur qu'il n'y ait une offense ; mais on me fera provision des choses que j'ai apportées.
\par 3 Alors Holopherne lui dit : Si ta provision venait à manquer, comment te donnerions-nous une pareille ? car il n'y a chez nous aucun membre de ta nation.
\par 4 Alors Judith lui dit : Tant que ton âme est vivante, mon seigneur, ta servante ne dépensera pas ce que j'ai, avant que le Seigneur n'ait accompli par ma main les choses qu'il a déterminées.
\par 5 Alors les serviteurs d'Holoferne la conduisirent dans la tente, et elle dormit jusqu'à minuit, et elle se leva vers la veille du matin,
\par 6 Et envoyé à Holopherne, sauvant : Que mon seigneur ordonne maintenant que ta servante aille à la prière.
\par 7 Alors Holopherne ordonna à ses gardes de ne pas l'arrêter. Elle resta ainsi trois jours dans le camp, sortit de nuit dans la vallée de Béthulie et se lavait dans une fontaine d'eau près du camp.
\par 8 Et quand elle sortit, elle supplia l'Éternel, le Dieu d'Israël, de diriger son chemin vers l'élévation des enfants de son peuple.
\par 9 Elle entra donc pure et resta dans la tente jusqu'à ce qu'elle mangeât sa viande le soir.
\par 10 Et le quatrième jour, Holopherne fit un festin uniquement à ses propres serviteurs, et n'invita aucun des officiers au festin.
\par 11 Alors il dit à Bagoas, l'eunuque, qui avait la charge de tout ce qui lui appartenait : Va maintenant, et persuade cette femme hébraïque qui est avec toi, qu'elle vienne vers nous, mange et boive avec nous.
\par 12 Car voici, ce serait une honte pour notre personne, si nous laissons partir une telle femme sans avoir eu sa compagnie ; car si nous ne l’attirons pas vers nous, elle se moquera de nous.
\par 13 Alors Bagoas sortit de devant Holopherne, et vint vers elle, et il dit : Que cette belle demoiselle n'ait pas peur de venir vers mon seigneur, et d'être honorée en sa présence, et de boire du vin, et de se réjouir avec nous. et sois devenue aujourd'hui l'une des filles des Assyriens qui servent dans la maison de Nabuchodonosor.
\par 14 Alors Judith lui dit : Qui suis-je maintenant, pour contredire mon seigneur ? sûrement, tout ce qui lui plaira, je le ferai promptement, et ce sera ma joie jusqu'au jour de ma mort.
\par 15 Elle se leva donc, et se para de ses vêtements et de tous ses vêtements de femme, et sa servante alla et posa pour elle sur le sol, face à Holopherne, des peaux douces qu'elle avait reçues de Bagoa pour son usage quotidien, afin qu'elle puisse asseyez-vous et mangez dessus.
\par 16 Or, lorsque Judith entra et s'assit, Holopherne eut le cœur ravi d'elle, et son esprit fut ému, et il désirait grandement sa compagnie ; car il a attendu un certain temps pour la tromper, depuis le jour où il l'avait vue.
\par 17 Alors Holopherne lui dit : Bois maintenant et réjouis-toi avec nous.
\par 18 Alors Judith dit : Je vais boire maintenant, mon seigneur, parce que ma vie est magnifiée en moi aujourd'hui plus que tous les jours depuis ma naissance.
\par 19 Puis elle prit, mangea et but devant lui ce que sa servante avait préparé.
\par 20 Et Holopherne prit beaucoup de plaisir en elle, et but plus de vin qu'il n'en avait jamais bu en un seul jour depuis sa naissance.

\chapitre{13}

\par 1 Le soir étant venu, ses serviteurs se hâtèrent de partir, et Bagoas ferma sa tente dehors, et renvoya les serviteurs de la présence de son maître ; et ils se dirigèrent vers leurs lits, car ils étaient tous fatigués, parce que le festin avait été long.
\par 2 Et Judith resta dans la tente, et Holopherne couché sur son lit, car il était rassasié de vin.
\par 3 Or Judith avait ordonné à sa servante de se tenir hors de sa chambre et de l'attendre. elle sortait, comme elle le faisait quotidiennement : car elle dit qu'elle sortirait pour ses prières, et elle parla à Bagoas dans le même but.
\par 4 Ainsi tous sortirent et il ne resta personne dans la chambre à coucher, ni petit ni grand. Alors Judith, debout près de son lit, dit en son cœur : Seigneur Dieu de toute puissance, regarde ce présent sur les œuvres de mes mains pour l'exaltation de Jérusalem.
\par 5 Car le moment est venu de sauver ton héritage et d'exécuter tes entreprises pour la destruction des ennemis qui se sont levés contre nous.
\par 6 Puis elle s'approcha du pilier du lit qui était à la tête d'Holoferne, et en descendit son fauchion,
\par 7 Et il s'approcha de son lit, et saisit les cheveux de sa tête, et dit : Fortifie-moi, ô Seigneur Dieu d'Israël, aujourd'hui.
\par 8 Et elle lui frappa deux fois le cou de toutes ses forces, et elle lui ôta la tête.
\par 9 Et il fit tomber son corps du lit, et arracha le dais des piliers ; Et peu de temps après, elle sortit et donna la tête d'Holoferne à sa servante ;
\par 10 Et elle le mit dans son sac de viande ; celui-ci.
\par 11 Alors Judith dit au loin, aux sentinelles qui étaient à la porte : Ouvrez, ouvrez maintenant la porte : Dieu, notre Dieu, est avec nous, pour montrer encore sa puissance à Jérusalem et ses forces contre l'ennemi, comme il l'a fait. je l'ai même fait ce jour-là.
\par 12 Or, lorsque les hommes de sa ville entendirent sa voix, ils descendirent en toute hâte à la porte de leur ville, et ils appelèrent les anciens de la ville.
\par 13 Et alors ils coururent tous ensemble, petits et grands, car cela leur était étrange qu'elle vienne ; alors ils ouvrirent la porte, et les reçurent, et allumèrent un feu pour s'éclairer, et se tinrent autour d'eux.
\par 14 Alors elle leur dit à haute voix : Louez, louez Dieu, louez Dieu, dis-je, car il n'a pas ôté sa miséricorde à la maison d'Israël, mais il a détruit nos ennemis par mes mains cette nuit.
\par 15 Alors elle sortit la tête du sac, la montra et leur dit : voici la tête d'Holoferne, le chef de l'armée d'Assur, et voici le dais où il gisait dans son ivresse ; et l'Éternel l'a frappé par la main d'une femme.
\par 16 Aussi vrai que l'Éternel est vivant, lui qui m'a gardé dans le chemin que j'ai suivi, mon visage l'a séduit jusqu'à sa destruction, et pourtant il n'a pas commis de péché avec moi, pour me souiller et me faire honte.
\par 17 Alors tout le peuple fut merveilleusement étonné, et s'inclina et adora Dieu, et dit d'un commun accord : Béni sois-tu, ô notre Dieu, qui as aujourd'hui réduit à néant les ennemis de ton peuple.
\par 18 Alors Ozias lui dit : Ô ma fille, tu es bénie du Dieu Très-Haut entre toutes les femmes de la terre ; et béni soit le Seigneur Dieu, qui a créé les cieux et la terre, qui t'a ordonné de couper la tête du chef de nos ennemis.
\par 19 C'est pourquoi ta confiance ne s'éloignera pas du cœur des hommes, qui se souviennent à jamais de la puissance de Dieu.
\par 20 Et Dieu tourne ces choses vers toi pour une louange perpétuelle, pour te visiter en bonnes choses, parce que tu n'as pas épargné ta vie pour l'affliction de notre nation, mais que tu as vengé notre ruine, en marchant droit devant notre Dieu. Et tout le monde disait : Ainsi soit-il, ainsi soit-il.

\chapitre{14}

\par 1 Alors Judith leur dit : Écoutez-moi maintenant, mes frères, et prenez cette tête, et accrochez-la au plus haut point de vos murs.
\par 2 Et dès que le matin paraîtra et que le soleil apparaîtra sur la terre, prenez chacun ses armes, et sortez de la ville tout homme vaillant, et établissez-vous un capitaine sur eux, comme si vous descendriez dans les champs à la garde des Assyriens ; mais ne descends pas.
\par 3 Alors ils prendront leurs armures, et entreront dans leur camp, et lèveront les capitaines de l'armée d'Assur, et ils courront à la tente d'Holoferne, mais ils ne le trouveront pas : alors la peur s'abattra sur eux, et ils fuiront devant toi.
\par 4 Ainsi, vous et tous ceux qui habitent la côte d'Israël, vous les poursuivrez et vous les renverserez sur leur passage.
\par 5 Mais avant de faire ces choses, appelez-moi Achior l'Ammonite, afin qu'il voie et connaisse celui qui a méprisé la maison d'Israël et qui l'a envoyé vers nous pour ainsi dire jusqu'à sa mort.
\par 6 Alors ils appelèrent Achior hors de la maison d'Ozias ; Et quand il fut arrivé, et qu'il vit la tête d'Holoferne dans la main d'un homme dans l'assemblée du peuple, il tomba la face contre terre, et son esprit s'effondra.
\par 7 Mais quand ils l'eurent récupéré, il tomba aux pieds de Judith, et la révéra, et dit : Tu es bénie dans tous les tabernacles de Juda et dans toutes les nations qui, entendant ton nom, seront étonnées.
\par 8 Maintenant donc, raconte-moi tout ce que tu as fait ces jours-ci. Alors Judith lui raconta au milieu du peuple tout ce qu'elle avait fait, depuis le jour où elle était sortie jusqu'à l'heure où elle leur avait parlé.
\par 9 Et quand elle eut fini de parler, le peuple cria à haute voix et fit un bruit joyeux dans leur ville.
\par 10 Et quand Achior eut vu tout ce que le Dieu d'Israël avait fait, il crut beaucoup en Dieu, et circoncit la chair de son prépuce, et fut attaché à la maison d'Israël jusqu'à ce jour.
\par 11 Et dès que le matin se leva, ils pendirent la tête d'Holoferne au mur, et chacun prit ses armes, et ils sortirent en bandes vers le détroit de la montagne.
\par 12 Mais quand les Assyriens les virent, ils envoyèrent vers leurs chefs, qui vinrent vers leurs capitaines et leurs tribuns, et vers chacun de leurs chefs.
\par 13 Ils arrivèrent donc à la tente d'Holopherne, et dirent à celui qui avait la garde de toutes ses affaires : Réveillez maintenant notre seigneur, car les esclaves ont eu l'audace de descendre contre nous pour nous combattre, afin qu'ils soient entièrement détruits.
\par 14 Puis il entra dans Bagoas et frappa à la porte de la tente ; car il pensait avoir couché avec Judith.
\par 15 Mais comme personne ne répondait, il l'ouvrit et entra dans la chambre à coucher, et le trouva mort par terre, et sa tête lui fut arrachée.
\par 16 C'est pourquoi il cria à haute voix, avec des pleurs, des soupirs et un grand cri, et il déchira ses vêtements.
\par 17 Après être entré dans la tente où logeait Judith, et ne la trouvant pas, il sauta vers le peuple et cria :
\par 18 Ces esclaves ont agi perfidement ; Une femme des Hébreux a fait honte à la maison du roi Nabuchodonosor ; car voici, Holopherne est couché à terre, sans tête.
\par 19 Lorsque les chefs de l'armée assyrienne entendirent ces paroles, ils déchirèrent leurs manteaux et leur esprit fut merveilleusement troublé, et il y eut un cri et un très grand bruit dans tout le camp.

\chapitre{15}

\par 1 Et quand ceux qui étaient dans les tentes entendirent, ils furent étonnés de ce qui se passait.
\par 2 Et la peur et le tremblement les saisirent, de sorte qu'aucun homme n'osa demeurer à la vue de son prochain, mais se précipitant tous ensemble, ils s'enfuirent dans tous les coins de la plaine et des montagnes.
\par 3 Ceux aussi qui avaient campé dans les montagnes autour de Béthulie s'enfuirent. Alors les enfants d'Israël, tous ceux d'entre eux qui étaient guerriers, se précipitèrent sur eux.
\par 4 Alors Ozias envoya à Betomasthem, et à Bebai, et Chobai, et Cola et dans toutes les côtes d'Israël, pour raconter les choses qui s'étaient faites, et pour que tous se précipitent sur leurs ennemis pour les détruire.
\par 5 Or, lorsque les enfants d'Israël l'apprirent, ils se précipitèrent tous d'un commun accord sur eux et les tuèrent jusqu'à Chobaï ; de même aussi ceux qui venaient de Jérusalem et de toute la région montagneuse, (car des hommes leur avaient dit quelles choses ont été commis dans le camp de leurs ennemis) et ceux qui étaient en Galaad et en Galilée les ont poursuivis avec un grand massacre jusqu'à ce qu'ils aient dépassé Damas et ses frontières.
\par 6 Et le reste qui habitait à Béthulie tomba sur le camp d'Assur, et le pilla, et s'enrichit grandement.
\par 7 Et les enfants d'Israël qui revenaient du massacre avaient ce qui restait ; Les villages et les villes qui étaient dans les montagnes et dans la plaine rapportèrent beaucoup de butin, car la multitude était très grande.
\par 8 Alors Joacim, le grand prêtre, et les anciens des enfants d'Israël qui habitaient à Jérusalem, vinrent voir les bonnes choses que Dieu avait montrées à Israël, et voir Judith, et la saluer.
\par 9 Et lorsqu'ils s'approchèrent d'elle, ils la bénirent d'un commun accord, et lui dirent : Tu es l'exaltation de Jérusalem, tu es la grande gloire d'Israël, tu es la grande réjouissance de notre nation.
\par 10 Tu as fait toutes ces choses par ta main ; tu as fait beaucoup de bien à Israël, et Dieu en est satisfait : tu seras béni par l'Éternel tout-puissant pour toujours. Et tout le monde disait : Qu’il en soit ainsi.
\par 11 Et le peuple pilla le camp pendant trente jours ; et ils donnèrent à Judith Holopherne sa tente, et toute sa vaisselle, et ses lits, et ses ustensiles, et toutes ses affaires ; et elle la prit et la posa sur sa mule. ; elle prépara ses charrettes et les y déposa.
\par 12 Alors toutes les femmes d'Israël accoururent pour la voir, la bénirent et firent danser parmi elles pour elle. Elle prit des branches dans sa main et les donna aussi aux femmes qui étaient avec elle.
\par 13 Et ils mirent une guirlande d'olivier sur elle et sur sa servante qui était avec elle, et elle marcha devant tout le peuple dans la danse, conduisant toutes les femmes. Et tous les hommes d'Israël la suivirent dans leurs armures avec des guirlandes, et avec des chansons dans la bouche.

\chapitre{16}

\par 1 Alors Judith se mit à chanter cette action de grâces dans tout Israël, et tout le peuple chanta après elle ce chant de louange.
\par 2 Et Judith dit : Commencez à mon Dieu avec des tambourins, chantez à mon Seigneur avec des cymbales : accordez-lui un nouveau psaume ; exaltez-le et invoquez son nom.
\par 3 Car Dieu brise les combats ; car au milieu des camps, au milieu du peuple, il m'a délivré des mains de ceux qui me persécutaient.
\par 4 Assur est sorti des montagnes du nord, il est venu avec dix mille hommes de son armée, dont la multitude a arrêté les torrents, et leurs cavaliers ont couvert les collines.
\par 5 Il se vantait de brûler mes frontières, de tuer mes jeunes gens avec l'épée, et de briser les enfants qui allaitaient contre terre, et de faire de mes enfants une proie, et de mes vierges un butin.
\par 6 Mais le Seigneur Tout-Puissant les a déçus par la main d'une femme.
\par 7 Car le puissant n'est pas tombé devant les jeunes gens, ni les fils des Titans ne l'ont frappé, ni les grands géants ne se sont attaqués à lui; mais Judith, fille de Merari, l'a affaibli par la beauté de son visage.
\par 8 Car elle ôta le vêtement de son veuvage, pour l'exaltation de ceux qui étaient opprimés en Israël, et oignit son visage d'onguent, et attacha ses cheveux avec un pneu, et prit un vêtement de lin pour le tromper.
\par 9 Ses sandales ravissaient ses yeux, sa beauté lui rendait l'esprit prisonnier, et le fauchion lui traversait le cou.
\par 10 Les Perses furent ébranlés par sa hardiesse, et les Mèdes furent intimidés par sa hardiesse.
\par 11 Alors mes affligés poussèrent des cris de joie, et mes faibles crièrent à haute voix ; mais ils furent étonnés : ceux-ci élevèrent la voix, mais ils furent renversés.
\par 12 Les fils des jeunes filles les ont transpercés et blessés comme des enfants de fuyards : ils ont péri dans la bataille de l'Éternel.
\par 13 Je chanterai au Seigneur un chant nouveau : Seigneur, tu es grand et glorieux, merveilleux en force et invincible.
\par 14 Que toutes les créatures te servent : car tu as parlé, et elles ont été créées, tu as envoyé ton esprit, et il les a créées, et il n'y en a personne qui puisse résister à ta voix.
\par 15 Car les montagnes seront ébranlées par les eaux, les rochers fondront comme de la cire devant toi ; mais tu es miséricordieux envers ceux qui te craignent.
\par 16 Car tout sacrifice est trop peu pour toi d'agréable odeur, et toute la graisse ne suffit pas pour ton holocauste; mais celui qui craint l'Éternel est grand en tout temps.
\par 17 Malheur aux nations qui se soulèvent contre mes frères ! le Seigneur Tout-Puissant se vengera d'eux au jour du jugement, en mettant du feu et des vers dans leur chair ; et ils les sentiront et pleureront pour toujours.
\par 18 Dès qu'ils furent entrés à Jérusalem, ils adorèrent l'Éternel ; et aussitôt que le peuple fut purifié, il offrit ses holocaustes, ses offrandes gratuites et ses présents.
\par 19 Judith consacra également tous les objets d'Holopherne que le peuple lui avait donnés, et donna le dais qu'elle avait retiré de sa chambre à coucher, en cadeau au Seigneur.
\par 20 Le peuple continua donc à faire la fête à Jérusalem devant le sanctuaire pendant trois mois et Judith resta avec eux.
\par 21 Après ce temps, chacun retourna dans son héritage, et Judith partit pour Béthulie, et resta dans sa propre possession, et fut en son temps honorable dans tout le pays.
\par 22 Et beaucoup la désiraient, mais personne ne la connut tous les jours de sa vie, après quoi Manassé, son mari, mourut et fut rassemblé auprès de son peuple.
\par 23 Mais elle grandit de plus en plus en honneur, et vieillit dans la maison de son mari, étant âgée de cent cinq ans, et affranchit sa servante ; elle mourut donc à Béthulie, et on l'enterra dans la grotte de son mari Manassé.
\par 24 Et la maison d'Israël la lamenta sept jours; et avant de mourir, elle distribua ses biens à tous ceux qui étaient les plus proches parents de Manassé, son mari, et à ceux qui étaient les plus proches de sa famille.
\par 25 Et il n'y en eut personne qui fit plus peur aux enfants d'Israël du temps de Judith, ni longtemps après sa mort.


\end{document}