\begin{document}

\title{Deuxième livre d'Hénoch}

\chapter{1}

\par \textit{Un récit du mécanisme du monde montrant la machinerie du soleil et de la lune en fonctionnement. Astronomie et calendrier ancien intéressant. Voir également les chapitres 15 à 17. 21. À quoi ressemblait le monde avant la création, voir le chapitre 24. Le chapitre 26 est particulièrement pittoresque. Un récit unique sur la façon dont Satan a été créé (Chapitre 29.)}

\par 1 Il y avait un homme sage, un grand artisan, et le Seigneur conçut de l'amour pour lui et le reçut, afin qu'il contemple les demeures les plus élevées et soit un témoin oculaire du royaume sage, grand, inconcevable et immuable de Dieu Tout-Puissant. , de la station très merveilleuse et glorieuse et brillante et aux yeux multiples des serviteurs du Seigneur, et du trône inaccessible du Seigneur, et des degrés et manifestations des armées incorporelles, et du ministère ineffable de la multitude des éléments , et des diverses apparitions et chants inexprimables de l'armée des Chérubins et de la lumière illimitée.

\par 2 A cette époque, dit-il, lorsque ma 165ème année fut accomplie, j'engendra mon fils Mathusal.

\par 3 Après cela aussi, j'ai vécu deux cents ans et j'ai accompli de toutes les années de ma vie trois cent soixante-cinq ans.

\par 4 Le premier jour du premier mois, j'étais seul dans ma maison, je me reposais sur mon canapé et je dormais.

\par 5 Et quand je dormais, une grande détresse montait dans mon cœur, et je pleurais avec mes yeux dans le sommeil, et je ne pouvais pas comprendre ce qu'était cette détresse, ni ce qui m'arriverait.

\par 6 Et il m'apparut deux hommes, extrêmement grands, de sorte que je n'en ai jamais vu sur terre ; leurs visages brillaient comme le soleil, leurs yeux aussi étaient comme une lumière brûlante, et de leurs lèvres sortait du feu avec des vêtements et des chants de diverses sortes d'apparence pourpre, leurs ailes étaient plus brillantes que l'or, leurs mains plus blanches que la neige.

\par 7 Ils se tenaient à la tête de mon canapé et ont commencé à m'appeler par mon nom.

\par 8 Et je me levai de mon sommeil et vis clairement ces deux hommes qui se tenaient devant moi.

\par 9 Et je les ai salués et j'ai été saisi de peur et l'apparence de mon visage a été changée par la terreur, et ces hommes m'ont dit :

\par 10 « Aie courage, Hénoc, ne crains pas ; le Dieu éternel nous a envoyé vers toi, et voilà ! tu monteras aujourd'hui avec nous au ciel, et tu diras à tes fils et à toute ta maison tout ce qu'ils feront sans toi sur la terre dans ta maison, et que personne ne te cherche jusqu'à ce que le Seigneur te revienne à eux.

\par 11 Je me hâtai de leur obéir, je sortis de ma maison, je me rendis aux portes, comme on me l'avait ordonné, j'appelai mes fils Mathusal, Regim et Gaidad, et je leur fis part de toutes les merveilles que ces hommes m'avaient racontées.

\chapter{2}

\par \textit{L'Instruction. Comment Enoch a instruit ses fils.}

\par 1 Écoutez-moi, mes enfants, je ne sais où je vais, ni ce qui m'arrivera ; maintenant donc, mes enfants, je vous le dis : ne vous détournez pas de Dieu devant les vains, qui n'ont pas fait le ciel et la terre, car ceux-ci périront ainsi que ceux qui les adorent, et que le Seigneur rende vos cœurs confiants dans la crainte de lui. Et maintenant, mes enfants, que personne ne pense à me chercher, jusqu'à ce que le Seigneur me ramène à vous.

\chapter{3}

\par \textit{De l'hypothèse d'Enoch ; comment les anges l'ont emmené au premier ciel.}

\par 1 Il arriva, quand Enoch eut dit à ses fils, que les anges le prirent sur leurs ailes et l'emportèrent jusqu'au premier ciel et le placèrent sur les nuées. Et là, j'ai regardé, et encore une fois, j'ai regardé plus haut, et j'ai vu l'éther, et ils m'ont placé au premier ciel et m'ont montré une très grande mer, plus grande que la mer terrestre.

\chapter{4}

\par \textit{Des anges régnant sur les étoiles.}

\par 1 ILS ont amené devant moi les anciens et les dirigeants des ordres stellaires, et m'ont montré deux cents anges, qui gouvernent les étoiles et leurs services aux cieux, et volent de leurs ailes et entourent tous ceux qui naviguent.

\chapter{5}

\par \textit{De la façon dont les Anges gardent les réserves de neige.}

\par 1 ET ici, j'ai regardé en bas et j'ai vu les trésors de la neige, et les anges qui gardent leurs terribles magasins, et les nuages ​​d'où ils sortent et dans lesquels ils entrent.

\chapter{6}

\par \textit{De la rosée et de l'huile d'olive, et diverses fleurs.}

\par 1 ILS m'ont montré le trésor de la rosée, comme l'huile de l'olive, et l'apparence de sa forme, comme celle de toutes les fleurs de la terre ; de plus, de nombreux anges gardent les trésors de ces choses, et comment ils sont faits pour se fermer et s'ouvrir.

\chapter{7}

\par \textit{De la façon dont Enoch a été emmené au deuxième ciel.}

\par 1 ET ces hommes m'ont pris et m'ont conduit au deuxième ciel, et m'ont montré des ténèbres, plus grandes que les ténèbres terrestres, et là j'ai vu des prisonniers pendus, surveillés, attendant le jugement grand et sans limites, et ces anges étaient sombres- regardant, plus que les ténèbres terrestres, et faisant sans cesse pleurer à toutes heures.

\par 2 Et je dis aux hommes qui étaient avec moi : 'Pourquoi sont-ils sans cesse torturés ?' ils m'ont répondu : « Ce sont des apostats de Dieu, qui n'ont pas obéi aux commandements de Dieu, mais ont pris conseil avec leur propre volonté et se sont détournés avec leur prince, qui est également attaché au cinquième ciel.

\par 3 Et j'eus une grande pitié pour eux, et ils me saluèrent, et me dirent : 'Homme de Dieu, prie pour nous le Seigneur' ; et je leur répondis : « Qui suis-je, un mortel, pour que je prie pour les anges ? qui sait où je vais, et que m'arrivera-t-il ? ou qui priera pour moi ?

\chapter{8}

\par \textit{De l'assomption d'Enoch au troisième ciel.}

\par 1 ET ces hommes m'ont pris de là, et m'ont conduit jusqu'au troisième ciel, et m'y ont placé ; et j'ai regardé en bas, et j'ai examiné les produits de ces lieux, tels qu'on n'en a jamais connu pour leur bonté.

\par 2 Et je vis tous les arbres aux fleurs douces, et je vis leurs fruits, qui étaient odorants, et tous les aliments qu'ils portaient bouillonnaient d'une exhalation odorante.

\par 3 Et au milieu des arbres celui de la vie, dans ce lieu sur lequel repose le Seigneur, lorsqu'il monte au paradis ; et cet arbre est d'une bonté et d'un parfum ineffables, et orné plus que tout ce qui existe ; et de tous côtés il a une forme dorée, vermillon et semblable à du feu, et il couvre tout, et il produit des produits de tous les fruits.

\par 4 Sa racine est dans le jardin au bout de la terre.

\par 5 Et le paradis est entre la corruptibilité et l'incorruptibilité.

\par 6 Et deux sources sortent qui jettent du miel et du lait, et leurs sources envoient de l'huile et du vin, et elles se séparent en quatre parties, et tournent avec un cours tranquille, et descendent dans le PARADIS D'EDEN, entre la corruptibilité et en corruptibilité.

\par 7 Et de là ils s'étendent le long de la terre, et font une révolution vers leur cercle comme les autres éléments.

\par 8 Et ici il n'y a pas d'arbre infructueux, et tout lieu est béni.

\par 9 Et il y a trois cents anges très brillants, qui gardent le jardin, et avec des chants doux et incessants et des voix jamais silencieuses, servent le Seigneur à tous les jours et à toutes les heures.

\par 10 Et je dis : « Comme cet endroit est doux ! » Et ces hommes me dirent :

\chapter{9}

\par \textit{La démonstration à Enoch de la place des justes et des compatissants.}

\par 1 CE lieu, ô Enoch, est préparé pour les justes, qui endurent toutes sortes d'offenses de la part de ceux qui exaspèrent leur âme, qui détournent leurs yeux de l'iniquité, et rendent un jugement juste, et donnent du pain à ceux qui ont faim, et couvrent le nus avec des vêtements, et relever ceux qui sont tombés, et aider les orphelins blessés, et qui marchent sans faute devant la face du Seigneur, et le servent seul, et pour eux est préparé ce lieu pour l'héritage éternel.

\chapter{10}

\par \textit{Ici, ils montrèrent à Enoch l'endroit terrible et diverses tortures.}

\par 1 Ces deux hommes me conduisirent du côté du septentrion, et me montrèrent un lieu terrible, où il y avait toutes sortes de tortures : des ténèbres cruelles et des ténèbres sans lumière ; il n'y a pas de lumière, mais un feu trouble qui flambe constamment en haut, et il y a un fleuve de feu qui sort, et tout ce lieu est partout en feu, et partout il y a du gel et de la glace, de la soif et des frissons, tandis que les liens sont très cruels, et les anges effrayants et impitoyables, portant des armes furieuses, des tortures impitoyables, et j'ai dit :

\par 2 «Malheur, malheur, comme cet endroit est terrible.»

\par 3 Et ces hommes me dirent : Ce lieu, ô Hénoc, est préparé pour ceux qui déshonorent Dieu, qui pratiquent sur terre le péché contre nature, qui est la corruption des enfants à la manière sodomitique, la magie, les enchantements et les sorcelleries diaboliques. , et qui se vantent de leurs mauvaises actions, vols, mensonges, calomnies, envie, rancune, fornication, meurtre, et qui, maudits, volent les âmes des hommes, qui, voyant les pauvres emporter leurs biens et s'enrichir eux-mêmes, leur nuisant pour les biens d'autres hommes; qui, pouvant satisfaire le vide, a fait mourir la faim ; être capable de vêtir, de déshabiller les personnes nues ; et qui n'ont pas connu leur créateur, et se sont prosternés devant des dieux sans âme (sc. sans vie), qui ne peuvent ni voir ni entendre, des dieux vains, qui ont aussi construit des images taillées et se sont inclinés devant des ouvrages impurs, car tout cela est préparé cette place parmi eux, pour un héritage éternel.

\chapter{11}

\par \textit{Ici, ils emmenèrent Enoch au quatrième ciel où se trouve la course du soleil et de la lune.}

\par 1 CES hommes m'ont pris et m'ont conduit jusqu'au quatrième ciel, et m'ont montré tous les déplacements successifs et tous les rayons de la lumière du soleil et de la lune.

\par 2 Et j'ai mesuré leurs déplacements et comparé leur lumière, et j'ai vu que la lumière du soleil est plus grande que celle de la lune.

\par 3 Son cercle et les roues sur lesquelles il va toujours, comme un vent qui passe avec une vitesse très merveilleuse, et jour et nuit il n'a pas de repos. 1

\par 4 Son passage et son retour sont accompagnés de quatre grandes étoiles, et chaque étoile a sous elle mille étoiles, à droite de la roue solaire, et quatre à gauche, chacune ayant sous elle mille étoiles, en tout huit mille. , émettant continuellement avec le soleil.

\par 5 Et le jour, quinze myriades d'anges y assistent, et la nuit mille.

\par 6 Et ceux à six ailes sortent avec les anges devant la roue du soleil dans les flammes ardentes, et cent anges allument le soleil et l'allument.

\par \textit{85:1 Cf. «Transport rapide.»}

\chapter{12}

\par \textit{Des éléments très merveilleux du soleil.}

\par 1 ET j'ai regardé et j'ai vu d'autres éléments volants du soleil, dont les noms sont Phénix et Chalkydri, merveilleux et merveilleux, avec des pieds et des queues en forme de lion, et une tête de crocodile, leur apparence est pourpre, comme l'arc-en-ciel. ; leur taille est de neuf cents mesures, leurs ailes sont comme celles des anges, chacun en a douze, et ils assistent et accompagnent le soleil, portant chaleur et rosée, comme cela leur est ordonné de Dieu.

\par 2 Ainsi le soleil tourne et va, et se lève sous le ciel, et sa course va sous la terre avec la lumière de ses rayons sans cesse.

\chapter{13}

\par \textit{Les anges prirent Enoch et le placèrent à l'est, aux portes du soleil.}

\par 1 CES hommes m'ont emmené à l'orient, et m'ont placé aux portes du soleil, là où le soleil se lève selon la règle des saisons et le circuit des mois de l'année entière, et le nombre des heures du jour et de la nuit,

\par 2 Et j'ai vu six portes ouvertes, chaque porte ayant soixante et un stades et un quart de stade, et je les ai mesurés avec précision, et j'ai compris que leur taille était telle, par laquelle le soleil sort et va vers le à l'ouest, et est égalisé, et s'élève pendant tous les mois, et revient des six portes selon la succession des saisons ; ainsi la période de l'année entière se termine après le retour des quatre saisons,

\chapter{14}

\par \textit{Ils ont emmené Enoch vers l'Ouest.}

\par 1 ET encore ces hommes m'emmenèrent vers les parties occidentales, et me montrèrent six grandes portes ouvertes correspondant aux portes orientales, en face de l'endroit où le soleil se couche, selon le nombre des jours trois cent soixante-cinq et un quart.

\par 2 Ainsi encore, il descend vers les portes occidentales et emporte sa lumière, la grandeur de son éclat, sous la terre ; car puisque la couronne de son éclat est dans le ciel avec le Seigneur et gardée par quatre cents anges, tandis que le soleil tourne sur une roue sous la terre et reste sept grandes heures dans la nuit et passe la moitié de sa course sous la terre, quand il arrive à l'approche orientale à la huitième heure de la nuit, il apporte ses lumières et sa couronne d'éclat, et le soleil brille plus que le feu.

\chapter{15}

\par \textit{Les éléments Soleil, Phénix et Chalkydri se mirent à chanter.}

\par 1 ALORS les éléments du soleil, appelés Phénix et Chalkydri, se mettent à chanter, c'est pourquoi chaque oiseau bat des ailes, se réjouissant du donneur de lumière, et ils se mettent à chanter sur l'ordre du Seigneur.

\par 2 Le donneur de lumière vient donner de la clarté au monde entier, et la garde du matin prend forme, ce sont les rayons du soleil, et le soleil de la terre sort, et reçoit sa clarté pour éclairer toute la surface de la terre, et ils m'ont montré ce calcul de la marche du soleil.

\par 3 Et les portes par lesquelles il entre, ce sont les grandes portes du calcul des heures de l'année ; c'est pourquoi le soleil est une grande création, dont le circuit dure vingt-huit ans et recommence depuis le commencement.

\chapter{16}

\par \textit{Ils prirent Enoch et le placèrent de nouveau à l'est au cours de la lune.}

\par 1 CES hommes m'ont montré l'autre cours, celui de la lune, douze grandes portes, couronnées d'ouest en est, par lesquelles la lune entre et sort des temps coutumiers.

\par 2 Il entre par la première porte vers les lieux occidentaux du soleil, par la première porte avec trente et un jours exactement, par la deuxième porte avec trente et un jours exactement, par la troisième avec trente jours exactement, par la le quatrième avec trente jours exactement, le cinquième avec trente et un jours exactement, le sixième avec trente et un jours exactement, le septième avec trente jours exactement, le huitième avec trente et un jours parfaitement, le neuvième avec trente- un jour exactement, le dixième avec trente jours exactement, le onzième avec trente et un jours exactement, le douzième avec vingt-huit jours exactement.

\par 3 Et il passe par les portes occidentales dans l'ordre et le nombre de celles de l'Orient, et accomplit les trois cent soixante-cinq jours et quart de l'année solaire, tandis que l'année lunaire en compte trois cent cinquante-quatre, et il lui manque douze jours du cercle solaire, qui sont les événements lunaires de toute l'année.

\par 4 (Ainsi aussi, le grand cercle contient cinq cent trente-deux ans.)

\par 5 Le quart de jour est omis pendant trois ans, le quatrième l'accomplit exactement.

\par 6 C'est pourquoi ils sont retirés du ciel pendant trois ans et ne sont pas ajoutés au nombre des jours, parce qu'ils changent le temps des années en deux nouveaux mois vers l'achèvement, en deux autres vers la diminution.

\par 7 Et quand les portes occidentales sont terminées, il revient et va à l'est vers les lumières, et il va ainsi jour et nuit autour des cercles célestes, plus bas que tous les cercles, plus rapide que les vents célestes, et les esprits et les éléments et les anges qui volent ; chaque ange a six ailes.

\par 8 Il a un cours septuple en dix-neuf ans.

\chapter{17}

\par \textit{Des chants des anges, qu'il est impossible de décrire.}

\par 1 AU milieu des cieux j'ai vu des soldats armés, servant le Seigneur, avec des tympans et des orgues, à voix incessante, à voix douce, à voix douce et incessante et des chants divers, qu'il est impossible de décrire, et qui étonne chaque esprit, tant le chant de ces anges est merveilleux et merveilleux, et j'étais ravi de l'écouter.

\chapter{18}

\par \textit{De la montée d'Enoch au cinquième ciel.}

\par 1 LES hommes m'emmenèrent au cinquième ciel et me placèrent, et là je vis de nombreux et innombrables soldats, appelés Grigori, d'apparence humaine, et leur taille était plus grande que celle des grands géants et leurs visages flétris, et le silence de leur bouche perpétuelle, et il n'y avait pas de service au cinquième ciel, et j'ai dit aux hommes qui étaient avec moi :

\par 2 Pourquoi ceux-ci sont-ils très flétris et leurs visages mélancoliques, et leurs bouches silencieuses, et pourquoi n'y a-t-il aucun service sur ce ciel ?

\par 3 Et ils me dirent : "Ce sont les Grigori qui, avec leur prince Satanail, ont rejeté le Seigneur de la lumière : Voici les Grigori qui, avec leur prince Satanail, ont rejeté le Seigneur de la lumière, et après eux ceux qui sont retenus dans de grandes ténèbres au second ciel, et trois d'entre eux sont descendus sur la terre depuis le trône du Seigneur, à l'endroit de l'Ermon, ils ont rompu leurs voeux sur l'épaule de la colline d'Ermon, ils ont vu les filles des hommes comme elles sont bonnes, ils ont pris des femmes, et ils ont souillé la terre par leurs oeuvres, eux qui, à toutes les époques de leur âge, ont fait l'anarchie et le mélange, et des géants sont nés, et des hommes d'une taille merveilleuse, et une grande inimitié.

\par 4 Et c'est pourquoi Dieu les a jugés avec un grand jugement, et ils pleurent sur leurs frères et ils seront punis au grand jour du Seigneur.

\par 5 Et je dis aux Grigori : 'J'ai vu vos frères et leurs œuvres, et leurs grands tourments, et j'ai prié pour eux, mais le Seigneur les a condamnés à être sous la terre jusqu'à la fin du ciel et de la terre pour toujours.'

\par 6 Et je dis : « Pourquoi attendez-vous, frères, et ne servez-vous pas devant la face du Seigneur, et n'avez-vous pas mis vos services devant la face du Seigneur, de peur que vous n'irritiez complètement votre Seigneur ?

\par 7 Et ils écoutèrent mon avertissement, et parlèrent aux quatre rangs du ciel, et voici ! tandis que je me tenais avec ces deux hommes, quatre trompettes sonnèrent ensemble à grande voix, et les Grigori se mirent à chanter d'une seule voix, et leur voix s'éleva devant le Seigneur d'une manière pitoyable et touchante.

\par \textit{87:1 Comparez le deuxième livre d'Adam et Ève. Type. XX.}}

\chapter{19}

\par \textit{De la montée d'Enoch au sixième ciel.}

\par 1 ET de là ces hommes m'ont pris et m'ont porté jusqu'au sixième ciel, et là j'ai vu sept bandes d'anges, très brillantes et très glorieuses, et leurs visages brillaient plus que celui du soleil, luisant, et il n'y a aucun différence dans leurs visages, ou leur comportement, ou leur manière de s'habiller ; et ceux-ci donnent les ordres et apprennent le mouvement des étoiles, l'altération de la lune, ou la révolution du soleil, et le bon gouvernement du monde.

\par 2 Et quand ils voient une mauvaise action, ils font des commandements et des instructions, et des chants doux et forts, et tous les chants de louange.

\par 3 Ce sont les archanges qui sont au-dessus des anges, qui mesurent toute vie dans le ciel et sur la terre, et les anges qui sont assignés aux saisons et aux années, les anges qui sont au-dessus des fleuves et de la mer, et qui sont au-dessus des fruits de la terre. , et les anges qui sont sur toute herbe, donnant de la nourriture à tous, à tout être vivant, et les anges qui écrivent toutes les âmes des hommes, et toutes leurs actions, et leur vie devant la face du Seigneur ; au milieu d'eux se trouvent six Phénix, six Chérubins et six à six ailes, continuellement d'une seule voix chantant d'une seule voix, et il n'est pas possible de décrire leur chant, et ils se réjouissent devant le Seigneur sur son marchepied.

\chapter{20}

\par \textit{C'est pourquoi ils emmenèrent Enoch au Septième Ciel.}

\par 1 ET ces deux hommes m'ont élevé de là jusqu'au septième Ciel, et j'ai vu là une très grande lumière et des troupes enflammées de grands archanges, des forces incorporelles et des dominations, des ordres et des gouvernements, des chérubins et des séraphins, des trônes et de nombreux -yeux, neuf régiments, les stations de lumière Ioanit, et j'ai eu peur et j'ai commencé à trembler d'une grande terreur, et ces hommes m'ont pris et m'ont conduit après eux et m'ont dit :

\par 2 'Aie courage, Hénoc, ne crains pas', et me montra de loin le Seigneur, assis sur son très haut trône. Car qu’y a-t-il au dixième ciel, puisque le Seigneur habite ici ?

\par 3 Au dixième ciel est Dieu, en langue hébraïque il est appelé Aravat.

\par 4 Et toutes les troupes célestes venaient se tenir sur les dix marches selon leur rang, et s'inclinaient devant le Seigneur, et retournaient à leur place dans la joie et la félicité, chantant des chants dans la lumière sans limites avec de petits et des voix tendres, le servant glorieusement.

\chapter{21}

\par \textit{De la façon dont les anges ici ont quitté Enoch, à la fin du septième ciel, et se sont éloignés de lui sans être vus.}

\par 1 ET les chérubins et les séraphins debout autour du trône, ceux à six ailes et aux yeux multiples ne s'éloignent pas, se tenant devant la face du Seigneur faisant sa volonté, et couvrent tout son trône, chantant d'une voix douce devant la face du Seigneur : «Saint, saint, saint, Seigneur Souverain de Sabaoth, les cieux et la terre sont pleins de ta gloire.»

\par 2 Quand j'ai vu toutes ces choses, ces hommes m'ont dit : « Hénoc, jusqu'ici il nous a été ordonné de marcher avec toi », et ces hommes s'éloignèrent de moi et là-dessus je ne les vis pas.

\par 3 Et je restai seul au bout du septième ciel et j'eus peur, je tombai la face contre terre et je me dis : « Malheur à moi, que m'est-il arrivé ? »

\par 4 Et le Seigneur envoya un de ses glorieux, l'archange Gabriel, et il me dit : 'Aie courage, Hénoc, ne crains pas, lève-toi devant la face du Seigneur dans l'éternité, lève-toi, viens avec moi.'

\par 5 Et je lui répondis et dis en moi-même : « Mon Seigneur, mon âme s'est éloignée de moi, de terreur et de tremblement », et j'ai appelé les hommes qui m'ont conduit jusqu'à cet endroit, sur eux je me suis appuyé, et c'est avec eux que je vais devant la face du Seigneur.

\par 6 Et Gabriel me saisit, comme une feuille emportée par le vent, et me plaça devant la face du Seigneur.

\par 7 Et j'ai vu le huitième Ciel, qui est appelé en langue hébraïque Muzaloth, changeur des saisons, de la sécheresse et de l'humidité, et des douze signes du zodiaque, qui sont au-dessus du septième Ciel.

\par 8 Et j'ai vu le neuvième Ciel, qui est appelé en hébreu Kuchavim, où sont les demeures célestes des douze signes du zodiaque.

\chapter{22}

\par \textit{Au dixième Ciel, l'archange Michel conduisit Enoch devant la face du Seigneur.}

\par 1 SUR le dixième Ciel, Aravoth, j'ai vu l'apparition du visage du Seigneur, comme du fer fait pour briller dans le feu, et sorti, émettant des étincelles, et il brûle.

\par 2 Ainsi j'ai vu le visage du Seigneur, mais le visage du Seigneur est ineffable, merveilleux et très affreux, et très, très terrible.

\par 3 Et qui suis-je pour parler de l'être ineffable du Seigneur et de son visage très merveilleux ? Et je ne peux pas dire la quantité de ses nombreuses instructions et de ses diverses voix, le trône du Seigneur très grand et non fait de mains, ni la quantité de ceux qui se tenaient autour de lui, les troupes de chérubins et de séraphins, ni leurs chants incessants, ni sa beauté immuable. , et qui racontera la grandeur ineffable de sa gloire ?

\par 4 Et je tombai à terre et je m'inclinai devant l'Éternel, et l'Éternel de ses lèvres me dit :

\par 5 'Aie courage, Enoch, ne crains pas, lève-toi et tiens-toi devant ma face pour l'éternité.'

\par 6 Et l'archistratège Michel m'a élevé et m'a conduit devant la face du Seigneur.

\par 7 Et le Seigneur dit à ses serviteurs pour les tenter : « Laissez Enoch se tenir devant ma face pour l'éternité », et les glorieux se prosternèrent devant le Seigneur et dirent : « Laissez Enoch partir selon ta parole.

\par 8 Et le Seigneur dit à Michel : 'Va et retire Hénoc de ses vêtements terrestres, et oins-le de mon doux onguent, et mets-le dans les vêtements de ma gloire.'

\par 9 Et Michel fit ainsi, comme le Seigneur lui avait dit. Il m'a oint et m'a habillé, et l'apparence de cet onguent est plus que la grande lumière, et son onguent est comme une douce rosée, et son odeur est douce, brillante comme le rayon du soleil, et je me suis regardé, et j'étais comme un de ses glorieux.

\par 10 Et le Seigneur appela un de ses archanges nommé Pravuil, dont la connaissance était plus rapide en sagesse que les autres archanges, qui écrivaient toutes les actions du Seigneur ; et le Seigneur dit à Pravuil :

\par 11 'Sortez les livres de mes magasins et un roseau à écriture rapide, et donnez-le à Enoch, et remettez-lui les livres de choix et de réconfort de votre main.'



\chapter{23}

\par \textit{Des écrits d'Enoch, comment il a écrit ses merveilleux voyages et ses apparitions célestes et comment il a lui-même écrit trois cent soixante-six livres.}

\par 1 ET il me racontait toutes les œuvres du ciel, de la terre et de la mer, et de tous les éléments, leurs passages et allées, et les tonnerres des tonnerres, du soleil et de la lune, les allées et venues des étoiles, les saisons , les années, les jours et les heures, les levers du vent, le nombre des anges et la formation de leurs chants, et toutes les choses humaines, la langue de chaque chant et vie humaine, les commandements, les instructions et les voix douces les chants et tout ce qu'il convient d'apprendre.

\par 2 Et Pravuil me dit : 'Toutes les choses que je t'ai dites, nous les avons écrites. Asseyez-vous et écrivez toutes les âmes de l’humanité, quel que soit le nombre d’entre elles nées, et les lieux qui leur sont préparés pour l’éternité ; car toutes les âmes sont préparées pour l'éternité, avant la formation du monde.

\par 3 Et le tout double trente jours et trente nuits, et j'ai tout écrit exactement, et j'ai écrit trois cent soixante-six livres.

\chapter{24}

\par \textit{Des grands secrets de Dieu, que Dieu révéla et dit à Enoch, et lui parla face à face.}

\par 1 ET le Seigneur m'appela, et me dit : 'Enoch, assieds-toi à ma gauche avec Gabriel.'

\par 2 Et je me suis prosterné devant le Seigneur, et le Seigneur m'a parlé : Hénoc, bien-aimé, tout ce que tu vois, tout ce qui est debout, je te le dis avant même le commencement, tout ce que j'ai créé à partir du non-être. , et les choses visibles d'invisibles.

\par 3 Écoute, Énoch, et reçois mes paroles, car je n'ai pas révélé mon secret à mes anges, je ne leur ai pas parlé de leur ascension ni de mon royaume sans fin, et ils n'ont pas compris ma création, que je te dis aujourd'hui.

\par 4 Car avant que toutes choses fussent visibles, moi seul circulais dans les choses invisibles, comme le soleil de l'est à l'ouest et de l'ouest à l'est.

\par 5 Mais même le soleil a la paix en lui-même, tandis que moi, je n'ai trouvé aucune paix, parce que je créais toutes choses, et j'ai conçu l'idée de poser des fondations et de créer une création visible.

\chapter{25}

\par \textit{Dieu se rapporte à Enoch, comment des plus basses ténèbres descendent le visible et l'invisible.}

\par 1 J'ai commandé dans les parties les plus basses, que les choses visibles descendent de l'invisible, et Adoil est descendu très grand, et je l'ai vu, et voilà ! il avait un ventre d'une grande lumière.

\par 2 Et je lui dis : « Défait-toi, Adoil, et que le visible sorte de toi. »

\par 3 Et il se défait, et une grande lumière sortit. Et j'étais au milieu de la grande lumière, et comme la lumière naît de la lumière, un grand âge survint et montra toute la création que j'avais pensé créer.

\par 4 Et j'ai vu que c'était bien.

\par 5 Et je me suis placé un trône, et je m'y suis assis, et j'ai dit à la lumière : « Monte plus haut et place-toi bien au-dessus du trône, et sois un fondement pour les choses les plus élevées. »

\par 6 Et au-dessus de la lumière, il n'y a rien d'autre, et alors je me suis penché et j'ai levé les yeux de mon trône.

\chapter{26}

\par \textit{Dieu appelle du plus bas une seconde fois qu'Archas, lourd et très rouge, sorte.}

\par 1 ET J'ai convoqué le plus bas une seconde fois, et j'ai dit : 'Laisse Archas sortir avec force,' et il est sorti avec force de l'invisible.

\par 2 Et Archas sortit, dur, lourd et très rouge.

\par 3 Et j'ai dit : 'Ouvre-toi, Archas, et qu'il naisse de toi', et il s'est défait, un âge est apparu, très grand et très obscur, portant la création de toutes les choses inférieures, et j'ai vu que c'était bon et lui dit :

\par 4 'Descends en bas, et affermis-toi, et sois pour un fondement pour les choses inférieures', et cela arriva et il descendit et se fixa, et devint le fondement pour les choses inférieures, et en-dessous les ténèbres là n'est rien d'autre.



\chapter{27}

\par \textit{De la façon dont Dieu fonda l'eau, l'entoura de lumière et y établit sept îles.}

\par 1 AND I commanded that there should be taken from light and darkness, and I said: ‘Be thick,’ and it became thus and I spread it out with the light, and it became water, and I spread it out over the darkness, below the light, and then I made firm the waters, that is to say the bottomless, and I made foundation of light around the water, and created seven circles from inside, and imaged it (sc. the water) like crystal wet and dry, that is to say like glass, and the circumcession of the waters and the other elements, and I showed each one of them its road, and the seven stars each one of them in its heaven, that they go thus, and I saw that it was good.

\par 2 Je séparai la lumière et les ténèbres, c'est-à-dire au milieu de l'eau, de part et d'autre, et je dis à la lumière qu'il ferait jour, et aux ténèbres qu'il ferait nuit ; et il y eut un soir et il y eut un matin, le premier jour.

\chapter{28}

\par \textit{La semaine au cours de laquelle Dieu a montré à Enoch toute sa sagesse et sa puissance, tout au long des sept jours, comment il a créé toutes les forces célestes et terrestres et toutes les choses motrices jusqu'à l'homme.}

\par 1 ET puis j'ai consolidé le cercle céleste, et j'ai fait que les eaux inférieures qui sont sous le ciel se rassemblent en un tout, et que le chaos devienne sec, et il le devint.

\par 2 Des vagues j'ai créé un rocher dur et gros, et du rocher j'ai entassé le sec, et le sec j'ai appelé terre, et le milieu de la terre j'ai appelé abîme, c'est-à-dire ce qui est sans fond, j'ai rassemblé la mer en un seul endroit et il la lia avec un joug.

\par 3 Et je dis à la mer : 'Voici, je te donne tes limites éternelles, et tu ne te détacheras pas de tes parties composantes.'

\par 4 Ainsi j'ai consolidé le firmament. Ce jour-là, je m'appelais le premier créé.

\chapter{29}

\par \textit{Puis ce fut le soir, puis de nouveau le matin, et ce fut le deuxième jour. (Lundi est le premier jour.) L'Essence ardente.}

\par 1 ET pour toutes les troupes célestes j'ai imaginé l'image et l'essence du feu, et mon œil a regardé le rocher très dur et ferme, et de la lueur de mon œil l'éclair a reçu sa nature merveilleuse, qui est à la fois le feu dans l'eau et l'eau dans le feu, et l'un n'éteint pas l'autre, et l'un n'assèche pas l'autre, c'est pourquoi l'éclair est plus brillant que le soleil, plus doux que l'eau et plus ferme que le rocher dur.

\par 2 Du rocher je fis jaillir un grand feu, et du feu je créai les ordres des incorporels, dix troupes d'anges, et leurs armes sont ardentes et leur vêtement une flamme ardente, et j'ordonnai que chacun se tînt dans son ordre. C'est ici que Satanail et ses anges furent précipités du haut de la montagne.

\par 3 Et quelqu'un de l'ordre des anges, s'étant détourné de l'ordre qui était sous lui, conçut une pensée impossible, placer son trône plus haut que les nuages ​​​​au-dessus de la terre, afin de devenir égal en rang à ma puissance. .

\par 4 Et je l'ai jeté du haut des hauteurs avec ses anges, et il volait continuellement dans les airs au-dessus de l'infini.

\chapter{30}

\par \textit{Et puis j'ai créé tous les cieux, et le troisième jour fut (mardi.)}

\par 1 Le troisième jour, j'ai ordonné à la terre de faire pousser de grands arbres féconds, des collines et des graines à semer, et j'ai planté le paradis, je l'ai clôturé, et j'ai placé comme gardiens armés des anges flamboyants, et ainsi j'ai créé le renouveau.

\par 2 Puis vint le soir, et vint le matin du quatrième jour.

\par 3 (mercredi). Le quatrième jour, j'ordonnai qu'il y ait de grandes lumières sur les cercles célestes.

\par 4 Sur le premier cercle supérieur, j'ai placé les étoiles, Kruno, et sur le deuxième Aphrodit, sur le troisième Aris, sur le cinquième Zeus, sur le sixième Ermis, sur le septième plus petit la lune, et je l'ai orné des petites étoiles. .

\par 5 Et en bas, j'ai placé le soleil pour l'éclairage du jour, et la lune et les étoiles pour l'éclairage de la nuit.

\par 6 Le soleil devait aller selon chaque animal (sc. signes du zodiaque), douze, et j'ai fixé la succession des mois et leurs noms et vies, leurs tonnerres et leurs heures, comment ils devraient réussir.

\par 7 Puis le soir vint et le matin vint le cinquième jour.

\par 8 (jeudi). Le cinquième jour, j'ai ordonné à la mer de produire des poissons, des oiseaux à plumes de toutes sortes, et tous les animaux rampant sur la terre, s'avançant sur la terre sur quatre pattes et planant dans les airs, sexes mâle et femelle. , et chaque âme respire l'esprit de vie.

\par 9 Et le soir vint, et le matin vint le sixième jour.

\par 10 (vendredi). Le sixième jour, j'ai ordonné à ma sagesse de créer l'homme à partir de sept consistances : l'une, sa chair tirée de la terre ; deuxièmement, son sang provenant de la rosée ; troisièmement, ses yeux du soleil ; quatre, ses os en pierre ; cinquièmement, son intelligence issue de la rapidité des anges et des nuages ; six, ses veines et ses cheveux de l'herbe de la terre ; sept, son âme de mon souffle et du vent.

\par 11 Et je lui ai donné sept natures : à la chair ouïe, aux yeux pour la vue, à l'âme pour l'odorat, aux veines pour le toucher, au sang pour le goût, aux os pour l'endurance, à l'intelligence pour la douceur (sc. jouissance).

\par 12 J'ai conçu un dicton astucieux pour dire : J'ai créé l'homme à partir de la nature invisible et de la nature visible, les deux sont sa mort, sa vie et son image, il connaît la parole comme une chose créée, petite en grandeur et encore grande en petitesse, et je je l'ai placé sur la terre, un deuxième ange, honorable, grand et glorieux, et je l'ai établi comme dirigeant pour régner sur la terre et avoir ma sagesse, et il n'y avait aucun semblable à lui sur la terre parmi toutes mes créatures existantes.

\par 13 Et je lui ai donné un nom, à partir des quatre parties composantes, de l'est, de l'ouest, du sud, du nord, et je lui ai donné quatre étoiles spéciales, et j'ai appelé son nom Adam, et je lui ai montré les deux chemins. , la lumière et les ténèbres, et je lui ai dit :

\par 14 « Ceci est bien et cela est mauvais », afin que je sache s'il a de l'amour ou de la haine envers moi, afin qu'il soit clair lequel de sa race m'aime.

\par 15 Car j'ai vu sa nature, mais il n'a pas vu sa propre nature, donc s'il ne voit pas, il péchera encore plus, et j'ai dit : « Après le péché, qu'y a-t-il sinon la mort ?

\par 16 Et je l'ai endormi et il s'est endormi. Et je lui ai pris une côte, et je lui ai créé une femme, pour que la mort lui vienne par sa femme, et j'ai pris son dernier mot et je l'ai appelée mère, c'est-à-dire Eva.

\chapter{31}

\par \textit{Dieu donne le paradis à Adam et lui donne l'ordre de voir les cieux ouverts et de voir les anges chanter le chant de la victoire.}

\par 1 ADAM a la vie sur terre, et j'ai créé un jardin en Eden à l'est, afin qu'il observe le testament et garde le commandement.

\par 2 Je lui ai ouvert les cieux, afin qu'il voie les anges chanter le chant de la victoire et la lumière sans ténèbres.

\par 3 Et il était continuellement au paradis, et le diable a compris que je voulais créer un autre monde, parce qu'Adam était seigneur sur terre, pour le gouverner et le contrôler.

\par 4 Le diable est l'esprit mauvais des lieux inférieurs, en tant que fugitif il a fait Sotona des cieux comme son nom était Satanail, ainsi il est devenu différent des anges, mais sa nature n'a pas changé son intelligence en ce qui concerne sa compréhension des choses justes et des choses pécheresses.

\par 5 Et il comprit sa condamnation et le péché qu'il avait commis auparavant, c'est pourquoi il conçut une pensée contre Adam, sous une telle forme qu'il entra et séduisit Eva, mais ne toucha pas Adam.

\par 6 Mais j'ai maudis l'ignorance, mais ce que j'avais béni auparavant, ceux que je n'ai pas maudis, je n'ai maudis ni l'homme, ni la terre, ni les autres créatures, mais le mauvais fruit de l'homme et ses œuvres.

\chapter{32}

\par \textit{Après le péché d'Adam, Dieu le renvoie sur la terre «d'où je t'ai pris», mais ne veut pas le ruiner pour toutes les années à venir.}

\par 1 Je lui dis : Tu es terre, et tu iras dans la terre d'où je t'ai pris, et je ne te ruinerai pas, mais je t'enverrai d'où je t'ai pris.

\par 2 Alors Je pourrai à nouveau te prendre lors de Ma seconde venue !

\par 3 Et j'ai béni toutes mes créatures visibles et invisibles. Et Adam a passé cinq heures et demie au paradis.

\par 4 Et j'ai béni le septième jour, qui est le sabbat, pendant lequel il se reposait de toutes ses œuvres.

\chapter{33}

\par \textit{Dieu montre à Hénoc l'âge de ce monde, son existence de sept mille ans, et le huitième mille est la fin, ni années, ni mois, ni semaines, ni jours.}

\par 1 ET J'ai fixé le huitième jour aussi, que le huitième jour soit le premier créé après mon travail, et que les sept premiers tournent sous la forme du septième mille, et qu'au début du huitième mille il y ait soit un temps sans compter, sans fin, sans années ni mois ni semaines ni jours ni heures.

\par 2 Et maintenant, Hénoc, tout ce que je t'ai dit, tout ce que tu as compris, tout ce que tu as vu des choses célestes, tout ce que tu as vu sur la terre, et tout ce que j'ai écrit dans les livres par ma grande sagesse. , toutes ces choses, j'ai conçu et créé depuis le fondement le plus élevé jusqu'au fondement le plus bas et jusqu'à la fin, et il n'y a ni conseiller ni héritier pour mes créations.

\par 3 Je suis moi-même éternel, non fait de mains et sans changement.

\par 4 Ma pensée est ma conseillère, ma sagesse et ma parole sont faites, et mes yeux observent toutes choses comme elles se tiennent ici et tremblent de terreur.

\par 5 Si je détourne mon visage, alors toutes choses seront détruites.

\par 6 Et applique ton esprit, Hénoc, et connais celui qui te parle, et prends les livres que tu as toi-même écrit.

\par 7 And I give thee Samuil and Raguil, who led thee up, and the books, and go down to earth, and tell thy sons all that I have told thee, and all that thou hast seen, from the lower heaven up to my throne, and all the troops.

\par 8 Car j'ai créé toutes les forces, et il n'y en a aucune qui me résiste ou qui ne se soumette à moi. Car tous se soumettent à ma monarchie et travaillent pour mon seul règne.

\par 9 Donnez-leur les livres d'écriture, et ils les liront et me connaîtront comme le créateur de toutes choses, et comprendront qu'il n'y a pas d'autre Dieu que moi.

\par 10 Et qu'ils distribuent les livres de ta main, d'enfants en enfants, de génération en génération, de nations en nations.

\par 11 Et je te donnerai, Enoch, mon intercesseur, l'archistratège Michel, pour les écritures de tes pères Adam, Seth, Enos, Caïnan, Mahaleleel et Jared ton père.

\chapter{34}

\par \textit{Dieu convainc les idolâtres et les fornicateurs sodomitiques, et fait donc tomber sur eux un déluge.}

\par 1 ILS ont rejeté mes commandements et mon joug, une postérité sans valeur a germé, sans craindre Dieu, et ils n'ont pas voulu se prosterner devant moi, mais ont commencé à se prosterner devant des dieux vains, et ont nié mon unité, et ont chargé le toute la terre avec des contrevérités, des offenses, des luxures abominables, notamment les unes avec les autres, et toutes sortes d'autres méchancetés impures, qui sont dégoûtantes à raconter.

\par 2 Et c'est pourquoi je ferai tomber un déluge sur la terre et je détruirai tous les hommes, et la terre entière s'effondrera dans de grandes ténèbres.

\chapter{35}

\par \textit{Dieu laisse un juste de la tribu d'Enoch avec toute sa maison, qui a fait le plaisir de Dieu selon sa volonté.}

\par 1 VOICI que de leur postérité naîtra une autre génération, bien plus tard, mais beaucoup d'entre eux seront très insatiables.

\par 2 Celui qui suscitera cette génération leur révélera les livres de ton écriture, de tes pères, à ceux à qui il doit indiquer la garde du monde, aux hommes fidèles et aux ouvriers de mon plaisir, qui ne reconnaissent pas mon nom en vain.

\par 3 Et ils le diront à une autre génération, et ceux qui auront lu seront ensuite glorifiés plus que la première.

\chapter{36}

\par \textit{Dieu ordonna à Hénoc de vivre sur terre trente jours, pour donner l'instruction à ses fils et aux enfants de ses enfants. Après trente jours, il fut de nouveau emmené au ciel.}

\par 1 Enoch, je te donne un délai de trente jours à passer dans ta maison, et tu le diras à tes fils et à toute ta famille, afin que tous entendent de ma bouche ce que tu leur diras, afin qu'ils lisent et comprennent qu'il n'y a pas d'autre Dieu que moi.

\par 2 Et afin qu'ils gardent toujours mes commandements, et qu'ils commencent à lire et à lire les livres de ton écriture.

\par 3 Et après trente jours, j'enverrai mon ange pour toi, et il te prendra de la terre et de tes fils vers moi.

\chapter{37}

\par 1 ET le Seigneur appela un des anges les plus âgés, terrible et menaçant, et le plaça près de moi, d'apparence blanche comme la neige, et ses mains comme de la glace, ayant l'apparence d'une grande gelée, et il me gela le visage, parce que Je ne pourrais pas supporter la terreur du Seigneur, tout comme il n'est pas possible de supporter le feu d'un poêle, la chaleur du soleil et le gel de l'air.

\par 2 Et l'Éternel me dit : 'Hénoc, si ton visage n'est pas gelé ici, personne ne pourra voir ton visage.'

\chapter{38}

\par \textit{Mathusal continuait d'avoir espoir et d'attendre son père Enoch sur son canapé jour et nuit.}

\par 1 ET le Seigneur dit à ces hommes qui m'ont fait monter les premiers : 'Laissez Enoch descendre sur terre avec vous, et attendez-le jusqu'au jour déterminé.'

\par 2 Et ils m'ont placé la nuit sur mon lit.

\par 3 Et Mathusal attendait mon arrivée, veillant jour et nuit sur mon lit, fut rempli d'effroi quand il entendit mon arrivée, et je lui dis : « Que toute ma maison se réunisse, que je leur raconte tout. »

\chapter{39}

\par \textit{L'exhortation pitoyable d'Hénoch à ses fils avec des pleurs et de grandes lamentations, pendant qu'il leur parlait.}

\par 1 Oh mes enfants, mes bien-aimés, écoutez l'exhortation de votre père, autant que cela est selon la volonté du Seigneur.

\par 2 J'ai été autorisé à venir à vous aujourd'hui et à vous annoncer, non de mes lèvres, mais de celles du Seigneur, tout ce qui est et était et tout ce qui est maintenant, et tout ce qui arrivera jusqu'au jour du jugement. .

\par 3 Car le Seigneur m'a laissé venir à vous, vous entendez donc les paroles de mes lèvres, d'un homme fait grand pour vous, mais j'ai vu la face du Seigneur, comme le fer fait briller par le feu qu'il envoie des étincelles et des brûlures,

\par 4 Tu regardes maintenant mes yeux, les yeux d'un homme grand qui a du sens pour toi, mais j'ai vu les yeux du Seigneur, brillant comme les rayons du soleil et remplissant les yeux de l'homme de crainte.

\par 5 Vous voyez maintenant, mes enfants, la main droite d'un homme qui vous aide, mais j'ai vu la main droite du Seigneur remplir le ciel pour m'aider.

\par 6 Vous voyez l'étendue de mon œuvre comme la vôtre, mais j'ai vu l'étendue illimitée et parfaite du Seigneur, qui n'a pas de fin.

\par 7 Vous entendez les paroles de mes lèvres, comme j'ai entendu les paroles du Seigneur, comme un grand tonnerre sans cesse avec des jets de nuages.

\par 8 Et maintenant, mes enfants, écoutez les discours du père de la terre, combien il est effrayant et terrible de se retrouver devant la face du souverain de la terre, combien il est encore plus terrible et affreux de se retrouver devant la face du souverain des cieux, le contrôleur des vivants et des morts, et des troupes célestes. Qui peut supporter cette douleur infinie ?

\chapter{40}

\par \textit{Enoch avertit vraiment ses enfants de toutes choses venant des lèvres du Seigneur, comment il a vu, entendu et écrit.}

\par 1 ET maintenant, mes enfants, je sais toutes choses, car cela sort des lèvres du Seigneur, et cela mes yeux l'ont vu, du début à la fin.

\par 2 Je connais toutes choses, et j'ai écrit toutes choses dans des livres, les cieux et leur fin, et leur plénitude, et toutes les armées et leurs marches.

\par 3 J'ai mesuré et décrit les étoiles, leur grande multitude innombrable.

\par 4 Quel homme a vu leurs révolutions, et leurs entrées ? Car même les anges ne voient pas leur nombre, alors que j'ai écrit tous leurs noms.

\par 5 Et j'ai mesuré le cercle du soleil, et j'ai mesuré ses rayons, j'ai compté les heures, j'ai aussi écrit tout ce qui passe sur la terre, j'ai écrit les choses qui sont nourries, et toute semence semée et non semée, que la terre produits et toutes les plantes, et chaque herbe et chaque fleur, et leurs odeurs douces, et leurs noms, et les demeures des nuages, et leur composition, et leurs ailes, et comment ils portent la pluie et les gouttes de pluie.

\par 6 Et j'ai étudié toutes choses, et j'ai écrit le chemin du tonnerre et de l'éclair, et ils m'ont montré les clés et leurs gardiens, leur montée, le chemin qu'ils vont ; on le laisse sortir avec mesure (sc. doucement) par une chaîne, de peur que, par une lourde chaîne et par la violence, il ne précipite les nuages ​​​​en colère et ne détruise toutes choses sur terre.

\par 7 J'ai écrit les trésors de la neige et les trésors du froid et de l'air glacial, et j'ai observé le détenteur des clés de leur saison, il en remplit les nuages ​​et n'épuise pas les trésors. .

\par 8 Et j'ai écrit les lieux de repos des vents et j'ai observé et vu comment leurs détenteurs de clés portent des balances et des mesures ; Ils les placèrent d'abord dans une balance, puis dans l'autre les poids et les étalèrent astucieusement selon la mesure sur toute la terre, de peur que par une respiration lourde ils ne fassent trembler la terre.

\par 9 Et j'ai mesuré toute la terre, ses montagnes, et toutes les collines, les champs, les arbres, les pierres, les rivières, tout ce qui existe, j'ai écrit, depuis la terre jusqu'au septième ciel, et en bas jusqu'à l'enfer le plus bas, et le lieu du jugement, et le très grand enfer, ouvert et pleurant.

\par 10 Et j'ai vu combien les prisonniers souffrent, attendant le jugement sans limites.

\par 11 Et j'ai écrit tous ceux qui étaient jugés par le juge, et tous leurs jugements (sc. sentences) et toutes leurs œuvres.

\chapter{41}

\par \textit{De la façon dont Enoch a déploré le péché d'Adam.}

\par 1 ET j'ai vu tous les ancêtres de tous les temps avec Adam et Eva, et j'ai soupiré et j'ai fondu en larmes et j'ai dit à propos de la ruine de leur déshonneur :

\par 2. « Malheur à moi pour mon infirmité et pour celle de mes ancêtres », et j'ai pensé dans mon cœur et j'ai dit :

\par 3 'Bienheureux l'homme qui n'est pas né, ou qui est né et qui ne péchera pas devant la face du Seigneur, de peur qu'il ne vienne dans ce lieu, et qu'il n'apporte le joug de ce lieu !

\chapter{42}

\par \textit{De la façon dont Enoch a vu debout les détenteurs des clés et les gardes des portes de l'enfer.}

\par 1 J'ai vu les détenteurs des clés et les gardes des portes de l'enfer debout, comme de grands serpents, et leurs visages comme des lampes éteintes, et leurs yeux de feu, leurs dents acérées, et j'ai vu toutes les œuvres du Seigneur, combien elles sont justes. , tandis que les œuvres de l'homme sont certaines bonnes et d'autres mauvaises, et dans leurs œuvres sont connus ceux qui mentent mal.

\chapter{43}

\par \textit{Enoch montre à ses enfants comment il mesurait et écrivait les jugements de Dieu.}

\par 1 Moi, mes enfants, j'ai mesuré et écrit chaque œuvre, chaque mesure et chaque jugement juste.

\par 2 Comme une année est plus honorable qu'une autre, ainsi un homme est plus honorable qu'un autre, les uns pour de grandes possessions, les autres pour la sagesse du cœur, les autres pour l'intelligence particulière, les autres pour la ruse, l'un pour le silence des lèvres, l'autre pour la propreté. , un pour la force, un autre pour la beauté, un pour la jeunesse, un autre pour l'esprit vif, un pour la forme du corps, un autre pour la sensibilité, que cela soit entendu partout, mais il n'y a personne de meilleur que celui qui craint Dieu, il sera plus glorieux dans le temps à venir.

\chapter{44}

\par \textit{Hénoch ordonne à ses fils de ne pas insulter la face de l'homme, petit ou grand.}

\par 1 LE Seigneur avec ses mains ayant créé l'homme, à l'image de son propre visage, le Seigneur l'a fait petit et grand.

\par 2 Celui qui insulte la face du chef, et qui a en horreur la face de l'Eternel, qui a méprisé la face de l'Eternel, et qui déverse sa colère sur quelqu'un sans lui faire de mal, la grande colère de l'Eternel le coupera, celui qui crache sur la face d'un homme avec reproche. , sera abattu au grand jugement du Seigneur.

\par 3 Bienheureux est l'homme qui ne dirige pas son cœur avec méchanceté contre qui que ce soit, qui aide les blessés et les condamnés, qui relève ceux qui sont brisés, et qui fait la charité aux nécessiteux, car au jour du grand jugement tout fardeau , chaque mesure et chaque poids seront comme au marché, c'est-à-dire qu'ils sont suspendus à une balance et placés sur le marché, et chacun apprendra sa propre mesure, et selon sa mesure recevra sa récompense.



\chapter{45}

\par \textit{Dieu montre comment il ne veut pas des hommes des sacrifices, ni des holocaustes, mais des cœurs purs et contrits.}

\par 1 Quiconque s'empresse de faire une offrande devant la face du Seigneur, le Seigneur, pour sa part, hâtera cette offrande en lui accordant son œuvre.

\par 2 Mais quiconque élève sa lampe devant la face du Seigneur et ne porte pas un jugement véritable, le Seigneur n'augmentera pas son trésor dans le royaume des cieux.

\par 3 Quand le Seigneur demande du pain, ou des bougies, ou de la chair (sc. du bétail), ou tout autre sacrifice, alors ce n'est rien ; mais Dieu exige des cœurs purs, et avec tout cela, il ne fait que tester le cœur de l'homme.

\chapter{46}

\par \textit{De la façon dont un dirigeant terrestre n'accepte pas de l'homme des dons abominables et impurs, combien plus Dieu a-t-il abominable les dons impurs, mais il les renvoie avec colère et n'accepte pas ses dons.}

\par 1 Écoutez, mon peuple, et recevez les paroles de mes lèvres.

\par 2 Si quelqu'un apporte des cadeaux à un dirigeant terrestre et a des pensées déloyales dans son cœur, et que le dirigeant le sache, ne sera-t-il pas en colère contre lui, ne refusera-t-il pas ses cadeaux et ne le livrera-t-il pas au jugement ?

\par 3 Ou si un homme se fait paraître bon à un autre par la tromperie de la langue, mais a du mal dans son cœur, alors l'autre ne comprendra-t-il pas la trahison de son cœur, et ne sera-t-il pas lui-même condamné, puisque son mensonge était évident à tous ?

\par 4 Et quand le Seigneur enverra une grande lumière, alors il y aura un jugement pour les justes et les injustes, et là personne n'échappera à l'attention.

\chapter{47}

\par \textit{Enoch instruit ses fils de la bouche de Dieu et leur remet l'écriture de ce livre.}

\par 1 ET maintenant, mes enfants, pensez à vos cœurs, notez bien les paroles de votre père, qui vous sont toutes sorties des lèvres du Seigneur.

\par 2 Prends ces livres écrits par ton père et lis-les.

\par 3 Car les livres sont nombreux, et dans eux vous apprendrez toutes les œuvres du Seigneur, tout ce qui a été depuis le commencement de la création et qui durera jusqu'à la fin des temps.

\par 4 Et si vous observez mon écriture, vous ne pécherez pas contre l'Éternel ; car il n'y en a d'autre que le Seigneur, ni dans le ciel, ni sur la terre, ni dans les lieux les plus bas, ni dans l'unique fondement.

\par 5 Le Seigneur a posé les fondements dans l'inconnu, et a étendu les cieux visibles et invisibles ; il a fixé la terre sur les eaux et a créé d'innombrables créatures, et qui a compté l'eau et le fondement du non fixé, ou la poussière de la terre, ou le sable de la mer, ou les gouttes de pluie, ou le matin la rosée ou les respirations du vent ? Qui a rempli la terre et la mer, et l'hiver indissoluble ?

\par 6 J'ai découpé les étoiles du feu, j'ai décoré le ciel et je l'ai mis au milieu d'elles.

\chapter{48}

\par \textit{Du passage du soleil le long des sept cercles.}

\par 1 Que le soleil parcourt les sept cercles célestes, qui sont le rendez-vous de cent quatre-vingt-deux trônes, qu'il se couche un jour court, et encore cent quatre-vingt-deux, qu'il se couche un jour long, et qu'il a deux trônes sur lesquels il repose, tournant de-ci de-là au-dessus des trônes des mois,

\par 2 Du dix-septième jour du mois de Tsivan il descend jusqu'au mois de Thévan, à partir du dix-septième de Thévan il monte.

\par 3 Et ainsi il s'approche de la terre, alors la terre est et fait pousser ses fruits, et quand elle s'éloigne, alors la terre est triste, et les arbres et tous les fruits n'ont aucune floraison.

\par 4 Il a mesuré tout cela, avec une bonne mesure des heures, et a fixé une mesure par sa sagesse, du visible et de l'invisible.

\par 5 De l'invisible, il a rendu toutes choses visibles, lui-même étant invisible.

\par 6 Ainsi je vous fais connaître, mes enfants, et je distribue les livres à vos enfants, dans toutes vos générations et parmi les nations qui auront le sens de craindre Dieu, qu'ils les reçoivent, et qu'ils arrivent à aimer eux plus que n'importe quelle nourriture ou douceur terrestre, et lisez-les et appliquez-vous à eux.

\par 7 Et ceux qui ne comprennent pas le Seigneur, qui ne craignent pas Dieu, qui ne les acceptent pas, mais les rejettent, qui ne les reçoivent pas (voir les livres), un jugement terrible les attend.

\par 8 Bienheureux est l'homme qui portera leur joug et les entraînera, car il sera libéré le jour du grand jugement.

\chapter{49}

\par \textit{Enoch demande à ses fils de ne jurer ni par le ciel ni par la terre, et montre la promesse de Dieu, même dans le ventre de la mère.}

\par 1 JE vous le JURE, mes enfants, mais je ne jure par aucun serment, ni par le ciel, ni par la terre, ni par aucune autre créature que Dieu a créée.

\par 2 Le Seigneur a dit : 'Il n'y a en moi ni serment, ni injustice, mais la vérité.'

\par 3 S'il n'y a pas de vérité chez les hommes, qu'ils jurent par les mots « oui, oui », ou bien « non, non !

\par 4 Et je vous jure, oui, oui, qu'il n'y a eu aucun homme dans le ventre de sa mère, mais que déjà auparavant, même pour chacun il y a une place préparée pour le repos de l'âme, et une mesure fixée comment il est bien entendu qu'un homme soit jugé dans ce monde.

\par 5 Oui, mes enfants, ne vous y trompez pas, car une place a été préparée d'avance pour chaque âme d'homme.

\chapter{50}

\par \textit{De la façon dont aucun né sur terre ne peut rester caché ni son œuvre rester cachée, mais il (sc. Dieu) nous ordonne d'être doux, d'endurer les attaques et les insultes, et de ne pas offenser les veuves et les orphelins.}

\par 1 J'AI mis le travail de chacun par écrit et aucun né sur terre ne peut rester caché ni ses œuvres rester cachées.

\par 2 Je vois toutes choses.

\par 3 Maintenant donc, mes enfants, passez avec patience et douceur le nombre de vos jours, afin que vous héritiez de la vie sans fin.

\par 4 Supportez pour l'amour du Seigneur toute blessure, toute injure, toute mauvaise parole et toute attaque.

\par 5 Si des malheurs vous arrivent, ne les rendez ni à votre prochain ni à vos ennemis, car l'Éternel les rendra pour vous et sera votre vengeur au jour du grand jugement, afin qu'il n'y ait pas de vengeance ici parmi les hommes.

\par 6 Celui d'entre vous qui dépense de l'or ou de l'argent pour l'amour de son frère recevra de grands trésors dans le monde à venir.

\par 7 Ne faites pas de mal aux veuves, ni aux orphelins, ni aux étrangers, de peur que la colère de Dieu ne s'abatte sur vous.

\chapter{51}

\par \textit{Enoch ordonne à ses fils de ne pas cacher de trésors dans la terre, mais leur ordonne de faire l'aumône aux pauvres.}

\par 1 étendez vos mains aux pauvres selon vos forces.

\par 2 Ne cache pas ton argent dans la terre.

\par 3 Aidez l'homme fidèle dans l'affliction, et l'affliction ne vous trouvera pas au temps de votre détresse.

\par 4 Et tout joug douloureux et cruel qui s'abat sur vous, supportez tout à cause du Seigneur, et ainsi vous trouverez votre récompense au jour du jugement.

\par 5 Il est bon d'entrer matin, midi et soir dans la demeure du Seigneur, pour la gloire de ton créateur.

\par 6 Parce que tout ce qui respire le glorifie, et que toute créature visible et invisible lui rend sa louange.

\chapter{52}

\par \textit{Dieu indique à ses fidèles comment ils doivent louer son nom.}

\par 1 BÉNI soit l'homme qui ouvre ses lèvres pour louer Dieu de Sabaoth et loue le Seigneur avec son cœur.

\par 2 Maudit tout homme qui ouvre la bouche pour mépriser et calomnier son prochain, parce qu'il méprise Dieu.

\par 3 Bienheureux celui qui ouvre ses lèvres en bénissant et en louant Dieu.

\par 4 Maudit soit celui devant l'Éternel tous les jours de sa vie, qui ouvre ses lèvres à la malédiction et à l'injure.

\par 5 Bienheureux celui qui bénit toutes les œuvres du Seigneur.

\par 6 Maudit soit celui qui méprise la création du Seigneur.

\par 7 Bienheureux celui qui regarde en bas et relève ceux qui sont tombés.

\par 8 Maudit soit celui qui regarde et désire la destruction de ce qui n'est pas à lui.

\par 9 Bienheureux est celui qui maintient fermes les fondements de ses pères dès le commencement.

\par 10 Maudit soit celui qui pervertit les décrets de ses ancêtres.

\par 11 Bienheureux celui qui implante la paix et l'amour.

\par 12 Maudit soit celui qui dérange ceux qui aiment leur prochain.

\par 13 Bienheureux celui qui parle à tous avec une langue et un cœur humbles.

\par 14 Maudit soit celui qui parle de paix avec sa langue, alors que dans son coeur il n'y a pas de paix, mais une épée.

\par 15 Car toutes ces choses seront mises à nu dans les balances et dans les livres, le jour du grand jugement.

\chapter{53}

\par \textit{(Ne disons pas : 'Notre père est devant Dieu, il se présentera pour nous au jour du jugement', car là le père ne peut pas aider le fils, ni encore le fils le père.)}

\par 1 ET maintenant, mes enfants, ne dites pas : « Notre père se tient devant Dieu et prie pour nos péchés », car il n'y a aucun secours pour aucun homme qui a péché.

\par 2 Vous voyez comment j'ai écrit toutes les œuvres de chaque homme, avant sa création, tout ce qui se fait parmi tous les hommes depuis toujours, et personne ne peut dire ou raconter mon écriture, parce que le Seigneur voit toutes les imaginations de l'homme, comment elles sont vains là où ils reposent dans les trésors du cœur.

\par 3 Et maintenant, mes enfants, notez bien toutes les paroles de votre père, que je vous dis, de peur que vous ne regrettiez, en disant : 'Pourquoi notre père ne nous l'a-t-il pas dit ?'

\chapter{54}

\par \textit{Enoch demande à ses fils de remettre également les livres à d'autres.}

\par 1 En ce temps-là, ne comprenant pas cela, que ces livres que je vous ai donnés soient en héritage de votre paix.

\par 2 Donnez-les à tous ceux qui en veulent, et instruisez-les, afin qu'ils voient les œuvres très grandes et merveilleuses du Seigneur.



\chapter{55}

\par \textit{Ici Hénoch montre ses fils et leur dit en pleurant : 'Mes enfants, l'heure est proche pour moi de monter au ciel ; voici, les anges se tiennent devant moi.'}

\par 1 MES enfants, voici, le jour de mon terme et le temps sont approchés.

\par 2 Car les anges qui m'accompagneront se tiennent devant moi et me poussent à m'éloigner de toi ; ils se tiennent ici sur terre, attendant ce qui leur a été dit.

\par 3 Car demain je monterai au ciel, à la Jérusalem la plus haute, vers mon héritage éternel.

\par 4 C'est pourquoi je vous ordonne de faire devant la face du Seigneur tout son bon plaisir.

\chapter{56}

\par \textit{Methosalam demande à son père la bénédiction, afin qu'il (sc. Methosalam) puisse lui préparer (sc. Enoch) de la nourriture à manger.}

\par 1 METHOSALAM ayant répondu à son père Enoch, dit : 'Ce qui est agréable à tes yeux, père, que je puisse le faire devant ta face, afin que tu bénisses nos demeures et tes fils, et que ton peuple soit rendu glorieux à travers toi, et ensuite que tu puisses partir ainsi, comme le Seigneur l'a dit ?,

\par 2 Hénoc répondit à son fils Méthosalam et dit : « Écoute, mon enfant, depuis le temps où le Seigneur m'a oint du parfum de sa gloire, il n'y a plus eu de nourriture en moi, et mon âme ne se souvient pas des jouissances terrestres, et mon âme ne se souvient pas non plus des jouissances terrestres. Je veux quelque chose de terrestre !

\chapter{57}

\par \textit{Hénoch demanda à son fils Methosalam de convoquer tous ses frères.}

\par 1 MON enfant Methosalam, appelle tous tes frères et notre maison et les anciens du peuple, afin que je puisse leur parler et m'en aller, comme cela est prévu pour moi.'

\par 2 Methosalam se hâta de convoquer ses frères, Regim, Riman, Uchan, Chermion, Gaidad, et tous les anciens du peuple, devant son père Enoch ; il les bénit et leur dit :

\chapter{58}

\par \textit{Instruction d'Énoch à ses fils.}

\par 1 Écoutez-moi, mes enfants, aujourd'hui.

\par 2 En ces jours où le Seigneur descendit sur terre à cause d'Adam et visita toutes ses créatures, qu'il créa lui-même, après tout cela, il créa Adam, et le Seigneur appela toutes les bêtes de la terre, tous les reptiles , et tous les oiseaux qui planent dans les airs, et nous les avons tous amenés devant la face de notre père Adam.

\par 3 Et Adam donna des noms à tout ce qui vit sur la terre.

\par 4 Et l'Éternel l'établit chef sur tous, et lui soumit toutes choses sous ses mains, et les rendit muets et ennuyés afin qu'ils soient commandés par l'homme, et qu'ils lui soient soumis et obéissance.

\par 5 Ainsi aussi l'Éternel a créé chaque homme seigneur sur tous ses biens.

\par 6 Le Seigneur ne jugera pas une seule âme de bête à cause de l'homme, mais il jugera les âmes des hommes selon leurs bêtes dans ce monde ; car les hommes ont une place particulière.

\par 7 Et comme chaque âme de l'homme est selon le nombre, de même les bêtes ne périront pas, ni toutes les âmes des bêtes que le Seigneur a créées, jusqu'au grand jugement, et ils accuseront l'homme s'il les nourrit mal.

\chapter{59}

\par \textit{Enoch enseigne à ses fils pourquoi ils ne doivent pas toucher au bœuf à cause de ce qui en découle.}

\par 1 Celui qui souille l'âme des bêtes, souille sa propre âme.

\par 2 Car l'homme amène des animaux purs pour les offrir en sacrifice pour le péché, afin d'avoir la guérison de son âme.

\par 3 Et s'ils apportent en sacrifice des animaux purs et des oiseaux, l'homme a la guérison, il guérit son âme.

\par 4 Tout vous est donné en nourriture, liez-le par les quatre pieds, c'est-à-dire pour guérir la guérison, il guérit son âme.

\par 5 Mais celui qui tue une bête sans la blesser tue sa propre âme et souille sa propre chair.

\par 6 Et celui qui fait quelque mal à quelque bête que ce soit, en secret, c'est une mauvaise pratique, et il souille son âme.

\chapter{60}

\par \textit{Celui qui fait du tort à l'âme de l'homme, fait du tort à sa propre âme, et il n'y a pas de remède pour sa chair, ni de pardon pour toujours. Comment il n'est pas convenable de tuer l'homme ni par les armes ni par la langue.}

\par 1 Celui qui travaille à tuer l'âme d'un homme, tue sa propre âme et tue son propre corps, et il n'y a pas de remède pour lui pour toujours.

\par 2 Celui qui met un homme dans un piège, y restera lui-même, et il n'y aura pas de remède pour lui pour toujours.

\par 3 Celui qui met un homme dans un vase quelconque, sa rétribution ne manquera pas au grand jugement pour toujours.

\par 4 Celui qui agit mal ou dit du mal contre quelqu'un ne se fera pas justice pour toujours.

\chapter{61}

\par \textit{Enoch demande à ses fils de se garder de l'injustice et souvent de tendre la main aux pauvres, pour donner une part de leurs travaux.}

\par 1 ET maintenant, mes enfants, gardez vos cœurs de toute injustice, que le Seigneur déteste. Tout comme un homme demande (sc. quelque chose) à Dieu pour sa propre âme, qu'il le fasse ainsi à toute âme vivante, car je sais toutes choses, comment dans le grand temps (sc. à venir) de nombreuses demeures sont préparées pour les hommes, bon pour le bien et mauvais pour le mauvais, sans nombre.

\par 2 Bienheureux ceux qui entrent dans les bonnes maisons, car dans les mauvaises (sc. maisons) il n'y a pas de paix, ni de retour (sc. d'elles).

\par 3 Écoutez, mes enfants, petits et grands ! Quand l'homme met une bonne pensée dans son cœur, apporte les dons de ses travaux devant la face du Seigneur et que ses mains ne les ont pas faits, alors le Seigneur détournera son visage du travail de ses mains, et il (sc. l'homme) ne peut pas trouver le travail de ses mains.

\par 4 Et si ses mains l'ont fait, mais que son cœur murmure, et que son cœur ne cesse de murmurer sans cesse, il n'a aucun avantage.

\chapter{62}

\par \textit{De la façon dont il convient d'apporter son don avec foi, car il n'y a pas de repentir après la mort.}

\par 1 BÉNI soit l'homme qui dans sa patience présente ses dons avec foi devant la face du Seigneur, car il trouvera le pardon des péchés.

\par 2 Mais s'il retire ses paroles avant le temps, il n'y a pas de repentir pour lui ; et si le temps passe et qu'il ne fait pas de sa propre volonté ce qui est promis, il n'y a pas de repentir après la mort.

\par 3 Parce que toute œuvre que l'homme fait avant le temps est toute tromperie devant les hommes, et péché devant Dieu.



\chapter{63}

\par \textit{De comment ne pas mépriser les pauvres, mais partager également avec eux, de peur d'être murmuré devant Dieu.}

\par 1 QUAND l'homme habille celui qui est nu et rassasie celui qui a faim, il trouvera une récompense de la part de Dieu.

\par 2 Mais si son cœur murmure, il commet un double mal : ruine de lui-même et de ce qu'il donne ; et pour lui, il n'y aura aucune récompense à cause de cela.

\par 3 Et si son propre cœur est rempli de sa nourriture et sa propre chair (sc. vêtue) de ses vêtements, il commet du mépris, et perd toute son endurance de pauvreté, et ne trouvera pas de récompense pour ses bonnes actions.

\par 4 Tout homme orgueilleux et magnifique est odieux au Seigneur, et tout discours faux est revêtu de mensonge ; il sera coupé avec la lame de l'épée de la mort, jeté au feu et brûlera pour toujours.

\chapter{64}

\par \textit{De la façon dont le Seigneur appelle Enoch et que les gens décident d'aller l'embrasser à l'endroit appelé Achuzan.}

\par 1 QUAND Enoch eut dit ces paroles à ses fils, tous les gens de loin et d'près entendirent comment le Seigneur appelait Enoch. Ils prirent conseil ensemble :

\par 2 'Allons embrasser Enoch' et deux mille hommes se rassemblèrent et arrivèrent au lieu d'Achuzan où se trouvaient Enoch et ses fils.

\par 3 Et les anciens du peuple, toute l'assemblée, vinrent, se prosternèrent et commencèrent à embrasser Hénoc et lui dirent :

\par 4 'Notre père Enoch, sois béni du Seigneur, le souverain éternel, et maintenant bénis tes fils et tout le peuple, afin que nous soyons glorifiés aujourd'hui devant ta face.

\par 5 Car tu seras glorifié devant la face du Seigneur pour toujours, puisque le Seigneur t'a choisi, plutôt que tous les hommes sur la terre, et t'a désigné écrivain de toute sa création, visible et invisible, et rédempteur des péchés de l'homme, et aide de ta maison.

\chapter{65}

\par \textit{De l'instruction d'Enoch à ses fils.}

\par 1 ET Hénoc répondit à tout son peuple en disant : 'Écoutez, mes enfants, avant que toutes les créatures soient créées, le Seigneur a créé les choses visibles et invisibles.

\par 2 Et autant qu'il y a eu et s'est écoulé, comprenez qu'après cela, il a créé l'homme à la ressemblance de sa propre forme, et lui a mis des yeux pour voir, des oreilles pour entendre, un cœur pour réfléchir, et un intellect avec lequel délibérer.

\par 3 Et l'Éternel vit toutes les œuvres de l'homme, et créa toutes ses créatures, et divisa le temps, à partir du temps il fixa les années, et à partir des années il fixa les mois, et à partir des mois il fixa les jours, et des jours il fixa nommé sept.

\par 4 Et dans celles-ci, il fixa les heures, les mesura exactement, afin que l'homme puisse réfléchir sur le temps et compter les années, les mois et les heures, leur alternance, leur commencement et leur fin, et qu'il puisse compter sa propre vie, à partir du en commençant jusqu'à la mort, et réfléchir à son péché et écrire son œuvre, mauvaise et bonne ; parce qu'aucune œuvre n'est cachée devant le Seigneur, afin que chacun connaisse ses œuvres et ne transgresse jamais tous ses commandements, et garde mon écriture de génération en génération.

\par 5 Quand toute la création visible et invisible, telle que le Seigneur l'a créée, prendra fin, alors tout homme ira au grand jugement, et alors tous les temps périront, ainsi que les années, et désormais il n'y aura plus ni mois, ni jours, ni heures. , ils seront collés ensemble et ne seront pas comptés.

\par 6 Il y aura un seul éon, et tous les justes qui échapperont au grand jugement du Seigneur seront rassemblés dans le grand éon. Pour les justes, le grand éon commencera, et ils vivront éternellement, et alors aussi il n'y aura parmi eux ni travail, ni maladie, ni humiliation, ni angoisse, ni besoin, ni violence, ni nuit, ni ténèbres, mais une grande lumière.

\par 7 Et ils auront une grande muraille indestructible, et un paradis lumineux et incorruptible, car toutes choses corruptibles passeront, et il y aura la vie éternelle.

\chapter{66}

\par \textit{Hénoch enseigne à ses fils et à tous les anciens du peuple comment ils doivent marcher avec terreur et tremblement devant le Seigneur, le servir seul et ne pas se prosterner devant les idoles, mais devant Dieu, qui a créé le ciel et la terre et toutes les créatures, et devant son image.}

\par 1 ET maintenant, mes enfants, gardez vos âmes de toute injustice, telle que le Seigneur hait.

\par 2 Marchez devant sa face avec terreur et tremblement et servez-le seul.

\par 3 Inclinez-vous devant le vrai Dieu, non devant des idoles muettes, mais inclinez-vous devant son image, et présentez toutes les offrandes justes devant la face de l'Éternel. Le Seigneur déteste ce qui est injuste.

\par 4 Car le Seigneur voit toutes choses ; quand l'homme a une pensée dans son cœur, alors il conseille l'intellect, et chaque pensée est toujours devant le Seigneur, qui a affermi la terre et y a placé toutes les créatures.

\par 5 Si vous regardez au ciel, le Seigneur est là ; si vous pensez aux profondeurs de la mer et à tout ce qui est sous la terre, le Seigneur est là.

\par 6 Car le Seigneur a créé toutes choses. Ne vous inclinez pas devant les choses faites par l'homme, en quittant le Seigneur de toute la création, car aucune œuvre ne peut rester cachée devant la face du Seigneur.

\par 7 Marchez, mes enfants, dans la patience, dans la douceur, dans l'honnêteté, dans la provocation, dans la douleur, dans la foi et dans la vérité, dans la confiance dans les promesses, dans la maladie, dans les abus, dans les blessures, dans la tentation, dans la nudité, dans la privation. , aimez-vous les uns les autres, jusqu'à ce que vous sortiez de cet âge de maux, afin que vous deveniez les héritiers d'un temps sans fin.

\par 8 Bienheureux les justes qui échapperont au grand jugement, car ils brilleront sept fois plus que le soleil, car dans ce monde la septième partie est retranchée de tout, lumière, ténèbres, nourriture, jouissance, tristesse, paradis, la torture, le feu, le gel et d'autres choses ; il a tout mis par écrit, afin que vous puissiez lire et comprendre.

\chapter{67}

\par \textit{Le Seigneur laissa tomber les ténèbres sur la terre et couvrit le peuple et Enoch, et il fut élevé en haut, et la lumière revint dans le ciel.}

\par 1 QUAND Enoch avait parlé au peuple, le Seigneur envoya des ténèbres sur la terre, et il y eut des ténèbres, et elles couvrirent ces hommes qui se tenaient avec Enoch, et ils emmenèrent Enoch au plus haut des cieux, où est le Seigneur. ; et il le reçut et le plaça devant sa face, et les ténèbres disparurent de la terre, et la lumière revint.

\par 2 Et le peuple ne vit et ne comprit pas comment Enoch avait été pris, et avait glorifié Dieu, et avait trouvé un rouleau dans lequel était tracé « le Dieu invisible » ; et tous rentrèrent chez eux.

\chapter{68}

\par 1 HÉNOCH naquit le sixième jour du mois de Tsivan et vécut trois cent soixante-cinq ans.

\par 2 Il fut élevé au ciel le premier jour du mois de Tsivan et resta au ciel soixante jours.

\par 3 Il écrivit tous ces signes de toute la création que le Seigneur créa, et écrivit trois cent soixante-six livres, et les remit à ses fils et resta sur terre trente jours, et fut de nouveau enlevé au ciel le sixième jour du mois Tsivan, le jour et l'heure mêmes de sa naissance.

\par 4 De même que la nature de chaque homme dans cette vie est sombre, sa conception, sa naissance et son départ de cette vie le sont également.

\par 5 A quelle heure il a été conçu, à cette heure il est né, et à cette heure aussi il est mort.

\par 6 Methosalam et ses frères, tous les fils d'Enoch, se hâtèrent et érigèrent un autel au lieu appelé Achuzan, d'où et où Enoch avait été enlevé au ciel.

\par 7 Et ils prirent des bœufs de sacrifice, convoquèrent tout le monde et sacrifièrent le sacrifice devant la face de l'Éternel.

\par 8 Tout le peuple, les anciens du peuple et toute l'assemblée vinrent à la fête et apportèrent des présents aux fils d'Hénoc.

\par 9 Ils firent un grand festin, se réjouirent et se livrèrent à des réjouissances pendant trois jours, louant Dieu qui leur avait donné un tel signe par l'intermédiaire d'Hénoc, qui avait trouvé grâce à ses yeux, et qu'ils devaient transmettre à leurs fils de génération en génération, d'âge en âge.

\par 10 Amen.



\end{document}