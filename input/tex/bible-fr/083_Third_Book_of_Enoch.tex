\begin{document}

\title{Troisième Livre d'Enoch}

\chapter{1}

\par \textit{INTRODUCTION : R. Ismaël monte au ciel pour voir la vision de la Merkaba et est chargé de Métatron}

\par \textit{ET ÉNOCH MARCHÉ AVEC DIEU : ET IL NE L'ÉTAIT PAS ; CAR DIEU L'A PRIS (Gen. v.4)}

\par 1 Rabbi Ismaël a dit : Quand je suis monté en haut pour contempler la vision de la Merkaba et que je suis entré dans les six salles, l'une dans l'autre :

\par 2 Dès que j'atteignis la porte de la septième salle, je m'arrêtai en prière devant le Saint, béni soit-Il, et, levant les yeux en haut (c'est-à-dire vers la Divine Majesté), je dis :

\par 3 « Seigneur de l'univers, je te prie, que le mérite d'Aaron, le fils d'Amram, l'amoureux de la paix et le poursuivant de la paix, qui a reçu la couronne de la prêtrise de ta Gloire sur la montagne du Sinaï, soit valide pour moi en cette heure, afin que Qafsiel, le prince, et les anges avec lui n'obtiennent pas de pouvoir sur moi et ne me jettent pas du haut des cieux. ».

\par 4 Le Saint, béni soit-il, m'envoya Metatron, son Serviteur ('Ebed) l'ange, le Prince de la Présence, et celui-ci, déployant ses ailes, vint avec une grande joie à ma rencontre pour me sauver de leur main.

\par 5 Et il me prit par la main devant eux, me disant : « Entre en paix devant le Roi haut et exalté et regarde l'image de la Merkaba ».

\par 6 Puis je suis entré dans la septième salle, et il m'a conduit au(x) camp(s) de Shekina et m'a placé devant le Saint, béni soit-Il, pour contempler la Merkaba.

\par 7 Dès que les princes de la Merkaba et les Séraphins enflammés m'aperçurent, ils fixèrent leurs yeux sur moi. Instantanément, un tremblement et un frisson me saisirent et je tombai et fus engourdi par l'image radieuse de leurs yeux et l'aspect splendide de leurs visages ; jusqu'à ce que le Saint, béni soit-Il, les réprimanda en disant :

\par 8 « Mes serviteurs, mes Séraphins, mes Kérubim et mes f Ophannim ! Couvrez vos yeux devant Ismaël, mon fils, mon ami, mon bien-aimé et ma gloire, afin qu'il ne tremble ni ne frémisse !

\par 9 Immédiatement Métatron, le Prince de la Présence, est venu et m'a restauré l'esprit et m'a remis sur pied.

\par 10 Après ce moment, je n'avais plus la force de chanter une chanson devant le trône de gloire du roi glorieux, le plus puissant de tous les rois, le plus excellent de tous les princes, jusqu'à ce que l'heure soit écoulée.

\par 11 Après une heure (passée) le Saint, béni soit-Il, m'a ouvert les portes de Shekina, les portes de la Paix, les portes de la Sagesse, les portes de la Force, les portes du Pouvoir, les portes de la Parole. (Dibbur)> les portes du Chant, les portes de Qedushsha, les portes du Chant.

\par 12 Et il a éclairé mes yeux et mon cœur par des paroles de psaume, de chant, de louange, d'exaltation, d'action de grâces, d'exaltation, de glorification, d'hymne et d'éloge funèbre. Et alors que j'ouvrais la bouche, poussant un chant devant le Saint, béni soit-Il, le Saint Chayyoth au-dessous et au-dessus du Trône de Gloire répondit et dit : « saint » et « bénie soit la gloire de Yhwh de sa place ! (c'est-à-dire chanté le Qedushsha).

\chapter{2}

\par \textit{Les plus hautes classes d'anges font des recherches sur R. Ismaël, auxquelles Métatron répond}

\par 1 Rabbi Ismaël dit : A cette heure-là, les aigles de la Merkaba, les 'Ophannim enflammés et les Séraphins du feu dévorant interrogeèrent Métatron, lui disant :

\par 2 « Jeunesse ! Pourquoi permets-tu à celui qui est né d'une femme d'entrer et de contempler la Merkaba ? De quelle nation, de quelle tribu est-il ? Quel est son caractère ?

\par 3 Métatron leur répondit : «De la nation d'Israël que le Saint, béni soit-il, a choisie pour son peuple parmi soixante-dix langues (nations), de la tribu de Lévi, qu'il a mise à part pour contribuer à son nom, et de la descendance d'Aaron que le Saint, béni soit-il, a choisie pour son serviteur et lui a donné la couronne de la prêtrise sur le Sinaï».

\par 4 Aussitôt ils parlèrent et dirent : « En effet, celui-ci est digne de voir la Merkaba ». Et ils dirent : « Heureux les gens qui se trouvent dans un tel cas ! »

\chapter{3}

\par \textit{Métatron a 70 noms, mais Dieu l'appelle 'Jeunesse'}

\par 1 R. Ismaël a dit : À cette heure-là, j'ai demandé à Métatron, l'ange, le Prince de la Présence : « Quel est ton nom ?

\par 2 Il me répondit : « J'ai soixante-dix noms, correspondant aux soixante-dix langues du monde et tous sont basés sur le nom Métatron, ange de la Présence ; mais mon Roi m'appelle « Jeunesse » (Na'ar) ».

\chapter{4}

\par \textit{Métatron est identique à Enoch qui fut transporté au ciel au moment du Déluge}

\par 1 R. Ismaël dit : J'ai interrogé Métatron et je lui ai dit : « Pourquoi es-tu appelé du nom de ton Créateur, par soixante-dix noms ? Tu es plus grand que tous les princes, plus haut que tous les anges, aimé plus que tous les serviteurs, honoré par-dessus tous les puissants en royauté, en grandeur et en gloire : pourquoi t'appelle-t-on « Jeune » dans les cieux élevés ?

\par 2 Il répondit et me dit : « Parce que je suis Enoch, le fils de Jared.

\par 3 Car lorsque la génération du déluge pécha et fut confondue dans ses actes, disant à Dieu : « Éloigne-toi de nous, car nous ne désirons pas connaître tes voies », alors le Saint, béni soit-Il, m'a éloigné de leur au milieu d'eux, pour être un témoin contre eux dans les cieux élevés, auprès de tous les habitants du monde, afin qu'ils ne disent pas : « Le Miséricordieux est cruel ».

\par 4 ADEL : Qu'est-ce qui a péché toutes ces multitudes, leurs femmes, leurs fils et leurs filles, leurs chevaux, leurs mulets et leur bétail et leurs biens, et tous les oiseaux du monde, dont tous le Saint, béni soit-Il , détruits du monde avec eux dans les eaux du déluge ? On ne peut pas non plus dire : « Et si la génération du déluge a péché ? les bêtes et les oiseaux, qu'avaient-ils péché pour périr avec eux ? BC : « Quels péchés avaient-ils commis, toutes ces multitudes ? Ou bien, s'ils ont péché, qu'ont péché leurs fils et leurs filles, leurs mulets et leur bétail ? Et de même, tous les animaux domestiques et sauvages, ainsi que les oiseaux du monde, que Dieu a exterminés du monde ?

\par 5 C'est pourquoi le Saint, béni soit-Il, m'a élevé de leur vivant devant leurs yeux pour être un témoin contre eux dans le monde futur. Et le Saint, béni soit-Il, m'a désigné comme prince et chef parmi les anges au service.

\par 6 À cette heure-là, trois des anges au service, 'uzza, 'azza et 'azzael, sortirent et portèrent plainte contre moi dans les cieux élevés, disant devant le Saint, béni soit-il : « Les Anciens n'ont pas dit (Premièrement Ceux-là) juste devant Toi : « Ne crée pas l'homme ! », Le Saint, béni soit-Il, répondit et leur dit : « J'ai fait et je porterai, oui, je porterai et délivrerai ».

\par 7 Dès qu'ils m'ont vu, ils ont dit devant Lui : « Seigneur de l'Univers ! Quel est celui-ci pour qu'il monte au sommet des hauteurs ? N'est-il pas l'un des fils de ceux qui ont péri aux jours du déluge ? « Que fait-il dans la Raqia ? »

\par 8 Encore une fois, le Saint, béni soit-Il, répondit et leur dit : « Qu'êtes-vous, pour que vous entriez et parliez en ma présence ? Celui-ci me plaît plus qu'en vous tous, et c'est pourquoi il sera votre prince et votre chef dans les cieux élevés.

\par 9 Aussitôt tous se levèrent et sortirent à ma rencontre, se prosternèrent devant moi et dirent : « Heureux es-tu et heureux ton père car ton Créateur te favorise ».

\par 10 Et parce que je suis petit et un jeune parmi eux en jours, en mois et en années, c'est pourquoi ils m'appellent « Jeune » (Na'ar).

\chapter{5}

\par \textit{L'idolâtrie de la génération d'Enosh amène Dieu à retirer la Shekina de la terre. L'idolâtrie inspirée par 'Azza, 'Uzza et 'Azziel}

\par 1 R. Ismaël a dit : Métatron, le Prince de la Présence, m'a dit : Depuis le jour où le Saint, béni soit-Il, a expulsé le premier Adam du Jardin d'Eden (et au-delà), Shekina demeurait sur un Kerubin sous l'Arbre de Vie.

\par 2 Et les anges au service se rassemblaient et descendaient du ciel en groupes, de la Raqia en groupes et du ciel en camps pour faire sa volonté dans le monde entier.

\par 3 Et le premier homme et sa génération étaient assis devant la porte du Jardin pour contempler l'apparence radieuse de la Shekina.

\par 4 Car la splendeur de la Shekina a parcouru le monde d'un bout à l'autre (avec une splendeur) 365 000 fois (celle) du globe du soleil. Et quiconque profitait de la splendeur de la Shekina, aucune mouche ni moucheron ne se reposait sur lui, il n'était ni malade ni souffrant. Aucun démon n’avait de pouvoir sur lui, et ils n’étaient pas non plus capables de le blesser.

\par 5 Quand le Saint, béni soit-Il, sortit et entra : du jardin à l'Eden, de l'Eden au jardin, du jardin à Raqia et de Raqia au jardin d'Eden alors tous et chacun virent la splendeur de Sa Shekina et ils n'ont pas été blessés ;

\par 6 jusqu'à l'époque de la génération d'Enosh qui était le chef de tous les adorateurs d'idoles du monde.

\par 7 Et qu'a fait la génération d'Enosh ? Ils allaient d'un bout à l'autre du monde, et chacun apportait de l'argent, de l'or, des pierres précieuses et des perles en tas comme des montagnes et des collines, pour en faire des idoles dans le monde entier. Et ils érigèrent des idoles dans toutes les parties du monde : la taille de chaque idole était de parasanges.

\par 8 Et ils firent descendre le soleil, la lune, les planètes et les constellations, et les placèrent devant les idoles à leur droite et à leur gauche, pour les accompagner comme ils s'occupent du Saint, béni soit-Il, tel qu'il est. écrit : « Et toute l’armée des cieux se tenait près de lui, à sa droite et à sa gauche ».

\par 9 Quelle puissance y avait-il en eux pour qu'ils puissent les faire tomber ? Ils n'auraient pas pu les faire tomber sans 'uzza, 'azza et 'azziel qui leur ont enseigné des sorcelleries par lesquelles ils les ont fait tomber et les ont utilisés.

\par 10 En ce temps-là, les anges au service portèrent des accusations (contre eux) devant le Saint, béni soit-Il, disant devant lui : « Maître du monde ! Qu'as-tu à faire avec les enfants des hommes ? Comme il est écrit « Qu'est-ce que l'homme (Enosh) pour que tu te souviennes de lui ? » « Mah Adam » n'est pas écrit ici, mais « Mah Enosh », car il (Enosh) est le chef des adorateurs d'idoles.

\par 11 Pourquoi as-tu quitté [ADE : le plus haut des cieux élevés, la demeure de ton nom glorieux, et le trône élevé et exalté dans 'Arabotk en haut] [B : les 'Araboth Raqia qui sont pleins de ta gloire, puissant et élevé, et le Trône élevé et exalté dans le 'Araboth Raqia au plus haut] [CL : le plus haut des cieux élevés qui sont remplis de la majesté de ta gloire et sont hauts, élevés et exaltés, et le haut et Trône exalté dans la Raqia 'Araboth en haut] et tu es parti et tu habites avec les enfants des hommes qui adorent les idoles et t'égalent aux idoles.

\par 12 Maintenant tu es sur terre et les idoles aussi. Qu’as-tu à voir avec les habitants de la terre qui adorent des idoles ?

\par 13 Aussitôt le Saint, béni soit-Il, éleva sa Shekina de la terre, du milieu d'eux.

\par 14 À ce moment arrivèrent les anges au service, les troupes des armées et les armées des 'Araboth en mille camps et dix mille armées : ils allèrent chercher des trompettes et prirent les cors dans leurs mains et entourèrent la Shekina de toutes sortes de chants. Et il monta vers les cieux, comme il est écrit : « Dieu est monté au son de la trompette, le Seigneur est monté au son de la trompette ».

\chapter{6}

\par \textit{Enoch s'est élevé au ciel avec la Shekina. Dieu répond aux protestations des anges}

\par 1 R. Ismaël a dit : Métatron, l'Ange, le Prince de la Présence, m'a dit : Quand le Saint, béni soit-Il, désira m'élever en haut, Il envoya d'abord 'Anaphid H (H= Tetrqgrgmmatfm ), le Prince, et il m'a pris du milieu d'eux sous leurs yeux et m'a porté dans une grande gloire sur un char de feu avec des chevaux de feu, serviteurs de gloire. Et il m'a élevé vers les cieux avec la Shekina.

\par 2 Dès que j'ai atteint les cieux élevés, les Saints Chayyoth, les 'Ophannim, les Séraphins, les Kerubim, les Roues de la Merkaba (les Galgallim) et les ministres du feu dévorant, percevant mon odeur de loin de 365 000 myriades de parasanges, a dit : [R : « Quelle odeur d'un être né d'une femme et quel goût de goutte blanche (est-ce) qui monte dans les hauteurs, et (voici, il n'est qu') un moucheron parmi ceux qui « divisent » flammes (de feu) ? »] [B : « Qu'est-ce qu'une personne née d'une femme entre (parmi) nous ? Le goût d'une goutte blanche qui monte vers les cieux pour servir parmi ceux qui « divisent les flammes du feu ».] [CDEL : « Quelle odeur de femme née est-ce et quel goût d'une goutte blanche qui monte vers les cieux » cieux pour servir parmi les diviseurs de flammes.]

\par 3 Le Saint, béni soit-il, leur répondit et leur parla : «Mes serviteurs, mes hôtes, mes Kerubim, mes 'Ophannim, mes Séraphins ! Ne soyez pas mécontents à cause de cela ! Puisque tous les enfants des hommes m'ont renié, moi et mon grand royaume, et qu'ils sont allés adorer des idoles, j'ai enlevé ma Shékina du milieu d'eux et je l'ai élevée en haut. Mais celui que j'ai pris parmi eux est un élu parmi les habitants du monde, et il est égal à eux tous par la foi, la justice et la perfection des actes ; je l'ai pris comme tribut de mon monde sous tous les cieux.»

\chapter{7}

\par \textit{Enoch élevé sur les ailes de la Shekina à la place du Trône, de la Merkaba et des armées angéliques}

\par 1 R. Ismaël a dit : Métatron, l'Ange, le Prince de la Présence, m'a dit : Lorsque le Saint, béni soit-il, m'a enlevé à la génération du déluge, il m'a soulevé sur les ailes du vent de Shekina jusqu'au plus haut des cieux et m'a amené dans les grands palais des 'Araboth Raqia dans les hauteurs, où se trouvent le glorieux trône de la Shékina, la Merkaba, les troupes de la colère, les armées de la véhémence, les Shin'anim enflammés, les Kerubim enflammés, les 'Ophannim brûlants, les serviteurs enflammés, les Chashmallim étincelants et les Séraphins étincelants. Et il m'a placé là pour assister au Trône de Gloire jour après jour.

\chapitre{8}

\par \textit{Les portes (des trésors du ciel) ouvertes à Métatron}

\par 1 R. Ismaël a dit : Métatron, le Prince de la Présence, m'a dit : Avant de me désigner pour assister au Trône de Gloire, le Saint, béni soit-Il, m'a ouvert

\par trois cent mille portes de la Compréhension

\par trois cent mille portes de la subtilité

\par trois cent mille portes de la Vie

\par trois cent mille portes de « grâce et bonté de cœur »

\par trois cent mille portes de l'amour

\par trois cent mille portes de Tora

\par trois cent mille portes de la douceur

\par trois cent mille portes de maintenance

\par trois cent mille portes de la miséricorde

\par trois cent mille portes de la crainte du ciel.

\par 2 En cette heure-là, le Saint, béni soit-Il, a ajouté en moi sagesse à sagesse, compréhension à compréhension, subtilité à subtilité, connaissance à connaissance, miséricorde à miséricorde, instruction à instruction, amour à amour, bonté de cœur à amour. - bonté, bonté en bonté, douceur en douceur, puissance en puissance, force en force, puissance en puissance, éclat en éclat, beauté en beauté, splendeur en splendeur, et j'ai été honoré et orné de toutes ces choses bonnes et louables plus que tous les enfants du ciel.


\chapitre{9}

\par \textit{Enoch reçoit les bénédictions du Très-Haut et se pare d'attributs angéliques}

\par 1 R. Ismaël a dit : Métatron, le Prince de la Présence, m'a dit : Après toutes ces choses, le Saint, béni soit-Il, a posé sa main sur moi et m'a béni de1 bénédictions.

\par 2 Et j'ai été élevé et agrandi à la taille de la longueur et de la largeur du monde.

\par 3 Et Il me fit pousser des ailes de chaque côté. Et chaque aile était comme le monde entier.

\par 4 Et Il fixa sur moi 365 yeux : chaque œil était comme un grand luminaire.

\par 5 Et Il n'a laissé aucune sorte de splendeur, d'éclat, de rayonnement, de beauté dans (de) toutes les lumières de l'univers qu'Il n'ait fixé sur moi.

\chapitre{10}

\par \textit{Dieu place Métatron sur un trône à l'entrée de la Septième Salle et annonce par le Héraut que Métatron est désormais le représentant de Dieu et le chef de tous les princes des royaumes et de tous les enfants des cieux, à l'exception des huit grands princes appelés YHWH par le nom de leur Roi.}

\par 1 R. Ismaël a dit : Métatron, le Prince de la Présence, m'a dit : Toutes ces choses que le Saint, béni soit-Il, a faites pour moi : Il m'a fait un Trône, semblable au Trône de Gloire. Et Il étendit sur moi un rideau de splendeur et d'apparence brillante, de beauté, de grâce et de miséricorde, semblable au rideau du Trône de Gloire ; et dessus étaient fixées toutes sortes de lumières de l’univers.

\par 2 Et Il le plaça à la porte de la Septième Salle et m'y fit asseoir.

\par 3 Et le héraut se rendit dans tous les cieux, disant : Voici Métatron, mon serviteur. J'ai fait de lui un prince et un chef sur tous les princes de mes royaumes et sur tous les enfants du ciel, à l'exception des huit grands princes, ceux qui sont honorés et vénérés, appelés YHWH, du nom de leur roi.

\par 4 Et tout ange et tout prince qui a une parole à dire en ma présence (devant moi) ira en sa présence (devant lui) et lui parlera (à la place).

\par 5 Et vous observerez et exécuterez tout commandement qu'il vous dira en mon nom. Car je lui ai confié le Prince de la Sagesse et le Prince de l'Intelligence pour l'instruire dans la sagesse des choses célestes et des choses terrestres, dans la sagesse de ce monde et du monde à venir.

\par 6 Et je l'ai établi sur tous les trésors des palais d'Araboth et sur tous les trésors de vie qui se trouvent dans les cieux élevés.

\chapitre{11}

\par \textit{Dieu révèle tous les mystères et secrets à Métatron}

\par 1 R. Ismaël dit : Métatron, l'ange, le Prince de la Présence, m'a dit : Désormais le Saint, béni soit-Il, m'a révélé tous les mystères de Tora et tous les secrets de la sagesse et toutes les profondeurs de la Loi Parfaite ; et les pensées du cœur de tous les êtres vivants et tous les secrets de l'univers et tous les secrets de la Création m'ont été révélés tout comme ils sont révélés au Créateur de la Création.

\par 2 Et j'ai regardé attentivement pour contempler les secrets de la profondeur et le merveilleux mystère. [ABL : Avant qu'un homme ne pense en secret, je l'ai vu et avant qu'un homme ne fasse une chose, je l'ai vu.] [C : Avant qu'un homme ne pense, je savais ce qu'il y avait dans sa pensée.]

\par 3 [ABL : Et il n’y avait rien de caché pour moi en haut ni dans les profondeurs du monde.] [C : Et il n’y avait rien de caché pour moi en haut ni en bas dans les profondeurs.]

\chapitre{12}

\par \textit{Dieu habille Métatron d'un vêtement de gloire, met une couronne royale sur sa tête et l'appelle « le Petit YHWH »}

\par 1 R. Ismaël dit : Métatron, le Prince de la Présence, m'a dit : En raison de l'amour avec lequel le Saint, béni soit-Il, m'a aimé plus que tous les enfants du ciel, Il m'a fait un vêtement de gloire sur laquelle étaient fixées toutes sortes de lumières, et Il m'en revêtit.

\par 2 Et Il m'a fait une robe d'honneur sur laquelle étaient fixées toutes sortes de beauté, de splendeur, d'éclat et de majesté.

\par 3 Et il me fit une couronne royale dans laquelle étaient fixées quarante-neuf pierres précieuses semblables à la lumière du globe du soleil.

\par 4 Car sa splendeur s'étendait dans les quatre parties de la 'Araboth Raqia, et dans (à travers) les sept cieux, et dans les quatre parties du monde. Et il me l'a mis sur la tête.

\par 5 Et Il m'a appelé le moindre Yhwh en présence de toute sa maison céleste ; comme il est écrit : « Car mon nom est en lui ».

\chapitre{13}

\par \textit{Dieu écrit avec un style flamboyant sur la couronne de Métatron les lettres cosmiques par lesquelles le ciel et la terre ont été créés}

\par 1 R. Ismaël a dit : Métatron, l'ange, le Prince de la Présence, la Gloire de tous les cieux, m'a dit : En raison du grand amour et de la miséricorde avec lesquels le Saint, béni soit-il, m'a aimé et chéri plus que tous les enfants du ciel, il a écrit avec son doigt, avec un style flamboyant sur la couronne de ma tête, les lettres par lesquelles ont été créés le ciel et la terre, les mers et les rivières, les montagnes et les collines, les planètes et les constellations, les éclairs, les vents, les tremblements de terre et les voix (tonnerres), la neige et la grêle, le vent de tempête et la tempête ; les lettres par lesquelles ont été créés tous les besoins du monde et tous les ordres de la Création.

\par 2 Et chaque lettre a envoyé successivement des éclairs, des torches, des flammes de feu, des rayons semblables au lever du soleil, de la lune et des planètes.

\chapitre{14}

\par \textit{Tous les princes les plus élevés, les anges élémentaires et les anges planétaires et sidériques craignent et tremblent à la vue de Métatron couronné}

\par 1 R. Ismaël a dit : Métatron, l'Ange, le Prince de la Présence, m'a dit : Quand le Saint, béni soit-Il, a mis cette couronne sur ma tête, (alors) tremblèrent devant moi tous les Princes des Royaumes qui sont à la hauteur d'Araboth Raqia et de toutes les armées de tous les cieux ; et même les princes (des) 'Elim, les princes (des) 'Er'ellim et les princes (des) Tafsarim, qui sont plus grands que tous les anges au service qui servent devant le Trône de Gloire, ont été secoués, craints et ils tremblèrent devant moi en me voyant.

\par 2 Même Sammael, le prince des accusateurs, qui est plus grand que tous les princes des royaumes d'en haut, craignait et tremblait devant moi.

\par 3 Et même l'ange de feu, et l'ange de la grêle, et l'ange du vent, et l'ange de la foudre, et l'ange de la colère, et l'ange du tonnerre, et l'ange de la neige, et l'ange de la pluie ; et l'ange du jour, et l'ange de la nuit, et l'ange du soleil et l'ange de la lune et l'ange des planètes et l'ange des constellations qui gouvernent le monde sous leurs mains, furent craints et tremblaient et ont été effrayés devant moi, quand ils m'ont vu.

\par 4 Voici les noms des chefs du monde : Gabriel, l'ange du feu, Baradiel, l'ange de la grêle, Ruchiel qui est préposé au vent, Baraqiel qui est préposé aux éclairs, Za'amiel qui est préposé à la véhémence, Ziqiel qui est préposé aux étincelles, Zi'iel qui est préposé à l'agitation, Za'aphiel qui est préposé au vent de tempête, Ra'amiel qui est préposé aux tonnerres, Ra'ashiel pour le tremblement de terre, Shalgiel pour la neige, Matariel pour la pluie, Shimshiel pour le jour, Lailiel pour la nuit, Galgalliel pour le globe du soleil, Ophanniel pour le globe de la lune, Kokbiel pour les planètes, Rahatiel pour les constellations.

\par 5 Et ils se prosternèrent tous en me voyant. Et ils n’ont pas pu me voir à cause de la gloire majestueuse et de la beauté de l’apparition de la lumière brillante de la couronne de gloire sur ma tête.

\chapitre{15}

\par \textit{Métatron transformé en feu}

\par 1 R. Ismaël a dit : Métatron, l'ange, le Prince de la Présence, la Gloire de tous les cieux, m'a dit : Dès que le Saint, béni soit-Il, m'a pris à (Son) service pour assister au Trône de la Gloire et des Roues (Galgallim) de la Merkaba et des besoins de Shekina, aussitôt ma chair fut transformée en flammes, mes tendons en feu flamboyant, mes os en charbons de genévrier brûlant, la lumière de mes paupières en splendeur d'éclairs , mes globes oculaires en tisons de feu, les cheveux de ma tête en flammes brûlantes, tous mes membres en ailes de feu ardent et tout mon corps en feu rougeoyant.

\par 2 Et à ma droite il y avait des divisions de flammes ardentes, à ma gauche des tisons brûlaient, autour de moi soufflaient des vents de tempête et des tempêtes et devant moi et derrière moi il y avait un grondement de tonnerre avec un tremblement de terre.

\chapitre{15b}

\par \textit{Addition se produisant en B et L}

\par 1 [B : R. Ismaël dit : M'a dit Métatron, le Prince de la Présence et le prince de tous les princes — et il se tient devant ] [L : Métatron, il est prince de tous les princes — et il se tient devant ] Celui qui est plus grand que tous les Elohim. Et il entre sous le Trône de Gloire. Et il a un grand tabernacle de lumière en haut. Et il fait jaillir le feu de la surdité et le met dans les oreilles du Saint Chayyoth, afin qu'ils n'entendent pas la voix de la Parole (Dibbur) qui sort de la bouche de la Divine Majesté.

\par 2 Et quand Moïse monta en haut, il jeûna jusqu'à ce que les habitations du chasmal lui soient ouvertes z ; et il [B : vit le cœur dans le cœur du Lion] [L : vit qu'il était blanc comme le cœur du Lion ] et il vit les innombrables groupes d'armées autour de lui. Et ils voulaient le brûler. Moïse demanda miséricorde, d'abord pour Israël, puis pour lui-même. Et celui qui était assis sur la Merkaba ouvrit les fenêtres qui sont au-dessus des têtes des Kerubim. Et une foule d’avocats – et le prince de la Présence, Métatron, avec eux – allèrent à la rencontre de Moïse. Et ils prirent les prières d'Israël et les mirent comme couronne sur la tête du Saint, béni soit-Il.

\par 3 Et ils dirent : « Écoute, ô Israël ! le Seigneur notre Dieu est un seul Seigneur » [B : et leur visage brillait et se réjouissait sur Shekina ] [L : et le visage de Shekina brillait et se réjouissait ] et ils dirent à Metatron : « Qu'est-ce que c'est ? Et à qui donnent-ils tout cet honneur et cette gloire ? Et ils répondirent : « Au Glorieux Seigneur d’Israël ». Et ils parlèrent : [B : « Écoute, ô Israël : l’Éternel, notre Dieu, est un seul Seigneur. À qui sera donnée l’abondance d’honneur et de majesté sinon à Toi YHWH, la Divine Majesté, le Roi vivant et éternel ». ] [L : « YHWH le Vivant et l'Éternel ». ]

\par 4 À ce moment-là, Akatriel Yah Yehod Sebaoth parla et dit à Métatron, le Prince de la Présence : « Qu'aucune prière qu'il prie devant moi ne revienne (à lui) nulle. Écoute sa prière et exauce son désir, qu'il soit grand ou petit ».

\par 5 Aussitôt Métatron, le Prince de la Présence, dit à Moïse : « Fils d'Amram ! Ne crains rien, car maintenant Dieu prend plaisir en toi. Et demande ton désir de Gloire et de Majesté. Car ton visage brille d’un bout à l’autre du monde ». Mais Moïse lui répondit : « (Je crains) de ne pas m'attirer la culpabilité ». Métatron lui dit : « Reçois les lettres du serment par lequel il n'y a pas de rupture de l'alliance » (ce qui exclut toute rupture de l'alliance),

\chapitre{16}

\par \textit{Métatron a perdu son privilège de présider son propre trône en raison de l'erreur d'Acher de le prendre pour un second pouvoir divin}

\par 1 R. Ismaël a dit : Métatron, l'Ange, le Prince de la Présence, la Gloire de tous les cieux, m'a dit : "Au début, j'étais assis sur un grand Trône à l'entrée de la Septième Salle : Au début, j'étais assis sur un grand Trône à l'entrée de la Septième Salle, et je jugeais les enfants du ciel, la maison d'en haut, par l'autorité du Saint, béni soit-il. Et j'ai partagé la grandeur, la royauté, la dignité, le pouvoir, l'honneur et la louange, le diadème et la couronne de gloire à tous les princes des royaumes, alors que je présidais (lit. assis) dans la Cour céleste (Yeshiba), et les princes des royaumes se tenaient devant moi, à ma droite et à ma gauche - par l'autorité du Saint, béni soit-Il.

\par 2 Mais quand Acher est venu voir la vision de la Merkaba et a fixé ses yeux sur moi, il a eu peur et a tremblé devant moi et son âme a été effrayée jusqu'à s'éloigner de lui, à cause de la peur, de l'horreur et de la crainte de moi, quand il me vit assis sur un trône comme un roi, avec tous les anges au service de moi qui étaient mes serviteurs et tous les princes des royaumes ornés de couronnes qui m'entouraient :

\par 3 À ce moment-là, il ouvrit la bouche et dit : « En effet, il y a deux puissances divines dans le ciel !

\par 4 Immédiatement Bath Qol (la Voix Divine) sortit du ciel devant la Shekina et dit : « Revenez, enfants rétrogrades, sauf Acher !

\par 5 Puis vint 'Aniyel, le Prince, l'honoré, glorifié, bien-aimé, merveilleux, vénéré et craintif, par commission du Saint, béni soit-Il et m'a donné soixante coups de fouets de feu et m'a fait me tenir debout sur mon pieds.

\chapitre{17}

\par \textit{Les princes des sept cieux, du soleil et de la lune, des planètes et des constellations et leurs suites d'anges}

\par 1 R. Ismaël a dit : Métatron, l'ange, le Prince de la Présence, la gloire de tous les cieux, m'a dit : Sept (sont les) princes, les grands, les beaux, les vénérés, les merveilleux et les honorés qui sont nommés sur les sept cieux. Et ce sont eux : [A : MIKAEL, GABRIEL, SHATQIEL, SHACHAQIEL, BAKARIEL, BADARIEL, PACHRIEL.] [D : MIKAEL et GABRIEL, SHATQIEL et BARADIEL et SHACHAQIEL et BARAQIEL et SIDRIEL. ]

\par 2 Et chacun d'eux est le prince de l'armée du (un) ciel. Et chacun d’eux est accompagné de 496 000 myriades d’anges au service.

\par 3 Mikaël, le grand prince, est placé sur le septième ciel, le plus haut, qui est dans les Arabes.

\par Gabriel, le prince de l'armée, est nommé sur le sixième ciel qui est à Makon.

\par Shataqiel, prince de l'armée, est nommé sur le cinquième ciel qui est à Ma'on.

\par Shahaqi'el, prince de l'armée, est établi sur le quatrième ciel qui est à Zebul.

\par Badariel, prince de l'armée, est nommé sur le troisième ciel qui est à Shehaqim.

\par Barakiel, prince de l'armée, est établi sur le deuxième ciel qui est à la hauteur de (Merom) Raqia.

\par Pazriel, prince de l'armée, est établi sur le premier ciel qui est à Wilon, qui est à Shamayim.

\par 4 Sous eux se trouve Galgalliel, le prince qui est établi sur le globe (galgal) du soleil, et avec lui sont des anges grands et honorés qui déplacent le soleil à Raqia.

\par 5 Au-dessous d'eux se trouve 'Ophanniel, le prince qui est placé sur le globe ('ophan) de la lune. Et avec lui sont des anges qui déplacent le globe de la lune de mille parasanges chaque nuit au moment où la lune se trouve à l'Est à son tournant. Et quand la lune se trouve-t-elle à l’Est à son tournant ? Réponse : le quinzième jour de chaque mois.

\par 6 Sous eux se trouve Rahatiel, le prince qui est nommé sur les constellations. Et il est accompagné de 72 anges grands et honorés. Et pourquoi s'appelle-t-il Rahatiel ? Parce qu'il fait courir (marhit) les étoiles sur leurs orbites et fait parcourir 339 mille parasanges chaque nuit de l'Est à l'Ouest et de l'Ouest à l'Est. Car le Saint, béni soit-Il, a fait pour eux tous une tente, pour le soleil, la lune, les planètes et les étoiles dans lesquelles ils voyagent la nuit de l'ouest à l'est.

\par 7 Sous eux se trouve Kokbiel, le prince qui est établi sur toutes les planètes. Et avec lui se trouvent 365 000 myriades d'anges au service, des êtres grands et honorés qui déplacent les planètes de ville en ville et de province en province dans la Raqia des cieux.

\par 8 Et sur eux se trouvent soixante-douze princes des royaumes d'en haut correspondant aux langues du monde. Et tous sont couronnés de couronnes royales, vêtus de vêtements royaux et enveloppés de manteaux royaux. Et tous montaient sur des chevaux royaux et tenaient dans leurs mains des sceptres royaux. Et devant chacun d'eux lorsqu'il voyage à Raqia, les serviteurs royaux courent avec une grande gloire et majesté [A : de même que sur terre ils (les princes) voyagent sur des chars avec des cavaliers et de grandes armées et dans la gloire et la grandeur avec louange, chant et honneur. ] [D : et devant chacun d'eux, lorsqu'ils voyagent à Raqia', il y a de grandes armées qui courent, comme (c'est la coutume) sur terre, avec des chars, dans la gloire et la grandeur, la louange, le chant et l'honneur. ]


\chapitre{18}

\par \textit{L'ordre des rangs des anges et l'hommage reçu par les rangs supérieurs de la part des rangs inférieurs}

\par 1 R. Ismaël a dit : Métatron, l'Ange, le Prince de la Présence, la gloire de tous les cieux, m'a dit : "Les anges du premier ciel, lorsqu'ils voient leur prince, descendent de cheval et tombent sur le visage : Les anges du premier ciel, lorsqu'ils voient leur prince, descendent de cheval et tombent sur leur visage.

\par Et le prince du premier ciel, lorsqu'il aperçut le prince du second ciel, descendit de cheval, ôta de sa tête la couronne de gloire et tomba sur sa face.

\par Et le prince du deuxième ciel, lorsqu'il voit le prince du troisième ciel, ôte de sa tête la couronne de gloire et tombe sur sa face.

\par Et le prince du troisième ciel, lorsqu'il voit le prince du quatrième ciel, ôte de sa tête la couronne de gloire et tombe sur sa face.

\par Et le prince du quatrième ciel, quand il voit le prince du cinquième ciel, il ôte de sa tête la couronne de gloire et tombe sur sa face.

\par Et le prince du cinquième ciel, lorsqu'il aperçut le prince du sixième ciel, ôta de sa tête la couronne de gloire et tomba sur sa face.

\par Et le prince du sixième ciel, quand il voit le prince du septième ciel, il ôte de sa tête la couronne de gloire et tombe sur sa face.

\par 2 Et le prince du septième ciel, quand il voit les soixante-douze princes des royaumes, il ôte de sa tête la couronne de gloire et tombe sur sa face.

\par Et les soixante-douze princes des royaumes, lorsqu'ils aperçurent les portiers de la première salle de l'Araboth Raqia, au plus haut, ôtèrent la couronne royale de leur tête et tombèrent sur leur face.

\par 3 Et les portiers de la première salle, quand ils aperçurent les portiers de la seconde salle, ils enlevèrent la couronne de gloire de leur tête et tombèrent sur leur face.

\par Et les portiers de la deuxième salle, lorsqu'ils aperçurent les portiers de la troisième salle, ôtèrent de leur tête la couronne de gloire et tombèrent sur leur face.

\par Et les portiers de la troisième salle, lorsqu'ils aperçurent les portiers de la quatrième salle, ôtèrent de leur tête la couronne de gloire et tombèrent sur leur face.

\par Et les portiers de la quatrième salle, quand ils aperçurent les portiers de la cinquième salle, ils enlevèrent la couronne de gloire de leur tête et tombèrent sur leur face.

\par Et les portiers de la cinquième salle, lorsqu'ils aperçurent les portiers de la sixième salle, ôtèrent de leur tête la couronne de gloire et tombèrent sur leur face.

\par Et les portiers de la sixième salle, lorsqu'ils aperçurent les portiers de la septième salle, ôtèrent de leur tête la couronne de gloire et tombèrent sur leur face.

\par 4 Et les gardiens de la porte de la septième salle, quand ils voient les quatre grands princes, les honorés, qui sont nommés sur les quatre camps de shekina, ils enlèvent la(les) couronne(s) de gloire de leur tête et tombent sur leur visages.

\par 5 Et les quatre grands princes, lorsqu'ils voient tag' comme prince, grand et honoré par des chants et des louanges, à la tête de tous les enfants du ciel, enlèvent la couronne de gloire de leur tête et tombent sur leur visage.

\par 6 Et Tag'as, le grand et honoré prince, quand il voit Barattiel, le grand prince à trois doigts à la hauteur d'Araboth, le ciel le plus haut, il enlève la couronne de gloire de sa tête et tombe sur sa face. .

\par 7 Et Barattiel, le grand prince, quand il voit Hamon, le grand prince, le redoutable et honoré, agréable et terrible, qui fait trembler tous les enfants du ciel, quand le temps (qui est fixé) pour la parole du (Trois fois) Saint telle qu'elle est écrite : « Au bruit du tumulte (hamon) les peuples s'enfuient ; quand tu t'élèves, les nations sont dispersées » — il enlève la couronne de gloire de sa tête et tombe sur sa face.

\par 8 Et Hamon, le grand prince, quand il voit Tutresiel, le grand prince, il ôte de sa tête la couronne de gloire et tombe sur sa face.

\par 9 Et TutresielIT, le grand prince, quand il voit Atrugiel, le grand prince, il enlève la couronne de gloire de sa tête et tombe sur sa face.

\par 10 Et Atrugiel le grand prince, quand il voit Na'aririel H', le grand prince, il enlève la couronne de gloire de sa tête et tombe sur sa face.

\par 11 Et Na'aririel H', le grand prince, quand il voit Sasnigiel, le grand prince, il enlève la couronne de gloire de sa tête et tombe sur sa face.

\par 12 Et Sasnigiel H', lorsqu'il voit Zazriel H', le grand prince, il enlève la couronne de gloire de sa tête et tombe sur sa face.

\par 13 Et Zazriel H', le prince, quand il voit Geburatiel H', le prince, il enlève la couronne de gloire de sa tête et tombe sur sa face.

\par 14 Et Geburatiel H', le prince, quand il voit 'Araphiel H', le prince, il enlève la couronne de gloire de sa tête et tombe sur sa face.

\par 15 Et 'Araphiel H', le prince, quand il voit 'Ashruylu, le prince, qui préside à toutes les séances des enfants du ciel, il enlève la couronne de gloire de sa tête et tombe sur sa face.

\par 16 Et Ashruylu H', le prince, quand il voit Gallisur H', le Prince, qui révèle tous les secrets de la loi (Tora), il enlève la couronne de gloire de sa tête et tombe sur son visage.

\par 17 Et Gallisur H', le prince, quand il voit Zakzakiel H', le prince chargé d'écrire les mérites d'Israël sur le trône de gloire, il enlève la couronne de gloire de sa tête et tombe sur sa face. .

\par 18 Et Zakzakiel H', le grand prince, quand il voit 'Anaph(i)el H', le prince qui garde les clés des salles célestes, il enlève la couronne de gloire de sa tête et tombe sur sa face. Pourquoi est-il appelé Anaphiel ? Parce que le rameau de son honneur et de sa majesté et de sa couronne et de sa splendeur et de son éclat couvre (éclipse) toutes les chambres d'Araboth Raqia d'en haut, tout comme le Créateur du monde (les éclipse). De même qu'il est écrit à propos du Créateur du monde : « Sa gloire couvrait les cieux et la terre était pleine de ses louanges », de même l'honneur et la majesté d'Anaphiel couvrent toutes les gloires d'Araboth le plus haut. .

\par 19 Et quand il voit Sother 'Ashiel H', le prince, le grand, craintif et honoré, il enlève la couronne de gloire de sa tête et tombe sur sa face. Pourquoi s'appelle-t-il Sother Ashiel ? Parce qu'il est établi sur les quatre sources du fleuve de feu, en face du trône de gloire ; et tout prince qui sort ou entre avant la Shekina ne sort ou n'entre qu'avec sa permission. Car les sceaux du fleuve ardent lui sont confiés. Et de plus, sa taille est de 7 000 myriades de parasanges. Et il attise le feu du fleuve ; et il sort et entre devant la Shekina pour expliquer ce qui est écrit (enregistré) concernant les habitants du monde. Selon comme il est écrit : « le jugement fut prononcé et les livres furent ouverts ».

\par 20 Et Sother 'Ashiel le prince, quand il voit Shoqed Chozi, le grand prince, le puissant, terrible et honoré, il enlève la couronne de gloire de sa tête et tombe sur son visage. Et pourquoi s'appelle-t-il Shoqed Chozi ? Parce qu'il pèse tous les mérites (de l'homme) dans une balance en présence du Saint, béni soit-Il.

\par 21 Et quand il voit Zehanpuryu H', le grand prince, le puissant et terrible, honoré, glorifié et craint dans toute la maison céleste, il enlève la couronne de gloire de sa tête et tombe sur sa face. Pourquoi s'appelle-t-il Zehanpuryu ? Parce qu'il repousse le fleuve de feu et le repousse à sa place.

\par 22 Et quand il voit 'Azbuga H', le grand prince, glorifié, vénéré, honoré, orné, merveilleux, exalté, aimé et redouté parmi tous les grands princes qui connaissent le mystère du Trône de Gloire, il enlève la couronne de la gloire descend de sa tête et tombe sur sa face. Pourquoi s'appelle-t-il 'Azbuga ? Parce qu'à l'avenir, il ceindra (habillera) les justes et les pieux du monde des vêtements de la vie et les enveloppera du manteau de la vie, afin qu'ils puissent y vivre une vie éternelle.

\par 23 Et quand il voit les deux grands princes, les forts et les glorifiés qui se tiennent au-dessus de lui, il ôte de sa tête la couronne de gloire et tombe sur sa face. Et voici les noms des deux princes :

\par Sopheriel H' (qui) tue, (Sopheriel H' le Tueur), le grand prince, l'honoré, glorifié, irréprochable, vénérable, ancien et puissant ; (et) Sopheriel H' (qui) rend vivant (Sopheriel H' le Donateur de vie), le grand prince, l'honoré, glorifié, irréprochable, ancien et puissant.

\par 24 Pourquoi s'appelle Sopheriel H' qui tue (Sopheriel H' le Tueur) ? Parce qu'il est chargé des livres des morts : [afin que] chacun, lorsque le jour de sa mort approche, l'écrive dans les livres des morts.

\par Pourquoi s'appelle-t-il Sopheriel H' qui donne la vie (Sopheriel H' le Donateur de vie) ? Parce qu'il est chargé des livres des vivants (de la vie), afin que quiconque que le Saint, béni soit-Il, fasse revivre, il l'écrit dans le livre des vivants (de la vie), par autorité de MAQOM. Vous pourriez peut-être dire : « Puisque le Saint, béni soit-Il, est assis sur un trône, eux aussi sont assis lorsqu'ils écrivent ». (Réponse) : L'Écriture nous enseigne : « Et toute l'armée des cieux se tient à ses côtés ». «L'armée des cieux» (dit-on) afin de nous montrer que même les Grands Princes, qui ne ressemblent pas à ceux qui existent dans les cieux élevés, n'accomplissent pas les demandes de la Shekina autrement que de se tenir debout. Mais comment est-il (possible) qu’ils (savent) écrire lorsqu’ils sont debout ? C'est comme ça:

\par 25 L'un se tient sur les roues de la tempête, et l'autre se tient sur les roues du vent de tempête.

\par L'un est vêtu d'habits royaux, l'autre est vêtu d'habits royaux.

\par L'un est enveloppé d'un manteau de majesté et l'autre est enveloppé d'un manteau de majesté.

\par L'un est couronné d'une couronne royale, et l'autre est couronné d'une couronne royale.

\par Le corps de l'un est plein d'yeux, et le corps de l'autre est plein d'yeux.

\par L'apparence de l'un est semblable à l'apparence des éclairs, et l'apparence de l'autre est semblable à l'apparence des éclairs.

\par Les yeux de l'un sont comme le soleil dans sa puissance, et les yeux de l'autre sont comme le soleil dans sa puissance.

\par La hauteur de l'un est comme la hauteur des sept cieux, et la hauteur de l'autre est comme la hauteur des sept cieux.

\par Les ailes de l'un représentent autant de jours de l'année, et les ailes de l'autre représentent autant de jours de l'année.

\par Les ailes de l'un s'étendent sur la largeur de Raqict, et les ailes de l'autre s'étendent sur la largeur de Raqia !.

\par Les lèvres de l'un sont comme les portes de l'Orient, et les lèvres de l'autre sont comme les portes de l'Orient.

\par La langue de l'un est aussi haute que les vagues de la mer, et la langue de l'autre est aussi haute que les vagues de la mer.

\par De la bouche de l'un sort une flamme, et de la bouche de l'autre sort une flamme.

\par De la bouche de l'un sortent des éclairs, et de la bouche de l'autre sortent des éclairs.

\par De la sueur d'un feu s'allume, et de la sueur de l'autre le feu s'allume.

\par De la langue de l'un brûle une torche, et de la langue de l'autre une torche brûle.

\par Sur la tête de l'un il y a une pierre de saphir, et sur la tête de l'autre il y a une pierre de saphir.

\par Sur les épaules de l'un il y a une roue d'un chérubin rapide, et sur les épaules de l'autre il y a une roue d'un chérubin rapide.

\par L'un a dans la main un parchemin brûlant, l'autre a dans la main un parchemin brûlant.

\par L'un a dans la main un style flamboyant, l'autre a dans la main un style flamboyant.

\par La longueur du parchemin est une myriade de parasanges ; la taille du style est de 3000 myriades de parasanges ; la taille de chaque lettre qu’ils écrivent est de 365 parasanges.

\chapitre{19}

\par \textit{Rikbiel, le prince des roues de la Merkaba. Les environs du Merkaba. L'agitation parmi les armées angéliques au moment de Qedushsha}

\par 1 R. Ismaël dit : Métatron, l'Ange, le Prince de la Présence, m'a dit : Au-dessus de ces trois anges, ces grands princes, il y a un Prince, distingué, honoré, noble, glorifié, orné, craintif, vaillant, fort, grand, magnifié, glorieux, couronné, merveilleux, exalté, irréprochable, bien-aimé, seigneurial, haut et élevé, ancien et puissant, comme nul parmi les princes. Son nom est Rikbiel H', le grand et vénéré prince qui se tient près de la Merkaba.

\par 2 Et pourquoi s'appelle-t-il rikbiel ? Parce qu'il est nommé sur les roues de la Merkaba, et qu'elles sont confiées à sa charge.

\par 3 Et combien y a-t-il de roues ? Huit; deux dans chaque direction. Et il y a quatre vents qui les entourent tout autour. Et voici leurs noms : « le Vent de Tempête », « la Tempête », « le Vent Fort » et « le Vent de Tremblement de Terre ».

\par 4 Au-dessous d'eux, quatre fleuves de feu coulent continuellement, un fleuve de feu de chaque côté. Autour d'eux, entre les fleuves, quatre nuages sont plantés, et ce sont des «nuages de feu», des «nuages de lampes», des «nuages de charbon», des «nuages de soufre», et ils se dressent contre [leurs] roues.

\par 5 Et les pieds du Chayyoth reposent sur les roues. Et entre une roue et l’autre le tremblement de terre gronde et le tonnerre gronde.

\par 6 Et quand l'heure de réciter le chant approche, (alors) les multitudes de roues sont déplacées, la multitude de nuages ​​tremble, tous les chefs (shallishim) ont peur, tous les cavaliers (parashim) font rage , tous les puissants (gtbborim) sont excités, toutes les armées (seba'im) sont effrayées, toutes les troupes (gedudim) ont peur, tous les nommés (memunnim) se précipitent, tous les princes (sarim) et armées (chayyelim) sont consternés, tous les serviteurs (mesharetim) s'évanouissent et tous les anges (maVakim) et divisions (degalim) travaillent de douleur.

\par 7 Et une roue fait un bruit qui se fait entendre à l'autre et un Kerubin à un autre, un Chayya à un autre, un Séraphin à un autre (en disant) «Exaltez celui qui chevauche à Araboth, par son nom Jah et réjouissez-vous devant lui!»

\chapitre{20}

\par \textit{Chayyliel, le prince des Chayyoth}

\par 1 R. Ismaël dit : Métatron, l'ange, le Prince de la Présence, m'a dit : Au-dessus de ceux-ci, il y a un grand et puissant prince. Son nom est Chayyliel H', un prince noble et vénéré, un prince glorieux et puissant, un prince grand et vénéré, un prince devant lequel tremblent tous les enfants du ciel, un prince capable d'engloutir la terre entière en un instant ( d'une bouchée).

\par 2 Et pourquoi s'appelle-t-il Chayyliel H' ? Parce qu'il est nommé sur le Saint Chayyoth et qu'il frappe les Chayyoth avec des coups de feu ; de chez lui ! (c'est-à-dire le Qedushsha).

\chapitre{21}

\par \textit{Les Chayyoth}

\par 1 R. Ismaël dit : Métatron, l'ange, le Prince de la Présence, m'a dit : Quatre (sont) les Chayyoth correspondant aux quatre vents. Chaque Chayya est comme l’espace du monde entier. Et chacun a quatre visages ; et chaque visage est comme le visage de l'Orient.

\par 2 Chacun a quatre ailes et chaque aile est comme la couverture (toit) de l'univers.

\par 3 Et chacun a des faces au milieu des faces et des ailes au milieu des ailes. La taille des visages est (comme la taille des) visages, et la taille des ailes est (comme la taille des) ailes.

\par 4 Et chacun est couronné de couronnes sur la tête. Et chaque couronne est comme un arc dans la nuée. Et sa splendeur est semblable à la splendeur du globe solaire. Et les étincelles qui jaillissent de chacun sont comme la splendeur de l’étoile du matin (planète Vénus) à l’Est.

\chapitre{22}

\par \textit{Kérubiel, le prince des Kérubim. Description des Kérubim}

\par 1 R. Ismaël a dit : Métatron, l'ange, le Prince de la Présence, m'a dit : Au-dessus de ceux-ci, il y a un prince, noble, merveilleux, fort et loué de toutes sortes de louanges. Son nom est Kerubiel H', un prince puissant, plein de pouvoir et de force.

\par [AD : un prince d'altesse, et l'altesse (est) avec lui, un prince juste, et la justice (est) avec lui, un prince saint, et la sainteté (est) avec lui, un prince] [B : un prince de grandeur, et avec lui (il y a) un prince juste, de justice, et avec lui un prince saint, de sainteté, et avec lui (il y a) un prince ] glorifié dans (par) mille armées, exalté par dix mille armées .

\par 2 À sa colère la terre tremble, à sa colère les camps sont ébranlés, de peur de lui les fondations sont ébranlées, à sa menace les Arabes tremblent.

\par 3 Sa stature est pleine de charbons (ardents). La hauteur de sa stature est comme la hauteur des sept cieux, la largeur de sa stature est comme la largeur des sept cieux et l'épaisseur de sa stature est comme les sept cieux.

\par 4 L'ouverture de sa bouche est comme une lampe de feu. Sa langue est un feu dévorant. Ses sourcils sont comme la splendeur de l'éclair. Ses yeux sont comme des étincelles de brillance. Son visage est comme un feu brûlant.

\par 5 Et il y a une couronne de sainteté sur sa tête sur laquelle (couronne) le Nom Explicite est gravé, et des éclairs en sortent. Et l'arc de Shekina est entre ses épaules.

\par 6 [AD : Et son épée est sur ses reins et ses flèches sont comme des éclairs dans sa ceinture. Et un bouclier de feu dévorant (est) sur son cou et des charbons de genévrier sont tout autour de lui.] [B : Et son épée est comme un éclair ; et sur ses reins il y a des flèches semblables à une flamme, et sur son armure et son bouclier il y a un feu dévorant, et sur son cou il y a des charbons de genévrier ardent et (aussi) autour de lui (il y a des charbons de genévrier ardent). ]

\par 7 Et la splendeur de Shekina est sur son visage ; et les cornes de majesté sur ses roues ; et un diadème royal sur son crâne.

\par 8 Et son corps est plein d'yeux. Et les ailes couvrent toute sa haute stature (lit. la hauteur de sa stature est constituée de toutes les ailes).

\par 9 A sa droite une flamme brûle, et à sa gauche un feu rougeoie ; et des charbons en brûlent. Et des tisons sortent de son corps. Et des éclairs jaillissent de son visage. Avec lui, il y a toujours tonnerre sur tonnerre, à ses côtés il y a toujours tremblement de terre sur tremblement de terre.

\par 10 Et les deux princes de la Merkaba sont avec lui.

\par 11 Pourquoi s'appelle-t-il Kerubiel H', le Prince. Parce qu'il est nommé sur le char des Kerubim. Et les puissants Kerubim sont confiés à sa charge. Et il orne les couronnes de leurs têtes et polit le diadème de leur crâne.

\par 12 Il magnifie la gloire de leur apparence. Et il glorifie la beauté de leur majesté. Et il augmente la grandeur de leur honneur. Il fait chanter le chant de leur louange. Il intensifie leur belle force. Il fait resplendir l’éclat de leur gloire. Il embellit leur miséricorde et leur bonté de cœur. Il cadre la justesse de leur rayonnement. Il rend leur beauté miséricordieuse encore plus belle. Il glorifie leur droite majesté. Il exalte l'ordre de leur louange, pour établir la demeure de celui « qui habite sur les Kerubim ».

\par 13 Et les Kerubim se tiennent près du Saint Chayyoth, et leurs ailes sont levées jusqu'à leur tête (lit. sont comme la hauteur de leur tête)
\par et Shekina est (repose) sur eux
\par et l'éclat de la Gloire est sur leurs visages
\par et chant et louange dans leur bouche
\par et leurs mains sont sous leurs ailes
\par et leurs pieds sont couverts par leurs ailes
\par et des cornes de gloire sont sur leurs têtes
\par et la splendeur de Shekina sur leur visage
\par et Shekina est (repose) sur eux
\par et des pierres de saphir sont autour d'eux
\par et des colonnes de feu sur leurs quatre côtés
\par et des colonnes de brandons à côté d'eux.

\par 14 Il y a un saphir d'un côté et un autre saphir de l'autre côté, et sous les saphirs il y a des charbons de genévrier ardent.

\par 15 Et un Kerubin se tient dans chaque direction, mais les ailes des Kerubim s'entourent au-dessus de leurs crânes en gloire ; et ils les étendirent pour chanter avec eux un chant à celui qui habite les nuées et pour louer avec eux la redoutable majesté du roi des rois.

\par 16 Et Kerubiel H', le prince qui est établi sur eux, il les dispose dans des ordres avenants, beaux et agréables et il les exalte dans toutes sortes d'exaltation, de dignité et de gloire. Et il les pousse – avec gloire et puissance – à faire à chaque instant la volonté de leur Créateur. Car au-dessus de leurs têtes élevées demeure continuellement la gloire du grand roi « qui habite sur les Kerubim ».

\chapitre{22b}

\par 1 [L(mr), suivant le rec. du ch. XXIIe s. vs.1-3 (milieu) : Et il y a un parvis devant le Trône de Gloire,] [B : R. Ismaël m'a dit : Métatron, l'ange, le Prince de la Présence, m'a dit : Comment vont les anges debout en haut ? Il dit : Comme un pont placé sur une rivière pour que chacun puisse y passer, de même un pont est placé depuis le début de l'entrée jusqu'à la fin.]

\par 2 [Lmr dans lequel aucun séraphin ni ange ne peut entrer, et ce sont 6 000 myriades de parasanges, comme il est écrit : « et les Séraphins se tiennent au-dessus de lui » (le dernier mot du passage scripturaire étant [valeur numérique : 36] ).] [B: Et trois anges au service l'entourent et poussent un chant devant YHWH, le Dieu d'Israël. Et se tiennent devant lui des seigneurs de frayeur et des capitaines de frayeur, au nombre de mille et dix mille dix mille, et ils chantent des louanges et des hymnes devant YHWH, le Dieu d'Israël. ]

\par 3 [Lmr : Comme la valeur numérique de (36) est le nombre de ponts qui s'y trouvent. ] [B : Il y a de nombreux ponts : des ponts de feu et de nombreux ponts de grêle. Et aussi de nombreuses rivières de grêle, de nombreux trésors de neige et de nombreuses roues de feu. ]

\par 4 [Lmr : Et il y a des myriades de roues de feu. Et les anges au service sont au nombre de 12 000 myriades. Et il y a 12 000 rivières de grêle et 12 000 trésors de neige. Et dans les sept salles se trouvent des chars de feu et de flammes, sans compter, ni fin, ni recherche. (Lm. se termine ici.) ] [B : Et combien sont les anges au service ? 12 000 myriades : six (mille myriades) en haut et six (mille myriades) en bas. Et 12 000 sont les trésors de neige, six en haut et six en bas. Et des myriades de roues de feu, 12 (myriades) en haut et 12 (myriades) en bas. Et ils entourent les ponts, les fleuves de feu et les fleuves de grêle. Et il y a de nombreux anges au service, formant des entrées, pour toutes les créatures qui se tiennent au milieu d'elles, correspondant aux chemins de RaqtaShamayim. ]

\par 5 Que fait YHWH, le Dieu d'Israël, le Roi de Gloire ? Le Dieu grand et craintif, puissant en force, couvre son visage.

\par 6 À Araboth se trouvent 660 000 myriades d'anges de gloire debout devant le trône de gloire et les divisions de feu flamboyant. Et le Roi de Gloire se couvre le visage ; car sinon l'Araboth Raqia' serait déchiré en son sein à cause de la majesté, de la splendeur, de la beauté, du rayonnement, de la beauté, de l'éclat, de l'éclat et de l'excellence de l'apparence (du Saint), béni soit-Il.

\par 7 Il y a de nombreux anges au service qui accomplissent sa volonté, de nombreux rois, de nombreux princes dans l'Arabot de ses délices, des anges qui sont vénérés parmi les dirigeants du ciel, distingués, ornés de chants et rappelant l'amour : (qui) sont effrayés par la splendeur de la Shekina, et leurs yeux sont éblouis par la beauté éclatante de leur roi, leurs visages deviennent noirs et leur force s'épuise.

\par 8 Des fleuves de joie, des fleuves d'allégresse, des fleuves de réjouissance, des fleuves de triomphe, des fleuves d'amour, des fleuves d'amitié - (autre lecture : ) de l'agitation - et ils débordent et s'avancent devant le Trône de Gloire et deviennent grands et traversent les portes des chemins de 'Araboth Raqia à la voix des cris et de la musique du Chayyoth, à la voix de l'allégresse des tambours de ses 'Ophannim et à la mélodie des cymbales de ses Kerubim. Ils s'agrandissent et s'agitent au son de l'hymne : «Saint, saint, saint ! Le Seigneur des armées est saint, saint, saint ; toute la terre est remplie de sa gloire.»

\chapitre{22c}

\par \textit{(en B, Lo et Lmr)}

\par 1 Ismaël dit : Métatron y le Prince de la Présence m'a dit : Quelle est la distance entre un pont et un autre ? 12 myriades de parasanges. Leur ascension est constituée de myriades de parasanges et leur descente de 12 myriades de parasanges.

\par 2 (La distance) entre les fleuves d'effroi et les fleuves de peur est de 22 myriades de parasanges ; entre les fleuves de grêle et les fleuves de ténèbres, 36 myriades de parasanges ; entre les chambres des éclairs et les nuages ​​de compassion 42 myriades de parasanges ; entre les nuages ​​de compassion et la Merkaba 148 myriades de parasanges ; entre la Merkaba et les Kerubim, 148 myriades de parasanges ; entre les Kerubim et les 'Ophannim, 24 myriades de parasanges ; entre les Ophannim et les chambres des chambres, 24 myriades de parasanges ; entre les chambres des chambres et le Saint Chayyoth, 40 000 myriades de parasanges ; entre un whig (du Chayyoth) et 12 autres myriades de parasanges ; et la largeur de chaque aile est de la même mesure ; et la distance entre le Saint Chayyoth et le Trône de Gloire est de 30 000 myriades de parasangs.

\par 3 Et depuis le pied du trône jusqu'au siège, il y a 40 000 myriades de parasanges. Et le nom de Celui qui est assis dessus : que le nom soit sanctifié !

\par 4 [Et les arcs de l'arc sont placés au-dessus des 'Araboth, et ils sont hauts de 1 000 mille et 10 000 fois dix mille (de parasanges). Leur mesure est après la mesure des Irin et des Qaddishin (les Veilleurs et les Saints). Comme il est écrit « Mon arc, je l'ai placé dans la nuée ». Il n'est pas écrit ici « je fixerai » mais « j'ai fixé », (c'est-à-dire) déjà ; nuages ​​qui entourent le Trône de Gloire. Alors que ses nuages ​​passent, les anges de la grêle (se transforment en) charbon ardent.

\par 5 Et un feu de voix descend du Saint Chayyoth. Et à cause du souffle de cette voix, ils « courent » vers un autre endroit, craignant qu'elle ne leur ordonne d'y aller ; et ils « reviennent » de peur que cela ne les blesse de l’autre côté. Donc « ils courent et reviennent ».

\par 6 Et ces arcs de l'Arc sont plus beaux et plus radieux que l'éclat du soleil pendant le solstice d'été. Et ils sont plus blancs qu’un feu flamboyant et ils sont grands et beaux.

\par 7 Au-dessus des arcs de l'Arc sont les roues des 'Ophannim. Leur hauteur est de 1 000 mille 10 000 fois 10 000 unités de mesure après la mesure des Séraphins et des Troupes (Gedudim).]

\chapitre{23}

\par \textit{Les vents soufflant 'sous les ailes des Kerubim'}

\par 1 R. Ismaël dit : Métatron, l'Ange, le Prince de la Présence, m'a dit : Il y a de nombreux vents qui soufflent sous les ailes des Kérubim. Là souffle « le vent couvant », comme il est écrit : « et le vent de Dieu couvait sur la face des eaux ».

\par 2 Là souffle « le vent fort », comme il est dit : « et l'Éternel fit reculer la mer par un fort vent d'est toute la nuit ».

\par 3 Là souffle « le vent d'Est » comme il est écrit : « le vent d'Est a amené les sauterelles ».

\par 4 Là souffle « le vent des cailles » comme il est écrit : « Et il sortit un vent de la part du Seigneur et apporta des cailles ».

\par 5 Là souffle « le vent de la jalousie » comme il est écrit : « Et le vent de la jalousie tomba sur lui ».

\par 6 Là souffle le « Vent de tremblement de terre » comme il est écrit : « et après cela le vent de tremblement de terre ; mais le Seigneur n'était pas dans le tremblement de terre ».

\par 7 Là souffle le « Vent de H' » comme il est écrit : « et il m'emporta par le vent de H' et me déposa ».

\par 8 Là souffle le « vent mauvais » comme il est écrit : « et le vent mauvais s'éloigna de lui ».

\par 9 Là soufflent le « Vent de Sagesse » et le « Vent de Compréhension » et le « Vent de Connaissance » et le « Vent de la Peur de H' » comme il est écrit : « Et le vent de H' reposera sur lui ;le vent de la sagesse et de la compréhension, le vent du conseil et de la puissance, le vent de la connaissance et de la peur”.

\par 10 Là souffle le « Vent de Pluie », comme il est écrit : « le vent du nord produit la pluie ».

\par 11 Là souffle le « Vent des éclairs », comme il est écrit : « il fait des éclairs pour la pluie et fait sortir le vent de ses trésors ».

\par 12 Là souffle le « Vent qui brise les rochers », comme il est écrit : « l'Eternel passa et un vent grand et fort (déchira les montagnes et brisa les rochers devant l'Eternel) ».

\par 13 Là souffle le « Vent d'apaisement de la mer », comme il est écrit : « et Dieu fit passer un vent sur la terre, et les eaux apaisent ».

\par 14 Là souffle le « Vent de colère », comme il est écrit : « et voici, un grand vent vint du désert et frappa les quatre coins de la maison et elle tomba ».

\par 15 Là souffle le « Vent de Tempête », comme il est écrit : « Vent de Tempête, accomplissant sa parole ».

\par 16 Et Satan se tient parmi ces vents, car « vent de tempête » n'est rien d'autre que « Satan », et tous ces vents ne soufflent que sous les ailes des Kérubim, comme il est écrit : « et il chevaucha sur un chérubin et il a volé, oui, et il a volé rapidement sur les ailes du vent ».

\par 17 Et où vont tous ces vents ? L'Écriture nous enseigne qu'ils sortent de dessous les ailes des Kerubim et descendent sur le globe du soleil, comme il est écrit : « Le vent va vers le midi et tourne vers le nord ; il tourne continuellement dans sa course et le vent revient dans ses circuits ». Et du globe du soleil, ils reviennent et descendent sur [les fleuves et les mers, sur] les montagnes et sur les collines, comme il est écrit : « Car voici, celui qui a formé les montagnes et créé le vent ».

\par 18 Et des montagnes et des collines ils reviennent et descendent vers les mers et les fleuves ; et des mers et des fleuves ils reviennent et descendent sur (les) villes et provinces ; et des villes et des provinces ils reviennent et descendent dans le Jardin, et du Jardin ils reviennent et descendent à l'Eden, comme il est écrit : « marchant dans le Jardin au vent du jour ». Et au milieu du Jardin, ils se réunissent et soufflent d'un côté à l'autre et sont parfumés des épices du Jardin même depuis ses parties les plus reculées, jusqu'à ce qu'ils se séparent les uns des autres, et, remplis du parfum des épices pures. , ils apportent l'odeur des parties les plus reculées de l'Eden et les épices du Jardin aux justes et aux pieux qui dans les temps à venir hériteront du Jardin d'Eden et de l'Arbre de Vie, comme il est écrit : « Réveille-toi, ô nord vent; et viens vers le sud ; souffle sur mon jardin, pour que ses épices en coulent. Que mon bien-aimé vienne dans son jardin et mange ses précieux fruits ».

\chapitre{24}

\par \textit{Les différents chars du Saint, béni soit-Il}

\par 1 R. Ismaël a dit : Métatron, l'Ange, le Prince de la Présence, la gloire de tous les cieux, m'a dit : De nombreux chars ont le Saint, béni soit-Il : Il a les « Chars de (les) Kerubim », comme il est écrit : « Et il monta sur un chérubin et vola ».

\par 2 Il a les « Chariots du Vent », comme il est écrit (ib.) : « et il vola rapidement sur les ailes du vent

\par 3 Il a les « Chariots de (la) Nuée Rapide », comme il est écrit : « Voici, le Seigneur chevauche sur une nuée rapide ».

\par 4 Il a « les Chariots des Nuées », comme il est écrit : « Voici, je viens à toi dans une nuée ».

\par 5 Il a les « Chars de l'Autel », comme il est écrit : « J'ai vu le Seigneur debout sur l'Autel ».

\par 6 Il a les « Chars de Ribbotaïm », comme il est écrit : « Les chars de Dieu sont Ribbotaïm ; des milliers d'anges ».

\par 7 Il a les « Chars de la Tente », comme il est écrit : « Et l'Éternel apparut dans la Tente dans une colonne de nuée ».

\par 8 Il a les « Chars du Tabernacle », comme il est écrit : « Et l'Éternel lui parla hors du tabernacle ».

\par 9 Il a les « Chars du propitiatoire », comme il est écrit : « alors il entendit la Voix qui lui parlait depuis le propitiatoire ».

\par 10 Il avait les « Chariots de pierre de saphir », comme il est écrit : « et il y avait sous ses pieds comme un ouvrage pavé de pierre de saphir ».

\par 11 Il a les « Chariots des Aigles », comme il est écrit : « Je vous ai portés sur des ailes d'aigle ». Littéralement, il ne s’agit pas ici de « ceux qui volent rapidement comme des aigles ».

\par 12Il a les « chars du Cri », comme il est écrit : « Dieu est monté au cri ».

\par 13 Il a les « Chars des 'Araboth », comme il est écrit : « Exaltez Celui qui monte sur les 'Araboth ».

\par 14 Il a les « Chariots des Nuées épaisses », comme il est écrit : « qui fait des nuées épaisses Son char ».

\par 15 Il a les « Chars des Chayyoth » comme il est écrit : « et les Chayyoth coururent et revinrent ». Ils courent avec autorisation et reviennent avec autorisation, car Shekina est au-dessus de leurs têtes.

\par 16 Il a les « Chariots à Roues (Galgallim) », comme il est écrit : « Et il dit : Passez entre les roues qui tournent ».

\par 17 Il a les « Chars d'un Kerubin rapide », comme il est écrit (?) : « chevauchant un chérubin rapide ».

\par Et lorsqu'il chevauche un Kerubin rapide, lorsqu'il pose un de ses pieds sur lui, avant de poser l'autre pied sur son dos, il regarde d'un seul coup d'œil à travers dix-huit mille mondes. Et il discerne et voit en chacun d’eux et sait ce qu’il y a en chacun d’eux – et alors il pose l’autre pied sur lui, selon qu’il est écrit : « Environ dix-huit mille ».

\par D'où savons-nous qu'Il examine chacun d'eux chaque jour ? Il est écrit : « Il regarda du ciel les enfants des hommes pour voir s'il y en avait qui comprenaient, qui cherchaient Dieu ».

\par 18 Il a les « chars des Ophannim », comme il est écrit : « et les Ophannim étaient pleins d'yeux tout autour ».

\par 19 Il a les « Chars de Son Saint Trône », comme il est écrit : « Dieu est assis sur son saint trône ».

\par 20 Il a les « chars du Trône de Yah », comme il est écrit : « Parce qu'une main est levée sur le Trône de Jah ».

\par 21 Il a les « Chars du Trône du Jugement », comme il est écrit : « mais l'Éternel des armées sera exalté en jugement ».

\par 22 Il a les « Chariots du Trône de Gloire », comme il est écrit : « Le Trône de Gloire, élevé dès le commencement, est le lieu de notre sanctuaire ».

\par 23 Il a les « Chars du Trône Haut et Exalté », comme il est écrit : « J'ai vu le Seigneur assis sur le trône haut et exalté ».

\chapitre{25}

\par \textit{'Ophphanniel y le prince des 'Ophannirn. Description des 'Ophannim}

\par 1 R. Ismaël a dit : Métatron, l'Ange, le Prince de la Présence, m'a dit : Au-dessus de ceux-ci il y a un grand prince, vénéré, haut, seigneurial, craintif, ancien et fort, 'ophphanniel H' est son nom .

\par 2 Il a seize faces, quatre faces de chaque côté, (aussi) cent ailes de chaque côté. Et il a des yeux correspondant aux jours de l'année. [A : 2190 — et certains disent 2116 — de chaque côté.] [DE : 2191 (E:196) et seize de chaque côté. ]

\par 3 Et dans ces deux yeux de sa face, dans chacun d'eux des éclairs jaillissent, et de chacun d'eux brûlent des tisons ; et aucune créature ne peut les voir, car quiconque les regarde est brûlé à l'instant.

\par 4 Sa taille est (comme) la distance d'années de voyage. Aucun œil ne peut voir et aucune bouche ne peut dire la puissance de sa force, sauf le Roi des rois, le Saint, béni soit-Il, seul.

\par 5 Pourquoi s'appelle-t-il 'Ophphanniel ?

\par Parce qu'il est nommé sur les 'Ophannim et que les 'Ophannim sont confiés à sa charge. Il se tient chaque jour et les soigne et les embellit. Et il exalte et ordonne leur appartement (DE : runnings) et polit leur position debout et illumine leurs habitations, aplanit leurs coins et nettoie leurs sièges. Et il les attend tôt et tard, de jour comme de nuit, pour accroître leur beauté, pour accroître leur dignité et pour les rendre diligents à la louange de leur Créateur.

\par 6 Et tous les 'Ophannim sont pleins d'yeux, et ils sont tous pleins d'éclat; soixante-douze pierres de saphir sont fixées sur leurs vêtements à leur côté droit et soixante-douze pierres de saphir sont fixées sur leurs vêtements à leur côté gauche.

\par 7 Et quatre pierres d'anthrax sont fixées sur la couronne de chacun, dont la splendeur se propage dans les quatre directions d'Araboth, tout comme la splendeur du globe du soleil se propage dans toutes les directions de l'univers. Et pourquoi est-ce que ça s'appelle Carbuncle (Bareqet) ? Parce que sa splendeur est comme l’apparition d’un éclair (Baraq). Et des tentes de splendeur, des tentes d'éclat, des tentes d'éclat comme le saphir et l'escarboucle les enferment à cause de l'aspect brillant de leurs yeux.

\chapitre{26}

\par \textit{Séraphiel, le Prince des Séraphins. Description des Séraphins}

\par 1 R. Ismaël a dit : Métatron, l'Ange, le Prince de la Présence, m'a dit : Au-dessus de ceux-ci, il y a un prince, merveilleux, noble, grand, honorable, puissant, terrible, un chef et un chef x et un rapide scribe, glorifié, honoré et aimé.

\par 2 Il est tout à fait rempli de splendeur, plein de louange et brillant ; et il est tout entier plein d'éclat, de lumière et de beauté ; et tout en lui est rempli de bonté et de grandeur.

\par 3 Son visage est tout à fait semblable à celui des anges, mais son corps est semblable à celui d'un aigle.

\par 4 Sa splendeur est comme des éclairs, son apparence comme des tisons de feu, sa beauté comme des étincelles, son honneur comme des charbons ardents, sa majesté comme des chaskmals, son éclat comme la lumière de la planète Vénus. Son image est semblable à la Grande Lumière. Sa hauteur est comme les sept cieux. La lumière de ses sourcils est comme la lumière septuple.

\par 5 La pierre de saphir sur sa tête est aussi grande que l'univers entier et semblable à la splendeur des cieux mêmes en éclat.

\par 6 Son corps est plein d'yeux comme les étoiles du ciel, innombrables et insondables. Chaque œil est comme la planète Vénus. Pourtant, il y en a certains qui ressemblent à la Petite Lumière et d’autres qui ressemblent à la Grande Lumière. De ses chevilles à ses genoux (ils sont) comme des étoiles d'éclair, de ses genoux à ses cuisses comme la planète Vénus, de ses cuisses à ses reins comme la lune, de ses reins à son cou comme le soleil, de son cou contre son crâne comme la Lumière Impérissable.

\par 7 La couronne sur sa tête est semblable à la splendeur du trône de gloire. La mesure de la couronne est la distance des années de voyage. Il n’y a aucune sorte de splendeur, aucune sorte d’éclat, aucune sorte de rayonnement, aucune sorte de lumière dans l’univers qui ne soit fixée sur cette couronne.

\par 8 Le nom de ce prince est Séraphiel H'. Et la couronne sur sa tête, son nom est « le Prince de la Paix ». Et pourquoi est-il appelé du nom de séraphiel H' ? Parce qu'il est nommé sur les Séraphins. Et les Séraphins enflammés sont confiés à sa charge. Et il les préside jour et nuit et leur enseigne le chant, la louange, la proclamation de la beauté, de la puissance et de la majesté ; afin qu'ils puissent proclamer la beauté de leur roi par toutes sortes de louanges et de sanctifications (Qedushsha).

\par 9 Combien sont les Séraphins ? Quatre, correspondant aux quatre vents du monde. Et combien d’ailes ont-ils chacun ? Six, correspondant aux six jours de la Création. Et combien ont-ils de visages ? Chacun d’eux a quatre visages.

\par 10 La mesure des Séraphins et la hauteur de chacun d'eux correspondent à la hauteur des sept cieux. La taille de chaque aile est comme la mesure de toute Raqia. La taille de chaque visage est semblable à celle du visage de l’Orient.

\par 11 Et chacun d'eux dégage une lumière semblable à la splendeur du Trône de Gloire : de sorte que même le Saint Chayyoth, les honorés 'Ophannim, ni les majestueux Kerubim ne peuvent le voir. Tous ceux qui le contemplent ont les yeux assombris à cause de sa grande splendeur.

\par 12 Pourquoi sont-ils appelés Séraphins ? Parce qu'ils brûlent (saraph) les tables à écrire de Satan : Chaque jour, Satan est assis avec Sammael, le prince de Rome, et avec Dubbiel, prince de Perse, et ils écrivent les iniquités d'Israël sur des tables à écrire qu'ils remettent aux Séraphins, afin qu'ils les présentent devant le Saint, béni soit-Il, afin qu'il détruise Israël du monde. Mais les Séraphins savent, par les secrets du Saint, béni soit-Il, qu'il ne désire pas que ce peuple Israël périsse. Que font les Séraphins ? Chaque jour, ils les reçoivent (les acceptent) de la main de Satan et les brûlent dans le feu ardent contre le Trône élevé et exalté, afin qu'ils ne puissent pas se présenter devant le Saint, béni soit-Il, au moment où il est assis sur le trône du jugement, jugeant le monde entier en vérité.



\chapitre{27}

\par \textit{Radweriel, le gardien du Livre des Records}

\par 1 R. Ismaël dit : Métatron, l'Ange x de H', le Prince de la Présence, m'a dit : Au dessus des Séraphins il y a un prince, exalté au-dessus de tous les princes, plus merveilleux que tous les serviteurs. Son nom est Radweriel H', qui est nommé au trésor des livres.

\par 2 Il récupère le dossier des écrits (avec) le livre des archives dedans et l'apporte devant le Saint, béni soit-Il. Et il brise les sceaux de l'étui, l'ouvre, en sort les livres et les remet devant le Saint, béni soit-Il. Et le Saint, béni soit-Il, les reçoit de sa main et les remet sous ses yeux aux scribes, afin qu'ils les lisent dans la grande Beth Dinin, sur les hauteurs d'Araboth Raqia, devant la maison céleste.

\par 3 Et pourquoi s'appelle-t-il Radweriel ? Parce que de chaque parole qui sort de sa bouche, un ange est créé ; et il se tient dans les chants (dans la compagnie qui chante) des anges au service et pousse un chant devant le Saint, béni soit-Il, quand le temps approche pour la récitation du (Trois fois) Saint.

\chapitre{28}

\par \textit{Les 'Irin et Qaddishin}

\par 1 R. Ismaël a dit : Métatron, l'Ange, le Prince de la Présence, m'a dit : Au-dessus de tous ceux-là, il y a quatre grands princes, Irin et Qaddishin de nom : hauts, honorés, vénérés, bien-aimés, merveilleux et glorieux. , plus grand que tous les enfants du ciel. Il n’y en a aucun comme eux parmi tous les princes célestes et aucun n’est égal parmi tous les Serviteurs. Car chacun d’eux est égal à tous les autres ensemble.

\par 2 Et leur demeure est face au trône de gloire, et leur place est face au Saint, béni soit-il, de sorte que l'éclat de leur demeure est un reflet de l'éclat du trône de gloire. Et la splendeur de leur visage est le reflet de la splendeur de Shekina.

\par 3 Et ils sont glorifiés par la gloire de la Majesté Divine (Gebura) et loués par (à travers) la louange de Shekina.

\par 4 Et non seulement cela, mais le Saint, béni soit-Il, ne fait rien dans son monde sans les consulter d'abord, mais après cela il le fait. Comme il est écrit : « La sentence est par le décret du 'Irin et la demande par la parole du Qaddishin »

\par 5 Les 'Irin sont deux et les Qaddishin sont deux. Et comment se tiennent-ils devant le Saint, béni soit-Il ? Il faut comprendre qu'un 'Ir se tient d'un côté et l'autre 'Ir de l'autre côté, et qu'un Qaddish se tient d'un côté et l'autre de l'autre côté.

\par 6 Et toujours ils élèvent les humbles, et ils abaissent jusqu'à terre ceux qui sont orgueilleux, et ils élèvent jusqu'aux hauteurs ceux qui sont humbles.

\par 7 Et chaque jour, comme le Saint, béni soit-Il, est assis sur le trône du jugement et juge le monde entier, et que les livres des vivants et les livres des morts sont ouverts devant lui, alors tous les enfants du ciel se tiennent devant lui dans la crainte, l'effroi, la crainte et le tremblement. En ce temps-là, (quand) le Saint, béni soit-Il, est assis sur le trône du jugement pour exécuter le jugement, son vêtement est blanc comme la neige, les cheveux de sa tête comme de la laine pure et tout son manteau est comme le brillant lumière. Et il est partout couvert de justice, comme d'une cotte de mailles.

\par 8 Et ces 'Irin et Qaddishin se tiennent devant lui comme des officiers de justice devant le juge. Et ils soulèvent et débattent chaque cas et clôturent le cas qui se présente devant le Saint, béni soit-Il, en jugement, selon qu'il est écrit : « La sentence est par le décret de l'Irin et la demande par la parole du 'Irin. Qaddishin »

\par 9 Certains d'entre eux argumentent et d'autres prononcent la sentence dans le Grand Beth Din à 'Araboth. Certains d'entre eux font les requêtes devant la Divine Majesté et d'autres clôturent les dossiers devant le Très-Haut. D'autres finissent par descendre et (confirmant =) exécuter les phrases sur terre ci-dessous. Selon comme il est écrit : « Voici, un 'Ir et un Qaddish descendirent du ciel et crièrent à haute voix et dirent ainsi : Abattez l'arbre, coupez ses branches, secouez ses feuilles et dispersez ses fruits : que les bêtes s'éloignent de dessous, et les oiseaux de ses branches ».

\par 10 Pourquoi sont-ils appelés 'Irin et Qaddishin ? C'est pourquoi ils sanctifient le corps et l'esprit avec des coups de feu le troisième jour du jugement, comme il est écrit : « Au bout de deux jours il nous ressuscitera ; le troisième il nous relèvera, et nous vivrons avant lui.»

\chapitre{29}

\par \textit{Description d'une classe d'anges}

\par 1 R. Ismaël dit : Métatron, l'Ange, le Prince de la Présence, m'a dit : Chacun des thetaa nas soixante-dix noms correspondant aux soixante-dix langues du monde. Et tous sont basés sur le nom du Saint, béni soit-Il. Et chaque nom est écrit avec un style flamboyant sur la couronne effrayante (Kether Nora) qui est sur la tête du roi haut et exalté.

\par 2 Et de chacun d'eux sortent des étincelles et des éclairs. Et chacun d’eux est entouré de cornes de splendeur. Des lumières brillent de chacun, et chacun est entouré de tentes brillantes, de sorte que même les Séraphins et les Chayyoth, qui sont plus grands que tous les enfants du ciel, ne peuvent pas les voir.

\chapitre{30}

\par \textit{Les princes des Royaumes et le Prince du Monde officiant au Grand Sanhédrin au ciel}

\par 1 R. Ismaël a dit : Métatron, l'Ange, le Prince de la Présence, m'a dit : Chaque fois que le Grand Beth Din est assis dans l'Araboth Raqia en haut, il n'y a d'ouverture de bouche pour personne dans le monde sauf ces grands princes qui sont appelés H'f du nom du Saint, béni soit-Il.

\par 2 Combien sont ces princes ? Soixante-douze princes des royaumes du monde, outre le Prince du monde, qui parle (plaie) en faveur du monde devant le Saint, béni soit-Il, chaque jour, à l'heure où s'ouvre le livre dans lequel sont enregistrés toutes les actions du monde, selon qu'il est écrit : « Le jugement fut prononcé et les livres furent ouverts. »

\chapitre{31}

\par \textit{(Les attributs de) Justice, Miséricorde et Vérité par le Trône du Jugement}

\par 1 R. Ismaël a dit : Métatron, l'Ange, le Prince de la Présence, m'a dit : Au moment où le Saint, béni soit-Il, est assis sur le Trône du Jugement, (alors) la Justice est debout à sa droite et la miséricorde à sa gauche et la vérité devant sa face.

\par 2 Et quand l'homme entre devant Lui pour le jugement, (alors) la splendeur de la Miséricorde sort vers lui comme (c'était) un bâton et se tient devant lui. Aussitôt l'homme tombe sur sa face, (et) tous les anges de destruction craignent et tremblent devant lui, selon qu'il est écrit : « Et avec miséricorde le trône sera affermi, et il s'assiéra dessus en vérité. »

\chapitre{32}

\par \textit{L'exécution du jugement sur les méchants. L'épée de Dieu}

\par 1 R. Ismaël a dit : Métatron, l'Ange, le Prince de la Présence, m'a dit : Quand le Saint, béni soit-Il, ouvre le Livre dont la moitié est feu et moitié flamme, (alors) ils sortent de devant Lui à chaque instant pour exécuter le jugement sur les méchants par Son épée (c'est-à-dire) sortie de son fourreau et dont la splendeur brille comme un éclair et imprègne le monde d'un bout à l'autre, comme il est écrit : «Car c'est par le feu que l'Éternel plaidera (et par son épée avec toute chair).»

\par 2 Et tous les habitants du monde (lit. ceux qui viennent au monde) craignent et tremblent devant Lui, quand ils voient Son épée aiguisée comme un éclair d'un bout du monde à l'autre, et des étincelles et des éclairs de la taille des étoiles de Raqict en sortent ; selon qu'il est écrit : « Si j'aiguise l'éclair de mon épée ».


\chapitre{33}

\par \textit{Les anges de la Miséricorde, de la Paix et de la Destruction près du Trône du Jugement. Les scribes, (v. i,) Les anges près du trône de gloire et les fleuves de feu en dessous. (vs.-5)}

\par 1 R. Ismaël dit : Métatron, l'Ange, le Prince de la Présence, m'a dit : Au moment où le Saint, béni soit-Il, est assis sur le Trône du Jugement,(alors) les anges de la Miséricorde se tiennent à sa droite, les anges de la Paix se tiennent à sa gauche et les anges de la Destruction se tiennent devant Lui.

\par 2 Et un scribe se tient au-dessous de lui, et un autre scribe au-dessus de lui.

\par 3 Et les glorieux Séraphins [A : entourez-les comme des tisons de feu autour du Trône de Gloire.] [E : entourez le Trône sur ses quatre côtés de murs d'éclairs, et les 'Ophannim les entourent de tisons de feu tout autour du Trône de gloire. ] Et des nuées de feu et des nuées de flammes les entourent à droite et à gauche ; et les Saints Chayyoth portent le Trône de Gloire d'en bas : chacun avec trois doigts. La mesure des doigts de chacun est de 800 000 fois cent, (et) 66 000 parasanges.

\par 4 Et sous les pieds des Chayyoth sept rivières de feu coulent et coulent. Et la largeur de chaque rivière est de mille parasanges, et sa profondeur est de mille myriades de parasanges. Sa longueur est insondable et incommensurable.

\par 5 Et chaque rivière tourne en arc dans les quatre directions de 'Araboth Raqia, et (de là) elle descend jusqu'à Mot on et est arrêtée (?), et de Ma'on à Zebul y de Zebul à Shechaqim , de Shechaqim à Raqia, de Raqia à Shamayim et de Shamayim sur la tête des méchants qui sont dans la Géhenne, comme il est écrit : « Voici, un tourbillon de l'Éternel, même sa fureur, est parti, oui, une tempête tourbillonnante ; il éclatera sur la tête des méchants ».

\chapitre{34}

\par \textit{Les différents cercles concentriques autour du Chayyoth, constitués de feu, d'eau, de grêle, etc. et des anges prononçant le responsorium Qedushsha}

\par 1 R. Ismaël a dit : Métatron ; l'Ange, le Prince de la Présence, me dit : Les sabots du Chayyoth sont entourés de sept nuages ​​de charbons ardents. Les nuages ​​de charbons ardents sont entourés à l’extérieur de sept murs de flamme(s). Les sept murs de flamme(s) sont entourés extérieurement par sept murs de grêlons (pierres de 'El-gabishy). Les grêlons sont entourés à l'extérieur de pierres de grêle (pierre de Bar ad). Les pierres de grêle sont entourées à l'extérieur par des pierres des « ailes de la tempête ». Les pierres des « ailes de la tempête » sont entourées à l’extérieur de flammes de feu. Les flammes du feu sont entourées des chambres du tourbillon. Les chambres du tourbillon sont entourées à l’extérieur par le feu et l’eau.

\par 2 Autour du feu et de l'eau sont ceux qui prononcent le « Saint ». Autour de ceux qui prononcent le « Saint », il y a ceux qui prononcent le « Bienheureux ». Autour de ceux qui prononcent le « Bienheureux », se trouvent des nuages ​​brillants. Les nuages ​​brillants sont entourés à l'extérieur de charbons de genévrier brûlants ; et à l'extérieur, entourant les cQals de genévrier brûlant, il y a mille camps de feu et dix mille armées de flammes. Et entre chaque camp et chaque armée, il y a une nuée, afin qu'ils ne soient pas brûlés par le feu.

\chapitre{35}

\par \textit{Les camps d'anges dans 'Araboth Raqia' : anges exécutant la Qedushsha}

\par 1 R. Ismaël dit : Métatron, l'Ange, le Prince de la Présence, m'a dit : 506 mille myriades de camps ont le Saint, béni soit-Il, dans les hauteurs d'Araboth Raqia. Et chaque camp est (composé de) 496 mille anges.

\par 2 Et chaque ange, la hauteur de sa stature est comme la grande mer ; et l'apparence de leur visage était comme l'apparence d'un éclair, et leurs yeux étaient comme des lampes de feu, et leurs bras et leurs pieds avaient la couleur de l'airain poli, et la voix rugissante de leurs paroles était comme la voix d'une multitude.

\par 3 Et ils se tiennent tous debout devant le trône de gloire sur quatre rangées. Et les princes de l'armée se tiennent à la tête de chaque rang.

\par 4 Et certains d'entre eux prononcent le « Saint » et d'autres prononcent le « Bienheureux », certains d'entre eux courent comme messagers, d'autres se tiennent debout, selon qu'il est écrit : « Des milliers de milliers le servaient, et dix mille fois dix mille se tenaient devant lui : le jugement était prononcé et les livres étaient ouverts ».

\par 5 Et à l'heure où le temps de dire le «Saint» approche, (alors) d'abord un tourbillon sort de devant le Saint, béni soit-Il, et fait irruption sur le camp de Shekina et là se lève une grande agitation parmi eux, comme il est écrit : « Voici, le tourbillon du Seigneur sort avec fureur, une agitation continue ».

\par 6 À ce moment-là, des milliers de milliers d'entre eux sont changés en étincelles, des milliers de milliers en brandons, des milliers de milliers en éclairs, des milliers de milliers en flammes, des milliers de milliers en mâles, des milliers de milliers en femelles, des milliers de milliers en vents, des milliers de milliers en des feux brûlants, des milliers de milliers en flammes, des milliers de milliers en étincelles, des milliers de milliers en chashmals de lumière ; jusqu'à ce qu'ils prennent sur eux le joug du royaume des cieux, le plus haut et le plus élevé, du Créateur de tous, avec crainte, effroi, crainte et tremblement, avec agitation, angoisse, terreur et appréhension. Ensuite, ils reprennent leur forme antérieure pour avoir toujours la crainte de leur Roi devant eux, car ils ont résolu de réciter continuellement le Cantique, comme il est écrit : « Et l’un cria à l’autre et dit (Saint, Saint, Saint, etc.) ».


\chapitre{36}

\par \textit{Les anges se baignent dans la rivière ardente avant de réciter le 'Chanson'}

\par 1 R. Ismaël a dit : Métatron, l'Ange, le Prince de la Présence, m'a dit : Au moment où les anges au service désirent réciter (le) Cantique, (alors) Nehar di-Nur (le courant ardent) s'élève avec plusieurs « milliers de milliers et myriades de myriades » (d'anges) de puissance et de force de feu et il court et passe sous le Trône de Gloire, entre les camps des anges au service et les troupes d'Araboth.

\par 2 Et tous les anges au service descendent d'abord dans Nehar di-Nur, et ils se plongent dans le feu et trempent leur langue et leur bouche sept fois ; et après cela, ils montent et revêtent le vêtement de 'Machaqe Samal' et se couvrent de manteaux de chasmal et se tiennent sur quatre rangées face au trône de gloire, dans tous les cieux.


\chapitre{37}

\par \textit{Les quatre camps de Shekina et leurs environs}

\par 1 R. Ismaël a dit : Métatron, l'Ange, le Prince de la Présence, m'a dit : Dans les sept Salles se tiennent quatre chars de Shekina, et devant chacun se tiennent les quatre camps de Shekina. Entre chaque camp, une rivière de feu coule continuellement.

\par 2 Entre chaque rivière il y a des nuages ​​brillants [qui les entourent], et entre chaque nuage sont dressées des colonnes de soufre. Entre un pilier et un autre se trouvent des roues flamboyantes qui les entourent. Et entre une roue et une autre, il y a des flammes de feu tout autour. Entre une flamme et une autre, il y a des trésors d'éclairs ; derrière les trésors d'éclairs se trouvent les ailes du vent de tempête. Derrière les ailes du vent de tempête se trouvent les chambres de la tempête ; derrière les chambres de la tempête il y a des vents, des voix, des tonnerres, des étincelles sur étincelles et des tremblements de terre sur tremblements de terre.


\chapitre{38}

\par \textit{La peur qui s'abat sur tous les cieux au son du «Saint», esp. les corps célestes. Ceux-ci apaisés par le Prince du Monde}

\par 1 R. Ismaël a dit : Métatron, l'Ange, le Prince de la Présence, m'a dit : Au moment où les anges au service prononcent (les Trois Fois) Saints, alors tous les piliers des cieux et leurs bases tremblent. , et les portes des salles de 'Araboth Raqia sont ébranlées et les fondations de Shechaqim et de l'Univers (Tebel) sont déplacées, et les ordres de Ma'on et les chambres de Makon frémissent, et tous les ordres de Raqia et les constellations et les planètes sont consternées, et les globes du soleil et de la lune se précipitent et fuient hors de leur course et parcourent 12 000 parasanges et cherchent à se jeter du ciel,

\par 2 à cause de la voix rugissante de leur chant, et du bruit de leurs louanges, et des étincelles et des éclairs qui sortent de leurs visages ; comme il est écrit : « La voix de ton tonnerre était dans le ciel (les éclairs éclairaient le monde, la terre tremblait et tremblait) ».

\par 3 Jusqu'à ce que le prince du monde les appelle et leur dise : « Soyez tranquilles à votre place ! Ne craignez rien à cause des anges au service qui chantent le Cantique devant le Saint, béni soit-Il ». Comme il est écrit : « Quand les étoiles du matin chantaient ensemble et que tous les enfants du ciel criaient de joie ».



\chapitre{39}

\par \textit{Les noms explicites s'envolent du Trône et toutes les différentes armées angéliques se prosternent devant lui au moment de Qedushsha}

\par 1 R. Ismaël a dit : Métatron, l'Ange, le Prince de la Présence, m'a dit : Lorsque les anges au service prononcent le « Saint », alors tous les noms explicites gravés avec un style flamboyant sur le Trône de Gloire s'envolent comme des aigles, avec seize ailes. Et ils entourent et entourent le Saint, béni soit-Il, des quatre côtés de la place de Sa Shekina.

\par 2 Et les anges de l'armée, et les serviteurs enflammés, et les puissants 'Ophanmm, et les Kerubim de la Shekina, et les saints Chayyoth, et les Séraphins, et les 'Er'ellim, et les Taphsarim et les troupes de le feu dévorant, et les armées de feu, et les armées enflammées, et les saints princes, ornés de couronnes, vêtus de majesté royale, enveloppés de gloire, ceints d'élévation, tombent trois fois sur leurs faces, en disant : « Béni soit le nom de Son royaume glorieux pour toujours et à jamais ».


\chapitre{40}

\par \textit{Les anges au service sont récompensés par des couronnes lorsqu'ils prononcent le ii Holyy dans son bon ordre, et punis par un feu consumant sinon. De nouveaux créés à la place des anges consumés}

\par 1 R. Ismaël a dit : Métatron, l'Ange, le Prince de la Présence, m'a dit : Quand les anges au service disent « Saint » devant le Saint, béni soit-Il, de la manière appropriée, alors les serviteurs de Son Trône, les serviteurs de Sa Gloire sortent avec une grande joie de sous le Trône de Gloire.

\par 2 Et ils portent tous dans leurs mains, chacun d'eux mille mille dix mille fois dix mille couronnes d'étoiles, semblables en apparence à la planète Vénus, et les mettent sur les anges ministériels et les grands princes qui prononcent le « Saint». Ils mettent sur chacun d'eux trois couronnes : une couronne parce qu'ils disent «Saint», une autre couronne parce qu'ils disent «Saint, Saint», et une troisième couronne parce qu'ils disent «Saint, Saint, Saint est le Seigneur des Armées». »,

\par 3 Et au moment où ils ne prononcent pas le « Saint » dans le bon ordre, un feu dévorant sort du petit doigt du Saint, béni soit-Il, et tombe au milieu de leurs rangs et est divisé en mille parties correspondant aux quatre camps des anges au service, et les consume en un instant, comme il est écrit : « Un feu marche devant lui et brûle ses adversaires tout autour ».

\par 4 Après cela, le Saint, béni soit-Il, ouvre la bouche et prononce une parole et en crée d'autres à leur place, de nouvelles comme elles. Et chacun se tient devant son Trône de Gloire, prononçant le « Saint », comme il est écrit (Lam. iii.3) : « Ils sont nouveaux chaque matin ; grande est ta fidélité ».


\chapitre{41}

\par \textit{Métatron montre à R. Ismaël les lettres gravées sur le Trône de Gloire par lesquelles tout a été créé dans le ciel et sur la terre}

\par 1 R. Ismaël dit : Métatron, l'Ange, le Prince de la Présence, m'a dit : Viens et contemple les lettres par lesquelles le ciel et la terre ont été créés,
\par les lettres par lesquelles furent créées les montagnes et les collines,
\par les lettres par lesquelles furent créées les mers et les fleuves,
\par les lettres par lesquelles furent créés les arbres et les herbes,
\par les lettres par lesquelles furent créées les planètes et les constellations,
\par les lettres par lesquelles furent créés le globe de la lune et le globe du soleil, Orion, les Pléiades et tous les différents luminaires de Raqia.

\par 2 les lettres par lesquelles furent créées le Trône de Gloire et les Roues de la Merkaba, les lettres par lesquelles furent créées les nécessités des mondes,

\par 3 les lettres par lesquelles ont été créées la sagesse, l'intelligence, la connaissance, la prudence, la douceur et la justice par lesquelles le monde entier est soutenu.

\par 4 Et j'ai marché à ses côtés et il m'a pris par la main et m'a élevé sur ses ailes et m'a montré ces lettres, toutes, qui sont gravées avec un style flamboyant sur le Trône de Gloire : et des étincelles en sortent et couvrez toutes les chambres d'Araboth.


\chapitre{42}

\par \textit{Instances d'opposés polaires maintenus en équilibre par plusieurs Noms Divins et autres merveilles similaires}

\par 1 R. Ismaël dit : Métatron, l'Ange, le Prince de la Présence, m'a dit : Viens et je te montrerai, où les eaux sont suspendues au plus haut, où le feu brûle au milieu de la grêle, où des éclairs jaillissent du milieu des montagnes enneigées, où les tonnerres grondent dans les hauteurs célestes, où une flamme brûle au milieu du feu brûlant et où des voix se font entendre au milieu du tonnerre et du tremblement de terre.

\par 2 Alors je suis allé à ses côtés et il m'a pris par la main et m'a soulevé sur ses ailes et m'a montré toutes ces choses. J'ai vu les eaux suspendues en haut à 'Araboth Raqia par (la force du) nom YAH 'EH YE 'ASHER 'EH YE (Jah, je suis celui que je suis), et leurs fruits descendaient du ciel et arrosaient la face du monde, comme il est écrit : « (Il arrose les montagnes de ses chambres :) la terre est rassasiée du fruit de ton œuvre ».

\par 3 Et je vis du feu, de la neige et de la grêle qui étaient mêlés les uns aux autres et pourtant intacts, par (la force du) nom 'ESH 'OKELA (feu dévorant), comme il est écrit : « Pour l'Éternel, ton Dieu est un feu dévorant ».

\par 4 Et j'ai vu des éclairs qui jaillissaient des montagnes de neige et pourtant n'étaient pas endommagés (éteints), par (la force du) nom YAH SUR 'OLAMIM (Jah, le rocher éternel), comme il est écrit : « Car en Jah, YHWH, le rocher éternel ».

\par 5 Et je vis des tonnerres et des voix qui rugissaient au milieu des flammes ardentes et qui n'étaient pas endommagées (réduites au silence), par (la force du) nom 'EL-SHADDAI RABBA (le Grand Dieu Tout-Puissant) comme il est écrit : « Je suis Dieu Tout-Puissant ».

\par 6 Et j'ai vu une flamme (et) une lueur (flammes rougeoyantes) qui flambaient et brillaient au milieu d'un feu brûlant, et pourtant n'étaient pas endommagées (dévorées), par (la force du) nom YAD AL KES YAH ( la main sur le Trône du Seigneur) comme il est écrit : « Et il dit : car la main est sur le Trône du Seigneur ».

\par 7 Et je vis des fleuves de feu au milieu des fleuves d'eau et ils ne furent pas endommagés (éteints) par (la force du) nom 'OSE SHALOM (artisan de la paix) comme il est écrit : « Il fait la paix dans son hauts lieux». Car il fait la paix entre le feu et l'eau, entre la grêle et le feu, entre le vent et la nuée, entre le tremblement de terre et les étincelles.

\chapitre{43}

\par \textit{Métatron montre à R. Ismaël la demeure des esprits à naître et des esprits des justes morts}

\par 1 R. Ismaël dit : Métatron m'a dit : Viens et je te montrerai où sont les esprits des justes qui ont été créés et qui sont revenus, et les esprits des justes qui n'ont pas encore été créés.

\par 2 Et il m'a soulevé à ses côtés, m'a pris par la main et m'a élevé près du Trône de Gloire à la place de la Shekina ; et il me révéla le Trône de Gloire, et il me montra les esprits qui avaient été créés et qui étaient revenus : et ils volaient au-dessus du Trône de Gloire devant le Saint, béni soit-Il.

\par 3 Après cela je suis allé interpréter le verset suivant de l'Écriture et j'ai trouvé dans ce qui est écrit : « car l'esprit s'est revêtu devant moi, et les âmes j'ai fait » que (« car l'esprit s'est revêtu devant moi ») désigne les esprits qui ont été créés dans la chambre de création des justes et qui sont revenus devant le Saint, béni soit-Il ; (et les mots :) et les âmes que j'ai créées font référence aux esprits des justes qui n'ont pas encore été créés dans la chambre (GUPH).


\chapitre{44}

\par \textit{Métatron montre à R. Ismaël la demeure des méchants et des intermédiaires dans le Shéol (vs.-6) Les Patriarches prient pour la délivrance d'Israël (vs. 7-10)}

\par 1 R. Ismaël dit : Métatron, l'Ange, le Prince de la Présence, m'a dit : Viens et je te montrerai les esprits des méchants et les esprits des intermédiaires là où ils se tiennent, et les esprits des intermédiaires, où ils descendent, et les esprits des méchants, où ils descendent.

\par 2 Et il me dit : Les esprits des méchants descendent au She'ol par les mains de deux anges de destruction : za'aphiel et simkiel sont leurs noms.

\par 3 Simkiel est désigné sur l'intermédiaire pour les soutenir et les purifier à cause de la grande miséricorde du Prince du Lieu (Maqom). Za'aphiel est désigné sur les esprits des méchants afin de les chasser de la présence du Saint, béni soit-Il, et de la splendeur de la Shekina jusqu'au She'ol, pour être punis dans le feu de la Géhenne avec des bâtons de charbon brûlant.

\par 4 Et je suis allé à ses côtés, et il m'a pris par la main et me les a tous montrés avec ses doigts.

\par 5 Et je vis l'apparence de leurs visages (et voici, c'était) comme l'apparence des enfants des hommes, et leurs corps comme des aigles. Et non seulement cela, mais (en outre) la couleur du visage de l'intermédiaire était comme du gris pâle à cause de leurs actes, car il y a des taches sur eux jusqu'à ce qu'ils soient purifiés de leur iniquité dans le feu.

\par 6 Et la couleur des méchants était comme le fond d'un pot, à cause de la méchanceté de leurs actes.

\par 7 Et j'ai vu les esprits des patriarches Abraham Isaac et Jacob et du reste des justes qu'ils ont fait sortir de leurs tombeaux et qui sont montés au ciel (Raqia). Et ils priaient devant le Saint, béni soit-Il, disant dans leur prière : « Seigneur de l'Univers ! Combien de temps vas-tu t'asseoir sur (ton) trône comme un pleureur dans les jours de son deuil avec ta main droite derrière toi et ne pas délivrer tes enfants et révéler ton royaume dans le monde ? Et pendant combien de temps n'auras-tu aucune pitié pour tes enfants qui sont fait esclave parmi les nations du monde ? Et sur ta main droite qui est derrière toi, avec laquelle tu as étendu les cieux et la terre et les cieux des cieux ? Quand auras-tu compassion ?

\par 8 Alors le Saint, béni soit-Il, répondit à chacun d'eux, disant : « Puisque ces méchants pèchent ainsi et ainsi, et transgressent par telles et telles transgressions contre moi, comment pourrais-je livrer ma grande main droite dans le chute par leurs mains (causée par eux).

\par 9 À ce moment-là, Métatron m'a appelé et m'a dit : « Mon serviteur ! Prenez les livres et lisez leurs mauvaises actions ! Immédiatement, j'ai pris les livres et j'ai lu leurs actes et il y avait 36 ​​transgressions (écrites) à l'égard de chaque méchant et en plus, ils ont transgressé toutes les lettres de la Tora, comme il est écrit : « Oui, tout Israël j’ai transgressé ta loi ». Il n'est pas écrit 'al torateka mais 'et torateka, car ils ont transgressé de y Aleph à Taw, ils ont transgressé 40 statuts pour chaque lettre.

\par 10 Aussitôt Abraham, Isaac et Jacob pleurèrent. Alors le Saint, béni soit-Il, leur dit : « Abraham, mon bien-aimé, Isaac, mon élu, Jacob, mon premier-né ! Comment puis-je maintenant les délivrer du milieu des nations de le monde?» Et aussitôt Mikaël, le prince d'Israël, cria et pleura d'une voix forte et dit : « Pourquoi te tiens-tu loin, ô Seigneur ? »


\chapitre{45}

\par \textit{Metatron montre les événements passés et futurs de R. Ismaël enregistrés sur le rideau du trône}

\par 1 R. Ismaël a dit : Métatron m'a dit : Viens, et je te montrerai le Rideau de MAQOM (la Divine Majesté) qui est déployé devant le Saint, béni soit-Il, (et) sur lequel sont gravées toutes les générations du monde et de toutes leurs actions, de ce qu'ils ont fait et de ce qu'ils feront jusqu'à la fin de toutes les générations.

\par 2 Et j'y suis allé, et il me l'a montré en le montrant avec ses doigts comme un père qui enseigne à ses enfants les lettres de Tora. Et j'ai vu chaque génération, les dirigeants de chaque génération,
\par et les chefs de chaque génération,
\par les bergers de chaque génération,
\par les oppresseurs (conducteurs) de chaque génération,
\par les gardiens de chaque génération,
\par les fléaux de chaque génération,
\par les surveillants de chaque génération,
\par les juges de chaque génération,
\par les huissiers de justice de chaque génération,
\par les enseignants de chaque génération,
\par les supporters de chaque génération,
\par les chefs de chaque génération,
\par les présidents d'académies de chaque génération,
\par les magistrats de chaque génération,
\par les princes de chaque génération,
\par les conseillers de chaque génération,
\par les nobles de chaque génération,
\par et les hommes forts de chaque génération,
\par les anciens de chaque génération,
\par et les guides de chaque génération.

\par 3 Et j'ai vu Adam, sa génération, leurs actions et leurs pensées,
\par Noé et sa génération, leurs actes et leurs pensées,
\par et la génération du déluge, leurs actions et leurs pensées,
\par Sem et sa génération, leurs actes et leurs pensées,
\par Nimrod et la génération de la confusion des langues, et son
\par génération, leurs actes et leurs pensées,
\par Abraham et sa génération, leurs actions et leurs pensées,
\par Isaac et sa génération, leurs actions et leurs pensées,
\par Ismaël et sa génération, leurs actes et leurs pensées,
\par Jacob et sa génération, leurs actions et leurs pensées,
\par Joseph et sa génération, leurs actions et leurs pensées,
\par les tribus et leur génération, leurs actions et leurs pensées,
\par Amram et sa génération, leurs actes et leurs pensées,
\par Moïse et sa génération, leurs actions et leurs pensées,

\par 4 Aaron et Mirjam, leurs oeuvres et leurs actions,
\par les princes et les anciens, leurs œuvres et leurs actions,
\par Josué et sa génération, leurs œuvres et leurs actions,
\par les juges et leur génération, leurs œuvres et leurs actions,
\par Eli et sa génération, leurs oeuvres et leurs actions,
\par Phinées, leurs (?) œuvres et actions,
\par Elkana et sa génération, leurs oeuvres et leurs actions,
\par Samuel et sa génération, leurs œuvres et leurs actions,
\par les rois de Juda avec leurs générations, leurs oeuvres et leurs actions,
\par les rois d'Israël et leurs générations, leurs oeuvres et leurs actions,
\par les princes d'Israël, leurs œuvres et leurs actions ; les princes des nations du monde, leurs œuvres et leurs actions,
\par les chefs des conseils d'Israël, leurs œuvres et leurs actions ;
\par les chefs des (conseils dans) les nations du monde, leurs générations, leurs œuvres et leurs actions ;
\par les dirigeants d'Israël et leur génération, leurs oeuvres et leurs actions ;
\par les nobles d'Israël et leur génération, leurs œuvres et leurs actions ; les nobles des nations du monde et leur(s) génération(s), leurs œuvres et leurs actes ;
\par les hommes de renom en Israël, leur génération, leurs œuvres et leurs actes ;
\par les juges d'Israël, leur génération, leurs œuvres et leurs actions ;
\par les juges des nations du monde et leur génération, leurs œuvres et leurs actes ;
\par les enseignants des enfants en Israël, leurs générations, leurs œuvres et leurs actions ; les enseignants des enfants dans les nations du monde, leurs générations, leurs œuvres et leurs actions ;
\par les conseillers (interprètes) d'Israël, leur génération, leurs œuvres et leurs actes ; les conseillers (interprètes) des nations du monde, leur génération, leurs œuvres et leurs actes ;
\par tous les prophètes d'Israël, leur génération, leurs œuvres et leurs actions ; tous les prophètes des nations du monde, leur génération, leurs œuvres et leurs faits ;

\par 5 et tous les combats et guerres que les nations du monde ont menés contre le peuple d'Israël au temps de leur royaume.

\par Et je vis le Messie, fils de Joseph, et sa génération, et leurs oeuvres et leurs actions qu'ils feront contre les nations du monde. Et je vis le Messie, fils de David, et sa génération, et tous les combats et guerres, et leurs œuvres et leurs actions qu'ils feront envers Israël, tant en bien qu'en mal. Et j'ai vu tous les combats et guerres que Gog et Magog mèneront aux jours du Messie, et tout ce que le Saint, béni soit-Il, fera avec eux dans les temps à venir.

\par 6 Et tout le reste de tous les chefs des générations et toutes les œuvres des générations, tant en Israël que dans les nations du monde, ce qui est fait et ce qui sera fait désormais à toutes les générations jusqu'à la fin des temps, (tous) ont été gravés sur le rideau de MAQOM. Et j'ai vu toutes ces choses de mes yeux ; et après l'avoir vu, j'ouvris la bouche pour louer MAQOM (la Divine Majesté) (en disant ainsi) : « Car la parole du Roi a du pouvoir (et qui lui dira : Que fais-tu ?) Quiconque garde les commandements le devra je ne connais rien de mal ». Et j’ai dit : « Ô Seigneur, comme tes œuvres sont nombreuses ! »



\chapitre{46}

\par \textit{La place des étoiles montrée à R. Ismaël}

\par 1 R. Ismaël a dit : Métatron m'a dit : (Viens et je te montrerai) l'espace des étoiles qui se tiennent à Raqta'nuit la nuit dans la peur du Tout-Puissant (MAQOM) et (je te montrerai) où elles vont et où ils en sont.

\par 2 J'ai marché à ses côtés, et il m'a pris par la main et m'a tout montré avec ses doigts. Et ils se tenaient sur des étincelles de flammes autour de la Merkaba du Tout-Puissant (MAQOM). Qu'a fait Métatron ? À ce moment-là, il frappa dans ses mains et les chassa de leur place. Immédiatement, ils s'envolèrent de leurs ailes flamboyantes, se levèrent et s'enfuirent des quatre côtés du trône de la Merkaba, et (pendant qu'ils volaient) il me dit les noms de chacun. Comme il est écrit : « Il donne le nombre des étoiles ; il leur donne à tous leurs noms », enseignant que le Saint, béni soit-Il, a donné un nom à chacun d'eux.

\par 3 Et ils entrent tous dans l'ordre compté sous la direction de (lit. à travers, par les mains de) Rahatiel à Raqia ha-shSHamayim pour servir le monde. Et ils sortent en ordre compté pour louer le Saint, béni soit-Il, avec des chants et des hymnes, selon qu'il est écrit : « Les cieux racontent la gloire de Dieu ».

\par 4 Mais dans les temps à venir, le Saint, béni soit-Il, les créera de nouveau 9, comme il est écrit : « Ils sont nouveaux chaque matin ». Et ils ouvrent la bouche et poussent une chanson. Quelle est la chanson qu'ils poussent ? : « Quand je considère ton ciel ».

\chapitre{47}

\par \textit{Métatron montre à R. Ismaël les esprits des anges punis}

\par 1 R. Ismaël a dit : Métatron m'a dit : Viens et je te montrerai les âmes x des anges et les esprits des serviteurs dont les corps ont été brûlés dans le feu de MAQOM (le Tout-Puissant) qui sort de son petit doigt. Et ils ont été transformés en charbons ardents au milieu du fleuve ardent (Nehar di-Nur). Mais leurs esprits et leurs âmes se tiennent derrière la Shekina.

\par 2 Chaque fois que les anges au service prononcent un chant au mauvais moment ou comme il n'est pas prévu qu'il soit chanté, ils sont brûlés et consumés par le feu de leur Créateur et par une flamme de leur Créateur, [A : dans les lieux (chambres) du tourbillon , car il souffle sur eux et les chasse] [E : à leur place (= sur place) ; et un tourbillon souffle sur eux et les jette ] dans le Nehar di-Nur ; et là, ils se transforment en de nombreuses montagnes de charbon ardent. Mais leur esprit et leur âme retournent vers leur Créateur, et tous se tiennent derrière leur Maître.

\par 3 Et je suis allé à ses côtés et il m'a pris par la main ; et il me montra toutes les âmes des anges et les esprits des serviteurs qui se tenaient derrière les Shekinau sur les ailes du tourbillon et des murs de feu qui les entouraient.

\par 4 À ce moment-là, Métatron m'ouvrit les portes des murs à l'intérieur desquels ils se tenaient derrière la Shekina, et je levai les yeux et les vis, et voici, la ressemblance de chacun était comme celle des anges et leurs ailes étaient comme des ailes d'oiseaux, faites de flammes, ouvrage d'un feu ardent. À ce moment-là, j’ai ouvert la bouche pour faire l’éloge de MAQOM et j’ai dit : « Comme tes œuvres sont grandes, ô Seigneur ».

\chapitre{48a}

\par \textit{Métatron montre à Rabbi Ismaël la Main Droite du Très-Haut, désormais inactive derrière Lui, mais destinée dans le futur à œuvrer à la délivrance d'Israël}

\par 1 R. Ismaël a dit : Métatron m'a dit : Viens, et je te montrerai la Main Droite de MAQOM, posée derrière (Lui) à cause de la destruction du Saint Temple ; d'où brillent toutes sortes de splendeurs et de lumières par lesquelles les cieux ont été créés ; et que même les Séraphins et les 'Ophannim ne sont pas autorisés (à voir), jusqu'à ce que le jour du salut vienne.

\par 2 Et je suis allé à ses côtés et il m'a pris par la main et m'a montré (la main droite de MAQOM) y avec toutes sortes de louanges, de réjouissances et de chants : et aucune bouche ne peut dire sa louange, et aucun œil ne peut le voir , en raison de sa grandeur, de sa dignité, de sa majesté, de sa gloire et de sa beauté.

\par 3 Et non seulement ce 4, mais toutes les âmes des justes qui sont jugées dignes de 4a voient la joie de Jérusalem, ils se tiennent à côté d'elle, la louant et priant devant elle trois fois par jour, disant : « Réveillez-vous, réveillez-vous. , fortifie-toi, ô bras du Seigneur » selon qu'il est écrit : « Il fit aller son bras glorieux à la droite de Moïse ».

\par 4 A ce moment la Main Droite de MAQOM pleurait. Et cinq fleuves de larmes sortirent de ses cinq doigts et tombèrent dans la grande mer et ébranlèrent le monde entier, selon qu'il est écrit : « La terre est entièrement brisée, la terre est pure dissoute, la terre est extrêmement émue, la terre chancellera comme un homme ivre et sera déplacée d'avant en arrière comme une hutte », cinq fois correspondant aux doigts de sa grande main droite.

\par 5 Mais quand le Saint, béni soit-Il, voit qu'il n'y a pas d'homme juste dans la génération, ni d'homme pieux (Hasid) sur terre, et pas de justice entre les mains des hommes ; et (qu'il n'y a) aucun homme comme Moïse, ni aucun intercesseur comme Samuel qui pourrait prier devant MAQOM pour le salut et pour la délivrance, et pour Son Royaume, afin qu'il soit révélé dans le monde entier ; et pour sa grande main droite, qu'il l'a mise de nouveau devant lui pour opérer par elle un grand salut pour Israël,

\par 6 Alors aussitôt le Saint, béni soit-Il, se souviendra de sa propre justice, de sa faveur, de sa miséricorde et de sa grâce : et il délivrera par lui-même son grand bras, et sa justice le soutiendra. Selon qu'il est écrit : « Et il vit qu'il n'y avait personne » — (c'est-à-dire :) comme Moïse qui pria d'innombrables fois pour Israël dans le désert et détourna d'eux les décrets (divins) — « et il s'étonna : qu'il n'y avait pas d'intercesseur » - comme Samuel qui implora le Saint, béni soit-Il, et l'appela et il lui répondit et exauça son désir, même si cela n'était pas convenable (conformément au plan Divin), selon comme il est écrit : « N'est-ce pas aujourd'hui la moisson du blé ? J'invoquerai le Seigneur ».

\par 7 Et pas seulement cela, mais il s'est joint à Moïse en tout lieu, comme il est écrit : « Moïse et Aaron parmi ses prêtres ». Et encore il est écrit : « Bien que Moïse et Samuel se tenaient devant moi » : « À moi mon propre bras m’a apporté le salut ».

\par 8 Le Saint, béni soit-Il, dit à cette heure-là : « Jusqu'à quand attendrai-je que les enfants de Menth opèrent le salut selon leur justice pour mon bras ? Pour mon propre bien et pour mon mérite et ma justice, je délivrerai mon bras et par lui je rachèterai mes enfants parmi les nations du monde. Comme il est écrit : « Je le ferai pour moi-même. Car comment mon nom devrait-il être profané ? »

\par 9 À ce moment-là, le Saint, béni soit-Il, révélera son grand bras et le montrera aux nations du monde : car sa longueur est comme la longueur du monde et sa largeur est comme la largeur du monde. Et l’apparence de sa splendeur est semblable à la splendeur du soleil dans sa puissance, au solstice d’été.

\par 10 Aussitôt Israël sera sauvé du milieu des nations du monde. Et le Messie leur apparaîtra et Il les fera monter à Jérusalem avec une grande joie. Et pas seulement cela, mais [A : ils mangeront et boiront car ils glorifieront le Royaume du Messie, de la maison de David, dans les quatre parties du monde. Et les nations du monde ne prévaudront pas contre eux, ] [E : Israël viendra des quatre coins du monde et mangera avec le Messie. Mais les nations du monde ne mangeront pas avec eux, ] comme il est écrit : « L'Éternel a mis à nu son bras saint aux yeux de toutes les nations ; et toutes les extrémités de la terre verront le salut de notre Dieu ». Et encore : « Le Seigneur seul le conduisait, et il n'y avait pas de dieu étranger avec lui ». : « Et le Seigneur sera roi sur toute la terre ».



\chapitre{48b}

\par \textit{Les Noms Divins qui sortent du Trône de Gloire, couronnés et escortés par de nombreuses armées angéliques à travers les cieux et reviennent au Trône — les anges chantent le 'Saint' et le 'Bienheureux'}

\par 1 [AEFGH : Ce sont les noms du Saint, béni soit-Il] [K : Ce sont les soixante-douze noms écrits sur le cœur du Saint, béni soit-Il : SS, SeDeQ (justice), SaHI 'eL SUR, SBI, SaDdlQ {juste}, S'Ph, SHN, SeBa'oTh (Seigneur des armées), ShaDdaY (Dieu Tout-Puissant), 'eLoHIM (Dieu), YHWH, SH, DGUL, WDOM, SSS'', 'YW, y T, 'HW, HB, YaH, HW, WWW, SSS, PPP, NN, HH, HaY (vivant), HaY, ROKeB 'aRaBOTh (chevauchant sur l'Araboth), YH, HH, WH, MMM, NNN, HWW, YH, YHH, HPhS, H'S, W, S'Z' QQQ (Saint, Saint, Saint), QShR, BW, ZK, GINUR, GINURYa', T, YOD, 'aLePh, H'N, P 'P, R'W, YYW, YYW, BBB, DDD, TTT, KKK, KLL, SYS, TT BShKMLW (= béni soit le nom de son glorieux royaume pour toujours et à jamais), complété pour MeLeK Ha'OLaM (le roi de l'Univers), BRH LB' (le début de la Sagesse pour les enfants des hommes), BNLK W''Y (béni soit Celui qui donne de la force à ceux qui sont fatigués et augmente la force à ceux qui n'ont pas de force.)] qui vont de l'avant (orné) de nombreuses couronnes de feu, de nombreuses couronnes de flammes, de nombreuses couronnes de chashmal, de nombreuses couronnes d'éclairs devant le trône de gloire. Et avec eux (il y a) des milliers de centaines de puissances (c'est-à-dire des anges puissants) qui les escortent comme un roi [AE : avec honneur et colonnes de feu et nuage(s), et colonnes de flammes, et avec des éclairs de rayonnement et avec l'image de (le) chashmal.] [FG : avec tremblement et effroi, avec crainte et frisson, avec honneur et majesté et peur, avec terreur, avec grandeur et dignité, avec gloire et force, avec compréhension et connaissance et avec une colonne de feu et une colonne de flammes et d'éclairs - et leur lumière est comme des éclairs de lumière - et avec la ressemblance du hashmal. ]

\par 2 Et ils leur rendent gloire et ils répondent et crient devant eux : Saint, Saint, Saint. Et ils les roulent (convoyent) à travers tous les cieux comme des princes puissants et honorés. Et quand ils les ramènent tous au lieu du Trône de Gloire, alors tous les Chayyoth près de la Merkaba ouvrent la bouche pour louer Son nom glorieux, en disant : « Béni soit le nom de Son glorieux royaume pour les siècles des siècles. ».



\chapitre{48c}

\par \textit{Une pièce Enoch-Métatron}

\par 1 [AEFGH : Aleph l'a rendu fort, je l'ai pris, je l'ai nommé : (à savoir) Metatron, mon serviteur qui est un (unique) parmi tous les enfants du ciel. Je l'ai rendu fort dans la génération du premier Adam. Mais quand j'ai vu les hommes de la génération du déluge, qu'ils étaient corrompus, alors je suis allé et j'ai retiré ma Shekina du milieu d'eux. Et il l'éleva en haut au son d'une trompette et avec un cri, comme il est écrit : « Dieu est monté avec un cri, le Seigneur avec un son de trompette ». ] [K : « Je l'ai saisi, je l'ai pris et je l'ai établi » — c'est Enoch, le ]


\par 2 [AEFGH : « Et je l'ai pris » : (c'est-à-dire) Enoch, le fils de Jared, parmi eux. Et je l'ai élevé au son d'une trompette et avec un teru'a (cri) vers les cieux élevés, pour être mon témoin avec les Chayyoth, par la Merkaba dans le monde à venir. ] [K : fils de Jared, dont le nom est Metatron (2) et je l'ai pris parmi les enfants des hommes (5) et j'ai fait de lui un trône contre mon trône. Quelle est la taille de ce trône ? Soixante-dix mille parasangs (tous) de feu. 9 Je lui ai confié 70 anges correspondant aux nations (du monde) et je lui ai confié toute la maison d'en haut et d'en bas. (7) Et je lui ai confié la Sagesse et l'Intelligence plus qu'à tous les anges. Et j'ai appelé son nom « le moindre yah », dont le nom est par Guématrie 71. Et j'ai arrangé pour lui toutes les œuvres de la Création. Et j’ai fait en sorte que sa puissance transcende (j’ai fait pour lui une puissance plus grande que) tous les anges au service. (Fin K).]

\par 3 [AEFGH : Je l'ai établi sur tous les trésors et magasins que j'ai dans tous les cieux. Et je remis entre ses mains les clés de chacun. ] [Lm (commence ici) : Il confia à Métatron — c'est-à-dire Enoch, le fils de Jared — tous les trésors. Et je l'ai établi sur tous les magasins que j'ai dans tous les cieux. Et j'ai remis entre ses mains les clés de chaque magasin céleste. ]

\par 4 [AEFGH : J'ai fait (de) prince sur tous les princes et ministre du Trône de Gloire (et) des Salles d'Araboth : pour m'ouvrir leurs portes, et (du) Trône de Gloire, pour exaltez-le et arrangez-le ; (et je l'ai nommé) le Saint Chayyoth pour couronner leurs têtes ; les majestueux Ophannim pour les couronner de force et de gloire ; les honorés Kerubim, pour les revêtir de majesté ; sur les étincelles rayonnantes, pour les faire briller de splendeur et d'éclat ; sur les Séraphins enflammés, pour les couvrir de grandeur ; les Chashmallim de lumière, pour les faire rayonner de lumière et préparer le siège pour chaque matin ] [Lm : J'ai fait (de) lui le prince de tous les princes, et j'ai fait (de) lui un ministre de mon Trône de Gloire, pour pourvoir et arranger les Saints Chayyoth, pour leur couronner des couronnes (pour les couronner de couronnes), pour les revêtir d'honneur et de majesté pour leur préparer un siège] [R : alors que je suis assis sur le Trône de Gloire. Et pour exalter et magnifier ma gloire au sommet de ma puissance ; (et je lui ai confié) les secrets d'en haut et les secrets d'en bas (secrets célestes et secrets terrestres). ] [FGH : quand je serai assis sur mon Trône dans la gloire et la dignité afin qu'il puisse voir ma gloire dans la hauteur de ma puissance, dans les secrets d'en haut et dans les secrets d'en bas. ] [Lm : quand il est assis sur son trône pour magnifier sa gloire en hauteur]

\par 5 [AFGH : Je l'ai rendu plus haut que tous. De la hauteur de sa stature, au milieu de tous (qui sont) de haute stature (j'ai fait) soixante-dix mille parasanges. J'ai rendu son trône grand par la majesté de mon trône. Et j'ai augmenté sa gloire par l'honneur de ma gloire.] [Lm : La hauteur de sa stature parmi tous ceux (qui sont) de haute stature (est) de soixante-dix mille parasanges. Et j'ai rendu sa gloire grande comme la majesté de ma gloire ]

\par 6 [AFGH : J'ai transformé sa chair en torches de feu, et tous les os de son corps en charbons ardents ; et j'ai fait apparaître ses yeux comme un éclair, et la lumière de ses sourcils comme une lumière impérissable. J'ai rendu son visage brillant comme la splendeur du soleil, et ses yeux comme la splendeur du Trône de Gloire. ] [Lm : et l'éclat de ses yeux comme la splendeur du Trône de Gloire ]

\par 7 [AFGH : J'ai fait honneur et majesté à ses vêtements, beauté et altesse, son manteau de couverture et une couronne royale de 500 par (fois) parasanges (son) diadème. ] [Lm : son vêtement honneur et majesté, sa couronne royale 500 par 500 parasanga.] [AFGHLm : Et je l'ai revêtu de mon honneur, de ma majesté et de la splendeur, de ma gloire qui est sur mon Trône de Gloire. Je l'ai appelé Yhwh le moindre, le Prince de la Présence, le Connaisseur des secrets : car je lui ai révélé chaque secret comme un père et je lui ai déclaré tous les mystères avec droiture. ]

\par 8 J'ai placé son trône à la porte de ma salle pour qu'il puisse s'asseoir et juger la maison céleste d'en haut. Et j'ai placé chaque prince devant lui, pour recevoir de lui le pouvoir d'accomplir sa volonté.

\par 9 J'ai pris soixante-dix noms de (mes) noms et je l'ai appelé par eux pour rehausser sa gloire.

\par 10 Soixante-dix princes remirent entre ses mains, pour leur commander mes préceptes et mes paroles en toutes langues ; [AFGH : pour abaisser par sa parole les orgueilleux jusqu'à terre, et pour élever par la parole de ses lèvres les humbles jusqu'aux hauteurs ; pour frapper les rois par son discours, pour détourner les rois de leurs sentiers, pour établir (les) dirigeants sur leur domination comme il est écrit : « et il change les temps et les saisons, et pour donner la sagesse à tous les sages du monde » et la compréhension (et) la connaissance à tous ceux qui comprennent la connaissance », comme il est écrit : « et la connaissance à ceux qui savent comprendre », pour leur révéler les secrets de mes paroles et enseigner le décret de mon juste jugement, tel qu'il est écrit: ] [Lm: et pour abaisser les orgueilleux et pour élever les humbles à la hauteur et pour frapper les rois et pour abaisser les dirigeants et pour établir des rois et des dirigeants et il change les temps et les saisons il renverse les rois et établit des rois, il donne la sagesse aux sages et la connaissance à ceux qui savent comprendre et je l'ai chargé de révéler les secrets et d'enseigner le jugement et la justice, ] « ainsi sera ma parole qui sort de ma bouche ; il ne me reviendra pas vide mais accomplira (ce que je veux), «E'ie'seh y (j'accomplirai) n'est pas écrit ici, mais ll asah y (il accomplira), ce qui signifie que quel que soit le mot et quoi que ce soit La parole sort de devant le Saint, béni soit-Il, Métatron se lève et l'exécute. Et il établit les décrets du Saint, béni soit-Il. (Ici se termine la version Lm du fragment c.)

\par 11 [« Et il fera prospérer ce que j'ai envoyé » . 'Asli'a'h (je ferai prospérer) n'est pas écrit ici, mais w'e'hisli'a'h (et il fera prospérer), enseignant que tout décret venant de devant le Saint, Béni soit-Il, concernant un homme, dès qu'il se repent, ils « ne l'exécutent pas (sur lui) mais sur un autre homme méchant, comme il est écrit : » Le juste est délivré de la détresse, et le méchant vient. à sa place».]

\par 12 Et non seulement cela, mais Métatron reste assis trois heures chaque jour dans les cieux élevés, et il rassemble toutes les âmes de ceux qui sont morts dans le ventre de leur mère, et des nourrissons qui sont morts sur les seins de leur mère, et des érudits qui sont morts au cours des cinq livres de la loi. Et il les amène sous le Trône de Gloire et les place en groupes, divisions et classes autour de la Présence : et il leur enseigne la Loi, et (les livres) de Sagesse, et la Haggada et la Tradition et termine (complète) leur instruction (éducation) [pour eux]. Comme il est écrit : « À qui enseignera-t-il la connaissance ? et à qui fera-t-il comprendre la tradition ? ceux qui sont sevrés du lait et tirés des mamelles ».


\chapitre{48d}

\par \textit{Les noms de Métatron. Les trésors de la Sagesse ouverts à Moïse sur le mont Sinaï. Les anges protestent contre Métatron pour avoir révélé les secrets à Moïse et Dieu leur répond et les réprimande. La chaîne de la tradition et le pouvoir des mystères transmis pour guérir les maladies}

\par 1 Soixante-dix noms ont Metatron que le Saint, béni soit-Il, a pris de son propre nom et l'a revêtu. Et ce sont :

\par YeHOEL YaH, YeHOEL, YOPHIEL et Yophphiel, et 5 'APHPHIEL et MaRGeZIEL,GIPpUYEL, Pa'aZIEL, A'aH, PerIEL, TaTRIEL, TaBKIEL, 'W, YHWH, DH WHYH,'eBeD, DiBbURIEL, 'aPH' apIEL, SPPIEL, PasPaSIEL, SeneGRON, MeTaTRON, SOGDIN,'A- DRIGON, 'ASUM, SaQPaM, SaQTaM, MIGON, MITTON, MOTTRON, ROSPHIM, QINOTh,ChaTaTYaH, DeGaZYaH, PSPYaH, BSKNYH, MZRG, BaRaD, MKRKK, MSPRD, ChShG, ChShB, MNRTTT, BSYRYM, MITMON, TITMON, PiSQON, SaPhSaPhYaH, ZRCh, ZRChYaH, BeYaH, HBH BeYaH, PeLeT, PLTYaH, RaBRaBYaH, ChaS, ChaSYaH, TaPhTaPhYaH,TaMTaMYaH, SeHaSYaH, IR'URYaH, L'aLYaH, BaZRIDY Ah , SaTSaTKYaH, SaSDYaH, RaZRaZYAH, BaZRaZYaH, aRIMYaH, SBHYaH, SBIBKHYH, StMKaM, YaHSeYaH, SSBIBYaH, SaBKaSBeYaH, QeLILQaLYaH, KIHHH, HHYH, ZoWH, POURQUOI, ZaKklKYaH, TUTRISYaH, SURYaH, ZeH, PenIRHYaH, 'ZrH, GaL RaZaYYa, MaMLIKYaH , TTYaH, eMeQ, QaMYaH, MeKaPpeR YaH, PERISHYAH, SePhaM, GBIR, GiBbORYaH, GOR, GOR YaH, ZIW, OKBaR, le MOINS YHWH, d'après le nom de son Maître, « car mon nom est en lui », RaBIBIEL, TUMIEL , Segansakkiel, le Prince de la Sagesse.

\par 2 Et pourquoi est-il appelé par le nom de Sagnesakiel ? Parce que tous les trésors de la sagesse sont entre ses mains.

\par 3 Et tous furent ouverts à Moïse sur le Sinaï, de sorte qu'il les apprit pendant les quarante jours, pendant qu'il était debout (restant) : la Torah dans les soixante-dix aspects des soixante-dix langues, les Prophètes dans les soixante-dix aspects des soixante-dix langues, les Écrits dans les soixante-dix aspects des soixante-dix langues, les Halakas dans les soixante-dix aspects des soixante-dix langues, les Traditions dans les soixante-dix aspects des soixante-dix langues, les Haggadas dans les soixante-dix aspects des soixante-dix langues et les Toseftas dans les soixante-dix aspects des soixante-dix langues.

\par 4 Mais dès que les quarante jours furent écoulés, il les oublia tous en un instant. Alors le Saint, béni soit-Il, appela Yephiphyah, le Prince de la Loi, et (par son intermédiaire) ils furent donnés en cadeau à Moïse. Comme il est écrit : « et le Seigneur me les donna ». Et après cela, il est resté avec lui. Et d'où savons-nous qu'il est resté (dans sa mémoire) ? Parce qu'il est écrit : « Souvenez-vous de la loi de Moïse, mon serviteur12, que je lui ai commandée en Horeb pour tout Israël, mes statuts et mes jugements ». « La Loi de Moïse » : c'est le Tor a, les Prophètes et les Écrits, les « statuts » : c'est les Halakas et les Traditions, les « jugements » ; ce sont les Haggadas et les Toseftas. Et tous furent donnés à Moïse dans les hauteurs du Sinaï,

\par 5 Ces soixante-dix noms (sont) le reflet du ou des noms explicites sur la Merkaba qui sont gravés sur le Trône de Gloire. Car le Saint, béni soit-Il, a pris de Son(ses) Nom(s) explicite(s) et a mis le nom de Métatron : Soixante-dix noms par lesquels les anges au service appellent le Roi des rois des rois, béni soit-Il, dans les cieux élevés, et vingt-deux lettres qui sont sur l'anneau à son doigt avec lesquelles sont scellées les destinées des princes de royaumes d'en haut en grandeur et en puissance et avec lesquels sont scellés le sort de l'Ange de la Mort et les destinées de chaque nation et langue.

\par 6 Dit Métatron, l'Ange, le Prince de la Présence ; l'Ange, le Prince de la Sagesse ; l'Ange, le Prince de l'Intelligence ; l'Ange, le Prince des Rois ; l'Ange, le Prince des Souverains ; l'ange, le Prince de la Gloire ; l'ange, le prince des hauts et des princes, les élevés, les grands et les honorés, dans le ciel et sur la terre :

\par 7 H, Dieu d'Israël, m'est témoin de cette chose, (que) lorsque j'ai révélé ce secret à Moïse, alors toutes les armées de tous les cieux d'en haut se sont déchaînées contre moi et m'ont dit :

\par 8 Pourquoi révèles-tu ce secret à un fils d'homme, né d'une femme, souillé et impur, un homme d'une goutte putréfiante, le secret par lequel ont été créés le ciel et la terre, la mer et la terre ferme, les montagnes et les collines, les rivières et les sources, la Géhenne de feu et de grêle, le Jardin d'Eden et l'Arbre de Vie ; et par lequel furent formés Adam et Ève, et le bétail, et les bêtes sauvages, et les oiseaux du ciel, et les poissons de la mer, et Béhémoth et Léviathan, et les reptiles, les vers, les dragons de la mer, et les choses rampantes des déserts ; et la Tora et la Sagesse et la Connaissance et la Pensée et la Gnose des choses d'en haut et la peur du ciel. Pourquoi révèles-tu cela à la chair et au sang ? [R : As-tu obtenu l’autorité de MAQOM ? Et encore : As-tu reçu la permission ? Les Noms Explicites sortirent de devant moi] [FG : Je leur répondis : Parce que le Saint, béni soit-Il, m'a donné l'autorité, Et en outre, j'ai obtenu la permission du Trône haut et exalté, d'où tous les Des noms explicites sortent ] avec des éclairs de feu et des chashmallim enflammés.

\par 9 Mais ils ne furent pas apaisés, jusqu'à ce que le Saint, béni soit-Il, les réprimanda et les chassa de devant lui avec des reproches, en leur disant : «Je prends plaisir à, et j'ai mis mon amour, et je leur ai confié et confié à Métatron, mon Serviteur, seul, car il est Un (unique) parmi tous les enfants du ciel.

\par 10 Et Métatron les fit sortir de sa maison des trésors et les confia à Moïse, et Moïse à Josué, et Josué aux anciens, et les anciens aux prophètes et les prophètes aux hommes de la Grande Synagogue, et les hommes de la Grande Synagogue à Esdras et Esdras le Scribe à Hillel l'ancien, et Hillel l'ancien à Rabbi Abbahu et Rabbi Abbahu à Rabbi Zera, et Rabbi Zera aux hommes de foi et aux hommes de foi (les engagea ) pour avertir et guérir par eux toutes les maladies qui font rage dans le monde, comme il est écrit : « Si tu écoutes diligemment la voix du Seigneur, ton Dieu, et si tu fais ce qui est droit à ses yeux, et Si tu prêtes l'oreille à ses commandements et si tu observes toutes ses prescriptions, je ne te ferai subir aucune des maladies que j'ai infligées aux Égyptiens, car je suis l'Éternel, qui te guéris. (Terminé et terminé. Loué soit le Créateur du monde.)




\end{document}