\begin{document}

\title{Lettre d'Aristée}

\chapter{1}

\par \textit{Au moment de la captivité juive en Egypte, Ptolémée Philadelphe se révèle comme le premier grand bibliophile. Il désire avoir tous les livres du monde dans sa bibliothèque ; Afin d'obtenir les lois de Moïse, il propose d'échanger 100 000 captifs contre ce travail en s'exclamant : «C'est vraiment une petite aubaine !»}

\par 1 DEPUIS que j'ai rassemblé des matériaux pour une histoire mémorable de ma visite à Éléazar, le grand prêtre des Juifs, et parce que toi, Philocrate, comme tu ne perds aucune occasion de me le rappeler, tu as attaché une grande importance à recevoir un récit des motifs et objet de ma mission, j'ai tenté de rédiger pour vous un exposé clair du sujet, car je vois que vous possédez un amour naturel pour apprendre, qualité qui est la plus haute possession de l'homme, et que vous essayez constamment « d'ajouter à son capital de connaissances et d'acquis, que ce soit à travers l'étude de l'histoire ou en participant effectivement aux événements eux-mêmes.

\par 2 C'est par ce moyen, en reprenant en elle les éléments les plus nobles, que l'âme s'établit dans la pureté, et ayant fixé son but sur la piété, le but le plus noble de tous, elle s'en sert comme guide infaillible et acquiert ainsi un but précis.

\par 3 C'est mon dévouement à la recherche des connaissances religieuses qui m'a amené à entreprendre l'ambassade auprès de l'homme que j'ai mentionné, qui était tenu en la plus haute estime par ses propres citoyens et par les autres, tant pour sa vertu que pour sa majesté, et qui avait en sa possession des documents de la plus haute valeur pour les Juifs de son propre pays et des pays étrangers pour l'interprétation de la loi divine, car leurs lois sont écrites sur des parchemins de cuir en caractères juifs.

\par 4 J'entrepris donc cette ambassade avec enthousiasme, ayant d'abord trouvé l'occasion de plaider auprès du roi en faveur des captifs juifs qui avaient été transportés de Judée en Égypte par le père du roi, lorsqu'il prit pour la première fois possession de cette ville et conquit la terre d'Egypte.

\par 5 Cela vaut la peine que je vous raconte également cette histoire, car je suis convaincu que vous, avec votre disposition à la sainteté et votre sympathie pour les hommes qui vivent selon la sainte loi, écouterez d'autant plus volontiers le récit. C'est ce que j'ai l'intention de vous exposer, puisque vous êtes vous-même venu récemment de l'île et que vous désirez entendre tout ce qui tend à édifier l'âme.

\par 6 Une fois également, je vous ai envoyé un rapport sur les faits que j'ai jugé dignes d'être racontés sur la race juive, un rapport que j'avais obtenu des grands prêtres les plus savants du pays le plus savant d'Égypte.

\par 7 Comme vous êtes si désireux d'acquérir la connaissance de ces choses qui peuvent bénéficier à l'esprit, je sens qu'il m'incombe de vous communiquer toutes les informations en mon pouvoir.

\par 8 Je devrais ressentir le même devoir envers tous ceux qui possédaient la même disposition mais je le ressens spécialement envers vous puisque vous avez des aspirations si nobles, et que vous n'êtes pas seulement mon frère de caractère, pas moins que de sang, mais nous ne faisons qu'un avec moi dans la quête du bien.

\par 9 Car ni le plaisir tiré de l'or ni aucun autre bien prisé par les esprits superficiels ne confère le même bénéfice que la poursuite de la culture et l'étude que nous dépensons pour l'acquérir.

\par 10 Mais pour ne pas vous fatiguer par une trop longue introduction, je passerai tout de suite au fond de mon récit.

\par 11 Démétrius de Phalère, président de la bibliothèque du roi, reçut une somme d'argent considérable pour rassembler, autant qu'il le pouvait, tous les livres du monde.

\par 12 Au moyen d'achats et de transcriptions, il exécuta, au mieux de ses capacités, le dessein du roi.

\par 13 Un jour, alors que j'étais présent, on lui a demandé : Combien de milliers de livres y a-t-il dans la bibliothèque ? et il répondit : « Plus de deux cent mille, ô roi, et je m'efforcerai dans un avenir immédiat de rassembler également le reste, afin que le total de cinq cent mille puisse être atteint. On me dit que les lois des Juifs valent la peine d'être transcrites et méritent une place dans votre bibliothèque !

\par 14 'Qu'est-ce qui vous empêche de faire cela ?' répondit le roi. « Tout ce qui est nécessaire a été mis à votre disposition !

\par 15 «Ils ont besoin d'être traduits», répondit Démétrius, «car dans le pays des Juifs, ils utilisent un alphabet particulier (tout comme les Égyptiens ont aussi une forme particulière de lettres) et parlent un dialecte particulier.»

\par 16 'Ils sont censés utiliser la langue syriaque, mais ce n'est pas le cas ; leur langage est tout à fait différent.

\par 17 Et le roi, après avoir compris tous les faits de l'affaire, ordonna qu'une lettre soit écrite au grand prêtre juif afin que son dessein (qui a déjà été décrit) puisse être accompli.

\par 18 Pensant que le moment était venu d'insister sur la demande que j'avais souvent présentée à Sosibius de Tarente et à Andreas, le chef des gardes du corps, concernant l'émancipation des Juifs déportés de Judée par le père du roi, car quand par un Grâce à sa bonne fortune et à son courage, il avait mené à bien son attaque contre tout le district de Coele-Syrie et de Phénicie. En terrorisant le pays et en le soumettant, il transporta certains de ses ennemis et en réduisit d'autres en captivité.

\par 19 Le nombre de ceux qu'il transporta du pays des Juifs en Égypte ne s'élevait pas à moins de cent mille.

\par 20 Parmi eux, il arma trente mille hommes d'élite et les installa en garnisons dans les campagnes.

\par 21 (Et même avant cette époque, un grand nombre de Juifs étaient venus en Égypte avec les Perses, et dans une période antérieure encore d'autres avaient été envoyés en Égypte pour aider Psammétique dans sa campagne contre le roi des Éthiopiens. Mais ceux-ci n'étaient pas aussi nombreux que les captifs que Ptolémée, fils de Lagus, transporta.)

\par 22 Comme je l'ai déjà dit, Ptolémée sélectionnait les meilleurs d'entre eux, ceux qui étaient dans la fleur de l'âge et se distinguaient par leur courage, et les armait, mais la grande masse des autres, ceux qui étaient trop vieux ou trop vieux. Il réduisit à cet effet les jeunes et les femmes en esclavage, non pas qu'il souhaitât le faire de son plein gré, mais il y fut contraint par ses soldats qui les réclamaient en récompense des services qu'ils avaient rendus à la guerre. .

\par 23 Ayant, comme je l'ai déjà dit, obtenu une occasion d'obtenir leur émancipation, je m'adressai au roi avec les arguments suivants. «Ne soyons pas déraisonnables au point de permettre à nos actes de démentir nos paroles.»

\par 24 « Puisque la loi que nous voulons non seulement transcrire mais aussi traduire appartient à toute la race juive, quelle justification pourrons-nous trouver à notre ambassade pendant qu'un si grand nombre d'entre eux restent en état d'esclavage dans votre pays ? Royaume?'

\par 25 'Dans la perfection et la richesse de ta clémence, libère ceux qui sont tenus dans un esclavage si misérable, car comme j'ai eu soin de le découvrir, le Dieu qui leur a donné leur loi est le Dieu qui maintient votre royaume.'

\par 26 «Ils adorent le même Dieu, le Seigneur et Créateur de l'Univers, comme tous les autres hommes, comme nous-mêmes, ô roi, bien que nous l'appelions sous des noms différents, tels que Zeus 1 ou Dis.»

\par 27 «Ce nom lui a été très bien attribué par nos premiers ancêtres, afin de signifier que Celui, par qui toutes choses sont dotées de vie et naissent, est nécessairement le Cavalier et le Seigneur de l'Univers.»

\par 28 'Donnez à toute l'humanité un exemple de magnanimité en libérant ceux qui sont retenus en esclavage.'

\par 29 Après un bref intervalle, pendant que j'adressais une prière sincère à Dieu pour qu'il dispose l'esprit du roi de manière à ce que tous les captifs soient mis en liberté (car le genre humain, étant la création de Dieu, est influencé et influencé par Lui.

\par 30 C'est pourquoi, avec de nombreuses prières diverses, j'ai invoqué Celui qui dirige les cœurs, afin que le roi soit contraint d'accéder à ma demande.

\par 31 Car j'avais de grands espoirs quant au salut des hommes puisque j'étais assuré que Dieu exaucerait l'accomplissement de ma prière.

\par 32 Car lorsque des hommes, pour des motifs purs, planifient une action dans l'intérêt de la justice et dans l'accomplissement d'actes nobles, Dieu Tout-Puissant mène leurs efforts et leurs desseins à un résultat réussi) - le roi leva la tête et me regarda avec un visage joyeux et demanda : « Combien de milliers pensez-vous qu'ils seront au nombre ? »

\par 33 Andreas, qui se tenait à proximité, répondit : « Un peu plus de cent mille. »

\par 34 «C'est en effet une petite faveur», dit le roi, «ce qu'Aristée nous demande!»

\par 35 Alors Sosibius et quelques autres qui étaient présents dirent : «Oui, mais ce serait un hommage digne de votre magnanimité que d'offrir l'affranchissement de ces hommes comme un acte de dévotion au Dieu suprême.»

\par 36 «Vous avez été grandement honoré par Dieu Tout-Puissant et exalté au-dessus de tous vos ancêtres dans la gloire et il est tout à fait approprié que vous lui rendiez la plus grande offrande de remerciement en votre pouvoir.»

\par 37 Extrêmement satisfait de ces arguments, il ordonna d'ajouter aux salaires des soldats le montant de l'argent de rachat, de payer vingt drachmes aux propriétaires pour chaque esclave, d'émettre un ordre public et de les registres des captifs doivent y être joints.

\par 38 Il montra le plus grand enthousiasme dans cette entreprise, car c'était Dieu qui avait réalisé notre objectif dans son intégralité et l'avait contraint à racheter non seulement ceux qui étaient venus en Égypte avec l'armée de son père, mais tous ceux qui étaient venus avant cette époque ou avait été introduit par la suite dans le royaume.

\par 39 On lui fit remarquer que l'argent de la rançon dépasserait quatre cents talents.

\par 40 Je pense qu'il sera utile d'insérer une copie du décret, car de cette manière la magnanimité du roi, à qui Dieu a donné le pouvoir de sauver de si vastes multitudes, sera rendue plus claire et plus manifeste.

\par 41 Le décret du roi était ainsi rédigé : « Tous ceux qui ont servi dans l'armée de notre père dans la campagne contre la Syrie et la Phénicie et dans l'attaque contre le pays des Juifs et qui sont devenus possédés de captifs juifs et les ont ramenés à la ville d'Alexandrie et le pays d'Egypte ou les ont vendus à d'autres - et de la même manière tous les captifs qui se trouvaient dans notre pays avant cette époque ou qui y ont été amenés ici après - tous ceux qui possèdent de tels captifs sont tenus de les remettre en liberté immédiatement , recevant vingt drachmes par tête en guise de rançon.

\par 42 'Les soldats recevront cet argent en cadeau ajouté à leur salaire, les autres du trésor du roi.'

\par 43 «Nous pensons que c'était contre la volonté de notre père et contre toute convenance qu'ils auraient dû être faits captifs et que la dévastation de leur pays et le transport des Juifs en Egypte étaient un acte de folie militaire.»

\par 44 'Le butin qui tombait aux soldats sur le champ de bataille était tout le butin qu'ils auraient dû réclamer.'

\par 45 «Réduire en outre le peuple à l'esclavage était un acte d'injustice absolue.»

\par 46 « C'est pourquoi, puisqu'il est reconnu que nous sommes habitués à rendre justice à tous les hommes et spécialement à ceux qui sont injustement dans un état de servitude, et puisque nous nous efforçons de traiter équitablement tous les hommes selon les exigences de la justice et piété, nous avons décrété, en référence aux personnes des Juifs qui sont dans une condition de servitude dans n'importe quelle partie de notre domination, que ceux qui les possèdent recevront la somme d'argent stipulée et les mettront en liberté et que personne ne devra ne montrer aucun retard dans l'exécution de ses obligations.

\par 47 'Dans les trois jours qui suivront la publication de ce décret, ils devront dresser des listes d'esclaves pour les officiers chargés d'exécuter notre volonté, et présenter immédiatement les personnes des captifs.'

\par 48 «Car nous considérons qu'il sera avantageux pour nous et pour nos affaires que l'affaire soit menée à son terme.»

\par 49 « Quiconque le souhaite peut donner des renseignements sur quiconque désobéit au décret, à condition que si l'homme est reconnu coupable, il deviendra son esclave ; ses biens seront cependant remis au trésor royal.

\par 50 Lorsque le décret fut soumis au roi pour approbation, il contenait toutes les autres dispositions, à l'exception de la phrase "tous les captifs qui se trouvaient dans le pays avant cette époque ou qui y furent amenés par la suite", et dans sa magnanimité et sa grandeur de son cœur, le roi inséra cette clause et ordonna que l'octroi d'argent nécessaire au rachat soit déposé en totalité auprès des payeurs des forces armées et des banquiers royaux. Ainsi, l'affaire fut tranchée et le décret ratifié dans les sept jours.

\par 51 La dotation pour le rachat s'élevait à plus de six cent soixante talents ; car de nombreux enfants au sein furent émancipés avec leurs mères.

\par 52 Lorsqu'on demanda si la somme de vingt talents devait être payée pour cela, le roi ordonna que cela soit fait, et il exécuta ainsi sa décision de la manière la plus complète.

\par \textit{Notes de bas de page}
\par \textit{143:1 Une comparaison importante entre Dieu et Zeus.}

\chapter{2}

\par \textit{Montrant comment les archives les plus minutieuses étaient tenues sur les affaires de l'État. La bureaucratie gouvernementale. Un comité de six personnes est nommé pour se rendre chez le Grand Prêtre à Jérusalem et organiser l'échange. Aristée est chargé de la délégation.}

\par 1 Une fois cela fait, il ordonna à Démétrius de rédiger un mémoire concernant la transcription des livres juifs.

\par 2 Car toutes les affaires de l'État étaient exécutées au moyen de décrets et avec la plus grande exactitude par ces rois égyptiens, et rien n'était fait de manière négligente ou aléatoire.

\par 3 J'ai donc inséré des copies du mémorial et des lettres, le nombre des présents envoyés et la nature de chacun, puisque chacun d'eux excellait en magnificence et en habileté technique.

\par 4 Ce qui suit est une copie du mémorial. La mémoire de Démétrius au grand roi. « Depuis que vous m'avez donné instruction, ô roi, que les livres nécessaires pour compléter votre bibliothèque soient rassemblés et que ceux qui sont défectueux soient réparés, je me consacre avec le plus grand soin à l'accomplissement de vos vœux, et j'ai maintenant la proposition suivante à vous présenter.

\par 5 'Les livres de la loi des Juifs (avec quelques autres) sont absents de la bibliothèque.'

\par 6 'Ils sont écrits en caractères et en langue hébraïque et ont été interprétés avec négligence, et ne représentent pas le texte original comme je l'ai informé par ceux qui le savent ; car ils n'ont jamais eu le soin d'un roi pour les protéger.

\par 7 'Il est nécessaire que ceux-ci soient rendus précis pour votre bibliothèque puisque la loi qu'ils contiennent, en tant qu'elle est d'origine divine, est pleine de sagesse et exempte de toute tare.'

\par 8 «C'est pour cette raison que les hommes de lettres, les poètes et la masse des écrivains historiques se sont abstenus de faire référence à ces livres et aux hommes qui ont vécu et vivent conformément à eux, parce que leur conception de la vie est si sacrée et religieuse, comme le dit Hécatée d'Abdère.

\par 9 « S'il te plaît, ô roi, une lettre sera écrite au grand prêtre de Jérusalem, lui demandant d'envoyer six anciens de chaque tribu, des hommes qui ont vécu la vie la plus noble et qui sont les plus habiles dans leur loi, afin que nous puissions découvrez les points sur lesquels la majorité d'entre eux sont d'accord, et ainsi, après avoir obtenu une traduction précise, vous pourrez la placer à un endroit bien en vue, d'une manière digne de l'ouvrage lui-même et de votre objectif.»

\par 10 « Que la prospérité continue soit la vôtre ! »

\par 11 Après que ce mémoire eut été présenté, le roi ordonna qu'une lettre soit écrite à Éléazar à ce sujet, décrivant également l'émancipation des captifs juifs.

\par 12 Et il donna cinquante talents d'or au poids et soixante-dix talents d'argent et une grande quantité de pierres précieuses pour faire des coupes et des coupes et une table et des coupes de libation.

\par 13 Il ordonna également à ceux qui avaient la garde de ses coffres de permettre aux artisans de faire une sélection de tous les matériaux dont ils pourraient avoir besoin à cet effet, et d'envoyer cent talents en argent pour fournir des sacrifices pour le temple et pour d'autres besoins.

\par 14 Je vous rendrai un compte rendu complet de l'exécution après avoir mis devant vous des copies des lettres. La lettre du roi était ainsi rédigée :

\par 15 «Le roi Ptolémée envoie salutations et salutations au grand prêtre Éléazar.»

\par 16 «Comme il y a beaucoup de Juifs installés dans notre royaume qui ont été enlevés de Jérusalem par les Perses à l'époque de leur pouvoir et beaucoup d'autres qui sont venus avec mon père en Égypte comme captifs, il en a placé un grand nombre dans l'armée et les a payés des salaires plus élevés que d'habitude, et après avoir prouvé la loyauté de leurs dirigeants, il construisit des forteresses et les confia à leur garde afin que les Égyptiens indigènes puissent être intimidés par elles.»

\par 17 « Et moi, lorsque je suis monté sur le trône, j'ai adopté une attitude bienveillante envers tous mes sujets, et plus particulièrement envers ceux qui étaient vos citoyens : j'ai mis en liberté plus de cent mille captifs, en payant pour eux à leurs propriétaires le prix du marché. , et si jamais du mal a été fait à votre peuple à cause des passions de la foule, je lui ai fait réparation«»

\par 18 'Le motif qui a poussé mon action a été le désir d'agir pieusement et de rendre au Dieu suprême une offrande de remerciement pour avoir maintenu mon royaume dans la paix et une grande gloire dans le monde entier.'

\par 19 J'ai enrôlé dans mon armée ceux de votre peuple qui étaient dans la fleur de l'âge, et ceux qui étaient aptes à être attachés à ma personne et dignes de la confiance de la cour, j'ai établi des postes officiels. '

\par 20 « Maintenant que je tiens à montrer ma gratitude à ces hommes et aux Juifs du monde entier et aux générations à venir, j'ai décidé que votre loi sera traduite de la langue hébraïque qui est en usage parmi vous vers la langue grecque. , afin que ces livres puissent être ajoutés aux autres livres royaux de ma bibliothèque.»

\par 21 «Ce sera une bonté de votre part et une récompense pour mon zèle si vous choisissez six anciens de chacune de vos tribus, des hommes de vie noble et compétents dans votre loi et capables de l'interpréter, qu'en matière de litige nous pourrons peut-être découvrir le verdict sur lequel la majorité est d'accord, car l'enquête est de la plus haute importance possible.

\par 22 'J'espère acquérir une grande renommée par l'accomplissement de ce travail.'

\par 23 J'ai envoyé Andréas, le chef de ma garde du corps, et Aristée, hommes que j'estime en haute estime, pour vous présenter l'affaire et vous présenter cent talents d'argent, prémices de mon offrande pour le temple et les sacrifices et autres rites religieux.

\par 24 'Si vous m'écrivez concernant vos souhaits en ces matières, vous m'accorderez une grande faveur et m'offrirez un nouveau gage d'amitié, car tous vos souhaits seront exaucés le plus rapidement possible. Adieu!'

\par 25 À cette lettre, Éléazar répondit de manière appropriée comme suit : « Éléazar, le grand prêtre, envoie ses salutations au roi Ptolémée, son véritable ami. »

\par 26 «Mes vœux les plus sincères vont à votre bien-être et à celui de la reine Arsinoé, de votre sœur et de vos enfants.»

\par 27 'Moi aussi je vais bien. J'ai reçu votre lettre et je suis très heureux de votre dessein et de vos nobles conseils.

\par 28 'J'ai convoqué tout le peuple et je le leur ai lu afin qu'ils connaissent votre dévotion envers notre Dieu.'

\par 29 Je leur ai montré aussi les coupes que vous aviez envoyées, vingt d'or et trente d'argent, les cinq coupes et la table de dédicace, et les cent talents d'argent pour l'offrande des sacrifices et pour pourvoir aux choses dont le le temple est dans le besoin.

\par 30 «Ces cadeaux m'ont été apportés par Andreas, l'un de vos serviteurs les plus honorés, et par Aristée, tous deux hommes bons et vrais, distingués par leur savoir et dignes à tous égards d'être les représentants de vos principes élevés et justes fins.

\par 31 'Ces hommes m'ont transmis votre message et ont reçu de moi une réponse en accord avec votre lettre. Je consentirai à tout ce qui vous sera avantageux, même si votre demande est très inhabituelle.

\par 32 «Car vous avez accordé à nos citoyens de grands avantages qui ne seront jamais oubliés à bien des égards.»

\par 33 « Aussitôt j'ai offert des sacrifices en faveur de vous, de votre sœur, de vos enfants et de vos amis, et tout le peuple a prié pour que vos projets prospèrent continuellement et que Dieu Tout-Puissant puisse conserver votre royaume en paix et avec honneur, et afin que la traduction de la sainte loi puisse vous être avantageuse et être exécutée avec succès.

\par 34 « En présence de tout le peuple, j'ai choisi dans chaque tribu six anciens, des hommes bons et fidèles, et je vous les ai envoyés avec une copie de notre loi. »

\par 35 'Ce serait une bonté, ô roi juste, si tu donnes l'instruction que dès que la traduction de la loi sera achevée, les hommes nous seront de nouveau rendus en sécurité. Adieu!'

\par 36 Voici les noms des anciens : De la première tribu, Joseph, Ezéchias, Zacharie, Jean, Ezéchias, Elisée.

\par 37 De la seconde tribu, Judas, Simon, Samuel, Adaeus, Mattathias, Eschlemias.

\par 38 De la troisième tribu, Néhémie, Joseph, Théodose, Baseas, Ornias, Dakis.

\par 39 De la quatrième tribu, Jonathan, Abraeus, Elisée, Ananias, Chabrias. . . .

\par 40 De la cinquième tribu, Isaac, Jacob, Jésus, Sabbatée, Simon, Lévi.

\par 41 De la sixième tribu, Judas, Joseph, Simon, Zacharie, Samuel, Sélémas.

\par 42 De la septième tribu, Sabbatée, Sédécias, Jacob, Isaac, Jesias, Natthée.

\par 43 De la huitième tribu, Théodose, Jason, Jésus, Théodote, Jean, Jonathan.

\par 44 De la neuvième tribu : Théophile, Abraham, Arsamos, Jason, Endemias, Daniel.

\par 45 De la dixième tribu, Jérémie, Éléazar, Zacharie, Baneas, Élisée, Dathaeus.

\par 46 De la onzième tribu, Samuel, Joseph, Judas, Jonathes, Chabu, Dositheus.

\par 47 De la douzième tribu, Isaël, Jean, Théodose, Arsamos, Abietes, Ezéchiel.

\par 48 Ils étaient soixante-douze en tout. Telle fut la réponse qu'Éléazar et ses amis donnèrent à la lettre du roi.

\chapter{3}

\par \textit{Dans lequel est décrite la table la plus exquise et la plus belle jamais produite. Et d'autres riches cadeaux, intéressants à la lumière des récentes fouilles en Égypte.}

\par 1 Je vais maintenant procéder au rachat de ma promesse et donner une description des œuvres d'art.

\par 2 Ils étaient travaillés avec une habileté exceptionnelle, car le roi n'épargnait aucune dépense et surveillait personnellement les ouvriers individuellement.

\par 3 Ils ne pouvaient donc dérober aucune partie de l'ouvrage ni l'achever avec négligence.

\par 4 Tout d'abord je vais vous donner une description du tableau.

\par 5 Le roi tenait à ce que cet ouvrage soit d'une dimension exceptionnellement grande, et il fit s'enquérir auprès des Juifs de la localité au sujet de la dimension de la table déjà dans le temple de Jérusalem.

\par 6 Et quand ils décrivèrent les mesures, il se demanda s'il pouvait faire une structure plus grande.

\par 7 Et certains des prêtres et les autres Juifs répondirent que rien ne l'en empêchait.

\par 8 Et il dit qu'il avait hâte de le rendre cinq fois plus grand, mais il hésitait de peur qu'il ne se révèle inutile pour les services du temple.

\par 9 Il désirait que son présent ne soit pas simplement placé dans le temple, car cela lui procurerait un bien plus grand plaisir si les hommes dont le devoir était d'offrir les sacrifices appropriés pouvaient le faire de manière appropriée sur la table qu'il avait fait.

\par 10 Il ne supposait pas que c'était à cause du manque d'or que l'ancienne table avait été faite de petite taille, mais il semble y avoir eu, dit-il, une raison pour laquelle elle était faite de cette dimension.

\par 11 Car si l'ordre avait été donné, les moyens n'auraient pas manqué.

\par 12 C'est pourquoi nous ne devons pas transgresser ni dépasser la juste mesure.

\par 13 En même temps, il leur ordonna de mettre en service toutes les diverses formes d'art, car il était un homme des conceptions les plus élevées et la nature l'avait doté d'une imagination vive qui lui permettait de se représenter l'apparence qui serait présenté par l’œuvre finie.

\par 14 Il a également donné l'ordre que là où il n'y avait aucune instruction dans les Écritures juives, tout soit rendu aussi beau que possible.

\par 15 Lorsque de telles instructions étaient établies, elles devaient être exécutées à la lettre.

\par 16 Ils firent la table avec deux coudées de longueur, une coudée de largeur et une coudée et demie en or massif et pur.

\par 17 Ce que je décris n'était pas de l'or mince posé sur une autre fondation, mais toute la structure était d'or massif soudé ensemble.

\par 18 Et ils firent tout autour une bordure de la largeur d'une main.

\par 19 Et il y avait une couronne de vagues, gravée en relief sous la forme de cordes merveilleusement travaillées sur ses trois côtés.

\par 20 Car il était de forme triangulaire et le style de l'ouvrage était exactement le même sur chacun des côtés, de sorte que quel que soit le côté où ils étaient tournés, ils présentaient le même aspect.

\par 21 Des deux côtés sous la bordure, celui qui descendait jusqu'à la table était une très belle œuvre, mais c'était le côté extérieur qui attirait le regard du spectateur.

\par 22 Or le bord supérieur des deux côtés, étant élevé, était aigu puisque, comme nous l'avons dit, le bord était à trois côtés, de quelque point de vue qu'on l'approchait.

\par 23 Et il y avait des couches de pierres précieuses dessus au milieu du cordage en relief, et elles étaient entrelacées les unes avec les autres par un dispositif artistique inimitable.

\par 24 Par souci de sécurité, ils étaient tous fixés par des aiguilles d'or qui étaient insérées dans des perforations des pierres.

\par 25 Sur les côtés, ils étaient serrés ensemble par des attaches pour les maintenir fermement.

\par 26 Sur la partie de la bordure autour de la table qui s'inclinait vers le haut et rencontrait les yeux, il y avait un motif d'œufs en pierres précieuses, minutieusement gravé par un morceau continu de relief cannelé, étroitement reliés entre eux autour de toute la table. .

\par 27 Et sous les pierres qui avaient été disposées pour représenter des œufs, les artistes firent une couronne contenant toutes sortes de fruits, ayant à son sommet des grappes de raisins et des épis de maïs, des dattes aussi et des pommes, des grenades et autres, bien visibles. .

\par 28 Ces fruits étaient façonnés avec des pierres précieuses, de la même couleur que les fruits eux-mêmes, et ils les attachaient sur tous les côtés de la table avec une bande d'or.

\par 29 Et après que la couronne de fruits eut été mise, en dessous fut inséré un autre modèle d'œufs en pierres précieuses, et d'autres ouvrages de cannelures et de repoussés, afin que les deux côtés de la table puissent être utilisés, selon les souhaits des propriétaires et c'est pour cette raison que le travail des vagues et la bordure furent prolongés jusqu'aux pieds de la table.

\par 30 Ils fabriquèrent et fixèrent sous toute la largeur de la table une plaque massive de quatre doigts d'épaisseur, pour que les pieds puissent y être insérés, et ils la fixèrent fermement avec des goupilles qui s'enfonçaient dans des douilles sous le bord, de sorte que de chaque côté de la table que les gens préfèrent, pourrait être utilisée.

\par 31 Ainsi, il est devenu manifestement clair que l'œuvre était destinée à être utilisée dans un sens ou dans l'autre.

\par 32 Sur la table elle-même, ils gravèrent un méandre, au milieu duquel se détachaient des pierres précieuses, des rubis, des émeraudes, un onyx et bien d'autres sortes de pierres qui excellent en beauté.

\par 33 Et à côté du «méandre», il y avait un merveilleux morceau de réseau, qui faisait apparaître le centre de la table comme un losange en forme de losange, et sur lequel avait été travaillé un cristal et de l'ambre, comme on l'appelle, ce qui a produit une impression incomparable sur les spectateurs.

\par 34 Ils firent les pieds de la table avec des têtes semblables à des lys, de sorte qu'ils ressemblaient à des lys penchés sous la table, et les parties visibles représentaient des feuilles dressées.

\par 35 La base du pied au sol était constituée d'un rubis et mesurait la largeur d'une main tout autour.

\par 36 Il avait l'apparence d'un soulier et avait huit doigts de large.

\par 37 Sur lui reposait toute l'étendue du pied.

\par 38 Et ils firent apparaître le pied comme du lierre sortant de la pierre, entrelacé d'akanthus et entouré d'une vigne qui l'entourait de grappes de raisin, qui étaient travaillées dans des pierres jusqu'au sommet du pied.

\par 39 Tous les quatre pieds étaient faits dans le même style, et tout était travaillé et ajusté si habilement, et une habileté et une connaissance si remarquables étaient dépensées pour le rendre fidèle à la nature, que lorsque l'air était agité par un souffle de vent, le mouvement était imprimé aux feuilles, et tout était façonné pour correspondre à la réalité réelle qu'il représentait.

\par 40 Et ils firent le dessus de la table en trois parties comme un triptyque, et elles étaient si bien ajustées et assemblées en queue d'aronde avec des embouts sur toute la largeur de l'ouvrage, que la rencontre des joints ne pouvait être vue ni même découverte.

\par 41 L'épaisseur de la table n'était pas inférieure à une demi-coudée, de sorte que tout l'ouvrage a dû coûter beaucoup de talents.

\par 42 Car comme le roi ne voulait pas augmenter sa taille, il dépensa pour les détails la même somme d'argent qui eût été nécessaire si la table avait pu être de plus grandes dimensions.

\par 43 Et tout fut achevé conformément à son plan, d'une manière des plus merveilleuses et des plus remarquables, avec un art inimitable et une beauté incomparable.

\par 44 Parmi les bols à mélanger, deux étaient travaillés en or, et de la base jusqu'au milieu étaient gravés d'un motif en relief en forme d'écailles, et entre les écailles des pierres précieuses étaient insérées avec une grande habileté artistique.

\par 45 Ensuite, il y avait un « méandre » d’une coudée de hauteur, avec sa surface sculptée de pierres précieuses de nombreuses couleurs, démontrant un grand effort artistique et une grande beauté.

\par 46 Il y avait dessus une mosaïque, travaillée en forme de losange, ayant l'apparence d'un filet et s'étendant jusqu'au bord.

\par 47 Au milieu, de petits boucliers faits de différentes pierres précieuses, placés alternativement et de nature variable, larges d'au moins quatre doigts, rehaussaient la beauté de leur apparence.

\par 48 Au sommet du bord, il y avait un ornement de lys en fleurs, et des grappes de raisins entrelacées étaient gravées tout autour.

\par 49 Telle était donc la construction des coupes d'or, et elles contenaient chacune plus de deux sapins.

\par 50 Les bols d'argent avaient une surface lisse et étaient merveilleusement faits comme s'ils étaient destinés à des miroirs, de sorte que tout ce qui s'approchait d'eux se reflétait encore plus clairement que dans les miroirs.

\par 51 Mais il est impossible de décrire l'impression réelle que ces œuvres d'art produisirent sur l'esprit une fois achevées.

\par 52 Car, lorsque ces vases furent achevés et placés côte à côte, d'abord un bol d'argent, puis un bol d'or, puis un autre d'argent, puis un autre d'or, l'aspect qu'ils présentaient est tout à fait indescriptible, et ceux qui venaient les voir n'étaient pas capables de s'arracher à ce spectacle brillant et envoûtant.

\par 53 Les impressions produites par le spectacle étaient de diverses natures.

\par 54 Quand les hommes regardaient les vases d'or, et que leur esprit faisait un aperçu complet de chaque détail de l'ouvrage, leurs âmes étaient ravies d'émerveillement.

\par 55 De plus, lorsqu'un homme voulait diriger son regard vers les vases d'argent, alors qu'ils se tenaient devant lui, tout semblait briller de lumière autour de l'endroit où il se tenait, et procurait un plaisir encore plus grand aux spectateurs.

\par 56 De sorte qu'il est vraiment impossible de décrire la beauté artistique des œuvres.

\par 57 Ils gravèrent au centre les coupes d'or avec des couronnes de vigne.

\par 58 Et autour des bords, ils tissaient une couronne de lierre, de myrte et d'olivier en relief et y inséraient des pierres précieuses.

\par 59 Les autres parties des reliefs, ils les firent selon des modèles différents, car ils mettaient un point d'honneur à tout achever d'une manière digne de la majesté du roi.

\par 60 En un mot, on peut dire que ni dans le trésor du roi ni dans aucun autre, il n'y avait aucune œuvre qui les égalait en coût ou en habileté artistique.

\par 61 Car le roi n'y prêtait pas peu d'attention, car il aimait à se glorifier pour l'excellence de ses desseins.

\par 62 Car souvent il négligeait ses affaires officielles et passait son temps avec les artistes dans son souci qu'ils terminent tout d'une manière digne du lieu où les cadeaux devaient être envoyés.

\par 63 Ainsi tout se fit en grand, d'une manière digne du roi qui envoyait les présents et du grand prêtre qui était le chef du pays.

\par 64 Il n'y avait pas de production de pierres précieuses, car on n'en utilisait pas moins de cinq mille, et elles étaient toutes de grande taille.

\par 65 L'habileté artistique la plus exceptionnelle était employée, de sorte que le coût des pierres et du travail était cinq fois plus élevé que celui de l'or.

\par \textit{Notes de bas de page}

\par \textit{148:1 Une coudée équivaut à 18 pouces.}

\chapter{4}

\par \textit{Détails éclatants du sacrifice. La précision infaillible des prêtres est remarquable. Une orgie sauvage. Une description du temple et de ses installations hydrauliques.}

\par 1 Je vous ai donné cette description des cadeaux parce que je pensais que c'était nécessaire.

\par 2 Le point suivant du récit est le récit de notre voyage à Éléazar, mais je vais d'abord vous donner une description de tout le pays.

\par 3 Lorsque nous sommes arrivés au pays des Juifs, nous avons vu la ville située au milieu de toute la Judée, au sommet d'une montagne d'une altitude considérable.

\par 4 Au sommet le temple avait été bâti dans toute sa splendeur.

\par 5 Elle était entourée de trois murs de plus de soixante-dix coudées de haut et en longueur et en largeur correspondant à la structure de l'édifice.

\par 6 Tous les bâtiments étaient caractérisés par une magnificence et une valeur tout à fait sans précédent.

\par 7 Il était évident qu'aucune dépense n'avait été épargnée pour la porte et les fixations qui la reliaient aux montants de la porte, ainsi que pour la stabilité du linteau.

\par 8 Le style du rideau était également tout à fait proportionné à celui de l'entrée.

\par 9 Son tissu, à cause du courant d'air, était en perpétuel mouvement, et comme ce mouvement se communiquait par le bas et que le rideau se déployait jusqu'à son point le plus élevé, il offrait un spectacle agréable dont un homme pouvait à peine s'arracher.

\par 10 La construction de l'autel était en harmonie avec le lieu lui-même et avec les holocaustes qui y étaient consumés par le feu, et l'approche de celui-ci était d'une dimension similaire.

\par 11 Il y avait une pente graduelle qui y montait, convenablement aménagée dans un but de décence, et les prêtres au service étaient vêtus de vêtements de lin jusqu'aux chevilles.

\par 12 Le Temple est tourné vers l'est et son dos est vers l'ouest.

\par 13 Tout le sol est pavé de pierres et descend aux endroits désignés, afin que l'eau puisse être amenée pour laver le sang des sacrifices, car plusieurs milliers d'animaux y sont sacrifiés les jours de fête.

\par 14 Et il y a une réserve d'eau inépuisable, car une source naturelle abondante jaillit de l'intérieur du temple.

\par 15 Il y a en outre des citernes merveilleuses et indescriptibles sous terre, comme ils me l'ont fait remarquer, à une distance de cinq stades tout autour de l'emplacement du temple, et chacune d'elles a d'innombrables tuyaux pour que les différents ruisseaux convergent ensemble.

\par 16 Et tout cela était fixé avec du plomb au fond et sur les parois latérales, et dessus une grande quantité de plâtre avait été étalée, et chaque partie du travail avait été exécutée avec le plus grand soin.

\par 17 Il y a de nombreuses ouvertures pour l'eau au pied de l'autel qui sont invisibles pour tous sauf pour ceux qui sont occupés au ministère, de sorte que tout le sang des sacrifices qui est recueilli en grande quantité est lavé en un clin d'œil.

\par 18 Telle est mon opinion quant au caractère des réservoirs et je vais maintenant vous montrer comment elle a été confirmée.

\par 19 Ils m'ont conduit à plus de quatre stades hors de la ville et m'ont ordonné de regarder vers un certain endroit et d'écouter le bruit qui était fait par la rencontre des eaux, de sorte que la grande taille des réservoirs m'est apparue clairement comme cela a déjà été souligné.

\par 20 Le ministère des prêtres est en tous points inégalé, tant par son endurance physique que par son service ordonné et silencieux.

\par 21 Car ils travaillent tous spontanément, même si cela exige un effort très pénible, et chacun a une tâche spéciale qui lui est assignée.

\par 22 Le service se fait sans interruption : les uns fournissent le bois, les autres l'huile, d'autres la fine farine de blé, d'autres les épices ; d'autres encore apportent les morceaux de chair pour l'holocauste, faisant preuve d'un degré de force merveilleux.

\par 23 Car ils prennent à deux mains les membres d'un veau, chacun pesant plus de deux talents, et les jettent de chaque main d'une manière merveilleuse sur le haut lieu de l'autel et ne manquent jamais de les placer sur le endroit approprié.

\par 24 De la même manière, les morceaux de mouton et aussi de chèvre sont merveilleux tant par leur poids que par leur graisse.

\par 25 Car ceux dont c'est l'affaire choisissent toujours les bêtes sans défaut et particulièrement grasses, et c'est ainsi que s'accomplit le sacrifice que j'ai décrit.

\par 26 Il y a un endroit spécialement réservé pour leur repos, où s'assoient ceux qui sont relevés du service.

\par 27 Lorsque cela arrive, ceux qui se sont déjà reposés et sont prêts à reprendre leurs fonctions se lèvent spontanément puisqu'il n'y a personne pour donner des ordres quant à la disposition des sacrifices.

\par 28 Le silence le plus complet règne, de sorte qu'on pourrait croire qu'il n'y avait pas une seule personne présente, alors qu'il y a en réalité sept cents hommes occupés à l'ouvrage, sans compter le grand nombre de ceux qui sont occupés à monter les sacrifices.

\par 29 Tout est fait avec révérence et d'une manière digne du grand Dieu.

\par 30 Nous avons été très étonnés, lorsque nous avons vu Éléazar occupé au ministère, de la façon dont il était habillé et de la majesté de son apparence, qui se révélait dans la robe qu'il portait et les pierres précieuses sur sa personne.

\par 31 Il y avait des clochettes d'or sur le vêtement qui descendait jusqu'à ses pieds, produisant une sorte de mélodie particulière, et des deux côtés il y avait des grenades avec des fleurs panachées d'une teinte merveilleuse.

\par 32 Il était ceint d'une ceinture d'une beauté remarquable, tissée des plus belles couleurs.

\par 33 Il portait sur sa poitrine l'oracle de Dieu, comme on l'appelle, sur lequel étaient incrustées douze pierres de différentes sortes, liées ensemble avec de l'or, contenant les noms des chefs des tribus, selon leur ordre originel. , chacun exhibant d'une manière indescriptible sa couleur particulière.

\par 34 Sur sa tête, il portait une tiare, comme on l'appelle, et sur celle-ci, au milieu de son front, un turban inimitable, le diadème royal plein de gloire avec le nom de Dieu inscrit en lettres sacrées sur une plaque d'or. . . ayant été jugé digne de porter ces emblèmes dans les ministères.

\par 35 Leur apparition créait une telle crainte et une telle confusion d'esprit qu'on avait l'impression que l'on était entré en présence d'un homme qui appartenait à un monde différent.

\par 36 Je suis convaincu que quiconque prendra part au spectacle que j'ai décrit sera rempli d'un étonnement et d'un émerveillement indescriptible et sera profondément touché dans son esprit à la pensée de la sainteté qui s'attache à chaque détail du service.

\par 37 Mais pour avoir des renseignements complets, nous montâmes au sommet de la citadelle voisine et regardâmes autour de nous.

\par 38 Il est situé dans un endroit très élevé et est fortifié par de nombreuses tours qui ont été construites jusqu'au sommet avec d'immenses pierres, dans le but, comme nous l'avons appris, de garder l'enceinte du temple, afin que s'il y avait une attaque, ou une insurrection ou un assaut de l'ennemi, personne ne pourrait forcer l'entrée à l'intérieur des murs qui entourent le temple.

\par 39 Sur les tours de la citadelle étaient placés des engins de guerre et différentes sortes de machines, et la position était beaucoup plus élevée que le cercle de murs dont j'ai parlé.

\par 40 Les tours étaient également gardées par les hommes les plus fidèles qui avaient donné la plus grande preuve de leur loyauté envers leur pays.

\par 41 ces hommes n'étaient jamais autorisés à sortir de la citadelle, sauf les jours de fête et alors seulement par détachements, et ils ne permettaient pas non plus à aucun étranger d'y entrer.

\par 42 Ils étaient également très prudents lorsqu'un ordre venait de l'officier en chef d'admettre des visiteurs pour inspecter les lieux, comme notre propre expérience nous l'a appris.

\par 43 Ils étaient très réticents à nous admettre, bien que nous n'étions que deux hommes sans armes, pour assister à l'offrande des sacrifices.

\par 44 Et ils affirmèrent qu'ils étaient liés par un serment lorsque le dépôt leur était confié, car ils avaient tous juré et étaient tenus d'exécuter sacrément le serment à la lettre, que bien qu'ils soient au nombre de cinq cents, ils ne permettraient pas plus que cinq hommes à entrer en même temps.

\par 45 La citadelle était la protection spéciale du temple et son fondateur l'avait fortifiée si fortement qu'elle pouvait la protéger efficacement.

\chapter{5}

\par \textit{Une description de la ville et de la campagne. Comparez le verset 11 avec les conditions d’aujourd’hui. Les versets 89 à 41 révèlent comment les anciens estimaient un érudit et un gentleman.}

\par 1 LA taille de la ville est de dimensions modérées.

\par 2 Il a environ quarante stades 1 de circonférence, autant qu'on puisse le conjecturer.

\par 3 Elle a ses tours disposées en forme de théâtre, avec des voies de circulation menant entre elles désormais les carrefours des tours inférieures sont visibles mais ceux des tours supérieures sont plus fréquentés.

\par 4 Car la terre monte, puisque la ville est bâtie sur une montagne.

\par 5 Il y a aussi des marches qui mènent au carrefour, et certains montent toujours, et d'autres descendent et ils se tiennent le plus loin possible les uns des autres sur la route à cause de ceux qui sont liés par les règles de pureté. , de peur qu’ils ne touchent à quelque chose d’illégal.

\par 6 Ce n'est pas sans raison que les premiers fondateurs de la ville l'ont construite dans les proportions voulues, car ils possédaient une vision claire des besoins.

\par 7 Car le pays est vaste et beau.

\par 8 Certaines parties sont plates, notamment les districts qui appartiennent à la Samarie, comme on l'appelle, et qui bordent le pays des Iduméens, d'autres parties sont montagneuses, surtout celles qui sont contiguës au pays de Judée.

\par 9 Les gens sont donc tenus de se consacrer à l'agriculture et à la culture du sol afin d'avoir ainsi des récoltes abondantes.

\par 10 De cette manière, on cultive toutes sortes de cultures et on récolte une récolte abondante dans tout le pays susmentionné.

\par 11 Les villes qui sont grandes et jouissent d'une prospérité correspondante sont bien peuplées, mais elles négligent les campagnes, car tous les hommes sont enclins à une vie de plaisir, car chacun a une tendance naturelle à rechercher le plaisir.

\par 12 La même chose s'est produite à Alexandrie, qui surpasse toutes les villes en taille et en prospérité.

\par 13 Les ruraux, en quittant les campagnes et en s'installant dans la ville, jetèrent le discrédit sur l'agriculture ; et pour les empêcher de s'établir dans la ville, le roi ordonna qu'ils n'y séjournent pas plus de vingt jours. 2

\par 14 Et de la même manière, il donna des instructions écrites aux juges, selon lesquelles s'il était nécessaire de délivrer une assignation contre quelqu'un qui habitait dans la campagne, l'affaire devait être réglée dans les cinq jours.

\par 15 Et comme il considérait la question comme étant d'une grande importance, il nomma également des officiers légaux pour chaque district avec leurs assistants, afin que les agriculteurs et leurs avocats ne puissent pas, dans l'intérêt des affaires, vider les greniers de la ville, je veux dire, de le produit de l'élevage.

\par 16 J'ai permis cette digression parce que c'est Éléazar qui a souligné avec une grande clarté les points qui ont été mentionnés.

\par 17 Car grande est l'énergie qu'ils dépensent pour le travail du sol.

\par 18 Car le pays est couvert d'une multitude d'oliviers, de blé et de légumes secs, de vignes, et le miel est abondant.

\par 19 Les autres espèces d'arbres fruitiers et les dattes ne comptent pas à côté de ceux-ci.

\par 20 Il y a du bétail de toute espèce en grande quantité et de riches pâturages pour lui.

\par 21 C'est pourquoi ils reconnaissent avec raison que les campagnes ont besoin d'une population nombreuse, et que les relations entre la ville et les villages sont convenablement réglées.

\par 22 Une grande quantité d'épices, de pierres précieuses et d'or est introduite dans le pays par les Arabes.

\par 23 Car le pays est bien adapté non seulement à l'agriculture mais aussi au commerce, et la ville est riche en arts et ne manque d'aucune des marchandises qui traversent la mer.

\par 24 Elle possède des ports trop convenables et commodes à Askalon, Joppé et Gaza, ainsi qu'à Ptolémaïs, qui a été fondée par le roi et occupe une position centrale par rapport aux autres lieux nommés, n'étant pas loin d'aucun d'eux.

\par 25 Le pays produit de tout en abondance, puisqu'il est bien arrosé de toutes parts et bien protégé des tempêtes.

\par 26 Le fleuve Jourdain, comme on l'appelle, qui ne tarit jamais, traverse le pays.

\par 27 A l'origine, le pays ne contenait pas moins de 60 millions d'acres — bien que par la suite les peuples voisins firent des incursions contre lui — et 600 000 hommes y étaient installés dans des fermes de cent acres chacune.

\par 28 Le fleuve, comme le Nil, monte au moment des récoltes et irrigue une grande partie des terres.

\par 29 Près du territoire des habitants de Ptolémaïs, elle se jette dans un autre fleuve, qui se jette dans la mer.

\par 30 D'autres torrents de montagne, comme on les appelle, coulent dans la plaine et englobent les environs de Gaza et le district d'Ashdod.

\par 31 Le pays est encerclé par une clôture naturelle et est très difficile à attaquer et ne peut pas être assailli par de grandes forces, en raison des passages étroits, des précipices surplombant les précipices et des ravins profonds, et du caractère accidenté des régions montagneuses qui entourent tout la terre.

\par 32 On nous a dit qu'on tirait autrefois du cuivre et du fer des montagnes voisines de l'Arabie.

\par 33 Ceci fut cependant stoppé à l'époque de la domination perse, car les autorités de l'époque répandirent à l'étranger une fausse nouvelle selon laquelle l'exploitation des mines était inutile et coûteuse, afin d'empêcher que leur pays ne soit détruit par l'exploitation minière dans ces districts et peut-être leur furent retirés à cause de la domination perse, car grâce à ce faux rapport, ils trouvèrent une excuse pour entrer dans ce district.

\par 34 Mon cher frère Philocrate, je vous ai maintenant donné, sous une forme brève, toutes les informations essentielles à ce sujet.

\par 35 Je décrirai le travail de traduction dans la suite.

\par 36 Le Grand Prêtre sélectionnait des hommes dotés du meilleur caractère et de la plus haute culture, tels qu'on peut s'y attendre de leurs nobles parents.

\par 37 C'étaient des hommes qui non seulement avaient acquis la maîtrise de la littérature juive, mais qui avaient également étudié avec le plus grand soin celle des Grecs.

\par 38 Ils étaient donc spécialement qualifiés pour servir dans les ambassades et ils assumaient ce devoir chaque fois que cela était nécessaire.

\par 39 Ils possédaient une grande facilité pour les conférences et la discussion des problèmes liés au droit.

\par 40 Ils ont adopté la voie du milieu, et c'est toujours la meilleure voie à suivre.

\par 41 Ils abjuraient les manières rudes et grossières, mais ils étaient tout à fait au-dessus de l'orgueil et ne prenaient jamais un air de supériorité sur les autres, et dans la conversation ils étaient prêts à écouter et à donner une réponse appropriée à chaque question.

\par 42 Et tous observaient attentivement cette règle et étaient soucieux par-dessus tout de se surpasser les uns les autres dans son observance et ils étaient tous dignes de leur chef et de sa vertu.

\par 43 Et on pouvait observer combien ils aimaient Éléazar par leur refus de se laisser arracher à lui et combien il les aimait.

\par 44 Outre la lettre qu'il a écrite au roi concernant leur retour sain et sauf, il a également prié instamment Andreas de travailler dans le même but et m'a également exhorté à l'aider au mieux de mes capacités.

\par 45 Et bien que nous ayons promis d'accorder toute notre attention à cette affaire, il a dit qu'il était toujours très affligé, car il savait que le roi, par bonté de nature, considérait comme son plus grand privilège, chaque fois qu'il entendait parler d'un homme supérieur. à ses semblables en culture et en sagesse, pour le convoquer à sa cour.

\par 46 Car j'ai entendu parler de sa belle parole selon laquelle, en s'assurant auprès de lui des hommes justes et prudents, il assurerait la plus grande protection à son royaume, puisque de tels amis lui donneraient sans réserve les conseils les plus bénéfiques.

\par 47 Et les hommes qui lui étaient maintenant envoyés par Éléazar possédaient sans aucun doute ces qualités.

\par 48 Et il affirmait fréquemment sous serment qu'il ne laisserait jamais partir ces hommes si c'était simplement un intérêt privé qui constituait le motif le plus important, mais c'était pour l'avantage commun de tous les citoyens qu'il les envoyait.

\par 49 Car, expliquait-il, la bonne vie consiste dans l'observation des textes de la loi, et ce but s'obtient bien plus par l'écoute que par la lecture.

\par 50 De cette déclaration et d'autres déclarations similaires, il ressortait clairement quels étaient ses sentiments à leur égard.

\par \textit{Notes de bas de page}

\par \textit{154:1 Un stade équivaut à 1/8 de mile (soit 220 mètres).}

\par \textit{154:2 Ce récit des mesures adoptées à Alexandrie pour empêcher le dépeuplement des campagnes par les migrations vers la ville est une révélation intéressante car la question était aussi aiguë il y a 2000 ans qu'elle l'est aujourd'hui.}

\chapter{6}

\par \textit{Explications des coutumes du peuple montrant ce que signifie le mot « impur ». L’essence et l’origine de la « croyance en Dieu ». Les versets 48 à 44 donnent une description pittoresque de la Divinité de la physiologie.}

\par 1 Il convient de mentionner brièvement les informations qu'il a données en réponse à nos questions.

\par 2 Car je suppose que la plupart des gens éprouvent de la curiosité à l'égard de certaines dispositions de la loi, en particulier celles concernant les viandes, les boissons et les animaux reconnus comme impurs.

\par 3 Quand nous avons demandé pourquoi, puisqu'il n'y a qu'une seule forme de création, certains animaux sont considérés comme impurs à manger, et d'autres impurs même au toucher (car bien que la loi soit scrupuleuse sur la plupart des points, elle est particulièrement scrupuleuse sur des questions comme celles-ci) il a commencé sa réponse comme suit :

\par 4 « Vous voyez, dit-il, quel effet nos modes de vie et nos associations produisent sur nous ; en s'associant aux méchants, les hommes attrapent leurs dépravations et deviennent malheureux tout au long de leur vie ; mais s'ils vivent avec des sages et des prudents, ils trouvent le moyen d'échapper à l'ignorance et de modifier leur vie.

\par 5 Notre législateur a d'abord posé les principes de piété et de justice et les a inculqués point par point, non seulement par des interdictions mais aussi par l'utilisation d'exemples, démontrant les effets néfastes du péché et les châtiments infligés par Dieu aux coupable.

\par 6 Car il prouva tout d'abord qu'il n'y a qu'un seul Dieu et que sa puissance se manifeste dans tout l'univers, puisque chaque lieu est rempli de sa souveraineté et qu'aucune des choses qui s'opèrent en secret par les hommes sur la terre n'échappe à sa connaissance.

\par 7 Car tout ce qu'un homme fait et tout ce qui doit arriver dans l'avenir lui est manifeste.

\par 8 Élaborant soigneusement ces vérités et les ayant exposées clairement, il montra que même si un homme pensait à faire le mal, sans parler de le faire réellement, il n'échapperait pas à la détection, car il montra clairement que le pouvoir de Dieu imprégnait toute la loi.

\par 9 Partant de son point de départ, il a poursuivi en montrant que tous les hommes, sauf nous, croient à l'existence de nombreux dieux, bien qu'ils soient eux-mêmes beaucoup plus puissants que les êtres qu'ils adorent en vain.

\par 10 Car lorsqu'ils ont fait des statues de pierre et de bois, ils disent qu'elles sont les images de ceux qui ont inventé quelque chose d'utile à la vie et ils les adorent, bien qu'ils aient la preuve évidente qu'ils n'ont aucun sentiment.

\par 11 Car il serait tout à fait insensé de supposer que quelqu'un devienne un dieu en vertu de ses inventions.

\par 12 Car les inventeurs ont simplement pris certains objets déjà créés et, en les combinant entre eux, ont montré qu'ils possédaient une nouvelle utilité : ils n'ont pas eux-mêmes créé la substance de la chose, et c'est donc une chose vaine et insensée que les gens fassent des dieux des hommes comme eux.

\par 13 Car de nos jours, il y en a beaucoup qui sont beaucoup plus inventifs et beaucoup plus savants que les hommes d'autrefois qui ont été déifiés, et pourtant ils ne viendraient jamais les adorer.

\par 14 Les créateurs et les auteurs de ces mythes pensent qu'ils sont les plus sages des Grecs.

\par 15 Pourquoi avons-nous besoin de parler d'autres peuples infatués, Égyptiens et semblables, qui comptent sur les bêtes sauvages et sur la plupart des espèces de reptiles et de bétail, et les adorent et leur offrent des sacrifices tant vivants que morts ?

\par 16 Or, notre législateur, étant un homme sage et spécialement doté par Dieu pour comprendre toutes choses, a pris en compte chaque détail particulier et nous a entouré de remparts imprenables et de murs de fer, afin que nous ne puissions nous mêler du tout aux autres nations, mais restent purs de corps et d’âme, libres de toute vaine imagination, adorant le Dieu Tout-Puissant unique au-dessus de toute la création.

\par 17 C'est pourquoi les principaux prêtres égyptiens, après avoir examiné attentivement de nombreuses questions et connaissant nos affaires, nous appellent « hommes de Dieu ».

\par 18 C'est un titre qui n'appartient pas au reste des humains mais seulement à ceux qui adorent le vrai Dieu.

\par 19 Les autres ne sont pas des hommes de Dieu, mais des hommes de viande, de boisson et de vêtements.

\par 20 Car tout leur tempérament les amène à trouver du réconfort dans ces choses qui ne comptent pas, mais dans toutes leurs choses.

\par 21 Parmi notre peuple, toute sa vie, sa principale considération est la souveraineté de Dieu.

\par 22 C'est pourquoi, afin que nous ne soyons pas corrompus par une abomination quelconque, ou que nos vies ne soient perverties par de mauvaises communications, il nous a entourés de tous côtés par des règles de pureté, affectant également ce que nous mangeons, ou buvons, ou touchons, ou entendons, ou voir.

\par 23 Car, bien que, d'une manière générale, toutes choses soient semblables dans leur constitution naturelle, puisqu'elles sont toutes gouvernées par une seule et même puissance, il y a cependant une raison profonde dans chaque cas individuel pour laquelle nous nous abstenons de l'usage de certaines choses et profiter de l’usage commun des autres.

\par 24 Par souci d'illustration, je vais passer en revue un ou deux points et vous les expliquer.

\par 25 Car vous ne devez pas tomber dans l'idée dégradante que c'est par égard pour les souris, les belettes et autres choses semblables que Moïse a rédigé ses lois avec un soin si extrême. 1

\par 26 Toutes ces ordonnances ont été faites dans un souci de justice pour aider à la quête de la vertu et au perfectionnement du caractère.

\par 27 Car tous les oiseaux que nous utilisons sont apprivoisés et distingués par leur propreté, se nourrissant de diverses espèces de céréales et de légumineuses, comme par exemple les pigeons, les tourterelles, les sauterelles, les perdrix, les oies aussi, et tous les autres oiseaux de cette classe.

\par 28 Mais les oiseaux qui sont interdits, vous les trouverez sauvages et carnivores, tyrannisant les autres par la force qu'ils possèdent, et obtenant cruellement de la nourriture en s'attaquant aux oiseaux apprivoisés énumérés ci-dessus.

\par 29 Et non seulement cela, mais ils s'emparent des agneaux et des chevreaux, et blessent aussi les êtres humains, morts ou vivants, et ainsi, en les qualifiant d'impurs, il a donné par eux un signe que ceux pour lesquels la législation a été ordonnée, doivent pratiquer la droiture dans leur cœur et ne tyranniser personne en s'appuyant sur leurs propres forces ni leur voler quoi que ce soit, mais diriger leur vie conformément à la justice, tout comme les oiseaux apprivoisés, déjà mentionnés, consomment le différentes sortes de légumineuses qui poussent sur la terre et ne tyrannisent pas jusqu'à la destruction de leurs propres parents.

\par 30 Notre législateur nous a donc enseigné que c'est par de telles méthodes que l'on donne aux sages des indications, qu'ils doivent être justes et ne rien faire par la violence, et s'abstenir de tyranniser les autres en s'appuyant sur leur propre force.

\par 31 Car puisqu'il est considéré comme inconvenant même de toucher des animaux aussi impurs, comme nous l'avons mentionné, à cause de leurs habitudes particulières, ne devrions-nous pas prendre toutes les précautions pour que notre propre caractère ne soit pas détruit dans la même mesure ?

\par 32 C'est pourquoi toutes les règles qu'il a établies concernant ce qui est permis dans le cas de ces oiseaux et autres animaux, il les a édictées dans le but de nous enseigner une leçon de morale.

\par 33 Car la division du sabot et la séparation des griffes sont destinées à nous apprendre que nous devons distinguer nos actions individuelles en vue de la pratique de la vertu.

\par 34 Car la force de tout notre corps et son activité dépendent de nos épaules et de nos membres.

\par 35 C'est pourquoi il nous oblige à reconnaître que nous devons accomplir toutes nos actions avec discernement selon les normes de la justice, - plus particulièrement parce que nous avons été distinctement séparés du reste de l'humanité.

\par 36 Car la plupart des autres hommes se souillent par la promiscuité, commettant ainsi de grandes iniquités, et des pays et des villes entières se glorifient de tels vices.

\par 37 Car non seulement ils ont des relations sexuelles avec des hommes, mais ils souillent leurs propres mères et même leurs filles.

\par 38 Mais nous avons été tenus à l'écart de tels péchés.

\par 39 Et les personnes qui ont été séparées de la manière susmentionnée sont également caractérisées par le Législateur comme possédant le don de mémoire.

\par 40 Car tous les animaux « qui ont les pieds fourchus et qui ruminent » représentent aux initiés le symbole de la mémoire.

\par 41 Car l'acte de ruminer n'est rien d'autre que la réminiscence de la vie et de l'existence.

\par 42 Car la vie est habituellement soutenue par la nourriture, c'est pourquoi il nous exhorte également dans l'Écriture en ces termes : « Tu te souviendras sûrement du Seigneur qui a opéré en toi ces choses grandes et merveilleuses. »

\par 43 Car lorsqu’ils sont correctement conçus, ils sont manifestement grands et glorieux ; d'abord la construction du corps et la disposition des aliments et la séparation de chaque membre individuel et, plus encore, l'organisation des sens, le fonctionnement et le mouvement invisible de l'esprit, la rapidité de ses actions particulières et sa découverte des arts, témoignent d'une infinie ingéniosité.

\par 44 C'est pourquoi il nous exhorte à nous rappeler que les parties susmentionnées sont maintenues ensemble par la puissance divine avec une habileté consommée.

\par 45 Car il a marqué chaque temps et chaque lieu pour que nous puissions continuellement nous souvenir du Dieu qui nous gouverne et nous préserve.

\par 46 Car, en ce qui concerne les viandes et les boissons, il nous ordonne d'abord d'en offrir une partie en sacrifice, puis de prendre immédiatement notre repas.

\par 47 De plus, il nous a donné sur nos vêtements un symbole de souvenir, et de la même manière il nous a ordonné de mettre les oracles divins sur nos portes et nos portes en souvenir de Dieu.

\par 48 Et sur nos mains aussi, il ordonne expressément que le symbole soit attaché, montrant clairement que nous devons accomplir chaque acte avec justice, en nous souvenant de notre propre création et par-dessus tout de la crainte de Dieu.

\par 49 Il ordonne aussi aux hommes, lorsqu'ils se couchent pour dormir et se relèvent, de méditer sur les œuvres de Dieu, non seulement en paroles, mais en observant distinctement le changement et l'impression produits sur eux lorsqu'ils s'endorment, et aussi leur réveil, combien est divin et incompréhensible le passage de l'un de ces états à l'autre.

\par 50 L'excellence de l'analogie en ce qui concerne la discrimination et la mémoire vous a maintenant été soulignée, selon notre interprétation du « sabot fendu et de la rumination ».

\par 51 Car nos lois n'ont pas été élaborées au hasard ou selon la première pensée fortuite venue à l'esprit, mais en vue de la vérité et de l'indication de la juste raison.

\par 52 Car, par les instructions qu'il donne concernant les viandes et les boissons et les cas particuliers d'attouchements, il nous ordonne de ne rien faire ni d'écouter inconsidérément, ni de recourir à l'injustice par abus de la puissance de la raison.

\par 53 Dans le cas des animaux sauvages aussi, le même principe peut être découvert.

\par 54 Car le caractère de la belette, des souris et des animaux de ce genre, qui sont expressément mentionnés, est destructeur.

\par 55 Les souris souillent et endommagent tout, non seulement pour leur propre nourriture, mais même au point de rendre absolument inutile à l'homme tout ce qu'elles peuvent endommager.

\par 56 La classe des belettes est également particulière : car outre ce qui a été dit, elle a un caractère souillé : elle conçoit par les oreilles et met bas par la bouche.

\par 57 Et c'est pour cette raison qu'une pratique semblable est déclarée impure chez les hommes.

\par 58 Car en incarnant dans la parole tout ce qu'ils reçoivent par les oreilles, ils entraînent les autres dans le mal et ne commettent aucune impureté ordinaire, étant eux-mêmes entièrement souillés par la pollution de l'impiété.

\par 59 Et votre roi, comme nous le savons, a tout à fait raison de détruire de tels hommes.

\par 60 Puis j'ai dit : «Je suppose que vous voulez dire les informateurs, car il les expose constamment à des tortures et à des formes douloureuses de mort.»

\par 61 « Oui, répondit-il, ce sont les hommes dont je parle ; car veiller à la destruction des hommes est une chose impie ».

\par 62 Et notre loi nous défend de nuire à qui que ce soit, ni en paroles, ni en actes.

\par 63 Mon bref exposé de ces questions aurait dû vous convaincre que tous nos règlements ont été rédigés en vue de la justice, et que rien n'a été édicté dans l'Écriture sans réfléchir ni sans raison, mais son but est de permettre nous, tout au long de notre vie et dans toutes nos actions, à pratiquer la justice devant tous les hommes, en nous souvenant du Dieu Tout-Puissant.

\par 64 Et ainsi concernant les viandes et les choses impures, les reptiles et les bêtes sauvages, tout le système vise la justice et les relations justes entre l'homme et l'homme.

\par 65 Il m'a semblé avoir fait une bonne défense sur tous les points ; car en référence également aux veaux, aux béliers et aux chèvres qui sont offerts, il dit qu'il était nécessaire de les prendre parmi les troupeaux et les troupeaux, de sacrifier des animaux apprivoisés et de n'offrir rien de sauvage, afin que ceux qui offrent les sacrifices puissent comprendre la signification symbolique du législateur et ne pas être sous l'influence d'une conscience de soi arrogante.

\par 66 Car celui qui offre un sacrifice, fait aussi une offrande de sa propre âme dans toutes ses humeurs.

\par 67 Je pense que ces détails concernant notre discussion méritent d'être racontés, et à cause du caractère sacré et du sens naturel de la loi, j'ai été amené à vous les expliquer clairement, Philocrate, à cause de votre propre dévouement à l'étude. .

\par \textit{Notes de bas de page}

\par \textit{158:1 Comparez cette idée pittoresque avec 1 Corinthiens, IX, 9.}

\chapter{7}

\par \textit{L'arrivée des envoyés avec le manuscrit du précieux livre et des cadeaux. Préparatifs d'un banquet royal. Dès qu'il est assis à table, l'hôte divertit ses invités avec des questions et des réponses. Quelques commentaires sages sur la sociologie.}

\par 1 ET Éléazar, après avoir offert le sacrifice, choisi les envoyés et préparé de nombreux présents pour le roi, nous envoya en voyage en grande sécurité.

\par 2 Et lorsque nous arrivâmes à Alexandrie, le roi fut aussitôt informé de notre arrivée.

\par 3 Lors de notre admission au palais, Andreas et moi avons chaleureusement salué le roi et lui avons remis la lettre écrite par Éléazar.

\par 4 Le roi était très désireux de rencontrer les envoyés, et ordonna que tous les autres fonctionnaires soient renvoyés et que les envoyés soient convoqués immédiatement devant lui.

\par 5 Or, cela suscita la surprise générale, car il est d'usage que ceux qui viennent chercher une audience avec le roi sur des questions importantes soient admis en sa présence le cinquième jour, tandis que les envoyés des rois ou des villes très importantes ont peine à obtenir admis à la Cour dans trente jours - mais il jugeait ces hommes dignes d'un plus grand honneur, car il tenait leur maître en si haute estime, et c'est pourquoi il renvoya immédiatement ceux dont il considérait la présence comme superflue et continua à se promener jusqu'à ce qu'ils entrent et il a pu les accueillir.

\par 6 Lorsqu'ils entrèrent avec les cadeaux qui avaient été envoyés avec eux et les précieux parchemins sur lesquels la loi était inscrite en or en caractères juifs, car le parchemin était merveilleusement préparé et la connexion entre les pages avait été effectuée de manière à être invisible, le roi dès qu'il les vit commença à leur poser des questions sur les livres.

\par 7 Et quand ils eurent retiré les rouleaux de leurs couvertures et déplié les pages, le roi resta immobile pendant un long moment, puis se prosternant environ sept fois, il dit :

\par 8 'Je vous remercie, mes amis, et je remercie celui qui vous a envoyé encore plus, et surtout Dieu, à qui sont ces oracles.'

\par 9 Et quand tous, les envoyés et les autres qui étaient présents aussi, crièrent en même temps et d'une seule voix : « Dieu sauve le roi ! il fondit en larmes de joie.

\par 10 L'exaltation de son âme et le sentiment de l'immense honneur qui lui avait été rendu le contraignaient à pleurer sur sa bonne fortune.

\par 11 Il leur ordonna de remettre les rouleaux à leur place, puis après avoir salué les hommes, il dit : « Il était juste, hommes de Dieu, que je rende avant tout mon respect aux livres pour lesquels je Je vous ai convoqué ici et là, après avoir fait cela, pour vous tendre la main droite de l'amitié.

\par 12 'C'est pour cette raison que j'ai fait cela en premier.'

\par 13 'J'ai décrété que ce jour où vous êtes arrivé sera considéré comme un grand jour et qu'il sera célébré chaque année tout au long de ma vie.'

\par 14 « Il se trouve aussi que c'est l'anniversaire de ma victoire navale sur Antigone. C'est pourquoi je serai heureux de festoyer avec vous aujourd'hui.

\par 15 'Tout ce dont vous pourrez avoir besoin,' dit-il, 'sera préparé pour vous d'une manière convenable et pour moi aussi avec vous.'

\par 16 Après qu'ils eurent exprimé leur joie, il ordonna qu'on leur assignât les meilleurs quartiers près de la citadelle, et qu'on fît les préparatifs pour le banquet.

\par 17 Et Nicanor appela le seigneur grand intendant Dorothée, qui était l'officier spécial chargé de veiller sur les Juifs, et lui ordonna de faire les préparatifs nécessaires pour chacun.

\par 18 Car cet arrangement avait été pris par le roi et c'est un arrangement que vous voyez maintenu aujourd'hui.

\par 19 Car toutes les villes qui ont des coutumes particulières en matière de boire, de manger et de se coucher, ont des officiers spéciaux nommés pour veiller à leurs besoins.

\par 20 Et chaque fois qu'ils viennent rendre visite aux rois, les préparatifs sont faits conformément à leurs propres coutumes, afin qu'il n'y ait aucun inconvénient qui puisse perturber la jouissance de leur visite.

\par 21 La même précaution fut prise dans le cas des envoyés juifs.

\par 22 Or Dorothée, qui était le patron chargé de s'occuper des invités juifs, était un homme très consciencieux.

\par 23 Il fit sortir pour la fête tous les magasins qui étaient sous son contrôle et réservés à la réception de ces invités.

\par 24 Il disposa les sièges sur deux rangées, conformément aux instructions du roi.

\par 25 Car il lui avait ordonné de faire asseoir la moitié des hommes à sa droite et le reste derrière lui, afin de ne pas leur refuser le plus grand honneur possible.

\par 26 Lorsqu'ils furent assis, il ordonna à Dorothée de tout faire conformément aux coutumes en usage parmi ses invités juifs.

\par 27 C'est pourquoi il renonça aux services des hérauts sacrés, des prêtres sacrificateurs et des autres qui avaient l'habitude d'offrir les prières, et il appela l'un d'entre nous, Éléazar, le plus âgé des prêtres juifs, pour qu'il prie à sa place.

\par 28 Et il se leva et fit une prière remarquable. «Que Dieu Tout-Puissant t'enrichisse, ô roi, de toutes les bonnes choses qu'il a faites et puisse-t-il t'accorder, à toi, à ta femme, à tes enfants et à tes camarades, la possession continue d'eux aussi longtemps que tu vivras!»

\par 29 A ces paroles, de vifs et joyeux applaudissements éclatèrent qui durent un temps considérable, puis ils se tournèrent vers la jouissance du banquet qui avait été préparé.

\par 30 Toutes les dispositions pour le service à table furent exécutées conformément à l'injonction de Dorothée.

\par 31 Parmi les assistants se trouvaient les pages royales et d'autres personnes qui occupaient des places d'honneur à la cour du roi.

\par 32 Profitant d'une pause dans le banquet, le roi demanda à l'envoyé qui occupait le siège d'honneur (car ils étaient disposés selon l'ancienneté), comment il pourrait maintenir son royaume intact jusqu'à la fin ?

\par 33 Après avoir réfléchi un instant, il répondit : « Vous pourriez mieux établir sa sécurité si vous deviez imiter la bienveillance incessante de Dieu. Car si vous faites preuve de clémence et infligez des châtiments légers à ceux qui les méritent selon leurs mérites, vous les détournerez du mal et les conduirez à la repentance. 1

\par 34 Le roi loua la réponse puis demanda à l'homme suivant, comment il pouvait tout faire pour le mieux dans toutes ses actions ?

\par 35 Et il répondit : « Si un homme garde une attitude juste envers tous, il agira toujours correctement en toute occasion, se souvenant que chaque pensée est connue de Dieu. Si vous prenez la crainte de Dieu comme point de départ, vous ne manquerez jamais le but.

\par 36 Le roi complimenta également cet homme pour sa réponse et demanda à un autre comment il pouvait avoir des amis partageant les mêmes idées que lui ?

\par 37 Il répondit : « S'ils vous voient étudier les intérêts des multitudes sur lesquelles vous gouvernez ; vous ferez bien d'observer comment Dieu accorde ses bienfaits à la race humaine, lui fournissant la santé, la nourriture et tout le reste au moment opportun.

\par 38 Après avoir exprimé son accord avec la réponse, le roi demanda à l'invité suivant, comment, en donnant des audiences et en rendant des jugements, il pourrait gagner les éloges même de ceux qui n'avaient pas gagné leur procès ?

\par 39 Et il dit : « Si vous êtes équitables dans vos paroles envers tous et que vous n'agissez jamais de manière insolente ni tyrannique dans votre traitement des délinquants. Et vous y parviendrez si vous observez la méthode par laquelle Dieu agit. Les requêtes des dignes sont toujours exaucées, tandis que ceux qui n'obtiennent pas de réponse à leurs prières sont informés par le biais de rêves ou d'événements de ce qui était nuisible dans leurs requêtes et que Dieu ne les frappe pas selon leurs péchés ou la grandeur de leur péché. Sa force, mais il agit avec patience envers eux.

\par 40 Le roi félicita chaleureusement l'homme pour sa réponse et demanda au suivant dans l'ordre, comment pouvait-il être invincible dans les affaires militaires ?

\par 41 Et il répondit : « S’il ne se fiait pas entièrement à ses multitudes ou à ses forces guerrières, mais invoquait continuellement Dieu pour mener à bien ses entreprises, alors que lui-même s’est acquitté de toutes ses fonctions dans un esprit de justice.

\par 42 Saluant cette réponse, il demanda à un autre comment il pourrait devenir un objet de crainte pour ses ennemis.

\par 43 Et il répondit : « Si, tout en maintenant une vaste réserve d'armes et de forces, il se souvenait que ces choses étaient impuissantes à obtenir un résultat permanent et concluant. Car même Dieu insuffle la peur dans l'esprit des hommes en accordant des sursis et en faisant simplement une démonstration de la grandeur de sa puissance.

\par 44 Le roi loua cet homme et dit ensuite au suivant : « Quel est le plus grand bien dans la vie ? »

\par 45 Et il répondit : 'Savoir que Dieu est le Seigneur de l'Univers et que dans nos plus belles réalisations, ce n'est pas nous qui atteignons le succès mais Dieu qui, par sa puissance, amène toutes choses à leur accomplissement et nous conduit au but.'

\par 46 Le roi s'écria que l'homme avait bien répondu puis demanda au suivant comment il pouvait conserver tous ses biens intacts et enfin les transmettre à ses successeurs dans le même état ?

\par 47 Et il répondit : « En priant constamment Dieu pour qu'il vous inspire de hautes motivations dans toutes vos entreprises et en avertissant vos descendants de ne pas se laisser éblouir par la renommée ou la richesse, car c'est Dieu qui accorde tous ces dons et les hommes ne conquièrent jamais à eux seuls la suprématie.»

\par 48 Le roi exprima son accord avec la réponse et demanda à l'invité suivant, comment il pouvait supporter avec sérénité tout ce qui lui arrivait ?

\par 49 Et il dit : « Si vous comprenez fermement la pensée que tous les hommes sont désignés par Dieu pour partager le plus grand mal aussi bien que le plus grand bien, puisqu'il est impossible à celui qui est un homme d'être exempté de ces. Mais Dieu, que nous devons toujours prier, nous inspire le courage d'endurer.

\par 50 Ravi de la réponse de l'homme, le roi dit que toutes leurs réponses avaient été bonnes. «Je vais me poser une question», ajouta-t-il, «et ensuite je m'arrêterai pour le moment : afin que nous puissions tourner notre attention vers la jouissance de la fête et passer un moment agréable.»

\par 51 Là-dessus, il demanda à l'homme : « Quel est le véritable but du courage ?

\par 52 Et il répondit : « Si un bon plan est exécuté à l'heure du danger conformément à l'intention initiale. Car tout est accompli par Dieu à votre avantage, ô roi, puisque votre dessein est bon.

\par 53 Quand tous eurent signifié par leurs applaudissements leur accord avec la réponse, le roi dit aux philosophes (car bon nombre d'entre eux étaient présents) : « J'estime que ces hommes excellent en vertu et possèdent des connaissances extraordinaires, puisque sur un coup de tête, ils ont donné des réponses appropriées aux questions que je leur ai posées et ont tous fait de Dieu le point de départ de leurs paroles.

\par 54 Et Ménédème, le philosophe d'Érétrie, dit : « C'est vrai, ô roi ; car puisque l'univers est géré par la providence et puisque nous percevons avec raison que l'homme est la création de Dieu, il s'ensuit que toute la puissance et la beauté de la parole procèdent de Dieu.'

\par 55 Lorsque le roi eut acquiescé à ce sentiment, les paroles cessèrent et ils commencèrent à s'amuser. Le soir venu, le banquet se termina.

\par \textit{Notes de bas de page}

\par \textit{162:1 Comparez cette attitude envers les criminels avec celle de la vision humanitaire dite moderne. Aussi Bee Chapitre VIII. 11.}

\chapter{8}

\par \textit{Plus de questions et réponses. Notez le verset 20 avec sa référence au vol dans les airs écrit en 150 avant JC}

\par 1 Le lendemain, ils se mirent à table et continuèrent le banquet selon les mêmes arrangements.

\par 2 Lorsque le roi pensa que l'occasion était venue de poser des questions à ses invités, il posa d'autres questions aux hommes qui étaient assis à côté, dans l'ordre de ceux qui avaient donné des réponses la veille.

\par 3 Il commença à engager la conversation avec le onzième homme, car il y en avait dix à qui on avait posé des questions la première fois.

\par 4 Lorsque le silence fut établi, il demanda comment il pouvait continuer à être riche ?

\par 5 Après une brève réflexion, l'homme à qui on avait posé la question répondit : « S'il n'a rien fait d'indigne de sa position, n'a jamais agi de manière licencieuse, n'a jamais dépensé sans compter dans des activités vides et vaines, mais par des actes de bienveillance a fait de tous ses sujets bien disposé envers lui-même. Car c'est Dieu qui est l'auteur de toutes les bonnes choses et c'est à lui que l'homme doit obéir.

\par 6 Le roi l'a loué et a ensuite demandé à un autre comment il pouvait maintenir la vérité ?

\par 7 En réponse à la question, il dit : « En reconnaissant qu'un mensonge apporte une grande honte à tous les hommes, et plus particulièrement aux rois. Car puisqu’ils ont le pouvoir de faire ce qu’ils veulent, pourquoi devraient-ils recourir au mensonge ? En plus de cela, tu dois toujours te rappeler, ô Roi, que Dieu aime la vérité.

\par 8 Le roi reçut la réponse avec une grande joie et, regardant un autre, dit : «Qu'est-ce que l'enseignement de la sagesse ?»

\par 9 Et l'autre répondit : « De même que vous souhaitez qu'aucun mal ne vous arrive, mais que vous ayez part à toutes les bonnes choses, vous devez donc agir selon le même principe envers vos sujets et vos délinquants, et vous devez réprimander avec douceur les nobles et les bons. Car Dieu attire tous les hommes à lui par sa bienveillance.»

\par 10 Le roi le loua et demanda au suivant comment il pouvait être l'ami des hommes ?

\par 11 Et il répondit : « En observant que le genre humain grandit et naît avec beaucoup de troubles et de grandes souffrances : c'est pourquoi vous ne devez pas les punir à la légère ni leur infliger des tourments, puisque vous savez que la vie des hommes est faite de peines et de peines. Car si tu comprenais tout tu serais rempli de pitié, pour Dieu aussi c'est pitoyable !»

\par 12 Le roi reçut la réponse avec approbation et demanda au suivant : « Quelle est la qualification la plus essentielle pour gouverner ?

\par 13 « Se garder, répondit-il, à l'abri de la corruption et pratiquer la sobriété pendant la plus grande partie de sa vie, honorer la justice par-dessus toutes choses et se lier d'amitié avec des hommes de ce type. Car Dieu aussi est un ami de la justice ! Après avoir signifié son approbation, le roi dit à un autre : « Quelle est la véritable marque de piété ?

\par 14 Et il répondit : « Percevoir que Dieu agit constamment dans l'univers et connaît toutes choses, et qu'aucun homme qui agit injustement et commet le mal ne peut échapper à son attention. De même que Dieu est le bienfaiteur du monde entier, vous aussi devez l'imiter et ne pas vous offenser !

\par 15 Le roi signifia son accord et dit à un autre : « Quelle est l'essence de la royauté ?

\par 16 Et il répondit : « Bien se gouverner et ne pas se laisser égarer par la richesse ou la renommée vers des désirs immodérés ou inconvenants, telle est la vraie façon de gouverner si vous raisonnez bien. Car tout ce dont vous avez réellement besoin est à vous, et Dieu est libre du besoin et en même temps bienveillant. Que vos pensées soient telles qu'elles conviennent à un homme, et ne désirez pas beaucoup de choses mais seulement celles qui sont nécessaires pour gouverner !

\par 17 Le roi le loua et demanda à un autre homme, comment ses délibérations pourraient-elles être pour le mieux ?

\par 18 Et il répondit : « S'il mettait constamment la justice devant lui en tout et pensait que l'injustice équivalait à la privation de la vie. Car Dieu promet toujours aux justes les plus grandes bénédictions !

\par 19 Après l'avoir loué, le roi demanda ensuite : comment pourrait-il être libre de pensées troublantes dans son sommeil ?

\par 20 Et il répondit : «Vous m'avez posé une question à laquelle il est très difficile de répondre, car nous ne pouvons pas mettre en jeu notre véritable moi pendant les heures de sommeil, mais nous y sommes retenus par des imaginations que la raison ne peut contrôler. Car nos âmes ont le sentiment de voir réellement les choses qui entrent dans notre conscience pendant le sommeil. Mais nous commettons une erreur si nous supposons que nous naviguons réellement sur la mer à bord de bateaux, ou que nous volons dans les airs 1, ou que nous voyageons vers d'autres régions ou quoi que ce soit de ce genre. Et pourtant, nous imaginons réellement que de telles choses se produisent.»

\par 21 Dans la mesure où il me est possible de décider, je suis parvenu à la conclusion suivante. Vous devez de toutes les manières possibles, ô Roi, gouverner vos paroles et vos actions selon la règle de la piété afin que vous puissiez avoir la conscience que vous maintenez la vertu et que vous ne choisissez jamais de vous satisfaire aux dépens de la raison et jamais en abusant de votre pouvoir faites du mal à la justice.

\par 22 Car l'esprit s'occupe principalement pendant le sommeil des mêmes choses dont il s'occupe pendant l'éveil. Et celui qui dirige toutes ses pensées et ses actions vers les fins les plus nobles s’établit dans la justice aussi bien lorsqu’il est éveillé que lorsqu’il dort. C’est pourquoi vous devez être ferme dans la discipline constante de vous-même.

\par 23 Le roi fit l'éloge de cet homme et dit à un autre : Puisque tu es le dixième à répondre, quand tu auras parlé, nous nous consacrerons au banquet. Et puis il a posé la question : comment puis-je éviter de faire quelque chose d’indigne de moi ?

\par 24 Et il répondit : 'Veillez toujours à votre propre renommée et à votre propre position suprême, afin que vous ne puissiez parler et penser que les choses qui y sont cohérentes, sachant que tous vos sujets pensent et parlent de vous.' Car vous ne devez pas paraître pire que les acteurs, qui étudient soigneusement le rôle qu'ils doivent jouer et façonnent toutes leurs actions en fonction de celui-ci. Vous ne jouez pas un rôle, mais vous êtes réellement un roi, puisque Dieu vous a conféré une autorité royale conforme à votre caractère.

\par 25 Lorsque le roi eut applaudi fort et longuement de la manière la plus gracieuse, les invités furent invités à chercher du repos. Aussi, lorsque la conversation cessa, ils se consacrèrent à la suite du festin.

\par 26 Le lendemain, la même disposition fut observée, et lorsque le roi trouva l'occasion d'interroger les hommes, il interrogea le premier de ceux qui étaient restés pour l'interrogatoire suivant : Quelle est la forme la plus élevée de gouvernement?

\par 27 Et il répondit : « Se gouverner soi-même et ne pas se laisser emporter par des impulsions. Car tous les hommes possèdent une certaine disposition d’esprit naturelle. Il est probable que la plupart des hommes ont un penchant pour la nourriture, la boisson et le plaisir, et que les rois sont enclins à l'acquisition de territoires et d'une grande renommée. Mais il est bon qu'il y ait de la modération en toutes choses.

\par 28 'Ce que Dieu donne, vous devez le prendre et le garder, mais ne aspirez jamais à des choses qui sont hors de votre portée.'

\par 29 Satisfait de ces paroles, le roi demanda ensuite : comment pourrait-il être libre de l'envie ?

\par 30 Et il répondit après une brève pause : « Si vous considérez d'abord que c'est Dieu qui accorde à tous les rois la gloire et de grandes richesses et que personne n'est roi par son propre pouvoir. Tous les hommes souhaitent partager cette gloire mais ne le peuvent pas, car c'est le don de Dieu !'

\par 31 'Le roi a loué l'homme dans un long discours puis a demandé à un autre, comment il pouvait mépriser ses ennemis ?'

\par 32 Et il répondit : « Si vous faites preuve de bonté envers tous les hommes et gagnez leur amitié, vous n'avez à craindre personne. Être populaire auprès de tous les hommes est le meilleur des bons cadeaux à recevoir de Dieu !

\par 33 Après avoir loué cette réponse, le roi ordonna à l'homme suivant de répondre à la question : comment pourrait-il maintenir sa grande renommée ?

\par 34 Et il répondit : « Si vous êtes généreux et au grand cœur en accordant des bontés et des actes de grâce aux autres, vous ne perdrez jamais votre renommée, mais si vous souhaitez que les grâces susmentionnées continuent les vôtres, vous devez continuellement invoquer Dieu. .'

\par 35 Le roi exprima son approbation et demanda au suivant : À qui doit-on faire preuve de libéralité ?

\par 36 Et il répondit : « Tous les hommes reconnaissent que nous devons faire preuve de libéralité envers ceux qui sont bien disposés à notre égard, mais je pense que nous devons montrer le même esprit de générosité envers ceux qui nous sont opposés que par ce cela signifie que nous pouvons les gagner à droite et à ce qui nous est avantageux. Mais nous devons prier Dieu pour que cela soit accompli, car il dirige l'esprit de tous les hommes.

\par 37 Après avoir exprimé son accord avec la réponse, le roi demanda au sixième de répondre à la question : à qui devons-nous montrer notre gratitude ?

\par 38 Et il répondit : « À nos parents continuellement, car Dieu nous a donné un commandement très important concernant l'honneur dû aux parents. Ensuite, il considère l'attitude d'un ami envers un ami, car il parle d'un « ami qui est comme ton âme ». Vous faites bien d’essayer de rapprocher tous les hommes de votre amitié.

\par 39 Le roi lui parla gentiment, puis demanda au suivant : Qu'est-ce qui ressemble à la beauté en valeur ?

\par 40 Et il dit : « La piété, car c'est la forme prééminente de la beauté, et sa puissance réside dans l'amour, qui est le don de Dieu. Ceci, vous l'avez déjà acquis et avec lui toutes les bénédictions de la vie.

\par 41 Le roi applaudit de la manière la plus gracieuse à la réponse et demanda à un autre : comment, s'il échouait, il pourrait retrouver sa réputation au même degré ?

\par 42 Et il dit : « Il ne vous est pas possible d'échouer, car vous avez semé en tous les hommes les graines de gratitude qui produisent une moisson de bonne volonté, et celle-ci est plus puissante que les armes les plus puissantes et garantit la plus grande sécurité. Mais si quelqu’un échoue, il ne doit plus jamais refaire les choses qui ont causé son échec, mais il doit nouer des amitiés et agir avec justice. Car c'est le don de Dieu de pouvoir faire de bonnes actions et non le contraire.

\par 43 Ravi de ces paroles, le roi demanda à un autre, comment pourrait-il être libéré du chagrin ?

\par 44 Et il répondit : S'il n'a jamais fait de mal à personne, mais qu'il a fait du bien à tout le monde et qu'il a suivi le chemin de la justice, car ses fruits libèrent du chagrin. Mais nous devons prier Dieu pour que des maux inattendus tels que la mort, la maladie, la douleur ou quoi que ce soit de ce genre ne nous arrivent pas et ne nous blessent pas. Mais puisque vous êtes dévoué à la piété, aucun malheur de ce genre ne vous arrivera jamais.

\par 45 Le roi lui fit de grands éloges et demanda au dixième : Quelle est la plus haute forme de gloire ?

\par 46 Et il dit : « Honorer Dieu, et cela ne se fait pas par des dons et des sacrifices mais avec une âme pure et une sainte conviction, puisque toutes choses sont façonnées et gouvernées par Dieu conformément à sa volonté. Vous êtes en possession constante de cet objectif, comme tous les hommes peuvent le faire grâce à vos réalisations passées et présentes.

\par 47 D'une voix forte, le roi les salua tous et leur parla gentiment, et tous ceux qui étaient présents exprimèrent leur approbation, surtout les philosophes. Car ils leur étaient de loin supérieurs [c'est-à-dire les philosophes] tant en conduite qu'en argumentation, puisqu'ils partaient toujours de Dieu.

\par 48 Après cela, le roi, pour montrer sa bonne humeur, se mit à boire à la santé de ses invités.

\par \textit{Notes de bas de page}

\par \textit{165:1 Écrit vers 150 avant JC !}

\chapter{9}

\par \textit{Le verset 8 incarne la valeur de la connaissance. Verset 28, affection parentale. Notez particulièrement la question du verset 26 et la réponse. Notez également la question du verset 47 et la réponse. C'est un sage conseil pour les hommes d'affaires.}

\par 1 Le lendemain, les mêmes dispositions furent prises pour le banquet, et le roi, dès que l'occasion se présenta, commença à poser des questions aux hommes qui étaient assis à côté de ceux qui avaient déjà répondu, et il dit au premier « La sagesse peut-elle être enseignée ?

\par 2 Et il dit : 'L'âme est ainsi constituée qu'elle est capable par la puissance divine de recevoir tout le bien et de rejeter le contraire.'

\par 3 Le roi exprima son approbation et demanda à l'homme suivant : Qu'est-ce qui est le plus bénéfique pour la santé ?

\par 4 Et il dit : 'La tempérance, et il n'est pas possible de l'acquérir à moins que Dieu ne crée une disposition à son égard.'

\par 5 Le roi parla gentiment à l'homme et dit à un autre : « Comment un homme peut-il dignement payer la dette de gratitude envers ses parents ?

\par 6 Et il dit : 'En ne leur causant jamais de souffrance, et cela n'est possible que si Dieu dispose l'esprit à la poursuite des fins les plus nobles.'

\par 7 Le roi exprima son accord et demanda au suivant : comment pourrait-il devenir un auditeur attentif ?

\par 8 Et il dit : 'En vous rappelant que toute connaissance est utile, car elle vous permet, avec l'aide de Dieu, en cas d'urgence, de sélectionner certaines des choses que vous avez apprises et de les appliquer à la crise à laquelle vous êtes confronté. Ainsi les efforts des hommes sont comblés par l'assistance de Dieu.

\par 9 Le roi le loua et demanda au suivant : Comment pourrait-il éviter de faire quelque chose de contraire à la loi ?

\par 10 Et il dit : Si vous reconnaissez que c'est Dieu qui a mis dans le cœur des législateurs la pensée de préserver la vie des hommes, vous les suivrez.

\par 11 Le roi reconnut la réponse de l'homme et dit à un autre : « Quel est l'avantage de la parenté ?

\par 12 Et il répondit : « Si nous considérons que nous sommes nous-mêmes affligés par les malheurs qui tombent sur nos proches et si leurs souffrances deviennent les nôtres, alors la force de la parenté apparaît immédiatement, car ce n'est que lorsqu'un tel sentiment est montré que nous gagnerons honneur et estime à leurs yeux. Car l'aide, lorsqu'elle est liée à la bonté, est en soi un lien tout à fait indissoluble. Et au jour de leur prospérité, nous ne devons pas convoiter leurs biens, mais prier Dieu de leur accorder toutes sortes de biens.

\par 13 Et lui ayant accordé les mêmes éloges qu'aux autres, le roi demanda à un autre : comment pourrait-il se libérer de la peur ?

\par 14 Et il dit : 'Quand l'esprit est conscient qu'il n'a fait aucun mal, et quand Dieu l'oriente vers tous les nobles conseils.'

\par 15 Le roi exprima son approbation et demanda à un autre, comment pouvait-il toujours maintenir un jugement juste ?

\par 16 Et il répondit : 'S'il mettait constamment devant ses yeux les malheurs qui arrivent aux hommes et reconnaissait que c'est Dieu qui enlève la prospérité aux uns et amène les autres à un grand honneur et à une grande gloire.'

\par 17 Le roi reçut aimablement l'homme et demanda au suivant de répondre à la question : comment pourrait-il éviter une vie de facilité et de plaisir ?

\par 18 Et il répondit : « S'il se souvenait continuellement qu'il était le chef d'un grand empire et le seigneur de vastes multitudes, et que son esprit ne devait pas être occupé par d'autres choses, mais il devrait toujours considérer comment c'est lui qui pouvait le mieux promouvoir leur bien-être. Il doit aussi prier Dieu pour qu'aucun devoir ne soit négligé.

\par 19 Après l'avoir loué, le roi demanda au dixième, comment pouvait-il reconnaître ceux qui traitaient avec trahison à son égard ?

\par 20 Et il répondit à la question : « S'il observait si l'attitude de ceux qui l'entouraient était naturelle et s'ils maintenaient la règle de préséance appropriée dans les réceptions et les conseils, et dans leurs relations générales, sans jamais dépasser les limites de la convenance dans les félicitations ou dans d'autres questions de comportement. Mais Dieu inclinera ton esprit, ô Roi, vers tout ce qui est noble.»

\par 21 Lorsque le roi eut exprimé sa vive approbation et les loua tous individuellement (au milieu des applaudissements de tous ceux qui étaient présents), ils se tournèrent vers la jouissance du festin.

\par 22 Et le lendemain, lorsque l'occasion se présenta, le roi demanda à l'homme suivant : Quelle est la forme de négligence la plus grossière ?

\par 23 Et il répondit : « Si un homme ne prend pas soin de ses enfants et ne consacre pas tous ses efforts à leur éducation. Car nous prions toujours Dieu, non pas tant pour nous-mêmes que pour nos enfants, afin que toutes les bénédictions soient les leurs. Notre désir que nos enfants puissent posséder la maîtrise de soi n'est réalisé que par la puissance de Dieu.

\par 24 Le roi dit qu'il avait bien parlé puis demanda à un autre, comment pouvait-il être patriote ?

\par 25 « En gardant à l'esprit, répondit-il, la pensée qu'il est bon de vivre et de mourir dans son propre pays. La résidence à l'étranger 1 apporte le mépris aux pauvres et la honte aux riches, comme s'ils avaient été bannis pour un crime. Si vous accordez des bienfaits à tous, comme vous le faites continuellement, Dieu vous accordera la faveur de tous et vous serez considéré comme un patriote.

\par 26 Après avoir écouté cet homme, le roi demanda au suivant dans l'ordre, comment pourrait-il vivre amicalement avec sa femme ?

\par 27 Et il répondit : « En reconnaissant que les femmes sont par nature têtues et énergiques dans la poursuite de leurs propres désirs, et sujettes à des changements d’opinion soudains à cause de raisonnements fallacieux, et que leur nature est essentiellement faible. Il est nécessaire de les traiter avec sagesse et de ne pas provoquer de conflits. Pour mener sa vie avec succès, le timonier doit connaître le but vers lequel il doit diriger sa route. Ce n’est qu’en faisant appel à l’aide de Dieu que les hommes peuvent à tout moment suivre le véritable chemin de leur vie.»

\par 28 Le roi exprima son accord et demanda ensuite : comment pourrait-il être exempt de toute erreur ?

\par 29 Et il répondit : « Si vous agissez toujours avec délibération et n'accordez jamais de crédit aux calomnies, mais prouvez par vous-même ce qu'on vous dit et décidez par votre propre jugement des demandes qui vous sont faites et exécutez tout dans les règles. à la lumière de ton jugement, tu seras libre de toute erreur, ô Roi. Mais la connaissance et la pratique de ces choses sont l'œuvre de la puissance divine.

\par 30 Ravi de ces paroles, le roi demanda à un autre, comment pourrait-il être libéré de la colère ?

\par 31 Et il répondit à la question : S'il reconnaissait qu'il a le pouvoir sur tous même de leur infliger la mort, s'il se laissait aller à la colère, et qu'il serait inutile et pitoyable qu'il, juste parce qu'il était seigneur, a privé beaucoup de vies.

\par 32 « Quel besoin y avait-il de colère, quand tous les hommes étaient soumis et que personne ne lui était hostile ? Il est nécessaire de reconnaître que Dieu gouverne le monde entier dans un esprit de bonté et sans aucune colère, et toi, dit-il, ô roi, tu dois nécessairement copier son exemple.

\par 33 Le roi dit qu'il avait bien répondu, puis il demanda à son voisin : Qu'est-ce qu'un bon conseil ?

\par 34 « Agir toujours bien et avec réflexion, expliqua-t-il, en comparant ce qui est avantageux pour notre propre politique avec les effets préjudiciables qui résulteraient de l'adoption du point de vue opposé, afin qu'en pesant chaque point nous pouvons être bien conseillés et notre objectif peut être atteint. Et le plus important de tout, par la puissance de Dieu, chacun de vos projets trouvera son accomplissement parce que vous pratiquez la piété.

\par 35 Le roi dit que cet homme avait bien répondu, et il demanda à un autre : Qu'est-ce que la philosophie ?

\par 36 Et il expliqua : « Bien délibérer sur toute question qui se présente et ne jamais se laisser emporter par des impulsions, mais réfléchir aux blessures qui résultent des passions et agir correctement selon que les circonstances l'exigent, en pratiquant la modération. . Mais nous devons prier Dieu d'inculquer dans notre esprit le respect de ces choses.

\par 37 Le roi signifia son consentement et demanda à un autre, comment il pourrait être reconnu lorsqu'il voyageait à l'étranger ?

\par 38 « En étant juste envers tous les hommes, répondit-il, et en paraissant inférieur plutôt que supérieur à ceux parmi lesquels il voyageait. Car c’est un principe reconnu que Dieu, de par sa nature même, accepte les humbles. Et la race humaine aime ceux qui sont prêts à se soumettre à eux.

\par 39 Ayant exprimé son approbation à cette réponse, le roi demanda à un autre, comment il pourrait construire de manière à ce que ses structures perdurent après lui ?

\par 40 Et il répondit à la question : « Si ses créations étaient à une grande et noble échelle, de manière à ce que les spectateurs les épargnent à cause de leur beauté, et s'il n'a jamais renvoyé aucun de ceux qui ont réalisé de telles œuvres et n'a jamais forcé les autres à subvenir à ses besoins sans salaire.

\par 41 Pour avoir observé comment Dieu pourvoit à la race humaine, en lui accordant la santé et la capacité mentale et pour tous les autres dons, il devrait lui-même suivre son exemple en rendant aux hommes une récompense pour leur dur labeur. 1 Car ce sont les œuvres accomplies dans la justice qui demeurent continuellement !

\par 42 Le roi dit que cet homme aussi avait bien répondu et demanda à la dixième question : Quel est le fruit de la sagesse ?

\par 43 Et il répondit : « Afin qu'un homme soit conscient en lui-même qu'il n'a fait aucun mal et qu'il vive sa vie dans la vérité. » Puisque c'est de ceux-ci, ô puissant roi, que vous reviennent la plus grande joie, la plus grande fermeté d'âme et la plus grande foi en Dieu, si vous dirigez votre royaume avec piété.

\par 44 Et quand ils entendirent la réponse, ils poussèrent tous de grands acclamations, et ensuite le roi, dans la plénitude de sa joie, se mit à boire à leur santé.

\par 45 Et le lendemain, le banquet suivit le même déroulement que les occasions précédentes, et lorsque l'occasion se présenta, le roi entreprit de poser des questions aux autres invités, et il dit au premier : « Comment un homme peut-il se tenir par fierté ?

\par 46 Et il répondit : 'S'il maintient l'égalité et se souvient en toutes occasions qu'il est un homme qui règne sur les hommes.' Et Dieu réduit à néant les orgueilleux et exalte les doux et les humbles !

\par 47 Le roi lui parla gentiment et demanda au suivant : Qui doit-on choisir pour conseillers ?

\par 48 Et il répondit : « Ceux qui ont été éprouvés dans de nombreuses affaires et qui maintiennent une bonne volonté sans mélange à son égard et qui participent à son propre tempérament. Et Dieu se manifeste à ceux qui sont dignes que ces fins soient atteintes.

\par 49 Le roi le loua et demanda à un autre : Quel est le bien le plus nécessaire à un roi ?

\par 50 « L'amitié et l'amour de ses sujets, répondit-il, car c'est par cela que le lien de bonne volonté est rendu indissoluble. Et c'est Dieu qui veille à ce que cela se réalise conformément à votre souhait.

\par 51 Le roi le loua et demanda à un autre : Quel est le but de la parole ? Et il a répondu : « Convaincre votre adversaire en lui montrant ses erreurs dans une armée d'arguments bien ordonnés. »

\par 52 « Car ainsi vous gagnerez votre auditeur, non pas en vous opposant à lui, mais en lui accordant des louanges en vue de le persuader. Et c'est par la puissance de Dieu que la persuasion s'accomplit.

\par 53 Le roi dit qu'il avait donné une bonne réponse, et demanda à une autre comment il pourrait vivre amicalement avec les nombreuses races différentes qui formaient la population de son royaume ?

\par 54 «En agissant comme il convient envers chacun», répondit-il, «et en prenant la justice comme guide, comme vous le faites maintenant avec l'aide de la perspicacité que Dieu vous accorde.»

\par 55 Le roi fut ravi de cette réponse et demanda à un autre : « Dans quelles circonstances un homme doit-il souffrir ?

\par 56 « Dans les malheurs qui arrivent à nos amis, répondit-il, quand on voit qu'ils sont prolongés et irrémédiables. La raison ne nous permet pas de pleurer ceux qui sont morts et libérés du mal, mais tous les hommes les pleurent parce qu'ils ne pensent qu'à eux-mêmes et à leur propre avantage. C'est par la seule puissance de Dieu que nous pouvons échapper à tout mal !

\par 57 Le roi dit qu'il avait donné une réponse appropriée, et demanda à un autre : comment se perd la réputation ?

\par 58 Et il répondit : « Lorsque l'orgueil et une confiance en soi illimitée règnent, le déshonneur et la perte de réputation sont engendrés. Car Dieu est le Seigneur de toute réputation et il l'accorde là où Il veut.

\par 59 Le roi confirma la réponse et demanda à l'homme suivant : À qui les hommes doivent-ils se confier ?

\par 60 «À ceux», répondit-il, «qui vous servent par bonne volonté et non par peur ou par intérêt personnel, en ne pensant qu'à leur propre gain.» Car l’un est le signe de l’amour, l’autre la marque de la mauvaise volonté et du temps passé.

\par 61 'Car l'homme qui cherche toujours son propre gain est un traître dans l'âme. Mais vous possédez l'affection de tous vos sujets grâce aux bons conseils que Dieu vous donne.

\par 62 Le roi dit qu'il avait répondu avec sagesse, et il demanda à un autre : Qu'est-ce qui assure la sécurité d'un royaume ?

\par 63 Et il répondit à la question : « Prenez soin et prévoyez qu'aucun mal ne puisse être fait par ceux qui sont placés en position d'autorité sur le peuple, et vous le faites toujours avec l'aide de Dieu qui vous inspire un jugement grave. .'

\par 64 Le roi lui adressa des paroles d'encouragement, et demanda à un autre : Qu'est-ce qui maintient la gratitude et l'honneur ?

\par 65 Et il répondit : « La vertu, car elle est créatrice des bonnes actions, et par elle le mal est détruit, de même que vous faites preuve de noblesse de caractère envers tous par le don que Dieu vous accorde. »

\par 66 Le roi accepta gracieusement la réponse et demanda au onzième (puisqu'il y en avait deux de plus de soixante-dix), comment pouvait-il, en temps de guerre, maintenir la tranquillité d'âme ?

\par 67 Et il répondit : 'En se souvenant qu'il n'avait fait de mal à aucun de ses sujets, et que tous se battraient pour lui en échange des bienfaits qu'ils avaient reçus, sachant que même s'ils perdent la vie, vous prendre soin de ceux qui en dépendent. Car vous ne manquez jamais de réparer à qui que ce soit : telle est la bonté que Dieu vous a inspirée.

\par 68 Le roi les applaudit tous très fort et leur parla très gentiment puis but une longue gorgée à la santé de chacun, s'adonnant à la jouissance et prodiguant à ses invités l'amitié la plus généreuse et la plus joyeuse.

\par \textit{Notes de bas de page}

\par \textit{169:1 Il y avait aussi des résidents étrangers à cette époque.}

\par \textit{170:1 La politique d'un salaire équitable pour une journée de travail équitable apparaît ici comme n'étant pas aussi moderne qu'on le pense parfois dans ce que nous nous plaisons à appeler cette époque éclairée.}

\chapter{10}

\par \textit{Les questions et réponses continuent. Montrer comment les officiers de l'armée doivent être sélectionnés. Ce que l'homme est digne d'admiration et d'autres problèmes de la vie quotidienne sont aussi vrais aujourd'hui qu'il y a 2000 ans. Les versets 15 à 17 sont remarquables pour recommander le théâtre. Les versets 2i-22 décrivent la sagesse d'élire un président ou d'avoir un roi.}

\par 1 Le septième jour, des préparatifs beaucoup plus étendus furent faits, et de nombreuses autres personnes étaient présentes des différentes villes (parmi lesquelles un grand nombre d'ambassadeurs).

\par 2 Lorsqu'une opportunité se présentait, le roi demanda au premier de ceux qui n'avaient pas encore été interrogés, comment il pourrait éviter d'être trompé par des raisonnements fallacieux ?

\par 3 Et il répondit : « En remarquant soigneusement l'orateur, la chose dite et le sujet en discussion, et en posant à nouveau les mêmes questions après un intervalle sous des formes différentes. Mais posséder un esprit alerte et être capable de former un jugement sûr dans chaque cas est l'un des bons dons de Dieu, et tu le possèdes, ô Roi.

\par 4 Le roi applaudit bruyamment à la réponse et demanda à un autre : Pourquoi la majorité des hommes ne deviennent-ils jamais vertueux ?

\par 5 « Parce que, répondit-il, tous les hommes sont par nature intempérants et enclins au plaisir. De là surgissent l’injustice et un flot d’avarice. L'habitude de la vertu est un obstacle pour ceux qui se consacrent à une vie de plaisir, car elle leur impose la préférence de la tempérance et de la droiture. Car c'est Dieu qui est le maître de ces choses.

\par 6 Le roi dit qu'il avait bien répondu, et demanda : À quoi doivent obéir les rois ? Et il dit : « Les lois, afin que, par des actes justes, elles rétablissent la vie des hommes. De même que, par une telle conduite, en obéissance au commandement divin, vous vous êtes réservé un mémorial perpétuel.

\par 7 Le roi dit que cet homme aussi avait bien parlé, et il demanda au suivant : Qui devons-nous nommer comme gouverneurs ?

\par 8 Et il répondit : « Tous ceux qui haïssent la méchanceté et qui imitent votre propre conduite agissent avec justice afin de conserver constamment une bonne réputation. Car c'est ce que tu fais, ô puissant roi, dit-il, et c'est Dieu qui t'a accordé la couronne de justice.

\par 9 Le roi acclama bruyamment la réponse, puis, regardant l'homme suivant, il dit : « Qui devrions-nous nommer comme officiers à la tête des forces ?

\par 10 Et il expliqua : « Ceux qui excellent en courage et en droiture et ceux qui sont plus soucieux de la sécurité de leurs hommes que de remporter la victoire en risquant leur vie par témérité. Car, de même que Dieu agit bien envers tous les hommes, de même, à son imitation, vous êtes le bienfaiteur de tous vos sujets.

\par 11 Le roi dit qu'il avait donné une bonne réponse et demanda à un autre : Quel homme est digne d'admiration ?

\par 12 Et il répondit : « L'homme qui est doté de réputation, de richesse et de pouvoir et qui possède une âme égale à tout cela. Vous montrez vous-même par vos actions que vous êtes le plus digne d'admiration grâce à l'aide de Dieu qui vous fait prendre soin de ces choses.

\par 13 Le roi exprima son approbation et dit à un autre : « À quelles affaires les rois doivent-ils consacrer le plus de temps ?

\par 14 Et il répondit : « À la lecture et à l'étude des récits de voyages officiels, qui sont écrits en référence aux différents royaumes, en vue de la réforme et de la conservation des sujets. Et c'est par une telle activité que vous avez atteint une gloire qui n'a jamais été approchée par d'autres, grâce à l'aide de Dieu qui exauce tous vos désirs.

\par 15 Le roi parla à l'homme avec enthousiasme et demanda à un autre : comment un homme doit-il s'occuper pendant ses heures de détente et de récréation ?

\par 16 Et il répondit : « Regarder ces pièces qui peuvent être jouées avec convenance et mettre sous les yeux des scènes tirées de la vie et jouées avec dignité et décence est profitable et approprié.

\par 17 « Car il y a une certaine édification même dans ces divertissements, car souvent une leçon désirable est enseignée par les affaires les plus insignifiantes de la vie. Mais en pratiquant la plus grande convenance dans toutes vos actions, vous avez montré que vous êtes philosophe et que vous êtes honoré de Dieu à cause de votre vertu.

\par 18 Le roi, satisfait des paroles qui venaient d'être prononcées, dit au neuvième homme : comment doit-on se comporter dans les festins ?

\par 19 Et il répondit : « Vous devez appeler à vos côtés des hommes savants et ceux qui sont capables de vous donner des indications utiles sur les affaires de votre royaume et la vie de vos sujets (car vous ne pouviez trouver aucun thème plus approprié ni plus éducatif que celui-ci) car de tels hommes sont chers à Dieu parce qu’ils ont entraîné leur esprit à contempler les thèmes les plus nobles – comme vous le faites vous-même, puisque toutes vos actions sont dirigées par Dieu.»

\par 20 Ravi de cette réponse, le roi demanda à l'homme suivant : Qu'est-ce qui est le mieux pour le peuple ? Qu'un citoyen privé devrait être nommé roi d'eux ou membre de la famille royale ?

\par 21 Et il répondit : « Celui qui est le meilleur par nature. Car les rois issus de la lignée royale sont souvent durs et sévères envers leurs sujets. Et cela est plus encore le cas de certains de ceux qui sont sortis des rangs des simples citoyens, qui, après avoir connu le mal et supporté leur part de pauvreté, lorsqu'ils règnent sur des multitudes, se révèlent plus cruels que les tyrans impies.

\par 22 «Mais, comme je l'ai dit, une bonne nature bien formée est capable de gouverner, et vous êtes un grand roi, non pas tant parce que vous excellez dans la gloire de votre règne et de votre richesse, mais plutôt parce que vous vous avez surpassé tous les hommes en clémence et en philanthropie, grâce à Dieu qui vous a doté de ces qualités.

\par 23 Le roi passa un certain temps à louer cet homme, puis demanda au dernier de tous : Quelle est la plus grande réussite dans la direction d'un empire ?

\par 24 Et il répondit : « Que les sujets demeurent continuellement en paix, et que la justice soit promptement rendue en cas de litige. »

\par 25 'Ces résultats sont obtenus grâce à l'influence du dirigeant, lorsqu'il est un homme qui déteste le mal et aime le bien et consacre ses énergies à sauver la vie des hommes, tout comme vous considérez l'injustice comme la pire forme du mal et par votre juste administration vous a façonné une réputation éternelle, puisque Dieu vous accorde un esprit pur et exempt de tout mal.

\par 26 Et quand il eut fini, des applaudissements bruyants et joyeux éclatèrent pendant un temps considérable. Lorsqu'il s'arrêta, le roi prit une coupe et porta un toast en l'honneur de tous ses invités et des paroles qu'ils avaient prononcées.

\par 27 « Puis, en conclusion, il dit : J'ai tiré le plus grand bénéfice de votre présence. J'ai beaucoup profité de la sage mise en cache que vous m'avez donnée en référence à l'art de gouverner.''¡

\par 28 Puis il ordonna que trois talents d'argent soient offerts à chacun d'eux, et il chargea un de ses serviteurs de remettre l'argent.

\par 29 Tout à coup ils crièrent leur approbation, et le banquet devint une scène de joie, tandis que le roi se livrait à une ronde continue de festivités.

\chapter{11}

\par \textit{Pour un commentaire sur la sténographie ancienne, voir le verset 7. La traduction est soumise pour approbation et acceptée telle que lue, et (verset 23) un vote d'approbation croissant est pris et adopté à l'unanimité.}

\par 1 J'ai longuement écrit et je dois vous demander pardon, Philocrate.

\par 2 J'ai été étonné au-delà de toute mesure par les hommes et par la façon dont, sur un coup de tête, ils ont donné des réponses qui ont vraiment nécessité beaucoup de temps à élaborer.

\par 3 Car, bien que celui qui posait la question ait longuement réfléchi à chaque question particulière, ceux qui répondaient l'un après l'autre avaient leurs réponses aux questions prêtes à la fois et ainsi ils semblaient à moi et à tous ceux qui étaient présents et surtout aux philosophes être digne d'admiration.

\par 4 Et je suppose que la chose paraîtra incroyable à ceux qui liront mon récit dans le futur.

\par 5 Mais il est inconvenant de déformer les faits enregistrés dans les archives publiques.

\par 6 Et il ne serait pas juste que je transgresse dans une affaire pareille. Je raconte l'histoire telle qu'elle s'est produite, en évitant consciencieusement toute erreur.

\par 7 J'ai été si impressionné par la force de leurs paroles, que j'ai fait un effort pour consulter ceux dont la tâche était de consigner tout ce qui se passait lors des audiences et des banquets royaux.

\par 8 Car c'est la coutume, comme vous le savez, depuis le moment où le roi commence à traiter ses affaires jusqu'au moment où il se couche pour se reposer, qu'un registre soit tenu de toutes ses paroles et de toutes ses actions - un arrangement des plus excellents et des plus utiles. .

\par 9 Car le lendemain, avant le début des affaires, on relit les procès-verbaux des faits et des paroles de la veille, et s'il y a eu quelque irrégularité, l'affaire est immédiatement réglée.

\par 10 J'ai donc obtenu, comme on l'a dit, des informations précises dans les archives publiques, et j'ai exposé les faits dans l'ordre approprié, car je sais combien vous êtes désireux d'obtenir des informations utiles.

\par 11 Trois jours plus tard, Démétrius prit les hommes et longea la digue longue de sept stades jusqu'à l'île, traversa le pont et se dirigea vers les districts septentrionaux de Pharos.

\par 12 Là, il les rassembla dans une maison bâtie au bord de la mer, d'une grande beauté et située dans un endroit isolé, et les invita à accomplir le travail de traduction, car tout ce dont ils avaient besoin à cet effet était mis à leur disposition.

\par 13 Ils se mirent donc au travail en comparant leurs différents résultats et en les mettant d'accord, et tout ce sur quoi ils s'accordèrent fut convenablement copié sous la direction de Démétrius.

\par 14 Et la séance dura jusqu'à la neuvième heure ; après cela, ils furent libres de subvenir à leurs besoins physiques.

\par 15 Tout ce qu'ils désiraient leur fut fourni en grande quantité. En outre, Dorothée faisait quotidiennement pour eux les mêmes préparatifs que pour le roi lui-même, car c'est ainsi que le roi lui avait ordonné.

\par 16 Dès le matin, ils se présentaient quotidiennement à la cour, et après avoir salué le roi, ils retournaient chez eux.

\par 17 Et comme c'est la coutume de tous les Juifs, ils se lavèrent les mains dans la mer et prièrent Dieu puis se consacrèrent à lire et à traduire le passage particulier sur lequel ils étaient occupés, et je leur posai la question : Pourquoi c'est qu'ils se lavaient les mains avant de prier ?

\par 18 Et ils expliquèrent que c'était un signe qu'ils n'avaient fait aucun mal (car toute forme d'activité s'accomplit au moyen des mains) puisque, dans leur manière noble et sainte, ils considèrent tout comme un symbole de justice et de vérité.

\par 19 Comme je l'ai déjà dit, ils se réunissaient quotidiennement dans un lieu délicieux par son calme et sa luminosité et s'appliquaient à leur tâche.

\par 20 Et il se trouve que le travail de traduction fut achevé en soixante-douze jours, comme si cela avait été arrangé à dessein.

\par 21 L'ouvrage terminé, Démétrius rassembla la population juive dans le lieu où la traduction avait été faite et la lut à tous, en présence des traducteurs, qui furent également très bien accueillis par le peuple, à cause de la grands avantages qu'ils leur avaient conférés.

\par 22 Ils louèrent également Démétrius et l'exhortèrent à faire transcrire toute la loi et à en présenter une copie à leurs chefs.

\par 23 Après que les livres eurent été lus, les prêtres et les anciens des traducteurs et la communauté juive et les chefs du peuple se levèrent et dirent que, puisqu'une traduction si excellente, si sacrée et si précise avait été faite, ce n'était que il est juste qu'il reste tel qu'il était et qu'aucune modification ne soit apportée

\par 24 Et lorsque toute la compagnie exprima son approbation, ils leur ordonnèrent de prononcer une malédiction, conformément à leur coutume, sur quiconque apporterait quelque altération, soit en ajoutant quelque chose, soit en changeant de quelque manière que ce soit l'un des mots qui avaient été écrits, ou en faisant quelque omission.

\par 25 C'était une précaution très sage pour garantir que le livre puisse être conservé inchangé pour toutes les époques futures.

\par 26 Lorsque l'affaire fut rapportée au roi, il se réjouit grandement, car il sentait que le dessein qu'il avait formé avait été exécuté en toute sécurité.

\par 27 On lui lut tout le livre et il fut très étonné de l'esprit du législateur.

\par 28 Et il dit à Démétrius : « Comment se fait-il qu'aucun des historiens ni des poètes n'ait jamais jugé utile de faire allusion à une œuvre aussi merveilleuse ?

\par 29 Et il répondit : « Parce que la loi est sacrée et d'origine divine. Et certains de ceux qui ont eu l'intention de s'en occuper ont été frappés par Dieu et ont donc renoncé à leur projet.

\par 30 Il dit qu'il avait entendu dire par Théopompe qu'il avait été rendu fou pendant plus de trente jours parce qu'il avait l'intention d'insérer dans son histoire quelques-uns des incidents des traductions antérieures et peu fiables de la loi.

\par 31 Lorsqu'il fut un peu rétabli, il supplia Dieu de lui faire comprendre pourquoi le malheur lui était arrivé.

\par 32 Et il lui fut révélé dans un rêve que, par vaine curiosité, il souhaitait communiquer des vérités sacrées aux hommes ordinaires, et que s'il y renonçait, il recouvrerait la santé.

\par 33 J'ai aussi entendu dire de la bouche de Théodectès, l'un des poètes tragiques, que lorsqu'il était sur le point d'adapter pour une de ses pièces quelques-uns des incidents rapportés dans le livre, il fut atteint de cataracte dans les deux sens yeux.

\par 34 Et lorsqu'il comprit la raison pour laquelle le malheur lui était arrivé, il pria Dieu pendant plusieurs jours et fut ensuite rétabli.

\par 35 Et après que le roi, comme je l'ai déjà dit, eut reçu l'explication de Démétrius sur ce point, il rendit hommage et ordonna qu'on prenne grand soin des livres et qu'ils fussent sacrément gardés.

\par 36 Et il exhorta les traducteurs à lui rendre visite fréquemment après leur retour en Judée, car il était juste, dit-il, qu'il les renvoie maintenant chez eux.

\par 37 Mais à leur retour, il les traitait comme des amis, comme il se doit, et ils recevaient de lui de riches présents.

\par 38 Il ordonna de préparer leur retour chez eux et les traita avec la plus grande générosité.

\par 39 Il offrit à chacun d'eux trois robes de la plus belle qualité, deux talents d'or, un buffet pesant un talent, tous les meubles pour trois canapés.

\par 40 Et avec son escorte, il envoya à Éléazar dix divans aux pieds d'argent et tout l'équipement nécessaire, un buffet valant trente talents, dix robes de pourpre et une magnifique couronne, et cent pièces de lin le plus fin, ainsi que des bols et plats et deux coupes en or à consacrer à Dieu.

\par 41 Il l'a également exhorté dans une lettre à ne pas l'en empêcher si l'un des hommes préférait revenir vers lui.

\par 42 Car il considérait comme un grand privilège de jouir de la compagnie de tels hommes savants, et il préférait prodiguer ses richesses sur eux plutôt que sur des vanités.

\par 43 Et maintenant Philocrate, tu as l'histoire complète conformément à ma promesse.

\par 44 Je pense que vous trouvez plus de plaisir dans ces matières que dans les écrits des mythologues.

\par 45 Car vous vous consacrez à l'étude de ce qui peut profiter à l'âme et vous y consacrez beaucoup de temps. J'essaierai de raconter tous les autres événements qui méritent d'être enregistrés, afin qu'en les parcourant, vous puissiez obtenir la plus haute récompense pour votre zèle.


\end{document}