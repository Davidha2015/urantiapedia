\begin{document}

\title{Testament de Dan}

\chapter{1}

\par \textit{Le septième fils de Jacob et de Bilhah. Le jaloux. Il déconseille la colère en disant qu'elle « donne une vision particulière ». C'est une thèse remarquable sur la colère.}

\par 1 LA copie des paroles que Dan dit à ses fils dans ses derniers jours, la cent vingt-cinquième année de sa vie.

\par 2 Car il rassembla sa famille et dit : Écoutez mes paroles, vous, fils de Dan ! et prête attention aux paroles de ton père.

\par 3 J'ai prouvé dans mon cœur et dans toute ma vie que la vérité la justice est bonne et agréable à Dieu, et le mensonge et la colère sont mauvais, parce qu'ils enseignent à l'homme toute méchanceté.

\par 4 Je vous avoue donc aujourd'hui, mes enfants, que dans mon cœur j'ai résolu la mort de Joseph, mon frère, l'homme vrai et bon. .

\par 5 Et je me suis réjoui qu'il ait été vendu, parce que son père l'aimait plus que nous.

\par 6 Car l'esprit de jalousie et de vaine gloire m'a dit : Toi aussi, tu es son fils.

\par 7 Et l'un des esprits de Beliar m'excita, disant : Prends cette épée, et avec elle tue Joseph ; ainsi ton père t'aimera-t-il quand il sera mort.

\par 8 Or, c'est l'esprit de colère qui m'a persuadé d'écraser Joseph comme un léopard écrase un chevreau.

\par 9 Mais le Dieu de mes pères n'a pas permis qu'il tombe entre mes mains, afin que je le trouve seul, que je le tue et que je fasse détruire une seconde tribu en Israël.

\par 10 Et maintenant, mes enfants, voici, je meurs, et je vous dis en vérité que si vous ne vous gardez pas de l'esprit de mensonge et de colère, si vous n'aimez pas la vérité et la patience, vous périrez.

\par 11 Car la colère est aveugle, et elle ne permet pas à quelqu'un de voir le visage d'un homme avec vérité.

\par 12 Car, qu'il soit père ou mère, il se comporte envers eux comme des ennemis ; bien que ce soit un frère, il ne le connaît pas ; bien que ce soit un prophète du Seigneur, il lui désobéit ; bien qu'il soit un homme juste, il ne le considère pas ; bien qu'il soit ami, il ne le reconnaît pas.

\par 13 Car l'esprit de colère l'entoure du filet de tromperie, et aveugle ses yeux, et à cause du mensonge, il obscurcit son esprit et lui donne sa propre vision particulière.

\par 14 Et de quoi l'entoure-t-il ses yeux ? Avec haine de cœur, au point d'envier son frère.

\par 15 Car la colère est une mauvaise chose, mes enfants, car elle trouble même l'âme elle-même.

\par 16 Et il s'approprie le corps de l'homme en colère, et il prend possession de son âme, et il confère au corps le pouvoir de commettre toute iniquité.

\par 17 Et quand le corps fait toutes ces choses, l'âme justifie ce qu'elle fait, puisqu'elle ne voit pas bien.

\par 18 C'est pourquoi celui qui est en colère, s'il est un homme fort, a une triple puissance dans sa colère : une par l'aide de ses serviteurs ; et un second par sa richesse, par laquelle il persuade et vainc injustement ; et troisièmement, ayant sa propre puissance naturelle, il produit ainsi le mal.

\par 19 Et bien que l'homme colérique soit faible, il a néanmoins une puissance double de celle qui est naturelle ; car la colère aide toujours ceux qui sont dans l'iniquité.

\par 20 Cet esprit va toujours avec le mensonge à la droite de Satan, afin que ses œuvres soient accomplies avec cruauté et mensonge.

\par 21 Comprenez donc que la puissance de la colère est vaine.

\par 22 Car il donne d'abord une provocation par la parole ; puis, par les actes, cela fortifie celui qui est en colère, et, par de graves pertes, il perturbe son esprit et excite ainsi son âme d'une grande colère.

\par 23 C'est pourquoi, quand quelqu'un parle contre vous, ne vous mettez pas en colère, et si quelqu'un vous loue comme des hommes saints, ne vous enorgueillissez pas : ne soyez pas ému ni par plaisir ni par dégoût..

\par 24 Car d'abord cela plaît à l'ouïe, et ainsi rend l'esprit vif à percevoir les motifs de la provocation ; puis, étant en colère, il pense qu'il est à juste titre en colère.

\par 25 Si vous tombez dans quelque perte ou ruine, mes enfants, ne soyez pas affligés ; car cet esprit même fait désirer à l'homme ce qui est périssable, afin qu'il soit enragé par l'affliction.

\par 26 Et si vous subissez une perte volontairement ou involontairement, ne vous inquiétez pas ; car du dépit naît la colère et le mensonge.

\par 27 Or, un double méfait est la colère contre le mensonge ; et ils s'entraident pour troubler le cœur ; et quand l'âme est continuellement troublée, le Seigneur s'en va, et Beliar règne sur elle.



\chapter{2}

\par \textit{Une prophétie des péchés, de la captivité, des fléaux et de la restitution ultime de la nation. Ils parlent encore d'Eden (voir le verset 18). Le verset 23 est remarquable à la lumière de la prophétie.}

\par 1 OBSERVEZ donc, mes enfants, les commandements du Seigneur, et gardez sa loi ; éloignez-vous de la colère et détestez le mensonge, afin que l'Éternel habite parmi vous et que Beliar fuie loin de vous.

\par 2 Dites la vérité chacun à son prochain. Ainsi ne tomberez-vous pas dans la colère et la confusion ; mais vous serez en paix, ayant le Dieu de paix, et aucune guerre ne prévaudra sur vous.

\par 3 Aimez le Seigneur toute votre vie, et les uns les autres d'un cœur sincère.

\par 4 Je sais que dans les derniers jours vous vous éloignerez de l'Éternel, et que vous provoquerez la colère de Lévi et que vous combattrez Juda ; mais vous ne prévaudrez pas contre eux, car un ange du Seigneur les guidera tous deux ; car c'est près d'eux qu'Israël se tiendra.

\par 5 Et chaque fois que vous vous éloignerez du Seigneur, vous marcherez dans toutes sortes de méchancetés et vous commettrez les abominations des païens, vous prostituant après les femmes des injustes, tandis qu'en toute méchanceté les esprits de méchanceté agiront en vous.

\par 6 Car j'ai lu dans le livre d'Hénoc, le juste, que ton prince est Satan, et que tous les esprits de méchanceté et d'orgueil conspireront pour s'occuper constamment des fils de Lévi, pour les faire pécher devant l'Éternel. .

\par 7 Et mes fils s'approcheront de Lévi, et pécheront avec eux en toutes choses ; et les fils de Juda seront cupides, pillant les biens d'autrui comme des lions.

\par 8 C'est pourquoi vous serez emmenés avec eux en captivité, et là vous recevrez toutes les plaies de l'Égypte et tous les maux des païens.

\par 9 Et ainsi, lorsque vous reviendrez au Seigneur, vous obtiendrez miséricorde, et il vous fera entrer dans son sanctuaire, et il vous donnera la paix.

\par 10 Et de la tribu de Juda et de Lévi vous surgendra le salut de l'Éternel ; et il fera la guerre à Beliar.

\par 11 Et exécutez une vengeance éternelle sur nos ennemis ; et il enlèvera de Beliar les âmes des saints en captivité, il ramènera les cœurs désobéissants au Seigneur et donnera à ceux qui l'invoquent la paix éternelle.

\par 12 Et les saints se reposeront en Éden, et dans la nouvelle Jérusalem les justes se réjouiront, et ce sera à la gloire de Dieu pour toujours.

\par 13 Et Jérusalem ne subira plus la désolation, et Israël ne sera plus emmené captif ; car l'Éternel sera au milieu d'elle [vivant parmi les hommes], et le Saint d'Israël y régnera dans l'humilité et dans la pauvreté ; et celui qui croit en lui régnera parmi les hommes en vérité.

\par 14 Et maintenant, craignez le Seigneur, mes enfants, et méfiez-vous de Satan et de ses esprits.

\par 15 Approchez-vous de Dieu et de l'ange qui intercède pour vous, car il est médiateur entre Dieu et l'homme, et pour la paix d'Israël il se dressera contre le royaume de l'ennemi.

\par 16 C'est pourquoi l'ennemi est désireux de détruire tous ceux qui invoquent le Seigneur.

\par 17 Car il sait que le jour où Israël se repentira, le royaume de l'ennemi prendra fin.

\par 18 Car l'ange de paix fortifiera Israël, afin qu'il ne tombe pas dans l'extrême du mal.

\par 19 Et il arrivera au temps de l'iniquité d'Israël, que l'Éternel ne s'éloignera pas d'eux, mais les transformera en une nation qui fait sa volonté, car aucun des anges ne lui sera égal.

\par 20 Et son nom sera partout en Israël et parmi les païens.

\par 21 Gardez-vous donc vous-mêmes, mes enfants, de toute mauvaise œuvre, rejetez la colère et tout mensonge, et aimez la vérité et la patience.

\par 22 Et ce que vous avez entendu de votre père, communiquez-le aussi à vos enfants, afin que le Sauveur des païens vous reçoive ; car il est vrai et patient, doux et humble, et il enseigne par ses œuvres la loi de Dieu.

\par 23 Éloignez-vous donc de toute injustice et attachez-vous à la justice de Dieu, et votre race sera sauvée pour toujours.

\par 24 Et enterre-moi près de mes pères.

\par 25 Et après avoir dit ces choses, il les baisa et s'endormit après une bonne vieillesse.

\par 26 Et ses fils l'enterrèrent, et après cela ils emportèrent ses ossements et les déposèrent près d'Abraham, d'Isaac et de Jacob.

\par 27 Néanmoins Dan leur prophétisa qu'ils oublieraient leur Dieu et seraient éloignés du pays de leur héritage, de la race d'Israël et de la famille de leur postérité.



\end{document}