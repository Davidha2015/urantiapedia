\begin{document}

\title{Testament de Nephtali}

\chapter{1}

\par \textit{Naphtali, le huitième fils de Jacob et de Bilhah. Le coureur. Une leçon de physiologie.}

\par 1 LA copie du testament de Nephtali, qu'il a ordonné au moment de sa mort, la cent trentième année de sa vie.

\par 2 Lorsque ses fils furent rassemblés le septième mois, le premier jour du mois, alors qu'ils étaient encore en bonne santé, il leur fit un festin de nourriture et de vin.

\par 3 Et après s'être réveillé le matin, il leur dit : Je meurs ; et ils ne le crurent pas.

\par 4 Et tandis qu'il glorifiait le Seigneur, il se fortifiait et disait qu'après la fête d'hier il mourrait.

\par 5 Et il commença alors à dire : Écoutez, mes enfants, vous, fils de Nephtali, écoutez les paroles de votre père.

\par 6 Je suis né de Bilhah, et parce que Rachel a agi astucieusement, et a donné Bilhah à sa place à Jacob, et qu'elle a conçu et m'a enfanté sur les genoux de Rachel, c'est pourquoi elle m'a appelé Nephtali.

\par 7 Car Rachel m'a beaucoup aimé parce que je suis né sur ses genoux ; et quand j'étais encore jeune, elle avait l'habitude de m'embrasser et de dire : Puissé-je avoir un frère à toi dès mon ventre, comme toi.

\par 8 C'est pourquoi aussi Joseph était semblable à moi en toutes choses, selon les prières de Rachel.

\par 9 Or, ma mère était Bilhah, fille de Rothée, frère de Débora, nourrice de Rébecca, née le même jour avec Rachel.

\par 10 Et Rothée était de la famille d'Abraham, un Chaldéen, craignant Dieu, né libre et noble.

\par 11 Et il fut fait captif et acheté par Laban ; et il lui donna pour femme Euna, sa servante, et elle enfanta une fille, et elle l'appela Zilpah, d'après le nom du village dans lequel il avait été emmené captif.

\par 12 Et ensuite elle enfanta Bilhah, en disant : Ma fille se hâte après ce qui est nouveau, car aussitôt qu'elle est née, elle a saisi le sein et s'est empressée de le téter.

\par 13 Et j'étais rapide sur mes pieds comme le cerf, et mon père Jacob m'a désigné pour tous les messages, et comme un cerf il m'a donné sa bénédiction.

\par 14 Car, de même que le potier connaît le vase, combien il doit contenir, et apporte de l'argile en conséquence, de même le Seigneur fait le corps à l'image de l'esprit, et selon la capacité du corps, il implante l'esprit.

\par 15 Et l'un ne manque pas à l'autre d'un tiers de cheveu ; car toute la création a été faite au poids, à la mesure et à la règle.

\par 16 Et comme le potier connaît l'usage de chaque vase et à quoi il convient, de même le Seigneur connaît le corps, jusqu'où il persistera dans le bien et quand il commencera dans le mal.

\par 17 Car il n'y a aucune inclination ou pensée que le Seigneur ne connaisse, car il a créé tout homme à sa propre image.

\par 18 Car comme la force d'un homme, ainsi en est-il de son travail ; comme son œil, ainsi aussi dans son sommeil ; comme son âme, ainsi aussi dans sa parole soit dans la loi du Seigneur, soit dans la loi de Beliar.

\par 19 Et comme il y a une division entre la lumière et les ténèbres, entre la vue et l'ouïe, de même y a-t-il une division entre l'homme et l'homme, et entre la femme et la femme ; et il ne faut pas dire que l’un soit comme l’autre, ni de visage ni d’esprit.

\par 20 Car Dieu a fait toutes choses bonnes dans leur ordre, les cinq sens dans la tête, et il a joint le cou à la tête, y ajoutant aussi les cheveux pour la beauté et la gloire, puis le cœur pour l'intelligence, le ventre pour les excréments, et l'estomac pour broyer, la trachée pour inspirer, le foie pour la colère, le fiel pour l'amertume, la rate pour le rire, les rênes pour la prudence, les muscles des reins pour la puissance, les poumons pour aspirer, les reins pour la force, et ainsi de suite.

\par 21 Ainsi donc, mes enfants, que toutes vos œuvres soient faites avec ordre, avec une bonne intention, dans la crainte de Dieu, et ne faites rien de désordonné par mépris ou hors du temps.

\par 22 Car si tu ordonnes à l'œil d'entendre, il ne le peut pas ; ainsi, pendant que vous êtes dans les ténèbres, vous ne pouvez pas faire les œuvres de lumière.

\par 23 Ne vous empressez donc pas de corrompre vos actions par la convoitise ou par de vaines paroles pour tromper vos âmes ; car si vous gardez le silence dans la pureté du cœur, vous comprendrez comment retenir fermement la volonté de Dieu et rejeter la volonté de Beliar.

\par 24 Soleil, lune et étoiles, ne changez pas leur ordre ; de même, ne changez pas la loi de Dieu dans le désordre de vos actions.

\par 25 Les païens se sont égarés, ont abandonné l'Éternel, ont ordonné leur ordre, et ont obéi aux ceps et aux pierres, esprits de tromperie.

\par 26 Mais vous ne le ferez pas, mes enfants, en reconnaissant dans le firmament, dans la terre, dans la mer et dans toutes les choses créées, le Seigneur qui a fait toutes choses, afin que vous ne deveniez pas comme Sodome, qui a changé le monde ordre de la nature.

\par 27 De la même manière, les Veilleurs ont également changé l'ordre de leur nature, que le Seigneur a maudit lors du déluge, à cause de laquelle il a rendu la terre sans habitants et stérile.

\par 28 Je vous dis ces choses, mes enfants, car j'ai lu dans les écrits d'Hénoc que vous aussi vous éloignerez du Seigneur, en marchant selon toutes les iniquités des païens, et que vous agirez selon toutes les méchanceté de Sodome.

\par 29 Et l'Éternel amènera sur vous la captivité, et là vous servirez vos ennemis, et vous serez courbés par toute affliction et tribulation, jusqu'à ce que l'Éternel vous ait tous consumés.

\par 30 Et après avoir été diminués et réduits en petit nombre, vous revenez et reconnaissez l'Éternel, votre Dieu ; et il vous ramènera dans votre pays, selon sa grande miséricorde.

\par 31 Et il arrivera qu'après leur retour dans le pays de leurs pères, ils oublieront de nouveau l'Éternel et deviendront impies.

\par 32 Et l'Éternel les dispersera sur la face de toute la terre, jusqu'à ce que vienne la compassion de l'Éternel, un homme exerçant la justice et faisant miséricorde à tous ceux qui sont loin et à ceux qui sont proches.



\chapter{2}

\par \textit{Il plaide pour une vie ordonnée. Les versets 27 à 40 sont remarquables pour leur sagesse éternelle.}

\par 1 CAR dans la quarantième année de ma vie, j'ai eu une vision sur le mont des Oliviers, à l'est de Jérusalem, que le soleil et la lune s'arrêtaient.

\par 2 Et voici, Isaac, le père de mon père, nous a dit : Courez et saisissez-les, chacun selon ses forces ; et à celui qui les saisira appartiendra le soleil et la lune.

\par 3 Et nous avons tous couru ensemble, et Lévi a saisi le soleil, et Juda a devancé les autres et s'est emparé de la lune, et ils ont été tous deux élevés avec eux.

\par 4 Et quand Lévi devint comme un soleil, voici, un jeune homme lui donna douze branches de palmier ; et Juda était brillant comme la lune, et sous leurs pieds il y avait douze rayons.

\par 5 Et les deux, Lévi et Juda, coururent et les saisirent.

\par 6 Et voici, un taureau sur la terre, avec deux grandes cornes, et des ailes d'aigle sur le dos ; et nous voulions le saisir, mais nous ne le pouvions pas.

\par 7 Mais Joseph vint, le saisit, et monta avec lui dans les hauteurs.

\par 8 Et je vis, car j'étais là, et voici, une écriture sainte nous apparut, disant : Assyriens, Mèdes, Perses, Chaldéens, Syriens posséderont en captivité les douze tribus d'Israël.

\par 9 Et de nouveau, sept jours plus tard, je vis notre père Jacob debout au bord de la mer de Jamnia, et nous étions avec lui.

\par 10 Et voici, un navire passait par là, sans matelots ni pilote ; et il y avait écrit sur le bateau : La Nef de Jacob.

\par 11 Et notre père nous dit : Venez, embarquons sur notre navire.

\par 12 Et quand il fut monté à bord, il y eut une violente tempête et une forte tempête de vent ; et notre père, qui tenait le gouvernail, nous quitta.

\par 13 Et nous, étant frappés par la tempête, avons été emportés au-dessus de la mer ; et le navire fut rempli d'eau et fut battu par de puissantes vagues, jusqu'à ce qu'il soit brisé.

\par 14 Et Joseph s'enfuit sur un petit bateau, et nous étions tous répartis sur neuf planches, et Lévi et Juda étaient ensemble.

\par 15 Et nous étions tous dispersés jusqu'aux extrémités de la terre.

\par 16 Alors Lévi, ceint d'un sac, pria pour nous tous l'Éternel.

\par 17 Et lorsque la tempête eut cessé, le navire atteignit la terre comme en paix.

\par 18 Et voici, notre père est venu, et nous nous sommes tous réjouis d'un commun accord.

\par 19 Ces deux rêves, je les ai racontés à mon père ; et il me dit : Ces choses doivent s'accomplir en leur temps, après qu'Israël ait enduré beaucoup de choses.

\par 20 Alors mon père me dit : Je crois en Dieu que Joseph est vivant, car je vois toujours que l'Éternel le compte parmi vous.

\par 21 Et il dit en pleurant : Ah moi, mon fils Joseph, tu vis, même si je ne te vois pas, et que tu ne vois pas Jacob qui t'a engendré.

\par 22 Il m'a donc fait pleurer aussi par ces paroles, et j'ai brûlé dans mon cœur de déclarer que Joseph avait été vendu, mais j'avais peur de mes frères.

\par 23 Et voilà ! mes enfants, je vous ai montré dans les derniers temps comment tout arrivera en Israël.

\par 24 Ordonnez-vous donc aussi à vos enfants de s'unir à Lévi et à Juda ; car c'est par eux que le salut d'Israël surgira, et en eux Jacob sera béni.

\par 25 Car Dieu apparaîtra à travers leurs tribus, habitant parmi les hommes sur la terre, pour sauver la race d'Israël et pour rassembler les justes parmi les païens.

\par 26 Si vous faites ce qui est bon, mes enfants, les hommes et les anges vous béniront ; et Dieu sera glorifié parmi les païens à travers vous, et le diable fuira loin de vous, et les bêtes sauvages vous craindront, et l'Éternel vous aimera, et les anges s'attacheront à vous.

\par 27 Comme un homme qui a bien élevé un enfant est gardé dans un bon souvenir ; de même, pour une bonne œuvre, il y a un bon souvenir devant Dieu.

\par 28 Mais celui qui ne fait pas ce qui est bon, les anges et les hommes le maudiront, et Dieu sera déshonoré parmi les païens à travers lui, et le diable fera de lui son instrument particulier, et toute bête sauvage le maîtrisera. , et l'Éternel le haïra.

\par 29 Car les commandements de la loi sont doubles, et il faut les accomplir par la prudence.

\par 30 Car il y a un temps pour qu'un homme embrasse sa femme, et un temps pour s'en abstenir pour sa prière.

\par 31 Ainsi donc, il y a deux commandements ; et, à moins qu’ils ne soient faits dans l’ordre voulu, ils entraînent un très grand péché sur les hommes.

\par 32 Il en est de même des autres commandements.

\par 33 Soyez donc sages en Dieu, mes enfants, et prudents, comprenant l'ordre de ses commandements et les lois de chaque parole, afin que le Seigneur vous aime,

\par 34 Et après leur avoir adressé de nombreuses paroles semblables, il les exhorta à transporter ses ossements à Hébron et à l'enterrer avec ses pères.

\par 35 Et après avoir mangé et bu de bon cœur, il se couvrit le visage et mourut.

\par 36 Et ses fils firent selon tout ce que Nephtali, leur père, leur avait ordonné.



\end{document}