\begin{document}

\title{Testament de Gad}

\chapter{1}

\par \textit{Gad, le neuvième fils de Jacob et de Zilpa. Berger et homme fort mais meurtrier dans l’âme. Le verset 25 est une définition remarquable de la haine.}

\par 1 LA copie du testament de Gad, ce qu'il dit à ses fils, la cent vingt-cinquième année de sa vie, leur disant :

\par 2 Écoutez, mes enfants, j'étais le neuvième fils de Jacob, et j'étais vaillant dans la garde des troupeaux.

\par 3 C'est pourquoi je gardais le troupeau la nuit ; et chaque fois que le lion, ou le loup, ou quelque bête sauvage venait contre le troupeau, je le poursuivais, et l'atteignant, je saisissais son pied avec ma main et le jetais à un jet de pierre, et ainsi je le tuais.

\par 4 Or Joseph, mon frère, faisait paître le troupeau avec nous pendant plus de trente jours, et étant jeune, il tomba malade à cause de la chaleur.

\par 5 Et il retourna à Hébron auprès de notre père, qui le fit coucher près de lui, parce qu'il l'aimait beaucoup.

\par 6 Et Joseph dit à notre père que les fils de Zilpa et de Bilhah tuaient les meilleurs du troupeau et les mangeaient contre le jugement de Ruben et de Juda.

\par 7 Car il a vu que j'avais délivré un agneau de la gueule d'un ours, et que j'avais fait mourir l'ours ; mais nous avions tué l'agneau, étant attristés de ce qu'il ne pouvait pas vivre et que nous l'avions mangé.

\par 8 Et à ce propos, j'étais en colère contre Joseph jusqu'au jour où il fut vendu.

\par 9 Et un esprit de haine était en moi, et je ne voulais ni entendre parler de Joseph des oreilles, ni le voir des yeux, parce qu'il nous réprimandait en face, disant que nous mangions du troupeau sans Juda.

\par 10 Tout ce qu'il disait à notre père, il le croyait.

\par 11 J'avoue maintenant mon gin, mes enfants, que souvent j'ai voulu le tuer, parce que je le haïssais de tout mon cœur.

\par 12 De plus, je le haïssais encore plus à cause de ses rêves ; et j'ai voulu l'arracher du pays des vivants, comme un bœuf lèche l'herbe des champs.

\par 13 Et Juda le vendit secrètement aux Ismaélites.

\par 14 Ainsi le Dieu de nos pères l'a délivré de nos mains, afin que nous ne commettions pas de grandes iniquités en Israël.

\par 15 Et maintenant, mes enfants, écoutez les paroles de la vérité pour pratiquer la justice et toute la loi du Très-Haut, et ne vous égarez pas par l'esprit de haine, car il est mauvais dans toutes les actions des hommes.

\par 16 Tout ce qu'un homme fait, celui qui le déteste l'abomine ; et même si un homme pratique la loi de l'Éternel, il ne le loue pas ; Même si un homme craint le Seigneur et prend plaisir à ce qui est juste, il ne l'aime pas.

\par 17 Il méprise la vérité, il envie celui qui prospère, il accueille les mauvaises paroles, il aime l'orgueil, car la haine aveugle son âme ; comme je regardais aussi Joseph.

\par 18 Prenez donc garde, mes enfants, à la haine, car elle produit l'iniquité même contre l'Éternel lui-même.

\par 19 Car il n'écoute pas les paroles de ses commandements concernant l'amour de son prochain, et il pèche contre Dieu.

\par 20 Car si un frère trébuche, il se plaît immédiatement à le proclamer à tous les hommes, et il est urgent qu'il soit jugé pour cela, qu'il soit puni et mis à mort.

\par 21 Et si c'est un serviteur, il l'excite contre son maître, et avec toutes les afflictions il complote contre lui, s'il est possible qu'il soit mis à mort.

\par 22 Car la haine agit avec envie aussi contre ceux qui prospèrent ; aussi longtemps qu'elle entend parler ou voit leur succès, elle languit toujours.

\par 23 Car, de même que l'amour vivifierait même les morts et rappellerait ceux qui sont condamnés à mourir, de même la haine tuerait les vivants, et ceux qui avaient péché véniellement ne souffriraient pas de vivre.

\par 24 Car l'esprit de haine travaille de concert avec Satan, par la précipitation des esprits, en toutes choses à la mort des hommes ; mais l'esprit d'amour travaille de concert avec la loi de Dieu dans la longanimité pour le salut des hommes.

\par 25 La haine est donc mauvaise, car elle s'associe constamment au mensonge, au discours contre la vérité ; et cela rend les petites choses grandes, et rend la lumière ténèbres, et rend le doux amer, et enseigne la calomnie, et allume la colère, et attise la guerre, la violence et toute convoitise ; cela remplit le cœur de maux et de poison diabolique.

\par 26 C'est pourquoi je vous dis ces choses par expérience, mes enfants, afin que vous chassiez la haine qui vient du diable et que vous vous attachiez à l'amour de Dieu.

\par 27 La justice chasse la haine, l'humilité détruit l'envie.

\par 28 Car celui qui est juste et humble a honte de faire ce qui est injuste, n'étant pas repris par un autre, mais par son propre cœur, parce que le Seigneur regarde son inclination.

\par 29 Il ne parle pas contre un saint homme, parce que la crainte de Dieu l'emporte sur la haine.

\par 30 Car craignant d'offenser le Seigneur, il ne fera de mal à personne, même en pensée.

\par 31 J'ai enfin appris ces choses, après m'être repenti concernant Joseph.

\par 32 Car la vraie repentance, selon Dieu, détruit l'ignorance, chasse les ténèbres, éclaire les yeux, donne la connaissance à l'âme et conduit l'esprit au salut.

\par 33 Et ce qu'il n'a pas appris de l'homme, il le sait par la repentance.

\par 34 Car Dieu m'a fait venir une maladie du foie ; et si les prières de Jacob mon père ne m'avaient pas secouru, cela n'aurait guère échoué mais mon esprit était parti.

\par 35 Car les choses qu'un homme transgresse sont aussi punies.

\par 36 Puisque donc mon foie était impitoyablement opposé à Joseph, dans mon foie aussi j'ai souffert sans pitié, et j'ai été jugé pendant onze mois, aussi longtemps que j'avais été irrité contre Joseph.

\par \textit{Notes de bas de page}

\par \textit{254:1 Même notre argot actuel est vieux de plusieurs siècles.}

\chapter{2}

\par \textit{Gad exhorte ses auditeurs contre la haine en montrant comment elle lui a causé tant de problèmes. Les versets 8 à 11 sont mémorables.}

\par 1 ET maintenant, mes enfants, je vous exhorte, aimez chacun son frère, et éloignez la haine de vos cœurs, aimez-vous les uns les autres en actes, en paroles et selon les inclinations de l'âme.

\par 2 Car, en présence de mon père, j'ai parlé paisiblement à Joseph ; et quand je fus sorti, l'esprit de haine obscurcit mon esprit et poussa mon âme à le tuer.

\par 3 Aimez-vous les uns les autres de tout votre cœur ; et si quelqu'un pèche contre toi, parle-lui paisiblement, et ne garde pas dans ton âme la fraude ; et s'il se repent et se confesse, pardonne-lui.

\par 4 Mais s'il le nie, ne te mets pas en colère contre lui, de peur qu'il ne prenne le poison de toi et qu'il ne se mette à jurer et que tu ne péches ainsi doublement.

\par 5 Qu'un autre homme n'entende pas tes secrets lorsqu'il est engagé dans une querelle légale, de peur qu'il ne vienne à te haïr, à devenir ton ennemi et à commettre un grand péché contre toi ; car souvent il s'adresse à toi avec ruse ou s'occupe de toi avec de mauvaises intentions.

\par 6 Et s'il le nie et qu'il éprouve néanmoins un sentiment de honte lorsqu'il est réprimandé, renoncez à le réprimander.

\par 7 Car celui qui nie peut se repentir afin de ne plus te faire du tort ; oui, il peut aussi t'honorer, craindre et être en paix avec toi.

\par 8 Et s'il est sans vergogne et persiste dans son méfait, pardonne-lui néanmoins de tout cœur, et laisse à Dieu le soin de se venger.

\par 9 Si un homme prospère plus que vous, ne vous inquiétez pas, mais priez aussi pour lui, afin qu'il ait une parfaite prospérité.

\par 10 car cela vous convient.

\par 11 Et s'il s'élève encore davantage, ne l'envie pas, rappelant que toute chair mourra ; et louez Dieu, qui donne à tous les choses bonnes et utiles.

\par 12 Recherchez les jugements du Seigneur, et votre esprit se reposera et sera en paix.

\par 13 Et même si un homme s'enrichit par de mauvais moyens, comme Ésaü, le frère de mon père, ne sois pas jaloux ; mais attendez la fin du Seigneur.

\par 14 Car s'il enlève à un homme une richesse acquise par de mauvais moyens, il lui pardonne s'il se repent, mais celui qui ne se repent pas est réservé au châtiment éternel.

\par 15 Car le pauvre, s'il est libre de toute envie et plaît au Seigneur en toutes choses, est béni plus que tous les hommes, parce qu'il n'a pas les travaux des hommes vains.

\par 16 Éloignez donc de vos âmes la jalousie, et aimez-vous les uns les autres avec droiture de cœur.

\par 17 Dites donc aussi ces choses à vos enfants, afin qu'ils honorent Juda et Lévi, car d'eux l'Éternel suscitera le salut pour Israël.

\par 18 Car je sais qu'à la fin vos enfants s'éloigneront de lui et marcheront dans la méchanceté, l'affliction et la corruption devant l'Éternel.

\par 19 Et après s'être reposé un peu de temps, il dit encore : Mes enfants, obéissez à votre père et enterrez-moi près de mes pères.

\par 20 Et il leva les pieds et s'endormit en paix.

\par 21 Et après cinq ans, ils le transportèrent à Hébron, et le couchèrent avec ses pères.



\end{document}