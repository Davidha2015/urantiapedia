\begin{document}

\title{Ajouts à Esther}

\chapter{1}

\par 1 Alors Mardochée dit : Dieu a fait ces choses.
\par 2 Car je me souviens d'un rêve que j'ai vu concernant ces choses, et rien de celui-ci n'a échoué.
\par 3 Une petite fontaine est devenue un fleuve, et il y eut de la lumière, et du soleil, et beaucoup d'eau : ce fleuve est Esther, que le roi épousa et fit reine :
\par 4 Et les deux dragons sont moi et Aman.
\par 5 Et les nations étaient celles qui s'étaient rassemblées pour détruire le nom des Juifs :
\par 6 Et ma nation, c'est cet Israël, qui a crié à Dieu et qui a été sauvé ; car l'Éternel a sauvé son peuple, et l'Éternel nous a délivrés de tous ces maux, et Dieu a opéré des signes et de grands prodiges qui n'ont pas été accomplis. a été fait parmi les Gentils.
\par 7 C'est pourquoi il a fait deux sorts, un pour le peuple de Dieu, et un autre pour tous les païens.
\par 8 Et ces deux sorts arrivèrent à l'heure, au temps et au jour du jugement, devant Dieu parmi toutes les nations.
\par 9 Ainsi Dieu se souvint de son peuple et justifia son héritage.
\par 10 C'est pourquoi ces jours seront pour eux le mois d'Adar, le quatorzième et le quinzième jour du même mois, avec assemblée, joie et allégresse devant Dieu, selon les générations, pour toujours parmi son peuple.

\chapter{2}

\par 1 La quatrième année du règne de Ptolémée et Cléopâtre, Dosithée, qui se disait prêtre et Lévite, et Ptolémée son fils, apportèrent cette épître de Phurim, qu'ils disaient être la même, et que Lysimaque, fils de Ptolémée, qui était à Jérusalem, l'avait interprété.
\par 2 La deuxième année du règne d'Artexerxès le grand, le premier jour du mois de Nisan, Mardocheus, fils de Jaïrus, fils de Sémei, fils de Cisaï, de la tribu de Benjamin, eut un songe ;
\par 3 qui était Juif, et demeurait dans la ville de Suse, un grand homme, serviteur à la cour du roi.
\par 4 Il était aussi l'un des captifs que Nabuchodonosor, roi de Babylone, emporta de Jérusalem avec Jechonias, roi de Judée ; et c'était son rêve :
\par 5 Voici un bruit de tumulte, avec du tonnerre, des tremblements de terre et du tumulte dans le pays :
\par 6 Et voici, deux grands dragons sortirent prêts à combattre, et leur cri était grand.
\par 7 Et à leur cri, toutes les nations étaient prêtes à combattre, afin de combattre les justes.
\par 8 Et voici un jour de ténèbres et d'obscurité, de tribulation et d'angoisse, d'affliction et de grand tumulte sur la terre.
\par 9 Et toute la nation juste était troublée, craignant ses propres maux, et était prête à périr.
\par 10 Alors ils crièrent vers Dieu, et à leur cri, comme s'il sortait d'une petite fontaine, il se fit un grand déluge, même beaucoup d'eau.
\par 11 La lumière et le soleil se levèrent, et les humbles furent exaltés, et dévorèrent les glorieux.
\par 12 Or, lorsque Mardochée, qui avait vu ce rêve et ce que Dieu avait décidé de faire, était éveillé, il gardait ce rêve à l'esprit et, jusqu'à la nuit, il désirait le connaître.

\chapter{3}

\par 1 Et Mardochée se reposa dans la cour avec Gabatha et Tharra, les deux eunuques du roi et gardiens du palais.
\par 2 Et il entendit leurs desseins, et scruta leurs desseins, et apprit qu'ils étaient sur le point de mettre la main sur le roi Artexerxès ; et c'est pourquoi il en a certifié le roi.
\par 3 Alors le roi examina les deux eunuques, et après qu'ils eurent avoué, ils furent étranglés.
\par 4 Et le roi fit un récit de ces choses, et Mardochée l'écrivit aussi.
\par 5 Le roi ordonna donc à Mardochée de servir à la cour, et il le récompensa pour cela.
\par 6 Cependant Aman, fils d'Amadathe, l'Agaguite, qui était en grande honneur auprès du roi, cherchait à molester Mardochée et son peuple à cause des deux eunuques du roi.

\chapter{4}

\par 1 La copie des lettres était celle-ci : Le grand roi Artexerxès écrit ces choses aux princes et gouverneurs qui sont sous lui depuis l'Inde jusqu'à l'Éthiopie dans cent vingt-sept provinces.
\par 2 Après cela, je suis devenu seigneur de nombreuses nations et j'ai eu la domination sur le monde entier, sans m'élever avec présomption de mon autorité, mais en me comportant toujours avec équité et douceur, j'ai résolu d'installer continuellement mes sujets dans une vie tranquille, et rendant mon royaume paisible et ouvert au passage jusqu'aux côtes les plus éloignées, pour renouveler la paix, qui est désirée par tous les hommes.
\par 3 Or, lorsque j'ai demandé à mes conseillers comment cela pourrait être réalisé, Aman, qui excellait en sagesse parmi nous, et qui était approuvé pour sa bonne volonté constante et sa fidélité inébranlable, et qui avait l'honneur de la deuxième place dans le royaume,
\par 4 Il nous a été déclaré que dans toutes les nations du monde, il y avait dispersé un certain peuple malveillant, qui avait des lois contraires à toutes les nations, et méprisait continuellement les commandements des rois, ainsi que l'unification de nos royaumes, honorablement voulue par nous. ne peut pas avancer.
\par 5 Voyant donc, nous comprenons que ce peuple seul est continuellement en opposition avec tous les hommes, différant par la manière étrange de leurs lois, et le mal affecté à notre état, faisant tout le mal qu'il peut pour que notre royaume ne soit pas fermement établi :
\par 6 C'est pourquoi nous avons ordonné que tous ceux qui vous sont signifiés par écrit par Aman, qui est ordonné sur les affaires et qui est le plus proche de nous, seront tous complètement détruits par l'épée de leurs ennemis, sans toute miséricorde ni pitié, le quatorzième jour du douzième mois d'Adar de cette année :
\par 7 Afin que ceux qui, autrefois et maintenant, sont méchants, puissent en un jour entrer avec violence dans la tombe, et ainsi faire en sorte que nos affaires soient toujours bien réglées et sans problème.
\par 8 Alors Mardochée réfléchit à toutes les œuvres du Seigneur et lui fit sa prière :
\par 9 En disant : Seigneur, Seigneur, roi tout-puissant, car le monde entier est en ton pouvoir, et si tu as décidé de sauver Israël, personne ne peut te contredire :
\par 10 Car tu as fait le ciel et la terre, et toutes les choses merveilleuses qui sont sous le ciel.
\par 11 Tu es le Seigneur de toutes choses, et il n'y a personne qui puisse te résister, qui est le Seigneur.
\par 12 Tu sais toutes choses, et tu sais, Seigneur, que ce n'était ni par mépris, ni par orgueil, ni par aucun désir de gloire, que je ne me suis pas prosterné devant l'orgueilleux Aman.
\par 13 Car j'aurais pu me contenter de bonne volonté pour le salut d'Israël en baisant la plante de ses pieds.
\par 14 Mais j'ai fait cela afin de ne pas préférer la gloire de l'homme à la gloire de Dieu : je n'adorerai que toi, ô Dieu, et je ne le ferai pas avec orgueil.
\par 15 Et maintenant, Seigneur Dieu et Roi, épargne ton peuple ; car leurs yeux sont sur nous pour nous réduire à néant ; oui, ils désirent détruire l’héritage qui t’appartient depuis le commencement.
\par 16 Ne méprise pas la part que tu as délivrée d'Egypte pour toi-même.
\par 17 Écoute ma prière et aie pitié de ton héritage : change notre tristesse en joie, afin que nous puissions vivre, Seigneur, et louer ton nom ; et ne détruis pas la bouche de ceux qui te louent, Seigneur.
\par 18 De la même manière, tout Israël criait avec insistance à l'Éternel, parce que sa mort était devant ses yeux.

\chapter{5}

\par 1 La reine Esther aussi, craignant la mort, recourut au Seigneur :
\par 2 Et elle déposa ses vêtements glorieux, et revêtit les vêtements d'angoisse et de deuil ; et au lieu d'onguents précieux, elle se couvrit la tête de cendre et de fumier, et elle humilia grandement son corps, et tous les lieux de sa joie elle rempli de ses cheveux déchirés.
\par 3 Et elle pria l'Éternel, le Dieu d'Israël, en disant : Ô mon Seigneur, tu es seul notre roi : aide-moi, femme désolée, qui n'as d'autre secours que toi :
\par 4 Car mon danger est entre mes mains.
\par 5 Dès ma jeunesse, j'ai entendu dans la tribu de ma famille que toi, Éternel, tu as pris Israël d'entre tous les peuples, et nos pères de tous leurs prédécesseurs, pour un héritage perpétuel, et tu as accompli tout ce que tu avais promis. eux.
\par 6 Et maintenant nous avons péché devant toi : c'est pourquoi tu nous as livrés entre les mains de nos ennemis,
\par 7 Parce que nous avons adoré leurs dieux : Ô Seigneur, tu es juste.
\par 8 Néanmoins, cela ne les satisfait pas que nous soyons dans une amère captivité ; mais ils se sont frappés la main avec leurs idoles,
\par 9 Afin qu'ils abolissent ce que tu as ordonné de ta bouche, qu'ils détruiront ton héritage, qu'ils fermeront la bouche de ceux qui te louent, et qu'ils éteignent la gloire de ta maison et de ton autel,
\par 10 Et ouvrez la bouche des païens pour proclamer les louanges des idoles, et pour magnifier à jamais un roi charnel.
\par 11 O Seigneur, ne donne pas ton sceptre à ceux qui ne sont rien, et qu'ils ne se moquent pas de notre chute ; mais retournez leur stratégie contre eux-mêmes et faites de celui qui a commencé cela contre nous un exemple.
\par 12 Souviens-toi, Seigneur, fais-toi connaître au temps de notre affliction, et donne-moi de l'assurance, ô Roi des nations et Seigneur de toute puissance.
\par 13 Donne-moi un discours éloquent dans ma bouche devant le lion; tourne son cœur à haïr celui qui nous combat, afin qu'il y ait une fin pour lui et pour tous ceux qui pensent comme lui.
\par 14 Mais délivre-nous par ta main, et aide-moi, moi qui suis désolé et qui n'ai d'autre secours que toi.
\par 15 Tu sais toutes choses, ô Seigneur ; tu sais que je hais la gloire des injustes, et que j'ai horreur du lit des incirconcis et de tous les païens.
\par 16 Tu connais ma nécessité, car j'ai horreur du signe de ma grandeur qui est sur ma tête les jours où je me montre, et que je l'ai en horreur comme un vêtement menstruel, et que je ne le porte pas quand je suis. privé par moi-même.
\par 17 Et que ta servante n'a pas mangé à la table d'Aman, et que je n'ai pas beaucoup estimé le festin du roi, ni bu le vin des libations.
\par 18 Ta servante n'a eu aucune joie depuis le jour où j'ai été amené ici jusqu'à présent, mais en toi, Seigneur Dieu d'Abraham.
\par 19 Ô toi, Dieu puissant au-dessus de tout, écoute la voix des abandonnés et délivre-nous des mains des méchants, et délivre-moi de ma peur.

\chapter{6}

\par 1 Et le troisième jour, après avoir terminé ses prières, elle ôta ses vêtements de deuil et revêtit ses vêtements glorieux.
\par 2 Et étant glorieusement parée, après avoir invoqué Dieu, qui est le spectateur et le sauveur de toutes choses, elle prit avec elle deux jeunes filles :
\par 3 Et sur celui-là, elle s'appuyait, comme si elle se comportait avec délicatesse ;
\par 4 Et l'autre la suivit, portant sa suite.
\par 5 Et elle était rouge à cause de la perfection de sa beauté, et son visage était joyeux et très aimable ; mais son cœur était angoissé par la peur.
\par 6 Alors ayant franchi toutes les portes, elle se présenta devant le roi, qui était assis sur son trône royal, et était vêtue de toutes ses robes de majesté, toutes étincelantes d'or et de pierres précieuses ; et il était très affreux.
\par 7 Alors, levant son visage qui brillait de majesté, il la regarda avec un regard très féroce ; et la reine tomba, et devint pâle, et s'évanouit, et se pencha sur la tête de la servante qui la précédait.
\par 8 Alors Dieu changea l'esprit du roi en douceur, qui, effrayé, sauta de son trône et la prit dans ses bras, jusqu'à ce qu'elle revienne à elle-même, et la réconforta par des paroles d'amour et lui dit :
\par 9 Esther, qu'y a-t-il ? Je suis ton frère, prends courage :
\par 10 Tu ne mourras pas, même si notre commandement est général : approche-toi.
\par 11 Et ainsi il leva son sceptre d'or et le posa sur son cou,
\par 12 Et il l'embrassa, et dit : Parle-moi.
\par 13 Alors elle lui dit : Je t'ai vu, mon seigneur, comme un ange de Dieu, et mon cœur a été troublé par la crainte de ta majesté.
\par 14 Car tu es merveilleux, Seigneur, et ton visage est plein de grâce.
\par 15 Et pendant qu'elle parlait, elle tomba de faiblesse.
\par 16 Alors le roi fut troublé, et tous ses serviteurs la consolèrent.

\chapter{7}

\par 1 Le grand roi Artexerxès, aux princes et gouverneurs de cent vingt-sept provinces depuis l'Inde jusqu'à l'Éthiopie, et à tous nos fidèles sujets, salut.
\par 2 Beaucoup, plus ils sont souvent honorés de la grande générosité de leurs gracieux princes, plus ils sont fiers,
\par 3 Et efforcez-vous non seulement de nuire à nos sujets, mais, ne pouvant supporter l'abondance, prenez en main la pratique également contre ceux qui leur font du bien :
\par 4 Et ôtez non seulement la reconnaissance parmi les hommes, mais encore les hommes obscènes, élevés par les paroles glorieuses, qui n'ont jamais été bons, qui pensent échapper à la justice de Dieu, qui voit tout et hait le mal.
\par 5 Souvent aussi les paroles justes de ceux à qui on a confié la gestion des affaires de leurs amis, ont amené beaucoup de ceux qui détiennent l'autorité à participer au sang innocent et les ont enveloppés dans des calamités sans remède :
\par 6 En séduisant par le mensonge et la tromperie de leur tempérament obscène l'innocence et la bonté des princes.
\par 7 Maintenant, vous pouvez le constater, comme nous l'avons déclaré, non pas tant par les histoires anciennes, que si vous recherchez ce qui a été mal fait ces derniers temps à travers le comportement pestilentiel de ceux qui sont indignement placés en autorité.
\par 8 Et nous devons veiller au temps à venir, afin que notre royaume soit calme et paisible pour tous les hommes,
\par 9 À la fois en changeant nos desseins et en jugeant toujours les choses qui sont évidentes d'une manière plus égale.
\par 10 Car Aman, Macédonien, fils d'Amadatha, étant vraiment étranger du sang perse, et très éloigné de notre bonté, et comme un étranger reçu de nous,
\par 11 Il avait obtenu jusqu'à présent la faveur que nous montrons envers chaque nation, au point qu'il était appelé notre père, et qu'il était continuellement honoré par tous ceux qui suivaient le roi.
\par 12 Mais lui, ne supportant pas sa grande dignité, a entrepris de nous priver de notre royaume et de notre vie :
\par 13 Ayant, par des tromperies multiples et astucieuses, cherché à nous faire perdre, aussi bien à Mardochée, qui nous a sauvé la vie et sans cesse assuré notre bien, qu'à Esther irréprochable, participante de notre royaume, avec toute leur nation.
\par 14 Car c'est par ces moyens qu'il pensait, nous trouvant dépourvus d'amis, avoir transféré le royaume des Perses aux Macédoniens.
\par 15 Mais nous constatons que les Juifs, que ce méchant misérable a livrés à une destruction totale, ne sont pas des malfaiteurs, mais vivent selon les lois les plus justes :
\par 16 Et qu'ils soient les enfants du Dieu vivant, le plus haut et le plus puissant, qui a ordonné le royaume à nous et à nos ancêtres de la manière la plus excellente.
\par 17 C'est pourquoi vous feriez bien de ne pas mettre à exécution les lettres que vous a envoyées Aman, fils d'Amadatha.
\par 18 Car celui qui a fait ces choses est pendu aux portes de Suse avec toute sa famille : Dieu, qui gouverne toutes choses, lui rend promptement vengeance selon ses mérites.
\par 19 C'est pourquoi vous publierez partout la copie de cette lettre, afin que les Juifs puissent vivre librement selon leurs propres lois.
\par 20 Et vous les aiderez, afin que le même jour, qui est le treizième jour du douzième mois d'Adar, ils se vengent de ceux qui, au temps de leur affliction, s'en prendront à eux.
\par 21 Car Dieu Tout-Puissant a tourné vers eux en joie le jour où le peuple élu aurait dû périr.
\par 22 Vous donc, parmi vos fêtes solennelles, vous célébrerez un grand jour avec tous les festins :
\par 23 Afin que, maintenant et dans l'avenir, nous puissions être en sécurité ainsi que pour les Perses bien traités ; mais à ceux qui conspirent contre nous, un mémorial de destruction.
\par 24 C'est pourquoi toute ville et tout pays qui n'agira pas selon ces choses sera détruit sans miséricorde par le feu et l'épée, et sera rendu non seulement infranchissable pour les hommes, mais aussi très odieux aux bêtes sauvages et aux oiseaux à jamais. .

\end{document}