\begin{document}

\title{Lettre de Jérémie}

\chapter{1}


\par 1 Copie d'une épître que Jérémie envoya à ceux qui devaient être emmenés captifs à Babylone par le roi des Babyloniens, pour les certifier, comme cela lui avait été ordonné de Dieu. À cause des péchés que vous avez commis devant Dieu, vous serez emmenés captifs à Babylone par Nabuchodonosor, roi des Babyloniens.
\par 2 Ainsi, lorsque vous serez arrivés à Babylone, vous y resterez de nombreuses années et pendant une longue période, à savoir sept générations ; et après cela, je vous en ramènerai en paix.
\par 3 Maintenant vous verrez à Babylone des dieux d'argent, d'or et de bois, portés sur les épaules, qui font craindre les nations.
\par 4 Gardez-vous donc de ressembler en aucune manière aux étrangers, ni vous ni l'un d'eux, lorsque vous voyez la multitude devant eux et derrière eux, les adorant.
\par 5 Mais dites dans vos cœurs : Seigneur, nous devons t'adorer.
\par 6 Car mon ange est avec vous, et moi-même je prends soin de vos âmes.
\par 7 Quant à leur langue, elle est polie par l'ouvrier, et eux-mêmes sont dorés et recouverts d'argent ; pourtant ils ne sont que faux et ne peuvent pas parler.
\par 8 Et prenant de l'or, comme pour une vierge qui aime s'égayer, ils font des couronnes pour les têtes de leurs dieux.
\par 9 Parfois aussi les prêtres apportent de l'or et de l'argent de leurs dieux et se les accordent à eux-mêmes.
\par 10 Oui, ils en donneront aux prostituées ordinaires, et les pareront comme des hommes avec des vêtements, [étant] des dieux d'argent, et des dieux d'or et de bois.
\par 11 Mais ces dieux ne peuvent se sauver de la rouille et de la teigne, même s'ils sont couverts de vêtements pourpres.
\par 12 Ils s'essuient le visage à cause de la poussière du temple, quand il y en a beaucoup sur eux.
\par 13 Et celui qui ne peut faire mourir celui qui l'offense tient un sceptre, comme s'il était juge du pays.
\par 14 Il a aussi dans sa main droite un poignard et une hache ; mais il ne peut se délivrer de la guerre et des voleurs.
\par 15 C'est pourquoi on sait qu'ils ne sont pas des dieux : ne les craignez donc pas.
\par 16 Car comme un vase dont un homme se sert ne vaut rien lorsqu'il est brisé ; il en est de même de leurs dieux : lorsqu'ils sont installés dans le temple, leurs yeux sont pleins de poussière par les pieds de ceux qui entrent.
\par 17 Et comme les portes sont fermées de tous côtés à celui qui offense le roi, comme étant voué à souffrir la mort, de même les prêtres ferment leurs temples avec des portes, des serrures et des barreaux, de peur que leurs dieux ne soient gâtés par des voleurs.
\par 18 Ils allument ces bougies, oui, plus que pour eux-mêmes, dont ils ne peuvent en voir une.
\par 19 Ils sont comme l'une des poutres du temple, et pourtant ils disent que leurs cœurs sont rongés par les choses qui rampent hors de la terre ; et quand ils les mangent ainsi que leurs vêtements, ils ne le sentent pas.
\par 20 Leurs visages sont noircis par la fumée qui sort du temple.
\par 21 Sur leurs corps et leurs têtes sont assis des chauves-souris, des hirondelles et des oiseaux, ainsi que des chats.
\par 22 Par ceci vous saurez qu'ils ne sont pas des dieux : ne les craignez donc pas.
\par 23 Malgré l'or qui les entoure pour les rendre beaux, s'ils n'enlèvent pas la rouille, ils ne brilleront pas ; car ils ne le sentaient pas non plus lorsqu'ils étaient fondus.
\par 24 Les choses qui ne respirent pas s'achètent au prix le plus élevé.
\par 25 Ils sont portés sur les épaules, n'ayant pas de pieds par lesquels ils déclarent aux hommes qu'ils ne valent rien.
\par 26 Ceux aussi qui les servent ont honte : car s'ils tombent à terre à quelque moment que ce soit, ils ne peuvent pas se relever d'eux-mêmes ; ni, si on les redresse, ils ne peuvent bouger d'eux-mêmes ; et s’ils sont courbés, ils ne peuvent pas non plus se redresser ; mais ils présentent des présents devant eux comme à des morts.
\par 27 Quant aux choses qui leur sont sacrifiées, leurs prêtres les vendent et en abusent ; de la même manière, leurs femmes en mettent une partie dans du sel ; mais ils n'en donnent rien aux pauvres et aux impuissants.
\par 28 Les femmes en règles et les femmes en couches mangent leurs sacrifices : à ces choses vous pouvez reconnaître qu'elles ne sont pas des dieux : ne les craignez pas.
\par 29 Car comment peut-on les appeler dieux ? parce que les femmes mettent la viande devant les dieux de l'argent, de l'or et du bois.
\par 30 Et les prêtres sont assis dans leurs temples, leurs vêtements déchirés, la tête et la barbe rasées, et rien sur la tête.
\par 31 Ils rugissent et crient devant leurs dieux, comme le font les hommes lors d'une fête où l'on est mort.
\par 32 Les prêtres aussi enlèvent leurs vêtements, et habillent leurs femmes et leurs enfants.
\par 33 Qu'on leur fasse du mal ou du bien, ils ne peuvent pas le récompenser : ils ne peuvent ni établir un roi, ni l'abattre.
\par 34 De la même manière, ils ne peuvent donner ni richesses ni argent : même si un homme leur fait un vœu et ne le tient pas, ils n'en exigeront pas.
\par 35 Ils ne peuvent sauver personne de la mort, ni délivrer le faible du puissant.
\par 36 Ils ne peuvent rendre la vue à un aveugle, ni secourir aucun homme dans sa détresse.
\par 37 Ils ne peuvent faire preuve de miséricorde envers la veuve, ni faire du bien à l'orphelin.
\par 38 Leurs dieux de bois, recouverts d'or et d'argent, sont comme les pierres taillées dans la montagne : ceux qui les adorent seront confus.
\par 39 Comment donc pourrait-on penser et dire qu'ils sont des dieux, alors que même les Chaldéens eux-mêmes les déshonorent ?
\par 40 Et s'ils voient un muet qui ne peut pas parler, ils l'amènent et supplient Bel de parler, comme s'il pouvait comprendre.
\par 41 Mais ils ne peuvent pas comprendre cela eux-mêmes et les quitter, car ils n'ont aucune connaissance.
\par 42 Les femmes aussi, avec des cordes autour d'elles, assises dans les chemins, brûlent du son pour se parfumer ; mais si l'une d'entre elles, attirée par un passant, couche avec lui, elle reproche à son prochain de n'avoir pas été jugée aussi digne qu'elle, ni son cordon cassé.
\par 43 Tout ce qui se fait parmi eux est faux : comment peut-on donc penser ou dire qu'ils sont des dieux ?
\par 44 Ils sont faits de charpentiers et d'orfèvres : ils ne peuvent être que ce que les ouvriers veulent qu'ils soient.
\par 45 Et ceux-là mêmes qui les ont faits ne pourront jamais durer longtemps ; Comment donc les choses qui en sont faites devraient-elles être des dieux ?
\par 46 Car ils ont laissé des mensonges et des reproches à ceux qui sont venus après.
\par 47 Car quand survient une guerre ou une peste sur eux, les prêtres se consultent entre eux pour savoir où ils peuvent se cacher avec eux.
\par 48 Comment donc les hommes ne peuvent-ils pas comprendre qu'ils ne sont pas des dieux, qui ne peuvent se sauver ni de la guerre, ni de la peste ?
\par 49 Car, puisqu'ils sont en bois et recouverts d'argent et d'or, on saura désormais qu'ils sont faux.
\par 50 Et il apparaîtra manifestement à toutes les nations et à tous les rois qu'ils ne sont pas des dieux, mais des œuvres de mains d'homme, et qu'il n'y a aucune œuvre de Dieu en eux.
\par 51 Qui donc ne saurait savoir qu'ils ne sont pas des dieux ?
\par 52 Car ils ne peuvent pas non plus établir un roi dans le pays, ni donner de la pluie aux hommes.
\par 53 Ils ne peuvent pas non plus juger leur propre cause, ni réparer un tort, faute de quoi ils sont incapables : car ils sont comme des corbeaux entre le ciel et la terre.
\par 54 Alors, quand le feu tombera sur la maison des dieux en bois, ou recouverte d'or ou d'argent, leurs prêtres s'enfuiront et s'échapperont ; mais eux-mêmes seront brûlés comme des poutres.
\par 55 De plus, ils ne peuvent résister à aucun roi ni à aucun ennemi : comment peut-on alors penser ou dire qu'ils sont des dieux ?
\par 56 Ces dieux de bois et recouverts d'argent ou d'or ne peuvent échapper ni aux voleurs ni aux brigands.
\par 57 Ceux qui sont forts prennent l'or, l'argent et les vêtements dont ils sont vêtus, et s'en vont avec, et ils ne peuvent s'aider eux-mêmes.
\par 58 Il vaut donc mieux être un roi qui montre sa puissance, ou bien un instrument utile dans une maison, dont le propriétaire aura l'usage, que de tels faux dieux ; ou être une porte dans une maison, pour y garder des choses qui ne sont pas de tels faux dieux, ou une colonne de bois dans un palais que de tels faux dieux.
\par 59 Car le soleil, la lune et les étoiles, étant brillants et envoyés pour accomplir leurs offices, sont obéissants.
\par 60 De la même manière, l'éclair lorsqu'il éclate est facile à voir ; et de la même manière le vent souffle dans tous les pays.
\par 61 Et quand Dieu commande aux nuages ​​de parcourir le monde entier, ils font ce qui leur est ordonné.
\par 62 Et le feu envoyé d'en haut pour consumer les collines et les bois fait ce qui est commandé ; mais ceux-ci ne leur ressemblent ni en apparence ni en puissance.
\par 63 C'est pourquoi il ne faut ni supposer ni dire qu'ils sont des dieux, puisqu'ils ne sont capables ni de juger des causes, ni de faire du bien aux hommes.
\par 64 Sachant donc qu'ils ne sont pas des dieux, ne les craignez pas,
\par 65 Car ils ne peuvent ni maudire ni bénir les rois :
\par 66 Ils ne peuvent non plus montrer des signes dans les cieux parmi les païens, ni briller comme le soleil, ni donner de la lumière comme la lune.
\par 67 Les bêtes valent mieux qu'eux, car elles savent se mettre à l'abri et se servir elles-mêmes.
\par 68 Il ne nous est donc en aucun cas évident qu'ils sont des dieux : ne les craignez donc pas.
\par 69 Car, comme un épouvantail dans un jardin de concombres ne garde rien, ainsi sont leurs dieux en bois, recouverts d'argent et d'or.
\par 70 Et de même leurs dieux de bois, recouverts d'argent et d'or, sont semblables à une épine blanche dans un verger, sur laquelle tout oiseau est assis ; ainsi qu'à un cadavre, c'est-à-dire à l'est dans l'obscurité.
\par 71 Et vous reconnaîtrez qu'ils ne sont pas des dieux à la pourpre éclatante qui pourrit dessus ; et eux-mêmes seront ensuite mangés et seront un opprobre dans le pays.
\par 72 Mieux vaut donc le juste qui n'a pas d'idoles, car il sera loin de l'opprobre.



\end{document}