\begin{document}

\title{3 Macchabées}


\chapter{1}

\par 1 Lorsque Philopator apprit de ceux qui revenaient que les régions qu'il contrôlait avaient été saisies par Antiochus, il donna des ordres à toutes ses forces, tant d'infanterie que de cavalerie, emmena avec lui sa sœur Arsinoé et marcha vers la région proche. Raphia, où campaient les partisans d'Antiochus.
\par 2 Mais un certain Théodote, déterminé à mettre à exécution le complot qu'il avait conçu, prit avec lui les meilleures armes ptolémaïques qui lui avaient été délivrées auparavant, et passa de nuit jusqu'à la tente de Ptolémée, avec l'intention d'avoir un seul homme. pour le tuer et ainsi mettre fin à la guerre.
\par 3 Mais Dositheus, connu comme le fils de Drimylus, un Juif de naissance qui plus tard changea de religion et apostasia les traditions ancestrales, avait emmené le roi et avait fait en sorte qu'un certain homme insignifiant dorme dans la tente ; et c'est ainsi que cet homme encourut la vengeance destinée au roi.
\par 4 Lorsqu'un combat acharné s'ensuivit et que les choses tournaient plutôt en faveur d'Antiochus, Arsinoé alla vers les troupes avec des lamentations et des larmes, ses cheveux ébouriffés, et les exhorta à se défendre eux-mêmes et leurs enfants et femmes avec courage, promettant de le faire. donnez-leur chacun deux mines d'or s'ils gagnent la bataille.
\par 5 Et ainsi il arriva que l'ennemi fut mis en déroute dans l'action, et de nombreux captifs furent également faits.
\par 6 Maintenant qu'il avait déjoué le complot, Ptolémée décida de visiter les villes voisines et de les encourager.
\par 7 Ce faisant, et en dotant de présents leurs enceintes sacrées, il renforça le moral de ses sujets.
\par 8 Puisque les Juifs avaient envoyé des membres de leur conseil et des anciens pour le saluer, lui apporter des cadeaux de bienvenue et le féliciter de ce qui s'était passé, il était d'autant plus désireux de leur rendre visite le plus tôt possible.
\par 9 Après son arrivée à Jérusalem, il offrit des sacrifices au Dieu suprême, fit des offrandes de remerciement et fit ce qui convenait au lieu saint. Puis, en entrant dans les lieux et étant impressionné par son excellence et sa beauté,
\par 10 il s'émerveilla du bon ordre du temple et conçut le désir d'entrer dans le saint des saints.
\par 11 Lorsqu'ils disaient que cela n'était pas permis, parce que même les membres de leur propre nation n'étaient pas autorisés à entrer, ni même tous les prêtres, mais seulement le grand prêtre qui prédominait sur tout, et il n'était admis qu'une seule fois. L’année suivante, le roi n’était nullement convaincu.
\par 12 Même après qu'on lui eut lu la loi, il ne cessa de soutenir qu'il devait entrer, disant : Même si ces hommes étaient privés de cet honneur, je ne devrais pas l'être.
\par 13 Et il demanda pourquoi, lorsqu'il entrait dans tous les autres temples, personne ne l'y avait arrêté.
\par 14 Et quelqu'un a dit inconsidérément que c'était une erreur de prendre cela comme un signe en soi.
\par 15 «Mais puisque cela est arrivé», dit le roi, «pourquoi n'entrerais-je pas au moins, qu'ils le veuillent ou non ?»
\par 16 Alors les prêtres, dans tous leurs vêtements, se prosternèrent et supplièrent le Dieu suprême de l'aider dans la situation présente et de conjurer la violence de ce mauvais dessein, et ils remplirent le temple de cris et de larmes ;
\par 17 et ceux qui restaient dans la ville s'agitèrent et se précipitèrent dehors, supposant que quelque chose de mystérieux se passait.
\par 18 Les vierges qui avaient été enfermées dans leurs chambres se précipitèrent avec leurs mères, saupoudrèrent leurs cheveux de poussière et remplirent les rues de gémissements et de lamentations.
\par 19 Les femmes récemment préparées pour le mariage abandonnèrent les chambres nuptiales préparées pour l'union conjugale et, négligeant la modestie appropriée, dans une précipitation désordonnée se rassemblèrent dans la ville.
\par 20 Les mères et les nourrices abandonnaient même les enfants nouveau-nés ici et là, certaines dans les maisons et d'autres dans les rues, et sans se retourner, elles se rassemblaient au temple le plus haut.
\par 21 Diverses étaient les supplications de ceux qui étaient rassemblés là, à cause des complots profanes du roi.
\par 22 De plus, les citoyens les plus audacieux ne toléreraient pas l'achèvement de ses plans ou la réalisation de son objectif.
\par 23 Ils crièrent à leurs semblables de prendre les armes et de mourir courageusement pour la loi ancestrale, et créèrent un désordre considérable dans le lieu saint ; et étant à peine retenus par les vieillards et les anciens, ils recoururent à la même posture de supplication que les autres.
\par 24 Pendant ce temps, la foule, comme auparavant, était engagée dans la prière,
\par 25 tandis que les anciens près du roi essayaient de diverses manières de détourner son arrogance du plan qu'il avait conçu.
\par 26 Mais lui, dans son arrogance, ne prêta attention à rien et commença à s'approcher, déterminé à mener à bien le plan susmentionné.
\par 27 Quand ceux qui étaient autour de lui remarquèrent cela, ils se tournèrent, avec notre peuple, pour invoquer celui qui a tout pouvoir pour les défendre dans la détresse actuelle et pour ne pas négliger cet acte illégal et hautain.
\par 28 Le cri continu, véhément et concerté des foules aboutit à un immense tumulte ; 
\par 29 car il semblait que non seulement les hommes mais aussi les murs et toute la terre alentour résonnaient, car en effet tous à cette époque préféraient la mort à la profanation du lieu.

\chapter{2}

\par 1 Alors le grand prêtre Simon, faisant face au sanctuaire, fléchissant les genoux et étendant les mains avec une dignité calme, pria ainsi :
\par 2 « Seigneur, Seigneur, roi des cieux et souverain de toute la création, saint parmi les saints, seul souverain, tout-puissant, prête attention à nous qui souffrons cruellement à cause d'un homme impie et profane, enflé dans son audace et puissance. »
\par 3 « Car toi, le créateur de toutes choses et le gouverneur de tous, tu es un dirigeant juste, et tu juges ceux qui ont fait quelque chose avec insolence et arrogance. »
\par 4 «Tu as détruit ceux qui dans le passé avaient commis l'injustice, parmi lesquels se trouvaient même des géants qui avaient confiance en leur force et leur audace, que tu as détruits en faisant venir sur eux un déluge sans limites.»
\par 5 « Vous avez consumé par le feu et le soufre les hommes de Sodome qui se comportaient avec arrogance, qui étaient connus pour leurs vices ; et tu en as fait un exemple pour ceux qui viendront ensuite.
\par 6 « Tu as fait connaître ta grande puissance en infligeant des châtiments nombreux et variés à l'audacieux Pharaon qui avait asservi ton saint peuple Israël. »
\par 7 «Et quand il les poursuivait avec des chars et une masse de troupes, tu l'as accablé dans les profondeurs de la mer, mais tu as fait passer en toute sécurité ceux qui avaient mis leur confiance en toi, le Souverain de toute la création.»
\par 8 « Et quand ils eurent vu les œuvres de tes mains, ils te louèrent, le Tout-Puissant. »
\par 9 « Toi, ô roi, lorsque tu as créé la terre sans limites et incommensurable, tu as choisi cette ville et tu as sanctifié ce lieu pour ton nom, bien que tu n'aies besoin de rien ; et après l'avoir glorifié par votre magnifique manifestation, vous en avez fait un fondement solide pour la gloire de votre grand et honoré nom.
\par 10 «Et parce que vous aimez la maison d'Israël, vous avez promis que si nous avions des revers et que des tribulations nous surprenaient, vous écouteriez notre requête lorsque nous viendrions à cet endroit et prierions.»
\par 11 « Et en effet, tu es fidèle et véritable. »
\par 12 «Et parce que souvent, lorsque nos pères étaient opprimés, vous les avez secourus dans leur humiliation et vous les avez délivrés de grands maux»,
\par 13 «Vois maintenant, ô saint Roi, qu'à cause de nos nombreux et grands péchés, nous sommes écrasés de souffrance, soumis à nos ennemis et dépassés par l'impuissance.»
\par 14 « Dans notre chute, cet homme audacieux et profane entreprend de violer le lieu saint sur terre dédié à ton nom glorieux. »
\par 15 «Car votre demeure, le ciel des cieux, est inaccessible à l'homme.»
\par 16 «Mais parce que tu as gracieusement accordé ta gloire à ton peuple Israël, tu as sanctifié ce lieu.»
\par 17 «Ne nous punissez pas pour la souillure commise par ces hommes, et ne nous demandez pas compte de cette profanation, de peur que les transgresseurs ne se vantent de leur colère ou n'exultent dans l'arrogance de leur langue, en disant:»
\par 18 « 'Nous avons piétiné la maison du sanctuaire comme sont piétinées les maisons offensantes.' »
\par 19 « Essuie nos péchés et disperse nos erreurs, et révèle ta miséricorde à cette heure. »
\par 20 « Que vos miséricordes nous atteignent rapidement, et mettent des louanges dans la bouche de ceux qui sont abattus et brisés, et donnez-nous la paix. »
\par 21 Alors Dieu, qui veille sur toutes choses, le premier Père de tous, saint parmi les saints, ayant entendu la supplication légitime, fouetta celui qui s'était élevé dans l'insolence et l'audace.
\par 22 Il le secoua d'un côté et de l'autre comme un roseau secoué par le vent, de sorte qu'il resta impuissant sur le sol et, en plus d'être paralysé dans ses membres, il était même incapable de parler, car il avait été frappé par un juste. jugement.
\par 23 Alors les amis et les gardes du corps, voyant le châtiment sévère qui l'avait frappé, et craignant qu'il ne perde la vie, l'empressèrent de l'entraîner dehors, affolés par leur très grande peur.
\par 24 Au bout d'un moment, il se rétablit, et bien qu'il ait été puni, il ne se repentit en aucune façon, mais s'en alla en proférant d'amères menaces.
\par 25 Lorsqu'il arriva en Égypte, il multiplia ses actes de méchanceté, encouragé par les compagnons de beuverie et les camarades mentionnés précédemment, qui étaient étrangers à tout ce qui est juste.
\par 26 Il ne se contenta pas de ses actes licencieux innombrables, mais il continua aussi avec une telle audace qu'il fit de mauvais bruits dans les différentes localités ; et beaucoup de ses amis, observant attentivement le dessein du roi, suivirent eux-mêmes sa volonté.
\par 27 Il proposa de déshonorer publiquement la communauté juive, et il dressa une pierre sur la tour dans la cour avec cette inscription :
\par 28 « Aucun de ceux qui ne sacrifient pas n'entrera dans leurs sanctuaires, et tous les Juifs seront soumis à un enregistrement avec capitation et au statut d'esclaves. Ceux qui s’y opposeront seront emmenés de force et mis à mort ; »
\par 29 «Ceux qui sont enregistrés doivent également être marqués au feu sur leur corps avec le symbole de la feuille de lierre de Dionysos, et ils seront également réduits à leur ancien statut limité.»
\par 30 Afin qu'il ne paraisse pas un ennemi à tous, il écrivit ci-dessous : « Mais si quelqu'un d'entre eux préfère se joindre à ceux qui ont été initiés aux mystères, il aura l'égalité de citoyenneté avec les Alexandrins. »
\par 31 Or, certains, cependant, avec une horreur évidente du prix à exiger pour le maintien de la religion de leur ville, se rendirent volontiers, car ils espéraient rehausser leur réputation par leur future association avec le roi.
\par 32 Mais la majorité a agi avec fermeté, avec un esprit courageux et ne s'est pas éloignée de sa religion ; et en payant de l'argent en échange de leur vie, ils tentèrent avec confiance de se sauver de l'enregistrement.
\par 33 Ils gardaient résolument l'espoir d'obtenir de l'aide, et ils abhorraient ceux qui se séparaient d'eux, les considérant comme des ennemis de la nation juive et les privant de communion commune et d'entraide.

\chapter{3}

\par 1 Lorsque le roi impie comprit cette situation, il devint si furieux que non seulement il fut enragé contre les Juifs qui vivaient à Alexandrie, mais il se montra encore plus amèrement hostile envers ceux des campagnes ; et il ordonna que tous soient promptement rassemblés en un seul endroit et mis à mort par les moyens les plus cruels.
\par 2 Pendant que ces affaires étaient réglées, une rumeur hostile circulait contre la nation juive par des hommes qui conspiraient pour lui faire du mal, sous prétexte étant donné par un bruit selon lequel ils empêchaient les autres d'observer leurs coutumes.
\par 3 Les Juifs, cependant, ont continué à maintenir une bonne volonté et une loyauté inébranlable envers la dynastie ;
\par 4 mais parce qu'ils adoraient Dieu et se conduisaient selon sa loi, ils gardaient leur séparation en ce qui concerne les aliments. C’est pour cette raison qu’ils paraissaient haineux à certains ;
\par 5 mais comme ils ornaient leur style de vie des bonnes actions des hommes droits, ils étaient établis en bonne réputation parmi tous les hommes.
\par 6 Néanmoins, ceux des autres races ne prêtèrent aucune attention à leur bon service envers leur nation, ce qui était un sujet de conversation commun parmi tous ;
\par 7 au lieu de cela, ils bavardaient sur les différences de culte et de nourriture, alléguant que ces gens n'étaient loyaux ni envers le roi ni envers ses autorités, mais qu'ils étaient hostiles et fortement opposés à son gouvernement. Ils ne leur attachèrent donc aucun reproche ordinaire.
\par 8 Les Grecs qui se trouvaient dans la ville, bien qu'ils n'aient subi aucun tort, lorsqu'ils virent un tumulte inattendu autour de ces gens et les foules qui se formaient tout à coup, n'étaient pas assez forts pour les secourir, car ils vivaient sous la tyrannie. Ils essayèrent de les consoler, attristés par la situation et s'attendant à ce que les choses changent ;
\par 9 car une si grande communauté ne devait pas être abandonnée à son sort alors qu'elle n'avait commis aucune offense.
\par 10 Et déjà certains de leurs voisins, amis et associés avaient pris certains d'entre eux à part en privé et s'engageaient à les protéger et à déployer des efforts plus sérieux pour leur aide.
\par 11 Alors le roi, se vantant de sa bonne fortune présente, et ne considérant pas la puissance du Dieu suprême, mais pensant qu'il persévérerait constamment dans son même dessein, écrivit cette lettre contre eux :
\par 12 « Roi Ptolémée Philopator à ses généraux et soldats en Égypte et dans tous ses districts, salutations et bonne santé. »
\par 13 « Moi-même et notre gouvernement nous portons bien. »
\par 14 «Lorsque notre expédition a eu lieu en Asie, comme vous le savez vous-mêmes, elle a été achevée, comme prévu, par l'alliance délibérée des dieux avec nous dans la bataille»,
\par 15 «et nous avons pensé que nous ne devions pas gouverner les nations habitant la Coele-Syrie et la Phénicie par la puissance de la lance, mais que nous devions les chérir avec clémence et une grande bienveillance, en les traitant volontiers bien.»
\par 16 «Et après avoir accordé de très grands revenus aux temples des villes, nous sommes également allés à Jérusalem, et sommes montés pour honorer le temple de ces gens méchants, qui ne cessent jamais de leur folie.»
\par 17 «Ils ont accepté notre présence en parole, mais sans sincérité par les actes, car lorsque nous avons proposé d'entrer dans leur temple intérieur et de l'honorer avec des offrandes magnifiques et les plus belles»,
\par 18 « ils ont été emportés par leur vanité traditionnelle et nous ont exclus de l'entrée ; mais ils ont été épargnés de l’exercice de notre pouvoir à cause de la bienveillance que nous avons envers tous.
\par 19 « En maintenant leur mauvaise volonté manifeste à notre égard, ils deviennent le seul peuple parmi toutes les nations qui lève la tête haute au mépris des rois et de ses propres bienfaiteurs, et qui ne veut considérer aucune action comme sincère. »
\par 20 «Mais nous, lorsque nous sommes arrivés victorieux en Égypte, nous nous sommes accommodés de leur folie et avons fait ce qui était convenable, puisque nous traitons toutes les nations avec bienveillance.»
\par 21 « Entre autres choses, nous avons fait connaître à tous notre amnistie envers leurs compatriotes d'ici, à la fois en raison de leur alliance avec nous et des myriades d'affaires qui leur ont été libéralement confiées dès le début ; et nous avons osé faire un changement, en décidant à la fois de les juger dignes de la citoyenneté alexandrine et de les faire participer à nos rites religieux réguliers.
\par 22 « Mais dans leur méchanceté innée, ils ont pris cela dans un esprit contraire et ont dédaigné ce qui est bon. Puisqu’ils inclinent constamment au mal, »
\par 23 « non seulement ils méprisent la citoyenneté inestimable, mais aussi par la parole et par le silence ils abominent le petit nombre d'entre eux qui sont sincèrement disposés envers nous ; dans toutes les situations, conformément à leur mode de vie infâme, ils soupçonnent secrètement que nous pourrions bientôt modifier notre politique.
\par 24 « C'est pourquoi, pleinement convaincus par ces indications qu'ils sont mal disposés à notre égard à tous égards, nous avons pris des précautions de peur que, si un désordre soudain survenait plus tard contre nous, nous n'ayons derrière notre dos ces impies comme des traîtres. et des ennemis barbares.
\par 25 « C'est pourquoi nous avons donné l'ordre que, dès que cette lettre arrivera, vous nous envoyiez ceux qui vivent parmi vous, ainsi que leurs femmes et leurs enfants, avec des traitements insultants et durs, et solidement liés par des chaînes de fer. , pour subir la mort sûre et honteuse qui sied aux ennemis.
\par 26 « Car lorsque tout cela aura été puni, nous sommes sûrs que pour le temps restant, le gouvernement sera établi pour nous en bon ordre et dans le meilleur état. »
\par 27 «Mais quiconque hébergera des Juifs, des vieillards, des enfants ou même des nourrissons, sera torturé à mort avec les tourments les plus odieux, ainsi que sa famille.»
\par 28 «Quiconque voudra donner des informations recevra les biens de celui qui encourt la punition, ainsi que deux mille drachmes du trésor royal, et obtiendra sa liberté.»
\par 29 « Tout endroit détecté abritant un Juif doit être rendu inaccessible et brûlé au feu, et deviendra inutile à jamais à toute créature mortelle. »
\par 30 La lettre a été écrite sous la forme ci-dessus.

\chapter{4}

\par 1 Partout donc où ce décret arrivait, une fête aux frais de l'État était organisée pour les Gentils avec des cris et de la joie, car l'inimitié invétérée qui avait longtemps été dans leur esprit était maintenant rendue évidente et franche.
\par 2 Mais parmi les Juifs, il y avait des deuils, des lamentations et des cris incessants ; partout, leur cœur brûlait et ils gémissaient à cause de la destruction inattendue qui leur était soudainement décrétée.
\par 3 Quel quartier ou ville, ou quel lieu habitable, ou quelles rues n'étaient pas remplies de deuil et de lamentations pour eux ?
\par 4 Car avec un esprit si dur et si impitoyable, ils étaient tous envoyés par les généraux dans les différentes villes, qu'à la vue de leurs châtiments inhabituels, même certains de leurs ennemis, s'apercevant de l'objet commun de pitié avant eux, leurs yeux réfléchissaient sur l'incertitude de la vie et versaient des larmes devant l'expulsion la plus misérable de ces gens.
\par 5 Car une multitude de vieillards aux cheveux gris, apathiques et courbés par l'âge, étaient emmenés, forcés de marcher à un pas rapide par la violence avec laquelle ils étaient conduits d'une manière si honteuse.
\par 6 Et les jeunes femmes qui venaient d'entrer dans la chambre nuptiale pour partager la vie conjugale échangeaient la joie contre des lamentations, leurs cheveux parfumés de myrrhe parsemés de cendre, et étaient emportées découvertes, poussant toutes ensemble une lamentation au lieu d'un chant de noces, tandis qu'elles ont été déchirés par le mauvais traitement infligé aux païens.
\par 7 Enchaînés et à la vue du public, ils furent violemment entraînés jusqu'au lieu d'embarquement.
\par 8 Leurs maris, dans la fleur de l'âge, le cou entouré de cordes au lieu de guirlandes, passaient le reste de leur fête de mariage en lamentations au lieu de bonne humeur et de réjouissances juvéniles, voyant la mort juste devant eux.
\par 9 Ils furent amenés à bord comme des bêtes sauvages, contraints par des liens de fer ; les uns étaient attachés par le cou aux bancs des bateaux, les autres avaient les pieds fixés par des fers incassables,
\par 10 et en outre ils étaient enfermés sous un pont solide, afin que, les yeux dans l'obscurité totale, ils subissent pendant tout le voyage un traitement digne des traîtres.
\par 11 Lorsque ces hommes furent amenés au lieu appelé Schedia, et que le voyage fut terminé comme le roi l'avait décrété, il ordonna qu'ils soient enfermés dans l'hippodrome qui avait été construit avec un monstrueux mur d'enceinte devant la ville. , et qui convenait bien pour en faire un spectacle évident pour tous ceux qui revenaient dans la ville et pour ceux de la ville qui sortaient à la campagne, de sorte qu'ils ne pouvaient ni communiquer avec les forces du roi ni prétendre en aucune manière être à l'intérieur du circuit de la ville.
\par 12 Et lorsque cela arriva, le roi, apprenant que les compatriotes des Juifs de la ville sortaient souvent en secret pour déplorer amèrement le malheur ignoble de leurs frères,
\par 13 ordonna dans sa colère que ces hommes soient traités exactement de la même manière que les autres, sans omettre aucun détail de leur châtiment.
\par 14 La race entière devait être enregistrée individuellement, non pour les travaux forcés brièvement mentionnés auparavant, mais pour être torturée avec les outrages qu'il avait ordonnés, et finalement être détruite en l'espace d'un seul jour. .
\par 15 L'enregistrement de ces personnes fut donc effectué avec une hâte amère et une attention zélée depuis le lever du soleil jusqu'à son coucher, et bien qu'inachevé, il s'arrêta au bout de quarante jours.
\par 16 Le roi était grandement et continuellement rempli de joie, organisant des fêtes en l'honneur de toutes ses idoles, avec un esprit étranger à la vérité et avec une bouche profane, louant des choses muettes qui ne sont même pas capables de communiquer ou de venir en aide. , et prononçant des paroles inappropriées contre le Dieu suprême.
\par 17 Mais après l'intervalle de temps mentionné ci-dessus, les scribes déclarèrent au roi qu'ils ne pouvaient plus recenser les Juifs à cause de leur multitude innombrable,
\par 18 bien que la plupart d'entre eux soient encore à la campagne, certains résidant encore dans leurs maisons, et certains sur place ; la tâche était impossible pour tous les généraux d'Egypte.
\par 19 Après les avoir sévèrement menacés, les accusant d'avoir été soudoyés pour trouver un moyen de s'échapper, il était clairement convaincu de l'affaire.
\par 20 lorsqu'ils dirent et prouvèrent que le papier et les plumes qu'ils utilisaient pour écrire avaient déjà été distribués.
\par 21 Mais c'était un acte de la providence invincible de celui qui aidait les Juifs du ciel.

\chapter{5}

\par 1 Alors le roi, complètement inflexible, fut rempli d'une colère et d'une colère accablantes ; alors il fit venir Hermon, le gardien des éléphants,
\par 2 et lui ordonna le lendemain de droguer tous les éléphants - au nombre de cinq cents - avec de grandes poignées d'encens et beaucoup de vin pur, et de les chasser, rendus fous par l'abondance somptueuse d'alcool, afin que les Juifs pourraient connaître leur perte.
\par 3 Après avoir donné ces ordres, il retourna à ses festins, avec ceux de ses amis et de l'armée qui étaient particulièrement hostiles aux Juifs.
\par 4 Et Hermon, gardien des éléphants, se mit fidèlement à exécuter les ordres.
\par 5 Les serviteurs chargés des Juifs sortaient le soir, liaient les mains des malheureux et organisaient leur garde pendant la nuit, convaincus que la nation entière connaîtrait sa destruction définitive.
\par 6 Car aux Gentils, il apparut que les Juifs étaient laissés sans aucune aide,
\par 7 parce que dans leurs liens ils étaient enfermés de force de tous côtés. Mais avec des larmes et une voix difficile à faire taire, ils invoquèrent tous le Seigneur Tout-Puissant et Souverain de tout pouvoir, leur Dieu et Père miséricordieux, priant
\par 8 qu'il déjoue par vengeance le mauvais complot contre eux et qu'il les délivre, dans une manifestation glorieuse, du sort maintenant préparé pour eux.
\par 9 Leur supplication monta donc avec ferveur jusqu'au ciel.
\par 10 Hermon, cependant, après avoir drogué les éléphants impitoyables jusqu'à ce qu'ils soient rassasiés d'une grande abondance de vin et rassasiés d'encens, se présenta de bon matin dans la cour pour rendre compte au roi de ces préparatifs.
\par 11 Mais l'Éternel envoya au roi une part du sommeil, ce bienfait qui dès le commencement, nuit et jour, est accordé par celui qui l'accorde à qui il veut.
\par 12 Et par l'action du Seigneur, il fut envahi par un sommeil si agréable et si profond qu'il échoua complètement dans son dessein anarchique et fut complètement frustré dans son plan inflexible.
\par 13 Alors les Juifs, ayant échappé à l'heure fixée, louèrent leur Dieu saint et supplièrent de nouveau celui qui se réconcilie facilement de montrer la puissance de sa main toute-puissante aux païens arrogants.
\par 14 Mais maintenant, comme il était presque le milieu de la dixième heure, la personne qui était chargée des invitations, voyant que les invités étaient assemblés, s'approcha du roi et lui donna un coup de coude.
\par 15 Et après l'avoir réveillé avec peine, il lui fit remarquer que l'heure du banquet s'écoulait déjà, et il lui rendit compte de la situation.
\par 16 Le roi, après avoir réfléchi à cela, retourna à sa boisson et ordonna à ceux qui étaient présents au banquet de s'asseoir en face de lui.
\par 17 Lorsque cela fut fait, il les exhorta à se livrer aux réjouissances et à rendre la partie actuelle du banquet joyeuse en célébrant d'autant plus.
\par 18 Après que la fête eut duré quelque temps, le roi convoqua Hermon et, avec de vives menaces, lui demanda pourquoi les Juifs avaient été autorisés à rester en vie jusqu'à présent.
\par 19 Mais quand lui, avec l'accord de ses amis, fit remarquer que pendant qu'il faisait encore nuit, il avait exécuté complètement l'ordre qui lui avait été donné,
\par 20 Le roi, possédé d'une sauvagerie pire que celle de Phalaris, dit que les Juifs bénéficiaient du sommeil d'aujourd'hui, « mais, ajouta-t-il, demain, préparez sans tarder les éléphants de la même manière à la destruction des sans-loi. Les Juifs!»
\par 21 Lorsque le roi eut parlé, tous ceux qui étaient présents donnèrent d'un commun accord leur approbation, et chacun s'en alla chez lui.
\par 22 Mais ils n'employaient pas tant la durée de la nuit à dormir qu'à inventer toutes sortes d'insultes à l'encontre de ceux qu'ils croyaient condamnés.
\par 23 Alors, dès que le coq eut chanté au petit matin, Hermon, après avoir équipé les bêtes, se mit à les faire avancer dans la grande colonnade.
\par 24 Les foules de la ville s'étaient rassemblées pour ce spectacle des plus pitoyables et attendaient avec impatience le lever du jour.
\par 25 Mais les Juifs, dans leur dernier soupir, puisque le temps était écoulé, étendirent leurs mains vers le ciel et, avec des supplications pleines de larmes et des chants lugubres, implorèrent le Dieu suprême de les secourir de nouveau immédiatement.
\par 26 Les rayons du soleil n'étaient pas encore répandus, et pendant que le roi recevait ses amis, Hermon arriva et l'invita à sortir, indiquant que ce que le roi désirait était prêt à être exécuté.
\par 27 Mais lui, après avoir reçu le rapport et frappé par l'invitation inhabituelle à sortir - car il était complètement envahi par l'incompréhension - demanda quelle était l'affaire pour laquelle cela avait été si zélé pour lui.
\par 28 C'était l'acte de Dieu qui règne sur toutes choses, car il avait implanté dans l'esprit du roi l'oubli des choses qu'il avait imaginées auparavant.
\par 29 Alors Hermon et tous les amis du roi firent remarquer que les bêtes et les forces armées étaient prêtes : « Ô roi, selon ton désir ardent. »
\par 30 Mais à ces paroles, il fut rempli d'une colère irrésistible, parce que, par la providence de Dieu, tout son esprit avait été dérangé à l'égard de ces questions ; et avec un regard menaçant il dit :
\par 31 « Si vos parents ou vos enfants étaient présents, je les aurais préparés pour qu'ils soient un riche festin pour les bêtes sauvages au lieu des Juifs, qui ne me donnent aucun motif de plainte et ont montré à un degré extraordinaire une loyauté pleine et ferme envers mes ancêtres.»
\par 32 «En fait, vous auriez été privés de la vie à la place de celles-ci, sans une affection née de notre éducation commune et de votre utilité.»
\par 33 Alors Hermon reçut une menace inattendue et dangereuse, et ses yeux vacillèrent et son visage tomba.
\par 34 Les amis du roi, un à un, s'éloignèrent d'un air maussade et renvoyèrent le peuple assemblé, chacun à sa propre occupation.
\par 35 Alors les Juifs, après avoir entendu ce que le roi avait dit, louèrent le Seigneur Dieu manifeste, le Roi des rois, car c'était aussi l'aide qu'ils avaient reçue.
\par 36 Le roi cependant convoqua de nouveau la fête de la même manière et exhorta les invités à retourner à leur fête.
\par 37 Après avoir convoqué Hermon, il dit d'un ton menaçant : « Combien de fois, pauvre malheureux, dois-je te donner des ordres sur ces choses ?
\par 38 « Équipez à nouveau les éléphants pour la destruction des Juifs demain ! »
\par 39 Mais les fonctionnaires qui étaient à table avec lui, s'étonnant de son instabilité d'esprit, lui remontrèrent ainsi :
\par 40 «Ô roi, jusqu'à quand vas-tu nous éprouver, comme si nous étions des idiots, en ordonnant maintenant pour la troisième fois qu'ils soient détruits, et en révoquant de nouveau ton décret à ce sujet ?»
\par 41 « En conséquence, la ville est en tumulte à cause de son attente ; il est rempli de foules et est également constamment en danger d'être pillé.
\par 42 Là-dessus, le roi, Phalaris en tout et rempli de folie, ne tint aucun compte des changements d'esprit qui s'étaient produits en lui pour la protection des Juifs, et il jura fermement et irrévocablement qu'il les enverrait. à mort sans délai, mutilé par les genoux et les pieds des bêtes,
\par 43 et marcherait également contre la Judée et la raserait rapidement avec le feu et la lance, et en brûlant jusqu'au sol le temple qui lui était inaccessible le rendrait rapidement vide à jamais de ceux qui y offraient des sacrifices.
\par 44 Alors les amis et les officiers partirent avec une grande joie, et ils postèrent avec confiance les forces armées aux endroits de la ville les plus propices à la garde.
\par 45 Or, alors que les bêtes avaient été amenées pratiquement à un état de folie, pour ainsi dire, par les breuvages très parfumés de vin mélangé à de l'encens et qu'elles avaient été équipées d'appareils effrayants, le gardien des éléphants
\par 46 Vers l'aube, il entra dans la cour — la ville étant maintenant remplie d'innombrables masses de gens se pressant vers l'hippodrome — et pressa le roi de se pencher sur l'affaire en question.
\par 47 Alors lui, après avoir rempli son esprit impie d'une profonde rage, se précipita en pleine force avec les bêtes, voulant assister, d'un cœur invulnérable et de ses propres yeux, à la destruction douloureuse et pitoyable du peuple susmentionné. .
\par 48 Et lorsque les Juifs virent la poussière soulevée par les éléphants sortant par la porte et par les forces armées qui les suivaient, ainsi que par le piétinement de la foule, et entendirent le bruit grand et tumultueux,
\par 49 ils pensèrent que c'était le dernier moment de leur vie, la fin de leur attente la plus misérable, et cédant à la lamentation et aux gémissements, ils s'embrassèrent, embrassant leurs proches et tombant dans les bras les uns des autres — parents et enfants, mères et filles. , et d'autres avec des bébés au sein qui tiraient leur dernier lait.
\par 50 Non seulement cela, mais, considérant le secours qu'ils avaient reçu auparavant du ciel, ils se prosternèrent d'un commun accord à terre, retirant les bébés de leurs seins,
\par 51 et cria d'une voix très forte, implorant le Souverain de tous les pouvoirs de se manifester et d'être miséricordieux envers eux, alors qu'ils se tenaient maintenant aux portes de la mort.

\chapter{6}

\par 1 Alors un certain Éléazar, célèbre parmi les prêtres du pays, qui avait atteint un âge avancé et qui avait été orné toute sa vie de toutes les vertus, ordonna aux anciens autour de lui de cesser d'invoquer le Dieu saint et pria ainsi : :
\par 2 « Roi d'une grande puissance, Dieu Tout-Puissant Très-Haut, gouvernant toute la création avec miséricorde »,
\par 3 « regarde les descendants d'Abraham, ô Père, les enfants du saint Jacob, un peuple de ta part consacrée qui périt comme des étrangers dans un pays étranger. »
\par 4 « Pharaon avec son abondance de chars, l'ancien dirigeant de cette Égypte, exalté par une insolence anarchique et une langue vantardise, tu as détruit avec son armée arrogante en les noyant dans la mer, manifestant la lumière de ta miséricorde sur la nation de Israël.»
\par 5 « Sennachérib exultant dans ses innombrables forces, roi oppressif des Assyriens, qui avait déjà pris le contrôle du monde entier par la lance et s'est élevé contre ta ville sainte, prononçant des paroles douloureuses avec vantardise et insolence, toi, ô Seigneur , brisé en morceaux, montrant ta puissance à de nombreuses nations.
\par 6 « Les trois compagnons de Babylone qui avaient volontairement livré leur vie aux flammes pour ne pas servir des choses vaines, tu les as sauvés indemnes, jusqu'à un cheveu, en humidifiant de rosée la fournaise ardente et en tournant la flamme contre tous leurs ennemis. »
\par 7 «Daniel, qui, par des calomnies envieuses, a été jeté en terre aux lions pour servir de pâture aux bêtes sauvages, tu l'as ramené à la lumière indemne.»
\par 8 «Et Jonas, dépérissant dans le ventre d'un énorme monstre né de la mer, toi, Père, tu as veillé et rendu indemne toute sa famille.»
\par 9 «Et maintenant, vous qui détestez l'insolence, qui êtes tout miséricordieux et protecteur de tous, révélez-vous rapidement à ceux de la nation d'Israël - qui sont outrageusement traités par les Gentils abominables et sans loi.»
\par 10 «Même si nos vies sont devenues empêtrées dans les impiétés de notre exil, délivre-nous de la main de l'ennemi et détruis-nous, Seigneur, quel que soit le sort que tu choisis.»
\par 11 « Que les vains ne louent pas leurs vanités lors de la destruction de votre peuple bien-aimé, en disant : 'Même leur dieu ne les a pas secourus.' »
\par 12 «Mais toi, ô Éternel, qui as toute puissance et tout pouvoir, veille sur nous maintenant et aie pitié de nous qui, par l'insolence insensée des sans-loi, sommes privés de la vie à la manière des traîtres.»
\par 13 «Et que les Gentils se recroquevillent aujourd'hui dans la peur de ta puissance invincible, ô Honoré, qui as le pouvoir de sauver la nation de Jacob.»
\par 14 « Toute la foule des enfants et de leurs parents vous supplie avec des larmes. »
\par 15 « Qu'il soit montré à tous les païens que tu es avec nous, ô Seigneur, et que tu ne t'es pas détourné de nous ; mais comme tu as dit : « Je ne les ai pas négligés, même lorsqu'ils étaient dans le pays de leurs ennemis », accomplis-le donc, ô Seigneur.
\par 16 Au moment où Éléazar terminait sa prière, le roi arriva à l'hippodrome avec les bêtes et toute l'arrogance de ses forces.
\par 17 Et quand les Juifs remarquèrent cela, ils poussèrent de grands cris vers le ciel, si bien que même les vallées voisines résonnèrent d'eux et provoquèrent une terreur incontrôlable sur l'armée.
\par 18 Alors le Dieu le plus glorieux, le plus tout-puissant et le véritable révéla sa face sainte et ouvrit les portes célestes, d'où descendirent deux anges glorieux, d'apparence effrayante, visibles de tous sauf des Juifs.
\par 19 Ils s'opposèrent aux forces ennemies et les remplirent de confusion et de terreur, les liant avec des chaînes inamovibles.
\par 20 Même le roi commença à frémir physiquement, et il oublia sa maussade insolence.
\par 21 Les bêtes se retournèrent contre les forces armées qui les suivaient et commencèrent à les piétiner et à les détruire.
\par 22 Alors la colère du roi se changea en pitié et en larmes à cause des choses qu'il avait imaginées d'avance.
\par 23 Car lorsqu'il entendit les cris et les vit tous tomber précipitamment vers la destruction, il pleura et menaça avec colère ses amis, disant :
\par 24 « Vous commettez une trahison et surpassez les tyrans en cruauté ; et même moi, votre bienfaiteur, vous essayez maintenant de me priver de la domination et de la vie en concevant secrètement des actes sans avantage pour le royaume.
\par 25 « Qui est-ce qui a enlevé chacun de chez lui et a rassemblé ici sans raison ceux qui ont fidèlement tenu les forteresses de notre pays ? »
\par 26 « Qui est-ce qui a si injustement traité d'un traitement outrageant ceux qui, dès le début, différaient de toutes les nations par leur bonne volonté à notre égard et qui ont souvent accepté de bon gré le pire des dangers humains ? »
\par 27 « Dénouez et dénouez leurs liens injustes ! Renvoyez-les chez eux en paix, en implorant pardon pour vos actions passées !
\par 28 « Libérez les fils du Dieu tout-puissant et vivant du ciel, qui, depuis le temps de nos ancêtres jusqu'à présent, a accordé à notre gouvernement une stabilité sans entrave et notable. »
\par 29 Voici donc ce qu'il dit ; et les Juifs, immédiatement libérés, louèrent leur saint Dieu et Sauveur, puisqu'ils avaient maintenant échappé à la mort.
\par 30 Alors le roi, de retour dans la ville, convoqua le fonctionnaire chargé des revenus et lui ordonna de fournir aux Juifs les vins et tout le reste nécessaire pour une fête de sept jours, décidant qu'ils célébreraient leur sauvetage en toute joie dans ce même endroit où ils s'attendaient à rencontrer leur destruction.
\par 31 En conséquence, ceux qui étaient humiliés et proches de la mort, ou plutôt qui se tenaient à ses portes, organisèrent un banquet de délivrance au lieu d'une mort amère et lamentable, et pleins de joie, ils distribuèrent aux célébrants la place qui avait été préparée pour eux. leur destruction et leur enterrement.
\par 32 Ils cessèrent de chanter des chants funèbres et reprirent le chant de leurs pères, louant Dieu, leur Sauveur et faiseur de prodiges. Mettant fin à tout deuil et à tous gémissements, ils formèrent des chœurs en signe de joie paisible.
\par 33 De même, le roi, après avoir convoqué un grand banquet pour célébrer ces événements, rendit grâce sans cesse et généreusement au ciel pour le sauvetage inattendu qu'il avait éprouvé.
\par 34 Et ceux qui croyaient auparavant que les Juifs seraient détruits et deviendraient de la nourriture pour les oiseaux, et qui les avaient enregistrés avec joie, gémissaient alors qu'ils étaient eux-mêmes accablés par la honte, et leur audace cracheuse de feu était ignominieusement éteinte.
\par 35 Mais les Juifs, après avoir constitué la chorale susmentionnée, comme nous l'avons déjà dit, passaient le temps à festoyer au son de joyeuses actions de grâces et de psaumes.
\par 36 Et après avoir ordonné un rite public pour ces choses dans toute leur communauté et pour leurs descendants, ils instituèrent l'observance de ces jours comme une fête, non pour la boisson et la gourmandise, mais à cause de la délivrance qui était venue pour eux. eux à travers Dieu.
\par 37 Alors ils adressèrent une requête au roi, demandant qu'on les renvoie chez eux.
\par 38 Ainsi leur enregistrement s'effectua du vingt-cinquième Pachon au quatrième Épiph, pendant quarante jours ; et leur destruction fut fixée du cinquième au septième jour d'Épiph, les trois jours
\par 39 sur lequel le Seigneur de tous a révélé très glorieusement sa miséricorde et les a tous sauvés ensemble et indemnes.
\par 40 Puis ils festoyèrent, munis de tout par le roi, jusqu'au quatorzième jour, jour où ils demandèrent également leur renvoi.
\par 41 Le roi accéda aussitôt à leur demande et écrivit pour eux la lettre suivante aux généraux des villes, exprimant magnanimement son inquiétude :

\chapter{7}

\par 1 « Roi Ptolémée Philopator aux généraux en Égypte et à tous ceux qui détiennent l'autorité dans son gouvernement, salutations et bonne santé. »
\par 2 «Nous-mêmes et nos enfants nous portons bien, le grand Dieu guidant nos affaires selon notre désir.»
\par 3 « Certains de nos amis, nous pressant fréquemment avec une intention malveillante, nous ont persuadés de rassembler les Juifs du royaume en un corps et de les punir de peines barbares comme traîtres ; »
\par 4 «car ils déclarèrent que notre gouvernement ne serait jamais solidement établi tant que cela ne serait pas accompli, à cause de la mauvaise volonté que ces gens avaient envers toutes les nations.»
\par 5 «Ils les ont également fait sortir avec des traitements sévères comme esclaves, ou plutôt comme traîtres, et, se ceignant d'une cruauté plus sauvage que celle de la coutume scythe, ils ont essayé, sans aucune enquête ni examen, de les mettre à mort.»
\par 6 « Mais nous les avons très sévèrement menacés pour ces actes, et conformément à la clémence que nous avons envers tous les hommes, nous avons à peine épargné leur vie. Depuis que nous avons compris que le Dieu du ciel défend certainement les Juifs, prenant toujours leur part comme un père le fait pour ses enfants, »
\par 7 «et puisque nous avons tenu compte de la bonne volonté amicale et ferme qu'ils avaient envers nous et nos ancêtres, nous les avons justement acquittés de toute accusation, quelle qu'en soit la nature.»
\par 8 «Nous avons également ordonné à chacun de rentrer chez lui, sans que personne, en aucun lieu, ne leur fasse du mal ou ne leur reproche les choses irrationnelles qui se sont produites.»
\par 9 « Car vous devez savoir que si nous projetons quelque mal contre eux ou si nous leur causons quelque chagrin que ce soit, nous n'aurons toujours pas d'homme, mais le Souverain de toute puissance, le Dieu Très-Haut, en toutes choses et inévitablement comme un antagoniste de tout pouvoir. venger de tels actes. Adieu.»
\par 10 En recevant cette lettre, les Juifs ne se précipitèrent pas immédiatement pour partir, mais ils demandèrent au roi que de leurs propres mains ceux de la nation juive qui avaient volontairement transgressé le Dieu saint et la loi de Dieu reçoivent le punition qu'ils méritaient.
\par 11 Car ils déclaraient que ceux qui, à cause du ventre, avaient transgressé les commandements divins ne seraient jamais favorablement disposés envers le gouvernement du roi.
\par 12 Le roi alors, reconnaissant et approuvant la vérité de ce qu'ils disaient, leur accorda une licence générale afin que librement et sans autorité ni surveillance royale, ils puissent exterminer ceux qui, partout dans son royaume, avaient transgressé la loi de Dieu.
\par 13 Après l'avoir applaudi comme il se doit, leurs prêtres et toute la foule crièrent Alléluia et s'en allèrent joyeusement.
\par 14 Et ainsi, en chemin, ils punissaient et mettaient à une mort publique et honteuse tous ceux qu'ils rencontraient parmi leurs compatriotes qui s'étaient souillés.
\par 15 En ce jour-là, ils tuèrent plus de trois cents hommes ; et ils célébrèrent ce jour comme une fête joyeuse, puisqu'ils avaient exterminé les profanateurs.
\par 16 Mais ceux qui s'étaient attachés à Dieu jusqu'à la mort et avaient reçu la pleine jouissance de la délivrance commencèrent à quitter la ville, couronnés de toutes sortes de fleurs très odorantes, remerciant joyeusement et haut et fort le Dieu unique de leurs pères. , le Sauveur éternel d'Israël, en paroles de louange et en toutes sortes de chants mélodieux.
\par 17 Lorsqu'ils furent arrivés à Ptolémaïs, appelée « la rose » à cause d'une caractéristique du lieu, la flotte les attendit, selon le désir commun, pendant sept jours.
\par 18 Là, ils célébrèrent leur délivrance, car le roi leur avait généreusement pourvu de tout pour leur voyage, chacun jusqu'à sa maison.
\par 19 Et lorsqu'ils furent débarqués en paix avec des actions de grâces appropriées, là aussi ils décidèrent d'observer ces jours comme une fête joyeuse pendant le temps de leur séjour.
\par 20 Puis, après les avoir inscrits comme saints sur une colonne et avoir consacré un lieu de prière sur le lieu de la fête, ils repartirent sains et saufs, libres et ravis, car sur l'ordre du roi, ils avaient été amenés sains et saufs par terre et par mer et rivière chacun à sa place.
\par 21 Ils possédaient également un plus grand prestige parmi leurs ennemis, étant tenus en honneur et en crainte ; et ils n'étaient pas du tout sujets à la confiscation de leurs biens par qui que ce soit.
\par 22 En outre, ils récupérèrent tous tous leurs biens, conformément à l'enregistrement, de sorte que ceux qui en détenaient les leur restituèrent avec une extrême crainte. Ainsi, le Dieu suprême a parfaitement accompli de grandes actions pour leur délivrance.
\par 23 Béni soit le Libérateur d'Israël à travers tous les temps ! Amen.

\end{document}