\begin{document}

\title{4 Macchabées}

\chapter{1}

\par \textit{Un aperçu de la philosophie des temps anciens concernant la Raison Inspirée. La civilisation n’a jamais atteint une pensée plus élevée. Une discussion sur les « répressions ». Le verset 48 résume toute la philosophie de l'humanité.}

\par 1 PHILOSOPHIQUE au plus haut degré est la question que je me propose de discuter, à savoir si la Raison Inspirée est la maîtresse suprême des passions ; et quant à sa philosophie, j'implorerais sérieusement votre sérieuse attention.

\par 2 Car non seulement le sujet est généralement nécessaire en tant que branche de la connaissance, mais il inclut l'éloge de la plus grande des vertus, par laquelle j'entends la maîtrise de soi.

\par 3 C'est-à-dire que s'il est prouvé que la raison contrôle les passions opposées à la tempérance, à la gourmandise et à la luxure, il est également clairement démontré qu'elle est maîtresse des passions, comme la malveillance, opposées à la justice et à celles opposées à la virilité. à savoir la rage, la douleur et la peur.

\par 4 Mais, diront certains, si la Raison est maîtresse des passions, pourquoi ne maîtrise-t-elle pas l'oubli et l'ignorance ? leur but étant de jeter le ridicule.

\par 5 La réponse est que la raison n'est pas maîtresse des défauts inhérents à l'esprit lui-même, mais des passions ou des défauts moraux qui sont contraires à la justice, à la virilité, à la tempérance et au jugement ; et son action dans leur cas n'est pas d'extirper les passions, mais de nous permettre de leur résister avec succès.

\par 6 Je pourrais vous présenter de nombreux exemples, tirés de diverses sources, où la raison s'est montrée maîtresse des passions, mais le meilleur exemple de loin que je puisse donner est la noble conduite de ceux qui sont morts pour l'amour de la vertu. Eléazar, les Sept Frères et la Mère.

\par 7 Car tous, par leur mépris des douleurs, même jusqu'à la mort, ont prouvé que la raison s'élève au-dessus des passions.

\par 8 Je pourrais développer ici l'éloge de leurs vertus, eux, les hommes avec la Mère, mourant en ce jour que nous célébrons pour l'amour de la beauté et de la bonté morales, mais je les féliciterais plutôt pour les honneurs qu'ils ont obtenus.

\par 9 Car l'admiration ressentie pour leur courage et leur endurance, non seulement par le monde en général mais par leurs bourreaux eux-mêmes, a fait d'eux les auteurs de la chute de la tyrannie sous laquelle se trouvait notre nation, ils ont vaincu le tyran par leur endurance, de sorte que par eux leur pays fut purifié.

\par 10 Mais je profiterai tout à l'heure de l'occasion pour discuter de cela, après avoir commencé par la théorie générale, comme j'ai l'habitude de le faire, et je passerai ensuite à leur histoire, rendant gloire au Dieu tout sage.

\par 11 Notre recherche est donc de savoir si la Raison est le maître suprême des passions.

\par 12 Mais nous devons définir exactement ce qu'est la Raison et ce qu'est la passion, et combien il y a de formes de passion, et si la Raison est suprême sur toutes.

\par 13 La raison, je considère, est l'esprit préférant, avec une délibération claire, la vie de la sagesse.

\par 14 La sagesse est, selon moi, la connaissance des choses divines et humaines, et de leurs causes.

\par 15 C'est ce que je considère comme la culture acquise sous la Loi, à travers laquelle nous apprenons avec le respect dû les choses de Dieu et, pour notre profit mondain, les choses de l'homme.

\par 16 Or la sagesse se manifeste sous les formes du jugement et de la justice, du courage et de la tempérance.

\par 17 Mais le jugement ou la maîtrise de soi est celui qui les domine tous, car par lui, en vérité, la Raison affirme son autorité sur les passions.

\par 18 Mais parmi les passions, il y a deux sources globales, à savoir le plaisir et la douleur, et l'une ou l'autre appartient essentiellement aussi bien à l'âme qu'au corps.

\par 19 Et en ce qui concerne tant le plaisir que la douleur, il existe de nombreux cas où les passions ont certaines séquences.

\par 20 Ainsi, tandis que le désir précède le plaisir, la satisfaction suit après, et tandis que la peur précède la douleur, après la douleur vient la tristesse.

\par 21 La colère, encore une fois, si l'on veut retracer le cours de ses sentiments, est une passion dans laquelle se mélangent à la fois le plaisir et la douleur.

\par 22 Sous le plaisir vient aussi cet avilissement moral qui présente la plus grande variété de passions.

\par 23 Cela se manifeste dans l'âme comme l'ostentation, la convoitise, la vaine gloire, la querelle et la médisance, et dans le corps comme le fait de manger des viandes étrangères, et la gourmandise, et la gourmandise en secret.

\par 24 Or, le plaisir et la douleur étant comme deux arbres, poussant du corps et de l'âme, de nombreuses ramifications de ces passions poussent ; et la raison de chaque homme, en tant que maître-jardinier, désherbant, taillant et liant, ouvrant l'eau et la dirigeant çà et là, amène le bosquet des dispositions et des passions sous la domestication.

\par 25 Car si la Raison est le guide des vertus, elle est la maîtresse des passions.

\par 26 Observez maintenant, en premier lieu, que la raison devient suprême sur les passions en vertu de l'action inhibitrice de la tempérance.

\par 27 La tempérance, je suppose, est la répression des désirs ; mais parmi les désirs, certains sont mentaux et d'autres physiques, et les deux sortes sont clairement contrôlées par la Raison ; lorsque nous sommes tentés par les viandes interdites, comment pouvons-nous renoncer aux plaisirs qui en découlent ?

\par 28 N'est-ce pas que la Raison a le pouvoir de réprimer les appétits ? À mon avis, c'est le cas.

\par 29 C'est pourquoi, lorsque nous éprouvons le désir de manger des animaux aquatiques, des oiseaux, des bêtes et des viandes de toute sorte qui nous sont interdites par la loi, nous nous abstenons par la prédominance de la raison.

\par 30 Car les propensions de nos appétits sont freinées et inhibées par l'esprit tempéré, et tous les mouvements du corps obéissent à la bride de la Raison.

\par 31 Et de quoi s'étonner si le désir naturel de l'âme de jouir du fruit de la beauté est éteint ?

\par 32 C'est certainement pour cela que nous louons le vertueux Joseph, car par sa Raison, avec un effort mental, il a freiné l'impulsion charnelle. 1 Car lui, jeune homme à l'âge où le désir physique est fort, a éteint par sa Raison l'impulsion de ses passions.

\par 33 Et il est prouvé que la raison maîtrise non seulement l'impulsion du désir sexuel, mais celle de toutes sortes de convoitises.

\par 34 Car la loi dit : Tu ne convoiteras pas la femme de ton prochain, ni rien de ce qui appartient à ton prochain.

\par 35 En vérité, lorsque la Loi nous ordonne de ne pas convoiter, elle devrait, je pense, confirmer fortement l'argument selon lequel la Raison est capable de contrôler les désirs avares, tout comme elle le fait pour les passions qui militent contre la justice.

\par 36 Comment autrement un homme, naturellement gourmand, avide et ivre, pourrait-il apprendre à changer de nature, si la Raison n'est pas manifestement la maîtresse des passions ?

\par 37 Certes, dès qu'un homme ordonne sa vie selon la loi, s'il est avare, il agit contrairement à sa nature, et prête de l'argent aux nécessiteux sans intérêt, et à la septième année, il annule la dette.

\par 38 Et s'il est parcimonieux, il se laisse vaincre par la Loi par l'action de la Raison, et s'abstient de glaner ses chaumes ou de cueillir les derniers raisins de ses vignes.

\par 39 Et pour tout le reste on peut reconnaître que la Raison est en position de maître sur les passions ou affections.

\par 40 Car la loi est au-dessus de l'affection des parents, afin qu'un homme ne puisse pas renoncer à sa vertu à cause d'eux, et elle prime sur l'amour pour une femme, de sorte qu'elle transgresse un homme qui doit la réprimander, et elle régit l'amour pour les enfants, de sorte que s'ils sont méchants, un homme doit les punir, et cela contrôle les droits de l'amitié, de sorte qu'un homme doit réprimander ses amis s'ils font le mal.

\par 41 Et ne pensez pas qu'il soit paradoxal que la raison, par la loi, soit capable de vaincre même la haine, de sorte qu'un homme s'abstienne de couper les vergers de l'ennemi, protège les biens de l'ennemi des pilleurs et rassemble ses biens. biens dispersés.

\par 42 Et il est également prouvé que le règne de la raison s'étend aux passions ou vices les plus agressifs, à l'ambition, à la vanité, à l'ostentation, à l'orgueil et à la médisance.

\par 43 Car l'esprit tempéré repousse toutes ces passions avilies, comme il repousse la colère, car il vainc même celle-là.

\par 44 Oui, Moïse, lorsqu'il était en colère contre Dathan et Abiram, n'a pas donné libre cours à sa colère, mais a gouverné sa colère par sa raison.

\par 45 Car l'esprit tempéré est capable, comme je l'ai dit, de remporter la victoire sur les passions, en modifiant les unes, en écrasant absolument les autres.

\par 46 Sinon, pourquoi notre sage père Jacob a-t-il blâmé les maisons de Siméon et de Lévi pour le massacre irraisonné de la tribu des Sichémites, en disant : « Maudite soit leur colère !

\par 47 Car si la Raison n'avait pas eu le pouvoir de retenir leur colère, il n'aurait pas parlé ainsi.

\par 48 Car le jour où Dieu créa l'homme, il implanta en lui ses passions et ses inclinations, et aussi, en même temps, plaça l'esprit sur un trône au milieu des sens pour être son guide sacré en toutes choses ; et à l'esprit il a donné la loi, par laquelle si un homme se commande, il régnera sur un royaume tempéré, juste, vertueux et courageux.

\par \textit{Notes de bas de page}

\par \textit{179:1 Voir Le Testament de Joseph, page 260.}

\chapter{2}

\par \textit{Le règne du Désir et de la Colère. L'histoire de la soif de David. Des chapitres émouvants de l’histoire ancienne. Tentatives sauvages pour faire manger du porc aux Juifs. Références intéressantes à une ancienne banque (Verset 21.)}

\par 1 BIEN alors, quelqu'un pourra se demander, si la Raison est maîtresse des passions pourquoi n'est-elle pas maîtresse de l'oubli et de l'ignorance ?

\par 2 Mais l'argument est suprêmement ridicule. Car la raison ne se montre pas maîtresse des passions ou des défauts en elle-même, mais de celles du corps.

\par 3 Par exemple, aucun de vous n'est capable d'extirper notre désir naturel, mais la Raison peut lui permettre d'échapper à l'esclavage du désir.

\par 4 Aucun de vous n'est capable d'extirper la colère de l'âme, mais il est possible que la Raison lui vienne en aide contre la colère.

\par 5 Aucun d'entre vous ne peut extirper une disposition malveillante, mais la Raison peut être son puissant allié pour ne pas se laisser influencer par la malveillance.

\par 6 La raison n'est pas l'extirpée des passions, mais leur antagoniste.

\par 7 Le cas de la soif du roi David peut servir au moins à rendre cela plus clair.

\par 8 Car après que David eut combattu toute la journée contre les Philistins, et qu'avec l'aide des guerriers de notre pays il en eut tué beaucoup, il arriva au soir, tout pardonné de sueur et de labeur, à la tente royale, autour de laquelle était campée toute l'armée de nos ancêtres.

\par 9 Ainsi toute l'armée se mit à manger du soir ; mais le roi, rongé par une soif intense, bien qu'il ait beaucoup d'eau, ne put l'apaiser.

\par 10 Au lieu de cela, un désir irrationnel pour l'eau qui était en possession de l'ennemi avec une intensité croissante l'a brûlé et sans équipage et l'a consumé.

\par 11 Alors, comme ses gardes du corps murmuraient contre l'avidité du roi, deux jeunes gens, de puissants guerriers, honteux que leur roi ne satisfasse pas à ses désirs, revêtirent toutes leurs armures, prirent un vase d'eau et escaladèrent les remparts de l'ennemi. ; et, passant inaperçus devant les gardes à la porte, ils fouillèrent tout le camp ennemi.

\par 12 Et ils trouvèrent courageusement la source, et en tirèrent une potion pour le roi.

\par 13 Mais David, bien que brûlant encore de soif, considérait qu'un tel breuvage, considéré comme équivalent au sang, représentait un grave danger pour son âme.

\par 14 C'est pourquoi, opposant sa raison à son désir, il versa l'eau en offrande à Dieu.

\par 15 Car l'esprit tempéré est capable de vaincre les préceptes des passions, d'éteindre les feux du désir, et de lutter victorieusement contre les douleurs de nos corps, bien qu'elles soient extrêmement fortes, et par la beauté morale et la bonté de la Raison. défier avec mépris toute la domination des passions.

\par 16 Et maintenant l'occasion nous appelle à exposer l'histoire de la Raison auto-contrôlée.

\par 17 À une époque où nos pères jouissaient d'une grande paix grâce à l'observance de la loi, et étaient dans une situation heureuse, de sorte que Séleucus Nicanor, roi d'Asie, autorisait l'impôt pour le service du temple et reconnaissait notre gouvernement, précisément alors, certains hommes, agissant de manière factieuse contre la concorde générale, nous ont entraînés dans des calamités nombreuses et diverses.

\par 18 Onias, homme de la plus haute considération, étant alors grand prêtre et ayant l'office pour sa vie, un certain Simon souleva une faction contre lui, mais comme, malgré toutes sortes de calomnies, il ne parvint pas à lui nuire à cause du peuple. , il s'enfuit à l'étranger avec l'intention de trahir son pays.

\par 19 Alors il vint trouver Apollonius, gouverneur de Syrie, de Phénicie et de Cilicie, et lui dit : « Étant fidèle au roi, je suis ici pour vous informer que dans les trésors de Jérusalem sont stockés plusieurs milliers de dépôts privés, n'appartenant pas à au compte du temple, et légitimement la propriété du roi Séleucus.

\par 20 Apollonius s'étant renseigné sur les détails de l'affaire, loua Simon pour ses loyaux services envers le roi, et se précipitant à la cour de Séleucus, lui révéla le précieux trésor ; puis, après avoir reçu l'autorité de s'occuper de l'affaire, il entra promptement dans notre pays, accompagné du maudit Simon et d'une armée très puissante, et annonça qu'il était là par ordre du roi pour prendre possession des dépôts privés du trésor.

\par 21 Notre peuple fut profondément irrité par cette annonce et protesta vivement, considérant qu'il s'agissait d'une chose scandaleuse pour ceux qui avaient confié leurs dépôts au trésor du temple de s'en faire voler, et ils jetèrent tous les obstacles possibles sur son chemin.

\par 22 Mais Apollonius, menaçant, entra dans le temple.

\par 23 Alors les prêtres dans le temple, les femmes et les enfants supplièrent Dieu de venir au secours de son Lieu Saint qui était violé ; Et quand Apollonius et son armée armée entraient pour saisir l'argent, des anges apparurent du ciel, montés sur des chevaux, avec des éclairs jaillissant de leurs bras, et jetèrent sur eux une grande peur et un grand tremblement.

\par 24 Et Apollonius tomba à moitié mort dans la cour des Gentils, et étendit les mains vers le ciel, et avec des larmes il supplia les Hébreux d'intercéder pour lui et de retenir la colère de l'armée céleste.

\par 25 Car il disait qu'il avait péché et qu'il méritait même la mort, et que si on lui donnait la vie, il louerait à tous les hommes la bénédiction du Lieu Saint.

\par 26 Ému par ces paroles, le grand prêtre Onias, bien que très scrupuleux dans d'autres cas, intercéda pour lui afin que le roi Séleucus ne pense pas qu'Apollonios avait été renversé par une invention humaine et non par la justice divine.

\par 27 Apollonius partit donc, après son étonnante délivrance, pour rapporter au roi ce qui lui était arrivé.

\par 28 Mais Séleucus mourant, son successeur sur le trône fut son fils Antiochus Épiphane, un homme extrêmement terrible ; qui renvoya Onias de son office sacré et nomma son frère Jason grand prêtre à la place, à la condition qu'en échange de cette nomination, Jason lui paierait trois mille six cent soixante talents par an.

\par 29 Il établit donc Jason comme grand prêtre et il l'établit chef du peuple.

\par 30 Et il (Jason) a introduit à notre peuple un nouveau mode de vie et une nouvelle constitution au mépris total de la Loi ; de sorte que non seulement il aménagea un gymnase sur le mont de nos pères, mais qu'il abolit même le service du temple.

\par 31 C'est pourquoi la justice divine s'est enflammée de colère et a amené Antiochus lui-même comme notre ennemi.

\par 32 Pour quand. Alors qu'il faisait la guerre à Ptolémée en Égypte et qu'il apprit que les habitants de Jérusalem s'étaient extrêmement réjouis de l'annonce de sa mort, il se retira immédiatement contre eux.

\par 33 Et après avoir pillé la ville, il publia un décret dénonçant la peine de mort pour quiconque serait vu vivre selon la loi de nos pères.

\par 34 Mais il a trouvé tous ses décrets inutiles pour briser la fidélité de notre peuple à la loi, et il a vu toutes ses menaces et punitions complètement méprisées, de sorte que même les femmes qui circoncissaient leurs fils, bien qu'elles savaient d'avance ce qui se passerait, quel que soit leur sort, ont été jetés, avec leur progéniture, tête baissée des rochers.

\par 35 Alors que ses décrets continuaient à être méprisés par la masse du peuple, il essaya personnellement de forcer par des tortures chaque homme séparément à manger des viandes impures et ainsi à abjurer la religion juive.

\par 36 En conséquence, le tyran Antiochus, accompagné de ses conseillers, siégeait pour juger sur un certain haut lieu, avec ses troupes rangées autour de lui en armure complète, et il ordonna à ses gardes d'y traîner tous les Hébreux et de les contraindre. manger de la chair de porc et des choses offertes aux idoles ; mais si quelqu'un refusait de se souiller avec des choses impures, il devait être torturé et mis à mort.

\par 37 Et après que beaucoup eurent été emmenés de force, un premier homme du groupe fut amené devant Antiochus, un Hébreu nommé Eléazar, prêtre de naissance, instruit dans la connaissance de la loi, homme avancé en âge et en bonne santé. connu de nombreux membres de la cour du tyran pour sa philosophie.

\par 38 Et Antiochus, le regardant, dit : Avant de permettre que les tourments commencent pour toi, ô vénérable homme, je voudrais te donner ce conseil, que tu manges de la chair de porc et que tu sauves ta vie ; car je respecte votre âge et vos cheveux gris, même si les avoir portés si longtemps et rester attaché à la religion juive me fait penser que vous n'êtes pas un philosophe.

\par 39 « Car la viande de cet animal que la nature nous a gracieusement accordée est excellente, et pourquoi l'aboriez-vous ? En vérité, c'est une folie de ne pas jouir de plaisirs innocents, et c'est une erreur de rejeter les faveurs de la nature.

\par 40 «Mais ce serait encore plus folie, je pense, de votre part si, en vapotant vainement sur la vérité, vous vous mettiez même à me défier à votre propre châtiment.»

\par 41 'Ne vous réveillerez-vous pas de votre philosophie absurde ? Ne mettrez-vous pas de côté les absurdités de vos calculs et, adoptant un autre état d'esprit qui convient à vos années de maturité, n'apprendrez-vous pas la véritable philosophie de l'opportunité, comment suivre mes conseils charitables, et n'aurez-vous pas pitié de votre propre âge vénérable ?

\par 42 «Car considérez ceci aussi, que même s'il y a quelque Puissance dont les yeux sont sur votre religion, il vous pardonnera toujours pour une transgression commise sous la contrainte.»

\par 43 Comme poussé par le tyran à manger illégalement de la viande impure, Éléazar demanda la permission de parler ; et en le recevant, il commença son discours devant le tribunal ainsi :

\par 44 «Nous, ô Antiochus, ayant accepté la loi divine comme loi de notre pays, ne croyons pas qu'une nécessité plus forte nous soit imposée que celle de notre obéissance à la loi.»

\par 45 'C'est pourquoi nous estimons sûrement que cela n'est pas vrai. de quelque manière que ce soit pour transgresser la Loi.

\par 46 «Et pourtant, si notre Loi, comme vous le suggérez, n'était pas vraiment divine, alors que nous croyions en vain qu'elle était divine, il ne serait pas non plus juste que nous détruisions notre réputation de piété.»

\par 47 Ne pensez donc pas que le fait de manger une chose impure soit un petit péché, car la transgression de la loi, que ce soit dans les petites choses ou dans les grandes, est également odieuse ; car dans les deux cas également, la loi est méprisée.

\par 48 «Et vous vous moquez de notre philosophie, comme si nous vivions sous elle d'une manière contraire à la raison.»

\par 49 «Ce n'est pas le cas, car la Loi nous enseigne la maîtrise de soi, de sorte que nous sommes maîtres de tous nos plaisirs et désirs et que nous soyons parfaitement formés à la virilité afin d'endurer toute douleur avec empressement ; et il enseigne la justice, de sorte qu'avec toutes nos diverses dispositions, nous agissions équitablement, et il enseigne la justice, de sorte que, avec le respect qui lui est dû, nous adorons uniquement le Dieu qui est.

\par 50 « C'est pourquoi nous ne mangeons pas de viande impure ; car croyant que notre Loi est donnée par Dieu, nous savons aussi que le Créateur du monde, en tant que Législateur, ressent pour nous selon notre nature.

\par 51 'Il nous a ordonné de manger les choses qui conviennent à nos âmes, et il nous a interdit de manger des viandes qui seraient le contraire.'

\par 52 «Mais c'est l'acte d'un tyran que vous nous obligeiez non seulement à transgresser la loi, mais aussi à nous faire manger de telle manière que vous puissiez vous moquer de cette souillure qui nous est si abominable.»

\par 53 «Mais vous ne vous moquerez pas de moi ainsi, et je ne romprai pas les serments sacrés de mes ancêtres d'observer la Loi, même si vous m'arrachez les yeux et brûlez mes entrailles.»

\par 54 «Je ne suis pas si dépourvu d'équipage par la vieillesse que lorsque la justice est en jeu, la force de la jeunesse revient à ma Raison.»

\par 55 'Alors tordez fort vos grilles et faites chauffer votre four plus fort. Je n'ai pas assez pitié de ma vieillesse pour enfreindre la loi de mes pères sur ma propre personne.

\par 56 'Je ne te mentirai pas, ô Loi qui fut mon maître ; Je ne t'abandonnerai pas, ô bien-aimée maîtrise de soi ; Je ne te ferai pas honte, ô Raison aimant la sagesse, et je ne te renierai pas, ô vénéré sacerdoce et connaissance de la Loi.

\par 57 Tu ne souilleras pas non plus la bouche pure de ma vieillesse et ma constance à la loi toute ma vie. Mes pères me recevront purs, sans craindre tes tourments jusqu'à la mort.

\par 58 'Car tu peux certes être tyran sur les hommes injustes, mais tu ne domineras pas ma résolution en matière de justice, ni par tes paroles ni par tes actes.'

\chapter{3}

\par \textit{Éléazar, le vieil homme à l'esprit doux, fait preuve d'un tel courage que même lorsque nous lisons ces mots 2000 ans plus tard, ils semblent comme un feu inextinguible.}

\par 1 MAIS quand Éléazar répondit ainsi avec éloquence aux exhortations des tyrans, les gardes autour de lui le traînèrent brutalement jusqu'au lieu de torture.

\par 2 Et d'abord ils dévêtirent le vieil homme, qui était paré de la beauté de la sainteté.

\par 3 Alors, lui liant les bras de chaque côté, ils le flagellèrent, un héraut se tenant debout et criant contre lui : « Obéis aux ordres du roi !

\par 4 Mais l'homme noble et noble, un Éléazar en vérité, n'était pas plus ému dans son esprit que s'il était tourmenté dans un rêve ; oui, le vieil homme, gardant les yeux fermement levés vers le ciel, laissa sa chair être déchirée par les fléaux jusqu'à ce qu'il soit baigné de sang et que ses côtés deviennent une masse de blessures ; et même lorsqu'il tombait à terre parce que son corps ne pouvait plus supporter la douleur, il gardait toujours sa Raison droite et inflexible.

\par 5 Alors, avec son pied, un des gardes de la burette, alors qu'il tombait, lui donna sauvagement un coup de pied dans le côté pour le faire se relever.

\par 6 Mais il a enduré l'angoisse, et a méprisé la contrainte, et a supporté les tourments, et comme un brave athlète subissant le châtiment, le vieil homme a surpassé ses bourreaux.

\par 7 La sueur lui coulait au front, et il respirait à grands coups de souffle, jusqu'à ce que sa noblesse d'âme lui arrache l'admiration de ses bourreaux eux-mêmes.

\par 8 Alors, en partie par pitié pour sa vieillesse, en partie par sympathie pour leur ami, en partie par admiration pour son courage, quelques courtisans du roi allèrent vers lui et lui dirent :

\par 9 'Pourquoi, ô Éléazar, te détruis-tu follement dans cette misère ? Nous t'apporterons des viandes cuites, mais fais semblant de manger seulement de la chair de porc et sauve-toi ainsi.

\par 10 Et Éléazar, comme si leur conseil ne faisait qu'ajouter à ses tortures, s'écria d'une voix forte : « Non. Puissions-nous, fils d'Abraham, n'avoir jamais la pensée si mauvaise que de contrefaire, avec un cœur faible, une partie qui ne nous convient pas.

\par 11 « Contrairement à la raison, en effet, si c'était pour nous, après avoir vécu selon la vérité jusqu'à un âge avancé, et gardé sous une forme légitime la réputation de ceux qui vivent ainsi, de changer maintenant et de devenir en nous-mêmes un modèle pour les jeunes de impiété, afin que nous les incitions à manger de la viande impure.

\par 12 «Il serait dommage que nous vivions un peu plus longtemps, pendant que nous nous moquions de tous les hommes pour leur lâcheté, et que pendant que nous sommes méprisés par le tyran comme étant peu virils, nous ne parvenions pas à défendre la loi divine jusqu'à la mort.»

\par 13 'C'est pourquoi, ô fils d'Abraham, mourez noblement à cause de la justice' ; mais quant à vous, ô serviteurs du tyran, pourquoi vous arrêtez-vous dans votre travail ?

\par 14 Alors eux, le voyant ainsi triomphant des tortures et insensible même à la pitié de ses bourreaux, le traînèrent au feu.

\par 15 Là, ils le jetèrent dessus, le brûlèrent avec des procédés cruels et astucieux, et ils lui versèrent un bouillon de mauvaise odeur dans ses narines.

\par 16 Mais alors que le feu atteignait déjà ses os et qu'il était sur le point de rendre l'âme, il leva les yeux vers Dieu et dit :

\par 17 Tu sais, ô Dieu, que même si je peux me sauver, je meurs dans des tourments enflammés à cause de ta loi. Sois miséricordieux envers ton peuple, et que notre châtiment soit une satisfaction en sa faveur. Faites de mon sang leur purification, et prenez mon âme pour racheter leurs âmes,'

\par 18 «Et avec ces paroles, le saint homme a noblement rendu son esprit sous la torture et pour l'amour de la loi tenue par sa raison même contre les tourments jusqu'à la mort.»

\par 19 Sans aucun doute donc, la Raison Inspirée est maîtresse des passions ; car si ses passions ou ses souffrances avaient prévalu sur sa raison, nous leur aurions attribué cette preuve de leur puissance supérieure.

\par 20 Mais maintenant que sa Raison a vaincu ses passions, on lui attribue à juste titre le pouvoir de les commander.

\par 21 Et il est juste que nous admettions que la maîtrise appartient à la Raison, au moins dans les cas où elle vainc les douleurs qui viennent du dehors ; car il serait ridicule de le nier.

\par 22 Et ma preuve couvre non seulement la supériorité de la raison sur les peines, mais aussi sa supériorité sur les plaisirs ; il ne leur cède pas non plus.

\chapter{4}

\par \textit{Ce soi-disant « Âge de la Raison » peut indiquer dans ce chapitre que la Philosophie de la Raison a 2000 ans. L'histoire de sept fils et de leur mère.}

\par 1 CAR la Raison de notre père Éléazar, comme un bon timonier dirigeant le navire de la sainteté sur la mer des passions, quoique secoué par les menaces du tyran et balayé par les vagues gonflées des tourments, n'a jamais bougé un seul instant. le gouvernail de la sainteté jusqu'à ce qu'il navigue dans le port de la victoire sur la mort.

\par 2 Aucune ville assiégée par de nombreux et astucieux engins ne s'est jamais défendue aussi bien que ce saint homme lorsque son âme sacrée a été attaquée par le fléau, le fléau et les flammes, et il a déplacé ceux qui assiégeaient son âme par sa Raison qui était le bouclier de la sainteté.

\par 3 Car notre père Eléazar, plaçant son film mental comme une falaise rocheuse, a brisé l'apparition folle des élans des passions.

\par 4 Ô prêtre digne de ton sacerdoce, tu n'as pas souillé tes dents saintes, et tu n'as pas non plus souillé par des viandes impures ton ventre qui n'avait de place que pour la piété et la pureté.

\par 5 Ô confesseur de la Loi et philosophe de la vie divine ! Tels devraient être ceux dont la fonction est de servir la Loi et de la défendre avec leur propre sang et leur sueur honorable face aux souffrances jusqu'à la mort.

\par 6 Toi, ô père, tu as fortifié notre fidélité à la loi par ta fermeté pour la gloire ; et après avoir parlé en l'honneur de la sainteté, tu n'as pas démenti ton discours, et tu as confirmé les paroles de la philosophie divine par tes actes, ô vieillard qui étais plus puissant que les tortures.

\par 7 Ô révérend ancien, qui étais plus tendu que la flamme, toi, grand roi des passions, Éléazar.

\par 8 Car, comme notre père Aaron, armé de l'encensoir, courut à travers la congrégation massive contre l'ange de feu et le vainquit, ainsi le fils d'Aaron, Éléazar, consumé par la chaleur fondante du feu, resta inébranlable dans sa raison. .

\par 9 Et pourtant, le plus merveilleux de tout, lui, étant un vieil homme, avec les tendons de son corps détendus et ses muscles détendus et ses nerfs affaiblis, est redevenu un jeune homme dans l'esprit de sa Raison et avec une Raison semblable à celle d'Isaac. a transformé la torture à tête d'hydre en impuissance.

\par 10 Ô âge béni, ô révérend tête grise, ô vie fidèle à la Loi et perfectionnée par le sceau de la mort !

\par 11 Assurément donc, si un vieil homme a méprisé les tourments jusqu'à la mort pour l'amour de la justice, il faut admettre que la raison inspirée est capable de guider les passions.

\par 12 Mais certains répondront peut-être que tous les hommes ne sont pas maîtres des passions parce que tous les hommes n'ont pas leur Raison éclairée.

\par 13 Mais tous ceux qui de tout leur cœur font de la justice leur première pensée, ceux-là seuls sont capables de maîtriser la faiblesse de la chair, croyant qu'à Dieu ils ne meurent pas, comme nos patriarches, Abraham, Isaac et Jacob, ne sont pas morts, mais qu'ils vivent pour Dieu.

\par 14 Il n'y a donc rien de contradictoire à ce que certaines personnes paraissent esclaves de la passion par suite de la faiblesse de leur raison.

\par 15 Car qui est-ce qu'étant un philosophe suivant justement toute la règle de la philosophie, et ayant mis sa confiance en Dieu, et sachant que c'est une chose bénie d'endurer toutes les duretés pour l'amour de la vertu, ne vaincrait pas ses passions ? pour le bien de la justice ?

\par 16 Car seul l'homme sage et maître de lui-même est le vaillant maître des passions.

\par 17 Oui, par ce moyen même les jeunes garçons, étant philosophes en vertu de la raison qui est selon la justice, ont triomphé de tortures encore plus graves.

\par 18 Car lorsque le tyran se trouva notablement vaincu dans sa première tentative, et incapable de contraindre un vieil homme à manger de la viande impure, alors véritablement dans une violente rage, il ordonna aux gardes d'amener d'autres jeunes gens des Hébreux, et si ils mangeaient de la viande impure pour les libérer après l'avoir mangée, mais s'ils refusaient, ils les torturaient encore plus sauvagement.

\par 19 Et sous ces ordres du tyran, sept frères et leur vieille mère furent amenés prisonniers devant lui, tous beaux, modestes, bien nés, et généralement attirants.

\par 20 Et quand le tyran les vit là, debout comme s'ils formaient un chœur de fête avec leur mère au milieu, il les remarqua et, frappé par leur noble et distinguée allure, il leur sourit et les appelant plus près, dit :

\par 21 'Ô jeunes hommes, je souhaite bonne chance à chacun de vous, j'admire votre beauté et j'honore hautement un si grand groupe de frères ; ainsi, non seulement je vous conseille de ne pas persister dans la folie de ce vieillard qui a déjà souffert, mais je vous supplie même de vous céder à moi et de devenir participants de mon amitié.

\par 22 'Car, de même que je peux punir ceux qui désobéissent à mes ordres, ainsi puis-je faire progresser ceux qui m'obéissent.'

\par 23 'Soyez alors assuré que vous obtiendrez des positions importantes et d'autorité à mon service si vous rejetez la loi ancestrale de votre régime politique.'

\par 24 « Partagez la vie hellénique, marchez d'une manière nouvelle, et prenez du plaisir dans votre jeunesse ; car si vous me mettez en colère par votre désobéissance, vous m'obligerez à recourir à des peines terribles et à mettre chacun d'entre vous à mort par la torture.

\par 25 «Ayez donc pitié de vous-mêmes, que moi aussi, votre adversaire, j'ai pitié de votre jeunesse et de votre beauté.»

\par 26 'Ne considérerez-vous pas en vous-mêmes que si vous me désobéissez, il n'y a rien d'autre devant vous que la mort dans les tourments ?'

\par 27 Avec ces paroles, il ordonna d'apporter les instruments de torture afin de les persuader par la peur de manger de la viande impure.

\par 28 Mais lorsque les gardes eurent produit des roues, et des dislocateurs d'articulations, et des crémaillères, et des broyeurs d'os, et des catapultes, et des chaudrons, et des braseros, et des vis à pouce, et des griffes de fer, et des coins, et des fers à marquer, les Le tyran reprit la parole et dit :

\par 29 «Vous feriez mieux d'avoir peur, mes enfants, et la justice que vous adorez vous pardonnera votre transgression involontaire.»

\par 30 Mais eux, entendant ses convictions et voyant ses terribles engins, non seulement ne montrèrent aucune crainte, mais opposèrent même leur philosophie au tyran, et, par leur raison légitime, avilirent sa tyrannie.

\par 31 Et pourtant, réfléchissez : en supposant que certains d’entre eux aient été timides et lâches, quel genre de langage auraient-ils utilisé ? cela n'aurait-il pas été le cas ?

\par 32 « Hélas ! misérables créatures que nous sommes et insensées au-delà de toute mesure ! Lorsque le roi nous invite et nous demande de nous traiter avec bienveillance, ne lui obéirons-nous pas ?

\par 33 'Pourquoi nous encourageons-nous par de vains désirs et osons-nous une désobéissance qui doit nous coûter la vie ? Ne devrions-nous pas, ô hommes, mes frères, craindre les instruments redoutables et peser bien ses menaces de tortures, et abandonner ces vaines vantardises et cette vantardise fatale ?

\par 34 « Ayons pitié de notre propre jeunesse et ayons compassion de l'âge de notre mère ; et prenons à cœur que si nous désobéissons, nous mourrons.

\par 35 « Et même la justice divine aura pitié de nous, si nous y sommes contraints par la nécessité, nous nous soumettons au roi avec crainte. Pourquoi devrions-nous rejeter cette chère vie et nous priver de ce doux monde ?

\par 36 «Ne luttons pas contre la nécessité et n'invitons pas avec une vaine confiance notre torture.»

\par 37 'Même la loi elle-même ne nous condamne pas volontairement à la mort, nous avons peur des instruments de torture.'

\par 38 'Pourquoi une telle querelle nous enflamme-t-elle et une obstination fatale trouve-t-elle grâce auprès de nous, alors que nous pourrions avoir une vie paisible en obéissant au roi ?'

\par 39 Mais de tels mots n'ont pas échappé à ces jeunes hommes à la perspective du supplice, et de telles pensées ne sont pas non plus entrées dans leur esprit.

\par 40 Car ils méprisaient les passions et étaient maîtres de la douleur.

\chapter{5}

\par \textit{Un chapitre d'horreur et de torture révélant l'ancienne tyrannie dans sa plus grande sauvagerie. Le verset 26 est une vérité profonde.}

\par 1 ET ainsi, à peine le tyran eut-il terminé de les inciter à manger de la viande impure, que tous d'une seule voix et comme d'une seule âme, lui dirent :

\par 2 'Pourquoi tardes-tu, ô tyran ? Nous sommes prêts à mourir plutôt que de transgresser les commandements de nos pères.

\par 3 'Car nous ferions aussi honte à nos ancêtres, si nous ne marchions pas dans l'obéissance à la Loi et si nous ne prenions pas Moïse pour conseiller.'

\par 4 'Ô tyran qui nous conseille de transgresser la Loi, ne nous déteste pas, ne nous plains pas au-delà de nous-mêmes.'

\par 5 'Car nous estimons ta miséricorde, ton don. nous notre vie en échange d'une violation de la Loi, une chose plus difficile à supporter que la mort elle-même.

\par 6 «Tu nous terrifierais avec tes menaces de mort sous la torture, comme si tout à l'heure tu n'avais rien appris d'Éléazar.»

\par 7 «Mais si les vieillards des Hébreux ont enduré les tourments pour l'amour de la justice, oui, jusqu'à ce qu'ils meurent, nous, les jeunes hommes, mourrons plus convenablement, en méprisant les tourments de ta contrainte, dont lui, notre vieux professeur, a également triomphé.»

\par 8 « Fais donc l'épreuve, ô tyran. Et si tu nous donnes la vie pour la justice, ne pense pas que tu nous fasses du mal par tes tortures.

\par 9 'Car, grâce à cela, notre mauvais traitement et notre endurance à celui-ci remporteront le prix de la vertu ; mais toi, pour notre meurtre cruel, tu souffriras de la part de la justice divine un tourment suffisant par le feu pour toujours.

\par 10 Ces paroles des jeunes redoublèrent la colère du tyran, non seulement contre leur désobéissance, mais contre ce qu'il considérait comme leur ingratitude.

\par 11 Ainsi, sur ses ordres, les fouetteurs firent avancer l'aîné d'entre eux, le dépouillèrent de ses vêtements et lui lièrent les mains et les bras de chaque côté avec des lanières.

\par 12 Mais après l'avoir fouetté jusqu'à ce qu'ils en soient fatigués, et n'y gagnant rien, ils le jetèrent sur la roue.

\par 13 Et là-dessus, le noble jeune homme fut torturé jusqu'à ce que ses os soient brisés. Et tandis que les joints cédaient, il dénonça le tyran en ces termes :

\par 14 'Ô toi, tyran le plus abominable, ennemi de la justice du ciel et esprit sanglant, tu me tourmentes de cette façon, non pas pour meurtre ni pour impiété, mais pour avoir défendu la loi de Dieu.'

\par 15 Et quand les gardes lui dirent : « Consentez à manger, afin que vous soyez délivrés de vos tourments », il leur dit : « Votre méthode, ô misérables serviteurs, n'est pas assez forte pour conduire captive ma Raison. Coupez-moi les membres, brûlez ma chair et tordez mes articulations ; à travers tous les tourments, je vous montrerai qu'en faveur de la vertu, seuls les fils des Hébreux sont invincibles.

\par 16 Pendant qu'il parlait ainsi, ils lui mirent en outre des charbons ardents, et intensifiant la torture, ils le tendirent encore plus fort sur la roue.

\par 17 Et toute la roue était tachée de son sang, et les charbons accumulés étaient éteints par les humeurs de son corps tombant, et la chair déchirée coulait autour des essieux de la machine.

\par 18 Et avec son corps déjà en dissolution, ce jeune à la grande âme, comme un vrai fils d'Abraham, ne gémissait pas du tout ; mais comme s'il souffrait d'une transformation par le feu jusqu'à l'incorruption, il endura noblement le tourment, en disant :

\par 19 « Suivez mon exemple, ô frères. Ne m'abandonnez pas pour toujours et ne renoncez pas à notre fraternité dans la noblesse d'âme.

\par 20 «La guerre est une guerre sainte et honorable au nom de la justice, par laquelle la juste Providence qui a veillé sur nos pères devient miséricordieuse envers son peuple et se venge du tyran maudit.»

\par 21 Et avec ces paroles, le saint jeune homme rendit l'âme.

\par 22 Mais pendant que tous s'étonnaient de sa constance d'âme, les gardes avancèrent le deuxième âge du. fils, et le saisissant avec des mains de fer aux griffes acérées, ils l'attaquèrent aux moteurs et à la catapulte.

\par 23 Mais lorsqu'ils entendirent sa noble résolution en réponse à leur question : « Mangerait-il plutôt que de torturer ? ces bêtes ressemblant à des panthères lui déchiraient les tendons avec des griffes de fer, arrachaient toute la chair de ses joues et lui arrachaient la peau de la tête.

\par 24 Mais lui, endurant fermement cette agonie, dit : « Comme toute forme de mort est douce à cause de la justice de nos pères !

\par 25 Et il dit au tyran : « Ô le plus impitoyable des tyrans, ne te semble-t-il pas qu'en ce moment tu souffres toi-même des tortures pires que les miennes en voyant les intentions arrogantes de ta tyrannie vaincues par mon endurance pour l'amour de la justice ?

\par 26 « Car je suis soutenu dans la douleur par les joies qui viennent de la vertu, tandis que tu es dans le tourment en te glorifiant de ton impiété ; tu n'échapperas pas non plus, ô tyran très abominable, aux châtiments de la colère divine.

\par 27 Et quand il eut courageusement rencontré sa mort glorieuse, le troisième fils fut amené et beaucoup le supplièrent instamment de goûter et ainsi de se sauver.

\par 28 Mais il répondit d'une voix forte : « Ignorez-vous que le même père m'a engendré, moi et mes frères morts, et que la même mère nous a donné naissance, et que j'ai été élevé dans les mêmes doctrines ?

\par 29 'Je ne renonce pas au noble lien de fraternité.'

\par 30 « Par conséquent, si vous avez un instrument de tourment, appliquez-le à mon corps ; car vous ne pouvez pas atteindre mon âme, même si vous le vouliez.

\par 31 Mais ils furent très irrités par le discours hardi de cet homme, et ils lui déboîtèrent les mains et les pieds avec leurs moteurs à luxation, et lui arrachèrent les membres de leurs orbites, et les dénouèrent ; et ils s'enroulèrent autour de ses doigts, et de ses bras, et de ses jambes, et de ses coudes.

\par 32 Et, ne pouvant en aucun cas étrangler son esprit, ils lui arrachèrent la peau, emportant avec elle la pointe des doigts, arrachèrent à la manière scythe le cuir chevelu de sa tête, et l'emmenèrent aussitôt au volant.

\par 33 Et là-dessus, ils lui tordirent la colonne vertébrale, jusqu'à ce qu'il voie sa propre chair pendue en lanières et de grandes gouttes de sang coulant de ses entrailles.

\par 34 Et sur le point de mourir, il dit : « Nous, ô tyran très abominable, souffrons ainsi pour notre éducation et notre vertu qui sont de Dieu ; mais toi, à cause de ton impiété et de ta cruauté, tu endureras des tourments sans fin.

\par 35 Et quand cet homme fut mort dignement de ses frères, ils amenèrent le quatrième et lui dirent : Ne sois pas fou de la même folie que tes frères, mais obéis au roi et sauve-toi.

\par 36 Mais il leur dit : Vous n'avez pas pour moi de feu si ardent qu'il fasse de moi un lâche.

\par 37 'Par la mort bénie de mes frères, par le destin éternel du tyran et par la vie glorieuse des justes, je ne renierai pas ma noble fraternité.'

\par 38 «Invente des tortures, ô tyran, afin que tu saches par là que je suis le frère de ceux qui ont déjà été torturés.»

\par 39 En entendant cela, Antiochus, assoiffé de sang, meurtrier et tout à fait abominable, leur ordonna de lui couper la langue.

\par 40 Mais il dit : Même si tu enlèves mon organe de la parole, Dieu entend aussi ceux qui sont muets.

\par 41 «Voici, je tire ma langue tout prêt : coupe-la, car tu ne feras pas ainsi taire ma Raison.»

\par 42 Nous donnons volontiers nos membres corporels à la mutilation pour la cause de Dieu.

\par 43 Mais Dieu te poursuivra promptement ; car tu as coupé la langue qui lui chantait des chants de louange.

\par 44 Mais lorsque cet homme fut également mis à mort par les tortures, le cinquième s'élança en disant : « Je n'hésite pas, ô tyran, à exiger la torture pour l'amour de la vertu.

\par 45 « Oui, je m'avance de moi-même, afin qu'en me tuant aussi, tu puisses, par de nouveaux méfaits, augmenter le châtiment que tu dois à la justice du Ciel.

\par 46 'Ô ennemi de la vertu et ennemi de l'homme, pour quel crime nous détruis-tu de cette manière ?'

\par 47 'Est-ce que cela te semble mal que nous adorions le Créateur de tous et que nous vivions selon sa loi vertueuse ?'

\par 48 «Mais ces choses sont dignes d'honneurs et non de tortures, si tu comprenais les aspirations humaines et si tu avais l'espoir d'être sauvé devant Dieu.»

\par 49 «Voici, maintenant tu es l'ennemi de Dieu et tu fais la guerre à ceux qui adorent Dieu.»

\par 50 Pendant qu'il parlait ainsi, les gardes le ligotèrent et l'amenèrent devant la catapulte ; et ils l'y attachèrent sur ses genoux, et, les y attachant avec des pinces de fer, ils enroulèrent ses reins sur le « coin » roulant, de sorte qu'il était complètement enroulé en arrière comme un scorpion et que chaque jointure était disjointe.

\par 51 Et ainsi, dans un essoufflement douloureux et une angoisse corporelle, il s'écria : « Glorieuses, ô tyran, glorieuses contre ta volonté sont les bienfaits que tu m'accordes, me permettant de montrer ma fidélité à la Loi par des tortures encore plus honorables. .'

\par 52 Et quand cet homme aussi fut mort, on amena le sixième, un simple garçon, qui, répondant à la question du tyran s'il était disposé à manger et à être relâché, dit :

\par 53 'Je ne suis pas aussi vieux en âge que mes frères, mais je suis aussi vieux d'esprit. Car nous sommes nés et avons grandi dans le même but et sommes également tenus de mourir pour la même cause ; donc si tu choisis de nous torturer parce que nous ne mangeons pas de viande impure, torture-toi.

\par 54 Tandis qu'il prononçait ces paroles, ils l'amenèrent à la roue, et avec soin ils l'étendirent, lui déboîtèrent les os du dos et mirent le feu sous lui.

\par 55 Et ils firent des brochettes pointues, chauffées au rouge, et les lui enfoncèrent dans le dos, et lui perçant les côtés, ils brûlèrent aussi ses entrailles.

\par 56 Mais lui, au milieu de ses tortures, s'écria : « Ô concours digne des saints, dans lequel tant d'entre nous, frères, dans la cause de la justice, avons été engagés dans une compétition de tourments et n'avons pas été vaincus !

\par 57 Car l'intelligence juste, ô tyran, est invincible.

\par 58 Dans l'armure de la vertu, je vais rejoindre mes frères dans la mort, et ajouter en moi un puissant vengeur de plus pour te punir, ô inventeur de tourments et ennemi des vrais justes.

\par 59 Nous, six jeunes gens, avons renversé ta tyrannie. « Car ton impuissance à altérer notre raison ou à nous forcer à manger de la viande impure n'est-elle pas un renversement pour toi ? »

\par 60 «Ton feu est froid pour nous, tes engins de torture ne tourmentent pas, et ta violence est impuissante.»

\par 61 « Car les gardes ont été pour nous des officiers, non d'un tyran, mais de la loi divine ; et c'est pourquoi nous avons encore notre Raison invaincue.

\chapter{6}

\par \textit{Des liens fraternels et un amour maternel.}

\par 1 ET quand celui-ci mourut aussi d'une mort bienheureuse, étant jeté dans le chaudron, le septième fils, le plus jeune de tous, s'avança.

\par 2 Mais le tyran, bien que furieusement exaspéré par ses frères, eut pitié du garçon, et le voyant déjà là, il le fit approcher et chercha à le persuader, en disant :

\par 3 'Tu vois la fin de la folie de tes frères ; car à cause de leur désobéissance, ils ont été torturés à mort. Toi aussi, si tu n'obéis pas, tu seras toi aussi misérablement torturé et mis à mort avant ton heure ; mais si tu obéis, tu seras mon ami et tu accéderas à de hautes fonctions dans les affaires du royaume.

\par 4 Et tout en lui faisant ainsi appel, il envoya chercher la mère du garçon, afin que, dans son chagrin pour la perte de tant de fils, elle puisse exhorter le survivant à obéir et à être sauvé.

\par 5 Mais la mère, parlant en langue hébraïque, comme je le dirai plus tard, encouragea le garçon, et il dit aux gardes : « Détachez-moi, et je parlerai au roi et à tous ses amis avec lui. '

\par 6 Et eux, se réjouissant de la demande du garçon, se hâtèrent de le libérer.

\par 7 Et courant vers le brasier chauffé au rouge : « Ô tyran impie », s'écria-t-il, « et le plus impie de tous les pécheurs, n'as-tu pas honte de prendre tes bénédictions et ta royauté entre les mains de Dieu, et de tuer ses serviteurs et torturer les adeptes de la justice ?

\par 8 'Pour quelles choses la justice divine te livre à un feu plus rapide et éternel et à des tourments qui ne te lâcheront pas de toute l'éternité.'

\par 9 « N'as-tu pas honte, étant un homme, ô misérable au cœur de bête sauvage, de prendre avec toi des hommes partageant les mêmes sentiments, faits des mêmes éléments, et de leur arracher la langue, et de les fouetter et de les torturer dans de cette manière ?

\par 10 «Mais pendant qu'ils auront accompli leur justice envers Dieu dans leur noble mort, tu crieras misérablement» Malheur est rencontré «pour ton meurtre injuste des champions de la vertu.»

\par 11 Et puis, se tenant au seuil de la mort, il dit : « Je ne suis pas un renégat du témoignage rendu par mes frères.

\par 12 'Et j'invoque le Dieu de mes pères pour qu'il soit miséricordieux envers ma nation.'

\par 13 'Et il te punira à la fois dans cette vie présente et après cela tu seras mort.'

\par 14 Et avec cette prière, il se jeta dans le brasier chauffé au rouge et rendit ainsi l'âme.

\par 15 Si donc les sept frères ont méprisé les tourments jusqu'à la mort, il est universellement prouvé que la Raison Inspirée est le maître suprême des passions.

\par 16 Car s'ils avaient cédé à leurs passions ou à leurs souffrances et mangé de la viande impure, nous aurions dit qu'ils en avaient été vaincus.

\par 17 Mais dans cette ville-là, il n'en était pas ainsi ; au contraire, par leur raison, qui était louée devant Dieu, ils s'élevaient au-dessus de leurs passions.

\par 18 Et il est impossible de nier la suprématie de l'esprit ; car ils ont remporté la victoire sur leurs passions et leurs douleurs.

\par 19 Comment faire autrement que d'admettre la juste maîtrise de la Raison sur la passion chez ces hommes qui n'ont pas reculé devant les angoisses de l'incendie ?

\par 20 Car de même que les tours sur les môles du port repoussent les assauts des vagues et offrent une entrée calme à ceux qui entrent dans le port, ainsi la droite raison aux sept tours des jeunes défendait le port de la justice et repoussait la tempête des passions. .

\par 21 Ils formaient un saint chœur de justice et s'encourageaient les uns les autres, disant :

\par 22 'Mourons comme des frères, ô frères, pour la Loi.'

\par 23 'Imitons les trois enfants de la cour assyrienne qui méprisaient cette même épreuve de la fournaise.'

\par 24 'Ne devenons pas lâches avant la preuve de la justice.'

\par 25 Et l'un dit : « Frère, prends courage », et l'autre : « Supporte-le noblement » ; et un autre rappelant le passé : « Souvenez-vous de quelle souche vous êtes, et sous la main de laquelle Isaac, pour l'amour de la justice, s'est livré pour être un sacrifice. »

\par 26 Et chacun d'eux ensemble, se regardant avec un regard brillant et très hardi, dit : « De tout notre cœur, nous nous consacrerons à Dieu qui nous a donné nos âmes, et prêtons nos corps à la garde du Loi.'

\par 27 « Ne craignons pas celui qui pense qu'il tue ; car une grande lutte et un grand péril pour l'âme attendent dans un tourment éternel ceux qui transgressent l'ordonnance de Dieu.

\par 28 « Armons-nous donc de la maîtrise des passions de la Raison divine.

\par 29 'Après cela, notre passion, Abraham, Isaac et Jacob nous recevront, et tous nos ancêtres nous loueront.'

\par 30 Et à chacun des frères, tandis qu'on les emmenait, ceux dont le tour était encore venu dirent : Ne nous déshonore pas, frère, et ne trahis pas nos frères déjà morts.

\par 31 Vous n'ignorez pas l'amour des frères, dont la divine et toute sage Providence a donné un héritage à ceux qui sont engendrés par leurs pères, l'implantant en eux dès le sein de leur mère ; dans lequel les frères demeurent pendant la même période, prennent leur forme pendant le même temps, se nourrissent du même sang, sont vivifiés par la même âme, sont mis au monde après le même espace et tirent du lait du mêmes sources, par lesquelles leurs âmes fraternelles sont allaitées ensemble dans les bras au sein ; et ils sont encore plus étroitement liés par une éducation commune, une compagnie quotidienne et d'autres formes d'éducation, ainsi que par notre discipline sous la Loi de Dieu.

\par 32 Le sentiment d'amour fraternel étant ainsi naturellement fort, les sept frères virent leur concorde mutuelle renforcée encore. Car formés dans la même loi, disciplinés dans les mêmes vertus et élevés ensemble dans la vie droite, ils s'aimèrent les uns les autres plus abondamment. Leur zèle commun pour la beauté et la bonté morales augmentait leur concorde mutuelle, car, alliée à leur piété, elle rendait leur amour fraternel plus fervent.

\par 33 Mais bien que la nature, la camaraderie et leur disposition vertueuse aient accru l'ardeur de leur amour fraternel, néanmoins les fils survivants, par leur religion, ont soutenu la vue de leurs frères, qui étaient en flagrant délit, étant torturés à mort ; bien plus, ils les encourageaient même à affronter l'agonie, afin non seulement de mépriser leurs propres tortures, mais aussi de vaincre leur passion d'affection fraternelle pour leurs frères.

\par 34 Ô esprits raisonnés, plus royaux que les rois, que hommes libres plus libres, de l'harmonie des sept frères, saints et bien accordés à la note fondamentale de la piété !

\par 35 Aucun des sept jeunes gens ne s'est montré lâche, aucun n'a reculé devant la mort, mais tous se sont précipités vers la mort par la torture comme s'ils couraient sur le chemin de l'immortalité.

\par 36 Car, de même que les mains et les pieds bougent en harmonie avec les inspirations de l'âme, de même ces saints jeunes, comme poussés par l'âme immortelle de la religion, allèrent en harmonie vers la mort à cause d'elle.

\par 37 Ô toute sainte septuple compagnie de frères en harmonie !

\par 38 Car, de même que les sept jours de la création du monde ont entouré la religion, de même les jeunes, semblables à un chœur, ont entouré leur septuple compagnie, et ont rendu sans compte la terreur des tortures.

\par 39 Nous frémissons maintenant lorsque nous entendons parler de la souffrance de ces jeunes ; mais eux, non seulement le voyant de leurs yeux, ni simplement entendre la menace imminente et prononcée, mais ressentant réellement la douleur, l'endurèrent jusqu'au bout ; et cela dans la torture par le feu, que quelle plus grande agonie peut-on trouver ?

\par 40 Car la puissance du feu est aiguë et rigoureuse, et elle a rapidement amené leurs corps à la dissolution.

\par 41 Et ne trouvez-vous pas étonnant que, chez ces hommes, la raison triomphe des tourments, alors que même l'âme d'une femme méprisait une plus grande diversité de souffrances ; car la mère des sept jeunes gens endura les tourments infligés à chacun de ses enfants.

\par 42 Mais considérez combien les aspirations du cœur d'une mère sont multiples, de sorte que ses sentiments pour sa progéniture deviennent le centre de tout son monde ; et en effet, \p

\par Ici, même les animaux irrationnels ont pour leurs petits une affection et un amour semblables à ceux des hommes.

\par 43 Par exemple, parmi les oiseaux, les apprivoisés qui s'abritent sous nos toits défendent leurs oisillons ; et ceux qui nichent sur les sommets des montagnes, et dans les fentes des rochers, et dans les trous des arbres, et dans les branches, et y font éclore leurs petits, chassent aussi l'intrus.

\par 44 Et puis, s'ils ne peuvent le chasser, ils s'agitent autour des petits avec une passion d'amour, les appelant dans leur propre langage, et ils secourent leurs petits de toutes les manières possibles.

\par 45 Et qu'avons-nous besoin d'exemples d'amour pour la progéniture parmi les animaux irrationnels, quand même les abeilles, à l'époque de la fabrication du rayon, repoussent les intrus et poignardent de leur aiguillon, comme avec une épée, ceux-là. qui s'approchent de leurs enfants et les combattent jusqu'à la mort ?

\par 46 Mais elle, la mère de ces jeunes gens, avec une âme comme Abraham, n'a pas été détournée de son dessein par son affection pour ses enfants.

\chapter{7}

\par \textit{Une comparaison des affections d'une mère et d'un père, dans ce chapitre se trouvent quelques sommets d'éloquence.}

\par 1 RAISON des fils, seigneur des passions ! Ô religion, qui était plus chère à la mère que ses enfants !

\par 2 La mère, ayant deux choix devant elle, la religion et le présent salut de ses sept fils selon la promesse du tyran, aimait plutôt la religion, qui sauve pour la vie éternelle selon Dieu.

\par 3 O comment puis-je exprimer l'amour passionné des parents pour leurs enfants ? Nous imprimons une merveilleuse ressemblance de notre âme et de notre forme sur la nature tendre de l'enfant, et surtout parce que la sympathie de la mère pour ses enfants est plus profonde que celle du père.

\par 4 Car les femmes sont plus douces d'âme que les hommes, et plus elles ont d'enfants, plus elles abondent en amour pour eux.

\par 5 Mais de toutes les mères, celle des sept fils était plus amoureuse que les autres, car, ayant eu sept grossesses, elle avait ressenti une tendresse maternelle pour le fruit de ses entrailles, et ayant été contrainte à cause des nombreuses douleurs dans qu'elle portait à chacun avec une étroite affection, elle rejeta néanmoins, par crainte de Dieu, la sécurité actuelle de ses enfants.

\par 6 Oui, et plus encore, grâce à la beauté morale et à la bonté de ses fils et à leur obéissance à la Loi, son amour maternel pour eux fut rendu plus fort.

\par 7 Car ils étaient justes, modérés, vaillants et grands d'âme, et amoureux les uns des autres et de leur mère, de telle manière qu'ils lui obéirent dans l'observation de la loi jusqu'à la mort.

\par 8 Mais néanmoins, bien qu'elle ait eu tant de tentations de céder à ses instincts maternels, dans aucun cas la terrible variété de tortures n'a eu le pouvoir d'altérer sa raison ; mais la mère exhortait chaque fils séparément et tous ensemble à mourir pour leur religion.

\par 9 Ô sainte nature, et amour parental, et désir des parents pour leur progéniture, et salaire des soins infirmiers, et affection invincible des mères !

\par 10 La mère, les voyant un à un déchirés et brûlés, resta inébranlable dans son âme à cause de la religion.

\par 11 Elle vit la chair de ses fils consumée par le feu, et les extrémités de leurs mains et de leurs pieds éparpillées sur le sol, et les couvertures de chair arrachées de la tête jusqu'aux joues, éparpillées comme des masques.

\par 12 Ô mère, qui connaissais maintenant des douleurs plus aiguës que celles de l'accouchement ! Ô femme, seule parmi les femmes, dont le fruit du sein était une religion parfaite !

\par 13 Ton premier-né, rendant l'âme, n'a pas changé ta résolution, ni ton second, te regardant avec des yeux de pitié sous ses tourments, ni ton troisième, expirant son esprit.

\par 14 Tu n'as pas non plus pleuré en voyant les yeux de chacun, au milieu des tourments, regardant hardiment la même angoisse, et tu as vu dans leurs narines frémissantes les signes de la mort prochaine.

\par 15 Quand tu as vu la chair d'un fils coupée après la chair d'un autre, et main après main coupée, tête après tête écorchée, cadavre jeté sur cadavre, et la place remplie de spectateurs à cause de la tortures de tes enfants, tu ne verses pas une larme.

\par 16 Ni les mélodies des sirènes ni les chants des cygnes au son doux ne charment autant les oreilles de celui qui les entend, comme résonnaient les voix des fils, parlant à la mère au milieu des tourments.

\par 17 Combien et combien grandes étaient les tortures avec lesquelles la mère était tourmentée tandis que ses fils étaient torturés avec les tourments du feu et du feu !

\par 18 Mais la raison inspirée prêta à son cœur une force d'homme sous sa passion de souffrance, et l'exalta au point de ne pas tenir compte des aspirations actuelles de l'amour maternel.

\par 19 Et bien qu'elle ait vu la destruction de ses sept enfants et les formes nombreuses et variées de leurs tourments, la noble mère les a abandonnés volontairement par la foi en Dieu.

\par 20 Car elle voyait dans son propre esprit, comme s'il s'agissait d'avocats rusés dans une chambre du conseil, la nature, la parentalité, l'amour maternel, et ses enfants à l'épreuve, et c'était comme si elle, la mère , ayant le choix entre deux voix dans le cas de ses enfants, une pour leur mort et une pour les sauver en vie, n'envisagea alors pas de sauver ses sept fils pendant un court moment, mais, comme une vraie fille d'Abraham, appelée à faites attention à son courage qui craint Dieu.

\par 21 Ô mère de la race, justificatrice de notre Loi, défenseur de notre religion et gagnante du prix dans la lutte intérieure !

\par 22 Ô femme, plus noble pour résister que les hommes, et plus courageuse que les guerriers pour endurer !

\par 23 Car, de même que l'Arche de Noé, avec le monde vivant tout entier pour fardeau dans le déluge qui ravageait le monde, a résisté aux puissantes vagues, ainsi toi, le gardien de la Loi, battu de toutes parts par les vagues déferlantes du Passionnés, et tendu comme par de forts souffles par les tortures de tes fils, tu as noblement résisté aux tempêtes qui t'assaillirent à cause de la religion.

\par 24 Ainsi donc, si une femme âgée, mère de sept fils, supportait la vue de ses enfants torturés à mort, il faut reconnaître que la Raison inspirée est la maîtresse suprême des passions.

\par 25 J'ai donc prouvé que non seulement les hommes ont triomphé de leurs souffrances, mais qu'une femme aussi a méprisé les tortures les plus terribles.

\par 26 Et les lions autour de Daniel n'étaient pas si féroces, ni la fournaise ardente de Mishael n'était pas si brûlante, qu'elle brûlait l'instinct de maternité à la vue de ses sept fils torturés.

\par 27 Mais par sa raison guidée par la religion, la mère a apaisé ses passions, si nombreuses et si fortes soient-elles.

\par 28 Car il y a aussi ceci à considérer : si la femme avait été faible d'esprit, malgré sa maternité, elle aurait pu pleurer sur eux, et peut-être parler ainsi :

\par 29 « Ah, trois fois misérable, et plus de trois fois misérable ! » J'ai mis au monde sept enfants et je me retrouve sans enfant ! »

\par 30 «J'ai été enceinte sept fois en vain, et sept fois le fardeau de mes dix mois a été supporté en vain, et mes tétées ont été vaines et mes nourrissons tristes.»

\par 31 «C'est en vain pour vous, ô mes fils, que j'ai enduré les nombreuses douleurs du travail et les soins les plus difficiles de votre éducation.»

\par 32 Hélas ! mes fils, certains n'étaient pas encore mariés, et ceux qui étaient mariés n'avaient pas engendré d'enfants ; Je ne verrai jamais vos enfants et je ne serai jamais appelé du nom de grand-parent.

\par 33 « Ah moi, qui ai eu beaucoup de beaux enfants, et je suis veuve et désolée dans mon malheur ! » Il n'y aura pas non plus de fils pour m'enterrer quand je serai mort !''

\par 34 Mais la sainte et pieuse mère ne se lamentait pas ainsi sur aucun d'eux, ne suppliait personne d'échapper à la mort, ni ne se lamentait sur eux comme des mourants ; mais, comme si elle avait une âme inflexible et qu'elle faisait naître une seconde fois le nombre de ses fils dans la vie immortelle, elle les supplia plutôt et les supplia de mourir pour l'amour de la religion.

\par 35 Ô mère, guerrière de Dieu pour la cause de la religion, vieille et femme, tu as vaincu le tyran par ton endurance, et tu as été trouvée plus forte qu'un homme, en actes comme en paroles.

\par 36 Car en vérité, lorsque tu étais lié avec tes fils, tu étais là, voyant Éléazar être torturé, et tu parlais à tes fils en langue hébraïque :

\par 37 'Mes fils, noble est le combat ; et vous, étant appelés à témoigner pour notre nation, combattez-y avec zèle en faveur de la loi de nos pères.

\par 38 «Car il serait honteux que, pendant que ce vieil homme endurait l'agonie à cause de la religion, vous, jeunes hommes, reculiez devant la douleur.»

\par 39 'Rappelez-vous que c'est pour l'amour de Dieu que vous êtes venus dans le monde et que vous avez joui de la vie, et que c'est pourquoi vous devez à Dieu d'endurer toute douleur pour lui ; pour lequel aussi notre père Abraham s'est empressé de sacrifier son fils Isaac, l'ancêtre de notre nation ; et Isaac, voyant la main de son père lever le couteau contre lui, ne recula pas.

\par 40 «Et Daniel, le juste, fut jeté aux lions, et Ananias, Azarias et Mishael furent jetés dans la fournaise de feu, et ils endurèrent à cause de Dieu.»

\par 41 « Et vous aussi, ayant la même foi en Dieu, ne soyez pas troublés ; car il serait contraire à la raison que vous, connaissant la justice, ne supportiez pas les douleurs.

\par 42 Par ces paroles, la mère des sept encouragea chacun de ses fils à mourir plutôt que de transgresser l'ordonnance de Dieu ; Eux-mêmes savent aussi bien que les hommes mourant pour Dieu vivent pour Dieu, comme vivent Abraham, Isaac, Jacob et tous les patriarches.



\chapitre{8}

\par \textit{Les célèbres « Athlètes de la justice ». Ici se termine l'histoire du courage appelée le Quatrième Livre des Macchabées.}

\par 1 CERTAINS gardes déclarèrent que lorsqu'elle aussi allait être saisie et mise à mort, elle se jeta sur le bûcher afin que personne ne puisse toucher son corps.

\par 2 Ô mère, qui, avec tes sept fils, as brisé la force du tyran, mis à néant ses mauvais desseins, et donné un exemple de la noblesse de la foi.

\par 3 Tu étais noblement placé comme un toit sur tes fils comme des colonnes, et le tremblement de terre des tourments ne t'ébranlait pas du tout.

\par 4 Réjouis-toi donc, mère à l'âme pure, ayant l'espérance de ton endurance assurée de la part de Dieu.

\par 5 La lune n'est pas si majestueuse au milieu des étoiles dans le ciel que toi, ayant éclairé le chemin de tes sept fils semblables à des étoiles vers la justice, tu es en honneur devant Dieu ; et tu es placé au ciel avec eux.

\par 6 Car tu as enfanté du fils d'Abraham.

\par 7 Et s'il nous avait été permis de peindre, comme le ferait un artiste, l'histoire de ta piété, les spectateurs n'auraient-ils pas frémi devant la mère de sept fils souffrant pour l'amour de la justice d'innombrables tortures jusqu'à la mort ?

\par 8 Et en effet, il convenait d'inscrire ces mots sur leur lieu de repos, en guise de mémorial pour les générations futures de notre peuple :

\par ICI LIT UN PRÊTRE VIEILLI
\par ET UNE FEMME PLEINE D'ANNÉES
\par ET SES SEPT FILS
\par PAR LA VIOLENCE D'UN TYRAN
\par DÉSIRANT DE DÉTRUIRE LA NATION HÉBRAÏQUE.
\par ILS ONT VALIDÉ LES DROITS DE NOTRE PEUPLE
\par REGARDER SUR DIEU ET PERMETTRE
\par LES TOURMENTS MÊME
\par LA MORT.

\par 9 Car en vérité, c'était une guerre sainte qu'ils menaient. Car ce jour-là, la vertu, les éprouvant par l'endurance, leur proposa le prix de la victoire dans l'incorruption dans la vie éternelle.

\par 10 Mais le premier dans le combat fut Éléazar, et la mère des sept fils joua son rôle, et les frères combattirent.

\par 11 Le tyran était leur adversaire et le monde et la vie de l'homme étaient les spectateurs.

\par 12 Et la justice gagna le vainqueur et donna la couronne à ses athlètes. Qui ne s’interrogeait sur les athlètes de la vraie Loi ?

\par 13 Qui n'en a pas été étonné ? Le tyran lui-même et tout son conseil admiraient leur endurance, grâce à laquelle ils se tiennent maintenant à côté du trône de Dieu et vivent l'âge béni.

\par 14 Car Moïse dit : Tous ceux qui se sont sanctifiés sont également entre tes mains.

\par 15 Et ces hommes donc, s'étant sanctifiés pour l'amour de Dieu, ont non seulement reçu cet honneur, mais aussi l'honneur que, par leur intermédiaire, l'ennemi n'avait plus de pouvoir sur notre peuple, et que le tyran a subi un châtiment, et que notre pays a été purifiés, ils étant pour ainsi dire devenus une rançon pour le péché de notre nation ; et par le sang de ces justes et la propitiation de leur mort, la Divine Providence délivra Israël qui auparavant était mal supplié.

\par 16 Car le tyran Antiochus, voyant l'héroïsme de leur vertu et leur endurance sous les tortures, il donna publiquement en exemple à ses soldats leur endurance ; et il inspira ainsi à ses hommes un sentiment d'honneur et d'héroïsme sur le champ de bataille et dans les travaux de siège, de sorte qu'il pilla et renversa tous ses ennemis.

\par 17 Ô Israélites, enfants nés de la postérité d'Abraham, obéissez à cette loi et soyez justes en toutes choses, reconnaissant que la raison inspirée est maîtresse des passions et des douleurs, non seulement du dedans, mais du dehors de nous-mêmes ; par quel moyen ces hommes, livrant leurs corps à la torture pour l'amour de la justice, non seulement gagnèrent l'admiration de l'humanité, mais furent jugés dignes d'un héritage divin.

\par 18 Et grâce à eux, la nation a obtenu la paix et le rétablissement de l'observance de la loi dans notre pays a pris la ville à l'ennemi.

\par 19 Et la vengeance a poursuivi le tyran Antiochus sur la terre, et il subit le châtiment dans la mort.

\par 20 Car n'ayant absolument pas réussi à contraindre les habitants de Jérusalem à vivre comme les païens et à abandonner les coutumes de nos pères, il quitta alors Jérusalem et partit contre les Perses.

\par 21 Or, voici les paroles que la mère des sept fils, la femme juste, dit à ses enfants :

\par 22 'J'étais une jeune fille pure, et je ne m'éloignais pas de la maison de mon père, et j'ai gardé la côte qui a été bâtie en Ève.'

\par 23 « Aucun séducteur du désert, aucun trompeur dans les champs ne m'a corrompu ; le faux et séduisant Serpent n’a pas non plus souillé la pureté de ma virginité ; J'ai vécu avec mon mari tous les jours de ma jeunesse ; mais quand mes fils furent grands, leur père mourut.

\par 24 'Heureux était-il; car il a vécu une vie bénie avec des enfants et il n'a jamais connu la douleur de leur perte.

\par 25 « Qui, alors qu'il était encore avec nous, vous a enseigné la loi et les prophètes. Il nous a lu Abel qui fut tué par Caïn, et Isaac qui fut offert en holocauste, et Joseph en prison.

\par 26 «Et il nous a parlé de Phinéas, le prêtre zélé, et il vous a enseigné le chant d'Ananias, d'Azarias et de Mishael dans le feu.»

\par 27 « Et il glorifia aussi Daniel dans la fosse aux lions, et le bénit ; et il vous a rappelé la parole d'Isaïe :

\par 28 '«Oui, même si tu passes par le feu, la flamme ne te fera pas de mal.»'

\par 29 « Il nous a chanté les paroles du psalmiste David : « Nombreuses sont les afflictions des justes. »

\par 30 «Il nous a cité le proverbe de Salomon : »Il est un arbre de vie pour tous ceux qui font sa volonté.»

\par 31 « Il confirma les paroles d'Ézéchiel : « Ces ossements desséchés vivront-ils ? Car il n’a pas oublié le cantique que Moïse a enseigné, qui enseigne : « Je tuerai et je ferai vivre. C'est votre vie et la bénédiction de vos jours.

\par 32 Ah ! cruel fut le jour, et pourtant non cruel, où le cruel tyran des Grecs alluma le feu de ses brasiers barbares, et où, avec ses passions bouillantes, il conduisit à la catapulte et retourna à ses tourments les sept fils de la fille d'Abraham, et ils leur aveuglèrent les globes oculaires, et leur coupèrent la langue, et les tuèrent de toutes sortes de tourments.

\par 33 C'est pour cette raison que le jugement de Dieu a poursuivi et poursuivra le misérable maudit.

\par 34 Mais les fils d'Abraham, avec leur mère victorieuse, sont rassemblés à la place de leurs ancêtres, ayant reçu de Dieu des âmes pures et immortelles, à qui soit gloire pour les siècles des siècles.


\end{document}