\begin{document}

\title{Livre des Jubilés}

\chapter{1}

\par \textit{Moïse reçoit les tables de la loi et les instructions sur l'histoire passée et future qu'il doit inscrire dans un livre, 1-4. Apostasie d'Israël,5-9. Captivité d'Israël et de Juda, 10-13. Retour de Juda et reconstruction du temple, 15-18. Prière de Moïse pour Israël, 19-21. La promesse de Dieu de les racheter et de demeurer avec eux, 22-5, 28. Moïse a ordonné d'écrire l'histoire future du monde (le Livre des Jubilés ?), 26. Et un ange d'écrire la loi, 27. Cet ange prend les tablettes chronologiques célestes pour en dicter à Moïse, 29.}

\par C'est l'histoire de la division des jours de la loi et du témoignage, des événements des années, de leurs (années) semaines, de leurs jubilés au cours de toutes les années du monde, comme le Seigneur l'a dit à Moïse sur le mont Sinaï lorsqu'il monta pour recevoir les tables de la loi et des commandements, selon la voix de Dieu lorsqu'il lui dit : « Monte au sommet de la montagne ».

\par 1 Et il arriva la première année de l'exode des enfants d'Israël hors d'Egypte, le troisième mois, le seizième jour du mois, [2450 Anno Mundi] que Dieu parla à Moïse, disant : «Monte à moi sur la montagne, et je te donnerai deux tables de pierre de la loi et du commandement que j'ai écrit, afin que tu les enseignes.»
\par 2 Et Moïse monta sur la montagne de Dieu, et la gloire de l'Éternel demeura sur la montagne de Sinaï, et une nuée l'éclipsa pendant six jours.
\par 3 Et il appela Moïse le septième jour du milieu de la nuée, et l'apparition de la gloire de l'Éternel fut comme un feu flamboyant au sommet de la montagne.
\par 4 Et Moïse resta sur la montagne quarante jours et quarante nuits, et Dieu lui enseigna l'histoire ancienne et postérieure de la division de tous les jours de la loi et du témoignage.
\par 5 Et Il dit : 'Incline ton cœur à chaque parole que je te dirai sur cette montagne, et écris-les dans un livre afin que leurs générations voient comment je ne les ai pas abandonnés à cause de tout le mal qu'ils ont eu. j'ai travaillé en transgressant l'alliance que j'établis aujourd'hui entre moi et toi pour leurs générations sur le mont Sinaï.
\par 6 'Et ainsi il arrivera que lorsque toutes ces choses leur arriveront, ils reconnaîtront que je suis plus juste qu'eux dans tous leurs jugements et dans toutes leurs actions, et ils reconnaîtront que j'ai été vraiment avec eux.'
\par 7 « Et écris pour toi toutes ces paroles que je te déclare aujourd'hui, car je connais leur rébellion et leur raideur de la nuque, avant de les amener dans le pays dont j'ai juré à leurs pères, à Abraham et à à Isaac et à Jacob, en disant : Je donnerai à ta postérité un pays où coulent le lait et le miel.
\par 8 'Et ils mangeront et seront rassasiés, et ils se tourneront vers des dieux étrangers, vers (des dieux) qui ne peuvent les délivrer de rien de leur tribulation : et ce témoignage sera entendu comme témoignage contre eux. Car ils oublieront tous mes commandements, (même) tout ce que je leur commande, et ils marcheront selon les païens, et après leur impureté et après leur honte, et serviront leurs dieux, et cela leur sera une offense et une tribulation, une affliction et un piège.
\par 9 'Et beaucoup périront et seront faits prisonniers et tomberont entre les mains de l'ennemi, parce qu'ils ont abandonné mes ordonnances et mes commandements, et les fêtes de mon alliance, et mes sabbats, et mon lieu saint. que je me suis sanctifié au milieu d'eux, et mon tabernacle et mon sanctuaire, que je me suis sanctifié au milieu du pays, afin que j'y établisse mon nom et qu'il y habite.'
\par 10 « Et ils se feront des hauts lieux, des bosquets et des images taillées, et ils adoreront chacun la sienne (image taillée), de manière à s'égarer, et ils sacrifieront leurs enfants aux démons et à tous les œuvres de l'erreur de leur cœur.
\par 11 'Et je leur enverrai des témoins, afin que je puisse témoigner contre eux, mais ils n'écouteront pas, et tueront aussi les témoins, et ils persécuteront ceux qui recherchent la loi, et ils abrogeront et changeront tout ainsi quant à faire le mal sous mes yeux.
\par 12 « Et je leur cacherai ma face, et je les livrerai entre les mains des païens pour les captifs, les proies et les dévorants, et je les éloignerai du milieu du pays, et je les dispersera parmi les païens.
\par 13 'Et ils oublieront toute ma loi et tous mes commandements et tous mes jugements, et s'égareront en ce qui concerne les nouvelles lunes, les sabbats, les fêtes, les jubilés et les ordonnances.'
\par 14 « Et après cela, ils se tourneront vers moi du milieu des païens de tout leur cœur, de toute leur âme et de toutes leurs forces, et je les rassemblerai du milieu de tous les païens, et ils me chercheront, afin que Ils me trouveront quand ils me chercheront de tout leur cœur et de toute leur âme.
\par 15 'Et je leur révélerai une paix abondante avec la justice, et je leur ôterai la plante de la droiture, de tout mon cœur et de toute mon âme, et ils seront pour une bénédiction et non pour une malédiction, et ils ce sera la tête et non la queue.
\par 16 'Et je bâtirai mon sanctuaire au milieu d'eux, et j'habiterai avec eux, et je serai leur Dieu et ils seront mon peuple dans la vérité et la justice.'
\par 17 'Et je ne les abandonnerai pas ni ne les laisserai tomber ; car je suis l'Éternel, leur Dieu.
\par 18 Et Moïse tomba sur sa face et pria et dit : « Ô Seigneur mon Dieu, n'abandonne pas ton peuple et ton héritage, afin qu'ils errent dans l'erreur de leur cœur, et ne les livre pas entre les mains de leurs ennemis, les païens, de peur qu'ils ne dominent sur eux et ne les incitent à pécher contre toi.
\par 19 'Que ta miséricorde, ô Seigneur, s'élève sur ton peuple et crée en lui un esprit droit, et que l'esprit de Beliar ne règne pas sur lui pour l'accuser devant toi et le piéger de tous les sentiers. de justice, afin qu'ils périssent devant ta face.
\par 20 « Mais ils sont ton peuple et ton héritage, que tu as délivré par ta grande puissance des mains des Égyptiens : crée en eux un cœur pur et un esprit saint, et qu'ils ne soient désormais plus pris au piège dans leurs péchés. » jusqu'à l'éternité.
\par 21 Et le Seigneur dit à Moïse : 'Je connais leur contradiction, leurs pensées et leur raideur du cou, et ils ne seront pas obéissants jusqu'à ce qu'ils confessent leur propre péché et celui de leurs pères.'
\par 22 « Et après cela, ils se tourneront vers moi en toute droiture et de tout (leur) cœur et de toute (leur) âme, et je circoncirai le prépuce de leur cœur et le prépuce du cœur de leur postérité, et Je créerai en eux un esprit saint et je les purifierai afin qu'ils ne se détournent pas de moi à partir de ce jour et pour l'éternité.
\par 23 'Et leurs âmes s'attacheront à moi et à tous mes commandements, et ils accompliront mes commandements, et je serai leur Père et ils seront mes enfants.'
\par 24 'Et ils seront tous appelés enfants du Dieu vivant, et tout ange et tout esprit sauront, oui, ils sauront que ceux-ci sont mes enfants, et que je suis leur Père en droiture et en justice, et que je suis leur Père. les aime.'
\par 25 'Et écris pour toi toutes ces paroles que je te déclare sur cette montagne, la première et la dernière, qui s'accompliront dans toutes les divisions des jours dans la loi et dans le témoignage et dans les semaines et les jubilés jusqu'à l'éternité, jusqu'à ce que je descende et demeure avec eux pendant toute l'éternité.
\par 26 Et Il dit à l'ange de la présence : 'Écris pour Moïse depuis le commencement de la création jusqu'à ce que mon sanctuaire soit bâti parmi eux pour toute l'éternité.'
\par 27 'Et l'Éternel apparaîtra aux yeux de tous, et tous sauront que je suis le Dieu d'Israël et le père de tous les enfants de Jacob, et le roi du mont Sion pour l'éternité. Et Sion et Jérusalem seront saintes.
\par 28 Et l'ange de la présence qui marchait devant le camp d'Israël prit les tables des divisions des années - depuis le temps de la création - de la loi et du témoignage des semaines des jubilés, selon le années individuelles, selon le nombre total des jubilés [selon les années individuelles], à partir du jour de la [nouvelle] création, où les cieux et la terre seront renouvelés et toute leur création selon les puissances des cieux, et selon toute la création de la terre, jusqu'à ce que le sanctuaire du Seigneur soit fait à Jérusalem sur le mont Sion, et que tous les luminaires soient renouvelés pour la guérison, la paix et la bénédiction pour tous les élus d'Israël, et qu'ainsi il peut-être depuis ce jour-là et jusqu'à tous les jours de la terre.

\chapter{2}

\par \textit{L'histoire des vingt-deux actes distincts de création sur les six jours, 1-16. Institution du sabbat : son observance par les anges les plus élevés, avec lesquels Israël doit ensuite être associé, 17-32. (cf. Gen. I-II. 3.)}

\par 1 Et l'ange de la présence parla à Moïse selon la parole du Seigneur, disant : Écrivez l'histoire complète de la création, comment en six jours le Seigneur Dieu a achevé toutes ses œuvres et tout ce qu'il a créé, et a observé le sabbat. le septième jour, il le sanctifia pour tous les âges et l'établit comme signe de toutes ses œuvres.
\par 2 Car le premier jour, il a créé les cieux qui sont en haut et la terre et les eaux et tous les esprits qui servent devant lui - les anges de la présence, et les anges de la sanctification, et les anges [de l'esprit de feu et les anges] de l'esprit des vents, et les anges de l'esprit des nuages, et des ténèbres, et de la neige, et de la grêle et du givre, et les anges des voix, et du tonnerre et du éclairs, et les anges des esprits du froid et de la chaleur, et de l'hiver et du printemps et de l'automne et de l'été et de tous les esprits de ses créatures qui sont dans les cieux et sur la terre, (Il créa) les abîmes et les ténèbres, le soir (et la nuit), et la lumière, l'aube et le jour, qu'Il a préparées dans la connaissance de son cœur.
\par 3 Et là-dessus nous avons vu ses œuvres, et nous l'avons loué, et nous avons loué devant lui à cause de toutes ses œuvres ; car il créa sept grandes œuvres le premier jour.
\par 4 Et le deuxième jour, Il créa le firmament au milieu des eaux, et les eaux furent divisées ce jour-là - la moitié d'entre elles montèrent au-dessus et l'autre moitié descendit au-dessous du firmament (qui était) au milieu. sur la face de la terre entière. Et c'était la seule œuvre (Dieu) créée le deuxième jour.
\par 5 Et le troisième jour, il ordonna aux eaux de passer de la surface de toute la terre en un seul lieu, et à la terre ferme d'apparaître.
\par 6 Et les eaux firent ce qu'il leur avait commandé, et elles se retirèrent de la surface de la terre dans un seul lieu en dehors de ce firmament, et la terre ferme apparut.
\par 7 Et ce jour-là, Il créa pour eux toutes les mers selon leurs lieux de rassemblement séparés, et tous les fleuves, et les rassemblements d'eaux dans les montagnes et sur toute la terre, et tous les lacs, et tous les rosée de la terre, et les semences semées, et toutes choses qui poussent, et les arbres fruitiers, et les arbres des bois, et le jardin d'Éden, en Éden et toutes les plantes selon leur espèce.
\par 8 Ces quatre grandes œuvres que Dieu a créées le troisième jour. Et le quatrième jour, il créa le soleil, la lune et les étoiles, et les plaça dans le firmament du ciel, pour éclairer toute la terre, et pour régner sur le jour et la nuit, et pour séparer la lumière du ciel. obscurité.
\par 9 Et Dieu fit du soleil un grand signe sur la terre pour les jours et pour les sabbats et pour les mois et pour les fêtes et pour les années et pour les sabbats d'années et pour les jubilés et pour toutes les saisons des années.
\par 10 Et il sépare la lumière des ténèbres [et] pour la prospérité, afin que prospèrent toutes choses qui poussent et poussent sur la terre.
\par 11 Il fit ces trois sortes le quatrième jour. Et le cinquième jour, Il créa de grands monstres marins dans les profondeurs des eaux, car ce furent les premières choses de chair qui furent créées par ses mains, les poissons et tout ce qui bouge dans les eaux, et tout ce qui vole, les oiseaux et tous leurs genres.
\par 12 Et le soleil se leva au-dessus d'eux pour prospérer, et au-dessus de tout ce qui était sur la terre, de tout ce qui pousse de la terre, de tous les arbres fruitiers et de toute chair.
\par 13 Ces trois sortes, Il les créa le cinquième jour. Et le sixième jour, Il créa tous les animaux de la terre, tout le bétail et tout ce qui bouge sur la terre.
\par 14 Et après tout cela, il créa l'homme, un homme et une femme, il les créa, et lui donna domination sur tout ce qui est sur la terre et dans les mers, et sur tout ce qui vole, et sur les bêtes et sur le bétail, et sur tout ce qui bouge sur la terre, et sur toute la terre, et sur tout cela, il lui a donné la domination.
\par 15 Et ces quatre sortes, Il les créa le sixième jour. Et il y en avait en tout vingt-deux sortes.
\par 16 Et il acheva tout son ouvrage le sixième jour, tout ce qui est dans les cieux et sur la terre, et dans les mers et dans les abîmes, et dans la lumière et dans les ténèbres, et en tout.
\par 17 Et Il nous a donné un grand signe, le jour du sabbat, pour que nous travaillions six jours, mais que nous gardions le sabbat le septième jour pour tout travail.
\par 18 Et tous les anges de la présence et tous les anges de la sanctification, ces deux grandes classes, Il nous a ordonné d'observer le sabbat avec Lui dans le ciel et sur la terre.
\par 19 Et Il nous dit : Voici, je me séparerai d'un peuple parmi tous les peuples, et ceux-ci observeront le jour du sabbat, et je me les sanctifierai comme mon peuple, et je les bénirai ; comme j'ai sanctifié le jour du sabbat et que je me le sanctifie, de même je les bénirai, et ils seront mon peuple et je serai leur Dieu.'
\par 20 Et j'ai choisi la postérité de Jacob parmi tout ce que j'ai vu, et je l'ai écrit comme mon fils premier-né, et je l'ai sanctifié pour moi pour les siècles des siècles ; et je leur enseignerai le jour du sabbat, afin qu'ils observent le sabbat lors de tout travail.
\par 21 Et ainsi il créa là un signe selon lequel ils devraient observer le sabbat avec nous le septième jour, manger et boire, et bénir celui qui a créé toutes choses, comme il s'est béni et sanctifié pour lui-même. peuple au-dessus de tous les peuples, et qu'ils devraient observer le sabbat avec nous.
\par 22 Et il fit monter ses commandements comme une douce odeur agréable devant lui tous les jours. . .
\par 23 Il y eut vingt-deux têtes d'humanité depuis Adam jusqu'à Jacob, et vingt-deux sortes d'ouvrages furent exécutés jusqu'au septième jour ; c'est béni et saint ; et le premier est aussi béni et saint ; et celui-ci sert avec celui-là pour la sanctification et la bénédiction.
\par 24 Et à celui-ci (Jacob et sa postérité), il fut accordé qu'ils seraient toujours les bienheureux et les saints du premier témoignage et de la première loi, tout comme Il avait sanctifié et béni le jour du sabbat le septième jour.
\par 25 Il a créé le ciel et la terre et tout ce qu'il a créé en six jours, et Dieu a sanctifié le septième jour, pour toutes ses œuvres ; c'est pourquoi il a ordonné en sa faveur que quiconque y ferait un ouvrage mourrait, et que celui qui le souillerait mourrait sûrement.
\par 26 C'est pourquoi ordonne aux enfants d'Israël d'observer ce jour, afin qu'ils le sanctifient, qu'ils n'y fassent aucun ouvrage et qu'ils ne le souillent pas, car il est plus saint que tous les autres jours.
\par 27 Et quiconque le profanera mourra sûrement, et quiconque y fera un ouvrage quelconque mourra sûrement éternellement, afin que les enfants d'Israël puissent observer ce jour à travers leurs générations et ne soient pas déracinés du pays ; car c'est un jour saint et un jour béni.
\par 28 Et quiconque l'observe et observe le sabbat pendant toute son œuvre sera saint et béni à tout moment, comme nous.
\par 29 Déclarez et dites aux enfants d'Israël la loi de ce jour, soit qu'ils doivent observer le sabbat ce jour-là, et qu'ils ne doivent pas l'abandonner dans l'égarement de leur cœur ; (et) qu'il n'est pas permis d'y faire un travail inconvenant, d'y faire leur propre plaisir, et qu'ils ne doivent pas y préparer quoi que ce soit à manger ou à boire, et (qu'il n'est pas permis) d'y puiser de l'eau, ou apporter ou emporter par leurs portes tout fardeau qu'ils ne s'étaient pas préparés le sixième jour dans leurs habitations.
\par 30 Et ils n'apporteront ni ne sortiront de maison en maison ce jour-là ; car ce jour est plus saint et plus béni que n'importe quel jour de jubilé parmi les jubilés ; c'est pourquoi nous avons observé le sabbat dans les cieux avant qu'il soit fait connaître à toute chair d'observer le sabbat sur la terre.
\par 31 Et le Créateur de toutes choses l'a béni, mais il n'a pas sanctifié tous les peuples et toutes les nations pour qu'ils y observent le sabbat, mais seulement Israël : à eux seuls il a permis de manger et de boire et d'observer le sabbat sur la terre.
\par 32 Et le Créateur de toutes choses bénit ce jour qu'il avait créé pour la bénédiction, la sainteté et la gloire par-dessus tous les jours.
\par 33 Cette loi et ce témoignage ont été donnés aux enfants d'Israël comme une loi éternelle pour leurs générations.

\chapter{3}

\par \textit{Adam nomme toutes les créatures, 1-3. Création d'Ève et promulgation des lois lévitiques de purification, 4-14. Adam et Ève au Paradis : leur péché et leur expulsion, 15-29. Loi de couvrir sa honte édictée, 30-2. Adam et Ève habitent à Êldâ, 32-5. (Cf.Gen. ii.18-25, iii.)}

\par 1 Et les six jours de la deuxième semaine, nous avons amené, selon la parole de Dieu, à Adam toutes les bêtes, et tout le bétail, et tous les oiseaux, et tout ce qui bouge sur la terre, et tout ce qui bouge dans l'eau, selon leurs espèces et selon leurs types : les bêtes le premier jour ; le bétail le deuxième jour ; les oiseaux le troisième jour ; et tout ce qui bouge sur la terre le quatrième jour ; et celui qui bouge dans l'eau le cinquième jour.
\par 2 Et Adam les nomma tous par leurs noms respectifs, et comme il les appelait, ainsi était leur nom.
\par 3 Et pendant ces cinq jours, Adam vit tous ces hommes et femmes, selon toutes les espèces qui étaient sur la terre, mais il était seul et ne trouva aucune aide pour lui.
\par 4 Et le Seigneur nous a dit : 'Il n'est pas bon que cet homme soit seul : faisons-lui une aide.'
\par 5 Et l'Éternel notre Dieu fit tomber sur lui un profond sommeil, et il s'endormit, et il prit pour la femme une côte d'entre ses côtes, et cette côte était l'origine de la femme d'entre ses côtes, et il il a bâti la chair à sa place, et il a bâti la femme.
\par 6 Et Il réveilla Adam de son sommeil et au réveil il se leva le sixième jour, et Il la lui amena, et il la connut et lui dit : « Ceci est maintenant l'os de mes os et la chair de mes chair; elle sera appelée [ma] femme ; parce qu'elle a été enlevée à son mari.
\par 7 C'est pourquoi l'homme et la femme seront un et c'est pourquoi l'homme quittera son père et sa mère et s'attachera à sa femme, et ils seront une seule chair.
\par 8 La première semaine, Adam fut créé, et la côte fut sa femme ; la deuxième semaine, il la lui montra ; et pour cette raison, le commandement fut donné de garder dans leur souillure, pour un mâle sept jours, et pour une femelle deux fois sept jours.
\par 9 Et après qu'Adam eut accompli quarante jours dans le pays où il avait été créé, nous l'avons amené dans le jardin d'Eden pour le cultiver et le garder, mais ils ont amené sa femme le quatre-vingtième jour, et après cela elle est entrée dans le jardin d'Eden.
\par 10 Et c'est pour cette raison que le commandement est écrit sur les tablettes célestes concernant celle qui enfante : « Si elle enfante un mâle, elle restera dans son impureté sept jours selon la première semaine des jours, et trente-trois jours ». elle restera dans le sang de sa purification pendant plusieurs jours, et elle ne touchera à aucune chose sainte, ni n'entrera dans le sanctuaire, jusqu'à ce qu'elle ait accompli ces jours qui (sont enjoints) dans le cas d'un enfant mâle.'
\par 11 «Mais s'il s'agit d'une fille, elle restera dans son impureté deux semaines de jours, selon les deux premières semaines, et soixante-six jours dans le sang de sa purification, et ils seront en tout quatre-vingts jours». .'
\par 12 Et quand elle eut accompli ces quatre-vingts jours, nous l'amenâmes dans le jardin d'Eden, car il est plus saint que toute la terre et tout arbre qui y est planté est saint.
\par 13 C'est pourquoi il a été prescrit à celle qui enfantera un mâle ou une femelle, la loi de ces jours-là, qu'elle ne touchera à aucune chose sainte, ni n'entrera dans le sanctuaire, jusqu'à ce que ces jours pour l'enfant mâle ou femelle soient accomplis.
\par 14 Ceci est la loi et le témoignage qui ont été écrits pour Israël, afin qu'ils l'observent tous les jours.
\par 15 Et dans la première semaine du premier jubilé, [1-7 heures du matin] Adam et sa femme étaient dans le jardin d'Eden pendant sept ans à le cultiver et à l'entretenir, et nous lui avons donné du travail et nous lui avons demandé de faire tout ce qui était nécessaire. est adapté au travail du sol.
\par 16 Et il labourait (le jardin), et il était nu et ne le savait pas, et n'avait pas honte, et il protégeait le jardin des oiseaux, des bêtes et du bétail, et en récoltait les fruits, et en mangeait, et mettait de côté le reste. pour lui et pour sa femme [et mettre de côté ce qui était gardé].
\par 17 Et après l'accomplissement des sept années qu'il avait accomplies là, sept années exactement, [8 heures du matin] et le deuxième mois, le dix-septième jour (du mois), le serpent vint et s'approcha de la femme, Et le serpent dit à la femme : Dieu vous a-t-il ordonné de ne pas manger de tous les arbres du jardin ?
\par 18 Et elle lui dit : De tous les fruits des arbres du jardin, Dieu nous a dit : Mangez ; mais du fruit de l'arbre qui est au milieu du jardin, Dieu nous a dit : Vous n'en mangerez pas et vous n'y toucherez pas, de peur que vous ne mourriez.
\par 19 Et le serpent dit à la femme : Vous ne mourrez certainement pas ; car Dieu sait que le jour où vous en mangerez, vos yeux s'ouvriront, et vous serez comme des dieux, et vous connaîtrez le bien et le bien. mal.'
\par 20 Et la femme vit que l'arbre était agréable et agréable à la vue, et que son fruit était bon à manger, et elle en prit et mangea.
\par 21 Et quand elle eut d'abord couvert sa honte avec des feuilles de figuier, elle en donna à Adam et il mangea, et ses yeux s'ouvrirent, et il vit qu'il était nu.
\par 22 Et il prit des feuilles de figuier, les cousit ensemble, et se fit un tablier, et couvrit sa honte.
\par 23 Et Dieu maudit le serpent, et il fut irrité contre lui pour toujours. . .
\par 24 Et il fut irrité contre la femme, parce qu'elle avait écouté la voix du serpent et qu'elle mangeait ; et il lui dit : « Je multiplierai grandement ta tristesse et tes douleurs : dans la tristesse tu enfanteras des enfants, et ton retour sera vers ton mari, et il dominera sur toi.
\par 25 Et il dit aussi à Adam : Parce que tu as écouté la voix de ta femme et que tu as mangé de l'arbre dont je t'ai commandé de ne pas en manger, maudit soit le sol à cause de toi : les épines et les épines. il te donnera des chardons, et tu mangeras ton pain à la sueur de ton visage, jusqu'à ce que tu retournes à la terre d'où tu as été pris ; car tu es terre, et tu retourneras sur terre.
\par 26 Et Il leur fit des tuniques de peau, et les vêtit, et les fit sortir du jardin d'Eden.
\par 27 Et le jour où Adam sortit du Jardin, il offrit comme une douce odeur une offrande, de l'encens, du galbanum, du stacte et des épices le matin au lever du soleil depuis le jour où il couvrit son corps. honte.
\par 28 Et ce jour-là, la gueule de toutes les bêtes, du bétail, des oiseaux, de tout ce qui marche et de tout ce qui bouge, fut fermée, de sorte qu'ils ne pouvaient plus parler, car ils s'étaient tous parlé les uns aux autres avec une lèvre et une langue.
\par 29 Et Il fit sortir du jardin d'Eden toute chair qui était dans le jardin d'Eden, et toute chair fut dispersée selon ses espèces et selon ses types dans les lieux qui avaient été créés pour eux.
\par 30 Et il a donné à Adam seul (de quoi) pour couvrir sa honte, de toutes les bêtes et du bétail.
\par 31 C'est pourquoi il est prescrit sur les tablettes célestes, concernant tous ceux qui connaissent le jugement de la loi, qu'ils doivent couvrir leur honte et ne doivent pas se découvrir comme les païens se découvrent.
\par 32 Et à la nouvelle lune du quatrième mois, Adam et sa femme sortirent du jardin d'Eden, et ils habitèrent au pays d'Elda, dans le pays de leur création.
\par 33 Et Adam appela le nom de sa femme Ève.
\par 34 Et ils n'eurent pas de fils jusqu'au premier jubilé, [8 heures du matin] et après cela, il la connut.
\par 35 Maintenant il labourait la terre comme il avait été instruit dans le jardin d'Eden.

\chapter{4}

\par \textit{Caïn et Abel et les autres enfants d'Adam, 1-12. Enos, Kenan, Mahalalel, Jared, 13-15. Enoch et son histoire, 16-25. Quatre lieux sacrés, 26. Mathusalem, Lamec, Noé, 27, 28. Mort d'Adam et Caïn, 29-32. Sem, Cham et Japhet, 32. (Cf. Gen. iv-v.)}

\par 1 Et la troisième semaine du deuxième jubilé [64-70 AM] elle donna naissance à Caïn, et la quatrième [71-77 AM] elle enfanta Abel, et la cinquième [78-84 AM] elle a donné naissance à sa fille Âwân.
\par 2 Et la première (année) du troisième jubilé [99-105 AM], Caïn tua Abel parce que (Dieu) accepta le sacrifice d'Abel et n'accepta pas l'offrande de Caïn.
\par 3 Et il le tua dans les champs ; et son sang cria de la terre jusqu'au ciel, se plaignant de ce qu'il l'avait tué.
\par 4 Et l'Éternel réprimanda Caïn à cause d'Abel, parce qu'il l'avait tué, et il le rendit fugitif sur la terre à cause du sang de son frère, et il le maudit sur la terre.
\par 5 Et c'est pour cela qu'il est écrit sur les tables célestes : Maudit soit celui qui frappe traîtreusement son prochain, et que tous ceux qui ont vu et entendu disent : Ainsi soit-il ; et celui qui l'a vu et ne l'a pas déclaré, qu'il soit maudit comme l'autre.
\par 6 Et c'est pour cela que nous annonçons, lorsque nous nous présentons devant le Seigneur notre Dieu, tous les péchés qui se commettent dans le ciel et sur la terre, dans la lumière et dans les ténèbres, et partout.
\par 7 Et Adam et sa femme pleurèrent Abel pendant quatre semaines d'années, [99-127 AM] et la quatrième année de la cinquième semaine [130 AM] ils devinrent joyeux, et Adam connut de nouveau sa femme, et elle lui enfanta un fils, et il appela son nom Seth ; car il dit : « DIEU nous a suscité une seconde semence sur la terre à la place d'Abel ; car Caïn l'a tué.
\par 8 Et la sixième semaine [134-40 AM] il engendra sa fille Azûrâ.
\par 9 Et Caïn prit pour femme Awân, sa sœur, et elle lui enfanta Enoch à la fin du quatrième jubilé. [190-196 AM] Et la première année de la première semaine du cinquième jubilé, [197 AM] des maisons furent construites sur la terre, et Caïn bâtit une ville et lui donna le nom du nom de son fils Enoch.
\par 10 Et Adam connut Ève, sa femme, et elle enfanta encore neuf fils.
\par 11 Et dans la cinquième semaine du cinquième jubilé [225-31 AM] Seth prit Azûrâ sa sœur pour femme, et la quatrième (année de la sixième semaine) [235 AM] elle lui enfanta Enos.
\par 12 Il commença à invoquer le nom du Seigneur sur la terre.
\par 13 Et au septième jubilé de la troisième semaine [309-15 AM] Enos prit Nôâm sa sœur pour femme, et elle lui enfanta un fils la troisième année de la cinquième semaine, et il appela son nom Kenan.
\par 14 Et à la fin du huitième jubilé [325, 386-3992 AM] Kenan prit Mûalêlêth sa sœur pour femme, et elle lui enfanta un fils au neuvième jubilé, la première semaine de la troisième année de cette année. semaine, [395 AM] et il appela son nom Mahalalel.
\par 15 Et dans la deuxième semaine du dixième jubilé [449-55 AM] Mahalalel prit pour femme DinaH, la fille de Barakiel, fille du frère de son père, et elle lui enfanta un fils la troisième semaine de la sixième. année, [461 AM] et il appela son nom Jared, car en son temps les anges du Seigneur descendirent sur la terre, ceux qu'on appelle les Veilleurs, afin qu'ils instruisent les enfants des hommes, et qu'ils fassent jugement et la droiture sur la terre.
\par 16 Et au onzième jubilé [512-18 AM] Jared prit pour lui une femme, et son nom était Baraka, la fille de Râsûjâl, fille du frère de son père, dans la quatrième semaine de ce jubilé, [522 AM] ] et elle lui enfanta un fils la cinquième semaine, la quatrième année du jubilé, et il appela son nom Enoch.
\par 17 Et il fut le premier parmi les hommes nés sur la terre qui apprit l'écriture, la connaissance et la sagesse et qui écrivit les signes du ciel selon l'ordre de leurs mois dans un livre, afin que les hommes puissent connaître les saisons des années. selon l'ordre de leurs mois séparés.
\par 18 Et il fut le premier à écrire un témoignage et il rendit témoignage aux fils des hommes parmi les générations de la terre, et raconta les semaines des jubilés, et leur fit connaître les jours des années, et les mit en ordre. les mois et lui raconta les sabbats des années comme nous les lui faisions connaître.
\par 19 Et ce qui était et ce qu'il verra dans une vision de son sommeil, comme cela arrivera aux enfants des hommes à travers leurs générations jusqu'au jour du jugement ; il a tout vu et tout compris, et a écrit son témoignage, et a placé le témoignage sur terre pour tous les enfants des hommes et pour leurs générations.
\par 20 Et au douzième jubilé, [582-88] à la septième semaine de celui-ci, il prit pour lui une femme, et elle s'appelait Edna, la fille de Danel, la fille du frère de son père, et la sixième année cette semaine [587 AM], elle lui enfanta un fils et il l'appela Mathusalem.
\par 21 Et il fut en outre avec les anges de Dieu pendant six jubilés d'années, et ils lui montrèrent tout ce qui est sur la terre et dans les cieux, le règne du soleil, et il écrivit tout.
\par 22 Et il rendit témoignage aux veilleurs, qui avaient péché avec les filles des hommes ; car celles-ci avaient commencé à s'unir, au point d'être souillées, avec les filles des hommes, et Hénoc témoigna contre (elles) toutes.
\par 23 Et il a été pris parmi les enfants des hommes, et nous l'avons conduit dans le jardin d'Eden avec majesté et honneur, et voici, il y écrit la condamnation et le jugement du monde, et toute la méchanceté des enfants de Hommes.
\par 24 Et à cause de cela (Dieu) fit venir les eaux du déluge sur tout le pays d'Eden ; car là, il fut placé comme un signe et pour qu'il témoigne contre tous les enfants des hommes, pour qu'il raconte toutes les actions des générations jusqu'au jour de la condamnation.
\par 25 Et il brûla l'encens du sanctuaire, des aromates agréables devant l'Éternel sur la montagne.
\par 26 Car l'Éternel a quatre lieux sur la terre, le jardin d'Eden et la montagne de l'Orient, et cette montagne sur laquelle tu es aujourd'hui, le mont Sinaï et le mont Sion (qui) sera sanctifié dans le nouveau création pour une sanctification de la terre ; grâce à elle, la terre sera sanctifiée de toute (sa) culpabilité et de ses impuretés à travers les générations du monde.
\par 27 Et au quatorzième jubilé [652 AM] Mathusalem prit pour femme, Edna, fille d'Azrial, fille du frère de son père, dans la troisième semaine, la première année de cette semaine, [701-7 AM ] et il engendra un fils et il appela son nom Lémec.
\par 28 Et au quinzième jubilé, la troisième semaine, Lémec prit une femme, et son nom était Betenos, fille de Baraki'il, fille du frère de son père, et cette semaine-là, elle lui enfanta un fils et il appela son nom Noé, en disant : « Celui-ci me consolera de ma peine et de tout mon travail, et pour le sol que l'Éternel a maudit. »
\par 29 Et à la fin du dix-neuvième jubilé, la septième semaine de la sixième année [930 AM] de celui-ci, Adam mourut, et tous ses fils l'enterra dans le pays de sa création, et il fut le premier à être enterré. dans la terre.
\par 30 Et il lui manquait soixante-dix ans sur mille ans ; car mille ans sont comme un jour dans le témoignage des cieux et c'est pourquoi il a été écrit à propos de l'arbre de la connaissance : « Le jour où vous en mangerez, vous mourrez ». C'est pour cette raison qu'il n'a pas accompli les années de ce jour ; car il est mort pendant cela.
\par 31 A la fin de ce jubilé, Caïn fut tué après lui la même année ; car sa maison tomba sur lui et il mourut au milieu de sa maison, et il fut tué par ses pierres ; car avec une pierre il avait tué Abel, et par une pierre il fut tué dans un juste jugement.
\par 32 C'est pour cette raison qu'il a été ordonné sur les tablettes célestes : « Avec l'instrument avec lequel un homme tue son prochain, il sera tué ; de la manière dont il l'a blessé, ils le traiteront de la même manière.
\par 33 Et au vingt-cinquième [1205 AM] jubilé, Noé prit pour lui une femme, et son nom était \`Emzârâ, la fille de Râkê'êl, la fille du frère de son père, la première année du cinquième semaine [1207 AM] : et la troisième année elle lui enfanta Sem, la cinquième année [1209 AM] elle lui enfanta Cham, et la première année de la sixième semaine [1212 AM] elle lui enfanta Japhet.

\chapter{5}

\par \textit{Les Anges de Dieu épousent les filles des hommes, 1. Corruption de toute la création, 2-3. Punition des anges déchus et de leurs enfants, 4-9a. Jugement final annoncé, 9b-16. Jour des Expiations, 17-18. Le déluge annoncé, Noé construit l'arche, le déluge, 19-32. (Cf. Gen.vi-viii.19.)}

\par 1 Et il arriva que lorsque les enfants des hommes commencèrent à se multiplier sur la face de la terre et que des filles leur naquirent, que les anges de Dieu les virent une certaine année de ce jubilé, qu'ils étaient beaux à regarder. sur; et ils prirent pour femmes parmi toutes celles qu'ils choisirent, et ils leur enfantèrent des fils et elles furent des géants.
\par 2 Et l'iniquité s'est accrue sur la terre et toute chair a corrompu ses voies, aussi bien les hommes que le bétail et les bêtes et les oiseaux et tout ce qui marche sur la terre - tous ont corrompu leurs voies et leurs ordres, et ils ont commencé à se dévorer les uns les autres, et l'iniquité augmentait sur la terre et toute imagination des pensées de tous les hommes (était) ainsi continuellement mauvaise.
\par 3 Et Dieu regarda la terre, et voici, elle était corrompue, et toute chair avait corrompu ses ordres, et tous ceux qui étaient sur la terre avaient fait toutes sortes de mal sous ses yeux.
\par 4 Et Il dit qu'Il détruirait l'homme et toute chair sur la face de la terre qu'Il avait créée.
\par 5 Mais Noé trouva grâce devant les yeux du Seigneur.
\par 6 Et contre les anges qu'il avait envoyés sur la terre, il fut extrêmement en colère, et il donna l'ordre de les déraciner de toute leur domination, et il nous ordonna de les lier dans les profondeurs de la terre, et voici, ils sont liés au milieu d'eux et sont (gardés) séparés.
\par 7 Et contre leurs fils un ordre sortit devant sa face : ils devaient être frappés par l'épée et enlevés de dessous le ciel.
\par 8 Et Il dit : « Mon esprit ne demeurera pas toujours sur l'homme ; car eux aussi sont chair et leurs jours seront de cent vingt ans.
\par 9 Et Il envoya Son épée au milieu d'eux pour que chacun tue son prochain, et ils commencèrent à s'entre-tuer jusqu'à ce qu'ils tombèrent tous par l'épée et furent détruits de la terre.
\par 10 Et leurs pères furent témoins (de leur destruction), et après cela ils furent liés dans les profondeurs de la terre pour toujours, jusqu'au jour de la grande condamnation, où le jugement sera exécuté sur tous ceux qui ont corrompu leurs voies et leurs œuvres devant le Seigneur.
\par 11 Et il les détruisit tous de leurs lieux, et il n'en resta pas un seul qu'il ne jugeât selon toute sa méchanceté.
\par 12 Et il a créé pour toutes ses œuvres une nature nouvelle et juste, afin qu'ils ne pèchent pas dans toute leur nature pour toujours, mais qu'ils soient tous justes, chacun selon son espèce, à tout moment.
\par 13 Et le jugement de tous est ordonné et écrit avec justice sur les tablettes célestes, même (le jugement de) tous ceux qui s'écartent du chemin qui leur est ordonné de marcher ; et s'ils n'y marchent pas, le jugement est écrit pour toute créature et pour toute espèce.
\par 14 Et il n'y a rien dans le ciel, ni sur la terre, ni dans la lumière, ni dans les ténèbres, ni dans le shéol, ni dans les profondeurs, ni dans le lieu des ténèbres (qui ne soit jugé) ; et tous leurs jugements sont ordonnés, écrits et gravés.
\par 15 Il jugera de tous, le grand selon sa grandeur, et le petit selon sa petitesse, et chacun selon sa voie.
\par 16 Et Il n'est pas quelqu'un qui s'occupera de la personne (d'aucun), ni celui qui recevra des cadeaux, s'Il dit qu'Il exécutera le jugement sur chacun : si quelqu'un a donné tout ce qui est sur la terre, Il le fera. ne faites pas attention aux cadeaux ou à la personne (d'aucun), et n'acceptez rien de sa main, car Il est un juge juste.
\par 17 [Et des enfants d'Israël, il a été écrit et ordonné : S'ils se tournent vers lui avec justice, il pardonnera toutes leurs transgressions et pardonnera tous leurs péchés.
\par 18 Il est écrit et ordonné qu'Il fera miséricorde à tous ceux qui se détournent de toute leur culpabilité une fois par an.]
\par 19 Et quant à tous ceux qui ont corrompu leurs voies et leurs pensées avant le déluge, personne n'a été accepté, sauf celui de Noé seul ; car sa personne a été acceptée au nom de ses fils, que (Dieu) a sauvés des eaux du déluge à cause de lui ; car son cœur était juste dans toutes ses voies, selon ce qui lui avait été ordonné, et il ne s'était pas écarté de tout ce qui lui était ordonné.
\par 20 Et l'Éternel dit qu'il détruirait tout ce qui était sur la terre, tant les hommes que le bétail, et
\par 21 les bêtes et les oiseaux du ciel, et tout ce qui se meut sur la terre. Et il ordonna à Noé de lui faire une arche, afin qu'il puisse se sauver des eaux du déluge.
\par 22 Et Noé fit l'arche en tous points comme il le lui avait commandé, au vingt-septième jubilé des années, la cinquième semaine de la cinquième année (à la nouvelle lune du premier mois). [13h07]
\par 23 Et il entra la sixième (année), [1308 AM] le deuxième mois, à la nouvelle lune du deuxième mois, jusqu'au seizième ; et il entra, avec tout ce que nous lui avions apporté, dans l'arche, et le Seigneur la ferma de l'extérieur le dix-septième soir.
\par 24 Et l'Éternel ouvrit les sept écluses du ciel,  
\par     Et les bouches des fontaines du grand abîme, au nombre de sept bouches.
\par 25 Et les écluses commencèrent à faire couler l'eau du ciel pendant quarante jours et quarante nuits,  
\par     Et les sources des profondeurs firent jaillir aussi des eaux, jusqu'à ce que le monde entier soit rempli d'eau.
\par 26 Et les eaux augmentèrent sur la terre :  
\par     Les eaux s'élevaient de quinze coudées au-dessus de toutes les hautes montagnes,  
\par     Et l'arche s'élevait au-dessus de la terre,  
\par     Et il se déplaçait à la surface des eaux.
\par 27 Et l'eau régna sur la surface de la terre pendant cinq mois, cent cinquante jours.
\par 28 Et l'arche alla et s'arrêta au sommet de Lubar, une des montagnes d'Ararat.
\par 29 Et (à la nouvelle lune) au quatrième mois, les sources du grand abîme furent fermées et les écluses du ciel furent fermées ; et à la nouvelle lune du septième mois, toutes les bouches des abîmes de la terre s'ouvrirent, et l'eau commença à descendre dans les profondeurs.
\par 30 Et à la nouvelle lune du dixième mois, les sommets des montagnes furent visibles, et à la nouvelle lune du premier mois, la terre devint visible.
\par 31 Et les eaux disparurent du dessus de la terre la cinquième semaine de la septième année [1309 AM], et le dix-septième jour du deuxième mois, la terre était sèche.
\par 32 Et le vingt-septième jour, il ouvrit l'arche et en fit sortir des bêtes, du bétail, des oiseaux et tout ce qui bouge.

\chapter{6}

\par \textit{Sacrifice de Noé, 1-3 (cf. Gen. vii.20-2). Alliance de Dieu avec Noé, consommation de sang interdite, 4-10 (cf. Gen. ix. 1-17). Moïse a ordonné de renouveler cette loi contre la consommation de sang, 11-14. Arc placé dans les nuages ​​pour un signe, 15-16. Fête des semaines instituée, histoire de ses observances, 17-22. Fête des nouvelles lunes, 23-8. Division de l'année en 364 jours, 29-38.}

\par 1 Et à la nouvelle lune du troisième mois, il sortit de l'arche et bâtit un autel sur cette montagne.
\par 2 Et il fit l'expiation pour la terre, et prit un chevreau et fit l'expiation par son sang pour toute la culpabilité de la terre ; car tout ce qui s'y trouvait avait été détruit, sauf ce qui était dans l'arche avec Noé.
\par 3 Et il en plaça la graisse sur l'autel, et il prit un bœuf, et une chèvre, et un mouton et des chevreaux, et du sel, et une tourterelle, et des petits de colombe, et il plaça un holocauste. sur l'autel, et il versa dessus une offrande mélangée à de l'huile, et aspergea le vin et répandit de l'encens sur tout, et fit naître une bonne odeur, agréable devant l'Éternel.
\par 4 Et l'Éternel sentit la bonne odeur, et il fit alliance avec lui qu'il n'y aurait plus de déluge pour détruire la terre ; que tous les jours de la terre, le temps des semailles et de la récolte ne devraient jamais cesser ; le froid et la chaleur, l'été et l'hiver, le jour et la nuit ne devraient pas changer d'ordre ni cesser pour toujours.
\par 5 'Et vous, multipliez-vous et multipliez-vous sur la terre, et devenez nombreux sur elle, et soyez une bénédiction sur elle.' J'inspirerai votre crainte et votre crainte dans tout ce qui est sur terre et dans la mer.'
\par 6 'Et voici, je vous ai donné toutes les bêtes, et tous les oiseaux, et tout ce qui se meut sur la terre, et les poissons dans les eaux, et tout ce qui vous sert de nourriture ; comme les herbes vertes, je vous ai donné tout à manger.
\par 7 « Mais vous ne mangerez pas de chair, ni de vie, ni de sang ; car la vie de toute chair est dans le sang, de peur que le sang de vos vies ne soit requis. De la main de chaque homme, de la main de chaque (bête), je demanderai le sang de l'homme.
\par 8 «Quiconque verse le sang de l'homme par l'homme, son sang sera versé, car il a été fait homme à l'image de Dieu.»
\par 9 'Et vous, multipliez-vous et multipliez-vous sur la terre.'
\par 10 Et Noé et ses fils jurèrent de ne manger du sang d'aucune chair, et il fit une alliance devant l'Éternel Dieu pour toujours, pour toutes les générations de la terre, au cours de ce mois.
\par 11 C'est pourquoi il t'a dit que tu ferais ce mois-ci une alliance avec les enfants d'Israël sur la montagne, avec un serment, et que tu leur aspergerais de sang à cause de toutes les paroles de l'alliance que les Seigneur a fait avec eux pour toujours.
\par 12 Et ce témoignage est écrit à votre sujet afin que vous l'observiez continuellement, afin que vous ne mangiez à aucun jour du sang de bêtes, d'oiseaux ou de bétail pendant tous les jours de la terre, et que l'homme qui mange le sang de bête ou de bétail ou d'oiseaux pendant tous les jours de la terre, lui et sa postérité seront arrachés du pays.
\par 13 Et tu ordonneras aux enfants d'Israël de ne pas manger de sang, afin que leurs noms et leurs postérité soient continuellement devant l'Éternel notre Dieu.
\par 14 Et pour cette loi, il n'y a pas de limite de jours, car elle est éternelle. Ils le feront de génération en génération, afin qu'ils continuent à invoquer pour toi avec du sang devant l'autel ; chaque jour, à chaque heure du matin et du soir, ils chercheront perpétuellement ton pardon devant l'Éternel, afin de le garder et de ne pas être déracinés.
\par 15 Et Il donna à Noé et à ses fils un signe pour qu'il n'y ait plus de déluge sur la terre.
\par 16 Il a placé son arc dans la nuée en signe de l'alliance éternelle selon laquelle il n'y aurait plus de déluge sur la terre pour la détruire tous les jours de la terre.
\par 17 C'est pourquoi il est ordonné et écrit sur les tablettes célestes qu'ils doivent célébrer la fête des semaines en ce mois une fois par an, afin de renouveler l'alliance chaque année.
\par 18 Et toute cette fête fut célébrée dans le ciel depuis le jour de la création jusqu'aux jours de Noé - vingt-six jubilés et cinq semaines d'années [1309-1659 AM] : et Noé et ses fils l'observèrent pendant sept jubilés et un semaine d'années, jusqu'au jour de la mort de Noé, et depuis le jour de la mort de Noé, ses fils l'ont détruit jusqu'aux jours d'Abraham, et ils ont mangé du sang.
\par 19 Mais Abraham l'a observé, et Isaac et Jacob et ses enfants l'ont observé jusqu'à tes jours, et de ton temps les enfants d'Israël l'ont oublié jusqu'à ce que vous le célébriez de nouveau sur cette montagne.
\par 20 Et tu ordonneras aux enfants d'Israël d'observer cette fête dans toutes leurs générations, c'est un commandement pour eux : un jour de l'année, au cours de ce mois, ils célébreront la fête.
\par 21 Car c'est la fête des semaines et la fête des prémices : cette fête est double et de double nature : selon ce qui est écrit et gravé à son sujet, célébrez-la.
\par 22 Car j'ai écrit dans le livre de la première loi, dans celui que j'ai écrit pour toi, que tu la célébreras en son temps, un jour de l'année, et je t'ai expliqué ses sacrifices que les enfants de Israël devrait s'en souvenir et le célébrer à travers ses générations au cours de ce mois, un jour par an.
\par 23 Et à la nouvelle lune du premier mois, et à la nouvelle lune du quatrième mois, et à la nouvelle lune du septième mois, et à la nouvelle lune du dixième mois, ce sont les jours de souvenir, et le jours des saisons dans les quatre divisions de l'année. Ceux-ci sont écrits et ordonnés comme témoignage pour toujours.
\par 24 Et Noé les a ordonnés pour lui-même comme des fêtes pour les générations à jamais, de sorte qu'ils sont devenus ainsi un mémorial pour lui.
\par 25 Et à la nouvelle lune du premier mois, il lui fut ordonné de se faire une arche, et ce jour-là, la terre devint sèche et il ouvrit (l'arche) et vit la terre.
\par 26 Et à la nouvelle lune du quatrième mois, les bouches des profondeurs de l'abîme en bas étaient fermées. Et à la nouvelle lune du septième mois, toutes les bouches des abîmes de la terre s'ouvrirent, et les eaux commencèrent à y descendre.
\par 27 Et à la nouvelle lune du dixième mois, les sommets des montagnes apparurent, et Noé se réjouit.
\par 28 Et c'est pour cette raison qu'il les a ordonnés pour lui-même comme des fêtes en mémoire pour toujours, et c'est ainsi qu'ils sont ordonnés.
\par 29 Et ils les placèrent sur les tablettes célestes, chacune ayant treize semaines ; de l'un à l'autre (passé) leur mémorial, du premier au deuxième, et du deuxième au troisième, et du troisième au quatrième.
\par 30 Et tous les jours du commandement seront cinquante-deux semaines de jours, et (ceux-ci complèteront) l'année entière. Ainsi il est gravé et ordonné sur les tablettes célestes.
\par 31 Et il n'y a pas de négligence (ce commandement) pendant une seule année ou d'année en année.
\par 32 Et commande aux enfants d'Israël qu'ils observent les années selon ce calcul - trois cent soixante-quatre jours, et (ceux-ci) constitueront une année complète, et ils ne perturberont pas son temps de ses jours et de ses fêtes ; car tout se passera en eux selon leur témoignage, et ils ne laisseront de côté aucun jour ni ne troubleront aucune fête.
\par 33 Mais s'ils les négligent et ne les observent pas selon Son commandement, alors ils perturberont toutes leurs saisons et les années seront délogées de cet (ordre), [et ils perturberont les saisons et les années seront délogées ] et ils négligeront leurs ordonnances.
\par 34 Et tous les enfants d'Israël oublieront et ne trouveront pas le chemin des années, et oublieront les nouvelles lunes, les saisons et les sabbats et ils se tromperont quant à tout l'ordre des années.
\par 35 Car je le sais, et désormais je te l'annoncerai, et ce n'est pas de ma propre volonté ; car le livre (mensonge) écrit devant moi, et sur les tablettes célestes la division des jours est ordonnée, de peur qu'ils n'oublient les fêtes de l'alliance et ne marchent selon les fêtes des païens après leur erreur et après leur ignorance.
\par 36 Car il y aura certainement ceux qui feront des observations de la lune, comment (elle) perturbe les saisons et arrive d'année en année dix jours trop tôt.
\par 37 C'est pourquoi les années viendront sur eux où ils perturberont (l'ordre), et feront un (jour) abominable le jour du témoignage, et un jour impur un jour de fête, et ils confondront tous les jours, le saint avec les impurs, et le jour impur avec les saints ; car ils se tromperont quant aux mois, aux sabbats, aux fêtes et aux jubilés.
\par 38 C'est pourquoi je te commande et je te témoigne que tu puisses leur témoigner ; car après ta mort, tes enfants les dérangeront, de sorte qu'ils ne feront pas l'année trois cent soixante-quatre jours seulement, et pour cette raison ils se tromperont quant aux nouvelles lunes, aux saisons, aux sabbats et aux fêtes, et ils mangeront toute sorte de sang avec toute sorte de chair.

\chapter{7}

\par \textit{Noé plante une vigne et offre un sacrifice, 1-5. S'enivre et expose sa personne, 6-9. La malédiction de Canaan et la bénédiction de Sem et Japeth, 10-12 (cf. Gen. ix.20-8). Les fils et petits-fils de Noé et leurs villes, 13-19. Noé enseigne à ses fils les causes du déluge et les exhorte à éviter de manger du sang et de tuer, à observer la loi concernant les arbres fruitiers et à laisser la terre en jachère tous les sept ans, comme Enoch l'avait ordonné, 20-39.}

\par 1 Et la septième semaine de la première année [1317 AM] de ce jubilé, Noé planta des vignes sur la montagne sur laquelle l'arche avait reposé, nommée Lubar, une des montagnes d'Ararat, et elles produisirent des fruits dans le la quatrième année, [1320 AM] et il garda leurs fruits, et les récolta cette année-là au septième mois.
\par 2 Et il en fit du vin, et le mit dans un vase, et le garda jusqu'à la cinquième année, [1321 AM] jusqu'au premier jour, à la nouvelle lune du premier mois.
\par 3 Et il célébra avec joie le jour de cette fête, et il fit un holocauste à l'Éternel, un jeune bœuf et un bélier, et sept moutons âgés chacun d'un an, et un chevreau, afin qu'il puisse faire ainsi l'expiation pour lui et ses fils.
\par 4 Et il prépara d'abord le chevreau, et il mit un peu de son sang sur la chair qui était sur l'autel qu'il avait fait, et il déposa toute la graisse sur l'autel où il faisait l'holocauste, et le bœuf et le le bélier et les brebis, et il déposa toute leur chair sur l'autel.
\par 5 Et il y déposa toutes leurs offrandes mélangées à de l'huile, puis il aspergea du vin sur le feu qu'il avait préalablement allumé sur l'autel, et il plaça de l'encens sur l'autel et fit monter une douce odeur agréable devant l'Éternel. son Dieu.
\par 6 Et il se réjouit et but de ce vin, lui et ses enfants avec joie.
\par 7 Et c'était le soir, et il entra dans sa tente, et étant ivre, il se coucha et dormit, et se découvrit dans sa tente pendant qu'il dormait.
\par 8 Et Cham vit Noé, son père, nu, et sortit et le informa dehors de ses deux frères.
\par 9 Et Sem prit son vêtement et se leva, lui et Japhet, et ils mirent le vêtement sur leurs épaules et allèrent en arrière et couvrirent la honte de leur père, et leurs visages étaient en arrière.
\par 10 Et Noé se réveilla de son sommeil et comprit tout ce que son plus jeune fils lui avait fait, et il maudit son fils et dit : « Maudit soit Canaan ! il sera un esclave pour ses frères.
\par 11 Et il bénit Sem, et dit : 'Béni soit le Seigneur, le Dieu de Sem, et Canaan sera son serviteur.'
\par 12 'Dieu agrandira Japhet, et Dieu habitera dans la demeure de Sem, et Canaan sera son serviteur.'
\par 13 Et Cham savait que son père avait maudit son plus jeune fils, et il était mécontent d'avoir maudit son fils. et il se sépara de son père, lui et ses fils avec lui, Cush et Mizraïm et Put et Canaan.
\par 14 Et il se bâtit une ville et lui donna le nom du nom de sa femme Ne'elatama'uk.
\par 15 Et Japhet le vit, et devint jaloux de son frère, et lui aussi se bâtit une ville, et il lui donna le nom du nom de sa femme 'Adataneses.
\par 16 Et Sem habita avec son père Noé, et il bâtit une ville près de son père sur la montagne, et lui aussi appela son nom d'après le nom de sa femme Sedeqetelebab.
\par 17 Et voici, ces trois villes sont près du mont Lubar ; Sedeqetelebab faisant face à la montagne à l'est ; et Na'eltama'uk au sud ; 'Adatan'es vers l'ouest.
\par 18 Et voici les fils de Sem : Élam, Assur, et Arpachshad – ce (fils) est né deux ans après le déluge – et Lud, et Aram.
\par 19 Les fils de Japhet : Gomer et Magog et Madaï et Javan, Tubal et Méschec et Tiras : ce sont les fils de Noé.
\par 20 Et au vingt-huitième jubilé [1324-1372 AM] Noé commença à imposer aux fils de ses fils les ordonnances et les commandements, et tous les jugements qu'il connaissait, et il exhorta ses fils à observer la justice et à se couvrir. la honte de leur chair, et pour bénir leur Créateur, et honorer leur père et leur mère, et aimer leur prochain, et protéger leurs âmes de la fornication, de l'impureté et de toute iniquité.
\par 21 Car c'est à cause de ces trois choses que le déluge est arrivé sur la terre, à savoir à cause de la fornication dans laquelle les observateurs, contre la loi de leurs ordonnances, se prostituaient après les filles des hommes et prenaient femmes parmi toutes celles qu'ils voulaient : et ils firent le commencement de l'impureté.
\par 22 Et ils engendrèrent des fils, les Naphidim, et ils étaient tous différents, et ils se dévorèrent les uns les autres ; et les géants tuèrent les Naphil, et les Naphil tuèrent les Eljo, et les humains Eljo, et les uns les autres.
\par 23 Et chacun se vendit pour commettre l'iniquité et pour verser beaucoup de sang, et la terre fut remplie d'iniquité.
\par 24 Et après cela, ils péchèrent contre les bêtes et les oiseaux, et contre tout ce qui bouge et marche sur la terre ; et beaucoup de sang fut versé sur la terre, et toute imagination et tout désir des hommes imaginait continuellement la vanité et le mal.
\par 25 Et l'Éternel détruisit tout sur la surface de la terre ; à cause de la méchanceté de leurs actes et à cause du sang qu'ils avaient versé au milieu de la terre, il détruisit tout.
\par 26 Et nous sommes restés, moi et vous, mes fils, et tout ce qui était entré avec nous dans l'arche, et voici, je vois vos œuvres devant moi, afin que vous ne marchiez pas dans la justice : car vous avez commencé sur le chemin de la destruction. marcher, et vous vous séparez les uns des autres, et vous enviez les uns les autres, et (c'est ce qui arrive) que vous n'êtes pas d'accord, mes fils, chacun avec son frère.
\par 27 Car je vois, et voici, les démons ont commencé (leurs) séductions contre vous et contre vos enfants et maintenant je crains pour vous qu'après ma mort vous ne versiez le sang des hommes sur la terre, et que vous, aussi, sera détruit de la surface de la terre.
\par 28 Car quiconque verse le sang d'un homme, et quiconque mange le sang d'une chair quelconque, tous seront détruits de la terre.
\par 29 Et il ne restera aucun homme qui mange du sang, ou qui verse le sang des hommes sur la terre, et il ne lui restera pas non plus de postérité ou de descendance vivant sous le ciel ; Car ils entreront dans le shéol, et ils descendront dans le lieu de condamnation, et dans les ténèbres de l'abîme ils seront tous emportés par une mort violente.
\par 30 On ne verra pas de sang sur vous de tout le sang qu'il y aura tous les jours où vous aurez tué des bêtes ou du bétail ou tout ce qui vole sur la terre, et faites une bonne œuvre pour vos âmes en couvrant ce qui a été répandu sur la surface de la terre.
\par 31 Et vous ne serez pas comme celui qui mange avec du sang, mais prenez garde à ce que personne ne mange du sang devant vous : couvrez le sang, car ainsi m'a-t-il été commandé de témoigner envers vous et vos enfants, ainsi que toute chair.
\par 32 Et ne permettez pas que l'âme soit mangée avec la chair, afin que votre sang, qui est votre vie, ne soit pas exigé par aucune chair qui le répand sur la terre.
\par 33 Car la terre ne sera pas pure du sang qui a été répandu sur elle ; car (seulement) par le sang de celui qui l'a versé, la terre sera purifiée à travers toutes ses générations.
\par 34 Et maintenant, mes enfants, écoutez : pratiquez le jugement et la justice, afin que vous puissiez semer la justice sur toute la surface de la terre, et que votre gloire soit élevée devant mon Dieu, qui m'a sauvé des eaux du déluge.
\par 35 Et voici, vous allez vous bâtir des villes, et y planter toutes les plantes qui sont sur la terre, et surtout tous les arbres fruitiers.
\par 36 Pendant trois ans, les fruits de tout ce qui est mangé ne seront pas récoltés ; et la quatrième année, leurs fruits seront considérés comme saints [et ils offriront les prémices], agréables devant le Dieu Très-Haut, qui a créé les cieux. et la terre et toutes choses. Qu'ils offrent en abondance les prémices du vin et de l'huile (comme) prémices sur l'autel du Seigneur, celui qui le reçoit, et que ce qui reste, que les serviteurs de la maison du Seigneur le mangent devant l'autel qui le reçoit. ).
\par 37 Et la cinquième année, faites la dissémination afin que vous la libériez avec justice et droiture, et vous serez justes, et tout ce que vous plantez prospérera.
\par 38 Car ainsi Hénoc, le père de ton père, a ordonné à Mathusalem, son fils, et à Mathusalem, son fils, Lamec, et Lémec m'a ordonné tout ce que ses pères lui avaient commandé.
\par 39 Et je vous donnerai aussi des commandements, mes fils, comme Hénoc l'a ordonné à son fils lors des premiers jubilés : alors qu'il était encore vivant, le septième de sa génération, il a commandé et témoigné à son fils et aux fils de son fils jusqu'au jour de sa mort.

\chapitre{8}

\par \textit{Kâinâm découvre une inscription relative au soleil et aux étoiles, 1-4. Ses fils, 5-8 ans. Les fils de Noé et Noé se partagent la terre, 10-11. L'héritage de Sem, 12-21 : celui de Cham, 22-4 : celui de Japhet, 25-30. (Cf. Gén. x.)}

\par 1 Au vingt-neuvième jubilé, la première semaine, [1373 AM] au début de celui-ci, Arpachshad prit pour lui une femme et son nom était Rasu'eja, la fille de Susan, la fille d'Elam, et elle enfanta lui un fils en troisième année cette semaine, [1375 AM] et il appela son nom Kainam.
\par 2 Et le fils grandit, et son père lui apprit à écrire, et il partit chercher un endroit où il pourrait s'emparer d'une ville.
\par 3 Et il trouva une écriture que les (générations) précédentes avaient gravée sur le rocher, et il lut ce qu'il y avait dessus, et il la transcrivit et pécha à cause d'elle ; car il contenait l'enseignement des Veilleurs selon lequel ils observaient les présages du soleil, de la lune et des étoiles dans tous les signes du ciel.
\par 4 Et il l'écrivit et ne dit rien à ce sujet ; car il avait peur d'en parler à Noé, de peur qu'il ne se fâche contre lui à cause de cela.
\par 5 Et au trentième jubilé, [1429 AM] dans la deuxième semaine, la première année de celle-ci, il prit pour lui une femme, et elle s'appelait Melka, la fille de Madaï, le fils de Japhet, et dans le la quatrième année [1432 AM], il engendra un fils et appela son nom Shelah ; car il dit : « En vérité, j'ai été envoyé. »
\par 6 [Et il naquit la quatrième année], et Schéla grandit et prit pour lui une femme, et son nom était Mu'ak, la fille de Kesed, le frère de son père, au un et trentième jubilé, en la cinquième semaine, la première année [1499 AM] de celle-ci.
\par 7 Et elle lui enfanta un fils la cinquième année [1503 AM], et il appela son nom Eber; et il prit pour lui une femme, et son nom était 'Azûrâd, fille de Nébrod, au trente-septième siècle. deuxième jubilé, la septième semaine, la troisième année. [1564 AM]
\par 8 Et la sixième année [1567 AM], elle lui enfanta un fils, et il appela son nom Peleg ; car à l'époque où il naquit, les enfants de Noé commencèrent à se partager la terre : c'est pour cette raison qu'il appela son nom Peleg.
\par 9 Et ils le partagèrent secrètement entre eux, et le rapportèrent à Noé.
\par 10 Et il arriva qu'au début du trente-troisième jubilé [1569 AM] ils divisèrent la terre en trois parties, pour Sem, Cham et Japhet, selon l'héritage de chacun, la première année du la première semaine, quand l'un de nous qui avait été envoyé, était avec eux.
\par 11 Et il appela ses fils, et ils s'approchèrent de lui, eux et leurs enfants, et il partagea la terre en lots, dont ses trois fils devaient prendre possession, et ils étendirent leurs mains et prirent le écrivant du sein de Noé, leur père.
\par 12 Et il ressortit sur l'écriture comme sort pour Sem le milieu de la terre qu'il devait prendre en héritage pour lui et pour ses fils pour les générations de l'éternité, du milieu de la chaîne de montagnes de Rafa, de la bouche de l'eau de la rivière Tina, et sa partie va vers l'ouest à travers le milieu de cette rivière, et elle s'étend jusqu'à atteindre l'eau des abîmes, d'où cette rivière sort et verse ses eaux dans la mer. à, et ce fleuve se jette dans la grande mer. Et tout ce qui est vers le nord appartient à Japhet, et tout ce qui est vers le sud appartient à Sem.
\par 13 Et il s'étend jusqu'à atteindre Karaso : c'est au sein de la langue qui regarde vers le sud.
\par 14 Et sa partie s'étend le long de la grande mer, et elle s'étend en ligne droite jusqu'à ce qu'elle atteigne l'ouest de la langue qui regarde vers le sud : car cette mer est appelée la langue de la mer d'Égypte.
\par 15 Et d'ici elle tourne vers le sud, vers l'embouchure de la grande mer, sur le bord de (ses) eaux, et elle s'étend à l'ouest jusqu'à 'Afra, et elle s'étend jusqu'à atteindre les eaux du fleuve Gihon. et au sud des eaux de Gihon, jusqu'aux rives de ce fleuve.
\par 16 Et il s'étend vers l'est, jusqu'à atteindre le jardin d'Eden, au sud de celui-ci, [au sud] et depuis l'est de tout le pays d'Eden et de tout l'est, il se tourne vers l'est et il continue jusqu'à atteindre l'est de la montagne nommée Rafa, et descend jusqu'à la rive de l'embouchure de la rivière Tina.
\par 17 Cette part fut tirée au sort pour Sem et ses fils, afin qu'ils la possèdent pour toujours, pour ses générations, pour toujours.
\par 18 Et Noé se réjouit de ce que cette part soit sortie pour Sem et pour ses fils, et il se souvint de tout ce qu'il avait dit de sa bouche dans la prophétie ; car il avait dit : « Béni soit le Seigneur Dieu de Sem et que le Seigneur habite dans la demeure de Sem. »
\par 19 Et il savait que le jardin d'Éden est le saint des saints et la demeure de l'Éternel, et le mont Sinaï le centre du désert, et le mont Sion le centre du nombril de la terre : ces trois-là ont été créés comme des lieux saints se faisant face.
\par 20 Et il bénit le Dieu des dieux, qui avait mis la parole de l'Éternel dans sa bouche, et l'Éternel pour toujours.
\par 21 Et il savait qu'une part bénie et une bénédiction étaient venues à Sem et à ses fils pour les générations à jamais, tout le pays d'Eden et tout le pays de la mer Rouge, et tout le pays de l'Orient et de l'Inde, et sur la mer Rouge et ses montagnes, et tout le pays de Basan, et tout le pays du Liban et les îles de Kaftur, et toutes les montagnes de Sanir et d'Amana, et les montagnes d'Assur au nord, et tout le pays d'Elam, d'Assur, de Babel, de Susan et de Ma'edai, et toutes les montagnes d'Ararat, et toute la région au-delà de la mer, qui est au-delà des montagnes d'Assur, vers le nord, une terre bénie et spacieuse, et tout ce qu'il y a dedans est très bon.
\par 22 Et pour Cham sortait la deuxième portion, au-delà du Gihon, vers le sud, à droite du Jardin, et elle s'étend vers le sud et elle s'étend vers toutes les montagnes de feu, et elle s'étend vers l'ouest jusqu'à la mer. de 'Atel et il s'étend vers l'ouest jusqu'à atteindre la mer de Ma'uk - cette (mer) dans laquelle descend tout ce qui n'est pas détruit.
\par 23 Et elle s'étend vers le nord jusqu'aux limites de Gadir, et elle s'étend jusqu'au bord des eaux de la mer jusqu'aux eaux de la grande mer jusqu'à ce qu'elle s'approche du fleuve Gihon, et longe le fleuve. Gihon jusqu'à ce qu'il atteigne la droite du jardin d'Eden.
\par 24 Et c'est ici le pays qui fut attribué à Cham, comme la part qu'il devait occuper pour toujours pour lui et ses fils, pour leurs générations à jamais.
\par 25 Et pour Japhet sortait la troisième partie au-delà du fleuve Tina, au nord de l'embouchure de ses eaux, et elle s'étendait au nord-est jusqu'à toute la région de Gog et tout le pays à l'est de celle-ci.
\par 26 Et il s'étend du nord vers le nord, et il s'étend vers les montagnes de Qelt vers le nord, et vers la mer de Ma'uk, et il s'étend à l'est de Gadir jusqu'à la région des eaux de la mer.
\par 27 Et il s'étend jusqu'à ce qu'il s'approche de l'ouest de Fara et qu'il retourne vers 'Aferag, et il s'étend vers l'est jusqu'aux eaux de la mer de Me'at.
\par 28 Et elle s'étend vers le nord-est jusqu'à la région de la rivière Tina, jusqu'à ce qu'elle s'approche de la limite de ses eaux vers la montagne Rafa, et elle tourne vers le nord.
\par 29 C'est ici le pays qui a été attribué à Japhet et à ses fils, comme la part de son héritage qu'il devait posséder pour lui et ses fils, pour leurs générations et pour toujours ; cinq grandes îles et un grand pays au nord.
\par 30 Mais il fait froid, et le pays de Cham est chaud, et le pays de Sem n'est ni chaud ni froid, mais c'est un mélange de froid et de chaleur.

\chapitre{9}

\par \textit{Subdivision des trois parts entre les petits-enfants de Noé. Parmi les enfants de Cham, 1 : ceux de Sem, 2-6 : ceux de Japhet, 7-13. Serment prêté par les fils de Noé, 14-15.}

\par 1 Et Cham se divisa entre ses fils, et la première part sortit pour Cusch vers l'est, et à l'ouest de lui vers Mizraïm, et à l'ouest de lui vers Put, et à l'ouest de lui [et vers l'ouest] à l'ouest de celui-ci] sur la mer pour Canaan.
\par 2 Et Sem se divisa aussi entre ses fils, et la première part sortit pour Cham et ses fils, à l'est du fleuve Tigre jusqu'à ce qu'il s'approche de l'est, tout le pays de l'Inde et sur la mer Rouge sur sa côte. , et les eaux de Dedan, et toutes les montagnes de Mebri et Ela, et tout le pays de Susan et tout ce qui est du côté de Pharnak jusqu'à la mer Rouge et le fleuve Tina.
\par 3 Et pour Assur sortait la deuxième portion, tout le pays d'Assur et Ninive et Shinar et jusqu'à la frontière de l'Inde, et elle monte et longe le fleuve.
\par 4 Et pour Arpachshad sortit la troisième partie, tout le pays de la région des Chaldées à l'est de l'Euphrate, bordant la mer Rouge, et toutes les eaux du désert près de la langue de la mer qui regarde vers l'Egypte, tout le pays du Liban et Sanir et 'Amana jusqu'à la frontière de l'Euphrate.
\par 5 Et pour Aram sortit la quatrième partie, tout le pays de Mésopotamie entre le Tigre et l'Euphrate, au nord de la Chaldée, jusqu'à la frontière des montagnes d'Assur et du pays d'Arara.
\par 6 Et la cinquième partie sortit pour Lud, les montagnes d'Assur et tout ce qui s'y rapporte jusqu'à ce qu'elle atteigne la Grande Mer, et jusqu'à ce qu'elle atteigne l'est d'Assur, son frère.
\par 7 Et Japhet partagea aussi le pays de son héritage entre ses fils.
\par 8 Et la première partie sortait vers Gomer, à l'est, depuis le côté nord jusqu'à la rivière Tina ; et au nord sortaient vers Magog toutes les parties intérieures du nord jusqu'à la mer de Méat.
\par 9 Et Madaï sortit comme sa part qu'il devait posséder depuis l'ouest de ses deux frères jusqu'aux îles et jusqu'aux côtes des îles.
\par 10 Et pour Javan, la quatrième partie sortit, chaque île et les îles qui sont vers la frontière de Lud.
\par 11 Et pour Tubal sortait la cinquième portion au milieu de la langue qui s'approche vers la limite de la portion de Lud jusqu'à la deuxième langue, jusqu'à la région au-delà de la deuxième langue jusqu'à la troisième langue.
\par 12 Et pour Mésec sortit la sixième portion, toute la région au-delà de la troisième langue jusqu'à ce qu'elle s'approche de l'est de Gadir.
\par 13 Et pour Tiras sortit la septième portion, quatre grandes îles au milieu de la mer, qui s'étendent jusqu'à la portion de Cham [et les îles de Kamaturi furent tirées au sort pour les fils d'Arpachshad comme son héritage].
\par 14 Et ainsi les fils de Noé se partagèrent entre leurs fils en présence de Noé, leur père, et il les lia tous par un serment, implorant une malédiction sur quiconque cherchait à s'emparer de la part qui ne lui était pas revenue. par son sort.
\par 15 Et ils dirent tous : « Ainsi soit-il ; qu'il en soit ainsi pour eux-mêmes et leurs fils pour toujours à travers leurs générations jusqu'au jour du jugement, où le Seigneur Dieu les jugera par l'épée et par le feu pour toute la méchanceté impure de leurs erreurs, dont ils ont rempli la terre de la transgression, l'impureté, la fornication et le péché.

\chapitre{10}

\par \textit{Les mauvais esprits égarent les fils de Noé, 1-2. La prière de Noé, 3-6. Mastêmâ a permis de conserver un dixième de ses esprits sujets, 7-11. Noé a enseigné l'utilisation des herbes par les anges pour résister aux démons, 12-14. Noé meurt entre 15 et 17 ans. Construction de Babel et confusion des langues, 18-27. Canaan s'empare de la Palestine, 29-34. Madai reçoit Media, 33-6.}

\par 1 Et dans la troisième semaine de ce jubilé, les démons impurs commencèrent à égarer les enfants des fils de Noé, et à les égarer et à les détruire.
\par 2 Et les fils de Noé vinrent vers Noé, leur père, et ils lui parlèrent des démons qui égaraient et aveuglaient et tuaient les fils de ses fils.
\par 3 Et il pria devant l'Éternel son Dieu, et dit :
\par    
\par     'Dieu des esprits de toute chair, qui m'as fait miséricorde  
\par     Et tu m'as sauvé, moi et mes fils, des eaux du déluge,  
\par     Et tu ne m’as pas fait périr comme tu as fait périr les fils de perdition ;
\par    
\par     Car ta grâce a été grande envers moi,  
\par     Et ta miséricorde envers mon âme a été grande ;
\par    
\par     Que ta grâce s'élève sur mes fils,  
\par     Et que les mauvais esprits ne règnent pas sur eux  
\par     De peur qu’ils ne les détruisent de la terre.
\par    
\par 4 Mais bénis-moi, moi et mes fils, afin que nous puissions croître, nous multiplier et remplir la terre.
\par 5 Et tu sais comment tes veilleurs, les pères de ces esprits, ont agi à mon époque : et quant à ces esprits qui sont vivants, emprisonne-les et retiens-les fermement dans le lieu de condamnation, et qu'ils ne apportent pas la destruction sur le fils de ton serviteur, mon Dieu; car ceux-ci sont malins et créés pour détruire.
\par 6 Et qu'ils ne dominent pas sur les esprits des vivants ; car Toi seul peux exercer leur domination sur eux. Et qu'ils n'aient plus pouvoir sur les fils des justes, désormais et à jamais.
\par 7 Et l'Éternel notre Dieu nous a ordonné de tous lier.
\par 8 Et le chef des esprits, Mastêmâ, vint et dit : 'Seigneur, Créateur, que quelques-uns d'entre eux restent devant moi, et qu'ils écoutent ma voix, et fassent tout ce que je leur dirai ; car si quelques-uns d’entre eux ne me sont pas laissés, je ne pourrai pas exécuter le pouvoir de ma volonté sur les fils des hommes ; car ceux-ci sont pour la corruption et l'égarement devant mon jugement, car grande est la méchanceté des fils des hommes.
\par 9 Et Il dit : 'Que le dixième d'entre eux reste devant lui, et que neuf parts descendent dans le lieu de condamnation.'
\par 10 Et à l'un de nous, il ordonna que nous enseignions à Noé tous leurs médicaments ; car Il savait qu’ils ne marcheraient pas dans la droiture, ni ne lutteraient dans la justice.
\par 11 Et nous avons agi selon toutes ses paroles : nous avons lié tous les méchants méchants dans un lieu de condamnation et nous avons laissé un dixième d'entre eux afin qu'ils puissent être soumis devant Satan sur la terre.
\par 12 Et nous avons expliqué à Noé tous les remèdes contre leurs maladies, ainsi que leurs séductions, comment il pouvait les guérir avec les herbes de la terre.
\par 13 Et Noé écrivit toutes choses dans un livre, comme nous lui instruisions concernant toutes sortes de médicaments. Ainsi, les mauvais esprits étaient empêchés de (blesser) les fils de Noé.
\par 14 Et il donna tout ce qu'il avait écrit à Sem, son fils aîné ; car il l'aimait extrêmement plus que tous ses fils.
\par 15 Et Noé s'endormit avec ses pères, et fut enterré sur le mont Lubar, au pays d'Ararat.
\par 16 Il accomplit neuf cent cinquante ans dans sa vie, dix-neuf jubilés et deux semaines et cinq ans. [16 h 59]
\par 17 Et dans sa vie sur terre, il surpassait les enfants des hommes, sauf Enoch, à cause de la justice dans laquelle il était parfait. Car la fonction d'Enoch a été ordonnée pour rendre témoignage aux générations du monde, afin qu'il raconte tous les actes de génération en génération, jusqu'au jour du jugement.
\par 18 Et au trente-troisième jubilé, la première année de la deuxième semaine, Peleg prit pour lui une femme, dont le nom était Lomna, fille de Sinaar, et elle lui enfanta un fils la quatrième année de cette année. semaine, et il appela son nom Reu; car il dit : « Voici, les enfants des hommes sont devenus méchants à cause du mauvais dessein de se construire une ville et une tour au pays de Shinar. »
\par 19 Car ils partirent du pays d'Ararat vers l'est, vers Shinar ; car de son temps ils bâtirent la ville et la tour, en disant : « Allez, montons ainsi au ciel. »
\par 20 Et ils commencèrent à construire, et la quatrième semaine ils fabriquèrent des briques au feu, et les briques leur servèrent de pierre, et l'argile avec laquelle ils les cimentèrent ensemble était de l'asphalte qui sort de la mer et de la mer. fontaines d'eau au pays de Shinar.
\par 21 Et ils l'ont construit : quarante-trois ans [1645-1688 AM] ils l'ont construit ; sa largeur était de 203 briques, et la hauteur (d'une brique) était le tiers d'une ; sa hauteur s'élevait à 5 433 coudées et 2 palmes, et (l'étendue d'un mur était) de treize stades (et des trente autres stades).
\par 22 Et le Seigneur notre Dieu nous a dit : « Voici, ils sont un seul peuple, et (cela) ils ont commencé à faire, et maintenant rien ne leur sera refusé. Allez, descendons et confondons leur langage, afin qu'ils ne comprennent pas le langage des autres, et qu'ils soient dispersés en villes et en nations, et qu'un seul objectif ne demeurera plus en eux jusqu'au jour du jugement.
\par 23 Et le Seigneur est descendu, et nous sommes descendus avec lui pour voir la ville et la tour que les enfants des hommes avaient bâties.
\par 24 Et il confondit leur langage, et ils ne s'entendirent plus les uns les autres, et ils cessèrent alors de bâtir la ville et la tour.
\par 25 C'est pourquoi tout le pays de Shinar est appelé Babel, parce que l'Éternel y a confondu toute la langue des enfants des hommes, et de là ils ont été dispersés dans leurs villes, chacun selon sa langue et sa nation.
\par 26 Et l'Éternel envoya un vent puissant contre la tour et la renversa sur la terre. Et voici, c'était entre Assur et Babylone, au pays de Shinar, et ils appelèrent son nom « Renversement ».
\par 27 La quatrième semaine de la première année [1688 AM] au début de celle-ci, lors du trente-quatrième jubilé, furent-ils dispersés du pays de Shinar.
\par 28 Et Cham et ses fils entrèrent dans le pays qu'il devait occuper, et qu'il avait acquis pour sa part dans le pays du midi.
\par 29 Et Canaan vit que le pays du Liban jusqu'au fleuve d'Egypte était très bon, et il n'entra pas dans le pays de son héritage à l'ouest (c'est-à-dire vers) la mer, et il habita dans le pays de Liban, à l’est et à l’ouest depuis la frontière du Jourdain et depuis la frontière de la mer.
\par 30 Et Cham, son père, et Cush et Mizraïm ses frères lui dirent : Tu t'es établi dans un pays qui n'est pas à toi et qui ne nous est pas attribué par tirage au sort : ne le fais pas ; car si tu fais cela, toi et tes fils tomberez dans le pays et serez maudits par sédition ; car par la sédition vous vous êtes établis, et par la sédition vos enfants tomberont, et vous serez extirpé pour toujours.
\par 31 « N'habitez pas dans la demeure de Sem ; car c'est à Sem et à ses fils que cela est arrivé par leur sort.
\par 32 «Tu es maudit, et tu seras maudit plus que tous les fils de Noé, par la malédiction par laquelle nous nous sommes liés par serment en présence du saint juge et en présence de Noé notre père.»
\par 33 Mais il ne les écouta pas, et il habita au pays du Liban depuis Hamath jusqu'à l'entrée de l'Égypte, lui et ses fils jusqu'à ce jour.
\par 34 Et c'est pour cette raison que ce pays est nommé Canaan.
\par 35 Et Japhet et ses fils allèrent vers la mer et habitèrent dans le pays de leur part, et Madaï vit le pays de la mer et cela ne lui plut pas, et il demanda une (portion) à Cham, Assur et Arpacshad. le frère de sa femme, et il a habité jusqu'à ce jour dans le pays de Médie, près du frère de sa femme.
\par 36 Et il appela sa demeure et celle de ses fils Médie, du nom de leur père Madaï.

\chapitre{11}

\par \textit{Reu et Serug, 1 (cf. Gen. xi.20, 21). Montée de la guerre, effusion de sang, consommation de sang et idolâtrie, 2-7. Nachor et Terah, 8-14 (cf. Gen. xi.22-30). La connaissance d'Abram de Dieu et ses actions merveilleuses, 15-24.}

\par 1 Et au trente-cinquième jubilé, la troisième semaine de la première année [1681 AM] de celui-ci, Reu prit pour lui une femme, et son nom était 'Ôrâ, la fille de 'Ûr, le fils de Kesed. , et elle lui enfanta un fils, et il appela son nom Sêrôh, la septième année de cette semaine de ce jubilé. [1687 AM]
\par 2 Et les fils de Noé commencèrent à se faire la guerre, à se faire prisonniers et à s'entre-tuer, et à verser le sang des hommes sur la terre, et à manger du sang, et à bâtir des villes fortes, et des murailles, et les tours et les individus (commencèrent) à s'élever au-dessus de la nation, et à fonder les commencements de royaumes, et à faire la guerre peuple contre peuple, et nation contre nation, et ville contre ville, et tous (commencèrent) à faire le mal, et pour acquérir des armes et enseigner la guerre à leurs fils, ils commencèrent à s'emparer des villes et à vendre des esclaves, hommes et femmes.
\par 3 Et 'Ûr, fils de Kesed, bâtit la ville de 'Ara en Chaldée, et lui donna le nom de son propre nom et du nom de son père. Et ils se fabriquèrent des images en fonte, et ils adorèrent chacun l'idole, l'image en fonte qu'ils s'étaient faite, et ils commencèrent à fabriquer des images taillées et des simulacres impurs, et des esprits malins les aidèrent et les séduisirent à commettre la transgression et l'impureté. .
\par 4 Et le prince Mastêmâ s'efforça de faire tout cela, et il envoya d'autres esprits, ceux qui étaient mis sous sa main, pour commettre toute sorte de mal et de péché, et toute sorte de transgression, pour corrompre et détruire, et verser le sang sur la terre.
\par 5 C'est pour cette raison qu'il appela le nom de Sêrôh, Serug, car chacun se tournait vers toutes sortes de péchés et de transgressions.
\par 6 Et il grandit et habita à Ur en Chaldée, près du père de la mère de sa femme, et il adora des idoles, et il prit une femme au trente-sixième jubilé, la cinquième semaine, en la première année de celle-ci, [1744 AM] et elle s'appelait Melka, la fille de Kaber, la fille du frère de son père.
\par 7 Et elle lui enfanta Nahor, la première année de cette semaine, et il grandit et habita à Ur en Chaldée, et son père lui enseigna les recherches des Chaldéens pour deviner et augurer selon les signes du ciel.
\par 8 Et au trente-septième jubilé, dans la sixième semaine, la première année de celle-ci, [1800 AM] il prit pour lui une femme, et elle s'appelait 'Ijaska, fille de Nestag des Chaldéens.
\par 9 Et elle lui enfanta Térach la septième année de cette semaine. [1806 AM]
\par 10 Et le prince Mastêmâ envoya des corbeaux et des oiseaux dévorer la graine qui était semée dans le pays, afin de détruire le pays et de ravir les enfants des hommes de leurs travaux. Avant de pouvoir labourer la graine, les corbeaux la ramassaient à la surface du sol.
\par 11 Et c'est pour cette raison qu'il appela son nom Térach, parce que les corbeaux et les oiseaux les réduisaient à la misère et dévoraient leur semence.
\par 12 Et les années commencèrent à être stériles, à cause des oiseaux, et ils dévorèrent tous les fruits des arbres : ce n'est qu'avec de grands efforts qu'ils purent conserver un peu de tous les fruits de la terre dans leur jours.
\par 13 Et en ce trente-neuvième jubilé, la deuxième semaine de la première année, [1870 AM] Térah prit pour lui une femme, et elle s'appelait 'Edna, la fille d'Abram, la fille de la sœur de son père. . Et la septième année de cette semaine [1876 AM], elle lui enfanta un fils, et il l'appela Abram, du nom du père de sa mère ;
\par 14 car il était mort avant que sa fille ait conçu un fils.
\par 15 Et l'enfant commença à comprendre les erreurs de la terre qui s'égaraient tous après les images taillées et après l'impureté, et son père lui apprit à écrire, et il avait deux semaines d'années, [1890 AM] et il se sépara de son père, afin qu'il n'adore pas les idoles avec lui.
\par 16 Et il commença à prier le Créateur de toutes choses afin qu'il le sauve des erreurs des enfants des hommes, et que sa part ne tombe pas dans l'erreur après l'impureté et la vilenie.
\par 17 Et le temps des semailles arriva pour semer la semence sur la terre, et ils sortirent tous ensemble pour protéger leur semence contre les corbeaux, et Abram sortit avec ceux qui partaient, et l'enfant était un garçon de quatorze ans.
\par 18 Et une nuée de corbeaux vint pour dévorer la graine, et Abram courut à leur rencontre avant qu'ils ne s'installent sur le sol, et leur cria avant qu'ils ne s'installent sur le sol pour dévorer la graine, et dit : « Ne descendez pas : revenez. vers l'endroit d'où vous êtes venus », et ils rebroussèrent chemin.
\par 19 Et il fit revenir les nuées de corbeaux ce jour-là soixante-dix fois, et de tous les corbeaux de tout le pays où Abram était là, pas un seul ne s'y installa.
\par 20 Et tous ceux qui étaient avec lui dans tout le pays le virent crier, et tous les corbeaux revinrent, et son nom devint grand dans tout le pays des Chaldéens.
\par 21 Cette année, tous ceux qui voulaient semer vinrent vers lui, et il les accompagna jusqu'à ce que le temps des semailles soit terminé. Ils semèrent leur terre, et cette année-là ils rapportèrent chez eux assez de blé, en mangèrent et furent rassasiés.
\par 22 Et la première année de la cinquième semaine [1891 AM] Abram enseigna à ceux qui fabriquaient des instruments pour les bœufs, les artisans du bois, et ils fabriquèrent un vase au-dessus du sol, face au cadre de la charrue, afin de mettre la graine était dessus, et la graine tombait de là sur le soc de la charrue, et était cachée dans la terre, et ils ne craignaient plus les corbeaux.
\par 23 Et de cette manière, ils firent des récipients au-dessus du sol sur tous les cadres des charrues, et ils semèrent et labourèrent toute la terre, comme Abram le leur avait ordonné, et ils ne craignirent plus les oiseaux.

\chapitre{12}

\par \textit{Abram cherche à chasser Térah de l'idolâtrie, 1-8. Épouse Sarai, 9 ans. Haran et Nachor, 9-11 ans. Abram brûle les idoles : mort d'Haran, 12-14 (cf. Gen. xi.28). Térah et sa famille se rendent à Haran, le 15. Abram observe les étoiles et prie, le 16-21. Est invité à aller en Canaan et béni, 22-4. Pouvoir de parler hébreu qui lui a été donné , 25-7. Quitte Haran pour Canaan, 28-31. (Cf. Gén. xi.31-xii.3.)}

\par 1 Et il arriva, la sixième semaine, la septième année de celle-ci, [1904 AM] qu'Abram dit à Térach, son père, en disant : « Père !
\par 2 Et il dit : Voici, me voici, mon fils. Et il a dit,
\par    
\par     «Quelle aide et quel profit avons-nous de ces idoles que tu adores,  
\par     Et devant quoi tu t'incline ?
\par    
\par 3 Car il n'y a pas d'esprit en eux,  
\par     Car ce sont des formes muettes et trompeuses du cœur.  
\par     Ne les adorez pas :
\par    
\par 4 Adorez le Dieu du ciel,  
\par     Qui fait descendre la pluie et la rosée sur la terre  
\par     Et fait tout sur terre,
\par    
\par     Et il a tout créé par sa parole,  
\par     Et toute vie vient de devant Sa face.
\par    
\par 5 Pourquoi adorez-vous des choses qui n'ont pas d'esprit en elles ?  
\par     Car ils sont l'ouvrage de mains (d'hommes),
\par    
\par     Et vous les portez sur vos épaules,  
\par     Et vous n'avez aucune aide de leur part,
\par    
\par     Mais ils sont une grande honte pour ceux qui les fabriquent,  
\par     Et une tromperie du cœur pour ceux qui les adorent :  
\par     Ne les adorez pas.
\par    
\par 6 Et son père lui dit : Moi aussi, je le sais, mon fils, mais que ferai-je d'un peuple qui m'a fait servir devant eux ?
\par 7 Et si je leur dis la vérité, ils me tueront ; car leur âme s'attache à eux pour les adorer et les honorer.
\par 8 Garde le silence, mon fils, de peur qu'ils ne te tuent. Et il dit ces paroles à ses deux frères, et ils furent en colère contre lui et il garda le silence.
\par 9 Et au quarantième jubilé, la deuxième semaine de la septième année, Abram prit pour lui une femme, et elle s'appelait Saraï, la fille de son père, et elle devint sa femme.
\par 10 Et Haran, son frère, prit une femme la troisième année de la troisième semaine, [1928 AM] et elle lui enfanta un fils la septième année de cette semaine, [1932 AM] et il appela son nom Parcelle.
\par 11 Et Nahor, son frère, prit une femme.
\par 12 Et la soixantième année de la vie d'Abram, c'est-à-dire la quatrième semaine, la quatrième année de celle-ci, [1936 AM] Abram se leva de nuit et brûla la maison des idoles, et il brûla tout cela. était dans la maison et personne ne le savait.
\par 13 Et ils se levèrent pendant la nuit et cherchèrent à sauver leurs dieux du milieu du feu.
\par 14 Et Haran se hâta de les sauver, mais le feu flamba sur lui, et il fut brûlé dans le feu, et il mourut à Ur en Chaldée devant Térah son père, et ils l'enterrant à Ur en Chaldée.
\par 15 Et Térach sortit d'Ur en Chaldée, lui et ses fils, pour aller au pays du Liban et au pays de Canaan, et il habita au pays de Haran, et Abram habita avec Térach, son père, à Haran. deux semaines d'années.
\par 16 Et la sixième semaine, la cinquième année de celle-ci, [1951 AM] Abram resta assis toute la nuit à la nouvelle lune du septième mois pour observer les étoiles du soir au matin, afin de voir ce qui se passerait. être le caractère de l'année en ce qui concerne les pluies, et il était seul alors qu'il était assis et observait.
\par 17 Et une parole lui vint au cœur et il dit : Tous les signes des étoiles, et les signes de la lune et du soleil sont tous dans la main du Seigneur. Pourquoi est-ce que je (les) recherche ?
\par    
\par 18 S'Il le veut, Il fait pleuvoir matin et soir ;  
\par     Et s’Il le désire, Il le retient,  
\par     Et tout est entre ses mains.
\par    
\par 19 Et il pria cette nuit-là et dit :  
\par     « Mon Dieu, Dieu Très-Haut, Toi seul es mon Dieu,  
\par     Et c'est toi et ta domination que j'ai choisis.  
\par     Et tu as créé toutes choses,  
\par     Et tout ce qui est l'ouvrage de tes mains.
\par    
\par 20 Délivre-moi des mains des mauvais esprits qui dominent sur les pensées du cœur des hommes,  
\par     Et qu'ils ne m'éloignent pas de Toi, mon Dieu.
\par    
\par     Et établis-toi moi et ma postérité pour toujours  
\par     Afin que nous ne nous égarons plus désormais et pour toujours.
\par    
\par 21 Et il dit : Dois-je retourner à Ur des Chaldéens qui cherchent ma face, afin que je puisse revenir vers eux, dois-je rester ici en ce lieu ? Le droit chemin devant Toi prospère entre les mains de Ton serviteur afin qu'il puisse l'accomplir et que je ne puisse pas marcher dans la tromperie de mon cœur, ô mon Dieu.'
\par 22 Et il finit de parler et de prier, et voici, la parole du Seigneur lui fut envoyée par moi, disant : « Lève-toi de ton pays, et de ta parenté et de la maison de ton père, vers un pays que je te montrerai, et je ferai de toi une nation grande et nombreuse.
\par    
\par 23 Et je te bénirai  
\par     Et je rendrai ton nom grand,  
\par     Et tu seras béni sur la terre,  
\par     Et en Toi seront bénies toutes les familles de la terre,  
\par     Et je bénirai ceux qui te bénissent,  
\par     Et maudis ceux qui te maudissent.
\par    
\par 24 Et je serai ton Dieu, toi et ton fils, et le fils de ton fils, et toute ta postérité : ne crains rien, désormais et pour toutes les générations de la terre, je suis ton Dieu.
\par 25 Et le Seigneur Dieu dit : « Ouvrez sa bouche et ses oreilles, afin qu'il entende et qu'il parle avec sa bouche, avec la langue qui a été révélée » ; car cela avait cessé de la bouche de tous les enfants des hommes depuis le jour de la chute (de Babel).
\par 26 Et j'ouvris sa bouche, et ses oreilles et ses lèvres, et je commençai à lui parler en hébreu dans la langue de la création.
\par 27 Et il prit les livres de ses pères, et ceux-ci étaient écrits en hébreu, et il les transcrivit, et il commença désormais à les étudier, et je lui fis connaître ce qu'il ne pouvait pas (comprendre), et il les étudia pendant les six mois pluvieux.
\par 28 Et il arriva la septième année de la sixième semaine [1953 AM] qu'il parla à son père et l'informa qu'il quitterait Haran pour aller au pays de Canaan pour le voir et revenir vers lui.
\par 29 Et Térach, son père, lui dit : Vas en paix:
\par    
\par     Que le Dieu éternel aplanisse ton chemin.  
\par     Et le Seigneur [(soit) avec toi, et] te protège de tout mal,  
\par     Et accorde-toi grâce, miséricorde et faveur devant ceux qui te voient,  
\par     Et qu'aucun des enfants des hommes n'ait pouvoir sur toi pour te faire du mal ;  
\par     Vas en paix.
\par    
\par 30 Et si tu vois un pays agréable à tes yeux pour y habiter, alors lève-toi et emmène-moi vers toi et prends avec toi Lot, le fils de Haran, ton frère, comme ton propre fils : l'Éternel soit avec toi.
\par 31 Et Nahor, ton frère, part avec moi jusqu'à ce que tu reviennes en paix, et nous partirons tous ensemble avec toi.

\chapitre{13}

\par \textit{Abram voyage de Haran à Sichem en Canaan, de là à Hébron et de là en Egypte, 1-14a. Retourne à Canaan où Lot se sépare de lui, reçoit la promesse de Canaan et se rend à Hébron, 14b-21. Attaque de Chedorlaomer sur Sodome et Gomorrhe : Lot fait prisonnier, 22-4. Loi des dîmes promulguée, 25-9. (Cf. Gen. XII.4-10, 15-17, 19-20 ; XIII.11-18 ; XIV.8-14 ; 21-4.)}

\par 1 Et Abram partit de Haran, et il emmena Saraï, sa femme, et Lot, le fils de son frère Haran, au pays de Canaan, et il entra en Assur, et se rendit à Sichem, et habita près d'un chêne élevé.
\par 2 Et il vit, et voici, le pays était très agréable depuis l'entrée de Hamath jusqu'au chêne élevé.
\par 3 Et l'Éternel lui dit : 'Je donnerai ce pays à toi et à ta postérité.'
\par 4 Et il bâtit là un autel, et il offrit dessus un holocauste à l'Éternel, qui lui était apparu.
\par 5 Et il partit de là vers la montagne. . . Béthel à l'ouest et Aï à l'est, et il y dressa sa tente.
\par 6 Et il vit et voici, le pays était très vaste et beau, et tout y poussait : des vignes, des figuiers et des grenadiers, des chênes et des ilex, et des térébinthes et des arbres à huile, et des cèdres, et des cyprès et des dattiers, et tous les arbres de les champs, et il y avait de l'eau sur les montagnes.
\par 7 Et il bénit l'Éternel qui l'avait fait sortir d'Ur en Chaldée et l'avait amené dans ce pays.
\par 8 Et il arriva la première année, la septième semaine, à la nouvelle lune du premier mois, 1954 AM] qu'il bâtit un autel sur cette montagne et invoqua le nom du Seigneur : « Toi , le Dieu éternel, tu es mon Dieu.
\par 9 Et il offrit sur l'autel un holocauste à l'Éternel, afin qu'il soit avec lui et ne l'abandonne pas tous les jours de sa vie.
\par 10 Et il partit de là et se dirigea vers le sud, et il arriva à Hébron et Hébron fut bâtie à cette époque-là, et il y demeura deux ans, et il partit (de là) dans le pays du sud, à Bealoth, et il y eut une famine dans le pays.
\par 11 Et Abram entra en Egypte la troisième année de la semaine, et il demeura en Egypte cinq ans avant que sa femme ne lui soit arrachée.
\par 12 C'est à cette époque-là que Tanaïs, en Égypte, fut bâtie, sept ans après Hébron.
\par 13 Et il arriva que lorsque Pharaon s'empara de Saraï, la femme d'Abram, l'Éternel frappa Pharaon et sa maison de grandes plaies à cause de Saraï, la femme d'Abram.
\par 14 Et Abram était très glorieux en raison de ses possessions en moutons, en bœufs, en ânes, en chevaux, en chameaux, en serviteurs et en servantes, et en argent et en or en abondance. Et Lot, également le fils de son frère, était riche.
\par 15 Et Pharaon rendit Saraï, la femme d'Abram, et il l'envoya hors du pays d'Égypte, et il se rendit au lieu où il avait dressé sa tente au début, au lieu de l'autel, avec Ai. à l'est, et Béthel à l'ouest, et il bénit l'Éternel, son Dieu, qui l'avait ramené en paix.
\par 16 Et il arriva au quarante et unième jubilé, la troisième année de la première semaine, [1963 AM] qu'il revint à ce lieu et y offrit un holocauste, et invoqua le nom du Seigneur, et a dit : « Toi, le Dieu Très-Haut, tu es mon Dieu pour toujours et à jamais. »
\par 17 Et la quatrième année de cette semaine [1964 AM] Lot se sépara de lui, et Lot habita à Sodome, et les hommes de Sodome étaient extrêmement pécheurs.
\par 18 Et il était affligé dans son cœur que le fils de son frère se soit séparé de lui ; car il n'avait pas d'enfants.
\par 19 L'année où Lot fut emmené captif, l'Éternel dit à Abram, après que Lot se fut séparé de lui, la quatrième année de cette semaine : « Leve tes yeux du lieu où tu habites, vers le nord et vers le sud. , et vers l’ouest et vers l’est.
\par 20 Car tout le pays que tu vois, je le donnerai à toi et à ta postérité pour toujours, et je rendrai ta semence comme le sable de la mer ; même si un homme peut compter la poussière de la terre, ta semence ne sera pas perdue. ne pas être numéroté.
\par 21 Lève-toi, marche (à travers le pays) dans sa longueur et dans sa largeur, et vois tout ; car je le donnerai à ta postérité. Et Abram partit pour Hébron et y demeura.
\par 22 Et cette année-là, Kedorlaomer, roi d'Élam, et Amraphel, roi de Shinar, et Arioch, roi de Sellasar, et Tergal, roi des nations, tuèrent le roi de Gomorrhe, et le roi de Sodome s'enfuit, et beaucoup tombé à cause de ses blessures dans la vallée de Siddim, au bord de la mer Salée.
\par 23 Et ils prirent captifs Sodome, Adam et Zeboim, et ils capturèrent aussi Lot, le fils du frère d'Abram, et tous ses biens, et ils allèrent vers Dan.
\par 24 Et quelqu'un qui s'était échappé vint et dit à Abram que le fils de son frère avait été fait prisonnier et (Abram) arma ses serviteurs. . .
\par 25 . . . . pour Abram et pour sa postérité, un dixième des prémices au Seigneur, et le Seigneur l'a ordonné comme ordonnance pour toujours afin qu'ils le donnent aux prêtres qui servaient devant Lui, afin qu'ils le possèdent pour toujours.
\par 26 Et pour cette loi il n'y a pas de limite de jours ; car il a ordonné que les générations donnent à jamais à l'Éternel le dixième de tout, de la semence, du vin, de l'huile, du bétail et des brebis.
\par 27 Et il le donna à ses prêtres pour qu'ils mangent et boivent avec joie devant lui.
\par 28 Et le roi de Sodome vint vers lui et se prosterna devant lui, et dit : « Notre Seigneur Abram, donne-nous les âmes que tu as sauvées, mais que le butin soit à toi.
\par 29 Et Abram lui dit : « Je lève mes mains vers le Dieu Très-Haut, pour que, du fil au lanière de chaussure, je ne prendrai rien de ce qui est à toi, de peur que tu ne dises : J'ai enrichi Abram ; sauf ce qu'ont mangé les jeunes gens, et la part des hommes qui m'accompagnaient, Aner, Eschol et Mamré. Ceux-ci prendront leur part.

\chapitre{14}

\par \textit{Abram reçoit la promesse d'un fils et d'une descendance innombrable, 1-7. Offre un sacrifice et on lui dit que sa postérité se trouve en Égypte, 8-17. L'alliance de Dieu avec Abram, 18-20. Agar enfante Ismaël, 21-4. (Cf. Gen. XV.; XVI.1-4, 11.)}


\par 1 Après ces choses, la quatrième année de cette semaine, à la nouvelle lune du troisième mois, la parole de l'Éternel fut adressée à Abram en songe, disant : « Ne crains rien, Abram ; Je suis ton défenseur, et ta récompense sera extrêmement grande.
\par 2 Et il dit : 'Seigneur, Seigneur, que me donneras-tu, puisque je pars d'ici sans enfants, et que le fils de Maseq, le fils de ma servante, est le Dammasek Eliezer : il sera mon héritier, et pour moi tu n'as donné aucune semence.
\par 3 Et il lui dit : « Celui-ci (l'homme) ne sera pas ton héritier, mais celui-ci qui sortira de tes propres entrailles ; il sera ton héritier.
\par 4 Et il le fit sortir et lui dit : «Regarde vers le ciel et compte les étoiles, si tu es capable de les compter.»
\par 5 Et il regarda vers le ciel, et vit les étoiles. Et Il lui dit : « Ainsi en sera-t-il de ta postérité. »
\par 6 Et il crut au Seigneur, et cela lui fut imputé à justice.
\par 7 Et Il lui dit : « Je suis l'Éternel qui t'ai fait sortir d'Ur en Chaldée, pour te donner le pays des Cananéens pour que tu le possèdes à jamais ; et je serai Dieu pour toi et pour ta postérité après toi.
\par 8 Et il dit : 'Seigneur, Seigneur, par quoi saurai-je que j'en hériterai ?'
\par 9 Et Il lui dit : 'Prends-moi une génisse de trois ans, et une chèvre de trois ans, et une brebis de trois ans, et une tourterelle et un pigeon.'
\par 10 Et il prit tout cela au milieu du mois et il habita près du chêne de Mamré, qui est près d'Hébron.
\par 11 Et il bâtit là un autel, et sacrifia tout cela ; et il versa leur sang sur l'autel, les divisa au milieu et les opposa les uns aux autres ; mais les oiseaux ne le divisèrent pas.
\par 12 Et les oiseaux tombèrent sur les morceaux, et Abram les chassa, et ne permit pas aux oiseaux de les toucher.
\par 13 Et il arriva, quand le soleil se coucha, qu'une extase tomba sur Abram, et voici ! une horreur de grandes ténèbres s'abattit sur lui, et il fut dit à Abram : 'Sache avec certitude que ta postérité sera étrangère dans un pays (qui n'est) pas le leur, et ils les réduiront en esclavage et les affligeront quatre. Cent ans.'
\par 14 'Et je jugerai aussi la nation à laquelle ils seront esclaves, et après cela ils en sortiront avec beaucoup de biens.'
\par 15 'Et tu iras en paix vers tes pères, et tu seras enterré dans une bonne vieillesse.'
\par 16 'Mais à la quatrième génération, ils reviendront ici ; car l'iniquité des Amoréens n'est pas encore pleine.
\par 17 Et il se réveilla de son sommeil, et il se leva, et le soleil se coucha ; et il y avait une flamme, et voici ! un four fumait et une flamme de feu passait entre les morceaux.
\par 18 Et ce jour-là, l'Éternel fit une alliance avec Abram, disant : Je donnerai à ta postérité ce pays, depuis le fleuve d'Égypte jusqu'au grand fleuve, le fleuve Euphrate, les Kéniens, les Kéniziens, les Kadmonites, les Phéréziens, les Rephaïm, les Phakorites, les Héviens, les Amoréens, les Cananéens, les Guirgashites et les Jébusiens.
\par 19 Et le jour passa, et Abram offrit les morceaux, et les oiseaux, et leurs offrandes de fruits et leurs libations, et le feu les dévora.
\par 20 Et ce jour-là, nous avons conclu une alliance avec Abram, comme nous l'avions conclue avec Noé ce mois-ci ; et Abram renouvela pour lui-même la fête et l'ordonnance pour toujours.
\par 21 Et Abram se réjouit et fit connaître toutes ces choses à Saraï, sa femme ; et il croyait qu'il aurait de la semence, mais elle ne la porta pas.
\par 22 Et Saraï conseilla son mari Abram, et lui dit : « Va vers Agar, ma servante égyptienne ; il se peut que je te construise une postérité par elle.
\par 23 Et Abram écouta la voix de Saraï, sa femme, et lui dit : « Fais (ainsi). » Et Saraï prit Agar, sa servante, l'Égyptienne, et la donna pour femme à Abram, son mari.
\par 24 Et il entra vers elle, et elle conçut et lui enfanta un fils, et il appela son nom Ismaël, la cinquième année de cette semaine [1965 AM] ; et c'était la quatre-vingt-sixième année de la vie d'Abram.

\chapitre{15}

\par \textit{Abram célèbre la fête des prémices, 1-2 : son nom change et la circoncision est instituée, 3-14. Le nom de Sarai a changé et Isaak a promis, 15-21. Abraham, Ismaël et toute sa maison circoncis, 22-4. La circoncision est une ordination éternelle, 25, 26. Israël partage cet honneur avec les anges les plus élevés qui ont été créés circoncis, 27-9. Israël soumis à Dieu seul : les autres nations aux anges, 30-2. Infidélité future d'Israël, 33-4. (Cf. Gen. XVII.)}

\par 1 Et la cinquième année de la quatrième semaine de ce jubilé, [1979 AM] au troisième mois, au milieu du mois, Abram célébra la fête des prémices de la moisson des céréales.
\par 2 Et il offrit à l'Éternel des offrandes nouvelles sur l'autel, les prémices des produits, une génisse, une chèvre et un agneau sur l'autel, en holocauste à l'Éternel; il offrit leurs offrandes de fruits et leurs libations sur l'autel avec de l'encens.
\par 3 Et l'Éternel apparut à Abram et lui dit : « Je suis Dieu Tout-Puissant ; approuve-toi devant moi et sois parfait.
\par 4 'Et je ferai mon alliance entre moi et toi, et je te multiplierai extrêmement.'
\par 5 Et Abram tomba sur sa face, et Dieu lui parla et dit :
\par    
\par 6 Voici, mon ordonnance est avec toi,  
\par     Et tu seras le père de nombreuses nations.
\par    
\par 7 Et ton nom ne sera plus appelé Abram,  
\par     Mais ton nom, désormais et pour toujours, sera Abraham.  
\par     Car je t’ai fait père de nombreuses nations.
\par    
\par 8 Et je te rendrai très grand,  
\par     Et je ferai de toi des nations,  
\par     Et des rois sortiront de toi.
\par    
\par 9 Et j'établirai mon alliance entre moi et toi, et ta postérité après toi, à travers leurs générations, pour une alliance éternelle, afin que je sois un Dieu pour toi et pour ta postérité après toi.
\par 10 (Et je te donnerai, ainsi qu'à ta postérité après toi) le pays où tu as séjourné, le pays de Canaan, afin que tu le possèdes pour toujours, et je serai leur Dieu.
\par 11 Et l'Éternel dit à Abraham : « Et quant à toi, garde mon alliance, toi et ta postérité après toi ; et circoncis tout mâle parmi vous, et circoncis tes prépuces, et ce sera un signe d'un alliance éternelle entre moi et vous.
\par 12 Et vous circoncirez l'enfant le huitième jour, tous les mâles de vos générations, ceux qui sont nés dans la maison, ou que vous avez achetés à prix d'argent à un étranger, que vous avez acquis et qui n'est pas de votre postérité. .'
\par 13 'Celui qui est né dans ta maison sera sûrement circoncis, et ceux que tu as achetés avec de l'argent seront circoncis, et mon alliance sera dans ta chair pour une ordonnance éternelle.'
\par 14 «Et le mâle incirconcis qui n'est pas circoncis dans la chair de son prépuce le huitième jour, cette âme sera retranchée de son peuple, car il a rompu mon alliance.»
\par 15 Et Dieu dit à Abraham : « Quant à Saraï, ta femme, son nom ne s'appellera plus Saraï, mais Sarah sera son nom.
\par 16 'Et je la bénirai, et je te donnerai un fils par elle, et je le bénirai, et il deviendra une nation, et des rois des nations sortiront de lui.'
\par 17 Et Abraham tomba sur sa face et se réjouit, et dit dans son cœur : « Celui qui a cent ans naîtra-t-il un fils, et Sarah, qui a quatre-vingt-dix ans, enfantera-t-elle ?
\par 18 Et Abraham dit à Dieu : 'Ô qu'Ismaël puisse vivre devant toi !'
\par 19 Et Dieu dit : Oui, et Sara aussi t'enfantera un fils, et tu lui donneras le nom d'Isaac, et j'établirai mon alliance avec lui, une alliance éternelle, et pour sa postérité après lui.
\par 20 «Et quant à Ismaël aussi, je t'ai entendu, et voici, je vais le bénir, et le rendre grand, et le multiplier extrêmement, et il engendrera douze princes, et je ferai de lui une grande nation.»
\par 21 «Mais j'établirai mon alliance avec Isaac, que Sarah t'enfantera, en ces jours-ci, l'année prochaine.»
\par 22 Et il cessa de lui parler, et Dieu partit d'Abraham.
\par 23 Et Abraham fit ce que Dieu lui avait dit, et il prit Ismaël, son fils, et tous ceux qui étaient nés dans sa maison et qu'il avait achetés avec son argent, tous les mâles de sa maison, et circoncit la chair des leur prépuce.
\par 24 Et le même jour Abraham fut circoncis, et tous les hommes de sa maison (et ceux nés dans la maison), et tous ceux qu'il avait achetés avec de l'argent aux enfants de l'étranger, furent circoncis avec lui. .
\par 25 Cette loi est pour toutes les générations pour toujours, et il n'y a pas de circoncision des jours, ni d'omission d'un jour sur les huit jours ; car c'est une ordonnance éternelle, ordonnée et écrite sur les tablettes célestes.
\par 26 Et quiconque naît dont la chair du prépuce n'est pas circoncise le huitième jour, n'appartient pas aux enfants de l'alliance que l'Éternel a faite avec Abraham, mais aux enfants de la destruction ; et il n'y a aucun signe sur lui qu'il appartient au Seigneur, mais (il est destiné) à être détruit et tué de la terre, et à être déraciné de la terre, car il a rompu l'alliance du Seigneur notre Dieu.
\par 27 Car tous les anges de la présence et tous les anges de la sanctification ont été ainsi créés dès le jour de leur création, et devant les anges de la présence et les anges de la sanctification il a sanctifié Israël, afin qu'ils soient avec lui. et avec ses saints anges.
\par 28 Et tu ordonneras aux enfants d'Israël et qu'ils observent le signe de cette alliance pour leurs générations comme une ordonnance éternelle, et ils ne seront pas déracinés du pays.
\par 29 Car le commandement est établi comme une alliance, afin qu'ils l'observent pour toujours parmi tous les enfants d'Israël.
\par 30 Pour Ismaël, ses fils, ses frères et Ésaü, l'Éternel ne les a pas fait approcher, et il ne les a pas choisis parce qu'ils sont les enfants d'Abraham, parce qu'il les connaissait, mais il a choisi Israël pour être son peuple.
\par 31 Et il le sanctifia, et le rassembla parmi tous les enfants des hommes ; car il y a beaucoup de nations et beaucoup de peuples, et tous lui appartiennent, et sur tout il a placé des esprits avec autorité pour les éloigner de lui.
\par 32 Mais sur Israël, il n'a établi aucun ange ni aucun esprit, car lui seul est leur chef, et il les préservera et les exigera de la main de ses anges et de ses esprits, et de la main de toutes ses puissances, afin afin qu'Il les préserve et les bénisse, et qu'ils soient à Lui et qu'Il soit à eux désormais pour toujours.
\par 33 Et maintenant je t'annonce que les enfants d'Israël ne resteront pas fidèles à cette ordonnance, et qu'ils ne circonciront pas leurs fils selon toute cette loi ; car dans la chair de leur circoncision, ils omettront cette circoncision de leurs fils, et tous, fils de Beliar, laisseront leurs fils incirconcis comme ils sont nés.
\par 34 Et il y aura une grande colère de la part de l'Éternel contre les enfants d'Israël. parce qu'ils ont abandonné son alliance et se sont détournés de sa parole, et ont provoqué et blasphémé, dans la mesure où ils n'observent pas l'ordonnance de cette loi ; car ils ont traité leurs membres comme les païens, afin qu'ils soient expulsés et déracinés du pays. Et il n’y aura plus de pardon pour eux [afin qu’il y ait pardon et pardon] pour tout le péché de cette erreur éternelle.

\chapitre{16}

\par \textit{Des anges apparaissent à Abraham à Hébron et Isaac a de nouveau promis, 1-4. Destruction de Sodome et délivrance de Lot, 5-9. Abraham à Beer Sheva : naissance et circoncision d'Isaac, dont la postérité devait être la part de Dieu, 10-19. Institution de la fête des Tabernacles, 20-31. (Cf. Gen. XVIII.1, 10, 12 ; XIX.24, 29, 33-7 ; XX.1, 4, 8 ; XXI. 1-4.)}

\par 1 Et à la nouvelle lune du quatrième mois, nous sommes apparus à Abraham, au chêne de Mamré, et nous avons parlé avec lui, et nous lui avons annoncé qu'un fils lui serait donné par Sarah, sa femme.
\par 2 Et Sarah rit, car elle avait appris que nous avions dit ces paroles avec Abraham, et nous l'avons réprimandée, et elle a eu peur et a nié avoir ri à cause de ces paroles.
\par 3 Et nous lui avons dit le nom de son fils, comme son nom est ordonné et écrit dans les tablettes célestes (c'est-à-dire) Isaac,
\par 4 Et (que) lorsque nous reviendrons vers elle à une époque fixée, elle aurait conçu un fils.
\par 5 Et ce mois-ci, l'Éternel a exécuté ses jugements sur Sodome, et Gomorrhe, et Zeboim, et toute la région du Jourdain, et il les a brûlés par le feu et le soufre, et les a détruits jusqu'à ce jour, comme [lo] Je t'ai déclaré toutes leurs œuvres, qu'ils sont méchants et extrêmement pécheurs, qu'ils se souillent et commettent la fornication dans leur chair et qu'ils commettent des impuretés sur la terre.
\par 6 Et de la même manière, Dieu exécutera un jugement sur les lieux où ils ont agi selon l'impureté des Sodomites, comme le jugement de Sodome.
\par 7 Mais Lot, nous avons sauvé ; car Dieu se souvint d'Abraham et le renvoya du milieu du bouleversement.
\par 8 Et lui et ses filles commettèrent sur la terre un péché tel qu'il n'y en avait pas eu sur la terre depuis les jours d'Adam jusqu'à son époque ; car l'homme couchait avec ses filles.
\par 9 Et voici, il fut commandé et gravé concernant toute sa postérité, sur les tablettes célestes, de les enlever et de les déraciner, et d'exécuter sur eux un jugement comme le jugement de Sodome, et de ne laisser aucune postérité de l'homme. sur terre au jour de la condamnation.
\par 10 Et ce mois-là, Abraham quitta Hébron, partit et habita entre Kadesh et Shur, dans les montagnes de Guérar.
\par 11 Et au milieu du cinquième mois, il quitta de là et habita au puits du Serment.
\par 12 Et au milieu du sixième mois, le Seigneur visita Sarah et lui fit ce qu'il avait dit et elle conçut.
\par 13 Et elle enfanta un fils au troisième mois, et au milieu du mois, au temps dont l'Éternel avait parlé à Abraham, à la fête des prémices de la moisson, Isaac naquit.
\par 14 Et Abraham circoncit son fils le huitième jour : il fut le premier qui fut circoncis selon l'alliance qui est établie pour toujours.
\par 15 Et la sixième année de la quatrième semaine, nous sommes arrivés chez Abraham, au puits du serment, et nous lui sommes apparus [comme nous avions dit à Sarah que nous devions retourner vers elle, et qu'elle aurait conçu un fils.
\par 16 Et nous sommes revenus au septième mois, et avons trouvé Sarah enceinte devant nous] et nous l'avons béni, et nous lui avons annoncé toutes les choses qui avaient été décrétées à son sujet, afin qu'il ne meure pas avant d'avoir engendré six fils. plus, et devrait les voir avant de mourir ; mais (que) en Isaac son nom et sa postérité devraient être appelés :
\par 17 Et (que) toute la postérité de ses fils serait des Gentils, et serait comptée avec les Gentils ; mais des fils d'Isaac, un seul devrait devenir une postérité sainte et ne devrait pas être compté parmi les Gentils.
\par 18 Car il devait devenir la part du Très-Haut, et toute sa postérité était tombée en possession de Dieu, afin qu'elle devienne pour l'Éternel un peuple pour (sa) possession au-dessus de toutes les nations et qu'elle devienne un royaume. et des prêtres et une nation sainte.
\par 19 Et nous sommes partis, et nous avons annoncé à Sara tout ce que nous lui avions dit, et ils se sont tous deux réjouis d'une très grande joie.
\par 20 Et il bâtit là un autel à l'Éternel qui l'avait délivré et qui le faisait se réjouir dans le pays de son séjour, et il célébra une fête de joie en ce mois sept jours, près de l'autel qu'il avait bâti. au Puits du Serment.
\par 21 Et il bâtit des tentes pour lui et pour ses serviteurs lors de cette fête, et il fut le premier à célébrer la fête des tabernacles sur la terre.
\par 22 Et pendant ces sept jours, il apporta chaque jour à l'autel un holocauste à l'Éternel, deux bœufs, deux béliers, sept brebis et un bouc, en sacrifice d'expiation, afin qu'il puisse ainsi expier pour lui-même et pour sa graine.
\par 23 Et, en offrande de remerciement, sept béliers, sept chevreaux, sept brebis et sept boucs, et leurs offrandes de fruits et leurs libations ; Et il brûla toute la graisse sur l'autel, offrande choisie à l'Éternel, d'odeur agréable.
\par 24 Et matin et soir il brûlait des substances odorantes, de l'encens et du galbanum, et du stackte, et du nard, et de la myrrhe, et des épices, et du costum ; il offrit tous ces sept, écrasés, mélangés ensemble à parts égales (et) purs.
\par 25 Et il célébra cette fête pendant sept jours, se réjouissant de tout son cœur et de toute son âme, lui et tous ceux qui étaient dans sa maison, et il n'y avait aucun étranger avec lui, ni aucun incirconcis.
\par 26 Et il bénit son Créateur qui l'avait créé dans sa génération, car il l'avait créé selon son bon plaisir ; car il savait et comprit que de lui naîtrait la plante de la justice pour les générations éternelles, et de lui une semence sainte, afin qu'elle devienne semblable à celui qui avait fait toutes choses.
\par 27 Et il bénit et se réjouit, et il appela cette fête le nom de fête du Seigneur, une joie agréable au Dieu Très-Haut.
\par 28 Et nous l'avons béni pour toujours, ainsi que toute sa postérité après lui, à travers toutes les générations de la terre, parce qu'il célébrait cette fête en son temps, selon le témoignage des tablettes célestes.
\par 29 C'est pourquoi il est écrit sur les tablettes célestes concernant Israël qu'ils célébreront avec joie la fête des tabernacles pendant sept jours, le septième mois, ce qui sera agréable devant l'Éternel, statut perpétuel pour leurs générations, chaque année.
\par 30 Et pour cela, il n'y a pas de limite de jours ; car il est ordonné à jamais à Israël de la célébrer et d'habiter dans des tentes, de mettre des couronnes sur sa tête, et de prendre des branches feuillues et des saules du ruisseau.
\par 31 Et Abraham prit des branches de palmiers et des fruits de beaux arbres, et chaque jour faisant le tour de l'autel avec les branches sept fois [par jour] le matin, il loua et rendit grâces à son Dieu pour tout ce qui se passait en joie.

\chapitre{17}

\par \textit{Expulsion d'Agar et d'Ismaël, 1-14. Mastêmâ propose que Dieu exige d'Abraham qu'il sacrifie Isaac afin de tester son amour et son obéissance : les dix épreuves d'Abraham, 15-18. (Cf. Gen.xxi.8-21.)}

\par 1 Et la première année de la cinquième semaine, Isaac fut sevré lors de ce jubilé, [1982 AM] et Abraham fit un grand banquet le troisième mois, le jour où son fils Isaac fut sevré.
\par 2 Et Ismaël, fils d'Agar, l'Égyptienne, était devant Abraham, son père, à sa place, et Abraham se réjouit et bénit Dieu parce qu'il avait vu ses fils et n'était pas mort sans enfants.
\par 3 Et il se souvint des paroles qu'il lui avait dites le jour où Lot s'était séparé de lui, et il se réjouit parce que l'Éternel lui avait donné une semence sur la terre pour hériter de la terre, et il bénit de toute sa bouche. le Créateur de toutes choses.
\par 4 Et Sarah vit Ismaël jouer et danser, et Abraham se réjouir d'une grande joie, et elle devint jalouse d'Ismaël et dit à Abraham : « Chasse cette esclave et son fils ; car le fils de cette esclave n'héritera pas avec mon fils Isaac.
\par 5 Et la chose était pénible aux yeux d'Abraham, à cause de sa servante et à cause de son fils, qu'il les chasse de lui.
\par 6 Et Dieu dit à Abraham : Que cela ne te soit pas pénible à cause de l'enfant et de la servante ; Dans tout ce que Sarah t'a dit, écoute ses paroles et mets-les en pratique. car c'est en Isaac que ton nom et ta postérité seront appelés.
\par 7 «Mais quant au fils de cette esclave, je ferai de lui une grande nation, car il est de ta postérité.»
\par 8 Et Abraham se leva de bon matin, prit du pain et une bouteille d'eau, les plaça sur les épaules d'Agar et de l'enfant, et la renvoya.
\par 9 Et elle partit et erra dans le désert de Beer-Sheva, et l'eau de la bouteille était épuisée, et l'enfant avait soif, et ne pouvait pas avancer, et tomba.
\par 10 Et sa mère le prit et le jeta sous un olivier, et alla s'asseoir contre lui, à distance d'un coup d'arc ; car elle dit : « Ne me laisse pas voir la mort de mon enfant », et tandis qu'elle était assise, elle pleurait.
\par 11 Et un ange de Dieu, l'un des saints, lui dit : Pourquoi pleures-tu, Agar ? Lève-toi, prends l’enfant et tiens-le dans ta main ; car Dieu a entendu ta voix et a vu l'enfant.
\par 12 Et elle ouvrit les yeux, et elle vit un puits d'eau, et elle alla remplir sa outre d'eau, et elle donna à boire à son enfant, et elle se leva et partit vers le désert de Paran.
\par 13 Et l'enfant grandit et devint archer, et Dieu était avec lui, et sa mère lui prit une femme parmi les filles d'Égypte.
\par 14 Et elle lui enfanta un fils, et il appela son nom Nebaioth ; car elle dit : « Le Seigneur était près de moi lorsque je l'invoquai. »
\par 15 Et il arriva que la septième semaine de la première année de celle-ci, [2003 AM] le premier mois de ce jubilé, le douzième de ce mois, des voix se firent dans le ciel concernant Abraham, qu'il était fidèle. dans tout ce qu'il lui disait, et qu'il aimait le Seigneur, et que dans toutes les afflictions il était fidèle.
\par 16 Et le prince Mastêmâ vint et dit devant Dieu : Voici, Abraham aime Isaac son fils, et il prend plaisir en lui par-dessus tout ; ordonne-lui de l'offrir en holocauste sur l'autel, et tu verras s'il exécutera cet ordre, et tu sauras s'il est fidèle dans tout ce que tu l'éprouves.
\par 17 Et l'Éternel savait qu'Abraham était fidèle dans toutes ses afflictions ; car il l'avait éprouvé par son pays et par la famine, et il l'avait éprouvé par les richesses des rois, et il l'avait encore éprouvé par sa femme, lorsqu'elle avait été arrachée (à lui), et par la circoncision ; et il l'avait éprouvé par l'intermédiaire d'Ismaël et d'Agar, sa servante, lorsqu'il les renvoya.
\par 18 Et dans tout ce qu'il l'avait éprouvé, il fut trouvé fidèle, et son âme n'était pas impatiente, et il ne tardait pas à agir ; car il était fidèle et ami du Seigneur.

\chapitre{18}

\par \textit{Sacrifice d'Isaac : Mastêmâ honteux, 1-13. Abraham à nouveau béni : retourne à Beer Sheva 14-19. (Cf. Gen. XXII. 1-19.)}

\par 1 Et Dieu lui dit : Abraham, Abraham ; et il dit : « Me voici. »
\par 2 Et il dit : Prends ton fils bien-aimé que tu aimes, Isaac, et va dans les hauteurs, et offre-le sur l'une des montagnes que je te montrerai.
\par 3 Et il se leva de bon matin et sella son âne, et prit avec lui ses deux jeunes hommes et Isaac son fils, et il fendit le bois de l'holocauste, et il se rendit sur place le troisième jour, et il a vu l'endroit de loin.
\par 4 Et il arriva à un puits d'eau, et il dit à ses jeunes gens : « Demeurez ici avec l'âne, et moi et le garçon irons (là-bas), et quand nous aurons adoré, nous reviendrons vers vous. .'
\par 5 Et il prit le bois de l'holocauste et le déposa sur Isaac, son fils, et il prit dans sa main le feu et le couteau, et ils partirent tous deux ensemble vers ce lieu.
\par 6 Et Isaac dit à son père : « Père ; et il dit : « Me voici, mon fils. Et il lui dit : Voici le feu, le couteau et le bois ; mais où est la brebis pour l'holocauste, père ?
\par 7 Et il dit : Dieu se procurera une brebis pour l'holocauste, mon fils. Et il s'approcha du lieu de la montagne de Dieu.
\par 8 Et il bâtit un autel, et il plaça le bois sur l'autel, et lia Isaac son fils, et le plaça sur le bois qui était sur l'autel, et étendit sa main pour prendre le couteau pour tuer Isaac son fils. .
\par 9 Et je me présentai devant lui, et devant le prince Mastêmâ, et l'Éternel dit : 'Dites-lui de ne pas porter la main sur l'enfant, ni de lui faire quoi que ce soit, car j'ai montré qu'il craint l'Éternel.'
\par 10 Et je l'appelai du ciel, et je lui dis : Abraham, Abraham ; et il fut terrifié et dit : « Me voici. »
\par 11 Et je lui dis : Ne pose pas la main sur l'enfant et ne lui fais rien ; car maintenant j'ai montré que tu crains l'Éternel, et que tu ne m'as pas refusé ton fils, ton premier-né.
\par 12 Et le prince Mastêmâ fut honteux ; Et Abraham leva les yeux et regarda, et voici, un bélier était attrapé. . . par ses cornes, et Abraham alla prendre le bélier et l'offrit en holocauste à la place de son fils.
\par 13 Et Abraham appela ce lieu «L'Éternel a vu», de sorte qu'il est dit \textit{dans la montagne} l'Éternel a vu : c'est le mont Sion.
\par 14 Et le Seigneur appela Abraham par son nom une seconde fois du ciel, alors qu'il nous faisait apparaître pour lui parler au nom du Seigneur.
\par 15 Et il dit : J'ai juré par moi-même, dit l'Éternel,
\par    
\par     Parce que tu as fait cette chose,  
\par     Et tu ne m'as pas refusé ton fils, ton fils bien-aimé,  
\par     Qu'en bénédiction je te bénirai,
\par    
\par     Et en multipliant je multiplierai ta semence  
\par     Comme les étoiles du ciel, Et comme le sable qui est au bord de la mer.
\par    
\par     Et ta postérité héritera des villes de ses ennemis,  
\par    
\par 16 Et en ta postérité seront bénies toutes les nations de la terre ;
\par    
\par     Parce que tu as obéi à ma voix,  
\par     Et j’ai montré à tous que tu m’es fidèle dans tout ce que je t’ai dit :
\par    
\par     Vas en paix.'
\par    
\par 17 Et Abraham alla vers ses jeunes hommes, et ils se levèrent et allèrent ensemble à Beer Sheva, et Abraham [2010 AM] habita près du puits du Serment.
\par 18 Et il célébrait cette fête chaque année, sept jours avec joie, et il l'appelait la fête du Seigneur selon les sept jours pendant lesquels il allait et revenait en paix.
\par 19 Et en conséquence, il a été ordonné et écrit sur les tablettes célestes concernant Israël et sa postérité qu'ils devraient observer cette fête pendant sept jours avec la joie de la fête.

\chapitre{19}

\par \textit{Retour d'Abraham à Hébron. Mort et enterrement de Sarah, 1-9. Mariage d'Isaac et second mariage d'Abraham. Naissance d'Ésaü et Jacob, 10-14. Abraham recommande Jacob à Rébecca et le bénit, 15-31. (Cf. Gen. xxiii.1-4, 11-16 ; xxiv.15 ; xxv.1-2, 25-7 ; xiii. 16.)}

\par 1 Et la première année de la première semaine du quarante-deuxième jubilé, Abraham revint et habita en face d'Hébron, qui est Kirjath Arba, pendant deux semaines d'années.
\par 2 Et la première année de la troisième semaine de ce jubilé, les jours de la vie de Sara furent accomplis, et elle mourut à Hébron.
\par 3 Et Abraham est allé la pleurer et l'a enterrée, et nous l'avons essayé [pour voir] si son esprit était patient et s'il n'était pas indigné dans les paroles de sa bouche ; et il fut trouvé patient en cela et ne fut pas dérangé.
\par 4 Car, avec patience d'esprit, il s'entretenait avec les enfants de Heth, dans l'intention qu'ils lui donnaient un lieu où enterrer ses morts.
\par 5 Et l'Éternel lui fit grâce devant tous ceux qui le voyaient, et il supplia avec douceur les fils de Heth, et ils lui donnèrent le pays de la double grotte, en face de Mamré, c'est-à-dire Hébron, pour quatre cents pièces d'argent.
\par 6 Et ils le supplièrent en disant : Nous te le donnerons gratuitement ; mais il ne voulut pas le leur enlever pour rien, car il leur donna le prix du lieu, l'argent en totalité, et il se prosterna devant eux deux fois, et après cela il enterra ses morts dans la double grotte.
\par 7 Et tous les jours de la vie de Sara furent de cent vingt-sept ans, soit deux jubilés et quatre semaines et un an : ce sont les jours des années de la vie de Sara.
\par 8 C'est la dixième épreuve par laquelle Abraham fut éprouvé, et il fut trouvé fidèle, patient d'esprit.
\par 9 Et il ne dit pas un seul mot concernant la rumeur qui courait dans le pays selon laquelle Dieu avait dit qu'il le donnerait à lui et à sa postérité après lui, et il demanda là un endroit pour enterrer ses morts ; car il fut trouvé fidèle et fut enregistré sur les tablettes célestes comme l'ami de Dieu.
\par 10 Et la quatrième année de celui-ci, il prit une femme pour son fils Isaac et elle s'appelait Rebecca [2020 AM] [la fille de Bethuel, le fils de Nahor, le frère d'Abraham] la sœur de Laban et la fille de Bethuel ; et Bethuel était le fils de Melca, qui était la femme de Nahor, le frère d'Abraham.
\par 11 Et Abraham prit une troisième femme, et elle s'appelait Ketura, parmi les filles de ses domestiques, car Agar était morte avant Sara. Et elle lui enfanta six fils, Zimram, et Jokshan, et Medan, et Madian, et Ishbak, et Shuah, en deux semaines d'années.
\par 12 Et la sixième semaine de la deuxième année, Rébecca enfanta à Isaac deux fils, Jacob et Ésaü,
\par 13 et [2046 AM] Jacob était un homme lisse et droit, et Ésaü était féroce, un homme des champs et poilu, et Jacob habitait sous des tentes.
\par 14 Et les jeunes gens grandissaient, et Jacob apprit à écrire ; mais Ésaü ne l'apprit pas, car il était homme des champs et chasseur, et il apprit la guerre, et toutes ses actions étaient féroces.
\par 15 Et Abraham aimait Jacob, mais Isaac aimait Ésaü.
\par 16 Et Abraham vit les actions d'Ésaü, et il sut que c'était en Jacob que son nom et sa postérité seraient appelés ; et il appela Rébecca et lui donna des commandements concernant Jacob, car il savait qu'elle (aussi) aimait Jacob beaucoup plus qu'Ésaü.
\par 17 Et il lui dit :
\par    
\par     Ma fille, veille sur mon fils Jacob,  
\par     Car il sera à ma place sur la terre,  
\par     Et pour une bénédiction au milieu des enfants des hommes,  
\par     Et pour la gloire de toute la postérité de Sem.
\par    
\par 18 Car je sais que le Seigneur le choisira pour être son peuple de possession, entre tous les peuples qui sont sur la face de la terre.
\par 19 Et voici, Isaac, mon fils, aime Ésaü plus que Jacob, mais je vois que tu aimes vraiment Jacob.
\par    
\par 20 Augmente encore ta bonté envers lui,  
\par     Et que tes yeux soient sur lui avec amour ;  
\par     Car il sera désormais pour nous une bénédiction sur la terre et pour toutes les générations de la terre.
\par    
\par 21 Que tes mains soient fortes  
\par     Et que ton cœur se réjouisse en ton fils Jacob ;  
\par     Car je l'ai aimé bien plus que tous mes fils.
\par    
\par     Il sera béni pour toujours,  
\par     Et sa semence remplira toute la terre.
\par    
\par 22 Si un homme peut compter le sable de la terre,  
\par     Sa postérité sera également comptée.
\par    
\par 23 Et toutes les bénédictions dont l'Éternel m'a béni, ainsi que ma postérité, appartiendront toujours à Jacob et à sa postérité.
\par 24 Et en sa postérité mon nom sera béni, ainsi que le nom de mes pères, Sem, et Noab, et Enoch, et Mahalalel, et Enos, et Seth, et Adam.
\par 25 Et ceux-ci serviront
\par    
\par     Pour poser les fondations du ciel,  
\par     Et pour fortifier la terre,  
\par     Et pour renouveler tous les luminaires qui sont au firmament.
\par    
\par 26 Et il appela Jacob devant les yeux de Rébecca, sa mère, et l'embrassa, et le bénit, et dit :
\par 27 « Jacob, mon fils bien-aimé, que mon âme aime, que Dieu te bénisse du haut du firmament, et qu'il te donne toutes les bénédictions dont il a béni Adam, Enoch, Noé et Sem ; et tout ce qu'il m'a dit, et tout ce qu'il a promis de me donner, qu'il fasse s'attacher à toi et à ta postérité pour toujours, selon les jours du ciel au-dessus de la terre.
\par 28 'Et les esprits de Mastêmâ ne domineront pas sur toi ni sur ta postérité pour te détourner du Seigneur, qui est ton Dieu désormais et pour toujours.'
\par 29 «Et que le Seigneur Dieu soit pour toi et toi, le fils premier-né, un père, et pour le peuple à jamais.»
\par 30 'Va en paix, mon fils.' Et ils sortirent tous deux ensemble d’Abraham.
\par 31 Et Rébecca aimait Jacob de tout son cœur et de toute son âme, bien plus qu'Ésaü ; mais Isaac aimait Ésaü bien plus que Jacob.

\chapitre{20}

\par \textit{Abraham exhorte ses fils et les fils de ses fils à pratiquer la justice, à observer la circoncision et à s'abstenir de l'impureté et de l'idolâtrie, 1-10. Les renvoie avec des cadeaux, 11. Résidences des Ismaélites et des fils de Ketura, 12-13. (Cf. Gen. XXV. 5-6.)}

\par 1 Et au quarante-deuxième jubilé, la première année de la septième semaine, Abraham appela Ismaël, [2052 (2045?) AM] et ses douze fils, et Isaac et ses deux fils, et les six fils de Ketura. , et leurs fils.
\par 2 Et il leur commanda d'observer la voie du Seigneur ; qu'ils devraient pratiquer la justice, aimer chacun son prochain, et agir de cette manière parmi tous les hommes ; afin qu'ils marchent chacun à leur égard de manière à exercer le jugement et la justice sur la terre.
\par 3 Afin qu'ils circoncissent leurs fils, selon l'alliance qu'il avait conclue avec eux, et qu'ils ne s'écartent ni à droite ni à gauche de tous les sentiers que l'Éternel nous avait commandés ; et que nous devrions nous garder de toute fornication et impureté, [et renoncer parmi nous à toute fornication et impureté].
\par 4 Et si une femme ou une servante parmi vous se livre à la fornication, brûlez-la au feu et qu'elle ne commette pas de fornication avec elle selon leurs yeux et leur cœur ; et qu'ils ne prennent pas pour femmes des filles de Canaan ; car la graine de Canaan sera arrachée du pays.
\par 5 Et il leur raconta le jugement des géants et le jugement des Sodomites, comment ils avaient été jugés à cause de leur méchanceté et étaient morts à cause de leur fornication, de leur impureté et de leur corruption mutuelle par fornication.
\par    
\par 6 « Et gardez-vous de toute fornication et impureté,  
\par     Et de toute pollution du péché,
\par    
\par     De peur que vous ne fassiez de notre nom une malédiction,  
\par     Et toute ta vie un sifflement,
\par    
\par     Et tous tes fils seront détruits par l'épée,  
\par     Et vous deviendrez maudits comme Sodome,  
\par     Et tout votre reste comme fils de Gomorrhe.
\par    
\par 7 Je vous en supplie, mes fils, aimez le Dieu du ciel  
\par     Et attachez-vous à tous ses commandements.
\par    
\par     Et ne marchez pas après leurs idoles et après leurs impuretés,  
\par    
\par 8 Et ne vous faites pas des dieux en fonte ou gravés ;
\par    
\par     Car ils sont vanité,  
\par     Et il n’y a pas d’esprit en eux ;
\par    
\par     Car ils sont l'ouvrage de mains (d'hommes),  
\par     Et tous ceux qui leur font confiance ne font confiance à rien.
\par    
\par 9 Ne les servez pas et ne les adorez pas,  
\par     Mais servez le Dieu Très-Haut et adorez-le continuellement :  
\par     Et espère toujours son visage,  
\par     Et pratiquez l'intégrité et la justice devant Lui,
\par    
\par     Afin qu'il prenne plaisir en vous et vous accorde sa miséricorde,  
\par     Et envoie de la pluie sur toi matin et soir,
\par    
\par     Et bénissez toutes vos œuvres que vous avez faites sur la terre,  
\par     Et bénis ton pain et ton eau,
\par    
\par     Et bénis le fruit de tes entrailles et le fruit de ta terre,  
\par     Et les troupeaux de ton bétail et les troupeaux de tes brebis.
\par    
\par 10 Et vous serez une bénédiction sur la terre,  
\par     Et toutes les nations de la terre te désireront,
\par    
\par     Et bénis tes fils en mon nom,  
\par     Afin qu'ils soient bénis comme moi.
\par    
\par 11 Et il donna à Ismaël et à ses fils, et aux fils de Ketura, des cadeaux, et les renvoya de chez Isaac, son fils, et il donna tout à Isaac, son fils.
\par 12 Et Ismaël et ses fils, et les fils de Ketura et leurs fils, partirent ensemble et habitèrent depuis Paran jusqu'à l'entrée de Babylone dans tout le pays qui est à l'est, face au désert.
\par 13 Et ceux-ci se mêlèrent les uns aux autres, et leur nom fut appelé Arabes et Ismaélites.

\chapitre{21}

\par \textit{Les dernières paroles d'Abraham à Isaac concernant l'idolâtrie, la consommation de sang, l'offrande de divers sacrifices et l'usage du sel, 1-11. Concernant également les bois à utiliser pour le sacrifice et le devoir de se laver avant le sacrifice et de couvrir le sang, etc., 12-25.}

\par 1 Et la sixième année de la septième semaine de ce jubilé, Abraham appela Isaac son fils, et [2057 (2050?) AM] lui commanda : en disant : « Je suis devenu vieux, et je ne connais pas le jour de ma mort, et je suis rempli de mes journées.
\par 2 'Et voici, j'ai cent soixante-quinze ans, et pendant tous les jours de ma vie je me suis souvenu du Seigneur et j'ai cherché de tout mon cœur à faire sa volonté et à marcher droit dans toutes ses façons.'
\par 3 'Mon âme a haï les idoles (et j'ai méprisé ceux qui les servaient, et j'ai donné mon cœur et mon esprit) afin de pouvoir faire la volonté de Celui qui m'a créé.'
\par 4 'Car Il est le Dieu vivant, et Il est saint et fidèle, et Il est juste entre tous, et il n'y a avec Lui aucune acceptation des personnes (des hommes) ni aucune acceptation des dons ; car Dieu est juste et il exécute son jugement sur tous ceux qui transgressent ses commandements et méprisent son alliance.
\par 5 'Et toi, mon fils, observe ses commandements, ses ordonnances et ses jugements, et ne marche pas après les abominations, ni après les images taillées, ni après les images en fonte.'
\par 6 'Et ne mangez aucun sang d'animaux, ni de bétail, ni de tout oiseau qui vole dans le ciel.'
\par 7 «Et si vous tuez une victime comme offrande de paix agréable, tuez-la et versez son sang sur l'autel, et toute la graisse de l'offrande sur l'autel avec de la farine fine et l'offrande de viande mêlée à l'huile et sa libation - offrez-les tous ensemble sur l'autel des holocaustes ; c'est une douce saveur devant le Seigneur.
\par 8 « Et tu offriras la graisse du sacrifice de remerciement sur le feu qui est sur l'autel, et la graisse qui est sur le ventre, et toute la graisse des entrailles et des deux rognons, et toute la graisse c'est sur eux, et tu enlèveras les reins et le foie, ainsi que les rognons.
\par 9 'Et offrez tout cela, d'agréable odeur, agréable devant l'Éternel, avec son offrande de viande et avec sa libation, d'agréable odeur, le pain de l'offrande à l'Éternel.'
\par 10 Et mangez sa viande ce jour-là et le deuxième jour, et que le soleil ne se couche pas sur elle le deuxième jour jusqu'à ce qu'elle soit mangée, et qu'il ne reste rien pour le troisième jour ; car ce n'est pas acceptable [car cela n'est pas approuvé] et qu'on n'en mange plus, et tous ceux qui en mangent attireront sur eux le péché ; car c'est ainsi que je l'ai trouvé écrit dans les livres de mes ancêtres, dans les paroles d'Hénoc et dans les paroles de Noé.
\par 11 'Et tu répandras du sel sur toutes tes oblations, et que le sel de l'alliance ne manque pas dans toutes tes oblations devant l'Éternel.'
\par 12 «Et quant au bois des sacrifices, prends garde que tu n'apportes (d'autres) bois pour l'autel en plus de ceux-ci : cyprès, laurier, amandier, sapin, pin, cèdre, savin, figuier, olivier, myrrhe, laurier.» , aspalathus.'
\par 13 « Et déposez sur l'autel sous le sacrifice ces sortes de bois dont l'apparence a été testée, et ne posez pas (dessus) de bois fendu ou foncé, (mais) dur et propre, sans faute. , une croissance saine et nouvelle ; et ne posez pas dessus du vieux bois, [car son parfum a disparu] car il n'y a plus de parfum dedans comme avant.'
\par 14 'En dehors de ces espèces de bois, il n'y en a pas d'autre que tu placeras (sur l'autel), car son parfum est dispersé, et l'odeur de son parfum ne monte pas jusqu'au ciel.'
\par 15 'Observe ce commandement et fais-le, mon fils, afin que tu sois droit dans toutes tes actions.'
\par 16 Et sois en tout temps pur dans ton corps, et lave-toi avec de l'eau avant de t'approcher pour offrir sur l'autel, et lave-toi les mains et les pieds avant de t'approcher de l'autel ; et quand tu auras fini de sacrifier, lave-toi encore les mains et les pieds.
\par 17 « Et qu'aucun sang n'apparaisse sur vous ni sur vos vêtements ; sois sur tes gardes, mon fils, contre le sang, sois sur tes gardes extrêmement ; couvre-le de poussière.
\par 18 'Et ne mange pas de sang car c'est l'âme ; ne mange pas de sang du tout.
\par 19 « Et n'acceptez pas de dons pour le sang de l'homme, de peur qu'il ne soit versé impunément, sans jugement ; car c'est le sang versé qui fait pécher la terre, et la terre ne peut être purifiée du sang de l'homme que par le sang de celui qui l'a versé.
\par 20 'Et n'accepte aucun présent ou présent pour le sang de l'homme : sang pour sang, afin que tu sois accepté devant l'Éternel, le Dieu Très-Haut ; car Il est la défense des bons, et afin que tu sois préservé de tout mal, et qu'Il puisse te sauver de toute sorte de mort.
\par    
\par 21 Je vois, mon fils,  
\par     Que toutes les œuvres des enfants des hommes sont péché et méchanceté,  
\par     Et toutes leurs actions sont des impuretés, des abominations et des souillures,  
\par     Et il n’y a pas de justice chez eux.
\par    
\par 22 Prends garde, de peur que tu ne marches dans leurs voies  
\par     Et marche sur leurs chemins,  
\par     Et péchez un péché jusqu'à la mort devant le Dieu Très-Haut.
\par    
\par     Sinon, il te cachera sa face  
\par     Et] te remets entre les mains de ta transgression,  
\par     Et je t'arracherai du pays, ainsi que ta semence de dessous les cieux,  
\par     Et ton nom et ta postérité périront de toute la terre.
\par    
\par 23 Détournez-vous de toutes leurs actions et de toutes leurs impuretés,  
\par     Et observez l'ordonnance du Dieu Très-Haut,  
\par     Et faites sa volonté et soyez honnête en toutes choses.
\par    
\par 24 Et il te bénira dans toutes tes actions,  
\par     Et fera germer de toi une plante de justice sur toute la terre, dans toutes les générations de la terre,  
\par     Et mon nom et ton nom ne seront pas oubliés à jamais sous le ciel.
\par    
\par 25 Va, mon fils en paix.  
\par     Que le Dieu Très-Haut, mon Dieu et ton Dieu, te fortifie pour faire sa volonté,  
\par     Et puisse-t-il bénir toute ta postérité et le reste de ta postérité pour les générations et pour toujours, avec toutes les justes bénédictions,  
\par     Afin que tu sois une bénédiction sur toute la terre.
\par    
\par 26 Et il sortit de chez lui en se réjouissant.

\chapitre{22}

\par \textit{Isaac, Ismaël et Jacob célèbrent la fête des prémices à Beer Sheva avec Abraham, 1-5. Prière d'Abraham, 6-9. Les dernières paroles d'Abraham et les bénédictions de Jacob, 10-30.}

\par 1 Et il arriva la première semaine du quarante-quatrième jubilé, la deuxième année, c'est-à-dire l'année où Abraham mourut, qu'Isaac et Ismaël sortirent du Puits du Serment pour célébrer la fête de semaines, c'est-à-dire la fête des prémices de la moisson, à Abraham, leur père, et Abraham se réjouit parce que ses deux fils étaient venus.
\par 2 Car Isaac avait de nombreuses possessions à Beer Sheva, et Isaac avait l'habitude d'aller voir ses possessions et de retourner chez son père.
\par 3 Et en ces jours-là, Ismaël vint voir son père, et ils s'assemblèrent tous deux, et Isaac offrit un sacrifice en holocauste, et le présenta sur l'autel de son père qu'il avait fait à Hébron.
\par 4 Et il offrit une offrande de remerciement et fit un festin de joie devant Ismaël, son frère. Et Rébecca fit de nouveaux gâteaux avec le nouveau grain, et les donna à Jacob, son fils, pour les apporter à Abraham, son père, de les prémices du pays, afin qu'il puisse manger et bénir le Créateur de toutes choses avant sa mort.
\par 5 Et Isaac envoya aussi par la main de Jacob à Abraham une meilleure offrande de remerciement, afin qu'il puisse manger et boire.
\par 6 Et il mangea et but, et bénit le Dieu Très-Haut,
\par    
\par     Qui a créé le ciel et la terre,  
\par     Qui a créé toutes les graisses de la terre,  
\par     Et je les ai donnés aux enfants des hommes  
\par     Afin qu'ils puissent manger et boire et bénir leur Créateur.
\par    
\par 7 « Et maintenant je te rends grâce, mon Dieu, parce que tu m'as fait voir ce jour : voici, j'ai cent soixante-quinze ans, un vieillard et rassasié de jours, et tous mes jours ont la paix m'a été accordée.
\par 8 'L'épée de l'adversaire ne m'a pas vaincu dans tout ce que tu m'as donné, à moi et à mes enfants, tous les jours de ma vie jusqu'à ce jour.'
\par 9 « Mon Dieu, que ta miséricorde et ta paix soient sur ton serviteur et sur la postérité de ses fils, afin qu'ils soient pour toi une nation choisie et un héritage parmi toutes les nations de la terre, désormais pour tous. » les jours des générations de la terre, jusqu'à tous les âges.
\par 10 Et il appela Jacob et dit : « Mon fils Jacob, que le Dieu de tous te bénisse et te fortifie pour faire la justice et sa volonté devant lui, et qu'il te choisisse, toi et ta postérité, afin que vous deveniez un peuple pour Son héritage selon Sa volonté toujours.
\par 11 'Et toi, mon fils, Jacob, approche-toi et embrasse-moi.' Et il s'approcha et l'embrassa, et il dit :
\par    
\par     'Béni soit mon fils Jacob  
\par     Et tous les fils du Dieu Très-Haut, dans tous les âges :
\par    
\par     Que Dieu te donne une semence de justice ;  
\par     Et qu'il sanctifie certains de tes fils au milieu de toute la terre ;
\par    
\par     Que les nations te servent,  
\par     Et toutes les nations s'inclinent devant ta postérité.
\par    
\par 12 Soyez forts devant les hommes,  
\par     Et exercez votre autorité sur toute la postérité de Seth.
\par    
\par     Alors tes voies et celles de tes fils seront justifiées,  
\par     Pour qu'ils deviennent une nation sainte.
\par    
\par 13 Que le Dieu Très Haut te donne toutes les bénédictions  
\par     Avec quoi il m'a béni
\par    
\par     Et avec quoi Il a béni Noé et Adam ;  
\par     Puissent-ils reposer sur la tête sacrée de ta semence de génération en génération pour toujours.
\par    
\par 14 Et qu'il te purifie de toute injustice et de toute impureté,  
\par     Afin que toutes tes transgressions te soient pardonnées ; que tu as commis par ignorance.
\par    
\par     Et qu'Il te fortifie,  
\par     Et te bénisse.  
\par     Et puisses-tu hériter de toute la terre,
\par    
\par 15 Et qu'il renouvelle son alliance avec toi.  
\par     Afin que tu sois pour lui une nation pour son héritage pour tous les siècles,  
\par     Et afin qu'il soit pour toi et pour ta postérité un Dieu en vérité et en justice pendant tous les jours de la terre.
\par    
\par 16 Et toi, mon fils Jacob, souviens-toi de mes paroles,  
\par     Et observe les commandements d'Abraham, ton père :
\par    
\par     Sépare-toi des nations,  
\par     Et ne mange pas avec eux :
\par    
\par     Et ne faites pas selon leurs œuvres,  
\par     Et ne devenez pas leur associé ;
\par    
\par     Car leurs œuvres sont impures,  
\par     Et toutes leurs voies sont une pollution, une abomination et une impureté.
\par    
\par 17 Ils offrent leurs sacrifices aux morts  
\par     Et ils adorent les mauvais esprits,
\par    
\par     Et ils mangent sur les tombes,  
\par     Et toutes leurs œuvres ne sont que vanité et néant.
\par    
\par 18 Ils n'ont pas de cœur pour comprendre  
\par     Et leurs yeux ne voient pas quelles sont leurs œuvres,
\par    
\par     Et comme ils se trompent en disant à un morceau de bois : « Tu es mon Dieu ».  
\par     Et à une pierre : « Tu es mon Seigneur et tu es mon libérateur. »  
\par     [Et ils n'ont pas de cœur.]
\par    
\par 19 Et quant à toi, mon fils Jacob,  
\par     Que le Dieu Très-Haut t'aide  
\par     Et que le Dieu du ciel te bénisse  
\par     Et éloigne-toi de leur impureté et de toute leur erreur.
\par    
\par 20 Garde-toi, mon fils Jacob, de prendre une femme parmi n'importe quelle postérité des filles de Canaan ;
\par    
\par     Car toute sa semence doit être arrachée de la terre.
\par    
\par 21 Car, à cause de la transgression de Cham, Canaan s'est égaré,  
\par     Et toute sa semence sera détruite de dessus la terre et tout ce qui en reste,  
\par     Et aucun de ses descendants ne sera sauvé au jour du jugement.
\par    
\par 22 Et quant à tous les adorateurs des idoles et des profanes  
\par     (b) Il n'y aura aucun espoir pour eux dans le pays des vivants ;  
\par     (c) Et il n'y aura aucun souvenir d'eux sur la terre ;  
\par     (c) Car ils descendront au schéol,  
\par     (d) Et ils iront au lieu de condamnation,
\par    
\par     Comme les enfants de Sodome furent enlevés de la terre  
\par     Ainsi seront expulsés tous ceux qui adorent les idoles.
\par    
\par 23 Ne crains rien, mon fils Jacob,  
\par     Et ne sois pas effrayé, ô fils d'Abraham !
\par    
\par     Que le Dieu Très-Haut te préserve de la destruction,  
\par     Et de tous les sentiers de l'erreur, qu'il te délivre.
\par    
\par 24 Je me suis bâti cette maison afin que je puisse y mettre mon nom sur la terre : [elle est donnée à toi et à ta postérité pour toujours], et elle sera appelée la maison d'Abraham ; il est donné à toi et à ta postérité pour toujours ; car tu bâtiras ma maison et établiras mon nom devant Dieu pour toujours : ta postérité et ton nom subsisteront à travers toutes les générations de la terre.
\par    
\par 25 Et il cessa de lui commander et de le bénir.
\par 26 Et tous deux se couchèrent ensemble sur un seul lit, et Jacob dormit dans le sein d'Abraham, le père de son père et il l'embrassa sept fois, et son affection et son cœur se réjouirent à son sujet.
\par 27 Et il le bénit de tout son cœur et dit : « Le Dieu Très-Haut, le Dieu de tous et le Créateur de tous, qui m'a fait sortir d'Ur en Chaldée afin de me donner ce pays pour en hériter pour pour toujours, et afin que je puisse établir une sainte semence, bénie soit le Très-Haut pour toujours.
\par 28 Et il bénit Jacob et dit : Mon fils, pour lequel je me réjouis de tout mon cœur et de toute mon affection, que ta grâce et ta miséricorde s'élèvent toujours sur lui et sur sa postérité.
\par 29 «Et ne l'abandonne pas, et ne le méprise pas désormais jusqu'aux jours de l'éternité, et que tes yeux soient ouverts sur lui et sur sa postérité, afin que tu puisses le préserver, le bénir et le sanctifier.» comme une nation pour ton héritage ;
\par 30 'Et bénis-le de toutes tes bénédictions dès désormais jusqu'à tous les jours de l'éternité, et renouvelle ton alliance et ta grâce avec lui et avec sa postérité selon tout ton bon plaisir pour toutes les générations de la terre.'

\chapitre{23}

\par \textit{Mort et enterrement d'Abraham, 1-8 (cf. Gen. xxv.7-10). Années décroissantes et corruption croissante de l'humanité : Malheurs messianiques : lutte universelle : les fidèles se lèvent en armes pour ramener les infidèles : Israël envahi par les pécheurs des Gentils, 11-25. Étude renouvelée de la loi et renouveau de l'humanité : Royaume messianique : immortalité bénie des justes, 26-31.}

\par 1 Et il plaça deux doigts de Jacob sur ses yeux, et il bénit le Dieu des dieux, et il se couvrit le visage et étendit ses pieds et dormit du sommeil de l'éternité, et fut recueilli auprès de ses pères.
\par 2 Et malgré tout cela, Jacob était couché dans son sein, et ne savait pas qu'Abraham, le père de son père, était mort.
\par 3 Et Jacob se réveilla de son sommeil, et voici, Abraham était froid comme la glace, et il dit : « Père, père » ; mais personne ne parlait, et il savait qu'il était mort.
\par 4 Et il se leva de son sein et courut le dire à Rébecca, sa mère ; Et Rébecca alla trouver Isaac pendant la nuit et lui rapporta ce qui se passait. Et ils s'en allèrent ensemble, et Jacob avec eux, et il avait une lampe à la main, et quand ils furent entrés, ils trouvèrent Abraham mort.
\par 5 Et Isaac tomba sur la face de son père, pleura et l'embrassa.
\par 6 Et des voix furent entendues dans la maison d'Abraham, et Ismaël son fils se leva, et alla vers Abraham son père, et pleura sur Abraham son père, lui et toute la maison d'Abraham, et ils pleurèrent avec de grands pleurs.
\par 7 Et ses fils Isaac et Ismaël l'enterrèrent dans la double grotte, près de Sarah sa femme, et ils le pleurèrent quarante jours, tous les hommes de sa maison, et Isaac et Ismaël, et tous leurs fils, et tous les fils. de Ketura à leur place ; et les jours où on pleurait Abraham furent terminés.
\par 8 Et il vécut trois jubilés et quatre semaines d'années, cent soixante-quinze ans, et accomplit les jours de sa vie, étant vieux et rempli de jours.
\par 9 Car les jours de la vie des ancêtres furent dix-neuf jubilés ; et après le déluge, ils commencèrent à croître de moins de dix-neuf jubilés, et à diminuer en jubilés, et à vieillir rapidement, et à être remplis de leurs jours à cause des multiples tribulations et de la méchanceté de leurs voies, à l'exception d'Abraham.
\par 10 Car Abraham était parfait dans toutes ses actions envers le Seigneur, et agréable en justice tous les jours de sa vie ; et voici, il n'accomplit pas quatre jubilés dans sa vie, alors qu'il était devenu vieux à cause de la méchanceté et qu'il était rassasié de ses jours.
\par 11 Et toutes les générations qui surgiront à partir de ce moment jusqu'au jour du grand jugement vieilliront rapidement, avant d'avoir accompli deux jubilés, et leur connaissance les abandonnera à cause de leur vieillesse. Terre toute leur connaissance sera disparaître].
\par 12 Et en ces jours-là, si un homme vit un jubilé et demi, on dira de lui : Il a vécu longtemps, et la plupart de ses jours sont des douleurs, des chagrins et des tribulations, et il y a pas de paix:'
\par 13 Car calamité sur calamité, et blessure sur blessure, et tribulation sur tribulation, et mauvaises nouvelles sur mauvaises nouvelles, et maladie sur maladie, et tous les mauvais jugements comme ceux-ci, les uns avec les autres, la maladie et le renversement, et la neige. et le gel, et la glace, et la fièvre, et les frissons, et la torpeur, et la famine, et la mort, et l'épée, et la captivité, et toutes sortes de calamités et de douleurs.
\par 14 Et tout cela viendra sur une génération méchante, qui transgresse sur la terre : leurs œuvres sont l'impureté et la fornication, et la pollution et les abominations.
\par 15 Alors ils diront : « Les jours des ancêtres ont été nombreux (même), jusqu'à mille ans, et ils ont été bons ; mais voici, les jours de notre vie, si un homme a vécu plusieurs, sont de soixante-dix ans, et s'il est fort, de quatre-vingts ans, et ces mauvais, et il n'y a pas de paix aux jours de cette génération méchante. .'
\par 16 Et dans cette génération, les fils convaincront leurs pères et leurs aînés de péché et d'injustice, et des paroles de leur bouche et des grandes méchancetés qu'ils commettent, et concernant leur abandon de l'alliance que l'Éternel a faite entre eux et lui. , afin qu'ils observent et mettent en pratique tous ses commandements, ses ordonnances et toutes ses lois, sans s'écarter ni à droite ni à gauche.
\par 17 Car tous ont fait le mal, et toute bouche parle d'iniquité et toutes leurs œuvres sont une impureté et une abomination, et toutes leurs voies sont pollution, impureté et destruction.
\par 18 Voici, la terre sera détruite à cause de tous leurs travaux, et il n'y aura ni semence de vigne, ni huile ; car leurs œuvres sont totalement infidèles, et ils périront tous ensemble, bêtes, bétail, oiseaux et tous les poissons de la mer, à cause des enfants des hommes.
\par 19 Et ils lutteront les uns contre les autres, les jeunes avec les vieux, et les vieux avec les jeunes, les pauvres avec les riches, les petits avec les grands, et les mendiants avec les princes, à cause de la loi et de la loi. engagement; car ils ont oublié les commandements, les alliances, les fêtes, les mois, les sabbats, les jubilés et tous les jugements.
\par 20 Et ils se tiendront debout (avec des arcs et) des épées et combattront pour les ramener sur le chemin ; mais ils ne reviendront que lorsque beaucoup de sang aura été versé les uns après les autres sur la terre.
\par 21 Et ceux qui se sont échappés ne retourneront pas de leur méchanceté dans la voie de la justice, mais ils s'exalteront tous jusqu'à la tromperie et à la richesse, afin que chacun prenne tout ce qui est à son prochain, et ils nommeront un grand nom, mais pas en vérité ni en justice, et ils souilleront le saint des saints par leur impureté et la corruption de leur souillure.
\par 22 Et un grand châtiment s'abattra sur les actions de cette génération de la part de l'Éternel, et il les livrera à l'épée, au jugement, à la captivité, et au pillage et à la dévoration.
\par 23 Et Il réveillera contre eux les pécheurs des Gentils, qui n'ont ni miséricorde ni compassion, et qui ne respecteront la personne de personne, ni vieux ni jeune, ni personne, car ils sont plus méchants et plus forts à faire. méchant que tous les enfants des hommes.
\par    
\par     Et ils useront de violence contre Israël et de transgression contre Jacob,  
\par     Et beaucoup de sang sera versé sur la terre,  
\par     Et il n’y aura personne à rassembler ni personne à enterrer.
\par    
\par 24 En ces jours-là, ils crieront à haute voix :  
\par     Et appelez et priez pour qu'ils soient sauvés de la main des pécheurs, les Gentils ;  
\par     Mais personne ne sera sauvé.
\par    
\par 25 Et les têtes des enfants seront blanches avec des cheveux gris,  
\par     Et un enfant de trois semaines paraîtra vieux comme un homme de cent ans,  
\par     Et leur stature sera détruite par la tribulation et l'oppression.
\par    
\par 26 Et en ces jours-là, les enfants commenceront à étudier les lois,  
\par     Et rechercher les commandements,  
\par     Et revenir sur le chemin de la justice.
\par    
\par 27 Et les jours commenceront à croître et à augmenter parmi ces enfants des hommes  
\par     Jusqu'à ce que leurs jours approchent de mille ans.  
\par     Et il y a eu un plus grand nombre d'années qu'avant le nombre des jours.
\par    
\par 28 Et il n'y aura pas de vieillard  
\par     Ni celui qui n'est (pas) satisfait de ses jours,  
\par     Car tous seront (comme) des enfants et des adolescents.
\par    
\par 29 Et ils accompliront tous leurs jours et vivront dans la paix et dans la joie,  
\par     Et il n’y aura ni Satan ni aucun destructeur malfaisant ;  
\par     Car tous leurs jours seront des jours de bénédiction et de guérison.
\par    
\par 30 Et à ce moment-là, l'Éternel guérira ses serviteurs,  
\par     Et ils se lèveront et verront une grande paix,  
\par     Et chasser leurs adversaires.
\par    
\par     Et les justes verront et seront reconnaissants,  
\par     Et réjouis-toi avec joie pour toujours et à jamais,  
\par     Et il verra tous leurs jugements et toutes leurs malédictions sur leurs ennemis.
\par    
\par 31 Et leurs os reposeront dans la terre,  
\par     Et leurs esprits connaîtront une grande joie,  
\par     Et ils sauront que c'est le Seigneur qui exécute le jugement,  
\par     Et fait preuve de miséricorde envers des centaines et des milliers et envers tous ceux qui l'aiment
\par    
\par 32 Et toi, Moïse, écris ces paroles ; car ainsi sont-ils écrits, et ils les inscrivent sur les tablettes célestes pour un témoignage pour les générations à jamais.

\chapitre{24}

\par \textit{Isaac au Puits de Vision, 1 (cf. Gen. xxv. 11). Ésaü vend son droit d'aînesse, 2-7 (cf. Gen. xxv.29-34).}

\par 1 Et il arriva après la mort d'Abraham, que l'Éternel bénit Isaac son fils, et il se leva d'Hébron et alla habiter au Puits de la Vision la première année de la troisième semaine [2073 AM] de ce jubilé, sept ans.
\par 2 Et la première année de la quatrième semaine, une famine commença dans le pays, [2080 AM] outre la première famine qui avait eu lieu du temps d'Abraham.
\par 3 Et Jacob mangea du potage de lentilles, et Ésaü revint des champs affamé. Et il dit à Jacob son frère : « Donne-moi de ce potage rouge. » Et Jacob lui dit : « Vends-moi ton droit d'aînesse et je te donnerai du pain et aussi un peu de ce potage de lentilles.
\par 4 Et Ésaü dit en son cœur : « Je mourrai ; à quoi me sert ce droit de naissance ?
\par 5 Et il dit à Jacob : 'Je te le donne.' Et Jacob dit : «Jure-le-moi aujourd'hui», et il lui jura.
\par 6 Et Jacob donna du pain et des lentilles à son frère Ésaü, et il mangea jusqu'à ce qu'il soit rassasié, et Ésaü méprisa son droit d'aînesse ; c'est pour cette raison que le nom d'Esaü était appelé Edom, à cause du potage rouge que Jacob lui avait donné en guise de droit d'aînesse.
\par 7 Et Jacob devint l'aîné, et Ésaü fut déchu de sa dignité.
\par 8 Et la famine s'abattit sur le pays, et Isaac partit pour descendre en Égypte la deuxième année de cette semaine, et se rendit chez le roi des Philistins à Guérar, vers Abimélec.
\par 9 Et le Seigneur lui apparut et lui dit : « Ne descends pas en Égypte ; habite dans le pays que je te dirai, et séjourne dans ce pays, et je serai avec toi et je te bénirai.
\par 10 Car à toi et à ta postérité je donnerai tout ce pays, et j'établirai mon serment que j'ai fait à Abraham ton père, et je multiplierai ta postérité comme les étoiles du ciel, et je donnerai à ta postérité toute cette terre.
\par 11 'Et en ta postérité toutes les nations de la terre seront bénies, parce que ton père a obéi à ma voix et a observé mes ordres et mes commandements, et mes lois, et mes ordonnances, et mon alliance ; et maintenant, obéissez à ma voix et demeurez dans ce pays.
\par 12 Et il demeura à Guélar trois semaines d'années.
\par 13 Et Abimélec donna des ordres à son sujet, [2080-2101 AM] et à propos de tout ce qui lui appartenait, en disant : 'Tout homme qui le touchera ou quoi que ce soit qui lui appartient mourra sûrement.'
\par 14 Et Isaac devint fort parmi les Philistins, et il acquit de nombreux biens, des bœufs et des moutons, des chameaux et des ânes, et une grande maison.
\par 15 Et il sema au pays des Philistins et rapporta au centuple, et Isaac devint extrêmement grand, et les Philistins l'envièrent.
\par 16 Tous les puits que les serviteurs d'Abraham avaient creusés pendant la vie d'Abraham, les Philistins les avaient bouchés après la mort d'Abraham et les avaient remplis de terre.
\par 17 Et Abimélec dit à Isaac : « Va-t-en de nous, car tu es beaucoup plus puissant que nous. » Et Isaac partit de là la première année de la septième semaine, et séjourna dans les vallées de Guérar.
\par 18 Et ils creusèrent de nouveau les puits d'eau que les serviteurs d'Abraham, son père, avaient creusés, et que les Philistins avaient fermés après la mort d'Abraham, son père, et il leur donna les noms comme Abraham, son père, les avait nommés.
\par 19 Et les serviteurs d'Isaac creusèrent un puits dans la vallée et trouvèrent de l'eau vive, et les bergers de Guérar se disputèrent avec les bergers d'Isaac, disant : L'eau est à nous ; et Isaac appela le nom du puits « Perversité », parce qu'ils avaient été pervers envers nous.
\par 20 Et ils creusèrent un deuxième puits, et ils s'efforçèrent également de l'obtenir, et il appela son nom «Inimitié». Et il se leva de là et ils creusèrent un autre puits, et pour cela ils ne cherchèrent pas, et il appela le nom de celui-ci « Chambre », et Isaac dit : « Maintenant, l'Éternel nous a fait de la place, et nous avons augmenté dans le pays. .'
\par 21 Et il monta de là au Puits du Serment la première année de la première semaine du quarante-quatrième jubilé.
\par 22 Et le Seigneur lui apparut cette nuit-là, à la nouvelle lune du premier mois, et lui dit : « Je suis le Dieu d'Abraham ton père ; ne crains rien, car je suis avec toi, je te bénirai et je multiplierai sûrement ta semence comme le sable de la terre, à cause d'Abraham mon serviteur.
\par 23 Et il bâtit là un autel qu'Abraham, son père, avait bâti le premier, et il invoqua le nom de l'Éternel, et il offrit un sacrifice au Dieu d'Abraham, son père.
\par 24 Et ils creusèrent un puits et ils trouvèrent de l'eau vive.
\par 25 Et les serviteurs d'Isaac creusèrent un autre puits et ne trouvèrent pas d'eau, et ils allèrent dire à Isaac qu'ils n'avaient pas trouvé d'eau, et Isaac dit : « J'ai juré aujourd'hui aux Philistins et cette chose a été annoncée à nous.'
\par 26 Et il appela le nom de cet endroit le Puits du Serment ; car là, il avait juré à Abimélec et à Ahuzzath son ami et à Phicol le préfet ou son hôte.
\par 27 Et Isaac comprit ce jour-là que, sous la contrainte, il leur avait juré de faire la paix avec eux.
\par 28 Et Isaac, ce jour-là, maudit les Philistins et dit : Maudits soient les Philistins jusqu'au jour de la colère et de l'indignation du milieu de toutes les nations ; que Dieu fasse d'eux une dérision et une malédiction et un objet de colère et d'indignation entre les mains des pécheurs les Gentils et entre les mains des Kittim.
\par 29 Et quiconque échappera à l'épée de l'ennemi et de Kittim, que la nation juste soit extirpée par le jugement de dessous le ciel ; car ils seront les ennemis et les ennemis de mes enfants à travers leurs générations sur la terre.
\par    
\par 30 Et il ne leur restera aucun reste,  
\par     Ni celui qui sera sauvé le jour de la colère du jugement ;  
\par     Car la destruction, l'extermination et l'expulsion de la terre sont toute la semence des Philistins (réservée),  
\par     Et il ne restera plus à ces Caphtorim ni nom ni graine sur la terre.
\par    
\par 31 Car s'il monte au ciel,  
\par     De là il sera descendu,
\par    
\par     Et bien qu'il se renforce sur terre,  
\par     De là il sera traîné,
\par    
\par     Et bien qu'il se cache parmi les nations,  
\par     Même de là il sera déraciné ;
\par    
\par     Et bien qu'il descende au schéol,  
\par     Là aussi sa condamnation sera grande,  
\par     Et là aussi, il n'aura pas de paix.  
\par    
\par 32 Et s'il va en captivité,  
\par     Ceux qui en veulent à sa vie le tueront en chemin,  
\par     Et ni nom ni postérité ne lui seront laissés sur toute la terre ;  
\par     Car il s'en ira dans la malédiction éternelle.
\par    
\par 33 Et ainsi est-il écrit et gravé à son sujet sur les tablettes célestes, de lui faire au jour du jugement, afin qu'il soit déraciné de la terre.

\chapitre{25}

\par \textit{Rebecca a averti Jacob de ne pas épouser une femme cananéenne, 1-3. Jacob promet d'épouser une fille de Laban malgré les demandes urgentes d'Ésaü lui demandant d'épouser une femme cananéenne, 4-10. Rebecca bénit Jacob, 11-23. (Cf. Gen. xxviii.1-4.)}

\par 1 Et la deuxième année de cette semaine, en ce jubilé, Rébecca appela Jacob son fils, et lui parla, disant : « Mon fils, ne te prends pas pour femme parmi les filles de Canaan, comme Ésaü , ton frère, qui lui a pris deux femmes parmi les filles de Canaan, et elles ont amer mon âme par toutes leurs actions impures ; car toutes leurs actions sont fornication et luxure, et il n'y a pas de justice chez elles, car (leurs actions) sont mal.'
\par 2 Et moi, mon fils, je t'aime extrêmement, et mon cœur et mon affection te bénissent à chaque heure du jour et à chaque veille de la nuit.
\par 3 Et maintenant, mon fils, écoute ma voix, et fais la volonté de ta mère, et ne te prends pas pour femme parmi les filles de ce pays, mais seulement de la maison de mon père et de la parenté de mon père. . Tu prendras une femme de la maison de mon père, et le Dieu Très-Haut te bénira, et tes enfants seront une génération juste et une postérité sainte.
\par 4 Alors Jacob parla à Rébecca, sa mère, et lui dit : Voici, mère, j'ai neuf semaines, et je ne connais ni n'ai touché aucune femme, et je ne me suis fiancé à aucune. je ne pense même pas à me prendre pour femme parmi les filles de Canaan.
\par 5 « Car je me souviens, mère, des paroles d'Abraham, notre père, car il m'a commandé de ne pas prendre une femme parmi les filles de Canaan, mais de me prendre une femme de la postérité de la maison de mon père et de ma parenté. .'
\par 6 'J'ai entendu dire auparavant que Laban, ton frère, avait donné naissance à des filles, et j'ai décidé de prendre une femme parmi elles.'
\par 7 'Et c'est pourquoi je me suis gardé dans mon esprit de pécher ou de me corrompre dans toutes mes voies, pendant tous les jours de ma vie ; car en ce qui concerne la luxure et la fornication, Abraham, mon père, m'a donné de nombreux commandements.
\par 8 «Et malgré tout ce qu'il m'a ordonné, depuis vingt-deux ans, mon frère a lutté avec moi, et il m'a souvent parlé et m'a dit : 'Mon frère, prends pour femme une sœur de mes deux femmes' ; mais je refuse de faire comme lui.
\par 9 'Je jure devant toi, mère, que tous les jours de ma vie je ne prendrai pas de femme parmi les filles de la postérité de Canaan, et que je n'agirai pas méchamment comme mon frère a fait.'
\par 10 « Ne crains rien, mère ; sois assuré que je ferai ta volonté et que je marcherai dans la droiture, et que je ne corromprai pas mes voies pour toujours.
\par 11 Et là-dessus elle leva son visage vers le ciel et étendit les doigts de ses mains, et ouvrit la bouche et bénit le Dieu Très-Haut, qui avait créé le ciel et la terre, et elle lui rendit grâces et louanges.
\par 12 Et elle dit : « Béni soit le Seigneur Dieu, et que son saint nom soit béni pour toujours et à jamais, qui m'a donné Jacob comme un fils pur et une sainte postérité ; car il est à toi, et sa postérité sera à toi continuellement et à travers toutes les générations pour toujours.
\par 13 « Bénis-le, Seigneur, et mets dans ma bouche la bénédiction de la justice, afin que je puisse le bénir.
\par 14 Et à cette heure-là, quand l'esprit de justice descendit dans sa bouche, elle posa ses deux mains sur la tête de Jacob, et dit :
\par    
\par 15 Béni sois-tu, Seigneur de justice et Dieu des siècles  
\par     Et qu'Il te bénisse au-delà de toutes les générations d'hommes.
\par    
\par     Qu'il te donne, mon Fils, le chemin de la justice,  
\par     Et révèle la justice à ta postérité.
\par    
\par 16 Et qu'il rende tes fils nombreux pendant ta vie,  
\par     Et qu'ils surviennent selon le nombre des mois de l'année.  
\par     Et que leurs fils deviennent nombreux et grands au-delà des étoiles du ciel,  
\par     Et leur nombre est plus grand que le sable de la mer.
\par    
\par 17 Et puisse-t-il leur donner ce beau pays - comme il a dit qu'il le donnerait toujours à Abraham et à sa postérité après lui -  
\par     Et puissent-ils le conserver comme une possession pour toujours.
\par    
\par 18 Et puissé-je te voir (né), mon fils, des enfants bénis pendant ma vie,  
\par     Et que toute ta semence soit une semence bénie et sainte.
\par    
\par 19 Et comme tu as rafraîchi l'esprit de ta mère pendant sa vie,  
\par     Le sein de celle qui t'a enfanté te bénit ainsi,
\par    
\par     [Mon affection] et mes seins te bénissent  
\par     Et ma bouche et ma langue te louent grandement.
\par    
\par 20 Augmente et s'étend sur la terre,  
\par     Et que ta postérité soit parfaite dans la joie du ciel et de la terre pour toujours ;
\par    
\par     Et que ta semence se réjouisse,  
\par     Et au grand jour de paix, qu'il y ait la paix.
\par    
\par 21 Et que ton nom et ta postérité subsistent à tous les âges,  
\par     Et que le Dieu Très-Haut soit leur Dieu,
\par    
\par     Et que le Dieu de justice habite avec eux,  
\par     Et c'est par eux que son sanctuaire sera bâti pour tous les âges.
\par    
\par 22 Béni soit celui qui te bénit,  
\par     Et toute chair qui te maudira faussement, qu'elle soit maudite.
\par    
\par 23 Et elle l'embrassa, et lui dit :  
\par     'Que le Seigneur du monde t'aime'  
\par     «Comme le cœur de ta mère et son affection se réjouissent en toi et te bénissent.»
\par     Et elle cessa de bénir.

\chapitre{26}

\textit{Isaac chasse Ésaü pour le gibier, 1-4. Rebecca demande à Jacob d'obtenir la bénédiction, 5-9. Jacob sous la personne d'Ésaü l'obtient, 10-24. Ésaü apporte sa venaison et par son importunité obtient une bénédiction , 25-34. Menace Jacob, 35. (Cf. Gen.xxvii.)}

\par 1 Et la septième année de cette semaine, Isaac appela Ésaü, son Fils aîné, et lui dit : « Je suis vieux, mon fils, et voici, mes yeux sont obscurs pour voir, et je ne connais pas le le jour de ma mort.
\par 2 'Et maintenant, prends tes armes de chasse, ton carquois et ton arc, et va dans les champs, chasse-moi et attrape-moi (le gibier), mon fils, et prépare-moi une viande savoureuse, telle que mon âme aime, et apporte-la. à moi afin que je puisse manger et que mon âme te bénisse avant de mourir.
\par 3 Mais Rébecca entendit Isaac parler à Ésaü.
\par 4 Et Ésaü partit de bonne heure aux champs pour chasser et attraper et ramener à la maison son père.
\par 5 Et Rébecca appela Jacob, son fils, et lui dit : « Voici, j'ai entendu Isaac, ton père, parler à Ésaü, ton frère, en disant : « Chasse-moi, et prépare-moi de la viande savoureuse, et apporte-la. ) pour moi ça'
\par 6 «Je peux manger et te bénir devant l'Éternel avant de mourir.» Et maintenant, mon fils, obéis à ma voix dans ce que je te commande : va vers ton troupeau et ramène-moi deux bons chevreaux, et je tu en feras un mets savoureux pour ton père, tel qu'il aime, et tu l'apporteras à ton père afin qu'il le mange et te bénisse devant l'Éternel avant de mourir, et que tu sois béni.
\par 7 Et Jacob dit à Rébecca, sa mère : « Mère, je ne refuserai rien de ce que mon père voudrait manger et qui lui plairait ; seulement je crains, ma mère, qu'il ne reconnaisse ma voix et ne veuille me toucher. '
\par 8 «Et tu sais que je suis lisse, et qu'Ésaü, mon frère, est poilu, et que j'apparaîtrai à ses yeux comme un malfaiteur, et que je ferai une action qu'il ne m'avait pas ordonnée, et il sera irrité de moi, et j'attirerai sur moi une malédiction et non une bénédiction.
\par 9 Et Rébecca, sa mère, lui dit : « Que ta malédiction soit sur moi, mon fils, obéis seulement à ma voix. »
\par 10 Et Jacob obéit à la voix de Rébecca, sa mère, et alla chercher deux bons et gros chevreaux, et les amena à sa mère, et sa mère leur fit une viande savoureuse comme il aimait.
\par 11 Et Rébecca prit les beaux vêtements d'Ésaü, son fils aîné, qui était avec elle dans la maison, et elle en vêtit Jacob, son plus jeune fils, (avec eux), et elle mit les peaux des chevreaux sur ses mains et sur les parties exposées de son cou.
\par 12 Et elle remit la viande et le pain qu'elle avait préparés entre les mains de son fils Jacob.
\par 13 Et Jacob alla vers son père et lui dit : Je suis ton fils. J'ai fait ce que tu m'as dit : lève-toi, assieds-toi et mange de ce que j'ai attrapé, père, afin que ton âme me bénisse.
\par 14 Et Isaac dit à son fils : 'Comment as-tu trouvé si vite, mon fils ?'
\par 15 Et Jacob dit : 'Parce que (l'Eternel) ton Dieu m'a fait trouver.'
\par 16 Et Isaac lui dit : Approche-toi, et je te sentirai, mon fils, si tu es mon fils Ésaü ou non.
\par 17 Et Jacob s'approcha d'Isaac, son père, et il le tâta et dit : « La voix est la voix de Jacob, mais les mains sont les mains d'Ésaü. »
\par 18 et il ne le discerna pas, parce que c'était une dispense du ciel pour lui retirer son pouvoir de perception et Isaac ne le discerna pas, car ses mains étaient velues comme celles de son frère Ésaü, de sorte qu'il le bénit.
\par 19 Et il dit : « Es-tu mon fils Ésaü ? « Et il dit : » Je suis ton fils « ; et il dit : » Approche-toi de moi et je mangerai de ce que tu as attrapé, mon fils, afin que mon âme te bénisse. »
\par 20 Et il s'approcha de lui, et il mangea, et il lui apporta du vin et il but.
\par 21 Et Isaac, son père, lui dit : Approche-toi et embrasse-moi, mon fils.
\par 22 Et il s'approcha et l'embrassa. Et il sentit l'odeur de ses vêtements, et il le bénit et dit : « Voici, l'odeur de mon fils est comme l'odeur d'un champ (plein) que l'Éternel a béni.
\par    
\par 23 Et que le Seigneur te donne de la rosée du ciel  
\par     Et de la rosée de la terre, et de beaucoup de blé et d'huile :
\par    
\par     Que les nations te servent,  
\par     Et les peuples se prosternent devant toi.
\par    
\par 24 Sois seigneur sur tes frères,  
\par     Et que les fils de ta mère se prosternent devant toi ;
\par    
\par     Et que toutes les bénédictions par lesquelles le Seigneur m'a béni et a béni Abraham, mon père ;  
\par     Soit communiqué à toi et à ta postérité pour toujours :
\par    
\par     Maudit soit celui qui te maudit,  
\par     Et béni soit celui qui te bénit.
\par    
\par 25 Et dès qu'Isaac eut fini de bénir son fils Jacob, et que Jacob fut sorti d'Isaac, son père, il se cacha et Ésaü, son frère, revint de sa chasse.
\par 26 Et il prépara aussi un mets savoureux, et l'apporta à son père, et dit à son père : « Que mon père se lève et mange de mon venaison afin que ton âme me bénisse. »
\par 27 Et Isaac, son père, lui dit : Qui es-tu ? 'Et il lui dit : 'Je suis ton premier-né, ton fils Ésaü : j'ai fait ce que tu m'as commandé.'
\par 28 Et Isaac fut très étonné et dit : « Qui est celui qui a chassé, attrapé et me l'a apporté, et j'ai mangé de tout avant que tu viennes, et je l'ai béni : (et) il le fera. soit béni, ainsi que toute sa postérité pour toujours.
\par 29 Et il arriva que lorsque Ésaü entendit les paroles de son père Isaac, il poussa un cri extrêmement grand et amer, et dit à son père : « Bénis-moi, (même) moi aussi, père.
\par 30 Et il lui dit : Ton frère est venu avec ruse et il t'a enlevé ta bénédiction. Et il dit : « Maintenant, je sais pourquoi son nom s'appelle Jacob : voici, il m'a supplanté ces deux fois : il m'a enlevé mon droit d'aînesse, et maintenant il m'a enlevé ma bénédiction.
\par 31 Et il dit : « Ne m'as-tu pas réservé une bénédiction, père ? » Et Isaac répondit et dit à Ésaü :
\par    
\par     «Voici, je l'ai établi ton seigneur,  
\par     Et je lui ai donné tous ses frères pour serviteurs,  
\par     Et avec beaucoup de blé, de vin et d'huile, je l'ai fortifié :
\par    
\par     Et que dois-je faire maintenant pour toi, mon fils ?
\par    
\par 32 Et Ésaü dit à Isaac, son père :  
\par     « N'as-tu qu'une seule bénédiction, ô père ?  
\par     Bénis-moi, (même) moi aussi, père : '
\par    
\par 33 Et Ésaü éleva la voix et pleura. Et Isaac répondit et lui dit :
\par    
\par     «Voici, loin de la rosée de la terre sera ta demeure,  
\par     Et loin de la rosée du ciel d'en haut.
\par    
\par 34 Et tu vivras par ton épée,  
\par     Et tu serviras ton frère.
\par    
\par     Et cela arrivera quand tu deviendras grand,  
\par     Et tu secoues son joug de dessus ton cou,  
\par     Tu pécheras un péché complet jusqu'à la mort,  
\par     Et ta semence sera déracinée de dessous le ciel.
\par    
\par 35 Et Ésaü menaçait Jacob à cause de la bénédiction dont son père l'avait béni, et il disait dans son cœur : « Que les jours de deuil pour mon père viennent maintenant, afin que je puisse tuer mon frère Jacob. »

\chapitre{27}

\par \textit{Rebecca alarmée par les menaces d'Ésaü pousse Isaac à envoyer Jacob en Mésopotamie, 1-12. Isaac réconforte Rébecca lors du départ de Jacob, 13-18. Le rêve et le vœu de Jacob à Béthel, 19-27. (Cf. Gén. xxviii.)}

\par 1 Et les paroles d'Esaü, son fils aîné, furent racontées en songe à Rébecca, et Rébecca envoya appeler Jacob son plus jeune fils,
\par 2 et lui dit : Voici, Ésaü, ton frère, va se venger de toi pour te tuer.
\par 3 «Maintenant donc, mon fils, obéis à ma voix, et lève-toi et fuis vers Laban, mon frère, à Haran, et reste avec lui quelques jours jusqu'à ce que la colère de ton frère se détourne et qu'il éloigne de toi sa colère. , et oublie tout ce que tu as fait ; alors je t'enverrai te chercher de là.
\par 4 Et Jacob dit : 'Je n'ai pas peur ; s'il veut me tuer, je le tuerai.
\par 5 Mais elle lui dit : 'Que je ne sois pas privée de mes deux fils un seul jour.'
\par 6 Et Jacob dit à Rébecca sa mère : 'Voici, tu sais que mon père est devenu vieux et ne voit pas parce que ses yeux sont ternes, et si je le quitte, ce sera mal à ses yeux, parce que je le quitte. et éloigne-toi de toi, et mon père se mettra en colère et me maudira. Je n'irai pas; quand il m’enverra, alors seulement j’irai.
\par 7 Et Rébecca dit à Jacob : J'entrerai et je lui parlerai, et il te renverra.
\par 8 Et Rébecca entra et dit à Isaac : « Je déteste ma vie à cause des deux filles de Heth, qu'Esaü a prises pour femmes ; et si Jacob prend une femme comme celles-ci parmi les filles du pays, dans quel but vivrais-je, car les filles de Canaan sont mauvaises.
\par 9 Et Isaac appela Jacob et le bénit, et le réprimanda et lui dit : « Ne te prends pas pour femme d'aucune des filles de Canaan ;
\par 10 lève-toi et va en Mésopotamie, à la maison de Bethuel, le père de ta mère, et prends-en une femme parmi les filles de Laban, le frère de ta mère.
\par 11 'Et que Dieu Tout-Puissant te bénisse et t'augmente et te multiplie afin que tu deviennes une communauté de nations, et te donne les bénédictions de mon père Abraham, pour toi et pour ta postérité après toi, afin que tu puisses hériter du pays de ton séjours et tout le pays que Dieu a donné à Abraham : va, mon fils, en paix.
\par 12 Et Isaac renvoya Jacob, et il se rendit en Mésopotamie, chez Laban, fils de Bethuel le Syrien, frère de Rébecca, mère de Jacob.
\par 13 Et il arriva, après que Jacob se leva pour aller en Mésopotamie, que l'esprit de Rébecca fut attristé après son fils, et elle pleura.
\par 14 Et Isaac dit à Rébecca : « Ma sœur, ne pleure pas à cause de Jacob, mon fils ; car il part en paix, et il reviendra en paix.
\par 15 « Le Dieu Très-Haut le préservera de tout mal et sera avec lui ; car il ne l'abandonnera pas tous ses jours;
\par 16 'Car je sais que ses voies prospéreront en toutes choses partout où il ira, jusqu'à ce qu'il revienne en paix vers nous, et que nous le voyions en paix.'
\par 17 « Ne crains rien à cause de lui, ma sœur, car il est sur le bon chemin et c'est un homme parfait ; et il est fidèle et ne périra pas. Ne pleure pas.
\par 18 Et Isaac consola Rébecca à cause de son fils Jacob, et le bénit.
\par 19 Et Jacob partit du Puits du Serment pour se rendre à Haran la première année de la deuxième semaine du quarante-quatrième jubilé, et il arriva à Luz sur les montagnes, c'est-à-dire Béthel, à la nouvelle lune de le premier mois de cette semaine, [21 h 15] et il arriva à cet endroit au soir et s'éloigna du chemin à l'ouest de la route cette nuit-là : et il dormit là ; car le soleil s'était couché.
\par 20 Et il prit une des pierres de ce lieu et la posa (à sa tête) sous l'arbre, et il voyageait seul, et il dormait.
\par 21 Et il rêva cette nuit-là, et voici, une échelle était dressée sur la terre, et son sommet atteignait le ciel, et voici, les anges du Seigneur montaient et descendaient dessus ; et voici, le Seigneur se tenait dessus. .
\par 22 Et il parla à Jacob et dit : « Je suis le Seigneur, le Dieu d'Abraham, ton père, et le Dieu d'Isaac ; le pays sur lequel tu dors, je le donnerai à toi, ainsi qu'à ta postérité après toi.
\par 23 'Et ta postérité sera comme la poussière de la terre, et tu augmenteras à l'ouest et à l'est, au nord et au sud, et en toi et en ta postérité seront toutes les familles des nations. béni.'
\par 24 'Et voici, je serai avec toi, et je te garderai partout où tu iras, et je te ramènerai dans ce pays en paix ; car je ne te quitterai pas avant d'avoir fait tout ce que je t'ai dit.
\par 25 Et Jacob se réveilla de son sommeil et dit : En vérité, cet endroit est la maison de l'Éternel, et je ne le savais pas. Et il eut peur et dit : « Affreux est cet endroit qui n'est autre que la maison de Dieu, et ceci est la porte du ciel. »
\par 26 Et Jacob se leva de bon matin, et prit la pierre qu'il avait mise sous sa tête et la dressa comme une statue pour un signe, et il versa de l'huile dessus. Et il donna à ce lieu le nom de Béthel ; mais le nom du lieu était au début Luz.
\par 27 Et Jacob fit un vœu au Seigneur, disant : « Si le Seigneur est avec moi et me garde pendant le chemin que je pars, et me donne du pain à manger et des vêtements pour me vêtir, afin que je vienne. je retournerai en paix à la maison de mon père, alors l'Éternel sera mon Dieu, et cette pierre que j'ai dressée comme monument pour signe en ce lieu, sera la maison de l'Éternel, et de tout ce que tu me donnes, je le ferai donne-toi le dixième, mon Dieu.

\chapitre{28}

\par \textit{Jacob épouse Léa et Rachel, 1-10. Ses enfants par Léa et Rachel et par leurs servantes, 11-24. Jacob cherche à quitter Laban, 25 : mais reste avec un certain salaire, 26-8. Jacob devient riche, 29-30. (Cf. Gen. xxix.1, 17, 18, 21-35 ; xxx.1-13,17-22, 24, 25, 28, 32, 39, 43 ; xxxi.1, 2.)}

\par 1 Et il partit en voyage, et arriva dans le pays de l'Orient, vers Laban, le frère de Rébecca, et il était avec lui, et le servit pour Rachel, sa fille, une semaine.
\par 2 Et la première année de la troisième semaine [2122 AM], il lui dit : « Donne-moi ma femme, pour laquelle je t'ai servi sept ans » ; et Laban dit à Jacob : « Je te donnerai ta femme. »
\par 3 Et Laban fit un festin, et prit Léa, sa fille aînée, et la donna pour femme à Jacob, et lui donna Zilpah, sa servante, pour servante ; et Jacob ne le savait pas, car il pensait que c'était Rachel.
\par 4 Et il entra vers elle, et voici, c'était Léa ; et Jacob se mit en colère contre Laban, et lui dit : « Pourquoi m'as-tu agi ainsi ? Ne t'ai-je pas servi pour Rachel et non pour Léa ? Pourquoi m'as-tu fait du tort ?
\par 5 Prends ta fille, et j'irai ; car tu m'as fait du mal. Car Jacob aimait Rachel plus que Léa ; car les yeux de Léa étaient faibles, mais sa forme était très belle ; mais Rachel avait de beaux yeux et une forme belle et très belle.
\par 6 Et Laban dit à Jacob : 'Il n'est pas ainsi fait dans notre pays de donner le plus jeune avant l'aîné.' Et ce n’est pas bien de faire cela ; car ainsi il est ordonné et écrit dans les tablettes célestes, que personne ne doit donner sa plus jeune fille avant l'aînée ; mais on donne d'abord l'aînée, et ensuite la plus jeune. Et celui qui fait cela, on le condamne au ciel, et personne n'est juste qui fait cela, car cette action est mauvaise devant le Seigneur.
\par 7 Et commande aux enfants d'Israël de ne pas faire cela ; qu'ils ne prennent ni ne donnent le plus jeune avant d'avoir donné l'aîné, car c'est très méchant.
\par 8 Et Laban dit à Jacob : Laisse passer les sept jours de la fête de celui-ci, et je te donnerai Rachel, et tu me serviras encore sept ans, afin que tu fasses paître mes brebis comme tu le faisais au début. la semaine précédente.
\par 9 Et le jour où furent écoulés les sept jours de la fête de Léa, Laban donna Rachel à Jacob, pour qu'il le serve encore sept ans, et il donna à Rachel Bilhah, sœur de Zilpa, comme servante.
\par 10 Et il servit encore sept années pour Rachel, car Léa lui avait été donnée pour rien.
\par 11 Et l'Éternel ouvrit le sein de Léa, et elle conçut et enfanta à Jacob un fils, et il appela son nom Ruben, le quatorzième jour du neuvième mois, la première année de la troisième semaine. [21 h 22]
\par 12 Mais le sein de Rachel était fermé, car l'Éternel voyait que Léa était haïe et que Rachel aimait.
\par 13 Et Jacob alla de nouveau vers Léa, et elle conçut, et enfanta à Jacob un deuxième fils, et il appela son nom Siméon, le vingt et unième du dixième mois, et la troisième année de cette semaine. [21 h 24]
\par 14 Et Jacob entra de nouveau vers Léa, et elle conçut, et lui enfanta un troisième fils, et il appela son nom Lévi, à la nouvelle lune du premier mois de la sixième année de cette semaine. [21 h 27]
\par 15 Et Jacob revint vers elle, et elle conçut, et lui enfanta un quatrième fils, et il appela son nom Juda, le quinzième du troisième mois, la première année de la quatrième semaine. [21 h 29]
\par 16 Et à cause de tout cela, Rachel enviait Léa, car elle n'enfantait pas, et elle dit à Jacob : « Donne-moi des enfants » ; et Jacob dit : « Est-ce que je t'ai refusé les fruits de tes entrailles ? Est-ce que je t'ai abandonné ?
\par 17 Et quand Rachel vit que Léa avait enfanté quatre fils à Jacob, Ruben et Siméon et Lévi et Juda, elle lui dit : 'Va vers Bilhah, ma servante, et elle deviendra enceinte et m'enfantera un fils.' (Et elle lui donna Bilhah, sa servante, pour épouse).
\par 18 Et il entra vers elle, et elle conçut, et lui enfanta un fils, et il appela son nom Dan, le neuvième du sixième mois, la sixième année de la troisième semaine. [21 h 27]
\par 19 Et Jacob revint à Bilhah une seconde fois, et elle conçut, et enfanta à Jacob un autre fils, et Rachel appela son nom Nephtali, le cinquième du septième mois, la deuxième année de la quatrième semaine. [21h30]
\par 20 Et quand Léa vit qu'elle était devenue stérile et qu'elle n'avait pas enfanté, elle envia Rachel, et elle donna aussi sa servante Zilpah à Jacob pour femme, et elle conçut et enfanta un fils, et Léa appela son nom Gad, sur le douzième du huitième mois, la troisième année de la quatrième semaine. [21h31]
\par 21 Et il revint vers elle, et elle conçut, et lui enfanta un deuxième fils, et Léa appela son nom Aser, le deuxième du onzième mois, la cinquième année de la quatrième semaine. [21h33]
\par 22 Et Jacob entra chez Léa, et elle conçut et enfanta un fils, et elle appela son nom Issacar, le quatrième du cinquième mois, la quatrième année de la quatrième semaine, [2132 AM] et elle donna le à une infirmière.
\par 23 Et Jacob revint vers elle, et elle conçut et enfanta deux (enfants), un fils et une fille, et elle appela le nom du fils Zabulon et le nom de la fille Dinah, le septième jour. le septième mois, la sixième année de la quatrième semaine. [21h34]
\par 24 Et l'Éternel eut pitié de Rachel, et ouvrit son sein, et elle conçut et enfanta un fils, et elle appela son nom Joseph, à la nouvelle lune du quatrième mois, la sixième année de cette quatrième semaine. [21h34]
\par 25 Et au temps où Joseph naquit, Jacob dit à Laban : « Donne-moi mes femmes et mes fils, et laisse-moi aller vers mon père Isaac, et laisse-moi me bâtir une maison ; car j'ai accompli les années pendant lesquelles je t'ai servi pour tes deux filles, et j'irai dans la maison de mon père.
\par 26 Et Laban dit à Jacob : « Reste avec moi pour ton salaire, et fais encore paître mon troupeau pour moi, et prends ton salaire. »
\par 27 Et ils convinrent entre eux qu'il lui donnerait comme salaire ceux des agneaux et des chevreaux nés noirs, tachetés et blancs, qui seraient son salaire.
\par 28 Et toutes les brebis donnèrent naissance à des agneaux tachetés, marquetés et noirs, diversement marqués, et elles enfantèrent de nouveau des agneaux semblables à eux, et tous ceux qui étaient tachetés appartenaient à Jacob et ceux qui ne l'étaient pas étaient à Laban.
\par 29 Et les possessions de Jacob se multiplièrent extrêmement, et il possédait des bœufs et des moutons, des ânes et des chameaux, et des serviteurs et des servantes.
\par 30 Et Laban et ses fils envièrent Jacob, et Laban lui reprit ses brebis, et il l'observa avec une mauvaise intention.

\chapitre{29}

\par \textit{Jacob, part secrètement, 1-4. Laban le poursuit, 5-6. Alliance de Jacob et Laban, 7-8. Demeures des Amoréens (anciennement des Rephaïm) détruites à l'époque de l'écrivain, 9-11. Laban s'en va, 12. Jacob se réconcilie avec Ésaü, 13. Jacob envoie des provisions de nourriture à ses parents quatre fois par an à Hébron, 14-17, 19-20. Ésaü se remarie, le 18. (Cf. Gen. xxxi.3, 4, 10, 13, 19, 21, 23, 24, 46, 47 ; xxxii.22 ; xxxiii.10, 16.)}

\par 1 Et il arriva que lorsque Rachel eut enfanté Joseph, Laban alla tondre ses brebis ; car ils étaient éloignés de lui à trois jours de voyage.
\par 2 Et Jacob vit que Laban allait tondre ses brebis, et Jacob appela Léa et Rachel, et leur dit gentiment qu'elles devaient venir avec lui au pays de Canaan.
\par 3 Car il leur raconta qu'il avait tout vu en rêve, même tout ce qu'il lui avait dit pour qu'il retourne à la maison de son père, et ils dirent : « Partout où tu iras, nous irons avec toi. '
\par 4 Et Jacob bénit le Dieu d'Isaac, son père, et le Dieu d'Abraham, le père de son père, et il se leva et monta sur ses femmes et ses enfants, et prit tous ses biens et traversa le fleuve, et arriva au pays de Galaad. , et Jacob cacha son intention à Laban et ne le lui dit pas.
\par 5 Et la septième année de la quatrième semaine, Jacob se tourna vers Galaad le vingt et unième mois du premier mois. [2135 AM] Et Laban le poursuivit et rattrapa Jacob dans la montagne de Galaad le troisième mois, le treizième.
\par 6 Et l'Éternel ne lui permit pas de faire du mal à Jacob ; car il lui apparut en songe la nuit. Et Laban parla à Jacob.
\par 7 Et le quinzième de ces jours, Jacob fit un festin pour Laban et pour tous ceux qui l'accompagnaient, et Jacob jura à Laban ce jour-là, et Laban aussi à Jacob, que ni l'un ni l'autre ne traverserait la montagne de Galaad pour se rendre à l'autre. avec un mauvais dessein.
\par 8 Et il fit là un monceau pour témoin ; c'est pourquoi le nom de cet endroit est appelé : « Le tas du témoignage », d'après ce tas.
\par 9 Mais auparavant, on appelait le pays de Galaad le pays des Rephaïm ; car c'était le pays des Rephaïm, et c'est là que naquirent les Rephaïm, des géants dont la hauteur était de dix, neuf, huit jusqu'à sept coudées.
\par 10 Et leur habitation s'étendait depuis le pays des enfants d'Ammon jusqu'au mont Hermon, et les sièges de leur royaume étaient Karnaim et Ashtaroth, et Edrei, et Misur, et Beon.
\par 11 Et l'Éternel les détruisit à cause de la méchanceté de leurs actions ; car ils étaient très méchants, et les Amoréens demeuraient à leur place, méchants et pécheurs, et il n'y a aucun peuple aujourd'hui qui ait accompli pleinement tous ses péchés, et ils n'ont plus de durée de vie sur la terre.
\par 12 Et Jacob renvoya Laban, et il partit en Mésopotamie, le pays de l'Orient, et Jacob retourna au pays de Galaad.
\par 13 Et il passa le Jabbok le neuvième mois, le onzième. Et ce jour-là, Ésaü, son frère, vint vers lui, et il se réconcilia avec lui, et il le quitta pour le pays de Séir, mais Jacob habita sous des tentes.
\par 14 Et la première année de la cinquième semaine de ce jubilé [2136 AM], il passa le Jourdain, et habita au-delà du Jourdain, et il fit paître ses brebis depuis la mer du tas jusqu'à Bethshan, et jusqu'à Dothan et jusqu'au forêt d'Akrabbim.
\par 15 Et il envoya à son père Isaac tous ses biens, des vêtements, et de la nourriture, et de la viande, et des boissons, et du lait, et du beurre, et du fromage, et quelques dattes de la vallée.
\par 16 Et à sa mère Rébecca aussi quatre fois par an, entre les périodes des mois, entre les labours et la moisson, et entre l'automne et la pluie (saison) et entre l'hiver et le printemps, à la tour d'Abraham.
\par 17 Car Isaac était revenu du puits du Serment et était monté à la tour de son père Abraham, et il y demeurait séparé de son fils Ésaü.
\par 18 Car à l'époque où Jacob partait en Mésopotamie, Ésaü prit pour femme Mahalath, fille d'Ismaël, et il rassembla tous les troupeaux de son père et de ses femmes, et monta et habita sur la montagne de Séir, et il laissa Isaac son père seul au Puits du Serment.
\par 19 Et Isaac monta du puits du Serment et habita dans la tour d'Abraham son père sur les montagnes d'Hébron,
\par 20 Et là Jacob envoya tout ce qu'il envoyait de temps en temps à son père et à sa mère, tout ce dont ils avaient besoin, et ils bénirent Jacob de tout leur cœur et de toute leur âme.

\chapitre{30}

\par \textit{Dinah ravie, 1-3. Massacre des Sichémites, 4-6. Lois contre les mariages mixtes entre Israël et les païens, 7-17. Lévi choisi pour le sacerdoce en raison de son massacre des Sichémites, 18-23. Dinah récupéré, 24. Réprimande de Jacob, 25-6. (Cf. Gen. xxxiii.18, xxxiv.2, 4, 7, 13-14, 25-30, xxxv.5.)}

\par 1 Et la première année de la sixième semaine [2143 AM], il monta en paix à Salem, à l'est de Sichem, au quatrième mois.
\par 2 Et là, ils emmenèrent Dinah, fille de Jacob, dans la maison de Sichem, fils de Hamor, le Hivite, prince du pays, et il coucha avec elle et la souilla, et elle était petite. fille, une enfant de douze ans.
\par 3 Et il supplia son père et ses frères de lui donner pour femme. Et Jacob et ses fils furent irrités à cause des hommes de Sichem ; car ils avaient souillé Dina, leur sœur, et ils leur parlaient avec de mauvaises intentions, les traitaient de manière trompeuse et les séduisaient.
\par 4 Et Siméon et Lévi arrivèrent à l'improviste à Sichem et exécutèrent le jugement sur tous les hommes de Sichem, et tuèrent tous les hommes qu'ils y trouvèrent, et n'en laissèrent pas un seul. Ils les tuèrent tous dans les tourments parce qu'ils avaient déshonoré leur sœur Dinah.
\par 5 Et qu'il ne se reproduise plus désormais qu'une fille d'Israël soit souillée ; car le jugement est ordonné dans le ciel contre eux, afin qu'ils fassent périr par l'épée tous les hommes des Sichémites, parce qu'ils avaient fait honte à Israël.
\par 6 Et l'Éternel les livra entre les mains des fils de Jacob afin qu'ils puissent les exterminer par l'épée et exécuter le jugement contre eux, et afin qu'il ne se reproduise plus en Israël qu'une vierge d'Israël soit souillée.
\par 7 Et si quelqu'un en Israël veut donner sa fille ou sa sœur à un homme de la race des Gentils, il mourra sûrement, et on le lapidera; car il a fait honte à Israël; et on brûlera au feu la femme, parce qu'elle a déshonoré le nom de la maison de son père, et elle sera arrachée d'Israël.
\par 8 Et qu'il ne se trouve pas de femme adultère ni d'impureté en Israël pendant tous les jours des générations de la terre ; car Israël est saint pour l'Éternel, et quiconque l'aura souillé mourra sûrement: on le lapidera avec des pierres.
\par 9 Car ainsi a été ordonné et écrit dans les tablettes célestes concernant toute la postérité d'Israël : celui qui la souillera mourra sûrement, et il sera lapidé avec des pierres.
\par 10 Et à cette loi, il n'y a ni limite de jours, ni rémission, ni aucune expiation ; mais l'homme qui a souillé sa fille sera déraciné au milieu de tout Israël, parce qu'il a donné de sa postérité à Moloch. , et a travaillé impie de manière à le souiller.
\par 11 Et toi, Moïse, commande aux enfants d'Israël et exhorte-les à ne pas donner leurs filles aux païens, et à ne prendre pour leurs fils aucune des filles des païens, car cela est abominable devant l'Éternel.
\par 12 C'est pourquoi j'ai écrit pour toi, dans les paroles de la loi, toutes les actions des Sichémites qu'ils ont commises contre Dina, et comment les fils de Jacob ont dit : « Nous ne donnerons pas notre fille à un homme. qui n'est pas circoncis; car c'était un reproche pour nous.
\par 13 Et c'est un opprobre pour Israël, pour ceux qui vivent et pour ceux qui prennent les filles des Gentils ; car cela est impur et abominable pour Israël.
\par 14 Et Israël ne sera pas exempt de cette impureté s'il a une femme parmi les filles des Gentils, ou s'il a donné l'une de ses filles à un homme qui est d'un Gentil.
\par 15 Car il y aura plaie sur plaie, et malédiction sur malédiction, et tous les jugements, plaies et malédictions viendront sur lui : s'il fait cela, ou s'il cache ses yeux à ceux qui commettent l'impureté, ou à ceux qui profanent le sanctuaire du Seigneur, ou ceux qui profanent son saint nom, (alors) toute la nation sera jugée ensemble pour toutes les impuretés et profanations de cet homme.
\par 16 Et il n'y aura aucun respect pour les personnes [et aucune considération pour les personnes] et il n'y aura pas de réception de sa part de fruits, d'offrandes, d'holocauste et de graisse, ni d'odeur de bonne odeur, de manière à l'accepter. traitez tout homme ou toute femme en Israël qui souille le sanctuaire.
\par 17 C'est pourquoi je te l'ai ordonné, en disant : Témoigne ce témoignage à Israël : vois comment se sont comportés les Sichémites et leurs fils : comment ils ont été livrés entre les mains de deux fils de Jacob, et ils les ont tués sous les tourments, et cela leur a été (compté) à justice, et cela leur a été écrit à justice.
\par 18 Et la postérité de Lévi a été choisie pour le sacerdoce et pour être Lévites, afin qu'ils puissent servir devant l'Éternel, comme nous, continuellement, et que Lévi et ses fils soient bénis pour toujours ; car il était zélé pour exécuter la justice, le jugement et la vengeance contre tous ceux qui s'élevaient contre Israël.
\par 19 Et c'est ainsi qu'ils inscrivent en témoignage en sa faveur sur les tablettes célestes la bénédiction et la justice devant le Dieu de tous :
\par 20 Et nous nous souvenons de la justice que l'homme a accomplie au cours de sa vie, à toutes les époques de l'année ; jusqu'à mille générations, ils l'enregistreront, et cela lui parviendra ainsi qu'à ses descendants après lui, et il a été enregistré sur les tablettes célestes comme un ami et un homme juste.
\par 21 Tout ce récit, je l'ai écrit pour toi, et je t'ai commandé de dire aux enfants d'Israël, qu'ils ne doivent pas commettre de péché, ni transgresser les ordonnances, ni rompre l'alliance qui a été ordonnée pour eux, (mais) qu'ils devraient le remplir et être enregistrés comme amis.
\par 22 Mais s'ils transgressent et commettent l'impureté de toute manière, ils seront inscrits sur les tablettes célestes comme des adversaires, et ils seront détruits du livre de vie, et ils seront inscrits dans le livre de ceux qui seront détruits et avec ceux qui seront déracinés de la terre.
\par 23 Et le jour où les fils de Jacob tuèrent Sichem, un écrit fut écrit en leur faveur dans le ciel, disant qu'ils avaient exécuté la justice, l'intégrité et la vengeance contre les pécheurs, et cela fut écrit pour une bénédiction.
\par 24 Et ils firent sortir Dinah, leur sœur, de la maison de Sichem, et ils prirent captifs tout ce qui était à Sichem, leurs brebis, et leurs bœufs, et leurs ânes, et toutes leurs richesses, et tous leurs troupeaux, et les amenèrent tout à Jacob leur père.
\par 25 Et il leur fit des reproches parce qu'ils avaient passé la ville par l'épée, car il craignait les habitants du pays, les Cananéens et les Phéréziens.
\par 26 Et la crainte de l'Éternel était sur toutes les villes qui sont autour de Sichem, et elles ne se levèrent pas pour poursuivre les fils de Jacob ; car la terreur était tombée sur eux.

\chapitre{31}

\par \textit{Jacob se rend à Béthel pour offrir un sacrifice, 1-3 (cf. Gen. xxxv.2-4, 7, 14). Isaac bénit Lévi, 4-17, et Juda, 18-22. Jacob raconte à Isaac comment Dieu l'a fait prospérer, 24. Jacob se rend à Béthel avec Rébecca et Déborah, 26-30. Jacob bénit le Dieu de ses pères, 31-2.}

\par 1 Et à la nouvelle lune du mois, Jacob parlait à tous les gens de sa maison. en disant : Purifiez-vous et changez de vêtements, et levons-nous et montons à Béthel, où je lui ai fait un vœu le jour où j'ai fui devant Ésaü, mon frère, parce qu'il était avec moi et m'a amené entrez dans ce pays en paix, et éloignez les dieux étrangers qui sont parmi vous.
\par 2 Et ils abandonnèrent les dieux étrangers, ce qu'ils avaient dans les oreilles et ce qu'ils avaient au cou, et les idoles que Rachel avait volées à Laban, son père, et qu'elle avait entièrement données à Jacob. Il les brûla, les brisa, les détruisit et les cacha sous un chêne qui est au pays de Sichem.
\par 3 Et il monta à Béthel à la nouvelle lune du septième mois. Et il bâtit un autel à l'endroit où il avait dormi, et il y dressa une stèle, et il envoya dire à son père Isaac de venir vers lui pour son sacrifice, ainsi qu'à sa mère Rébecca.
\par 4 Et Isaac dit : 'Laisse venir mon fils Jacob, et laisse-moi le voir avant de mourir.'
\par 5 Et Jacob partit vers son père Isaac et vers sa mère Rébecca, dans la maison de son père Abraham, et il prit avec lui deux de ses fils, Lévi et Juda, et il vint vers son père Isaac et vers sa mère Rébecca. .
\par 6 Et Rébecca sortit de la tour, devant elle, pour embrasser Jacob et l'embrasser ; car son esprit s'était réveillé lorsqu'elle entendit : « Voici, Jacob, ton fils, est venu » ; et elle l'a embrassé.
\par 7 Et elle vit ses deux fils, et elle les reconnut, et lui dit : « Est-ce que ce sont tes fils, mon fils ? et elle les embrassa, les baisa et les bénit, en disant : « En vous la postérité d'Abraham s'illustrera, et vous serez une bénédiction sur la terre.
\par 8 Et Jacob entra vers Isaac, son père, dans la chambre où il couchait, et ses deux fils étaient avec lui, et il prit la main de son père, et se baissant, il l'embrassa, et Isaac s'accrocha au cou de Jacob, son fils, et pleura sur son cou.
\par 9 Et les ténèbres quittèrent les yeux d'Isaac, et il vit les deux fils de Jacob, Lévi et Juda, et il dit : « Sont-ce là tes fils, mon fils ? car ils sont comme toi.
\par 10 Et il lui dit qu'ils étaient vraiment ses fils : 'Et tu as vraiment vu qu'ils sont vraiment mes fils'.
\par 11 Et ils s'approchèrent de lui, et il se retourna et les embrassa et les embrassa tous deux ensemble.
\par 12 Et l'esprit de prophétie descendit dans sa bouche, et il prit Lévi par sa main droite et Juda par sa gauche.
\par 13 Et il se tourna d'abord vers Lévi, et commença à le bénir d'abord, et lui dit : Que le Dieu de tous, le Seigneur même de tous les âges, te bénisse, toi et tes enfants, dans tous les âges.
\par 14 Et que le Seigneur te donne, à toi et à ta postérité, la grandeur et une grande gloire, et te fasse, toi et ta postérité, d'entre toute chair, vous approcher de lui pour servir dans son sanctuaire comme les anges de la présence et comme les saints. . (Même) comme eux, la postérité de tes fils sera pour la gloire, la grandeur et la sainteté, et puisse-t-Il les rendre grands dans tous les âges.
\par 15 Et ils seront juges et princes, et chefs de toute la postérité des fils de Jacob ;
\par    
\par     Ils diront la parole du Seigneur avec justice,  
\par     Et ils jugeront tous ses jugements avec justice.
\par    
\par     Et ils raconteront mes voies à Jacob  
\par     Et mes chemins vers Israël.
\par    
\par     La bénédiction du Seigneur sera donnée dans leur bouche  
\par     Pour bénir toute la semence du bien-aimé.
\par    
\par 16 Ta mère t'a appelé Lévi,  
\par     Et c’est à juste titre qu’elle a appelé ton nom ;
\par    
\par     Tu seras uni au Seigneur  
\par     Et sois le compagnon de tous les fils de Jacob ;
\par    
\par     Que sa table soit à toi,  
\par     Et toi et tes fils, mangez-en ;
\par    
\par     Et que ta table soit pleine pour toutes les générations,  
\par     Et ta nourriture ne manquera pas à tous les âges.
\par    
\par 17 Et que tous ceux qui te haïssent tombent devant toi,  
\par     Et que tous tes adversaires soient déracinés et périssent ;
\par    
\par     Et béni soit celui qui te bénit,  
\par     Et maudite soit toute nation qui te maudira.
\par    
\par 18 Et il dit à Juda :  
\par     'Que le Seigneur te donne force et puissance
\par    
\par     Pour fouler à pied tous ceux qui te haïssent ;  
\par     Tu seras prince, toi et l'un de tes fils, sur les fils de Jacob ;
\par    
\par     Que ton nom et celui de tes fils se répandent et traversent chaque pays et chaque région.  
\par     Alors les païens craindront devant ta face,
\par    
\par     Et toutes les nations trembleront  
\par     [Et tous les peuples trembleront].
\par    
\par 19 En toi sera le secours de Jacob,  
\par     Et en toi se trouve le salut d’Israël.
\par    
\par 20 Et quand tu es assis sur le trône d'honneur de ta justice  
\par     Il y aura une grande paix pour toute la postérité des fils du bien-aimé ;
\par    
\par     Béni soit celui qui te bénit,  
\par     Et tous ceux qui te haïssent, t'affligent et te maudissent  
\par     Sera déraciné et détruit de la terre et sera maudit.
\par    
\par 21 Et se retournant, il l'embrassa de nouveau et l'embrassa, et se réjouit grandement ; car il avait vu les fils de Jacob, son fils, en toute vérité.
\par 22 Et il sortit d'entre ses pieds, tomba et se prosterna devant lui, et il les bénit et se reposa là avec Isaac, son père, cette nuit-là, et ils mangèrent et burent avec joie.
\par 23 Et il fit dormir les deux fils de Jacob, l'un à sa droite et l'autre à sa gauche, et cela lui fut imputé à justice.
\par 24 Et Jacob raconta tout à son père pendant la nuit, comment l'Éternel lui avait montré une grande miséricorde, et comment il l'avait fait prospérer dans toutes ses voies et l'avait protégé de tout mal.
\par 25 Et Isaac bénit le Dieu de son père Abraham, qui n'avait pas retiré sa miséricorde et sa justice aux fils de son serviteur Isaac.
\par 26 Et le matin, Jacob raconta à son père Isaac le vœu qu'il avait fait au Seigneur, et la vision qu'il avait eue, et qu'il avait bâti un autel, et que tout était prêt pour le sacrifice qui devait être fait avant. le Seigneur comme il l'avait promis, et qu'il était venu le mettre sur un âne.
\par 27 Et Isaac dit à Jacob son fils : « Je ne peux pas aller avec toi ; car je suis vieux et je ne peux pas supporter le chemin : va, mon fils, en paix ; car j'ai aujourd'hui cent soixante-cinq ans ; Je ne peux plus voyager ; mets ta mère (sur un âne) et laisse-la partir avec toi.
\par 28 «Et je sais, mon fils, que tu es venu à cause de moi, et que soit béni ce jour où tu m'as vu vivant, et moi aussi je t'ai vu, mon fils.»
\par 29 Puisses-tu prospérer et accomplir le vœu que tu as fait ; et ne remets pas ton vœu ; car tu seras appelé à rendre compte du vœu ; Maintenant donc hâte-toi de l'accomplir, et qu'il soit satisfait de celui qui a tout fait, et de celui à qui tu as fait ce vœu.
\par 30 Et il dit à Rébecca : « Va avec Jacob ton fils » ; Rébecca partit avec Jacob, son fils, et Débora avec elle, et ils arrivèrent à Béthel.
\par 31 Et Jacob se souvint de la prière par laquelle son père l'avait béni, lui et ses deux fils, Lévi et Juda, et il se réjouit et bénit le Dieu de ses pères, Abraham et Isaac.
\par 32 Et il dit : « Maintenant, je sais que j'ai une espérance éternelle, et mes fils aussi, devant le Dieu de tous » ; et c'est ainsi qu'il est ordonné concernant les deux ; et ils l'enregistrent comme un témoignage éternel sur les tablettes célestes comment Isaac les a bénis.

\chapitre{32}

\par \textit{Le rêve de Lévi à Béthel, 1. Lévi choisi pour le sacerdoce, comme dixième fils, 2-3. Jacob célèbre la fête des tabernacles et offre la dîme par l'intermédiaire de Lévi : aussi la deuxième dîme, 4-9. Loi des dîmes ordonnée, 10-15. Visions de Jacob dans lesquelles Jacob lit sur les tablettes célestes son propre avenir et celui de ses descendants, 16-26. Célèbre les quatre-vingts jours de la fête des tabernacles, 27-9. Mort de Déborah, 30 ans. Naissance de Benjamin et mort de Rachel, 33-4. (Cf. Gen. xxxv.8,10, 11, 13, 16-20.)}

\par 1 Et il passa cette nuit-là à Béthel, et Lévi rêva qu'ils l'avaient ordonné et établi prêtre du Dieu Très-Haut, lui et ses fils pour toujours ; et il se réveilla de son sommeil et bénit le Seigneur.
\par 2 Et Jacob se leva tôt le matin, le quatorze de ce mois, et il donna la dîme de tout ce qui était venu avec lui, tant des hommes que du bétail, de l'or et de tous les ustensiles et vêtements, oui, il donna la dîme de tout.
\par 3 Et en ces jours-là, Rachel devint enceinte de son fils Benjamin. Et Jacob compta ses fils à partir de lui et Lévi tomba dans la part du Seigneur, et son père le revêtit des vêtements de la prêtrise et remplit ses mains.
\par 4 Et le quinzième de ce mois, il apporta à l'autel quatorze bœufs du milieu du bétail, vingt-huit béliers, quarante-neuf moutons, sept agneaux et vingt et un chevreaux comme brûlés. -offrande sur l'autel du sacrifice, agréable et d'une douce saveur devant Dieu.
\par 5 C'était là son offrande, en conséquence du vœu qu'il avait fait de donner le dixième, avec leurs offrandes de fruits et leurs libations.
\par 6 Et lorsque le feu l'eut consumé, il brûla de l'encens sur le feu, sur le feu, et en offrande de remerciement, deux bœufs, quatre béliers, quatre moutons, quatre boucs et deux moutons d'un an, et deux chevreaux; et il fit ainsi chaque jour pendant sept jours.
\par 7 Et lui et tous ses fils et ses hommes mangèrent là avec joie pendant sept jours et bénirent et remercièrent le Seigneur, qui l'avait délivré de toutes ses tribulations et lui avait fait son vœu.
\par 8 Et il donna la dîme à tous les animaux purs, et fit un holocauste, mais il ne donna pas les animaux impurs à Lévi, son fils, et il lui donna toutes les âmes des hommes.
\par 9 Et Lévi exerça la fonction sacerdotale à Béthel devant Jacob son père, de préférence à ses dix frères, et il y fut prêtre, et Jacob fit son vœu : ainsi il remit la dîme à l'Éternel et la sanctifia, et elle est devenu saint pour Lui.
\par 10 Et c'est pour cette raison qu'il est ordonné sur les tablettes célestes comme une loi pour la dîme, encore une fois la dîme à manger devant le Seigneur d'année en année, dans le lieu où il est choisi pour que son nom habite, et à cette loi il n'y a pas de limite de jours pour toujours.
\par 11 Cette ordonnance est écrite pour qu'elle s'accomplisse d'année en année en mangeant la deuxième dîme devant le Seigneur au lieu où elle aura été choisie, et qu'il n'en restera rien de cette année à l'année suivante.
\par 12 Car dans son année, la semence sera mangée jusqu'aux jours de la récolte de la semence de l'année, et le vin jusqu'aux jours du vin, et l'huile jusqu'aux jours de sa saison.
\par 13 Et tout ce qui en reste et devient vieux, qu'il soit considéré comme pollué ; qu'il soit brûlé au feu, car il est impur.
\par 14 Et qu'ils le mangent ainsi ensemble dans le sanctuaire, et qu'ils ne le laissent pas vieillir.
\par 15 Et toutes les dîmes des bœufs et des moutons seront consacrées à l'Éternel, et appartiendront à ses prêtres, qu'ils mangeront devant lui d'année en année ; car ainsi est-il ordonné et gravé concernant la dîme sur les tablettes célestes.
\par 16 Et la nuit suivante, le vingt-deuxième jour de ce mois, Jacob résolut de rebâtir ce lieu et d'entourer le parvis d'un mur, et de le sanctifier et de le rendre saint pour toujours, pour lui et son des enfants après lui.
\par 17 Et l'Éternel lui apparut pendant la nuit, le bénit et lui dit : « Ton nom ne sera pas appelé Jacob, mais Israël donnera ton nom. »
\par 18 Et Il lui dit encore : « Je suis l'Éternel qui ai créé le ciel et la terre, et je t'augmenterai et te multiplierai extrêmement, et des rois sortiront de toi, et ils jugeront partout où le pied de les fils des hommes ont marché.
\par 19 'Et je donnerai à ta postérité toute la terre qui est sous les cieux, et ils jugeront toutes les nations selon leurs désirs, et après cela ils prendront possession de toute la terre et en hériteront pour toujours.'
\par 20 Et il acheva de lui parler, et il s'éloigna de lui. et Jacob regarda jusqu'à ce qu'il soit monté au ciel.
\par 21 Et il vit dans une vision de la nuit, et voici, un ange descendait du ciel avec sept tablettes dans ses mains, et il les donna à Jacob, et il les lut et connut tout ce qui y était écrit qui lui arriverait et ses fils à travers tous les âges.
\par 22 Et il lui montra tout ce qui était écrit sur les tablettes, et lui dit : Ne bâtis pas ce lieu, n'en fais pas un sanctuaire éternel, et n'habite pas ici ; car ce n'est pas le lieu. Va dans la maison d'Abraham ton père et demeure avec Isaac ton père jusqu'au jour de la mort de ton père.
\par 23 Car en Égypte tu mourras en paix, et dans ce pays tu seras enterré avec honneur dans le sépulcre de tes pères, avec Abraham et Isaac.
\par 24 'Ne crains rien, car comme tu l'as vu et lu, ainsi tout sera ; et écris tout comme tu as vu et lu.
\par 25 Et Jacob dit : « Seigneur, comment puis-je me souvenir de tout ce que j'ai lu et vu ? « Et il lui dit : « Je te rappellerai toutes choses. »
\par 26 Et il se releva, et il se réveilla de son sommeil, et il se souvint de tout ce qu'il avait lu et vu, et il écrivit toutes les paroles qu'il avait lu et vu.
\par 27 Et il y célébra encore un autre jour, et il y sacrifia selon tout ce qu'il avait sacrifié les jours précédents, et il appela son nom « Addition », car ce jour a été ajouté et les jours précédents il a appelé « La Fête ».
\par 28 Et ainsi il fut manifesté qu'il en serait ainsi, et cela est écrit sur les tables célestes : c'est pourquoi il lui fut révélé qu'il devait le célébrer et l'ajouter aux sept jours de la fête.
\par 29 Et son nom était appelé Addition, parce qu'il était inscrit parmi les jours des fêtes, selon le nombre des jours de l'année.
\par 30 Et dans la nuit, le vingt-trois de ce mois, la nourrice de Déborah Rébecca mourut, et on l'enterra sous la ville sous le chêne de la rivière, et il appela le nom de cet endroit : « La rivière de Déborah ». ,' et le chêne, 'Le chêne du deuil de Déborah.'
\par 31 Et Rébecca partit et retourna dans sa maison vers son père Isaac, et Jacob envoya par elle des béliers, des moutons et des boucs, afin qu'elle prépare un repas pour son père tel qu'il désirait.
\par 32 Et il suivit sa mère jusqu'à ce qu'il arrivât au pays de Kabratan, et il y demeura.
\par 33 Et Rachel enfanta un fils pendant la nuit, et elle appela son nom « Fils de ma douleur » ; car elle souffrit en l'accouchant ; mais son père l'appela Benjamin, le onzième du huitième mois, la première de la sixième semaine de ce jubilé. [21 h 43]
\par 34 Et Rachel mourut là et elle fut enterrée dans le pays d'Ephrath, c'est Bethléem, et Jacob bâtit une stèle sur le tombeau de Rachel, sur le chemin au-dessus de son tombeau.

\chapitre{33}

\par \textit{Ruben pèche avec Bilhah, 1-9 (cf. Gen. xxxv.21, 22). Lois concernant l'inceste, 10-20. Enfants de Jacob, 22. (Cf. Gen. xxxv.23-7.)}

\par 1 Et Jacob partit et habita au sud de Magdaladra'ef. Et il alla vers son père Isaac, lui et Léa, sa femme, à la nouvelle lune du dixième mois.
\par 2 Et Ruben vit Bilhah, la servante de Rachel, la concubine de son père, se baigner dans l'eau dans un lieu secret, et il l'aimait.
\par 3 Et il se cacha la nuit, et il entra dans la maison de Bilhah [la nuit], et il la trouva endormie seule sur un lit dans sa maison.
\par 4 Et il coucha avec elle, et elle se réveilla et vit, et voici Ruben était couché avec elle dans le lit, et elle découvrit le bord de sa couverture et le saisit, et cria, et découvrit que c'était Ruben.
\par 5 Et elle eut honte à cause de lui, et elle relâcha sa main, et il s'enfuit.
\par 6 Et elle se lamenta extrêmement à cause de cette chose, et n'en parla à personne.
\par 7 Et quand Jacob revint et la chercha, elle lui dit : Je ne suis pas pure pour toi, car je me suis souillé à ton égard ; car Ruben m'a souillé, et il a couché avec moi pendant la nuit, et je dormais, et il ne l'a pas découvert jusqu'à ce qu'il découvre ma jupe et couche avec moi.
\par 8 Et Jacob fut extrêmement irrité contre Ruben, parce qu'il avait couché avec Bilhah, parce qu'il avait découvert le pan de son père.
\par 9 Et Jacob ne s'approcha plus d'elle parce que Ruben l'avait souillée. Et quant à tout homme qui découvre les jupes de son père, son acte est extrêmement mauvais, car il est abominable devant l'Éternel.
\par 10 C'est pourquoi il est écrit et ordonné sur les tablettes célestes qu'un homme ne doit pas coucher avec la femme de son père, et qu'il ne doit pas découvrir le vêtement de son père, car cela est impur : ils mourront sûrement ensemble, l'homme qui couche avec la femme de son père et la femme aussi, car ils ont commis des impuretés sur la terre.
\par 11 Et il n'y aura rien d'impur devant notre Dieu dans la nation qu'il s'est choisie pour possession.
\par 12 Et encore, il est écrit une seconde fois : Maudit soit celui qui couche avec la femme de son père, car il a découvert la honte de son père ; et tous les saints du Seigneur dirent : « Ainsi soit-il ; ainsi soit-il.'
\par 13 Et toi, Moïse, ordonne aux enfants d'Israël d'observer cette parole ; car cela (implique) une peine de mort ; et c'est impur, et il n'y a pas d'expiation éternelle pour l'homme qui a commis cela, mais il doit être mis à mort et tué, et lapidé avec des pierres, et déraciné du milieu du peuple de notre Dieu. .
\par 14 Car il n'est permis à personne qui agit ainsi en Israël de rester en vie un seul jour sur la terre, car il est abominable et impur.
\par 15 Et qu'on ne dise pas : Ruben a reçu la vie et le pardon après avoir couché avec la concubine de son père, et à elle aussi, bien qu'elle ait eu un mari, et que son mari Jacob, son père, soit encore en vie.
\par 16 Car jusqu'à ce moment-là, l'ordonnance, le jugement et la loi n'avaient pas été révélés dans leur intégralité pour tous, mais en tes jours (elle a été révélée) comme une loi des saisons et des jours, et une loi éternelle pour l'éternité. générations.
\par 17 Et pour cette loi, il n'y a ni consommation de jours, ni expiation pour elle, mais ils doivent tous deux être déracinés au milieu de la nation : le jour où ils l'ont commise, ils les tueront.
\par 18 Et toi, Moïse, écris-le pour Israël afin qu'ils l'observent et fassent selon ces paroles, et ne commettent pas de péché qui mène à la mort ; car le Seigneur notre Dieu est juge, il ne respecte pas les personnes et n'accepte aucun don.
\par 19 Et dis-leur ces paroles de l'alliance, afin qu'ils les entendent et les observent, et soient sur leurs gardes à leur égard, et qu'ils ne soient pas détruits et déracinés du pays ; car c'est une impureté, une abomination, une souillure et une souillure, que sont tous ceux qui commettent cela sur la terre devant notre Dieu.
\par 20 Et il n'y a pas de plus grand péché que la fornication qu'ils commettent sur la terre ; car Israël est une nation sainte pour l'Éternel son Dieu, et une nation d'héritage, et une nation sacerdotale et royale et pour (sa propre) possession ; et une telle impureté n'apparaîtra pas au milieu de la nation sainte.
\par 21 Et la troisième année de cette sixième semaine [2145 AM] Jacob et tous ses fils allèrent habiter dans la maison d'Abraham, près d'Isaac son père et de Rébecca sa mère.
\par 22 Et voici les noms des fils de Jacob : le premier-né Ruben, Siméon, Lévi, Juda, Issacar, Zabulon, les fils de Léa ; et les fils de Rachel, Joseph et Benjamin ; et les fils de Bilhah, Dan et Nephtali ; et les fils de Zilpa, Gad et Aser; et Dina, fille de Léa, fille unique de Jacob.
\par 23 Et ils vinrent et se prosternèrent devant Isaac et Rébecca, et quand ils les virent, ils bénirent Jacob et tous ses fils, et Isaac se réjouit extrêmement, car il vit les fils de Jacob, son plus jeune fils, et il les bénit.

\chapitre{34}

\par \textit{Guerre des rois amoréens contre Jacob et ses fils, 1-9. Jacob envoie Joseph rendre visite à ses frères, 10. Joseph vendu et emmené en Egypte, 11-12 (cf. Gen. xxxvii.14, 17, 18, 25, 32-6). Décès de Bilhah et Dinah, 15. Jacob pleure Joseph, 13, 14, 17. Institution du Jour des Expiations le jour où la nouvelle de la mort de Joseph est arrivée, 18-19. Épouses du fils de Jacob, 20-1.}

\par 1 Et la sixième année de cette semaine de ce quarante-quatrième jubilé [2148 AM] Jacob envoya ses fils faire paître leurs brebis, et ses serviteurs avec eux dans les pâturages de Sichem.
\par 2 Et les sept rois des Amoréens se rassemblèrent contre eux, pour les tuer, se cachant sous les arbres, et prendre leur bétail comme proie.
\par 3 Et Jacob, Lévi, Juda et Joseph étaient dans la maison avec Isaac, leur père ; car son esprit était triste, et ils ne pouvaient pas le quitter ; et Benjamin était le plus jeune, et c'est pour cette raison qu'il resta avec son père.
\par 4 Et vinrent les rois de Taphu et les rois de 'Aresa, et les rois de Seragan, et les rois de Selo, et les rois de Ga' comme, et le roi de Bethoron, et le roi de Ma'anisakir, et tous ceux qui habitent dans ces montagnes (et) qui habitent dans les bois du pays de Canaan.
\par 5 Et ils annoncèrent ceci à Jacob en disant : « Voici, les rois des Amoréens ont entouré tes fils et pillé leurs troupeaux.
\par 6 Et il se leva de sa maison, lui et ses trois fils et tous les serviteurs de son père et ses propres serviteurs, et il marcha contre eux avec six mille hommes qui portaient l'épée.
\par 7 Et il les tua dans les pâturages de Sichem, et poursuivit ceux qui fuyaient, et il les tua au fil de l'épée, et il tua 'Aresa et Taphu et Saregan et Selo et 'Amani-sakir et Ga[ga. ]'as, et il récupéra ses troupeaux.
\par 8 Et il les domina et leur imposa un tribut pour qu'ils lui payent un tribut, cinq produits fruitiers de leur pays, et il bâtit Robel et Tamnatares.
\par 9 Et il revint en paix, et fit la paix avec eux, et ils devinrent ses serviteurs, jusqu'au jour où lui et ses fils descendirent en Egypte.
\par 10 Et la septième année de cette semaine [2149 AM], il envoya Joseph s'informer du bien-être de ses frères de sa maison au pays de Sichem, et il les trouva au pays de Dothan.
\par 11 Et ils l'ont trahi et ont formé un complot contre lui pour le tuer, mais changeant d'avis, ils l'ont vendu à des marchands ismaélites, et ils l'ont fait descendre en Égypte, et ils l'ont vendu à Potiphar, l'eunuque de Pharaon, chef des cuisiniers, prêtre de la ville d'Elew.
\par 12 Et les fils de Jacob égorgeèrent un chevreau, et trempèrent la tunique de Joseph dans le sang, et l'envoyèrent à Jacob, leur père, le dixième du septième mois.
\par 13 Et il pleura toute la nuit, car on le lui avait apporté le soir, et il devint fiévreux de deuil à cause de sa mort, et il dit : « Une méchante bête a dévoré Joseph » ; et tous les membres de sa maison [pleurèrent avec lui ce jour-là, et ils] furent dans le deuil et dans le deuil avec lui tout ce jour-là.
\par 14 Et ses fils et sa fille se levèrent pour le consoler, mais il refusa d'être consolé pour son fils.
\par 15 Et ce jour-là, Bilha apprit que Joseph avait péri, et elle mourut en le pleurant, et elle habitait à Qafratef, et Dinah aussi, sa fille, mourut après la mort de Joseph.
\par 16 Et ces trois deuils tombèrent sur Israël en un mois. Et ils enterrèrent Bilha, près du tombeau de Rachel, et Dina aussi. sa fille, ils l'ont enterré là.
\par 17 Et il pleura Joseph pendant un an, et ne cessa pas, car il dit : «Laisse-moi descendre au tombeau pour pleurer mon fils».
\par 18 C'est pourquoi il est ordonné aux enfants d'Israël de s'affliger le dixième du septième mois, le jour où la nouvelle qui le fit pleurer Joseph parvint à Jacob son père, afin qu'ils fassent l'expiation. pour eux-mêmes avec un chevreau le dixième du septième mois, une fois par an, pour leurs péchés ; car ils avaient attristé l'affection de leur père à l'égard de Joseph, son fils.
\par 19 Et ce jour a été fixé pour qu'ils s'affligent de leurs péchés, de toutes leurs transgressions et de toutes leurs erreurs, afin qu'ils puissent se purifier ce jour-là une fois par an.
\par 20 Et après la mort de Joseph, les fils de Jacob prirent des femmes. Le nom de la femme de Ruben est 'Ada ; et le nom de la femme de Siméon est 'Adlba'a, une Cananéenne ; et le nom de la femme de Lévi est Melka, des filles d'Aram, de la postérité des fils de Térah ; et le nom de la femme de Juda, Betasu'el, une Cananéenne ; et le nom de la femme d'Issacar, Hezaqa ; et le nom de la femme de Zabulon, Ni'iman ; et le nom de la femme de Dan, « Egla » ; et le nom de la femme de Nephtali, Rasu'u, de Mésopotamie ; et le nom de la femme de Gad, Maka ; et le nom de la femme d'Aser, 'Ijona ; et le nom de la femme de Joseph, Asenath, l'Égyptienne ; et le nom de la femme de Benjamin, 'Ijasaka.
\par 21 Et Siméon se repentit et prit pour frères une seconde femme de Mésopotamie.

\chapitre{35}

\par \textit{Avertissement de Rebecca à Jacob et sa réponse, 1-8. Rebecca demande à Isaac de faire jurer à Ésaü qu'il ne blessera pas Jacob, 9-12. Isaac consent, 13-17. Ésaü prête serment ainsi que Jacob, 18-26. Mort de Rebecca, 27.}

\par 1 Et la première année de la première semaine du quarante-cinquième jubilé [2157 AM] Rébecca appela Jacob, son fils, et lui commanda concernant son père et son frère, de les honorer tous les jours de sa vie. vie.
\par 2 Et Jacob dit : « Je ferai tout comme tu me l'as commandé ; car cela sera pour moi un honneur et une grandeur, et une justice devant l'Éternel, que je les honorerai.
\par 3 «Et toi aussi, mère, tu sais depuis ma naissance jusqu'à ce jour, toutes mes actions et tout ce qui est dans mon cœur, que je pense toujours au bien de tout.»
\par 4 'Et comment ne ferais-je pas ce que tu m'as commandé, que j'honore mon père et mon frère !'
\par 5 'Dites-moi, mère, quelle perversité as-tu vue en moi et je m'en détournerai, et la miséricorde sera sur moi.'
\par 6 Et elle lui dit : « Mon fils, je n'ai vu en toi de tous mes jours aucune action perverse mais (seulement) des actions droites. Et pourtant je te dirai la vérité, mon fils : je mourrai cette année, et je ne survivrai pas cette année de ma vie ; car j'ai vu en songe le jour de ma mort, que je ne vivrais pas au-delà de cent cinquante-cinq ans ; et voici, j'ai accompli tous les jours de ma vie que je dois vivre.
\par 7 Et Jacob se moqua des paroles de sa mère. parce que sa mère lui avait dit qu'elle devait mourir ; et elle était assise en face de lui, en possession de sa force, et elle n'était pas affaiblie dans sa force ; car elle entrait et sortait et voyait, et ses dents étaient fortes, et aucune maladie ne l'avait touchée tous les jours de sa vie.
\par 8 Et Jacob lui dit : « Bienheureuse je suis, mère, si mes jours approchent des jours de ta vie, et si ma force reste avec moi comme ta force ; et tu ne mourras pas, car tu te moques inutilement avec moi. concernant ta mort.
\par 9 Et elle alla vers Isaac et lui dit : Je te fais une seule requête : fais jurer à Ésaü qu'il ne fera pas de mal à Jacob et qu'il ne le poursuivra pas avec inimitié ; car tu connais les pensées d'Ésaü, qu'elles sont perverses dès sa jeunesse, et qu'il n'y a aucune bonté en lui ; car il désire, après ta mort, le tuer.
\par 10 Et tu sais tout ce qu'il a fait depuis le jour où Jacob, son frère, est allé à Haran jusqu'à ce jour : comment il nous a abandonnés de tout son cœur et nous a fait du mal ; il s'est emparé de tes troupeaux et a emporté devant toi tous tes biens.
\par 11 'Et lorsque nous l'avons imploré et supplié pour ce qui nous appartenait, il a agi comme un homme qui avait pitié de nous.'
\par 12 « Et il est amer contre toi parce que tu as béni Jacob ton fils parfait et droit ; car il n'y a pas de mal mais seulement du bien en lui, et depuis qu'il est venu de Haran jusqu'à ce jour, il ne nous a rien volé, car il nous apporte toujours chaque chose en son temps et se réjouit de tout son cœur lorsque nous prenons entre ses mains. et il nous bénit et ne s'est pas séparé de nous depuis qu'il est venu de Haran jusqu'à ce jour, et il reste continuellement avec nous à la maison, nous honorant.
\par 13 Et Isaac lui dit : Moi aussi, je connais et je vois les actions de Jacob qui est avec nous, comment il nous honore de tout son cœur ; mais j'aimais autrefois Ésaü plus que Jacob, parce qu'il était le premier-né ; mais maintenant j'aime Jacob plus qu'Ésaü, car il a commis de nombreuses mauvaises actions, et il n'y a pas de justice en lui, car toutes ses voies sont injustice et violence, [et il n'y a pas de justice autour de lui.]'
\par 14 'Et maintenant mon cœur est troublé à cause de toutes ses actions, et ni lui ni sa postérité ne doivent être sauvés, car ce sont ceux qui seront détruits de la terre et qui seront déracinés de dessous le ciel, car il Il a abandonné le Dieu d'Abraham et s'est lancé après ses femmes, après leur impureté et après leur erreur, lui et ses enfants.
\par 15 Et tu m'ordonnes de lui faire jurer qu'il ne tuera pas Jacob son frère ; même s'il jure, il ne respectera pas son serment et il ne fera pas le bien mais seulement le mal.
\par 16 'Mais s'il veut tuer Jacob, son frère, il sera livré entre les mains de Jacob, et il n'échappera pas à ses mains, [car il descendra entre ses mains.]'
\par 17 « Et ne crains rien à cause de Jacob ; car le gardien de Jacob est grand, puissant, honoré et plus loué que le gardien d'Ésaü.
\par 18 Et Rébecca envoya appeler Ésaü, et il vint vers elle, et elle lui dit : « J'ai une requête, mon fils, à te faire, et tu promets de la faire, mon fils.
\par 19 Et il dit : Je ferai tout ce que tu me diras, et je ne refuserai pas ta requête.
\par 20 Et elle lui dit : Je te demande que le jour de ma mort, tu me prennes et que tu m'enterres près de Sarah, la mère de ton père, et que toi et Jacob vous aimerez l'un l'autre et que ni l'un ni l'autre ne désirerez du mal contre l'autre, mais l'amour mutuel seulement, et (ainsi) vous prospérerez, mes fils, et serez honorés au milieu du pays, et aucun ennemi ne se réjouira à votre sujet, et vous serez une bénédiction et une miséricorde aux yeux de tous ceux qui t'aiment.
\par 21 Et il dit : Je ferai tout ce que tu m'as dit, et je t'enterrerai le jour où tu mourras près de Sarah, la mère de mon père, comme tu as désiré que ses os soient près des tiens.
\par 22 Et Jacob, mon frère, j'aimerai aussi par-dessus toute chair ; car je n'ai pas de frère sur toute la terre, mais lui seul : et ce n'est pas un grand mérite pour moi si je l'aime ; car il est mon frère, et nous avons été semés ensemble dans ton corps, et nous sommes sortis ensemble de ton sein, et si je n'aime pas mon frère, qui aimerai-je ?
\par 23 «Et moi-même, je te prie d'exhorter Jacob à mon sujet et à celui de mes fils, car je sais qu'il sera assurément roi sur moi et sur mes fils, car le jour où mon père l'a béni, il l'a élevé au-dessus et moi le plus bas.
\par 24 Et je te jure que je l'aimerai et que je ne désirerai pas de mal contre lui tous les jours de ma vie, mais que je ne désirerai que du bien.
\par 25 Et il lui jura sur toute cette affaire. Et elle appela Jacob devant les yeux d'Ésaü, et lui donna le commandement selon les paroles qu'elle avait dites à Ésaü.
\par 26 Et il dit : « Je ferai ton bon plaisir ; croyez-moi, aucun mal ne viendra de moi ou de mes fils contre Ésaü, et je ne serai premier en rien qu'en amour seulement.
\par 27 Et ils mangèrent et burent, elle et ses fils cette nuit-là, et elle mourut, cette nuit-là, à l'âge de trois jubilés et d'une semaine et d'un an, et ses deux fils, Ésaü et Jacob, l'enterrèrent dans la double grotte proche. Sarah, la mère de leur père.

\chapitre{36}

\par \textit{Isaac donne des directives à ses fils quant à son enterrement : les exhorte à s'aimer les uns les autres et leur fait imprécher la destruction de celui qui blesse son frère, 1-11. Partage ses possessions, donnant la plus grande part à Jacob, et meurt, 12-18. Léa meurt : les fils de Jacob viennent le réconforter, 21-4.}

\par 1 Et la sixième année de cette semaine [2162 AM] Isaac appela ses deux fils Ésaü et Jacob, et ils vinrent vers lui, et il leur dit : « Mes fils, je vais par le chemin de mes pères, vers la maison éternelle où sont mes pères.
\par 2 C'est pourquoi, enterre-moi près d'Abraham, mon père, dans la double grotte du champ d'Éphron le Hittite, où Abraham a acheté un sépulcre pour y enterrer ; dans le sépulcre que j'ai creusé pour moi-même, enterre-moi là.
\par 3 'Et ceci, je vous commande, mes fils, de pratiquer la justice et l'intégrité sur la terre, afin que l'Éternel fasse venir sur vous tout ce que l'Éternel a dit qu'il ferait à Abraham et à sa postérité.'
\par 4 'Et aimez-vous les uns les autres, mes fils, vos frères comme un homme qui aime son âme, et que chacun cherche en quoi il peut profiter à son frère, et agissez ensemble sur la terre ; et qu'ils s'aiment comme leur propre âme.
\par 5 'Et concernant la question des idoles, je vous commande et vous exhorte à les rejeter, à les haïr et à ne pas les aimer, car elles sont pleines de tromperie pour ceux qui les adorent et pour ceux qui se prosternent devant elles.'
\par 6 « Souvenez-vous, mes fils, du Seigneur Dieu d'Abraham, votre père, et de la façon dont moi aussi je l'ai adoré et que je l'ai servi dans la justice et dans la joie, afin qu'il vous multiplie et augmente votre postérité comme les étoiles du ciel en multitude, et établis-toi sur la terre comme une plante de justice qui ne sera pas déracinée de génération en génération.
\par 7 'Et maintenant je vais vous faire prêter un grand serment - car il n'y a pas de serment qui soit plus grand que celui du nom glorieux et honoré et grand et splendide et merveilleux et puissant, celui qui a créé les cieux et la terre et toutes choses. ensemble, afin que vous le craigniez et que vous l'adoriez.
\par 8 'Et que chacun aimera son frère avec affection et justice, et que personne ne désirera du mal contre son frère désormais et pour toujours, tous les jours de votre vie, afin que vous puissiez prospérer dans toutes vos actions et ne pas être détruits.'
\par 9 « Et si l'un de vous médite du mal contre son frère, sachez que désormais quiconque médite du mal contre son frère tombera entre ses mains, et sera déraciné du pays des vivants, et sa postérité sera détruite. » de dessous le ciel.
\par 10 «Mais au jour de la turbulence, de l'exécration, de l'indignation et de la colère, avec un feu flamboyant et dévorant, comme il a brûlé Sodome, de même il brûlera son pays et sa ville et tout ce qui lui appartient, et il sera effacé du livre de discipline des enfants des hommes, et ne sera pas enregistré dans le livre de vie, mais dans celui qui est destiné à la destruction, et il s'en ira à l'exécration éternelle ; afin que leur condamnation soit toujours renouvelée dans la haine et dans l'exécration et dans la colère et dans le tourment et dans l'indignation et dans les plaies et dans la maladie pour toujours.
\par 11 'Je vous le dis et vous témoigne, mes fils, selon le jugement qui s'abattra sur l'homme qui veut faire du mal à son frère.'
\par 12 « Et il partagea tous ses biens entre les deux ce jour-là et il donna la plus grande part à celui qui était le premier-né, et la tour et tout ce qui était autour, et tout ce qu'Abraham possédait au puits de le serment.'
\par 13 Et il dit : 'Cette part plus grande, je la donnerai au premier-né.'
\par 14 Et Ésaü dit : J'ai vendu à Jacob et j'ai donné mon droit d'aînesse à Jacob ; qu'il lui soit donné, et je n'ai pas un seul mot à dire à ce sujet, car il est à lui.
\par 15 Et Isaac dit : Qu'une bénédiction repose aujourd'hui sur vous, mes fils, et sur votre postérité, car vous m'avez donné du repos, et mon cœur n'est pas peiné au sujet du droit d'aînesse, de peur que vous ne commettiez le mal à cause de il.'
\par 16 'Que le Dieu Très-Haut bénisse l'homme qui pratique la justice, lui et sa postérité pour toujours.'
\par 17 Et il finit de leur donner des ordres et de les bénir, et ils mangèrent et burent ensemble devant lui, et il se réjouit parce qu'il y avait une seule âme entre eux, et ils sortirent de lui, se reposèrent ce jour-là et dormirent.
\par 18 Et Isaac dormit ce jour-là sur son lit, se réjouissant ; et il dormit du sommeil éternel, et mourut à l'âge de cent quatre-vingts ans. Il a accompli vingt-cinq semaines et cinq ans ; et ses deux fils Ésaü et Jacob l'enterrèrent.
\par 19 Et Ésaü partit pour le pays d'Edom, vers les montagnes de Séir, et il y demeura.
\par 20 Et Jacob habitait dans les montagnes d'Hébron, dans la tour du pays de séjour de son père Abraham, et il adorait l'Éternel de tout son cœur et selon les commandements visibles, selon qu'il avait divisé les jours de son générations.
\par 21 Et Léa, sa femme, mourut la quatrième année de la deuxième semaine du quarante-cinquième jubilé, [2167 AM] et il l'enterra dans la double grotte près de Rébecca, sa mère, à gauche du tombeau de Sarah, la mère de son père. mère
\par 22 Et tous ses fils et ses fils vinrent pleurer avec lui Léa, sa femme, et le consoler à son sujet, car il la pleurait, car il l'aimait extrêmement après la mort de Rachel, sa sœur.
\par 23 Car elle était parfaite et droite dans toutes ses voies et honorait Jacob, et pendant tout le temps qu'elle vécut avec lui, il n'entendit pas de sa bouche une parole dure, car elle était douce et paisible, droite et honorable.
\par 24 Et il se souvint de toutes ses actions qu'elle avait faites au cours de sa vie et il la déplora extrêmement; car il l'aimait de tout son cœur et de toute son âme.

\chapitre{37}

\par \textit{Les fils d'Ésaü lui reprochent sa subordination à Jacob, et le contraignent à faire la guerre avec l'aide de 4 000 mercenaires contre Jacob, 1-15. Jacob réprimande Ésaü, 16-17. Réponse d'Ésaü, 18-25.}

\par 1 Et le jour où mourut Isaac, le père de Jacob et d'Ésaü, [2162 AM] les fils d'Ésaü apprirent qu'Isaac avait donné la part de l'aîné à son plus jeune fils Jacob et ils furent très en colère.
\par 2 Et ils disputèrent avec leur père, disant : Pourquoi ton père a-t-il donné à Jacob la part de l'aîné et t'a-t-il ignoré, alors que tu es l'aîné et Jacob le plus jeune ?
\par 3 Et il leur dit : « Parce que j'ai vendu mon droit d'aînesse à Jacob pour un petit plat de lentilles, et le jour où mon père m'a envoyé chasser et attraper et lui apporter quelque chose qu'il devrait manger et me bénir, il est venu avec par ruse et j'ai apporté à manger et à boire à mon père, et mon père l'a béni et m'a mis sous sa main.
\par 4 'Et maintenant notre père nous a fait jurer, lui et moi, que nous ne méditerons pas mutuellement de mal, ni contre son frère, et que nous continuerons dans l'amour et la paix chacun avec son frère et que nous ne ferons pas nos chemins. corrompu.'
\par 5 Et ils lui dirent : Nous ne t'écouterons pas pour faire la paix avec lui ; car notre force est plus grande que sa force, et nous sommes plus puissants que lui ; nous irons contre lui, nous le tuerons, et nous le détruirons ainsi que ses fils. Et si tu ne viens pas avec nous, nous te ferons aussi du mal.
\par 6 'Et maintenant écoutez-nous : envoyons en Aram et en Philistie et en Moab et Ammon, et choisissons pour nous-mêmes des hommes choisis qui sont ardents pour la bataille, et allons contre lui et combattons contre lui, et que exterminons-le de la terre avant qu'il ne devienne fort.
\par 7 Et leur père leur dit : N'allez pas et ne lui faites pas la guerre, de peur que vous ne tombiez devant lui.
\par 8 Et ils lui dirent : C'est aussi exactement ta façon d'agir depuis ta jeunesse jusqu'à ce jour, et tu mets ton cou sous son joug.
\par 9 Nous n'écouterons pas ces paroles. Et ils envoyèrent en Aram et à Aduram chez l'ami de leur père, et ils engagèrent avec eux mille combattants, hommes de guerre choisis.
\par 10 Et vinrent de Moab et des enfants d'Ammon, ceux qui avaient été embauchés, mille hommes d'élite, et de Philistie, mille hommes de guerre d'élite, et d'Edom et des Horites mille combattants d'élite. et de Kittim, de puissants hommes de guerre.
\par 11 Et ils dirent à leur père : 'Sort avec eux et conduis-les, sinon nous te tuerons.'
\par 12 Et il fut rempli de colère et d'indignation, en voyant que ses fils le forçaient à marcher devant (eux) pour les conduire contre Jacob son frère.
\par 13 Mais ensuite il se souvint de tout le mal qui était caché dans son cœur contre Jacob son frère ; et il ne se souvint pas du serment qu'il avait juré à son père et à sa mère de ne méditer aucun mal pendant toute sa vie contre son frère Jacob.
\par 14 Et malgré tout cela, Jacob ne savait pas qu'ils allaient lui attaquer pour le combattre, et il pleura Léa, sa femme, jusqu'à ce qu'ils s'approchèrent tout près de la tour avec quatre mille guerriers et hommes de guerre d'élite.
\par 15 Et les hommes d'Hébron lui envoyèrent dire : Voici, ton frère est venu contre toi pour te combattre, avec quatre mille ceints de l'épée, et ils portent des boucliers et des armes. car ils aimaient Jacob plus qu'Ésaü. Alors ils le lui dirent ; car Jacob était un homme plus libéral et plus miséricordieux qu'Ésaü.
\par 16 Mais Jacob ne crut pas jusqu'à ce qu'ils soient très près de la tour.
\par 17 Et il ferma les portes de la tour ; Et il se tint sur les créneaux et parla à son frère Ésaü et dit : « Noble est la consolation avec laquelle tu es venu me consoler pour ma femme qui est décédée. Est-ce le serment que tu as prêté à ton père et de nouveau à ta mère avant leur mort ? Tu as rompu le serment, et au moment où tu as juré à ton père, tu as été condamné.
\par 18 Alors Ésaü répondit et lui dit : Ni les enfants des hommes ni les bêtes de la terre n'ont de serment de justice qu'ils ont juré en jurant (serment valable) pour toujours ; mais chaque jour, ils inventent du mal les uns contre les autres, et comment chacun peut tuer son adversaire et son ennemi.
\par 19 Et tu me hais, moi et mes enfants, pour toujours. Et il n’y a pas de lien de fraternité avec toi.
\par 20 Écoute ces paroles que je te déclare,
\par    
\par     Si le sanglier peut changer de peau et rendre ses poils aussi doux que de la laine,  
\par     Ou s'il peut faire germer des cornes sur sa tête, comme celles d'un cerf ou d'un mouton,  
\par     Alors j'observerai le lien de fraternité avec toi  
\par     Et si les seins se sont séparés de leur mère, car tu n'es pas pour moi un frère.  
\par    
\par 21 Et si les loups font la paix avec les agneaux pour ne pas les dévorer ni leur faire violence,  
\par     Et si leur cœur est tourné vers eux pour le bien,  
\par     Alors il y aura la paix dans mon cœur envers toi
\par    
\par 22 Et si le lion devient l'ami du bœuf et fait la paix avec lui  
\par     Et s’il est lié avec lui sous un même joug et qu’il laboure avec lui,  
\par     Alors je ferai la paix avec toi.  
\par    
\par 23 Et quand le corbeau devient blanc comme le raza,  
\par     Alors sache que je t'ai aimé  
\par     Et je ferai la paix avec toi  
\par     Tu seras déraciné,  
\par     Et tes fils seront déracinés,  
\par     Et il n'y aura pas de paix pour toi'
\par    
\par 24 Et quand Jacob vit qu'il était (si) méchant envers lui de tout son cœur et de toute son âme au point de le tuer, et qu'il était venu bondir comme le sanglier qui vient sur la lance qui transperce et tue il, et ne recule pas devant lui ;
\par 25 Alors il dit aux siens et à ses serviteurs de l'attaquer, lui et tous ses compagnons.

\chapitre{38}

\par \textit{Guerre entre Jacob et Ésaü. Mort d'Ésaü et renversement de ses forces, 1-10. Edom réduit en servitude « jusqu'à ce jour », 11-14. Rois d'Édom, 15-24. (Cf. Gen. xxxvi.31-9.)}

\par 1 Et après cela Juda parla à Jacob, son père, et lui dit : 'Bende ton arc, père, et envoie tes flèches et renverse l'adversaire et tue l'ennemi ; et puisses-tu avoir le pouvoir, car nous ne tuerons pas ton frère, car il est comme toi, et il est comme toi, donnons-lui (cet) honneur.
\par 2 Alors Jacob tendit son arc et envoya la flèche et frappa Ésaü, son frère (sur son sein droit) et le tua.
\par 3 Et il lança de nouveau une flèche et frappa 'Adoran l'Araméen au sein gauche, puis le repoussa en arrière et le tua.
\par 4 Alors sortirent les fils de Jacob, eux et leurs serviteurs, se divisant en groupes sur les quatre côtés de la tour.
\par 5 Et Juda sortit en tête, et Nephtali et Gad avec lui et cinquante serviteurs avec lui du côté sud de la tour, et ils tuèrent tout ce qu'ils trouvèrent devant eux, et pas un seul d'entre eux n'échappa.
\par 6 Et Lévi, Dan et Aser sortirent du côté est de la tour, et cinquante (hommes) avec eux, et ils frappèrent les combattants de Moab et d'Ammon.
\par 7 Et Ruben, Issacar et Zabulon sortirent du côté nord de la tour, et cinquante hommes avec eux, et ils frappèrent les combattants des Philistins.
\par 8 Et Siméon, Benjamin et Enoch, fils de Ruben, sortirent du côté occidental de la tour, et cinquante (hommes) avec eux, et ils tuèrent quatre cents hommes d'Edom et des Horiens, vaillants guerriers ; et six cents s'enfuirent, et quatre des fils d'Ésaü s'enfuirent avec eux, et laissèrent leur père tué, comme il était tombé sur la colline qui est à Aduram.
\par 9 Et les fils de Jacob les poursuivirent jusqu'aux montagnes de Séir. Et Jacob enterra son frère sur la colline qui est à Aduram, et il retourna dans sa maison.
\par 10 Et les fils de Jacob pressèrent fortement les fils d'Ésaü dans les montagnes de Séir, et inclinèrent le cou, de sorte qu'ils devinrent les serviteurs des fils de Jacob.
\par 11 Et ils envoyèrent vers leur père (pour savoir) s'ils devaient faire la paix avec eux ou les tuer.
\par 12 Et Jacob envoya dire à ses fils qu'ils devaient faire la paix, et ils firent la paix avec eux, et placèrent sur eux le joug de la servitude, de sorte qu'ils payèrent toujours un tribut à Jacob et à ses fils.
\par 13 Et ils continuèrent à payer un tribut à Jacob jusqu'au jour où il descendit en Égypte.
\par 14 Et les fils d'Edom ne sont pas délivrés du joug de servitude que les douze fils de Jacob leur avaient imposé jusqu'à ce jour.
\par 15 Et ce sont là les rois qui régnaient à Édom avant qu'aucun roi ne règne sur les enfants d'Israël [jusqu'à ce jour] dans le pays d'Édom.
\par 16 Et Balaq, fils de Beor, régna à Edom, et le nom de sa ville était Danaba.
\par 17 Et Balaq mourut, et Jobab, fils de Zara de Boser, régna à sa place.
\par 18 Et Jobab mourut, et 'Asam, du pays de Théman, régna à sa place.
\par 19 Et 'Asam mourut, et 'Adath, fils de Barad, qui tua Madian dans le champ de Moab, régna à sa place, et le nom de sa ville était Avith.
\par 20 Et 'Adath mourut, et Salman, de 'Amaseqa, régna à sa place.
\par 21 Et Salman mourut, et Saül de Ra'aboth (près du) fleuve régna à sa place.
\par 22 Et Saül mourut, et Ba'elunan, fils d'Achbor, régna à sa place.
\par 23 Et Ba'elunan, fils d'Achbor, mourut, et 'Adath régna à sa place, et le nom de sa femme était Maitabith, fille de Matarat, fille de Metabedza'ab.
\par 24 Ce sont là les rois qui régnaient au pays d'Édom.

\chapitre{39}

\par \textit{Joseph s'installa sur la maison de Potiphar, 1-4. Sa pureté et son emprisonnement , 5-13. Emprisonnement du chef échanson et du chef boulanger de Pharaon dont Joseph interprète les rêves, 14-18. (Cf. Gen.xxxvii.2 ; xxxix.3-8, 12-15, 17-23 ; xl.1-5, 21-3 ; xli.1.)}

\par 1 Et Jacob habita dans le pays où son père séjournait, au pays de Canaan. Ce sont les générations de Jacob.
\par 2 Et Joseph avait dix-sept ans lorsqu'on l'emmena au pays d'Égypte, et Potiphar, eunuque de Pharaon, le chef cuisinier, l'acheta.
\par 3 Et il établit Joseph sur toute sa maison et la bénédiction de l'Éternel tomba sur la maison de l'Égyptien à cause de Joseph, et l'Éternel lui fit prospérer dans tout ce qu'il faisait.
\par 4 Et l'Égyptien remit tout entre les mains de Joseph ; car il voyait que le Seigneur était avec lui et que le Seigneur lui faisait prospérer dans tout ce qu'il faisait.
\par 5 Et l'apparence de Joseph était belle [et son apparence était très belle], et la femme de son maître leva les yeux et vit Joseph, et elle l'aima et le supplia de coucher avec elle.
\par 6 Mais il ne rendit pas son âme, et il se souvint du Seigneur et des paroles que Jacob, son père, lisait parmi les paroles d'Abraham, selon lesquelles aucun homme ne devait se livrer à la fornication avec une femme qui a un mari ; que pour lui le châtiment de mort a été ordonné dans les cieux devant le Dieu Très-Haut, et que le péché sera enregistré contre lui dans les livres éternels continuellement devant le Seigneur.
\par 7 Et Joseph se souvint de ces paroles et refusa de coucher avec elle.
\par 8 Et elle le supplia pendant un an, mais il refusa et ne voulut pas écouter.
\par 9 Mais elle l'embrassa et le retint dans la maison pour le forcer à coucher avec elle, et ferma les portes de la maison et le retint; mais il laissa son vêtement entre ses mains, franchit la porte et s'enfuit dehors, loin d'elle.
\par 10 Et la femme vit qu'il ne voulait pas coucher avec elle, et elle le calomnia en présence de son maître, disant : « Ton serviteur hébreu, que tu aimes, a cherché à me forcer à coucher avec moi ; et lorsque j'ai élevé la voix, il s'est enfui et a laissé son vêtement dans mes mains lorsque je le tenais, et il a forcé la porte.
\par 11 Et l'Égyptien vit le vêtement de Joseph et la porte cassée, et entendit les paroles de sa femme, et jeta Joseph en prison dans le lieu où étaient gardés les prisonniers que le roi avait emprisonnés.
\par 12 Et il était là, dans la prison ; et le Seigneur donna à Joseph faveur aux yeux du chef des gardiens de la prison et compassion devant lui, car il voyait que le Seigneur était avec lui et que le Seigneur faisait prospérer tout ce qu'il faisait.
\par 13 Et il remit toutes choses entre ses mains, et le chef des gardiens de la prison ne savait rien de ce qui était avec lui, car Joseph faisait tout, et le Seigneur l'a parfait.
\par 14 Et il resta là deux ans. Et en ce temps-là, Pharaon, roi d'Égypte, était irrité contre ses deux eunuques, contre le grand échanson et contre le chef panetier, et il les mit en garde dans la maison du chef cuisinier, dans la prison où était détenu Joseph.
\par 15 Et le chef des gardiens de la prison désigna Joseph pour les servir ; et il servit devant eux.
\par 16 Et ils eurent tous deux un songe, le grand échanson et le chef panetier, et ils le racontèrent à Joseph.
\par 17 Et comme il leur avait expliqué ce qui leur arriva, Pharaon rétablit le chef des échanson dans ses fonctions et il tua le (chef) panetier, comme Joseph le leur avait interprété.
\par 18 Mais le grand échanson oublia Joseph dans la prison, bien qu'il l'ait informé de ce qui lui arriverait, et il ne se souvint pas d'informer Pharaon de la manière dont Joseph le lui avait dit, car il avait oublié.

\chapitre{40}

\par \textit{Les rêves de Pharaon et leur interprétation, 1-4. Élévation et mariage de Joseph, 5-13. (Cf. Gen. xli.1-5, 7-9, 14 suiv., 25, 29-30, 34, 36, 38-43, 45-6, 49.)}

\par 1 Et en ces jours-là, Pharaon fit deux rêves en une nuit concernant une famine qui devait sévir dans tout le pays, et il se réveilla de son sommeil et appela tous les interprètes de rêves qui étaient en Égypte et les magiciens, et leur dit ses deux rêves, et ils n'ont pas pu les déclarer.
\par 2 Et alors le grand échanson se souvint de Joseph et parla de lui au roi, et il le fit sortir de la prison, et il raconta devant lui ses deux rêves.
\par 3 Et il dit devant Pharaon que ses deux rêves n'en faisaient qu'un, et il lui dit : « Sept années viendront (au cours desquelles il y aura) d'abondance dans tout le pays d'Égypte, et après cela sept années de famine, telles que une famine comme il n'y en a pas eu dans tout le pays.
\par 4 «Et maintenant, que Pharaon établisse des surveillants dans tout le pays d'Égypte, et qu'ils accumulent de la nourriture dans chaque ville pendant les jours des années d'abondance, et il y aura de la nourriture pour les sept années de famine, et le pays ne périra pas à cause de la famine, car elle sera très grave.
\par 5 Et l'Éternel fit à Joseph faveur et miséricorde aux yeux de Pharaon, et Pharaon dit à ses serviteurs. «Nous ne trouverons pas d'homme aussi sage et avisé que cet homme, car l'esprit du Seigneur est avec lui.»
\par 6 Et il l'établit comme second dans tout son royaume et lui donna autorité sur toute l'Égypte, et le fit monter sur le deuxième char de Pharaon.
\par 7 Et il le vêtit de vêtements en byssus, et il lui mit une chaîne d'or autour du cou, et (un héraut) proclama devant lui 'El 'El wa 'Abirer, et lui plaça un anneau à la main et le fit régner sur tout. sa maison, il le magnifia et lui dit. «Ce n'est que sur le trône que je serai plus grand que toi.»
\par 8 Et Joseph régna sur tout le pays d'Égypte, et tous les princes de Pharaon, et tous ses serviteurs, et tous ceux qui s'occupaient des affaires du roi, l'aimèrent, car il marchait dans la droiture, car il était sans orgueil ni arrogance, et il n'avait aucun respect pour les personnes et n'acceptait pas de cadeaux, mais il jugeait avec intégrité tous les habitants du pays.
\par 9 Et le pays d'Égypte était en paix devant Pharaon à cause de Joseph, car l'Éternel était avec lui et lui accordait faveur et miséricorde pour toutes ses générations devant tous ceux qui le connaissaient et ceux qui entendaient parler de lui, et du royaume de Pharaon. était bien ordonné, et il n'y avait ni Satan ni personne méchante (à l'intérieur).
\par 10 Et le roi appela Joseph du nom de Séphantiphane, et il donna à Joseph pour femme la fille de Potiphar, fille du prêtre d'Héliopolis, chef cuisinier.
\par 11 Et le jour où Joseph se présenta devant Pharaon, il avait trente ans [quand il se présenta devant Pharaon].
\par 12 Et cette année-là, Isaac mourut. Et il arriva, comme Joseph l'avait dit dans l'interprétation de ses deux songes, comme il l'avait dit, qu'il y eut sept années d'abondance dans tout le pays d'Égypte, et que le pays d'Égypte produisit en abondance, une mesure (produisant) dix-huit cents mesures.
\par 13 Et Joseph rassembla de la nourriture dans chaque ville jusqu'à ce qu'elles soient pleines de blé jusqu'à ce qu'elles ne puissent plus le compter et le mesurer pour sa multitude.

\chapitre{41}

\par \textit{Les fils de Juda et Tamar, 1-7. L'inceste de Juda avec Tamar, 8-18. Tamar donne naissance à des jumeaux, 21-2. Juda a pardonné, parce qu'il a péché par ignorance et s'est repenti une fois reconnu coupable, et parce que le mariage de Tamar avec ses fils n'avait pas été consommé, 23-8. (Cf. Gen. xxxviii.6-18, 20-6, 29-30 ; xli.13.)}

\par 1 Et au quarante-cinquième jubilé, la deuxième semaine, (et) la deuxième année, [2165 AM] Juda prit pour son premier-né Er, une femme d'entre les filles d'Aram, nommée Tamar.
\par 2 Mais il la haït et ne coucha pas avec elle, parce que sa mère était d'une des filles de Canaan, et il voulait lui prendre une femme parmi les parents de sa mère, mais Juda, son père, ne le permettait pas.
\par 3 Et cet Er, le premier-né de Juda, était méchant, et l'Éternel le fit mourir.
\par 4 Et Juda dit à Onan, son frère : «Va vers la femme de ton frère et accomplis envers elle le devoir de frère de ton mari, et suscite une postérité à ton frère.»
\par 5 Et Onan savait que la semence ne serait pas la sienne, (mais) celle de son frère seulement, et il entra dans la maison de la femme de son frère, et répandit la semence sur le sol, et il était méchant aux yeux du Seigneur. , et Il l'a tué.
\par 6 Et Juda dit à Tamar, sa belle-fille : « Reste dans la maison de ton père comme veuve jusqu'à ce que Schéla, mon fils, soit grand, et je te lui donnerai pour femme.
\par 7 Et il grandit ; mais Bedsu'el, la femme de Juda, ne permit pas à son fils Schéla de se marier. Et Bedsu'el, la femme de Juda, mourut [2168 AM] la cinquième année de cette semaine.
\par 8 Et la sixième année, Juda monta pour tondre ses brebis à Timnah. [2169 AM] Et ils dirent à Tamar : 'Voici ton beau-père monte à Timnah pour tondre ses brebis.'
\par 9 Et elle ôta ses vêtements de veuve, et mit un voile, et se para, et s'assit à la porte qui est près du chemin de Timna.
\par 10 Et tandis que Juda s'en allait, il la trouva et la prit pour une prostituée, et il lui dit : « Laisse-moi entrer vers toi » ; et elle lui dit : « Entre », et il entra.
\par 11 Et elle lui dit : « Donne-moi mon salaire » ; et il lui dit : « Je n'ai rien dans ma main, sauf l'anneau qui est à mon doigt, et mon collier, et mon bâton qui est dans ma main.
\par 12 Et elle lui dit : « Donne-les-moi jusqu'à ce que tu m'envoies mon salaire », et il lui dit : « Je t'enverrai un chevreau » ; et il les lui donna, \textit{et il entra vers elle}, et elle conçut par lui.
\par 13 Et Juda alla vers ses brebis, et elle alla à la maison de son père.
\par 14 Et Juda envoya un chevreau par la main de son berger, un Adullamite, et il ne la trouva pas ; et il interrogea les gens du lieu, disant : « Où est la prostituée qui était ici ? Et ils lui dirent : «Il n'y a pas de prostituée ici avec nous.»
\par 15 Et il revint et l'informa, et lui dit qu'il ne l'avait pas trouvée : 'J'ai interrogé les gens du lieu, et ils m'ont dit : 'Il n'y a pas de prostituée ici.' '
\par 16 Et il dit : « Qu'elle les garde, de peur que nous ne devenions une cause de dérision. » Et lorsqu'elle eut accompli trois mois, il fut manifeste qu'elle était enceinte, et ils le dirent à Juda: «Voici Tamar, ta belle-fille, est enceinte par prostitution.»
\par 17 Et Juda se rendit à la maison de son père, et dit à son père et à ses frères : Faites-la sortir, et qu'ils la brûlent, car elle a commis des impuretés en Israël.
\par 18 Et il arriva, lorsqu'on l'amena pour la brûler, qu'elle envoya à son beau-père l'anneau, le collier et le bâton, en disant : « Discerne à qui sont ceux-ci, car c'est par lui que je suis avec enfant.'
\par 19 Et Juda reconnut et dit : Tamar est plus juste que moi.
\par 20 'Et donc qu'ils ne la brûlent pas' Et c'est pour cette raison qu'elle ne fut pas donnée à Schéla, et il ne s'approcha plus d'elle.
\par 21 Et après cela, elle enfanta deux fils, Perez [2170 AM] et Zerah, la septième année de cette deuxième semaine.
\par 22 Et alors furent accomplies les sept années de fécondité dont Joseph parla à Pharaon.
\par 23 Et Juda reconnut que l'action qu'il avait commise était mauvaise, car il avait couché avec sa belle-fille, et il estima cela odieux à ses yeux, et il reconnut qu'il avait transgressé et s'était égaré, car il avait découvert le vêtement de son fils, et il commença à se lamenter et à supplier devant le Seigneur à cause de sa transgression.
\par 24 Et nous lui avons dit dans un rêve que cela lui était pardonné parce qu'il avait supplié avec ferveur, et se lamenté, et ne l'avait plus commis.
\par 25 Et il a reçu le pardon parce qu'il s'est détourné de son péché et de son ignorance, car il a beaucoup transgressé devant notre Dieu ; et quiconque agit ainsi, quiconque couche avec sa belle-mère, qu'on le brûle au feu, afin qu'il y brûle, car il y a sur eux des impuretés et des souillures, qu'ils les brûlent au feu.
\par 26 Et tu ordonneras aux enfants d'Israël qu'il n'y ait aucune impureté parmi eux, car quiconque couche avec sa belle-fille ou avec sa belle-mère a commis une impureté ; qu'on brûle au feu l'homme qui a couché avec elle, ainsi que la femme, et il détournera d'Israël la colère et le châtiment.
\par 27 Et nous avons dit à Juda que ses deux fils n'avaient pas couché avec elle, et c'est pour cette raison que sa postérité était établie pour une seconde génération et ne serait pas déracinée.
\par 28 Car, avec un regard simple, il était allé chercher un châtiment, à savoir que, selon le jugement d'Abraham, qu'il avait ordonné à ses fils, Juda avait cherché à la brûler par le feu.

\chapitre{42}

\par \textit{En raison de la famine, Jacob envoie ses fils en Égypte chercher du blé, 1-4. Joseph les reconnaît et retient Siméon, et leur demande d'amener Benjamin à leur retour, 5-12. Malgré la réticence de Jacob, ses fils emmènent Benjamin avec eux lors de leur deuxième voyage et sont divertis par Joseph, 13-25. (Cf. Gen. xli.54, 56 ; xlii.7-9, 13, 17, 20, 24-5, 29-30, 34-8 ; xliii.1-2, 4-5, 8-9, 11 , 15, 23, 26, 29, 34 ; xliv. 1-2.)}

\par 1 Et la première année de la troisième semaine du quarante-cinquième jubilé, la famine commença à s'abattre sur le pays, et la pluie refusa de tomber sur la terre, car il n'en tomba rien.
\par 2 Et la terre devint stérile, mais dans le pays d'Égypte il y avait de la nourriture, car Joseph avait récolté la semence de la terre pendant les sept années d'abondance et l'avait conservée.
\par 3 Et les Egyptiens vinrent vers Joseph pour qu'il leur donne de la nourriture, et il ouvrit les magasins où se trouvait le grain de la première année, et il le vendit aux gens du pays contre de l'or.
\par 4 (Or la famine était très grande au pays de Canaan), et Jacob apprit qu'il y avait de la nourriture en Égypte, et il envoya ses dix fils pour qu'ils lui procurent de la nourriture en Égypte ; mais il n'envoya pas Benjamin, et (les dix fils de Jacob) arrivèrent (en Égypte) parmi ceux qui y allaient.
\par 5 Et Joseph les reconnut, mais ils ne le reconnurent pas, et il leur parla et les interrogea, et il leur dit : « N'êtes-vous pas des espions et n'êtes-vous pas venus explorer les abords du pays ? «Et il les a mis en cellule.
\par 6 Et après cela il les relâcha de nouveau, et retint Siméon seul et renvoya ses neuf frères.
\par 7 Et il remplit leurs sacs de blé, et il mit leur or dans leurs sacs, et ils ne le savaient pas.
\par 8 Et il leur ordonna d'amener leur jeune frère, car ils lui avaient dit que leur père était vivant et leur jeune frère.
\par 9 Et ils remontèrent du pays d'Égypte et arrivèrent au pays de Canaan ; Ils racontèrent à leur père tout ce qui leur était arrivé, et comment le seigneur du pays leur avait parlé durement et s'était emparé de Siméon jusqu'à ce qu'ils amènent Benjamin.
\par 10 Et Jacob dit : « Vous m'avez privé de mes enfants ! Joseph n'est pas là et Siméon n'est pas non plus, et vous emmènerez Benjamin. Votre méchanceté est venue sur moi.
\par 11 Et il dit : Mon fils ne descendra pas avec vous, de peur qu'il ne tombe malade ; car leur mère a donné naissance à deux fils, et un est péri, et vous m'enlèverez aussi celui-là. Si par hasard il prenait de la fièvre en route, vous feriez descendre ma vieillesse avec tristesse jusqu'à la mort.
\par 12 Car il voyait que leur argent avait été restitué à chacun dans son sac, et c'est pour cette raison qu'il craignait de l'envoyer.
\par 13 Et la famine augmenta et devint aiguë dans le pays de Canaan et dans tous les pays sauf au pays d'Égypte, car beaucoup d'enfants des Égyptiens avaient emmagasiné leur semence pour se nourrir depuis le moment où ils virent Joseph rassembler. semez ensemble, et mettez-les dans des magasins et conservez-les pour les années de famine.
\par 14 Et le peuple égyptien s'en nourrit pendant la première année de sa famine.
\par 15 Mais quand Israël vit que la famine était très grande dans le pays et qu'il n'y avait pas de délivrance, il dit à ses fils : « Retournez et procurez-nous de la nourriture afin que nous ne mourrions pas.
\par 16 Et ils dirent : « Nous n'irons pas ; Si notre plus jeune frère ne nous accompagne pas, nous n'irons pas.
\par 17 Et Israël vit que s'il ne l'envoyait pas avec eux, ils périraient tous à cause de la famine.
\par 18 Et Ruben dit : Livre-le entre mes mains, et si je ne te le ramène pas, tue mes deux fils au lieu de son âme.
\par 19 Et il lui dit : Il n'ira pas avec toi. Et Juda s'approcha et dit : « Envoie-le avec moi, et si je ne te le ramène pas, que j'en porte la responsabilité devant toi tous les jours de ma vie. »
\par 20 Et il l'envoya avec eux la deuxième année de cette semaine, le premier jour du mois [2172 Am], et ils arrivèrent au pays d'Égypte avec tous ceux qui y étaient allés, et (ils avaient) des cadeaux dans leur mains, stacte et amandes et noix de térébinthe et miel pur.
\par 21 Et ils allèrent et se tinrent devant Joseph, et il vit Benjamin son frère, et il le connut et leur dit : « Est-ce votre plus jeune frère ? Et ils lui dirent : « C'est lui. » Et il dit : « Que le Seigneur te fasse grâce, mon fils !
\par 22 Et il l'envoya dans sa maison et leur amena Siméon et il leur fit un festin, et ils lui présentèrent le présent qu'ils avaient apporté entre leurs mains.
\par 23 Et ils mangèrent devant lui, et il leur donna à tous une part, mais la part de Benjamin était sept fois plus grande que celle de n'importe lequel des leurs.
\par 24 Et ils mangèrent et burent, puis se levèrent et restèrent avec leurs ânes.
\par 25 Et Joseph imagina un plan par lequel il pourrait connaître leurs pensées pour savoir si les pensées de paix prévalaient parmi eux, et il dit à l'intendant qui était sur sa maison : « Remplissez tous leurs sacs de nourriture et rendez-leur leur argent. dans leurs vases, et ma coupe, la coupe d'argent dans laquelle je bois, mets-la dans le sac des plus jeunes, et renvoie-les.

\chapitre{43}

\par \textit{Le plan de Joseph pour rester ses frères, 1-10. Supplication de Juda, 11-13. Joseph se fait connaître de ses frères et les renvoie chercher son père, 14-24. (Cf. Gen. xliv.3-10, 12-18, 27-8, 30-2 ; xlv.1-2, 5-9, 12, 18, 20-1, 23, 25-8.)}

\par 1 Et il fit ce que Joseph lui avait dit, et remplit pour eux tous leurs sacs de nourriture, et mit leur argent dans leurs sacs, et mit la coupe dans le sac de Benjamin.
\par 2 Et ils partirent de bon matin, et il arriva que, lorsqu'ils furent partis de là, Joseph dit à l'intendant de sa maison : « Poursuivez-les, courez et saisissez-les, en disant : « Pour votre bien, vous avez tu m'as rendu du mal ; tu m'as volé la coupe d'argent dans laquelle boit mon seigneur. Et ramène-moi leur plus jeune frère, et va-le vite chercher avant que je me rende à mon siège de jugement.
\par 3 Et il courut après eux et leur dit selon ces paroles.
\par 4 Et ils lui dirent : « À Dieu ne plaise que tes serviteurs fassent cela et volent dans la maison de ton seigneur tout ustensile, ainsi que l'argent que nous avons trouvé dans nos sacs la première fois, nous, tes serviteurs, l'avons ramené. du pays de Canaan.
\par 5 « Comment alors devrions-nous voler un ustensile ? Voici, nous sommes en train de fouiller avec nos sacs, et partout où tu trouveras la coupe dans le sac de quelqu'un d'entre nous, qu'il soit tué, et nous et nos ânes servirons ton seigneur.
\par 6 Et il leur dit : 'Non, l'homme chez qui je trouve, je le prendrai seul pour serviteur, et vous retournerez en paix dans votre maison.'
\par 7 Et comme il fouillait dans leurs vases, en commençant par l'aîné et en terminant par le plus jeune, on le trouva dans le sac de Benjamin.
\par 8 Et ils déchirèrent leurs vêtements, et chargèrent leurs ânes, et retournèrent à la ville et arrivèrent à la maison de Joseph, et ils se prosternèrent tous le visage contre terre devant lui.
\par 9 Et Joseph leur dit : 'Vous avez fait du mal.' Et ils dirent : « Que dirons-nous et comment nous défendrons-nous ? Notre seigneur a découvert la transgression de ses serviteurs ; voici, nous sommes les serviteurs de notre seigneur, et nos ânes aussi.
\par 10 Et Joseph leur dit : Moi aussi, je crains l'Éternel ; quant à vous, rentrez chez vous et que votre frère soit mon serviteur, car vous avez fait le mal. Ne savez-vous pas qu'un homme se réjouit de sa coupe comme moi de cette coupe ? Et pourtant vous me l’avez volé.
\par 11 Et Juda dit : 'Ô mon seigneur, que ton serviteur, je te prie, dise un mot à l'oreille de mon seigneur. La mère de ton serviteur a donné deux frères à notre père : l'un s'en est allé, s'est perdu et n'a pas été retrouvé. , et lui seul reste de sa mère, et ton serviteur notre père l'aime, et sa vie aussi est liée à la vie de ce (garçon).'
\par 12 «Et il arrivera que, lorsque nous irons vers ton serviteur notre père, et que le garçon ne soit pas avec nous, il mourra, et nous ferons tomber notre père avec tristesse jusqu'à la mort.»
\par 13 Maintenant, laisse-moi plutôt, ton serviteur, demeurer à la place du garçon comme esclave de mon seigneur, et laisse le garçon partir avec ses frères, car je me suis porté garant pour lui auprès de ton serviteur notre père, et si Je ne le ramène pas, ton serviteur entendra toujours le blâme envers notre père.
\par 14 Et Joseph vit qu'ils étaient tous d'accord en bonté les uns avec les autres, et il ne put s'en empêcher, et il leur dit qu'il était Joseph.
\par 15 Et il s'entretenait avec eux en langue hébraïque, se jeta à leur cou et pleura.
\par 16 Mais ils ne le connaissaient pas et ils se mirent à pleurer. Et il leur dit : « Ne pleurez pas sur moi, mais dépêchez-vous et amenez-moi mon père ; et vous voyez que c'est ma bouche qui parle et que les yeux de mon frère Benjamin voient.
\par 17 'Car voici, c'est la deuxième année de famine, et il y a encore cinq années sans récolte ni fruit d'arbres ni labourage.'
\par 18 « Descendez vite, vous et vos maisons, afin que vous ne périssiez pas par la famine, et que vous ne soyez pas attristés à cause de vos biens, car l'Éternel m'a envoyé devant vous pour mettre les choses en ordre afin que beaucoup de gens vivent. »
\par 19 «Et dites à mon père que je suis encore en vie, et vous voyez, vous voyez que l'Éternel m'a établi comme père de Pharaon et comme chef de sa maison et de tout le pays d'Égypte.»
\par 20 'Et raconte à mon père toute ma gloire, ainsi que toutes les richesses et la gloire que l'Éternel m'a données.'
\par 21 Et sur l'ordre de la bouche de Pharaon, il leur donna des chars et des provisions pour le chemin, et il leur donna à tous des vêtements multicolores et de l'argent.
\par 22 Et il envoya à leur père des vêtements, de l'argent et dix ânes qui transportaient du blé, et il les renvoya.
\par 23 Et ils montèrent et dirent à leur père que Joseph était vivant, qu'il mesurait le blé pour toutes les nations de la terre, et qu'il était le chef de tout le pays d'Égypte.
\par 24 Et leur père ne le crut pas, car il était hors de lui dans son esprit ; mais quand il vit les chariots que Joseph avait envoyés, la vie de son esprit revint, et il dit : « Cela me suffit si Joseph vit ; Je descendrai le voir avant de mourir.

\chapitre{44}

\par \textit{Jacob célèbre la fête des prémices, et encouragé par une vision descend en Egypte, 1-10. Noms de ses descendants, 11-34. (Cf. Gen. XLVI.1-28.)}

\par 1 Et Israël partit de Haran depuis sa maison à la nouvelle lune du troisième mois, et il suivit le chemin du Puits du Serment, et il offrit un sacrifice au Dieu de son père Isaac le septième. de ce mois.
\par 2 Et Jacob se souvint du rêve qu'il avait vu à Béthel, et il craignait de descendre en Égypte.
\par 3 Et pendant qu'il pensait faire dire à Joseph de venir vers lui, et qu'il ne descendrait pas, il resta là sept jours, si par hasard il pouvait avoir une vision pour savoir s'il devait rester ou descendre.
\par 4 Et il célébra la fête de la moisson des prémices avec les vieux grains, car dans tout le pays de Canaan il n'y avait pas une poignée de semence [dans le pays], car la famine était sur toutes les bêtes, le bétail et les oiseaux. , et aussi sur l'homme.
\par 5 Et le seizième, le Seigneur lui apparut et lui dit : Jacob, Jacob ; et il dit : « Me voici. » Et il lui dit : « Je suis le Dieu de tes pères, le Dieu d'Abraham et d'Isaac ; ne crains pas de descendre en Égypte, car je ferai de toi une grande nation. Je descendrai avec toi et je te ferai remonter, et dans ce pays tu seras enterré, et Joseph mettra ses mains sur tes yeux.
\par 6 'Ne craignez rien ; descends en Égypte.
\par 7 Et ses fils se levèrent, ainsi que les fils de ses fils, et ils mirent leur père et leurs biens sur des chariots.
\par 8 Et Israël se leva du puits du Serment le seizième de ce troisième mois, et il se rendit au pays d'Égypte.
\par 9 Et Israël envoya Juda devant lui vers son fils Joseph pour examiner le pays de Goshen, car Joseph avait dit à ses frères qu'ils devaient venir y habiter afin d'être près de lui.
\par 10 Et c'était le plus beau (pays) du pays d'Egypte et à proximité de lui, pour tous (eux) et aussi pour le bétail.
\par 11 Et voici les noms des fils de Jacob qui allèrent en Égypte avec Jacob leur père.
\par 12 Ruben, le premier-né d'Israël; et voici les noms de ses fils Enoch, Pallu, Hetsron et Carmi-cinq.
\par 13 Siméon et ses fils ; Et voici les noms de ses fils : Jemuel, et Jamin, et Ohad, et Jakin, et Zohar, et Shaul, fils de la femme de Zephathite, sept.
\par 14 Lévi et ses fils ; et voici les noms de ses fils : Guershon, et Kehath, et Merari-quatre.
\par 15 Juda et ses fils ; et voici les noms de ses fils : Schéla, Perez et Zérach-quatre.
\par 16 Issacar et ses fils ; et voici les noms de ses fils : Tola, et Phua, et Jasub, et Shimron-cinq.
\par 17 Zabulon et ses fils ; et voici les noms de ses fils : Sered, Elon et Jahleel-quatre.
\par 18 Et voici les fils de Jacob et leurs fils que Léa enfanta à Jacob en Mésopotamie, six, et leur sœur unique, Dina, et toutes les âmes des fils de Léa et leurs fils, qui allèrent avec Jacob, leur père, en Egypte, ils étaient vingt-neuf, et Jacob, leur père, étant avec eux, ils étaient trente.
\par 19 Et les fils de Zilpa, servante de Léa, femme de Jacob, qui enfanta à Jacob Gad et Assur.
\par 20 Et voici les noms de leurs fils qui sont allés avec lui en Égypte. Fils de Gad : Ziphion, Haggi, Shuni, Ezbon (et Eri), Areli et Arodi, huit.
\par 21 Et les fils d'Aser : Jimna, et Ishva, (et Ishvi), et Beriah, et Serah, leur sœur unique, six.
\par 22 Toutes les âmes étaient quatorze, et toutes celles de Léa étaient quarante-quatre.
\par 23 Et les fils de Rachel, femme de Jacob : Joseph et Benjamin.
\par 24 Et Joseph naquit en Égypte, avant que son père vienne en Égypte, ceux que lui enfantèrent Asenath, fille de Potiphar, prêtre d'Héliopolis, Manassé et Éphraïm, trois.
\par 25 Et les fils de Benjamin : Béla, Beker et Ashbel, Guéra, et Naaman, et Ehi, et Rosh, et Muppim, et Huppim, et Ard-onze.
\par 26 Et toutes les âmes de Rachel étaient quatorze.
\par 27 Et les fils de Bilhah, servante de Rachel, femme de Jacob, qu'elle enfanta à Jacob, furent Dan et Nephtali.
\par 28 Et voici les noms de leurs fils qui sont allés avec eux en Égypte. Et les fils de Dan furent Hushim, Samon et Asudi. et 'Ijaka et Salomon-six.
\par 29 Et ils moururent l'année où ils entrèrent en Égypte, et il ne resta là que Dan Hushim.
\par 30 Et voici les noms des fils de Nephtali, Jahziel, Guni, Jezer, Shallum et 'Iv.
\par 31 Et 'Iv, qui était né après les années de famine, mourut en Egypte.
\par 32 Et toutes les âmes de Rachel étaient vingt-six.
\par 33 Et toutes les âmes de Jacob qui allèrent en Égypte étaient soixante-dix âmes. Ce sont là ses enfants et les enfants de ses enfants, au total soixante-dix, mais cinq moururent en Égypte avant Joseph et n'eurent pas d'enfants.
\par 34 Et dans le pays de Canaan, deux fils de Juda moururent, Er et Onan, et ils n'eurent pas d'enfants, et les enfants d'Israël enterrèrent ceux qui périrent, et ils furent comptés parmi les soixante-dix nations païennes.

\chapitre{45}

\par \textit{Joseph reçoit Jacob, et lui donne Goshen, 1-7. Joseph acquiert toute la terre et ses habitants pour Pharaon, 8-12. Jacob meurt et est enterré à Hébron, 13-15. Ses livres donnés à Lévi, 16. (Cf. Gen. xlvi.28-30 ; xlvii.11-13, 19, 20, 23, 24, 28 ; l.13.)}

\par 1 Et Israël entra dans le pays d'Égypte, dans le pays de Goshen, à la nouvelle lune du quatrième [2172 AM]. mois, la deuxième année de la troisième semaine du quarante-cinquième jubilé.
\par 2 Et Joseph alla à la rencontre de son père Jacob, au pays de Goshen, et il tomba au cou de son père et pleura.
\par 3 Et Israël dit à Joseph : « Maintenant, laisse-moi mourir depuis que je t'ai vu, et maintenant que le Seigneur Dieu d'Israël soit béni, le Dieu d'Abraham et le Dieu d'Isaac qui n'a pas refusé sa miséricorde et sa grâce à son serviteur Jacob.
\par 4 'Il me suffit d'avoir vu ton visage alors que je suis encore en vie ; oui, la vision que j'ai eue à Béthel est vraie. Béni soit le Seigneur mon Dieu pour toujours et à jamais, et béni soit son nom.
\par 5 Et Joseph et ses frères mangèrent du pain devant leur père et burent du vin, et Jacob se réjouit d'une très grande joie parce qu'il vit Joseph manger avec ses frères et boire devant lui, et il bénit le Créateur de toutes choses qui l'avait préservé, et il lui avait conservé ses douze fils.
\par 6 Et Joseph avait donné en cadeau à son père et à ses frères le droit de résider dans le pays de Goshen et à Ramsès et dans toute la région avoisinante, sur laquelle il régnait devant Pharaon. Et Israël et ses fils habitèrent dans le pays de Goshen, la meilleure partie du pays d'Égypte et Israël avait cent trente ans lorsqu'il entra en Égypte.
\par 7 Et Joseph nourrit son père et ses frères ainsi que leurs biens avec du pain autant qu'il leur en suffisait pour les sept années de famine.
\par 8 Et le pays d'Égypte souffrit à cause de la famine, et Joseph acquit tout le pays d'Égypte pour Pharaon en échange de nourriture, et il prit possession du peuple et de son bétail et de tout pour Pharaon.
\par 9 Et les années de famine furent accomplies, et Joseph donna au peuple du pays des semences et de la nourriture pour qu'ils puissent semer (le pays) la huitième année, car le fleuve avait inondé tout le pays d'Égypte.
\par 10 Car au cours des sept années de famine, elle n'avait pas débordé et n'avait irrigué que quelques endroits sur les rives du fleuve, mais maintenant elle a débordé et les Egyptiens ont semé la terre, et elle a produit beaucoup de blé cette année-là.
\par 11 Et c'était la première année de [2178 AM] la quatrième semaine du quarante-cinquième jubilé.
\par 12 Et Joseph prit du grain de la moisson un cinquième pour le roi et leur laissa quatre parts pour la nourriture et les semences, et Joseph en fit une ordonnance pour le pays d'Égypte jusqu'à ce jour.
\par 13 Et Israël vécut dix-sept ans dans le pays d'Égypte, et tous les jours qu'il vécut furent trois jubilés, cent quarante-sept ans, et il mourut la quatrième [2188 AM] année de la cinquième semaine du quarante-cinquième jubilé.
\par 14 Et Israël bénit ses fils avant de mourir et leur annonça tout ce qui leur arriverait au pays d'Égypte ; et il leur fit connaître ce qui leur arriverait dans les derniers jours, et il les bénit et donna à Joseph deux parts du pays.
\par 15 Et il dormit avec ses pères, et il fut enterré dans la double grotte du pays de Canaan, près d'Abraham, son père, dans la tombe qu'il s'était creusée dans la double grotte du pays d'Hébron.
\par 16 Et il donna tous ses livres et les livres de ses pères à Lévi, son fils, afin qu'il les conserve et les renouvelle pour ses enfants jusqu'à ce jour.

\chapitre{46}

\par \textit{Prospérité d'Israël en Egypte, 1-2. Mort de Joseph, 3-5. Guerre entre l'Egypte et Canaan durant laquelle les ossements de tous les fils de Jacob sauf Joseph sont enterrés à Hébron, 6-11. L'Égypte opprime Israël, 12-16. (Cf. Gen. l.22, 25-6 ; Exode i.6-14.)}

\par 1 Et il arriva qu'après la mort de Jacob, les enfants d'Israël se multiplièrent dans le pays d'Égypte, et ils devinrent une grande nation, et ils étaient d'un seul cœur, de sorte que le frère aimait son frère et que chacun aidait son frère. , et ils croissèrent abondamment et se multiplièrent extrêmement, dix [2242 AM] semaines d'années, tous les jours de la vie de Joseph.
\par 2 Et il n'y eut ni Satan ni aucun mal pendant tous les jours de la vie de Joseph qu'il vécut après son père Jacob, car tous les Egyptiens honorèrent les enfants d'Israël pendant tous les jours de la vie de Joseph.
\par 3 Et Joseph mourut âgé de cent dix ans ; Il vécut dix-sept ans dans le pays de Canaan, et dix ans il fut serviteur, et trois ans en prison, et quatre-vingts ans il fut sous le roi, régnant sur tout le pays d'Égypte.
\par 4 Et il mourut, ainsi que tous ses frères et toute cette génération.
\par 5 Et il ordonna aux enfants d'Israël, avant de mourir, de porter ses ossements avec eux lorsqu'ils sortiraient du pays d'Égypte.
\par 6 Et il leur fit jurer sur ses os, car il savait que les Égyptiens ne le ramèneraient plus et ne l'enterreraient plus au pays de Canaan, car Makamaron, roi de Canaan, alors qu'il habitait au pays d'Assyrie, combattait au pays de Canaan. il y tua le roi d'Égypte et poursuivit les Égyptiens jusqu'aux portes d'Ermon.
\par 7 Mais il ne put entrer, car un autre, un nouveau roi, était devenu roi d'Egypte, et il était plus fort que lui, et il retourna au pays de Canaan, et les portes de l'Egypte furent fermées, et aucun sont sortis et personne n’est entré en Égypte.
\par 8 Et Joseph mourut au quarante-sixième jubilé, la sixième semaine de la deuxième année, et on l'enterra au pays d'Égypte, et [2242 AM] tous ses frères moururent après lui.
\par 9 Et le roi d'Égypte partit en guerre contre le roi de Canaan [2263 AM] au quarante-septième jubilé, la deuxième semaine de la deuxième année, et les enfants d'Israël rapportèrent tous les ossements des enfants. de Jacob, sauf les ossements de Joseph, et ils les enterrèrent dans un champ, dans la double grotte de la montagne.
\par 10 Et la plupart (d'entre eux) retournèrent en Égypte, mais quelques-uns d'entre eux restèrent dans les montagnes d'Hébron, et Amram, ton père, resta avec eux.
\par 11 Et le roi de Canaan fut victorieux sur le roi d'Egypte, et il ferma les portes de l'Egypte.
\par 12 Et il forma un mauvais plan contre les enfants d'Israël pour les affliger et il dit au peuple d'Égypte : 'Voici, le peuple des enfants d'Israël s'est accru et s'est multiplié plus que nous.'
\par 13 « Venez et traitons-les avec sagesse avant qu'ils ne deviennent trop nombreux, et affligeons-les de esclavage avant que la guerre ne nous frappe et avant qu'ils ne nous combattent eux aussi ; sinon ils se joindront à nos ennemis et les feront sortir de notre pays, car leur cœur et leur visage sont tournés vers le pays de Canaan.
\par 14 Et il leur donna des ordonnateurs pour les affliger de servitude ; et ils bâtirent des villes fortes pour Pharaon, Pithom et Raamsès et ils bâtirent tous les murs et toutes les fortifications qui étaient tombées dans les villes d'Égypte.
\par 15 Et ils les faisaient servir avec rigueur, et plus ils les traitaient mal, plus ils croissaient et se multipliaient.
\par 16 Et le peuple d'Égypte avait en abomination les enfants d'Israël

\chapitre{47}

\par \textit{Naissance de Moïse, 1-4. Adopté par la fille de Pharaon, 5-9 ans. Tue un Égyptien et s'enfuit (vers Madian), 10-12. (Cf. Exode. i.22 ; ii. 2-15.)}

\par 1 Et la septième semaine de la septième année, au quarante-septième jubilé, ton père sortit [2303 AM] du pays de Canaan, et tu es né la quatrième semaine, la sixième année de celui-ci, au [23 h 30] quarante-huitième jubilé ; c'était le temps de tribulation pour les enfants d'Israël.
\par 2 Et Pharaon, roi d'Egypte, leur ordonna de jeter dans le fleuve tous leurs enfants mâles qui naissaient.
\par 3 Et ils les ont jetés pendant sept mois jusqu'au jour de ta naissance
\par 4 Et ta mère t'a caché pendant trois mois, et ils ont parlé d'elle. Et elle fit une arche pour toi, et la recouvrit de poix et d'asphalte, et la plaça dans les drapeaux sur la rive du fleuve, et elle t'y plaça sept jours, et ta mère vint de nuit et t'allaita, et par jour, Miriam, ta sœur, t'a gardé des oiseaux.
\par 5 Et en ce temps-là, Tharmuth, fille de Pharaon, vint se baigner dans le fleuve, et elle entendit ta voix qui criait, et elle dit à ses servantes de te faire sortir, et elles t'amenèrent vers elle.
\par 6 Et elle t'a fait sortir de l'arche, et elle a eu compassion de toi.
\par 7 Et ta sœur lui dit : « Dois-je aller t'appeler une des femmes hébraïques pour allaiter et allaiter ce bébé pour toi ?
\par 8 Et elle dit (lui) : 'Va.' Et elle alla appeler ta mère Jokébed, et elle lui donna son salaire, et elle t'allaita.
\par 9 Et ensuite, quand tu fus grand, ils t'amèrent auprès de la fille de Pharaon, et tu devins son fils, et Amram, ton père, t'apprit à écrire, et après que tu eus accompli trois semaines, ils t'amenèrent dans la cour royale. .
\par 10 Et tu restas trois semaines d'années à la cour jusqu'au moment [2351-] où tu sortis de la cour royale et où tu vis un Égyptien frapper ton ami qui était [2372 AM] des enfants d'Israël, et tu tu l'as tué et tu l'as caché dans le sable.
\par 11 Et le deuxième jour, tu te disputais avec deux des enfants d'Israël, et tu disais à celui qui faisait le mal : Pourquoi frappes-tu ton frère ?
\par 12 Et il fut en colère et indigné, et dit : « Qui t'a établi prince et juge sur nous ? Penses-tu me tuer comme tu as tué l'Égyptien hier ? Et tu as eu peur et tu as fui à cause de ces paroles.

\chapitre{48}

\par \textit{Moïse revient de Madian en Egypte. Mastêmâ cherche à le tuer en chemin, 1-3. Les dix plaies, 4-11. Israël sort d'Egypte : la destruction des Egyptiens sur la mer Rouge, 12-19. (Cf. Exode ii.15 ; iv.19, 24 ; vii. suiv.)}

\par 1 Et la sixième année de la troisième semaine du quarante-neuvième jubilé tu partis et tu habitas (au [2372 AM] le pays de Madian), cinq semaines et un an. Et tu es retourné en Égypte la deuxième semaine de la deuxième année du cinquantième jubilé.
\par 2 Et tu sais toi-même ce qu'il t'a dit sur le mont Sinaï, et ce que le prince Mastêmâ désirait faire de toi quand tu retournais en Egypte (en chemin, quand tu l'as rencontré au lieu d'hébergement) .
\par 3 N'a-t-il pas cherché de toutes ses forces à te tuer et à délivrer les Égyptiens de ta main, quand il a vu que tu étais envoyé pour exécuter le jugement et la vengeance contre les Égyptiens ?
\par 4 Et je t'ai délivré de sa main, et tu as accompli les signes et les prodiges que tu avais été envoyé accomplir en Égypte contre Pharaon et contre toute sa maison, et contre ses serviteurs et son peuple.
\par 5 Et l'Éternel exerça une grande vengeance contre eux, à cause d'Israël, et les frappa par le sang et les grenouilles, les poux et les mouches des chiens, et les furoncles malins qui éclataient en plaies ; et leur bétail par la mort ; et par la grêle, il détruisit ainsi tout ce qui poussait pour eux ; et par les sauterelles qui dévorèrent les restes laissés par la grêle et par les ténèbres ; et (par la mort) des premiers-nés des hommes et des animaux, et de toutes leurs idoles, l'Éternel se vengea et les brûla au feu.
\par 6 Et tout t'a été envoyé par ta main, pour que tu déclares (ces choses) avant qu'elles soient accomplies, et tu as parlé avec le roi d'Egypte devant tous ses serviteurs et devant son peuple.
\par 7 Et tout s'est passé selon tes paroles ; dix jugements grands et terribles sont tombés sur le pays d'Égypte, afin que tu puisses y exercer ta vengeance pour Israël.
\par 8 Et l'Éternel fit tout pour l'amour d'Israël, et selon son alliance, qu'il avait établie avec Abraham, selon laquelle il se vengerait d'eux, car ils les avaient réduits en esclavage par la force.
\par 9 Et le prince Mastêmâ se dressa contre toi et chercha à te jeter entre les mains de Pharaon, et il secourut les sorciers égyptiens,
\par 10 Et ils se sont levés et ont fait devant toi les maux que nous leur avons permis de faire, mais les remèdes que nous n'avons pas permis qu'ils soient faits par leurs mains.
\par 11 Et l'Éternel les frappa d'ulcères malins, et ils ne purent tenir debout, car nous les détruisîmes de sorte qu'ils ne purent accomplir un seul signe.
\par 12 Et malgré tous (ces) signes et prodiges, le prince Mastêmâ n'a pas été honteux parce qu'il a pris courage et a crié aux Égyptiens de te poursuivre avec toutes les puissances des Égyptiens, avec leurs chars et avec leurs chevaux, et avec toutes les armées des peuples d'Égypte.
\par 13 Et je me tenais entre les Égyptiens et Israël, et nous délivrâmes Israël de sa main et de la main de son peuple, et l'Éternel les fit traverser la mer comme s'il s'agissait d'une terre ferme.
\par 14 Et tous les peuples qu'il a amenés à poursuivre après Israël, l'Éternel notre Dieu les a jetés au milieu de la mer, dans les profondeurs de l'abîme, sous les enfants d'Israël, comme le peuple d'Égypte avait jeté ses enfants. Dans le fleuve, il se vengea d'un million d'entre eux, et mille hommes forts et énergiques furent détruits à cause d'un seul nourrisson des enfants de ton peuple qu'ils avaient jetés dans le fleuve.
\par 15 Et le quatorzième jour et le quinzième et le seizième et le dix-septième et le dix-huitième, le prince Mastêmâ fut lié et emprisonné derrière les enfants d'Israël pour ne pas les accuser.
\par 16 Et le dix-neuvième, nous les avons relâchés pour qu'ils puissent aider les Égyptiens et poursuivre les enfants d'Israël.
\par 17 Et il endurcit leurs cœurs et les rendit obstinés, et l'Éternel notre Dieu conçut un plan pour frapper les Égyptiens et les jeter à la mer.
\par 18 Et le quatorzième, nous l'avons lié afin qu'il ne puisse pas accuser les enfants d'Israël le jour où ils demandaient aux Égyptiens des ustensiles et des vêtements, des ustensiles d'argent, des ustensiles d'or et des ustensiles d'airain, afin de les piller. les Égyptiens en échange de la servitude dans laquelle ils les avaient forcés à servir.
\par 19 Et nous n'avons pas fait sortir les enfants d'Israël d'Egypte les mains vides.

\chapitre{49}

\par \textit{La Pâque : règles concernant sa célébration. (Cf. Exode XII.6, 9, 11, 13, 22-3, 30, 46 ; XV.22.)}

\par 1 Souviens-toi du commandement que l'Éternel t'a ordonné concernant la Pâque, que tu la célèbres en sa saison, le quatorzième du premier mois, que tu la tueras avant le soir, et qu'ils la mangeront la nuit. le 15 au soir à compter du coucher du soleil.
\par 2 Car cette nuit-là - le début de la fête et le début de la joie - vous mangiez la Pâque en Egypte, quand toutes les puissances de Mastêmâ avaient été lâchées pour tuer tous les premiers-nés du pays d'Egypte. , depuis le premier-né de Pharaon jusqu'au premier-né de la servante captive du moulin, et jusqu'au bétail.
\par 3 Et voici le signe que le Seigneur leur a donné : Dans chaque maison sur les linteaux de laquelle ils ont vu le sang d'un agneau d'un an, dans (cette) maison ils ne devaient pas entrer pour tuer, mais devaient passer par là. (il), afin que soient sauvés tous ceux qui étaient dans la maison parce que le signe du sang était sur ses linteaux.
\par 4 Et les puissances de l'Éternel firent tout selon ce que l'Éternel leur avait ordonné, et ils passèrent par tous les enfants d'Israël, et la peste ne vint pas sur eux pour détruire du milieu d'eux quiconque soit du bétail, soit un homme, soit chien.
\par 5 Et la peste était très grave en Egypte, et il n'y avait pas de maison en Egypte où il n'y ait pas un mort, avec des pleurs et des lamentations.
\par 6 Et tout Israël mangeait la chair de l'agneau pascal et buvait le vin, et louait, bénissait et rendait grâces au Seigneur, le Dieu de leurs pères, et était prêt à sortir du joug de l'Égypte. , et de la mauvaise servitude.
\par 7 Et souviens-toi de ce jour tous les jours de ta vie, et observe-le d'année en année tous les jours de ta vie, une fois par an, à son jour, selon toute sa loi, et ne l'ajourne pas (il ) de jour en jour ou de mois en mois.
\par 8 Car c'est une ordonnance éternelle, et gravée sur les tablettes célestes concernant tous les enfants d'Israël qu'ils doivent l'observer chaque année, à son jour, une fois par an, dans toutes leurs générations ; et il n'y a pas de limite de jours, car cela est ordonné pour toujours.
\par 9 Et l'homme qui est exempt d'impureté et qui ne vient pas la célébrer à l'occasion de son jour, afin d'apporter une offrande agréable devant l'Éternel, et de manger et de boire devant l'Éternel le jour de son fête, cet homme qui est pur et à portée de main sera retranché ; parce qu'il n'a pas offert l'offrande de l'Éternel au temps fixé, il en portera la culpabilité sur lui.
\par 10 Que les enfants d'Israël viennent célébrer la Pâque le jour de son heure fixée, le quatorzième jour du premier mois, entre les soirées, depuis la troisième partie du jour jusqu'à la troisième partie de la nuit, car deux parties du jour sont consacrées à la lumière et une troisième partie au soir.
\par 11 C'est ce que le Seigneur t'a commandé de l'observer entre les soirs.
\par 12 Et il n'est pas permis de le tuer à aucune période de la lumière, mais pendant la période limitrophe du soir, et qu'ils le mangent à l'heure du soir, jusqu'à la troisième partie de la nuit, et quoi qu'il arrive. Le reste de toute sa chair depuis le tiers de la nuit et au-delà, qu'on le brûle au feu.
\par 13 Et ils ne le feront pas cuire avec de l'eau, ni le mangeront cru, mais ils le rôtiront sur le feu ; ils le mangeront avec diligence, ils rôtiront sa tête avec l'intérieur et ses pieds au feu, et ne le briseront pas. tout os de celui-ci ; car aucun os des enfants d’Israël ne sera brisé.
\par 14 C'est pourquoi l'Éternel a ordonné aux enfants d'Israël d'observer la Pâque le jour de son heure fixée, et ils n'en briseront aucun os ; car c'est un jour de fête, et un jour commandé, et il ne peut y avoir de passage d'un jour à l'autre, ni d'un mois à l'autre, sans que le jour de sa fête soit observé.
\par 15 Et tu ordonneras aux enfants d'Israël d'observer la Pâque tout au long de leurs jours, chaque année, une fois par an, au jour de son heure fixée, et elle viendra comme un mémorial agréable devant l'Éternel, et aucune plaie ne viendra. venez sur eux pour tuer ou frapper l'année où ils célèbrent la Pâque en son temps, à tous égards selon son commandement.
\par 16 Et ils ne le mangeront pas hors du sanctuaire de l'Éternel, mais devant le sanctuaire de l'Éternel, et tout le peuple de la congrégation d'Israël le célébrera au temps fixé.
\par 17 Et tout homme qui arrivera à son jour en mangera dans le sanctuaire de ton Dieu devant l'Éternel, depuis l'âge de vingt ans et au-dessus ; car ainsi est-il écrit et ordonné qu'ils en mangent dans le sanctuaire du Seigneur.
\par 18 Et lorsque les enfants d'Israël entreront dans le pays qu'ils doivent posséder, dans le pays de Canaan, et établiront le tabernacle de l'Éternel au milieu du pays dans l'une de leurs tribus jusqu'au sanctuaire de l'Éternel. a été bâti dans le pays, qu'ils viennent célébrer la Pâque au milieu du tabernacle de l'Éternel, et qu'ils l'égorgent devant l'Éternel d'année en année.
\par 19 Et les jours où la maison sera bâtie au nom de l'Éternel dans le pays de leur héritage, ils y iront et célébreront la Pâque le soir, au coucher du soleil, à la troisième partie du jour.
\par 20 Et ils offriront son sang sur le seuil de l'autel, et placeront sa graisse sur le feu qui est sur l'autel, et ils mangeront sa chair rôtie au feu dans le parvis de la maison qui a été sanctifiée en le nom du Seigneur.
\par 21 Et ils ne pourront célébrer la Pâque dans leurs villes, ni en aucun autre lieu que devant le tabernacle de l'Éternel, ou devant sa maison où son nom a habité ; et ils ne s'éloigneront pas du Seigneur.
\par 22 Et toi, Moïse, ordonne aux enfants d'Israël d'observer les ordonnances de la Pâque, comme cela t'a été commandé ; déclare-leur chaque année et le jour de ses jours, ainsi que la fête des pains sans levain, qu'ils mangeront des pains sans levain sept jours, (et) qu'ils observeront sa fête, et qu'ils apporteront une offrande chaque jour pendant ces sept jours. des jours de joie devant le Seigneur sur l'autel de ton Dieu.
\par 23 Car vous avez célébré cette fête en toute hâte lorsque vous êtes sortis d'Égypte jusqu'à ce que vous soyez entrés dans le désert de Shur ; car sur le bord de la mer vous l'avez achevé.

\chapitre{50}

\par \textit{Lois concernant les jubilés, 1-5, et le sabbat, 6-13.}

\par 1 Et après cette loi, je t'ai fait connaître les jours des sabbats dans le désert de Sin[ai], qui est entre Elim et Sinaï.
\par 2 Et je t'ai parlé des sabbats du pays du mont Sinaï, et je t'ai parlé des années de jubilé dans les sabbats des années ; mais je ne t'en ai pas indiqué l'année jusqu'à ce que vous entriez dans le pays que vous devez posséder. .
\par 3 Et le pays aussi observera ses sabbats pendant qu'ils y habiteront, et ils connaîtront l'année du jubilé.
\par 4 C'est pourquoi je t'ai ordonné les années-semaines, les années et les jubilés : il y a quarante-neuf jubilés depuis les jours d'Adam jusqu'à ce jour, [2410 AM] et une semaine et deux ans : et il y a encore quarante ans à venir (lit. 'lointain') pour apprendre les commandements [2450 AM] du Seigneur, jusqu'à ce qu'ils passent au pays de Canaan, traversant le Jourdain à l'ouest.
\par 5 Et les jubilés passeront, jusqu'à ce qu'Israël soit purifié de toute culpabilité de fornication, et d'impureté, et de pollution, et de péché, et d'erreur, et habite avec confiance dans tout le pays, et il n'y aura plus de Satan ou tout méchant, et le pays sera pur à partir de ce moment-là pour toujours.
\par 6 Et voici le commandement concernant les sabbats - je les ai écrits pour toi - et tous les jugements de ses lois.
\par 7 Tu travailleras six jours, mais le septième jour est le sabbat de l'Éternel, ton Dieu. Vous n'y ferez aucun travail, vous et vos fils, vos serviteurs et vos servantes, et tout votre bétail, ainsi que l'étranger qui est avec vous.
\par 8 Et l'homme qui y fera un ouvrage mourra : quiconque profanera ce jour-là, quiconque couchera avec (sa) femme, ou quiconque dit qu'il fera quelque chose dessus, qu'il entreprendra un voyage là-dessus en ce qui concerne tout achat ou vente : et quiconque y puisera de l'eau qu'il ne s'était pas préparée le sixième jour, et quiconque prend un fardeau pour la transporter hors de sa tente ou de sa maison, mourra.
\par 9 Vous ne ferez aucun travail le jour du sabbat, sauf ce que vous vous êtes préparé le sixième jour, afin de manger, de boire, de vous reposer, et d'observer le sabbat de tout travail ce jour-là, et de bénir le Seigneur ton Dieu, qui t'a donné un jour de fête et un jour saint ; et un jour du royaume saint pour tout Israël est ce jour parmi leurs jours pour toujours.
\par 10 Car grand est l'honneur que l'Éternel a donné à Israël de pouvoir manger, boire et se rassasier en ce jour de fête, et se reposer de tout travail qui appartient au travail des enfants des hommes, sauf brûler de l'encens et apporter oblations et sacrifices devant le Seigneur pendant les jours et les sabbats.
\par 11 Ce travail seul sera fait les jours de sabbat dans le sanctuaire de l'Éternel, ton Dieu ; afin qu'ils expient continuellement Israël par des sacrifices de jour en jour, en souvenir qui plaît à l'Éternel, et qu'il les reçoive toujours de jour en jour, selon ce qui t'a été commandé.
\par 12 Et quiconque y fait un travail quelconque, ou part en voyage, ou cultive (sa) ferme, que ce soit dans sa maison ou en tout autre lieu, et quiconque allume un feu, ou monte sur un animal, ou voyage en bateau sur la mer, et quiconque frappe ou tue quelque chose, ou tue une bête ou un oiseau, ou quiconque attrape un animal ou un oiseau ou un poisson, ou quiconque jeûne ou fait la guerre les jours du sabbat :
\par 13 L'homme qui fera l'une de ces choses le jour du sabbat mourra, de sorte que les enfants d'Israël observeront les sabbats selon les commandements concernant les sabbats du pays, comme il est écrit sur les tablettes qu'il a données. mes mains pour que je t'écrive les lois des saisons, et les saisons selon la division de leurs jours.
\par    
\par     Ci-après est complété le récit de la division des jours.

\end{document}