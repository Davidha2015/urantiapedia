\begin{document}

\title{Testament d'Abraham}

\part{Version 1}

\chapter{1}

\par 1 Abraham vécut la mesure de sa vie, neuf cent quatre-vingt-quinze ans, et ayant vécu toutes les années de sa vie dans le calme, la douceur et la justice, le juste était extrêmement hospitalier ;

\par 2 car, plantant sa tente au carrefour près du chêne de Mamré, il recevait tout le monde, riches et pauvres, rois et dirigeants, estropiés et faibles, amis et étrangers, voisins et voyageurs, tous le faisaient également le dévot, le tout saint, le juste et l'hospitalier Abraham divertissent.

\par 3 Même sur lui, cependant, vint le sort commun, inexorable et amer de la mort et la fin incertaine de la vie.

\par 4 C'est pourquoi le Seigneur Dieu, appelant son archange Michel, lui dit : Descends, capitaine Michel, vers Abraham et parle-lui de sa mort, afin qu'il mette de l'ordre dans ses affaires,

\par 5 Car je l'ai béni comme les étoiles du ciel et comme le sable au bord de la mer, et il a une longue vie en abondance et de nombreuses possessions, et il devient extrêmement riche. Au-delà de tous les hommes, il est juste en toute bonté, hospitalier et aimant jusqu'à la fin de sa vie ;

\par 6 mais va, archange Michel, vers Abraham, mon ami bien-aimé, et annonce-lui sa mort et assure-le ainsi :

\par 7 À ce moment-là, tu quitteras ce monde vain, tu quitteras le corps et tu iras vers ton propre Seigneur parmi les bons.

\chapter{2}

\par 1 Et le capitaine en chef partit devant la face de Dieu, et descendit vers Abraham au chêne de Mamré, et trouva le juste Abraham dans le champ voisin, assis à côté des attelages de bœufs pour labourer, avec les fils de Masek et d'autres serviteurs, au nombre de douze.

\par 2 Et voici, le capitaine en chef s'approcha de lui, et Abraham, voyant le capitaine en chef Michael venir de loin, comme un très beau guerrier, se leva et le rencontra comme c'était sa coutume, rencontrant et recevant tous les étrangers.

\par 3 Et le capitaine en chef le salua et dit : « Salut, père très honoré, âme juste choisie de Dieu, vrai fils du Céleste. »

\par 4 Abraham dit au capitaine en chef : « Salut, guerrier très honoré, brillant comme le soleil et le plus beau entre tous les fils des hommes ; je vous en prie;

\par 5 C'est pourquoi, je vous en supplie, dites-moi d'où vient la jeunesse de votre âge ; apprenez-moi, votre suppliante, d'où, de quelle armée et de quel voyage votre beauté est venue ici.

\par 6 Le capitaine en chef dit : « Moi, ô juste Abraham, je viens de la grande ville. J'ai été envoyé par le grand roi pour remplacer un de ses bons amis, car le roi l'a convoqué.

\par 7 Et Abraham dit : « Viens, mon Seigneur, viens avec moi jusqu'à mon champ ». Le capitaine en chef dit : « Je viens » ;

\par 8 et étant entrés dans le champ de labour, ils s'assirent à côté de la troupe.

\par 9 Et Abraham dit à ses serviteurs, les fils de Masek : « Allez au troupeau de chevaux et amenez deux chevaux calmes, doux et apprivoisés, afin que moi et cet étranger puissions nous asseoir dessus. »

\par 10 Mais le capitaine en chef dit : « Non, mon Seigneur, Abraham, qu'ils n'amènent pas de chevaux, car je m'abstiens de jamais m'asseoir sur une bête à quatre pattes.

\par 11 Mon roi n'est-il pas riche en marchandises, ayant pouvoir sur les hommes et sur toute espèce de bétail ? Mais je m'abstiens de jamais m'asseoir sur une bête à quatre pattes.

\par 12 Partons donc, ô âme juste, en marchant légèrement jusqu'à ce que nous arrivions à ta maison. Et Abraham dit : « Amen, qu’il en soit ainsi. »


\chapter{3}

\par 1 Et comme ils sortaient des champs vers sa maison,

\par 2 à côté de ce chemin se dressait un cyprès,

\par 3 et par l'ordre du Seigneur, l'arbre s'écria d'une voix humaine, disant : « Saint, saint, saint est le Seigneur Dieu qui s'appelle à ceux qui l'aiment ; »

\par 4 mais Abraham cacha le mystère, pensant que le capitaine en chef n'avait pas entendu la voix de l'arbre.

\par 5 Et s'approchant de la maison, ils s'assirent dans la cour, et Isaac, voyant le visage de l'ange, dit à Sarah, sa mère : « Ma dame, mère, voici, l'homme assis avec mon père Abraham n'est pas un fils du race de ceux qui habitent sur la terre.

\par 6 Et Isaac courut et le salua, et tomba aux pieds de l'Incorporel, et l'Incorporel le bénit et dit : « Le Seigneur Dieu t'accordera la promesse qu'il a faite à ton père Abraham et à sa postérité, et t'accordera également la précieuse prière de ton père et de ta mère.

\par 7 Abraham dit à Isaac, son fils : « Mon fils Isaac, puise de l'eau du puits et apporte-la-moi dans le vase, afin que nous puissions laver les pieds de cet étranger, car il est fatigué, étant venu vers nous de loin un long voyage.»

\par 8 Et Isaac courut au puits, puisa de l'eau dans le vase et le leur apporta.

\par 9 Et Abraham monta et lava les pieds du capitaine en chef Michel, et le cœur d'Abraham fut ému, et il pleura sur l'étranger.

\par 10 Et Isaac, voyant son père pleurer, pleura aussi, et le capitaine en chef, les voyant pleurer, pleura aussi avec eux,

\par 11 et les larmes du capitaine en chef tombèrent sur le navire dans l'eau du bassin et devinrent des pierres précieuses.

\par 12 Et Abraham voyant le prodige et étant étonné, prit les pierres en secret, et cacha le mystère, le gardant pour lui dans son cœur.

\chapter{4}

\par 1 Et Abraham dit à Isaac son fils : « Va, mon fils bien-aimé, dans la chambre intérieure de la maison et embellis-la. Étendez-nous là deux canapés, un pour moi et un pour cet homme qui est avec nous aujourd'hui.

\par 2 Préparez-nous là un siège, un chandelier et une table avec une abondance de toutes bonnes choses. Embellis la chambre, mon fils, et étends sous nous du lin, de la pourpre et du fin lin.

\par 3 Brûlez-y tous les encens précieux et excellents, rapportez des plantes odorantes du jardin et remplissez-en notre maison. Allumez sept lampes pleines d'huile, afin que nous puissions nous réjouir, car cet homme qui est notre jour est plus glorieux que les rois ou les dirigeants, et son apparence surpasse tous les fils des hommes.

\par 4 Et Isaac prépara tout bien, et Abraham, prenant l'archange Michel, entra dans la chambre, et ils s'assirent tous deux sur les canapés, et entre eux il plaça une table avec une abondance de toutes bonnes choses.

\par 5 Alors le capitaine en chef se leva et sortit, comme si son ventre était contraint de faire sortir de l'eau, et monta au ciel en un clin d'œil, et se tint devant l'Éternel et lui dit :

\par 6 « Seigneur et Maître, fais savoir à ta puissance que je suis incapable de rappeler sa mort à cet homme juste, car je n'ai pas vu sur la terre un homme comme lui, pitoyable, hospitalier, juste, véridique, pieux, s'abstenant de chaque mauvaise action. Et maintenant, sache, Seigneur, que je ne peux pas lui rappeler sa mort.

\par 7 Et le Seigneur dit : « Descendez, capitaine en chef Michel, chez mon ami Abraham, et faites tout ce qu'il vous dira, et mangez avec lui tout ce qu'il mange.

\par 8 Et j'enverrai mon Saint-Esprit sur son fils Isaac, et je mettrai le souvenir de sa mort dans le cœur d'Isaac, afin que même lui, dans un rêve, puisse voir la mort de son père, et Isaac racontera le rêve. , et vous l’interpréterez, et lui-même connaîtra sa fin.

\par 9 Et le capitaine en chef dit : « Seigneur, tous les esprits célestes sont incorporels, et ne mangent ni ne boivent, et cet homme a dressé devant moi une table avec une abondance de toutes les bonnes choses terrestres et corruptibles. Maintenant, Seigneur, que dois-je faire ? Comment lui échapper, assis à la même table avec lui ? »

\par 10 Le Seigneur dit : « Descendez vers lui et n'y pensez pas, car lorsque vous vous asseoirez avec lui, j'enverrai sur vous un esprit dévorant, et il consumera tout de vos mains et par votre bouche c'est sur la table. Réjouissez-vous avec lui en tout,

\par 11 Seulement, tu interpréteras bien les choses de la vision, afin qu'Abraham connaisse la faucille de la mort et la fin incertaine de la vie, et qu'il dispose de tous ses biens, car je l'ai béni au-dessus du sable de la mer et comme les étoiles du ciel.

\chapter{5}

\par 1 Alors le capitaine en chef descendit à la maison d'Abraham, et s'assit à table avec lui, et Isaac les servit.

\par 2 Et quand le souper fut terminé, Abraham pria selon sa coutume, et le capitaine en chef pria avec lui, et chacun se coucha pour dormir sur son lit.

\par 3 Et Isaac dit à son père : « Père, moi aussi je voudrais coucher avec toi dans cette chambre, afin que moi aussi j'entende ton discours, car j'aime entendre l'excellence de la conversation de cet homme vertueux. »

\par 4 Abraham dit : « Non, mon fils, mais va dans ta chambre et dors sur ton propre lit, de peur que nous ne gênions cet homme. »

\par 5 Alors Isaac, ayant reçu d'eux la prière et les ayant bénis, alla dans sa propre chambre et se coucha sur son lit.

\par 6 Mais le Seigneur jeta la pensée de la mort dans le cœur d'Isaac comme dans un rêve,

\par 7 Et vers la troisième heure de la nuit, Isaac se réveilla, se leva de son lit et courut vers la chambre où son père dormait avec l'archange.

\par 8 Isaac donc, en arrivant à la porte, s'écria, disant : « Mon père Abraham, lève-toi et ouvre-moi vite, afin que j'entre, que je m'accroche à ton cou et que je t'embrasse avant qu'ils ne t'éloignent de moi.

\par 9 Abraham se leva donc et s'ouvrit à lui, et Isaac entra et se pendit à son cou, et se mit à pleurer à haute voix.

\par 10 Abraham donc, ému dans son cœur, pleura aussi à haute voix, et le capitaine en chef, les voyant pleurer, pleura aussi.

\par 11 Sarah étant dans sa chambre, entendit leurs pleurs, courut vers eux et les trouva embrassant et pleurant.

\par 12 Et Sarah dit en pleurant : « Mon Seigneur Abraham, pourquoi pleures-tu ?

\par 13 Dis-moi, mon Seigneur, ce frère que nous avons reçu aujourd'hui t'a-t-il annoncé la mort de Lot, le fils de ton frère ? Est-ce pour cela que vous êtes ainsi affligé ?

\par 14 Le capitaine en chef répondit et lui dit : « Non, ma sœur Sarah, ce n'est pas comme tu le dis, mais il me semble que ton fils Isaac a eu un songe et est venu vers nous en pleurant, et nous, le voyant, avons été émus dans nos cœurs et nous avons pleuré.

\chapter{6}

\par 1 Alors Sarah, entendant l'excellence de la conversation du capitaine en chef, comprit aussitôt que c'était un ange du Seigneur qui parlait.

\par 2 Sarah fit donc signe à Abraham de sortir vers la porte, et lui dit : « Mon Seigneur Abraham, sais-tu qui est cet homme ?

\par 3 Abraham dit : « Je ne sais pas. »

\par 4 Sarah dit : « Tu connais, mon Seigneur, les trois hommes du ciel que nous avons reçus dans notre tente près du chêne de Mamré, lorsque tu as tué le chevreau sans défaut et que tu as dressé une table devant eux.

\par 5 Après que la chair eut été mangée, le chevreau se releva et tétait sa mère avec une grande joie. Ne sais-tu pas, mon Seigneur Abraham, que par promesse ils nous ont donné Isaac comme fruit des entrailles ? De ces trois saints hommes, celui-ci n’en est qu’un.

\par 6 Abraham dit : « Ô Sarah, en cela tu dis la vérité. Gloire et louange de notre Dieu et Père. Car tard le soir, quand je lui lavais les pieds dans le bassin, je disais dans mon cœur : Ce sont les pieds de l'un des trois hommes que j'ai lavé alors ;

\par 7 et ses larmes qui tombaient dans le bassin devinrent alors des pierres précieuses. Et les secouant de ses genoux, il les donna à Sara, en disant : Si tu ne me crois pas, regarde maintenant ceci.

\par 8 Et Sarah les recevant, se prosterna et salua et dit : « Gloire à Dieu qui nous montre des choses merveilleuses. Et maintenant, sache, mon Seigneur Abraham, qu'il y a parmi nous la révélation de quelque chose, qu'elle soit mauvaise ou bonne !


\chapter{7}

\par 1 Et Abraham quitta Sara, entra dans la chambre et dit à Isaac : Viens ici, mon fils bien-aimé, dis-moi la vérité, ce que tu as vu et ce qui t'est arrivé pour que tu sois venu si précipitamment vers nous.

\par 2 Et Isaac, répondant, commença à dire : « J'ai vu, mon Seigneur, cette nuit-là, le soleil et la lune au-dessus de ma tête, m'entourant de ses rayons et m'éclairant.

\par 3 Tandis que je regardais cela et me réjouissais, je vis le ciel ouvert, et un homme porteur de lumière en descendre, brillant plus que sept soleils.

\par 4 Et cet homme semblable au soleil est venu et a ôté le soleil de ma tête, et est monté dans les cieux d'où il était venu, mais j'ai été très affligé qu'il m'ait ôté le soleil.

\par 5 Peu de temps après, comme j'étais encore triste et très troublé, je vis cet homme sortir du ciel une seconde fois, et il m'enleva aussi la lune de dessus ma tête,

\par 6 Et j'ai pleuré beaucoup et j'ai invoqué cet homme de lumière, et je lui ai dit : Ne m'enlève pas ma gloire, mon Seigneur ; ayez pitié de moi et écoutez-moi, et si vous m'enlevez le soleil, laissez-moi la lune.

\par 7 Il dit : « Laissez-les monter vers le roi d'en haut, car il veut qu'ils soient là-bas ». Et il me les a enlevés, mais il a laissé les rayons sur moi.

\par 8 Le capitaine en chef dit : « Écoute, ô juste Abraham ! le soleil que ton fils a vu, c'est toi son père, et la lune aussi, c'est Sarah, sa mère. L’homme porteur de lumière qui est descendu du ciel, c’est celui envoyé de Dieu qui doit vous enlever votre âme juste.

\par 9 Et maintenant, sache, ô très honoré Abraham, qu'à ce moment-là tu quitteras cette vie mondaine et que tu t'en iras vers Dieu.

\par 10 Abraham dit au capitaine en chef : « Ô la plus étrange des merveilles ! Et maintenant, es-tu celui qui me prendra mon âme ?

\par 11 Le capitaine en chef lui dit : « Je suis le capitaine en chef Michel, qui me tiens devant le Seigneur, et j'ai été envoyé vers toi pour te rappeler ta mort, puis je m'en irai vers lui comme il m'a été commandé. .»

\par 12 Abraham dit : « Maintenant, je sais que tu es un ange du Seigneur et que tu as été envoyé pour prendre mon âme, mais je n'irai pas avec toi ; mais faites tout ce qu’on vous commande.

\chapter{8}

\par 1 Le capitaine en chef, entendant ces paroles, disparut aussitôt, et montant au ciel, se tint devant Dieu et raconta tout ce qu'il avait vu dans la maison d'Abraham ;

\par 2 Et le capitaine en chef dit aussi cela à son Seigneur : « Ainsi parle ton ami Abraham : Je n'irai pas avec toi, mais je ferai tout ce qu'on t'ordonnera ;

\par 3 et maintenant, ô Seigneur Tout-Puissant, est-ce que ta gloire et ton royaume immortel ordonnent quelque chose ?

\par 4 Dieu dit au capitaine en chef Michel : « Va encore une fois trouver mon ami Abraham et parle-lui ainsi :

\par 5 Ainsi parle l'Éternel, ton Dieu, celui qui t'a fait entrer dans le pays promis, qui t'a béni au-dessus du sable de la mer et au-dessus des étoiles du ciel,

\par 6 qui a ouvert le sein stérile de Sara, et qui t'a donné Isaac comme fruit du sein dans la vieillesse,

\par 7 En vérité, je vous le dis, en vous bénissant, je vous bénirai, et en multipliant, je multiplierai votre semence, et je vous donnerai tout ce que vous me demanderez, car je suis l'Éternel votre Dieu, et en dehors de moi il n'y a pas d'autre autre.

\par 8 Dis-moi pourquoi tu t'es rebellé contre moi, et pourquoi il y a du chagrin en toi, et pourquoi tu t'es rebellé contre mon archange Michel ?

\par 9 Ne savez-vous pas que tous ceux qui sont issus d'Adam et d'Ève sont morts, et qu'aucun des prophètes n'a échappé à la mort ? Aucun de ceux qui règnent en tant que rois n’est immortel ; aucun de vos ancêtres n’a échappé au mystère de la mort. Ils sont tous morts, ils sont tous partis dans l’Hadès, ils sont tous rassemblés par la faucille de la mort.

\par 10 Mais je n'ai pas envoyé sur vous la mort, je n'ai pas laissé tomber sur vous aucune maladie mortelle, je n'ai pas permis à la faucille de la mort de vous rencontrer, je n'ai pas permis aux filets de l'Hadès de vous envelopper, j'ai Je n'ai jamais souhaité que vous rencontriez un quelconque mal.

\par 11 Mais pour vous consoler, je vous ai envoyé mon capitaine en chef Michel, afin que vous sachiez votre départ du monde, que vous mettiez de l'ordre dans votre maison et tout ce qui vous appartient, et que vous bénissiez Isaac, votre fils bien-aimé. Et maintenant, sachez que j'ai fait cela sans vouloir vous attrister.

\par 12 Pourquoi donc as-tu dit à mon capitaine en chef : Je n'irai pas avec toi ? Pourquoi as-tu parlé ainsi ? Ne sais-tu pas que si j'autorise la mort et qu'il vient sur toi, je verrai si tu viendras ou non ?

\chapter{9}

\par 1 Et le capitaine en chef, recevant les exhortations de l'Éternel, descendit vers Abraham, et le voyant, le juste tomba la face contre terre comme un mort,

\par 2 et le capitaine en chef lui raconta tout ce qu'il avait entendu du Très-Haut. Alors le saint et juste Abraham, se levant avec beaucoup de larmes, tomba aux pieds de l'Incorporel et le supplia, disant :

\par 3 « Je t'en supplie, chef des armées d'en haut, puisque tu as entièrement daigné venir toi-même à moi pécheur et en toutes choses ton indigne serviteur, je t'en supplie même maintenant, ô chef capitaine, de porter mon parle encore une fois au Très-Haut, et tu lui diras :

\par 4 Ainsi parle Abraham ton serviteur : Seigneur, Seigneur, dans chaque œuvre et parole que je t'ai demandée, tu m'as entendu et tu as accompli tous mes conseils.

\par 5 Maintenant, Seigneur, je ne résiste pas à ta puissance, car moi aussi je sais que je ne suis pas immortel mais mortel. Puisque donc toutes choses se soumettent à ton commandement, et que je crains et tremble devant ta puissance, je crains aussi, mais je ne te demande qu'une seule requête :

\par 6 et maintenant, Seigneur et Maître, écoute ma prière, car pendant que je suis encore dans ce corps, je désire voir toute la terre habitée et toutes les créations que tu as établies par une seule parole, et quand je les verrai, alors si je veux quitte la vie, je serai sans chagrin.

\par 7 Alors le capitaine en chef revint et se tint devant Dieu et lui raconta tout, en disant : Ainsi parle ton ami Abraham : J'ai désiré voir toute la terre de mon vivant avant de mourir.

\par 8 Et le Très-Haut entendant cela, commanda de nouveau au capitaine en chef Michel, et lui dit : Prends une nuée de lumière et les anges qui ont pouvoir sur les chars, et descends, prends le juste Abraham sur un char des chérubins, et élève-le dans les airs du ciel afin qu'il puisse voir toute la terre.

\chapter{10}

\par 1 Et l'archange Michel descendit et prit Abraham sur le char des chérubins, et l'exalta dans les airs du ciel, et le conduisit sur la nuée avec soixante anges, et Abraham monta sur le char sur toute la terre.

\par 2 Et Abraham vit le monde tel qu'il était en ce jour-là, les uns labourant, d'autres conduisant des chariots, ici des hommes gardant les troupeaux, et ailleurs les gardant la nuit, et dansant, jouant et jouant de la harpe, ailleurs des hommes s'efforçant et se disputant en justice, ailleurs des hommes pleurant et se souvenant des morts.

\par 3 Il vit aussi les nouveaux mariés reçus avec honneur, et en un mot il vit tout ce qui se fait dans le monde, tant bien que mal.

\par 4 Abraham donc, passant au-dessus d'eux, vit des hommes portant des épées, brandissant dans leurs mains des épées aiguisées, et Abraham demanda au capitaine en chef : « Qui sont ceux-là ?

\par 5 Le capitaine en chef dit : « Ce sont des voleurs qui ont l'intention de commettre un meurtre, de voler, de brûler et de détruire. »

\par 6 Abraham dit : « Seigneur, Seigneur, écoute ma voix et commande que les bêtes sauvages sortent du bois et les dévorent. »

\par 7 Et pendant qu'il parlait, des bêtes sauvages sortirent du bois et les dévorèrent.

\par 8 Et il vit dans un autre endroit un homme et une femme se livrant à la fornication l'un avec l'autre,

\par 9 et il dit : « Seigneur, Seigneur, ordonne que la terre s'ouvre et les engloutisse, et aussitôt la terre se fend et les engloutit. »

\par 10 Et il vit ailleurs des hommes qui creusaient une maison et emportaient les biens d'autrui,

\par 11 et il dit : « Seigneur, Seigneur, ordonne que le feu descende du ciel et les consume. Et au moment même où il parlait, le feu descendit du ciel et les consuma.

\par 12 Et aussitôt une voix vint du ciel au capitaine en chef, disant ainsi : « Ô capitaine en chef Michel, ordonne au char de s'arrêter, et détourne Abraham afin qu'il ne voie pas toute la terre,

\par 13 car s'il voit tous ceux qui vivent dans le mal, il détruira toute la création. Car voici, Abraham n’a pas péché et n’a aucune pitié pour les pécheurs,

\par 14 mais j'ai créé le monde, et je ne désire détruire aucun d'entre eux, mais j'attends la mort du pécheur, jusqu'à ce qu'il se convertisse et vive.

\par 15 Mais emmenez Abraham jusqu'à la première porte du ciel, afin qu'il y voie les jugements et les récompenses, et qu'il se repente des âmes des pécheurs qu'il a détruits.

\chapter{11}

\par 1 Alors Michel fit tourner le char et conduisit Abraham vers l'est, jusqu'à la première porte du ciel ;

\par 2 Et Abraham vit deux chemins, l'un étroit et resserré, l'autre large et spacieux,

\par 3 et là il vit deux portes, l'une large sur le chemin large, et l'autre étroite sur le chemin étroit.

\par 4 Et là, hors des deux portes, il aperçut un homme assis sur un trône doré, et l'apparence de cet homme était terrible, comme celle de l'Éternel.

\par 5 Et ils virent beaucoup d'âmes conduites par les anges et conduites par la porte large, et d'autres âmes, en petit nombre, qui furent emmenées par les anges par la porte étroite.

\par 6 Et quand l'homme merveilleux qui était assis sur le trône d'or vit peu d'entre eux entrer par la porte étroite, et beaucoup entrer par la porte large, aussitôt cet homme merveilleux s'arracha les cheveux de sa tête et les côtés de sa barbe, et se jeta. à terre depuis son trône, pleurant et se lamentant.

\par 7 Mais quand il vit beaucoup d'âmes entrer par la porte étroite, alors il se leva de terre et s'assit sur son trône dans une grande joie, se réjouissant et exultant.

\par 8 Et Abraham demanda au capitaine en chef : « Mon Seigneur, capitaine en chef, quel est cet homme le plus merveilleux, orné d'une telle gloire, et tantôt il pleure et se lamente, tantôt il se réjouit et exulte ?

\par 9 L'incorporel dit : « Celui-ci est l'Adam premier-créé qui est dans une telle gloire, et il regarde le monde parce que tous sont nés de lui,

\par 10 et quand il voit beaucoup d'âmes passer par la porte étroite, alors il se lève et s'assied sur son trône se réjouissant et exultant de joie, car cette porte étroite est celle des justes, qui mènent à la vie, et de ceux qui entrent par elle allez au Paradis. C'est pourquoi Adam, le premier créé, se réjouit, car il voit les âmes être sauvées.

\par 11 Mais quand il voit beaucoup d'âmes entrer par la porte large, alors il s'arrache les cheveux de la tête et se jette à terre en pleurant et en se lamentant amèrement, car la porte large est celle des pécheurs, qui mène à la perdition et à la destruction châtiment éternel. Et pour cela, Adam, le premier formé, tombe de son trône en pleurant et en se lamentant sur la destruction des pécheurs, car il y en a beaucoup qui sont perdus, et il y en a peu qui sont sauvés.

\par 12 car sur sept mille, à peine se trouve-t-on une âme sauvée, juste et sans souillure.

\chapter{12}

\par 1 Pendant qu'il me disait encore ces choses, voici deux anges, d'aspect enflammé, d'esprit impitoyable et sévère de regard, et ils se précipitaient sur des milliers d'âmes, les fouettant impitoyablement avec des lanières de feu.

\par 2 L'ange s'empara d'une seule âme, et ils poussèrent toutes les âmes par la grande porte de la perdition.

\par 3 Nous aussi, nous partîmes avec les anges, et nous entrâmes par cette large porte,

\par 4 et entre les deux portes se dressait un trône d'apparence terrible, d'un cristal terrible, brillant comme le feu,

\par 5 et dessus était assis un homme merveilleux, brillant comme le soleil, semblable au Fils de Dieu.

\par 6 Devant lui se tenait une table semblable à du cristal, toute d'or et de fin lin,

\par 7 Et sur la table il y avait un livre ayant six coudées d'épaisseur et dix coudées de largeur.

\par 8 et à droite et à gauche se tenaient deux anges tenant du papier, de l'encre et une plume.

\par 9 Devant la table était assis un ange de lumière, tenant dans sa main une balance,

\par 10 et à sa gauche était assis un ange tout enflammé, impitoyable et sévère, tenant dans sa main une trompette, ayant en elle un feu dévorant pour éprouver les pécheurs.

\par 11 L'homme merveilleux qui était assis sur le trône lui-même jugeait et condamnait les âmes,

\par 12 et les deux anges de droite et de gauche écrivirent, celui de droite la justice et celui de gauche la méchanceté.

\par 13 Celui qui était devant la table, qui tenait la balance, pesait les âmes,

\par 14 et l'ange de feu, qui tenait le feu, essayait les âmes.

\par 15 Et Abraham demanda au capitaine en chef Michel : « Qu'est-ce que nous voyons ? Et le capitaine en chef dit : « Ces choses que tu vois, saint Abraham, sont le jugement et la récompense.

\par 16 Et voici, l'ange tenait l'âme dans sa main, et il l'amena devant le juge,

\par 17 et le juge dit à l'un des anges qui le servaient : Ouvre-moi ce livre, et trouve-moi les péchés de cette âme.

\par 18 Et ouvrant le livre, il trouva que ses péchés et sa justice étaient également équilibrés, et il ne le donna ni aux bourreaux, ni à ceux qui étaient sauvés, mais le plaça au milieu.


\chapter{13}

\par 1 Et Abraham dit : « Mon Seigneur, capitaine en chef, qui est ce juge le plus merveilleux ? Et qui sont les anges qui écrivent ? Et qui est l’ange comme le soleil, qui tient la balance ? Et qui est l’ange de feu qui tient le feu ?

\par 2 Le capitaine en chef dit : « Vois-tu, très saint Abraham, l'homme terrible assis sur le trône ? C'est le fils du premier Adam créé, appelé Abel, que le méchant Caïn a tué,

\par 3 et il s'assied ainsi pour juger toute la création, et examine les justes et les pécheurs. Car Dieu a dit : Je ne vous jugerai pas, mais tout homme né d'un homme sera jugé.

\par 4 C'est pourquoi il lui a donné le jugement, pour juger le monde jusqu'à sa grande et glorieuse venue, et alors, ô juste Abraham, sera le jugement et la récompense parfaits, éternels et immuables, que personne ne peut modifier.

\par 5 Car chaque homme est issu des premiers créés, et c'est pourquoi ils sont ici jugés d'abord par son fils,

\par 6 et à la seconde venue, ils seront jugés par les douze tribus d'Israël, de tout souffle et de toute créature.

\par 7 Mais la troisième fois, ils seront jugés par le Seigneur, le Dieu de tous, et alors, en effet, la fin de ce jugement est proche, et le jugement est terrible, et il n'y a personne pour délivrer.

\par 8 Et maintenant, c'est par trois tribunaux que le jugement du monde et la récompense sont rendus, et c'est pourquoi une chose n'est pas définitivement confirmée par un ou deux témoins, mais par trois témoins tout sera établi.

\par 9 Les deux anges à droite et à gauche, ce sont ceux qui écrivent les péchés et la justice, celui de droite écrit la justice, et celui de gauche les péchés.

\par 10 L'ange comme le soleil, tenant la balance dans sa main, est l'archange, Dokiel le juste peseur, et il pèse les justices et les péchés avec la justice de Dieu.

\par 11 L'ange ardent et impitoyable, tenant le feu dans sa main, est l'archange Puruel, qui a pouvoir sur le feu, et qui éprouve les œuvres des hommes par le feu,

\par 12 et si le feu consume l'œuvre de quelqu'un, l'ange du jugement le saisit aussitôt et l'emmène au lieu des pécheurs, un lieu de châtiment des plus amers.

\par 13 Mais si le feu approuve l'œuvre de quelqu'un et ne s'en empare pas, cet homme est justifié, et l'ange de justice le prend et l'élève pour être sauvé dans le sort des justes.

\par 14 Et ainsi, très juste Abraham, tout chez tous les hommes est éprouvé par le feu et la balance.

\chapter{14}

\par 1 Et Abraham dit au capitaine en chef : « Mon Seigneur, le capitaine en chef, l'âme que l'ange tenait dans sa main, pourquoi a-t-il été décidé de la placer au milieu ?

\par 2 Le capitaine en chef dit : « Écoute, juste Abraham. Parce que le juge a constaté ses péchés et ses justices étant égales, il ne l'a pas livré au jugement ni au salut, jusqu'à ce que le juge de tous vienne.

\par 3 Abraham dit au capitaine en chef : « Et que manque-t-il encore pour que l'âme soit sauvée ?

\par 4 Le capitaine en chef dit : « S'il obtient une justice par rapport à ses péchés, il entre dans le salut. »

\par 5 Abraham dit au capitaine en chef : « Viens ici, capitaine en chef Michel, prions pour cette âme, et voyons si Dieu nous exaucera. Le capitaine en chef dit : Amen, qu'il en soit ainsi ;

\par 6 et ils firent des prières et des supplications pour l'âme, et Dieu les exauça, et quand ils se levèrent de leur prière, ils ne virent pas l'âme qui se tenait là.

\par 7 Et Abraham dit à l'ange : « Où est l'âme que tu tenais au milieu ?

\par 8 Et l'ange répondit : «Il a été sauvé par votre juste prière, et voici, un ange de lumière l'a pris et l'a transporté au Paradis.»

\par 9 Abraham dit : « Je glorifie le nom de Dieu, le Très-Haut, et sa miséricorde incommensurable. »

\par 10 Et Abraham dit au capitaine en chef : « Je t'en supplie, archange, écoute ma prière, et invoquons encore le Seigneur,

\par 11 et implore sa compassion, et implore sa miséricorde pour les âmes des pécheurs que jadis, dans ma colère, j'ai maudis et détruits, que la terre a dévorées, et que les bêtes sauvages ont déchirées, et que le feu a consumé à travers mes paroles. .

\par 12 Maintenant, je sais que j'ai péché devant le Seigneur notre Dieu. Viens donc, ô Michel, capitaine en chef des armées d'en haut, viens, invoquons Dieu avec des larmes afin qu'il me pardonne mes péchés et me les accorde.

\par 13 Et le capitaine en chef l'entendit, et ils supplièrent l'Éternel, et après l'avoir longuement invoqué, une voix vint du ciel disant :

\par 14 « Abraham, Abraham, j'ai écouté ta voix et ta prière, et je te pardonne ton péché, et ceux que tu penses que j'ai détruits, je les ai rappelés et je les ai ramenés à la vie par ma grande bonté, car pour un temps Je les ai punis par le jugement, et ceux que je détruis vivant sur terre, je ne les rendrai pas par la mort.

\chapter{15}

\par 1 Et la voix de l'Éternel dit aussi au capitaine en chef Michel : Michel, mon serviteur, ramène Abraham à sa maison, car voici, sa fin est proche, et la mesure de sa vie est accomplie, afin qu'il puisse mets tout en ordre, puis prends-le et amène-le-moi.

\par 2 Alors le capitaine en chef, faisant tourner le char et la nuée, amena Abraham dans sa maison,

\par 3 et entrant dans sa chambre, il s'assit sur son lit.

\par 4 Et Sarah, sa femme, vint et embrassa les pieds de l'Incorporel, et parla humblement, disant : « Je te rends grâce, mon Seigneur, de ce que tu as amené mon Seigneur Abraham, car voici, nous pensions qu'il avait été enlevé de nous. .»

\par 5 Et son fils Isaac vint aussi et se jeta à son cou, et de la même manière tous ses esclaves et esclaves entourèrent Abraham et l'embrassèrent, glorifiant Dieu.

\par 6 Et l'Incorporel leur dit : « Écoutez, juste Abraham. Voici ta femme Sarah, vois aussi ton fils bien-aimé Isaac, vois aussi tous tes serviteurs et servantes autour de toi.

\par 7 Disposez de tout ce que vous avez, car le jour est proche où vous quitterez votre corps et irez au Seigneur une fois pour toutes.

\par 8 Abraham dit : « Est-ce que le Seigneur l'a dit, ou est-ce vous qui dites cela de vous-même ?

\par 9 Le capitaine en chef répondit : « Écoute, juste Abraham. Le Seigneur l'a commandé, et je vous le dis.

\par 10 Abraham dit : « Je n'irai pas avec toi. »

\par 11 Le capitaine en chef, entendant ces paroles, sortit aussitôt de devant Abraham, monta aux cieux et se tint devant Dieu le Très-Haut.

\par 12 et il dit : « Seigneur Tout-Puissant, voici, j'ai écouté ton ami Abraham dans tout ce qu'il t'a dit et j'ai exaucé ses demandes. Je lui ai montré ta puissance, ainsi que toute la terre et la mer qui sont sous le ciel. Je lui ai montré le jugement et la récompense au moyen de nuées et de chars, et il dit encore : Je n'irai pas avec toi.

\par 13 Et le Très-Haut dit à l'ange : « Est-ce que mon ami Abraham dit encore ainsi : Je n'irai pas avec toi ? »

\par 14 L'archange dit : « Seigneur Tout-Puissant, dit-il ainsi, et je m'abstiens de lui imposer les mains, car dès le commencement il est ton ami et il a fait tout ce qui te plaît. »

\par 15 Il n'y a aucun homme comme lui sur la terre, pas même Job l'homme merveilleux, et c'est pourquoi je m'abstiens de lui imposer les mains. Ordonne donc, Roi Immortel, ce qui doit être fait.

\chapter{16}

\par 1 Alors le Très-Haut dit : « Appelez-moi ici Mort, qu'on appelle le visage impudique et le regard impitoyable. »

\par 2 Et Michel l'Incorporel alla et dit à la Mort : « Viens ici ; le Seigneur de la création, le roi immortel, vous appelle.

\par 3 Et la Mort, entendant cela, frissonna et trembla, étant possédée d'une grande terreur, et venant avec une grande peur, elle se tint devant le père invisible, frissonnant, gémissant et tremblant, attendant l'ordre du Seigneur.

\par 4 C'est pourquoi le Dieu invisible dit à la mort : « Viens ici, nom amer et féroce du monde, cache ta férocité, couvre ta corruption, et chasse de toi ton amertume, et revêts ta beauté et toute ta gloire,

\par 5 et descends vers Abraham mon ami, prends-le et amène-le-moi. Mais maintenant aussi, je vous dis de ne pas l'effrayer, mais de l'amener avec un langage honnête, car il est mon propre ami.

\par 6 Ayant entendu cela, la mort sortit de la présence du Très-Haut, et revêtit une robe d'une grande luminosité, et apparut comme le soleil, et devint belle et belle au-dessus des fils des hommes, prenant la forme de un archange, dont les joues étaient enflammées de feu, et il s'en alla vers Abraham.

\par 7 Or le juste Abraham sortit de sa chambre et s'assit sous les arbres de Mamré, tenant son menton dans sa main, et attendant l'arrivée de l'archange Michel.

\par 8 Et voici, une odeur agréable lui parvint, et un éclair de lumière, et Abraham se retourna et vit la mort venir vers lui dans une grande gloire et beauté. Et Abraham se leva et alla à sa rencontre, pensant que c'était le chef de Dieu,

\par 9 Et la mort le voyant le salua, disant : Réjouis-toi, précieux Abraham, âme juste, véritable ami du Dieu Très-Haut et compagnon des saints anges.

\par 10 Abraham dit à la mort : « Je te salue, d'apparence et de forme semblables au soleil, aide très glorieuse, porteur de lumière, homme merveilleux, d'où nous vient ta gloire, et qui es-tu, et d'où viens-tu ? »

\par 11 Alors la Mort dit : « Très juste Abraham, voici, je vous dis la vérité. Je suis le sort amer de la mort.

\par 12 Abraham lui dit : « Non, mais tu es la beauté du monde, tu es la gloire et la beauté des anges et des hommes, tu es plus beau de forme que tous les autres, et dis-tu : Je suis le sort amer de la mort, et pas plutôt, je suis plus juste que toute bonne chose.

\par 13 La mort dit : « Je vous dis la vérité. Ce que le Seigneur m’a nommé, je vous le dis aussi.

\par 14 Abraham dit : « Pourquoi es-tu venu ici ?

\par 15 La mort dit : « Je suis venu pour ton âme sainte. »

\par 16 Alors Abraham dit : « Je sais ce que tu veux dire, mais je n'irai pas avec toi ; et la Mort se tut et ne lui répondit pas un mot.

\chapter{17}

\par 1 Alors Abraham se leva et entra dans sa maison, et là aussi la mort l'accompagna. Et Abraham monta dans sa chambre, et la Mort monta avec lui. Et Abraham se coucha sur son lit, et la Mort vint s'asseoir à ses pieds.

\par 2 Alors Abraham dit : « Éloignez-vous de moi, car je désire me reposer sur mon lit. »

\par 3 La mort a dit : « Je ne partirai pas tant que je ne t'aurai pas enlevé ton esprit. »

\par 4 Abraham lui dit : Par le Dieu immortel, je t'ordonne de me dire la vérité. Êtes-vous la mort ?

\par 5 La Mort lui dit : « Je suis la Mort. Je suis le destructeur du monde.

\par 6 Abraham dit : « Je t'en supplie, puisque tu es la Mort, dis-moi si tu te présentes ainsi envers tous dans une telle équité, une telle gloire et une telle beauté ?

\par 7 La mort dit : « Non, mon Seigneur Abraham, à cause de ta justice, et de la mer sans limites de ton hospitalité, et de la grandeur de ton amour envers Dieu, est devenue une couronne sur ma tête, et dans la beauté, la grande paix et la douceur, je approchez-vous des justes,

\par 8 mais pour les pécheurs, je viens avec une grande corruption et une grande férocité et la plus grande amertume et avec un regard féroce et impitoyable.

\par 9 Abraham dit : « Je t’en supplie, écoute-moi et montre-moi ta férocité, toute ta corruption et toute ton amertume. »

\par 10 Et la Mort dit : « Tu ne peux pas voir ma férocité, très juste Abraham. »

\par 11 Abraham dit : « Oui, je pourrai contempler toute votre férocité au moyen du nom du Dieu vivant, car la puissance de mon Dieu qui est dans les cieux est avec moi. »

\par 12 Alors la mort dépouille toute sa beauté et sa beauté, et toute sa gloire et la forme semblable au soleil dont il était vêtu,

\par 13 et il revêtit une robe de tyran, et se rendit sombre et plus féroce que toutes sortes de bêtes sauvages, et plus impur que toute impureté.

\par 14 Et il montra à Abraham sept têtes de serpents ardentes et quatorze faces, (une) de feu flamboyant et d'une grande férocité, et une face de ténèbres, et une face de vipère très sombre, et une face d'une espèce très terrible précipice, et une face plus féroce qu'un aspic, et une face d'un lion terrible, et une face de cérastes et de basilic.

\par 15 Il lui montra aussi une face de cimeterre enflammé, et une face portant une épée, et une face d'éclair, éclairant terriblement, et un bruit de tonnerre épouvantable.

\par 16 Il lui montra aussi une autre face d'une mer ardente et orageuse, et d'un fleuve impétueux, et d'un terrible serpent à trois têtes, et d'une coupe mêlée de poisons,

\par 17 et en bref il lui montra une grande férocité et une amertume insupportable, et toutes les maladies mortelles comme une odeur de mort.

\par 18 Et à cause de la grande amertume et de la férocité, moururent serviteurs et servantes, au nombre d'environ sept mille,

\par 19 et le juste Abraham tomba dans l'indifférence face à la mort, de sorte que son esprit lui fit défaut.

\chapter{18}

\par 1 Et le très saint Abraham, voyant ces choses ainsi, dit à la mort : « Je te supplie, mort destructrice de tout, de cacher ta férocité et de revêtir ta beauté et la forme que tu avais auparavant. »

\par 2 Et aussitôt la Mort cacha sa férocité et revêtit sa beauté qu'il avait auparavant.

\par 3 Et Abraham dit à la Mort : « Pourquoi as-tu fait cela, que tu as tué tous mes serviteurs et mes servantes ? Est-ce que Dieu vous a envoyé ici aujourd’hui à cette fin ?

\par 4 La mort dit : « Non, mon Seigneur Abraham, ce n'est pas comme tu le dis, mais c'est à cause de toi que j'ai été envoyé ici. »

\par 5 Abraham dit à la Mort : « Comment donc sont-ils morts ? Le Seigneur ne l’a-t-il pas dit ?

\par 6 La mort dit : « Crois, très juste Abraham, que ceci aussi est merveilleux, que toi aussi tu n'aies pas été emmené avec eux. Néanmoins je vous dis la vérité,

\par 7 car si la main droite de Dieu n'avait pas été avec vous à ce moment-là, vous auriez aussi dû quitter cette vie.

\par 8 Le juste Abraham dit : « Maintenant, je sais que je suis devenu indifférent à la mort, de sorte que mon esprit défaille,

\par 9 mais je t'en supplie, mort destructrice de tout, puisque mes serviteurs sont morts avant leur temps, venons prier l'Éternel notre Dieu afin qu'il nous exauce et relève ceux qui sont morts avant leur temps à cause de ta férocité.

\par 10 Et la Mort dit : « Amen, qu'il en soit ainsi. » C'est pourquoi Abraham se leva et tomba à terre en prière, et la Mort avec lui,

\par 11 et le Seigneur envoya un esprit de vie sur ceux qui étaient morts et ils revinrent à la vie. Alors le juste Abraham rendit gloire à Dieu.

\chapter{19}

\par 1 Et montant dans sa chambre, il se coucha, et la Mort vint et se tint devant lui.

\par 2 Et Abraham lui dit : « Éloigne-toi de moi, car je désire me reposer, parce que mon esprit est dans l'indifférence. »

\par 3 La mort a dit : « Je ne te quitterai pas tant que je n'aurai pas pris ton âme. »

\par 4 Et Abraham, avec un visage austère et un regard courroucé, dit à la Mort : « Qui t'a ordonné de dire cela ?

\par 5 Vous dites ces paroles de vous-même avec vantardise, et je n'irai pas avec vous jusqu'à ce que le capitaine en chef Michael vienne à moi, et j'irai avec lui. Mais je te le dis aussi, si tu désires que je t'accompagne, explique-moi tous tes changements, les sept têtes de serpents ardentes et quelle est la face du précipice, et ce qu'est l'épée tranchante, et ce qu'est le grand rugissement rivière, et quelle mer tumultueuse qui fait rage avec tant de fureur.

\par 6 Enseigne-moi aussi le tonnerre insupportable, et les éclairs terribles, et la coupe nauséabonde mêlée de poisons. Instruisez-moi sur tout cela.

\par 7 Et la Mort répondit : « Écoute, juste Abraham. Pendant sept âges, je détruis le monde et je fais descendre tout le monde dans l'Hadès, rois et dirigeants, riches et pauvres, esclaves et hommes libres, je convoye au fond de l'Hadès, et pour cela je vous ai montré les sept têtes de serpents.

\par 8 Je vous ai montré la face de feu, parce que beaucoup meurent consumés par le feu, et voient la mort à travers une face de feu.

\par 9 Je vous ai montré la face du précipice, car beaucoup d'hommes meurent en descendant de la cime des arbres ou de terribles précipices et perdent la vie, et voient la mort sous la forme d'un terrible précipice.

\par 10 Je vous ai montré la face de l'épée, parce que beaucoup sont tués dans les guerres par l'épée, et voient la mort comme une épée.

\par 11 Je vous ai montré la face du grand fleuve impétueux, car beaucoup se noient et périssent, emportés par la traversée de nombreuses eaux et emportés par de grands fleuves, et voient la mort avant leur temps.

\par 12 Je vous ai montré le visage de la mer furieuse et déchaînée, parce que beaucoup dans la mer tombant dans de grandes vagues et faisant naufrage sont engloutis et voient la mort comme la mer.

\par 13 Le tonnerre insupportable et les éclairs terribles que je vous ai montrés parce que beaucoup d'hommes dans le moment de colère rencontrent le tonnerre insupportable et les éclairs terribles venant s'emparer des hommes, et voient ainsi la mort.

\par 14 Je vous ai aussi montré les bêtes sauvages venimeuses, les aspics et les basilics, les léopards et les lions et les lionceaux, les ours et les vipères, et enfin je vous ai montré la face de chaque bête sauvage, la plus juste, car beaucoup d'hommes sont détruits par les bêtes sauvages,

\par 15 et d'autres par les serpents venimeux, les serpents et les aspics et les cérastes et les basilics et les vipères, expirent leur vie et meurent.

\par 16 Je vous ai aussi montré les coupes destructrices mêlées de poison, car beaucoup d'hommes, à qui d'autres hommes ont donné du poison à boire, s'en vont aussitôt à l'improviste.

\chapter{20}

\par 1 Abraham dit : « Je vous en supplie, y a-t-il aussi une mort inattendue ? Dites-moi.»

\par 2 La mort dit : « En vérité, en vérité, je vous dis dans la vérité de Dieu qu'il y a soixante-douze morts. L'une est la mort juste, achetant son temps fixe, et beaucoup d'hommes en une heure entrent dans la mort et sont livrés à la tombe.

\par 3 Voici, je t'ai dit tout ce que tu as demandé, maintenant je te dis, très juste Abraham, de rejeter tout conseil et de cesser de demander quoi que ce soit une fois pour toutes, et de venir, de venir avec moi, comme Dieu et juge de tout ce qui m’a été commandé.

\par 4 Abraham dit à la mort : « Éloigne-toi encore un peu de moi, afin que je puisse me reposer sur mon lit, car j'ai le cœur très fatigué,

\par 5 Car depuis que je t'ai vu de mes yeux, ma force m'a fait défaut, tous les membres de ma chair me semblent un poids de plomb, et mon esprit est extrêmement affligé. Partez un peu ; car j’ai dit que je ne pouvais pas supporter de voir ta forme.

\par 6 Alors Isaac, son fils, vint et tomba sur sa poitrine en pleurant, et sa femme Sarah vint et embrassa ses pieds, se lamentant amèrement.

\par 7 Là aussi ses hommes et ses femmes esclaves vinrent et entourèrent son lit en se lamentant grandement. Et Abraham devint indifférent à la mort,

\par 8 et la Mort dit à Abraham : « Viens, prends ma main droite, et que la gaieté, la vie et la force te viennent. »

\par 9 Car la mort a trompé Abraham, et il a pris sa main droite, et aussitôt son âme a adhéré à la main de la mort.

\par 10 Et aussitôt l'archange Michel vint avec une multitude d'anges et prit son âme précieuse entre ses mains dans un drap de lin divinement tissé,

\par 11 et ils entretinrent le corps du juste Abraham avec des onguents et des parfums divins jusqu'au troisième jour après sa mort, et ils l'enterrèrent dans la terre promise, le chêne de Mamré,

\par 12 mais les anges reçurent son âme précieuse et montèrent au ciel, chantant l'hymne trois fois saint au Seigneur le Dieu de tous, et ils la placèrent là pour adorer le Dieu et Père.

\par 13 Et après qu'une grande louange et une grande gloire eurent été données au Seigneur, et qu'Abraham se prosterna pour l'adorer, la voix pure de Dieu et Père se fit entendre ainsi :

\par 14 Emmène donc mon ami Abraham au paradis, où sont les tabernacles de mes justes, et les demeures de mes saints Isaac et Jacob dans son sein, où il n'y a ni trouble, ni chagrin, ni soupir, mais paix et réjouissance et la vie sans fin.

\par 15 (Et nous aussi, mes frères bien-aimés, imitons l'hospitalité du patriarche Abraham et atteignons sa voie de vie vertueuse, afin que nous soyons jugés dignes de la vie éternelle, glorifiant le Père, le Fils et le Saint-Esprit. ; à qui soient la gloire et la puissance pour toujours. Amen.).

\part{Version 2}

\chapter{21}

\par 1 Il arriva, alors que les jours de la mort d'Abraham approchaient, que le Seigneur dit à Michel :

\par 2 Lève-toi et va vers Abraham, mon serviteur, et dis-lui : Tu quitteras la vie, car voici !

\par 3 Les jours de ta vie temporelle sont accomplis : afin qu'il puisse mettre de l'ordre dans sa maison avant de mourir.

\chapter{22}

\par 1 Et Michel alla et vint vers Abraham, et le trouva assis devant ses bœufs pour labourer ; il était extrêmement vieux en apparence, et avait son fils dans ses bras.

\par 2 Abraham donc, voyant l'archange Michel, se leva de terre et le salua, ne sachant pas qui il était,

\par 3 et lui dit : « Le Seigneur te préserve. Que votre voyage soit prospère avec vous.

\par 4 Et Michel lui répondit : « Tu es bon, bon père. »

\par 5 Abraham répondit et lui dit : « Viens, approche-toi de moi, frère, et assieds-toi un peu, afin que je fasse amener une bête pour que nous puissions aller chez moi, et que tu te reposes avec moi. , car c'est vers le soir,

\par 6 et le matin, lève-toi et va où tu veux, de peur qu'une mauvaise bête ne te rencontre et ne te fasse du mal.

\par 7 Et Michel s'enquit auprès d'Abraham, disant : « Dis-moi ton nom, avant que j'entre dans ta maison, de peur que je ne te sois à charge. »

\par 8 Abraham répondit et dit : « Mes parents m'ont appelé Abram, et l'Éternel m'a nommé Abraham, en disant : Lève-toi et pars de ta maison et de ta parenté, et va dans le pays que je te montrerai.

\par 9 Et quand je m'en allai dans le pays que l'Éternel m'a montré, il me dit : Ton nom ne sera plus appelé Abram, mais ton nom sera Abraham.

\par 10 Michel répondit et lui dit : « Pardonne-moi, mon père, homme de Dieu expérimenté, car je suis un étranger, et j'ai entendu dire que tu as parcouru quarante stades, que tu as amené une chèvre et que tu l'as tuée, en divertissant les anges ta maison, afin qu'ils puissent s'y reposer.

\par 11 Ainsi parlant ensemble, ils se levèrent et se dirigèrent vers la maison.

\par 12 Et Abraham appela un de ses serviteurs et lui dit : « Va, amène-moi une bête pour que l'étranger puisse s'asseoir dessus, car il est fatigué du voyage. »

\par 13 Et Michel dit : « Ne dérangez pas les jeunes, mais allons doucement jusqu'à ce que nous arrivions à la maison, car j'aime votre compagnie ».

\chapter{23}

\par 1 Et ils se levèrent et continuèrent leur route, et comme ils approchaient de la ville,

\par 2 à environ trois stades de là, ils trouvèrent un grand arbre ayant trois cents branches, semblable à un tamaris.

\par 3 Et ils entendirent une voix venant de ses branches qui chantait : Saint es-tu, parce que tu as gardé le dessein pour lequel tu as été envoyé.

\par 4 Et Abraham entendit la voix, et cacha le mystère dans son cœur, disant en lui-même : Quel est le mystère que j'ai entendu ?

\par 5 Comme il entrait dans la maison, Abraham dit à ses serviteurs : Levez-vous, sortez vers les troupeaux, et amenez trois brebis, et égorgez-les rapidement, et préparez-les afin que nous puissions manger et boire, car ce jour est un jour fête pour nous.

\par 6 Et les serviteurs amenèrent les brebis, et Abraham appela son fils Isaac, et lui dit : Mon fils Isaac, lève-toi et mets de l'eau dans le vase afin que nous puissions laver les pieds de cet étranger. Et il l'apporta comme on lui avait ordonné,

\par 7 Et Abraham dit : Je m'aperçois, et il en sera ainsi, que dans ce bassin je ne laverai plus jamais les pieds d'aucun homme venant chez nous comme hôte.

\par 8 Et Isaac, entendant son père dire cela, pleura et lui dit : Mon père, qu'est-ce que tu dis ? C'est la dernière fois que je lave les pieds d'un inconnu ? Et Abraham, voyant son fils pleurer, pleura aussi beaucoup,

\par 9 et Michel les voyant pleurer, pleura aussi, et les larmes de Michel tombèrent sur le vase et devinrent une pierre précieuse.

\chapter{24}

\par 1 Lorsque Sarah, étant à l'intérieur de sa maison, entendit leurs pleurs, elle sortit et dit à Abraham : « Seigneur, pourquoi pleures-tu ainsi ?

\par 2 Abraham répondit et lui dit : « Ce n'est pas un mal. Entrez dans votre maison et faites votre propre travail, de peur que nous ne gênions cet homme.

\par 3 Et Sarah s'en alla, étant sur le point de préparer le souper.

\par 4 Et le soleil approchait du coucher, et Michel sortit de la maison et fut élevé au ciel pour se prosterner devant Dieu,

\par 5 car au coucher du soleil tous les anges adorent Dieu et Michel lui-même est le premier des anges.

\par 6 Et ils l'adorèrent tous, et allèrent chacun chez lui,

\par 7 mais Michel parla devant le Seigneur et dit : Seigneur, ordonne-moi d'être interrogé devant ta sainte gloire !

\par 8 Et le Seigneur dit à Michel : « Annonce ce que tu veux ! »

\par 9 Et l'Archange répondit et dit : « Seigneur, tu m'as envoyé vers Abraham pour lui dire : Quitte ton corps et quitte ce monde ; le Seigneur vous appelle ;

\par 10 et je n'ose pas, Seigneur, me révéler à lui, car il est ton ami, et un homme juste et qui reçoit des étrangers.

\par 11 Mais je te prie, Seigneur, ordonne que le souvenir de la mort d'Abraham entre dans son cœur, et ne m'ordonne pas de le lui dire,

\par 12 car c'est une grande brusquerie de dire : Quitter le monde, et surtout de quitter son propre corps,

\par 13 car tu l'as créé dès le commencement pour avoir pitié de l'âme de tous les hommes.

\par 14 Alors le Seigneur dit à Michel : « Lève-toi et va vers Abraham, et passe la nuit chez lui,

\par 15 et tout ce que vous le voyez manger, mangez-le aussi, et partout où il dormira, dormez-y aussi.

\par 16 Car je jetterai en songe la pensée de la mort d'Abraham dans le cœur d'Isaac, son fils.

\chapter{25}

\par 1 Ce soir-là, Michel entra dans la maison d'Abraham et les trouva en train de préparer le souper. Ils mangèrent et burent et furent joyeux.

\par 2 Et Abraham dit à son fils Isaac : « Lève-toi, mon fils, et étends le lit de l'homme pour qu'il dorme, et pose la lampe sur le support. »

\par 3 Et Isaac fit ce que son père lui avait ordonné,

\par 4 et Isaac dit à son père : « Moi aussi, je viens dormir à côté de toi. »

\par 5 Abraham lui répondit : « Non, mon fils, de peur que nous ne gênant cet homme, mais va dans ta chambre et dors. »

\par 6 Et Isaac, ne voulant pas désobéir au commandement de son père, s'en alla et dormit dans sa propre chambre.

\chapter{26}

\par 1 Et il arriva vers la septième heure de la nuit qu'Isaac se réveilla et vint à la porte de la chambre de son père, criant et disant : « Ouvre, père, afin que je te touche avant qu'ils ne t'éloignent de moi. »

\par 2 Abraham se leva et s'ouvrit à lui, et Isaac entra et se pencha au cou de son père en pleurant, et l'embrassa avec des lamentations.

\par 3 Et Abraham pleura avec son fils, et Michel les vit pleurer et pleura aussi.

\par 4 Et Sarah, les entendant pleurer, appela de sa chambre à coucher,

\par 5 disant : « Mon Seigneur Abraham, pourquoi pleure-t-il ? L'étranger vous a-t-il dit que Lot, le fils de votre frère, était mort ? Ou est-ce qu’il nous est arrivé autre chose ?

\par 6 Michel répondit et dit à Sarah : « Non, Sarah, je n'ai apporté aucune nouvelle de Lot, mais je connaissais toute ta bonté de cœur, que en cela tu surpasses tous les hommes sur la terre, et que le Seigneur s'est souvenu de toi.

\par 7 Alors Sara dit à Abraham : Comment oses-tu pleurer, quand l'homme de Dieu est entré vers toi ?

\par 8 et pourquoi vos yeux ont-ils versé des larmes car aujourd'hui il y a une grande réjouissance ? Abraham lui dit :

\par 9 Comment savez-vous que c'est un homme de Dieu ?

\par 10 Sarah répondit et dit : Parce que je dis et déclare que c'est l'un des trois hommes que nous recevions au chêne de Mamré, lorsqu'un des serviteurs est allé amener un chevreau et que vous l'avez tué,

\par 11 et il me dit : Lève-toi, prépare-toi à manger avec ces hommes dans notre maison.

\par 12 Abraham répondit et dit : Tu as bien compris, ô femme,

\par 13 Car moi aussi, quand je lui lavais les pieds, je savais dans mon cœur que c'étaient les pieds que j'avais lavés au chêne de Mamré, et quand j'ai commencé à m'enquérir de son voyage, il m'a dit : Je vais préserver Lot ton frère des hommes de Sodome, et alors j'ai connu le mystère.


\chapter{27}

\par 1 Et Abraham dit à Michel : « Dis-moi, homme de Dieu,

\par 2 et montre-moi pourquoi tu es venu ici.

\par 3 Et Michel dit : « Ton fils Isaac te le montrera. »

\par 4 Et Abraham dit à son fils : « Mon fils bien-aimé, raconte-moi ce que tu as vu aujourd'hui dans ton rêve et ce que tu as eu peur. Racontez-le-moi.

\par 5 Isaac répondit à son père : « J'ai vu dans mon rêve le soleil et la lune, et il y avait une couronne sur ma tête,

\par 6 et vint du ciel un homme de grande taille et brillant comme la lumière qu'on appelle le père de la lumière.

\par 7 Il a ôté le soleil de ma tête, et pourtant il a laissé les rayons derrière moi.

\par 8 Et j'ai pleuré et j'ai dit : Je t'en supplie, mon Seigneur, ne m'enlève pas la gloire de ma tête, ni la lumière de ma maison, et toute ma gloire.

\par 9 Et le soleil, la lune et les étoiles se lamentaient, disant : N'ôtez pas la gloire de notre puissance.

\par 10 Et cet homme brillant répondit et me dit : Ne pleure pas parce que je prends la lumière de ta maison, car elle est passée des ennuis au repos, d'un état bas à un état élevé ;

\par 11 ils l'élèvent d'un lieu étroit à un lieu large ; ils l'élèvent des ténèbres à la lumière.

\par 12 Et je lui dis : Je t'en supplie, Seigneur, prends aussi les rayons avec.

\par 13 Il me dit : Il y a douze heures dans le jour, et alors je prendrai tous les rayons.

\par 14 Comme disait ainsi l'homme brillant, je vis le soleil de ma maison monter au ciel, mais cette couronne, je ne la vis plus,

\par 15 et ce soleil était comme toi mon père.

\par 16 Et Michel dit à Abraham : Ton fils Isaac a dit la vérité, car tu iras et tu seras élevé au ciel.

\par 17 mais ton corps restera sur terre, jusqu'à ce que sept mille siècles soient accomplis, car alors toute chair surgira.

\par 18 Maintenant donc, Abraham, mets de l'ordre dans ta maison et dans tes enfants, car tu as pleinement entendu ce qui est décrété à ton sujet.

\par 19 Abraham répondit et dit à Michel : « Je te prie, Seigneur, si je quitte mon corps, j'ai désiré être enlevé dans mon corps afin de voir les créatures que l'Éternel mon Dieu a créées dans les cieux et sur terre.

\par 20 Michel répondit et dit : Ce n'est pas à moi de le faire, mais j'irai le dire au Seigneur, et si on me l'ordonne, je vous montrerai toutes ces choses.

\chapter{28}

\par 1 Et Michel monta au ciel et parla devant l'Éternel au sujet d'Abraham,

\par 2 et le Seigneur répondit à Michel : « Va prendre Abraham dans son corps, et montre-lui toutes choses, et tout ce qu'il te dira, fais-le comme à mon ami. »

\par 3 Alors Michel sortit et prit Abraham avec son corps sur une nuée, et le conduisit au fleuve de l'Océan.

\chapter{29}

\par \textit{Version du chapitre 10 en version 1}

\par 1 Et après qu'Abraham eut vu le lieu du jugement, la nuée le descendit sur le firmament en bas,

\par 2 et Abraham, regardant la terre, vit un homme commettre adultère avec une femme mariée.

\par 3 Et Abraham se retournant dit à Michel : « Vois-tu cette méchanceté ? Mais, Seigneur, envoie du feu du ciel pour les consumer.

\par 4 Et aussitôt le feu descendit et les consuma,

\par 5 car le Seigneur avait dit à Michel : fais tout ce qu'Abraham te demandera de faire pour lui.

\par 6 Abraham regarda de nouveau, et vit d'autres hommes insulter leurs compagnons,

\par 7 et il dit : « Que la terre s'ouvre et les engloutisse »

\par 8 et pendant qu'il parlait, la terre les engloutit vivants.

\par 9 De nouveau, la nuée le conduisit vers un autre endroit, et Abraham vit que certains allaient dans un lieu désert pour commettre un meurtre,

\par 10 et il dit à Michel : « Vois-tu cette méchanceté ? Mais que les bêtes sauvages sortent du désert et les mettent en pièces. »

\par 11 et à cette même heure des bêtes sauvages sortirent du désert et les dévorèrent.

\par 12 Alors le Seigneur Dieu parla à Michel en disant : « Renvoie Abraham dans sa propre maison, et qu'il ne fasse pas le tour de toute la création que j'ai faite, car il n'a aucune compassion pour les pécheurs.

\par 13 mais j'ai compassion des pécheurs afin qu'ils se convertissent et vivent, se repentent de leurs péchés et soient sauvés.

\chapter{30}

\par \textit{Version du chapitre 11 en version 1}

\par 1 Et Abraham regarda et vit deux portes, l'une petite et l'autre grande,

\par 2 et entre les deux portes était assis un homme sur un trône d'une grande gloire, et une multitude d'anges autour de lui,

\par 3 et il pleurait et riait encore, mais ses pleurs dépassaient sept fois son rire.

\par 4 Et Abraham dit à Michel : « Qui est celui qui est assis entre les deux portes dans une grande gloire ? parfois il rit, et parfois il pleure, et ses pleurs dépassent sept fois son rire ?

\par 5 Et Michel dit à Abraham : « Ne sais-tu pas qui c'est ?

\par 6 Et il dit : « Non, Seigneur. »

\par 7 Et Michel dit à Abraham : « Vois-tu ces deux portes, la petite et la grande ?

\par 8 Ce sont eux qui mènent à la vie et à la destruction.

\par 9 Cet homme qui est assis entre eux est Adam, le premier homme que le Seigneur a créé,

\par 10 et place-le dans ce lieu pour voir chaque âme qui quitte le corps, sachant que toutes viennent de lui.

\par 11 Quand donc vous le voyez pleurer, sachez qu'il a vu beaucoup d'âmes être conduites à la perdition,

\par 12 mais quand vous le voyez rire, il a vu beaucoup d'âmes être conduites à la vie.

\par 13 Voyez-vous comme ses pleurs surpassent son rire ? Puisqu’il voit la plus grande partie du monde être emmenée par la large porte de la destruction, ses pleurs sont sept fois plus grands que son rire.

\chapter{31}

\par 1 Et Abraham dit : « Et celui qui ne peut entrer par la porte étroite, ne peut-il pas entrer dans la vie ? »

\par 2 Alors Abraham pleura, disant : « Malheur à moi, que dois-je faire ?

\par 3 Car je suis un homme au corps large, et comment pourrai-je entrer par la porte étroite, par laquelle un garçon de quinze ans ne peut entrer ?

\par 4 Michel répondit et dit à Abraham : « Ne crains pas, père, et ne t'afflige pas, car tu y entreras sans obstacle, ainsi que tous ceux qui te ressemblent. »

\par 5 Et pendant qu'Abraham se tenait debout et était étonné, voici un ange du Seigneur conduisant soixante mille âmes de pécheurs à la destruction.

\par 6 Et Abraham dit à Michel : « Est-ce que tout cela va à la destruction ?

\par 7 Et Michel lui dit : « Oui, mais allons chercher parmi ces âmes, s'il y a parmi elles même un juste. »

\par 8 Et pendant qu'ils s'en allaient, ils trouvèrent un ange tenant dans sa main l'âme d'une femme parmi ces soixante mille, parce qu'il avait trouvé que ses péchés pesaient également avec toutes ses œuvres, et qu'ils n'étaient ni en mouvement ni en repos, mais dans un état intermédiaire ;

\par 9 mais il emmena les autres âmes à la destruction.

\par 10 Abraham dit à Michel : « Seigneur, est-ce là l'ange qui enlève les âmes du corps ou non ? Michel répondit et dit : « C'est la mort, et il les conduit au lieu de jugement, afin que le juge les juge. »

\chapter{32}

\par 1 Et Abraham dit : « Mon Seigneur, je te supplie de me conduire au lieu du jugement afin que moi aussi je puisse voir comment ils sont jugés. »

\par 2 Alors Michel prit Abraham sur une nuée et le conduisit au Paradis,

\par 3 et quand il arriva au lieu où était le juge, l'ange vint et donna cette âme au juge.

\par 4 Et l'âme dit : « Seigneur, aie pitié de moi. »

\par 5 Et le juge dit : « Comment aurais-je pitié de vous, alors que vous n'avez pas eu pitié de votre fille que vous aviez, le fruit de vos entrailles ? Pourquoi l’as-tu tuée ?

\par 6 Il répondit : « Non, Seigneur, ce n'est pas moi qui ai commis le massacre, mais ma fille m'a menti. »

\par 7 Mais le juge lui ordonna de venir et de rédiger les actes,

\par 8 et voici des chérubins portant deux livres. Et il y avait avec eux un homme d'une très grande stature, ayant sur la tête trois couronnes,

\par 9 et une couronne était plus haute que les deux autres. C'est ce qu'on appelle les couronnes de témoignage.

\par 10 Et l'homme avait dans sa main une plume d'or, et le juge lui dit : Montre le péché de cette âme.

\par 11 Et cet homme, ouvrant un des livres des chérubins, chercha le péché de l'âme de la femme et le trouva.

\par 12 Et le juge dit : « Ô misérable, pourquoi dis-tu que tu n'as pas commis de meurtre ?

\par 13 N'êtes-vous pas allé, après la mort de votre mari, commettre adultère avec le mari de votre fille, et ne l'avez-vous pas tué ?

\par 14 Et il la convainquit aussi de ses autres péchés, de tout ce qu'elle avait fait dès sa jeunesse.

\par 15 En entendant ces choses, la femme s'écria, disant : « Malheur à moi, tous les péchés que j'ai commis dans le monde, j'ai oublié, mais ici ils n'ont pas été oubliés. »

\par 16 Puis ils l'emmenèrent aussi et la livrèrent aux bourreaux.

\chapter{33}

\par 1 Et Abraham dit à Michel : « Seigneur, qui est ce juge, et qui est l'autre, qui convainc des péchés ?

\par 2 Et Michel dit à Abraham : « Voyez-vous le juge ? C'est Abel, qui a témoigné le premier, et Dieu l'a amené ici pour juger,

\par 3 et celui qui rend témoignage ici est le docteur du ciel et de la terre, et le scribe de la justice, Hénoc,

\par 4 car le Seigneur les a envoyés ici pour écrire les péchés et les justices de chacun.

\par 5 Abraham dit : « Et comment Enoch peut-il supporter le poids des âmes, n'ayant pas vu la mort ? Ou comment peut-il condamner toutes les âmes ?

\par 6 Michel dit : « S'il donne une sentence concernant les âmes, cela n'est pas permis ; mais Enoch lui-même ne prononce pas de sentence,

\par 7 mais c'est le Seigneur qui le fait, et il n'a plus qu'à écrire.

\par 8 Car Hénoc priait le Seigneur en disant : Je ne veux pas, Seigneur, condamner les âmes, de peur de causer de la peine à quelqu'un ;

\par 9 et le Seigneur dit à Enoch : Je te commanderai d'écrire les péchés de l'âme qui fait l'expiation et elle entrera dans la vie,

\par 10 et si l'âme ne fait pas l'expiation et ne se repent pas, vous trouverez ses péchés écrits et elle sera jetée en punition. Et vers la neuvième heure, Michel ramena Abraham dans sa maison. Mais Sarah, sa femme, ne voyant pas ce qui était arrivé à Abraham, fut consumée par le chagrin et rendit l'esprit. Après le retour d'Abraham, il la trouva morte et l'enterra.

\chapter{34}

\par 1 Mais comme le jour de la mort d'Abraham approchait, le Seigneur Dieu dit à Michel :

\par 2 La mort n'osera pas s'approcher pour enlever l'âme de mon serviteur, parce qu'il est mon ami, mais va parer la mort d'une grande beauté, et envoie-la ainsi à Abraham, afin qu'il le voie de ses yeux.

\par 3 Et aussitôt Michel, comme on lui l'avait ordonné, orna la Mort d'une grande beauté, et l'envoya ainsi vers Abraham pour qu'il le voie.

\par 4 Et il s'assit près d'Abraham, et Abraham, voyant la mort assise près de lui, eut une grande frayeur.

\par 5 Et la Mort dit à Abraham : « Salut, âme sainte ! Salut, ami du Seigneur Dieu ! Salut, consolation et divertissement des voyageurs !

\par 6 Et Abraham dit : « De rien, serviteur du Très-Haut. Dieu. Je t'en supplie, dis-moi qui tu es ; et entre dans ma maison, mange et bois, et sors de moi, car depuis que je t'ai vu assis près de moi, mon âme est troublée.

\par 7 Car je ne suis pas du tout digne de m'approcher de toi, car tu es un esprit élevé et je suis chair et sang,

\par 8 et c'est pourquoi je ne peux pas supporter ta gloire, car je vois que ta beauté n'est pas de ce monde.

\par 9 Et la Mort dit à Abraham : Je te le dis, dans toute la création que Dieu a faite, on n'en a pas trouvé un comme toi,

\par 10 car même le Seigneur lui-même, en le cherchant, n'en a pas trouvé sur toute la terre.

\par 11 Et Abraham dit à la Mort : « Comment oses-tu mentir ? Car je vois que ta beauté n’est pas de ce monde.

\par 12 Et la Mort dit à Abraham : « Ne pense pas, Abraham, que cette beauté est à moi, ou que je viens ainsi à tout homme. Non, mais si quelqu'un est juste comme toi, je prends ainsi des couronnes et je viens à lui, mais si c'est un pécheur, je viens dans une grande corruption, et à cause de son péché, je me fais une couronne pour ma tête et je le secoue avec grande peur, au point qu’ils sont consternés.

\par 13 Abraham lui dit donc : Et d'où vient ta beauté ?

\par 14 Et la Mort dit : « Il n'y a personne d'autre plus corrompu que moi. »

\par 15 Abraham lui dit : Et es-tu bien celui qu'on appelle Mort ?

\par 16 Il lui répondit et dit : « Je suis le nom amer. Je pleure… »

\chapter{35}

\par 1 Et Abraham dit à la Mort : « Montre-nous ta corruption. »

\par 2 Et la mort a manifesté sa corruption ; et il avait deux têtes,

\par 3 l'un avait la face d'un serpent et par elle certains meurent aussitôt par des aspics,

\par 4 et l'autre tête était comme une épée ; par elle, certains meurent par l'épée comme par les arcs.

\par 5 En ce jour-là, les serviteurs d'Abraham moururent de peur de la mort, et Abraham, les voyant, pria l'Éternel, et il les ressuscita.

\par 6 Mais Dieu revint et enleva l'âme d'Abraham comme dans un rêve, et l'archange Michel l'emporta dans les cieux.

\par 7 Et Isaac enterra son père à côté de sa mère Sarah, glorifiant et louant Dieu, car à lui sont dus la gloire, l'honneur et l'adoration du Père, du Fils et du Saint-Esprit, maintenant et toujours et pour toute l'éternité. Amen.

\end{document}