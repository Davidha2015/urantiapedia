\begin{document}

\title{Testament d'Adam}

\chapter{1}

\par \textit{[Le texte éthiopien et une version arabe ont été publiés par Bezold dans le Festschrift de Nöldeke, Gieszen, 1906.]}

\par LES HEURES DE LA JOURNEE.

\par 1 Et, de plus, comprends les heures du jour et de la nuit, et comment il convient que vous suppliiez Dieu et que vous le priiez à chacune de ses saisons. Car mon Créateur m'a appris tout cela, et il m'a donné les noms de tous les animaux et bêtes sauvages, et des oiseaux du ciel, et alors Dieu m'a fait comprendre le nombre des heures du jour et de la nuit, et Il m'a raconté comment les anges louent Dieu. Comprenez donc, ô mon fils, qu'à la première heure du jour la prière de mes enfants monte vers Dieu. Et à la deuxième heure ont lieu la prière et la supplication des anges. A la troisième heure, les oiseaux du ciel le louent. Et à la quatrième heure, les êtres spirituels l'adorent. Et à la cinquième heure, toutes les bêtes sauvages et tous les animaux le saluent. A la sixième heure a lieu la requête des Kîrûbêl (Chérubins). Et à la septième heure, tous les anges entrent dans la présence de Dieu et en sortent, car à cette heure la prière de tout être vivant monte vers Dieu. À la huitième heure, les brillants habitants du ciel le louent. Et à la neuvième heure, les anges de Dieu qui se tiennent devant le trône du Très-Haut lui rendent hommage. Et à la dixième heure, le Saint-Esprit éclipse les eaux, et les démons s'enfuient et s'éloignent des eaux. Et si le Saint-Esprit n'éclipsait pas les eaux à cette heure chaque jour, personne ne pourrait boire de ces eaux, [car s'il le faisait] sa chair (c'est-à-dire son corps) serait détruite par les méchants démons. Et si le prêtre prend de l'eau à cette heure-ci, y mélange de l'huile sainte, et oint avec ce mélange les malades et les possédés d'esprits impurs, ils seront guéris de leur maladie. Et à la onzième heure ont lieu les glorifications des justes. Et à la douzième heure, Dieu, le Très-Haut, reçoit les prières et les supplications des enfants des hommes.

\chapter{2}

\par LES HEURES DE LA NUIT.

\par 1 Et dès la première heure de la nuit, les démons rendent grâces et louanges au Dieu Très-Haut, et il n'y a en eux aucun mal ni aucun mal pour personne jusqu'à ce qu'ils aient fini leur service d'hommage. Et à la deuxième heure de la nuit, les poissons et tous les animaux qui se trouvent dans les eaux louent Dieu, ainsi que les bêtes sauvages et les baleines. Et à la troisième heure, le feu le loue ; maintenant il est au plus profond, et à cette heure-là personne ne peut s'adresser à lui (?). Et à la quatrième heure les Sûrâfêl (Séraphins) Le proclament Saint. Et à la cinquième heure, les eaux qui sont au-dessus des cieux le louent. Il y a longtemps, j'étais assis et j'écoutais les anges à cette heure-là, et je m'émerveillais de la façon dont ils criaient ; [leur cri] était comme le bruit d'une grande roue, et ils criaient comme les vagues de la mer avec une voix de louange à Dieu. Et à la sixième heure, les nuages ​​louèrent Dieu avec crainte et tremblement. Et à la septième heure, la terre et toutes les créatures qui s'y trouvaient se turent, et les eaux s'endormirent. Et si à cette heure le prêtre prend de l'eau et y mélange de l'huile sainte, et qu'il en oigne les malades et ceux qui ne peuvent dormir la nuit à cause de [leur] douleur, ceux qui sont malades seront guéris et ceux qui veillent va s'endormir. A la huitième heure, la terre fait pousser de l'herbe et des herbes vertes, et les arbres produisent des feuilles et des fruits. Et à la neuvième heure, les anges accomplissent leur service d'hommage à Dieu, et la prière des enfants des hommes vient devant Dieu le Très-Haut. Et à la dixième heure, les portes du ciel s'ouvrent, et Dieu entend la prière des enfants des croyants, et la requête qu'ils demandent à Dieu leur est accordée ; Et au bruit des ailes des Séraphins, à ce moment-là, les coqs chantent et louent Dieu. Et à la onzième heure, il y a de la joie et de l'allégresse sur toute la terre, car le soleil entre dans le Jardin (c'est-à-dire le Paradis), et sa lumière se lève à toutes les extrémités du monde et éclaire toute chose créée. Et à la douzième heure, il convient que mes enfants se lèvent devant Dieu et lui rendent hommage, car à cette heure il y a un grand silence sur tous les êtres célestes.

\chapter{3}

\par ADAM PRÉDIT LA VENUE DU CHRIST.

\par 1 Maintenant donc, sache tout cela, et écoute ma parole, et comprends que la Parole de Dieu, le Très-Haut, descendra sur la terre, comme il me l'a dit au moment où il m'a chassé du monde. Jardin (Paradis). Car Il m'a dit que Sa Parole, dans les jours ultérieurs, deviendrait homme à partir d'une femme vierge nommée Marie, qu'elle se cacherait en elle, qu'elle revêtirait de chair et qu'elle naîtrait comme un homme doté d'une grande puissance, d'une grande habileté et compétences et connaissances opérationnelles. Personne ne le connaîtra sauf lui-même et celui à qui il s'est manifesté. Et Dieu dit qu'Il devait parcourir la terre avec les hommes, grandir en jours et en années, et accomplir ouvertement des signes et des prodiges, et marcher sur la mer comme sur la terre ferme, et menacer ouvertement la mer et les vents, et ils devraient lui être soumis, et qu'il crierait aux vagues de la mer et qu'ils lui répondraient rapidement. Et qu'Il fasse voir les aveugles, et que les lépreux soient purifiés, et que les sourds entendent, et que les muets parlent, et qu'Il relève les paralytiques, et fasse marcher les boiteux, et qu'Il fasse en sorte que beaucoup de gens passent de l'erreur à l'erreur la connaissance de Dieu, et devrait chasser les démons des hommes.

\par 2 Et en plus [ces choses] Dieu m'a parlé, disant : « Ne sois pas triste, ô Adam, car tu as voulu devenir un dieu et tu as transgressé mon commandement. Voici, je ne t'établirai pas maintenant, mais dans quelques jours. Et il me parla encore, disant : « Je suis Dieu qui t'ai fait sortir du jardin de joie jusqu'à la terre qui poussera des épines et des ronces, et tu y habiteras. Courbe ton dos et fais chanceler tes genoux dans la vieillesse, et je ferai de ta chair une pâture aux vers. Et après cinq jours et demi-journée, j'aurai compassion de toi, et je te ferai miséricorde dans l'abondance de ma compassion et de ma miséricorde. Et je descendrai dans ta maison, et j'habiterai dans ta chair, et à cause de toi je serai heureux de naître comme un enfant [ordinaire]. Et pour toi, j'aurai plaisir à me promener sur la place du marché. Et c'est à cause de toi que je jeûnerai quarante jours. Et pour toi, je serai heureux d'accepter le baptême. Et pour toi, je serai heureux d'endurer la souffrance. Et pour toi, il me fera plaisir de m'accrocher au bois de la Croix. Toutes ces choses [ferai-je] pour toi, ô Adam.

\par 3 A Lui soient la louange, et la majesté, et la domination, et la gloire, et l'adoration, et les hymnes, avec son Père et le Saint-Esprit, à partir de maintenant et pour toujours et à jamais. Amen.

\par 4 De plus, tu dois savoir, ô mon fils, Seth, voici qu'un déluge viendra et lavera toute la terre à cause des enfants de Kâyal (Caïn), le meurtrier, qui a tué son frère par jalousie, à cause de sa sœur. Lud. Et après le déluge et plusieurs semaines, les derniers jours viendront, et tout sera accompli, et son temps viendra et le feu consumera tout ce qui se trouve devant Dieu, et la terre sera sanctifiée, et le Seigneur des seigneurs marchera dessus.

\par 5 Et Seth écrivit ce commandement et le scella de son sceau, et du sceau de son père Adam, qu'il avait emporté avec lui du Jardin (Paradis), et du sceau d'Ève, sa mère.

\end{document}