\begin{document}

\title{3Baruch}

\chapter{1}

\par \textit{Prologue}

\par 1 Un récit et une révélation de Baruch, concernant ces choses ineffables qu'il a vues par ordre de Dieu. Bénis-toi, ô Seigneur.

\par 2 Une révélation de Baruch, qui se tenait sur la rivière Gel, pleurant sur la captivité de

\par 3 Jérusalem, quand aussi Abimélec fut préservé par la main de Dieu, à la ferme d'Agrippa. Et il était ainsi assis devant les belles portes où se trouvait le Saint des Saints.

\chapitre{1b}

\par \textit{Prologue}

\par 1 En vérité, moi, Baruch, je pleurais dans mon esprit et j'étais attristé à cause du peuple, et cela

\par 2 Dieu permit au roi Nabuchodonosor de détruire sa ville, en disant : Seigneur, pourquoi as-tu incendié ta vigne et l'as-tu dévastée ? Pourquoi as-tu fait cela et pourquoi, Seigneur, ne nous as-tu pas récompensé par un autre châtiment ? , mais tu nous as livrés à des nations comme celles-ci, afin qu'elles

\par 3 Fais-nous des reproches et dis : Où est leur Dieu. Et voici, pendant que je pleurais et disais de telles choses, j'ai vu un ange du Seigneur venir et me dire : Comprenez, ô homme bien-aimé, et ne vous inquiétez pas trop concernant le salut de Jérusalem, car ainsi parle le Seigneur Dieu :

\par 4 le Tout-Puissant. Car Il m'a envoyé devant toi pour te faire connaître et te montrer toutes (les choses)

\par 5 [...]

\par 6 de Dieu. Car ta prière a été entendue devant lui et est entrée dans les oreilles du Seigneur Dieu. Et quand il m'avait dit ces choses, je me taisais. Et l'ange m'a dit : Cesse de provoquer

\par 7 Dieu, et je te montrerai d'autres mystères, plus grands que ceux-ci. Et moi, Baruch, je dis : Aussi vrai que le Seigneur Dieu est vivant, si tu me le montres et que j'entends une de tes paroles, je ne continuerai plus à parler.

\par 8 Dieu ajoutera à mon jugement au jour du jugement, si je parle par la suite. Et l'ange des puissances me dit : Viens, et je te montrerai les mystères de Dieu.

\chapter{2}

\par \textit{Le Premier Ciel.}

\par 1 Et il m'a pris et m'a conduit là où le firmament a été fixé, et où il y avait une rivière que personne ne peut traverser, ni aucune brise étrangère de toutes celles que Dieu a créées. Et il m'a pris et m'a conduit au premier ciel, et m'a montré une porte de grande taille. Et il m'a dit : Entrons

\par 2 [...]

\par 3 nous l'avons traversé, et nous sommes entrés comme portés par des ailes, sur une distance d'environ trente jours de voyage. Et il m'a montré dans le ciel une plaine ; et il y avait des hommes qui y habitaient, avec des visages de

\par 4 des bœufs, des cornes de cerf, des pieds de boucs et des cuisses d'agneaux. Et moi, Baruch, je demandai à l'ange : Fais-moi connaître, je te prie, quelle est l'épaisseur du ciel dans lequel nous avons voyagé,

\par 5 ou quelle est son étendue, ou quelle est la plaine, afin que je puisse aussi le dire aux fils des hommes. Et l'ange dont le nom est Phamaël me dit : Cette porte que tu vois est la porte du ciel, et comme Si grande est la distance de la terre au ciel, si grande est aussi son épaisseur ; et encore une fois, aussi grande est la distance (du nord au sud, si grande) est la longueur de la plaine que tu as vue. Et encore une fois l'ange des puissances me dit : Viens, et je te montrerai de plus grands mystères. Mais

\par 6 [...]

\par 7 J'ai dit, je te prie, montre-moi quels sont ces hommes. Et il me dit : Ce sont eux qui ont bâti la tour de discorde contre Dieu, et l'Éternel les a expulsés.

\chapter{3}

\par \textit{Le Deuxième Ciel.}

\par 1 Et l'ange du Seigneur me prit et me conduisit à un second ciel. Et il m'a montré là

\par 2 aussi une porte comme la première et il dit : Entrons par là. Et nous sommes entrés, portés par des ailes

\par 3 une distance d'environ soixante jours de voyage. Et il m'a montré là aussi une plaine, et elle était pleine de

\par 4 des hommes dont l'apparence était comme celle des chiens, et dont les pieds étaient comme ceux des cerfs. Et j'ai demandé

\par 5 l'ange : Je te prie, Seigneur, dis-moi qui sont ceux-ci. Et il dit : Ce sont eux qui ont conseillé de construire la tour, car ceux que tu vois ont chassé des multitudes d'hommes et de femmes pour fabriquer des briques ; parmi lesquels, une femme fabriquant des briques ne pouvait pas être relâchée au moment de l'accouchement, mais elle était mise au monde pendant qu'elle fabriquait des briques et portait son enfant dans son tablier, et

\par 6 a continué à fabriquer des briques. Et le Seigneur leur apparut et confondit leur discours, quand ils

\par 7 avait bâti la tour à une hauteur de quatre cent soixante-trois coudées. Et ils prirent une vrille et cherchèrent à percer le ciel, en disant : Voyons (si) le ciel est fait d'argile ou de

\par 8 en laiton ou en fer. Quand Dieu a vu cela, il ne leur a pas permis, mais il les a frappés d'aveuglement et de confusion de langage, et les a rendus comme tu les vois.

\chapter{4}

\par \textit{Le Troisième Ciel.}

\par 1 Et moi, Baruch, je dis : Voici, Seigneur, tu m'as montré des choses grandes et merveilleuses ; et maintenant

\par 2 montre-moi toutes choses à cause du Seigneur. Et l'ange m'a dit : Viens, allons-y. (Et j'ai continué) avec l'ange depuis cet endroit environ cent quatre-vingt-cinq jours.

\par 3 voyage. Et il me montra une plaine et un serpent, qui paraissait avoir deux cents pléthres de longueur.

\par 4 Et il me montra Hadès, et son aspect était sombre et abominable. Et j'ai dit,

\par 5 Qui est ce dragon, et qui est ce monstre qui l'entoure Et l'ange dit : C'est le dragon

\par 6 qui mange les corps de ceux qui vivent méchamment, et il se nourrit d'eux. Et c'est Hadès, qui lui-même lui ressemble beaucoup, en ce sens qu'il boit aussi environ une coudée de

\par 7 la mer, qui ne coule pas du tout. Baruch dit : Et comment (cela arrive-t-il) Et l'ange dit : Écoute, le Seigneur Dieu a fait trois cent soixante rivières, dont le chef des

\par 8 tous sont Alphias, Abyrus et Gericus ; et c'est à cause de cela que la mer ne coule pas. Et j’ai dit : je te prie, montre-moi quel est l’arbre qui a égaré Adam. Et l'ange me dit : C'est la vigne que l'ange Sammael a plantée, contre laquelle le Seigneur Dieu s'est mis en colère, et il l'a maudit ainsi que sa plante, et c'est aussi pour cette raison qu'il n'a pas permis à Adam d'y toucher, et c'est pourquoi

\par 9 Le diable étant envieux, le séduisit par sa vigne. [Et moi, Baruch, je dis : Puisque aussi la vigne a été la cause d'un si grand mal, et qu'elle est sous le jugement de la malédiction de Dieu, et qu'elle était la

\par 10 la destruction des premiers créés, en quoi est-elle si utile maintenant Et l'ange dit : Tu demandes bien. Lorsque Dieu provoqua le déluge sur la terre et détruisit toute chair et quatre cent neuf mille géants, et que l'eau s'élevait de quinze coudées au-dessus des plus hautes montagnes, alors l'eau entra dans le paradis et détruisit toutes les fleurs ; mais il a entièrement supprimé les limites du tournage

\par 11 de la vigne et je la jetai dehors. Et quand la terre sortit de l'eau et que Noé sortit

\par 12 de l'arche, il commença à planter les plantes qu'il trouva. Mais il trouva aussi le sarment de la vigne ; et il l'a pris et a raisonné en lui-même : Qu'est-ce que c'est donc ? Et je suis venu et j'ai parlé à

\par 13 lui les choses qui le concernent. Et il dit : Dois-je le planter, ou que dois-je faire ? Puisque Adam a été détruit à cause de cela, que je ne rencontre pas non plus la colère de Dieu à cause de cela. Et en disant

\par 14 ces choses, il pria pour que Dieu lui révèle ce qu'il devait faire à ce sujet. Et après avoir accompli la prière qui dura quarante jours, après avoir prié et pleuré beaucoup de choses,

\par 15 il dit : Seigneur, je te supplie de me révéler ce que je ferai concernant cette plante. Mais Dieu envoya son ange Sarasaël et lui dit : Lève-toi, Noé, et plante le sarment de la vigne, car ainsi parle l'Éternel : Son amertume se changera en douceur, et sa malédiction deviendra une bénédiction, et ce qui est ce qui en sera produit deviendra le sang de Dieu ; et comme par elle le genre humain a obtenu la condamnation, de même encore par Jésus-Christ l'Emmanuel recevront-ils en Lui la

\par 16 l'appel vers le haut et l'entrée au paradis]. Sachez donc, ô Baruch, que, de même qu'Adam par cet arbre a obtenu la condamnation et a été dépouillé de la gloire de Dieu, de même les hommes qui boivent maintenant insatiablement le vin qui en est engendré transgressent pire qu'Adam et sont loin d'être le

\par 17 gloire de Dieu, et ils s'abandonnent au feu éternel. Car (aucun) bien n’en sort. Car ceux qui en boivent pour se rassasier font ces choses : ni un frère n'a pitié de son frère, ni un père de son fils, ni des enfants leurs parents ; mais de la consommation de vin naissent tous les maux, tels que les meurtres, les adultères, les fornications, les parjures, vols, etc. Et rien de bon n’en est établi.

\chapter{5}

\par 1 Et moi, Baruch, je dis à l'ange :

\par 2 Laisse-moi te demander une chose, Seigneur. Depuis que tu m'as dit

\par 3 que le dragon boive une coudée de la mer, dis-moi aussi quelle est la taille de son ventre. Et l'ange dit : Son ventre est Hadès ; et autant un plomb est lancé par trois cents hommes, autant son ventre est gros. Viens donc, afin que je te montre aussi des œuvres plus grandes que celles-ci.

\chapter{6}

\par 1 Et il m'a pris et m'a conduit là où le soleil se lève ;

\par 2 et il me montra un char et quatre, sous lesquels brûlait un feu, et dans le char était assis un homme, portant une couronne de feu, (et) le char (était) tiré par quarante anges. Et voici, un oiseau tournant autour du soleil, vers neuf heures

\par 3 coudées de distance. Et j'ai dit à l'ange : Quel est cet oiseau ? Et il m'a dit : C'est le

\par 4 [...]

\par 5 gardien de la terre. Et j'ai dit : Seigneur, comment est-il le gardien de la terre, apprends-moi. Et l'ange me dit : Cet oiseau vole à côté du soleil, et ses ailes déployées reçoivent son feu ardent.

\par 6 rayons. Car s'il ne les recevait pas, la race humaine ne serait pas préservée, ni aucune autre

\par 7 créature vivante. Mais Dieu a désigné cet oiseau pour cela. Et il déploya ses ailes, et je vis sur son aile droite de très grandes lettres, aussi grandes que l'espace d'une aire, de la taille d'environ quatre personnes.

\par 8 mille modii; et les lettres étaient d'or. Et l'ange m'a dit : Lis-les. Et j'ai lu

\par 9 et ils coururent ainsi : Ni la terre ni le ciel ne me font sortir, mais des ailes de feu me font sortir. Et j'ai dit : Seigneur, quel est cet oiseau, et quel est son nom ? Et l'ange m'a dit : On l'appelle son nom.

\par 10 [...]

\par 11 Phénix. (Et j'ai dit) : Et que mange-t-il ? Et il m'a dit : La manne du ciel et

\par 12 la rosée de la terre. Et je dis : L'oiseau excrète-t-il ? Et il me dit : Il excrète un ver, et les excréments du ver sont de la cannelle, dont se servent les rois et les princes. Mais attends et tu le feras

\par 13 voyez la gloire de Dieu. Et pendant qu'il parlait avec moi, il y eut comme un coup de tonnerre, et le lieu sur lequel nous nous trouvions fut ébranlé. Et j'ai demandé à l'ange, Mon Seigneur, quel est ce son Et l'ange m'a dit : Même maintenant, les anges ouvrent les trois cent soixante-cinq portes

\par 14 du ciel, et la lumière est séparée des ténèbres. Et une voix se fit entendre qui dit : Lumière

\par 15 donneur, donne au monde l'éclat. Et quand j'ai entendu le bruit de l'oiseau, j'ai dit : Seigneur, qu'est-ce que c'est ?

\par 16 bruit Et il dit : C'est l'oiseau qui réveille du sommeil les coqs sur la terre. Car, comme les hommes le font par la bouche, ainsi le coq signifie-t-il aux gens du monde, dans son propre langage. Car le soleil est préparé par les anges et le coq chante.

\chapter{7}

\par 1 Et j'ai dit : Et où le soleil commence-t-il ses travaux, après le chant du coq

\par 2 Et l'ange me dit : Écoute, Baruch : Tout ce que je t'ai montré est dans le premier et le deuxième ciel, et dans le troisième ciel le soleil passe et éclaire le monde. Mais attends, et toi

\par 3 tu verras la gloire de Dieu. Et pendant que je causais avec lui, j'ai vu l'oiseau, et il est apparu

\par 4 en avant, et grandit de moins en moins, et revint enfin à sa pleine taille. Et derrière lui, je vis le soleil brillant, et les anges qui le dessinaient, et une couronne sur son collier, dont nous étions en vue.

\par 5 incapable de regarder et de voir. Et dès que le soleil brillait, le Phénix déployait également ses ailes. Mais moi, quand j'ai vu une si grande gloire, j'ai été abattu par une grande peur, et j'ai fui et

\par 6 caché dans les ailes de l'ange. Et l'ange me dit : Ne crains pas, Baruch, mais attends et tu verras aussi leur coucher.

\chapitre{8}

\par 1 Et il me prit et me conduisit vers l'ouest ; et quand le moment du coucher arriva, je vis de nouveau l'oiseau venir devant lui, et dès que le coucher fut venu, je vis les anges, et ils soulevèrent la couronne.

\par 2 [...]

\par 3 de sa tête. Mais l’oiseau se tenait épuisé et les ailes contractées. Et voyant ces choses, je dis : Seigneur, pourquoi ont-ils soulevé la couronne de la tête du soleil, et pourquoi est-ce que

\par 4 l'oiseau si épuisé Et l'ange me dit : La couronne du soleil, quand elle a couru tout le jour - quatre anges la prennent, la portent jusqu'au ciel et la renouvellent, car elle et ses rayons ont été souillé sur terre; d'ailleurs il se renouvelle ainsi chaque jour. Et moi, Baruch, j'ai dit : Seigneur, et pourquoi

\par 5 Ses rayons sont-ils souillés sur la terre Et l'ange m'a dit : Parce qu'il voit l'iniquité et l'injustice des hommes, à savoir la fornication, les adultères, les vols, les extorsions, les idolâtries, l'ivrognerie, les meurtres, les querelles, les jalousies, les calomnies, les murmures. , chuchotements, divinations, etc., qui ne plaisent pas à Dieu. C'est à cause de ces choses qu'il est souillé et donc renouvelé.

\par 6 Mais tu demandes concernant l'oiseau, comment il est épuisé. Parce qu'en retenant les rayons du soleil à travers le feu et la chaleur brûlante de toute la journée, il s'épuise ainsi. Car, comme nous l'avons dit précédemment, si ses ailes ne masquaient les rayons du soleil, aucune créature vivante ne serait préservée.

\chapitre{9}

\par 1 Et ils s'étant retirés, la nuit tomba aussi, et en même temps arriva le char de la lune, avec les étoiles.

\par 2 Et moi, Baruch, je dis : Seigneur, montre-moi aussi, je te prie, comment

\par 3 il sort, où il part et sous quelle forme il se déplace. Et l'ange dit : Attends, et tu le verras aussi bientôt. Et le lendemain, je le vis aussi sous la forme d'une femme assise sur un char à roues. Et il y avait devant lui des bœufs et des agneaux dans le char, et une multitude de

\par 4 anges de la même manière. Et j'ai dit : Seigneur, que sont les bœufs et les agneaux ? Et il m'a dit :

\par 5 Ce sont aussi des anges. Et encore une fois, j'ai demandé : Pourquoi est-ce que cela augmente à un moment donné, mais à un autre

\par 6 le temps diminue Et (il me dit) : Écoute, ô Baruch : Ce que tu vois avait été écrit

\par 7 par Dieu beau comme aucun autre. Et lors de la transgression du premier Adam, Sammael était proche lorsqu'il prit le serpent comme vêtement. Et cela ne s'est pas caché mais a augmenté, et Dieu a été

\par 8 il s'est mis en colère contre lui, il l'a affligé et a abrégé ses jours. Et je dis : Et comment ne brille-t-il pas aussi toujours, mais seulement la nuit ? Et l'ange dit : Écoute : de même qu'en présence d'un roi les courtisans ne peuvent pas parler librement, de même la lune et les étoiles ne peuvent pas briller en présence d'un roi du soleil; car les étoiles sont toujours suspendues, mais elles sont masquées par le soleil, et la lune, bien qu'indemne, est consumée par la chaleur du soleil.

\chapitre{10}

\par \textit{Le Quatrième Ciel.}

\par 1 Et quand j'eus appris toutes ces choses de l'archange, il me prit et me conduisit dans un quatrième

\par 2 [...]

\par 3 le paradis. Et j'ai vu une plaine monotone, et au milieu une mare d'eau. Et il y avait là une multitude d’oiseaux de toutes sortes, mais pas comme ceux d’ici sur terre. Mais j'ai vu une grue aussi grande que

\par 4 grands bœufs ; et tous les oiseaux étaient plus grands que ceux du monde. Et j'ai demandé à l'ange : Qu'est-ce que

\par 5 est la plaine, et qu'est-ce que l'étang, et quelles sont les multitudes d'oiseaux autour d'elle. Et l'ange dit : Écoute, Baruch : La plaine qui contient en elle l'étang et d'autres merveilles est l'endroit où le

\par 6 Les âmes des justes viennent, lorsqu'elles conversent, vivant ensemble dans des chœurs. Mais l'eau est

\par 7 ce que reçoivent les nuages, et la pluie sur la terre, et les fruits augmentent. Et je dis encore à l'ange du Seigneur : Mais (que) sont ces oiseaux ? Et il me dit : Ce sont ceux qui

\par 8 chantez continuellement des louanges au Seigneur. Et j'ai dit : Seigneur, et comment les hommes disent-ils que l'eau qui

\par 9 qui descend sous la pluie vient de la mer. Et l'ange dit : L'eau qui descend sous la pluie, elle aussi, vient de la mer et des eaux de la terre ; mais ce qui stimule les fruits vient (uniquement) de

\par 10 cette dernière source. Sachez donc désormais que de cette source vient ce qu'on appelle la rosée du ciel.

\chapitre{11}

\par \textit{Le Cinquième Ciel.}

\par 1 Et l'ange m'a pris et m'a conduit de là vers un cinquième ciel. Et la porte était fermée. Et j'ai dit : Seigneur, cette porte n'est-elle pas ouverte pour que nous puissions entrer ? L'ange m'a dit : Nous ne pouvons pas entrer jusqu'à ce que vienne Michel, qui détient les clés du Royaume des Cieux ; mais attends et tu verras

\par 2 [...]

\par 3 la gloire de Dieu. Et il y eut un grand bruit, comme celui du tonnerre. Et j'ai dit, Seigneur, quel est ce son

\par 4 Et il me dit : Même maintenant, Michel, le commandant des anges, descend pour recevoir le

\par 5 prières des hommes. Et voici, une voix se fit entendre : Que les portes s'ouvrent. Et ils les ouvrirent, et

\par 6 il y eut un rugissement comme celui du tonnerre. Et Michael est venu, et l'ange qui était avec moi s'est présenté face à moi.

\par 7 lui fit face et dit : Salut, mon commandant, et celui de tout notre ordre. Et le commandant Michel dit : Je te salue aussi, notre frère, et l'interprète des révélations pour ceux qui passent par la vie.

\par 8 vertueusement. Et s'étant ainsi salués, ils s'arrêtèrent. Et je vis le commandant Michel, tenant un très grand vaisseau ; sa profondeur était aussi grande que la distance du ciel au

\par 9 la terre, et sa largeur était aussi grande que la distance du nord au sud. Et je dis : Seigneur, qu'est-ce que tient l'archange Michel ? Et il me dit : C'est ici qu'entrent les mérites des justes et les bonnes œuvres qu'ils font, qui sont escortées devant le Dieu céleste.

\chapitre{12}

\par 1 Et pendant que je causais avec eux, voici, des anges sont venus portant des paniers pleins de fleurs. Et

\par 2 ils les ont donnés à Michael. Et j'ai demandé à l'ange, Seigneur, qui sont ceux-ci et quelles sont les choses

\par 3 amené ici d'à côté d'eux Et il me dit : Ce sont des anges (qui) sont au-dessus des

\par 4 [...]

\par 5 juste. Et l'archange prit les paniers et les jeta dans le vase. Et l'ange

\par 6 m'a dit : Ces fleurs sont les mérites des justes. Et j'ai vu d'autres anges portant des paniers qui n'étaient (ni) vides ni pleins. Et ils commencèrent à se lamenter, et n'osèrent pas s'approcher,

\par 7 parce qu'ils n'avaient pas les prix complets. Et Michael pleura et dit : Venez ici aussi, vous

\par 8 anges, apportez ce que vous avez apporté. Et Michel fut extrêmement attristé, ainsi que l'ange qui était avec moi, de ce qu'ils n'avaient pas rempli le vase.

\chapitre{13}

\par 1 Et puis vinrent de la même manière d'autres anges pleurant et se lamentant, et disant avec crainte : Vois comme nous sommes couverts de nuages, ô Seigneur, car nous avons été livrés à des hommes méchants, et nous voulons nous en éloigner.

\par 2 eux. Et Michel dit : Vous ne pouvez vous éloigner d'eux, afin que l'ennemi ne prévienne pas.

\par 3 la fin ; mais dites-moi ce que vous demandez. Et ils dirent : Nous te prions, Michel notre commandant, éloigne-nous d'eux, car nous ne pouvons pas demeurer avec des hommes méchants et insensés, car il n'y a rien de bon.

\par 4 en eux, mais toute sorte d'injustice et d'avidité. Car nous ne les voyons pas entrer [dans l’Église, ni parmi les pères spirituels, ni] dans aucune bonne œuvre. Mais là où il y a des meurtres, il y en a aussi au milieu, et où sont les fornications, les adultères, les vols, les calomnies, les parjures, les jalousies, l'ivresse, les querelles, l'envie, les murmures, les chuchotements, l'idolâtrie, la divination et autres choses semblables,

\par 5 donc sont-ils des ouvriers de telles œuvres, et d'autres pires. C'est pourquoi nous vous supplions de nous en éloigner. Et Michel dit aux anges : Attendez que j'apprenne du Seigneur ce qui arrivera.

\chapitre{14}

\par 1 Et à cette heure même Michel partit, et les portes furent fermées. Et il y eut un bruit comme

\par 2 tonnerre. Et j'ai demandé à l'ange : Quel est le son ? Et il m'a dit : Michael présente même maintenant les mérites des hommes à Dieu.

\chapitre{15}

\par 1 Et à cette heure même Michel descendit, et la porte s'ouvrit ; et il apporta de l'huile.

\par 2 Et quant aux anges qui apportaient les paniers pleins, il les remplit d'huile, en disant : Enlevez-la, récompensez au centuple nos amis et ceux qui ont travaillé laborieusement de bonnes œuvres.

\par 3 Car celui qui a semé vertueusement, récoltera aussi vertueusement. Et il dit aussi à ceux qui apportaient les paniers à moitié vides : Venez aussi ici ; emportez la récompense selon ce que vous avez apporté, et

\par 4 livrez-le aux fils des hommes. [Puis il dit aussi à ceux qui apportaient les paniers pleins et à ceux qui apportaient les paniers à moitié vides : Allez bénir nos amis, et dites à ceux qui ainsi dit le Seigneur : Vous êtes fidèles en peu de choses, je vous mettrai en place sur beaucoup de choses; entre dans la joie de ton Seigneur.]

\chapitre{16}

\par 1 Et se retournant, il dit aussi à ceux qui n'avaient rien apporté : Ainsi dit l'Éternel : Ne vous attristez pas de

\par 2 Tenez bon, et ne pleurez pas, et ne laissez pas les fils des hommes tranquilles. Mais puisqu'ils m'ont irrité par leurs œuvres, va les rendre envieux, irrités et provoqués contre un peuple qui n'est pas un peuple, un

\par 3 personnes qui n'ont aucune compréhension. En plus de cela, envoyez la chenille et la sauterelle sans ailes, la moisissure et la sauterelle commune (et) la grêle avec des éclairs et de la colère, et

\par 4 punissez-les sévèrement par l'épée et par la mort, et leurs enfants par les démons. Car ils n'ont pas écouté ma voix, ils n'ont pas observé mes commandements et ne les ont pas observés, mais ils ont méprisé mes commandements et ont été insolents envers les prêtres qui leur annonçaient mes paroles.

\chapitre{17}

\par 1 Et pendant qu'il parlait encore, la porte fut fermée, et nous nous retirâmes.

\par 2 Et l'ange m'a pris et

\par 3 m'a ramené à l'endroit où j'étais au début. Et étant revenu à moi, j'ai rendu gloire

\par 4 à Dieu, qui m'a jugé digne d'un tel honneur. C'est pourquoi vous aussi, frères, qui avez obtenu une telle révélation, glorifiez vous-mêmes Dieu, afin qu'il vous glorifie aussi, maintenant et toujours, et pour toute l'éternité. Amen.

\end{document}