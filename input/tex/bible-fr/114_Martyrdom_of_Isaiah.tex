\begin{document}

\title{Martyre d'Isaïe}

\chapter{1}

\par 1 Et il arriva la vingt-sixième année du règne d'Ézéchias, roi de Juda, qu'il

\par 2 a appelé Manassé son fils. Maintenant, il était son seul. Et il l'appela devant Isaïe, fils d'Amots, le prophète; et en présence de Josab, fils d'Isaïe.

\par 3 [...]

\par 4 [...]

\par 5 [...]

\par 6 Et pendant qu'il (Ézéchias) donnait des ordres, Josab, fils d'Isaïe, se tenant là, Isaïe dit au roi Ézéchias, mais ce n'est pas seulement en présence de Manassé qu'il lui dit : «Aussi vrai que l'Éternel est vivant, dont le nom n'a pas été envoyé dans ce monde, [et tant que le Bien-aimé de mon Seigneur est vivant] et tant que l'Esprit qui parle en moi est vivant, tous ces commandements et ces paroles seront rendus sans effet par Manassé, ton fils, et par l'intermédiaire de ses mains, je partira au milieu de la torture de

\par 7 [...]

\par 8 mon corps. Et Sammael Malchira servira Manassé et exécutera tout son désir, et il

\par 9 devenez un disciple de Beliar plutôt que de moi. Et il fera abandonner beaucoup de gens à Jérusalem et en Judée, la vraie foi, et Beliar habitera à Manassé, et par ses mains je serai

\par 10 scié en morceaux.' Et quand Ezéchias entendit ces paroles, il pleura amèrement et déchira ses vêtements,

\par 11 et il mit de la terre sur sa tête, et il tomba sur sa face. Et Isaïe lui dit : « Le conseil de

\par 12 Sammael contre Manassé est consommé : rien ne te servira.' Et ce jour-là, Ezéchias

\par 13 résolut dans son cœur de tuer Manassé, son fils. Et Isaïe dit à Ézéchias : ['Le Bien-aimé a rendu ton dessein inutile, et] le dessein de ton cœur ne s'accomplira pas, car avec cet appel j'ai été appelé [et j'hériterai de l'héritage du Bien-aimé'] .

\chapter{2}

\par 1 Et il arriva après qu'Ézéchias mourut et que Manassé devint roi, qu'il ne se souvint pas des commandements d'Ézéchias son père mais les oublia, et Sammael demeura à Manassé.

\par 2 et s'accrocha fermement à lui. Et Manassé abandonna le service du Dieu de son père, et il servit

\par 3 Satan et ses anges et ses puissances. Et il détourna la maison de son père qui avait été

\par 4 devant la face d'Ézéchias les paroles de sagesse et du service de Dieu. Et Manassé détourna son cœur pour servir Beliar ; car l'ange de l'iniquité, qui est le chef de ce monde, est Beliar, dont le nom est Matanbuchus. Et il se plaisait à Jérusalem à cause de Manassé, et il le rendait fort dans l'apostasie (d'Israël) et dans l'iniquité qui se répandait à Jérusalem.

\par 5 Et la sorcellerie et la magie se multiplièrent, la divination, l'augulation, la fornication, [et l'adultère], et la persécution des justes par Manassé et [Belachira, et] Tobia le Cananéen et Jean.

\par 6 d'Anathoth, et par (Tsadok) le chef des ouvrages. Et le reste des actes, voici, ils sont écrits

\par 7 dans le livre des rois de Juda et d'Israël. Et quand Ésaïe, le fils d'Amoz, vit l'iniquité qui se commettait à Jérusalem, le culte de Satan et son insouciance, il

\par 8 se retira de Jérusalem et s'installa à Bethléem de Juda. Et là aussi il y avait beaucoup

\par 9 l'anarchie, et se retirant de Bethléem, il s'établit sur une montagne dans un lieu désert. [Et Michée le prophète, et le vieil Ananias, et Joël et Habacuc, et son fils Josab, et beaucoup de fidèles qui croyaient à l'ascension au ciel, se retirèrent et s'installèrent sur la montagne.]

\par 10 Ils étaient tous vêtus de vêtements en poil, et ils étaient tous prophètes. Et ils n'avaient rien avec eux, mais ils étaient nus, et ils se lamentèrent tous avec une grande lamentation à cause du voyage.

\par 11 égaré d'Israël. Et ceux-ci ne mangeaient que des herbes sauvages qu'ils cueillaient sur les montagnes, et les ayant cuisinées, ils en vivaient avec Isaïe le prophète. Et ils ont passé deux ans

\par 12 jours en montagne et collines. Et après cela, pendant qu'ils étaient dans le désert, il y avait à Samarie un certain homme nommé Belchlra, de la famille de Sédécias, fils de Chenaan, un faux prophète dont la demeure était à Bethléem. Or Ezéchias, fils de Chanani, frère de son père, et qui, du temps d'Achab, roi d'Israël, avait été le docteur des quatre cents prophètes de Baal,

\par 13 se fit frapper et réprimanda Michée, fils d'Amada, le prophète. Et lui, Michée, avait été réprimandé par Achab et jeté en prison. (Et il était) avec Sédécias le prophète : ils étaient

\par 14 avec Achazia, fils d'Achab, roi de Samarie. Et Elie, le prophète de Tebon de Galaad, réprimandait Achazia et Samarie, et prophétisait concernant Achazia qu'il mourrait sur son lit de maladie, et que Samarie serait livrée entre les mains de Leba Nasr parce qu'il avait tué

\par 15 les prophètes de Dieu. Et quand les faux prophètes, qui étaient avec Achazia, fils d'Achab et

\par 16 Leur maître Guemarias, du mont Joël, l'avait appris, et il était frère de Sédécias. Lorsqu'ils l'eurent appris, ils persuadèrent Achazia, roi d'Aguaron, et tuèrent Michée.

\chapter{3}

\par 1 Et Belchlra reconnut et vit la place d'Isaïe et des prophètes qui étaient avec lui ; car il habitait dans la région de Bethléem et était un partisan de Manassé. Et il prophétisa faussement à Jérusalem, et plusieurs habitants de Jérusalem étaient alliés avec lui, et il était Samaritain.

\par 2 Et il arriva quand Alagar Zagar, roi d'Assyrie, vint et prit Samarie et fit captives les neuf tribus (et demie) et les emmena vers les montagnes des Mèdes et du

\par 3 rivières de Tazon ; Celui-ci (Belchira), alors qu'il était encore jeune, s'était échappé et était venu à Jérusalem à l'époque d'Ézéchias, roi de Juda, mais il ne marchait pas dans les voies de son père de Samarie ; car il craignait

\par 4 Ézéchias. Et on le trouva aux jours d'Ézéchias prononçant des paroles iniques à Jérusalem.

\par 5 Et les serviteurs d'Ézéchias l'accusèrent, et il s'enfuit vers la région de Bethléem.

\par 6 Et ils ont persuadé . . . Et Belchlra accusa Isaïe et les prophètes qui étaient avec lui, en disant : « Isaïe et ceux qui sont avec lui prophétisent contre Jérusalem et contre les villes de Juda qu'elles seront dévastées et (contre les enfants de Juda et) Benjamin aussi qu'ils ira en captivité, et aussi contre toi, ô seigneur le roi, que tu iras (lié) avec des crochets

\par 7 [...]

\par 8 et des chaînes de fer' : Mais ils prophétisent faussement contre Israël et Juda. Et Isaïe lui-même a

\par 9 dit : 'Je vois plus que Moïse le prophète.' Mais Moïse a dit : « Aucun homme ne peut voir Dieu et vivre » :

\par 10 et Isaïe a dit : 'J'ai vu Dieu et voici, je vis.' Sachez donc, ô roi, qu'il ment. Et il a aussi appelé Jérusalem Sodome, et il a déclaré les princes de Juda et Jérusalem comme étant le peuple de Gomorrhe. Et il porta de nombreuses accusations contre Isaïe et le

\par 11 prophètes avant Manassé. Mais Beliar habitait au cœur de Manassé et au cœur du

\par 12 les princes de Juda et de Benjamin, les eunuques et les conseillers du roi. Et les paroles de Belchira lui plurent [extrêmement], et il envoya et saisit Isaïe.

\chapter{4}

\par \textit{Aucun contenu n'existe pour ce chapitre}

\par 1 [...]

\chapter{5}

\par 1 Et il le scia avec une scie à bois. Et quand Isaïe fut scié et divisé, Balchlra se leva pour l'accuser, et tous les faux prophètes se levèrent, riant et se réjouissant parce que

\par 2 [...]

\par 3 d’Isaïe. Et Balchlra, avec l'aide de Mécembèques, se leva devant Isaïe, [riant]

\par 4 se moquer; Et Belchlra dit à Isaïe : « Dis : « J'ai menti dans tout ce que j'ai dit, et de même

\par 5 les voies de Manassé sont bonnes et droites. Et les voies de Balchlra et de ses associés sont également

\par 6 bon. » Et il lui dit cela lorsqu'il commença à être scié en morceaux. Mais Isaïe était (absorbé)

\par 7 [...]

\par 8 dans une vision du Seigneur, et bien que ses yeux fussent ouverts, il les vit. Et Balchlra parla ainsi à Isaïe : « Dis ce que je te dis et je retournerai leur cœur, et je contraindrai Manassé

\par 9 et les princes de Juda, le peuple et tout Jérusalem pour te révérer.' Et Isaïe répondit et dit : « Pour autant que je puisse exprimer (je dis) : damné et maudit sois-tu, ainsi que toutes tes puissances et

\par 10 toute ta maison. Car tu ne peux rien prendre (de moi) sauf la peau de mon corps. Et ils

\par 11 [...]

\par 12 Il saisit et scia Isaïe, fils d'Amoz, avec une scie à bois. Et Manassé et

\par 13 Balchlra et les faux prophètes et les princes et le peuple [et] tous regardaient. Et aux prophètes qui étaient avec lui, avant qu'il ne soit scié, il dit : « Allez dans la région

\par 14 de Tyr et de Sidon ; car Dieu a mélangé la coupe pour moi seul. Et quand Isaïe fut scié en deux, il ne cria ni ne pleura, mais ses lèvres parlèrent avec le Saint-Esprit jusqu'à ce qu'il fut scié en deux.

\end{document}