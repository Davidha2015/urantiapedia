\begin{document}

\title{Psaumes de David}

\chapitre{152}

\par \textit{Parlé par David lorsqu'il combattait le lion et le loup qui prenaient une brebis de son troupeau.}

\par 1 Ô Dieu, ô Dieu, viens à mon secours ; aide-moi et sauve-moi ; délivre mon âme du meurtrier.

\par 2 Dois-je descendre au Schéol par la gueule du lion ? ou le loup me confondra-t-il ?

\par 3 Ne leur suffisait-il pas d'attendre le troupeau de mon père et de déchirer en morceaux un mouton du troupeau de mon père, mais ils voulaient aussi détruire mon âme ?

\par 4 Aie pitié, Seigneur, et sauve ton saint de la destruction ; afin qu'il puisse répéter tes gloires en tout son temps, et qu'il puisse louer ton grand nom :

\par 5 quand tu l'auras délivré des mains du lion destructeur et du loup ravisseur, et quand tu auras délivré ma captivité des mains des bêtes sauvages.

\par 6 Vite, ô mon Seigneur (Adonaï), envoie de devant toi un libérateur, et tire-moi de la fosse béante, qui m'enferme dans ses profondeurs.

\chapitre{153}

\par \textit{Parlé par David en rendant grâce à Dieu, qui l'avait délivré du lion et du loup et qui les avait tués tous deux.}

\par 1 Louez l'Éternel, vous toutes, nations ! glorifiez-le et bénissez son nom :

\par 2 Qui a délivré l'âme de ses élus des mains de la mort et délivré son saint de la destruction :

\par 3 et il m'a sauvé des filets du schéol, et mon âme de la fosse insondable.

\par 4 Parce qu'avant que ma délivrance puisse sortir devant lui, j'ai failli être déchiré en deux par deux bêtes sauvages.

\par 5 Mais il a envoyé son ange, et m'a fermé la bouche béante, et a délivré ma vie de la destruction.

\par 6 Mon âme le glorifiera et l'exaltera, à cause de toutes ses bontés qu'il m'a faites et qu'il me fera.

\chapitre{154}

\par \textit{La prière d'Ézéchias quand les ennemis l'entouraient.}

\par 1 A haute voix, glorifiez Dieu ; dans l’assemblée de plusieurs, proclamez sa gloire.

\par 2 Au milieu de la multitude des hommes droits, glorifiez sa louange ; et parle de sa gloire avec les justes.

\par 3 Unissez-vous (littéralement, votre âme) au bien et au parfait, pour glorifier le Très-Haut.

\par 4 Rassemblez-vous pour faire connaître sa force ; et ne tardez pas à manifester sa délivrance [et sa force] et sa gloire à tous les enfants.

\par 5 Afin que l'honneur du Seigneur soit connu, la sagesse a été donnée ; et pour raconter ses œuvres, cela a été fait connaître aux hommes :

\par 6 pour faire connaître aux enfants sa force, et pour faire comprendre sa gloire à ceux qui manquent de compréhension (littéralement, de cœur) ;

\par 7 qui sont loin de ses entrées et éloignés de ses portes :

\par 8 parce que le Seigneur de Jacob est exalté et que sa gloire est sur toutes ses œuvres.

\par 9 Et celui qui glorifie le Très-Haut prendra plaisir en lui ; comme celui qui offre un bon repas, et comme celui qui offre des boucs et des veaux ;

\par 10 et comme celui qui engraisse l'autel avec une multitude d'holocauste; et comme l'odeur de l'encens qui sort des mains des justes.

\par 11 De tes portes droites on entendra sa voix, et de la voix d'une droite exhortation.

\par 12 Et ce qu'ils mangeront sera rassasié en vérité, et ce qu'ils boivent, lorsqu'ils partagent ensemble.

\par 13 Leur demeure est dans la loi du Très-Haut, et leur parole est pour faire connaître sa force.

\par 14 Comme il est loin des méchants de parler de lui, et de tous les transgresseurs de le connaître !

\par 15 Voici, l'œil du Seigneur a pitié des bons, et il multipliera la miséricorde pour ceux qui le glorifient, et du temps du mal il délivrera leur âme.

\par 16 Béni soit l'Éternel, qui a délivré les malheureux de la main des méchants ; qui suscite une corne de Jacob et un juge des nations d'Israël;

\par 17 afin qu'il prolonge sa demeure à Sion et qu'il orne notre siècle à Jérusalem.

\chapitre{155}

\par \textit{Quand le Peuple obtint de Cyrus la permission de rentrer chez lui.}

\par 1 O Seigneur, j'ai crié vers toi ; écoute-moi.

\par 2 J'ai levé mes mains vers ta sainte demeure ; incline vers moi ton oreille.

\par 3 Et accorde-moi ma demande ; ma prière ne me refuse pas.

\par 4 Bâtissez mon âme, et ne la détruisez pas ; et ne le dévoile pas devant les méchants.

\par 5 Ceux qui récompensent les mauvaises choses, détourne-toi de moi, ô juge de la vérité.

\par 6 O Seigneur, ne me juge pas selon mes péchés, car aucune chair n'est innocente devant toi.

\par 7 Expose-moi, Seigneur, ta loi, et enseigne-moi tes jugements ;

\par 8 et beaucoup entendront parler de tes œuvres, et les nations loueront ton honneur.

\par 9 Souviens-toi de moi et ne m'oublie pas ; et ne me conduis pas à des choses trop difficiles pour moi.

\par 10 Les péchés de ma jeunesse te font passer loin de moi, et mon châtiment qu'ils ne se souviennent pas de moi.

\par 11 Purifie-moi, Seigneur, de la mauvaise lèpre, et qu'elle ne m'arrive plus.

\par 12 Séchez ses racines en (littéralement, de) moi, et ne laissez pas ses feuilles germer en moi.

\par 13 Tu es grand, ô Seigneur ; c'est pourquoi ma demande sera exaucée devant toi.

\par 14 A qui me plaindrais-je pour qu'il me donne ? et que peut m'ajouter la force des hommes ?

\par 15 Devant toi, Seigneur, est ma confiance ; J'ai crié vers le Seigneur et il m'a entendu et a guéri mon cœur brisé.

\par 16 J'ai dormi et j'ai dormi ; J'ai rêvé et j'ai été aidé, et le Seigneur m'a soutenu.

\par 17 Ils m'ont fait beaucoup de peine au cœur ; Je vous rendrai grâce car le Seigneur m'a délivré.

\par 18 Maintenant je me réjouirai de leur honte ; J'ai espéré en toi et je n'aurai pas honte.

\par 19 Donne-toi honneur pour toujours, même pour toujours et à jamais.

\par 20 Délivre Israël, tes élus, et ceux de la maison de Jacob, ton éprouvé.

\end{document}