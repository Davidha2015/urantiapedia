\begin{document}

\title{Apocalypse grecque d'Esdras}

\chapter{1}

\par \textit{Introduction et prière}

\par 1 La trentième année, le vingt-deuxième jour du mois, j'étais dans ma maison et je criai, disant au Très-Haut : « Seigneur, accorde-moi la gloire afin que

\par 2 [...]


\par 3 afin que je puisse voir tes mystères. À la tombée de la nuit, l’ange Michel, l’archange, est venu et m’a dit : « Prophète Esdras, mets de côté le pain pour soixante-dix semaines. » Et

\par 4 J'ai jeûné comme il me l'a dit. Et l'archistratège Raphaël est venu et m'a donné un bâton de storax, et j'ai jeûné deux fois soixante semaines, et j'ai vu les mystères de Dieu.

\par 5 [...]

\par 6 et ses anges. Et je leur ai dit : « Je souhaite plaider auprès de Dieu concernant le peuple chrétien. Il vaudrait mieux que l’homme ne naisse pas plutôt qu’il entre dans le monde. »

\par \textit{Esdras élevé au ciel : sa prière de miséricorde}

\par 7 C'est pourquoi j'ai été enlevé au ciel et j'ai vu dans le premier ciel un grand

\par 8 commandement des anges et ils m'ont conduit aux jugements. Et j'ai entendu une voix disant

\par 9 à moi : « Aie pitié de nous, Esdras élu de Dieu. » Alors j’ai commencé à dire : « Malheur aux pécheurs lorsqu’ils voient le juste (élevé) au-dessus des anges, et ils sont

\par 10 pour la Géhenne ardente. Et Esdras dit : « Aie pitié de l’œuvre de tes mains,

\par 11 quelqu'un de miséricordieux et de très compatissant. Condamnez-moi plutôt que les âmes des pécheurs, car il vaut mieux punir une seule âme et ne pas amener le monde entier à

\par 12 destruction. Et Dieu dit : « Je donnerai du repos aux justes au Paradis et

\par 13 Je suis miséricordieux. Et Esdras dit : « Seigneur, pourquoi fais-tu preuve de faveur envers les justes ?

\par 14 Car, comme un salarié termine son service et s'en va, et que de nouveau un esclave sert ses maîtres pour recevoir son salaire, ainsi le juste reçoit sa récompense dans les cieux. Mais ayez pitié des pécheurs car nous savons que

\par 15 [...]

\par 16 tu es miséricordieux. Et Dieu dit : « Je n’ai aucun moyen d’être miséricordieux envers eux. » Et

\par 17 [...]

\par 18 Esdras dit : « (Soyez miséricordieux) car ils ne peuvent pas soutenir votre colère. » Et Dieu dit : « (Je suis en colère) parce que tels (sont les mérites) de tels (hommes) comme ceux-ci. »

\par 19 Et Dieu dit : « Je souhaite vous garder comme Paul et Jean. Tu m'as donné intact le trésor inviolé, le trésor de la virginité, le mur des hommes.

\par 20 [...]

\par \textit{Deuxième prière d'Esdras}

\par 21 Et Esdras dit : « Il vaudrait mieux que l'homme ne naisse pas ; ce serait bien s'il l'était

\par 22 pas vivant. Les bêtes muettes valent mieux que l'homme, car elles n'ont pas

\par 23 punition. Vous nous avez pris et livrés au jugement. Malheur aux pécheurs dans le monde à venir, car leur condamnation est sans fin et la flamme ne s’éteint pas. »

\chapter{2}

\par \textit{Esdras fait des remontrances à Dieu : le péché d'Adam}

\par 1 Comme je lui disais cela, Michel et Gabriel et tous les apôtres vinrent et dirent :

\par 2 « Salutations ! » lEt Esdras dit : « Homme fidèle à Dieu ! Lève-toi et viens ici avec moi, Seigneur, pour le jugement. Et Dieu dit : « Voici, je vous donne mon

\par 3 [...]

\par 4 [...]

\par 5 alliance, la mienne et la vôtre, afin que vous l'acceptiez. Et Esdras dit : « Nous

\par 6 plaidera notre cause à vos oreilles. Et Dieu dit : « Demande à Abraham, ton père, quel genre de fils porte plainte contre son père et viens plaider la cause auprès de lui.

\par 7 nous. Et Esdras dit : « Tant que le Seigneur est vivant, je ne cesserai jamais de plaider ma cause.

\par 8 avec vous, à cause du peuple chrétien. Où sont tes anciennes miséricordes, ô

\par 9 Seigneur ? Où est ta patience ? Et Dieu dit : « Comme j’ai fait la nuit et le jour, j’ai fait le juste et le pécheur et il convenait de vous conduire comme le

\par 10 homme juste. Et le prophète a dit : « Qui a créé Adam, le protoplaste, le

\par 11 le premier ? Et Dieu dit : « Mes mains immaculées, et je l'ai placé au Paradis

\par 12 pour garder la région de l'arbre de vie. « Puisque celui qui a établi

\par 13 la désobéissance a fait pécher cet (homme). Et le prophète dit : « N'était-il pas gardé

\par 14 par un ange ? Et la vie n'a-t-elle pas été préservée (par) les chérubins pour les siècles sans fin ?

\par 15 Et comment a-t-il été trompé celui qui était gardé par des anges (auxquels) tu as ordonné

\par 16 être présent quoi qu'il arrive ? Faites aussi attention à ce que je dis ! Si tu avais

\par 17 ne lui avait pas donné Ève, le serpent ne l'aurait jamais trompée. Si vous sauvez qui vous voulez, vous détruirez aussi qui vous voulez.

\par \textit{Esdras fait des remontrances à Dieu : les péchés des hommes}

\par 18 Et le prophète dit : « Ô mon Seigneur, passons à un second jugement. »

\par 19 Et Dieu dit : « J'ai jeté le feu sur Sodome et Gomorrhe. » Et le prophète dit :

\par 20 [...]

\par 21 « Seigneur, tu nous fais venir ce que nous méritons. » Et Dieu dit : « Vos péchés dépassent

\par 22 ma gentillesse. Et le prophète dit : « Souviens-toi de l’Écriture, mon père, qui

\par 23 J'ai mesuré Jérusalem et je l'ai reconstruite. Pitié, Seigneur, les pécheurs, plains les tiens

\par 24 façonnage, ayez pitié de vos ouvrages. Alors Dieu se souvint de ses œuvres et

\par 25 dit au prophète : « Comment puis-je avoir pitié d'eux ? Ils donnèrent à boire du vinaigre et du fiel et [...] ils se repentirent.

\par \textit{Le jour du jugement}

\par 26 Et le prophète dit : Révélez votre chérubin et allons ensemble au jugement,

\par 27 et montre-moi quel sera le jour du jugement. Et Dieu dit : « Vous

\par 28 [...]

\par 29 J'ai fait une digression, Esdras, car tel est le jour du jugement où il n'y a pas de pluie

\par 30 sur la terre, car c'est un jugement miséricordieux en ce jour-là ». Et le prophète dit : « Je ne cesserai jamais de plaider votre cause jusqu'à ce que je voie le jour de

\par 31 [...]

\par 32 consommation. (Et Dieu dit :) « Comptez les étoiles et le sable de la mer et si vous savez compter cela, vous pourrez aussi plaider ma cause avec moi. »

\chapter{3}

\par 1 Et le prophète dit : « Seigneur, tu sais que je porte une chair humaine. Et comment peut-on

\par 2 [...]

\par 3 Je compte les étoiles du ciel et le sable de la mer ? Et Dieu dit : « Ô mon prophète élu, personne ne connaîtra ce grand jour et la manifestation qui prévaudra pour

\par 4 juger le monde. Pour toi, ô mon prophète, je t'ai dit le jour, mais l'heure

\par 5 Je ne vous l'avais pas dit. Et le prophète dit : « Seigneur, dis-moi aussi les années. » Et Dieu dit : « Si je vois que la justice du monde est devenue abondante, je serai patient envers eux. Sinon, j'étendrai la main et je saisirai le monde habité par ses quatre coins et je les rassemblerai tous dans la vallée de Josaphat et j'exterminerai le genre humain et le monde ne sera plus.

\par 6 [...]

\par 7 de plus. Et le prophète dit : « Et comment ta droite sera-t-elle glorifiée ? »

\par 8 Et Dieu dit : « Je serai glorifié par mes anges. »

\par \textit{Pourquoi l'homme a-t-il été créé ?}

\par 9 Et le prophète dit : « Seigneur, si tel était ton calcul, pourquoi as-tu formé

\par 10 mec ? Tu as dit à Abraham, notre père : « Je multiplierai ta postérité comme les étoiles du ciel et comme le sable au bord de la mer. » Et où est ta promesse ?

\par \textit{Signes de la fin}

\par 11 Et Dieu dit : « Premièrement, je ferai tomber en ébranlant les quadrupèdes et

\par 12 hommes. Et quand tu vois ce frère livre son frère à la mort et ses enfants

\par 13 se soulèvera contre les parents et une femme abandonne son propre mari, et quand une nation se soulèvera contre une nation dans une guerre, alors vous saurez que la fin est proche.

\par 14 Et alors le frère n'aura pas pitié de son frère, ni l'homme de sa femme, ni

\par 15 enfants sur parents, ni amis sur amis, ni esclave sur maître. Car l'adversaire des hommes lui-même viendra du Tartare et montrera beaucoup de choses

\par 16 aux hommes. Que dois-je te faire, Ezra, et vas-tu plaider ma cause avec moi ?

\chapter{4}

\par \textit{Esdras descend au Tartare}

\par 1 Et le prophète dit : « Seigneur, je ne cesserai jamais de discuter avec toi. »

\par 2 Et Dieu dit : « Comptez les fleurs de la terre. Si vous pouvez les compter, vous pourrez également plaider votre cause avec moi.

\par 3 [...]

\par 4 Et le prophète dit : « Seigneur, je ne peux pas les compter, je porte de la chair humaine, mais

\par 5 Je ne cesserai pas non plus de discuter cette affaire avec vous. Je souhaite, Seigneur, voir les parties inférieures

\par 6 du Tartare. Et Dieu dit : « Descendez et voyez ! » Et il m'a donné Michael

\par 7 [...]

\par 8 et Gabriel et trente-quatre autres anges, et moi, je descendis quatre-vingt-cinq marches et ils me firent descendre cinq cents marches.

\par \textit{Le châtiment d'Hérode}

\par 9 Et je vis un trône de feu et un vieil homme assis dessus, et son châtiment fut

\par 10 impitoyable. Et je dis aux anges : « Qui est-ce et quel est son péché ? » Et ils m'ont dit :

\par 11 « Voici Hérode, qui fut roi pendant un temps, et il ordonna de tuer

\par 12 les nourrissons âgés de deux ans ou moins. Et j’ai dit : « Malheur à son âme ! » Le désobéissant et l'abîme

\par 13 Et encore une fois ils m'ont fait descendre trente marches. Et j'y ai vu des feux bouillants, et un

\par 14 une multitude de pécheurs en eux. Et j'ai entendu leurs voix, mais je n'ai pas perçu leurs

\par 15 formulaires. Et ils m'ont fait descendre plus profondément de nombreuses marches que je ne pouvais pas compter.

\par 16, Et je vis là des vieillards, et des axes de feu tournaient sur leurs oreilles.

\par 17 [...]

\par 18 J'ai dit : « Qui sont-ils et quel est leur péché ? Et ils m'ont dit : « Ce sont

\par 19 les oreilles indiscrètes. Et encore une fois, ils m'ont fait descendre cinq cents autres marches. Et

\par 20 [...]

\par 21 Là, j'ai vu le ver qui ne dormait pas et le feu qui dévorait les pécheurs. Et ils m'ont conduit jusqu'à la fondation d'Apoleia (Destruction) et là j'ai vu les douze

\par 22 coup d'abîme. Et ils m'ont emmené vers le sud et là j'ai vu un homme

\par 23 pendait à ses paupières et les anges le frappaient. Et j'ai demandé : « Qui

\par 24 est-ce cela et quel est son péché ? Et Michel, l'archistratège, me dit : « Cet homme est incestueux ; après avoir accompli un petit désir, cet homme reçut l'ordre d'être pendu.

\par \textit{L'Antéchrist}

\par 25 Et ils m'ont emmené vers le nord et j'y ai vu un homme retenu avec du fer

\par 26 mesures. Et j'ai demandé : « Qui est-ce ? » Et il m'a dit : « C'est celui qui dit : 'Je suis le fils de Dieu et celui qui a fait les pierres, le pain et l'eau, le vin.' »

\par 27 [...]

\par 28 Et le prophète dit : « Faites-moi connaître quelle sorte d'apparence il a et

\par 29 J'informerai la race des hommes de peur qu'ils ne croient en lui. Et il me dit : « Son visage est celui d’un homme sauvage. Son œil droit est comme une étoile qui se lève

\par 30 l'aube et l'autre est immobile. Sa bouche est d'une coudée, ses dents sont d'un empan

\par 31 long, ses doigts comme des faux, la plante de ses pieds à deux envergures, et sur son front

\par 32 une inscription 'Antéchrist.' Il a été exalté jusqu'au ciel, il descendra aussi loin

\par 33 comme Hadès. Une fois, il sera un enfant, une autre, un vieil homme. Et le prophète

\par 34 [...]

\par 35 dit : « Seigneur, comment permets-tu à la race des hommes de s'égarer ? Et Dieu dit : « Écoute, mon prophète ! Il devient un enfant et un vieil homme et que personne ne le croie

\par 36 c'est mon fils bien-aimé. Et après ces choses, la trompette sonnera, et les tombeaux seront

\par 37 s'ouvrira et les morts ressusciteront sans corruption. Alors l'adversaire, ayant entendu

\par 38 la terrible menace, se cachera dans les ténèbres extérieures. Puis le ciel et

\par 39 la terre et la mer périront. Alors je brûlerai le ciel pendant quatre-vingts coudées

\par 40 et la terre sur huit cents coudées. Et le prophète dit : « Et (dans) quoi

\par 41 le ciel a-t-il péché ? Et Dieu dit : « Puisque [...] c'est le mal. » Et le prophète

\par 42 [...]

\par 43 dit : « Seigneur, (en) qu'est-ce que la terre a péché ? Et Dieu dit : « Puisque l’adversaire ayant entendu ma terrible menace se cachera (dedans), et à cause de cela je ferai fondre la terre et avec elle le rebelle du genre humain. »

\chapter{5}

\par \textit{Autres sanctions}

\par 1 Et le prophète dit : « Pitié, Seigneur, race des chrétiens. » Et j'ai vu un

\par 2 [...]

\par 3 femme suspendue et quatre bêtes sauvages lui tétaient les seins. Et les anges m'ont dit : « Elle a refusé de lui donner du lait, mais elle a aussi jeté les enfants dans le

\par 4 rivières. Et j'ai vu des ténèbres terribles et une nuit sans étoiles ni lune. Il n'y a ni jeune ni vieux, ni frère avec frère, ni mère avec enfant, ni

\par 5 [...]

\par 6 femme et mari. Et j’ai pleuré et j’ai dit : « Ô Seigneur, Seigneur, aie pitié des pécheurs. »

\par \textit{Ezra emmené au paradis}

\par 7 Et pendant que je disais ces choses, une nuée vint et me saisit et me reprit

\par 8 vers les cieux. Et j'ai vu beaucoup de jugements et j'ai pleuré amèrement et j'ai dit : « Cela

\par 9 [...]

\par 10 serait meilleur si l'homme ne sortait pas du ventre de sa mère. Ceux qui étaient en punition criaient en disant : « Depuis que tu es venu ici, saint de Dieu, nous

\par 11 ont obtenu un léger répit. Et le prophète a dit : « Bienheureux ceux qui déplorent leurs propres péchés. »

\par \textit{La naissance et son but}

\par 12 Et Dieu dit : « Écoute Esdras, bien-aimé ! Tout comme un agriculteur jette la graine

\par 13 du blé en terre, ainsi l'homme jette sa semence à la place de la femme. Le premier (mois) c'est un tout, le deuxième il est enflé, le troisième il pousse des cheveux, le quatrième il pousse des ongles, le cinquième il devient laiteux, le sixième il est prêt et vivifié, le le septième, il est préparé, [au huitième...], le neuvième les barreaux des portes de la femme sont ouverts et elle naît saine sur la terre.

\par 14 Et le prophète dit : « Il vaudrait mieux que l'homme ne naisse pas. Hélas, ô

\par 15 [...]

\par 16 race humaine, à l'heure où tu viens au jugement ! Et j'ai dit au Seigneur,

\par 17 « Seigneur, pourquoi as-tu créé l'homme et l'as-tu livré au jugement ? » Et Dieu dit dans sa déclaration exaltée : « Je ne pardonnerai pas à ceux qui transgressent mes

\par 18 alliance. Et le prophète dit : « Seigneur, où est ta bonté ? Et Dieu dit : J'ai tout préparé à cause de l'homme et l'homme ne garde pas mes commandements.

\par 19 [...]

\par \textit{Punitions et récompenses}

\par 20 Et le prophète dit : « Seigneur, révèle-moi les châtiments et le Paradis. »

\par 21 Et les anges m'ont emmené vers l'est et j'ai vu l'arbre de vie. Et j'ai vu là

\par 22 Hénoc et Élie et Moïse et Pierre et Paul et Luc et Matthieu et tous

\par 23 les justes et les patriarches.g Et j'ai vu là le châtiment de l'air et le souffle des vents et les entrepôts de glace et l'éternel.

\par 24 punitions. Et j'y ai vu un homme pendu par le crâne. Et ils m'ont dit :

\par 25 [...]

\par 26 « Celui-ci a transféré les frontières. » Et là, j'ai vu de grands jugements et j'ai dit à

\par 27 le Seigneur : « Seigneur, Seigneur, lequel des hommes, étant né, n'a pas péché ? Et ils m'ont conduit plus loin dans le Tartare et j'ai vu tous les pécheurs se lamenter et

\par 28 des pleurs et un deuil mauvais. Et moi aussi, j'ai pleuré en voyant la race humaine ainsi punie.

\chapter{6}

\par 1 Alors Dieu me dit : « Esdras, connais-tu les noms des anges qui sont

\par 2 sur la consommation : Michel, Gabriel, Uriel, Raphaël, Gabuthelon, Aker, Arphugitonos, Beburos, Zebuleon ?

\par \textit{Ezra lutte pour son âme}

\par 3 Alors une voix me parvint : « Viens ici, meurs, Esdras, mon bien-aimé ! Rends ça

\par 4 qui vous a été confié. Et le prophète dit : « Et d’où peut-on

\par 5 tu fais naître mon âme ? Et les anges dirent : « Nous pouvons le jeter à travers

\par 6 ta bouche. Et le prophète a dit : « J’ai parlé bouche à bouche avec Dieu et cela

\par 7 ne sortira pas de là. Et les anges ont dit. «Nous le ferons sortir à travers

\par 8 vos narines. Et le prophète dit : « Mes narines sentaient la gloire de Dieu. »

\par 9 Et les anges dirent : « Nous pouvons le faire naître à travers vos yeux. » Et le prophète

\par 10 [...]

\par 11 dit : « Mes yeux ont vu le dos de Dieu. » Et les anges dirent : « Nous pouvons apporter

\par 12 cela passe par ta tête. Et les anges dirent : « Nous pouvons le faire sortir par vos pieds. » Et le prophète dit : « J’ai marché avec Moïse sur la montagne,

\par 13 et il ne sortira pas de là. Et les anges dirent : « Nous pouvons le jeter

\par 14 par le bout de vos ongles (d'orteils). Et le prophète a dit : « Mes pieds sont entrés dans le sanctuaire. » Et les anges s'en allèrent sans succès, en disant : « Seigneur, nous

\par 15 [...]

\par 16 ne peut pas recevoir son âme. Puis il dit à son fils unique : « Descends, mon fils bien-aimé, avec une multitude d'anges, et prends l'âme de mon Esdras bien-aimé. »

\par 17 Car le Seigneur, ayant pris une armée nombreuse de plusieurs anges, dit au prophète : « Donne-moi le dépôt que je t'ai confié. La couronne est prête pour vous.

\par 18 Et le prophète dit : « Seigneur, si tu me prends mon âme, qui auras-tu ?

\par 19 reste-t-il pour plaider en faveur de la race des hommes ? Et Dieu dit : « Toi qui es mortel

\par 20 et terrestre, ne plaidez pas ma cause. Et le prophète a dit : « Je ne pourrai jamais

\par 21 cessez de plaider. Et Dieu dit : « Donnez, en attendant, ce qui vous a été confié.

\par 22 (à vous). La couronne est prête pour vous. Viens ici, meurs, afin que tu puisses atteindre

\par 23 cela. Alors le prophète commença à parler en pleurant : « Ô Seigneur, à quoi sert-il que je

\par 24 plaider la cause avec toi, et je vais tomber à terre ? Malheur, malheur ! car je le ferai

\par 25 être consommé par les vers. Déplorez-moi, vous tous, saints et pieux, je vous en supplie grandement et

\par 26 suis livré à la mort ! Déplorez-moi, vous tous saints et justes, parce que je suis entré dans le bol de l'Hadès.

\chapter{7}

\par \textit{Âme et corps}

\par 1 Et Dieu lui dit : « Écoute, Esdras. ma bien-aimée. Moi, étant immortel, j'ai reçu

\par 2 une croix, j'ai goûté du vinaigre et du fiel, j'ai été déposé dans un tombeau. Et j'ai suscité mes élus et j'ai invoqué Adam de l'Hadès pour que la race des hommes

\par 3 [...]. Ne craignez donc pas la mort. Car ce qui vient de moi, c'est-à-dire l'âme, s'en va vers le ciel. Ce qui vient de la terre, c'est-à-dire le corps, s'en va vers le

\par 4 la terre d'où il a été tiré. Et le prophète dit : « Malheur, malheur ! Que dois-je faire ? Comment dois-je agir ? Je ne sais pas.»

\par \textit{Prière de clôtureb}

\par 5 Et alors le bienheureux Esdras commença à dire : « Ô Dieu éternel, Créateur de toute la création, qui a mesuré le ciel avec une envergure et a contenu la terre dans son

\par 6 main, qui chasse les chérubins, qui a emmené le prophète Élie au ciel dans

\par 7 un char de feu, qui nourrit toute chair, que toutes choses craignent et tremblent

\par 8 Devant ta puissance, écoute moi qui supplie grandement et donne à tous ceux qui

\par 9 copiez ce livre et conservez-le et rappelez-vous mon nom et conservez pleinement ma mémoire,

\par 10 donne-leur la bénédiction du ciel. Et bénis toutes ses choses, tout comme les fins de

\par 11 Joseph.d Et ne te souviens pas de ses péchés antérieurs le jour de son jugement. Ceux

\par 12 [...]

\par 13 Ceux qui ne croiront pas à ce livre seront brûlés comme Sodome et Gomorrhe. Et une voix lui fit entendre : « Esdras. ma bien-aimée, j’accorderai à chacun ce que tu as demandé.

\par \textit{Mort et enterrement d'Esdras}

\par 14 Et aussitôt il donna sa précieuse âme avec beaucoup d'honneur le dix-huitième

\par 15 du mois d'octobre. Et ils l'enterrèrent avec de l'encens et des psaumes. Son corps précieux et saint renforce sans cesse les âmes et les corps de ceux qui s’approchent de lui de bon gré.

\par 16 Gloire, puissance, honneur et adoration à celui à qui cela convient, au Père et au Fils et au Saint-Esprit, maintenant et toujours et pour toujours et à jamais. Amen.

\end{document}