\begin{document}


\title{Testament d'Isaac}

\chapter{1}

\par 1 Au nom du Père, du Fils et du Saint-Esprit, le Dieu Unique.

\par 2 Nous commençons avec l'aide de Dieu et par sa médiation de célébrer la mort du patriarche Isaac, fils du patriarche Abraham, et son ascension de son

\par 3 corps en ce même jour, qui est le vingt-huitième du mois de Misri. Que la bénédiction de son intercession soit avec nous et nous protège des tentations de l’ennemi ! Amen!

\par 4 [Le patriarche Isaac rédigea son testament et adressa ses paroles d'instruction à Jacob son fils et à tous ceux qui étaient assemblés avec lui. ]

\par 5 Il dit : « Écoutez, mes frères et mes bien-aimés, l'instruction de cet orateur et cette médecine curative.

\par 6 Parce que le chemin de Dieu continue éternellement, écoutez non seulement avec des oreilles corporelles chastes, mais aussi avec la profondeur du cœur et avec une foi vraie sans aucun doute, comme il est écrit : Voici, vous avez entendu une parole ferme. quant à ce qu'un homme devrait devenir. S’il l’a entendu avec un cœur pur, Dieu lui accordera de la compassion lorsqu’il lui demandera quelque chose.’”

\par 7 « Et il est aussi écrit : 'Il n'y a aucun profit à quelqu'un qui demande à Dieu ce que les êtres humains sollicitent sur terre.' Et si Dieu nous a donné la maîtrise sur la terre, alors quel avantage a celui qui a été ferme dans la foi en la parole de Dieu et qui s'est tenu fermement sans doute et d'un cœur droit à la connaissance des commandements de Dieu. Dieu et les histoires de ses saints ; car il sera l’héritier du royaume de Dieu.

\par 8 « Car voici, Dieu est compatissant et miséricordieux, celui qui a reçu auprès de lui les voleurs et les publicains dans le passé à cause de la sincérité de leur foi qui vient de Dieu. Et Dieu, en outre, est avec les siècles à venir.

\chapter{2}

\par 1 Il arriva que, lorsque le temps approchait pour notre père Isaac, le père des pères, de quitter ce monde et de sortir de son corps, le Compatissant, le Miséricordieux lui envoya le chef des anges, Michel, celui qu'il avait envoyé à son père Abraham, le matin du vingt-huitième jour du mois de Misri.

\par 2 L'ange lui dit : « La paix soit sur toi, ô fils choisi, notre père Isaac !

\par 3 Or, il était d'usage que les saints anges lui parlent chaque jour. Alors il se prosterna et vit que l'ange ressemblait à son père Abraham.

\par 4 Alors il ouvrit la bouche, cria d'une voix forte et dit avec joie et exultation : « Voici, j'ai vu ton visage comme si j'avais vu le visage du Créateur miséricordieux. »

\par 5 Alors l'ange lui dit : « Ô mon bien-aimé Isaac, je t'ai été envoyé de la présence du Dieu vivant pour t'emmener au ciel pour être avec ton père Abraham et tous les saints.

\par 6 Car ton père Abraham t'attend ; lui-même est sur le point de venir vous chercher, mais maintenant il se repose.

\par 7 Un trône t'a été préparé à côté de ton père Abraham ; de même pour ton fils bien-aimé Jacob.

\par 8 Et vous serez tous au-dessus de tous dans le royaume des cieux dans la gloire du Père et du Fils et du Saint-Esprit.

\par 9 Ce nom vous sera confié pour toutes les générations futures : Les Patriarches. Ainsi vous serez les pères du monde entier, ô ancien fidèle, notre père Isaac.

\par 10 Isaac répondit, disant à l'ange : « Je suis vraiment étonné à ton sujet. N'es-tu pas mon père Abraham ?

\par 11 Alors l'ange lui dit : « Je ne suis pas ton père Abraham, mais je suis celui qui sert ton père Abraham.

\par 12 Alors maintenant, réjouissez-vous et soyez dans l'allégresse ; car tu ne seras pas frappé (de maladie) ? et ne sera pas pris (dans la mort) avec douleur mais avec joie.

\par 13 Vous atteindrez les bénédictions et le repos pour toujours et sortirez de l'emprisonnement vers l'espace.

\par 14 Aussi vous partirez vers une réjouissance qui n'a pas de fin, et vers la lumière et la félicité qui n'ont pas de limite, et pour acclamer et ravir sans cesse.

\par 15 « Et maintenant, faites votre testament et mettez de l'ordre dans votre maison ; car tu es sur le point de t'en aller au (dernier) repos.

\par 16 Néanmoins, la bénédiction sera sur le père qui t'a engendré et sur ta postérité qui viendra après toi !

\par 17 Or, lorsque notre père Jacob les entendit parler ainsi entre eux, il se mit à les écouter, mais il ne parla pas.

\par 18 Alors notre père Isaac dit à l'ange avec patience et humilité : « Que dois-je faire maintenant concernant la lumière de mes yeux, mon Jacob bien-aimé ?

\par 19 Je crains pour lui à cause d'Ésaü. Bien sûr, vous connaissez toute l’histoire.

\par 20 Alors l'ange lui dit : Mon bien-aimé Isaac, tous les peuples du monde, s'ils étaient rassemblés en un seul lieu, ne pourraient pas annuler ta bénédiction sur Jacob ; parce que, au moment où tu l'as béni, il a été béni par le Dieu suprême, aussi par le Fils et le Saint-Esprit, et par ton père Abraham ; tous répondirent en disant : « Amen ».

\par 21 Le fer (l'épée ?) ne l'effrayera pas, mais il sera extrêmement fort et obtiendra la souveraineté.

\par 22 Alors il deviendra le père de nombreuses nations, et douze tribus sortiront de lui.

\par 23 Alors Isaac dit à l'ange : « Tu m'as informé et tu m'as apporté une bonne nouvelle.

\par 24 Mais que Jacob n'écoute pas, car il sera triste et troublé ; car je ne lui ai jamais fait souffrir le cœur.

\par 25 Alors l'ange du Seigneur dit : « Ô mon bien-aimé Isaac, tous les justes qui sortent de leur corps sont bénis, et ils sont heureux quand ils voient Dieu, le Miséricordieux, le Compatissant.

\par 26 Mais malheur, malheur trois fois au pécheur lorsqu'il est né sur la terre, car il a beaucoup de souffrances.

\par 27 Tu enseigneras à tes fils tes voies et les commandements de ton père, tous ceux qu'il t'a prescrits.

\par 28 Et ne cachez pas ces choses à Jacob, afin qu'elles soient un rappel aux générations de sa postérité après lui, afin que les fidèles les observent et atteignent par elles la vie éternelle, qui est éternelle.

\par 29 Mais je tiendrai compte de votre préoccupation.

\par 30 Voici, je suis venu vers vous avec joie et rapidement. La paix que le Seigneur a donnée, je vous la donne.

\par 31 Et maintenant, je vais vite vers Celui qui m'a envoyé.

\chapter{3}

\par 1 Quand l'ange eut dit cela, il se leva du lit de notre père Isaac et s'éloigna de lui.

\par 2 Isaac le regardait et était étonné de ce qu'il avait entendu et vu.

\par 3 Alors il entreprit de dire : « Je ne verrai pas la lumière jusqu'à ce que vous m'envoyiez chercher. »

\par 4 Pendant qu'il méditait cela, Jacob s'était avancé à la porte de la chambre de son père.

\par 5 L'ange l'avait déjà endormi pour l'empêcher de les entendre.

\par 6 Alors, lorsqu'il entra dans le lieu de repos de son père, il dit : « Père, avec qui parlais-tu ?

\par 7 Isaac, son père, lui dit : « Maintenant, tu dois m'écouter, mon fils. Une nouvelle a été envoyée à ton vénérable père qu'il te sera enlevé, ô mon fils Jacob.

\par 8 Alors Jacob embrassa son père et pleura, disant : Ma force m'a quitté ; Veux-tu me rendre orphelin, ô mon père, afin que je devienne aujourd'hui malheureux ?

\par 9 Il embrassa encore notre père Isaac et le baisa ; tous deux pleurèrent jusqu'à ce qu'ils soient épuisés et las.

\par 10 Alors Jacob dit : « Ô père, je m'en irai avec toi et je ne t'abandonnerai pas. »

\par 11 Mais Isaac lui dit : Mon garçon, ce n'est pas à moi de faire cela, ô mon enfant et mon Jacob bien-aimé ; mais je remercie Dieu de ce que toi aussi tu es devenu père et que tu le resteras jusqu'à ce que tu sois appelé. . .?

\par 12 Comme mon père Abraham me l'a informé, je ne peux annuler aucune partie du décret, qui est valable pour tout le monde ; ainsi cela arrivera, car ce qui est écrit ne sera pas frustré.

\par 13 Mais Dieu sait, mon fils, que mon cœur est fatigué à cause de toi. Pourtant, je suis heureux de aller vers le Seigneur.

\par 14 Ainsi donc, maintenant que vous avez connu la croissance dans l'Esprit, éloignez de vous ces pleurs et ces lamentations.

\par 15 Écoute, mon garçon, afin que je puisse te parler et te faire comprendre le premier homme, je veux dire notre père Adam, le créé, que Dieu a formé de sa propre main ; de même notre mère Eve ; aussi Abel et Seth et notre père Enoch (Enosh ?) et Mahalalel, le père de Methusaleh, et Lamec, le père de Jared, et Enosh (Enoch ?), le père de notre père Noé et ses fils, Sem, Cham et Japhet ; et après eux Phinées et Kenan et Noé (?) et Eber et Reu et Terah et Nahor et mon père Abraham et Lot le fils de son frère.

\par 16 Tous ces morts ont emporté sauf notre père Enoch, le parfait qui est monté au ciel.

\par 17 « Et après cela sortiront douze géants.

\par 18 Alors viendra Jésus le Messie, de votre descendance, d'une vierge nommée Marie.

\par 19 Et Dieu s'incarnera en lui jusqu'à l'accomplissement de cent ans.

\chapter{4}

\par 1 Or Isaac jeûnait tous les jours et ne rompait son jeûne que le soir.

\par 2 Il offrirait des sacrifices pour lui et pour tous les gens de sa maison, pour le salut de leurs âmes.

\par 3 Il se levait pour prier au milieu de la nuit, et pendant la journée il priait Dieu. Il a continué à faire cela pendant de nombreuses années.

\par 4 Il jeûnait également les trois périodes de quarante jours, chaque fois que la période de quarante jours arrivait.'

\par 5 Et il ne mangerait pas de viande ni ne boirait de vin toute sa vie.

\par 6 Il ne goûtait pas non plus le goût des fruits et ne dormait pas sur un lit, car il se consacrait à la prière chaque jour et à la supplication à Dieu toute sa vie.

\par 7 Ainsi, lorsque les foules apprirent qu'un homme de Dieu était apparu, elles affluèrent vers lui de tous les districts et de tous les lieux pour entendre ses instructions et ses recommandations vivifiantes et pour être assurées que l'esprit de Dieu parlait en lui.

\par 8 Alors les grands qui s'étaient rassemblés vers lui dirent : « Quelle est cette puissance qui est descendue sur toi après le temps où l'éclat de ta vue t'a quitté, et comment as-tu eu un sursis pour voir maintenant ?

\par 9 Alors le vieillard fidèle sourit et leur dit : « Quant à ceux qui se sont présentés, je leur dirai que Dieu m'a guéri lorsqu'il a vu que je m'approchais de la porte de la mort.

\par 10 Il m'a accordé cet honneur dans ma vieillesse afin que je sois prêtre du Seigneur.

\par 11 Alors quelqu'un (Jacob ?) lui dit : « Commence pour moi un discours afin que j'en sois consolé et que je m'y tienne fermement. »

\par 12 Alors notre père Isaac lui dit : « Si tu parles avec colère, garde-toi des calomnies et garde-toi des vaines vantardises.

\par 13 Veillez à ne pas converser seul (avec une femme).

\par 14 Prenez garde qu'une mauvaise parole ne sorte de votre bouche.

\par 15 Gardez votre corps, afin qu'il soit pur, car c'est le temple du Saint-Esprit qui habite en lui.

\par 16 Prenez soin des moindres fonctions de votre corps, afin qu'il soit pur et sanctifié.

\par 17 Gardez-vous de vous amuser avec votre langue, de peur qu'une mauvaise parole ne sorte de votre bouche.

\par 18 « Gardez-vous de tendre la main vers ce que vous ne possédez pas.

\par 19 Ne présentez pas d'offrande lorsque vous n'êtes pas rituellement pur ; baignez-vous dans l'eau lorsque vous comptez vous approcher de l'autel.

\par 20 Ne mélangez pas vos pensées avec les pensées du monde, lorsque vous vous tenez à l'autel en présence de Dieu.

\par 21 Faites votre offrande afin que vous soyez un pacificateur entre les hommes.

\par 22 Comme tu vas présenter ton offrande à Dieu, quand tu t'avanceras pour t'approcher de l'autel, tu prieras Dieu cent fois sans cesse.

\par 23 « Au début, tu exprimeras cette action de grâce comme suit : 'Ô Dieu, l'incompréhensible, qui ne peut être exploré, le détenteur du pouvoir, la source de la pureté, purifie-moi par ta miséricorde, un don gratuit de ta part. tome.

\par 24 Car je suis une créature de chair et de sang, fuyant vers toi.

\par 25 Je connais mon impureté, et tu me purifieras sûrement, ô Seigneur.

\par 26 « Car voici, ma cause est entre vos mains et mon recours est vers vous.

\par 27 Je connais mon péché, alors purifie-moi, Seigneur, afin que j'entre en ta présence avec respect.

\par 28 Maintenant mes offenses sont graves; Je me suis approché du feu qui brûle.

\par 29 Ta miséricorde est sur toutes choses, afin que tu puisses ôter toutes mes transgressions.

\par 30 Pardonne-moi, même à moi, le pécheur.

\par 31 Et pardonne à toutes tes créatures que tu as façonnées, mais qui n'ont pas entendu et appris de toi.

\par 32 « 'Je suis comme tous ceux qui sont à ton image. Je me suis tourné vers l'accomplissement de ce qui m'est interdit.

\par 33 Je suis venu vers toi et je suis ton serviteur et le fils pécheur de ta nation, mais tu es celui qui pardonne.

\par 34 Pardonne-moi par la grâce qui vient de toi, et écoute ma supplication afin que je sois digne de me tenir devant ton saint autel.

\par 35 Que cet holocauste vous soit agréable.

\par 36 Ne me ramène pas à mon ignorance à cause de mes péchés. Recevez-moi comme la brebis perdue.

\par 37 Que le Dieu qui a pourvu à notre père Adam, à Abel et à Noé, et à notre père Abraham, soit avec toi, ô Jacob, et avec moi aussi.

\par 38 Recevez de ma part mon offrande.

\par 39 « Ainsi, si vous vous êtes approché et avez fait cela avant de monter à l'autel, alors offrez votre sacrifice.

\par 40 Mais vous prendrez garde et veillerez à ne pas attrister l'esprit du Seigneur.

\par 41 Car l'œuvre du sacerdoce n'est pas facile, puisqu'il incombe à chaque prêtre, d'aujourd'hui jusqu'à l'achèvement de la dernière des générations et la fin du monde, de ne pas se rassasier en buvant du vin. ni être rassasié en mangeant du pain ; et qu'il ne devrait pas en parler

\par 42 les soucis du monde ni écouter celui qui en parle. Mais les prêtres doivent consacrer tous leurs efforts et leur vie à la prière, à la vigilance et à la persévérance dans la piété, afin que chacun puisse adresser avec succès sa requête au Seigneur.

\par 43 « De plus, tout homme sur terre, qu'il soit malheureux ou heureux, lui incombe d'observer les commandements appropriés.

\par 44 Car les hommes, après peu de temps, seront éloignés de ce monde et de son intense anxiété.

\par 45 Alors ils seront engagés dans un service saint et angélique en raison de leur pureté.

\par 46 Ils seront présentés devant le Seigneur et ses anges à cause de leurs offrandes pures et de leur service angélique.

\par 47 Car leur conduite terrestre se reflétera dans le ciel, et les anges seront leurs amis à cause de leur foi et de leur pureté parfaites.

\par 48 Grande est leur estime devant le Seigneur, et il n'y a personne, ni petit ni grand, en qui le Seigneur ne fasse du progrès ; car le Seigneur veut que chacun soit sans faute ni offense.

\par 49 « Et maintenant, continuez à implorer Dieu avec repentance pour vos péchés passés, et ne commettez plus de péché.

\par 50 C'est pourquoi ne tuez pas avec l'épée, ne tuez pas avec la langue, ne forniquez pas avec votre corps et ne vous mettez pas en colère jusqu'au coucher du soleil.

\par 51 Ne vous laissez pas recevoir de louanges injustifiées, et ne vous réjouissez pas de la chute de vos ennemis ou de vos frères.

\par 52 Ne blasphème pas ; méfiez-vous des calomnies.

\par 53 Ne regardez pas une femme avec un œil lubrique.

\par 54 Vous vous garderez de ces choses et de ce qui leur ressemble, afin que chacun de vous soit sauvé de la colère qui se manifestera du ciel. »

\chapter{5}

\par 1 Quand les foules qui les entouraient entendirent cela, elles s'écrièrent d'un commun accord, disant : « En vérité, tout ce qu'a dit cet homme vénérable est digne d'attention. »

\par 2 Mais il resta silencieux, remonta son manteau et se couvrit le visage.

\par 3 L'assemblée et le prêtre qui était présent, après un silence, dirent : « Laissez-le se reposer un peu. »

\par 4 Alors l'ange de Dieu vint vers lui et l'emmena au ciel.

\par 5 Là, il vit certaines choses avec crainte.

\par 6 De nombreuses bêtes sauvages (?) étaient à portée de main.

\par 7 Les côtés . . . (?) comme les frères pour qu'ils ne puissent pas se voir.

\par 8 Leurs visages étaient comme des visages de chameaux et certains étaient comme des visages de chiens.

\par 9 D'autres ressemblaient à des visages de lions, d'hyènes et de tigres ; et certains n'avaient qu'un seul œil.

\par 10 Isaac dit : « J'ai regardé et voici, ils s'étaient mis d'accord sur une personne et ils l'empressaient d'avancer.

\par 11 Et après qu'ils eurent fait un signe aux lions, ceux qui marchaient avec lui se retirèrent de lui.

\par 12 Alors les lions se tournèrent vers lui, le déchirèrent par le milieu, le démembrèrent, le mâchèrent et l'engloutirent.

\par 13 Après cela, ils l'expulsèrent de leur bouche et il revint à son état originel.

\par 14 Et après les lions, les autres s'avancèrent et lui firent la même chose.

\par 15 L'un après l'autre, ils le prendraient, et chacun d'eux le mâcherait, l'avalerait et l'expulserait, et il reviendrait à son état originel.

\par 16 Alors je dis à l'ange : « Ô mon seigneur, quel est le péché que cet homme a commis pour qu'il ait à supporter un fardeau pareil ?

\par 17 L'ange me dit : « C'est parce que cet homme, que tu vois, a été en inimitié avec son prochain pendant cinq heures, et il est mort sans s'être réconcilié avec lui.

\par 18 Il fut donc livré à cinq des bourreaux afin qu'ils le tourmentent pendant une année entière pour chacune des cinq heures qu'il passait comme ennemi de son ami.

\par 19 Alors l'ange me dit : « Ô mon bien-aimé Isaac, vois ici les soixante myriades qui infligent la torture pour chaque heure où l'homme reste hostile à son prochain.

\par 20 Il est amené ici vers ces créatures qui le torturent, chacune d'elles pendant une heure jusqu'à ce qu'une année complète soit accomplie s'il n'avait pas fait la paix et ne s'était pas repenti de son péché avant son enlèvement et sa séparation d'avec son corps.

\par 21 Puis il m'a amené à une rivière de feu. Je l'ai vu palpiter, avec ses vagues s'élevant jusqu'à environ trente coudées ; et son bruit était comme celui du tonnerre.

\par 22 J'ai vu beaucoup d'âmes y être immergées jusqu'à une profondeur d'environ neuf coudées.

\par 23 Ceux qui étaient dans ce fleuve pleuraient et criaient à haute voix et avec de grands gémissements.

\par 24 Et ce fleuve avait la sagesse dans son feu : il ne ferait pas de mal aux justes, mais seulement aux pécheurs en les brûlant.

\par 25 Cela les brûlerait tous à cause de la puanteur et de la répugnance de l'odeur qui entourait les pécheurs.

\par 26 Alors j'ai observé le fleuve profond ? dont la fumée était montée devant moi, et j'ai vu un groupe de gens au fond, criant, pleurant, chacun se lamentant.

\par 27 L'ange m'a dit : « Regarde au fond pour observer ceux que tu vois au plus bas fond. Ce sont eux qui ont commis le péché de Sodome ; en réalité, ils devaient subir une punition sévère. »

\par 28 Puis j'ai vu le surveillant du châtiment et il était tout de feu.

\par 29 Il frappait les myrmidons de l'enfer (ses aides) et leur disait : « Tuez-les afin que l'on sache que Dieu existe pour toujours. »

\par 30 Alors l'ange me dit : « Leve les yeux et regarde toute la gamme des châtiments. »

\par 31 Mais je dis à l'ange : « Ma vue ne peut pas les embrasser à cause de leur grand nombre ; mais je désire comprendre combien de temps ces gens vont subir cette torture.

\par 32 Il m'a dit : « Jusqu'à ce que le Dieu de miséricorde devienne miséricordieux et ait pitié d'eux. »

\chapter{6}

\par 1 Après cela, l'ange m'a emmené au ciel et j'ai vu Abraham.

\par 2 Alors je me suis prosterné devant lui et il m'a reçu avec grâce, lui et tous les pieux.

\par 3 Puis ils se sont tous réunis et m'ont fait honneur à cause de mon père.

\par 4 Alors ils me prirent par la main et me conduisirent jusqu'au rideau devant le trône du Père.

\par 5 Alors je me suis prosterné devant lui et je l'ai adoré avec mon père et tous les saints, tandis que nous poussions des louanges et criions à haute voix, en disant : « Très saint, très saint, très saint est le Seigneur Sabaoth ! Le ciel et la terre sont remplis de ta gloire sanctifiée.

\par 6 Alors le Seigneur me dit du haut de sa sainteté : « Quant à quiconque donnera à son fils le nom de mon bien-aimé Isaac, ma bénédiction reposera sur lui et sera dans sa maison pour toujours.

\par 7 Excellente est ta venue, ô Abraham, fidèle ; excellente est votre lignée, et excellente est la présence ici de cette lignée bénie.

\par 8 Ainsi maintenant, tout ce que vous demanderez au nom de votre fils bien-aimé Isaac, vous l'aurez aujourd'hui comme alliance pour toujours.

\par 9 Alors mon père Abraham répondit et dit : « À toi est la souveraineté, ô Seigneur, chef de l'univers. »

\par 10 Le Seigneur, du haut de sa sainteté, dit à mon père Abraham : « Tout homme qui appellera son fils du nom de mon bien-aimé Isaac, ou qui rédigera son propre testament, aura une bénédiction qui ne finira pas, et ma bénédiction sur sa maison ne cessera pas.

\par 11 Ou si quelqu'un donne à manger à un pauvre le jour de la fête de mon bien-aimé Isaac, je te le donnerai dans mon royaume.

\par 12 Alors mon père Abraham dit : « Ô Père, Dieu, maître de l'univers, même s'il n'est pas capable d'écrire son testament ou son alliance, que ta bénédiction et ta miséricorde l'enveloppent, car tu es le miséricordieux. »

\par 13 L'Éternel dit à Abraham : Qu'il nourrisse de pain celui qui a faim et je lui donnerai une place dans mon royaume et il sera présent avec toi dès le premier moment du banquet millénaire.

\par 14 Le Dieu salvateur dit aussi à mon père Abraham : « Et s'il est si pauvre qu'il ne trouve pas de pain dans sa maison, alors qu'il passe une nuit entière à commémorer mon bien-aimé Isaac sans dormir et je lui accorderai un héritage dans mon royaume.

\par 15 Mon père Abraham a dit : « Et s'il est faible et ne peut supporter la veillée, alors que votre miséricorde et votre compassion l'enveloppent encore. »

\par 16 Alors l'Éternel lui dit : « Alors qu'il offre un peu d'encens en mon nom, le jour de la mémoire de mon bien-aimé Isaac, ton fils.

\par 17 Et s'il ne sait pas lire, qu'il aille entendre la lecture de celui qui sait la lire.

\par 18 S'il ne peut faire aucune de ces choses, alors qu'il entre dans sa maison, ferme la porte derrière lui et prie cent prières de repentance ; alors je te le donnerai pour fils dans mon royaume.

\par 19 Mais par-dessus tout cela, qu'il apporte une offrande au jour de la mémoire de mon bien-aimé Isaac.

\par 20 Et tous ceux qui feront tout ce que j'ai dit recevront l'héritage du royaume dans mes cieux.

\par 21 Et tous ceux qui ont pris la peine d'écrire leurs testaments, leurs alliances et leurs histoires de vie, et ont fait preuve de miséricorde ne serait-ce que (en donnant) un verre d'eau froide, et ont cru de tout leur cœur, avec eux seront ma force et mon Saint-Esprit. pour la prospérité de leurs affaires dans le monde.

\par 22 Il n'y aura aucun problème lors de leur départ (de ce monde), je leur accorderai une vie dans mon royaume, et ils seront présents dès le premier instant du banquet millénaire.

\par 23 La paix soit sur vous, ô mes bien-aimés, les saints !'

\par 24 Lorsqu'il eut terminé tout ce discours, les êtres célestes commencèrent à crier, disant : « Très saint, très saint, très saint est le Seigneur, Sabaoth ! Le ciel et la terre sont remplis de ta gloire sanctifiée.

\par 25 Le Père qui contrôle tout répondit depuis ce lieu saint et dit : « Ô Michel, mon fidèle serviteur, appelle tous les anges et tous les saints. »

\par 26 Puis il monta sur le char des séraphins, tandis que les chérubins marchaient devant [avec les anges.

\par 27 Et lorsqu'ils furent arrivés au lit de notre père Isaac, notre père Isaac vit aussitôt le visage de notre Seigneur, plein de joie envers lui.

\par 28 Il s'écria : « C'est bien que tu sois venu, mon Seigneur, avec ton grand archange Michel. C'est bien que tu sois venu, mon Père, avec tous les saints. »].?

\par 29 Lorsqu'il eut dit cela, Jacob fut très troublé et il s'accrocha à son père et l'embrassa en pleurant.

\par 30 Alors notre père Isaac le releva et lui fit un signe avec ses yeux, disant : « Tais-toi, mon garçon. »

\par 31 Alors Abraham dit à l'Éternel : « Ô Seigneur, souviens-toi aussi de mon (petit)fils Jacob. »

\par 32 Alors l'Éternel lui dit : « Mon pouvoir sera avec lui, il glorifiera mon nom, il deviendra maître du pays promis, et l'ennemi n'aura plus d'emprise sur lui. »

\par 33 Et notre père Isaac dit : « Jacob, mon fils bien-aimé, observe mon commandement que je te donne aujourd'hui : garde mon corps.

\par 34 Ne profanez pas l'image de Dieu par la façon dont vous la traitez ; car l’image de l’homme a été faite comme l’image de Dieu ; et Dieu vous traitera en conséquence au moment où vous le rencontrerez et le verrez face à face.

\par 35 Il est le premier et le dernier, comme l'ont dit les prophètes.

\chapter{7}

\par 1 Quand Isaac eut dit cela, le Seigneur ôta son âme de son corps et elle était blanche comme neige ; il en prit possession, le transporta avec lui sur son char sacré et monta avec lui au ciel, pendant que les chérubins chantaient devant lui, ainsi que ses saints anges.

\par 2 Le Seigneur lui a accordé le royaume des cieux ; et tout ce que notre père désirait parmi l'abondance des bénédictions de Dieu qu'il avait, y compris l'accomplissement de son alliance pour toujours.

\chapter{8}

\par 1 Tel fut le décès de notre père Abraham et de notre père Isaac, fils d'Abraham, le vingt-huitième jour du mois de Misri, aujourd'hui même. Ce jour, nous l'avons consacré et désigné.

\par 2 Et le jour où notre père Abraham offrit le sacrifice à Dieu, le vingt-huitième jour du mois d'Amshir, les cieux et la terre furent remplis du doux parfum de sa manière de vivre devant l'Éternel.

\par 3 Et notre père Isaac était comme l'argent qui est brûlé, fondu, purifié et affiné au feu ; de même tous ceux qui sortiront de notre père Isaac, le père des pères.

\par 4 Le jour où Abraham, le père des pères, l'offrit en sacrifice à Dieu, le parfum de son sacrifice monta jusqu'au voile du rideau de celui qui contrôle tout.

\par 5 Bienheureux quiconque fera preuve de miséricorde en ce jour de commémoration du père des pères, notre père Abraham et notre père Isaac, car chacun d'eux aura une demeure dans le royaume des cieux, parce que notre Seigneur a fait avec eux son véritable alliance pour toujours.

\par 6 Et il le gardera pour eux et pour ceux qui viendront après eux, en leur disant : Celui qui aura fait preuve de miséricorde au nom de mon Isaac bien-aimé, voici, je vous le donnerai dans le royaume des cieux et il sera présent avec eux au premier moment du banquet millénaire pour célébrer avec eux dans la lumière éternelle dans le royaume de notre Maître et notre Dieu et de notre Roi et notre Sauveur, Jésus le Messie.

\par 7 Il est celui à qui sont dus la gloire, la dignité, la majesté, la domination, la révérence, l'honneur, la louange et l'adoration, avec le Père miséricordieux et le Saint-Esprit maintenant et pour toujours. , et pour toute l'éternité et pour toujours et à jamais, amen !

\chapter{9}

\par 1 Les obsèques de notre père Isaac sont terminées. Merci et louange à Dieu, toujours, pour toujours et éternellement.



\end{document}