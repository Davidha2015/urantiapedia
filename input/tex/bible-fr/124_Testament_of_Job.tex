\begin{document}


\title{Testament de Job}

\chapter{1}

\par 1 Le jour où il tomba malade et (il) savait qu'il devrait quitter sa demeure corporelle, il convoqua ses sept fils et ses trois filles et leur parla ainsi :

\par 2 « Faites un cercle autour de moi, les enfants, et écoutez, et je vous raconterai ce que le Seigneur a fait pour moi et tout ce qui m'est arrivé.

\par 3 Car je suis Job, ton père.

\par 4 Sachez donc mes enfants, que vous êtes la génération d'un élu et prenez garde à votre noble naissance.

\par 5 Car je suis des fils d'Ésaü. Mon frère s'appelle Nahor et ta mère s'appelle Dinah. Par elle je suis devenu ton père.

\par 6 Car ma première femme est morte avec mes dix autres enfants dans une mort amère.

\par 7 Écoutez maintenant, mes enfants, et je vous révélerai ce qui m'est arrivé.

\par 8 J'étais un homme très riche, vivant à l'Est dans le pays d'Ausitis (Utz) et avant que l'Éternel ne m'appelle Job, on m'appelait Jobab.

\par 9 Le début de mon épreuve fut ainsi. Près de ma maison, il y avait l'idole d'une personne adorée par le peuple ; et je voyais constamment des holocaustes qui lui étaient apportés comme à un dieu.

\par 10 Alors j'ai réfléchi et je me suis dit : 'Est-ce lui qui a fait le ciel et la terre, la mer et nous tous Comment saurai-je la vérité'

\par 11 Et cette nuit-là, alors que je dormais, une voix vint et cria : « Jobab ! Jobab! lève-toi, et je te dirai qui est celui que tu veux connaître.

\par 12 Mais celui à qui le peuple apporte des holocaustes et des libations n'est pas Dieu, mais c'est la puissance et l'œuvre du Séducteur (Satan) par lequel il séduit le peuple.

\par 13 Et quand j'entendis cela, je tombai sur la terre et je me prosternais en disant :

\par 14 'Ô mon Seigneur qui parles pour le salut de mon âme. Je te prie, si c'est l'idole de Satan, je te prie, laisse-moi partir d'ici, la détruire et purifier cet endroit.

\par 15 Car personne ne peut m'interdire de faire cela, puisque je suis le roi de ce pays, afin que ceux qui l'habitent ne soient plus égarés.

\par 16 Et la voix qui parlait hors de la flamme me répondit : 'Tu peux purifier cet endroit.'

\par 17 Mais voici, je t'annonce ce que le Seigneur m'a ordonné de te dire : Car je suis l'archange de Dieu.

\par 18 Et je dis : 'Tout ce qui sera dit à son serviteur.' J'entendrai ».

\par 19 Et l'archange me dit : Ainsi parle le Seigneur : Si tu entreprends de détruire et d'enlever l'image de Satan, il se mettra avec colère à te faire la guerre, et il déploiera contre toi tous ses moyens. malice.

\par 20 Il fera venir sur toi de nombreux fléaux graves et t'enlèvera tout ce que tu as.

\par 21 Il enlèvera tes enfants et t'infligera de nombreux maux.

\par 22 Alors tu devras lutter comme un athlète et résister à la douleur, sûr de ta récompense, surmonter les épreuves et les afflictions.

\par 23 Mais quand tu persévéreras, je ferai connaître ton nom à travers toutes les générations de la terre jusqu'à la fin du monde.

\par 24 Et je te rendrai tout ce que tu avais, et la double partie de ce que tu perdras te sera donnée afin que tu saches que Dieu ne considère pas la personne mais donne à chacun qui mérite le bien. .

\par 25 Et aussi cela te sera donné, et tu mettras une couronne d'amarant.

\par 26 Et à la résurrection tu te réveilleras pour la vie éternelle. Alors tu sauras que le Seigneur est juste, vrai et puissant.

\par 27 Sur quoi, mes enfants, je répondis : « Par amour de Dieu, je supporterai jusqu'à la mort tout ce qui m'arrivera, et je ne reculerai pas ».

\par 28 Alors l'ange m'a mis son sceau et m'a quitté.

\chapter{2}

\par 1 Après cela, je me suis levé pendant la nuit, j'ai pris cinquante esclaves, je suis allé au temple de l'idole et je l'ai entièrement détruit.

\par 2 Et je suis donc retourné chez moi et j'ai donné l'ordre que la porte soit bien verrouillée ; disant à mes portiers :

\par 3 'Si quelqu'un me demande, ne me rapporte rien, mais dis-lui : Il enquête sur les affaires urgentes.' Il est à l'intérieur.

\par 4 Alors Satan se déguisa en mendiant et frappa lourdement à la porte, disant au portier :

\par 5 'Faites un rapport à Job et dites que je désire le rencontrer',

\par 6 Et le portier entra et me dit cela, mais il apprit de moi que j'étudiais.

\par 7 Le Malin, ayant échoué, s'en alla et prit sur son épaule un vieux panier déchiré et entra et parla au portier en disant : « Dis à Job : Donne-moi du pain de tes mains pour que je mange. »

\par 8 Et quand j'entendis cela, je lui donnai du pain brûlé pour le lui donner, et je lui fis savoir : 'Ne t'attends pas à manger de mon pain, car cela t'est interdit'.

\par 9 Mais la portière, ayant honte de lui remettre le pain brûlé et cendré, comme elle ne savait pas que c'était Satan, prit de son propre pain fin et le lui donna.

\par 10 Mais il le prit et, sachant ce qui s'était passé, dit à la jeune fille : 'Va d'ici, mauvaise servante, et apporte-moi le pain qu'on t'a donné à me remettre.'

\par 11 Et le serviteur cria et dit avec tristesse : « Tu dis la vérité, en disant que je suis un mauvais serviteur. parce que je n'ai pas fait ce que mon maître m'a demandé de faire.

\par 12 Et il se retourna et lui apporta le pain brûlé et lui dit : 'Ainsi parle mon seigneur : Tu ne mangeras plus de mon pain, car cela t'est interdit.'

\par 13 Et il m'a donné cela [en disant : Ceci, je le donne] afin qu'on ne me reproche pas que je n'ai pas donné à l'ennemi qui l'a demandé.)

\par 14 Et quand Satan entendit cela, il me renvoya le serviteur, disant : 'Comme tu vois ce pain tout brûlé, ainsi je brûlerai bientôt ton corps pour le rendre ainsi'.

\par 15 Et je répondis : 'Fais ce que tu désires faire et accomplis tout ce que tu complotes.' Car je suis prêt à endurer tout ce que tu m'apporteras.

\par 16 Et quand le diable entendit cela, il me quitta, et, montant sous le ciel [le plus haut], il prit de la part de l'Éternel le serment qu'il aurait pouvoir sur tous mes biens.

\par 17 Et après avoir pris le pouvoir, il s'en alla et m'enleva aussitôt toutes mes richesses.

\chapter{3}

\par 1 Car j'avais cent trente mille brebis, et j'en ai séparé sept mille pour le vêtement des orphelins et des veuves, des nécessiteux et des malades.

\par 2 J'avais un troupeau de huit cents chiens qui gardaient mes moutons et en plus deux cents pour garder ma maison.

\par 3 Et j'avais neuf moulins en activité pour toute la ville et des navires pour transporter les marchandises, et j'en installais dans chaque ville et dans les villages pour les faibles, les malades et les malheureux.

\par 4 Et j'avais trois cent quarante mille ânes nomades, et j'en ai mis de côté cinq cents, et j'ordonne que leurs descendants soient vendus et que le produit soit donné aux pauvres et aux nécessiteux.

\par 5 Car de tous les pays les pauvres venaient à ma rencontre.

\par 6 Car les quatre portes de ma maison étaient ouvertes, chacune ayant sous la garde d'un gardien qui devait voir s'il y avait des gens qui venaient demander l'aumône, et s'ils me verraient assis à l'une des portes pour pouvoir sortir par l'autre et prendre tout ce dont ils avaient besoin.

\par 7 J'avais aussi trente tables immobiles dressées à toute heure pour les seuls étrangers, et j'avais aussi douze tables dressées pour les veuves.

\par 8 Et si quelqu'un venait demander l'aumône, il trouvait sur ma table à manger pour emporter tout ce dont il avait besoin, et je ne refoulais personne pour sortir de ma porte le ventre vide.

\par 9 J'avais aussi trois mille cinq cents paires de bœufs, et j'en choisis cinq cents et je les fis travailler au labourage.

\par 10 Et avec ceux-là, j'avais fait tous les travaux dans chaque champ par ceux qui voulaient s'en charger, et le revenu de leurs récoltes, je le mettais de côté pour les pauvres sur leur table.

\par 11 J'avais aussi cinquante boulangeries d'où j'envoyais [le pain] à la table des pauvres.

\par 12 Et j'ai fait choisir des esclaves pour leur service.

\par 13 Il y avait aussi des étrangers qui virent ma bonne volonté ; ils voulaient eux-mêmes servir de serveurs.

\par 14 D'autres, étant dans la détresse et incapables de gagner leur vie, vinrent avec la requête en disant :

\par 15 « Nous te prions, puisque nous aussi pouvons remplir cette fonction de serviteurs (diacres) et n'avoir aucune possession, aie pitié de nous et avance-nous de l'argent afin que nous puissions aller dans les grandes villes et vendre des marchandises.

\par 16 Et nous pouvons donner le surplus de notre profit comme aide aux pauvres, et ensuite nous te rendrons le tien (l'argent).

\par 17 Et quand j'ai entendu cela, j'ai été heureux qu'ils m'enlèvent tout cela pour cultiver la charité envers les pauvres.

\par 18 Et de bon cœur, je leur ai donné ce qu'ils voulaient, et j'ai accepté leur caution écrite, mais je n'ai accepté d'eux aucune autre garantie que le document écrit.

\par 19 Et ils partaient à l'étranger et donnaient du temps pauvre dans la mesure où ils réussissaient.

\par 20 Mais souvent, une partie de leurs biens se perdait sur la route ou sur la mer, ou on les leur volait.

\par 21 Alors ils venaient et disaient : 'Nous te prions, agis généreusement envers nous afin que nous puissions voir comment nous pouvons te restituer les tiens'.

\par 22 Et quand j'ai entendu cela, j'ai eu de la sympathie pour eux, et je leur ai remis leur caution, et l'ayant souvent lu devant eux, je l'ai déchirée et je les ai libérés de leur dette en leur disant :

\par 23 'Ce que j'ai consacré au profit des pauvres, je ne vous le retirerai pas'.

\par 24 C'est pourquoi je n'ai rien accepté de mon débiteur.

\par 25 Et quand un homme au cœur joyeux est venu me dire : Je n'ai pas besoin d'être contraint d'être un travailleur rémunéré pour les pauvres.

\par 26 Mais je veux servir les nécessiteux à ta table », et il consentit à travailler, et il mangea sa part.

\par 27 Je lui ai donc donné néanmoins son salaire, et je suis rentré chez moi joyeux.

\par 28 Et comme il ne voulait pas le prendre, je le forçai à le faire, en disant : « Je sais que tu es un travailleur qui cherche et attend son salaire, et tu dois le prendre. »

\par 29 Jamais je n'ai différé le paiement du salaire du mercenaire ou de tout autre, ni retenu dans ma maison un seul soir le salaire qui lui était dû.

\par 30 Ceux qui traitaient les vaches et les brebis faisaient signe aux passants qu'ils devaient prendre leur part.

\par 31 Car le lait coulait en si grande quantité qu'il se transformait en beurre sur les collines et au bord des chemins ; et près des rochers et des collines gisaient les troupeaux qui avaient donné naissance à leur progéniture.

\par 32 Car mes serviteurs se lassèrent de garder la viande des veuves et des pauvres et de la diviser en petits morceaux.

\par 33 Car ils maudissaient et disaient : « Oh, si nous avions de sa chair pour que nous puissions être rassasiés », bien que j'étais très bon envers eux,

\par 34 J'avais aussi six harpes [et six esclaves pour jouer des harpes] et aussi une cithare, un décaccorde, et je la jouais pendant le jour.

\par 35 Et j'ai pris la cithare, et les veuves ont répondu après leurs repas.

\par 36 Et avec l'instrument de musique, je leur ai rappelé Dieu qu'ils devaient louer le Seigneur.

\par 37 Et quand mes esclaves murmuraient, alors je prenais les instruments de musique et j'en jouais autant qu'elles l'auraient fait pour leur salaire, et je leur donnais un répit de leur travail et de leurs soupirs.

\chapter{4}

\par 1 Et mes enfants, après avoir pris charge du service, prenaient chaque jour leurs repas avec leurs trois sœurs en commençant par le frère aîné, et faisaient un festin.

\par 2 Et je me levai le matin et leur offris en sacrifice pour le péché cinquante béliers et dix-neuf brebis, et ce qui restait comme reste fut consacré aux pauvres.

\par 3 Et je leur dis : « Prenez ceci comme résidu et priez pour mes enfants.

\par 4 Peut-être que mes fils ont péché devant le Seigneur, en disant avec hauteur d'esprit : Nous sommes les enfants de cet homme riche. Tous ces biens nous appartiennent ; pourquoi devrions-nous être les serviteurs des pauvres'

\par 5 Et en parlant ainsi avec un esprit hautain, ils ont peut-être provoqué la colère de Dieu, car un orgueil excessif est une abomination devant l'Éternel.

\par 6 J'ai donc apporté des bœufs en offrande au prêtre près de l'autel en disant : « Que mes enfants ne pensent jamais de mal à Dieu dans leur cœur. »

\par 7 Pendant que je vivais de cette manière, le Séducteur ne pouvait pas supporter de voir le bien [que je faisais], et il a exigé la guerre de Dieu contre moi.

\par 8 Et il s'est attaqué à moi cruellement.

\par 9 Il a d'abord brûlé le grand nombre de moutons, puis les chameaux, puis il a brûlé le bétail et tous mes troupeaux ; ou bien ils étaient capturés non seulement par des ennemis, mais aussi par ceux qui avaient reçu des bénéfices de ma part.

\par 10 Et les bergers sont venus et m'ont annoncé cela.

\par 11 Mais quand je l'ai entendu, j'ai loué Dieu et je n'ai pas blasphémé.

\par 12 Et quand le Séducteur apprit ma force d'âme, il complota de nouvelles choses contre moi.

\par 13 Il s'est déguisé en roi de Perse et a assiégé ma ville, et après avoir emmené tous ceux qui s'y trouvaient, il leur a parlé avec méchanceté, en disant dans un langage vantard :

\par 14 « Cet homme Job, qui a obtenu tous les biens de la terre et n'a rien laissé aux autres, il a détruit et démoli le temple de Dieu.

\par 15 C'est pourquoi je lui rendrai ce qu'il a fait à la maison du grand dieu.

\par 16 Maintenant, viens avec moi et nous pillerons tout ce qui reste dans sa maison.

\par 17 Et ils lui répondirent : « Il a sept fils et trois filles.

\par 18 Prenez garde qu'ils ne s'enfuient dans d'autres pays et qu'ils ne deviennent nos tyrans, puis ne nous envahissent par la force et ne nous tuent.''

\par 19 Et il dit : N'ayez pas du tout peur. J'ai détruit par le feu ses troupeaux et ses richesses, et j'ai capturé le reste, et voici, je tuerai ses enfants.

\par 20 Et après avoir parlé ainsi, il alla jeter la maison sur mes enfants et les tua.

\par 21 Et mes concitoyens, voyant que ce qu'il disait était devenu réalité, sont venus et m'ont poursuivi, et m'ont dépouillé de tout ce qui était dans ma maison.

\par 22 Et j'ai vu de mes propres yeux le pillage de ma maison, et des hommes sans culture et sans honneur étaient assis à ma table et sur mes lits, et je ne pouvais pas leur faire de remontrances.

\par 23 Car j'étais épuisée comme une femme dont les reins se déchaînent à cause d'une multitude de douleurs, me souvenant principalement que ce combat m'avait été prédit par le Seigneur par l'intermédiaire de son ange.

\par 24 Et je suis devenu comme quelqu'un qui, voyant la mer agitée et les vents contraires, tandis que le chargement du navire au milieu de l'océan est trop lourd, jette le fardeau dans la mer, en disant :

\par 25 «Je souhaite détruire tout cela uniquement pour rentrer sain et sauf dans la ville et pouvoir profiter du navire sauvé et du meilleur de mes affaires.»

\par 26 C'est ainsi que j'ai géré mes propres affaires.

\par 27 Mais un autre messager est venu et m'a annoncé la ruine de mes propres enfants, et j'ai été secoué de terreur.

\par 28 Et j'ai déchiré mes vêtements et j'ai dit : L'Éternel a donné, l'Éternel a pris. Comme le Seigneur l’a jugé bon, ainsi cela s’est produit. Que le nom du Seigneur soit béni.

\chapter{5}

\par 1 Et quand Satan vit qu'il ne pouvait pas me désespérer, il alla demander mon corps au Seigneur afin de m'infliger la peste, car le Malin ne pouvait supporter ma patience.

\par 2 Alors le Seigneur m'a livré entre ses mains pour qu'il utilise mon corps comme il le voulait, mais il ne lui a donné aucun pouvoir sur mon âme.

\par 3 Et il est venu vers moi alors que j'étais assis sur mon trône, pleurant toujours mes enfants.

\par 4 Et il ressemblait à un grand ouragan, il renversa mon trône et me jeta à terre.

\par 5 Et je suis resté allongé sur le sol pendant trois heures. et il me frappa d'une forte plaie depuis le sommet de la tête jusqu'à la pointe des pieds.

\par 6 Et je quittai la ville dans une grande terreur et dans un grand malheur, et je m'assis sur un fumier, mon corps étant vermoulu.

\par 7 Et j'ai mouillé la terre avec l'humidité de mon corps malade, car de la matière coulait de mon corps, et de nombreux vers le couvraient.

\par 8 Et lorsqu'un seul ver se glissait hors de mon corps, je le remettais en disant : « Reste à l'endroit où tu as été placé jusqu'à ce que Celui qui t'a envoyé t'ordonne ailleurs. »

\par 9 J'ai donc souffert pendant plusieurs années, assis sur un fumier à l'extérieur de la ville, tandis que j'étais frappé par la peste.

\par 10 Et j'ai vu de mes propres yeux mes enfants tant désirés [portés par les anges au ciel]

\par 11 Et ma humble épouse qui avait été amenée dans sa chambre nuptiale dans un si grand luxe et avec des lanciers comme gardes du corps. Je l'ai vue faire le travail de porteuse d'eau comme une esclave dans la maison d'un homme ordinaire pour gagner du pain et me l'apporter.

\par 12 Et dans ma douloureuse affliction, je dis : « Oh, que ces dirigeants vantards de la ville, que je n'ai pas pensé être égaux à mes chiens de berger, emploient maintenant ma femme comme servante !

\par 13 Et après cela, j'ai repris courage.

\par 14 Mais ensuite, ils lui retirèrent même le pain, afin qu'elle ne soit que sa propre nourriture.

\par 15 Mais elle le prit et le partagea entre elle et moi, en disant tristement : « Malheur à moi ! Désormais il ne peut plus se nourrir de pain, et il ne peut pas aller au marché demander du pain aux marchands de pain pour me l'apporter pour qu'il le mange. »

\par 16 Et quand Satan apprit cela, il prit l'apparence d'un vendeur de pain, et ce fut comme par hasard que ma femme le rencontra et lui demanda du pain pensant que c'était ce genre d'homme.

\par 17 Mais Satan lui dit : « Donne-moi le prix, et prends ensuite ce que tu veux. »

\par 18 Sur quoi elle répondit : Où trouverai-je de l'argent ? Ne sais-tu pas quel malheur m'est arrivé. Si tu as de la pitié, montre-la-moi ; sinon, tu verras.

\par 19 Et il répondit en disant : « Si tu n'avais pas mérité ce malheur, tu n'aurais pas souffert tout cela.

\par 20 Maintenant, s'il n'y a pas de pièce d'argent dans ta main, donne-moi les cheveux de ta tête et prends trois miches de pain pour cela, afin que tu y vives pendant trois jours.

\par 21 Alors elle se dit : «Qu'est-ce que les cheveux de ma tête en comparaison de mon mari affamé»

\par 22 Et ainsi, après avoir réfléchi, elle lui dit : « Lève-toi et coupe-moi les cheveux ».

\par 23 Puis il prit une paire de ciseaux, lui ôta les cheveux de la tête en présence de tous, et lui donna trois miches de pain.

\par 24 Puis elle les prit et me les apporta. Et Satan la suivait sur le chemin, se cachant pendant qu'il marchait et troublant beaucoup son cœur.

\chapter{6}

\par 1 Et aussitôt ma femme s'est approchée de moi et a crié à haute voix et en pleurant, elle a dit : « Job ! Emploi! Combien de temps resteras-tu assis sur le fumier à l'extérieur de la ville, réfléchissant encore un moment et attendant d'obtenir le salut que tu espères !

\par 2 Et j'ai erré de lieu en lieu, errant comme un mercenaire, voici, leur souvenir a déjà disparu de la terre.

\par 3 Et mes fils et les filles que je portais sur mon sein et les travaux et les douleurs que j'ai soutenus n'ont servi à rien.

\par 4 Et tu restes assis dans un état malodorant de douleurs et de vers, passant les nuits dans l'air froid.

\par 5 Et j'ai enduré toutes les épreuves, les ennuis et les douleurs, jour et nuit, jusqu'à ce que je réussisse à t'apporter du pain.

\par 6 Car ton surplus de pain ne m'est plus permis ; et comme je peux à peine prendre ma propre nourriture et la partager entre nous, j'ai réfléchi dans mon cœur qu'il n'était pas juste que tu souffres et que tu aies faim de pain.

\par 7 Et c'est ainsi que j'ai osé aller au marché sans pudeur. et quand le marchand de pain me disait : « Donne-moi de l'argent. et tu auras du pain ». Je lui ai révélé notre état de détresse.

\par 8 Alors je l'entendis dire : « Si tu n'as pas d'argent, donne-moi les cheveux de ta tête, et prends trois miches de pain afin que vous puissiez en vivre pendant trois jours ».

\par 9 Et j'ai cédé au tort et je lui ai dit : « Lève-toi et coupe-moi les cheveux ! et il se leva et, en disgrâce, me coupa les cheveux avec des ciseaux sur la place du marché, tandis que la foule restait là et s'interrogeait.

\par 10 Qui ne s'étonnerait alors de dire : « Est-ce là Sitis, la femme de Job, qui avait quatorze rideaux pour couvrir son salon intérieur, et des portes dans les portes, afin qu'il soit grandement honoré celui qui s'approchait d'elle, et maintenant voici, elle troque ses cheveux contre du pain !

\par 11 Qui avait des chameaux chargés de marchandises. et ils ont été amenés dans des pays reculés pour les pauvres, et maintenant elle vend ses cheveux contre du pain !

\par 12 Voici celle qui avait sept tables dressées immobiles dans sa maison, où mangeaient chaque pauvre et chaque étranger, et maintenant elle vend ses cheveux contre du pain !

\par 13 Voici celle qui avait le bassin d'or et d'argent pour se laver les pieds, et maintenant elle marche sur la terre et [vend ses cheveux contre du pain !]

\par 14 Voici celle qui avait ses vêtements faits de byssus entrelacés d'or, et maintenant elle échange ses cheveux contre du pain !

\par 15 Voici celle qui avait des couches d'or et d'argent, et maintenant elle vend ses cheveux contre du pain !

\par 16 Bref donc, Job, après toutes les choses qui m'ont été dites, je te dis maintenant en un seul mot :

\par 17 « Puisque la faiblesse de mon cœur m'a brisé les os, lève-toi donc, prends ces miches de pain et profite-en, puis dis quelque chose contre le Seigneur et meurs !

\par 18 Car moi aussi, j'échangerais la torpeur de la mort contre la subsistance de mon corps ».

\par 19 Mais je lui répondis : « Voici, depuis sept ans, je suis frappé par la peste, et j'ai résisté aux vers de mon corps, et je n'ai pas été accablé dans mon âme par toutes ces douleurs.

\par 20 Et quant à la parole que tu dis : « Dis quelque chose contre Dieu et meurs ! », je soutiendrai avec toi le mal que tu vois. et supportons la ruine de tout ce que nous possédons.

\par 21 Pourtant tu désires que nous disons quelque mot contre Dieu et qu'il soit échangé contre le grand Pluton [le dieu du monde inférieur.]

\par 22 Pourquoi ne te souviens-tu pas de ces grands biens que nous possédons ? Si ces biens viennent des terres du Seigneur, ne devrions-nous pas aussi endurer les maux et avoir un esprit élevé en toutes choses jusqu'à ce que le Seigneur ait à nouveau pitié de nous et nous fasse preuve de pitié.

\par 23 Ne vois-tu pas le Séducteur se tenir derrière toi et confondre tes pensées afin que tu me séduises

\par 24 Et il se tourna vers Satan et dit : « Pourquoi ne viens-tu pas ouvertement vers moi ? Arrête de te cacher, misérable,

\par 25 Le lion montre-t-il sa force dans la cage de la belette Ou l'oiseau vole-t-il dans le panier ? Je te le dis maintenant : Va-t-en et fais ta guerre contre moi ».

\par 26 Alors il sortit de derrière ma femme et se plaça devant moi en criant et il dit : Voici, Job, je cède et m'abandonne à toi qui n'es que chair alors que je suis esprit.

\par 27 Tu es frappé par la peste, mais moi, je suis dans une grande détresse.

\par 28 Car je suis comme un lutteur en lutte avec un lutteur qui, dans un combat à une main, a démoli son adversaire, l'a couvert de poussière et lui a brisé tous les membres, tandis que l'autre qui se trouve en dessous, ayant déployé son bravoure, émet des sons de triomphe témoignant de sa propre excellence supérieure.

\par 29 Ainsi, toi, ô Job, tu es déprimé et frappé de peste et de douleur, et pourtant tu as remporté la victoire dans le combat de lutte avec moi, et voici, je me soumets à toi.

\par 30 Puis il m'a laissé confus.

\par 31 Maintenant, mes enfants, montrez-vous aussi un cœur ferme dans tout le mal qui vous arrive, car plus grande que tout est la fermeté du cœur.

\chapter{7}

\par 1 À ce moment-là, les rois apprirent ce qui m'était arrivé et se levèrent et vinrent vers moi. chacun de son pays pour me rendre visite et me réconforter.

\par 2 Et lorsqu'ils s'approchèrent de moi, ils crièrent à haute voix et chacun déchira ses vêtements.

\par 3 Et après s'être prosternés, touchant la terre avec leur tête, ils s'assirent à côté de moi pendant sept jours et sept nuits, et personne ne prononça un mot.

\par 4 Ils étaient au nombre de quatre : Eliplaz, roi de Théman, Balad, Sophar et Elilhu.

\par 5 Et lorsqu'ils furent assis, ils conversèrent de ce qui m'était arrivé.

\par 6 Or, quand pour la première fois ils étaient venus vers moi et que je leur avais montré mes pierres précieuses, ils furent étonnés et dirent :

\par 7 « Si de nous trois rois tous nos biens étaient réunis en un seul, cela n'atteindrait pas les pierres précieuses du royaume (couronne) de Jobab. Car tu es d'une plus grande noblesse que tous les peuples de l'Orient.

\par 8 Et quand donc ils arrivèrent au pays d'Ausitis «Uz» pour me rendre visite, ils demandèrent dans la ville : «Où est Jobab, le chef de tout ce pays»

\par 9 Et ils leur dirent à mon sujet : « Il est assis sur le fumier, à l'extérieur de la ville, car il n'est pas entré dans la ville depuis sept ans ».

\par 10 Et puis ils s'enquirent de nouveau de mes biens, et tout ce qui m'était arrivé leur fut révélé.

\par 11 Et quand ils eurent appris cela, ils sortirent de la ville avec les habitants, et mes concitoyens me leur montrèrent.

\par 12 Mais ceux-ci protestèrent et dirent : « Certainement, ce n'est pas Jobab ».

\par 13 Et pendant qu'ils hésitaient, Eliphaz dit alors. le roi de Théman : « Venez, approchons-nous et voyons. »

\par 14 Et quand ils s'approchèrent, je me souvins d'eux, et j'ai beaucoup pleuré quand j'ai appris le but de leur voyage.

\par 15 Et je jetai de la terre sur ma tête, et tout en secouant la tête, je leur révélai que j'étais [Job].

\par 16 Et quand ils m'ont vu secouer la tête, ils se sont jetés à terre, tous saisis d'émotion.

\par 17 Et pendant que leurs armées se tenaient là, je vis les trois rois étendus par terre pendant trois heures comme morts.

\par 18 Alors ils se levèrent et se dirent : Nous ne pouvons pas croire que celui-ci soit Jobab.

\par 19 Et finalement, après avoir passé sept jours à s'enquérir de tout ce qui me concernait et à chercher mes troupeaux et mes autres biens, ils dirent :

\par 20 « Ne savons-nous pas combien de biens il a envoyé dans les villes et les villages des environs pour les donner aux pauvres, sans compter tout ce qu'il a donné dans sa propre maison ? Comment donc aurait-il pu tomber dans quel état de perdition et de misère !

\par 21 Et après les sept jours, Elihu dit aux rois : « Approchons-nous et examinons-le soigneusement, s'il est vraiment Jobab ou non. »

\par 22 Et eux, n'étant pas éloignés d'un demi-mile (stade) de son corps malodorant, ils se levèrent et s'approchèrent, portant du parfum dans leurs mains, tandis que leurs soldats allaient avec eux et jetaient de l'encens odorant autour d'eux pour qu'ils puissent venir. proche de moi.

\par 23 Et après avoir passé ainsi trois heures, couvrant le chemin d'arômes, ils approchèrent.

\par 24 Et Eliphaz commença et dit : « C'est vraiment toi, Job, notre compagnon-roi, c'est toi qui possédais la grande gloire

\par 25 Es-tu celui qui brillait autrefois comme le soleil du jour sur toute la terre ? Es-tu celui qui ressemblait autrefois à la lune et aux étoiles resplendissantes toute la nuit »

\par 26 Et je lui répondis et dis : « Je suis », et sur ce, tous pleurèrent et se lamentèrent, et ils chantèrent un chant royal de lamentation, toute leur armée se joignant à eux en chœur.

\par 27 Et encore Eliphaz me dit : « C'est toi qui avais ordonné que sept mille brebis soient données pour l'habillement des pauvres. Où, alors la gloire de ton trône a disparu.

\par 28 C'est toi qui avais ordonné à trois mille bêtes de labourer les champs du pauvre Wither, alors ta gloire est partie !

\par 29 Tu es celui qui avait des couches d'or, et maintenant tu es assis sur une colline de fumier [« Où donc est allée ta gloire ! »]

\par 30 Es-tu celui qui avait soixante tables dressées pour les pauvres ? Es-tu celui qui avait des encensoirs pour le parfum raffiné fait de pierres précieuses, et maintenant tu es dans un état malodorant. Où donc est allée ta gloire ?

\par 31 Es-tu celui qui avait des candélabres d'or posés sur des supports d'argent ? et maintenant tu dois aspirer à l'éclat naturel de la lune [« Où donc est passée ta gloire ! »]

\par 32 C'est toi qui avais fait faire un onguent avec des épices d'encens, et maintenant tu es dans un état de répugnance ! [« Où donc est passée ta gloire ! »]

\par 33 Es-tu celui qui a ridiculisé les méchants et les pécheurs et maintenant tu es devenu la risée de tous ! [«Où donc est passée ta gloire»]

\par 34 Et comme Eliphaz pleurait et se lamentait depuis longtemps, tandis que tous les autres se joignaient à lui, de sorte que le tumulte était très grand, je leur dis :

\par 35 Tais-toi et je te montrerai mon trône et la gloire de sa splendeur : ma gloire sera éternelle.

\par 36 Le monde entier périra, et sa gloire disparaîtra, et tous ceux qui s'y accrochent resteront en dessous, mais mon trône est dans le monde supérieur et sa gloire et sa splendeur seront à la droite du Sauveur dans Le Paradis.

\par 37 Mon trône existe dans la vie des « saints » et sa gloire dans le monde impérissable.

\par 38 Car les rivières seront asséchées et leur arrogance descendra jusqu'au fond de l'abîme, mais les ruisseaux de mon pays dans lequel mon trône est érigé ne se tariront pas, mais resteront ininterrompus en force.

\par 39 Les rois périssent et les dirigeants disparaissent, et leur gloire et leur orgueil sont comme l'ombre dans un miroir, mais mon Royaume dure aux siècles des siècles, et sa gloire et sa beauté sont dans le char de mon Père.

\chapitre{8}

\par 1 Quand je leur parlai ainsi, Ehiphaz se mit en colère et dit aux autres amis : « Dans quel but sommes-nous venus ici avec nos hôtes pour le consoler ? Voici, il nous fait des reproches. Revenons donc à nos pays.

\par 2 Cet homme est assis ici dans une misère vermoulue au milieu d'un état de putréfaction insupportable, et pourtant il défie son salut : 'Les royaumes périront et leurs dirigeants, mais mon royaume, dit-il, durera pour toujours' ».

\par 3 Éliphaz se leva alors en grande agitation, et, se détournant d'eux avec une grande fureur, dit : « Je m'en vais. Nous sommes bien venus le réconforter, mais il nous déclare la guerre face à nos armées ».

\par 4 Mais alors Baldad le saisit par la main et dit : « Il ne faut pas parler ainsi à un homme affligé, et surtout à celui qui est frappé par tant de plaies.

\par 5 Voici, nous, étant en bonne santé, n'osions l'approcher à cause de l'odeur désagréable, qu'avec l'aide d'une abondance d'arômes parfumés. Mais toi, Eliphaz. j’oublie tout cela.

\par 6 Laissez-moi parler clairement. Soyons magnanimes et apprenons quelle en est la cause. Doit-il, en se souvenant de ses anciens jours de bonheur, ne pas devenir fou dans son esprit.

\par 7 Qui ne devrait pas être tout à fait perplexe en se voyant ainsi tomber dans le malheur et les fléaux. Mais permettez-moi de m'approcher de lui afin de découvrir pour quelle raison il est ainsi.

\par 8 Et Baldad se leva et s'approcha de moi en disant : «Es-tu Job» et il dit : «Ton cœur est-il toujours bien gardé»

\par 9 Et je dis : « Je ne me suis pas attaché aux choses terrestres, car la terre et tout ce qui l'habite est instable. Mais mon cœur s’attache au ciel, car il n’y a pas de trouble au ciel ».

\par 10 Alors Baldad reprit et dit : « Nous savons que la terre est instable, car elle change selon les saisons. Parfois c’est en état de paix, et parfois c’est en état de guerre. Mais du ciel, nous entendons dire qu'il est parfaitement stable.

\par 11 Mais es-tu vraiment dans un état de calme. C'est pourquoi laisse-moi demander et parler, et quand tu me répondras à mon premier mot, j'aurai une seconde question à poser, et si tu réponds encore avec des mots bien prononcés, cela il sera manifeste que ton cœur n’a pas été déséquilibré ».

\par 12 Et je dis : « Sur quoi mets-tu ton espérance ? » Et je dis : « Sur le Dieu vivant ».

\par 13 Et il me dit : « Qui t'a dépouillé de tout ce que tu possédais et qui t'a infligé ces fléaux » Et j'ai dit : « Dieu ».

\par 14 Et il dit : « Si tu places encore ton espoir en Dieu, comment pourrait-il faire du mal en matière de jugement, après t'avoir attiré ces fléaux et ces malheurs, et t'avoir ôté tous tes biens.

\par 15 Et puisqu'Il les a pris, il est clair qu'Il ne t'a rien donné. Aucun roi ne déshonorera son soldat qui lui a bien servi comme garde du corps. »

\par 16 [Et je répondis en disant] : « Qui comprend les profondeurs du Seigneur et de sa sagesse pour pouvoir accuser Dieu d'injustice »

\par 17 [Et Baldad dit] : « Réponds-moi, ô Job, à ceci. Je te le dis encore : « Si tu es dans un état de raison calme, apprends-moi si tu as de la sagesse :

\par 18 Pourquoi voyons-nous le soleil se lever à l'Est et se coucher à l'Ouest Et encore une fois, en se levant le matin, nous le trouvons se lever à l'Est Dis-moi ce que tu en penses »

\par 19 Alors je dis : « Pourquoi trahirais-je (bavarder) les puissants mystères de Dieu Et si ma bouche trébuchait en révélant des choses appartenant au Maître Jamais !

\par 20 Qui sommes-nous pour nous intéresser aux affaires du monde supérieur alors que nous ne sommes que chair, voire terre et cendre !

\par 21 Afin que vous sachiez que mon cœur est sain, écoutez ce que je vous demande :

\par 22 Par l'estomac vient la nourriture, et l'eau que l'on boit par la bouche, puis elle coule par la même gorge, et quand les deux descendent pour devenir des excréments, ils se séparent à nouveau ; qui effectue cette séparation ».

\par 23 Et Baldad dit : « Je ne sais pas ». Et je le rejoignis et lui dis : « Si tu ne comprends pas même les sorties du corps, comment peux-tu comprendre les circuits célestes »

\par 24 Alors Sophar reprit et dit : « Nous ne nous enquêtons pas sur nos propres affaires, mais nous désirons savoir si tu es en bonne santé, et voici, nous voyons que ta raison n'a pas été ébranlée.

\par 25 Que veux-tu maintenant que nous fassions pour toi ? Voici, nous sommes venus ici et avons amené les médecins de trois rois, et si tu le souhaites, tu pourras qu'il soit guéri par eux.

\par 26 Mais je répondis et dis : « Ma guérison et ma restauration viennent de Dieu, le Créateur des médecins ».

\chapitre{9}

\par 1 Et comme je leur parlais ainsi, voici, ma femme Sitis accourut, vêtue de haillons. du service du maître par lequel elle était employée comme esclave, bien qu'il lui ait été interdit de sortir, de peur que les rois, en la voyant, ne la prennent comme captive.

\par 2 Et quand elle arriva, elle se jeta à leurs pieds, criant et disant : « Souviens-toi ». Eliphaz et vous autres amis, ce que j'étais autrefois avec vous, et comment j'ai changé, comment je suis maintenant habillé pour vous rencontrer''

\par 3 Alors les rois éclatèrent en grandes larmes et, étant dans une double perplexité, ils gardèrent le silence. Mais Eliphaz prit son manteau de pourpre et l'enveloppa autour d'elle pour s'en envelopper.

\par 4 Mais elle lui demanda en disant : « Je vous demande une faveur, mes seigneurs, d'ordonner à vos soldats de creuser parmi les ruines de notre maison qui sont tombées sur mes enfants, afin que leurs ossements puissent être ramenés dans un endroit sûr. parfait état des tombeaux.

\par 5 Sapin comme nous n'avons, à cause de notre malheur, aucun pouvoir, et ainsi nous pouvons au moins voir leurs os.

\par 6 Car j'ai, comme une brute, le sentiment maternel des bêtes sauvages que mes dix enfants auraient péri en un seul jour et que je ne pourrais donner à aucun d'eux un enterrement décent »

\par 7 Et les rois ordonnèrent de déterrer les ruines de ma maison. Mais je l'ai interdit, économisant

\par 8 « Ne vous donnez pas de peine en vain ; car mes enfants ne seront pas retrouvés, car ils sont sous la garde de leur Créateur et Souverain''.

\par 9 Et les rois répondirent et dirent : « Qui contestera qu'il est fou et qu'il s'extasie ?

\par 10 Car tandis que nous désirons rapporter les ossements de ses enfants, il nous l'interdit en disant : 'Ils ont été pris et placés sous la garde de leur Créateur'. Prouve-nous donc la vérité ».

\par 11 Mais je leur ai dit : « Relevez-moi pour que je puisse me lever, et ils m'ont soulevé en levant mes bras des deux côtés.

\par 12 Et je me suis levé, et j'ai prononcé d'abord la louange de Dieu et après la prière, je leur ai dit : « Regardez avec vos yeux vers l'Orient ».

\par 13 Et ils regardèrent et virent mes enfants avec des couronnes proches de la gloire du Roi, le Souverain des cieux.

\par 14 Et quand ma femme Sitis vit cela, elle tomba à terre et se prosterna devant Dieu, en disant : « Maintenant, je sais que ma mémoire reste auprès du Seigneur ».

\par 15 Et après avoir dit cela, et le soir étant venu, elle partit pour la ville, retourna chez le maître dont elle servait comme esclave, et se coucha à la mangeoire du bétail et y mourut d'épuisement.

\par 16 Et comme son maître despotique la cherchait et ne la trouvait pas, il arriva au troupeau de ses troupeaux, et là il la vit étendue sur la mangeoire, morte, tandis que tous les animaux autour criaient autour d'elle.

\par 17 Et tous ceux qui la voyaient pleuraient et se lamentaient, et le cri s'étendait dans toute la ville.

\par 18 Et le peuple la fit descendre, l'enveloppa et l'enterra près de la maison qui était tombée sur ses enfants.

\par 19 Et les pauvres de la ville firent un grand deuil pour elle et dirent : « Voici cette Sitis dont on ne trouve l'égale en noblesse et en gloire chez aucune femme. Hélas ! elle n’a pas été jugée digne d’un vrai tombeau !

\par 20 Le chant funèbre pour elle, vous le trouverez dans le dossier.

\chapitre{10}

\par 1 Mais Eliphaz et ceux qui étaient avec lui furent étonnés de ces choses, et ils s'assirent avec moi et, me répondant, parlèrent à mon sujet avec des paroles vantardes pendant vingt-sept jours.

\par 2 Ils répétaient sans cesse que j'avais ainsi souffert à juste titre pour avoir commis de nombreux péchés et qu'il ne me restait plus aucun espoir, mais j'ai moi-même rétorqué à ces hommes dans un enthousiasme de dispute.

\par 3 Et ils se levèrent avec colère, prêts à se séparer dans un esprit courroucé. Mais Elihu les conjura de rester encore un peu de temps jusqu'à ce qu'il leur ait montré ce que c'était.

\par 4 « Car, dit-il, vous avez passé tant de jours pour permettre à Job de se vanter d'être juste. Mais je ne le souffrirai plus.

\par 5 Car dès le début j'ai continué à pleurer sur lui, me souvenant de son ancien bonheur. Mais maintenant, il parle avec vantardise et avec un orgueil autoritaire, il dit qu'il a son trône dans les cieux.

\par 6 C'est pourquoi, écoutez-moi, et je vous dirai quelle est la cause de sa destinée.

\par 7 Puis, imprégné de l'esprit de Satan. Elihu a prononcé des paroles dures qui sont écrites dans les archives laissées par Elihu.

\par 8 Et après qu'il eut fini, Dieu m'apparut dans la tempête et dans les nuées, et parla en accusant Elihu et en me montrant que celui qui avait parlé n'était pas un homme, mais une bête sauvage.

\par 9 Et quand Dieu eut fini de me parler, le Seigneur dit à Eliphaz : « Toi et tes amis avez péché en ce que vous n'avez pas dit la vérité concernant mon serviteur Job.

\par 10 C'est pourquoi lève-toi et fais-lui apporter un sacrifice pour ton péché, afin que tes péchés soient pardonnés ; car sans lui, je t’aurais détruit ».

\par 11 Et ils m'apportèrent donc tout ce qui appartenait à un sacrifice, et je le pris et leur apportai un sacrifice pour le péché, et l'Éternel l'accueillit favorablement et leur pardonna leur tort.

\par 12 Alors, quand Eliphaz, Baldad et Sophar virent que Dieu avait gracieusement pardonné leur péché par l'intermédiaire de son serviteur Job, mais qu'il ne daignait pas pardonner à Elihu, alors Eliphaz se mit à chanter un hymne, tandis que les autres répondaient, leurs soldats aussi. se joindre en se tenant près de l’autel.

\par 13 Et Eliphaz parla ainsi : « Le péché et notre injustice ont disparu ;

\par 14 Mais Elihu, le méchant, n'aura aucun souvenir parmi les vivants ; son luminaire s'est éteint et a perdu sa lumière.

\par 15 La gloire de sa lampe s'annoncera pour lui, car il est le fils des ténèbres. et non de lumière.

\par 16 Les portiers du lieu des ténèbres lui donneront en partage leur gloire et leur beauté ; Son royaume a disparu, son trône a moisi et l'honneur de sa stature est dans (le schéol) Hadès.

\par 17 Car il a aimé la beauté du serpent et les écailles (peaux) du dracon, son fiel et son venin appartiennent à Celui du Nord (Zphuni = Adder).

\par 18 Car il ne s'est pas reconnu devant l'Éternel et il ne l'a pas craint, mais il a haï ceux qu'il a choisis (connus).

\par 19 Ainsi Dieu l'a oublié, et 'les saints' l'ont abandonné, sa colère et sa colère seront pour lui une désolation et il n'aura ni miséricorde ni paix dans son cœur, parce qu'il avait le venin d'une vipère sur sa langue .

\par 20 L'Éternel est juste, et ses jugements sont vrais. Chez lui, il n'y a pas de préférence de personne, car il juge tous de la même manière.

\par 21 Voici, le Seigneur vient ! Voici, les « saints » sont préparés : les couronnes et les prix des vainqueurs les précèdent !

\par 22 Que les saints se réjouissent, et que leurs cœurs exultent de joie ; car ils recevront la gloire qui leur est réservée.

\par \textit{Refrain.}

\par 23 Nos péchés sont pardonnés, notre injustice a été purifiée, mais Elihu n'a aucun souvenir parmi les vivants ».

\par 24 Après qu'Éliphaz eut fini le cantique, nous nous levâmes et retournâmes à la ville, chacun dans la maison où il habitait.

\par 25 Et le peuple me fit un festin en signe de gratitude et de joie pour Dieu, et tous mes amis revinrent vers moi.

\par 26 Et tous ceux qui m'avaient vu dans mon ancien état de bonheur, m'interrogeaient en disant : «Quelles sont ces trois choses ici parmi nous»

\chapitre{11}

\par 1 Mais étant désireux de reprendre mon œuvre de bienveillance envers les pauvres, je leur demandai en disant :

\par 2 « Donnez-moi chacun un agneau pour l'habillement des pauvres dans leur état de nudité, et quatre drachmes (pièces de monnaie) d'argent ou d'or »

\par 3 Alors le Seigneur bénit tout ce qui me restait, et au bout de quelques jours je redevins riche en marchandises, en troupeaux et de toutes choses que j'avais perdues, et je reçus de nouveau tout en double nombre.

\par 4 Alors j'ai aussi pris pour femme votre mère et je suis devenu votre père, dix à la place des dix enfants qui étaient morts.

\par 5 Et maintenant, mes enfants, laissez-moi vous avertir : « Voici, je meurs. Tu prendras ma place.

\par 6 Seulement, n'abandonnez pas le Seigneur. Soyez charitable envers les pauvres ; Ne négligez pas les faibles. Ne prenez pas pour femmes des femmes étrangères.

\par 7 Voici, mes enfants, je partagerai entre vous ce que je possède, afin que chacun puisse contrôler le sien et avoir le plein pouvoir de faire le bien avec sa part.

\par 8 Et après avoir ainsi parlé, il apporta tous ses biens et les partagea entre ses sept fils, mais il ne donna rien de ses biens à ses filles.

\par 9 Alors ils dirent à leur père : « Notre seigneur et père ! Ne sommes-nous pas aussi tes enfants ? Pourquoi donc ne nous donnes-tu pas aussi une part de tes biens »

\par 10 Alors Job dit à ses filles : « Ne vous mettez pas en colère, mes filles. Je ne t'ai pas oublié. Voici, je vous ai conservé un bien meilleur que celui que vos frères ont pris. »

\par 11 Et il appela sa fille dont le nom était Day (Yemima) et lui dit : « Prends ce double anneau qui sert de clé et va au trésor et apporte-moi le coffret d'or, afin que je te donne ton bien. ».

\par 12 Et elle alla le lui apporter, et il l'ouvrit et en sortit des ceintures à trois cordes dont personne ne peut parler de l'apparence.

\par 13 Car ce n'étaient pas des œuvres terrestres, mais des étincelles de lumière céleste les traversaient comme les rayons du soleil.

\par 14 Et il donna un cordon à chacune de ses filles et dit : « Mets-les comme ceintures autour de toi, afin que tous les jours de ta vie ils t'entourent et te dotent de tout ce qui est bon ».

\par 15 Et l'autre fille qui s'appelait Kassiah dit : « Est-ce la possession dont tu dis qu'elle est meilleure que celle de nos frères. Que pouvons-nous maintenant vivre de cela ? »

\par 16 Et leur père leur dit : « Non seulement vous avez ici de quoi vivre, mais cela vous amène dans un monde meilleur où vivre, dans les cieux.

\par 17 Ou ne connaissez-vous pas, mes enfants, la valeur de ces choses ici. Écoutez donc ! Lorsque le Seigneur m'eut jugé digne d'avoir compassion de moi et d'ôter de mon corps les fléaux et les vers, Il m'appela et me remit ces trois ficelles.

\par 18 Et Il m'a dit : « Lève-toi et ceins tes reins comme un homme, je te l'exigerai et tu me le déclareras ».

\par 19 Et je les ai pris et je les ai ceints autour de mes reins, et aussitôt les vers ont quitté mon corps, ainsi que les plaies, et tout mon corps a pris une nouvelle force par le Seigneur, et ainsi j'ai continué, comme si j'avais jamais souffert.

\par 20 Mais aussi dans mon cœur j'ai oublié les douleurs. Alors le Seigneur m'a parlé dans sa grande puissance et m'a montré tout ce qui était et ce qui sera.

\par 21 Maintenant donc, mes enfants, en gardant cela, vous n'aurez pas l'ennemi complotant contre vous ni de [mauvaises] intentions dans votre esprit, car c'est un charme (Phylacterion) du Seigneur.

\par 22 Levez-vous donc et ceignez-les autour de vous avant que je meure, afin que vous puissiez voir les anges venir à mon adieu afin que vous puissiez contempler avec émerveillement les puissances de Dieu ».

\par 23 Alors se leva celle dont le nom était Day (Yemima) et se ceignit ; et aussitôt elle quitta son corps, comme son père l'avait dit, et elle revêtit un autre cœur, comme si elle ne se souciait jamais des choses terrestres.

\par 24 Et elle chantait des hymnes angéliques avec la voix des anges, et elle chantait la louange angélique de Dieu en dansant.

\par 25 Alors l'autre fille, nommée Kassia, mit la ceinture, et son cœur fut transformé, de sorte qu'elle ne désira plus les choses du monde.

\par 26 Et sa bouche a pris le dialecte des dirigeants célestes (Archontes) et elle a chanté la donologie du travail du Haut Lieu et si quelqu'un souhaite connaître le travail des cieux, il peut avoir un aperçu des hymnes de Kassia. .

\par 27 Alors l'autre fille, nommée Corne d'Amalthée (Keren Happukh), se ceignit et sa bouche parla dans la langue de ceux d'en haut ; car son cœur fut transformé, élevé au-dessus des choses du monde.

\par 28 Elle parlait dans le dialecte des Chérubins, chantant les louanges du Souverain des puissances (vertus) cosmiques et exaltant leur (Sa) gloire.

\par 29 Et celui qui désire suivre les vestiges de la « Gloire du Père » les trouvera écrits dans les Prières de la Corne d'Amalthée.

\chapitre{12}

\par 1 Après que ces trois-là eurent fini de chanter des hymnes. Est-ce que moi, Nahor (Néros), frère de Job, je me suis assis à côté de lui, alors qu'il se couchait.

\par 2 Et j'entendis les (grandes) choses merveilleuses des trois filles de mon frère, l'une se succédant toujours l'autre au milieu d'un silence affreux.

\par 3 Et j'ai écrit ce livre contenant les hymnes à l'exception des hymnes et des signes de la [sainte] Parole, car c'étaient les grandes choses de Dieu.

\par 4 Et Job, malade, se coucha sur son lit, sans douleur ni souffrance, parce que sa douleur ne le saisit pas fortement à cause du charme de la ceinture qu'il s'était enroulée autour de lui.

\par 5 Mais après trois jours, Job vit les saints anges venir chercher son âme, et aussitôt il se leva et prit la cithare et la donna à sa fille Day (Yemima).

\par 6 Et à Kassia il donna un encensoir (avec parfum = Kassia), et à la corne d'Amalthée (= musique) il donna un tambourin afin qu'ils puissent bénir les saints anges qui venaient chercher son âme.

\par 7 Et ils prirent cela, chantèrent, jouèrent du psaltérion et louèrent et glorifièrent Dieu dans le saint dialecte.

\par 8 Et après cela vint Celui qui était assis sur le grand char et qui baisa Job, tandis que ses trois filles regardaient, mais les autres ne le voyaient pas.

\par 9 Et Il prit l'âme de Job et Il s'envola vers le haut, la prenant (l'âme) par le bras et la portant sur le char, et Il partit vers l'Orient.

\par 10 Mais son corps fut transporté au tombeau tandis que les trois filles marchaient en tête, ayant mis leurs ceintures et chantant des hymnes à la louange de Dieu.

\par 11 Alors Nahor (Néréos), son frère, et ses sept fils, avec le reste du peuple et les pauvres, les orphelins et les faibles, lui firent un grand deuil, en disant :

\par 12 « Malheur à nous, car aujourd'hui nous a été enlevée la force des faibles, la lumière des aveugles, le père des orphelins ;

\par 13 Le receveur des étrangers a été ôté du chef des égarés, la couverture des nus. le bouclier des veuves. Qui ne pleurerait pas l’homme de Dieu !

\par 14 Et comme ils pleuraient sous telle et telle forme, ils ne voulaient pas qu'il soit mis dans la tombe.

\par 15 Mais après trois jours, il fut finalement mis au tombeau, comme quelqu'un dans un doux sommeil, et il reçut le nom du bon (beau) qui restera célèbre à travers toutes les générations du monde.

\par 16 Il laissa sept fils et trois filles, et il n'y eut pas de filles sur terre aussi belles que les filles de Job.

\par 17 Le nom de Job était autrefois Jobab, et il fut appelé Job par l'Éternel.

\par 18 Avant sa plaie, il avait vécu quatre-vingt-cinq ans, et après la plaie, il prit la double part de tous ; de là aussi il a doublé son année, qui est de 170 ans. Il vécut ainsi au total 255 ans.

\par 19 Et il vit les fils de ses fils jusqu'à la quatrième génération. Il est écrit qu'il se lèvera avec ceux que le Seigneur réveillera. À notre Seigneur par gloire. Amen.

\end{document}