\begin{document}


\title{Joseph et Aseneth}

\chapter{1}

\par \textit{La confession et la prière d'Asenath, fille du prêtre Pentephres}

\par \textit{Asenath est recherché en mariage par le Fils du Roi et bien d'autres.}

\par 1 La première année d'abondance, le deuxième mois, le cinquième du mois, Pharaon envoya Joseph parcourir tout le pays d'Égypte ; et le quatrième mois de la première année, le dix-huitième jour du mois,

\par 2 Joseph arriva aux confins d'Héliopolis,

\par 3 et il récoltait le blé de ce pays comme le sable de la mer.

\par 4 Et il y avait dans cette ville un homme nommé Pentephres, qui était prêtre d'Héliopolis et satrape de Pharaon, et chef de tous les satrapes et princes de Pharaon ;

\par 5 et cet homme était extrêmement riche et très sage et doux, et il était aussi un conseiller de Pharaon, parce qu'il était prudent plus que tous les princes de Pharaon.

\par 6 Et il eut une fille vierge, nommée Asenath, âgée de dix-huit ans, grande et jolie, et d'une beauté incomparablement plus grande que toutes les vierges de la terre.

\par 7 Or Asenath elle-même ne ressemblait en rien aux vierges, filles des Égyptiens, mais elle ressemblait en toutes choses aux filles des Hébreux,

\par 8 étant grande comme Sarah et belle comme Rébecca et belle comme Rachel ;

\par 9 et la renommée de sa beauté se répandit dans tout ce pays et jusqu'aux extrémités du monde, de sorte que c'est pour cette raison que tous les fils des princes et des satrapes désiraient la courtiser, et même les fils des princes et des satrapes. des rois aussi, tous jeunes et puissants,

\par 10 Et il y eut une grande querelle parmi eux à cause d'elle, et ils essayèrent de se battre les uns contre les autres.

\par 11 Et le fils premier-né de Pharaon entendit aussi parler d'elle, et il continua à supplier son père de la lui donner pour femme.

\par 12 et lui disant : « Donne-moi, père, Asenath, fille de Pentephres, le premier homme d'Héliopolis, pour femme ».

\par 13 Et son père, Pharaon, lui dit : Pourquoi cherches-tu une femme inférieure à toi, alors que tu es roi de tout ce pays ?

\par 14 Non, mais voilà ! la fille de Joacim, roi de Moab, t'est fiancée, et elle est elle-même une reine et extrêmement belle à voir. Prends donc celle-ci pour femme.

\chapter{2}

\par \textit{La tour dans laquelle vit Asenath est décrite.}

\par 1 Mais Asenath méprisait et méprisait tout le monde, étant vantarde et hautaine, et jamais personne ne l'avait vue, car Pentephres avait dans sa maison une tour adjacente, grande et extrêmement haute,

\par 2 et au-dessus de la tour se trouvait un grenier contenant dix chambres.

\par 3 Et la première chambre était grande et très belle, pavée de pierres pourpres, et ses murs étaient recouverts de pierres précieuses et multicolores.

\par 4 et le toit aussi de cette chambre était d'or. Et dans cette chambre, les dieux des Égyptiens, dont il n’y avait pas de nombre, étaient fixés en or et en argent,

\par 5 et tous ceux qu'Asenath adorait, et elle les craignait, et elle leur offrait des sacrifices chaque jour.

\par 6 Et la deuxième chambre contenait également tous les ornements et les coffres d'Asenath,

\par 7 et il y avait de l'or dedans, et beaucoup d'argent et des vêtements tissés d'or en quantité illimitée, et des pierres de choix et de grand prix,

\par 8 et de beaux vêtements de lin, et tous les ornements de sa virginité étaient là.

\par 9 Et la troisième chambre était le magasin d'Asenath, contenant toutes les bonnes choses de la terre.

\par 10 Et les sept chambres restantes étaient occupées par les sept vierges qui servaient à Asenath,

\par 11 Chacun ayant une chambre, car ils étaient du même âge, nés la même nuit qu'Asenath, et elle les aimait beaucoup ; et ils étaient aussi extrêmement beaux comme les étoiles du ciel, et jamais un homme ne parlait avec eux ni avec un enfant mâle.

\par 12 Or, la grande chambre d'Asenath, où sa virginité était entretenue, avait trois fenêtres ;

\par 13 et la première fenêtre était très grande, donnant sur la cour à l'orient ; et le deuxième regardait vers le sud, et le troisième regardait vers la rue.

\par 14 Et un lit d'or se dressait dans la chambre, tournée vers l'orient ;

\par 15 et le lit fut fait d'étoffe pourpre entrelacée d'or, le lit étant tissé d'étoffe écarlate et pourpre et de fin lin.

\par 16 Asenath seul dormait sur ce lit, et jamais homme ni femme ne s'y était assis.

\par 17 Et il y avait aussi une grande cour attenante à la maison tout autour, et un mur très haut autour de la cour, construit en grosses pierres rectangulaires ;

\par 18 et il y avait aussi quatre portes dans le parvis recouvertes de fer, et celles-ci étaient gardées chacune par dix-huit jeunes hommes forts et armés ;

\par 19 Et on planta aussi le long du mur des arbres beaux de toutes espèces et tous portant des fruits, leurs fruits étant mûrs, car c'était la saison de la moisson ;

\par 20 et il y avait aussi une riche source d'eau jaillissant à la droite du même parvis ; et au-dessous de la fontaine il y avait une grande citerne recevant l'eau de cette fontaine, d'où sortait pour ainsi dire une rivière qui traversait le milieu de la cour et qui arrosait tous les arbres de cette cour.

\chapter{3}

\textit{Joseph annonce sa frappe à Pentéphres.}

\par 1 Et il arriva la première année des sept années d'abondance, le quatrième mois, le vingt-huitième jour du mois, que Joseph arriva aux frontières d'Héliopolis pour ramasser le blé de ce district.

\par 2 Et, lorsque Joseph approcha de cette ville, il envoya douze hommes devant lui vers Pentephres, prêtre d'Héliopolis, disant : « J'entrerai chez toi aujourd'hui, car c'est l'heure de midi et du soleil. repas de midi,

\par 3 et il y a une grande chaleur du soleil, et pour que je puisse me rafraîchir sous le toit de ta maison.

\par 4 Et Pentephrès, quand il entendit ces choses, se réjouit d'une très grande joie, et dit :

\par 5 «Béni soit le Seigneur Dieu de Joseph, parce que mon seigneur Joseph m'a jugé digne.» Pentephrès appela le surveillant de sa maison et lui dit :

\par 6 « Dépêchez-vous de préparer ma maison et préparez un grand dîner, car Joseph, le puissant de Dieu, vient à nous aujourd'hui. »

\par 7 Et quand Asenath apprit que son père et sa mère renonçaient à la possession de leur héritage,

\par 8 elle se réjouit beaucoup et dit : « J'irai voir mon père et ma mère, car ils sont sortis de la possession de notre héritage » (car c'était la saison des moissons).

\par 9 Et Asenath se précipita dans sa chambre où reposaient ses robes et revêtit une robe de lin fin, faite d'étoffe cramoisie et entrelacée d'or, et se ceignit d'une ceinture d'or et de bracelets autour de ses mains ; et à ses pieds elle mit des cothurnes d'or,

\par 10 et elle jeta autour de son cou un ornement de grand prix et des pierres précieuses, qui étaient ornées de tous côtés, et sur lesquelles étaient également gravés partout les noms des dieux des Égyptiens, tant sur les bracelets que sur les pierres ;

\par 11 et elle mit aussi une tiare sur sa tête, et attacha un diadème autour de ses tempes et se couvrit la tête d'un manteau.

\chapter{4}

\textit{Pentephres propose de donner Asenath à Joseph en mariage.}

\par 1 Et là-dessus, elle descendit en toute hâte les escaliers de son loft et vint vers son père et sa mère et les embrassa.

\par 2 Et Pentephrès et sa femme se réjouirent de leur fille Asenath avec une très grande joie, car ils la voyaient parée et embellie comme l'épouse de Dieu ;

\par 3 et ils rapportèrent tous les biens qu'ils avaient rapportés de la possession de leur héritage et les donnèrent à leur fille ;

\par 4 et Asenath se réjouissait de toutes les bonnes choses, des fruits de la fin de l'été, des raisins, des dattes, des colombes, des mûres et des figues, parce qu'elles étaient toutes belles et agréables au goût.

\par 5 Et Pentephrès dit à sa fille Asenath : « Enfant. » Et elle dit : « Me voici, mon seigneur. »

\par 6 Et il lui dit : « Assieds-toi entre nous, et je te dirai mes paroles. » Et elle s'assit entre son père et sa mère,

\par 7 et Pentéphrès, son père, lui saisit la main droite avec sa main droite, la baisa tendrement et dit : « Très chère enfant. » Et elle lui dit : « Me voici, mon seigneur père. »

\par 8 Et Pentephrès lui dit : « Voici ! Joseph, le puissant de Dieu, vient à nous aujourd'hui, et cet homme est le chef de tout le pays d'Égypte ; et le roi Pharaon l'établit chef de tout notre pays et roi, et il donna lui-même du blé à tout ce pays et le sauva de la famine à venir ;

\par 9 et ce Joseph est un homme qui adore Dieu, discret et vierge comme tu l'es aujourd'hui, et un homme puissant en sagesse et en connaissance, et l'esprit de Dieu est sur lui et la grâce du Seigneur est en lui. lui.

\par 10 Viens, mon enfant très cher, et je te lui donnerai pour femme, et tu seras pour lui une épouse, et lui-même sera ton époux pour toujours.

\par 11 Et quand Asenath entendit ces paroles de son père, une grande sueur coula sur son visage, et elle devint irritée d'une grande colère,

\par 12 et elle regarda son père de travers et dit : « Pourquoi, mon seigneur père, dis-tu ces paroles ? Veux-tu me livrer comme captif à un étranger et un fugitif et qui a été vendu ?

\par 13 N'est-ce pas le fils du berger du pays de Canaan ? et lui-même a été laissé par lui.

\par 14 N'est-ce pas celui qui coucha avec sa maîtresse, et son seigneur le jeta dans la prison des ténèbres, et Pharaon le fit sortir de la prison dans la mesure où il interpréta son rêve, comme l'interprètent aussi les femmes âgées des Égyptiens ?

\par 15 Non, mais je me marierai avec le fils premier-né du roi, car lui-même est roi de tout le pays.

\par 16 Lorsqu'il entendit ces choses, Pentéphrès eut honte de parler davantage de Joseph à sa fille Asenath, car elle lui répondit avec vantardise et colère.

\chapter{5}

\textit{Joseph arrive à la maison des Pentéphres.}

\par 1 Et voilà ! un jeune homme, serviteur de Pentéphre, entra et lui dit :

\par 2 « Voilà ! Joseph se tient devant les portes de notre tribunal. Et quand Asenath entendit ces paroles, elle s'enfuit devant son père et sa mère et monta dans le grenier, et elle entra dans sa chambre et se tint à la grande fenêtre qui regardait vers l'est pour voir Joseph entrer dans la maison de son père.

\par 3 Et Pentephrès sortit avec sa femme et toute leur parenté et leurs serviteurs à la rencontre de Joseph ;

\par 4 et, lorsque les portes du parvis qui regardaient vers l'orient furent ouvertes, Joseph entra assis sur le deuxième char de Pharaon ;

\par 5 Et il y avait quatre chevaux attelés, blancs comme neige, avec des mors d'or, et le char était tout fait d'or pur.

\par 6 Et Joseph était vêtu d'une tunique blanche et rare, et la robe qui l'entourait était de pourpre, faite de fin lin entrelacé d'or, et une couronne d'or était sur sa tête, et autour de sa couronne étaient douze pierres de choix. , et au-dessus des pierres douze rayons d'or,

\par 7 et dans sa main droite un bâton royal sur lequel était tendu un rameau d'olivier, et il y avait dessus des fruits en abondance.

\par 8 Lorsque donc Joseph fut entré dans le parvis et que les portes en furent fermées,

\par 9 et tous les hommes et toutes les femmes étrangers restèrent hors du parvis, car les gardes des portes se rapprochèrent et fermèrent les portes,

\par 10 Pentephres et sa femme et tous leurs parents, à l'exception de leur fille Asenath, vinrent et ils rendirent hommage à Joseph sur leur face terrestre ;

\par 11 et Joseph descendit de son char et les salua de la main.

\chapter{6}

\textit{Asenath voit Joseph depuis la fenêtre.}

\par 1 Et quand Asenath vit Joseph, son âme fut irritée et son cœur fut brisé,

\par 2 et ses genoux étaient relâchés et tout son corps tremblait et elle eut une grande peur, puis elle gémit et dit dans son cœur : « Hélas moi misérable ! où vais-je maintenant m'en aller, le misérable ? ou où serai-je caché de sa face ? ou comment Joseph, fils de Dieu, me verra-t-il, puisque de ma part j'ai dit du mal de lui ? Hélas moi misérable !

\par 3 Où irai-je et me cacher, parce que lui-même voit chaque cachette et connaît toutes choses, et que rien de caché ne lui échappe à cause de la grande lumière qui est en lui ?

\par 4 Et maintenant que le Dieu de Joseph me fasse grâce, parce que dans l'ignorance j'ai prononcé de mauvaises paroles contre lui.

\par 5 Que dois-je maintenant suivre, moi le misérable ? N'ai-je pas dit : « Joseph vient, le fils du berger du pays de Canaan » ? Maintenant donc, il est venu vers nous sur son char comme le soleil du ciel, et il est entré aujourd'hui dans notre maison, et il y brille comme une lumière sur la terre.

\par 6 Mais je suis insensé et audacieux, parce que je l'ai méprisé et que j'ai dit de mauvaises paroles à son sujet et que je ne savais pas que Joseph est fils de Dieu.

\par 7 Car qui parmi les hommes engendrera jamais une telle beauté, ou quel ventre de femme enfantera une telle lumière ? Je suis malheureux et insensé, parce que j'ai dit de mauvaises paroles à mon père.

\par 8 Maintenant donc, que mon père me donne plutôt à Joseph comme servante et comme esclave, et je serai sa servitude pour toujours.

\chapter{7}

\par \textit{Joseph voit Asenath à la fenêtre.}

\par 1 Et Joseph entra dans la maison de Pentephres et s'assit sur une chaise. Et ils lui lavèrent les pieds et dressèrent une table devant lui à part, car Joseph ne mangeait pas avec les Egyptiens, car cela lui était une abomination.

\par 2 Et Joseph leva les yeux et vit Asenath regarder dehors, et il dit à Pentephres : « Qui est cette femme qui se tient dans le grenier, près de la fenêtre ? Qu'elle s'éloigne de cette maison.

\par 3 Car Joseph craignait, disant : « De peur qu'elle aussi ne m'irrite. » Car toutes les femmes et filles des princes et des satrapes de tout le pays d'Égypte l'ennuyaient pour pouvoir coucher avec lui ;

\par 4 Mais beaucoup de femmes et de filles d'Égyptiens aussi, tous ceux qui virent Joseph, furent affligées à cause de sa beauté ;

\par 5 et les envoyés que les femmes lui envoyèrent avec de l'or, de l'argent et des présents précieux, Joseph les renvoya avec des menaces et des insultes, en disant : « Je ne pécherai pas devant l'Éternel Dieu et devant mon père Israël. »

\par 6 Car Joseph avait Dieu toujours devant ses yeux et se souvenait toujours des injonctions de son père ; car Jacob disait souvent et réprimandait son fils Joseph et tous ses fils : « Gardez-vous, mes enfants, à l'abri d'une femme étrangère afin de ne pas avoir de communion avec elle, car la communion avec elle est ruine et destruction. »

\par 7 C'est pourquoi Joseph dit : « Que cette femme s'éloigne de cette maison. »

\par 8 Et Pentephrès lui dit : « Monseigneur, cette femme que tu as vue debout dans le grenier n'est pas une étrangère, mais c'est notre fille, une qui déteste tout homme, et aucun autre homme ne l'a jamais vue, sauf toi seul. aujourd'hui;

\par 9 et si tu le veux, seigneur, elle viendra te parler, car notre fille est comme ta sœur.

\par 10 Et Joseph se réjouit d'une très grande joie, car Pentephres disait : «C'est une vierge qui déteste tout homme.»

\par 11 Et Joseph dit à Pentephres et à sa femme : « Si c'est votre fille et qu'elle est vierge, qu'elle vienne, car elle est ma sœur, et je l'aime dès aujourd'hui comme ma sœur. »

\chapter{8}


\par \textit{Joseph bénit Asenath.}

\par 1 Alors sa mère monta dans le grenier et amena Asenath à Joseph, et Pentephres lui dit : « Embrasse ton frère, car lui aussi est vierge comme toi aujourd'hui, et il déteste toute femme étrangère comme tu détestes. tout homme étrange.

\par 2 Et Asenath dit à Joseph : « Je te salue, Seigneur, béni du Dieu Très-Haut. » Et Joseph lui dit : « Dieu qui vivifie toutes choses te bénira, jeune fille. »

\par 3 Pentephrès dit alors à sa fille Asenath : « Viens et embrasse ton frère. »

\par 4 Quand Asenath s'approcha alors pour embrasser Joseph, Joseph étendit sa main droite et la posa sur sa poitrine entre ses deux mamelles (car ses mamelles ressortaient déjà comme de belles pommes), et Joseph dit :

\par 5 « Il n'est pas convenable qu'un homme qui adore Dieu, qui bénit de sa bouche le Dieu vivant, qui mange le pain béni de la vie, qui boit la coupe bénie de l'immortalité, et qui est oint de l'onction bénie de l'incorruption, embrasser une femme étrangère, qui bénit de sa bouche des idoles mortes et sourdes et mange sur leur table le pain d'étranglement et boit de leur libation la coupe de tromperie et est ointe de l'onction de destruction ;

\par 6 mais l'homme qui adore Dieu embrassera sa mère et la sœur née de sa mère et la sœur née de sa tribu et la femme qui partage son lit, qui bénissent de leur bouche le Dieu vivant.

\par 7 De même, il n'est pas convenable qu'une femme qui adore Dieu embrasse un homme étranger, car c'est une abomination aux yeux du Seigneur Dieu.

\par 8 Et, quand Asenath entendit ces paroles de Joseph, elle fut très affligée et gémit ; et, comme elle regardait fixement Joseph, les yeux ouverts, ils étaient remplis de larmes.

\par 9 Et Joseph, la voyant pleurer, la plaignit extrêmement, car il était doux et miséricordieux et craignait l'Éternel.

\par 10 Alors il leva sa main droite au-dessus de sa tête et dit :

\par « Seigneur Dieu de mon père Israël, le Dieu Très-Haut et Puissant,
\par qui a vivifié toutes choses et qui t'a appelé des ténèbres à la lumière
\par et de l'erreur à la vérité et de la mort à la vie,
\par bénis aussi cette vierge,

\par 11 et revigore-la, et renouvelle-la par ton esprit saint,
\par et qu'elle mange le pain de ta vie et boive la coupe de ta bénédiction,
\par et compte-la parmi ton peuple, que tu as choisi avant que toutes choses soient faites,
\par et qu'elle entre dans ton repos que tu as préparé pour tes élus,
\par et laisse-la vivre éternellement dans ta vie éternelle.

\chapter{9}

\par \textit{Asenath se retire et Joseph se prépare à partir.}

\par 1 Et Asenath se réjouit de la bénédiction de Joseph avec une très grande joie. Alors elle se hâta et monta seule dans son grenier, et tomba sur son lit dans une infirmité, car il y avait en elle de la joie, de la tristesse et une grande peur ; et une sueur continue coula sur elle lorsqu'elle entendit ces paroles de Joseph, et lorsqu'il lui parla au nom du Dieu Très-Haut.

\par 2 Alors elle pleura de grands et amers pleurs, et elle se détourna, en pénitence, de ses dieux qu'elle avait l'habitude d'adorer, et des idoles qu'elle méprisait, et elle attendit que le soir vienne.

\par 3 Mais Joseph mangea et but ; et il dit à ses serviteurs d'atteler les chevaux à leurs chars et de parcourir tout le pays.

\par 4 Et Pentephres dit à Joseph : « Que mon seigneur passe ici aujourd'hui, et demain tu partiras. »

\par 5 Et Joseph dit : « Non, mais je m'en irai aujourd'hui, car c'est le jour où Dieu a commencé à faire toutes ses choses créées, et le huitième jour, je reviendrai aussi vers vous et je logerai ici. .»

\chapter{10}

\par \textit{Asenath rejette les dieux égyptiens et s'abaisse.}

\par 1 Et lorsque Joseph eut quitté la maison, Pentephres et toute sa parenté s'en allèrent aussi vers leur héritage,

\par 2 et Asenath resta seule avec les sept vierges, apathiques et pleurant jusqu'au coucher du soleil ; et elle ne mangeait ni de pain ni ne buvait d'eau, mais pendant que tout dormait, elle seule était éveillée et pleurait et se frappait fréquemment la poitrine avec sa main.

\par 3 Et après ces choses, Asenath se leva de son lit et descendit tranquillement les escaliers du grenier, et en arrivant à la porte, elle trouva la portière endormie avec ses enfants ;

\par 4 et elle s'empressa d'enlever de la porte le revêtement de cuir du rideau et le remplit de cendres et le porta jusqu'au grenier et le posa sur le sol.

\par 5 Et là-dessus elle ferma solidement la porte et la ferma sur le côté avec le verrou de fer et poussa de grands gémissements accompagnés de beaucoup et de très grands pleurs.

\par 6 Mais la vierge qu'Asenath aimait plus que toutes les vierges, l'ayant entendu gémir, se hâta et vint à la porte après avoir réveillé aussi les autres vierges et la trouva fermée.

\par 7 Et, après avoir entendu les gémissements et les pleurs d'Asenath, elle lui dit, debout dehors : « Qu'y a-t-il, ma maîtresse, et pourquoi es-tu triste ? Et qu'est-ce qui te trouble ?

\par 8 Ouvre-nous et laisse-nous te voir. Et Asenath lui dit, étant enfermée à l'intérieur : « Une douleur grande et douloureuse m'a attaqué la tête, et je me repose dans mon lit, et je ne peux pas me lever et m'ouvrir à toi, à cause de cela je suis infirme de tous mes membres. Allez donc chacune de vous dans sa chambre et dormez, et laissez-moi me taire.

\par 9 Et, quand les vierges furent parties, chacune dans sa chambre, Asenath se leva et ouvrit doucement la porte de sa chambre, et s'en alla dans sa seconde chambre où étaient les coffres de sa parure, et elle ouvrit son coffre et prit une tunique noire et sombre qu'elle a enfilée et qu'elle a pleurée à la mort de son frère aîné.

\par 10 Ayant donc pris cette tunique, elle la porta dans sa chambre, referma bien la porte et ferma le verrou sur le côté.

\par 11 Alors Asenath ôta sa robe royale, et mit la tunique de deuil, et dénoua sa ceinture d'or et se ceignit d'une corde et ôta la tiare, c'est-à-dire la mitre, de sa tête, ainsi que la tiare. le diadème et les chaînes de ses mains et de ses pieds étaient également posés sur le sol.

\par 12 Alors elle prit sa robe de prédilection, et la ceinture d'or, et la mitre, et son diadème, et elle les jeta par la fenêtre qui regardait vers le nord, vers les pauvres.

\par 13 Et alors elle prit tous ses dieux qui étaient dans sa chambre, les dieux d'or et d'argent dont il n'y avait pas de nombre, et les brisa en morceaux, et les jeta par la fenêtre aux pauvres et aux mendiants.

\par 14 Et de nouveau Asenath prit son dîner royal et les animaux gras, et les poissons et la chair des génisses, et tous les sacrifices de ses dieux, et les vases de vin de libation, et les jeta tous par la fenêtre qui regardait vers le nord, comme nourriture pour les chiens.

\par 15 Et après ces choses, elle prit le couvercle de cuir contenant les cendres et les versa sur le sol ;

\par 16 Et là-dessus elle prit un sac et ceint ses reins ; et elle dénoua aussi le filet de ses cheveux et répandit de la cendre sur sa tête. Et elle répandit aussi des cendres sur le sol,

\par 17 et tomba sur les cendres et se frappa constamment la poitrine avec ses mains et pleura toute la nuit en gémissant jusqu'au matin.

\par 18 Et, quand Asenath se leva le matin et vit, et voilà ! les cendres étaient sous elle comme l'argile de ses larmes,

\par 19 Elle retomba la face contre terre sur les cendres jusqu'au coucher du soleil.

\par 20 Asenath fit ainsi pendant sept jours, sans rien goûter.

\chapter{11}

\par \textit{Asenath décide de prier le Dieu des Hébreux.}

\par 1 Et le huitième jour, quand l'aube arriva et que les oiseaux gazouillaient déjà et que les chiens aboyaient contre les passants, Asenath releva un peu la tête du sol et des cendres sur lesquelles elle était assise, pour cela elle elle était extrêmement fatiguée et avait perdu la puissance de ses membres à cause de sa grande humiliation ;

\par 2 Car Asenath était devenue lasse et faible et ses forces faiblissaient, et là-dessus elle se tourna vers le mur, assise sous la fenêtre qui regardait vers l'est ;

\par 3 et elle posa sa tête sur son sein, enroulant les doigts de ses mains sur son genou droit ;

\par 4 et sa bouche fut fermée, et elle ne l'ouvrit pas pendant les sept jours et les sept nuits de son humiliation.

\par 5 Et elle dit dans son cœur, sans ouvrir la bouche : « Que dois-je faire, moi l'humble, ou où irai-je ? Et chez qui trouverai-je désormais refuge ? ou à qui parlerai-je, à la vierge orpheline, désolée, abandonnée de tous et haïe ?

\par 6 Maintenant, tous me haïssent, et parmi eux même mon père et ma mère, car j'ai méprisé les dieux avec dégoût, je les ai éliminés et je les ai donnés aux pauvres pour qu'ils soient détruits par les hommes. Car mon père et ma mère disaient : « Asenath n’est pas notre fille. »

\par 7 Mais tous mes parents aussi en sont venus à me haïr, ainsi que tous les hommes, à cause de cela, j'ai livré leurs dieux à la destruction. Et j'ai haï tous les hommes et tous ceux qui m'ont courtisé, et maintenant, dans cette humiliation, j'ai été haï de tous et ils se réjouissent de ma tribulation.

\par 8 Mais l'Éternel et Dieu du puissant Joseph hait tous ceux qui adorent les idoles, car il est un Dieu jaloux et terrible, comme je l'ai entendu, contre tous ceux qui adorent des dieux étrangers ; c'est pourquoi il m'a haï aussi, parce que j'ai adoré des idoles mortes et sourdes et que je les ai bénies.

\par 9 Mais maintenant j'ai évité leur sacrifice, et ma bouche s'est éloignée de leur table, et je n'ai pas le courage d'invoquer le Seigneur Dieu des cieux, le Très-Haut et le Tout-Puissant du puissant Joseph, pour que ma bouche est pollué par les sacrifices des idoles.

\par 10 Mais j'ai entendu beaucoup dire que le Dieu des Hébreux est un vrai Dieu, et un Dieu vivant, et un Dieu miséricordieux et pitoyable et patient, plein de miséricorde et doux, et qui ne compte pas le péché de un homme humble, et surtout celui qui pèche dans l'ignorance, et qui ne se rend pas coupable d'iniquité au temps de l'affliction d'un homme affligé ;

\par 11 C'est pourquoi moi aussi, l'humble, je serai audacieux et je me tournerai vers lui et chercherai refuge auprès de lui et lui confesserai tous mes péchés et déverserai ma requête devant lui, et il aura pitié de ma misère.

\par 12 Car qui sait s'il verra cette humiliation et la désolation de mon âme et aura pitié de moi, et s'il verra aussi l'orphelinat de ma misère et de ma virginité et me défendra ?

\par 13 car, à ce que j'entends, il est lui-même père d'orphelins, consolateur pour les affligés et secours pour les persécutés. Mais en tout cas, moi aussi, l'humble, je serai audacieux et je crierai vers lui.

\par 14 Alors Asenath se leva du mur où elle était assise, et se redressa sur ses genoux vers l'est et dirigea ses yeux vers le ciel et ouvrit la bouche et dit à Dieu :

\chapter{12}

\par \textit{Prière d'Asenath}

\par 1 La prière et la confession d'Asenath :

\par 2 « Seigneur Dieu des justes, qui a créé les siècles et qui a donné la vie à toutes choses,
\par qui as donné le souffle de vie à toute ta création,
\par qui a mis en lumière les choses invisibles,
\par qui a fait toutes choses et a rendu manifeste les choses qui ne paraissaient pas,

\par 3 qui as élevé les cieux et fondé la terre sur les eaux,
\par qui a fixé les grandes pierres sur l'abîme de l'eau,
\par qui ne seront pas submergés mais qui feront jusqu'à la fin ta volonté,
\par car c'est toi, Seigneur, qui as dit la parole et toutes choses ont été créées, et ta parole, Seigneur, est la vie de toutes tes créatures, vers toi je me réfugie pour me réfugier,

\par 4 Seigneur mon Dieu, désormais vers toi je crierai, Seigneur,
\par et je te confesserai mes péchés, je te déverserai ma requête, Maître,
\par et je te révélerai mes iniquités.

\par 5 Épargne-moi, Seigneur, épargne-moi, car j'ai commis beaucoup de péchés contre toi,
\par J'ai commis l'iniquité et l'impiété,
\par J'ai dit des choses qui ne doivent pas être dites et qui sont mauvaises à tes yeux ;
\par ma bouche, Seigneur, a été souillée par les sacrifices des idoles des Égyptiens,
\par et de la table de leurs dieux :

\par 6 J'ai péché, Seigneur, j'ai péché devant toi, tant par la connaissance que par l'ignorance
\par J'ai commis l'impiété en adorant des idoles mortes et sourdes,
\par et je ne suis pas digne de t'ouvrir la bouche, Seigneur,

\par 7 Moi, la misérable Asenath, fille du prêtre Pentephrès, vierge et reine,
\par qui était autrefois fier et hautain et qui a prospéré plus que tous les hommes dans les richesses de mon père,
\par mais maintenant orphelin, désolé et abandonné de tous les hommes.
\par Vers toi je fuis, Seigneur, et vers toi je présente ma requête,
\par et vers toi je crierai.

\par 8 Délivre-moi de ceux qui me poursuivent, Maître, avant que je sois pris par eux ;
\par car, comme un enfant, craignant quelqu'un, s'enfuit vers son père et sa mère,
\par et son père étendit les mains et le saisit contre sa poitrine,
\par ainsi toi aussi, Seigneur, étends sur moi tes mains pures et terribles comme un père qui aime les enfants,
\par et arrache-moi de la main de l'ennemi suprasensuel.

\par 9 Pour voilà ! le lion ancien, sauvage et cruel me poursuit,
\par car il est le père des dieux des Égyptiens,
\par et les dieux des idolâtres sont ses enfants,
\par et j'en suis venu à les haïr, et je me suis débarrassé d'eux,
\par parce qu'ils sont les enfants d'un lion,
\par et j'ai chassé loin de moi tous les dieux des Égyptiens et je les ai fait disparaître,
\par et le lion, ou leur père le diable, en colère contre moi, cherche à m'engloutir.

\par 10 Mais toi, Seigneur, délivre-moi de ses mains,
\par et je serai délivré de sa bouche,
\par de peur qu'il ne me déchire et ne me jette dans la flamme du feu,
\par et le feu m'a jeté dans la tempête,
\par et la tempête s'est emparée de moi dans les ténèbres et m'a jeté dans les profondeurs de la mer,
\par et la grande bête qui est éternelle m'engloutit,
\par et je péris pour toujours.

\par 11 Délivre-moi, Seigneur, avant que toutes ces choses m'arrivent ;
\par délivre-moi, Maître, le désolé et sans défense,
\par car mon père et ma mère m'ont renié et ont dit :
\par «Asenath n'est pas notre fille»,
\par parce que j'ai brisé leurs dieux et que je les ai détruits,
\par comme les avoir totalement détestés. Et maintenant, je suis orphelin et désolé, et je n'ai d'autre espoir que toi.

\par 12 Seigneur, ni autre refuge que ta miséricorde, toi ami des hommes,
\par parce que tu es le seul père des orphelins, le défenseur des persécutés et le secours des affligés.
\par Ayez pitié de moi. Seigneur, et garde-moi pur et vierge,
\par l'abandonné et l'orphelin, pour cela toi seul.
\par Seigneur, tu es un père doux, bon et gentil.
\par Car quel père est doux et bon comme toi, Seigneur ?
\par Pour voilà ! toutes les maisons de mon père Pentephres
\par qu'il m'a donné en héritage sont pour un temps et disparaissent ;
\par mais les maisons de ton héritage, Seigneur, sont incorruptibles et éternelles.

\chapter{13}

\par \textit{Prière d'Asenath (suite).}

\par 1 « Visite, Seigneur, mon humiliation et aie pitié de mon orphelinat et prends pitié de moi, l'affligé. Pour voilà ! Moi, Maître, j'ai fui tout et j'ai cherché refuge auprès de toi, le seul ami des hommes.

\par 2 Lo ! J'ai quitté toutes les bonnes choses de la terre et j'ai cherché refuge auprès de toi. Seigneur, vêtu de sacs et de cendres, nu et solitaire.

\par 3 Lo ! maintenant j'ai enlevé ma robe royale de fin lin et d'étoffe cramoisie entrelacée d'or et j'ai mis une tunique noire de deuil. Lo ! J'ai détaché ma ceinture d'or, je l'ai jetée loin de moi et je me suis ceint de corde et de sac.

\par 4 Lo ! J'ai jeté mon diadème et ma mitre de ma tête et je me suis saupoudré de cendres,

\par 5 Lo ! le sol de ma chambre, qui était pavé de pierres multicolores et violettes, qui était autrefois humidifié avec des onguents et séché avec des linges brillants, est maintenant humidifié de mes larmes et a été déshonoré en ce qu'il est jonché de cendres.

\par 6 Voici, mon Seigneur, des cendres et de mes larmes, beaucoup d'argile s'est formée dans ma chambre comme sur un large chemin.

\par 7 Voici !, mon Seigneur, mon dîner royal et les viandes que j'ai données aux chiens.

\par 8 Lo ! Moi aussi, Maître, j'ai jeûné sept jours et sept nuits et je n'ai ni mangé de pain ni bu d'eau, et ma bouche est sèche comme une roue et ma langue comme une corne et mes lèvres comme un tesson, et mon visage est rétréci et mes yeux sont rétrécis. ont échoué à force de verser des larmes.

\par 9 Mais toi, Seigneur mon Dieu, délivre-moi de mes nombreuses ignorances, et pardonne-moi de cela, étant vierge et ignorant, je me suis égaré. Lo ! maintenant, tous les dieux que j'adorais auparavant dans l'ignorance, je sais maintenant qu'ils étaient des idoles sourdes et mortes, et je les ai brisés en morceaux et je les ai donnés pour être piétinés par tous les hommes, et les voleurs les ont dépouillés, qui étaient de l'or et de l'argent. , et auprès de toi j'ai cherché refuge. Seigneur Dieu, le seul compatissant et ami des hommes.

\par 10 Pardonne-moi, Seigneur, de ce que j'ai commis de nombreux péchés contre toi par ignorance et que j'ai prononcé des paroles blasphématoires contre mon seigneur Joseph, et que je ne savais pas, misérable, qu'il était ton fils. Seigneur, puisque les méchants poussés par l'envie m'ont dit : Joseph est fils d'un berger du pays de Canaan, et moi, le misérable, je les ai crus et je me suis égaré, et je l'ai méprisé et j'ai dit des choses mauvaises à son sujet. , ne sachant pas qu'il est ton fils.

\par 11 Car qui parmi les hommes a engendré ou engendrera jamais une telle beauté ? ou qui d'autre est tel que lui, sage et puissant comme le tout beau Joseph ? Mais à toi. Seigneur, je l'engage, car pour ma part je l'aime plus que mon âme.

\par 12 Garde-le en sécurité dans la sagesse de ta grâce, et confie-moi à lui comme servante et esclave, afin que je puisse lui laver les pieds et faire son lit et le servir et le servir, et je serai esclave pour lui pour les moments de ma vie.

\chapter{14}

\par \textit{L'archange Michel visite Asenath.}

\par 1 Et, quand Asenath eut cessé de se confesser au Seigneur, voici ! l'étoile du matin s'est également levée du ciel à l'est ;

\par 2 et Asenath le vit et se réjouit et dit : « Le Seigneur Dieu a-t-il alors entendu ma prière ? car cette étoile est une messagère et une annonciatrice de la lumière du grand jour.

\par 3 Et voilà ! près de l’étoile du matin, le ciel se déchira et une grande et ineffable lumière apparut.

\par 4 Et quand elle le vit, Asenath tomba la face contre terre sur les cendres, et aussitôt un homme vint vers elle du ciel, envoyant des rayons de lumière, et se tint au-dessus de sa tête. Et, alors qu'elle était couchée sur son visage, l'ange divin lui dit : « Asenath, lève-toi. »

\par 5 Et elle dit : « Qui est celui qui m'a appelé parce que la porte de ma chambre est fermée et que la tour est haute, et comment alors est-il entré dans ma chambre. »

\par 6 Et il l'appela encore une seconde fois, disant : « Asenath, Asenath. » Et elle dit : « Me voici, Seigneur, dis-moi qui tu es. »

\par 7 Et il dit : « Je suis le capitaine en chef du Seigneur Dieu et le commandant de toute l'armée du Très-Haut : lève-toi et tiens-toi debout, pour que je puisse te dire mes paroles. »

\par 8 Et elle leva son visage et vit, et voilà ! un homme en toutes choses semblable à Joseph, en robe, en couronne et en bâton royal,

\par 9 sauf que son visage était comme l'éclair, et ses yeux comme la lumière du soleil, et les cheveux de sa tête comme la flamme du feu d'une torche allumée, et ses mains et ses pieds comme du fer brillant du feu, car comme des étincelles sortaient de ses mains et de ses pieds.

\par 10 Voyant ces choses, Asenath eut peur et tomba sur sa face, incapable même de se tenir debout, car elle eut très peur et tous ses membres tremblaient.

\par 11 Et l'homme lui dit : « Prends courage, Asenath, et ne crains rien ; mais lève-toi et tiens-toi debout, afin que je puisse te dire mes paroles.

\par 12 Alors Asenath se leva et se leva, et l'ange lui dit :

\par 13 « Va sans obstacle dans ta deuxième chambre et met de côté la tunique noire dont tu es vêtu, et jette le sac de tes reins, et secoue les cendres de ta tête, et lave ton visage et tes mains avec de l'eau pure. et revêts une robe blanche intacte et ceins tes reins de la ceinture lumineuse de la virginité, la double,

\par 14 et reviens vers moi, et je te dirai les paroles qui te sont envoyées de la part du Seigneur.

\par 15 Alors Asenath se précipita et entra dans sa deuxième chambre, où étaient ses coffres parés, et ouvrit son coffre et prit une robe blanche, belle et intacte et l'enfila, après avoir ôté d'abord la robe noire,

\par 16 et elle détacha aussi la corde et le sac de ses reins et se ceignit d'une double ceinture brillante de sa virginité, une ceinture autour de ses reins et une autre ceinture autour de sa poitrine.

\par 17 Et elle secoua aussi les cendres de sa tête et se lava les mains et le visage avec de l'eau pure, et elle prit un manteau très beau et très fin et se voila la tête.

\chapter{15}

\par \textit{Michael dit à Asenath qu'elle sera la femme de Joseph.}

\par 1 Et là-dessus, elle s'approcha du divin capitaine en chef et se tint devant lui, et l'ange du Seigneur lui dit : « Enlève maintenant le manteau de ta tête, car tu es aujourd'hui une vierge pure, et ta tête est comme celui d’un jeune homme.

\par 2 Et Asenath l'enleva de sa tête. Et encore l'ange divin lui dit : « Prends courage, Asenath, la vierge et pure, car voici ! le Seigneur Dieu a entendu toutes les paroles de ta confession et de ta prière, et il a vu aussi l'humiliation et l'affliction du sept jours d'abstinence, car de tes larmes beaucoup d'argile s'est formée devant ton visage sur ces cendres.

\par 3 En conséquence, prends courage, Asenath, la vierge et pure, car voici ! ton nom a été écrit dans le livre de vie et ne sera pas effacé à jamais ;

\par 4 mais à partir de ce jour tu seras renouvelé, remodelé et vivifié, et tu mangeras le pain béni de la vie et tu boiras une coupe remplie d'immortalité et tu seras oint de l'onction bénie de l'incorruption.

\par 5 Prends courage, Asenath, la vierge et pure, voici ! le Seigneur Dieu t'a donné aujourd'hui à Joseph pour épouse, et lui-même sera ton époux pour toujours.

\par 6 Et désormais tu ne seras plus appelé Asenath, mais ton nom sera Ville de Refuge, car en toi beaucoup de nations chercheront refuge et logeront sous tes ailes, et beaucoup de nations trouveront refuge par tes moyens, et sur tes murs, ceux qui s'attachent au Dieu Très-Haut par la pénitence seront gardés en sécurité ;

\par 7 Car la pénitence est fille du Très-Haut, et elle implore elle-même le Dieu Très-Haut pour toi à chaque heure et pour tous ceux qui se repentent, puisqu'il est père de la pénitence,

\par 8 Et elle-même est l'achèvement et la surveillante de toutes les vierges, vous aimant extrêmement et implorant le Très-Haut pour vous à chaque heure, et pour tous ceux qui se repentent, elle fournira un lieu de repos dans les cieux, et elle renouvelle tous ceux qui l'ont fait. repenti. Et la pénitence est extrêmement belle, une vierge pure, douce et douce ; et c'est pourquoi Dieu Très-Haut l'aime, et tous les anges la vénèrent, et je l'aime extrêmement, car elle aussi est ma sœur, et comme elle vous aime, les vierges, je vous aime aussi.

\par 9 Et voilà ! pour ma part, je vais vers Joseph et je lui dirai toutes ces paroles à ton sujet, et il viendra à toi aujourd'hui et te verra et se réjouira de toi et t'aimera et sera ton époux, et tu seras son épouse bien-aimée pour jamais.

\par 10 En conséquence, écoute-moi, Asenath, et revêts une robe de mariée, l'ancienne et première robe qui est encore déposée dans ta chambre depuis les temps anciens, et mets aussi sur toi tous les ornements de ton choix, et pare-toi comme une bonne épouse. et prépare-toi à le rencontrer ;

\par 11 pour voilà ! lui-même vient à toi aujourd'hui, te verra et se réjouira.

\par 12 Et, lorsque l'ange du Seigneur sous forme d'homme eut fini de dire ces paroles à Asenath, elle se réjouit d'une grande joie de tout ce qu'il avait dit,

\par 13 et tomba la face contre terre, et se prosterna devant ses pieds et lui dit : « Béni soit l'Éternel, ton Dieu, qui t'a envoyé pour me délivrer des ténèbres et pour me faire sortir des fondements de l'abîme lui-même. dans la lumière, et ton nom est béni pour toujours. Si donc j'ai trouvé grâce, mon seigneur, à tes yeux et si je sais que tu exécuteras toutes les paroles que tu m'as dites afin qu'elles s'accomplissent, que ta servante te parle. Et l'ange lui dit : «Parle».

\par 14 Et elle dit : « Je te prie, Seigneur, assieds-toi un peu de temps sur ce lit, car ce lit est pur et sans souillure, car aucun homme ni aucune autre femme ne s'est jamais assis dessus, et je mettrai devant toi un table et du pain, et tu mangeras, et je t'apporterai aussi du vin vieux et bon, dont l'odeur atteindra jusqu'au ciel, et tu en boiras et ensuite tu partiras sur ton chemin. Et il lui dit : « Dépêche-toi et apporte-le vite. »

\chapter{16}

\par \textit{Asenath trouve un nid d'abeille dans son entrepôt.}

\par 1 Et Asenath se hâta et dressa devant lui une table vide ; et, comme elle commençait à aller chercher du pain, l'ange divin lui dit : « Apporte-moi aussi un rayon de miel. » Et elle resta immobile, perplexe et affligée de ce qu'elle n'avait pas un rayon d'abeille dans son entrepôt. Et l'ange divin lui dit : «Pourquoi restes-tu tranquille ?»

\par 2 Et elle dit : « Monseigneur, j'enverrai un garçon dans la banlieue, parce que la possession de notre héritage est proche, et il viendra et en ramènera rapidement un, et je le mettrai devant toi.

\par 3 L'ange divin lui dit : « Entre dans ton magasin et tu trouveras un rayon d'abeille posé sur la table ; prends-le et amène-le ici. Et elle dit : « Seigneur, il n’y a pas de rayon d’abeille dans mon entrepôt. » Et il dit : « Va et tu trouveras. »

\par 4 Et Asenath entra dans son magasin et trouva un rayon de miel posé sur la table ; et le rayon était grand et blanc comme la neige et plein de miel, et ce miel était comme la rosée du ciel, et son odeur était comme l'odeur de la vie. Alors Asenath s'interrogea et dit en elle-même : « Ce peigne vient-il de la bouche de cet homme lui-même ?

\par 5 Et Asenath prit ce peigne, l'apporta et le posa sur la table, et l'ange lui dit : « Pourquoi as-tu dit : 'Il n'y a pas de rayon de miel dans ma maison', et voilà ! tu me l'as apporté ?

\par 6 Et elle dit : « Seigneur, je n'ai jamais mis de nid d'abeilles dans ma maison, mais comme tu l'as dit, il a été fait. Est-ce que cela est sorti de ta bouche ? car son odeur est comme l’odeur d’un onguent.

\par 7 Et l'homme sourit de la compréhension de la femme. Alors il l'appela à lui et, quand elle arriva, il étendit sa main droite et lui saisit la tête, et, quand il secoua la tête avec sa main droite, Asenath craignit beaucoup la main de l'ange, car des étincelles sortaient de ses mains étaient à la manière d'un fer chauffé au rouge, et c'est pourquoi elle regardait tout le temps avec beaucoup de crainte et de tremblement la main de l'ange.

\par 8 Et il sourit et dit : « Tu es béni, Asenath, parce que les mystères ineffables de Dieu t'ont été révélés ; et bienheureux tous ceux qui s'attachent au Seigneur Dieu dans la pénitence, parce qu'ils mangeront de ce rayon, car ce rayon est l'esprit de vie, et cela les abeilles du paradis des délices l'ont fait de la rosée des roses de la vie. qui sont dans le paradis de Dieu et de toutes les fleurs, et qui en mangent les anges et tous les élus de Dieu et tous les fils du Très-Haut, et quiconque en mangera ne mourra pas pour toujours.

\par 9 Alors l'ange divin étendit sa main droite et prit un petit morceau du peigne et mangea, et de sa propre main plaça ce qui restait dans la bouche d'Asenath et lui dit : « Mange », et elle mangea. Et l'ange lui dit : « Voilà ! maintenant tu as mangé le pain de vie et tu as bu la coupe de l'immortalité et tu as été oint de l'onction d'incorruption ;

\par 10 voilà ! maintenant, aujourd'hui, ta chair produit des fleurs de vie de la fontaine du Très-Haut, et tes os seront engraissés comme les cèdres du paradis des délices de Dieu et des puissances infatigables te soutiendront ;

\par 11 C'est pourquoi ta jeunesse ne verra pas la vieillesse, et ta beauté ne disparaîtra pas pour toujours, mais tu seras comme la ville mère fortifiée de tous.

\par 12 Et l'ange incita le rayon, et de nombreuses abeilles surgirent des cellules de ce rayon, et les cellules étaient innombrables, des dizaines de milliers de dizaines de milliers et des milliers de milliers.

\par 13 Et les abeilles aussi étaient blanches comme la neige, et leurs ailes étaient comme une étoffe pourpre, pourpre et cramoisi ; et ils avaient aussi des piqûres aiguës et ne blessaient personne.

\par 14 Alors toutes ces abeilles encerclèrent Asenath des pieds à la tête, et d'autres grandes abeilles comme leurs reines sortirent des cellules, et elles tournèrent autour de son visage et sur ses lèvres, et firent un peigne sur sa bouche et sur ses lèvres comme le peigne posé devant l'ange ; et toutes ces abeilles mangeaient du rayon qui était sur la bouche d'Asenath.

\par 15 Et l'ange dit aux abeilles : « Allez maintenant chez vous. »

\par 16 Alors toutes les abeilles se levèrent, s'envolèrent et s'en allèrent au ciel ; mais tous ceux qui voulaient blesser Asenath tombèrent tous sur terre et moururent. Et là-dessus l'ange étendit son bâton sur les abeilles mortes

\par 17 et il leur dit : « Levez-vous et partez, vous aussi, chez vous. » Alors toutes les abeilles mortes se levèrent et s'en allèrent dans la cour qui jouxtait la maison d'Asenath et s'installèrent sur les arbres fruitiers.

\chapter{17}

\par \textit{Michael s'en va.}

\par 1 Et l'ange dit à Asenath : « As-tu vu cette chose ? Et elle dit : « Oui, mon seigneur, j'ai vu toutes ces choses. »

\par 2 L'ange divin lui dit : « Telles seront toutes mes paroles, autant que je t'en ai parlé aujourd'hui. »

\par 3 Alors l'ange du Seigneur étendit pour la troisième fois sa main droite et toucha le côté du peigne, et aussitôt un feu sortit de la table et dévora le peigne, mais il ne blessa pas du tout la table.

\par 4 Et, quand beaucoup de parfum sortit de la combustion du peigne et remplit la chambre, Asenath dit à l'ange divin : « Seigneur, j'ai sept vierges qui ont été élevées avec moi dès ma jeunesse et qui sont nées sur un seul enfant. nuit avec moi, qui me servent, et je les aime toutes comme mes sœurs. Je les appellerai et tu les béniras aussi, comme tu me bénis.

\par 5 Et l'ange lui dit : «Appelle-les.» Alors Asenath appela les sept vierges et les plaça devant l'ange, et l'ange leur dit : « Le Seigneur Dieu Très-Haut vous bénira, et vous serez [piliers] de refuge pour sept villes, et tous les élus de cette ville. qui habitent ensemble [sur vous] se reposeront pour toujours.

\par 6 Et après ces choses, l'ange « divin » dit à Asenath : « Enlève cette table. » Et, quand Asenath se tourna pour enlever la table, aussitôt il se détourna de ses yeux, et Asenath vit comme un char avec quatre chevaux qui se dirigeaient vers l'est vers le ciel, et le char était comme une flamme de feu, et les chevaux comme un éclair. , et l'ange se tenait au-dessus de ce char.

\par 7 Alors Asenath dit : « Je suis stupide et insensé, moi l'humble, car j'ai dit qu'un homme est entré du ciel dans ma chambre ! je ne savais pas que Dieu y était entré; et voilà ! maintenant il retourne au ciel à sa place. Et elle dit en elle-même : « Aie pitié, Seigneur, de ta servante, et épargne ta servante, car, pour ma part, j'ai, par ignorance, dit devant toi des choses téméraires. »

\chapter{18}

\par \textit{Le visage d'Asenath est transformé.}

\par 1 Et, pendant qu'Asenath se disait encore ces paroles, voici ! un jeune homme, l'un des serviteurs de Joseph, disant : « Joseph, l'homme puissant de Dieu, vient vers vous aujourd'hui. »

\par 2 Et aussitôt Asenath appela le surveillant de sa maison et lui dit : « Dépêche-toi, prépare ma maison et prépare un bon dîner, car Joseph, le puissant homme de Dieu, vient chez nous aujourd'hui.

\par 3 Et le surveillant de la maison, quand il la vit (car son visage avait rétréci à cause des sept jours d'affliction, de pleurs et d'abstinence), fut attristé et pleura ; Il lui prit la main droite, la baisa tendrement et dit : « Qu'as-tu, ma dame, pour que ton visage soit ainsi rétréci ? Et elle dit : « J'ai eu une grande douleur à la tête et le sommeil a disparu de mes yeux. » Alors le surveillant de la maison s'en alla et prépara la maison et le dîner.

\par 4 Et Asenath se souvint des paroles de l'ange et de ses injonctions, et se hâta d'entrer dans sa deuxième chambre, où se trouvaient les coffres de sa parure, et ouvrit son grand coffre et sortit sa première robe comme un éclair pour la voir et la mettre ; et elle se ceignit aussi d'une ceinture éclatante et royale, qui était d'or et de pierres précieuses,

\par 5 Elle mit à ses mains des bracelets d'or, et à ses pieds des cothurnes d'or, et un ornement précieux autour de son cou, et une couronne d'or autour de sa tête ; et sur la couronne, comme sur son devant, il y avait une grande pierre de saphir, et autour de la grande pierre six pierres de grand prix, et avec un manteau très merveilleux, elle se voilait la tête. Et quand Asenath se souvint des paroles du surveillant de sa maison, qui lui avait dit que son visage avait rétréci, elle fut extrêmement triste, et gémit et dit : « Malheur à moi, la petite, puisque mon visage est rétréci. Joseph me verra ainsi et je serai méprisé par lui.

\par 6 Et elle dit à sa servante : Apportez-moi de l'eau pure de la fontaine. Et, après l'avoir apporté, elle le versa dans le bassin, et, se penchant pour se laver le visage, elle vit son propre visage brillant comme le soleil, et ses yeux comme l'étoile du matin quand elle se lève, et ses joues. .comme une étoile du ciel, et ses lèvres comme des roses rouges, les cheveux de sa tête étaient comme la vigne qui fleurit parmi ses fruits dans le paradis de Dieu, son cou comme un cyprès tout bigarré. Et Asenath, quand elle vit ces choses, fut étonnée en elle-même de ce spectacle et se réjouit d'une très grande joie et ne se lava pas le visage, car elle dit : « De peur que je ne lave cette grande et jolie beauté. »

\par 7 Le surveillant de sa maison revint alors pour lui dire : « Tout ce que tu as ordonné a été fait » ; et, lorsqu'il la vit, il eut une grande crainte et fut saisi d'un tremblement prolongé pendant un long moment, et il tomba à ses pieds et commença à dire : « Qu'est-ce que c'est, ma maîtresse ? Quelle est cette beauté qui t'entoure, qui est grande et merveilleuse ? Le Seigneur Dieu du Ciel t'a-t-il choisi comme épouse pour son fils Joseph ?

\chapter{19}

\par \textit{Joseph revient et est reçu par Asenath.}

\par 1 Et, pendant qu'ils parlaient encore ces choses, un garçon vint dire à Asenath : « Voici ! Joseph se tient devant les portes de notre tribunal. Alors Asenath se hâta de descendre les escaliers de son grenier avec les sept vierges pour rencontrer Joseph et se tint sous le porche de sa maison.

\par 2 Et Joseph étant entré dans le parvis, les portes furent fermées et tous les étrangers restèrent dehors. Et Asenath sortit du porche pour rencontrer Joseph, et quand il la vit, il s'émerveilla de sa beauté et lui dit : « Qui es-tu, demoiselle ? Dis-le-moi vite.

\par 3 Et elle lui dit : « Moi, Seigneur, je suis ta servante Asenath ; J'ai rejeté toutes les idoles loin de moi et elles ont péri. Et un homme est venu du ciel vers moi aujourd'hui et m'a donné du pain de vie et j'en ai mangé et j'ai bu une coupe bénie, et il m'a dit : Je t'ai donnée pour épouse à Joseph, et lui-même sera ton époux pour toujours ; et ton nom ne sera pas appelé Asenath, mais on l'appellera « Ville de refuge », et l'Éternel Dieu régnera sur de nombreuses nations, et par toi ils chercheront refuge auprès du Dieu Très-Haut. Et l’homme dit : « J’irai aussi vers Joseph pour lui dire à ses oreilles ces paroles à ton sujet. Et maintenant tu sais, Seigneur, si cet homme est venu vers toi et s'il t'a parlé de moi.

\par 4 Alors Joseph dit à Asenath : « Tu es bénie, femme, du Dieu Très-Haut, et ton nom est béni pour toujours, car le Seigneur Dieu a posé les fondements de tes murs, et les fils du Dieu vivant habite dans ta ville de refuge, et le Seigneur Dieu régnera sur eux pour toujours. Car cet homme est venu aujourd’hui du ciel vers moi et m’a dit ces paroles à ton sujet. Et maintenant, viens vers moi, vierge et pure, et pourquoi te tiens-tu loin ?

\par 5 Alors Joseph étendit les mains et embrassa Asenath et Asenath Joseph, et ils s'embrassèrent longuement, et tous deux revivirent dans leur esprit. Et Joseph embrassa Asenath et lui donna l'esprit de vie, puis la deuxième fois il lui donna l'esprit de sagesse, et la troisième fois il l'embrassa tendrement et lui donna l'esprit de vérité.

\chapter{20}

\par \textit{Pentephres revient et souhaite fiancer Asenath à Joseph, mais Joseph décide de demander sa main à Pharaon.}

\par 1 Et, après qu'ils eurent longtemps serré les uns les autres et entrelacé les chaînes de leurs mains, Asenath dit à Joseph : « Viens ici, Seigneur, et entre dans notre maison, car de ma part j'ai préparé notre maison. et un excellent dîner.

\par 2 Et elle lui saisit la main droite, et le conduisit dans sa maison, et l'assit sur la chaise de Pentephres, son père ; et elle apporta de l'eau pour lui laver les pieds. Et Joseph dit : « Qu’une des vierges vienne me laver les pieds. »

\par 3 Et Asenath lui dit : « Non, seigneur, car désormais tu es mon seigneur et je suis ta servante. Et pourquoi demandes-tu qu’une autre vierge te lave les pieds ? car tes pieds sont mes pieds, et tes mains mes mains, et ton âme mon âme, et aucun autre ne te lavera les pieds. Et elle le contraignit et lava son corps.

\par 4 Alors Joseph saisit sa main droite et la baisa tendrement et Asenath lui baisa tendrement la tête, et là-dessus il la fit asseoir à sa droite.

\par 5 Son père et sa mère et toute sa famille revinrent alors de la possession de leur héritage, et ils la virent assise avec Joseph et vêtue d'un vêtement de noces. Et ils s'émerveillèrent de sa beauté, se réjouirent et glorifièrent Dieu qui vivifie les morts. Et après cela, ils mangèrent et burent ;

\par 6 Et, tous s'étant réjouis, Pentephrès dit à Joseph : « Demain, j'appellerai tous les princes et satrapes de tout le pays d'Égypte, et je te ferai des noces, et tu prendras pour femme ma fille Asenath. »

\par 7 Mais Joseph dit : « Je vais demain trouver le roi Pharaon, car lui-même est mon père et m'a établi chef sur tout ce pays, et je lui parlerai d'Asenath, et il me la donnera pour femme. .» Et Pentephrès dit à Hini : « Va en paix. »

\chapter{21}

\par \textit{Le mariage de Joseph et Asenath.}

\par 1 Et Joseph resta ce jour-là avec Pentephres, et il n'entra pas à Asenath, car il avait l'habitude de dire : « Il n'est pas convenable qu'un homme qui adore Dieu couche avec sa femme avant son mariage. » Et Joseph se leva de bonne heure et partit vers Pharaon et lui dit : « Donne-moi pour femme Asenath, fille de Pentephres, prêtre d'Héliopolis. » Et Pharaon se réjouit d'une grande joie, et il dit à Joseph : « Voici ! celle-ci n'est-elle pas ta fiancée pour épouse de toute éternité ? Qu'elle soit donc ta femme désormais et pour les temps éternels.

\par 2 Alors Pharaon envoya appeler Pentephres, et Pentephres amena Asenath et la plaça devant Pharaon ;

\par 3 et Pharaon quand il la vit s'émerveilla de sa beauté et dit : « Le Seigneur Dieu de Joseph te bénira, mon enfant, et ta beauté restera pour l'éternité, car le Seigneur Dieu de Joseph t'a choisie pour épouse pour lui. : car Joseph est comme le fils du Très-Haut, et tu seras appelé « son épouse désormais et pour toujours ».

\par 4 Et après ces choses, Pharaon prit Joseph et Asenath et mit sur leurs têtes des couronnes d'or, qui étaient dans sa maison depuis les temps anciens et anciens, et Pharaon plaça Asenath à la droite de Joseph. Et Pharaon leur posa les mains sur la tête et dit :

\par 5 « Le Seigneur Dieu Très-Haut vous bénira et vous multipliera, vous magnifiera et vous glorifiera pour les temps éternels. »

\par 6 Alors Pharaon les retourna l'un en face de l'autre et les approcha bouche à bouche, et ils s'embrassèrent. Et Pharaon fit des noces pour Joseph et un grand dîner et beaucoup de boissons pendant sept jours, et il convoqua tous les chefs d'Égypte et tous les rois des nations,

\par 7 ayant fait une proclamation dans le pays d'Égypte, disant : « Tout homme qui travaillera pendant les sept jours des noces de Joseph et d'Asenath mourra sûrement. »

\par 8 Et pendant que les noces avaient lieu et que le dîner était terminé, Joseph entra à Asenath, et Asenath conçut par Joseph et enfanta Manassé et Ephraïm son frère dans la maison de Joseph.

\chapter{22}

\par \textit{Asenath est présenté à Jacob.}


\par 1 Et, lorsque les sept années d'abondance furent passées, les sept années de famine commencèrent à venir.

\par 2 Et lorsque Jacob entendit parler de Joseph, son fils, il vint en Égypte avec toute sa parenté la deuxième année de la famine, le deuxième mois, le vingt et un du mois, et s'établit à Guéshem.

\par 3 Et Asenath dit à Joseph : « J'irai voir ton père, car ton père Israël est comme mon père et Dieu. » Et Joseph lui dit : « Tu viendras avec moi voir mon père. »

\par 4 Et Joseph et Asenath vinrent vers Jacob dans le pays de Guéshem, et les frères de Joseph les rencontrèrent et leur rendirent hommage sur leur face sur la terre. Alors tous deux entrèrent chez Jacob ; et Jacob était assis sur son lit, et lui-même était un vieil homme d'une vieillesse vigoureuse. Et, quand Asenath le vit, elle fut émerveillée par sa beauté, car Jacob était extrêmement beau à voir et sa vieillesse comme la jeunesse d'un homme beau, et toute sa tête était blanche comme la neige, et les cheveux de sa tête étaient tous serré et extrêmement épais, et sa barbe blanche atteignant sa poitrine, ses yeux joyeux et brillants, ses tendons et ses épaules et ses bras comme ceux d'un ange, ses cuisses et ses mollets et ses pieds comme ceux d'un géant. Alors Asenath, quand elle le vit ainsi, fut étonnée et tomba et lui rendit hommage face contre terre. Et Jacob dit à Joseph : « Est-ce ma belle-fille, ta femme ? Bénie sera-t-elle du Dieu Très-Haut.

\par 5 Alors Jacob appela Asenath, la bénit et l'embrassa tendrement ; Et Asnath étendit les mains, saisit le cou de Jacob, s'accrocha à son cou et l'embrassa tendrement.

\par 6 Et après ces choses, ils mangèrent et burent.

\par 7 Et là-dessus Joseph et Asenath rentrèrent dans leur maison ; et Siméon et Lévi, les fils de Léa, les conduisirent seuls, mais les fils de Balla et Zelpha, les servantes de Léa et de Rachel, ne se joignirent pas à leur conduite, car ils les enviaient et les détestaient. Et Lévi était à la droite d'Asenath et Siméon à sa gauche.

\par 8 Et Asenath saisit la main de Lévi, parce qu'elle l'aimait extrêmement plus que tous les frères de Joseph et comme un prophète et un adorateur de Dieu et qui craignait l'Éternel. Car il était un homme intelligent et un prophète du Très-Haut, et lui-même vit des lettres écrites dans le ciel, les lut et les révéla en secret à Asenath ; car Lévi lui-même aimait beaucoup Asenath et voyait le lieu de son repos au plus haut des lieux.

\chapter{23}

\par \textit{Le Fils de Pharaon tente d'inciter Siméon et Lévi à tuer Joseph.}


\par 1 Et il arriva que, comme Joseph et Asenath passaient, alors qu'ils allaient vers Jacob, le fils premier-né de Pharaon les vit du mur, et,

\par 2 Lorsqu'il vit Asenath, il devint fou d'elle à cause de sa beauté incomparable. Alors le fils de Pharaon envoya des messagers et appela Siméon et Lévi ; et, quand ils arrivèrent et se tinrent devant lui,

\par 3 Le fils premier-né de Pharaon leur dit : « Pour ma part, je sais que vous êtes aujourd'hui des hommes vaillants entre tous les hommes de la terre, et que c'est par vos mains droites que la ville des Sicimites a été renversée, et que par vos deux mains Avec les épées, 30 000 guerriers furent abattus.

\par 4 Aujourd'hui, je vous prendrai pour mes compagnons, je vous donnerai beaucoup d'or et d'argent, des serviteurs, des servantes, des maisons et de grands héritages, et vous lutterez pour moi, et vous me ferez du bien ; C'est pour cela que j'ai reçu un grand mécontentement de la part de ton frère Joseph, puisqu'il a lui-même pris Asenath pour femme, et que cette femme m'était fiancée depuis longtemps.

\par 5 Et maintenant venez avec moi, et je combattrai Joseph pour le tuer avec mon épée, et je prendrai Asenath pour femme, et vous serez pour moi comme des frères et des amis fidèles.

\par 6 Mais si vous n’écoutez pas mes paroles, je vous tuerai avec mon épée. Et après avoir dit ces choses, il sortit son épée et la leur montra.

\par 7 Et Siméon était un homme audacieux et audacieux, et il pensa mettre sa main droite sur le pommeau de son épée et la retirer du fourreau et frapper le fils de Pharaon pour avoir dit des paroles dures envers eux.

\par 8 Lévi vit alors la pensée de son cœur, parce qu'il était prophète, et il foula de son pied le pied droit de Siméon et le pressa, afin qu'il cesse sa colère.

\par 9 Et Lévi disait doucement à Siméon : « Pourquoi es-tu irrité contre cet homme ? Nous sommes des hommes qui adorent Dieu et il n’est pas convenable pour nous de rendre le mal pour le mal. »

\par 10 Alors Lévi dit ouvertement au fils de Pharaon avec douceur de cœur : « Pourquoi notre seigneur dit-il ces paroles ? Nous sommes des hommes qui adorent Dieu, et notre père est un ami du Dieu Très-Haut, et notre frère est comme un fils de Dieu. Et comment ferons-nous cette mauvaise chose, pour pécher aux yeux de notre Dieu et de notre père Israël et aux yeux de notre frère Joseph ? Et maintenant, écoutez mes paroles. Il n'est pas convenable qu'un homme qui adore Dieu fasse du mal à quelqu'un de quelque manière que ce soit ; et si quelqu'un veut blesser un homme qui adore Dieu, cet homme qui adore Dieu ne se vengera pas de lui, car il n'a pas d'épée dans ses mains.

\par 11 Et garde-toi de dire plus ces paroles à propos de notre frère Joseph.

\par 12 Mais si tu continues dans ton mauvais conseil, voilà ! nos épées sont tirées contre toi.

\par 13 Alors Siméon et Lévi tirèrent leurs épées de leur fourreau et dirent : « Vois-tu maintenant ces épées ? Avec ces deux épées, l'Éternel a puni les Sicimites, par lesquels ils ont insulté les enfants d'Israël par l'intermédiaire de notre sœur Dina, que Sychem, fils de Hamor, avait souillée.

\par 14 Et le fils de Pharaon, quand il vit les épées tirées, eut une grande crainte et trembla sur tout son corps, car elles brillaient comme une flamme de feu, et ses yeux devinrent obscurs, et il tomba la face contre terre sous leurs armes. pieds.

\par 15 Alors Lévi étendit sa main droite et le saisit, en disant : « Lève-toi et ne crains rien, mais prends garde de ne plus prononcer de mauvaises paroles concernant notre frère Joseph. »

\par 16 Et ainsi Siméon et Lévi sortirent devant sa face.

\chapter{24}

\par \textit{Le Fils de Pharaon conspire avec Dan et Gad pour tuer Joseph et s'emparer d'Asenath.}


\par 1 Le fils de Pharaon continua alors à être plein de crainte et de chagrin à cause de sa crainte des frères de Joseph, et de nouveau il devint extrêmement fou à cause de la beauté d'Asenath et fut grandement affligé.

\par 2 Alors ses serviteurs lui disent à l'oreille : « Voilà ! les fils de Balla et les fils de Zelpha, les servantes de Léa et Rachel, les femmes de Jacob, sont en grande inimitié contre Joseph et Asenath et les haïssent ; cela te sera accordé en toutes choses selon ta volonté.

\par 3 Aussitôt le fils de Pharaon envoya des messagers et les appela, et ils vinrent vers lui à la première heure de la nuit, et ils se tinrent en sa présence, et il leur dit : « J’ai appris de beaucoup que vous êtes des hommes puissants. »

\par 4 Et Dan et Gad, les frères aînés, lui dirent : « Que mon seigneur dise maintenant à ses serviteurs ce qu'il veut, afin que tes serviteurs entendent et que nous fassions selon ta volonté. »

\par 5 Alors le fils de Pharaon se réjouit d'une très grande joie et dit à ses serviteurs : « Éloignez-vous de moi pour un court instant, car j'ai une conversation secrète à tenir avec ces hommes. » Et ils se sont tous retirés.

\par 6 Alors le fils de Pharaon mentit, et il leur dit : « Voici ! maintenant la bénédiction et la mort sont devant vos visages ; acceptez-vous donc la bénédiction plutôt que la mort,

\par 7 parce que vous êtes des hommes vaillants et que vous ne mourrez pas comme des femmes ; mais soyez courageux et vengez-vous de vos ennemis.

\par 8 Car j'ai entendu Joseph, ton frère, dire à Pharaon, mon père : Dan, Gad, Nephtalim et Aser ne sont pas mes frères, mais les enfants des servantes de mon père.

\par 9 J'attends donc la mort de mon père, et je les exterminerai de la terre et de tous leurs descendants, de peur qu'ils n'héritent avec nous, parce qu'ils sont enfants de servantes.

\par 10 Car ceux-ci m'ont aussi vendu aux Ismaélites, et je leur rendrai selon la méchanceté qu'ils ont commise contre moi ; seul mon père mourra.

\par 11 Et mon père Pharaon le félicita pour ces choses et lui dit : Tu as bien parlé, mon enfant. En conséquence, prends-moi des hommes forts et procède contre eux selon ce qu'ils ont fait contre toi, et je serai pour toi ton aide.'»

\par 12 Et lorsque Dan et Gad entendirent ces choses du fils de Pharaon, ils furent très troublés et extrêmement attristés, et ils lui dirent : « Nous te prions, Seigneur, aide-nous ; Car désormais nous sommes tes esclaves et tes serviteurs et nous mourrons avec toi. Et le fils de Pharaon dit : « Je vous serais à l'aide si vous aussi écoutez mes paroles. » Et ils lui dirent : « Commande-nous ce que tu veux et nous ferons selon ta volonté. »

\par 13 Et le fils de Pharaon leur dit : « Cette nuit, je tuerai mon père Pharaon, car ce Pharaon est comme le père de Joseph et il lui a dit qu'il aiderait contre vous ; et vous tuerez Joseph, et je prendrai Asenath pour femme, et vous serez mes frères et cohéritiers de tous mes biens. Fais seulement cette chose.

\par 14 Et Dan et Gad lui dirent : « Nous sommes aujourd'hui tes serviteurs et nous ferons tout ce que tu nous as commandé. Et nous avons entendu Joseph dire à Asenath : « Va demain en possession de notre héritage, car c'est le temps de la vendange » ; et il envoya six cents hommes vaillants pour lui faire la guerre, ainsi que cinquante précurseurs. Maintenant donc, écoute-nous et nous parlerons à notre seigneur. Et ils lui dirent toutes leurs paroles secrètes.

\par 15 Alors le fils de Pharaon donna aux quatre frères cinq cents hommes chacun et leur donna leurs chefs et leurs chefs.

\par 16 Et Dan et Gad lui dirent : « Nous sommes tes serviteurs aujourd'hui et ferons tout ce que tu nous as commandé, et nous partirons de nuit et nous nous embûcherons dans le ravin et nous nous cacherons. dans le bosquet des roseaux ;

\par 17 et prends avec toi cinquante archers à cheval et pars loin devant nous, et Asenath viendra et tombera entre nos mains, et nous massacrerons les hommes qui sont avec elle,

\par 18 Et elle-même s'enfuira avec son char et tombera entre tes mains et tu lui feras ce que ton âme désire ;

\par 19 et après ces choses, nous tuerons aussi Joseph pendant qu'il pleure Asenath ; de même, nous tuerons ses enfants sous ses yeux.

\par 20 Alors le fils premier-né de Pharaon, quand il entendit ces choses, se réjouit extrêmement, et il les envoya ainsi que deux mille combattants avec eux. Et quand ils arrivèrent au ravin, ils se cachèrent dans le bosquet de roseaux, et se divisèrent en quatre compagnies, et prirent position de l'autre côté du ravin comme à l'avant, cinq cents hommes de ce côté de la route. et là-dessus, et du côté voisin du ravin également, le reste resta,

\par 21 Et eux aussi se postèrent dans le bosquet des roseaux, cinq cents hommes de chaque côté du chemin ; et entre eux il y avait une route large et large.


\chapter{25}

\par \textit{Le Fils de Pharaon va tuer son Père, mais n'est pas admis. Nephthalim et Asher protestent auprès de Dan et Gad contre la Conspiration.}


\par 1 Alors le fils de Pharaon se leva la même nuit et vint dans la chambre de son père pour le tuer avec l'épée. Les gardes de son père l'empêchèrent alors d'entrer chez son père et lui dirent : « Que commandes-tu, seigneur ?

\par 2 Et le fils de Pharaon leur dit : « Je veux voir mon père, c'est pour cela que je vais récolter la vendange de ma vigne nouvellement plantée. »

\par 3 Et les gardes lui dirent : « Ton père souffre et est resté éveillé toute la nuit et maintenant il se repose, et il nous a dit que personne ne devait entrer chez lui, même si c'était mon fils premier-né. »

\par 4 Et il, ayant entendu ces choses, s'en alla en colère et prit aussitôt des archers à cheval, au nombre de cinquante, et s'en alla devant eux comme Dan et Gad le lui avaient dit.

\par 5 Et les jeunes frères Nephthalim et Asher parlèrent à leurs frères aînés Dan et Gad, en disant : « Pourquoi, de votre côté, faites-vous encore du mal contre votre père Israël et contre votre frère Joseph ? Et Dieu le préserve comme la prunelle de ses yeux. Lo !

\par 6 N'avez-vous pas vendu Joseph une seule fois ? et il est aujourd'hui roi de tout le pays d'Égypte et donneur de nourriture.

\par 7 Maintenant donc, si vous voulez encore faire du mal contre lui, il criera au Très-Haut et il enverra du ciel un feu qui vous dévorera, et les anges de Dieu combattront contre vous.

\par 8 Alors les frères aînés se mirent en colère contre eux et dirent : « Et mourrons-nous comme des femmes ? Loin de là. Et ils sortirent à la rencontre de Joseph et d'Asenath.

\chapter{26}

\par \textit{Les conspirateurs tuent les gardes d'Asenath et elle s'enfuit.}


\par 1 Et Asenath se leva le matin et dit à Joseph : « Je vais prendre possession de notre héritage comme tu l'as dit ; mais mon âme craint extrêmement que tu te sépares de moi.

\par 2 Et Joseph lui dit : « Prends courage et n'aie pas peur, mais va-t'en plutôt joyeux, sans crainte de personne, car l'Éternel est avec toi et lui-même te gardera comme la pomme d'un arbre. œil de tout mal.

\par 3 Et je partirai pour donner de la nourriture et je la donnerai à tous les hommes de la ville, et personne ne périra de faim au pays d'Égypte.

\par 4 Alors Asenath partit, ainsi que Joseph pour son don de nourriture.

\par 5 Et quand Asenath atteignit le lieu du ravin avec les six cents hommes, soudain ceux qui étaient avec le fils de Pharaon sortirent de leur embuscade et joignirent le combat à ceux qui étaient avec Asenath, et les frappèrent tous avec leurs épées. et ils tuèrent tous ses précurseurs,

\par 6 mais Asenath s'enfuit avec son char. Alors Lévi, fils de Léa, connut toutes ces choses par l'esprit comme un prophète et informa ses frères du danger d'Asenath,

\par 7 et aussitôt chacun d'eux prit son épée sur sa cuisse et ses boucliers sur ses bras et les lances dans sa main droite et poursuivirent Asenath avec une grande rapidité.

\par 8 Et, comme Asenath s'enfuyait auparavant, voici ! Le fils de Pharaon vint à sa rencontre, ainsi que cinquante cavaliers avec lui. Asenath, lorsqu'elle le vit, fut prise d'une très grande peur et tremblante, et elle invoqua le nom de l'Éternel, son Dieu.

\chapter{27}

\par \textit{Les hommes avec le Fils de Pharaon et ceux avec Dan et Gad sont tués ; et les quatre Frères fuient vers le Ravin et leurs épées sont arrachées de leurs mains.}


\par 1 Et Benjamin était assis avec elle sur le char du côté droit ; et Benjamin était un robuste garçon d'environ dix-neuf ans,

\par 2 et sur lui se trouvait une beauté et une puissance ineffables comme celles d'un jeune lion, et il était aussi quelqu'un qui craignait Dieu extrêmement.

\par 3 Alors Benjamin sauta du char, et prit une pierre ronde du ravin et remplit sa main et la lança sur le fils de Pharaon et lui frappa la tempe gauche, et le blessa d'une blessure grave, et il tomba de son cheval sur le la terre à moitié morte.

\par 4 Et là-dessus Benjamin, ayant couru sur un rocher, dit au char d'Asenath : « Donne-moi des pierres du ravin. » Et il lui donna cinquante pierres.

\par 5 Et Benjamin lança les pierres et tua les cinquante hommes qui étaient avec le fils de Pharaon, toutes les pierres s'enfonçant dans leurs temples.

\par 6 Alors les fils de Léa, Ruben et Siméon, Lévi et Juda, Issacar et Zabulon, poursuivirent les hommes qui avaient dressé une embuscade contre Asenath et tombèrent sur eux à l'improviste et les tuèrent tous ; et les six hommes tuèrent deux mille soixante-seize hommes.

\par 7 Et les fils de Balla et de Zelpha s'enfuirent devant eux et dirent : « Nous avons péri par la main de nos frères, et le fils de Pharaon est également mort par la main du jeune Benjamin, et tous ceux qui étaient avec lui ont péri par la main de Benjamin. la main du garçon Benjamin. Venez donc, tuons Asenath et Benjamin, et fuyons vers le bosquet de ces roseaux.

\par 8 Et ils arrivèrent contre Asenath, tenant leurs épées nues et couvertes de sang. Et Asenath, quand elle les vit, eut une grande crainte et dit : « Seigneur Dieu, qui m'as vivifié et qui m'as délivré des idoles et de la corruption de la mort, comme tu m'as dit que mon âme vivrait pour toujours, délivre-moi maintenant aussi de ces derniers. des hommes méchants. » Et l'Éternel Dieu entendit la voix d'Asenath, et aussitôt les épées des adversaires tombèrent de leurs mains sur la terre et furent transformées en cendres.

\chapter{28}

\par \textit{Dan et Gad sont épargnés grâce à la supplication d'Asenath.}


\par 1 Et les fils de Balla et de Zelpha, voyant l'étrange miracle qui s'était produit, eurent peur et dirent : « L'Éternel combat contre nous en faveur d'Asenath. »

\par 2 Alors ils tombèrent la face contre terre et rendirent hommage à Asenath et dirent : « Aie pitié de nous, tes serviteurs, car tu es notre maîtresse et notre reine. Nous avons commis de mauvaises actions contre toi et contre notre frère Joseph, mais le Seigneur nous a rendu selon nos œuvres.

\par 3 C'est pourquoi nous, tes serviteurs, te prions, aie pitié de nous, les humbles et les misérables, et délivre-nous des mains de nos frères, car ils se feront vengeurs du mépris qui t'a été fait et leurs épées sont contre nous. Sois donc aimable envers tes serviteurs, maîtresse, devant eux.

\par 4 Et Asenath leur dit : « Prenez courage et n'ayez pas peur de vos frères, car eux-mêmes sont des hommes qui adorent Dieu et craignent l'Éternel ;

\par 5 mais allez dans le bosquet de ces roseaux jusqu'à ce que je les apaise en votre faveur et que je retienne leur colère à cause des grands crimes que vous, de votre part, avez osé commettre contre eux. Mais le Seigneur voit et juge entre moi et vous.

\par 6 Alors Dan et Gad s'enfuirent dans le bosquet des roseaux ; et leurs frères, les fils de Léa, accoururent comme des cerfs en toute hâte contre eux.

\par 7 Et Asenath descendit du char qui lui servait de couverture et leur tendit sa main droite en pleurant,

\par 8 Et eux-mêmes tombèrent, lui rendirent hommage sur la terre et pleurèrent à haute voix ;

\par 9 et ils continuèrent à demander à leurs frères, les fils des servantes, de les faire mourir.

\par 10 Et Asenath leur dit : « Je vous prie, épargnez vos frères, et ne leur rendez pas mal pour mal. Car l'Éternel m'a sauvé d'eux et a brisé leurs poignards et leurs épées de leurs mains, et voilà ! ils ont fondu et ont été réduits en cendres sur la terre, comme la cire d'avant le feu,

\par 11 et cela nous suffit que le Seigneur combatte pour nous contre eux. Epargnez donc vos frères, car ils sont vos frères et le sang de votre père Israël.

\par 12 Et Siméon lui dit : « Pourquoi notre maîtresse dit-elle de bonnes paroles en faveur de ses ennemis ? Non, mais nous allons plutôt les couper membre par membre avec nos épées,

\par 13 parce qu'ils ont conçu des choses mauvaises contre notre frère Joseph et notre père Israël et contre toi, notre maîtresse, aujourd'hui.

\par 14 Alors Asenath étendit sa main droite et toucha la barbe de Siméon et l'embrassa tendrement et dit : « En aucune manière, frère, ne rends à ton prochain mal pour mal, car l'Éternel vengera cet affront. Vous le savez, eux-mêmes sont vos frères et les descendants d'Israël, votre père, et ils ont fui de loin devant vous. Accordez-leur donc le pardon.

\par 15 Alors Lévi s'approcha d'elle et lui baisa tendrement la main droite, car il savait qu'elle voulait sauver les hommes de la colère de leurs frères afin qu'ils ne les tuent pas. Et eux-mêmes étaient à proximité, dans le bosquet de la roselière. Et Lévi, son frère, sachant cela, ne le déclara pas à ses frères, car il craignait que, dans leur colère, ils ne massacrent leurs frères.

\chapter{29}

\par \textit{Le Fils de Pharaon meurt. Pharaon meurt aussi et Joseph lui succède.}

\par 1 Et le fils de Pharaon se leva de terre, se redressa et cracha du sang de sa bouche ; car le sang coulait de sa tempe dans sa bouche.

\par 2 Et Benjamin courut vers lui, prit son épée et la tira du fourreau du fils de Pharaon (car Benjamin ne portait pas d'épée sur sa cuisse) et voulut frapper le fils de Pharaon sur la poitrine.

\par 3 Alors Lévi courut vers lui, lui prit la main et dit : « Ne fais en aucun cas cela, frère, car nous sommes des hommes qui adorent Dieu, et il n'est pas convenable qu'un homme qui adore Dieu rendre le mal pour le mal, ni piétiner celui qui est tombé, ni écraser complètement son ennemi jusqu'à la mort. Et maintenant, remets l'épée à sa place,

\par 4 et viens m'aider, et guérissons-le de cette blessure ; et s'il vit, il sera notre ami et son père Pharaon sera notre père.

\par 5 Alors Lévi releva de terre le fils de Pharaon, lava le sang de son visage, et attacha un pansement sur sa blessure, le fit monter sur son cheval et le conduisit vers son père, Pharaon.

\par 6 lui racontant tout ce qui était arrivé et arrivé.

\par 7 Et Pharaon se leva de son trône et rendit hommage à Lévi sur la terre et le bénit.

\par 8 Puis, le troisième jour passé, le fils de Pharaon mourut à cause de la pierre avec laquelle il avait été blessé par Benjamin.

\par 9 Et Pharaon pleura extrêmement son fils premier-né,

\par 10 C'est pourquoi, de chagrin, Pharaon tomba malade et mourut à 109 ans, et il laissa son diadème au tout beau Joseph.

\par 11 Et Joseph régna seul en Egypte 48 ans ; Après cela, Joseph rendit le diadème au plus jeune enfant de Pharaon, qui était au sein lorsque le vieux Pharaon mourut.

\par 12 Et Joseph fut désormais le père du plus jeune enfant de Pharaon en Égypte jusqu'à sa mort, glorifiant et louant Dieu.


\end{document}