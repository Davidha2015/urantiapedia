\begin{document}

\title{Pseudo-Philo}

\chapter{1}

\par 1 Le début du monde. Adam a engendré trois fils et une fille, Caïn, Noaba, Abel et Seth.

\par 2 Et Adam vécut après avoir engendré Seth 700 ans, et engendra 12 fils et 8 filles.

\par 3 Et voici les noms des mâles : Eliseel, Suris, Elamiel, Brabal, Naat, Zarama, Zasam, Maathal et Anath.

\par 4 Et voici ses filles : Phua, Iectas, Arebica, Sifa, Tecia, Saba, Asin.

\par 5 Et Seth vécut 105 ans et engendra Enos. Et Seth vécut après avoir engendré Enos 707 ans et engendra 3 fils et 2 filles.

\par 6 Et voici les noms de ses fils : Elidia, Phonna et Matha ; et de ses filles, Malida et Thila.

\par 7 Et Enos vécut 180 ans et engendra Caïnan. Et Enos vécut après avoir engendré Caïnan 715 ans, et engendra 2 fils et une fille.

\par 8 Et voici les noms de ses fils : Phoë et Thaal ; et de la fille, Catennath.

\par 9 Et Caïnan vécut 520 ans et engendra Malalech. Et Caïnan vécut, après avoir engendré Malalech, 730 ans, et il engendra 3 fils et 2 filles.

\par 10 Et voici les noms des mâles : Athach, Socer, Lopha ; et les noms des filles, Ana et Leua.

\par 11 Et Malalech vécut 165 ans et engendra Jareth. Et Malalech vécut après avoir engendré Jareth 730 ans, et engendra 7 fils et 5 filles.

\par 12 Et voici les noms des mâles : Leta, Matha, Cethar, Melie, Suriel, Lodo, ​​Othim. Et voici les noms des filles : Ada et Noa, Iebal, Mada, Sella.

\par 13 Et Jareth vécut 172 ans et engendra Enoch. Et Jareth vécut après avoir engendré Enoch 800 ans et engendra 4 fils et 2 filles.

\par 14 Et voici les noms des mâles : Lead, Anac, Soboac et Iectar ; et des filles, Tetzeco, Lesse.

\par 15 Et Enoch vécut 165 ans et engendra Matusalam. Et Enoch vécut après avoir engendré Matusalam 200 ans et engendra 5 fils et 3 filles.

\par 16 Mais Hénoc plut à Dieu en ce temps-là et il ne fut pas trouvé, car Dieu le traduisit.

\par 17 Les noms de ses fils sont : Anaz, Zeum, Achaun, Pheledi, Elith ; et des filles, Theiz, Lefith, Leath.

\par 18 Et Mathusalam vécut 187 ans et engendra Lémec. Et Mathusalam vécut après avoir engendré Lémec 782 ans, et il engendra 2 fils et 2 filles.

\par 19 Et voici les noms des mâles : Inab et Rapho ; et des filles, Aluma et Amuga.

\par 20 Et Lémec vécut 182 ans et engendra un fils, et il l'appela selon sa naissance Noé, en disant : Cet enfant nous donnera du repos, ainsi qu'à la terre, de ceux qui y sont, sur qui (ou au jour où) une visite sera faite à cause de l'iniquité de leurs mauvaises actions.

\par 21 Et Lémec vécut, après avoir engendré Noé, 585 ans.

\par 22 Et Noé vécut 300 ans et engendra 3 fils, Sem, Cham et Japhet.

\chapter{2}

\par 1 Mais Caïn demeura sur la terre en tremblant, comme Dieu lui avait ordonné après qu'il eut tué Abel son frère ; et le nom de sa femme était Themech.

\par 2 Et Caïn connut Themech, sa femme, et elle conçut et enfanta Enoch.

\par 3 Or Caïn avait 15 ans lorsqu'il fit ces choses ; et à partir de ce moment-là, il commença à bâtir des villes, jusqu'à en fonder sept villes. Et voici les noms des villes : Le nom de la première ville d'après le nom de son fils Enoch. Le nom de la deuxième ville Mauli, et de la troisième Leeth, et le nom de la quatrième Teze, et le nom de la cinquième Iesca ; le nom du sixième Céléth et le nom du septième Iebbath.

\par 4 Et Caïn vécut après avoir engendré Enoch 715 ans et engendra 3 fils et 2 filles. Et voici les noms de ses fils : Olad, Lizaph, Fosal ; et de ses filles, Citha et Maac. Et tous les jours de Caïn furent de 730 ans, et il mourut.

\par 5 Alors Hénoc prit une femme parmi les filles de Seth, qui lui enfantèrent Ciram, Cuuth et Madab. Mais Ciram engendra Matusael, et Matusael engendra Lémec.

\par 6 Mais Lémec prit deux femmes : le nom de l'une était Ada et le nom de l'autre Sella.

\par 7 Et Ada lui enfanta Yobab : il fut le père de tous ceux qui habitaient sous des tentes et des troupeaux de troupeaux. Et encore une fois, elle lui enfanta Iobal, qui fut le premier à enseigner tous les instruments (lit. tous les psaumes des orgues).

\par 8 Et en ce temps-là, quand les habitants de la terre commençaient à faire le mal, chacun avec la femme de son prochain, les souillant, Dieu se mit en colère. Et il commença à jouer du luth (kinnor) et de la harpe et de tous les instruments de douce psalmodie (lit. psaltérion), et à corrompre la terre.

\par 9 Mais Sella enfanta Tubal, Misa et Theffa, et c'est ce Tubal qui montra aux hommes les arts du plomb, de l'étain, du fer, du cuivre, de l'argent et de l'or. Alors les habitants de la terre commencèrent à fabriquer des images taillées et à adorer. eux.

\par 10 Or Lémec dit à ses deux femmes Ada et Sella : Écoutez ma voix, femmes de Lémec, prêtez attention à mon précepte ; car j'ai corrompu les hommes pour moi-même et j'ai ôté les nourrissons des mamelles, afin de montrer mes fils, comment faire le mal, et les habitants de la terre. Et maintenant la vengeance sera prise sept fois contre Caïn, mais contre Lémec soixante-dix fois sept fois.

\chapter{3}

\par 1 Et il arriva, lorsque les hommes commencèrent à se multiplier sur la terre, que de belles filles leur naquirent. Et les fils de Dieu virent que les filles des hommes étaient extrêmement belles, et ils les prirent pour femmes parmi toutes celles qu'ils avaient choisies.

\par 2 Et Dieu dit : Mon esprit ne jugera pas éternellement parmi ces hommes, parce qu'ils sont de chair ; mais leurs années seront de 120. Sur qui il a imposé (ou sur quoi j'ai fixé) les extrémités du monde, et entre leurs mains les méchancetés n'ont pas été éteintes (ou la loi ne sera pas éteinte).

\par 3 Et Dieu vit que chez tous les habitants de la terre les œuvres mauvaises s'accomplissaient ; et comme leur pensée était tournée vers l'iniquité tous leurs jours, Dieu dit : J'effacerai l'homme et tout ce qui a germé sur la terre, car je me repens de l'avoir créé.

\par 4 Mais Noé a trouvé grâce et miséricorde devant le Seigneur, et ce sont ses générations. Noé, qui était un homme juste et sans tache dans sa génération, plut au Seigneur. A qui Dieu dit : Le temps de tous les hommes qui habitent sur la terre est venu, car leurs actions sont très mauvaises. Et maintenant, fais-toi une arche en bois de cèdre, et tu la feras ainsi. Sa longueur sera de 300 coudées, sa largeur de 30 coudées et sa hauteur de 30 coudées. Et tu entreras dans l'arche, toi, ta femme, tes fils et les femmes de tes fils avec toi. Et je ferai mon alliance avec toi, pour détruire tous les habitants de la terre. Maintenant, parmi les bêtes pures et parmi les oiseaux purs du ciel, tu prendras par sept, mâles et femelles, afin que leur postérité soit conservée vivante sur la terre. Mais tu prendras des bêtes impures et des oiseaux par deux, mâles et femelles, et tu prendras soin de toi et d'eux aussi.

\par 5 Et Noé fit ce que Dieu lui avait ordonné et entra dans l'arche, lui et tous ses fils avec lui. Et il arriva après 7 jours que l'eau du déluge commença à être sur la terre. Et ce jour-là, toutes les profondeurs furent ouvertes, ainsi que la grande source d'eau et les écluses du ciel, et il y eut de la pluie sur la terre pendant 40 jours et 40 nuits.

\par 6 Et c'était alors la 1652ème (1656ème) année depuis le temps où Dieu avait fait les cieux et la terre, au jour où la terre fut corrompue avec ses habitants à cause de l'iniquité de leurs œuvres.

\par 7 Et lorsque le déluge dura 140 jours sur la terre, Noé seul et ceux qui étaient avec lui dans l'arche restèrent en vie ; et quand Dieu se souvint de Noé, il fit diminuer l'eau.

\par 8 Et il arriva le 90ème jour que Dieu dessécha la terre, et dit à Noé : Sors de l'arche, toi et tous ceux qui sont avec toi, et croisse et multiplie sur la terre. Et Noé sortit de l'arche, lui, ses fils et les femmes de ses fils, et toutes les bêtes, les reptiles, les oiseaux et le bétail, il les fit sortir avec lui, comme Dieu le lui avait ordonné. Alors Noé bâtit un autel à l'Éternel, et il prit de tout le bétail et des oiseaux purs et offrit des holocaustes sur l'autel : et cela fut accepté par l'Éternel comme une odeur de repos.

\par 9 Et Dieu dit : Je ne maudirai plus la terre à cause de l'homme, car l'apparence du cœur de l'homme a disparu dès sa jeunesse. C'est pourquoi je ne détruirai plus tous les êtres vivants, comme je l'ai fait. Mais il arrivera que, lorsque les habitants de la terre auront péché, je les jugerai par la famine ou par l'épée ou par le feu ou par la peste (lit. mort), et il y aura des tremblements de terre, et ils seront dispersés dans des lieux inhabités. (ou, les lieux de leur habitation seront dispersés). Mais je ne gâterai plus la terre avec l'eau du déluge, et pendant tous les jours de la terre, le temps des semailles et la récolte, le froid et la chaleur, l'été et l'automne, le jour et la nuit ne cesseront pas, jusqu'à ce que je me souvienne de ceux qui habitent là-bas. la terre, même jusqu'à ce que les temps soient accomplis.

\par 10 Mais quand les années du monde seront accomplies, alors la lumière cessera et les ténèbres s'éteindront ; et je ressusciterai les morts et je ressusciterai de la terre ceux qui dorment ; et l'enfer paiera sa dette et la destruction donnera restitue ce qui lui a été confié, afin que je puisse rendre à chacun selon ses œuvres et selon le fruit de ses imaginations, jusqu'à ce que je juge entre l'âme et la chair. Et le monde se reposera, et la mort sera éteinte, et l'enfer fermera sa bouche. Et la terre ne sera pas sans naissance, ni stérile pour ceux qui l'habitent ; et aucun de ceux qui auront été justifiés en moi ne sera pollué. Et il y aura une autre terre et un autre ciel, même une habitation éternelle.

\par 11 Et l'Éternel parla encore à Noé et à ses fils, disant : Voici, je ferai mon alliance avec vous et avec votre postérité après vous, et je ne ravagerai plus la terre avec l'eau du déluge. Et tout ce qui y vit et s'y déplace vous servira de nourriture. Mais vous ne mangerez pas la chair et le sang de l'âme. Car celui qui verse le sang de l’homme, son sang sera versé ; car l’homme a été créé à l’image de Dieu. Et vous, grandissez et multipliez et remplissez la terre comme la multitude de poissons qui se multiplient dans les eaux. Et Dieu dit : C'est ici l'alliance que j'ai conclue entre moi et vous ; et ce sera quand je couvrirai le ciel de nuages, que mon arc apparaîtra dans la nuée, et ce sera pour un mémorial de l'alliance entre moi et vous, et tous les habitants de la terre.

\chapter{4}

\par 1 Et les fils de Noé qui sortirent de l'arche furent Sem, Cham et Japhet.

\par 2 Les fils de Japhet : Gomer, Magog et Madaï, Nidiazech, Tubal, Mocteras, Cenez, Riphath et Thogorma, Elisa, Dessin, Cethin, Tudant.

\par Et les fils de Gomer : Thélez, Lud, Deberlet.

\par Et les fils de Magog : Cesse, Thipha, Pharuta, Ammiel, Phimei, Goloza, Samanach.

\par Et les fils de Duden : Sallus, Phelucta Phallita.

\par Et les fils de Tubal : Phanatonova, Eteva.

\par Et les fils de Tyras : Maac, Tabel, Ballana, Samplameac, Elaz.

\par Et les fils de Mellech : Amboradat, Urach, Bosara.

\par Et les fils de [As]cenez : Jubal, Zaraddana, Anac.

\par Et les fils de Héri : Phuddet, Doad, Dephadzeat, Enoc.

\par Et les fils de Togorma : Abiud, Saphath, Asapli, Zepthir.

\par Et les fils d'Élisa : Etzaac, Zenez, Mastisa, Rira.

\par Et les fils de Zepti : Macziel, Temna, Aela, Phinon.

\par Et les fils de Tessis : Meccul, Loon, Zelataban.

\par Et les fils de Duodennin : Itheb, Beath, Phenech.

\par 3 Et ce sont ceux qui ont été dispersés à l'étranger, et qui ont habité sur la terre avec les Perses et les Mèdes, et dans les îles qui sont dans la mer. Et Phenech, fils de Dudeni, monta et ordonna qu'on construise des navires de mer ; et alors le tiers de la terre fut divisé.

\par 4 Domereth et ses fils prirent Ladech ; et Magog et ses fils prirent Degal ; Madame et ses fils prirent Besto ; Iuban (sc. Javan) et ses fils prirent Ceel ; Tubal et ses fils prirent Pheed ; Misech et ses fils prirent Nepthi ; [T]iras et ses fils prirent [Rôô] ; Duodennut et ses fils prirent Goda ; Riphath et ses fils prirent Bosarra ; Torgoma et ses fils prirent Fud ; Elisa et ses fils prirent Thabola ; Thèse (sc. Tarshish) et ses fils prirent Marecham ; Cethim et ses fils prirent Thaan ; Dudennin et ses fils prirent Caruba.

\par 5 Et alors ils commencèrent à labourer la terre et à semer dessus ; et quand la terre eut soif, les habitants crièrent au Seigneur et il les entendit et donna de la pluie abondamment, et il en fut ainsi lorsque la pluie tomba sur la terre, que l'arc apparut dans la nuée, et que les habitants de la terre virent le mémorial de l'alliance et tombèrent sur leurs faces et sacrifièrent, offrant des holocaustes à l'Éternel.

\par 6 Les fils de Cham furent Chus, Mestra, Phuni et Chanaan.

\par Et les fils de Chus : Saba, et . . . Tudan.

\par Et les fils de Phuni : [Effuntenus], Zeleutelup, Geluc, Lephuc.

\par Et les fils de Chanaan furent Sydona, Endain, Racin, Simmin, Uruin, Nenugin, Amathin, Nephiti, Telaz, Elat, Cusin.

\par 7 Et Chus engendra Nembroth. Il commença à être fier devant le Seigneur.

\par Mais Mestram engendra Ludin et Megimin et Labin et Latuin et Petrosonoin et Ceslun : de là sortirent les Philistins et les Cappadociens.

\par 8 Et alors ils commencèrent aussi à bâtir des villes : et voici les villes qu'ils bâtirent : Sydon et les parties qui l'entourent, c'est-à-dire Resun, Beosa, Maza, Gerara, Ascalon, Dabir, Camo, Tellun, Lacis, Sodome et Gomorrhe, Adama et Seboim.

\par 9 Et les fils de Sem : Élam, Assur, Arphaxa, Luzi, Aram. Et les fils d'Aram : Gedrum, Ese. Et Arphaxa engendra Salé, Salé engendra Héber, et à Héber naquirent deux fils : le nom de l'un était Phalech, car de son temps la terre était divisée, et le nom de son frère était Jectan.

\par 10 Et Jectan engendra Helmadam et Salastra et Mazaam, Rea, Dura, Uzia, Deglabal, Mimoel, Sabthphin, Evilac, Iubab.

\par Et les fils de Phalech : Ragau, Rephuth, Zepheram, Aculon, Sachar, Siphaz, Nabi, Suri, Seciur, Phalacus, Rapho, Phalthia, Zaldephal, Zaphis et Arteman, Heliphas. Ce sont là les fils de Phalech, et voici leurs noms. Ils prirent pour femmes des filles de Jectan, engendrèrent des fils et des filles et remplirent la terre.

\par 11 Mais Ragau le prit pour femme Melcha, fille de Ruth, et elle lui engendra Seruch. Et quand le jour de son accouchement arriva, elle dit : De cet enfant naîtra à la quatrième génération celui qui élèvera sa demeure et sera appelé parfait et sans souillure, et il sera le père des nations, et son l'alliance ne sera pas rompue, et sa postérité sera multipliée pour toujours.

\par 12 Et Ragau vécut, après avoir engendré Seruch, 119 ans et engendra 7 fils et 5 filles. Et voici les noms de ses fils : Abiel, Obed, Salma, Dedasal, Zeneza, Accur, Nephes. Et voici les noms de ses filles : Cedema, Derisa, Seipha, Pherita, Theila.

\par 13 Et Seruch vécut 29 ans et engendra Nachor. Et Seruch vécut, après avoir engendré Nachor, 67 ans et engendra 4 fils et 3 filles. Et voici les noms des mâles : Zela, Zoba, Dica et Phodde. Et voici ses filles : Tephila, Oda, Selipha.

\par 14 Et Nachor vécut 34 ans et engendra Thara. Et Nachor vécut, après avoir engendré Thara, 200 ans et engendra 8 fils et 5 filles. Et voici les noms des mâles : Recap, Dediap, Berechap, Iosac, Sithal, Nisab, Nadab, Camoel. Et voici ses filles : Esca, Thipha, Bruna, Ceneta.

\par 15 Et Thara vécut 70 ans et engendra Abram, Nachor et Aram. Et Aram engendra Loth.

\par 16 Alors ceux qui habitaient sur la terre commencèrent à regarder les étoiles, et commencèrent à pronostiquer par elles et à faire de la divination, et à faire passer leurs fils et leurs filles par le feu. Mais Seruch et ses fils ne marchèrent pas selon eux.

\par 17 Et ce sont là les générations de Noé sur la terre, selon leurs langues et leurs tribus, d'où les nations furent divisées sur la terre après le déluge.

\chapter{5}

\par 1 Alors les fils de Cham arrivèrent et établirent Nembroth comme leur prince ; mais les fils de Japhet établirent Phenech leur chef ; et les fils de Sem se rassemblèrent et établirent sur eux Jectan pour être leur prince.

\par 2 Et quand ces trois-là se furent réunis, ils décidèrent qu'ils examineraient et tiendraient compte du peuple de leurs partisans. Et cela se fit du vivant de Noé, afin que tous les hommes fussent rassemblés ; et ils vécurent ensemble, et la terre fut en paix.

\par 3 Or, la 340e année de la sortie de Noé de l'arche, après que Dieu ait séché le déluge, les princes ont tenu compte de leur peuple.

\par 4 Et le premier Phenech, fils de Japhet, les regarda.

\par Les fils de Gomer, tous passant par là, selon les sceptres de leurs capitaineries, étaient au nombre de 5 800.

\par Mais parmi les fils de Magog, tous passant par là, selon les sceptres de leurs dirigeants, le nombre était de 6 200.

\par Et parmi les fils de Madaï, tous ceux qui passaient par là, selon les sceptres de leurs capitaineries, étaient au nombre de 5 700.

\par Et les fils de Tubal. tous ceux qui passaient selon les sceptres de leurs capitaineries étaient au nombre de 9 400.

\par Et les fils de Mesca, tous passant par là, selon les sceptres de leurs capitaineries, étaient au nombre de 5 600.

\par Les fils de Thiras, tous passant par là, selon les sceptres de leurs capitaineries, étaient au nombre de 12 300.

\par Et les fils de Ripha[th] passant par là, selon les sceptres de leurs capitaineries, étaient au nombre de 14 500.

\par Et les fils de Thogorma passant par là, selon les sceptres de leur capitainerie, étaient au nombre de 14 400.

\par Mais les fils d'Élisa passant par là, selon les sceptres de leur capitainerie, étaient au nombre de 14 900.

\par Et les fils de Thersis, tous passant par là, selon les sceptres de leur capitainerie, étaient au nombre de 12 100.

\par Les fils de Cethin, tous passant par là, selon les sceptres de leur capitainerie, étaient au nombre de 17 300.

\par Et les fils de Doin passant par là, selon les sceptres de leurs capitaineries, étaient au nombre de 17 700.

\par Et le nombre du camp des fils de Japhet, tous hommes vaillants et tous ceints de leurs armes, qui étaient placés sous les yeux de leurs capitaines, était de 140 202, sans compter les femmes et les enfants.

\par Le récit complet de Japhet était au nombre de 142 000.

\par 5 Et Nembroth passa par là, lui et le(s) fils de Cham, tous passant par là selon les sceptres de leurs capitaineries furent trouvés au nombre de 24 800.

\par Les fils de Phua, tous passant par là, selon les sceptres de leurs capitaineries, étaient au nombre de 27 700.

\par Et les fils de Canaan, tous passant par là, selon les sceptres de leurs capitaineries, furent trouvés au nombre de 32 800.

\par Les fils de Soba, tous passant par là selon les sceptres de leurs capitaineries, furent trouvés au nombre de 4 300.

\par Les fils de Lebilla, tous passant par là selon les sceptres de leurs capitaineries, furent trouvés au nombre de 22 300.

\par Et les fils de Sata, tous passant par là selon les sceptres de leurs capitaineries, furent trouvés au nombre de 25 300.

\par Et les fils de Remma, tous passant par là, selon les sceptres de leurs capitaineries, furent trouvés au nombre de 30 600.

\par Et les fils de Sabaca, tous passant par là, selon les sceptres de leurs capitaineries, furent trouvés au nombre de 46 400.

\par Et le nombre du camp des fils de Cham, tous hommes vaillants et équipés d'armures, qui étaient placés devant leurs capitaineries, était au nombre de 244 900, sans compter les femmes et les enfants.

\par 6 Et Jectan, fils de Sem, regarda les fils d'Elam, et ils étaient tous de passage, selon le nombre des sceptres de leurs capitaineries, au nombre de 47 000.

\par Et les fils d'Assur, tous passant par là, selon les sceptres de leurs capitaineries, furent trouvés au nombre de 73 000.

\par Et les fils d'Aram, tous passant par là, selon les sceptres de leurs capitaineries, furent trouvés au nombre de 87 300.

\par Les fils de Lud, tous passant par là, selon les sceptres de leurs capitaineries, furent trouvés au nombre de 30 600.

\par [Le nombre des fils de Cham était de 73 000.]

\par Les fils d'Arfaxat, tous passant par là, selon les sceptres de leurs capitaineries, étaient au nombre de 114 600.

\par Et leur nombre total était de 347 600.

\par 7 Le nombre du camp des fils de Sem, tous partis avec vaillance et commandement de la guerre, devant leurs capitaineries, était de † ix † sans compter les femmes et les enfants.

\par 8 Et ce sont les générations de Noé présentées séparément, dont le nombre total était de 914 000. Et tout cela fut compté du vivant de Noé, et en présence de Noé 350 ans après le déluge. Et tous les jours de Noé furent de 950 ans, et il mourut.

\chapter{6}

\par 1 Alors tous ceux qui avaient été divisés et qui habitaient sur la terre se rassemblèrent ensuite et habitèrent ensemble ; Et ils partirent de l'Orient et trouvèrent une plaine au pays de Babylone ; et ils y demeurèrent, et ils dirent chacun à son prochain : Voici, il arrivera que nous serons chacun dispersés. de son frère, et dans les derniers jours nous nous battrons les uns contre les autres. Maintenant donc, venons et construisons-nous une tour dont la tête atteindra le ciel, et nous ferons de nous un nom et une renommée sur la terre.

\par 2 Et ils dirent chacun à son voisin : Prenons des briques (lit. pierres), et écrivons chacun notre nom sur les briques et brûlons-les au feu ; et ce qui est complètement brûlé sera pour mortier. et la brique. (Peut-être que ce qui n'est pas complètement brûlé servira de mortier, et ce qui l'est, de brique.)

\par 3 Et ils prirent chacun leurs briques, sauf 12 hommes qui ne voulaient pas les prendre, et voici leurs noms : Abraham, Nachor, Loth, Ruge, Tenute, Zaba, Armodath, Iobab, Esar, Abimahel, Saba, Auphin. .

\par 4 Et le peuple du pays imposa la main à eux et les conduisit devant leurs princes et dit : Ce sont ces hommes qui ont transgressé nos conseils et qui ne marchent pas dans nos voies. Et les princes leur dirent : Pourquoi ne mettriez-vous pas chacun vos briques avec le peuple du pays ? Et ils répondirent et dirent : Nous ne poserons pas de briques avec vous, et nous ne nous joindrons pas non plus à votre désir. Nous connaissons un seul Seigneur et nous l’adorons. Et si vous nous jetez au feu avec vos briques, nous ne vous consentirons pas.

\par 5 Et les princes furent irrités et dirent : Faites-leur comme ils l'ont dit, et s'ils ne consentent pas à mettre des briques chez vous, vous les brûlerez au feu avec vos briques.

\par 6 Alors Jectan, qui était le premier prince des capitaines, répondit : Non, mais il leur sera donné un espace de 7 jours. Et s’ils se repentent de leurs mauvais conseils et posent des briques avec nous, ils vivront ; sinon, qu'ils soient brûlés selon ta parole. Mais il cherchait comment les sauver des mains du peuple ; car il était de leur tribu, et il servait Dieu.

\par 7 Et après avoir ainsi dit, il les prit et les enferma dans la maison du roi. Et quand ce fut le soir, le prince ordonna d'appeler près de lui 50 hommes vaillants et vaillants, et leur dit : Sortez et emmenez-les. -la nuit, ces hommes qui sont enfermés dans ma maison, et mettez pour eux des provisions de ma maison sur 10 bêtes, et les hommes vous m'amènent, et leurs provisions avec les bêtes, vous les emmenez dans les montagnes et vous les attendez là : et sachez ceci, que si quelqu'un sait ce que je vous ai dit, je vous brûlerai par le feu.

\par 8 Et les hommes partirent et firent tout ce que leur prince leur commandait, et ils sortirent les hommes de sa maison pendant la nuit ; Il prit des provisions, en chargea les bêtes et les emmena dans les montagnes, comme il le leur avait ordonné.

\par 9 Et le prince appela ces 12 hommes et leur dit : Ayez bon courage et ne craignez pas, car vous ne mourrez pas. Car Dieu en qui vous avez confiance est puissant, et c'est pourquoi soyez affermis en lui, car il vous délivrera et vous sauvera. Et maintenant, voici, j'ai ordonné à des hommes de vous prendre avec des provisions dans ma maison, de marcher devant vous dans la montagne et de vous attendre dans la vallée. Et je vous donnerai 50 autres hommes qui vous y conduiront. allez donc et cachez-vous là dans la vallée, en buvant de l'eau qui coule des rochers. Tenez-vous là pendant 30 jours, jusqu'à ce que la colère des gens du pays soit apaisée et que Dieu envoie sa colère sur eux et brise eux. Car je sais que le conseil d’iniquité qu’ils ont accepté d’accomplir ne tiendra pas, car leur pensée est vaine. Et quand 7 jours seront expirés et qu'ils vous chercheront, je leur dirai : Ils sont sortis et ont brisé la porte de la prison dans laquelle ils étaient enfermés et se sont enfuis de nuit, et j'ai envoyé 100 des hommes pour les chercher. Ainsi je les détournerai de la folie qui les assaille.

\par 10 Et là, 11 des hommes lui répondirent, disant : Tes serviteurs ont trouvé grâce à tes yeux, en ce sens que nous sommes libérés des mains de ces hommes orgueilleux.

\par 11 Mais Abram garda seulement le silence, et le prince lui dit : Pourquoi ne me réponds-tu pas, Abram, serviteur de Dieu ? Abram répondit et dit : Voici, je m'enfuis aujourd'hui dans la région montagneuse, et si j'échappe au feu, des bêtes sauvages sortiront des montagnes et nous dévoreront. Ou bien nos vivres manqueront et nous mourrons de faim ; et nous serons trouvés en train de fuir les gens du pays et nous tomberons dans nos péchés. Et maintenant, tant que vit celui en qui j'ai confiance, je ne quitterai pas la place où ils m'ont mis ; et s'il y a un péché de ma part qui me fait vraiment brûler, la volonté de Dieu soit faite. Et le prince lui dit : Que ton sang soit sur ta tête, si tu refuses de sortir avec ceux-ci. Mais si tu y consents, tu seras délivré. Mais si tu veux demeurer, demeure tel que tu es. Et Abram dit : Je ne sortirai pas, mais je demeurerai ici.

\par 12 Et le prince prit ces 11 hommes et en envoya 50 autres avec eux, et leur commanda en disant : Attendez, vous aussi, dans la montagne, pendant 15 jours avec les 50 qui ont été envoyés avant vous ; et après cela vous reviendrez et direz : Nous ne les avons pas trouvés, comme je l'ai dit aux premiers. Et sachez que si quelqu’un transgresse une de toutes ces paroles que je vous ai dites, il sera brûlé par le feu. Alors les hommes sortirent, et il prit Abram seul et l'enferma là où il avait été enfermé auparavant.

\par 13 Et après sept jours, le peuple se rassembla et parla à son prince en disant : Rends-nous les hommes qui ne voulaient pas nous consentir, afin que nous les brûlions au feu. Et ils envoyèrent des capitaines pour les amener, et ils ne les trouvèrent qu'Abram. Et ils les rassemblèrent tous auprès de leur prince en disant : Les hommes que vous avez enfermés se sont enfuis et ont échappé à ce que nous leur avions conseillé.

\par 14 Et Phenech et Nemroth dirent à Jectan : Où sont les hommes que tu as enfermés ? Mais il dit : Ils ont brisé la prison et se sont enfuis de nuit. Mais j'ai envoyé 100 hommes pour les chercher, et je leur ai ordonné, s'ils les trouvaient, non seulement de les brûler au feu, mais de donner leurs corps aux oiseaux du ciel et alors détruisez-les.

\par 15 Alors ils dirent : Cet homme qui se trouve seul, brûlons-le. Et ils prirent Abram et l'amenèrent devant leurs princes et lui dirent : Où sont ceux qui étaient avec toi ? Et il dit : En vérité, la nuit, j'ai dormi, et quand je me suis réveillé, je ne les ai pas trouvés.

\par 16 Et ils le prirent, bâtirent un fourneau, l'allumèrent au feu, et mirent des briques brûlées au feu dans le fourneau. Alors Jectan le prince, stupéfait (lit. fondu) dans son esprit, prit Abram et le mit avec les briques dans la fournaise de feu.

\par 17 Mais Dieu provoqua un grand tremblement de terre, et le feu jaillit de la fournaise et se transforma en flammes et en étincelles de feu et consuma tous ceux qui se tenaient autour en vue de la fournaise ; et tous ceux qui furent brûlés ce jour-là furent 83 500. Mais Abram ne fut pas le moins du monde blessé par la combustion du feu.

\par 18 Et Abram se leva de la fournaise, et la fournaise ardente tomba, et Abram fut sauvé. Et il alla vers les 11 hommes qui étaient cachés dans la montagne et leur raconta tout ce qui lui était arrivé. Ils descendirent avec lui de la montagne en se réjouissant au nom de l'Éternel, et personne ne les rencontra pour les effrayer. Ce jour là. Et ils appelèrent ce lieu du nom d'Abram, et dans la langue des Chaldéens Deli, ce qui est interprété, Dieu.



\chapter{7}

\par 1 Et il arriva après ces choses, que le peuple du pays ne se détourna pas de ses mauvaises pensées ; et ils se rassemblèrent de nouveau vers leurs princes et dirent : Le peuple ne sera pas vaincu pour toujours ; et maintenant venons-en. ensemble et construisons-nous une ville et une tour qui ne seront jamais enlevées.

\par 2 Et quand ils eurent commencé à bâtir, Dieu vit la ville et la tour que bâtissaient les enfants des hommes, et il dit : Voici, celui-ci est un seul peuple et leur parole est une, et ce qu'ils ont commencé à bâtir. la terre ne le supportera pas, et le ciel ne le supportera pas non plus, en le voyant ; et il arrivera, s'ils ne sont pas maintenant empêchés, qu'ils oseront tout ce qu'ils auront à l'esprit de faire.

\par 3 C'est pourquoi voici, je diviserai leur langage, et je les disperserai dans tous les pays, afin qu'ils ne connaissent pas chacun leur frère, et que chacun ne comprenne pas le langage de son prochain. Et je les livrerai aux rochers, et ils se construiront des tabernacles de chaume et de paille, et se creuseront des grottes et y vivront comme les bêtes des champs, et ainsi ils demeureront devant ma face pour toujours, afin qu'ils ne puissent jamais inventer de telles choses. Et je les considérerai comme une goutte d'eau, et je les comparerai à de la salive ; et certains d'entre eux connaîtront la fin par l'eau, et d'autres seront desséchés par la soif.

\par 4 Et avant eux tous, je choisirai mon serviteur Abram, et je le ferai sortir de leur pays, et je le conduirai dans le pays que mes yeux ont regardé dès le commencement, lorsque tous les habitants de la terre ont péché devant ma face. , et j'ai fait venir sur eux l'eau du déluge ; et alors je n'ai pas détruit ce pays, mais je l'ai préservé. C'est pourquoi les sources de ma colère n'y ont pas jailli, et les eaux de ma destruction n'y sont pas tombées. Car c'est là que j'établirai mon serviteur Abram, et je ferai mon alliance avec lui, je bénirai sa postérité, et je serai appelé son Dieu pour toujours.

\par 5 Cependant, lorsque les habitants du pays eurent commencé à construire la tour, Dieu divisa leur langage et changea leur image. Et ils ne connaissaient pas chacun leur frère, et chacun ne comprenait pas non plus le langage de son voisin. Ainsi, lorsque les maçons ordonnaient à leurs aides d'apporter des briques, ils apportaient de l'eau, et s'ils demandaient de l'eau, les autres leur apportaient de la paille. Et ainsi leur conseil fut rompu et ils cessèrent de bâtir la ville ; et Dieu les dispersa de là sur toute la surface de la terre. C'est pourquoi le nom de ce lieu fut appelé Confusion, parce que là Dieu confondit leur parole et les dispersa de là sur la face de toute la terre.

\chapter{8}

\par 1 Mais Abram partit de là et s'établit dans le pays de Chanaan, et prit avec lui Loth, le fils de son frère, et Saraï, sa femme. Et comme Saraï était stérile et n'avait pas de descendance, Abram prit Agar, sa servante, et elle lui enfanta Ismahel. Et Ismahel engendra 12 fils.

\par 2 Alors Loth quitta Abram et habita à Sodome [mais Abram habita au pays de Cam]. Et les hommes de Sodome étaient très méchants et extrêmement pécheurs.

\par 3 Et Dieu apparut à Abraham disant : Je donnerai ce pays à ta postérité ; et ton nom sera appelé Abraham, et Saraï, ta femme, sera appelée Sara. Ana, je te donnerai d'elle une postérité éternelle et je ferai mon alliance avec toi. Et Abraham connut Sara, sa femme, et elle conçut et enfanta Isaac.

\par 4 Et Isaac prit pour femme une Mésopotamie, fille de Bathuel, qui conçut et lui enfanta Ésaü et Jacob.

\par 5 Et Ésaü prit pour femmes Judin, fille de Bereu, et Basmath, fille d'Elon, et Elibema, fille d'Anan, et Manem, fille de Samahel.

\par Et [Basemath] lui enfanta Adelifan, et les fils d'Adelifan furent Temar, Omar, Seffor, Getan, Tenaz, Amalec. Et Judin enfanta Tenacis, Ieruebemas, Bassemen, Rugil ; et les fils de Rugil furent Naizar, Samaza ; et Elibema enfanta Auz, Iollam, Coro.

\par Manem nu Tenetde, Thenatela.

\par 6 Et Jacob prit pour femmes les filles Gen. de Laban le Syrien, Lia et Rachel, et deux concubines, Bala et Zelpha. Et Lia lui enfanta Ruben, Siméon, Lévi, Juda, Isachar, Zabulon et Dina, leur sœur.

\par Mais Rachel enfanta Joseph et Benjamin.

\par Bala enfanta Dan et Neptalim, et Zelpha enfanta Gad et Aser.

\par Ce sont les 12 fils de Jacob et une fille.

\par 7 Et Jacob habita dans le pays de Chanaan, et Sichem, fils d'Emor le Corréen, força sa fille Dina et l'humilia. Siméon et Lévi, fils de Jacob, entrèrent et frappèrent toute leur ville au tranchant de l'épée ; ils prirent Dina, leur sœur, et sortirent de là.

\par 8 Et ensuite Job la prit pour femme et engendra de ses 14 fils et 6 filles, même 7 fils et 3 filles avant d'être frappé d'affliction, et par la suite, lorsqu'il fut rétabli, 7 fils et 3 filles. Et voici leurs noms : Eliphac, Érinoé, Diasat, Philias, Diffar, Zellud, Thelon ; et ses filles Meru, Litaz, Zeli. Et tels qu'avaient été les noms des premiers, ils l'étaient aussi pour les seconds.

\par 9 Or Jacob et ses douze fils habitèrent au pays de Chanaan. Et ses fils haïrent leur frère Joseph, qu'ils livrèrent aussi en Égypte à Pétéphrès, le chef des cuisiniers de Pharaon, et il resta avec lui 14 ans.

\par 10 Et il arriva, après que le roi d'Egypte eut eu un songe, qu'on lui parla de Joseph, et il lui raconta les songes. Après avoir raconté ses rêves, Pharaon l'établit prince de tout le pays d'Égypte. A cette époque, il y avait une famine dans tout le pays, comme Joseph l'avait prévu. Et ses frères descendirent en Égypte pour acheter de la nourriture, car il n'y avait qu'en Égypte qu'il y avait de la nourriture. Et Joseph connaissait ses frères, il se faisait connaître d'eux et ne leur traitait pas de mal. Et il envoya appeler son père du pays de Chanaan, et il descendit vers lui.

\par 11 Et voici les noms des fils d'Israël qui descendirent en Égypte avec Jacob, chacun avec sa maison. Les fils de Ruben, Enoch et Phallud, Esrom et Carmin ; les fils de Siméon, Namuhel et Iamin, Dot et Iachin, et Saül, fils d'une Cananéenne.

\par Les fils de Lévi, Gerson, Caat et Merari ; mais les fils de Juda, Auna, Selon, Phares, Zerami.

\par Les fils d'Isacar, Tola et Phua, Job et Sombram. Fils de Zabulon, Sarelon et Jaillil. Et Dina, leur sœur, eut 14 fils et 6 filles.

\par Et ce sont là les générations de Lia qu'elle enfanta à Jacob. Toutes les âmes des fils et des filles étaient au nombre de 72.

\par 12 Les fils de Dan étaient Usinam. Les fils de Neptalim, Betaal, Neemmu, Surem, Optisariel. Et ce sont les générations de Balla qu'elle
\par nu à Jacob. Toutes les âmes étaient 8.

\par 13 Mais les fils de Gad : . . . Sariel, Sua, Visui, Mophat et Sar : leur sœur la fille de Seriebel, Melchiel. Telles sont les générations de Zelpha, femme de Jacob, qu'elle lui enfanta. Et toutes les âmes des fils et des filles étaient au nombre de 10.

\par 14 Et les fils de Joseph, Éphraïm et Manassen. Benjamin engendra Géla, Esbel, Abocméphec et Utundeus. Et ce sont là les âmes que Rachel enfanta à Jacob, 14.

\par Et ils descendirent en Égypte et y restèrent 210 ans.

\chapter{9}

\par 1 Et il arriva qu'après le départ de Joseph, les enfants d'Israël se multiplièrent et s'accrurent considérablement. Et il se leva en Égypte un autre roi qui ne connaissait pas Joseph ; et il dit à son peuple : Voici, ce peuple est plus nombreux que nous. Venez, prenons conseil contre eux afin qu'ils ne se multiplient pas. Et le roi d'Egypte ordonna à tout son peuple de dire : Tout fils qui naîtra du lait maternel, jetez-le dans le fleuve, mais gardez les femelles en vie. Et les Égyptiens répondirent à leur roi en disant : Tuons leurs mâles et gardons leurs femelles, pour les donner à nos esclaves pour femmes ; et celui qui naîtra d'eux sera esclave et nous servira. Et c’est là ce qui a paru le plus mauvais devant le Seigneur.

\par 2 Alors les anciens du peuple rassemblèrent le peuple dans le deuil, et se lamentèrent et se lamentèrent en disant : Les ventres de nos femmes ont souffert d'une naissance prématurée. Nos fruits sont livrés à nos ennemis et maintenant nous sommes retranchés. Or, donnons-nous une ordonnance afin que personne ne s'approche de sa femme, de peur que le fruit de ses entrailles ne soit souillé et que nos entrailles ne servent d'idoles ; car il vaut mieux mourir sans enfants, jusqu'à ce que nous sachions ce que Dieu fera.

\par 3 Et Amram répondit et dit : Il arrivera plutôt que le monde soit complètement aboli et que le monde incommensurable tombe, ou que le cœur des abîmes touche les étoiles, que que la race des enfants d'Israël soit diminuée. . Et cela arrivera lorsque sera accomplie l'alliance dont Dieu, lorsqu'il l'a faite, a dit à Abraham en disant : Assurément, tes fils habiteront dans un pays qui n'est pas le leur, et ils seront réduits en esclavage et affligés pendant 400 ans. — Et voici, depuis la parole que Dieu a dite à Abraham a été transmise, il y a 350 ans. (Et) depuis que nous sommes en esclavage en Égypte, cela fait 130 ans.

\par 4 Maintenant donc, je n'observerai pas ce que vous ordonnez, mais j'entrerai, je prendrai ma femme et j'engendrerai des fils, afin que nous soyons nombreux sur la terre. Car Dieu ne persistera pas dans sa colère, il n'oubliera pas toujours son peuple, il ne rejettera pas non plus la race d'Israël sur la terre, et il n'a pas non plus conclu en vain son alliance avec nos pères : oui, quand nous n'étions pas encore , Dieu a parlé de ces choses.

\par 5 Maintenant donc j'irai prendre ma femme, et je n'accepterai pas le commandement de ce roi. Et si cela est juste à vos yeux, faisons-le nous tous, car il en sera ainsi, lorsque nos femmes seront enceintes, elles ne seront pas connues pour être grandes enceintes avant que 3 mois ne soient accomplis, comme l'a également fait notre mère Thamar. car son intention n'était pas de fornication, mais comme elle ne voulait pas se séparer des enfants d'Israël, elle réfléchit et dit : Mieux vaut pour moi mourir pour mes péchés avec mon beau-père que d'être unie aux païens. Et elle cacha le fruit de ses entrailles jusqu'au troisième mois, car c'est alors qu'il fut perçu. Et comme elle allait être mise à mort, elle l'affirma en disant : L'homme dont sont ce bâton, cet anneau et cette peau de chèvre, c'est de lui que j'ai conçu. Et son appareil la délivra de tout péril.

\par 6 Maintenant donc faisons aussi ainsi. Et ce sera quand le temps des enfants sera venu, si cela est possible, nous ne rejetterons pas le fruit de nos entrailles. Et qui sait si Dieu sera ainsi provoqué pour nous délivrer de notre humiliation ?

\par 7 Et la parole qu'Amram avait dans son cœur était agréable devant Dieu. Et Dieu dit : Parce que la pensée d'Amram me plaît, et qu'il n'a pas méprisé l'alliance faite entre moi et ses pères, c'est pourquoi voici. maintenant, ce qui est engendré de lui me servira pour toujours, et par lui je ferai des merveilles dans la maison de Jacob, et je ferai par lui des signes et des prodiges pour mon peuple que je n'ai faits pour aucun autre et que j'accomplirai dans leur ma gloire et leur déclarer mes voies.

\par 8 Moi, l'Éternel, j'allumerai pour lui ma lampe pour habiter en lui, et je lui montrerai mon alliance que personne n'a vue, et je lui manifesterai ma grande excellence, ma justice et mes jugements et je ferai briller pour lui une lumière perpétuelle. . Car autrefois, je pensais à lui, en disant : Mon esprit ne sera pas toujours médiateur parmi ces hommes, car ils sont chair, et leurs jours seront de 120 ans.

\par 9 Et Amram, de la tribu de Lévi, sortit et prit une femme de sa tribu, et quand il la prit, le reste le poursuivit et prit leurs femmes. Il avait maintenant un fils et une fille, et ils s'appelaient Aaron et Maria.

\par 10 Et l'esprit de Dieu tomba sur Maria pendant la nuit, et elle fit un rêve, et le matin elle le dit à ses parents en disant : J'ai vu cette nuit, et voici, un homme vêtu d'un vêtement de lin se tenait là et me dit : Va et dis à tes parents : voici, celui qui naîtra de toi sera jeté dans l'eau, car par lui l'eau tarira, et par lui je ferai des miracles, et je sauverai mon peuple, et il aura la capitainerie. celui-ci toujours. Et quand Maria lui avait raconté son rêve, ses parents ne la croyaient pas.

\par 11 Mais la parole du roi d'Egypte prévalut contre les enfants d'Israël et ils furent humiliés et opprimés dans le travail des briques.

\par 12 Mais Jochabeth conçut d'Amram et cacha l'enfant dans son ventre pendant 3 mois, car elle ne pouvait pas le cacher plus longtemps, parce que le roi d'Égypte avait nommé des surveillants de la région, afin que lorsque les femmes hébraïques enfanteraient, elles rejettent les mâles. dans la rivière immédiatement. Et elle prit son enfant et lui fit une arche avec de l'écorce de pin et plaça l'arche au bord du fleuve.

\par 13 Or, le garçon est né dans l'alliance de Dieu et dans l'alliance de sa chair.

\par 14 Et il arriva, quand ils le chassèrent, tous les anciens se rassemblèrent et se disputèrent avec Amram en disant : Ne sont-ce pas là les paroles que nous avons prononcées en disant : « Il vaut mieux pour nous mourir sans enfants que de laisser notre fruit mourir ? être jeté à l’eau ? Et quand ils dirent cela, Amram ne les écouta pas.

\par 15 Mais la fille de Pharaon descendit pour se laver dans le fleuve comme elle l'avait vu en songe, et ses servantes virent l'arche, et elle envoya l'une d'elles, la prit et l'ouvrit. Et quand elle vit l'enfant et regarda l'alliance, c'est-à-dire le testament dans sa chair, elle dit : Il est des enfants des Hébreux.

\par 16 Et elle le prit et le nourrit et il devint son fils, et elle l'appela du nom de Moïse. Mais sa mère l'appelait Melchiel. Et l'enfant fut nourri et devint glorieux au-dessus de tous les hommes, et par lui Dieu délivra les enfants d'Israël, comme il l'avait dit.

\chapter{10}

\par 1 Or, lorsque le roi d'Égypte était mort, un autre roi se leva et affligea tout le peuple d'Israël. Mais ils crièrent à l'Éternel, et il les entendit, et il envoya Moïse et les délivra du pays d'Égypte. Et Dieu envoya aussi sur eux dix plaies et les frappa. Or, c'étaient là les fléaux, à savoir le sang, les grenouilles, et toutes sortes de mouches, la grêle et la mort du bétail, les sauterelles et les moucherons, et les ténèbres qu'on pouvait sentir, et la mort des premiers-nés.

\par 2 Et lorsqu'ils furent partis de là et qu'ils étaient en route, le cœur des Égyptiens s'endurcit encore une fois, et ils continuèrent à les poursuivre et les trouvèrent près de la mer Rouge. Et les enfants d'Israël crièrent à leur Dieu et parlèrent à Moïse en disant : Voici, maintenant est venu le temps de notre destruction, car la mer est devant nous et la multitude des ennemis derrière nous, et nous au milieu. Était-ce pour cela que Dieu nous a fait sortir, ou est-ce là les alliances qu'il a conclues avec nos pères, disant : Je donnerai à votre postérité le pays dans lequel vous habitez ? et maintenant qu'il fasse de nous ce qui lui semble bon.

\par 3 Alors les enfants d'Israël divisèrent leurs conseils en trois divisions de conseils, à cause de la crainte du temps. Pour la tribu de Ruben et d'Isachar et. de Zabulon et de Syméon dirent : Venez, jetons-nous à la mer, car il vaut mieux pour nous mourir dans l'eau que d'être tués par nos ennemis. Et la tribu de Gad et d'Aser et de Dan et Neptalim dirent : Non, mais retournons avec eux, et s'ils nous donnent notre vie, nous les servirons. Mais la tribu de Lévi, de Juda, de Joseph et la tribu de Benjamin dirent : Non, mais prenons nos armes et combattons-les, et Dieu sera avec nous.

\par 4 Moïse cria aussi au Seigneur et dit : Seigneur Dieu de nos pères, ne m'as-tu pas dit : Va dire aux fils de Lia : Dieu m'a envoyé vers toi ? Et maintenant, voici, tu as amené ton peuple au bord de la mer, et les ennemis les poursuivent; mais toi, Seigneur, souviens-toi de ton nom.

\par 5 Et Dieu dit : Puisque tu m'as crié, prends ton bâton et frappe la mer, et elle sera asséchée. Et quand Moïse fit tout cela, Dieu menaça la mer, et la mer fut asséchée : les mers d'eaux s'arrêtèrent et les profondeurs de la terre apparurent, et les fondations de la demeure furent mises à nu au bruit de la peur. de Dieu et au souffle de la colère de mon Seigneur.

\par 6 Et Israël passa à sec, au milieu de la mer. Et les Égyptiens les virent et se mirent à leur poursuite. Dieu endurcit leur esprit, et ils ne savaient pas qu'ils entraient dans la mer. Et c'est ainsi que pendant que les Égyptiens étaient dans la mer, Dieu ordonna encore une fois à la mer et dit à Moïse : Frappez la mer encore une fois. Et il l’a fait. Et l'Éternel commanda à la mer, et elle retourna vers ses vagues, et couvrit les Égyptiens, leurs chars et leurs cavaliers jusqu'à ce jour.

\par 7 Mais quant à son propre peuple, il les conduisit dans le désert : quarante ans il fit pleuvoir pour eux du pain du ciel, et il leur fit sortir des cailles de la mer, et un puits d'eau qui les suivait leur fit sortir pour eux. . Et dans une colonne de nuée il les conduisait le jour et dans une colonne de feu la nuit il les éclairait.

\chapter{11}

\par 1 Et au troisième mois du voyage des enfants d'Israël hors du pays d'Égypte, ils arrivèrent dans le désert du Sinaï. Et Dieu se souvint de sa parole et dit : J'éclairerai le monde, j'éclairerai les lieux habitables, je ferai mon alliance avec les enfants des hommes, et je glorifierai mon peuple plus que toutes les nations, car je lui proposerai une exaltation éternelle. ce qui sera pour eux une lumière, mais pour les impies un châtiment.

\par 2 Et il dit à Moïse : Voici, je t'appellerai demain : sois prêt et dis à mon peuple : « Pendant trois jours, qu'un homme ne s'approche pas de sa femme », et le 3ème jour je parlerai à Moïse. toi et vers eux, et après cela tu monteras vers moi. Et je mettrai mes paroles dans ta bouche et tu éclaireras mon peuple. Car j'ai remis entre tes mains une loi éternelle par laquelle je jugerai le monde entier. Car ceci sera pour témoignage. Car si les hommes disent : « Nous ne t’avons pas connu, et c’est pourquoi nous ne t’avons pas servi », alors je me vengerai d’eux, parce qu’ils n’ont pas connu ma loi.

\par 3 Et Moïse fit ce que Dieu lui avait commandé, et sanctifia le peuple et leur dit : Soyez prêts le 3ème jour, car après 3 jours Dieu fera son alliance avec vous. Et le peuple fut sanctifié.

\par 4 Et il arriva le troisième jour que voici, il y eut des voix de tonnerres (lit. ceux qui sonnaient) et l'éclat des éclairs et la voix des instruments résonnant à haute voix. Et la peur envahit tous ceux qui étaient dans le camp. Et Moïse emmena le peuple à la rencontre de Dieu.

\par 5 Et voici, les montagnes brûlèrent de feu et la terre trembla et les collines furent enlevées et les montagnes renversées ; les profondeurs bouillonnèrent, et tous les lieux habitables furent ébranlés ; et les cieux se replièrent et les nuages ​​tirèrent de l'eau. Et des flammes de feu brillèrent et les tonnerres et les éclairs se multiplièrent et les vents et les tempêtes rugirent : les étoiles se rassemblèrent et les anges coururent devant, jusqu'à ce que Dieu établisse la loi d'une alliance éternelle avec les enfants d'Israël et leur donna un commandement éternel qui ne doit pas passer.

\par 6 Et à ce moment-là, l'Éternel dit à son peuple toutes ces paroles, disant : Je suis l'Éternel, ton Dieu, qui t'ai fait sortir du pays d'Égypte, de la maison de servitude. Tu ne te feras pas de dieux gravés, ni aucune image abominable du soleil ou de la lune, ni aucun des ornements du ciel, ni la ressemblance de toutes les choses qui sont sur la terre ni de celles qui rampent dans les eaux. ou sur la terre. Je suis l'Éternel, ton Dieu, un Dieu jaloux, qui punis les péchés de ceux qui dorment sur les enfants vivants des impies, s'ils marchent dans les voies de leurs pères ; jusqu'à la troisième et quatrième génération, faisant (ou montrant) miséricorde jusqu'à 1000 générations envers ceux qui m'aiment et gardent mes commandements.

\par 7 Tu ne prendras pas en vain le nom de l'Éternel, ton Dieu, afin que mes voies ne soient pas vaines. Car Dieu abomine celui qui prend son nom en vain.

\par 8 Observez le jour du sabbat pour le sanctifier. Six jours font ton travail, mais le septième jour est le sabbat du Seigneur. Tu n'y feras aucun ouvrage, toi et tous tes ouvriers, sauf pour y louer le Seigneur dans la congrégation des anciens et glorifier le Puissant dans le siège des vieillards. Car en six jours, l'Éternel a fait les cieux et la terre, la mer et tout ce qu'ils contiennent, et tout le monde, le désert inhabité, et tout ce qui travaille, et tout l'ordre du ciel, et Dieu s'est reposé. le septième jour. C'est pourquoi Dieu a sanctifié le septième jour, parce qu'il s'y reposait.

\par 9 Tu aimeras ton père et ma mère et tu les craindras ; et alors ta lumière se lèvera, et je commanderai au ciel et il te paiera la pluie, et la terre hâtera ses fruits et tes jours seront nombreux. et tu habiteras dans ton pays, et tu ne resteras pas sans enfants, car ta postérité ne manquera pas, même celle de ceux qui y habitent.

\par 10 Tu ne commettras pas d'adultère, car tes ennemis n'ont pas commis d'adultère avec toi, mais tu es sorti avec la main haute.

\par 11 Tu ne tueras pas, parce que tes ennemis n'ont pas eu le pouvoir de te tuer, mais tu as vu leur mort.

\par 12 Tu ne porteras pas de faux témoignage contre ton prochain, en parlant faussement, de peur que tes sentinelles ne parlent faussement contre toi.

\par 13 Tu ne convoiteras pas la maison de ton prochain, ni ce qu'il possède, de peur que d'autres ne convoitent aussi ton pays.

\par 14 Et quand l'Eternel cessa de parler, le peuple fut saisi d'une grande frayeur ; et ils virent la montagne brûlante avec des torches de feu, et ils dirent à Moïse : Parle-nous, et que Dieu ne nous parle pas, de peur que par hasard nous mourons. Car voici, aujourd’hui nous savons que Dieu parle à l’homme face à face, et que l’homme vivra. Et maintenant, nous avons vraiment perçu comment la terre portait la voix de Dieu en tremblant. Et Moïse leur dit : Ne craignez rien, car cette voix vous est parvenue, afin que vous ne péchiez pas (ou, à cause de cela, afin qu'il vous éprouve). Dieu est venu vers vous, afin que vous receviez sa crainte. à vous, afin que vous ne péchiez pas).

\par 15 Et tout le peuple se tenait au loin, mais Moïse s'approcha de la nuée, sachant que Dieu était là. Et alors Dieu lui annonça sa justice et ses jugements, et le garda près de lui 40 jours et 40 nuits. Et là il lui commanda beaucoup de choses, et lui montra l'arbre de vie, dont il coupa et prit et le mit dans Mara, et l'eau de Mara fut adoucie et les suivit dans le désert pendant 40 ans, et monta dans le avec eux les collines et descendirent dans la plaine. Il lui commanda aussi concernant le tabernacle et l'arche de l'Éternel, et le sacrifice d'holocauste et d'encens, et l'ordonnance de la table et du chandelier, et concernant la cuve et son socle, et l'épaulette et le il leur montra leur image, pour les faire selon le modèle qu'il avait vu. Et lui dit : Fais-moi un sanctuaire et le tabernacle de ma gloire sera parmi toi.

\chapter{12}

\par 1 Et Moïse descendit ; et tandis qu'il était couvert d'une lumière invisible, car il était descendu dans le lieu où est la lumière du soleil et de la lune, la lumière de son visage l'emporta sur l'éclat du soleil et de la lune, et il ne le savait pas. Et il en fut ainsi lorsqu'il descendit vers les enfants d'Israël, ils le virent et ne le connurent pas. Mais quand il parlait, ils le connaissaient. Et c'était comme ce qui se passait en Égypte lorsque Joseph connaissait ses frères mais qu'ils ne le connaissaient pas. Et il arriva après cela, lorsque Moïse comprit que son visage était devenu glorieux, il lui fit un voile pour couvrir son visage.

\par 2 Mais pendant qu'il était sur la montagne, le cœur du peuple se corrompit, et ils se rassemblèrent vers Aaron et dirent : Faites de nous des dieux afin que nous puissions les servir, comme l'ont aussi fait les autres nations. Car ce Moïse par qui les prodiges ont été accomplis avant nous, nous est enlevé. Et Aaron leur dit : Ayez patience, car Moïse viendra et fera approcher le jugement de nous, et nous éclairera une loi, et fera ressortir de sa bouche la grande excellence de Dieu, et fixera des jugements pour notre peuple.

\par 3 Et quand il disait cela, ils ne l'écoutèrent pas, afin que s'accomplisse la parole qui a été prononcée au jour où le peuple pécha en construisant la tour, quand Dieu dit : Et maintenant, si je ne le leur défends pas, ils le feront. aventure tout ce qu'ils envisagent de faire, et pire encore. Mais Aaron eut peur, car le peuple était fort fortifié, et il leur dit : Apportez-nous les boucles d'oreilles de vos femmes. Et les hommes cherchèrent chacun sa femme, et ils la leur donnèrent aussitôt, et ils les mirent au feu et elles devinrent une figure, et il en sortit un veau en fusion.

\par 4 Et l'Éternel dit à Moïse : Dépêche-toi, car le peuple est corrompu et a agi de manière trompeuse dans mes voies que je leur ai commandées. Et si les promesses que j'avais faites à leurs pères lorsque je disais : À votre postérité je donnerai ce pays dans lequel vous habitez ? Car voici, le peuple n'est pas encore entré dans le pays, même s'il supporte mes jugements, mais il m'a abandonné. Et c'est pourquoi je sais que s'ils entrent dans le pays, ils commettront des iniquités encore plus grandes. Maintenant donc, moi aussi, je les abandonnerai ; et je reviendrai et je ferai la paix avec eux, afin qu'une maison me soit bâtie parmi eux ; et cette maison aussi sera supprimée, parce qu'ils pécheront contre moi, et la race des hommes sera pour moi comme une goutte de cruche et sera considérée comme de la salive.

\par 5 Et Moïse descendit en toute hâte et vit le veau, et il regarda les tables et vit qu'elles n'étaient pas écrites ; et il se hâta de les briser ; et ses mains s'ouvrirent et il devint comme une femme qui accouche de son premier-né ; lorsqu'elle est prise dans ses douleurs, ses mains sont sur son sein, et elle n'a aucune force pour l'aider à accoucher.

\par 6 Et il arriva qu'au bout d'une heure il dit en lui-même : L'amertume ne prévaut pas pour toujours, et le mal n'a pas toujours la domination. Maintenant donc je vais me lever et fortifier mes reins ; car même s'ils ont péché, ces choses qui m'ont été déclarées d'en haut ne seront pas vaines.

\par 7 Et il se leva, brisa le veau, le jeta dans l'eau, et fit boire le peuple. Et il en était ainsi : si quelqu'un voulait que le veau soit fabriqué, sa langue était coupée, mais si quelqu'un y était contraint par la peur, son visage brillait.

\par 8 Et alors Moïse monta sur la montagne et pria l'Éternel, disant : Voici maintenant, tu es Dieu qui as planté cette vigne et qui en a mis les racines dans l'abîme, et qui en a étendu les pousses jusqu'à ton siège le plus élevé. . Regardez-le en ce moment, car la vigne a produit ses fruits et n'a pas connu celui qui la labourait. Et maintenant, si tu es irrité contre ta vigne et que tu l'arraches de l'abîme, et que tu dessèches les sarments de ton siège éternel le plus élevé, l'abîme ne viendra plus pour la nourrir, ni ton trône pour rafraîchir ta vigne que tu as créée. as brûlé.

\par 9 Car tu es celui qui es toute lumière, et tu as orné ta maison de pierres précieuses et d'or et de parfums et d'épices (ou et de jaspe), et de bois de baume et de cannelle, et de racines de myrrhe et de costum tu as parsemé ton maison, et tu l'as rassasié de viandes diverses et de boissons sucrées. Si donc tu n'as pas pitié de ta vigne, toutes ces choses sont faites en vain, Seigneur, et tu n'auras personne pour te glorifier. Car même si tu plantes une autre vigne, celle-là ne se confiera pas non plus en toi, parce que tu as détruit la première. Car si vraiment tu abandonnes le monde, qui fera pour toi ce que tu as dit comme Dieu ? Et maintenant, que ta colère soit retenue contre ta vigne, d'autant plus [à cause de] ce que tu as dit et de ce qui reste à dire, et que ton travail ne soit pas vain, et que ton héritage ne soit pas déchiré dans l'humiliation.

\par 10 Et Dieu lui dit : Voici, je suis devenu miséricordieux selon tes paroles. Taille donc deux tables de pierre à l'endroit d'où tu as taillé la première, et écris de nouveau dessus mes jugements qui portaient sur la première.

\chapter{13}

\par 1 Et Moïse se hâta et fit tout ce que Dieu lui avait commandé, et descendit et fit les tables [et le tabernacle] et ses ustensiles, et l'arche et les lampes et la table et l'autel des holocaustes et l'autel. d'encens, et l'épaulette, et le pectoral, et les pierres précieuses, et la cuve, et les bases, et tout ce qui lui fut montré. Et il ordonna tous les vêtements des prêtres, les ceintures et tout le reste, la mitre, la plaque d'or et la sainte couronne ; il fit aussi l'huile d'onction pour les prêtres, et il sanctifia les prêtres eux-mêmes. Et quand tout fut terminé, la nuée les recouvrit tous.

\par 2 Alors Moïse cria à l'Éternel, et Dieu lui parla depuis le tabernacle en disant : C'est la loi de l'autel, par laquelle vous me sacrifierez et prierez pour vos âmes. Mais quant à ce que vous m'offrirez, offrez du bétail le veau, le mouton et la chèvre, et parmi les oiseaux la tortue et la colombe.

\par 3 Et s'il y a la lèpre dans ton pays, et que le lépreux soit purifié, qu'on prenne pour l'Éternel deux oisillons vivants, du bois de cèdre, de l'hysope et du cramoisi ; et il viendra chez le prêtre, et il tuera l'un et gardera l'autre. Et il soignera le lépreux selon tout ce que j'ai prescrit dans ma loi.

\par 4 Et quand les temps viendront pour vous, vous me sanctifierez par un jour de fête et vous vous réjouirez devant moi à la fête des pains sans levain, et vous présenterez du pain devant moi, célébrant une fête de souvenir, car à ce moment-là jour où vous êtes sortis du pays d’Égypte.

\par 5 Et à la fête des semaines, vous me proposerez du pain et vous me ferez une offrande pour vos fruits.

\par 6 Mais la fête des trompettes sera une offrande pour vos veilleurs, car c'est là que j'ai surveillé ma création, afin que vous vous souveniez du monde entier. Au commencement de l'année, quand vous me les montrerez, je connaîtrai le nombre des morts et de ceux qui sont nés, ainsi que le jeûne de la miséricorde. Car vous jeûnerez avec moi pour vos âmes, afin que les promesses de vos pères s'accomplissent.

\par 7 Amenez-moi aussi la fête des tabernacles : vous prendrez pour moi les fruits agréables de l'arbre, et des branches de palmier, de saules et de cèdres, et des branches de myrrhe ; et je me souviendrai de toute la terre sous la pluie. , et la mesure des saisons sera établie, et j'ordonnerai les étoiles et commanderai les nuages, et les vents retentiront et les éclairs courront au loin, et il y aura une tempête de tonnerre, et ce sera pour un signe perpétuel . Les nuits produiront de la rosée, comme je l'ai dit après le déluge de la terre.

\par 8 quand je (ou alors lui) lui donnai un précepte concernant l'année de la vie de Noé, et lui dis : Ce sont les années que j'ai fixées après les semaines où j'ai visité la cité des hommes, à quelle époque je leur (ou lui) montra le lieu de naissance et la couleur (ou et le serpent), et je (ou lui) dis : Ceci est. C'est le lieu que j'ai enseigné au premier homme en disant : Si tu ne transgresses pas ce que je t'ai ordonné, toutes choses te seront soumises. Mais il a transgressé mes voies et a été persuadé de sa femme, et elle a été trompée par le serpent. Et puis la mort fut ordonnée aux générations d’hommes.

\par 9 Et de plus, le Seigneur lui montra (ou, Et le Seigneur dit plus loin : Je lui montrai) les voies du paradis et lui dit : Ce sont les voies que les hommes ont perdues en n'y marchant pas, parce qu'ils ont péché contre moi. .

\par 10 Et le Seigneur lui commanda concernant le salut des âmes du peuple et dit : S'ils marchent dans mes voies, je ne les abandonnerai pas, mais je leur serai toujours miséricordieux, et je bénirai leur postérité et la terre. se hâtera de produire ses fruits, et il y aura de la pluie pour eux pour augmenter leurs gains, et la terre ne sera pas stérile. Mais en vérité, je sais qu'ils corrompent leurs voies, que je les abandonnerai et qu'ils oublieront les alliances que j'ai conclues avec leurs pères. Mais je ne les oublierai pas pour toujours ; car dans les derniers jours, ils sauront qu'à cause de leurs péchés, leur postérité a été abandonnée ; car je suis fidèle dans mes voies.

\chapter{14}

\par 1 En ce temps-là, Dieu lui dit : Commencez à dénombrer mon peuple à partir de 20 ans et plus jusqu'à 40 ans, afin que je puisse montrer à vos tribus tout ce que j'ai déclaré à leurs pères dans un pays étranger. Car je les ai ressuscités du pays d'Égypte sur la cinquantième partie, mais 40 et 9 parties d'entre eux sont mortes dans le pays d'Égypte.

\par 2 Quand tu les auras ordonnés et dénombrés (ou, Pendant que vous y demeurez. Et quand vous les aurez dénombrés, etc.), écrivez leur histoire, jusqu'à ce que j'accomplisse tout ce que j'ai dit à leurs pères, et que je les place fermement dans leur propre pays, car je ne diminuerai aucune parole de ceux que j'ai dit à leurs pères, même de ceux que je leur ai dit : Votre postérité sera comme les étoiles du ciel en multitude. Ils entreront en nombre dans le pays, et en peu de temps ils deviendront sans nombre.

\par 3 Alors Moïse descendit et les dénombra, et le nombre du peuple était de 604 550. Mais la tribu de Lévi ne le comptait pas parmi eux, car tel lui avait été commandé ; seulement il dénombra ceux qui avaient plus de 50 ans, dont le nombre était de 47 300. Il dénombra aussi ceux qui avaient moins de 20 ans, et leur nombre fut de 850 850. Et il examina la tribu de Lévi, et leur nombre total était de CXX. CCXD. DCXX. CC. DCCC.

\par 4 Et Moïse en déclara le nombre à Dieu ; Et Dieu lui dit : Ce sont les paroles que j'ai dites à leurs pères au pays d'Égypte, et j'ai fixé un nombre de 210 ans à tous ceux qui ont vu mes merveilles. Or, leur nombre était de 9 000 fois 10 000, 200 fois 95 000 hommes, sans compter les femmes, et j'ai fait mourir toute la multitude d'entre eux parce qu'ils ne m'ont pas cru, et il m'en est resté la cinquantième partie et je les ai sanctifiés pour moi. . C'est pourquoi j'ordonne à la génération de mon peuple de me donner la dîme de ses fruits, pour qu'elle soit devant moi en souvenir de la grande oppression que j'ai ôtée d'eux.

\par 5 Et lorsque Moïse descendit et annonça ces choses au peuple, ils pleurèrent et se lamentèrent et demeurèrent dans le désert deux ans.

\chapter{15}

\par 1 Et Moïse envoya des espions pour explorer le pays, même 12 hommes, car tel lui avait été commandé. Et après être montés et avoir vu le pays, ils revinrent vers lui, lui apportant les fruits du pays, et troublèrent le cœur du peuple, en disant : Vous ne pourrez pas hériter du pays, car il est fermé par du fer. bars par leurs hommes puissants.

\par 2 Mais deux hommes sur les 12 ne parlèrent pas ainsi, mais dirent : De même que le fer dur peut vaincre les étoiles, ou comme les armes peuvent vaincre les éclairs, ou que les oiseaux du ciel éteignent le tonnerre, ainsi ces hommes peuvent résister. le Seigneur. Car ils virent qu'à mesure qu'ils montaient, les éclairs des étoiles brillaient et les tonnerres suivaient, résonnant avec elles.

\par 3 Et voici les noms des hommes : Chaleb, fils de Jephone, fils de Beri, fils de Batuel, fils de Galipha, fils de Zenen, fils de Selimun, fils de Selon, fils de Juda. L'autre, Jésus, fils de Naue, fils d'Eliphat, fils de Gal, fils de Néphélien, fils d'Emon, fils de Saül, fils de Dabra, fils d'Effrem, fils de Joseph.

\par 4 Mais le peuple ne voulut pas entendre la voix des deux, mais fut très troublé et parla en disant : Sont-ce là les paroles que Dieu nous a dites en disant : Je vous amènerai dans un pays où coulent le lait et le miel ? Et comment maintenant nous élève-t-il afin que nous tombions sous l'épée et que nos femmes partent en captivité ?

\par 5 Et quand ils parlèrent ainsi, la gloire de Dieu apparut tout à coup, et il dit à Moïse : Ce peuple persiste-t-il ainsi à ne pas m'écouter du tout ? Voici maintenant, les conseils que j'ai émanés ne seront pas vains. J'enverrai sur eux l'ange de ma colère pour briser leurs corps par le feu dans le désert. Et je commanderai à mes anges qui veillent sur eux de ne pas prier pour eux, car j'enfermerai leurs âmes dans les trésors des ténèbres, et je dirai à mes serviteurs, leurs pères : Voici, ceci est la semence à laquelle J'ai dit : Votre postérité viendra dans un pays qui n'est pas le sien, et je jugerai la nation qu'elle servira. Et j'ai accompli mes paroles et j'ai fait fondre leurs ennemis, et j'ai soumis des anges sous leurs pieds, et j'ai mis une nuée pour couvrir leurs têtes, et j'ai commandé à la mer, et les profondeurs ont été brisées avant que leur face et les murs d'eau ne se dressent. .

\par 6 Et il n'y a pas eu de pareille à cette parole depuis le jour où j'ai dit : Que les eaux sous le ciel soient rassemblées en un seul lieu, jusqu'à ce jour. Et je les ai fait sortir, j'ai tué leurs ennemis et je les ai conduits devant moi sur la montagne Sina. Et j'inclinai les cieux et je descendis pour allumer une lampe pour mon peuple et pour fixer des limites à toutes les créatures. Et je leur ai appris à me faire un sanctuaire pour que j'habite parmi eux. Mais ils m'ont abandonné et sont devenus infidèles dans mes paroles, et leur esprit s'est évanoui, et maintenant voici, les jours viendront où je leur ferai ce qu'ils ont désiré et je jetterai leurs corps dans le désert.

\par 7 Et Moïse dit : Avant que tu ne prennes la semence pour faire l'homme sur la terre, ai-je ordonné ses voies ? c'est pourquoi maintenant, que ta miséricorde nous souffre jusqu'à la fin, et ta pitié pendant toute la durée des jours.

\chapter{16}

\par 1 A ce moment-là, il lui donna des commandements concernant les marges ; puis Choreb se révolta et 200 hommes avec lui et parlèrent en disant : Et si une loi que nous ne pouvons pas supporter nous était ordonnée ?

\par 2 Et Dieu fut en colère et dit : J'ai commandé à la terre et elle m'a donné l'homme, et de lui sont nés les deux premiers fils. Et l'aîné se leva et tua le plus jeune, et la terre se hâta d'avaler son sang. Mais j'ai chassé Caïn, j'ai maudi la terre et j'ai dit à Sion : Tu n'avaleras plus de sang. Et maintenant les pensées des hommes sont grandement polluées.

\par 3 Voici, je commanderai à la terre, et elle engloutira corps et âme ensemble, et leur demeure sera dans les ténèbres et dans la destruction, et ils ne mourront pas mais dépériront jusqu'à ce que je me souvienne du monde et renouvelle la terre. . Et alors ils mourront et ne vivront plus, et leur vie sera ôtée du nombre de tous les hommes ; l’enfer ne les vomira plus, et la destruction ne se souviendra pas d’eux, et leur départ sera comme celui de la tribu de les nations dont j'ai dit : « Je ne me souviendrai pas d'elles », c'est-à-dire le camp des Égyptiens et le peuple que j'ai détruit avec l'eau du déluge. Et la terre les engloutira, et je ne leur ferai plus rien.

\par 4 Et lorsque Moïse prononça toutes ces paroles au peuple, Choreb et ses hommes étaient encore incrédules. Et Choreb envoya appeler ses sept fils qui n'avaient pas conseil avec lui.

\par 5 Mais ils lui envoyèrent en réponse, disant : De même que le peintre ne montre pas une image faite par son art à moins qu'il ne soit d'abord instruit, ainsi nous aussi, lorsque nous avons reçu la loi du Très-Puissant qui nous enseigne ses voies, nous n'avons pas entrer . là-dedans, sauf pour que nous puissions y marcher. Notre père ne nous a pas engendrés, mais le Tout-Puissant nous a formés, et maintenant, si nous marchons dans ses voies, nous serons ses enfants. Mais si tu ne crois pas, va ton propre chemin. Et ils ne s'approchèrent pas de lui.

\par 6 Et il arriva après cela que la terre s'ouvrit devant eux, et ses fils lui envoyèrent dire : Si ta folie est tranquille sur toi, qui t'aidera au jour de ta destruction ? et il ne les écouta pas. Et la terre ouvrit la bouche et les engloutit ainsi que leurs maisons, et quatre fois les fondements de la terre furent ébranlés pour engloutir les hommes, comme cela lui avait été ordonné. Et ensuite Choreb et sa compagnie gémirent, jusqu'à ce que le firmament de la terre soit rendu.

\par 7 Mais les assemblées du peuple dirent à Moïse : Nous ne pouvons pas rester aux alentours de ce lieu où Choreb et ses hommes ont été engloutis. Et il leur dit. Élevez vos tentes tout autour d'eux, et ne vous joignez pas à leurs péchés. Et ils l’ont fait.

\chapter{17}

\par 1 Alors la lignée des prêtres de Dieu fut déclarée par le choix d'une tribu, et il fut dit à Moïse : Prends dans chaque tribu une verge et mets-les dans le tabernacle, et alors la verge de celui qui sera à qui que ce soit mon la gloire parlera, fleurira, et j'ôterai les murmures de mon peuple.

\par 2 Et Moïse fit ainsi et posa 12 verges, et la verge d'Aaron sortit, et produisit des fleurs et donna des graines d'amandes.

\par 3 Et cette ressemblance qui est née là était semblable à l'ouvrage qu'Israël faisait alors qu'il était en Mésopotamie avec Laban le Syrien, lorsqu'il prit des tiges d'amandes et les plaça au point de collecte des eaux, et que le bétail venait boire. et ils furent répartis parmi les tiges épluchées, et ils enfantèrent des [chevreaux] blancs, tachetés et bariolés.

\par 4 C'est pourquoi la synagogue du peuple fut rendue semblable à un troupeau de moutons, et comme le bétail naissait selon les verges d'amandes, de même le sacerdoce fut établi au moyen des verges d'amandes.

\chapter{18}

\par 1 En ce temps-là, Moïse tua Séon et Og, les rois des Amoréens, et partagea tout leur pays entre ses gens, et ils y demeurèrent.

\par 2 Mais Balac était le roi de Moab, qui vivait en face d'eux. Il eut très peur et envoya vers Balaam, fils de Beor, l'interprète des rêves, qui habitait en Mésopotamie, et lui dit: Voici, je sais comment que sous le règne de mon père Sefor, lorsque les Amoréens combattaient contre lui, tu les maudis et ils furent livrés devant lui. Et maintenant, viens et maudis ce peuple, car ils sont nombreux, plus nombreux que nous, et cela te fera un grand honneur.

\par 3 Et Balaam dit : Voici, cela est bon aux yeux de Balac, mais il ne sait pas que le conseil de Dieu n'est pas comme le conseil de l'homme. Et il ne sait pas que l'esprit qui nous est donné est donné pour un temps, et que nos voies ne sont guidées que si Dieu le veut. Maintenant donc, restez ici, et je verrai ce que le Seigneur me dira cette nuit.

\par 4 Et pendant la nuit, Dieu lui dit : Qui sont les hommes qui sont venus vers toi ? Et Balaam dit : Pourquoi, Seigneur, tentes-tu la race humaine ? Ils ne peuvent donc pas le soutenir, car tu connaissais plus qu'eux tout ce qu'il y avait dans le monde, avant de le fonder. Et maintenant, éclaire ton serviteur s'il est juste que je les accompagne.

\par 5 Et Dieu lui dit : N'est-ce pas à propos de ce peuple que j'ai dit à Abraham dans une vision : Ta postérité sera comme les étoiles du ciel, quand je l'ai élevé au-dessus du firmament et lui ai montré tous les ordres de les étoiles, et lui demanda-t-il son fils en holocauste ? et il l'amena pour le déposer sur l'autel, mais je le rendis à son père. Et comme il n'a pas résisté, son offrande a été agréable à mes yeux, et c'est pour son sang que j'ai choisi ce peuple. Et puis j'ai dit aux anges qui travaillent subtilement : Je n'ai pas dit de lui : À Abraham révélerai-je tout ce que je fais ?

\par 6 Jacob aussi, lorsqu'il luttait dans la poussière avec l'ange qui était au-dessus des louanges, ne le laissa pas partir jusqu'à ce qu'il l'ait béni. Et maintenant, voici, tu penses aller avec eux et maudir ceux que j'ai choisis. Mais si tu les maudis, qui est celui qui te bénira ?

\par 7 Et Balaam se leva le matin et dit : Va, car Dieu ne veut pas que je vienne avec toi. Et ils allèrent rapporter à Balac tout ce qui avait été dit de Balaam. Et Balac envoya encore d'autres hommes vers Balaam pour dire : Voici, je sais que lorsque tu offriras des holocaustes à Dieu, Dieu se réconciliera avec l'homme, et maintenant demande encore une fois à ton Seigneur, et implore par des holocaustes, autant qu'il volonté. Car si, par hasard, il se montre apaisé dans mes besoins, tu auras ta récompense, si Dieu accepte tes offrandes.

\par 8 Et Balaam leur dit : Voici, le fils de Séphor est insensé, et il ne sait pas qu'il habite à proximité (lit. autour) des morts. Et maintenant, restez ici cette nuit et je verrai ce que Dieu dira à moi. Et Dieu lui dit : Va avec eux, et ton voyage sera un scandale, et Balac lui-même ira à la perdition. Et il se leva et partit avec eux.

\par 9 Et son ânesse arriva par le chemin du désert et vit l'ange, et il ouvrit les yeux de Balaam et il vit l'ange et l'adora sur la terre. Et l'ange lui dit : Dépêche-toi et va, car ce que tu dis arrivera pour lui.

\par 10 Et il vint au pays de Moab, bâtit un autel et offrit des sacrifices. Et après avoir vu une partie du peuple, l'esprit de Dieu ne demeura pas en lui, et il reprit sa parabole et dit : Voici, Balac m'a amené ici sur la montagne, en disant : Viens, cours au feu de ces hommes. [Lo] Je ne peux pas supporter ce [feu] que les eaux éteignent, mais ce feu qui consume l'eau, qui le supportera ? Et il lui dit : Il est plus facile d'enlever les fondations et toute leur partie supérieure, d'éteindre la lumière du soleil et d'obscurcir l'éclat de la lune, que pour celui qui veut déraciner la plantation du Le plus puissant ou gâter sa vigne. Et Balac lui-même ne le sait pas, parce que son esprit est enflé dans l'intention que sa destruction puisse survenir rapidement.

\par 11 Car voici, je vois l'héritage que le Tout-Puissant m'a montré pendant la nuit, et voici les jours viennent où Moab sera étonné de ce qui lui arrive, car Balac désirait persuader le Tout-Puissant par des cadeaux et acheter une décision. Avec de l'argent. N'aurais-tu pas dû lui demander ce qu'il avait envoyé sur Pharaon et sur son pays pour les asservir ? Voici une vigne qui porte de l'ombre, extrêmement désirable, et qui serait jaloux d'elle, car elle ne se dessèche pas ? Mais si quelqu'un dit dans son conseil que le Tout-Puissant a travaillé en vain ou l'a choisi en vain, voici maintenant je vois le salut de la délivrance qui doit lui parvenir. Je suis retenu dans le discours de ma voix et je ne peux pas exprimer ce que je vois de mes yeux, car il ne me reste qu'un peu du Saint-Esprit qui demeure en moi, puisque je sais qu'en cela j'ai été persuadé de Balac. j'ai perdu les jours de ma vie:

\par 12 Voici, je vois encore une fois l'héritage de la demeure de ce peuple, et sa lumière brille plus que l'éclat de l'éclair, et sa course est plus rapide que les flèches. Et le temps viendra où Moab gémira, et ceux qui servent Cham (Kemosh ?) seront faibles, même ceux qui ont pris ce conseil contre eux. Mais je grincerai des dents parce que j'ai été trompé et que j'ai transgressé ce qui m'a été dit pendant la nuit. Pourtant, ma prophétie restera manifeste, et mes paroles vivront, et les sages et les prudents se souviendront de mes paroles, car lorsque j'ai maudis, j'ai péri, et bien que j'ai béni, je n'ai pas été béni. Et après avoir dit cela, il se tut. Et Balac dit : Ton Dieu t'a volé de nombreux dons de ma part.

\par 13 Alors Balaam lui dit : Viens et dis-nous ce que tu vas leur faire. Choisissez les femmes les plus jolies qui sont parmi vous et qui sont à Madian et présentez-les devant elles nues et parées d'or et de bijoux, et quand ils les verront et coucheront avec elles, ils pécheront contre leur Seigneur et tomber entre vos mains, car autrement vous ne pourrez pas les soumettre.

\par 14 Et en disant cela, Balaam se détourna et retourna à sa place. Et ensuite le peuple s'égara après les filles de Moab, car Balac fit tout ce que Balaam lui avait montré.

\chapter{19}

\par 1 En ce temps-là, Moïse tua les nations et donna la moitié du butin au peuple, et il commença à leur annoncer les paroles de la loi que Dieu leur avait dites à Oreb.

\par 2 Et il leur parla, disant : Voici, je couche avec mes pères, et j'irai vers mon peuple. Mais je sais que vous vous lèverez et abandonnerez les paroles qui vous ont été ordonnées par moi, et que Dieu sera en colère contre vous et vous abandonnera et quittera votre pays, et amènera contre vous ceux qui vous haïssent, et ils domineront. sur vous, mais pas jusqu'à la fin, car il se souviendra de l'alliance qu'il a conclue avec vos pères.

\par 3 Mais alors vous et vos fils et toutes vos générations après vous vous lèverez et chercherez le jour de ma mort et direz dans leur cœur : Qui nous donnera un berger comme Moïse, ou un autre juge pour les enfants de Israël, pour prier pour nos péchés à tout moment et pour être entendu pour nos iniquités ?

\par 4 Cependant, aujourd'hui, j'en prends à témoin contre vous le ciel et la terre, car le ciel entendra ceci et la terre comprendra de ses oreilles que Dieu a révélé la fin du monde, afin qu'il puisse faire alliance avec vous. sur ses hauts lieux, et il a allumé parmi vous une lampe éternelle. Rappelez-vous, vous méchants, que lorsque je vous ai parlé, vous avez répondu en disant : Tout ce que Dieu nous a dit, nous l'écouterons et le ferons. Mais si nous transgressons ou corrompons nos voies, il prendra un témoin contre nous et nous retranchera.

\par 5 Mais sachez que vous avez mangé du pain des anges pendant 40 ans. Et maintenant voici, je bénis vos tribus, avant que ma fin vienne. Mais vous, connaissez le travail dans lequel j'ai travaillé avec vous depuis le jour où vous êtes sortis du pays d'Égypte.

\par 6 Et après avoir dit cela, Dieu lui parla pour la troisième fois, disant : Voici, tu t'endors avec tes pères, et ce peuple se lèvera et me cherchera, et oubliera ma loi par laquelle je l'ai éclairé. et j'abandonnerai leur semence pendant un temps.

\par 7 Mais je te montrerai le pays avant que tu meures, mais tu n'y entreras pas dans ce siècle, de peur que tu ne voies les images taillées par lesquelles ce peuple sera trompé et détourné du chemin. Je te montrerai le lieu où ils me serviront 740 (l. 850) ans. Et ensuite, il sera livré entre les mains de leurs ennemis, et ils le détruiront, et des étrangers l'entoureront, et il en sera ce jour-là comme il était au jour où je brisai les tables de l'alliance que j'ai faite. avec toi à Oreb ; et quand ils ont péché, ce qui y était écrit a disparu. Or, ce jour était le 17ème jour du 4ème mois.

\par 8 Et Moïse monta sur le mont Oreb, comme Dieu le lui avait ordonné, et pria, disant : Voici, j'ai accompli le temps de ma vie, même 120 ans. Et maintenant, je te prie, que ta miséricorde soit avec ton peuple et que ta compassion se poursuive envers ton héritage, Seigneur, et ta longanimité à ta place envers la race de ton choix, car tu les as aimés plus que tous.

\par 9 Et tu sais que j'étais un berger de moutons, et que lorsque je faisais paître le troupeau dans le désert, je les amenais sur ton mont Oreb, et alors pour la première fois j'ai vu ton ange dans le feu hors du buisson ; mais tu m'as appelé hors du buisson, et j'ai eu peur et j'ai détourné mon visage, et tu m'as envoyé vers eux, et tu les as délivrés d'Egypte, et tu as coulé leurs ennemis dans l'eau. Et tu leur as donné une loi et des jugements selon lesquels ils devaient vivre. Car quel est l'homme qui n'a pas péché contre toi ? Comment ton héritage sera-t-il établi si tu n’as pas pitié d’eux ? Ou qui naîtra sans péché ? Pourtant tu les corrigeras pour un temps, mais pas avec colère.

\par 10 Alors l'Éternel lui montra le pays et tout ce qu'il contient, et dit : C'est ici le pays que je donnerai à mon peuple. Et il lui montra le lieu d'où les nuages ​​puisent de l'eau pour arroser toute la terre, et le lieu d'où le fleuve reçoit son eau, et le pays d'Égypte, et le lieu du firmament, d'où seule la terre sainte boit. Il lui montra aussi les lieux d'où il pleuvait la manne pour le peuple, et même jusqu'aux sentiers du paradis. Et il lui montra les mesures du sanctuaire, et le nombre des offrandes, et le signe par lequel les hommes interpréteront (lit. commenceront à regarder) le ciel, et dit : Ce sont les choses qui étaient interdites aux fils de les hommes parce qu'ils ont péché.

\par 11 Et maintenant, ton bâton avec lequel les signes ont été accomplis sera pour témoin entre moi et mon peuple. Et quand ils pécheront, je serai en colère contre eux et je me souviendrai de mon bâton, et je les épargnerai selon ma miséricorde, et ton bâton sera à mes yeux en souvenir tous les jours, et sera semblable à l'arc avec lequel j'ai fait une alliance. avec Noé quand il sortit de l'arche, disant : Je placerai mon arc dans la nuée, et ce sera un signe entre moi et les hommes que l'eau du déluge ne sera plus sur la terre.

\par 12 Mais je vais te prendre d'ici et te donner le sommeil avec tes pères et te donner du repos dans ton sommeil, et t'enterrer en paix, et tous les anges se lamenteront sur toi, et les armées des cieux seront tristes. Mais aucun ange ou homme ne connaîtra ton sépulcre dans lequel tu vas être enterré, mais tu y reposeras jusqu'à ce que je visite le monde et que je te fasse sortir, toi et tes pères, de la terre [d'Égypte] où vous serez. dormez, et vous vous réunirez et habiterez dans une demeure immortelle qui n'est pas soumise au temps.

\par 13 Mais ce ciel sera à mes yeux comme un nuage passager, et comme hier quand il sera passé, et ce sera quand je m'approcherai pour visiter le monde, je commanderai les années et fixerai les temps, et ils sera raccourcie, et les étoiles se hâteront, et la lumière du soleil accélérera son coucher, et la lumière de la lune ne durera pas non plus, parce que je me hâterai de vous réveiller qui dormez, dans le lieu de sanctification que j'ai montré toi, tous ceux qui peuvent vivre peuvent y habiter.

\par 14 Et Moïse dit : Si je peux encore te demander une chose, ô Seigneur, selon la multitude de ta miséricorde, ne sois pas en colère contre moi. Et montre-moi quelle mesure du temps s'est écoulée et ce qui reste.

\par 15 Et le Seigneur lui dit : Un instant, la partie supérieure d'une main, la plénitude d'un instant et la goutte d'une coupe. Et le temps a tout accompli. Car 4½ sont passés et 2½ restent.

\par 16 Et Moïse, lorsqu'il entendit, fut rempli de compréhension, et sa ressemblance fut glorieusement changée ; et il mourut dans la gloire selon la bouche de l'Éternel, et il l'enterra comme il le lui avait promis, et les anges se lamentèrent à cause de son la mort, et les éclairs, les torches et les flèches marchaient devant lui d'un commun accord. Et ce jour-là, le cantique des armées ne fut pas dit à cause du départ de Moïse. Il n'y a pas non plus eu de jour semblable à celui-ci depuis que le Seigneur a créé l'homme sur la terre, et il n'y en aura pas non plus jamais pour lequel il fasse cesser l'hymne des anges à cause d'un homme ; car il l'aimait beaucoup; et il l'enterra de ses propres mains sur un lieu élevé de la terre, et à la lumière du monde entier.

\chapter{20}

\par 1 Et à ce moment-là, Dieu fit son alliance avec Jésus, fils de Naue, qui restait des hommes qui explorèrent le pays ; car le sort leur était tombé sur qu'ils ne verraient pas le pays parce qu'ils en avaient dit du mal, et c'est pour cette raison que cette génération est morte.

\par 2 Alors Dieu dit à Jésus, fils de Naue : Pourquoi pleures-tu, et pourquoi espères-tu en vain, pensant que Moïse vivra encore ? Maintenant donc tu attends en vain, car Moïse est mort. Prends les vêtements de sa sagesse et mets-les sur toi, et ceins tes reins de la ceinture de sa connaissance, et tu seras changé et tu deviendras un autre homme. N'ai-je pas parlé pour toi à Moïse, mon serviteur, en disant : « Il conduira mon peuple après toi, et je livrerai entre ses mains les rois des Amoréens » ?

\par 3 Et Jésus prit les vêtements de la sagesse, les revêtit, et ceint ses reins de la ceinture de l'intelligence. Et il arriva qu'après qu'il l'eut mis, son esprit s'enflamma et son esprit s'éveilla, et il dit au peuple : Voici, la génération précédente est morte dans le désert parce qu'elle a parlé contre son Dieu. Et voici maintenant, sachez, vous tous, capitaines, que si vous marchez dans les voies de votre Dieu, vos sentiers seront aplanis.

\par 4 Mais si vous n'obéissez pas à sa voix et si vous êtes comme vos pères, vos œuvres seront gâtées, et vous-mêmes brisés, et votre nom périra du pays, et alors où seront les paroles que Dieu a dites à vos des pères ? Car même si les païens disent : Il se peut que Dieu ait échoué parce qu'il n'a pas délivré son peuple, mais s'ils perçoivent qu'il s'est choisi d'autres peuples et qu'il a accompli pour eux de grands prodiges, ils comprendront que le Tout-Puissant n'accepte pas personnes. Mais parce que vous avez péché par vanité, il vous a retiré son pouvoir et vous a soumis. Et maintenant, lève-toi et mets ton cœur à marcher dans les voies de ton Seigneur et il te dirigera.

\par 5 Et le peuple lui dit : Voici, nous voyons aujourd'hui ce qu'Eldad et Modat ont prophétisé du temps de Moïse, disant : Après que Moïse se reposera, la capitainerie de Moïse sera donnée à Jésus, fils de Naue. Et Moïse n'était pas envieux, mais il se réjouissait quand il les entendait ; et désormais tout le peuple crut que tu devais les conduire et leur partager le pays en paix. Et maintenant aussi, s'il y a un conflit, sois fort et agis vaillamment, car tu es le seul à être le chef d'Israël.

\par 6 Et quand il entendit cela, Jésus pensa envoyer des espions à Jéricho. Et il appela Cénez et Sénamias, son frère, les deux fils de Caleph, et leur parla, disant : Moi et votre père avons été envoyés par Moïse dans le désert et nous sommes montés avec dix autres hommes ; et ils sont revenus et ont dit du mal du pays. et il fit fondre le cœur du peuple, et ils furent dispersés et le cœur du peuple avec eux. Mais moi et ton père n'avons fait qu'accomplir la parole du Seigneur, et voici, nous sommes vivants aujourd'hui. Et maintenant je vais vous envoyer explorer le pays de Jéricho. Faites comme votre père et vous vivrez aussi.

\par 7 Et ils montèrent et explorèrent la ville. Et quand ils eurent rapporté la nouvelle, le peuple monta, assiégea la ville et la brûla au feu.

\par 8 Et après que Moïse fut mort, la manne cessa de descendre pour les enfants d'Israël, et alors ils commencèrent à manger les fruits du pays. Et ce sont là les trois choses que Dieu a données à son peuple pour trois personnes, c'est-à-dire le puits d'eau de Mara pour l'amour de Maria, et la colonne de nuée pour l'amour d'Aaron, et la manne pour l'amour de Moïse. Et quand ces trois-là prirent fin, ces trois dons leur furent retirés.

\par 9 Or, le peuple et Jésus combattirent contre les Amoréens, et lorsque la bataille devint intense contre leurs ennemis pendant toute la vie de Jésus, 30 et 9 rois qui habitaient dans le pays furent retranchés. Et Jésus donna le pays par tirage au sort au peuple, à chaque tribu, selon le sort, selon qu'il avait reçu le commandement.

\par 10 Alors Caleph vint vers lui et lui dit : Tu sais comment nous avons tous deux été envoyés par Moïse pour aller avec les espions, et parce que nous avons accompli la parole de l'Éternel, voici, nous sommes vivants aujourd'hui ; et maintenant si si cela te plaît, que soit donné à mon fils Cénez, en guise de portion, le territoire des trois (ou la tribu des) tours. Et Jésus le bénit et il le fit.

\chapter{21}

\par 1 Et quand Jésus fut devenu vieux et prospère en années, Dieu lui dit : Voici, tu vieillis et tu vieillis en jours, et le pays est devenu très grand, et il n'y a personne pour le diviser (ou tire-le au sort), et ce sera après ton départ que ce peuple se mêlera aux habitants du pays et s'égarera après d'autres dieux, et je les abandonnerai comme je l'ai témoigné dans ma parole à Moïse ; mais rends-leur témoignage avant de mourir.

\par 2 Et Jésus dit : Tu sais mieux que tout, ô Seigneur, ce qui émeut le cœur de la mer avant qu'elle ne se déchaîne, et tu as traqué les constellations et compté les étoiles, et ordonné la pluie. Tu connais l'esprit de toutes les générations avant leur naissance. Et maintenant, Seigneur, donne à ton peuple un cœur de sagesse et un esprit de prudence, et quand tu donneras ces ordonnances à ton héritage, ils ne pécheront pas devant toi et tu ne seras pas en colère contre eux.

\par 3 Ne sont-ce pas là les paroles que j'ai dites devant toi, Seigneur, quand Achar a dérobé la malédiction et que le peuple a été livré devant toi, et j'ai prié devant toi et j'ai dit : N'était-ce pas mieux pour nous, ô Seigneur ? , si nous étions morts dans la mer Rouge, où tu as noyé nos ennemis ? ou si nous étions morts dans le désert, comme nos pères, pour être livrés entre les mains des Amoréens et que nous soyons exterminés pour toujours ?

\par 4 Mais si ta parole s'adresse à nous, aucun mal ne nous arrivera ; car, même si notre fin est éloignée de la mort, toi qui es avant le monde et après le monde, tu vis ; et tandis qu'un homme ne peut pas imaginer comment placer une génération avant une autre, il dit : « Dieu a détruit son peuple qu'il a choisi » : et voici, nous serons en enfer : pourtant tu rendras vivante ta parole. Et maintenant, que la plénitude de ta miséricorde soit patiente envers ton peuple, et choisis pour ton héritage un homme qui régnera sur ton peuple, lui et sa génération.

\par 5 N'est-ce pas pour cela que notre père Jacob a dit : Un prince ne quittera pas Juda, ni un chef de ses reins. Et maintenant, confirme les paroles prononcées autrefois, afin que les nations de la terre et les tribus du monde apprennent que tu es éternel.

\par 6 Et il dit en outre : Ô Seigneur, voici, les jours viendront et la maison d'Israël sera comme une colombe couvante qui couche ses petits dans le nid et ne les abandonnera pas ni n'oubliera sa place. De même, ceux-ci se détourneront de leurs actes et lutteront contre le salut qui leur naîtra.

\par 7 Et Jésus descendit de Galgala et bâtit un autel de très grandes pierres, et n'apporta pas de fer dessus, comme Moïse l'avait ordonné, et dressa de grandes pierres sur le mont Gebal, et les blanchit et écrivit dessus les paroles du la loi très clairement ; il rassembla tout le peuple et lisa à ses oreilles toutes les paroles de la loi.

\par 8 Et il descendit avec eux et offrit des offrandes de paix sur l'autel, et ils chantèrent de nombreuses louanges, et élevèrent l'arche de l'alliance du Seigneur hors du tabernacle avec des tambourins et des danses et des luths et des harpes et des psalteries et tous les instruments au son doux.

\par 9 Et les prêtres et les Lévites montaient devant l'arche et se réjouissaient avec des psaumes, et ils placèrent l'arche devant l'autel, et y élevèrent encore une fois des offrandes de paix en très grand nombre, et toute la maison d'Israël chantait ensemble avec une voix forte disant : Voici, notre Seigneur a accompli ce qu'il avait dit à nos pères en disant : Je donnerai à ta semence un pays pour habiter, un pays où coulent le lait et le miel. Et voici, il nous a amenés. dans le pays de nos ennemis et il les a livrés devant nous le cœur brisé, et il est le Dieu qui a envoyé à nos pères dans les lieux secrets des âmes, disant : Voici, l'Éternel a fait tout ce qu'il nous a dit. Et maintenant, nous savons en vérité que Dieu a confirmé toutes les paroles de la loi qu'il nous a dites à Oreb ; et si notre cœur garde ses voies, tout ira bien pour nous, ainsi que pour nos fils après nous.

\par 10 Et Jésus les bénit et dit : Le Seigneur accorde à votre cœur de demeurer en lui (ou en lui) tous les jours, et si vous ne vous écartez pas de son nom, l'alliance du Seigneur durera avec vous. Et il accorde qu'il ne soit pas corrompu, mais que la demeure de Dieu soit bâtie parmi vous, comme il l'a dit lorsqu'il vous a envoyé dans son héritage avec joie et allégresse.



\chapter{22}

\par 1 Et il arriva après ces choses, lorsque Jésus et tout Israël apprirent que les enfants de Ruben et les enfants de Gad et la demi-tribu de Manassé qui habitaient autour du Jourdain leur avaient bâti un autel et y offraient des sacrifices et Ayant nommé des prêtres pour le sanctuaire, tout le peuple fut troublé au-delà de toute mesure et vint vers eux à Silon.

\par 2 Et Jésus et tous les anciens leur parlèrent en disant : Quelles sont ces œuvres qui se font parmi vous, alors que nous ne sommes pas encore établis dans notre pays ? Ne sont-ce pas là les paroles que Moïse vous a dites dans le désert, disant : Gardez-vous, lorsque vous entrez dans le pays, de ne pas gâter vos actions et de ne corrompre pas tout le peuple ? Et maintenant, pourquoi nos ennemis ont-ils tant abondé, sauf parce que vous avez corrompu vos voies et causé tous ces troubles, et c'est pourquoi ils se rassembleront contre nous et nous vaincront.

\par 3 Et les enfants de Ruben et les enfants de Gad et la demi-tribu de Manassé dirent à Jésus et à tout le peuple d'Israël : Voici maintenant, Dieu a élargi le fruit du ventre des hommes, et il a dressé une lumière pour que celui qui est dans les ténèbres peut voir, car il sait ce qu'il y a dans les lieux secrets de l'abîme, et avec lui la lumière demeure. Or, le Seigneur Dieu de nos pères sait si l'un d'entre nous ou si nous avons nous-mêmes fait cela par iniquité, mais seulement à cause de notre postérité, afin que leur cœur ne se sépare pas du Seigneur notre Dieu, de peur qu'ils ne nous disent : Voici maintenant, nos frères qui sont au-delà du Jourdain ont un autel pour y faire des offrandes, mais nous, dans ce lieu qui n'a pas d'autel, éloignons-nous de l'Éternel notre Dieu, parce que notre Dieu nous a éloignés de ses voies, que nous ne devrions pas le servir.

\par 4 Et alors, en vérité, nous avons dit entre nous : Faisons-nous un autel, afin qu'ils aient le zèle de chercher le Seigneur. Et en vérité, certains d’entre nous sont là et savent que nous sommes vos frères et que nous sommes innocents devant vous. Faites donc ce qui plaît au Seigneur.

\par 5 Et Jésus dit : Le Seigneur notre roi n'est-il pas plus puissant que de courtiser les sacrifices ? Et pourquoi n'avez-vous pas enseigné à vos fils les paroles du Seigneur que vous avez entendues à notre sujet ? Car si vos fils avaient été occupés à méditer la loi du Seigneur, leur esprit ne se serait pas détourné après un sanctuaire fait de main d'homme. Ou ne savez-vous pas que lorsque le peuple fut abandonné un instant dans le désert lorsque Moïse monta pour recevoir les tables, leur esprit s'égara et se fit des idoles ? Et si la miséricorde du Dieu de vos pères ne nous avait gardés, toutes les synagogues auraient dû devenir un sujet de vanité, et tous les péchés du peuple auraient été consumés à cause de votre folie.

\par 6 C'est pourquoi maintenant, allez creuser les sanctuaires que vous vous avez bâtis, et enseignez la loi à vos fils, et ils y méditeront jour et nuit, afin que l'Éternel soit avec eux pour témoin et juge pour tous. les jours de leur vie. Et Dieu sera témoin et juge entre moi et vous, et entre mon cœur et votre cœur, que si vous avez fait cela avec subtilité, vous serez vengés, parce que vous voudriez détruire vos frères ; mais si vous l'avez fait par ignorance comme vous le dites, Dieu vous fera miséricorde à cause de vos fils. Et tout le peuple répondit : Amen, Amen.

\par 7 Et Jésus et tout le peuple d'Israël offrèrent pour eux 1 000 béliers en sacrifice pour le péché (lit. la parole d'excuse), et prièrent pour eux et les renvoyèrent en paix : et ils allèrent et détruisirent le sanctuaire, et ils jeûnèrent et pleurèrent, eux et leurs fils, et prièrent et dirent : Ô Dieu de nos pères, qui sais devant le cœur de tous les hommes, tu sais que nos voies n'ont pas été tracées dans l'iniquité devant toi, et que nous ne nous sommes pas non plus écartés de ton mais je t'ai tous servi, car nous sommes l'ouvrage de tes mains. Souviens-toi donc maintenant de ton alliance avec les fils de tes serviteurs.

\par 8 Et après cela, Jésus monta à Galgala, et éleva le tabernacle du Seigneur, et l'arche de l'alliance et tous ses ustensiles, et l'installa à Silo, et y mit la démonstration et la vérité (c'est-à-dire l'Urim et le Thummim). Et à ce moment-là, Éléazar, le prêtre qui servait l'autel, enseignait par la démonstration tous ceux du peuple qui venaient consulter le Seigneur, car cela leur avait été montré, mais dans le nouveau sanctuaire qui était à Galgala, Jésus désigna même jusqu'à ce jour, les holocaustes offerts chaque année par les enfants d'Israël.

\par 9 Car jusqu'à ce que la maison de l'Éternel fût bâtie à Jérusalem, et aussi longtemps que les offrandes étaient faites dans le nouveau sanctuaire, il n'était pas interdit au peuple d'y offrir, parce que la vérité et la démonstration révélaient toutes choses à Silo. Et jusqu'à ce que Salomon place l'arche dans le sanctuaire de l'Éternel, ils y sacrifièrent jusqu'à ce jour-là. Mais Éléazar, fils d'Aaron, prêtre de l'Éternel, servait à Silo.



\chapter{23}

\par 1 Et Jésus, fils de Naué, ordonna au peuple et lui partagea le pays, étant un homme vaillant et vaillant. Et pendant que les adversaires d'Israël étaient encore dans le pays, les jours de Jésus approchaient où il devait mourir. Il envoya appeler tout Israël dans tout son pays, avec leurs femmes et leurs enfants, et leur dit : Rassemblez-vous devant l'arche de l'alliance du Seigneur à Silo et je ferai alliance avec toi avant de mourir.

\par 2 Et quand tout le peuple fut rassemblé le 16ème jour du 3ème mois devant la face du Seigneur à Silo avec leurs femmes et leurs enfants, Jésus leur dit : Écoutez, ô Israël, voici, je fais avec vous le alliance de cette loi que le Seigneur a ordonnée avec nos pères à Oreb, et restez donc ici cette nuit et voyez ce que Dieu me dira à votre sujet.

\par 3 Et pendant que le peuple attendait là cette nuit-là, le Seigneur apparut à Jésus dans une vision et dit : C'est selon toutes ces paroles que je parlerai à ce peuple.

\par 4 Et Jésus vint le matin et rassembla tout le peuple et leur dit : Ainsi parle le Seigneur : Il y avait là un rocher d'où j'ai extrait votre père, et la taille de ce rocher a produit deux hommes, dont les noms étaient Abraham et Nachor, et du cisaillement de ce lieu naquirent deux femmes nommées Sara et Melcha. Et ils habitèrent ensemble au-delà du fleuve. Et Abraham prit Sara pour femme et Nachor prit Melcha.

\par 5 Et lorsque les gens du pays étaient égarés, chacun selon ses propres projets, Abraham a cru en moi et ne s'est pas laissé égarer après eux. Et je l'ai sauvé du feu, je l'ai pris et je l'ai amené dans tout le pays de Chanaan. Et je lui parlai dans une vision, disant : Je donnerai ce pays à ta postérité. Et il me dit : Voici, tu m'as donné une femme et elle est stérile. Et comment aurai-je la semence de ce ventre enfermé ?

\par 6 Et je lui dis : Prends pour moi un veau de trois ans et une chèvre de trois ans et un bélier de trois ans, une tourterelle et un pigeon. Et il les a pris comme je lui ai commandé. Et je l'endormis et je l'entourai de crainte, et je plaçai devant lui le lieu de feu où les œuvres de ceux qui commettent l'iniquité contre moi seront vengées, et je lui montrai les torches de feu par lesquelles les justes qui ont qui a cru en moi sera éclairé.

\par 7 Et je lui dis : Ceux-ci serviront de témoignage entre moi et toi que je te donnerai le germe d'un ventre enfermé. Et je te comparerai à la colombe, parce que tu m'as reçu la ville que tes fils (commenceront) à bâtir sous mes yeux. Mais la tourterelle, je la comparerai aux prophètes qui naîtront de toi. Et je comparerai le bélier aux sages qui naîtront de toi et qui éclaireront tes fils. Mais je comparerai le veau à la multitude des peuples qui se multiplieront par toi. Et la chèvre, je la comparerai aux femmes dont j'ouvrirai le ventre et qu'elles enfanteront. Ces choses serviront de témoignage entre nous pour que je ne transgresse pas mes paroles.

\par 8 Et je lui ai donné Isaac et je l'ai formé dans le sein de celle qui l'a enfanté, et je lui ai ordonné qu'il le rende promptement et me le rende au 7ème mois. Et c’est pour cela que toute femme qui enfante au septième mois vivra son enfant ; car c’est sur lui que j’ai invoqué ma gloire et que j’ai manifesté l’ère nouvelle.

\par 9 Et je donnai à Isaac Jacob et Ésaü, et à Ésaü je donnai le pays de Séir en héritage. Et Jacob et ses fils descendirent en Égypte. Et les Égyptiens ont humilié vos pères, comme vous le savez, et je me suis souvenu de vos pères, j'ai envoyé Moïse, mon ami, je les ai délivrés de là et j'ai frappé leurs ennemis.

\par 10 Et je les ai fait sortir d'une main haute et je les ai conduits à travers la mer Rouge, et j'ai mis la nuée sous leurs pieds, et je les ai fait sortir à travers les profondeurs, et je les ai amenés sous la montagne Sina, et j'ai courbé les cieux et je suis descendu, j'ai figé la flamme du feu, j'ai bouché les sources de l'abîme, j'ai empêché le cours des étoiles, j'ai apprivoisé le bruit du tonnerre et j'ai éteint la plénitude ; du vent, et j'ai réprimandé la multitude des nuages, et j'ai arrêté leurs mouvements, et j'ai interrompu la tempête des armées, afin que je ne rompe pas mon alliance, car toutes choses ont été émues à ma descente, et toutes choses ont été vivifiées à ma descente. l'avènement, et je n'ai pas permis que mon peuple soit dispersé, mais je lui ai donné ma loi et je l'ai éclairé, afin que s'ils faisaient ces choses, ils puissent vivre et avoir de longs jours et ne pas mourir.

\par 11 Et je vous ai amené dans ce pays et je vous ai donné des vignes. Vous habitez dans des villes que vous n'avez pas bâties. Et j'ai accompli l'alliance que j'avais dite à vos pères.

\par 12 Et maintenant, si vous obéissez à vos pères, je mettrai mon cœur sur vous pour toujours, et je vous éclipserai, et vos ennemis ne combattront plus contre vous, et votre pays sera renommé dans le monde entier et votre postérité sera des élus au milieu des peuples, qui diront : Voici le peuple fidèle ; parce qu'ils ont cru au Seigneur, c'est pourquoi le Seigneur les a délivrés et les a plantés. Et c'est pourquoi je te planterai comme une vigne désirable et je te gouvernerai comme un troupeau bien-aimé, et je chargerai la pluie et la rosée, et elles te rassasieront tous les jours de ta vie.

\par 13 Et à la fin, le sort de chacun de vous sera dans la vie éternelle, à la fois pour vous et pour votre postérité, et je recevrai vos âmes et les laisserai en paix, jusqu'au temps du siècle. s'accomplira, et je vous rendrai à vos pères et vos pères à vous, et ils sauront par vous que ce n'est pas en vain que je vous ai choisis. Ce sont les paroles que le Seigneur m’a dites cette nuit.

\par 14 Et tout le peuple répondit et dit : L'Éternel est notre Dieu, et nous le servirons lui seul. Et tout le peuple fit ce jour-là une grande fête et sa renouvellement pendant 28 jours.

\chapter{24}

\par 1 Et après ces jours, Jésus, fils de Naue, rassembla encore tout le peuple et leur dit : Voici, le Seigneur vous a témoigné aujourd'hui : j'ai pris à témoin le ciel et la terre, si vous continuez pour servir le Seigneur, vous serez pour lui un peuple particulier. Mais si vous ne le servez pas et si vous obéissez aux dieux des Amoréens dans le pays desquels vous habitez, dites-le aujourd'hui devant l'Éternel et partez. Mais moi et ma maison servirons le Seigneur.

\par 2 Et tout le peuple éleva la voix et pleura en disant : Peut-être que l'Éternel nous en jugera dignes, et il vaut mieux pour nous mourir dans sa crainte que d'être détruits hors du pays.

\par 3 Et Jésus, fils de Naue, bénit le peuple, les baisa et leur dit : Que vos paroles soient pour la miséricorde devant notre Seigneur, et qu'il envoie son ange et vous préserve : Souvenez-vous de moi après ma mort, et souvenez-vous de vous. Moïse l'ami du Seigneur. Et que les paroles de l'alliance qu'il a conclue avec toi ne s'éloignent pas de toi tous les jours de ta vie. Et lui les renvoya, et ils laissèrent chacun vers son héritage.

\par 4 Mais Jésus se coucha sur son lit, et envoya appeler Phinées, fils du prêtre Éléazar, et lui dit : Voici maintenant, je vois de mes yeux la transgression de ce peuple par laquelle ils commenceront à tromper ; mais toi, fortifie tes mains pendant le temps que tu es avec eux, et il le baisa, ainsi que son père et ses fils, et le bénit et dit : Le Seigneur, le Dieu de tes pères, dirige tes voies et celles de ce peuple.

\par 5 Et quand il cessa de leur parler, il avança ses pieds dans le lit et coucha avec ses pères. Et ses fils lui posèrent les mains sur les yeux.

\par 6 Et alors tout Israël se rassembla pour l'enterrer, et ils le lamentèrent avec une grande lamentation, et dirent ainsi dans leur lamentation : Pleurez à cause de l'aile de cet aigle rapide, car il s'est envolé loin de nous. Et pleurez sur la force de ce petit lion, car il nous est caché. Qui ira maintenant rapporter à Moïse le juste que nous avons eu quarante ans un chef semblable à lui ? Et ils accomplirent leur deuil et l'enterrèrent de leurs propres mains sur la montagne d'Effraïm et retournèrent chacun à sa tente. Et après la mort de Jésus, le pays d'Israël était en repos.

\chapter{25}

\par 1 Et les Philistins cherchèrent à combattre les hommes d'Israël ; et ils consultèrent l'Éternel et dirent : Montons-nous et combattons-nous contre les Philistins ? et Dieu leur dit : Si vous montez avec un cœur pur, combattez ; mais si ton coeur est souillé, n'y monte pas. Et ils demandèrent encore une fois, disant : Comment saurons-nous si tout le cœur du peuple est pareil ? et Dieu leur dit : Tirez au sort entre vos tribus, et chaque tribu qui sera tirée au sort sera mise à part en un seul sort, et alors vous saurez quel cœur est pur et lequel est souillé.

\par 2 Et le peuple dit : Donnons d'abord un prince sur nous, et tirons au sort. Et l'ange du Seigneur leur dit : Nommez. Et le peuple dit : Qui en sera-t-on digne, Seigneur ? Et l'ange de l'Éternel leur dit : Jetez le sort sur la tribu de Caleb, et celui qui sera désigné par le sort sera votre prince. Et ils jetèrent le sort pour la tribu de Caleb, et le sort tomba sur Cénez, et ils l'établirent chef d'Israël.

\par 3 Et Cénez dit au peuple : Amenez-moi vos tribus et écoutez la parole de l'Éternel. Et le peuple se rassembla et Cénez leur dit : Vous savez ce que Moïse, ami de l'Éternel, vous a ordonné de ne transgresser la loi ni à droite ni à gauche. Et Jésus aussi, qui était après lui, vous a donné le même commandement. Et maintenant, voici, nous avons entendu de la bouche du Seigneur que votre cœur est souillé. Et l'Éternel nous a ordonné de tirer au sort parmi vos tribus pour savoir quel cœur s'est éloigné de l'Éternel notre Dieu. La fureur de la colère ne s'abattra-t-elle pas sur le peuple ? Mais je vous promets aujourd'hui que même si un homme de ma maison sort dans le lot du péché, il ne sera pas sauvé vivant, mais sera brûlé au feu. Et le peuple dit : Tu as prononcé un bon conseil pour l'exécuter.

\par 4 Et les tribus furent amenées devant lui, et on trouva de la tribu de Juda 345 hommes, et de la tribu de Ruben 560, et de la tribu de Siméon 775, et de la tribu de Lévi 150, et de la de la tribu de Zabulon 655 (ou 645), et de la tribu d'Isachar 665, et de la tribu de Gad 380, de la tribu d'Aser 665, et de la tribu de Manassé 480, et de la tribu d'Effraim 468, et de la tribu de Benjamin 267 Et le nombre total de ceux qui furent trouvés par le sort du péché était de 6110. Cénez les prit tous et les enferma en prison, jusqu'à ce qu'on sache ce qu'on devait faire d'eux.

\par 5 Et Cénez dit : N'est-ce pas de cela que Moïse, l'ami de l'Éternel, a parlé, disant : Il y a parmi vous une racine forte, qui produit du fiel et de l'amertume ? Maintenant béni soit le Seigneur qui a révélé tous les desseins de ces hommes, et il ne leur a pas non plus permis de corrompre son peuple par leurs mauvaises œuvres. Apportez donc ici la démonstration et la vérité, et appelez le prêtre Éléazar, et consultons le Seigneur par lui.

\par 6 Alors Cénez et Éléazar et tous les anciens et toute la synagogue prièrent d'un commun accord en disant : Seigneur, Dieu de nos pères, révèle la vérité à tes serviteurs, car nous ne croyons pas aux merveilles que tu as faites pour nos pères depuis tu les as fait sortir du pays d'Égypte jusqu'à ce jour. Et le Seigneur répondit et dit : Interrogez d'abord ceux qui ont été trouvés, et qu'ils confessent les actes qu'ils ont commis subtilement, et ensuite ils seront brûlés au feu.

\par 7 Et Cenez les fit sortir et leur dit : Voici, vous savez maintenant comment Achiar a avoué lorsque le sort est tombé sur lui et a déclaré tout ce qu'il avait fait. Et maintenant, déclarez-moi toute votre méchanceté et vos inventions : qui sait, si vous nous dites la vérité, même si vous mourez maintenant, Dieu aura néanmoins pitié de vous quand il ressuscitera les morts ?

\par 8 Et l'un d'eux, nommé Élas, lui dit : La mort ne viendra-t-elle pas maintenant sur nous, et nous mourrons par le feu ? Néanmoins, je te le dis, mon Seigneur, il n'y a pas d'inventions semblables à celles-ci que nous avons faites méchamment. Mais si tu veux rechercher clairement la vérité, interroge-le individuellement auprès des hommes de chaque tribu, et ainsi quelqu'un d'entre eux qui sera là percevra la différence de ses péchés.

\par 9 Et Cénez les interrogea sur sa propre tribu et ils lui dirent : Nous désirions imiter et fabriquer le veau qu'ils fabriquaient dans le désert. Et après cela, il interrogea les hommes de la tribu de Ruben, qui dirent : Nous désirions offrir des sacrifices aux dieux des habitants du pays. Et il interrogea les hommes de la tribu de Lévi, qui dirent : Nous voudrions vérifier si le tabernacle est saint. Et il interrogea le reste de la tribu d'Isacar, qui dit : Nous examinerions les mauvais esprits des idoles, pour voir s'ils se révélaient clairement. Et il interrogea les hommes de la tribu de Zabulon, qui dirent : Nous désirions manger la chair de nos enfants et pour savoir si Dieu prend soin d'eux. Et il interrogea le reste de la tribu de Dan, qui dit : Les Amoréens nous ont enseigné ce qu'ils faisaient, afin que nous puissions l'enseigner à nos enfants. Et voici, ils sont cachés sous la tente d'Élas, qui t'a dit de nous consulter. Envoie donc et tu les trouveras. Et Cenez les a envoyés et les a trouvés.

\par 10 Puis il interrogea ceux qui restaient de la tribu de Gad, et ils dirent : Nous avons commis adultère les unes avec les autres. Et il interrogea ensuite les hommes de la tribu d'Aser, qui dirent : Nous avons trouvé sept images d'or que les Amoréens appelaient les saintes Nymphes, et nous les avons prises avec les pierres précieuses qui étaient serties dessus, et nous les avons cachées. ils sont déposés sous le sommet du mont Sychem. Envoie donc et tu les trouveras. Et Cénez envoya des hommes et les fit partir de là.

\par 11 Or, ce sont les nymphes qui, lorsqu'elles étaient invoquées, montraient aux Amoréens leurs œuvres à chaque heure. Car ce sont là ceux qui ont été imaginés par sept hommes méchants après le déluge, dont les noms sont ceux-ci : [? Cham] Chanaan, Phuth, Selath, Nembroth, Elath, Desuath. Il n'y aura plus aucune similitude semblable dans le monde gravée par la main de l'artisan et ornée de diverses peintures, mais elles ont été érigées et fixées pour la consécration (c'est-à-dire le lieu saint ?) des idoles. Or, les pierres étaient précieuses, apportées du pays d'Euilath, parmi lesquelles se trouvaient un cristal et une prase (ou une cristalline et une verte), et elles montraient leur mode, étant taillées à la manière d'une pierre percée d'un ajouré, et une autre était gravée sur le dessus, et une autre comme marquée de taches (ou comme une chrysoprase tachetée) si brillante avec sa gravure comme si elle montrait l'eau des profondeurs situées en dessous.

\par 12 Et ce sont là les pierres précieuses que les Amoréens possédaient dans leurs lieux saints, et leur prix était au-dessus de toute estimation. Car quand quelqu'un entrait la nuit, il n'avait pas besoin de la lumière d'une lanterne, tant la lumière naturelle des pierres brillait. Celui-là donnait la plus grande lumière, celui qui était taillé selon la forme d'une pierre percée d'ajourés et nettoyé avec des poils ; car si quelqu'un des Amoréens était aveugle, il allait y poser ses yeux et recouvrait la vue. Lorsque Cénez les trouva, il les mit à part et les rangea jusqu'à ce qu'il sache ce qu'ils deviendraient.

\par 13 Et après cela, il interrogea ceux qui restaient de la tribu de Manassé, et ils dirent : Nous n'avons fait que profaner les sabbats de l'Éternel. Et il interrogea les abandonnés de la tribu d'Effraïm, qui dirent : Nous désirions faire passer nos fils et nos filles par le feu, afin de savoir si ce qui a été dit était manifeste. Et il interrogea les abandonnés de la tribu de Benjamin, qui dit : Nous désirions en ce moment examiner le livre de la loi, si Dieu avait clairement écrit ce qui y était, ou si Moïse l'avait enseigné de lui-même.

\chapter{26}

\par 1 Et lorsque Cénez eut pris toutes ces paroles et les eut écrites dans un livre et les eut lues devant le Seigneur, Dieu lui dit : Prends les hommes et ce qui était trouvé avec eux et tous leurs biens et mets-les dans le lit de le fleuve Phison, et je les brûlerai au feu afin que ma colère cesse d'eux.

\par 2 Et Cénez dit : Devons-nous aussi brûler ces pierres précieuses au feu, ou te les sanctifier, car parmi nous il n'y en a pas comme elles ? Et Dieu lui dit : Si Dieu devait recevoir en son nom quelqu'un de ce qui est maudit, que ferait l'homme ? Prends donc maintenant ces pierres précieuses et tout ce qui a été trouvé, tant les livres que les hommes ; et quand tu agiras ainsi avec les hommes, mets à part ces pierres avec les livres, car le feu ne suffira pas à les brûler, et ensuite je te montrerai comment tu dois les détruire. Mais tu brûleras au feu les hommes et tout ce qui a été trouvé. Et tu rassembleras tout le peuple, et tu leur diras : Ainsi sera-t-il fait à tout homme dont le cœur se détourne de son Dieu.

\par 3 Et quand le feu aura consumé ces hommes, alors les livres et les pierres précieuses qui ne peuvent être brûlées par le feu, ni taillées avec du fer, ni effacées par l'eau, les déposeront sur le sommet de la montagne, à côté du nouvel autel ; et je commanderai à une nuée, et elle ira prendre la rosée et la répandra sur les livres, et effacera ce qui y est écrit, car ils ne peuvent être effacés avec aucune autre eau que celle qui n'a jamais servi les hommes. Et ensuite j'enverrai mon éclair, et il brûlera les livres eux-mêmes.

\par 4 Mais quant aux pierres précieuses, j'ordonnerai à mon ange et il les prendra et ira les jeter dans les profondeurs de la mer, et je chargerai les profondeurs et il les engloutira, car elles ne pourront pas continuer. dans le monde parce qu'ils ont été pollués par les idoles des Amoréens. Et j'ordonnerai à un autre ange, et il prendra pour moi douze pierres du lieu d'où ces sept ont été prises ; et toi, quand tu les trouveras au sommet de la montagne où il les posera, prends-les et mets-les sur l'épaulette contre les douze pierres que Moïse y a placées dans le désert, et sanctifie-les dans le pectoral (litt. oracle) selon les douze tribus ; et ne dis pas : Comment saurai-je quelle pierre je dresserai pour quelle tribu ? Voici, je te dirai le nom de la tribu qui répond au nom de la pierre, et tu trouveras l'une et l'autre gravées.

\par 5 Et Cénez s'en alla et prit tout ce qui avait été trouvé et les hommes avec cela, et rassembla de nouveau tout le peuple, et leur dit : Voici, vous avez vu toutes les merveilles que Dieu nous a montrées jusqu'à ce jour, et voici. , lorsque nous avons recherché tous ceux qui avaient subtilement comploté du mal contre l'Éternel et contre Israël, Dieu les a révélés selon leurs œuvres, et maintenant maudit soit tout homme qui projette de faire de même parmi vous, frères. Et tout le peuple répondit Amen, Amen. Et après avoir dit cela, il brûla au feu tous les hommes et tout ce qui se trouvait avec eux, sauf les pierres précieuses.

\par 6 Et après cela, Cenez voulut prouver si les pierres pouvaient être brûlées au feu, et il les jeta au feu. Et c'est ainsi que lorsqu'ils y tombèrent, le feu s'éteignit aussitôt. Et Cénez prit du fer pour les briser, et quand l'épée les toucha, le fer fondit ; et ensuite il effaçait au moins les livres avec de l'eau ; mais il arriva que l'eau, lorsqu'elle tomba sur eux, se figea. Et voyant cela, il dit : Béni soit Dieu qui a fait de si grandes merveilles pour les enfants des hommes, qui a fait Adam le premier créé et qui lui a tout montré ; que lorsqu'Adam aurait péché ainsi, il lui refuserait toutes ces choses, de peur que s'il les montrait à la race des hommes, ils n'en auraient la maîtrise.

\par 7 Et après avoir dit cela, il prit les livres et les pierres et les déposa au sommet de la montagne, près du nouvel autel, comme l'Éternel le lui avait ordonné, et prit une offrande de paix et des holocaustes, et offrit sur le nouvel autel 2000, les offrant tous en holocauste. Et ce jour-là, ils célébrèrent une grande fête, lui et tout le peuple ensemble.

\par 8 Et Dieu fit cette nuit-là ce qu'il avait dit à Cénez, car il commanda à une nuée, et elle alla prendre la rosée de la glace du paradis et la répandit sur les livres et les effaça. Et après cela, un ange est venu et les a brûlés, et un autre ange a pris les pierres précieuses et les a jetées au cœur de la mer, et il a chargé la profondeur de la mer, et elle les a engloutis. Et un autre ange alla et apporta douze pierres et les posa solidement à l'endroit d'où il avait pris ces sept. Et il y grava les noms des douze tribus.

\par 9 Et Cénez se leva le lendemain et trouva ces douze pierres au sommet de la montagne où lui-même avait posé ces sept. Et leur taille était telle que si la forme d'yeux y était représentée.

\par 10 Et la première pierre, sur laquelle était écrit le nom de la tribu de Ruben, était comme une pierre à sardine. La deuxième pierre était gravée d'une dent (ou d'ivoire), et là était gravé le nom de la tribu de Siméon, et l'on y voyait l'image d'une topaze ; et sur la troisième pierre était gravé le nom de la tribu de Lévi, et elle ressemblait à une émeraude. Mais la quatrième pierre était appelée cristal, sur laquelle était gravé le nom de la tribu de Juda, et elle était comparée à un anthrax. La cinquième pierre était verte, et dessus était gravé le nom de la tribu d'Isacar, et la couleur d'une pierre de saphir y était. Et sur la sixième pierre, la gravure était comme si elle avait été inscrite, (ou comme une chrysoprase) tachetée de diverses marques, et dessus était écrite la tribu de Zabulon, et la pierre de jaspe lui était comparée.

\par 11 Sur la septième pierre, la tombe brillait et montrait en elle-même, comme si elle enfermait l'eau de l'abîme, et là était écrit le nom de la tribu de Dan, laquelle pierre était comme une ligure. Mais la huitième pierre était taillée en diamant, et là était écrit le nom de la tribu de Neptalim, et elle ressemblait à une améthyste. Et la neuvième pierre fut percée d'une tombe, et elle provenait du mont Ophir, et là-dedans était écrite la tribu de Gad, et une pierre d'agate lui était comparée. Et sur la dixième pierre, la tombe était creusée et ressemblait à une pierre de Théman, et là était écrite la tribu d'Aser, et une chrysolite lui était comparée. Et la onzième pierre était une pierre élue du Liban, sur laquelle était écrit le nom de la tribu de Joseph, et il y avait un béryl. assimilé à cela. Et la douzième pierre fut taillée dans la hauteur de Sion (ou la carrière), et dessus était écrite la tribu de Benjamin ; et la pierre d'onyx lui fut comparée.

\par 12 Et Dieu dit à Cenez : Prends ces pierres et mets-les dans l'arche de l'alliance de l'Éternel avec les tables de l'alliance que j'ai données à Moïse à Oreb, et elles y seront avec elles jusqu'à ce que Jahel se lève pour bâtir. une maison en mon nom, puis il les placera devant moi sur les deux chérubins, et ils seront à mes yeux en mémorial de la maison d'Israël.

\par 13 Et ce sera lorsque les péchés de mon peuple seront accomplis et que leurs ennemis auront le contrôle de leur maison, que je prendrai ces pierres et les premières avec les tables, et les déposerai à l'endroit d'où ils sont partis. ont été engendrés au commencement, et ils seront là jusqu'à ce que je me souvienne du monde et que je visite les habitants de la terre. Et alors je les prendrai, ainsi que bien d'autres, meilleurs qu'eux, de ce lieu que l'œil n'a pas vu, que l'oreille n'a pas entendu et qui n'est pas monté dans le cœur de l'homme, jusqu'à ce que quelque chose de semblable arrive dans le monde, et que les justes auront nul besoin de la lumière du soleil ni de l’éclat de la lune, car la lumière des pierres précieuses sera leur lumière.

\par 14 Et Cénez se leva et dit : Voyez quels biens Dieu a fait pour les hommes, et à cause de leurs péchés ils en ont tous été privés. Et maintenant, je sais aujourd’hui que la race des hommes est faible et que leur vie ne comptera pour rien.

\par 15 Et en disant cela, il prit les pierres de l'endroit où elles étaient posées, et pendant qu'il les prenait, la lumière du soleil se répandait sur elles, et la terre brillait de leur lumière. Et Cénez les plaça dans l'arche de l'alliance de l'Éternel avec les tables, comme cela lui avait été commandé, et ils y sont encore aujourd'hui.

\chapter{27}

\par 1 Après cela, il arma 300 000 hommes du peuple et monta combattre les Amoréens ; il tua le premier jour 800 000 hommes, et le deuxième jour il en tua environ 500 000.

\par 2 Et le troisième jour étant venu, certains hommes du peuple dirent du mal de Cénez, en disant : Voici, Cénez seul couche dans sa maison avec sa femme et ses concubines, et il nous envoie au combat, afin que nous soyons détruits avant. nos ennemis.

\par 3 Et quand les serviteurs de Cénez l'eurent entendu, ils lui apportèrent un message. Et il commanda un capitaine de cinquante personnes, et il fit venir d'eux trente-sept hommes qui parlèrent contre lui et les enfermèrent dans une salle.

\par 4 Et leurs noms sont ceux-ci : Lé et Uz, Betul, Ephal, Dealma, Anaph, Desac, Besac, Gethel, Anael, Anazim, Noac, Cehec, Boac, Obal, Iabal, Enath, Beath, Zelut, Ephor, Ezeth. , Desaph, Abidan, Esar, Moab, Duzal, Azath, Phelac, Igat, Zophal, Eliesor, Ecar, Zebath, Sebath, Nesach et Zere. Et quand le chef des cinquante les eut enfermés comme Cénez l'avait ordonné, Cénez dit : Quand l'Éternel aura opéré le salut de son peuple par ma main, alors je punirai ces hommes.

\par 5 Et en disant cela, Cénez ordonna au chef des cinquante, en disant : Allez choisir parmi mes serviteurs 300 hommes et autant de chevaux, et que personne du peuple ne sache l'heure à laquelle j'irai au combat ; mais seulement à quelle heure je te le dirai, prépare les hommes pour qu'ils soient prêts cette nuit.

\par 6 Et Cénez envoya des messagers, des espions, pour voir où était la multitude du camp des Amoréens. Et les messagers allèrent et espionnèrent, et virent que la multitude du camp des Amoréens se déplaçait parmi les rochers, projetant de venir combattre contre Israël. Et les messagers revinrent et lui rapportèrent cette parole. Et Cénez se leva de nuit, lui et 300 cavaliers avec lui, prit une trompette à la main et commença à descendre avec les 300 hommes. Et il arriva, alors qu'il était près du camp des Amoréens, qu'il dit à ses serviteurs : Restez ici et je descendrai seul et visiterai le camp des Amoréens. Et il en sera ainsi, si je sonne de la trompette, vous descendrez, mais sinon, attendez-moi ici.

\par 7 Et Cénez descendit seul, et avant de descendre, il pria et dit : Seigneur, Dieu de nos pères, tu as montré à ton serviteur les choses merveilleuses que tu as préparées à faire par ton alliance dans les derniers jours : et maintenant, envoie à ton serviteur une de tes merveilles, et je vaincrai tes adversaires, afin qu'eux, ainsi que toutes les nations et ton peuple, sachent que l'Éternel ne délivre pas par la multitude d'une armée, ni par la force des cavaliers, quand ils comprendront le signe de délivrance que tu feras pour moi aujourd'hui (ou les cavaliers, et que toi, Seigneur, tu feras un signe de salut avec moi ce jour). Voici, je vais tirer mon épée du fourreau et elle brillera dans le camp des Amoréens ; et si les Amoréens s'aperçoivent que c'est moi, Cenez, alors je saurai que tu les as livrés entre mes mains. . Mais s’ils ne s’aperçoivent pas que c’est moi et pensent que c’est un autre, alors je saurai que tu ne m’as pas écouté, mais que tu m’as livré à mes ennemis. Mais si je suis effectivement livré à la mort, je saurai qu'à cause de mes iniquités, l'Éternel ne m'a pas entendu et m'a livré à mes ennemis ; mais il ne détruira pas son héritage par ma mort.

\par 8 Et il partit après avoir prié, et entendit la multitude des Amoréens dire : Levons-nous et combattons Israël ; car nous savons que nos saintes Nymphes sont là parmi eux et qu'elles les livreront entre nos mains.

\par 9 Et Cénez se leva, car l'esprit de l'Éternel le revêtit comme un vêtement, et il tira son épée, et quand la lumière de celle-ci brillait sur les Amoréens comme un éclair aigu, ils le virent et dirent : N'est-ce pas là le épée de Cénez qui a rendu nos blessés nombreux ? Maintenant la parole que nous avons prononcée est justifiée, disant que nos saintes Nymphes les ont livrés entre nos mains. Voici, aujourd'hui il y aura une fête pour les Amoréens, quand notre ennemi nous sera livré. Maintenant donc, levez-vous, que chacun ceigne son épée et commence le combat.

\par 10 Et il arriva que lorsque Cénez entendit leurs paroles, il fut revêtu d'un esprit de force et se changea en un autre homme, et descendit dans le camp des Amoréens et commença à les frapper. Et le Seigneur envoya devant lui l'ange Ingethel (ou Gethel), qui est chargé des choses cachées et qui agit de manière invisible, (et un autre) ange puissant pour l'aider. Et Ingethel frappa d'aveuglement les Amoréens, de sorte que tout homme qui voyait son voisin, les considérait comme ses adversaires, et ils s'entre-tuaient. Et l'ange Zérouel, qui est chargé de la force, dénuda les bras de Cénez, de peur qu'ils ne l'aperçoivent ; Et Cénez frappa quarante-cinq mille hommes des Amoréens, et eux-mêmes se frappèrent les uns les autres, et tombèrent quarante-cinq mille hommes.

\par 11 Et lorsque Cénez avait frappé une grande multitude, il aurait desserré sa main de son épée, car le manche de l'épée était clavelé, de sorte qu'il ne pouvait pas être desserré, et sa main droite y avait pris la force de l'épée. .

\par Alors ceux qui restaient des Amoréens s'enfuirent dans les montagnes ; Mais Cénez cherchait comment il pourrait lâcher sa main. Il regarda de ses yeux et vit un homme des Amoréens qui s'enfuyait. Il l'attrapa et lui dit : Je sais que les Amoréens sont rusés. Maintenant donc, montre-moi comment je peux détacher sa main. ma main de cette épée, et je te laisserai partir. Et l'Amoréen dit : Va, prends un homme des Hébreux et tue-le ; et pendant que son sang est encore chaud, mets ta main en dessous et reçois son sang, ainsi ta main sera déliée. Et Cénez dit : Tant que l'Éternel est vivant, si tu avais dit : Prends un homme des Amoréens, j'aurais pris l'un d'eux et je t'aurais sauvé la vie ; mais d'autant que tu as dit « des Hébreux » pour montrer ta haine, ta bouche sera contre toi, et je te ferai selon ce que tu as dit. Et après avoir dit cela, Cénez le tua, et tandis que son sang était encore chaud, il passa sa main en dessous et la reçut dedans, et elle se détacha.

\par 12 Et Cénez partit et ôta ses vêtements, se jeta dans la rivière et se lava, puis remonta et changea de vêtements, et retourna vers ses jeunes hommes. Le Seigneur les plongea dans un profond sommeil pendant la nuit. Ils dormirent et ne savaient rien de tout ce que Cenez avait fait. Et Cénez vint et les réveilla du sommeil ; Et ils le regardèrent de leurs yeux et virent, et voici, le champ était plein de cadavres. Et ils furent étonnés dans leur esprit, et regardèrent chacun son prochain. Et Cénez leur dit : Pourquoi vous étonnez-vous ? Les voies du Seigneur sont-elles celles des hommes ? Car chez les hommes la multitude prévaut, mais devant Dieu ce qu'il prescrit. Et donc, si Dieu a voulu opérer la délivrance de ce peuple par mes mains, pourquoi vous étonnez-vous ? Levez-vous et ceignez chacun de vos épées, et nous rentrerons chez nos frères.

\par 13 Et lorsque tout Israël apprit la délivrance opérée par les mains de Cénez, tout le peuple sortit d'un commun accord à sa rencontre et dit : Béni soit l'Éternel, qui t'a établi chef sur son peuple et qui t'a montré que les choses qu'il t'a dites sont sûres : ce que nous avons entendu par la parole, nous le voyons maintenant de nos yeux, car l'œuvre de la parole de Dieu est manifeste.

\par 14 Et Cénez leur dit : Interrogez maintenant vos frères, et qu'ils vous disent combien ils ont travaillé avec moi dans la bataille. Et les hommes qui étaient avec lui dirent : Aussi vrai que le Seigneur est vivant, nous n'avons pas combattu et nous ne savions rien, sauf qu'à notre réveil, nous avons vu le champ plein de cadavres. Et le peuple répondit : Maintenant, nous savons que lorsque le Seigneur charge d'opérer la délivrance de son peuple, il n'a pas besoin d'une multitude, mais seulement de la sanctification.

\par 15 Et Cénez dit au chef des cinquante qui avait enfermé ces hommes en prison : Faites sortir ces hommes pour que nous puissions entendre leurs paroles. Et après les avoir fait sortir, Cénez leur dit : Dites-moi, qu'avez-vous vu en moi pour que vous murmuriez parmi le peuple ? Et ils dirent : Pourquoi nous demandes-tu ? Pourquoi nous demandes-tu ? Ordonne maintenant que nous soyons brûlés par le feu, car nous ne mourrons pas à cause du péché dont nous avons parlé maintenant, mais à cause du premier dans lequel ont été pris ces hommes qui ont été brûlés dans leurs péchés ; car alors nous avons consenti à leur péché, en disant : Peut-être que le peuple ne nous apercevra pas ; et puis nous avons échappé aux gens. Mais maintenant, nous sommes (à juste titre) devenus un exemple public par nos péchés en ce sens que nous sommes tombés dans la calomnie de toi. Et Cenez dit : Si donc vous-mêmes témoignez contre vous-mêmes, comment aurais-je compassion de vous ? Et Cénez ordonna de les brûler au feu, et de jeter leurs cendres à l'endroit où ils avaient brûlé la multitude des pécheurs, dans le ruisseau Phison.

\par 16 Et Cénez régna sur son peuple cinquante-sept ans, et il y eut de la crainte sur tous ses ennemis pendant toute sa vie.

\chapter{28}

\par 1 Et lorsque les jours de Cénez approchaient où il devait mourir, il envoya appeler tous les hommes (ou tous les anciens), ainsi que les deux prophètes Jabis et Phinées, et Phinées, fils du prêtre Éléazar, et leur dit : : Voici maintenant, le Seigneur m'a montré toutes ses merveilles qu'il a préparées à accomplir pour son peuple dans les derniers jours.

\par 2 Et maintenant, je ferai aujourd'hui mon alliance avec vous, afin que vous n'abandonniez pas l'Éternel, votre Dieu, après mon départ. Car vous avez vu tous les prodiges qui sont survenus à ceux qui ont péché, et tout ce qu'ils ont déclaré, confessant leurs péchés de leur propre gré, et comment le Seigneur notre Dieu a mis fin à eux parce qu'ils avaient transgressé son alliance. C'est pourquoi maintenant, épargnez-leur votre maison et vos fils, et demeurez dans les voies de l'Éternel, votre Dieu, afin que l'Éternel ne détruise pas son héritage.

\par 3 Et Phinées, fils du prêtre Éléazar, dit : Si Cenez, le chef, me l'ordonne, ainsi que les prophètes, le peuple et les anciens, je dirai une parole que j'ai entendue de mon père lorsqu'il mourait, et je ne garderai pas silence sur le commandement qu'il m'a ordonné lors de la réception de son âme. Et Cenez, le gouverneur, et les prophètes dirent : Que Phinées continue à parler. Un autre parlera-t-il devant le prêtre qui garde les commandements du Seigneur notre Dieu, et cela, puisque la vérité sort de sa bouche et que de son cœur une lumière brillante ?

\par 4 Alors Phinées dit : Mon père, alors qu'il était mourant, m'a commandé en disant : Ainsi tu diras aux enfants d'Israël lorsqu'ils seront rassemblés pour l'assemblée : L'Éternel m'est apparu le troisième jour avant cela. dans un rêve pendant la nuit, et il me dit : Voici, tu as vu, toi et ton père avant toi, combien j'ai travaillé pour mon peuple ; et ce sera après ta mort que ce peuple se lèvera et corrompra ses voies, s'écartant de mes commandements, et je serai extrêmement en colère contre lui. Pourtant, je me souviendrai du temps qui était avant les siècles, même du temps où il n'y avait pas d'homme et où il n'y avait pas d'iniquité, quand je disais que le monde devrait exister et que ceux qui viendraient me loueraient en lui, et je Je planterai une grande vigne, et parmi elle je choisirai un plant, je l'ordonnerai et je l'appellerai par mon nom, et il sera à moi pour toujours. Mais quand j'aurai fait tout ce que j'ai dit, néanmoins ma plantation, qui porte mon nom, ne me connaîtra pas, moi qui l'ai planté, mais elle corrompra son fruit et ne me donnera pas son fruit. Voilà les choses que mon père m'a ordonné de dire à ce peuple.

\par 5 Et Cénez éleva la voix, les anciens et tout le peuple d'un commun accord, et pleurèrent avec de grandes lamentations jusqu'au soir et dit : Le berger détruira-t-il son troupeau en vain, s'il ne continue pas à pécher contre lui? Et ne sera-ce pas lui qui épargnera selon l'abondance de sa miséricorde, puisqu'il a dépensé pour nous un grand travail ?

\par 6 Or, pendant qu'ils étaient couchés, l'esprit saint qui demeurait en Cenez sauta sur lui et lui ôta sa sensation corporelle, et il se mit à prophétiser, disant : Voici maintenant, je vois ce que je n'attendais pas, et je comprends que Je ne le savais pas. Écoutez maintenant, vous qui habitez sur la terre, comme ceux qui y ont séjourné ont prophétisé avant moi, quand ils ont vu cette heure, avant même que la terre ne soit corrompue, afin que vous connaissiez les prophéties fixées d'avance, vous tous qui y habitez.

\par 7 Voici maintenant, je vois des flammes qui ne brûlent pas, et j'entends des sources d'eau s'éveiller du sommeil, et elles n'ont aucun fondement, et je ne vois pas non plus les sommets des montagnes, ni la voûte du firmament, mais toutes choses qui ne paraissent pas. et invisibles, qui n'ont aucune place, et bien que mon œil ne sache pas ce qu'il voit, mon cœur découvrira ce qu'il peut apprendre (ou dire).

\par 8 Or, de la flamme que j'ai vue, et elle ne brûlait pas, j'ai vu, et voici, une étincelle est sortie et comme elle s'est construite un aire sous le ciel, et la ressemblance de l'aire de celui-ci était comme une araignée qui tourne. , à la manière d'un bouclier. Et quand le fondement fut posé, je vis, et de cette source s'éleva comme une écume bouillante, et voici, elle se transforma pour ainsi dire en un autre fondement ; et entre les deux fondations, la supérieure et la inférieure, sortaient de la lumière du lieu invisible comme des formes d'hommes, et ils allaient et venaient. Et voici, une voix disait : Celles-ci seront pour un fondation aux hommes et ils y demeureront 7000 ans.

\par 9 Et le fondement inférieur était un pavé et le fondement supérieur était d'écume, et ceux qui sont sortis de la lumière du lieu invisible, ce sont ceux qui y habiteront, et le nom de cet homme est [Adam]. Et ce sera, quand il (ou ils auront) péché contre moi et que le temps sera accompli, que l'étincelle s'éteindra et que la source cessera, et ainsi ils seront changés.

\par 10 Et il arriva, après que Cénez eut dit ces paroles, qu'il se réveilla et que sa raison lui revint ; mais il ne savait pas ce qu'il avait dit ni ce qu'il avait vu, mais ceci seulement il dit au peuple : Si Si le reste des justes le sont après leur mort, il vaut mieux qu'ils meurent dans le monde corruptible, afin qu'ils ne voient pas le péché. Et Cénez ayant dit cela, il mourut et s'endormit avec ses pères, et le peuple le pleura pendant 30 jours.



\chapter{29}

\par 1 Et après ces choses, le peuple établit Zebul comme chef sur eux, et à ce moment-là il rassembla le peuple et leur dit : Voici, nous connaissons tout le travail avec lequel Cénez a travaillé avec nous pendant les jours de sa vie. Or, s'il avait eu des fils, ils auraient dû être princes du peuple. Mais si ses filles sont encore en vie, qu'elles reçoivent un plus grand héritage parmi le peuple, car leur père, de son vivant, a refusé de le leur donner, de peur qu'il ne le leur donne. devrait être qualifié de cupide et avide de gain. Et le peuple dit : Fais tout ce qui te paraît bon.

\par 2 Or Cenez eut trois filles dont les noms sont ceux-ci : Ethema, la première-née, la deuxième Pheila, la troisième Zelpha. Et Zebul donna au premier-né tout ce qui était autour du pays des Phéniciens, et au second il donna le verger d'oliviers d'Accaron, et au troisième tout le champ labouré qui était autour d'Azotus. Et il leur donna des maris, à savoir au premier-né Elisephan, au deuxième Odiel et au troisième Doel.

\par 3 Or, en ces jours-là, Zebul érigea un trésor pour l'Éternel et dit au peuple : Voici, si quelqu'un veut sanctifier pour l'Éternel de l'or et de l'argent, qu'il l'apporte au trésor de l'Éternel à Sylo ; qui a des objets appartenant aux idoles, pensez à les sanctifier dans les trésors de l'Éternel, car l'Éternel ne désire pas les abominations des choses maudites, de peur que vous ne troubliez la synagogue de l'Éternel, car la colère qui passe suffit. Et tout le peuple apporta ce qu'il désirait apporter, hommes et femmes, même de l'or et de l'argent. Et tout ce qu'on apporta fut pesé, et c'était 20 talents d'or et 250 talents d'argent.

\par 4 Et Zebul jugea le peuple pendant vingt-cinq ans. Et quand il eut accompli son temps, il envoya et appela tout le peuple et dit : Voici, maintenant je pars pour mourir. Regardez les témoignages qu'ont rendus ceux qui nous ont précédés, et que votre cœur ne soit pas comme les vagues de la mer, mais comme la vague de la mer ne comprend que ce qui est dans la mer, ainsi votre cœur ne pense également qu’à ce qui appartient à la loi. Et Zebul dormit avec ses pères, et fut enterré dans le sépulcre de son père.

\chapter{30}

\par 1 Alors les enfants d'Israël n'avaient personne qu'ils pussent établir pour juger sur eux ; et leur cœur s'effondra, et ils oublièrent la promesse, et transgressèrent les voies que Moïse et Jésus, les serviteurs du Seigneur, leur avaient commandées, et Ils furent emmenés après les filles des Amoréens et servirent leurs dieux.

\par 2 Et le Seigneur fut en colère contre eux, et envoya son ange et dit : Voici, je m'ai choisi un seul peuple entre toutes les tribus de la terre, et j'ai dit que ma gloire demeurerait avec eux dans ce monde, et je je leur ai envoyé Moïse, mon serviteur, pour leur déclarer ma grande majesté et mes jugements, et ils ont transgressé mes voies. Maintenant donc, voici, je vais attiser leurs ennemis et ils domineront sur eux, et alors tout le peuple dira : Parce que nous avons transgressé les voies de Dieu et de nos pères, c'est pourquoi ces choses nous arrivent. Mais il y aura une femme qui les gouvernera et qui leur donnera de la lumière pendant 40 ans.

\par 3 Et après ces choses, l'Éternel excita contre eux Jabin, roi d'Asor, et il commença à combattre contre eux, et il avait pour chef de sa force Sisara, qui avait 8 000 chars de fer. Et il arriva à la montagne d'Effrem et combattit le peuple. Israël le craignit beaucoup, et le peuple ne put tenir debout pendant toute la durée de Sisara.

\par 4 Et quand Israël fut très humilié, tous les enfants d'Israël se rassemblèrent d'un commun accord sur la montagne de Juda et dirent : Nous nous considérions bienheureux plus que tous les peuples, et maintenant, voici, nous sommes si humiliés, plus que toutes les nations, que nous ne pouvons pas habiter dans notre pays, et que nos ennemis nous dominent. Et maintenant, qui nous a fait tout cela ? N'est-ce pas nos iniquités, parce que nous avons abandonné le Seigneur, le Dieu de nos pères, et que nous avons marché dans des choses qui ne pouvaient nous servir ? Venez donc, jeûnons sept jours, hommes et femmes, et depuis le plus petit jusqu'à l'enfant qui allaite. Qui sait si Dieu se réconciliera avec son héritage et ne détruira pas la plantation de sa vigne ?

\par 5 Et après que le peuple eut jeûné 7 jours, assis dans des sacs, le Seigneur leur envoya le 7ème jour Debbora, qui leur dit : La brebis destinée à l'abattoir peut-elle répondre devant celui qui la tue, quand les deux celui qui tue [ . . . ] et celui qui est tué garde le silence, quand il est parfois irrité contre cela ? Or, vous êtes nés pour être un troupeau devant notre Seigneur. Et il vous a conduit au sommet des nuages, et a soumis des anges sous vos pieds, et vous a établi une loi, et vous a donné des commandements par des prophètes, et vous a châtié par des dirigeants, et vous a montré de nombreux prodiges, et à cause de vous. a ordonné aux luminaires et ils se sont arrêtés aux endroits où ils étaient invités, et lorsque vos ennemis sont tombés sur vous, il a fait pleuvoir sur eux des grêlons et les a détruits, et Moïse et Jésus et Cénez et Zebul vous ont donné des commandements. Et vous ne leur avez pas obéi.

\par 6 Car pendant qu'ils vivaient, vous vous êtes montrés comme obéissants à votre Dieu, mais quand ils sont morts, votre cœur est mort aussi. Et vous êtes devenus semblables au fer qu'on jette dans le feu, qui, lorsqu'il est fondu par la flamme, devient comme de l'eau, mais qui, lorsqu'il sort du feu, redevient dur. De même, vous aussi, pendant que ceux qui vous avertissent vous brûlent, montrez-en l'effet, et quand ils sont morts, vous oubliez toutes choses.

\par 7 Et maintenant, voici, l'Éternel aura compassion de vous aujourd'hui, non à cause de vous, mais à cause de son alliance qu'il a conclue avec vos pères et à cause du serment qu'il a juré de ne pas vous abandonner à cause de vous. jamais. Mais sachez qu'après mon décès, vous commencerez à pécher dans vos derniers jours. C'est pourquoi l'Éternel accomplira parmi vous des choses merveilleuses et livrera vos ennemis entre vos mains. Car vos pères sont morts, mais Dieu, qui a fait alliance avec eux, est vie.

\chapter{31}

\par 1 Et Debbora envoya appeler Barach et lui dit : Lève-toi et ceins tes reins comme un homme, et descends et combats contre Sisara, car je vois les constellations très émues dans leurs rangs et se préparant à combattre pour toi. Je vois aussi les éclairs immobiles dans leur course, et s'avançant pour arrêter les roues des chars de ceux qui se vantent de la puissance de Sisara, qui dit : Je descendrai sûrement dans le bras de ma force pour combattre Israël, et je partagerai leur butin entre mes serviteurs, et je prendrai leurs belles femmes pour concubines. C'est pourquoi l'Éternel a dit à son sujet que le bras d'une femme faible le vaincre, que les jeunes filles prendront son butin, et que lui aussi tombera entre les mains d'une femme.

\par 2 Et lorsque Debbora, le peuple et Barach descendirent à la rencontre de leurs ennemis, aussitôt le Seigneur perturba le mouvement de ses étoiles et leur parla en disant : Hâtez-vous et partez, car nos (ou vos) ennemis tombent sur vous : confondez leurs bras et brisez la force de leurs cœurs, car je suis venu pour que mon peuple triomphe. Car même si mon peuple a péché, j'aurai néanmoins pitié de lui. Et quand cela fut dit, les étoiles sortirent comme cela leur avait été commandé et brûlèrent leurs ennemis. Et le nombre d'entre eux qui furent rassemblés (ou brûlés) et tués en une heure était de 97 000 hommes. Mais ils ne détruisirent pas Sisara, car ainsi cela leur avait été ordonné.

\par 3 Et comme Sisara s'était enfui sur son cheval pour délivrer son âme, Jahel, la femme d'Aber le Cinéen, se para de ses ornements et sortit à sa rencontre. Or, la femme était très belle, et quand elle le vit, elle dit : Entre, mange et dors ; et le soir, j'enverrai mes serviteurs avec toi, car je sais que tu te souviendras de moi et que tu me récompenseras. Et Sisara entra, et quand il vit des roses éparpillées sur le lit, il dit : Si je suis délivré, ô Jahel, j'irai vers ma mère et tu seras (ou Jahel sera) ma femme.

\par 4 Et ensuite Sisara eut soif et il dit à Jahel : Donne-moi un peu d'eau, car je suis faible et mon âme brûle à cause de la flamme que je voyais dans les étoiles. Et Jahel lui dit : Repose-toi un peu et ensuite tu boiras.

\par 5 Et quand Sisara s'endormit, Jahel alla vers le troupeau et en tira le lait. Et pendant qu'elle traitait, elle dit : Voici maintenant, souviens-toi, ô Seigneur, quand tu as divisé chaque tribu et chaque nation sur la terre, tu n'as pas choisi Israël seulement, et tu ne l'as comparé à aucune bête, sauf au bélier qui va. devant le troupeau et le conduit-il ? Voici donc et voyez comment Sisara a pensé dans son cœur en disant : J'irai punir le troupeau du Très-Puissant. Et voici, je prendrai du lait des bêtes auxquelles tu as comparé ton peuple, et j'irai lui donner à boire, et quand il aura bu, il deviendra faible, et après cela je le tuerai. Et ceci sera le signe que tu me donneras, ô Seigneur, que, pendant que Sisara dort, quand j'entre, s'il se réveille et me demande aussitôt en disant : Donne-moi de l'eau à boire, alors je saurai que ma prière a été été entendu.

\par 6 Alors Jahel revint et entra, et Sisara se réveilla et lui dit : Donne-moi à boire, car je brûle fortement et mon âme est enflammée. Et Jahel prit du vin, le mêla avec du lait et lui donna à boire. Il but et s'endormit.

\par 7 Mais Jahel prit un pieu dans sa main gauche et s'approcha de lui en disant : Si l'Éternel me donne ce signe, je saurai que Sisara tombera entre mes mains. Voici, je vais le jeter à terre du lit sur lequel il dort, et s'il ne s'en aperçoit pas, je saurai qu'il est livré. Et Jahel prit Sisara et le poussa du lit sur la terre, mais il ne s'en rendit pas compte, car il était extrêmement affaibli. Et Jahel dit : Fortifie en moi, Seigneur, mon bras aujourd'hui, à cause de toi et de ton peuple, et de ceux qui se confient en toi. Et Jahel prit le pieu, le plaça sur sa tempe et frappa avec le marteau. Et tandis qu'il mourait, Sisara dit à Jahel : Voici, la douleur m'est venue, Jahel, et je meurs comme une femme. Et Jahel lui dit : Va te vanter devant ton père en enfer, et dis-lui que tu es tombé (ou dis : j'ai été livré entre) les mains d'une femme. Et elle y mit fin, le tua et y déposa son corps jusqu'au retour de Barach.

\par 8 Or la mère de Sisara s'appelait Themech, et elle envoya dire à ses amis : Venez, sortons ensemble à la rencontre de mon fils, et vous verrez les filles des Hébreux que mon fils amènera ici pour être son fils. concubines.

\par 9 Mais Barach revint après avoir suivi Sisara et fut très irrité de ne pas l'avoir trouvé, et Jahel sortit à sa rencontre et dit : Viens, entre, toi béni de Dieu, et je te délivrerai ton ennemi que tu as. j'ai suivi après et je n'ai pas trouvé. Et Barach entra et trouva Sisara morte, et dit : Béni soit le Seigneur qui envoya son esprit et dit : Sisara sera livrée entre les mains d'une femme. Et après avoir dit cela, il coupa la tête de Sisara et l'envoya à sa mère, et lui donna un message disant : Reçois ton fils que tu attendais pour qu'il vienne avec du butin.



\chapter{32}

\par 1 Alors Debbora et Barach, fils d'Abino, et tout le peuple ensemble chantèrent un hymne à l'Éternel en ce jour-là, disant : Voici, d'en haut l'Éternel nous a montré sa gloire, comme il l'a fait autrefois lorsqu'il envoya fait entendre sa voix pour confondre les langues des hommes. Et il a choisi notre nation, et il a retiré Abraham notre père du feu, et l'a choisi avant tous ses frères, et l'a gardé du feu et l'a délivré des briques de la construction de la tour, et lui a donné un fils en les derniers jours de sa vieillesse, et il le fit sortir du sein stérile, et tous les anges furent jaloux de lui, et les chefs des armées l'envièrent.

\par 2 Et il arriva qu'ils étaient jaloux de lui, Dieu lui dit : Tue pour moi le fruit de ton ventre et offre pour moi ce que je t'ai donné. Et Abraham ne le contredit pas et partit immédiatement. Et comme il sortait, il dit à son fils : Voici, maintenant, mon fils, je t'offre en holocauste et je te livre entre les mains de celui qui t'a donné à moi.

\par 3 Et le fils dit à son père : Écoute-moi, père. Si un agneau du troupeau est accepté en offrande à l'Éternel pour son odeur douce, et si, pour les iniquités des hommes, des brebis sont destinées à l'abattoir, mais que l'homme soit destiné à hériter du monde, comment me dis-tu maintenant : Venez hériter d'une vie sécurisée, et d'un temps qui ne se mesure pas ? Et si je n'étais pas né dans le monde pour être offert en sacrifice à celui qui m'a créé ? Et ce sera ma bénédiction plus grande que celle de tous les hommes, car il n'y aura rien d'autre de pareil ; et en moi les générations seront instruites, et par moi les peuples comprendront que l'Éternel a rendu l'âme d'un homme digne de lui être un sacrifice.

\par 4 Et lorsque son père l'eut offert sur l'autel et lui eut lié les pieds pour le tuer, le Très-Puissant se hâta et envoya sa voix d'en haut disant : Ne tue pas ton fils, et ne détruis pas le fruit de ton corps ; car maintenant je me suis manifesté pour pouvoir apparaître à ceux qui ne me connaissent pas, et j'ai fermé la bouche à ceux qui parlent toujours mal de toi. Et ton souvenir sera devant moi pour toujours, ainsi que ton nom et le nom de ton fils, d'une génération à l'autre.

\par 5 Et il donna à Isaac deux fils, qui étaient également issus d'un ventre enfermé, car à cette époque leur mère était dans la troisième année de son mariage. Et il n'en sera pas ainsi d'aucune autre femme, et aucune femme ne se vantera de la sorte si elle s'approche de son mari la troisième année. Et il lui naquit deux fils, Jacob et Ésaü. Et Dieu aimait Jacob, mais il haïssait Ésaü à cause de ses actes.

\par 6 Et il arriva, dans la vieillesse de leur père, qu'Isaac bénit Jacob et l'envoya en Mésopotamie, et là il engendra 12 fils, et ils descendirent en Egypte et y demeurèrent.

\par 7 Et lorsque leurs ennemis les traitaient mal, le peuple cria au Seigneur, et leur prière fut exaucée, et il les fit sortir de là, et les conduisit au mont Sina, et leur apporta le fondement de la compréhension qui il s'était préparé dès la naissance du monde ; et alors les fondations furent déplacées, les armées projetèrent des éclairs sur leurs parcours, et les vents retentirent de leurs entrepôts, et la terre fut remuée depuis ses fondations, et les montagnes et les rochers tremblèrent dans leurs attaches, et les nuages ​​​​se soulevèrent. ils lèvent leurs vagues contre la flamme du feu, afin qu'il ne consume pas le monde.

\par 8 Alors l'abîme se réveilla de ses sources, et toutes les vagues de la mer se rassemblèrent. Alors le Paradis rendit le souffle de ses fruits, et les cèdres du Liban furent arrachés de leurs racines. Et les bêtes des champs furent terrifiées dans les habitations des forêts, et toutes ses œuvres se rassemblèrent pour contempler l'Éternel lorsqu'il ordonna une alliance avec les enfants d'Israël. Et tout ce que le Tout-Puissant a dit, il l'a observé, ayant pour témoin Moïse, son bien-aimé.

\par 9 Et quand il mourait, Dieu lui assigna le firmament, et lui montra ces témoins que nous avons maintenant, disant : Que le ciel dans lequel tu es entré et la terre dans laquelle tu as marché jusqu'à présent soient témoins entre moi et toi. et mon peuple. Car le soleil, la lune et les étoiles seront nos serviteurs (ou vous).

\par 10 Et lorsque Jésus se leva pour gouverner le peuple, il arriva, le jour où il combattait les ennemis, que le soir approchait, alors que la bataille était encore forte, et Jésus dit au soleil et à la lune : Ô vous, ministres qui avez été établis entre le Très-Puissant et ses fils, voici maintenant, la bataille continue, et abandonnez-vous votre fonction ? Arrêtez-vous donc aujourd'hui et donnez la lumière à ses fils, et mettez les ténèbres sur nos ennemis. Et ils l’ont fait.

\par 11 Et maintenant, en ces jours-là, Sisara se leva pour faire de nous ses esclaves, et nous criâmes vers l'Éternel notre Dieu, et il commanda aux étoiles et dit : Sortez de vos rangs et brûlez mes ennemis, afin qu'ils connaissent ma puissance. . Et les étoiles sont descendues et ont renversé leur camp et nous ont gardés en sécurité sans aucun travail.

\par 12 C'est pourquoi nous ne cesserons pas de chanter des louanges, et nos bouches ne garderont pas le silence pour raconter ses merveilles ; car il s'est souvenu de ses promesses, nouvelles et anciennes, et nous a montré sa délivrance ; et c'est pourquoi Jahel se vante parmi les femmes, parce qu'elle seule a ouvert cette bonne voie au succès, en ce sens qu'elle a tué Sisara de ses propres mains.

\par 13 Ô terre, allez, allez, cieux et éclairs, allez, anges et armées, [allez] et parlez aux pères dans les trésors de leurs âmes, et dites : Le Tout-Puissant n'a pas oublié le et encore moins les promesses qu'il nous a faites, en disant : Je ferai beaucoup de prodiges pour vos fils. Et maintenant, à partir de ce jour, on saura que tout ce que Dieu a dit aux hommes qu'il fera, il l'accomplira, même si l'homme meurt.

\par 14 Chante des louanges, chante des louanges, ô Debbora (ou, si l'homme tarde à chanter des louanges à Dieu, chante pourtant, ô Debbora), et que la grâce d'un esprit saint s'éveille en toi, et commence à louer les œuvres de l'Éternel, car il ne se lèvera plus un jour où les étoiles annonceront la nouvelle et vaincraront les ennemis d'Israël, comme il leur a été commandé. Désormais, si Israël tombe dans une situation difficile, qu'il appelle ces témoins avec leurs ministres, et ils se rendront en ambassade auprès du Très-Haut, et il se souviendra de ce jour et enverra une délivrance à son alliance. .

\par 15 Et toi, Debbora, commence à parler de ce que tu as vu dans les champs : comment le peuple marchait et sortait sain et sauf, et les étoiles combattaient de leur côté (ou comment, comme des peuples marchant, ainsi sortait le étoiles et combattu). Réjouis-toi, ô terre, de ceux qui habitent en toi, car en toi est la connaissance du Seigneur qui bâtit en toi sa forteresse. Car il était juste que Dieu ait retiré de toi la côte de celui qui était le premier formé, sachant que de sa côte naîtrait Israël. Et ta formation servira de témoignage de ce que l'Éternel a fait pour son peuple.

\par 16 Attendez, ô heures du jour, et ne vous hâtez pas, afin que nous puissions annoncer ce que notre entendement peut produire, car la nuit viendra sur nous. Et ce sera comme la nuit où Dieu frappa le premier-né des Egyptiens à cause de son premier-né.

\par 17 Et alors je cesserai de mon hymne parce que le temps sera accéléré (ou préparé) pour ses justes. Car je lui chanterai comme au renouvellement de la création, et les gens se souviendront de cette délivrance, et cela leur servira de témoignage. Que la mer aussi et ses profondeurs en témoignent, car non seulement Dieu l'a asséchée devant nos pères, mais il a aussi renversé le camp de son emplacement et a vaincu nos ennemis.

\par 18 Et lorsque Débora eut fini de parler, elle monta ensemble avec le peuple à Silo, et ils offrirent des sacrifices et des holocaustes et sonnèrent des larges trompettes. Et quand ils sonnèrent de la trompette et eurent offert les sacrifices, Debbora dit : Ceci sera pour le témoignage des trompettes entre les étoiles et leur Seigneur.

\chapter{33}

\par 1 Et Debbora descendit de là, et jugea Israël pendant 40 ans. Et il arriva que, lorsque le jour de sa mort approchait, elle envoya rassembler tout le peuple et leur dit : Écoutez maintenant, mon peuple. Voici, je vous exhorte comme une femme de Dieu, et je vous éclaire comme une femme de la race des femmes ; obéissez-moi maintenant comme à votre mère, et prêtez l'oreille à mes paroles, comme des hommes qui mourront vous-mêmes.

\par 2 Voici, je pars pour mourir par la voie de toute chair, par laquelle vous aussi allez : dirigez seulement votre cœur vers l'Éternel, votre Dieu, au temps de votre vie, car après votre mort vous ne pourrez pas vous repentir de ces choses dans lesquelles vous vivez.

\par 3 Car la mort est maintenant scellée et accomplie, et la mesure, le temps et les années ont restauré ce qui leur avait été confié. Car même si vous cherchez à faire le mal en enfer après votre mort, vous ne le pourrez pas, parce que le désir du péché cessera, et la mauvaise création perdra sa puissance, et l'enfer, qui reçoit ce qui lui est confié, il ne la restituera pas à moins que celui qui l'a commis ne l'exige. Maintenant donc, mes fils, obéissez à ma voix pendant que vous avez le temps de la vie et la lumière de la loi, et dirigez vos voies.

\par 4 Et quand Debora prononça ces paroles, tout le peuple éleva la voix ensemble et pleura, disant : Voici maintenant, mère, tu meurs et tu abandonnes tes fils ; et à qui les confies-tu ? Priez donc pour nous, et après votre départ, votre âme se souviendra de nous pour toujours.

\par 5 Et Debbora répondit et dit au peuple : Tant qu'un homme vit, il peut prier pour lui-même et pour ses fils ; mais après sa mort, il ne pourra plus implorer ni se souvenir de personne. N’espérez donc pas en vos pères, car ils ne vous serviront à rien si vous n’êtes pas trouvés semblables à eux. Mais alors votre ressemblance sera comme les étoiles du ciel qui vous ont été manifestées en ce moment.

\par 6 Et Debbora mourut et s'endormit avec ses pères et fut enterrée dans la ville de ses pères, et le peuple la pleura pendant 70 jours. Et tandis qu'ils la pleuraient, ils prononçaient ainsi une lamentation, disant : Voici, une mère d'Israël a péri, et une sainte qui avait régné dans la maison de Jacob, qui a fixé la clôture autour de sa génération, et sa génération sera la chercher. Et après sa mort, le pays fut en repos pendant sept ans.

\chapter{34}

\par 1 Et à ce moment-là arriva un certain Aod des prêtres de Madian, et il était un sorcier, et il parla à Israël, disant : Pourquoi prêtez-vous l'oreille à votre loi ? Venez et je vous montrerai une chose que votre loi n'est pas. Et le peuple dit : Que peux-tu nous montrer que notre loi ne montre pas ? Et il dit au peuple : Avez-vous déjà vu le soleil la nuit ? Et ils dirent : Non. Et il dit : Chaque fois que vous le voudrez, je vous le montrerai, afin que vous sachiez que nos dieux ont du pouvoir et que vous ne séduisez pas ceux qui les servent. Et ils dirent : Montre-nous.

\par 2 Et il s'en alla et opéra avec sa magie, ordonnant aux anges qui étaient chargés des sorcelleries, parce que pendant longtemps il leur offrait des sacrifices.

\par 3 [Car cela était autrefois au pouvoir des anges et était] accompli par les anges avant d'être jugés, et ils auraient détruit le monde incommensurable ; et parce qu'ils avaient transgressé, il arriva que les anges n'avaient plus le pouvoir. Car lorsqu'ils ont été jugés, alors la puissance n'a pas été confiée aux autres ; et c'est par ces signes (ou puissances) qu'agissent ceux qui exercent la sorcellerie auprès des hommes, jusqu'à ce que vienne un âge incommensurable.

\par 4 Et à ce moment-là, Aod, par magie, montra au peuple le soleil la nuit. Et le peuple fut étonné et dit : Voici, que de grandes choses peuvent faire les dieux des Madianites, et nous ne le savions pas !

\par 5 Et Dieu, voulant éprouver Israël s'il était encore dans l'iniquité, laissa les anges, et leur œuvre eut un bon succès, et le peuple d'Israël fut trompé et commença à servir les dieux des Madianites. Et Dieu dit : Je les livrerai entre les mains des Madianites, dans la mesure où c'est par eux qu'ils ont été trompés. Et il les livra entre leurs mains, et les Madianites commencèrent à asservir Israël.

\chapter{35}

\par 1 Or Gédéon était fils de Joath, l'homme le plus vaillant de tous ses frères. Et quand arriva le temps de l'été, il arriva à la montagne, ayant des gerbes avec lui, pour les y battre et échapper aux Madianites qui le pressaient. Et l'ange du Seigneur le rencontra et lui dit : D'où viens-tu et où entres-tu ?

\par 2 Il lui dit : Pourquoi me demandes-tu d'où je viens ? car la détresse m'entoure, car Israël est tombé dans l'affliction, et ils sont en vérité livrés entre les mains des Madianites. Et où sont les prodiges que nos pères nous ont racontés, disant : L'Éternel a choisi Israël seul avant tous les peuples de la terre ? Voici, maintenant il nous a livrés, et il a oublié les promesses qu'il avait faites à nos pères. Car nous aurions préféré être livrés à la mort une fois pour toutes, plutôt que que son peuple soit ainsi puni à maintes reprises.

\par 3 Et l'ange du Seigneur lui dit : Ce n'est pas pour rien que vous êtes livrés, mais vos propres inventions ont amené ces choses sur vous, car de même que vous avez abandonné les promesses que vous avez reçues du Seigneur, ces maux vous sont arrivés, et vous n'avez pas tenu compte des commandements de Dieu que ceux qui vous ont précédés vous ont prescrits. C’est pourquoi vous êtes tombés dans le mécontentement de votre Dieu. Mais il aura pitié de vous, comme personne n'a pitié, même de la race d'Israël, et cela non à cause de vous, mais à cause de ceux qui s'endorment.

\par 4 Et maintenant viens, je t'enverrai, et tu délivreras Israël de la main des Madianites. Car ainsi parle l'Éternel : Bien qu'Israël ne soit pas juste, mais parce que les Madianites sont des pécheurs, c'est pourquoi, connaissant l'iniquité de mon peuple, je leur pardonnerai, et après cela je leur réprimanderai pour avoir fait le mal, mais sur le Madianites, je serai vengé tout à l'heure.

\par 5 Et Gédéon dit : Qui suis-je et quelle est la maison de mon père, pour que j'aille combattre les Madianites ? Et l'ange lui dit : Peut-être penses-tu que la voie de Dieu est telle que la voie de l'homme. Car les hommes regardent la gloire du monde et les richesses, mais Dieu regarde ce qui est droit et bon et la douceur. Maintenant donc, va, ceins tes reins, et l'Éternel sera avec toi, car c'est toi qu'il a choisi pour te venger de ses ennemis, comme voici, il te l'a ordonné.

\par 6 Et Gédéon lui dit : Que mon Seigneur ne soit pas irrité si je dis une parole. Voici, Moïse, le premier de tous les prophètes, demanda au Seigneur un signe, et celui-ci lui fut donné. Mais qui suis-je, sinon le Seigneur qui m'a choisi, donne-moi un signe afin que je sache que je vais bien. Et l'ange du Seigneur lui dit : Cours et prends pour moi de l'eau dans la fosse là-bas et verse-la sur ce rocher, et je te donnerai un signe. Et il alla le prendre comme il le lui avait ordonné.

\par 7 Et l'ange lui dit : Avant de verser l'eau sur le rocher, demande ce que tu veux qu'il devienne, soit du sang, soit du feu, ou qu'il n'apparaisse pas du tout. Et Gédéon dit : Que cela devienne moitié sang et moitié feu. Et Gédéon versa l'eau sur le rocher, et quand il l'eut répandue, la moitié devint flamme et la moitié sang, et ils furent mélangés ensemble, c'est-à-dire le feu et le sang. mais le sang n’a pas éteint le feu, et le feu n’a pas non plus consumé le sang. Et quand Gédéon vit cela, il demanda encore d'autres signes, et ils lui furent donnés. Cela n'est-il pas écrit dans le livre des Juges ?

\chapter{36}

\par 1 Et Gédéon prit 300 hommes et partit et arriva à l'extrémité du camp de Madian, et il entendit chacun parler à son voisin et dire : Vous verrez une confusion au-delà de toute estimation, de l'épée de Gédéon venir sur nous, car Dieu a livré entre ses mains le camp des Madianites, et il commencera à en finir avec nous, même la mère et les enfants, parce que nos péchés sont comblés, comme aussi nos dieux nous l'ont montré et nous je ne les croyais pas. Et maintenant, levons-nous, secourons nos âmes et fuyons.

\par 2 Et lorsque Gédéon entendit ces paroles, aussitôt il fut revêtu de l'esprit du Seigneur, et, étant doté de puissance, il dit aux 300 hommes : Levez-vous et que chacun de vous ceigne son épée, pour les Madianites. sont livrés entre nos mains. Et les hommes descendirent avec lui, et il s'approcha et commença à combattre. Et ils sonnèrent de la trompette et crièrent ensemble et dirent : L'épée du Seigneur est sur nous. Et ils tuèrent environ 120 000 hommes des Madianites, et le reste des Madianites s'enfuit.

\par 3 Et après ces choses, Gédéon vint et rassembla le peuple d'Israël et leur dit : Voici, l'Éternel m'a envoyé pour combattre votre bataille, et j'y suis allé selon ce qu'il m'a ordonné. Et maintenant, je vous demande une seule requête : ne détournez pas votre visage ; et que chacun d'entre vous me donne les bracelets d'or que vous avez aux mains. Et Gédéon étendit un manteau, et chacun y jeta ses brassards, et ils furent tous pesés, et on trouva que leur poids était de 12 talents (ou 12 000 sicles). Et Gédéon les prit, et il en fit des idoles et les adora.

\par 4 Et Dieu dit : Une voie est en vérité fixée, pour que je ne réprimande pas Gédéon de son vivant, même parce que lorsqu'il détruisit le sanctuaire de Baal, alors tous les hommes dirent : Que Baal se venge. Maintenant donc, si je le châtie pour avoir fait du mal contre moi, vous direz : Ce n'est pas Dieu qui l'a châtié, mais Baal, parce qu'il a péché autrefois contre lui. C'est pourquoi Gédéon mourra maintenant dans une bonne vieillesse, afin qu'ils n'aient plus de quoi parler. Mais après que Gédéon sera mort, je le punirai une fois, parce qu'il a transgressé contre moi. Et Gédéon mourut dans une bonne vieillesse et fut enterré dans sa propre ville.



\chapter{37}

\par 1 Et il eut un fils d'une concubine dont le nom était Abimélec; Celui-ci tua tous ses frères, voulant régner sur le peuple.

\par [Une feuille disparue.]

\par 2 Alors tous les arbres des champs se rassemblèrent vers le figuier et dirent : Venez, régnez sur nous. Et le figuier dit : Suis-je vraiment né dans le royaume ou dans la domination des arbres ? ou étais-je habitué à cela et à régner sur toi ? Et c'est pourquoi, même si je ne peux pas régner sur vous, Abimélec n'obtiendra pas non plus la continuité de sa domination. Après cela, les arbres se rassemblèrent autour de la vigne et dirent : Viens, règne sur nous. Et la vigne dit : J'ai été plantée pour donner aux hommes la douceur du vin, et je me conserve en leur rendant mon fruit. Mais de même que je ne peux pas régner sur vous, de même le sang d'Abimélec sera exigé de vos mains. Et après cela, les arbres s'approchèrent du pommier et dirent : Viens, règne sur nous. Et il dit : Il m'a été commandé de donner aux hommes un fruit de bonne odeur. C'est pourquoi je ne peux pas régner sur vous, et Abimélec mourra lapidé.

\par 3 Alors les arbres s'approchèrent des ronces et dirent : Venez, régnez sur nous. Et la ronce dit : Quand l'épine est née, la vérité a brillé sous la forme d'une épine. Et lorsque notre premier père fut condamné à mort, la terre fut condamnée à produire des épines et des chardons. Et quand la vérité a éclairé Moïse, c’est par un buisson épineux qu’elle l’a éclairé. Maintenant donc, c'est par moi que la vérité sera entendue à votre sujet. Maintenant, si vous avez dit sincèrement à la ronce qu'elle devait vraiment régner sur vous, asseyez-vous à son ombre ; mais si vous faites semblant, alors que le feu sorte et dévore et consume les arbres des champs. Car le pommier a été fait pour ceux qui châtiaient, et le figuier a été fait pour le peuple, et la vigne a été faite pour ceux qui étaient avant nous.

\par 4 Et maintenant, la ronce sera pour vous comme Abimélec, qui a tué injustement ses frères et qui veut régner sur vous. Si Abimélec est digne de ceux qu'il désire gouverner (ou qu'Abimélec soit un feu pour eux), qu'il soit comme la ronce qui a été faite pour réprimander les insensés parmi le peuple. Et un feu sortit des ronces et dévora les arbres qui sont dans les champs.

\par 5 Après cela, Abimélec régna sur le peuple pendant un an et six mois, et il mourut durement près d'une certaine tour, d'où une femme jeta sur lui la moitié d'une meule.

\par [Un espace de longueur incertaine dans le texte.]

\chapter{38}

\par 1 (Puis Jair jugea Israël pendant 22 ans.) Celui-ci bâtit un sanctuaire à Baal et égara le peuple en disant : Tout homme qui ne sacrifiera pas à Baal mourra. Et quand tout le peuple sacrifiait, sept hommes seulement ne voulaient pas sacrifier, dont les noms sont : Dephal, Abiesdrel, Getalibal, Selumi, Assur, Jonadali, Memihel.

\par 2 Celui-ci répondit et dit à Jair : Voici, nous nous souvenons des préceptes que ceux qui étaient avant nous nous ont commandés, ainsi que Debbora notre mère, en disant : Prenez garde à ne pas détourner votre cœur à droite ou à gauche. , mais observez la loi du Seigneur jour et nuit. Maintenant donc, pourquoi corrompt-tu le peuple de l’Éternel et le trompes-tu, en disant : Baal est Dieu, adorons-le ? Et maintenant, s'il est Dieu comme tu le dis, qu'il parle comme un Dieu, et alors nous lui offrirons des sacrifices.

\par 3 Et Jaïr dit : Brûlez-les au feu, car ils ont blasphémé Baal. Et ses serviteurs les prirent pour les brûler au feu. Et quand ils les jetèrent sur le feu, Nathaniel, l'ange qui est au-dessus du feu, sortit, et éteignit le feu et brûla les serviteurs de Jaïr. Mais il fit échapper les sept hommes, de sorte qu'aucun homme du peuple ne les vit. , car il avait frappé le peuple d'aveuglement.

\par 4 Et quand Jaïr arriva au lieu (ou il arriva au lieu de Jaïr), lui aussi fut brûlé. Mais avant de le brûler, l'ange du Seigneur lui dit : Écoute la parole du Seigneur avant de mourir. Ainsi parle l'Éternel : Je t'ai fait sortir du pays d'Égypte et je t'ai établi chef de mes peuples. Mais tu t'es ressuscité et tu as corrompu mon alliance, et tu les as égarés, et tu as cherché à brûler mes serviteurs dans la flamme, parce qu'ils t'avaient réprimandé, qui, bien qu'ils soient brûlés par un feu corruptible, sont maintenant vivifiés par un feu vivant et sont livré. Mais tu mourras, dit l'Éternel, et c'est dans le feu où tu mourras que tu habiteras. Et ensuite il le brûla, et s'approcha même de la colonne de Baal, la renversa et incendia Baal avec le peuple qui se tenait là, soit mille hommes.

\chapter{39}

\par 1 Et après ces choses arrivèrent les enfants d'Ammon et commencèrent à combattre Israël et prirent plusieurs de leurs villes. Et comme le peuple était dans une grande difficulté, ils se rassemblèrent à Masphath, disant chacun à ses voisins : Voici maintenant, nous voyons la détroit qui nous entoure, et l'Éternel s'est éloigné de nous, et n'est plus avec nous, ni nos ennemis. ont pris nos villes, et il n'y a pas de chef pour entrer et sortir devant nous. Voyons maintenant qui nous pouvons mettre à notre tête pour mener notre bataille.

\par 2 Or Jepthan le Galaadite était un homme vaillant et vaillant, et parce qu'il était jaloux de ses frères, ils l'avaient chassé de son pays, et il partit et habita dans le pays de Tobi. Et les vagabonds se rassemblèrent près de lui et demeurèrent avec lui.

\par 3 Et il arriva, quand Israël fut vaincu au combat, qu'ils arrivèrent au pays de Tobi, près de Jepthan, et lui dirent : Viens, domine sur le peuple. Car qui sait si tu as été préservé jusqu'à ce jour, ou si tu as été délivré des mains de tes frères, afin que tu puisses en ce moment régner sur ton peuple ?

\par 4 Et Jepthan leur dit : L'amour revient-il ainsi après la haine, ou le temps l'emporte-t-il sur toutes choses ? Car vous m'avez chassé de mon pays et de la maison de mon père ; et maintenant, venez-vous vers moi alors que vous êtes dans une situation difficile ? Et ils lui dirent : Si le Dieu de nos pères ne s'est pas souvenu de nos péchés, mais s'il nous a délivrés lorsque nous avons péché contre lui et qu'il nous a livrés devant nos ennemis, et que nous avons été opprimés par eux, pourquoi veux-tu cela ? Est-ce qu'un mortel se souvient des iniquités qui nous sont arrivées au temps de notre affliction ? Qu’il n’en soit donc pas ainsi devant toi, Seigneur.

\par 5 Et Jepthan dit : Dieu est en effet capable de ne pas se soucier de nos péchés, puisqu'il a le temps et le lieu pour se reposer de sa longanimité, car il est Dieu ; mais je suis mortel, fait de la terre : vers quoi retournerai-je, et où rejetterai-je ma colère et le tort par lequel vous m'avez blessé ? Et le peuple lui dit : Que la colombe t'instruise, à laquelle Israël était comparé, car, même si ses petits lui sont enlevés, elle ne s'éloigne pas de sa place, mais elle rejette sa faute et l'oublie comme si elle était dans le temps. fond des profondeurs.

\par 6 Et Jepthan se leva et alla avec eux et rassembla tout le peuple, et leur dit : Vous savez comment, lorsque nos princes étaient en vie, ils nous ont exhortés à suivre notre loi. Et Ammon et ses fils détournèrent le peuple du chemin où il marchait, pour servir d'autres dieux qui devaient le détruire. Maintenant donc, attachez votre cœur à la loi de l’Éternel, votre Dieu, et implorons-le d’un commun accord. Et ainsi nous lutterons contre nos adversaires, et nous ferons confiance et espérerons dans le Seigneur qu’il ne nous livrera pas pour toujours. Car, même si nos péchés abondent, sa miséricorde remplit néanmoins toute la terre.

\par 7 Et tout le peuple priait d'un commun accord, hommes et femmes, garçons et nourrissons. Et quand ils priaient, ils disaient : Regarde, Seigneur, le peuple que tu as choisi, et ne détruit pas la vigne que ta droite a plantée ; afin que ce peuple soit devant toi pour héritage, que tu possèdes depuis le commencement, et que tu as toujours préféré, et pour lequel tu as fait des lieux habitables et les as amenés dans le pays que tu leur avais juré ; ne nous livre pas devant ceux qui te haïssent, ô Seigneur.

\par 8 Et Dieu se repentit de sa colère et fortifia l'esprit de Jepthan. Et il envoya un message à Getal, roi des enfants d'Ammon, et dit : Pourquoi troublez-vous notre pays et avez-vous pris mes villes, ou pourquoi nous affligez-vous ? Le Dieu d'Israël ne t'a pas ordonné de détruire les habitants du pays. Maintenant, rends-moi mes villes, et ma colère cessera de toi. Sinon, sache que je monterai vers toi et te rendrai ce que tu as fait auparavant, et que je rendrai ta méchanceté sur ta tête. Ne te souviens-tu pas de la manière dont tu as agi en tromperie envers le peuple d'Israël dans le désert ? Et les messagers de Jepthan dirent ces paroles au roi des enfants d'Ammon.

\par 9 Et Guétal dit : Israël a-t-il réfléchi lorsqu'il a pris le pays des Amoréens ? Dis donc : Sache que maintenant je t'enlèverai le reste de tes villes, je te rendrai justice pour ta méchanceté et je me vengerai des Amoréens que tu as offensés. Et Jepthan envoya de nouveau au roi des enfants d'Ammon pour lui dire : En vérité, je comprends que Dieu t'a amené ici pour que je puisse te détruire, à moins que tu ne te reposes de l'iniquité avec laquelle tu veux tourmenter Israël. Et c'est pourquoi je viendrai à toi et me montrerai à toi. Car ce ne sont pas, comme vous le dites, des dieux qui vous ont donné l'héritage que vous possédez. Mais parce que vous vous êtes égarés après les pierres, le feu vous suivra pour vous venger.

\par 10 Et comme le roi des enfants d'Ammon ne voulait pas entendre la voix de Jepthan, Jepthan se leva et arma tout le peuple pour sortir et combattre dans les frontières en disant : Quand les enfants d'Ammon seront livrés entre mes mains et que je serai de retour, celui qui me rencontrera le premier sera offert en holocauste à l'Éternel.

\par 11 Et l'Éternel fut très irrité et dit : Voici, Jepthan a juré de m'offrir ce qui lui conviendra en premier. Maintenant donc, si un chien rencontre Jepthan le premier, un chien me sera-t-il offert ? Et maintenant, que le vœu de Jepthan soit sur son premier-né, sur le fruit de son corps, et que sa prière soit sur sa fille unique. Mais en vérité, je délivrerai mon peuple en ce moment, non à cause de lui, mais à cause de la prière qu'Israël a priée.

\chapter{40}

\par 1 Et Jepthan vint et combattit les enfants d'Ammon, et l'Éternel les livra entre ses mains, et il frappa soixante de leurs villes. Et Jepthan revint en paix. Et les femmes venaient à sa rencontre avec des danses. Et il avait une fille unique; la même est sortie la première dans les danses pour rencontrer son père. Et quand Jepthan la vit, il s'évanouit et dit : C'est à juste titre que ton nom est appelé Seila, pour que tu sois offert en sacrifice. Et maintenant, qui mettra mon cœur dans la balance et pèsera mon âme ? et je me tiendrai debout et verrai si l'un l'emportera sur l'autre, la joie qui est venue ou l'affliction qui m'arrive ? car en ayant ouvert la bouche à mon Seigneur dans le chant de mes vœux, je ne peux plus le rappeler.

\par 2 Et Séila, sa fille, lui dit : Et qui peut être attristé par sa mort en voyant le peuple délivré ? Ne te souviens-tu pas de ce qui se passait du temps de nos pères, lorsque le père offrait son fils en holocauste et qu'il ne le contestait pas, mais consentait à ce qu'il se réjouisse ? Et celui à qui l'on offrait était prêt, et celui qui l'offrait était content.

\par 3 Maintenant donc n'annule rien de ce que tu as fait vœu, mais accorde-moi une seule prière. Je te demande avant de mourir une petite requête : je te supplie qu'avant de rendre mon âme, je puisse aller dans les montagnes et errer (ou demeurer) parmi les collines et me promener parmi les rochers, moi et les vierges qui sont mes amis, et y verser mes larmes et raconter l'affliction de ma jeunesse ; et les arbres des champs me pleureront et les bêtes des champs se lamenteront sur moi ; car je ne suis pas triste de mourir, et cela ne me chagrine pas non plus de donner mon âme. Mais alors que mon père a été dépassé dans son vœu, [et] si je ne m'offre pas volontairement en sacrifice, je crains ma mort. ne serait pas acceptable et que je perdrais la vie en vain. Je dirai ces choses aux montagnes, et après cela je reviendrai. Et son père dit : Vas-y.

\par 4 Et Seila, fille de Jepthan, sortit, elle et les vierges qui étaient ses semblables, et vinrent et le rapportèrent aux sages du peuple. Et aucun homme ne pouvait répondre à ses paroles. Et après cela, elle se rendit au mont Stelac, et pendant la nuit, le Seigneur pensa à elle et dit : Voici, maintenant j'ai fermé la langue des sages parmi mon peuple devant cette génération, afin qu'ils ne puissent pas répondre à la parole du fille de Jepthan, afin que ma parole s'accomplisse et que mon conseil que j'avais conçu ne soit pas détruit. Et j'ai vu qu'elle est plus sage que son père, et une jeune fille intelligente plus que tous les sages qui sont ici. Et maintenant, que sa vie lui soit donnée à sa demande, et sa mort sera toujours précieuse à mes yeux.

\par 5 Et lorsque la fille de Jepthan arriva au mont Stelac, elle se lamenta. Et voici sa lamentation avec laquelle elle s'est pleurée et s'est lamentée avant de partir, et elle a dit : Écoutez, ô montagnes, mes lamentations, et regardez, ô collines, les larmes de mes yeux, et soyez témoins, ô rochers, dans le pleurant mon âme. Voyez comme je suis accusé, mais mon âme ne sera pas enlevée en vain. Que mes paroles s'élèvent dans les cieux et que mes larmes soient écrites sur la face du firmament, afin que le père ne vainque pas (ou ne combatte pas) sa fille qu'il a juré d'offrir, afin que son chef entende que son la fille unique est promise en sacrifice.

\par 6 Pourtant je n'ai pas été satisfait de mon lit conjugal, ni rempli des guirlandes de mes noces. Car je n'ai pas été revêtue d'éclat, assise dans ma virginité ; Je n'ai pas utilisé mon précieux onguent, et mon âme n'a pas non plus apprécié l'huile d'onction qui a été préparée pour moi. Ô ma mère, c'est en vain que tu as enfanté ton unique et que tu l'as engendrée sur la terre, car l'enfer est devenu ma chambre nuptiale. Que tout le mélange d'huile que tu m'as préparé soit versé, et que la robe blanche que ma mère m'a tissée, que le papillon de nuit la mange, et que la couronne de fleurs que ma nourrice m'avait autrefois tressée, qu'elle se fane, et le couvre-lit qu'elle a tissé de violette et de pourpre pour ma virginité, que le ver le gâte ; et quand les vierges, mes semblables, parleront de moi, qu'elles me lamentent en gémissant pendant plusieurs jours.

\par 7 Inclinez vos branches, ô arbres, et déplorez ma jeunesse. Venez, bêtes des forêts, et foulez aux pieds ma virginité. Car mes années sont retranchées, et les jours de ma vie vieillissent dans les ténèbres.

\par 8 Et après avoir dit cela, Seila retourna vers son père, et il fit tout ce qu'il avait juré, et offrit des holocaustes. Alors toutes les jeunes filles d'Israël se rassemblèrent et enterrèrent la fille de Jepthan et la pleurèrent. Et les enfants d'Israël firent une grande lamentation et décidèrent ce mois-là, le 14e jour du mois, qu'ils se rassembleraient chaque année et se lamenteraient sur la fille de Jepthan pendant quatre jours. Et ils appelèrent son sépulcre d'après son propre nom, Seila.

\par 9 Et Jepthan jugea les enfants d'Israël dix ans, puis mourut et fut enterré avec ses pères.

\chapter{41}

\par 1 Et après lui se leva un juge en Israël, Addo, fils d'Elech de Praton, et il jugea aussi les enfants d'Israël pendant huit ans. De son temps, le roi de Moab lui envoya des messagers pour lui dire : Voici, tu sais qu'Israël a pris mes villes ; maintenant donc, restitue-les en récompense. Et Addo dit : N'êtes-vous pas encore instruits par ce qui est arrivé aux enfants d'Ammon, à moins que, par hasard, les péchés de Moab ne soient comblés ? Et Addo envoya et prit du peuple 20 000 hommes et vint contre Moab, et combattit contre eux et tua d'eux 45 000 hommes. Et le reste s'enfuit devant lui. Et Addo revint en paix et offrit des holocaustes et des sacrifices à son Seigneur, puis mourut et fut enterré à Ephrata, sa ville.

\par 2 Et à cette époque-là, le peuple choisit Elon et le fit juge sur eux, et il jugea Israël pendant vingt ans. En ce temps-là, ils combattirent contre les Philistins et leur prirent douze villes. Et Elon mourut et fut enterré dans sa ville.

\par 3 Mais les enfants d'Israël oublièrent l'Éternel, leur Dieu, et servirent les dieux des habitants du pays. C'est pourquoi ils furent livrés aux Philistins et les servirent quarante ans.

\chapter{42}

\par 1 Or il y avait un homme de la tribu de Dan, dont le nom était Manue, fils d'Edoc, fils d'Odo, fils d'Eriden, fils de Phadesur, fils de Dema, fils de Susi, le fils de Dan. Il avait une femme qui s'appelait Eluma, fille de Remac. Et elle était stérile et ne lui donna aucun enfant. Et quand Manue, son mari, lui disait jour après jour : Voici, l'Éternel t'a fermé le ventre pour que tu n'enfantes pas ; libère-moi donc, afin que je puisse prendre une autre femme, de peur de mourir sans issue. Et elle dit : Ce n'est pas le Seigneur qui m'a empêché de porter du fruit, mais toi, pour que je ne porte aucun fruit. Et il lui dit : Que la loi expose clairement notre épreuve.

\par 2 Et comme ils se disputaient jour après jour et qu'ils étaient tous deux très attristés parce qu'ils manquaient de fruits, une certaine nuit, la femme monta dans la chambre haute et pria en disant : Toi, Seigneur Dieu de toute chair, révèle-le à que ce soit à mon mari ou à moi qu'il n'est pas donné d'engendrer des enfants, ou à qui il est interdit ou à qui il est permis de porter du fruit, afin que celui à qui il est interdit de pleurer ses péchés, parce qu'il continue sans fruit . Ou si nous sommes tous deux privés, révèle-le-nous aussi, afin que nous puissions porter notre péché et garder le silence devant toi.

\par 3 Et le Seigneur écouta sa voix et lui envoya son ange le matin, et lui dit : Tu es la stérile qui ne produit pas, et tu es le ventre qui ne peut pas porter de fruit. Mais maintenant le Seigneur a entendu ta voix, il a regardé tes larmes et a ouvert ton ventre. Et voici, tu concevras et tu enfanteras un fils et tu lui donneras le nom de Samson, car il sera saint pour ton Seigneur. Mais prenez garde qu'il ne goûte aucun fruit de la vigne, et qu'il ne mange rien d'impur, car, comme lui-même l'a dit, il délivrera Israël de la main des Philistins. Et après que l'ange du Seigneur eut prononcé ces paroles, il la quitta.

\par 4 Et elle vint vers son mari dans la maison et lui dit : Voici, je mets la main sur ma bouche et je garderai silence devant toi tous mes jours, parce que c'est en vain que je me suis vanté et que je n'ai pas cru en ton mots. Car l'ange du Seigneur est venu vers moi aujourd'hui et me l'a montré, disant : Eluma, tu es stérile, mais tu concevras et tu enfanteras un fils.

\par 5 Et Manue ne croyait pas sa femme. Et il fut honteux et attristé et monta, lui aussi, dans la chambre haute et pria en disant : Voici, je ne suis pas digne d'entendre les signes et les prodiges que Dieu a opérés en nous, ni de voir le visage de son messager.

\par 6 Et il arriva, pendant qu'il parlait ainsi, que l'ange du Seigneur revint vers sa femme. Maintenant, elle était aux champs et Manue était chez lui. Et l'ange lui dit : Cours et appelle ton mari, car Dieu l'a jugé digne d'entendre ma voix.

\par 7 Et la femme courut et appela son mari, et il se hâta et vint vers l'ange dans le champ à Ammo (?), qui lui dit : Va vers ta femme et fais vite toutes ces choses. Mais il lui dit : Mais veille, Seigneur, à ce que ta parole s'accomplisse sur ton serviteur. Et il dit : Il en sera ainsi.

\par 8 Et Manue lui dit : Si je le pouvais, je te persuaderais d'entrer dans ma maison et de manger du pain avec moi, et sache que quand tu partiras, je te donnerai des cadeaux à emporter avec toi afin que tu puisses offrir un sacrifiez au Seigneur votre Dieu. Et l'ange lui dit : Je n'entrerai pas avec toi dans ta maison, je ne mangerai pas ton pain et je ne recevrai pas tes cadeaux. Car si tu offres un sacrifice de ce qui ne t'appartient pas, je ne peux pas te montrer de faveur.

\par 9 Et Manue bâtit un autel sur le rocher, et offrit des sacrifices et des holocaustes. Et il arriva qu'après avoir découpé la chair et l'avoir déposée sur le lieu saint, l'ange étendit la main et la toucha du bout de son sceptre. Et du feu sortit du rocher et consuma les holocaustes et les sacrifices. Et l'ange s'éloigna de lui avec la flamme du feu.

\par 10 Mais Manue et sa femme, voyant cela, tombèrent la face contre terre et dirent : Nous mourrons sûrement, parce que nous avons vu le Seigneur face à face. Et il ne me suffisait pas de le voir, mais je lui demandais aussi son nom, ne sachant pas qu'il était le ministre de Dieu. L'ange qui venait s'appelait Phadahel.



\chapter{43}

\par 1 Et il arriva, à l'époque de ces jours-là, qu'Eluma conçut et enfanta un fils et appela son nom Samson. Et le Seigneur était avec lui. Et quand il commença à grandir et qu'il chercha à combattre les Philistins, il prit une femme des Philistins. Et les Philistins la brûlèrent au feu, car ils avaient été très humiliés par Samson.

\par 2 Et après cela, Samson entra (ou fut en colère contre) Azotus. Et ils l'enfermèrent et firent le tour de la ville et dirent : Voici, maintenant notre adversaire est livré entre nos mains. Maintenant donc, rassemblons-nous et secourons-nous les uns les autres. Et quand Samson se leva pendant la nuit et vit la ville fermée, il dit : Voici, maintenant, ces puces m'ont enfermé dans leur ville. Et maintenant, l'Éternel sera avec moi, et je sortirai par leurs portes et je les combattrai.

\par 3 Et il alla mettre sa main gauche sous la barre de la porte, la secoua et renversa la porte du mur. Il tenait l'une des portes dans sa main droite comme bouclier, et il posa l'autre sur ses épaules et la démonta ; et comme il n'avait pas d'épée, il poursuivit les Philistins avec et tua avec elle 25 000 hommes. Et il éleva tous les dépendances de la porte et les plaça sur une montagne.

\par 4 Or, concernant le lion qu'il a tué, et la mâchoire de l'âne avec laquelle il a frappé les Philistins, et les liens qu'il a arrachés de ses bras comme s'ils étaient d'eux-mêmes, et les renards qu'il a capturés, ne sont pas ces choses. écrit dans le livre des Juges ?

\par 5 Alors Samson descendit à Gérara, une ville des Philistins, et y vit une prostituée nommée Dalila, et il fut emmené après elle et la prit pour femme. Et Dieu dit : Voici, maintenant Samson est égaré par ses yeux et il a oublié les merveilles que j'ai faites avec lui, et il s'est mêlé aux filles des Philistins, et il n'a pas considéré mon serviteur Joseph qui était dans un pays étranger. et il est devenu une couronne pour ses frères parce qu'il n'a pas voulu affliger sa postérité. Maintenant donc, sa concupiscence sera une pierre d'achoppement pour Samson, et son mélange sera sa destruction, et je le livrerai à ses ennemis et ils l'aveugleront. Pourtant, à l'heure de sa mort, je me souviendrai de lui, et je le vengerai encore une fois sur les Philistins.

\par 6 Et après ces choses, sa femme lui fut importune, lui disant : Montre-moi ta force, et en quoi est ta puissance. Alors je saurai que tu m'aimes. Et comme Samson l'avait trompée trois fois, et qu'elle continuait à l'importuner chaque jour, la quatrième fois il lui montra son cœur. Mais elle l'enivra, et pendant qu'il dormait, elle appela un barbier, et il lui rasa les sept mèches de la tête, et sa force le quitta, car ainsi lui-même le lui avait révélé. Et elle appela les Philistins, et ils frappèrent Samson, l'aveuglèrent et le mirent en prison.

\par 7 Et il arriva, le jour de leur festin, qu'ils appelèrent Samson pour se moquer de lui. Et lui, étant lié entre deux colonnes, pria en disant : Seigneur, Dieu de mes pères, écoute-moi encore une fois, et fortifie-moi afin que je puisse mourir avec ces Philistins ; car cette vue des yeux qu'ils m'ont ôtés m'a été donnée gratuitement. à moi par toi. Et Samson ajouta en disant : Va, ô mon âme, et ne sois pas attristé. Meurs, ô mon corps, et ne pleure pas sur toi-même.

\par 8 Et il saisit les deux piliers de la maison et les secoua. Et la maison tomba et tout ce qui s'y trouvait, et ils tuèrent tous ceux qui l'entouraient, et leur nombre fut de 40 000 hommes et femmes. Et les frères de Samson descendirent, ainsi que toute la maison de son père, et le prirent et l'enterrèrent dans le sépulcre de son père. Et il jugea Israël pendant vingt ans.

\chapter{44}

\par 1 Et en ce temps-là, il n'y avait pas de prince en Israël ; mais chacun faisait ce qui lui plaisait.

\par 2 A cette époque-là se leva Michas, fils de Dedila, mère d'Héliu, et il avait mille drachmes d'or et quatre coins d'or en fusion, et 40 didrachmes d'argent. Et sa mère Dedila lui dit : Mon fils, écoute ma voix et tu te feras un nom avant ta mort : prends cet or et fais-le fondre, et tu te feras des idoles, et elles seront pour toi des dieux, et toi tu deviendras pour eux leur prêtre.

\par 3 Et il arrivera que quiconque s'informera par eux viendra à toi et tu leur répondras. Et il y aura dans ta maison un autel et une stèle bâtis, et avec l'or que tu as, tu t'achèteras de l'encens pour les brûler et des moutons pour les sacrifices. Et quiconque offrira un sacrifice donnera sept didrachmes pour un mouton, et pour l'encens, s'il le brûle, il donnera un didrachme d'argent de tout son poids. Et ton nom sera Prêtre, et tu seras appelé adorateur des dieux.

\par 4 Et Michas lui dit : Tu m'as bien conseillé, ma mère, sur la façon dont je peux vivre ; et maintenant ton nom sera plus grand que mon nom, et dans les derniers jours ces choses te seront exigées.

\par 5 Et Michas s'en alla et fit tout ce que sa mère lui avait commandé. Et il sculpta et se fit trois images d'enfants, et de veaux, et un lion, et un aigle, et un dragon, et une colombe. Et c'est ainsi que tous ceux qui étaient égarés venaient à lui, et si quelqu'un voulait demander des femmes, ils le consultaient par la colombe ; et s'ils demandaient des fils, par l'image des garçons : mais celui qui demandait des richesses prenait conseil par la ressemblance de l'aigle, et celui qui demandait de la force par l'image du lion : encore, s'ils demandaient des hommes et des jeunes filles ils s'enquéraient par les images de veaux, mais si pendant des jours, ils s'enquéraient par l'image du dragon. Et son iniquité était de plusieurs formes, et son impiété était pleine de ruse.

\par 6 C'est pourquoi donc, lorsque les enfants d'Israël s'éloignèrent de l'Éternel, l'Éternel dit : Voici, je vais déraciner la terre et détruire toute la race des hommes, parce que lorsque j'ai fixé de grandes choses sur la montagne de Sina, je me suis montré au enfants d'Israël dans la tempête et j'ai dit qu'ils ne devaient pas fabriquer d'idoles, et ils ont consenti à ne pas sculpter l'image des dieux. Et je leur ai ordonné de ne pas prendre mon nom en vain, et ils ont choisi de ne pas même prendre mon nom en vain. Et je leur commandai d'observer le jour du sabbat, et ils consentirent à ce que je me sanctifie. Et je leur ai dit qu'ils devaient honorer leur père et leur mère : et ils ont promis qu'ils le feraient. Et je leur ai ordonné de ne pas voler, et ils ont consenti. Et je leur ai ordonné de ne pas commettre de meurtre, et ils l'ont reçu, qu'ils ne le devraient pas. Et je leur ai commandé de ne pas commettre d'adultère, et ils n'ont pas refusé. Et je leur ai donné l'ordre de ne pas porter de faux témoignage, et de ne pas convoiter à chacun la femme de son prochain, ou sa maison, ou tout ce qui lui appartient ; et ils l'ont accepté.

\par 7 Et maintenant, alors que je leur ai dit de ne pas faire d'idoles, ils ont fait les œuvres de tous ces dieux qui sont nés de la corruption sous le nom d'image taillée. Et aussi de ceux par qui toutes choses ont été corrompues. Car les hommes mortels les ont créés, et le feu a servi à les fondre : l'action des hommes les a fait naître, et les mains les ont façonnés, et l'intelligence les a façonnés. Et bien qu'ils les aient reçus, ils ont pris mon nom en vain, et ont donné mon nom à des images taillées, et le jour du sabbat qu'ils ont accepté pour l'observer, ils ont commis des abominations. Parce que je leur ai dit qu'ils devaient aimer leur père et leur mère, ils m'ont déshonoré, leur Créateur. Et c'est pour cela que je leur ai dit qu'ils ne devaient pas voler, car ils ont agi de manière voleuse dans leur compréhension des images gravées. Et alors que j’ai dit qu’ils ne devraient pas tuer, ils les tuent quand ils trompent. Et quand je leur avais commandé de ne pas commettre d'adultère, ils ont joué à l'adultère avec leur jalousie. Et là où ils ont choisi de ne pas rendre de faux témoignage, ils ont reçu de faux témoignages de la part de ceux qu'ils avaient chassés, et ils ont convoité des femmes étrangères.

\par 8 C'est pourquoi voici, j'ai horreur de la race des hommes, et afin d'extirper ma création, ceux qui mourront seront multipliés au-dessus du nombre de ceux qui sont nés. Car la maison de Jacob est souillée par les iniquités et les impiétés d'Israël se sont multipliées et je ne peux pas [quelques mots perdus] détruire entièrement la tribu de Benjamin, parce qu'elle a d'abord été emmenée après Michas. Et le peuple d'Israël ne restera pas non plus impuni, mais ce sera pour lui une offense à jamais, dans la mémoire de toutes les générations.

\par 9 Mais je livrerai Michas au feu. Et sa mère dépérira à ses yeux, vivant sur la terre, et des vers sortiront de son corps. Et quand ils se parleront, elle dira comme une mère réprimandant son fils : Voici quel péché tu as commis. Et il répondra comme à un fils obéissant à sa mère et agissant avec ruse : Et tu as commis une iniquité encore plus grande. Et l'image de la colombe qu'il a faite sera pour lui crever les yeux, et l'image de l'aigle sera pour que ses ailes répandent du feu, et les images des enfants qu'il a faits seront pour lui gratter les côtés, et car l'image du lion qu'il a fait lui sera comme des puissants qui le tourmentent.

\par 10 Et ainsi je ferai non seulement à Michas, mais aussi à tous ceux qui pèchent contre moi. Et maintenant, laissez la course. des hommes savent qu’ils ne me provoqueront pas par leurs propres inventions. Ce châtiment ne viendra pas seulement à ceux qui fabriquent des idoles, mais il appartiendra à chaque homme de savoir avec quel péché il a commis ce péché. C'est pourquoi s'ils disent des mensonges devant moi, j'ordonnerai au ciel et il leur privera de pluie. Et si quelqu'un convoite les biens de son prochain, j'ordonnerai la mort et elle lui refusera le fruit de son corps. Et s’ils jurent faussement par mon nom, je ne supporterai pas leur prière. Et quand l'âme se séparera du corps, alors ils diront : Ne pleurons pas les choses que nous avons souffertes, mais parce que tout ce que nous avons imaginé, nous le recevrons aussi.

\chapter{45}

\par 1 Et il arriva à ce moment-là qu'un certain homme de la tribu de Lévi arriva à Gabaon, et comme il désirait y demeurer, le soleil se coucha. Et quand il entrait là, ceux qui y habitaient ne le souffraient pas. Et il dit à son garçon : Va, conduis le mulet, et nous irons à la ville de Noba, peut-être qu'on nous permettra d'y entrer. Et il vint là et s'assit dans la rue de la ville. Et personne ne lui dit : Viens dans ma maison.

\par 2 Mais il y avait là un certain Lévite nommé Béthac. Celui-ci le vit et lui dit : Es-tu Beel de ma tribu ? Et il a dit : je le suis. Et il lui dit : Ne connais-tu pas la méchanceté des habitants de cette ville ? Qui t'a conseillé d'entrer ici ? Hâtez-vous, sortez d'ici, et entrez dans ma maison où j'habite, et j'y demeure aujourd'hui, et le Seigneur fermera leur cœur devant nous, comme il a enfermé les hommes de Sodome devant la face de Lot. Et il entra dans la ville et y resta cette nuit-là.

\par 3 Et tous les habitants de la ville se rassemblèrent et dirent à Béthac : Fais sortir ceux qui sont venus vers toi aujourd'hui, sinon nous les brûlerons, eux et toi, au feu. Et il sortit vers eux et leur dit : Ne sont-ils pas nos frères ? Ne les traitons pas mal, de peur que nos péchés ne se multiplient contre nous. Et ils répondirent : Il n'a jamais été admis que des étrangers donnaient des ordres aux habitants. Et ils entrèrent avec violence et l'enlevèrent lui et sa concubine et les chassèrent dehors, et ils dirent : «Laissez partir l'homme, mais ils ont maltraité sa concubine jusqu'à ce qu'elle meure ; car elle avait autrefois transgressé son mari en péchant avec les Amalécites, et c'est pourquoi le Seigneur Dieu l'a livrée entre les mains des pécheurs.

\par 4 Et le jour étant venu, Béel sortit et trouva sa concubine morte. Et il la déposa sur le mulet et sortit en toute hâte et arriva à Gadès. Et il prit son corps, le divisa et l'envoya en toutes parties (ou par portions) dans les douze tribus, en disant : Ces choses m'ont été faites dans la ville de Noba, car les habitants de là se sont soulevés contre moi pour me tuer et a pris ma concubine, m'a fait taire et l'a tuée. Et si cela vous plaît, gardez le silence et laissez le Seigneur juger ; mais si vous voulez le venger, le Seigneur vous aidera.

\par 5 Et tous les hommes, même les douze tribus, furent confondus. Et ils se rassemblèrent à Silo et dirent chacun à son prochain : Une telle iniquité a-t-elle été commise en Israël ?

\par 6 Et l'Éternel dit à l'Adversaire : Vois-tu comment ce peuple insensé est troublé ? À l'heure où ils auraient dû mourir, même lorsque Michas agissait astucieusement pour tromper le peuple avec ces choses, c'est-à-dire avec la colombe et l'aigle et avec l'image d'hommes et de veaux et d'un lion et d'un dragon, alors étaient-ils pas bougé. Et c'est pourquoi, parce qu'ils n'ont pas été irrités, que leurs conseils soient maintenant vains et que leur cœur soit ému, afin que ceux qui permettent le mal soient consumés comme les pécheurs.

\chapter{46}

\par 1 Et quand le jour fut venu, le peuple d'Israël fut très ému et dit : Montons et examinons le péché qui a été commis, afin que l'iniquité nous soit ôtée. Et ils parlèrent ainsi, et dirent : Examinons d'abord le Seigneur et sachons s'il livrera nos frères entre nos mains. Et sinon, abstenons-nous. Et Phinées leur dit : Offrons la démonstration et la vérité. Et le Seigneur leur répondit et dit : Montez, car je les livrerai entre vos mains. Mais il les a trompés, afin d'accomplir sa parole.

\par 2 Et ils montèrent au combat et arrivèrent à la ville de Benjamin et envoyèrent des messagers disant : Envoyez-nous les hommes qui ont commis ce mal et nous vous épargnerons, mais rendez à chacun sa mauvaise action. Et les fils de Benjamin endurcirent leur cœur et dirent aux fils d'Israël : Pourquoi devrions-nous vous livrer nos frères ? Si vous ne les épargnez pas, nous combattrons même contre vous. Et les fils de Benjamin sortirent contre les enfants d'Israël et les poursuivirent, et les enfants d'Israël tombèrent devant eux et en frappèrent 45 000 hommes.

\par 3 Et le cœur du peuple fut très tourmenté, et ils vinrent à Silo en pleurs et en deuil et dirent : Voici, l'Éternel nous a livrés devant les habitants de Noba. Maintenant, demandons au Seigneur lequel d'entre nous a péché. Et ils consultèrent le Seigneur et il leur dit : Si vous le voulez, montez et combattez, et ils seront livrés entre vos mains ; et alors on vous dira pourquoi vous êtes tombés devant eux. Et ils partirent le deuxième jour pour lutter contre eux. Et les enfants de Benjamin sortirent et poursuivirent Israël et en frappèrent 46 000 hommes.

\par 4 Et le cœur du peuple fut tout à fait fondu et ils dirent : Dieu a-t-il voulu tromper son peuple ? ou a-t-il ordonné, à cause du mal qui est fait, que les innocents tombent aussi bien que ceux qui font le mal ? Et quand ils parlèrent ainsi, ils se prosternèrent devant l'arche de l'alliance de l'Éternel, déchirèrent leurs vêtements et mirent de la cendre sur leurs têtes, eux et Phinées, fils du prêtre Éléazar, qui prièrent et dit : Quelle est cette tromperie avec laquelle tu as nous a trompés, Seigneur ? Si ce que les enfants de Benjamin ont fait est juste devant toi, pourquoi ne nous l'as-tu pas dit, afin que nous puissions y réfléchir ? Mais si cela ne te plaisait pas, pourquoi nous as-tu laissé tomber devant eux ?



\chapter{47}

\par 1 Et Phinées ajouta et dit : Ô Dieu de nos pères, écoute ma voix, et dis aujourd'hui à ton serviteur si c'est bien fait à tes yeux, ou si par hasard le peuple a péché et que tu veux détruire son mal, que tu tu pourras corriger parmi nous aussi ceux qui ont péché contre toi. Car je me souviens que dans ma jeunesse, Jambri avait péché aux jours de Moïse, ton serviteur. En vérité, j'y suis entré, j'étais zélé dans mon âme, et je les ai élevés tous les deux sur mon épée, et le reste se serait levé contre moi pour mettre tu m'as tué, et tu as envoyé ton ange, tu as frappé 24 000 hommes d'entre eux et tu m'as délivré de leurs mains.

\par 2 Et maintenant tu as envoyé les onze tribus et tu les as amenés ici en disant : Va et frappe-les. Et quand ils partirent, ils furent livrés. Et maintenant, ils disent que les déclarations de ta vérité sont devant toi. Et maintenant, Seigneur, Dieu de nos pères, ne le cache pas à ton serviteur, mais dis-nous pourquoi tu as commis cette iniquité contre nous.

\par 3 Et quand l'Eternel vit que Phinées priait avec ferveur devant lui, il lui dit : J'ai juré par moi-même, dit l'Eternel, que si je n'avais pas juré, je ne me serais pas souvenu de toi pendant que tu as parlé, et je ne me serais pas souvenu de toi dans tes paroles. Je vous ai répondu aujourd'hui. Et maintenant, dis au peuple : Levez-vous et écoutez la parole du Seigneur,

\par 4 Ainsi parle l'Éternel : Il y avait un certain lion puissant au milieu de la forêt, et toutes les bêtes lui confièrent la forêt afin qu'il la garde par sa puissance, de peur que par hasard d'autres bêtes ne viennent la ravager. Et tandis que le lion le gardait, des bêtes des champs arrivèrent d'une autre forêt et dévorèrent tous les petits des bêtes et dévastèrent le fruit de leur corps, et le lion le vit et se tut. Or les bêtes étaient en paix, car elles avaient confié la forêt au lion et ne s'apercevaient pas que leurs petits étaient détruits.

\par 5 Et après un certain temps, une toute petite bête se leva, parmi celles qui avaient confié la forêt au lion, et dévora le moindre des petits d'une autre bête très méchante. Et voici, le lion cria et attira toutes les bêtes de la forêt, et elles se battaient entre elles, et chacun se battait contre son prochain.

\par 6 Et comme beaucoup de bêtes étaient détruites, un autre petit, sorti d'une autre forêt semblable à elle, le vit et dit : N'as-tu pas détruit autant de bêtes ? Quelle iniquité est-ce qu'au commencement, alors que de nombreuses bêtes et leurs petits étaient injustement détruits par d'autres bêtes méchantes, et que toutes les bêtes auraient dû être poussées à se venger, voyant que le fruit de leur corps était dépouillé en vain, alors tu tu as gardé le silence et tu n'as rien dit, mais maintenant un petit d'une bête méchante a péri, et tu as excité toute la forêt pour que toutes les bêtes se dévorent les unes les autres sans cause, et que la forêt soit diminuée. Maintenant donc tu dois d'abord être détruit, et ainsi le reste soit établi. Et quand les petits des bêtes entendirent cela, ils tuèrent d'abord le lion, et mirent sur eux le petit à sa place, et ainsi le reste des bêtes fut soumis ensemble.

\par 7 Michas s'est levé et vous a enrichis par ce qu'il a commis, lui et sa mère. Et il y avait des choses mauvaises et mauvaises que personne n'avait imaginées avant eux, mais dans sa subtilité il a fait des images taillées qui n'avaient pas été faites jusqu'à ce jour, et personne n'a été irrité, mais vous avez tous été égarés et vous avez vu le fruit. de ton corps gâté, et tu as gardé le silence comme ce méchant lion.

\par 8 Et maintenant, quand vous avez vu comment la concubine de cet homme, qui avait souffert du malheur, était morte, vous avez tous été émus et êtes venus vers moi en disant : Veux-tu livrer les enfants de Benjamin entre nos mains ? C'est pourquoi je vous ai trompé et j'ai dit : Je vous les livrerai. Et maintenant, j'ai détruit ceux qui se taisaient alors, et ainsi je vais me venger de tous ceux qui ont fait du mal contre moi. Mais vous, montez maintenant, car je vous les livrerai.

\par 9 Et tout le peuple se leva d'un commun accord et s'en alla. Et les enfants de Benjamin se précipitèrent contre eux et pensèrent qu'ils les vaincre comme auparavant. Et ils ne savaient pas que leur méchanceté s'était accomplie sur eux. Et quand ils furent arrivés comme d'abord et les poursuivaient, le peuple s'enfuit devant eux pour leur céder la place, puis ils sortirent de leurs embuscades, et les enfants de Benjamin étaient au milieu d'eux.

\par 10 Alors ceux qui fuyaient rebroussèrent chemin, et les hommes de la ville de Noba furent tués, hommes et femmes, soit 85 000 hommes, et les enfants d'Israël incendièrent la ville et prirent le butin et détruisirent tout avec le bord. de l'épée. Et il ne resta plus aucun homme des enfants de Benjamin, à l'exception de 600 hommes qui s'enfuirent et ne furent pas retrouvés dans la bataille. Et tout le peuple retourna à Silo et avec eux Phinées, fils du prêtre Éléazar.

\par 11 Or, voici ceux qui restaient de la race de Benjamin, les chefs de la tribu, de dix familles dont les noms sont ceux-ci : de la première famille : Ezbaile, Zieb, Balac, Reindebac, Belloch ; et de la 2ème famille : Nethac, Zenip, Phenoch, Demech, Geresaraz ; et de la 3ème famille : Jerimuth, Veloth, Amibel, Genuth, Nephuth, Phienna ; et de la 4ème ville : Gemuph, Eliel, Gemoth, Soleph, Raphaph et Doffo ; et de la 5ème famille : Anuel, Code, Fretan, Remmon, Peccan, Nabath ; et de la 6ème famille : Rephaz, Sephet, Araphaz, Metach, Adhoc, Balinoc ; et de la 7ème famille : Bénin, Mephiz, Araph, Ruimel, Belon, Iaal, Abac ; et (de) la (8ème, 9ème et) 10ème famille : Enophlasa, Melec, Meturia, Meac ; et les autres princes de la tribu qui restaient, au nombre de soixante.

\par 12 Et à ce moment-là, l'Éternel rendit à Michas et à sa mère tout ce qu'il avait dit. Et Michas fut fondu dans le feu et sa mère se languit, comme l'Éternel l'avait dit à leur sujet.

\chapter{48}

\par 1 À ce moment-là aussi Phinées se coucha pour mourir, et le Seigneur lui dit : Voici, tu as dépassé les 120 ans qui étaient ordonnés à tous les hommes. Et maintenant, lève-toi et va d'ici et habite sur la montagne Danaben et demeure-y de nombreuses années, et j'ordonnerai à mon aigle et il te nourrira là, et tu ne descendras plus vers les hommes jusqu'à ce que le moment soit venu et que tu sois éprouvé dans le temps. Et alors tu fermeras le ciel, et sur ta parole il s'ouvrira. Et après cela, tu seras élevé à l'endroit où ont été élevés ceux qui étaient avant toi, et tu y seras jusqu'à ce que je me souvienne du monde. Et alors je vous amènerai et vous goûterez à ce qu'est la mort.

\par 2 Et Phinées monta et fit tout ce que l'Éternel lui avait ordonné. Or, au temps où il le nomma prêtre, il l'oignit à Silo.

\par 3 Et à ce moment-là, quand il montait, alors il arriva que les enfants d'Israël, pendant qu'ils célébraient la Pâque, commandèrent aux enfants de Benjamin en disant : Montez et prenez des femmes de force.. parce que nous ne pouvons pas donner vous, nos filles, car nous avons juré au temps de notre colère; et il ne se peut pas qu'une tribu périsse parmi Israël. Et les enfants de Benjamin montèrent et prirent des femmes, bâtirent Gabaon et commencèrent à y habiter.

\par 4 Et tandis que pendant ce temps les enfants d'Israël étaient en repos, ils n'avaient pas de prince en ce temps-là, et chacun faisait ce qui lui semblait droit.

\par 5 Tels sont les commandements, les jugements, les témoignages et les manifestations qui existaient du temps des juges d'Israël, avant qu'un roi ne règne sur eux.

\chapter{49}

\par 1 Et à ce moment-là, les enfants d'Israël commencèrent à consulter l'Éternel, et dirent : Laissons-nous ; tous tirons au sort, afin que nous puissions voir qui peut nous gouverner comme Cénez, car peut-être trouverons-nous un homme qui pourra nous délivrer de nos afflictions, car il n'est pas avantageux que le peuple soit sans prince.

\par 2 Et ils jetèrent le sort et ne trouvèrent personne ; Et le peuple fut très attristé et dit : Le peuple n'est pas digne d'être entendu par l'Éternel, car il ne nous a pas répondu. Maintenant donc, tirons au sort même par tribus, si par hasard Dieu veut être apaisé par une multitude, car nous savons qu'il se réconciliera avec ceux qui sont dignes de lui. Et ils tirèrent au sort par tribus, et le sort ne tomba sur aucune tribu. Et Israël dit : Choisissons-en un parmi nous, car nous sommes dans une situation difficile, car nous comprenons que Dieu abhorre son peuple et que son âme est mécontente de nous.

\par 3 Et quelqu'un répondit et dit au peuple, dont le nom était Nethez : Ce n'est pas lui qui nous hait, mais c'est nous-mêmes qui nous sommes fait haïr, pour que Dieu nous abandonne. C'est pourquoi, même si nous mourons, ne l'abandonnons pas, mais fuyons vers lui pour trouver refuge ; car nous avons marché dans nos mauvaises voies et n'avons pas connu celui qui nous a créés, et donc notre dessein sera vain. Car je sais que Dieu ne nous rejettera pas pour toujours, et qu'il ne haïra pas son peuple de génération en génération. C'est pourquoi maintenant, soyez forts et prions encore et tirons au sort par ville, car même si nos péchés s'agrandissent, son la patience ne faillit pas.

\par 4 Et ils tirèrent au sort les villes, et le sort tomba sur Armathem. Et le peuple dit : Armathem est-il considéré comme juste entre toutes les villes d'Israël, pour qu'il l'ait ainsi choisie devant toutes les villes ? Et chacun dit à son prochain : Dans cette même ville tirée au sort, tirons le sort entre hommes, et voyons qui l'Éternel a choisi parmi elle.

\par 5 Et ils jetèrent le sort par hommes, et il ne prit personne sauf Elchana, car le sort tomba sur lui, et le peuple le prit et dit : Venez et soyez notre chef. Et Elchana dit au peuple : Je ne peux pas être prince sur ce peuple, et je ne peux pas non plus juger qui peut être prince sur vous. Mais si mes péchés m'ont découvert, et que le sort tombe sur moi, je me tuerai, afin que vous ne me souilliez pas ; car il est juste que je meure uniquement pour mes propres péchés et que je n'aie pas à supporter le poids du peuple.

\par 6 Et quand le peuple vit que ce n'était pas la volonté d'Elchana de prendre le commandement sur eux, ils prièrent de nouveau l'Éternel en disant : Ô Seigneur, Dieu d'Israël, pourquoi as-tu abandonné ton peuple dans la victoire de l'ennemi et avez-vous négligé votre héritage au temps des difficultés ? Voici, même celui qui a été tiré au sort n'a pas accompli ton commandement ; mais seulement ceci est arrivé, que le sort s'est jeté sur lui, et nous avons cru que nous avions un prince. Et voici, il lutte aussi contre le sort. De qui aurons-nous encore besoin, ou vers qui fuirons-nous, et où est le lieu de notre repos ? Car si les ordonnances que tu as faites avec nos pères, disant : J'agrandirai ta postérité, et qu'ils le sauront, sont vraies, alors il vaudrait mieux que tu nous dises : Je retrancherai ta postérité, plutôt que de le faire. sans égard à nos racines.

\par 7 Et Dieu leur dit : Si effectivement je vous rétribuais selon vos mauvaises actions, je ne devrais pas prêter l'oreille à votre peuple ; mais que dois-je faire, parce que mon nom vient pour être invoqué sur vous ? Et maintenant, sachez qu'Elchana, sur qui le sort est tombé, ne peut pas régner sur vous, mais c'est plutôt son fils qui naîtra de lui ; il sera votre prince et prophétisera ; et désormais il ne vous manquera plus de prince pendant de nombreuses années.

\par 8 Et le peuple dit : Voici, Seigneur, Elchana a dix fils, et lequel d'entre eux sera prince ou prophétisera ? Et Dieu dit : Aucun des fils de Phénenne ne peut être prince du peuple, mais celui qui est né de la femme stérile que je lui ai donnée pour femme, il sera prophète devant moi, et je l'aimerai même. comme j'ai aimé Isaac, et son nom sera devant moi pour toujours. Et le peuple dit : Voici, il se peut que Dieu se souvienne de nous pour nous délivrer de la main de ceux qui nous haïssent. Et ce jour-là, ils offraient des offrandes de paix et faisaient un festin selon leurs ordres.

\chapter{50}

\par 1 Or [alors que] Elchana avait deux femmes, le nom de l'une était Anna et le nom de l'autre Phenenna. Et comme Phénenne avait des fils et qu'Anne n'en avait pas, Phénenne lui fit des reproches, disant : Que te sert-il qu'Elchana, ton mari, t'aime ? mais tu es un arbre sec. Je sais en outre qu'il m'aimera, car il se réjouit de voir mes fils debout autour de lui comme la plantation d'un olivier.

\par 2 Et ainsi, quand elle lui faisait des reproches tous les jours, et qu'Anne avait le cœur très douloureux, et qu'elle craignait Dieu dès sa jeunesse, il arriva que lorsque le bon jour de la Pâque approchait, et que son mari monta faire des sacrifices, que Phenenna injuria Anna en disant : Une femme n'est vraiment pas aimée même si son mari l'aime ou sa beauté. Qu'Anne ne se vante donc pas de sa beauté, mais que celui qui se vante se vante lorsqu'il voit sa postérité devant lui ; et s'il n'en est pas ainsi entre les femmes, même le fruit de leurs entrailles, alors l'amour ne comptera plus. À quoi sert Rachel que Jacob l’aime ? si le fruit de ses entrailles ne lui avait été donné, son amour n'aurait sûrement servi à rien ? Et quand Anna entendit cela, son âme fondit en elle et ses yeux se remplirent de larmes.

\par 3 Et son mari la vit et dit : Pourquoi es-tu triste et ne manges-tu pas, et pourquoi ton cœur au-dedans de toi est-il abattu ? Ta conduite n'est-elle pas meilleure que celle des dix fils de Phenenna ? Anna l'écouta, se leva après avoir mangé, et se rendit à Silo, dans la maison de l'Éternel, où demeurait Héli, le prêtre, que Phinées, fils d'Éléazar, le prêtre, avait présenté comme cela lui avait été ordonné.

\par 4 Et Anna pria et dit : N'as-tu pas, Seigneur, examiné le cœur de toutes les générations avant de former le monde ? Mais qu'est-ce que le ventre qui naît ouvert, ou celui qui est enfermé qui meurt, si tu ne le veux ? Et maintenant, que ma prière monte aujourd'hui devant toi, de peur que je ne descende d'ici vide, car tu connais mon cœur, comment j'ai marché devant toi depuis les jours de ma jeunesse.

\par 5 Et Anna ne voulait pas prier à haute voix comme le font tous les hommes, car elle pensait à ce moment-là en disant : De peur que, par hasard, je ne sois pas digne d'être entendue, et il arrivera que Phénenne m'enviera encore plus et me reprochera comme elle le fait quotidiennement. dit : Où est ton Dieu en qui tu as confiance ? Et je sais que ce n'est pas celle qui a beaucoup de fils qui est enrichie, ni celle qui en manque n'est pauvre, mais que celle qui abonde dans la volonté de Dieu, elle est enrichie. Car ceux qui savent pourquoi j'ai prié, s'ils s'aperçoivent que je ne suis pas exaucé dans ma prière, blasphémeront. Et je n'aurai pas seulement un témoignage dans mon âme, car mes larmes sont aussi les servantes de mes prières.

\par 6 Et pendant qu'elle priait, Héli le prêtre, voyant qu'elle était affligée dans son esprit et se comportait comme une ivrogne, lui dit : Va, retire ton vin de chez toi. Et elle dit : Ma prière est-elle tellement entendue qu'on me traite d'ivrogne ? En vérité, je suis ivre de tristesse et j'ai bu la coupe de mes pleurs.

\par 7 Et Héli le prêtre lui dit : Raconte-moi ton opprobre. Et elle lui dit : Je suis la femme d'Elchana, et parce que Dieu a sûrement fermé mon ventre, c'est pourquoi j'ai prié devant lui afin que je ne puisse pas quitter ce monde pour lui sans fruit, ni mourir sans laisser ma propre image. Et Héli le prêtre lui dit : Va, car je sais pourquoi tu as prié, et. ta prière est entendue.

\par 8 Mais Héli, le prêtre, ne voulut pas lui dire qu'un prophète devait naître d'elle d'avance, car il avait entendu quand l'Éternel avait parlé de lui. Et Anna entra dans sa maison et fut consolé de son chagrin, mais elle ne parla à personne de ce pour lequel elle avait prié.

\chapter{51}

\par 1 Et à cette époque-là, elle conçut et enfanta un fils et appela son nom Samuel, ce qui signifie Puissant, selon le nom que Dieu lui donna lorsqu'il prophétisa à son sujet. Et Anna s'assit et allaita l'enfant jusqu'à ce qu'il ait deux ans, et après l'avoir sevré, elle monta avec lui, portant des cadeaux dans ses mains. L'enfant était très beau et le Seigneur était avec lui.

\par 2 Et Anna plaça l'enfant devant le visage d'Héli et lui dit : Ceci est le désir que j'ai désiré, et ceci est la demande que j'ai recherchée. Et Héli lui dit : Non seulement tu l'as cherché, mais le peuple a aussi prié pour cela. Ce n'est pas seulement ta demande, mais elle a été promise autrefois aux tribus ; et par cet enfant ton ventre est justifié, pour que tu établisses la prophétie devant le peuple, et que tu fasses du lait de tes seins une source pour les douze tribus.

\par 3 Et quand Anna entendit cela, elle pria et dit : Venez à ma voix, vous tous, peuples, et prêtez l'oreille à ma parole, vous tous, royaumes, car ma bouche est ouverte pour que je puisse parler, et mes lèvres sont commandées. afin que je puisse chanter les louanges du Seigneur. Laissez tomber, ô mes seins, et donnez vos témoignages, car il vous est réservé de téter. Car celui que vous allaitez sera établi, et par ses paroles les peuples seront éclairés, et il montrera aux nations leurs frontières, et sa corne sera grandement exaltée.

\par 4 Et c'est pourquoi je prononcerai ouvertement mes paroles, car de moi surgira l'ordonnance du Seigneur, et tous les hommes trouveront la vérité. Ne vous hâtez pas de parler avec orgueil, ni de prononcer de grandes paroles de votre bouche, mais prenez plaisir à vous vanter lorsque la lumière sortira d'où naîtra la sagesse, afin que ceux qui ont le plus de biens ne soient pas appelés riches, ni ceux qui possèdent le plus de biens. celles qui ont enfanté abondamment sont appelées mères : car la stérile est rassasiée, et celle qui s'est multipliée en fils est devenue vide ;

\par 5 Car l'Éternel tue par jugement et vivifie par miséricorde ; car les impies sont dans ce monde ; c'est pourquoi il vivifie le juste quand il veut, mais il enferme les impies dans les ténèbres. Mais il réserve aux justes leur lumière, et quand les impies seront morts, alors ils périront, et quand les justes s'endormiront, alors ils seront délivrés. Et ainsi tout jugement durera jusqu'à ce que soit révélé celui qui le détient.

\par 6 Parle, parle, ô Anne, et ne garde pas le silence : chante des louanges, ô fille de Bathuel, à cause des merveilles que Dieu a faites avec toi. Qui est Anna pour qu'un prophète sorte d'elle ? ou qui est la fille de Bathuel, pour qu'elle enfante une lumière qui déjoue les peuples ? Lève-toi aussi, Elchana, et ceins tes reins. Chante des louanges pour les signes du Seigneur : Car à propos de ton fils Asaph a prophétisé dans le désert en disant : Moïse et Aaron parmi ses prêtres et Samuel parmi eux. Voici, la parole s'accomplit et la prophétie s'accomplit. Et ces choses durent ainsi, jusqu'à ce qu'elles donnent une corne à son oint, et que la puissance s'attache au trône de son roi. Pourtant, que mon fils se tienne ici et fasse son ministère, jusqu'à ce qu'une lumière apparaisse pour ce peuple.

\par 7 Et ils partirent de là et partirent avec joie, se réjouissant et exultant de cœur pour toute la gloire que Dieu avait façonnée avec eux. Mais le peuple descendit d'un commun accord à Silo avec des tambourins et des danses, avec des luths et des harpes, et vint vers Héli le prêtre et lui offrit Samuel, qu'ils placèrent devant la face du Seigneur, l'oignirent et dirent : Que le prophète vive parmi le peuple, et qu'il soit longtemps une lumière pour cette nation.

\chapter{52}

\par 1 Mais Samuel était un très jeune enfant et ne savait rien de toutes ces choses. Et tandis qu'il servait devant l'Éternel, les deux fils d'Héli, qui ne marchaient pas dans les voies de leurs pères, commencèrent à faire du mal au peuple et multiplièrent leurs iniquités. Et ils habitaient près de la maison de Béthac, et quand le peuple se rassemblait pour sacrifier, Ophni et Phinées vinrent et irritèrent le peuple, s'emparant des offrandes avant que les choses saintes fussent offertes au Seigneur.

\par 2 Et cela n'a plu ni à l'Éternel, ni au peuple, ni à leur père. Et leur père leur dit ainsi : Quel est ce bruit que j'entends parler de vous ? Ne savez-vous pas que j'ai reçu la place que Phinées m'a confiée ? Et si nous gaspillons ce que nous avons reçu, que dirons-nous si celui qui l'a commis le réclame à nouveau et nous contrarie pour ce qu'il nous a confié ? Maintenant donc aplanissez vos chemins et marchez dans les bons sentiers, et vos actions dureront. Mais si vous me contredisez et ne vous abstenez pas de vos mauvaises intentions, vous vous détruirez vous-mêmes, et le sacerdoce sera vain, et ce qui était sanctifié ne servira à rien. Et alors on dira : Le bâton d’Aaron n’a pas germé en vain, et la fleur qui en est née est devenue vaine.

\par 3 C'est pourquoi, pendant que vous le pouvez encore, mes fils, corrigez ce que vous avez mal fait, et les hommes contre lesquels vous avez péché prieront pour vous. Mais si vous ne le voulez pas, mais persistez dans vos iniquités, je serai innocent, et non seulement je serai triste de peur (ou et maintenant je n'effacerai pas ces grands maux en vous, de peur) d'apprendre le jour de votre mort avant Je meurs, mais aussi si cela arrive (ou même si cela n'arrive pas), je serai exempt de tout blâme : et même si je suis affligé, vous périrez néanmoins.

\par 4 Et ses fils ne lui obéirent pas, car l'Éternel leur avait ordonné de mourir, parce qu'ils avaient péché. Car quand leur père leur dit : Repentez-vous de votre mauvaise voie, ils dirent : Quand nous vieillirons. , alors nous nous repentirons. Et c'est pour cette raison qu'il ne leur fut pas donné de se repentir lorsqu'ils furent réprimandés par leur père, parce qu'ils avaient toujours été rebelles et avaient agi très injustement en spoliant Israël. Mais le Seigneur était en colère contre Héli.

\chapter{53}

\par 1 Mais Samuel servait devant l'Éternel et ne savait pas encore quels étaient les oracles de l'Éternel; car il n'avait pas encore entendu les oracles de l'Éternel, car il avait 8 ans.

\par 2 Mais quand Dieu se souvenait d'Israël, il révélait ses paroles à Samuel, et Samuel dormait dans le temple de l'Éternel. Et il arriva que lorsque Dieu l'appela, il réfléchit d'abord et dit : Regarde maintenant, Samuel est jeune pour qu'il soit (ou s'il soit) bien-aimé à mes yeux ; néanmoins, parce qu'il n'a pas encore entendu la voix du Seigneur, il n'est pas non plus confirmé à la voix du Très-Haut, et pourtant il est semblable à Moïse, mon serviteur. Mais j'ai parlé à Moïse quand il avait 80 ans, mais Samuel en a 8. ans. Et Moïse vit le feu le premier et son cœur eut peur. Et si Samuel voit le feu maintenant, comment le supportera-t-il ? C’est pourquoi maintenant une voix lui parviendra comme celle d’un homme, et non comme celle de Dieu. Et quand il comprendra, alors je lui parlerai comme à Dieu.

\par 3 Et à minuit, une voix venant du ciel l'appela ; et Samuel se réveilla et perçut comme la voix d'Héli, et courut vers lui et dit : Pourquoi m'as-tu réveillé, père ? Car j'avais peur, parce que tu ne m'as jamais appelé pendant la nuit. Et Héli dit : Malheur à moi, se pourrait-il qu'un esprit impur ait séduit mon fils Samuel ? Et il lui dit : Va et dors, car je ne t'ai pas appelé. Cependant, dis-moi, si tu te souviens, combien de fois celui qui t'a appelé a pleuré. Et il a dit : Deux fois. Et Héli lui dit : Dis maintenant, de quelle voix as-tu entendu parler, mon fils ? Et il dit : C'est à toi que j'ai couru vers toi.

\par 4 Et Héli dit : En toi je vois le signe que les hommes auront à partir de ce jour pour toujours, que si l'un s'appelle deux fois dans la nuit ou à midi, ils sauront que c'est un mauvais esprit. Mais s'il appelle une troisième fois, ils sauront que c'est un ange. Et Samuel s'en alla et s'endormit.

\par 5 Et il entendit une seconde fois une voix venant du ciel, et il se leva et courut vers Héli et lui dit : Pourquoi m'a-t-il appelé, car j'ai entendu la voix d'Elchana, mon père ? C'est alors qu'Héli comprit que Dieu commençait à l'appeler. Et Héli dit : Dans ces deux voix avec lesquelles Dieu t'a appelé, il s'est comparé à ton père et à ton maître, mais maintenant, pour la troisième fois, il parlera comme Dieu.

\par 6 Et il lui dit : Avec ton oreille droite, écoute et avec ta gauche abstiens-toi. Car le prêtre Phinées nous l'a ordonné en disant : L'oreille droite entend le Seigneur la nuit, et l'oreille gauche un ange. C'est pourquoi, si tu entends de ton oreille droite, dis ainsi : Dis ce que tu veux, car je t'entends, car tu m'as formé ; mais si tu entends de l'oreille gauche, viens me le dire. Samuel s'en alla et s'endormit comme Héli le lui avait ordonné.

\par 7 Et l'Éternel ajouta et parla encore une troisième fois, et l'oreille droite de Samuel fut remplie de la voix. Et lorsqu'il s'aperçut que la parole de son père lui était parvenue, Samuel se tourna de l'autre côté et dit : Si je le peux, parle, car tu m'as formé (ou tu me connais bien).

\par 8 Et Dieu lui dit : En vérité, j'ai éclairé la maison d'Israël en Égypte et j'ai choisi pour moi en ce temps-là Moïse, mon serviteur, pour prophète, et par lui j'ai fait des merveilles pour mon peuple et je l'ai vengé de mes ennemis comme je l'ai fait. Je le ferais, et j'emmenai mon peuple dans le désert et je l'éclairai selon ce qu'il voyait.

\par 9 Et quand une tribu s'élevait contre une autre tribu, disant : Pourquoi les prêtres seuls sont-ils saints ? Je ne voulais pas les détruire, mais je leur ai dit : Donnez à chacun son bâton, et celui dont le bâton fleurit, je le choisirai pour le sacerdoce. Et quand ils eurent tous donné leurs verges comme je l'avais ordonné, alors j'ordonnai à la terre du tabernacle que la verge d'Aaron fleurisse, afin que sa lignée soit manifestée pendant plusieurs jours. Et maintenant, ceux qui ont prospéré ont abhorré mes choses saintes.

\par 10 C'est pourquoi voici, les jours viendront où je couperai (lit. arrêter) la fleur qui est sortie à ce moment-là, et j'irai contre eux parce qu'ils transgressent la parole que j'ai dite à mon serviteur Moïse. , disant : Si tu rencontres un nid, tu ne prendras pas la mère avec les petits, c'est pourquoi il leur arrivera que les mères mourront avec les enfants, et les pères périront avec les fils.

\par 11 Et quand Samuel entendit ces paroles, son cœur fut fondu, et il dit : M'est-il arrivé ainsi dans ma jeunesse que je prophétise pour la destruction de celui qui m'a nourri ? et comment alors ai-je été exaucé à la demande de ma mère ? et qui est celui qui m'a élevé ? comment m'a-t-il chargé de porter de mauvaises nouvelles ?

\par 12 Et Samuel se leva le matin et ne voulut pas le dire à Héli. Et Héli lui dit : Écoute maintenant, mon fils. Voici, avant ta naissance, Dieu a promis à Israël qu'il t'enverrait vers eux pour prophétiser. Et maintenant, quand ta mère est venue ici et a prié, car elle ne savait pas ce qui avait été fait, je lui ai dit : Va, car ce qui naîtra de toi sera pour moi un fils. Ainsi ai-je parlé à ta mère, et ainsi le Seigneur a dirigé ta voie. Et même si tu châties ton père qui nourrit, comme le Seigneur est vivant, ne me cache pas les choses que tu as entendues.

\par 13 Alors Samuel eut peur et lui rapporta toutes les paroles qu'il avait entendues. Et il dit : La chose formée peut-elle répondre à celui qui l'a formée ? De même, je ne puis pas répondre quand il enlèvera ce qu'il a donné, même le donateur fidèle, le saint qui a prophétisé, car je suis soumis à sa puissance.



\chapter{54}

\par 1 Et en ces jours-là, les Philistins rassemblèrent leur camp pour combattre contre Israël, et les enfants d'Israël sortirent pour combattre contre eux. Et lorsque le peuple d'Israël fut mis en fuite lors de la première bataille, ils dirent : Faisons monter l'arche de l'alliance du Seigneur, peut-être qu'elle combattra contre nous, car dans elle se trouvent les témoignages du Seigneur qu'il a donnés. ordonné à nos pères à Oreb.

\par 2 Et tandis que l'arche montait avec eux, lorsqu'elle entra dans le camp, l'Éternel tonna et dit : Cette fois sera comparée à ce qui était dans le désert, lorsqu'ils prirent l'arche sans mon ordre, et la destruction. leur est arrivé. De même, en ce temps-là, le peuple tombera, et l'arche sera prise, afin que je puisse punir les adversaires de mon peuple à cause de l'arche, et réprimander mon peuple parce qu'il a péché.

\par 3 Et lorsque l'arche fut entrée dans la bataille, les Philistins sortirent à la rencontre des enfants d'Israël et les frappèrent. Et il y avait là un certain Golia, un Philistin, qui vint jusqu'à l'arche, et Ophni et Phinées, fils d'Héli, et Saül, fils de Cis, tenaient l'arche. Et Golia le prit avec sa main gauche et tua Ophni et Phinées.

\par 4 Mais Saül, parce qu'il avait les pieds légers, s'enfuit devant lui ; Il déchira ses vêtements, mit de la cendre sur sa tête et vint vers le prêtre Héli. Et Héli lui dit : Dis-moi, qu'est-il arrivé au camp ? Et Saül lui dit : Pourquoi me demandes-tu ces choses ? car le peuple est vaincu, et Dieu a abandonné Israël. Oui, et les prêtres aussi sont tués par l'épée, et l'arche est livrée aux Philistins.

\par 5 Et quand Héli apprit la prise de l'arche, il dit : Voici, Samuel a prophétisé sur moi et mes fils que nous mourrions ensemble, mais il ne m'a pas nommé l'arche. Et maintenant les témoignages sont livrés à l’ennemi, et que puis-je dire de plus ? Voici, Israël a péri à cause de la vérité, car les jugements lui sont retirés. Et comme Héli était complètement désespéré, il tomba de son siège. Et ils moururent en un jour, Héli, Ophni et Phinées, ses fils.

\par 6 Et la femme du fils d'Héli était assise et travaillait ; et quand elle entendit ces choses, tous ses entrailles fondirent. Et la sage-femme lui dit : Prends courage, et ne laisse pas ton âme se lasser, car un fils t'est né. Et elle lui dit : Voici maintenant qu'une seule âme est née et nous mourons tous les quatre, c'est-à-dire mon père et ses deux fils et sa belle-fille. Et elle l'appela : Où est la gloire ? disant : La gloire de Dieu a péri en Israël parce que l'arche de l'Éternel a été emmenée captive. Et quand elle eut dit cela, elle rendit l'âme.

\chapter{55}

\par 1 Mais Samuel ne savait rien de toutes ces choses, car trois jours avant la bataille, Dieu le renvoya, lui disant : Va et regarde le lieu d'Arimatha, là sera ta demeure. Et quand Samuel apprit ce qui était arrivé à Israël, il vint et pria l'Éternel, disant : Voici, maintenant, c'est en vain qu'on me refuse la compréhension, afin que je puisse voir la destruction de mon peuple. Et maintenant, je crains que mes jours ne vieillissent dans le mal et que mes années ne se terminent dans le chagrin, car si l'arche du Seigneur n'est pas avec moi, pourquoi vivrais-je encore ?

\par 2 Et l'Éternel lui dit : Ne sois pas attristé, Samuel, que l'arche ait été enlevée. Je le ramènerai, et je renverserai ceux qui l'ont pris, et je vengerai mon peuple de ses ennemis. Et Samuel dit : Voici, même si tu les venge à temps, selon ta longanimité, que ferons-nous, nous qui mourrons maintenant ? Et Dieu lui dit : Avant de mourir, tu verras la fin que j'apporterai à mes ennemis, par laquelle les Philistins périront et seront tués par les scorpions et par toutes sortes de bestioles rampantes.

\par 3 Et lorsque les Philistins eurent placé l'arche de l'Éternel qui avait été prise dans le temple de Dagon, leur dieu, et furent venus s'enquérir de Dagon au sujet de leur sortie, ils le trouvèrent tombé sur sa face et ses mains et ses pieds posés. devant l'arche. Et ils sortirent le premier matin, après avoir crucifié ses prêtres. Et le deuxième jour, ils arrivèrent et trouvèrent comme la veille, et la destruction fut grandement multipliée parmi eux.

\par 4 C'est pourquoi les Philistins se rassemblèrent à Accaron, et dirent chacun à son prochain : Voici maintenant, nous voyons que la destruction s'agrandit parmi nous, et que le fruit de notre corps périt, car les reptiles qui nous sont envoyés détruisent. celles qui sont enceintes, celles qui allaitent et celles qui allaitent. Et ils dirent : Voyons pourquoi la main du Seigneur est forte contre nous. Est-ce pour le bien de l'arche ? car chaque jour notre dieu est retrouvé tombé sur sa face devant l'arche, et nous avons tué nos prêtres sans raison à maintes reprises.

\par 5 Et les sages des Philistins dirent : Voici, maintenant par ceci pouvons-nous savoir si l'Éternel nous a envoyé la destruction à cause de son arche ou si une affliction fortuite nous arrive pour un temps ?

\par 6 Et maintenant, tandis que toutes celles qui sont enceintes et qui allaitent meurent, et que celles qui allaitent sont rendues sans enfants, et que celles qui sont allaitées périssent, nous prendrons aussi les vaches qui allaitent et les attelerons à une charrette neuve, et tu placeras l'arche dessus, et tu enfermeras les petits des vaches. Et si les vaches sortent et ne retournent pas vers leurs petits, nous saurons que nous avons souffert ces choses à cause de l'arche ; mais s'ils refusent de partir, désireux de retrouver leurs petits, nous saurons que le moment de notre chute est venu pour nous.

\par 7 Et certains des sages et des devins répondirent : Essayez non seulement cela, mais plaçons les vaches à la tête des trois chemins qui entourent Accaron. Car le chemin du milieu mène à Accaron, et le chemin à droite mène à la Judée, et le chemin à gauche mène à Samarie. Et dirigez les vaches qui portent l’arche sur la voie médiane. Et s’ils partent par le droit chemin directement vers la Judée, nous saurons qu’en vérité le Dieu des Juifs nous a ravagés ; mais s'ils empruntent ces autres voies, nous saurons qu'un temps mauvais (lit. puissant) nous est arrivé, car maintenant nous avons renié nos dieux.

\par 8 Et les Philistins prirent des vaches laitières et les attelèrent à un chariot neuf et y placèrent l'arche, et les placèrent à la tête des trois voies, et ils enfermèrent leurs petits à la maison. Et les vaches, bien qu'elles beuglaient et aspiraient à leurs petits, avançaient néanmoins par le chemin droit qui mène à la Judée. Et alors ils savaient qu’ils étaient dévastés à cause de l’arche.

\par 9 Et tous les Philistins se rassemblèrent et ramenèrent l'arche à Silo avec des tambourins, des flûtes et des danses. Et à cause des insectes nuisibles qui les ravageaient, ils fabriquèrent des sièges en or et sanctifièrent l'arche.

\par 10 Et dans ce fléau des Philistins, le nombre de celles qui moururent enceintes était de 75 000, et celui des enfants qui allaitaient 65 000, et celui de celles qui allaitaient 55 000, et celui des hommes 25 000. Et le pays se reposa pendant sept ans.

\chapter{56}

\par 1 Et à cette époque-là, les enfants d'Israël avaient besoin d'un roi dans leur juste. Et ils se rassemblèrent auprès de Samuel et lui dirent : Voici, maintenant tu es vieux, et tes fils ne marchent pas dans les voies de l'Éternel ; Maintenant donc, établis sur nous un roi pour juger entre nous, car s'accomplit la parole que Moïse a dite à nos pères dans le désert, en disant : Tu établiras sûrement sur toi un prince d'entre tes frères.

\par 2 Et quand Samuel entendit parler du royaume, il fut très attristé dans son cœur, et dit : Voici maintenant, je vois qu'il n'y a plus (ou pas encore) pour nous de temps de royaume perpétuel, ni de construction du royaume. maison du Seigneur notre Dieu, dans la mesure où ceux-ci désirent un roi avant le temps. Et maintenant, si le Seigneur le refuse complètement (ou même si le Seigneur le veut), il me semble qu'un roi ne peut pas être établi.

\par 3 Et l'Éternel lui dit pendant la nuit : Ne sois pas attristé, car je leur enverrai un message qui les ravagera, et lui-même sera ensuite dévasté. Or, celui qui viendra vers toi demain à la sixième heure, c'est lui qui régnera sur eux.

\par 4 Et le lendemain, Saül, fils de Cis, revenait du mont Effrem, cherchant les ânes de son père ; Et quand il fut arrivé à Armathem, il entra pour consulter Samuel au sujet des ânes. Or, il marchait près de Baam, et Saül lui dit : Où est celui qui voit ? Car à cette époque, un prophète s’appelait Voyant. Et Samuel lui dit : Je suis celui qui voit. Et il dit : Peux-tu me parler des ânes de mon père ? car ils sont perdus.

\par 5 Et Samuel lui dit : Rafraîchis-toi avec moi aujourd'hui, et demain matin je te dirai ce pour quoi tu es venu te renseigner. Et Samuel dit au Seigneur : Dirige, Seigneur, ton peuple, et révèle-moi ce que tu as décidé à son sujet. Et Saül se rafraîchit avec Samuel ce jour-là et se leva le matin. Et Samuel lui dit : Voici, sache que l'Éternel t'a choisi pour être prince de son peuple en ce moment, et qu'il a élevé tes voies, et que ton temps sera dirigé.

\par 6 Et Saül dit à Samuel : Qui suis-je, et quelle est la maison de mon père, pour que mon seigneur me parle ainsi ? Car je ne comprends pas ce que tu dis, parce que je suis un jeune homme. Et Samuel dit à Saül : Qui fera que ta parole s'accomplisse d'elle-même, afin que tu vives plusieurs jours ? mais considère ceci, que tes paroles seront comparées aux paroles d'un prophète, dont le nom sera Hieremias.

\par 7 Et comme Saül s'éloignait ce jour-là, le peuple vint vers Samuel, disant : Donne-nous un roi, comme tu nous l'as promis. Et il leur dit : Voici, le roi viendra vers vous après trois jours. Et voici, Saül est venu. Et il lui arriva tous les signes que Samuel lui avait annoncés. Ces choses ne sont-elles pas écrites dans le livre des Rois ?

\chapter{57}

\par 1 Et Samuel envoya rassembler tout le peuple, et leur dit : Voici, vous et votre roi êtes ici, et je suis entre vous, comme l'Éternel me l'a ordonné.

\par 2 Et c'est pourquoi je vous le dis, devant la face de votre roi, comme mon seigneur Moïse : le serviteur de Dieu, a dit à vos pères dans le désert, lorsque la synagogue de Core s'est levée contre lui : Vous savez que je ne vous ai rien pris et que je n'ai fait de tort à aucun d'entre vous ; et parce que certains mentirent alors et dirent : Tu as pris, la terre les engloutit.

\par 3 Maintenant donc, vous que l'Éternel n'a pas puni, répondez devant l'Éternel et devant son oint, si c'est pour cette raison que vous avez demandé un roi, parce que je vous ai mal supplié, et l'Éternel sera votre témoin. Mais si maintenant la parole du Seigneur s'accomplit, je suis libre, ainsi que la maison de mon père.

\par 4 Et le peuple répondit : Nous sommes tes serviteurs et notre roi avec nous ; parce que nous ne sommes pas dignes d'être jugés par un prophète, c'est pourquoi nous avons dit : Établissez sur nous un roi pour nous juger. Et tout le peuple et le roi pleurèrent avec une grande lamentation, et dirent : Que Samuel, le prophète, vive. Et lorsque le roi fut nommé, ils offrirent des sacrifices au Seigneur.

\par 5 Après cela, Saül combattit contre les Philistins pendant un an, et la bataille prospéra grandement.

\chapter{58}

\par 1 Et à ce moment-là, l'Éternel dit à Samuel : Va et dis à Saül : Tu es envoyé pour détruire Amalech, afin que s'accomplissent les paroles que Moïse, mon serviteur, a dites en disant : Je détruirai du pays le nom d'Amalech. dont j'ai parlé dans ma colère. Et n’oublie pas de détruire chacune d’entre elles comme cela te l’est ordonné.

\par 2 Et Saül partit et combattit Amalech, et il sauva vivant Agag, roi d'Amalech, parce qu'il lui avait dit : Je te montrerai les trésors cachés. C'est pourquoi il l'épargna, le sauva vivant et l'amena à Armathem.

\par 3 Et Dieu dit à Samuel : As-tu vu comment le roi s'est corrompu même en un instant avec l'argent, et as-tu sauvé la vie Agag, roi d'Amalech, et sa femme ? Maintenant donc, permets qu'Agag et sa femme se réunissent cette nuit, et demain tu le tueras ; mais ils garderont sa femme jusqu'à ce qu'elle enfante un enfant mâle, et alors elle aussi mourra, et celui qui naîtra d'elle sera une offense à Saül. Mais toi, lève-toi demain et tue Agag ; car le péché de Saül est toujours écrit devant ma face.

\par 4 Et quand Samuel se leva le lendemain, Saül sortit à sa rencontre et lui dit : L'Éternel a livré nos ennemis entre nos mains, comme il l'a dit. Et Samuel dit à Saül : À qui Israël a-t-il fait du tort ? car avant que le temps soit venu pour qu'un roi doive régner sur lui, il t'a demandé pour son roi, et toi, quand tu as été envoyé pour faire la volonté du Seigneur, tu l'as transgressée. C'est pourquoi celui que tu as sauvé vivant mourra maintenant, et il ne te montrera pas les trésors cachés dont il a parlé, et celui qui est né de lui te sera une offense. Et Samuel vint vers Agag avec une épée, le tua et retourna dans sa maison.

\chapter{59}

\par 1 Et l'Éternel lui dit : Va, oins celui à qui je te dirai, car le temps est accompli où viendra son royaume. Et. Samuel dit : Voici, vas-tu maintenant effacer le royaume de Saül ? Et il a dit : je vais l’effacer.

\par 2 Et Samuel sortit à Béthel, et sanctifia les anciens, Jessé et ses fils. Et Eliab, le premier-né de Jessé, vint. Et Samuel dit : Voici maintenant le Saint, l'oint du Seigneur. Et le Seigneur lui dit : Où est ta vision que ton cœur a vue ? N'est-ce pas toi qui as dit à Saül : Je suis celui qui voit ? Et comment ne sais-tu pas qui tu dois oindre ? Et maintenant, que cette réprimande te suffise, cherche le berger, le plus petit de tous, et oins-le.

\par 3 Et Samuel dit à Jessé : Écoute, Jessé, envoie et amène ici ton fils du troupeau, car Dieu l'a choisi. Et Jessé envoya chercher David, et Samuel l'oignit au milieu de ses frères. Et le Seigneur était avec lui à partir de ce jour.

\par 4 Alors David se mit à chanter ce psaume, et dit : Aux extrémités de la terre je commencerai à le glorifier, et je chanterai des louanges jusqu'aux jours éternels. Au début, Abel, lorsqu'il nourrissait les brebis, son sacrifice était plus acceptable que celui de son frère. Et son frère l'enviait et le tuait. Mais il n'en est pas ainsi pour moi, car Dieu m'a gardé et m'a livré à ses anges et à ses veilleurs pour qu'ils me gardent, car mes frères m'enviaient, et mon père et ma mère ne m'ont pas compté, et quand le prophète Quand ils sont venus, ils ne m'ont pas appelé, et quand l'oint du Seigneur a été proclamé, ils m'ont oublié. Mais Dieu s'est approché de moi avec sa main droite et avec sa miséricorde : c'est pourquoi je ne cesserai de chanter des louanges tous les jours de ma vie.

\par 5 Et comme David parlait encore, voici, un lion féroce sorti de la forêt et une ourse de la montagne prirent les taureaux de David. Et David dit : Voici, ceci sera pour moi un signe pour un puissant début de ma victoire dans la bataille. Je sortirai après eux, je délivrerai ceux qui ont été emportés et je les tuerai. Et David sortit après eux, prit des pierres du bois et les tua. Et Dieu lui dit : Voici, c'est par des pierres que je t'ai délivré ces bêtes sous tes yeux. Et ceci sera pour toi un signe que désormais tu tueras à coups de pierres l'adversaire de mon peuple.

\chapter{60}

\par 1 Et à ce moment-là, l'esprit du Seigneur fut ôté de Saül, et un mauvais esprit l'opprima (litt. l'étouffa). Et Saül envoya chercher David, et il joua pendant la nuit un psaume sur sa harpe. Et c'est le psaume qu'il chanta à Saül pour que le mauvais esprit se retire de lui.

\par 2 Il y avait des ténèbres et du silence avant que le monde soit, et le silence parlait, et les ténèbres devenaient visibles. Et alors ton nom fut créé, même lors du rapprochement de ce qui était étendu, dont le haut était appelé ciel et le bas était appelé terre. Et il fut ordonné à la partie supérieure qu'il pleuve selon sa saison, et à la partie inférieure qu'elle produise la nourriture pour l'homme qui devait être préparée. Et après cela fut créée la tribu de vos esprits.

\par 3 Maintenant donc, ne sois pas nuisible, alors que tu es une seconde création, mais sinon, souviens-toi de l'Enfer (lit. souviens-toi du Tartare) dans lequel tu as marché. Ou ne te suffit-il pas d'entendre que par ce qui résonne devant toi, je chante à plusieurs ? Ou oublies-tu que d'un écho rebondissant dans l'abîme (ou le chaos) ta création est née ? Mais ce nouveau ventre, dont je suis né, et de qui naîtra après un temps de mes reins celui qui vous soumettra.

\par Et quand David chantait des louanges, l'esprit épargnait Saül.

\chapter{61}

\par 1 Et après ces choses, les Philistins vinrent combattre Israël. Et David retourna dans le désert pour paître ses brebis, et les Madianites arrivèrent et voulurent prendre ses brebis, et il descendit vers eux et les combattit et tua d'eux 15 000 hommes. C'est la première bataille que David livra, dans le désert.

\par 2 Un homme du nom de Golia sortit du camp des Philistins. Il regarda Saül et Israël et dit : N'est-ce pas toi, Saül, qui as fui devant moi lorsque je t'ai pris l'arche et que j'ai tué tes prêtres ? Et maintenant que tu règnes, descendras-tu vers moi comme un homme et un roi et vas-tu nous combattre ? Sinon, je viendrai vers toi, et je te ferai captiver, et ton peuple servira nos dieux. Et quand Saül et Israël entendirent cela, ils furent très effrayés. Et le Philistin dit : Selon le nombre de jours pendant lesquels Israël a fait la fête lorsqu'ils ont reçu la loi dans le désert, même 40 jours, je leur ferai des reproches, et après cela je combattrai avec eux.

\par 3 Et il arriva que lorsque les 40 jours furent accomplis, et que David fut venu pour voir la bataille de ses frères, qu'il entendit les paroles que disait le Philistin, et dit : Est-ce peut-être le moment dont Dieu m'a dit ? : Je livrerai l'adversaire de mon peuple entre tes mains par des pierres ?

\par 4 Et Saül entendit ces paroles et l'envoya le prendre et dit : Quel a été le discours que tu as dit au peuple ? Et David dit : Ne crains rien, ô roi, car j'irai combattre les Philistins, et Dieu enlèvera la haine et l'opprobre d'Israël.

\par 5 Et David sortit et prit 7 pierres et écrivit dessus les noms de ses pères, Abraham, Isaac et Jacob, Moïse et Aaron, et son propre nom et le nom du Très-Puissant. Et Dieu envoya Cervihel, l'ange qui est au-dessus des forces.

\par 6 Et David sortit vers Golia et lui dit : Écoute une parole avant de mourir. Les deux femmes dont toi et moi sommes nés n'étaient-elles pas sœurs ? et ta mère était Orpha et ma mère était Ruth. Et Orpha choisit les dieux des Philistins et partit à leur poursuite, tandis que Ruth choisit les voies du Tout-Puissant et les suivit. Et maintenant, toi et tes frères êtes nés d'Orpha, et comme tu es ressuscité aujourd'hui et que tu es venu ravager Israël, voici, moi aussi, moi qui suis né de ta parenté, je suis venu pour venger mon peuple. Car tes trois frères tomberont aussi entre mes mains après ta mort. Et alors vous direz à votre mère : Celui qui est né de votre sœur ne nous a pas épargné.

\par 7 Et David mit une pierre dans sa fronde et frappa le Philistin au front, et courut sur lui, sortit son épée du fourreau et lui ôta la tête. Et Golia lui dit alors que sa vie était encore en lui : Hâte-toi, tue-moi et réjouis-toi.

\par 8 Et David lui dit : Avant de mourir, ouvre les yeux et regarde ton meurtrier qui t'a tué. Et le Philistin regarda et vit l'ange et dit : Ce n'est pas toi qui m'as tué, mais celui qui était avec toi, dont la forme n'est pas celle d'un homme. Et puis David lui a arraché la tête.

\par 9 Et l'ange du Seigneur leva la face de David et personne ne le connut. Et quand Saül vit David, il lui demanda qui il était, et personne ne savait qui il était.

\chapter{62}

\par 1 Et après ces choses, Saül enviait David et cherchait à le tuer. Mais David et Jonathan, le fils de Saül, conclurent une alliance ensemble. Et quand David vit que Saül cherchait à le tuer, il s'enfuit à Armathem ; et Saül sortit après lui.

\par 2 Et l'esprit demeura en Saül, et il prophétisa, disant : Pourquoi es-tu trompé, ô Saül, ou qui persécutes-tu en vain ? Le temps de ton royaume est accompli. Va chez toi, car tu mourras et David régnera. Ne mourras-tu pas ensemble, toi et ton fils ? Et alors le royaume de David apparaîtra. Et l'esprit se retira de Saül, et il ne sut pas ce qu'il avait prophétisé.

\par 3 Mais David vint vers Jonathan et lui dit : Viens et faisons une alliance avant de nous séparer l'un de l'autre. Car Saül, ton père, cherche à me tuer sans motif. Et depuis qu'il a compris que tu m'aimes, il ne te dit pas ce qu'il médite à mon sujet.

\par 4 Mais c'est pour cette raison qu'il me hait, parce que tu m'aimes, et de peur que je ne règne à sa place. Et tandis que je lui ai fait du bien, il me rend du mal. Et tandis que j'ai tué Golia par la parole du Tout-Puissant, vois quelle fin il me propose. Car il a décidé de détruire la maison de mon père. Et que le jugement de vérité soit mis en balance, et que la multitude des prudents entende la sentence.

\par 5 Et maintenant je crains qu'il ne me tue et ne perde sa propre vie à cause de moi. Car il ne versera jamais de sang innocent sans punition. Pourquoi mon âme devrait-elle être persécutée ? Car j'étais le plus petit d'entre mes frères, je paissiais les brebis, et pourquoi suis-je en danger de mort ? Car je suis juste et je n'ai aucune iniquité. Et pourquoi ton père me déteste-t-il ? Mais la justice de mon père m'aidera à ne pas tomber entre les mains de ton père. Et comme je suis jeune et tendre, ce n’est pas pour rien que Saül m’envie.

\par 6 Si je lui avais fait du tort, je le prierais de me pardonner le péché. Car si Dieu pardonne l'iniquité, combien plus à ton père qui est chair et sang ? J'ai marché dans sa maison avec un cœur parfait, oui, j'ai grandi devant sa face comme un aigle rapide, j'ai mis mes mains sur la harpe et je l'ai béni par des chants, et il a imaginé de me tuer, et comme un moineau qui J'ai fui devant la face du faucon, ainsi j'ai fui devant sa face.

\par 7 À qui ai-je dit cela, ou à qui ai-je raconté les choses que j'ai souffertes, sinon à toi et à Melchol, ta sœur ? Car nous deux, marchons ensemble dans la vérité.

\par 8 Et il vaudrait mieux, mon frère, que je sois tué au combat plutôt que de tomber entre les mains de ton père ; car dans le combat mes yeux regardaient de tous côtés, afin que je puisse le défendre de ses ennemis. Ô mon frère Jonathan, écoute mes paroles, et s'il y a de l'iniquité en moi, reprends-moi.

\par 9 Et Jonathan répondit et dit : Viens à moi, mon frère David, et je te dirai ta justice. Mon âme se languit de ta tristesse parce que maintenant nous sommes séparés l'un de l'autre. Et nos péchés sont contraints à cela, afin que nous soyons séparés les uns des autres. Mais souvenons-nous les uns des autres jour et nuit pendant que nous vivons. Et même si la mort nous sépare, je sais que nos âmes se connaîtront. Car à toi appartient le royaume dans ce monde, et c'est de toi que viendra le commencement du royaume, et il viendra en son temps.

\par 10 Et maintenant, comme un enfant qui est sevré de sa mère, ainsi sera notre séparation. Que le ciel soit témoin et que la terre soit témoin de ce dont nous avons parlé ensemble. Et pleurons l'un avec l'autre et Jay versons nos larmes dans un seul vase et confions le vase à la terre, et cela nous servira de témoignage. ii. Et ils se lamentèrent douloureusement et s'embrassèrent. Mais Jonathan eut peur et dit à David : Souvenons-nous, ô mon frère, de l'alliance qui a été conclue entre nous et du serment qui est inscrit dans notre cœur. Et si je meurs avant toi et que tu règnes effectivement, comme l'Éternel l'a dit, ne te souviens pas de la colère de mon père, mais de l'alliance qui a été conclue entre moi et toi. Ne pense pas non plus à la haine avec laquelle mon père te hait en vain, mais à mon amour avec lequel je t'ai aimé. Ne pense pas non plus à ce pour quoi mon père ne t'a pas été reconnaissant, mais souviens-toi de la table où nous avons mangé ensemble. Ne te souviens pas non plus de l'envie avec laquelle mon père t'enviait méchamment, mais de la foi que moi et toi gardons. Ne te soucie pas non plus du mensonge par lequel Saül a menti, mais des serments que nous nous sommes prêtés l'un à l'autre. Et ils s'embrassèrent. Après cela, David partit dans le désert, et Jonathan entra dans la ville.

\chapter{63}

\par 1 En ce temps-là, les prêtres qui habitaient à Noba profanaient les choses saintes de l'Éternel et rendaient les prémices en opprobre au peuple. Et Dieu se mit en colère et dit : Voici, je vais exterminer les prêtres qui habitent à Noba, parce qu'ils marchent dans la voie des fils d'Héli.

\par 2 Et à ce moment-là, Doech le Syrien, qui était conducteur des mulets de Saül, vint vers Saül et lui dit : Ne sais-tu pas qu'Abimelec, le sacrificateur, a pris conseil avec David, lui a donné une épée et l'a renvoyé en paix ? Et Saül envoya appeler Abimélec et lui dit : Tu mourras sûrement, parce que tu as pris conseil avec mon ennemi. Et Saül tua Abimelec et toute la maison de son père, et il n'y eut pas un seul membre de sa tribu délivré, sauf Abiathar, son fils. Celui-ci vint trouver David et lui raconta tout ce qui lui était arrivé.

\par 3 Et Dieu dit : Voici, l'année où Saül commença à régner, quand Jonathan avait péché et qu'il voulait le faire mourir, ce peuple se leva et ne le souffrit pas, et maintenant, quand les prêtres furent tués, même 385 les hommes, ils gardaient le silence et ne disaient rien. C'est pourquoi voici, les jours viendront bientôt où je les livrerai entre les mains de leurs ennemis et ils tomberont blessés, eux et leur roi.

\par 4 Et à Doech le Syrien ainsi dit l'Éternel : Voici, les jours viendront bientôt où le ver montera sur sa langue et le fera dépérir, et sa demeure sera avec Jair pour toujours dans le feu qui n'est pas éteint.

\par 5 Or, tout ce que Saül a fait, et le reste de ses paroles, et comment il a poursuivi David, n'est-ce pas écrit dans le livre des rois d'Israël ?

\par 6 Et après ces choses, Samuel mourut, et tout Israël se rassembla et le pleura, et l'enterra.

\chapter{64}

\par 1 Alors Saül réfléchit, disant : J'enlèverai sûrement les sorciers du pays d'Israël. Ainsi les hommes se souviendront de moi après mon départ. Et Saül dispersa tous les sorciers du pays. Et Dieu dit : Voici, Saül a fait sortir les sorciers du pays, non par crainte de moi, mais pour se faire un nom. Voici, celui qu'il a dispersé, qu'il ait recours à eux et obtienne d'eux la divination, car il n'a pas de prophètes.

\par 2 En ce temps-là, les Philistins disaient chacun à son prochain : Voici, Samuel le prophète est mort et il n'y a personne qui prie pour Israël. David aussi, qui combattait pour eux, est devenu l'adversaire de Saül et n'est plus avec eux. Maintenant donc, levons-nous, combattons-les avec force et vengeons le sang de nos pères. Et les Philistins se rassemblèrent et arrivèrent au combat.

\par 3 Et quand Saül vit que Samuel était mort et que David n'était pas avec lui, ses mains se relâchèrent. Et il consulta le Seigneur, et il ne l'écouta pas. Et il cherchait des prophètes, et aucun ne lui apparut. Et Saül dit au peuple : Cherchons un devin et demandons-lui ce que j'ai en tête. Et le peuple lui répondit : Voici, il y a maintenant une femme nommée Sédecla, fille de Debin (ou Adod) le Madianite, qui a trompé le peuple d'Israël par des sorcelleries ; et voici, elle habite à Endor.

\par 4 Et Saül revêtit des vêtements infâmes et alla vers elle, lui et deux hommes avec lui, pendant la nuit et lui dit : Lève-moi vers moi Samuel. Et elle dit : J'ai peur du roi Saül. Et Saül lui dit : Saül ne te fera aucun mal dans cette affaire. Et Saül dit en lui-même : Quand j'étais roi en Israël, même si les païens ne me voyaient pas, ils savaient pourtant que j'étais Saül. Et Saül interrogea la femme, disant : As-tu déjà vu Saül ? Et elle a dit : Souvent. Et Saül sortit et pleura et dit : Voici, maintenant je sais que ma beauté a changé et que la gloire de mon royaume est passée loin de moi.

\par 5 Et il arriva que, lorsque la femme vit Samuel monter, et vit Saül avec lui, elle s'écria et dit : Voici, tu es Saül, pourquoi m'as-tu trompé ? Et il lui dit : Ne crains rien, mais dis-moi ce que tu as vu. Et elle dit : Voici, cela fait 40 ans que j'ai ressuscité les morts pour les Philistins, mais cette apparition n'a pas été vue, et elle ne se verra plus non plus par la suite.

\par 6 Et Saül lui dit : Quelle est sa forme ? Et elle dit : Tu me demandes des renseignements sur les dieux. Car voici, sa forme n’est pas la forme d’un homme. Car il est vêtu d'une robe blanche et porte un manteau dessus, et deux anges le conduisent. Et Saül se souvint du manteau que Samuel avait déchiré de son vivant, et il joignit les mains et se jeta à terre.

\par 7 Et Samuel lui dit : Pourquoi m'as-tu inquiété pour me faire monter ? Je pensais que le moment était venu pour moi de recevoir la récompense de mes actes. Ne te vante donc pas, ô roi, ni toi, ô femme. Car ce n'est pas vous qui m'avez élevé, mais le précepte que Dieu m'a dit alors que j'étais encore en vie, que je vienne te dire que tu avais péché une seconde fois en négligeant Dieu. C'est pourquoi mes os sont perturbés après que j'ai rendu mon âme, afin que je te parle, et qu'étant mort, je sois entendu comme un vivant.

\par 8 Maintenant donc, demain, toi et tes fils serez avec moi, lorsque le peuple sera livré entre les mains des Philistins. Et parce que tes entrailles ont été émues par la jalousie, ce qui est à toi te sera ôté. Et Saül entendit les paroles de Samuel, et son âme fondit et il dit : Voici, je pars mourir avec mes fils, si par hasard ma destruction peut être une expiation pour mes iniquités. Et Saül se leva et partit de là.

\chapter{65}

\par 1 Et les Philistins combattirent contre Israël. Et Saül partit au combat. Et Israël s'enfuit devant les Philistins. Et quand Saül vit que la bataille devenait extrêmement dure, il dit dans son cœur : Pourquoi te fortifies-tu pour vivre, puisque Samuel a annoncé la mort pour toi et pour tes fils ?

\par 2 Et Saül dit à celui qui portait son armure : Prends ton épée et tue-moi avant que les Philistins ne viennent m'insulter. Et celui qui dénudait son armure ne voulait pas mettre la main sur lui.

\par 3 Et lui-même s'inclina sur son épée, et il ne put mourir. Et il regarda derrière lui et vit un homme courir et l'appela et dit : Prends mon épée et tue-moi. Car ma vie est encore en moi.

\par 4 Et il vint pour le tuer. Et Saül lui dit : Avant de me tuer, dis-moi, qui es-tu ? Et il lui dit : Je suis Edab, fils d'Agag, roi des Amalécites. Et Saül dit : Voici, maintenant les paroles de Samuel sont venues sur moi, comme il a dit : Celui qui naîtra d'Agag te sera une offense.

\par 5 Mais va et dis à David : J'ai tué ton ennemi. Et tu lui diras : Ainsi parle Saül : Ne te souviens pas de ma haine, ni de mon injustice. . . .



\end{document}