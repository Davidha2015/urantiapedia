\begin{document}

\title{Apocalypse de Sédrach}

\chapter{1}

\par \textit{Un sermon du saint et bienheureux Sédrach sur l'amour, la repentance, les chrétiens orthodoxes et la seconde venue de notre Seigneur Jésus-Christ. Maître, accorde (ta) bénédiction.}

\par 1 Bien-aimés, nous ne devons préférer rien de plus que l'amour non feint.

\par 2 Nous commettons de nombreuses fautes à chaque heure, jour et nuit, et pour cette raison acquérons l'amour, car il couvre une multitude de péchés.

\par 3 Que gagnons-nous, mes enfants, si nous possédons tout mais n'avons pas l'amour salvateur ?

\par 4 Quel profit, mes enfants, si l'on donne un grand banquet, si l'on invite roi et noble et si l'on prépare toutes sortes de mets coûteux pour que rien ne manque ; néanmoins, s’il n’y a pas de sel, ce banquet ne peut être mangé ; et non seulement on en supporte la dépense, mais on gaspille aussi ses efforts et on est déshonoré par les invités.

\par 5 Il en est de même dans notre situation, mes frères ; à quoi profiterons-nous, car quelle grâce possédons-nous sans amour ?

\par 6 Chacune de nos actions est fausse, même si quelqu'un est vierge, jeûne, veille, prie et donne un banquet pour les pauvres.

\par 7 Et si quelqu'un apporte des dons à Dieu, ou offre les prémices de tous ses biens, ou bâtit des églises ou fait autre chose sans amour, cela sera considéré par Dieu comme rien, car (ces choses) ne sont pas acceptables.

\par 8 Ainsi le prophète dit : « Le sacrifice des impies est une abomination à l'Éternel. »

\par 9 On ne vous conseille pas de faire quoi que ce soit sans amour.

\par 10 Si vous dites : « Je hais mon frère mais j'aime le Christ », vous êtes un menteur, et Jean le Théologien vous réprimande, car comment celui qui n'aime pas son frère qu'il a vu peut-il aimer Dieu qu'il a vu. pas vu?

\par 11 Il est clair que quiconque hait son frère mais pense qu'il aime le Christ est un menteur et se fait des illusions.

\par 12 Car Jean le Théologien dit que nous avons ce commandement de Dieu, que celui qui aime Dieu aime aussi son frère.

\par 13 Et encore le Seigneur lui-même dit : « De ces deux (commandements) dépendent toute la loi et les prophètes. »

\par 14 Oh, combien extraordinaire et paradoxal est le miracle que celui qui a l'amour accomplit toute la loi ; l'amour est l'accomplissement de la loi.

\par 15 Oh, puissance de l'amour au-delà de l'imagination ; oh, puissance de l'amour au-delà de toute mesure !

\par 16 Il n'y a rien de plus honorable que l'amour, et il n'y a rien de plus grand ni au ciel ni sur la terre.

\par 17 Cet amour divin est la capitale (la vertu) ; parmi toutes les vertus, l’amour est la plus haute perfection du monde.

\par 18 Elle habitait au cœur d'Abel ; il a travaillé en collaboration avec les Patriarches ; il gardait Moïse ; cela a fait de David la demeure du Saint-Esprit ; cela a fortifié Joseph.

\par 19 Mais pourquoi est-ce que je dis ces choses ?

\par 20 Le plus important est que cet amour a fait descendre du ciel le Fils de Dieu.

\par 21 Par l'amour toutes les bonnes choses ont été révélées ; la mort a été foulée aux pieds, Hadès a été rendu captif, Adam a été rappelé (de la mort), et par l'amour, un seul troupeau a été constitué par la suite, composé d'anges et d'hommes.

\par 22 Par l'amour le Paradis a été ouvert ; le royaume des cieux est promis ; il transforma les déserts en villes, et remplit de chants les montagnes et les grottes ; il a enseigné aux hommes et aux femmes qui suivaient le chemin étroit et douloureux.

\par 23 Mais combien de temps prolongerons-nous ce sermon sur les réalisations de l'amour que même les anges ne peuvent accomplir ?

\par 24 Oh, amour béni qui accorde toutes les bonnes choses !

\par 25 Bienheureux est l'homme qui possède la vraie foi et l'amour sincère ; car, comme l'a dit le Maître, rien n'est plus grand que l'amour pour lequel un homme donne sa vie pour ses amis.

\chapter{2}

\par 1 Et il a entendu un caché ? voix à ses oreilles : « Ici, Sédrach, toi qui souhaites et désires parler avec Dieu et lui demander de te révéler les choses que tu souhaites demander. »

\par 2 Et Sédrach dit : « Qu'est-ce que c'est, mon Seigneur ?

\par 3 Et la voix lui dit : « Je t'ai été envoyé pour te transporter au ciel. »

\par 4 Et il dit : « Je veux parler à Dieu face à face, mais je ne peux pas, Seigneur, monter aux cieux. »

\par 5 Mais l'ange, ayant déployé ses ailes, le prit et monta dans les cieux, et l'emmena jusqu'au troisième ciel, et la flamme de la divinité se tenait là.

\chapter{3}

\par 1 Et le Seigneur lui dit : « Bienvenue, mon cher Sédrach.

\par 2 Quel genre de plainte avez-vous contre le Dieu qui vous a créé, car vous avez dit : « Je veux parler avec Dieu face à face » ?

\par 3 Sédrach lui dit : « En effet, le fils a une plainte contre le Père : Mon Seigneur, pourquoi as-tu créé la terre ?

\par 4 Le Seigneur lui dit : « Pour l'homme ».

\par 5 Sédrach dit : « Pourquoi as-tu créé la mer et pourquoi as-tu répandu toutes sortes de bonnes choses sur la terre ? »

\par 6 Le Seigneur a dit : « Pour l’homme ».

\par 7 Sédrach lui dit : Si tu as fait ces choses, pourquoi as-tu détruit l'homme ?

\par 8 Et le Seigneur dit : « L'homme est mon ouvrage et la créature de mes mains, et je le corrige comme je le trouve juste. »

\chapter{4}

\par 1 Sédrach lui dit : « Ta discipline ? est-ce une punition ? et le feu; et ils sont très amers, mon Seigneur.

\par 2 Il vaudrait mieux pour l'homme qu'il ne naisse pas.

\par 3 En effet, qu'avez-vous fait, mon Seigneur ; Pourquoi avez-vous travaillé de vos mains immaculées et créé l'homme, puisque vous ne vouliez pas avoir pitié de lui ?

\par 4 Dieu lui dit : « J'ai créé le premier homme, Adam, et je l'ai placé au paradis au milieu de (qui est) l'arbre de vie, et je lui ai dit : 'Mange de tous les fruits, mais prends garde.' de l'arbre de vie, car si vous en mangez, vous mourrez sûrement.

\par 5 Cependant, il a désobéi à mon commandement et, trompé par le diable, il a mangé de l'arbre.

\chapter{5}

\par 1 Sédrach lui dit : « C'est par ta volonté qu'Adam a été trompé, mon Maître.

\par 2 Vous avez ordonné à vos anges d'adorer Adam, mais celui qui était le premier parmi les anges a désobéi à votre ordre et ne l'a pas adoré ; et ainsi tu l'as banni, parce qu'il a transgressé ton commandement et n'est pas sorti (pour adorer) la création de tes mains.

\par 3 Si vous avez aimé l'homme, pourquoi n'avez-vous pas tué le diable, l'artisan de toute iniquité ?

\par 4 Qui peut lutter contre un esprit invisible ?

\par 5 Il entre dans le cœur des hommes comme une fumée et leur enseigne toutes sortes de péchés.

\par 6 Il combat même contre toi, le Dieu immortel, et alors que peut faire un homme pitoyable contre lui ?

\par 7 Mais aie pitié, Maître, et détruis le châtiment ; sinon, reçois-moi aussi avec les pécheurs, car si tu ne veux pas être miséricordieux envers les pécheurs, où sont tes miséricordes et où est ta compassion, ô Seigneur ?


\chapter{6}

\par 1 Et Dieu lui dit : « Sachez que tout ce que j'ai commandé à l'homme de faire était à sa portée.

\par 2 Je l'ai fait sage et héritier du ciel et de la terre, et je lui ai tout soumis et tout être vivant fuit devant lui et devant sa face.

\par 3 Mais après avoir reçu mes dons, il est devenu un étranger, un adultère et un pécheur.

\par 4 Dites-moi, quel genre de père donnerait un héritage à son fils, et après avoir reçu l'argent (le fils) s'en va, laissant son père, et devient un étranger et au service des étrangers.

\par 5 Le père alors, voyant que le fils l'a abandonné (et s'en est allé), obscurcit son cœur et s'en allant, il récupère sa richesse et bannit son fils de sa gloire parce qu'il a abandonné son père.

\par 6 Comment se fait-il que moi, le Dieu merveilleux et jaloux, je lui ai tout donné, mais que lui, les ayant reçus, soit devenu adultère et pécheur ?

\chapter{7}

\par 1 Sédrach lui dit : « Toi, Maître, tu as créé l'homme ; vous connaissez l'état bas de sa volonté et de ses connaissances ? et vous envoyez l'homme au châtiment sous un faux prétexte ; alors retirez-le.

\par 2 Suis-je seul censé remplir les royaumes célestes ?

\par 3 S'il n'en est pas ainsi, Seigneur, sauve aussi l'homme.

\par 4 Un homme pitoyable a transgressé par ta volonté, ô Seigneur.

\par 5 « Pourquoi me jettes-tu des mots comme s'ils étaient un filet, Sédrach ?

\par 6 J'ai créé Adam et sa femme et le soleil et j'ai dit : 'Regardez-vous (pour voir) qui est illuminé.'

\par 7 Et le soleil et Adam étaient d'un même caractère, mais la femme d'Adam était plus brillante que la lune en beauté, et elle lui a donné la vie.

\par 8 Sédrach dit : « À quoi servent les belles choses si elles se dessèchent en poussière ?

\par 9 Comment se fait-il que tu aies dit : Seigneur : « Ne rends pas le mal pour le mal » ?

\par 10 Comment se fait-il, Maître, car la parole de votre divinité ne ment jamais ?

\par 11 Et pourquoi avez-vous ainsi rendu à l'homme, si vous ne voulez pas (rendre) le mal pour le mal ?

\par 12 Je sais que parmi les quadrupèdes le mulet est un animal rusé, il n'en est aucun autre ; pourtant, avec la bride, on la tourne où l'on veut.

\par 13 Vous avez des anges ; envoie-les surveiller (sur l'homme) et quand il fait un geste vers le péché, retiens son pied, et il n’ira pas où il veut.

\chapitre{8}

\par 1 Dieu lui dit : « Si je tiens son pied, il dit : 'Tu ne m'as donné aucune grâce au monde', alors je l'ai laissé à ses propres désirs parce que je l'aimais et c'est pourquoi j'ai envoyé mes anges justes. pour le surveiller nuit et jour.

\par 2 Sédrach dit : « Je sais que parmi tes propres créatures, Maître, tu as aimé l'homme en premier ; parmi les quadrupèdes, les moutons ; parmi les arbres, l'olivier ; parmi les plantes qui portent du fruit, la vigne ; parmi les choses qui volent, l'abeille ; parmi les fleuves, le Jourdain ; parmi les villes, Jérusalem.

\par 3 Mais l’homme aussi aime tout cela, Maître.

\par 4 Dieu dit à Sédrach : « Je vais te demander une chose, Sédrach ; si vous pouvez me répondre, c’est à juste titre que vous m’avez défié, même si vous avez tenté votre créateur.

\par 5 Sédrach dit : « Parle. »

\par 6 Le Seigneur Dieu a dit : « Depuis que j'ai tout créé, combien de personnes sont nées, et combien sont mortes et combien mourront et combien de cheveux ont-elles ?

\par 7 Dis-moi, Sédrach, depuis que le ciel et la terre ont été créés, combien d'arbres ont été créés dans le monde, et combien tomberont et combien seront créés, et combien de feuilles ont-ils ?

\par 8 Dis-moi, Sédrach, depuis que j'ai fait la mer, combien de vagues ont gonflé, et combien ont légèrement roulé, et combien s'élèveront, et combien de vents soufflent près du rivage de la mer ?

\par 9 Dis-moi, Sédrach, depuis la création du monde des âges où l'air est plein de pluie, combien de gouttes sont tombées sur le monde et combien tomberont ?

\par 10 Et Sédrach dit : Toi seul sais toutes ces choses, Seigneur ; vous seul connaissez tout cela ; Je vous supplie seulement de libérer l'homme du châtiment, car sinon je vais moi-même subir le châtiment et je ne suis pas séparé de notre race.

\chapitre{9}

\par 1 Et Dieu dit à son Fils unique : « Va, prends l'âme de mon bien-aimé Sédrach et mets-la au Paradis. »

\par 2 Le Fils unique dit à Sédrach : « Donne-moi ce que notre Père a déposé dans le ventre de ta mère, dans ta sainte demeure, depuis ta naissance. »

\par 3 Sédrach dit : « Je ne te donnerai pas mon âme. »

\par 4 Dieu lui dit : « Et pourquoi ai-je été envoyé, et pourquoi suis-je venu ici, et tu fais semblant ? tome?

\par 5 Mon père m'a ordonné de ne pas hésiter à prendre votre âme ; par conséquent, donnez-moi votre âme la plus désirée.

\chapitre{10}

\par 1 Et Sédrach dit à Dieu : « D'où prendras-tu mon âme, de quel membre ?

\par 2 Et Dieu lui dit : « Ne sais-tu pas qu'il est placé au milieu de tes poumons et de ton cœur et qu'il s'étend à tous les membres ?

\par 3 Il est retiré par le pharynx et le larynx et par la bouche ; et chaque fois qu'il doit sortir (du corps), il est difficilement tiré au début et, à mesure qu'il se rassemble des ongles et de tous les membres, il y a nécessairement une grande tension ? en étant séparé du corps et détaché du cœur.

\par 4 Après avoir entendu toutes ces choses et rappelé le souvenir de la mort, Sédrach fut très troublé et dit à Dieu : « Seigneur, donne-moi un peu de temps pour que je pleure, car j'ai entendu dire que les larmes accomplissent beaucoup et peut devenir un remède suffisant pour le corps humble de vos créatures.

\chapitre{11}


\par 1 Et pleurant et se lamentant, il commença à dire : « Ô tête merveilleuse, ornée comme le ciel ; Ô soleil sur le ciel et la terre ; vos cheveux sont connus de Theman, vos yeux de Bosra, vos oreilles du tonnerre, votre langue du clairon, et votre cerveau est une petite création ; la tête, le mouvement de tout le corps, est fiable et très belle, aimée de tous mais dès qu'elle tombe en terre, elle est méconnue.

\par 2 Ô mains qui tiennent si bien, qui s'enseignent facilement et qui travaillent dur, par lesquelles le corps se nourrit.

\par 3 Ô mains si habiles, rassemblant les matériaux, vous avez orné ensemble les maisons.

\par 4 Ô doigts embellis et ornés d'or et d'argent ; même les grandes structures sont réalisées avec les doigts ; les trois articulations étendent les paumes et rassemblent les bonnes choses ensemble ; mais maintenant vous êtes devenus étrangers à ce monde.

\par 5 O pieds qui marchent si bien, se déplaçant tout seuls si vite et infatigables.

\par 6 Ô genoux ainsi joints, sans vous le corps ne bouge pas ; les pieds courent ensemble avec le soleil et la lune, nuit et jour, rassemblant toutes choses, nourriture et boisson qui nourrissent le corps.

\par 7 Ô pieds si rapides et si agiles, remuant la surface de la terre et ornant les maisons de toutes bonnes choses.

\par 8 O pieds qui portent tout le corps, qui marchent droit vers les temples ? faites pénitence et suppliez les saints, et maintenant tout à coup vous devez rester impassibles.

\par 9 Ô tête, mains et pieds, jusqu'à présent je vous ai retenus.

\par 10 Ô âme, qu'est-ce qui t'a placé dans le corps humble et misérable ?

\par 11 Et maintenant, séparé d'elle, vous montez là où le Seigneur vous appelle et le corps misérable s'en va pour le jugement.

\par 12 Ô beau corps, cheveux perdus par les étoiles, tête ornée comme le ciel.

\par 13 Ô visage odorant, des yeux comme des fenêtres, une voix comme le son d'un clairon, une langue qui parle si facilement, une barbe bien taillée, des cheveux comme les étoiles, la tête haute comme le ciel, un corps paré, l'enlumineur élégant et célèbre, mais maintenant, après être tombé dans la terre, ta beauté sous la terre est invisible.

\chapitre{12}

\par 1 Le Christ lui dit : « Arrête, Sédrach, jusques à quand vas-tu verser des larmes et gémir ?   

\par 2 Le paradis vous a été ouvert, et après votre mort vous vivrez.

\par 3 Sédrach lui dit : « Une fois de plus, je te parlerai, Seigneur, pendant que je vis, avant de mourir ; et n’ignorez pas ma supplication.

\par 4 Le Seigneur lui dit : « Parle, Sédrach. »

\par 5 (Et Sédrach dit :) « Si l'homme vit quatre-vingts, quatre-vingt-dix ou cent ans et les vit dans le péché mais qu'à la fin il se convertit et que l'homme vit dans la repentance, pendant combien de jours de repentance lui pardonnez-vous ? ) ses péchés ?

\par 6 Dieu lui dit : « S'il revient après avoir vécu cent ou quatre-vingts ans et qu'il se repent pendant trois ans et porte le fruit de la justice et que la mort l'atteint, alors je ne me souviendrai pas de tous ses péchés. »

\chapitre{13}

\par 1 Sédrach lui dit : « Trois ans, c'est trop, mon Seigneur.

\par 2 Sa mort arrivera peut-être et il n'accomplira pas son repentir.

\par 3 Aie pitié, Seigneur, de ton image et sois compatissant, car trois ans, c'est trop.

\par 4 Dieu lui dit : « Si, après cent ans, un homme vit et se souvient de sa mort et se confesse devant les hommes, et que je le retrouve, après un an ? Je pardonnerai tous ses péchés.

\par 5 Sédrach dit encore : « Seigneur, je implore encore ta miséricorde envers ta créature ; une année, c'est beaucoup, et sa mort viendra peut-être et l'enlèvera tout à coup.

\par 6 Le Sauveur lui dit : « Sédrach, mon bien-aimé, je vais te poser une question, puis tu pourras reprendre tes recherches ; si le pécheur se repent pendant quarante jours, ne me souviendrai-je pas en effet de tous les péchés qu'il a commis ?

\chapitre{14}

\par 1 Et Sédrach dit à l'archange Michel : « Écoute-moi, fort protecteur ; aide-moi et intercède pour que Dieu soit miséricordieux envers le monde.

\par 2 Et tombant sur leurs faces, ils supplièrent Dieu et dirent : « Seigneur, enseigne-nous de quelle manière et par quel repentir l'homme peut être sauvé, ou par quel travail. »

\par 3 Dieu a dit : « Par les repentances, les supplications et les liturgies, par les larmes drainantes et les gémissements fervents.

\par 4 Ne savez-vous pas que mon prophète David (a été sauvé) à cause des larmes, et que les autres ont été sauvés en un instant ?

\par 5 Tu sais, Sédrach, qu'il y a des nations qui n'ont pas de loi et qui pourtant observent la loi ; ils ne sont pas baptisés, mais mon esprit divin entre en eux et ils se convertissent à mon baptême, et je les reçois avec mes justes dans le sein d'Abraham.

\par 6 Et il y en a qui ont été baptisés de mon baptême et oints de ma divine myrrhe, mais ils sont devenus pleins de désespoir et ils ne changeront pas d'avis.

\par 7 Pourtant je les attends avec beaucoup de pitié et une grande miséricorde, afin qu'ils se repentent.

\par 8 Mais ils font ce que ma divinité déteste, et ils n'ont pas entendu le sage qui demandait et disait : « Nous ne justifions en aucune façon le pécheur.

\par 9 Ne savez-vous pas du tout qu'il est écrit : « Et ceux qui se repentent ne verront pas de châtiment » ?

\par 10 Et ils n'ont entendu ni les apôtres ni ma parole dans les Évangiles et ils causent du chagrin à mes anges, et certainement dans mes réunions et dans mes liturgies ils n'écoutent pas mon ange et ils ne se tiennent pas dans mes saintes églises ; ils se tiennent debout et ne se prosternent pas de peur et de tremblement, mais ils prononcent de longues paroles que ni moi ni mes anges n'acceptons.

\chapitre{15}

\par 1 Sédrach dit à Dieu : « Seigneur, toi seul es sans péché et très miséricordieux, faisant preuve de pitié et de grâce envers les pécheurs, mais ta divinité a dit : 'Je ne suis pas venu appeler les justes mais les pécheurs à la repentance.' »

\par 2 Et l'Éternel dit à Sédrach : « Ne sais-tu pas, Sédrach, qu'après avoir changé d'avis, le voleur fut sauvé en un instant ?

\par 3 Ne savez-vous pas que même mon apôtre et évangéliste a été sauvé en un instant ? [. . . mais les pécheurs ne sont pas sauvés] parce que leur cœur est comme une pierre pourrie ; ce sont ceux qui marchent dans des sentiers impies et qui périssent avec l’Antéchrist.

\par 4 Sédrach dit : « Mon Seigneur, tu as aussi dit : 'Mon esprit divin est entré dans les nations qui, bien qu'elles n'aient pas de loi, font pourtant les choses de la loi.'

\par 5 Cependant, comme le voleur, l'apôtre, l'évangéliste et les autres qui ont trébuché dans ton royaume, mon Seigneur, pardonne de la même manière à ceux qui ces derniers jours ? J'ai péché contre toi, Seigneur, parce que la vie est pleine de labeur et (est) obstinée.

\chapitre{16}

\par 1 Le Seigneur dit à Sédrach : « J'ai créé l'homme en trois étapes ; quand il est jeune, je néglige ses erreurs à cause de sa jeunesse ; encore une fois, quand il est un homme, je veille sur son esprit ; encore une fois, quand il vieillit, je le préserve afin qu'il se repente.

\par 2 Sédrach dit : « Seigneur, tu sais et tu connais tout cela ; mais ayez compassion des pécheurs.

\par 3 Le Seigneur lui dit : « Mon bien-aimé Sédrach, je promets d'avoir compassion même moins de quarante jours, jusqu'à vingt, et quiconque se souviendra de ton nom ne verra pas le lieu du châtiment mais il sera avec les justes. dans un lieu de rafraîchissement et de repos, et le péché de celui qui copie cet admirable sermon ne sera pas compté pour toujours.

\par 4 Et Sédrach dit : « 'Seigneur, quiconque célèbre une liturgie en l'honneur de ton serviteur, délivre-le, Seigneur, de tout mal.' e Et le serviteur de Dieu, Sédrach, dit : « Maintenant, Maître, prends mon âme. »

\par 5 Et Dieu le prit et le mit au Paradis avec tous les saints. eÀ lui soient la gloire et la puissance pour toujours et à jamais, amen.




\fr{document}