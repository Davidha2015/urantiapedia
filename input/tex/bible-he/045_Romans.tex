\begin{document}

\title{האיגרת אל הרומאים}


\chapter{1}

\par 1 פולוס עבד ישוע המשיח מקרא להיות שליח ונבדל לבשורת אלהים׃
\par 2 אשר הבטיח אתה מראש על־ידי נביאיו בכתבי הקדש׃
\par 3 על־דבר בנו אשר מזרע דוד לפי הבשר׃
\par 4 אשר הוכן לבן־האלהים בגבורה לפי רוח הקדשה בתחיתו מבין המתים הוא ישוע המשיח אדנינו׃
\par 5 אשר־על־ידו נתן לנו חסד ושליחות להקים משמעת האמונה בכל־הגוים למען שמו׃
\par 6 ובתוכם הנכם גם־אתם קרואי ישוע המשיח׃
\par 7 כל־ידידי האלהים ומקראים להיות קדושים אשר ברומי חסד ושלום לכם מאת האלהים אבינו ואדנינו ישוע המשיח׃
\par 8 בראשונה מודה אני לאלהי בישוע המשיח על־כלכם אשר אמונתכם מודעת בכל־העולם׃
\par 9 כי עד האלהים אשר אני עבד אתו ברוחי בבשורת בנו כי תמיד אני מזכיר אתכם׃
\par 10 ומתחנן אני בכל־עת בתפילותי אשר אצליח לבוא אליכם רק־הפעם בחפץ האלהים׃
\par 11 כי כלתה נפשי לראותכם ולהאציל אליכם מתת רוח למען חזק לבכם׃
\par 12 להתנחם עמכם אני באמונתכם ואתם באמונתי׃
\par 13 ולא־אכחד מכם אחי כי־פעמים רבות שמתי על־לבי לבוא אליכם להיות לי פרי גם־בכם כמו ביתר הגוים ולא־עלתה בידי עד־הנה׃
\par 14 מחיב אנכי ליונים וללעזים גם לחכמים ולפתאים׃
\par 15 לכן נדבני לבי להשמיע את־הבשורה גם־אתכם בני רומי׃
\par 16 כי אינני בוש מבשורת המשיח אשר גבורת אלהים היא לתשועה לכל־המאמין ליהודי ראשונה וגם־ליוני׃
\par 17 כי־בה נגלתה צדקת אלהים מאמונה אל־אמונה ככתוב וצדיק באמונתו יחיה׃
\par 18 כי נגלה חרון אלהים מן־השמים על כל־רשעת בני אדם ועולתם אשר יעצרו את־האמת בעולה׃
\par 19 יען אשר דעת האלהים גלויה בקרבם כי האלהים גלה להם׃
\par 20 כי מהותו הנעלמה היא כחו תודע במעשיו ותראה בהם גבורתו הנצחית ואלהותו מעת נברא העולם עד־אשר אין להם פתחון פה להתנצל׃
\par 21 כי הכירו את־האלהים ולא־כבדהו כאלהים וגם־לא הודו לו כי אם־הלכו אחרי ההבל במועצותיהם ויחשך לבם הנבער׃
\par 22 ובאמרם חכמים אנחנו היו לכסילים׃
\par 23 וימירו את־כבוד האלהים אשר הוא חי וקים בדמות צלם אדם אשר הוא כלה והולך צלם כל־עוף והולך על־ארבע ורמש האדמה׃
\par 24 על־כן גם־האלהים נתנם לטמאה בתאות לבם לנבל גויותיהם איש ברעהו׃
\par 25 אשר המירו אמתו של האלהים בשקר ויכבדו את־הבריה לעבדה תחת בראה המברך לעולמים אמן׃
\par 26 בעבור זאת נתנם האלהים לתאות בושה כי־נשיהם החליפו את־דרך ארץ בשלא כדרך ארץ׃
\par 27 וכן גם־הזכרים עזבו דרך גבר באשה ויחמו זה בזה בתאותם ויעשו תועבה זכר עם־זכר ויקחו שכר משובתם הראוי להם בעצם גופם׃
\par 28 וכאשר מאסו דעת אלהים נתנם האלהים בידי דעה נמאסה לעשות את אשר־לא יעשה׃
\par 29 וירב בקרבם כל־חמס זנות ורשע בצע ואון וימלאו קנאה ורצח ומריבה ומרמה ותהפכות׃
\par 30 הלכי רכיל ומלשינים שנאי אלהים וגאים וזדים ומתהללים וחשבי און ולא שמעים בקול אבותם׃
\par 31 נבערים מדעת ובגדים אכזרים נטרי שנאה ולא רחמנים׃
\par 32 יודעים המה את־משפט אלהים כי־עשי אלה בני־מות הם ולא לבד שיעשו את־אלה כי גם־רוצים בעשיהם׃

\chapter{2}

\par 1 לכן כל־בן־האדם הדן אין לך התנצלות כי בדבר אשר תדין את־חברך תחיב נפשך באשר אתה הדן תעשה כמעשהו׃
\par 2 וידענו כי־משפט אלהים משפט אמת על־עשי אלה׃
\par 3 ואתה בן־אדם הדן את אשר־פעלו כאלה ואתה עשה כמעשיהם התאמר להמלט ממשפט האלהים׃
\par 4 או תבוז לרוב טובו ולחמלתו ולארך רוחו ולא תדע כי־טובת האלהים מביאה אתך לידי תשובה׃
\par 5 ובקשי לבבך הממאן לשוב תצבר לך עברה ליום עברת האלהים והגלות משפט צדקו׃
\par 6 אשר ישלם לאיש כמעשהו׃
\par 7 חיי עולם לשקדים לעשות הטוב ושחרי כבוד והדר אשר איננו עבר׃
\par 8 ועל־בני המרי ואשר לא־שמעו לאמת כי אם שמעו־לעולה עליהם חרון־אף וחמה׃
\par 9 צרה ומצוקה על־כל־נפש אדם עשה הרע על־היהודי בתחלה וגם־על־היוני׃
\par 10 וכבוד והדר ושלום לכל־עשה הטוב ליהודי בתחלה וגם ליוני׃
\par 11 כי אין משא פנים עם־האלהים׃
\par 12 כי כל־אשר חטאו ואין להם תורה גם בבלי־תורה יאבדו ואשר חטאו ולהם תורה על־פי התורה ישפטו׃
\par 13 כי לא שמעי התורה צדיקים לפני האלהים כי אם־עשי התורה הם יצדקו׃
\par 14 כי הגוים אשר אין־להם תורה בעשותם כדברי התורה מאליהם גם־באין תורה הם תורה לנפשם׃
\par 15 בהראותם מעשה התורה כתוב על־לבם ודעתם מעידה בהם ומחשבותם בקרבם מחיבות זאת את־זאת או מזכות׃
\par 16 ביום אשר ישפט האלהים את־כל־תעלמות בני האדם ביד ישוע המשיח כפי בשורתי׃
\par 17 הן אתה נקרא בשם יהודי ונשענת על־התורה ותתהלל באלהים׃
\par 18 וידעת את־רצונו ותבין בין־טוב לרע בהשכילך בתורה׃
\par 19 ובטחת בנפשך להיות מוליך העורים ואור לאשר בחשך׃
\par 20 אמן לחסרי לב ומורה הפתאים ויש לך צורת המדע והאמת בתורה׃
\par 21 ואתה התורה אחרים ונפשך לא תורה התאמר לא תגנב והנך גנב׃
\par 22 התאמר לא תנאף ואתה נאף תשקץ את־האלילים ואתה גזל את־הקדשים׃
\par 23 תתהלל בתורה ותנבל את־האלהים בעברך את־התורה׃
\par 24 כי בגללכם שם האלהים מחלל בגוים ככתוב׃
\par 25 הן המילה תועיל אם־תשמר את־התורה אבל אם־עבר אתה את־התורה מילתך היתה־לך לערלה׃
\par 26 ואם־ישמר הערל את־משפטי התורה הלא תחשב־לו ערלתו למילה׃
\par 27 והערל מלדה המקים את־התורה הוא ישפט אתך אשר־לך הכתב והמילה ועברת את־התורה׃
\par 28 כי לא־המצין למראה עינים הוא היהודי ולא האות הנראה בבשר היא המילה׃
\par 29 כי אם־תוכו של אדם הוא יהודי ומילה היא בלב כפי הרוח ולא כפי הכתב אשר־לא מבני אדם תהלתו כי אם־מאת האלהים׃

\chapter{3}

\par 1 אם כן מה־הוא יתרון היהודי ומה־היא תועלת המילה׃
\par 2 הרבה מכל־פנים תחלתו שבידם הפקדו דברי אלהים׃
\par 3 ואם־מקצתם לא האמינו מה־בכך היבטל חסרון אמונתם את־אמונת אלהים׃
\par 4 חלילה אבל האל הוא הנאמן וכל־האדם כזב ככתוב למען תצדק בדברך תזכה בשפטך׃
\par 5 ואם־עולתנו תודיע את־צדקת האלהים מה־נאמר היש־עול באלהים המשלח חרון אפו כדבר בני־אדם אני מדבר׃
\par 6 חלילה שאם־כן איך ישפט האלהים את־העולם׃
\par 7 כי אם־בכזבי תרבה ותפרץ אמתו של אלהים לתהלה לו למה אשפט עוד כחוטא׃
\par 8 ולמה לא נעשה כדבר מחרפינו ומוציאי דבה עלינו כאלו אמרים אנחנו נעשה הרע למען יבא הטוב אלה הם אשר עליהם יבא דינם בצדק׃
\par 9 ועתה מה היש־לנו מעלה יתרה לא במאומה כבר הוכחנו שגם־היהודים גם־היונים כלם תחת החטא׃
\par 10 ככתוב אין צדיק אין גם־אחד׃
\par 11 אין משכיל אין־דרש את־אלהים׃
\par 12 הכל סר יחדו נאלחו אין עשה־טוב אין גם־אחד׃
\par 13 קבר פתוח גרונם לשונם יחליקון חמת עכשוב תחת שפתימו׃
\par 14 אשר אלה פיהם מלא ומררות׃
\par 15 רגליהם ימהרו לשפך־דם׃
\par 16 שד ושבר במסלותם׃
\par 17 ודרך שלום לא ידעו׃
\par 18 אין פחד אלהים לנגד עיניהם׃
\par 19 ואנחנו ידענו כי כל־מה־שאמרה התורה מדברת לאלה שעול התורה עליהם כדי שיסכר כל־פה ויהי כל־העולם חיב לפני אלהים׃
\par 20 מפני שממעשי התורה לא־יצדק לפניו כל־בשר כי על־ידי התורה דעת החטא׃
\par 21 ועתה בבלי תורה צדקת אלהים יצאה לאור אשר העידו עליה התורה והנביאים׃
\par 22 והיא צדקת אלהים באמונת ישוע המשיח אל־כל ועל־כל אשר האמינו בו כי אין להבדיל׃
\par 23 כי־כלם חטאו וחסרי־כבוד אלהים המה׃
\par 24 ונצדקו חנם בחסדו על־ידי הפדות אשר היתה במשיח ישוע׃
\par 25 אשר שמו האלהים לפנינו לכפרת על־ידי האמונה בדמו להראות את־צדקתו אחרי אשר העביר את־החטאים הראשנים בעת חמלתו׃
\par 26 להראות את־צדקתו בעת הזאת כי צדיק הוא ומצדיק את־בן־אמונת ישוע׃
\par 27 ובכן איה תהלת המתהלל הלא אבדה ועל־ידי איזו תורה העל־ידי־תורת המעשים לא כי על־ידי תורת האמונה׃
\par 28 לכן דנים אנחנו שבאמונה יצדק האדם בבלי מעשי תורה׃
\par 29 או הרק אלהי היהודים האלהים הלא גם אלהי הגוים אכן גם־אלהי הגוים הוא׃
\par 30 כי אחד האלהים המצדיק את־המולים מתוך האמונה ואת־הערלים על־ידי האמונה׃
\par 31 המבטלים אפוא אנחנו את־התורה על־ידי האמונה חלילה אך מקימים אנחנו את־התורה׃

\chapter{4}

\par 1 ומה־נאמר אפוא על־אברהם אבינו מה־זה השיג לפי הבשר׃
\par 2 כי אם־נצדק אברהם מתוך המעשים לו התהילה אבל לא לפני האלהים׃
\par 3 כי מה אמר הכתוב והאמן אברהם ביהוה ויחשבה לו צדקה׃
\par 4 הנה הפעל לא־יחשב לו שכרו לפי החסד כי אם־לפי החובה׃
\par 5 אבל לאשר איננו פעל כי אם־מאמין במצדיק את־הרשע אמונתו תחשב לו לצדקה׃
\par 6 כאשר גם־דוד מאשר את־האדם אשר האלהים יחשב־לו צדקה בלא מעשים באמרו׃
\par 7 אשרי נשוי־פשע כסוי חטאה׃
\par 8 אשרי אדם לא־יחשב יהוה לו עון׃
\par 9 ובכן האשור הזה העל־המילה הוא או־גם על־הערלה הלא אמרנו כי לאברהם נחשבה אמונתו לצדקה׃
\par 10 ואימתי נחשבה־לו בהיותו נמול או בעודנו ערל הן לא בהיותו נמול כי אם בעודנו ערל׃
\par 11 ואות המילה נתן לו לחותם צדקת האמונה אשר היתה־לו והוא ערל להיות לאב לכל־אשר יאמינו והם ערלים למען תחשב הצדקה אף־להם׃
\par 12 ולהיות לאב גם־למולים אך־לא לאשר אינם אלא נמולים כי אם־גם־הלכים בעקבות האמונה שהיתה־לו לאברהם אבינו בעודנו ערל׃
\par 13 כי לא על־ידי תורה היתה ההבטחה לאברהם או לזרעו להיות ירש העולם כי אם־על־ידי צדקת האמונה׃
\par 14 כי אלו היתה הירשה לבני־התורה האמונה תהיה לריק וההבטחה בטלה׃
\par 15 יען אשר התורה מביאה קצף כי באין תורה אין עברה׃
\par 16 על־כן מאמונה למען תהיה לפי־חסד בעבור אשר תכון ההבטחה לכל הזרע לא לבני התורה לבדם כי־גם לבני אמונת אברהם אשר הוא אב לכלנו׃
\par 17 ככתוב כי אב־המון גוים נתתיך והוא האמין כי נגד פניו אלהים המחיה את־המתים והקורא למה־שלא־היה כמו הוה׃
\par 18 באפס תקוה האמין בתקוה למען אשר יהיה לאב המון גוים כמו שנאמר כה יהיה זרעך׃
\par 19 ולא רפתה אמונתו בהתבוננו אל־בשרו אשר היה כמת בהיותו כבן־מאת שנה ואל־רחם שרה אשר בלה׃
\par 20 ולא־חלק לבו בהבטחת האלהים כמחסר אמונה כי אם־התחזק באמונתו ויתן כבוד לאלהים׃
\par 21 ונפשו ידעת מאד כי את־אשר הבטיח גם־יכל לעשותו׃
\par 22 על־כן גם־נחשבה־לו לצדקה׃
\par 23 ולא־למענו לבד כתוב הדבר הזה שנחשבה לו׃
\par 24 כי אם־גם למעננו אשר עתידה להחשב לנו המאמינים במי שהעיר את־ישוע אדנינו מן־המתים׃
\par 25 אשר נמסר בעבור פשעינו ונעור לבעבור צדקנו׃

\chapter{5}

\par 1 לכן אחרי נצדקנו באמונה שלום לנו עם־האלהים באדנינו ישוע המשיח׃
\par 2 אשר בידו מצאנו באמונה גם־מבוא החסד הזה אשר אנחנו עמדים בו ונתהלל בתקות כבוד האלהים׃
\par 3 ולא־עוד אלא שנתהלל בצרות מפני שידענו כי־הצרה מביאה לידי סבלנות׃
\par 4 וסבלנות לידי עמידה בנסיון ועמידה בנסיון לידי תקוה׃
\par 5 ותקוה היא לא תביש כי הוצקה בלבבנו אהבת אלהים על־ידי רוח הקדש הנתן לנו׃
\par 6 כי המשיח בעודנו חלשים מת בעתו בעד הרשעים׃
\par 7 לא במהרה ימות איש בעד הצדיק אבל אפשר שישאהו לבו למות בעד הטוב׃
\par 8 ובזאת הודיע האלהים את־אהבתו אלינו אשר משיח מת בעדנו ואנחנו עוד חטאים׃
\par 9 ועתה אשר נצדקנו בדמו על אחת כמה וכמה שנושע בו מן־הקצף׃
\par 10 כי הנה הרצינו לאלהים במות בנו בהיותנו איבים אף כי־נושע עתה בחייו אחרי אשר הרצינו׃
\par 11 ולא־עוד אלא שמתהללים אנחנו באלהים על־יד אדנינו ישוע המשיח אשר בו עתה היה לנו הרצוי׃
\par 12 לכן כאשר על־ידי אדם אחד בא החטא לעולם והמות בעקב החטא וכן עבר המות על־כל־בני אדם מפני אשר כלם חטאו׃
\par 13 כי לפני מתן תורה כבר היה חטא בעולם אלא שלא יחשב חטא באין תורה׃
\par 14 אף על־פי כן משל המות מאדם עד־משה גם על־אותם שלא חטאו כפשעו של־אדם הראשון אשר בדמותו הוא העתיד לבוא׃
\par 15 אבל לא כפשע המתנה כי הנה בפשע האחד מתו הרבים אף כי־חסד אלהים ומתנתו רבו לרבים בחסד האדם האחד ישוע המשיח׃
\par 16 ואין המתנה כדבר שהיה על־ידי אחד שחטא כי הדין בא מאחד לחיב ומתנת החסד היא לזכות מפשעים רבים׃
\par 17 כי אם־בפשע האחד מלך המות על־ידי האחד אף כי־מקבלי שפעת החסד ומתנת הצדקה ימלכו בחיים על־ידי האחד ישוע המשיח׃
\par 18 לכן כאשר בפשע אחד נאשמו כל־בני־אדם כן בזכות אחת יזכו כל־בני־אדם לחיים׃
\par 19 כי כאשר במרי האדם האחד היו הרבים לחטאים כן במשמעת האחד יהיו הרבים לצדיקים׃
\par 20 והתורה נכנסה למען ירבה הפשע ובאשר רבה החטא עדף עליו החסד׃
\par 21 למען ימלך החסד על־ידי הצדקה לחיי עולם בישוע המשיח אדנינו כאשר מלך החטא במות עד־הנה׃

\chapter{6}

\par 1 אם־כן מה־נאמר הנעמד בחטא למען ירבה החסד׃
\par 2 חלילה לנו כי מתנו לחטא ואיך נוסיף לחיות בו׃
\par 3 או האינכם ידעים כי כלנו הנטבלים למשיח ישוע למותו נטבלנו׃
\par 4 לכן נקברנו עמו בטבילה למות למען נתהלך בחיים מחדשים כאשר המשיח נעור מן־המתים על־ידי כבוד האב׃
\par 5 כי אם־נדבקנו בדמיון מותו אכן דבוקים נהיה גם־לתחיתו׃
\par 6 באשר ידעים אנחנו כי־נצלב עמו האדם הישן אשר בנו למען יאבד גוף החטא ולא נהיה עוד עבדים לחטא׃
\par 7 כי המת נקה מן־החטא׃
\par 8 והנה אם־מתנו עם־המשיח נאמין כי־גם־נחיה עמו׃
\par 9 באשר ידענו כי המשיח אחרי אשר נעור מן־המתים לא ימות עוד ולא ישלט־בו עוד המות׃
\par 10 כי אשר מת מת לחטא פעם אחד ואשר חי חי הוא לאלהים׃
\par 11 וכן גם־אתם היו בעיניכם כמתים לחטא וחיים לאלהים במשיח ישוע אדנינו׃
\par 12 אם־כן אפוא אל־תשלט החטאת בגופכם אשר ימות להטות לבבכם אחרי תאותיו׃
\par 13 ואל־תתנו את־אבריכם להיות לכלי־עול לחטא אבל תנו עצמכם לאלהים כחיים מעם המתים ואבריכם לכלי צדקה לאלהים׃
\par 14 כי החטא לא ישתרר עוד עליכם מפני שאינכם תחת התורה כי אם־תחת החסד׃
\par 15 ועתה הנחטא מפני שאין אנחנו תחת התורה כי אם־תחת החסד חלילה׃
\par 16 הלא ידעתם כי אשר תתנו נפשכם לו להיות עבדיו לסור למשמעתו עבדים אתם לו לשמע בקולו אם־לחטא אלי־מות אם־למשמעת אלי־צדקה׃
\par 17 אבל תודת לאלהים כי־הייתם עבדי החטא ואחר שמעתם בכל־לבבכם לצורת הלקח אשר חנכתם בה׃
\par 18 שחררתם מידי החטא לכן השתעבדתם לצדקה׃
\par 19 כדרך בני־אדם אני מדבר מפני בשרכם החלוש כי כאשר לפנים הכינותם את־אבריכם לעבודת הטמאה והרשע להרשיע כן עתה הכינו את־אבריכם לעבודת הצדקה להתקדש׃
\par 20 כי־בעת היותכם עבדי החטא חפשים הייתם מן־הצדקה׃
\par 21 ומה אפוא הפרי שהיה לכם אז מן־המעשים אשר עתה תבשו מהם כי אחריתם המות׃
\par 22 אכן עתה בהיותכם משחררים מידי החטא ומשעבדים לאלהים יש לכם פריכם לקדשה ואחריתו חיי עולם׃
\par 23 כי־שכר החטא הוא המות ומתנת חסד אלהים היא חיי העולמים במשיח ישוע אדנינו׃

\chapter{7}

\par 1 או הלא ידעתם אחי כי לידעי התורה אני מדבר כי התורה תשלט על־האדם כל־ימי חייו׃
\par 2 כי אשת איש מן התורה זקוקה לבעלה בחייו ובמות בעלה פטורה היא מדין בעלה׃
\par 3 ועל־כן אם־תהיה לאיש אחר בחיי בעלה נאפת יקרא לה ובמות בעלה חפשית היא מן־התורה ואיננה נאפת בהיותה לאיש אחר׃
\par 4 וכן אחי גם־אתם הייתם כמתים לתורה בגוית המשיח להיות לאחר לאשר נעור מן־המתים למען נעשה־פרי לאלהים׃
\par 5 כי בעת היותנו בבשר תשוקות החטאים אשר התעוררו על־ידי התורה היו פעלות באברינו לעשות פרי למות׃
\par 6 אבל עתה פטורים אנחנו מן־התורה כי מתנו לאשר היינו זקוקים לו למען נעבד מעתה לפי חדוש הרוח ולא לפי־ישן הכתב׃
\par 7 אם־כן הנאמר שהתורה חטא היא חלילה אלא לא ידעתי את־החטא בלתי על־ידי התורה כי לא־הייתי יודע החמוד לולי אמרה התורה לא תחמד׃
\par 8 והחטא מצא לו סבה במצוה לעורר בקרבי כל־חמוד כי מבלעדי התורה החטא מת הוא׃
\par 9 ואני הייתי חי מלפנים בלא תורה וכשבאה המצוה ויחי החטא׃
\par 10 ואני מתי ונמצא שהמצוה אשר נתנה לחיים היתה לי למות׃
\par 11 כי־מצא החטא סבה במצוה להתעות אתי וימיתני על־ידה׃
\par 12 ובכן התורה היא קדושה והמצוה קדושה וישרה וטובה׃
\par 13 הכי הטובה היתה־לי למות חלילה אלא החטא כדי שיראה החטא בהביאו לי המות מן־הטובה כדי שיהיה החטא לחטאה יתרה על־ידי המצוה׃
\par 14 כי ידעים אנחנו שהתורה רוחנית ואני בשר ונמכר ביד־החטא׃
\par 15 כי את־אשר אני פעל לא ידעתי כי אינני עשה את אשר־אני רצה בו כי אם־אשר שנאתי אתו אני עשה׃
\par 16 ובעשותי את אשר לא־רציתי הנני מודה כי התורה טובה היא׃
\par 17 ועתה לא־אני עוד הפעל אתו כי אם־החטא השכן בקרבי׃
\par 18 כי ידעתי אשר־בי בבשרי לא ישכן טוב כי רצה אני לעשות הטוב ולא אמצא׃
\par 19 כי אינני עשה הטוב אשר־אני רצה כי אם־הרע אשר אינני רצה אותו אני עשה׃
\par 20 ואם את־אשר לא־רציתי אני עשה לא־עוד אני הפעל כי אם־החטא השכן בקרבי׃
\par 21 ובכן מצא־אני בי זה החק אנכי רצה לעשות הטוב ודבק־בי הרע׃
\par 22 כי לפי האדם הפנימי חפצתי בתורת אלהים׃
\par 23 אבל ראה־אני באברי חק אחר הלחם לחק־שכלי ויוליכני שבי לתורת החטא אשר באברי׃
\par 24 אוי־לי האדם העני מי יצילני מגוף המות הזה׃
\par 25 אברכה את־האלהים בישוע המשיח אדנינו׃ (7:26) ובכן בשכלי הנני עבד לתורת האלהים ובבשרי אני עבד לתורת החטא׃

\chapter{8}

\par 1 על־כן עתה אין־אשמה באלה אשר הם במשיח ישוע (המתהלכים שלא כבשר אלא לפי הרוח)׃
\par 2 כי תורת רוח החיים אשר במשיח ישוע הוציאה אתי לחפשי מתורת החטא והמות׃
\par 3 כי מה־שלא יכלה התורה לעשות מפני שנחלש כחה על־ידי הבשר אתו עשה האלהים בשלחו את־בנו בדמיון בשר החטא ובעד החטא וירשיע את־החטא בבשר׃
\par 4 כדי שתקים צדקת התורה בנו ההלכים לא־כדרך הבשר כי אם־לפי הרוח׃
\par 5 כי בני הבשר יהגו בדברי הבשר ובני הרוח בדברי הרוח׃
\par 6 כי־מחשבת הבשר היא המות ומחשבת הרוח היא החיים והשלום׃
\par 7 מפני שמחשבת הבשר רק שנאת אלהים היא באשר לא תשתעבד לתורת האלהים ואף לא תוכל׃
\par 8 כל־אשר בבשר יסודם לא יוכלו להיות רצוים לאלהים׃
\par 9 ואתם אינכם בבשר כי אם־ברוח אם־אמנם רוח האלהים שכן בקרבכם כי מי שאין־בו רוח המשיח הוא איננו שלו׃
\par 10 ואם־המשיח בקרבכם הגוף מת בגלל החטא והרוח חיים בגלל הצדקה׃
\par 11 ואם־ישכן בקרבכם רוחו של המעיר את־ישוע מן־המתים המעיר את־המשיח מן־המתים הוא גם את־גויותיכם המתות יחיה על־ידי רוחו השכן בקרבכם׃
\par 12 לכן אחי חיבים אנחנו לא לבשר לחיות לפי הבשר׃
\par 13 כי אם־תחיו לפי הבשר מות תמתון ואם־על־ידי הרוח תמיתו את־מעללי הבשר חיה תחיו׃
\par 14 כי־כל אשר רוח אלהים ינהגם בני אלהים המה׃
\par 15 כי לא קבלתם רוח עבדות לשוב לירא כי אם־קבלתם רוח משפט בנים אשר בו קראים אנחנו אבא אבינו׃
\par 16 והרוח ההוא מעיד ברוחנו כי־בני אלהים אנחנו׃
\par 17 ואם־בנים אנחנו גם־ירשים נהיה ירשי נחלת אלהים וחברי המשיח בירשה אם־נתענה אתו למען גם־אתו נכבד׃
\par 18 כי אמר אני שענויי הזמן הזה אינם שקולים כנגד הכבוד הבא להגלות עלינו׃
\par 19 כי הבריאה תערג ותצפה למועד אשר יתגלו בני האלהים׃
\par 20 כי־נכנעה הבריאה להבל לא מרצונה כי אם־למען המכניע אתה ולא באין תקוה׃
\par 21 כי הבריאה גם־היא תצא מעבדות הכליון אל־חרות כבוד בני האלהים׃
\par 22 כי ידענו אשר הבריאה כלה תאנח ותחיל עד־הנה׃
\par 23 ולא־עוד אלא שגם־אנחנו אף על פי שיש־לנו בכורי הרוח נאנח בנפשנו ונחכה למשפט הבנים לפדות גויתנו׃
\par 24 כי נושענו בתקוה אבל התקוה הנראה לעינים איננה תקוה כי איך ייחל איש לדבר אשר־הוא ראה׃
\par 25 אלא אם־נקוה למה־שלא ראינהו נחכה לו ונוחיל׃
\par 26 וכן גם־הרוח תמך אתנו בחלשותינו כי לא ידענו להתפלל כראוי אכן הרוח הוא מפגיע בעדנו באנחות עמקות מדבר׃
\par 27 והחקר לבבות יודע את־מחשבות הרוח כי כרצון האלהים יפגיע בעד הקדושים׃
\par 28 והנה ידענו כי אהבי אלהים הקרואים בעצתו הכל יעזר לטוב להם׃
\par 29 כי את אשר ידעם מקדם אתם גם־יעד להיות דומים לצלם בנו למען יהיה הבכור בתוך אחים רבים׃
\par 30 ואת אשר־יעד מקדם אתם גם־קרא ואת־אשר קרא אתם גם־הצדיק ואת אשר הצדיק אתם גם פאר׃
\par 31 ועתה מה־נאמר על־זאת אם־האלהים לנו מי יריב אתנו׃
\par 32 אשר־על־בנו שלו לא חס כי אם־נתנו בעד כלנו הלא יתן לנו עמו את־הכל׃
\par 33 מי יענה בבחירי אלהים הן אלהים הוא המצדיק׃
\par 34 ומי־הוא יאשימם הן המשיח אשר מת ואשר נעור מעם המתים הוא מימין האלהים והוא יפגיע בעדנו׃
\par 35 מי יפרידנו מאהבת האלהים הצרה או מצוקה או משטמה או רעב אם־עריה או סכנה או־חרב׃
\par 36 ככתוב כי־עליך הרגנו כל־היום נחשבנו כצאן טבחה׃
\par 37 אבל בכל־אלה גברנו מאד על־ידי האהב אתנו׃
\par 38 ובטוח אני שלא המות ולא החיים לא מלאכים ולא שררות ולא גבורות לא ההוה ולא העתיד׃
\par 39 לא הרום ולא העמק ולא כל־בריה יוכלו להפרידנו מאהבת האלהים אשר היא במשיח ישוע אדנינו׃

\chapter{9}

\par 1 אמת אני מדבר במשיח ולא אשקר ודעתי מעידה לי ברוח הקדש׃
\par 2 כי־גדול עצבוני ואין־קץ לדאבון לבי׃
\par 3 כי מי־יתן היותי אני לחרם מן־המשיח בעד אחי שארי ובשרי׃
\par 4 אשר הם בני ישראל ולהם משפט הבנים והכבוד והבריתות ומתן התורה והעבודה וההבטחות׃
\par 5 ולהם האבות ומהם יצא המשיח לפי בשרו אשר־הוא אלהים על־הכל מברך לעולמים אמן׃
\par 6 אבל לא שנפל דבר אלהים ארצה כי לא־כל אשר מישראל ישראל המה׃
\par 7 ולא מפני שהם זרע אברהם כלם בנים כי ביצחק יקרא לך זרע׃
\par 8 כלומר לא בני־הבשר המה בני האלהים כי אם־בני ההבטחה הם הנחשבים לזרע׃
\par 9 כי־דבר ההבטחה הוא מה־שנאמר למועד אשוב ולשרה בן׃
\par 10 ולא־עוד אלא שהיה גם־ברבקה והיא הרה לאחד ליצחק אבינו׃
\par 11 כי בטרם ילדו בניה ועוד לא־עשו טוב או־רע למען תקום עצת האלהים כפי בחירתו לא מתוך מעשים כי אם־כרצון הקרא׃
\par 12 נאמר לה כי־רב יעבד צעיר׃
\par 13 ככתוב ואהב את־יעקב ואת־עשו שנאתי׃
\par 14 אם־כן הנאמר שיש־עול באלהים חלילה׃
\par 15 כי למשה אמר וחנתי את־אשר אחן ורחמתי את־אשר ארחם׃
\par 16 ועל־כן אין הדבר לא־ביד הרצה ולא־ביד הרץ כי אם־ביד האלהים המרחם׃
\par 17 כי־כן הכתוב אמר לפרעה בעבור זאת העמדתיך בעבור הראתך את־כחי ולמען ספר שמי בכל־הארץ׃
\par 18 ויודע בזה שמי שיחפץ יחננו ומי שיחפץ יקשה לבו׃
\par 19 ואם תאמר למה־זה יפקד עון כי נגד רצונו מי יתיצב׃
\par 20 אבל בן־אדם מי אתה כי תריב את־האלהים היאמר יצר ליצרו מדוע ככה עשיתני׃
\par 21 אם־אין רשות ליצר על־החמר לעשות הגלם האחד כלי כבוד או כלי קלון׃
\par 22 ומה אפוא אם־האלהים החפץ להראות זעמו ולהודיע גבורתו נשא בכל־ארך רוחו את־כלי הזעם הנכונים לאבדון׃
\par 23 להודיע גם־את־עשר כבודו בכלי החנינה אשר הכין לכבוד׃
\par 24 והם אנחנו אשר קראנו לא מן־היהודים לבדם כי אף מן־הגוים׃
\par 25 כאמרו בהושע אקרא ללא־עמי עמי וללא־רחמה רחמה׃
\par 26 והיה במקום אשר־יאמר להם לא־עמי אתם יאמר להם בני אל־חי׃
\par 27 וישעיהו קרא על־ישראל כי אם־יהיה מספר בני ישראל כחול הים שאר ישוב בו (כליון חרוץ שוטף צדקה)׃
\par 28 כי כלה ונחרצה אדני עשה בקרב הארץ׃
\par 29 וכאשר אמר ישעיהו לפני מזה לולי יהוה צבאות הותיר לנו שריד כמעט כסדם היינו לעמרה דמינו׃
\par 30 ועתה הנאמר שהגוים אשר לא רדפו אחרי הצדקה השיגו את־הצדקה היא הצדקה אשר מתוך האמונה׃
\par 31 וישראל ברדפו תורת צדקה לתורת הצדקה לא הגיע׃
\par 32 ועל־מה על־אשר־לא מאמונה דרשוה כי אם־ממעשים כי התנגפו באבן נגף׃
\par 33 ככתוב הנני יסד בציון אבן נגף וצור מכשול וכל־המאמין בו לא יבוש׃

\chapter{10}

\par 1 אחי חפץ לבבי ותפילתי לאלהים בעד ישראל אשר יושעו׃
\par 2 כי מעיד אני עליהם שמקנאים לאלהים אבל לא־בדעת׃
\par 3 כי את־צדקת אלהים לא ידעו ויבקשו להקים את־צדקתם ולצדקת אלהים לא נכנעו׃
\par 4 כי המשיח סוף התורה לצדקה לכל־המאמין בו׃
\par 5 כי־משה כתב על־דבר הצדקה מתוך התורה אשר יעשה אתם האדם וחי בהם׃
\par 6 והצדקה אשר מתוך האמונה אמרת אל־תאמר בלבבך מי־יעלה השמימה להוריד את־המשיח׃
\par 7 או מי ירד לתהום להעלות את־המשיח מן־המתים׃
\par 8 אבל מה־תאמר קרוב אליך הדבר בפיך ובלבבך הוא דבר האמונה אשר אנחנו מבשרים׃
\par 9 כי אם־בפיך תודה שישוע הוא האדון ותאמין בלבבך שהאלהים העירו מן־המתים תושע׃
\par 10 כי בלבבו יאמין האדם והיתה לו לצדקה ובפיהו יודה והיתה־לו לישועה׃
\par 11 כי הכתוב אמר כל־המאמין בו לא יבוש׃
\par 12 ואין הפרש בין היהודי ליוני כי אדון אחד לכלם והוא עשיר לכל־קראיו׃
\par 13 כי־כל אשר־יקרא בשם יהוה ימלט׃
\par 14 ועתה איך יקראו אל־אשר לא־האמינו בו ואיך יאמינו במי שלא שמעו את־שמעו ואיך ישמעו ואין מגיד׃
\par 15 ואיך יגידו כי אם־שלוחים ככתוב מה־נאוו רגלי מבשר שלום מבשר טוב׃
\par 16 אבל לא־כלם שמעו לקול הבשורה כי ישעיהו אמר יהוה מי האמין לשמעתנו׃
\par 17 אם־כן האמונה באה מתוך השמועה והשמועה על־ידי דבר־המשיח׃
\par 18 ואמר הכי לא שמעו אמנם בכל־הארץ יצא קום ובקצה תבל מליהם׃
\par 19 ואמר הכי ישראל לא ידע הנה־כבר משה אמר אני אקניאכם בלא־עם בגוי נבל אכעיסכם׃
\par 20 וישעיהו מלאו לבו לאמר נמצאתי ללא בקשני נדרשתי ללוא שאלו׃
\par 21 ועל־ישראל הוא אמר פרשתי ידי כל־היום אל־עם סורר ומרה׃

\chapter{11}

\par 1 ובכן אמר אני הזנח האלהים את־עמו חלילה כי גם־אנכי ישראלי מזרע אברהם למטה בנימין׃
\par 2 לא־זנח האלהים את־עמו אשר ידעו מקדם הלא תדעו את־אשר הכתוב אמר באליהו והוא צעק אל־האלהים על־ישראל לאמר׃
\par 3 יהוה את־נביאיך הרגו ואת־מזבחתיך הרסו ואותר אני לבדי ויבקשו את־נפשי׃
\par 4 ומה־ענה אתו דבר אלהים השארתי לי שבעת אלפים איש אשר לא־כרעו לבעל׃
\par 5 וכן גם־בימינו נשארה שארית כבחירת החסד׃
\par 6 ואם על־ידי החסד לא היתה מתוך המעשים כי אם־כן החסד איננו־עוד חסד (ואם־היתה מתוך המעשים איננו־עוד חסד כי אם־כן המעשה יחדל להיות מעשה)׃
\par 7 ועתה מה־הוא את אשר־בקש ישראל לא השיג רק הנבחרים הם השיגו והאחרים השמינו לבבם׃
\par 8 ככתוב נתן להם האלהים רוח תרדמה עינים לא לראות ואזנים לא לשמע עד־היום הזה׃
\par 9 ודוד אמר יהי שלחנם לפח ולרשת ולמוקש ולשלומים להם׃
\par 10 תחשכנה עיניהם מראות ומתניהם תמיד המעד׃
\par 11 ועתה אני אמר הנכשלו למען יפלו חלילה כי בפשעם באה הישועה לגוים למען הקניאם׃
\par 12 ואם־פשעם היה לעשר העולם ונזקם לעשר הגוים מלאם על־אחת כמה וכמה׃
\par 13 ואליכם הגוים אני מדבר וכפי היותי שליח לגוים את־שרותי אכבד׃
\par 14 אולי אוכל להקניא את בני־עמי ולהושיע מקצתם׃
\par 15 כי אם־דחיתם רצוי לעולם מה־אפוא תהיה אספתם הלא חיים מן־המתים׃
\par 16 ואם־התרומה קדש העסה קדש כמוה ואם־השרש קדש הענפים קדש כמוהו׃
\par 17 וכי נקפו מקצת הענפים ואתה זית היער הרכבת תחתיהם ונתחברת לשרש הזית ולדשנו׃
\par 18 אל־תתפאר על־הענפים ואם־תתפאר דע שאתה לא תשא את־השרש כי אם־השרש נשא אותך׃
\par 19 וכי תאמר הלא נקפו הענפים למען ארכב אני׃
\par 20 כן הדבר המה נקפו על־אשר לא האמינו ואתה הנך קים על־ידי האמונה אל־תתגאה כי אם־ירא׃
\par 21 כי הנה האלהים לא־חס על־הענפים הנולדים מן־העץ ואולי לא־יחוס גם־עליך׃
\par 22 לכן ראה־נא טובת אלהים וזעמו זעמו על־הנפלים ועליך טובתו אם־תעמד בטובתו ואם־אין גם־אתה תכרת׃
\par 23 וגם־המה אם־לא יעמדו במרים ירכבו כי־יכל האלהים לשוב להרכיבם׃
\par 24 הן אתה נגזרת מעץ אשר בטבעו זית יער והרכבת שלא כטבע בזית טוב על־אחת כמה וכמה שירכבו אלה כטבעם בזית אשר יצאו ממנו׃
\par 25 כי לא־אכחד מכם אחי את־הסוד הזה פן־תהיו חכמים בעיניכם שישראל בא לידי טמטום הלב למקצתו עד כי־יכנס מלא הגוים׃
\par 26 וכן כל־ישראל יושע ככתוב ובא לציון גואל וישיב פשע מיעקב׃
\par 27 וזאת בריתי אשר אכרת אתם כי אסלח לעונם׃
\par 28 הן בדבר הבשורה איבים הם בגללכם ובדבר הבחירה חביבים הם בגלל האבות׃
\par 29 כי לא־ינחם האלהים על־מתנותיו ולא על־קריאתו׃
\par 30 כי כאשר גם־אתם מלפנים ממרים הייתם את־פי אלהים ועתה הוחנתם במרים של־אלה׃
\par 31 כן גם־אלה עתה ממרים למען יחנו גם־הם על־ידי חנינתכם׃
\par 32 כי־האלהים הסגיר את־כלם למרי למען יחן את־כלם׃
\par 33 מה־עמק עשר חכמת אלהים ועשר דעתו משפטיו מי יחקר ודרכיו מי ימצא׃
\par 34 כי מי־תכן את־רוח יהוה ואיש עצתו יודיענו׃
\par 35 או מי הקדים אתו בדבר וישלם לו׃
\par 36 הלא ממנו הכל ועל־ידו הכל ואליו הכל ולו הכבוד לעולמים אמן׃

\chapter{12}

\par 1 ועתה הנני מעורר אתכם אחי ברחמי אלהים אשר תגישו את־גויותיכם קרבן חי וקדוש ונרצה לאלהים והיתה זאת עבודתכם השכלית׃
\par 2 ואל־תדמו לעולם הזה כי אם־התחלפו להיות לכם לב חדש לבחן מה־הוא רצון האלהים הטוב והנחמד והשלם׃
\par 3 כי על־פי החסד הנתן לי אמר אני לכל־איש בכם לבלתי רום־לבבו למעלה מן־הראוי כי אם־יהי צנוע במחשבותיו כמדת האמונה אשר־חלק לו האלהים׃
\par 4 כי כאשר בגוף אחד יש־לנו אברים הרבה ולא כל־האברים ישמשו שמוש אחד׃
\par 5 כן אנחנו הרבים גוף אחד במשיח וכל־אחד ואחד ממנו אבר לחברו׃
\par 6 ויש־לנו מתנות שנות כחסד הנתן לנו אם־נבואה היא תהי כמדת האמונה׃
\par 7 ואם־שמוש לאיש יעסק בשמושו ואם־מורה בהוראתו׃
\par 8 ואם־מוכיח בתוכחתו הנותן יעשה בתם־לבב והמנהיג בשקידה והגמל חסד בסבר פנים יפות׃
\par 9 אהבתכם תהי בלי חנפה שנאו את־הרע ודבקו בטוב׃
\par 10 אהבו את־אחיכם מחבבים זה את־זה הקדימו איש את־רעהו לנהוג בו כבוד׃
\par 11 שקדו ואל־תעצלו התלהבו ברוח והיו עבדים לאדון׃
\par 12 שמחו בתקוה הוחילו בצרה שקדו על־התפלה׃
\par 13 התנדבו צרכי הקדושים רדפו להכניס ארחים׃
\par 14 ברכו את־רדפיכם ברכו ואל־תקללו׃
\par 15 שמחו עם־השמחים ובכו עם־הבכים׃
\par 16 לב אחד יהי לכלכם אל־תהלכו בגדלות כי אם־התנהגו עם־השפלים אל־תהיו חכמים בעיניכם׃
\par 17 אל־תשלמו לאיש רעה תחת רעה דרשו הטוב בעיני כל־אדם׃
\par 18 אם־תוכלו ככל־אשר תמצא ידכם יהי לכם שלום עם־כל־אדם׃
\par 19 אל־תנקמו נקם ידידי כי אם־תנו מקום לחרון־אף כי כתוב לי נקם ושלם אמר יהוה׃
\par 20 לכן אם־רעב שנאך האכילהו לחם ואם־צמא השקהו מים כי גחלים אתה חתה על־ראשו׃
\par 21 אל־נא יכבשך הרע כבוש אתה את־הרע בטוב׃

\chapter{13}

\par 1 כל־נפש תכנע לגדלת הרשיות כי־אין רשות כי אם־מאת האלהים והרשיות הנמצאות על־יד אלהים נתמנו׃
\par 2 לכן כל־המתקומם לרשות ממרה את־פי האלהים והממרים ישאו את־עונם׃
\par 3 כי אין פחד השליטים על עשי הטוב כי אם־על עשי הרע ועל־כן אם־רצונך שלא תירא מן־הרשות עשה הטוב והיה־לך שבח מאתה׃
\par 4 כי משרתת אלהים היא לטוב לך אבל אם־הרע תעשה ירא כי לא לחנם חגרת־חרב היא כי־משרתת אלהים היא לשלם גמול ולשפך חמה על עשי הרע׃
\par 5 על־כן עלינו להכנע לא בעבור הקצף בלבד כי־גם מדעת חובתנו׃
\par 6 על־כן נתנים אתם את־המס כי משרתי אלהים הם עמדים לזאת על משמרתם׃
\par 7 לכן תנו לכל־איש מה־שאתם חיבים לו המס לאשר־לו המס והמכס לאשר־לו המכס והמורא לאשר־לו המורא והכבוד לאשר־לו הכבוד׃
\par 8 ואל־תהיו חיבים לאיש דבר זולתי אהבת איש את־רעהו כי האהב את־חברו קים את־התורה׃
\par 9 כי מצות לא תנאף לא תרצח לא תגנב לא תענה עד שקר לא תחמד עם כל־מצות אחרות כלן הנה בכלל המאמר הזה ואהבת לרעך כמוך׃
\par 10 האהבה לא תרע לרע על־כן האהבה קיום התורה כלה׃
\par 11 וכזאת עשו מפני שאתם ידעים את־השעה כי־כבר עת להקיץ מן־השנה כי ישועתנו קרובה עתה מהיום אשר באנו להאמין׃
\par 12 הלילה חלף והיום קרב לכן נסירה־נא את־מעשי החשך ונלבשה את־כלי נשק האור׃
\par 13 וכהתהלך באור היום נתהלכה בצניעות לא בזוללות ובשכרון ולא בגילוי עריות ועשות זמה ולא במריבה וקנאה׃
\par 14 כי אם־לבשו את־האדון ישוע המשיח ודאגו לבשרכם אך־לא להגביר התאות׃

\chapter{14}

\par 1 ואת־החלוש באמונה אותו קבלו ולא לדין את־המחשבות׃
\par 2 יש מאמין שמתר לאכל כל־דבר והחלוש לא יאכל כי אם־ירק׃
\par 3 האכל אל־יבז את־אשר לא יאכל ואשר לא יאכל אל־ידין את־האכל כי־קבל אתו האלהים׃
\par 4 מי אתה כי תדין עבד שאינו שלך הן לאדניו הוא אם יקום ואם יפל אבל יוקם כי־יכל האלהים להקימו׃
\par 5 יש מבדיל בין־יום ליום ויש אשר כל־הימים דמים בעיניו יהי כל־איש נכון בדעתו׃
\par 6 השמר את־היום לקדשו שמר אתו לאדון ואשר איננו שמר לאדון איננו שמר האכל אכל לשם האדון כי מודה הוא לאלהים ואשר איננו אכל לשם האדון איננו אכל ומודה הוא לאלהים׃
\par 7 כי אין־איש ממנו חי לנפשו ואין איש מת לנפשו׃
\par 8 כי אם־נחיה נחיה לאדון ואם נמות נמות לאדון לכן בין חיים ובין מתים לאדון הננו׃
\par 9 כי בעבור זאת מת המשיח (ויקם) ויחי למען יהיה אדון גם על־המתים גם על־החיים׃
\par 10 ואתה מה־לך כי תדין את־אחיך ומה־לך כי תבוז לאחיך הלא כלנו עתידים לעמד לפני כסא דין אלהים׃
\par 11 כי כתוב חי־אני נאם־יהוה כי לי תכרע כל־ברך וכל־לשון תודה לאלהים׃
\par 12 הנה כל־אחד ממנו על־נפשו יתן חשבון לאלהים׃
\par 13 לכן אל־נדין עוד איש את־רעהו כי אם־זה יהי דינכם שלא־יתן איש לפני אחיו מכשול או מוקש׃
\par 14 אני ידעתי וברור לי הדבר באדון ישוע כי־אין טמא בפני עצמו ורק־טמא הוא למי שיחשבנו לו לטמא׃
\par 15 ואם־יעצב אחיך על־דבר מאכל אינך הלך דרך אחוה אל־נא תאבד באכלך את־אשר בעדו מת המשיח׃
\par 16 לכן הזהרו פן־יהיה טובכם לגדופים׃
\par 17 כי־מלכות האלהים איננה אכילה ושתיה כי־צדקה היא ושלום ושמחה ברוח הקדש׃
\par 18 והעבד באלה את־המשיח רצוי הוא לאלהים ובחון לאנשים׃
\par 19 ועתה נרדפה־נא דרכי שלום ואשר נכונן בו איש את־רעהו׃
\par 20 אל־תהרוס את־מעשה האלהים על־דבר מאכל הן הכל טהור ורע הוא לאדם אשר יאכלנו למכשל׃
\par 21 טוב שלא־תאכל בשר ולא־תשתה יין ולא־תעשה דבר אשר יתנגף־בו אחיך והיה לו למכשל ולפוקה׃
\par 22 אם יש לך אמונה תהי־לך לבדך לפני האלהים אשרי העשה הכשר בעיניו ואין לבו נקפו׃
\par 23 ואשר ספק לו באכלו נאשם כי לא עשה מאמונה וכל־הנעשה שלא מאמונה חטא הוא׃

\chapter{15}

\par 1 ואנחנו החזקים עלינו לשאת חלשות הרפים ואל־נבקש הנאת עצמנו׃
\par 2 כי כל־אחד ממנו יבקש הנאת חברו לטוב לו למען יתכונן׃
\par 3 כי גם־המשיח לא בקש הנאת עצמו אלא ככתוב חרפות חורפיך נפלו עלי׃
\par 4 כי כל־אשר נכתב מלפנים נכתב ללמדנו למען תהיה־לנו תקוה בסבלנות ובתנחומות הכתובים׃
\par 5 ואלהי הסבלנות והנחמה הוא יתן והייתם כלכם לב אחד כרצון המשיח ישוע׃
\par 6 אשר תכבדו בנפש אחת ובפה אחד את־האלהים אבי אדנינו ישוע המשיח׃
\par 7 על־כן קבלו־נא איש את־אחיו כאשר גם־המשיח קבל אתנו לכבוד האלהים׃
\par 8 ואני אמר כי ישוע המשיח נולד להיות משרת הנמולים למען אמתו של האלהים לקים את־ההבטחות אשר לאבות׃
\par 9 והגוים המה יכבדו את־האלהים למען רחמיו ככתוב על־כן אודך בגוים ולשמך אזמרה׃
\par 10 ואומר הרנינו גוים עמו׃
\par 11 ואומר הללו את־יהוה כל־גוים שבחוהו כל־האמים׃
\par 12 וישעיהו אמר והיה שרש ישי אשר עמד לנשיא עמים אליו גוים יקוו׃
\par 13 ואלהי התקוה הוא ימלא אתכם כל־שמחה ושלום באמונה למען תעדף תקותכם בגבורת רוח הקדש׃
\par 14 והנה אחי מבטח אני בכם כי מלאי אהבת חסד אתם וממלאים כל־דעת וידעים להוכיח איש את־רעהו׃
\par 15 אף־על־פי כן העזתי מעט לכתב אליכם הנה והנה כמזכיר אתכם על־פי החסד הנתון לי מאת האלהים׃
\par 16 להיות משרת ישוע המשיח לגוים ולכהן בבשורת האלהים למען יהיה קרבן הגוים רצוי ומקדש ברוח הקדש׃
\par 17 על־כן יש־לי להתהלל במשיח ישוע בעניני האלהים׃
\par 18 כי לא־אעז פני לדבר דבר זולתי אשר עשה המשיח בידי למען הטות באמר ובמעשה את־לב הגוים לסור למשמעתו׃
\par 19 בגבורת אתות ומופתים ובגבורת רוח אלהים עד־כי מירושלים וסביבותיה ועד לאלוריקון מלאתי את־בשורת המשיח׃
\par 20 בהשתדלי להגיד את־הבשורה לא במקמות אשר־שם כבר נקרא שם המשיח שלא אבנה על־יסוד אחרים׃
\par 21 אלא ככתוב אשר לא־ספר להם ראו ואשר לא־שמעו התבוננו׃
\par 22 והוא הדבר אשר־בגללו נעצרתי פעם ושתים מבוא אליכם׃
\par 23 אבל עתה שאין־לי עוד מקום בגלילות האלה ואני נכסף לבא אליכם זה שנים רבות׃
\par 24 אבוא אליכם בלכתי לאספמיא כי מקוה אנכי לראותכם בעברי ואתם תשלחוני שמה ואשבעה מעט מכם בראשונה׃
\par 25 אמנם עתה אלכה ירושלימה לעזור לקדושים׃
\par 26 כי מקדוניא ואכיא הואילו לתרם תרומה לאביוני הקדושים אשר בירושלים׃
\par 27 כי הואילו ואף־מחיבים הם להם הנה לגוים היה חלק בדברי הרוח אשר להם והלא עליהם לתמכם גם בדברי הגוף׃
\par 28 לכן כשגמרתי את־זאת וחתמתי להם הפרי הזה אז אעברה דרך ארצכם לאספמיא׃
\par 29 ויודע אנכי כי בבאי אליכם אבוא במלא ברכתה של־בשורת המשיח׃
\par 30 ואני מעורר אתכם אחי באדנינו ישוע המשיח ובאהבת הרוח להתחזק עמי בהעתירכם בעדי אל־האלהים׃
\par 31 למען אשר אנצל מהסוררים בארץ יהודה ויערב על־הקדושים שמושי לשם ירושלים׃
\par 32 ואשר אבוא אליכם בשמחה ברצון אלהים ואנפש עמכם׃
\par 33 ואלהי השלום עם־כלכם אמן׃

\chapter{16}

\par 1 והנני מזכיר לכם לטוב את־פובי אחותינו שהיא משמשת הקהלה אשר בקנכרי׃
\par 2 אשר תקבלוה באדנינו כראוי לקדושים ותתמכו בה לכל אשר־תצטרך לכם כי־גם־היא היתה עזרת לרבים וגם לנפשי׃
\par 3 שאלו לשלום פריסקלא ועקילס שהם חברי בעבודת המשיח ישוע׃
\par 4 אשר נתנו את־צוארם בעד נפשי ולא־אני לבדי אודה להם כי גם־כל־קהלות הגוים וגם לקהלה בביתם תשאלו לשלום׃
\par 5 שאלו לשלום אפינטוס חביבי שהוא ראשית אסיא למשיח׃
\par 6 שאלו לשלום מרים שעמלה עמל רב בעבורכם׃
\par 7 שאלו לשלום אנדרוניקוס ויוניס קרובי ואשר היו אסורים אתי ולהם שם בשליחים ולפני היו במשיח׃
\par 8 שאלו לשלום אמפליאס חביבי באדנינו׃
\par 9 שאלו לשלום אורבנוס חברנו בעבודת המשיח ולשלום אסטכיס חביבי׃
\par 10 שאלו לשלום אפליס הבחון במשיח שאלו לשלום בני־ביתו של־אריסטובלוס׃
\par 11 שאלו לשלום הורודיון קרובי שאלו לשלום בני־ביתו של־נרקיסוס אשר־הם באדנינו׃
\par 12 שאלו לשלום טרופינה וטרופסה העמלות באדנינו שאלו לשלום פרסיס החביבה שעמלה עמל רב באדנינו׃
\par 13 שאלו לשלום רופוס הנבחר באדנינו ולשלום אמו שהיא כאם לי׃
\par 14 שאלו לשלום אסונקריטוס ופליגון והרמס ופטרובס והרמיס והאחים אשר אתם׃
\par 15 שאלו לשלום פילולוגוס ויוליא נירוס ואחותו ואולומפס וכל־הקדושים אשר אתם׃
\par 16 שאלו איש לרעהו לשלום בנשיקה הקדושה קהלות המשיח שאלות לשלומכם׃
\par 17 ואני מזהיר אתכם אחי לשום פניכם במשלחי מדנים ומכשולים שלא כלקח אשר למדתם וסורו מהם׃
\par 18 כי אנשים כאלה אינם עבדים את־אדנינו ישוע המשיח כי אם־את־כרשם ובאמרי נעם ושפת חלקות יתעו את־לב הפתאים׃
\par 19 כי משמעתכם נודעת לכל לכן אני שמח עליכם אבל רצוני שתהיו חכמים להטיב ותמימים לבלתי הרע׃
\par 20 ואלהי השלום הוא ידכא את־השטן במהרה תחת רגליכם חסד אדנינו ישוע המשיח עמכם׃
\par 21 טימותיוס חברי ולוקיוס ויסון וסוספטרוס קרובי שאלים לשלומכם׃
\par 22 אני טרטיוס הכותב את־האגרת הזאת שאל לשלומכם באדנינו׃
\par 23 גיוס המארח אותי ואת כל־הקהלה שאל לשלומכם ארסטוס סכן העיר וקורטוס אחינו שאלים לשלומכם׃
\par 24 חסד אדנינו ישוע המשיח עם־כלכם אמן׃
\par 25 ואשר יכל לחזק אתכם כבשורתי וכקריאת ישוע המשיח כפי גלוי הסוד אשר־היה מכסה מימות עולם׃
\par 26 ועתה נתפרסם ונודע על־ידי כתבי הנביאים כמצות אלהי עולם לכל הגוים להביאם למשמעת האמונה׃
\par 27 האלהים החכם האחד לו הכבוד בישוע המשיח לעולמים אמן׃


\end{document}