\begin{document}

\title{האיגרת הראשונה אל הקורינתים}


\chapter{1}

\par 1 פולוס שליח מקרא של־ישוע המשיח ברצון אלהים וסוסתניס אחינו׃
\par 2 אל־קהלת אלהים אשר בקורנתוס למקדשים במשיח ישוע הקדשים הקרואים עם כל־הקראים בשם אדנינו ישוע המשיח בכל־מקום שלהם ושלנו׃
\par 3 חסד לכם ושלום מאת האלהים אבינו ואדנינו ישוע המשיח׃
\par 4 אודה לאלהי בעבורכם בכל־עת על־חסד האלהים הנתן לכם במשיח ישוע׃
\par 5 אשר עשרתם־בו בכל בדבור ובדעת׃
\par 6 באשר עדות המשיח התחזקה בכם׃
\par 7 עד אשר־לא חסרתם כל־מתן החסד והנכם מחכים להתגלות אדנינו ישוע המשיח׃
\par 8 והוא גם־יכונן אתכם עד־עת קץ להיות נקיים ביום אדנינו ישוע המשיח׃
\par 9 נאמן האלהים אשר על־פיו נקראתם לחברת בנו ישוע המשיח אדנינו׃
\par 10 והנני מזהיר אתכם אחי בשם אדנינו ישוע המשיח להיות כלכם פה אחד ולא תהיינה מחלקות ביניכם כי אם־תכוננו יחד בלב אחד ובעצה אחת׃
\par 11 כי בני־בית כלואה הגידו לי עליכם אחי כי מריבות ביניכם׃
\par 12 וזאת אני אמר מפני שאיש איש מכם אמר אני לפולוס ואני לאפולוס ואני לכיפא ואני למשיח הנני׃
\par 13 הכי חלק המשיח הכי פולוס נצלב בעדכם אם לשם פולוס נטבלתם׃
\par 14 אודה לאלהים שלא טבלתי איש מכם בלתי קרספוס וגיוס׃
\par 15 פן־יאמרו כי־לשמי טבלתי׃
\par 16 וטבלתי גם את־בני־בית אסטפנוס ומלבד אלה לא אדע אם־טבלתי איש אחר׃
\par 17 כי לא שלחני המשיח לטבול כי אם־לבשר לא־בחכמת דברים למען אשר לא־יהיה צלוב המשיח לריק׃
\par 18 כי־דבר הצלוב סכלות הוא לאבדים אבל לנו הנושעים גבורת אלהים׃
\par 19 כי־כן כתוב אאבד חכמת חכמים ובינת נבונים אסתיר׃
\par 20 איה חכם איה ספר איה דרש העולם הזה הלא סכל האלהים את־חכמת העולם הזה׃
\par 21 כי אחרי אשר בחכמת האלהים לא ידע העולם את־האלהים בחכמה היה רצון מלפניו להושיע בסכלות הקריאה את המאמינים׃
\par 22 כי היהודים שאלים להם אות והיונים מבקשים חכמה׃
\par 23 ואנחנו משמיעים את־המשיח הצלוב מכשול ליהודים וסכלות ליונים׃
\par 24 אבל למקראים בין מיהודים בין מיונים את־המשיח אשר הוא גבורת אלהים וחכמת אלהים׃
\par 25 יען כי סכלות האל חכמה היא מאדם וחלשת האל חזקה היא מאדם׃
\par 26 כי ראו־נא אחי את־קריאתכם שלא רבים החכמים מן־הבשר לא רבים השליטים לא רבים רמי היחש׃
\par 27 כי אם־בסכל שבעולם בחר האל למען ביש את־החכמים ובחלוש שבעולם בחר האל למען ביש את־החזק׃
\par 28 ובדלת העולם בחר האל ובנמאס ובאשר כאין למען בטל את אשר ישנו׃
\par 29 כדי שלא־יתהלל לפניו כל־בשר׃
\par 30 וממנו אתם במשיח ישוע אשר היה לנו לחכמה מאת האלהים ולצדקה ולקדשה ולפדיום׃
\par 31 ויהי ככתוב המתהלל יתהלל ביהוה׃

\chapter{2}

\par 1 וגם אנכי בבאי אליכם אחי לא באתי בגאות הדבור והחכמה להגיד לכם את־עדות האלהים׃
\par 2 כי לא־אמרתי לדעת בתוככם דבר בלתי אם־ישוע המשיח והוא הנצלב׃
\par 3 ואהי עמכם בחלשה וביראה ובחלחלה רבה׃
\par 4 ודברי וקריאתי לא לפתות באמרי חכמת בני־אדם כי אם־בתוכחת הרוח והגבורה׃
\par 5 למען אשר לא־תהיה אמונתכם בחכמת בני אדם כי אם־בגבורת אלהים׃
\par 6 אבל חכמה אנחנו מדברים בקרב השלמים לא חכמת העולם הזה גם־לא של־שרי העולם הזה אשר יאבדו׃
\par 7 כי אם־נדבר בסוד חכמת האלהים הנסתרה אשר האלהים יעדה לכבודנו לפני ימות עולם׃
\par 8 אשר לא ידעה איש משרי העולם הזה כי אלו ידעוה לא צלבו את־אדון הכבוד׃
\par 9 כי אם־ככתוב אשר־עין לא ראתה ואזן לא שמעה ולא עלה על־לב אדם את אשר־הכין האלהים לאהביו׃
\par 10 ולנו גלה האלהים ברוחו כי הרוח חוקר את־הכל גם את־מעמקי האלהים׃
\par 11 כי מי הוא מבני אדם ידע את אשר באדם כי אם־רוח האדם אשר בקרבו וכן אין איש ידע את אשר באלהים כי אם־רוח האלהים׃
\par 12 ואנחנו לא לקחנו את־רוח העולם כי אם־הרוח מאת האלהים למען נדע את־אשר נתן לנו מאת האלהים בחסדו׃
\par 13 ואת־זאת לא נוציא בלשון אשר תלמד חכמת בני אדם כי אם־בלשון אשר רוח הקדש תלמד ונבאר רוחניות בדברים רוחניים׃
\par 14 הן האדם הטבעי איננו מקבל את־דברי רוח אלהים כי־המה סכלות לו ולא יוכל להבינם באשר המה נדונים בדרך הרוח׃
\par 15 אבל האדם הרוחני ידין הכל ואותו לא־ידין איש׃
\par 16 כי מי־תכן את־רוח יהוה ומי יודיענו ואנחנו הנה יש־לנו רוח המשיח׃

\chapter{3}

\par 1 ואני לא יכלתי לדבר עמכם אחי כדבר עם אנשי רוח כי אם כדבר עם אנשי בשר כדבר עם עוללים במשיח׃
\par 2 חלב השקיתי אתכם ולא־מאכל כי עדין לא הייתם יכולים וגם־עתה אינכם יכולים כי עודכם של־הבשר׃
\par 3 כי באשר קנאה ביניכם ומריבה ומחלקות הלא של־הבשר אתם ונהגים מנהג בני אדם׃
\par 4 הן באמר זה אני לפולוס וזה אני לאפולוס הלא של־הבשר אתם׃
\par 5 מי אפוא פולוס ומי־הוא אפולוס אך־משרתים אשר על־ידם באתם להאמין איש איש כמתנת האדון אשר נתן לו׃
\par 6 אני נטעתי ואפולוס השקה והאלהים הוא הצמיח׃
\par 7 על־כן הנטע כאין והמשקה כאין כי אם־אלהים המצמיח׃
\par 8 והנטע והמשקה כאחד המה וכל־איש יקבל את־שכרו כפי עמלו׃
\par 9 כי עזרים לאלהים במלאכתו אנחנו שדה אלהים אתם בנין אלהים אתם׃
\par 10 ואני כפי חסד אלהים הנתן לי כאמן חכם שתי יסוד ואחר בונה עליו אך־ירא כל־איש איך הוא־בנה עליו׃
\par 11 כי לא־יוכל איש לשית יסוד אחר זולתי המוסד שהוא ישוע המשיח׃
\par 12 ואם־יבנה הבונה על־היסוד הזה זהב או כסף או אבנים יקרות או־עץ או חציר או קש׃
\par 13 יגלה מעשה כל־איש כי־היום הוא יבררהו כי־באש יראה ומה־מעשה כל־איש ואיש האש תבחננו׃
\par 14 אם־יעמד מעשה איש אשר בנה עליו יקבל שכרו׃
\par 15 ואם־ישרף מעשהו יפסידנו והוא יושע אך כאוד מצל מאש׃
\par 16 הלא ידעתם כי היכל אלהים אתם ורוח אלהים שכן בקרבכם׃
\par 17 ואיש אשר ישחית את־היכל אלהים האלהים ישחית אתו כי היכל אלהים קדוש ואתם היכלו׃
\par 18 אל־ישיא איש נפשו מי אשר חכם יחשב בעולם הזה יהי לסכל למען יחכם׃
\par 19 כי־חכמת העולם הזה סכלות לפני האלהים ככתוב לכד חכמים בערמם׃
\par 20 ועוד כתוב יהוה ידע מחשבות חכמים כי המה הבל׃
\par 21 על־כן אל־יתהלל איש באדם כי הכל הוא שלכם׃
\par 22 אם־פולוס אם־אפולוס ואם־כיפא אם־העולם אם־החיים ואם־המות אם־ההוה ואם־העתיד הכל הוא שלכם׃
\par 23 ואתם של־המשיח והמשיח של־אלהים׃

\chapter{4}

\par 1 כמשרתי המשיח וסכני רזי אל כן יחשב־איש אתנו׃
\par 2 ולזה עוד יבקש מן־הסכנים להמצא נאמן׃
\par 3 ואני נקלה היא בעיני שאתם דנים אתי או־יום דין של־בני אדם גם־אני לא אדין נפשי׃
\par 4 כי אינני יודע בנפשי מאומה רע ובכל־זאת לא אצדק כי האדון הוא הדן אתי׃
\par 5 על־כן אל־תשפטו דבר לפני עתו עד כי־יבוא האדון והוא יוציא לאור את־תעלמות החשך ויגלה את־מזמות הלבבות ואז תהיה תהלה לכל־איש מאת האלהים׃
\par 6 ואת־זאת אחי הסבתי על־עצמי ועל־אפולוס בעבורכם למען תלמדו בנו שלא־יתגדל איש על־מה־שכתוב פן־תתגאו איש בשם איש לנגד רעהו׃
\par 7 כי מי הפליא אותך ומה בידך ולא נתן לך ואם־נתן לך למה תתהלל כמי שלא נתן לו׃
\par 8 הן כבר שבעתם כבר עשרתם ובלעדינו מלכתם ולו מלכתם למען נמלך אתכם גם־אנחנו׃
\par 9 כי אמר אני שהאלהים הציג אתנו השליחים שפלי השפלים כבני־תמותה כי היינו לראוה לעולם גם־למלאכים גם־לבני אדם׃
\par 10 אנחנו סכלים למען המשיח ואתם חכמים במשיח אנחנו חלשים ואתם גבורים אתם נכבדים ואנחנו נקלים׃
\par 11 ועד־השעה הזאת הננו רעבים גם־צמאים וערמים ומכים באגרף ואין מנוח לנו׃
\par 12 ויגעים אנחנו בעמל ידינו מקללים אותנו ונברך מחרפים אתנו ונסבל׃
\par 13 גדפו אתנו ונתחנן ונהי כגללי העולם ולסחי לכלם עד־היום הזה׃
\par 14 ולא כתבתי הדברים האלה לביש אתכם כי אם־מזהיר אני אתכם כבני האהובים׃
\par 15 כי גם־אם־היו לכם רבבות אמנים במשיח אין לכם אבות רבים כי אנכי ילדתי אתכם בישוע המשיח על־ידי הבשורה׃
\par 16 על־כן אני מבקש מכם ללכת בעקבותי׃
\par 17 ובעבור זאת שלחתי אליכם את־טימותיוס בני האהוב והנאמן באדון והוא יזכיר לכם את־דרכי במשיח כאשר מלמד אנכי בכל־מקום בכל־קהלה וקהלה׃
\par 18 הן־יש מתנשאים כאלו לא־אבוא אליכם׃
\par 19 אבל בוא אבוא אליכם בזמן קרוב אם־ירצה יהוה ולא את־דברי המתגאים אדעה כי אם־את־גבורתם׃
\par 20 כי לא בדבר שפתים תכן מלכות האלהים כי אם־בגבורה׃
\par 21 ומה־אתם רוצים האבוא אליכם בשבט אם באהבה וברוח ענוה׃

\chapter{5}

\par 1 הקול נשמע בכל־מקום שזנות ביניכם וזנות אשר אין כמוה בגוים עד־שיקח איש את־אשת אביו׃
\par 2 ואתם מתגאים תחת אשר תתאבלו להסיר מקרבכם עשה המעשה הזה׃
\par 3 ואנכי הרחוק מכם בגופי וקרוב ברוחי כבר דנתי כאלו הייתי אצלכם על־האיש אשר־עשה כדבר הזה׃
\par 4 בשם אדנינו ישוע המשיח בהקהלכם יחד ורוחי אתכם עם־גבורת אדנינו ישוע המשיח׃
\par 5 למסר את־האיש ההוא לשטן לאבד את־הבשר למען יושע הרוח ביום האדון ישוע׃
\par 6 לא־טוב התהללכם הלא ידעתם כי מעט שאר מחמץ את־כל־העסה׃
\par 7 בערו את־השאור הישן למען תהיו עסה חדשה הלא לחם מצות אתם כי גם־לנו פסחנו הנזבח בעדנו הוא המשיח׃
\par 8 על־כן נחגה־נא החג לא־בשאר ישן ולא־בשאר רעה ורשע כי אם־במצות התם והאמת׃
\par 9 כתבתי לכם באגרת שלא תתערבו עם־הזנים׃
\par 10 ואין־דעתי על־הזנים בעולם או על־בצעי בצע וגזלנים ועבדי אלילים כי אם־כן סופכם לצאת מן־העולם׃
\par 11 אך־זאת כתבתי לכם לבלתי התערב עם־מי שנקרא אח והוא זנה או־בצע בצע או־עבד אלילים או מגדף או סבא או גזלן ואף לא לאכל עם־האיש אשר כזה׃
\par 12 כי מה־לי לשפט את־אשר בחוץ הלא תשפטו את אשר בבית׃
\par 13 ואשר בחוץ האלהים ישפטם ואתם תבערו את־הרע מקרבכם׃

\chapter{6}

\par 1 היריב איש מכם עם־רעהו ויזיד להביא דינו לפני הרשעים ולא לפני הקדשים׃
\par 2 הלא ידעתם כי הקדשים ידינו את־העולם ואם־העולם ידון על־ידכם הלא ראוים אתם לדין דינים קלים׃
\par 3 הלא ידעתם כי נדין דין־המלאכים אף כי־דיני ממונות׃
\par 4 ואתם כשיש־לכם דיני ממונות מושיבים אתם את־הנמאס בקהל לשפטים עליכם׃
\par 5 לבשתכם אני אמר את־זאת הכי אין בכם חכם ידע להוכיח בין איש לאחיו׃
\par 6 כי אח בא לדין עם־אחיו ובא לפני בלי־מאמינים׃
\par 7 אף־זאת ירידה היא לכם שתריבו זה עם־זה ולמה לא תבחרו להיות מן־העלובים ואינם עולבים ומן־העשוקים ואינם עושקים׃
\par 8 אבל עולבים אתם ועשקים אף את־אחיכם׃
\par 9 הלא ידעתם כי הרשעים לא יירשו את־מלכות האלהים אל־תשיאו נפשותיכם לא הזנים לא עבדי אלילים לא המנאפים ולא הקדשים ולא השכבים את־זכר׃
\par 10 לא הגנבים ולא־בצעי בצע לא הסבאים ולא המגדפים ולא הגזלנים כל־אלה לא יירשו את־מלכות האלהים׃
\par 11 וכאלה לפנים היו מקצתכם אבל רחצתם אבל קדשתם אבל הצדקתם בשם האדון ישוע וברוח אלהינו׃
\par 12 הכל רשות לי אבל לא כל־דבר מועיל הכל רשות לי אבל לא ישעבדני דבר׃
\par 13 המאכל לכרש והכרש למאכל והאלהים את־זה ואת־זה יכלה והגוף אל־יהי לזנות כי אם־לאדון והאדון לגוף׃
\par 14 והאלהים העיר גם את־אדנינו ויעיר גם־אתכם בגבורתו׃
\par 15 הלא ידעתם כי גופתיכם אברי המשיח המה האקח את־אברי המשיח ואעשה אתם לאברי זונה חלילה׃
\par 16 או הלא ידעתם כי הדבק בזונה גוף אחד הוא עמה כי הכתוב אמר והיו שניהם לבשר אחד׃
\par 17 אבל הדבק באדון רוח אחד הוא עמו׃
\par 18 רחקו מן־הזנות כל־חטא אשר־יחטא האדם מחוץ לגופו הוא והזונה חטא בעצם גופו׃
\par 19 או הלא־ידעתם כי גופכם הוא היכל רוח הקדש השכן בקרבכם אשר היה לכם מאת האלהים ולא־שלכם אתם׃
\par 20 כי במחיר נקניתם על־כן כבדו את־האלהים בגופכם (וברוחכם אשר לאלהים המה)׃

\chapter{7}

\par 1 ולענין אשר כתבתם אלי הנה טוב לאדם שלא יגע באשה׃
\par 2 אך מפני הזנות תהי לכל־איש אשתו ויהי לכל־אשה בעלה׃
\par 3 האיש יהי יוצא ידי חובתו עם אשתו וכמו־כן האשה עם בעלה׃
\par 4 האשה אין גופה ברשותה אלא ברשות בעלה וכמו־כן האיש אין גופו ברשותו אלא ברשות אשתו׃
\par 5 אל־תפרדו זה מזה כי אם מדעת שניכם לפי שעה לעמוד (בתענית ו) בתפלה ותשובו ותתאחדו פן־ינסה אתכם השטן בפריצותכם׃
\par 6 ואני אמר זאת בדרך רשות ולא בדרך מצוה׃
\par 7 כי מי יתן והיה כל־אדם כמני אבל כל־אדם יש־לו מתנתו מאת האלהים זה בכה וזה בכה׃
\par 8 ואל־הפנוים ואל־האלמנות אמר אני כי־טוב להם לעמד ככה כמו גם־אני׃
\par 9 אך אם־לא יוכלו לכבש את־יצרם ישאו כי־טובים נשואין מאיש להוט אחר עברה׃
\par 10 ועל־הנשואים אני מצוה ולא מעצמי כי אם־מדעת האדון שלא־תפרש אשה מבעלה׃
\par 11 ואם־פרש תפרש ממנו תשב בלא איש או תתרצה לבעלה ואיש אל־ישלח את־אשתו׃
\par 12 ואל־האחרים אמר אני שלא מדעת האדון כי־תהיה לאח אשה אשר איננה מאמינה ורצונה שתעמד עמו אל־ישלחנה׃
\par 13 ואשת איש אשר איננו מאמין ורצונו שישב עמה אל־תעזבנו׃
\par 14 כי האיש אשר איננו מאמין יקדש באשה והאשה אשר איננה מאמינה תקדש באיש שאם־לא כן הדבר בניכם טמאים ועתה קדושים המה׃
\par 15 ומי שאיננו מאמין אם בא לפרש יפרש והאח או האחות אינם זקוקים לאלה ואנחנו לשלום קראנו האלהים׃
\par 16 כי מה־תדעי את האשה אם־תושיעי את־האיש ומה־תדע אתה האיש אם־תושיע את־האשה׃
\par 17 רק יתהלך כל־איש כפי מה־שחלק לו האלהים וכפי מה־שקרא אתו האדון וכן־מתקן אני בכל־הקהלות׃
\par 18 אם־נמול המקרא אל־ימשך לו ערלה ואם־ערל הוא אל־ימול׃
\par 19 אין־המילה נחשבה ואין־הערלה נחשבה כי אם־לשמר מצות האלהים׃
\par 20 איש איש במשמרתו שמתוכה נקרא בה יעמד׃
\par 21 אם־נקראת ואתה עבד על־ירע בעינך אלא אם־תשיג ידך לצאת לחפשי בחר בזה׃
\par 22 כי־הקרוא באדון בהיותו עבד משחרר הוא לאדון וכן הקרוא בהיותו חפשי עבד הוא למשיח׃
\par 23 במחיר נקניתם אל־תהיו עבדים לבני־אדם׃
\par 24 אחי איש איש במשמרת שמתוכה נקרא בה יעמד לפני האלהים׃
\par 25 ועל־דבר הבתולות אין־לי מצוה מפי האדון רק אחוה דעתי אחרי אשר־חנני האדון להיות נאמן׃
\par 26 ואני אמר כי־טוב לאדם מפני הצרה הקרובה כי־טוב לו לעמד כך׃
\par 27 אם־זקוק אתה לאשה אל־תבקש להפטר ואם־נפטרת אל־תבקש אשה׃
\par 28 וגם כי־תקח אשה אין בך חטא והבתולה כי־תהיה לאיש אין בה חטא אבל יבאום צרות בבשרם ואני חס עליכם׃
\par 29 וזאת אני אמר אחי כי השעה דחוקה מעתה על־כן יהיו הנשואים כאלו אין־להם נשים׃
\par 30 והבכים כאינם בכים והשמחים כאינם שמחים והקונים כאלו אין־קנין בידם׃
\par 31 והנהנים מן־העולם הזה כאלו אין להם הנאה ממנו כי תעבר צורת העולם הזה׃
\par 32 ואני רצוני שלא תהיו נטרדים מי שאין־לו אשה טרוד באשר לאדון איך ייטב בעיני האדון׃
\par 33 ובעל אשה טרוד בחפצי העולם איך ייטב בעיני האשה ואין לבו תמים׃
\par 34 ואשה פנויה ובתולה טרודה באשר לאדון ולהיות קדושה גם בגופה גם ברוחה ובעולת בעל טרודה היא בחפצי העולם שתיטב בעיני בעלה׃
\par 35 וכן אני אמר לטוב לכם ולא להשליך פח עליכם כי אם־להנהגה טובה ולמען תהיו נכונים תמיד לקראת האדון באין מעצור׃
\par 36 וכי־יאמר איש שהוא עשה בבתו הבתולה שלא כהגן אם־יעבר עליה פרקה ודבר צרך הוא אז יעשה לה כרצונו אין בו חטא ישיאנה׃
\par 37 ומי שהוא נכון בלבו ואיננו מכרח כי אם־יכל לעשות כרצונו וגמר בלבו לשמר את בתו הבתולה טוב הוא עשה׃
\par 38 לכן המשיא אתה עשה טוב ואשר איננו משיא עשה טוב ממנו׃
\par 39 האשה זקוקה לבעלה מן־התורה כל־זמן שהוא חי וכשמת בעלה מתרת היא להנשא למי שתרצה ובלבד שתהיה באדון׃
\par 40 ואשריה אם־תעמד פנויה זאת דעתי ואמר כי רוח אלהים גם־בי׃

\chapter{8}

\par 1 ועל־דבר זבחי האלילים ידענו שכלנו יש־לנו דעת הדעת תגביה לב והאהבה היא הבונה׃
\par 2 האמר שהוא ידע דבר עודנו לא־ידע כאשר ראוי לו׃
\par 3 אבל האהב את־האלהים הוא נודע לו׃
\par 4 ועל־דבר אכילת זבחי האלילים ידענו כי־אין אליל בעולם ואין אלהים בלתי אחד׃
\par 5 ואף כי־יש מי שנקראים אלהים בין־בשמים בין־בארץ כאשר יש אלהים רבים ואדנים רבים׃
\par 6 אמנם לנו רק־אל אחד האב אשר הכל ממנו ואנחנו אליו ואדון אחד ישוע המשיח אשר הכל על־ידו ואנחנו על־ידו׃
\par 7 אך לא בכלם הדעת כי יש זכרים עוד את־האליל ואכלים כאכל זבח אליל ולבם החלוש יתגאל׃
\par 8 והמאכל לא יקרב אתנו לאלהים כי אם־נאכל אין־לנו יתרון ואם־לא נאכל לא נגרע׃
\par 9 אבל הזהרו פן־יהיה אתו הרשיון שלכם למכשל לחלשים׃
\par 10 כי הראה אתך אשר לך הדעת מסב בבית אלילים הלא החלש יעז ברוחו לאכל מזבחי אלילים׃
\par 11 ויאבד על־ידי דעתך אחיך החלש אשר למענו מת המשיח׃
\par 12 ואם־ככה תחטאו לאחיכם ותכאיבו את־רוחם החלוש למשיח אתם חטאים׃
\par 13 על־כן אם־מאכלי מכשיל את־אחי לא־אכל בשר לעולם פן־אכשיל את־אחי׃

\chapter{9}

\par 1 טהלא שליח אנכי הלא חפשי אנכי הלא ראיתי את־ישוע המשיח אדנינו הלא פעלי אתם באדנינו׃
\par 2 ואם־אינני שליח לאחרים לכם שליח אני כי חותם שליחותי אתם באדנינו׃
\par 3 וכנגד הדנים אותי אני אמר׃
\par 4 האין רשות בידנו לאכל ולשתות׃
\par 5 האין רשות בידנו להוליך עמנו אחות לאשה כשליחים האחרים וכאחי האדון וכמו כיפא׃
\par 6 אם־לי לבדי ולבר־נבא לא נתנה רשות לחדל לעשות מלאכה׃
\par 7 מי יצא בצבא ופרנסתו עליו מי נטע כרם ולא יאכל את־פריו מי רעה עדר ומחלב העדר לא יאכל׃
\par 8 הכדרך בני אדם אני מדבר כזאת הלא גם־התורה אמרת כן׃
\par 9 כי כתוב בתורת משה לא־תחסם שור בדישו׃
\par 10 הלשורים חושש האלהים או רק־למעננו מדבר אכן למעננו נכתב כי החרש יחרש אלי־תקוה והדש ידוש אלי־תקוה לקחת חלקו בתקוה׃
\par 11 אם־זרענו בכם עניני הרוח הדבר גדול הוא שנקצר מכם עניני הבשר׃
\par 12 ואם לאחרים יש רשות עליכם הלא יותר לנו אבל לא עשינו כרשות הזאת כי אם־סבלנו את־הכל לבלתי־שום מעצר לבשורת המשיח׃
\par 13 הלא ידעתם כי עבדי עבדת הקדש אכלים מן־הקדשים ומשרתי המזבח לקחים חלקם במזבח׃
\par 14 כן תקן אדנינו גם־הוא שיחיו המבשרים מן־הבשורה׃
\par 15 ואנכי לא עשיתי כאחת מאלה וגם לא־כתבתי זאת למען יעשה־לי כן כי־טוב לי המות מאשר ישים איש את־תפארתי לריק׃
\par 16 אם־אבשר את־הבשורה אין־לי להתפאר כי־החובה מטלת עלי ואוי לי אם־לא אבשר׃
\par 17 כי אם־ברצוני אעשה יהיה־לי שכר ואם־שלא ברצוני פקדת משמרתי היא׃
\par 18 ועתה מה־שכרי הלא שאבשר בשורת המשיח בלא־מחיר לבלתי השתמש להנאת עצמי ברשות הנתנה--לי בבשורה׃
\par 19 כי בהיותי חפשי מכל עשיתי עצמי עבד לכל־אדם לקנות את־הרבים׃
\par 20 ואהי ליהודים כיהודי לקנות היהודים אשר הם תחת התורה להם הייתי כמי שתחת התורה אף כי־אני אינני תחת התורה למען קנות אותם שהם תחת התורה׃
\par 21 לאתם שאין להם תורה הייתי כמי שאין לו תורה אף על־פי שאינני בלא־תורת אלהים כי־תורת המשיח תורתי למען קנות אותם שאין להם תורה׃
\par 22 ולחלשים הייתי כחלש לקנות את־החלשים הכל לכלם נהייתי למען אושיע אחדים על כל־פנים׃
\par 23 ואת־זאת אני עשה בעבור הבשורה למען יהיה חלקי בה׃
\par 24 הלא ידעתם כי־הרצים באצטדיון כלם רצים ואחד יזכה בשכר הנצחון כן רוצו למען תזכו בו׃
\par 25 וכל־העמד להתגושש ינזר מכל־דבר המה לקחת כתר נפסד ואנחנו לקחת כתר אשר איננו נפסד׃
\par 26 לכן הנני רץ לא כמו בחשכה הנני נלחם לא כהולם רוח׃
\par 27 כי אם־אדכא את־גופי ואשעבדנו שלא־אהיה אני הקורא לאחרים נאלח בעצמי׃

\chapter{10}

\par 1 ולא אכחד מכם אחי שאבותינו היו כלם תחת הענן וכלם עברו בתוך הים׃
\par 2 וכלם נטבלו למשה בענן ובים׃
\par 3 וכלם אכלו מאכל אחד רוחני׃
\par 4 וכלם שתו משקה אחד רוחני כי שתו מן־הצור הרוחני ההלך עמהם והצור ההוא המשיח׃
\par 5 אבל רבם לא רצה בם האלהים ופגריהם נפלו במדבר׃
\par 6 וכל־זאת היתה־לנו למופת לבלתי התאות לרעה כאשר התאוו גם־המה׃
\par 7 ולא תהיו עבדי אלילים כאשר היו מקצתם כמו שכתוב וישב העם לאכל ושתו ויקמו לצחק׃
\par 8 ולא־נהיה זנים כאשר זנו מקצתם ויפלו ביום אחד שלשה ועשרים אלף איש׃
\par 9 ולא־ננסה את־יהוה כאשר נסוהו מקצתם ויאבדום הנחשים׃
\par 10 גם־לא תלינו כאשר הלינו מקצתם וימותו ביד המשחית׃
\par 11 כל־זאת מצאתם להיות למופת ותכתב למוסר לנו אשר־הגיעו אלינו קצי עולמים׃
\par 12 לכן האמר בנפשו אני עמד ירא פן־יפול׃
\par 13 עדין לא־בא עליכם נסיון בלתי כדרך בני־אדם כי־נאמן הוא האלהים אשר לא יניח לנסות אתכם יותר על כחכם כי אם־יתן עם־הנסיון גם־אחריתו כדי שתוכלו שאת׃
\par 14 על־כן חביבי רחקו מעבודת אלילים׃
\par 15 כדבר אל־נבונים אני מדבר ואתם בינו את אשר אמר׃
\par 16 כוס של־ברכה אשר אנחנו מברכים עליו הלא הוא מחבר אתנו לדמו של־המשיח והלחם אשר אנחנו בצעים הלא הוא מחבר אתנו לגופו של־המשיח׃
\par 17 כי־לחם אחד הוא לכן גוף אחד אנחנו הרבים מפני שחלק לכלנו בלחם האחד׃
\par 18 הביטו אל־ישראל שלפי הבשר הלא אכלי הזבחים חברי המזבח המה׃
\par 19 ועתה מה אמר היש ממש באליל אם־יש ממש בזבחי אלילים׃
\par 20 אלא מה־שיזבחו הגוים לשדים הם זבחים ולא לאלהים ואני אין רצוני שתהיו חברים לשדים׃
\par 21 לא תוכלו לשתות כוס אדנינו וכוס השדים יחד ולא יהיה חלק לכם בשלחן אדנינו ובשלחן השדים׃
\par 22 הנעז להקניא את־אדנינו הכי חזקים אנחנו ממנו׃
\par 23 הכל רשות לי אבל לא כל־דבר מועיל הכל רשות לי אבל לא כל־דבר בנה׃
\par 24 איש אל־יבקש דבר לעצמו כי אם־לרעהו׃
\par 25 כל־הנמכר בשוק אתו תאכלו ואל־תחקרו מפני מכשל הלב׃
\par 26 כי ליהוה הארץ ומלואה׃
\par 27 ואם־יקרא אתכם איש מאשר אינם מאמינים ורצונכם ללכת אליו אכול תאכלו מכל אשר־ישימו לפניכם ואל־תחקרו מפני מכשול הלב׃
\par 28 וכי־יאמר לכם איש זה הוא זבח אלילים אל־תאכלו מפני אתו המודיע ומפני מכשול הלב (כי ליהוה הארץ ומלואה)׃
\par 29 והלב שאני אמר לא לבך כי אם־לב רעך כי למה־זה תדון חרותי על־ידי לב האחר׃
\par 30 ואם־אכל אני בברכה למה יצא לי שם רע על־הדבר שאני מברך עליו׃
\par 31 לכן אם תאכלו ואם־תשתו או־תעשו דבר עשו הכל לכבוד אלהים׃
\par 32 ואל־תתנו מכשל לא ליהודים ולא ליונים ולא לקהלת אלהים׃
\par 33 כאשר גם־אנכי מבקש להיות רצוי לכל בכל ולא אבקש תועלת עצמי כי־אם תועלת הרבים למען יושעו׃

\chapter{11}

\par 1 לכו בעקבותי כאשר גם־אני הלך בעקבות המשיח׃
\par 2 ועל־זאת אני משבח אתכם אחי שזכרתם אתי בכל לשמר את־הקבלות כאשר מסרתי לכם׃
\par 3 ורצוני שתהיו ידעים שראש כל־איש המשיח וראש האשה האיש וראש המשיח הוא האלהים׃
\par 4 כל־איש אשר יתפלל או יתנבא וראשו מכסה מנול הוא את־ראשו׃
\par 5 וכל־אשה אשר תתפלל או תתנבא וראשה פרוע את־ראשה היא מנולת כי שוה היא למגלחה׃
\par 6 כי האשה אם־לא תתכסה גם תתגלח ואם־בזיון הוא לאשה לגז או לגלח את־שערה תתכסה׃
\par 7 אמנם האיש איננו חיב לכסות את־ראשו כי הוא צלם אלהים וכבודו והאשה היא כבוד האיש׃
\par 8 כי אין־האיש מן־האשה כי אם־האשה מן־האיש׃
\par 9 גם־לא־נברא האיש בעבור האשה כי אם־האשה בעבור האיש׃
\par 10 על־כן האשה חיבת להיות אות משמעתה על־ראשה בעבור המלאכים׃
\par 11 אבל אין האיש בלא אשה ואין האשה בלא איש באדון׃
\par 12 כי כאשר האשה מן־האיש כן גם־האיש על־ידי האשה וכל־אלה מאלהים׃
\par 13 שפטו־נא בנפשכם הנאוה לאשה להתפלל אל־האלהים וראשה מגלה׃
\par 14 והלא תלמדו מנוהג שבעולם כי איש אשר יגדל פרע שער ראשו חרפה היא לו׃
\par 15 אבל האשה כי תגדל שערה פאר הוא לה כי־נתן לה השער לצניף׃
\par 16 ואם־יאהב איש לריב לא זו דרכנו ולא דרך קהלות האלהים׃
\par 17 והנה בצותי את־זאת לא אוכל לשבח אתכם על־אשר תאספו יחד לא לטובה כי אם־לרעה׃
\par 18 כי שמעתי שיש מחלקות ביניכם כשתועדו בקהל ומקצת הדבר אני מאמין׃
\par 19 כי כתות צריכות להיות ביניכם למען יודעו הנאמנים שבכם׃
\par 20 ועתה כאשר תאספו יחד אין־זה לאכל סעודתו של־האדון׃
\par 21 כי כל־אחד מקדים לקחת סעודתו בשעת האכילה וזה ירעב וזה ישתכר׃
\par 22 הכי אין לכם בתים לאכל וזשתות או התבוזו את־קהל אלהים ותבישו את־מי שאין־לו מה אמר לכם העל־זאת אשבח אתכם אינני משבח׃
\par 23 כי־כן קבלתי אני מן־האדון ומסרתי לכם כי האדון ישוע בלילה אשר־נמסר בו לקח את־הלחם׃
\par 24 ויברך ויבצע ויאמר קחו אכלו זה גופי הנבצע בעדכם עשו־זאת לזכרי׃
\par 25 וכמו־כן את־הכוס אחר הסעודה ויאמר הכוס הזאת היא הברית החדשה בדמי עשו־זאת לזכרי בכל־זמן שתשתו׃
\par 26 כי בכל־זמן שתאכלו את־הלחם הזה ותשתו את־הכוס הזאת הזכר תזכירו את־מות אדנינו עד כי יבוא׃
\par 27 לכן מי שיאכל מן־הלחם הזה או־ישתה מכוס האדון שלא כראוי יאשם לגוף אדנינו ולדמו׃
\par 28 יבחן האיש את־נפשו ואז יאכל מן־הלחם וישתה מן־הכוס׃
\par 29 כי האכל והשתה שלא כראוי אכל ושתה דין לנפשו לפי שלא־הפלה את־גוף האדון׃
\par 30 בגלל הדבר הזה יש־בכם חולים וחלשים רבים והרבה ישנו המות׃
\par 31 כי אם־נבחן את־נפשנו לא נהיה נדונים׃
\par 32 וכשאנו נדונים נוסר על־יד האדון כדי שלא נחיב עם־העולם׃
\par 33 על־כן אחי בהועדכם יחד לאכל המתינו זה לזה׃
\par 34 וכי־ירעב איש יאכל בביתו פן־תועדו לאשמה ויתר הדברים אתקן בבאי׃

\chapter{12}

\par 1 ובענין מתנות הרוח אחי לא־אכחד מכם דבר׃
\par 2 הלא ידעתם כי לפנים גוים הייתם ואחרי האלילים האלמים הובלתם וגם נמשכתם׃
\par 3 לכן אודיע אתכם כי אין איש דבר ברוח אלהים ויאמר לישוע חרם וגם לא יקרא איש לישוע אדון בלתי אם־ברוח הקדש׃
\par 4 והמתנות שנות אבל אחד הוא הרוח׃
\par 5 ושנים השמושים ואחד הוא האדון׃
\par 6 והפעלות שנות אבל האלהים אחד והוא הפעל את־הכל בכל׃
\par 7 ולכל־איש ואיש נתנה התגלות הרוח להועיל׃
\par 8 כי האחד נתן־לו על־ידי הרוח דבור החכמה ולאחר דבור הדעת כפי הרוח ההוא׃
\par 9 לאחר האמונה ברוח ההוא ולאחר מתנות הרפאות ברוח ההוא׃
\par 10 ולאחר לפעל גבורות ולאחר נבואה ולאחר להבחין בין הרוחות ולאחר מיני לשנות ולאחר באור לשנות׃
\par 11 וכל־אלה יפעל הרוח האחד ההוא החלק לאיש כרצונו׃
\par 12 כי כאשר הגוף אחד ובו אברים הרבה וכל־אברי הגוף אף כי־רבים הם כלם גוף אחד כן גם המשיח׃
\par 13 כי ברוח אחד נטבלנו כלנו לגוף אחד אם־יהודים ואם־יונים אם־עבדים ואם־בני חורין וכלנו לרוח אחד השקינו׃
\par 14 כי גם־הגוף לא אבר אחד הוא כי אם־רבים׃
\par 15 אם־תאמר הרגל לא יד אני על־כן אינני מן־הגוף הלזאת לא מן־הגוף היא׃
\par 16 ואם־תאמר האזן לא עין אני על־כן אינני מן־הגוף הלזאת לא מן־הגוף היא׃
\par 17 אם־הגוף כלו יהיה עין איה השמע ואם־כלו שמע איה הריח׃
\par 18 ועתה האלהים שת את־האברים כל־אחד ואחד מהם בגוף כפי רצונו׃
\par 19 ואלו־היו כלם אבר אחד איה הגוף׃
\par 20 הנה רבים הם האברים והגוף אחד׃
\par 21 העין לא־תוכל דבר אל־היד לאמר לא אצטרך לך וגם־הראש לא־יוכל דבר אל־הרגלים לאמר לא אצטרך לכן׃
\par 22 כי להפך אברי הגוף הנראים רפים צריכים אנו להם ביותר׃
\par 23 והנראים לנו נקלים בגוף אתם נלביש יתר כבוד ואשר לבשת לנו מרבים אנחנו את עדים׃
\par 24 כי האברים ההגונים אשר בנו אין צרך לתת כבוד להם וכן מזג האלהים את הגוף לתת כבוד יותר לגרוע׃
\par 25 כדי שלא־תהיה מחלקת בגוף כי אם־ידאגו כל־האברים יחד זה לזה׃
\par 26 ואם־יכאב אבר אחד יכאבו אתו כל־האברים ואם־יכבד אבר אחד ישמחו אתו כל־האברים׃
\par 27 ואתם גוף המשיח אתם ואבריו כל־אחד לפי חלקו׃
\par 28 ומהם שם האלהים בקהל ראשונה לשליחים ושנית לנביאים ושלישית למלמדים ויתן גבורות אף־מתנות הרפאות ותמיכות והנהגות ומיני לשנות׃
\par 29 הכלם שליחים אם־כלם נביאים או כלם מלמדים הכלם עשי גבורות׃
\par 30 הלכלם מתנות רפאות הכלם מדברים בלשנות הכלם מבארי לשנות׃
\par 31 ואתם השתדלו להשיג המתנות הטובות ביותר ואני הנני מורה אתכם דרך נעלה על־כלנה׃

\chapter{13}

\par 1 אם־בלשנות אנשים ומלאכים אדבר ואין־בי אהבה הייתי כנחשת המה או כצלצל תרועה׃
\par 2 ואם תהיה־לי נבואה ואדע כל־הסודות וכל־הדעת ואם תהיה־לי אמונה רבה עד להעתיק הרים ממקומם ואין־בי האהבה הייתי כאין׃
\par 3 ואם־אחלק את־כל־הוני ואם־אתן את־גופי לשרפה ואין־בי האהבה כל־זאת לא תועילני׃
\par 4 האהבה מארכת־אף ועשה חסד האהבה לא תקנא האהבה לא תתפאר ולא תתרומם׃
\par 5 לא תעשה דבר־תפלה ולא־תבקש את אשר־לה ולא תתמרמר ולא תחשב הרעה׃
\par 6 לא תשמח בעולה כי עם־האמת תשמח׃
\par 7 את־כל תשא את־כל תאמין את־כל תקוה ואת־כל תסבל׃
\par 8 האהבה לא־תבל לעולם אבל הנבואות תבטלנה והלשנות תכלינה והדעת תבטל׃
\par 9 כי־קצת הוא שידענו וקצת הוא שנבאנו׃
\par 10 וכבוא התמים אז עבור תעבר הקצת׃
\par 11 כאשר הייתי עולל כעולל דברתי כעולל הגיתי כעולל חשבתי וכאשר הייתי לאיש הסירתי דברי העולל׃
\par 12 כי כעת מביטים אנחנו במראה ובחידות ואז פנים אל־פנים׃
\par 13 כעת יודע אני קצתו ואז כאשר נודעתי אדע אף־אני׃ (13:14) ועתה שלש־אלה תעמדנה האמונה והתקוה והאהבה והגדולה שבהן היא האהבה׃

\chapter{14}

\par 1 רדפו אחרי האהבה והשתדלו להשיג מתנות הרוח וביותר שתתנבאו׃
\par 2 כי המדבר בלשון איננו מדבר לאדם כי אם־לאלהים כי אין־איש שמעו וברוח הוא מדבר סודות׃
\par 3 והמתנבא הוא מדבר לבני אדם לבנותם וליסרם ולנחמם׃
\par 4 המדבר בלשון בונה את־נפשו והמתנבא בונה את־העדה׃
\par 5 ומי יתן וכלכם תדברו בלשנות וביותר כי תתנבאו כי גדול המתנבא מן־המדבר בלשנות בלתי אם־יפרש למען תבנה העדה׃
\par 6 ועתה אחי כי־אבוא אליכם ואדבר בלשנות מה־אועיל לכם אם־לא אדבר אליכם בחזון או בדעת או בנבואה או בהוראה׃
\par 7 הלא מה־שאין בו רוח חיים ונתן קול חליל או כנור אם־לא ישמיעו קלות אשר תוכל האזן להבחין איכה יודע מה־יזמר ומה־ינגן׃
\par 8 גם השופר אם־לא יתן קולו קול ברור מי יחלץ למלחמה׃
\par 9 כן גם־אתם אם לא־תוציאו בלשונכם דבור מפרש איכה יודע המדבר הלא תהיו כמדברים לרוח׃
\par 10 הן כמה מיני לשנות יש בעולם ואין־אחת מהן בלי קול׃
\par 11 לכן אם־אינני ידע פשר הקול אהיה כלעז בעיני המדבר והמדבר יהיה כלעז בעיני׃
\par 12 כן גם־אתם לפי שמתאוים אתם לכחות רוחניות בקשו להעדיף במה־שיבנה את־העדה׃
\par 13 על־כן יתפלל המדבר בלשון וגם יפרשנה׃
\par 14 כי אם־אתפלל בלשון רוחי מתפלל ושכלי איננו עשה פרי׃
\par 15 ועתה מה־אעשה אתפללה ברוחי ואתפללה גם־בשכלי אזמרה ברוחי ואזמרה גם־בשכלי׃
\par 16 כי אם־תברך ברוח איך יענה מי שהוא נתון במקום ההדיוטות אמן אחר ברכתך והוא איננו ידע מה אתה אמר׃
\par 17 הן אתה תברך כראוי אבל רעך לא יבנה׃
\par 18 אודה לאלהי שיותר מכלכם אני מדבר בלשנות׃
\par 19 אכן בקהל אבחר לדבר חמש מלין בשכלי כדי להורת גם את־האחרים מלדבר רבבות מלין בלשון׃
\par 20 אחי אל־תהיו כקטנים בבינה אלא היו תינוקות לרעה ושלמים בבינה׃
\par 21 הן כתוב בתורה כי־בלעגי שפה ובלשון אחרת אדבר אל־העם הזה וגם בזאת לא־אבו שמע־לי אמר יהוה׃
\par 22 לכן הלשנות לא למאמינים הנה אות כי אם־למחסרי אמונה אבל הנבואה איננה למחסרי אמונה כי אם־למאמינים׃
\par 23 והנה אם־תקהל כל־העדה יחד וכלם מדברים בלשנות ויבואו הדיוטות או מחסרי אמונה הלא יאמרו שמשגעים אתם׃
\par 24 אבל אם־יתנבאו כלם ובא איש מחסר אמונה או הדיוט יוכח על־ידי כלם וידון על־ידי כלם׃
\par 25 ובכן יגלו תעלמות לבבו ויפל על־פניו וישתחוה לאלהים ויענה ויאמר באמת האלהים בקרבכם׃
\par 26 ועתה מה־לעשות אחי בהקהלכם יחד כל־אחד ואחד מכם יש־לו מזמור יש־לו הוראה יש־לו לשון יש־לו חזון יש־לו באור וכל יעשה להבנותכם׃
\par 27 כי־ידבר איש בלשון יהיו המדברים שנים שנים או שלשה ולא יותר וזה אחר זה ואחד יפרש׃
\par 28 ואם־אין מפרש אז ידם בקהל וידבר לנפשו ולאלהים׃
\par 29 והנביאים הם ידברו שנים או שלשה והאחרים יבחנו׃
\par 30 וכי נגלה חזון לאחר מן־הישבים שם ידם הראשון׃
\par 31 כי תוכלו להתנבא כלכם זה אחר זה למען ילמדו כלם וכלם יזהרו׃
\par 32 ורוחות הנביאים ברשות הנביאים המה׃
\par 33 כי לא אלהי מבוכה האלהים כי אם־אלהי השלום כאשר בכל־קהלות הקדשים׃
\par 34 נשיכם בכנסיות תשתקנה כי לא־נתנה להן רשות לדבר כי אם־להכנע כאשר אמרה התורה׃
\par 35 ואם־חפצן ללמד דבר תשאלנה את־בעליהן בביתן כי־חרפה היא לנשים לדבר בקהל׃
\par 36 או המכם יצא דבר אלהים אם־אליכם לבדכם הגיע׃
\par 37 אם יאמר איש שהוא נביא או־איש הרוח בין יבין את אשר־אני כתב לכם כי־מצות האדון הנה׃
\par 38 ומי אשר לא ידע אל־ידע׃
\par 39 לכן אחי השתדלו להתנבא ואל־תכלאו מלדבר בלשנות׃
\par 40 הכל יעשה כהגן וכשורה׃

\chapter{15}

\par 1 ואני מזכירכם אחי את־הבשורה אשר בשרתי אתכם ואתם קבלתם ועמדתם בה׃
\par 2 וגם תושעו בה אם־תחזיקו בדבר אשר בשרתי אתכם רק אם לא־האמנתם לשוא׃
\par 3 כי ראשית כל־דבר מסרתי לכם מה־שקבלתי כי־המשיח מת לכפר על־חטאתינו ככתוב׃
\par 4 ונקבר והוקם ביום השלישי ככתוב׃
\par 5 ונראה אל־כיפא ואחריו אל־שנים העשר׃
\par 6 ואחרי־כן נראה ליותר מחמש מאות אחים כאחד אשר רבם עודם בחיים ומקצתם ישנו׃
\par 7 ואחרי־כן נראה אל־יעקב ואחריו אל־כל־השליחים׃
\par 8 ואחרי כלם נראה גם־אלי הדמה לנפל׃
\par 9 כי אני הצעיר בשליחים וקטנתי מהקרא שליח כי־רדפתי את־קהל האלהים׃
\par 10 אבל בחסד אלהים הייתי מה־שהייתי וחסדו עלי לא־היה לריק כי־יותר מכלם עמלתי ולא אני כי־אם־חסד אלהים אשר עמדי׃
\par 11 והנה גם־אני גם־המה ככה משמיעים וככה האמנתם׃
\par 12 ואם־הגד כי־הוקם המשיח מן־המתים איך יאמרו אנשים מכם אין תחיה למתים׃
\par 13 אם־אין תחיה למתים גם־המשיח לא הוקם׃
\par 14 ואם־המשיח לא הוקם ריק שמועתנו וריק אמונתכם׃
\par 15 וגם־נמצא שעדי שקר אנחנו לאלהים יען אשר־העידנו את־האלהים כי הקים את־המשיח והוא לא הקימו אם כן הדבר שהמתים לא יקומו׃
\par 16 כי אם־המתים לא יקומו גם־המשיח לא קם׃
\par 17 ואם־המשיח לא קם הבל אמונתכם ועודכם בחטאתיכם׃
\par 18 אם־כן גם־הישנים במשיח אבדו׃
\par 19 ואם־בחיים האלה בלבד בטחים אנחנו במשיח אמללים מכל־אדם אנחנו׃
\par 20 אבל עתה המשיח הוקם מן־המתים ראשית הישנים׃
\par 21 כי אחרי אשר־בא המות על־ידי אדם גם־תחית המתים באה על־ידי אדם׃
\par 22 כי כאשר באדם הראשון מתים כלם כן יחיו כלם במשיח׃
\par 23 וכל־אחד ואחד בסדרו ראשית כלם המשיח ואחרי־כן אתם שהם למשיח בבואו׃
\par 24 ואחרי כן הקץ כשימסר את־המלכות לאלהים האב אחרי השביתו כל־משרה וכל־שלטן וגבורה׃
\par 25 כי־הוא מלך ימלך עד כי־ישית את־כל־איביו תחת רגליו׃
\par 26 ואחרון האיבים אשר יכחד הוא המות׃
\par 27 כי־כל שת תחת רגליו ובאמרו כל הושת תחתיו ברור הוא שהשת כל תחתיו איננו בכלל׃
\par 28 וכאשר יושת הכל תחתיו אז יושת הבן גם־הוא תחת השת־כל תחתיו למען יהיה האלהים הכל בכל׃
\par 29 כי מה־יעשו הנטבלים בעד המתים אם־אמת הוא שהמתים לא־יקומו למה־זה יטבלו בעד המתים׃
\par 30 ולמה זה מסתכנים אנחנו בכל־שעה׃
\par 31 בתהלתכם אשר יש־לי במשיח ישוע אדנינו מעיד אני עלי אם־לא מת אני בכל־יום ויום׃
\par 32 אם־כדרך כל־אדם נלחמתי עם־החיות הרעות באפסוס מה־היא תועלתי אם־המתים לא יקומו נאכלה ונשתה כי־מחר נמות׃
\par 33 אל־נא תתעו נפשותיכם חברת אנשים רעים תשחית מדות טבות׃
\par 34 הקיצו במישרים ואל־תחטאו כי־יש אנשים אשר אין־בהם דעת אלהים אני אמר זאת לבשתכם׃
\par 35 וכי־יאמר איש איך יקומו המתים וכשישובו מה־גופם׃
\par 36 אתה הסכל הן מה־שתזרע לא יחיה בלתי אם־ימות׃
\par 37 וכשתזרע אינך זרע את־הגוף אשר יהיה כי אם־גרגר ערם של־חטה או של־אחד הזרעים׃
\par 38 והאלהים יתן־לו גוף כרצונו ולכל־זרע וזרע את־גופו למינהו׃
\par 39 לא כל־הבשר בשר אחד כי מין אחר הוא בשר האדם ומין אחר בשר הבהמה ומין אחר בשר הדגה ומין אחר בשר העוף׃
\par 40 ויש גופות שבשמים וגופות שבארץ אבל אחר הוא כבוד הגופות שבשמים ואחר הוא כבוד הגופות שבארץ׃
\par 41 אחר הוא כבוד השמש ואחר הוא כבוד הירח ואחר הוא כבוד הכוכבים כי־כוכב מכוכב שנה לכבוד׃
\par 42 וכן תחית המתים הזריעה לכליון והתקומה לחיי עד׃
\par 43 יזרע בבזיון ויקום בכבוד יזרע בחלשה ויקום בגבורה׃
\par 44 יזרע גוף נפשי ויקום גוף רוחני אם־יש גוף נפשי גם יש גוף רוחני׃
\par 45 וכן כתוב ויהי האדם אדם הראשון לנפש חיה אדם האחרון לרוח מחיה׃
\par 46 אבל לא של־הרוח היא הראשונה אלא של־הנפש ואחרי־כן של־הרוח׃
\par 47 האדם הראשון מן־האדמה הוא של־עפר והאדם השני הוא האדון מן־השמים׃
\par 48 וכמדת האחד שהוא של־עפר כן מדת כל־אשר של־עפר הם וכמדת האחד שהוא של־השמים כן מדת כל־אשר של־השמים הם׃
\par 49 וכאשר לבשנו צלם האדם שהוא של־עפר כן נלבש גם־צלם האדם שהוא של־השמים׃
\par 50 וזאת אני אמר אחי כי־בשר ודם לא־יוכל לרשת את־מלכות האלהים ואשר יכלה לא יירש את אשר לא־יכלה׃
\par 51 הנה סוד אגלה לכם אנחנו לא כלנו נישן המות אבל כלנו נתחלף׃
\par 52 ברגע אחד כהרף עין כתקע השופר האחרון כי יתקע בשופר והמתים יחיו בלי כליון ואנחנו נתחלף׃
\par 53 כי מה־שעתה סופו לכליון ילבש אל־כליון ומה־שעתה סופו למות ילבש אל־מות׃
\par 54 ומה־שעתה סופו לכליון כשילבש אל־כליון ומה־שעתה סופו למות כשילבש אל־מות אז יבא דבר־הכתוב בלע המות לנצח׃
\par 55 איה עקצך המות איה נצחונך שאול׃
\par 56 עקץ המות הוא החטא וכח החטא היא התורה׃
\par 57 אבל תודות לאלהים אשר נתן־לנו הנצחון על־ידי אדנינו ישוע המשיח׃
\par 58 על־כן אחי חביבי התכוננו בל־תמוטו והעדיפו בכל־עת במעשה אדנינו מפני שידעים אתם כי־לא לריק עמלכם באדנינו׃

\chapter{16}

\par 1 ועל־דבר גבוי הצדקה לעזרת הקדושים כאשר תקנתי לקהלות אשר בגלטיא כן תעשו גם־אתם׃
\par 2 בכל־אחד בשבת איש איש מכם כאשר תשיג ידו יניח אצלו ויאצר למען אשר אבא ולא יקבץ עוד׃
\par 3 ואני אבא ואשר תמצאו נאמנים אשלח אתם עם־אגרות להביא את־נדבתכם לירושלים׃
\par 4 ואם־חשוב הוא שאלך גם אני אתי ילכו׃
\par 5 ואני אבא אליכם אחרי עברי את־מקדוניא כי את־מקדוניא אעברה׃
\par 6 ואולי אשב עמכם ימים אחדים או כל־ימי הסתו למען תלוני אל־אשר אלך שמה׃
\par 7 כי כעת אין רצוני שאראה אתכם כעבר בתוככם כי־אקוה לשבת אצלכם ימים אם־יתן יהוה׃
\par 8 אבל אשב באפסוס עד־חג השבועות׃
\par 9 כי־נפתח לי פתח גדול ורב־פעלים והמתקוממים רבים׃
\par 10 וכי יבוא אליכם טימותיוס ראו־נא שיהיה עמכם בלי פחד כי־מלאכת יהוה הוא עשה כמוני׃
\par 11 על־כן איש אל־יבז אתו ושלחהו בשלום למען יבא אלי כי אחכה־לו אני והאחים׃
\par 12 ואפולוס אחינו הפצרתי־בו לבוא אליכם עם־האחים ולא־רצה לבוא עתה ויבוא כשיעלה בידו׃
\par 13 שקדו ועמדו באמונה התאששו והתחזקו׃
\par 14 וכל־דבריכם יעשו באהבה׃
\par 15 ואבקשה מכם אחי הלא ידעתם את־בית אסטפנס שהוא ראשית אכיא ויתנו נפשם לשרות הקדשים׃
\par 16 לכן הכנעו גם־אתם מפני האנשים ההם ומפני־כל־אשר יעבד ויעמל עמהם׃
\par 17 והנני שמח בביאת אסטפנס ופרטונטוס ואכיקוס כי המה מלאו את־חסרנכם׃
\par 18 ויניחו את־רוחי ואת־רוחכם על־כן הכירו האנשים ההם׃
\par 19 הקהלות אשר באסיא שאלות לשלומכם עקילס ופרסקלא וגם־הקהלה אשר בביתם מרבים לשאל לשלומכם באדון׃
\par 20 האחים כלם שאלים לשלומכם שאלו לשלום איש את־רעהו בנשיקה הקדושה׃
\par 21 שאל לשלומכם בכתב ידי אני פולוס׃
\par 22 מי שלא יאהב את־האדון ישוע המשיח יחרם מרן אתא׃
\par 23 חסד ישוע המשיח אדנינו יהי עמכם׃
\par 24 ואהבתי את־כלכם במשיח ישוע אמן׃


\end{document}