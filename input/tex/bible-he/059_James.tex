\begin{document}

\title{איגרת יעקב}


\chapter{1}

\par 1 יעקב עבד אלהים ואדנינו ישוע המשיח שאל לשלום שנים עשר השבטים שבגולה׃
\par 2 אך לשמחה חשבו לכם אחי כאשר תבאו בנסינות שונים׃
\par 3 מפני שידעים אתם כי־בחן אמונתכם מביא לידי סבלנות׃
\par 4 והסבלנות שלמה תהיה בפעלה למען תהיו שלמים ותמימים ולא תחסרו כל־דבר׃
\par 5 ואיש מכם כי יחסר חכמה יבקשנה מאלהים הנותן לכל בנדיבה ובלא הונאת דברים ותנתן לו׃
\par 6 ובלבד שיבקש באמונה ובבלי ספק כי־בעל ספק דומה לגל הים נשא ומטרף ברוח׃
\par 7 והאיש ההוא אל־ידמה בנפשו כי־ישא דבר מאת יהוה׃
\par 8 איש אשר חלק לבו הפכפך הוא בכל־דרכיו׃
\par 9 אבל האח השפל יתהלל ברוממתו׃
\par 10 והעשיר יתהלל בשפלותו כי יעבר כציץ החציר׃
\par 11 כי זרח השמש בחמתו וייבש את־החציר ויבל ציצו וחן מראהו אבד כן יבול העשיר בהליכותיו׃
\par 12 אשרי האיש העמד בנסיונו כי כאשר נבחן ישא עטרת החיים אשר־הבטיח יהוה לאהביו׃
\par 13 אל־יאמר המנסה האלהים נסני כי האלהים איננו מנסה ברע והוא לא־ינסה איש׃
\par 14 כי אם־ינסה כל־איש בתאות נפשו אשר תסיתהו ותפתהו׃
\par 15 ואחרי־כן הרתה התאוה ותלד חטא והחטא כי נגמר יוליד את־המות׃
\par 16 אל־תתעו אחי אהובי׃
\par 17 כל־מתת טובה וכל־מתנה שלמה תרד ממעל מאת אבי האורות אשר חלוף וכל־צל שנוי אין־עמו׃
\par 18 הוא בחפצו ילד אותנו בדבר האמת להיות כמו ראשית בכורי יצוריו׃
\par 19 על־כן אחי אהובי יהי כל־איש מהיר לשמע קשה לדבר וקשה לכעוס׃
\par 20 כי־כעס אדם לא יפעל צדקת אלהים׃
\par 21 לכן הסירו מעליכם כל־טנוף ותרבות רעה וקבלו בענוה את־הדבר הנטוע בכם אשר יכל להושיע את־נפשתיכם׃
\par 22 והיו עשי הדבר ולא שמעיו בלבד לרמות את־נפשכם׃
\par 23 כי האיש השמע את־הדבר ואין עשהו דמה לאיש מביט את־תאר הויתו במראה׃
\par 24 כי הביט אל־מראהו וילך לו וברגע שכח מה־תארו׃
\par 25 אבל המשקיף בתורה השלמה תורת החרות ומחזיק בה אשר איננו שמע ושכח כי אם־עשה בפעל אשרי האיש ההוא במעשהו׃
\par 26 איש מכם אם־ידמה להיות עבד אלהים ואיננו שם רסן ללשנו כי אם־מתעה הוא את־לבבו עבדתו אך־לריק תהיה׃
\par 27 זאת היא העבודה הטהורה והברה לפני האלהים אבינו לפקד את־היתומים והאלמנות בצרתם ולשמר נפשו בנקיון מחלאת העולם׃

\chapter{2}

\par 1 אחי אל־יהי משא פנים באמונתכם בישוע המשיח אדנינו אדון הכבוד׃
\par 2 כי אם־יבוא איש לבית הכנסת שלכם וטבעת זהב על־ידיו והוא לבוש לבשי מכלול ובא שמה גם־איש עני בבגדים צואים׃
\par 3 ופניתם אל־הלבוש לבשי מכלול ואמרתם לו שב־לך הנה בכבוד ולעני תאמרו עמד־שם או שב־פה מתחת להדם רגלי׃
\par 4 הלא לב ולב לכם והנכם שפטים בעלי מחשבות רעות׃
\par 5 שמעו אחי אהובי הלא בעניי העולם הזה בחר האלהים להיות עשירים באמונה וירשי המלכות אשר הבטיח לאהביו׃
\par 6 ואתם הכלמתם את־העני הלא העשירים הם העשקים אתכם והם הסחבים אתכם אל־בתי דין׃
\par 7 הלא הם המגדפים את־השם הטוב הנקרא עליכם׃
\par 8 הן בעשותכם את־המצוה המלכת על־כלן כפי הכתוב ואהבת לרעך כמוך תיטיבו לעשות׃
\par 9 אולם אם־תשאו פנים חטאים אתם והתורה תוכיחכם כעברים עליה׃
\par 10 כי איש אשר יקים את־כל־התורה ונכשל באחת ממצותיה נדון על־כלן׃
\par 11 כי האמר לא תנאף הוא האמר לא תרצח ואם־אינך נאף ואתה רוצח הנך עבר על־התורה׃
\par 12 כן דברו וכן עשו כמי שעתידים להשפט על־פי תורת החרות׃
\par 13 כי אין חסד במשפט לאשר לא־עשה חסד והחסד יתגאה על־המשפט׃
\par 14 אחי מה־יועיל לאיש שיאמר כי אמונה בו ומעשים אין בו התוכל האמונה להושיעו׃
\par 15 אח או אחות כי־יהיו בעירם ואין להם לחם חקם׃
\par 16 ואיש מכם יאמר אליהם לכו לשלום והתחממו ושבעו ולא־תתנו להם צרכי גופם מה־תועיל זאת׃
\par 17 ככה גם־האמונה אם־אין בה מעשים מתה היא בעצמה׃
\par 18 ואם־יאמר איש אתה אמונה בך ולי מעשים הראני נא את־אמונתך בבלי מעשים ואראך אני מתוך מעשי את־אמונתי׃
\par 19 אתה מאמין שהאלהים אחד הוא הטיבות להאמין גם השדים מאמינים בו ורעדים׃
\par 20 ואתה איש־בער התחפץ לדעת כי האמונה באין מעשים מתה היא׃
\par 21 אברהם אבינו הלא במעשיו נצדק בהעלתו את־יצחק בנו על־המזבח׃
\par 22 הנך ראה כי־האמונה עזרת למעשיו ומתוך המעשים השלמה האמונה׃
\par 23 וימלא הכתוב האמר והאמן אברהם ביהוה ותחשב־לו לצדקה ויקרא אהב יהוה׃
\par 24 הנכם ראים כי במעשים יצדק האיש ולא באמונה לבדה׃
\par 25 וכן גם־רחב הזונה הלא נצדקה במעשים באספה את־המלאכים אל־ביתה ותשלחם בדרך אחר׃
\par 26 כי כאשר הגוף מבלי נשמה מת הוא כן גם־האמונה מבלי־מעשים מתה׃

\chapter{3}

\par 1 אחי אל־יהיו רבים מכם למורים באשר ידעתם כי בזאת נחמיר עלינו את־הדין׃
\par 2 כי כלנו נכשלים הרבה ואשר לא־יכשל בדבור הוא איש תמים ויכל לשום רסן לכל־גופו׃
\par 3 הנה בפי הסוסים נשים את־הרסן למען אשר ישמעו לנו ונהגנו בו את כל־גויתם׃
\par 4 והנה האניות אף־כי גדלות הנה ונהדפות ברוח עזה משוט קטן ינהג אתן אל־כל אשר־יחפץ החבל כן גם־הלשון אבר קטן היא וגדלות תדבר׃
\par 5 הנה מה־גדול היער ואש קטנה תבעירנו גם־הלשון אש היא עולם מלא עולה׃
\par 6 (כן) הלשון נצבת בין אברינו המגאלת את־כל־הגוף ומלהטת את־גלגל הויתנו והיא להוטה באש גיהנם׃
\par 7 כי כל־מין בהמה ועוף ורמש וחיות הים יכבש ונכבשים הם על־ידי מין האדם׃
\par 8 אבל הלשון אין אדם יכל לכבשה אין מעצור לרעה הזאת ומלאה חמת המות׃
\par 9 בה נברך את־האלהים אבינו ובה נקלל את־האנשים העשוים בצלם אלהים׃
\par 10 מפה אחד יצאת ברכה וקללה וכן לא־יעשה אחי׃
\par 11 היביע המעין מתוקים ומרים ממוצא אחד׃
\par 12 אחי היוכל עץ התאנה לעשות זיתים או התוכל הגפן לעשות תאנים כן גם־מעין אחד לא יוכל לנבע מים מלוחים ומתוקים׃
\par 13 מי בכם חכם ונבון יראה בדרכו הטובה את־מעשיו בענות החכמה׃
\par 14 ואם־קנאה מרה ומריבה בלבבכם אל־תתהללו ואל־תשקרו באמת׃
\par 15 לא זאת החכמה הירדת ממעל כי אם־חכמת החלד והיצר והשדים׃
\par 16 כי־במקום קנאה ומריבה שם מהומה וכל־מעשה רע׃
\par 17 אבל החכמה אשר ממעל טהורה היא אף־אהבת שלום ומכרעת לכף־זכות ולא עמדת על־דעתה ומלאה רחמים ופרי טוב בלא־לב ולב ואין חנפה בה׃
\par 18 ופרי הצדקה בשלום יזרע לעשי השלום׃

\chapter{4}

\par 1 המלחמות והמדנים אשר ביניכם מאין המה הלא מתוך התאות המתגרות באבריכם׃
\par 2 אתם מתאוים ואין לכם תרצחו ותקנאו והשג לא תשיגו תריבו ותלחמו ואין לכם מפני שלא־שאלתם׃
\par 3 הן שאלים אתם ולא ינתן לכם על־אשר שאלתם ברעה למען תבלו בתאותיכם׃
\par 4 (הנאפים ו) הנאפות הלא ידעתם כי־אהבת העולם איבת אלהים היא ועתה החפץ להיות אהב העולם יהיה איב לאלהים׃
\par 5 התדמו בנפשכם כי לריק אמר הכתוב בקנאה יתאוה לרוח אשר השכין בקרבנו׃
\par 6 וגם יגדיל לתת־חן על־כן הכתוב אומר אלהים ללצים יליץ ולענוים יתן־חן׃
\par 7 לכן הכנעו לפני האלהים התיצבו נגד השטן ויברח מפניכם׃
\par 8 קרבו לאלהים ויקרב אליכם רחצו ידיכם החטאים טהרו לבבכם חלוקי הלבב׃
\par 9 התענו והתאבלו ובכו שחוקכם יהפך לאבל ושמחתכם ליגון׃
\par 10 השפלו לפני יהוה והוא ירים אתכם׃
\par 11 אחי אל־תדברו איש ברעהו המחרף את־רעהו ודן את־אחיו את־התורה הוא מחרף ואת־התורה הוא דן ואם־תדין את־התורה אינך מקים התורה כי אם־דנה׃
\par 12 אחד הוא המחקק (והשפט) אשר יכול להושיע ולאבד ומי אתה כי תדין את־עמיתך׃
\par 13 הוי האמרים נלכה היום ומחר לעיר פלונית אלמונית ונעשה־שם שנה אחת לסחר בה ולהרבות הון׃
\par 14 ולא תדעו מה־ילד יום מחר כי מה חייכם עשן הנראה כמעט־רגע ואחר כלה וילך׃
\par 15 תחת אשר תאמרו אם־ירצה יהוה ונחיה נעשה כזה וכזה׃
\par 16 עתה תתהללו בגאותכם וכל־תהלה כזאת רעה היא׃
\par 17 לכן היודע לעשות הטוב ולא יעשנו חטא הוא לו׃

\chapter{5}

\par 1 הוי העשירים בכו והילילו על־הצרות אשר תבאנה עליכם׃
\par 2 עשרכם בלה ובגדיכם אכלם עש׃
\par 3 זהבכם וכספכם כסתם חלאה והיתה חלאתם בכם לעדות ואכלה כמו־אש את־בשרכם זה הוא האוצר אצרתם לכם לקץ הימים׃
\par 4 הנה שכר הפעלים אספי קציר שדתיכם אשר עשקתם צעק עליכם וצעקת הקוצרים באה באזני יהוה צבאות׃
\par 5 התעדנתם בארץ והתענגתם והשמנתם את־לבכם כמו ליום טבחה׃
\par 6 הרשעתם והמתם את־הצדיק והוא נענה לא־יפתח פיו׃
\par 7 לכן אחי דמו והוחילו עד־בוא האדון הנה האכר מחכה לטוב תבואת האדמה ומיחל כי־ירד עליה גשם יורה ומלקוש׃
\par 8 כן הוחילו גם־אתם ואמצו לבבכם כי קרוב האדון לבא׃
\par 9 אחי אל־תתאוננו איש על־רעהו פן־תשפטו הנה הדין עמד בפתח׃
\par 10 אחי הנביאים אשר דברו בשם יהוה הם יהיו לכם למופת העני והתוחלת׃
\par 11 הנה מאשרים אנחנו את־הסבלים שמעתם סבלנות איוב ואת־אחרית האדון ראיתם כי־רחום וחנון יהוה׃
\par 12 וראש דבר אחי לא תשבעו לא בשמים ולא בארץ ולא כל־שבועה אחרת ויהי הן שלכם הן ולא שלכם לא פן־תפלו בידי הדין׃
\par 13 איש מכם כי־יצר לו יתפלל ואשר ייטב לבו הוא יזמר׃
\par 14 איש מכם כי־יחלה יקרא את־זקני הקהלה ויתפללו בעדו ויסוכהו שמן בשם יהוה׃
\par 15 ותפלת האמונה תושיע את־החולה ויהוה יקימנו ואשר חטא יסלח לו׃
\par 16 התודו עונותיכם איש לפני רעהו והתפללו איש בעד רעהו למען תרפאו כי־גדול כח תפלת הצדיק הקרא אל־אלהים בחזקה׃
\par 17 אליהו אנוש אנוש היה כמנו והתפלל תפלה שלא יהיה מטר ולא־היה מטר בארץ שלש שנים וששה חדשים׃
\par 18 וישב ויתפלל והשמים נתנו מטר והארץ הצמיחה את־פריה׃
\par 19 אחי כי יתעה איש בכם מן־האמת ואיש אחר ישיבנו׃
\par 20 ידע־נא כי המשיב את־החוטא מעקשות דרכו הוא יושיע את־נפשו ממות ויכסה על־המון פשעים׃


\end{document}