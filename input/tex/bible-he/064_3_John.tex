\begin{document}

\title{איגרת יוחנן השלישית}


\chapter{1}

\par 1 הזקן אל־גיוס ידידי אשר אני אהב אתו באמת׃
\par 2 ידידי רצוני שייטב לך לכל־דבר ותהיה בריא כטוב לנפשך׃
\par 3 כי שמחתי מאד כאשר באו אחים ויעידו על־אמתך אשר מתהלך אתה באמת׃
\par 4 אין לי שמחה גדולה מלשמע את־בני מתהלכים באמת׃
\par 5 ידידי באמונה כל־מעשיך עם־האחים ועם־הארחים׃
\par 6 אשר העידו על־אהבתך בפני הקהל ויפה תעשה לשלח אתם כראוי לפני אלהים׃
\par 7 כי למען שמו יצאו ולא לקחו דבר מן־הגוים׃
\par 8 על־כן עלינו לקבל אתם למען נהיה עזרים לאמת׃
\par 9 אני כתבתי אל־הקהלה אבל דיוטריפס המתאוה להיות להם לראש איננו מקבל אתנו׃
\par 10 על־כן בבאי אזכיר את־מעשיו אשר הוא עשה לבטא עלינו דברים רעים ולא די־לו בזה כי גם את־האחים לא יקבל וימנע את־החפצים לקבלם ויגרשם מתוך הקהל׃
\par 11 ידידי אל־תלך בעקבות הרעה כי אם־בעקבות הטוב העשה טוב הוא מאלהים והעשה רע לא ראה את־האלהים׃
\par 12 על־דמטריוס העידו הכל וגם־האמת עצמה וגם־אנחנו מעידים עליו וידעתם כי עדותנו נאמנה׃
\par 13 הרבה יש־לי לכתב ולא חפצתי לכתב אליך בדיו ובקולמוס׃
\par 14 אבל אקוה לראותך במהרה ופה אל־פה נדבר׃ (1:15) שלום לך הרעים שאלים לשלומך שאל לשלום הרעים לאיש איש בשמו׃


\end{document}