\begin{document}

\title{Leviticus}


\chapter{1}

\par 1 Szólítá Mózest és beszéle vele az Úr a gyülekezet sátorából, mondván:
\par 2 Szólj Izráel fiainak, és mondd meg nékik: Ha valaki közületek áldozni akar az Úrnak: barmokból, tulok- és juhfélékbõl áldozzatok.
\par 3 Ha tulokfélébõl áldozik egészen égõáldozattal: hímmel és éppel áldozzék. A gyülekezet sátorának ajtajához vigye azt, hogy kedvessé legyen az Úr elõtt.
\par 4 És tegye kezét az égõáldozat fejére, hogy kedves legyen õ érette, hogy engesztelést szerezzen az õ számára.
\par 5 És ölje meg a tulkot az Úr elõtt, az Áron fiai pedig, a papok, vigyék köröskörûl az oltárra, a mely a gyülekezet sátorának nyílásánál van.
\par 6 Azután vonja le az égõáldozat bõrét, azt pedig vagdalja el tagjaira.
\par 7 És az Áron pap fiai gerjeszszenek tüzet az oltáron, és rakjanak fát a tûzre.
\par 8 Azután rakják az Áron fiai, a papok, a tagokat: a fejet és a kövérjét a fára, a mely az oltáron lévõ tûzön van.
\par 9 A belét pedig és lábszárait mossa meg vízben, és füstölögtesse el a pap az egészet az oltáron egészen égõáldozatul. Ez a tûzáldozat kedves illatú az Úrnak.
\par 10 Ha pedig a juhfélékbõl: bárányokból vagy kecskékbõl akar valaki áldozni égõáldozatul, hímmel és éppel áldozzék.
\par 11 És ölje meg azt az oltár északi oldalánál az Úr elõtt, az Áron fiai pedig, a papok, hintsék annak vérét az oltárra köröskörül.
\par 12 Azután vagdalja el azt tagjaira, a fejével és kövérjével együtt, a pap pedig rakja azokat a fára, a mely az oltáron lévõ tûzön van.
\par 13 A belet és a lábszárakat mossa meg vízben, a pap pedig áldozza meg az egészet és füstölögtesse el az oltáron égõáldozatul. Ez a tûzáldozat kedves illatú az Úrnak.
\par 14 Ha pedig madárfélébõl akar áldozni égõáldozatul az Úrnak, gerliczékbõl vagy galambfiakból áldozzék.
\par 15 És vigye azt a pap az oltárhoz, és tekerje ki annak a fejét, és füstölögtesse el az oltáron, a vérét pedig bocsássa ki az oltár falára.
\par 16 A begyét pedig rútságával egyben vegye ki, és vesse azt az oltár keleti oldalára, a hamu helyére.
\par 17 És hasítsa meg azt a szárnyain, de el ne szakaszsza egymástól, és füstölögtesse el azt a pap az oltáron a fán, a mely a tûzõn van, égõáldozatul. Tûzáldozat ez, kedves illatú az Úrnak.

\chapter{2}

\par 1 Mikor valaki ételáldozatot akar áldozni az Úrnak, lisztlángból áldozzék, és öntsön arra olajat, és temjént is tegyen arra.
\par 2 És vigye azt az Áron fiaihoz, a papokhoz, és vegyen ki abból egy tele marokkal; annak lisztlángjából és olajából, az egész temjénével együtt, és füstölögtesse el a pap az oltáron annak emlékeztetõ részéül. Ez a tûzáldozat kedves illatú az Úrnak.
\par 3 A mi pedig megmarad az ételáldozatból, Ároné és az õ fiaié legyen; szentséges áldozat ez az Úrnak tûzáldozatai között.
\par 4 Hogyha pedig ételáldozatot kemenczében sültbõl akar áldozni valaki lisztlángból, olajjal elegyített kovásztalan lepényei, vagy olajjal megkent kovásztalan pogácsái legyenek.
\par 5 Ha pedig a te áldozatod serpenyõben sült ételáldozat, olajjal elegyített kovásztalan lisztlángból legyen.
\par 6 Darabold azt el, és önts reá olajat, ételáldozat ez.
\par 7 Ha pedig a te ételáldozatod rostélyon sült, lisztlángból, olajjal készíttessék.
\par 8 És vidd el az ételáldozatot, a mi ezekbõl készült, az Úrnak, és azt a papnak bemutatván, az vigye azt el az oltárhoz.
\par 9 És vegyen a pap az ételáldozatból emlékeztetõ részt, és füstölögtesse el az oltáron; tûzáldozat ez, kedves illatú az Úrnak.
\par 10 A mi pedig megmarad az ételáldozatból, Ároné és az õ fiaié legyen; szentséges áldozat ez az Úrnak tûzáldozatai között.
\par 11 Semmi ételáldozat, a mit az Úrnak áldoztok, kovászszal ne készüljön; mert kovászból és mézbõl semmit se füstölögtethettek az Úrnak a tûzáldozatok között.
\par 12 Zsengeáldozatul felvihetitek azokat az Úrnak, de az oltárra nem juthatnak fel kedves illatul.
\par 13 Minden te ételáldozatodat pedig sózd meg sóval, és a te ételáldozatodból soha el ne maradjon a te Istened szövetségének sója; minden te áldozatodhoz sót adj.
\par 14 Ha zsengékbõl való ételáldozatot áldozol az Úrnak, tûznél pergelt kalászból, és zsenge gabona-darából áldozzad a te zsengéidnek áldozatát.
\par 15 Adj hozzá olajat, és tégy reá tömjént; ételáldozat ez.
\par 16 A pap pedig füstölögtesse el annak emlékeztetõ részét a darából és olajból, az egész tömjénnel együtt. Tûzáldozat ez az Úrnak.

\chapter{3}

\par 1 Hogyha hálaáldozattal áldozik valaki, ha tulokfélébõl, akár hímmel, akár nõsténynyel áldozik: ép barmot vigyen az Úr elé.
\par 2 És tegye a kezét az õ áldozatjának fejére, és ölje meg azt a gyülekezet sátorának nyílásánál, és Áron fiai, a papok, öntsék a vért köröskörül az oltárra.
\par 3 Azután vigyen a hálaáldozatból tûzáldozatot az Úrnak: a kövérjét, a mely a belet takarja, és mindazt a kövérjét, a mely a belek között van.
\par 4 A két vesét is, és a rajtuk lévõ kövérséget, a mely a véknyaknál van, és a májon lévõ hártyát a vesékkel együtt vegye el.
\par 5 És füstölögtessék el azt Áron fiai az oltáron az égõáldozattal együtt a tûzön lévõ fán: tûzáldozat ez, kedves illatú az Úrnak.
\par 6 Hogyha pedig valaki juhfélébõl áldozik hálaáldozatot az Úrnak, akár hím, akár nõstény legyen az, épekkel áldozzék.
\par 7 Ha bárányt visz õ az õ áldozatául, vigye azt az Úr elébe.
\par 8 És tegye a kezét az õ áldozatjának fejére, azután ölje meg azt a gyülekezet sátora elõtt, és az Áron fiai öntsék a vérét az oltárra köröskörül.
\par 9 Azután áldozzék az Úrnak e hálaáldozatból tûzáldozatot: a kövérjét, a farkát egészen, a melyet a hátagerézdi végénél vágjon el, és a belét takaró kövérjét, és minden kövérjét, a mely a belek közt van.
\par 10 És a két vesét is, és a rajtuk lévõ kövérséget, a mely a véknyaknál van, és a májon lévõ hártyát a vesékkel együtt vegye el.
\par 11 És füstölögtesse el azt a pap az oltáron: tûzáldozati eledel ez az Úrnak.
\par 12 Hogyha kecskével áldozik valaki, azt is az Úr elé vigye.
\par 13 És tegye kezét annak fejére, és ölje meg azt a gyülekezetnek sátora elõtt, és öntsék Áron fiai annak vérét az oltárra körül.
\par 14 És áldozzék abból tûzáldozatot az Úrnak: a kövérjét, a mely betakarja a belet, és mindazt a kövérjét, a mely a belek között van.
\par 15 A két vesét is, és a rajtuk lévõ kövérséget, a mely a véknyaknál van, és a májon lévõ hártyát a vesékkel együtt vegye el.
\par 16 És füstölögtesse el azokat a pap az oltáron tûzáldozati eledelül, kedves illatul. A kövérje mind az Úré legyen.
\par 17 Örökkévaló rendtartás legyen a ti nemzetségeiteknél minden ti lakhelyeteken: semmi kövért és semmi vért meg ne egyetek!

\chapter{4}

\par 1 Szóla ismét az Úr Mózesnek, mondván:
\par 2 Szólj az Izráel fiainak, mondván: Ha valaki tévedésbõl vétkezik az Úrnak valamely parancsolata ellen, úgy a mint nem kellene cselekednie, és a parancsolatok közül valamelyiknek ellene cselekszik:
\par 3 Ha a felkent pap vétkezik, a népnek romlására: hozzon az õ bûnéért, a melyet elkövetett, egy tulkot, fiatal ép marhát az Úrnak bûnáldozatul.
\par 4 És vigye a tulkot a gyülekezet sátorának nyílásához, az Úr elé, és tegye a kezét a tuloknak fejére, és a tulkot ölje meg az Úr elõtt.
\par 5 És vegyen a felkent pap a tuloknak vérébõl, és vigye azt a gyülekezet sátrába.
\par 6 És mártsa be a pap az õ újját a vérbe, és hintsen a vérbõl hétszer az Úr elõtt a szent hajléknak függönye felé.
\par 7 És tegyen a pap a vérbõl az Úr elõtt a fûszerekbõl való füstölõ oltár szarvaira, a mely ott van a gyülekezet sátorában; a tulok vérét pedig mind öntse az egészen égõáldozat oltárának aljára, a mely a gyülekezet sátorának nyílásánál van.
\par 8 Azután a tuloknak, a mely bûnáldozatra való, minden kövérjét szedje ki belõle: azt a kövérjét, amely betakarja a belet, és mindazt a kövérjét, a mi a belek között van.
\par 9 A két vesét is, és a rajtuk lévõ kövérséget, a mely a véknyaknál van, és a májon lévõ hártyát a vesékkel együtt vegye el,
\par 10 A miképen kiszedik a hálaáldozatra való tulokból; és füstölögtesse el azokat a pap az égõáldozatnak oltárán.
\par 11 A tuloknak pedig bõrét és minden húsát, fejével és lábszáraival együtt, bélit és ganéját,
\par 12 És mind az egész tulkot vigye ki a táboron kivül tiszta helyre, a hová a hamut öntik: és égesse el azt a fán, tûzben; ott égessék meg, a hová a hamut öntik.
\par 13 Hogyha pedig az Izráel fiainak egész közönsége megtéved, és a gyülekezet elõtt rejtve marad e dolog; és valami olyat cselekesznek az Úrnak valamelyik parancsolatja ellen, a mit nem kellett volna cselekedni, és bûnösökké lesznek:
\par 14 Mikor kitudódik a bûn, a melyet elkövettek: akkor áldozzék a gyülekezet egy tulkot, fiatal marhát a bûnért, és vigye azt a gyülekezetnek sátora elé.
\par 15 És a gyülekezet vénei tegyék kezeiket a tuloknak fejére az Úr elõtt, és ölje meg a tulkot a pap az Úr elõtt.
\par 16 Azután vigyen be a felkent pap a tulok vérébõl a gyülekezet sátorába.
\par 17 És mártsa be a pap az õ újját a vérbe, és hintsen abból hétszer az Úr elõtt a függöny felé.
\par 18 És tegyen a vérbõl az oltár szarvaira is, a mely az Úr elõtt, a gyülekezet sátorában van; a vért pedig mind öntse az egészen égõáldozat oltárának aljára, a mely a gyülekezet sátorának nyílása elõtt van.
\par 19 Minden kövérjét pedig szedje ki belõle, és füstölögtesse el az oltáron.
\par 20 És úgy cselekedjék azzal a tulokkal, mint a bûnért való tulokkal cselekvék, úgy cselekedjék vele, és engesztelést szerez számukra a pap, és megbocsáttatik nékik.
\par 21 És vigye ki a tulkot a táboron kivül, és égesse el azt, miképen elégette az elsõ tulkot. A gyülekezet bûnéért való áldozat ez.
\par 22 Ha fejedelem vétkezik, és cselekeszik valamit az Úrnak, az õ Istenének parancsolata ellen, a mit nem kellett volna cselekedni, és bûnössé lesz tévedésbõl:
\par 23 Ha tudtára esik néki a bûne, a melyet elkövetett, akkor vigyen áldozatul egy ép kecskebakot,
\par 24 És tegye kezét a baknak fejére, és ölje meg azt azon a helyen, a hol megölik az egész égõáldozatot, az Úr elõtt; bûnért való áldozat ez.
\par 25 És vegyen a pap bûnért való áldozatnak vérébõl az õ újjával, és tegyen az égõáldozat oltárának szarvaira; a vérét pedig öntse az égõáldozatok oltárának aljára.
\par 26 A kövérjét pedig mind füstölögtesse el az oltáron, mint a hálaáldozatnak kövérjét. Így szerezzen néki a pap engesztelést az õ bûnéért, és megbocsáttatik néki.
\par 27 Ha pedig a föld népe közül vétkezik valaki tévedésbõl, mivelhogy az Úrnak valamelyik parancsolatja ellen olyat cselekeszik, a mit nem kellett volna cselekedni, és bûnössé lesz;
\par 28 Ha tudtára esik néki az õ bûne, a melyet elkövetett; vigyen áldozatul az õ bûnéért, a melyet elkövetett, egy ép nõstény kecskét.
\par 29 És tegye a kezét a bûnért való áldozat fejére, és ölje meg a bûnért való áldozatot az egészen égõáldozat helyén.
\par 30 És vegyen a pap annak vérébõl az újjával, és tegyen az égõáldozat oltárának szarvaira, a vérét pedig mind öntse az oltárnak aljára.
\par 31 Azután vegye el minden kövérjét, a mint elveszik a hálaáldozatnak kövérjét: és füstölögtesse el a pap az oltáron, kedves illatul az Úrnak. Ekképen szerezzen néki engesztelést a pap, és megbocsáttatik annak.
\par 32 Ha pedig bárányt visz az õ bûnért való áldozatául, nõstényt és épet vigyen.
\par 33 És tegye a kezét annak a bûnért való áldozatnak fejére, és ölje meg azt bûnért való áldozatul azon a helyen, a hol megölik az egészen égõáldozatot.
\par 34 És vegyen a pap a bûnért való áldozat vérébõl az újjával, és tegyen az égõáldozat oltárának szarvaira; a vért pedig mind öntse az oltárnak aljára.
\par 35 Azután vegye el minden kövérjét, a mint elveszik a hálaáldozatra való báránynak kövérjét, és füstölögtesse el azokat a pap az oltáron az Úrnak tûzáldozataival. Ekképen szerezzen néki engesztelést a pap az õ bûnéért, a melyet elkövetett, és megbocsáttatik néki.

\chapter{5}

\par 1 Ha azzal vétkezik valaki, hogy hallotta a káromló beszédet, és bizonyság lehetne, hogy látta, vagy tudja: ha meg nem jelenti azt, de hordozza az õ vétségének terhét;
\par 2 Vagy ha valaki illet akármely tisztátalan dolgot, akár tisztátalan vadnak, akár tisztátalan baromnak, akár tisztátalan féregnek holttestét, és nem tud arról, hogy tisztátalanná lesz és vétkezik;
\par 3 Vagy ha illeti az ember tisztátalanságát, akármi tisztátalanságát, a mely tisztátalanná tesz, és nem tudta azt, hanem azután értette meg, hogy vétkezett;
\par 4 Vagy ha valaki hitetlenkedve tesz esküt az õ ajkaival rosszra vagy jóra, vagy akármi az, a mire hirtelenkedve esküszik az ember, ha nem tudott arról, de azután megértette, hogy azok közül valamelyikben vétkezett:
\par 5 Akkor, mivelhogy vétkezett ezek közül valamelyikben, vallja meg, hogy mi az, a miben bûnössé lett;
\par 6 És vigyen az õ vétkéért az Úrnak, az õ bûnéért, a melyet elkövetett, egy nõstény bárányt vagy kecskét a nyájból, bûnért való áldozatul. Ekképen szerezzen néki engesztelést a pap az õ bûnéért.
\par 7 Ha pedig nincsen egy bárányhoz való módja: vigyen az õ vétkéért való áldozatra két gerliczét vagy két galambfiat az Úrnak, az egyiket a bûnért való áldozatul, a másikat egészen égõáldozatul.
\par 8 És vigye azokat a paphoz; az pedig áldozza meg elõször azt, a mely bûnért való áldozat, és tekerje ki annak fejét nyakcsigájánál, úgy, hogy el ne szakaszsza.
\par 9 És hintsen a bûnért való áldozat vérébõl az oltár oldalára, a többi vért pedig nyomják ki az oltár aljára; bûnért való áldozat ez.
\par 10 A másikat készítse el egészen égõáldozatul, úgy, a mint szokás; ekképen szerez néki engesztelést a pap az õ bûnéért, a melyet elkövetett, és megbocsáttatik néki.
\par 11 Ha pedig nincs módja két gerliczéhez vagy két galambfihoz sem: vigyen áldozatul az, a ki vétkezett, egy efa lánglisztnek tizedrészét bûnért való áldozatul; ne tegyen ahhoz olajat, és ne adjon ahhoz tömjént, mert bûnért való áldozat ez.
\par 12 És vigye azt a paphoz, és markolja ki abból a pap teli marokkal az emlékeztetõ részét, és füstölögtesse el az oltáron az Úrnak tûzáldozataival. Bûnért való áldozat az.
\par 13 Így szerezzen néki engesztelést a pap az õ bûnéért, a melyet elkövetett valamivel ama bûnök közül, és megbocsáttatik néki. És legyen a papé, mint az ételáldozat.
\par 14 Azután szóla az Úr Mózesnek, mondván:
\par 15 Ha valaki hûtlenséget követ el, és tévedésbõl vétkesen elvesz az Úrnak szentelt dolgokból: vigyen az õ vétkéért való áldozatot az Úrnak, egy ép kost a nyájból, a mint te becsülöd, ezüst siklusokban, a szent siklus szerint, vétekért való áldozatul.
\par 16 És a mit vétkesen elvett a szent dologból, azt fizesse meg, és tegye hozzá az ötödrészét, és adja azt a papnak. Így szerez néki engesztelést a pap a vétekért való áldozat kosával, és megbocsáttatik néki.
\par 17 Hogyha vétkezik valaki, és cselekszik valamit az Úrnak valamely parancsolata ellen, a mit nem kell cselekedni, ha nem tudta is: vétkessé lesz, és hordozza az õ vétségének terhét;
\par 18 Vigyen azért a nyájból egy ép kost a paphoz, a te becslésed szerint, bûnért való áldozatul, és szerezzen néki engesztelést a pap az õ tévedéséért, a melylyel tudatlanságból tévedett, és megbocsáttatik néki.
\par 19 Bûnért való áldozat ez, mivelhogy bûnt követett el az Úr ellen.

\chapter{6}

\par 1 Szóla ismét az Úr Mózesnek, mondván:
\par 2 Mikor vétkezik valaki, és hûtlenséget követ el az Úr ellen, tudniillik eltagadja felebarátjának reábízott vagy kezébe adott holmiját, vagy megrabolja vagy zsarolja felebarátját;
\par 3 Vagy ha elveszett holmit talált, és eltagadja, vagy valami miatt hamisan esküszik, akármi is az, a mit az ember úgy cselekszik, hogy vétkezik vele:
\par 4 Mivelhogy azért bûnössé lett és vétkezett, térítse vissza az elrablottat, a mit rabolt, vagy a zsaroltat, a mit zsarolt, vagy a reá bízottat, a mi reá bízatott, vagy az elveszettet, a mit megtalált;
\par 5 Vagy akármi legyen, a mire hamisan esküdött, fizesse meg azt teljes értéke szerint, és hozzátoldva az ötödrészét, adja azt annak, a kié volt, bûnbevallásának napján.
\par 6 Az õ bûnéért pedig vigyen az Úrnak a nyájból egy ép kost a paphoz, a te becslésed szerint, bûnért való áldozatul.
\par 7 Így szerezzen néki engesztelést a pap az Úr elõtt, és megbocsáttatik néki mindaz, a mit cselekedett, és a miben vétkezett.
\par 8 Szóla ismét az Úr Mózesnek, mondván:
\par 9 Parancsolj Áronnak és az õ fiainak, mondván: Ez az egészen égõáldozat törvénye: Legyen az egészen égõáldozat az oltáron levõ tüzelõhelyen egész éjszaka, mind reggelig, és az oltárnak tüze égve maradjon azon.
\par 10 A pap öltse fel az õ gyolcs ruháját, és a gyolcs lábravalót is öltse fel az õ testére, és szedje el a hamut, a mivé égette a tûz az égõáldozatot az oltáron, és töltse azt az oltár mellé.
\par 11 Azután vesse le azt a ruháját, és öltözzék más ruhába, és vigye ki a hamut a táboron kivül, tiszta helyre.
\par 12 Az oltáron lévõ tûz pedig égve maradjon azon, el ne aludjék, hanem égessen fát rajta a pap minden reggel, és rakja reá az egészen égõáldozatot, és azon füstölögtesse el a hálaáldozat kövérjét is.
\par 13 A tûz szüntelen égve maradjon az oltáron, és el ne aludjék.
\par 14 Ez pedig az ételáldozatnak törvénye: az Áron fiai áldozzák azt az Úr elõtt az oltáron.
\par 15 És vegyen valaki közülök az ételáldozat lisztlángjából egy marokkal és annak olajából is, a tömjént pedig, a mely az ételáldozathoz való, mind; és égesse el az oltáron; annak emlékeztetõ része kedves illat az Úr elõtt.
\par 16 A mi pedig megmarad belõle, egyék meg Áron és az õ fiai, kovásztalanul egyék meg szenthelyen, a gyülekezet sátorának pitvarában egyék meg azt.
\par 17 Ne süssék azt kovászszal, mert nékik adtam azt, részökül az én tûzáldozataimból; igen szentséges az, mint a bûnért és vétekért való áldozat.
\par 18 Az Áron fiai közül minden férfiú egye azt. Örökkévaló rendtartás legyen ez a ti nemzetségeiteknél az Úrnak tûzáldozatai felõl. Valaki illeti azokat, szent legyem.
\par 19 Szóla ismét az Úr Mózesnek, mondván:
\par 20 Ez Áronnak és az õ fiainak áldozatjok, a melyet az Úrnak áldozzanak, mikor felkenik õket: Egy efa lisztlángnak tizedrésze mindenkor ételáldozatul, fele reggel, fele pedig estve.
\par 21 Serpenyõben készíttessék, olajjal összegyúrva vidd el azt, az ételáldozati süteményeket darabokban áldozd az Úrnak kedves illatul.
\par 22 A mely pap az õ helyébe kenetik fel az õ fiai közül, az mívelje ezt. Örökkévaló rendtartás ez, az Úrnak mindenestõl füstölögtessék el.
\par 23 Mert a pap minden ételáldozatának mindenestõl meg kell égettetni, nem kell abból semmit megenni.
\par 24 Szóla ismét az Úr Mózesnek, mondván:
\par 25 Szólj Áronnak és az õ fiainak, mondván: Ez a bûnért való áldozat törvénye: a mely helyen meg szokták ölni az egészen égõáldozatot, azon a helyen öljék meg a bûnért való áldozatot az Úr elõtt; igen szentséges az.
\par 26 A mely pap megáldozza azt a bûnért, az egye meg azt, szent helyen egye meg, a gyülekezet sátorának pitvarában.
\par 27 Valami annak a húsát érinti, szent legyen, és ha annak vérébõl valami a ruhájára esik valakinek, azt, a mire a vér esett, mosd meg szenthelyen.
\par 28 És a cserépedényt, a melyben azt fõzték, törjék el; hogyha pedig érczfazékban fõzték, súrolják meg, és mossák meg vízzel.
\par 29 A papok között minden férfiú eheti azt; igen szentséges az.
\par 30 Valamely bûnért való áldozat vérébõl bevisznek a gyülekezet sátorába, a szenthelyen való engesztelés végett, az meg nem ehetõ tûzzel égettessék meg.

\chapter{7}

\par 1 Ez pedig a vétekért való áldozatnak törvénye; igen szentséges az.
\par 2 A mely helyen megölik az egészen égõáldozatot, ott öljék meg a vétekért való áldozatot is, és a vérét hintsék az oltárra köröskörül.
\par 3 A kövérjét pedig áldozzák meg mind, a farkát is, és a mely kövér a belet takarja.
\par 4 A két veséjét és a rajtuk lévõ kövérséget, a mely a véknyaknál van, úgyszintén a májon lévõ hártyát a vesékkel együtt szedje ki.
\par 5 És füstölögtesse el azokat a pap az oltáron tûzáldozatul az Úrnak. Vétekért való áldozat ez.
\par 6 A papok között minden férfiú eheti azt, szenthelyen egyék meg; igen szentséges az.
\par 7 A milyen a bûnért való áldozat, olyan a vétekért való áldozat is, egy törvényök van nékik. Azé a papé az, a ki engesztelést szerez vele.
\par 8 Ha a pap egészen égõáldozatot áldoz valakiért, az égõáldozat bõre, a melyet megáldoz, azé a papé legyen.
\par 9 Minden ételáldozat is, a melyet kemenczében sütnek és minden, a mit rostélyon vagy serpenyõben készítenek, a papé legyen, a ki azt áldozta.
\par 10 Az olajjal elegyített és száraz ételáldozat is mind az Áron fiaié legyen közösen, egyiké úgy, mint a másiké.
\par 11 Ez pedig a hálaáldozat törvénye, a melyet áldoznak az Úrnak.
\par 12 Ha dicsõítésül áldozza azt valaki, a dicsõítés áldozatjával együtt áldozzék olajjal elegyített kovásztalan lepényeket, és olajjal megkent kovásztalan pogácsákat, és lisztlángból gyúrt, olajjal elegyített lepényeket.
\par 13 A lepényeken kivül kovászos kenyeret is vigyen áldozatául, dicsõítõ hálaáldozatjával együtt.
\par 14 És mindezekbõl az áldozatokból áldozzék egyet-egyet az Úrnak felmutatott áldozatul; és legyen azé a papé, a ki a hálaáldozatnak vérét elhinti.
\par 15 És az õ dicsõítõ hálaáldozatjának húsát, az õ áldozásának napján egyék meg; ne hagyjon abból reggelig.
\par 16 Hogyha fogadásból vagy szabadakaratból áldozza valaki az õ áldozatát, a mely napon áldozza azt, azon a napon egyék meg az õ áldozatát; a mi pedig megmarad abból, másnap egyék meg.
\par 17 A mi pedig annak az áldozatnak húsából tovább is megmarad, harmadnapon tûzzel égettessék meg.
\par 18 Mert ha az õ hálaadó áldozatának húsából harmadnapon eszik valaki, nem lesz az kedves; a ki áldozta azt, annak nem számíttatik az õ javára, sõt útálatos lesz; és valaki eszik abból, hordozza az õ vétségének terhét.
\par 19 Azt a húst pedig, a mely valami tisztátalanhoz ér, meg ne egyék, hanem tûzzel égessék meg. A mi különben a húst illeti, mindenki ehetik húst, a ki tiszta;
\par 20 De az, a ki eszik a békeáldozatnak húsából, a mely az Úré, noha az õ tisztátalansága rajta van, az ilyen ember gyomláltassék ki az õ népe közül.
\par 21 Ha valaki akármi tisztátalanhoz ér, tisztátalan emberhez, vagy tisztátalan baromhoz, vagy akármihez, a mi tisztátalan útálatosság, és eszik a hálaadó áldozatnak húsából, a mely az Úré, gyomláltassék ki az az ember az õ népe közül.
\par 22 Szóla ismét az Úr Mózesnek, mondván:
\par 23 Szólj az Izráel fiainak, mondván: Az ökörnek, a báránynak és a kecskének semmi kövérjét meg ne egyétek.
\par 24 A hullott állatnak kövérje, és a vadtól megszaggatottnak kövérje, akármi munkához felhasználható, de enni meg ne egyétek!
\par 25 Mert akárki egyék is az aféle állatoknak kövérjébõl, a melyekbõl tûzáldozatot visznek az Úrnak, az az ember a ki ilyet eszik, gyomláltassék ki az õ népe közül.
\par 26 És semmi vért se egyetek meg bármely lakhelyeteken: se madárnak, se baromnak vérét.
\par 27 Valaki megeszik valami féle vért, az az ember gyomláltassék ki az õ népe közül.
\par 28 Szóla ismét az Úr Mózesnek és monda:
\par 29 Szólj az Izráel fiainak, mondván: A ki hálaáldozattal áldozik az Úrnak, maga vigye az Úrnak az õ áldozatát az õ hálaáldozatából.
\par 30 A maga keze vigye az Úrnak tûzáldozatát: a kövérjét a szegyével együtt vigye el, a szegyét azért, hogy meglóbálják azt az Úr elõtt.
\par 31 A pap pedig füstölögtesse el azt a kövért az oltáron, a szegye pedig legyen Ároné és az õ fiaié.
\par 32 A jobblapoczkát is a papnak adjátok a hálaadó áldozatból, hogy azt felmutassa.
\par 33 A ki Áron fiai közül a hálaáldozat vérét és kövérjét megáldozza, a jobblapoczka annak része legyen.
\par 34 Mert a meglóbált szegyet és a felmutatott lapoczkát elveszem Izráel fiaitól az õ hálaadó áldozataikból, és adom azokat Áron papnak és az õ fiainak, örökre kiszabott részül, az Izráel fiaitól.
\par 35 Ez az Áron felkenetési része, és az õ fiainak felkenetési része az Úrnak tûzáldozataiból, a naptól fogva, a melyen elõállítá õket, hogy papi szolgálatot tegyenek az Úrnak;
\par 36 A melyet parancsolt az Úr, hogy adják nékik az Izráel fiai, a mely napon felkente õket, örökre kiszabott részül az õ nemzetségökben.
\par 37 Ez az egészen égõáldozatnak, az ételáldozatnak, a bûnért és a vétekért való áldozatnak, a felavatási áldozatnak és a hálaáldozatnak törvénye,
\par 38 A melyet parancsolt az Úr Mózesnek a Sinai hegyen, a mely napon parancsolta az Izráel fiainak a Sinai pusztában, hogy áldozzanak az Úrnak.

\chapter{8}

\par 1 Szóla továbbá az Úr Mózesnek, mondván:
\par 2 Vegyed Áront és az õ fiait is vele, és az öltözeteket, a kenetnek olaját, és bûnért való áldozati tulkot, két kost és egy kosár kovásztalan kenyeret.
\par 3 És az egész gyülekezetet gyûjtsd egybe a gyülekezet sátorának nyílásához.
\par 4 És a képen cselekedék Mózes, a mint az Úr parancsolta vala néki, és egybe gyûle a gyülekezet a gyülekezet sátorának nyílásához.
\par 5 Akkor monda Mózes a gyülekezetnek: Ez a dolog, a mit az Úr parancsolt cselekedni.
\par 6 És elõállatá Mózes Áront és az õ fiait, és megmosá õket vízzel.
\par 7 És reá adá Áronra a köntöst, és felövezé õt az övvel, és reáveté a palástot, az efódot is reáadá, és felövezé az efód övével, és megerõsíté azt rajta.
\par 8 És feltevé arra a hósent, és betevé a hósenbe az Urimot és a Thummimot.
\par 9 Azután feltevé fejére a süveget, és elõl odatevé a süvegre az arany lapot, a szent koronát, a mint parancsolta vala az Úr Mózesnek.
\par 10 Vevé Mózes a kenetnek olaját is, és megkené a hajlékot minden bennevalóval egybe, és felszentelé azokat.
\par 11 És hinte abból az oltárra is hétszer, és felkené az oltárt és annak minden edényét, a mosdómedenczét is a lábával együtt, hogy azokat megszentelje.
\par 12 Az Áron fejére is tölte a kenetnek olajából, és megkené õt, hogy felszentelje õt.
\par 13 És elõállatá Mózes az Áron fiait is, és felöltözteté azokat is az õ köntöseikbe, és felövezé õket övvel, felköté nékik a süvegeket is, a mint az Úr parancsolta vala Mózesnek.
\par 14 Azután elõhoza egy tulkot a bûnért való áldozatra, és Áron az õ fiaival egybe a bûnért való áldozat tulkának fejére tevé az õ kezét.
\par 15 És miután megölték azt, võn Mózes annak vérébõl, és tõn az újjával az oltárnak szarvaira köröskörül, és megtisztítá az oltárt, a többi vért pedig önté az oltárnak aljára; és felszentelte azt, hogy engesztelést szerezzen rajta.
\par 16 Azután vevé mindazt a kövérséget, a mely annak a bélin vala, és a máj hártyáját, és a két veséjét és azoknak kövérségét, és elfüstölögteté Mózes azokat az oltáron.
\par 17 A tulkot pedig, azaz annak bõrét, húsát és ganéját megégeté tûzzel a táboron kivül, a mint az Úr parancsolta vala Mózesnek.
\par 18 Azután elõállatá az égõáldozatra való kost, és Áron és az õ fiai rátevék kezeiket a kos fejére.
\par 19 És megölék azt; Mózes pedig elhinté a vért az oltárra köröskörül.
\par 20 És a kost tagjaira vagdalák, és Mózes elfüstölögteté annak a fejét, a tagjait és a kövérjét.
\par 21 A beleket pedig és a lábszárakat megmosta vízben, és elfüstölögteté Mózes az egész kost az oltáron. Kedves illatú égõáldozat ez, tûzáldozat ez az Úrnak, a mint megparancsolta vala az Úr Mózesnek.
\par 22 Azután elõállatá a másik kost, és rátevék Áron és az õ fiai kezeiket a kos fejére.
\par 23 És megölék azt; Mózes pedig võn annak vérébõl, és tõn abból az Áron jobb fülének czimpáján, és jobb kezének hüvelykére és jobb lábának hüvelykére.
\par 24 Elõállatá az Áron fiait is, és tõn Mózes a vérbõl azok jobb fülének czimpájára, és jobb kezöknek hüvelykére és jobb láboknak hüvelykére; azután oda tölté Mózes a vért az oltárra köröskörül.
\par 25 És vevé a kövérjét és a farkát és mindazt a kövérjét, a mely a bélen van, továbbá a máj hártyáját, és a két vesét azoknak kövérjével egybe, és a jobb lapoczkát;
\par 26 És a kovásztalan kenyerek kosarából, a mely az Úr elõtt vala, võn egy kovásztalan lepényt, egy olajos kalácsot, és egy pogácsát, és raká azokat a kövérségekre és a jobb lapoczkára;
\par 27 És tevé mindezeket az Áron kezeire és az õ fiainak kezeire, és meglóbáltatá azokat az Úr elõtt.
\par 28 Azután elvevé azokat Mózes az õ kezeikbõl, és elfüstölögteté az oltáron az egészen égõáldozattal egybe. Felavatási áldozatok ezek, kedves illatú tûzáldozat ez az Úrnak.
\par 29 Elvevé pedig Mózes a szegyet és meglóbálá azt az Úr elõtt; a felavatási kosból a Mózes része lõn ez, a mint megparancsolta az Úr Mózesnek.
\par 30 Azután võn Mózes a kenetnek olajából és a vérbõl, a mely az oltáron vala, és meghinté Áront és az õ ruháit, az õ fiait és az õ fiainak ruháit õ vele együtt, és megszentelé Áront és az õ ruháit, és az õ fiait és az õ fiainak ruháit õ vele együtt.
\par 31 És monda Mózes Áronnak és az õ fiainak: Fõzzétek meg a húst a gyülekezet sátorának nyílásánál, és ott egyétek meg azt és a kenyeret, a mely a felavatási áldozat kosarában van, a miképen megparancsoltam, mondván: Áron és az õ fiai egyék meg azt.
\par 32 A mi pedig megmarad a húsból és kenyérbõl, tûzzel égessétek meg.
\par 33 De a gyülekezet sátorának nyílásán ki ne menjetek hét napig, addig a napig, a melyen betelnek a ti felavatástoknak napjai, mert hét nap avat fel benneteket az Úr.
\par 34 A miképen e napon cselekedett, úgy parancsolta az Úr hogy cselekedjünk, hogy néktek engesztelést szerezzünk.
\par 35 A gyülekezet sátorának nyílásánál maradjatok éjjel és nappal hét napig, és tartsátok meg, a mit megtartani rendelt az Úr, hogy meg ne haljatok; mert így parancsolta vala nékem.
\par 36 Áron azért és az õ fiai mind akképen cselekedének, a mint megparancsolta vala nékik az Úr Mózes által.

\chapter{9}

\par 1 És lõn a nyolczadik napon, hogy szólítá Mózes Áront és az õ fiait, és Izráelnek véneit.
\par 2 És monda Áronnak: Végy egy borjút, fiatal bikát bûnért való áldozatul, és egy kost egészen égõáldozatul, épek legyenek, és vidd az Úr elé.
\par 3 Szólj Izráel fiainak is mondván: Vegyetek egy kecskebakot bûnért való áldozatul, és egy esztendõs borjút és bárányt, épek legyenek, egészen égõáldozatul.
\par 4 És egy ökröt és egy kost hálaáldozatul, hogy megáldozzátok az Úr elõtt, és olajjal elegyített ételáldozatot; mert ma az Úr megjelenik néktek.
\par 5 Elvivék azért, a miket Mózes parancsolt vala, a gyülekezet sátora elé, és odajárula az egész gyülekezet, és megálla ott az Úr elõtt.
\par 6 És monda Mózes: Ez a dolog, a melyet az Úr parancsolt, cselekedjétek meg; és megjelenik néktek az Úr dicsõsége.
\par 7 Áronnak pedig monda Mózes: Járulj az oltárhoz, és készítsd el a te bûnért való áldozatodat és egészen égõáldozatodat, és végezz engesztelést magadért és a népért. Késyítsd el a nép áldozatatát is, és végezz engesztelést érettök is a mint megparancsolta az Úr.
\par 8 Járula azért Áron az oltárhoz, és megölé a borjút, a mely az övé, bûnért való áldozatul.
\par 9 Áron fiai pedig odavivék õ hozzá a vért, és õ bemártá az újját a vérbe, és tõn abból az oltár szarvaira, a többi vért pedig kiönté az oltár aljához.
\par 10 A kövérjét pedig és a veséket és a máj hártyáját elfüstölögteté a bûnért való áldozatból az oltáron, a mint az Úr parancsolta vala Mózesnek.
\par 11 A húst pedig és a bõrt tûzzel égeté meg a táboron kivül.
\par 12 Azután megölé az egészen égõáldozatot, Áron fiai pedig odavivék õ hozzá a vért, és õ elhinté azt az oltáron köröskörül.
\par 13 Az egészen égõáldozatot is odavivék hozzá darabonként, a fejével együtt, és elfüstölögteté az oltáron.
\par 14 És megmosá a belet és a lábszárakat, és elfüstölögteté az oltáron az egészen égõáldozattal egybe.
\par 15 Azután megáldozá a nép áldozatát; véve ugyanis a bûnért való áldozat bakját, a mely a népé, és megölé azt, és megáldozá azt bûnért való áldozatul, mint az elõbbit.
\par 16 Azután elõhozá az egészen égõáldozatot, és elkészíté azt szokás szerint.
\par 17 Elõhozá az ételáldozatot is, és võn abból egy teli marokkal, és elfüstölögteté az oltáron a reggeli egészen égõáldozaton kivül.
\par 18 Azután megölé az ökröt és a kost hálaadó áldozatul a népért, és Áron fiai odavivék õ hozzá a vért, és elhinté azt az oltáron köröskörül.
\par 19 Az ökörbõl és a kosból való kövérségeket pedig, a farkat, a béltakarót, a veséket és a máj hártyáját;
\par 20 Odahelyezték e kövérségeket a szegyekre, és elfüstölögteté e kövérségeket az oltáron.
\par 21 De a szegyeket és a jobb lapoczkát meglóbálá Áron az Úr elõtt, a mint parancsolta vala Mózes.
\par 22 Azután felemelé kezeit Áron a népre, és megáldá azt és leszálla, miután elvégezte vala a bûnért való áldozatot, az egészen égõáldozatot és hálaáldozatot.
\par 23 És beméne Mózes és Áron a gyülekezetnek sátorába, azután kijövének és megáldák a népet, az Úrnak dicsõsége pedig megjelenék az egész népnek.
\par 24 Tûz jöve ugyanis ki az Úr elõl, és megemészté az oltáron az égõáldozatot és a kövérségeket. És látá ezt az egész nép, és ujjongának és arczra esének.

\chapter{10}

\par 1 Nádáb pedig és Abihu, Áronnak fiai, vevék egyen-egyen az õ temjénezõjöket, és tõnek azokba szenet és rakának arra füstölõ szert, és vivének az Úr elé idegen tüzet, a melyet nem parancsolt vala nékik.
\par 2 Tûz jöve azért ki az Úr elõl, és megemészté õket, és meghalának az Úr elõtt.
\par 3 És monda Mózes Áronnak: Ez az, a mit szólt vala az Úr, mondván: A kik hozzám közel vannak, azokban kell megszenteltetnem, és az egész nép elõtt megdicsõíttetnem.
\par 4 Áron pedig mélyen hallgata. Szólítá azért Mózes Misáelt és Elsafánt, Uzzielnek, az Áron nagybátyjának fiait, és monda nékik: Jertek ide, vigyétek ki atyátokfiait a szenthely elõl, a táboron kivül.
\par 5 És odamenének, és kivivék õket az õ köntöseikben a táboron kivül, a mint szólott vala Mózes.
\par 6 Azután monda Mózes Áronnak és az õ fiainak, Eleázárnak és Ithamárnak: Fejeteket meg ne meztelenítsétek, ruháitokat meg ne szaggassátok, hogy meg ne haljatok és haragra ne gerjedjen az Úr az egész gyülekezet ellen; a ti atyátokfiai pedig, Izráelnek egész háza sirassák az égést, a melyet égetett az Úr.
\par 7 A gyülekezet sátorának nyílásán se menjetek ki, hogy meg ne haljatok; mert az Úr kenetének olaja van rajtatok. És cselekvének a Mózes beszéde szerint.
\par 8 Áronnak pedig szóla az Úr, mondván:
\par 9 Bort és szeszes italt ne igyatok te és a te fiaid veled, mikor bementek a gyülekezet sátorába, hogy meg ne haljatok. Örökkévaló rendtartás legyen ez a ti nemzetségeitekben.
\par 10 Hogy különbséget tehessetek a szent és közönséges között, a tiszta és tisztátalan között.
\par 11 És hogy taníthassátok Izráel fiait mindazokra a rendelésekre, a melyeket az Úr szólott vala nékik Mózes által.
\par 12 Mózes pedig szóla Áronnak és az õ megmaradt fiainak, Eleázárnak és Ithamárnak: Vegyétek az ételáldozatot, a mely megmaradt az Úrnak tûzáldozatiból, és egyétek meg azt kovásztalan kenyerekkel az oltár mellett; mert igen szentséges az.
\par 13 Azért egyétek azt szent helyen, mert kiszabott részed, és fiaidnak is kiszabott része az, az Úrnak tûzáldozatiból; mert így parancsolta nékem.
\par 14 A meglóbált szegyet pedig, és a felmutatott lapoczkát egyétek meg tiszta helyen, te és a te fiaid és leányaid is veled, mert kiszabott részül adattak azok néked s kiszabott részül a te fiaidnak is, Izráel fiainak hálaadó áldozataiból.
\par 15 A felmutatott lapoczkát és meglóbált szegyet a tûzáldozat kövérségeivel együtt vigyék be, hogy meglóbálják az Úr elõtt, és ez lesz a te kiszabott részed, és veled a te fiaidé örökké, a mint megparancsolta vala az Úr.
\par 16 Azután szorgalmasan tudakozódék Mózes a bûnáldozatra való bak felõl, de ímé elégett vala. Haragra gerjede azért Eleázár és Ithamár ellen, Áronnak megmaradt fiai ellen, mondván:
\par 17 Miért nem ettétek meg a bûnért való áldozatot a szenthelyen? Hiszen igen szentséges az, és néktek adta azt az Úr a gyülekezet vétkének hordozásáért, hogy engesztelést szerezzetek annak az Úr elõtt.
\par 18 Ímé, nem vitetett be annak vére a szenthely belsejébe, meg kellett volna azért ennetek a szenthelyen, a mint megparancsoltam vala.
\par 19 Áron pedig szóla Mózesnek: Ímé ma áldozták meg az õ bûnért való áldozatukat és egészen égõáldozatukat az Úr elõtt, engem pedig ilyen keserûségek értek: Ha megettem volna ma a bûnért való áldozatot, vajjon jó lett volna-é az Úr elõtt?
\par 20 Mikor ezt hallotta vala Mózes, jónak tetszék ez néki.

\chapter{11}

\par 1 Szóla ismét az Úr Mózesnek és Áronnak, mondván nékik:
\par 2 Szóljatok Izráel fiainak, mondván: Ezek azok az állatok, a melyeket megehettek minden barmok közül, a melyek vannak e földön:
\par 3 Mindazt, a minek hasadt a körme, és egészen ketté hasadt körme van, és kérõdzõ a barmok közt, megehetitek.
\par 4 De a kérõdzõk és a hasadt körmûek közül ne egyétek meg ezeket: A tevét, mert az kérõdzõ ugyan, de nincs hasadt körme, tisztátalan ez néktek.
\par 5 A hörcsököt, mert kérõdzõ ugyan, de nem hasadt a körme, tisztátalan ez néktek.
\par 6 A nyulat, mert kérõdzõ ugyan, de nem hasadt a körme; tisztátalan ez néktek.
\par 7 És a disznót, mert hasadt körmû ugyan és egészen ketté hasadt körme van, de nem kérõdzik; tisztátalan ez néktek.
\par 8 Ezeknek húsából ne egyetek, és holttestöket se illessétek; tisztátalan ez néktek.
\par 9 Mindazokból, a melyek a vizekben élnek, ezeket ehetitek meg: A minek úszószárnya és pikkelye van a vizekben, tengerekben és folyóvizekben, azokat mind egyétek meg.
\par 10 A minek pedig nincsen úszószárnya és pikkelye a tengerekben és folyóvizekben, legyen az akármely vízben nyüzsgõ, és akármely vízben élõ állat; mind útálatos az néktek.
\par 11 De legyenek is útálatosak néktek; azoknak húsából ne egyetek, és holttestöket is útáljátok.
\par 12 A minek nincs úszószárnya és pikkelye a vizekben, mind útálatos az néktek.
\par 13 A szárnyas állatok közül pedig ezeket útáljátok: meg ne egyétek, útálatosak ezek: a sas, a saskeselyû és a halászó sas.
\par 14 A sólyom és a héja az õ nemével.
\par 15 Minden holló az õ nemével.
\par 16 A strucz, a bagoly, a kakuk és a karvaly az õ nemével.
\par 17 A kuvik, a hattyú és a füles bagoly.
\par 18 A bölömbika, a pelikán és a gém.
\par 19 Az eszterág és a szarka az õ nemével, a büdös banka és a denevér.
\par 20 Minden szárnyas féreg, a mely négy lábon jár, útálatos néktek.
\par 21 Csak azt ehetitek meg a négylábú szárnyas férgek közül, a melynek lábain felûl szökõ-szárai vannak, hogy szökdécselhessen azokkal a földön.
\par 22 Ezeket egyétek meg azok közül: az arbé-sáskát az õ nemével, a szolám-sáskát az õ nemével, a khargol-sáskát az õ nemével és a khagab-sáskát az õ nemével.
\par 23 Minden egyéb négylábú szárnyas féreg pedig útálatos legyen néktek.
\par 24 És ezekkel tisztátalanokká teszitek magatokat; mindaz, a ki illeti holttestüket, tisztátalan legyen estvéig.
\par 25 Mindaz pedig, a ki hordozza azoknak holttestét, mossa meg az õ ruháit, és tisztátalan legyen estvéig.
\par 26 Minden barom, a melynek hasadt a körme, de nincs egészen ketté hasadva, és nem kérõdzik, tisztátalan legyen néktek; valaki illeti azt, tisztátalan legyen.
\par 27 Minden állat, a mely a négylábúak között a talpán jár, tisztátalan legyen néktek; mindaz, a ki azoknak holttestét illeti, tisztátalan legyen estvéig.
\par 28 A ki pedig hordozza azoknak holttestét, mossa meg az õ ruháit, és tisztátalan legyen estvéig. Tisztátalanok azok néktek.
\par 29 A földön csúszó-mászó állatok között pedig ezek legyenek tisztátalanok: a menyét; az egér és a gyík az õ nemével.
\par 30 A sündisznó, a kaméleon, a tarka gyík, a csiga és a vakondok.
\par 31 Ezek tisztátalanok néktek minden csúszó-mászó között; valaki illeti ezeket holtuk után, tisztátalan legyen estvéig.
\par 32 És minden, a mire ezek közül holtuk után esik valamelyik, tisztátalan legyen; akármely faedény, akár ruha, vagy bõr, vagy zsák; akármely eszköz, a mivel dolgozni szoktak, vízbe tétessék, és tisztátalan legyen estvéig, ezután tiszta legyen.
\par 33 Akármely cserépedény pedig, a melybe beleesik valami azokból, mindazzal együtt, a mi benne van, tisztátalan legyen, és az edényt törjétek el.
\par 34 Minden megehetõ eledel, a melyhez az ilyen edénybõl víz jut, tisztátalan, és minden megiható ital is minden ilyen edényben tisztátalan legyen.
\par 35 És minden, a mire azoknak holttestébõl esik valami, tisztátalan; kemencze és tûzhely lerontassék; tisztátalanok azok és tisztátalanok legyenek néktek.
\par 36 De a forrás, a kút, az egybegyûlt víz tiszta legyen; de a mi azoknak holttesttéhez ér, tisztátalan.
\par 37 Hogyha azoknak holttestébõl ráesik is valamely vetõmagra, a mely elvetendõ, tiszta legyen az.
\par 38 De ha vizet töltenek a magra, és úgy esik rá azoknak a holttesttébõl, tisztátalan az ilyen néktek.
\par 39 Hogyha olyan hullik el a barmok közül, a mely eledeletek néktek; a ki annak holttestét illeti, tisztátalan legyen estvéig.
\par 40 A ki pedig eszik annak holttesttébõl, mossa meg az õ ruháit, és tisztátalan legyen estvéig. És mossa meg ruháit az is, a ki hordozta annak holttestét, és tisztátalan legyen estvéig.
\par 41 Mindaz is, a mi csúszik-mászik a földön, útálatos legyen, meg ne egyétek.
\par 42 Mindazt, a mi hason csúszik, és mindazt, a mi négy, sõt mindazt, a mi több lábon jár, a földön csúszó-mászó bármely állatot, meg ne egyétek ezeket, mert útálatosak ezek.
\par 43 Meg ne fertéztessétek magatokat semmiféle csúszó-mászó állattal, és meg ne tisztátalanítsátok magatokat azokkal, hogy tisztátalanokká legyetek általok.
\par 44 Mert én, az Úr, vagyok a ti Istenetek; szenteljétek meg azért magatokat, és szentek legyetek, mert én szent vagyok, és meg ne tisztátalanítsátok magatokat semmiféle állat által, a mely csúszik-mászik a földön.
\par 45 Mert én vagyok az Úr, a ki felhoztalak titeket Égyiptom földébõl, hogy Istenetekké legyek néktek; legyetek azért szentek, mert én szent vagyok.
\par 46 Ez a törvény a baromfélékrõl, a szárnyas állatokról, minden élõ állatról, a mely nyüzsög a vizekben, és minden állatról, a mely csúszik-mászik a földön.
\par 47 Hogy különbséget tehessetek a tisztátalan és tiszta között, az olyan állat között, a mely megehetõ, és az olyan állat között, a mely meg nem ehetõ.

\chapter{12}

\par 1 Szóla ismét az Úr Mózesnek, mondván:
\par 2 Szólj Izráel fiainak, mondván: Ha az asszony lebetegszik, és fiat szül, tisztátalan legyen hét napig; az õ havi betegségének ideje szerint legyen tisztátalan.
\par 3 A nyolczadik napon pedig metéljék körül a fiú férfitestének bõrét.
\par 4 Azután harminczhárom napig maradjon otthon a vértõl való tisztulás miatt; semmi szent dolgot ne illessen, a szent helyre se menjen be, míg el nem telnek az õ tisztulásának napjai.
\par 5 Ha pedig leányt szül, két hétig legyen tisztátalan, mint havi betegségekor, és hatvanhat napig maradjon otthon a vértõl való tisztulása végett.
\par 6 Mikor pedig letelnek az õ tisztulásának napjai, fiú miatt vagy leány miatt, hozzon egészen égõáldozatul esztendõs bárányt, galambfiat vagy gerliczét bûnért való áldozatul, a gyülekezet sátorának nyílása elé a paphoz.
\par 7 És áldozza meg azt az Úr elõtt; és szerezzen néki engesztelést; így lesz tisztává az õ vérfolyása után. Ez a törvénye a fiút vagy leányt szülõ asszonynak.
\par 8 Ha pedig nincs elég módja bárányhoz, vigyen két gerliczét vagy két galambfiat, egyiket egészen égõáldozatul, a másikat bûnért való áldozatul, és szerezzen néki engesztelést a pap, és tiszta lesz.

\chapter{13}

\par 1 Szóla ismét az Úr Mózesnek és Áronnak, mondván:
\par 2 Ha valamely ember testének bõrén daganat, vagy tarjagosság, vagy fehér folt támad, és az õ testének bõrén poklos fakadékká lehet: vigyétek el az ilyet Áronhoz, a paphoz, vagy egyvalamelyikhez az õ papfiai közül.
\par 3 És nézze meg a pap azt a test bõrén lévõ fakadékot. Ha a szõr a fakadékban fehérré változott, és ha a fakadéknak felülete mélyebben van az õ testének bõrénél: akkor poklos fakadék az. Mihelyt látja ezt a pap, tisztátalannak ítélje azt.
\par 4 Ha pedig fehér folt van a teste bõrén, de annak felülete nincs mélyebben a bõrnél, és a szõre sem változott meg fehérré, akkor rekeszsze külön a pap a fakadékos embert hét napig.
\par 5 A hetedik napon pedig nézze meg õt a pap, s ha szerinte a fakadék egy állapotban van, át nem terjedt tovább a fakadék a bõrön, a pap másodszor is rekeszsze õt külön hét napig.
\par 6 Nézze meg õt azután a pap a hetedik napon másodszor is, és ha a fakadék meghalványodott, és nem terjedt tovább a bõrön a fakadék, tisztának ítélje õt a pap; tarjagosság az, mossa meg azért a ruháit és legyen tiszta.
\par 7 De ha a tarjagosság tovább terjedt a bõrön, miután a papnál az õ megtisztulása végett jelentkezett: akkor másodszor is jelentse magát a papnál.
\par 8 És ha látja a pap, hogy ímé tovább terjedt a tarjagosság a bõrön; ítélje azt a pap tisztátalannak, poklosság az.
\par 9 Ha poklos fakadék lesz az emberen, vigyék tehát azt a paphoz.
\par 10 És ha látja a pap, hogy ímé fehér daganat van a bõrön, és az a szõrt fehérré változtatta, és vadhús van a daganatban
\par 11 Idûlt poklosság az az õ testének bõrén; azért tisztátalannak ítélje azt a pap, ne is rekesztesse külön azt, mert tisztátalan az.
\par 12 Ha pedig folyton fejlõdik a poklosság a bõrön, és a poklos fakadék elborítja a fakadékosnak egész bõrét tetõtõl talpig mindenfelé, a merre a pap szemei látnak;
\par 13 És ha látja a pap, hogy ímé elborította a poklosság annak egész testét: akkor ítélje tisztának a fakadékot; mivelhogy egészen fehérré változott az, tiszta az.
\par 14 De mihelyt vadhús mutatkozik abban, tisztátalan legyen.
\par 15 Ha meglátja a pap a vadhúst, tisztátalannak ítélje azt; a vadhús tisztátalan legyen, poklosság az.
\par 16 De ha a vadhús eltûnik, és fehérré változik, akkor menjen a paphoz.
\par 17 És ha megnézi azt a pap, és a fakadék csakugyan fehérré változott, akkor ítélje a pap a fakadékot tisztának, tiszta az.
\par 18 Ha pedig valakinek a teste bõrén kelevény volt, de begyógyult,
\par 19 És a kelevény helyén fehér daganat, vagy vörhenyes fehér folt támad, jelentse magát a papnál.
\par 20 És ha látja a pap, hogy íme annak felülete alább esik a bõrnél, és a szõre fehérré változott; akkor ítélje azt a pap tisztátalannak, poklos fakadék az, a mi a kelevényben fakadt.
\par 21 Ha pedig megnézi azt a pap, és íme nincs abban fehér szõr, és nem is esett alább a bõrnél, sõt meghalványodott az: akkor rekeszsze külön azt a pap hét napig.
\par 22 Ha azonban tovább terjed a bõrön; akkor tisztátalannak ítélje azt a pap, poklos fakadék az.
\par 23 De ha elébbi állapotában marad a folt, nem terjedt: a kelevény forradása az, azért tisztának ítélje azt a pap.
\par 24 Vagy ha valaki testének bõrén égési seb lesz, és a seb helyén vörhenyesfehér folt támad vagy fehér;
\par 25 És megnézi azt a pap, és íme a szõr fehérré változott a foltban, és annak felülete mélyebben van a bõrnél: poklosság az, a mi a seben fakadt, azért ítélje azt a pap tisztátalannak, poklos fakadék az.
\par 26 De ha megnézi azt a pap, és ímé nincs a foltban fehér szõr, és nem is esett az alább a bõrnél, sõt meg is halványodott az: akkor rekeszsze azt külön a pap hét napig.
\par 27 És nézze meg azt a pap a hetedik napon; ha tovább terjedt a bõrön, akkor tisztátalannak ítélje azt a pap, poklos fakadék az.
\par 28 Ha pedig elébbi állapotában van a folt, nem terjedt a bõrön, sõt meg is halványodott, égési daganat az; azért tisztának ítélje azt a pap, mert égési seb forradása az.
\par 29 És ha valamely férfiúnak vagy asszonynak fakadéka támad a fején vagy a szakállában,
\par 30 És a pap megnézi a fakadékot, és íme annak felülete mélyebben van a bõrnél; és abban a szõr sárga és vékony: akkor ítélje azt a pap tisztátalannak, var az, fejnek vagy szakállnak poklossága az.
\par 31 Ha pedig megnézi a pap a varas fakadékot, és ímé annak felülete nincsen mélyebben a bõrnél, és nincsen abban fekete szõr: akkor rekeszsze külön a pap azt, a kinek varas fakadéka van, hét napig.
\par 32 És nézze meg a pap a fakadékot a hetedik napon, és ha nem terjedt a var, és nem lett benne sárga szõr, és a varnak felülete nem lett mélyebbé a bõrnél:
\par 33 Akkor borotválkozzék meg, de a varat le ne borotválja; a pap pedig másodszor is rekeszsze külön a varas embert hét napig.
\par 34 És a pap nézze meg a varat a hetedik napon, és ha nem terjed a var a bõrön, és annak felülete nincsen mélyebben a bõrnél: akkor tisztátalannak ítélje azt a pap és mossa meg a ruháit, és tiszta lesz.
\par 35 De ha tovább terjed a var a bõrön, miután tisztának ítélte azt,
\par 36 És megnézi azt a pap, és csakugyan terjedt a var a bõrön: akkor ne is kutasson a pap a sárga szõr után, tisztátalan az.
\par 37 De ha a var szerinte egy állapotban van, és fekete szõr indult abban, meggyógyult az a var, tiszta az; tisztának ítélje azt a pap.
\par 38 Ha pedig valamely férfi vagy asszony testének bõrén foltok, fehér foltok támadnak,
\par 39 És megnézi a pap, és ímé a testök bõrén lévõ foltok halavány fehérek: a bõrön fakadt sömör az, tiszta az az ember.
\par 40 Hogyha valamely embernek egészen elhull a haja, kopasz az, és tiszta az.
\par 41 És ha elõlrõl hull el a haja, elõl kopasz az, és tiszta az.
\par 42 Hogyha pedig az õ egész kopasz, vagy elõl kopasz fején vörhenyes fehér fakadék támad, poklos fakadék az az õ egész kopasz vagy elõl kopasz fején.
\par 43 Ha megnézi azt a pap, és ímé a fakadék daganata vörhenyes fehér az õ egész kopasz vagy elõl kopasz fején, olyanforma, mint a test bõrén való poklosság:
\par 44 Poklos ember az, tisztátalan az, igen tisztátalannak ítélje azt a pap, mivelhogy fején van a fakadékja.
\par 45 A poklos ember pedig, a kin a fakadék van, megszaggatott ruhában és mezítelen fõvel legyen, és a bajuszát fedezze be, és ezt kiáltsa: Tisztátalan, tisztátalan!
\par 46 Mindaddig tisztátalan legyen, a míg rajta van a fakadék, tisztátalan az; csak õ maga lakjék, a táboron kivül legyen az õ lakása.
\par 47 Hogyha pedig valami ruhán van a poklos fakadék, gyapjú ruhán vagy len ruhán,
\par 48 Vagy lenbõl és gyapjúból készült fonadékon vagy szöveten; vagy bõrön, vagy valamely bõrbõl való készítményen;
\par 49 És ha az a fakadék zöld vagy vörhenyes színû a ruhán, vagy bõrön, vagy szöveten, vagy fonadékon, vagy akármely bõrbõl való eszközön: poklos fakadék az, mutassák azért meg a papnak.
\par 50 És nézze meg a pap a fakadékot és rekeszsze külön azt a min a fakadék van, hét napig.
\par 51 A hetedik napon pedig nézze meg a fakadékot. Ha terjed a fakadék a ruhán vagy szöveten, vagy fonadékon vagy bõrön, és akármilyen készítményen, a mivé a bõr feldolgoztatik: emésztõ poklosság az a fakadék, tisztátalan az.
\par 52 Azért égesse meg azt a ruhát, vagy a gyapjúból vagy a lenbõl készült szövetet vagy fonadékot, vagy akármely bõrbõl való eszközt, a melyen a fakadék leend; mert emésztõ poklosság az, tûzön égessék meg azt.
\par 53 De ha megnézi a pap, és ímé nem terjedt a fakadék a ruhán, vagy szöveten, vagy fonadékon, vagy akármely bõrbõl való eszközön:
\par 54 Akkor parancsolja meg a pap, hogy mossák meg azt, a min a fakadék van, és rekeszsze külön azt másodszor is hét napig.
\par 55 Ha pedig megnézi azt a pap a mosás után, és ímé a fakadék nem változtatta meg a színét; ha nem terjedt is a fakadék, tisztátalan az, tûzben égesd meg azt; beevõdés az, akár a fonákján, akár a színén legyen az.
\par 56 De ha látja a pap, hogy ímé meghalványodék a fakadék a megmosatása után: akkor szakaszsza el azt a ruhától vagy bõrtõl, vagy a szövettõl vagy a fonadéktól.
\par 57 És ha mégis mutatkozik a ruhán vagy szöveten, vagy fonadékon, vagy akármely bõrbõl való eszközön: akkor kiújulás az; tûzzel égesd meg azt, a min a fakadék van.
\par 58 Azt a ruhát pedig, vagy szövetet, vagy fonadékot, vagy akármely bõrbõl való eszközt, a melyet megmostál, ha eltávozik rólok a fakadék, mosd meg másodszor is, és tiszta lesz.
\par 59 Ez a törvénye a poklos fakadéknak, akár gyapjú-, akár lenruhán, akár szöveten, akár fonadékon, vagy akármely bõrbõl való eszközön legyen, hogy tisztának vagy tisztátalannak ítéltessék.

\chapter{14}

\par 1 Szóla ismét az Úr Mózesnek, mondván:
\par 2 Ez legyen a poklos embernek törvénye az õ megtisztulásának napján, hogy vigyék a paphoz.
\par 3 A pap pedig menjen ki a táboron kivül, és nézze meg a pap, és ha meggyógyult a pokloson a poklos fakadék:
\par 4 Akkor parancsolja meg a pap, hogy hozzanak a megtisztulandó emberért két élõ, tiszta madarat, czédrusfát, karmazsint és izsópot;
\par 5 Azután parancsolja meg a pap, hogy az egyik madarat öljék meg cserépedényben, forrásvíz felett.
\par 6 Az élõ madarat pedig, vegye azt és a czédrusfát, a karmazsint és az izsópot; és mártsa be azokat és az élõ madarat a megölt madárnak vérébe a a forrásvíz felett.
\par 7 És hintse meg a poklosságból megtisztulandó embert hétszer, és ítélje azt tisztának; az élõ madarat pedig bocsássa szabadon a mezõre.
\par 8 Azután a megtisztulandó ember mossa ki az õ ruháit, borotválja le minden szõrét, és mosódjék meg vízben, és tiszta lesz; így menjen be azután a táborba, de a sátorán kivül maradjon hét napig.
\par 9 A hetedik napon pedig borotválja le minden szõrét: a haját, szakállát, szemöldökeit és minden egyéb szõrét borotválja le, és mossa ki ruháit, és mossa meg a testét vízben, és tiszta lesz.
\par 10 A nyolczadik napon pedig vegyen két ép bárányt, meg egy ép nõstény bárányt, egy esztendõst, és olajjal elegyített három tized efa lisztlángot ételáldozatul, és egy lóg olajt.
\par 11 A pap pedig, a ki a tisztítást végzi, állassa a megtisztulandó embert mindezekkel együtt az Úr elébe, a gyülekezet sátorának nyílásához.
\par 12 És vegye a pap az egyik bárányt, és áldozza meg azt vétekért való áldozatul a lóg olajjal együtt, és lóbálja meg azokat az Úr elõtt.
\par 13 A bárányt pedig ott ölje meg, a hol a bûnért való áldozatot és az egészen égõáldozatot ölik meg a szent helyen; mert a miképen a bûnért, azonképen a vétekért való áldozat is a papé; igen szentséges az.
\par 14 És vegyen a pap a vétekért való áldozatnak vérébõl, és kenje meg azzal a megtisztulandó ember jobb fülének czimpáját, és az õ jobb kezének hüvelykét, és jobb lábának hüvelykét.
\par 15 Vegyen a pap a lóg olajból is, és töltsön a papnak a bal tenyerére.
\par 16 És mártsa be a pap az õ jobb kezének újját az olajba, mely az õ bal tenyerén van, és hintsen az olajból az újjával hétszer az Úr elõtt.
\par 17 A maradék olajból pedig, a mely az õ tenyerén van, kenje meg a pap a megtisztulandó ember jobb fülének czimpáját, a jobb kezének hüvelykét és a jobb lábának hüvelykét a vétekért való áldozat vérén felül.
\par 18 És a mi megmarad az olajból, a mely a pap tenyerén van, kenje a megtisztulandó ember fejére; így szerezzen néki engesztelést a pap az Úr elõtt.
\par 19 Azután készítsen a pap bûnért való áldozatot, és szerezzen engesztelést a tisztátalanságából megtisztulónak; azután ölje meg az égõáldozatot.
\par 20 És vigye fel a pap az égõáldozatot és az ételáldozatot az oltárra; így szerezzen néki engesztelést a pap, és tiszta lesz.
\par 21 Hogyha pedig szegény õ, és nincs módja azokhoz, akkor vegyen egy bárányt vétekért való áldozatul meglóbálás végett, hogy engesztelésül legyen érte; meg egy tized efa lisztlángot, olajjal elegyítve, ételáldozatul, és egy lóg olajt.
\par 22 És két gerliczét vagy két galambfiat, a mint a módja engedi, és legyen az egyik bûnért való áldozat, a másik pedig egészen égõáldozat.
\par 23 És vigye azokat az õ tisztulásának nyolczadik napján a paphoz, a gyülekezet sátorának nyílásához, az Úr elébe.
\par 24 És vegye a pap a vétekért való áldozat bárányát, és a lóg olajt, és lóbálja meg azokat a pap az Úr elõtt.
\par 25 Azután ölje meg a vétekért való áldozat bárányát, és vegyen a pap a vétekért való áldozat vérébõl, és kenjen a megtisztulandó ember jobb fülének czimpájára, és jobb kezének hüvelykére, és jobb lábának hüvelykére.
\par 26 Az olajból pedig töltsön a pap a papnak baltenyerére.
\par 27 És hintsen a pap az õ jobb kezének újjával az olajból, a mely az õ bal tenyerén van, hétszer az Úr elõtt.
\par 28 Azután kenje meg a pap a tenyerén levõ olajból a megtisztulandó jobb fülének czimpáját, a jobb kezének hüvelykét és a jobb lábának hüvelykét a vétekért való áldozat vérének helyén.
\par 29 A mi pedig megmarad a pap tenyerén levõ olajból, kenje a megtisztulandó fejére, hogy engesztelésül legyen érette az Úr elõtt.
\par 30 És készítse el az egyiket a gerliczék közül, vagy galambfiak közül, a melyiket az õ módjától telik.
\par 31 Azt, a mi kitelik az õ módjától: az egyiket bûnért való áldozatul, a másikat pedig egészen égõáldozatul az ételáldozattal egybe; így szerezzen engesztelést a pap a megtisztulandó embernek az Úr elõtt.
\par 32 Ez a törvénye annak, a kin poklos fakadék van, de a kinek nincs módja az õ megtisztulásánál.
\par 33 Szóla ismét az Úr Mózesnek és Áronnak, mondván:
\par 34 Mikor bementek majd a Kanaán földére, a melyet én adok néktek birtokul, és a ti birtokotokban levõ föld valamelyik házára poklosságot bocsátok:
\par 35 Akkor menjen el az, a kié a ház, és jelentse meg a papnak, mondván: Mint a poklosság, olyan mutatkozik nálam a házban.
\par 36 A pap pedig parancsolja meg, hogy takarítsák ki a házat, mielõtt oda menne a pap a poklosság megnézésére, hogy semmi se legyen tisztátalanná, a mi a házban van; és csak azután menjen be a pap a ház megnézésére.
\par 37 És ha látja a poklosságot, hogy ímé a poklosság a háznak falain zöld vagy vörhenyes horpadásokban mutatkozik; és annak felülete alább esik a falnál:
\par 38 Akkor menjen ki a pap a házból, a háznak ajtaja elé, és zárja be a házat hét napra.
\par 39 A hetedik napon pedig térjen vissza a pap, és ha látja, hogy ímé elterjedt a poklosság a ház falán:
\par 40 Akkor parancsolja meg a pap, hogy szedjék ki a köveket, a melyeken a poklosság van, és vessék azokat a városon kivül tisztátalan helyre;
\par 41 A házat pedig vakartassa le belül köröskörül, és a tapasztékot, a melyet levakartak, töltsék a városon kivül tisztátalan helyre.
\par 42 És vegyenek elõ más köveket, és illeszszék be ama kövek helyére; tapasztékot is mást vegyenek, és tapasszák be a házat.
\par 43 Hogyha a poklosság visszatér, és kiújul a házon, miután kiszedték a köveket, és miután levakarták a házat, és miután be is tapasztották azt:
\par 44 Akkor menjen be a pap, és nézze meg, és ha ímé tovább terjedt a poklosság a házon: emésztõ poklosság az a házon, tisztátalan az;
\par 45 Rontsák azért le a házat köveivel együtt, és a fáit is, a háznak minden tapasztékát is; és vigyék a városon kívül tisztátalan helyre.
\par 46 A ki pedig bemegy a házba akármikor, a míg az zárva van, tisztátalan legyen az estvéig.
\par 47 És a ki meghál abban a házban, mossa meg a ruháit, és a ki eszik abban a házban, az is mossa meg a ruháit.
\par 48 Ha pedig bemegy a pap és látja, hogy ímé nem terjedt a poklosság a házon, miután megtapasztották a házat: akkor tisztának ítélje a pap a házat, mert megszûnt a poklosság.
\par 49 A háznak megtisztítása végett pedig vegyen elõ két madarat, czédrusfát, karmazsint és izsópot.
\par 50 És ölje meg az egyik madarat cserépedényben, forrásvíz felett.
\par 51 Azután vegye a czédrusfát, az izsópot, a karmazsint és az élõ madarat, és mártsa be azokat a megölt madár vérébe és a forrásvízbe, és hintse meg azzal a házat hétszer.
\par 52 És tisztítsa meg a házat a madár vérével, a forrásvízzel, az élõ madárral, a czédrusfával, az izsóppal és a karmazsinnal.
\par 53 Az élõ madarat pedig bocsássa el a városon kivül a mezõre, így szerezzen engesztelést a házért, és tiszta lesz.
\par 54 Ez a törvénye mindenféle poklos fakadéknak és varnak.
\par 55 A ruha és a ház poklosságának is.
\par 56 A daganatnak, a tarjagosságnak és a fehér foltnak;
\par 57 Hogy megtudhassák: mikor tiszta és mikor tisztátalan valami? Ez a poklosság törvénye.

\chapter{15}

\par 1 Szóla ismét az Úr Mózesnek s Áronnak, mondván:
\par 2 Szóljatok Izráel fiainak és mondjátok meg nékik: Ha valamely férfiúnak magfolyása támad, az õ folyása tisztátalan.
\par 3 És pedig tisztátalan lesz e folyás miatt: akár folytonos az õ magfolyása, akár megreked testében ez a folyás; tisztátalanság ez õ nála.
\par 4 Minden ágy, a melyen fekszik a magfolyós, tisztátalan, és minden holmi is, a melyre ráül, tisztátalan lesz.
\par 5 Valaki azért illeti az õ ágyát, mossa meg a ruháit, és mosódjék meg vízben, és tisztátalan legyen estvéig.
\par 6 Az is, aki a holmira ül, a melyen a magfolyós ült vala, mossa meg ruháit, és mosódjék meg vízben, és tisztátalan legyen estvéig.
\par 7 Az is, a ki illeti a magfolyósnak testét, mossa meg a ruháit, és mosódjék meg vízben, és tisztátalan legyen estvéig.
\par 8 És ha ráköp a magfolyós a tiszta emberre, mossa meg ez a ruháját, és mosódjék meg vízben, és tisztátalan legyen estvéig.
\par 9 És minden nyereg is, a melyre a magfolyós ráült, tisztátalan legyen.
\par 10 És akárki is, a ki illet valamit, a mi annak alatta vala, tisztátalan legyen estvéig, és a ki hordozza azokat, mossa meg ruháit, és mosódjék meg vízben, és tisztátalan legyen estvéig.
\par 11 És mindaz, a kit illet a magfolyós, úgy hogy kezeit le nem öblíti vízzel, mossa meg ruháit, és mosódjék meg vízben, és tisztátalan legyen estvéig.
\par 12 A cserépedény pedig, a melyet a magfolyós illet, törettessék el, minden faedény pedig öblíttessék ki vízzel.
\par 13 Mikor pedig megtisztul a magfolyós az õ folyásából, akkor számláljon hét napot az õ tisztulására, és mossa meg ruháit, és a testét is mossa le forrásvízben, és tiszta lesz.
\par 14 És a nyolczadik napon vegyen elõ két gerliczét vagy két galambfiat, és menjen el az Úr elé, a gyülekezet sátorának nyílásához, és adja azokat a papnak.
\par 15 És készítse el azokat a pap; az egyiket bûnért való áldozatul, a másikat egészen égõáldozatul; így szerezzen néki engesztelést a pap az Úr elõtt az õ magfolyása miatt.
\par 16 Ha valamely férfiúnak magömlése van, mossa meg az egész testét vízben, és tisztátalan legyen estvéig.
\par 17 És minden ruha, és minden bõr, a melyre a magömlés kihat, mosattassék meg vízben, és tisztátalan legyen estvéig,
\par 18 És az asszony is, a kivel férfiú hál magömléssel. Mosódjanak meg vízben, és tisztátalanok legyenek estvéig.
\par 19 Mikor asszony lesz magfolyóssá és véressé lesz az õ magfolyása a testén, hét napig legyen az õ havi bajában, és valaki illeti azt, tisztátalan legyen estvéig.
\par 20 Mindaz is, a min hál az õ havi bajában, tisztátalan legyen és mindaz is, a min ül, tisztátalan legyen.
\par 21 És mindaz, a ki illeti az õ ágyát, mossa meg ruháit, és mosódjék meg vízben, és tisztátalan legyen estvéig.
\par 22 És mindaz is, a ki illet bármely holmit, a melyen ült, mossa meg ruháit, és mosódjék meg vízben, és tisztátalan legyen estvéig.
\par 23 Sõt ha valaki az õ ágyán, vagy a holmikon illet is valamit, a melyeken õ ült, tisztátalan legyen estvéig.
\par 24 Ha pedig vele hál valaki, és reá ragad arra az õ havi baja: tisztátalan legyen hét napig, és minden ágy is, a melyen fekszik, tisztátalan legyen.
\par 25 És hogyha sok napig tart az asszonynak az õ vérfolyása a havi bajának idején kivül, vagy ha a folyás a havi bajon túl tart: valameddig az õ tisztátalanságának folyása tart, úgy legyen, mint havi bajának idején, tisztátalan az.
\par 26 Minden ágy, a melyen fekszik az õ folyásának egész ideje alatt, olyan legyen, mint a havi baja idejében lévõ ágya, és minden holmi is, a melyre ráül, tisztátalan legyen, mint az õ havi bajának tisztátalansága miatt.
\par 27 És mindaz is, a ki illeti azokat, tisztátalan lesz, mossa meg azért a ruháit, és mosódjék meg vízben, és tisztátalan legyen estvéig.
\par 28 Ha pedig megtisztul az õ folyásából: számláljon hét napot, és azután tiszta legyen.
\par 29 A nyolczadik napon pedig vegyen elõ két gerliczét vagy két galambfiat, és vigye el azokat a papnak, a gyülekezet sátorának nyílásához.
\par 30 És készítse el a pap az egyiket bûnért való áldozatul, a másikat pedig egészen égõáldozatul; így szerezzen néki engesztelést a pap az Úr elõtt az õ tisztátalanságának folyása miatt.
\par 31 Így tartsátok vissza Izráel fiait az õ tisztátalanságuktól, hogy meg ne haljanak az õ tisztátalanságuk miatt, megfertõztetvén az én hajlékomat, a mely közöttök van.
\par 32 Ez a törvénye a magfolyósnak, és annak, a kinek magömlése van, a mi által tisztátalanná lesz;
\par 33 És a havi bajban szenvedõnek és a magfolyásban lévõnek, férfinak és asszonynak, és annak a férfiúnak a ki tisztátalan asszonynyal hál.

\chapter{16}

\par 1 És szóla az Úr Mózesnek, az Áron két fiának halála után, a kik akkor haltak meg, a mikor az Úrhoz járultak vala.
\par 2 És monda az Úr Mózesnek: Szólj a te atyádfiának, Áronnak, hogy ne menjen be akármikor a szenthelyre a függönyön belül a fedél elé, a mely a láda felett van, hogy meg ne haljon, mert felhõben jelenek meg a fedél felett.
\par 3 Ezzel menjen be Áron a szenthelyre: egy fiatal tulokkal bûnért való áldozatul, és egy kossal égõáldozatul.
\par 4 Gyolcsból készült szent köntöst öltsön magára, és gyolcs lábravaló legyen a testén, gyolcs övvel övezze be magát, és gyolcs süveget tegyen fel; szent ruhák ezek; mossa meg azért a testét vízben, és úgy öltse fel ezeket.
\par 5 Izráel fiainak gyülekezetétõl pedig vegyen át két kecskebakot bûnért való áldozatul, és egy kost egészen égõáldozatul.
\par 6 És áldozza meg Áron a bûnért való áldozati tulkot, a mely az övé, és végezzen engesztelést magáért és háza népéért.
\par 7 Azután vegye elõ a két kecskebakot, és állassa azokat az Úr elé a a gyülekezet sátorának nyílásához,
\par 8 És vessen sorsot Áron a két fiatal bakra; egyik sorsot az Úrért, a másik sorsot Azázelért.
\par 9 És áldozza meg Áron azt a bakot, a melyre az Úrért való sors esett, és készítse el azt bûnért való áldozatul.
\par 10 Azt a bakot pedig, a melyre az Azázelért való sors esett, állassa elevenen az Úr elé, hogy engesztelés legyen általa, és hogy elküldje azt Azázelnek a pusztába.
\par 11 Áron pedig úgy áldozza meg a bûnért való áldozati tulkot, a mely az övé, és úgy szerezzen engesztelést magáért és háza népéért, hogy ölje meg a bûnért való áldozati tulkot, a mely az övé.
\par 12 És vegye tele a tömjénezõt eleven szénnel az oltárról, a mely az Úr elõtt van, és vegye tele a két markát a porrá tört fûszerekbõl való füstölõbõl, és vigye be a függönyön belõl.
\par 13 És vesse a füstölõt a tûzre az Úr elõtt, hogy befedje a füstölõ felhõje a fedelet, a mely a bizonyság felett van, hogy meg ne haljon.
\par 14 Azután vegyen a tuloknak vérébõl és hintsen újjával a fedél felsõ színére napkelet felé; a fedél elõtt pedig hétszer hintsen újjával a vérbõl.
\par 15 És ölje meg a bûnért való áldozati bakot, a mely a népé, és vigye be annak vérét a függönyön belõl, és úgy cselekedjék annak vérével, a mint a tuloknak vérével cselekedett: hintse ugyanis azt a fedélre és a fedél elé.
\par 16 Így szerezzen engesztelést a szenthelynek Izráel fiainak tisztátalanságai és vétkei miatt; mindenféle bûnei miatt; így cselekedjék a gyülekezet sátorával is, a mely közöttök van, az õ tisztátalanságaik közepette.
\par 17 Senki se legyen a gyülekezet sátorában, a mikor bemegy a szenthelybe, hogy engesztelést szerezzen, egészen az õ kijöveteléig; és végezzen engesztelést magáért, házanépéért, és Izráelnek egész gyülekezetéért.
\par 18 Azután menjen ki az oltárhoz, a mely az Úr elõtt van, és végezzen engesztelést azért is; vegyen ugyanis a tuloknak vérébõl és a baknak vérébõl, és kenje meg az oltárnak szarvait köröskörül.
\par 19 És hintsen arra a vérbõl az õ újjával hétszer; így tegye tisztává, és így szentelje meg azt Izráel fiainak tisztátalanságaitól.
\par 20 Miután pedig elvégezi a szenthelyért, a gyülekezet sátoráért és az oltárért való engesztelést; hozza elõ az élõ bakot.
\par 21 És tegye Áron mind a két kezét az élõ baknak fejére, és vallja meg felette Izráel fiainak minden hamisságát és minden vétkét, mindenféle bûneit: és rakja azokat a baknak fejére, azután küldje el az arravaló emberrel a pusztába,
\par 22 Hogy vigye el magán a bak minden õ hamisságukat kietlen földre, és hogy bocsássa el a bakot a pusztában.
\par 23 Azután menjen be Áron a gyülekezet sátorába, és vesse le a gyolcs ruhákat, a melyeket felöltött, mikor bement a szenthelybe, és hagyja ott azokat.
\par 24 És mossa meg a testét vízben szent helyen, és öltse fel a maga ruháit, úgy menjen ki, és készítse el a maga egészen égõáldozatát és a nép egészen égõáldozatát, és végezzen engesztelést magáért és a népért.
\par 25 A bûnért való áldozat kövérjét pedig füstölögtesse el az oltáron.
\par 26 Az pedig, a ki elvitte az Azázelnek való bakot, mossa meg ruháit, és a testét is mossa le vízben, és azután menjen be a táborba.
\par 27 A bûnért való áldozati tulkot pedig, és a bûnért való áldozati bakot, a melyeknek vére engesztelés végett bevitetett a szenthelyre, vigye ki a táboron kivül, és égessék meg azoknak bõrét, húsát és ganéját tûzzel.
\par 28 És a ki elégeti ezeket, mossa meg ruháit, és a testét is mossa le vízben, és azután így menjen be a táborba.
\par 29 Örökkévaló rendtartás legyen ez nálatok: a hetedik hónapban, a hónapnak tizedikén sanyargassátok meg magatokat és semmi munkát ne végezzetek, se a benszülött, se a közöttetek tartózkodó jövevény.
\par 30 Mert ezen a napon engesztelés lesz értetek, hogy megtisztítson titeket; minden bûnötöktõl megtisztultok az Úr elõtt.
\par 31 Szombatok szombatja ez néktek, sanyargassátok meg azért magatokat; örökkévaló rendtartás ez.
\par 32 És végezzen engesztelést a pap, a kit felkennek, és a kit az õ tisztére felavatnak, hogy paposkodjék az õ atyja helyett, és öltözködjék a gyolcs ruhákba, a szent ruhákba:
\par 33 És végezzen engesztelést a szentek szentjéért, és a gyülekezet sátoráért, és az oltárért is végezzen engesztelést, sõt a papokért és az egész összegyülekezett népért is engesztelést végezzen.
\par 34 És örökkévaló rendtartás legyen ez nálatok, hogy egyszer egy esztendõben engesztelést végezzenek Izráel fiainak minden bûnéért. És úgy cselekedék, a mint megparancsolta vala az Úr Mózesnek.

\chapter{17}

\par 1 Szóla ismét az Úr Mózesnek, mondván:
\par 2 Szólj Áronnak és az õ fiainak és Izráel minden fiának, és mondd nékik: Ez az a dolog, a mit megparancsolt az Úr, mondván:
\par 3 Ha valaki Izráel házából ökröt, vagy bárányt, vagy kecskét öl le a táborban, vagy a ki öl a táboron kivül,
\par 4 És nem viszi azt a gyülekezet sátorának nyílásához, hogy áldozattal járuljon az Úrhoz, az Úrnak hajléka elõtt: vérontásul tulajdoníttassék az annak az embernek; vért ontott, töröltessék ki azért az ilyen ember az õ népe közûl:
\par 5 Azért hogy vigyék el Izráel fiai az õ véres áldozataikat, a melyeket áldoznak vala a mezõn, vigyék el azokat az Úrnak, a gyülekezet sátorának nyílásához, a paphoz, és áldozzák meg azokat hálaáldozatul az Úrnak.
\par 6 És hintse a pap a vért az Úr oltárára, a mely a gyülekezet sátorának nyílásánál van, a kövérjét pedig füstölögtesse el kedves illatul az Úrnak.
\par 7 És ne áldozzák többé véres áldozataikat az ördögöknek, a kikkel õk paráználkodnak. Örökkévaló rendtartás legyen ez nékik nemzetségrõl nemzetségre.
\par 8 Mondjad nékik ezt is: Valaki az Izráel házából, vagy a köztök tartózkodó jövevények közül, egészen égõáldozatot áldoz vagy véres áldozatot,
\par 9 És nem viszi azt a gyülekezet sátorának nyílásához, hogy elkészítse azt az Úrnak: irtassék ki az ilyen ember az õ népe közül.
\par 10 És ha valaki Izráel házából, vagy a köztök tartózkodó jövevények közül valamiféle vért megeszik: kiontom haragomat az ellen, a ki a vért megette és kiirtom azt az õ népei közül.
\par 11 Mert a testnek élete a vérben van, én pedig az oltárra  adtam azt néktek, hogy engesztelésül legyen a ti életetekért, mert a vér a benne levõ élet által szerez engesztelést.
\par 12 Azért mondom Izráel fiainak: Egy lélek se egyék vért közületek; a köztetek tartózkodó jövevény se egye meg a vért.
\par 13 És ha valaki Izráel fiai közül, vagy a köztök tartózkodó jövevények közül vadászásban vadat vagy madarat fog, a mely megehetõ: ontsa ki annak vérét, és fedje be azt földdel.
\par 14 Mert minden testnek élete az õ vére a benne levõ élettel. Azért mondom Izráel fiainak: Semmiféle testnek a vérét meg ne egyétek, mert minden testnek élete az õ vére; valaki megeszi azt, irtassék ki.
\par 15 Ha pedig valaki elhullott, vagy vadtól megszaggatott állatot eszik, akár benszülött, akár jövevény: mossa meg ruháit, és mosódjék meg vízben, és tisztátalan legyen estvéig, azután tiszta.
\par 16 Hogyha meg nem mossa ruháit, sem a testét le nem mossa: viselje az õ vétségének terhét.

\chapter{18}

\par 1 Szóla ismét az Úr Mózesnek, mondván:
\par 2 Szólj Izráel fiaihoz, és mondd nékik: Én vagyok az Úr, a ti Istenetek.
\par 3 Ne cselekedjetek úgy, a mint Égyiptom földén cselekesznek, a hol laktatok; úgy se cselekedjetek, a mint Kanaán földén cselekesznek, a hová beviszlek titeket; se azoknak rendtartásai szerint ne járjatok.
\par 4 Az én végzéseim szerint cselekedjetek, és az én rendeleteimet tartsátok meg, azok szerint járván. Én vagyok az Úr, a ti Istenetek.
\par 5 Tartsátok meg azért az én rendeleteimet és az én végzéseimet, a melyeket ha megcselekszik az ember, él azok által. Én vagyok az Úr.
\par 6 Senki se közelgessen valamely vér szerint való rokonához, hogy felfedje annak szemérmét. Én vagyok az Úr.
\par 7 A te atyádnak szemérmét és a te anyádnak szemérmét fel ne fedd; a te anyád õ, fel ne fedd az õ szemérmét.
\par 8 A te atyád feleségének szemérmét fel ne fedd, a te atyádnak szemérme az.
\par 9 A te atyád leányának, vagy a te anyád leányának, a te leánytestvérednek szemérmét, akár otthon született, akár kivül született legyen; fel ne fedd szemérmöket.
\par 10 A te fiad leányának szemérmét, vagy a te leányod leányáét, ezeknek szemérmét fel ne fedd, mert a te szemérmeid azok.
\par 11 A te atyád felesége leányának szemérmét, a ki a te atyádnak magzatja, leánytestvéred õ, fel ne fedd ennek szemérmét.
\par 12 A te atyád leánytestvérének szemérmét fel ne fedd, a te atyádnak vér szerint való rokona õ.
\par 13 A te anyád leánytestvérének szemérmét fel ne fedd, mert a te anyádnak vér szerint való rokona õ.
\par 14 A te atyád fiútestvérének szemérmét fel ne fedd, annak feleségéhez ne közelgess, nagynénéd õ.
\par 15 A te menyednek szemérmét fel ne fedd; a te fiadnak felesége õ: ne fedd fel az õ szemérmét.
\par 16 A te fiútestvéred feleségének szemérmét fel ne fedd; a te fiútestvérednek szemérme az.
\par 17 Valamely asszonynak és az õ leányának szemérmét fel ne fedd; az õ fiának leányát, vagy leányának leányát el ne vedd, hogy annak szemérmét felfedjed; mert vér szerint való rokonok õk; fajtalankodás ez.
\par 18 De feleségül se végy senkit az õ leánytestvére mellé, hogy ellenkezés ne legyen, ha felfeded õ mellette amannak szemérmét az õ életében.
\par 19 Asszonyhoz ne közelgess, az õ havi tisztátalansága alatt, hogy felfedje az õ szemérmét.
\par 20 És a te felebarátodnak feleségéhez se add magad közösülésre, hogy azzal magadat megfertõztessed.
\par 21 A te magzatodból ne adj, hogy oda áldozzák a Moloknak, és meg ne szentségtelenítsd a te Istenednek nevét. Én vagyok az Úr.
\par 22 Férfiúval ne hálj úgy, a mint asszonynyal hálnak: útálatosság az.
\par 23 És semmiféle barommal se közösülj, hogy azzal magadat megfertõztessed, és asszony se álljon meg barom elõtt, hogy meghágja õt; fertelmesség az.
\par 24 Egyikkel se fertõztessétek meg magatokat ezek közül; mert mindezekkel ama pogányok fertõztették meg magokat, kiket én kiûzök ti elõletek.
\par 25 És fertõzötté lett az a föld, de meglátogatom azon az õ gonoszságát, mert kiokádja az a föld az õ lakosait.
\par 26 Tartsátok meg azért ti az én rendeléseimet és végzéseimet, és ez útálatosságok közül semmit meg ne cselekedjetek, se a benszülött, se a közöttetek tartózkodó jövevény:
\par 27 (Mert mindezeket az útálatosságokat megcselekedték annak a földnek lakosai, a mely elõttetek van; és fertelmessé lõn az a föld),
\par 28 Hogy ki ne okádjon titeket az a föld, ha megfertõztetitek azt, a mint kiokádja azt a népet, a mely elõttetek van.
\par 29 Mert a ki megcselekszik valamit ez útálatosságokból, mind kiirtatik az így cselekvõ ember az õ népe közül.
\par 30 Tartsátok meg azért a mit én megtartani rendelek, hogy egyet se kövessetek amaz útálatos szokásokból, a melyeket követtek ti elõttetek, és meg ne fertõztessétek magatokat azokkal. Én, az Úr, vagyok a ti Istenetek.

\chapter{19}

\par 1 Szóla ismét az Úr Mózesnek, mondván:
\par 2 Szólj Izráel fiainak egész gyülekezetéhez, és mondd nékik: Szentek legyetek mert én az Úr, a ti Istenetek szent vagyok.
\par 3 Az õ anyját és atyját minden ember tisztelje, és az én  szombatjaimat megtartsátok. Én vagyok az Úr, a ti Istenetek.
\par 4 Ne hajoljatok a bálványokhoz, és ne csináljatok magatoknak öntött isteneket. Én vagyok az Úr, a ti Istenetek.
\par 5 Hogyha hálaadó áldozatot áldoztok az Úrnak, úgy áldozzátok, hogy kedvesen fogadtassatok.
\par 6 A ti áldozástok napján és a következõn egyétek meg; a mi pedig harmadnapra marad, égessétek meg tûzben.
\par 7 Ha pedig harmadnapra eszik valaki abból, útálatos az, nem lehet kedves.
\par 8 És a ki eszi azt, viselje az õ álnokságának terhét; mivelhogy megfertõztette az Úrnak szentségét, irtassék ki az ilyen ember az õ népe közül.
\par 9 Mikor a ti földetek termését learatjátok, ne arasd le egészen a te mezõdnek szélét, és az elhullott gabonafejeket fel ne szedd.
\par 10 Szõlõdet se mezgéreld le, és elhullott szemeit se szedd fel szõlõdnek, a szegénynek és a jövevénynek hagyd meg azokat. Én vagyok az Úr, a ti Istenetek.
\par 11 Ne orozzatok, se ne hazudjatok és  senki meg ne csalja az õ felebarátját.
\par 12 És ne esküdjetek hamisan az én nevemre, mert megfertõzteted a te Istenednek nevét. Én vagyok az Úr.
\par 13 A te felebarátodat ne zsarold, se ki ne rabold. A napszámos bére ne maradjon nálad reggelig.
\par 14 Siketet ne szidalmazz, és vak elé gáncsot ne vess; hanem félj a te Istenedtõl. Én vagyok az Úr.
\par 15 Ne kövessetek el igazságtalanságot az ítéletben; ne nézd a szegénynek személyét, se a hatalmas személyét ne becsüld; igazságosan ítélj a te felebarátodnak.
\par 16 Ne járj rágalmazóként a te néped között; ne támadj fel a te felebarátodnak vére ellen. Én vagyok az Úr.
\par 17 Ne gyûlöld a te atyádfiát szívedben; fedd meg  a te felebarátodat nyilván, hogy ne viseljed az õ bûnének terhét.
\par 18 Bosszúálló ne légy, és haragot ne tarts a te néped fiai ellen, hanem szeressed  felebarátodat, mint magadat. Én vagyok az Úr.
\par 19 Az én rendeléseimet megtartsátok: Barmodat másféle állattal ne párosítsd, szántóföldedbe kétféle magot ne vess, és kétféle szövetû ruha ne legyen rajtad.
\par 20 És ha valaki asszonynyal hál és közösül, és az valamely férfi hatalma alatt lévõ rabnõ, és sem ki nem váltatott, sem szabadon nem bocsáttatott: büntetés érje, de meg ne ölettessenek, mert nem volt szabad az asszony.
\par 21 A férfiú pedig vigye el az õ vétkéért való áldozatát az Úrnak a gyülekezet sátorának nyílásához: egy kost vétekért való áldozatul.
\par 22 És a pap szerezzen néki engesztelést, a vétekért való áldozat kosával az Úr elõtt, az õ bûnéért, a melyet elkövetett, és megbocsáttatik néki az õ bûne, a melyet elkövetett.
\par 23 Mikor pedig bementek arra a földre, és plántáltok ott mindenféle gyümölcstermõ fát, annak gyümölcsét körülmetéletlennek tartsátok, három esztendeig legyen az néktek körülmetéletlen: meg ne egyétek.
\par 24 A negyedik esztendõben pedig annak minden gyümölcse szent legyen, hálaáldozatul az Úrnak.
\par 25 Csak az ötödik esztendõben egyétek meg annak gyümölcsét, és annak termését magatoknak gyüjtsétek. Én vagyok az Úr, a ti Istenetek.
\par 26 Ne egyetek vérrel valót ne varázsoljatok és ne  bûvészkedjetek.
\par 27 A ti hajatokat kerekdedre ne nyírjátok, a szakállad végét se csúfítsd el.
\par 28 Testeteket a holt emberért meg ne hasogassátok, se égetéssel magatokat meg ne bélyegezzétek. Én vagyok az Úr.
\par 29 A te leányodat meg ne becstelenítsd, paráznaságra adván azt; hogy paráznává ne legyen a föld, és be ne teljék a föld fajtalansággal.
\par 30 Az én szombatjaimat megtartsátok; szenthelyemet tiszteljétek. Én vagyok az Úr.
\par 31 Ne menjetek ígézõkhöz, és a jövendõmondókat ne tudakozzátok, hogy magatokat azokkal megfertõztessétek. Én vagyok az Úr, a ti Istenetek.
\par 32 Az õsz ember elõtt kelj fel, és a vén ember orczáját becsüld meg, és félj a te Istenedtõl. Én vagyok az Úr.
\par 33 Hogyha jövevény tartózkodik nálad, a ti földeteken, ne nyomorgassátok õt.
\par 34 Olyan legyen néktek a jövevény, a ki nálatok tartózkodik, mintha közületek való benszülött volna, és szeressed azt mint magadat, mert jövevények voltatok Égyiptom földén. Én vagyok az Úr, a ti Istenetek.
\par 35 Ne kövessetek el igazságtalanságot az ítéletben, a hosszmértékben, súlymértékben és ürmértékben.
\par 36 Igaz mérték, igaz font, igaz efa, és igaz hin legyen közöttetek. Én vagyok az Úr, a ti Istenetek, a ki kihoztalak titeket Égyiptom földébõl.
\par 37 Tartsátok meg azért minden rendelésemet és minden végzésemet, és cselekedjetek azok szerint. Én vagyok az Úr.

\chapter{20}

\par 1 Szóla ismét az Úr Mózesnek, mondván:
\par 2 Izráel fiainak pedig mondd meg: Valaki Izráel fiai közül és az Izráelben tartózkodó jövevények közül odaadja az õ magzatát a Moloknak, halállal lakoljon, a földnek népe kövezze agyon kõvel.
\par 3 Én is kiontom haragomat az ilyen emberre, és kiirtom azt az õ népe közül, mivelhogy adott az õ magzatából a Moloknak, hogy megfertõztesse az én szentségemet, és megszentségtelenítse az én szent nevemet.
\par 4 Ha pedig a föld népe behúnyja szemeit az ilyen ember elõtt, a mikor az oda adja az õ magzatát a Moloknak, és azt meg nem öli:
\par 5 Akkor én ontom ki haragomat arra az emberre és annak házanépére, és kiirtom azt és mindazokat, a kik õ utána paráználkodnak, hogy a Molokkal paráználkodjanak, az õ népök közül.
\par 6 A mely ember pedig az ígézõkhöz és a jövendõmondókhoz fordul, hogy azok után paráználkodjék, arra is kiontom haragomat, és kiirtom azt az õ népe közül.
\par 7 Szenteljétek meg azért magatokat, és szentek legyetek, mert én, az Úr, vagyok a ti Istenetek.
\par 8 És tartsátok meg az én rendeléseimet, és cselekedjétek azokat. Én vagyok az Úr, a ti megszentelõtök.
\par 9 Mert valaki szidalmazza az õ atyját vagy anyját, halállal lakoljon; atyját és anyját szidalmazta; vére rajta.
\par 10 Ha valaki más ember feleségével paráználkodik, mivelhogy az õ felebarátjának feleségével paráználkodik: halállal lakoljon a parázna férfi és a parázna nõ.
\par 11 Ha valaki az õ atyjának feleségével  hál, az õ atyjának szemérmét fedi fel: halállal lakoljanak mindketten; vérök rajtok.
\par 12 Ha valaki az õ menyével hál, halállal lakoljanak mindketten, fertelmességet követtek el; vérök rajtok.
\par 13 És ha valaki férfival hál, úgy a mint asszonynyal hálnak: útálatosságot követtek el mindketten, halállal lakoljanak; vérök rajtok.
\par 14 És ha valaki feleségül vesz valamely asszonyt annak anyjával egybe: fajtalankodás ez; tûzzel égessék meg azt és azokat, hogy ne legyen köztetek fajtalankodás.
\par 15 Ha pedig valaki barommal közösül, halállal lakoljon, és a barmot is öljétek meg.
\par 16 Ha valamely asszony akármely baromhoz járul, hogy az meghágja õt: öld meg mind az asszonyt, mind a barmot, halállal lakoljanak; vérök rajtok.
\par 17 És ha valaki feleségül veszi az õ leánytestvérét, atyjának leányát, vagy anyjának leányát, és meglátja annak szemérmét és az is meglátja az õ szemérmét: gyalázatosság ez; azért irtassanak ki népök fiainak láttára, az õ leánytestvérének szemérmét fedte fel: viselje gonoszságának terhét.
\par 18 És ha valaki havi bajos asszonynyal hál, és felfedi annak szemérmét, és forrását feltakarja, és az asszony is felfedi az õ vérének forrását: mindketten irtassanak ki az õ népökbõl.
\par 19 A te anyád leánytestvérének, vagy az atyád leánytestvérének szemérmét se fedd fel; mivelhogy az õ vérrokonát takarja ki: viseljék gonoszságuk terhét.
\par 20 És ha valaki az õ nagynénjével hál, az õ nagybátyjának szemérmét fedte fel: viseljék gonoszságuk terhét, magtalanul haljanak meg.
\par 21 Ha pedig elveszi valaki az õ fiútestvérének feleségét: vérfertõzés az; az õ fiútestvérének szemérmét fedte fel; magtalanok legyenek.
\par 22 Tartsátok meg minden rendelésemet és minden végzésemet, és azokat cselekedjétek, hogy ki ne okádjon titeket az a föld, a melybe én viszlek be titeket, hogy ott lakjatok.
\par 23 És ne járjatok annak a népnek rendtartási szerint, a melyet kiûzök én elõletek. Mivelhogy mindezeket cselekedték, azért megútáltam õket.
\par 24 Néktek pedig mondom: Ti örökölni fogjátok az õ földüket, mert én néktek adom azt örökségül, azt a tejjel és mézzel folyó földet. Én vagyok az Úr, a ti Istenetek, a ki kiválasztottalak titeket a népek közül.
\par 25 Tegyetek különbséget azért a tiszta és tisztátalan barmok között, a tiszta és tisztátalan szárnyas állatok között, és ne fertõztessétek meg magatokat barommal vagy szárnyas állattal, sem semmiféle földön csúszó állattal, a melyeket megkülönböztettem elõttetek, mint tisztátalanokat.
\par 26 És legyetek nékem szentek, mert én, az Úr, szent vagyok, a ki kiválasztottalak titeket a népek közül, hogy enyéim legyetek.
\par 27 És akár férfi, akár asszony, hogyha ígézõ vagy jövendõmondó lesz közöttök, halállal lakoljanak; kõvel kövezzétek azokat agyon; vérök rajtok.

\chapter{21}

\par 1 Szóla ismét az Úr Mózesnek: Szólj a papoknak, Áron fiainak, és mondd meg nékik: Senki közülök meg ne fertõztesse magát halottal az õ népe között;
\par 2 Hanem ha a hozzá legközelebb álló vérrokonával: anyjával, atyjával, fiával, leányával és fiútestvérével,
\par 3 Vagy a hozzá legközelebb álló hajadon leánytestvérével, a ki még nem ment férjhez: ezt megérintheti.
\par 4 Mint fõ-ember ne fertõztesse meg magát az õ népe között, hogy szentségtelenné ne legyen.
\par 5 Ne nyírjanak kopaszságot a fejükön, szakálluk szélét le ne messék, és a testükbe vágásokat ne vágjanak.
\par 6 Szentek legyenek Istenöknek, és az õ Istenöknek nevét meg ne szentségtelenítsék, mert az Úrnak tûzáldozatait, Istenöknek kenyerét õk áldozzák; azért szentek legyenek.
\par 7 Parázna és megszeplõsített asszonyt el ne vegyenek, se olyan asszonyt, a ki elûzetett az õ férjétõl, el ne vegyenek; mert a pap az õ Istenének van szentelve.
\par 8 Te is szentnek tartsad õt, mert Istenednek kenyerét õ áldozza: szent legyen azért elõtted, mert szent vagyok én, az Úr, a ti megszentelõtök.
\par 9 Hogyha valamely papnak leánya vetemedik paráznaságra, megszentségteleníti az õ atyját, azért tûzzel égettessék meg.
\par 10 A ki pedig fõpap az õ attyafiai között, a kinek fejére töltötték a kenetnek olaját, és a kit felavattak az õ szolgálatára, hogy a szent ruhákba felöltözzék: fejét meg ne meztelenítse, se ruháit meg ne szaggassa.
\par 11 És semmiféle holttesthez be ne menjen: atyjával és anyjával se fertõztesse meg magát.
\par 12 És a szenthelybõl ki ne menjen, hogy az õ Istenének szenthelyét meg ne szentségtelenítse, mert korona, az õ Istenének kenet-olaja van õ rajta. Én vagyok az Úr.
\par 13 Hajadont vegyen feleségül.
\par 14 Özvegyet, elûzöttet, megszeplõsítettet, paráznát: ilyeneket el ne vegyen, hanem hajadont vegyen feleségül az õ népe közül.
\par 15 Hogy meg ne fertõztesse az õ magzatát az õ népe között; mert én, az Úr vagyok az õ megszentelõje.
\par 16 Szóla ismét az Úr Mózesnek, mondván:
\par 17 Szólj Áronnak, mondván: Ha lesz valaki a te magod közül, az õ nemzetségökben, a kiben fogyatkozás leend, ne áldozza áldozatul az õ Istenének kenyerét.
\par 18 Mert senki sem áldozhat, a kiben fogyatkozás van: vagy vak, vagy sánta vagy csonka orrú, vagy hosszú tagú.
\par 19 Sem az, a ki törött lábú vagy törött kezû,
\par 20 Vagy púpos, vagy törpe, vagy szemfájós, vagy viszketeges, vagy sömörgös, vagy a ki megszakadott.
\par 21 Senki, a kiben fogyatkozás van, elõ ne álljon Áronnak, a papnak fiai közül, hogy tûzáldozatot vigyen fel az Úrnak; fogyatkozás van õ benne, ne álljon elõ, hogy megáldozza az õ Istenének kenyerét.
\par 22 Az õ Istenének kenyerébõl, a legszentségesebbikbõl és a szentségesbõl ehetik.
\par 23 Csak a függönyhöz be ne menjen, és az oltárhoz ne közelítsen, mert fogyatkozás van õ benne, hogy meg ne fertõztesse az én szenthelyemet. Én vagyok az Úr, az õ megszentelõjök.
\par 24 És elmondá Mózes Áronnak, és az õ fiainak, és Izráelnek minden fiainak.

\chapter{22}

\par 1 Szóla ismét az Úr Mózesnek, mondván:
\par 2 Szólj Áronnak és az õ fiainak, hogy tartóztassátok meg magokat Izráel fiainak szent adományaitól, hogy meg ne fertõztessék az én szent nevemet azokkal, a miket nékem szentelnek. Én vagyok az Úr.
\par 3 Mondd meg nékik: Ha valaki a ti nemzetségetekbõl, a ti összes magzataitok közül hozzájárul a szent dolgokhoz, a melyeket Izráel fiai szentelnek az Úrnak, mikor rajta van az õ tisztátalansága: az ilyen ember irtassék ki én elõlem. Én vagyok az Úr.
\par 4 Ha valaki az Áron fiai közül poklos, vagy magfolyós, a szent dolgokból ne egyék, míg meg nem tisztul. A ki pedig valamely halott által megfertõzöttet illet, vagy valakit, a kinek magömlése van,
\par 5 Vagy ha valaki valamely férget illet, a mely által tisztátalanná lesz, vagy embert, a kitõl tisztátalanná lesz annak valamilyen tisztátalanságához képest:
\par 6 Az ilyen ember, a ki effélét illet, tisztátalan legyen estvéig, és a szent dolgokból ne egyék, hanem ha megmosta a testét vízzel;
\par 7 De mikor lemegy a nap, tiszta lesz, és azután ehetik a szent dolgokból, mert az õ eledele az.
\par 8 Elhullott vagy széttépett állatot ne egyék, hogy tisztátalanná ne legyen általa. Én vagyok az Úr.
\par 9 Az én rendelésemet pedig megtartsák, hogy bûnbe ne essenek miatta, és meg ne haljanak a miatt, hogy megrontották azt. Én vagyok az Úr, az õ megszentelõjök.
\par 10 Idegen ember ne egyék szenteltet, a papnak zsellére és bérese se egyék szenteltet.
\par 11 De ha megvásárol valakit a pap a maga pénzén, az ehetik abból, és a ki házánál született: ezek ehetnek az õ eledelébõl.
\par 12 De a pap leánya, ha idegennek lesz a felesége, nem ehetik a szent áldozatból.
\par 13 Ha azonban a pap leánya özvegygyé lesz vagy elválik, de magzata nincsen, és visszatér az õ atyjának házához, mint leánykorában: akkor ehetik az õ atyjának eledelébõl; de idegen nem ehetik abból.
\par 14 Ha pedig tévedésbõl eszik valaki szenteltet, tegye ahhoz annak ötödrészét; így adja meg a papnak a szenteltet.
\par 15 És meg ne fertõztessék Izráel fiainak szent dolgait, a melyeket áldoznak az Úrnak,
\par 16 Hogy vétkes hamissággal ne terheljék magokat, ha esznek azoknak szent dolgaiból; mert én vagyok az Úr, az õ megszentelõjük.
\par 17 Szóla ismét az Úr Mózesnek, mondván:
\par 18 Szólj Áronnak és az õ fiainak és Izráel minden fiának, és mondd meg nékik: ha valaki Izráel házából, és az Izráelben levõ jövevények közül felviszi a maga áldozatát, akár fogadásból akár szabad akaratból, a miket felvisznek az Úrnak egészen égõáldozatul,
\par 19 Hogy kedvesen fogadtassanak: épek és hímek legyenek, akár tulkok, akár bárányok, akár kecskék.
\par 20 A miben pedig fogyatkozás van, abból semmit se áldozzatok, mert nem lesz kedvessé ti érettetek.
\par 21 És ha valaki hálaáldozattal áldozik az Úrnak, akár fogadásának teljesítésére, akár szabad akaratból, akár tulokfélébõl, akár juhfélébõl: ép legyen, hogy kedves legyen; semmi fogyatkozás ne legyen abban.
\par 22 Vakot, vagy rokkantat, vagy csonkát, vagy fekélyest, vagy viszketegest, vagy varast, ilyeneket ne áldozzatok az Úrnak, és tûzáldozatul ne tegyetek ezekbõl az oltárra az Úrnak.
\par 23 Hosszú, vagy kurta tagú ökröt, vagy bárányt szabad akaratból való áldozatul vihetsz ugyan, de fogadási áldozatul nem lesz kedves.
\par 24 Szétnyomott, összezúzott, megszakadt, vagy kimetszett heréjût se áldozzatok az Úrnak. Se a ti földeteken ne cselekedjétek ezt,
\par 25 Se idegen ember kezébõl ne áldozzatok semmi ilyenbõl a ti Istenetek eledeléül; mert romlás van bennök, fogyatkozás van bennök: nem fogadtatnak kedvesen érettetek.
\par 26 Szóla ismét az Úr Mózesnek, mondván:
\par 27 Borjú, bárány és kecske, ha megelletett, legyen az anyja alatt hét napig, a nyolczadik naptól fogva és azon túl kedves lesz az tûzáldozatul az Úrnak.
\par 28 De tehenet és juhot, azt és annak fiát ne öljétek meg egy napon.
\par 29 Hogyha dicsõítõ áldozattal áldoztok az Úrnak, úgy áldozzatok, hogy kedvesen fogadtassatok.
\par 30 Azon a napon egyétek meg, ne hagyjatok abból reggelig. Én vagyok az Úr.
\par 31 Tartsátok meg azért az én parancsolataimat, és azokat cselekedjétek. Én vagyok az Úr.
\par 32 És meg ne fertõztessétek az én szent nevemet, hogy megszenteltessem Izráel fiai között. Én vagyok az Úr, a ti megszentelõtök,
\par 33 A ki kihoztalak titeket Égyiptom földébõl, hogy Istenetek legyek néktek. Én vagyok az Úr.

\chapter{23}

\par 1 Szóla ismét az Úr Mózesnek, mondván:
\par 2 Szólj Izráel fiainak, és mondd meg nékik az Úrnak ünnepeit, a melyeken szent gyülekezésekre kell összegyülekeznetek. Ezek azok az én ünnepeim:
\par 3 Hat napon át munkálkodjatok, a hetedik napon nyugodalomnak, szent gyülekezésnek szombatja van, semmi dolgot ne végezzetek: az Úrnak szombatja legyen az minden lakhelyeteken.
\par 4 Ezek az Úrnak ünnepei, szent gyülekezések napjai, a melyekre szabott idejökben kell összegyülekeznetek.
\par 5 Az elsõ hónapban, a hónapnak tizennegyedikén, estennen az Úrnak páskhája.
\par 6 E hónapnak tizenötödik napján pedig az Úr kovásztalan kenyerének ünnepe. Hét napig egyetek kovásztalan kenyeret.
\par 7 Az elsõ napon szent gyülekezéstek legyen, semmi robota munkát ne végezzetek.
\par 8 Hét napon át pedig tûzáldozatot áldozzatok az Úrnak, és a hetedik napon szent gyülekezéstek is legyen: semmi robota munkát ne végezzetek.
\par 9 Szóla ismét az Úr Mózesnek, mondván:
\par 10 Szólj Izráel fiainak és mondd meg nékik: Mikor bementek a földre, a melyet én adok néktek, és megaratjátok annak vetését: a ti aratástok zsengéjének elsõ kévéjét vigyétek a papnak.
\par 11 Az pedig lóbálja meg a kévét az Úr elõtt, hogy kedvesen fogadtassék érettetek; a szombat után való napon lóbálja azt meg a pap.
\par 12 A mely napon pedig meglóbáltatjátok a kévét, áldozzatok az Úrnak egy ép, esztendõs bárányt egészen égõáldozatul.
\par 13 Ahhoz pedig ételáldozatul két tized efa lánglisztet, olajjal elegyítve; tûzáldozatul az Úrnak, kedves illatul; italáldozatul pedig egy hin bornak negyedrészét.
\par 14 Új kenyeret pedig és pergelt búzaszemeket és zsenge kalászokat ne egyetek mind a napig, a míg be nem viszitek a ti Isteneteknek áldozatját. Örök rendtartás ez nemzetségrõl nemzetségre minden lakóhelyeteken.
\par 15 Számláljátok azután a szombatra következõ naptól, attól a naptól, a melyen beviszitek a meglóbálni való kévét, hét hetet, egészek legyenek azok.
\par 16 A hetedik hétre következõ napig számláljatok ötven napot, és akkor járuljatok új ételáldozattal az Úrhoz.
\par 17 A ti lakóhelyeitekbõl hozzatok fel két meglóbálni való kenyeret; két tized efa lisztlángból legyenek azok, kovászszal sütve, zsengékül az Úrnak.
\par 18 A kenyérrel együtt pedig áldozzatok meg hét bárányt, épeket, esztendõsöket, és egy tulkot, fiatal bikát, és két kost; egészen égõáldozatul legyenek ezek az Úrnak, étel- és italáldozatjokkal egybe; kedves illatú tûzáldozat ez az Úrnak.
\par 19 Készítsetek el egy kecskebakot is bûnért való áldozatul, és két bárányt, esztendõsöket, hálaadó áldozatul.
\par 20 És lóbálja meg azokat a pap a zsengékbõl való kenyérrel az Úr elõtt való lóbálással a két báránynyal egybe. Szentek legyenek ezek az Úrnak a pap számára.
\par 21 És gyülekezzetek egybe ugyanazon a napon; szent gyülekezéstek legyen néktek, semmi robota munkát ne végezzetek. Örök rendtartás ez minden lakóhelyeteken a ti nemzetségeitek szerint.
\par 22 Mikor pedig földetek termését learatjátok: ne arasd le egészen a mezõdnek széleit, és az elhullott gabonafejeket fel ne szedd; a szegénynek és jövevénynek hagyd azokat. Én vagyok az Úr, a ti Istenetek.
\par 23 Szóla ismét az Úr Mózesnek, mondván:
\par 24 Szólj Izráel fiaihoz, mondván: A hetedik hónapban, a hónap elsõ napján ünnepetek legyen néktek, emlékeztetõ kürtzengéssel, szent gyülekezéssel.
\par 25 Semmi robota munkát ne végezzetek, és tûzáldozattal áldozzatok az Úrnak.
\par 26 Szóla ismét az Úr Mózesnek, mondván:
\par 27 Ugyanennek a hetedik hónapnak tizedikén az engesztelés napja van: szent gyülekezéstek legyen néktek, és sanyargassátok meg magatokat, és tûzáldozattal áldozzatok az Úrnak.
\par 28 Semmi dolgot ne végezzetek azon a napon, mert engesztelésnek napja az, hogy engesztelés legyen érettetek az Úr elõtt, a ti Istenetek elõtt.
\par 29 Mert ha valaki nem sanyargatja meg magát ezen a napon, irtassék ki az õ népe közül.
\par 30 És ha valaki valami dolgot végez ezen a napon, elvesztem az ilyent az õ népe közül.
\par 31 Semmi dolgot ne végezzetek; örök rendtartás legyen ez nemzetségrõl nemzetségre minden lakhelyeteken.
\par 32 Ünnepek ünnepe ez néktek, sanyargassátok meg azért magatokat. A hónap kilenczedikének estvéjén, egyik estvétõl a másik estvéig ünnepeljétek a ti ünnepeteket.
\par 33 Szóla ismét az Úr Mózesnek, mondván:
\par 34 Szólj Izráel fiainak, mondván: Ugyanennek a hetedik hónapnak tizenötödikén a sátorok ünnepe legyen az Úrnak hét napig.
\par 35 Az elsõ napon szent gyülekezés legyen, semmi robota munkát ne végezzetek.
\par 36 Hét napon áldozzatok az Úrnak tûzáldozatot, a nyolczadik napon pedig szent gyülekezéstek legyen és újra tûzáldozattal áldozzatok az Úrnak; berekesztõ ünnep ez, semmi robota munkát ne végezzetek azon.
\par 37 Ezek az Úrnak ünnepei, a melyeken szent gyülekezésekre kell gyülekeznetek, hogy áldozzatok az Úrnak tûzáldozattal, egészen égõáldozattal, ételáldozattal, véres- és italáldozattal: minden napét a maga napján.
\par 38 Az Úrnak szombatjain kivül, adományaitokon kivül, fogadásból és szabad akaratból való minden ajándékaitokon kivül a melyeket adni szoktatok az Úrnak,
\par 39 Ugyancsak a hetedik hónapnak tizenötödik napján, a mikor a földnek termését betakarjátok, az Úrnak ünnepét ünnepeljétek hét napig: az elsõ napon nyugodalom napja, és a nyolczadik napon is nyugodalom napja legyen.
\par 40 És vegyetek magatoknak az elsõ napon szép fának gyümölcsét, pálmafa ágait, sûrû levelû fa lombját, és patak mellett való fûzgalyakat, és örvendezzetek az Úr elõtt, a ti Istenetek elõtt hét napig.
\par 41 Így ünnepeljétek meg azt az Úrnak ünnepét minden esztendõben hét napig. Örökkévaló rendtartás legyen ez a ti nemzetségeiteknél; a hetedik hónapban ünnepeljétek azt.
\par 42 Sátorokban lakjatok hét napig, minden benszülött sátorokban lakjék Izráelben.
\par 43 Hogy megtudják a ti nemzetségeitek, hogy sátorban lakattam Izráel fiait, a mikor kihoztam õket Égyiptom földérõl. Én vagyok az Úr, a ti Istenetek.
\par 44 És szóla Mózes Izráel fiainak az Úrnak ünnepei felõl.

\chapter{24}

\par 1 Szóla ismét az Úr Mózesnek, mondván:
\par 2 Parancsold meg Izráel fiainak, hogy hozzanak néked tiszta faolajat, a melyet a világításhoz sajtoltak, hogy szünet nélkül égõ lámpákat gyújthassanak.
\par 3 A bizonyság függönyén kivül, a gyülekezet sátorában úgy helyheztesse el azokat Áron, hogy estvétõl fogva reggelig az Úr elõtt legyenek. Örökkévaló rendtartás legyen ez a ti nemzetségeiteknél.
\par 4 A tiszta arany gyertyatartóra rakja fel a mécseket; az Úr elõtt legyenek szüntelen.
\par 5 És végy lisztlángot, és süss abból tizenkét lepényt; két tized efából legyen egy lepény.
\par 6 És helyheztesd el azokat két rendben; hatot egy rendbe, a tiszta arany asztalra az Úr elé.
\par 7 És tégy mindenik rendhez tiszta tömjént, és legyen emlékeztetõül a kenyér mellett, tûzáldozatul az Úrnak.
\par 8 Szombat napról szombat napra rakja fel azt a pap az Úr elé szüntelen; örök szövetség ez Izráel fiaival.
\par 9 Azután legyen az Ároné és az õ fiaié, a kik egyék meg azokat szent helyen, mert mint igen szentséges, az övé az, az Úrnak tûzáldozataiból, örök rendelés szerint.
\par 10 Kiméne pedig egy izráelbeli asszonynak fia, a ki égyiptomi férfiútól való vala, Izráel fiai közé, és versengének a táborban az izráelbeli asszonynak fia és egy izráelbeli férfi.
\par 11 És káromlá az izráelbeli asszony fia az Isten nevét és átkozódék; elvivék azért azt Mózeshez. Az õ anyjának neve pedig Selomith vala, Dibrinek leánya, Dán nemzetségébõl.
\par 12 És õrizet alá veték azt, míg kijelentést nyernének az Úr akarata felõl.
\par 13 Szóla azért az Úr Mózesnek, mondván:
\par 14 Vidd ki az átkozódót a táboron kivül, és mindazok, a kik hallották, tegyék kezeiket annak fejére és kövezze agyon azt az egész gyülekezet.
\par 15 Izráel fiainak pedig szólj, ezt mondván: Ha valaki az õ Istenét átkozza, viselje az õ bûnének terhét.
\par 16 És a ki szidalmazza az Úrnak nevét, halállal lakoljon, kövezze azt agyon az egész gyülekezet; akár jövevény, akár benszülött, ha szidalmazza az Úrnak nevét, halállal lakoljon.
\par 17 Ha valaki agyon üt valamely embert, halállal lakoljon.
\par 18 Ha pedig barmot üt agyon valaki, fizesse meg azt: barmot baromért.
\par 19 És ha valaki sérelmet ejt a felebarátján, a mint õ cselekedett, vele is úgy cselekedjenek:
\par 20 Törést törésért, szemet szemért, fogat fogért; a milyen sérelmet õ ejtett máson, olyan ejtessék rajta is.
\par 21 A ki barmot üt agyon, fizesse meg azt, de a ki embert üt agyon, halállal lakoljon.
\par 22 Egy törvény legyen nálatok: a jövevény olyan legyen, mint a benszülött, mert én vagyok az Úr, a ti Istenetek.
\par 23 Szóla azért Mózes Izráel fiainak, és kivivék az átkozódót a táboron kivül, és agyonverék azt kõvel. És úgy cselekedének Izráel fiai, a mint parancsolta vala az Úr Mózesnek.

\chapter{25}

\par 1 Szóla ismét az Úr Mózesnek a Sinai hegyen, mondván:
\par 2 Szólj Izráel fiainak és mondd meg nékik: Mikor bementek a földre, a melyet én adok néktek, nyugodjék meg a föld az Úrnak szombatja szerint.
\par 3 Hat esztendõn át vesd a te szántóföldedet, és hat esztendõn át messed a te szõlõdet, és takarítsd be annak termését;
\par 4 A hetedik esztendõben pedig szombati nyugodalma legyen a földnek, az Úrnak szombatja: szántóföldedet ne vesd be, és szõlõdet meg ne mesd.
\par 5 A mi a te tarló földeden magától terem, le ne arasd, és a mi a te metszetlen szõlõdön terem, meg ne szedjed; mert nyugalom esztendeje lesz az a földnek.
\par 6 És a mit a föld az õ szombatján terem, legyen az eledelül néktek: néked, szolgádnak, szolgáló leányodnak, béresednek és zsellérednek, a kik nálad tartózkodnak;
\par 7 A te barmodnak is és a vadnak, a mely a te földeden van, legyen annak minden termése eledelül.
\par 8 Számlálj azután hét szombat-esztendõt, hétszer hét esztendõt, úgy hogy a hét szombat-esztendõnek ideje negyvenkilencz esztendõ legyen:
\par 9 Akkor fúvasd végig a riadó kürtöt a hetedik hónapban, a hónap tizedikén az engesztelés napján fúvasd végig a kürtöt a ti egész földeteken.
\par 10 És szenteljétek meg az ötvenedik esztendõt, és hirdessetek szabadságot a földön, annak minden lakójának; kürtölésnek esztendeje legyen ez néktek, és kapja vissza kiki az õ birtokát, és térjen vissza kiki az õ nemzetségéhez.
\par 11 Kürtölésnek esztendeje ez, az legyen néktek az ötvenedik esztendõ; ne vessetek és le se arassátok, a mit önként terem a föld, és a metszetlen szõlõjét se szedjétek meg annak.
\par 12 Mert kürtölésnek esztendeje ez, szent legyen néktek, a mezõrõl egyétek meg annak termését.
\par 13 A kürtölésnek ebben az esztendejében, kapja vissza ismét kiki az õ birtokát.
\par 14 Ha pedig eladsz valami eladni valót a te felebarátodnak, vagy vásárolsz valamit a te felebarátodtól: egymást meg ne csaljátok.
\par 15 A kürtölés esztendejét követõ esztendõk száma szerint vásárolj a te felebarátodtól; a terméses esztendõk száma szerint adjon el néked.
\par 16 Az esztendõk nagyobb számához képest nagyobb árt adj azért, a mi veszesz, az esztendõk kisebb számához képest pedig kisebb árt adj azért, a mit veszesz, mert a termések számát adja õ el néked.
\par 17 Egymást azért meg ne csaljátok, hanem félj a te Istenedtõl: mert én vagyok az Úr, a ti Istenetek.
\par 18 Ha teljesítitek azért az én rendeléseimet, és megtartjátok végzéseimet, és teljesítitek azokat, bátorságosan lakhattok a földön.
\par 19 És megtermi a föld az õ gyümölcsét, hogy eleget ehessetek, és bátorságosan lakhattok azon.
\par 20 Ha pedig azt mondjátok: Mit eszünk a hetedik esztendõben, ha nem vetünk, és termésünket be nem takarítjuk?
\par 21 Én rátok bocsátom majd az én áldásomat a hatodik esztendõben, hogy három esztendõre való termés teremjen.
\par 22 És mikor a nyolczadik esztendõkre vettek, akkor is az ó termésbõl esztek egészen a kilenczedik esztendeig; mindaddig ó gabonát esztek, míg ennek termése be nem jön.
\par 23 A földet pedig senki el ne adja örökre, mert enyém a föld; csak jövevények és zsellérek vagytok ti nálam.
\par 24 Azért a ti birtokotoknak egész földén megengedjétek, hogy a föld kiváltható legyen.
\par 25 Ha elszegényedik a te atyádfia, és elad valamit az õ birtokából, akkor álljon elõ az õ rokona, a ki közel van õ hozzá, és váltsa ki, a mit eladott az õ atyjafia.
\par 26 Ha pedig nincs valakinek kiváltó rokona, de maga tesz szert annyira, hogy elege van annak megváltásához:
\par 27 Számlálja meg az eladása óta eltelt esztendõket, a felül lévõt pedig térítse meg annak, a kinek eladta volt, és újra övé legyen az õ birtoka.
\par 28 Ha pedig nincsen módjában, hogy visszatéríthesse annak, akkor maradjon az õ eladott birtoka annál, a ki megvette azt, egészen a kürtölésnek esztendejéig: a kürtölésnek esztendejében pedig szabaduljon fel, és újra övé legyen az õ birtoka.
\par 29 Ha valaki lakó-házat ad el kerített városban, az kiválthatja azt az eladás esztendejének elteléséig; egy esztendõn át válthatja ki azt.
\par 30 Ha pedig ki nem váltják az esztendõnek teljes elteléséig, akkor a ház, a mely kerített városban van, örökre azé és annak nemzetségeié marad, a ki megvette azt; nem szabadul fel a kürtölésnek esztendejében.
\par 31 Az olyan falvak házai pedig, a melyek nincsenek körülkerítve, mezei földek gyanánt számíttassanak, kiválthatók legyenek, és a kürtölésnek esztendejében felszabaduljanak.
\par 32 A mi pedig a léviták városait illeti, az õ birtokukban lévõ városok házai a léviták által mindenkor kiválthatók legyenek,
\par 33 De a mit ki nem vált is valaki a léviták közül, szabaduljon fel a kürtölésnek esztendejében, az eladott ház, és az õ birtokának városa; mert a léviták városainak házai tulajdon birtokuk nékik Izráel fiai között.
\par 34 De a városaikhoz tartozó szántóföldeket el ne adják, mert örök birtokuk az nékik.
\par 35 Ha a te atyádfia elszegényedik, és keze erõtlenné lesz melletted, segítsd meg õt, akár jövevény, akár zsellér, hogy megélhessen melletted.
\par 36 Ne végy õ tõle kamatot vagy uzsorát, hanem félj a te Istenedtõl, hogy megélhessen melletted a te atyádfia.
\par 37 Pénzedet ne add néki kamatra, se uzsoráért ne add a te eleségedet.
\par 38 Én vagyok az Úr, a ti Istenetek, a ki kihoztalak titeket Égyiptom földérõl, hogy néktek adjam Kanaán földét, és Istenetek legyek néktek.
\par 39 Ha pedig elszegényedik melletted a te atyádfia, és eladja magát néked: ne szolgáltassad úgy mint rabszolgát.
\par 40 Mint béres, mint zsellér legyen nálad; a kürtölésnek esztendejéig szolgáljon nálad.
\par 41 Azután menjen el tõled õ és vele az õ gyermekei, és térjen vissza az õ nemzetségéhez, és térjen vissza az õ atyáinak örökségébe.
\par 42 Mert az én szolgáim õk, a kiket kihoztam Égyiptom földérõl: nem adathatnak el, mint rabszolgák.
\par 43 Ne uralkodjál rajta kegyetlenül, hanem félj a te Istenedtõl.
\par 44 Mind szolgád, mind szolgálóleányod, a kik lesznek néked, a körületek lévõ népek közül legyenek: azokból vásárolj szolgát és szolgálóleányt;
\par 45 Meg a zsellérek gyermekei közül is, a kik nálatok tartózkodnak, azokból is vásárolhattok, és azoknak nemzetségébõl, a kik veletek vannak, a kiket a ti földeteken nemzettek; és legyenek a ti tulajdonotok.
\par 46 És örökül hagyhatjátok azokat a ti utánnatok való fiaitoknak, hogy örökségül bírják azokat, örökké dolgoztathattok velük; de a ti atyátokfiain, az Izráel fiain, egyik a másikán senki ne uralkodjék kegyetlenül.
\par 47 És ha a jövevény vagy zsellér vagyonra tesz szert melletted, a te atyádfia pedig elszegényedik mellette, és eladja magát a melletted lévõ jövevénynek, zsellérnek, vagy jövevény nemzetségébõl való sarjadéknak:
\par 48 Mindamellett is, hogy eladta magát, megváltható legyen; akárki megválthassa azt az õ atyjafiai közül.
\par 49 Vagy nagybátyja, vagy nagybátyjának fia váltsa meg azt, vagy az õ nemzetségébõl való vérrokona váltsa meg azt, vagy, ha módja van hozzá, maga váltsa meg õmagát.
\par 50 És vessen számot azzal, a ki megvette õt, attól az esztendõtõl kezdve, a melyen eladta magát annak, a kürtölésnek esztendejéig, és az õ eladásának ára az esztendõk száma szerint legyen, a béres ideje szerint legyen nála.
\par 51 Ha még sok esztendõ van hátra, azokhoz képest térítse meg annak a váltságot az õ megvásárlásának árából.
\par 52 Ha pedig kevés esztendõ van hátra a kürtölésnek esztendejéig, akkor is vessen számot vele, és az évek számához képest fizesse vissza az õ váltságát.
\par 53 Mint esztendõrõl esztendõre fogadott béres legyen nála; ne uralkodjék kegyetlenül rajta te elõtted.
\par 54 Ha pedig ilyen módon meg nem váltatik, a kürtölésnek esztendejében szabaduljon fel: õ és vele az õ gyermekei.
\par 55 Mert az én szolgáim Izráel fiai, az én szolgáim õk, a kiket kihoztam Égyiptom földérõl. Én vagyok az Úr, a ti Istenetek.

\chapter{26}

\par 1 Ne csináljatok magatoknak bálványokat, se faragott képet, se oszlopot ne emeljetek magatoknak, se kõszobrokat ne állítsatok fel a ti földeteken, hogy meghajoljatok elõtte, mert én vagyok az Úr, a ti Istenetek.
\par 2 Az én szombatjaimat megtartsátok, és az én szenthelyemet tiszteljétek. Én vagyok az Úr.
\par 3 Ha az én rendeléseim szerint jártok, és az én parancsolataimat megtartjátok, és azokat megcselekszitek:
\par 4 Esõt adok néktek idejében, és a föld megadja az õ termését, a mezõ fája is megtermi gyümölcsét.
\par 5 És a ti csépléstek ott éri a szüretet, és a szüret ott éri a vetést, és elégségig ehetitek kenyereteket, és bátorságosan lakhattok a ti földeteken.
\par 6 Mert békességet adok azon a földön, hogy mikor lefeküsztök, senki fel ne rettentsen; és kipusztítom az ártalmas vadat arról a földrõl, és fegyver sem megy át a ti földeteken.
\par 7 Sõt elûzitek ellenségeiteket, és elhullanak elõttetek fegyver által.
\par 8 És közületek öten százat elûznek, és közületek százan elûznek tízezeret, és elhullanak elõttetek a ti ellenségeitek fegyver által.
\par 9 És hozzátok fordulok, és megszaporítlak titeket, és megsokasítlak titeket és szövetségemet megerõsítem veletek.
\par 10 És réginél régibbet ehettek, és az új elõl is régit kell kihordanotok.
\par 11 És az én hajlékomat közétek helyezem, és meg nem útál titeket az én lelkem.
\par 12 És közöttetek járok, és a ti Istenetek leszek, ti pedig az én népem lesztek.
\par 13 Én vagyok az Úr, a ti Istenetek, a ki kihoztalak titeket Égyiptom földérõl, hogy ne legyetek azoknak rabjai, és összetörtem a ti igátok szegeit, és egyenesen járattalak titeket.
\par 14 Ha pedig nem hallgattok reám, és mind e parancsolatokat meg nem cselekeszitek;
\par 15 És ha megvetitek rendeléseimet, és ha az én végzéseimet megútálja a ti lelketek, azáltal, hogy nem cselekszitek meg minden én parancsolatomat, hanem felbontjátok az én szövetségemet:
\par 16 Bizony azt cselekszem én veletek, hogy rettenetességet bocsátok reátok: a száraz betegséget és a forrólázt, a melyek szemeket égetnek és lelket epesztenek, és a ti  magotokat hiába vetitek el, mert ellenségeitek emésztik meg azt.
\par 17 És kiontom haragomat reátok, hogy elhulljatok a ti ellenségeitek elõtt, és uralkodjanak rajtatok a ti gyûlölõitek, és fussatok, mikor senki nem kerget is titeket.
\par 18 Ha pedig ezek után sem hallgattok reám, hétszerte keményebben megostorozlak titeket a ti bûneitekért;
\par 19 És megtöröm a ti megátalkodott kevélységeteket, és olyanná teszem az eget felettetek, mint a vas, a földeteket pedig olyanná, mint a réz.
\par 20 És hiába fogy a ti erõtök, mert földetek nem adja meg az õ termését, s a föld fája sem adja meg az õ gyümölcsét.
\par 21 Ha mégis ellenemre jártok, és nem akartok reám hallgatni: hétszeres csapást borítok reátok a ti bûneitekért.
\par 22 És reátok bocsátom a mezei vadakat, hogy megfoszszanak titeket gyermekeitektõl, kiirtsák barmaitokat, és elfogyaszszanak titeket, hogy pusztákká legyenek a ti útaitok.
\par 23 És ha ezek által sem jobbultok meg, hanem ellenemre jártok:
\par 24 Én is bizony ellenetekre járok, és hétszeresen sújtalak titeket a ti bûneitekért.
\par 25 És hozok reátok bosszuló fegyvert, a mely bosszút álljon a szövetség megrontásáért. Ha városaitokba gyülekeztek össze, akkor döghalált bocsátok reátok, és az ellenség kezébe adattok.
\par 26 Mikor eltöröm nálatok a kenyérnek botját, tíz asszony süti majd a ti kenyereteket egy kemenczében, és megmérve viszik vissza a ti kenyereiteket, és esztek, de  nem elégesztek meg.
\par 27 És ha e mellett sem hallgattok reám, hanem ellenemre jártok:
\par 28 Én is ellenetekre járok búsult haragomban, és bizony hétszeresen megostorozlak titeket a ti bûneitekért.
\par 29 És megeszitek a ti fiaitok húsát, és megeszitek a ti leányaitok húsát.
\par 30 És lerontom a ti magaslataitokat, és kiirtom a ti nap-oszlopaitokat, és a ti holttesteiteket bálványaitok holttetemeire hányatom, és megútál titeket az én lelkem.
\par 31 És városaitokat sivataggá teszem; szenthelyeiteket is elpusztítom, és nem lesz kedves nékem a ti jóillatú áldozatotok.
\par 32 És elpusztítom ezt a földet, hogy álmélkodjanak rajta a ti ellenségeitek, a kik letelepednek ebbe.
\par 33 Titeket pedig elszélesztelek a pogány népek közé, és kivont fegyverrel ûzetlek titeket, és pusztasággá lesz a ti földetek, városaitok pedig sivataggá.
\par 34 Akkor örül a föld az õ szombatjainak az õ pusztaságának egész ideje alatt, ti pedig a ti ellenségeitek földjén lesztek; akkor nyugodni fog a föld és örül az õ szombatjainak.
\par 35 Pusztaságának egész ideje alatt nyugodni fog, mivelhogy nem nyugodott a ti szombatjaitokon, mikor rajta laktatok.
\par 36 A kik pedig megmaradnak közületek, azoknak szívébe gyávaságot öntök az õ ellenségeiknek földén, és megkergeti õket a szállongó falevél zörrenése, és futnak, mintha fegyver elõl futnának, és elhullanak, ha senki nem kergeti is õket.
\par 37 És egymásra hullanak, mint a fegyver elõtt, pedig senki sem kergeti õket, és nem lesz megállásotok a ti ellenségeitek elõtt.
\par 38 És elvesztek a pogány népek között, és a ti ellenségeitek földe megemészt titeket.
\par 39 A kik pedig megmaradnak közületek, elsenyvednek az õ hamisságuk miatt a ti ellenségeitek földén, sõt az õ atyáiknak hamissága miatt is azokkal együtt elsenyvednek.
\par 40 Akkor megvallják az õ hamisságukat, és atyáiknak hamisságát az õ hûtelenségökben, a melylyel hûtelenkedtek ellenem, és hogy mivel ellenemre jártak.
\par 41 Bizony én is ellenökre járok, és beviszem õket az õ ellenségeik földjére; akkor talán megalázódik az õ körülmetéletlen szívök, és akkor az õ bûnöknek büntetését békével szenvedik:
\par 42 Én pedig megemlékezem Jákóbbal kötött szövetségemrõl, Izsákkal kötött szövetségemrõl is, Ábrahámmal kötött szövetségemrõl is megemlékezem, és e földrõl is megemlékezem.
\par 43 A föld tehát pusztán hagyatik tõlük, és örül az õ szombatjainak, a míg pusztán marad tõlük, õk pedig békével szenvedik bûnöknek büntetését, azért, mert megvetették az én ítéleteimet, és megútálta lelkök az én rendeléseimet.
\par 44 De mindamellett is, ha az õ ellenségeik földén lesznek is, akkor sem vetem meg õket, és nem útálom meg õket annyira, hogy mindenestõl elveszítsem õket, felbontván velök való szövetségemet, mert én, az Úr, az õ Istenök vagyok.
\par 45 Sõt megemlékezem érettök az elõdökkel kötött szövetségrõl, a kiket kihoztam Égyiptom földérõl, a pogány népek láttára, hogy Istenök legyek nékik. Én vagyok az Úr.
\par 46 Ezek a rendelések, a végzések és a törvények, a melyeket szerzett az Úr õ maga között és Izráel fiai között a Sinai hegyen Mózes által.

\chapter{27}

\par 1 Szóla ismét az Úr Mózesnek, mondván:
\par 2 Szólj Izráel fiainak, és mondd meg nékik: Ha valaki fogadásul a te becslésed szerint való személyeket szentel az Úrnak:
\par 3 Akkor, ha férfit kell becslened, húsz esztendõstõl hatvan esztendõsig ötven ezüst siklusra becsüljed, a szent siklus szerint.
\par 4 Ha pedig asszony-személy az, harmincz siklusra becsüljed.
\par 5 Ha pedig öt esztendõstõl húsz esztendõsig való, akkor a fiúgyermeket húsz siklusra becsüljed, a leányt pedig tíz siklusra.
\par 6 Ha pedig egy hónapostól öt esztendõsig való, akkor a fiút öt ezüst siklusra becsüld, a leányt pedig három ezüst siklusra.
\par 7 Ha pedig hatvan esztendõs és azon felül való, ha férfi, akkor becsüljed tizenöt siklusra, az asszony-személyt pedig tíz siklusra.
\par 8 Ha pedig szegényebb az, mint te becsülted, akkor állassák a pap elé, és becsülje meg azt a pap; a szerint becsülje azt a pap, a milyen módja van a fogadást tevõnek.
\par 9 Ha pedig olyan barom az, a mibõl áldozni szoktak az Úrnak: mindaz, a mit az efélébõl ád valaki az Úrnak, szent legyen.
\par 10 Ne adjon mást helyette, és ki ne cserélje azt: jót hitványért, vagy hitványat jóért, de ha mégis kicserélne barmot barommal: mind ez, mind az, a mivel kicserélte, szent legyen.
\par 11 Ha pedig valamely tisztátalan barom az, a melybõl nem vihetnek áldozatot az Úrnak: állassák azt a barmot a pap elé.
\par 12 És becsülje meg azt a pap akár jó, akár hitvány, és a mint becsüli a pap, úgy legyen.
\par 13 Ha pedig meg akarja váltani, adja hozzá annak ötödrészét a te becsléseden felül.
\par 14 És ha valaki az õ házát szenteli az Úrnak szentségül, azt is becsülje meg a pap: akár jó, akár hitvány, és a mennyire a pap becsüli azt, úgy maradjon.
\par 15 Ha pedig az, a ki odaszentelte, megváltja az õ házát: akkor adja ahhoz a te becslésed szerint való árnak ötödrészét, és legyen az övé.
\par 16 És ha valaki az õ mezei birtokából szentel valamit az Úrnak, akkor a mag szerint becsüld meg, a mely abba megy: egy hómer árpa-mag után ötven ezüst siklusra.
\par 17 Ha a kürtölésnek esztendejétõl szenteli oda az õ mezejét, a mint becsülted, úgy maradjon.
\par 18 Ha pedig a kürtölés esztendeje után szenteli oda az õ mezejét, a pap számítsa fel néki az árt az esztendõk száma szerint, a melyek hátra vannak a kürtölés esztendejéig, és szállíttassék le a te becslésed.
\par 19 És ha meg akarja váltani a mezõt az, a ki odaszentelte, akkor adja ahhoz az általad becsült árnak ötödrészét, és maradjon az övé.
\par 20 Ha pedig nem váltja meg azt a mezõt, és ha eladja azt a mezõt más valakinek, többé meg nem válthatja azt.
\par 21 És az a föld, mikor a kürtölésnek esztendejében felszabadul, az Úrnak szenteltessék, mint valamely néki szentelt mezõ, papok birtokává legyen az.
\par 22 Ha pedig pénzen vett mezejét, a mi nem az õ birtokának mezejébõl való, valaki az Úrnak szenteli:
\par 23 Akkor számolja fel néki a pap a te becslésed szerint való összeget a kürtölés esztendejéig, és adja oda azt, a mire te becsülted, azon a napon szentségül az Úrnak.
\par 24 A kürtölésnek esztendejében visszaszáll a mezõ arra, a kitõl vette azt, a kinek birtoka volt az a föld.
\par 25 Minden becslésed pedig a szent siklus szerint legyen; húsz géra legyen a siklus.
\par 26 Csak elsõszülöttet, a mely elsõszülött úgyis az Úré, azt ne szenteljen a baromfélébõl senki; akár borjú akár bárány, az Úré az.
\par 27 Ha pedig tisztátalan baromból való az, váltsa meg a te becslésed szerint, és adja hozzá annak ötödrészét; ha pedig nem váltják meg, adassék el a te becslésed szerint.
\par 28 De semmi, a mit valaki teljesen az Úrnak szentelt, mind abból a mije van, akár ember vagy barom, akár mezei birtokából való, el ne adassék, és meg se váltassék; minden, a mi teljesen néki szenteltetett, igen szentséges az Úrnak.
\par 29 Senki, a ki teljesen az Úrnak szenteltetett az emberek közül, meg ne váltassék, hanem halálra adassék bizonynyal.
\par 30 A földnek minden tizede, a föld vetésébõl, a fa gyümölcsébõl az Úré; szentség az az Úrnak.
\par 31 És ha valaki meg akar valamit váltani az õ tizedébõl: adja hozzá annak ötödrészét.
\par 32 És minden tizede a baromnak és juhnak, mindabból, a mi a vesszõ alatt átmegy, a tizedik az Úrnak legyen szentelve.
\par 33 Ne tudakozódjék, ha jó-e vagy hitvány, és el se cserélje azt; de ha mégis elcseréli azt, akkor az, és a mit cserébe adott azért, szent legyen, és meg se váltassék.
\par 34 Ezek azok a parancsolatok, a melyeket Mózes által parancsolt az Úr Izráel fiainak a Sinai hegyen.


\end{document}