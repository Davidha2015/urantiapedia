\begin{document}

\title{Numbers}


\chapter{1}

\par 1 Szóla pedig az Úr Mózesnek a Sinai pusztájában, a gyülekezet sátorában, a második hónapnak elsején, az Egyiptom földébõl való kijövetelök után a második esztendõben, mondván:
\par 2 Vegyétek számba Izráel fiainak egész gyülekezetét, az õ nemzetségeik szerint, az õ atyáiknak háznépe szerint, a neveknek száma szerint, minden férfiút fõrõl fõre,
\par 3 Húsz esztendõstõl fogva és feljebb, mindent, a ki hadba mehet Izráelben; számláljátok meg õket az õ seregök szerint, te és Áron.
\par 4 És legyen veletek egy-egy férfiú mindenik törzsbõl, mindenik feje legyen az õ atyái házának.
\par 5 Ezek pedig a férfiak nevei, a kik veletek legyenek: Rúbenbõl Elisúr, Sedeúrnak fia.
\par 6 Simeonból Selúmiel, Surisaddainak fia.
\par 7 Júdából Naasson, Amminádábnak fia.
\par 8 Izsakhárból Néthánéel, Suárnak fia.
\par 9 Zebulonból Eliáb, Hélonnak fia.
\par 10 József fiai közül: Efraimból Elisama, Ammihudnak fia; Manasséból Gámliel, Pédasurnak fia.
\par 11 Benjáminból Adibán, Gideóni fia.
\par 12 Dánból Ahiézer, Ammisaddai fia.
\par 13 Áserbõl Págiel, Okránnak fia.
\par 14 Gádból Eleásaf, Déhuelnek fia.
\par 15 Nafthaliból Akhira, Enánnak fia.
\par 16 Ezek a gyülekezetnek hivatalosai, az õ atyjok törzseinek fejei, Izráel ezereinek is fejei õk.
\par 17 Maga mellé vevé azért Mózes és Áron e férfiakat, a kik név szerint is elõszámláltattak vala.
\par 18 És összegyûjték az egész gyülekezetet a második hónapnak elsõ napján; és vallást tõnek az õ születésökrõl, az õ nemzetségeik szerint, az õ atyáiknak háznépe szerint, a neveknek száma szerint, húsz esztendõstõl fogva és feljebb fõrõl fõre.
\par 19 A miképen megparancsolta vala az Úr Mózesnek, úgy számlálá meg õket a Sinai pusztájában.
\par 20 Valának pedig Rúbennek, Izráel elsõszülöttének fiai, azoknak szülöttei az õ nemzetségeik szerint, az õ atyáiknak háznépe szerint, a nevek száma szerint, fõrõl fõre, minden férfiú, húsz esztendõstõl fogva és feljebb, minden hadba mehetõ;
\par 21 A kik megszámláltattak a Rúben törzsébõl: negyvenhat ezer és ötszáz.
\par 22 Simeon fiai közül azoknak szülöttei az õ nemzetségeik szerint, az õ atyáiknak háznépe szerint, az õ megszámláltjai, a neveknek száma szerint, fõrõl fõre, minden férfiú, húsz esztendõstõl fogva és feljebb, minden hadba mehetõ;
\par 23 A kik megszámláltattak Simeon törzsébõl; ötvenkilencz ezer és háromszáz.
\par 24 Gád fiai közül azoknak szülöttei az õ nemzetségeik szerint, az õ atyáiknak háznépe szerint, a neveknek száma szerint, húsz esztendõstõl fogva és feljebb, minden hadba mehetõ;
\par 25 A kik megszámláltattak Gád törzsébõl: negyvenöt ezer és hatszáz ötven.
\par 26 Júda fiai közül azoknak szülöttei az õ nemzetségeik szerint, az õ atyáiknak háznépe szerint, a nevek száma szerint, húsz esztendõstõl fogva és feljebb, minden hadba mehetõ;
\par 27 A kik megszámláltattak Júda törzsébõl: hetvennégy ezer és hatszáz.
\par 28 Izsakhár fiai közül azoknak szülöttei az õ nemzetségeik szerint, az õ atyáiknak háznépe szerint, a nevek száma szerint, húsz esztendõstõl fogva és feljebb, minden hadba mehetõ;
\par 29 A kik megszámláltattak Izsakhár törzsébõl, ötvennégy ezer és négyszáz.
\par 30 Zebulon fiai közül azoknak szülöttei az õ nemzetségeik szerint, az õ atyáiknak háznépe szerint, a nevek száma szerint, húsz esztendõstõl fogva és feljebb, minden hadba mehetõ;
\par 31 A kik megszámláltattak Zebulon törzsébõl; ötvenhét ezer és négyszáz.
\par 32 József fiaiból Efraim fiai közül azoknak szülöttei az õ nemzetségeik szerint, az õ atyáiknak háznépe szerint, a neveknek száma szerint, húsz esztendõstõl fogva és feljebb, minden hadba mehetõ;
\par 33 A kik megszámláltattak Efraim törzsébõl; negyvenezer és ötszáz.
\par 34 Manasse fiai közül azoknak szülöttei az õ nemzetségeik szerint, az õ atyáiknak háznépe szerint, a nevek száma szerint, húsz esztendõstõl fogva és feljebb, minden hadba mehetõ;
\par 35 A kik megszámláltattak Manasse törzsébõl; harminczkét ezer és kétszáz.
\par 36 Benjámin fiai közül azoknak szülöttei az õ nemzetségeik szerint, az õ atyáiknak háznépe szerint, a nevek száma szerint, húsz esztendõstõl fogva és feljebb, minden hadba mehetõ;
\par 37 A kik megszámláltattak Benjámin törzsébõl; harminczöt ezer és négyszáz.
\par 38 Dán fiai közül azoknak szülöttei az õ nemzetségeik szerint, az õ atyáiknak háznépe szerint, a nevek száma szerint, húsz esztendõstõl fogva és feljebb, minden hadba mehetõ;
\par 39 A kik megszámláltattak Dán törzsébõl; hatvankét ezer és hétszáz.
\par 40 Áser fiai közül azoknak szülöttei az õ nemzetségeik szerint, az õ atyáiknak háznépe szerint, a neveknek száma szerint, húsz esztendõstõl fogva és feljebb, minden hadba mehetõ;
\par 41 A kik megszámláltattak Áser törzsébõl; negyvenegy ezer és ötszáz.
\par 42 A Nafthali fiainak szülöttei az õ nemzetségeik szerint, az õ atyáiknak háznépe szerint, a neveknek száma szerint, húsz esztendõstõl fogva és feljebb, minden hadba mehetõ;
\par 43 A kik megszámláltattak a Nafthali törzsébõl; ötvenhárom ezer és négyszáz.
\par 44 Ezek azok a megszámláltattak, a kiket megszámláltak Mózes és Áron és Izráel fejedelmei, tizenkét férfiú; egy-egy férfiú vala az õ atyáiknak házanépébõl.
\par 45 Valának azért mindnyájan, a kik megszámláltattak az Izráel fiai közûl az õ atyáiknak háznépe szerint, húsz esztendõstõl fogva és feljebb, minden hadba mehetõ az Izráelben;
\par 46 Valának mindnyájan a megszámláltattak: hatszáz háromezer és ötszáz ötven.
\par 47 De a léviták az õ atyáiknak háznépe szerint nem számláltattak közéjök.
\par 48 Mert szólott vala az Úr Mózesnek, mondván:
\par 49 Csak a Lévi törzsét ne vedd számba, és azokat ne számláld Izráel fiai közé;
\par 50 Hanem a lévitákat rendeld a bizonyság hajlékához, és minden edényéhez, és minden ahhoz valókhoz; õk hordozzák a hajlékot, és annak minden edényét, és õk szolgáljanak mellette, és a hajlék körül táborozzanak.
\par 51 És mikor a hajléknak elébb kell indulni, a léviták szedjék azt szét, mikor pedig megáll a hajlék, a léviták állassák azt fel, az idegen pedig, a ki oda járul, meghaljon.
\par 52 És tábort járjanak az Izráel fiai kiki az õ táborában, és kiki az õ zászlója alatt, az õ seregeik szerint.
\par 53 A léviták pedig tábort járjanak a bizonyság hajléka körül, hogy ne legyen harag Izráel fiainak gyülekezetén; és megtartsák a léviták a bizonyság hajlékának õrizetét.
\par 54 Cselekedének azért az Izráel fiai mind a szerint, a mint parancsolta vala az Úr Mózesnek, úgy cselekedének.

\chapter{2}

\par 1 És szóla az Úr Mózesnek és Áronnak, mondván:
\par 2 Az Izráel fiai, kiki az õ zászlója alatt, az õ atyáik háznépének jeleivel járjon tábort, a gyülekezet sátora körül járjon tábort annak oldalai felõl.
\par 3 Így járjanak pedig tábort: Keletre naptámadat felõl Júda táborának zászlója az õ seregeivel; és Júda fiainak fejedelme Naasson, Amminádáb fia.
\par 4 Az õ serege pedig, vagyis az õ megszámláltjaik: hetvennégy ezer és hatszáz.
\par 5 Mellette pedig tábort járjon Izsakhár törzse, és Izsakhár fiainak fejedelme Néthánéel, Suárnak fia.
\par 6 Az õ serege pedig, vagyis az õ megszámláltjaik: ötvennégy ezer és négyszáz.
\par 7 Zebulon törzse, és Zebulon fiainak fejedelme Eliáb, Hélon fia.
\par 8 Az õ serege pedig, vagyis az õ megszámláltjaik: ötvenhét ezer és négyszáz.
\par 9 Mindnyájan, a kik megszámlálva voltak Júda táborában: száz nyolcvanhat ezer és négyszáz, az õ seregeik szerint. Ezek induljanak elõre.
\par 10 Rúben táborának zászlója legyen dél felõl az õ seregeivel, és Rúben fiainak fejedelme Elisúr, Sedeúr fia.
\par 11 Az õ serege pedig, vagyis az õ megszámláltjaik: negyvenhat ezer és ötszáz.
\par 12 Mellette pedig tábort járjon Simeon törzse, és Simeon fiainak fejedelme: Selúmiel, Surisaddai fia.
\par 13 Az õ serege pedig, vagyis az õ megszámláltjaik: ötvenkilencz ezer és háromszáz.
\par 14 Azután Gád törzse, és Gád fiainak fejedelme: Eliásáf, a Réuel fia.
\par 15 Az õ serege pedig, vagyis az õ megszámláltjaik: negyvenöt ezer és hatszáz ötven.
\par 16 Mindnyájan, a kik megszámlálva voltak és Rúben táborában: száz ötvenegy ezer és négyszáz ötven az õ seregeik szerint. Ezek másodszorra induljanak.
\par 17 Azután induljon a gyülekezet sátora, a léviták táborával, a táboroknak közepette: A miképen tábort járnak, a képen induljanak, kiki az õ helyén, az õ zászlója mellett.
\par 18 Efraim táborának zászlója az õ seregeivel nyugot felé legyen, és Efraim fiainak fejedelme: Elisáma, Ammihud fia.
\par 19 Az õ serege pedig, vagyis az õ megszámláltjaik: negyven ezer és ötszáz.
\par 20 És õ mellette legyen Manasse törzse, és Manasse fiainak fejedelme Gámliel, Pédasur fia.
\par 21 Az õ serege pedig, vagyis az õ megszámláltjaik, harminczkét ezer és hétszáz.
\par 22 Azután legyen Benjámin törzse és Benjámin fiainak fejedelme: Abidám Gideóni fia.
\par 23 Az õ serege pedig, vagyis az õ megszámláltjaik: harminczöt ezer és négyszáz.
\par 24 Mindnyájan a kik megszámlálva voltak Efraim táborában, száz nyolcezer és száz az õ seregeik szerint. Ezek harmadszorra induljanak.
\par 25 Dán táborának zászlója legyen észak felõl az õ seregeivel, és a Dán fiainak fejedelme: Ahiézer, Ammisaddai fia.
\par 26 Az õ serege pedig, vagyis az õ megszámláltjaik: hatvankét ezer és hétszáz.
\par 27 Mellette pedig tábort járjon Áser törzse, és Áser fiainak fejedelme: Págiel, az Okhrán fia.
\par 28 Az õ serege pedig, vagyis az õ megszámláltjaik: negyvenegy ezer és ötszáz.
\par 29 Azután legyen Nafthali törzse, és Nafthali fiainak fejedelme: Akhira, Enán fia.
\par 30 Az õ serege pedig, vagyis az õ megszámláltjaik: ötvenhárom ezer és négyszáz.
\par 31 Mindnyájan, a kik megszámlálva voltak Dán táborában, száz ötvenhét ezer és hatszáz; utóljára induljanak az õ zászlóik szerint.
\par 32 Ezek Izráel fiainak megszámláltjai az õ atyáiknak háznépei szerint; mindnyájan, a kik megszámlálva voltak táboronként, az õ seregeik szerint: hatszáz háromezer ötszáz és ötven.
\par 33 De a léviták nem voltak számba véve Izráel fiai között, a mint megparancsolta volt az Úr Mózesnek.
\par 34 És cselekedének Izráel fiai mind a szerint, a mint parancsolta volt az Úr Mózesnek: aképen járának tábort, az õ zászlóik szerint, és úgy indulának, kiki az õ nemzetsége szerint.

\chapter{3}

\par 1 Ezek pedig Áronnak és Mózesnek szülöttei azon a napon, a melyen szólott az Úr Mózesnek a Sinai hegyen;
\par 2 Ezek az Áron fiainak nevei: Az elsõszülött Nádáb, azután Abihú, Eleázár és Ithamár.
\par 3 Ezek Áron fiainak, a felkenetett papoknak nevei, a kiket papi szolgálatra avattak fel.
\par 4 De Nádáb és Abihú meghala az Úr elõtt, mikor idegen tûzzel áldozának az Úr elõtt a Sinai pusztájában, fiaik pedig nem valának nékik. Eleázár és Ithamár viselék azért a papságot, Áronnak, az õ atyjoknak színe elõtt.
\par 5 Szóla pedig az Úr Mózesnek, mondván:
\par 6 Hozd elõ Lévi törzsét, és állassad Áron pap elé, hogy szolgáljanak néki.
\par 7 És ügyeljenek az õ ügyére, és az egész gyülekezet ügyére a gyülekezet sátora elõtt, hogy végezhessék a hajlék körül való  szolgálatot.
\par 8 Ügyeljenek pedig a gyülekezet sátorának minden eszközére is, és Izráel fiainak ügyeire is, hogy végezhessék a hajlék körül való szolgálatot.
\par 9 És adjad a lévitákat Áronnak és az õ fiainak; mert valóban néki adattak Izráel fiaitól.
\par 10 Áront pedig és az õ fiait rendeld föléjök, hogy õrizzék az õ papságukat; és ha idegen járulna oda, haljon meg.
\par 11 Szóla azután az Úr Mózesnek, mondván:
\par 12 Ímé én kiválasztottam a lévitákat Izráel fiai közül, minden elsõszülött helyett, a mely az õ anyjának méhét megnyitja  Izráel fiai között: azért legyenek a léviták enyéim.
\par 13 Mert enyém minden elsõszülött: a mikor megöltem minden elsõszülöttet Égyiptom földén, magamnak szenteltem minden elsõszülöttet Izráelben; akár ember, akár barom, enyéim legyenek: én vagyok az Úr.
\par 14 Szóla azután az Úr Mózesnek a Sinai pusztájában, mondván:
\par 15 Számláld meg Lévi fiait az õ atyáiknak háznépe szerint, az õ nemzetségeik szerint; egy hónapostól fogva, és azon felül minden finemût számlálj meg.
\par 16 Megszámlálá azért Mózes õket az Úr szava szerint, a miképen meghagyatott vala néki.
\par 17 És ezek voltak a Lévi fiai az õ neveik szerint: Gerson, Kéhát és Mérári.
\par 18 Ezek pedig a Gerson fiainak nevei az õ nemzetségök szerint: Libni és Simhi.
\par 19 Továbbá a Kéhát fiai az õ nemzetségök szerint: Amrám és Iczhár, Hebron és Uzziél.
\par 20 A Mérári fiai pedig az õ nemzetségök szerint: Makhli és Músi. Ezek a Lévi nemzetségei, az õ atyáiknak háznépe szerint.
\par 21 Gersontól valók a Libni nemzetsége és a Simhi nemzetsége; ezek a Gersoniták nemzetségei.
\par 22 Az õ megszámláltjaik, az egy hónapostól fogva és feljebb minden finemûnek száma szerint, az õ megszámláltjaik: hétezer és ötszáz.
\par 23 A Gersoniták nemzetségei a hajlék megett járjanak tábort nyugot felõl.
\par 24 És a Gersoniták atyái háznépének fejedelme legyen Eliásáf, a Láél fia.
\par 25 A Gerson fiainak tiszte pedig: ügyelni a gyülekezet sátorában, a hajlékra, a sátorra, annak takarójára, és a gyülekezet sátora nyílásának leplére.
\par 26 Továbbá a pitvarnak szõnyegeire, és a pitvar nyílásának leplére, a mely van a hajlékon és az oltáron köröskörül, és annak köteleire, és minden azzal járó szolgálatra.
\par 27 Kéháttól való pedig az Amrám nemzetsége, az Iczhár nemzetsége, a Hebron nemzetsége és az Uzziél nemzetsége: Ezek Kéhátnak nemzetségei.
\par 28 Minden finemûnek száma szerint, egy hónapostól fogva és feljebb, nyolczezeren és hatszázan valának a szenthelynek õrizõi.
\par 29 A Kéhát fiainak nemzetségei a hajlék oldala mellett dél felõl járjanak tábort.
\par 30 És a Kéhátiták nemzetségének, az õ atyái háznépének fejedelme legyen Elisáfán, Uzziélnek fia.
\par 31 Az õ tisztök pedig, ügyelni a ládára, az asztalra, a gyertyatartóra, az oltárokra és a szenthelynek edényeire, a melyekkel szolgálnak, és a takaróra, és minden azzal járó szolgálatra.
\par 32 Továbbá a léviták fejedelmeinek fejedelme legyen Eleázár, Áron pap fia: a szenthelyre ügyelõknek elõljárója.
\par 33 Méráritól való a Makhli és Músi nemzetségei; ezek a Mérári nemzetségei.
\par 34 Az õ megszámláltjaik pedig minden finemûnek száma szerint, egy hónapostól fogva és feljebb: hatezer és kétszáz.
\par 35 És a Mérári nemzetségének, az õ atyái háznépének fejedelme legyen Suriel, az Abihail fia; a hajléknak észak felõl való oldala mellett járjanak tábort.
\par 36 A Mérári fiainak pedig tisztök legyen felügyelni; a hajlék deszkáira, annak reteszrúdaira, oszlopaira és annak talpaira, minden edényeire és minden azzal járó szolgálatra;
\par 37 Továbbá a pitvar körül való oszlopokra és azoknak talpaira, szegeire és köteleire.
\par 38 A hajlék elõtt keletre, a gyülekezet sátora elõtt naptámadat felõl, Mózes, Áron és az õ fiai járjanak tábort, a kik felügyelnek a szenthely szolgálatára, és Izráel fiainak ügyeire; ha pedig idegen járulna oda, haljon meg.
\par 39 A léviták minden megszámláltja, a kiket Mózes és Áron az Úr rendeletére nemzetségenként számláltak vala meg, minden finemû, az egy hónapostól fogva és feljebb: huszonkét ezer.
\par 40 És monda az Úr Mózesnek: Számláld meg Izráel fiainak minden finemû elsõszülöttét, egy hónapostól fogva és feljebb, és pedig névszerint számláld meg õket.
\par 41 És válaszd a lévitákat nékem (én vagyok az Úr) az Izráel fiai közül való minden elsõszülött helyett; és a léviták barmait, Izráel fiai barmainak minden elsõ fajzása helyett.
\par 42 Megszámlálá azért Mózes, a mint parancsolta vala néki az Úr, Izráel fiainak minden elsõszülöttét.
\par 43 És lõn minden finemû elsõszülött a nevek száma szerint, egy hónapostól fogva és feljebb, az õ megszámláltjaik: huszonkét ezer kétszáz és hetvenhárom.
\par 44 És szóla az Úr Mózesnek, mondván:
\par 45 Válaszd a lévitákat az Izráel fiai közül való minden elsõszülött helyett; és a léviták barmait az õ barmaik helyett, és legyenek enyéim a léviták. Én vagyok az Úr.
\par 46 A mi pedig a kétszáz és hetvenháromnak megváltását illeti, a kik felül vannak a lévitákon Izráel fiainak elsõszülöttei közül:
\par 47 Végy öt-öt siklust fejenként; a szent  siklus szerint vedd azt, (húsz géra egy siklus).
\par 48 És add azt a pénzt Áronnak és az õ fiainak, váltságul a köztök lévõ számfelettiekért.
\par 49 Bevevé azért Mózes a váltságpénzt azoktól, a kik felül voltak a lévitáktól megváltottakon.
\par 50 Izráel fiainak elsõszülöttitõl vevé be e pénzt: ezer háromszáz és hatvanöt siklust, a szent siklus szerint.
\par 51 És adá Mózes a megváltottaknak pénzét Áronnak és az õ fiainak, az Úr rendelete szerint, a miképen parancsolta az Úr Mózesnek.

\chapter{4}

\par 1 És szóla az Úr Mózesnek és Áronnak, mondván:
\par 2 Vedd számba a Kéhát fiait Lévi fiai közül nemzetségenként, az õ atyjoknak háznépe szerint.
\par 3 Harmincz esztendõstõl és azon felül az ötven esztendõsökig mindenkit, a ki szolgálatra való, hogy munkálkodjék a gyülekezet sátorában.
\par 4 Ez a tisztök Kéhát fiainak a gyülekezet sátorában: a legszentségesebbekrõl való gondviselés.
\par 5 Áron és az õ fiai pedig, mikor indulni akar a tábor, menjenek be, és vegyék le a takaró függönyt, és takarják be azzal a bizonyság ládáját.
\par 6 És tegyenek arra borzbõrbõl csinált takarót, és borítsák azt be egészen kékszínû ruhával felülrõl, és dugják belé a rúdjait is.
\par 7 A szent kenyerek asztalát is borítsák be kékszínû ruhával, azon felül tegyék rá a tálakat, a csészéket, a kelyheket, és az italáldozathoz való kancsókat, és ama szüntelen való kenyér is rajta legyen.
\par 8 Azután borítsanak azokra karmazsinszínû ruhát, és takarják be azt borzbõrbõl való takaróval, és dugják belé a rúdjait is.
\par 9 Vegyenek azután kékszínû ruhát, és takarják be a világításra való gyertyatartót és annak mécseit, hamvvevõit, hamutartóit és minden olajos edényét, a melyekkel szolgálnak körülte.
\par 10 És tegyék azt és minden edényét borzbõrbõl csinált takaróba, és tegyék saroglyára.
\par 11 Az arany oltárt is borítsák be kékszínû ruhával, és takarják be borzbõrbõl csinált takaróval, és dugják belé a rúdjait is.
\par 12 Vegyék elõ azután a szolgálatnak minden eszközét, a melyekkel szolgálni fognak a szenthelyen, és tegyék kékszínû ruhába, és takarják be azokat borzbõrbõl csinált takaróval, és tegyék a saroglyára.
\par 13 Azután takarítsák el a hamvat az oltárról, és borítsanak arra bíborpiros színû ruhát.
\par 14 És tegyék rá arra minden õ eszközét, a melyekkel szolgálnak azon: a szenes serpenyõket, a villákat, a lapátokat és a medenczéket, az oltárnak minden eszközét; és borítsanak arra borzbõrbõl csinált takarót, és dugják belé a rúdjait is.
\par 15 Ha pedig elvégezi Áron és az õ fiai a szenthelynek és a szenthely minden edényének betakarását, mikor el akar indulni a tábor: akkor jöjjenek el Kéhát fiai, hogy elvigyék azokat; de ne illessék a szenthelynek semmi edényét, hogy meg ne haljanak. Ezek Kéhát fiainak terhei a gyülekezet sátorában.
\par 16 Eleázárnak pedig, az Áron pap fiának tiszte: a világító olajra, a füstölõ szerekre, a szüntelen való ételáldozatra és a kenetnek olajára; az egész hajlékra és mindenre, a mi abban van, mind a szenthelyre, mind annak edényeire való gondviselés.
\par 17 Szóla azután az Úr Mózesnek és Áronnak mondván:
\par 18 A Kéhátiták nemzetségének törzsét ne engedjétek kiirtani a léviták közül;
\par 19 Hanem ezt míveljétek õ velök, hogy éljenek és meg ne haljanak, mikor járulnak a szentségek szentségéhez: Áron és az õ fiai jöjjenek el, és rendeljék el õket, kit-kit az õ szolgálatára és az õ terhére.
\par 20 Egy pillanatra se menjenek be, hogy meglássák a szenthelyet, hogy meg ne haljanak.
\par 21 Szóla ezután az Úr Mózesnek, mondván:
\par 22 Vedd számba a Gerson fiait is az õ atyjoknak háznépe szerint, az õ nemzetségök szerint.
\par 23 A harmincz esztendõstõl és azon felül az ötven esztendõsig számláld meg õket; mindenkit, a ki szolgálatot teljesíteni való, hogy szolgáljon a gyülekezet sátorában.
\par 24 Ez legyen munkájok a Gersoniták nemzetségeinek a szolgálatban és a teherhordozásban:
\par 25 Hordozzák a hajlék kárpitjait és a gyülekezet sátorát, annak takaróját és borzbõrbõl csinált takarót, a mely rajta van felül és a gyülekezet sátora nyilásának leplét.
\par 26 És a pitvar szõnyegeit, és a pitvar kapunyílásának leplét, a melyek a hajlékon és az oltáron köröskörül vannak, azoknak köteleit, és az õ szolgálatjokhoz való minden edényt; és mindazt, a mi tenni való azokkal, õk teljesítsék.
\par 27 Áronnak és az õ fiainak beszéde szerint legyen a Gersoniták fiainak minden szolgálatjok, minden terhökre és minden szolgálatjokra nézve; minden terhöket pedig az õ õrizetökre bízzátok.
\par 28 Ez a Gersoniták fiai nemzetségének szolgálata a gyülekezet sátorában; az õ szolgálatuk pedig legyen az Áron pap fiának, Ithamárnak keze alatt.
\par 29 A Mérári fiait is az õ nemzetségeik szerint, az õ atyjoknak háznépei szerint számláld meg.
\par 30 A harmincz esztendõstõl és annál feljebb az ötven esztendõsig számlált meg õket; mindenkit, a ki szolgálatra való, hogy szolgálatot végezzen a gyülekezet sátorában.
\par 31 Ezek pedig, a mikre ügyelniök kell a továbbvitelnél és a gyülekezet sátorában való minden szolgálatuknál: a hajlék deszkái, annak reteszrúdjai, oszlopai és talpai.
\par 32 A pitvarnak is köröskörül való oszlopai és azok talpai, szegei és kötelei, azoknak minden szerszámai, és minden szolgálatjokhoz való. Név szerint is megszámláljátok az edényeket, a mikre a továbbvitelnél ügyelniök kell.
\par 33 Ez legyen szolgálatuk a Mérári fiai nemzetségeinek, a melylyel szolgáljanak a gyülekezet sátorában, Itharmárnak, az Áron pap fiának vezetése alatt.
\par 34 És megszámlálá Mózes és Áron és a gyülekezet fejedelmei a Kéhátiták fiait, az õ nemzetségök szerint és az õ atyjoknak háznépe szerint,
\par 35 A harmincz esztendõstõl és annál feljebb, az ötven esztendõsig mindenkit, a ki szolgálatra való, hogy szolgáljon a gyülekezet sátorában.
\par 36 Vala pedig azoknak száma az õ nemzetségeik szerint: kétezer hétszáz és ötven.
\par 37 Ezek a Kéhátiták nemzetségeinek megszámláltjai, mindazok, a kik szolgálnak vala a gyülekezet sátorában, a kiket megszámlált Mózes és Áron az Úrnak parancsolatja szerint, a Mózes keze által.
\par 38 A Gerson fiainak száma az õ nemzetségeik szerint, és az õ atyáiknak háznépe szerint.
\par 39 A harmincz esztendõstõl és azon felül, az ötven esztendõsig mindazok, a kik szolgálatra valók, hogy szolgáljanak a gyülekezet sátorában.
\par 40 Azoknak száma az õ nemzetségök szerint, az õ atyjoknak háznépe szerint: kétezer hatszáz és harmincz.
\par 41 Ezek a Gerson fiai nemzetségeinek megszámláltjai, a kik mindnyájan szolgálnak a gyülekezet sátorában, a kiket megszámlált Mózes és Áron az Úrnak beszéde szerint.
\par 42 A Mérári fiai nemzetségeinek száma az õ nemzetségei szerint, az õ atyjoknak háznépe szerint;
\par 43 A harmincz esztendõstõl és azon felül, az ötven esztendõsig mindazok, a kik szolgálatra valók, hogy szolgáljanak a gyülekezet sátorában.
\par 44 Ezeknek száma az õ nemzetségeik szerint: háromezer és kétszáz.
\par 45 Ezek a Mérári fiai nemzetségeinek megszámláltjai, a kiket megszámlált Mózes és Áron és az Úrnak beszéde szerint, a Mózes keze által.
\par 46 Mindnyájan a megszámláltatott léviták, a kiket megszámlált Mózes és Áron, és Izráel fejedelmei az õ nemzetségeik és atyjoknak háznépe szerint;
\par 47 A harmincz esztendõstõl és azon felül, az ötven esztendõsig, a kik csak alkalmatosak a szolgálatra, a szolgálatnak gyakorlására, és a továbbvitelnél való szolgálatra a gyülekezetnek sátorában;
\par 48 Ezeknek száma lõn nyolczezer ötszáz és nyolczvan.
\par 49 Az Úrnak beszéde szerint számlálták meg õket a Mózes keze által kit-kit az õ szolgálata szerint, és az õ vinni való terhe szerint. Ez az a számlálás, a melyet az Úr parancsolt vala Mózesnek.

\chapter{5}

\par 1 És szóla az Úr Mózesnek, mondván:
\par 2 Parancsold meg Izráel fiainak, hogy ûzzenek ki a táborból minden poklost, és minden  magfolyóst, és mindenkit, a ki holttest miatt lett tisztátalanná.
\par 3 Ûzzétek azt ki akár férfi, akár asszony; a táboron kivül ûzzétek õket, hogy tisztátalanná ne tegyék az õ táborukat, mivelhogy én közöttök lakozom.
\par 4 És úgy cselekedének Izráel fiai, és kiûzék azokat a táboron kivül; a miképen meghagyta valal az Úr Mózesnek, úgy cselekedtek Izráel fiai.
\par 5 Szóla azután az Úr Mózesnek, mondván:
\par 6 Szólj Izráel fiainak: Ha akár férfi, akár asszony, akármi emberi bûnt követ el, a mely által hûtelenné válik az Úrhoz; az a lélek vétkessé lesz.
\par 7 Vallja meg azért az õ bûnét, a melyet elkövetett, és fizesse meg a kárt, a mit okozott, teljes értékében, azután toldja ahhoz annak ötödrészét, és adja annak, a kinek kárt tett.
\par 8 Ha pedig nincs az embernek atyjafia, a kinek megfizetné a kárt: a megtérített kár az Úré legyen a pap számára, az engesztelésre való koson kivül, a melylyel engesztelés tétetik érte.
\par 9 Izráel fiainak minden szent dolgaiból mindent felmutatott áldozat azé a papé legyen, a kihez viszik azt.
\par 10 És kinek-kinek az õ szenteltje a magáé legyen; a mit pedig kiki a papnak ád, azé legyen.
\par 11 Szóla azután az Úr Mózesnek, mondván:
\par 12 Szólj Izráel fiainak, és mondd meg nékik: Ha elhajol valakinek a felesége, és hûtelenné válik iránta;
\par 13 És hál valaki õ vele közösülve, és az õ férje elõtt titok marad, és elrejtetik, hogy õ megfertõztetett; bizonyság pedig nincs ellene, sem a bûnön nem kapták;
\par 14 De felgerjed õ benne a féltékenység lelke, és féltékenykedik a feleségére, mivelhogy az megfertõztetett; vagy felgerjed õ benne a féltékenység lelke, és féltékenykedik a feleségére, jóllehet az meg nem fertõztetett.
\par 15 Akkor vigye a férfiú az õ feleségét a paphoz; és vigye el azzal együtt az érette való áldozatot egy efa árpalisztnek tizedrészét, de ne öntsön arra olajt, és ne tegyen arra temjént, mert féltékenységi ételáldozat ez, emlékeztetõ ételáldozat ez, a mely hamisságra emlékeztet.
\par 16 A pap pedig léptesse elõ az asszonyt, és állassa õt az Úr elé.
\par 17 És vegyen a pap szent vizet cserépedénybe; a hajlék pádimentomán levõ porból is vegyen a pap, és tegye azt a vízbe.
\par 18 És állassa a pap az asszonyt az Úr elé és fejét meztelenítse meg az asszonynak, és tegye annak kezére az emlékeztetõ ételáldozatot, féltékenységi ételáldozat az; és a papnak kezében legyen átokhozó keserû víz.
\par 19 És eskesse meg õt a pap, és ezt mondja az asszonynak: Ha nem hált veled senki, és ha el nem hajoltál tisztátalanságra a te férjed mellett, ne ártson néked ez az átokhozó, keserû víz.
\par 20 Ha pedig elhajoltál a te férjed mellõl, és megfertõztetted magadat, és valaki közösült veled a te férjedem kívül,
\par 21 Miután megeskette a pap az asszonyt az átoknak esküjével, ezt mondja az asszonynak: Tegyen tégedet az Úr átokká, és eskü-példává a te néped között, megszárasztván az Úr a te tomporodat, és a te méhedet dagadtá tévén.
\par 22 És menjen be az átokhozó víz a te belsõ részeidbe, hogy megdagadjon a te méhed, és megszáradjon a te tomporod. Az asszony pedig mondja: Ámen! Ámen!
\par 23 És írja fel a pap ez átkokat levélre, azután törölje le a keserû vízzel.
\par 24 És itassa meg az asszonynyal az átokhozó keserû vizet, hogy bemenjen õ belé az átokhozó víz az õ keserûségére.
\par 25 Azután vegye el a pap az asszony kezébõl a féltékenységi ételáldozatot, és lóbálja meg azt az ételáldozatot az Úr elõtt, és áldozzék azzal az oltáron.
\par 26 És vegyen egy marokkal a pap az ételáldozatból emlékeztetõûl, és füstölögtesse el az oltáron, és azután itassa meg az asszonynyal a vizet.
\par 27 És ha megitatta vele a vizet, akkor lészen, hogy, ha megfertõztette magát, és hütelenné lett az õ férjéhez, bemegy az átokhozó víz õ belé az õ keserûségére, és megdagad az õ méhe, és megszárad az õ tompora, és az az asszony átokká lesz az õ népe között.
\par 28 Ha pedig nem fertõztette meg magát az asszony, hanem tiszta: akkor ártatlan, és termékeny lészen.
\par 29 Ez a féltékenységi törvény, mikor elhajol az asszony az õ férje mellõl, és megfertõzteti magát;
\par 30 Vagy mikor valaki, a kiben felgerjed a féltékenység lelke annyira, hogy féltékenykedik a feleségére; az õ feleségét állatja az Úr elé. És cselekedjék vele a pap mind e törvény szerint.
\par 31 És ártatlan lesz a férfi a bûntõl, az asszony pedig viseli az õ bûnének terhét.

\chapter{6}

\par 1 És szóla az Úr Mózesnek, mondván:
\par 2 Szólj Izráel fiainak, és mondd meg nékik: Mikor férfi vagy asszony külön fogadást tesz, nazireusi fogadást, hogy így az Úrnak szentelje magát:
\par 3 Bortól és részegítõ italtól szakassza el magát: boreczetet és részegítõ italból való eczetet ne igyék, és semmi szõlõbõl csinált italt se igyék, se új, se asszú szõlõt ne egyék.
\par 4 Az õ nazireusságának egész idején át semmit a félét ne igyék, a mi a szõlõtõrõl kerül, a szõlõ magvától fogva a szõlõ héjáig.
\par 5 Az õ nazireusi fogadásának egész idején, beretva az õ fejét ne járja; míg be nem teljesednek a napok, a melyekre az Úrnak szentelte magát, szent legyen, hagyja növekedni az õ fejének hajfürtjeit.
\par 6 Az egész idõn át, a melyre az Úrnak szentelte magát, megholtnak testéhez be ne menjen.
\par 7 Se atyjának, se anyjának, se fiú- se leánytestvéreinek holttestével meg ne fertõztesse magát, mikor meghalnak, mert az õ Istenének nazireussága van az õ fején.
\par 8 Az õ nazireusságának egész idejében szent legyen az Úrnak.
\par 9 Ha pedig meghal valaki õ nála hirtelenséggel, és megfertõzteti az õ nazireus fejét: nyírja meg a fejét az õ tisztulásának napján, a hetedik napon nyírja meg azt.
\par 10 A nyolczadik napon pedig vigyen két gerliczét vagy két galambfiat a papnak a gyülekezet sátorának nyílásához.
\par 11 És készítse el a pap egyiket bûnért való áldozatul, a másikat pedig egészen égõáldozatul, és szerezzen néki engesztelést, a miért vétkezett a holttest miatt; és szentelje meg annak fejét azon napon.
\par 12 És az õ nazireusságának napjait szentelje újra az Úrnak, és vigyen az õ vétkéért való áldozatul egy esztendõs bárányt; az elébbi napok pedig essenek el, mert megfertõztette az õ nazireusságát.
\par 13 Ez pedig a nazireus törvénye: A mely napon betelik az õ nazireusságának ideje, vigyék õt a gyülekezet sátorának nyílásához.
\par 14 Õ pedig vigye fel az õ áldozatját az Úrnak: egy esztendõs, ép hím bárányt egészen égõáldozatul, és egy esztendõs, ép nõstény bárányt bûnért való áldozatul, és egy ép kost hálaáldozatul.
\par 15 Továbbá egy kosár kovásztalan kenyeret, olajjal elegyített lánglisztbõl való lepényeket, és olajjal megkent kovásztalan pogácsákat, a hozzájok való étel- és italáldozatokkal.
\par 16 És vigye azokat a pap az Úr elé, és készítse el annak bûnéért való áldozatát és egészen égõáldozatát.
\par 17 A kost is készítse el hálaadó áldozatul az Úrnak, a kosárban lévõ kovásztalan kenyerekkel egybe, és készítse el a pap az ahhoz való étel- és italáldozatot is.
\par 18 A nazireus pedig nyírja meg az õ nazireus fejét a gyülekezet sátorának nyílásánál, és vegye az õ nazireus fejének haját, és tegye azt a tûzre, a mely van a hálaadó áldozat alatt.
\par 19 Vegye azután a pap a kosnak megfõtt lapoczkáját, és egy kovásztalan lepényt a kosárból, és egy kovásztalan pogácsát, és tegye a nazireus tenyerére, minekutána megnyirta az õ nazireus fejét.
\par 20 És lóbálja meg a pap azokat áldozatul az Úr elõtt; a papnak szenteltetett ez, a meglóbált szegyen és a felemelt lapoczkán felül. Azután igyék bort a nazireus.
\par 21 Ez a nazireus törvénye, a ki fogadást tett, és az õ áldozata az õ nazireusságáért az Úrnak, azonkivül, a mihez módja van. Az õ fogadása szerint, a melyet fogadott, a képen cselekedjék, az õ nazireusságának törvénye szerint.
\par 22 És szóla az Úr Mózesnek, mondván:
\par 23 Szólj Áronnak és az õ fiainak, mondván: Így áldjátok meg Izráel fiait, mondván nékik:
\par 24 Áldjon meg tégedet az Úr, és õrizzen meg tégedet.
\par 25 Világosítsa meg az Úr az õ orczáját te rajtad, és könyörüljön te rajtad.
\par 26 Fordítsa az Úr az õ orczáját te reád, és adjon békességet néked.
\par 27 Így tegyék az én nevemet Izráel fiaira, hogy én megáldjam õket.

\chapter{7}

\par 1 És lõn, hogy a mely napon elvégezé Mózes a sátor felállítását, és felkené azt, és megszentelé azt, minden edényével egybe, és az oltárt és annak minden edényét; és felkené és megszentelé azokat:
\par 2 Akkor elõjövének az Izráel fejedelmei, az õ atyjok házának fejei; ezek a törzsek fejedelmei, és ezek a megszámláltattak felügyelõi:
\par 3 És vivék az õ áldozatukat az Úr elé: hat borított szekeret, és tizenkét ökröt; egy-egy szekeret két-két fejedelemért, éd mindenikért egy-egy ökröt; és odavivék azokat a sátor elébe.
\par 4 És szóla az Úr Mózesnek, mondván:
\par 5 Vedd el õ tõlük, és legyenek azok felhasználva a gyülekezet sátorának szolgálatában; és add azokat a lévitáknak, mindeniknek az õ szolgálata szerint.
\par 6 Elvevé azért Mózes a szekereket és ökröket, és adá azokat a lévitáknak.
\par 7 Két szekeret és négy ökröt ada a Gerson fiainak, az õ szolgálatuk szerint.
\par 8 Négy szekeret pedig és nyolcz ökröt ada a Mérári fiainak, az õ szolgálatuk szerint, Ithamárnak, Áron pap fiának keze alá.
\par 9 A Kéhát fiainak pedig semmit nem ada; mert a szentség szolgálata illette vala õket, a melyet vállon hordoznak vala.
\par 10 Vivének pedig a fejedelmek az oltár felszentelésére valókat azon napon, a melyen az felkenetett, és vivék a fejedelmek az õ áldozatukat az oltár elébe.
\par 11 És monda az Úr Mózesnek: Egyik napon egyik fejedelem, másik napon másik fejedelem vigye az õ áldozatát az oltár felszentelésére.
\par 12 És vivé elsõ napon az õ áldozatát Naasson, az Amminádáb fia, Júda nemzetségébõl.
\par 13 Vala pedig az õ áldozata, egy ezüst tál, száz és harmincz siklus súlyú, egy ezüst medencze hetven siklus súlyú, a szent siklus szerint, és mind a kettõ telve vala olajjal elegyített lisztlánggal, ételáldozatul.
\par 14 Egy arany csésze, tíz siklus súlyú, füstölõ szerekkel telve.
\par 15 Egy fiatal tulok, egy kos, egy esztendõs bárány egészen égõáldozatul.
\par 16 Egy kecskebak bûnért való áldozatul.
\par 17 Hálaadó áldozatul pedig két ökör, öt kos, öt kecskebak, öt bárány, esztendõsök. Ez Naassonnak, Amminádáb fiának áldozata.
\par 18 Másodnapon vivé Néthanéel, Suárnak fia, Izsakhár nemzetségének fejedelme.
\par 19 Vive az õ áldozatául egy ezüst tálat, száz és harmincz siklus súlyút, egy ezüst medenczét, hetven siklus súlyút, a szent siklus szerint, mind a kettõ telve vala olajjal elegyített lisztlánggal, ételáldozatul.
\par 20 Egy arany csészét, tíz siklus súlyút, füstölõ szerekkel telve.
\par 21 Egy fiatal tulkot, egy kost, egy esztendõs bárányt egészen égõáldozatul.
\par 22 Egy kecskebakot bûnért való áldozatul.
\par 23 Hálaadó áldozatul pedig két ökröt, öt kost, öt bakot, öt bárányt, esztendõsöket. Ez Néthanéelnek a Suár fiának áldozata.
\par 24 Harmadik napon a Zebulon fiainak fejedelme: Eliáb, Hélon fia.
\par 25 Az õ áldozata volt egy ezüst tál, száz harmincz siklus súlyú, egy ezüst medencze, hetven siklus súlyú, a szent siklus szerint, mind a kettõ telve vala olajjal elegyített lisztlánggal, ételáldozatul.
\par 26 Egy arany csésze, tíz siklus súlyú füstölõ szerekkel telve.
\par 27 Egy fiatal tulok, egy kos, egy esztendõs bárány, egészen égõáldozatul.
\par 28 Egy kecskebak, bûnért való áldozatul.
\par 29 Hálaadó áldozatul pedig két ökör, öt kos, öt bak, öt bárány, esztendõsök. Ez Eliábnak, Hélon fiának áldozata.
\par 30 Negyedik napon a Rúben fiainak fejedelme: Elisúr, Sedeúrnak fia.
\par 31 Az õ áldozata volt egy ezüst tál, száz harmincz siklus súlyú, egy ezüst medencze, hetven siklus súlyú, a szent siklus szerint, mind a kettõ telve vala olajjal elegyített lisztlánggal, ételáldozatul.
\par 32 Egy arany csésze, tíz siklus súlyú, füstölõ szerekkel telve.
\par 33 Egy fiatal tulok, egy kos, egy esztendõs bárány, egészen égõáldozatul.
\par 34 Egy kecskebak bûnért való áldozatul.
\par 35 Hálaadó áldozatul pedig két ökör, öt kos, öt bak, öt bárány, esztendõsök. Ez Elisúrnak, Sedeúr fiának áldozata.
\par 36 Ötödnapon a Simeon fiainak fejedelme: Selúmiel, Surisaddai fia.
\par 37 Az õ áldozata volt egy ezüst tál, száz harcmincz siklus súlyú, egy ezüst medencze, hetven siklus súlyú, a szent siklus szerint, mind a kettõ telve vala olajjal elegyített lisztlánggal, ételáldozatul.
\par 38 Egy arany csésze, tíz siklus súlyú, füstölõ szerekkel telve.
\par 39 Egy fiatal tulok, egy kos, egy esztendõs bárány, egészen égõáldozatul.
\par 40 Egy kecskebak bûnért való áldozatul.
\par 41 Hálaadó áldozatul pedig két ökör, öt kos, öt bak, öt bárány, esztendõsök. Ez Selúmielnek, Surisaddai fiának áldozata.
\par 42 Hatodnapon a Gád fiainak fejedelme: Éliásáf, a Dehuél fia.
\par 43 Az õ áldozata volt egy ezüst tál, száz harcmincz siklus súlyú, egy ezüst medencze, hetven siklus súlyú, a szent siklus szerint, mind a kettõ telve vala olajjal elegyített lisztlánggal, ételáldozatul.
\par 44 Egy arany csésze, tíz siklus súlyú, füstölõ szerekkel telve.
\par 45 Egy fiatal tulok, egy kos, egy esztendõs bárány, egészen égõáldozatul.
\par 46 Egy kecskebak bûnért való áldozatul.
\par 47 Hálaadó áldozatul pedig két ökör, öt kos, öt bak, öt bárány, esztendõsök. Ez Eliásáfnak, Dehuél fiának áldozata.
\par 48 Hetednapon az Efraim fiainak fejedelme: Elisáma, az Ammihúd fia.
\par 49 Az õ áldozata volt egy ezüst tál, száz harcmincz siklus súlyú, egy ezüst medencze, hetven siklus súlyú, a szent siklus szerint, mind a kettõ telve vala olajjal elegyített lisztlánggal, ételáldozatul.
\par 50 Egy arany csésze, tíz siklus súlyú, füstölõ szerekkel telve.
\par 51 Egy fiatal tulok, egy kos, egy esztendõs bárány, egészen égõáldozatul.
\par 52 Egy kecskebak bûnért való áldozatul.
\par 53 Hálaadó áldozatul pedig két ökör, öt kos, öt bak, öt bárány, esztendõsök. Ez Elisámának, Ammihúd fiának áldozata.
\par 54 Nyolczadnapon a Manasse fiainak fejedelme: Gamliél, Pédasúr fia.
\par 55 Az õ áldozata volt egy ezüst tál, száz harcmincz siklus súlyú, egy ezüst medencze, hetven siklus súlyú, a szent siklus szerint, mind a kettõ telve vala olajjal elegyített lisztlánggal, ételáldozatul.
\par 56 Egy arany csésze, tíz siklus súlyú, füstölõ szerekkel telve.
\par 57 Egy fiatal tulok, egy kos, egy esztendõs bárány egészen égõáldozatul.
\par 58 Egy kecskebak bûnért való áldozatul.
\par 59 Hálaadó áldozatul pedig két ökör, öt kos, öt bak, öt bárány, esztendõsök. Ez Gámliélnek, a Pédasúr fiának áldozata.
\par 60 Kilenczednapon a Benjámin fiainak fejedelme: Abidán, a Gideóni fia.
\par 61 Az õ áldozata volt egy ezüst tál, száz harcmincz siklus súlyú, egy ezüst medencze, hetven siklus súlyú, a szent siklus szerint, mind a kettõ telve vala olajjal elegyített lisztlánggal, ételáldozatul.
\par 62 Egy arany csésze, tíz siklus súlyú, füstölõ szerekkel telve.
\par 63 Egy fiatal tulok, egy kos, egy esztendõs bárány egészen égõáldozatul.
\par 64 Egy kecskebak bûnért való áldozatul.
\par 65 Hálaadó áldozatul pedig két ökör, öt kos, öt bak, öt bárány, esztendõsök. Ez Abidánnak, Gideóni fiának áldozata.
\par 66 Tizedik napon a Dán fiainak fejedelme: Ahiézer, az Ammisaddai fia.
\par 67 Az õ áldozata volt egy ezüst tál, száz harcmincz siklus súlyú, egy ezüst medencze, hetven siklus súlyú, a szent siklus szerint, mind a kettõ telve vala olajjal elegyített lisztlánggal, ételáldozatul.
\par 68 Egy arany csésze, tíz siklus súlyú, füstölõ szerekkel telve.
\par 69 Egy fiatal tulok, egy kos, egy esztendõs bárány egészen égõáldozatul.
\par 70 Egy kecskebak bûnért való áldozatul.
\par 71 Hálaadó áldozatul pedig két ökör, öt kos, öt bak, öt bárány, esztendõsök. Ez Ahiézernek, az Ammisaddai fiának áldozata.
\par 72 Tizenegyedik napon az Áser fiainak fejedelme: Págiel, Okrán fia.
\par 73 Az õ áldozata volt egy ezüst tál, száz harcmincz siklus súlyú, egy ezüst medencze, hetven siklus súlyú, a szent siklus szerint, mind a kettõ telve vala olajjal elegyített lisztlánggal, ételáldozatul.
\par 74 Egy arany csésze, tíz siklus súlyú, füstölõ szerekkel telve.
\par 75 Egy fiatal tulok, egy kos, egy esztendõs bárány egészen égõáldozatul.
\par 76 Egy kecskebak bûnért való áldozatul.
\par 77 Hálaadó áldozatul pedig két ökör, öt kos, öt bak, öt bárány, esztendõsök. Ez Págielnek, Okrán fiának áldozata.
\par 78 Tizenkettedik napon a Nafthali fiainak fejedelme: Ahira, Enán fia.
\par 79 Az õ áldozata volt egy ezüst tál, száz harcmincz siklus súlyú, egy ezüst medencze, hetven siklus súlyú, a szent siklus szerint, mind a kettõ telve vala olajjal elegyített lisztlánggal, ételáldozatul.
\par 80 Egy arany csésze, tíz siklus súlyú, füstölõ szerekkel telve.
\par 81 Egy fiatal tulok, egy kos, egy esztendõs bárány egészen égõáldozatul.
\par 82 Egy kecskebak bûnért való áldozatul.
\par 83 Hálaadó áldozatul pedig két ökör, öt kos, öt bak, öt bárány, esztendõsök. Ez Ahirának, az Enán fiának áldozata.
\par 84 Ez volt az áldozat az oltár felszentelésére, a napon, a melyen felkenetett, az Izráelnek fejedelmeitõl. Tizenkét ezüst tál, tizenkét ezüst medencze, tizenkét arany csésze.
\par 85 Száz és harmincz siklus súlyú vala egy ezüst tál, egy ezüst medencze pedig hetven siklus súlyú; az edények minden ezüstje: kétezer négyszáz siklus, a szent siklus szerint;
\par 86 Tizenkét arany csésze, füstölõ szerekkel, tíz-tíz siklus súlyú vala egy-egy csésze, a szent siklus szerint: A csészéknek minden aranya: száz húsz siklus.
\par 87 Az egészen égõáldozatra való minden barom: tizenkét tulok, tizenkét kos, esztendõs bárány tizenkettõ, a hozzájok való ételáldozatokkal, és tizenkét kecskebak bûnért való áldozatul.
\par 88 A hálaadó áldozatra való minden barom pedig: huszonnégy tulok, hatvan kos, hatvan bak, esztendõs bárány hatvan. Ez volt az oltár felszentelésére való áldozat, minekutána felkenetett volt.
\par 89 Mikor pedig bemegy vala Mózes a gyülekezet sátorába, hogy szóljon õ vele, hallja vala annak szavát, a ki szól vala vele a fedél felõl, amely van a bizonyság ládáján, a két Kérub közûl; és szól vala vele.

\chapter{8}

\par 1 És szóla az Úr Mózesnek, mondván:
\par 2 Szólj Áronnak és mondd meg néki: Mikor felrakod a mécseket, a gyertyatartó elé világítson a hét mécs.
\par 3 És úgy cselekedék Áron, és úgy raká fel mécseit, hogy a gyertyatartónak ellenébe tegyenek világosságot, a miképen parancsolta vala az Úr Mózesnek.
\par 4 A gyertyatartó pedig ilyen alkotású: vert aranyból vala mind a szára, mind a virága; vert mû az; ama forma szerint,  a melyet mutatott volt az Úr Mózesnek, úgy készíték a gyertyatartót.
\par 5 Szóla továbbá az Úr Mózesnek, mondván:
\par 6 Vedd a lévitákat Izráel fiai közül, és tisztítsd meg õket.
\par 7 Így cselekedjél pedig velök, hogy megtisztítsad õket: hintsd reájok a tisztulás vizét, és az egész testöket beretválják meg, és mossák meg ruháikat, hogy tiszták legyenek.
\par 8 Azután vegyenek egy fiatal tulkot és hozzávaló ételáldozatul olajjal elegyített lisztlángot; egy másik fiatal tulkot pedig vegyenek bûnért való áldozatul.
\par 9 Akkor vidd a lévitákat a gyülekezet sátora elé, és gyûjtsd egybe Izráel fiainak egész gyülekezetét.
\par 10 Ezután vidd a lévitákat az Úr elé, és Izráel fiai tegyék kezeiket a lévitákra.
\par 11 Áron pedig lóbálja meg a lévitákat, áldozatul az Úr elõtt, Izráel fiai részérõl, hogy szolgáljanak az Úr szolgálatában.
\par 12 A léviták pedig tegyék kezeiket a tulkok fejére; azután készítsd el az egyiket bûnért való áldozatul, a másikat pedig egészen égõáldozatul az Úrnak, a lévitákért engesztelésül.
\par 13 És állassad a lévitákat Áron elé, és az õ fiai elé, és lóbáld meg õket áldozatul az Õrnak.
\par 14 És válaszd külön a lévitákat Izráel fiai közül, hogy a léviták legyenek az enyéim.
\par 15 És azután menjenek el a léviták a gyülekezet sátorában való szolgálatra. Így tisztítsd meg õket, és lóbáld meg õket áldozatul.
\par 16 Mert bizony nékem adattak õk Izráel fiai közül; mind azok helyett, a kik az õ anyjok méhét megnyitják; Izráelnek elsõszülöttei helyett vettem õket magamnak.
\par 17 Mert enyém minden elsõszülött Izráel fiai között, emberbõl és baromból; a naptól fogva, hogy megöltem minden elsõszülöttet Égyiptom földén, magamnak szenteltem azokat.
\par 18 A lévitákat pedig minden elsõszülött helyett vettem magamnak Izráel fiai között.
\par 19 És odaadtam a lévitákat adományul Áronnak és az õ fiainak Izráel fiai közül, hogy Izráel fiainak tisztében járjanak el a gyülekezet sátorában, és hogy engesztelést végezzenek Izráel fiaiért, és ne legyen Izráel fiai között csapás, ha a szenthelyhez közelednek Izráel fiai.
\par 20 Úgy cselekedék azért Mózes és Áron, és Izráel fiainak egész gyülekezete a lévitákkal; a mint parancsolt vala az Úr Mózesnek a léviták felõl, mindazt úgy cselekedék õ velök Izráel fiai.
\par 21 És megtisztíták magokat a léviták, és megmosák ruháikat, és meglóbálá õket Áron áldozatul az Úr elõtt, és engesztelést szerze nékik Áron, hogy tisztákká tegye õket.
\par 22 Azután pedig bemenének a léviták, hogy végezzék az õ szolgálatukat a gyülekezetnek sátorában Áron elõtt és az õ fiai elõtt; a miképen parancsolt vala az Úr Mózesnek a léviták felõl, akképen cselekedének velök.
\par 23 Szóla pedig az Úr Mózesnek, mondván:
\par 24 Ez is a lévitákra tartozik: Huszonöt esztendõs korától fogva és azon felül, menjen be, hogy szolgálatot teljesítsen a gyülekezet sátorának szolgálatában.
\par 25 Ötven esztendõs korától pedig lépjen ki e szolgálatnak sorából, és ne szolgáljon többé.
\par 26 Hanem segítse az õ atyafiait a gyülekezet sátorában, hogy azok az õ tisztökben eljárjanak; de szolgálatot ne teljesítsen. E képen cselekedjél a lévitákkal az õ szolgálatukban.

\chapter{9}

\par 1 Szóla pedig az Úr Mózesnek, a Sinai pusztájában, az Égyiptom földébõl való kijövetelöknek második esztendejében, az elsõ hónapban, mondván:
\par 2 Izráel fiai pedig készítsék el a páskhát a maga idejében.
\par 3 E hónapnak tizennegyedik napján, estennen készítsék el azt a maga idejében. Minden õ rendtartása szerint, és minden õ szertartása szerint készítsétek el azt.
\par 4 Szóla azért Mózes Izráel fiainak, hogy készítsék el a páskhát.
\par 5 És elkészíték a páskhát az elsõ hónapban, a hónapnak tizennegyedik napján, estennen, a Sinai pusztájában; mindent úgy cselekedének Izráel fiai, a mint az Úr parancsolta vala Mózesnek.
\par 6 Valának pedig némelyek, a kik tisztátalanok valának holtember illetése miatt, és nem készítheték meg a páskhát azon a napon: járulának azért Mózes elé és Áron elé azon a napon.
\par 7 És mondának azok az emberek néki: Mi tisztátalanok vagyunk holtember illetése miatt; miért tiltatunk meg, hogy ne vigyünk áldozatot az Úrnak a maga idejében Izráel fiai között?
\par 8 És monda nékik Mózes: Legyetek veszteg, míglen megértem: mit parancsol az Úr ti felõletek?
\par 9 Szóla pedig az Úr Mózesnek, mondván:
\par 10 Szólj Izráel fiainak, mondván: Ha valaki holtember illetése miatt tisztátalan, vagy messze úton leend közületek, vagy a ti utódaitok közül, mindazáltal készítsen páskhát az Úrnak.
\par 11 A második hónapnak tizennegyedik napján, estennen, készítsék el azt; kovásztalan kenyérrel és keserû füvekkel egyék meg azt.
\par 12 Ne hagyjanak meg abból semmit reggelig, és annak csontját meg ne törjék; a páskhának minden rendtartása szerint készítsék el azt.
\par 13 A mely ember pedig tiszta, vagy nincs útban, és elmulasztja a páskhát elkészíteni: irtassék ki az ilyen az õ népe közül: mert az Úrnak áldozatát nem vitte fel annak idejében; viselje az ilyen ember az õ bûnének terhét.
\par 14 Hogyha pedig jövevény tartózkodik köztetek, és páskhát készít az Úrnak, a páskhának rendtartása szerint és annak szertartásai szerint készítse azt; egy rendtartástok legyen néktek, mind a jövevénynek, mind a föld lakosának.
\par 15 A hajlék felállításának napján pedig befedezé a felhõ a hajlékot, a bizonyság sátorát; és estve a hajlék felett vala, tûznek ábrázatjában reggelig.
\par 16 Úgy vala szüntelen: A felhõ borítja vala azt; és tûznek ábrázatja éjjel.
\par 17 Mihelyt pedig felszáll vala a felhõ a sátorról, azonnal elindulnak vala Izráel fiai; és azon a helyen, a hol a felhõ megáll vala, ott ütnek vala tábort Izráel fiai.
\par 18 Az Úr szava szerint mennek vala Izráel fiai, és az Úr szava szerint táboroznak vala; mind addig, míg a felhõ áll vala a hajlékon, táborban maradnak vala.
\par 19 Még ha a felhõ sok napig nyugszik vala a hajlékon, akkor is megtartják vala Izráel fiai az Úr rendelését, és nem indulának.
\par 20 Megesék, hogy a felsõ kevés napon át lõn a hajlékon: akkor is az Úr szava szerint maradnak vala táborban, és az Úr szava szerint indulnak vala.
\par 21 Megesék, hogy a felhõ estvétõl fogva ott lõn reggelig; mikor azért reggel a felhõ felszáll vala, akkor indulnak vala el; vagy egy nap és egy éjjel lõn ott; mikor azért a felhõ felszáll vala, õk is indulának.
\par 22 Vagy két napig, vagy egy hónapig, vagy hosszabb ideig lõn ott; a meddig késik vala a felhõ a hajlékon, vesztegelvén azon, táborban maradnak Izráel fiai is, és nem indulnak vala tovább; mikor pedig az felszáll vala, õk is indulnak vala.
\par 23 Az Úr szava szerint járnak vala tábort, és az Úr szava szerint indulnak vala. Az Úrnak rendelését megtartják vala, az Úrnak Mózes által való szava szerint.

\chapter{10}

\par 1 Ismét szóla az Úr Mózesnek, mondván:
\par 2 Csináltass magadnak két kürtöt, vert ezüstbõl csináltasd azokat, és legyenek azok néked a gyülekezet összegyûjtésére, és a táborok megindítására.
\par 3 És mikor megfújják azokat, gyüljön te hozzád az egész gyülekezet, a gyülekezet sátorának nyílása elé.
\par 4 Ha csak egyet fújnak meg, akkor gyûljenek hozzád a fejedelmek, Izráel ezereinek fejei.
\par 5 Ha pedig riadót fújtok meg, akkor induljon azok tábora, a kik napkelet felõl táboroznak.
\par 6 Mikor pedig másodszor fújtok riadót, akkor induljon azok tábora, a kik dél felõl táboroznak. Riadót fújjanak azok indulására.
\par 7 Mikor pedig összegyûjtitek a gyülekezetet, egyszerûen kürtöljetek, és ne fújjatok riadót.
\par 8 A kürtöket pedig Áron fiai, a papok fújják; és legyen ez néktek örökkévaló rendtartás a ti nemzetségeitek között.
\par 9 És mikor viadalra mentek a ti földetekben, a titeket háborító ellenségetek ellen, akkor is azokkal a kürtökkel fújjatok riadót, és emlékezetben lesztek az Úr elõtt, a ti Istenetek elõtt, és megszabadultok a ti ellenségeitektõl.
\par 10 A ti vígasságtoknak napján, és a ti ünnepeiteken, és a ti hónapjaitok kezdetén is fújjátok meg a kürtöket, a ti egészen égõáldozataitokra, és a ti hálaáldozatitokra: és lesznek néktek emlékeztetõül a ti Istenetek elõtt. Én vagyok az Úr, a ti Istenetek.
\par 11 Vala pedig a második esztendõben a második hónapban, a hónapnak huszadik napján, felszálla a felhõ a bizonyság hajlékáról.
\par 12 És elindulának Izráel fiai az õ menésöknek rendje szerint a Sinai pusztájából; és megállapodék a felhõ Párán pusztájában.
\par 13 Elindulának azért elõször az Úrnak Mózes által való parancsolatja szerint.
\par 14 Elindula pedig elõször a Júda fiai táborának zászlója az õ serege szerint; és az õ seregének feje vala Naasson, az Amminádáb fia.
\par 15 Az Izsakhár fiai törzsébõl való seregnek feje pedig Néthanéel vala, Suárnak fia.
\par 16 És a Zebulon fiai törzsébõl való seregnek feje vala Eliáb, Hélonnak fia.
\par 17 És elbontatván a hajlék, elindulának Gersonnak és Mérárinak fiai, a hajlék hordozói.
\par 18 Azután indula a Rúben táborának zászlója az õ seregeik szerint, és az õ seregének feje vala Elisur, Sedeúrnak fia.
\par 19 A Simeon fiai törzsébõl való seregnek pedig feje vala Selúmiel, Surisaddainak fia.
\par 20 És a Gád törzsébõl való seregnek feje vala Eliásáf, Dehuélnek fia.
\par 21 Elindulának a Kéhátiták is, a szentség hordozói, és amazok felállíták vala a hajlékot, míg ezek oda jutnak vala.
\par 22 Azután elindula az Efraim fiai táborának zászlója az õ seregei szerint, és az õ seregének feje vala Elisáma, Ammihúdnak fia.
\par 23 A Manasse fiai törzsébõl való seregnek feje vala Gámliél, Pédasúrnak fia.
\par 24 A Benjámin fiai törzsébõl való seregnek feje vala Abidán, Gideóninak fia.
\par 25 Utolszor indula el a Dán fiai táborának zászlója, mint az egész tábornak utócsapata az õ seregei szerint; és az õ seregének feje vala Ahiézer; az Ammisaddai fia.
\par 26 Az Áser fiai törzsébõl való seregnek pedig feje vala Págiel, Okhránnak fia.
\par 27 És a Nafthali fiai törzsébõl való seregnek feje vala Akhira, az Enán fia.
\par 28 Ilyen vala Izráel fiainak menetele az õ seregeik szerint: ekképen mentek.
\par 29 Monda pedig Mózes Hóbábnak, a ki fia vala a Midiánból való Reuélnek, a Mózes ipának: Arra a helyre indulunk mi, a mely felõl azt mondta vala az Úr: néktek adom. Jer el velünk, és jól teszünk veled, mert az Úr jót igért Izráelnek.
\par 30 Az pedig felele néki: Nem megyek, hanem az én földemre és az én rokonaim közé megyek.
\par 31 És monda Mózes: Kérlek, ne hagyj el minket: mert te tudod, hol kell megszállanunk e pusztában, és légy nékünk szemünk gyanánt.
\par 32 És ha eljösz velünk, a mi jót cselekszik velünk az Úr, közöljük azt veled.
\par 33 Elmenének azért az Úr hegyétõl háromnapi járásnyira, és az Úr szövetségének ládája megyen vala õ elõttök háromnapi járásnyira, hogy kiszemelje nékik: hol kelljen megszállaniok.
\par 34 És az Úr felhõje vala õ rajtok nappal, mikor elindulának a táborból.
\par 35 Mikor pedig el akarták indítani a ládát, ezt mondja vala Mózes: Kelj fel Uram,  és széledjenek el a te ellenségeid, és fussanak el elõled a te gyûlölõid.
\par 36 Mikor pedig megáll vala, ezt mondja vala: Fordulj vissza Uram Izráelnek tízezerszer való ezereihez.

\chapter{11}

\par 1 És lõn, hogy panaszolkodék a nép az Úr hallására, hogy rosszul van dolga. És meghallá az Úr, és haragra gerjede, és felgyullada ellenök az Úrnak tüze és megemészté a tábornak szélét.
\par 2 Kiálta azért a nép Mózeshez, és könyörge Mózes az Úrnak, és megszünék a tûz.
\par 3 És nevezé azt a helyet Thaberának; mert felgyulladt vala ellenök az Úrnak tüze.
\par 4 De a gyülevész nép, a mely köztök vala, kívánságba esék, és Izráel fiai is újra síránkozni kezdének, és mondának: Kicsoda ád nékünk húst ennünk?
\par 5 Visszaemlékezünk a halakra, a melyeket ettünk Égyiptomban ingyen, az ugorkákra és dinnyékre, a párhagymákra, vereshagymákra és a foghagymákra.
\par 6 Most pedig a mi lelkünk eleped, mindennek híjával lévén; szemünk elõtt nincs egyéb mint manna.
\par 7 (A manna pedig olyan vala mint a kóriándrum magva, a színe pedig mint a bdelliomnak színe.
\par 8 Kiomol vala pedig a nép, és szedik vala a mannát, és õrlik vala kézimalmokban, vagy megtörik vala mozsárban, és megfõzik vala fazékban, és csinálnak vala abból pogácsákat: az íze pedig olyan vala, mint az olajos kalácsé.
\par 9 Mikor pedig a harmat leszáll vala a táborra éjjel, a manna is mindjárt leszáll vala arra.)
\par 10 És meghallá Mózes, hogy sír a nép, az õ nemzetsége szerint, kiki az õ sátorának nyílása elõtt; és igen felgerjede az Úr haragja, és nem tetszék az Mózesnek.
\par 11 És monda Mózes az Úrnak: Miért nyomorítád meg a te szolgádat? és miért nem találék kegyelmet a te szemeid elõtt, hogy ez egész népnek terhét én reám vetéd?
\par 12 Avagy tõlem fogantatott-e mind az egész nép? avagy én szûltem-e õt, hogy azt mondod nékem: Hordozd õt a te kebleden, a miképen hordozza a dajka a csecsemõt, arra a földre, a mely felõl megesküdtél az õ atyáinak?
\par 13 Hol vegyek én húst, hogy adjam azt mind ez egész népnek? mert reám sírnak, mondván: Adj nékünk húst, hadd együnk!
\par 14 Nem viselhetem én magam mind ez egész népet; mert erõm felett van.
\par 15 Ha így cselekszel velem, kérlek ölj meg engemet, ölj meg ha kedves vagyok elõtted, hogy ne lássam az én nyomorúságomat.
\par 16 Monda azért az Úr Mózesnek: Gyûjts egybe nékem hetven férfiút Izráel vénei közül, a kikrõl tudod, hogy vénei a népnek és annak elõljárói, és vidd õket a gyülekezet sátorához, és álljanak ott veled.
\par 17 Akkor alá szállok, és szólok ott veled, és elszakasztok abból a lélekbõl, a mely te benned van, és teszem õ beléjök, hogy viseljék te veled a népnek terhét, és ne viseljed te magad.
\par 18 A népnek pedig mondd meg: Készítsétek el magatokat holnapra, és húst esztek; mert sírtatok az Úr hallására, mondván: Kicsoda ád nékünk húst ennünk? mert jobban vala nékünk dolgunk Égyiptomban. Azért az Úr ád néktek hústl és enni fogtok.
\par 19 Nem csak egy napon esztek, sem két napon, sem öt napon, sem tíz napon, sem húsz napon;
\par 20 Hanem egy egész hónapig, míglen kijön az orrotokon, és útálatossá lesz elõttetek; mivelhogy megvetettétek az Urat, a ki közöttetek van; és sírtatok õ elõtte mondván: Miért jöttünk ide ki Égyiptomból?
\par 21 És monda Mózes: Hatszáz ezer gyalogos e nép, a mely között én vagyok, és te azt mondod: Húst adok nékik, és esznek egy egész hónapig?!
\par 22 Nemde juhok és ökrök vágattatnak-é nékik, hogy elég legyen nékik? vagy a tengernek minden hala összegyûjtetik-é nékik, hogy elég legyen nékik?
\par 23 Akkor monda az Úr Mózesnek: Avagy megrövidült-é az Úrnak keze? Majd meglátod: beteljesedik-é néked az én beszédem vagy nem?
\par 24 Kiméne azért Mózes, és elmondá a népnek az Úr beszédét, és összegyûjte hetven férfiút a nép vénei közül, és állatá õket a sátor körül.
\par 25 Akkor leszálla az Úr felhõben, és szóla néki, és elszakaszta abból a lélekbõl, a mely vala õ benne, és adá a hetven vén férfiúba. Mihelyt pedig megnyugovék õ rajtok a lélek, menten prófétálának, de nem többé.
\par 26 Két férfiú azonban elmaradt vala a táborban; egyiknek neve Eldád, a másiknak neve Médád, és ezeken is megnyugodott vala a lélek; mert azok is az összeírottak közül valók, de nem mentek vala el a sátorhoz, és mégis prófétálának a táborban.
\par 27 Elfutamodék azért egy ifjú, és megjelenté Mózesnek, és monda: Eldád és Médád prófétálnak a táborban.
\par 28 Akkor felele Józsué, a Nún fia, Mózes szolgája, az õ választottai közül való, és monda: Uram, Mózes, tiltsd meg õket!
\par 29 És felele néki Mózes: Avagy érettem buzgólkodol-é? Vajha az Úrnak minden népe próféta volna, hogy adná az Úr az õ lelkét õ beléjök.
\par 30 Ezután visszatére Mózes a táborba, õ és az Izráel vénei.
\par 31 És szél jöve ki az Úrtól, és hoza fürjeket a tengertõl, és bocsátá a táborra egynapi járásnyira egy felõl, és egynapi járásnyira más felõl a tábor körül, és mintegy két sing magasságnyira a földnek színén.
\par 32 Akkor felkele a nép és azon az egész napon, és egész éjjel, és az egész következõ napon gyûjtének magoknak fürjeket, a ki keveset gyûjtött is, gyûjtött tíz hómert, és kiteregeték azokat magoknak a tábor körül.
\par 33 A hús még foguk között vala, és meg sem emésztették vala, a mikor az Úrnak haragja felgerjede a népre és megveré az Úr a népet igen nagy csapással.
\par 34 És elnevezék azt a helyet Kibrot-thaavának: mert ott temeték el a mohó népet.
\par 35 Kibrot-thaavától elméne a nép Haseróthba; és ott valának Haseróthban.

\chapter{12}

\par 1 Miriám pedig és Áron szólának Mózes ellen a kúsita asszony miatt, a kit feleségül võn, mert kúsita asszonyt vett vala feleségül.
\par 2 És mondának: Avagy csak Mózes által szólott-é az Úr? avagy nem szólott-é mi általunk is? És meghallá az Úr.
\par 3 (Az az ember pedig, Mózes, igen szelíd vala, minden embernél inkább, a kik e föld színén vannak.)
\par 4 Mindjárt monda azért az Úr Mózesnak, Áronnak és Miriámnak: Menjetek ki ti hárman a gyülekezetnek sátorába; és kimenének õk hárman.
\par 5 Akkor leszálla az Úr felhõnek oszlopában, és megálla a sátornak nyílásánál; és szólítá Áront és Miriámot, és kimenének mindketten.
\par 6 És monda: Halljátok meg most az én beszédeimet: Ha valaki az Úr prófétája közöttetek, én megjelenek annak látásban, vagy álomban szólok azzal.
\par 7 Nem így az én szolgámmal, Mózessel, a ki az én egész házamban hív.
\par 8 Szemtõl szembe szólok õ vele, és nyilvánvaló látásban; nem homályos beszédek által, hanem az Úrnak hasonlatosságát látja. Miért nem féltetek hát szólani az én szolgám ellen, Mózes ellen?
\par 9 És felgyullada az Úr haragja õ reájok, és elméne.
\par 10 És a felhõ is eltávozék a sátor felül, és ímé Miriám poklos vala, fejér mint a hó; és rátekinte Áron Miriámra, és ímé poklos vala.
\par 11 Monda azért Áron Mózesnek: Kérlek Uram, ne tulajdonítsad nékünk e bûnt; mert bolondul cselekedtünk és vétkeztünk!
\par 12 Kérlek, ne legyen olyan Miriám mint a holt, a melynek húsa félig megemésztetik, mikor kijõ az õ anyjának méhébõl.
\par 13 Kiálta azért Mózes az Úrhoz, mondván: Isten, kérlek, gyógyítsd meg õt!
\par 14 Az Úr pedig monda Mózesnek: Ha csak az atyja pökött volna is az õ orczájára, avagy nem kellene-é szégyenkeznie hetednapig? Rekesztessék ki hét napig a táboron kívül, és azután hívassék vissza.
\par 15 Kirekeszteték azért Miriám a táboron kivül hét napig. És a nép nem indula tovább, míg vissza nem hívaték Miriám.

\chapter{13}

\par 1 Azután pedig elindula a nép Haseróthból, és tábort ütének Párán pusztájában.
\par 2 És szóla az Úr Mózesnek, mondván:
\par 3 Küldj férfiakat, hogy kémleljék meg a Kanaán földét, a melyet én adok Izráel fiainak; az õ atyáiknak mindenik törzsébõl egy-egy férfiút küldjetek, mind olyat, a ki fõember közöttök.
\par 4 Elküldé azért õket Mózes Párán pusztájából az Úr rendelése szerint; és azok a férfiak mindnyájan fõemberek valának Izráel fiai között.
\par 5 Ezek pedig azoknak nevei: A Rúben nemzetségébõl Sámmua, a Zakhúr fia.
\par 6 A Simeon nemzetségébõl Safát, a Hóri fia.
\par 7 A Júda nemzetségébõl Káleb, a Jefunné fia.
\par 8 Az Izsakhár nemzetségébõl Igál, a József fia.
\par 9 Az Efraim nemzetségébõl Hósea, a Nún fia.
\par 10 A Benjámin nemzetségébõl Pálthi, a Rafú fia.
\par 11 A Zebulon nemzetségébõl Gaddiel, a Szódi fia.
\par 12 A József nemzetségébõl, a Manasse nemzetségébõl Gaddi, a Szúszi fia.
\par 13 A Dán nemzetségébõl Ammiél, a Gemálli fia.
\par 14 Az Áser nemzetségébõl Szenthúr, a Mikaél fia.
\par 15 A Nafthali nemzetségébõl Nahbi, a Vofszi fia.
\par 16 A Gád nemzetségébõl Géuel, a Mákhi fia.
\par 17 Ezek ama férfiaknak nevei, a kiket elküldött Mózes a földnek megkémlelésére. És nevezé Mózes Hóseát, a Nún fiát Józsuénak.
\par 18 És mikor elküldé õket Mózes a Kanaán földének megkémlelésére, monda nékik: Menjetek fel erre dél felõl, és hágjatok fel a hegyre.
\par 19 És nézzétek meg a földet, hogy milyen az; és a népet, a mely lakozik azon: erõs-é az vagy erõtlen, kevés-é az vagy sok?
\par 20 És milyen a föld, a melyben lakozik az: jó-é az vagy hitvány; és milyenek a városok, a melyekben lakozik: táborokban vagy erõsségekben lakozik-é?
\par 21 És milyen a föld: kövér-é az, vagy sovány; van-é abban élõfa vagy nincs? És legyetek bátorságosok, és szakaszszatok a földnek gyümölcsébõl. Azok a napok pedig a szõlõzsendülés napjai valának.
\par 22 Felmenének azért, és megkémlelék a földet a Czin pusztájától fogva mind Rehóbig, odáig, a hol Hamáthba mennek.
\par 23 És felmenének dél felõl, és jutának Hebronig; valának pedig ott Ahimán, Sésai és Thalmai, az Anák fiai: (Hebron pedig hét esztendõvel építtetett elébb, mint az égyiptomi Czoán).
\par 24 Mikor pedig eljutának Eskol völgyéig, lemetszének ott egy szõlõvesszõt egy szõlõfürttel, és ketten vivék azt rúdon; és a gránátalmákból és a figékbõl is szakasztának.
\par 25 Azt a helyet elnevezék Eskol völgyének, a szõlõfürtért, a melyet lemetszettek vala onnét Izráel fiai.
\par 26 És visszatérének a föld megkémlelésébõl negyven nap mulva.
\par 27 És menének, és jutának Mózeshez és Áronhoz, és Izráel fiainak egész gyülekezetéhez, Párán pusztájába, Kádesbe; és hírt vivének nékik és az egész gyülekezetnek, és megmutaták nékik a földnek gyümölcsét.
\par 28 Így beszélének néki, és ezt mondák: Elmentünk vala arra a földre, a melyre küldöttél vala minket, és bizonyára tejjel és mézzel folyó az, és ez annak gyümölcse!
\par 29 Csakhogy erõs az a nép, a mely lakja azt a földet, és a városok erõsítve vannak, és felette nagyok; sõt még Anák fiait is láttuk ott!
\par 30 Amálek lakik a dél felõl való földön, és Khitteus, Jebuzeus és Emoreus lakik a hegyeken; a tenger mellett, és a Jordán partján pedig Kananeus lakik.
\par 31 És jóllehet Káleb csendesíté a Mózes ellen háborgó népet, és azt mondja vala: Bátran felmehetünk, és elfoglalhatjuk azt a földet, mert kétség nélkül megbírunk azzal;
\par 32 Mindazáltal a férfiak, a kik felmentek vala vele, azt mondják vala: Nem mehetünk fel az ellen a nép ellen, mert erõsebb az nálunknál.
\par 33 És rossz hírét vivék annak a földnek, a melyet megkémleltek volt, Izráel fiaihoz, mondván: Az a föld, a melyen általmentünk, hogy megkémleljük azt, olyan föld, a mely megemészti az õ lakóit; az egész nép is, a melyet láttunk azon, szála emberekbõl áll.
\par 34 És láttunk ott óriásokat is, az óriások közül való Anáknak fiait, és olyanok valánk a magunk szemében, mint a sáskák, és az õ szemeikben is olyanok valánk.

\chapter{14}

\par 1 És felemelé szavát az egész gyülekezet, és síra a nép azon az éjszakán.
\par 2 És mindnyájan zúgolódának Mózes ellen és Áron ellen Izráel fiai, és monda nékik az egész gyülekezet: Vajha megholtunk volna Égyiptom földén! vagy ebben a pusztában vajha meghalnánk!
\par 3 Miért is visz be minket az Úr arra a földre? hogy fegyver miatt hulljunk el? feleségeink és a kicsinyeink prédára legyenek? Nem jobb volna-é nékünk visszatérnünk Égyiptomba?
\par 4 És mondának egymásnak: Szerezzünk elõttünk járót, és térjünk vissza Égyiptomba.
\par 5 Akkor arczczal leborulának Mózes és Áron Izráel fiai gyülekezetének egész községe elõtt.
\par 6 Józsué pedig, a Nún fia, és Káleb, a Jefunné fia, a kik a földnek kémlelõi közül valók valának, meghasogaták ruháikat.
\par 7 És szólának Izráel fiai egész gyülekezetének, mondván: A föld, a melyen általmentünk, hogy megkémleljük azt, igen-igen jó föld.
\par 8 Ha az Úrnak kedve telik bennünk, akkor bevisz minket arra a földre, és nékünk adja azt, mely tejjel és mézzel folyó föld.
\par 9 Csakhogy ne lázongjatok az Úr ellen, se ne féljetek annak a földnek népétõl; mert õk nekünk csak olyanok, mint a kenyér; eltávozott tõlök az õ oltalmok, de az Úr velünk van: ne féljetek tõlök!
\par 10 Mikor pedig az egész gyülekezet azon tanakodék, hogy megkövezze õket: megjelenék az Úrnak dicsõsége a gyülekezet sátorában Izráel minden fiának.
\par 11 És monda az Úr Mózesnek: Meddig gyaláz engemet ez a nép? Meddig nem hisznek nékem, mind ama csudatételeim mellett sem, a melyeket cselekedtem közöttök?
\par 12 Megverem õket döghalállal, és elvesztem õket; téged pedig nagy néppé teszlek, és õ nálánál erõsebbé.
\par 13 És monda Mózes az Úrnak: Ha meghallják az égyiptombeliek (mert közülök hoztad fel e népet a te hatalmad által):
\par 14 Elmondják majd e föld lakosainak, a kik hallották, hogy te Uram e nép között vagy, hogy szemtõl szembe megjelentetted magadat te Uram, és hogy a te felhõd megállott õ rajtok, és felhõoszlopban jársz te õ elõttök nappal, éjjel pedig tûzoszlopban.
\par 15 Hogyha mind egyig elveszted e népet, így szólanak majd e népek, a melyek hallották a te híredet, mondván:
\par 16 Mivelhogy nem vihette be az Úr e népet a földre, a mely felõl megesküdött nékik, azért öldöste le õket a pusztában.
\par 17 Most azért hadd magasztaltassék fel az Úrnak ereje, a miképen szólottál, mondván:
\par 18 Az Úr késedelmes a haragra, nagy irgalmasságú, megbocsát hamisságot és vétket, de a bûnöst nem hagyja büntetlenül;  megbünteti az atyák álnokságait a fiakban harmad és negyed íziglen.
\par 19 Kérlek, kegyelmezz meg e nép hamisságának a te irgalmasságod nagy volta szerint, a miképen megbocsátottál e népnek Égyiptomtól fogva mind eddig.
\par 20 És monda az Úr: Megkegyelmeztem a te beszéded szerint.
\par 21 De bizonnyal élek én, és betölti az Úr dicsõsége az egész földet.
\par 22 Hogy mindazok az emberek, a kik látták az én dicsõségemet és csudáimat, a melyeket cselekedtem Égyiptomban és e pusztában, és megkisértettek engemet immár tízszer, és nem engedtek az én szómnak:
\par 23 Nem látják meg azt a földet, a mely felõl megesküdtem az õ atyáiknak; senki nem látja azt azok közöl, a kik gyaláztak engem.
\par 24 De az én szolgámat, Kálebet, mivelhogy más lélek volt vele, és tökéletességgel követett engem, beviszem õt arra a földre, a melyre bement vala, és örökségül bírja azt az õ magva.
\par 25 De Amálek és Kananeus lakik a völgyben; holnap forduljatok meg, és induljatok a pusztába, a veres tenger útján.
\par 26 Szóla annakfelette az Úr Mózesnek és Áronnak, mondván:
\par 27 Meddig tûrjek e gonosz gyülekezetnek, a mely zúgolódik ellenem? Hallottam Izráel fiainak zúgolódásait, a kik zúgolódnak ellenem!
\par 28 Mondd meg nékik: Élek én, azt mondja az Úr, hogy épen úgy cselekszem veletek, a miképen szólottatok az én füleim hallására!
\par 29 E pusztában hullanak el a ti holttesteitek, és pedig mindazok, a kik megszámláltattak a ti teljes számotok szerint, húsz esztendõstõl fogva és azon felül, a kik zúgolódtatok ellenem.
\par 30 És nem mentek be arra a földre, a melyre nézve felemeltem az én kezemet, hogy lakosokká teszlek abban titeket; Kálebet, a Jefunné fiát és Józsuét, a Nún fiát kivéve.
\par 31 De kicsinyeiteket, a kik felõl azt mondtátok, hogy prédára lesznek; õket beviszem, és megismerik azt a földet, a melyet megútáltatok.
\par 32 A ti holttesteitek azért a pusztában hullanak el.
\par 33 A ti fiaitok pedig, mint a pásztorok, bujdosnak e pusztában negyven esztendeig, és viselik a ti paráználkodásitoknak büntetését, míglen megemésztetnek a ti holttesteitek e pusztában.
\par 34 A napok száma szerint, a melyeken megkémleltétek a földet, (tudniillik negyven napon, egy-egy napért egy-egy esztendõ), negyven esztendeig hordozzátok a ti hamisságotoknak büntetését, és megismeritek az én elfordulásomat.
\par 35 Én, az Úr, szólottam. Bizonyára ezt mívelem az egész gonosz gyülekezettel, a mely összegyülekezett vala ellenem; ebben a pusztában emésztetnek meg, és ugyanott halnak meg.
\par 36 A férfiak azért, a kiket elküldött vala Mózes a földnek megkémlelésére, és visszatérének és felzúdíták ellene az egész gyülekezetet, rossz hírt terjesztvén arról a földrõl;
\par 37 Azok a férfiak azért, a kik rossz hírt terjesztének a földrõl, meghalának az Úr elõtt csapás által.
\par 38 Csak Józsué, a Nún fia, és Káleb, a Jefunné fia, maradának életben ama férfiak közül, a kik mentek vala a földet megkémlelni.
\par 39 A mint pedig elbeszélé Mózes e beszédeket Izráel fiainak, a nép felette igen keserge.
\par 40 És felkelének reggel, és felmenének a hegy tetejére, mondván: Ímé készek vagyunk elmenni a helyre, a melyrõl szólott az Úr, mert vétkeztünk.
\par 41 És monda Mózes: Miért hágjátok át ilyen módon az Úr akaratát, holott nem sikerülhet az néktek.
\par 42 Fel ne menjetek, mert nem lesz közöttetek az Úr, hogy el ne hulljatok a ti ellenségeitek elõtt.
\par 43 Mert az Amálek és a Kananeus van ott elõttetek, és fegyver által hulltok el. Mivelhogy elfordultatok az Úrtól, nem is lesz az Úr veletek.
\par 44 Mindazonáltal merészkedének felmenni a hegy tetejére; de az Úr szövetségének ládája és Mózes meg sem mozdulának a táborból.
\par 45 Alászálla azért az Amálek és a Kananeus, a ki lakik vala azon a hegyen, és megverék õket, és vágák õket mind Hormáig.

\chapter{15}

\par 1 Szóla pedig az Úr Mózesnek, mondván:
\par 2 Szólj Izráel fiainak, és mondd nékik: Mikor bementek a ti lakó földetekre, a melyet én adok néktek.
\par 3 És tûzáldozatot akartok készíteni az Úrnak, egészen égõ- vagy véres áldozatot, vagy fogadás teljesítése végett, vagy szabadakarat szerint, vagy a ti ünnepiteken, hogy tulok- vagy juhfélébõl kedves illatot készítsetek az Úrnak:
\par 4 Akkor, a ki áldozza az õ áldozatját, vigyen az Úrnak ételáldozatul egy tizedrész efa lisztlángot, egy negyedrész hin olajjal elegyítve.
\par 5 Italáldozatul pedig egy negyedrész hin bort adj az egészen égõ- vagy véres áldozathoz egy-egy bárány mellé.
\par 6 Vagy ha kossal áldozol, készíts ételáldozatul két tizedrész efa lisztlángot megelegyítve egy hin olajnak harmadrészével;
\par 7 Italáldozatul pedig egy hin bornak harmadrészét. Így viszel az Úrnak jóillatú áldozatot.
\par 8 Hogyha fiatal tulkot akarsz készíteni egészen égõ- vagy véres áldozatul fogadás teljesítése végett, vagy hálaáldozatul az Úrnak:
\par 9 Akkor vígy a fiatal tulokkal egyetemben ételáldozatul három tizedrész efa lisztlángot, megelegyítve egy hin olajnak felével.
\par 10 Italáldozatul pedig vigy fel fél hin bort; kedves illatú tûzáldozat ez az Úrnak.
\par 11 Eképpen cselekedjetek mindenik ökörnél, mindenik kosnál, mind a juhoknak, mind a kecskéknek bárányainál.
\par 12 A barmok száma szerint, a melyeket áldozatra készítetek: így cselekedjetek mindenikkel, az õ számok szerint.
\par 13 Minden benszülött így cselekedjék ezekkel, hogy kedves illatú tûzáldozatot vigyen az Úrnak.
\par 14 Ha pedig jövevény tartózkodik nálatok, vagy lesz köztetek a ti nemzetségeitek szerint, és kedves illatú tûzáldozatot készít az Úrnak: a miképen ti cselekesztek, a képen cselekedjék az is.
\par 15 Óh, község! néktek és a köztetek lakozó jövevénynek egy rendtartástok legyen; örökkévaló törvény legyen a ti nemzetségeiteknél, hogy az Úr elõtt olyan legyen a jövevény, mint ti.
\par 16 Egy törvényetek legyen, és egy szabályotok néktek és a jövevénynek, a mely közöttetek lakik.
\par 17 Szóla azután az Úr Mózesnek, mondván:
\par 18 Szólj Izráel fiainak, és mondd meg nékik: Mikor bementek a földre, a melyre én viszlek be titeket:
\par 19 Akkor, ha majd esztek a földnek kenyerébõl, adjatok felemelt áldozatot az Úrnak.
\par 20 A ti tésztátok zsengéjét lepényként mutassátok fel áldozatul; mint a szérûrõl felemelt áldozatot, úgy mutassátok fel azt.
\par 21 A ti tésztátok zsengéjébõl adjatok az Úrnak felemelt áldozatot, a ti nemzetségeitek szerint.
\par 22 És ha megtévelyedtek, és nem cselekszitek meg mind e parancsolatokat, a melyeket szólott vala az Úr Mózesnek;
\par 23 Mind azokat, a melyeket parancsolt az Úr néktek Mózes által, az naptól fogva, a melyen parancsolta azokat az Úr, és azután a ti nemzetségeiteknek is:
\par 24 Akkor, hogyha a gyülekezet tudtán kivül esik a tévedés, az egész gyülekezet áldozzék egy fiatal tulkot egészen égõáldozatul, kedves illatul az Úrnak, és étel- és italáldozatot is hozzá szokás szerint, és egy kecskebakot bûnért való áldozatul.
\par 25 És végezzen engesztelést a pap Izráel fiainak egész gyülekezetéért, és megbocsáttatik nékik, mert tévedés volt az; õk pedig vigyék be az õ áldozatjokat tûzáldozatul az Úrnak, és a bûnért való áldozatjokat az Úr elé az õ tévedésökért.
\par 26 És megbocsáttatik Izráel fiai egész gyülekezetének, és a közöttök tartózkodó jövevénynek; mert az egész nép tévedésben volt.
\par 27 Hogyha csak egy ember vétkezik tévedésbõl: áldozzék egy esztendõs kecskét bûnért való áldozatul.
\par 28 A pap végezzen engesztelést azért a tévedõ emberért, a ki tévedésbõl vétkezett az Úr elõtt, és ha engesztelést szerez néki, megbocsáttatik néki.
\par 29 A benszülöttnek Izráel fiai között és a jövevénynek, a ki közöttök tartózkodik: egy törvényetek legyen néktek a felõl, a ki tévedésbõl cselekszik.
\par 30 De a mely ember felemelt kézzel cselekszik, akár benszülött akár jövevény, az Urat illeti az szidalommal, vágassék ki azért az az õ népe közül;
\par 31 Mivelhogy az Úrnak szavát megvetette, és az õ parancsolatját megszegte, kiirtatván kiirtassék az az ember, az õ hamissága legyen õ rajta.
\par 32 Mikor pedig Izráel fiai a pusztában valának, találának egy férfiat, ki fád szedeget vala szombatnapon.
\par 33 És elvivék azt, a kik találták vala azt fát szedegetni, Mózeshez és Áronhoz és az egész gyülekezethez.
\par 34 És õrizet alá adák azt, mert nem vala kijelentve, mit kelljen vele cselekedni.
\par 35 És monda az Úr Mózesnek: Halállal lakoljon az a férfi, kövezze õt agyon az egész gyülekezet a táboron kivül.
\par 36 Kivivé azért õt az egész gyülekezet a táboron kivül, és agyon kövezék õt, és meghala, a miképen parancsolta vala az Úr Mózesnek.
\par 37 És szóla az Úr Mózesnek, mondván:
\par 38 Szólj Izráel fiainak, és mondjad nékik, hogy készítsenek magoknak bojtokat az õ ruháik szegleteire az õ nemzetségeik szerint, és tegyenek a szeglet bojtjára kék zsinórt.
\par 39 És arra való legyen néktek a bojt, hogy mikor látjátok azt, megemlékezzetek az Úrnak minden parancsolatjáról, hogy megcselekedjétek azokat; és ne nézzetek a ti szívetek után, és a ti szemeitek után, a melyek után ti paráználkodtok.
\par 40 Hogy megemlegessétek, és megcselekedjétek minden én parancsolatomat, és legyetek szentek a ti Istenetek elõtt.
\par 41 Én vagyok az Úr, a ti Istenetek, a ki kihoztalak titeket Égyiptom földébõl, hogy legyek néktek Istenetekké. Én vagyok az Úr, a ti Istenetek.

\chapter{16}

\par 1 Kóré pedig az Iczhár fia, a ki a Lévi fiának, Kéhátnak fia vala; és Dáthán és Abirám, Eliábnak fiai; és On, a Péleth fia, a kik Rúben fiai valának, fogták magokat;
\par 2 És támadának Mózes ellen, és velök Izráel fiai közül kétszáz és ötven ember, a kik a gyülekezetnek fejedelmei valának, tanácsbeli híres neves emberek.
\par 3 És gyülekezének Mózes ellen és Áron ellen, és mondának nékik: Sokat tulajdonítotok magatoknak, holott az egész gyülekezet, ezek mindnyájan szentek, és közöttök van az Úr: miért emelitek azért fel magatokat az Úr gyülekezete fölé?
\par 4 És mikor hallá ezt Mózes, arczra borula,
\par 5 És szóla Kórénak és az õ egész gyülekezetének, mondván: Reggel megmutatja az Úr: ki az övé és ki a szent, és kit fogadott magához; mert a kit magának választott, magához fogadja azt.
\par 6 Ezt cselekedjétek azért: Vegyetek magatoknak temjénezõket, Kóré és az õ egész gyülekezete!
\par 7 És tegyetek azokba tüzet, és rakjatok rá füstölõ szert az Úr elõtt holnap, és az a férfiú legyen szent, a kit kiválaszt az Úr. Sokat tulajdonítotok magatoknak, Lévi fiai!
\par 8 És monda Mózes Kórénak: Halljátok meg, kérlek, Lévi fiai:
\par 9 Avagy keveslitek-é azt, hogy titeket Izráel Istene külön választott Izráel gyülekezetétõl, hogy magához fogadjon titeket, hogy szolgáljatok az Úr sátorának szolgálatában, hogy álljatok e gyülekezet elõtt, és szolgáljatok néki?
\par 10 És hogy magának fogadott tégedet, és minden atyádfiát, a Lévi fiait te veled; hanem még a papságot is kivánjátok?
\par 11 Azért hát te és a te egész gyülekezeted az Úr ellen gyülekeztetek össze; mert Áron micsoda, hogy õ ellene zúgolódtok?
\par 12 Elkülde azután Mózes, hogy hívják elõ Dáthánt és Abirámot, az Eliáb fiait. Azok pedig felelének: Nem megyünk fel!
\par 13 Avagy kevesled-é azt, hogy felhozál minket a tejjel és mézzel folyó földrõl, hogy megölj minket a pusztában; hanem még uralkodni is akarsz rajtunk?
\par 14 Éppen nem tejjel és mézzel folyó földre hoztál be minket, sem szántóföldet és szõlõt nem adtál nékünk örökségül! Avagy ki akarod-é szúrni az emberek szemeit? Nem megyünk fel!
\par 15 Megharaguvék azért Mózes igen, és monda az Úrnak: Ne tekints az õ áldozatjokra! Egy szamarat sem vettem el tõlök, és egyet sem bántottam közülök.
\par 16 Azután monda Mózes Kórénak: Te és a te egész gyülekezeted legyetek az Úr elõtt; te és azok és Áron, holnap.
\par 17 És kiki vegye az õ temjénezõjét, és tegyetek abba füstölõ szert, és vigyétek az Úr elé, kiki az õ temjénezõjét; kétszáz és ötven temjénezõt. Te is, és Áron is, kiki az õ temjénezõjét.
\par 18 Vevé azért kiki az õ temjénezõjét, és tevének abba tüzet, és rakának arra füstölõ szert, és megállának a gyülekezet sátorának nyílása elõtt, Mózes is és Áron.
\par 19 Kóré pedig összegyûjtötte vala ellenök az egész gyülekezetet a gyülekezet sátorának nyílásához, és megjelenék az Úrnak dicsõsége az egész gyülekezetnek.
\par 20 És szóla az Úr Mózesnek és Áronnak, mondván:
\par 21 Váljatok külön e gyülekezettõl, hogy megemészszem  õket egy szempillantásban.
\par 22 Õk pedig arczukra borulának, és mondának: Isten, minden test lelkének Istene! nem egy férfiú vétkezett-é, és az egész gyülekezetre haragszol-é?
\par 23 Akkor szóla az Úr Mózesnek, mondván:
\par 24 Szólj a gyülekezetnek, mondván: Menjetek el a Kóré, Dáthán és Abirám hajléka mellõl köröskörül.
\par 25 Felkele azért Mózes, és elméne Dáthán és Abirám felé, követék õt Izráel vénei.
\par 26 És szóla a gyülekezetnek, mondván: Kérlek, távozzatok el ez istentelen emberek sátorai mellõl, és semmit ne illessetek abból, a mi az övék, hogy el ne veszszetek az õ bûneik miatt.
\par 27 És elmenének a Kóré, Dáthán és Abirám hajlékai mellõl köröskörül; Dáthán pedig és Abirám kimenének, megállván az õ sátoraiknak nyílásánál feleségeikkel, fiaikkal és kisdedeikkel.
\par 28 Akkor monda Mózes: Ebbõl tudjátok meg, hogy az Úr küldött engemet, hogy cselekedjem mind e dolgokat, hogy nem magamtól indultam:
\par 29 Ha úgy halnak meg ezek, a mint meghal minden más ember, és ha minden más ember büntetése szerint büntettetnek meg ezek: akkor nem az Úr küldött engemet.
\par 30 Ha pedig az Úr valami új dolgot cselekszik, és a föld megnyitja az õ száját, és elnyeli õket, és mindazt, a mi az övék, és elevenen szállanak alá pokolba: akkor megismeritek, hogy gyalázták ezek az emberek az Urat.
\par 31 És lõn, a mint elvégezé mind e beszédeket, meghasada a föld alattok.
\par 32 És megnyitá a föld az õ száját, és elnyelé õket és az õ háznépeiket: és minden embert, a kik Kóréé valának, és minden jószágukat.
\par 33 És alászállának azok és mindaz, a mi az övék, elevenen a pokolra: és befedezé õket a föld, és elveszének a község közül.
\par 34 Az Izraeliták pedig, a kik körülöttök valának, mind elfutának azoknak kiáltására; mert azt mondják vala: netalán elnyel minket a föld!
\par 35 És tûz jöve ki az Úrtól, és megemészté ama kétszáz és ötven férfiút, a kik füstölõ szerekkel áldoznak vala.
\par 36 Szóla pedig az Úr Mózesnek, mondván:
\par 37 Mondd meg Eleázárnak, Áron pap fiának, hogy szedje ki a temjénezõket a tûzbõl, a tüzet pedig hintsd széjjel, mert megszenteltettek a temjénezõk;
\par 38 Ezeknek temjénezõi, a kik a magok lelke ellen vétkeztek. És csináljanak azokból vékonyra vert lapokat az oltár beborítására. Mivelhogy járultak azokkal az Úr elé, és megszenteltettek: legyenek jegyül Izráel fiainak.
\par 39 Felszedé azért Eleázár pap a réz temjénezõket, a melyekkel a megégettek járultak vala oda, és vékonyra verette azokat az oltár beborítására.
\par 40 Emlékeztetõül Izráel fiainak, hogy senki idegen, a ki nem az Áron magvából való, ne járuljon az Úr elé füstölõ szerrel füstölögtetni, hogy úgy ne járjon, mint Kóré és mint az õ gyülekezete, a miképen megmondotta vala néki az Úr Mózes által.
\par 41 És másnap felzúdula Izráel fiainak egész gyülekezete Mózes ellen és Áron ellen, mondván: Ti öltétek meg az Úrnak népét!
\par 42 Mikor pedig egybegyûle a gyülekezet Mózes ellen és Áron ellen, akkor fordulának a gyülekezet sátora felé: és íme befedezte vala azt a felhõ, és megjelenék az Úrnak dicsõsége.
\par 43 Mózes azért és Áron menének a gyülekezet sátora elé.
\par 44 És szóla az Úr Mózesnek, mondván:
\par 45 Menjetek ki e gyülekezet közül, hogy megemészszem õket egy szempillantásban; õk pedig orczájokra borulának.
\par 46 És monda Mózes Áronnak: Fogd a temjénezõt, és tégy abba tüzet az oltárról, és rakj reá füstölõ szert, és vidd hamar a gyülekezethez, és végezz engesztelést értök, mert kijött az Úrtól a nagy harag, elkezdõdött a csapás.
\par 47 Vevé azért Áron a temjénezõt, a mint mondotta vala Mózes, és futa a község közé, és ímé elkezdõdött vala a csapás a nép között. És füstölõ áldozatot tõn, és engesztelést szerze a népnek.
\par 48 És megálla a megholtak között és élõk között; és megszûnék a csapás.
\par 49 És valának, a kik megholtak vala e csapás alatt, tizennégy ezer hétszázan; azokon kivül, a kik megholtak vala a Kóré dolgáért.
\par 50 És visszatére Áron Mózeshez a gyülekezet sátorának nyílásához. Így szûnék meg a csapás.

\chapter{17}

\par 1 És szóla az Úr Mózesnek, mondván:
\par 2 Szólj Izráel fiainak, és végy tõlök egy-egy vesszõt az õ atyáiknak háza szerint; az õ atyáik házának valamennyi fejedelmétõl tizenkét vesszõt; és kinek-kinek a nevét írd fel az õ vesszejére.
\par 3 Az Áron nevét pedig írd a Lévi vesszejére: mert egy vesszõ esik az õ atyjok házának fejéért.
\par 4 És tedd le azokat a gyülekezet sátorában a bizonyság ládája elé, a hol megjelenek néktek.
\par 5 És lesz, hogy annak a férfiúnak vesszeje, a kit elválasztok, kihajt; így hárítom el magamról Izráel fiainak zúgolódásait, a melyekkel zúgolódnak ti ellenetek.
\par 6 Szóla azért Mózes Izráel fiainak, és adának néki mind az õ fejedelmeik egy-egy vesszõt egy-egy fejedelemért; az õ atyáiknak háza szerint tizenkét vesszõt; az Áron vesszeje is azok között a vesszõk között vala.
\par 7 És letevé Mózes a vesszõket az Úr elé a bizonyság sátorában.
\par 8 És lõn másnap, hogy beméne Mózes a bizonyság sátorába; és ímé kihajtott vala a Lévi házából való Áronnak vesszeje, és hajtást hajtott, és virágot növelt, és mandolát érlelt.
\par 9 És kihozá Mózes mind azokat a vesszõket az Úr színe elõl mind az Izráel fiai elé, és miután megnézték vala, vevé kiki az õ vesszejét.
\par 10 És monda az Úr Mózesnek: Vidd vissza az Áron vesszejét a bizonyság ládája elé, hogy õriztessék ott a lázadó fiaknak jegyül, hogy megszünjék az én ellenem való zúgolódások, hogy meg ne haljanak.
\par 11 És megcselekedé Mózes; a mint parancsolta vala az Úr néki, akképen cselekedék.
\par 12 És szólának Izráel fiai Mózesnek, mondván: Ímé pusztulunk, veszünk, mi mindnyájan elveszünk!
\par 13 Valaki közel járul az Úrnak sátorához, meghal. Avagy mindenestõl elpusztulunk-é?!

\chapter{18}

\par 1 És monda az Úr Áronnak: Te és a te fiaid, és a te atyádnak háza te veled, hordozzátok a szent hajlék körül való hamisság büntetését. Te és a te fiaid te veled, hordozzátok a ti papságtok hamisságának büntetését.
\par 2 A te atyádfiait, Lévinek törzsét, a te atyádnak nemzetségét is vedd magad mellé, melletted legyenek, és néked szolgáljanak; te pedig és a te fiaid veled, szolgáljatok a bizonyság sátora elõtt.
\par 3 És ügyeljenek a te ügyedre és az egész sátornak ügyére; de a szenthajlék edényeihez és az oltárhoz ne járuljanak, hogy meg ne haljanak mind õk, mind ti.
\par 4 És melletted legyenek, és ügyeljenek a gyülekezet sátorának ügyére, a sátornak minden szolgálatja szerint; de idegen ne járuljon ti hozzátok.
\par 5 Ügyeljetek azért a szenthajléknak ügyére, és az oltárnak ügyére, hogy ne legyen ezután harag Izráel fiai ellen.
\par 6 Mert íme én választottam a ti atyátokfiait, a lévitákat Izráel fiai közül, mint az Úrnak adottakat, néktek, ajándékul, hogy szolgáljanak a gyülekezet sátorának szolgálatában.
\par 7 Te pedig és a te fiaid te veled, ügyeljetek a ti papságotokra mindenben, a mik az oltárhoz tartoznak, és a függönyön belõl vannak, hogy azokban szolgáljatok; a ti papságotoknak tisztét adtam néktek ajándékul, azért az idegen, a ki oda járul, haljon meg.
\par 8 Szóla azután az Úr Áronnak: Ímé én néked adtam az én felemelt áldozataimra való ügyelést is, valamit Izráel fiai nékem szentelnek, néked és a te fiaidnak adtam azokat felkenetési díjul, örökkévaló rendelés szerint.
\par 9 Ez legyen tiéd a legszentségesebbekbõl, a melyek tûzzel meg nem égettetnek: Minden õ áldozatjok, akár ételáldozatjok, akár bûnért való áldozatjok, akár pedig vétekért való áldozatjok, a miket nékem adnak, mint legszentségesebbek, tiéid legyenek és a te fiaidé.
\par 10 A legszentségesebb helyen egyed meg azt, csak a férfiak egyék azt; szentség legyen tenéked.
\par 11 Ez is tiéd legyen, mint felemelt áldozat az õ ajándékukból, Izráel fiainak minden meglóbált áldozataival együtt; néked adtam azokat és a te fiaidnak, és a te leányaidnak, a kik veled vannak, örökkévaló rendelés szerint; mindenki, aki tiszta a te házadban, eheti azt:
\par 12 Az olajnak minden kövérjét, és minden kövérjét a mustnak és gabonának, azoknak zsengéit, a melyeket az Úrnak adnak, tenéked adtam.
\par 13 Mindennek elsõ gyümölcsei az õ földjökön, a melyeket az Úrnak visznek, tiéid legyenek; mindenki, a ki tiszta a te házadban, egye azokat.
\par 14 Minden, a mi teljesen Istennek szenteltetik Izráelben, tiéd legyen.
\par 15 Minden, ami az élõ állatok közül az anyaméhet megnyitja, a melyet az Úrnak visznek, mind emberekbõl s mind barmokból a tiéd legyen; csakhogy az embernek elsõszülöttét váltasd meg, és a tisztátalan állatnak elsõszülöttét is megváltassad.
\par 16 A melyek pedig váltságosok, egy hónapostól kezdve váltasd meg a te becslésed szerint, öt ezüst sikluson a szent siklus szerint:  húsz géra az.
\par 17 De a tehénnek elsõ fajzását, vagy a juhnak elsõ fajzását, vagy a kecskének elsõ fajzását meg ne váltasd; mert szentelni valók azok; a véröket hintsd az oltárra, és azoknak kövérit füstölögtesd el kedves illatú tûzáldozatul az Úrnak.
\par 18 Azoknak a húsa pedig tiéd legyen, a miképen a meglóbált szegy, és a mint a jobb lapoczka is tiéd lesz.
\par 19 Minden felemelt áldozatot a szent dolgokból, a melyeket az Úrnak áldoznak Izráel fiai, néked adtam örökkévaló rendelés szerint, és a te fiaidnak, és a te leányaidnak, a kik veled vannak; sónak szövetsége ez, örökkévaló az Úr elõtt, néked és a te magodnak veled.
\par 20 Monda pedig az Úr Áronnak: Az õ földjökbõl örökséged nem lesz, sem osztályrészed nem lesz néked õ közöttök: Én vagyok a te osztályrészed és a te örökséged Izráel fiai között.
\par 21 De ímé a Lévi fiainak örökségül adtam minden tizedet Izráelben; az õ szolgálatjokért való osztályrész ez a melylyel teljesítik õk a gyülekezet sátorának szolgálatát.
\par 22 És ne járuljanak ezután Izráel fiai a gyülekezetnek sátorához, hogy ne vétkezzenek, és meg ne haljanak.
\par 23 De a léviták teljesítsék a gyülekezet sátorának szolgálatát, és õk viseljék az õ bûnüket; örökkévaló rendelés legyen a ti nemzetségeiteknél, hogy Izráel fiai között ne birjanak örökséget.
\par 24 Mivelhogy Izráel fiainak tizedét, a mit felemelt áldozatul visznek fel az Úrnak, adtam a lévitáknak örökségül: azért végeztem õ felõlök, hogy Izráel fiai között ne birjanak örökséget.
\par 25 És szóla az Úr Mózesnek, mondván:
\par 26 A lévitáknak pedig szólj, és mondd meg nékik: Mikor beszeditek Izráel fiaitól a tizedet, a melyet örökségtekül adtam néktek azoktól, akkor áldozzatok abból felemelt áldozatot az Úrnak; a tizedbõl tizedet.
\par 27 És ez a ti felemelt áldozatotok olyanul tulajdoníttatik néktek, mint a szérûrõl való gabona és mint a sajtóból kiömlõ bor.
\par 28 Így vigyetek ti is felemelt áldozatot az Úrnak minden ti tizedetekbõl, a melyet beszedtek Izráel fiaitól, és adjátok abból az Úrnak felemelt áldozatot Áronnak, a papnak.
\par 29 Minden ti ajándékotokból vigyétek a felemelt áldozatokat az Úrnak, mindenbõl a kövérjét, a mi abból szentelni való.
\par 30 Mondd meg azt is nékik: Mikor a kövérjét áldozzátok abból, olyanul tulajdoníttatik az a lévitáknak, mintha a szérûrõl és sajtóból adnák.
\par 31 Megehetitek pedig azt minden helyen, ti és a ti házatok népe; mert jutalmatok ez néktek a gyülekezet sátorában való szolgálatokért.
\par 32 És nem lesztek bûnösök a miatt, ha abból a kövérjét áldozzátok: és Izráel fiainak szent dolgait sem fertõztetitek meg, és meg sem haltok.

\chapter{19}

\par 1 Szóla azután az Úr Mózesnek és Áronnak, mondván:
\par 2 Ez a törvény rendelése, a melyet parancsolt az Úr, mondván: Szólj Izráel fiainak, hogy hozzanak hozzád egy veres tehenet, épet, a melyben ne legyen hiba, a melynek nyakán iga nem volt.
\par 3 És adjátok azt Eleázárnak, a papnak, és õ vitesse ki azt a táboron kivül, és öljék meg azt õ elõtte.
\par 4 És vegyen Eleázár, a pap annak vérébõl az õ újjával, és hintsen a gyülekezet sátorának eleje felé annak vérébõl hétszer.
\par 5 Azután égessék meg azt a tehenet az õ szemei elõtt; annak bõrét, húsát és vérét a ganéjával együtt égessék meg.
\par 6 Akkor vegyen a pap czédrusfát, izsópot és karmazsint, és vesse a tehénnek égõ részei közé.
\par 7 És mossa meg a pap az õ ruháit, az õ testét is mossa le vízzel, és azután menjen be a táborba, és tisztátalan legyen a pap estvéig.
\par 8 Az is, a ki megégeti azt, mossa meg az õ ruháit vízzel, és az õ testét is mossa le vízzel, és tisztátalan legyen estvéig.
\par 9 Valamely tiszta ember pedig szedje fel annak a tehénnek hamvát, és helyezze el azt a táboron kivûl tiszta helyre, hogy legyen az Izráel fiai gyülekezetének szolgálatára a tisztulásnak vizéhez; bûnért való áldozat ez.
\par 10 És az, a ki felszedi a tehénnek hamvát, mossa meg az õ ruháit, és tisztátalan legyen estvéig; és legyen ez Izráel fiainak és a köztök tartózkodó jövevénynek örök rendelésül.
\par 11 A ki illeti akármely embernek a holttestét, és tisztátalanná lesz hét napig:
\par 12 Az olyan tisztítsa meg magát azzal a vízzel harmadnapon és hetednapon, és tiszta lesz; ha pedig nem tisztítja meg magát harmadnapon és hetednapon, akkor nem lesz tiszta.
\par 13 Valaki holtat illet, bármely embert a ki megholt, és meg nem tisztítja magát, az megfertézteti az Úrnak hajlékát; és irtassék ki az a lélek Izráelbõl mivelhogy tisztulásnak vize nem hintetett õ reá, tisztátalan lesz; még rajta van az õ tisztátalansága.
\par 14 Ez legyen a törvény, mikor valaki sátorban hal meg. Mindaz, a ki bemegy a sátorba, és mindaz, a ki ott van a sátorban, tisztátalan legyen hét napig.
\par 15 Minden nyitott edény is, a melyen nincs lezárható fedél, tisztátalan.
\par 16 És mindaz, a ki illet a mezõn fegyverrel megöletettet, vagy megholtat, vagy emberi csontot, vagy sírt, tisztátalan legyen hét napig.
\par 17 És vegyenek a tisztátalanért a bûnért való megégetett áldozatnak hamvából, és töltsenek arra élõ vizet edénybe.
\par 18 Valamely tiszta ember pedig vegyen izsópot, és mártsa azt vízbe, és hintse meg a sátort és minden edényt, és minden embert, a kik ott lesznek; és azt is, a ki a csontot, vagy a megöltet, vagy a megholtat, vagy a koporsót illette.
\par 19 Hintse pedig meg a tiszta a tisztátalant harmadnapon és hetednapon, és tisztítsa meg õt hetednapon, azután mossa meg az õ ruháit, mossa le magát is vízzel, és tiszta lesz estve.
\par 20 Ha pedig valaki tisztátalanná lesz, és nem tisztítja meg magát, az a lélek irtassék ki a község közûl; mivelhogy az Úrnak szenthelyét megfertéztette, a tisztulásnak vize nem hintetett õ reá, tisztátalan az.
\par 21 Ez legyen õ nálok örök rendelésül: mind az, a ki hinti a tisztulásnak vizét, tisztátalan legyen estvéig.
\par 22 És valamit illet a tisztátalan, tisztátalan legyen az; és az a lélek is, a ki illeti azt, tisztátalan legyen estvéig.

\chapter{20}

\par 1 És eljutának Izráel fiai, az egész gyülekezet Czin pusztájába az elsõ hónapban és megtelepedék a nép Kádesben, és meghala ott Miriám, és eltemetteték ott.
\par 2 De nem vala vize a gyülekezetnek, összegyûlének azért Mózes és Áron ellen.
\par 3 És feddõzék a nép Mózessel, és szólának mondván: Vajha holtunk volna meg, mikor megholtak a mi atyánkfiai az Úr elõtt!
\par 4 És miért hoztátok az Úrnak gyülekezetét e pusztába, hogy meghaljunk itt mi, és a mi barmaink?
\par 5 És miért hoztatok fel minket Égyiptomból, hogy e rossz helyre hozzatok minket, hol nincs vetés, sem füge, sem szõlõ, sem gránátalma, és inni való víz sincsen!
\par 6 Elmenének azért Mózes és Áron a gyülekezetnek színe elõl, a gyülekezet sátorának nyílása elé, és arczukra borulának; és megjelenék nékik az Úrnak dicsõsége.
\par 7 És szóla az Úr Mózesnek, mondván:
\par 8 Vedd ezt a vesszõt, és gyûjtsd össze a gyülekezetet te, és Áron, a te atyádfia, és szóljatok ím e kõsziklának az õ szemeik elõtt, hogy adjon vizet; és fakaszsz vizet nékik e kõsziklából, és adj inni a gyülekezetnek és az õ barmaiknak.
\par 9 Vevé azért Mózes azt a vesszõt az Úrnak színe elõl a mint parancsolta vala néki.
\par 10 És összegyûjték Mózes és Áron a gyülekezetet a kõszikla elé, és monda nékik: Halljátok meg most, ti lázadók! Avagy e  kõsziklából fakasszunk-é néktek vizet?
\par 11 És felemelé Mózes az õ kezét, és megüté a kõsziklát az õ vesszejével két ízben; és sok víz ömle ki, és ivék a gyülekezet és az õ barmai.
\par 12 És monda az Úr Mózesnek és Áronnak: Mivelhogy nem hittetek nékem, hogy megdicsõítettetek volna engem Izráel fiainak szemei elõtt: azért  nem viszitek be e községet a földre, a melyet adtam nékik.
\par 13 Ezek a versengésnek vizei, a melyekért feddõztek Izráel fiai az Úrral; és megdicsõítette magát õ bennök.
\par 14 És külde Mózes követeket Kádesbõl Edom királyához, kik így szólának: Ezt mondja a te atyádfia az Izráel: Te tudod mindazt a nyomorúságot, a mely mi rajtunk esett:
\par 15 Hogy a mi atyáink alámentek Égyiptomba, és sok ideig laktunk Égyiptomban, és hogy nyomorgatának minket az égyiptombeliek, és a mi atyáinkat.
\par 16 És kiáltottunk az Úrhoz, és meghallgatta a mi szónkat, és  angyalt külde és kihozott minket Égyiptomból; és ímé Kádesben vagyunk a te határodnak végvárosában.
\par 17 Hadd mehessünk át a te földeden! Nem megyünk át sem mezõn, sem szõlõn, és kútvizet sem iszunk; az országúton megyünk, és nem térünk sem jobbra, sem balra, míg általmegyünk a te határodon.
\par 18 Felele pedig Edom: Nem mehetsz át az én földemen, hogy fegyverrel ne menjek ellened!
\par 19 És mondának néki Izráel fiai: A járt úton megyünk fel, és ha a te vizedet iszszuk, én és az én barmaim, megadom annak az árát. Nincs egyéb szándékom csak hogy gyalog mehessek át.
\par 20 Az pedig monda: Nem mégy át. És kiméne õ ellene Edom sok néppel és nagy erõvel.
\par 21 Mivel nem akará Edom megengedni Izráelnek, hogy általmenjen az õ országán; azért eltére Izráel õ tõle.
\par 22 Majd elindulának Kádesbõl, és jutának Izráel fiai, az egész gyülekezet a Hór hegyére.
\par 23 És szóla az Úr Mózesnek és Áronnak, a Hór hegyénél, Edom földének határán, mondván:
\par 24 Áron az õ népeihez takaríttatik: mert nem megy be a földre, a melyet Izráel fiainak adtam, mivelhogy ellenszegültetek az én szómnak a versengésnek vizénél.
\par 25 Vedd Áront és Eleázárt, az õ fiát, és vezesd fel õket a Hór hegyére.
\par 26 És vetkeztesd le Áront az õ ruháiból, és öltöztesd fel azokba Eleázárt, az õ fiát: mert Áron az õ népéhez takaríttatik, és meghal ott.
\par 27 És úgy cselekedék Mózes, a mint parancsolta vala az Úr, és felmenének a Hór hegyére az egész gyülekezet láttára.
\par 28 És Mózes levetkezteté Áront az õ ruháiból, és felöltözteté azokba Eleázárt, az õ fiát. És meghala Áron ott a hegynek tetején, Mózes pedig és Eleázár leszálla a hegyrõl.
\par 29 És látá az egész gyülekezet, hogy meghalt vala Áron, és siratá Áront harmincz napig Izráelnek egész háza.

\chapter{21}

\par 1 Mikor pedig meghallotta a Kananeus, Arad királya, a ki lakozik vala dél felõl, hogy jön Izráel a kémeknek útán: megütközék Izráellel, és foglyokat ejte közülök.
\par 2 Fogadást tõn azért Izráel az Úrnak, és monda: Ha valóban kezembe adod e népet, eltörlöm az õ városait.
\par 3 És meghallgatá az Úr Izráelnek szavát, és kézbe adá a Kananeust, és eltörlé õket, és azoknak városait. És nevezé azt a helyet Hormának.
\par 4 És elindulának a Hór hegyétõl a veres tengerhez vivõ úton, hogy megkerüljék Edom földét. És a népnek lelke megkeseredék útközben.
\par 5 És szóla a nép Isten ellen és Mózes ellen: Miért hoztatok fel minket Égyiptomból, hogy meghaljunk e pusztában? Mert nincsen kenyér, víz sincsen, és e hitvány eledelt útálja a  mi lelkünk.
\par 6 Bocsáta azért az Úr a népre tüzes kigyókat, és megmardosák a népet, és sokan meghalának Izráel népébõl.
\par 7 Akkor méne a nép Mózeshez, és mondának: Vétkeztünk, mert szólottunk az Úr ellen és te ellened; imádkozzál az Úrhoz, hogy vigye el rólunk a kígyókat. És imádkozék Mózes a népért.
\par 8 És monda az Úr Mózesnek: Csinálj magadnak tüzes kígyót, és tûzd fel azt póznára: és ha valaki megmarattatik, és feltekint arra, életben maradjon.
\par 9 Csinála azért Mózes rézkígyót, és feltûzé azt póznára. És lõn, hogy ha a kígyó valakit megmar vala, és az feltekinte a rézkígyóra, életben marada.
\par 10 Azután elindulának Izráel fiai, és tábort ütének Obóthban.
\par 11 Obóthból is elindulának, és tábort ütének Hije-Abarimban, abban a pusztában, a mely Moáb elõtt vala napkelet felõl.
\par 12 Onnét elindulának, és tábort ütének Zéred völgyében.
\par 13 Onnét elindulának, és tábort ütének az Arnon vizén túl, a mely van a pusztában, és kijõ az Emoreus határából. Mert az Arnon Moábnak határa Moáb között és Emoreus között.
\par 14 Azért van szó az Úr hadainak könyvében: Vahébrõl Szúfában, és a patakokról Arnonnál,
\par 15 És a patakok folyásáról, a mely Ar város felé hajol és Moáb határára dûl.
\par 16 És onnét Béérbe menének. Ez az a kút, a melynél mondotta vala az Úr Mózesnek: Gyûjtsd össze a népet, és adok nékik vizet.
\par 17 Akkor éneklé az Izráel ez éneket: Jõjj fel óh kút! énekeljetek néki!
\par 18 Kút, a melyet fejedelmek ástak; a nép elõkelõi vájtak, kormánypálczával, vezérbotjaikkal. És a pusztából Mathanába menének.
\par 19 És Mathanából Nahaliélbe, és Nahaliélbõl Bámóthba.
\par 20 Bámóthból pedig abba a völgybe, a mely Moáb mezején van, onnan a Piszga tetejére, a mely a sivatagra néz.
\par 21 És külde Izráel követeket Szíhonhoz, az Emoreusok királyához, mondván:
\par 22 Hadd mehessek át a te földeden! Nem hajlunk mezõre, sem szõlõre, kútvizet sem iszunk; az országúton megyünk, míg átmegyünk a te határodon.
\par 23 De nem engedé Szíhon Izráelnek, hogy átmenjen az õ határán, sõt egybegyûjté Szihon minden népét, és kiméne Izráel ellen a pusztába, és jöve Jaháczba; és megütközék Izráellel.
\par 24 És megveré õt Izráel fegyvernek élivel, és elfoglalá annak földét Arnontól Jabbókig, az Ammon fiaiig; mert erõs vala az Ammon fiainak határa.
\par 25 És elfoglalá Izráel mind e városokat, és megtelepedék Izráel az Emoreusok minden városában, Hesbonban és annak minden városában;
\par 26 Mert Hesbon Szíhonnak az Emoreusok királyának városa vala, ki is hadakozott vala Moábnak elõbbi királyával, és elvett vala minden földet annak kezébõl Arnonig.
\par 27 Azért mondják a példabeszédmondók: Jõjjetek Hesbonba! Építtessék és erõsíttessék Szíhon városa!
\par 28 Mert tûz jött ki Hesbonból, láng Szíhon városából; megemésztette Art, Moábnak városát, Arnon magaslatainak urait.
\par 29 Jaj néked Moáb! Elvesztél Kámosnak népe! Futásra adta õ fiait, leányait fogságra Szíhonnak, az emoreus királynak.
\par 30 De mi lövöldöztük õket, elveszett Hesbon Dibonig, és elpusztítottuk Nofáig, a mely Medebáig ér.
\par 31 Megtelepedék azért Izráel az Emoreusok földén.
\par 32 És elkülde Mózes, hogy megkémleljék Jázert, és bevevék annak városait, és kiûzé az Emoreust, a ki ott vala.
\par 33 Majd megfordulának, és felmenének a Básánba vivõ úton. És kijöve Og, Básán királya õ ellenök, õ és egész népe, hogy megütközzenek Hedreiben.
\par 34 Akkor monda az Úr Mózesnek: Ne félj tõle, mert a te kezedbe adtam õt, és egész népét, és az õ földét. És úgy cselekedjél vele, a miképen cselekedtél Szíhonnal az Emoreusok királyával, a ki lakik vala Hesbonban.
\par 35 Megverék azért õt és az õ fiait és egész népét annyira, hogy egy sem marada belõle; és elfoglalák az õ földét.

\chapter{22}

\par 1 És tovább menének Izráel fiai, és tábort ütének Moáb mezõségen a Jordánon túl, Jérikhó ellenében.
\par 2 És mikor látta Bálák, a Czippór fia mind azokat, a melyeket cselekedett vala Izráel az Emoreussal:
\par 3 Igen megrémüle Moáb a néptõl, mivelhogy sok vala az, és búsula Moáb Izráel fiai miatt.
\par 4 Monda azért Moáb Midián véneinek: Most elnyalja e sokaság minden mi környékünket, a miképen elnyalja az ökör a mezõnek pázsitját. (Bálák pedig, Czippórnak fia Moáb királya vala abban az idõben.)
\par 5 Külde azért követeket Bálámhoz, a Beór fiához Péthorba, a mely vala a folyóvíz mellett, az õ népe fiainak földére, hogy hívják õt, mondván: Ímé nép jött ki Égyiptomból, és ímé ellepte e földnek színét, és megtelepszik én ellenemben.
\par 6 Most azért kérlek jöjj el, átkozd meg érettem e népet, mert erõsebb nálamnál; talán erõt vehetek rajta, megverjük õt, és kiûzhetem õt e földbõl; mert jól tudom, hogy a kit megáldasz, meg lesz áldva, és a kit megátkozol, átkozott lesz.
\par 7 Elmenének azért Moábnak vénei és Midiánnak vénei, és a jövendõmondásnak jutalma kezeikben vala, és jutának Bálámhoz, és megmondák néki Bálák izenetét.
\par 8 Õ pedig monda nékik: Háljatok itt ez éjjel, és feleletet adok néktek, a miképen szól nékem az Úr. És Moábnak fejedelmei ott maradának Bálámnál.
\par 9 Eljöve pedig az Isten Bálámhoz, és monda: Kicsodák ezek a férfiak te nálad?
\par 10 És monda Bálám az Istennek: Bálák, Czippór fia, Moáb királya, küldött én hozzám, ezt mondván:
\par 11 Ímé e nép, a mely kijött Égyiptomból, ellepte a földnek színét; most azért jöjj el, átkozd meg azt érettem, talán megharczolhatok vele, és kiûzhetem õt.
\par 12 És monda Isten Bálámnak: Ne menj el õ velök, ne átkozd meg azt a népet, mert áldott az.
\par 13 Felkele azért Bálám reggel, és monda a Bálák fejedelmeinek: Menjetek el a ti földetekre; mert nem akarja az Úr megengedni nékem, hogy elmenjek veletek.
\par 14 És felkelének Moáb fejedelmei, és jutának Bálákhoz, és mondának: Nem akart Bálám eljönni velünk.
\par 15 Elkülde azért ismét Bálák több fejedelmet, amazoknál elõkelõbbeket.
\par 16 És eljutának Bálámhoz, és mondának néki: "Ezt mondja Bálák, Czippórnak fia: Kérlek, ne vonakodjál eljönni hozzám!"
\par 17 Mert igen-igen megtisztellek téged, és akármit mondasz nékem, megcselekszem. Jöjj el azért kérlek, és átkozd meg értem e népet!
\par 18 Bálám pedig felele és monda a Bálák szolgáinak: Ha Bálák az õ házát aranynyal és ezüsttel tele adná is nékem, nem hághatom át az Úrnak, az én Istenemnek szavát, hogy valamit míveljek; kicsinyt vagy nagyot.
\par 19 Most mindazonáltal maradjatok itt kérlek ti is ez éjjel, hadd tudjam meg, mit szól ismét az Úr nékem?
\par 20 És eljöve Isten Bálámhoz éjjel, és monda néki: Ha azért jöttek e férfiak, hogy elhívjanak téged: kelj fel, menj el velök; de mindazáltal azt cselekedjed, a mit mondok majd néked.
\par 21 Felkele azért Bálám reggel, és megnyergelé az õ szamarát, és elméne a Moáb fejedelmeivel.
\par 22 De megharaguvék Isten, hogy elmegy vala õ. És megálla az Úrnak angyala az útban, hogy ellenkezzék vele; õ pedig üget vala az õ szamarán, és két szolgája vala vele.
\par 23 És meglátá a szamár az Úrnak angyalát, a mint áll vala az úton, és mezítelen fegyvere a kezében; letére azért a szamár az útról, és méne a mezõre; Bálám pedig veré az õ szamarát, hogy visszatérítse az útra.
\par 24 Azután megálla az Úrnak angyala a szõlõk ösvényén, holott innen is garád, onnan is garád vala.
\par 25 A mint meglátá a szamár az Úrnak angyalát, a falhoz szorula, és a Bálám lábát is oda szorítá a falhoz; ezért ismét megveré azt.
\par 26 Az Úr angyala pedig ismét tovább méne, és megálla szoros helyen, hol nem volt út a kitérésre, sem jobbra, sem balra.
\par 27 A mint meglátá a szamár az Úrnak angyalát, lefeküvék Bálám alatt, azért megharaguvék Bálám, és megveré a szamarat a bottal.
\par 28 És megnyitá az Úr a szamárnak száját, és monda a szamár Bálámnak: Mit vétettem néked, hogy immár háromszor vertél meg engem?
\par 29 Bálám pedig monda a szamárnak: Mert megcsúfoltál engem! Vajha volna fegyver a kezemben, nyilván megölnélek most téged.
\par 30 És monda a szamár Bálámnak: Avagy nem te szamarad vagyok-é, a melyen járni szoktál, a mióta megvagy, mind e napig? Avagy szoktam volt-é veled e képen cselekedni? Az pedig felele: Nem.
\par 31 És megnyitá az Úr a Bálám szemeit, és látá az Úr angyalát, a mint áll vala az útban, és mezítelen fegyverét az õ kezében; akkor meghajtá magát és arczra borula.
\par 32 Az Úrnak angyala pedig monda néki: Miért verted meg a te szamaradat immár három ízben? Ímé én jöttem ki, hogy ellenkezzem veled, mert veszedelmes ez az út én elõttem.
\par 33 És meglátott engem a szamár, és kitért én elõttem immár három ízben; ha ki nem tért volna elõlem, most meg is öltelek volna téged, õt pedig életben hagytam volna.
\par 34 Monda azért Bálám az Úr angyalának: Vétkeztem, mert nem tudtam, hogy te állasz elõttem az útban. Most azért, ha nem tetszik ez néked, visszatérek.
\par 35 Az Úrnak angyala pedig monda Bálámnak: Menj el e férfiakkal; mindazáltal a mit én mondok majd néked, azt mondjad. Elméne azért Bálám a Bálák fejedelmeivel.
\par 36 Mikor pedig meghallá Bálák, hogy jön Bálám, kiméne elébe Moábnak egyik városába, a mely az Arnon vidékén, a határ szélén vala.
\par 37 És monda Bálák Bálámnak: Avagy nem küldözgettem-é hozzád, hogy hívjanak téged? Miért nem jösz vala én hozzám? Avagy valóban nem tisztelhetnélek meg téged?
\par 38 Bálám pedig monda Báláknak: Ímé eljöttem hozzád. Most pedig szólhatok-é magamtól valamit? A mi mondani valót Isten ád az én számba, azt mondom.
\par 39 És elméne Bálám Bálákkal, és eljutának Kirjat-Husótba.
\par 40 És vágata Bálák ökröket és juhokat, és küldé Bálámnak és a fejedelmeknek, a kik vele valának.
\par 41 Reggel pedig magához vevé Bálák Bálámot és felvivé õt a Baál magas hegyére, hogy meglássa onnét a népnek valami részét.

\chapter{23}

\par 1 És monda Bálám Báláknak: Építs itt nékem hét oltárt, és készíts el ide nékem hét tulkot és hét kost.
\par 2 És úgy cselekedék Bálák a miképen mondotta vala Bálám. És áldozék Bálák és Bálám mindenik oltáron egy-egy tulkot és egy-egy kost.
\par 3 És monda Bálám Báláknak: Állj meg a te égõáldozatod mellett, én pedig elmegyek; talán elõmbe jõ az Úr nékem, és a mit mutat majd nékem, megjelentem néked. Elméne azért egy kopasz oromra.
\par 4 És elébe méne Isten Bálámnak, és monda néki Bálám: A hét oltárt elrendeztem, és áldoztam mindenik oltáron egy-egy tulkot és egy-egy kost.
\par 5 Az Úr pedig ígét ada Bálámnak szájába, és monda néki: Térj vissza Bálákhoz, és így szólj.
\par 6 És visszatére õ hozzá, és ímé áll vala az õ égõáldozata mellett; õ maga és Moábnak minden fejedelme.
\par 7 És elkezdé az õ példázó beszédét, és monda: Siriából hozatott engem Bálák, Moábnak királya kelet hegyeirõl, mondván: Jöjj, átkozd meg nékem Jákóbot, és jöjj, szidalmazd meg Izráelt;
\par 8 Mit átkozzam azt, a kit Isten nem átkoz, és mit szidalmazzam azt, a kit az Úr nem szidalmaz?
\par 9 Mert sziklák tetejérõl nézem õt, és halmokról tekintem õt; ímé oly nép, a mely maga fog lakni, és nem számláltatik a nemzetek közé.
\par 10 Ki számlálhatja meg a Jákób porát, és Izráel negyedrészének számát? Haljon meg az én lelkem az igazak halálával, és legyen az én utolsó napom, mint az övé!
\par 11 És monda Bálák Bálámnak: Mit cselekeszel én velem? Hogy megátkozzad ellenségeimet, azért hoztalak téged, és ímé igen megáldád õket.
\par 12 Ez pedig felele és monda: Avagy nem arra kell-é vigyáznom, hogy azt szóljam, a mit az Úr adott az én számba?
\par 13 Monda azután néki Bálák: Kérlek, jöjj velem más helyre, honnét meglássad õt, de csak valamely részét látod annak, és õt mindenestõl nem látod, és átkozd meg õt onnét nékem.
\par 14 És vivé õt az õrállók helyére, a Piszga tetejére, és építe hét oltárt, és áldozék egy-egy tulkot és egy-egy kost mindenik oltáron.
\par 15 És monda Báláknak: Állj meg itt a te égõáldozatod mellett, én pedig elébe megyek amoda.
\par 16 Elébe méne azért az Úr Bálámnak, és ígét ada az õ szájába, és monda: Térj vissza Bálákhoz, és így szólj.
\par 17 Méne azért õ hozzá, és ímé õ áll vala az õ égõáldozata mellett, és Moáb fejedelmei is õ vele. És monda néki Bálák: Mit szóla az Úr?
\par 18 Akkor elkezdé az õ példázó beszédét, és monda: Kelj fel Bálák, és halljad; figyelj rám Czippórnak fia!
\par 19 Nem ember az Isten, hogy hazudjék és nem embernek fia, hogy megváltozzék. Mond-é õ valamit, hogy meg ne tenné? Igér-é valamit, hogy azt ne teljesítené?
\par 20 Ímé parancsolatot vettem, hogy áldjak; ha õ áld, én azt meg nem fordíthatom.
\par 21 Nem vett észre Jákóbban hamisságot, és nem látott gonoszságot Izráelben. Az Úr, az õ Istene van õ vele; és királynak szóló rivalgás hangzik õ benne.
\par 22 Isten hozta ki õket Égyiptomból, az õ ereje mind a  vad bivalyé.
\par 23 Mert nem fog varázslás Jákóbon, sem jövendõmondás Izráelen. Idején adatik tudtára Jákóbnak és Izráelnek: mit mívelt Isten!
\par 24 Ímé e nép felkél mint nõstény oroszlán, és feltámad mint hím oroszlán; nem nyugszik, míg prédát nem eszik, és elejtettek vérét nem iszsza.
\par 25 Akkor monda Bálák Bálámnak: Se ne átkozzad, se ne áldjad õt.
\par 26 Felele pedig Bálám, és monda Báláknak: Avagy nem szólottam volt-é néked, mondván: Valamit mond nékem az Úr, azt mívelem?
\par 27 És monda Bálák Bálámnak: Jöjj, kérlek, elviszlek téged más helyre: talán tetszeni fog az Istennek, hogy megátkozzad onnét e népet én érettem.
\par 28 Elvivé azért Bálák Bálámot a Peór tetejére, a mely a puszta felé néz.
\par 29 És monda Bálám Báláknak: Építtess itt nékem hét oltárt, és készíts el ide nékem hét tulkot és hét kost.
\par 30 Úgy cselekedék azért Bálák, a mint mondotta volt Bálám, és áldozék minden oltáron egy-egy tulkot és egy-egy kost.

\chapter{24}

\par 1 Mikor pedig látta Bálám, hogy tetszik az Úrnak, hogy megáldja Izráelt, nem indula, mint az elõtt, varázslatok után, hanem fordítá az õ orczáját a puszta felé.
\par 2 És mikor felemelte Bálám az õ szemeit, látá Izráelt, a mint letelepedett az õ nemzetségei szerint; és Istennek lelke vala õ rajta.
\par 3 Akkor elkezdé az õ példázó beszédét és monda: Bálámnak, Beór fiának szózata, a megnyílt szemû embernek szózata.
\par 4 Annak szózata, a ki hallja Istennek beszédét, a ki látja a Mindenhatónak látását, leborulva, de nyitott szemekkel:
\par 5 Mily szépek a te sátoraid óh Jákób! a te hajlékaid óh Izráel!
\par 6 Mint kiterjesztett völgyek, mint kertek a folyóvíz mellett, mint az Úr plántálta áloék, mint czédrusfák a vizek mellett!
\par 7 Víz ömledez az õ vedreibõl, vetését bõ víz öntözi; királya nagyobb Agágnál, és felmagasztaltatik az õ országa.
\par 8 Isten hozta ki Égyiptomból, az õ ereje mint a vad bivalyé: megemészti a pogányokat, az õ ellenségeit; csontjaikat megtöri, és nyilaival által veri.
\par 9 Lehever, nyugszik, mint hím oroszlán, és mint nõstény oroszlán; ki serkenti fel õt? A ki áld téged, áldott lészen, és ki átkoz téged, átkozott lészen.
\par 10 És felgerjede Báláknak haragja Bálám ellen, és egybeüté kezeit, és monda Bálák Bálámnak: Azért hívtalak téged, hogy átkozd meg ellenségeimet; és ímé igen megáldottad immár három ízben.
\par 11 Most azért fuss a te helyedre. Mondottam vala, hogy igen megtisztellek  téged; de ímé megfosztott téged az Úr a tisztességtõl.
\par 12 És monda Bálám Báláknak: A te követeidnek is, a kiket küldöttél volt hozzám, nem így szólottam-é, mondván:
\par 13 Ha Bálák az õ házát ezüsttel és aranynyal tele adná is nékem, az Úrnak beszédét által nem hághatom, hogy magamtól jót, vagy rosszat cselekedjem. A mit az  Úr szól, azt szólom.
\par 14 Most pedig én elmegyek ímé az én népemhez; jõjj, hadd jelentsem ki néked, mit fog cselekedni e nép a te népeddel, a következõ idõben.
\par 15 És elkezdé az õ példázó beszédét, és monda: Bálámnak, Beór fiának szózata, a megnyilt szemû ember szózata,
\par 16 Annak szózata, a ki hallja Istennek beszédét, és a ki tudja a Magasságosnak tudományát, és a ki látja a Mindenhatónak látását, leborulva, de nyitott szemekkel.
\par 17 Látom õt, de nem most; nézem õt, de nem közel. Csillag származik Jákóbból, és királyi pálcza támad Izráelbõl; és általveri Moábnak oldalait, és összetöri Sethnek minden fiait.
\par 18 És Edom más birtoka lesz, Szeir az õ ellensége is másnak birtoka lesz; de hatalmasan cselekszik Izráel.
\par 19 És uralkodik a Jákóbtól való, és elveszti a városból a megmaradtat.
\par 20 És mikor látja vala Amáleket, elkezdé az õ példázó beszédét, és monda: Amálek elsõ a nemzetek között, de végezetre mindenestõl elvész.
\par 21 És mikor látja vala a Keneust, elkezdé példázó beszédét, és monda: Erõs a te lakhelyed, és sziklára raktad fészkedet;
\par 22 Mégis el fog pusztulni Kain; a míg Assur téged fogva viszen.
\par 23 Újra kezdé az õ példázó beszédét, és monda: Óh, ki fog élni még, a mikor véghez viszi ezt az Isten?
\par 24 És Kittim partjairól hajók jõnek, és nyomorgatják Assúrt, nyomorgatják Ébert is, és ez is mindenestõl elvész.
\par 25 Felkele azért Bálám, és elméne, hogy visszatérjen az õ helyére. És Bálák is elméne az õ útján.

\chapter{25}

\par 1 Mikor pedig Sittimben lakozik vala Izráel, kezde a nép paráználkodni Moáb leányaival.
\par 2 Mert hívogaták a népet az õ isteneik áldozataira; és evék a nép, és imádá azoknak isteneit.
\par 3 És odaszegõdék Izráel Bál-Peórhoz: az Úr haragja pedig felgerjede Izráel ellen.
\par 4 És monda az Úr Mózesnek: Vedd elõ e népnek minden fõemberét, és akasztasd fel õket az Úrnak fényes nappal; hogy elforduljon az Úr haragjának gerjedezése Izráeltõl.
\par 5 Monda azért Mózes Izráel bíráinak: Kiki ölje meg az õ embereit, a kik odaszegõdtek Bál-Peórhoz.
\par 6 És ímé eljöve valaki Izráel fiai közül, és hoza az õ atyjafiai felé egy midiánbeli asszonyt, Mózes szeme láttára, és Izráel fiai egész gyülekezetének láttára; õk pedig sírnak vala a gyülekezet sátorának nyilásánál.
\par 7 És mikor látta vala Fineás, Eleázár fia, Áron papnak unokája, felkele a gyülekezet közül, és dárdáját vevé kezébe.
\par 8 És beméne az izráelita férfi után a sátorba, és általdöfé mindkettõjöket, mind az izráelita férfit, mind az asszonyt hason. És megszünék a csapás Izráel fiai között.
\par 9 De meghaltak vala a csapás miatt huszonnégy ezeren.
\par 10 Akkor szóla az Úr Mózesnek, mondván:
\par 11 Fineás, Eleázár fia, Áron pap unokája, elfordította az én haragomat Izráel fiaitól, mivelhogy az én bosszúmat megállotta õ közöttök; ezért nem pusztítom ki bosszúmban Izráel fiait.
\par 12 Mondd azért: Ímé én az én szövetségemet, a békesség szövetségét adom õ néki.
\par 13 És lészen õ nála és az õ magvánál õ utána az örökkévaló papságnak szövetsége; mivelhogy bosszút állott az õ Istenéért, és engesztelést végze Izráel fiaiért.
\par 14 A megöletett izráelita férfinak neve pedig, a ki a midiánbeli asszonynyal együtt öletett meg, Zimri, a Szálu fia vala, a Simeon-nemzetség háznépének fejedelme.
\par 15 A megöletett midiánbeli asszony neve pedig Kozbi, Czúr leánya, a ki a Midiániták között az õ atyja háza nemzetségeinek fejedelme vala.
\par 16 És szóla az Úr Mózesnek, mondván:
\par 17 Támadjátok meg a Midiánitákat, és verjétek meg õket.
\par 18 Mert õk megtámadtak titeket az õ cselszövéseikkel, a melyeket Peórért és az õ hugokért Kozbiért, a midián fejedelem leányáért szõttek ellenetek, a ki megöletett a Peór miatt való csapásnak napján.

\chapter{26}

\par 1 És lõn a csapás után, szóla az Úr Mózesnek és Eleázárnak, az Áron pap fiának, mondván:
\par 2 Vegyétek számba Izráel fiainak egész gyülekezetét, húsz esztendõstõl fogva és feljebb, az õ atyáiknak háznépe szerint; mindenkit, a ki hadba mehet Izráelben.
\par 3 Szóla azért velök Mózes és Eleázár, a pap, a Moáb mezõségében a Jordán mellett, Jérikhó ellenében, mondván:
\par 4 Vegyétek számba a népet, húsz esztendõstõl fogva és feljebb, a miképen parancsolta vala az Úr Mózesnek és Izráel fiainak, a kik kijöttek volt Égyiptom földébõl.
\par 5 Rúben elsõszülötte Izráelnek. Rúben fiai ezek: Hánoktól a Hánokiták, Pallutól a Palluiták nemzetsége.
\par 6 Heczrontól a Heczroniták nemzetsége, Kármitól a Kármiták nemzetsége.
\par 7 Ezek a Rúbeniták nemzetségei. És lõn az õ számok negyvenhárom ezer, hétszáz és harmincz.
\par 8 És a Pallu fiiiai valának: Eliáb.
\par 9 Eliáb fiai pedig, Nemuél, Dáthán és Abirám. Ez a Dáthán és Abirám a gyülekezet elõljárói valának, a kik feltámadtak vala Mózes ellen és Áron ellen.
\par 10 És megnyitá a föld az õ száját, és elnyelé õket és Kórét, meghalván az a gyülekezet, mivelhogy megemészte a tûz kétszáz és ötven férfiút, a kik intõpéldául lõnek.
\par 11 Kóré fiai pedig nem halának meg.
\par 12 Simeon fiai az õ nemzetségeik szerint ezek: Nemuéltõl a Nemuéliták nemzetsége, Jámintól a Jáminiták nemzetsége, Jákintól a Jákiniták nemzetsége.
\par 13 Zerákhtól a Zerákhiták nemzetsége. Saultól a Sauliták nemzetsége.
\par 14 Ezek a Simeoniták nemzetségei: huszonkét ezer és kétszáz.
\par 15 Gád fiai az õ nemzetségeik szerint ezek: Sefontól a Sefoniták nemzetsége, Haggitól a Haggiták nemzetsége, Súnitól a Súniták nemzetsége.
\par 16 Oznitól az Ozniták nemzetsége, Éritõl az Ériták nemzetsége.
\par 17 Arodtól az Aroditák nemzetsége, Arélitõl az Aréliták nemzetsége.
\par 18 Ezek Gád fiainak nemzetségei; az õ számok szerint negyven ezer és ötszáz.
\par 19 Júda fiai: Ér és Onán. És meghala Ér és Onán a Kanaán földén.
\par 20 Júda fiai pedig az õ nemzetségeik szerint ezek valának: Séláhtól a Séláhiták nemzetsége, Pérecztõl a Pérecziták nemzetsége, Zerákhtól a Zerákhiták nemzetsége.
\par 21 Pérecz fiai valának pedig: Heczrontól a Heczroniták nemzetsége; Hámultól a Hámuliták nemzetsége.
\par 22 Ezek Júda nemzetségei az õ számok szerint: hetvenhat ezer ötszáz.
\par 23 Izsakhár fiai az õ nemzetségeik szerint ezek: Thólától a Thóláiták nemzetsége; Puvától a Puviták nemzetsége;
\par 24 Jásubtól a Jásubiták nemzetsége; Simrontól a Simroniták nemzetsége.
\par 25 Ezek Izsakhár nemzetségei az õ számok szerint: hatvannégy ezer háromszáz.
\par 26 Zebulon fiai az õ nemzetségeik szerint ezek: Szeredtõl a Szerediták nemzetsége, Élontól az Éloniták nemzetsége, Jahleéltõl a Jahleéliták nemzetsége.
\par 27 Ezek a Zebuloniták nemzetségei az õ számok szerint: hatvan ezer ötszáz.
\par 28 József fiai az õ nemzetségeik szerint ezek: Manasse és Efraim.
\par 29 Manasse fiai: Mákirtól a Mákiriták nemzetsége. Mákir nemzé Gileádot; Gileádtól a Gileáditák nemzetsége.
\par 30 Ezek Gileád fiai: Jezertõl a Jezeriták nemzetsége, Hélektõl a Hélekiták nemzetsége.
\par 31 És Aszriéltõl az Aszriéliták nemzetsége; Sekemtõl a Sekemiták nemzetsége.
\par 32 És Semidától a Semidáták nemzetsége és Héfertõl a Héferiták nemzetsége.
\par 33 Czélofhádnak pedig, a Héfer fiának nem voltak fiai, hanem leányai; és a Czélofhád leányainak nevei ezek: Makhla, Nóa, Hogla, Milkha és Thircza.
\par 34 Ezek Manasse nemzetségei, és az õ számok ötvenkét ezer és hétszáz.
\par 35 Ezek Efraim fiai az õ nemzetségeik szerint: Suthelákhtól a Suthelákhiták nemzetsége, Békertõl a Békeriták nemzetsége, Tahántól a Tahániták nemzetsége.
\par 36 Ezek pedig a Suthelákh fiai: Érántól az Érániták nemzetsége.
\par 37 Ezek Efraim fiainak nemzetségei az õ számok szerint: harminczkét ezer és ötszáz. Ezek József fiai az õ nemzetségeik szerint.
\par 38 Benjámin fiai az õ nemzetségeik szerint ezek: Belától a Belaiták nemzetsége; Asbéltõl az Asbéliták nemzetsége; Ahirámtól az Ahirámiták nemzetsége.
\par 39 Sefufámtól a Sefufámiták nemzetsége; Hufámtól a Hufámiták nemzetsége.
\par 40 Bela fiai pedig valának: Ard és Naamán: Ardtól az Arditák nemzetsége, Naamántól a Naamániták nemzetsége.
\par 41 Ezek Benjámin fiai az õ nemzetségeik szerint, és számok: negyvenöt ezer és hatszáz.
\par 42 Ezek Dán fiai az õ nemzetségeik szerint: Suhámtól a Suhámiták nemzetsége. Ezek Dán nemzetségei, az õ nemzetségeik szerint.
\par 43 A Suhamiták minden nemzetsége az õ számok szerint: hatvannégy ezer és négyszáz.
\par 44 Áser fiai az õ nemzetségeik szerint ezek: Jimnától a Jimnaiták nemzetsége; Jisvitõl a Jisviták nemzetsége; Bériától a Bérihiták nemzetsége.
\par 45 A Béria fiaitól: Khébertõl a Khéberiták nemzetsége, Malkiéltõl a Malkiéliták nemzetsége.
\par 46 Áser leányának pedig neve vala Sérah.
\par 47 Ezek Áser fiainak nemzetségei az õ számok szerint: ötvenhárom ezer és négyszáz.
\par 48 Nafthali fiai az õ nemzetségeik szerint ezek: Jakhczeéltõl a Jakhczeéliták nemzetsége, Gúnitól a Gúniták nemzetsége.
\par 49 Jéczertõl a Jéczeriták nemzetsége; Sillémtõl a Sillémiták nemzetsége.
\par 50 Ezek Nafthali nemzetségei az õ nemzetségeik szerint: az õ számok pedig negyvenöt ezer és négyszáz.
\par 51 Ezek Izráel fiainak megszámláltjai: hatszáz egyezer hétszáz harmincz.
\par 52 Szóla pedig az Úr Mózesnek, mondván:
\par 53 Ezeknek osztassék el az a föld örökségül, az õ neveiknek száma szerint.
\par 54 A nagyobb számúnak adj nagyobb örökséget, a kisebb számúnak pedig tedd kisebbé az õ örökségét; mindeniknek az õ száma szerint adattassék az õ öröksége.
\par 55 De sorssal osztassék el a föld; az õ atyjok törzseinek nevei szerint örököljenek.
\par 56 A sors szerint osztassék el az örökség, mind a sok és mind a kevés között.
\par 57 Ezek pedig Lévi megszámláltjai az õ nemzetségeik szerint: Gérsontól a Gérsoniták nemzetsége; Kéháttól a Kéhátiták nemzetsége; Méráritól a Méráriták nemzetsége.
\par 58 Ezek Lévi nemzetségei: a Libniták nemzetsége, a Hébroniták nemzetsége, a Makhliták nemzetsége, a Músiták nemzetsége, a Kórahiták nemzetsége. Kéhát pedig nemzé Amrámot.
\par 59 Amrám feleségének neve pedig Jókebed, a Lévi leánya, a ki Égyiptomban született Lévinek; és õ szülte Amrámnak Áront, Mózest, és Miriámot, az õ leánytestvéröket.
\par 60 És születének Áronnak: Nádáb és Abihu, Eleázár és Ithamár.
\par 61 És meghalának Nádáb és Abihu, mikor idegen tûzzel áldozának az Úr elõtt.
\par 62 És vala azoknak száma: huszonhárom ezer, mind férfiak, egy hónapostól fogva és feljebb; mert nem voltak beszámlálva az Izráel fiai közé, mivel nem adatott nékik örökség Izráel fiai között.
\par 63 Ezek Mózesnek és Eleázárnak, a papnak megszámláltjai, a kik megszámlálák Izráel fiait a Moáb mezõségében, Jérikhó átellenében a Jordán mellett.
\par 64 Ezek között pedig nem volt senki a Mózestõl és Áron paptól megszámláltattak közûl, mikor megszámlálták vala Izráel fiait a Sinai pusztájában.
\par 65 Mert az Úr mondotta vala nékik: Bizonynyal meghalnak a pusztában; és senki nem maradt meg azok közül, hanem csak Káleb, a Jefunné fia, és Józsué, a Nún fia.

\chapter{27}

\par 1 Elõállának pedig a Czélofhád leányai, a ki Héfer fia vala, a ki Gileád fia, a ki Mákir fia, a ki Manasse fia vala, József fiának, Manassénak nemzetségei közül. Ezek pedig az õ leányainak neveik: Makhla, Nóa, Hogla, Milkha és Thircza.
\par 2 És megállának Mózes elõtt, és Eleázár pap elõtt, és a fejedelmek elõtt, és az egész gyülekezet elõtt a gyülekezet sátorának nyilásánál, mondván:
\par 3 A mi atyánk meghalt a pusztában, de õ nem volt azoknak seregében, a kik összeseregeltek volt az Úr ellen a Kóré seregében; hanem az õ bûnéért  holt meg, és fiai nem voltak néki.
\par 4 Miért töröltetnék el a mi atyánknak neve az õ nemzetsége közül, azért, hogy nincsen õnéki fia? Adj örökséget nékünk a mi atyánknak atyjafiai között.
\par 5 Mózes pedig vivé ezeknek ügyét az Úr elé.
\par 6 És szóla az Úr Mózesnek, mondván:
\par 7 Igazat szólnak a Czélofhád leányai: Adj nékik örökségi birtokot az õ atyjoknak atyjafiai között, és szállítsd rájok az õ atyjoknak örökségét.
\par 8 Izráel fiainak pedig szólj, mondván: Mikor valaki meghal, és fia nem leend annak, akkor adjátok annak örökségét az õ leányának.
\par 9 Ha pedig nem leend néki leánya, akkor adjátok az õ örökségét az õ testvéreinek.
\par 10 Ha pedig nem leendenek néki testvérei, akkor adjátok az õ örökségét az õ atyja testvéreinek.
\par 11 Ha pedig nem leendenek az õ atyjának testvérei, akkor adjátok az õ örökségét annak, a ki legközelebbi atyjafia az õ nemzetségébõl, és bírja azt. Legyen pedig az Izráel fiainak végezett törvényök, a miképen megparancsolta az Úr Mózesnek.
\par 12 Monda azután az Úr Mózesnek: Menj fel az Abarim hegyére, és lásd meg a földet, a melyet Izráel fiainak adtam.
\par 13 Miután pedig megláttad azt, takaríttatol te is a te népedhez, a miképen oda takaríttatott Áron, a te testvéred.
\par 14 Mivelhogy nem engedétek az én beszédemnek a Czin pusztájában, a gyülekezet versengésének idején, hogy megdicsõítettetek volna engemet ama vizeknél az õ szemeik elõtt. Ezek a versengésnek vizei Kádesnél a Czin pusztájában.
\par 15 Szóla azért Mózes az Úrnak, mondván:
\par 16 Az Úr, a minden test lelkének Istene, rendeljen férfiút a gyülekezet fölé.
\par 17 A ki kimenjen õ elõttök, és a ki bemenjen õ elõttök, a ki kivigye õket, és a ki bevigye õket, hogy ne legyen az Úr gyülekezete olyan mint a juhok, a melyeknek nincsen pásztoruk.
\par 18 Az Úr pedig monda Mózesnek: Vedd melléd Józsuét a Nún fiát, a férfiút, a kiben lélek van, és tedd õ reá a te kezedet.
\par 19 És állasd õt Eleázár pap elé, és az egész gyülekezet elé, és adj néki parancsolatokat az õ szemeik elõtt.
\par 20 És a te dicsõségedet közöld õ vele, hogy hallgassa õt Izráel fiainak egész gyülekezete.
\par 21 Azután pedig álljon Eleázár pap elé, és kérdje meg õt az Urimnak ítélete felõl az Úr elõtt. Az õ szava szerint menjenek ki, és az õ szava szerint menjenek be, õ és Izráel minden fia õ vele, és az egész gyülekezet.
\par 22 Úgy cselekedék azért Mózes, a miképen parancsolta vala az Úr néki; mert vevé Józsuét, és állatá õt Eleázár pap elé, és az egész gyülekezet elé.
\par 23 És tevé az õ kezét õ reá, és ada néki parancsolatokat, a miképen szólott vala az Úr Mózes által.

\chapter{28}

\par 1 És szóla az Úr Mózesnek, mondván:
\par 2 Parancsold meg Izráel fiainak, és mondd meg nékik: Ügyeljetek, hogy az én áldozatomat, kenyeremet kedves illatú tûzáldozatul a maga idejében áldozzátok nékem.
\par 3 És mondd meg nékik: Ez a tûzáldozat, a melyet áldozzatok az Úrnak: egy esztendõs, ép bárányokat, naponként kettõt, szüntelen való egészen égõáldozatul.
\par 4 Egyik bárányt reggel készítsd el, a másik bárányt pedig estennen készítsd el.
\par 5 És egy efa lánglisztnek tizedrészét ételáldozatul, megelegyítve egy hin sajtolt olajnak negyedrészével.
\par 6 Ez a szüntelen való egészen égõáldozat, a mely Sinai hegyen szereztetett kedves illatú tûzáldozatul az Úrnak.
\par 7 Annak italáldozatja pedig egy hinnek negyedrésze egy-egy bárányhoz. A szenthelyen adj italáldozatot, jó borból az Úrnak.
\par 8 A másik bárányt készítsd el estennen, a reggeli ételáldozat és és annak italáldozatja szerint készítsd el azt, jóillatú tûzáldozatul az Úrnak.
\par 9 Szombatnapon pedig áldozz két ép bárányt, esztendõsöket, és két tized efa lisztlángot olajjal elegyített ételáldozatul, annak italáldozatjával egybe.
\par 10 Ez a szombati egészen égõáldozat szombatonként, a szüntelen való egészen égõáldozaton és annak italáldozatján kivül.
\par 11 A ti hónapjaitok kezdetén is áldozzatok az Úrnak egészen égõáldozatul két fiatal tulkot, egy kost, és hét egyesztendõs, ép bárányt.
\par 12 És egy-egy tulok mellé olajjal elegyített három tized efa lánglisztet ételáldozatul, egy-egy kos mellé pedig olajjal elegyített két tized efa lánglisztet ételáldozatul.
\par 13 És egy-egy bárány mellé egy-egy tized efa lisztlángot olajjal elegyítve ételáldozatul; égõáldozat ez, kedves illatú tûzáldozat ez az Úrnak.
\par 14 Az azokhoz való italáldozat pedig legyen fél hin bor egy tulok mellé, egy harmad hin egy kos mellé, és egy negyed hin egy bárány mellé. Ez a hónapos egészen égõáldozat minden hónap kezdetén, az esztendõnek hónapjai szerint.
\par 15 És egy kecskebakot is készítsetek el bûnért való áldozatul az Úrnak, a szüntelen való egészen égõáldozaton kívül, annak italáldozatával egyben.
\par 16 Az elsõ hónapban pedig, a hónapnak tizennegyedik napján az Úrnak páskhája van.
\par 17 És e hónapnak tizenötödik napján is ünnep van; hét napon át kovásztalan kenyeret egyetek.
\par 18 Elsõ napon legyen szent gyülekezés; semmi robota munkát ne végezzetek,
\par 19 Hanem tûzáldozatul egészen égõáldozatot vigyetek az Úrnak: két fiatal tulkot, egy kost, egy esztendõs bárányt hetet, a melyek épek legyenek.
\par 20 És azoknak ételáldozata olajjal elegyített lisztláng legyen; három tized efát tegyetek egy tulok mellé, és két tizedrészt egy kos mellé.
\par 21 Egy-egy bárány mellé egy-egy tizedrészt tegyetek, a hét bárány szerint.
\par 22 És bûnért való áldozatul egy bakot, hogy engesztelés legyen értetek.
\par 23 A reggeli egészen égõáldozaton kívül, (a mely szüntelen való égõáldozat) készítsétek el ezeket.
\par 24 Ezek szerint készítsetek naponként hét napon át kenyeret, kedves illatú tûzáldozatul az Úrnak; a szüntelen való egészen égõáldozaton kívül készíttessék az, annak italáldozatjával egyben.
\par 25 A hetedik napon is legyen néktek szent gyülekezésetek; semmi robota munkát ne végezzetek.
\par 26 A zsengék napján is, a mikor új ételáldozatot visztek az Úrnak a ti hetes ünnepeteken: szent gyülekezésetek legyen néktek; semmi robota munkát ne végezzetek.
\par 27 Hanem vigyetek kedves illatul egészen égõáldozatot az Úrnak: két fiatal tulkot, egy kost, és esztendõs bárányt hetet.
\par 28 És azokhoz való ételáldozatul olajjal elegyített háromtized efa lisztlángot egy tulok mellé, két tizedet egy kos mellé.
\par 29 Egy-egy bárány mellé egy-egy tizedrészt, a hét bárány szerint.
\par 30 Egy kecskebakot is, hogy engesztelés legyen értetek.
\par 31 A szüntelen való egészen égõáldozaton és annak égõáldozatján kivül készítsétek ezeket: épek legyenek, azoknak italáldozatjokkal egyben.

\chapter{29}

\par 1 A hetedik hónapban pedig a hónapnak elsõ napján szent gyülekezésetek legyen néktek. Semmi robota munkát ne végezzetek; kürt zengésének napja legyen az néktek.
\par 2 És készítsetek el kedves illatul egészen égõáldozatot az Úrnak, egy fiatal tulkot, egy kost, esztendõs ép bárányt hetet.
\par 3 És azokhoz ételáldozatul olajjal elegyített három tized efa lisztlángot a tulok mellé, két tizedrészt a kos mellé;
\par 4 És egy-egy tizedrészt egy-egy bárány mellé, a hét bárány szerint;
\par 5 És egy kecskebakot bûnért való áldozatul, hogy engesztelés legyen értetek,
\par 6 az újhold egészen égõáldozatján és annak ételáldozatján kivül, és a szöntelen való egészen égõáldozaton és annak ételáldozatján, és azoknak italáldozatjokon kivül, az õ rendjök szerint, kedves illatú tûzáldozatul az Úrnak.
\par 7 E hetedik hónapnak tizedik napján is szent gyülekezésetek legyen néktek; és sanyargassátok meg magatokat, semmi munkát ne végezzetek;
\par 8 És vigyetek az Úrnak kedves illatú egészen égõáldozatul: egy fiatal tulkot, egy kost, esztendõs bárányt hetet, épek legyenek;
\par 9 És azoknak ételáldozatául olajjal elegyített három tized efa lisztlángot a tulok mellé; két tizedrészt az egy kos mellé;
\par 10 Egy-egy tizedrészt egy-egy bárány mellé, a hét bárány szerint.
\par 11 Egy kecskebakot bûnért való áldozatul, az engesztelésre való bûnáldozaton kivül, és a szüntelen való egészen égõáldozaton, annak ételáldozatán és azoknak italáldozatjakon kívül.
\par 12 A hetedik hónapnak tizenötödik napján is szent gyülekezésetek legyen néktek; semmi robota munkát ne végezzetek, és az Úrnak ünnepét hét napon ünnepeljétek.
\par 13 És vigyetek az Úrnak egészen égõáldozatot, kedves illatú tûzáldozatul: tizenhárom fiatal tulkot, két kost, esztendõs bárányt tizennégyet, épek legyenek.
\par 14 És azoknak ételáldozatául olajjal elegyített lisztlángból három tizedrész efát egy-egy tulok mellé, a tizenhárom tulok szerint, két tizedrészt egy-egy kos mellé, a két kos szerint.
\par 15 És egy-egy tizedrészt minden bárány mellé, a tizennégy bárány szerint.
\par 16 És egy kecskebakot bûnért való áldozatul, a szüntelen való egészen égõáldozaton, annak ételáldozatján és italáldozatján kívül.
\par 17 Másodnapon pedig tizenkét fiatal tulkot, két kost, tizennégy ép bárányt, esztendõsöket.
\par 18 És azoknak ételáldozatjokat és italáldozatjokat, a tulkok mellé, a kosok mellé, a bárányok mellé, az õ számok szerint, a szokás szerint.
\par 19 És egy kecskebakot bûnért való áldozatul, a szüntelen való egészen égõáldozaton, annak ételáldozatján és azoknak italáldozatjokon kivül.
\par 20 Harmadnap pedig tizenegy fiatal tulkot, két kost, tizennégy ép bárányt, esztendõsöket.
\par 21 És azoknak ételáldozatjokat és italáldozatjokat, a tulkok mellé, a kosok mellé és a bárányok mellé, az õ számok szerint, a szokás szerint;
\par 22 És egy kecskebakot bûnért való áldozatul, a szüntelen való egészen égõáldozaton és annak ételáldozatján és italáldozatján kivül.
\par 23 Negyednapon pedig tíz fiatal tulkot, két kost, tizennégy ép bárányt, esztendõsöket.
\par 24 Azoknak ételáldozatjokat és italáldozatjokat a tulkok mellé, a kosok mellé és a bárányok mellé, az õ számok szerint, a szokás szerint;
\par 25 És egy kecskebakot bûnért való áldozatul, a szüntelen való egészen égõáldozaton, annak ételáldozatján és italáldozatján kívül.
\par 26 És ötödnapon kilencz fiatal tulkot, két kost, tizennégy ép bárányt, esztendõsöket:
\par 27 És azoknak ételáldozatjokat és italáldozatjokat, a tulkok mellé, a kosok mellé és a bárányok mellé, azoknak számok szerint, a szokás szerint.
\par 28 És egy bakot bûnért való áldozatul, a szüntelen való egészen égõáldozaton, annak ételáldozatján és italáldozatján kívül.
\par 29 És hatodnapon nyolcz tulkot, két kost, tizennégy ép bárányt, esztendõsöket.
\par 30 És azoknak ételáldozatjokat és italáldozatjokat a tulkok mellé, a kosok mellé és a bárányok mellé, az õ számok szerint, a szokás szerint;
\par 31 És egy bakot bûnért való áldozatul, a szüntelen való egészen égõáldozaton kívül, annak ételáldozatján és italáldozatján kívül.
\par 32 És hetedik napon hét tulkot, két kost, tizennégy ép bárányt, esztendõsöket;
\par 33 És azoknak ételáldozatjokat és italáldozatjokat a tulkok mellé, a kosok mellé és a bárányok mellé, az õ számok szerint, a szokás szerint;
\par 34 És egy bakot bûnért való áldozatul, a szüntelen való egészen égõáldozaton kívül, annak ételáldozatján és italáldozatján kívül.
\par 35 Nyolczadnapon bezáró ünnepetek legyen néktek, semmi robota munkát ne végezzetek;
\par 36 Hanem vigyetek az Úrnak egészen égõáldozatot, kedves illatú tûzáldozatul: egy tulkot, egy kost, hét ép bárányt, esztendõsöket;
\par 37 Azoknak ételáldozatjokat és italáldozatjokat, a tulok mellé, a kos mellé, és a bárányok mellé, az õ számok szerint, a szokás szerint;
\par 38 És egy bakot bûnért való áldozatul, a szüntelen való egészen égõáldozaton, annak ételáldozatján és italáldozatján kívül.
\par 39 Ezeket áldozzátok az Úrnak a ti ünnepeiteken, azokon kívül, a miket fogadásból, és szabad akaratból áldoztok, a ti egészen égõáldozataitokul, ételáldozataitokul, és italáldozataitokul, és hálaáldozataitokul.
\par 40 És megmondá Mózes Izráel fiainak mind azokat, a melyeket az Úr parancsolt vala Mózesnek.

\chapter{30}

\par 1 És szóla Mózes az Izráel fiai közt lévõ törzsek fejeinek, mondván: Ez az a beszéd, a melyet parancsolt az Úr:
\par 2 Ha valamely férfi fogadást fogad az Úrnak, vagy esküt tesz, hogy lekötelezze magát valami kötésre: meg ne szegje az õ szavát; a mint az õ szájából kijött, egészen úgy cselekedjék.
\par 3 Ha pedig asszony fogad fogadást az Úrnak, és kötésre kötelezi le magát az õ atyjának házában az õ fiatalságában;
\par 4 És hallja az õ atyja az õ fogadását, vagy kötelezését, a melylyel lekötelezte magát, és nem szól arra az õ atyja: akkor megáll minden õ fogadása, és minden kötelezés is, a melylyel lekötelezte magát, megálljon.
\par 5 Ha pedig megtiltja azt az õ atyja azon a napon, a melyen hallotta: nem áll meg semmi fogadása és kötelezése, a melylyel lekötelezte magát, és az Úr is megbocsát néki, mert az õ atyja tiltotta meg azt.
\par 6 Ha pedig férjhez megy, és így terhelik õt az õ fogadásai vagy ajkán kiszalasztott szava, a melylyel lekötelezte magát;
\par 7 És hallja az õ férje, és nem szól néki azon a napon, a melyen hallotta azt: akkor megállanak az õ fogadásai, és az õ kötelezései is, a melyekkel lekötelezte magát, megálljanak.
\par 8 Ha pedig azon a napon, a melyen hallja a férje, megtiltja azt: akkor erõtlenné teszi annak fogadását, a melyet magára vett, és az õ ajakinak kiszalasztott szavát, a melylyel lekötelezte magát; és az Úr is megbocsát néki.
\par 9 Az özvegy asszonynak pedig és az elváltnak minden fogadása, a melylyel lekötelezi magát, megáll.
\par 10 Ha pedig az õ férjének házában tesz fogadást, vagy esküvéssel kötelezi magát valami kötésre:
\par 11 Ha hallotta az õ férje és nem szólt arra, nem tiltotta meg azt: akkor annak minden fogadása megáll, és minden kötelezése, a melylyel lekötelezte magát, megálljon.
\par 12 De ha a férje teljesen erõtlenné teszi azokat azon a napon, a melyen hallotta: nem áll meg semmi, a mi az õ ajakin kijött, sem fogadása, sem az õ maga lekötelezése; az õ férje erõtlenné tette azokat, és az Úr megbocsát néki.
\par 13 Minden fogadását, és minden esküvéssel való kötelezését a maga megsanyargattatására, a férje teszi erõssé, és a férje teszi azt erõtlenné.
\par 14 Hogyha nem szólván nem szól néki a férje egy naptól fogva más napig, akkor megerõsíti minden fogadását, vagy minden kötelezését, a melyeket magára vett; megerõsíti azokat, mert nem szól néki azon a napon, a melyen hallotta azt.
\par 15 Ha pedig azután teszi erõtelenekké azokat, minekutána hallotta vala: õ hordozza az õ bûnének terhét.
\par 16 Ezek azok a rendelések, a melyeket parancsolt az Úr Mózesnek, a férj és az õ felesége között, az atya és leánya között, mikor még fiatalságában az õ atyjának házában van.

\chapter{31}

\par 1 És szóla az Úr Mózesnek, mondván:
\par 2 Állj bosszút Izráel fiaiért a Midiánitákon, azután a te népeidhez takaríttatol.
\par 3 Szóla azért Mózes a népnek, mondván: Készítsétek fel magatok közül a viadalra való embereket, és induljanak Midián ellen, hogy bosszút álljanak az Úrért Midiánon.
\par 4 Ezret-ezret egy-egy törzsbõl, Izráelnek minden törzsébõl küldjetek a hadba.
\par 5 Kiválogatának azért Izráel ezereibõl, ezeret törzsenként, tizenkét ezeret, viadalra készet.
\par 6 És elküldé õket Mózes, törzsenként ezeret-ezeret a hadba, és velök Fineást, Eleázár papnak fiát is a hadba; és a szent edények és a riadó kürtök valának az õ keze alatt.
\par 7 És harczolának Midián ellen, a miképen megparancsolta vala az Úr Mózesnek, és minden férfiút megölének.
\par 8 A Midián királyait is megölék, azoknak levágott népeivel egybe: Evit, Rékemet, Czúrt, Húrt, és Rebát, Midiánnak öt királyát; és Bálámot a Beór fiát is megölék fegyverrel.
\par 9 És fogságba vivék Izráel fiai a Midiániták feleségeit és azoknak kisdedeit, és azoknak minden barmát és minden nyáját, és minden vagyonát prédára veték.
\par 10 Minden városukat pedig az õ lakhelyeik szerint, és minden falvaikat tûzzel megégeték.
\par 11 És elvivének minden ragadományt és minden prédát mind emberekbõl, mind barmokból.
\par 12 És vivék Mózeshez és Eleázár paphoz, és Izráel fiainak gyülekezetéhez a foglyokat, a prédát és a ragadományokat a táborba, a mely a Moáb mezõségén vala a Jordán mellett, Jérikhó átellenében.
\par 13 Kimenének azért Mózes és Eleázár pap, és a gyülekezetnek minden fejedelme õ eléjök a táboron kívül.
\par 14 És megharaguvék Mózes a hadnak vezetõire, az ezredesekre és századosokra, a kik megjöttek vala a harczról.
\par 15 És monda nékik Mózes: Megtartottátok-é életben mind az asszonyokat?
\par 16 Ímé õk voltak, a kik Izráel fiait Bálám tanácsából hûtlenségre bírták az Úr ellen a Peór dolgában; és lõn csapás az Úr gyülekezetén.
\par 17 Most azért öljetek meg a kisdedek közül minden finemût; és minden asszonyt is, a ki férfit ismert azzal való hálás végett, megöljetek.
\par 18 Minden leánygyermeket pedig, a kik nem háltak férfiúval, tartsatok életben magatoknak.
\par 19 Ti pedig maradjatok a táboron kivül hét napig; a ki megölt valakit, és a ki hullát érintett, mind tisztítsátok meg magatokat harmad és hetednapon, magatokat és foglyaitokat.
\par 20 Minden ruhát, minden bõrbõl való eszközt, minden kecskeszõrbõl való készítményt és minden faedényt; tisztítsátok meg magatokat.
\par 21 És monda Eleázár pap a vitézeknek, a kik elmentek vala a hadba: Ez a törvény rendelése, a melyet parancsolt vala az Úr Mózesnek:
\par 22 Az aranyat, ezüstöt, rezet, vasat, az ónt és ólmot bizonyára;
\par 23 Minden egyebet is, a mi állja a tüzet, vigyetek át a tûzön, és megtisztíttatik, de a tisztító vízzel is tisztíttassék meg; mindazt pedig, a mi nem állja a tûzet, vízen vigyétek át.
\par 24 Ruháitokat pedig mossátok meg a hetedik napon, és tiszták lesztek: és ezután bemehettek a táborba.
\par 25 Újra szóla az Úr Mózesnek, mondván:
\par 26 Vedd számba az elfoglalt prédát, mind emberben, mind baromban, te és Eleázár, a pap, és a gyülekezet atyáinak fejei.
\par 27 És oszszad a prédát két részre: a hadakozók között, a kik hadba mentek, és az egész gyülekezet között.
\par 28 És végy részt az Úrnak a hadakozó férfiaktól, akik hadba mentek, ötszázból egy lelket, az emberek közül, az ökrök közül, a szamarak közül, és a juhok közül.
\par 29 Azoknak fele részébõl vegyétek, és adjad Eleázárnak, a papnak, felemelt áldozatul az Úrnak.
\par 30 Az Izráel fiainak járó fele részbõl pedig egy elfogottat végy ötvenbõl: emberekbõl, ökrökbõl, szamarakból, juhokból, és minden baromból; és adjad azokat a lévitáknak, a kik ügyelnek az Úr hajlékának ügyére.
\par 31 És úgy cselekedék Mózes és Eleázár, a pap, a miképen parancsolta vala az Úr Mózesnek.
\par 32 És vala az a préda azaz annak a zsákmánynak maradéka, a mit a hadakozó nép zsákmányolt: hatszáz hetvenöt ezer juh.
\par 33 És hetvenhét ezer ökör.
\par 34 És hatvanegy ezer szamár.
\par 35 Emberi lélek pedig: a leányok közül, a kik nem ismertek vala férfival való egyesülést, ilyen lélek összesen harminczkét ezer.
\par 36 Vala pedig az egyik fele, azoknak része, a kik hadba mentek vala: számszerint háromszáz harminczhét ezer és ötszáz juh.
\par 37 Vala pedig az Úrnak része a juhokból: hatszáz és hetvenöt.
\par 38 Az ökör pedig: harminczhat ezer; és azokból az Úrnak része: hetvenkettõ.
\par 39 És a szamár: harmincz ezer és ötszáz; azokból az Úrnak része: hatvanegy.
\par 40 Emberi lélek pedig tizenhat ezer; és azokból az Úrnak része: harminczhét lélek.
\par 41 És adá Mózes az Úrnak részét felemelt áldozatul Eleázárnak, a papnak, a miképen parancsolta vala az Úr Mózesnek.
\par 42 Az Izráel fiainak esõ másik fele részbõl pedig, a melyet elválasztott Mózes a hadakozó férfiakétól.
\par 43 (Vala pedig a gyülekezetre esõ felerész juhokból: háromszáz harminczhét ezer és ötszáz;
\par 44 Ökör: harminczhat ezer;
\par 45 Szamár: harmincz ezer és ötszáz;
\par 46 Emberi lélek: tizenhat ezer.)
\par 47 Az Izráel fiainak esõ fele részbõl pedig egy elfogottat vett Mózes ötvenbõl, az emberekbõl és a barmokból; és adá azokat a lévitáknak, a kik ügyelnek az Úr hajlékának ügyére, a miképen parancsolta vala az Úr Mózesnek.
\par 48 És járulának Mózeshez a had ezereinek vezetõi, az ezredesek és századosok.
\par 49 És mondának Mózesnek: Szolgáid megszámlálták a hadakozó férfiakat, a kik a mi kezünk alatt voltak, és senki közülünk el nem veszett.
\par 50 Hoztunk azért az Úrnak való áldozatul kiki a mit talált: aranyeszközöket, karlánczokat, karpereczeket, gyûrûket, fülönfüggõket, nyaklánczokat, hogy engesztelést végezzünk a mi lelkünkért az Úr elõtt.
\par 51 És elvevé Mózes és Eleázár, a pap, az aranyat õ tõlök, és a megkészített eszközöket is mind.
\par 52 És mindaz az arany, a melyet felemelt áldozatul vivének az Úrnak, tizenhat ezer hétszáz és ötven siklus vala az ezeredesektõl és századosoktól.
\par 53 A hadakozó férfiak közül kiki magának zsákmányolt.
\par 54 Miután elvette vala Mózes és Eleázár, a pap, az aranyat az ezeredesektõl és századosoktól, bevivék azt a gyülekezetnek sátorába, Izráel fiaira való emlékeztetõül az Úr elé.

\chapter{32}

\par 1 Rúben fiainak pedig és Gád fiainak igen sok barmok vala. Mikor látták a Jázér földét és a Gileád földét, hogy ímé az a hely marhatartani való hely:
\par 2 Eljövének Gád fiai és Rúben fiai, és szólának Mózesnek és Eleázárnak, a papnak, és a gyülekezet fejedelmeinek, mondván:
\par 3 Atárót, Díbon, Jázér, Nimra, Hesbon, Elealé, Sebám, Nébó és Béon:
\par 4 A föld, a melyet megvert az Úr az Izráel fiainak gyülekezete elõtt, baromtartó föld az; a te szolgáidnak pedig sok marhájok van.
\par 5 És mondának: Hogyha kedvet találtunk a te szemeid elõtt, adassék ez a föld a te szolgáidnak örökségül, ne vígy minket át a Jordánon.
\par 6 Mózes pedig monda a Gád fiainak és Rúben fiainak: Avagy a ti atyátokfiai hadakozni menjenek, ti pedig itt maradjatok?
\par 7 És miért idegenítitek el Izráel fiainak szívét, hogy által ne menjenek a földre, a melyet adott nékik az Úr?
\par 8 A ti atyáitok cselekedtek így, mikor elbocsátám õket Kádes-Bárneából, hogy nézzék meg azt a földet;
\par 9 Mert felmentek az Eskól völgyéig, és megnézék a földet, és elidegeníték Izráel fiainak szívét, hogy be ne menjenek arra a földre, a melyet adott vala nékik az Úr.
\par 10 Azért megharaguvék az Úr azon a napon, és megesküvék, mondván:
\par 11 Nem látják meg azok az emberek, a kik feljöttek Égyiptomból, húsz esztendõstõl fogva és feljebb, azt a földet, a mely felõl megesküdtem Ábrahámnak, Izsáknak és Jákóbnak, mivelhogy nem tökéletesen jártak én utánam;
\par 12 Kivéve a Kenizeus Kálebet, a Jefunné fiát és Józsuét, a Nún fiát, mivelhogy tökéletesen jártak az Úr után.
\par 13 És megharaguvék az Úr Izráelre, és bujdostatá õket a pusztában negyven esztendeig; míg megemészteték az egész nemzetség, a mely gonoszt cselekedett vala az Úr szemei elõtt.
\par 14 És ímé feltámadtatok a ti atyáitok helyett, bûnös emberek maradékai, hogy az Úr haragjának tüzét még öregbítsétek Izráel ellen.
\par 15 Hogyha elfordultok az õ utaitól, még tovább is otthagyja õt a pusztában, és elvesztitek mind az egész népet.
\par 16 És járulának közelebb õ hozzá, és mondának: Barmainknak juhaklokat építünk itt, és a mi kicsinyeinknek városokat;
\par 17 Magunk pedig felfegyverkezve, készséggel megyünk Izráel fiai elõtt, míg beviszszük õket az õ helyökre; gyermekeink pedig a kerített városokban maradnak e  földnek lakosai miatt.
\par 18 Vissza nem térünk addig a mi házainkhoz, a míg Izráel fiai közül meg nem kapja kiki az õ örökségét.
\par 19 Mert nem veszünk mi részt az örökségben õ velök a Jordánon túl és tovább, mivelhogy meg van nékünk a mi örökségünk a Jordánon innen napkelet felõl.
\par 20 És monda nékik Mózes: Ha azt cselekszitek, a mit szóltok; ha az Úr elõtt készültök fel a hadra;
\par 21 És átmegy közületek minden fegyveres a Jordánon az Úr elõtt, míg kiûzi az õ ellenségeit maga elõtt;
\par 22 És csak azután tértek vissza, ha az a föld meghódol az Úr elõtt: akkor ártatlanok lesztek az Úr elõtt, és Izráel elõtt, és az a föld birtokotokká lesz néktek az Úr elõtt.
\par 23 Hogyha nem így cselekesztek, ímé vétkeztek az Úr ellen; és gondoljátok meg, hogy a ti bûnötöknek büntetése utólér benneteket!
\par 24 Építsetek magatoknak városokat a ti kicsinyeitek számára, és a ti juhaitoknak aklokat; de a mit fogadtatok, azt megcselekedjétek.
\par 25 És szólának Gád fiai és Rúben fiai Mózesnek, mondván: A te szolgáid úgy cselekesznek, a mint az én Uram parancsolja.
\par 26 A mi kicsinyeink, feleségeink, juhaink és mindenféle barmaink ott lesznek Gileád városaiban;
\par 27 A te szolgáid pedig átmennek mindnyájan hadra felkészülve, harczolni az Úr elõtt, a miképen az én Uram szól.
\par 28 Parancsot ada azért õ felõlök Mózes Eleázárnak, a papnak, és Józsuénak a Nún fiának, és az Izráel fiai törzseibõl való atyák fejeinek.
\par 29 És monda nékik Mózes: Ha átmennek a Gád fiai és a Rúben fiai veletek a Jordánon, mindnyájan hadakozni készen az Úr elõtt, és meghódol a föld ti elõttetek: akkor adjátok nékik a Gileád földét birtokul.
\par 30 Ha pedig nem mennek át fegyveresen veletek, akkor veletek kapjanak birtokot a Kanaán földén.
\par 31 És felelének a Gád fiai és a Rúben fiai, mondván: A mit mondott az Úr a te szolgáidnak, akképen cselekeszünk.
\par 32 Mi átmegyünk fegyveresen az Úr elõtt a Kanaán földére, de miénk legyen a mi örökségünknek birtoka a Jordádon innen.
\par 33 Nékik adá azért Mózes, tudniillik a Gád fiainak, a Rúben fiainak és a József fiának, Manasse féltörzsének, Szihonnak, az Emoreusok királyának országát, és Ógnak, Básán királyának országát; azt a földet az õ városaival, határaival, annak a földnek városait köröskörül.
\par 34 És megépítik a Gád fiai Dibont, Ataróthot, Aroért,
\par 35 Atróth-Sofánt, Jázért, Jogbehát,
\par 36 Beth-Nimrát, és Beth-Haránt, kerített városokat juhaklokkal egyben.
\par 37 A Rúben fiai pedig megépíték Hesbont, Elealét, Kirjáthaimot,
\par 38 Nébót, Baál-Meont, (változtatott néven nevezve) és Sibmát. És új neveket adának a városoknak, a melyeket építének.
\par 39 Mákirnak, a Manasse fiának fiai pedig Gileádba vonulának, és bevevék azt, és kiûzék az Emoreust, a ki ott vala.
\par 40 Adá azért Mózes Gileádot Mákirnak, a Manasse fiának, és lakozék abban.
\par 41 Jáir pedig, a Manasse fia elméne, és bevevé azoknak falvait, és hívá azokat Jáir falvainak.
\par 42 Nóbah is elméne, és bevevé Kenáthot és annak városait, és hívá azt Nóbáhnak, a maga nevérõl.

\chapter{33}

\par 1 Ezek Izráel fiainak szállásai, a kik kijövének Égyiptom földébõl az õ seregeik szerint Mózes és Áron vezetése alatt.
\par 2 (Megírá pedig Mózes az õ kijövetelöket az õ szállásaik szerint, az Úrnak rendeletére.) Ezek az õ szállásaik, az õ kijövetelök szerint.
\par 3 Elindulának Rameszeszbõl az elsõ hónapban, az elsõ hónap tizenötödik napján. A páskha után való napon jövének ki Izráel fiai felemelt kézzel, az egész Égyiptom láttára.
\par 4 (Az égyiptombeliek pedig temetik vala azokat a kiket megölt vala az Úr õ közöttök, minden elsõszülöttjöket; az isteneiken is ítéletet tartott vala az Úr.)
\par 5 És elindulának Izráel fiai Rameszeszbõl, és tábort ütének Szukkótban.
\par 6 És elindulának Szukkótból, és tábort ütének Ethámban, a mely van a pusztának szélén.
\par 7 És elindulának Ethámból, és fordulának Pihahiróth felé, a mely van Baál-Czefon elõtt, és tábort ütének Migdol elõtt,
\par 8 És elindulának Pihahiróthból, és átmenének a tenger közepén a pusztába; és menének három napi járásnyira az Ethám pusztáján; és tábort ütének Márában.
\par 9 És elindulának Márából, és jutának Élimbe; Élimben pedig vala tizenkét kútfõ és hetven pálmafa; és tábort ütének ott.
\par 10 És elindulának Élimbõl, és tábort ütének a Veres tenger mellett.
\par 11 És elindulának a Veres tengertõl, és tábort ütének a Szin pusztájában.
\par 12 És elindulának a Szin pusztájából, és tábort ütének Dofkában.
\par 13 És elindulának Dofkából, és tábort ütének Álúsban.
\par 14 És elindulának Álúsból, és tábort ütének Refidimben, de nem vala ott a népnek inni való vize.
\par 15 És elindulának Refidimbõl, és tábort ütének a Sinai pusztájában.
\par 16 És elindulának a Sinai pusztájából, és tábort ütének Kibrót-Thaavában.
\par 17 És elindulának Kibrót-Thaavából, és tábort ütének Haserótban.
\par 18 És elindulának Haserótból, és tábort ütének Rithmában.
\par 19 És elindulának Rithmából, és tábort ütének Rimmon-Péreczben.
\par 20 És elindulának Rimmon-Péreczbõl, és tábort ütének Libnában.
\par 21 És elindulának Libnából, és tábort ütének Risszában.
\par 22 És elindulának Risszából, és tábort ütének Kehélátban.
\par 23 És elindulának Kehélátból, és tábort ütének a Séfer hegyénél.
\par 24 És elindulának a Séfer hegyétõl és tábort ütének Haradában.
\par 25 És elindulának Haradából, és tábort ütének Makhélótban.
\par 26 És elindulának Makhélótból, és tábort ütének Tháhátban.
\par 27 És elindulának Tháhátból, és tábort ütének Thárakhban.
\par 28 És elindulának Thárakhból, és tábort ütének Mitkában.
\par 29 És elindulának Mitkából, és tábort ütének Hasmonában.
\par 30 És elindulának Hasmonából, és tábort ütének Moszérótban.
\par 31 És elindulának Moszérótból, és tábort ütének Bené-Jaakánban.
\par 32 És elindulának Bené-Jaakánból, és tábort ütének Hór-Hagidgádban.
\par 33 És elindulának Hór-Hagidgádból, és tábort ütének Jotbathában.
\par 34 És elindulának Jotbathából, és tábort ütének Abronában.
\par 35 És elindulának Abronából, és tábort ütének Eczjon-Geberben.
\par 36 És elindulának Eczjon-Geberbõl, és tábort ütének Czin pusztájában; ez Kádes.
\par 37 És elindulának Kádesbõl, és tábort ütének a a Hór hegyénél, az Edom földének szélén.
\par 38 Felméne pedig Áron, a pap, a Hór hegyére az Úr rendelése szerint, és meghala ott a negyvenedik esztendõben, Izráel fiainak Égyiptom földébõl való kijövetelök után, az ötödik hónapban, a hónap elsején.
\par 39 Áron pedig száz és huszonhárom esztendõs vala, mikor meghala a Hór hegyén.
\par 40 Hallott pedig a Kananeus, Arad királya (ez pedig dél felõl lakozik vala a Kanaán földén) Izráel fiainak jövetele felõl.
\par 41 És elindulának a Hór hegyétõl, és tábort ütének Czalmonában.
\par 42 És elindulának Czalmonából, és tábort ütének Púnonban.
\par 43 És elindulának Púnonból, és tábort ütének Obothban.
\par 44 És elindulának Obothból, és tábort ütének Ijé-Abárimban, a Moáb határán.
\par 45 És elindulának Ijé-Abárimból, és tábort ütének Dibon-Gádban.
\par 46 És elindulának Dibon-Gádból, és tábort ütének Almon-Diblathaimban.
\par 47 És elindulának Almon-Diblathaimból, és tábort ütének az Abarim hegységnél, Nébó ellenében.
\par 48 És elindulának az Abarim hegységétõl, és tábort ütének a Moáb mezõségén, a Jordán mellett, Jérikhó ellenében.
\par 49 Tábort ütének pedig a Jordán mellett, Beth-Hajjesimóthtól Abel-Hassittimig, a Moáb mezõségén.
\par 50 És szóla az Úr Mózesnek a Moáb mezõségén, a Jordán mellett, Jerikhó ellenében, mondván:
\par 51 Szólj Izráel fiainak, és mondd meg nékik: Mikor átmentek ti a Jordánon a Kanaán földére:
\par 52 Ûzzétek ki akkor a földnek minden lakosát a ti színetek elõl, és veszessétek el minden írott képeiket, és minden õ  öntött bálványképeiket is elveszessétek, és minden magaslataikat rontsátok el!
\par 53 Ûzzétek ki a földnek lakóit, és lakozzatok abban; mert néktek adtam azt a földet, hogy bírjátok azt.
\par 54 Azt a földet pedig sorsvetés által vegyétek birtokotokba a ti nemzetségeitek szerint. A nagyobb számúnak nagyobbítsátok az õ örökségét, a kisebb számúnak kisebbítsd az õ örökségét. A hová a sors esik valaki számára, az legyen az övé. A ti atyáitoknak törzsei szerint vegyétek át birtokotokat.
\par 55 Ha pedig nem ûzitek ki annak a földnek lakosait a ti színetek elõl, akkor, a kiket meghagytok közülök, szálkákká lesznek a ti szemeitekben, és tövisekké a ti oldalaitokban, és ellenségeitek lesznek néktek a földön, a melyen lakoztok.
\par 56 És akkor a miképen gondoltam, hogy cselekszem azokkal, úgy cselekszem majd veletek.

\chapter{34}

\par 1 És szólal az Úr Mózesnek, mondván:
\par 2 Parancsold meg Izráel fiainak, és mondd meg nékik: Hogyha bementek ti a Kanaán földére; (ez a föld, a mely örökségûl esik néktek, tudniillik a Kanaán földe az õ határai szerint),
\par 3 Akkor legyen a ti déli oldalatok a Czin pusztájától fogva Edom határáig, és legyen a ti déli határotok a Sós tenger végétõl napkelet felé.
\par 4 És kerüljön a határ dél felõl az Akrabbim hágójáig, és menjen át Czinig, és a vége legyen Kádes-Barneától délre; és menjen tova Haczár-Adárig, és menjen Aczmonig.
\par 5 Azután kerüljön a határ Aczmontól Égyiptom patakáig, a vége pedig a tengernél legyen.
\par 6 A napnyugoti határotok pedig legyen néktek a nagy tenger; ez legyen néktek a napnyugoti határotok.
\par 7 Ez legyen pedig a ti északi határotok: a nagy tengertõl fogva vonjatok határt a Hór hegyének.
\par 8 A Hór hegyétõl vonjatok határt a Hamáthba való bejárásig; a határnak vége pedig Czedádnál legyen.
\par 9 És tovamenjen a határ Zifronig, a vége pedig Haczar-Enán legyen. Ez legyen néktek az északi határotok.
\par 10 A napkeleti határt pedig vonjátok Haczar-Enántól Sefámig.
\par 11 És hajoljon le a határ Sefámtól Ribláig, Aintól napkeletre; és újra hajoljon le a határ, és érje a Kinnéreth tenger partját napkelet felé.
\par 12 Azután hajoljon le a határ a Jordán felé, a vége pedig a Sós tenger legyen. Ez legyen a ti földetek az õ határai szerint köröskörül.
\par 13 És parancsot ada Mózes Izráel fiainak, mondván: Ez az a föld, a melyet sors által vesztek birtokotokba, a mely felõl parancsot ada az Úr, hogy adjam azt kilencz törzsnek, és fél törzsnek;
\par 14 Mert megkapták a Rúbeniták fiainak törzse az õ atyáiknak háza szerint, a Gáditák fiainak törzse, az õ atyáiknak háza szerint, és a Manasse fél törzse is, megkapták az õ örökségöket.
\par 15 Két törzs és egy fél törzs megkapta az õ örökségét a Jordánon túl, Jérikhó ellenében napkelet felõl.
\par 16 Szóla azután az Úr Mózesnek, mondván:
\par 17 Ezek azoknak a férfiaknak nevei, a kik örökségül fogják néktek elosztani azt a földet: Eleázár, a pap, és Józsué, a Nún fia.
\par 18 És törzsenként egy-egy fejedelmet vegyetek mellétek a földnek örökségül való elosztására.
\par 19 Ezek a férfiaknak nevei: a Júda törzsébõl Káleb, a Jefunné fia.
\par 20 A Simeon fiainak törzsébõl Sámuel, az Ammihúd fia.
\par 21 A Benjámin törzsébõl Elidád, a Kiszlon fia.
\par 22 A Dán fiainak törzsébõl Bukki fejedelem, a Jógli fia.
\par 23 A József fia közül, a Manasse fiainak törzsébõl Hanniél fejedelem, az Efód fia.
\par 24 Az Efraim fiainak törzsébõl Kemuél fejedelem, a Siftán fia.
\par 25 És a Zebulon fiainak törzsébõl Eliczáfán fejedelem, a Parnák fia.
\par 26 És az Izsakhár fiainak törzsébõl Paltiél fejedelem, az Azzán fia.
\par 27 És az Áser fiainak törzsébõl Akhihúd fejedelem, a Selómi fia.
\par 28 És a Nafthali fiainak törzsébõl Pédahél fejedelem, az Ammihúd fia.
\par 29 Ezek azok, a kiknek megparancsolá az Úr, hogy örökséget oszszanak Izráel fiainak a Kanaán földén.

\chapter{35}

\par 1 És szóla az Úr Mózesnek a Moáb mezõségén a Jordán mellett, Jérikhó ellenében, mondván:
\par 2 Parancsold meg Izráel fiainak, hogy adjanak a lévitáknak az õ örökségi birtokukból lakásra való városokat; a városokhoz pedig adjatok azok környékén legelõt is a lévitáknak;
\par 3 Hogy legyenek nékik a városok lakóhelyekül, a legelõk pedig legyenek az õ barmaiknak, jószágaiknak és mindenféle állatjoknak.
\par 4 És azoknak a városoknak legelõi, a melyeket a lévitáknak adtok, a város falától és azon kivül, ezer singnyire legyenek köröskörül.
\par 5 Mérjetek azért a városon kivül, napkelet felõl két ezer singet, dél felõl is kétezer singet, napnyugot felõl kétezer singet, és észak felõl kétezer singet; és a város legyen középben. Ez legyen számukra a városok legelõje.
\par 6 A városok közül pedig, a melyeket a lévitáknak adtok, hat legyen menedékváros, a melyeket azért adjatok, hogy oda szaladjon a gyilkos; és  azokon kivül adjatok negyvenkét várost.
\par 7 Mind a városok, a melyeket adnotok kell a lévitáknak, negyvennyolc város, azoknak legelõivel egyben.
\par 8 A mely városokat pedig Izráel fiainak örökségébõl adtak, azokhoz attól, a kinek több van, többet vegyetek, és attól, a kinek kevesebb van, kevesebbet vegyetek; mindenik az õ örökségéhez képest, a melyet örökül kapott, adjon az õ városaiból a lévitáknak.
\par 9 És szóla az Úr Mózesnek, mondván:
\par 10 Szólj Izráel fiainak, és mondd meg nékik: Mikor átmentek ti a Jordánon a Kanaán földére:
\par 11 Válaszszatok ki magatoknak városokat, a melyek menedékvárosaitok legyenek, hogy oda szaladjon a gyilkos, a ki  történetbõl öl meg valakit.
\par 12 És legyenek azok a ti városaitok menedékül a vérbosszúló ellen, és ne haljon meg a gyilkos, míg ítéletre nem áll a gyülekezet elé.
\par 13 A mely városokat pedig oda adtok, azok közül hat legyen néktek menedékvárosul.
\par 14 Három várost adjatok a Jordánon túl, és három várost adjatok a Kanaán földén; menedékvárosok legyenek azok.
\par 15 Izráel fiainak és a jövevénynek, és az õ közöttök lakozónak menedékül legyen az a hat város, hogy oda szaladjon az, a ki  történetbõl öl meg valakit.
\par 16 De ha valaki vaseszközzel úgy üt meg valakit, hogy meghal, gyilkos az; halállal lakoljon a gyilkos.
\par 17 És ha kézben levõ kõvel, a melytõl meghalhat, üti meg úgy, hogy meghal, gyilkos az; halállal lakoljon a gyilkos.
\par 18 Vagy ha kézben lévõ faeszközzel, a melytõl meghalhat, üti meg úgy, hogy meghal, gyilkos az; halállal lakoljon a gyilkos.
\par 19 A vérbosszuló rokon ölje meg a gyilkost; mihelyt találkozik vele, ölje meg azt.
\par 20 Hogyha gyûlölségbõl taszítja meg õt, vagy szántszándékkal úgy hajít valamit reá, hogy meghal;
\par 21 Vagy ellenségeskedésbõl kezével üti meg azt úgy, hogy meghal: halállal lakoljon az, a ki ütötte; gyilkos az, a vérbosszuló rokon ölje meg azt a gyilkost, mihelyt találkozik vele.
\par 22 Ha pedig hirtelenségbõl, ellenségeskedés nélkül taszítja meg õt; vagy nem szántszándékból hajít reá akármiféle eszközt;
\par 23 Vagy akármiféle követ, a melytõl meghalhat, úgy ejt valakire, a kit nem látott, hogy meghal, holott nem volt õ annak ellensége, sem nem kereste annak vesztét:
\par 24 Akkor ítéljen a gyülekezet az agyonütõ között és a vérbosszuló rokon között e törvények szerint.
\par 25 És mentse ki a gyülekezet a gyilkost a vérbosszuló rokonnak kezébõl, és küldje vissza azt a gyülekezet az õ menedékvárosába, a melybe szaladott vala, és lakozzék abban, míg meghal a fõpap, a ki felkenetett a szent olajjal.
\par 26 Ha pedig kimegy a gyilkos az õ menedékvárosának határából, a melybe szaladott vala;
\par 27 És találja õt a vérbosszuló rokon az õ menedékvárosának határán kivül, és megöli az a vérbosszuló rokon a gyilkost; nem lesz annak vére õ rajta;
\par 28 Mert az õ menedékvárosában kell laknia a fõpap haláláig; a fõpap halála után pedig visszatérhet a gyilkos az õ örökségének földére.
\par 29 És legyenek ezek néktek ítéletre való rendelések a ti nemzetségeitek szerint, minden lakhelyeteken.
\par 30 Ha valaki megöl valakit, tanúk szavára gyilkolják meg a gyilkost; de egy tanú nem lehet elég tanú senki ellen, hogy meghaljon.
\par 31 Az olyan gyilkos életéért pedig ne vegyetek el váltságot, a ki halálra való gonosz, hanem halállal lakoljon.
\par 32 Attól se vegyetek váltságot, a ki az õ menedékvárosába szaladt, hogy visszamehessen és otthon lakozzék a fõpap haláláig.
\par 33 És meg ne fertõztessétek a földet, a melyben lesztek; mert a vér, az megfertézteti a földet, és a földnek nem szerezhetõ engesztelés a vér miatt, a mely kiontatott azon, csak annak vére által, a ki kiontotta azt.
\par 34 Ne tisztátalanítsd meg azért azt a földet, a melyben laktok, a melyben én is lakozom; mert én, az Úr, Izráel fiai között lakozom.

\chapter{36}

\par 1 Járulának pedig Mózeshez az atyák fejei a Gileád fiainak nemzetségébõl, a ki Mákirnak, Manasse fiának fia vala, a József fiainak nemzetségébõl, és szólának Mózes elõtt és Izráel fiai atyáinak fejedelmei elõtt.
\par 2 És mondának: Az én uramnak megparancsolta az Úr, hogy sors által adja ezt a földet örökségül Izráel fiainak; de azt is megparancsolta az Úr az én Uramnak, hogy a mi atyánkfiának, Czélofhádnak örökségét adja oda az õ leányainak.
\par 3 És ha Izráel fiai közül valamely más törzs fiaihoz mennek feleségül: elszakasztatik az õ örökségök a mi atyáink örökségétõl, és oda csatoltatik az annak a törzsnek örökségéhez, a melynél feleségül lesznek; a mi örökségünk pedig megkisebbedik.
\par 4 És mikor Izráel fiainál a kürtzengés ünnepe lesz, az õ örökségök akkor is annak a törzsnek örökségéhez csatoltatik, a melynél feleségül lesznek. Eképen a mi atyáink törzseinek örökségébõl vétetik el azok öröksége.
\par 5 Parancsot ada azért Mózes Izráel fiainak az Úr rendelése szerint, mondván: A József fiainak törzse igazat szól.
\par 6 Ez az, a mit parancsolt az Úr a Czélofhád leányai felõl, mondván: A kihez jónak látják, ahhoz menjenek feleségül, de csak az õ atyjok törzsének háznépébõl valóhoz menjenek feleségül,
\par 7 Hogy át ne szálljon Izráel fiainak öröksége egyik törzsrõl a másik törzsre; hanem Izráel fiai közül kiki ragaszkodjék az õ atyái törzsének örökségéhez.
\par 8 És minden leányzó, a ki örökséget kap Izráel fiainak törzsei közül, az õ atyja törzsebeli háznépbõl valóhoz menjen feleségül, hogy Izráel fiai közül kiki megtartsa az õ atyáinak örökségét.
\par 9 És ne szálljon az örökség egyik törzsrõl a másik törzsre, hanem Izráel fiainak törzsei közül kiki ragaszkodjék az õ örökségéhez.
\par 10 A miképen megparancsolta vala az Úr Mózesnek, a képen cselekedtek vala a Czélofhád leányai.
\par 11 Mert Makhla, Thircza, Hogla, Milkha, Hogla, Milkha és Nóa, a Czélofhád leányai, az õ nagybátyjok fiaihoz mentek vala feleségül;
\par 12 A József fiának, Manassé fiainak nemzetségébõl valókhoz mentek vala feleségül; és lõn azoknak öröksége, az õ atyjok nemzetségének törzsénél.
\par 13 Ezek a parancsolatok és vegzések, a melyeket Mózes által parancsolt az Úr Izráel fiainak, Moáb mezõségén a Jordán mellett, Jerikhó ellenében.


\end{document}