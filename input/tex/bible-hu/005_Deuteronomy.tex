\begin{document}

\title{Második Törvénykönyv}


\chapter{1}

\par 1 Ezek az igék, a melyeket szólott Mózes az egész Izráelnek, a Jordánon túl a pusztában, a mezõségen, a Veres tenger ellenében, Párán és Tófel, és Lábán és Haczéróth, és Dizaháb között.
\par 2 Tizenegy napi járóföldön Hórebtõl fogva, a Szeir hegyének  menve, Kádes-Barneáig.
\par 3 Lõn pedig a negyvenedik esztendõben, a tizenegyedik hónapban, a hónapnak elsõ napján, szóla Mózes az Izráel fiainak mind a szerint, a mint parancsolt vala az Úr néki azok felõl.
\par 4 Minekutána megverte vala Szihont az Emoreusok királyát, a ki lakik vala Hesbonban, és Ógot, Básánnak királyát Edreiben, a ki lakik vala Asthárótban:
\par 5 A Jordánon túl, a Moáb földén, kezdé Mózes magyarázni ezt a törvényt, mondván:
\par 6 Az Úr, a mi Istenünk szólott nékünk a Hóreben, ezt mondván: Elég ideig laktatok e hegy alatt;
\par 7 Forduljatok meg, és induljatok, és menjetek az Emoreusok hegyére, és annak minden szomszéd vidékére, a mezõségen, és a hegyes-völgyes földön, és dél felé, és a tengernek partján, a Kananeusok földére és a Libánusra, a nagy folyóvízig, az Eufrates folyóig.
\par 8 Ímé elõtökbe adtam a földet; menjetek be, és bírjátok azt a földet, a mely felõl megesküdt az Úr a ti atyáitoknak, Ábrahámnak, Izsáknak és Jákóbnak, hogy nékik adja, és õ utánok az õ magvoknak.
\par 9 És szólottam vala néktek abban az idõben, ezt mondván: Nem viselhetlek egymagam titeket;
\par 10 Az Úr, a ti Istenetek megsokasított titeket elannyira, hogy oly sokan vagytok ti most, mint az égnek csillagai.
\par 11 Az Úr, a ti atyáitoknak Istene szaporítson meg titeket ezerszerte is inkább mint most vagytok, és áldjon meg titeket, a miképen igérte néktek!
\par 12 Miképen viselhetném én egymagam a terheiteket és a ti bajaitokat és a ti pereiteket?
\par 13 Válaszszatok magatoknak bölcs, értelmes és a ti törzseitekben ismeretes férfiakat, és én azokat elõljáróitokká teszem.
\par 14 És felelétek nékem, és mondátok: Jó dolog, a mit mondál, hogy míveled azt.
\par 15 Vevém azért a ti törzseiteknek fõbbjeit, a bölcs és ismeretes férfiakat, és tevém õket elõljáróitokká: ezredesekké, századosokká, ötvenedesekké, tizedesekké és tiszttartókká, a ti törzseitek szerint.
\par 16 És megparancsolám abban az idõben a ti biráitoknak, mondván: Hallgassátok ki atyátokfiait, és ítéljetek igazságosan mindenkit, a õ atyjafiaival és jövevényeivel egyben.
\par 17 Ne legyetek személyválogatók az ítéletben: kicsinyt úgy, mint nagyot hallgassatok ki; ne féljetek senkitõl, mert az ítélet az Istené; a mi pedig nehéznek tetszik néktek, én elõmbe hozzátok, és én meghallgatom azt.
\par 18 És megparancsoltam néktek abban az idõben mindent, a mit cselekedjetek.
\par 19 Azután elindulánk a Hórebtõl, és bejártuk ama nagy és rettenetes pusztát, a melyet láttatok, az Emoreusok hegyének menve, a miképen parancsolta volt az Úr, a mi Istenünk nékünk; és eljutánk Kádes-Barneához.
\par 20 És mondám néktek: Eljutottatok az Emoreusok hegyéig, a melyet az Úr, a mi Istenünk ád nékünk.
\par 21 Ímé elõdbe adta az Úr, a te Istened ezt a földet: menj fel, bírjad azt, a miképen megmondotta az Úr, a te atyáidnak Istene néked; ne félj, és meg ne rettenj!
\par 22 Ti pedig mindnyájan hozzám járulátok és mondátok: Küldjünk embereket elõre, hogy kémleljék meg nékünk azt a földet, és hozzanak nékünk hírt az út felõl, a melyen felmenjünk, és a városok felõl, a melyekbe bevonuljunk.
\par 23 És tetszék nékem ez a beszéd, és vevék közületek tizenkét férfiút, minden törzsbõl egyet-egyet.
\par 24 És fordulának és felmenének a hegyre, és eljutának az Eskól völgyéig, és kikémlelék azt.
\par 25 És võnek kezeikbe annak a földnek gyümölcsébõl, és alá hozák hozzánk, és hírt hozának nékünk és mondának: Jó az a föld, a melyet az Úr, a mi Istenünk ád nékünk.
\par 26 De ti nem akartatok felmenni; hanem pártot ütétek az Úrnak, a ti Isteneteknek parancsa ellen.
\par 27 És zúgolódátok a ti sátoraitokban, és mondátok: mivelhogy gyûlöl minket az Úr, azért hozott ki minket Égyiptom földébõl, hogy adjon minket az Emoreus kezébe, és elpusztítson minket.
\par 28 Hová mennénk fel mi? A mi atyánkfiai megrettenték a mi szíveinket, mondván: Az a nép nagyobb és szálasabb nálunknál; a városok nagyok és megerõsíttettek az égig; még Anák fiakat is láttunk ott!
\par 29 Akkor mondám néktek: Ne rettegjetek és ne féljetek azoktól;
\par 30 Az Úr, a ti Istenetek, a ki elõttetek megy, õ hadakozik ti érettetek mind a szerint, a mint cselekedett vala veletek Égyiptomban a ti szemeitek elõtt;
\par 31 És a pusztában, a hol láttad, hogy úgy hordozott téged az Úr, a te Istened, a miképen hordozza az ember az õ fiát, mind az egész úton, a melyen jártatok, míg jutátok e helyre.
\par 32 Mindazáltal nem hivétek az Úrnak, a ti Isteneteknek.
\par 33 A ki elõttetek jár vala az úton, hogy helyet szemeljen ki néktek, a hol táborozzatok, éjjel tûzben, hogy megmutassa néktek az útat, a melyen járjatok, és felhõben nappal.
\par 34 Meghallá pedig az Úr beszédetek szavát, és megharaguvék, és megesküvék, mondván:
\par 35 E gonosz nemzetségbõl való emberek közül egy sem látja meg azt a jó földet, a mely felõl megesküdtem, hogy a ti atyáitoknak adom;
\par 36 Kivéve Kálebet, a Jefunné fiát; õ meglátja azt, és õ néki adom azt a földet, a melyet tapodott, és az õ fiainak, mert tökéletességgel követte az Urat.
\par 37 Még én reám is megharaguvék az Úr miattatok, mondván: Te sem mégy oda be!
\par 38 Józsué, a Nún fia, a ki áll te elõtted, õ megy be oda; azért biztassad õt, mert õ  osztja el az örökséget Izráelnek.
\par 39 És a ti kicsinyeitek, a kikrõl szólátok hogy prédául lesznek, és a ti fiaitok, a kik nem tudnak most sem jót, sem gonoszt, azok mennek be oda, mert nékik adom azt, és õk bírják azt.
\par 40 Ti pedig forduljatok vissza, és induljatok a pusztába, a Veres tenger felé.
\par 41 És azt felelétek, és azt mondátok nékem: Vétkeztünk az Úr ellen, mi felmegyünk és hadakozunk mind a szerint, a mint parancsolta nékünk az Úr, a mi Istenünk! És felövezétek magatokat, kiki az õ harczi eszközeivel, és készek valátok felmenni a hegyre.
\par 42 Monda pedig az Úr nékem: Mondd meg nékik: Ne menjetek fel, és ne harczoljatok, mert nem vagyok közöttetek: hogy meg ne verettessetek a ti ellenségeitektõl.
\par 43 És megmondanám néktek, de nem hallgattatok rám, hanem pártot ütöttetek az Úr parancsolata ellen, és vakmerõsködétek, és felmenétek a hegyre.
\par 44 De kijöve az Emoreus, a ki lakik vala azon a hegyen, ti ellenetek, és megkergetének titeket, mint a méhek szokták cselekedni, és vagdaltak vala titeket Szeirtõl Hormáig.
\par 45 És visszatérétek onnét, és sírátok az Úr elõtt, de nem hallgatá meg az Úr a ti szavatokat, és nem figyele rátok.
\par 46 És sok idõn át lakozátok Kádesben, a meddig ott lakozátok.

\chapter{2}

\par 1 Annakutána megfordulánk, és indulánk a pusztába a Veres tenger felé, a miképen szólott vala nékem az Úr, és kerülgettük a Szeir hegyét sok ideig.
\par 2 De szóla az Úr nékem, mondván:
\par 3 Elég már e hegyet kerülgetnetek, forduljatok észak felé.
\par 4 Parancsolj azért a népnek, mondván: Mikor általmentek a ti atyátokfiainak, az Ézsau fiainak határán, a kik Szeirben lakoznak: jóllehet félnek tõletek, mindazáltal igen vigyázzatok!
\par 5 Ne ingereljétek õket, mert nem adok az õ földjükbõl néktek egy talpalatnyit sem; mert Ézsaúnak adtam a Szeir hegyét örökségül.
\par 6 Pénzen vásároljatok tõlük enni valót, hogy egyetek, és vizet is pénzen vegyetek tõlük, hogy igyatok,
\par 7 Mert az Úr, a te Istened megáldott téged a te kezednek minden munkájában; tudja, hogy e nagy pusztaságon jársz; immár negyven esztendeje veled van az Úr, a te Istened; nem szûkölködtél semmiben.
\par 8 És általmenénk a mi atyánkfiai között, az Ézsaú fiai között, a kik lakoznak vala Szeirben, a síkság útján Eláthtól és Éczjon-Gebertõl fogva. Aztán megfordulánk és általmenénk a Moáb pusztájának útjára.
\par 9 És monda az Úr nékem: Ne hadakozzál Moáb ellen, és ne ingereld azt hadra, mert nem adok az õ földébõl néked semmi örökséget; mert a Lót fiainak adtam Art örökségül.
\par 10 (Az Emeusok laktak abban annak elõtte, nagy nép, sok és szálas, mint az Anákok.
\par 11 Óriásoknak állíttatnak vala azok is, mint az Anákok, és a Moábiták Emeknek hívták õket.
\par 12 Szeirben pedig Horeusok laktak az elõtt, a kiket az Ézsaú fiai kiûztek, és kiirtottak színök elõl, és azoknak helyén laktak, a miképen cselekedék Izráel is az õ örökségének földén, a melyet adott néki az Úr.)
\par 13 Most keljetek fel, és menjetek át a Záred patakán; és átkelénk a Záred patakán.
\par 14 Az idõ pedig, a melyet eljáránk Kádes-Barneától, míg általmenénk a Záred patakán, harmincznyolcz esztendõ, a mely alatt kiveszett a hadra való férfiak egész nemzetsége a táborból, a mint megesküdt vala az Úr nékik.
\par 15 E felett az Úrnak keze is vala õ rajtok, hogy elveszítse õket a táborból az õ kipusztulásukig.
\par 16 És lõn, hogy a mint a hadakozó férfiak mind elpusztulának, kihalván a nép közül.
\par 17 Így szóla az Úr nékem, mondván:
\par 18 Ma te általmégy a Moáb határán Ar felé,
\par 19 És mikor közel jutsz az Ammon fiaihoz, ne háborgasd õket, ne is ingereld õket, mert nem adok néked az Ammon fiainak földjébõl örökséget, mert a Lót fiainak adtam azt örökségül.
\par 20 (Óriások földének tartották azt is; óriások laktak azon régenten, a kiket az Ammoniták Zanzummoknak hívtak.
\par 21 Ez a nép nagy, sok és szálas volt, valamint az Anákok, de kivesztette õket az Úr azok színe elõl, hogy bírják azoknak örökségét, és lakjanak azoknak helyén;
\par 22 A miképen cselekedett az Ézsaú fiaival is, a kik Szeir hegyén laknak, a mikor kiveszté elõlök a Horeusokat, hogy bírják azoknak örökségét, és lakjanak azoknak helyén mind e mai napig.
\par 23 Az Avveusokat, a kik falvakban laknak vala Gázáig, kiirtották a Káftoreusok a kik kijöttek volt Káftorból, és lakának azoknak helyén.)
\par 24 Keljetek fel azért, induljatok, menjetek át az Arnon patakán; lásd: kezedbe adtam Szihont, Hesbonnak királyát: az Emoreust, és annak földét; kezdj hozzá, foglald el azt, és hadakozzál õ ellene.
\par 25 E napon kezdem rábocsátani a népekre, hogy féljenek és rettegjenek tõled az egész ég alatt, és a kik híredet hallják, rendüljenek meg és reszkessenek te elõtted.
\par 26 És követeket küldék a Kedemót pusztából Szihonhoz, Hesbon királyához békességes beszéddel ezt izenvén:
\par 27 Hadd menjek át a te földeden! Útról-útra megyek, nem térek le se jobbra, se balra.
\par 28 Eleséget pénzen adj nékem, hogy egyem; vizet is pénzen adj nékem, hogy igyam; csak gyalog hadd megyek át:
\par 29 A miképen cselekedtek én velem az Ézsaú fiai, a kik Szeirben laknak; és a Moábiták, a kik Arban laknak; míglen átmegyek a Jordánon arra a földre, a melyet az Úr, a mi Istenünk ád nékünk!
\par 30 De nem akarta Szihon, Hesbon királya, hogy átmenjünk õ rajta, mert megkeményítette volt az Úr, a te Istened az õ lelkét, és engedetlenné tette az Õ szívét, hogy a te kezedbe adja õt, a mint nyilván van e mai napon.
\par 31 Monda pedig az Úr nékem: Lásd: elkezdem átadni néked Szihont és az õ földét; kezdj hozzá, foglald el azt, hogy az õ földe örököd legyen.
\par 32 És kijöve Szihon mi elõnkbe minden népével, hogy megvívjon velünk Jahácznál.
\par 33 De az Úr, a mi Istenünk kezünkbe adá õt, és levertük õt és az õ fiait és minden õ népét.
\par 34 És elfoglaltuk minden õ városát abban az idõben, és fegyverre hánytuk az egész várost: férfiakat, asszonyokat, és kisdedeket; nem hagytunk menekülni senkit.
\par 35 De a barmokat prédára vetettük közöttünk, és a városokból való ragadományokat, a melyeket elfoglaltunk volt.
\par 36 Aróertõl fogva, a mely van az Arnon patakának partján, és a völgyben lévõ várostól fogva Gileádig, egy város sem volt, a melylyel ne bírtunk volna. Mind azokat az Úr, a mi Istenünk adta a mi kezünkbe.
\par 37 De az Ammon fiainak földéhez nem közeledtél, sem a Jabbók patak egész oldalához, sem a hegyen lévõ városokhoz, sem semmi olyanhoz, a melyektõl  eltiltott téged az Úr, a mi Istenünk.

\chapter{3}

\par 1 És megfordulánk, és felmenénk Básán felé, és kijöve elõnkbe Óg, Básánnak királya, õ és minden népe, hogy megvívjon velünk Edreiben.
\par 2 De az Úr monda nékem: Ne félj tõle, mert a te kezedbe adtam õt és minden õ népét és földjét; és úgy cselekedjél vele, a mint cselekedtél Szihonnal, az Emoreusok királyával, a ki Hesbonban lakik vala.
\par 3 És kezünkbe adá az Úr, a mi Istenünk Ógot is, Básánnak királyát és minden õ népét, és úgy megvertük õt, hogy menekülni való sem maradt belõle.
\par 4 És abban az idõben elfoglaltuk minden városát; nem volt város, a melyet el nem vettünk volna tõlök: hatvan várost, Argóbnak egész vidékét, a Básenbeli Ógnak országát.
\par 5 Ezek a városok mind meg valának erõsítve magas kõfalakkal, kapukkal és zárokkal, kivévén igen sok kerítetlen várost.
\par 6 És fegyverre hánytuk azokat, a mint cselekedtünk vala Szihonnal, Hesbon királyával, fegyverre hányván az egész várost: férfiakat, asszonyokat és a kisdedeket is.
\par 7 De a barmokat és a városokból való ragadományokat mind magunk közt vetettük prédára.
\par 8 És elvettük abban az idõben az Emoreusok két királyának kezébõl azt a földet, a mely a Jordánon túl vala, az Arnon pataktól fogva a Hermon hegyéig.
\par 9 (A Sidoniak a Hermont Szirjonnak, az Emoreusok pedig Szenirnek hívják.)
\par 10 A síkságnak minden városát, és az egész Gileádot, meg az egész Básánt Szalkáig és Edreig, a melyek a Básánbeli Óg országának városai voltak.
\par 11 Mert egyedül Óg, Básánnak királya maradt meg az óriások maradéka közül. Ímé az õ ágya vas-ágy, nemde Rabbátban az Ammon fiainál van-é? Kilencz sing a hosszasága és négy sing a szélessége, férfi könyök szerint.
\par 12 Ezt a földet pedig, a melyet abban az idõben örökségünkké tettünk, Aróertõl fogva, a mely az Arnon patak mellett van, és a Gileád hegyének felét, és annak városait odaadtam a Rúbenitáknak és Gáditáknak.
\par 13 A Gileád többi részét pedig, és az egész Básánt, az Óg országát odaadtam a Manassé fél törzsének, Argóbnak egész vidékét. Ezt az egész Básánt óriások földének hívták.
\par 14 Jair, Manassénak fia, kapta Argóbnak egész vidékét, a Gessuriták és Maakátiták határáig; és azokat a Básánnal együtt az õ nevérõl Jair faluinak hívják mind e mai napig.
\par 15 Mákirnak pedig adtam Gileádot.
\par 16 A Rúbenitáknak és a Gáditáknak pedig adtam Gileádtól fogva Arnonnak patakáig (a határ pedig a patak közepe) és a Jabbók patakáig, a mely az Ammon fiainak határa;
\par 17 És a síkságot és határul a Jordánt, a Kinnerettõl a Síkság tengeréig, a Sóstengerig, a mely a Piszga-hegy lába alatt van napkelet felõl.
\par 18 És parancsolék abban az idõben néktek, mondván: Az Úr, a ti Istenetek adta néktek ezt a földet, hogy bírjátok azt; felfegyverkezvén, menjetek át a ti atyátok fiai, Izráel fiai elõtt mind, a kik hadakozásra valók vagytok.
\par 19 Csak feleségeitek, kicsinyeitek és barmaitok (mert tudom, hogy sok barmotok van) maradjanak a ti városaitokban, a melyeket én adtam néktek.
\par 20 Mindaddig, a míg nyugodalmat ád az Úr a ti atyátokfiainak, mint néktek, és azok is bírálhatják a földet, a melyet az Úr, a ti Istenetek ád nékik a Jordánon túl. Azután térjetek vissza, kiki az õ örökségébe, a melyet adtam néktek.
\par 21 Józsuénak is parancsolék abban az idõben, mondván: Szemeiddel láttad mindazt, a mit cselekedett az Úr, a ti Istenetek ama két királylyal; így cselekszik az Úr minden országgal, a melyen átmégy.
\par 22 Ne féljetek tõlök, mert az Úr, a ti Istenetek, maga hadakozik ti érettetek!
\par 23 Könyörgék is az Úrnak abban az idõben, mondván:
\par 24 Uram, Isten, te elkezdetted megmutatni a te szolgádnak a te nagyságodat és hatalmas kezedet! Mert kicsoda olyan Isten mennyben és földön, a ki cselekedhetnék a te cselekedeteid és hatalmad szerint?
\par 25 Hadd menjek át kérlek, és hadd lássam meg azt a jó földet, a mely a Jordánon túl van, és azt a jó hegyet, és a Libanont!
\par 26 De megharaguvék az Úr én reám ti miattatok, és nem hallgatott meg engem; hanem ezt mondá az Úr nékem: Elég ez néked, ne szólj többet már nékem e dolog felõl!
\par 27 Menj fel a Piszga tetejére, és emeld fel a te szemeidet napnyugot felé és észak felé, dél felé és napkelet felé, és nézz szét a te szemeiddel, mert nem mégy át ezen a Jordánon.
\par 28 Józsuénak pedig parancsolj, és bátorítsd õt, és erõsítsd õt, mert õ megy át e nép elõtt, és õ teszi õket örököseivé annak a földnek, a melyet meglátsz.
\par 29 És ott maradánk a völgyben, Berth-Peórral szemben.

\chapter{4}

\par 1 Most pedig hallgass ó Izráel a rendelésekre és végzésekre, a melyekre én tanítlak titeket, hogy azok szerint cselekedjetek, hogy élhessetek, és bemehessetek, és bírhassátok a földet, a melyet az Úr, a ti atyáitoknak Istene ád néktek.
\par 2 Semmit se tegyetek az ígéhez, a melyet én parancsolok néktek, se el ne vegyetek abból, hogy megtarthassátok az Úrnak, a ti Isteneteknek parancsolatait, a melyeket én parancsolok néktek.
\par 3 Szemeitekkel láttátok, a mit cselekedett az Úr Baal-Peór miatt; hogy minden ember, a ki Baal-Peór után járt, kipusztított az Úr, a te Istened te közüled.
\par 4 Ti pedig, a kik ragaszkodtatok az Úrhoz, a ti Istenetekhez, mindnyájan éltek e napig.
\par 5 Lássátok, tanítottalak titeket rendelésekre és végzésekre, a mint megparancsolta nékem az Úr, az én Istenem, hogy azok szerint cselekedjetek azon a földön, a melybe bementek, hogy bírjátok azt.
\par 6 Megtartsátok azért és megcselekedjétek! Mert ez lesz a ti bölcseségtek és értelmetek a népek elõtt, a kik meghallják majd mind e rendeléseket, és ezt mondják: Bizony bölcs és értelmes nép ez a nagy nemzet!
\par 7 Mert melyik nagy nemzet az, a melyhez olyan közel volna az õ Istene, mint mi hozzánk az Úr, a mi Istenünk, valahányszor hozzá kiáltunk?
\par 8 És melyik nagy nemzet az, a melynek olyan rendelései és igazságos végzései volnának, mint ez az egész törvény, a melyet én ma adok elétek?!
\par 9 Csak vigyázz magadra, és õrizd jól a te lelkedet, hogy el ne felejtkezzél azokról, a melyeket láttak a te szemeid, és hogy el ne távozzanak a te szívedtõl teljes életedben, hanem ismertesd meg azokat a te fiaiddal és fiaidnak  fiaival.
\par 10 El ne felejtkezzél a napról, a melyen az Úr elõtt, a te Istened elõtt állottál a Hóreben, a mikor azt mondta nékem az Úr: Gyûjtsd egybe nékem a népet, hogy hallassam véle beszédeimet, hogy tanuljanak félni engem, minden idõben, a míg e földön élnek, és tanítsák meg fiaikat is.
\par 11 És elõjárulátok, és megállátok a hegy alatt; a hegy pedig tûzben ég vala mind az ég közepéig, mindamellett sötétség, köd és homályosság vala.
\par 12 És szóla az Úr néktek a tûz közepébõl. A szavak hangját ti is halljátok vala, de csak a hangot; alakot azonban nem láttok vala.
\par 13 És kijelenté néktek az õ szövetségét, a melyre nézve utasított titeket a tíz ige teljesítésére, és felírá azokat két kõtáblára.
\par 14 Engem is utasított az Úr abban az idõben, hogy tanítsalak meg titeket a rendelésekre és végzésekre, hogy azok szerint cselekedjetek azon a földön, a melyre átmentek, hogy bírjátok azt.
\par 15 Õrizzétek meg azért jól a ti lelketeket, mert semmi alakot nem láttatok akkor, a mikor a tûznek közepébõl szólott hozzátok az Úr a Hóreben;
\par 16 Hogy el ne vetemedjetek, és faragott képet, valamely bálványféle alakot ne csináljatok magatoknak, férfi vagy asszony képére;
\par 17 Képére valamely baromnak, a mely van a földön; képére valamely repdesõ madárnak, a mely röpköd a levegõben;
\par 18 Képére valamely földön csúszó-mászó állatnak; képére valamely halnak, a mely van a föld alatt lévõ vizekben.
\par 19 Se szemeidet fel ne emeld az égre, hogy meglásd a napot, a holdat és a csillagokat, az égnek minden seregét, hogy meg ne tántorodjál, és le ne borulj azok elõtt, és ne tiszteljed azokat, a melyeket az Úr, a te Istened minden néppel közlött, az egész ég alatt.
\par 20 Titeket pedig kézen fogott az Úr, és kihozott titeket a vas kemenczébõl, Égyiptomból, hogy legyetek néki örökös népe, miképen e mai napon vagytok.
\par 21 De én reám megharaguvék az Úr ti miattatok, és megesküvék, hogy nem megyek át a Jordánon, és hogy nem megyek be arra a jó földre, a melyet az Úr, a te Istened, ád néked örökségül.
\par 22 Miután én meghalok e földön, nem megyek át a Jordánon; ti pedig átmentek, és bírjátok azt a jó földet;
\par 23 Vigyázzatok, hogy az Úrnak, a ti Isteneteknek szövetségérõl, a melyet kötött veletek, el ne felejtkezzetek, és ne csináljatok magatoknak faragott képet, akármihez is hasonlót, a miképen megparancsolta az Úr, a te Istened.
\par 24 Mert az Úr, a te Istened emésztõ tûz, féltõn szeretõ Isten õ.
\par 25 Hogyha majd fiakat és unokákat nemzesz, és megvénhedtek azon a földön, és elvetemedtek, és csináltok magatoknak faragott képet, akármihez is hasonlót, és gonoszt cselekesztek az Úrnak, a te Istenednek szemei elõtt, haragra ingerelvén õt:
\par 26 Bizonyságul hívom ti ellenetek e mai napon a mennyet és a földet, hogy elveszvén hamarsággal elvesztek a földrõl, a melyre átmentek a Jordánon, hogy bírjátok azt; nem laktok sok ideig azon, hanem pusztára kipusztultok róla.
\par 27 És az Úr szétszór titeket a népek közé, és szám szerint kevesen maradtok meg a népek között, a kik közé visz titeket az Úr.
\par 28 És szolgáltok ott emberi kéz által csinált isteneknek: fának és kõnek, a melyek nem látnak, nem is hallanak, nem is esznek, nem is szagolnak.
\par 29 De ha onnan keresed meg az Urat, a te Istenedet, akkor is megtalálod, hogyha teljes szívedbõl és teljes lelkedbõl keresed õt.
\par 30 Mikor nyomorúságban leéndesz, és utólérnek téged mindezek az utolsó idõkben, és megtérsz az Úrhoz, a te Istenedhez, és hallgatsz az õ szavára:
\par 31 (Mert irgalmas Isten az Úr, a te Istened), nem hágy el téged, sem el nem veszít, sem el nem felejtkezik a te atyáidnak szövetségérõl, a mely felõl megesküdt nékik.
\par 32 Mert tudakozzál csak a régi idõkrõl, a melyek te elõtted voltak, ama naptól fogva, a melyen az Isten embert teremtett e földre, és pedig az égnek egyik szélétõl az égen másik széléig, ha történt-é e nagy dologhoz hasonló, vagy hallatszott-é ehhez fogható?
\par 33 Hallotta-é valamely nép a tûz közepébõl szóló Istennek szavát, a miképen hallottad te, hogy életben maradt volna?
\par 34 Avagy próbálta-é azt Isten, hogy elmenjen és válaszszon magának népet valamely nemzetség közül, kisértésekkel: jelekkel, csudákkal, haddal, hatalmas kézzel, kinyújtott karral, és nagy rettenetességek által, a miképen cselekedte mind ezeket ti érettetek az Úr, a ti Istenetek Égyiptomban, szemeitek láttára?
\par 35 Csak néked adatott láthatóan tudnod, hogy az Úr az Isten és nincsen kivüle több!
\par 36 Az égbõl hallatta veled az õ szavát, hogy tanítson téged, a földön pedig mutatta néked amaz õ nagy tüzét, és hallottad beszédét a tûz közepébõl.
\par 37 És mivel szerette a te atyáidat, és kiválasztotta az õ magvokat is õ utánok, és kihozott téged az õ orczájával Égyiptomból, az õ nagy erejével:
\par 38 Hogy kiûzzön náladnál nagyobb és erõsebb népeket elõled, hogy bevigyen téged, és adja néked az õ földjöket örökségül, mint a mai napon van:
\par 39 Tudd meg azért e mai napon, és vedd szívedre, hogy az Úr az Isten, fent a mennyben, és alant e földön, és nincsen több!
\par 40 Tartsd meg azért az õ rendeléseit és parancsolatait, a melyeket én parancsolok ma néked, hogy jól legyen dolgod és a te fiaidnak te utánad, és hogy mindenkor hosszú ideig élj azon a földön, a melyet az Úr, a te Istened ád néked.
\par 41 Akkor választa Mózes három várost a Jordánon túl napkelet felé;
\par 42 Hogy oda fusson a gyilkos, a ki nem akarva gyilkolta meg az õ felebarátját, a ki az elõtt nem gyûlölte vala, és hogy életben maradjon, ha befutott valamelyikbe e városok közül.
\par 43 Tudniillik Beczert a pusztában, a sík földön a Ruben fiainak, Rámótot Gileádban a Gád fiainak, és Gólánt Básánban a Manassé fiainak.
\par 44 Ez pedig a törvény, a melyet Mózes adott az Izráel fiai elé.
\par 45 Ezek a bizonyságtételek, a rendelések és a végzések, a melyeket szóla Mózes Izráel fiainak, mikor Égyiptomból kijöttek vala.
\par 46 A Jordánon túl a völgyben, Berth-Peórnak átellenében, Szihonnak, az Emoreusok királyának földjén, a ki lakozik vala Hesbonban, a kit megvert vala Mózes, az Izráel fiaival egyben, mikor Égyiptomból kijöttek vala.
\par 47 És elfoglalák az õ földét, és Ógnak, Básán királyának földét, az Emoreusok két királyáét, a kik a Jordánon túl laknak vala napkelet felõl.
\par 48 Aróertõl fogva, mely az Arnon patakának partján van, a Sion hegyéig, a mely a Hermon;
\par 49 És az egész síkságot a Jordánon túl napkelet felé, a síkság tengeréig, a Piszga hegy aljáig.

\chapter{5}

\par 1 És szólítá Mózes az egész Izráelt, és monda nékik: Hallgasd meg Izráel a rendeléseket és a végzéseket, a melyeket elmondok én ma fületek hallására, és tanuljátok meg azokat, és ügyeljetek azokra, megcselekedvén azokat!
\par 2 Az Úr, a mi Istenünk szövetséget kötött velünk a Hóreben.
\par 3 Nem a mi atyáinkkal kötötte az Úr e szövetséget, hanem mi velünk, a kik íme itt vagyunk e mai napon mindnyájan és élünk.
\par 4 Színrõl színre szólott veletek az Úr a hegyen, a tûz közepébõl.
\par 5 (Én pedig az Úr között és ti közöttetek állok vala abban az idõben, hogy megjelentsem néktek az Úr beszédét; mert ti a tûztõl féltek vala, és nem menétek fel a hegyre) mondván:
\par 6 Én, az Úr, vagyok a te Istened, a ki kihoztalak téged Égyiptomnak földébõl, a szolgálatnak házából.
\par 7 Ne legyenek néked idegen isteneid én elõttem.
\par 8 Ne csinálj magadnak faragott képet, és semmi hasonlót azokhoz, a melyek fenn az égben, vagy a melyek alant a földön, vagy a melyek a vizekben a föld alatt vannak.
\par 9 Ne imádd és ne tiszteld azokat; mert én, az Úr, a te Istened, féltõn szeretõ Isten vagyok, a ki megbüntetem az atyák vétkét a fiakban, harmad és negyedíziglen, a kik engem gyûlölnek;
\par 10 De irgalmasságot cselekeszem ezeríziglen azokkal, a kik engem szeretnek, és az én parancsolataimat megtartják.
\par 11 Az Úrnak, a te Istenednek nevét hiába fel ne vedd; mert nem hagyja azt az Úr büntetés nélkül, a ki az õ nevét hiába felveszi.
\par 12 Vigyázz a szombatnak napjára, hogy megszenteld azt, a miképen megparancsolta  néked az Úr, a te Istened.
\par 13 Hat napon át munkálkodjál és végezd minden dolgodat.
\par 14 De a hetedik nap az Úrnak, a te Istenednek szombatja: semmi dolgot se tégy azon, se magad, se fiad, se leányod, se szolgád, se szolgálóleányod, se ökröd, se szamarad, és semminémû barmod, se jövevényed, a ki a te kapuidon belõl van, hogy megnyugodjék a te szolgád és szolgálóleányod, mint te magad;
\par 15 És megemlékezzél róla, hogy szolga voltál Égyiptom földén, és kihozott onnan téged az Úr, a te Istened erõs kézzel és kinyújtott karral. Azért parancsolta néked az Úr, a te Istened, hogy a szombat napját megtartsad.
\par 16 Tiszteld atyádat és anyádat, a mint megparancsolta néked az Úr, a te Istened;  hogy hosszú ideig élj, és hogy jól legyen dolgod azon a földön, a melyet az Úr, a te Istened ád te néked.
\par 17 Ne ölj
\par 18 És ne paráználkodjál.
\par 19 És ne lopj.
\par 20 És ne tégy a te felebarátod ellen hamis tanubizonyságot.
\par 21 És ne kívánd a te felebarátodnak feleségét; és ne áhítsd a te felebarátodnak házát, szántóföldét; se szolgáját, se szolgálóleányát, se ökrét, se szamarát és semmit, a mi a te felebarátodé.
\par 22 Ez ígéket szólá az Úr a ti egész gyülekezeteteknek a hegyen a tûz, a felhõ és a homályosság közepébõl nagy felszóval, és nem többet; és felírá azokat két kõtáblára,  és adá azokat nékem.
\par 23 És lõn, mikor a szót a setétség közepébõl halljátok vala, és a hegy tûzzel ég vala, hozzám jövétek a ti törzseiteknek minden fejedelmével és vénjével;
\par 24 És mondátok: Ímé az Úr, a mi Istenünk megmutatta nékünk az õ dicsõségét és nagyságát; és az õ szavát hallottuk a tûznek közepébõl; e mai napon pedig láttuk, hogy az Isten emberrel szól, és ez mégis él.
\par 25 Most hát miért haljunk meg? Mert megemészt e nagy tûz minket. Ha még tovább halljuk az Úrnak, a mi Istenünknek szavát, meghalunk!
\par 26 Mert kicsoda az, az összes halandók közül, a ki a tûznek közepébõl szóló élõ Istennek szavát hallotta, mint mi, hogy megélt volna?
\par 27 Járulj oda te, és hallgasd meg mind azt, a mit mond az Úr, a mi Istenünk, és te majd beszéld el nékünk mind, a mit néked mond az Úr, a mi Istenünk, és mi meghallgatjuk, és megcselekeszszük.
\par 28 És meghallá az Úr a ti beszédetek szavát, a mikor beszéltek vala velem, és monda nékem az Úr: Hallottam e nép beszédének szavát, a mint beszéltek vala hozzád; mind jó, a mit beszéltek vala.
\par 29 Vajha így maradna az õ szívök, hogy félnének engem, és megtartanák minden parancsolatomat minden idõben, hogy jól legyen dolguk nékik és az õ gyermekeiknek mindörökké!
\par 30 Menj el, és mondd meg nékik: Térjetek vissza a ti sátoraitokba.
\par 31 Te pedig állj ide mellém, hogy elmondjam néked minden parancsolatomat, rendelésemet és végzésemet, a melyekre tanítsd meg õket, hogy cselekedjék azon a földön, a melyet én adok néktek örökségül.
\par 32 Vigyázzatok azért, hogy úgy cselekedjetek, a mint az Úr, a ti Istenetek parancsolta néktek; ne térjetek se jobbra, se balra.
\par 33 Mindig azon az úton járjatok, a melyet az Úr, a ti Istenetek parancsolt néktek, hogy éljetek, és jó legyen dolgotok, és hosszú ideig élhessetek a földön, a melyet bírni fogtok.

\chapter{6}

\par 1 Ezek pedig a parancsolatok, a rendelések és a végzések, a melyek felõl parancsolta az Úr, a ti Istenetek, hogy megtanítsam azokat néktek, hogy azok szerint cselekedjetek azon a földön, a melyre átmenõben vagytok, hogy bírjátok azt;
\par 2 Hogy féljed az Urat, a te Istenedet, és megtartsad minden õ rendelését és parancsolatát, a melyeket én parancsolok néked: te és a te fiad, és a te unokád, teljes életedben, és hogy hosszú ideig élhess.
\par 3 Halld meg azért Izráel, és vigyázz, hogy megcselekedjed, hogy jól legyen dolgod, és hogy igen megsokasodjál a tejjel és mézzel folyó földön, a miképen megigérte az Úr, a te atyáidnak Istene néked.
\par 4 Halld Izráel: az Úr, a mi Istenünk, egy Úr!
\par 5 Szeressed azért az Urat, a te Istenedet teljes szívedbõl, teljes lelkedbõl és teljes erõdbõl.
\par 6 És ez ígék, a melyeket e mai napon parancsolok néked, legyenek a te szívedben.
\par 7 És gyakoroljad ezekben a te fiaidat, és szólj ezekrõl, mikor a te házadban ülsz, vagy mikor úton jársz, és mikor lefekszel, és mikor felkelsz.
\par 8 És kössed azokat a te kezedre jegyül, és legyenek homlokkötõül a te szemeid között.
\par 9 És írd fel azokat a te házadnak ajtófeleire, és a te kapuidra.
\par 10 És mikor bevisz téged az Úr, a te Istened a földre, a mely felõl megesküdt a te atyáidnak, Ábrahámnak, Izsáknak és Jákóbnak, hogy ad néked nagy és szép városokat, a melyeket nem építettél;
\par 11 És minden jóval telt házakat, a melyeket nem te töltöttél meg, és ásott kutakat, a melyeket nem te ástál, szõlõ- és olajfakerteket, a melyeket nem te plántáltál; és eszel és megelégszel:
\par 12 Vigyázz magadra, hogy el ne felejtkezzél az Úrról, a ki kihozott téged Égyiptom földérõl, a szolgaságnak házából!
\par 13 Féljed az Urat, a te Istenedet, õ néki szolgálj, és az õ nevére  esküdjél.
\par 14 Ne járjatok idegen istenek után, azoknak a népeknek istenei közül, a kik körültetek vannak;
\par 15 (Mert az Úr, a te Istened féltõn szeretõ Isten te közötted), hogy az Úrnak, a te Istenednek haragja fel ne gerjedjen reád, és el ne törüljön téged a föld színérõl.
\par 16 Meg ne kísértsétek az Urat, a ti Isteneteket, miképen  megkísértettétek Maszszában!
\par 17 Szorosan megtartsátok az Úrnak, a ti Isteneteknek parancsolatait, az õ bizonyságait és rendeléseit, a melyeket parancsolt néked.
\par 18 És azt cselekedjed, a mi igaz és jó az Úrnak szemei elõtt, hogy jól legyen dolgod, és bemehess és bírhasd azt a jó földet, a mely felõl megesküdött az Úr a te atyáidnak.
\par 19 Hogy elûzzed minden ellenségedet a te színed elõl, a mint megmondotta az Úr.
\par 20 Ha a te fiad megkérdez téged ezután, mondván: Mire valók e bizonyságtételek, rendelések és végzések, a melyeket az Úr, a mi Istenünk parancsolt néktek:
\par 21 Akkor azt mondjad a te fiadnak: Szolgái voltunk Égyiptomban a Faraónak; de kihozott minket az Úr Égyiptomból hatalmas kézzel.
\par 22 És tõn az Úr Égyiptomban nagy és veszedelmes jeleket és csudákat a Faraón és az õ egész háznépén, a mi szemeink láttára.
\par 23 Minket pedig kihoza onnét, hogy bevigyen minket, és nékünk adja azt a földet, a mely felõl megesküdött a mi atyáinknak.
\par 24 És megparancsolta nékünk az Úr, hogy cselekedjünk mind e rendelések szerint, hogy féljük az Urat, a mi Istenünket, hogy jó dolgunk legyen teljes életünkben, hogy megtartson minket az életben, mint e mai napon.
\par 25 És ez lesz nékünk igazságunk, ha vigyázunk arra, hogy megtartsuk mind e parancsolatokat, az Úr elõtt, a mi Istenünk elõtt, a miképen megparancsolta nékünk.

\chapter{7}

\par 1 Mikor bevisz téged az Úr, a te Istened a földre, a melyre te bemenendõ vagy, hogy bírjad azt; és sok népet kiûz te elõled a Khitteust, a Girgazeust, az Emoreust, a Kananeust, a Perizeust, a Khivveust, és a Jebuzeust: hétféle népet, náladnál nagyobbakat és erõsebbeket;
\par 2 És adja õket az Úr, a te Istened a te hatalmadba, és megvered õket: mindenestõl veszítsd ki õket; ne köss velök szövetséget, és ne könyörülj rajtok.
\par 3 Sógorságot se szerezz õ velök, a leányodat se adjad az õ fioknak, és az õ leányukat se vegyed a te fiadnak;
\par 4 Mert elpártoltatja a te fiadat én tõlem, és idegen isteneknek szolgálnak; és felgerjed az Úrnak haragja reátok, és hamar kipusztít titeket.
\par 5 Hanem így cselekedjetek velök: Oltáraikat rontsátok le, oszlopaikat törjétek össze, berkeiket vágjátok ki, faragott képeiket pedig tûzzel égessétek meg.
\par 6 Mert az Úrnak, a te Istenednek szent népe vagy te; téged választott az Úr, a te Istened, hogy saját népe légy néki, minden nép közül e föld színén.
\par 7 Nem azért szeretett titeket az Úr, sem nem azért választott titeket, hogy minden népnél többen volnátok; mert ti minden népnél kevesebben vagytok;
\par 8 Hanem mivel szeretett titeket az Úr, és hogy megtartsa az esküt, a melylyel megesküdt volt a ti atyáitoknak; azért hozott ki titeket az Úr hatalmas kézzel, és szabadított meg téged a szolgaságnak házából, az égyiptombeli Faraó királynak kezébõl.
\par 9 És hogy megtudjad, hogy az Úr, a te Istened, õ az Isten, a hívséges Isten, a ki megtartja a szövetséget és az irgalmasságot ezeríziglen azok iránt, a kik õt szeretik, és az õ parancsolatait megtartják.
\par 10 De megfizet azoknak személy szerint, a kik õt gyûlölik, elvesztvén õket; nem késlekedik az ellen, a ki gyûlöli õt, megfizet annak személy szerint.
\par 11 Tartsd meg azért a parancsolatot, a rendeléseket és végzéseket, a melyeket én e mai napon parancsolok néked, hogy azokat cselekedjed.
\par 12 Ha pedig engedelmeskedtek e végzéseknek, és megtartjátok, és teljesítitek azokat: az Úr, a te Istened is megtartja néked a szövetséget és irgalmasságot, a mely felõl megesküdött a te atyáidnak.
\par 13 És szeretni fog téged, és megáld téged, és megsokasít téged; és megáldja a te méhednek gyümölcsét, a te földednek gyümölcsét: gabonádat, mustodat és olajodat; teheneid fajzását és juhaidnak ellését azon a földön, a mely felõl megesküdt a te atyáidnak, hogy néked adja azt.
\par 14 Áldottabb lészesz minden népnél; nem lészen közötted magtalan férfi és asszony, sem barmaid között meddõ.
\par 15 És távol tart az Úr te tõled minden betegséget, és Égyiptomnak minden gonosz nyavalyáját, a melyeket ismersz; nem veti azokat te reád, hanem mind azokra, a kik gyûlölnek téged.
\par 16 És megemészted mind a népeket, a melyeket néked ád az Úr, a te Istened; ne kedvezzen a te szemed nékik, és ne tiszteld az õ isteneit; mert tõr gyanánt volna az néked.
\par 17 Ha azt mondod a te szívedben: Többen vannak e népek, mint én, miképen ûzhetem én ki õket?
\par 18 Ne félj tõlök; emlékezzél meg csak azokról, a miket cselekedett az Úr, a te Istened a Faraóval és mind az égyiptombeliekkel:
\par 19 A nagy kisértésekrõl, a melyeket láttak a te szemeid, és a jelekrõl és csudákról; az erõs  kézrõl, és a kinyujtott karról, a melylyel kihozott téged az Úr, a te Istened! Így cselekeszik az Úr, a te Istened minden néppel, a melytõl te félsz.
\par 20 Sõt még a darázsokat is rájok bocsátja az Úr, a te Istened mind addig, míglen elvesznek azok is, a kik megmaradtak, és a kik elrejtõztek te elõled.
\par 21 Ne rettenj meg azok elõtt, mert közötted van az Úr, a te Istened, nagy és rettenetes Isten!
\par 22 És lassan-lassan kiûzi az Úr, a te Istened e népeket te elõled. Nem lehet õket hirtelen kipusztítanod, hogy a mezei vadak meg ne sokasodjanak ellened!
\par 23 De az Úr, a te Istened elõdbe veti õket, és nagy romlással rontja meg õket, míglen elvesznek.
\par 24 Az õ királyaikat is kezedbe adja, hogy eltöröljed az õ nevöket az ég alól; senki ellened nem állhat, míglen elveszted õket.
\par 25 Az õ isteneiknek faragott képeit tûzzel égesd meg; az azokon lévõ ezüstöt és aranyat meg ne kívánd, és magadnak el ne vedd, hogy tõrbe ne essél miatta; mert útálatosság az az Úr elõtt, a te Istened elõtt.
\par 26 Útálatosságot pedig ne vígy be a te házadba, hogy átokká ne légy, mint az, hanem megvetvén vesd meg azt, és útálván útáld meg azt, mert átkozott.

\chapter{8}

\par 1 Mind azt a parancsolatot, a melyet én e mai napon parancsolok néked, tartsátok meg és teljesítsétek, hogy élhessetek és megsokasodhassatok, bemehessetek és bírhassátok a földet, a mely felõl megesküdött az Úr a ti atyáitoknak.
\par 2 És emlékezzél meg az egész útról, a melyen hordozott téged az Úr, a te Istened immár negyven esztendeig a pusztában, hogy megsanyargasson és megpróbáljon  téged, hogy nyilvánvaló legyen, mi van a te szívedben; vajjon megtartod-é az õ parancsolatait vagy nem?
\par 3 És megsanyargata téged, és megéheztete, azután pedig enned adá a mannát, a melyet nem ismertél, sem a te atyáid nem ismertek, hogy tudtodra adja néked, hogy az ember nem csak kenyérrel  él, hanem mind azzal él az ember, a mi az Úrnak szájából származik.
\par 4 A te ruházatod le nem kopott rólad, sem a te lábad meg nem  dagadott immár negyven esztendõtõl fogva.
\par 5 Gondold meg azért a te szívedben, hogy a miképen megfenyíti az ember az õ gyermekét, úgy fenyít meg téged az Úr, a te Istened;
\par 6 És õrizd meg az Úrnak, a te Istenednek parancsolatait, hogy az õ útján járj, és õt féljed.
\par 7 Mert az Úr, a te Istened jó földre visz be téged; bõvizû patakoknak, forrásoknak és mély vizeknek földére, a melyek a völgyekben és a hegyeken fakadnak.
\par 8 Búza-, árpa-, szõlõtõ- fige- és gránátalma-termõ földre, faolaj- és méz-termõ földre.
\par 9 Oly földre, a melyen nem nyomorogva eszed kenyeredet, és a hol semmiben sem szûkölködöl; oly földre, a melynek kövei vas, és a melynek hegyeibõl rezet vághatsz!
\par 10 Ha azért eszel majd és megelégszel: dícsérjed az Urat, a te Istenedet azért a jó földért, a melyet néked adott.
\par 11 Vigyázz magadra, hogy el ne felejtkezzél az Úrról, a te Istenedrõl, meg nem tartván az õ parancsolatait, végzéseit, rendeléseit, a melyeket én parancsolok néked e mai napon;
\par 12 Hogy mikor eszel és jól lakol, és szép házakat építesz, és lakozol azokban;
\par 13 És mikor a te barmaid és juhaid megsokasodnak, és ezüstöd és aranyad is megsokasodik, és minden jószágod megszaporodik:
\par 14 Fel ne fuvalkodjék akkor a te szíved, és el ne felejtkezzél az Úrról, a te Istenedrõl, a ki kihozott téged Égyiptom földébõl, a szolgaságnak házából;
\par 15 A ki vezérlett téged a tüzes kígyóknak, skorpióknak, és szomjúságnak nagy és rettenetes pusztáján, a melyben víz nem vala; a ki vizet ada néked a kemény kõsziklából;
\par 16 A ki mannával étete téged a pusztában, a mit nem ismertek a te atyáid, hogy megsanyargasson és hogy megpróbáljon  téged, és jól tegyen veled azután:
\par 17 És ne mondjad ezt a te szívedben: Az én hatalmam, és az én kezemnek ereje szerzette nékem e gazdagságot!
\par 18 Hanem emlékezzél meg az Úrról, a te Istenedrõl, mert õ az, a ki erõt ád néked a gazdagságnak megszerzésére, hogy megerõsítse az õ szövetségét, a mely felõl megesküdt a te atyáidnak, miképen e mai napon van.
\par 19 Ha pedig teljesen megfelejtkezel az Úrról, a te Istenedrõl, és idegen istenek után jársz, és azoknak szolgálsz; és meghajtod magadat azoknak; bizonyságot tészek e mai napon ti ellenetek, hogy végképen elvesztek.
\par 20 Mint azok a nemzetek, a kiket az Úr elveszt elõletek, azonképen vesztek el; azért mert nem hallgattok az Úrnak, a ti Isteneteknek szavára.

\chapter{9}

\par 1 Halljad Izráel, te általmégy ma a Jordánon, hogy bemenvén, örökségül bírj náladnál nagyobb és erõsebb népeket, nagy és az égig megerõsített városokat;
\par 2 Nagy és szálas népet, Anák-fiakat, a kikrõl magad is tudod és magad is hallottad: kicsoda állhat meg Anák fiai elõtt?
\par 3 Tudd meg azért e mai napon, hogy az Úr, a te Istened az, a ki átmegy elõtted mint emésztõ tûz, õ törli el azokat, és õ alázza meg azokat te elõtted; és kiûzöd, és hamar elveszted õket, a miképen az Úr megmondotta néked.
\par 4 Mikor azért kiûzi az Úr, a te Istened azokat te elõled, ne szólj a te szívedben, mondván: Az én igazságomért hozott be engem az Úr, hogy örökségül bírjam ezt a földet; holott e népeket az õ  istentelenségökért ûzi ki te elõled az Úr;
\par 5 Nem a te igazságodért, sem a te szívednek igaz voltáért mégy te be az õ földük bírására; hanem az Úr, a te Istened e népeknek istentelenségéért ûzi ki õket elõled, hogy megerõsítse az ígéretet, a mely felõl megesküdt az Úr a te atyáidnak: Ábrahámnak, Izsáknak és Jákóbnak.
\par 6 Tudd meg azért, hogy az Úr, a te Istened nem a te igazságodért adja néked ezt a jó földet birtokul, mert kemény nyakú nép vagy te!
\par 7 Emlékezzél meg róla, és el ne felejtsed azokat, a mikkel haragra indítottad az Urat, a te Istenedet a pusztában! A naptól fogva, a melyen kijöttél Égyiptom földébõl, mind addig, míglen e helyre jutottatok, az Úr ellen tusakodtatok vala.
\par 8 Már a Hóreben haragra indítátok az Urat, és annyira megharaguvék reátok az Úr, hogy el akara veszteni titeket.
\par 9 Mikor felmegyek vala a hegyre, hogy átvegyem a kõtáblákat, a szövetségnek tábláit, a melyet az Úr kötött vala veletek, és a hegyen maradtam vala negyven nap és negyven éjjel: kenyeret nem ettem, sem vizet nem ittam vala.
\par 10 Akkor átadá nékem az Úr a két kõtáblát, a melyek az Isten ujjával valának beírva, és rajtok valának mind amaz ígék, a melyeket mondott vala az Úr néktek a  hegyen a tûz közepébõl, a gyülekezésnek napján.
\par 11 És mikor a negyven nap és negyven éj elmultával átadá az Úr nékem a két kõtáblát, a szövetségnek tábláit;
\par 12 Akkor monda az Úr nékem: Kelj fel, hamar menj innen alá; mert elvetemedett a te néped, a kit kihoztál Égyiptomból; hamar eltértek az útról, a melyet parancsoltam vala nékik, öntött bálványt készítettek magoknak.
\par 13 Ismét szóla nékem az Úr, mondván: Láttam e népet, és bizony kemény nyakú nép ez!
\par 14 Hagyj békét nékem, hadd pusztítsam el õket, és töröljem el az õ nevöket az ég alól, és teszlek téged ennél nagyobb és erõsebb néppé!
\par 15 Megfordulék azért, és alájövék a hegyrõl, a hegy pedig tûzzel ég vala, és a szövetségnek két táblája az én két kezemben vala.
\par 16 És mikor látám, hogy ímé vétkeztetek vala az Úr ellen, a ti Istenetek ellen; öntött borjút csináltatok vala magatoknak; hamar letértetek vala az útról, a melyet az Úr parancsolt vala néktek:
\par 17 Akkor megragadám a két táblát, és elhajítám a két kezembõl, és összetörém azokat a ti szemeitek láttára.
\par 18 És leborulék az Úr elõtt, mint annakelõtte, negyven nap és negyven éjjel, kenyeret nem ettem és vizet sem ittam; minden ti bûnötökért, a melyeket elkövettetek vala, azt cselekedvén, a mi gonosz az Úr elõtt, hogy ingereljétek õt.
\par 19 Mert félek vala a haragtól és búsulástól, a melylyel ti reátok úgy megharagudt vala az Úr, hogy el akart vala pusztítani titeket. És meghallgata az Úr engem akkor is.
\par 20 Áronra is igen megharagudt vala az Úr, és el akará õt is pusztítani; de ugyanakkor imádkozám Áronért is.
\par 21 A ti bûnötöket pedig, a borjút, a melyet készítettetek, megragadám, és megégetém azt tûzzel; és összetörém azt, jól megõrölvén, mígnem porrá morzsolódék, azután bevetém annak porát a patakba, a mely a hegyrõl foly vala alá.
\par 22 És Thaberában, Massában,  és Kibrot-Taavában is haragra indítátok az Urat.
\par 23 És mikor az Úr elküldött vala titeket Kádes-Barneából, mondván: Menjetek fel, és bírjátok örökségül a földet, a melyet néktek adtam: akkor is tusakodtatok vala az Úrnak, a ti Isteneteknek beszéde ellen, nem hittetek néki, és nem hallgattatok az õ szavára.
\par 24 Tusakodók voltatok az Úr ellen, a mióta ismerlek titeket.
\par 25 És leborulék az Úr elõtt azon a negyven napon és negyven éjjel, a melyeken leborultam vala; mert azt mondotta vala az Úr, hogy elveszt titeket.
\par 26 Akkor imádkozám az Úrhoz, és mondék: Uram, Isten! ne rontsd meg a te népedet, és a te örökségedet, a kit a te nagyságoddal szabadítottál meg, a kit erõs kézzel hoztál ki Égyiptomból.
\par 27 Emlékezzél meg a te szolgáidról: Ábrahámról, Izsákról és Jákóbról; ne nézzed e népnek keménységét, istentelenségét és bûnét!
\par 28 Hogy ne mondja a föld népe, a honnét kihoztál minket: mivelhogy az Úr nem vihette be õket a földre, a melyet igért volt nékik, és mivelhogy gyûlölte õket, azért hozta ki õket, hogy megölje õket a pusztában.
\par 29 Pedig õk a te néped és a te örökséged, a melyet kihoztál a te nagy erõddel, és a te kinyujtott karoddal!

\chapter{10}

\par 1 Abban az idõben monda az Úr nékem: Faragj magadnak két kõtáblát, az elõbbiekhez hasonlókat, és jõjj fel hozzám a hegyre, és csinálj  faládát.
\par 2 És felírom a táblákra azokat az ígéket, a melyek az elõbbi táblákon valának, a melyeket széttörtél; és tedd azokat a ládába.
\par 3 Csinálék azért ládát sittim-fából, és faragék két kõtáblát is, az elõbbiekhez hasonlókat; és felmenék a hegyre, és a két kõtábla kezemben vala.
\par 4 És felírá a táblákra az elõbbi írás szerint a tíz ígét, a melyeket szólott vala az Úr ti hozzátok a hegyen, a tûznek közepébõl a gyülekezésnek napján, és átadá az Úr azokat nékem.
\par 5 Akkor megfordulék és alájövék a hegyrõl, és betevém a táblákat a ládába, a melyet csináltam vala, hogy ott legyenek, a miképen az Úr parancsolta vala nékem.
\par 6 Izráel fiai pedig elindulának Beeróthból, a mely a Jákán fiaié, Moszérába. Ott  halt meg Áron, és ugyanott el is temetteték, és Eleázár, az õ fia lõn pappá helyette.
\par 7 Innét indulának Gudgódba, Gudgódból pedig Jotbatába, bõvízû patakok földére.
\par 8 Abban az idõben választá ki az Úr a Lévi törzsét, hogy hordozza az Úr szövetségének ládáját, és hogy az Úr elõtt álljon, és néki szolgáljon, és hogy áldjon az õ nevében mind e napig.
\par 9 Ezért nem volt része és öröksége a Lévinek az õ atyafiaival; az Úr az õ öröksége, a miképen megmondotta vala néki az Úr, a te Istened.
\par 10 Én pedig ott állottam a hegyen, mint az elõbbi napokban, negyven nap és negyven éjjel: és meghallgata engem az Úr akkor is, és nem akara téged elveszteni az Úr.
\par 11 És monda az Úr nékem: Kelj fel, menj és járj a nép elõtt, hogy bemenjenek és bírják a földet, a mely felõl megesküdtem az õ atyáiknak, hogy nékik adom.
\par 12 Most pedig, óh Izráel! mit kíván az Úr, a te Istened tõled? Csak azt, hogy féljed az Urat, a te Istenedet; hogy minden õ utain járj, és szeresd õt, és tiszteljed az Urat, a te Istenedet teljes szívedbõl, és teljes lelkedbõl,
\par 13 Megtartván az Úrnak parancsolatait és rendeléseit, a melyeket én ma parancsolok néked, hogy jól legyen dolgod!
\par 14 Ímé az Úréi, a te Istenedéi az egek, és az egeknek egei, a föld, és minden, a mi rajta van!
\par 15 De egyedül a ti atyáitokat kedvelte az Úr, hogy szeresse õket, és az õ magvokat: titeket választott ki õ utánok minden nép közül, a mint e mai napon is látszik.
\par 16 Metéljétek azért körül a ti szíveteket, és ne legyetek  ezután keménynyakúak;
\par 17 Mert az Úr, a ti Istenetek, isteneknek Istene, és uraknak Ura; nagy, hatalmas és rettenetes Isten, a ki nem személyválogató, sem ajándékot el nem fogad.
\par 18 Igazságot szolgáltat az árvának és az özvegynek; szereti a jövevényt, adván néki kenyeret és ruházatot.
\par 19 Szeressétek azért a jövevényt; mert ti is jövevények voltatok Égyiptom földén.
\par 20 Az Urat, a te Istenedet féljed, õt tiszteljed, õ hozzá ragaszkodjál, és az õ nevére esküdjél.
\par 21 Õ a te dícséreted, és a te Istened, a ki azokat a nagy és rettenetes dolgokat cselekedte veled, a melyeket láttak a te szemeid.
\par 22 A te atyáid hetvenen mentek vala alá Égyiptomba; most pedig az Úr, a te Istened megsokasított téged, mint az égnek csillagait!

\chapter{11}

\par 1 Szeresd azért az Urat, a te Istenedet, tartsd meg az õ megtartani valóit: rendeléseit, végzéseit és parancsolatait minden idõben.
\par 2 És tudjátok meg ma (mert nem a ti fiaitokkal szólok, a kik nem tudják és nem látták) az Úrnak, a te Isteneteknek fenyítését, nagyságát, erõs kezét és kinyújtott karját,
\par 3 Jeleit és cselekedeteit, a melyeket Égyiptomban cselekedett a Faraóval, az égyiptombeliek királyával, és az õ egész földével;
\par 4 És a melyeket cselekedett az égyiptombeliek seregével, lovaival és szekereivel, mivelhogy reájok árasztá a Veres tenger vizeit, mikor üldözének titeket, és elveszté õket az Úr mind e mai napig;
\par 5 És a melyeket cselekedett veletek a pusztában, a míg e helyre jutátok;
\par 6 És a melyeket cselekedett Dáthánnal és Abirámmal, Eliábnak a Rúben fiának fiaival, mikor a föld megnyitá az õ száját, és elnyelé õket háznépeikkel, sátoraikkal, és minden marhájokkal egyetemben, a mely az övék vala, az egész Izráel között.
\par 7 Mert saját szemeitekkel láttátok az Úrnak minden nagy cselekedetét, a melyeket cselekedett.
\par 8 Tartsátok meg azért mind a parancsolatokat, a melyeket én ma parancsolok néked, hogy megerõsödjetek, bemenjetek és bírjátok a földet, a melyre átmentek, hogy bírjátok azt.
\par 9 És hogy sok ideig élhessetek a földön, a mely felõl megesküdt az Úr a ti atyáitoknak, hogy nékik és az õ magvoknak adja azt a tejjel és mézzel folyó földet.
\par 10 Mert a föld, a melyre te bemégy, hogy bírjad azt, nem olyan az, mint Égyiptomnak földe, a honnan kijöttetek; a melyben elveted vala a te magodat és a te lábaddal kell vala megöntöznöd, mint egy veteményes kertet;
\par 11 Hanem az a föld, a melyre átmentek, hogy bírjátok azt, hegyes-völgyes föld, az égnek esõjébõl iszik vizet.
\par 12 Oly föld az, a melyre az Úr, a te Istened visel gondot; mindenkor rajta függenek az Úrnak, a te Istenednek szemei az esztendõ kezdetétõl az esztendõ végéig.
\par 13 Lészen azért, hogyha valóban engedelmeskedtek az én parancsolataimnak, a melyeket én ma parancsolok néktek, úgy hogy az Urat, a ti Istenteket szeretitek, és néki szolgáltok teljes szívetekbõl és teljes lelketekbõl:
\par 14 Esõt adok a ti földetekre alkalmatos idõben: korai és kései esõt, hogy betakaríthasd a te gabonádat, borodat és olajodat;
\par 15 Füvet is adok a te mezõdre a te barmaidnak; te pedig eszel és megelégszel.
\par 16 Vigyázzatok azért, hogy a ti szívetek meg ne csalattassék, és el ne térjetek, és ne tiszteljetek idegen isteneket, és ne boruljatok le elõttök.
\par 17 Különben az Úrnak haragja felgerjed reátok, és bezárja az eget, hogy esõ ne legyen, és a föld az õ gyümölcsét meg ne teremje; és hamarsággal elvesztek a jó földrõl, a melyet az Úr ád néktek.
\par 18 Vegyétek azért szívetekre és lelketekre e szavaimat, és kössétek azokat jegyül a ti kezetekre, és homlokkötõkül legyenek a ti szemeitek között;
\par 19 És tanítsátok meg azokra a ti fiaitokat, szólván azokról, mikor házadban ülsz, mikor úton jársz, mikor fekszel és mikor felkelsz.
\par 20 És írd fel azokat a te házadnak ajtófeleire és a te kapuidra;
\par 21 Hogy megsokasodjanak a ti napjaitok és fiaitoknak napjai azon a földön, a mely felõl megesküdt az Úr a ti atyáitoknak, hogy nékik adja mindaddig, a míg az ég a föld felett lészen.
\par 22 Mert ha szorosan megtartjátok mind e parancsolatot, a melyet én parancsolok néktek, hogy a szerint cselekedjetek; ha szeretitek az Urat, a ti Istenteket, ha minden õ útain jártok, és õ hozzá ragaszkodtok:
\par 23 Akkor kiûzi az Úr mind azokat a nemzeteket ti elõletek, és úrrá lesztek nálatoknál nagyobb és erõsebb nemzeteken.
\par 24 Minden hely, a melyet lábatok talpa megnyom, tiétek lesz, a pusztától a Libanonig, és a folyóvíztõl, az Eufrátes folyóvizétõl a nyugoti tengerig lesz a ti határotok.
\par 25 Nem állhat meg senki elõttetek; azt míveli az Úr, a ti Istenetek, hogy féljenek és rettegjenek titeket az egész föld színén, a melyre rátapostok, a mint megmondotta néktek.
\par 26 Lásd, én adok ma elõtökbe áldást és átkot!
\par 27 Az áldást, ha engedelmeskedtek az Úrnak, a ti Istenetek parancsolatainak, a melyeket én e mai napon parancsolok néktek;
\par 28 Az átkot pedig, ha nem engedelmeskedtek az Úrnak, a ti Istenetek parancsolatainak, és letértek az útról, a melyet én ma parancsolok néktek, és idegen istenek után jártok, a kiket nem ismertetek.
\par 29 És mikor bevisz téged az Úr, a te Istened arra a földre, a melyre te bemégy, hogy bírjad azt: akkor mondd el az áldást a Garizim hegyén, az átkot pedig az Ebál hegyén.
\par 30 Nemde azok túl vannak a Jordánon, a napnyugoti út megett, a Kananeusok földén, a kik a síkságon laknak Gilgálnak átellenében, a Móré tölgyei mellett?!
\par 31 Mert ti átmentek a Jordánon, hogy bemenvén bírhassátok a földet, a melyet az Úr, a ti Istenetek ád néktek. Ha bírni fogjátok azt, és lakni fogtok abban:
\par 32 Vigyázzatok, hogy mind e rendelések és végzések szerint cselekedjetek, a melyeket én ma adok ti elõtökbe.

\chapter{12}

\par 1 Ezek a rendelések és a végzések, a melyeket meg kell tartanotok, azok szerint cselekedvén azon a földön, a melyet az Úr, a te atyáidnak Istene ád néked, hogy bírjad azt minden idõben a míg éltek a földön:
\par 2 Pusztára pusztítsátok el mind azokat a helyeket, a hol azok a nemzetek, a kiknek ti urai lesztek, szolgáltak az õ isteneiknek a magas hegyeken, a halmokon, és minden zöldelõ fa alatt.
\par 3 És rontsátok el azoknak oltárait, törjétek össze oszlopaikat, tûzzel égessétek meg berkeiket, és vagdaljátok szét az õ isteneiknek faragott képeit, a nevöket is pusztítsátok ki arról a helyrõl.
\par 4 Ne cselekedjetek így az Úrral, a ti Istenetekkel;
\par 5 Hanem azt a helyet, a melyet kiválaszt az Úr, a ti Istenetek minden ti törzsetek közül, hogy az õ nevét oda helyezze és ott lakozzék, azt gyakoroljátok és oda menjetek.
\par 6 És oda vigyétek egészen égõáldozataitokat, véres áldozataitokat, tizedeiteket, kezeiteknek felemelt áldozatát, mind fogadásból, mind szabad akaratból való adományaitokat, és a ti barmaitoknak és juhaitoknak elsõ fajzását.
\par 7 És ott egyetek az Úrnak, a ti Isteneteknek színe elõtt, és örvendezzetek ti és a ti házatok minden népe, kezetek minden keresményének, a melyekkel megáld téged az Úr, a te Istened.
\par 8 Ne cselekedjetek ott mind a szerint, a mint mi cselekszünk itt most, mindenik azt, a mi jónak látszik néki.
\par 9 Mert még ez ideig nem mentetek be a nyugodalmas helyre, és az örökségbe, a melyet az Úr, a te Istened ád néked.
\par 10 Mikor pedig átmentek a Jordánon, és lakozni fogtok azon a földön, a melyet az Úr, a ti Istenetek ád néktek örökségül, és megnyugtat titeket minden ellenségetektõl, a kik körületek vannak, és bátorsággal fogtok lakozni:
\par 11 Akkor arra a helyre, a melyet kiválaszt az Úr, a ti Istenetek, hogy ott lakozzék az õ neve, oda vigyetek mindent, a mit én parancsolok néktek: egészen égõáldozataitokat, véres áldozataitokat, tizedeiteket és kezeiteknek felemelt áldozatát, és minden megkülönböztetett fogadástokat, a melyeket fogadtok az Úrnak.
\par 12 És örvendezzetek az Úrnak, a ti Isteneteknek színe elõtt, mind ti, mind a ti fiaitok, és leányaitok, mind a ti szolgáitok és szolgálóleányaitok, mind a lévita, a ki a ti kapuitokon belõl lészen; mert nincs néki része vagy  öröksége ti veletek.
\par 13 Vigyázz, hogy a te egészen égõáldozataidat ne áldozzad minden helyen, a melyet meglátsz;
\par 14 Hanem azon a helyen, a melyet kiválaszt az Úr a te törzseid közül valamelyikben: ott áldozzad a te egészen égõáldozatidat, és ott cselekedjél mindent, a mit én parancsolok néked.
\par 15 Mindazáltal a te lelkednek teljes kívánsága szerint vághatsz barmot és ehetel húst, minden te kapuidon belõl, az Úrnak, a te Istenednek áldásához képest, a melyet ád néked; mind a tisztátalan, mind a tiszta eheti azt, ép úgy mint az õzet és a szarvast.
\par 16 Csakhogy a vért meg ne egyétek; földre öntsd azt, mint a vizet.
\par 17 Nem eheted meg a te kapuidon belõl sem gabonádnak, sem mustodnak, sem olajodnak tizedét, sem barmaidnak és juhaidnak elsõ fajzását; sem semmi fogadási áldozatodat, a melyet fogadsz, sem szabad akarat szerint való adományaidat, sem a te kezednek felemelt áldozatát;
\par 18 Hanem az Úrnak a te Istenednek színe elõtt egyed azokat azon a helyen, a melyet kiválaszt az Úr a te Istened: te és a te fiad, leányod, szolgád, szolgálóleányod és a lévita, a ki a te kapuidon belõl van; és örvendezzél az Úrnak, a te Istenednek színe elõtt mindenben, a mire kezedet veted.
\par 19 Vigyázz, hogy el ne hagyjad a lévitát valameddig élsz a te földeden.
\par 20 Mikor az Úr, a te Istened kiszélesíti a te határodat, a miképen igérte vala néked, és ezt mondod: Húst ehetném! mivelhogy a te lelked húst kíván enni: egyél húst a te lelkednek teljes kívánsága szerint.
\par 21 Ha messze van tõled a hely, a melyet az Úr, a te Istened választ, hogy oda helyezze az õ nevét, és leölsz, a mint parancsoltam néked, a te barmaidból és juhaidból, a melyeket az Úr ád majd néked: akkor egyél azokból a te kapuidban a te lelkednek teljes kívánsága szerint.
\par 22 De, a mint az õzet és a szarvast eszik, úgy egyed azokat; a tisztátalan és a tiszta egyaránt ehetik abból.
\par 23 Csakhogy abban állhatatos légy, hogy a vért meg ne egyed; mert a vér, az a lélek: azért a lelket a hússal együtt meg ne egyed!
\par 24 Meg ne egyed azt, a földre öntsd azt, mint a vizet.
\par 25 Meg ne egyed azt, hogy jól legyen dolgod néked és a te gyermekeidnek is utánad, mivelhogy azt cselekszed, a mi igaz az Úr szemei elõtt.
\par 26 De ha valamit megszentelsz a tiéidbõl, vagy fogadást teszesz: vedd fel, és vidd azt arra a helyre, a melyet kiválaszt az Úr.
\par 27 És az Úrnak, a te Istenednek oltárán áldozd meg a te egészen égõáldozataidat, azoknak húsát és vérét; egyéb áldozataidnak vérét azonban öntsd az Úrnak, a te Istenednek oltárára, a húsát pedig megeheted.
\par 28 Vigyázz és hallgasd meg mindezeket az ígéket, a melyeket én parancsolok néked, hogy jól legyen dolgod, néked és a te gyermekeidnek utánad mind örökké; mivelhogy azt cselekszed, a mi jó és igaz az Úrnak, a te Istenednek szemei elõtt.
\par 29 Mikor kiirtja elõled az Úr, a te Istened a nemzeteket, a kikhez bemégy, hogy bírjad õket, és bírni fogod õket, és lakozol majd az õ földükön:
\par 30 Vigyázz magadra, hogy õket követvén tõrbe ne essél, miután már kivesztek elõled; és ne tudakozzál az õ isteneik felõl, mondván: Miképen tisztelik e nemzetek az õ isteneiket? én is akképen cselekszem.
\par 31 Ne cselekedjél így az Úrral, a te Isteneddel, mert mind azt az útálatosságot, a mit gyûlöl az Úr, megcseledték az õ isteneikkel; mert még fiaikat és leányaikat is megégetik vala tûzzel az õ isteneiknek.
\par 32 Mindazt, a mit én parancsolok néktek, megtartsátok, és a szerint cselekedjetek: semmit ne tégy ahhoz, és el se végy abból!

\chapter{13}

\par 1 Mikor te közötted jövendõmondó, vagy álomlátó támad és jelt vagy csodát ád néked;
\par 2 Ha bekövetkezik is az a jel vagy a csoda, a melyrõl szólott vala néked, mondván: Kövessünk idegen isteneket, a kiket te nem ismersz, és tiszteljük azokat:
\par 3 Ne hallgass efféle jövendõmondónak beszédeire, vagy az efféle álomlátóra; mert az Úr, a ti Istenetek kísért titeket, hogy megtudja, ha szeretitek-é az Urat, a ti Isteneteket teljes szívetekbõl, és teljes lelketekbõl?
\par 4 Az Urat, a ti Isteneteket kövessétek, és õt féljétek, és az õ parancsolatait tartsátok meg, és az õ szavára hallgassatok, õt tiszteljétek, és õ hozzá ragaszkodjatok.
\par 5 Az a jövendõmondó pedig vagy álomlátó ölettessék meg; mert pártütést hirdetett az Úr ellen, a ti Istenetek ellen, a ki kihozott titeket Égyiptom földébõl, és megszabadított téged a szolgaságnak házából; hogy elfordítson téged arról az útról, a melyet parancsolt néked az Úr, a te Istened, hogy azon járj. Gyomláld ki azért a gonoszt magad közül.
\par 6 Ha testvéred, a te anyádnak fia, vagy a te fiad vagy leányod, vagy a kebleden lévõ feleség, vagy lelki-testi barátod titkon csalogat, mondván: Nosza menjünk és tiszteljünk idegen isteneket, a kiket nem ismertetek sem te, sem atyáid.
\par 7 Ama népek istenei közül, a kik körültetek vannak, közel hozzád vagy távol tõled, a földnek egyik végétõl a másik végéig;
\par 8 Ne engedj néki, és ne hallgass reá, ne nézz reá szánalommal, ne kíméld és ne rejtegesd õt;
\par 9 Hanem megölvén megöljed õt; a te kezed legyen elõször rajta az õ megölésére, azután pedig az egész népnek keze.
\par 10 Kövekkel kövezd meg õt, hogy meghaljon; mert azon igyekezett, hogy elfordítson téged az Úrtól, a te Istenedtõl, a ki kihozott téged Égyiptomnak földébõl, a szolgaságnak házából;
\par 11 És hallja meg az egész Izráel, és féljenek és ne cselekedjenek többé ahhoz hasonló gonoszt te közötted.
\par 12 Ha valamelyikben a te városaid közül, a melyeket az Úr, a te Istened ád néked, hogy ott lakjál, ezt hallod mondani:
\par 13 Emberek jöttek ki közüled, istentelenségnek fiai, és elfordítják városuk lakosait, mondván: Nosza, menjünk és tiszteljünk idegen isteneket, a kiket nem ismertetek:
\par 14 Akkor keress, kutass és szorgalmatosan tudakozódjál, és ha igaz, és bizonyos a dolog, és megtörtént az efféle útálatosság közötted:
\par 15 Hányd kard élére annak a városnak lakosait; áldozd fel azt mindenestõl, a mi benne van; a barmát is kard élére hányd.
\par 16 És a mi préda van benne, hordd mind együvé az õ piaczának közepére, és égesd meg a várost tûzzel és annak minden prédáját is teljes áldozatul az Úrnak, a te Istenednek, és legyen rom mind örökké, és soha többé fel ne építtessék.
\par 17 Ne ragadjon semmi a te kezedhez abból az átokra valóból, hogy szûnjék meg az Úr haragjának búsulása, és könyörületes legyen hozzád, és könyörüljön rajtad, és megszaporítson téged, a miképen megesküdt a te atyáidnak,
\par 18 Ha hallgatsz az Úrnak, a te Istenednek szavára, megtartván minden õ parancsolatát, a melyeket én ma parancsolok néked, hogy azt cselekedjed, a mi igaz az Úrnak, a te Istenednek szemei elõtt.

\chapter{14}

\par 1 Ti az Úrnak, a ti Isteneteknek fiai vagytok; ne  vagdaljátok meg magatokat, se szemeitek között ne csináljatok kopaszságot, a halottért,
\par 2 Mert szent népe vagy te az Úrnak, a te Istenednek, és az Úr választott téged, hogy légy néki tulajdon népe minden nép közül, a melyek a föld színén vannak.
\par 3 Semmi útálatosságot meg ne egyél.
\par 4 Ezek azok az állatok, a melyeket megehettek: az ökör, juh, kecske,
\par 5 Szarvas, õz, bival, vadkecske, zerge, vad bika és jávor.
\par 6 És mindazt az állatot, a melynek hasadt a körme és egészen ketté hasadt körme van, és kérõdzõ az állatok között, megehetitek.
\par 7 De a kérõdzõk és hasadt körmûek közül ne egyétek meg ezeket: a tevét, a nyulat és hörcsököt, mert kérõdznek ugyan, de körmük nem hasadt; tisztátalanok legyenek ezek néktek.
\par 8 És a disznót, mert hasadt ugyan a körme, de nem kérõdzik; tisztátalan legyen ez néktek. Ezeknek húsából ne egyetek, holttestöket se illessétek.
\par 9 Ezeket ehetitek meg mindazokból, melyek vízben élnek: a minek úszó szárnya és pikkelye van, mind megegyétek;
\par 10 Valaminek pedig nincsen úszó szárnya és pikkelye, meg ne egyétek; tisztátalan az néktek.
\par 11 Minden tiszta madarat megehettek.
\par 12 Ezek pedig, a melyeket meg ne egyetek közülök: a sas, a saskeselyû és a halászó sas.
\par 13 A keselyû, a héja és a sólyom az õ nemével.
\par 14 Minden holló az õ nemével.
\par 15 A strucz, a bagoly, a kakuk és a karvaly az õ nemével.
\par 16 A kuvik, a fülesbagoly és a bölömbika.
\par 17 A pelikán, a gém és a hattyú.
\par 18 Az eszterág és a szarka az õ nemével; a büdösbanka és a denevér.
\par 19 Minden szárnyas féreg is tisztátalan legyen néktek; meg ne egyétek.
\par 20 Minden tiszta szárnyast megehettek.
\par 21 Semmi holttestet meg ne egyetek; a jövevénynek, a ki a te kapuidon belõl van, adjad azt, hogy egye meg azt, vagy add el az idegennek, mert szent népe vagy te az Úrnak, a te Istenednek. Ne fõzd a  gödölyét az õ anyja tejében.
\par 22 Esztendõrõl esztendõre tizedet végy a te magodnak minden termésébõl, a mely a te mezõdön terem.
\par 23 És egyed az Úrnak, a te Istenednek színe elõtt azon, a helyen, a melyet kiválaszt, hogy ott lakozzék az õ neve, gabonádnak, mustodnak, olajodnak tizedét, a te barmaidnak és juhaidnak elsõ fajzását; hogy tanuljad félni az Urat, a te Istenedet minden idõben.
\par 24 Ha pedig hosszabb néked az út, hogysem oda vihetnéd azokat, mivelhogy távol esik tõled az a hely, a melyet az Úr, a te Istened választ, hogy oda helyezze az õ nevét, téged pedig megáldott az Úr, a te Istened:
\par 25 Akkor add el pénzen, és kösd a kezedbe a pénzt, és menj el arra a helyre, a melyet kiválaszt az Úr, a te Istened;
\par 26 És adjad a pénzt mind azért, a mit kíván a te lelked: ökrökért, juhokért, borért és vidámító italért és mindenért, a mit megáhít néked a te lelked, és egyél ott az Úrnak, a te Istenednek színe elõtt, és örvendezzél te és  a te házadnak népe.
\par 27 A lévitát pedig, a ki a te kapuidon belõl van, ne hagyd el, mert nincsen néki része, sem öröksége veled.
\par 28 A harmadik esztendõ végén vidd ki annak az esztendõ termésének minden tizedét, és rakd le a te kapuidba.
\par 29 És eljön a lévita (a kinek nincsen része és öröksége te veled), és a jövevény, árva és özvegy, a kik a te kapuidon belõl vannak, és esznek és megelégesznek, hogy megáldjon téged az Úr, a te Istened a te kezednek minden munkájában, a melyet végzesz.

\chapter{15}

\par 1 A hetedik esztendõ végén elengedést mívelj.
\par 2 Ez pedig az elengedésnek módja: Minden kölcsönadó ember engedje el, a mit kölcsönadott az õ felebarátjának; ne hajtsa be az õ felebarátján és atyjafián; mert elengedés hirdettetett az Úrért.
\par 3 Az idegenen hajtsd be, de a mid a te atyádfiánál lesz, engedje el néki a te kezed.
\par 4 De nem is lesz közötted szegény, mert igen megáld téged az Úr azon a földön, a melyet az Úr, a te Istened ád néked örökségül, hogy bírjad azt.
\par 5 De csak úgy lesz ez, ha hallgatsz az Úrnak, a te Istenednek szavára, megtartván és teljesítvén mind azt a parancsolatot, a melyet én ma parancsolok néked.
\par 6 Mert az Úr, a te Istened megáld téged, a miképen megmondotta néked; és sok népnek adsz zálogos kölcsönt, te pedig nem kérsz kölcsönt, és sok népen fogsz uralkodni, és te rajtad nem uralkodnak.
\par 7 Ha mégis szegénynyé lesz valaki a te atyádfiai közül valamelyikben a te kapuid közül a te földeden, a melyet az Úr, a te Istened ád néked: ne keményítsd meg a te szívedet, be se zárjad kezedet a te szegény atyádfia elõtt;
\par 8 Hanem örömest nyisd meg a te kezedet néki, és örömest adj kölcsön néki, a mennyi elég az õ szükségére, a mi nélkül szûkölködik.
\par 9 Vigyázz magadra, hogy ne legyen a te szívedben valami istentelenség, mondván: Közelget a hetedik esztendõ, az elengedésnek esztendeje; és elfordítsd szemedet a te szegény atyádfiától, hogy ne adj néki; mert õ ellened kiált az Úrhoz, és bûn lesz benned.
\par 10 Bizonyára adj néki, és meg ne háborodjék azon a te szíved, mikor adsz néki; mert az ilyen dologért áld meg téged az Úr, a te Istened minden munkádban, és mindenben, a mire kezedet veted.
\par 11 Mert a szegény nem fogy ki a földrõl, azért én parancsolom néked, mondván: Örömest nyisd meg kezedet a te szûkölködõ és  szegény atyádfiának a te földeden.
\par 12 Hogyha eladja magát néked a te atyádfia, a zsidó férfi és zsidó asszony, és szolgál téged hat esztendeig: a hetedik esztendõben bocsássad õt szabadon mellõled.
\par 13 És mikor szabadon bocsátod õt mellõled, ne bocsásd el õt üresen;
\par 14 Hanem terheld meg õt bõven a te juhaidból, a te szérûdrõl, és a te sajtódból; a mivel megáldott téged az Úr, a te Istened, adj néki abból.
\par 15 És emlékezzél meg róla, hogy te is szolga voltál Égyiptom földén, és megszabadított téged az Úr, a te Istened; azért parancsolom én ma ezt néked.
\par 16 Ha pedig ezt mondja néked: Nem megyek el tõled, mert szeret téged és a te házadat, mivelhogy jól van néki te nálad dolga:
\par 17 Akkor vedd az árat, és fúrd a fülébe és az ajtóba; és legyen szolgáddá mindvégig; így cselekedjél szolgálóleányoddal is.
\par 18 Ne essék nehezedre, hogy szabadon bocsátod õt mellõled; (hiszen két annyi bérre valót szolgált néked hat éven át, mint a béres-munkás) és megáld téged az Úr, a te Istened mindenben, a mit cselekszel.
\par 19 Barmaid és juhaid elsõ fajzásának minden hímjét az Úrnak, a te Istenednek szenteljed. Ne munkálkodjál a te tehenednek elsõ fajzásán, és meg ne nyírjad a te juhaidnak elsõ fajzását.
\par 20 Az Úrnak, a te Istenednek színe elõtt edd meg azt esztendõrõl esztendõre, te és a te házad népe, azon a helyen, a melyet kiválaszt az Úr.
\par 21 Hogyha valami fogyatkozás lesz benne; sánta vagy vak lesz, vagy akármely fogyatkozásban szenvedõ: meg ne áldozd azt az Úrnak, a te Istenednek.
\par 22 A te kapuidon belõl edd meg azt; a tisztátalan és a tiszta egyaránt, mintha õz volna az vagy szarvas.
\par 23 Csakhogy a vérét meg ne edd, hanem a földre öntsd azt, mint a vizet.

\chapter{16}

\par 1 Ügyelj az Abib hónapra, és készíts az Úrnak, a te Istenednek páskhát; mert az Abib hónapban hozott ki téged az Úr, a te Istened Égyiptomból éjjel.
\par 2 Páskha gyanánt pedig ölj az Úrnak, a te Istenednek juhot és ökröt azon a helyen, a melyet kiválaszt az Úr, a te Istened, hogy oda helyezze az õ nevét.
\par 3 Ne egyél azzal semmi kovászost, hanem hét napon át egyél azzal kovásztalan lepényeket, nyomorúságnak kenyerét, (mert sietséggel jöttél ki Égyiptom földérõl) hogy megemlékezzél arról a napról életednek minden idejében, a melyen kijöttél Égyiptom földérõl.
\par 4 És ne lásson senki kovászt hét napon át sehol a te határodban; és a húsból, a melyet az elsõ napon estve megáldozol, semmi ne maradjon reggelig.
\par 5 Nem ölheted le a páskhát akármelyikben a te városaid közül, a melyeket az Úr, a te Istened ád néked;
\par 6 Hanem azon a helyen, a melyet kiválaszt az Úr, a te Istened, hogy az õ nevét oda helyezze: ott öld le a páskhát estve, napnyugtakor, abban az idõben, a mikor kijöttél Égyiptomból.
\par 7 Azon a helyen süsd és edd is meg, a melyet kiválaszt az Úr, a te Istened; reggel pedig fordulj vissza és menj haza a te hajlékodba.
\par 8 Hat napon át egyél kovásztalant; hetednapon pedig az Úrnak, a te Istenednek berekesztõ ünnepe lévén, ne munkálkodjál azon.
\par 9 Számlálj azután magadnak hét hetet; attól fogva kezdjed számlálni a hét hetet, hogy sarlódat a vetésbe bocsátod.
\par 10 És tarts hetek ünnepét az Úrnak, a te Istenednek, a te kezednek szabad akarat szerint való adományával, a melyet ahhoz képest adj, a mint megáldott téged az Úr, a te Istened.
\par 11 És örvendezz az Úrnak, a te Istenednek színe elõtt, te és a te fiad, és leányod, szolgád, szolgálóleányod, és a lévita, a ki a te kapuidon belõl van, és a jövevény, az árva és az özvegy, a kik te közötted vannak, azon a helyen, a melyet kiválasztott az Úr, a te Istened, hogy oda helyezze az õ nevét.
\par 12 És emlékezzél meg róla, hogy te is szolga voltál Égyiptomban; és tartsd meg, és teljesítsd e rendeléseket.
\par 13 A sátorok ünnepét hét napig tartsd, mikor begyûjtöd a termést a te szérûdrõl és sajtódról.
\par 14 És örvendezz a te ünnepeden, te és a te fiad, a te leányod, szolgád és szolgálóleányod, a lévita, a jövevény, az árva és az özvegy, akik belõl vannak a te kapuidon.
\par 15 Hét napig ünnepelj az Úrnak, a te Istenednek azon a helyen, a melyet kiválaszt az Úr, mert megáld téged az Úr, a te Istened minden termésedben, és kezeidnek minden munkájában; azért  örvendezz igen!
\par 16 Minden esztendõben háromszor jelenjen meg közüled minden férfiú az Úrnak, a te Istenednek színe elõtt azon a helyen, a melyet kiválaszt; a kovásztalan kenyerek ünnepén, a hetek ünnepén, és a sátorok ünnepén. De üres kézzel  senki se jelenjen meg az Úr elõtt!
\par 17 Kiki az õ képessége szerint adjon, az Úrnak, a te Istenednek áldása szerint, a melyet ád néked.
\par 18 Bírákat és felügyelõket állíts minden kapudba, a melyeket az Úr, a te Istened ád néked, a te törzseid szerint, hogy ítéljék a népet igaz ítélettel.
\par 19 El ne fordítsd az itéletet; személyt se válogass; ajándékot  se végy; mert az ajándék megvakítja a bölcsek szemeit, és elfordítja az igazak beszédét.
\par 20 Igazságot, igazságot kövess, hogy élhess és örökségül bírhasd azt a földet, a melyet az Úr, a te Istened ád néked.
\par 21 Ne plántálj magadnak berket semmiféle fából, az Úrnak, a te Istenednek oltára mellé, a melyet készítesz magadnak.
\par 22 Oszlopot se emelj magadnak, a mit gyûlöl az Úr, a te Istened.

\chapter{17}

\par 1 Ne áldozzál az Úrnak, a te Istenednek olyan ökröt és juhot, a melyen valami fogyatkozás van, akármi hiba; mert útálatosság az az Úrnak, a te Istenednek.
\par 2 Hogyha találtatik közötted valamelyikben a te kapuid közül, a melyeketek az Úr, a te Istened ád néked, vagy férfiú vagy asszony, a ki gonoszt cselekszik az Úrnak, a te Istenednek szemei elõtt, megszegvén az õ szövetségét;
\par 3 És elmegy és szolgál idegen istenek és imádja azokat, akár a napot, akár a holdat, vagy akármelyet az égnek seregei közül, a melyet nem parancsoltam;
\par 4 És megjelentetik néked, és meghallod: jól megtudakozd; és hogyha igaz, és bizonyos a dolog, és megtörtént ez az útálatosság Izráelben:
\par 5 Akkor vidd ki azt a férfiút vagy azt az asszonyt, a ki azt a gonoszságot mívelte, a te kapuidba, (a férfiút vagy az asszonyt) és kövezd agyon õket, hogy meghaljanak.
\par 6 Két tanú vagy három tanú szavára halállal lakoljon a halálra való; de egy tanú szavára meg ne haljon.
\par 7 A tanúk keze legyen elsõ rajta, hogy megölettessék, és azután mind az egész nép keze. Így tisztítsd ki magad közül a gonoszt.
\par 8 Ha megfoghatatlan valami elõtted, a mikor ítélned kell vér és vér között, ügy és ügy között, sérelem és sérelem között, vagy egyéb versengések között a te kapuidban: akkor kelj fel, és menj el arra a helyre, a melyet kiválaszt az Úr, a te Istened.
\par 9 És menj be a Lévita-papokhoz és a bíróhoz, a ki lesz majd abban az idõben; és kérdezd meg õket és õk tudtul adják néked az ítéletmondást.
\par 10 És annak a mondásnak értelme szerint cselekedjél, a melyet tudtul adnak néked azon a helyen, a melyet kiválaszt az Úr; és vigyázz, hogy mind a szerint cselekedjél, a mint tanítanak téged.
\par 11 A törvény szerint cselekedjél, a melyre tanítanak téged, és az ítélet szerint, a melyet mondanak néked; el ne hajolj attól a mondástól, a melyet tudtul adnak néked, se jobbra, se balra.
\par 12 Ha pedig elbizakodottságból azt cselekszi valaki, hogy nem hallgat a papra, a ki ott áll, szolgálván az Urat, a te Istenedet, vagy a bíróra: haljon meg az ilyen ember. Így tisztísd ki a gonoszt Izráelbõl.
\par 13 És mind az egész nép hallja, és féljen, hogy elbizakodottan senki ne cselekedjék többé.
\par 14 Mikor bemégy arra a földre, a melyet az Úr, a te Istened ád néked, és bírod azt, és lakozol abban, és ezt mondod: Királyt teszek magam fölé, miképen egyéb népek, a melyek körültem vannak:
\par 15 Azt emeld magad fölé királyul, a kit az Úr, a te Istened választ. A te atyádfiai közül emelj magad fölé királyt; nem tehetsz magad fölé idegent, a ki nem atyádfia.
\par 16 Csak sok lovat ne tartson, és a népet vissza ne vigye Égyiptomba, hogy lovat sokasítson, mivelhogy az Úr megmondta néktek: Ne térjetek többé vissza azon az úton!
\par 17 Sok feleséget se tartson, hogy a szíve el ne hajoljon; se ezüstjét, se aranyát felettébb meg ne sokasítsa.
\par 18 És mikor az õ országának királyi székére ül, írja le magának könyvbe e törvénynek mását a Lévita-papokéból.
\par 19 És legyen az õ nála, és olvassa azt életének minden idejében, hogy tanulja félni az Urat, a te Istenedet, hogy megtartsa e törvénynek minden ígéjét, és e rendeléseket, hogy azokat cselekedje.
\par 20 Fel ne fuvalkodjék az õ szíve az õ atyjafiai ellen, se el ne hajoljon a parancsolattól jobbra vagy balra, hogy hosszú ideig éljen az õ országában õ és az õ fiai Izráelben.

\chapter{18}

\par 1 A Lévita-papoknak, a Lévi egész nemzetségének ne legyen se része, se öröksége Izráellel, hanem éljenek az Úrnak tüzes áldozatjaiból és örökségébõl.
\par 2 Annakokáért ne legyen néki öröksége az õ atyjafiai között: Az Úr az õ öröksége, a mint megmondotta néki.
\par 3 És ez legyen a papoknak törvényes része a néptõl, azoktól, a kik áldoznak akár ökörrel, akár juhval, hogy a papnak adják a lapoczkát, a két állat és a gyomrot.
\par 4 A te gabonád, mustod és olajod zsengéjét, és a te juhaid gyapjának zsengéjét néki adjad;
\par 5 Mert õt választotta ki az Úr, a te Istened minden te nemzetséged közül, hogy álljon szolgálatra az Úrnak nevében, õ és az õ fiai minden idõben.
\par 6 Mikor pedig eljön a Lévita valamelyikbõl a te egész Izráelben lévõ kapuid közül, a hol õ lakik és bemegy az õ lelkének teljes kívánsága szerint arra a helyre, a melyet kiválaszt az Úr:
\par 7 Szolgáljon az Úrnak az õ Istenének nevében, mint az õ többi atyjafiai, a Léviták, a kik ott állanak az Úr elõtt.
\par 8 Az eledelekben egyenlõképen részesedjenek, kivéve azt, a mit eladott valaki az õ atyai örökségébõl.
\par 9 Mikor te bemégy arra a földre, a melyet az Úr, a te Istened ád néked: ne tanulj cselekedni azoknak a népeknek útálatosságai szerint.
\par 10 Ne találtassék te közötted, a ki az õ fiát vagy leányát átvigye a , tûzön se  jövendõmondó, se igézõ, se jegymagyarázó, se varázsló;
\par 11 Se bûbájos, se ördöngõsöktõl tudakozó, se titok-fejtõ, se halottidézõ;
\par 12 Mert mind útálja az Úr, a ki ezeket míveli, és ez ilyen útálatosságokért ûzi ki õket az Úr, a te Istened te elõled.
\par 13 Tökéletes légy az Úrral, a te Isteneddel.
\par 14 Mert ezek a nemzetek, a kiket te elûzesz, igézõkre és jövendõmondókra hallgatnak; de tenéked nem engedett ilyet az Úr, a te Istened.
\par 15 Prófétát támaszt néked az Úr, a te Istened te közüled, a te atyádfiai közül, olyat mint én: azt hallgassátok!
\par 16 Mind a szerint, amint kérted az Úrtól, a te Istenedtõl a Hóreben a gyülekezésnek napján mondván: Ne halljam többé az Úrnak, az én Istenemnek szavát, és ne lássam többé ezt a nagy tüzet, hogy meg ne haljak!
\par 17 Az Úr pedig monda nékem: Jól mondták a mit mondtak.
\par 18 Prófétát támasztok nékik az õ atyjokfiai közül, olyat mint te, és az én ígéimet  adom annak szájába, és megmond nékik mindent, a mit parancsolok néki.
\par 19 És ha valaki nem hallgat az én ígéimre, a melyeket az én nevemben szól, én megkeresem azon!
\par 20 De az a próféta, a ki olyat mer szólani az én nevemben, a mit én nem parancsoltam néki szólani, és a ki idegen istenek nevében szól: haljon meg az a próféta.
\par 21 Ha pedig azt mondod a te szívedben: miképen ismerhetjük meg az ígét, a melyet nem mondott az Úr?
\par 22 Ha a próféta az Úr nevében szól, és nem lesz meg, és nem teljesedik be a dolog: ez az a szó, a melyet nem az Úr szólott; elbizakodottságból mondotta azt a próféta; ne félj attól!

\chapter{19}

\par 1 Mikor kiírtja az Úr, a te Istened a nemzeteket, a kiknek földjét néked adja az Úr, a te Istened, és bírni fogod õket, és lakozol az õ városaikban és az õ házaikban:
\par 2 Válaszsz ki magadnak három várost a te földeden, a melyet az Úr, a te Istened ád néked, hogy örököljed azt.
\par 3 Készítsd meg magadnak az utat azokhoz, és oszszad három részre a te országod határát, a melyet az Úr, a te Istened ád néked örökségül; azok arra valók legyenek, hogy minden gyilkos oda meneküljön.
\par 4 Ez pedig a gyilkos törvénye, a ki oda menekül, hogy élve maradjon: A ki nem szándékosan öli meg az õ felebarátját, és nem gyûlöli vala azt azelõtt;
\par 5 A ki például elmegy az õ felebarátjával az erdõre fát vágni, és meglódul a keze a fejszével, hogy levágja a fát, és leesik a vas a nyelérõl, és úgy találja az õ felebarátját, hogy az meghal: az ilyen meneküljön e városok egyikébe, hogy élve maradjon.
\par 6 Különben a vérbosszuló rokon ûzõbe veszi az õ szívének búsulásában, és eléri, ha az út hosszú leend, és agyon üti õt, holott nem méltó a halálra, mivel azelõtt nem gyûlölte azt.
\par 7 Azért én parancsolom néked, mondván: Három várost válaszsz magadnak.
\par 8 Ha pedig az Úr, a te Istened kiterjeszti a te határodat, a mint megesküdt a te atyáidnak, és néked adja mind az egész földet, a melynek megadását megígérte volt a te atyáidnak;
\par 9 (Hogyha megtartod mind e parancsolatot, megtévén azt, a mit én ma parancsolok néked, tudniillik, hogy szeressed az Urat, a te Istenedet, és járj az õ utain minden idõben): akkor e háromhoz szerezz még három várost.
\par 10 Hogy ártatlan vér ne ontassék ki a te földeden, a melyet az Úr, a te Istened ád néked örökségül, és hogy a vér ne legyen rajtad.
\par 11 De hogyha lesz valaki, a ki gyûlöli az õ felebarátját, és meglesi azt, és reá támad és úgy üti meg, hogy meghal, és bemenekül valamelyikbe e városok közül:
\par 12 Akkor az õ városának vénei küldjenek embereket, és vonják ki azt onnét, és adják azt a vérbosszúló rokon kezébe, hogy meghaljon.
\par 13 Ne nézz reá szánalommal, hanem tisztítsd ki az ártatlan vérontást Izráelbõl, hogy jól legyen dolgod.
\par 14 A te felebarátodnak határát el ne told, a mely határt az õsök vetettek a te örökségedben, a melyet örökölni fogsz azon a földön, a melyet az Úr, a te Istened ád néked, hogy bírjad azt.
\par 15 Ne álljon elõ egy tanú senki ellen semmiféle hamisság és semmiféle bûn miatt; akármilyen bûnben bûnös valaki, két tanú szavára vagy három tanú szavára álljon a dolog.
\par 16 Ha valaki ellen gonosz tanú áll elõ, hogy pártütéssel vádolja õt:
\par 17 Akkor álljon az a két ember, a kiknek ilyen perök van, az Úr elé, a papok és a bírák elé, a kik abban az idõben lesznek;
\par 18 És a bírák vizsgálják meg jól a dolgot, és ha hazug tanú lesz a tanú, a ki hazugságot szólott az õ atyjafia ellen:
\par 19 Úgy cselekedjetek azzal, a mint õ szándékozott cselekedni az õ atyjafiával. Így tisztítsd ki közüled a gonoszt;
\par 20 Hogy a kik megmaradnak, hallják meg, és féljenek, és többször ne cselekedjenek te közötted ilyen gonosz dolgot.
\par 21 Ne nézz reá szánalommal; lelket lélekért, szemet szemért, fogat fogért, kezet kézért, lábat lábért.

\chapter{20}

\par 1 Mikor hadba mégy ellenséged ellen, és látsz lovakat, szekereket, náladnál nagyobb számú népet: ne félj tõlök, mert veled van az Úr, a te Istened, a ki felhozott téged Égyiptom földérõl.
\par 2 És mikor az ütközethez készültök, álljon elõ a pap, és szóljon a népnek;
\par 3 És ezt mondja nékik: Hallgasd meg Izráel! Ti ma készültök megütközni ellenségeitekkel: A ti szívetek meg ne lágyuljon, ne féljetek, és meg ne rettenjetek, se meg ne rémüljetek elõttök;
\par 4 Mert az Úr, a ti Istenetek veletek megy, hogy harczoljon érettetek a ti ellenségeitekkel, hogy megtartson titeket.
\par 5 Az elõljárók pedig szóljanak a népnek, mondván: Kicsoda az olyan férfi, a ki új házat épített, de még fel nem avatta azt? Menjen el, és térjen vissza az õ házába, hogy meg ne haljon a harczban, és más valaki avassa fel azt.
\par 6 És kicsoda olyan férfi, a ki szõlõt ültetett és nem vette el annak hasznát? Menjen el, és térjen vissza az õ házába, hogy meg ne haljon a harczban, és más valaki vegye el annak hasznát.
\par 7 És kicsoda olyan férfi, a ki feleséget jegyzett el magának, de még el nem vette? Menjen el, és térjen vissza házába, hogy meg ne haljon a harczban, és más valaki vegye azt el.
\par 8 Még tovább is szóljanak az elõljárók a néphez, és ezt mondják: Kicsoda olyan férfi, a ki félénk és lágy szívû? Menjen el, és térjen vissza az õ házába, hogy az õ atyjafiainak szíve úgy meg ne olvadjon, mint az õ szíve.
\par 9 És mikor elvégzik az elõljárók beszédöket a néphez, állítsanak seregvezéreket a nép élére.
\par 10 Mikor valamely város alá mégy, hogy azt megostromold, békességgel kínáld meg azt.
\par 11 És ha békességgel felel néked, és kaput nyit, akkor az egész nép, a mely találtatik abban, adófizetõd legyen, és szolgáljon néked.
\par 12 Ha pedig nem köt békességet veled, hanem harczra kél veled, akkor zárd azt körül;
\par 13 És ha az Úr, a te Istened kezedbe adja azt: vágj le abban minden finemût fegyver élével;
\par 14 De az asszonyokat, a kis gyermekeket, a barmokat és mind azt a mi lesz a városban, az egész zsákmányolni valót magadnak prédáljad; és fogyaszd el a te ellenségeidtõl való zsákmányt, a kiket kezedbe ad néked az Úr, a te Istened.
\par 15 Így cselekedjél mindazokkal a városokkal, a melyek igen messze esnek tõled, a melyek nem e nemzetek városai közül valók.
\par 16 De e népek városaiban, a melyeket örökségül ad néked az Úr, a te Istened, ne hagyj élni csak egy lelket is;
\par 17 Hanem mindenestõl veszítsd el õket: a Khittheust, az Emoreust, a Kananeust, a Perizeust, a Khivveust és a Jebuzeust, a mint megparancsolta néked az Úr, a te Istened;
\par 18 Azért, hogy meg ne tanítsanak titeket az õ mindenféle útálatosságaik szerint cselekedni, a melyeket õk cselekesznek az õ isteneiknek, és hogy ne vétkezzetek az Úr ellen, a ti  Istenetek ellen.
\par 19 Mikor valamely várost hosszabb ideig tartasz körülzárva, hadakozván az ellen, hogy bevegyed azt: ki ne veszítsd annak egy élõfáját sem, fejszével vágván azt; hanem egyél arról, és azt magát ki ne irtsad; mert ember-é a mezõnek fája, hogy ostrom alá jusson miattad?
\par 20 Csak a mely fáról tudod, hogy nem gyümölcstermõ, azt veszítsd el és irtsd ki, és abból építs erõsséget az ellen a város ellen, a mely te ellened hadakozik, mind addig, a míg leomlik az.

\chapter{21}

\par 1 Hogyha agyonütöttet találnak azon a földön, a melyet az Úr, a te Istened ád néked birtokul, és az a mezõn fekszik, és nem tudható: ki ölte meg azt;
\par 2 Akkor a te véneid és a te bíráid menjenek ki, és mérjék meg a földet a városokig, a melyek az agyonütött körül vannak;
\par 3 És a mely város legközelebb esik az agyonütötthöz, annak a városnak a vénei vegyenek egy üszõtinót, a melylyel még nem dolgoztattak, és a mely nem vont még jármot;
\par 4 És annak a városnak vénei vigyék az üszõt valamely folyóvíz völgyébe, a melyet nem szántanak és nem vetnek, és szegjék nyakát az üszõnek ott a völgyben.
\par 5 És járuljanak oda a papok, Lévi fiai; (mert õket választotta az Úr, a te Istened, hogy szolgáljanak néki, és hogy áldjanak az Úr  nevében, és az õ ítéletök szerint legyen minden per és minden sérelem).
\par 6 És mindnyájan annak a városnak vénei, a kik legközelebb esnek az agyonütötthöz, mossák meg kezeiket a nyakaszegett üszõ felett a völgyben;
\par 7 És szóljanak és mondják: A mi kezeink nem ontották ki ezt a vért, sem a mi szemeink nem látták.
\par 8 Bocsáss meg a te népednek, Izráelnek, a melyet megváltottál, Uram, és ne engedd, hogy ártatlan vér ontása legyen Izráel között, a te néped között! És bocsánatot nyernek a vérért.
\par 9 Te azért tisztítsd ki közüled az ártatlan vérnek kiontását, mert így cselekszed azt, a mi igaz az Úr elõtt.
\par 10 Mikor hadba mégy ellenségeid ellen, és kezedbe adja õket az Úr, a te Istened, és azok közül foglyokat ejtesz;
\par 11 És meglátsz a foglyok között egy szép ábrázatú asszonyt, és megszereted azt, úgy hogy elvennéd feleségül:
\par 12 Vidd be õt a te házadba, hogy nyirja meg a fejét, és messe le körmeit.
\par 13 És az õ fogoly ruháját vesse le magáról, és maradjon a te házadban, hogy sirassa az õ atyját és anyját egész hónapig; és csak azután menj be hozzá, és légy az õ férje, és legyen õ a te feleséged.
\par 14 Hogyha pedig nem tetszik néked, bocsásd el õt az õ kivánsága szerint; de pénzért semmiképen el ne add õt; ne hatalmaskodj rajta, miután megrontottad õt.
\par 15 Ha valakinek két felesége van, az egyik szeretett, a másik gyûlölt, és szülnek néki fiakat, mind a szeretett, mind a gyûlölt, és a gyûlöltnek fia lesz az elsõszülött:
\par 16 Azon a napon, a melyen az õ fiait örökösökké teszi a maga jószágában, nem teheti elsõszülötté a szeretettnek fiát a gyûlöltnek fia felett, a ki elsõszülött;
\par 17 Hanem az elsõszülöttet, a gyûlöltnek fiát ismerje el, két részt adván néki mindenbõl, a mije van; mert az az õ erejének  zsengéje, övé az elsõszülöttség joga.
\par 18 Ha valakinek pártütõ és makacs fia van, a ki az õ atyja szavára és anyja szavára nem hallgat, és ha megfenyítik, sem engedelmeskedik nékik:
\par 19 Az ilyet fogja meg az õ atyja és anyja, és vigyék azt az õ városának véneihez és az õ helységének kapujába,
\par 20 És ezt mondják a város véneinek: Ez a mi fiunk pártütõ és makacs, nem hallgat a mi szónkra, tobzódó és részeges:
\par 21 Akkor az õ városának minden embere kövekkel kövezze meg azt, hogy meghaljon. Így tisztítsd ki közüled a gonoszt, és az egész Izráel hallja meg, és féljen!
\par 22 Ha valakiben halálos ítéletre való bûn van, és megölik, és felakasztatod azt fára:
\par 23 Ne maradjon éjjel az õ holtteste a fán, hanem temesd el azt még azon a napon; mert átkozott Isten elõtt  a ki fán függ; és meg ne fertéztessed azt a földet, a melyet az Úr, a te Istened ád néked örökségül.

\chapter{22}

\par 1 Ne nézd el, ha a te atyádfiának ökre vagy juha tévelyeg, és ne fordulj el azoktól, hanem bizony tereld vissza azokat a te atyádfiához.
\par 2 Hogyha pedig nincs közel hozzád a te atyádfia, vagy nem is ismered õt: hajtsd a barmot a magad házához, és legyen nálad, míg keresi azt a te atyádfia, és akkor add vissza néki.
\par 3 És ekképen cselekedjél szamarával, ekképen cselekedjél ruhájával és ekképen cselekedjél a te atyádfiának minden elveszett holmijával, a mi elveszett tõle és te megtaláltad; nem szabad félrevonulnod.
\par 4 Ha látod, hogy a te atyádfiának szamara vagy ökre az úton eldûlve fekszik, ne fordulj el azoktól, hanem vele együtt emeld fel azokat.
\par 5 Asszony ne viseljen férfiruházatot, se férfi ne öltözzék asszonyruhába; mert mind útálatos az Úr elõtt, a te Istened elõtt, a ki ezt míveli.
\par 6 Ha madárfészek akad elédbe az úton valamely fán vagy a földön, madárfiakkal vagy tojásokkal, és az anya rajta ül a fiakon vagy a tojásokon: meg ne fogd az anyát a fiakkal egyben;
\par 7 Hanem bizony bocsásd el az anyát, és a fiakat fogd el magadnak, hogy jól legyen dolgod, és hosszú ideig élj.
\par 8 Ha új házat építesz, házfedeledre korlátot csinálj, hogy vérrel ne terheld a te házadat, ha valaki leesik arról.
\par 9 Ne vess a te szõlõdbe kétféle magot, hogy fertõzötté ne legyen az egész: a mag, a melyet elvetsz és a szõlõnek termése.
\par 10 Ne szánts ökrön és szamáron együtt.
\par 11 Ne öltözzél vegyes szövésû azaz gyapjúból és lenbõl szõtt ruhába.
\par 12 A te felsõruhádnak négy szegletére, a melyet felülre öltesz, bojtokat csinálj magadnak.
\par 13 Ha valaki feleséget vesz, és bemegy hozzá, és meggyûlöli azt,
\par 14 És szégyenletes dolgokkal vádolja, és rossz hírbe keveri azt, mert ezt mondja: E feleséget vettem magamnak, és hozzá mentem, de nem találtam õ benne szûzességet:
\par 15 Akkor vegye azt a leánynak atyja és anyja, és vigyék a leány szûzességének jeleit a város vénei elé a kapuba;
\par 16 És mondja a leánynak atyja a véneknek: Leányomat feleségül adtam e férfiúnak, de gyûlöli õt;
\par 17 És íme szégyenletes dolgokkal vádolja, mondván: Nem találtam a te leányodban szûzességet; pedig ímhol vannak az én leányom szûzességének jelei! És terítsék ki a ruhát a város vénei elé.
\par 18 Akkor a város vénei fogják meg azt a férfit, és ostorozzák meg õt;
\par 19 És bírságolják meg száz ezüst siklusra, és adják ezt a leány atyjának; mert rossz hírbe kevert egy izráelita szûzet; és legyen annak felesége, és el nem bocsáthatja azt teljes életében.
\par 20 Ha pedig igaz lesz a vádolás, és nem találtatik szûzesség a leányban:
\par 21 Akkor vigyék ki a leányt az õ atyjának háza elé, és az õ városának emberei kövezzék meg kõvel, hogy meghaljon; mert gyalázatosságot cselekedett Izráelben, paráználkodván az az õ atyjának házánál. Így tisztítsd ki közüled a gonoszt.
\par 22 Ha rajtakapnak valamely férfit, hogy férjes asszonynyal hál, õk mindketten is meghaljanak: a férfi,  a ki az asszonynyal hált, és az asszony is. Így tisztítsd ki a gonoszt Izráelbõl.
\par 23 Ha szûz leány van jegyben egy férfiúval, és megtalálja azt valaki a városban, és vele hál:
\par 24 Vigyétek ki mindkettõjöket annak a városnak kapuja elé, és kövezzétek meg õket kõvel, hogy meghaljanak. A leányt azért, hogy nem kiáltott a városban, a férfit pedig  azért, hogy meggyalázta az õ felebarátjának feleségét. Így tisztísd ki a gonoszt Izráelbõl.
\par 25 De hogyha mezõn találja a férfi a jegyben járó leányt, és erõszakoskodik rajta a férfi és vele hál: csak maga a férfi haljon meg, a ki azzal hált;
\par 26 A leányt pedig ne bántsd, mert a leánynak nincsen halálos bûne, mivel olyan ez a dolog, mint a mikor valaki felebarátjára támad és azt agyonüti.
\par 27 Mert a mezõn találta õt; kiálthatott a jegyben járó leány, de nem volt a ki megoltalmazza õt.
\par 28 Ha valaki el nem jegyzett szûz leánynyal találkozik, és megragadja azt, és vele hál, és rajta kapják õket:
\par 29 Akkor a férfi, a ki vele hált, adjon a leány atyjának ötven ezüst siklust, a leány pedig legyen feleségévé. Mivelhogy meggyalázta azt, nem bocsáthatja el azt teljes életében.
\par 30 Ne vegye el senki az õ atyjának feleségét, és az õ atyjának takaróját fel ne takarja!

\chapter{23}

\par 1 Akinek szeméremteste zúzott vagy megcsonkított, ne menjen be az Úrnak községébe.
\par 2 A fattyú se menjen be az Úrnak községébe; még tizedízig se menjen be az Úrnak községébe.
\par 3 Az Ammoniták és Moábiták se menjenek be az Úrnak községébe; még tizedízig se menjenek be az Úrnak községébe, soha örökké.
\par 4 Azért, mert nem jöttek elõtökbe kenyérrel és vízzel az úton, mikor kijöttetek Égyiptomból; és mivelhogy felbérlette ellened Bálámot, a Beór fiát, a mesopotámiabeli Péthorból valót, hogy megátkozzon téged.
\par 5 De az Úr, a te Istened nem akarta meghallgatni Bálámot; hanem fordította az Úr, a te Istened az átkot néked áldásodra, mivelhogy szeretett téged az Úr, a te Istened.
\par 6 Ne keresd az õ békességöket és az õ javokat teljes életedben, soha.
\par 7 Ne útáld az Edomitát; mert atyádfia az; ne útáld az égyiptombelit, mert jövevény voltál az õ földén.
\par 8 Az olyan fiak, a kik harmadízen születnek nékik, bemehetnek az Úrnak községébe.
\par 9 Ha táborba szállsz a te ellenséged ellen: õrizkedjél minden gonosztól.
\par 10 Ha volna valaki közötted, a ki nem volna tiszta valami éjszakai véletlenség miatt: menjen ki a táborból, és ne menjen vissza a táborba;
\par 11 És mikor eljõ az estve, mossa meg magát vízzel, és a nap lementével menjen be a táborba.
\par 12 A táboron kívül valami helyed is legyen, hogy kimehess oda.
\par 13 És legyen ásócskád a fegyvered mellett, hogy mikor leülsz kivül, gödröt áss azzal és ha felkelsz, betakarhassad azt, a mi elment tõled;
\par 14 Mert az Úr, a te Istened, a te táborodban jár, hogy megszabadítson téged, és elõdbe vesse a te ellenségedet: legyen azért a te táborod szent, hogy ne lásson te közted valami rútságot, és el ne forduljon tõled.
\par 15 Ne add ki a szolgát az õ urának, a ki az õ urától hozzád menekült.
\par 16 Veled lakjék, te közötted, azon a helyen, a melyet választ valamelyikben a te városaid közül, a hol néki tetszik: ne nyomorgasd õt.
\par 17 Ne legyen felavatott paráznanõ Izráel leányai közûl; se felavatott paráznaférfi ne legyen Izráel fiai közül.
\par 18 Ne vidd be a paráznanõ bérét és az eb-bért az Úrnak, a te Istenednek házába akárminémû fogadás fejében; mert mind a kettõt útálja az Úr, a te Istened.
\par 19 A te atyádfiától ne végy kamatot: se pénznek kamatját, se eleségnek kamatját, se semmi egyébnek kamatját, a mit kamatra szokás adni.
\par 20 Az idegentõl vehetsz kamatot, de a te atyádfiától ne végy kamatot, hogy megáldjon téged az Úr, a te Istened mindenben, a mire kinyujtod kezedet, azon a földön, a melyre bemégy, hogy bírjad azt.
\par 21 Ha fogadással ígérsz valamit az Úrnak, a te Istenednek: ne halogasd annak megadását; mert bizony megkeresi azt rajtad az Úr, a te Istened, és bûnül tulajdoníttatik az néked.
\par 22 Ha pedig nem teszesz fogadást, bûn sem tulajdoníttatik néked.
\par 23 Ügyelj arra, a mi ajkaidon kijön, és úgy teljesítsd, a mit száddal ígérsz, mint mikor szabad akaratból teszesz fogadást az Úrnak, a te Istenednek.
\par 24 Ha bemégy a te felebarátodnak szõlõjébe, egyél szõlõt kívánságod szerint jóllakásodig, de edényedbe ne rakj.
\par 25 Ha bemégy a te felebarátod vetésébe, kezeddel szaggass kalászokat, de sarlóval ne vágj be a te felebarátod vetésébe.

\chapter{24}

\par 1 Ha valaki asszonyt vesz magához, és feleségévé teszi azt, és ha azután nem találja azt kedvére valónak, mivelhogy valami illetlenséget talál benne, és ír néki váló levelet, és kezébe adja azt annak, és elküldi õt házától;
\par 2 És kimegy az õ házából, és elmegy és más férfiúé lesz;
\par 3 És a második férfiú is meggyûlöli õt, és ír néki váló levelet, és kezébe adja azt, és elküldi õt házától; vagy ha meghal az a második férfi, a ki elvette azt magának feleségül;
\par 4 Az elsõ férje, a ki elküldte õt, nem veheti õt másodszor is magához, hogy feleségévé legyen, minekutána megfertéztetett; mert útálatosság ez az Úr elõtt; te pedig ne tedd bûnössé a földet, a melyet az Úr, a te Istened ád néked örökségül.
\par 5 Hogyha valaki újonnan vesz feleséget, ne menjen hadba, és ne vessenek reá semmiféle terhet; egy esztendeig szabad legyen az õ házában, és vidámítsa a feleségét, a kit elvett.
\par 6 Zálogba senki ne vegyen kézimalmot vagy malomkövet, mert életet venne zálogba.
\par 7 Hogyha rajtakapnak valakit, a ki embert lop az õ atyjafiai közül, Izráel fiai közül, és hatalmaskodik rajta, vagy eladja azt: haljon meg az a tolvaj. Így tisztítsd ki a gonoszt te közüled.
\par 8 A poklosság csapásában vigyázz, hogy szorgalmatosan megtartsad és megcselekedjed mindazt, a mire a Lévita-papok tanítanak titeket; vigyázzatok, hogy a miképen megparancsoltam nékik, a képen cselekedjetek.
\par 9 Emlékezzél meg arról, a mit cselekedett az Úr, a te Istened Miriámmal az úton, mikor kijöttetek Égyiptomból.
\par 10 Ha kölcsön adsz valamit a te felebarátodnak: ne menj be az õ házába, hogy magad végy zálogot tõle;
\par 11 Kivül állj meg, és az ember, a kinek kölcsönt adsz, maga vigye ki hozzád az õ zálogát.
\par 12 Hogyha szegény ember az, ne feküdjél le az õ zálogával;
\par 13 Bizony add vissza néki azt a zálogot napnyugtakor, hogy az  õ ruhájában feküdjék le, és áldjon téged. És igazságul lesz ez néked az Úr elõtt, a te Istened elõtt.
\par 14 A szegény és szûkölködõ napszámoson ne erõszakoskodjál, akár atyádfiai, akár a te jövevényeid azok, a kik a te földeden a te kapuid között vannak.
\par 15 Azon a napon add meg az õ bérét, és le se menjen felette a nap; mert szegény õ, és kivánkozik az után az õ lelke, hogy ellened ne kiáltson az Úrhoz, és bûn ne legyen rajtad.
\par 16 Meg ne ölettessenek az atyák a fiakért, se a fiak meg ne ölettessenek az atyákért; kiki az õ bûnéért haljon meg.
\par 17 A jövevénynek és az árvának igazságát el ne csavard; és az özvegynek ruháját ne vedd zálogba.
\par 18 Hanem emlékezzél vissza, hogy szolga voltál Égyiptomban, és megváltott téged az Úr, a te Istened, onnét. Azért parancsolom én néked, hogy így cselekedjél.
\par 19 Mikor learatod aratni valódat a te mezõdön, és kévét felejtesz a mezõn, ne térj vissza annak felvételére; a jövevényé, az árváé és az özvegyé legyen az, hogy megáldjon téged az Úr, a te Istened, kezeidnek minden munkájában.
\par 20 Ha olajfád termését lerázod, ne szedd le, a mi még utánad marad; a jövevényé, árváé és az özvegyé legyen az.
\par 21 Ha szõlõdet megszeded, ne mezgeréld le, a mi utánad marad; a jövevényé, árváé és az özvegyé legyen az.
\par 22 És emlékezzél vissza, hogy szolga voltál Égyiptomnak földén. Azért parancsolom néked, hogy így cselekedjél.

\chapter{25}

\par 1 Ha per támad férfiak között, és törvény elé mennek, és megítélik õket, és igazat adnak az igaznak és bûnösnek mondják a bûnöst:
\par 2 Akkor, ha a bûnös ütleget érdemel, vonassa le azt a bíró, és üttessen arra maga elõtt annak bûnössége szerint való számban.
\par 3 Negyvenet üttessen rá, ne többet, hogy netalán, ha ennél több ütést üttet reá, alávalóvá legyen elõtted a te atyádfia.
\par 4 Ne kösd be az ökörnek száját, mikor nyomtat!
\par 5 Ha testvérek laknak együtt, és meghal egy közülök, és nincs annak fia: a megholtnak felesége ne menjen ki a háztól idegen férfiúhoz; hanem a sógora menjen be hozzá, és vegye el õt magának feleségül, és éljen vele sógorsági házasságban.
\par 6 És majd az elsõszülött, a kit szülni fog, a megholt testvér nevét kapja, hogy annak neve ki ne töröltessék Izráelbõl.
\par 7 Hogyha a férfinak nincs kedve elvenni az õ ángyát, menjen el az õ ángya a kapuba a vénekhez, és mondja: Sógorom vonakodik fentartani az õ testvérének nevét Izráelben, nem akar velem sógorsági házasságban élni.
\par 8 Akkor hívják azt az õ városának vénei, és beszéljenek vele; és ha megáll és ezt mondja: nincs kedvem õt elvenni:
\par 9 Akkor járuljon hozzá az õ ángya a vének szemei elõtt, és húzza le saruját annak lábáról, és köpjön az arczába, és szóljon, és ezt mondja: Így kell cselekedni azzal a férfival, a ki nem építi az õ testvérének házát.
\par 10 És "lehúzott sarujú háznép"-nek nevezzék az õ nevét Izráelben.
\par 11 Ha két férfi összevesz egymással, és az egyiknek felesége oda járul, hogy megszabadítsa az õ férjét annak kezébõl, a ki veri azt, és kinyujtja kezét, és megfogja annak szeméremtestét:
\par 12 Vágd el annak kezét, meg ne szánja szemed.
\par 13 Ne legyen a te zsákodban kétféle font: nagyobb és kisebb.
\par 14 Ne legyen a te házadban kétféle éfa: nagyobb és kisebb.
\par 15 Teljes és igaz fontod legyen néked; teljes és igaz éfád legyen néked; hogy hosszú ideig élj azon a földön, a melyet az Úr, a te Istened ád néked.
\par 16 Mert az Úr elõtt, a te Istened elõtt útálni való mindaz, a ki ezeket cselekszi; és mindaz, aki hamisságot mível.
\par 17 Megemlékezzél arról, a mit Amálek cselekedett rajtad az úton, a mikor Égyiptomból kijöttetek:
\par 18 Hogy reád támadt az úton, és megverte a seregnek utolsó részét, mind az erõtleneket, a kik hátul valának, a mikor magad is fáradt és lankadt voltál, és nem félte az Istent.
\par 19 Mikor azért megnyugtat majd téged az Úr, a te Istened, minden te köröskörül lévõ ellenségedtõl azon a földön, a melyet az Úr, a te Istened ád néked örökségül, hogy bírjad azt: töröld el Amálek emlékezetét az ég alól; el ne felejtsd!

\chapter{26}

\par 1 Mikor pedig bemégy arra a földre, a melyet az Úr, a te Istened ád néked örökségül, és bírni fogod azt, és lakozol abban:
\par 2 Akkor végy a föld minden gyümölcsének zsengéjébõl, a melyet szerezz a te földedbõl, a melyet az Úr, a te Istened ád néked; és tedd kosárba és menj oda a helyre, a melyet kiválaszt az Úr, a te Istened, hogy ott lakozzék az õ neve;
\par 3 És menj be a paphoz, a ki abban az idõben lesz, és mondjad néki: Vallást teszek ma az Úr elõtt, a te Istened elõtt, hogy bejöttem a földre, a mely felõl megesküdt az Úr a mi atyáinknak, hogy nékünk adja.
\par 4 És a pap vegye el a kosarat kezedbõl, és tegye azt az Úrnak, a te Istenednek oltára elé.
\par 5 És szólj, és mondjad az Úr elõtt, a te Istened elõtt: Veszendõ mesopotámiai vala az atyám, és aláment vala Égyiptomba, és jövevény volt ott kevesed magával; nagy, erõs és temérdek néppé lõn ottan.
\par 6 Bosszúsággal illetének pedig minket az Égyiptombeliek, és nyomorgatának minket, és vetének reánk kemény szolgálatot.
\par 7 Kiáltánk azért az Úrhoz, a mi atyáink Istenéhez, és meghallgatta az Úr a mi szónkat, és megtekintette a mi nyomorúságunkat, kínunkat és szorongattatásunkat:
\par 8 És kihozott minket az Úr Égyiptomból erõs kézzel, kinyújtott karral, nagy rettentéssel, jelekkel és csudákkal;
\par 9 És behozott minket e helyre, és adta nékünk ezt a földet, a tejjel és mézzel folyó földet.
\par 10 Most azért ímé elhoztam ama föld gyümölcsének zsengéjét, a melyet nékem adtál Uram. És rakd le azt az Úr elõtt, a te Istened elõtt, és imádkozzál az Úr elõtt, a te Istened elõtt;
\par 11 És örömet találj mindabban a jóban, a melyet ád néked az Úr, a te Istened, és a te házadnépének; te és a lévita, és a jövevény, a ki te közötted van.
\par 12 Ha a harmadik esztendõben, a tizednek esztendejében, minden termésedbõl egészen megadod a tizedet, és adod a lévitának, a jövevénynek, az árvának és özvegynek, hogy egyenek a te kapuid között, és jól lakjanak:
\par 13 Akkor ezt mondjad az Úr elõtt, a te Istened elõtt: Kitakarítottam a szent részt a házból, és oda adtam azt a lévitának, a jövevénynek, az árvának és az özvegynek minden te parancsolatod szerint, a melyet parancsoltál nékem; nem hágtam át egyet sem a te parancsolataidból, sem el nem felejtettem!
\par 14 Nem ettem belõle gyászomban, nem pusztítottam belõle tisztátalanul, és halottra sem adtam belõle. Hallgattam az Úrnak, az én Istenemnek szavára; a szerint cselekedtem, a mint parancsoltad nékem.
\par 15 Tekints alá a te szentségednek lakóhelyébõl a mennyekbõl, és áldd meg Izráelt, a te népedet, és a földet, a melyet nékünk adtál, a mint megesküdtél vala a mi atyáinknak, a tejjel és mézzel folyó földet.
\par 16 E mai napon az Úr, a te Istened parancsolja néked, hogy e rendelések és végzések szerint cselekedjél: tartsd meg azért és cselekedjed azokat teljes szívedbõl és teljes lelkedbõl!
\par 17 Azt kívántad ma kimondatni az Úrral, hogy Isteneddé lesz néked, hogy járhass az õ útain, megtudhassad az õ rendeléseit, parancsolatait és végzéseit, és engedhess az õ szavának;
\par 18 Az Úr pedig azt kívánja ma kimondatni veled, hogy az õ tulajdon népévé leszesz, a miképen szólott néked, és minden õ parancsolatát megtartod.
\par 19 Hogy feljebb valóvá tegyen téged minden nemzetnél, a melyeket teremtett, dícséretben, névben és dicsõségben, és hogy szent népévé lehess az Úrnak, a te Istenednek, a mint megmondta vala.

\chapter{27}

\par 1 Mózes pedig és Izráel vénei parancsot adának a népnek, mondván: Tartsátok meg mind e parancsolatot, a melyet én parancsolok ma néktek.
\par 2 És a mely napon általmentek a Jordánon arra a földre, a melyet az Úr, a te Istened ád néktek: nagy köveket állíts fel, és meszeld be azokat mészszel.
\par 3 És mihelyt általmégy, írd fel azokra e törvénynek minden ígéjét, hogy bemehess arra a földre, a melyet az Úr, a te Istened ád néked, a tejjel és mézzel folyó földre, a miképen megígérte néked az Úr, a te atyáidnak Istene.
\par 4 Mihelyt azért általmentek a Jordánon, állítsátok fel azokat a köveket, a melyeket én e mai napon parancsolok néktek, az Ebál hegyén; és meszeld be azokat mészszel.
\par 5 És építs ott oltárt az Úrnak, a te Istenednek; olyan kövekbõl való oltárt, a melyeket vassal meg ne faragj.
\par 6 Ép kövekbõl építsd az Úrnak, a te Istenednek oltárát; és áldozzál azon egészen égõáldozatokat az Úrnak, a te Istenednek.
\par 7 Áldozzál hálaáldozatokat is, és egyél ott, és vigadozzál az Úr elõtt, a te Istened elõtt.
\par 8 És írd fel a kövekre e törvénynek minden ígéjét igen világosan!
\par 9 És szóla Mózes és a Lévi nemzetségébõl való papok az egész Izráelnek, mondván: Figyelj és hallgass Izráel! E mai napon lettél az Úrnak, a te Istenednek népévé.
\par 10 Hallgass azért az Úrnak, a te Istenednek szavára, és cselekedjél az õ parancsolatai és rendelései szerint, a melyeket, én parancsolok ma néked.
\par 11 És parancsola Mózes azon a napon a népnek, mondván:
\par 12 Ezek álljanak fel a népnek megáldására a Garizim hegyén, mikor általmentek a Jordánon: Simeon, Lévi, Júda, Izsakhár, József és Benjámin.
\par 13 Ezek pedig az átkozásra álljanak fel az Ebál hegyén: Rúben, Gád, Áser, Zebulon, Dán és Nafthali.
\par 14 Szóljanak pedig a léviták, és ezt mondják az egész Izráel népének felszóval:
\par 15 Átkozott az ember, a ki faragott és öntött képet csinál, útálatára az Úrnak, mesterember kezének munkáját, és rejtve tartja azt! És feleljen az egész nép és mondja: Ámen!
\par 16 Átkozott a ki kevésre becsüli az õ atyját vagy anyját! És mondja az egész nép: Ámen!
\par 17 Átkozott, a ki elmozdítja az õ felebarátjának határát! És mondja az egész nép: Ámen!
\par 18 Átkozott, a ki félrevezeti a vakot az úton! És mondja az egész nép: Ámen!
\par 19 Átkozott, a ki elfordítja a jövevénynek, árvának és özvegynek igazságát! És mondja az egész nép: Ámen!
\par 20 Átkozott, a ki az õ atyjának feleségével hál, mert feltakarja az õ atyjának takaróját! És mondja az egész nép: Ámen!
\par 21 Átkozott, a ki közösül valamely barommal! És mondja az egész nép: Ámen!
\par 22 Átkozott, a ki az õ leánytestvérével hál, az õ atyjának leányával vagy az õ anyjának leányával! És mondja az egész nép: Ámen!
\par 23 Átkozott, a ki az õ napával hál! És mondja az egész nép: Ámen!
\par 24 Átkozott, a ki megöli az õ felebarátját titkon! És mondja az egész nép: Ámen!
\par 25 Átkozott, a ki ajándékot fogad el, hogy ártatlan lélek vérét ontsa! És mondja az egész nép: Ámen!
\par 26 Átkozott, a ki meg nem tartja e törvények ígéit, hogy cselekedje azokat! És mondja az egész nép: Ámen!

\chapter{28}

\par 1 Ha pedig szorgalmatosan hallgatsz az Úrnak, a te Istenednek szavára, és megtartod és teljesíted minden õ parancsolatát, a melyeket én parancsolok ma néked: akkor e földnek minden népénél feljebbvalóvá tesz téged az Úr, a te Istened;
\par 2 És reád szállanak mind ez áldások, és megteljesednek rajtad, ha hallgatsz az Úrnak, a te Istenednek szavára.
\par 3 Áldott leszesz a városban, és áldott leszesz a mezõben.
\par 4 Áldott lesz a te méhednek gyümölcse és a te földednek gyümölcse, és a te barmodnak gyümölcse, a te teheneidnek fajzása és a te juhaidnak ellése.
\par 5 Áldott lesz a te kosarad és a te sütõ tekenõd.
\par 6 Áldott leszesz bejöttödben, és áldott leszesz kimentedben.
\par 7 Az Úr megszalasztja elõtted a te ellenségeidet, a kik reád támadnak; egy úton jõnek ki reád, és hét úton futnak elõled.
\par 8 Áldást parancsol melléd az Úr a te csûreidben és mindenben, a mire ráteszed kezedet; és megáld téged azon a földön, a melyet az Úr, a te Istened ád néked.
\par 9 Az Úr felkészít téged magának szent néppé, a miképen megesküdt néked, ha megtartod az Úrnak, a te Istenednek parancsolatait, és az õ útain jársz.
\par 10 És megérti majd a földnek minden népe, hogy az Úrnak nevérõl neveztetel, és félnek tõled.
\par 11 És bõvölködõvé tesz téged az Úr minden jóban: a te méhednek gyümölcsében és a te földednek gyümülcsében, a te barmodnak gyümölcsébe azon a földön, a mely felõl megesküdt az Úr a te atyáidnak, hogy néked adja azt.
\par 12 Megnyitja néked az Úr az õ drága kincsesházát, az eget, hogy esõt adjon a te földednek alkalmas idõben, és megáldja kezednek minden munkáját, és kölcsönt adsz sok népnek, te pedig nem veszesz kölcsönt.
\par 13 És fejjé tesz téged az Úr és nem farkká, és mindinkább feljebbvaló leszesz és nem alábbvaló, ha hallgatsz az Úrnak, a te Istenednek parancsolataira, a melyeket én parancsolok ma néked, hogy tartsd meg és teljesítsd azokat;
\par 14 És ha el nem térsz egyetlen ígétõl sem, a melyeket én parancsolok néktek, se jobbra se balra, járván idegen istenek után, hogy azokat tiszteljétek.
\par 15 Ha pedig nem hallgatsz az Úrnak, a te Istenednek szavára, hogy megtartsad és teljesítsed minden parancsolatát és rendelését, a melyeket én parancsolok ma néked: reád jõnek mind ez átkok, és megteljesednek rajtad:
\par 16 Átkozott leszesz a városban, és átkozott a mezõn.
\par 17 Átkozott lesz a te kosarad és a te sütõ tekenõd.
\par 18 Átkozott lesz a te méhednek gyümölcse és a te földednek gyümölcse, a te teheneidnek fajzása és a te juhaidnak ellése.
\par 19 Átkozott leszesz bejöttödben, és átkozott leszesz kimentedben.
\par 20 Bocsát az Úr te reád átkot, bomlást és romlást mindenben, a mit kezdesz vagy cselekszel; mígnem eltöröltetel és mígnem gyorsasággal elveszesz a te cselekedeteidnek gonoszsága miatt, a melyekkel elhagytál engem.
\par 21 Hozzád ragasztja az Úr a döghalált, mígnem elemészt téged arról a földrõl, a melyre bemégy, hogy bírjad azt.
\par 22 Megver téged az Úr szárazbetegséggel hidegleléssel, gyulasztó és izzasztó betegséggel, aszálylyal és szárazsággal és ragyával; és üldöznek téged, mígnem elveszesz.
\par 23 Eged, a mely fejed felett van, rézzé, a föld pedig, a mely lábad alatt van, vassá válik.
\par 24 Az Úr esõ helyett port és hamut ád a te földedre; az égbõl száll reád, mígnem elpusztulsz.
\par 25 Az Úr megszalaszt téged a te ellenségeid elõtt; egy úton mégy ki õ reá, és hét úton futsz elõtte, és a föld minden országának rettentésére leszel.
\par 26 És eledelévé lesz a te holttested az ég minden madarának és a föld vadainak, és nem lesz, a ki elûzze azokat.
\par 27 Megver téged az Isten Égyiptomnak fekélyével, és sülylyel, varral és viszketegséggel, a melyekbõl ki nem gyógyíttathatol.
\par 28 Megver téged az Úr tébolyodással, vaksággal és elme-zavarodással;
\par 29 És tapogatni fogsz délben, a mint tapogat a vak a setétségben; és szerencsétlen leszel a te útaidban, sõt elnyomott és kifosztott leszel minden idõben, és nem lesz, a ki megszabadítson.
\par 30 Feleséget jegyzesz magadnak, de más férfi hál azzal; házat építesz, de nem lakol benne; szõlõt ültetsz, de nem veszed annak hasznát.
\par 31 A te ökröd szemed elõtt vágatik le, és nem eszel abból; a te szamarad elragadtatik elõled, és nem tér vissza hozzád; a te juhaid ellenségeidnek adatnak, és nem lesz, a ki megszabadítson.
\par 32 A te fiaid és leányaid más népnek adatnak, és a te szemeid néznek és epekednek utánok egész napon, és nem lesz erõ a te kezedben.
\par 33 A te földednek gyümölcsét, és minden fáradságos szerzeményedet oly nép emészti fel, a melyet nem ismertél, sõt elnyomott és megnyomorított leszel minden idõben.
\par 34 És megtébolyodol a látványtól, a melyet látni fognak a te szemeid.
\par 35 Megver téged az Úr gonosz kelésekkel a te térdeiden és czombjaidon, a melyekbõl ki nem gyúgyíttathatol, talpadtól fogva a koponyádig.
\par 36 Az Úr elvisz téged és a te királyodat, a kit magad fölé emelsz, oly nép közé, a melyet nem ismertél sem te, sem a te atyáid; és szolgálni fogsz ott idegen isteneket: fát és követ.
\par 37 És iszonyattá, példabeszéddé és gúnynyá leszel minden népnél, a melyek közé elûz téged az Úr.
\par 38 Sok magot viszel ki a mezõre, de keveset takarsz be, mert felemészti azt a sáska.
\par 39 Szõlõket ültetsz és míveled azokat, de bort nem iszol, meg sem szeded, mert elemészti azokat a féreg.
\par 40 Olajfáid lesznek minden határodban, de nem kened magadat olajjal, mert olajfádnak gyümölcse lehull.
\par 41 Fiakat és leányokat nemzesz, de nem lesznek tiéid; mert fogságra jutnak.
\par 42 Minden fádat és földednek minden gyümölcsét megemészti a sáska.
\par 43 A jövevény, a ki közötted van, feljebb-feljebb emelkedik feletted, te pedig alább-alább szállasz.
\par 44 Õ fog néked kölcsönt adni, és nem te kölcsönzöl néki; õ fej lesz, te pedig fark leszel.
\par 45 És ez átkok mind reád szállanak, és üldöznek téged és megteljesednek rajtad, míglen elpusztulsz; mert nem hallgattál az Úrnak, a te Istenednek szavára, hogy megtartottad volna az õ parancsolatait és rendeléseit, a melyeket parancsolt néked;
\par 46 És rajtad lesznek jelül és csudául, és a te magodon mind örökké.
\par 47 A miatt, hogy nem szolgáltad az Urat, a te Istenedet örömmel és jó szívvel, mindennel bõvölködvén:
\par 48 Szolgálod majd a te ellenségeidet, a kiket reád bocsát az Úr, éhen és szomjan, mezítelen és mindennek szûkiben; és vasigát vet a te nyakadra, míglen elpusztít téged.
\par 49 Hoz az Úr ellened népet meszszünnen, a földnek szélérõl, nem különben, a mint repül a  sas; oly népet, a melynek nyelvét nem érted;
\par 50 Vad tekintetû népet, a mely nem tiszteli a vén embert, és a gyermeknek nem kedvez:
\par 51 És felemészti a te barmodnak tenyészését és a te földednek gyümölcsét, mígnem kipusztulsz; a mely nem hágy néked a te gabonádból, borodból, olajodból, és a te teheneidnek fajzásából, juhaidnak ellésébõl, mígnem kiveszít téged.
\par 52 És megszáll téged minden városodban, míglen leomolnak a te magas és erõs kõfalaid, a melyekben bízol, minden te földeden: megszáll téged minden városodban, minden te földeden, a melyet az Úr, a te Istened ád néked.
\par 53 És megeszed a te méhednek gyümölcsét, a te fiaidnak és leányidnak húsát, a kiket ád néked az Úr, a te Istened - a megszállás és szorongattatás alatt, a melylyel megszorongat téged a te ellenséged.
\par 54 A te közötted való finnyás és igen kedvére nevekedett férfi is irígy szemmel tekint az õ atyjafiára, az õ szeretett feleségére és fiainak maradékrészére, a kik megmaradtak még.
\par 55 Hogy ne kelljen adnia azok közül senkinek az õ fiainak húsából, a mit eszik, mivelhogy semmi egyebe nem marad a megszállás és szorongattatás alatt, a melylyel megszorongat téged a te ellenséged minden városodban.
\par 56 A közötted való finnyás és kedvére nevekedett asszony (a ki meg se próbálta talpát a földre bocsátani az elkényesedés és finnyásság miatt) irígy szemmel tekint az õ szeretett férjére, fiára, leányára.
\par 57 Az õ mássa miatt, a mely elmegy tõle és gyermekei miatt, a kiket megszül; mert megeszi ezeket titkon, mikor mindenbõl kifogy, a megszállás és szorongattatás alatt, a melylyel megszorongat téged a te ellenséged a te városaidban.
\par 58 Hogyha meg nem tartod és nem teljesíted e törvény minden ígéjét, a melyek meg vannak írva e könyvben, hogy féljed e dicsõséges és rettenetes nevet, az Úrét, a te Istenedét:
\par 59 Csudálatosakká teszi az Úr a te csapásaidat, és a te magodnak csapásait: nagy és maradandó csapásokká, gonosz és maradandó betegségekké.
\par 60 És reád fordítja Égyiptomnak minden nyavalyáját, a melyektõl irtóztál vala, és hozzád ragadnak azok.
\par 61 Mindazt a betegséget és mindazt a csapást is, a melyek nincsenek megírva e törvénynek könyvében, reád rakja az Úr, míglen kipusztulsz.
\par 62 És kevesen maradtok meg, a kik annak elõtte oly sokan  voltatok, mint az égnek csillagai; mivelhogy nem hallgattál az Úrnak, a te Istenednek szavára.
\par 63 És a miképen örvendezett az Úr rajtatok, hogy jót tett veletek és megsokasított titeket: akképen fog örvendezni az Úr rajtatok, hogy kiveszt és kipusztít titeket; és ki fogtok gyomláltatni arról a földrõl, a melyre te bemégy, hogy bírjad azt.
\par 64 És szétszór téged az Úr minden nép közé, a földnek egyik végétõl a földnek másik végéig; és szolgálni fogsz ott idegen isteneket, a kiket sem te nem ismertél, sem a te atyáid: fát és követ.
\par 65 De e nemzetek között sem pihensz meg, és nem lesz a te talpadnak nyugodalma; mert rettegõ szívet, epedõ szemeket és sóvárgó lelket ád ott néked az Úr.
\par 66 És a te életed kétséges lesz majd elõtted: és rettegni fogsz éjjel és nappal, és nem bízol életedben.
\par 67 Reggel azt mondod: Bárcsak estve volna! estve pedig azt mondod: Bárcsak reggel volna! - a te szívednek rettegései miatt, a melylyel rettegsz, és a te szemeidnek látása miatt, a melyet látsz.
\par 68 És visszavisz téged az Úr Égyiptomba hajókon, azon az úton, a melyrõl azt mondtam néked, hogy nem fogod azt többé meglátni! És áruljátok ott magatokat a ti ellenségeiteknek szolgákul és szolgálóleányokul, de nem lesz, a ki megvegyen.

\chapter{29}

\par 1 Ezek annak a szövetségnek ígéi, a mely felõl megparancsolta az Úr Mózesnek, hogy kösse meg azt Izráel fiaival Moábnak földén, azon a szövetségen kivül, a melyet kötött vala velök a Hóreben.
\par 2 És elõhivatá Mózes az egész Izráelt, és monda nékik: Ti láttátok mind azt, a mit szemeitek elõtt cselekedett az Úr Égyiptom földén a Faraóval és minden õ szolgájával, és egész földével:
\par 3 A nagy kísértéseket, a melyeket láttak a te szemeid, a jeleket és ama nagy  csudákat.
\par 4 De nem adott az Úr néktek szívet, hogy jól értsétek, szemeket, hogy lássátok, és füleket, hogy halljatok, mind e mai napig.
\par 5 Mindamellett is vezérlettelek titeket a pusztában negyven esztendeig; nem koptak le a ti ruháitok rólatok, és a te sarud sem kopott le lábadról.
\par 6 Kenyeret nem ettetek, sem bort, sem részegítõ italt nem ittatok, hogy megtudjátok, hogy én vagyok az Úr, a ti Istenetek.
\par 7 És eljutottatok e helyre, és kijöve elõnkbe Szíhon, Hesbonnak királya, és Óg, Básánnak királya, hogy megütközzenek velünk, de megvertük õket.
\par 8 És elvettük az õ földjöket, és odaadtuk örökségül a Rubenitáknak, Gáditáknak és a Manassé-törzs felének.
\par 9 Tartsátok meg azért e szövetségnek ígéit, és a szerint cselekedjetek, hogy szerencsések legyetek mindenben, a mit cselekesztek.
\par 10 Ti e napon mindnyájan az Úr elõtt, a ti Istenetek elõtt álltok: a ti fõembereitek, törzseitek, véneitek és a ti tiszttartóitok, Izráelnek minden férfia;
\par 11 A ti kicsinyeitek, feleségeitek és a te jövevényed, a ki a te táborodban van, sõt favágóid és vízmerítõid is;
\par 12 Hogy szövetségre lépjetek az Úrral, a ti Istenetekkel, és pedig az õ esküjével erõsített kötésre, a melyet ma köt meg veled az Úr, a te Istened;
\par 13 Hogy az õ népévé emeljen ma téged, õ pedig legyen néked Istened, a miképen szólott néked, és a miképen megesküdt  a te atyáidnak, Ábrahámnak, Izsáknak és Jákóbnak.
\par 14 És nem csak ti veletek kötöm én e szövetséges, és ez esküvéses kötést,
\par 15 Hanem azzal, a ki itt van velünk, és itt áll e mai napon az Úr elõtt a mi Istenünk elõtt, és azzal is, a ki nincsen e mai napon itt velünk.
\par 16 (Mert ti tudjátok miképen laktunk Égyiptomnak földén, és miképen jöttünk által a nemzetek között, a kiken általjöttetek.
\par 17 És láttátok az õ undokságaikat és bálványaikat: fát és követ, ezüstöt és aranyat, a melyek nálok vannak.)
\par 18 Vajha ne lenne közöttetek férfi vagy asszony, nemzetség vagy törzs, a kinek szíve elforduljon e mai napon az Úrtól, a mi Istenünktõl, hogy elmenjen és szolgáljon e nemzetek isteneinek; vajha ne lenne köztetek méreg- és ürömtermõ gyökér!
\par 19 És ha lesz, a ki hallja ez esküvéses kötésnek ígéit, és boldognak állítja magát az õ szívében, ezt mondván: Békességem lesz nékem, ha a szívem gondolata szerint járok is, (hogy a részeg és a szomjas együtt veszszenek):
\par 20 Nem akar majd az Úr annak megbocsátani, sõt felgerjed akkor az Úrnak haragja és búsulása az ilyen ember ellen, és rászáll arra minden átok, a mely meg van írva e könyvben, és eltörli az Úr annak nevét az ég alól.
\par 21 És kiválasztja azt az Úr veszedelemre, Izráelnek minden törzse közül, a szövetségnek minden átka szerint, a melyek meg vannak írva e törvénykönyvben.
\par 22 És ezt fogja mondani a következõ nemzedék, a ti fiaitok, a kik ti utánatok támadnak, és az idegen, a ki messze földrõl jön el, ha látni fogják e földnek csapásait és nyomorúságait, a melyekkel megnyomorította azt az Úr:
\par 23 Kénkõ és só égette ki egész földjét, be sem vethetõ, semmit nem terem, és semmi fû sem nevelkedik rajta; olyan, mint Sodomának, Gomorának, Ádmának és Czeboimnak elsülyesztett helye, a melyeket elsülyesztett az Úr haragjában és búsulásában.
\par 24 Azt fogják majd kérdezni mind a nemzetek: Miért cselekedett az Úr így ezzel a földdel? Micsoda nagy felgerjedése ez a haragnak?
\par 25 És ezt mondják majd: Azért, mert elhagyták az Úrnak, az õ atyáik Istenének szövetségét, a melyet akkor kötött velök, a mikor kihozta õket Égyiptom földérõl;
\par 26 És elmentek, és szolgáltak idegen isteneket, és imádták azokat; olyan isteneket, a kiket nem ismertek volt és nem adott nékik az Isten.
\par 27 És felgerjedett az Úrnak haragja e föld ellen, hogy reá hozza mindazt az átkot, a mely meg van írva e könyvben.
\par 28 És kigyomlálta õket az Úr az õ földjökrõl haragjában, búsulásában és nagy indulatjában; és vetette õket más földre, a mint mai nap is van.
\par 29 A titkok az Úréi, a mi Istenünkéi; a kinyilatkoztatott dolgok pedig a miénk és a mi fiainké mind örökké, hogy e törvénynek minden ígéjét beteljesítsük.

\chapter{30}

\par 1 És ha majd elkövetkeznek reád mind ezek: az áldás és az átok, a melyet elõdbe adtam néked; és szívedre veszed azt ama nemzetek között, a kik közé oda taszított téged az Úr, a te Istened;
\par 2 És megtérsz az Úrhoz, a te Istenedhez, és hallgatsz az õ szavára mind a szerint, a mint én parancsolom néked e napon, te és a te fiaid teljes szívedbõl és teljes lelkedbõl:
\par 3 Akkor visszahozza az Úr, a te Istened a te foglyaidat, és könyörül rajtad, és visszahozván, összegyûjt majd téged minden  nép közül, a kik közé oda szórt téged az Úr, a te Istened.
\par 4 Ha az ég szélére volnál is taszítva, onnét is összegyûjt téged az Úr, a te Istened, és onnét is felvesz téged;
\par 5 És elhoz téged az Úr, a te Istened a földre, a melyet bírtak a te atyáid, és bírni fogod azt; és jól tesz veled, és inkább megsokasít téged, mint a te atyáidat.
\par 6 És körülmetéli az Úr, a te Istened a te szívedet, és a te magodnak szívét, hogy szeressed az Urat, a te Istenedet teljes szívedbõl és teljes lelkedbõl, hogy élj.
\par 7 Mind ez átkokat pedig rábocsátja az Úr, a te Istened a te ellenségeidre és gyûlölõidre, a kik üldöztek téged.
\par 8 Te azért térj meg, és hallgass az Úr szavára, és teljesítsd minden parancsolatát, a melyeket én e mai napon parancsolok néked.
\par 9 És bõvölködõvé tesz téged az Úr, a te Istened kezeidnek minden munkájában, a te méhednek gyümölcsében, a te barmodnak gyümölcsében és a te földednek gyümölcsében, a te jódra. Mert hozzád fordul az Úr és öröme lesz benned a te jódra, a miképen öröme volt a te atyáidban.
\par 10 Hogyha hallgatsz az Úrnak, a te Istenednek szavára, megtartván az õ parancsolatait és rendeléseit, a melyek meg vannak írva e törvénykönyvben, és ha teljes szívedbõl és teljes lelkedbõl megtérsz az Úrhoz, a te Istenedhez.
\par 11 Mert e parancsolat, a melyet én e mai napon parancsolok néked, nem megfoghatatlan elõtted; sem távol nincs tõled.
\par 12 Nem a mennyben van, hogy azt mondanád: Kicsoda hág fel érettünk a mennybe, hogy elhozza azt nékünk, és hallassa azt velünk, hogy teljesítsük azt?
\par 13 Sem a tengeren túl nincsen az, hogy azt mondanád: Kicsoda megy át érettünk a tengeren, hogy elhozza azt nékünk és hallassa azt velünk, hogy teljesítsük azt?
\par 14 Sõt felette közel van hozzád ez íge: a te szádban és szívedben van, hogy teljesítsed azt.
\par 15 Lám elõdbe adtam ma néked az életet és a jót: a halált és a gonoszt.
\par 16 Mikor én azt parancsolom néked ma, hogy szeressed az Urat, a te Istenedet, hogy járj az õ útain, és tartsd meg az õ parancsolatait, rendeléseit és végzéseit, hogy élj és szaporodjál, és megáldjon téged az Úr, a te Istened a földön, a melyre bemégy, hogy bírjad azt.
\par 17 Ha pedig elfordul a te szíved, és nem hallgatsz meg, sõt elhajolsz és idegen isteneket imádsz, és azoknak szolgálsz;
\par 18 Tudtotokra adom ma néktek, hogy bizony elvesztek: nem éltek sok ideig azon a földön, a melyre a Jordánon átkelvén, bemégy, hogy bírjad azt.
\par 19 Bizonyságul hívom ellenetek ma a mennyet és a földet, hogy az életet és  a halált adtam elõtökbe, az áldást és az átkot: válaszd azért az életet, hogy élhess mind te, mind a te magod;
\par 20 Hogy szeressed az Urat, a te Istenedet, és hogy hallgass az õ szavára, és ragaszkodjál hozzá; mert õ a te életed és a te életednek hosszúsága; hogy lakozzál azon a földön, a mely felõl megesküdt az Úr a te atyáidnak, Ábrahámnak, Izsáknak és Jákóbnak, hogy nékik adja azt.

\chapter{31}

\par 1 És méne Mózes, és ez ígéket mondotta vala az egész Izráelnek;
\par 2 Monda pedig nékik: Száz és húsz esztendõs vagyok ma, nem járhatok többé ki és be: az Úr pedig azt mondá nékem: Nem mégy át ezen a Jordánon.
\par 3 Az Úr, a te Istened maga megy át elõtted, õ pusztítja el e nemzeteket elõtted, hogy bírjad õket; Józsué az, a ki átmegy elõtted, a mint megmondotta az Úr.
\par 4 És akképen cselekeszik azokkal az Úr, a miképen cselekedett Szíhonnal és Óggal az Emoreusok királyaival, és azoknak földjökkel, a melyeket elpusztított vala.
\par 5 Ha azért elõtökbe adja õket az Úr, egészen a szerint a parancsolat szerint cselekedjetek velök, a mint parancsoltam néktek.
\par 6 Legyetek erõsek és bátrak, ne féljetek és ne rettegjetek tõlök, mert az Úr, a te Istened maga megy veled; nem  marad el tõled, sem el nem hágy téged.
\par 7 Szólítá azért Mózes Józsuét, és monda néki az egész Izráel szemei elõtt: Légy erõs és bátor, mert te mégy be e néppel a földre, a mely felõl megesküdt az Úr az õ atyáiknak, hogy nékik adja, és te osztod el azt nékik örökségül.
\par 8 Az Úr, õ az, a ki elõtted megy, õ lesz te veled; el nem marad tõled, sem el nem hágy téged: ne félj és ne rettegj!
\par 9 És megírá Mózes e törvényt, és adá azt a papoknak, a Lévi fiainak, a kik hordozzák az Úr szövetségének ládáját, és Izráel minden vénjének.
\par 10 És megparancsolá nékik Mózes, mondván: A hetedik esztendõ végén, az elengedés esztendejének idejében, a sátorok innepén;
\par 11 Mikor eljön az egész Izráel, hogy megjelenjék az Úr elõtt, a te Istened elõtt azon a helyen, a melyet kiválaszt: olvasd fel e törvényt az egész Izráel elõtt fülök hallására.
\par 12 Gyûjtsd egybe a népet, a férfiakat, az asszonyokat, a kicsinyeket és a te jövevényedet, a ki a te kapuidon belõl van, hogy hallják és tanuljanak, és féljék az Urat, a ti Isteneteket, és tartsák meg és teljesítsék e törvénynek minden ígéjét.
\par 13 És az õ fiaik is, a kik nem tudják még, hallják és tanulják meg, hogy az Urat, a ti Isteneteket kell félni mind addig, a míg éltek azon a földön, a melyre általkeltek a Jordánon, hogy bírjátok azt.
\par 14 Monda azután az Úr Mózesnek: Ímé elközelgettek a te napjaid, hogy meghalj; hívd elõ Józsuét, és álljatok fel a gyülekezetnek sátorában, hogy parancsolatokat adjak néki. Elméne azért Mózes és Józsué, és felállának a gyülekezet sátorában.
\par 15 És megjelenék az Úr a sátorban, felhõoszlopban, és megálla a felhõoszlop a sátor nyílása felett,
\par 16 És monda az Úr Mózesnek: Ímé te elaluszol a te atyáiddal, és ez a nép felkél, és idegen istenek után jár és paráználkodik azon a földön, a melyre bemegy, hogy lakozzék azon; és elhágy engem, és felbontja az én szövetségemet, a melyet én õ vele kötöttem.
\par 17 De felgerjed az én haragom õ ellene azon a napon, és elhagyom õt, és elrejtem  az én orczámat õ elõle, hogy megemésztessék. És mikor utóléri a sok baj és nyomorúság, mondani fogja azon a napon: Avagy nem azért értek-é engem ezek a bajok, hogy nincsen az én Istenem én közöttem?
\par 18 Én pedig valóban elrejtem az én orczámat azon a napon az õ minden gonoszsága miatt, a melyet cselekedett, mivelhogy más istenekhez fordult.
\par 19 Most pedig írjátok fel magatoknak ez éneket, és tanítsd meg arra Izráel fiait; adjad azt szájokba, hogy legyen nékem ez ének bizonyságul Izráel fiai ellen.
\par 20 Mert beviszem õt arra a földre, amely felõl megesküdtem az õ atyáinak, a tejjel és mézzel folyó földre; és eszik, jóllakik és meghízik, azután pedig  más istenekhez fordul, és azoknak szolgál, és meggyaláz engem, és felbontja az én szövetségemet.
\par 21 Mikor pedig utóléri õt a sok baj és nyomorúság: akkor szóljon ez az ének elõtte bizonyságképen (mert nem megy feledésbe az õ maradékának szájából), mert tudom az õ gondolatát, a mely szerint cselekszik már most is, minekelõtte bevinném õt arra a földre, a mely felõl megesküdtem vala.
\par 22 Megírá azért Mózes ezt az éneket azon a napon, és megtanítá arra Izráel fiait.
\par 23 Azután parancsola az Úr Józsuénak, a Nún fiának, és monda: Légy erõs  és bátor, mert te viszed be Izráel fiait arra a földre, a mely felõl megesküdtem nékik; és én veled leszek.
\par 24 Mikor pedig teljesen és mind végig beírta Mózes e törvény ígéit könyvbe:
\par 25 Parancsola Mózes a lévitáknak, a kik hordozzák vala az Úr szövetségének ládáját, mondván:
\par 26 Vegyétek e törvénykönyvet, és tegyétek ezt az Úrnak, a ti isteneteknek szövetségládája oldalához, és legyen ott ellened bizonyságul;
\par 27 Mert én ismerem a te pártos voltodat, és kemény nyakadat. Ímé most is, holott még köztetek élek, pártot ütöttetek az Úr ellen; mennyivel inkább halálom után?
\par 28 Gyûjtsétek én hozzám a ti törzseiteknek minden vénjét és a ti elõljáróitokat, hadd mondjam el ez ígéket az õ füleik hallására, és hadd hívjam bizonyságul ellenök a mennyet és földet.
\par 29 Mert tudom, hogy halálom után mind inkább-inkább megromoltok és eltértek az útról, a melyet parancsoltam néktek; és utólér majd titeket a veszedelem a késõbbi idõben, mivelhogy gonoszt cselekesztek az Úrnak szemei elõtt, bosszantván õt kezeiteknek csinálmányával.
\par 30 Azután elmondá Mózes Izráel egész gyülekezetének füle hallására ez éneknek ígéit, mind végig.

\chapter{32}

\par 1 Figyeljetek egek, hadd szóljak! Hallgassa a föld is számnak beszédeit!
\par 2 Csepegjen tanításom, mint esõ; hulljon mint harmat a beszédem; mint langyos zápor a gyenge fûre, s mint permetezés a pázsitra!
\par 3 Mert az Úr nevét hirdetem: magasztaljátok Istenünket!
\par 4 Kõszikla! Cselekedete tökéletes, mert minden õ úta igazság! Hûséges Isten és nem csalárd; igaz és egyenes õ!
\par 5 Gonoszak voltak hozzá, nem fiai, a magok gyalázatja; romlott és elvetemült  nemzedék.
\par 6 Így fizettek-é az Úrnak: balga és értelmetlen nép?! Nem atyád-é õ, a ki teremtett? Õ alkotott és erõsített meg.
\par 7 Emlékezzél meg az õs idõkrõl; gondoljátok el annyi nemzedék éveit! Kérdezd meg atyádat és megjelenti néked, a te véneidet és megmondják néked!
\par 8 Mikor a Felséges örökséget osztott a népeknek; mikor szétválasztá az ember fiait: megszabta a népek határait, Izráel fiainak  száma szerint,
\par 9 Mert az Úrnak része az õ népe, Jákób néki sorssal jutott öröksége.
\par 10 Puszta földön találta vala õt, zordon, sivatag vadonban; körülvette õt, gondja volt reá, õrizte, mint a szeme fényét;
\par 11 Mint a fészkén felrebbenõ sas, fiai felett lebeg, kiterjeszti felettök szárnyait, felveszi õket, és tollain emeli õket:
\par 12 Egymaga vezette õt az Úr; idegen Isten nem volt õ vele.
\par 13 A föld magaslatain járatta õt, mezõk terméseivel étette, kõsziklából is mézet szopatott vele, kovaszirtbõl is olajat;
\par 14 Tehenek vaját, és juhok tejét bárányok kövérjével, básáni kosokat és bakkecskéket a buza java kövérjével; és szõlõ vérét, bort ittál.
\par 15 És meghízott Jesurun, és rúgódozott. Meghíztál, megkövéredtél, elhájasodtál. És elhagyá Istent, teremtõjét, és megveté az õ  üdvösségének kõszikláját.
\par 16 Idegen istenekkel ingerelték, útálatosságokkal bosszantották.
\par 17 Ördögöknek áldoztak, nem Istennek; isteneknek, a kiket nem ismertek; újaknak, a kik csak most támadtak, a kiket nem rettegtek a ti atyáitok.
\par 18 A Kõsziklát, a ki szült téged, elfeledted; megfelejtkeztél Istenrõl, a ki nemzett téged.
\par 19 Látta ezt az Úr és megútálta bosszúságában az õ fiait és leányait.
\par 20 És monda: Elrejtem orczámat elõlök, hadd látom, mi lesz a végök? Mert elzüllött nemzetség ez, fiak, akikben nincs hûség!
\par 21 Azzal ingereltek õk, a mi nem isten; hiábavalóságaikkal bosszantottak engem; én pedig azzal ingerlem õket, a mi nem népem: bolond nemzettel  bosszantom õket.
\par 22 Mert tûz lobban fel haragomban és leég a Seol fenekéig; megemészti a földet és gyümölcsét, és felgyújtja a hegyek alapjait.
\par 23 Veszedelmeket halmozok reájok, nyilaimat mind rájok fogyasztom.
\par 24 Éhségtõl aszottan, láztól emésztetten és keserû dögvésztõl - a vadak fogait is rájok bocsátom, a porban csúszók mérgével együtt.
\par 25 Kivül fegyver pusztít, az ágyasházakban rettegés: ifjat és szûzet, csecsszopót a vén emberrel együtt.
\par 26 Mondom: Elfuvom õket, eltörlöm emlékezetöket az emberek közül.
\par 27 Ha nem tartanék az ellenség bosszantásától, hogy szorongatóik a dolgot félremagyarázzák, és hogy ezt mondják: A mi kezünk a hatalmas, és nem az Úr cselekedte mind ezt! -
\par 28 Mert tanács-vesztett nép ez, és nincs bennök értelem.
\par 29 Vajha eszesek volnának, megértenék ezt, meggondolnák, hogy mi lesz a végök!
\par 30 Miképen kergethetne egy ezeret, és kettõ hogyan ûzhetne tízezeret, ha az õ  Kõsziklájok el nem adja õket, és ha az Úr kézbe nem adja õket?!
\par 31 Mert a mi Kõsziklánk nem olyan, mint az õ kõsziklájok; ellenségeink is megítélhetik!
\par 32 Mert az õ szõlõjök Sodoma szõlõje és Gomora mezõsége; bogyóik mérges bogyók, keserûek a gerézdjeik.
\par 33 Sárkányok mérge az õ boruk, áspiskígyóknak kegyetlen epéje.
\par 34 Nincsen-é ez elrejtve nálam, lepecsételve az én kincseim között?
\par 35 Enyém a bosszúállás és megfizetés, a mikor lábuk megtántorodik: mert közel van az õ veszedelmök napja, és siet, a mi rájok vár!
\par 36 Mert megítéli az Úr az õ népét, és megkönyörül az õ szolgáin, ha látja, hogy elfogyott az erõ, s védett és védtelen oda van.
\par 37 És ezt mondja: Hol az õ istenök? a Kõszikla, a melyben bizakodtak?
\par 38 A kik megették az õ véres áldozataik kövérjét, megitták az õ italáldozatuk borát: keljenek fel és segítsenek meg titeket, és oltalmazzanak meg titeket!
\par 39 Most lássátok meg, hogy én vagyok, és nincs Isten kivülem! Én ölök  és elevenítek, én sebesítek és én gyógyítok, és nincs, a ki kezembõl megszabadítson.
\par 40 Mert felemelem kezemet az égre, és ezt mondom: Örökké élek én!
\par 41 Ha megélesítem fényes kardomat és ítélethez fog kezem: bosszút állok ellenségeimen és megfizetek gyûlölõimnek.
\par 42 Megrészegítem nyilaimat vérrel, és kardom jól lakik hússal: a legyilkoltak és foglyok vérével, az ellenség vezéreinek fejébõl!
\par 43 Ujjongjatok ti nemzetek, õ népe! Mert õ megtorolja az õ szolgáinak vérét,  bosszút áll az õ ellenségein, földjének és népének megbocsát!
\par 44 Elméne azért Mózes és elmondá ez éneknek minden ígéjét a nép füle hallására, õ és Józsué a Nún fia.
\par 45 És mikor végig elmondá Mózes mind ez ígéket az egész Izráelnek,
\par 46 Monda nékik: Vegyétek szívetekre mind ezeket az ígéket, amelyekkel én bizonyságot teszek ellenetek e mai napon: és parancsoljátok meg fiaitoknak, hogy tartsák meg és teljesítsék e törvénynek minden ígéjét;
\par 47 Mert nem hiábavaló íge ez néktek; hanem ez a  ti életetek, és ez íge által hosszabbítjátok meg napjaitokat azon a földön, a melyre általmentek a Jordánon, hogy bírjátok azt.
\par 48 És ugyanezen a napon szóla az Úr Mózesnek, mondván:
\par 49 Menj fel ebbe az Abarim hegységbe, a Nébó hegyére, a mely Moáb földén van és pedig Jérikhóval átellenben; és nézd meg a Kanaán földét, a melyet én Izráel fiainak adok örökségül.
\par 50 És halj meg a hegyen, a melyre felmégy, és takaríttassál a te népedhez, a miképen meghalt Áron, a te testvéred a Hór hegyén, és takaríttatott az õ népeihez;
\par 51 Mivelhogy vétkeztetek ellenem Izráel fiai között a versengésnek vizénél, a Czin pusztájában Kádesnél: mert nem szenteltetek meg engem Izráel fiai között.
\par 52 Mert szemközt látod a földet; de arra a földre, a melyet én adok Izráel fiainak, oda nem mégy be.

\chapter{33}

\par 1 Ez pedig az áldás, a melylyel megáldá Mózes, az Istennek embere, Izráel fiait az õ halála elõtt.
\par 2 Monda ugyanis: Az Úr a Sinai hegyrõl jött, és Szeirbõl támadt fel nékik; Párán hegyérõl ragyogott elõ, tízezer szent közül jelent meg, jobbja felõl tüzes törvény vala számukra.
\par 3 Bizony szereti õ a népeket! Mind kezednél vannak az õ szentjei, oda szegõdnek, a te lábaidhoz, és hallgatják a te beszédeidet.
\par 4 Törvényt parancsolt nékünk Mózes, örökségül  Jákób községének.
\par 5 És király lõn Jesurunban, mikor összegyûltek a népnek fejei, és együtt voltak Izráel törzsei.
\par 6 Éljen Rúben és meg ne haljon; és száma legyen embereinek.
\par 7 Ez pedig a Júda áldása; és monda: Hallgasd meg Uram a Júda szavát, és vidd be õt az õ népéhez. Az õ keze elégséges legyen néki, de légy segítsége az õ szorongatói ellen.
\par 8 Lévirõl pedig monda: A te Thummimod és Urimod a te kegyes férfiadé, akit megkísértél Masszában, a kivel perbe szálltál Mériba vizeinél.
\par 9 A ki azt mondta az õ atyjáról és anyjáról: Nem láttam õt; és az õ atyjafiait nem ismerte, fiaival sem gondolt; mert megtartották a te beszédedet, és ragaszkodtak szövetségedhez.
\par 10 Tanítják a te végzéseidre Jákóbot, és a te törvényedre Izráelt; füstölõt tesznek a te orczád elé, és égõáldozatot a te oltárodra.
\par 11 Áldd meg Uram az õ erejét, és az õ kezének munkája legyen kedves elõtted! Törd meg derekukat a reá támadóknak és az õ gyûlölõinek, hogy fel ne kelhessenek!
\par 12 Benjáminról monda: Az Úrnak kedveltje! Bátorságban lakozik mellette, fedezi õt minden idõben, és az õ vállai között lakik.
\par 13 Józsefrõl pedig monda: Áldott az Úrtól az õ földje az égnek kincseivel, a harmattal és az alant elterülõ mélységes vizekkel;
\par 14 A nap érlelte drága terméssel; és a hold sarjasztotta drágaságokkal;
\par 15 És az õs hegyek javaival, és az örök halmok drágaságaival;
\par 16 A földnek drágaságaival és bõségével. A csipkebokorban lakozónak jó kedve szálljon Józsefnek fejére, az õ atyjafiai közül kiválasztottnak koponyájára!
\par 17 Tehenének elsõ fajzása dicsõségére van; szarvai bivalyszarvak; népeket öklel azokkal mindenfelé a földnek széléig. És ezek Efraim tízezrei és Manassé ezrei.
\par 18 És Zebulonról monda: Örvendj Zebulon a te kimentedben, és te Izsakhár a te sátraidban.
\par 19 Népeket hívogatnak a hegyre, igaz áldozattal áldoznak ott; mert a tengerek bõségét szopják, és a fövénynek rejtett kincsét.
\par 20 És Gádról monda: Áldott az, a ki kiterjeszti Gádot! Mint nõstény oroszlán, úgy lakik, és szétszaggat kart és koponyát.
\par 21 Az elejét nézte ki magának, mert ott volt elrejtve a törvényadó osztályrésze. De elméne a népnek fejedelmeivel, az Úrnak igazságát cselekedte, és az õ végzését Izráellel együtt.
\par 22 És Dánról monda: Dán oroszlánnak kölyke, a mely Básánból szökik ki.
\par 23 És Nafthaliról monda; Ó Nafthali, a ki az Úrnak jó kedvével bõvölködöl és áldásával vagy teljes! Vedd birtokba a tengert és a délt.
\par 24 És Áserrõl monda: Áldott a többi fiak felett Áser! Legyen az õ atyjafiai elõtt kedves, és áztassa lábát olajban.
\par 25 Vas és réz legyenek a te záraid; és élteden át tartson erõd.
\par 26 Nincs olyan, mint Jesurun Istene! Az egeken száguld segítségedre, és fenségében a felhõkön.
\par 27 Hajlék az örökkévaló Isten, alant vannak örökkévaló karjai; elûzi elõled az ellenséget, és ezt mondja: Pusztítsd!
\par 28 És bátorságban lakozik Izráel, egymaga  lesz Jákób forrása a gabona és a bor földén, és az õ egei harmatot csepegnek.
\par 29 Boldog vagy Izráel! Kicsoda olyan mint te? Nép, a kit az Úr véd, a te segítségednek pajzsa, és a ki a te dicsõségednek fegyvere! Hízelegnek majd néked a te ellenségeid, és te azoknak magaslatait taposod.

\chapter{34}

\par 1 És felméne Mózes a Moáb mezõségérõl a Nébó hegyére, a Piszga tetejére, a mely átellenben van Jérikhóval; és megmutatá néki az Úr az egész földet, a Gileádot Dánig;
\par 2 És az egész Nafthalit, Efraim és Manassé földét, az egész Júda földét a túlsó tengerig;
\par 3 És a déli tartományt, és Jérikhónak, a pálmafák városa völgyének környékét, Czoárig.
\par 4 És monda néki az Úr: Ez a föld az, a mely felõl megesküdtem Ábrahámnak, Izsáknak, Jákóbnak, mondván: a te magodnak adom azt. Megengedtem néked, hogy szemeiddel lásd, de  oda nem mégy át.
\par 5 És meghala ott Mózes, az Úrnak szolgája a Moáb földén, az Úr szava szerint.
\par 6 És eltemeték õt a völgyben, a Moáb földén, Béth-Peórral átellenben; és senki sen tudja az õ temetésének helyét e mai napig.
\par 7 Mózes pedig száz és húsz esztendõs volt, mikor meghalt; nem homályosodott vala meg az õ szeme, sem el nem fogyatkozott vala az õ ereje.
\par 8 És siraták Izráel fiai Mózest a Moáb mezõségén harmincz napig; és eltelének a Mózes siratásának, azaz gyászolásának napjai.
\par 9 Józsué, a Nún fia pedig beteljesedék bölcseségnek lelkével; mert Mózes tette vala õ reá kezeit; és hallgatának reá Izráel fiai, és úgy cselekedének, a mint parancsolta vala az Úr Mózesnek.
\par 10 És nem támadott többé Izráelben olyan próféta, mint Mózes, a kit ismert volna az Úr színrõl-színre:
\par 11 Mindazokban a jelekben és csudákban, a melyekért küldötte vala õt az Úr, hogy véghez vigye azokat Égyiptom földén, a Faraón, minden õ szolgáján, és az õ egész földén;
\par 12 És mindama hatalmas erõben, és mindama nagy rettenetességben, a melyeket véghez vitt Mózes az egész Izráel szemei elõtt.


\end{document}