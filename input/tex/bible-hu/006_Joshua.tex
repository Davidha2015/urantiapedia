\begin{document}

\title{Józsue könyve}


\chapter{1}

\par 1 És lõn Mózesnek, az Úr szolgájának halála után, szóla az Úr Józsuénak,  a Nún fiának, Mózes szolgájának, mondván:
\par 2 Mózes, az én szolgám meghalt; most azért kelj fel, menj át ezen a Jordánon, te és mind ez a nép arra a földre, a melyet én adok nékik, az Izráel fiainak.
\par 3 Minden helyet, a melyet talpatok érint, néktek adtam, a miképen szólottam Mózesnek.
\par 4 A pusztától és a Libánontól fogva a nagy folyóvízig, az Eufrates folyóvízig, a Khitteusoknak egész földe és a nagy tengerig napnyugat felé lesz a ti határotok.
\par 5 Meg nem áll senki elõtted életednek minden idejében: a miképen Mózessel vele voltam, teveled is veled leszek; el nem hagylak téged, sem el nem maradok  tõled.
\par 6 Légy bátor és erõs, mert te teszed majd e népet annak a földnek örökösévé,  a mely felõl megesküdtem az õ atyáiknak, hogy nékik adom azt.
\par 7 Csak légy bátor és igen erõs, hogy vigyázz és mindent ama törvény szerint cselekedjél, a melyet Mózes, az én szolgám szabott elõdbe; attól se jobbra, se balra ne hajolj, hogy jó szerencsés lehess mindenben, a miben jársz!
\par 8 El ne távozzék e törvénynek könyve a te szádtól, hanem gondolkodjál arról éjjel és nappal, hogy vigyázz és mindent úgy cselekedjél, a mint írva van abban, mert akkor leszel jó szerencsés a te utaidon és akkor boldogulsz.
\par 9 Avagy nem parancsoltam-é meg néked: légy bátor és erõs? Ne félj, és ne rettegj,  mert veled lesz az Úr, a te Istened mindenben, a miben jársz.
\par 10 Parancsola azért Józsué a nép elõljáróinak, mondván:
\par 11 Menjetek által a táboron, és parancsoljátok meg a népnek, mondván: Készítsetek magatoknak útravalót, mert harmadnap mulva átmentek ti ezen a Jordánon, hogy bemenjetek és bírjátok azt a földet, a melyet az Úr, a ti Istenetek ád néktek birtokul.
\par 12 A Rúben és Gád nemzetségének pedig, és a Manassé nemzetség felének szóla Józsué, mondván:
\par 13 Emlékezzetek meg a dologról, a mit megparancsolt néktek Mózes, az Úrnak szolgája, mondván: Az Úr, a ti Istenetek megnyugtatott titeket, és néktek adta ezt a földet.
\par 14 Feleségeitek, gyermekeitek és barmaitok maradjanak e földön, a melyet adott néktek Mózes a Jordánon túl; ti pedig fegyveres kézzel menjetek át atyátokfiai elõtt, mindnyájan, a kik közületek erõs vitézek, és segéljétek meg õket;
\par 15 A míg megnyugosztja az Úr a ti atyátokfiait is, mint titeket, és bírják õk is a földet, a melyet az Úr, a ti Istenetek ád nékik: akkor aztán térjetek vissza a ti örökségtek földére és bírjátok azt, a melyet Mózes, az Úr szolgája adott néktek a Jordánon túl, napkelet felõl.
\par 16 Azok pedig felelének Józsuénak, mondván: Mindent megcselekszünk, a mit parancsoltál nékünk, és a hová küldesz minket, oda megyünk.
\par 17 A mint Mózesre hallgattunk, épen úgy hallgatunk majd te reád, csak legyen veled az Úr, a te Istened, a miképen vele volt Mózessel.
\par 18 Mindenki, a ki ellene szegül a te szódnak, és nem hallgat a te beszédedre mindabban, a mit parancsolsz néki, megölettessék. Csak bátor légy és erõs!

\chapter{2}

\par 1 És Józsué, a Nún fia, elkülde Sittimbõl titkon két férfiút kémekül, mondván: Menjetek el, tekintsétek meg azt a földet és Jérikhót. Azok pedig elmenének, és bemenének egy parázna asszonynak házába, akinek Ráháb vala neve, és ott hálának.
\par 2 Mikor pedig megjelentették ezt Jérikhó királyának, mondván: Íme, férfiak jöttek ide ez éjszaka az Izráel fiai közül e földnek kémlelésére;
\par 3 Akkor külde Jérikhó királya Ráhábhoz, mondván: Hozd ki a férfiakat, a kik bementek hozzád, a kik házadba mentek, mert azért jöttek, hogy kikémleljék az egész földet.
\par 4 Az asszony pedig fogá a két férfiút, és elrejté vala õket, és monda: Úgy van! Bejöttek hozzám a férfiak, de azt sem tudom, honnan valók voltak.
\par 5 És kimentek e férfiak kapuzáráskor a setétben; nem tudom, hová mentek a férfiak; siessetek gyorsan utánok, mert utólérhetitek õket.
\par 6 Pedig õ felhágatta õket a házhéjára, és elbujtatta õket a száras len közé, a mely néki a házhéjára volt kirakva.
\par 7 És utánok sietének a férfiak a Jordánhoz vivõ úton a rév felé; a kaput pedig bezárták, azután, hogy kimenének azok, akik õket üldözik vala.
\par 8 Õk pedig még le sem feküvének, a mikor felméne õ hozzájok az asszony a házhéjára,
\par 9 És monda a férfiaknak: Tudom, hogy az Úr néktek adta ezt a földet, és hogy megszállt minket a félelem miattatok, és hogy e földnek minden lakosa megolvad elõttetek.
\par 10 Mert hallottuk, hogy megszárította az Úr a Veres tenger vizét elõttetek, a mikor kijöttetek Égyiptomból, és hogy mit cselekedtetek az Emoreusok két királyával  a kik túl voltak a Jordánon, Szíhonnal és Óggal, a kiket megöltetek.
\par 11 És a mint hallottuk, megolvadott a mi szívünk, és nem támadt többé bátorság senkiben sem miattatok. Bizony az Úr, a ti Istenetek  az Isten fenn az égben és alant a földön!
\par 12 Most azért esküdjetek meg kérlek, nékem az Úrra, hogy a mint én irgalmasságot cselekedtem veletek, ti is irgalmasságot cselekesztek majd az én atyámnak házával, és igaz jelt adtok nékem,
\par 13 Hogy életben hagyjátok az én atyámat és anyámat, férfi és nõtestvéreimet, és mindent, a mi az övék, és megmentitek a mi lelkünket a haláltól.
\par 14 És mondának néki a férfiak: A mi lelkünket érje helyettetek a halál, ha nem nem jelentitek ezt a mi dolgunkat. És ha nékünk adja az Úr ezt a földet, irgalmasságot és igazságot cselekszünk veled.
\par 15 Alábocsátá azért õket kötélen az ablakon (mert az õ háza a kõkerítés falán vala, és õ a kõkerítésen lakik vala).
\par 16 És monda nékik: A hegyre menjetek, hogy rátok ne találjanak az üldözõk, és ott rejtõzködjetek három napig, a míg visszatérnek az üldözõk; azután pedig menjetek a magatok útján.
\par 17 A férfiak pedig mondának néki: Ártatlanok leszünk ezen te esketésedtõl, a melylyel te megeskettél minket!
\par 18 Ha bejövünk mi erre a földre, kösd e veres fonalú zsinórt ahhoz az ablakhoz, a melyen alábocsátottál minket; atyádat, anyádat és atyádfiait pedig, és atyádnak egész háznépét gyûjtsd be magadhoz a házba.
\par 19 És akkor, ha akárki kijön a te házadnak ajtain, annak vére az õ fején lesz, mi pedig ártatlanok leszünk; mindannak vére pedig, a ki teveled lesz a házban, a mi fejünkön legyen, ha valakinek keze lenne azon.
\par 20 Ha pedig megjelented ezt a mi dolgunkat, akkor mi fel leszünk oldva az eskü alól, a melylyel te megeskettél minket.
\par 21 Monda pedig amaz: A mint szólottatok, úgy legyen! Ekkor elbocsátá õket, és elmenének; a veres zsinórt pedig reáköté az ablakra.
\par 22 És elmenének és eljutának a hegyre, és ott maradának három nap, mígnem visszatérének az  üldözõk. Keresék ugyanis õket az üldözõk minden úton, de nem találták vala.
\par 23 Akkor a két férfiú visszatére, és leszállának a hegyrõl, és átmenének a Jordánon, és eljutának Józsuéhoz, a Nún fiához, és elbeszélének néki mindent, a mi történt vala velök.
\par 24 És mondának Józsuénak: Bizony kezünkbe adta az Úr azt az egész földet; meg is olvadt már a földnek minden lakosa miattunk.

\chapter{3}

\par 1 Felkele azért Józsué jó reggel, és elindulának Sittimbõl, és eljutának a Jordánhoz, õ és Izráel fiai mindnyájan, és meghálának ott, mielõtt általmentek volna.
\par 2 Lõn pedig három nap mulva, hogy általmenének a vezérek a táboron;
\par 3 És parancsolának a népnek, mondván: Mihelyt meglátjátok az Úrnak, a ti Isteneteknek frigyládáját és a Lévi nemzetségébõl való papokat, a kik hordozzák azt, ti is induljatok meg a ti helyetekrõl és menjetek utána.
\par 4 Csakhogy legyen köztetek és a között mintegy kétezer singnyi távolság; közel ne menjetek ahhoz, hogy megismerhessétek az útat, a melyen mennetek kell, mert nem jártatok ezen az úton soha ez elõtt.
\par 5 Józsué pedig monda a népnek: Tisztítsátok meg magatokat, mert holnap az Úr csudákat cselekszik köztetek.
\par 6 A papoknak is szóla Józsué, mondván: Vegyétek fel a frigyládát, és menjetek át a nép elõtt. Felvevék azért a frigyládát, és mennek vala a nép elõtt.
\par 7 Az Úr pedig monda Józsuénak: E napon kezdelek téged felmagasztalni az egész Izráel szemei elõtt, hogy megtudják, hogy a miképen vele voltam Mózessel,  te veled is veled leszek.
\par 8 Te azért parancsolj a papoknak, a kik a frigyládát hordozzák, mondván: Mikor bementek a Jordán vizének szélébe, álljatok meg a Jordánban.
\par 9 Ekkor monda Józsué Izráel fiainak: Járuljatok ide, és halljátok meg az Úrnak, a ti Isteneteknek szavait!
\par 10 És monda Józsué: Ebbõl tudjátok meg, hogy az élõ Isten van köztetek, és hogy kétség nélkül elûzi elõletek a Kananeust, a Khitteust, a Khivveust, a Perizeust, a Girgazeust, az Emoreust és a  Jebuzeust:
\par 11 Íme az egész föld Urának frigyládája elõttetek megy át a Jordánon!
\par 12 Most azért válaszszatok magatoknak tizenkét férfiút Izráel nemzetségeibõl, egy-egy férfiút egy-egy nemzetségbõl.
\par 13 És mihelyt megnyugosznak majd a Jordán vizében a papok talpai, a kik az Úrnak, az egész föld Urának ládáját hordozzák: Jordán vize ketté szakad,  és a felülrõl aláfolyó víz megáll egy rakásban.
\par 14 És lõn, hogy a mint megindula a nép az õ sátraiból, hogy általmenjen a Jordánon, és a papok, a frigyládának hordozói, a nép elõtt:
\par 15 És a mint a láda hordozói a Jordánhoz jutának, és a ládahordozó papok bemárták lábaikat a víznek szélébe ( a Jordán pedig az egész aratási idõ alatt telve vala minden õ partja felett):
\par 16 Megálla a víz, a mely felülrõl foly vala alá, és álla egy rakásban, nagy messzire Ádám városánál, a mely Czarthan mellett vala; a puszta tengere, a Sóstenger felé aláfolyó víz pedig egészen elfuta, és általméne a nép Jérikhó elõtt.
\par 17 A papok pedig, az Úr frigyládájának hordozói, ott állának a szárazon a Jordán közepében bátorsággal, és az egész Izráel szárazon megy vala át, mindaddig, míg az egész nép teljesen általméne a Jordánon.

\chapter{4}

\par 1 Mikor pedig az egész nép teljesen általment vala a Jordánon, szóla az Úr Józsuénak, mondván:
\par 2 Vegyetek magatokhoz a népbõl tizenkét férfiút, egy-egy férfiút egy-egy nemzetségbõl.
\par 3 És parancsoljátok nékik, mondván: Vegyetek fel innét a Jordán közepébõl, arról a helyrõl, a hol bátorsággal állanak a papok lábai, tizenkét követ, és vigyétek át azokat magatokkal, és tegyétek le a szálláson, a hol megháltok ez éjszaka.
\par 4 Elõhivá azért Józsué a tizenkét férfiút, a kikez az Izráel fiai közül rendelt vala, egy-egy férfiút egy-egy nemzetségbõl;
\par 5 És monda nékik Józsué: Menjetek át az Úrnak, a ti Isteneteknek frigyládája elõtt a Jordán közepébe, s vegyen fel mindegyikõtök egy-egy követ az õ vállára, az Izráel fiai nemzetségeinek száma szerint,
\par 6 Hogy legyen ez jelül köztetek. Ha kérdezik majd ezután a ti fiaitok, mondván: Mire valók néktek ezek a kövek?
\par 7 Mondjátok meg nékik, hogy kétfelé vált a Jordánnak vize az Úr frigyládája elõtt, mikor általment a Jordánon; ketté vált a Jordánnak vize, és ezek a kövek emlékeztetõül lesznek az Izráel fiainak mindörökre.
\par 8 És úgy cselekedének Izráel fiai, a mint megparancsolta vala Józsué, és felvevének a Jordán közepébõl tizenkét követ, a mint megmondotta vala az Úr Józsuénak, az Izráel fiai nemzetségeinek száma szerint, és általhozták magokkal az éjjeli szállásra és ott letevék azokat.
\par 9 Tizenkét követ állíta fel Józsué a Jordán közepén is, azon a helyen, a hol állottak vala a frigyládát vivõ papok lábai; ott is vannak mind e mai napig.
\par 10 A papok pedig, a ládának hordozói, állanak vala a Jordán közepén, a míg végbe mene mind az a dolog, a mit parancsolt vala az Úr Józsuénak, hogy mondja meg a népnek, egészen úgy, a mint Mózes parancsolta vala Józsuénak. A nép pedig sietett és általméne.
\par 11 És a mint véget ére az egész nép általmenetele, általméne az Úr ládája is és a papok is, a nép elõtt.
\par 12 Általmentek vala a Rúben fiai, a Gád fiai és Manassé félnemzetsége is felfegyverkezve az Izráel fiai elõtt, a miképen szólt vala  nékik Mózes.
\par 13 Mintegy negyven ezernyi fegyveres vitéz ment vala át az Úr elõtt a harczra, Jérikhónak sík mezejére.
\par 14 Azon a napon felmagasztalá az Úr Józsuét az egész Izráel szemei elõtt, és félék õt, a mint félték vala Mózest, életének minden napjaiban.
\par 15 Mert szólt vala az Úr Józsuénak, mondván:
\par 16 Parancsold meg a papoknak, a kik a bizonyság ládáját hordozzák, hogy jõjjenek fel a Jordánból.
\par 17 És parancsola Józsué a papoknak, mondván: Jõjjetek fel a Jordánból!
\par 18 És lõn, hogy a mint a papok, az Úr frigyládájának hordozói, feljöttek vala a Jordán közepébõl, és érintik vala a papok talpai a szárazt: visszatére a Jordán vize az õ helyére, és folyt vala, mint az elõtt, minden partja felett.
\par 19 A nép pedig az elsõ hónak tizedik napján jöve fel a Jordánból, és tábort üte Gilgálban, Jérikhónak keleti határán.
\par 20 Azt a tizenkét követ is, a melyeket a Jordánból hoztak vala, Gilgálban állatá fel Józsué.
\par 21 És szóla Izráel fiainak, mondván: Ha fiaitok kérdezik majd apáiktól, mondván: Mire valók ezek a kövek?
\par 22 Tudassátok majd a ti fiaitokkal, mondván: Szárazon jött át Izráel ezen a Jordánon.
\par 23 Mert kiszárította az Úr, a ti Istenetek a Jordán vizét ti elõttetek, míg általjövétek rajta, a miképen cselekedett vala az Úr, a ti Istenetek a Veres tengerrel, a melyet megszárított elõttünk; míg általjövénk rajta.
\par 24 Hogy megismerje a földnek minden népe az Úrnak kezét, hogy bizony erõs az; hogy féljétek az Urat, a ti Isteneteket minden idõben.

\chapter{5}

\par 1 Lõn pedig, a mint meghallák az Emoreusok minden királyai, a kik a Jordánon túl laknak napnyugot felé, és a Kananeusok minden királyai, a kik a tenger mellett laknak vala, hogy kiszáraztotta az Úr a Jordánnak vizét az Izráel fiai elõtt, a míg általjöttünk vala; megolvada az õ szívök,  és nem vala többé bátorság bennök Izráel fiai miatt.
\par 2 Ez idõben monda az Úr Józsuénak: Csinálj magadnak kõkéseket és másodszor  is metéld körül az Izráel fiait.
\par 3 És csinála Józsué magának kõkéseket, és körülmetélé Izráel fiait a körülmetéletlenség halmán.
\par 4 Az ok pedig, a miért körülmetélé õket Józsué, ez: Mindaz a nép, a mely kijött vala Égyiptomból; a férfiak, a hadakozó emberek mindnyájan meghaltak vala a pusztában, útközben, a míg jöttek vala Égyiptomból.
\par 5 Mert körül volt ugyan metélve mindaz a nép, a mely kijött vala; de azt a népet, a mely a pusztában született, útközben, a míg jöttek vala Égyiptomból, nem metélték vala körül;
\par 6 Mert negyven esztendeig jártak Izráel fiai a pusztában, mialatt elemészteték  a hadakozó férfiaknak egész népsége, a kik Égyiptomból jöttek vala ki, mivelhogy nem hallgattak vala az Úr szavára; a kiknek megesküdt az Úr, hogy nem láttatja meg velök a földet, a mely felõl megesküdt vala az Úr az õ atyáiknak, hogy nékünk adja azt a tejjel és mézzel folyó földet.
\par 7 Fiaikat állította vala azért helyökbe, ezeket metélé körül Józsué, mert körülmetéletlenek valának, mivelhogy nem metélék vala körül õket az útban.
\par 8 Miután pedig mind az egész nép körülmetéltetett vala, veszteg lõnek az õ helyökön a táborban, míglen meggyógyulának.
\par 9 És monda az Úr Józsuénak: Ma fordítottam el rólatok Égyiptom gyalázatát; ezért hívják e hely nevét Gilgálnak mind a mai napig.
\par 10 És Gilgálban táborozának Izráel fiai, és megkészíték a páskhát a hónapnak tizennegyedik napján, estve, Jérikhónak mezején.
\par 11 És evének a föld termésébõl a páskhának másodnapján kovásztalan kenyeret és pörkölt gabonát, ugyanezen a napon.
\par 12 És másnaptól kezdve, hogy ettek vala a föld termésébõl, megszûnék a manna, és nem volt többé mannájok az Izráel fiainak; hanem Kanaán földének gyümölcsébõl evének abban az esztendõben.
\par 13 Lõn pedig, a mikor Józsué Jérikhónál vala, felemelé az õ szemeit, és látá, hogy íme egy férfiú áll elõtte meztelen karddal a kezében. És hozzá méne Józsué, és monda néki: Közülünk való vagy-é te, vagy ellenségeink közül?
\par 14 Az pedig monda: Nem, mert én az Úr seregének fejedelme vagyok, most jöttem. És leborula Józsué a földre arczczal, és meghajtá magát, és monda néki: Mit szól az én Uram az õ szolgájának?
\par 15 És monda az Úr seregének fejedelme Józsuénak: Oldd le a te saruidat lábaidról, mert szent a hely, a melyen állasz. És úgy cselekedék Józsué.

\chapter{6}

\par 1 Jérikhó pedig be- és elzárkózott vala az Izráel fiai miatt, se ki nem jöhetett, se be nem mehetett senki.
\par 2 És monda az Úr Józsuénak: Lásd! kezedbe adtam Jérikhót és királyát a sereg vitézeivel együtt.
\par 3 Azért járjátok körül a várost mind ti hadakozó emberek, megkerülvén egyszer a várost. Így cselekedjél hat napon át.
\par 4 És hét pap hordozzon hét kos-szarvból való kürtöt a láda elõtt; a hetedik napon azonban hétszer kerüljétek meg a várost, a papok pedig kürtöljenek a kürtökkel.
\par 5 És ha majd belefúnak a kos-szarvba, mihelyt meghalljátok a kürtnek szavát, kiáltson fel az egész nép nagy kiáltással, és leszakad a város kõfala magától, és felmegy arra a nép, kiki az elõtte való helyen.
\par 6 Elõhívá azért Józsué, a Nún fia, a papokat és monda nékik: Vegyétek fel a frigyládát, hét pap pedig vigyen hét kos-szarvból való kürtöt az Úr ládája elõtt.
\par 7 A népnek pedig monda: Menjetek el, és kerüljétek meg a várost, a fegyveresek pedig menjenek az Úr ládája elõtt.
\par 8 És úgy lõn, a mint mondotta vala Józsué a népnek. A hét pap ugyanis, a kik a kos-szarból való kürtöt vivék, az Úr elõtt megy vala, és kürtölt vala a kürtökkel, az Úrnak frigyládája pedig utánok megy vala.
\par 9 A fegyveresek pedig elõttök mennek vala a kürtölõ papoknak, és a köznép követi vala a ládát, menvén és kürtölvén kürtökkel.
\par 10 A népnek pedig parancsolt vala Józsué, mondván: Ne kiáltsatok, hangotokat se hallassátok, és szó se jõjjön ki szátokból addig a napig, a míg azt mondom néktek: Kiáltsatok; és akkor kiáltsatok.
\par 11 Körüljárák azért az Úrnak ládájával a várost, egyszer megkerülvén; azután visszatérének a táborba, és az éjszakát a táborban tölték.
\par 12 Józsué pedig felkele jó reggel, és felvevék a papok az Úrnak ládáját.
\par 13 És a hét pap, a kik a kos-szarvból való hét kürtöt vivék, az Úr ládája elõtt megy vala folyton, és kürtöl vala a kürtökkel, a fegyveresek pedig elõttök mennek vala, és a köznép követi az Úrnak ládáját, menvén és kürtökkel kürtölvén.
\par 14 A második napon is egyszer kerülék meg a várost, azután visszatérének a táborba. Így cselekedének hat napon át.
\par 15 És lõn a hetedik napon, hogy felkelének, mihelyt a hajnal feljöve, és megkerülék a várost a szokott módon hétszer; csak ezen a napon kerülék meg a várost hétszer.
\par 16 És lõn, hogy a hetedik forduláskor kürtölnek vala a papok a kürtökkel, Józsué pedig monda a népnek: Kiáltsatok, mert néktek adta az Úr a várost!
\par 17 És legyen a város maga, és minden, a mi benne van, teljesen az Úrnak szentelve; csak a parázna Ráháb maradjon életben, õ és mindazok, a kik vele vannak a házban, mert elrejtette a követeket, a kiket küldöttünk volt.
\par 18 Mindazáltal ti óvjátok meg magatokat a teljesen Istennek szentelt dolgoktól, hogy miután néki szentelitek, el ne vegyetek a teljesen néki szentelt dolgokból, hogy Izráel táborát átkozottá ne tegyétek, és bajba ne keverjétek azt.
\par 19 Hanem mivel minden ezüst- és arany-, meg réz- és vasedény az Úrnak van szentelve, az Úrnak kincse közé jusson.
\par 20 Kiálta azért a nép, mihelyt kürtölének a kürtökkel. Lõn ugyanis, a mint meghallá a nép a kürtnek szavát, kiálta a nép nagy kiáltással, és leszakada a kõfal magától, és felméne a nép a városba, kiki az elõtte való helyen,  és bevevék a várost.
\par 21 És teljesen kipusztítának mindent, a mi csak vala a városban, a férfitól az asszonyig, a gyermektõl az öregig, sõt az ökörig, juhig és a szamárig, fegyver élivel.
\par 22 A két férfiúnak pedig, a kik megkémlelték vala a földet, monda Józsué: Menjetek be a parázna asszonynak házába és hozzátok ki onnét az asszonyt és mindazt, a mije van, a miképen megesküdtetek néki.
\par 23 Bemenének azért a kémlõ ifjak, és kihozák Ráhábot, és az õ atyját, anyját és az õ atyjafiait, és mindazt, a mije vala, és minden cselédjét is kihozák, és helyezék õket Izráel táborán kivül.
\par 24 A várost pedig megégeték tûzzel és mind azt, a mi benne vala; csakis az ezüstöt és aranyat és a réz- és vasedényeket rakták az Úr házának kincsei közé.
\par 25 A parázna Ráhábot pedig, és az õ atyjának háznépét és mindenét, a mije vala, élni hagyta vala  Józsué, és ott lakik az Izráel között mind e mai napig; mert elrejtette vala a követeket, a kiket küldött volt Józsué, hogy kikémleljék Jérikhót.
\par 26 És átkot szóla Józsué azon a napon, mondván: Átkozott legyen az Úr elõtt az a férfiú, a ki felkél, hogy megépítse e várost, Jérikhót! Az õ elsõ szülöttjére rakja le annak alapját s legifjabb fiára állítsa fel annak kapuit!
\par 27 És vala az Úr Józsuéval, és lõn híre az egész földön.

\chapter{7}

\par 1 De az Izráel fiai hûtlenül bántak vala a teljesen Istennek szentelt dolgokkal, mert elvõn a teljesen Istennek szentelt dolgokból Ákán, Kárminak  fia (ki a Zabdi fia, ki a Zéra fia a Júda nemzetségébõl); felgerjede azért az Úrnak haragja Izráel fiai ellen.
\par 2 Külde ugyanis Józsué férfiakat Jérikhóból, Aiba, a mely Bethaven mellett van, Bétheltõl napkelet felé, és szóla nékik, mondván: Menjetek fel és kémleljétek ki azt a földet. És felmenének a férfiak és kikémlelék Ait.
\par 3 Majd visszatérének Józsuéhoz, és mondának néki: Ne menjen fel az egész nép; mintegy kétezer férfi, vagy mintegy háromezer férfi menjen fel, és megverik Ait. Ne fáraszd oda az egész népet, hiszen kevesen vannak azok!
\par 4 Felméne azért oda a népbõl mintegy háromezer férfi; de elfutának Ai férfiai elõl.
\par 5 És megölének közülök Ai férfiai mintegy harminczhat férfit, és üldözék õket a kaputól kezdve egész Sébarimig, és levágták õket a lejtõn. Azért megolvada a népnek szíve, és lõn olyanná, mint a víz.
\par 6 Józsué pedig megszaggatá az õ ruháit, és földre borula arczczal az Úrnak ládája elõtt mind estvéig, õ és Izráel vénei és port hintének a fejökre.
\par 7 És monda Józsué: Ah Uram Istenem! Miért is hozád által ezt a népet a Jordánon, hogyha az Emoreus kezébe adsz minket, hogy elveszítsen? Vajha úgy akartuk volna, hogy maradtunk volna túl a Jordánon!
\par 8 Óh Uram! mit mondjak, miután meghátrált Izráel az õ ellenségei elõtt!
\par 9 Ha meghallják a Kananeusok és e földnek minden lakói, és ellenünk fordulnak, és kiirtják nevünket e földrõl: mit cselekszel majd a te nagy nevedért?
\par 10 És monda az Úr Józsuénak: Kelj fel! Miért is borulsz te arczra?
\par 11 Vétkezett Izráel, és általhágták szövetségemet is, a melyet rendeltem nékik, mert elvettek a teljesen nékem szentelt dolgokból is, és loptak is és hazudtak is, és edényeik közé is dugdostak.
\par 12 Ezért nem bírtak megállni Izráel fiai az õ ellenségeik elõtt, hátat fordítottak ellenségeiknek, mert átkozottakká lettek. Nem leszek többé veletek, ha ki nem vesztitek magatok közül azt a nékem szentelt dolgot.
\par 13 Kelj fel, és tisztítsd meg a népet, és mondjad: Tisztítsátok meg magatokat holnapra, mert ezt mondá az Úr, Izráelnek Istene: Istennek szenteld dolog van közötted, Izráel! Nem állhatsz meg a te ellenségeid elõtt, míg el nem távolítjátok közületek az Istennek szentelt dolgot.
\par 14 Azért jõjjetek elõ reggel nemzetségeitek szerint; a nemzetség pedig, a melyet bûnösnek jelent az Úr, jõjjön elõ családonként; a család pedig, a melyet bûnösnek jelent az Úr, jõjjön elõ házanként, a ház pedig, a melyet bûnösnek jelent az Úr, jõjjön elõ férfianként.
\par 15 És lészen, hogy a ki az Istennek szentelt dologban bûnösnek találtatik, tûzzel égettessék meg, õ és mindene, a mije van, mivelhogy megszegte az Úrnak szövetségét, és mivel alávaló dolgot cselekedett Izráelben.
\par 16 Felkele azért Józsué és jó reggel elõállítá Izráelt az õ nemzetségei szerint, és bûnösnek jelenteték a Júda nemzetsége.
\par 17 Ekkor elõállítá a Júda családjait, és bûnösnek jelenteték a Zéra családja; azután elõállítá a Zéra családját férfianként, és bûnösnek jelenteték a Zabdi háza.
\par 18 És elõállítá az õ házát férfianként, és bûnösnek találtaték Ákán, Kárminak fia, a ki Zabdi fia, a ki a Júda nemzetségébõl való Zérának fia.
\par 19 Monda azért Józsué Ákánnak: Fiam, adj dicsõséget, kérlek, az Úrnak, Izráel Istenének, és tégy néki vallást, és add tudtomra, kérlek, nékem, mit cselekedtél,  és el ne titkoljad tõlem!
\par 20 Ákán pedig felele Józsuénak, és monda: Bizony én vétkeztem az Úr ellen, Izráel Istene ellen, és ezt s ezt cselekedtem!
\par 21 Láték ugyanis a zsákmány közt egy jó babiloni köntöst, kétszáz siklus ezüstöt és egy arany vesszõt, a melynek súlya ötven siklus vala, és megkivántam ezeket, és elvevém ezeket, és ímé elrejtve vannak a földben, a sátoromnak közepében, az ezüst pedig alatta van.
\par 22 Ekkor követeket külde Józsué s ezek a sátorba futának, és ímé, elrejtve vala az az õ sátorában, és az ezüst is alatta vala.
\par 23 Kivivék azért azokat a sátor közepébõl, és vivék Józsuéhoz és Izráelnek minden fiához, és lerakák azokat az Úr elõtt.
\par 24 Józsué pedig fogá Ákánt, a Zéra fiát, az ezüstöt, a köntöst és az aranyvesszõt, az õ fiait és leányait, az õ ökreit, szamarait és juhait, sátorát és mindent, a mije vala, és vele lévén az egész Izráel is, vivék azokat Akor völgyébe.
\par 25 És monda Józsué: Miért rontottál meg minket? Rontson meg téged e napon az Úr! És elborítá õt egész Izráel kövekkel, és megégeték õket tûzzel, miután megkövezték vala õket.
\par 26 És nagy kõhalmot rakának feléje; megvan mind e napig. És megszûnék az Úr haragjának gerjedezése. Ezért nevezik ezt a helyet Akor völgyének mind e napig.

\chapter{8}

\par 1 És monda az Úr Józsuénak: Ne félj és ne rettegj! Vedd magadhoz mind a fegyverfogható népet, és kelj fel és menj fel Aiba, meglásd: kezedbe adom Ainak királyát és az õ népét, városát és földét.
\par 2 Úgy cselekedjél Aival és az õ királyával, a mint cselekedtél Jérikhóval és az õ királyával; de zsákmányolni valóját és barmait magatoknak zsákmányolhatjátok! Vess lest a városnak, annak háta megett!
\par 3 Felkele azért Józsué és az egész fegyverfogható nép, hogy felmenjenek Aiba, és kiválaszta Józsué harminczezer erõs férfiút és elküldé õket éjjel.
\par 4 És parancsola nékik, mondván: Vigyázzatok! Ti lest vettek a városnak, a város háta megett; igen messzire ne menjetek a várostól, és mindnyájan készen legyetek!
\par 5 Én pedig és az egész nép, a mely velem van, megközelítjük a várost. És ha kijönnek ellenünk, mint elõször, akkor megfutamodunk elõttök.
\par 6 És utánunk jönnek, mígnem elszakasztjuk õket a várostól, mert azt fogják mondani: Futnak elõlünk, mint elõször. És a míg futunk elõttök,
\par 7 Ti támadjatok elõ a leshelybõl, és foglaljátok el a várost, mert az Úr, a ti Istenetek adja azt a ti kezetekbe.
\par 8 És ha majd beveszitek a várost, gyújtsátok fel a várost tûzzel. Az Úrnak szava szerint cselekedjetek, vigyázzatok, megparancsoltam néktek!
\par 9 Elküldé azért õket Józsué és elmenének a leshelyre, és megszállának Béthel és Ai között, Aitól napnyugat felé; Józsué pedig ez éjszakán a nép között hála.
\par 10 És felkele Józsué, jó reggel, és megszemlélé a népet, és felméne õ és Izráelnek vénei a nép elõtt Aiba.
\par 11 És felméne a fegyverfogható nép is mind, a mely vele vala, és elközelítének és jutának a város elé, és táborba szállának Aitól északra, a völgy pedig köztök és Ai közt vala.
\par 12 És võn mintegy ötezer férfiút, és lesbe állítá õket Béthel és Ai között a városnak nyugati részén.
\par 13 Így állíták fel a népet, az egész tábort, a mely Aitól észak felé, utócsapatja pedig a várostól napnyugot felé vala. Józsué pedig beméne ez éjszakán a völgynek közepébe.
\par 14 És lõn, hogy mikor meglátta vala Ainak királya, sietve felkelének, és kijövének a város férfiai Izráel ellen a harczra; õ és egész népe a megszabott helyre, a síkság elejére, mert nem tudja vala, hogy lest vetének néki a város háta mögött.
\par 15 Józsué pedig és az egész Izráel mintha megverettek volna elõttök futnak vala a puszta felé vivõ úton.
\par 16 És felriasztaték az egész nép, a mely a városban vala, hogy üldözze õket. És üldözék Józsué, és elszakadának a várostól.
\par 17 És nem marada ember Aiban, sem Béthelben, a ki nem jött volna Izráel után, és ott hagyák a várost kinyitva, és üldözék Izráelt.
\par 18 Az Úr pedig monda Józsuénak: Emeld fel a kopját, a mely kezedben van, Ai felé, mert kezedbe adom azt. És felemelé Józsué a kopját, a mely kezében vala, a város felé.
\par 19 A lesben levõk pedig nagy hamarsággal felkelének helyökrõl, és a mint felemelte vala kezét, futásnak eredének, és bemenének a városba, és bevevék azt, és nagy hamarsággal tûzbe boríták a várost.
\par 20 Ai férfiai pedig hátratekintének, és láták, hogy ímé a városnak füstje felszáll vala az ég felé, és hogy nincsen módjukban imide vagy amoda elfutni, mert a nép, a mely fut vala a pusztának, visszafordul vala az üldözõ felé.
\par 21 Józsué ugyanis és az egész Izráel látták vala, hogy a lesben levõk bevették a várost, és hogy a városnak füstje felszállott vala: visszafordulának azért és vágák Ainak férfiait.
\par 22 Amazok pedig a városból jövének ki ellenök, és így közben valának Izráelnek: ezek innen, amazok meg amonnan, és vágák õket mindaddig, a míg egy sem marada közülök  élve, vagy a ki elszaladt volna.
\par 23 Ainak királyát is elfogták élve, és elvivék õt Józsué elé.
\par 24 Mikor pedig leöldösé Izráel Ainak minden lakosát a mezõn, a pusztában, a hol üldözték vala õket, és mikor mindnyájan elhullottak vala az utolsóig fegyver éle alatt: visszafordula az egész Izráel Ai ellen, és vágá azt fegyver élével.
\par 25 Mindazok pedig, a kik e napon elhullottak, férfiak és asszonyok együtt, tizenkét ezeren valának; Ainak minden embere.
\par 26 Józsué ugyanis nem voná vissza a kezét, a melyet a kopjával együtt felemelt vala, míglen megölék Ainak minden lakosát.
\par 27 A barmot azonban és a mi zsákmányolni valója volt ennek a városnak, magának zsákmányolá el Izráel az Úr rendelete szerint. A mint utasította vala Józsuét.
\par 28 Ait pedig felgyujtatá Józsué és tevé örökkévaló kõhalommá, pusztassággá mind e napig.
\par 29 Ainak királyát pedig felakasztá fára és ott vala estvéig; de naplementekor parancsolt Józsué, és levevék annak holttestét a fáról, és veték a város kapujának bejáratához, és nagy rakás követ hordanak fölébe, mind e napig.
\par 30 Majd oltárt építe Józsué az Úrnak, Izráel Istenének Ebál hegyén.
\par 31 Miképen megparancsolta vala Mózes, az Úrnak szolgája Izráel fiainak, a mint meg van írva Mózes törvényének könyvében; oltárt ép kövekbõl, melyeket vas nem érintett és vivének arra egészen égõáldozatot az Úrnak, és áldozának  hálaáldozatokat.
\par 32 És felírá ott kövekre a Mózes törvényének mását, a melyet az írt vala Izráel fiai elé.
\par 33 Az egész Izráel pedig és az õ vénei, vezérei és bírái ott álltak vala kétfelõl a láda mellett, a Lévita-papok elõtt, a kik az Úrnak frigyládáját hordozták vala, úgy a jövevény, mint a benszülött, fele a Garizim hegye felé, fele pedig Ebál hegye felé, a mint megparancsolta Mózes az Úr szolgája, hogy megáldaná az Izráel népét elõször.
\par 34 Azután pedig felolvasta a törvénynek minden ígéjét, az áldást és az átkot, mind úgy, a mint meg van írva a törvény könyvében.
\par 35 Nem volt egy íge sem azok közül, a melyeket Mózes parancsolt vala, a melyet fel nem olvasott volna Józsué, az Izráelnek egész gyülekezete elõtt, még az asszonyok, gyermekek, jövevények elõtt is, a kik járnak vala közöttök.

\chapter{9}

\par 1 Lõn pedig, hogy mikor ezt meghallották mind ama királyok, a kik a Jordánon túl a hegyeken és síkon és a nagy tengernek egész partja-mentén valának a Libánon ellenében: a Khittheus, az Emoreus, a Kananeus, a Perizeus, a Khivveus és a Jebuzeus:
\par 2 Egybegyülekezének, hogy megvívjanak Józsuéval és Izráellel egy akarattal.
\par 3 De meghallák Gibeon lakosai is, a mit Józsué Jérikhóval és  Aival cselekedett vala.
\par 4 És õk is ravaszul cselekedének. Elmenének ugyanis és követekül adák ki magokat. Szerzének azért szamaraikra ócska zsákokat, és ócska, megrepedezett és összekötözött boros tömlõket;
\par 5 És ócska, megfoltozott sarukat lábaikra, és ócska ruhákat magokra; az útravaló kenyerök is mind száraz és penészes vala.
\par 6 Így menének el Józsuéhoz a táborba Gilgálba, és mondának néki és Izráel férfiainak: Messze földrõl jöttünk, most azért kössetek frigyet mi velünk.
\par 7 Izráel férfiai pedig mondának a Khivveusnak: Hátha közöttem lakol te; hogyan kössek azért frigyet te veled?
\par 8 Azok pedig mondának Józsuénak: Szolgáid vagyunk mi! És monda nékik Józsué: Kik vagytok, és honnan jöttetek?
\par 9 És mondának néki: Igen messze földrõl jöttek a te szolgáid, az Úrnak, a te Istenednek nevéért, mert hallottuk az õ hírét és mindazt, a mit Égyiptomban cselekedett;
\par 10 Mindazt is, a mit cselekedett az Emoreusok két királyával, a kik valának túl a Jordánon, Szíhonnal, Hesbon királyával és Óggal, Básán királyával, a ki Astarótban vala.
\par 11 Ezért szólának nékünk a mi véneink és földünk lakosai is mind, mondván: szerezzetek magatoknak eledelt az útra, és menjetek eléjök, és mondjátok nékik: Szolgáitok vagyunk mi, most azért kössetek frigyet mi velünk!
\par 12 Ez a mi kenyerünk meleg volt, a mikor eleségül elhoztuk a mi házainkból elindulván, hogy hozzátok jõjjünk: most ímé, száraz és penészes lett.
\par 13 Ezek a boros tömlõk is, a melyeket új korukban töltöttünk vala meg, íme, de megszakadoztak; e mi ruháink és saruink pedig megavultak az útnak igen hosszú volta miatt!
\par 14 És võnek a férfiak azoknak eledelébõl, az Úr tanácsát pedig nem kérték vala.
\par 15 És békességesen bánt velök Józsué, és frigyet köte velök, hogy életben hagyja õket, a gyülekezet fejedelmei pedig megesküdének nékik.
\par 16 De harmadnap múlva azután, hogy frigyet kötöttek vala velök, meghallják, hogy közel valók azok hozzájok, sõt közöttök lakoznak azok.
\par 17 Elindulának azért Izráel fiai, és eljutának azoknak városaihoz harmadnapon. Városaik pedig valának: Gibeon, Kefira, Beéróth és Kirjáth-Jeárim.
\par 18 De nem bánták õket Izráel fiai, mivelhogy megesküdtek vala nékik a gyülekezet fõemberei az Úrra, Izráel Istenére, és zúgolódék az egész gyülekezet a fõemberek ellen.
\par 19 Mondának azért mind a fõemberek az egész gyülekezetnek: Mi megesküdtünk nékik az Úrra, Izráel Istenére, most hát nem bánthatjuk õket.
\par 20 Ezt cselekedjük velök, hogy életben hagyjuk õket, és így nem lesz harag ellenünk az esküvésért, a melylyel megesküdtünk nékik.
\par 21 Mondának azért a fõemberek nékik: Ám éljenek! És lõnek favágóivá és vízhordozóivá az egész gyülekezetnek, a mint szólottak vala nékik a fõemberek.
\par 22 Hívatá ugyanis õket Józsué, és szóla nékik, mondván: Miért csaltatok meg minket, ezt mondván: Igen messzirõl valók vagyunk mi tõletek, holott ti közöttünk laktok?
\par 23 Most azért átkozottak legyetek, és ne fogyjon ki közületek a szolga, a favágó és vízhordó az én Istenemnek házába.
\par 24 Azok pedig felelének Józsuénak, és mondának: Mivelhogy nyilván tudtokra esett a te szolgáidnak az, a mit az Úr, a ti Istenetek parancsolt vala Mózesnek, az õ szolgájának, hogy néktek adja ezt az egész földet, és hogy kiirtja elõletek a földnek minden lakosát: igen féltettük  tõletek a mi életünket, ezért cselekedtük ezt a dolgot.
\par 25 És most ímé, a te kezedben vagyunk, a mint cselekedni jónak és igaznak tetszik elõtted, úgy cselekedjél mivelünk!
\par 26 És úgy cselekedék velök, és kiszabadítá õket Izráel fiainak kezébõl, és nem ölék meg õket.
\par 27 És tevé õket Józsué azon a napon favágókká és vízhordókká a gyülekezethez és az Úr oltárához, mind e napig, azon a helyen, a melyet választánd.

\chapter{10}

\par 1 Lõn pedig, hogy a mikor meghallá Adonisédek, Jeruzsálemnek királya, hogy bevette vala Józsué Ait, és elpusztította azt, és hogy a mint cselekedett vala Jérikhóval  és annak királyával, úgy cselekedett Aival és annak királyával, és hogy békességre léptek Gibeon lakói Izráellel, és közöttük vannak:
\par 2 Igen megijedének, mivelhogy nagy város vala Gibeon, olyan mint egy a királyi városok közül, sõt nagyobb vala az Ainál, férfiai pedig mind vitézek valának.
\par 3 Külde azért Adonisédek, Jeruzsálemnek királya Hohámhoz, Hebronnak királyához és Pireámhoz Jármutnak királyához és Jáfiához, Lákisnak királyához és Debirhez, Eglonnak királyához, mondván:
\par 4 Jõjjetek fel hozzám, és segéljetek meg engem, és verjük meg Gibeont, mert békességre lépett Józsuéval és Izráel fiaival!
\par 5 Összegyülének azért, és felméne az Emoreusoknak öt királya: Jeruzsálemnek királya, Hebronnak királya, Jármutnak királya, Lákisnak királya, Eglonnak királya, õk magok és minden seregök, és tábort ütének Gibeonnál, és hadakozának ellene.
\par 6 Küldének azért Gibeon férfiai Józsuéhoz a táborba, Gilgálba, mondván: Ne vond meg kezeidet a te szolgáidtól! Jõjj fel hozzánk hamar és ments meg minket, és segíts rajtunk, mert mind felgyûltek ellenünk az Emoreusok királyai, a kik a hegyen lakoznak.
\par 7 Felméne azért Józsué Gilgálból, õ maga és az egész hadakozó nép vele, és a seregnek minden vitéze.
\par 8 Monda pedig az Úr Józsuénak: Ne félj tõlök, mert kezedbe  adtam õket; senki sem áll meg közülök elõtted.
\par 9 És rájok töre Józsué nagy hirtelen, miután egész éjszaka ment vala Gilgálból.
\par 10 És megrettenté õket az Úr Izráel elõtt, és megveré õket Gibeonnál nagy vereséggel, és ûzé õket a Bethoronba vivõ úton, és vágá õket egészen Azekáig és Makkedáig.
\par 11 Mikor pedig futnak vala õk Izráel elõtt a bethoroni lejtõn, az Úr nagy köveket hullata rájok az égbõl egész Azekáig, és meghalának. Többen valának, a kik a jégesõ kövei miatt haltak vala meg, mint azok, a kiket fegyverrel öltek meg Izráel fiai.
\par 12 Akkor szóla Józsué az Úrnak azon a napon, a melyen odavetette az Úr az Emoreust Izráel fiai elé; ezt mondotta vala pedig Izráel szemei elõtt: Állj meg nap, Gibeonban, és hold az Ajalon völgyében!
\par 13 És megálla a nap, és vesztegle a hold is, a míg bosszút álla a nép az õ ellenségein. Avagy nincsen-é ez megírva a Jásár könyvében? És megálla a nap az égnek közepén és nem sietett lenyugodni majdnem teljes egy napig.
\par 14 És nem volt olyan nap, mint ez, sem annakelõtte, sem annakutána, hogy ember szavának engedett volna az Úr, mert az Úr hadakozik vala Izráelért.
\par 15 Ezután visszatére Józsué és vele az egész Izráel a táborba, Gilgálba.
\par 16 Ez az öt király pedig elfutott vala, és elrejtõzék Makkedában a barlangban.
\par 17 És megizenék ezt Józsuénak, mondván: Megtaláltatott az öt király, elrejtõzve a makkedai barlangban.
\par 18 És monda Józsué: Hengergessetek nagy köveket a barlang szájához és rendeljetek mellé férfiakat, hogy õrizzék õket.
\par 19 Ti pedig meg ne álljatok, nyomuljatok ellenségeitek után és vágjátok utócsapataikat, és ne engedjétek õket bejutni az õ városaikba, mert kezetekbe adta õket az Úr, a ti Istenetek.
\par 20 Minekutána pedig elvégezték Józsué és Izráelnek fiai azoknak igen nagy vereséggel való verését, egészen azok megsemmisítéséig, és az élvemaradtak a megerõsített városokba vonultak:
\par 21 Visszatére az egész nép Józsuéhoz a táborba, Makkedába békességgel; nyelvét se mozdította senki Izráel fiai ellen.
\par 22 És monda Józsué: Nyissátok fel a barlang száját, és hozzátok ki hozzám ezt az öt királyt a barlangból!
\par 23 És akképen cselekedének, és kihozák hozzá a barlangból ezt az öt királyt: Jeruzsálem királyát, Hebron királyát, Jármut királyát, Lákis királyát, Eglon királyát.
\par 24 Mikor pedig kihozták vala ezt az öt királyt Józsuéhoz, elõhívatá Józsué Izráelnek minden férfiát, és monda a hadakozó nép vezéreinek, a kik vele mentek vala: Jõjjetek elõ, tegyétek lábaitokat e királyoknak nyakára. Eljövének azért és tevék lábaikat azoknak nyakára.
\par 25 És monda nékik Józsué: Ne féljetek, és meg ne rettenjetek; legyetek bátrak és erõsek, mert ekképen cselekszik az Úr minden ellenségetekkel, a kik ellen ti hadakoztok.
\par 26 Azután pedig megveré õket Józsué, és megölé õket, és felakasztatá õket öt fára, és felakasztva maradtak a fákon mind estvéig.
\par 27 Lõn pedig a nap lementének idején, parancsola Józsué, és levevék õket a fákról,  és behányák õket a barlangba, a melyben elbújtak vala, és nagy köveket rakának a barlang szája elé, mind e napig.
\par 28 Makkedát is bevevé Józsué ugyanazon a napon, és fegyver élére hányá azt és annak királyát, és megölé õket és egy lelket sem engedett menekülni azokból, a melyek benne valának. Úgy cselekedék Makkedának királyával, a mint cselekedett vala Jérikhónak királyával.
\par 29 Általméne annakutána Józsué és vele az egész Izráel Makkedából Libnába, és hadakozék Libnával.
\par 30 És kezébe adá az Úr azt is Izráelnek, és annak királyát; és fegyver élére hányá azt, és egy lelket sem engede menekülni azokból a melyek benne valának. És úgy cselekedék annak királyával, a mint cselekedett vala Jérikhónak királyával.
\par 31 Libnából pedig általméne Józsué és vele az egész Izráel Lákisba, és táborba szálla mellette, és hadakozék ellene.
\par 32 És kezébe adá az Úr Izráelnek Lákist és bevevé azt másodnapon, és fegyver élére hányá azt, és minden lelket, a mely benne vala, egészen úgy, amint cselekedett vala Libnával.
\par 33 Akkor feljöve Hórám, Gézernek királya, hogy megsegélje Lákist, de megveré Józsué õt és az õ népét annyira, hogy egy menekülõt sem hagya meg néki.
\par 34 Lákisból pedig általméne Józsué és vele az egész Izráel Eglonba, és táborba szállának ellene, és hadakozának ellene.
\par 35 És bevevék azt ugyanazon a napon, és fegyver élére hányák azt; és megöle minden lelket, a mely benne vala, ugyanazon a napon, egészen úgy, amint cselekedett vala Lákissal.
\par 36 Felméne azután Józsué Eglonból és õ vele az egész Izráel Hebronba, és hadakozának az ellen.
\par 37 És bevevék azt, és fegyver élére hányák azt, és annak királyát és minden városát, és egy lelket sem engede menekülni azokból, a melyek benne valának, egészen úgy, amint cselekedett vala Eglonnal, elvesztvén azt és minden lelket, a mely benne vala.
\par 38 Fordula azután Józsué, és vele az egész Izráel Debirnek, és hadakozék az ellen.
\par 39 És bevevé azt és annak királyát és minden városát, és fegyver élére hányá õket, és megölének minden lelket, a mely benne vala, nem hagyott menekülni valót. A miképen cselekedett vala Hebronnal, úgy cselekedék Debirrel és annak királyával, avagy a miképen Libnával és annak királyával cselekedett vala.
\par 40 Megveré azért Józsué az egész földet; a hegységet és a déli vidéket, a síkságot és a lejtõket és mindazoknak királyait, menekülni valót sem hagyott; és megöle minden élõt, a mint megparancsolta  vala az Úr, az Izráelnek Istene.
\par 41 Megveré pedig õket Józsué Kádes Barneától fogva egész Gázáig, a Gósennek is minden földét, egész Gibeonig.
\par 42 És mindezeket a királyokat és azoknak földjét egy útban hódítá meg Józsué, mivelhogy az Úr, Izráelnek Istene hadakozott vala Izráelért.
\par 43 Azután visszatére Józsué és vele az egész Izráel a táborba, Gilgálba.

\chapter{11}

\par 1 Mikor pedig meghallotta ezt Jábin, Hásornak királya, külde Jobábhoz, Mádonnak királyához, és Simronnak királyához, És Aksáfnak királyához,
\par 2 És azokhoz a királyokhoz, a kik laknak vala észak felé a hegységben, és a pusztában Kinneróttól délre, és a síkságon, és Dór magaslatain a tenger felé;
\par 3 A Kananeushoz napkelet és napnyugat felé, és az Emoreushoz, a Khittheushoz, a Perizeushoz, a Jebuzeushoz a hegyek közé, és a Khivveushoz a Hermon alá, Mispának földére.
\par 4 És kijövének õk és velök az õ egész táboruk, sok nép, olyan sok, mint a föveny, a mely a tenger partján van, és igen sok ló és szekér.
\par 5 És összegyülének mindezek a királyok, és megindulának, és táborba szállának együttesen Méromnak vizeinél, hogy hadakozzanak Izráel ellen.
\par 6 Ekkor monda az Úr Józsuénak: Ne félj tõlök, mert holnap ilyenkorra mindnyájokat átdöfötten vetem az Izráel elé; lovaikat bénítsd meg, szekereiket  pedig égesd meg tûzzel.
\par 7 Elméne azért Józsué és vele az egész hadakozó nép azok ellen a Mérom vizeihez nagy hirtelen, és reájok rohanának.
\par 8 És adá õket az Úr Izráelnek kezébe, és verék õket és ûzék õket egészen a nagy Sidonig és Miszrefót-Majimig, és Mispának völgyéig napkelet felé, és leverék õket annyira, hogy senki sem maradt közülök életben.
\par 9 És úgy cselekedék velök Józsué, a mint megmondotta vala néki az Úr: az õ lovaikat megbénítá, szekereiket pedig tûzzel égeté el.
\par 10 Majd visszafordula Józsué ugyanazon idõben és bevevé Hásort, királyát pedig fegyverrel megölé (Hásor ugyanis mindezeknek az országoknak feje volt az elõtt);
\par 11 És levágának minden lelket, a mely benne vala, megölvén õket fegyver élével; nem maradt meg egy élõ sem; Hásort pedig tûzzel égeté meg.
\par 12 És e királyoknak minden városát és minden királyukat is meghódoltatá Józsué, és megölé õket fegyver élével, kipusztítván õket, a mint megparancsolta vala Mózes, az Úrnak szolgája.
\par 13 Csak épen azokat a városokat nem égeté meg Izráel, a melyek halmokon állottak vala, kivéve Hásort, egyedül ezt égeté meg Józsué.
\par 14 És e városoknak minden zsákmányolni valóját, és a barmokat is magoknak zsákmányolák el Izráel fiai; csak az embereket hányák mind fegyver élére, míglen kipusztíták õket. Nem hagytak meg egy élõt sem.
\par 15 A mint parancsolt az Úr Mózesnek, az õ szolgájának, úgy parancsolt Mózes  Józsuénak, és úgy cselekedék Józsué, semmit el nem hagyott mindabból, a mit az Úr parancsolt vala Mózesnek.
\par 16 És elfoglalá Józsué mindazt a földet, a hegységet, az egész déli vidéket, az egész Gósen földét, úgy a síkságot, mint a pusztát, és Izráel hegyét és annak síkságát.
\par 17 A kopasz hegytõl fogva, a mely Szeír felé emelkedik, egészen Baál-Gádig, a Libanon völgyében, a Hermon hegye alatt; királyaikat pedig mind elfogá és megveré és megölé õket.
\par 18 Sok napon át viselt hadat Józsué mindezekkel a királyokkal.
\par 19 Nem volt város, a mely békességre lépett volna Izráel fiaival, kivéve a Gibeonban lakó Khivveusokat; haddal vették azt meg mind.
\par 20 Mert az Úrtól volt az, hogy megkeményítvén szíveiket, haddal menjenek Izráel ellen, hogy eltörölje õket; hogy ne legyen nékik irgalom, hanem hogy elpusztítsa õket, a mint megparancsolta vala az Úr Mózesnek.
\par 21 Majd elméne Józsué ez idõben, és kiirtá az Anákokat a hegyek közül Hebronból, Debirbõl, Anábból és Júdának minden hegyébõl, és Izráelnek minden hegyébõl; városaikkal együtt törlé el õket Józsué.
\par 22 Nem maradtak Anákok Izráel fiainak földén, csak Gázában, Gáthban és Asdódban hagyattak meg.
\par 23 Elfoglalá azért Józsué az egész földet egészen úgy, a mint az Úr mondotta vala Mózesnek, és adá azt Józsué örökségül Izráelnek, osztályrészeikhez képest, nemzetségeik szerint. A föld pedig megnyugovék a harcztól.

\chapter{12}

\par 1 Ezek pedig ama földnek királyai, a kiket levertek Izráelnek fiai, és a kiknek földjét birtokba vették a Jordánon túl, napkelet felé, az Arnon pataktól fogva a Hermon hegyéig, és az egész mezõséget kelet felõl:
\par 2 Szíhon, az Emoreusok királya, a ki lakik vala Hesbonban, a ki uralkodik vala Aróertõl fogva, a mely van az Arnon patak partján, és a patak közepétõl és a Gileád felétõl  a Jabbok patakig, az Ammon fiainak határáig.
\par 3 És a mezõségtõl a Kinneróth tengerig napkelet felé, és a puszta tengeréig, a Sóstengerig napkelet felé, a Béth-Jesimothi útig, és délfelé a Piszga hegyoldalainak aljáig.
\par 4 És Ógnak a Básán királyának tartománya, a ki Refaim maradékai közül való, a ki Astarotban és Edreiben lakozik vala.
\par 5 És uralkodik vala Hermon hegyén, Szalkhában és az egész Básánban, a Gesurnak és Maakhátnak határáig, és a fél Gileádon, Szíhonnak, Hesbon királyának határáig.
\par 6 Mózes, az Úrnak szolgája és Izráelnek fiai verték le õket, és oda adta azt a földet Mózes, az Úrnak szolgája örökségül a Rúben és Gád nemzetségeknek és a Manassé nemzetség felének.
\par 7 Ezek pedig ama földnek királyai, a kiket levertek vala Józsué és az Izráelnek fiai a Jordán másik oldalán napnyugat felé, Baál-Gádtól fogva, a mely van a Libánon völgyében, egészen a  kopasz hegyig, a mely Szeír felé emelkedik. És oda adá azt Józsué örökségül az Izráel nemzetségeinek, az õ osztályrészeik szerint:
\par 8 A hegységben és a síkságon, a mezõségen és a hegyoldalakon, a pusztán és a déli vidéken, a  Khittheus, Emoreus, Kananeus, Perizeus, Khivveus és Jebuzeusok földjét.
\par 9 Jérikhónak királya egy; Ainak, a mely oldalra vala Béthel felé,  királya egy;
\par 10 Jeruzsálemnek királya egy, Hebronnak  királya egy;
\par 11 Jármutnak királya egy, Lákisnak királya egy;
\par 12 Eglonnak királya egy, Gézernek királya  egy;
\par 13 Debirnek királya egy, Gédernek királya egy;
\par 14 Hormáhnak királya egy, Aradnak királya egy;
\par 15 Libnának királya egy; Adullámnak királya egy;
\par 16 Makkedának királya egy, Béthelnek királya egy;
\par 17 Tappuáhnak királya egy, Héfernek királya egy;
\par 18 Afeknek királya egy, Lassáronnak királya egy;
\par 19 Mádonnak királya egy,  Hásornak királya egy;
\par 20 Simron Meronnak királya egy, Aksáfnak királya egy;
\par 21 Taanáknak királya egy, Megiddónak királya egy;
\par 22 Kedesnek királya egy, a Kármelen levõ Jokneámnak királya egy;
\par 23 A Dór magaslatán levõ Dórnak királya egy, a Gilgál népeinek királya egy;
\par 24 Tirczának királya egy; összesen harminczegy király.

\chapter{13}

\par 1 Mikor Józsué megvénhedett és igen megidõsödött vala, monda az Úr néki: Te megvénhedtél, igen megidõsödtél, pedig még igen sok föld maradt elfoglalni való.
\par 2 Ez az a föld, a mi fennmaradt: A Filiszteusoknak minden tartománya és az egész Gesur.
\par 3 A Sikhórtól fogva, a mely Égyiptom felett foly, egészen Ekronnak határáig északra, mely a Kananeushoz számíttatik; a Filiszteusok öt fejedelemsége: Gázáé, Asdódé, Askelóné, Gáthé, Ekroné és az Avveusoké.
\par 4 Délrõl a Kananeusok egész földe és Meára, a mely a Sídonbelieké, Afékáig, az Emoreusok határáig.
\par 5 Továbbá a Gibli földe és az egész Libanon napkelet felé, Baál-Gádtól fogva, a mely a Hermon hegy alatt van, egészen addig, a hol Hamáthba mennek.
\par 6 A hegységnek minden lakosát, a Libanontól Miszrefóth-Majimig, a Sídoniakat mind, magam ûzöm ki õket Izráel fiai elõl, csak sorsold ki Izráelnek örökségül, a mint megparancsoltam néked.
\par 7 Mostan azért oszd el ezt a földet örökségül kilencz nemzetségnek, és a Manassé nemzetség felének.
\par 8 Õ vele együtt a Rúben és Gád nemzetségek elvették örökségöket, a melyet adott vala nékik Mózes, túl a Jordánon, napkelet felé, a miképen adta vala nékik Mózes, az Úrnak szolgája.
\par 9 Aróertõl fogva, a mely az Arnon patak partján van, úgy a várost, a mely a völgynek közepette van, mint Medebának minden sík földét Dibonig.
\par 10 És minden városát Szíhonnak, az Emoreusok királyának, a ki uralkodik vala Hesbonban, az Ammon fiainak határáig.
\par 11 És Gileádot és Gesurnak és Maakátnak határát, az egész Hermon hegyet és az egész Básánt Szalkáig.
\par 12 Básánban Ógnak egész országát, a ki uralkodik vala Astarótban és Edreiben. Ez maradt vala meg a Refaim maradékai közül, de leveré és kiûzé õket Mózes.
\par 13 De Izráel fiai nem ûzék ki a Gesurit és a Maakhátit, és ott is lakik a Gesuri és Maakháti az Izráel között mind e mai napig.
\par 14 Csak Lévi nemzetségének nem adott örökséget; az Úrnak, Izráel Istenének tüzes áldozatai az õ öröksége, a mint szólott vala néki.
\par 15 Adott vala pedig Mózes örökséget a Rúben fiai nemzetségének az õ családjaik szerint.
\par 16 És lõn az õ határuk Aróertõl fogva, a mely az Arnon folyó partján van, úgy a város, a mely a völgy közepette van, mint az egész sík föld Medeba mellett;
\par 17 Hesbon és annak minden városa, a melyek a sík földön vannak; Dibon, Bámoth-Baal és Béth-Baál-Meon;
\par 18 És Jahcza, Kedemót és Méfaát;
\par 19 És Kirjáthaim, Szibma és Czeret-Sáhár a völgy mellett való hegyen:
\par 20 És Béth-Peór, a Piszga hegyoldalai, és Béth-Jesimóth.
\par 21 És a sík föld minden városa és Szíhonnak, az Emoreusok királyának egész országa, a ki uralkodik vala Hesbonban, a kit megvert vala Mózes, õt és a Midiánnak fejedelmeit: Evit és Rékemet, Czúrt és Húrt és Rébát, Szíhonnak fejedelmeit, a kik e földön laktak vala.
\par 22 A jövendõmondó Bálámot is, Beórnak fiát, megölék Izráel fiai fegyverrel, azokkal együtt, a kiket levágtak vala.
\par 23 Vala tehát a Rúben fiainak határa a Jordán és melléke. Ez a Rúben fiainak öröksége az õ családjaik szerint, a városok és azoknak falai.
\par 24 A Gád nemzetségének, a Gád fiainak is adott vala örökséget Mózes, az õ családjaik szerint.
\par 25 És lõn az õ határuk Jaázer és Gileádnak minden városa és az Ammon fiai földjének fele Aróerig, a mely Rabba felett van.
\par 26 És Hesbontól fogva Ramath-Miczpéig és Betónimig, meg Mahanáimtól Debir határáig.
\par 27 A völgyben pedig Béth-Harám, Béth-Nimra, Szukkóth és Czáfon, Szíhonnak, Hesbon királyának maradék országa, a Jordán és melléke, a Kinnereth tenger széléig a Jordánon túl napkelet felé.
\par 28 Ez a Gád fiainak örökségök, az õ családjaik szerint, a városok és azoknak falui.
\par 29 A Manassé nemzetség felének is adott vala Mózes örökséget. És lõn a Manassé fiainak félágáé, az õ családjaik szerint;
\par 30 És lõn az õ határuk: Mahanáimtól fogva az egész Básán, Ógnak, Básán királyának egész országa és Jairnak minden faluja, a melyek Básánban vannak, hatvan város.
\par 31 És pedig Gileádnak fele és Astarót és Edrei, Óg országának városai Básánban, a Manassé fiának, Mákirnak fiaié, Mákir feléé, az õ családjaik szerint.
\par 32 Ezek azok, a miket örökségül adott vala Mózes a Moáb mezõségén, a Jordánon túl Jérikhótól napkelet felé.
\par 33 A Lévi nemzetségének pedig nem adott vala Mózes örökséget. Az Úr, Izráelnek Istene az õ örökségök, a mint szólott vala nékik.

\chapter{14}

\par 1 Ezek pedig azok, a miket örökségül võnek el Izráel fiai a Kanaán földén, a miket örökségül adtak nékik Eleázár, a pap, Józsué, a Nún fia és az atyai fejedelmek, a kik valának Izráel fiainak nemzetségei felett;
\par 2 Sorsvetés által való örökségökül, (a mint megparancsolta vala az Úr Mózes által) a kilencz nemzetségnek és a félnemzetségnek:
\par 3 Mert két nemzetségnek és fél nemzetségnek a Jordánon túl adott vala Mózes örökséget, a Lévitáknak pedig nem adott vala örökséget õ közöttök.
\par 4 Mert a József fiai két nemzetség voltak: Manassé és Efraim; a Lévitáknak pedig nem adtak osztályrészt a földbõl, hanem csak városokat lakásul és az azokhoz való legelõket barmaik és marháik számára.
\par 5 A mint megparancsolta vala az Úr Mózesnek, úgy cselekedének az Izráel fiai, és úgy oszták fel a földet.
\par 6 Hozzámenének pedig Józsuéhoz Júdának fiai Gilgálba és monda néki a Kenizeus Káleb, Jefunné fia: Te tudod azt a dolgot, a melyet beszélt vala az Úr Mózesnek, az Isten emberének én felõlem és te felõled Kádes-Barneában.
\par 7 Negyven esztendõs valék én, mikor elküldött engem Mózes az Úrnak szolgája Kádes-Barneából, hogy kikémleljem a földet, és úgy hoztam néki hírt, a mint az én szívemben vala.
\par 8 Atyámfiai pedig, a kik feljöttek vala velem, elrémítették a népnek szívét, de én tökéletesen követtem az Urat, az én Istenemet.
\par 9 És megesküvék Mózes azon a napon, mondván: Bizony a föld, a melyet megtapodott a te lábad, tiéd lesz örökségül, és a te fiaidé mind örökké, mivelhogy tökéletesen követted az Urat, az én Istenemet.
\par 10 Most pedig, ímé megtartott engem az Úr életben, a mint szólott vala; most negyvenöt esztendeje, a mióta szólott vala az Úr e dologról Mózesnek, a mi alatt Izráel a pusztában bolyongott vala; és most ímé, nyolczvanöt esztendõs vagyok!
\par 11 Még ma is olyan erõs vagyok, a milyen azon a napon voltam, a mikor elküldött engem Mózes; a milyen akkor volt az én erõm, most is olyan az én erõm a harczoláshoz és járásra-kelésre.
\par 12 Most azért add nékem ezt a hegyet, a melyrõl szólt vala az Úr azon a napon; mert magad is hallottad azon a napon, hogy Anákok vannak ott, és nagy, erõsített városok; hátha velem lesz az Úr, és kiûzöm õket, a mint megmondotta az Úr.
\par 13 És megáldá õt Józsué, és odaadá Hebront Kálebnek, a Jefunné fiának örökségül.
\par 14 Azért lõn Hebron a Kenizeus Kálebé, a Jefunné fiáé, örökségül mind e mai napig, a miért hogy tökéletesen követte vala az Urat, Izráelnek Istenét.
\par 15 A Hebron neve pedig annakelõtte Kirjáth-Arba volt; a ki a legnagyobb ember volt az Anákok között. A föld pedig megnyugodott  a harcztól.

\chapter{15}

\par 1 A Júda fiai nemzetségének sors által való része pedig az õ családjaik szerint ez vala: Edomnak határa felé a Czin pusztája délre, a déli határnak végén.
\par 2 Vala pedig az õ déli határuk a Sóstengernek szélétõl, a tengernyelvtõl fogva, a mely délfelé fordul.
\par 3 És halad délre az Akrabbim hágónak, majd átmegy Czin felé, és felmegy délrõl Kádes-Barneának, átmegy Hesronnak, felmegy Adárnak és kerül Karka felé;
\par 4 Majd átmegy Asmonnak és halad Égyiptom patakának. A határ szélei pedig a tengernél vannak. Ez a ti határotok délre.
\par 5 Napkelet felé pedig a Sóstenger a határ a Jordán végéig; az északi rész határa pedig a tengernyelvtõl, a Jordán végétõl kezdõdik.
\par 6 És felmegy ez a határ Béth-Hoglának, és átmegy északra Béth-Arabán majd felmegy ez a határ Rúben fiának, Bohánnak kövéhez.
\par 7 És felmegy ez a határ Debirbe is az Akor völgyébõl, és északnak fordul Gilgál felé, a mely átellenében van az Adummim hágójának, a mely a pataktól délfelé esik. És átmegy a határ az Én-semes vizeire és tova halad a Rógel forrása felé.
\par 8 Azután felmegy a határ a Hinnom fiának völgyén, Jebuzeusnak, azaz Jeruzsálemnek déli oldala felé; felmegy továbbá e határ a hegynek tetejére, a mely átellenben van a Hinnom völgyével napnyugat felé, a mely északra van a Refaim völgyének szélén.
\par 9 És hajlik e határ a hegynek tetejétõl a Neftoáh víznek kútfejéhez és kimegy az Efron hegyének városai felé; majd hajlik e határ Baalának, azaz Kirjáth-Jeárimnak.
\par 10 Baalától pedig fordul e határ napnyugotnak a Szeír-hegy felé, és átmegy északnak a Jeárim-hegy oldala felé, azaz Kesalon felé és alámegy Béth-Semesnek és átmegy Timnának.
\par 11 Majd tova megy e határ Ekron északi oldala felé, és hajlik e határ Sikkeronnak, és átmegy a Baala hegynek, és tova megy Jabnéel felé. A határ szélei pedig a tengernél vannak.
\par 12 A napnyugati határ pedig a nagy tenger és melléke. Ez Júda fiainak határa köröskörül az õ házoknépe szerint.
\par 13 Kálebnek, a Jefunné fiának pedig a Júda fiai között ada részt, az Úrnak Józsuéhoz való szavai szerint; Kirjáth-Arbának, Anák atyjának városát, azaz Hebront.
\par 14 És kiûzé onnan Káleb Anáknak három fiát: Sésait, Ahimánt és Tálmait, Anák gyermekeit.
\par 15 És felméne innét Debir lakói ellen, Debirnek neve pedig azelõtt Kirjáth-Széfer volt.
\par 16 És monda Káleb: A ki megveri Kirjáth-Széfert és elfoglalja azt, néki adom Akszát, az én leányomat feleségül.
\par 17 Elfoglala pedig azt Othniél, Kénáznak, a Káleb testvérének fia; és néki adá Akszát, az õ leányát feleségül.
\par 18 És lõn, hogy a mikor eljöve az, biztatá õt, hogy kérjen az õ atyjától mezõt. Leszálla azért a szamárról; Káleb pedig monda néki: Mi bajod?
\par 19 Õ pedig monda: Adj áldást nékem! Mivelhogy száraz földre helyeztél engem, adj azért nékem vízforrásokat is. És néki adá a felsõ forrást és az alsó forrást.
\par 20 Ez a Júda fiai nemzetségének öröksége az õ családjaik szerint.
\par 21 A Júda fiai nemzetségének városai pedig a déli végtõl kezdve Edom határa felé valának: Kabseél, Éder és Jágur;
\par 22 Kina, Dimóna és Adada;
\par 23 Kedes, Hásor és Ithnán;
\par 24 Zif, Télem és Bealóth;
\par 25 Hásor-Hadatha és Kerioth-Hesron, azaz Hásor;
\par 26 Amam, Séma és Móláda;
\par 27 Hasar-Gaddah, Hesmón és Béth-Pelet;
\par 28 Hasar-Suál, Beer-Seba és Bizjotheja;
\par 29 Baála, Ijjim és Eczem;
\par 30 Elthólád, Keszil és Hormah;
\par 31 Siklág, Madmanna és Szanszanna;
\par 32 Lebaóth, Silhim, Ain és Rimmon. Összesen huszonkilencz város és ezek falui.
\par 33 A síkságon; Esthaól, Czórah és Asnáh;
\par 34 Zanoah, Én-Gannim, Tappuáh és Énám;
\par 35 Jármut, Adullám, Szókó és Azéka;
\par 36 Saáraim, Adithaim, Gedéra és Gederóthaim. Tizennégy város és azok falui.
\par 37 Senán, Hadása és Migdal-Gad;
\par 38 Dilán, Miczpe és Jokteél;
\par 39 Lákis, Boczkát és Eglon;
\par 40 Kabbon, Lahmász és Kitlis;
\par 41 Gedéróth, Béth-Dágon, Naama és Makkéda. Tizenhat város és ezeknek falui.
\par 42 Libna, Ether és Asán;
\par 43 Jifta, Asná és Neczib;
\par 44 Keila, Akzib és Marésa. Kilencz város és ezeknek falui.
\par 45 Ekron, ennek mezõvárosai és falui.
\par 46 Ekrontól fogva egész a tengerig mind azok, a melyek Asdód mellett vannak, és azoknak falui.
\par 47 Asdód, ennek mezõvárosai és falui; Gáza, ennek mezõvárosai és falui; Égyiptom patakjáig, és a nagy tenger és melléke.
\par 48 A hegységen pedig: Sámír, Jathír és Szókó;
\par 49 Danna, Kirjáth-Szanna, azaz Debir;
\par 50 Anáb, Estemót és Anim;
\par 51 Gósen, Hólon és Giló. Tizenegy város és ezeknek falui.
\par 52 Aráb, Dúma és Esán;
\par 53 Janum, Béth-Tappuah és Aféka;
\par 54 Humta, Kirjáth-Arba, azaz Hebron és Czihor. Kilencz város és ezeknek falui.
\par 55 Maón, Karmel, Zif és Júta;
\par 56 Jezréel, Jokdeám és Zánoah;
\par 57 Kajin, Gibea és Timna. Tíz város és ezeknek falui.
\par 58 Halhul, Béth-Czúr és Gedor;
\par 59 Maarát, Béth-Anóth és Elthekon. Hat város és ezeknek falui.
\par 60 Kirjáth-Baál, azaz Kirjáth-Jeárim, és Rabba. Két város és ezeknek falui.
\par 61 A pusztában: Béth-Arabá, Middin és Szekáka;
\par 62 Nibsán, Ir-Melah és Én-Gedi. Hat város és ezeknek falui.
\par 63 De a Jebuzeusokat, Jeruzsálemnek lakóit, Júda fiai nem bírták kiûzni, azért laknak ott a Jebuzeusok Júda fiaival együtt Jeruzsálemben, mind e mai napig.

\chapter{16}

\par 1 A József fiainak sors által való része pedig juta a jérikhói Jordántól fogva, Jérikhó vizei felé napkeletnek a pusztára, a mely felmegy Jérikhótól a Béthel hegyének.
\par 2 És tovamegy Béthelbõl Lúzba, és átmegy az Arkiták határára, Ataróthra.
\par 3 Majd lemegy a tenger felé a Jafleteus határának, az alsó Bethoronnak határáig és Gézerig, a szélei pedig a tengernél vannak.
\par 4 Elvevék azért az õ örökségöket Józsefnek fiai: Manassé és Efraim.
\par 5 Efraim fiainak határa is az õ családjaik szerint vala, és pedig az õ örökségöknek határa napkelet felé, Atróth-Adártól felsõ Bethoronig vala.
\par 6 És kimegy a határ a tengerre; Mikhmethattól észak felé, és fordul a határ kelet felé Thaanath-Silónak és átmegy azon napkelet felé Janoáhnak.
\par 7 Janoáhtól pedig lemegy Ataróthba és Naaróthba és éri Jérikhót, és kimegy a Jordánnak.
\par 8 Tappuahtól tovamegy a határ a tenger felé a Kána patakjának, a szélei pedig a tengernél vannak. Ez az Efraim fiai nemzetségének öröksége az õ családjaik szerint.
\par 9 És a városok, a melyek kiválasztattak Efraim fiai számára a Manassé fiai örökségének közepette, mind e városok és ezeknek falui.
\par 10 De ki nem ûzék a Kananeust, a ki lakik vala Gézerben;  azért ott lakik a Kananeus az Efraim között mind e napig, és lõn robotos szolgává.

\chapter{17}

\par 1 Lõn sors által való része Manassé nemzetségének is, mert õ vala elsõszülötte Józsefnek, Mákirnak, a Manassé  elsõszülöttének, Gileád atyjának, mivelhogy hadakozó férfiú vala, juta néki Gileád és Básán.
\par 2 Manassé többi fiainak is juta az õ családjaik szerint: Abiézer fiainak, Hélek fiainak, Aszriél fiainak, Sekem fiainak, Héfer fiainak és Sémida fiainak. Ezek Manassénak a József fiának fiúgyermekei, az õ családjaik szerint.
\par 3 De Czélofhádnak, Héfer fiának (ez Gileád fia, ez Mákir fia, ez Manassé fia) nem valának fiai, hanem csak leányai. Ezek valának pedig leányainak nevei: Makhla, Nóa, Khogla, Milkha és Thircza.
\par 4 Ezek odajárulának Eleázár pap elé, Józsuénak, a Nún fiának eleibe és a fejedelmek elé, mondván: Az Úr megparancsolta Mózesnek, hogy örökséget adjon nékünk a mi atyánkfiai között. Adott vala azért nékik örökséget az Úr beszéde szerint, az õ atyjoknak bátyjai között.
\par 5 Tíz rész esék azért Manassénak, Gileád és Básán földén kívül, a melyek túl vannak a Jordánon.
\par 6 Mert a Manassé leányai örökséget kaptak az õ fiai között; Gileád földe pedig a Manassé fiaié lõn.
\par 7 A Manassé határa pedig Ásertõl Mikmetháth felé vala, a mely Sikem elõtt van. És méne a határ jobbkéz felé, Én-Tappuahnak lakói felé.
\par 8 Manasséé volt a Tappuah földe, de Tappuah a Manassé határa felé az Efraim fiaié vala.
\par 9 És alámegy a határ a Kána patakjára, a pataknak déli részére. Ezek a városok Efraiméi a Manassé városai között. Manassé határa pedig a pataktól észak felé és a széle a tengernél vala.
\par 10 Délrõl Efraimé, északról pedig Manasséé, és ennek határa a tenger vala. De észak felõl Áserbe ütköznek, napkelet felõl pedig Issakhárba.
\par 11 És a Manasséé vala Issakhárban és Áserben: Béth-Seán és mezõvárosai, Jibleám és mezõvárosai, Dórnak  lakosai és mezõvárosai, Én-Dórnak lakosai és mezõvárosai, Thaanaknak lakosai és mezõvárosai, és Megiddó lakosai és mezõvárosai, három hegyi tartomány.
\par 12 De nem bírták elfoglalni Manassénak fiai ezeket a városokat, és sikerült a Kananeusnak ott maradni azon a földön.
\par 13 De mihelyt megerõsödtek vala Izráel fiai, robot alá veték a Kananeust, de teljességgel nem ûzék ki õket.
\par 14 József fiai pedig szóltak vala Józsuéval, mondván: Miért adtál nékem egy sors szerint való örökséget és egy osztályrészt, holott én sok nép vagyok, mivelhogy ez ideig megáldott engem az Úr?!
\par 15 Józsué pedig monda nékik: Ha sok nép vagy te, menj fel az erdõre, és írts ott magadnak a Perizzeusoknak és Refaimnak földén, ha szoros néked az Efraim hegye.
\par 16 És mondának a József fiai: Nem elegendõ nékünk ez a hegy, a Kananeusoknak pedig, a kik a völgyi síkon laknak, mindnek vas-szekerök van; úgy azoknak, akik Béth-Seanban és mezõvárosaiban, mint azoknak, a kik a Jezréel völgyében vannak.
\par 17 Szóla pedig Józsué a József házának, Efraimnak és Manassénak, mondván: Sok nép vagy te és nagy erõd van néked, nem lesz néked csak egy sors szerint való részed;
\par 18 Mert hegyed lesz néked. Ha erdõ az, úgy írtsd ki azt, és annak szélei is a tieid lésznek; mert kiûzöd a Kananeust, noha vas-szekere van néki, s noha erõs az.

\chapter{18}

\par 1 Izráel fiainak egész gyülekezete pedig összegyülekezék Silóban és oda helyhezteték a gyülekezetnek sátorát, minekutána meghódola elõttök a föld.
\par 2 De maradtak vala még Izráel fiai között, akiknek nem osztották vala ki az õ örökségöket; hét nemzetség.
\par 3 Monda azért Józsué Izráel fiainak: Meddig vonakodtok még elmenni, hogy elfoglaljátok a földet, a melyet néktek adott az Úr, a ti atyáitoknak Istene?
\par 4 Hozzatok elõ három-három férfiút nemzetségenként, és elküldöm õket, hogy keljenek fel és járják el a földet, és írják fel azt az õ örökségük szerint, és térjenek vissza hozzám;
\par 5 Azután oszszák fel azt magok közt hét részre. Júda maradjon meg a maga határaiban dél felõl, József háza pedig maradjon meg a maga határaiban észak felõl.
\par 6 És ti írjátok le a földet hét részre, és hozzátok ide hozzám, hogy sorsot vessek itt néktek az Úr elõtt, a mi Istenünk elõtt.
\par 7 Mert a Lévitáknak nincs részök ti közöttetek; mivelhogy az Úrnak papsága az õ örökségök;  Gád pedig és Rúben és Manassé fél nemzetsége megkapták az õ örökségöket a Jordánon túl napkelet felé, a mit Mózes, az Úrnak szolgája adott vala nékik.
\par 8 És felkelének azok a férfiak, és elmenének. Parancsola pedig Józsué azoknak, a kik elmenének, hogy leírják a földet, mondván: Menjetek el és járjátok el a földet, és írjátok le azt, azután térjetek vissza hozzám, és itt vetek néktek sorsot Silóban, az Úr elõtt.
\par 9 Elmenének azért a férfiak, és által menének a földön, és leírák azt városonként hét részre, könyvben, azután visszatérének Józsuéhoz a táborba, Silóba.
\par 10 És sorsot vete nékik Józsué Silóban az Úr elõtt, és elosztá ott Józsué a földet Izráel fiai között az õ osztályrészeik szerint.
\par 11 És kijöve a Benjámin fiai nemzetségének sors szerint való része az õ családjaik szerint; és pedig esék az õ sors szerint való részöknek határa a Júda fiai és a József fiai közé.
\par 12 Vala pedig az õ határok az északi oldalon a Jordántól fogva, és felméne a határ Jérikhó háta mögé észak felé, azután felméne a hegyre napnyugat felé, a szélei pedig Béth-Aven pusztájánál valának.
\par 13 Onnan pedig átmegy a határ Luz-felé, Lúznak azaz Béthelnek háta mögé dél felõl; azután alámegy a határ Ataroth-Adárnak a hegyen, a mely dél felõl van alsó Béth-Horontól.
\par 14 Majd tovább megy a határ, és kerül a nyugoti oldalnak dél felé, attól a hegytõl, a mely átellenben van Béth-Horonnal délrõl; a szélei pedig Kirjáth-Baál, azaz Kirjáth-Jeárim felé, a Júda fiainak városa felé vannak. Ez a napnyugoti határ.
\par 15 A déli oldala pedig van Kirjáth-Jeárim szélétõl kezdve, és megy a határ napnyugot felé, megy a Nefthoa vizének kútfejéhez.
\par 16 Azután alámegy a határ a hegynek széléhez, a mely átellenben van a Hinnom fiának völgyével, a mely észak felé van a Refaim völgyében; alámegy a Hinnom völgyébe is a Jebuzeus mellett dél felé, és alámegy a Rógel  forrásához.
\par 17 És kerül észak felõl, és megy Én-Semesnek, azután megy Gelilothnak, a mely átellenben van az Adummimba felvivõ úttal; majd alámegy Bohánnak, a Rúben fiának kövéhez.
\par 18 És átmegy az Arabával átellenben levõ oldalra észak felé, és alámegy Arabába is.
\par 19 Átmegy a határ azután Béth-Hogla oldalára észak felé; a határ szélei pedig északnak a Sóstenger csúcsánál, délnek a Jordán végénél vannak. Ez a dél felé való határ.
\par 20 A Jordán pedig határolja azt a napkelet felõl való oldalról. Ez a Benjámin fiainak öröksége az õ határaik szerint köröskörül, az õ családjaik szerint.
\par 21 A Benjámin fiai nemzetségének városai pedig az õ családjaik szerint ezek: Jérikhó, Béth-Hogla, és Emek-Keczicz;
\par 22 Béth-Arábá, Czemaraim és Béthel;
\par 23 Avvim, Pára és Ofra;
\par 24 Kefár-Amóni, Ofni és Gába. Tizenkét város és azoknak falui.
\par 25 Gibeon, Ráma és Beéroth;
\par 26 Miczpe, Kefira és Mócza;
\par 27 Rekem, Jirpeél és Thareala;
\par 28 Czéla, Elef és Jebuzeus, azaz Jeruzsálem, Gibeath, Kirjáth. Tizennégy város és ezeknek falui. Ez a Benjámin fiainak öröksége az õ családjaik szerint.

\chapter{19}

\par 1 A sors által való második rész juta Simeonnak, a Simeon fiai nemzetségének az õ családjaik szerint, és lõn az õ örökségök a Júda fiainak öröksége között.
\par 2 És lõn az övék az õ örökségökül: Beer-Seba, Seba és Móláda;
\par 3 Haczar-Sual, Bála és Eczem;
\par 4 Eltolád, Bethul és Horma;
\par 5 Cziklág, Béth-Markaboth és Haczar-Szusza;
\par 6 Béth-Lebaoth és Sarúhen. Tizenhárom város és ezeknek falui.
\par 7 Ain, Rimmon, Ether és Asán. Négy város és ezeknek falui;
\par 8 És mindazok a faluk, a melyek e városok körül valának Baalath-Beérig, délen Rámatig. Ez a Simeon fiai nemzetségének öröksége az õ családjaik szerint.
\par 9 A Júda fiainak osztályrészébõl lõn a Simeon fiainak örökségök, mert a Júda fiainak osztályrésze nagyobb vala mint illett volna nékik; ezért öröklének Simeon fiai azoknak öröksége között.
\par 10 A sors által való harmadik rész juta a Zebulon fiainak az õ családjaik szerint, és lõn az õ örökségöknek határa Száridig.
\par 11 És felmegy az õ határuk nyugotnak, Mareala felé, és éri Dabbasethet, és éri a folyóvizet is, a mely átellenben van Jokneámmal.
\par 12 Száridtól pedig napkelet felé fordul Kiszloth-Tábor határára, és tova megy Daberáthnak, és felmegy Jafiának.
\par 13 Innen pedig átmegy kelet felé Gittha-Héfernek és Ittha-Kaczinnak, és tova megy Rimmonnak, kerülvén Néa felé:
\par 14 És ennél kerül a határ északról Hannathonnak; a széle pedig a Jiftah-Él völgye.
\par 15 Továbbá Kattáth, Nahalál, Simron, Jidealá és Betlehem. Tizenkét város és azoknak falui.
\par 16 Ez a Zebulon fiainak öröksége az õ családjaik szerint; ezek a városok és ezeknek falui.
\par 17 A sors által való negyedik rész Issakhárnak, az Issakhár fiainak juta, az õ családjaik szerint.
\par 18 És lõn az õ határuk; Jezréel, Keszuloth és Súnem:
\par 19 Hafaráim, Sion és Anaharath;
\par 20 Rabbith, Kisjon és Ébecz;
\par 21 Remeth, Én-Gannim, Én-Hadda és Béth-Paczczécz.
\par 22 És éri a határ Tábort, Sahaczimát és Béth-Semest, a határuknak széle pedig a Jordán. Tizenhat város és ezeknek falui.
\par 23 Ez az Izsakhár fiai nemzetségének öröksége, az õ családjaik szerint: a városok és ezeknek falui.
\par 24 A sors által való ötödik rész pedig juta az Áser fiai nemzetségének az õ családjaik szerint.
\par 25 És lõn az õ határuk: Helkath, Háli, Beten és Aksáf;
\par 26 Alammelek, Ameád és Misál, és éri Karmelt nyugot felé és Sihór-Libnáthot.
\par 27 Azután visszafordul napkelet felé Béth-Dágonnak, és éri Zebulont és a Jiftah-Él völgyét észak felõl, Béth-Émeket és Neiélt, és tovamegy Kabulnak balkéz felõl;
\par 28 És Ebronnak, Rehobnak, Hammonnak és Kánának a nagy Czidonig.
\par 29 Azután visszafordul a határ Rámának, Tyrusnak erõs városáig; és újra fordul a határ Hósznak, a szélei pedig a tengernél vannak Akzib oldala felõl;
\par 30 És Umma, Afék és Rehób. Huszonkét város és ezeknek falui.
\par 31 Ez az Áser fiai nemzetségének öröksége az õ családjaik szerint: ezek a városok és ezeknek falui.
\par 32 A sors által való hatodik rész juta a Nafthali fiainak, a Nafthali fiainak az õ családjaik szerint.
\par 33 Lõn pedig a határuk: Heleftõl, Elontól fogva Czaanannimnál Adámi-Nekebig és Jabneél-Lakkumig; a széle pedig a Jordán vala.
\par 34 Azután fordul a határ nyugot felé Aznoth-Tábornak; innen pedig tovamegy Hukkóknak, és éri Zebulont dél felõl, Ásert pedig éri nyugot felõl, és a Júdát is; a Jordán napkelet felé vala.
\par 35 Erõsített városok ezek: Cziddim, Czér, Hammath, Rakkath és Kinnereth;
\par 36 Adáma, Ráma és Hásor;
\par 37 Kedes, Edrei és Én-Hásor;
\par 38 Jireon, Migdal-Él, Horem, Béth-Anath és Béth-Semes. Tizenkilencz város és ezeknek falui.
\par 39 Ez a Nafthali fiai nemzetségének öröksége az õ családjaik szerint: a városok és ezeknek falui.
\par 40 A sors által való hetedik rész juta a Dán fiai nemzetségének az õ családjaik szerint.
\par 41 És lõn az õ örökségüknek határa: Czóra, Estháol és Ir-Semes;
\par 42 Saalabbin, Ajjálon és Jithla;
\par 43 Élon, Timnatha és Ekrón;
\par 44 Eltheké, Gibbethon és Baaláth;
\par 45 Jehud, Bené-Bárak és Gath-Rimmon;
\par 46 Mé-Jarkon és Rakkon, a Jáfó átellenében levõ határral.
\par 47 De tovább méne ezeknél a Dán fiainak határa. Felmenének ugyanis a Dán fiai, és hadakozának Lesem ellen, és el is foglalák azt, és veték azt fegyver élére, és birtokba vevék azt, és lakozának benne, és nevezék Lesemet Dánnak, az õ atyjoknak Dánnak nevére.
\par 48 Ez a Dán fiai nemzetségének öröksége az õ családjaik szerint: ezek a városok és ezeknek falui.
\par 49 Mikor pedig elvégezték vala a földnek örökbe vételét annak határai szerint, akkor adának Izráel fiai örökséget Józsuénak, a Nún fiának õ közöttök.
\par 50 Az Úr rendelése szerint adák néki azt a várost, a melyet kért vala: Timnath-Szeráhot az Efraim hegyén, és megépíté azt a várost, és abban lakozék.
\par 51 Ezek azok az örökségek, a melyeket örökül adának Eleázár, a pap és Józsué, a Nún fia és az atyáknak fejei az Izráel fiai és nemzetségeinek sors szerint, Silóban, az Úr elõtt, a gyülekezet sátorának nyílásánál. Így végezék el a földnek felosztását.

\chapter{20}

\par 1 Majd szóla az Úr Józsuénak, mondván:
\par 2 Szólj az Izráel fiainak, mondván: Válaszszatok magatoknak menekülésre való városokat, a melyekrõl szóltam néktek Mózes által,
\par 3 Hogy oda szaladjon a gyilkos, a ki megöl valakit tévedésbõl, nem szándékosan, és legyenek azok menedékül nékik a vérbosszúló elõl.
\par 4 A ki pedig beszalad valamelyikbe e városok közül, álljon az a város kapujába, és beszélje el az õ dolgait a város véneinek hallatára, és vegyék be õt magok közé a városba, és adjanak néki helyet, hogy velök lakozzék.
\par 5 Hogyha pedig kergeti azt a vérbosszúló, ki ne adják a gyilkost annak kezébe, mert nem szándékosan ölte meg az õ felebarátját, és nem gyûlölte õ azt annakelõtte.
\par 6 És lakozzék abban a városban mindaddig, a míg ítéletre állhat a gyülekezet elé, a míg meghal a fõpap, a ki abban az idõben lesz, azután térjen vissza a gyilkos, és menjen haza az õ városába és az õ házába, abba a városba, a melybõl elfutott vala.
\par 7 És kiválaszták Kedest Galileában a Nafthali hegyén, Sikemet az Efraim hegyén és Kirjáth-Arbát,  azaz Hebront a Júda hegyén.
\par 8 Túl a Jordánon pedig Jérikhótól napkelet felé választák Beczert a pusztában, a sík földön Rúben nemzetségébõl; Rámothot Gileádban, a Gád nemzetségébõl, és Gólánt Básánban a Manassé nemzetségébõl.
\par 9 Ezek voltak a meghatározott városok mind az Izráel fiainak, mind a jövevényeknek, a ki közöttük tartózkodik vala, hogy oda szaladjon mindaz, a ki megöl valakit tévedésbõl, és meg ne haljon a vérbosszúlónak keze által, míg oda nem áll a gyülekezet elé.

\chapter{21}

\par 1 És hozzámenének a Léviták atyai fejedelmei Eleázárhoz, a paphoz, és Józsuéhoz, a Nún fiához és az atyai fejedelmekhez, a kik valának Izráel fiainak nemzetségei felett,
\par 2 És szólának nékik Silóban, a Kanaán földén, mondván: Az Úr megparancsolta Mózes által, hogy adjatok nékünk városokat lakóhelyül, és azokhoz való legelõket barmaink részére.
\par 3 Adák azért Izráel fiai a Lévitáknak az õ örökségökbõl, az Úrnak rendelése szerint, ezeket a városokat és azoknak legelõit.
\par 4 Esék pedig a sors a Kehátiták családjaira, és juta a Léviták közül való Áron pap fiainak a Júda nemzetségétõl, a Simeon nemzetségétõl és a Benjámin nemzetségétõl sors szerint tizenhárom város;
\par 5 Kehát többi fiainak pedig az Efraim nemzetségének családjaitól, Dán nemzetségétõl és Manassé fél nemzetségétõl sors szerint tíz város.
\par 6 A Gerson fiainak pedig az Issakhár nemzetségének családjaitól, az Áser nemzetségétõl, a Nafthali nemzetségétõl és Manassénak Básánban levõ fél nemzetségétõl sors szerint tizenhárom város;
\par 7 Mérári fiainak az õ családjaik szerint a Rúben nemzetségétõl, Gád nemzetségétõl és Zebulon nemzetségétõl tizenkét város.
\par 8 Adák azért Izráel fiai e városokat és ezeknek legelõit a Lévitáknak, a miképen megparancsolta vala az Úr Mózes által, sors szerint.
\par 9 Adák pedig a Júda fiainak nemzetségébõl és Simeon fiainak nemzetségébõl ezeket a városokat, a melyek névszerint olvashatók.
\par 10 És lõn, hogy esék a sors elõször a Lévi fiai közül való Kerátnak családjaiból az Áron fiaira.
\par 11 És adák nékik az Anák atyjának városát, Kirjáth-Arbát, azaz  Hebront a Júda hegyén, és annak körülötte levõ legelõit;
\par 12 A városnak szántóföldjét és faluit pedig Kálebnek, a Jefunné fiának adák birtokául.
\par 13 Az Áron pap fiainak pedig adák a gyilkosok menekülésének városát, Hebront és annak legelõjét, Libnát és annak legelõjét.
\par 14 Jatthirt és annak legelõjét; Estemoát és annak legelõjét;
\par 15 Holont és annak legelõjét, és Debirt és annak legelõjét;
\par 16 Aint és annak legelõjét, Juttát és annak legelõjét, Béth-Semest és annak legelõjét. Kilencz várost e két nemzetségbõl.
\par 17 A Benjámin nemzetségébõl pedig Gibeont és annak legelõjét, Gébát és annak legelõjét;
\par 18 Anathótot és annak legelõjét, Almont és annak legelõjét: négy várost.
\par 19 Az Áron fiainak, a papoknak városai összesen tizenhárom város és azoknak legelõi.
\par 20 A Kehát fiai családjainak pedig, s a Kehát fiai közül való többi Lévitáknak városai, az õ sorsuk szerint, az Efraim nemzetségébõl valának.
\par 21 Nékik adák ugyanis a gyilkosok menekülésének városát, Sikemet és annak legelõjét az Efraim hegyén, És Gezert és annak legelõjét;
\par 22 Kibczaimot és annak legelõjét, Béth-Horont és annak legelõjét: négy várost.
\par 23 A Dán nemzetségébõl pedig Elthekét és annak legelõjét, Gibbethont és annak legelõjét;
\par 24 Ajjálont és annak legelõjét; Gath-Rimmont és annak legelõjét; négy várost.
\par 25 A Manassé fél nemzetségébõl pedig: Taanákot és annak legelõjét, Gath-Rimmont és annak legelõjét: két várost.
\par 26 Összesen tíz várost és azoknak legelõit a Kehát fiai többi családjainak.
\par 27 A Gerson fiainak pedig, a kik a Léviták családjaiból valának, adák a Manassai fél nemzetségébõl a gyilkosok menekülésének városát,  Gólánt Básánban és annak legelõjét, és Beesterát és annak legelõjét: két várost.
\par 28 Az Issakhár nemzetségébõl pedig: Kisjont és annak legelõjét, Dobráthot, és annak legelõjét,
\par 29 Jármutot és annak legelõjét, és Én-Gannimot és annak legelõjét: négy várost.
\par 30 Az Áser nemzetségébõl pedig: Misált és annak legelõjét, Abdont és annak legelõjét,
\par 31 Helkathot és annak legelõjét, és Rehobot és annak legelõjét: négy várost.
\par 32 A Nafthali nemzetségébõl pedig: a gyilkosok menekülésének városát, Kedest Galileában és annak legelõjét, Hammonth-Dórt és annak legelõjét, és Karthant és annak legelõjét: három várost.
\par 33 A Gersoniták összes városai, az õ családjaik szerint, tizenhárom város és azoknak legelõi.
\par 34 A Mérári fiai családjainak pedig, e még hátralévõ Lévitáknak, adák a Zebulon nemzetségébõl: Jokneámot és annak legelõjét, Karthát és annak legelõjét,
\par 35 Dimnát és annak legelõjét, Nahalált és annak legelõjét; négy várost.
\par 36 A Rúben nemzetségébõl pedig: Béczert és annak legelõjét, Jahczát és annak legelõjét,
\par 37 Kedémothot és annak legelõjét, és Méfaathot és annak legelõjét: négy várost.
\par 38 A Gád nemzetségébõl pedig: a gyilkosok menekülésének városát, Rámothot Gileádban és annak legelõjét, Mahanaimot és annak legelõjét,
\par 39 Hesbont és annak legelõjét; Jaézert és annak legelõjét: összesen négy vársot.
\par 40 A Mérári fiainak városai az õ családjaik szerint, a kik a Léviták családjai közül maradtak fel, valának az õ sorsuk szerint összesen tizenkét város.
\par 41 A Lévitáknak összes városai az Izráel fiainak birtoka között: negyvennyolcz város azoknak legelõivel.
\par 42 Valának ezek a városok a tulajdonképeni város és körülöttök azoknak legelõi. Így vala ez mind e városoknál.
\par 43 Megadá azért az Úr Izráelnek mindazt a földet, a mely felõl megesküdött vala, hogy odaadja azt az õ atyáiknak. És bírák azt, és lakozának abban.
\par 44 Nyugodalmat is ada nékik az Úr mindenfelõl, szintén úgy, a mint megesküdött vala az õ atyáiknak. És senki nem állott meg ellenökben valamennyi ellenségeik közül; minden ellenségöket kezökbe adá az Úr.
\par 45 Nem esett el csak egy szó is mindama jó szóból, a melyet szólott vala az Úr az Izráel házának. Mindaz  betelt.

\chapter{22}

\par 1 Akkor hivatá Józsué a Rúbenitákat, a Gáditákat és a Manassé fél nemzetségét,
\par 2 És monda nékik: Ti megtartottátok mindazt, a mit Mózes, az Úrnak szolgája parancsolt néktek, és hallgattatok az én szómra mindenben, a mit parancsoltam néktek;
\par 3 Nem hagytátok el a ti atyátokfiait immár sok nap óta mind e mai napig, és megtartottátok a megtartandókat, az Úrnak, a ti Isteneteknek parancsolatját.
\par 4 Most pedig nyugodalmat adott az Úr, a ti Istenetek a ti atyátokfiainak, a mint megmondta vala nékik; most azért térjetek vissza, és menjetek a ti sátraitokba, a ti örökségteknek földére, a melyet Mózes, az Úrnak szolgája adott vala néktek túl a Jordánon.
\par 5 Csak igen vigyázzatok, hogy teljesítsétek a parancsolatot és a törvényt, a melyet parancsolt néktek Mózes, az Úrnak szolgája, hogy szeressétek az Urat, a ti Isteneteket, és járjatok minden õ útján, és tartsátok meg az õ parancsolatait, és ragaszkodjatok hozzá, és szolgáljatok néki teljes szívetekbõl és teljes lelketekbõl.
\par 6 És megáldá õket Józsué, azután elbocsátá õket, és elmenének az õ sátraikba.
\par 7 Mert a Manassé nemzetsége felének Mózes adott vala örökséget Básánban; a másik felének pedig Józsué adott az õ atyjafiaival a Jordánon innen napnyugot felõl. És a mikor elbocsátá is õket Józsué az õ sátraikba, akkor is megáldá õket.
\par 8 És szóla nékik, mondván: Nagy gazdagsággal térjetek vissza sátraitokba, és igen sok barommal, ezüsttel, arannyal, rézzel, vassal és igen sok ruhával. Az ellenségeitektõl vett zsákmányt oszszátok meg a ti atyátokfiaival.
\par 9 Visszatérének azért, és elmenének a Rúben fiai, a Gád fiai és a Manassé fél nemzetsége Izráel fiaitól, Silóból, a mely Kanaán földén van, hogy menjenek Gileád földére, az õ örökségüknek földébe, a melyben örökséget võnek az Úr rendelése szerint Mózes által.
\par 10 Mikor a Jordán mellékére értek, a mely még Kanaán földén van, építének a Rúben fiai, Gád fiai és Manassé fél nemzetsége ott a Jordán mellett oltárt, nagy, láttatós oltárt.
\par 11 Meghallák pedig Izráel fiai, hogy ezt mondják vala: Ímé a Rúben fiai, Gád fiai és Manassé fél nemzetsége oltárt építettek átellenben a Kanaán földével, a Jordán mellékén, Izráel fiainak oldala felõl.
\par 12 A mint meghallák Izráel fiai, egybegyûle Izráel fiainak egész gyülekezete Silóban, hogy hadakozni induljanak fel ellenök.
\par 13 Küldék pedig Izráel fiai a Rúben fiaihoz, Gád fiaihoz és a Manassé fél nemzetségéhez Gileád földére Fineást, Eleázárnak, a papnak fiát;
\par 14 Tíz fejedelmet is vele, egy-egy fejedelmet az õsök háza szerint Izráelnek minden nemzetségébõl. Ezek mind fejedelmei valának az õ õseik házának, Izráelnek ezerei között.
\par 15 És eljutának a Rúben fiaihoz, a Gád fiaihoz és a Manassé fél nemzetségéhez a Gileád földére, és szólának õ velök, mondván:
\par 16 Így szól az Úrnak egész gyülekezete: Micsoda vétek ez, a melylyel vétkeztetek Izráelnek Istene ellen, hogy immár elfordultatok az Úrtól, építvén magatoknak oltárt, hogy pártot üssetek immár az Úr ellen?!
\par 17 Avagy kevés-é nékünk a Peór miatt való hamisságunk, a mibõl ki sem tisztultunk még mind e napig, és a miért csapás lõn az Úrnak gyülekezetén?
\par 18 És ti elfordultatok immár az Úrtól; pedig ha ti pártot üttök ma az Úr ellen: holnap majd Izráelnek egész gyülekezetére haragszik meg.
\par 19 Ha pedig tisztátalannak tetszik elõttetek a ti örökségeteknek földje; úgy jõjjetek át az Úr örökségének földjére, a hol ott áll az Úrnak sátora, és legyetek örökösökké mi közöttünk; de az Úr ellen ne üssetek pártot, és ellenünk se üssetek pártot, építvén magatoknak oltárt, az Úrnak, a mi Istenünknek oltárán kivül.
\par 20 Avagy nem Ákán, a Zéra fia vétkezék-é vagy vétekkel az Istennek szentelt dolog ellen; mégis Izráelnek egész gyülekezete ellen lõn a harag! És az a férfiú nem egymaga halt meg az õ bûnéért!
\par 21 Felelének pedig a Rúben fiai, a Gád fiai és a Manassé fél nemzetsége, és szólának az Izráel ezereinek fejeihez:
\par 22 Az Istenek Istene az Úr, az Istenek Istene az Úr: õ tudja, és az Izráel is megtudja. Ha pártütésbõl és ha az Úr ellen való vétekbõl van ez: ne tartson meg minket e napon!
\par 23 Ha azért építettünk magunknak oltárt, hogy elforduljunk az Úrtól, vagy pedig hogy áldozzunk azon egészen égõáldozatot és ételáldozatot; vagy hogy tegyünk arra hálaadásnak áldozatját: - lássa meg ezt õ maga az Úr!
\par 24 Hát nem inkább a miatt való féltünkben cselekedtük-é ezt, hogy így gondolkoztunk: Maholnap így szólhatnak a ti fiaitok a mi fiainkhoz, mondván: Mi közötök van néktek az Úrhoz, Izráel Istenéhez?
\par 25 Hiszen határt vetett az Úr mi közöttünk és ti közöttetek Rúben fiai és Gád fiai - a Jordánt; nincsen néktek részetek az Úrban! És elidegeníthetnék a ti fiaitok a mi fiainkat, hogy ne féljék az Urat!
\par 26 Mondottuk azért: Nosza fogjunk hozzá, hogy építsünk oltárt, se nem egészen égõáldozatra, se nem véres áldozatra;
\par 27 Hanem hogy bizonyság legyen az mi közöttünk és ti közöttetek, és a mi utánunk való nemzetségeink között, hogy szolgálni akarjuk az Urat õ elõtte a mi egészen égõáldozatainkkal, véres áldozatainkkal és hálaáldozatainkkal, és hogy ne mondhassák a ti fiaitok maholnap a mi fiainknak: Nincs néktek részetek az Úrban.
\par 28 Mondottuk azért: Ha így szólának maholnap nékünk vagy a mi nemzetségeinknek, ezt mondjuk majd: Lássátok az Úr oltárának mássát, a melyet a mi atyáink készítettek, nem egészen égõáldozatra, sem nem véres áldozatra, hanem hogy bizonyság legyen mi közöttünk és ti közöttetek!
\par 29 Távol legyen tõlünk, hogy pártot üssünk az Úr ellen, és elforduljunk immár az Úrtól, építvén oltárt égõáldozatra, ételáldozatra és véres áldozatra az Úrnak, a mi Istenünknek oltárán kivül, a mely az õ sátora elõtt van!
\par 30 Hallván pedig Fineás, a pap és a gyülekezetnek fejedelmei és Izráel ezereinek fejei, a kik vele valának, a beszédeket, a melyeket szólottak vala a Rúben fiai, a Gád fiai és a Manassé fiai; tetszésre találtak vala elõttök.
\par 31 És monda Fineás, az Eleázár pap fia a Rúben fiainak, Gád fiainak és a Manassé fiainak: Ma tudtuk meg, hogy köztünk van az Úr, mivelhogy nem vétkeztetek e vétekkel az Úr ellen. Most megszabadítottátok Izráel fiait az Úr kezébõl.
\par 32 Visszatére azért Fienás, az Eleázár pap fia és a fejedelmek a Rúben fiaitól, a Gád fiaitól a Gileád földérõl a Kanaán földére Izráelnek fiaihoz, és megbeszélék nékik e dolgot.
\par 33 És tetszésre talált e dolog Izráel fiai elõtt is, és áldák az Istent Izráel fiai, és nem mondák, hogy hadakozni menjenek fel ellenök, hogy elveszessék a földet, a melyben lakoznak a Rúben fiai és a Gád fiai.
\par 34 Az oltárt pedig így nevezék Rúben fiai és a Gád fiai: "Bizonyság ez közöttünk, hogy az Úr az Isten."

\chapter{23}

\par 1 Lõn pedig sok nappal azután, hogy nyugodalmat adott vala az Úr Izráelnek minden õ körülötte lévõ ellenségeitõl: megvénhedék Józsué  és megidõsödék.
\par 2 Elõhívá ekkor Józsué az egész Izráelt, annak véneit, fejeit, biráit és felügyelõit, és monda nékik: Én megvénhedtem és megidõsödtem.
\par 3 Magatok is láttatok mindent, a mit az Úr, a ti Istenetek mind eme népekkel cselekedett ti elõttetek; mivelhogy maga az Úr, a ti Istenetek harczolt érettetek.
\par 4 Ímé elosztám néktek sors által a fennmaradt népeket örökségül a ti nemzetségeitek szerint, a Jordántól kezdve, és mindazokat a népeket, a melyeket kipusztítottam, és a nagy tengert is napnyugat felõl.
\par 5 Az Úr pedig, a ti Istenetek kiûzi õket a ti orczátok elõl, és kiûzi õket ti elõletek, hogy örököseivé legyetek az õ földüknek, a mint megmondotta vala néktek az Úr, a ti Istenetek.
\par 6 Legyetek azért igen erõsek, hogy megtartsátok és megcselekedjétek mind azt, a mi meg van írva a Mózes törvényének könyvében; hogy el ne távozzatok attól se jobbkézre, se balkézre;
\par 7 Hogy össze ne elegyedjetek ezekkel a népekkel, azokkal, a melyek fenmaradtak közöttetek; isteneiknek nevét ki se ejtsétek,  azokra ne esküdjetek, se ne szolgáljatok nékik, se elõttök meg ne hajoljatok;
\par 8 Hanem ragaszkodjatok az Úrhoz, a ti Istenetekhez, a miképen e mai napig cselekedtetek!
\par 9 Ezért ûzött ki az Úr elõletek nagy és erõs népeket: a mi pedig titeket illet, senki meg nem állhatott ellenetekben mind e napig.
\par 10 Egy férfiú közületek elûz ezeret, mert az Úr, a ti Istenetek az, a ki harczol érettetek, a miképen megmondta vala néktek.
\par 11 Azért igen vigyázzatok magatokra, hogy szeressétek az Urat, a ti Isteneteket!
\par 12 Mert ha elfordulván elfordultok tõle és ragaszkodtok e népek maradékaihoz, a melyek itt maradtak ti közöttetek; és sógorságot köttök õ velök és összeelegyedtek velök és õk veletek:
\par 13 Bizonynyal megtudjátok majd, hogy az Úr, a ti Istenetek ki nem ûzi többé e népeket ti elõletek; sõt inkább lésznek ti néktek tõrré és hurokká, oldalaitokban ostorrá, szemeitekben pedig tövisekké, míglen kivesztek errõl a jó földrõl, a melyet az Úr, a ti Istenetek adott ti néktek.
\par 14 Én pedig ímé megyek már a minden földinek útján: Tudjátok meg azért teljes szívetek és teljes lelketek szerint, hogy egy szó sem esett  el mindama jó szóból, a melyeket az Úr, a ti Istenetek szólott vala felõletek. Minden betelt rajtatok, egy szó sem esett el azokból!
\par 15 De a miképen betelt rajtatok mind az a jó szó, a melyeket az Úr, a ti Istenetek szólott vala felõletek: akképen teljesíti majd rajtatok mind a gonosz szót is az Úr, míglen kipusztít titeket e jó földrõl, a melyet az Úr, a ti Istenetek adott ti néktek.
\par 16 Ha általhágjátok az Úrnak, a ti Isteneteknek szövetségét, a melyet parancsolt néktek, és elmentek és szolgáltok idegen isteneknek, és meghajoltok azok elõtt: akkor felgerjed ellenetek az Úrnak haragja, és hamar kivesztek e jó földrõl, a melyet õ adott néktek.

\chapter{24}

\par 1 Józsué pedig összegyûjté Izráelnek minden nemzetségét Síkembe, és elõhívá Izráelnek véneit, fejeit, biráit és felügyelõit, és oda állának az Úr elébe,
\par 2 És monda Józsué az egész népnek: Ezt mondja az Úr, Izráelnek Istene: A folyóvizen túl lakoztak régenten a ti atyáitok: Tháré, Ábrahámnak atyja és Nákhórnak atyja, és idegen isteneknek szolgáltak vala.
\par 3 De áthozám a ti atyátokat, Ábrahámot a folyóvíz túlsó oldaláról, és elhordozám õt az egész Kanaán földén, és megsokasítám az õ magvát: és adám néki Izsákot.
\par 4 Izsáknak pedig adám Jákóbot és Ézsaut. És Ézsaunak a Szeír hegyét adám birtokul,  Jákób pedig és az õ fiai alámenének Égyiptomba.
\par 5 Majd elküldém Mózest és Áront, és megverém Égyiptomot úgy, a hogy cselekedtem vala közöttök: azután pedig kihoztalak vala titeket.
\par 6 És kihoztam a ti atyáitokat Égyiptomból, és jutátok a tengerhez, és ûzék az égyiptomiak a ti atyáitokat szekerekkel és lovasokkal a Veres tengerig.
\par 7 Akkor az Úrhoz kiáltának, és homályt vete ti közétek és az égyiptomiak közé, és rájok vivé a tengert és elborítá õket. Szemeitek is látták azt, a mit cselekedtem vala  az égyiptomiakkal: ti pedig sok ideig lakoztatok a pusztában.
\par 8 Majd elhozálak titeket az Emoreusok földére, a kik túl laktak vala a Jordánon, és hadakozának ellenetek; de kezetekbe adám õket, és bírátok az õ földüket, õket pedig eltörlém a ti orczátok elõl.
\par 9 Majd felkele Bálák, a Czippor fia, Moábnak királya, és hadakozék Izráellel, és elkülde és hivatá Bálámot, a Beór fiát, hogy átkozzon meg titeket.
\par 10 De nem akarám meghallgatni Bálámot, ezért áldva áldott vala  titeket, és megszabadítálak titeket az õ kezébõl.
\par 11 Majd általjövétek a Jordánon, és jutátok Jérikhóba,  és hadakozának veletek Jérikhónak urai; az Emoreus, Perizeus, Kananeus, Khittheus, Girgazeus, Khivveus és Jebuzeus; de kezetekbe adám õket.
\par 12 Mert elbocsátám elõttetek a darázsokat, és ûzék azokat elõletek: az Emoreusok két királyát, de nem a te fegyvered által,  de nem a te kézíved által!
\par 13 És adék néktek földet, a melyben nem munkálkodtál, és városokat, a melyeket nem építettetek vala, mégis bennök laktok; szõlõket és olajfákat, a melyeket nem ti ültettetek, de esztek azokról.
\par 14 Azért hát féljétek az Urat, és szolgáljatok néki tökéletességgel és hûséggel; és hányjátok el  az isteneket, a kiknek szolgáltak a ti atyáitok túl a folyóvizen és Égyiptomban, szolgáljatok az Úrnak.
\par 15 Hogyha pedig rossznak látjátok azt, hogy szolgáljatok az Úrnak: válaszszatok magatoknak még ma, a kit szolgáljatok; akár azokat az isteneket, a kiknek a ti atyáitok szolgáltak, a míg túl valának a folyóvizen, akár az Emoreusok isteneit, a kiknek földjén lakoztok: én azonban és az én házam az Úrnak szolgálunk.
\par 16 A nép pedig felele, és monda: Távol legyen tõlünk, hogy elhagyjuk az Urat, szolgálván idegen isteneknek!
\par 17 Sõt inkább az Úr, a mi Istenünk az, a ki felhozott minket és atyáinkat Égyiptom földébõl, a szolgák házából, és a ki ezeket a nagy jeleket tette a mi szemeink elõtt, és megtartott minket minden útunkban, a melyen jártunk, és mind ama népek között, a melyek között általjöttünk;
\par 18 És kiûzött az Úr minden népet, az Emoreust is, e földnek lakóját, a mi orczánk elõl: Mi is szolgálunk az Úrnak, mert õ a mi Istenünk!
\par 19 Józsué pedig monda a népnek: Nem szolgálhattok az Úrnak, mert szent Isten õ, féltõn szeretõ Isten õ: nem bocsátja meg a ti vétkeiteket és bûneiteket;
\par 20 Hogyha elhagyjátok az Urat, és szolgáltok idegen isteneknek: akkor elfordul és roszszal illet benneteket, és megemészt titeket, minekutána jól cselekedett veletek.
\par 21 Akkor monda a nép Józsuénak: Nem, mert mi az Úrnak szolgálunk!
\par 22 Józsué pedig monda a népnek: Bizonyságok vagytok magatok ellen, hogy ti választottátok magatoknak az Urat, hogy néki szolgáljatok. És mondának: Bizonyságok!
\par 23 Most azért hányjátok el az idegen isteneket, a kik köztetek vannak, és hajtsátok sziveteket az Úrhoz, Izráelnek Istenéhez.
\par 24 És monda a nép Józsuénak: Az Úrnak, a mi Istenünknek szolgálunk, és az õ szavára hallgatunk.
\par 25 Szerze azért Józsué szövetséget a néppel e napon, és ada eleibe rendelést és végzést Síkemben.
\par 26 És beírá Józsué e dolgokat az Isten törvényének könyvébe, és võn egy nagy követ, és oda helyhezteté azt a cserfa alá, a mely vala az Úrnak szent házában.
\par 27 És monda Józsué az egész népnek: Ímé ez a kõ lesz ellenünk bizonyságul; mert ez hallotta az Úrnak minden beszédét, a melyet szólott vala nékünk; és lesz ellenetek bizonyságul, hogy ne hazudjatok a ti Istenetek ellen.
\par 28 És elbocsátá Józsué a népet, kit-kit a maga örökségébe.
\par 29 És lõn e dolgok után, hogy meghala Józsué, a Nún fia, az Úrnak szolgája,  száztíz esztendõs korában,
\par 30 És eltemeték õt az õ örökségének határában Timnat-Szerában, a mely az Efraim hegyén van, a Gaas hegytõl észak felé.
\par 31 Izráel pedig az Urat szolgálta vala Józsuénak minden idejében, és a véneknek is minden idejökben, a kik hosszú ideig éltek Józsué után, és a kik tudják vala minden cselekedetét az Úrnak, a melyet cselekedett vala Izráellel.
\par 32 A József csontjait pedig, a melyeket felhoztak vala Izráel fiai Égyiptomból, eltemették Síkemben, a mezõnek abban a részében, a melyet szerzett vala  Jákób Hámornak, a Síken atyjának fiaitól száz pénzén: és lõnek a József fiainak örökségévé.
\par 33 Meghala Eleázár is, Áronnak fia, és eltemeték õt Gibeathban, az õ fiának Fineásnak városában, a mely néki adatott az Efraim hegyén.


\end{document}