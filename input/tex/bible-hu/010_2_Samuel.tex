\begin{document}

\title{2 Samuel}


\chapter{1}

\par 1 És lõn Saul halála után, mikor Dávid visszatért az Amálekiták legyõzésébõl, és Dávid két napig Siklágban idõzött:
\par 2 Ímé a harmadik napon egy férfi jöve a táborból Saultól, és az õ ruhái megszaggatva valának, fején pedig föld vala; és a mikor Dávidhoz ért, leesék a földre, és meghajtá magát.
\par 3 Monda pedig néki Dávid: Honnét jössz? Felele néki: Az Izráel táborából szaladék el.
\par 4 Monda néki Dávid: Mondd meg kérlek nékem, mint lõn a dolog? Õ pedig felele: megfutamodék a nép a harczból, és a nép közül nagy sokaság esett el, és meghalának. Sõt Saul is és Jonathán az õ fia meghalának.
\par 5 Dávid pedig mondá az ifjúnak, ki néki ezt elbeszélé: Honnan tudod, hogy meghalt Saul és Jonathán az õ fia?
\par 6 Felele az ifjú, ki a hírt hozta: Történetbõl felmenék a Gilboa hegyére, és ímé Saul az õ dárdájára támaszkodott vala, és ímé a szekerek és lovagok utólérék õtet.
\par 7 Hátratekintvén pedig Saul, megláta engem és szólíta, és mondék: Ímhol vagyok én.
\par 8 Monda pedig nékem: Ki vagy te? Felelék néki: Amálekita vagyok.
\par 9 Akkor monda nékem: Kérlek állj mellém és ölj meg engem, mert dermedtség fogott el engem, pedig a lélek még teljesen bennem van.
\par 10 Annakokáért én mellé állván, megölém õtet, mert tudtam, hogy meg nem él, miután elesett, és elhozám a koronát, mely az õ fején vala, és az aranypereczet, mely az õ karján volt, és azokat ímé ide hoztam az én uramnak.
\par 11 Akkor megragadá Dávid a maga ruháit és megszaggatá, úgyszintén a többi emberek is, a kik õ vele valának.
\par 12 És nagy zokogással sírának, és bõjtölének mind estvéig, Saulon és Jonathánon, az õ fián, és az Úrnak népén és Izráelnek házán, mivelhogy fegyver által hullottak el.
\par 13 És monda Dávid az ifjúnak, a ki ezt elbeszélé néki: Honnét való vagy te? Felele: Egy jövevény Amálekita férfi fia vagyok.
\par 14 Ismét monda néki Dávid: Hogy nem féltél felemelni kezedet az Úr felkentjének elvesztésére?
\par 15 És szólíta Dávid egyet az õ szolgái közül, kinek monda: Jõjj elõ és öld meg õt. Ki általüté azt, és meghala.
\par 16 És monda néki Dávid: A te véred legyen a te fejeden: mert a tennen nyelved vallása bizonyságot tesz ellened, mondván: Én öltem meg az Úrnak felkentjét.
\par 17 És keservesen síra Dávid ilyen sírással, Saulon és Jonathánon, az õ fián,
\par 18 És monda (íjdal ez, hogy megtanulják Júda fiai, mely be van írva a Jásár könyvébe):
\par 19 Izráel! a te ékességed elesett halmaidon: miként hullottak el a hõsök!
\par 20 Meg ne mondjátok Gáthban, ne hirdessétek Askelon utczáin, hogy ne örvendjenek a Filiszteusok leányai, és ne ujjongjanak a körülmetéletlenek leányai.
\par 21 Gilboa hegyei, se harmat, se esõ ti reátok ne szálljon, és mezõtök ne teremjen semmi áldozatra valót; mert ott hányatott el az erõs vitézek paizsa, Saulnak paizsa, mintha meg nem kenettetett volna  olajjal.
\par 22 A megöletteknek vérétõl és a hõsöknek kövérétõl Jonathán kézíve hátra nem tért, és a Saul fegyvere hiába nem járt.
\par 23 Sault és Jonathánt, a kik egymást szerették és kedvelték míg éltek, a halál sem szakította el; a saskeselyûknél gyorsabbak és az oroszlánoknál erõsebbek valának.
\par 24 Izráel leányai! sirassátok Sault, ki karmazsinba öltöztetett gyönyörûen, és aranynyal ékesíté fel ruhátokat.
\par 25 Oh, hogy elhullottak a hõsök a harczban! Jonathán halmaidon esett el!
\par 26 Sajnállak testvérem, Jonathán, kedves valál nékem nagyon, hozzám való szereteted csudálatra méltóbb volt az asszonyok szerelménél.
\par 27 Oh, hogy elhullottak a hõsök! És elvesztek a hadi szerszámok!

\chapter{2}

\par 1 Ezek után lõn, hogy megkérdezé Dávid az Urat, mondván: Felmenjek-é Júdának valamelyik városába? Kinek felele az Úr: Menj fel. És monda Dávid: Hová menjek? Felele: Hebronba.
\par 2 Felméne azért oda Dávid és az õ két felesége is, Ahinoám, a Jezréelbõl való, és Abigail, a Karmelbõl való, Nábál felesége.
\par 3 És embereit is, a kik vele valának, felvivé Dávid, kit-kit a maga házanépével, és lakának Hebron városaiban.
\par 4 És eljövének Júda férfiai, és ott felkenék Dávidot királynak a Júda házán. Mikor pedig megjelentették Dávidnak, mondván: A Gileádból való Jábes emberei voltak azok, kik eltemették Sault;
\par 5 Követeket külde Dávid a Gileádból való Jábes embereihez, és ezt izené néki: Áldottak vagytok ti az Úrtól, kik ezt az irgalmasságot cselekedtétek a ti uratokkal, Saullal, hogy eltemettétek;
\par 6 Annakokáért most az Úr cselekedjék veletek irgalmasságot és igazságot; sõt én is ezt a jót teszem veletek, hogy ezt cselekedtétek;
\par 7 Most azért a ti kezeitek erõsödjenek meg, és legyetek erõs férfiak, mert megholt a ti uratok Saul, és immár engem felkent királynak Júda háza önmagán.
\par 8 Abner pedig, a Nér fia, a ki Saulnak fõvezére vala, felvevé Isbósetet, a Saul fiát, és  elvivé Mahanáimba.
\par 9 És királylyá tette õt Gileádon, Asuron és Jezréel városán, és az Efraim és Benjámin nemzetségein, és az egész Izráelen.
\par 10 Negyven esztendõs vala Isbóset, Saul fia, mikor uralkodni kezde Izráelen, és két esztendeig uralkodék. Csak Júdának háza követé Dávidot.
\par 11 Lõn pedig az idõnek száma, míg Dávid király volt Hebronban Júdának házán, hét esztendõ és hat hónap.
\par 12 És elméne Abner, a Nér fia, és Isbósetnek, a Saul fiának szolgái Mahanáimból Gibeonba.
\par 13 Joáb is Sérujának fia, és a Dávid szolgái elmenvén, összetalálkozának azokkal a Gibeon halastavánál, és maradának ezek a halastón innét, amazok pedig a halastón túl.
\par 14 Akkor monda Abner Joábnak: Nosza álljanak elõ néhányan az ifjak, és viaskodjanak elõttünk. És monda Joáb: Hát álljanak elõ.
\par 15 Felkelének azért és egyenlõ számban általmenének a Benjámin nemzetébõl és a Saul fiának, Isbósetnek seregébõl tizenketten, és a Dávid szolgái közül is tizenketten.
\par 16 És kiki az õ társának fejét megragadá, és fegyverét oldalába üté; és egyenlõképen mind elhullának; és nevezék azt a mezõt Helkáth-Hassurimnak, mely Gibeonban van.
\par 17 És felette erõs harcz lõn azon a napon, és megvereték Abner és az Izráel népe a Dávid szolgái által.
\par 18 Három fia vala pedig ott Sérujának: Joáb,  Abisai és Asáel; Asáel pedig könnyû lábú vala, mint egy vadkecske, mely a mezõn lakik.
\par 19 És Asáel üldözé Abnert; és se jobbfelé, se balfelé nem tére ki Abner után való futásában.
\par 20 Hátratekintvén pedig Abner, monda: Te vagy-é Asáel? Felele: Én vagyok.
\par 21 És monda néki Abner: Térj másfelé, vagy jobbkézre vagy balkézre, és fogj meg egyet az ifjak közül, és foszd ki õt mindenébõl; de Asáel nem akarta õt elhagyni.
\par 22 Ismét monda Abner Asáelnek: Menj el hátam mögül. Miért verjelek téged a földhöz? És micsoda orczával menjek Joábhoz, a te bátyádhoz?
\par 23 Mikor azért semmiképen nem akart eltérni, általüté õt Abner a dárda végével, az ötödik oldalcsontja között, és hátul jöve ki a dárda; és elesék azon a helyen, és ugyanott meghala. És lõn, hogy mindazok, a kik arra a helyre érkezének, a hol Asáel elesett és meghalt vala, megállának.
\par 24 Joáb pedig és Abisai tovább üldözék Abnert; és midõn a nap lement, elérkezének az Amma halmára, mely Giah átellenében vala, a Gibeon pusztája melletti úton.
\par 25 Akkor egybegyülekezének az Abner után való Benjámin fiai, egy csoportot alkotva, és megállának egy halomnak tetején.
\par 26 És kiálta Abner Joábnak, és monda: Vajjon szûntelenül öldökölnie kell-é a fegyvernek? Nem tudod-é, hogy siralmas lesz ennek a vége? És meddig nem mondod a népnek, hogy térjenek vissza testvéreiknek hátukról?
\par 27 És monda Joáb: Él az Isten, hogy ha te nem szólottál volna, bizony már reggel eltávozott volna a nép, és nem kergette volna az õ atyjafiait.
\par 28 Trombitát fuvata azért Joáb, és megálla az egész nép, és nem üldözék tovább Izráelt, és nem harczolának tovább.
\par 29 Abner pedig és az õ vitézei azon az egész éjszakán mennek vala a mezõségen, és általkelének a Jordánon, és az egész vidéket bejárván, jutának Mahanáimba.
\par 30 Joáb pedig megtérvén Abner üldözésébõl, összegyûjté az egész népet, és hiányozának Dávid szolgái közül tizenkilenczen és Asáel.
\par 31 A Dávid szolgái pedig Benjámin nemzetébõl, az Abner szolgái közül háromszázhatvan embert ölének meg, a kik meghalának.
\par 32 És felvevén Asáelt, eltemeték õt atyjának sírboltjában, mely Bethlehemben vala. Joáb pedig és az õ vitézei egész éjjel menve, Hebronban virradának meg.

\chapter{3}

\par 1 Sok ideig tartó hadakozás lõn a Saul háznépe között és a Dávid háznépe között. Dávid pedig mind feljebb-feljebb emelkedik és erõsbödik vala; a Saul háza pedig alább-alább száll és fogy vala.
\par 2 Fiai születének Dávidnak Hebronban, kik között elsõszülött vala Ammon, a  Jezréelbõl való Ahinoámtól;
\par 3 A második pedig Kileáb, a Kármelbeli Nábál feleségétõl, Abigailtól való, és a harmadik Absolon, a Máákha fia, a ki a Gessurbeli Thalmai király leánya vala;
\par 4 És a negyedik Adónia, Haggitnak fia, és az ötödik Sefátia, Abitál fia,
\par 5 A hatodik Ithreám, Eglától, Dávid feleségétõl való: ezek születtek Dávidnak Hebronban.
\par 6 Míg a hadakozás tartott a Saul házanépe és a Dávid házanépe között, Abner igen ragaszkodék a Saul házanépéhez.
\par 7 Vala pedig Saulnak egy ágyasa, kinek neve vala Rizpa, Ajának leánya; és monda Isbóset Abnernek: Miért mentél be az én atyámnak ágyasához?
\par 8 Felette igen megharaguvék azért Abner az Isbóset szaván, és monda: Ebfej vagyok-é én, ki Júdával tart? Én most nagy irgalmasságot cselekedtem a te atyádnak, Saulnak házával, az õ atyjafiaival és rokonságaival, és nem adtalak téged Dávidnak kezébe; és te mégis ez asszonynak vétkét reám fogod mst.
\par 9 Úgy cselekedjék Isten Abnerrel most és ezután is, hogy a mint megesküdött az Úr Dávidnak, én is a szerint cselekeszem  vele:
\par 10 Hogy elveszem a királyságot Saulnak házától, és megerõsítem Dávidnak székét az Izráel és a Júda felett, Dántól fogva mind Bersebáig.
\par 11 És nem felelhete semmit erre Abnernek, mivelhogy igen fél vala tõle.
\par 12 Követeket külde azért Abner Dávidhoz maga helyett ilyen izenettel: Vajjon kié az ország? Azt mondván: Tégy frigyet velem, és ímé az én erõm is te melletted lesz, hogy az egész Izráelt hozzád hajtsam.
\par 13 Kinek felele Dávid: Jó, én frigyet kötök veled. De mindazáltal egyet kérek tõled, mondván: Addig ne lássad az én arczomat, míg el nem hozod nékem Mikált, a Saul leányát, mikor ide akarsz jõni, hogy arczomat lássad.
\par 14 És követeket külde Dávid Isbósethez, a Saul fiához, kik ezt mondják: Add vissza az én feleségemet, Mikált, kit én száz Filiszteus elõbõrével jegyeztem el magamnak.
\par 15 Elkülde azért Isbóset, és elvéteté õt az õ férjétõl Páltieltõl, Láis fiától.
\par 16 És vele ment az õ férje is, sírva követvén õt Bahurimig; és ott mondá néki Abner: Eredj, menj vissza; és haza tére.
\par 17 Annakutána Abner szóla az Izráel véneinek, mondván: Immár régtõl fogva kivántátok Dávidot, hogy királyotok legyen néktek:
\par 18 Azért most vigyétek véghez; mert az Úr szólott Dávidnak, ezt mondván: Az én szolgámnak, Dávidnak keze által szabadítom meg az én népemet Izráelt a Filiszteusok kezébõl, és minden ellenségeinek kezébõl.
\par 19 Azután szóla Abner a Benjámin nemzetségével is, és elméne Abner Dávidhoz is Hebronba, hogy megjelentse néki mindazt, a mi tetszenék Izráel népének és Benjámin egész nemzetségének.
\par 20 Mikor pedig eljutott Abner Dávidhoz Hebronba és õ vele együtt húsz ember, megvendégelé Dávid Abnert és a vele volt embereket.
\par 21 Ennekutána monda Abner Dávidnak: Felkelek és elmegyek, hogy az egész Izráelt ide gyûjtsem az én uram eleibe, a király eleibe, a kik frigyet kössenek te veled, és uralkodjál mindeneken úgy, a mint szívednek tetszik. És visszabocsátá Dávid Abnert, és elméne békével.
\par 22 És ímé a Dávid szolgái és Joáb jõnek vala a táborból, és sok prédát hoznak magukkal, de Abner nem volt már Dávidnál Hebronban, mert elbocsátotta õt és békével elment vala.
\par 23 Joáb pedig és az egész sereg, mely õ vele vala, a mint megérkezének, értesíték Joábot e dolog felõl, mondván: Ide jött Abner, Nérnek fia a királyhoz, és visszabocsátá õt, és békével hazatére.
\par 24 Beméne azért Joáb a királyhoz és monda: Mit cselekedtél? Ímé Abner hozzád jött, miért bocsátád el õt, hogy elmenjen?
\par 25 Ismered-é Abnert, a Nér fiát? Csak azért jött volt ide, hogy megcsalhasson, és kikémlelje a te kijövésedet és bemenésedet, és megtudjon mindent, a mit te cselekszel.
\par 26 És kimenvén Joáb Dávidtól, követeket külde Abner után, kik visszahozák õt a Sira kútjától; Dávid azonban nem tudja vala.
\par 27 Visszajövén Abner Hebronba, félreszólítá õt Joáb a kapu között, mintha titokban akarna vele beszélni, és általüté ott õt az ötödik oldalbordájánál, és meghala Asáelnek,  a Joáb atyjafiának véréért.
\par 28 Mely dolgot minekutána megtudott Dávid, monda: Ártatlan vagyok én és az én országom mindörökké az Úr elõtt, Abnernek, a Nér fiának vérétõl.
\par 29 Szálljon ez Joábnak fejére, és az õ atyjának egész háznépére; és el ne fogyjon a Joáb házából a folyásos, a bélpoklos, a mankón járó, és a ki fegyver miatt vész el, és a kenyér nélkül szûkölködõ.
\par 30 Joáb pedig és Abisai, az õ atyjafia azért ölék meg Abnert, mivelhogy megölte vala az õ atyjokfiát, Asáelt Gibeonban a harczon.
\par 31 Monda pedig Dávid Joábnak és mind az egész népnek, mely vele vala: Szaggassátok meg ruháitokat, és öltözzetek zsákba, és sírjatok Abner elõtt! Dávid király pedig megy vala a koporsó után;
\par 32 És eltemeték Abnert Hebronban. Akkor felkiálta a király, és igen síra az Abner koporsója felett, és síra az egész nép is.
\par 33 És a király gyászdalt szerezvén Abner felett, monda: Gaz halállal kelle kimulnia Abnernek?
\par 34 A te kezeid nem voltak megkötve, sem lábaid békóba verve; de úgy vesztél el, mint álnok ember miatt szokott elveszni az ember! És siratá õt ismét az egész nép.
\par 35 Elõjöve pedig mind az egész nép, hogy enni adjanak Dávidnak, mikor még a nap fenn vala, de megesküvék Dávid, ezt mondván: Úgy cselekedjék én velem az Isten most és ezután is, hogy míg a nap le nem megy, sem kenyeret, sem egyebet nem eszem.
\par 36 Mely dolgot mikor az egész község megértett, igen tetszék nékik; valamint a többi dolga is, a mit a király cselekeszik vala, igen tetszék nékik.
\par 37 És megértette azon a napon az egész nép és az egész Izráel, hogy nem a királytól volt, hogy Abnert, a Nér fiát megölték.
\par 38 Monda pedig a király az õ szolgáinak: Nem tudjátok-é, hogy nagy fejedelem esett ma el az Izráelben?
\par 39 Én pedig ma erõtelen, noha felkent király vagyok; ezek pedig a Sérujának fiai hatalmasabbak nálamnál. De fizessen meg az Úr annak, a ki  gonoszt cselekeszik, az õ gonoszsága szerint.

\chapter{4}

\par 1 Mikor pedig meghallotta Saul fia, hogy meghalt Abner Hebronban, igen megfogyatkozék az õ ereje. Sõt az egész Izráel megrémüle.
\par 2 Két fõvezére volt a Saul fiának, egyiknek neve Bahana és a másiknak neve Rékáb, Rimmonnak fiai, ki Beerótból való vala, a Benjámin fiai közül; mert Beerót is a Benjámin városai közé számláltatik.
\par 3 Elfutottak vala pedig a Beerótbeliek Gittáimba, és lõnek ott jövevények mind e mai napig.
\par 4 Jonathánnak pedig, a Saul fiának volt egy sánta fia (öt esztendõs vala, mikor Jezréelbõl hír jött Saul és Jonathán felõl, és felvevé õt a dajkája, hogy elszaladjon vele; lõn pedig, hogy mikor a dajka gyorsan futott, elesett, és megsántula), ennek neve vala Méfibóset.
\par 5 Elmenének azért a Beerótbeli Rimmonnak fiai, Rékáb és Bahana, és bemenének Isbósetnek házába, mikor a nap legmelegebb vala és õ déli álmát aluszsza vala.
\par 6 Bemenvén azért ezek a ház belsejébe, gabonát vive, általüték õt az ötödik oldalborda alatt; s Rékáb és az õ atyjafia, Bahana, elszaladának.
\par 7 Mikor azért ezek a házba bementek, és õ hálószobájában az ágyán feküvék, megsebesítvén megölték õk, s fejét levágva, felvették az õ fejét, és egész éjjel mennek vala a sík mezõn.
\par 8 És elvivék Isbóset fejét Dávidhoz Hebronba, és mondának a királynak: Ímhol Isbósetnek, Saul fiának, a te ellenségednek feje; a ki üldözé a te lelkedet, és az Úr bosszút állott e mai napon az én uramért, a királyért, Saulon és az õ maradékán.
\par 9 Felele pedig Dávid Rékábnak és Bahanának, az õ atyjafiának, a Beerótbeli Rimmon fiainak, és monda nékik: Él az Úr, ki az én életemet megszabadította minden nyomorúságból,
\par 10 Hogy én azt, a ki nékem hírt hozott vala, ezt mondván: Ímé meghalt Saul (és azt hitte, hogy azzal nékem örömet szerez), megragadván megöletém õt Siklágban, holott jutalmat kellett volna adnom néki hírmondásáért:
\par 11 Mennyivel inkább az istentelen embereket, kik ágyában, a maga házában ölték meg az igaz embert? Azért most vajjon ne kivánjam-é meg az õ vérét kezeitekbõl, hogy titeket eltöröljelek a földrõl?
\par 12 Parancsola azért Dávid az õ szolgáinak, hogy megöljék õket; és elvagdalák kezeiket és lábaikat; és felakaszták õket Hebronban, a halastó mellett. Isbósetnek pedig fejét felvevén, eltemeték az Abner  sírboltjába, Hebronban.

\chapter{5}

\par 1 Eljövének pedig Dávidhoz Hebronba Izráelnek minden nemzetségei, és szólának ilyenképen: Ímé mi a te csontodból és testedbõl valók vagyunk.
\par 2 Mert ennekelõtte is, mikor Saul uralkodott felettünk, te vezérelted ki s be Izráelt, és az Úr azt mondotta néked: Te legelteted az én népemet, az Izráelt, és te fejedelem leszel Izráel felett.
\par 3 Eljövének azért Izráelnek minden vénei a királyhoz Hebronba, és frigyet tõn velek Dávid király Hebronban az Úr elõtt, és királylyá kenék Dávidot Izráel felett.
\par 4 Harmincz esztendõs vala Dávid, mikor uralkodni kezde, és negyven esztendeig uralkodék.
\par 5 Hebronban uralkodék a Júda nemzetségén hét esztendeig és hat hónapig; és Jeruzsálemben uralkodék harminczhárom esztendeig az egész Izráel és Júda nemzetségein.
\par 6 Felméne pedig a király és az õ népe Jeruzsálembe a Jebuzeusok ellen, kik azt a földet lakják vala, õk azonban azt mondák Dávidnak: Nem jössz ide be, hanem a sánták és vakok elûznek téged! melylyel azt jelenték: Nem jõ ide be Dávid.
\par 7 Bevevé mindazáltal Dávid a Sion várát, és az immár a Dávid városa.
\par 8 Mert azt mondá Dávid ama napon: Mindenki, a ki vágja a Jebuzeusokat, menjen fel a csatornához és vágja ott a sántákat és a vakokat, a kiket  gyûlöl a Dávid lelke. Ezért mondják: Vak és sánta ne menjen be a házba!
\par 9 És lakozék Dávid abban a várban, és nevezé azt Dávid városának; és megépíté Dávid köröskörül, Millótól fogva befelé.
\par 10 Dávid pedig folytonosan emelkedék és növekedék, mert az Úr, a Seregeknek Istene vala õ vele.
\par 11 Követeket külde pedig Hirám, Tirusnak királya Dávidhoz, és czédrusfákat is, ácsmestereket és kõmíveseket, és építének házat Dávidnak.
\par 12 És belátta Dávid, hogy az Úr megerõsítette õt az Izráel felett való királyságában, és hogy felmagasztalta az õ királyságát az õ népéért, Izráelért.
\par 13 Võn pedig még magának Dávid ágyasokat, és  feleségeket Jeruzsálembõl, minekutána Hebronból oda ment; és lõnek még Dávidnak fiai és leányai.
\par 14 És ezek a nevei azoknak, a kik Jeruzsálemben születtek: Sammua, Sóbáb, Nátán, Salamon,
\par 15 Ibhár, Elisua, Néfeg, Jáfia,
\par 16 Elisáma, Eljada és Elifélet.
\par 17 Mikor pedig a Filiszteusok meghallották, hogy királylyá kenték Dávidot az Izráelen; felkelének mind a Filiszteusok, hogy Dávidot megkeressék; melyet megértvén Dávid, aláméne az erõsségbe.
\par 18 A Filiszteusok pedig elérkezének és elszéledének a Réfaim völgyében.
\par 19 Megkérdé azért Dávid az Urat ilyen szóval: Elmenjek-é a Filiszteusok ellen? Kezembe adod-é õket? Felele az Úr Dávidnak: Menj el, mert kétség nélkül kezedbe adom a Filiszteusokat.
\par 20 Elérkezék azért Dávid Baál Perázimba, és megveré ott õket Dávid, és monda: Szétszórta az Úr ellenségimet elõttem, mint a víz szokott eloszlani; azért nevezé azt a helyet Baál Perázimnak.
\par 21 És ott hagyák az õ bálványaikat, melyeket felszedének Dávid és az õ szolgái.
\par 22 Azután ismét feljövének a Filiszteusok, és elszéledének a Réfaim völgyében.
\par 23 Megkérdé azért Dávid az Urat, ki ezt felelé: Ne menj most reájok; hanem kerülj a hátuk mögé és a szederfák ellenében támadd meg õket.
\par 24 És mikor a szederfák tetején indulásnak zaját fogod hallani, akkor indulj meg, mert akkor kimegy te elõtted az Úr, hogy megverje a Filiszteusok táborát.
\par 25 És úgy cselekedék Dávid, a mint megparancsolta vala néki az Úr: és vágta a Filiszteusokat Gibeától fogva, mind addig, míg  Gézerbe mennél.

\chapter{6}

\par 1 Összegyûjté ezek után Dávid az egész Izráelnek színét, harmincezer embert.
\par 2 És felkelvén, elméne Dávid az egész néppel együtt, mely vele vala, Júdának városába, Bahalába, hogy elhozza onnét az Isten ládáját, mely Névrõl, a Seregek Urának nevérõl neveztetik, ki ül a Kérubok  között.
\par 3 És tevék az Isten ládáját új szekérre, és elhozák azt Abinádábnak házától, mely vala a dombon, és Uzza és Ahió, Abinádáb fiai vezetik az új szekeret.
\par 4 Elvivék azért azt az Abinádáb házából, mely a dombon vala, az Isten ládájával, és Ahió a láda elõtt megyen.
\par 5 Dávid pedig és az egész Izráel népe örvendeznek vala az Úr elõtt, jegenyefából való minden szerszámokkal, hegedûkkel, lantokkal, dobokkal, sípokkal és czimbalmokkal.
\par 6 És mikor Nákonnak szérûjére jutának, kinyújtá Uzza az õ kezét az Isten ládájára, és megtartá azt; mert az ökrök félremozdították vala.
\par 7 Ennekokáért felgerjede az Úr haragja Uzza ellen, és megölé ott õt az Isten vakmerõségéért: és meghala ott az Isten ládája mellett.
\par 8 És bosszankodék Dávid, hogy az Úr ilyen romlással rontotta meg Uzzát, és nevezé azt a helyet Péres Uzzának, mind e mai napig.
\par 9 Megfélemlék azért Dávid azon a napon az Úrtól, és monda: Hogyan jõjjön az Úr ládája én hozzám?
\par 10 És nem akará Dávid magához vinni az Úrnak ládáját Dávidnak városába, hanem letéteté azt Dávid a Gitteus Obed  Edom házánál.
\par 11 És lõn az Úrnak ládája a Gitteus Obed Edom házánál három hónapig, és megáldá az Úr Obed Edomot és egész háznépét.
\par 12 És megmondák Dávid királynak ilyen szóval: Megáldotta az Úr az Obed Edom házát és mindenét, a mije van, az Isten ládájáért. És elmenvén Dávid, elvivé  az Isten ládáját az Obed Edom házától Dávid városába  vígassággal.
\par 13 Mikor pedig azok, a kik az Úr ládáját vitték, hat lépést mentek, áldozék ott egy ökröt és hízott borjút.
\par 14 Dávid pedig teljes erejébõl tánczol vala az Úr elõtt, és gyolcs  efódot övedzett magára Dávid.
\par 15 Dávid azért és Izráelnek egész háza felvivék az Úr ládáját énekléssel és trombitaszóval.
\par 16 Lõn pedig, hogy mikor az Úr ládája Dávid városába ért, Mikál, Saulnak leánya kinéz vala az ablakon, és látván, hogy Dávid király ugrál és tánczol az Úr elõtt,  megútálá õt szívében.
\par 17 Mikor pedig bevitték az Úr ládáját, és elhelyezék azt az õ helyére a sátornak közepére, melyet számára Dávid felállíttatott: akkor áldozott Dávid egészen égõáldozatokat és hálaadó áldozatokat, az Úr elõtt.
\par 18 Mikor pedig elvégezte Dávid az egészen égõáldozatot, és a hálaadó áldozatot, megáldá a népet a Seregek Urának nevében.
\par 19 És osztogattata Dávid mind az egész népnek, Izráel egész seregének, férfiaknak és asszonyoknak, mindenkinek egy-egy lepényt, egy-egy darab húst, és egy-egy kalácsot; és elméne mind az egész nép, kiki az õ házához.
\par 20 Mikor pedig Dávid hazament, hogy megáldja az õ háznépét is: kijöve Mikál, a Saul leánya Dávid eleibe, és monda: Mily dicsõséges vala ma Izráel királya, ki az õ szolgáinak szolgálói elõtt felfosztózott vala ma, mint a hogy egy esztelen szokott felfosztózni!
\par 21 És monda Dávid Mikálnak: Az Úr elõtt, ki inkább engem választott, mint atyádat és az õ egész házanépét, hogy az Úr népének, Izráelnek fejedelme  legyek, igen, az Úr elõtt örvendezém.
\par 22 Sõt minél inkább megalázom magamat és minél alábbvaló leszek a magam szemei elõtt: annál dicséretreméltóbb leszek a szolgálók elõtt, a kikrõl te szólasz.
\par 23 Ennekokáért Mikálnak, Saul leányának nem lõn soha gyermeke, az õ halálának napjáig.

\chapter{7}

\par 1 Lõn pedig, hogy mikor a király az õ palotájában üle, és az Úr mindenfelõl békességet adott néki minden ellenségeitõl,
\par 2 Monda a király Nátán prófétának: Ímé lássad, én czédrusfából csinált palotában  lakom, az Istennek ládája pedig a kárpitok között van.
\par 3 És monda Nátán a királynak: Eredj, s valami a te szívedben van, vidd véghez, mert az Úr veled van.
\par 4 Azonban lõn az Úr szava Nátánhoz azon éjjel, mondván:
\par 5 Menj el, és mondd meg az én szolgámnak, Dávidnak: Ezt mondja az Úr: Házat akarsz-é nékem építeni, hogy abban lakjam?
\par 6 Mert én nem laktam házban attól a naptól fogva, hogy kihoztam Izráel fiait Égyiptomból, mind e mai napig, hanem szüntelen sátorban és hajlékban jártam.
\par 7 Valahol csak jártam Izráel fiai között, avagy szólottam-é csak egy szót is Izráel valamelyik nemzetségének, a kinek parancsoltam, hogy legeltesse az én népemet Izráelt, mondván: Miért nem építettetek nékem czédrusfából való házat?
\par 8 Most azért ezt mondd meg az én szolgámnak, Dávidnak: Ezt mondja a Seregeknek Ura: Én hoztalak ki téged a kunyhóból, a juhok mögül, hogy légy fejedelem az én népem felett, Izráel felett;
\par 9 És veled voltam mindenütt, valahová mentél, és kiirtottam minden ellenségeidet elõtted; és nagy hírnevet szerzettem néked, mint a nagyoknak hírneve, a kik e földön vannak.
\par 10 Helyet is szerzék az én népemnek Izráelnek, és ott elplántálám õt, és lakozék az õ helyében, és többé helyébõl ki nem mozdíttatik, és nem fogják többé az álnokságnak fiai nyomorgatni õt, mint  annakelõtte,
\par 11 Attól a naptól fogva, hogy bírákat rendeltem az én népem az Izráel felett. Békességet szereztem tehát néked minden ellenségeidtõl. És megmondotta néked az Úr, hogy házat csinál néked az  Úr.
\par 12 Mikor pedig a te napjaid betelnek, és elaluszol a te atyáiddal, feltámasztom utánad a te magodat, mely ágyékodból származik, és megerõsítem az õ királyságát:
\par 13 Az fog házat építeni az én nevemnek, és megerõsítem az õ királyságának trónját mindörökké.
\par 14 Én leszek néki atyja, és õ lészen nékem fiam, a ki mikor gonoszul cselekszik, megfenyítem õt  emberi veszszõvel és emberek fiainak büntetésével;
\par 15 Mindazáltal irgalmasságomat nem vonom meg tõle, miképen megvonám Saultól, a kit kivágék elõtted.
\par 16 És állandó lészen a te házad, és a te országod mindörökké tiéd lészen, és a te trónod erõs lészen mindörökké.
\par 17 Mind e beszéd szerint, és mind e látás szerint szóla Nátán Dávidnak.
\par 18 Bemenvén azért Dávid király, leborula az Úr elõtt, és monda: Micsoda vagyok én, Uram Isten! és micsoda az én házam népe, hogy engem ennyire elõvittél?
\par 19 És ez még csekélynek tetszett néked, Uram Isten! hanem ímé szólasz a te szolgádnak háza felõl  messze idõre valókat; és ez törvény az emberre nézve, Uram Isten!
\par 20 De mi szükség Dávidnak többet szólani néked, holott te jól ismered, Uram Isten, a te szolgádat?
\par 21 A te igéretedért, és a te jóakaratod szerint cselekedted mindezeket a nagy dolgokat, hogy megjelentsd a te szolgádnak.
\par 22 Ennekokáért felmagasztaltattál, Uram Isten: mert senki sincs olyan, mint te, és rajtad kivül nincsen Isten, mind a szerint, a mint hallottuk a mi füleinkkel.
\par 23 Mert melyik nép olyan, mint a te néped, az Izráel, melyért  elment volna az Isten, hogy megváltsa magának való népül, és magának nevet szerezzen, és érettetek oly nagyokat cselekedjék és országodért oly csudálatos dolgokat néped elõtt, melyet megszabadítottál Égyiptomból, a pogányoktól és isteneiktõl.
\par 24 Mert megerõsítetted magadnak a te népedet; az Izráelt, népedül mindörökké; és te voltál, Uram, nékik Istenök.
\par 25 Most annakokáért, Uram Isten, a mit szóltál a te szolgád felõl, és az õ háza felõl, teljesítsd be mindörökké; és cselekedjél úgy, a mint szólottál;
\par 26 Hogy felmagasztaltassék a te neved mindörökké, ilyen szókkal: A Seregeknek Ura az Izráelnek Istene, és Dávidnak, a te szolgádnak háza legyen állandó te elõtted!
\par 27 Mert mejelentetted a te szolgádnak fülébe, óh Seregeknek Ura és Izráelnek  Istene, ezt mondván: Házat építek néked. Ezért készteté szolgádat az õ szíve, hogy ilyen könyörgéssel könyörögjön hozzád.
\par 28 Most azért, óh Uram Isten, (mert te vagy az Isten, és a te beszéded igazság, és te mondottad ezt a jót a te szolgád felõl),
\par 29 Legyen a te jóakaratodnak áldása a te szolgádnak házán, hogy legyen elõtted mindörökké; mert te szólottál, Uram Isten, azért a te áldásoddal áldassék meg a te szolgádnak háza mindörökké.

\chapter{8}

\par 1 Ezek után megveré Dávid a Filiszteusokat, és megalázá õket: és elfoglalá Dávid a Filiszteusok kezébõl Méteg Ammát.
\par 2 És megveré a Moábitákat is, és kötéllel méré meg õket, lefektetvén õket a földre; két kötéllel méré azokat, a kik megölendõk, és egy teljes kötéllel azokat, a kik életben hagyandók valának; és a Moábiták Dávidnak  adófizetõ szolgái lõnek.
\par 3 Megveré Dávid Hadadézert is, Réhóbnak fiát, Czóbának királyát, mikor elméne, hogy hatalmát kiterjessze az Eufrátes folyó vizéig.
\par 4 És foglyul ejtett Dávid közülök ezer és hétszáz lovagot, és húszezer gyalog embert; és inaikat elvagdaltatá Dávid minden szekeres lovaknak és csak száz szekérbe valót hagyott meg közülök.
\par 5 Eljövének pedig a Siriabeliek Damaskusból, hogy megsegítsék Hadadézert, Czóbának királyát; és levága Dávid a Siriabeliek közül huszonkétezer férfit.
\par 6 És helyeze Dávid állandó sereget a damaskusi Siriába; és a Siriabeliek Dávidnak adófizetõ szolgái lettek. És megoltalmazá az Úr Dávidot valahová megyen vala.
\par 7 És elvevé Dávid az arany paizsokat is, melyek Hadadézer szolgáin valának, és bevivé azokat Jeruzsálembe.
\par 8 Hoza annakfelette Dávid király a Hadadézer városaiból, Bétákhból és Berótaiból felette igen sok rezet.
\par 9 Mikor pedig meghallá Tói, Hamát királya, hogy megverte Dávid Hadadézernek minden seregeit;
\par 10 Küldé Tói Jórámot, az õ fiát Dávid királyhoz, hogy békességesen köszöntse õt, és áldja, hogy harczolt Hadadézer ellen és megverte õt (mert Hadadézer Tóival is hadakozik vala). És a Tói fia kezében aranyból, ezüstbõl és rézbõl való edények valának,
\par 11 Melyeket Dávid király az Úrnak szentele, azzal az ezüsttel és aranynyal együtt, a melyet az Úrnak szentele mind ama népektõl, a kiket meghódoltatott;
\par 12 A Siriabeliektõl, Moábitáktól, az Ammon fiaitól, a Filiszteusoktól, az  Amálekitáktól és a Réhób fiától, Hadadézertõl, Czóbának királyától nyert prédából.
\par 13 És hírnevet szerze Dávid, a mikor visszatért, miután a Siriabelieket leverte a sós völgyben, tizenyolczezeret.
\par 14 Az Idumeusok közé is állandó sereget  rendele, egész Idumeába állandó sereget rendele, és az Idumeusok mind Dávid szolgái lettek. És megoltalmazá az Úr Dávidot, valahová megyen vala.
\par 15 Uralkodék azért Dávid az egész Izráelen, és szolgáltat vala Dávid az egész nép között ítéletet és igazságot.
\par 16 Fõvezére volt Joáb, Sérujának fia, Jósafát pedig, Ahiludnak fia, emlékíró vala.
\par 17 Sádók pedig, Akhitóbnak fia, és Akhimélek, Abjátárnak fia papok valának, és Sérája íródeák.
\par 18 Benája Jéhójadának fia, a Kereteusok  és Peleteusok elõljárója, a Dávid fiai pedig fõk valának.

\chapter{9}

\par 1 Monda pedig Dávid: Maradt-é még valaki a Saul házanépe közül, hogy irgalmasságot cselekedjem õ  vele Jonathánért?
\par 2 Vala pedig Saul házából egy szolga, kinek neve vala Siba, kit Dávidhoz szólítának, és monda a király néki: Te vagy-é Siba? Felele: Én vagyok a te szolgád.
\par 3 Akkor monda a király: Maradt-é még valaki a Saul házanépe közül, hogy cselekedjem vele az Isten irgalmasságát? Felele Siba a királynak: Van még Jonathánnak egy fia, ki mind a  két lábára sánta.
\par 4 Monda néki a király: Hol van? És monda Siba a királynak: Ímé az Ammiel fiának, Mákirnak házában van Ló-Debárban.
\par 5 Akkor elkülde Dávid király, és elhozatá õt a Mákirnak, az Ammiel fiának házából Ló-Debárból.
\par 6 És mikor megérkezett Dávidhoz Méfibóset, Jonathánnak a Saul fiának fia, arczczal leborula, és tisztességet tõn néki; és monda Dávid: Méfibóset! ki felele: Ímhol a te szolgád.
\par 7 Monda néki Dávid: Ne félj; mert kétség nélkül irgalmasságot cselekeszem veled Jonathánért, a te atyádért; és visszaadom néked Saulnak, a te nagyatyádnak minden majorságát, és néked mindenkor asztalomnál lesz ételed.
\par 8 Akkor fejet hajta Méfibóset, és monda: Micsoda a te szolgád, hogy a holt ebre reá tekintettél, minemû én vagyok?
\par 9 Szólítá azután a király Sibát, Saulnak szolgáját, és monda néki: Valamije volt Saulnak és egész háznépének, mindazokat a te urad fiának adtam.
\par 10 A majorságnak azért te viseljed gondját, és a te fiaid és szolgáid, és beszolgáltassad, hogy legyen a te urad fiának kenyere, melylyel éljen. Méfibóset pedig, a te uradnak fia, az én asztalomnál eszik mindenkor. Vala pedig Sibának tizenöt fia és húsz szolgája.
\par 11 Felele Siba a királynak: Valamit az én uram, a király parancsol nékem, szolgájának, a képen cselekeszik a te szolgád. És Méfibóset az én asztalomnál eszik, mint a király fiainak egyike.
\par 12 Vala pedig Méfibósetnek egy fiacskája, kinek neve Mika vala; a Siba házában lakozók pedig mindnyájan Méfibóset szolgái valának.
\par 13 Méfibóset tehát Jeruzsálemben lakozék, mert mindenkor a király asztalánál eszik vala; és õ mind a két lábára sánta vala.

\chapter{10}

\par 1 Lõn pedig azután, meghala az Ammon fiainak királya, és uralkodék helyette Hánon, az õ fia.
\par 2 És monda Dávid: Irgalmasságot teszek Hánonnal, Náhás fiával, miképen az õ atyja is velem irgalmasságot tett; és hozzá külde Dávid, hogy vigasztalja õt szolgái által atyja halála felõl. Mikor Dávid szolgái Ammon fiainak földére érkezének:
\par 3 Mondának az Ammon fiainak vezérei Hánonnan, az õ uroknak: Vajjon becsüli-é Dávid ezzel elõtted a te atyádat, hogy hozzád vigasztalókat küldött? Vajjon Dávid nem inkább azért küldötte-é hozzád szolgáit, hogy a várost megszemléljék és kikémleljék és teljesen elpusztítsák.
\par 4 Elfogatá azért Hánon a Dávid szolgáit, és szakáloknak felét lenyiratá, és ruháikat félben elmetszeté az alfelekig, és elbocsátá õket.
\par 5 A mint ezt Dávidnak hírül hozták, eleikbe külde, (mert ezek az emberek igen meggyaláztattak) és ezt izené a király: Maradjatok Jerikhóban mindaddig, míg megnevekedik szakálotok; és úgy jõjjetek haza.
\par 6 Látván pedig az Ammon fiai, hogy gyûlöltekké lettek Dávid elõtt, követet küldének az Ammon fiai, és felfogadák Siriában a Béth-Réhóbbelieket és ugyancsak Siriában a Czóbeusokat, húszezer gyalogot, és Maakának királyát ezer emberrel, és a Tóbbelieket, tizenkétezer embert.
\par 7 A mint ezt Dávid maghallá, elküldötte Joábot és a vitézeknek egész seregét.
\par 8 Kivonulának az Ammon fiai is, és csatarendbe állottak a kapu bejárata elõtt. A Siriabeli Czóbeusok, Réhóbbeliek, Tóbbeliek és Maaka magukban valának a mezõn.
\par 9 Látván pedig Joáb, hogy mind elõl, mind hátul ellenség van õ ellene, kiválogatá az Izráel népének színét, és a Siriabeliek ellenében állítá fel.
\par 10 A népnek maradékát az õ öcscsének, Abisainak vezetésére bízta, hogy felállítsa azt Ammon fiainak ellenében.
\par 11 És monda: Ha erõsebbek lesznek a Siriabeliek nálamnál, nékem segítségül légy. Ha pedig az Ammon fiai náladnál hatalmasabbak lesznek, elmegyek, hogy megsegítselek.
\par 12 Légy erõs és legyünk bátrak a mi nemzetségünkért és a mi Istenünknek városaiért; és az Úr cselekedjék úgy, a mint néki tetszik.
\par 13 Elérkezék azért Joáb és az õ vele való sereg a Siriabeliek ellen az ütközetre, és megfutamodának õ elõtte.
\par 14 Látván pedig az Ammon fiai, hogy megfutamodtak a Siriabeliek, megfutamodának õk is Abisai elõtt, és beszaladának a városba; és megtére Joáb az Ammon fiaitól és méne Jeruzsálembe.
\par 15 Mikor pedig látták a Siriabeliek, hogy megverettettek az Izráeliták által, ismét együvé gyülekezének.
\par 16 És elkülde Hadadézer, és elhozatá a Siriabelieket, a kik a vizen túl valának, és jövének Helámba; és Sobák, Hadadézernek fõvezére volt élükön.
\par 17 Megizenék pedig Dávidnak, a ki egybegyûjté egész Izráelt, és általméne a Jordánon, és juta Hélámhoz: és sereget rendelének a Siriabeliek Dávid ellen, és megütközének vele.
\par 18 De a Siriabeliek megfutamodtak Izráel elõtt, és levága Dávid a Siriabeliek közül hétszáz szekerest és negyvenezer lovagot, Sobákot is, a seregnek fõvezérét megölé; és ugyanott meghala.
\par 19 Mikor pedig látták mindazok a királyok, a kik Hadadézernek szolgái valának, hogy Izráel által megverettek, békét kötöttek Izráellel és nékik szolgálának; és nem merének többé a Siriabeliek kijõni az Ammon fiainak segítségére.

\chapter{11}

\par 1 Lõn pedig az esztendõ fordulásakor, mikor a királyok hadba szoktak menni, elküldé Dávid Joábot és az õ szolgáit õ vele, és mind az egész Izráelt, hogy elveszessék az Ammon fiait, és megszállják Rabba városát; Dávid pedig Jeruzsálemben maradt.
\par 2 És lõn estefelé, mikor felkelt Dávid az õ ágyából, és a királyi palota tetején sétála: láta a tetõrõl egy asszonyt fürdeni, ki igen szép termetû vala.
\par 3 És elkülde Dávid, és tudakozódék az asszony felõl, és monda egy ember: Nemde nem ez-é Bethsabé, Eliámnak leánya, a Hitteus Uriás felesége?
\par 4 Akkor követeket külde Dávid, és elhozatá õt; ki beméne õ hozzá, és vele hála, mikor az õ tisztátalanságából megtisztult; és annakutána visszaméne az õ házához.
\par 5 És fogada méhében az asszony, és elküldvén, megizené Dávidnak ilyen szóval: Teherbe estem!
\par 6 Akkor Dávid izené Joábnak: Küldd haza hozzám a Hitteus Uriást; és elküldé Joáb Uriást Dávidhoz.
\par 7 Mikor pedig eljutott Uriás õ hozzá, megkérdé õt Dávid Joábnak békessége felõl, a népnek békessége felõl és a harcz folyása felõl.
\par 8 Ezután monda Dávid Uriásnak: Menj haza, és mosd meg lábaidat; kimenvén pedig Uriás a király házából, a király ajándékot külde utána.
\par 9 Uriás azonban lefeküvék a királyi palota bejárata elõtt az õ urának minden szolgáival, és nem méne a maga házához.
\par 10 És megjelenték Dávidnak, mondván: Nem ment le Uriás az õ házához. Akkor monda Dávid Uriásnak: Nemde nem útról jöttél-é? Miért nem mentél le a te házadhoz?
\par 11 Felele Uriás Dávidnak: Az Isten ládája, Izráel és Júda nemzetsége sátorokban laknak, és az én uram Joáb, és az én uramnak szolgái a nyilt mezõn táboroznak: hogy mennék én be az én házamba, hogy egyem és igyam és feleségemmel háljak? Úgy élj  te és úgy éljen a te lelked, hogy nem mívelem azt!
\par 12 Monda azért Dávid Uriásnak: Maradj itt ma is, és holnap elbocsátlak. Ott marada azért Uriás Jeruzsálemben az nap és más napra kelve is.
\par 13 Ennekutána hivatá õt Dávid, és õ vele evék és ivék, és lerészegíté õt, és kiméne estve aludni az õ ágyára, az õ urának szolgáival együtt, de házához nem méne alá.
\par 14 Megvirradván pedig, levelet íra Dávid Joábnak, melyet Uriástól küldött el.
\par 15 Írá pedig a levélben, mondván: Állassátok Uriást legelõl, a hol a harcz leghevesebb és háta mögül fussatok el, hogy megölettessék és meghaljon.
\par 16 Lõn azért, hogy mikor ostromlá Joáb a várost, állatá Uriást arra a helyre, a hol tudja vala, hogy erõs vitézek vannak.
\par 17 Kijövén azért a városbeli nép, megharczolának Joábbal, és elhullának egyesek a nép közül, Dávid szolgái közül, és a Hitteus Uriás is meghala.
\par 18 Követet külde akkor Joáb, és megizené Dávidnak az ütközetnek egész lefolyását.
\par 19 És megparancsolá a követnek, mondván: Ha az ütközetnek egész lefolyását teljesen elõadtad a királynak,
\par 20 És ha a király haragra lobbanva azt mondja néked: Miért mentetek olyan közel a városhoz harczolni? Avagy nem tudtátok-é, hogy lelövöldöznek a kõfalról?
\par 21 Kicsoda ölte meg Abiméleket, a Jérubbóset fiát? Nemde nem egy asszony üté-é agyon a kõfalról egy malomkõdarabbal és meghala Tébesben? Azért miért mentetek közel a kõfalhoz? Akkor mondd meg: A te szolgád a Hitteus Uriás is meghalt.
\par 22 Elméne azért a követ, és mikor megérkezett, elbeszélé Dávidnak mindazt, a mivel megbízta volt õt Joáb.
\par 23 És monda a követ Dávidnak: Azok az emberek erõt vettek felettünk, és a mezõre kijöttek ellenünk, de mi visszaûztük õket a kapu bejáratáig;
\par 24 Azonban a kõfalról lövöldözének a nyilasok a te szolgáidra, és a király szolgái közül egyesek meghaltak; és a te szolgád, a Hitteus Uriás is meghalt.
\par 25 Monda azért Dávid a követnek: Ezt mondjad Joábnak: Ne bánkódjál a miatt; mert a fegyver úgy megemészt egyet, mint mást. Fokozzad azért támadásodat a város ellen, hogy elpusztítsad azt. Így biztasd õt.
\par 26 Meghallá pedig Uriásnak felesége, hogy meghalt Uriás, az õ férje, és siratá az õ férjét.
\par 27 És mikor a gyászolásnak ideje eltelt, érette külde Dávid és házába viteté õt és lõn néki felesége, és szüle néki egy fiat. De ez a dolog, a melyet Dávid cselekedett,  nem tetszék az Úrnak.

\chapter{12}

\par 1 Elküldé azért az Úr Dávidhoz Nátán prófétát, ki bemenvén hozzá, monda néki: Két ember vala egy városban, egyik gazdag, a másik szegény.
\par 2 A gazdagnak felette sok juhai és ökrei valának;
\par 3 A szegénynek pedig semmije nem vala egyéb egy kis nõstény báránykájánál, a melyet vett és táplált vala, s felnevelkedett nála gyermekeivel együtt; saját falatjából evett és poharából ivott és keblén aludt, és néki olyan vala, mintegy leánya.
\par 4 Mikor pedig utazó vendége érkezett a gazdagnak: sajnált az õ ökrei és juhai közül hozatni, hogy a vendégnek ételt készítsen belõle, a ki hozzá ment vala; hanem elvevé a szegénytõl az õ bárányát, és azt fõzeté meg a vendégnek, a ki hozzá ment.
\par 5 Akkor felgerjede Dávidnak haragja az ember ellen, és monda Nátánnak: Él az Úr, hogy halálnak fia az az ember, a ki azt cselekedte.
\par 6 A bárányért pedig négy annyit kell adnia, mivelhogy ezt mívelte, és annak nem kedvezett.
\par 7 És monda Nátán Dávidnak: Te vagy az az ember! Ezt mondja az Úr, Izráelnek Istene: Én  kentelek fel téged, hogy király légy Izráel felett, és megszabadítottalak téged a Saul kezébõl.
\par 8 És néked adtam a te urad házát, és a te uradnak feleségeit a te kebeledbe; ennek felette néked adtam Izráelnek és Júdának házát; és ha még ez kevés volt, ezt s ezt adtam volna néked.
\par 9 Miért vetetted meg az Úrnak beszédét, oly dolgot cselekedvén, mely útálatos õ elõtte? A Hitteus Uriást fegyverrel  ölted meg, és az õ feleségét magadnak vetted feleségül; magát pedig az Ammon fiainak fegyverével ölted meg.
\par 10 Most azért ne távozzék el a fegyver soha házadból, mivel megútáltál engem,  és a Hitteus Uriás feleségét elvetted, hogy feleséged legyen.
\par 11 Ezt mondja az Úr: Ímé én épen a saját házadból bocsátok reád csapásokat, és feleségeidet szemed láttára veszem el, és adom más felebarátodnak, és hál a te feleségeiddel fényes nappal.
\par 12 Mert te titkon cselekedtél; de én az egész Izráel elõtt és napvilágnál cselekeszem azt.
\par 13 Monda azért Dávid Nátánnak: Vétkeztem az Úr ellen! És monda Nátán Dávidnak: Az Úr is elvette a te bûnödet, nem fogsz meghalni.
\par 14 Mindazáltal mivel alkalmat adtál a gyalázásra az Úr ellenségeinek e dologban: a te fiad is, a ki lett néked, bizonynyal meghal.
\par 15 Ezekután elméne Nátán az õ házához. És megveré az Úr a gyermeket, a kit az Uriás felesége szült vala Dávidnak; és megbetegedék.
\par 16 És könyörge Dávid az Istennek a gyermekért, és bõjtöle is Dávid, és bemenvén, a földön feküvék éjjel.
\par 17 Felkelének azért az õ házának vénei és menének õ hozzá, hogy felemeljék õt a földrõl: de nem akará, és nem is evék õ velek kenyeret.
\par 18 Hetednapra azért meghala a gyermek, és nem merik vala a Dávid szolgái néki megmondani, hogy meghalt a gyermek, mert ezt mondják vala: Ímé, még mikor a gyermek élt, szólottunk néki és meg sem hallotta szónkat; hogyan mondanánk meg néki, hogy meghalt a gyermek, hogy magának bajt szerezzen?
\par 19 Látván pedig Dávid, hogy az õ szolgái suttognak, eszébe vevé Dávid, hogy meghalt a gyermek, és monda Dávid az õ szolgáinak: Meghalt-é a gyermek? Azok mondának: Meghalt.
\par 20 Felkelvén azért Dávid a földrõl, megmosdék és megkené magát és más ruhát võn magára, és bemenvén az Úr házába, imádkozék. Azután beméne a maga házába, és kérésére kenyeret vivének eleibe, és evék.
\par 21 Akkor mondának az õ szolgái néki: Mi dolog ez, a mit míveltél? Míg a gyermek élt, bõjtöltél és sírtál; most pedig, hogy meghalt a gyermek, felkelél és kenyeret ettél.
\par 22 Monda õ: Míg a gyermek élt, addig bõjtöltem és sírtam; mert ezt mondottam: Kit tudja, talán az Úr könyörül rajtam, és megél a gyermek.
\par 23 De most, hogy meghalt, vajjon miért bõjtölnék? Vajjon visszahozhatom-é azzal? Én megyek õ hozzá, de õ nem jõ ide vissza én hozzám.
\par 24 És megvigasztalá Dávid az õ feleségét, Bethsabét, és beméne hozzá, és vele hála. És õ szült fiat, és nevezé annak nevét Salamonnak, és az Úr szereté azt,
\par 25 A mint megizente vala Nátán próféta által, ki nevezé az õ nevét Jedidjának, az Úrért.
\par 26 Joáb pedig hadakozék az Ammon fiainak városa, Rabba ellen, és megvevé a királyi várost.
\par 27 És követeket külde Dávidhoz Joáb ilyen követséggel: Hadakoztam Rabba ellen, és meg is vettem a városnak egyik részét, a hol a víz van.
\par 28 Most azért gyûjtsd egybe a népnek maradékát, és szállj táborba a város ellen, foglald el azt, nehogy valamiképen, ha én foglalnám el, róla neveztessék az én nevem.
\par 29 Egybegyûjté azért Dávid mind az egész népet, és aláméne Rabba ellen, és harczola ellene, és elfoglalá azt.
\par 30 És elvevé az õ királyuknak koronáját annak fejérõl, melynek súlya egy tálentum arany volt, és drága kövekkel vala rakva, és lõn a Dávid fején, és a városból felette sok zsákmányt hozott el.
\par 31 A népet pedig, mely benne vala, kihozatá; és némelyét fûrész, némelyét vasborona alá, némelyét fejsze alá vetteté; némelyeket mészkemenczén vitt által, és így cselekedék az Ammon fiainak minden városával: Haza méne azután Dávid és az egész nép Jeruzsálembe.

\chapter{13}

\par 1 Lõn ennekutána hogy Absolonnak, Dávid fiának igen szép huga vala, kinek neve Támár vala; és Amnon, a Dávid fia megszereté õt.
\par 2 Igen nagy gyötrelemben vala pedig Amnon, úgy hogy beteggé lett az õ hugáért, Támárért; mert szûz vala, és Amnon elõtt lehetetlennek tünt fel, hogy rajta valamit elkövessen.
\par 3 Vala azonban Amnonnak egy barátja, kinek Jonadáb vala neve, Simeának, Dávid testvérének fia; Jonadáb pedig igen eszes ember vala.
\par 4 Ki monda néki: Mi az oka, hogy te naponként soványodol, királynak fia? Nem mondhatnád-é meg nékem? És monda néki Amnon: Támárt, Absolon öcsémnek hugát igen szeretem.
\par 5 És monda néki Jonadáb: Feküdj le ágyadba, és tedd betegnek magadat. És ha eljön atyád, hogy meglátogasson, mondd azt néki: Jõjjön ide, kérlek, Támár, az én hugom, hadd adjon ennem; és itt szemem elõtt készítse el az ételt, hogy én is lássam, és az õ kezébõl egyem.
\par 6 Lefeküvék azért Amnon, és tetteté, mintha beteg volna; mikor azután eljött a király, hogy õt meglátogassa, monda Amnon a királynak: Kérlek, hadd jõjjön ide hozzám Támár hugom, hadd csináljon elõttem egy pár bélest, és hadd egyem az õ kezébõl.
\par 7 Elkülde azért Dávid a Támár házához, ezt izenvén: Eredj el mindjárt az Amnon bátyád házához, és készíts valami ennivalót néki.
\par 8 Beméne azért Támár az Amnon bátyja házába, õ pedig fekszik vala. És lisztet vévén, meggyúrá és bélest csinála õ elõtte, és megfõzé a bélest.
\par 9 Elõhozá annakutána a serpenyõt, és kitölté eleibe; de õ nem akara enni. És monda Amnon: Küldjetek ki mellõlem mindenkit; és kimenének mindnyájan elõle.
\par 10 Akkor monda Amnon Támárnak: Hozd be a kamarába az étket, hadd egyem kezedbõl; vevé azért Támár a bélest, melyet készített vala, és bevivé Amnon bátyjának az ágyasházba.
\par 11 És eleibe vivé, hogy egyék, és megragadá õt, és monda néki: Jõjj, feküdj mellém, húgom.
\par 12 Õ pedig monda néki: Ne, bátyám, engem meg ne ronts, mert nem szoktak így cselekedni Izráelben, ne kövess el ilyen gyalázatot.
\par 13 És én, ugyan hová vigyem szégyenemet? Te pedig olyan leszel Izráelben, mint egy bolond. Azért kérlek, szólj a királynak, mert nem fog megtagadni engem tõled.
\par 14 Õ azonban nem akart szavára hallgatni, hanem erõsebb levén nála, erõt vett rajta, és vele feküvék.
\par 15 És meggyûlölé õt Amnon felette igen, mert nagyobb lõn gyûlölete, melylyel gyûlölte õt, a szeretetnél, melylyel õt megszerette vala. És monda néki Amnon: Kelj fel, eredj dolgodra.
\par 16 Ki felele néki: Ne kövess el nagyobb gonoszságot annál, a melyet rajtam véghezvittél, hogy elûzz engem. Õ azonban nem akart reá hallgatni,
\par 17 Hanem beszólítá szolgáját, a ki néki szolgál vala, és monda: Ûzd ki õt gyorsan elõlem, és zárd be az ajtót utána.
\par 18 Vala pedig õ rajta igen szép tarka szoknya, a milyenben a király leányai szoktak járni, míg szûzek valának. Kiûzé azért õt a szolga, és bezárá az ajtót utána.
\par 19 Hamut hinte azért Támár az õ fejére, és a tarka szoknyát, mely rajta volt;  meghasogatá, kezét pedig fejére tevén, jajgatva jár vala.
\par 20 És monda néki a bátyja, Absolon: Talán Amnon bátyád volt veled? Azért hallgass most, hugom, mert atyádfia, ne bánkódjál azon felettébb. Lakozék azért Támár nagy árvaságban az õ bátyjának, Absolonnak házában.
\par 21 Dávid király pedig hallván mindezeket, felette igen megharaguvék.
\par 22 Nem szóla pedig semmit felõle Absolon Amnonnak, sem jót, sem gonoszt; mert igen gyûlöli vala Absolon Amnont, mivelhogy megszeplõsítette az õ hugát, Támárt.
\par 23 És lõn két esztendõ mulva, mikor Absolonnak juhait nyírták Baál-Hásorban, mely Efraimban van, meghívá Absolon mind a király fiait.
\par 24 Beméne Absolon a királyhoz is, és monda: Ímé most nyírják a te szolgádnak juhait, azért jõjjön el kérem a király és az õ szolgái a te szolgáddal.
\par 25 Monda a király Absolonnak: Ne, fiam, ne menjünk el mindnyájan, hogy meg ne terheljünk téged. És ismét erõlteti vala õt, de nem akara elmenni, hanem megáldá õt.
\par 26 Monda mégis Absolon: Ha nem, úgy jõjjön el velünk Amnon, az én testvérem. Felele néki a király: Miért menne el veled?
\par 27 Mikor pedig erõltette õt Absolon, elbocsátá õ vele Amnont is és mind a király fiait.
\par 28 Parancsola pedig Absolon az õ szolgáinak, ezt mondván: Kérlek, vigyázzatok, és mikor Amnon a bortól jókedvû lesz és mondom néktek: akkor üssétek le Amnont és öljétek meg õt, semmit ne féljetek, hiszen én parancsoltam néktek, legyetek bátrak, ne féljetek.
\par 29 Úgy cselekedének azért az Absolon szolgái Amnonnal, a mint Absolon parancsolta vala. A király fiai pedig mindnyájan felkelének, és ki-ki öszvérére üle, és elszaladának.
\par 30 Mikor pedig még az útban voltak, a hír eljutott Dávidhoz, mondván: Mind megölte Absolon a király fiait, egy sem maradt meg közülök.
\par 31 Akkor felkele a király, megszaggatá ruháit, és a földre feküvék, és az õ szolgái mindnyájan megszaggatott ruhában állának vala elõtte.
\par 32 Szóla pedig Jonadáb, Simeának, a Dávid testvérének fia, és monda: Ne mondja azt az én uram, hogy a királynak minden fiait megölték, mert csak Amnon halt meg egyedül; mert attól a naptól fogva, hogy az õ hugát megszeplõsítette, Absolonnak mindig szájában volt ez a dolog.
\par 33 Ne vegye azért szívére az én uram, a király, azt gondolván, hogy a királynak minden fiai meghaltak, mert csak Amnon halt meg egyedül.
\par 34 Absolon pedig elmenekült. És felemelvén az õrálló az õ szemeit, látá, hogy sok ember jõ az úton õ mögötte, a hegyoldalon.
\par 35 És monda Jonadáb a királynak: Ímé jõnek a király fiai; a mint a te szolgád mondá, úgy történt.
\par 36 És lõn, a mint megszünt beszélni, megérkezének a király fiai, és szavokat felemelvén, sírának; és maga a király is és az õ szolgái mindnyájan felette igen sírának.
\par 37 Absolon pedig elfuta, és méne Talmaihoz, Ammihur fiához, Gessurnak királyához. És Dávid minden nap siratá az õ fiát.
\par 38 Absolon pedig, minekutána elfutott és Gessurba ment, három esztendeig volt ott.
\par 39 Dávid király pedig felhagyott azzal, hogy Absolon ellen menjen, mert megvigasztalódott Amnon felõl, hogy meghalt.

\chapter{14}

\par 1 Észrevevén Joáb, Sérujának fia, hogy a királynak szíve vágyakozik Absolon után,
\par 2 Elkülde Joáb Tékoa városába, és hozata onnét egy asszonyt, ki igen eszes vala, és monda néki: Kérlek tetessed, mintha nagy keserûséged volna, és öltözzél fel gyászruhába, és olajjal kend meg magadat; és légy olyan, mint aféle asszony, ki sokáig siratta halottját.
\par 3 És menj be a királyhoz, s így és így szólj hozzá. És Joáb szájába adta, hogy mit kelljen szólani.
\par 4 Szóla azért a Tékoabeli asszony a királynak, minekutána arczczal a földre leborult, és térdet-fejet hajtott, és monda: Segíts meg, óh király!
\par 5 És monda néki a király: Mi bajod van? Felele az: Óh, én özvegyasszony vagyok, mert az én férjem meghalt.
\par 6 És a te szolgálódnak két fia vala, kik összevesztek a mezõn, és mivel nem vala senki, a ki õket megvédte volna, az egyik megsérté a másikat és megölé.
\par 7 És ímé az egész háznép ellene támadott a te szolgálóleányodnak, és ezt mondják: Add kezünkbe az õ testvérének gyilkosát, hadd öljük meg õt az õ testvérének lelkéért, a kit megölt, és veszessük el az örököst is. Így akarják eloltani a kicsiny szikrácskát, a mely nékem megmaradott, hogy az én férjemnek ne maradjon se neve, se maradéka a föld színén.
\par 8 Monda azért a király az asszonynak: Menj el házadhoz, és parancsolok a te dolgod felõl.
\par 9 Felele pedig a Tékoából való asszony a királynak: Uram király, én rajtam legyen a bûn súlya és az én atyámnak házán, de a király és az õ trónja ártatlan leszen.
\par 10 Monda erre a király: A ki te ellened szól, hozd ide elõmbe, és többé nem fog illetni téged.
\par 11 Akkor õ monda: Emlékezzék meg kérlek, a király az Úrról, a te Istenedrõl: hogy a vérbosszúló ne szaporítsa a pusztulást, és hogy az én fiamat ne veszessék el. Felele a király: Él az Úr, hogy a te fiadnak egy hajszála sem esik le a földre.
\par 12 És monda az asszony: Kérlek, hadd szóljon a te szolgálód csak egy szót az én uramnak, a királynak; és õ monda: Szólj.
\par 13 Akkor monda az asszony: Miért gondoltál ehhez hasonló dolgot az Isten népe ellen (mert mivel a király ezt a szót szólotta, mintegy maga is bûnös), hogy a király azt, a kit eltaszított magától, nem hívatja vissza?
\par 14 Mert bizonyára meg kell halnunk, és olyanok vagyunk, mint a víz, mely a földre kiöntetvén, fel nem szedhetõ, és az Isten egy lelket sem akar elvenni, hanem azt a gondolatot gondolja magában, hogy ne legyen számkivetve elõtte az eltaszított sem.
\par 15 Most annakokáért azért jöttem ide, hogy én szólnék a királynak, az én uramnak, noha sokan rettentettek engem ettõl; mindazáltal azt mondotta a te szolgálód: Mégis beszélek a királylyal, hátha megteszi a király, a mit az õ szolgálóleánya mond.
\par 16 Igen, meghallgatja a király, és megszabadítja az õ szolgálóleányát annak kezébõl, a ki engem el akar veszteni és velem együtt az én fiamat az Istennek örökségébõl.
\par 17 Annakfelette ezt gondolta a te szolgálóleányod: Az én uramnak, a királynak beszéde szerezzen nyugodalmat, mert mint az Istennek angyala, olyan az én uram, a király, mivelhogy meghallgatja mind a jót, mind a gonoszt. És az Úr a te Istened legyen te veled.
\par 18 És felelvén a király, monda az asszonynak: Kérlek, ne tagadd meg, a mit tõled kérek. És monda az asszony: Mondja el az én uram, a király, kérlek!
\par 19 És monda a király: Vajjon mindezekben nem a Joáb keze van-é veled? Felele az asszony, és monda: Él a te lelked, óh uram, király, hogy sem jobbra, sem balra nem lehet térni attól, a mit az én uram, a király szól; mert a te szolgád Joáb  hagyta ezt nékem, és mindezeket a szókat õ adta a te szolgálóleányodnak szájába.
\par 20 Hogy a dolognak más fordulatot adjon, azért tette ezt Joáb, a te szolgád. De az én uram bölcs, az Isten angyalának bölcsesége szerint, hogy mindent észrevegyen, a mi a földön van.
\par 21 Akkor monda a király Joábnak: Ímé megteszem ezt a dolgot. Eredj el, és hozd haza az én fiamat, Absolont.
\par 22 És a földre arczczal leborula Joáb, és térdet-fejet hajtván, megköszöné a királynak, és monda Joáb: Ma ismerte meg, uram király, a te szolgád, hogy van valami becsületem elõtted; mert az õ szolgájának beszédét megcselekedte a király.
\par 23 Felkele azért Joáb, és elméne Gessurba, és haza hozá Absolont Jeruzsálembe.
\par 24 És monda a király: Menjen a maga házába, és az én színemet ne lássa. Tére azért Absolon az õ házába, és a királynak orczáját nem láthatá.
\par 25 Nem vala pedig az egész Izráelben olyan szép ember, mint Absolon, ki dicséretre olyan méltó volna; tetõtõl fogva talpig õ benne semmi hiba nem vala.
\par 26 És mikor a fejét megnyiratja (mert minden esztendõben megnyiratja vala, mivel igen nehéz volna, azért nyiratja le), az õ fejének haja nyom vala kétszáz siklust a királyi mérték szerint.
\par 27 Lõn pedig Absolonnak három fia és egy leánya, kinek neve Támár vala; ez igen szép termetû asszony vala.
\par 28 Két esztendõt töltött immár Absolon Jeruzsálemben, de a királynak színét még nem látta.
\par 29 Elkülde azért Absolon Joábhoz, hogy õt a királyhoz küldje, ki nem akara hozzá menni; és elkülde másodszor is, de õ még sem akart elmenni.
\par 30 Monda azért az õ szolgáinak: Nézzétek, a Joáb gazdasága az enyém mellett van, és ott van az õ árpája: menjetek el, és gyújtsátok fel tûzzel; és meggyújták az Absolon szolgái a gazdaságot tûzzel.
\par 31 Felkele azért Joáb, és méne Absolonhoz az õ házába, és monda néki: Mi az oka, hogy a te szolgáid az én gazdaságomat felgyújtották tûzzel?
\par 32 Felele Absolon Joábnak: Azért mert hozzád küldöttem ily szóval, hogy ide jõjj, hogy a királyhoz küldjelek, hogy ezt mondjad néki: Mi szükség volt hazajõnöm Gessurból? Jobb volna most is nékem ott lennem. Most azért szeretném a király arczát látni; ha van bennem álnokság, ölessen meg engem.
\par 33 Elméne azért Joáb a királyhoz, és megmondá néki. És akkor hivatá a király Absolont, és elméne a királyhoz, és fejet hajtván a király elõtt, arczczal a földre borula. És megcsókolá a király Absolont.

\chapter{15}

\par 1 Lõn pedig annakutána, szerze magának Absolon szekeret, lovakat és ötven embert, kik elõtte szaladjanak.
\par 2 És reggelenként felkelvén Absolon, megálla az útfélen a kapuban, és mindenkit, a kinek dolga lévén, a királyhoz megy ítélet végett, megszólíta Absolon, és megkérdé: Micsoda városból való vagy te? És ha azt mondá: Izráelnek egyik nemzetségébõl való a te szolgád;
\par 3 Monda néki Absolon: Ímé a te beszéded mind jó és mind igaz; de senki sincs, a ki téged meghallgatna a királynál.
\par 4 Monja vala ismét Absolon: Vajha valaki engem tenne ítélõbíróvá e földön, és én hozzám jõne minden ember, a kinek valami ügye és pere volna, igazat tennék néki.
\par 5 Mikor pedig valaki hozzá megy és fejet hajt vala néki, azonnal kezét nyujtja vala, és megfogván, megcsókolja vala õt.
\par 6 És e képen cselekedék Absolon egész Izráellel, valakik ítéletért a királyhoz mennek vala, és így Absolon az Izráel fiainak szíveket alattomban megnyeri vala.
\par 7 Lõn pedig negyven esztendõ mulván, monda Absolon a királynak: Hadd menjek el, és teljesítsem Hebronban azt a fogadást, melyet fogadtam az Úrnak;
\par 8 Mert fogadást tett a te szolgád, mikor Gessurban laktam, mely Siriában van, ezt mondván: Ha valóban hazavezérel engem az Úr Jeruzsálembe, az Úrnak szolgálok.
\par 9 Monda néki a király: Menj el békével; és felkelvén, elméne Hebronba.
\par 10 Hírnököket külde pedig Absolon Izráelnek minden nemzetségéhez, hogy megmondják: Mikor a trombitaszót halljátok, azt mondjátok: Absolon uralkodik Hebronban.
\par 11 És Absolonnal együtt kétszáz férfi is elméne Jeruzsálembõl, kiket meghívott, kik jóhiszemûleg menének, semmit nem tudván a dologról.
\par 12 És elküldvén Absolon, hivatá a Gilóból való Akhitófelt is, Dávidnak tanácsosát, az õ városából Gilóból, míg õ az áldozatot végezte. És igen nagy lõn az összeesküvés, és a nép gyülekezék és szaporodék Absolon mellett.
\par 13 És követ méne Dávidhoz ilyen izenettel: Az Izráel népének szíve Absolon felé hajlik.
\par 14 Akkor monda Dávid minden szolgáinak, kik vele valának Jeruzsálemben: Keljetek fel, és fussunk el, mert itt nincsen számunkra menekülés Absolon elõl; siessetek elmenni, hogy valamikép sietve utól ne érjen minket, s ne hozzon szerencsétlenséget reánk, és a várost le ne vágja fegyvernek élével.
\par 15 Mondának pedig a király szolgái a királynak: Minden úgy legyen, a mint tetszik a királynak, a mi urunknak, ímé itt vannak a te szolgáid.
\par 16 Kiméne azért a király és egész háznépe õ utána és otthon hagyá a király tíz ágyasát, hogy õrizzék a házat.
\par 17 És kimenvén a király és egész háznépe õ utána, megállapodának a legszélsõ háznál.
\par 18 És minden szolgái õ mellé menének; és a Kereteusok, a Peleteusok mind, a  Gitteusok is mind, az a hatszáz férfi, kik vele jövének Gáthból, elvonulának a király elõtt.
\par 19 Monda pedig a király a Gitteus Ittainak: Miért jössz el te is mi velünk? Menj vissza, és maradj a királynál, mert te idegen vagy és vissza is költözhetel szülõ helyedre.
\par 20 Csak tegnap jöttél, és már ma zaklassalak téged, hogy velünk jõjj? Én megyek oda, a hova mehetek; te pedig térj vissza, és vidd vissza testvéreidet is; irgalmasság és igazság legyen veled.
\par 21 És felel Ittai a királynak, mondván: Él az Úr és él az én uram, a király, hogy valahol lesz az én uram, a király, mind halálában, mind életében, ott lesz a te szolgád is.
\par 22 Monda azért Dávid Ittainak: Ám jõjj el és menjünk el. És elméne a Gitteus Ittai és az õ emberei együtt, még a kicsinyek is, valakik vele valának.
\par 23 És az egész föld népe nagy jajgatással sír vala, mikor az egész nép elméne. A király azért általméne a Kedron patakán, és a nép mind átméne az útra, a puszta felé.
\par 24 És ímé vele vala Sádók is, és a Léviták mind, kik az Isten szövetségének ládáját hordozzák vala; és letevék az Isten ládáját. Azonközben Abjátár  is felméne, míg a nép a városból mind kitakarodék.
\par 25 És monda a király Sádóknak: Vidd vissza az Isten ládáját a városba, ha én az Úr elõtt kedves leszek, engem ismét haza hoz, megmutatja nékem azt, és az õ sátorát.
\par 26 Ha pedig azt mondja: Nem gyönyörködöm benned: Ímhol vagyok, cselekedjék velem úgy, a mint néki tetszik.
\par 27 Monda annakfelette a király Sádók papnak: Nemde nem próféta vagy-é te? Azért menj haza békességben a városba, Akhimás is, a te fiad, és az Abjátár fia, Jonathán, a ti két fiatok, veletek együtt.
\par 28 Lássátok, ímé én itt idõzöm, ennek a pusztának sík mezején, míg tõletek hír jõ, és nékem izentek.
\par 29 Visszavivék azért Sádók és Abjátár Jeruzsálembe az Isten ládáját, és otthon maradának.
\par 30 Dávid pedig felméne az olajfáknak hegyén, mentében sírva, fejét beborítva, saru nélkül ment, és az egész nép, mely vele volt, kiki beborította fejét, és mentökben sírának.
\par 31 Megizenék azonközben Dávidnak, hogy Akhitófel is a pártosok között van Absolonnal, és monda Dávid: Kérlek, óh Uram, hiúsítsd meg az Akhitófel tanácsát.
\par 32 És a mint feljuta Dávid a hegy tetejére, hogy ott imádkozzék az Istennek, ímé eleibe jöve az Árkeából való Khúsai, ki ruháját megszaggatá és földet  hinte fejére.
\par 33 És monda néki Dávid: Ha eljösz velem, terhemre leszel nékem;
\par 34 De ha a városba visszatérsz, és ezt mondod Absolonnak: Te szolgád vagyok, óh király; ennekelõtte a te atyád szolgája voltam, most immár a te szolgád leszek, megronthatod Akhitófelnek ellenem való tanácsát.
\par 35 És ímé veled lesznek ott Sádók és Abjátár papok; azért minden dolgot, a mit hallándasz a király házából, mondj meg a papoknak, Sádóknak és Abjátárnak.
\par 36 Ímé ott van velök az õ két fiok is, Akhimás, a Sádók fia, és Jonathán, az Abjátár fia, kik által nékem mindjárt megizenhetitek, valamit hallotok.
\par 37 Elméne azért Khúsai, a Dávid barátja a városba, Absolon pedig bevonula Jeruzsálembe.

\chapter{16}

\par 1 Mikor pedig Dávid egy kevéssé alább ment a hegy tetejérõl, eleibe jöve Siba, a Mefibóset szolgája, két megnyergelt szamárral, melyeken kétszáz kenyér és száz kötés aszúszõlõ és száz csomó füge és egy tömlõ bor vala.
\par 2 És monda a király Sibának: Mit akarsz ezzel? Felele Siba: A szamarak a király háznépéé legyenek, hogy rajtok járjanak; a kenyér pedig és a füge, hogy a szolgák megegyék, és a bor, hogy igyék, a ki megfárad a pusztában.
\par 3 És monda a király: Hol van most a te uradnak fia? Felele Siba a királynak: Ímé Jeruzsálemben marada, mert azt mondja: Visszaadja  ma nékem Izráel háznépe az én atyámnak országát.
\par 4 És monda a király Sibának: Ímé minden tied legyen, valamije volt Mefibósetnek. Monda akkor Siba: Meghajtom magamat; vajha kegyelmet találnék elõtted, óh uram király!
\par 5 Elméne azután Dávid király Bahurimig, és ímé onnan egy férfi jöve ki a Saul nemzetségébõl való, kinek neve Sémei vala, Gérának fia, és kijövén szidalmazza vala õket.
\par 6 És kõvel hajigálá Dávidot és Dávid királynak minden szolgáit, jóllehet az egész nép és az erõs férfiak mindnyájan az õ jobb és balkeze felõl valának.
\par 7 És így szóla Sémei szitkozódása közben: Eredj, eredj te vérszopó és istentelen ember!
\par 8 Megfizet most az Úr néked Saul egész házanépének véréért, a ki helyett te uralkodol; és adta az Úr az országot a te fiadnak, Absolonnak: és ímé te nyomorúságban vagy, mert vérszopó ember vagy!
\par 9 Monda pedig Abisai, Sérujának fia, a királynak: Hogyan szidalmazhatja ez a holt eb az én uramat, a királyt? Majd én elmegyek és fejét veszem.
\par 10 Monda pedig a király: Mi közöm van veletek, Sérujának fiai? Hadd szidalmazzon, mert az Úr mondotta néki: Szidalmazzad Dávidot; és ki mondhatja néki: Miért míveled ezt?
\par 11 És monda Dávid Abisainak és minden szolgáinak: Íme az én fiam, ki az én ágyékomból származott, kergeti az én életemet: hogyne cselekedné tehát e Benjáminita? Hagyjatok békét néki, hadd szidalmazzon; mert az Úr mondotta néki.
\par 12 Netalán reá tekint az Úr az én nyomorúságomra, és jóval fizet még ma nékem az Úr az õ átka helyett.
\par 13 És megy vala Dávid és az õ népe az úton, Sémei pedig a hegyoldalon átellenében menvén, mentében átkozódik és köveket hajigál vala õ ellenébe, és port hány vala.
\par 14 Eljuta annakutána a király és az egész nép, mely vele vala, Ajefimbe, és ott megnyugovék.
\par 15 Absolon pedig és az egész nép, Izráelnek férfiai, bemenének Jeruzsálembe, és Akhitófel is õ vele.
\par 16 Mikor pedig az Arkeából való Khúsai, a Dávid barátja bement Absolonhoz, monda Khúsai Absolonnak: Éljen a  király, éljen a király!
\par 17 És monda Absolon Khúsainak: Ez-é a barátod iránti szereteted? Miért nem mentél el barátoddal?
\par 18 Felele Khúsai Absolonnak: Nem, hanem a kit az Úr és ez a nép választ, és az Izráelnek minden fiai: azé leszek és azzal maradok.
\par 19 Azután ugyan kinek szolgálnék örömestebb, mint az én barátom fiának? A mint szolgáltam a te atyádnak, szintén olyan leszek te hozzád is.
\par 20 Monda pedig Absolon Akhitófelnek: Adjatok tanácsot, mit cselekedjünk?
\par 21 Felele Akhitófel Absolonnak: Menj be a te atyádnak ágyasaihoz, a kiket itthon hagyott, hogy õriznék a házat: és megérti az egész Izráel, hogy te atyád elõtt gyûlöltté tetted magadat, és annál inkább megerõsödnek mindazoknak kezeik, a kik melletted vannak.
\par 22 Sátort vonának azért Absolonnak a tetõn, és beméne Absolon az õ atyjának ágyasaihoz, az egész Izráelnek szeme láttára.
\par 23 És Akhitófel tanácsa, melyet adott, olyannak tekintetett abban az idõben, mintha valaki az Isten szavát kérdezte volna; olyan volt Akhitófelnek minden tanácsa mind Dávid elõtt, mind Absolon elõtt.

\chapter{17}

\par 1 Monda azért Akhitófel Absolonnak: Engedj kiválasztanom tizenkétezer embert, hogy felkeljek, és üldözzem Dávidot ez éjjel.
\par 2 És megtámadom õt, míg fáradt és erõtlen kezû; megrettentem õt, és megfutamodik az egész nép, mely vele van; és a királyt magát megölöm.
\par 3 És visszavezetem te hozzád az egész népet, mert az egésznek visszatérése attól a férfiútól függ, a kit te üldözöl; és akkor az egész nép békességben lesz.
\par 4 Igen tetszék e beszéd Absolonnak és Izráel minden véneinek.
\par 5 És monda Absolon: Hívják ide mégis az Arkeabeli Khúsait is, és hallgassuk meg, mit szól õ is.
\par 6 És mikor megérkezék Khúsai Absolonhoz, monda néki Absolon ilyenképen: Akhitófel ilyen tanácsot ád, megfogadjuk-é az õ szavát, vagy ne? Szólj hozzá.
\par 7 Monda akkor Khúsai Absolonnak: Nem jó tanács az, a melyet Akhitófel ez egyszer adott.
\par 8 És monda Khúsai: Tudod magad, hogy a te atyád és az õ emberei igen erõs vitézek és igen elkeseredett szívûek, mint a kölykeitõl  megfosztott medve a mezõn. Annakfelette a te atyád igen hadakozó ember, ki nem alszik a néppel együtt.
\par 9 Ilyenkor õ valami barlangban lappang, vagy valami más helyen, és meglehet, hogy mindjárt kezdetben elesvén némelyek közülök, valaki meghallja, és azt kezdi mondani: Megveretett az Absolon népe.
\par 10 És még az erõs vitéz is, kinek szíve olyan, mint az oroszlánnak szíve, igen megrémül; mert az egész Izráel tudja, hogy a te atyád igen erõs vitéz, és azok is, a kik vele vannak, erõs vitézek.
\par 11 Én azért azt tanácsolom, hogy gyûjtsd magadhoz az egész Izráelt Dántól fogva mind Bersebáig, oly számban, mint a tenger partján való föveny; és magad is menj el a hadba.
\par 12 Akkor aztán támadjuk meg õt azon a helyen, a hol található, és úgy lepjük meg õt, miként a harmat a földre esik, hogy se közüle, se azok közül, a kik vele vannak, egy se maradjon meg.
\par 13 Ha pedig városba szaladna, mind az egész Izráel köteleket húzzon a város körül, és vonjuk azt a patakba, hogy még csak egy kövecskét se találjanak ott.
\par 14 És monda Absolon és Izráelnek minden férfia: Jobb az Arkeabeli Khúsainak tanácsa az Akhitófel tanácsánál. Az Úr parancsolta vala pedig, hogy az Akhitófel tanácsa elvettessék, mely jó vala, hogy veszedelmet hozzon az Úr Absolonra.
\par 15 Monda pedig Khúsai Sádók és Abjátár papoknak: Ilyen s ilyen tanácsot adott Akhitófel Absolonnak és Izráel véneinek; én pedig ilyen s ilyen tanácsot adtam.
\par 16 Azért sietve küldjetek el, és izenjétek meg Dávidnak ilyen szóval: Ne maradj ez éjjel a pusztának mezején, hanem inkább menj át, hogy valamiképen el ne nyelettessék a király és az egész nép, mely vele van.
\par 17 Jonathán pedig és Akhimás a Rógel  forrásánál állanak vala, a hová méne egy leányzó, ki megmondá ezeket nékik, és õk elmenvén, megmondák Dávid királynak, mert nem akarják vala magokat megmutatni, bemenvén a városba.
\par 18 Meglátá mindazáltal õket egy szolga, és megmondá Absolonnak: Elmenének azért õk ketten sietséggel, és menének egy ember házába Bahurim városában, kinek tornáczában volt egy kút, és oda szállának alá.
\par 19 Az asszony pedig egy terítõt võn és beteríté a kút száját, arra pedig darált árpát hintett; és nem vevék észre.
\par 20 Menének pedig az Absolon szolgái az asszonyhoz a házba, és mondának: Hol van Akhimás és Jonathán? Felele nékik az asszony: Általgázolának a patakon. És keresék õket, de mivel nem találták, visszatérének Jeruzsálembe.
\par 21 Mikor pedig elmentek, kijövének a kútból, és elmenvén, megmondák ezeket Dávid királynak, és mondának Dávidnak: Keljetek fel és menjetek által gyorsan a vizen; mert ilyen tanácsot adott ti ellenetek Akhitófel.
\par 22 Felkele azért Dávid és mind a vele való nép, és általkelének a Jordánon, míg megvirrada; egy sem hiányzék, a ki által nem ment volna a Jordánon.
\par 23 Látván pedig Akhitófel, hogy az õ tanácsát nem hajtották végre: megnyergelé szamarát, és felkelvén elméne házához, az õ városába; és elrendezvén háznépének dolgát, megfojtá  magát, és meghala; és eltemetteték az õ atyjának sírjába.
\par 24 És midõn Dávid Mahanáimba érkezett, átkele Absolon a Jordánon, õ és Izráelnek férfiai mindnyájan õ vele.
\par 25 És Amasát tevé Absolon fõvezérré a sereg felett Joáb helyett. Amasa pedig egy férfi vala, a kinek a neve az Izráelita Jithra vala, a ki beméne Abigailhoz,  a Náhás leányához, ki Sérujának, a Joáb anyjának nõvére vala.
\par 26 És tábort jára Izráel és Absolon a Gileád földén.
\par 27 Lõn pedig, hogy mikor Dávid Mahanáimba jutott, ímé Sóbi, az Ammoniták városából, Rabbából való Náhásnak fia, és a Ló-Debár városból való Ammielnek fia, Mákir, és a Gileád tartományában, Rógelimban lakozó  Barzillai.
\par 28 Ágynemût, medenczéket és cserépedényeket, búzát, árpát, lisztet, pergelt búzát, babot, lencsét, pergelt árpát,
\par 29 Mézet, vajat, juhot, ünõsajtokat hozának Dávidnak és az õ vele való népnek eleségül. Mert azt gondolják vala magokban: A nép éhes, fáradt és eltikkadt a pusztában.

\chapter{18}

\par 1 Megszámlálá pedig Dávid a népet, a mely vele vala, s ezredeseket és századosokat rendele föléjök.
\par 2 És hagyá Dávid a népnek harmadrészét Joáb keze alatt, és harmadrészét Séruja fiának,  Abisainak, Joáb atyjafiának keze alatt, harmadrészét pedig bízá a Gitteus Ittai kezére. És monda a népnek a király: Bizony én magam is elmegyek veletek.
\par 3 De a nép monda: Ne jõjj; mert ha netalán mi megfutamodunk is, velünk nem gondolnak, és ha felerészben meghalunk is, velünk semmit sem gondolnak; de te teszesz anynyit, mint mi tízezeren: jobb azért, hogy te a városból légy nekünk segítségül.
\par 4 Monda azért nékik a király: A mi néktek jónak tetszik, én azt mívelem. Megálla azért a király a kapuban, és az egész sereg megy vala ki százanként és ezerenként.
\par 5 Parancsola pedig a király Joábnak, Abisainak és Ittainak, mondván: Az én fiammal, Absolonnal én érettem kiméletesen bánjatok; hallá pedig ezt mind az egész had, mikor a király mindenik vezérnek parancsola Absolon felõl.
\par 6 Kiméne azért a nép a mezõre, az Izráel ellenébe, és megütközének az Efraim erdejénél.
\par 7 És megvereték ott az Izráel népe a Dávid szolgái által, és nagy veszteség volt ott azon a napon, mintegy húszezer emberé.
\par 8 És kiterjedt a harcz az egész vidékre, és a nép közül sokkal többet emészte meg az erdõ, mint a fegyver azon a napon.
\par 9 És találkozék Absolon Dávid szolgáival; Absolon pedig egy öszvéren ül vala. És beméne az öszvér a nagy cserfák sürû ágai alá, hol fennakadt fejénél fogva egy cserfán, függvén az ég és föld között, az öszvér pedig elszalad alóla.
\par 10 Kit mikor egy ember meglátott, hírül adá Joábnak, és monda: Ímé, láttam Absolont egy cserfán függeni.
\par 11 Monda pedig Joáb az embernek, a ki megmondotta vala néki: Ímé láttad, és miért nem ütötted le ott õt a földre? Az én dolgom lett volna azután, hogy megajándékozzalak tíz ezüst siklussal és egy övvel.
\par 12 Monda az ember Joábnak: Ha mindjárt ezer ezüstpénzt adnál is kezembe, nem ölném meg a király fiát; mert a mi fülünk hallására parancsolá a király néked, Abisainak és Ittainak, ezt mondván: Kiméljétek, bárki legyen, az ifjút, Absolont.
\par 13 Vagy ha orozva törtem volna életére - mivel a király elõtt semmi sem marad titokban - magad is ellenem támadtál volna.
\par 14 Monda azért Joáb: Nem akarok elõtted késedelmezni; és võn három nyilat kezébe és Absolonnak szívébe lövé, minthogy még élt a cserfán.
\par 15 Körülfogák akkor a Joáb fegyverhordozó szolgái tízen, és általverék Absolont, és megölék õt.
\par 16 Megfúvatá azután a trombitát Joáb, és megtére a nép Izráelnek ûzésébõl, mert kimélni akará Joáb a népet.
\par 17 Absolont pedig felvevék, és veték õt az erdõn egy nagy verembe, és igen nagy rakás követ hányának reá. És az egész Izráel elmeneküle, kiki az õ sátorába.
\par 18 Absolon pedig vett és még életében emelt magának emlékoszlopot, a mely a király völgyében van. Mert ezt mondja vala: Nincsen nékem oly fiam, a kin az én nevemnek emlékezete maradhatna; azért az oszlopot a maga nevére nevezé; és az Absolon oszlopának hívattatik mind e mai napig.
\par 19 Monda pedig Akhimás, Sádók fia: Majd elfutok, és megmondom a királynak, hogy megszabadította õt az Úr az õ ellenségeinek kezébõl.
\par 20 És monda néki Joáb: Ne légy ma hírmondó, hanem holnap mondd meg a hírt, ma pedig ne mondd meg; mivelhogy a király fia meghalt.
\par 21 Monda azonközben Joáb Kúsinak: Eredj el, mondd meg a királynak, a mit láttál; és meghajtá magát Kúsi Joáb elõtt, és elszaladt.
\par 22 És szóla ismét Akhimás, Sádók fia, és monda Joábnak: Bármint legyen, hadd fussak el én is Kúsi után! Monda Joáb: Miért futnál fiam; nem néked való ez a hírmondás?!
\par 23 Bármint legyen, hadd fussak el mégis! Õ pedig monda néki: Ám fuss el. Elfuta azért Akhimás a síkon való úton, és megelõzé Kúsit.
\par 24 Dávid pedig ül vala a két kapu között, és az õrálló felméne a kapu tetejére, a kõfalra, és felemelvén szemeit, látá, hogy egy ember igen fut egyedül.
\par 25 Kiálta azért az õrálló, és megmondá a királynak, és monda a király: Ha egyedül jõ, hír van az õ szájában. Amaz pedig mind közelebb jöve.
\par 26 Látá pedig az õrálló, és másik ember is fut, és lekiálta az õrálló a kapunállónak, mondván: Ímé más ember is fut egyedül. Akkor monda a király: Az is hírmondó.
\par 27 Monda ismét az õrálló: A mint látom, az elsõnek olyan a futása, mint Akhimásnak, a Sádók fiának; és monda a király: Jó ember az, és jó hírrel jõ.
\par 28 Kiáltván azért Akhimás, monda a királynak: Békesség! És meghajtá magát arczczal a földre a király elõtt, és monda: Áldott az Úr a te Istened, ki kezedbe adta az embereket, kik felemelték kezeiket az én uram ellen, a király ellen.
\par 29 Monda akkor a király: Hogy van Absolon fiam? Felele Akhimás: Látám a nagy sereglést, mikor elküldé Joáb a király szolgáját és a te szolgádat; de nem tudom, mi történt.
\par 30 És monda a király: Eredj tova, és állj meg ott; félreméne azért, és megálla ott.
\par 31 E közben Kúsi is megérkezett, és monda Kúsi: Ezt izenik a királynak, az én uramnak, hogy az Úr megszabadított ma téged mindeneknek kezébõl, a kik ellened támadtak volt.
\par 32 Monda a király Kúsinak: Hogy van Absolon fiam? Felele Kúsi: Úgy legyenek az én uramnak, a királynak minden ellenségei, és valakik te ellened gonoszul feltámadnak, mint a te fiad.
\par 33 És megháborodék a király, és felméne a kapu felett való házba, és síra, és ezt mondja vala mentében: Szerelmes fiam, Absolon! édes fiam, édes fiam, Absolon! bár én haltam volna meg te helyetted, Absolon, édes fiam, szerelmes fiam!

\chapter{19}

\par 1 És megjelenték Joábnak, hogy a király siratja és gyászolja Absolont.
\par 2 És gyászra fordult a szabadulás azon a napon az egész népre nézve, mert a nép azon a napon hallja vala, hogy beszélik: Így bánkódik a király az õ fián.
\par 3 És belopózkodék a nép azon a napon, bemenvén a városba, mint lopózkodni szokott a nép, mely szégyenli magát, hogy a harczból elmenekült.
\par 4 A király pedig eltakarván orczáját, fenszóval kiáltja vala a király: Édes fiam, Absolon! Absolon, édes fiam! szerelmes fiam!
\par 5 Akkor méne Joáb a házba a királyhoz, és monda: Megszégyenítetted e mai napon minden te szolgáidnak orczáját, kik e mai nap a te lelkedet megszabadították, és a te fiaidnak és leányidnak lelkeit, és feleségidnek lelkeit, és ágyasidnak lelkeit;
\par 6 Szeretvén azokat, a kik téged gyûlölnek és gyûlölvén azokat, a kik téged szeretnek; mert kijelentetted ma, hogy elõtted a vezérek és szolgák mind semmik; mert tapasztaltam ma, hogy csak élne Absolon, ha mi mindnyájan meghaltunk volna is ma, jobbnak tetszenék néked.
\par 7 Azért most kelj fel, menj ki, és szólj kedvök szerint a te szolgáidnak; mert esküszöm az Úrra, hogy ha ki nem jössz, ez éjjel egy ember sem marad melletted, és ez gonoszabb lesz reád nézve mindama nyomorúságnál, mely veled történt ifjúságodtól fogva, mind e mai napig.
\par 8 Felkele azért a király és leüle a kapuban, és hírül adták az egész népnek, mondván: Ímé a király a kapuban ül. Akkor mind az egész nép eleibe jöve a királynak. Izráel pedig elmenekült, kiki az õ sátorába.
\par 9 Mind az egész nép között pedig nagy veszekedés vala Izráelnek mindenik nemzetségében, kik ezt mondják vala: A király szabadított meg minket a mi ellenségeinknek kezébõl, a Filiszteusok kezébõl is õ szabadított meg; és most Absolon elõl elmenekült az országból.
\par 10 Absolon pedig, kit felkentünk vala, meghalt a hadban. Mit késtek azért hazahozni a királyt?
\par 11 Dávid király pedig elkülde Sádók és Abjátár papokhoz, mondván: Szóljatok a júdabeli véneknek, ezt mondván: Mi az oka, hogy utolsók akartok lenni a királynak hazahozásában? Mert az egész Izráelnek  szava eljutott a királynak házába.
\par 12 Én atyámfiai vagytok, én csontom és én testem vagytok; miért lesztek tehát utolsók a királynak hazahozásában?
\par 13 És Amasának azt mondjátok: Nemde te is nem én csontom és testem vagy-é? Úgy cselekedjék velem az Isten és úgy segéljen, ha nem leszel a sereg fõvezére minden idõben én elõttem Joáb  helyett.
\par 14 És megnyeré Júda minden emberének szívét, mint egy emberét, kik elküldének a királyhoz: Jõjj haza mind te, mind a te szolgáid.
\par 15 Hazatére azért a király; és a mint a Jordánhoz érkezék, Júda férfiai Gilgálba menének, hogy a királynak eleibe jussanak, és általvigyék a királyt a Jordánon.
\par 16 Siete pedig Sémei is, Gérának fia, a Bahurimból való Benjáminita, és õ is a Júda nemzetségével eleibe méne Dávid királynak.
\par 17 És ezer ember vala õ vele Benjámin nemzetségébõl, és Siba, a Saul házában való szolga, és tizenöt fia, és húsz szolgája is vele valának, és általmenének a Jordánon a király elõtt.
\par 18 Kompot is vivének által, hogy a király háznépét általhozzák, és hogy kedve szerint cselekedjenek. Sémei pedig Gérának fia, térdre esék a király elõtt, mikor a Jordánon általment.
\par 19 És monda a királynak: Ne tulajdonítsa vétkül nékem az én uram az én álnokságomat; és ne emlékezzél meg arról, hogy gonoszul cselekedék veled a te szolgád azon a napon, a melyen az én uram, a király Jeruzsálembõl kiméne, hogy szívére venné a király;
\par 20 Mert elismeri a te szolgád, hogy én vétkeztem, és ímé az egész József nemzetsége közül én jöttem ma elõször, hogy a királynak, az én uramnak eleibe menjek.
\par 21 És felele Abisai, Sárujának fia és monda: Nem ölöd-é meg azért Sémeit, hogy káromlásokat szólott az Úrnak felkentje ellen?
\par 22 Monda pedig Dávid: Mi közöm van veletek, Sérujának fiai, hogy ellenkezni akartok velem ma? Ma kellene-é embert ölni  Izráelben? Mintha nem tudnám, hogy ma lettem újonnan Izráel királya.
\par 23 És monda a király Sémeinek: Nem halsz meg; és megesküvék néki a király.
\par 24 Méfibóset is, Saulnak fia eleibe jöve a királynak, és sem lábait, sem szakálát meg nem tisztította, sem ruháját meg nem mosatta attól a naptól fogva, hogy a király elment vala, mindama napig, a melyen békességgel haza jöve.
\par 25 És lõn, hogy mikor Jeruzsálembõl kijött eleibe a királynak, monda néki a király: Miért nem jöttél volt el velem Méfibóset?
\par 26 És õ felele: Uram király, az én szolgám csalt meg engem; mert azt mondotta a te szolgád: Megnyergeltetem a szamarat, és felülök rá, és elmegyek a királylyal; mert sánta a te szolgád.
\par 27 És rágalmazott engem, a te szolgádat az én uram, a király elõtt; de az én uram, a király olyan,  mint az Istennek angyala: azért cselekedjél úgy, a mint néked tetszik!
\par 28 Mert jóllehet az én atyámnak egész háznépe csak halált érdemlett volna az én uramtól, a királytól, mégis a te szolgádat azok közé helyezéd, a kik asztalodnál esznek. Azért micsoda követelésem és micsoda kérésem volna még a király elõtt?
\par 29 Monda néki akkor a király: Mi szükség többet szólnod? Én megmondottam, hogy te és Siba osztozzatok meg a jószágon.
\par 30 És monda Méfibóset a királynak: Elveheti az egészet, csakhogy az én uram, a király békességben haza jöhete.
\par 31 A Gileádból való Barzillai is eljöve Rógelimból, és általméne a királylyal a Jordánon, kisérvén õt a Jordánon.
\par 32 Barzillai pedig igen vén ember vala, nyolczvan esztendõs, és õ táplálta vala a királyt, míg Mahanáimban lakott; mert igen gazdag ember vala.
\par 33 Monda pedig a király Barzillainak: Jere velem és eltartlak téged magamnál Jeruzsálemben.
\par 34 Barzillai pedig monda a királynak: Mennyi az én életem esztendeinek napja, hogy én felmehetnék a királylyal Jeruzsálembe?
\par 35 Nyolczvan esztendõs vagyok ma, avagy képes vagyok-é még különbséget tenni a jó és rossz között, vagy érzem-é, a te szolgád, ízét annak, a mit eszem és iszom, vagy gyönyörködhetem-é az éneklõ férfiak és asszonyok hangjaiban? Miért lenne terhére a te szolgád az én uramnak, a királynak?
\par 36 Egy kevés ideig óhajtana a Jordánon átmenni a királylyal a te szolgád; miért adna azért a király nékem ily nagy jutalmat?
\par 37 Hadd menjen vissza, kérlek, a te szolgád, és hadd haljak meg az én városomban, az én atyámnak és anyámnak temetõhelyében. Hanem inkább ímhol a te szolgád, Kimhám, menjen el õ a királylyal, az én urammal, és cselekedjél úgy vele, a mint néked jónak tetszik.
\par 38 És monda a király: Jõjjön el velem Kimhám és azt cselekszem vele, ami néked tetszik; és valamit tõlem kivánsz, megcselekeszem veled.
\par 39 Mikor pedig általkelt a Jordánon az egész nép, és a király is általment, megcsókolván a király Barzillait, megáldá õt, ki visszatére az õ helyére.
\par 40 Általméne pedig a király Gilgálba, és Kimhám õ vele megy vala és Júdának egész népe, a kik a királyt átszállították, sõt Izráel népének is fele.
\par 41 És ímé, Izráelnek minden férfiai eljövének a királyhoz és mondák a királynak: Miért loptak el téged a mi atyánkfiai, a Júda nemzetsége? És hozták által a királyt és az õ háznépét a Jordánon, és Dávidnak minden embereit vele egyetemben?
\par 42 Felelének pedig mindnyájan a Júda férfiai az Izráel férfiainak: Azért, mert a király közelebb áll hozzám; és miért neheztelsz ezért a dologért? Avagy megvendégelt-é azért a király? vagy megajándékozott-é valami ajándékkal?
\par 43 Felelvén pedig az Izráel férfiai a Júda férfiainak, mondának: Tízszeres részem van nékem a királyhoz, és Dávidra nézve is elsõbbségem van feletted. Miért tartottál azért semminek engem? Avagy nem én tettem-é legelõször szóvá, hogy hozzuk haza a mi királyunkat? De mindazáltal erõsebb lõn a Júda férfiainak szava, az Izráel férfiainak szavánál.

\chapter{20}

\par 1 Vala pedig ott történetesen egy istentelen ember, kinek neve Séba vala, Bikrinek fia, Benjámin nemzetségébõl való férfiú. Ez trombitát fuvata, és monda: Nincsen nékünk semmi közünk Dávidhoz, és semmi örökségünk nincsen az Isai fiában. Azért oh Izráel, oszoljatok el, kiki az õ sátorába.
\par 2 Eltávozék azért mind az Izráel népe Dávidtól Séba után, Bikri fia után. A Júda nemzetségébõl valók azonban mellette maradának az õ királyoknak, a Jordán vizétõl fogva mind Jeruzsálemig.
\par 3 Bemenvén pedig Dávid az õ házába Jeruzsálemben, elõhozatá a király azt a tíz ágyasát, a kiket otthon hagyott vala a ház õrzésére; és õrizet alá rekeszté és tápláltatá azokat; de hozzájok be nem méne. És õrizet alatt lõnek haláloknak napjáig, özvegységben élvén.
\par 4 Monda pedig a király Amasának: Gyûjtsd össze nékem a Júda nemzetségét három nap alatt, és magad is itt légy jelen.
\par 5 Elméne azért Amasa, hogy összegyûjtse Júdát, de több ideig késék, mint a hogy a király meghagyta vala néki.
\par 6 Monda akkor Dávid Abisainak: Ímé majd gonoszabbul cselekeszik velünk Séba, Bikrinek fia, Absolonnál: Vedd magad mellé a te uradnak szolgáit, és kergesd meg õt, nehogy erõs városokat szerezzen magának, és elmeneküljön elõlünk.
\par 7 Kimenének azért õ utána a Joáb emberei, és a Kereteusok  és Peleteusok és az erõsek mindnyájan, és kimenének Jeruzsálembõl, hogy üldözzék Sébát, Bikrinek fiát.
\par 8 Mikor pedig ama nagy kõsziklánál voltak, mely Gibeonnál van, Amasa jöve eleikbe; Joáb pedig ruhájába öltözvén, az õ ruháján felül derekához övedzett fegyvere a hüvelyében vala, mely mikor Joáb ment, leesék.
\par 9 És monda Joáb Amasának: Egészségben vagy-é atyámfia? És Joáb megfogá jobbkezével az Amasa szakálát, mintha meg akarná  csókolni.
\par 10 Amasa azonban nem õrizkedék a fegyvertõl, mely Joáb kezében vala, és általüté õt az ötödik oldalborda alatt, és kiontá a bélit a földre, és noha többször nem üté által, mindazáltal meghala; Joáb pedig és Abisai, az õ atyjafia, üldözék azután Sébát, a Bikri fiát.
\par 11 Megálla pedig az Amasa teste felett Joábnak egy szolgája, és monda: Valaki Joábbal tart és Dávidnak javát kívánja, Joáb után siessen.
\par 12 Amasa pedig fetreng vala a vérben az útnak közepén; látván pedig egy ember, hogy ott mindenki megálla, kivoná Amasát az útról a mezõre, és ruhát vete õ reá; mivel látta, hogy valaki csak arra ment, mind megállott.
\par 13 Minekutána pedig kivonták az útról, minden ember Joáb után siet vala, hogy üldözzék Sébát, Bikrinek fiát.
\par 14 És általméne Séba Izráelnek minden nemzetségein, Abelán és Béth-Maakán, és egész Béreán, kik egybegyûlvén, követik vala õt.
\par 15 És eljövén, körülfogták õt Abelában Béth-Maaka városában, és nagy töltést emelének a város ellen, mely a kõfallal egyenlõ volt; és az egész nép, mely Joábbal vala, rombolá, hogy ledöntse a kõfalat,
\par 16 Akkor kiálta egy eszes asszony a városból: Halljátok meg, halljátok meg! Mondjátok meg, kérlek, Joábnak: Jõjj ide hozzám, valamit mondok néked!
\par 17 Ki mikor hozzá ment, monda az asszony: Te vagy-é Joáb? Felele: Én vagyok. Ki monda néki: Hallgasd meg a te szolgálóleányodnak szavát. És felele: Meghallgatom.
\par 18 Akkor monda az asszony: Korábban azt szokták mondani: Kérdezzék meg Abelát, és úgy végeztek.
\par 19 Én Izráel békeszeretõ hívei közül való vagyok, te pedig el akarsz pusztítani egy várost és anyát Izráelben? Miért akarod elnyelni az Úrnak örökségét?
\par 20 Akkor felele Joáb, és monda: Távol legyen, távol legyen az én tõlem, hogy elnyeljem és elveszessem!
\par 21 Nem úgy van a dolog, hanem egy ember az Efraim hegységébõl való, a kinek neve Séba, a Bikri fia, felemelte kezét Dávid király ellen: adjátok kézbe azt egyedül, és elmegyek a város alól. És monda az asszony Joábnak: Ímé majd kivetjük a kõfalon a fejét néked.
\par 22 Elméne azért az asszony nagy eszesen az egész községhez, és elvágatá Sébának, a Bikri fiának fejét, és kiveték Joábnak. Akkor megfuvatá a trombitát, és hazaoszlának a város alól kiki az õ sátorába, Joáb pedig hazaméne Jeruzsálembe, a királyhoz.
\par 23 Joáb pedig Izráel egész serege felett való volt, Benája pedig, Jójadának fia, a Kereteusok és Peleteusok vezére vala.
\par 24 Adorám pedig adószedõ, és Jósafát, Ahiludnak fia, emlékíró vala.
\par 25 Séja íródeák, Sádók pedig és Abjátár papok valának.
\par 26 A Jairból való Ira is Dávidnak fõ embere vala.

\chapter{21}

\par 1 Lõn pedig nagy éhség Dávidnak idejében három egész esztendeig egymás után, és megkeresé e miatt Dávid az Urat, és monda az Úr: Saulért és az õ vérszopó háznépéért van ez: mivelhogy megölte a  Gibeonitákat.
\par 2 Hivatá azért a király a Gibeonitákat és szóla nékik: (A Gibeoniták pedig nem az Izráel fiai közül valók valának, hanem az Emoreusok maradékából, a kiknek az Izráel fiai megesküdtek vala; Saul mindazáltal alkalmat keresett vala, hogy azokat levágassa az Izráel és Júda nemzetsége iránti buzgalmából).
\par 3 És monda Dávid a Gibeonitáknak: Mit cselekedjem veletek, és mivel szerezzek engesztelést, hogy áldjátok az Úrnak örökségét?
\par 4 Felelének néki a Gibeoniták: Nincsen nékünk sem ezüstünk, sem aranyunk Saulnál és az õ házánál, és nem kell mi nékünk, hogy valaki megölettessék Izráelben. És monda: Valamit mondotok, megcselekszem veletek.
\par 5 Akkor mondának a királynak: Annak az embernek a házából, a ki megemésztett minket, és a ki ellenünk gonoszt gondola, hogy megsemmisíttessünk, és meg ne maradhassunk Izráel egész határában;
\par 6 Adj nékünk hét embert az õ maradékai közül, a kiket felakaszszunk az Úr elõtt, Saulnak, az Úr választottjának Gibeájában. És monda a király: Én átadom.
\par 7 És kedveze a király Méfibósetnek, Jonathán fiának, ki Saul fia vala, az Úr nevére tett esküvésért, mely közöttök tétetett, Dávid és Jonathán között, a Saul fia között.
\par 8 De elvevé a király Aja leányának, Rispának két fiát, kiket Saulnak szült vala, Armónit és Méfibósetet, és a Saul leányának, Mikálnak öt fiát, kiket szült Barzillai fiának,  Adrielnek, a Méholátból valónak.
\par 9 És adá azokat a Gibeoniták kezébe, a kik felakaszták õket a hegyen az Úr elõtt. Ezek tehát egyszerre heten pusztulának el, és az aratás elsõ napjaiban, az árpaaratás kezdetén ölettek meg.
\par 10 Võn pedig Rispa, Ajának leánya egy zsákruhát, és kiteríté azt magának a kõsziklán az aratás kezdetétõl fogva, míg esõ lõn reájok az égbõl: és nem engedé az égi madarakat azokra szállani nappal, sem pedig a mezei vadakat éjszaka.
\par 11 És megmondák Dávidnak, a mit Rispa, Ajának leánya, Saulnak ágyasa cselekedett.
\par 12 Akkor elméne Dávid, és elhozá a Saul és Jonathán tetemeit a a Jábes-Gileádbeliektõl, kik ellopták vala azokat a Bethsánnak utczájáról, a hol õket a Filiszteusok felakasztották vala, mikor a Filiszteusok megverték Sault a Gilboa hegyén.
\par 13 És elhozták onnét Saulnak és az õ fiának, Jonathánnak tetemeit, és összeszedték azoknak tetemeit is,  a kik felakasztattak.
\par 14 És eltemeték Saulnak és az õ fiának, Jonathánnak tetemeit a Benjámin földében, Sélában, az õ atyjának, Kisnek sírboltjában; és megtették mindazt, a mit a király parancsolt. És kiengesztelõdött ez által Isten az ország iránt.
\par 15 Ezután ismét háborút kezdének a Filiszteusok Izráel ellen; és elméne Dávid az õ szolgáival együtt, és harczolának a Filiszteusok ellen, annyira, hogy Dávid elfárada.
\par 16 Akkor Jisbi Bénób, ki az óriások maradékából való vala (kinek kopjavasa háromszáz rézsiklust nyomott, és új hadi szerszámmal volt felövezve), elhatározá magában, hogy megöli Dávidot;
\par 17 De Abisai, Sérujának fia segített néki és általüté a Filiszteust és megölé. Akkor esküvéssel fogadák néki a Dávid szolgái, ezt mondván: Soha többé velünk hadba nem jössz, hogy az Izráelnek szövétnekét el ne oltsad.
\par 18 Lõn azután is harczuk a Filiszteusokkal Gób városánál, Sibbékai Husát városból való, akkor megölé Sáfot, ki az óriások maradékai közül való vala.
\par 19 Azután újra lõn háború a Filiszteusokkal Góbnál, hol Elkhanán, a Bethlehembõl való Jaharé Oregimnek fia megölé a Gitteus Góliátot, kinek kopjanyele olyan vala, mint a szövõknek zugolyfája.
\par 20 Gáthban is volt háború, hol egy óriás férfi vala, kinek kezein és lábain hat-hat ujjai valának, azaz mindenestõl huszonnégy, és ez is óriástól származott vala.
\par 21 És szidalmazá Izráelt, de megölé Jonathán, Dávid bátyjának, Simeának fia.
\par 22 Ezek négyen származtak Gáthban az óriástól, kik mind Dávid keze által és az õ szolgáinak kezeik által estek el.

\chapter{22}

\par 1 Dávid pedig ezt az éneket mondotta az Úrnak azon a napon, mikor az Úr megszabadítá õt minden ellenségeinek kezébõl, és a Saul kezébõl.
\par 2 És monda: Az Úr az én kõsziklám és kõváram, és szabadítóm nékem.
\par 3 Az Isten az én erõsségem, õ benne bízom én. Paizsom nékem õ s idvességemnek szarva, erõsségem és oltalmam. Az én idvezítõm, ki megszabadítasz az erõszakosságtól.
\par 4 Az Úrhoz kiáltok, a ki dícséretreméltó; És megszabadulok ellenségeimtõl.
\par 5 Mert halál hullámai vettek engem körül, Az istentelenség árjai rettentettek engem;
\par 6 A pokol kötelei vettek körül, S a halál tõrei estek reám.
\par 7 Szükségemben az Urat hívtam, S az én Istenemhez kiáltottam: És meghallá lakóhelyérõl szavamat, S kiáltásom eljutott füleibe.
\par 8 Akkor rengett és remegett a föld, Az égnek fundamentumai inogtak, És megrendülének, mert haragudott Õ.
\par 9 Füst szállt fel orrából, És emésztõ tûz szájából, Izzószén gerjedt belõle.
\par 10 Lehajtá az eget és leszállt, És homály volt lábai alatt.
\par 11 A Khérubon ment és röpült, És a szelek szárnyain tünt fel.
\par 12 Sötétségbõl maga körül sátrakat emelt, Esõhullást, sürû felhõket.
\par 13 Az elõtte levõ fényességbõl Izzó szenek gerjedének.
\par 14 És dörgött az égbõl az Úr, És a Magasságos hangot adott.
\par 15 Ellövé nyilait és szétszórta azokat, Villámot, és összekeverte azokat.
\par 16 És meglátszottak a tenger örvényei, S a világ fundamentumai felszínre kerültek, Az Úrnak feddésétõl, Orra leheletének fúvásától.
\par 17 Lenyúlt a magasságból és felvett engem, S a mélységes vizekbõl kihúzott engem.
\par 18 Hatalmas ellenségimtõl megszabadított engem; Gyûlölõimtõl, kik hatalmasabbak valának nálam.
\par 19 Reámtörtek nyomorúságom napján, De az Úr gyámolóm volt nékem.
\par 20 Tágas helyre vitt ki engem, Kiragadott, mert jóakaróm nékem.
\par 21 Az Úr megfizetett nékem igazságom szerint, Kezeimnek tisztasága szerint fizetett meg nékem.
\par 22 Mert megõriztem az Úrnak utait, S gonoszul nem szakadtam el Istenemtõl.
\par 23 Mert ítéletei mind elõttem vannak, S rendeléseitõl nem távoztam el.
\par 24 Tökéletes voltam elõtte, s õrizkedtem rosszaságomtól.
\par 25 Ezért megfizet nékem az Úr igazságom szerint, Szemei elõtt való tisztaságom szerint.
\par 26 Az irgalmashoz irgalmas vagy, A tökéletes vitézhez tökéletes vagy.
\par 27 A tisztához tiszta vagy, A visszáshoz pedig visszás.
\par 28 Segítesz a nyomorult népen, Szemeiddel pedig megalázod a felfuvalkodottakat.
\par 29 Mert te vagy az én szövétnekem, Uram, S az Úr megvilágosítja az én sötétségemet.
\par 30 Mert veled harczi seregen is átfutok, Az én Istenemmel kõfalon is átugrom.
\par 31 Az Istennek útja tökéletes; Az Úrnak beszéde tiszta; Paizsa õ mindeneknek, a kik õ benne bíznak.
\par 32 Mert kicsoda volna Isten az Úron kivül? S kicsoda kõszikla a mi Istenünkön kivül?
\par 33 Isten az én erõs kõváram, Ki vezérli az igaznak útját.
\par 34 Lábait olyanná teszi, mint a szarvasé, S magas helyekre állít engem.
\par 35 Kezeimet harczra tanítja, Hogy az érczív karjaim által törik el.
\par 36 Idvességednek paizsát adtad nékem, S kegyelmed nagygyá tett engem.
\par 37 Lépéseimet kiszélesítetted alattam. És bokáim meg nem tántorodtak.
\par 38 Üldözöm ellenségeimet és elpusztítom õket, Nem térek vissza, míg meg nem semmisítem õket.
\par 39 Megsemmisítem, eltiprom õket, hogy fel nem kelhetnek, Lábaim alatt hullanak el.
\par 40 Mert te erõvel öveztél fel engem a harczra, Lenyomtad azokat, kik ellenem támadtak.
\par 41 Megadtad, hogy ellenségeim hátat fordítottak nékem, Gyülölõim, a kiket elpusztítottam én,
\par 42 Felnéztek, de nem volt, ki megszabadítsa, Az Úrhoz, de nem felelt nékik.
\par 43 Szétmorzsolom õket, mint a föld porát, Összezúzom, mint az utcza sarát, széttaposom õket.
\par 44 Megmentettél népemnek pártoskodásaitól, Népeknek fejéül tartasz fenn engemet, Oly nép szolgál nékem, melyet nem ismertem.
\par 45 Idegen fiak hizelkednek nékem, S egy hallásra engedelmeskednek,
\par 46 Idegen fiak elcsüggednek, S váraikból reszketve jõnek elõ.
\par 47 Él az Úr és áldott az én kõsziklám. Magasztaltassék az Isten, idvességem kõsziklája.
\par 48 Isten az, ki bosszút áll értem, S alám hajtja a népeket.
\par 49 Ki megment engem ellenségeimtõl, Te magasztalsz fel engem az ellenem támadók fölött, S az erõszakos embertõl megszabadítasz engem.
\par 50 Dícsérlek azért téged, Uram, a pogányok között, S nevednek dícséretet éneklek.
\par 51 Nagy segítséget ad az õ királyának, Irgalmasságot cselekszik felkentjével, Dáviddal és az õ magvával mindörökké!

\chapter{23}

\par 1 Ezek Dávidnak utolsó beszédei. Dávidnak, Isai fiának szózata, annak a férfiúnak szózata, a ki igen felmagasztaltaték, Jákób Istenének felkentje és Izráel dalainak kedvencze.
\par 2 Az Úrnak lelke szólott, én bennem, és az õ beszéde az én nyelvem által.
\par 3 Izráelnek Istene szólott, Izráelnek kõsziklája mondá nékem: A ki igazságosan uralkodik az emberek felett, a ki Isten félelmével uralkodik:
\par 4 Olyan az, mint a reggeli világosság, mikor a nap feljõ, mint a felhõtlen reggel; napsugártól, esõtõl sarjadzik a fû a földbõl.
\par 5 Avagy nem ilyen-é az én házam Isten elõtt? Mert örökkévaló szövetséget kötött velem, mindennel ellátva és állandót. Mert az én teljes idvességemet és minden kivánságomat  nem sarjadoztatja-é?
\par 6 De az istentelenek mindnyájan olyanok, mint a kitépett tövis, melyhez kézzel nem nyúlnak;
\par 7 Hanem a ki hozzá akar nyúlni is, fejszét és rudat vesz hozzá, hogy ugyanazon helyen tûzzel égettessék meg.
\par 8 Ezek pedig a Dávid erõs vitézeinek nevei: Joseb-Bassebet, a Tahkemonita, a ki a testõrök vezére; õ dárdáját forgatván, egy ízben nyolczszázat sebesített  meg.
\par 9 Õ utána volt Eleázár, Dódónak fia, ki Ahóhi fia vala; õ egyike a három hõsnek, a kik Dáviddal valának, mikor a Filiszteusok által kigúnyoltatának, összegyülekezvén ott a harczra, és az Izráeliták megfutamodtak volt.
\par 10 Õ megállván, vágta a Filiszteusokat mindaddig, míg a keze elfáradt, és a keze a fegyverhez ragadt. És nagy szabadulást szerze az Úr azon a napon, a nép pedig visszatére õ utána, de csak a fosztogatásra.
\par 11 Õ utána volt Samma, a Harárból való Agénak fia. És összegyûlének a Filiszteusok egy seregbe ott, a hol egy darab szántóföld volt tele lencsével, a nép pedig elfutott a Filiszteusok elõl:
\par 12 Akkor õ megálla annak a darab földnek közepén, és megoltalmazá azt, és megveré a Filiszteusokat, és az Úr nagy szabadítást szerze.
\par 13 A harmincz vezér közül is hárman lementek, és elérkezének aratáskor Dávidhoz az Adullám barlangjába, mikor a Filiszteusok táborban valának a Réfaim völgyében.
\par 14 Dávid akkor a sziklavárban volt, a Filiszteusok õrsége pedig Bethlehemnél.
\par 15 Vizet kivánt vala pedig Dávid, és monda: Kicsoda hozna nékem vizet innom a bethlehemi kútból, mely a kapu elõtt van?
\par 16 Akkor a három vitéz keresztül tört a Filiszteusok táborán, és merítének vizet a bethlehemi kútból, mely a kapu elõtt van, és elhozván, vivék Dávidnak. Õ azonban nem akará meginni, hanem kiönté azt az Úrnak.
\par 17 És monda: Távol legyen tõlem, Uram, hogy én ezt míveljem: avagy azoknak az embereknek vérét igyam-é meg, kik életöket halálra adva mentek el a vízért? És nem akará azt meginni. Ezt mívelte a három hõs.
\par 18 Továbbá Abisai, Joábnak atyjafia, Sérujának fia, a ki e háromnak feje volt, a ki háromszáz ellen felemelvén dárdáját, megölé azokat; és néki nagy híre vala a három  között.
\par 19 A három között bizonyára híres volt és azoknak vezére vala, mindazáltal ama hárommal nem ért fel.
\par 20 Benája is, Jójadának fia, vitéz ember, nagytehetségû, a ki  Kabséelbõl való vala; ez ölé meg a Moábitáknak két fõvitézét. Ugyanõ elmenvén, az oroszlánt is megölé a veremben, télen.
\par 21 Ugyanõ ölt meg egy Égyiptomból való tekintélyes embert. Az égyiptomi kezében dárda vala, és õ csak egy pálczával méne reá; és kivevé az égyiptomi kezébõl a dárdát, és megölé õt a maga dárdájával.
\par 22 Ezeket cselekedé Benája, Jójadának fia, és néki is jó híre vala a három erõs vitéz között.
\par 23 Híres volt õ a harmincz között, mindazáltal ama hárommal nem ért fel. És elõljáróvá tevé õt Dávid a tanácsosok között.
\par 24 Asáel is, Joáb atyjafia, e harmincz közül való, kik ezek: Elkhanán, a bethlehemi  Dódónak fia.
\par 25 Haród városbeli Samma; azon Haródból való Elika.
\par 26 Héles, Páltiból való; Híra, a Thékoából való Ikkes fia.
\par 27 Abiézer, Anathóthból; Mébunnai, Húsáthból való.
\par 28 Sálmon, Ahóhitból való; Maharai, Nétofátból.
\par 29 Héleb, Bahanának fia, Nétofátból való; Ittai, Ribainak fia, Gibeából való, mely Benjámin fiaié.
\par 30 Benája, Piráthonból való; Hiddai, a patak mellett való Gáhasbeli.
\par 31 Abiálbon, Árbátból való; Azmávet, Bárhumból való.
\par 32 Eljáhba, Sahalbomból való; Jásen fia, Jonathán.
\par 33 Samma, Harárból való; Ahiám, Arárból való Sarárnak fia.
\par 34 Elifélet, Ahásbainak fia, Maakátból való; Eliám, Gilóbeli  Akhitófelnek fia.
\par 35 Hesrai, Kármelbõl való; Paharai, Arbiból való.
\par 36 Jigeál, Sobabeli Nátánnak fia; Báni, Gádból való.
\par 37 Sélek, Ammonita; Naharai Beerótból való, Joábnak, a Séruja fiának fegyverhordozója.
\par 38 Ira, Jithribõl való; Gáreb, Jithribõl való.
\par 39 Hitteus Uriás. Mindössze harminczheten.

\chapter{24}

\par 1 Ismét felgerjede az Úrnak haragja Izráel ellen, és felingerlé Dávidot õ ellenek, ezt mondván: Eredj el, számláld meg Izráelt és Júdát.
\par 2 Monda azért Dávid Joábnak, az õ serege fõvezérének: Menj el, járd be Izráelnek minden nemzetségeit, Dántól fogva Beersebáig, és számláljátok meg a népet, hogy tudjam a népnek számát.
\par 3 És monda Joáb a királynak: Sokasítsa meg az Úr, a te Istened a népet, s adjon még százannyit, mint a mennyi most van, és lássák azt az én uramnak, a királynak szemei: de miért  akarja ezt az én uram, a király?
\par 4 Azonban hatalmasabb volt a királynak szava a Joábénál és a többi fõemberekénél. Kiméne azért Joáb és a többi vezérek a király elõl, hogy megszámlálják a népet, az Izráelt.
\par 5 És általkelének a Jordánon és tábort járának Aroer város mellett jobbkéz felõl, Gád völgyében, és Jáser mellett.
\par 6 Azután menének Gileádba, és a Hodsi alföldére. Innét menének a Dán Jáán mellé, és Sídon környékére.
\par 7 Ezután menének a Tírus erõs városához, és a Hivveusoknak és Kananeusoknak minden városaiba. Innét menének a Júdának dél felõl való részére, Beersebába.
\par 8 És mikor bejárták az egész országot, hazamenének Jeruzsálembe, kilencz hónap és húsz nap mulva.
\par 9 És beadá Joáb a megszámlált népnek számát a királynak; és Izráelben nyolczszázezer erõs fegyverfogható férfi, és Júda nemzetségében ötszázezer férfiú vala.
\par 10 Minekutána pedig Dávid a népet megszámlálta, megsebhedék az õ szíve, és monda Dávid az Úrnak: Igen vétkeztem abban, a mit cselekedtem. Azért most, óh Uram, vedd el, kérlek, a te szolgádnak álnokságát; mert felette esztelenül cselekedtem!
\par 11 És mikor felkelt reggel Dávid, szóla az Úr Gád prófétának, a ki Dávidnak látnoka vala, ezt mondván:
\par 12 Menj el, és szólj Dávidnak: Ezt mondja az Úr: Három dolgot adok elõdbe, válaszd egyiket magadnak ezek közül, hogy a szerint cselekedjem veled.
\par 13 Elméne azért Gád Dávidhoz, és tudtára adá néki, és monda néki: Akarod-é, hogy hét esztendeig való éhség szálljon földedre? Vagy hogy három hónapig ellenségeid elõtt bujdossál és ellenséged kergessen téged? Vagy hogy három napig döghalál legyen országodban? Most gondold meg és lásd meg, micsoda választ vigyek annak, a ki engem elküldött.
\par 14 És monda Dávid Gádnak: Felette igen szorongattatom; de mégis, hadd essünk inkább az Úr kezébe, mert nagy az õ irgalmassága, és ne essem ember kezébe.
\par 15 Bocsáta annakokáért az Úr döghalált Izráelre, reggeltõl fogva az elrendelt ideig, és meghalának a nép közül Dántól fogva  Beersebáig, hetvenezer férfiak.
\par 16 És mikor felemelte kezét az angyal Jeruzsálem ellen is, hogy azt is elpusztítsa, megelégelé az Úr a veszedelmet, és monda az angyalnak, a ki a népet öli vala: Elég immár, hagyd el. Az Úrnak angyala pedig vala a Jebuzeus  Arauna szérûje mellett.
\par 17 És szóla Dávid az Úrnak, mikor látta az angyalt, a ki a népet vágja vala, és monda: Ímé én vétkeztem és én cselekedtem hamisságot, de ezek a juhok ugyan mit cselekedtek? Kérlek inkább forduljon a te kezed ellenem és az én atyámnak háznépe ellen.
\par 18 Méne azért Gád Dávidhoz azon a napon, és monda néki: Eredj, menj el, és rakass oltárt az Úrnak a Jebuzeus Araunának szérûjén.
\par 19 És elméne Dávid Gádnak beszéde szerint, a mint az Úr angyala megparancsolta vala.
\par 20 És feltekintvén Arauna, látá, hogy a király az õ szolgáival õ hozzá megy; és Arauna kimenvén, meghajtá  magát a király elõtt, arczczal a föld felé.
\par 21 És monda Arauna: Mi az oka, hogy az én uram, a király az õ szolgájához jõ? Felele Dávid: Azért, hogy megvegyem tõled e szérût, és oltárt építsek ezen az Úrnak, hogy megszünjék a csapás a nép között.
\par 22 Monda Arauna Dávidnak: Vegye el hát az én uram, a király, és áldozza fel, a mi néki tetszik. Ímhol vannak az ökrök az áldozathoz; a boronák és az ökrök szerszámai pedig fa helyett;
\par 23 Arauna mindezt, óh király, a királynak adja. És monda Arauna a királynak: A te Urad Istened engeszteltessék meg általad.
\par 24 Monda pedig a király Araunának: Nem, hanem pénzen veszem meg tõled, mert nem akarok az Úrnak, az én Istenemnek ingyen való áldozatot áldozni. Megvevé azért Dávid azt a szérût és az ökröket ötven ezüst sikluson.
\par 25 És oltárt építe ott Dávid az Úrnak, és áldozék egészen égõ és hálaáldozattal;  és megkegyelmeze az Úr a földnek, és megszünék a csapás Izráelben.


\end{document}