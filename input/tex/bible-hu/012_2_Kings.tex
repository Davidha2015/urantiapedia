\begin{document}

\title{Királyok II. könyve}


\chapter{1}

\par 1 És elszakadt Moáb, Akháb halála után, az Izráeltõl.
\par 2 És mikor Akházia kibukott az õ felházának korlátján Samariában, és megbetegedék, követeket külde el, a kiknek azt parancsolá: Menjetek el, kérdjetek tanácsot a Baálzebubtól, az Ekron istenétõl, hogy meggyógyulok-é e betegségbõl?
\par 3 Az Úrnak angyala pedig szóla Thesbites Illésnek: Kelj fel, menj el eleibe a Samariabeli király követeinek, és szólj nékik: Nincs-é Isten Izráelben, hogy Baálzebubhoz, az Ekron istenéhez mentek tanácsot kérdeni?
\par 4 Azért azt mondja az Úr: Az ágyból, a melyben fekszel, fel nem kelsz, hanem kétség nélkül meghalsz. És elméne Illés.
\par 5 És mikor a követek visszajöttek õ hozzá, monda nékik: Miért jöttetek vissza?
\par 6 És felelének néki: Egy férfiú jöve elõnkbe, és monda nékünk: Menjetek el, térjetek vissza a királyhoz, a ki titeket elküldött, és mondjátok meg néki: Ezt mondja az Úr: Nincs-é Izráelben Isten, hogy te Baálzebubhoz, Ekron istenéhez küldesz tanácsot kérdeni? Azért az ágyból, a melyben fekszel, fel nem kelsz, hanem kétség nélkül meghalsz.
\par 7 És monda nékik: Milyen volt az a férfiú, a ki elõtökbe jött volt, és ezeket a beszédeket szólá néktek?
\par 8 És felelének néki: Egy szõr-ruhás ember, derekán bõr övvel felövezve. Akkor monda: Thesbites Illés volt.
\par 9 És hozzá külde egy ötven ember elõtt járó fõembert, az alatta való ötven emberrel. És mikor ez hozzá felment, ott ült fenn a hegy tetején, és monda néki: Isten embere, a király azt parancsolja: Jõjj le!
\par 10 Felelvén Illés, monda az ötven ember elõtt járó fõembernek: Ha én az Isten embere vagyok, szálljon tûz alá az égbõl, és emészszen meg téged és az alattad való ötven embert. És tûz szálla alá az égbõl, és megemészté õt és az õ ötven emberét.
\par 11 És ismét külde õ hozzá a király más ötven ember elõtt való fõembert, mind az ötven férfival egybe, a ki szóla és monda néki: Isten embere, a király azt parancsolja, hogy hamar jõjj alá!
\par 12 És felele Illés, és monda nékik: Ha Isten embere vagyok, szálljon tûz az égbõl alá, és emészszen meg téged és a te ötven emberedet. És Istennek tüze szálla alá az égbõl, és megemészté õt és az õ ötven emberét.
\par 13 És újra elküldött egy harmadik ötven ember elõtt járó fõembert is, az õ ötven emberével együtt, és felmenvén ez az ötven ember elõtt járó fõember, elméne és térdre borult Illés elõtt, és könyörögvén néki, monda: Óh Isten embere, kérlek, legyen becsülete a te szemeid elõtt az én lelkemnek, és ezeknek a te szolgáidnak, ez ötven ember lelkének!
\par 14 Ímé tûz szállott le az égbõl és megemészté az ötven ember elõtt való elébbi két fõembert, az õ ötven emberével egybe: Most azért legyen becsülete az én lelkemnek a te szemeid elõtt!
\par 15 Ekkor szóla az Úr angyala Illésnek: Menj alá vele, ne félj semmit tõle. És õ felkelvén aláméne vele a királyhoz.
\par 16 És szóla néki: Ezt mondja az Úr: A miért követeket küldöttél tanácsot kérdeni Baálzebubhoz, Ekron istenéhez, mintha nem volna Izráelben Isten, a kinek a beszédébõl tanácsot kérdhetnél: azért fel nem kelsz ez ágyból, a melyben fekszel, hanem kétség nélkül meghalsz.
\par 17 És meghalt az Úrnak beszéde szerint, a melyet szólott Illés, és uralkodék Jórám õ helyette, Jórámnak, a Josafát fiának, a Júdabeli királynak második esztendejében; mert Akháziának nem volt fia.
\par 18 Akháziának egyéb dolgai pedig, a melyeket cselekedett, vajjon nincsenek-é megírva az Izráel királyainak krónika-könyvében?

\chapter{2}

\par 1 És lõn mikor az Úr Illést fel akará vinni a szélvész által mennyekbe: elméne Illés és Elizeus Gilgálból.
\par 2 És monda Illés Elizeusnak: Maradj itt, kérlek, mert az Úr Béthelbe küldött engem. És felele Elizeus: Él az Úr, és a te lelked, hogy el nem hagylak téged! És mikor lemenének Béthel felé:
\par 3 Kijövének a próféták fiai, a kik Béthelben valának, Elizeushoz, és mondának néki: Nem tudod-é, hogy e mai napon az Úr elragadja a te uradat tõled? És monda: Tudom én is, hallgassatok.
\par 4 És monda néki Illés: Elizeus! Maradj itt, kérlek; mert az Úr engem Jérikhóba küldött. Õ azonban monda: Él az Úr és a te lelked, hogy el nem hagylak téged. És mikor elmenének Jérikhóba:
\par 5 A próféták fiai, a kik Jérikhóban valának, Elizeushoz jövének, és mondának néki: Nem tudod-é, hogy e mai napon az Úr a te uradat elragadja tõled? És õ monda: Én is tudom, hallgassatok.
\par 6 Azután monda néki Illés: Maradj itt, kérlek; mert az Úr engem a Jordán mellé küldött. De õ felele: Él az Úr és a te lelked, hogy el nem hagylak téged. És elmenének együtt mindketten.
\par 7 Ötven férfiú pedig a próféták fiai közül utánok menvén, velök szemben messze megállának, mikor õk ketten a Jordán mellett megállottak.
\par 8 És fogá Illés az õ palástját, és összehajtva azt, megüté azzal a vizet; és az kétfelé válék; úgy hogy mind a ketten szárazon menének át rajta.
\par 9 És mikor általmentek, monda Illés Elizeusnak: Kérj tõlem, mit cselekedjem veled, mielõtt tõled elragadtatom. És monda Elizeus: Legyen, kérlek, a te benned való léleknek kettõs mértéke én rajtam.
\par 10 És õ monda: Nehéz dolgot kértél; mégis, ha majd meglátándasz engem, mikor tõled elragadtatom, meglesz, a mit kérsz: ha pedig meg nem látándasz, nem lesz meg.
\par 11 És lõn, a mikor menének és menvén beszélgetének, ímé egy tüzes szekér tüzes lovakkal elválasztá õket egymástól. És felméne Illés a szélvészben az égbe.
\par 12 Elizeus pedig ezt látván, kiált vala: Édes atyám, édes atyám! Izráel szekerei és lovagjai! És nem látá õt többé. És vevé a maga ruháit, és két részre szakasztá azokat,
\par 13 És felemelé az Illés palástját, a mely róla leesett, és visszatért, és megállott a Jordán partján.
\par 14 És vevé az Illés palástját, a mely róla leesett, és azzal megüté a vizet, és monda: Hol van az Úr, az Illés Istene? És mikor õ is megütötte a vizet, kétfelé válék az; és általméne Elizeus.
\par 15 És mikor látták õt a próféták fiai, a kik átellenben Jérikhónál valának, mondának: Az Illés lelke megnyugodt Elizeuson. És eleibe menének néki, és meghajták magokat õ elõtte a földig;
\par 16 És mondának néki: Ímé a te szolgáid között van ötven ember, erõs férfiak, küldd el õket, hadd keressék meg a te uradat, hátha az Úrnak lelke ragadta el õt, és letette õt valamelyik hegyen, vagy völgyben. De õ monda: Ne küldjetek.
\par 17 De azok kényszerítették õt egész a megszégyenülésig, és monda: Hát küldjetek el! És elküldék az ötven férfiút, de harmadnapig keresvén sem találák meg õt.
\par 18 És mikor visszajöttek õ hozzá, mert Jérikhóban lakott, monda nékik: Nem mondottam-é, hogy ne menjetek el?
\par 19 És mondának a város férfiai Elizeusnak: Ímé e város jó lakóhely volna, a mint uram magad látod; de a vize ártalmas, és a föld gyümölcsét meg nem érleli.
\par 20 És monda: Hozzatok nékem egy új csészét, és tegyetek sót belé. És elhozák néki.
\par 21 Õ pedig kiment a forráshoz, és bele veté a sót, és monda: Ezt mondja az Úr: Meggyógyítottam e vizeket; nem származik ezután azokból halál és idétlen termés.
\par 22 És egészségesekké lõnek a vizek mind e mai napig, Elizeus beszéde szerint, a melyet szólott.
\par 23 Felméne azután onnét Béthelbe; és mikor az úton felfelé méne, apró gyermekek jövének ki a városból, a kik õt csúfolják vala, ezt mondván: Jõjj fel, kopasz, jõjj fel, kopasz!
\par 24 És hátratekintvén és meglátván õket, megátkozá õket az Úr nevében, és az erdõbõl két nõstény medve jövén ki, szétszaggata közülök negyvenkét gyermeket.
\par 25 Onnét azután felment a Kármel hegyére; onnét pedig Samariába tért vissza.

\chapter{3}

\par 1 Jórám pedig, az Akháb  fia kezde uralkodni Izráelen Samariában Josafátnak, a Júda királyának tizennyolczadik esztendejében, és uralkodék tizenkét esztendeig.
\par 2 És gonoszul cselekedék az Úrnak szemei elõtt, de nem annyira, mint az õ atyja és anyja; mert elrontá a Baál képét, a melyet az õ atyja készíttetett volt.
\par 3 De mindazáltal követé Jeroboámnak, a Nébát fiának bûneit, a ki vétekbe ejtette volt az Izráelt, és el nem távozék azoktól.
\par 4 És Mésának, a Moáb királyának nagyon sok juha volt, és az Izráel királyának adóban százezer bárány és százezer kos gyapját fizette.
\par 5 De mikor meghalt Akháb, a Moáb királya elszakadt az Izráel királyától.
\par 6 Kiméne azért Jórám király azon a napon Samariából, és megszámlálá az egész Izráelt;
\par 7 És elméne, és külde követeket Josafáthoz, Júda királyához, ezt izenvén néki: A Moáb királya elszakadt tõlem; eljösz-é velem a Moáb ellen a hadba? Felele: Felmegyek, úgy én, mint te; úgy az én népem, mint a te néped; úgy az én lovam, mint a te lovaid.
\par 8 És monda: Mely úton menjünk fel? Felele: Az Edom pusztájának útján.
\par 9 És elméne az Izráel királya és Júda királya és az Edom királya; és mikor hét napig bolyongtak az úton, nem volt vize sem a tábornak, sem a velök volt barmoknak.
\par 10 Akkor monda az Izráel királya; Jaj, jaj; az Úr azért hívta egybe e három királyt, hogy a Moáb kezébe adja õket!
\par 11 És monda Josafát: Nincs itt az Úr prófétái közül egy sem, hogy általa tanácsot kérhetnénk az Úrtól? És felele egy az Izráel királyának szolgái közül, és monda: Itt van Elizeus, a Sáfát fia, a ki Illésnek kezeire vizet tölt vala.
\par 12 És monda Josafát: Nála van az Úrnak beszéde. És alámenének hozzá az Izráel királya és Josafát és az Edom királya.
\par 13 És monda Elizeus az Izráel királyának: Mi közöm van hozzád? Menj a te atyádnak és anyádnak prófétáihoz! És monda néki az Izráel királya: Ne utasíts el; mert az Úr gyûjtötte össze ezt a három királyt, hogy a Moáb kezébe adja õket.
\par 14 És monda Elizeus: Él a Seregek Ura, a ki elõtt állok, ha nem nézném Josafátnak, a Júda királyának személyét, bizony téged nem néznélek, és rád sem tekintenék;
\par 15 Hozzatok ide egy énekest. És mikor énekelt elõtte az éneklõ, az Úrnak keze lõn õ rajta.
\par 16 És monda: Ezt mondja az Úr: Csináljatok itt és ott e patakon árkokat;
\par 17 Mert ezt mondja az Úr: Nem láttok sem szelet, sem esõt, és mégis e patak megtelik vízzel, hogy ihattok mind ti, mind a ti nyájatok és barmaitok.
\par 18 Kevés pedig még ez az Úr szemei elõtt, hanem a Moábot is kezetekbe adja.
\par 19 És megvesztek minden kulcsos várost és minden szép várost, és minden jótermõ fát kivagdaltok, és minden kútfõt betöltötök, és minden jó szántóföldet behánytok kövekkel.
\par 20 És lõn reggel, a mikor áldozatot szoktak tenni, ímé vizek jõnek vala Edom útjáról, és megtelék a föld vízzel.
\par 21 Mikor pedig meghallotta az egész Moáb, hogy feljöttek a királyok õ ellenök harczolni, egybegyûlének mindnyájan, a kik fegyvert foghattak, és megállottak az õ tartományuk határán.
\par 22 És mikor felkeltek reggel, és a nap feljött a vizekre, úgy láták a Moábiták, mintha õ ellenökbe az a víz vereslenék, mint a vér.
\par 23 És mondának: Vér ez! Megvívtak a királyok egymással, és megölte egyik a másikát. Prédára most, Moáb!
\par 24 De mikor az Izráel táborához jutottak, felkeltek az Izráeliták, és megverték a Moábitákat, és azok megfutamodtak elõttök; betörtek hozzájok és leverték Moábot.
\par 25 És városaikat lerontották, és a jó szántóföldekre kiki követ hányt, és elborították azt kövekkel, és minden kútfõt behánytak, és minden jótermõ fát kivágtak, úgy hogy csak Kir-Haréset kõfalait hagyták fenn; de körülvevék azt is a parittyások, és lerontották.
\par 26 Látván pedig a Moáb királya, hogy legyõzettetik a viadalban, maga mellé vett hétszáz fegyverfogó férfiút, hogy keresztül törjenek az Edom királyához; de nem bírtak.
\par 27 Akkor vevé az õ elsõszülött fiát, a ki õ helyette uralkodandó volt, és égõáldozatul megáldozá a kõfalon. Mely dolog felett az Izráel népe igen felháborodék, és elmenének onnét, és megtérének az õ földjökbe.

\chapter{4}

\par 1 És kiálta egy asszony a próféták fiainak feleségei közül Elizeushoz, és monda: A te szolgád, az én férjem meghalt; te tudod, hogy a te szolgád félte az Urat. Eljött pedig a hitelezõ, hogy elvigye mind a két gyermekemet, hogy néki szolgái legyenek.
\par 2 Monda néki Elizeus: Mit cselekedjem veled? Mondd meg nékem, mi van a házadban? Monda az: A te szolgálóleányod házában nincs egyéb, csak egy korsó olaj.
\par 3 Akkor monda: Menj el, kérj ott kinn minden te szomszédidtól üres edényeket, de ne keveset;
\par 4 És menj be és zárkózzál be magad a te fiaiddal, és tölts az olajból mindenik edénybe, és a tele edényt állítsd félre.
\par 5 És elment õ tõle, és bezárkózott az õ fiaival, azok hordták néki az edényeket, õ maga pedig csak töltögetett.
\par 6 És mikor megtölötte az edényeket, monda az õ fiának: Hozz ide még egy edényt. Felele az: Nincs több edény; és akkor megállott az olaj.
\par 7 És elment, és elmondá ezt az Isten emberének. Az pedig monda: Menj el, add el az olajat, és fizesd ki hitelezõdet; te pedig és a te fiaid éljetek a maradékából.
\par 8 És történt ebben az idõben, hogy Elizeus Súnembe ment. Ott volt egy gazdag asszony, aki tartóztatá õt, hogy nála egyék kenyeret. És lõn, hogy valamikor csak arra járt, betért oda, hogy kenyeret egyék.
\par 9 És monda az asszony a férjének: Ímé úgy veszem észre, hogy az az Isten embere, a ki szüntelen erre jár által, szent ember;
\par 10 Csináljunk, kérlek, egy kicsiny felházat, és tegyünk abba néki egy ágyat, asztalt, széket és gyertyatartót, hogy mikor hozzánk jön, hadd térjen oda.
\par 11 És történt egy napon, hogy oda ment Elizeus, és megszállott a felházban, és megpihent ott.
\par 12 És monda Géházinak, az õ szolgájának: Hívd ide azt a Súnemitát. Elõhívá azért azt, és eleibe álla.
\par 13 Megmondotta volt pedig néki: Mondd meg néki: Ímé nagy szorgalmatossággal szolgálsz nékünk, mit kivánsz, hogy cselekedjem veled? Nincs-é valami mondani valód a király elõtt, vagy a sereg fõvezére elõtt? És monda az: Én az én nemzetségem között békességgel lakom.
\par 14 Monda Elizeus: Mit lehetne tehát érette tennünk? Felele Géházi: Nincs fia és a férje vén ember.
\par 15 És monda: Hívd ide! És a mikor oda hívta, megállott az ajtóban.
\par 16 És monda Elizeus: Esztendõ ilyenkorra fiút fogsz ölelni. És monda az: Ne, édes uram, Isten embere, ne mondj képtelen dolgot a te szolgálóleányodnak!
\par 17 És teherbe esék az asszony és fiat szült abban az idõben, a melyet megmondott volt Elizeus.
\par 18 De mikor megnõtt a gyermek, történt, hogy egy napon kiment az õ atyjához, az aratókhoz.
\par 19 És monda az õ atyjának: Jaj fejem, jaj fejem! És monda az õ atyja a szolgának: Vidd el õt az anyjához.
\par 20 Ki mikor felvette õt, vivé az õ anyjához, és az az ölében tartá délig, és akkor meghalt.
\par 21 És felméne az asszony, és az Isten emberének ágyára tevé õt, és az ajtót bezárván kijöve onnét.
\par 22 És elõhívatá az õ férjét és monda: Kérlek, küldj ide nékem egyet a szolgák közül és egy szamarat, hadd menjek el hamar az Isten emberéhez, és mindjárt megjövök.
\par 23 És az monda: Miért mégy õ hozzá, ma nincs sem újhold, sem szombat? Felele az: Csak hagyd rám!
\par 24 És megnyergelé a szamarat, és monda a szolgának: Hajtsd és siess, ne késlelj engem a menésben, hanem ha mondándom néked.
\par 25 És elmenvén, juta az Isten emberéhez a Kármel hegyére. És mikor meglátá õt az Isten embere távolról, monda Géházinak, az õ szolgájának: Ímé a Súnemita ez!
\par 26 Fuss, kérlek, eleibe, és kérdezd meg tõle, ha békességben van-é mind õ, mind az õ férje, mind az õ gyermeke? Monda az: Békességben!
\par 27 Mikor pedig az Isten emberéhez jutott a hegyre, átölelte az õ lábait; de Géházi oda ment, hogy elûzze õt, az Isten embere azonban így szólt: Hagyj békét néki, mert megkeseredett az õ szíve, és az Úr eltitkolta tõlem, és nem jelentette meg nékem.
\par 28 És monda az: Vajjon én kértem-é fiat az én uramtól? Nemde nem mondám-é néked: Ne csalj meg engemet?
\par 29 És monda Elizeus Géházinak: Övezd fel derekadat, és vedd kezedbe az én pálczámat, és menj el, ha valakivel találkozol, ne köszönj néki, és annak is, a ki köszön néked, ne felelj, és az én pálczámat tedd a gyermek arczára.
\par 30 De a gyermeknek anyja monda: Él az Úr és él a te lelked, hogy el nem hagylak téged. Felkele azért és követé õt.
\par 31 Géházi pedig már elõttök elment volt, és a pálczát a gyermek arczára fekteté, de nem szólott és nem is eszmélt rá a gyermek. Azért visszatért eleibe, és megmondá néki, mondván: Nem támadt fel a gyermek.
\par 32 És bement Elizeus a házba, és ímé a gyermek ott feküdt halva az õ ágyán.
\par 33 És bement, és bezárta az ajtót magára és a gyermekre, és könyörgött az Úrnak.
\par 34 És az ágyra felhágván, a gyermekre feküdt, és az õ száját a gyermek szájára tevé, szemeit szemeire, kezeit kezeire, és ráborult, és megmelegedék a gyermek teste.
\par 35 Azután felállott, és egyszer alá és fel járt a házban, majd újra felment és reáborult. Akkor a gyermek prüsszente vagy hétszer, és felnyitá szemeit a gyermek.
\par 36 Õ pedig szólítá Géházit, és monda: Hívd ide a Súnemitát. És oda hívá azt. És mikor oda ment, monda: Vedd a te fiadat.
\par 37 Ki mikor bement, lábához esék, és leborula a földre, és az õ fiát fogván, kiméne.
\par 38 Azután visszament Elizeus Gilgálba. Éhség vala pedig akkor az országban, és a próféták fiai õ vele laknak vala. És monda az õ szolgájának: Tedd fel a nagy fazekat, és fõzz valami fõzeléket a próféták fiainak.
\par 39 Kiméne azért egy a mezõre, hogy paréjt szedjen. És holmi vad indákra találván, tele szedé az õ ruháját azokról sártökkel, és mikor hazament, belevagdalta a fazékba fõzeléknek; de nem tudta, hogy mi az?
\par 40 Mikor azután feladták a férfiaknak, hogy egyenek, és õk enni kezdének a fõzelékbõl, felkiáltának és mondának: Halál van a fazékban, Isten embere! És nem bírták megenni.
\par 41 Õ pedig monda: Hozzatok lisztet. És beleveté azt a fazékba, és monda: Add fel immár a népnek, hadd egyenek. És nem volt már semmi rossz a fazékban.
\par 42 Jöve pedig egy férfi Baál Sálisából és hoz vala az Isten emberének elsõ zsengék kenyereit, húsz árpakenyeret, és megzsendült gabonafejeket az õ ruhájában; de õ monda: Add a népnek, hadd egyenek.
\par 43 Felele az õ szolgája: Minek adjam ezt száz embernek? Õ pedig monda ismét: Add a népnek, hadd egyenek, mert ezt mondja az Úr: Esznek és még marad is.
\par 44 És õ eleikbe adá, és evének, és még maradt is belõle, az Úrnak beszéde szerint.

\chapter{5}

\par 1 És Naámán, a siriai király seregének fõvezére, az õ ura elõtt igen nagy férfiú és nagyrabecsült volt, mert általa szabadította volt meg az Úr Siriát; és az a férfi vitéz hõs, de bélpoklos volt.
\par 2 Egyszer portyázó csapatok mentek ki Siriából, és azok Izráel országából egy kis leányt vittek el foglyul, és ez Naámán feleségének szolgált.
\par 3 És monda ez az õ asszonyának: Vajha az én uram szembe lenne azzal a prófétával, a ki Samariában van, kétség nélkül meggyógyítaná õt az õ bélpoklosságából.
\par 4 És Naámán beméne, és elbeszélé az õ urának, mondván: Így s így szólott az Izráel országából való leány!
\par 5 Akkor monda Siria királya: Menj el, és ím levelet küldök az Izráel királyának. Elméne azért és võn magával tíz tálentom ezüstöt és hatezer aranyat, azon felül tíz öltözõ ruhát.
\par 6 És elvivé a levelet az Izráel királyának, ezt írván: Mikor e levél hozzád érkezik, ímé az én szolgámat, Naámánt azért küldöttem hozzád, hogy õt gyógyítsd meg bélpoklosságából.
\par 7 De a mikor elolvasta az Izráel királya a levelet, megszaggatá az õ ruháit és monda: Isten vagyok-e én, hogy öljek és elevenítsek, hogy ez én hozzám küld, hogy gyógyítsam meg e férfiút az õ bélpoklosságából? Vegyétek eszetekbe és lássátok,  hogy csak okot keres ellenem.
\par 8 Mikor pedig meghallotta Elizeus, az Isten embere, hogy az Izráel királya ruháit megszaggatta, külde a királyhoz ilyen izenettel: Miért szaggattad meg a te ruháidat? Hadd jõjjön hozzám, és tudja meg, hogy van próféta Izráelben.
\par 9 És elméne Naámán lovaival és szekereivel, és megálla az Elizeus házának ajtaja elõtt.
\par 10 És külde Elizeus követet õ hozzá, mondván: Menj el és fürödj meg hétszer a Jordánban, és megújul a te tested, és megtisztulsz.
\par 11 Akkor megharaguvék Naámán és elment, és így szólt: Íme én azt gondoltam, hogy kijõ hozzám, és elõállván, segítségül hívja az Úrnak, az õ Istenének nevét, és kezével megilleti a beteg helyeket, és úgy gyógyítja meg a kiütést.
\par 12 Avagy nem jobbak-é Abana és Párpár, Damaskus folyóvizei Izráel minden vizeinél? Avagy nem fürödhetném-é meg azokban, hogy megtisztuljak? Ilyen módon megfordulván, nagy haraggal elment.
\par 13 De hozzá menének az õ szolgái, és szólának néki, mondván: Atyám, ha valami nagy dolgot mondott volna e próféta neked, avagy nem tetted volna-é meg? Mennyivel inkább, a mikor csak azt mondja, hogy fürödj meg és megtisztulsz?
\par 14 Beméne azért a Jordánba, és belemeríté magát abba hétszer az Isten emberének beszéde szerint, és megújult az õ teste, mint egy kis gyermek teste, és megtisztult.
\par 15 Azután visszatért egész kiséretével az Isten emberéhez, és bemenvén megálla elõtte, és monda: Ímé, most tudom már, hogy nincsen az egész földön Isten, csak Izráelben! Azért most vedd el, kérlek, ez ajándékot a te szolgádtól.
\par 16 Õ pedig monda: Él az Úr, a ki elõtt állok, hogy el nem veszem. Kényszeríti vala pedig õt, hogy elvegye; de õ nem akará.
\par 17 És monda Naámán: Ha nem; adj kérlek a te szolgádnak e földbõl annyit, a mennyit elbír két öszvér; mert a te szolgád többé égõáldozattal, vagy egyéb áldozattal nem áldozik idegen isteneknek, hanem csak az Úrnak.
\par 18 Ebben a dologban legyen az Úr kegyelmes a te szolgádnak, hogy mikor bemegy az én uram a Rimmon templomába, hogy ott imádkozzék, és õ az én kezemre támaszkodik, ha akkor én is meghajlok a Rimmon templomában: azt, hogy én meghajlok a Rimmon templomában, bocsássa meg az Úr a te szolgádnak ebben a dologban.
\par 19 És monda néki Elizeus: Eredj el békességgel. És mikor elment õ tõle, úgy egy mértföldnyire,
\par 20 Géházi, Elizeusnak, az Isten emberének szolgája azt gondolta: Ímé az én uram megkimélé ezt a Siriabeli Naámánt, és nem akará tõle elvenni, a mit hozott volt; él az Úr, hogy utána futok, és valamit kérek tõle.
\par 21 És utána futott Géházi Naámánnak. Látván pedig Naámán õt, hogy utána fut, leugrott a szekérbõl és eleibe méne és monda: Rendben van minden?
\par 22 És monda: Rendben. Az én uram küldött engem, ezt mondván: Ímé most csak ez órában jött hozzám két ifjú az Efraim hegyérõl a próféták fiai közül: adj kérlek azoknak egy tálentom ezüstöt és két öltözõ ruhát.
\par 23 És monda Naámán: Kérlek végy két tálentomot. És kényszeríté õt és egybeköte két tálentom ezüstöt két zsákba, és két öltözõ ruhát, és azokat két szolgájának adá, a kik elõtte vitték azokat.
\par 24 De mikor a dombhoz ért, elvette tõlök azokat, és elrejté egy házban, és elbocsátá a férfiakat, és elmenének.
\par 25 Õ pedig bemenvén, megálla az õ ura elõtt, és monda néki Elizeus: Honnét, Géházi? Felele: Nem ment a te szolgád sehová.
\par 26 Õ pedig monda néki: Nem ment-é el az én szívem veled, mikor az a férfiú leszállott szekerébõl elõdbe? Most az ideje, hogy szerezz ezüstöt, és hogy végy ruhákat, olajfákat, szõlõket, juhokat, barmokat, szolgákat és szolgálóleányokat?!
\par 27 Azért rád és a te magodra ragad a Naámán bélpoklossága mindörökké. És kiment õ elõle megpoklosodva, mint a hó.

\chapter{6}

\par 1 És mondának a próféták fiai Elizeusnak: Ímé ez a hely, a hol nálad lakunk, igen szoros nékünk;
\par 2 Hadd menjünk el, kérlek, a Jordán mellé, hogy mindenikünk egy-egy fát hozzon onnét, hogy ott valami hajlékot építsünk magunknak, a melyben lakjunk. És monda: Menjetek el!
\par 3 És monda egy közülök: Nyugodj meg rajta, és jõjj el a te szolgáiddal. És monda: Én is elmegyek.
\par 4 És elméne velök. És menének a Jordán mellé, és ott fákat vágtak.
\par 5 És történt, hogy mikor egy közülök egy fát levágna, a fejsze beesék a vízbe. Akkor kiálta és monda: Jaj, jaj, édes uram! pedig ezt is kölcsön kértem!
\par 6 És monda az Isten embere: Hová esett? És mikor megmutatta néki a helyet, levágott egy fát és utána dobta, és a fejsze feljött a víz színére.
\par 7 És monda: Vedd ki. És kinyújtván kezét, kivevé azt.
\par 8 Siria királya pedig hadat indított Izráel ellen, és tanácsot tartván az õ szolgáival, monda: Itt meg itt lesz az én táborom.
\par 9 És elkülde az Isten embere az Izráel királyához, mondván: Vigyázz, ne hagyd el azt a helyet, mert ott akarnak a siriaiak betörni.
\par 10 És elkülde az Izráel királya arra a helyre, a melyrõl néki az Isten embere szólott, és õt megintette volt, és vigyázott magára nem egyszer, sem kétszer.
\par 11 És felháborodott ezen a siriai király szíve, és összegyûjtvén az õ szolgáit, monda nékik: Miért nem mondjátok meg nékem, ki tart közülünk az Izráel királyával?!
\par 12 Akkor monda egy az õ szolgái közül: Nem úgy, uram király, hanem Elizeus próféta, a ki Izráelben van, jelenti meg az Izráel királyának a beszédeket, a melyeket te a te titkos házadban beszélsz.
\par 13 És monda: Menjetek el és nézzétek meg, hol van, hogy utána küldjek és elhozassam õt. És megjelenték néki, mondván: Ímé Dótánban van.
\par 14 Akkor lovakat, szekereket és nagy sereget külde oda, a kik elmenének éjjel, és körülvevék a várost.
\par 15 Felkelvén pedig jókor reggel az Isten emberének szolgája, kiméne, és ímé seregek vették körül a várost, és lovak és szekerek. És monda néki az õ szolgája: Jaj, jaj, édes uram! mit cselekedjünk?
\par 16 Felele õ: Ne félj. Mert többen vannak, a kik velünk vannak, mint a kik õ velök.
\par 17 És imádkozott Elizeus és monda: Óh Uram! nyisd meg kérlek az õ szemeit, hadd lásson. És megnyitá az Úr a szolga szemeit és láta, és ímé a hegy rakva volt tüzes lovagokkal és szekerekkel Elizeus körül.
\par 18 És mikor azok hozzá lementek, könyörgött Elizeus az Úrnak, mondván: Verd meg ezt a népet vaksággal! És megveré õket vaksággal az Elizeus kivánsága szerint.
\par 19 És monda nékik Elizeus: Nem ez az út, sem ez a város; jertek el én utánam, és ahhoz a férfiúhoz vezetlek titeket, a kit kerestek. És elvezeté õket Samariába.
\par 20 És mikor bementek Samariába, monda Elizeus: Óh Uram, nyisd meg ezek szemeit, hogy lássanak. És megnyitá az Úr az õ szemeiket és látának, és ímé Samaria közepében voltak.
\par 21 Az Izráel királya pedig mikor látta õket, monda Elizeusnak: Vágván vágassam-é õket, atyám?
\par 22 És monda: Ne vágasd. Le szoktad-é vágatni azokat, a kiket karddal vagy kézívvel fogsz el? Adj nékik kenyeret és vizet, hogy egyenek és igyanak, és elmenjenek az õ urokhoz.
\par 23 És nagy lakomát szerzett nékik; és miután ettek és ittak, elbocsátá õket. Õk pedig elmenének az õ urokhoz; és ettõl fogva többé nem jöttek a siriai portyázó csapatok az Izráel földjére.
\par 24 És lõn ezek után, hogy Benhadád, Siria királya összegyûjté egész seregét, és felment és megszállotta Samariát.
\par 25 És igen nagy inség lett Samariában, mert addig tartották megszállva a várost, míg egy szamárfej nyolczvan ezüst, és egy véka galambganéj öt ezüst lett.
\par 26 És mikor az Izráel királya a kõfalon széjjeljára, egy asszony kiálta õ hozzá, mondván: Légy segítséggel, uram király!
\par 27 A király monda: Ha nem segít meg téged az Isten, hogyan segítselek én meg? A szérûrõl vagy a sajtóról?
\par 28 És monda néki a király: Mit akarsz? Monda az: Ez az asszony azt mondta nékem: Add ide a te fiadat, hogy együk meg õt ma, az én fiamat pedig holnap eszszük meg.
\par 29 És megfõztük az én fiamat, és megettük õt. Mikor azután másnap azt mondtam néki: Add ide a te fiadat, hogy azt is együk meg, õ elrejté az õ fiát.
\par 30 Mikor pedig hallotta a király az asszonynak beszédét, megszaggatá az õ ruháit, a mint a kõfalon járt, és meglátta a nép, hogy ímé alól zsákruha van az õ testén.
\par 31 És monda: Úgy cselekedjék velem az Isten és úgy segéljen, ha Elizeusnak, a Sáfát fiának feje ma rajta marad!
\par 32 Elizeus pedig ott ült az õ házában, és együtt ültek vele a vének. És elküldött a király egy férfiat maga elõtt. Mielõtt azonban hozzá jutott volna a követ, monda Elizeus a véneknek: Látjátok-é, hogy az a gyilkos hogyan küld ide, hogy a fejemet vétesse? Vigyázzatok, hogy mikor ide ér a követ, zárjátok be az ajtót és szorítsátok meg õt az ajtóban: Ímé az õ ura  lábainak dobogása követi õt.
\par 33 És mikor még így beszélne velök, már a követ leérkezett hozzá és nyomában a király, és monda: Ímé ilyen veszedelem származott az Úrtól; várjak-é még tovább az Úrra?

\chapter{7}

\par 1 És monda Elizeus: Halljátok meg az Úr beszédét. Ezt mondja az Úr: Holnap ilyenkor egy köböl zsemlyelisztet egy sikluson, és két köböl árpát egy sikluson vesznek Samaria kapujában.
\par 2 És felelvén egy fõember, a kinek kezére támaszkodott a király, az Isten emberének, monda: Hacsak az Úr ablakokat nem csinál az égen; akkor meglehet? És monda Elizeus: Ímé, te szemeiddel meg fogod látni, de nem eszel belõle.
\par 3 A kapu elõtt pedig volt négy bélpoklos férfi, a kik mondák egymásnak: Miért maradunk itt, hogy meghaljunk éhen?
\par 4 Ha azt határozzuk is, hogy bemegyünk a városba, ott is inség van, és akkor ott halunk meg; ha pedig itt maradunk, akkor itt halunk meg; jertek el azért, szökjünk el a Siriabeliek táborába, ha meghagyják életünket, élünk, ha megölnek, meghalunk.
\par 5 És felkeltek alkonyatkor, hogy a Siriabeliek táborába menjenek; és mikor odaértek a Siriabeliek táborának széléhez, ímé már nem volt ott senki.
\par 6 Mert az Úr azt cselekedte volt, hogy a Siriabeliek tábora szekerek zörgését és lovak dobogását, és nagy sereg robogását hallotta, és mondának egymásnak: Ímé az Izráel királya bérbe fogadta meg ellenünk a Hitteusok királyit és az Égyiptombeliek királyit, hogy ellenünk jõjjenek.
\par 7 És felkelvén elfutának alkonyatkor, és elhagyák mind sátoraikat, mind lovaikat, mind szamaraikat, a mint a tábor volt, és elfutának, csakhogy életöket megmenthessék.
\par 8 Mikor azért e bélpoklosok a tábor széléhez értek, bemenvén egy sátorba evének és ivának, és elvivének onnét ezüstöt, aranyat és ruhákat, és elmenvén elrejték azokat; és megtérvén más sátorba menének be, és abból is hozának és elmenvén, elrejték.
\par 9 És monda egyik a másiknak: Nem igazán cselekszünk: ez a mai nap örömmondás napja, ha mi hallgatunk, és a virradatot megvárjuk, büntetés ér bennünket; most azért jertek és menjünk el, és mondjuk meg a király házának.
\par 10 És elmenének, és kiáltának a város kapuján állónak, és elbeszélék nékik, mondván: Odamentünk volt a Siriabeliek táborába, és ímé már nem volt ott senki; emberek szava nem hallatszott, csak a lovak és szamarak vannak kikötve, és a sátorok úgy, a mint voltak.
\par 11 Kiáltának azért a kapunállók, és hírré tevék ott benn a király házában.
\par 12 És felkele éjszaka a király, és monda az õ szolgáinak: Megmondom néktek, mit csinálnak velünk a Siriabeliek. Tudják, hogy éhen vagyunk, és csak azért mentek ki a táborból, hogy elrejtõzzenek a mezõn, mondván: Mikor kijönnek a városból, megfogjuk õket elevenen, és bemegyünk a városba.
\par 13 Akkor felele egy az õ szolgái közül, és monda: Ki kell választani a megmaradt lovak közül, a melyek a városban megmaradtak, ötöt, - ímé épen olyanok ezek, mint Izráelnek egész sokasága, a mely megmaradt; ímé épen olyanok ezek, mint Izráel egész sokasága, a mely elpusztult, - és küldjük ki, hadd lássuk meg.
\par 14 És vevének két szekeret lovakkal, és kiküldé a király a siriaiak táborába, mondván: Menjetek el és nézzétek meg.
\par 15 És mikor utánuk mentek egész a Jordánig, ímé az egész út rakva volt ruhákkal és edényekkel, a melyeket a siriaiak a sietségben elhánytak. És mikor visszajöttek a követek, és elmondák ezt a királynak:
\par 16 Kiment a nép, és kirabolta a Siriabeliek táborát, és egy köböl zsemlyelisztet egy sikluson, és két köböl árpát egy sikluson vettek, az Úrnak beszéde szerint.
\par 17 A király pedig azt a fõembert, a kinek kezére szokott támaszkodni, oda rendelte a kapuhoz. És a nép eltapodá õt a kapuban, és meghala, a mint az Isten embere megmondotta, a ki megjövendölte ezt, mikor a király lement hozzá.
\par 18 Úgy történt, a mint az Isten embere a királynak jövendölte: Két köböl árpát egy sikluson és egy köböl zsemlyelisztet egy sikluson adnak holnap ilyenkor Samaria kapujában.
\par 19 És ezt felelte volt a fõember az Isten emberének, mondván: Hacsak az Úr ablakokat nem csinál az égen; akkor meglehet? és õ azt mondotta rá: Ímé te szemeiddel meg fogod látni; de nem eszel belõle.
\par 20 És teljesen így történt vele, mert eltapodá õt a nép a kapuban és meghalt.

\chapter{8}

\par 1 És Elizeus szólott volt annak az asszonynak, a kinek a fiát feltámasztotta volt, mondván: Kelj fel, és menj el a te házadnépével együtt, és tartózkodjál, a hol tartózkodhatol; mert az Úr éhséget hívott elõ, és el is jött a földre hét esztendeig.
\par 2 És felkelt az asszony, és az Isten emberének beszéde szerint cselekedék, és elment õ és az õ háznépe, és lakék a Filiszteusok földében hét esztendeig.
\par 3 Mikor pedig elmult a hét esztendõ, visszatért az asszony a Filiszteusok földébõl, és elment, hogy panaszolkodjék a királynak az õ házáért és szántóföldeiért.
\par 4 A király pedig beszélt Géházival, az Isten emberének szolgájával, mondván: Beszéld el, kérlek, nékem mindazokat a csudálatos dolgokat, a melyeket Elizeus cselekedett.
\par 5 És mikor elbeszélé a királynak, mimódon támasztotta fel a halottat, ímé az asszony, a kinek a gyermekét feltámasztotta volt, éppen akkor kiáltott a királyhoz az õ házáért és szántóföldeiért; és monda Géházi: Uram király, ez az az asszony és ez az õ fia, a kit feltámasztott Elizeus.
\par 6 És kikérdezé a király az asszonyt és az elbeszélé néki; és ada a király õ mellé egy udvari szolgát, mondván: Adasd vissza néki minden jószágát, és a szántóföldnek minden hasznát attól az idõtõl fogva, a mióta elhagyta a földet egész mostanig.
\par 7 És Elizeus elment Damaskusba, Benhadád pedig, Siria királya beteg volt, és hírül adák néki, mondván: Az Isten embere ide jött.
\par 8 És monda a király Hazáelnek: Végy ajándékot kezedbe, és menj eleibe az Isten emberének, és kérj tanácsot az Úrtól õ általa, mondván: Meggyógyulok-é ebbõl a betegségbõl?
\par 9 És eleibe ment Hazáel, és ajándékokat vitt kezében mindenféle drága damaskusi jószágból negyven teve terhét; és elment, és megállott elõtte, és így szólt: A te fiad, Benhadád, Siria királya, az küldött engem hozzád, mondván: Vajjon meggyógyulok-é ebbõl a betegségbõl?
\par 10 Felele néki Elizeus: Menj el, mondd meg néki: Nem maradsz életben, mert megjelentette nékem az Úr, hogy halált hal.
\par 11 És mereven ránézett Hazáelre, mígnem zavarba jött; végre sírni kezdett az Isten embere.
\par 12 És monda Hazáel: Miért sír az én uram? És felele: Mert tudom a veszedelmet, a melyet az Izráel fiaira hozol; az õ erõs városait megégeted, az õ ifjait fegyverrel levágatod, és kis gyermekeit a földhöz vered, és terhes asszonyait ketté vágod.
\par 13 És monda Hazáel: Kicsoda a te szolgád, ez az eb, hogy ilyen nagy dolgokat cselekednék? És felele Elizeus: Megjelentette nékem az Úr, hogy te leszel Siria királya.
\par 14 És elméne Elizeustól, és beméne az õ urához, és az monda néki: Mit mondott Elizeus? és monda: Azt mondta, hogy meggyógyulsz.
\par 15 Másnap azonban elõvett egy takarót és bemártván azt vízbe, ráteríté az õ arczára, és meghalt: és uralkodék Hazáel õ helyette.
\par 16 És Jórámnak, az Akháb fiának, az Izráel királyának ötödik esztendejében, mikor még Josafát vala Júda királya, uralkodni kezdett Jórám, a Josafát fia, Júda királya.
\par 17 Harminczkét esztendõs volt, mikor uralkodni kezdett, és nyolcz esztendeig uralkodott Jeruzsálemben.
\par 18 És járt az Izráel királyainak útján, a miképen cselekedtek volt az Akháb házából valók; mert az Akháb leánya volt a felesége, és gonoszul cselekedék az Úr szemei elõtt.
\par 19 De az Úr még sem akarta elveszteni Júdát Dávidért, az õ szolgájáért; a mint megigérte volt néki, hogy szövétneket ad néki és az õ fiainak mindörökké.
\par 20 Az õ idejében szakadt el Edom a Júda birodalmától, és választott királyt magának.
\par 21 És átment Jórám Seirbe és minden harczi szekere õ vele, s mikor éjjel felkelt és megtámadta az Edomitákat, a kik körülzárták õt, és a szekerek fejedelmeit, megfutott a nép, kiki az õ hajlékába.
\par 22 És elszakadt Edom a Júda birodalmától mind e mai napig; ugyanebben az idõben szakadt el Libna is.
\par 23 Jórámnak egyéb dolgai pedig és minden cselekedetei, vajjon nincsenek-é megírva a Júda királyainak krónika-könyvében?
\par 24 És elaluvék Jórám az õ atyáival, és eltemetteték az õ atyáival a Dávid városában, és uralkodék Akházia, az õ fia, õ helyette.
\par 25 Jórámnak az Akháb, az Izráel királya fiának tizenkettedik esztendejében kezdett uralkodni Akházia, Jórámnak, a Júdabeli királynak fia.
\par 26 Huszonkét esztendõs volt Akházia, mikor uralkodni kezdett, és egy esztendeig uralkodott Jeruzsálemben; az õ anyjának neve Athália volt, Omrinak, az Izráel királyának leánya.
\par 27 És járt az Akháb házának útján, és gonoszul cselekedék az Úr szemei elõtt, mint az Akháb háza; mert az Akháb házának veje vala.
\par 28 És hadba ment Jórámmal, az Akháb fiával, Siria királya, Hazáel ellen, Rámóth Gileádba; de a Siriabeliek megverték Jórámot.
\par 29 Akkor visszatért Jórám király, hogy meggyógyíttassa magát Jezréelben a sebekbõl, a melyeket rajta a Siriabeliek Ráma alatt ütöttek, mikor Hazáel, Siria királya ellen harczolt. Akházia pedig, a Jórám fia, Júda királya aláméne, hogy meglátogassa Jórámot, az Akháb fiát Jezréelben, a hol az betegen feküdt.

\chapter{9}

\par 1 Elizeus próféta pedig szólíta egyet a próféták fiai közül, és monda néki: Övezd fel derekadat, és vedd kezedbe e korsócska olajat, és menj el Rámóth Gileádba.
\par 2 És menj be oda, és nézd meg, hol van Jéhu, Josafátnak, a Nimsi fiának fia. Mikor pedig oda érsz, költsd fel õt az õ atyjafiai közül, és vidd be a belsõ kamarába,
\par 3 És vedd elõ e korsócska olajat, és töltsd az õ fejére, ezt mondván: Azt mondja az Úr: Téged kentelek királylyá Izráelen! És az ajtót kinyitván, fuss el, és  semmit ott ne idõzz.
\par 4 És elment az ifjú, a próféta tanítványa, Rámóth Gileádba.
\par 5 És mikor bement, ímé a seregek fejedelmei ott ültek együtt, és õ monda: Beszédem volna veled, fejedelem! És monda Jéhu: Kivel volna beszéded ennyiõnk közül? És monda: Te veled, fejedelem!
\par 6 Felkele azért, és bement a házba, és fejére tölté az olajat, és monda néki: Azt mondja az Úr, Izráel Istene: Királylyá kentelek téged az Úrnak népén, az Izráelen,
\par 7 Hogy elveszítsed Akhábnak, a te uradnak háznépét; mert bosszút állok az én szolgáimnak, a  prófétáknak véréért, és mind az Úr szolgáinak véréért Jézabelen.
\par 8 És kivész egészen az Akháb háza, és kigyomlálom mind az Akhábhoz tartozókat, még az ebet is, mind a berekesztettet, mind az elhagyottat Izráelben;
\par 9 És olyanná teszem az Akháb házát, mint Jeroboámnak, a Nébát fiának házát, és mint Baasának, az Ahija fiának házát.
\par 10 Jézabelt pedig az ebek eszik meg Jezréel mezején, és nem lesz, a ki eltemesse õt. És kinyitván az ajtót, elfutott.
\par 11 És mikor Jéhu kiment az õ urának szolgáihoz, mondának néki: Békességes-é a dolog? Miért jött e bolond hozzád? És felele nékik: Hiszen ismeritek ez embert és az õ beszédét!
\par 12 És mondának: Hazugság! Mondd meg az igazat. És monda: Így s így szóla nékem, mondván: Azt mondja az Úr: Királylyá kentelek téged Izráelen.
\par 13 Akkor nagy sietséggel kiki mind vevé az õ ruháját, és alája terítették a grádics felsõ részére és megfuvaták a harsonákat, és kikiálták: Jéhu uralkodik!
\par 14 Így ütött pártot Jéhu, Josafátnak, a Nimsi fiának fia, Jórám ellen. Jórám pedig ott táborozott volt Rámóth Gileád alatt az egész Izráellel, Hazáel, Siria királya ellen.
\par 15 De visszatért volt Jórám király, hogy magát Jezréelben gyógyíttassa a sebekbõl, a melyeket a Siriabeliek ütöttek rajta, mikor Hazáel, Siria királya ellen harczolt. És monda Jéhu: Ha néktek is úgy tetszik, ne engedjetek senkit kimenni a városból, a ki elmenjen és hírül vigye ezt Jezréelbe.
\par 16 És befogatott Jéhu, és elment Jezréelbe, a hol Jórám feküdt, és a hova Akházia is, a Júda királya lement volt, hogy meglátogassa Jórámot.
\par 17 Mikor pedig az õrálló, a ki Jezréelben a tornyon állott, meglátta Jéhu seregét, hogy jõ, monda: Valami sereget látok. Akkor monda Jórám: Válaszsz egy lovast és küldj eléjök, és mondja ezt: Békességes-é a dolog?
\par 18 És oda lovagolt a lovas eléjük, és monda: Azt kérdi a király: Békességes-é a dolog? Felele Jéhu: Mi gondod van a békességgel? Kerülj a hátam mögé. Megjelenté pedig ezt az õrálló, mondván: Hozzájok ment ugyan a követ; de nem tért vissza.
\par 19 Akkor elküldött egy másik lovast, a ki hozzájok ment, és monda: Azt kérdi a király: Békességes-é a dolog? Felele Jéhu: Mi gondod van a békességgel? Kerülj a hátam mögé.
\par 20 Hírül adá ezt is az õrálló, mondván: Hozzájok ment ugyan, de nem jött vissza. De a hajtás olyan, mint Jéhunak, a Nimsi fiának hajtása; mert mint egy õrült, úgy hajt.
\par 21 Akkor monda Jórám: Fogjatok be! És befogván az õ szekerét, kiméne Jórám, az Izráel királya, és Akházia, a Júda királya, mindenik a maga szekerén, Jéhu eleibe, és a Jezréelbeli Nábót mezején találkoztak vele.
\par 22 És mikor meglátta Jórám Jéhut, monda: Békességes-é a dolog, Jéhu? Felele õ: Mit békesség?! Mikor Jézabelnek, a te anyádnak paráznasága és varázslása mindig nagyobb lesz!
\par 23 Akkor megfordítá Jórám az õ kezét és futni kezde, és monda Akháziának: Árulás ez, Akházia?
\par 24 Jéhu pedig meghúzván az ívét, Jórámot hátba lövé a lapoczkák között úgy, hogy a szívén ment át a nyíl, és lerogyott a szekérben.
\par 25 És monda Bidkárnak, az õ hadnagyának: Fogd meg és vesd a Jezréelbeli Nábót mezejére; mert emlékezz csak vissza, mikor mi, én és te, ketten az õ atyja, Akháb után lovagoltunk, és az Úr õ felõle ezt a fenyegetést mondotta:
\par 26 Bizonyára megkeresem a Nábót vérét és az õ fiainak vérét, a melyet tegnap láttam, azt mondja az Úr: Azért megfizetek néked ezen a szántóföldön, azt mondja az Úr: Most azért fogjad õt, és vesd a szántóföldre, az Úr beszéde szerint.
\par 27 Akházia pedig, a Júda királya, mikor látta ezt, futni kezde a kert házának útján; de Jéhu utána menvén, monda: Ezt is vágjátok le a szekérben és megsebesíték õt a Gúr hágójánál, a mely Jibleám mellett van; és õ Megiddóba  menekülvén, ott meghalt.
\par 28 És az õ szolgái elvitték õt szekéren Jeruzsálembe, és eltemették az õ sirboltjába az õ atyáival a Dávid városában.
\par 29 Akházia pedig uralkodni kezdett Júdában Jórámnak, az Akháb fiának tizenegyedik esztendejében.
\par 30 És mikor Jéhu Jezréelbe ment és Jézabel ezt meghallotta, arczát megékesíté kenettel, felékesítette fejét, és kitámaszkodott az ablakon.
\par 31 És mikor Jéhu bevonult a kapun, monda: Békesség van-é, óh Zimri! uradnak gyilkosa?
\par 32 Õ pedig feltekintve az ablakra, monda: Ki van ott velem? Ki? És alátekintett két vagy három fõember.
\par 33 És monda azoknak: Vessétek alá õt. És aláveték, és az õ vére szétfrecskendezett a falra és a lovakra, és eltapodtatá õt.
\par 34 Bemenvén pedig oda, evett és ivott, és monda: Nézzetek utána annak az átkozottnak és temessétek el; hiszen mégis csak  király leánya.
\par 35 De mikor kimentek, hogy eltemetnék õt, már semmit sem találtak belõle, csak a koponyáját, a lábait és a keze fejeit.
\par 36 És visszamenvén, megmondák néki, és õ monda: Ez az Úr beszéde, a melyet szólott az õ szolgája, Thesbites Illés által, mondván: Az ebek eszik meg Jézabel testét a Jezréel földén,
\par 37 És olyan lesz Jezréel földjén a Jézabel teste, mint a mezõn a ganéj, úgy, hogy senki meg nem mondhatja: Ez Jézabel!

\chapter{10}

\par 1 Akhábnak pedig hetven fia volt Samariában. És levelet írt Jéhu, és elküldé azt Samariába Jezréel fejedelmeihez, a vénekhez, és az Akháb fiainak tútoraihoz, ilyen parancsolattal:
\par 2 Mihelyt e levél hozzátok jut, a kiknél vannak a ti uraitok fiai a szekerekkel, lovakkal, az erõs városokkal és fegyverekkel együtt,
\par 3 Nézzétek meg, melyik a legjobb és a legigazabb a ti uratok fiai közül, azt ültessétek az õ atyjának királyi székébe, és harczoljatok a ti uratok házáért.
\par 4 És megrettenének felette igen, és mondának; Ímé már két király nem maradhatott meg õ elõtte, mimódon maradhatnánk hát mi meg?
\par 5 És elküldének mind a király házának, mind a városnak fejedelmei, és a vének és a tútorok Jéhuhoz, mondván: A te szolgáid vagyunk mi, valamit parancsolsz nékünk, azt cselekeszszük; mi senkit királylyá nem teszünk, a mi néked tetszik, azt cselekedjed!
\par 6 És írt nékik levelet másodszor is, mondván: Ha velem tartotok és az én beszédemre hallgattok, vegyétek fejöket a férfiaknak, a ti uratok fiainak, és jõjjetek hozzám holnap ilyenkor Jezréelbe. És a király fiai: hetven férfiú, és a város nagyjaival tartottak, a kik nevelték õket.
\par 7 És mikor a levél hozzájok jutott, vevék a király fiait, és megölték a hetven férfiút, és fejeiket kosarakba rakták, és õ hozzá küldték Jezréelbe.
\par 8 És mikor odaérkezett a követ, és bejelenté néki, mondván: Elhozták a király fiainak fejeit, monda: Rakjátok két rakásba azokat a kapu elõtt reggelig.
\par 9 És mikor reggel kiment, megállott, és monda az egész népnek: Ti igazak vagytok. Ímé az én uram ellen én ütöttem pártot és én öltem meg õt; de ki ölte meg mind ezeket?
\par 10 Azért vegyétek eszetekbe ebbõl, hogy az Úrnak beszédébõl egyetlen egy sem esik a földre, a mit az Úr az Akháb háza ellen szólott, és az Úr véghezvitte, a mit az õ szolgája, Illés által mondott.
\par 11 Azután levágá Jéhu mindazokat, a kik megmaradtak volt az Akháb házából Jezréelben, és minden fõemberét, egész rokoságát és minden papját, míg csak ki nem irtotta a maradékát is.
\par 12 És felkelvén, elindult és elment Samariába. De útközben volt a pásztorok juhnyíró háza.
\par 13 És itt Akháziának, a Júda királyának atyjafiaira talált Jéhu, és monda: Kik vagytok? És felelének azok: Akházia atyjafiai vagyunk, és alájöttünk, hogy köszöntsük a király gyermekeit és a királyné fiait.
\par 14 Õ pedig monda: Fogjátok meg õket elevenen. És megfogták õket elevenen, és megölték õket a juhnyíróház kútja mellett, negyvenkét férfiút; és egyetlen egyet sem hagyott meg közülük.
\par 15 És mikor elment onnét, Jonadábbal, a Rekháb fiával találkozott, a ki elébe jött, és köszönté õt, és monda néki: Vajjon olyan igaz-é a te szíved, mint az én szívem a te szívedhez? És felele Jonadáb: Olyan. Ha így van, nyújts kezet. És õ kezet nyújta, és felülteté õt maga mellé a szekérbe.
\par 16 És monda: Jer velem és lásd meg, mint állok bosszút az Úrért. És vele együtt vitték õt az õ szekerén.
\par 17 És megérkezett Samariába, és levágta mindazokat, a kik az Akháb nemzetségébõl megmaradtak volt Samariában, míg ki nem veszté azt az Úr beszéde szerint, a melyet szólott Illésnek.
\par 18 És Jéhu összegyûjté az egész népet, és monda néki: Akháb kevéssé szolgálta Baált; Jéhu sokkal jobban akarja szolgálni.
\par 19 Most azért hívjátok hozzám a Baál minden prófétáit, minden papját és minden szolgáját; senki el ne maradjon; mert nagy áldozatot akarok tenni a Baálnak; valaki elmarad, meg kell halni annak. Jéhu pedig ezt álnokságból cselekedte, hogy elveszítse a Baál tisztelõit.
\par 20 És monda Jéhu: Szenteljetek ünnepet Baálnak. És kikiálták.
\par 21 És szétküldött Jéhu egész Izráelbe, és eljövének mind a Baál tisztelõi, és senki el nem maradt a ki el nem jött volna, és bemenének a Baál templomába, és megtelék a Baál temploma minden zugában.
\par 22 Akkor monda a ruhatárnoknak: Hozz ruhákat ki a Baál minden tisztelõinek. És hozott nékik ruhákat.
\par 23 És bement Jéhu és Jonadáb, a Rekháb fia a Baál templomába, és monda a Baál tisztelõinek: Tudakozzátok meg és lássátok meg, hogy valamiképen ne legyen itt veletek az Úr szolgái közül valaki, hanem csak a Baál tisztelõi.
\par 24 És mikor bementek, hogy ajándékokkal és égõáldozatokkal áldozzanak, Jéhu oda állított kivül nyolczvan embert, a kiknek azt mondta: A ki egyet elszalaszt azok közül, a kiket én kezetekbe adok, annak meg kell érette halni.
\par 25 Mikor pedig elvégezték az égõáldozatot, monda Jéhu a vitézeknek és hadnagyoknak: Menjetek be, vágjátok le õket, csak egy is közülök meg ne meneküljön! És levágták õket fegyver élével, és elhányták az õ holttestöket a vitézek és a hadnagyok. Azután elmentek a Baál templomának városába.
\par 26 És kihordván a Baál templomának bálványait, megégeték azokat.
\par 27 És lerontották a Baál képét is templomostól együtt, és azt árnyékszékké tették mind e mai napig.
\par 28 Így veszté ki Jéhu a Baált Izráelbõl.
\par 29 De Jeroboámnak, a Nébát fiának bûneitõl, a ki vétekbe ejté az Izráelt, nem szakadt el Jéhu, az arany borjúktól, melyek Béthelben és Dánban valának.
\par 30 És monda az Úr Jéhunak: A miért szorgalmatosan megcselekedted azt, a mi nékem tetszett, és az én szívem kívánsága szerint cselekedtél az Akháb házával: azért a te fiaid negyedízig ülnek az Izráel királyi székiben.
\par 31 De Jéhu még sem igyekezett azon, hogy az Úrnak, Izráel Istenének törvényében járjon teljes szívébõl, mert nem szakadt el a Jeroboám bûneitõl, a ki bûnbe ejtette az Izráelt.
\par 32 Abban az idõben kezdett az Úr pusztítani Izráelben, és megveré õket Hazáel, Izráel minden határában.
\par 33 A Jordántól egész napkeletig, a Gileádbelieknek, a Gád nemzetségének, a Rúben nemzetségének, Manasse nemzetségének egész földjét. Aroertõl fogva; mely az Arnon patak mellett van, mind Gileádot, mind Básánt.
\par 34 Jéhunak egyéb dolgai pedig és minden cselekedetei, és minden erõssége, vajjon nincsenek-é megírva az Izráel királyainak krónika-könyvében?
\par 35 És elaluvék Jéhu az õ atyáival, és eltemeték õt Samariában. És uralkodék Joakház, az õ fia, helyette.
\par 36 A napok pedig, a melyekben uralkodott Jéhu Izráelen Samariában, huszonnyolcz esztendõ.

\chapter{11}

\par 1 És Athália, Akházia anyja, mikor látta, hogy fia meghalt, felkelt  és az egész királyi magot megölette.
\par 2 De Jóseba, Jórám király leánya, Akházia huga, fogá Joást, Akházia fiát, és ellopván õt a király fiai közül, a kik megölettettek vala, elrejté az õ dajkájával együtt az ágyasházban; és elrejték õt Athália elõl úgy, hogy ez nem öletteték meg.
\par 3 És elrejtve vala Joás õ vele együtt az Úr házában hat esztendeig, és Athália uralkodott az országban.
\par 4 A hetedik esztendõben azonban elküldött Jójada, és magához hivatta a Káriánusok és a testõrök csapatainak vezéreit, és bevitte õket magával az Úr házába, és szövetséget kötött velök, és megesküdteté õket az Úr házában, és megmutatta nékik a király fiát.
\par 5 És megparancsolta nékik, mondván: Ezt kell cselekednetek: közületek a harmadrész, a kik szombaton jõnek õrségre, tartsa a király házának õrizetét;
\par 6 A harmadrésze legyen a Súr kapuban, és a harmadrésze a testõrök háta mögött való kapuban, és nagy szorgalmatossággal õrizzétek e házat, hogy senki reánk ne üthessen.
\par 7 Az a két rész pedig közületek, az egész, a mely szombaton az õrszolgálatból kilép, vigyázzon az Úr házának megõrzésére a királyhoz vezetõ részen.
\par 8 És vegyétek körül a királyt, mindenki kezében fegyverével, és a ki a sorokba benyomul, ölettessék meg, és a király körül legyetek kijövetelekor és bemenetelekor.
\par 9 És a csapatok vezérei úgy cselekedtek, a mint meghagyta volt nékik Jójada pap; és mindenik maga mellé vette az õ embereit, azokat, a kik bementek szombaton az õrségre, azokkal együtt, a kiket felváltottak szombaton, és Jójada paphoz mentek.
\par 10 És a pap a csapatok vezéreinek azokat a kopjákat és paizsokat adta oda, a melyek Dávid királyéi voltak, és az Úr házában voltak.
\par 11 És a testõrök ott állottak a király körül, mindenik fegyverrel kezében, a ház jobb szárnyától a bal szárnyáig, az oltár és a szenthely felé.
\par 12 És kivezeté Jójada a király fiát, és fejébe tevé a koronát, és kezébe adá a bizonyságtételt, és királylyá tevék õt és felkenék; és kezeikkel tapsolván, kiáltanak: Éljen a király!
\par 13 Mikor pedig Athália meghallotta a testõröknek és a községnek kiáltását, bement a néphez az Úr házába.
\par 14 És mikor szétnézett, ímé a király ott állott az emelvényen, a mint szokás, és a király elõtt állottak a fejedelmek és a trombitások, és az egész föld népe ujjongott, és trombitáltak. Akkor megszaggatá Athália az õ ruháit, és kiáltván, monda: Összeesküvés! Összeesküvés!
\par 15 De parancsola Jójada pap a csapatok vezéreinek, a sereg hadnagyainak, és monda nékik: Vezessétek ki õt a sorok között, és ha valaki õt követné, öljétek meg fegyverrel. Mert azt mondta a pap: Ne ölettessék meg az Úr házában.
\par 16 És útat nyitottak néki, és õ arra az útra ment, a merre a lovakat vezetik be a király házába, és ott megölték.
\par 17 És kötést tõn Jójada az Úr, a király és a nép között, hogy õk az Úrnak népe lesznek; és külön a király és a nép között.
\par 18 És elment az egész föld népe a Baál templomába, és azt lerombolta, és oltárait és bálványait teljesen összetörte, és Matthánt, a Baál  papját az oltárok elõtt ölték meg. És a pap gondviselõket rendele az Úr házában.
\par 19 És maga mellé vévén a csapatok vezéreit, a Káriánusokat és a testõröket és az egész föld népét, elvivék a királyt az Úrnak házából, és menének a testõrök kapujának útján a király házához. És Joás üle a királyok székébe.
\par 20 És örvendezett mind az egész föld népe, és megnyugovék a város, miután megölték Atháliát fegyverrel a király háza mellett.
\par 21 Joás hét esztendõs volt, mikor uralkodni kezdett.

\chapter{12}

\par 1 Jéhunak hetedik esztendejében tették Joást királylyá, és negyven esztendeig uralkodott Jeruzsálemben; az õ anyjának pedig Sibja vala neve, a ki Beersebából való volt.
\par 2 És cselekedék Joás kedves dolgot az Úr szemei elõtt mindaddig, a míg Jójada pap oktatá õt;
\par 3 Azonban a magaslatok oltárai nem rontattak le, a nép még mindig ott áldozott és tömjénezett a magaslatok oltárain.
\par 4 És megparancsolá Joás a papoknak, hogy minden pénzt, a mely Istennek szenteltetik és az Úr házába bevitetik, mind a megszámlálás pénzét és a személyek váltságának pénzét, mind pedig, a melyet kiki szabad akaratja szerint az Úr  házához bevisz.
\par 5 Vegyék magokhoz a papok, mindenik a maga ismerõsétõl, és építsék meg az Úr házának romlásait mindenütt, a hol romlást látnak rajta.
\par 6 De mikor Joás király huszonharmadik esztendejéig sem építették meg a papok az Úr házának romlásait,
\par 7 Elõhivattatá Joás király Jójada papot a többi papokkal együtt, és monda nékik: Miért nem építitek meg az Úr házának romlásait? Ezután ne vegyétek a pénzt magatokhoz a ti ismerõseitektõl, hanem az Úr háza romlásainak építésére adjátok azt.
\par 8 És a papok beleegyeztek, hogy többé semmi pénzt nem vesznek el a néptõl, és nem javítják ki a ház romlásait.
\par 9 És Jójada pap vett egy ládát, és annak a fedelén csinált egy nyílást, és helyhezteté azt az oltár mellé jobbfelõl, a mely felõl az Úr házába bemennek, hogy abba töltsenek a papok, a templom küszöbinek õrizõi minden pénzt, a melyet az Úr házába hoznak.
\par 10 És mikor látták, hogy sok pénz van a ládában, felment a király íródeákja a fõpappal együtt, csomóba kötötték és megszámlálták a pénzt, a mely az Úr házában találtatott.
\par 11 És a megmért pénzt a munka vezetõinek kezeibe adták, a kik az Úr házához voltak rendelve, és ezek kiadták azt az ácsoknak és építõknek, a kik a házon dolgoztak,
\par 12 És a kõmíveseknek és a kõfaragóknak, fa és faragott kõ vásárlására, az Úrnak háza romlásainak kijavítására, és mindarra, a mi a ház javítására szükséges volt.
\par 13 De abból a pénzbõl, a melyet bevittek az Úr házába, nem csináltattak az Úr házához sem ezüst poharakat, sem késeket, sem medenczéket, sem trombitákat, sem valami egyéb arany vagy ezüst edényt.
\par 14 Hanem a munkásoknak adták azt, és csak az Úr házát javították ki abból.
\par 15 Számot sem vettek azoktól az emberektõl, a kiknek keze által kiadták a pénzt a munkásoknak, mert híven jártak el.
\par 16 De a vétekért és a bûnért való pénzt nem vitték az Úr házába; az a papoké lett.
\par 17 Ebben az idõben jött fel Hazáel, Siria királya, és megszállotta Gáthot, és be is vette azt; azután megfordult Hazáel, hogy Jeruzsálem ellen menjen.
\par 18 De Joás, Júda királya, vevé mind a megszentelt ajándékokat, a melyeket az õ atyái, Josafát, Jórám és Akházia, Júda királyai az Úrnak szenteltek volt, és a melyeket õ maga szentelt néki, és minden aranyat, a mely találtaték mind az Úr házának, mind pedig a király házának kincsei között, és elküldé Hazáelnek, Siria királyának, és az elment Jeruzsálem alól.
\par 19 Joásnak egyéb dolgai és minden cselekedetei pedig, vajjon nincsenek-é megírva a Júda királyainak krónika-könyvében?
\par 20 Fellázadván pedig az õ szolgái, pártot ütének ellene, és megölték Joást Beth-Millóban, a merre Sillába megy az ember.
\par 21 Jozakhár, a Simeáth fia, és Jozadáb, a Sómer fia, az õ szolgái ölték meg õt, és meghalt és eltemették õt az õ atyáival, a Dávid városában. És Amásia, az õ fia lett a király õ helyette.

\chapter{13}

\par 1 Joásnak, az Akházia fiának, Júda királyának huszonharmadik esztendejében lett király Joákház, a Jéhu fia Izráelen Samariában tizenhét esztendeig.
\par 2 És gonoszul cselekedék az Úr szemei elõtt, és követé Jeroboámnak, a Nébát fiának bûneit, a ki bûnbe ejté az Izráelt, és nem szakadt el azoktól.
\par 3 És felgerjedt az Úr haragja Izráel ellen, és adá õket Hazáelnek, a Siria királyának kezébe, és Benhadádnak, a Hazáel fiának kezébe, az õ életének minden idejében.
\par 4 De könyörgött Joákház az Úrnak, és meghallgatá õt az Úr; mert megtekintette az Izráel nyomorúságát, hogy mikép nyomorgatja õket Siria királya.
\par 5 És az Isten szabadítót küldött Izráelnek, a ki megmentette õket a Siriabeliek hatalmától, hogy az Izráel fiai lakhassanak az õ sátoraikban, mint azelõtt.
\par 6 De még sem távoztak el a Jeroboám házának bûneitõl, a ki bûnbe ejté az Izráelt, hanem azokban jártak; sõt az Asera is helyén maradt Samariában;
\par 7 Pedig nem hagyattatott meg Joákháznak több nép, mint ötven lovag, tíz szekér és tízezer gyalogos; a többit mind megölte Siria királya, és olyanná tette, mint a cséplõ pora.
\par 8 Joákház egyéb dolgai pedig és minden cselekedetei és az õ ereje, vajjon nincsenek-é megírva az Izráel királyainak krónika-könyvében?
\par 9 És elaluvék Joákház az õ atyáival, és eltemették õt Samariában; és az õ fia, Joás, uralkodék õ helyette.
\par 10 Joásnak, a Júda királyának harminczhetedik esztendejében lett a király Joás, a Joákház fia, Izráelen Samariában, tizenhat esztendeig.
\par 11 És gonoszul cselekedék az Úr szemei elõtt, és nem távozék el Jeroboámnak, a Nébát fiának semmi bûnétõl, a ki az Izráelt bûnbe ejté, hanem azokban járt.
\par 12 Joásnak egyéb dolgai pedig és valamit cselekedett és az õ erõssége, a melylyel Amásia, a Júda királya ellen harczolt, vajjon nincsenek-é megírva az Izráel királyainak krónika-könyvében?
\par 13 És elaluvék Joás, az õ atyáival, és Jeroboám ült az õ királyi székibe; és eltemetteték Joás Samariában, az Izráel királyai mellé.
\par 14 És megbetegedett Elizeus olyan betegséggel, a melybe bele is halt, és lement hozzá Joás, az Izráel királya, és arczára borulván sírt és monda: Édes atyám, édes atyám! Izráel szekerei és lovagjai!
\par 15 És monda néki Elizeus: Végy kézívet és nyilakat. És õ kézívet és nyilakat vett kezébe.
\par 16 Akkor monda az Izráel királyának: Fogd meg kezeddel a kézívet. És õ megfogván azt kezével, Elizeus is rátette kezeit a király kezeire;
\par 17 És monda: Nyisd ki az ablakot napkelet felõl. És mikor kinyitotta, monda Elizeus: Lõjj! És lõtt. Akkor monda: Az Úrnak gyõzedelmes nyíla ez, gyõzedelmes nyíl a Siriabeliek ellen; mert megvered a Siriabelieket Afekben a megsemmisülésig.
\par 18 Monda azután: Vedd fel a nyilakat. És felvette. Õ pedig monda az Izráel királyának: Lõjj a földbe, és bele lõtt a földbe háromszor; azután abbahagyta.
\par 19 Akkor megharaguvék reá az Isten embere, és monda: Ötször vagy hatszor kellett volna lõnöd, akkor megverted volna a Siriabelieket a megsemmisülésig; de így már csak háromszor vered meg a Siriabelieket.
\par 20 Azután meghalt Elizeus, és eltemették. A moábita portyázó csapatok pedig az országba törtek a következõ esztendõben.
\par 21 És történt, hogy egy embert temettek, és mikor meglátták a csapatokat, gyorsan odatették azt az embert az Elizeus sírjába; de a mint odajutott és hozzáért az Elizeus tetemeihez, megelevenedett és lábaira állott.
\par 22 Hazáel pedig, Siria királya nyomorgatá az Izráelt Joákház minden napjaiban.
\par 23 De megkegyelmezett az Úr nékik, és megkönyörült rajtok, és hozzájok tért az õ szövetségéért, a melyet kötött volt Ábrahámmal, Izsákkal és Jákóbbal, és nem akarta õket elveszíteni, sem el nem vetette õket az õ orczája elõl mind ez ideig.
\par 24 És meghalt Hazáel, Siria királya, és az õ fia, Benhadád lett helyette a király.
\par 25 És visszavevé Joás, a Joákház fia Benhadádnak, a Hazáel fiának kezébõl a városokat, a melyeket erõvel elvett volt Joákháznak, az õ atyjának kezébõl. Háromszor veré meg õt Joás, és visszaszerezte Izráel városait.

\chapter{14}

\par 1 Joásnak, a Joákház fiának, Izráel királyának második esztendejében kezdett uralkodni Amásia, Joásnak, a Júda királyának fia.
\par 2 Huszonöt esztendõs volt, mikor uralkodni kezdett, és huszonkilencz esztendeig uralkodott Jeruzsálemben. Az õ anyjának pedig Joáddán volt a neve, Jeruzsálembõl való.
\par 3 És kedves dolgot cselekedett az Úr szemei elõtt, de még sem annyira, mint Dávid, az õ atyja, hanem csak úgy tett mindent, mint az õ atyja, Joás.
\par 4 Mert a magaslatokat nem rontották le, hanem a nép még mindig ott áldozott és ott tömjénezett a magaslatokon.
\par 5 Mikor azután a birodalom kezében megerõsödött, megölte az õ szolgáit, a kik a királyt, az õ atyját megölték volt.
\par 6 De a gyilkosok fiait nem ölte meg, a mint meg van írva a Mózes törvénykönyvében, a melyben az Úr megparancsolta és megmondotta: Meg ne ölettessenek az atyák a fiakért, se a fiak meg ne ölettessenek az atyákért; kiki az õ bûnéért haljon meg.
\par 7 Ugyanõ megvert az Edomiták közül tízezeret a sós völgyben; és bevette Sélát ostrommal, és ezt nevezé Joktéelnek mind e mai napig.
\par 8 Akkor követeket küldött Amásia Joáshoz, a Joákház fiához, a ki Jéhunak, az Izráel királyának volt a fia, ezt izenvén: Jere, nézzünk egymás szemébe!
\par 9 És elkülde Joás, Izráel királya Amásiához, Júda királyához, ezt válaszolván: A bogácskóró, mely a Libánonon van, elküldött a Libánon czédrusfájához, ezt izenvén: Add a te leányodat az én fiamnak feleségül; de a mezõ vadja, a mely a Libánonon van, átszaladt a bogácskórón és eltapodta azt.
\par 10 Megverted az Edomitákat, azért fuvalkodtál fel szívedben. Elégedjél meg a dicsõséggel és maradj otthon; miért akarsz a szerencsétlenséggel harczra kelni, hogy elessél magad is és Júda is veled együtt?
\par 11 De Amásia nem hallgatott reá, és felvonult Joás, Izráel királya, és szembe-szállottak egymással, õ és Amásia, Júda királya Bethsemesnél, amely Júdában van.
\par 12 De megveretett Júda Izráeltõl, és kiki elfutott az õ sátorába.
\par 13 És Amásiát, Júda királyát, Joásnak, az Akházia fiának fiát is elfogta Joás, Izráel királya Bethsemesnél; és Jeruzsálem ellen ment, és lerontotta Jeruzsálem kerítését, az Efraim kapujától egész a szegletkapuig, négyszáz singre.
\par 14 És elvitt minden aranyat, minden ezüstöt és minden edényt, a mely találtatott az Úr házában és a király házának kincsei között, és kezeseket is vévén, visszatért Samariába.
\par 15 Joás egyéb dolgai pedig, a melyeket cselekedett, és az õ erõssége, és miképen hadakozott Amásia, a Júda királya ellen, vajjon nincsenek-é megírva az Izráel királyainak krónika-könyvében?
\par 16 És elaluvék Joás az õ atyáival, és eltemetteték Samariában az Izráel királyaival együtt; és az õ fia, Jeroboám uralkodék õ helyette.
\par 17 Amásia pedig, Joásnak, a Júda királyának fia, Joásnak, a Joákház fiának, az Izráel királyának halála után még tizenöt esztendeig élt.
\par 18 És Amásiának egyéb dolgai vajjon nincsenek-é megírva a Júda királyainak krónika-könyvében?
\par 19 És pártot ütöttek õ ellene Jeruzsálemben, és õ Lákisba menekült; de utána küldtek Lákisba, és megölték ott.
\par 20 És visszahozták õt lovakon, és eltemetteték Jeruzsálemben, a Dávid városában az õ atyáival.
\par 21 És az egész Júda népe vevé Azáriát, a ki tizenhat esztendõs volt, és õt tette királylyá, az õ atyja, Amásia helyett.
\par 22 Õ építé meg Elátot, a melyet visszavett Júdának azután, hogy a király elaludt az õ atyáival.
\par 23 Amásiának, Joásnak, a Júda királya fiának tizenötödik esztendejétõl fogva uralkodott Jeroboám, Joásnak, az Izráel királyának fia Samariában, negyven esztendeig.
\par 24 És gonoszul cselekedék az Úr szemei elõtt, mert Jeroboámnak, a Nébát fiának semmi bûnétõl el nem távozott, a ki bûnbe ejtette az Izráelt.
\par 25 Õ szerezte vissza az Izráel határát Emáthtól fogva a pusztasági tengerig, az Úrnak, Izráel Istenének beszéde szerint, a melyet szólott volt az õ szolgája, Jónás próféta, az Amittai fia által, a ki Gáth-Kéferbõl való volt.
\par 26 Mert megtekintette az Úr az Izráel igen nagy nyomorúságát, hogy a berekesztetett és az elhagyatott is semmi és nincs senki, a ki az Izráelt megszabadítaná.
\par 27 És nem mondta azt az Úr, hogy az Izráel nevét eltörli az ég alól; annakokáért megszabadította õket Jeroboám, a Joás fia által.
\par 28 Jeroboám egyéb dolgai pedig és minden cselekedetei és az õ erõssége, hogy miképen hadakozott, és mi módon nyerte vissza Damaskust és a Júdához tartozó Hámátot az Izráelnek, vajjon nincsenek-é megírva az Izráel királyainak krónika-könyvében?
\par 29 És elaluvék Jeroboám az õ atyáival, Izráel királyaival, és uralkodék az õ fia, Zakariás, õ helyette.

\chapter{15}

\par 1 Jeroboámnak, az Izráel királyának huszonhetedik esztendejében kezdett uralkodni Azária, Amásiának, a Júda királyának fia.
\par 2 Tizenhat esztendõs volt, mikor uralkodni kezdett, és ötvenkét esztendeig uralkodott Jeruzsálemben, és az õ anyjának neve Jekólia, Jeruzsálembõl való.
\par 3 És kedves dolgot cselekedék az Úr szemei elõtt mind a szerint, a mint az õ atyja Amásia cselekedett;
\par 4 Csakhogy a magaslatok nem rontattak le, a nép még ott áldozott és tömjénezett a magaslatokon.
\par 5 És megverte az Úr a királyt, mert bélpoklos lett egész halála napjáig, és külön házban lakott, és Jótám, a király fia kormányozta a házat, és õ szolgáltatott törvényt a föld népének.
\par 6 Azáriának egyéb dolgai pedig és minden cselekedetei, vajjon nincsenek-é megírva a Júda királyainak krónika-könyvében?
\par 7 És elaluvék Azária az õ atyáival, és eltemeték õt az õ atyáival a Dávid városában, és az õ fia,  Jótám, uralkodék õ helyette.
\par 8 Azáriának, a Júda királyának harmincznyolczadik esztendejében kezdett uralkodni Zakariás, a Jeroboám fia Izráelen, Samariában hat hónapig.
\par 9 És gonoszul cselekedék az Úr szemei elõtt, miképen az õ atyái cselekedtek volt. Nem hagyta el Jeroboámnak, a Nébát fiának bûneit, a ki vétekbe ejtette az Izráelt.
\par 10 És összeesküdött ellene Sallum, a Jábes fia, és megölte õt a nép szeme láttára, és megölvén õt, õ uralkodott helyette.
\par 11 Zakariásnak egyéb dolgai pedig, ímé meg vannak írva az Izráel királyainak krónika-könyvében.
\par 12 Ez az Úr beszéde, a mit Jéhu felõl szólott, mondván: A te fiaid negyedízig ülnek Izráel trónján. És így történt.
\par 13 Sallum, a Jábes fia lett a király Uzziának, Júda királyának harminczkilenczedik esztendejében, de csak egy hónapig uralkodott Samariában;
\par 14 Mert Menáhem, a Thirsabeli Gádi fia feljött, és Samariába ment, és megölte Sallumot, a Jábes fiát Samariában, és megölvén õt, õ uralkodott helyette.
\par 15 Sallumnak egyéb dolgai pedig és összeesküvése, a melyet csinált, ímé meg vannak írva az Izráel királyainak krónika-könyvében.
\par 16 Akkor verte le Menáhem Tifsáhot és mindazokat, a kik benne voltak, és egész határát Thirsától fogva, mert nem bocsátották be, azért veré le, és még a terhes asszonyokat is mind felhasogatá benne.
\par 17 Azáriának, Júda királyának harminczkilenczedik esztendejétõl fogva uralkodott Menáhem, a Gádi fia Izráelen tíz esztendeig Samariában.
\par 18 És gonoszul cselekedék az Úr szemei elõtt; nem hagyta el Jeroboámnak, a Nébát fiának bûneit teljes életében, a ki vétekbe ejtette az Izráelt.
\par 19 És rátört Púl, Assiria királya az országra, és Menáhem Púlnak ezer tálentom ezüstöt adott azért, hogy legyen néki segítséggel az õ birodalmának megerõsítésében.
\par 20 És Menáhem adót vetett Izráelben a gazdagokra, ötven ezüst siklust minden egyes emberre, hogy azt Assiria királyának adja, és visszatért Assiria királya, és nem maradt ott az országban.
\par 21 Menáhemnek egyéb dolgai pedig és minden cselekedetei, vajjon nincsenek-é megírva az Izráel királyainak krónika-könyvében?
\par 22 És elaluvék Menáhem az õ atyáival, és uralkodék az õ fia, Pekája õ helyette.
\par 23 És Azáriának, Júda királyának ötvenedik esztendejében kezdett uralkodni Pekája, a Menáhem fia, Izráelen Samariában két esztendeig.
\par 24 És gonoszul cselekedék az Úr szemei elõtt, nem távozék el Jeroboámnak, a Nébát fiának bûneitõl, a ki vétekbe ejtette az Izráelt.
\par 25 És pártot ütött ellene Péka, a Remália fia, az õ hadnagya, és megölte õt Samariában, a király házának palotájában, Argóbbal és Arjéval és ötven emberrel a Gileádbeliek fiai közül, és megölte õt, és õ lett helyébe a király.
\par 26 Pekájának egyéb dolgai pedig és minden cselekedetei, ímé meg vannak írva az Izráel királyainak krónika-könyvében.
\par 27 Azáriának, Júda királyának ötvenkettedik esztendejében kezdett uralkodni Péka, a Remália fia, Izráelen Samariában húsz esztendeig.
\par 28 És gonoszul cselekedék az Úr szemei elõtt, nem távozék el Jeroboámnak, a Nébát fiának bûneitõl, a ki vétekbe ejtette az Izráelt.
\par 29 Pékának, az Izráel királyának idejében jött el Tiglát-Piléser, Assiria királya, és foglalta el Hijont, Abélt, Beth-Maakát, Jánoát, Kédest, Hásort, Gileádot, Galileát és a Nafthali egész földét, és hurczolta el õket, Assiriába.
\par 30 De Hósea, az Ela fia, pártot ütött Péka, a Remália fia ellen, és levágta és megölte õt, és õ lett helyébe a király Jótámnak, az Uzzia fiának huszadik esztendejében.
\par 31 Pékának egyéb dolgai pedig és minden cselekedetei, ímé meg vannak írva az Izráel királyainak krónika-könyvében.
\par 32 Pékának, a Remália fiának, az Izráel királyának második esztendejében kezdett uralkodni Jótám, Uzziának, a Júda királyának fia.
\par 33 Huszonöt esztendõs volt, a mikor uralkodni kezdett, és tizenhat esztendeig uralkodott Jeruzsálemben. Az õ anyjának Jérusa volt a neve, a Sádók leánya.
\par 34 És kedves dolgot cselekedék az Úr szemei elõtt, mind a szerint, a mint az õ atyja, Uzzia cselekedék;
\par 35 Csak a magaslatokat nem rontották le; még ott áldozott és tömjénezett a nép a magaslatokon. Õ építette meg az Úr házának felsõ kapuját.
\par 36 Jótámnak egyéb dolgai pedig és minden cselekedetei, vajjon nincsenek-é megírva a Júda királyainak krónika-könyvében?
\par 37 Ebben az idõben kezdte az Úr ráküldeni Júdára Réczint, Siria királyát, és Pékát, a Remália fiát.
\par 38 És elaluvék Jótám az õ atyáival, és eltemetteték az õ atyáival, Dávidnak, az õ atyjának városában, és Akház, az õ fia, uralkodék õ helyette.

\chapter{16}

\par 1 Pékának, a Remália fiának tizenhetedik esztendejében kezdett uralkodni Akház, Jótámnak, a Júda királyának fia.
\par 2 Húsz esztendõs volt Akház, mikor uralkodni kezdett, és tizenhat esztendeig uralkodott Jeruzsálemben; de nem azt cselekedé, a mi az Úrnak, az õ Istenének tetszett volna, a mint Dávid, az õ atyja,
\par 3 Hanem az Izráel királyainak útjokon járt, még az õ fiát is átvitte a tûzön, a pogányok útálatosságai szerint, a kiket az Úr az Izráel fiai elõtt kiûzött volt.
\par 4 És ott áldozott és tömjénezett a magaslatokon és a halmokon és minden zöld fa alatt.
\par 5 Abban az idõben jött fel Réczin, Siria királya, és Péka, a Remália fia, az Izráel királya, Jeruzsálemet vívatni, és körülzárták Akházt, de nem tudták  legyõzni.
\par 6 Akkor nyerte vissza Réczin, Siria királya Elátot a Siriabelieknek, és irtotta ki a zsidókat Elátból; és Siriabeliek jövének Elátba, és ott laknak mind e mai napig.
\par 7 És követeket küldött Akház Tiglát-Piléserhez, Assiria királyához, ezt izenvén: Te szolgád és a te fiad vagyok; jõjj fel és szabadíts meg engem Siria  királyának kezébõl és az Izráel királyának kezébõl, a kik reám támadtak.
\par 8 És vevé Akház az ezüstöt és aranyat, és a mely találtaték az Úr házában és a király házának kincsei között, és ajándékba küldé Assiria királyának.
\par 9 És engedett néki Assiria királya, és feljövén Assiria királya Damaskus ellen, bevette azt, és a benne levõket Kirbe hurczolta; Réczint pedig megölte.
\par 10 És eleibe ment Akház Tiglát-Pilésernek, Assiria királyának Damaskusba, és mikor meglátta Akház király azt az oltárt, a mely Damaskusban volt, elküldte Akház király annak az oltárnak hasonlatosságát és képét minden felszerelésével Uriás paphoz.
\par 11 És megépítette Uriás pap az oltárt arra a formára, a melyet Akház király Damaskusból küldött, és elkészíté azt Uriás pap, mire Akház király visszajött Damaskusból.
\par 12 És mikor megjött a király Damaskusból és meglátta a király az oltárt: az oltárhoz ment a király és áldozott rajta,
\par 13 És meggyújtotta az õ égõáldozatját és ételáldozatját, és áldozott italáldozattal is, és az õ hálaáldozatainak vérét elhintette az oltár körül.
\par 14 És a rézoltárt, a mely az Úr elõtt állott, elviteté a ház elõl, hogy ne álljon az õ oltára és az Úr háza között, és helyhezteté azt az oltár szegletéhez észak felõl.
\par 15 És megparancsolta Akház király Uriás papnak, mondván: E nagy oltáron gyújtsd meg a reggeli égõáldozatot és az esteli ételáldozatot, a király égõáldozatját is az õ ételáldozatával együtt, az egész föld népének is mind égõáldozatát, mind ételáldozatját, mind italáldozatját, és az égõáldozat minden vérét és egyéb áldozatnak minden vérét a körül hintsed el; a rézoltár felõl pedig még majd gondolkozom.
\par 16 És Uriás pap mind a szerint cselekedék, a mint Akház király megparancsolta.
\par 17 És letörte Akház király a talpak oldalait, és lehányta azokról a mosdómedenczéket; a tengert is ledobta a rézökrökrõl, a melyeken állott, és kõtalapzatra tette.
\par 18 És áthelyezte a szombati hajlékot, a melyet a házban építettek, és a király külsõ bejáró helyét az Úr házában, Assiria királya miatt.
\par 19 Akháznak egyéb dolgai pedig, a melyeket cselekedett, vajjon nincsenek-é megírva a Júda királyainak krónika-könyvében?
\par 20 És elaluvék Akház az õ atyáival, és eltemetteték az õ atyáival a Dávid városában, és uralkodék õ helyette az õ fia,  Ezékiás.

\chapter{17}

\par 1 Akháznak, a Júda királyának tizenkettedik esztendejében kezdett uralkodni Hóseás, az Ela fia, Samariában Izráelen, kilencz esztendeig.
\par 2 És gonoszul cselekedék az Úr szemei elõtt; de nem annyira, mint Izráel ama királyai, a kik elõtte éltek.
\par 3 Ez ellen jött Salmanassár, Assiria királya, és szolgájává lett Hóseás, és adót fizetett néki.
\par 4 De mikor Assiria királya megtudta, hogy Hóseás pártot ütött ellene és követeket küldött Suához, Égyiptom királyához, az évenkénti adót pedig nem küldötte meg Assiria királyának: elfogatta õt Assiria királya és börtönbe vetette.
\par 5 És felméne az assiriai király az egész ország ellen, és feljövén Samaria ellen, azt három esztendeig ostromolta.
\par 6 És Hóseásnak kilenczedik esztendejében bevette Assiria királya Samariát, és elhurczolta az Izráelt Assiriába; és Halában és Háborban a Gózán folyó mellett, és a Médek városaiban telepítette le õket.
\par 7 Ez pedig azért történt, mert az Izráel fiai vétkeztek az Úr ellen, az õ Istenök ellen, a ki õket kihozta Égyiptom földébõl, a Faraónak, az égyiptomi királynak kezébõl, és idegen isteneket tiszteltek.
\par 8 És jártak a pogányok szerzésekben, a kiket az Úr kiûzött volt az Izráel fiai elõl, és úgy cselekedtek, mint az Izráel királyai.
\par 9 És alattomban oly dolgokat cselekedtek az Izráel fiai, a melyek ellenére voltak az Úrnak, az õ Istenöknek, és építettek magoknak magaslatokat minden városaikban, az õrtornyoktól a kerített városokig.
\par 10 És emeltek magoknak faragott képeket és Aserákat minden magas halmokon és minden zöld fa alatt.
\par 11 És ott tömjéneztek minden magaslaton, mint a pogányok, a kiket az Úr kiûzött volt elõlök, és gonosz dolgokat cselekedtek, a melyekkel az Urat haragra indították.
\par 12 És a bálványoknak szolgáltak, a kik felõl az Úr azt parancsolta volt nékik: Ezt ne cselekedjétek.
\par 13 És mikor bizonyságot tett az Úr Izráelben és Júdában minden prófétája és minden látnoka által, ezt mondván: Térjetek meg  a ti gonosz útaitokról, és õrizzétek meg az én parancsolataimat és rendeléseimet, az egész törvény szerint, a melyet parancsoltam a ti atyáitoknak; és a melyet ti hozzátok küldöttem az én szolgáim, a próféták által:
\par 14 Akkor nem engedelmeskedtek, hanem megkeményítették az õ nyakokat atyáik makacssága szerint, a kik nem hittek az Úrban, az õ Istenökben.
\par 15 Sõt megvetették az õ rendeléseit és szövetségét, a melyet atyáikkal kötött, és az õ bizonyságtételeit, a melyekkel meghintette õket, és hiábavalóságot követvén, magok is hiábavalókká lettek, és a körültök való pogányok után indultak, a kik felõl pedig azt parancsolta nékik az Úr, hogy azokat ne utánozzák.
\par 16 És elhagyták az Úrnak, az õ Istenöknek minden parancsolatját, és öntött képeket csináltak magoknak, két borjút és készítettek Aserát, és meghajlottak az ég minden seregei elõtt, és a Baált tisztelték.
\par 17 És átvitték fiaikat és leányaikat a tûzön, jövendõket mondtak és varázslást ûztek, és magokat  teljesen eladták a bûnnek az Úr bosszantására.
\par 18 Azért igen megharaguvék az Úr Izráelre, és elveté õt színe elõl; és semmi nem maradt meg, hanem csak a Júda  nemzetsége egyedül.
\par 19 Sõt még a Júda sem õrizte meg az Úrnak, az õ Istenének parancsolatait, hanem az Izráel szerzésiben jártak, úgy cselekedvén, mint azok.
\par 20 Elidegenült azért az Úr az egész Izráel magvától, és megsanyargatta õket, és a kóborlók kezébe adta õket, mígnem mind elvetette az õ szemei elõl;
\par 21 Mert elszakasztotta Izráelt a Dávid házától, és királylyá tevék Jeroboámot, a Nébát fiát. És Jeroboám elcsalta az Izráelt az Úrtól, és igen nagy vétekbe ejtette õket.
\par 22 És az Izráel fiai követték Jeroboám minden bûnét, a melyeket elkövetett, és nem távoztak el azoktól;
\par 23 Míglen elveté az Úr az õ színe elõl az Izráelt, a mint megmondotta minden õ szolgái, a próféták által. Így hurczoltatott el fogságra az Izráel az õ földébõl Assiriába mind e mai napig.
\par 24 És más népet telepített be Assiria királya Babilóniából, Kutából, Avából, Hámátból és Sefárvaimból, és beszállítá õket Samaria városaiba az Izráel fiai helyett, a kik birtokba vették Samariát, és annak városaiban laktak.
\par 25 És történt ott lakásuk kezdetén, hogy nem az Urat félték, és az Úr oroszlánokat bocsátott reájok, a melyek többet megöltek közülök.
\par 26 Mondának azért Assiria királyának: A pogányok, a kiket ide hoztál és Samaria városaiba telepítettél, nem tudják e föld Istene tiszteletének módját, azért oroszlánokat bocsátott az rájok, a melyek megölik õket, mert nem tudják e föld Istene tiszteletének módját.
\par 27 Parancsola azért Assiria királya, mondván: Vigyetek oda egyet a papok közül, a kiket onnét elhoztatok, a ki menjen el és lakjék ott, és tanítsa meg õket ama föld Istene tiszteletének módjára.
\par 28 És elméne egy a papok közül, a kiket Samariából elvittek volt, és lakék Béthelben, és megtanította õket, hogy mikép tiszteljék az Urat.
\par 29 De azért mindenik nép külön isteneket csinált magának, és behelyezték azokat a magaslatok házaiba, a melyeket a Samaritánusok építettek, mindenik nemzetség a maga városában, a melyben lakott.
\par 30 A Babilonból való férfiak csinálták a Sukkót-Benótot, a Kutból való férfiak csinálták a Nérgelt, és a Hámátból való férfiak az Asimát csinálták;
\par 31 A Háveusok pedig a Nibeházt és a Tartákot csinálták, míg a Sefárvaimbeliek tûzzel égették meg az õ magzatjaikat az Adraméleknek és Anaméleknek, a Sefárvaimbeliek isteneinek.
\par 32 De miután az Urat is tisztelték, a magaslatokra papokat állítottak a maguk tömegébõl, a kik õ érettök áldoztak a magaslatok házaiban.
\par 33 Így tisztelték az Urat és szolgálták az õ isteneiket is, ama népek szokása szerint, a kik közül elhozták volt õket.
\par 34 És mind e mai napig az õ régi szokásuk szerint cselekesznek; nem tisztelik igazán az Urat, és nem cselekesznek az õ rendeléseik és szokásuk szerint,sem pedig ama törvény és parancsolat szerint, a melyet az Úr parancsolt a Jákób fiainak, a kiknek az Izráel nevet adta.
\par 35 És a kikkel az Úr szövetséget kötött, és megparancsolta nékik, mondván: Ne tiszteljetek idegen isteneket, és ne imádjátok õket, és ne szolgáljatok és ne áldozzatok nékik;
\par 36 Hanem csak az Urat tiszteljétek, õt imádjátok és néki áldozzatok, a ki titeket kihozott az Égyiptom földébõl nagy erõvel és kinyújtott karral.
\par 37 És õrizzétek meg az õ rendeléseit, ítéleteit, törvényét és parancsolatját, a melyeket megírt néktek, azokat cselekedvén minden idõben, és idegen isteneket ne tiszteljetek;
\par 38 És el ne felejtkezzetek a kötésrõl, a melyet veletek tettem, és ne tiszteljetek idegen isteneket;
\par 39 Hanem az Urat tiszteljétek, a ti Isteneteket, és õ megszabadít titeket minden ellenségetek kezébõl.
\par 40 De ezek nem engedelmeskedtek, hanem elõbbi szokásaik szerint cselekedtek.
\par 41 Így tisztelték ezek a pogányok az Urat és szolgálták az õ bálványaikat, és így cselekedtek az õ fiaik és unokáik is, a mint az õ eleik cselekedtek, mind e mai napig.

\chapter{18}

\par 1 És Hóseásnak, az Ela fiának, az Izráel királyának harmadik esztendejében kezdett uralkodni Ezékiás, Akháznak, a Júda királyának fia.
\par 2 Huszonöt esztendõs volt, mikor uralkodni kezdett, és huszonkilencz esztendeig urakodott Jeruzsálemben; anyjának neve Abi, a Zakariás leánya.
\par 3 És kedves dolgokat cselekedék az Úr szemei elõtt, a mint az õ atyja, Dávid cselekedett volt.
\par 4 Õ rontotta le a magaslatokat, törte el az oszlopokat, és vágta ki az Aserát, és törte össze az érczkígyót is, a melyet Mózes  csinált; mert mind az ideig az Izráel fiai jóillatot tettek annak, és nevezék azt Nékhustánnak.
\par 5 Egyedül az Úrban, Izráel Istenében bízott, és õ utána nem volt hozzá hasonló Júda minden királyai között sem azok között, a kik õ  elõtte voltak.
\par 6 Mert az Úrhoz ragaszkodott, és el nem hajlott õ tõle, és megõrizte az õ parancsolatait, a melyeket az Úr Mózesnek parancsolt vala.
\par 7 És vele volt az Úr mindenütt és a hova csak ment, elõmenetele volt. És elszakadt Assiria királyától, és nem szolgált néki.
\par 8 Megverte a Filiszteusokat is egész Gázáig és határukat, az õrtornyoktól a kerített városokig.
\par 9 Ezékiás királynak negyedik esztendejében - amely Hóseásnak, az Ela fiának, az Izráel királyának hetedik esztendeje - feljött Salmanassár, Assiria királya Samaria ellen, és megszállotta azt;
\par 10 És elfoglalta három esztendõ mulva, Ezékiás hatodik esztendejében - ez Hóseásnak, az Izráel királyának kilenczedik esztendeje - ekkor vették be Samariát.
\par 11 És elhurczolta Assiria királya az Izráelt Assiriába, és letelepíté Halába, Háborba, a Gózán folyó mellé, és a Médeusok városaiba;
\par 12 Azért, mert nem hallgattak az Úrnak, az õ Istenöknek szavára, hanem megszegték az õ szövetségét, mindazokat, a melyeket Mózes, az Úr szolgája parancsolt vala; sem nem hallgattak rájok, sem nem cselekedték azokat.
\par 13 Ezékiás királynak pedig tizennegyedik esztendejében feljött Sénakhérib, Assiria királya Júdának minden kerített városa ellen, és elfoglalta azokat.
\par 14 Akkor elküldött Ezékiás, Júda királya Assiria királyához Lákisba, ezt izenvén: Vétkeztem, térj el én rólam; a mit rám vetsz, elviselem! És Assiria királya Ezékiásra, Júda királyára háromszáz tálentom ezüstöt és harmincz tálentom aranyat vetett ki.
\par 15 És Ezékiás oda adott minden ezüstöt, a mely találtatott az Úr házában és a király házának kincsei között.
\par 16 Ugyanebben az idõben fosztotta meg Ezékiás az Úr templomának ajtajait és az ajtók félfáit, a melyeket Ezékiás, Júda királya maga boríttatott be, és oda adta azokat Assiria királyának.
\par 17 És Assiria királya mégis oda küldte Thartánt, Rabsárist és Rabsakét Lákisból Ezékiás királyhoz igen nagy haddal Jeruzsálembe, és felmenének és jutának Jeruzsálembe, és feljövén és oda érvén, megállának a felsõ halastó zsilipjénél, a mely a ruhafestõ útja mellett van.
\par 18 És kihívaták a királyt. És kiméne hozzájok Eliákim, a Hilkia fia, ki a király házának gondviselõje volt, és Sebna, az íródeák, és Joákh, az Asáf fia, az emlékíró.
\par 19 És monda nékik Rabsaké: Mondjátok meg Ezékiásnak: Ezt monda a király, Assiria nagy királya: Micsoda bizodalom ez, a melyben te bizakodtál?
\par 20 Azt mondád: csak szóbeszéd; tanács és erõ kell a hadakozáshoz; vajjon kihez bíztál, hogy fellázadtál ellenem?
\par 21 Ímé, ebben a törött nádszálban bizakodol, Égyiptomban, melyhez ha ki támaszkodik, bemegy az õ kezébe és általlikasztja? Ilyen a Faraó, Égyiptom királya mindenekhez, a kik õ hozzá bíznak!
\par 22 Vagy azt akarjátok nékem mondani: Mi az Úrban, a mi Istenünkben bízunk; hát nem ez-é az, a kinek magaslatait és oltárait lerontotta Ezékiás, és mind Júdának, mind Jeruzsálemnek megparancsolta: Ez oltár elõtt imádkozzatok Jeruzsálemben?
\par 23 Nosza, fogadj hát az én urammal, Assiria királyával: Én néked kétezer lovat adok, ha tudsz reájok adni annyi lovast.
\par 24 Hogy verhetnél hát vissza az én uram legkisebb hadnagyai közül csak egyet is, és hogyan bízhatol Égyiptomhoz a szekerek és a lovagok miatt?
\par 25 Vajjon az Úr tudta nélkül jöttem-é fel e hely ellen, hogy ezt elveszessem? Az Úr mondotta nékem: Menj fel e föld ellen, és veszesd el azt!
\par 26 És monda Eliákim, a Hiklia fia, és Sebna és Joákh Rabsakénak: Beszélj, kérlek, a te szolgáiddal siriai nyelven, mert jól értjük; és ne beszélj velünk zsidóul e nép füle hallatára, a mely a kõfalon van.
\par 27 És felele nékik Rabsaké: Vajjon a te uradhoz, vagy te hozzád küldött-é engem az én uram, hogy elmondjam e dolgokat, avagy nem inkább e férfiakhoz-é, akik a kõfalon ülnek, hogy azután veletek együtt egyék meg a saját ganéjokat, és igyák meg a saját vizeletöket?
\par 28 És oda állott Rabsaké, és hangosan kezdett zsidóul beszélni, mondván: Halljátok meg a királynak, Assiria nagy királyának beszédit!
\par 29 Azt mondja a király: Meg ne csaljon titeket Ezékiás; mert nem szabadíthat meg titeket az õ kezébõl.
\par 30 És ne biztasson titeket Ezékiás az Úrral, ezt mondván: Kétség nélkül megszabadít minket az Úr, és nem adatik e város az assiria király kezébe.
\par 31 Ne hallgassatok Ezékiásra; mert így szól Assiria királya: Békéljetek meg velem, és jõjjetek ki hozzám, és kiki egyék az õ szõlõjébõl és fügefájáról, és igyék az õ kútjának vizébõl;
\par 32 Míg eljövök és elviszlek titeket a ti földetekhez hasonló földre, gabonás és boros földre, kenyeres és szõlõs földre, olajfás, és mézes földre, hogy élhessetek és meg ne halljatok. Ne higyjetek Ezékiásnak, mert félrevezet benneteket, mikor azt mondja: Az Úr megszabadít minket!
\par 33 Vajjon megszabadították-é a pogányok istenei, mindenik a maga földjét Assiria királyának kezébõl?
\par 34 Hol vannak Hámáthnak és Arphádnak istenei? Hol Sefárvaimnak, Hénának és Hivvának istenei? És Samariát is megszabadították-é az én kezembõl?
\par 35 Kicsoda a földön való minden istenek közül, a ki megszabadíthatta volna az õ földjét az én kezembõl, hogy az Úr is kiszabadíthatná Jeruzsálemet az én kezembõl?
\par 36 De a nép csak hallgatott, és nem felelt néki csak egy igét sem; mert a király megparancsolta volt, mondván: Ne feleljetek néki.
\par 37 És elment Eliákim, a Hiklia fia, a király házának gondviselõje, és Sebna, az íródeák, és Joákh, az Asáf fia, az emlékíró, Ezékiáshoz megszaggatott ruhában, és elmondák néki Rabsaké beszédit.

\chapter{19}

\par 1 Mikor pedig ezeket hallotta Ezékiás király, megszaggatta az õ ruháit, és zsákba öltözék, és bement az Úr házába.
\par 2 És elküldte Eliákimot, a király házának gondviselõjét, és Sebnát, az íródeákot, és a papoknak véneit, a kik zsákba öltöztek, Ésaiás prófétához, az Ámós fiához.
\par 3 És mondának néki: Ezt mondja Ezékiás: E nap nyomorúságnak, szidalmazásnak és káromlásnak napja; mert a fiak a szülésre jutottak, de nincs erõ szüléshez.
\par 4 Netalán az Úr, a te Istened meghallotta Rabsakénak minden beszédit, a kit elküldött az õ ura, Assiria királya, hogy szidalommal illesse az élõ Istent és káromló beszédekkel, a melyeket az Úr, a te Istened meghallott: könyörögj azért azokért, a kik még megmaradtak.
\par 5 És elmenének az Ezékiás király szolgái Ésaiáshoz.
\par 6 És monda nékik Ésaiás: Ezt mondjátok a ti uratoknak: Ezt mondja az Úr: Ne félj a beszédektõl, a melyeket hallottál, a melyekkel szidalmaztak engem Assiria királyának szolgái.
\par 7 Ímé én oly lelket adok belé, hogy hírt hallván, visszatér az õ földjére és fegyverrel  vágatom le õt az õ földjében.
\par 8 És mikor visszatért Rabsaké, Assiria királyát Libna ellen harczolva találta, mert meghallotta volt, hogy Lákisból elment.
\par 9 És hallván Tirháka felõl, a szerecsen király felõl, ezt mondván: Ímé kijött, hogy hadakozzék te ellened; visszafordult és követeket küldött Ezékiáshoz, ezt izenvén:
\par 10 Így szóljatok Ezékiásnak, a Júda királyának, mondván: Meg ne csaljon téged a te Istened, a kiben bízol, ezt mondván: Nem adatik Assiria királyának kezébe Jeruzsálem.
\par 11 Ímé hallottad, mit cselekedtek Assiria királyai minden országokkal, elvesztvén azokat; és te megszabadulhatsz-é?
\par 12 Vajjon megszabadították-é a pogányok istenei azokat, a kiket elvesztettek az én atyáim: Gózánt, Aránt, Résefet és az Eden fiait, a kik Thelasárban voltak?
\par 13 Hol van Hámát királya, Arphád királya és Sefárvaim város királya? Héna és Hivva?
\par 14 És elvevé Ezékiás a levelet a követek kezébõl, és elolvasá azt, és felment az Úr házába, és kiterjeszté azt Ezékiás az Úr elõtt;
\par 15 És imádkozék Ezékiás az Úr elõtt, és monda: Uram, Izráel Istene! a ki a Kérubok között lakol, te vagy egyedül e föld minden országainak Istene, te teremtetted a mennyet és a földet;
\par 16 Hajtsd hozzám, Uram, a te füledet és halld meg; nyisd fel Uram, a te szemeidet és lásd meg: és halld meg a Sénakhérib beszédét, a ki ide küldött, hogy szidalommal illesse az élõ Istent.
\par 17 Igaz, Uram, Assiria királyai elpusztították a pogányokat és azok országait.
\par 18 És az õ isteneiket a tûzbe hányták; de azok nem voltak istenek, hanem csak emberi kéz alkotásai, fa és kõ, azért vesztették el õket.
\par 19 Most azonban, mi Urunk, Istenünk, szabadíts meg, kérlek, minket az õ kezébõl, hogy megismerje e föld minden országa, hogy te, az Úr, vagy az egyetlen Isten!
\par 20 Akkor elküldött Ésaiás, az Ámós fia, Ezékiáshoz, ezt mondván: Azt mondja az Úr, Izráel Istene: A te könyörgésedet, Assiria királya, Sénakhérib felõl, meghallgattam.
\par 21 Ez az, a mit az Úr õ felõle mondott: Megutál téged és megcsúfol téged Sionnak szûz leánya, utánad fejét hajtogatja Jeruzsálem leánya.
\par 22 Kit szidalmaztál és kit gyaláztál? És ki ellen emelted fel a te szódat, vagy ki ellen emelted fel a te szemeidet a magasságba? Az Izráel szentje ellen!
\par 23 Követeid által gúnyoltad az Urat, és mondád: Szekereim sokaságával meghágom a hegyek magasságait, a Libánon oldalait, és levágom magas czédrusait, legfelségesebb cziprusfáit, és bemegyek csúcsának lakóhelyébe, kertes erdejébe.
\par 24 Ástam és ittam idegen vizeket, és kiszárítom lábaim talpával Égyiptom minden folyóvizét.
\par 25 Avagy nem hallottad? Régen megcsináltam, õs idõktõl elvégeztem ezt! Most csak véghezvittem, hogy puszta kõhalmokká döntsd össze az erõs városokat;
\par 26 És hogy a benne lakók erejökben megfogyatkozzanak, megrontassanak és megszégyenüljenek, és olyanok legyenek mint a mezõ füve, fiatal paréj, a háztetõ füve és mint a kalászhajtás elõtt elszáradt gabona.
\par 27 És ismerem a te ülésedet, és járásodat kelésedet, és ellenem való tombolásodat;
\par 28 A te ellenem való tombolásodért és a te elbizakodásodért, a mely felhatott füleimbe,  az én karikámat orrodba vetem, és zabolámat szádba, és visszaviszlek azon az úton, a melyen eljöttél.
\par 29 Te néked pedig, Ezékiás, legyen ez jeled: Ez esztendõben táplál a hulladék termése, a második esztendõben, a mi magától terem; de a harmadik esztendõben már vettek és arattok, szõlõket plántáltok és azok gyümölcsét eszitek.
\par 30 És a Júda házából a kiszabadult és a megmaradt gyökeret ver alól, és gyümölcsöt terem felül.
\par 31 Mert Jeruzsálembõl fog származni a maradék, és a megszabadult a Sion hegyérõl; a Seregek Urának buzgó szerelme cselekszi ezt!
\par 32 Azért azt mondja az Úr Assiria királya felõl: Be nem jõ e városba, és nyilat sem lõ bele, sem paizs nem ostromolja azt, sem sánczot nem ás mellette.
\par 33 Azon az úton, a melyen jött, tér vissza, de e városba be nem jõ, azt mondja az Úr.
\par 34 És megoltalmazom e várost, hogy megtartsam azt, én érettem és Dávidért, az én szolgámért.
\par 35 És azon az éjszakán kijött az Úr angyala, és levágott az Assiriabeli táborban száznyolczvanötezeret, és mikor jó reggel felkeltek, ímé mindenütt holttestek hevertek.
\par 36 És elindult, és elment, és visszafordult Sénakhérib, Assiria királya, és Ninivében maradt.
\par 37 És lõn, mikor õ a Nisróknak, az õ Istenének templomában imádkozék, Adramélek és Sarézer, az õ fiai, levágták õt fegyverrel: magok pedig elszaladtak az Ararát földébe, és az õ fia Esárhaddon uralkodék helyette.

\chapter{20}

\par 1 Ebben az idõben halálosan megbetegedett Ezékiás, és hozzá menvén Ésaiás próféta, az Ámós fia, monda néki: Azt mondja az Úr: Rendeld el házadat, mert meghalsz és nem élsz.
\par 2 Akkor arczczal a falhoz fordult, és könyörgött az Úrnak, mondván:
\par 3 Óh Uram, emlékezzél meg róla, hogy te elõtted hûséggel és tökéletes szívvel jártam, és hogy azt cselekedtem, a mi jó volt a te szemeid elõtt. És sírt Ezékiás nagy sírással.
\par 4 Azonban Ésaiás még alig ért a város közepére, mikor az Úr beszéde lõn õ hozzá, mondván:
\par 5 Menj vissza és mondd meg Ezékiásnak, az népem fejedelmének: Azt mondja az Úr, Dávidnak, a te atyádnak Istene: Meghallgattam a te imádságodat, láttam a te könyhullatásidat, ímé én meggyógyítlak téged, harmadnapra felmégy az Úr házába;
\par 6 És a te idõdet tizenöt esztendõvel meghosszabbítom, és megszabadítlak téged és e várost Assiria királyának kezébõl, és megoltalmazom e várost én érettem és Dávidért, az én szolgámért.
\par 7 És monda Ésaiás: Hozzatok egy kötés száraz fügét ide. És hozának, és azt a kelevényre kötötték, és meggyógyult.
\par 8 És mikor azt kérdezé Ezékiás Ésaiástól: Mi lesz a jele, hogy meggyógyít engem az Úr, és hogy harmadnapra felmehetek az Úr házába?
\par 9 Felele Ésaiás: Ez legyen jeled az Úrtól, hogy õ megcselekeszi ezt a dolgot, a melyrõl szólott néked: Elõremenjen-é az árnyék tíz grádicscsal, vagy visszatérjen-é tíz grádicscsal?
\par 10 És felele Ezékiás: Könnyû az árnyéknak tíz grádicscsal alábbszállani. Ne úgy, hanem menjen hátra az árnyék tíz grádicscsal.
\par 11 És könyörgött Ésaiás próféta az Úrhoz, és visszatéríté az árnyékot Akház napóráján, azokon a grádicsokon, a melyeken már aláment volt, tíz grádicscsal.
\par 12 Ebben az idõben küldött Berodákh Baladán, Baladánnak, a babilóniai királynak fia levelet és ajándékot Ezékiásnak; mert meghallotta, hogy Ezékiás beteg volt.
\par 13 És meghallgatá õket Ezékiás, és megmutatta nékik az õ egész kincsesházát, az ezüstöt, az aranyat, a fûszerszámokat, a drága kenetet és az õ fegyveres házát és mindent, a mi csak találtatott az õ kicstáraiban, és nem volt semmi az õ házában és egész birodalmában, a mit meg nem mutatott volna Ezékiás.
\par 14 Ekkor jött Ésaiás próféta Ezékiás királyhoz, és monda néki: Mit mondtak ezek a férfiak, és honnét jöttek hozzád? És felele Ezékiás: Messze földrõl jöttek, Babilóniából.
\par 15 És monda: Mit láttak a te házadban? Felele Ezékiás: Mindent láttak, a mi csak van az én házamban, és nem volt semmi az én tárházamban, a mit nékik meg ne mutattam volna.
\par 16 Akkor monda Ésaiás Ezékiásnak: Halld meg az Úrnak beszédét:
\par 17 Ímé eljõ az idõ, a mikor mindaz, a mi a te házadban van, és a mit eltettek a te atyáid e mai napig, elvitetik Babilóniába, és semmi sem marad meg, azt mondja az Úr.
\par 18 És a te fiaid közül is, a kik tõled származnak és születnek, elhurczoltatnak és udvariszolgák lesznek a babilóniai király udvarában.
\par 19 Ezékiás pedig monda Ésaiásnak: Jó az Úr beszéde, a melyet szólál: És monda: Nem merõ jóság-é, ha békesség és hûség lesz az én napjaimban?
\par 20 Ezékiásnak egyéb dolgai pedig és minden erõssége, és hogy miképen csinálta a tavat és vízcsöveket, a melyekkel a vizet a városba vezette, vajjon nincsenek-é megírva a Júda királyainak krónika-könyvében?
\par 21 És elaluvék Ezékiás az õ atyáival, és az õ fia, Manasse uralkodék helyette.

\chapter{21}

\par 1 Manasse tizenkét esztendõs volt, mikor uralkodni kezdett, és ötvenöt esztendeig uralkodott Jeruzsálemben, és az õ anyjának Hefsiba volt a neve.
\par 2 És gonoszul cselekedék az Úr szemei elõtt, a pogányok útálatossága szerint, a kiket az Úr kiûzött volt az Izráel fiai elõl:
\par 3 Mert újra megépítette a magaslatokat, a melyeket Ezékiás, az õ atyja lerontott, és oltárokat emelt a Baálnak, és állított Aserát,  mint a hogy Akháb, az Izráel királya cselekedett volt, és imádta az összes mennyei seregeket és azoknak szolgált.
\par 4 És oltárokat is épített az Úr házában, a mely felõl azt mondotta volt az Úr: Jeruzsálemben helyheztetem az én nevemet!
\par 5 Oltárokat épített az egész mennyei seregnek, az Úr házának mind a két pitvarában.
\par 6 És átvivé a fiát a tûzön, és igézést és jegymagyarázást ûzött és ördöngösöket és titokfejtõket  tartott; sok gonosz dolgot cselekedék az Úr szemei elõtt, hogy õt haragra ingerelje.
\par 7 És az Asera-bálványt, a melyet készített, bevitte abba a házba, a mely felõl azt mondotta volt az Úr Dávidnak és az õ fiának, Salamonnak: Ebben a házban és Jeruzsálemben, a melyet magamnak választottam Izráelnek minden nemzetségei közül, helyheztetem az én nevemet mindörökké;
\par 8 És ki nem mozdítom többé Izráel lábát errõl a földrõl, a melyet adtam az õ eleiknek, ha szorgalmatosan az én parancsolatim szerint cselekesznek, és az egész törvény szerint, a melyet nékik Mózes, az én szolgám parancsolt.
\par 9 Õk azonban nem engedelmeskedtek, mert tévelygésbe ejté õket Manasse, hogy még gonoszabbul viseljék magokat azoknál a pogányoknál, a kiket az Úr kivesztett az Izráel fiai elõl.
\par 10 Akkor szóla az Úr az õ szolgái, a próféták által, mondván:
\par 11 Mivelhogy Manasse, Júda királya ezeket az útálatosságokat cselekedte, gonoszabb dolgokat cselekedvén mindazoknál, a melyeket az õ elõtte való Emoreusok cselekedtek vala, a Júdát is vétekbe ejtette az õ bálványai által:
\par 12 Azért ezt mondja az Úr, Izráel Istene: Íme, én oly veszedelmet hozok Jeruzsálemre és Júdára, hogy mindenkinek, a ki azt hallja, megcsendül bele mind a két füle.
\par 13 És kiterjesztem Jeruzsálemre a Samaria mérõ-zsinórját és az Akháb házának mértékét;  és kitörlöm Jeruzsálemet, mint kitörlik a tálat, és kitörölve leborítják azt,
\par 14 És elhagyom az én örökségem maradékát, és adom õt ellenségei kezébe, és zsákmánya és ragadománya lesz minden ellenségeinek;
\par 15 Azért, mert gonoszul cselekedtek én elõttem, és engem haragra ingerlettek az õ atyáiknak Égyiptomból való kijövetelök napjától fogva, mind e mai napig.
\par 16 És Manasse nagyon sok ártatlan vért is ontott ki, úgy hogy Jeruzsálem minden felõl megtelt vele, azon a vétkén  kivül, a melylyel vétekbe ejtette Júdát, gonoszul cselekedvén az Úr szemei elõtt.
\par 17 Manassénak egyéb dolgai pedig és minden cselekedetei és az õ vétke, a melyet cselekedett, vajjon nincsenek-é megírva a Júda királyainak krónika-könyvében?
\par 18 És elaluvék Manasse az õ atyáival, és eltemetteték az õ háza mellett lévõ kertben, az Uzza kertjében, és az õ fia, Amon uralkodék helyette.
\par 19 Amon huszonkét esztendõs volt, mikor uralkodni kezdett, és két esztendeig uralkodott Jeruzsálemben; az õ anyjának neve Mésullémet volt, a Jóthabeli Hárus leánya.
\par 20 És gonoszul cselekedék az Úr szemei elõtt, a mint cselekedett Manasse, az õ atyja.
\par 21 És tökéletesen azon az úton járt, a melyen járt volt az õ atyja, és szolgált a bálványoknak, a kiknek szolgált volt atyja, és azokat imádta.
\par 22 És elhagyta az Urat, atyái Istenét, és nem járt az Úrnak útában.
\par 23 És pártot ütöttek Amon ellen a maga szolgái, és megölték a királyt az õ házában.
\par 24 De a föld népe levágta mindazokat, a kik pártot ütöttek volt Amon király ellen, és a föld népe királylyá tevé az õ fiát, Jósiást, helyette.
\par 25 Amonnak egyéb dolgai pedig, a melyeket cselekedett, vajjon nincsenek-é megírva a Júda királyainak krónika-könyvében?
\par 26 És eltemeték õt az õ sírjába, Uzza kertjében, és fia, Jósiás lett a király õ helyette.

\chapter{22}

\par 1 Nyolcz esztendõs volt Jósiás, mikor uralkodni kezdett, és harminczegy esztendeig uralkodott Jeruzsálemben; az õ anyjának neve Jédida, a Boskátból való Adaja leánya.
\par 2 És kedves dolgot cselekedék az Úr szemei elõtt, és járt az õ atyjának, Dávidnak minden útaiban, és nem tért el sem jobbra, sem balra.
\par 3 És történt Jósiás király tizennyolczadik esztendejében, elküldte a király Sáfánt, Asaliának, a Messullám fiának fiát, az íródeákot, az Úr házához, mondván:
\par 4 Menj fel Hilkiához, a fõpaphoz, és számlálják meg az Úr házába begyült pénzt, a melyet gyûjtöttek az ajtóõrizõk a néptõl.
\par 5 És adják azt az Úr házában való mívesek pallérainak kezébe, hogy adják a munkásoknak, a kik az Úrnak házán dolgoznak, hogy a háznak romlásait kijavítsák;
\par 6 Az ácsoknak, az építõknek és a kõmíveseknek, hogy fákat és faragott köveket vásároljanak a ház kijavítására.
\par 7 De a számadást nem kell tõlök venni a pénzrõl, a mely kezökbe adatik, mert õk azt becsülettel végzik.
\par 8 És monda Hilkia, a fõpap, Sáfánnak, az íródeáknak: Megtaláltam a törvénykönyvet az Úr házában. És Hilkia oda adta a könyvet Sáfánnak, hogy olvassa el azt.
\par 9 És elméne Sáfán, az íródeák, a királyhoz, és megvivé a királynak a választ, és monda: A te szolgáid egybeszedék a pénzt, a mely a házban találtatott, és oda adták azt az Úr házában munkálkodók pallérainak kezébe.
\par 10 És megmondá Sáfán, az íródeák, a királynak, mondván: Egy könyvet adott nékem Hilkia pap. És felolvasá azt Sáfán a király elõtt.
\par 11 Mikor pedig hallotta a király a törvény könyvének beszédit, megszaggatá az õ ruháit.
\par 12 És megparancsolta a király Hilkia papnak és Ahikámnak, a Sáfán fiának, és Akbórnak, a Mikája fiának, és Sáfánnak, az íródeáknak, és Asájának, a király szolgájának, mondván:
\par 13 Menjetek el, kérdezzétek meg az Urat én érettem és a népért és az egész Júdáért, e könyvnek beszédei felõl, a mely megtaláltatott; mert nagy az Úr haragja, mely felgerjedett ellenünk, mivel a mi atyáink nem engedelmeskedtek e könyv beszédeinek, hogy cselekedtek volna mindent úgy, a mint megíratott nékünk.
\par 14 És elméne Hilkia pap és Ahikám, Akbór, Sáfán és Asája Hulda próféta asszonyhoz, Sallumnak, a Tikva fiának - a ki Harhásnak, a ruhák õrizõjének fia volt - feleségéhez, a ki Jeruzsálem más részében lakott, és beszéltek vele.
\par 15 És monda nékik: Azt mondja az Úr, Izráel Istene: Mondjátok meg a férfiúnak, a ki titeket hozzám küldött;
\par 16 Ezt mondja az Úr: Ímé én veszedelmet hozok e helyre és e helyen lakozókra a könyv minden beszédei szerint, a melyet olvasott a Júda királya;
\par 17 Mert elhagytak engem, és idegen isteneknek áldoztak jóillattal, hogy engem haragra indítsanak az õ kezöknek minden csinálmányával: azért felgerjed az én haragom e hely ellen, és meg sem oltatik.
\par 18 A Júda királyának pedig, a ki elküldött titeket, hogy megkérdezzétek az Urat, ezt mondjátok: Azt mondja az Úr, Izráel Istene: Mivelhogy e beszédekre, a melyeket hallottál,
\par 19 Meglágyult a te szíved, és magadat megaláztad az Úr elõtt, hallván azokat, a miket e hely és az ezen helyen lakók ellen szólottam, hogy pusztulássá és átokká lesznek, és megszaggattad a te ruháidat, és sírtál elõttem; azért én is meghallgattalak, azt mondja az Úr.
\par 20 Azért ímé én téged a te atyáidhoz gyûjtelek, és a te sírodba békességgel visznek téged, és meg nem látják a te szemeid azt a nagy veszedelmet, a melyet én e helyre hozok. És megvitték a királynak a választ.

\chapter{23}

\par 1 És elküldött a király, és hozzá gyûltek Júdának és Jeruzsálemnek minden vénei.
\par 2 És felment a király az Úr házába, és Júdából minden férfi és Jeruzsálem minden lakosa vele volt; a papok, a próféták és az egész nép kicsinytõl fogva nagyig. És minden beszédét elolvasá elõttök a szövetség könyvének, a mely megtaláltatott az Úr házában.
\par 3 És oda állott a király az emelvényre, és kötést tõn az Úr elõtt, hogy õk az Urat akarják követni, és az õ parancsolatait, bizonyságtételeit és rendeléseit teljes szívbõl és lélekbõl megõrizni, és beteljesíteni e szövetség beszédeit, a melyek meg vannak írva abban a könyvben; és ráállott az egész nép a kötésre.
\par 4 És megparancsolá a király Hilkiának, a fõpapnak és a másod rendbeli papoknak és az ajtóõrizõknek, hogy az Úr templomából hordjanak ki minden edényt, a melyet a Baálnak, az Aserának és az egész mennyei seregnek csináltak, és megégeté azokat Jeruzsálemen kivül és Kidron völgyében, és azok hamvait  Béthelbe vivé.
\par 5 És kiirtá a bálvány papokat is, a kiket Júda királyai állítottak be, hogy a magaslatokon tömjénezzenek Júda városaiban és Jeruzsálem körül, és mindazokat is, a kik a Baálnak, a napnak, holdnak, égi jeleknek és az egész mennyei  seregnek tömjéneztek.
\par 6 Kivivé az Úr házából az Aserát is Jeruzsálemen kivül a Kidron patakja mellé, és megégeté azt a Kidron völgyében, és porrá zúzta, és annak porát a község temetõhelyére hinté.
\par 7 És lerontá a férfi paráznák házait, a melyek az Úr háza mellett voltak, és a melyekben az asszonyok kárpitokat szõttek az Aserának.
\par 8 És behozatta az összes papokat Júda városaiból, és megfertéztette a magaslatokat, a melyeken a papok tömjéneztek, Gebától egész Beersebáig, és lerontotta a kapuk mellett levõ magaslatokat is, a melyek Józsuénak, a város fejedelmének kapuja elõtt voltak balkéz felõl, a város kapujában.
\par 9 De a magaslatok papjai soha sem áldoztak az Úr oltárán Jeruzsálemben, hanem a kovásztalan kenyeret az õ atyjokfiai között ették.
\par 10 Megfertõztette a Tófetet, a Hinnom fiainak völgyében, hogy senki az õ fiát, vagy leányát át ne vihesse a tûzön Moloknak.
\par 11 És eltávolította a lovakat, a melyeket Júda királyai a napnak szenteltek az Úr házának bemenetelénél, a Nátán-Mélek udvari szolga háza mellett, a mely a Pharvarimban volt, és a nap szekereit tûzzel elégette.
\par 12 És az oltárokat, a melyek az Akház palotájának tetején voltak, a melyeket a Júda királyai csináltak, és azokat az oltárokat is, a melyeket Manasse csinált volt az Úr házának mindkét pitvarában, lerontotta a király, és lehányván, porukat a Kidron patakjába szóratta.
\par 13 Azokat a magaslatokat is megfertõztette a király, a melyek Jeruzsálem elõtt voltak az Olajfák hegyének jobbfelõl levõ oldalán, a melyeket még Salamon, az Izráel királya épített volt Astóretnek, a Sídonbeliek útálatosságának, és Kámósnak, a Moábiták útálatosságának, és Milkómnak, az Ammon fiai útálatosságának.
\par 14 És összetörte az oszlopokat és kivágatta az Aserákat, és azok helyeit embercsontokkal töltötte meg.
\par 15 Még az oltárt is, a mely Béthelben volt, a magaslatot, a melyet Jeroboám, a Nébát fia csinált, a ki vétekbe ejté az Izráelt, - azt az oltárt is és azt a magaslatot is letörte, és a magaslatot felégette, porrá tette, és az Aserát megégette.
\par 16 És körültekintett Jósiás, és meglátta a sírokat, a melyek ott a hegyen voltak, és elküldött és elhozatta a csontokat a sírokból, és megégette az oltáron, és megfertõztette azt az Úr beszéde szerint, a melyet mondott volt az Isten embere, a ki e dolgot megjövendölte volt.
\par 17 És monda: Miféle síremlék ez, a melyet látok? És felelének néki a város férfiai: Az Isten emberének sírja ez, a ki Júdából jött volt és mindezeket megjövendölte a Béthelbeli oltár felõl, a miket most cselekedtél.
\par 18 És monda: Hagyjatok békét néki, és senki meg ne mozdítsa az õ tetemeit. És megmentették az õ tetemeit annak a prófétának a tetemeivel együtt, a ki Samariából jött volt.
\par 19 És lerontotta Jósiás a magaslatok minden házát is, a melyek Samária városaiban voltak, a melyeket az Izráel királyai csináltak volt, hogy az Urat haragra ingereljék, és épen úgy cselekedett azokkal mindenekben, a mint Béthelben cselekedett.
\par 20 És megáldozta a magaslatok összes papjait, a kik ott voltak, az oltárokon, és emberi csontokat égetett  meg azokon, és úgy tért vissza Jeruzsálembe.
\par 21 És parancsolt a király az egész népnek, és monda: Ünnepeljetek páskhát az Úrnak, a ti Istenteknek, a mint meg van írva e szövetség könyvében;
\par 22 Mert nem szereztetett olyan páskha a birák idejétõl fogva, a kik az Izráelt ítélték, sem pedig az Izráel és a Júda királyainak minden idejében,
\par 23 Hanem csak Jósiás király tizennyolczadik esztendejében szereztetett ilyen páskha az Úrnak Jeruzsálemben.
\par 24 És kivesztette Jósiás király az ördöngösöket és a titokfejtõket, a teráfokat és a bálványokat, és mindazokat az útálatosságokat, a melyek láttattak Júda földjén és Jeruzsálemben, hogy helyreállítsa a törvény beszédit, a melyek meg valának írva a könyvben, a melyet  Hilkia pap talált meg az Úr házában.
\par 25 Nem is volt õ hozzá hasonló király õ elõtte, a ki úgy megtért volna az Úrhoz teljes szívébõl és teljes lelkébõl és teljes erejébõl, Mózesnek minden törvénye szerint; de utána sem támadott hozzá hasonló.
\par 26 De az Úr még sem szünt meg az õ megbúsult nagy haragjától, a melylyel megharagudott volt Júdára mindazokért a bosszantásokért, a melyekkel bosszantotta õt Manasse.
\par 27 Mert azt mondá az Úr: Júdát is elvetem szemem elõl, mint ahogy az Izráelt elvetettem, és megútálom ezt a várost, a melyet választottam, Jeruzsálemet, és ezt a házat is,  a melyrõl azt mondottam volt: Ott legyen az én nevem!
\par 28 Jósiásnak egyéb dolgai pedig és minden cselekedetei, vajjon nincsenek-é megírva a Júda királyainak krónika-könyvében?
\par 29 Az õ idejében jött fel Faraó-Nékó, az égyiptomi király Assiria királya ellen az Eufrátes folyóvize mellé. És Jósiás király eleibe ment, de az megölte õt Megiddóban, a mint meglátta õt.
\par 30 És az õ szolgái szekérre tévén õt, halva vitték el Megiddóból és Jeruzsálembe hozván, eltemeték õt az õ sírboltjába. A föld népe pedig vevé Joákházt,  a Jósiás fiát, és felkenvén õt, királylyá tevé az õ atyja helyett.
\par 31 Huszonhárom esztendõs volt Joákház, mikor uralkodni kezdett, és három hónapig uralkodott Jeruzsálemben, és az õ anyjának neve Hamutál, a Libnabeli Jeremiás leánya.
\par 32 És gonoszul cselekedék az Úr szemei elõtt mind a szerint, a mint az õ atyái cselekedtek volt.
\par 33 De Faraó-Nékó bilincsekbe verte õt Riblában, Hámát földjén, a mikor Jeruzsálemben királylyá lett; és az országra adót vetett, száz tálentom ezüstöt és egy tálentom aranyat.
\par 34 És Faraó-Nékó Eliákimot, a Jósiás fiát tette királylyá, az õ atyja, Jósiás helyett, és nevét Joákimra változtatta; Joákházt pedig magával vivé, és az Égyiptomba megérkezvén, ott hala meg.
\par 35 Az ezüstöt és az aranyat megadta ugyan Joákim a Faraónak, de az országot sarczoltatta meg, hogy megadhassa az ezüstöt a Faraó parancsolata szerint; a föld népe közül mindenkitõl az õ értéke szerint hajtott be ezüstöt és aranyat, hogy Faraó-Nékónak adja.
\par 36 Joákim huszonöt esztendõs volt, mikor uralkodni kezdett, és tizenegy esztendeig uralkodott Jeruzsálemben; az õ anyjának neve Zébuda, a Rúmabeli Pedája leánya.
\par 37 És gonoszul cselekedék az Úr szemei elõtt mind a szerint, a mint az õ atyái cselekedtek.

\chapter{24}

\par 1 Joákim idejében jött fel Nabukodonozor, Babilónia királya, és Joákim szolgája lett három esztendeig; de azután elfordult és elpártolt tõle.
\par 2 És ráküldé az Úr a Káldeusok seregeit, a Siriabeliek seregeit, a Moábiták seregeit és az Ammon fiainak seregeit, és ráküldte õket Júdára, hogy elveszessék õt az Úr beszéde szerint, a melyet szólott szolgái, a próféták által.
\par 3 Csak az Úr beszéde szerint történt ez Júdán, hogy elvesse õt a maga orczája elõl Manasse bûneiért, mind a szerint, a mint cselekedett vala;
\par 4 És az ártatlan vérért is, a melyet kiontott, és hogy betöltötte volt Jeruzsálemet ártatlan vérrel; ezért nem akart az Úr megbocsátani néki.
\par 5 Joákimnak egyéb dolgai pedig és minden cselekedetei, vajjon nincsenek-é megírva a Júda királyainak krónika-könyvében?
\par 6 És elaluvék Joákim az õ atyáival, és az õ fia, Joákin uralkodék helyette,
\par 7 És Égyiptom királya többé nem jött ki az õ földébõl; mert Babilónia királya mindent elvett, a mi csak az égyiptomi királyé volt, Égyiptom folyóvizétõl az Eufrátes folyóvizéig.
\par 8 Tizennyolcz esztendõs volt Joákin, mikor királylyá lett, és három hónapig uralkodott Jeruzsálemben. Az õ anyjának neve Nékhusta, a Jeruzsálembeli Elnatán leánya.
\par 9 És gonoszul cselekedék az Úr szemei elõtt mind a szerint, a mint az õ atyja cselekedett volt.
\par 10 Ebben az idõben jöttek fel Nabukodonozornak, Babilónia királyának szolgái Jeruzsálem ellen, és szállották meg a várost.
\par 11 Maga Nabukodonozor, Babilónia királya is feljött a város ellen, a melyet már az õ szolgái megszállottak volt.
\par 12 És kiment Joákin, a Júda királya Babilónia királyához az õ anyjával, szolgáival, hadnagyaival és udvariszolgáival együtt, de fogságra vetette õt Babilónia királya az õ uralkodásának nyolczadik esztendejében.
\par 13 És elvitte onnét az Úr házának minden kincsét és a király házának kincsét, és összevagdalt minden arany edényt, a melyet Salamon, az Izráel királya csináltatott volt az Úr templomában, a mint az Úr megmondotta volt.
\par 14 És elhurczolta az egész Jeruzsálemet, összes fejedelmeit és minden vitézeit, tízezer foglyot, az összes mesterembereket és lakatosokat, úgy hogy a föld szegény népén kivül senki sem maradt ott.
\par 15 És elhurczolta Joákint is Babilóniába, és a király anyját, és a király feleségeit és udvariszolgáit, és az ország erõs vitézeit fogságba hurczolta Jeruzsálembõl Babilóniába;
\par 16 És az összes elõkelõ férfiakat, hétezeret, és a mesterembereket és a lakatosokat, ezeret, és az összes erõs, harczra termett férfiakat fogva vitte Babilóniába Babilónia királya.
\par 17 És Babilónia királya Mattaniát, nagybátyját, tette õ helyette királylyá, és nevét Sédékiásra változtatta.
\par 18 Huszonegy esztendõs volt Sédékiás, mikor uralkodni kezdett, és tizenegy esztendeig uralkodott Jeruzsálemben; az õ anyjának neve Hamutál, a Libnából való Jeremiás leánya.
\par 19 És gonoszul cselekedék az Úr szemei elõtt mind a szerint, a mint Joákim cselekedett volt;
\par 20 Mert az Úr haragja miatt történt ez így Jeruzsálemmel és Júdával, míg csak el nem vetette õket az õ orczája elõl. Sédékiás azonban elpártolt Babilónia királyától.

\chapter{25}

\par 1 És történt az õ uralkodásának kilenczedik esztendejében, a tizedik hónapban, és annak tizedik napján, hogy feljött Nabukodonozor, Babilónia királya minden õ seregével Jeruzsálem ellen, és táborba szállott ellene, és köröskörül ostromtornyokat építettek ellene.
\par 2 És megszállva tartatott a város Sédékiás király tizenegyedik esztendejéig;
\par 3 De a negyedik hónap kilenczedik napján akkora inség lett a városban, hogy nem volt mit enni a föld népének.
\par 4 És betöretett a város, és a harczosok mind futni kezdtek éjjel a kettõs kõfal között levõ kapu útján, a mely a király kertje mellett van; a Káldeusok pedig ott táboroztak a város körül. És a király is elfutott a puszta útján.
\par 5 De a Káldeusok hada ûzõbe vette a királyt, és utólérték õt Jerikhó mezején, és egész serege szétszóródott mellõle.
\par 6 És elfogták a királyt, és elvitték õt Babilónia királyához Riblába, a hol ítéletet tartottak fölötte.
\par 7 És Sédékiás fiait saját szeme láttára vágták le; Sédékiás szemeit pedig megvakították, és lánczokba verve vitték el õt Babilóniába.
\par 8 És az ötödik hónap hetedik napján - ez a Nabukodonozor, babilóniai király uralkodásának tizenkilenczedik esztendeje - feljött Nabuzár-Adán, a vitézek hadnagya, Babilónia királyának szolgája Jeruzsálembe;
\par 9 És felgyújtotta az Úr házát és a király házát, és Jeruzsálem összes házait és mind a nagy palotákat felégette tûzzel.
\par 10 És Jeruzsálem kõfalait köröskörül lerombolta a Káldeusok serege, a mely a vitézek hadnagyával volt.
\par 11 A többi népet pedig, a mely a városban még megmaradt volt, és azokat, a kik Babilónia királyához hajlottak, és a többi népet mind elhurczolta Nabuzár-Adán, a vitézek hadnagya.
\par 12 A föld népének csak a szegényébõl hagyott ott a vitézek hadnagya szõlõmíveseket és szántó-vetõ embereket.
\par 13 És a rézoszlopokat, a melyek az Úr házában voltak, és a mosdómedenczék talpait és a réztengert, a mely az Úr házában volt, összetörték a Káldeusok, és azok rezét Babilóniába vitték.
\par 14 Elvitték a fazekakat is, a lapátokat, a késeket, temjénezõket, és minden, szolgálatra rendelt, egyéb rézedényeket.
\par 15 És elvitte a vitézek hadnagya a serpenyõket, a medenczéket, a melyek közül némelyek aranyból, némelyek pedig ezüstbõl voltak,
\par 16 A két rézoszlopot, a réztengert és a mosdómedenczék talpait, a melyeket Salamon csinált volt az Úr házában; megmérhetetlen volt mindezeknek az edényeknek a reze.
\par 17 Az egyik oszlop magassága tizennyolcz sing volt, és egy rézgömb volt rajta és a gömb három sing magas volt, a gömbön köröskörül hálózat és gránátalmák mind érczbõl, és ugyanilyen volt a másik oszlop is a hálózattal együtt.
\par 18 És elhurczolta a vitézek hadnagya Serája papot is, az elsõ rendbõl és Sofóniás papot, a második rendbõl és a három ajtónállót,
\par 19 És a városból elvitt egy fõembert, ki a hadakozó férfiak elõljáró hadnagya volt, és öt férfiút, a kik a király körül forgolódtak volt, a kik találtattak a városban, és a sereg hadnagyának íródeákját, a ki sereget gyûjt vala a föld népe közül, és hatvan férfiakat a föld népe közül, a kik ott találtattak a városban.
\par 20 És vevé õket Nabuzár-Adán, a vitézek hadnagya, és elvitte Babilónia királyához Riblába,
\par 21 És levágta õket Babilóna királya, és megölte Riblában a Hámát földjén. És így viteték el Júda az õ földjérõl.
\par 22 A Júda földjén megmaradt népnek pedig, a melyet meghagyott Nabukodonozor, Babilónia királya, Gedáliát, Ahikámnak, a Sáfán fiának fiát, rendelte tiszttartóul.
\par 23 Mikor pedig meghallották a seregek hadnagyai mind, és az õ embereik, hogy Babilónia királya Gedáliát tette tiszttartóvá, elmentek Gedáliához Mispába, Izmáel, a Nétánia fia, és Johanán, a Kareáh fia, és Serája, a nétofáti Tánhumet fia, és Jahazánia, Maakáti fia, õk és az õ embereik;
\par 24 És megesküdött nékik Gedália és az õ embereiknek, és monda nékik: Ne féljetek a Káldeusoknak való szolgálattól. Maradjatok az országban, és szolgáljatok Babilónia királyának, és jó dolgotok lesz.
\par 25 A hetedik hónapban azonban elment Izmáel, Nétániának, az Elisáma fiának fia, a ki királyi magból volt, és vele tíz férfiú, és megölték Gedáliát és meghalt; és a Zsidókat és a Káldeusokat, a kik õ vele voltak Mispában.
\par 26 És felkelt az egész nép kicsinytõl nagyig és a seregek hadnagyai, és elmentek Égyiptomba; mert féltek a Káldeusoktól.
\par 27 És lõn a harminczhetedik esztendõben, Joákinnak, a Júda királyának fogságba hurczoltatása után, a tizenkettedik hónap huszonhetedik napján, kivette Evil-Merodák, Babilónia királya, az õ uralkodásának elsõ esztendejében Joákint, Júda királyát a fogházból;
\par 28 És kegyesen beszélt vele, és feljebb tette az õ székét a többi királyok székeinél, a kik nála voltak Babilóniában;
\par 29 És kicserélte fogsága ruháit, és mindenkor nála volt étele életének minden idejében.
\par 30 És mindenkor kijárt az õ része, a melyet a király adott néki napról-napra életének minden idejében.


\end{document}