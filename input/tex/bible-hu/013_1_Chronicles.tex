\begin{document}

\title{1 Chronicles}


\chapter{1}

\par 1 Ádám, Séth, Énós.
\par 2 Kénán, Mahalálél, Járed.
\par 3 Énókh, Methuséláh, Lámekh.
\par 4 Noé, Sém, Khám és Jáfet.
\par 5 Jáfetnek fiai: Gómer, Mágog, Madai, Jáván, Thubál, Mésekh és Thirász.
\par 6 A Gómer fiai pedig: Askhenáz, Rifáth és Tógármah.
\par 7 Jávánnak pedig fiai: Elisah, Thársis, Kitthim és Dodánim.
\par 8 Khámnak fiai: Khús, Miczráim, Pút és Kanaán.
\par 9 Khúsnak fiai: Széba, Hávilah, Szábthah, Rahmáh és Szabthékah; Rahmáhnak pedig fiai: Séba és Dédán.
\par 10 Khús nemzé Nimródot is; ez kezde hatalmassá lenni a földön.
\par 11 Miczráim pedig nemzé Lúdimot, Anámimot, Lehábimot és Naftukhimot.
\par 12 Pathruszimot és Kaszlukhimot, a kiktõl a Filiszteusok származtak, és Kafthorimot.
\par 13 Kanaán pedig nemzé Czídont, az õ elsõszülöttét és Khétet,
\par 14 És Jebuzeust, Emorreust és Girgazeust.
\par 15 Khivveust, Harkeust és Szineust.
\par 16 Arvadeust, Czemareust és Hamatheust.
\par 17 Sémnek fiai: Élám és Assur, Arpaksád, Lúd, Arám, Úcz, Húl, Gether és Mesek.
\par 18 Arpaksád pedig nemzé Séláht és Séláh nemzé Hébert.
\par 19 Hébernek is lett két fia, az egyiknek neve Péleg, mivelhogy az õ idejében osztatott el a föld; testvérének neve pedig Joktán.
\par 20 Joktán pedig nemzé Almodádot és Sélefet, Haczarmávethet és Jerákhot,
\par 21 Hadórámot, Úzált és Diklát,
\par 22 És Ebált, Abimáelt és Sébát,
\par 23 Ofirt, Havilát és Jóbábot. Ezek mind Joktán fiai.
\par 24 Sém, Arpaksád, Séláh.
\par 25 Héber, Péleg, Réu.
\par 26 Sérug, Nákhor, Tháré.
\par 27 Abrám, ez az Ábrahám.
\par 28 Ábrahám fiai: Izsák és Ismáel.
\par 29 Ezeknek nemzetségei pedig ezek: Ismáel elsõszülötte Nebájót és Kédar, Adbeél és Mibszám.
\par 30 Misma és Dúmah, Massza, Hadad és Théma.
\par 31 Jétur, Náfis és Kedmah; ezek az Ismáel fiai.
\par 32 Keturának pedig, az Ábrahám ágyasának fiai, kiket szüle, ezek: Zimrán, Joksán, Médán, Midián, Isbák és Suakh. És a Joksán fiai: Séba és Dédán.
\par 33 És a Midián fiai: Éfah, Héfer, Hánok, Abida és Eldáh. Mindezek a Keturáh fiai.
\par 34 Ábrahám pedig nemzé Izsákot; Izsák fiai pedig ezek: Ézsau és Izráel.
\par 35 Ézsaunak fiai: Elifáz, Réhuél, Jéhus, Jahlám és Korakh.
\par 36 Elifáz fiai: Thémán, Omár, Czefi, Gahtám, Kenáz és Timna és Amálek.
\par 37 Réhuél fiai: Nakhath, Zérakh, Samma és Mizza.
\par 38 Széir fiai: Lótán és Sóbál, Czibhón, Haná, Disón, Eczer és Disán.
\par 39 Lótán fiai pedig: Hóri és Hómám: Lótánnak huga pedig Timna.
\par 40 Sóbál fiai: Alján és Mánakháth, Hébál, Sefi és Onám; Czibhón fiai pedig: Aja és Haná.
\par 41 Haná fia: Disón, Disón fiai pedig: Hamrán és Esbán, Ithrán és Kherán.
\par 42 Eczer fiai: Bilhán és Zahaván, Jakán. Disán fiai: Húcz és Arán.
\par 43 Ezek pedig a királyok, a kik uralkodának Edom földén, mielõtt az Izráel fiai között király uralkodott volna: Bela, Behor fia, az õ városának neve Dinhába vala.
\par 44 Bela meghalván, uralkodék helyette Jóbáb, a Boczrából való Zerakh fia.
\par 45 És hogy Jóbáb meghala, uralkodék helyette a Témán földébõl való Khusám.
\par 46 Meghala Khusám is, és uralkodék helyette Hadád, a Bédád fia, ki megveré a Midiánitákat a Moáb mezején; és az õ városának neve Hávit vala.
\par 47 Hadád is, hogy meghala, uralkodék helyette a Masrekából való Szamlá.
\par 48 Szamlá holta után uralkodék helyette Saul, a folyóvíz mellett való Rékhobóthból.
\par 49 Saul is meghala, és uralkodék helyette Báhál-Khanán, az Akhbór fia.
\par 50 Báhál-Khanán holta után uralkodék helyette Hadád; és az õ városának neve Páhi, feleségének pedig neve Mehetábéel, ki Mézaháb leányának Matrédnak volt a leánya.
\par 51 Hadád halála után Edom fejedelmei valának: Timná fejedelem, Halvá fejedelem, Jetéth fejedelem,
\par 52 Ohólibámá fejedelem, Éla fejedelem és Pinon fejedelem,
\par 53 Kenáz fejedelem, Témán fejedelem és Mibczár fejedelem,
\par 54 Magdiél fejedelem és Hirám fejedelem. Ezek voltak Edom fejedelmei.

\chapter{2}

\par 1 Ezek az Izráel fiai: Rúben, Simeon, Lévi, Júda, Issakhár és Zebulon;
\par 2 Dán, József, Benjámin, Nafthali, Gád és Áser.
\par 3 Júdának fiai: Hér, Ónán és Séla; e három születék néki a Kanaánita Súahnak leányától. De Hér, Júdának elsõszülött fia gonosz vala az Úr szemei elõtt, és megölé õt az Úr.
\par 4 Thámár pedig, az õ menye szülé néki Péreczet és Zerákhot. Júdának fiai mindnyájan öten valának.
\par 5 Pérecz fiai: Kheczrón és Khámul.
\par 6 Zerákh fiai pedig: Zimri és Ethán, Hémán, Kálkól és Dára; mindenestõl öten.
\par 7 Kármi fiai; Ákán, Izráelnek megrontója, mivel lopott a zsákmányból.
\par 8 Ethán fia: Azária.
\par 9 Kheczrón fiai, kik születtenek néki: Jérakhméel, Rám és Kélubai.
\par 10 Rám nemzé Amminádábot, Amminádáb pedig nemzé Nakhsont, a Júda fiainak fejedelmét.
\par 11 Nakhson nemzé Szálmát, Szálma pedig nemzé Boázt;
\par 12 Boáz nemzé Obedet; Obed nemzé Isait;
\par 13 Isai pedig nemzé Eliábot, az õ elsõszülöttét, és Abinádábot, másodikat, Simeát,  harmadikat.
\par 14 Netanéelt, negyediket, és Raddait, ötödiket.
\par 15 Osemet, hatodikat és Dávidot, a hetedik fiát;
\par 16 És nõvéreiket, Séruját és Abigáilt. Sérujának pedig fiai voltak: Absai, Joáb  és Asáel, e három.
\par 17 Abigáil szülé Amasát; Amasának atyja pedig az Ismáel nemzetségébõl való Jéter vala.
\par 18 Káleb pedig, a Kheczrón fia nemzett vala az õ Azuba nevû feleségétõl és Jérióttól; és ezek az õ fiai: Jéser, Sobáb és Ardon.
\par 19 Azuba meghala, és Káleb vevé magának feleségül Efratát, és ez szülé néki Húrt.
\par 20 Húr nemzé Urit, Uri pedig nemzé Bésaléelt.
\par 21 Azután beméne Kheczrón Mákirnak, Gileád atyjának leányához; mert õ ezt elvette vala hatvan esztendõs korában, és szülé néki Ségubot.
\par 22 Ségub pedig nemzé Jáirt, kinek huszonhárom városa vala a Gileád földén.
\par 23 De Gesur és Árám elvették tõlük Jáir falvait, Kenáthot és mezõvárosait, hatvan várost. Mindezek Mákirnak, a Gileád atyjának fiaié.
\par 24 Minekutána pedig meghala Kheczrón Káleb-Efratában, akkor szülé néki Abija, a Kheczrón felesége, Ashúrt, Tékoa atyját.
\par 25 Jérakhméelnek, a Kheczrón elsõszülöttének fiai voltak: Rám, az elsõszülött, Búna, Orem, Osem és Akhija.
\par 26 Volt más felesége Jérakhméelnek, Atára nevû; ez az Onám anyja.
\par 27 Rámnak, Jérakhméel elsõszülöttének pedig fiai voltak: Maás, Jámin és Héker.
\par 28 Onám fiai voltak: Sammai és Jáda; és Sammai fiai: Nádáb és Abisúr.
\par 29 Abisúr feleségének neve Abihail, a ki szülé néki Akhbánt és Mólidot.
\par 30 Nádáb fiai: Széled és Appaim: Széled magtalanul halt meg.
\par 31 Appaim fia: Isi; Isi fia: Sésán; Sésán fia: Ahálai.
\par 32 Jáda fiai, a ki Sammai testvére volt: Jéter és Jonathán; Jéter magtalanul halt meg.
\par 33 Jonathán fiai: Pélet és Záza. Ezek voltak a Jérakhméel fiai.
\par 34 Sésánnak nem voltak fiai, hanem csak leányai; de volt Sésánnak egy Égyiptombeli szolgája, Járha nevû.
\par 35 És adá Sésán e Járha nevû szolgájának az õ leányát feleségül, a ki szülé néki Athait.
\par 36 Athai pedig nemzé Nátánt; Nátán nemzé Zabádot;
\par 37 Zabád nemzé Eflált; Eflál nemzé Obedet;
\par 38 Obed nemzé Jéhut; Jéhu nemzé Azáriát.
\par 39 Azária nemzé Hélest; Héles nemzé Elását;
\par 40 Elása nemzé Sisémait; Sisémai nemzé Sallumot;
\par 41 Sallum nemzé Jékámiát; Jékámia pedig nemzé Elisámát.
\par 42 A Káleb fiai pedig, a ki Jérakhméel testvére vala: elsõszülötte Mésa; ez volt Zifnek és Marésa fiainak atyjok, Hebronnak atyja.
\par 43 Hebron fiai: Kórah, Tappuah, Rékem és Séma.
\par 44 Séma pedig nemzé Rahámot, a Jorkeám atyját; és Rékem nemzé Sammait,
\par 45 A Sammai fia pedig: Máon; ez a Máon volt a Bethsúr atyja.
\par 46 Efa pedig, a Káleb ágyastársa, szülé Háránt és Mósát és Gázezt; és Hárán nemzé Gázezt.
\par 47 Jaddai fiai pedig: Régem, Jotám, Gésán, Pelet, Héfa és Saáf.
\par 48 A Káleb ágyasa Maaka szülé Sébert és Tirhánát.
\par 49 És szülé Saáfot, Madmanna atyját, Sevát, a Makbéna atyját és Gibea atyját; és Aksza a Káleb leánya vala.
\par 50 Ezek voltak Káleb fiai, a ki Húrnak, az Efrata elsõszülöttének fia volt: Sobál, Kirját-Jeárim atyja.
\par 51 Szálma, Bethlehem atyja; Háref, Bethgáder atyja.
\par 52 Voltak pedig Sobálnak is, a Kirját-Jeárim atyjának fiai: Haroé, a fél Menuhót ura.
\par 53 A Kirját-Jeárim háznépei: Jitreusok, Puteusok, Sumateusok, Misraiteusok; ezektõl származának a Sorateusok és az Estaoliteusok.
\par 54 Szálma fiai, Bethlehem és a Netofátbeliek, Atróth, Beht-Joáb, a Czórabeli Manahateusok fele.
\par 55 És a Jábesben lakozó tudós emberek háznépei: a Tirateusok, Simateusok, Sukateusok. Ezek a Kineusok, a kik Hámáttól, a Rékáb házának atyjától származtak.

\chapter{3}

\par 1 Ezek a Dávid fiai, a kik Hebronban születének néki: elsõszülötte vala Amnon, a Jezréelbõl való Ahinoámtól; második vala Dániel, a Kármelbõl való Abigailtól.
\par 2 Harmadik Absolon, a ki Maakának, a Gessurbeli Talmai király leányának volt a fia; negyedik Adónia, Haggittól való.
\par 3 Ötödik Sefátia, Abitáltól való; hatodik Jitreám, az õ feleségétõl, Eglától való.
\par 4 E hat fia születék néki Hebronban, a hol hét esztendeig és hat hónapig uralkodott; Jeruzsálemben pedig harminczhárom esztendeig uralkodék.
\par 5 Jeruzsálemben pedig ezek születének néki: Simea, Sóbáb, Nátán és Salamon, négyen, Bathsuától, az Ammiel leányától.
\par 6 És Ibhár, Elisáma és Elifélet.
\par 7 Nógah, Néfeg és Jáfia.
\par 8 És Elisáma, Eljada és Elifélet, kilenczen.
\par 9 Mindezek a Dávid fiai, az õ ágyastársainak fiain kivül; és ezeknek a huga, Támár.
\par 10 Salamonnak pedig fia volt Roboám; ennek fia Abija, ennek fia Asa, ennek fia Josafát.
\par 11 Ennek fia Jórám, ennek fia Akházia, ennek fia Joás.
\par 12 Ennek fia Amásia, ennek fia Azária; ennek fia Jótám.
\par 13 Ennek fia Akház, ennek  fia Ezékiás, ennek fia Manasse.
\par 14 Ennek fia Amon, ennek fia Jósiás.
\par 15 Jósiásnak pedig fiai: elsõszülötte Johanán, második Jójákim, harmadik Sédékiás, Sallum negyedik.
\par 16 Jójákim fiai: az õ fia Jékoniás, ennek fia Sédékiás.
\par 17 Jékóniásnak fiai: Assir, ennek fia Saálthiel.
\par 18 És Málkirám, Pedája, Sénasár, Jékámia, Hosáma és Nédábia.
\par 19 Pedája fiai: Zerubábel és Simei. Zerubábel fiai: Mesullám, Hanánia; és az õ hugok vala Selómit.
\par 20 És Hásuba, Ohel, Berekia, Hasádia és Jusáb-Hésed, ezek öten.
\par 21 Hanánia fiai: Pelátia és Jésaia; és Refája fiai és Arnám fiai, Obádia fiai, Sekánia fiai.
\par 22 És Sekánia fiai: Semája. Semája fiai: Hattus, Jigeál, Báriah, Neárja és Sáfát, ezek hatan.
\par 23 Neárja fiai: Eljohénai, Ezékiás és Azrikám, ezek hárman.
\par 24 Eljohénai fiai: Hódajéva, Eliásib, Pelája, Akkub, Johanán, Delája és Anáni; ezek heten.

\chapter{4}

\par 1 Júda fiai ezek: Pérecz, Kheczrón, Kármi, Húr és Sobál.
\par 2 Reája pedig, a Sobál fia nemzé Jáhátot; Jáhát nemzé Ahumáit és Lahádot; ezek a Sorateusok háznépei.
\par 3 Ezek Etám atyjától valók: Jezréel, Jisma, Jidbás; és az õ hugoknak neve Haslelponi.
\par 4 Pénuel pedig Gedor atyja, és Ezer Húsa atyja. Ezek Efrata elsõszülöttének, Húrnak fiai, a ki Bethlehem atyja vala.
\par 5 Ashúrnak pedig, a Tékoa atyjának volt két felesége, Heléa és Naára.
\par 6 És Naára szülé néki Ahuzámot, Héfert, Teménit és Ahastárit. Ezek a Naára fiai.
\par 7 Heléa fiai: Séret, Jésohár, Etnán és Kócz.
\par 8 Kócz pedig nemzé Hánubot, Hásobébát és Ahárhel háznépét, a ki Hárumnak fia vala.
\par 9 Jábes pedig testvéreinél tiszteletreméltóbb vala, és azért nevezé õt az õ anyja Jábesnek, mondván: Mivelhogy fájdalommal szülém õt.
\par 10 És Jábes az Izráel Istenét hívá segítségül, mondván: Ha engem megáldanál és az én határomat megszélesítenéd, és a te kezed én velem lenne, és engem minden veszedelemtõl megoltalmaznál, hogy bút ne lássak! És megadá Isten néki, a mit kért vala.
\par 11 Kélub pedig, a Súkha testvére, nemzé Méhirt; ez az Eston atyja.
\par 12 Eston nemzé Béth-Rafát, Paseákhot, Tehinnát, Ir-Náhás atyját. Ezek a Rékától való férfiak.
\par 13 Kénáz fiai: Othniel és Serája; Othniel fia: Hatát.
\par 14 Meonótai nemzé Ofrát; Serája pedig nemzé Joábot, a Gé-Harasimbeliek atyját, mert mesteremberek valának.
\par 15 Káleb fiai, ki Jefunné fia vala: Iru, Ela és Naám; és Ela fiai; és Kénáz.
\par 16 Jéhalélel fiai: Zif, Zifa, Tirja és Asárel.
\par 17 Ezra fiai: Jéter, Méred, Efer és Jálon; és szülé Mirjámot, Sammait és Isbát, Estemóa atyját.
\par 18 Ennek felesége pedig, Jehudéja szülé Jéredet, a Gedor atyját és Hébert, a Szókó atyját, Jékuthielt, a Zánoah atyját; ezek Bithiának, a Faraó leányának fiai, a kit Méred elvett vala.
\par 19 Hódia nevû feleségének pedig fiai, ki Nahamnak, Keila atyjának nõvére vala: Hagármi és a Maakátbeli Estemóa.
\par 20 Simon fiai: Amnon, Rinna, Benhanán és Thilon. Isi fiai: Zohét és Benzohét.
\par 21 Júda fiának, Sélának fiai: Er, Léka atyja, és Laáda, Marésa atyja, és a gyapotszövõk háznépe Béth-Asbeában;
\par 22 És Jókim és Kozeba lakosai, és Joás és Saráf, a kik Moáb urai voltak, és Jásubi, Léhem. De ezek már régi dolgok.
\par 23 Ezek voltak a fazekasok, és Netaimban és Gederában laktak. A királylyal laktak ott, az õ dolgáért.
\par 24 Simeon fiai: Némuel, Jámin, Járib, Zérah, Saul;
\par 25 Kinek fia, Sallum, kinek fia Mibsám, kinek fia Misma.
\par 26 Misma fiai: Hammúel az õ fia, Zakkur az õ fia, Simi az õ fia;
\par 27 Siminek tizenhat fia és hat leánya volt; de testvéreinek nem volt sok fia, s általában háznépök nem volt oly népes, mint Júda fiaié.
\par 28 Lakoznak vala pedig Beersebában és Móladában és Haczar-Suálban.
\par 29 Bilhában, Eczemben és Toládban,
\par 30 Bétuelben, Hormában és Cziklágban,
\par 31 Beth-Markabótban, Haczar-Szuszimban, Beth-Biriben és Saáraimban; ezek valának az õ városaik mindaddig, míg Dávid királylyá lett.
\par 32 Faluik pedig ezek: Etám, Ain, Rimmon, Tóken és Asán, öt város;
\par 33 És mindazok a falvaik, a melyek e városok körül voltak Bálig; ezek valának lakóhelyeik és nemzetségeik:
\par 34 Mesobáb, Jámlek és Jósa, Amásia fia.
\par 35 És Jóel és Jéhu, a Jósibia fia, ki Serája fia, ki Asiel fia vala;
\par 36 És Eljoénai, Jaákoba, Jésohája, Asája, Adiel, Jesiméel és Benája.
\par 37 Ziza, a Sifi fia, ki Alon fia, ki Jedája fia, ki Simri fia, ki Semája fia.
\par 38 Ezek a névszerint felsoroltak voltak a fõemberek nemzetségökben, a kik igen elszaporodtak volt atyjok házában,
\par 39 Azért elindulának Gedor felé, hogy a völgy keleti részére menjenek és ott barmaiknak legelõt keressenek.
\par 40 Találának is zsíros és jó legeltetõ helyet (az a föld pedig tágas, nyugodalmas és békességes), mert Khámból valók laktak ott azelõtt.
\par 41 Elmenvén pedig e névszerint megnevezettek Ezékiásnak, a Júda királyának idejében, lerombolták sátoraikat, és a Maonitákat, a kiket ott találtak, kiirtották mind e mai napig, s helyökbe letelepedének, mivel ott barmaik számára legelõhelyeket találtak.
\par 42 És közülök, a Simeon fiai közül, némelyek elmenének a Seir hegyére, úgymint ötszázan, a kiknek elõljáróik az Isi fiai, Pelátja, Nehárja, Refája és Uzriel voltak.
\par 43 És valakik az Amálek nemzetségébõl megmaradtak vala, mind levágák azokat, és ott letelepedének mind e mai napig.

\chapter{5}

\par 1 Rúbennek, Izráel elsõszülöttének fiai (mert õ volt az elsõszülött; mikor pedig megfertõztette az õ atyjának ágyasházát, az õ elsõszülöttségi joga a József fiainak adaték, a ki Izráel fia vala, mindazáltal nem úgy hogy õk neveztessenek származás szerint elsõszülötteknek,
\par 2 Mert Júda tekintélyesebb vala az õ testvérei között, és õ belõle való volt a fejedelem, hanem az elsõszülöttségnek haszna lõn Józsefé):
\par 3 Ezek Rúbennek, Izráel elsõszülöttének fiai: Khánokh, Pallu, Kheczrón és Kármi.
\par 4 Jóel fiai: Semája ennek fia, Góg ennek fia, Simei ennek fia.
\par 5 Mika ennek fia, Reája ennek fia, Baál ennek fia.
\par 6 Beéra ennek fia, a kit fogságba vitt Tiglát-Piléser, az Assiriabeli király; õ a Rúbeniták fejedelme vala.
\par 7 Testvérei pedig, családjaik, nemzetségük megszámlálása szerint ezek valának: a fõ Jéhiel és Zakariás,
\par 8 Bela, Azáz fia, ki Séma fia, ki Jóel fia vala, ki Aróerben lakott Nébóig és Baál-Meonig.
\par 9 Napkelet felé is lakik vala a pusztában való bemenetelig, az Eufrátes folyóvíztõl fogva; mert az õ barmai Gileád földén igen elszaporodtak.
\par 10 Saul királynak idejében pedig támasztának hadat a Hágárénusok ellen, és elhullának azok az õ kezeik által, és lakának azoknak sátoraikban, Gileádnak napkelet felé való egész részében.
\par 11 A Gád fiai pedig velök szemben a Básán földén laktak Szalkáig.
\par 12 Jóel vala elõljárójok, Sáfám második az után; Johánai és Sáfát Básánban.
\par 13 És testvéreik, családjaik szerint ezek: Mikáel, Mésullám, Séba, Jórai, Jaékán, Zia és Éber, heten.
\par 14 Ezek az Abihail fiai, ki Húri fia, ki Jároáh fia, ki Gileád fia, ki Mikáel fia, ki Jésisai fia, ki Jahadó fia, ki Búz fia.
\par 15 És Ahi, a Gúni fiának, Abdielnek a fia volt a család feje.
\par 16 Ezek Gileádban, Básánban és az ezekhez tartozó mezõvárosokban laktak, és Sáronnak minden legelõjén, határaikig.
\par 17 Kik mindnyájan megszámláltatának Jótámnak, a Júda királyának idejében, és Jeroboámnak,  az Izráel királyának idejében.
\par 18 A Rúben fiai közül és a Gáditák közül és a Manasse félnemzetsége közül erõs paizs- és fegyverhordozó férfiak, kézívesek és a hadakozásban jártasok, negyvennégyezerhétszázhatvanan harczra kész férfiak;
\par 19 Hadakozának a Hágárénusok ellen, Jétúr, Náfis és Nódáb ellen.
\par 20 És gyõzedelmesek levének azokon, és kezekbe adatának a Hágárénusok és mindazok, a kik ezekkel valának; mert az Istenhez kiáltának harcz közben, és õ meghallgatá õket, mert õ benne bíztak.
\par 21 És elvivék az õ barmaikat, ötvenezer tevét, kétszázötvenezer juhot, kétezer szamarat és százezer embert.
\par 22 A seb miatt pedig sokan elhullának; mert Istentõl vala az a harcz; és azok helyén lakának a fogságig.
\par 23 A Manasse nemzetsége felének fiai pedig azon a földön laktak, a mely Básántól Baál-Hermonig, Szenirig és Hermon hegyéig terjedt, mert igen megsokasodtak vala.
\par 24 És ezek voltak az õ atyjok háznépének fejedelmei: Efer, Isi, Eliel, Azriel, Irméja, Hodávia és Jahdiel, igen erõs férfiak, híres férfiak, a kik az õ atyjok háznépe között fõk voltak.
\par 25 Vétkezének pedig az õ atyjoknak Istene ellen; mert a föld lakóinak bálványisteneivel paráználkodának, a kiket az Isten szemök elõl elpusztított.
\par 26 Felindítá azért az Izráel Istene Pulnak, az Assiriabeli királynak szívét és Tiglát-Pilésernek, az assiriai királynak szívét, és fogva elvivé õket, a Rúbenitákat, a Gáditákat és a Manasse félnemzetségét is; és elvivé õket Haláhba és Háborba, Hárába és a Gózán folyóvizéhez mind e mai napig.

\chapter{6}

\par 1 Lévi fiai: Gerson, Kéhát és Mérári.
\par 2 Kéhát fiai pedig: Amrám, Ishár, Hebron és Uzziel.
\par 3 Amrám gyermekei: Áron, Mózes és Miriám; Áron fiai pedig: Nádáb,  Abihu, Eleázár és Ithamár.
\par 4 Eleázár nemzé Fineást, Fineás nemzé Abisuát;
\par 5 Abisua pedig nemzé Bukkit, Bukki nemzé Uzzit;
\par 6 Uzzi nemzé Zeráhiát, Zeráhia nemzé Mérajótot;
\par 7 Mérajót nemzé Amáriát, Amária nemzé Ahitúbot;
\par 8 Ahitúb nemzé Sádókot;  Sádók nemzé Ahimáhást;
\par 9 Ahimáhás nemzé Azáriát, Azária nemzé Jóhanánt;
\par 10 Jóhanán nemzé Azáriát, ez volt a pap abban a házban, a melyet Salamon Jeruzsálemben épített vala.
\par 11 Azária nemzé Amáriát; Amária nemzé Ahitúbot;
\par 12 Ahitúb nemzé Sádókot, Sádók nemzé Sallumot;
\par 13 Sallum nemzé Hilkiát; Hilkia nemzé Azáriát;
\par 14 Azária nemzé Séráját, Sérája nemzé Jéhozadákot;
\par 15 Jéhozadák pedig fogságba méne, mikor az Úr Júdát és Jeruzsálemet fogságba viteté Nabukodonozor által.
\par 16 Lévi fiai: Gerson, Kéhát és Mérári.
\par 17 Ezek Gerson fiainak nevei: Libni és Simhi.
\par 18 Kéhát fiai: Amrám, Jiczhár, Khebron és Huzziel.
\par 19 Mérári fiai: Makhli és Musi. Ezek a Lévi háznépei az õ nemzetségeik szerint.
\par 20 Gersonnak fiai: Ligni az õ fia, Jáhát ennek fia, Zima ennek fia.
\par 21 Jóah ennek fia, Iddó ennek fia, Zérah ennek fia és Jéathérai ennek fia.
\par 22 Kéhát fiai: Amminádáb az õ fia, Kórákh ennek fia és Asszir ennek fia;
\par 23 Elkána ennek fia, Ebiásáf ennek fia és Asszir ennek fia.
\par 24 Tákhát ennek fia, Uriel ennek fia, Uzzia ennek fia és Saul ennek fia.
\par 25 Elkána fiai: Amásai és Ahimót,
\par 26 Elkána. Elkána fia: Sófai az õ fia és Náhát ennek fia.
\par 27 Eliáb ennek fia, Jérohám ennek fia, Elkána ennek fia.
\par 28 Sámuel fiai pedig: az elsõszülött Vásni, a második Abija.
\par 29 Mérári fiai: Mahli, Libni ennek fia; Simhi ennek fia és Uzza ennek fia.
\par 30 Simea ennek fia, Haggija ennek fia és Asája ennek fia.
\par 31 Ezek azok, a kiket Dávid állított be az Úr házában az énekléshez, mikor az Isten ládája elhelyeztetett.
\par 32 És a míg Salamon felépíté az Úr házát Jeruzsálemben, addig a gyülekezet sátora elõtt szolgáltak énekléssel és állottak szolgálatban, kiki az õ rendje szerint.
\par 33 Ezek pedig a kik szolgáltak, és az õ fiaik: a Kéhátiták fiai közül Hémán fõéneklõ, Jóel fia, ki Sámuel fia.
\par 34 Ki Elkána fia, ki Jérohám fia, ki Eliél fia, ki Thóa fia.
\par 35 Ki Czúf fia, ki Elkána fia, ki Mahát fia, ki Amásai fia.
\par 36 Ki Elkána fia, ki Jóél fia, ki Azárja fia, ki Séfánia fia.
\par 37 Ki Tákhát fia, ki Asszir fia, ki Ebiásáf fia, ki Kórákh fia.
\par 38 Ki Jiczár fia, ki Kéhát fia, ki Lévi fia, ki Izráel fia.
\par 39 És ennek testvére, Asáf, a ki jobbkeze felõl áll vala; Asáf, a Berekiás fia, ki Simea fia vala.
\par 40 Ki Mikáel fia, ki Bahásia fia, ki Melkija fia.
\par 41 Ki Ethni fia, ki Zérah fia, ki Adája fia.
\par 42 Ki Etán fia, ki Zimma fia, ki Simhi fia.
\par 43 Ki Jáhát fia, ki Gerson fia, ki Lévi fia.
\par 44 Továbbá a Mérári fiai, a kik azokkal atyafiak valának, balkéz felõl állnak vala; Etán, Kisi fia, ki Abdi fia, ki Malluk fia.
\par 45 Ki Kasábja fia, ki Amásia fia, ki Hilkia fia.
\par 46 Ki Amsi fia, ki Báni fia, ki Sémer fia.
\par 47 Ki Makhli fia, ki Musi fia, ki Mérári fia, ki Lévi fia.
\par 48 És testvéreik, a Léviták rendeltetnek az Isten háza hajlékának egyéb szolgálatjára.
\par 49 Áron pedig és az õ fiai az egészen megégetendõ áldozatnak oltára mellé, és a füstölõ oltár mellé, a szentek-szentjének szolgálatja mellé, és az Izráel megszentelésére rendeltetének mind a szerint, a mint Mózes, az Isten szolgája megparancsolta volt.
\par 50 Áron fiai pedig ezek: Eleázár, az õ fia, ennek fia Fineás, ennek fia Abisua.
\par 51 Ennek fia Bukki, ennek fia Uzzi, ennek fia Zerája.
\par 52 Ennek fia Merájót, ennek fia Amárja, ennek fia Akhitúb.
\par 53 Ennek fia Sádók, ennek fia Akhimás.
\par 54 És az Áron fiainak, a Kéhátiták nemzetségébõl, ezek a lakhelyeik, letelepedésük szerint az õ vidékükön, mert ez jutott volt nékik sors által.
\par 55 Õk kapták Hebront, a Júda földében és a körülte való legelõket.
\par 56 De e város földjeit és annak faluit Kálebnek, a Jefunné fiának adák.
\par 57 Az Áron fiainak azért a Júda városai közül adák a menedékvárosokat, Hebront, Libnát és legelõit, Jatthirt és Esthemoát és ezeknek legelõit,
\par 58 És Hilent és annak legelõit, és Débirt és annak legelõit,
\par 59 Asánt és annak legelõit, és Béth-Semest és annak legelõit.
\par 60 A Benjámin nemzetségébõl: Gébát és annak legelõit, Allémetet és annak legelõit, Anatót várost is és annak legelõit. Ezeknek az õ nemzetségek szerint tizenhárom városuk volt.
\par 61 A Kéhát többi fiainak pedig az egy nemzetségnek családjaitól, és a félnemzetségbõl, a Manasse nemzetségének felétõl, sors által tíz várost adtak.
\par 62 Míg a Gerson fiainak meg az õ nemzetségök szerint az Izsakhár nemzetségébõl, az Áser nemzetségébõl, a Nafthali nemzetségébõl és a Manasse nemzetségébõl Básánban adtak tizenhárom várost.
\par 63 A Mérári fiainak az õ nemzetségök szerint a Rúben nemzetségébõl, a Gád nemzetségébõl és a Zebulon nemzetségébõl sors által tizenkét várost.
\par 64 Adának tehát az Izráel fiai a Lévitáknak városokat, azoknak legelõivel együtt.
\par 65 Sors által adták a Júda nemzetségébõl, a Simeon nemzetségébõl és a Benjámin nemzetségébõl ezeket a névszerint megnevezett városokat.
\par 66 Azoknak, a kik a Kéhát fiainak családjaiból valók voltak, és a határukban levõ városok az Efraim nemzetségébõl valának:
\par 67 Azoknak adák a menedékvárosokat, Sikemet és annak legelõit az Efraim hegyén, Gézert és annak legelõit.
\par 68 És Jokmeámot és annak legelõit, Bethoront és annak legelõit.
\par 69 Ajalont és annak legelõit; Gáthrimmont is és annak legelõit.
\par 70 A Manasse nemzetségének felébõl Anert és annak legelõit, Bileámot és annak legelõit, a Kéhát többi fiainak családjai részére.
\par 71 A Gerson fiainak pedig a Manasse félnemzetségébõl Gólánt Básánban és annak legelõit, és Astarótot és annak legelõit:
\par 72 Az Izsakhár nemzetségébõl adák Kédest és annak legelõit; Dobrátot és annak legelõit.
\par 73 Rámótot és annak legelõit, Anémet és annak legelõit.
\par 74 Az Áser nemzetségébõl Másált és annak legelõit, és Abdont és annak legelõit.
\par 75 Hukókot és annak legelõit; Réhobot és annak legelõit.
\par 76 A Nafthali nemzetségébõl Kédest Galileában és annak legelõit; Hammont és annak legelõit; és Kirjáthaimot és annak legelõit.
\par 77 A Mérári többi fiainak a Zebulon nemzetségébõl Rimmont és annak legelõit, és Thábort és annak legelõit.
\par 78 A Jordánon túl Jérikhó ellenében a Jordánnak napkelet felõl való részében a Rúben nemzetségébõl Bésert a pusztában és annak legelõit; Jahását és annak legelõit.
\par 79 Kedemótot és annak legelõit; Mefaátot és annak legelõit.
\par 80 A Gád nemzetségébõl Rámótot Gileádban és annak legelõit; Mahanaimot és annak legelõit.
\par 81 Hesbont és annak legelõit; Jaázert és annak legelõit.

\chapter{7}

\par 1 Izsakhár fiai: Thóla, Pua, Jásub és Simron, négyen.
\par 2 Thóla fiai pedig: Uzzi, Réfája, Jériel, Jakhmai, Jibsám és Sámuel, kik fejedelmek valának az õ atyjaiknak családjaiban. A Thóla fiai erõs hadakozók voltak nemzetségökben, kiknek száma Dávid király idejében huszonkétezerhatszáz vala.
\par 3 Uzzi fiai: Izráhja; és az Izráhja fiai: Mikáel, Obádia, Joel, Issia, öten, a kik mind fõemberek valának.
\par 4 És közöttük nemzetségeik és családjaik szerint harminczhatezer hadakozó férfi volt, mert sok feleségök volt és sok fiuk is.
\par 5 Ezeknek testvéreik is, Izsakhárnak egész nemzetsége szerint erõs hadakozó férfiak valának, nyolczvanhétezeren szám szerint mindenestõl.
\par 6 Benjámin fiai: Bela, Béker és Jediáel, hárman.
\par 7 Bela fiai: Esbon, Uzzi, Uzziel, Jérimót és Hiri, öt fõember az õ nemzetségökben, erõs hadakozó férfiak, a kik megszámláltatván, huszonkétezerharmincznégyen valának.
\par 8 Béker fiai: Zémira, Joás, Eliézer, Eljoénai, Omri, Jeremót, Abija, Anatót és Alémet; ezek mindnyájan Béker fiai.
\par 9 Azok megszámláltatván nemzetségeik szerint, családjuk fejei és az erõs hadakozó férfiak húszezerkétszázan valának.
\par 10 Továbbá Jédiáel fia: Bilhán; Bilhán fiai: Jéus, Benjámin, Éhud, Kénaána, Zetán, Társis és Ahisahár.
\par 11 Ezek mind Jédiáel fiai, családfõk, erõs hadakozó férfiak, a kik tizenhétezerkétszázan mehetnek vala ki a viadalra.
\par 12 Ir fiai: Suppim és Khuppim; Húsim Ahernek fia.
\par 13 Nafthali fiai: Jakhcziel, Gúni, Jéczer, Sallum, a Bilha fiai.
\par 14 Manasse fiai: Aszriel, kit szüle az õ felesége. Az õ ágyastársa pedig, a Siriabeli asszony szülé Mákirt, a Gileád atyját.
\par 15 Mákir pedig vevé feleségül Khuppimnak és Suppimnak hugát, kinek neve Maáka. A másiknak neve Czélofhád. Czélofhádnak leányai voltak.
\par 16 Maáka, a Mákir felesége szüle fiat, a kit neveze Péresnek: annak pedig öcscsét Séresnek. Ennek fiai: Ulám és Rékem.
\par 17 Ulám fia: Bédán. Ezek a Gileád fiai, ki Mákir fia volt, ki Manasse fia volt.
\par 18 Az õ huga, Moléket pedig szülé Ishodot, Abiézert és Makhlát.
\par 19 Semidának fiai voltak: Ahián, Sekem, Likhi és Aniám.
\par 20 Efraim fiai pedig: Sútelákh, kinek fia Béred, ennek fia Táhát, ennek fia Elhada, ennek fia Táhát.
\par 21 Ennek fia Zábád, ennek fia Sútelákh, Ezer és Elhád. És megölék ezeket a Gáthbeliek, a földnek lakosai; mert alámentek vala, hogy barmaikat elhajtsák.
\par 22 Efraim azért, az õ atyjok sokáig siratá õket, a kihez elmennek vala az õ atyjafiai, és õt vigasztalják vala.
\par 23 Beméne azért az õ feleségéhez, ki fogada méhében, és szüle fiat, és nevezé Bériának, mivelhogy szerencsétlenség történt az õ házában.
\par 24 Leánya pedig Seéra vala, a ki alsó és felsõ Bethoront és Uzen-Seérát építé.
\par 25 Réfah is az õ fia és Resef; ennek fia Théla, ennek fia Táhán.
\par 26 Ennek fia Laadán, ennek fia Ammihud, ennek fia Elisáma.
\par 27 Ennek fia Nún, ennek fia Józsué.
\par 28 Ezeknek birtokuk és lakóhelyük vala: Béthel és annak mezõvárosai; napkeletre Naarán, napenyészetre Gézer és ennek mezõvárosai, Sikem és ennek mezõvárosai, szinte Gázáig és az ehhez tartozó mezõvárosokig;
\par 29 És a Manasse fiai mellett: Béth-Seán és ennek mezõvárosai, Thaanák és ennek mezõvárosai, Megiddó és ennek mezõvárosai, Dór és ennek mezõvárosai. Ezekben laktak az Izráel fiának, Józsefnek fiai.
\par 30 Áser fiai: Jimnah, Jisvah, Jisvi, Beriha és Szerakh, az õ hugok.
\par 31 Beriha fiai: Khéber és Malkhiel, a ki Birzávit atyja volt.
\par 32 Khéber pedig nemzé Jaflétet, Sómert, Hótámot és Suát, az õ hugokat.
\par 33 Jaflét fiai: Pásák, Bimhál, Asvát; ezek Jaflét fiai.
\par 34 Sómer fiai: Ahi és Róhega, Jehubba és Arám.
\par 35 Testvérének, Hélemnek fia vala: Sófákh, Jimna, Séles és Amál.
\par 36 Sófákh fiai: Suákh, Harnéfer, Suál, Béri és Imra.
\par 37 Béser, Hód, Samma, Silsa, Itrán és Beéra.
\par 38 Jéter fiai: Jéfunné, Pispa és Ara.
\par 39 Ulla fiai: Ara, Hanniel és Risja.
\par 40 Ezek mind Áser fiai, családfõk, válogatott harczosok, a fejedelmek fejei; és megszámláltatván az õ nemzetségök rendje szerint, huszonhatezer harczra képes férfi volt.

\chapter{8}

\par 1 Benjámin pedig nemzé Belát, az õ elsõszülöttét, másodikat Asbélt, harmadikat Akhráhot,
\par 2 Negyediket Nóhát, és Ráját ötödiket.
\par 3 (Belának fiai voltak: Adár, Géra és Abihúd,
\par 4 Abisua, Naamán, Ahóah.)
\par 5 Gérát, Sefufánt és Hurámot.
\par 6 Ezek az Ehud fiai: (Ezek voltak fõemberek a Géba városában lakó nemzetség között, kiket fogságba vittek Manahátba.
\par 7 Nevezetesen Naamánt és Ahiját és Gérát vitték fogságba): nemzé pedig Uzzát és Akhihúdot.
\par 8 Saharáim pedig nemzé a Moáb mezején, minekutána eltaszítá feleségeit, Husimot és Baarát;
\par 9 Nemzé Hódes nevû feleségétõl Jobábot és Sibját, Mésát és Malkámot,
\par 10 Jéust is, Sokját és Mirmát. Ezek az õ fiai; fõemberek az õ nemzetségökben.
\par 11 Husimtól nemzé Abitúbot és Elpaált.
\par 12 Elpaál fiai: Eber, Miseám és Sémer; ez építé Onót és Lódot és ennek mezõvárosait.
\par 13 Béria, Séma, (ezek voltak fõk az Ajalonban lakozó nemzetségek közt, és ezek ûzték vala el Gáthnak lakóit),
\par 14 Ahio, Sasák, Jeremót,
\par 15 Zebádia, Arád, Ader.
\par 16 Mikáel, Ispa, Jóha, Béria fiai.
\par 17 Zebádia, Mésullám, Hizki, Héber.
\par 18 Ismérai, Izlia és Jobáb; Elpaál fiai.
\par 19 Jákim, Zikri, Zabdi,
\par 20 Eliénai, Silletai, Eliel,
\par 21 Adája, Berája és Simrát, Simhi fiai.
\par 22 Jispán, Eber, Eliel,
\par 23 Abdon, Zikri, Hanán,
\par 24 Hanánja, Elám, Anatótija,
\par 25 Ifdéja, Pénuel, Sasák fiai,
\par 26 Samsérai, Sehárja, Atália.
\par 27 Jaarésia, Elia, és Zikri, Jérohám fiai.
\par 28 Ezek voltak a családfõk az õ nemzetségök szerint, fõemberek és ezek laktak Jeruzsálemben.
\par 29 Gibeonban pedig laktak Gibeonnak atyja; az õ feleségének neve Maaka vala.
\par 30 És az õ elsõszülötte Abdon, azután Súr, Kis, Baál, Nádáb;
\par 31 Gedor, Ahio és Zéker,
\par 32 És Miklót, a ki nemzé Simámot. Ezek is testvéreiknek átellenében, Jeruzsálemben laktak testvéreikkel.
\par 33 Nér pedig nemzé Kist; Kis nemzé Sault, Saul nemzé Jonathánt, Malkisuát, Abinádábot és Esbaált.
\par 34 Jonathán fia: Méribbaál; Méribbaál nemzé Mikát.
\par 35 Mika fiai: Pitón, Mélek, Tárea és Akház.
\par 36 Akház pedig nemzé Jehoádát. Jehoáda pedig nemzé Alémetet és Azmávetet és Zimrit; Zimri nemzé Mósát;
\par 37 Mósa pedig Binát; ennek fia Ráfa, ennek fia Elása, ennek fia Asel.
\par 38 Továbbá Aselnek hat fia volt, kiknek ezek neveik: Azrikám, Bókru, Ismáel, Seárja, Obádia és Hanán; ezek mind Asel fiai.
\par 39 Az õ testvérének Eseknek fiai ezek: Ulám az õ elsõszülötte, Jéus második és Elifélet harmadik.
\par 40 És az Ulám fiai erõs hadakozó férfiak, kézívesek voltak, s gyermekeik és unokáik százötvenre szaporodtak. Mindezek a Benjámin fiai közül valók voltak.

\chapter{9}

\par 1 Az egész Izráel megszámláltatván nemzetségeik szerint, beirattak az Izráel és Júda királyainak könyvébe, a kik gonoszságukért Babilóniába vitetének.
\par 2 A legelsõ lakosok az õ jószágaikon s városaikban, ezek: Izráeliták, papok, Léviták és Nétineusok.
\par 3 Mert Jeruzsálemben laktak a Júda fiai közül s Benjámin fiai közül, Efraim és Manasse fiai közül valók:
\par 4 Utái, Ammihud fia, ki Omri fia, ki Imri fia, ki Báni fia, a Pérecz fiai közül való, ki Júda fia vala.
\par 5 A Silóniták közül, Asája, ki elsõszülött vala s ennek fiai.
\par 6 A Zérah fiai közül: Jéluel és az õ atyjafiai, hatszázkilenczvenen.
\par 7 A Benjámin fiai közül: Sallu, Mésullám fia, ki Hodávia fia, ki Hasénua fia.
\par 8 Ibnéja, Jérohám fia, és Ela, Uzzi fia, ki Mikri fia, és Mésullám, Séfátja fia, ki Reuel fia, ki Ibnija fia vala.
\par 9 Testvéreik, nemzetségeik szerint, kilenczszázötvenhatan valának; ezek mind családfõk voltak az õ atyjok háznépe szerint.
\par 10 A papok közül is: Jedája, Jéhojárib és Jákin.
\par 11 Azária, Hilkia fia, ki Mésullám fia, ki Sádók fia, ki Mérajót fia, ki Ahitubnak, az Isten háza fõgondviselõjének fia vala.
\par 12 És Adája Jérohám fia, ki Pashúr fia, ki Málkija fia; Maasai, Adiel fia, ki Jahzéra fia, ki Mésullám fia, ki Mésillémit fia, ki Immer fia.
\par 13 Ezeknek atyjokfiai az õ családjaiknak fejei, ezerhétszázhatvanan valának, buzgók az Isten háza dolgának munkájában.
\par 14 A Léviták közül Semája, Hásub fia, ki Azrikám fia, ki Hasábia fia, a Mérári fiai közül.
\par 15 Bakbakkár, Héres és Galál, és Mattánia, a Mika fia, ki Zikri fia, ki Asáf fia vala.
\par 16 Obádia, a Semája fia, Galál fia, ki Jédutun fia, és Bérékia, Asa fia, ki Elkána fia, a ki Nétofáti faluiban lakik vala.
\par 17 És az ajtónállók: Sallum, Akkúb, Talmon, Ahimán és ezek testvérei; Sallum pedig fõ vala.
\par 18 (És ennek mindez ideig a királykapun van helye napkelet felõl.) Ezek az ajtónállók a Léviták rendje szerint.
\par 19 Sallum pedig a Kóré fia, ki Ebiásáf fia, ki Kórákh fia, és ennek rokonságai a Kórakhiták családjából a szolgálat munkájában a hajlék ajtajának õrizõi valának, mint az õ atyáik az Úr seregében õrizték vala a bejáratot,
\par 20 Mikor Fineás, az Eleázár fia volt egykor az õ elõljárójok, kivel az Úr vala.
\par 21 Zakariás, a Meselémia fia, a gyülekezet sátorába való bejárat õrizõje.
\par 22 Ezek mind választott ajtónállók voltak, kétszáztizenketten; kik az õ faluikban, nemzetségök szerint, megszámláltattak volt; kiket Dávid rendelt és Sámuel próféta az õ tisztökbe.
\par 23 Õk és fiaik õrizik vala renddel az Úr házának, a sátornak kapuit.
\par 24 Négyfelé valának az õrizõk: napkeletre, napnyugotra, északra és délre.
\par 25 És ezeknek atyjokfiai az õ faluikban valának, de minden hetednap amazokhoz felmenének Jeruzsálembe bizonyos ideig.
\par 26 De a kapunállók négy fõemberének állandó megbizatásuk volt. Ezek a Léviták õrizik vala a kamarákat és az Isten házának  kincseit.
\par 27 És az Úrnak háza körül hálnak vala, mivelhogy az õrzés az õ tisztök volt, és minden reggel õk nyitják vala meg az ajtókat.
\par 28 És õ közülök némelyek a szolgálati edényekre viselnek vala gondot; mert mind a kivitelnél, mind a visszavitelnél számon veszik vala azokat.
\par 29 Ugyanazok közül választattak vala némelyek másféle edénynek, a szenthely minden eszközeinek, a lisztnek, bornak, olajnak, tömjénnek és fûszereknek gondviselésére.
\par 30 A papok fiai közül valók csinálják vala fûszerekbõl a drágakenetet is.
\par 31 Továbbá Mattitja, a Léviták közül való (ki elsõszülötte a Kórakhiták közül való Sallumnak), a serpenyõkre visel vala gondot.
\par 32 A Kéhátiták fiai és azok atyjokfiai közül rendeltettek volt a szent kenyérnek gondviselésére, hogy minden szombaton megkészítsék azt.
\par 33 Ezek közül valók valának az éneklõk is, a Léviták közül a családfõk, a kik szabadosok valának egyéb tiszttõl az õ kamarájokban; mert éjjel és nappal szolgálattal tartoznak vala.
\par 34 Ezek a Léviták között családfõk nemzetségök szerint, fõemberek, és ezek Jeruzsálemben laktak.
\par 35 Gibeonban pedig laktak a Gibeon atyja, Jéhiel, és az õ feleségének neve Maaka.
\par 36 Az õ elsõszülött fia Abdon, azután Súr, Kis, Baál, Nér és Nádáb.
\par 37 Gedor, Ahió, Zékária és Miklót.
\par 38 Miklót pedig nemzé Simámot. Ezek is az õ atyjokfiai átellenében laktak Jeruzsálemben az õ testvéreikkel.
\par 39 Nér nemzé Kist, Kis nemzé Sault, Saul pedig nemzé Jonathánt, Málkisuát, Abinádábot és Esbaált.
\par 40 Jonathán fia Méribbaál; és Méribbaál nemzé Mikát.
\par 41 Mika fiai: Piton, Mélek és Táréa.
\par 42 Akház pedig nemzé Jahrát, Jahra nemzé Alémetet, Azmávetet, Zimrit, Zimri pedig nemzé Mósát.
\par 43 Mósa nemzé Bineát, ennek fia Réfája, ennek fia Elása, ennek fia Ásel.
\par 44 Áselnek hat fia volt, kiknek neveik; Azrikám, Bókru, Ismáel, Seárja, Obádia és Hanán; ezek Ásel fiai.

\chapter{10}

\par 1 A Filiszteusok pedig hadakoznak vala az Izráellel, és megfutamodék Izráel népe a Filiszteusok elõtt, és néhányan a sebek miatt el is hullának a Gilboa hegyén.
\par 2 Elérék pedig a Filiszteusok Sault és az õ fiait, és megölék a Filiszteusok Jonathánt, Abinádábot és Malkisuát, a Saul fiait.
\par 3 És a viadal igen heves volt Saul körül, és rátalálván a kézívesek, nyilakkal megsebesíték õt.
\par 4 És monda Saul az õ fegyverhordozójának: Vond ki fegyveredet, és verj által engem vele, mert netalán eljõnek e körülmetéletlenek, és meggyaláznak engemet. De fegyverhordozója nem akará, mert igen fél vala. Ragadá azért Saul a fegyvert, és belé bocsátkozék.
\par 5 Látván pedig az õ fegyverhordozója, hogy Saul immár meghalt, õ is a fegyverbe bocsátkozék, és meghala.
\par 6 Meghala azért Saul és az õ három fia, és egész háznépe is egyetemben meghala.
\par 7 Mikor pedig meglátták Izráelnek minden férfiai, a kik a völgyben valának, hogy õk megfutamodtak, és hogy Saul és az õ fiai megholtak: pusztán hagyák városaikat és elfutának. Akkor eljövének a Filiszteusok, és azokba beszállának.
\par 8 Lõn pedig másodnap, eljövének a Filiszteusok, hogy a holtakat kifoszszák, és megtalálák Sault és az õ fiait, halva feküdvén a Gilboa hegyén.
\par 9 És kifoszták õt, és elvevék fejét és az õ fegyvereit, és elküldék a Filiszteusok minden tartományába köröskörül, hogy hírül adják bálványaiknak és a népnek.
\par 10 Az õ fegyvereit isteneik házába helyezék el, fejét pedig Dágon templomában akasztották fel.
\par 11 Mikor pedig meghallotta az egész Jábesgileád, hogy mit cselekedtek a Filiszteusok Saullal:
\par 12 Feltámadának mindnyájan az erõs férfiak, és elvivék Saulnak és az õ fiainak testét; és Jábesbe vivén, eltemeték azoknak csontjaikat a tölgyfa alatt Jábesben, és bõjtölének hetednapig.
\par 13 Meghala azért Saul az õ gonoszsága miatt, mivel vétkezett az Úr ellen, az Úrnak igéje ellen, melyet nem õrzött meg, sõt az ördöngöst is megkereste, hogy megkérdezze;
\par 14 És nem az Urat kérdé. Ezért elveszté õt, és adá az õ országát Dávidnak, az Isai fiának.

\chapter{11}

\par 1 És gyûlének az Izráeliták mindnyájan Dávidhoz Hebronba mondván:  Ímé mi a te csontod és a te tested vagyunk;
\par 2 Ezelõtt is, még mikor Saul volt a király, te voltál az, aki az Izráelt ki- és bevezetéd, és ezt mondotta a te Urad Istened néked: Te legelteted az én népemet, az Izráelt, és te leszel vezér az én népemen, az Izráelen.
\par 3 Elmenének azért mindnyájan az Izráel vénei a királyhoz Hebronba, és szövetséget tõn Dávid velök Hebronban az Úr elõtt; és királylyá kenték Dávidot Izráel felett, az Úrnak Sámuel által való beszéde szerint.
\par 4 Elméne akkor Dávid és az egész Izráel Jeruzsálembe; ez Jebus (ott a Jebuzeusok voltak a föld lakosai).
\par 5 És mondának Jebus lakói Dávidnak: Ide be nem jössz! De megvevé Dávid a Sion várát; ez a Dávid városa;
\par 6 Mert ezt megmondotta vala Dávid: A ki legelõször egy Jebuzeust levág, elõljáró és vezér legyen. Felméne azért legelõször Joáb, a Séruja fia; és lõn elõljáróvá.
\par 7 Azután lakék Dávid a várban, azért nevezék azt Dávid városának.
\par 8 És megépíté Dávid a várost Millótól fogva egészen körül; Joáb pedig megépíté a városnak maradékát.
\par 9 És folytonosan emelkedék Dávid; mert a Seregek Ura vala õ vele.
\par 10 Ezek pedig a hõsök elõljárói, a kik Dávid mellett valának, a kik erõsen forgolódának vele az õ királyságáért egész Izráellel, hogy királylyá válaszszák õt Izráel felett, az Úr beszéde szerint.
\par 11 Ezek számszerint a hõsök, a kik Dávid körül valának: Jásobeám, Hakhmoni fia, a harmincznak elõljárója; ez emelte vala fel az õ kopjáját háromszáz ellen  a kiket egyszerre megsebesíte.
\par 12 Ezután Eleázár, az Ahóhita Dódó fia; a ki a három hõs közül egy vala.
\par 13 Õ vala Dáviddal Pasdamimban, a hová összegyûlének a Filiszteusok viadalra. Egy darab föld árpával vala tele, és a nép elmenekült a Filiszteusok elõl.
\par 14 De õk megállának ott annak a darab földnek a közepén, és megtartották azt, és megverék a Filiszteusokat, és az Úr megszabadítá õket nagy szabadítással.
\par 15 Továbbá, mikor alámentek hárman a harmincz fõember közül Dávidhoz, a kõsziklához, az Adullám barlangjába; a Filiszteusok pedig tábort járának a Réfaim völgyben.
\par 16 (Dávid az erõsségben vala akkor, a Filiszteusok hada pedig Bethlehemnél.)
\par 17 Kivána Dávid vizet, és monda: Óh, ki adhatna nékem innom a Bethlehem kapuja elõtt való forrásnak vizébõl!
\par 18 Akkor keresztülvágták magokat hárman a Filiszteusok táborán, és vizet merítének a Bethlehem forrásából, mely a kapu elõtt vala és felvivék, és menének Dávidhoz; de Dávid nem akara inni, hanem kitölté azt az Úrnak.
\par 19 És monda: Távoztassa el tõlem az én Istenem, hogy ezt cselekedném. Avagy e férfiaknak vérét igyam-é meg, a kik életöket halálra vetették? Mert õk ezt életök veszedelmével hozták. És semmiképen nem akara inni. Ezt cselekedé a három hõs.
\par 20 És Abisai, a Joáb testvére vala e háromnak elõljárója; õ ragadott vala dárdát háromszáz  ellen, a kiket megsebesíte; és õ volt a leghíresebb a három között.
\par 21 A három közül a kettõnél híresebb vala, azért volt azok elõljárója; de azért egymaga még sem ért föl a hárommal.
\par 22 Benája, a Jojada fia, vitéz férfiúnak fia Kabséelbõl, a ki nagy dolgokat cselekedék; õ ölte meg Moáb két oroszlánját, és õ méne alá a verembe, s ölte meg az oroszlánt, mikor havas idõ volt.
\par 23 Ugyanez ölé meg az Égyiptombeli férfit, akinek magassága öt sing volt, és oly dárda vala az Égyiptombeli férfiú kezében, mint a szövõ zugoly. Szembeszálla vele egy pálczával, s kiragadá az Égyiptombeli ember kezébõl a dárdát, és saját dárdájával általveré.
\par 24 Ezeket cselekedte Benája, a Jojada fia, a ki híres vala a három hõs között.
\par 25 Híres vala õ a harmincz között, de azzal a hárommal nem ért fel. És elõljáróvá tevé õt Dávid a tanácsosok között.
\par 26 A seregnek pedig ezek vitézei: Asael, a Joáb testvére; Elhanán, Dódónak fia, ki Bethlehembeli vala.
\par 27 Haróritból való Sammót, Pélomból való Héles,
\par 28 Tékoabeli Hira, Ikkés fia, Anatótbeli Abiézer,
\par 29 Húsatbeli Sibbékai, Ahóhitbeli Hirai.
\par 30 Nétofátbeli Maharai, Nétofátbeli Héled, Bahána fia;
\par 31 Ittai, Ribai fia, a Benjámin fiainak Gibea városából való; Pirátonbeli Benája;
\par 32 Húrai, a Gaás völgyébõl való; Arbátbeli Abiel;
\par 33 Baharumbeli Azmávet, Saálbonitbeli Eliáhba;
\par 34 Gisonbeli Hásem fiai: Jonathán, Hararitbeli Ságé fia.
\par 35 Hararitbeli Ahiám, Sákár fia; Elifál, Úr fia,
\par 36 Mekerátbeli Héfer, Pélonbeli Ahija,
\par 37 Kármelbõl való Hésró, Naárai, Ezbái fia,
\par 38 Jóel, Nátán testvére; Mibhár, Géri fia;
\par 39 Sélek, Ammon nemzetségébõl való; Berótbeli Naárai, Joábnak, a ki Séruja fia vala, fegyverhordozója;
\par 40 Itrébeli Hira, Itrébeli Gáreb,
\par 41 Hitteus Uriás, Zabád, Ahlai fia;
\par 42 Hadina, a Rúben nemzetségébõl való Siza fia, ki a Rúbeniták elõljárója vala, és vele harminczan valának.
\par 43 Hanán, Maaka fia, és Mitnibeli Jósafát,
\par 44 Asterátbeli Uzzija; Sáma és Jéhiel, Aróerbeli Hótám fiai.
\par 45 Jidiháel, Simri fia, és az õ testvére, Joha, Tisibeli.
\par 46 Mihávimbeli Eliel, Jéribai és Jósávia, Elnaám fia, és Jitma, Moáb nemzetségébõl való.
\par 47 Eliel és Obed és Jaásiel, Mésóbájából valók.

\chapter{12}

\par 1 Ezek azok, a kik Dávidhoz menének Siklágba, mikor Saul, a Kis fia miatt még számkivetésben vala, a kik a hõsöknek a harczban segítõi voltak.
\par 2 Ívesek, akik mind jobb-, mind balkézre kõvel hajítanak és nyíllal lõnek vala, a kik Saul atyjafiai közül valók valának, Benjámin nemzetségébõl.
\par 3 Elõljáró vala Ahiézer és Joás, a Gibeabeli Semáa fiai és Jéziel és Pélet, Azmávet fiai, Beráka és Jéhu, Anatótból,
\par 4 És a Gibeonbeli Ismája, a harmincz közül való hõs, a kiknek elõljárójok is vala; Irméja, Jaháziel, Johanán és Gederátbeli Józabád,
\par 5 Elúzai, Jérimót, Behália, Semária és Hárufbeli Sefátja,
\par 6 Elkána, Isija, Azaréel, Jóézer és a Kóré nemzetébõl való Jásobéám.
\par 7 Joéla és Zebádja, a Gedorból való Jérohám fiai.
\par 8 A Gáditák közül is menének Dávidhoz, mikor a pusztában vala az erõsségben, erõs és hadakozó férfiak, paizsosok, dárdások, a kiknek orczájok, mint az oroszlánnak orczája és gyorsaságra hasonlók a hegyen lakozó  vadkecskékhez.
\par 9 Ézer az elsõ, Obádia második, Eliáb harmadik,
\par 10 Mismanna negyedik, Jirméja ötödik,
\par 11 Attai hatodik, Eliel hetedik,
\par 12 Nyolczadik Johanán, kilenczedik Elzabád,
\par 13 Tizedik Jirméja, tizenegyedik Makbánnai.
\par 14 Ezek voltak fõemberek a seregben a Gád fiai közül; a legkisebbek egyike száz ellen, a legnagyobbak egyike ezer ellen!
\par 15 Ezek azok, a kik a Jordánon átmentek volt az elsõ hónapban, noha az árvíz a partot felülmúlta, és elûzték mindazokat, a kik a völgyben valának napkelet felõl és napnyugot felõl.
\par 16 Jövének Dávidhoz a Benjámin és a Júda fiai közül is az erõsségbe.
\par 17 És kiméne Dávid elejökbe, és felelvén, monda nékik: Hogyha békesség okáért jöttök hozzám, hogy segítségemre legyetek, az én szívem egy lesz ti veletek; ha pedig meg akartok csalni, hogy eláruljatok az én ellenségeimnek, holott semmi gonoszságot nem követtem el: lássa meg a mi atyáink Istene és büntessen meg.
\par 18 A lélek pedig felindítá Amásait, a harmincznak  fejedelmét, s monda: Óh Dávid, tied vagyunk és te veled leszünk, Isai fia! Békesség, békesség néked, békesség a te segítõidnek is, mert megsegít téged a te Istened! Magához fogadá azért õket Dávid, és fõemberekké tevé a seregben.
\par 19 Ennekfelette Manasséból is hajlának Dávidhoz, mikor a Filiszteusokkal együtt Saul ellen ment volna harczolni; de nem segéllék õket; mert tanácsot tartván, haza küldék a Filiszteusok fejedelmei, mondván: A mi fejünk veszésével fog visszamenni az õ urához, Saulhoz.
\par 20 Mikor visszatére Siklágba, hajlának õ hozzá a Manassé fiai közül Adna, Józabád, Jediháel, Mikáel, Józabád, Elihu és Sillétai, a kik a Manasse nemzetségébõl való ezerek elõljárói voltak.
\par 21 És ezek Dávidnak segítségül voltak az ellenség seregei ellen; mert fejenként mind erõs vitézek valának, és vezérek a seregben.
\par 22 Annakfelette minden nap mennek vala Dávidhoz, hogy segítségére legyenek néki, míg serege nagygyá lõn, mint az Istennek tábora.
\par 23 Ezek pedig számszerint a viadalhoz készült elõljárók, a kik Dávidhoz mentek vala Hebronban, hogy õt Saul helyett az országban királylyá válaszszák, az Isten ígérete szerint.
\par 24 A Júda fiai közül, a kik paizst és kopját viselének hatezernyolczszáz vala harczra készen.
\par 25 A Simeon fiai közül vitéz férfiak a viadalra, hétezerszáz.
\par 26 A Lévi fiai közül négyezerhatszáz vala.
\par 27 Jojada is, ki az Áron fiai között elõljáró vala, és õ vele háromezerhétszáz.
\par 28 És az ifjú Sádók, a ki igen erõs vala, és az õ atyja házából huszonkét fõember.
\par 29 A Benjámin fiai közül, a kik Saul atyjafiai valának, háromezer; mert még azok közül sokan hûségesen õrizik vala a Saul házát.
\par 30 Az Efraim fiai közül húszezernyolczszáz, igen vitézek, a kik az õ nemzetségökben híres férfiak valának;
\par 31 Manassénak félnemzetségébõl pedig tizennyolczezer, kik névszerint kijelöltetének, hogy elmenjenek és Dávidot királylyá válaszszák.
\par 32 Az Izsakhár fiai közül, a kik felismerék az idõ alkalmatos voltát, hogy tudnák, mit kellene Izráelnek cselekednie, kétszáz fõember és az õ rokonaik mind hallgatnak vala beszédjökre.
\par 33 A Zebulon fiai közül a harczra kimenõk, minden hadiszerszámokkal felkészülve, ötvenezeren valának, készek a viadalra állhatatos szívvel.
\par 34 A Nafthali nemzetségébõl ezer fõember vala; és õ velek paizszsal s kopjával harminczhétezer vala.
\par 35 A Dániták közül, a kik a viadalhoz készek valának, huszonnyolczezerhatszázan voltak.
\par 36 És az Áser fiai közül a hadakozók és az ütközethez készek negyvenezeren valának.
\par 37 A Jordánon túl lakozók közül, azaz a Rúbeniták, Gáditák és a Manasse nemzetségének fele közül, minden viadalhoz való szerszámokkal egyetemben, jöttek százhúszezeren.
\par 38 Mindezek hadakozó férfiak, a viadalra elkészülve, egy értelemmel mentek vala Hebronba, hogy Dávidot az egész Izráel felett királylyá válaszszák, sõt ezeken kivül is az egész Izráel egy szívvel azon volt, hogy Dávidot királylyá válaszszák.
\par 39 És ott maradának Dáviddal harmadnapig, s esznek és isznak vala; mert az õ atyjokfiai készítettek vala nékik;
\par 40 És úgy a szomszédságukban levõk, mint mások Izsakhárig, Zebulonig és Nafthaliig hoznak vala kenyereket szamarakon, tevéken, öszvéreken és ökrökön, eleséget, lisztet, fügét, aszuszõlõt, bort, olajat, vágóbarmokat, juhokat számtalan sokat; mert nagy öröm vala Izráelben.

\chapter{13}

\par 1 Tanácsot tarta pedig Dávid az ezredeknek és századoknak fejeivel és minden elõljárókkal.
\par 2 És monda Dávid Izráel egész gyülekezetének: Ha néktek tetszik, és ha az Úrtól, a mi Istenünktõl van: küldjünk el mindenfelé a mi atyánkfiaihoz, a kik otthon maradtak Izráel minden tartományaiban, s velök együtt a papokhoz és Lévitákhoz, az õ városaik és vidékeik szerint, hogy õk is gyûljenek hozzánk.
\par 3 Hogy hozzuk ide a mi Istenünknek ládáját mi hozzánk, mert a Saul idejében nem törõdtünk vele.
\par 4 És monda az egész gyülekezet, hogy úgy kell cselekedni; mert igaznak láttaték e dolog az egész nép elõtt.
\par 5 Összegyûjté azért Dávid mind az Izráel népét Égyiptomnak Nilus folyóvizétõl fogva egészen Hámátig, hogy az Istennek ládáját elhozzák Kirjáth-Jeárimból.
\par 6 Felméne azért Dávid és vele az egész Izráel Baalába vagy Kirjáth-Jeárimba, a mely Júdában van, hogy onnan elhozzák az Úr Istennek ládáját, mely az õ nevérõl neveztetik, a ki a  Khérubok közt ül.
\par 7 És helyhezteték az Isten ládáját az Abinádáb házából egy új szekérre; Uzza és Ahió vezetik vala a szekeret.
\par 8 Dávid pedig és az egész Izráel tánczolnak vala az Isten elõtt teljes erõvel, énekekkel, cziterákkal, hegedûkkel, dobokkal, czimbalmokkal és kürtökkel.
\par 9 Mikor pedig jutottak a Kidon szérûjéhez, Uzza reá tevé kezét a ládára, hogy megtartsa, mert az ökrök félre tértek vala.
\par 10 És az Úrnak haragja felgerjede Uzza ellen, és õt megveré, hogy reá tevé kezét a ládára, s meghala ugyanott az Isten elõtt.
\par 11 Akkor Dávid igen megdöbbene, hogy az Úr ily csapással sujtá Uzzát. Azért azt a helyet mind e mai napig Péres-Uzzának nevezik.
\par 12 És félni kezde Dávid azon a napon Istentõl, mondván: Miképen merjem magamhoz bevinni az Isten ládáját?!
\par 13 És nem vivé be Dávid magához a ládát a Dávid városába, hanem elhelyezé azt a Gitteus Obed-Edom házában.
\par 14 És az Isten ládája Obed-Edom házában volt három hónapig; és megáldá az Úr Obed-Edom házát és mindenét, valamije volt.

\chapter{14}

\par 1 Külde pedig Hírám, Tírus királya Dávidhoz követeket: czédrusfákat, kõmíveseket és ácsmestereket, hogy néki házat csináljanak.
\par 2 És megismeré Dávid, hogy az Úr õt megerõsítette a királyságban Izráel felett, mivelhogy igen felmagasztaltatott vala az õ országa az õ népéért, az Izráelért.
\par 3 Võn pedig Dávid még több feleséget Jeruzsálemben, és nemze Dávid még több fiakat és leányokat.
\par 4 És ezek neveik azoknak, a kik születtek néki Jeruzsálemben: Sammua, Sobáb,  Nátán és Salamon,
\par 5 Ibhár, Elisua és Elpélet,
\par 6 Nógah, Néfeg és Jáfia,
\par 7 Elisáma, Beheljada és Elifélet.
\par 8 Megértvén pedig a Filiszteusok, hogy Dávid királylyá kenetett fel az egész Izráel felett, feljövének mind a Filiszteusok, hogy megkeressék Dávidot. Mely dolgot mikor Dávid meghallott, azonnal ellenök ment.
\par 9 A Filiszteusok pedig eljövén, elszéledének a Réfaim völgyében.
\par 10 Akkor Dávid megkérdé az Istent, mondván: Felmenjek-é a Filiszteusok ellen, és kezembe adod-é õket? És monda néki az Úr: Menj fel, és kezedbe adom õket.
\par 11 Felmenének azért Baál-perásimba, és ott Dávid õket megveré, és monda Dávid: Az Isten eloszlatá az én ellenségeimet az én kezem által, mint a vizeknek oszlását; azért nevezék azt a helyet Baál-perásimnak.
\par 12 És ott hagyák az õ bálvány isteneiket, és Dávid megparancsolá, hogy megégessék azokat.
\par 13 Ismét feltámadának a Filiszteusok és a völgyben szétterjeszkedtek.
\par 14 Azért Dávid ismét megkérdé az Istent, és monda az Isten néki: Ne menj fel utánok, hanem fordulj el tõlök, és menj reájok a szederfák irányában;
\par 15 És mikor hallándod a járás dobogását a szederfák tetején, akkor menj ki a viadalra; mert az Isten te elõtted megyen, hogy megverje a Filiszteusok táborát.
\par 16 Dávid azért úgy cselekedék, a mint néki Isten meghagyta volt, és vágák a Filiszteusok táborát Gibeontól fogva szinte Gézerig.
\par 17 És elterjede Dávid híre az országokban, és az Úr adá a tõle való félelmet minden pogányokra.

\chapter{15}

\par 1 Csináltata pedig Dávid magának házakat az õ városában; és helyet készített az Isten ládájának, és annak sátort állított fel.
\par 2 Akkor monda Dávid: Nem szabad másnak hordozni az Isten ládáját, hanem csak a Lévitáknak, mert az Úr õket választotta, hogy hordozzák az Isten ládáját, és néki szolgáljanak mindörökké.
\par 3 Összegyûjté azért Dávid Jeruzsálembe az egész Izráel népét, hogy az Úr ládáját az õ helyére vitesse, melyet számára csináltatott vala.
\par 4 Összegyûjté Dávid az Áron fiait is és a Lévitákat.
\par 5 A Kéhát fiai között fõ vala Uriel, és az õ atyjafiai százhúszan valának.
\par 6 A Mérári fiai között Asája volt a fõ, és az õ atyjafiai kétszázhúszan valának.
\par 7 A Gerson fiai között Jóel volt a fõ, és az õ atyjafiai százharminczan valának.
\par 8 Az Elisáfán fiai között Semája volt a fõ, és az õ atyjafiai kétszázan valának,
\par 9 A Hebron fiai között Eliel volt a fõ, és az õ atyjafiai nyolczvanan valának.
\par 10 Az Uzziel fiai között fõ vala Amminádáb, és az õ atyjafiai száztizenketten.
\par 11 Hivatá akkor Dávid Sádók és Abjátár papokat, a Léviták közül pedig Urielt, Asáját, Jóelt, Sémáját, Elielt és Amminádábot.
\par 12 És monda nékik: Ti vagytok a Léviták családfõi. Szenteljétek meg magatokat s a ti atyátokfiait, és vigyétek az Úrnak, Izráel Istenének ládáját arra a helyre, a melyet készítettem számára.
\par 13 Minthogy kezdettõl fogva nem mívelték ezt, az Úr, a mi Istenünk csapást bocsátott reánk, mert nem kerestük õt a rendtartás szerint.
\par 14 Megszentelék azért magokat a papok és a Léviták, hogy vigyék az Úrnak, Izráel Istenének ládáját.
\par 15 És felvevék a Léviták fiai az Isten ládáját, úgy, a mint Mózes meghagyta volt az Úrnak beszéde szerint, a rudakkal vállaikra.
\par 16 És monda Dávid a Léviták fejedelmeinek, hogy állítsanak az õ atyjokfiai közül éneklõket, éneklõszerszámokkal, lantokkal, cziterákkal és czimbalmokkal, hogy énekeljenek felemelt szóval, nagy örömmel.
\par 17 Választák azért a Léviták Hémánt a Jóel fiát, és az õ atyjafiai közül Asáfot, a Berekiás fiát, és a Mérári fiai közül, a kik azoknak atyjokfiai valának, Etánt, a Kúsája fiát.
\par 18 És õ velök együtt az õ atyjokfiait másod renden, Zakariást, Bént, Jeázielt, Semirámótot, Jéhielt, Unnit, Eliábot, Benáját, Maaséját, Mattithját, Elifélet, Miknéját, Obed-Edomot és Jehielt, a kik ajtónállók valának.
\par 19 Éneklõk: Hémán, Asáf  és Etán, réz czimbalmokkal, hogy zengedezzenek;
\par 20 Zakariás, Aziel, Semirámót, Jéhiel, Unni, Eliáb, Maaséja és Benája lantokkal a szûzek módjára;
\par 21 Mattithja, Eliféle, Miknéja, Obed-Edom, Jéhiel és Azaziás, hogy énekeljenek cziterákkal a nyolczhúrú szerint.
\par 22 És Kénániás volt a Léviták vezére az éneklésben, õ igazgatá az éneklést, mivel tudós vala.
\par 23 Berekiás és Elkána a láda elõtt való ajtónállók valának:
\par 24 Sébániás pedig és Jósafát, Nétenéel, Amásai, Zakariás, Benája és Eliézer papok kürtölnek vala az Isten ládája elõtt; Obed-Edom és Jéhija, a kik kapunállók valának, a láda után mennek vala.
\par 25 Dávid pedig s az Izráel vénei és az ezredek vezérei, a kik elmenének, hogy felvigyék az Úr szövetségének ládáját az Obed-Edom házából, nagy örömben valának.
\par 26 Lõn pedig, mikor az Isten megsegíté a Lévitákat, akik az Úr szövetségének ládáját viszik vala, áldozának hét tulokkal és hét kossal.
\par 27 Dávid pedig bíborból csinált ruhába öltözvén, valamint a Léviták mind, a kik a ládát viszik vala, az éneklõk is és Kénániás, a ki az éneklés vezére vala, énekelének (Dávidon pedig gyolcsból csinált efód vala).
\par 28 És az egész Izráel vivé az Úr szövetségének ládáját nagy örömmel, kürtökkel, trombitákkal, czimbalmokkal, zengedezvén lantokkal és cziterákkal.
\par 29 Mikor pedig immár az Úr szövetségének ládája a Dávid városába jutott, Mikál a Saul leánya kitekinte az ablakon, s látván, hogy Dávid király tánczol és vígad, szívében megutálá õt.

\chapter{16}

\par 1 Mikor pedig bevitték az Isten ládáját és elhelyezék azt a sátor közepén, a melyet Dávid annak számára felállított vala: áldozának egészen égõ- és hálaáldozatokkal az Isten elõtt.
\par 2 És mikor Dávid elvégezte az egészen égõáldozatot és a hálaáldozatot, az Úr nevében megáldá a népet.
\par 3 És osztogata minden Izráelitának, férfinak úgy, mint asszonynak egy-egy kenyeret, és egy-egy darab húst és egy-egy kalácsot.
\par 4 És rendele az Úr ládája elé a Léviták közül szolgákat, a kik hirdessék, tiszteljék és dícsérjék az Urat, Izráel Istenét.
\par 5 Asáf vala a fõ, utána másodrenden Zakariás, Jéhiel, Semirámót, Jékhiel, Mattithja, Eliáb, Benája és Obed-Edom. Jéhiel lantokkal és cziterákkal, Asáf pedig czimbalmokkal énekel vala;
\par 6 Továbbá Benája és Jaháziel papok kürtölnek vala szüntelen az Isten szövetségének ládája elõtt.
\par 7 Azon a napon adott Dávid elõször éneket az Úrnak dícséretére Asáfnak és az õ atyjafiainak kezébe.
\par 8 Dícsérjétek az Urat, hívjátok segítségül az õ nevét, hirdessétek minden népek között az õ nagy dolgait.
\par 9 Énekeljetek néki, mondjatok dícséretet néki, beszéljetek minden csudálatos dolgairól.
\par 10 Dicsekedjetek az õ szent nevében; örvendezzen szívök azoknak, a kik az Urat keresik.
\par 11 Keressétek az Urat és az õ erõsségét; keressétek az õ orczáját szüntelen.
\par 12 Emlékezzetek meg az õ csudálatos dolgairól, a melyeket cselekedett, az õ csudáiról és az õ szájának ítéletirõl.
\par 13 Óh Izráelnek, az õ szolgájának magva! Jákóbnak, az õ választottjának fiai!
\par 14 Ez az Úr, a mi Istenünk; az egész földön az õ ítéletei!
\par 15 Emlékezzetek meg örökké az õ szövetségérõl, és az õ beszédérõl, a melyet parancsolt, ezer nemzetségig;
\par 16 A melyet szerzett Ábrahámmal; és az  Izsáknak tett esküjérõl.
\par 17 Amelyet állíta Jákóbnak örök végzésül, Izráelnek örökkévaló szövetségül,
\par 18 Mondván: A Kanaán földét néked adom, hogy legyen néktek örökségtek.
\par 19 Midõn ti számszerint kevesen valátok, igen kevesen, és zsellérek azon a földön;
\par 20 Mert járnak vala egyik nemzetségtõl a másikhoz, és egyik országból más országba:
\par 21 Mégsem engedé senkinek õket bántani, sõt még a királyokat is megbünteté  érettök.
\par 22 Ezt mondván: Az én felkentjeimet ne bántsátok, prófétáimnak se ártsatok.
\par 23 Mind ez egész föld énekeljen az Úrnak, napról-napra hirdessétek az õ szabadítását.
\par 24 Beszéljétek a pogányok között az õ dicsõségét, minden népek között az õ csudálatos dolgait;
\par 25 Mert nagy az Úr és igen dícsérendõ, és rettenetes minden istenek felett;
\par 26 Mert a pogányoknak minden isteneik csak bálványok, de az Úr teremtette az egeket.
\par 27 Dicsõség és tisztesség van õ elõtte, erõsség és vígasság az õ helyén.
\par 28 Adjatok az Úrnak, népeknek nemzetségei, adjatok az Úrnak dicsõséget és erõsséget!
\par 29 Adjatok az Úr nevének dicsõséget, hozzatok ajándékot, és jõjjetek eleibe, imádjátok az Urat a szentség ékességében.
\par 30 Rettegjen az egész föld az õ orczájától; a föld kereksége is megerõsíttetik, hogy ne ingadozzék.
\par 31 Örüljenek az egek, és örvendezzen a föld, és mondják a pogányok között: az Úr uralkodik!
\par 32 Zengjen a tenger és az õ teljessége; örvendezzen a mezõ és minden, a mi azon van.
\par 33 Akkor örvendezni kezdenek az erdõnek fái az Úr elõtt, mikor eljövend megítélni a földet.
\par 34 Tiszteljétek az Urat, mert igen jó, mert örökkévaló az õ irgalmassága.
\par 35 És mondjátok: Tarts meg minket, mi szabadító Istenünk, gyûjts össze minket, és szabadíts meg a pogányoktól, hogy a te szent nevedet tisztelhessük, dicsekedhessünk a te dícséretedben!
\par 36 Áldott legyen az Úr, Izráel Istene öröktõl fogva mindörökké! És monda a sokaság: Ámen! és dícséré az Urat.
\par 37 Ott hagyá azért Dávid az Úr szövetségének ládájánál Asáfot és az õ atyjafiait, hogy a láda elõtt szüntelen minden napon szolgáljanak,
\par 38 Obed-Edomot és az õ hatvannyolcz atyjafiát (Obed-Edom pedig a Jedithun fia) és Hósát pedig ajtónállóknak.
\par 39 Sádók papot pedig és az õ pap atyjafiait, az Úr sátora elõtt hagyá a magaslaton, mely Gibeonban  vala;
\par 40 Hogy áldozzanak az Úrnak szüntelen égõáldozattal az égõáldozatnak oltárán minden reggel és estve, és hogy mindent a szerint cselekedjenek, a mint megiratott az Úr törvényében, melyet parancsolt vala az Izráelnek,
\par 41 Hémánt is és Jédutunt velök hagyá, és többeket is választott, a kik nevök szerint megneveztettek, hogy az Urat dícsérjék, mert az õ irgalmassága örökkévaló.
\par 42 És õ velök Hémánt és Jédutunt kürtökkel, czimbalmokkal és az Isten énekének szerszámaival. A Jédutun fiait pedig kapunállókká tevé.
\par 43 Akkor eltávozék az egész nép, kiki az õ házához. Dávid pedig visszatére, hogy az õ háznépét is megáldja.

\chapter{17}

\par 1 Lõn pedig, mikor Dávid az õ házában ülne, monda Nátán prófétának: Ímé én czédrusfából csinálta házban lakom, az Úr  szövetségének ládája pedig kárpitok alatt.
\par 2 Akkor monda Nátán Dávidnak: Valami a te szívedben van, cselekedd meg, mert az Isten veled leend.
\par 3 Azon éjjel pedig lõn az Istennek szava Nátánhoz, mondván:
\par 4 Menj el, és mondd meg az én szolgámnak, Dávidnak: Ezt mondja az Úr: Ne te építs nékem házat lakásul;
\par 5 Mert nem laktam én attól fogva házban, mióta az Izráel fiait kihoztam, mind e mai napig, hanem egy hajlékból más hajlékba mentem és sátorból sátorba.
\par 6 A mely helyeken jártam az Izráel egész népével, szólottam-é vagy egyszer valakinek az Izráel birái közül (a kiknek parancsoltam vala, hogy az én népemet legeltessék), mondván: Miért nem csináltatok nékem czédrusfából házat?
\par 7 Most azért ezt mondjad az én szolgámnak, Dávidnak: Ezt mondja a Seregek Ura: Én választottalak téged a juhok mellõl a pásztorkunyhóból, hogy légy vezére az én népemnek, az Izráelnek,
\par 8 És veled voltam mindenütt, valahová mentél, minden ellenségeidet is a te orczád elõl elvesztettem, ennekfelette oly hírt szerzettem néked, a minemû hírök van a hatalmasoknak, a kik a földön vannak;
\par 9 Lakóhelyet is adtam az én népemnek, az Izráelnek és elplántálám õt; és lakik az õ helyén, és ki nem mozdul többé,  s nem fogják az álnokságnak fiai sanyargatni, mint azelõtt.
\par 10 És attól az idõtõl fogva, hogy megparancsoltam volt, hogy birák legyenek az én népem, az Izráel felett, minden te ellenségeidet megalázám, és azt is jelentém néked, hogy az Úr házat épít néked.
\par 11 És lészen, mikor betelnek a te életed napjai, hogy a te atyáidhoz elmenj, a te magodat feltámasztom te utánad, mely a te fiaid közül való lesz, és az õ országát megerõsítem.
\par 12 Õ épít nékem házat, és megerõsítem az õ királyi székét mindörökké.
\par 13 Én leszek néki atyja, õ pedig fiam lészen, és az én irgalmasságomat õ tõle el nem veszem, mint a hogy a te elõtted  valótól elvettem;
\par 14 Hanem megerõsítem õt az én házamban és az én országomban mindörökké, és az õ királyiszéke erõs lesz mindörökké.
\par 15 Mind e beszédek szerint és mind e látás szerint szóla Nátán Dávidnak.
\par 16 Beméne azért Dávid király, és leüle az Úr elõtt, és monda: Ki vagyok én, óh Uram Isten, s micsoda az én házam is, hogy engemet eddig juttattál?
\par 17 Sõt még ez is kevés volt elõtted, óh Isten! hanem ennekfelette szólál jövendõt is a te szolgád háza felõl, és mint magas rangú embert, úgy tekintettél engemet, Úr Isten!
\par 18 És mit kérhetne Dávid többet te tõled, a te szolgádnak tisztességére, holott te jól ismered a te szolgádat?
\par 19 Óh Uram, a te szolgádért és a te szíved szerint cselekedéd mind e nagy dolgokat, hogy kijelentéd mindezeket a csudálatos dolgokat,
\par 20 (Óh Uram, nincsen senki hasonló hozzád, és nincsen Isten náladnál több), mind a szerint, a mint füleinkkel hallottuk.
\par 21 És kicsoda olyan, mint a te néped, az Izráel, egy nemzetség a földön, a melyért elment volna az Isten, hogy megváltaná magának népül; hogy magadnak nagy és rettenetes nevet szerezz, kiûzvén a pogányokat a te néped elõl, a melyet Égyiptomból megszabadítál!
\par 22 És az Izráel népét a te népeddé tevéd mindörökké, és te Uram, nékik Istenök lettél.
\par 23 Most azért Uram, a szó, a melyet szólál a te szolgád felõl és az õ háza felõl, erõsíttessék meg mindörökké, és úgy cselekedjél, a mint szóltál.
\par 24 Maradjon meg és magasztaltassék fel a te neved mindörökké, hogy mondhassák: A Seregek Ura az Izráel Istene, Istene Izráelnek; és Dávidnak, a te szolgádnak háza legyen állandó elõtted.
\par 25 Minthogy te, én Istenem, a te szolgádnak füle hallására megjelentetted, hogy néki házat csinálsz: ezokáért indula meg a te szolgád, hogy könyörögne elõtted.
\par 26 Most azért én Uram, te vagy az Isten, és te szólád e jó dolgot a te szolgád felõl.
\par 27 Most azért tessék néked megáldani a te szolgádnak házát, hogy legyen állandó mindörökké elõtted; mivelhogy te, Uram, megáldottad, legyen azért áldott mindörökké.

\chapter{18}

\par 1 Ezek után pedig megveré Dávid a Filiszteusokat, és megalázá õket, és Gáth városát faluival együtt elvevé a Filiszteusok kezébõl.
\par 2 A Moábitákat is megveré; és a Moábiták Dávidnak adófizetõ szolgái lettek.
\par 3 Továbbá megveré Dávid Hámátban Hadadézert is, Sóbának királyát, mikor elindult vala, hogy az Eufrátes folyóvízig vesse birodalmának határát.
\par 4 Dávid pedig nyere õ tõle ezer szekeret és hétezer lovast és húszezer gyalogost; és minden szekeres lovaknak inait elvagdaltatá Dávid, és csak száz szekérbe valót tarta meg azokból.
\par 5 És mikor eljövének Damaskusból a Siriabeliek, hogy megsegéllenék Hadadézert, a Sóba királyát: levága Dávid a Siriabeliek közül huszonkétezer férfit.
\par 6 És rendele Dávid tiszttartókat Siriában, a hol Damaskus van; és lettek a Siriabeliek Dávidnak adófizetõ szolgái. És megsegíté az Úr Dávidot mindenütt, a merre csak ment.
\par 7 És vevé Dávid az arany paizsokat, a melyeket a Hadadézer szolgái viselének, és vivé Jeruzsálembe.
\par 8 És hoza Dávid Tibhátból és Kúnból, a Hadadézer városaiból igen sok rezet. Abból csinálá Salamon a  réztengert, az oszlopokat és a rézedényeket.
\par 9 Mikor meghallotta volna pedig Tóhu, a Hamáthbeli király, hogy Dávid megverte Hadadézernek, a Sóba királyának egész hadát:
\par 10 Küldé Dávid királyhoz az õ fiát, Hadorámot, hogy köszöntené õt és megáldaná, mivelhogy megharczolt vala Hadadézerrel és megverte vala õt; mert Tóhu is hadakozik Hadadézer ellen. És külde Dávidnak sok arany, ezüst és rézedényt.
\par 11 És Dávid király ezeket is az Úrnak szentelé, az ezüsttel és az aranynyal együtt, a melyet nyert vala minden pogányoktól, az Edomitáktól, a Moábitáktól, az Ammon fiaitól, a Filiszteusoktól és Amálekitáktól.
\par 12 Abisai pedig, a Séruja fia, az Edoniták közül a Sónak völgyében megvere tizennyolczezeret.
\par 13 És rendele Edomban Dávid tiszttartókat, és lettek az Edombeliek mind Dávidnak szolgái. És megsegíté az Úr Dávidot mindenütt, a merre csak ment.
\par 14 Így uralkodék Dávid az egész Izráel felett, és ítéletet és igazságot szolgáltat vala egész népének.
\par 15 Joáb pedig, Séruja fia, a sereg elõljárója volt, Jósafát, az Ahilud fia pedig emlékíró.
\par 16 Sádók, az Akhitub fia és Abimélek, az Abjátár fia, voltak a papok; Sausa az íródeák.
\par 17 És Benája, a Jójada fia, a Kereteusok és Peleteusok elõljárója volt; a Dávid fiai pedig elsõk a király mellett.

\chapter{19}

\par 1 Történt ezután, hogy meghalt Náhás, az Ammon fiainak királya, és uralkodék az õ fia helyette.
\par 2 És monda Dávid: Minden jóval leszek a Náhás fiához, Hánunhoz; mert az õ atyja is jól tett volt velem. Ezért Dávid követeket külde õ hozzá, a kik vigasztalnák õt az atyjának halála miatt; és elmenének a Dávid szolgái az Ammon fiainak földére Hánunhoz, hogy õt vigasztalnák.
\par 3 Akkor mondának az Ammon fiainak fõemberei Hánunnak: Azt hiszed-é, hogy Dávid atyád iránti tiszteletbõl küldött hozzád vigasztalókat? Avagy nem azért jöttek-é az õ szolgái hozzád, hogy a földet megvizsgálják, elpusztítsák és kikémleljék?
\par 4 Megfogatá azért Hánun a Dávid szolgáit és azokat megnyiratá, és ruhájokat félig elmetszeté az õ derekukig, és úgy bocsátá el õket.
\par 5 Elmenének pedig és értesíték Dávidot, hogy mi történt a férfiakkal. És külde eléjök (mert azok a férfiak felette nagy gyalázattal illettettek vala) és ezt izené a király: Maradjatok Jérikhóban, míg szakállatok megnövénd, akkor jõjjetek hozzám.
\par 6 Látván pedig az Ammon fiai, hogy tisztességtelen dolgot cselekedtek vala Dáviddal, küldének Hánun és az Ammon fiai ezer tálentom ezüstöt, hogy fogadjanak szekereket és lovagokat Mésopotámiából, siriai Maakából és Sóbából.
\par 7 Fogadának azért magoknak harminczkétezer szekeret, Maakának királyát is az õ népével együtt, a kik eljövének és tábort járának Medeba elõtt. Az Ammon fiai is összegyûlének az õ városaikból és jövének az ütközetre.
\par 8 A mit mikor meghallott Dávid, elküldé Joábot és a vitézek egész seregét.
\par 9 És az Ammon fiai kimenvén, csatarendbe állának a város kapuja elõtt; a mely királyok pedig segítségre jöttek vala, külön valának a mezõn.
\par 10 Látván pedig Joáb, hogy mind elõl, mind hátul ellenség állana, kiválaszta az egész Izráel harczosai közül egynéhányat, és a Siriabeliek ellen rendelé.
\par 11 A nép többi részét pedig testvérére, Abisaira bízá, és ezek az Ammon fiaival állának szembe.
\par 12 És monda: Ha a Siriabeliek rajtam erõt vennének, légy segítségemre; ha pedig az Ammon fiai rajtad vennének erõt, én is megsegéllek.
\par 13 Légy erõs, sõt legyünk bátrak mindnyájan a mi népünkért és a mi Istenünk  városaiért; az Úr pedig cselekedje azt, a mi néki tetszik.
\par 14 Harczra indula azért Joáb és az õ hada a Siriabeliek ellen, a kik õ elõtte megfutamodának.
\par 15 Az Ammon fiai pedig mikor látták, hogy megfutamodának a Siriabeliek: õk is megfutamodának Abisai elõl az õ testvére elõl, és a városba menekülének; Joáb pedig visszatért Jeruzsálembe.
\par 16 Látván pedig a Siriabeliek, hogy az Izráel elõtt megverettetének, követeket küldének és kihozatták a Siriabelieket, a kik a folyóvizen túl laknak vala, és Sófák, a Hadadézer seregeinek vezére volt az elõljárójuk.
\par 17 Mikor pedig hírül adák Dávidnak, összegyûjté az egész Izráelt, és a Jordán vizén átmenvén, hozzájok érkezék és csatarendbe állott ellenök. És mikor csatarendbe állott Dávid a Siriabeliek ellen, õk is megütközének õ vele.
\par 18 De a Siriabeliek megfutamodának Izráel elõl, és levága Dávid a Siriabeliek közül hétezer szekeret és negyvenezer gyalogot; annakfelette Sófákot,  a sereg vezérét is megölé.
\par 19 Mikor pedig látták a Hadadézer szolgái, hogy Izráel elõtt legyõzetének: békét kötöttek Dáviddal és szolgálának néki, és nem akarák többször a Siriabeliek megsegélleni az Ammon fiait.

\chapter{20}

\par 1 És lõn az esztendõnek fordulásával, mikor a királyok harczba menni szoktak, elindítá Joáb a hadat, és elpusztítá az  Ammon fiainak földjét. És elmenvén megszállá Rabbát, (Dávid pedig Jeruzsálemben marada) és elfoglalá Joáb Rabbát és elrontá azt.
\par 2 És elvevé Dávid az õ királyuknak fejérõl a koronát, mely egy tálentom arany súlyú vala, s melyben drágakövek valának. És Dávid fejére tette azt, s a városból is temérdek zsákmányt vitt el.
\par 3 A város népét pedig kihozatá és fûrészszel vágatá, és vasboronákkal és fejszékkel. Így cselekedék Dávid az Ammon fiainak minden városával; azután megtére Dávid az egész néppel Jeruzsálembe.
\par 4 Ezután ismét had támada Gézerben a Filiszteusok ellen; és akkor ölé meg a Husátites Sibbékai az óriások nemzetségébõl való  Sippait: és ilyen módon megaláztatának.
\par 5 Ismét lõn had a Filiszteusok ellen, a melyben megölé Elhanán, a Jáir fia a Gáthbeli Lákhmit, a Góliát atyjafiát; és az õ dárdájának nyele hasonló vala a szövõk zúgolyfájához.
\par 6 Ezek után ismét versengés támadt Gáthban, hol egy magas ember vala, a kinek hat-hat, vagyis huszonnégy ujja volt; ez is óriás fia vala.
\par 7 És szidalommal illeté Izráelt, és megölé õt Jonathán, Dávid testvérének, Simeának fia.
\par 8 Ezek ugyanazon egy óriásnak fiai voltak Gáthban, a kik elveszének Dávidnak és az õ szolgáinak keze által.

\chapter{21}

\par 1 Támada pedig a Sátán Izráel ellen, és felindítá Dávidot, hogy megszámlálja Izráelt.
\par 2 Monda azért Dávid Joábnak és a nép elõljáróinak: Menjetek el, számláljátok meg Izráelt, Beersebától fogva Dánig: és hozzátok hozzám, hadd tudjam számát.
\par 3 Akkor monda Joáb: Az Úr száz ennyivel szaporítsa meg az õ népét, a mennyien vannak. Avagy nem mind a te szolgáid-é azok, uram, király? S miért tudakozza ezt az én uram? Miért lenne Izráelnek  büntetésére.
\par 4 De a király szava erõsebb volt a Joábénál. Elméne azért Joáb, és bejárta egész Izráelt; azután megtért Jeruzsálembe.
\par 5 És megjelenté Joáb a megszámlált népnek számát Dávidnak. És volt az egész Izráel népének száma ezerszer ezer és százezer fegyverfogható férfi. A Júda fiai közül pedig négyszázhetvenezer fegyverfogható férfi.
\par 6 A Lévi és Benjámin fiait azonban nem számlálta közéjök; mert sehogy sem tetszett Joábnak a király parancsolata.
\par 7 Sõt az Istennek sem tetszék e dolog, melyért meg is veré Izráelt.
\par 8 Monda pedig Dávid az Istennek: Igen vétkeztem, hogy ezt mûvelém: most azért bocsásd meg a te szolgád vétkét, mert felette esztelenül cselekedtem!
\par 9 Akkor szóla az Úr Gádnak, a Dávid prófétájának, mondván:
\par 10 Eredj el, szólj Dávidnak ekképen: Ezt mondja az Úr: Három dolgot teszek elõdbe, válaszsz magadnak azok közül egyet, hogy azt cselekedjem veled.
\par 11 Elméne azért Gád próféta Dávidhoz, és monda néki: Ezt mondja az Úr: válaszsz magadnak!
\par 12 Vagy három esztendeig való éhséget, vagy hogy három hónapig emésztessél ellenségeid által és a te ellenséged fegyvere legyen rajtad; vagy az Úr fegyvere és a döghalál három napig kegyetlenkedjék földedben és az Úr angyala pusztítson Izráel minden határaiban! Azért lássad most, mit feleljek annak, a ki engem hozzád küldött.
\par 13 És monda Dávid Gádnak: Nagy az én szorongattatásom! Hadd essem inkább az Úr kezébe (mert igen nagy az õ irgalmassága) és ne essem az emberek kezébe!
\par 14 Bocsáta azért az Úr döghalált Izráelre; és meghalának Izráel közül hetvenezeren.
\par 15 Bocsáta annakfelette angyalt az Úr Jeruzsálemre, hogy elpusztítaná azt. És mikor vágná a népet, meglátá az Úr és könyörüle az õ veszedelmökön; és monda a pusztító angyalnak: Elég immár, szünjél meg. Az Úrnak angyala pedig áll vala a Jebuzeus  Ornánnak szérûjénél.
\par 16 Akkor Dávid felemelé az õ szemeit, és látá az Úr angyalát állani a föld és az ég között, kivont kardja a kezében, a melyet Jeruzsálem ellen emelt vala fel. Leesék azért Dávid és a vének is, zsákba öltözvén, és az õ orczájokra.
\par 17 És monda Dávid az Istennek: Nemde nem én számláltattam-é meg a népet? Én vagyok, a ki vétkeztem és igen gonoszul cselekedtem! De ez a nyáj mit tett? óh én Uram Istenem forduljon ellenem a te kezed és az én házam népe ellen, és ne legyen a te népeden a csapás.
\par 18 Akkor szóla az úr angyala Gádnak, hogy megmondja Dávidnak, hogy menjen fel Dávid és építsen oltárt az Úrnak a Jebuzeus Ornán szérûjén.
\par 19 Felméne azért Dávid a Gád beszéde szerint, melyet az Úr nevében szólott vala.
\par 20 Mikor pedig Ornán hátratekintvén, látta az angyalt, õ és az õ négy fia, a kik vele valának, elrejtõzének (Ornán pedig búzát csépel vala).
\par 21 És juta Dávid Ornánhoz. Mikor pedig feltekintett vala Ornán, meglátta Dávidot, és kimenvén a szérûrõl, meghajtá magát Dávid elõtt, arczcal a földre.
\par 22 És monda Dávid Ornánnak: Add nékem e szérûhelyet, hogy építsek oltárt rajta az Úrnak; igaz árán adjad nékem azt, hogy megszünjék e csapás a népen.
\par 23 Monda Ornán Dávidnak: Legyen a tied, és az én uram, a király azt cselekedje, a mi néki tetszik; sõt az ökröket is oda adom égõáldozatul, és fa helyett a cséplõszerszámokat, a gabonát pedig ételáldozatul; mindezeket ajándékul adom.
\par 24 És monda Dávid király Ornánnak: Nem úgy, hanem igaz áron akarom megvenni tõled; mert a mi a tied, nem veszem el tõled az Úrnak és nem akarok égõáldozattal áldozni néki a máséból.
\par 25 Ada azért Dávid Ornánnak a szérûért hatszáz arany siklust.
\par 26 És építe ott oltárt Dávid az Úrnak, és áldozék égõ- és hálaáldozatokkal és segítségül hívá az Urat, a ki meghallgatá õt, mennybõl tüzet bocsátván az égõáldozat oltárára.
\par 27 És parancsola az Úr az angyalnak; és betevé az õ kardját hüvelyébe.
\par 28 Abban az idõben, mikor látta Dávid, hogy az Úr õt meghallgatta a Jebuzeus Ornán szérûjén, áldozék ott.
\par 29 (Az Úr sátora pedig, a melyet csinált vala Mózes a pusztában, és az égõáldozat oltára is akkor Gibeon magaslatán  vala.
\par 30 És Dávid nem mehetett fel oda, hogy az Istent megengesztelje, mert igen megrettent volt az Úr angyalának kardjától.)

\chapter{22}

\par 1 Monda Dávid: Ez az Úr Istennek háza és az égõáldozatnak oltára Izráel számára.
\par 2 Megparancsolá azért Dávid, hogy gyûjtsék össze az Izráel földén való jövevényeket, a kiket kõvágókká tõn, hogy faragni való köveket vágnának, hogy az Isten házát megcsinálnák.
\par 3 Továbbá sok vasat szerze Dávid szegeknek, az ajtókhoz és a foglalásokra; rezet is bõségesen minden mérték nélkül.
\par 4 Számtalan czédrusfát is; mert a Sídon és Tírus városbeliek czédrusfákat bõségesen szállítanak Dávidnak.
\par 5 Mert monda Dávid: Az én fiam, Salamon, gyermek és igen gyenge, az Úrnak pedig nagy házat kell építeni, mely híres legyen és ékesség az egész világon; elkészítek azért mindeneket néki. Dávid azért mindeneket nagy bõségesen megszerze, minekelõtte meghalna.
\par 6 Hivatá azért Dávid az õ fiát, Salamont, és meghagyá néki, hogy az Úrnak, Izráel Istenének házat csináltasson.
\par 7 És monda Dávid Salamonnak: Édes fiam, én elgondoltam vala szívemben, hogy az Úrnak, az én Istenemnek nevének házat építsek;
\par 8 De az Úr ekképen szóla nékem, mondván: Sok vért ontottál, és sokat hadakoztál; ne építs az én nevemnek házat, mert sok vért ontottál ki a földre én elõttem.
\par 9 Ímé fiad lészen néked, a kinek csendessége lészen, mert nyugodalmat adok néki minden körüle való ellenségeitõl; azért neveztetik Salamonnak, mert békességet és  nyugodalmat adok Izráelnek az õ idejében.
\par 10 Õ csinál az én nevemnek házat; õ lészen nékem fiam  és én néki atyja leszek, és megerõsítem az õ királyságának trónját Izráel felett mindörökké.
\par 11 Most édes fiam legyen az Úr veled, hogy sikerüljön néked házat építni az Úrnak, a te Istenednek, miképen szólott te felõled;
\par 12 De adjon az Úr néked értelmet és bölcseséget, mikor téged Izráel fölé helyez, hogy az Úrnak a te Istenednek törvényét megõrízzed.
\par 13 Akkor jól lesz dolgod, ha a rendeléseket és a végzéseket megtartod és teljesíted azokat, a melyeket az Úr Mózes által parancsolt volt Izráelnek. Légy bátor, légy erõs, ne félj, és ne rettegj!
\par 14 Ímé én is az én szegénységemben szereztem az Úr házának építésére százezer  tálentom aranyat és ezerszer ezer tálentom ezüstöt, és rezet, s vasat mérték nélkül, mert igen bõven van; fákat is, köveket is szerzettem; te is szerezz ezekhez.
\par 15 Van néked sok mívesed, kõvágód, kõ- és fafaragód és mindenféle dologban bölcs mesterembered.
\par 16 Az aranynak, ezüstnek, vasnak és réznek száma nincsen: azért kelj fel, láss hozzá, és az Úr legyen veled!
\par 17 Megparancsolá pedig Dávid Izráel összes fõembereinek, hogy õk is legyenek segítségül az õ fiának, Salamonnak, ezt mondván:
\par 18 Avagy nincsen-é az Úr a ti Istentek ti veletek, a ki néktek békességet adott köröskörül? Mert az én hatalmam alá adá e földnek lakosait, és meghódolt e föld az Úr elõtt és az õ népe elõtt.
\par 19 No azért keressétek az Urat a ti Istenteket teljes szívetek és lelketek szerint, és felkelvén, az Úr Isten szentséges helyét csináljátok meg, hogy vigyétek az Úr szövetségének ládáját és az Istennek szentelt edényeket a házba, a mely építtetik az Úr nevének.

\chapter{23}

\par 1 Megvénhedék pedig Dávid és mikor igen koros volna: királylyá tevé az õ fiát, Salamont Izráel felett.
\par 2 És összegyûjté Izráel összes fejedelmeit, a papokat és a Lévitákat is.
\par 3 És megszámláltatának a Léviták, harmincz esztendõstõl fogva, és a kik annál idõsebbek valának; és azok száma fejenként, férfianként harmincznyolczezer volt.
\par 4 Ezek közül huszonnégyezeren az Úr háza teendõinek gondviselõi valának, és hatezeren tiszttartók és birák.
\par 5 Négyezeren ajtónállók; négyezeren pedig dícsérik vala az Urat minden zengõ szerszámokkal, melyeket Dávid készíttetett a dícséretre.
\par 6 És Dávid õket csoportokba osztá a Lévi fiai szerint: Gersonitákra, Kéhátitákra és Méráritákra.
\par 7 A Gersoniták közül valók valának Lahdán és Simhi.
\par 8 Lahdán fiai: Jéhiel a fõ, Zétám és Joel hárman.
\par 9 És a Simhi fiai: Selómit, Hásiel és Hárán, hárman; ezek voltak a Lahdán családjának fejei.
\par 10 Simhi fiai: Jahát, Zina, Jéus és Béria; ezek négyen voltak Simhi fiai.
\par 11 Jahát volt a fõ, Zina második: de Jéus és Béria, mivel nem sok fiakat nemzének, az õ családjukban csak egy ágnak vétettek.
\par 12 Kéhát fiai: Amrám, Ishár, Hebron és Uzziel, négyen.
\par 13 Amrám fiai: Áron és Mózes, Áron kiválasztatott, hogy felszenteltetnék a szentek  szentje számára, õ és az õ fiai mindörökké, hogy jóillatot tennének az Úr elõtt, és szolgálnának néki, s az õ nevében a népet megáldanák mindörökké.
\par 14 Mózesnek, az Isten emberének fiai pedig számláltatának a Lévi nemzetségei közé.
\par 15 Mózes fiai: Gerson és Eliézer.
\par 16 Gerson fiai: Sébuel a fõ.
\par 17 Eliézer fiai voltak: Rehábia a fõ. Eliézernek nem volt több fia; de a Rehábia fiai igen megsokasodtak vala.
\par 18 Ishár fiai: Selómit, a ki fõ vala.
\par 19 Hebron fiai: Jéria a fõ, Amárja második, Jaháziel harmadik és Jékámám negyedik.
\par 20 Uzziel fiai: Mika a fõ, és Isija második.
\par 21 Mérári fiai: Mákhli és Músi: Mákhli fiai: Eleázár és Kis.
\par 22 Meghala pedig Eleázár és nem voltak néki fiai, hanem leányai, a kiket a saját atyjokfiai, a Kis fiai vettek el.
\par 23 Músi fiai: Mákhli, Eder és Jeremót, hárman.
\par 24 Ezek Lévi fiai családjaik szerint, a családfõk, az õ megszámláltatásuk szerint, neveik száma szerint fejenként, a kik az Úr házában szolgáltak, húsz éves korban és azon felül.
\par 25 Mert ezt mondotta vala Dávid: Az Úr, az Izráel Istene nyugodalmat adott az õ népének, és Jeruzsálemben lakozik mindörökké;
\par 26 Azért a Lévitáknak sem kell többé hordozni az Isten hajlékát, s mindazokat az eszközöket, a melyek az õ szolgálatához valók.
\par 27 Annakokáért Dávidnak utolsó rendelése szerint megszámláltattak a Lévi fiai, húsz éves korban és azon felül.
\par 28 Mert az õ helyök az Áron fiai mellett van, hogy szolgáljanak az Úr házában és annak pitvaraiban, kamaráiban, és mindenféle szent edények tisztítása és az Isten házának szolgálatja által.
\par 29 És hogy gondot viseljenek a szent kenyerekre, az ételáldozathoz  való lisztlángra, a kovász nélkül való lepényekre, a serpenyõben fõttre és pirítottra, minden mértékre és mérõre.
\par 30 És hogy álljanak az Úrnak tiszteletére és dícséretére, úgy reggel, mint este;
\par 31 És hogy áldozzanak az Úrnak minden égõáldozattal, minden szombaton, a hónapok elsõ napjain és a szokott ünnepeken bizonyos szám szerint, a mint szükség vala, szüntelen az úr elõtt;
\par 32 És hogy szorgalmasan õrizzék a gyülekezet sátorát, õrizzék a szenthelyet, és hogy vigyázzanak az õ atyjokfiainak, az Áron fiainak szolgálatjánál az Úr házában.

\chapter{24}

\par 1 Az Áron fiainak is voltak rendjeik. Áron fiai: Nádáb, Abihú, Eleázár és Itamár.
\par 2 Nádáb és Abihú még atyjuk elõtt meghaltak és fiaik nem valának, azért Eleázár és Itamár viselék a papságot.
\par 3 És elosztá õket Dávid és Sádók, ki az Eleázár fiai közül vala, és Ahimélek, ki az Itamár fiai közül vala, az õ tisztök szerint a szolgálatra.
\par 4 Eleázár fiai között pedig több fõember találtaték, mint az Itamár fiai között, mikor eloszták õket. Az Eleázár fiai között családjaik szerint tizenhat fõember volt; az Itamár fiai közül, családjaik szerint, nyolcz.
\par 5 Eloszták pedig õket sors által válogatás nélkül, mert a szenthelynek fejedelmei és Istennek fejedelmei valának úgy az Eleázár, mint az Itamár fiai közül valók.
\par 6 És beírá õket Semája, a Nétanéel fia, a Lévi nemzetségébõl való íródeák, a király elõtt és a fejedelmek elõtt, Sádók pap elõtt; Ahimélek elõtt, ki Abjátár fia vala, és a papoknak és Lévitáknak családfõi elõtt. Egy család sorsoltatott az Eleázár és egy az Itamár nemzetségébõl.
\par 7 Esék pedig az elsõ sors Jojáribra; a második Jedájára;
\par 8 Hárimra a harmadik; Seórimra a negyedik;
\par 9 Málkijára az ötödik; Mijáminra a hatodik;
\par 10 Hakkósra a hetedik; Abijára a nyolczadik;
\par 11 Jésuára a kilenczedik; Sekániára a tizedik;
\par 12 Eliásibra a tizenegyedik; Jákimra a tizenkettedik;
\par 13 Huppára a tizenharmadik; Jésebeábra a tizennegyedik;
\par 14 Bilgára a tizenötödik; Immérre a tizenhatodik;
\par 15 Hézirre a tizenhetedik; Hápisesre a tizennyolczadik;
\par 16 Petáhiára a tizenkilenczedik; Jéhezkelre a huszadik;
\par 17 Jákinra a huszonegyedik; Gámulra a huszonkettedik;
\par 18 Delájára a huszonharmadik; Maáziára a huszonnegyedik.
\par 19 Ez az õ hivatalos rendjök szolgálatukban, hogy bejárnának az Úr házába sorban, az õ atyjoknak Áronnak rendelése szerint, a mint megparancsolta volt néki az Úr, Izráel Istene.
\par 20 A mi Lévi többi fiait illeti: az Amrám fiai közül vala Subáel; a Subáel fiai közül Jehdéja.
\par 21 A mi illeti Rehábiát: A Rehábia fiai közül Issija vala fõ.
\par 22 Az Ishár fiai közül Selómót; és a Selómót fiai közül Jahát.
\par 23 A Hebron fiai közül elsõ vala Jérija, Amárja második, Jaháziel harmadik, Jekámhám negyedik.
\par 24 Uzziel fiai: Mika; a Mika fiai közül Sámir.
\par 25 Mika atyjafia Issija; Issija fiai közül Zekáriás.
\par 26 Mérári fiai: Mákhli és Músi; Jaázija fia, Bénó.
\par 27 Mérárinak Jaázijától, az õ fiától való fiai: Sohám, Zakkúr és Hibri.
\par 28 Mákhlitól vala Eleázár, és ennek nem valának fiai.
\par 29 Kistõl: a Kis fiai közül való volt Jérakhméel.
\par 30 Músi fiai: Mákhli, Eder és Jérimót. Ezek a Léviták fiai az õ családjaik szerint.
\par 31 Ezek is sorsot vetének az õ atyjokfiaival az Áron fiaival együtt Dávid király elõtt, Sádók és Ahimélek elõtt, és a papok és Léviták családfõi elõtt, a fõ a kisebbekkel egyformán.

\chapter{25}

\par 1 Dávid és a sereg fõvezérei a szolgálatra kijelölék az Asáf, Hémán és Jédutun fiait, hogy prófétáljanak  cziterákkal, lantokkal és czimbalmokkal. Azok száma, a kik e szolgálatra rendeltettek, az õ szolgálatuk szerint:
\par 2 Az Asáf fiai közül: Zakkúr, József, Nétánia és Asaréla, az Asáf fiai Asáf mellett, a ki a király mellett prófétál vala.
\par 3 A Jédutun fiai közül: A Jédutun fiai Gedália, Séri, Jésája, Hasábia, Mattithia és Simei, hatan cziterával az õ atyjok Jédutun mellett, a ki az Úr tiszteletére és dícséretére prófétál vala.
\par 4 A Hémán fiai közül: Hémán fiai: Bukkija, Mattánia, Uzziel, Sébuel, Jérimót, Hanánia, Hanáni, Eliáta, Giddálti, Romámti-Ezer, Josbekása, Mallóti, Hótir, Maháziót.
\par 5 Ezek mind Hémán fiai, a ki az Isten beszédeiben a király látnoka a hatalom szarvának emelésére. Az Isten Hémánnak tizennégy fiút és három leányt ada.
\par 6 Ezek mindnyájan az õ atyjuk mellett valának, a kik az Úr házában énekelnek vala czimbalmokkal, lantokkal és cziterákkal az Isten házának szolgálatában, a királynak, Asáfnak, Jédutunnak és Hémánnak parancsolata szerint.
\par 7 Ezeknek száma testvéreikkel együtt, akik jártasok valának az Úr énekében, mindnyájan tudósok, kétszáznyolczvannyolcz vala.
\par 8 És sorsot vetének a szolgálat sorrendjére nézve, kicsiny és nagy, tanító és tanítvány egyaránt.
\par 9 És esék az elsõ sors az Asáf fiára, Józsefre; Gedáliára a második. Õ és testvérei tizenketten valának.
\par 10 A harmadik Zakkúrra, kinek fiai és testvérei tizenketten valának.
\par 11 Negyedik Jisrire esék, kinek fiai és testvérei tizenketten valának.
\par 12 Ötödik Nétániára, kinek fiai és testvérei tizenketten valának.
\par 13 Hatodik Bukkijára, kinek fiai és testvérei tizenketten valának.
\par 14 Hetedik Jésarelára, kinek fiai és testvérei tizenketten valának.
\par 15 Nyolczadik Jésájára, kinek fiai és testvérei tizenketten valának.
\par 16 Kilenczedik Mattániára, kinek fiai és testvérei tizenketten valának.
\par 17 Tizedik Simeire, kinek fiai és testvérei tizenketten valának.
\par 18 Tizenegyedik Azárelre, kinek fiai és testvérei tizenketten valának.
\par 19 Tizenkettedik Hasábiára, kinek fiai és testvérei tizenketten valának.
\par 20 Tizenharmadik Subáelre, kinek fiai és testvérei tizenketten valának.
\par 21 Tizennegyedik Mattithiára, kinek fiai és testvérei tizenketten valának.
\par 22 Tizenötödik Jérimótra, kinek fiai és testvérei tizenketten valának.
\par 23 Tizenhatodik Hanániára, kinek fiai és testvérei tizenketten valának.
\par 24 Tizenhetedik Josbekására, kinek fiai és testvérei tizenketten valának.
\par 25 Tizennyolczadik Hanánira, kinek fiai és testvérei tizenketten valának.
\par 26 Tizenkilenczedik Mallótira, kinek fiai és testvérei tizenketten valának.
\par 27 Huszadik Eliátára, kinek fiai és testvérei tizenketten valának.
\par 28 Huszonegyedik Hótirra, kinek fiai és testvérei tizenketten valának.
\par 29 Huszonkettedik Giddáltire, kinek fiai és testvérei tizenketten valának.
\par 30 Huszonharmadik Maháziótra, kinek fiai és testvérei tizenketten valának.
\par 31 Huszonnegyedik Romámti-Ezerre, kinek fiai és testvérei tizenketten valának.

\chapter{26}

\par 1 Az ajtónállók rendje ez: A Kóriták közül Meselémiás, a Kóré fia, a ki az Asáf  fiai közül való volt.
\par 2 A Meselémiás fiai közül elsõszülött Zekária, Jédiáel második, Zebádia harmadik és Játniel negyedik,
\par 3 Elám ötödik, Jóhanán hatodik, Eljehoénai hetedik.
\par 4 Obed-Edom fiai közül: Semája elsõszülött, Józabád második, Jóah harmadik, Sákár negyedik, Nétanéel ötödik,
\par 5 Ammiel hatodik, Issakhár hetedik, Pehullétai nyolczadik, mert az Isten megáldotta vala õt.
\par 6 Az õ fiának, Semájának is születtek fiai, a kik családjukban uralkodtak, mert erõs vitézek valának.
\par 7 Semája fiai: Othni, Refáel, Obed, Elzabád, kinek testvérei igen erõs férfiak valának, Elihu és Sémákiás.
\par 8 Ezek mindnyájan Obed-Edom fiai közül valók; mind õ magok, mind fiaik és testvéreik derék férfiak valának, erõsek a szolgálatra; hatvanketten Obed-Edomtól valók.
\par 9 Meselémiásnak is fiai és testvérei tizennyolczan erõs férfiak valának.
\par 10 Hósásnak is, a Mérári fiai közül, valának fiai, kik között Simri volt a fõ, nem mintha elsõszülött volna, hanem mivel az õ atyja õt akará választani fõnek;
\par 11 Hilkiás a második, Tebáliás a harmadik, Zekária negyedik; mindnyájan a Hósa fiai és testvérei, tizenhárman.
\par 12 Ezek között osztaték el az ajtónállás tiszte a fõemberek által, hogy az õ atyjokfiaival együtt vigyáznának és szolgálnának az Úr házában.
\par 13 És sorsot vetének kicsiny és nagy egyaránt az õ családjaik szerint mindenik kapura.
\par 14 És esék a napkelet felé való kapunak õrizete sors szerint Selémiásra; Zekáriára pedig, az õ fiára, a ki bölcs tanácsadó vala, mikor sorsot vetének, esék az északi kapu sors szerint.
\par 15 Obed-Edomnak a déli kapu; az õ fiainak pedig a kincstartó ház.
\par 16 Suppimnak és Hósának a napnyugotra való kapu, a Salléketh kapuval együtt a felmenõ töltésen, egyik õrizõhely szemben a másikkal.
\par 17 Napkelet felé vala hat Lévita; észak felé minden napra négy; délre is minden napra négy; a kincstartó háznál kettõ-kettõ.
\par 18 A külsõ részen napnyugat felé: négy a felmenõ töltésen, kettõ a külsõ rész felé.
\par 19 Ezek az ajtónállók csoprtojai a Kóriták és Méráriták közül.
\par 20 A Léviták közül: Ahija volt az Isten háza kincsének, és az Istennek szentelt kincsnek fõgondviselõje.
\par 21 A Lahdán fiai, Gersoniták fiai Lahdántól; a Lahdán családfõi, Gersonita Jéhiéli.
\par 22 Jéhiéli fiai: Zétám és Joel az õ testvére, a kik az Úr háza kincseinek valának gondviselõi.
\par 23 Az Amrám, Ishár, Hebron és Aziel nemzetségébõl valók is.
\par 24 Sébuel pedig Gersonnak, a Mózes fiának fia a kincs fõgondviselõje volt.
\par 25 Az õ atyjafiai pedig, Eliézertõl valók, ezek: Rehábiás az õ fia, kinek fia Jésáia, kinek fia Jórám, kinek fia Zikri, kinek fia Selómit.
\par 26 Ez a Selómit és az õ testvérei valának gondviselõi minden megszentelt kincsnek, a melyet Dávid király, a nemzetségek fejedelmei, az ezredesek, századosok és a hadakozó nép fejedelmei szenteltek vala Istennek,
\par 27 A melyet a hadban való zsákmányból szenteltek vala az Úr házának építésére.
\par 28 És a mit csak szentelt Sámuel Próféta, és Saul, a Kis fia, és Abner, a Nér fia, és Joáb, a Séruja fia; és bárki szentelt is valamit Istennek: mind Selómitnak és az õ testvéreinek gondviselése alatt volt.
\par 29 Az Isháriták közül Kenániás és az õ fiai Izráel külsõ dolgaival bízattak meg, mint tiszttartók és birák.
\par 30 A Hebroniták küzöl Hasábiás és testvérei igen erõs férfiak, ezeren és hétszázan valának, a kik Izráel népe között, a Jordán vizén túl napenyészetre, gondviselõk voltak az Úr minden dolgában és a király szolgálatjában.
\par 31 A Hebroniták között Jerija vala a fõ (a Hebroniták nemzetségeik és családjaik szerint megkerestetének a Dávid királyságának negyvenedik esztendejében és találtak bátor férfiakat Gileád-Jaézerben).
\par 32 És az õ atyjafiai, erõs férfiak, kétezerhétszázan valának, családfõk; a kiket Dávid király gondviselõkké tett a Rubenitákon és Gáditákon és Manasse félnemzetségén, mind az Istennek, mind a királynak minden dolgaiban.

\chapter{27}

\par 1 Ezek az Izráel fiai szám szerint, családfõk, ezredesek, századosok és elõljáróik azoknak, a kik a király szolgálatában állottak a seregek minden dolgaiban, a kik be- és kimennek hónapról-hónapra az egész esztendõn át; a seregek egyenként huszonnégyezer emberbõl állának.
\par 2 Az elsõ csapatnak vezére az elsõ hónapban Jasobeám, a Zabdiel fia vala, mely csapat huszonnégyezerbõl állott;
\par 3 Õ a Péres fiai közül való volt és vezére az elsõ hónapra rendelt sereg minden elõljárójának.
\par 4 A második csapatnak vezére a második hónapban Dódai Ahohites vala és az õ csapatja; és Miklót az elõljáró. Az õ csapatja is huszonnégyezerbõl állott.
\par 5 A harmadik seregnek a harmadik hónapban Benája volt a vezére, a ki Jójada fõpapnak volt a fia; az õ csapatja is huszonnégyezerbõl állott.
\par 6 Ez a Benája hõs volt a harmincz között, sõt a harmincz felett való volt. Csapatjának elõljárója az õ fia, Ammizabád vala.
\par 7 A negyediknek vezére a negyedik hónapban Asáel vala, a Joáb testvére és utána az õ fia, Zebádia. Az õ csapatja is huszonnégyezerbõl állott.
\par 8 Az ötödiknek vezére az ötödik hónapban Jizráhites Samhut vala. Az õ csapatja is huszonnégyezerbõl állott.
\par 9 A hatodiknak vezére a hatodik hónapban a Tékoabeli Ira volt, az Ikkés fia. Az õ csapatja is huszonnégyezerbõl állott.
\par 10 A hetediknek vezére a hetedik hónapban a Pélonbeli Héles vala, az Efraim fiai közül. Az õ csapatja is huszonnégyezerbõl állott.
\par 11 A nyolczadiknak vezére a nyolczadik hónapban a Husátbeli Sibbékai vala, a Zárhiták közül. Az õ csapatja is huszonnégyezerbõl állott.
\par 12 A kilenczediknek vezére a kilenczedik hónapban az Anatótbeli Abiézer vala, a Benjáminiták közül. Az õ csapatja is huszonnégyezerbõl állott.
\par 13 A tizediknek vezére a tizedik hónapban a Nétófátbeli Maharai vala, a Zárhiták közül. Az õ csapatja is huszonnégyezerbõl állott.
\par 14 A tizenegyediknek vezére a tizenegyedik hónapban a Pirathonbeli Benája vala, az Efraimiták közül. Az õ csapatja is huszonnégyezerbõl állott.
\par 15 A tizenkettediknek vezére a tizenkettedik hónapban a Nétófátbeli Héldai vala, Othniel nemzetségébõl való. Az õ csapatja is huszonnégyezerbõl állott.
\par 16 Izráel nemzetségei között, a Rubeniták elõljárója Eliézer, a Zikri fia; a Simeonitáké Sefétiás, a Maaka fia.
\par 17 Lévinek Hasábia, a Kémuel fia; Áronnak Sádók;
\par 18 Júdának Elihu, a Dávid atyjafiai közül; Izsakhárnak Omri, a Mikáel fia;
\par 19 Zebulonnak Ismája, az Obádiás fia; Nafthalinak Jérimót, Azriel fia;
\par 20 Efraim fiainak Hósea, az Azáziás fia; a Manasse félnemzetségének Jóel, a Pedája fia;
\par 21 Manasse Gileád földén való félnemzetségének Iddó, a Zekáriás fia; Benjáminnak Jaasiel, az Abner fia;
\par 22 Dánnak Azaréel, a Jérohám fia. Ezek valának az Izráel nemzetségének fejedelmei.
\par 23 Nem számláltatá meg pedig Dávid azokat, a kik húsz esztendõn alól valának; mert az Úr megigérte, hogy megsokasítja Izráelt, mint az égnek csillagait.
\par 24 Joáb, a Séruja fia megkezdé a számlálást, de véghez nem vivé, mivelhogy e miatt Istennek haragja támadt Izráel ellen. Ez okból a szám nem vétetett fel a Dávid király krónikájának száma közé.
\par 25 A király kincsének gondviselõje vala Azmávet, az Adiel fia; a mezõkön, a városokban, a falvakban, a várakban való kincseknek pedig Jónatán, az Uzziás fia volt a gondviselõje;
\par 26 A föld mûvelésére rendelt mezei munkások felett volt Ezri, a Kélub fia.
\par 27 A szõlõmívesek felett volt a Rámátbeli Simei, és a szõlõkben levõ boros tárházak felett a Sifmitbeli Zabdi;
\par 28 Az olajfák és a mezõn való fügefák mívesei felett a Gideritbeli Baálhanán; az olajos tárházak felett Joás;
\par 29 A Sáron mezején legelõ barmok felett a Sáronbeli Sitrai; a völgyekben legelõ barmok felett pedig Sáfát, az Adlai fia.
\par 30 A tevék felett az Ismáel nemzetébõl való Obil; a szamarak felett a Méronótbeli Jehdéja;
\par 31 A juhok felett a Hágárénus Jaziz. Mindnyájan ezek valának a Dávid király jószágainak gondviselõi.
\par 32 Jónatán, a Dávid nagybátyja fõtanácsos vala, a ki értelmes és írástudó ember volt. Jéhiel, a Hakhmóni fia pedig a király fiaival vala.
\par 33 Akhitófel is a királynak tanácsosa volt. Arkites Khúsai pedig  a király barátja.
\par 34 Akhitófel után volt Jójada, a Benája fia, és Abjátár; a király hadának fõvezére pedig Joáb.

\chapter{28}

\par 1 Egybehívatá pedig Dávid Jeruzsálembe az Izráel összes fejedelmeit, a nemzetségek fejedelmeit, a király szolgálatában levõ csapatok elõljáróit, az ezredeseket, a századosokat, a király minden jószágának és marhájának gondviselõit, a maga fiait is, udvariszolgáival, a harczosokat minden erõs vitézivel egyetemben.
\par 2 És felálla Dávid király az õ lábaira, és monda: Halljátok meg szómat atyámfiai és én népem! Elvégezém szívem szerint, hogy az Úr szövetsége ládájának nyugodalmas házat csináltatnék és a mi Istenünk lábainak zsámolyt; hozzá is készítettem mindent az építéshez.
\par 3 De az Isten monda nékem: Ne te csinálj házat az én nevemnek; mert hadakozó ember vagy, sok vért is ontottál immár.
\par 4 Engem választa az Úr, Izráel Istene az én atyámnak egész házából, hogy lennék királya az Izráel népének mindörökké; mert Júdát választotta elõljárónak és a Júda törzsében az én atyámnak  háznépét; az én atyámnak fiai közül pedig engem méltóztatott királylyá tenni egész Izráel felett.
\par 5 És minden fiaim közül (mert az Úr sok fiakat adott nékem) választotta Salamont az én fiamat, hogy ülne az Úr királyságának székében Izráel felett.
\par 6 És monda nékem: Salamon a te fiad építi meg az én házamat és az én pitvarimat; mert én õt magamnak fiamul választottam, és én is néki atyja leszek;
\par 7 Megerõsítem az õ királyságát mindörökké; ha az én parancsolatimat és ítéletimet szorgalmatosan megtartándja, a mint e mai napon.
\par 8 Most azért az egész Izráelnek, az Úr gyülekezetének szeme elõtt és a mi Istenünk hallására: õrizzétek s keressétek az Úrnak a ti Istenteknek minden parancsolatit, hogy bírhassátok e jó földet, és hogy örökségül hagyhassátok fiaitokra is magatok után mindörökké.
\par 9 Te azért, fiam, Salamon, ismerd meg a te atyád Istenét, és szolgálj néki tökéletes szívvel és jó kedvvel; mert az úr minden szívbe belát és minden emberi gondolatot jól ért. Hogyha õt keresénded, megtalálod; ha ellenben õt elhagyándod, õ is elhagy téged mindörökké.
\par 10 Most azért, mivelhogy az Úr választott téged, hogy néki szent házat építs: légy erõs és készítsd meg azt.
\par 11 Akkor átadá Dávid az õ fiának, Salamonnak a tornácznak, annak házacskáinak, kincstartó helyeinek, felházainak és belsõ kamaráinak, a kegyelem táblája helyének is formáját,
\par 12 És mindennek formáját, a melyeket szívében elgondolt vala, az Úr házának pitvarai felõl, az Isten háza kincsének és a szent kincseknek számára köröskörül levõ minden kamarák felõl.
\par 13 A papok és Léviták csapatjainak háza, az Úr háza szolgálatának minden munkája és az Úr háza szolgálatára való minden eszköz felõl.
\par 14 Azután aranyat mérték szerint, a különbözõ szolgálatok eszközei számára; mértékkel minden ezüst eszközök és a különbözõ szolgálatok eszközei számára.
\par 15 Vagyis aranyat mértékkel az arany gyertyatartókra és azok arany szövétnekére, mértékkel mindenik gyertyatartóra és annak szövétnekire, s az ezüst gyertyatartókra mértékkel, a gyertyatartóra és annak szövétnekire, mindegyik gyertyatartó rendeltetése szerint.
\par 16 És aranyat mértékkel, a szent kenyerek asztalai, minden asztal számára, azonképen ezüstöt az ezüst asztalok számára.
\par 17 A villákra, medenczékre és kancsókra tiszta aranyat; az arany poharakra is bizonyos mérték szerint ada aranyat, minden pohárra; és az ezüst poharakra ezüstöt bizonyos mértékkel, minden pohárra.
\par 18 A füstölõ oltár számára megtisztított aranyat mérték szerint; aranyat a szekérnek azaz a  Kéruboknak mintázására, a kik kiterjesztett szárnyakkal az Úr szövetségének ládáját befedezik.
\par 19 Mindezek az Úr kezétõl írattattak meg, a ki engem megtanított az egész alkotmány formájára.
\par 20 Monda ezek után Dávid Salamonnak, az õ fiának: Légy bátor és erõs, és kezdj hozzá, semmit ne félj és ne rettegj; mert az Úr Isten az én Istenem veled lészen, téged el nem hagy, tõled el sem távozik, míglen elvégzed az Úr háza szolgálatának minden  mûvét.
\par 21 Ímé, itt vannak a papok és Léviták csapatjai az Isten házának minden szolgálatára, veled lesznek minden munkában készségesen bölcseséggel, a fejedelmek is és az egész nép minden dolgaidra nézve.

\chapter{29}

\par 1 Azután monda Dávid király az egész gyülekezetnek: Ím látjátok, hogy egyedül az én fiamat Salamont választotta Isten, a ki még gyermek és gyenge; a munka pedig nagy, mert nem emberé lészen az a ház, hanem az Úr Istené.
\par 2 Én pedig teljes tehetségem szerint az én Istenem háza számára bõségesen szereztem aranyat az arany szerszámokra, ezüstöt az ezüst szerszámokra, rezet a rézre, vasat a vasra, fákat a fákra, ónyxköveket, foglalni való köveket, veres köveket, különb-különbszínû köveket; mindenféle drágaköveket; márványköveket is bõséggel.
\par 3 Ezenfölül, mivel nagy kedvem van az én Istenem házához, a mi kincsem, aranyam és ezüstöm van, oda adom az én Istenem házának szükségére, azok mellett,  a melyeket szereztem a szentház számára.
\par 4 Háromezer tálentom aranyat Ofir aranyából és hétezer tálentom tiszta ezüstöt, a házak falainak beborítására.
\par 5 Aranyat az arany szerszámokra, ezüstöt az ezüstre és minden szükségre, a mesteremberek kezéhez. És ha valaki még adni akar, szabad akaratja szerint, töltse meg az õ kezét ma és adjon, a mit akar az Úrnak.
\par 6 Azért szabad akaratjok szerint adának ajándékokat az atyák fejedelmei, az Izráel nemzetségeinek fejedelmei, az ezredesek, századosok és a király dolgainak fejedelmei.
\par 7 És adának az Isten házának szükségére ötezer tálentom aranyat és tízezer dárikot; tízezer tálentom ezüstöt és tizennyolczezer tálentom rezet és százezer tálentom vasat.
\par 8 És valakinél drágakövek találtatának, adák az Úr házának kincséhez, a Gersoniták közül való Jéhiel kezébe.
\par 9 És örvendeze a sokaság, hogy szabad akaratjokból adának; mert tiszta szívökbõl adakozának az Úrnak; Dávid király is nagy örömmel örvendezett.
\par 10 Hálákat ada azért Dávid az Úrnak az egész gyülekezet elõtt, és monda Dávid: Áldott vagy te, Uram, Izráelnek, a mi atyánknak Istene, öröktõl fogva mind örökké!
\par 11 Oh Uram, tied a nagyság, hatalom, dicsõség, örökkévalóság és méltóság, sõt minden, valami a mennyben és a földön van, tied! Tied, oh Uram, az ország, te magasztalod fel magadat, hogy légy minden fejedelmek felett!
\par 12 A gazdagság és a dicsõség mind te tõled vannak, és te uralkodol mindeneken; a te kezedben van mind az erõsség és mind a birodalom; a te kezedben van mindeneknek felmagasztaltatása és megerõsíttetése.
\par 13 Most azért, oh mi Istenünk, vallást teszünk elõtted, és dícsérjük a te dicsõséges nevedet;
\par 14 Mert micsoda vagyok én, és micsoda az én népem, hogy erõnk lehetne a szabad akarat szerint való ajándék adására, a mint tettünk? Mert tõled van minden, és a miket a te kezedbõl vettünk, azokat adtuk most néked.
\par 15 Mert mi csak jövevények vagyunk te elõtted és zsellérek, a mint a mi atyáink is egyenként; a mi életünk napjai olyanok e földön, mint az árnyék, melyben  állandóság nincsen.
\par 16 Oh mi Urunk Istenünk! mind ez a gazdagság, amit gyûjtöttünk, hogy néked és a te szent nevednek házat építsünk, a te kezedbõl való és mindazok tiéid!
\par 17 Jól tudom, oh én Istenem, hogy te a szívet vizsgálod és az igazságot szereted. Én mindezeket tiszta szívembõl, nagy jó kedvvel adtam, s látom, hogy a te néped is, a mely itt jelen van, nagy örömmel, szabad akaratja szerint adta ezeket néked.
\par 18 Oh Uram, Ábrahámnak, Izsáknak és Izráelnek, a mi atyáinknak Istene, tartsd meg mindörökké ezt az  érzést a te néped szívében, és irányítsd az õ szívöket te feléd!
\par 19 Salamonnak pedig, az én fiamnak adj tökéletes szívet, hogy a te parancsolatidat, bizonyságtételeidet és rendelésidet megõrizhesse, s hogy mindazokat megcselekedhesse, s felépíthesse azt a házat, a melyet én megindítottam.
\par 20 És szóla Dávid az egész gyülekezethez: Kérlek, áldjátok az Urat, a ti Isteneteket. Áldá azért az egész gyülekezet az Urat, az õ atyáik Istenét; s leborulván, tisztelék az Urat és a királyt.
\par 21 És áldozának az Úrnak áldozatokkal, és áldozának egészen égõáldozatokkal az Úrnak másnapon, ezer tulkot, ezer kost, ezer bárányt italáldozataikkal, és sokféle áldozatokkal az egész Izráelért.
\par 22 S evének és ivának az Úr elõtt azon a napon nagy vígasságban. Azután Salamont, a Dávid fiát másodszor is királylyá tevék és felkenék õt az Úrnak, hogy lenne fejedelem; Sádókot is fölkenék a fõpapságra.
\par 23 És üle Salamon az Úr székébe, mint király az õ atyjának, Dávidnak helyében, és mindenben szerencsés vala, mert engedelmeskedék néki az egész Izráel.
\par 24 A fejedelmek és a hatalmasok, sõt a Dávid király fiai is mindnyájan kezet adának, hogy Salamon királynak engedelmeskedni fognak.
\par 25 És igen felmagasztalá az Úr Salamont mind az egész Izráel népe elõtt, és oly királyi méltóságot szerzett néki, a melyhez hasonló egy királynak sem volt õ elõtte Izráelben.
\par 26 Így uralkodott Dávid, az Isai fia az egész Izráel felett.
\par 27 Az idõ pedig, a melyben uralkodék Izráel felett, negyven esztendõ. Hebronban uralkodék hét esztendeig, Jeruzsálemben pedig harminczháromig uralkodék.
\par 28 És meghala jó vénségben, életével, gazdagságával és minden dicsõségével megelégedve. És uralkodék az õ  fia, Salamon helyette.
\par 29 Dávid királynak pedig úgy elsõ, mint utolsó dolgai, ímé megirattak a Sámuel próféta könyvében és a Nátán próféta könyvében és a Gád próféta könyvében;
\par 30 Az õ egész országlásával, hatalmasságával és mind azzal, a mi az õ idejében történt vele, Izráellel és a föld minden országaival.


\end{document}