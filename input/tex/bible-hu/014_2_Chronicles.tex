\begin{document}

\title{Krónikák II. könyve}


\chapter{1}

\par 1 Megerõsödék királyságában Salamon, a Dávid fia; és az Úr az õ Istene vele volt, és õt igen felmagasztalá.
\par 2 És szóla Salamon az egész Izráel népének, ezredeseknek, századosoknak, a bíráknak és az egész Izráel minden elõljáróinak és a családfõknek,
\par 3 Hogy elmennének Salamon és az egész gyülekezet õ vele a magaslatra, mely Gibeonban vala; mert ott volt az Isten gyülekezetének sátora, melyet Mózes, az Úr szolgája csinált vala a pusztában;
\par 4 De az Isten ládáját Dávid már felvitte volt Kirját-Jeárimból arra a helyre, melyet készített számára Dávid; mert sátort állított fel számára Jeruzsálemben.
\par 5 A rézoltár azonban, melyet Bésaléel, az Uri fia csinált, a ki a Huri fia volt, ott volt az Úr sátora elõtt; felkeresé tehát azt Salamon és a gyülekezet.
\par 6 És áldozék ott Salamon a rézoltáron az Úr elõtt, mely a gyülekezetnek sátorában vala, áldozék azon ezer égõáldozatot.
\par 7 Azon éjszaka megjelenék az Isten Salamonnak, és mondja néki: Kérj a mit akarsz, hogy adjak néked.
\par 8 És monda Salamon az Istennek: Te nagy irgalmasságot cselekedtél az én atyámmal, Dáviddal; és õ helyette engemet királylyá tettél.
\par 9 Most, oh Uram Isten, legyen állandó a te beszéded, melyet szólottál volt az én atyámnak, Dávidnak; mert te választottál engem királylyá e nép felett, mely oly sok, mint a földnek pora.
\par 10 Most azért adj nékem bölcseséget és tudományt, hogy a te néped elõtt mind ki-, mind bemehessek; mert vajjon kicsoda  kormányozhatja ezt a te nagy népedet?
\par 11 Akkor monda az Isten Salamonnak: Minthogy ez volt a te szívedben, és nem kértél tõlem gazdagságot, kincset és tisztességet, avagy a téged gyûlölõknek lelkét, sem hosszú életet magadnak nem kértél, hanem kértél magadnak bölcseséget és tudományt, hogy kormányozhasd az én népemet, mely felett királylyá tettelek téged:
\par 12 A bölcseséget és a tudományt megadtam néked, sõt gazdagságot, kincset és tisztességet is olyat adok néked, a melyhez hasonló nem volt sem az elõtted, sem az utánad való királyoknak.
\par 13 És visszatére Salamon Jeruzsálembe a Gibeon hegyérõl, a gyülekezet sátora elõl és uralkodék az Izráelen.
\par 14 Szerze Salamon szekereket és lovagokat; és vala néki ezernégyszáz szekere és tizenkétezer lovagja, a kiket helyheztete a szekerek városaiba és Jeruzsálembe a király mellé.
\par 15 És felhalmozá a király az ezüstöt és aranyat, mint a köveket, Jeruzsálemben; a czédrusfákat is felhalmozá, mint a vadfügefákat, melyek a lapályon nagy tömegben vannak.
\par 16 És Salamonnak hoznak vala lovakat Égyiptomból; mert a király kereskedõi sereggel vették volt meg a lovakat szabott áron.
\par 17 És mikor feljõnek vala, hozának Égyiptomból egy szekeret hatszáz ezüst siklusért, egy-egy lovat százötven ezüst siklusért; és ugyan csak õk szállították ezeket a Hitteusok minden királyainak és Siria királyainak.

\chapter{2}

\par 1 Elvégezé Salamon magában, hogy az Úr nevének házat építene, és magának királyi palotát.
\par 2 És rendele Salamon hetvenezer férfit teherhordásra, nyolczvanezer férfit favágásra a hegyen; háromezerhatszáz felügyelõt azokhoz.
\par 3 És külde Salamon a tírusbeli Hirám királyhoz, mondván: A mint az én atyámmal, Dáviddal cselekedtél, a kinek küldöttél czédrusfákat, hogy építene magának házat, melyben laknék;
\par 4 Úgy én az én Uram Istenem nevének akarok házat építeni, hogy néki szenteljem, és abban füstölõ szerekkel jó illatot gerjeszszek, hogy folytonosan szent kenyerek álljanak elõtte, s minden  reggel és este égõáldozatot áldozzak szombatnapokon, és új holdnak napjain, és az Úrnak a mi Istenünknek szentelt ünnepeken, melyeket örökké kell cselekedniök az Izráelitáknak.
\par 5 A ház pedig, a melyet építeni szándékozom, igen nagy lesz, mert a mi Istenünk nagyobb minden isteneknél.
\par 6 De kinek volna annyi ereje, hogy néki házat csinálhatna? Az ég és az egeknek egei õt be nem foghatják, s ki vagyok én is, hogy néki házat csinálhassak? hanem hogy csak jóillatot tegyenek abban õ elõtte.
\par 7 Most azért küldj hozzám tudós mesterembert, a ki tudjon készíteni aranyból, ezüstbõl, rézbõl, vasból, bíborból, karmazsinból és kék bíborból; a ki tudjon metszéseket metszeni az én mesterembereimmel együtt, a kik Júdában és Jeruzsálemben vannak, a kiket az én atyám, Dávid szerzett.
\par 8 Annakfelette küldj czédrusfákat, fenyõfákat és ébenfákat a Libánusról, mert tudom, hogy a te szolgáid tudják vágni a Libánusnak fáit; és ímé az én szolgáim is a te szolgáiddal lesznek.
\par 9 Hogy készítsenek nékem sok fát, mert a ház, a melyet építek, nagy és csudálatos lészen.
\par 10 És ímé a munkásoknak, a favágóknak, a te szolgáidnak adok húszezer véka cséplett búzát és húszezer véka árpát, húszezer báth bort és húszezer báth olajat.
\par 11 Felele pedig Hírám, a Tírus királya levélben, melyet külde Salamonnak: Mivelhogy az Úr szerette az õ népét, azért adott téged nékik királyul.
\par 12 Ismét monda Hírám: Áldott az Úr, az Izráel Istene, a ki mind a mennyet, mind a földet teremtette; a ki ilyen bölcs, tudós, okos és értelmes fiat adott volt Dávid királynak, a ki mind az Úrnak házat, mind magának királyi palotát akar építeni.
\par 13 Ímé küldöttem azért bölcs, tudós és értelmes férfit, a ki az én atyámé, Húrámé volt;
\par 14 A Dán nemzetségének leányai közül való asszony fiát (és az õ atyja Tírus városából való), a ki tud készíteni aranyból, ezüstbõl, rézbõl, vasból, kövekbõl, fákból, bíborból, kék bíborból, lenbõl és karmazsinból és mindenféle metszést metszeni és minden remekmûvet elkészíteni, a melyekkel megbízatik, a te tudós mesterembereiddel és az én uramnak, Dávid királynak, a te atyádnak tudós mestereivel egyben.
\par 15 Azért most, a mely búzát, árpát, olajat és bort igért az én uram, küldje el az õ szolgáinak;
\par 16 Mi pedig a Libánuson vágunk fát, a mennyire néked szükséged lesz, és elviszszük azokat szálakon a tengeren Joppéhoz, és te onnét vitessed Jeruzsálembe.
\par 17 Megszámláltata azért Salamon minden idegen férfit az Izráel földén, az õ atyjának, Dávidnak  megszámláltatása után és találtatának százötvenháromezerhatszázan.
\par 18 És választa azok közül hetvenezer teherhordót, favágásra pedig a hegyen nyolczvanezeret; pallérokul, a kik a népet szorgalmaztatnák, rendele háromezerhatszázat.

\chapter{3}

\par 1 Elkezdé építtetni Salamon az Úr házát Jeruzsálemben a Mórija hegyén, mely Dávidnak, az  õ atyjának megmutattatott, azon a helyen, melyet készített vala Dávid a Jebuzeus Ornán szérûjén.
\par 2 Elkezdé pedig az építést a második hónap második napján, királyságának negyedik esztendejében.
\par 3 És ilyen alapot vetett Salamon az Isten házának építésénél: hosszúsága a régi mérték szerint vala hatvan sing, szélessége húsz sing.
\par 4 És a tornácz a templom hosszában, a ház szélessége szerint, húsz sing volt, a magassága pedig százhúsz sing; és beborítá azt belõl tiszta aranynyal.
\par 5 A derék házat pedig megbélelteté fenyõfákkal és finom aranynyal borítá be, melyen pálmafákat és lánczokat metszete.
\par 6 És beborította a házat drágakövekkel ékességül, és az arany Párvaimból való arany volt.
\par 7 És beboríttatá a háznak gerendáit, ajtómellékit, falait és annak ajtait is aranynyal; és metszete a háznak falaira Kérubokat.
\par 8 Megcsináltatá a szentek-szentjét is, melynek hosszasága a derék háznak szélességével arányban húsz sing, szélessége is húsz sing volt, és beboríttatá azt hatszáz tálentom finom aranynyal.
\par 9 A szegeknek súlya ötven arany siklus volt. A felsõ helyiségeket is beboríttatá aranynyal.
\par 10 És csináltatott a szentek-szentjébe két Kérubot is, szobormûveket, és beboríták azokat aranynyal.
\par 11 A Kérubok szárnyainak hosszúsága húsz sing vala; egyik szárnya öt sing, és a háznak falát érinté; a másik szárnya is öt sing és a másik Kérub szárnyát érinté.
\par 12 A másik Kérubnak szárnya is, a mely öt sing volt, a háznak falát érinté; a másik szárnya pedig, a mely ismét öt sing volt, a másik Kérub szárnyát érinté.
\par 13 Úgy, hogy a Kérubok szárnyai húsz singnyire valának kiterjesztve; lábaikon állának, és arczuk befelé.
\par 14 Annakfelette függönyt is csináltata, kék és piros bíborból és karmazsinból és lenbõl, melyre Kérubokat csináltatott.
\par 15 Két oszlopot is csináltata a ház elõtt, a melyeknek hossza harminczöt sing vala, és gömöt felül mindenikre, a mely öt sing vala.
\par 16 És csináltatott lánczokat is, mint a belsõ részben, s az oszlopok tetejére helyhezteté; és száz gránátalmát is csináltatott, s a lánczokba helyhezteté.
\par 17 És felállítá ez oszlopokat a templom elõtt: egyiket jobbfelõl, a másikat balfelõl. És nevezé a jobbfelõl valót Jákinnak, a balfelõl valót pedig Boáznak.

\chapter{4}

\par 1 Csináltatott rézoltárt is, melynek hosszúsága húsz sing, szélessége is húsz sing, magassága pedig tíz sing vala.
\par 2 Csináltatott öntött tengert is, mely egyik szélétõl fogva a másik széléig tíz sing vala, köröskörül kerek, és öt sing magas, és harmincz sing zsinór érte be a kerületit.
\par 3 Az alatt ökör alakok valának köröskörül, tíz lévén egy singnyire, a melyek kört alkotának a tenger körül; az ökör alakok két renddel valának öntve ugyanazon öntésbõl.
\par 4 Tizenkét ökrön állott; három északra fordulva, három nyugotra, három délre és három napkeletre, és a tenger felül vala rajtok, hátuk pedig mind befelé.
\par 5 Vastagsága egy tenyérnyi volt, és karimája olyan, mint a pohár ajaka, vagy a liliom virága; és háromezer báth fért belé.
\par 6 Csináltatott annakfelette tíz mosdómedenczét, és ötöt helyheztete jobbkéz felõl, ötöt pedig balkéz felõl, hogy azokban mossanak, mossák azt, a mi áldozatra való; a tenger pedig a végre vala, hogy abban mosakodjanak a papok.
\par 7 Csináltatott tíz arany gyertyatartót is, az utasítás szerint, és helyhezteté a templomban, ötöt jobbkéz felõl, ötöt balkéz felõl.
\par 8 Csináltatott tíz asztalt is, a melyeket helyheztete a templomban, ötöt jobbkéz felõl, ötöt balkéz felõl; csináltatott száz arany medenczét is.
\par 9 Megcsináltatá a papok pitvarát is és a nagy tornáczot; és ajtókat a tornáczra, s azok ajtait rézzel borítá be.
\par 10 A tengert pedig helyhezteté jobbkéz felõl napkeletre, délnek ellenébe.
\par 11 Húrám fazekakat, lapátokat és medenczéket is csinált. És elvégezé Húrám a mívet, a melyet csinálnia kellett Salamon királynak, az Isten házában;
\par 12 Tudniillik a két oszlopot és a két kerek gömböt a két oszlop tetejére, és a két hálót a két kerek gömb befedezésére, a melyet az oszlopok tetején valának.
\par 13 És négyszáz gránátalmát a két hálóra; két rend gránátalmát minden hálóba a két kerek gömb befedezésére, a melyek az oszlopok tetején valának.
\par 14 Csinála talpakat is, és azokra mosdómedenczéket.
\par 15 Egy tengert és tizenkét ökröt az alá.
\par 16 Fazekakat, lapátokat és villákat. Mindezen eszközöket Húrám az õ atyja tiszta rézbõl csinálta Salamon királynak az Úr háza számára.
\par 17 A Jordán mezején önteté azokat a király az agyagos földben, Sukkót és Seredáta között.
\par 18 Mindezen eszközöket Salamon nagy mennyiségben csináltatá, mert nem tekinték a réznek súlyát.
\par 19 És megcsináltata Salamon minden egyéb felszerelést is, mely az Úr házához szükséges volt: az arany oltárt és az asztalokat, a melyeken a szent kenyerek voltak.
\par 20 A gyertyatartókat és azoknak szövétnekeit is, hogy égjenek azok rendeltetésök szerint a szentek-szentje elõtt, finom aranyból.
\par 21 Azok virágait, szövétnekeit és hamvvevõit is aranyból, és pedig tiszta aranyból.
\par 22 És az ollókat, medenczéket, tálakat és tömjénezõket finom aranyból, és a ház kapuját, a szentek-szentjéhez való bejárat belsõ ajtóit és a templom házának belsõ ajtóit aranyból.

\chapter{5}

\par 1 És elvégezteték az egész mû, a melyet Salamon király csinála az Úr házához. És bevivé  Salamon Dávidtól, az õ atyjától, Istennek szenteltetett jószágot, az ezüstöt, aranyat és az összes edényeket, és helyhezteté azokat az Isten házának kincsei közé.
\par 2 Akkor Salamon összegyûjté az Izráel véneit és a nemzetségek összes fejedelmeit és az Izráel háznépének fejedelmeit Jeruzsálembe, hogy felvigyék az Úr szövetségének ládáját a  Dávid városából, a mely a Sion.
\par 3 És felgyûlének Izráelnek minden férfiai a királyhoz, a hetedik hónak ünnepén.
\par 4 Mikor pedig eljöttek volna mindnyájan az Izráel vénei: felvevék a Léviták a ládát.
\par 5 És felvivék a ládát, a gyülekezet sátorát és minden szent edényeket, a melyek a sátorban valának; felvivék azokat a papok és Léviták.
\par 6 Salamon király pedig és az Izráel egész gyülekezete, a mely õ hozzá gyûlt, megy vala a láda elõtt, áldozván juhokkal és ökrökkel, a melyek meg sem számláltathatnának, sem meg nem irattathatnának sokaságuk miatt.
\par 7 És bevivék a papok az Úr szövetségének ládáját az õ helyére, az Isten házának belsõ részébe, a szentek-szentjébe, a Kérubok szárnyai alá.
\par 8 A Kérubok pedig szárnyaikat kiterjesztik vala a láda felett, és befedezik vala a Kérubok a ládát és annak rúdjait felülrõl.
\par 9 Azután kijebb vonták annak rúdjait, úgy hogy a rudaknak végei láthatók valának a ládán kivül, a legbelsõ rész felõl, de kivülrõl nem voltak láthatók. És ott volt mind e mai napig.
\par 10 Nem volt egyéb a ládában, hanem csak Mózes két táblája, melyeket õ a Hóreb hegyén tett vala abba, a mikor az Úr szövetséget kötött Izráel fiaival, mikor kijövének Égyiptomból.
\par 11 Lõn pedig, mikor a papok kijöttek a szent helybõl, (mert a papok mindnyájan, a kik ott valának, magokat megszentelték vala és akkor nem kellett megtartaniok az õ sorrendjöket,
\par 12 Annakokáért az énekes Léviták mind, a mennyien valának, Asáf, Hémán, Jedutun, az õ fiaik és testvéreik fehér ruhákban, czimbalmokkal, lantokkal és cziterákkal állanak vala napkelet felõl az oltárnál és õ velök százhúsz kürtölõ pap;
\par 13 Mert a kürtölõknek és éneklõknek tisztök vala egyenlõképen zengeni az Úrnak dícséretére és tiszteletére.) És mikor nagy felszóval énekelnének kürtökkel, czimbalmokkal és mindenféle zengõ szerszámokkal, dícsérvén az Urat, hogy õ igen jó és örökkévaló az õ irgalmassága: akkor a ház, az Úrnak háza megtelék köddel,
\par 14 Annyira, hogy meg sem állhattak a papok az õ szolgálatjukban a köd miatt, mert az Úr dicsõsége töltötte vala be az Istennek házát.

\chapter{6}

\par 1 Akkor monda Salamon: Az Úr mondotta, hogy õ lakoznék ködben.
\par 2 Én pedig lakóházat építettem néked, helyet, a hol örökké lakjál.
\par 3 Azután megfordult a király, és megáldá Izráel egész gyülekezetét; és Izráel egész gyülekezete felállott.
\par 4 És monda: Áldott az Úr, Izráel Istene, ki szólott volt az õ szája által az én atyámnak, Dávidnak, és hatalmas kezeivel beteljesítette, mondván:
\par 5 Attól a naptól fogva, a melyen kihozám az én népemet Égyiptom földébõl, soha nem választottam egyetlen várost sem az Izráel minden nemzetségei közül, hogy házat építenének, a melyben lenne az én nevem, sem férfit nem választottam, hogy az én népemnek Izráelnek vezére lenne.
\par 6 Hanem Jeruzsálemet választottam, hogy az én nevem abban lenne, és választám Dávidot, hogy vezére lenne az én népemnek, Izráelnek.
\par 7 Ámbár az én atyám, Dávid elvégezé magában, hogy házat építene az Úrnak, Izráel Istenének,
\par 8 De az Úr azt mondotta Dávidnak, az én atyámnak: Azt, hogy arra gondoltál, hogy az én nevemnek házat építs, jól tetted, hogy szívedben ezt elvégezted;
\par 9 Mégis nem te építesz házat nékem, hanem a te fiad, a ki a te ágyékodból származik, õ épít az én nevemnek házat.
\par 10 És beteljesíté most az Úr az õ beszédét, a melyet szólott; mert felkelék az én atyám Dávid helyett, és ülék az Izráel királyi székébe, a mint az Úr megmondotta volt; és megépítém a házat az Úrnak, Izráel Istene nevének.
\par 11 És abba helyeztem a ládát, a melyben az Úrnak szövetsége van, melyet szerzett volt az Izráel fiaival.
\par 12 És oda állott Salamon az Úr oltára elé, az Izráel egész gyülekezetével szemben, és kezeit kiterjesztette.
\par 13 Salamon pedig egy széket csináltatott vala rézbõl, a melyet a tornácznak közepén helyeztetett el, melynek hossza öt sing, szélessége is öt sing, magassága pedig három sing vala. Felálla abba, és térdeire esvén az egész Izráel gyülekezete elõtt, kezeit az ég felé kiterjeszté,
\par 14 És monda: Oh Uram, Izráelnek Istene! nincsen hozzád hasonló Isten sem mennyben, sem földön, a ki megtartod a te fogadásodat és irgalmasságodat a te szolgáidhoz, a kik te elõtted teljes szívvel járnak!
\par 15 Ki megtartottad a te szolgádnak, az én atyámnak, Dávidnak, a mit szólottál néki; mert te magad szólottál, és kezeiddel beteljesítetéd, a mint e mai napon megtetszik.
\par 16 Most azért, oh Uram, Izráelnek Istene, tartsd meg, a mit a te szolgádnak, Dávidnak, az én atyámnak igértél, mondván: Nem fogy el elõttem a te magodból való férfiú, a ki az Izráelnek királyiszékiben üljön; csakhogy  a te fiaid õrizzék meg az õ útjokat, hogy az én törvényemben járjanak, mint te én elõttem jártál.
\par 17 Most azért, oh Uram, Izráelnek Istene, bizonyosodjék meg a te beszéded, melyet szólottál volt a te szolgádnak, Dávidnak!
\par 18 (Avagy lakozhatnék-é valósággal az Isten a földön az emberek között? Ím az egek, és az egeknek egei téged be nem foghatnak, mennyivel kevésbbé e ház, a melyet én építettem.)
\par 19 És tekints a te szolgád könyörgésére és imádságára, oh én Uram Istenem, meghallgatván kiáltását és könyörgését, a melylyel könyörög a te szolgád elõtted!
\par 20 Hogy a te szemeid éjjel és nappal figyelmezzenek e házra, e helyre, a melyrõl azt mondottad, hogy nevedet abba helyezénded, meghallgatván könyörgését a te szolgádnak, a mikor e helyen könyörögne.
\par 21 Hallgasd meg azért a te szolgádnak és a te népednek Izráelnek könyörgését, a mikor könyörögni fognak e helyen; hallgasd meg a te mennyei lakhelyedbõl, és meghallgatván õket, légy kegyelmes!
\par 22 Mikor valaki vétkezéndik felebarátja ellen, és esküre köteleztetik, hogy megesküdjék, és õ ide jõ, megesküszik oltárod elõtt ebben a házban,
\par 23 Te hallgasd meg a mennybõl és vidd véghez, és tégy igaz ítéletet a te szolgáid között, az istentelent megbüntetvén, fejére fordítván az õ útját; és az igazat megigazítván, megfizetvén néki az õ igazsága szerint.
\par 24 Mikor pedig megverettetik a te néped, az Izráel, az õ ellenségeitõl, mivel te ellened vétkeztek és hozzád visszatérve vallást tesznek a te nevedrõl, könyörögnek és imádkoznak te elõtted e házban:
\par 25 Te hallgasd meg a mennybõl, és bocsásd meg a te népednek, az Izráelnek bûnét, és hozd vissza õket arra a földre, a melyet adtál nékik és az õ atyáiknak.
\par 26 Mikor az ég berekesztetik és nem lészen esõ, mivel vétkeztek te ellened; és imádkozni fognak e helyen, és vallást tesznek a te nevedrõl és megtérnek bûneikbõl, mert te sanyargatod õket:
\par 27 Te hallgasd meg a mennybõl, és légy kegyelmes a te szolgáidnak és a te népednek, az Izráelnek; minekutána õket megtanítándod az igaz útra, a melyen járjanak; és adj esõt a te földedre, a melyet adtál a te népednek örökségül.
\par 28 Éhség ha lesz a földön, ha döghalál, aszály, ragya, sáska, cserebogár; ha az õ ellensége  szorongatja az õ birodalmának földében; ha bármiféle csapás és nyomorúság jövend reájok:
\par 29 A ki akkor könyörög és imádkozik, legyen az bárki; vagy a te néped, az Izráel, ha elismeri kiki az õ csapását és fájdalmát, és kezeit e házban kiterjeszténdi:
\par 30 Te hallgasd meg a mennybõl, a te lakhelyedbõl és légy kegyelmes, és kinek-kinek fizess az õ útai szerint, a mint megismerted az õ szívét, mert egyedül csak te ismered az emberek fiainak szívét;
\par 31 Hogy féljenek téged, járván a te útaidon, míg élnek e föld színén, a melyet adtál volt a mi atyáinknak.
\par 32 Sõt még az idegen is, a ki nem a te néped, az Izráel közül való, ha eljövénd messze földrõl a te nagy nevedért és a te hatalmas kezedért és a te kiterjesztett karodért, mikor ide jutván, könyörögnek e házban:
\par 33 Te hallgasd meg a mennybõl, a te lakóhelyedbõl, és add meg az idegennek mindazt, a miért könyörög hozzád, hogy megismerjék a földnek minden népei a te nevedet, és tiszteljenek téged úgy, mint a te néped, az Izráel, és ismerjék meg, hogy a te nevedrõl neveztetik e ház, a melyet én építettem.
\par 34 Ha a te néped hadba megy ki az õ ellensége ellen azon az úton, a melyen elbocsátod õket; ha könyörögnek, hozzád fordulván e város felé, a melyet választottál magadnak, és e ház felé, a melyet a te nevednek építettem:
\par 35 Te hallgasd meg az égbõl az õ könyörgésöket és imádságukat, és szerezz nékik igazságot.
\par 36 Ha vétkeznek ellened (mert nincsen ember, a ki nem vétkeznék) és reájok megharagudván, az ellenség kezébe adándod és õket fogságba viendik azok, a kiktõl megfogattak, messze földre vagy közelre,
\par 37 És ha az idegen földön, a hol fogva tartatnak, magokba szállnak, és megtérvén, az õ fogságuk helyén könyörögnek hozzád, és ezt mondandják: Vétkeztünk, hamisan és gonoszul cselekedtünk;
\par 38 És megtérendenek te hozzád teljes szívökbõl és teljes lelkökbõl az õ fogságuk földében, a hol õket fogva tartják, és könyörögnek hozzád, az õ földjüknek útja felé fordulva, a melyet adtál az õ atyáiknak, és e város felé, a melyet választottál magadnak, és e ház felé, a melyet a te nevednek építettem:
\par 39 Hallgasd meg akkor az õ könyörgésöket és imádságukat a mennybõl, a te lakhelyedrõl, és szerezz nékik igazságot, és bocsásd meg a te népednek, hogy ellened vétkezett.
\par 40 Most azért, oh én Istenem, legyenek a te szemeid nyitva és füleid legyenek figyelmesek a könyörgésre ezen a helyen.
\par 41 És most kelj fel, oh Úr Isten, a te nyugodalmadba,  te és a te hatalmasságodnak ládája! A te papjaid, oh Úr Isten, öltöztessenek fel üdvösséggel, és a te szenteid örvendezzenek a jóban.
\par 42 Oh Úr Isten, ne utáld meg a te felkenetett királyod orczáját; emlékezzél meg Dávidhoz, a te szolgádhoz való nagy irgalmasságaidról!

\chapter{7}

\par 1 És mikor Salamon elvégezte a könyörgést, tûz szálla le az égbõl, és megemészté az egészen égõáldozatot és a véres áldozatot, és az Úr dicsõsége betölté a házat,
\par 2 Annyira, hogy még a papok sem mehettek be az Úr házába; mert az Úr dicsõsége betölté az Úr házát.
\par 3 És az Izráel fiai mindnyájan látták, a mikor alászálla a tûz és az Úr dicsõsége a házra, és arczczal leborulának a föld felé a padlózatra, s imádák és tisztelék az Urat, hogy jó és az õ kegyelme mindörökké való!
\par 4 A király pedig és az egész nép áldoznak vala áldozatokat az Úr elõtt.
\par 5 Áldozék Salamon király áldozatot huszonkétezer ökörrel és százhúszezer juhval; és ekképen szentelék fel az Úr házát a király és az egész nép.
\par 6 A papok pedig foglalatosok valának az õ tisztökben; a Léviták is az Úrnak minden zengõ szerszámaival, a melyeket Dávid király készíttetett, hogy az Urat dícsérnék (mert örökkévaló  az õ irgalmassága) a Dávid dicséretivel mely kezökbe adatott; a papok pedig trombitálának velök szemben, míg az egész Izráel ott álla.
\par 7 És felszentelé Salamon a középsõ pitvart, a mely az Úrnak háza elõtt vala; mert ott szerze égõáldozatokat és hálaadóáldozatok kövéreit; mert a rézoltárra, a melyet Salamon készíttete, nem fér vala az égõáldozat, az ételáldozat és a hálaadóáldozat kövére.
\par 8 És Salamon ünnepet szerze ebben az idõben hét napig, és vele együtt az egész Izráel, nagy gyülekezet, mely összegyülekezék Hámáttól fogva az Égyiptom patakáig.
\par 9 A nyolczadik napon pedig gyülekezést tartának, mert az oltár felszentelését hét napon át végezték és az ünnepet is hét napon.
\par 10 A hetedik hónapnak huszonharmadik napján elbocsátá a népet sátoraikba, vígan és megelégedve mindama jók felett, a melyeket cselekedett az Úr Dáviddal, Salamonnal és az õ népével Izráellel.
\par 11 És Salamon bevégezé az Úr házát és a királyi palotát, s mindazt, a mit magában elhatározott Salamon, hogy megcsinál az Úr házában és a maga házában; szerencsésen  bevégezé.
\par 12 Megjelenék pedig az Úr Salamonnak azon éjjel, s monda néki: Meghallgattam a te könyörgésedet, és e  helyet magamnak áldozat házául választottam.
\par 13 Ímé, a mikor az eget bezárandom, hogy ne legyen esõ; és a mikor parancsolok a sáskának, hogy a földet megemészsze; vagy a mikor döghalált bocsátandok az én népemre:
\par 14 És megalázza magát az én népem, a mely nevemrõl neveztetik, s könyörög és keresi az én arczomat, és felhagy az õ bûnös életmódjával: én is meghallgatom õket a mennybõl, megbocsátom bûneiket, és megszabadítom földjüket.
\par 15 Most már az én szemeim nyitva lesznek, és füleim figyelmesek lesznek e helyen a könyörgésre.
\par 16 Most választottam és megszenteltem e házat, hogy az én nevem abban legyen mindörökké, és ott lesznek az én szemeim és az én szívem mindenkor.
\par 17 És ha te elõttem járándasz, a mint járt Dávid, a te atyád, úgy cselekedvén mindenekben, a mint neked megparancsoltam, s az én rendelésimet és ítéletimet megtartándod:
\par 18 Megerõsítem a te birodalmad trónját, amint megigértem volt Dávidnak, a te atyádnak, mondván: Nem vétetik el a te nemzetségedbõl való férfiú az Izráel királyiszékibõl.
\par 19 De hogyha ti elszakadtok, s rendelésimet és parancsaimat, a melyeket elõtökbe adtam, elhagyjátok, s elmenvén, idegen isteneknek szolgálandotok és azok elõtt meghajoltok:
\par 20 Kiszaggatom õket az én földembõl, a melyet adtam volt nékik: és ezt a házat, a melyet az én nevemnek szenteltem, orczám elõl elvetem; tanulságul és példabeszédül adom õket minden nemzetségnek.
\par 21 És ezen a házon, a mely felséges vala, minden elmenõ álmélkodni fog, és azt mondja: Miért cselekedett így az Úr ezzel az országgal és ezzel a házzal?
\par 22 És azt felelik: Azért, mert elhagyták az Urat, atyáik Istenét, a ki õket kihozta volt Égyiptom földébõl, és idegen istenekhez hajlottak, azokat imádták és azoknak szolgáltak; azért hozta reájok mind e veszedelmet.

\chapter{8}

\par 1 Lõn pedig húsz esztendõ mulva, a mi alatt Salamon megépíté az Úr házát és a maga palotáját:
\par 2 Azokat a városokat, a melyeket Hirám adott Salamonnak, megépíté Salamon, és Izráel fiait telepíté oda.
\par 3 Azután elméne Salamon Hámát-Sobába és azt hatalmába keríté.
\par 4 És megépíté Tádmort a pusztában, és minden kincstartó városokat, a melyeket Hámátban épített.
\par 5 Annakfelette mind a felsõ, mind az alsó Bethoront megépíté s megerõsített városokká tette kõfalakkal, kapukkal és zárokkal.
\par 6 És Baalátot, s a tárházak minden városait, a melyek Salamonéi valának, a szekereknek és a lovagoknak minden városait; és mindent, a mihez Salamonnak kedve volt, megépíté Jeruzsálemben, a Libánuson  és az õ egész birodalmának földén.
\par 7 Mindazt a népet, a mely megmaradt a Hitteusok közül, az Emoreusok, Perizeusok, Hivveusok és a Jebuzeusok közül, a kik nem az Izráel közül valók;
\par 8 Hanem azoknak fiai közül, a kik azon a földön õ utánok maradtak volt, a kiket az Izráel fiai ki nem irthattak, azokat Salamon adófizetõkké tevé mind e mai napig.
\par 9 De az Izráel fiai közül Salamon senkit nem tett szolgává az õ külsõ dolgaiban; mert ezek hadakozó férfiak voltak, és az õ hadnagyinak vezérei, s szekereinek és lovagjainak vezérei.
\par 10 Azok, a Salamon király seregeinek vezérei, kétszázötvenen valának, a kik uralkodnak vala a népen.
\par 11 És felviteté Salamon a Faraó leányát a Dávid városából a házba, a melyet néki épített vala; mert ezt mondá: Nem lakhatik az én feleségem az Izráel királyának, Dávidnak házában; mert szentséges hely az, mivel az Úr ládája abba  vitetett.
\par 12 Akkor égõáldozatokat áldozék Salamon az Úrnak az Úr oltárán, a melyet rakatott vala a tornácz elõtt;
\par 13 Naponként áldozának azon, Mózes parancsolatja szerint, szombatnapokon, a hónapok elsõ napjain, és évenként a fõünnepeken  háromszor; tudniillik a kovásztalan kenyerek ünnepén, a hetek ünnepén és a sátorok ünnepén.
\par 14 És elrendelé az õ atyjának, Dávidnak rendelése szerint a papok tisztét az õ szolgálatjokban, a Lévitákat is az õ tisztök szerint, hogy dícsérnék az Istent és szolgálnának a papok mellett naponként; az ajtónállókat is, az õ csoportjaik szerint minden kapuhoz, mert így volt az Isten emberének, Dávidnak parancsolatja.
\par 15 És nem tértek el a király parancsolatjától, a melyet parancsolt vala a papoknak és Lévitáknak minden dolog és a kincsek felõl.
\par 16 Ilyen módon bevégzõdött Salamonnak minden munkája azon naptól, a mikor az Úr házának alapját letették, annak befejezéséig, a mikor immár az Úr háza elkészült.
\par 17 Azután elméne Salamon Esiongáberbe és Elótba, a mely a tenger partján, Edom földén volt.
\par 18 És külde Hirám az õ szolgái által néki hajókat és szolgákat, a kik a tengeren jártasok valának, a kik menének a Salamon szolgáival együtt Ofirba, honnan négyszázötven tálentom aranyat hozának és vivék Salamon királynak.

\chapter{9}

\par 1 A Séba királynéasszonya pedig hallván Salamon hírét, eljöve, hogy megkisértse Salamont nehéz kérdésekkel, Jeruzsálembe, igen nagy sereggel és tevékkel, a melyek hoznak vala fûszereket, igen sok aranyat és drágaköveket; és Salamonhoz méne, és beszélt vele mindenekrõl, a melyek szívén voltak.
\par 2 És Salamon megfelelt minden beszédeire, mert semmi sem volt Salamon elõl elrejtve, a melyet meg nem mondhatott volna néki.
\par 3 És mikor látta Séba királynéasszonya Salamon bölcseségét és a házat, a melyet épített vala;
\par 4 És az õ asztalának étkeit, szolgáinak lakását és hivatalnokainak állását és öltözékeiket, pohárnokait s azoknak öltözékeit, és az õ áldozatát, a melylyel az Úrnak házában áldozott: a lélekzete is elállott.
\par 5 És monda a királynak: Mind igaz volt, a mit hallottam volt az én lakóföldemben a te dolgaidról és bölcseségedrõl.
\par 6 De hinni sem akartam azoknak beszédeit, míg én magam el nem jöttem, és szemeimmel nem láttam. És ímé nékem a felét sem beszélték el a te bölcseséged nagyságának; felülmúltad a hírt, a melyet hallottam.
\par 7 Boldogok a te embereid és boldogok ezek a te szolgáid, a kik szüntelen udvarlanak néked, hogy hallhatják a te bölcseségedet!
\par 8 Legyen az Úr, a te Istened áldott, a ki téged annyira szeretett, hogy az õ székébe helyezett, hogy lennél az Úrnak, a te Istenednek királya; mivel a te Istened szerette Izráelt, hogy megerõsítse õt mindörökké, azért tett téged királylyá felettök, hogy szolgáltass ítéletet és igazságot.
\par 9 És ada a királynak százhúsz tálentom aranyat és felette sok fûszerszámot és drágaköveket. Nem is volt több olyan fûszerszám, mint a milyet Séba királynéasszonya ada Salamon királynak.
\par 10 És Hirám szolgái is és Salamon szolgái is, a kik hoztak vala aranyat Ofirból; hoztak ébenfát és drágaköveket is.
\par 11 És csinála a király az ébenfából lépcsõket az Úr házába és a király házába, s cziterákat és lantokat az éneklõknek. Ezekhez hasonlókat nem láttak azelõtt Júda országában.
\par 12 Salamon király pedig ada a Séba királynéasszonyának mindent, a mit csak kivánt és kért tõle, azon-kivül, a mit õ a királynak hozott. Azután megtére és méne az õ földébe mind õ, mind az õ szolgái.
\par 13 Vala pedig mértéke az aranynak, a mely esztendõnként bejõ vala Salamonnak hatszázhatvanhat tálentom arany,
\par 14 Azonkivül, a mit hoznak vala a kalmárok és kereskedõk; de még Arábia minden királyai és annak a földnek fejedelmei és hoznak vala aranyat és ezüstöt Salamonnak.
\par 15 És csináltata Salamon király kétszáz paizst vert aranyból, mindenik paizsra hatszáz vert arany siklus ment fel.
\par 16 Háromszáz kerek paizst is vert aranyból; mindenik paizsra háromszáz arany siklus megy vala, a melyeket a király helyeztete a Libánon erdejének házába.
\par 17 És csináltatott a király egy nagy királyiszéket elefántcsontból, és beboríttatá azt finom aranynyal.
\par 18 Hat lépcsõje volt a széknek, és arany zsámolya a székhez erõsítve, és támaszai valának mindkét felõl az ülés mellett, és két oroszlán állott a karok mellett.
\par 19 És tizenkét oroszlán áll vala ott a hat lépcsõn mindkét felõl. Senki soha olyant nem csinált egyetlen országban sem.
\par 20 És Salamon királynak összes ivóedényei és aranyból valának; a Libánon erdõ házának is összes edényei tiszta aranyból valának; nem vala az ezüstnek semmi becse a Salamon idejében;
\par 21 Mert hajói voltak a királynak, a melyek Társisba jártak a Hirám szolgáival együtt. Minden három esztendõben egyszer menének a hajók Társisba, honnan aranyat, ezüstöt, elefántcsontot, majmokat és pávákat hoznak vala.
\par 22 És felülmúlta Salamon király e földnek minden királyait gazdagságban és bölcseségben.
\par 23 És e földnek minden királyai kivánnak vala szembe lenni Salamonnal, hogy hallhatnák az õ bölcseségét, a melyet Isten adott vala az õ szívébe.
\par 24 És azok mindnyájan ajándékot visznek vala néki, arany és ezüst edényeket, ruhákat, fegyvert, fûszerszámokat, lovakat és öszvéreket esztendõnként.
\par 25 És Salamonnak négyezer lóistállói, szekerei és tizenkétezer lovagjai valának, a kiket helyheztete a szekerek városaiba és a király mellé Jeruzsálemben.
\par 26 És uralkodó vala minden király felett az Eufrátes folyóvíztõl a Filiszteusok földéig és az Égyiptom határáig.
\par 27 És a király Jeruzsálemben olyanná tevé az ezüstöt, mint a köveket, és a czédrusfákat úgy elszaporítá, mint a vad fügefákat, a melyek nevekednek a mezõségen bõséggel.
\par 28 Hordnak vala pedig Salamonnak lovakat Égyiptomból és minden országból.
\par 29 Salamonnak egyéb dolgai, úgy az elsõk, mint az utolsók, avagy nem írattak-é meg a Nátán próféta könyvében, és a Silóbeli Ahija prófécziájában, és Jehdó prófétának Jeroboám ellen, a Nébát fia ellen írt látásaiban?
\par 30 És uralkodék Salamon Jeruzsálemben az egész Izráel felett negyven esztendeig.
\par 31 És elaluvék Salamon az õ atyáival egyetemben, és eltemeték õt az õ atyjának, Dávidnak városában, és uralkodék helyette az õ fia Roboám.

\chapter{10}

\par 1 Elméne Roboám Síkembe; mert Síkembe gyûlt vala az egész Izráel, hogy õt királylyá választanák.
\par 2 Lõn pedig, mikor ezt meghallotta Jeroboám, a Nébát fia, a ki akkor Égyiptomban vala; mert oda futott volt Salamon király elõl, visszatére Jeroboám Égyiptomból.
\par 3 És hozzáküldvén, elhivaták õt. Eljöve azért Jeroboám és az egész Izráel, és szólának Roboámnak, mondván:
\par 4 A te atyád igen megnehezítette a mi igánkat, de te most könnyebbítsd meg atyádnak kemény szolgálatát és az õ nehéz igáját, a melyet reánk vetett, és szolgálunk néked.
\par 5 És monda nékik: Harmadnapig gondolkodom róla, azután jõjjetek hozzám. Elméne azért a nép.
\par 6 És tanácskozék Roboám király a vén emberekkel, a kik Salamon elõtt az õ atyja elõtt állottak vala életében, mondván: Mit tanácsoltok, mit válaszoljak e népnek?
\par 7 És azok ekképen szólának: Ha javára leendesz ennek a népnek, s kedvezel nékik és jó szóval beszélsz hozzájok; akkor te szolgáid lesznek mindenkor.
\par 8 De õ megvetette a vének tanácsát, a melyet néki tanácsoltak, és tanácsot tarta az ifjakkal, a kik õ vele nevekedtek volt fel és néki udvaroltak.
\par 9 És monda azoknak: Ti micsoda tanácsot adtok, hogy választ adjunk e népnek, a kik nékem így szólának: Könnyebbítsd meg az igát, a melyet a te atyád reánk vetett?
\par 10 Akkor felelének az ifjak, a kik õ vele együtt nevekedtek vala, mondván: Így szólj a népnek, a mely szólván néked, azt mondja: A te atyád megnehezítette a mi igánkat, te pedig könnyebbítsd meg nékünk; így szólj nékik: Az én legkisebb ujjam erõsebb atyám derekánál;
\par 11 Most azért, ha az én atyám nehéz igát vetett reátok, én még nehezebbé teszem igátokat; ha az én atyám ostorral vert titeket, én skorpiókkal.
\par 12 És elméne Jeroboám és az egész nép Roboámhoz harmadnap, a mint a király meghagyta, ezt mondván: Jõjjetek hozzám harmadnapon.
\par 13 És a király kemény választ adott nékik, megvetve Roboám király a vének tanácsát.
\par 14 És az ifjak tanácsa szerint szóla nékik, mondván: Ha az én atyám megnehezítette a ti igátokat, én még nehezebbé teszem azt; ha az én atyám ostorral vert titeket, én skorpiókkal.
\par 15 És a király nem hallgatá meg a népet; mert ezt az Úr fordította ekként, hogy megerõsítené az Úr az õ beszédét, a melyet szólott vala a Silóbeli Ahija által  Jeroboámnak, a Nébát fiának.
\par 16 Mikor pedig az egész Izráel látta, hogy nem hallgatá meg õket a király, felele a nép a királynak, mondván: Micsoda részünk van nékünk Dávidban? Nincsen nékünk örökségünk az Isai fiában! Menj el a te hajlékidba, oh Izráel! Ám viseld gondját a te házadnak, oh Dávid! Elméne azért hajlékiba az egész Izráel;
\par 17 Úgy, hogy Roboám csak azokon az Izráel fiain uralkodék, a kik Júda városaiban laktak.
\par 18 És mikor elküldé Roboám Adorámot, az adószedõt, megkövezék õt Izráel fiai, és meghala; Roboám király pedig siete szekerébe ülni, hogy Jeruzsálembe szaladjon.
\par 19 Így szakada el az Izráel népe a Dávid házától, mind e mai napig.

\chapter{11}

\par 1 Méne azért Roboám Jeruzsálembe, és összegyûjté a Júda és Benjámin házát, száznyolczvanezer válogatott hadviselõket, hogy hadakoznának Izráel ellen, és visszanyernék az országot Roboámnak.
\par 2 Szóla pedig az Semájának, az Isten emberének, mondván:
\par 3 Mondd meg Roboámnak, Salamon fiának, Júda királyának és az egész Izráelnek Júdában és Benjáminban, így szólván:
\par 4 Ezt mondja az Úr: Ne menjetek fel és ne hadakozzatok atyátokfiai ellen; térjetek meg ki-ki a maga házába, mert én tõlem lett e dolog. És engedének az Úr szavának, és megtérének a helyett, hogy Jeroboám ellen mennének.
\par 5 Lakozék azért Roboám Jeruzsálemben, és megerõsíté a városokat Júdában.
\par 6 Így megépíté Bethlehemet, Etámot és Tékoát.
\par 7 Bethsúrt, Sókót és Adullámot,
\par 8 Gátot, Marésát és Zifet,
\par 9 Adoráimot, Lákist és Azekát,
\par 10 Sorát, Ajalont és Hebront, melyek erõs városok valának Júdában és Benjáminban.
\par 11 És mikor megerõsítette ez erõsségeket, helyezett azokba elõljárókat és szerze tárházakat eleségnek és bornak és olajnak.
\par 12 És mindenik városban szerze paizsokat és kopjákat, és rendkivül megerõsíté azokat. És az övé lõn Júda és Benjámin.
\par 13 Továbbá a papok és a Léviták, a kik az egész Izráelben valának, õ hozzá csatlakozának minden õ határukból;
\par 14 Mert a Léviták elhagyták az õ faluikat és jószágukat, és Júdába és Jeruzsálembe menének, mert kiûzte vala õket Jeroboám és az õ fiai, hogy az Úrnak ne szolgálnának.
\par 15 És rendele magának papokat a magaslatokhoz, a bakokhoz és a borjúkhoz, a melyeket csináltatott vala.
\par 16 És utánuk Izráel minden nemzetségei közül azok, a kik szívök szerint keresték az Urat, Izráelnek Istenét, menének Jeruzsálembe, hogy áldoznának az Úrnak, az õ atyáik Istenének.
\par 17 És megerõsíték Júda országát, és megerõsíték Roboámot, a Salamon fiát három esztendeig; mert három esztendeig járának Dávidnak és Salamonnak útján.
\par 18 És feleségül vevé Roboám Mahalátát, Jérimótnak, a Dávid fiának leányát és Abihailt, Eliábnak, az Isai fiának leányát,
\par 19 A ki szüle néki fiakat: Jeust, Semáriát és Zahámot,
\par 20 És õ utána vevé Maakát, az Absolon leányát, a ki szülé néki Abiját, Attait, Zizát és Selómitot.
\par 21 Legjobban szereté pedig Roboám Maakát, az Absolon leányát minden feleségei és ágyasai között; mert tizennyolcz felesége és hatvan ágyasa volt. És nemze huszonnyolcz fiút és hatvan leányt.
\par 22 És Roboám Abiját, a Maaka fiát tette testvérei között vezérré és elõljáróvá, mert õt akará királylyá tenni.
\par 23 És okosan gondolkodván, szétosztá fiait mind Júda és Benjámin földén a megerõsített városokba, a kiknek bõségesen adott eleséget és sok feleséget szerzett számukra.

\chapter{12}

\par 1 Lõn pedig, mikor Roboám az õ királyságát megszilárdította és abban megerõsödött: elhagyta az Úr törvényét, és vele együtt az egész Izráel.
\par 2 Azért a Roboám királyságának ötödik esztendejében feljöve Sésák, az égyiptomi király Jeruzsálem ellen, (mert az Úr ellen vétkézének).
\par 3 Ezerkétszáz fegyveres szekérrel és hatvanezer lovaggal, és megszámlálhatlan vala a nép, a mely vele Égyiptomból feljött, a Libiabeliekkel, Sukkeusokkal és Szerecsenekkel;
\par 4 És elfoglalá Júdának erõs városait, azután méne Jeruzsálem alá.
\par 5 Akkor Semája próféta méne Roboámhoz és a Júda fejedelmeihez, a kik Jeruzsálembe gyûltek össze Sésáktól való féltökben, és monda nékik: Ezt mondja az Úr: Mivel ti engem elhagytatok, én is a Sésák kezébe bocsátlak titeket.
\par 6 Akkor megalázák magokat az Izráel fejedelmei és a király, s mondának: Az Úr igaz!
\par 7 És mikor az Úr látta, hogy megalázták magokat, ekképen szóla az Úr Semája prófétának: Megalázták magokat, nem vesztem el õket, hanem némi szabadulást szerzek nékik, és nem ontom ki az én haragomat Jeruzsálem ellen Sésák által;
\par 8 Mindazáltal szolgái lesznek néki, hogy megtudják a különbséget az én szolgálatom és más országok királyságainak szolgálatai között.
\par 9 Feljöve azért Sésák, az égyiptomi király Jeruzsálem ellen, és elvivé az Úr házának kincsét, s a király házának kincsét; mindazokat elvivé; az arany paizsokat is elvivé, a melyeket Salamon csináltatott vala.
\par 10 Ezek helyett Roboám király rézpaizsokat csináltata, és bízá azokat a gyalogosok fejedelminek kezére, a kik õrzik a király házának ajtaját.
\par 11 És mikor a király felmegy az Úr házába, mennek a gyalogosok is, és felviszik azokat, s azután visszahozzák a gyalogosok szobájába.
\par 12 Mikor azért megalázta magát Roboám, eltávozék az Úr haragja õ róla, hogy meg ne semmisülne mindenestõl, mert Júdában is volt még jó dolog.
\par 13 Megerõsödék azért Roboám király Jeruzsálemben és uralkodék; mert negyvenegy esztendõs vala Roboám, mikor uralkodni kezdett volt, és tizenhét esztendeig uralkodék Jeruzsálemben, a városban, a melyet az Úr választott vala az Izráel minden nemzetségei közül, hogy ott helyheztesse az õ nevét; és az õ anyjának neve vala Naáma, a ki Ammonita volt.
\par 14 Cselekedék pedig gonoszt, mert az Urat szíve szerint keresni nem akará.
\par 15 Roboámnak pedig elsõ és utolsó dolgai avagy nincsenek-é megírva a Semája próféta könyvében, és Iddónak, a látnoknak  könyvében, a nemzetségi lajstromban? És hadakozás volt Roboám és Jeroboám között egész éltökben.
\par 16 És elaluvék Roboám az õ atyáival, és eltemetteték a Dávid városában; és uralkodék az õ fia,  Abija, helyette.

\chapter{13}

\par 1 Jeroboám királynak tizennyolczadik esztendejében kezde uralkodni Abija Júdában.
\par 2 Három esztendeig uralkodék Jeruzsálemben. Az õ anyjának neve Mikája vala, a ki a Gibeából való Uriel leánya. És Abija és Jeroboám között háború vala.
\par 3 Azért felkészüle Abija a háborúra négyszázezer válogatott harczosból álló sereggel; és Jeroboám vele szembeszállott nyolczszázezer válogatott harczosból álló sereggel.
\par 4 Akkor felálla Abija a Semáraim Hegyének tetején, a mely az Efraim hegységében vala, és monda: Hallgassátok meg szómat, Jeroboám és az egész Izráel!
\par 5 Avagy nem kellene-é néktek meggondolnotok, hogy az Úr, az Izráel Istene Dávidnak adta volt a királyságot Izráel felett örökre; néki és fiainak, sónak szövetsége  által?
\par 6 Mindazáltal felkele Jeroboám, a Nébát fia, Salamonnak, a Dávid fiának szolgája, és támada az õ ura ellen;
\par 7 Azután gyûlének õ hozzá a haszontalan emberek, Beliál fiai, a kik ellene szegültek Roboámnak, a Salamon fiának, mikor Roboám gyermek és félénk szívû volt, és azok ellen magát nem oltalmazhatta.
\par 8 És most azt gondoljátok, hogy ti ellene állhattok az Úr királyságának, a mely a Dávid fiainak kezében van, mivel sokan vagytok, s veletek vannak az aranyborjúk is, a melyeket Jeroboám öntetett  néktek istenek gyanánt.
\par 9 Avagy nem ti ûztétek-é el az Úrnak papjait, az Áron fiait és a Lévitákat? És nem ti szerzettetek-é magatoknak papokat, mint egyéb országoknak nemzetségei, akárkit, a ki az õ szolgálatjának felszentelésére egy gyermekded tulokkal és hét kossal eljött, és lett a bálványok papja, a melyek nem istenek.
\par 10 Mi mellettünk van pedig az Úr, a mi Istenünk, a kit mi el nem hagytunk; a papok pedig, a kik az Úrnak szolgálnak, az Áron fiai, és vannak Léviták, a kik forgolódnak az õ tisztökben.
\par 11 És áldoznak az Úrnak égõáldozattal minden reggel és minden estve, és füstölõáldozattal, és a kenyérnek a tiszta asztalra való tételére és az arany gyertyatartóra, szövétnekeivel egybe gondot viselnek, meggyújtván azokat minden estve; mert mi megtartjuk az Úrnak, a mi Istenünknek rendelését: ti pedig elhagytátok õt.
\par 12 Azért ímé mi velünk van az Isten vezér gyanánt, és az õ papjai a riadó kürtökkel, hogy ti ellenetek kürtöljenek. Izráel fiai! Ne harczoljatok az Úr ellen, a ti atyáitok Istene ellen, mert nem lesztek szerencsések!
\par 13 Jeroboám pedig lest vetett ellenök, hogy hátuk mögé kerüljön, s ilyen módon Júda elõtt is õk legyenek, hátuk mögött is a les.
\par 14 Látván pedig Júda, hogy ímé mind elõl, mind hátul megtámadtatának: kiáltának az Úrhoz, a papok pedig trombitálnak vala a trombitákkal.
\par 15 És kiáltának Júda férfiai; és mikor kiáltának a Júda férfiai, az Isten megveré Jeroboámot és az egész Izráelt, Abija és Júda elõtt.
\par 16 És az Izráel fiai menekülének Júda elõl, de az Isten kezökbe adá õket;
\par 17 Mert megveré õket Abija és az õ népe nagy csapással, úgyannyira, hogy az Izráeliták közül seb miatt ötszázezer válogatott férfi esett el.
\par 18 És ilyen módon aláztatának meg az Izráel fiai abban az idõben; Júda fiai pedig megerõsödének, mert õk az Úrra, az õ atyáik Istenére támaszkodtak volt.
\par 19 És üldözé Abija Jeroboámot, és elfoglala õ tõle egynéhány várost, Béthelt és annak faluit, Jésanát és annak faluit, s Efrávint és annak faluit.
\par 20 És nem jutott többé erõhöz Jeroboám Abija idejében, hanem megveré õt az Úr, és meghala.
\par 21 Abija pedig hatalmassá lõn, és vett vala magának tizennégy feleséget, a kiktõl nemze huszonkét fiút és tizenhat leányt.
\par 22 Abijának pedig több dolgai, útai és beszédei megírattak az Iddó próféta könyvében.

\chapter{14}

\par 1 Elaluvék pedig Abija az õ atyáival, és eltemeték õt a Dávid városában, s uralkodék helyette az õ fia, Asa, a kinek idejében tíz esztendeig volt békesség a földön.
\par 2 És Asa mindazt cselekedé, a mi jó és igaz vala az Úr elõtt, az õ Istene elõtt;
\par 3 Elrontá az idegen istenek oltárait és a magaslatokat; a bálványokat eltöreté, és az Aserákat kivágatá;
\par 4 És megparancsolá Júdának, hogy az Urat, az õ atyáik Istenét keressék, és cselekedjék az Isten törvényét és parancsolatját.
\par 5 Kipusztítá Júda minden városaiból a magaslatokat és a nap-oszlopokat, és az ország csendes lõn alatta.
\par 6 És építtetett megerõsített városokat Júdában, mivelhogy nyugodalomban volt a föld, és senki sem folytatott ellene háborút azokban az esztendõkben, mert az Úr nyugodalmat adott vala néki.
\par 7 Mert ezt mondja vala Júdának: Építsük meg a városokat és vegyük körül kerítéssel, tornyokkal, kapukkal, zárokkal, míg a föld birodalmunkban van; mert megkerestük az Urat, a mi Istenünket, megkerestük és nyugodalmat adott nékünk minden felõl. Azért építének és lõn jó elõmenetelök.
\par 8 Vala pedig az Asa serege, a mely paizst és kopját visel vala, Júdából háromszázezer; és Benjáminból paizst viselõk és kézívesek kétszáznyolczvanezeren valának; mindezek erõs vitézek.
\par 9 És kijöve õ ellenök a szerecsen Zérah, ezerszer ezer emberrel és háromszáz szekérrel, és méne Marésáig.
\par 10 Kiméne Asa is õ ellene, és viadalhoz készülének a Sefáta völgyben, Marésa mellett.
\par 11 Akkor kiálta Asa az Úrhoz, az õ Istenéhez, és monda: Oh Uram, nincs különbség elõtted a sok között és az erõ nélkül való között, hogy megsegítsed! Segélj meg minket, oh mi Urunk Istenünk, mert benned bízunk, és a te nevedben jöttünk e sokaság ellen! oh Uram, te vagy a mi Istenünk, ne vegyen ember te rajtad erõt.
\par 12 Megveré azért az Úr a szerecseneket Asa és Júda elõtt, és elfutának a szerecsenek.
\par 13 És üldözé õket Asa az õ seregével Gerárig, és elhullának a szerecsenek közül sokan, hogy közülök senki sem marada életben, mert leverettek az Úr elõtt és az õ serege elõtt; és hozának nagy zsákmányt.
\par 14 És Gérár környékén elpusztítának minden várost, mert az Úrtól való rettegés szállotta meg õket. És a városokat mind feldúlták, mivelhogy sok ragadomány vala azokban.
\par 15 A barmok tanyáit is lerombolták, és sok juhot és tevét elhajtának, s úgy tértek vissza Jeruzsálembe.

\chapter{15}

\par 1 És Azáriást, az Obed fiát felindítá az Isten lelke:
\par 2 A ki Asa elé lépett, és monda: Hallgassatok meg engem, Asa s egész Júda és Benjámin! Az Úr van veletek, ha ti is õ vele lesztek; ha õt keresenditek, megtaláljátok; de ha õt elhagyándjátok, õ is elhágy titeket.
\par 3 Sok ideje, hogy Izráel az igaz Isten nélkül, tanító pap nélkül, és törvény nélkül van.
\par 4 Ha megtért volna az õ nyomorúságában az Úrhoz, Izráel Istenéhez: megtalálták volna azok, a kik õt keresik.
\par 5 De ezekben az idõkben nincs békessége sem a kimenõnek, sem a hazajövõnek, mivelhogy nagy a nyomorúsága mindazoknak, a kik e földön laknak:
\par 6 Annyira, hogy egy nemzetség a másik nemzetséget és egyik város a másik várost elpusztítja; mert az Isten gyötri õket minden sanyarúsággal.
\par 7 Ti azért bátorságosok legyetek, kezeiteket le ne ereszszétek, mert a ti munkátoknak jutalma van.
\par 8 Mikor pedig Asa meghallotta e beszédeket és az Obed prófétának prófécziáját, megbátorodék, és elpusztítá a Júda és Benjámin földérõl mindenestõl a bálványokat, a városokból is, a melyeket elfoglalt az Efraim hegységén, és megújítá az Úr oltárát is, a mely az Úr tornácza elõtt volt.
\par 9 És összegyûjté az egész Júda és Benjámin nemzetségét és azokat, a kik jövevények valának köztök az Efraim, Manasse és Simeon nemzetségébõl; mert az Izráel nemzetségébõl sokan csatlakozának õ hozzá, látván, hogy az Úr, az õ Istene, õ vele volt.
\par 10 Összegyûlének azért Jeruzsálembe a harmadik hónapban, Asa királyságának tizenötödik esztendejében.
\par 11 És áldozának az Úrnak azon a napon a nyert zsákmányból hétszáz ökörrel és hétezer juhval;
\par 12 És fogadást tettek, hogy ezután az Urat, az õ atyáik Istenét teljes szívvel és teljes lélekkel fogják keresni.
\par 13 És ha valaki nem keresné az Urat, Izráel Istenét, megölettessék kicsinytõl fogva nagyig, úgy a férfi, mint az asszony.
\par 14 És megesküvének az Úrnak felszóval, kiáltással, trombita- és kürtszókkal.
\par 15 És örvendezett az egész Júda az eskû felett; mert teljes szívökbõl esküdtek, és egyenlõ akarattal keresték az Urat; és megtaláltaték általok, és az Úr nyugodalmat szerze nékik minden felõl.
\par 16 De még Maakát, Asa király anyját is megfosztá a királynéságtól, mivel egy iszonyú bálványt emelt vala Aserának, és Asa elrontá és összetörte annak iszonyú bálványát, és a Kedron patakjánál  megégeté.
\par 17 Jóllehet Izráelbõl a magaslatokat nem irtották ki, mindazáltal Asának tiszta szíve vala egész életében.
\par 18 És bevivé az Isten házába, a mit atyja és õ megszentelének, ezüstöt, aranyat és edényeket.
\par 19 És nem volt háború Asa királyságának harminczötödik esztendejéig.

\chapter{16}

\par 1 Asa király uralkodásának harminczhatodik esztendejében feljöve Baása, az Izráel királya Júda ellen, és  megépíté Rámát, hogy ne engedjen senkit se kimenni, se bemenni Asához a Júda királyához.
\par 2 De Asa az Úrnak és a királynak tárházából hoza ki ezüstöt, aranyat, és küldé azt Benhadádnak, a Siriabeli királynak, a ki lakik vala Damaskusban, mondván:
\par 3 Szövetség van köztem és te közötted, a mint az én atyám és a te atyád között is volt azelõtt. Ímé küldök néked ezüstöt és aranyat. Menj el, bontsd fel a te szövetségedet Baásával, az Izráel királyával, hogy távozzék el tõlem.
\par 4 És engedvén Benhadád Asa királynak, elküldé az õ seregének vezéreit Izráel városai ellen, és bevevé Ijont, Dánt, Abelmáimot és Nafthali minden kincses városait.
\par 5 A mit mikor meghallott Baása, abbanhagyta Ráma építését és megszünteté munkáját.
\par 6 Akkor Asa király felvevé az egész Júda népét, és Rámából a köveket és a fákat mind elhordák, a melyekkel Baása a vársot építi vala, és azokból Gébát és Mispát építé.
\par 7 Az idõben méne Hanáni próféta Asához, a Júda királyához, és monda néki: Mivel a Siriabeli királyban volt bizodalmad, és nem az Úrban, a te Istenedben bízál: ezért szabadult meg a Siriabeli király hada a te kezedbõl.
\par 8 Avagy nem vala-é a Szerecseneknek és a Libiabelieknek nagy seregök, felette sok szekereik és lovagjaik? Mindazáltal, mivel az Úrban volt bizodalmad, kezedbe adá azokat;
\par 9 Mert az Úr szemei forognak az egész földön, hogy hatalmát megmutassa azokhoz, a kik õ hozzá teljes szívvel ragaszkodnak; bolondul  cselekedél ebben; azért mostantól kezdve háborúk lesznek te ellened.
\par 10 Akkor megharaguvék Asa a prófétára, és veté õt a tömlöczházba, mert igen megharagudott vala e szóért reá; és ugyanakkor Asa a nép közül is sokat megnyomoríta.
\par 11 De ímé Asának mind elsõ, mind utolsó dolgai meg vannak írva a Júda és az Izráel királyainak könyvében.
\par 12 És megbetegedék Asa, királyságának harminczkilenczedik esztendejében lábaira, annyira, hogy igen súlyos volt az õ betegsége; mindazáltal betegségében is nem az Urat keresé, hanem az orvosokat.
\par 13 És elaluvék Asa az õ atyáival, és meghala az õ királyságának negyvenegyedik esztendejében.
\par 14 És eltemeték õt az õ sírjába, a melyet magának vágatott vala a Dávid városában; és helyezék õt az ágyba, a melyet megtöltének drága fûszerekkel, s kenõcscsé feldolgozott jó illatokkal, és érette felette nagy égést rendezének.

\chapter{17}

\par 1 Uralkodék pedig õ helyette az õ fia, Jósafát, és megerõsíté magát Izráel ellen.
\par 2 És sereget helyezett Júda minden erõs városaiba, és õrségeket helyezett Júda országába és Efraim városaiba, a melyeket az õ atyja, Asa meghódított vala.
\par 3 És az Úr Jósafáttal vala, mivel az õ atyjának Dávidnak elõbbi útain jára, és nem kére segítséget a bálványoktól,
\par 4 Hanem az õ atyjának Istenét kereste, és az õ parancsolatiban járt vala, és nem Izráelnek cselekedetei szerint.
\par 5 Azért az Úr megerõsíté a királyságot az õ kezében, és az egész Júda ada Jósafátnak ajándékot, s gazdagsága és dicsõsége igen nagy volt.
\par 6 És az õ szíve felemelkedett az Úr útjain, és még jobban kiirtá Júdából a magaslatokat és az Aserákat.
\par 7 Királyságának harmadik esztendejében elküldé az õ vezérei közül Benhailát, Obádiást, Zakariást, Nétanéelt és Mikáját, hogy tanítsanak a Júda városaiban,
\par 8 És velök Lévitákat: Semája, Nétánia, Zebádia, Asáel, Semirámót, Jónatán, Adónia, Tóbiás és Tóbadónia Lévitákat, és velök Elisáma és Jórám papokat.
\par 9 Tanítának azért Júdában, és az Úr törvényének könyve velök vala, mikor jártak vala Júda városaiban, tanítván a népet.
\par 10 Ezért az Úr igen megrettenté a földnek minden országait, a melyek Júda körül valának, annyira, hogy nem merének Jósafát ellen hadakozni.
\par 11 A Filiszteusoktól is hoznak vala Jósafátnak ajándékot és adópénzt; az Arábiabeliek is hoznak néki nyájakat, hétezerhétszáz kost és hétezerhétszáz bakot.
\par 12 És Jósafát mindig nagyobb és hatalmasabb lõn, és építe Júdában kastélyokat és tárházakat.
\par 13 És sok munkája vala néki Júda városaiban, és erõs hadakozó férfiakból álló serege volt Jeruzsálemben.
\par 14 Ez pedig azoknak száma nemzetségeik szerint: Júdában az ezredesek: Adna, a fõvezér, és vele háromszázezer harczos.
\par 15 Mellette Johanán volt a vezér, és vele kétszáznyolczvanezer ember.
\par 16 Mellette Amásia, a Zikri fia, a ki magát szabadakaratjából az Úrnak kötelezte vala; és õ vele kétszázezer harczos.
\par 17 A Benjámin nemzetségébõl vitéz harczos vala Eljada, és vele a kézívesek és paizsosok kétszázezeren.
\par 18 Mellette Józabád, és vele száznyolczvanezeren harczra felszerelve.
\par 19 Ezek szolgálnak vala a királynak azokon kivül, a kiket a király egész Júdában a megerõsített városokba helyezett.

\chapter{18}

\par 1 Jósafátnak nagy gazdagsága és dicsõsége vala. Õ sógorságot  szerze Akhábbal.
\par 2 Néhány esztendõ mulva aláméne Akhábhoz Samariába, és levágatott Akháb az õ és a vele való nép számára sok juhot és ökröt, és rávette õt, hogy felmenjen vele Rámóth Gileádba.
\par 3 Mert ezt mondá Akháb, az Izráel királya Jósafátnak, a Júda királyának: Feljösz-é velem Rámóth Gileádba? Felele néki, és monda: Úgy én, mint te; úgy az én népem, mint a te néped együtt lesz a harczban.
\par 4 Azután monda Jósafát az Izráel királyának: Kérdezõsködjél még ma az Úr beszéde után.
\par 5 És összegyûjté az Izráel királya a prófétákat, mintegy négyszáz férfiút, és monda nékik: Elmenjünk-é Rámóth Gileád ellen hadba, vagy elhagyjam? Felelének: Menj el, és az Isten a király kezébe adja.
\par 6 És monda Jósafát: Nincsen-é itt több prófétája az Úrnak, hogy attól is tudakozódhatnánk?
\par 7 És monda az Izráel királya Jósafátnak: Van még egy férfiú, a ki által az Urat megkérdezhetjük, de én gyûlölöm õt, mert soha nem jövendöl nékem jót, hanem mindig csak rosszat; ez Mikeás, a Jimla fia. És monda Jósafát: Ne beszéljen így a király!
\par 8 Szólíta azért az Izráel királya egyet az õ szolgái közül, és monda: Hívd ide hamar Mikeást, a Jimla fiát.
\par 9 És az Izráel királya és Jósafát, a Júda királya ott ülnek vala, kiki az õ királyiszékében, királyi ruhákba öltözötten; ott ülnek vala Samaria kapuja elõtt, a térségen; és a próféták mind prófétálnak vala õ elõttök.
\par 10 Csináltatott vala pedig magának Sédékiás, a Kénaána fia vasszarvakat, és monda: Ezt mondja az Úr: Ezekkel ökleled a Siriabelieket, mígnem megemészted õket!
\par 11 A többi próféták is mind ekképen jövendöltek, mondván: Menj fel Rámóth Gileád ellen, szerencsés leszel; mert azt az Úr a király kezébe adja.
\par 12 A követ pedig, a ki elment volt, hogy elhívná Mikeást, szóla néki, mondván: Ímé a próféták egyenlõ akarattal jót jövendölnek a királynak; szólj, kérlek, te is úgy, mint azok közül egy, és jövendölj jót.
\par 13 Akkor monda Mikeás: Él az Úr, hogy csak azt fogom mondani, a mit az én Istenem nékem mondánd!
\par 14 Mikor azért a király elé jutott, akkor monda a király néki: Mikeás! elmenjünk-é Rámóth Gileád ellen hadba, vagy elhagyjam? És monda: Menjetek el, és jó szerencsétek leszen, kezetekbe adattatnak azok.
\par 15 És monda a király néki: Hányszor esküdtesselek meg téged, hogy az igaznál egyebet ne mondj nékem az Úr nevében?
\par 16 Akkor monda: Látám az egész Izráelt elszéledve a hegyeken, mint a juhokat, melyeknek pásztoruk nincsen. És azt mondá az Úr: Nincsen ezeknek urok. Térjen meg kiki az õ házához békességben.
\par 17 És monda az Izráel királya Jósafátnak: Nemde nem megmondottam-é, hogy nem fog nékem jót jövendölni, hanem rosszat?
\par 18 Ismét monda: Halljátok meg azért most az Úr szavát: Látám az Urat ülni az õ királyiszékében, és az egész mennyei sereget jobb és balkeze felõl mellette állani.
\par 19 És monda az Úr: Kicsoda csalja meg Akhábot, az Izráel királyát, hogy felmenjen, és elveszen Rámóth Gileádban? És ki egyet, ki mást szóla.
\par 20 Akkor eljöve egy lélek, a ki megállván az Úr elõtt, monda: Én akarom megcsalni õt. Az Úr pedig monda néki: Mimódon?
\par 21 És felele: Kimegyek és leszek hazug lélek az õ összes prófétái szájában. Monda azért: Csald meg és gyõzd meg, menj ki és cselekedjél úgy.
\par 22 Ímé azért most az Úr adta a hazugságnak lelkét ezen te prófétáid szájába, és az Úr szólott veszedelmes dolgot ellened.
\par 23 Akkor oda lépett Sédékiás, a Kénaána fia, és arczul csapá Mikeást, és monda: Melyik úton távozott el az Úrnak lelke tõlem, hogy csak néked szólana?
\par 24 Felele Mikeás: Ímé meglátod magad azon a napon, a mikor egyik kamarából a másik kamarába mégy, hogy elrejtõzhessél.
\par 25 Akkor monda az Izráel királya: Fogjátok meg Mikeást, és vigyétek Amonhoz, a város fejedelméhez, és Joáshoz, a király fiához.
\par 26 És mondjátok: Ezt mondja a király: Vessétek õt a tömlöczbe, és tápláljátok õt a nyomorúság kenyerével és a nyomorúság vizével, míg békességgel megjövök.
\par 27 És monda Mikeás: Ha békével térsz vissza, akkor nem az Úr szólott én általam: Ismét monda: Halljátok meg minden népek!
\par 28 És felvonult az Izráel királya és Jósafát a Júda királya Rámóth Gileád ellen.
\par 29 És monda az Izráel királya Jósafátnak: Ruhámat megváltoztatom, és úgy megyek a viadalra; te pedig öltözzél fel ruhádba. És megváltoztatá az Izráel királya az õ ruháját, és menének viadalra.
\par 30 Siria királya pedig meghagyta vala az õ szekerei fejedelmeinek, mondván: Ne harczoljatok se kicsiny, se nagy ellen, hanem csak az Izráel királya ellen.
\par 31 És a mikor meglátták a szekerek fejedelmei Jósafátot, mondának: Ez az Izráel királya! És körülfogták õt, hogy legyõzzék. Akkor felkiálta Jósafát, és az Úr megsegéllé õt, és az Isten azokat elfordítá tõle;
\par 32 Mert mikor látták a szekerek fejedelmei, hogy nem az Izráel királya, ott hagyták.
\par 33 Egy férfi pedig kifeszíté kézívét csak úgy találomra, és találá az Izráel királyát a pánczél és a kapocs között. És õ monda az õ kocsisának: Fordulj meg és vígy ki engem a táborból, mert megsebesültem.
\par 34 És az ütközet mind erõsebb lett azon a napon, és az Izráel királya az õ szekerében állott a Siriabeliek ellen estvéig, és naplementekor meghala.

\chapter{19}

\par 1 Megtére pedig Jósafát, a Júda királya az õ házához Jeruzsálembe békével.
\par 2 És eleibe méne Jéhu próféta, a Hanáni fia, és monda Jósafát királynak: Avagy az istentelennek kellett-é segítségül lenned, és az Úrnak gyûlölõit szeretned? Ezért nagy az Úrnak haragja ellened.
\par 3 Mindazáltal némi jó dolog találtatott benned, hogy e földrõl kivágattad az Aserákat és az Istennek keresésére adtad magadat.
\par 4 És Jósafát egy ideig Jeruzsálemben tartózkodék, azután pedig kiméne a nép közé, Beersebától fogva mind az Efraim hegységéig, és megtéríté õket az Úrhoz, az õ atyáiknak Istenéhez;
\par 5 És rendele bírákat azon a földön, Júdának minden erõs városaiba, városonként.
\par 6 És monda a bíráknak: Jól meglássátok, a mit cselekesztek; mert nem ember nevében ítéltekm hanem az Úrnak nevében, a ki az ítéletben veletek lesz.
\par 7 Azért az Úr félelme legyen rajtatok, vigyázzatok arra, a mit tesztek; mert az Úrnál, a mi Istenünknél nincsen hamisság, sem személyválogatás, sem ajándékvétel.
\par 8 Sõt Jeruzsálemben is beállíta némelyeket Jósafát a Léviták, papok és az Izráel nemzetségeibõl való fejedelmek közül az Úr ítéletére és a perlekedésekre. És õk Jeruzsálembe visszatérének.
\par 9 És meghagyá nékik, mondván: Így cselekedjetek az Úrnak félelmében hûséggel és tökéletes szívvel.
\par 10 Ha valamely pert elõtökbe hoznak a ti atyátokfiai, a kik lakoznak az õ városaikban, emberhalál, törvény és parancsolat, a rendtartások és ítéletek miatt: intsétek õket, hogy ne vétkezzenek az Úr ellen, és ne szálljon reátok és a ti atyáitokfiaira az Úr haragja. Így cselekedjetek és ne vétkezzetek.
\par 11 És ímé Amária pap lesz a fõ ti köztetek az Úrnak minden dolgaiban; és Zebádia, az Ismáel fia lesz a Júda házának vezére a király minden dolgában; a Léviták is elõljáróitok lesznek. Legyetek azért erõsek a ti tisztetekben és az Úr mellette lesz az igaznak.

\chapter{20}

\par 1 És lõn ezek után, eljövének a Moáb fiai és Ammon fiai, és velök mások is az Ammoniták közül, Jósafát ellen, hogy hadakozzanak vele.
\par 2 Eljövének pedig a hírmondók, és megmondák Jósafátnak, mondván: A tenger tulsó részérõl nagy sokaság jön ellened Siriából, és már Haséson-Tamárban vannak; ez az Engedi.
\par 3 Megfélemlék azért Jósafát, és az Urat kezdé keresni és hirdete az egész Júda országában bõjtöt.
\par 4 Azért felgyûlének a Júdabeliek, hogy az Úr segedelmét keressék, Júdának minden városaiból is jövének, hogy az Urat megkeressék.
\par 5 És megálla Jósafát Júda és Jeruzsálem gyülekezetiben, az Úr házában az új pitvar elõtt;
\par 6 És monda: Oh Uram, mi atyáink Istene! nem te vagy-é egyedül Isten a mennyben, a ki uralkodol a pogányoknak minden országain? A te kezedben van az erõ és hatalom, és senki nincsen, a ki ellened megállhatna.
\par 7 Oh mi Istenünk! nem te ûzéd-é ki e földnek lakóit a te néped az Izráel elõtt, és nem te adád-é azt Ábrahámnak, a te barátod magvának mindörökké?
\par 8 És lakának azon és építettek azon a te nevednek szentséges hajlékot, mondván:
\par 9 A mikor veszedelem jövend mi reánk, háború, ítélet, döghalál vagy éhség, megállunk e házban elõtted (mert a te neved  e házban van) és a mikor kiáltunk hozzád a mi nyomorúságainkban: hallgass meg és szabadíts meg minket.
\par 10 És most ímé az Ammoniták, a Moábiták és a Seir hegyén lakozók, a kiknek földjén nem akarád, hogy általmenjenek az Izráel fiai, mikor Égyiptom földébõl kijöttek, hanem mellettök menének el és nem pusztíták el õket;
\par 11 Ímé ezért azzal fizetnek nékünk, hogy ellenünk jönnek, hogy kiûzzenek a te örökségedbõl, melyet örökségül adtál nékünk.
\par 12 Oh mi Istenünk, nem ítéled-é meg õket? Mert nincsen mi bennünk erõ e nagy sokasággal szemben, mely ellenünk jön. Nem tudjuk, mit cselekedjünk, hanem csak te reád néznek a mi szemeink.
\par 13 És a Júdabeliek mindnyájan állanak vala az Úr elõtt, gyermekeikkel, feleségeikkel és fiaikkal egyetemben.
\par 14 Akkor Jaházielre, a Zakariás fiára (ki Benája fia vala, ki Jéhiel fia, ki Mattániás fia vala, és az Asáf fia közül vala Lévita vala) az Úrnak lelke szálla, a gyülekezet közepette,
\par 15 És monda: Mindnyájan, a kik Júdában és Jeruzsálemben lakoztok, és te Jósafát király, halljátok meg szómat! Így szól az Úr néktek: Ne féljetek és ne rettegjetek e nagy sokaság miatt; mert nem ti harczoltok velök, hanem az Isten.
\par 16 Holnap szálljatok szembe velök! Ímé õk a Czicz hágóján fognak felmenni, és rájok találtok a völgynek szélénél, a Jeruel pusztájával szemben.
\par 17 Nem kell néktek harczolnotok, hanem csak álljatok veszteg, és lássátok az Úrnak szabadítását rajtatok. Júda és Jeruzsálem, ne féljetek és ne rettegjetek! Holnap menjetek ellenök, mert az Úr veletek lesz.
\par 18 Akkor Jósafát meghajtá fejét a föld felé, s Júda és Jeruzsálem lakói leborulának az Úr elõtt és imádák az Urat.
\par 19 A Kéhátiták fiai közül és a Kóriták fiai közül való Léviták pedig felállának, hogy az Urat, Izráel Istenét nagy felszóval dícsérjék.
\par 20 És reggel felkészülvén, kimenének a Tékoa pusztájára; és mikor kiindulnának onnan, megálla Jósafát, és monda: Halljátok meg szómat, Júda és Jeruzsálemben lakozók! Bízzatok az Úrban a ti Istentekben, és megerõsíttettek; bízzatok az õ prófétáiban, és szerencsések lesztek!
\par 21 Tanácsot tartván pedig a néppel, elõállítá az Úr énekeseit, hogy dícsérjék a szentség ékességét, a sereg elõtt menvén, és mondják: Tiszteljétek az Urat, mert örökkévaló az õ irgalmassága;
\par 22 És a mint elkezdették az éneklést és a dícséretet: az Úr ellenséget szerze az Ammon fiai és a Moábiták és a Seir hegyén lakozók ellen, a kik Júdára jövének, és megverettetének.
\par 23 Mert az Ammon és a Moáb fiai a Seir hegyén lakozók ellen támadának, hogy õket levágnák és elvesztenék; és mikor mind elvesztették a Seir hegyén lakozókat, azután egymás elpusztítását segítették elõ.
\par 24 A Júda népe pedig méne Mispába a puszta felé, és mikor a sokaság felé fordulának: ímé csak elesett holttestek valának a földön, és senki sem menekült meg.
\par 25 Akkor elméne Jósafát és az õ népe, hogy azoknak jószágait megzsákmányolják, és találának nálok temérdek gazdagságot és a holttesteken drága szép ruhákat, melyeket lefosztának rólok, oly sokat, hogy alig vihették el, és harmadnapig kapdosták a zsákmányt, mert felette sok vala.
\par 26 Negyednapra pedig gyûlének a hálaadásnak völgyébe, mivel az Úrnak ott adának hálákat; azért azt a helyet hálaadás völgyének nevezék mind e mai napig.
\par 27 Megtére azért Júdának és Jeruzsálemnek egész népe Jósafáttal, az õ fejedelmökkel egybe, hogy visszamenjen Jeruzsálembe nagy örömmel; mert az Úr megvigasztalta vala õket az õ ellenségeik felett.
\par 28 És bemenének Jeruzsálembe, lantokkal, cziterákkal és trombitákkal, az Úr házához.
\par 29 És lõn az Istennek félelme az országok minden királyságain, mikor meghallották, hogy az Úr hadakozott vala az Izráel ellenségei ellen.
\par 30 Megnyugovék azért a Jósafát országa, és békességet ada néki az õ Istene minden felõl.
\par 31 És uralkodék Jósafát Júda felett. Harminczöt esztendõs vala, mikor uralkodni kezde, és uralkodék Jeruzsálemben huszonöt esztendeig; és az õ anyjának neve Azuba vala, a Silhi leánya.
\par 32 És jára Asának, az õ atyjának útján, el sem távozék attól, cselekedvén azt, a mi az Úr szeme elõtt kedves dolog vala.
\par 33 Csakhogy még a magaslatok nem rontattak le, és a nép nem készítette az õ szívét az õ atyái Istenéhez.
\par 34 Jósafátnak pedig elsõ és utolsó dolgai ímé meg vannak írva Jéhunak, a Hanáni fiának könyvében, a ki azokat beírta az Izráel királyainak könyvébe.
\par 35 Azután Jósafát, a Júda királya megbarátkozék Akháziával, az Izráel királyával, a ki gonoszul cselekedett vala;
\par 36 Mindazáltal vele megbarátkozék, hogy hajókat készítenének, melyeken Társisba mennének; és a hajókat Esiongáberben készíték.
\par 37 Jövendöle azért Eliézer, a Maresából való Dódava fia Jósafát ellen, mondván: Minthogy megbarátkozál Akháziával, az Úr megsemmisíti a te munkádat. És a hajók mind összetörének, és nem mehetének Társisba.

\chapter{21}

\par 1 És meghala Jósafát az õ atyáival egyetemben, és eltemetteték az õ atyáival a Dávid városában; és uralkodék helyette az õ fia, Jórám.
\par 2 És az õ testvérei, a Jósafát fiai ezek valának: Azária, Jéhiel, Zakariás, Azáriás, Mikáel és Sefátja. Ezek mind Jósafátnak, az Izráel királyának fiai voltak.
\par 3 És adott nékik az õ atyjok sok ajándékot ezüstben, aranyban és drágaságokban, Júdabeli megerõsített városokkal; de a királyságot Jórámnak adá, mivel õ vala elsõszülötte.
\par 4 Kezde azért Jórám az õ atyjának királyságában uralkodni, és mikor immár abban megerõsödött, az õ testvéreit mind megölé fegyverrel; sõt Izráel fejedelmei közül is némelyeket.
\par 5 Harminczkét esztendõs korában kezdett vala uralkodni Jórám, és nyolcz esztendeig uralkodék Jeruzsálemben.
\par 6 És jára az Izráel királyainak útján, a mint cselekszik vala az Akháb háznépe; mert az Akháb leányát vette vala magának feleségül; és az Úr szemei elõtt gonosz dolgot cselekedék.
\par 7 Nem akará mindazáltal az Úr a Dávid házát elveszteni a szövetségért, a melyet  Dáviddal kötött, és mivel igéretet tett vala, hogy szövétneket ad néki és az õ fiainak minden idõben.
\par 8 Az õ idejében szakada el Edom Júda keze alól, és királyt választának magoknak.
\par 9 Elméne ugyan Jórám az õ vezéreivel és a szekerek mind õ vele, és felkelvén éjjel, megveré az Edomitákat, a kik õt körülvették vala, és szekereiknek fejedelmeit;
\par 10 Mindazáltal Edom elszakada Júdának keze alól mind e mai napig. Ugyanakkor elszakada Libna is az õ keze alól, mivel elhagyta az Urat, atyái Istenét.
\par 11 Õ is csináltatott magaslatokat Júda hegyein, és azt mûvelé, hogy a Jeruzsálembeliek paráználkodának, sõt Júdát is felbiztatá erre.
\par 12 Juta pedig azonközben az Illés próféta írása hozzá, mondván: Ezt mondja az Úr, a te atyádnak, Dávidnak Istene: Mivel nem járál a te atyádnak, Jósafátnak útján, sem a Júda királyának, Asának útján;
\par 13 Hanem járál az Izráel királyainak útján, és azt mûveléd, hogy Júda és Jeruzsálem lakói paráználkodjanak, a mint az Akháb háza is paráználkodik; annakfelette testvéreidet, atyádnak házát megöléd, a kik jobbak voltak nálad:
\par 14 Ímé az Úr nagy csapást bocsát a te népedre, fiaidra, feleségeidre és minden jószágodra.
\par 15 Te pedig súlyos betegségbe, bélbajba esel, mindaddig, míg a te béled naponként kimegy a betegség miatt.
\par 16 Felindítá azért az Úr Jórám ellen a Filiszteusok és az Arábiabeliek elméjét, a kik a szerecsenekkel határosok valának.
\par 17 És feljövének Júda ellen és megtámadván õt, zsákmányul vivék mindazt a vagyont, a mi a király házában található volt, sõt fiait és feleségeit is, és nem maradt néki más fia, csak Joákház, a legkisebbik.
\par 18 És mindezek után megveré õt az Úr felette nagy bélbajjal, mely gyógyíthatatlan vala.
\par 19 És ez így volt napról-napra, egészen a második év végéig, midõn belei kifolytak a betegség miatt, és meghala nagy kínokban: és népe nem égete néki drága illatú fûszereket, mint az õ atyáinak égettek vala.
\par 20 Harminczkét esztendõs vala, mikor uralkodni kezde, és nyolcz esztendeig uralkodék Jeruzsálemben. És mikor minden részvét nélkül kimula, eltemeték õt a Dávid városában; de nem a királyok sírjába.

\chapter{22}

\par 1 És királylyá tevék Jeruzsálem lakosai helyette az õ legkisebb fiát Akháziát; mert az idõsebbeket mind megölték azok, a kik az Arábiabeliekkel jöttek  vala a táborba. Uralkodék azért Akházia, Jórámnak a Júda királyának fia.
\par 2 Akházia negyvenkét esztendõs volt, mikor királylyá lett, és egy esztendeig uralkodék Jeruzsálemben. Anyjának neve Athália volt, a ki az Omri leánya vala.
\par 3 Õ is az Akháb házának útjain jára; mert az õ anyja vala néki tanácsadója az istentelen cselekedetre.
\par 4 És gonoszul cselekedék az Úr szemei elõtt, miképen az Akháb háznépe, mert azok voltak tanácsadói atyja halála után, az õ veszedelmére.
\par 5 Sõt azoknak tanácsa után indulva, hadba méne Jórámmal az Akháb fiával az Izráel királyával Hazáel ellen a Siriabeli király ellen, Rámóth Gileádba; de a Siriabeliek Jórámot megverék.
\par 6 Visszatért azért Jórám, hogy meggyógyíttassa magát Jezréel városában; mert sebek valának rajta, melyekkel megsebesíttetett Rámában, mikor Hazáel ellen, Siria királya ellen harczolt. Akházia pedig, Jórámnak a Júda királyának fia, aláméne, hogy meglátogatná Jórámot, az Akháb fiát Jezréelben, mert beteg vala.
\par 7 Hogy Akházia Jórámhoz méne, Isten akaratából, az õ romlása vala, mert odaérkezvén, elméne Jórámmal Jéhu ellen, a ki a Nimsi fia volt, a kit az Úr felkenetett, hogy kiírtaná az Akháb háznépét.
\par 8 Lõn azért, mikor Jéhu az Akháb házán bosszút álla, rátalált a Júda fejedelmeire és az Akházia testvéreinek fiaira, a kik Akháziának szolgálnak vala, és megölé õket.
\par 9 Akháziát is keresé és megfogák õt, (ki Samariában rejtõzött vala el) és vivék õt Jéhuhoz, a ki megöleté õt. De eltemették, mert ezt mondják róla: Mégis a Jósafát fia volt, a ki teljes szívvel keresé az Urat. És nem vala immár senki az Akházia háznépe közül a ki képes lett volna a királyságra.
\par 10 És Athália az Akházia anyja, látván, hogy az õ fia meghalt: felkele és megölé a Júda háznépének minden királyi sarját.
\par 11 De Jósabát a király leánya vevé Joást az Akházia fiát, és kivivé õt titkon a király fiai közül, a kik megölettek, és elrejté õt és az õ dajkáját az ágyasházban. És elrejté õt Jósabát, a Jórám király leánya, a ki a Jójada pap felesége volt, (mert Akháziának huga vala) Athália elõl, hogy õt meg ne ölhesse.
\par 12 És náluk vala az Úr házában elrejtve hat esztendeig. Athália pedig uralkodék az ország felett.

\chapter{23}

\par 1 A hetedik esztendõben pedig felbátorodván Jójada, szövetséget kötött a századosokkal, Azáriával a Jérohám fiával, Ismáellel a Jóhanán fiával, Azáriával az Obed fiával, Maaséjával az Adája fiával, és Elisafáttal a Zikri fiával;
\par 2 A kik Júda országát körüljárván, összegyûjték Júdának minden városaiból a Lévitákat és az Izráel családfõit, és jövének Jeruzsálembe.
\par 3 Szövetséget tõn pedig mind az egész gyülekezet a királylyal az Isten házában, minekutána Jójada ekképen szólott nékik: Ímé a király fia fog uralkodni, a mint az Úr szólott volt a Dávid fiairól;
\par 4 Azért ez az a dolog, amit cselekednetek kell: Harmadrész közületek, a kik a szombatra szoktatok feljönni a papok és a Léviták közül, ajtónálló legyen;
\par 5 Harmadrész a király házánál, és harmadrész a fõkapunál álljon; az egész nép pedig legyen az Úr házának pitvariban;
\par 6 Senki ne menjen be az Úr házába, hanem csak a papok és a kik szolgálnak a Léviták közül, csak õk menjenek be, mert szent dologra rendeltettek; az egész nép tartsa magát az Úr parancsolatjához.
\par 7 És a Léviták vegyék körül a királyt, fegyvere mindenkinek kezében legyen, és ha valaki bemenne a házba, ott megölettessék; és a király mellett legyetek, mikor bemegy és mikor kijön.
\par 8 És mind a szerint cselekedének a Léviták és az egész Júda, a mint Jójada pap megparancsolá, és kiki maga mellé vevé az õ embereit, mindazokat, a kik felmennek vala a szombatra, mind a kik kijõnek vala szombaton; mert Jójada pap nem ereszté el a csapatokat.
\par 9 És Jójada pap a századosoknak azokat a dárdákat, paizsokat és pánczélokat adá, a melyek Dávid királyéi voltak, a melyek az Isten házában valának.
\par 10 És állítá az egész népet, mindenkinek fegyvere kezében lévén, a ház jobb oldalától a ház bal oldaláig, az oltár mellett és a ház mellett a király körül.
\par 11 Akkor kihozák a király fiát, és reá tevék a koronát és a bizonyságtételt, és királylyá tevék õt, és megkenék õt Jójada és az õ fiai, mondván: Éljen a király!
\par 12 Mikor pedig meghallotta Athália a futkosó nép szavát, a kik a királyt dícsérik vala; akkor õ is felméne a nép közé az Úr házába.
\par 13 És mikor látta, hogy a király az õ oszlopánál áll a bejáratnál, s a fejedelmek és a trombitások a király mellett vannak, és hogy az egész föld népe örül, trombitál, s az énekesek a zengõ szerszámokkal énekelnek, a kik tudósok valának az isteni dícséretben: megszaggatá Athália az õ ruháit, és monda: Árulás! Árulás!
\par 14 És kiküldé Jójada pap a századosokat, a sereg elõljáróit, mondván nékik: Vezessétek ki a rendek között, és ha valaki utána menne, fegyverrel ölettessék meg. Mert ezt mondja vala a pap: Ne öljétek meg õt az Úr házában.
\par 15 Helyet adának azért néki, hogy kimehessen, és mikor jutott volna a király háza felé a lovak kapujáig, ott megölték õt.
\par 16 Szövetséget tõn pedig Jójada õ maga között, az egész nép között és a király között, hogy õk az Úr népei legyenek.
\par 17 És beméne az egész sokaság a Baál házába, és azt elrontá, és annak mind oltárait, mind bálványait összetöré; Mattánt pedig, a Baál papját az oltárok elõtt  ölék meg.
\par 18 És gondviselõket állíta Jójada az Úr házába, a lévitai papok által, a kiket Dávid csoportokba osztott az Úr házában, hogy áldoznának égõáldozatokkal az Úrnak, a mint a Mózes törvényében megíratott, nagy vígassággal és énekszóval, a Dávid rendelése szerint.
\par 19 És állítá az ajtónállókat az Úr házának kapuihoz, hogy be ne mehessen, a ki tisztátalan bármely dolog által.
\par 20 Azután maga mellé vevé a századosokat és a fõembereket és a kik a nép felett uralkodnak, s az egész ország népét, és kivezetvén a királyt az Úr házából, bemenének a király házának felsõ kapuján, és ülteték a királyt a királyiszékbe.
\par 21 És örvendezett a föld minden népe, és a város megnyugovék; minekutána Atháliát megölék fegyverrel.

\chapter{24}

\par 1 Hét esztendõs vala Joás, mikor uralkodni kezde, és uralkodék Jeruzsálemben negyven esztendeig; az õ anyjának neve Sibia vala, Beersebából.
\par 2 És cselekedék Joás az Úr elõtt kedves dolgot, Jójada papnak teljes életében.
\par 3 Vett pedig néki Jójada két feleséget, és nemze fiakat és leányokat.
\par 4 Ezek után elvégezé magában Joás, hogy megújítja az Úrnak házát.
\par 5 És összehivatá a papokat és a Lévitákat, és monda nékik: Menjetek el a Júda városaiba, és szedjetek az Izráel népétõl fejenként pénzt, hogy a ti Istentek háza esztendõnként kijavíttassék. Ti pedig siessetek e dologgal; de a Léviták nem sietének.
\par 6 Akkor hivatá a király Jójadát a papifejedelmet, és monda néki: Miért nem gondoltál a Lévitákra, hogy behozzák Júdából és Jeruzsálembõl az ajándékot, a melyet rendelt Mózes az Úr szolgája és az Izráel gyülekezete, a gyülekezet sátorához?
\par 7 Mert az istentelen Athália és az õ fiai elpusztították az Isten házát, és mindazt, a mi az Úr házának vala szentelve, a bálványokra költötték.
\par 8 És mikor parancsolt a király, csinálának egy ládát, a melyet az Úr házának kapuja elõtt helyezének el, kivül.
\par 9 És kihirdeték Júdában és Jeruzsálemben, hogy hozzák el az Úrnak az ajándékot, a melyet az Isten szolgája Mózes parancsolt a pusztában Izráelnek.
\par 10 Akkor a vezérek mindnyájan és az egész nép örömmel vivék az õ ajándékaikat és veték a ládába, míg megtelék.
\par 11 És idõnként a Léviták által elviteték a ládát a király gondviselõjéhez, és mikor látták, hogy sok pénz van benne, eljövén a király íródeákja és a fõpap választott embere, kiüríték a ládát, s azután ismét visszavitték a helyére. Ezt mûvelék idõnként, és nagy összeg pénzt gyûjtének.
\par 12 És adá azt a király és Jójada az Úr háza körül való munka felügyelõjének; és fogadának favágókat és ácsokat az Úr házának újítására, vas- és rézmûveseket is az Úr házának megerõsítésére.
\par 13 Munkálkodának azért a mûvesek, és az õ kezök által a kijavítás elõrehaladt, s az Úrnak házát elõbbi állapotába helyezék, és megerõsíték azt.
\par 14 Mikor pedig elvégezték, a megmaradt pénzt vivék a királynak és Jójadának, melybõl csinálának az Úr háza számára edényeket, az isteni tisztelet és áldozat számára kanalakat, s arany és ezüst edényeket. És áldozának vala égõáldozatokkal szüntelen az Úrnak házában, Jójadának, teljes életében.
\par 15 Megvénhedék pedig Jójada, és megelégedvén életével, meghala. Százharmincz esztendõs korában hala meg.
\par 16 És eltemeték õt a Dávid városában a királyok között, mivel kedves dolgot cselekedett vala Izráelben mind Istennel s mind az õ házával.
\par 17 Minekutána pedig meghala Jójada, eljövének a Júda fejedelmei, és meghajták magokat a király elõtt; a király pedig hallgatott reájok.
\par 18 És elhagyák az Úrnak, atyáik Istenének házát, és szolgálának az Aseráknak és bálványoknak; mely vétkök miatt lõn az Úrnak haragja Júda és Jeruzsálem ellen.
\par 19 És külde hozzájuk prófétákat, hogy visszatérítenék õket az Úrhoz, a kik bizonyságot tevének ellenök, de nem hallgattak reájok.
\par 20 Az Isten lelke pedig felindítá Zakariást, a Jójada pap fiát, a ki felálla a nép között, és monda nékik: Ezt mondja az Isten: Miért szegtétek meg az Úrnak parancsolatait? - mert az nem használ néktek. Ha elhagytátok az Urat, õ is elhagy titeket.
\par 21 Amazok pedig reá támadván, ott az Úr háza pitvarában megkövezék õt a király parancsolatjából.
\par 22 És nem emlékezék meg Joás király a jótéteményrõl, a melylyel annak atyja, Jójada vala õ hozzá éltében, hanem megöleté a fiát. Mikor pedig meghalna, ezt mondá: Látja az Úr és bosszút áll!
\par 23 És már az esztendõ elmúltával feljöve ellene Siria királyának serege, és méne Júdára és Jeruzsálemre, és kiirtották a népnek minden vezéreit a nép közül, és minden zsákmányukat küldék Damaskusba a királynak;
\par 24 Mert noha kevés emberrel jött vala rájok a Siriabeli had, mindazáltal az Úr kezökbe adá Júdának nagy seregét, mivel az Urat, atyáiknak Istenét elhagyták; és Joáson is bosszút állának.
\par 25 És mikor tõle elmentek (súlyos betegségben hagyták hátra): pártot ütének ellene az õ szolgái, a Jójada pap fiának haláláért, és megölék õt ágyában, és meghala. És eltemeték õt a Dávid városában, de nem temeték õt a királyok sírjába.
\par 26 Ezek ütöttek ellene pártot: Zabád, az Ammonbeli Simeát asszony fia, és Józabád, a Moábbeli Simrith fia.
\par 27 Az õ fiai, és alatta az adónak megszaporodása, s az Isten házának kijavítása, ímé meg vannak írva a királyok könyvének magyarázatában. Uralkodék helyette az õ fia, Amásia.

\chapter{25}

\par 1 Huszonöt esztendõs korában kezdett Amásia uralkodni, és huszonkilencz esztendeig uralkodott Jeruzsálemben; az õ anyjának neve Jéhoaddán vala, Jeruzsálembõl való.
\par 2 És kedves dolgot cselekedék az Úr elõtt; de nem tiszta szívbõl.
\par 3 Lõn pedig azután, hogy országában megerõsödék, megölé az õ szolgáit, a kik a királyt, az õ atyját megölték vala.
\par 4 De azoknak fiait nem öleté meg, hanem a szerint cselekedék, a mint a törvényben, a Mózes könyvében megiratott, a melyben az Úr parancsolt volt, mondván: Meg ne ölettessenek az atyák a fiakért és a fiak se ölettessenek meg az atyákért, hanem kiki az õ saját bûnéért ölettessék meg.
\par 5 Összegyûjté annak felette Amásia a Júda népét, és választa közülök a nemzetségek szerint egész Júdában és Benjáminban ezredeseket és századosokat; és megszámlálá õket a húsz esztendõsöktõl fogva és a kik feljebb valának, és talála azok közül válogatott fegyverfoghatókat, kopjásokat és paizsosokat, háromszázezeret.
\par 6 Annakfelette az Izráeliták közül százezer erõs vitézt fogadott fel, száz talentom ezüstön.
\par 7 Eljöve pedig az Isten embere õ hozzá, mondván: Oh király! ne menjen el te veled Izráel serege, mert az Úr nem lesz Izráellel, Efraim minden fiaival.
\par 8 Ha nem hiszed, ám menj el, készülj a viadalhoz; de megver az Isten téged az ellenség elõtt; mert az Isten hatalmában van mind a segítség, mind a megveretés.
\par 9 Akkor monda Amásia az Isten emberének: De mit tegyünk a száz talentom ezüsttel, a melyet Izráel seregének adtam? És felele az Isten embere: Az Úr néked annál sokkal többet adhat.
\par 10 Kiválasztá azért Amásia azt a sereget, a mely Efraimból jött vala õ hozzá, hogy mennének helyökre; mely dologért igen megharagvának a Júda népére, és felgerjedt haraggal tértek vissza helyeikre.
\par 11 Amásia pedig felbátorodván, elindítá népét, és méne a sós völgybe; és megvere a Seir fiai közül tízezeret.
\par 12 És Júda fiai tízezeret élve fogtak el, a kiket egy magas kõszikla tetejére vivének, és letaszították õket a magas kõszikláról, és mindnyájan összeroncsoltattak.
\par 13 Annak a seregnek fiai pedig, a kiket visszakülde Amásia, hogy ne menjenek õ vele hadba: Júdának városaira ütének Samariától fogva mind Bethóronig; és levágván háromezeret azok közül, nagy zsákmányt vivének el.
\par 14 Lõn azután, hogy Amásia megtére az Edomiták megverésébõl, a Seir fiainak isteneit elhozá  és Isten gyanánt tisztelé azokat, a kik elõtt magát meghajtja vala, és nékik jóillatot gerjeszte.
\par 15 Ezért megharaguvék az Úr Amásiára, és prófétát külde hozzá, a ki monda néki: Miért imádod annak a népnek isteneit, a kik nem szabadíthatták meg az õ népöket a te kezedbõl?
\par 16 Lõn pedig, mikor ekképen szólott volna néki, monda néki a király: Vajjon te tanácsosa vagy-é a királynak? Hallgass, mert rosszul jársz. Megszünék azért a próféta, minekutána ezt mondotta volna: Látom, hogy az Isten el akar téged veszteni,  mivel ezt mûveléd, és tanácsomat nem fogadád meg.
\par 17 Amásia pedig, a Júda királya, tanácsot tartván, követet külde Joáshoz, a Joákház fiához, a ki Jéhunak fia vala, az Izráel királyához, mondván: Nosza, szálljunk szembe egymással!
\par 18 Akkor Joás, az Izráel királya ilyen választ ada Amásiának, a Júda királyának: A Libánus hegyén való tövis külde a Libánuson való czédrusfához, mondván: Add a te leányodat az én fiamnak feleségül; eközben azonban arra menvén egy fenevad, a mely a Libánuson lakik vala, eltapodá azt a tövist.
\par 19 Te magadban így gondolkodtál: Megveréd az Edomitákat, azért fuvalkodtál fel magadban, hogy dicsekedjél. Kérlek, maradj otthon, miért szereznél magadnak veszedelmet, hogy te és Júda elveszszen általam.
\par 20 De Amásia nem nyugodhatott, mert Isten elvégezte vala, hogy az ellenség kezébe adja õket, mivel az Edomiták isteneit keresték.
\par 21 Felindula azért Joás, az Izráel királya, és szembeszállának egymással õ és Amásia, a Júda királya Béth-Semesnél, a mely Júdában van.
\par 22 És Júda megveretteték Izráel által, és elmenekülének mindnyájan sátoraikba.
\par 23 Amásiát pedig, a Júda királyát, a Joás fiát, a ki Joákház fia volt, Joás, az Izráel királya elfogá Béth-Semesben, és vivé õt Jeruzsálembe, és Jeruzsálem kõfalát lerontá az Efraim kaputól fogva mind a szeglet kapujáig négyszáz singnyire.
\par 24 És az aranyat, az ezüstöt és mindenféle edényeket, a melyek az Isten házában, az Obed-Edom birtokában találtatának, és a király házának kincseit, s a kezesek fiait mind Samariába vivé.
\par 25 Amásia, a Joás fia, a Júda királya, minekutána meghala Joás, a Joákház fia, az Izráel királya, még tizenöt esztendeig éle.
\par 26 Amásiának pedig többi dolgai, az elsõk és utolsók, avagy nincsenek-é megírva a Júda és az Izráel királyainak könyvében?
\par 27 Azon idõtõl fogva pedig, hogy Amásia az Úrtól elszakada, összeesküvést szõttek ellene Jeruzsálemben; és elmeneküle Lákisba, de utána küldöttek Lákisba, és megölték ott õt.
\par 28 És elhozák onnét lovakon, és eltemeték õt az õ atyáival, Júdának városában.

\chapter{26}

\par 1 Akkor elõhozván az egész Júda nemzetsége Uzziást (ki tizenhat esztendõs vala), királylyá tevék õt az õ atyja Amásia helyett.
\par 2 Õ építé meg Elótot, és csatolta ismét Júdához, minekutána Amásia király meghalt az õ atyáival egybe.
\par 3 Tizenhat esztendõs korában kezdett vala uralkodni Uzziás, és ötvenkét esztendeig uralkodék Jeruzsálemben; az õ anyjának neve Jékólia vala, Jeruzsálembõl való.
\par 4 És az Úr elõtt kedves dolgot cselekedék, a mint az õ atyja, Amásia is cselekedett vala.
\par 5 És keresi vala az Istent Zakariás próféta idejében, a ki az isteni látásokban értelmes vala; és mindaddig, míg az Urat keresé, jó elõmenetelt adott néki Isten;
\par 6 Mert kimenvén, hadakozék a Filiszteusok ellen; és a Gáth kerítését, a Jabné kerítését és az Asdód kerítését letöré, és építe városokat Asdódban és a Filiszteusok tartományában.
\par 7 És megsegéllé õt az Isten a Filiszteusok ellen és az Arábiabeliek ellen, a kik lakoznak vala Gúr-Baálban, és Meunimban.
\par 8 És adának az Ammoniták Uzziásnak ajándékot, és elterjede az õ híre Égyiptomig; mert felette igen megnevekedett az õ hatalma;
\par 9 És építe Uzziás tornyokat Jeruzsálemben a szeglet kapuja felett, a völgy kapuja felett és a szegletek felett, és igen megerõsítteté azokat.
\par 10 A pusztában is tornyokat épített, és sok kutat ásatott; mert sok nyája vala mind a völgyekben, mind a lapályon, és szántóvetõ szolgái, vinczellérei a hegyeken és Kármelben, mert a földmûvelést kedvelte.
\par 11 Uzziásnak hadakozó serege is vala, a mely harczba mehetett csapatonként, a mint megszámláltatott Jéhiel íródeák és Maaséja elõljáró által Hanániás vezetése alatt, a ki a király vezérei közül való.
\par 12 A családfõk egész száma a hadakozó vitézek között kétezerhatszáz vala.
\par 13 És az õ kezök alatt levõ hadakozó sereg háromszázhétezerötszáz harczosból állott, a kik képesek valának megsegíteni a királyt ellenségei ellen.
\par 14 És készíttete Uzziás nékik, az egész sereg számára paizsokat, kopjákat, sisakokat, pánczélokat, íveket és parittyába való köveket.
\par 15 Készíttete annakfelette Jeruzsálemben értelmes mesteremberek által gépezeteket a tornyok tetején és a kõfal szegletein, nyilaknak és nagy köveknek kihajigálására. És az õ híre messzire elterjede; mert csudálatosan megsegítteték, míglen megerõsödék.
\par 16 Mikor pedig ilyen módon megerõsödött volna, felfuvalkodék, hogy megfertõztetné magát és vétkezék az Úr ellen, az õ Istene ellen. Beméne az Úr templomába, hogy a füstölõ oltáron füstölne.
\par 17 És beméne õ utána Azáriás pap, és vele az Úr papjai nyolczvanan, igen erõsek.
\par 18 És ellene állának Uzziás királynak, és mondának néki: Uzziás! nem a te dolgod az Úrnak füstölni, hanem az Áron pap fiaié, a kik felszenteltetének, hogy füstöljenek. Menj ki e szent helybõl; mert igen vétkeztél és dicsõségedre nem leend az Úr Istentõl.
\par 19 És megharaguvék Uzziás, a kinek kezében vala a fülstölõ szerszám, hogy füstölne; és mikor haragudnék a papokra, bélpoklosság támada a homlokán ott a papok elõtt, az Úr házában, a füstölõ oltár elõtt.
\par 20 És mikor tekintett volna õ reá Azáriás fõpap, és vele mind a többi papok, láták a bélpoklosságot az õ homlokán; és elûzék õt onnét, sõt maga is sietett kimenni, mert az Úr megverte vala õt.
\par 21 És lõn Uzziás király halála napjáig bélpoklos; és lakik vala egy elkülönített házban bélpoklosan, mert az Úr házából kivettetett vala. És Jótám, az õ fia vala a király házában, a ki ítélkezék az ország népe felett.
\par 22 Uzziásnak pedig elsõ és utolsó dolgait megírta Ésaiás próféta, az Ámós fia.
\par 23 És meghala Uzziás az õ atyáival egybe, és eltemeték õt az õ atyáival a temetõbe, a mely a királyoké vala, mert ezt mondják felõle: bélpoklos volt. És uralkodék helyette Jótám, az õ fia.

\chapter{27}

\par 1 Huszonöt esztendõs vala Jótám, mikor uralkodni kezde, és tizenhat esztendeig uralkodék Jeruzsálemben; az õ anyjának neve Jérusa vala, a Sádók leánya.
\par 2 És kedves dolgot cselekedék az Úr elõtt, a mint az õ atyja, Uzziás is cselekedett vala, csakhogy nem méne az Úr templomába; a nép azonban tovább is vétkezék.
\par 3 Õ építé meg az Úr házának felsõ kapuját; a vár kõfalán is sokat építe.
\par 4 Annakfelette a Júda hegyes földén városokat építe, és a ligetekben palotákat és tornyokat építe.
\par 5 Õ is hadakozott az Ammon fiainak királyai ellen, a kiket megvere; és adának néki az Ammon fiai azon esztendõben száz tálentom ezüstöt s tízezer véka búzát és tízezer véka árpát. Ezt fizették néki az Ammon fiai a második és harmadik esztendõben is.
\par 6 És hatalmassá lõn Jótám, mert útjában az Úr elõtt, az õ Istene elõtt járt.
\par 7 Jótámnak pedig több dolgait, minden hadakozásait és útjait, ímé megírták az Izráel és a Júda királyainak könyvében.
\par 8 Huszonöt esztendõs korában kezdett uralkodni, és tizenhat esztendeig uralkodék Jeruzsálemben.
\par 9 És elaluvék Jótám az õ atyjával, és eltemeték õt a Dávid városában; és uralkodék Akház, az õ fia helyette.

\chapter{28}

\par 1 Húsz esztendõs volt Akház, mikor uralkodni kezdett, és tizenhat esztendeig uralkodék Jeruzsálemben, és nem cselekedék kedves dolgot az Úr elõtt, mint Dávid, az õ atyja;
\par 2 Hanem az Izráel királyainak útján járt, és öntött bálványokat is csináltatott a Baál tiszteletére.
\par 3 Annakfelette tömjéneze a Hinnom fiának völgyében; fiait is megégeté tûzben, a pogányok útálatosságai szerint, a kiket az Úr az Izráel fiai elõtt kiûzött volt.
\par 4 Áldozék és tömjéneze a magaslatokon is, a halmokon is és minden zöld fa alatt.
\par 5 Ezért az Úr az õ Istene adá õt a Siriabeli király kezébe, és õt igen megverék, és sok foglyot hurczolának el õ tõle, a kiket Damaskusba vivének. Sõt még az Izráel királya kezébe is adaték, és az is igen megveré õt.
\par 6 Mert Pékah, a Rémália fia, Júdában egy nap levága százhúszezer embert, mind vitézeket; mivel elhagyták az Urat, atyáik Istenét.
\par 7 Annakfelette az Efraimbeli vitéz Zikri megölé Maásiát, a király fiát, Azrikámot, az õ házának gondviselõjét, és Elkánát, a ki a király után második vala.
\par 8 És elvivének az Izráel fiai az õ atyjokfiai közül kétszázezer asszonyt, fiút, leányt, és nagy vagyont rablának el tõlök, és azzal a zsákmánynyal mennek vala Samariába.
\par 9 Vala pedig ott az Úrnak egy prófétája, a kinek neve Odéd, a ki eleibe menvén a hadnak, a mely Samariába megy vala, monda nékik: Ímé, mivel az Úrnak, atyáitok Istenének haragja Júda ellen felgerjedett, õket kezetekbe adta, és ti sokat megölétek közülök haragotokban, a mely szintén az égig felhatott;
\par 10 És immár arra gondoltok, hogy Júdának és Jeruzsálemnek fiait megalázzátok, hogy néktek szolgáitok és szolgálóleányitok legyenek: avagy ezáltal nem teszitek-é magatokat bûnösökké az Úrnál, a ti Isteneteknél?
\par 11 Azért halljátok meg szómat: Vigyétek vissza a ti atyátokfiai közül való foglyokat, a kiket ide hoztatok; különben az Úrnak nagy haragja lészen rajtatok.
\par 12 Akkor felkelének némelyek az Efraimból való vezérek közül: Azáriás, a Jóhanán fia; Berékiás, a Mesillemót fia; Ezékiás, a Sallum fia, és Amása a Hadlai fia, azok ellen, a kik a viadalból jõnek vala.
\par 13 És mondának nékik: Ne hozzátok ide be a foglyokat, mert felette igen nagy bûn lesz az, a mit akartok mûvelni, és ezzel a mi bûneinket és vétkeinket megsokasítjátok; mert e nélkül is sok bûnünk van, és felgerjedt a harag Izráel ellen.
\par 14 Ott hagyá azért a sereg a foglyokat és a zsákmányt a vezérek és az egész gyülekezet elõtt.
\par 15 És felállának a névszerint megnevezett férfiak, s felvevék a foglyokat, és a kik mezítelenek valának közülök, felöltözteték a zsákmányból; felöltözteték azokat, sarukat is adának lábaikra; ételt és italt is adának nékik; sõt meg is kenék õket, és a gyengélkedõket szamarakra helyezék, és vivék õket a pálmafák városába, Jérikhóba, az õ atyjokfiaihoz; azután megtérének Samariába.
\par 16 Az idõben külde Akház király az Assiriabeli királyhoz, hogy megsegítené õt.
\par 17 Mert még az Edomiták is eljöttek vala, és a Júdabeliek közül sokat levágának, vagy rabságba hurczolának.
\par 18 A Filiszteusok is mind ellepék a lapályon való városokat és Júdának dél felõl való részét, és elfoglalák Béth-Semest, Ajalont, Gederótot, Sókot és annak faluit; Timnát és annak faluit; Gimzót és annak faluit, és ott laknak vala;
\par 19 Mert az Úr megalázta Júdát Akházért, az Izráel királyáért; mert arra indítá Júdát, hogy vétkezzék az Úr ellen.
\par 20 Eljöve azért õ ellene Tiglát-Piléser, Assiria királya, a ki sanyargatá õt, és nem segítette meg.
\par 21 Mert Akház kifosztá az Úr házát, a királyét, a fejedelmekét, és az Assiriabeli királynak adá, de azért nem lõn néki segítségére.
\par 22 Sõt még a szorongattatás idejében is tovább vétkezék az Úr ellen; ilyen vala Akház király.
\par 23 Mert áldozék Damaskus isteneinek, a kik õt megverték vala, ezt mondván: Mivel Siria királyainak istenei megsegítik õket, azért én is azoknak áldozom, hogy segéljenek engem is, holott mind néki, mind az egész Izráelnek azok okozták romlását.
\par 24 És összehordá Akház az Isten házának edényeit, és összetöré az Isten házának edényeit, és az Úr házának ajtait bezárá, és csinála a maga számára oltárokat Jeruzsálemnek minden szegletén;
\par 25 Júdának minden városaiban is magaslatokat építe, hogy az idegen isteneknek tömjénezzen, és haragra ingerlé az Urat, atyái Istenét.
\par 26 Az õ több dolgai pedig és útjai, úgy az elsõk, mint utolsók, ímé meg vannak írva a Júda és az Izráel királyainak könyvében.
\par 27 Meghala pedig Akház az õ atyáival, és eltemeték õt Jeruzsálem városában; mert nem vivék õt az Izráel királyainak sírjába. És uralkodék az õ fia, Ezékiás, õ helyette.

\chapter{29}

\par 1 Ezékiás huszonöt esztendõs korában kezdett uralkodni, és uralkodék huszonkilencz esztendeig Jeruzsálemben; az õ anyjának neve Abija, a Zakariás leánya.
\par 2 És kedves dolgot cselekedék az Úr elõtt, mind a szerint, a mint Dávid, az õ atyja is cselekedett vala.
\par 3 És az õ királyságának elsõ esztendejében, az elsõ hónapban kinyitá az Úr házának ajtait, és azokat megújíttatá.
\par 4 És egybehivatá a papokat és a Lévitákat, és összegyûjté õket a napkelet felõl való utczában;
\par 5 És monda nékik: Hallgassatok meg engem Léviták! Most szenteljétek meg magatokat, az Úrnak, atyáitok Istenének házát is szenteljétek meg, és hordjatok ki minden tisztátalanságot a szent helyrõl;
\par 6 Mert vétkeztek a mi atyáink, és az Úr elõtt, a mi Istenünk elõtt gonoszul cselekedének, és elhagyták õt, az Úr sátorától elfordították arczukat, hátat fordítván annak.
\par 7 A tornácz ajtait is bezárták, a szövétnekeket eloltották, és füstölõszert nem füstölögtettek és égõáldozatot nem áldoztak az Izráel Istenének a szent helyen.
\par 8 És ezért volt az Úrnak haragja Júdán és Jeruzsálemen, és adta volt õket rabságra és pusztulásra és kigunyoltatásra, a mint ti magatok is látjátok.
\par 9 És ímé a mi atyáink fegyver által hullottak el, fiaink, leányaink és feleségeink fogságba vitettek e dolog miatt.
\par 10 Most azért elvégeztem magamban, hogy az Úrral, Izráel Istenével szövetséget szerzek, hogy haragját tõlünk elfordítsa.
\par 11 Fiaim, most ne tévelyegjetek; mert az Úr választott titeket, hogy õ elõtte állván, néki szolgáljatok, hogy szolgái legyetek néki és jóillatot szerezzetek.
\par 12 Felkelének azért a Léviták: Máhát az Amásai fia, Jóel az Azárja fia, a kik Kéhátiták valának; a Mérári fiai közül pedig Kis az Abdi fia, és Azária a Jéhalélel fia, és a Gersoniták közül Joah a Zimma fia, és Éden a Joah fia;
\par 13 Az Elisáfán fiai közül Simri és Jéhiel; az Asáf fiai közül Zakariás és Mattánia;
\par 14 A Hemán fiai közül Jéhiel és Simei; a Jédutun fiai közül Semája és Uzziel.
\par 15 Összegyûjték az õ atyjokfiait, és megszentelék magokat, és bemenének a király parancsolatjából az Úrnak beszédei szerint, az Úr házának megtisztítására.
\par 16 És bemenének a papok az Úr házának belsõ részébe, hogy azt megtisztítsák; kihordának belõle minden tisztátalanságot, a melyet az Úr templomában találának, az Úr házának pitvarába; és a Léviták felszedék, hogy onnan kihordják a Kidron patakába.
\par 17 Elkezdék pedig a megszentelést az elsõ hónap elsõ napján, és a hónap nyolczadik napján bemenének az Úr házának tornáczába, és megszentelék az Úr házát nyolcz napon át, úgy hogy az elsõ hónap tizenhatodik napján végezték be.
\par 18 És akkor bemenének Ezékiás királyhoz, és mondának: Megtisztítottuk mindenestõl az Úr házát, az égõáldozat oltárát is, minden hozzá tartozó edényekkel egybe, a szent asztalt is, minden szerszámaival;
\par 19 Minden egyéb eszközöket is, a melyeket Akház király az õ királysága alatt megszentségtelenített, mikor Isten ellen vétkezett vala, helyreállítottunk és megszenteltünk, és ímé mind az Úr oltára elõtt vannak.
\par 20 Reggel azért felkele Ezékiás király, és összegyûjté a város fejedelmeit, és felméne az Úr házába.
\par 21 És vivének fel hét tulkot és hét kost, hét bárányt és hét bakot bûnért való áldozatra az országért, a szent hajlékért és Júdáért, és megparancsolá az Áron fiainak, a papoknak, hogy megáldozzák az Úr oltárán.
\par 22 Megölék azért a tulkokat, és a papok azoknak véröket vévén, elhinték az oltárra; hasonlatosképen megölvén a kosokat, elhinték azoknak véröket az oltárra; a bárányokat is megölvén, azoknak vérét az oltárra hinték.
\par 23 Azután elõhozák a bûnért való bakokat a király és a gyülekezet elé, és kezöket rájok tevék.
\par 24 Minekutána a papok azokat megölték, bûnért való áldozást végeztek a vérökkel az oltáron, az egész Izráel megtisztulására; mert az egész Izráelért parancsolta vala a király az égõáldozatot és a bûnért való áldozatot.
\par 25 És beállítá a Lévitákat az Úr házába czimbalmokkal, lantokkal és cziterákkal Dávidnak és Gádnak a király prófétájának, és Nátán prófétának parancsolatja szerint; mert az Úrtól volt a parancs az õ prófétái által.
\par 26 Elõállának azért a Léviták a Dávid zengõ szerszámaival; a papok is a trombitákkal.
\par 27 És megparancsolá Ezékiás, hogy egészen égõáldozatot áldozzanak az oltáron. És mikor megkezdõdött az áldozás ugyanakkor megkezdõdött az Úrnak éneke is és a trombiták harsonája Dávidnak az Izráel királyának szerszámaival.
\par 28 És az egész gyülekezet leborula, az énekesek énekelének, és a trombitások trombitálának mindaddig, míg az egészen égõáldozatnak vége lõn.
\par 29 És a mikor elvégezték az áldozatokat, a király és mindnyájan a vele valók leborulván arczczal, imádkozának.
\par 30 És megparancsolá Ezékiás király és a fejedelmek a Lévitáknak, hogy az Urat dícsérjék a Dávid és az Asáf próféta dicséreteivel; a kik, mikor nagy örömmel dícsérték az Urat, meghajoltak és leborulának.
\par 31 Azután szóla és monda Ezékiás: Most már felavattátok magatokat az Úrnak, azért jõjjetek, és hozzatok áldozatokat és dícsérõ áldozatokat az Úr házában. És az egész gyülekezet hoza áldozatokat és dícsérõ áldozatokat, és mindaz a kit a szíve indított, egészen égõáldozatot.
\par 32 És az égõáldozatra való barmok száma, a melyeket a gyülekezet hozott vala, ez volt: hetven tulok, száz kos, kétszáz bárány. Mindezek egészen égõáldozatul valának az Úrnak.
\par 33 Azonkivül szenteltek az Úrnak hatszáz tulkot és háromezer juhot.
\par 34 De mivel a papok kevesen valának, és nem gyõzték az áldozatokat mind megnyúzni, ezért az õ atyjokfiai, a Léviták segítségökre valának nékik mindaddig, míg azt a munkát elvégezék, és míg a többi papok magokat megszentelék; mert a Léviták igazabb szívûek valának a magok megszentelésében, mint a papok.
\par 35 Az egészen égõáldozat is igen sok volt a hálaáldozatok kövérivel és az egészen égõáldozatokhoz való italáldozatokkal; és ekképen helyreállíttatott az Úr házának szolgálata.
\par 36 Örvendeze azért Ezékiás és õ vele az egész gyülekezet, hogy az Isten erre hajlandóvá tette a népet; mert hirtelen történt ez a dolog.

\chapter{30}

\par 1 És külde Ezékiás az egész Izráelhez és Júdához, sõt Efraimnak és Manassének is íra leveleket, hogy jönnének el az Úr házához Jeruzsálembe, hogy megtartanák a páskhát az Úrnak, Izráel Istenének.
\par 2 És tanácsot tarta a király és fejedelmei s az egész gyülekezet Jeruzsálemben, hogy a második hónapban tartsák meg a páskhát.
\par 3 Mert azt akkor nem tarthatták meg, mert a papok nem szentelhették meg magokat kellõ számban, s a nép sem gyûlt vala össze Jeruzsálemben.
\par 4 És e dolog igen tetszék mind a királynak, mind az egész gyülekezetnek.
\par 5 Elhatározták tehát, hogy kihirdetik egész Izráelben Beersebától fogva Dánig, hogy jõjjenek Jeruzsálembe páskhát szentelni az Úrnak, Izráel Istenének, mert már régóta nem tartották meg úgy, a mint megiratott.
\par 6 Elmenének azért a híradók a király levelével és a fejedelmekével az egész Izráelhez és Júdához, és a király parancsolatjából ezt mondják vala: Izráel fiai! térjetek meg az Úrhoz, Ábrahámnak, Izsáknak és Izráelnek Istenéhez s õ is megtér a maradékhoz, mely megmaradt még köztetek az Assiriabeli királyok kezétõl.
\par 7 És ne legyetek olyanok, mint a ti atyáitok és atyátokfiai, a kik vétkeztek volt az Úr ellen, atyáik Istene ellen és elpusztította õket, a mint ti magatok látjátok.
\par 8 Most azért meg ne keményítsétek nyakatokat, mint a ti atyáitok; adjatok kezet az Úrnak, és bemenvén az õ szent helyébe, a melyet megszentelt örökké, szolgáljatok az Úrnak, a ti Istenteknek, és elfordul rólatok haragja.
\par 9 Mert hogy ha ti megtéréndetek az Úrhoz, a ti atyátokfiai és fiaitok kegyelmet találnak azoknál a kik õket fogságba vitték, és megtérnek  erre a földre, mert irgalmas és kegyelmes az Úr, a ti Istentek, és nem fordítja el orczáját tõletek, ha õ hozzá megtéréndetek.
\par 10 Mikor pedig a híradók városról-városra menének az Efraim és a Manasse földén, mind Zebulonig, nevetik és csúfolják vala õket;
\par 11 Mindazáltal némelyek az Áser, Manasse és a Zebulon nemzetségébõl megalázák magokat, és eljövének Jeruzsálembe.
\par 12 Júdában is ezt cselekedé az Istennek keze, adván beléjök egy akaratot, hogy engednének a király parancsolatjának és a fejedelmekének az Úr beszéde szerint.
\par 13 Sok nép gyûle azért Jeruzsálembe, hogy a kovásztalan kenyerek ünnepét megtartsák a második hónapban; igen nagy gyülekezet vala.
\par 14 És felkelének, és elronták az oltárokat, a melyek Jeruzsálemben valának, és a füstölõ oltárokat is mind elronták, és a Kidron patakába hányták.
\par 15 És megölék a páskhabárányt a második hónap tizennegyedik napján; s a papok és a Léviták megszégyenlék és megszentelék magokat, és égõáldozatokat vittek az Úr házába;
\par 16 És kiki álla az õ helyére az õ szokások szerint és Mózesnek az Isten emberének törvénye szerint, és a papok hintik vala a vért a Léviták kezébõl.
\par 17 Minthogy pedig igen sokan valának a gyülekezetben, a kik magokat meg nem szentelték vala, ezért a Léviták a páskhabárány megölésével foglalkozának azokért, a kik tisztátalanok voltak, hogy megszenteljék az Úrnak;
\par 18 Mert a népnek nagy része, sokan az Efraim, Manasse, Izsakhár és Zebulon nemzetségébõl, nem szentelték meg magokat, mindazáltal megevék a páskhabárányt, nem úgy, a mint megiratott; de könyörge Ezékiás érettök, mondván: A kegyes Úr tisztítsa meg azt;
\par 19 Mindenkit, a ki szívét elkészítette, hogy keresse az Istent, az Urat, az õ atyái Istenét, ha nem a szentségnek tisztasága szerint is.
\par 20 És meghallgatá az Úr Ezékiást, és a népnek megkegyelmeze.
\par 21 Megtartották tehát az Izráel fiai, a kik Jeruzsálemben találtatának, a kovásztalan kenyerek ünnepét hét napon át nagy örömmel, és dícsérik vala az Urat minden nap a Léviták és a papok, az Úr erejét éneklõ szerszámokkal.
\par 22 És kegyesen beszéle Ezékiás minden Léávitával, a kik értelmesek és jóindulattal valának az Úr iránt. És ünnepi lakomát tartottak hét napon át, áldozván hálaadó-áldozatokkal, és dícsérvén az Urat, atyáik Istenét.
\par 23 Tanácsot tarta pedig az egész gyülekezet, hogy még hét napig szentelne ünnepet; és így még hét napot töltöttek el vígasságban;
\par 24 Mert Ezékiás, a Júda királya ada a gyülekezetnek ezer tulkot és hétezer juhot, a fejedelmek is adának a gyülekezetnek ezer tulkot és tízezer juhot; és megszentelék magokat a papok elegen.
\par 25 Nagy örömben vala azért az egész Júda gyülekezete, a papok és a Léviták és azoknak egész gyülekezete, a kik Izráelbõl oda mentek vala; mind az idegenek, a kik az Izráel földjérõl jöttek vala, mind a Júdában lakozók.
\par 26 És volt nagy vígasság Jeruzsálemben; mert Salamonnak, az Izráel királyának, Dávid fiának idejétõl fogva nem volt ehez hasonló ünnep Jeruzsálemben.
\par 27 Ezek után felállának a papok és a Léviták, és megáldák a gyülekezetet; és meghallgatásra talált az õ szavok, és felhata az õ könyörgésök mennybe az Isten szentséges lakhelyébe.

\chapter{31}

\par 1 A mint mindezeknek vége lõn: kiméne az egész Izráel, a kik Júdának városaiban találtatának, s a bálványokat széttörték, az Aserákat kivagdalák, s leronták a magaslatokat és oltárokat egész Júdában, Benjáminban, Efraimban és Manasseban, mígnem befejezték; azután visszatérének az Izráel fiai mind, kiki az õ örökségébe és városaiba.
\par 2 És helyreállítá Ezékiás a papok és Léviták osztályait csoportjaik szerint, kit-kit az õ szolgálata szerint, a papokat és a Lévitákat az égõáldozatokra és a hálaadó-áldozatokra, a szolgálatra, hálaadásra és dícséretre az Úr táborinak kapuiban.
\par 3 És rendelé a király az õ jövedelmének egy részét az égõáldozatok számára, a reggeli és estvéli égõáldozatok, a szombatnapok, az újhold és az ünnepek égõáldozatai számára, a mint megiratott az Úr törvényében.
\par 4 Meghagyá a népnek is, Jeruzsálem lakosinak, hogy megadják a papok és a Léviták járandóságát, hogy az Úr törvényéhez annál inkább ragaszkodjanak.
\par 5 A mint ez intézkedés híre elterjedt: az Izráel fiai a búzának, a bornak és olajnak, a méznek és minden mezei termésnek zsengéjét bõséggel megadák; és mindenbõl a tizedet bõséggel meghozák.
\par 6 És az Izráel és Júda fiai, a kik lakoznak vala a Júda városaiban, õk is elhozák a barmokból és juhokból a tizedet, és annak tizedét, a mi az Úrnak, az õ Istenöknek szenteltetett, és rakásonként felhalmozák.
\par 7 A harmadik hónapban kezdék a rakásokat rakni, és a hetedik hónapban végezék el.
\par 8 Oda menvén pedig Ezékiás és a fejedelmek, és látván a rakásokat: áldák az Urat és az õ népét, Izráelt.
\par 9 Megkérdé pedig Ezékiás a papokat és a Lévitákat a rakások felõl.
\par 10 Kinek felelvén Azáriás a fõpap, a Sádók nemzetségébõl való, monda: Mióta az Úr házába kezdék az ajándékokat hozni: eleget ettünk és sok meg is maradt belõle, mert az Úr megáldotta az õ népét; és ez a rakás, a mi megmaradt.
\par 11 És monda Ezékiás, hogy az Úr házában csináljanak tárházakat, és megcsinálák.
\par 12 Behordák azért az ajándékokat, a tizedeket, és a mi megszenteltetett, nagy hûséggel, és ezeknek fõgondviselõje Konánia Lévita vala, a második pedig ennek atyjafia, Simei.
\par 13 Jéhiel pedig és Azáriás, Náhát, Asáel, Jérimot, Józabád, Eliel, Ismákia, Máhát és Benája, gondviselõk valának, Konániának és atyjafiának Simeinek keze alatt, Ezékiás királynak és Azáriásnak az Isten háza elõljárójának parancsolatából.
\par 14 Kóré Lévita pedig, a Jemna fia, a ki ajtónálló vala napkelet felõl, az Isten számára tett önkéntes adományok gondviselõje volt, hogy kiadja az Úrnak és a szentek szentjének áldozatát.
\par 15 Keze alatt valának: Eden, Minjámin, Jésua, Semája, Amária és Sekánia a papok városaiban, hogy hûségesen osztogassák a csapatok szerint az õ atyjokfiainak, úgy a nagynak, mint a kicsinynek,
\par 16 (Az õ nemzetségök férfiain kivül, a három esztendõs fiútól fogva feljebb) mindenkinek, a kinek bejárása vala az Úr házába a maga napján, a naponként való szolgálatra, rendtartásuk és csapatjaik szerint;
\par 17 A papok nemzetségének családjaik szerint; a Lévitáknak is, a kik húsz esztendõsök és feljebb valók volnának, az õ rendtartások és csapatjaik szerint.
\par 18 És azok családjának minden kisdeddel, feleségeikkel, fiaikkal, leányaikkal, s az egész gyülekezetnek; mert az õ hitök szerint szentelték vala magokat a szentségre;
\par 19 Az Áron fiainak is, a papoknak, az õ városaikhoz tartozó környékén, minden városban valának névszerint megnevezett emberek, hogy kiadják a részét minden férfinak a papok közül, és minden nemzetségnek a Léviták között.
\par 20 És így cselekedék Ezékiás egész Júdában. A mi jó, igaz és helyes vala az Úr elõtt, az õ Istene elõtt, azt mûvelé.
\par 21 És minden munkában, a melyet elkezdett az Isten házának szolgálatában, a törvényben és parancsolatban, keresvén az õ Istenét, teljes szívvel jár vala el, és ezért szerencsés vala.

\chapter{32}

\par 1 Ezen dolgok és igazságos cselekedetek után eljöve Sénakhérib, az Assiriabeli király, és Júdába menvén, megszállá a megerõsített városokat, azt mondván, hogy elfoglalja azokat magának.
\par 2 Mikor tehát Ezékiás látta, hogy Sénakhérib eljöve, és Jeruzsálemet meg akarná szállani:
\par 3 Tanácsot tarta vezéreivel és vitézeivel, hogy a városon kivül való forrásokat betöltsék; és azok segítségére lõnek néki;
\par 4 Mert összegyûlvén a sokaság, bedugának minden forrást és az ország közepén folyó patakot, mondván: Miért találjanak az assiriai királyok elegendõ vizet, ha eljõnek?!
\par 5 És felbátorodván, megépíté a város leromlott kerítését, felemelvén a tornyokig, és kivül másik kõfalat is emelt, s Millót a Dávid városában megerõsíté; ennekfelette szerze sok fegyvert és paizst.
\par 6 És a nép fölé seregvezéreket tett, és maga köré gyûjtvén õket a város kapujának utczájára, szóla az õ szívök szerint ekképen:
\par 7 Erõsek legyetek és bátrak, semmit se féljetek, meg se rettenjetek az assiriai királytól és a vele való egész sokaságtól, mert velünk többen vannak, hogynem õ vele.
\par 8 Õ vele testi erõ van, velünk pedig az Úr a mi Istenünk, hogy megsegéljen minket és érettünk hadakozzék. És megbátorodék a nép, ezt hallván Ezékiástól, a Júda királyától.
\par 9 Ezek után elküldé szolgáit Sénakhérib, az assiriai király Jeruzsálembe (õ pedig Lákis mellett volt egész seregével) Ezékiáshoz a Júda királyához, és az egész Júdához, mely Jeruzsálemben vala, mondván:
\par 10 Ezt mondja Sénakhérib, az assiriai király: Kiben bíztok, hogy Jeruzsálemben maradtok a megszállás idején?
\par 11 Avagy nem Ezékiás áltatott-é el titeket, hogy éhséggel és szomjúsággal ölne meg titeket, mondván: Az Úr, a mi Istenünk megszabadít minket az Assiriabeli király kezébõl!
\par 12 Vagy nem Ezékiás pusztította-é el az õ magaslatait és oltárait, mikor így szólott Júdához és Jeruzsálemhez, mondván: Csak egy oltár elõtt imádkozzatok, és csak azon tömjénezzetek?!
\par 13 Avagy nem tudjátok-é, mit mûveltem én és az én atyáim e föld minden népeivel? Vajjon e föld nemzetségeinek istenei megszabadíthatták-é az én kezembõl az õ földöket?
\par 14 És kicsoda e nemzetségek istenei közül az, a melyeket az én atyáim elvesztettek, a ki az én kezembõl az õ népét megszabadíthatta volna, hogy a ti Istenetek is az én kezembõl titeket megszabadíthatna?
\par 15 Most azért Ezékiás titeket el ne ámítson és meg ne csaljon ily módon; ne higyjetek néki, mert ha egy népnek és országnak istene sem szabadíthatta meg az õ népét kezembõl és az én atyáim kezébõl: mennyivel kevésbbé szabadíthat meg titeket a ti Istenetek az én kezembõl!
\par 16 Sõt ezenkivül az õ szolgái még sokat szólának az Úr Isten ellen, és az õ szolgája Ezékiás ellen.
\par 17 Levelet is íra, az Urat, Izráel Istenét káromlással illetvén, és szólván ellene ilyen módon: A mint e földön lakozó népek istenei meg nem szabadíthatták az õ népöket az én kezembõl: ekképen az Ezékiás Istene sem szabadíthatja meg az õ népét kezembõl.
\par 18 És kiáltnak nagy felszóval zsidó nyelven Jeruzsálem népe ellen, mely a kerítésen vala, hogy õket megrettentenék és megháborítanák, abban a reményben, hogy így a várost elfoglalhatják.
\par 19 És úgy szólának a Jeruzsálem Istenérõl, mint a föld népeinek istenei felõl, melyek emberi kézzel csináltattak.
\par 20 Akkor Ezékiás király könyörge, és õ vele Ésaiás próféta az Ámós fia e káromlásért, és felkiáltának az égre.
\par 21 És elbocsátá az Úr az õ angyalát, a ki megöle minden erõs vitézt, elõljárót és vezért az assiriai király táborában, és nagy szégyennel megtére az õ földébe. Bemenvén pedig az õ istenének házába, ott az õ saját fiai  fegyverrel ölék meg õt.
\par 22 Megszabadítá azért az Úr Ezékiást és a Jeruzsálem népét Sénakhéribtõl az assiriai királytól, és minden másoktól, és védelmezé õket mindenfelõl.
\par 23 És sokan ajándékokat hoznak vala Jeruzsálembe az Úrnak, Ezékiásnak is a Júda királyának drágaságokat, és õ felmagasztaltatott minden pogányok szemei elõtt azután.
\par 24 Az idõben Ezékiás halálos betegségbe esék; de könyörgött az Úrhoz, a ki szóla hozzá és csudajelt adott néki.
\par 25 De nem cselekedék Ezékiás az õ hozzá való jótétemény szerint, mert magában felfuvalkodék, azért Istennek haragja lõn rajta, Júdán és Jeruzsálemen.
\par 26 Azonban megalázta magát Ezékiás az õ felfuvalkodottságában, Jeruzsálem lakosaival egybe; ezért nem szálla többé reájok az Úrnak haragja Ezékiás életében.
\par 27 És igen nagy gazdagsága és dicsõsége vala Ezékiásnak. És csináltatott magának kincsesházat az ezüst, arany, drágakövek és drága fûszerszámok, paizsok és mindenféle drága szerszámok számára;
\par 28 És tárházakat jövedelmének a gabonának, bornak, olajnak számára, és mindenféle barom számára istállókat, a nyájaknak pedig aklokat.
\par 29 Városokat is építe magának, és szerze igen sok juhot és barmot, mert az Isten nagy gazdagságot adott néki.
\par 30 És Ezékiás volt az, a ki betömé a Gihon vizeinek felsõ forrását, és Dávid városának napnyugat felõl való részén vezeté lefelé. És minden dolgában igen szerencsés vala Ezékiás;
\par 31 De mivel a Babilóniabeli fejedelmek követeivel megbarátkozék, a kik õ hozzá küldettek, hogy megtudakoznák a csudajelt, mely a földön lõn; elhagyá õt az Isten, hogy megkisértené õt és meglátná, mi volna az õ szívében.
\par 32 Ezékiásnak pedig többi dolgai és jótéteményei ímé meg vannak írva az Ésaiás prófétának, az Ámós fiának látásában, és a Júda és Izráel királyainak könyvében.
\par 33 Meghala pedig Ezékiás az õ atyáival, és eltemeték õt a Dávid fiainak sírjaihoz vivõ feljárón, és mind az egész Júda és Jeruzsálem nagy tisztességet tettek néki az õ halálának idején. És uralkodék Manasse, az õ fia helyette.

\chapter{33}

\par 1 Tizenkét esztendõs vala Manasse, mikor uralkodni kezdett, és ötvenöt esztendeig uralkodott Jeruzsálemben.
\par 2 És gonoszul cselekedék az Úr szemei elõtt a pogányok útálatosságai szerint, a kiket az Úr az Izráel fiai elõl kiûzött vala.
\par 3 Mert a magaslatokat ismét megépíté, a melyeket Ezékiás az õ atyja azelõtt elrontott vala, és oltárokat emele Baálnak, Aserákat  is plántála, és tisztelé az ég minden seregeit, és szolgála azoknak.
\par 4 Sõt az Úr házában is építe oltárokat, a melyrõl az Úr azt mondotta volt: Jeruzsálemben  lészen az én nevem örökké.
\par 5 És építe oltárokat az ég minden seregeinek, az Úr házának mindkét pitvariban.
\par 6 És fiait átvitte a tûzön a Hinnom fiának völgyében; és az idõnek forgására ügyelt, jövendõmondásokat, varázslásokat  és szemfényvesztéseket ûzött, ördöngösöket és jövendõmondókat szerzett, és sok gonoszságot cselekedett az Úr szemei elõtt, hogy õt haragra indítaná.
\par 7 A faragott bálványt, a melyet csináltatott vala, az Úr házában állítá fel, a melyrõl azt mondá az Isten Dávidnak, és az õ fiának, Salamonnak: E házban és Jeruzsálemben, a melyet választottam az Izráel minden nemzetségei közül, helyheztetem az én nevemet mindörökké;
\par 8 És nem ûzöm ki az Izráelt e földrõl, melyet adtam volt a ti atyáitoknak; de csak úgy, ha õk is mind megtartándják,  a melyeket nékik Mózes által parancsoltam, minden törvényt, rendeléseket és ítéleteket;
\par 9 De Manasse elcsábítá Júdát és Jeruzsálem lakóit, hogy még gonoszabbul cselekedjenek, mint a pogányok, a kiket az Úr kigyomlált volt az Izráel fiai elõl.
\par 10 És noha megszólította az Úr Manassét és az õ népét; de nem figyelmezének reá.
\par 11 Reájok hozá azért az Úr az Assiriabeli király seregének vezéreit, a kik Manassét megfogták és vasba vervén megkötözék õt két lánczczal, és Babilóniába vivék.
\par 12 Mikor pedig nagy nyomorúságban volna, fohászkodék az Úrhoz az õ Istenéhez, és teljesen megalázta magát az õ atyáinak Istene elõtt.
\par 13 És könyörögvén hozzá megkegyelmeze néki, és meghallgatván könyörgését, visszahozá õt Jeruzsálembe, az õ országába. Akkor ismeré meg Manasse, hogy az Úr az igaz Isten.
\par 14 Ezek után a Dávid városának külsõ kõfalát felépíté Gihontól napnyugat felé a völgyben, a halkapu bemeneteléig; Ofelt is körülvéteté magas kerítéssel, és Júdának minden megerõsített városaiba seregvezéreket helyeze.
\par 15 És eltávolítá az idegen isteneket és a bálványt az Úr házából, és minden oltárt, a melyet az Úr házának hegyén és Jeruzsálemben emeltetett, kihányatá azokat a városon  kivül.
\par 16 És megépíté az Úr oltárát, és áldozék rajta hálaadó és dicsõítõ áldozatokkal, és megparancsolá Júdának, hogy szolgáljanak az Úrnak, Izráel Istenének.
\par 17 Mindazáltal még akkor a nép áldozik vala a magaslatokon; de csak az Úrnak, az õ Istenének.
\par 18 Manassénak pedig többi dolgai, Istenéhez való könyörgése, a próféták intése, a kik az Úrnak, Izráel Istenének nevében szólanának néki, ímé meg vannak írva az Izráel királyainak dolgai között.
\par 19 Az õ könyörgése pedig, és hogy az Isten mint kegyelmezett volt meg néki, s az õ minden bûne és vétke; és a helyek, a melyeken magaslatokat épített volt, s Aserákat és bálványokat állított fel, minekelõtte megalázta volna magát: ímé meg vannak írva a Hózai beszédei között.
\par 20 És meghala Manasse az õ atyáival, és eltemeték õt az õ házában; és uralkodék Amon, az õ fia helyette.
\par 21 Huszonkét esztendõs vala Amon, mikor uralkodni kezdett, és két esztendeig uralkodék Jeruzsálemben.
\par 22 És gonoszul cselekedék az Úr szemei elõtt, miként az õ atyja Manasse cselekedett volt, mert áldozék Amon mindama bálványoknak, a melyeket az õ atyja Manasse csináltatott, és azoknak szolgál vala.
\par 23 Meg sem alázá magát az Úr elõtt, mint az õ atyja Manasse megalázta magát; hanem még sokasítá Amon a bûnt.
\par 24 Pártot ütének pedig az õ szolgái õ ellene, és õt saját házában megölék.
\par 25 A föld népe pedig levágá mindazokat, a kik Amon király ellen pártot ütének, és királylyá tevé a föld népe Jósiást az õ fiát helyette.

\chapter{34}

\par 1 Nyolcz esztendõs vala Jósiás, mikor uralkodni kezdett volt, és uralkodott harminczegy esztendeig Jeruzsálemben.
\par 2 És jó dolgot cselekedék az Úr elõtt, és jára az õ atyjának Dávidnak útjain, és nem hajolt el sem jobbra, sem balra;
\par 3 Mert az õ királyságának nyolczadik esztendejében, mikor még gyermek volna, kezdé keresni az õ atyjának Dávidnak Istenét; tizenkettedik esztendejében pedig Júdát és Jeruzsálemet kezdé megtisztítani a magaslatoktól, Aseráktól, bálványoktól és öntött képektõl.
\par 4 Leronták õ elõtte a Baálok oltárait és az azokon levõ szobrokat lehányatá; az Aserákot is a bálványokkal és öntött képekkel egyetembe szétrombolá és apróra töreté, és elhinteté azoknak  temetõhelyén, a kik azoknak áldoztak vala.
\par 5 A papok csontjait megégeté azoknak oltárain, és megtisztítá Júdát és Jeruzsálemet.
\par 6 Így cselekedék Manassénak, Efraimnak és Simeonnak városaiban is, mind Nafthaliig, azoknak pusztáiban köröskörül.
\par 7 Lehányatá azért az oltárokat; az Aserákat és a bálványokat összetördelé mind porrá, és minden naposzlopot elpusztított Izráel egész földén; azután megtére Jeruzsálembe.
\par 8 Királyságának tizennyolczadik esztendejében pedig, minekutána a földet és a házat megtisztítá, elküldé Sáfánt az Asália fiát, és Maaséját a város elõljáróját, és Joát a Joákház fiát, az emlékírót, hogy kijavíttatnák az Úrnak, az õ Istenének házát.
\par 9 És menének Hilkiás fõpaphoz, és átadták néki az Isten házába begyült pénzt, a melyet a Léviták, az ajtónállók gyûjtöttek Manassétól, Efraimtól és az egész Izráel maradékaitól, és egész Júdától és Benjámintól, és Jeruzsálem lakóitól.
\par 10 És adák a munkavezetõknek, a kik felvigyáztak az Úr házában, hogy adják azt a munkásoknak, a kik dolgoztak az Úr házában, hogy kijavítsák és helyreállítsák a házat.
\par 11 És adának pénzt az ácsoknak és a kõmûveseknek, hogy vegyenek faragott köveket, és fákat a kapcsolásokra, és a házak gerendázására, a melyeket elrontottak a Júda királyai.
\par 12 És az emberek hûségesen végzik vala a munkát; a kikre felügyelének Jáhát és Obádia Léviták, a Mérári fiai közül valók; és Zakariás, Mésullám, a Kéhátiták fiai közül valók, a kik szorgalmazzák vala õket. E Léviták pedig mindnyájan mesterek valának az éneklõszerszámokban.
\par 13 A teherhordókat pedig (mert voltak felügyelõk a munkások felett mindenféle dologban) szorgalmazák a Léviták közül való íródeákok, igazgatók és ajtónállók.
\par 14 Mikor pedig kihozák azok a pénzt, a mely az Úr házában gyûlt össze: megtalálá Hilkia pap az Úr törvénykönyvét, a melyet Mózes által adott volt.
\par 15 Szóla pedig Hilkia, és monda Sáfánnak, az íródeáknak: A törvénykönyvet megtalálám az Úr házában. És adá Hilkia a könyvet Sáfánnak.
\par 16 És vivé Sáfán a könyvet a királyhoz, és elbeszélé néki a dolgot, mondván: Valamit bíztál a te szolgáidra, abban híven eljárnak;
\par 17 Mert az Úr házában talált pénzt összeszedvén, adák a felügyelõknek és munkásoknak kezébe.
\par 18 Továbbá jelenté Sáfán íródeák a királynak, mondván: Hilkia pap nékem egy könyvet ada; és olvasott abból Sáfán a király elõtt.
\par 19 Mikor pedig a király hallotta a törvény beszédit, ruháit megszaggatá.
\par 20 És parancsola a király Hilkia papnak és Ahikámnak a Sáfán fiának, Abdonnak a Mika fiának, és az íródeáknak Sáfánnak, és Asájának a király szolgájának, mondván:
\par 21 Menjetek el, és keressétek meg az Urat én érettem, s az Izráel és Júda maradékaiért, a könyv beszédei felõl, a mely megtaláltatott, mert nagy az Úr haragja, a mely mi reánk szállott azért, hogy a mi atyáink nem tartották meg az Úrnak beszédét, hogy mind a szerint cselekedtek volna, a mint e könyvben megíratott.
\par 22 Elméne azért Hilkia és a király embere Húlda prófétaasszonyhoz, a Sallum feleségéhez, aki Thókéhát fia vala, a ki Hasrának, a ruhák gondviselõjének fia vala (õ pedig Jeruzsálemben, a második utczában lakik vala) és a szerint szólának néki.
\par 23 Ki monda nékik: Így szól az Úr, Izráel Istene: Mondjátok meg a férfinak, a ki titeket hozzám küldött;
\par 24 Ezt mondja az Úr: Ímé én veszedelmet hozok e helyre és ennek lakosira, mindazokat az átkokat, a melyek meg vannak írva a könyvben, a melyet felolvastak a Júda királya elõtt;
\par 25 Mivel elhagytak engem, és idegen isteneknek tömjéneztek, hogy engem haragra gerjeszszenek az õ kezeiknek minden cselekedetei által. Felgerjedt az én haragom e hely ellen, és el sem oltatik.
\par 26 A Júda királyának pedig, a ki titeket küldött, hogy az Urat megkérdjétek, így szóljatok: Így szól az Úr, Izráel Istene: A mi a beszédeket illeti, a melyeket hallottál:
\par 27 Mivel a te szíved meglágyult, és magadat megaláztad az Isten elõtt, mikor hallád az õ beszédeit e hely ellen és az õ lakosai ellen; mivel magadat megaláztad elõttem, és ruháidat megszaggattad és sírtál elõttem: én is meghallgattalak, ezt mondja az Úr.
\par 28 Ímé én téged a te atyáid közé takarítlak, és tétetel a te sírodba békességben, és nem látják a te  szemeid azt a veszedelmet, a melyet én hozok e helyre és ennek lakóira. És e szerint beszélték el ezt a királynak.
\par 29 Akkor a király elkülde, és összegyûjteté Júdának és Jeruzsálemnek minden véneit.
\par 30 És felméne a király az Úr házába, és õ vele Júdának minden férfiai, és Jeruzsálem lakosai, a papok, a Léviták és az egész nép kicsinytõl fogva nagyig, és elolvasá fülök hallására a szövetség könyvének minden beszédeit, a melyet találtak vala az Úr házában.
\par 31 Azután felállván a király az õ helyén, fogadást tõn az Úr elõtt, hogy õ az Urat követendi, és hogy parancsolatait, bizonyságtételeit és rendeléseit teljes szívébõl és teljes lelkébõl megõrzendi, cselekedvén a szövetség beszédeit, a melyek megirattak abban a könyvben.
\par 32 És felfogadtatá e dolgot mindazokkal, a kik Jeruzsálembõl és Benjáminból valók. És Jeruzsálem lakói az Istennek, az õ atyáik Istenének szövetsége szerint cselekedének.
\par 33 Akkor elpusztított Jósiás minden útálatos  bálványt Izráel fiainak egész földjérõl, és kényszeríte mindenkit, valaki találtatik vala Izráelben, hogy szolgáljon az Úrnak, az õ Istenöknek; és az õ egész életében nem szakadának el az Úrtól, atyáik Istenétõl.

\chapter{35}

\par 1 És megtartá Jósiás Jeruzsálemben a húsvétünnepet az Úrnak; és megölék a páskhabárányt az elsõ hónap tizennegyedik napján.
\par 2 És állítá a papokat az õrállásukra, és buzdítá õket az Úr házának szolgálatára.
\par 3 És monda a Lévitáknak, a kik az egész Izráelt oktatják, és magokat az Úrnak szentelék: Helyheztessétek a szent ládát a házba, a melyet készített volt Salamon a Dávid fia, az Izráel királya; nem kell most vállatokon hordoznotok; hanem szolgáljatok az Úrnak, a ti Istenteknek, és az õ népének az Izráelnek.
\par 4 És készítsétek el magatokat a ti családjaitok és csoportjaitok szerint, a mint Dávid az Izráel királya elrendelte volt, és megírta az õ fia Salamon.
\par 5 És álljatok a szent helyen, a nép közül való testvéreitek családjainak csoportjai és a Léviták családjának egy-egy csoportja szerint.
\par 6 Azután öljétek meg a páskhabárányt, és szenteljétek meg magatokat, és készítsétek el azt a ti atyátokfiainak, hogy az Úrnak Mózes által tett beszéde szerint cselekedjetek.
\par 7 És ada Jósiás a nép fiai számára juhokat, bárányokat és gödölyéket, mindezeket a húsvéti áldozatokra; mindenkinek, valaki ott találtatik vala, szám szerint harminczezeret, és háromezer tulkot; ezek mind a királyéból valának.
\par 8 Az õ fejedelmei is szabad akaratjokból a községnek, a papoknak és a Lévitáknak adakozának; Hilkia, Zakariás és Jéhiel, az Isten házának fejedelmei, adának a papoknak a húsvét áldozatira kétezerhatszáz juhot és háromszáz tulkot.
\par 9 Konánia pedig és Semája és Nétanéel az õ atyjafiai, Hasábia, Jéhiel és Józabád, a Léviták fejedelmei, adának a Lévitáknak a húsvét áldozatira ötezer juhot és ötszáz tulkot.
\par 10 Készen levén a szolgálat, a papok helyeikre állának, a Léviták is csapatjaik szerint, a mint a király parancsolta.
\par 11 Megölék azért a páskhabárányt, és a papok hintik vala kezökbõl a vért, a Léviták pedig a bárányok bõrét húzzák le.
\par 12 És külön választák az egészen égõáldozatra valókat, hogy adnák azokat a nép közül való családok csoportjainak, hogy áldoznának az Úrnak, a mint Mózes könyvében megiratott; azonképen a tulkokból is elválaszták.
\par 13 És megsüték a páskhabárányt szokás szerint a tûznél; a megszenteltetett állatokat pedig megfõzék fazekakban, vasfazekakban és üstökben: és nagy hamarsággal adák az egész községnek.
\par 14 Ezek után magoknak és a papoknak is elkészíték a páskhabárányt, mert a papok, az Áron fiai az égõáldozatoknak és a kövérségeknek megáldozásával el valának foglalva késõ éjszakáig, azért a Léviták készíték el mind magoknak, mind a papoknak, az Áron fiainak.
\par 15 Az éneklõk is, az Asáf fiai, szolgálatukban valának, Dávidnak, Asáfnak, Hémánnak és Jédutunnak, a király prófétájának parancsolatja szerint; az ajtónállók is mindnyájan az ajtóknál valának; nem távozhatának szolgálatukból, hanem az õ atyjokfiai, a Léviták készíték vala el nékik.
\par 16 És elkészüle az Úrnak minden szolgálata azon a napon, hogy megtartanák a páskhát, egészen égõáldozatokkal áldozván az Úr oltárán, a Jósiás király parancsolatja szerint.
\par 17 Megtartották tehát az Izráel fiai, a kik jelen lehetének, a páskhát abban az idõben, és a kovász nélkül való kenyerek ünnepét hét napon át.
\par 18 Ehez hasonló páskhát nem tartottak Izráelben a Sámuel próféta idejétõl fogva; Izráel királyai közül is senki sem tartott olyan páskhát, a milyet Jósiás tartott és a papok, a Léviták, egész Júda, és a kik Izráelbõl jelen voltak, és Jeruzsálem lakosai.
\par 19 Jósiás királyságának tizennyolczadik évében tartatott ez a páskha.
\par 20 Mindezek után, hogy az Isten házát Jósiás helyreállítá, feljöve Nékó, Égyiptom királya, hogy az Eufrátes mellett való Kárkemis városát elfoglalná; és Jósiás ellene ment.
\par 21 És noha követeket külde õ hozzá Nékó, ezt mondván: Mi közöm te hozzád nékem, Júda királya? Mert én most nem ellened megyek, hanem az én országom ellensége ellen, és Isten parancsolta, hogy siessek; ne tusakodjál az Isten ellen, a ki én velem van, hogy el ne veszessen téged;
\par 22 Mindazáltal Jósiás nem tére ki õ elõle, hanem, hogy megütköznék vele, öltözetit megváltoztatá, és nem hallgatott Nékó beszédeire, a melyek az Isten szájából származtak vala. Elméne azért, hogy megütközzék vele Megiddó mezején.
\par 23 Akkor a kézívesek nyilakat lövének Jósiás királyra, és monda a király az õ szolgáinak: Vigyetek ki engem innét; mert nagyon megsebesültem.
\par 24 Levévén azért õt az õ szolgái a szekérbõl, másik szekerére helyezék, és vivék Jeruzsálembe: És meghala, és eltemetteték az õ atyáinak sírjába; s egész Júda és Jeruzsálem siránkozék Jósiás felett.
\par 25 Jerémiás is siratá Jósiást, és siralmas énekekkel siratják vala õt az éneklõ férfiak és asszonyok mindnyájan mind e mai napig, a melyek szokásossá lettek Izráelben, s ímé azok meg vannak írva a Jerémiás siralmaiban.
\par 26 Jósiásnak pedig többi dolgai, és a szerint való jó tettei, a mint az Úrnak törvényében meg van írva;
\par 27 És az õ elsõ és utolsó dolgai, ímé meg vannak írva az Izráel és Júda királyainak könyvében.

\chapter{36}

\par 1 És elõhozá a föld népe Joákházt a Jósiás fiát, és királylyá tette õt atyja helyett Jeruzsálemben.
\par 2 Huszonhárom esztendõs vala Joákház, mikor uralkodni kezde, és három hónapig uralkodék Jeruzsálemben.
\par 3 És letevé õt Égyiptom királya Jeruzsálemben, és a földre adót vetett, száz talentom ezüstöt és egy talentom aranyat.
\par 4 És Égyiptom királya Eliákimot, az õ testvérét tette királylyá Júda és Jeruzsálem felett, megváltoztatván nevét Joákimra; Joákházt pedig az õ testvérét fogá és elvivé Nékó Égyiptomba.
\par 5 Huszonöt esztendõs vala Joákim, mikor uralkodni kezdett, és tizenegy esztendeig uralkodék Jeruzsálemben; de gonoszul cselekedék az Úr elõtt, az õ Istene elõtt.
\par 6 És feljöve ellene Nabukodonozor a babilóniai király, és kettõs békót vete lábaira, hogy Babilóniába vinné õt.
\par 7 Az Úr házában való edények egy részét is elvivé Nabukodonozor Babilóniába, és helyezteté azokat az õ templomába, Babilóniában.
\par 8 Joákimnak pedig többi dolgai és az õ útálatosságai, a melyeket cselekedett, s a melyek találtattak õ benne, ímé meg vannak írva az Izráel és Júda királyainak könyvében; és uralkodék helyette Joákin, az õ fia.
\par 9 Nyolcz esztendõs korában kezdett uralkodni Joákin, és három hónapig és tíz napig uralkodék Jeruzsálemben; de õ is gonoszul cselekedék az Úr elõtt.
\par 10 Az esztendõ fordultával pedig elkülde Nabukodonozor király, és elviteté õt Babilóniába, az Úr házának drága edényeivel együtt, és királylyá tevé az õ testvérét Sédékiást Júda és Jeruzsálem felett.
\par 11 Huszonegy esztendõs korában kezdett uralkodni Sédékiás, és uralkodék tizenegy esztendeig Jeruzsálemben.
\par 12 És gonoszul cselekedék az Úr elõtt, az õ Istene elõtt, és nem alázta meg magát Jeremiás próféta elõtt, a ki az Úr képében szól vala néki.
\par 13 Sõt még Nabukodonozor király ellen is pártot ütött, a ki õt az Isten nevére megesküdtette vala, s makacscsá és önfejûvé lett, a helyett, hogy megtért volna az Úrhoz, Izráel Istenéhez.
\par 14 Sõt még a papok fejedelmei és a nép is, mindnyájan szaporították a bûnt a pogányok minden undokságai szerint, és megfertõztették az Úr házát, a melyet megszentelt vala Jeruzsálemben.
\par 15 És az Úr, az õ atyáiknak Istene elküldé hozzájok követeit jó idején, mert kedvez vala az õ népének és az õ lakhelyének.
\par 16 De õk az Isten követeit kigúnyolták, az õ beszédeit megvetették, és prófétáival gúnyt ûztek; míglen az Úrnak haragja felgerjede az õ népe ellen, s többé nem vala segítség.
\par 17 És reájok hozá a Káldeusok királyát, a ki fegyverrel ölé meg ifjaikat az õ szent hajlékukban, s nem kedveze sem az ifjaknak és szûzeknek, sem a vén és elaggott embereknek, mindnyájokat kezébe adá.
\par 18 És az Isten házának mindenféle edényeit, nagyokat, kicsinyeket, és az Úr házának kincseit, s a királynak és az õ vezéreinek kincseit, mindezeket Babilóniába viteté.
\par 19 Az Isten házát meggyújták, Jeruzsálem kõfalait lerontották, palotáit mind elégeték tûzzel, és minden drágaságait elpusztították.
\par 20 És a kik a fegyver elõl megmenekültek, azokat elhurczolta Babilóniába, és néki és fiainak szolgáivá lettek mindaddig, míg a persiai birodalom fel nem támadott;
\par 21 Hogy beteljesedjék az Úrnak Jeremiás szája által mondott beszéde, míg lerójja a föld az õ szombatjait, mert az elpusztulás egész ideje alatt nyugovék, hogy betelnének a  hetven esztendõk.
\par 22 És Czírus persa király elsõ esztendejében, hogy beteljesednék az Úrnak Jeremiás szája által mondott beszéde, az Úr felindítá Czírus persa király lelkét, és õ kihirdetteté az õ egész birodalmában, élõszóval és írásban is, mondván:
\par 23 Így szól Czírus, a persa király: Az Úr, a mennynek Istene e föld minden országait nékem adta, és Õ parancsolta meg nékem, hogy építsek néki házat Jeruzsálemben, a mely Júdában van; valaki azért ti köztetek az õ népe közül való, legyen vele az Úr, az õ Istene, és menjen fel.


\end{document}