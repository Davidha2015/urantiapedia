\begin{document}

\title{Ezra}


\chapter{1}

\par 1 Czírus persa király elsõ esztendejében, hogy beteljesednék az Úrnak Jeremiás szája által mondott beszéde, felindítá az Úr Czírus persa király lelkét, és õ kihirdetteté az õ egész birodalmában, élõszóval és írásban is, mondván:
\par 2 Így szól Czírus, a persa király: Az Úr, a mennynek Istene e föld minden országait nékem adta, és Õ parancsolta meg nékem, hogy építsek néki házat Jeruzsálemben, mely Júdában van;
\par 3 Valaki azért ti köztetek az õ népe közül való, legyen vele az õ Istene, és menjen fel Jeruzsálembe, mely Júdában van, és építse az Úrnak, Izráel Istenének házát, õ az Isten, ki Jeruzsálemben lakozik.
\par 4 És mindenkit, a ki még megmaradt, minden helyrõl, a hol lakik, segítsék azon helynek férfiai ezüsttel, aranynyal, jószággal és barommal, azzal együtt, a mit önkénytesen adnak az Isten házának, mely Jeruzsálemben van.
\par 5 Fölkelének azért Júda és Benjámin családfõi és a papok és a Léviták, és mindnyájan, a kiknek felindítá az Isten lelköket, hogy felmenjenek az Úr házának építésére, mely Jeruzsálemben van.
\par 6 És minden körültök lakók segíték õket ezüst edényekkel, aranynyal, jószággal, barommal, drágaságokkal, mindazon kivül, a mit önkénytesen adának.
\par 7 Czírus király pedig elõhozatá az Úr házának edényeit, melyeket Nabukodonozor hozatott vala el Jeruzsálembõl, s az õ isteneinek házába helyezett vala;
\par 8 Elõhozatá ezeket Czírus, a persák királya, Mithredáthes kincstartó kezeihez, a ki is átszámolá azokat Sesbassárnak, Júda fejedelmének.
\par 9 És számok ez vala: Harmincz arany medencze, ezer ezüst medencze, huszonkilencz kés,
\par 10 Harmincz arany pohár, négyszáztíz másrendbeli ezüst pohár, és ezer más edény.
\par 11 Minden arany és ezüst edényeknek száma ötezernégyszáz. Mindezt magával vivé Sesbassár, mikor a foglyok kijövének Babilóniából Jeruzsálembe.

\chapter{2}

\par 1 Ezek pedig a tartományok fiai, kik feljöttek vala a rabság foglyai közül, a kiket Nabukodonozor, Babilónia királya, fogva vitetett Babilóniába, s most visszajövének Jeruzsálembe és Júdába, kiki az õ városába.
\par 2 Kik jövének Zorobábellel, Jésuával, Nehémiással, Serájával, Rélájával, Mordokhaival, Bilsánnal, Miszpárral, Bigvaival, Rehummal, Baanával. Izráel népe férfiainak száma ez:
\par 3 Paros fiai kétezerszázhetvenkettõ;
\par 4 Sefátja fiai háromszázhetvenkettõ;
\par 5 Árah fiai hétszázhetvenöt;
\par 6 Pahath-Moáb fiai, Jésua és Joáb fiaitól, kétezernyolczszáztizenkettõ;
\par 7 Élám fiai ezerkétszázötvennégy;
\par 8 Zattu fiai kilenczszáznegyvenöt;
\par 9 Zakkai fiai hétszázhatvan;
\par 10 Báni fiai hatszáznegyvenkettõ;
\par 11 Bébai fiai hatszázhuszonhárom;
\par 12 Azgád fiai ezerkétszázhuszonkettõ;
\par 13 Adónikám fiai hatszázhatvanhat;
\par 14 Bigvai fiai kétezerötvenhat;
\par 15 Ádin fiai négyszázötvennégy;
\par 16 Áter fiai, Ezékiástól, kilenczvennyolcz;
\par 17 Bésai fiai háromszázhuszonhárom;
\par 18 Jórá fiai száztizenkettõ;
\par 19 Hásum fiai kétszázhuszonhárom;
\par 20 Gibbár fiai kilenczvenöt;
\par 21 Bethlehem fiai százhuszonhárom;
\par 22 Netófah férfiai ötvenhat;
\par 23 Anathóth férfiai százhuszonnyolcz;
\par 24 Azmáveth fiai negyvenkettõ;
\par 25 Kirjáth-Árim, Kefira és Beéróth fiai hétszáznegyvenhárom;
\par 26 Ráma és Géba fiai hatszázhuszonegy;
\par 27 Mikmás férfiai százhuszonkettõ;
\par 28 Béthel és Ái férfiai kétszázhuszonhárom;
\par 29 Nebó fiai ötvenkettõ;
\par 30 Magbis fiai százötvenhat;
\par 31 A másik Élám fiai ezerkétszázötvennégy;
\par 32 Hárim fiai háromszázhúsz;
\par 33 Lód, Hádid és Ónó fiai hétszázhuszonöt;
\par 34 Jérikó fiai háromszáznegyvenöt;
\par 35 Szenáa fiai háromezerhatszázharmincz;
\par 36 A papok: Jedája fiai, Jésua családjából, kilenczszázhetvenhárom;
\par 37 Immér fiai ezerötvenkettõ;
\par 38 Pashur fiai ezerkétszáznegyvenhét;
\par 39 Hárim fiai ezertizenhét;
\par 40 A Léviták: Jésuának és Kadmiélnek fiai, Hodávia fiaitól, hetvennégy.
\par 41 Az énekesek: Asáf fiai százhuszonnyolcz;
\par 42 A kapunállók fiai: Sallum fiai, Ater fiai, Talmón fiai, Akkub fiai, Hatita fiai, Sóbai fiai, összesen százharminczkilencz.
\par 43 A Léviták szolgái: Siha fiai, Hasufa fiai, Tabbaóth fiai;
\par 44 Kérósz fiai, Sziaha fiai, Pádón fiai,
\par 45 Lebána fiai, Hagába fiai, Akkub fiai,
\par 46 Hágáb fiai, Salmai fiai, Hanán fiai,
\par 47 Giddél fiai, Gahar fiai, Reája fiai,
\par 48 Resin fiai, Nekóda fiai, Gazzám fiai,
\par 49 Uzza fiai, Pászéah fiai, Bészai fiai,
\par 50 Aszna fiai, Meunim fiai, Nefiszim fiai.
\par 51 Bakbuk fiai, Hakufa fiai, Harhur fiai,
\par 52 Basluth fiai, Mehida fiai, Harsa fiai,
\par 53 Barkósz fiai, Sziszera fiai, Temah fiai,
\par 54 Nesiah fiai, Hatifa fiai;
\par 55 Salamon szolgáinak fiai: Szótai fiai, Haszszófereth fiai, Peruda fiai.
\par 56 Jaalá fiai, Darkón fiai, Giddél fiai,
\par 57 Sefátia fiai, Hattil fiai, Pókhereth-Hássebaim fiai, Ámi fiai.
\par 58 Összesen a Léviták szolgái és a Salamon szolgáinak fiai háromszázkilenczvenkettõ.
\par 59 És ezek, a kik feljövének Tél-Melahból, Tél-Harsából, Kerub-Addán-Immérbõl, de nem mondhatták meg családjukat és eredetüket, hogy Izráel közül valók-é:
\par 60 Delája fiai, Tóbiás fiai, Nekóda fiai hatszázötvenkettõ.
\par 61 És a papok fiai közül: Habája fiai, Hakkós fiai, Barzillai fiai, ki a Gileádbeli Barzillai leányai közül vett magának feleséget, és ezek nevérõl nevezteték;
\par 62 Ezek keresték írásukat, tudniillik a nemzetségök könyvét, de nem találák, miért is kirekesztetének a papságból;
\par 63 És megmondá nékik a király helytartója, hogy ne egyenek a szentséges áldozatból, mígnem pap ítél az Urimmal és  Tummimmal.
\par 64 Mind az egész gyülekezet együtt negyvenkétezerháromszázhatvan.
\par 65 Szolgáikon és szolgálóikon kivül - ezek száma hétezerháromszázharminczhét - valának nékik énekes férfiaik és asszonyaik kétszázan.
\par 66 Lovaik hétszázharminczhat, öszvéreik kétszáznegyvenöt;
\par 67 Tevéik négyszázharminczöt, hatezerhétszázhúsz szamárral.
\par 68 A családfõk közül pedig, mikor megérkezének az Úr házához, mely Jeruzsálemben van, némelyek önkénytesen adakozának az Isten házára, hogy fölépítenék azt az õ helyén;
\par 69 Tehetségök szerint adának az építés költségére aranyban hatvanegyezer dárikot, s ezüstben ötezer mánét, és száz papi ruhát.
\par 70 És lakozának mind a papok, mind a Léviták, mind a nép fiai, mind az énekesek, mind a kapunállók, mind a Léviták szolgái városaikban, s így az egész Izráel a maga városaiban vala.

\chapter{3}

\par 1 Mikor pedig eljöve a hetedik hónap, és Izráel fiai az õ városaiban lakozának, felgyûle a nép egyenlõképen Jeruzsálembe.
\par 2 És fölkele Jésua, Jósádák fia, és az õ atyjafiai, a papok, és  Zorobábel, Sealtiél fia s az õ atyjafiai, és megépíték Izráel Istenének oltárát, hogy áldozzanak rajta égõáldozatokat, a mint írva van Mózesnek, az Isten emberének törvényében.
\par 3 És erõs fundamentomra állíták fel az oltárt, mert félnek vala a földnek népétõl, és áldozának rajta égõáldozatokat az Úrnak, reggeli és estveli égõáldozatokat.
\par 4 És megülék a sátoros ünnepet, a mint írva van, s áldozának égõáldozatot napról-napra szám és szokás szerint minden napit az õ napján.
\par 5 És azután áldozák az állandó napi, továbbá a hónapok elsõ napjain s az Úr minden szent ünnep napjain viendõ égõáldozatot, és mindazokért valót, a kik önkénytesen ajándékozának az Úrnak.
\par 6 Tehát a hetedik hónap elsõ napjától fogva kezdének égõáldozatot áldozni az Úrnak, a mikor az Úr templomának alapköve még nem tétetett le.
\par 7 És adának pénzt a kõ- és favágóknak s a mesterembereknek, és ételt és italt és olajat a Sidonbelieknek és Tírusbelieknek, hogy hozzanak czédrusfákat a Libánonról a joppéi tenger felé, Czírus, persa király nékik adott engedelme szerint.
\par 8 Második esztendõben azután, hogy fölmenének az Isten házához Jeruzsálembe, a második hónapban megkezdték az építést Zorobábel, a Sealtiél fia és Jésua, a Jósádák fia és a többi atyjafiaik, a papok és Léviták és mindnyájan, a kik a fogságból visszajöttek vala Jeruzsálembe, és rendelék a Lévitákat, a kik húsz esztendõsök vagy azon felül valának, az Úr háza építésének vezetésére.
\par 9 És elõlálla Jésua, az õ fiai és atyjafiai, Kadmiél és fiai, a Júda fiai, egyenlõképen, hogy vezérei legyenek az Isten házát építõ munkásoknak, továbbá Hénádád fiai, fiaik és testvéreik, - mind Léviták.
\par 10 S midõn az építõk letették az Úr templomának alapkövét, Jésua és Zorobábel oda állaták a papokat öltözetükben kürtökkel, s a Lévitákat, Ásáf fiait czimbalmokkal, hogy dícsérjék az Urat, Izráel királyának, Dávidnak rendelete szerint.
\par 11 És énekelének, dícsérvén az Urat és hálát adván néki, mert jó mert mindörökké van az õ irgalmassága Izráelen! És mind az egész nép nagy felszóval kiált vala, dícsérvén az Urat, hogy az Úr házának alapköve immáron letétetett!
\par 12 Nagy sokan pedig a papok és a Léviták és a családfõk közül, a vének, a kik látták vala az elsõ házat, mikor alapot vetnek vala most e háznak az õ szemök elõtt, nagy felszóval sírnak vala, sokan pedig örömükben nagy felszóval kiáltanak vala;
\par 13 Úgy hogy a nép nem tudja vala megkülönböztetni az örömben való kiáltást a nép siralmának szavától, mert a nép kiált vala nagy felszóval, és ez a szó messze földre meghallatott.

\chapter{4}

\par 1 Mikor pedig meghallák Júda és Benjámin ellenségei, hogy a kik a fogságból visszatértek, templomot építenek az Úrnak, Izráel Istenének:
\par 2 Menének Zorobábelhez és a családfõkhöz, s mondának nékik: Hadd építsünk együtt veletek, mert miképen ti, úgy mi is a ti Isteneteket keressük s néki áldozunk Esárhaddon, Assiria királyának idejétõl fogva, s ki ide hozott fel minket!
\par 3 És monda nékik Zorobábel és Jésua és Izráel családainak többi fõi: Nem veletek együtt kell nékünk házat építeni a mi Istenünknek, hanem mi magunk fogunk építeni az Úrnak, Izráel Istenének, a miképen megparancsolta nékünk a király, Czírus, Persia királya.
\par 4 És igyekezék e tartomány népe megkötni Júda népének kezeit és elrémíteni azt az építéstõl.
\par 5 És felbérelni ellene tanácsosokat, hogy semmivé tegyék szándékát Czírusnak, Persia királyának egész idejében, Dárius persa király uralkodásáig.
\par 6 És Ahasvérus uralkodásakor uralkodásának kezdetén, vádolást írának Júda és Jeruzsálem lakói ellen.
\par 7 Artaxerxes napjaiban pedig Bislám, Mithredathes és Tábeél s ennek többi társai írtak vala Artaxerxeshez a persiai királyhoz, s e levelet arám betûkkel írták, mely azután arám nyelvre fordíttatott át.
\par 8 Rehum, a helytartó és Simsai, a kanczellár, írának egy levelet Jeruzsálem ellen Artaxerxes királyhoz ekképen.
\par 9 Írának pedig ekkor Rehum, a helytartó és Simsai, a kanczellár és ezeknek többi társaik: a Dineusok, Afársatakeusok, Tarpeleusok, Afárseusok, Arkeveusok, Babilóniabeliek, Susánkeusok, Dehaveusok és a Hélameusok.
\par 10 És a többi népek, a kiket a nagy és dicsõséges Asznapár vitt el és tett lakosokká Samária városában és a többi városokban, melyek a folyóvizen túl vannak és a többi.
\par 11 Ez mássa a levélnek, melyet hozzá, Artaxerxes királyhoz küldének: "A te szolgáid, a folyóvizen túl lévõ férfiak és, a többi.
\par 12 Tudtára légyen a királynak, hogy a zsidók, a kik feljöttek tõled, megérkezének hozzánk Jeruzsálembe; a visszavonó és gonosz várost építik, a kõfalakat készítik s a fundamentomokat javítgatják.
\par 13 Mostan tudtára légyen a királynak, hogy ha ez a város megépíttetik s a kõfalak elkészülnek, adót, rovást és úti vámot nem fognak fizetni, s a királyok jövedelmét megkárosítják.
\par 14 Mostan, mivel a palota savával sózunk s épen ezért a király kárát nem illik elnéznünk, ez okért küldjük e levelet és tudatjuk a királylyal,
\par 15 Hogy nézzen utána valaki atyáid történeteinek könyvében, és meg fogod találni a történetek könyvében, és megtudod, hogy e város visszavonó és királyokat és tartományokat megkárosító város volt, s hogy eleitõl fogva lázadások történtek abban, ezért pusztíttatott is el e város.
\par 16 Tudatjuk mi a királylyal, hogy ha ez a város megépülend és a kõfalak elkészülnek: ennek miatta birtokod a folyóvizen túl nem lészen."
\par 17 A király választ küldött Rehumnak, a helytartónak és Simsainak, a kanczellárnak és többi társaiknak, a kik Samariában laknak vala, és minden folyóvizen túl lakozóknak: "Békesség és a többi.
\par 18 A levél, melyet hozzánk küldtetek, világosan felolvastatott elõttem;
\par 19 És megparancsolám, hogy nézzenek utána, és úgy találák, hogy ez a város eleitõl fogva királyok ellen támadó volt, és hogy pártütés és lázadás történt vala benne;
\par 20 És hogy hatalmas királyok voltak Jeruzsálemben és uralkodtak minden a folyóvizen túl lakókon, és adó, rovás és úti vám fizettetett nékik.
\par 21 Mostan azért parancsoljátok meg, hogy akadályozzák meg a férfiakat, s e város ne építtessék addig, míg tõlem parancsolat nem jövend;
\par 22 És meglássátok, hogy ebben valami mulasztást el ne kövessetek, hogy ne nevekedjék a veszedelem a királyok megkárosítására!"
\par 23 Mihelyt Artaxerxes király levelének mássa felolvastaték Rehum és Simsai, a kanczellár és társaik elõtt, menének nagy hamarsággal Jeruzsálembe a zsidókhoz, és megakadályozák õket erõvel és hatalommal.
\par 24 Akkor megszünék az Úr házának építése, mely Jeruzsálemben van, és szünetelt Dárius, Persia királya uralkodásának második esztendejéig.

\chapter{5}

\par 1 És prófétálának a próféták, Aggeus, a próféta és Zakariás az  Iddó fia a zsidóknak, a kik valának Júdában és Jeruzsálemben, szólván nékik az Izráel Istenének nevében.
\par 2 Akkor fölkelének Zorobábel, Sealtiél fia és Jésua, a Jósádák fia, és hozzá kezdenének Isten háza építéséhez, mely Jeruzsálemben van, s velök valának Isten prófétái, támogatván õket.
\par 3 Abban az idõben jöve hozzájok Tattenai folyóvizen túli helytartó, és Sethar-bóznai és társaik, és így szólának nékik: Ki adott néktek szabadságot, hogy e házat építsétek s e kõfalat készítsétek?
\par 4 Ekkor megmondánk nékik ily módon, hogy kik ama férfiak névszerint, kik ez épületet építik.
\par 5 És az õ Istenök szeme vala a zsidók vénein, hogy nem akadályozzák meg õket az építésben, míg az ügy Dárius elébe jutand, a mikor is levélben fognak felelni e dologra nézve.
\par 6 Mássa a levélnek, melyet küldött Tattenai folyóvizen túli helytartó, és Sethar-bóznai és az õ társai, az Afarsakeusok, a kik a folyóvizen túl lakának, Dárius királyhoz.
\par 7 Tudósítást küldének ugyanis hozzá, ekként levén az írva: "Dárius királynak minden békesség!
\par 8 Tudtára legyen a királynak, hogy elmentünk Júda tartományába, a nagy Istennek házához, és az építtetik nagy kövekbõl, és fa rakatik a falakra, és e munka szorgalmatosan folyik, és jó szerencsés lészen az õ kezök által.
\par 9 Ekkor megkérdénk azokat a véneket, ily módon szólván hozzájok: Ki adott néktek szabadságot, hogy e házat építsétek, s hogy e kõfalat készítsétek?
\par 10 Sõt még neveiket is megkérdeztük tõlük, hogy tudassuk veled, hogy megírhassuk azon férfiak nevét, kik fejeik.
\par 11 És ekképen felelének nékünk, mondván: Mi az Õ, a menny és föld Istenének szolgái vagyunk, és építjük e házat, mely ennekelõtte sok esztendõkön át meg vala építve, és Izráelnek egy nagy királya építé és végezé be azt.
\par 12 De minekutána haragra ingerelték volt atyáink a mennynek Istenét, adá õket a Babilóniabeli Káldeus királynak, Nabukodonozornak kezébe, a ki e házat lerontotta, és a népet Babilóniába rabságra vitte.
\par 13 Azonban Czírusnak, Babilónia királyának elsõ esztendejében Czírus király szabadságot adott, hogy Istennek ezt a házát megépítenék.
\par 14 Sõt az Isten házához való arany és ezüst edényeket is, a melyeket Nabukodonozor hozott vala el a jeruzsálemi templomból s bevitte volt azokat a babilóniai templomba, kihozatá Czírus király a babilóniai templomból s adatá azokat annak a Sesbassár nevûnek, a kit helytartóul rendelt;
\par 15 És monda néki: Vedd ez edényeket, menj és helyezd el azokat a jeruzsálemi templomba, és az Isten háza építtessék meg elõbbi helyén.
\par 16 Ekkor ez a Sesbassár eljött, letevé az Isten házának alapkövét, mely Jeruzsálemben van, és attól fogva mindeddig építtetik és még sem végzõdött be.
\par 17 Mostan azért, ha tetszik a királynak, nézzen utána valaki a király kincstartó házában, ott Babilóniában, ha úgy van-é, hogy Czírus király szabadságot adott az Isten házának megépítésére, mely Jeruzsálemben van, s a király akaratját erre nézve küldje hozzánk."

\chapter{6}

\par 1 Ekkor Dárius király megparancsolá, hogy nézzenek utána a könyvek tárházában, hogy a kincseket tartják vala Babiloniában.
\par 2 És találtaték Akhméta várában, a mely Média tartományában van, egy tekercs, melyre emlékezetül ez vala írva:
\par 3 Czírus király elsõ esztendejében, Czírus király parancsolatot adott Isten házára nézve, mely Jeruzsálemben van, hogy e ház építtessék meg, oly helyül, hol áldozatokat áldoznak; s alapzati felemeltetvén, magassága hatvan sing, szélessége is hatvan sing legyen.
\par 4 A nagy kövek rétege három s a fa rétege legyen egy, a költség pedig a király házából adassék.
\par 5 Sõt az Isten házának arany és ezüst edényeit is, a melyeket Nabukodonozor hozott vala el a jeruzsále_{Š/
\par 6 Mostan azért Tattenai folyóvizen túli helytartó, Sethar-bóznai és társaik, és az Afarsakeusok, a kik a folyóvizen túl laknak, távol legyetek onnan!
\par 7 Hagyjátok, hadd épüljön meg Istennek ama háza! A zsidók helytartója és a zsidók vénei építsék meg Isten ama templomát a maga helyén!
\par 8 És megparancsolom azt is, hogy mit cselekedjetek a zsidók ama véneivel, hogy megépítsék Isten ama házát; tudniillik a király folyóvizen túl való adójának kincseibõl pontosan költség adassék ama férfiaknak, és pedig félbeszakítás nélkül,
\par 9 És a mik szükségesek, tulkok, kosok, bárányok, égõáldozatra a menny Istenének, búza, só, bor, és olaj, a jeruzsálemi papok szava szerint adassanak nékik naponként és pedig mulasztás nélkül,
\par 10 Hogy vihessenek jó illatú áldozatot a menny Istenének, és könyörögjenek a királynak és fiainak életéért.
\par 11 És megparancsolom azt is, hogy valaki e rendelést megszegi, annak házából szakíttassék ki egy gerenda s felállíttatván szegeztessék reá, háza pedig szemétdombbá tétessék e miatt.
\par 12 Az Isten pedig, a ki nevének ott szerze lakást, megrontson minden királyt és népet, a ki felemelendi kezét, hogy megszegje e rendelést, hogy elpusztítsa Istennek ama házát, mely Jeruzsálemben van. Én, Dárius parancsoltam, pontosan megtörténjék.
\par 13 Akkor Tattenai folyóvizen túli helytartó, Sethar-bóznai és társaik, miután Dárius ily rendelést küldött, ahhoz képest cselekedének pontosan.
\par 14 És a zsidók vénei építének és jó szerencsések valának Aggeus próféta és Zakariás, Iddó fia prófétálása folytán, és megépítették és elvégezték Izráel Istenének akaratjából, és Czírus, Dárius és Artaxerxes, persiai királyok parancsolatjából.
\par 15 Bevégezteték pedig e ház az Adár hó harmadik napjáig, mely Dárius király országlásának hatodik esztendejében vala.
\par 16 És megtarták Izráel fiai, a papok, a Léviták és a fogságból visszajött többiek Isten házának felszentelését örömmel.
\par 17 Vivének pedig Isten házának felszentelésénél áldozatul száz bikát, kétszáz kost, négyszáz bárányt és tizenkét kecskebakot bûnért való áldozatra egész Izráelért, Izráel nemzetségeinek száma szerint.
\par 18 És rendelék a papokat az õ osztályaik és a Lévitákat az õ rendjeik szerint Istennek szolgálatára Jeruzsálemben, Mózes könyvének rendelése szerint,
\par 19 És megtarták a rabságból visszajöttek a páskhát az elsõ hó tizennegyedik napján;
\par 20 Mert megtisztultak vala a papok és Léviták: egyetemben mindnyájan tiszták valának, és megölék a húsvéti bárányt mindazokért, a kik a rabságból megjöttek, és atyjokfiaiért a papokért és önmagokért;
\par 21 És megevék azt Izráel fiai, a kik a rabságból visszajöttek vala, és mindazok, a kik elkülönítették vala magokat a föld népeinek tisztátalanságától s hozzájok állottak vala, hogy keressék az Urat, Izráel Istenét.
\par 22 És azután megtarták a kovásztalan kenyerek ünnepét hét napon örömmel, mert megvidámította vala õket az Úr, hozzájok hajtván az Assiriabeli király szívét, hogy erõsítse kezöket az Istennek, Izráel Istenének háza építésében.

\chapter{7}

\par 1 És ezek után Artaxerxes persa király uralkodásakor Ezsdrás, a Serája fia, ki Azariás fia, ki Hilkiás fia,
\par 2 Ki Sallum fia, ki Sádók fia, ki Ahitub fia,
\par 3 Ki Amária fia, ki Azariás fia, ki Mérájóth fia,
\par 4 Ki Zerahja fia, ki Uzzi fia, ki Bukki fia,
\par 5 Ki Abisua fia, ki Fineás fia, ki Eleázár fia, ki pedig Áronnak, a fõpapnak fia volt.
\par 6 Ez az Ezsdrás feljöve Babilóniából, (õ, a ki bölcs írástudó vala a Mózes törvényében, melyet az Úr, Izráel Istene adott vala) és megadá néki a király az Úrnak, az õ Istenének rajta nyugovó kegyelme szerint minden kérését.
\par 7 Feljövének pedig az Izráel fiai, a papok, a Léviták, az énekesek, a kapunállók és a Léviták szolgái közül többen Jeruzsálembe Artaxerxes király hetedik esztendejében.
\par 8 És Ezsdrás megérkezék velök Jeruzsálembe az ötödik hónapban, a király hetedik esztendejében.
\par 9 Ugyanis az elsõ hónak elsõ napján határozá el a Babilóniából való feljövetelt, és az ötödik hó elsõ napján érkezett Jeruzsálembe, az õ Istenének rajta nyugvó jó kegyelme szerint;
\par 10 Mert Ezsdrás erõs szívvel törekedett keresni és cselekedni az Úr törvényét, és tanítani Izráelben a rendeléseket és ítéleteket.
\par 11 És ez mássa a levélnek, melyet adott vala Artaxerxes király Ezsdrásnak, az írástudó papnak, a ki írástudó vala az Úr parancsolatainak beszédiben, Izráelnek adott rendeléseiben:
\par 12 Artaxerxes, a királyok királya Ezsdrás papnak, ki a menny Istenének törvényében bölcs írástudó és a többi.
\par 13 Szabadságot adok, hogy valaki országomban Izráel népe, papjai és a Léviták közül Jeruzsálembe akar menni, veled elmehet.
\par 14 Minthogy te elbocsáttatol a királytól és hét tanácsosától, hogy utána nézz Júdának és Jeruzsálemnek, a te Istened törvénye szerint, mely kezedben van,
\par 15 És hogy elvigyed az ezüstöt és aranyat, melyet a király és tanácsosai önkénytesen ajándékoznak Izráel Istenének, kinek hajléka Jeruzsálemben van,
\par 16 És mindazt az ezüstöt és aranyat, melyet kapni fogsz Babilónia egész tartományában, együtt a nép és a papok önkénytes ajándékával, mit ezek önkénytesen ajándékoznak az õ Istenök házának, mely Jeruzsálemben van.
\par 17 Annálfogva gondosan végy e pénzen bikákat, kosokat és bárányokat s hozzájok való étel- és italáldozatokat, s áldozd meg azokat Istenetek házának oltárán, mely Jeruzsálemben van:
\par 18 A megmaradott ezüsttel és arannyal pedig, a mit jónak láttok cselekedni te és atyádfiai, Istenetek akaratja szerint, azt cselekedjétek.
\par 19 És az edényekkel, melyek néked adattak át a te Istened házának szolgálatára, számolj be Isten elõtt Jeruzsálemben.
\par 20 Egyéb szükségét pedig Istened házának, mit csak teljesítened kell, teljesítsd a király kincstartó házából.
\par 21 És én, Artaxerxes, a király, parancsolom minden a folyóvizen túl lakó kincstartóimnak, hogy minden, a mit csak kérend tõletek Ezsdrás pap, ki a menny Istenének törvényében írástudó, pontosan teljesíttessék,
\par 22 És pedig száz tálentom ezüstig, száz kór búzáig, száz báth borig, száz báth olajig és sóban, a mennyi elég.
\par 23 Minden, mi a menny Istenének akaratja szerint való, pontosan teljesíttessék a menny Istenének házáért, hogy meg ne haragudjék a királynak és fiainak országára.
\par 24 Veletek pedig tudatjuk, hogy sem a papokra, sem a Lévitákra, sem az énekesekre, sem a kapunállókra, sem a Léviták szolgáira, sem Isten e háza szolgáira, adót, rovást és úti vámot senkinek vetni nem szabad.
\par 25 És te Ezsdrás, a te Istened bölcs törvénye szerint, a mely kezedben van, rendelj ítélõket és birákat, a kik törvényt tegyenek minden a folyóvizen túl lakó nép között, mindazok között, a kik tudják Istenednek törvényeit, és a kik nem tudják, azokat tanítsátok!
\par 26 Valaki pedig nem fogja cselekedni a te Istenednek törvényét és a király törvényét, ítélet hozassék felette pontosan, vagy halálra, vagy számkivetésre, vagy jószágvesztésre, vagy fogságra.
\par 27 Áldott az Úr, atyáink Istene, a ki erre indítá a király szívét, hogy megékesítse az Úr házát, mely Jeruzsálemben van.
\par 28 És a ki hozzám fordítá irgalmasságát a király elõtt és tanácsosai elõtt és a király minden hatalmas fejedelmei elõtt! És én megerõsödvén az Úrnak az én Istenemnek rajtam nyugovó kegyelme által, családfõket gyûjték össze Izráelbõl, hogy feljönnének velem.

\chapter{8}

\par 1 Ezek pedig családfõik és nemzetségi eredetök azoknak, a kik feljövének velem Artaxerxes uralkodásakor Babilóniából:
\par 2 Fineás fiai közül: Gersóm; Ithamár fiai közül: Dániel; Dávid fiai közül: Hattus;
\par 3 A Sekánia fiai közül, Parós fiai közül: Zakariás s vele egy nemzetségben másfélszáz férfi;
\par 4 Pahath-Moáb fiai közül: Eljóénai, Zerahia fia s vele kétszáz férfi;
\par 5 Sekánia fiai közül a Jaháziel fia s õ vele háromszáz férfi;
\par 6 Ádin fiai közül: Ebed, Jónathán fia és vele ötven férfi;
\par 7 Élám fiai közül: Ésaiás, Athália fia és vele hetven férfi;
\par 8 Sefátia fiai közül: Zebádia, Mikháel fia és õ vele nyolczvan férfi;
\par 9 Joáb fiai közül: Obádia Jéhiel fia és vele kétszáztizennyolcz férfi;
\par 10 Selómith fiai közül a Jószifia fia és vele százhatvan férfi;
\par 11 Bébai fiai közül: Zakariás, Bébai fia és vele huszonnyolcz férfi;
\par 12 Azgád fiai közül: Jóhanán, Hakkátán fia és vele száztíz férfi;
\par 13 Adónikám fiai közül az utódok s neveik: Elifélet, Jeiél és Semája és velök hatvan férfi;
\par 14 És Bigvai fiai közül: Uthai és Zabbud és velök hetven férfi.
\par 15 Mikor pedig összegyûjtém õket az Aháva felé folyó folyóvízhez, hol három napig valánk, megnéztem jól a népet és a papokat, s a Lévi fiai közül nem találtam közöttük senkit.
\par 16 Elküldém annakokáért Eliézert, Arielt, Semáját, Elnáthánt, Járibot, Elnáthánt, Náthánt, Zakariást, Mesullámot, mint családfõket, és Jójáribot és Elnáthánt, mint eszes embereket.
\par 17 És rendelém õket Iddóhoz, a ki fõ vala a Kászifia nevû helyen, és betanítám õket, hogy mit szóljanak Iddónak és az õ atyjafiainak, a Léviták szolgáinak a Kászifia nevû helyen, - hogy hozzanak szolgákat a mi Istenünk házához.
\par 18 És elhozák nékünk, Istenünknek rajtunk nyugvó jó kegyelme szerint, Is-szekhelt, a Mahli fiai közül, a ki Lévi fia, a ki meg Izráel fia vala, és Serébiát, fiaival és tizennyolcz testvérével együtt.
\par 19 És Hasábiát s vele Ésaiást, a Mérári fiai közül, testvéreivel és húsz fiával együtt;
\par 20 És a Léviták szolgái közül, a kiket Dávid és a fejedelmek adának a Léviták szolgálatába, kétszázhuszat; mindnyájok nevei feljegyeztettek.
\par 21 Ekkor bõjtöt hirdeték ott az Aháva folyóvíz mellett, hogy megaláznók magunkat a mi Istenünk elõtt, hogy kérnénk tõle szerencsés utat magunknak, családainknak és minden marháinknak.
\par 22 Mert szégyeltem vala a királytól kérni sereget és lovagokat oltalmunkra a mi ellenségeink ellen ez útban, mert ezt mondtuk volt a királynak, mondván: A mi Istenünk kegyelme van mindazokon, a kik õt keresik, az õ javukra, és az õ hatalma és haragja van mindazokon, a kik elhagyják õt.
\par 23 És bõjtölénk a mi Istenünkhöz könyörgénk annakokáért, és meghallgatott minket.
\par 24 Ekkor különválaszték a papi fejedelmek közül tizenkettõt Serébiához és Hasábiához és hozzájok atyjokfiai közül tízet.
\par 25 És átmérém nékik az ezüstöt és aranyat és az edényeket, a mi Istenünk házának ajándékait, a melyeket a király és tanácsosai és fejedelmei és minden Babilóniában élõ Izráeliták ajándékozának.
\par 26 Kezökhöz mérék pedig hatszázötven tálentom ezüstöt, száz tálentomot érõ ezüst edényeket és száz tálentom aranyat.
\par 27 Húsz arany poharat, melyek ezer dárikot érnek vala, és két szép ragyogású rézedényt, melyek oly becsesek, mint az arany.
\par 28 És mondék nékik: Ti az Úrnak szentei vagytok, ez edények is szentek, és ez az ezüst és arany az Úrnak, atyáitok Istenének önkénytesen adott ajándék:
\par 29 Vigyázzatok azért reá és megõrizzétek, míg átméritek a papi fejedelmek, a Léviták és Izráel családainak fejedelmei elõtt Jeruzsálemben az Úr házának kamaráiban!
\par 30 És átvevék a papok és a Léviták az ezüstöt, aranyat és az edényeket súly szerint, hogy vigyék Jeruzsálembe Istenünk házához.
\par 31 Elindulánk pedig az Aháva folyóvíz mellõl az elsõ hó tizenkettedik napján, hogy Jeruzsálembe menjünk; s a mi Istenünk kegyelme nyugodott rajtunk, megszabadítván minket a ránk leselkedõ ellenségnek kezébõl ez úton.
\par 32 És megérkezénk Jeruzsálembe, és pihenénk ott három napig.
\par 33 Negyednap pedig átméretett az ezüst, az arany és az edények a mi Istenünk házában, Merémóth papnak Uriás fiának kezéhez, vele lévén Eleázár, a Fineás fia, és velök Józadáb, a Jésua fia és Nóadia, a Binnui fia, Léviták,
\par 34 Minden szám és súly szerint, s feljegyeztetett az egész súly abban az idõben.
\par 35 A kik pedig megérkeztek a fogságból, a rabságnak fiai, égõáldozatokat vivének Izráel Istenének: tizenkét tulkot egész Izráelért, kilenczvenhat kost, hetvenhét bárányt, tizenkét bakot bûnért való áldozatra, mindezt egészen égõáldozatul az Úrnak.
\par 36 És átadták a király parancsolatait a király helytartóinak, s a vizen túl lévõ helytartóknak, és ezek segíték a népet és az Isten házát.

\chapter{9}

\par 1 Minekutána ezek elvégezõdének, jövének hozzám a fõemberek, mondván: Izráel népe és a papok és a Léviták nem különíték el magokat e tartományok népeitõl, s miképen pedig azoknak, a Kananeusoknak, Hitteusoknak, Perizeusoknak, Jebuzeusoknak, Ammonitáknak, Moábitáknak, Égyiptomiaknak és Emoreusoknak útálatos vétke szerint el kellett volna,
\par 2 Mert ezek leányai közül vettek vala feleséget magoknak és fiaiknak, és megelegyedett a szent mag e tartományok népeivel; és pedig a fejedelmek és fõemberek valának elsõk e bûnben.
\par 3 Mihelyt e dolgot meghallottam, megszaggatám alsó- és felsõ ruhámat, s téptem fejem hajszálait és szakállamat, és veszteg ültem.
\par 4 És hozzám gyûlének mindnyájan, a kik reszketve gondoltak Izráel Istenének beszédeire azoknak vétke miatt, a kik a fogságból megjöttek vala; és én veszteg ülök vala mind az estvéli áldozatig.
\par 5 Az estvéli áldozatkor pedig felkeltem sanyargatásomból, megszaggatván alsó- és felsõ ruhámat; és térdeimre esvén, kiterjesztém kezeimet az Úrhoz, az én Istenemhez.
\par 6 És mondék: Én Istenem, szégyenlem és átallom felemelni, én Istenem az én orczámat te hozzád, mert a mi álnokságaink felülhaladtak fejünk fölött és a mi vétkeink mind az égig nevekedtek!
\par 7 A mi atyáink napjaitól fogva nagy vétekben vagyunk mi mind e mai napig, és a mi álnokságainkért adattunk vala mi, a királyaink és a mi papjaink a földi királyok kezébe, fegyver által rabságra és ragadományra és orczapirulásra, a mint ez a mai nap is van.
\par 8 És most nem sok ideje, hogy az Úr, a mi Istenünk rajtunk könyörült, hogy hagyjon minékünk maradékot, és hogy adjon nékünk egy szeget az õ szent helyén, hogy így megvilágosítsa szemeinket a mi Istenünk, s hogy megelevenítsen bennünket egy kissé a mi szolgaságunkban.
\par 9 Mert szolgák vagyunk mi, de szolgaságunkban nem hagyott el minket a mi Istenünk, hanem hozzánk fordítá irgalmasságát Persiának királyai elõtt, hogy megelevenítene bennünket, hogy felemelhessük a mi Istenünk házát s megépíthessük annak romjait, és hogy adjon nékünk bátorságos lakást Júdában és Jeruzsálemben.
\par 10 És most mit mondjunk, óh mi Istenünk, mindezek után? Azt, hogy mi mégis elhagytuk parancsolataidat,
\par 11 A melyeket parancsoltál szolgáid, a próféták által, mondván: A föld, melyre bementek, hogy bírjátok azt, tisztátalan föld, a tartományok népeinek tisztátalansága miatt, útálatosságaik miatt, melyekkel betölték azt egyik végétõl a másikig tisztátalanságukban;
\par 12 Annakokáért leányaitokat ne adjátok az õ fiaiknak és az õ leányaikat ne vegyétek fiaitoknak, és nem keressétek barátságukat, sem javokat soha, hogy megerõsödjetek és éljetek e föld javaival, és örökségképen adhassátok azt fiaitoknak mindörökké.
\par 13 Mindazok után pedig, a mik reánk jövének gonosz cselekedeteinkért és nagy vétkünkért, hiszen te, mi Istenünk, jobban kedveztél nékünk, sem mint bûneink miatt érdemeltünk volna, s adád nékünk e maradékot.
\par 14 Hát megrontjuk-é ismét parancsolataidat s összeházasodunk-é ez útálatos népekkel? Nem fogsz-é haragudni reánk mindaddig, míg megemésztetünk, hogy sem maradékunk, sem hírmondónk ne legyen?
\par 15 Oh Uram, Izráel Istene! igaz vagy te, mert meghagytál minket, maradék gyanánt, mint e mai nap bizonyítja. Ímé elõtted vagyunk vétkünkben, és nem állhatunk meg elõtted e miatt!

\chapter{10}

\par 1 És mikor imádkozék Ezsdrás és vallást tõn, sírva és földre borulva az Isten háza elõtt, sereglének õ hozzá Izráelbõl igen nagy gyülekezetben férfiak és asszonyok és gyermekek, mert síra a nép is sokat és keservesen.
\par 2 És szóla Sekhánia, Jéhielnek fia, az Élám fiai közül, és monda Ezsdrásnak: Mi vétkeztünk a mi Istenünk ellen, hogy idegen feleségeket vettünk magunknak e föld népei közül: mindazáltal van reménysége Izráelnek, mind e mellett is!
\par 3 Vessünk ugyanis frigyet a mi Istenünkkel, hogy elbocsátjuk mindazon asszonyokat és a tõlük szülötteket az én Uramnak és azoknak akaratja szerint, a kik reszketve gondolnak a mi Istenünk parancsolatára, és a törvény szerint cselekedjünk!
\par 4 Kelj fel! mert reád néz e dolog, s mi veled leszünk: légy erõs és láss hozzá!
\par 5 Fölkele annakokáért Ezsdrás és megesketé a papoknak, a Lévitáknak és az egész Izráelnek fejedelmeit, hogy e beszéd szerint fognak cselekedni; és megesküvének.
\par 6 És fölkele Ezsdrás az Isten háza elõl, és elméne Jóhanánnak, Eliásib fiának szobájához s bemenvén abba, kenyeret nem evék, sem vizet nem ivék, mert gyászol vala a rabságból hazajöttek vétke miatt.
\par 7 És meghirdeték Júdában és Jeruzsálemben mindazoknak, a kik a rabságból hazajöttek, hogy Jeruzsálembe gyûljenek;
\par 8 Valaki pedig el nem jön harmadnapra, a fejedelmek és vének tanácsa szerint, adassék minden vagyona a templomnak, és õ maga vettessék ki a rabságból hazajötteknek gyülekezetébõl.
\par 9 Összegyûlének azért Júdának és Benjáminnak minden férfiai harmadnapra Jeruzsálembe a kilenczedik hóban, e hó huszadik napján, és leüle az egész nép az Isten házának piaczán, aggódván e dolog és az esõzések miatt.
\par 10 Ekkor fölkele Ezsdrás, a pap, és monda nékik: Ti vétkeztetek, hogy idegen feleségeket vettetek magatoknak, hogy ezzel is többítenétek Izráel vétkét;
\par 11 Annakokáért tegyetek vallást az Úr elõtt, atyáitoknak Istene elõtt, és cselekedjetek az õ akaratja szerint, elkülönítvén magatokat e föld népeitõl és az idegen feleségektõl.
\par 12 És felele az egész gyülekezet, és monda felszóval: Ekképen a te beszéded szerint kell minékünk cselekednünk!
\par 13 De e nép igen sok és az esõs idõ miatt kivül nem állhatunk; annak felette ez nem egy, sem nem két napi munka, mert sokan vagyunk, a kik vétkeztünk e dologban;
\par 14 Hadd álljanak hát elõ fejedelmeink, az egész gyülekezeté, és mindenki, a ki idegen feleséget vett magának, városonként jõjjön elõ bizonyos idõben s velök a városoknak vénei és birái, hogy elfordítsuk magunkról a mi Istenünknek e dolog miatt való búsulásának haragját!
\par 15 Csak Jónathán, az Asáhel fia és Jahzéja, a Tikva fia állának fel ez ellen és Mésullám és Sabbethai, a Léviták támogaták õket.
\par 16 És cselekedének ekképen azok, a kik a rabságól hazajöttek; és választa magának Ezsdrás, a pap férfiakat, a családfõket, családjuk szerint, s mindnyájok nevei feljegyeztettek. És összeülének a tizedik hó elsõ napján, hogy nyomozzák e dolgot;
\par 17 És bevégezék azt mindazon férfiakra nézve, a kik idegen feleséget vettek vala magoknak, az elsõ hónak elsõ napjáig.
\par 18 Találtatának pedig a papok fiai közül a kik idegen feleségeket vettek vala magoknak, Jésuának, a Jósádák fiának és az õ testvéreinek fiai közül: Maaséja, Eliézer, Járib és Gedálja;
\par 19 És kezöket adák, hogy elbocsátják feleségeiket és egy kos áldozására ítéltetének vétkükért;
\par 20 Immér fiai közül pedig: Hanáni és Zebádja.
\par 21 Hárim fiai közül: Maaszéja, Elija, Semája, Jéhiel és Uzzia,
\par 22 Pashur fiai közül: Eljoénaj, Maaszéja, Ismáel, Nethanéel, Józabád és Elásza;
\par 23 Továbbá a Léviták közül: Józabád, Simei és Kélája. (az a Kelita), Pethahja, Júda és Eliézer;
\par 24 Az énekesek közül pedig: Eliásib, s a kapunállók közül: Sallum, Telem és Uri;
\par 25 Továbbá Izráel népébõl: Parós fiai közül: Ramina, Jezija, Malkija, Mijámin, Eleázár, Malkija és Benája;
\par 26 Elám fiai közül: Mattánia, Zakariás, Jéhiel, Abdi, Jerémóth és Élija;
\par 27 Zattu fiai közül: Eljóénai, Eljásib, Mattánia, Jerémóth, Zabád és Aziza;
\par 28 Bébai fiai közül: Jóhanán, Hanánia, Zabbai és Athlai;
\par 29 Báni fiai közül: Mesullám, Mallukh, Adája, Jásub, Seál és Rámóth;
\par 30 Pahath-Moáb fiai közül: Adna, Kelál, Benája, Maaszéja, Mattánia, Besaléel, Binnui és Manasse;
\par 31 A Hárim fiai pedig: Eliézer, Jissija, Malkija, Semája, Simeon;
\par 32 Benjámin, Mallukh, Semarja;
\par 33 Hásum fiai közül: Mattenai, Mattattá, Zabád, Elifélet, Jerémai, Manasse, Simei,
\par 34 Báni fiai közül: Maádai, Amrám és Uél;
\par 35 Benája, Bédéja, Keluhu;
\par 36 Vanja, Merémóth, Eliásib;
\par 37 Mattánia, Mattenai és Jaaszái;
\par 38 Báni, Binnui, Simei;
\par 39 Selemia, Náthán és Adája;
\par 40 Makhnadbai, Sásai, Sárai;
\par 41 Azareél, Selemia és Semária;
\par 42 Sallum, Amaria, József:
\par 43 A Nebó fiai közül: Jéiel, Mattithja, Zabád, Zebina, Jaddai, Jóel, Benája.
\par 44 Mindezek idegen feleségeket vettek vala magoknak. És valának e feleségek között olyanok is a kik már fiakat szültek.


\end{document}