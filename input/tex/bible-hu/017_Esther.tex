\begin{document}

\title{Esther}


\chapter{1}

\par 1 Ahasvérus idejében történt (ez az Ahasvérus az, a ki uralkodott Indiától fogva Szerecsenországig százhuszonhét tartományon),
\par 2 Azokban a napokban, a mikor Ahasvérus király üle királyi székében, a mely Susán várában volt,
\par 3 Uralkodása harmadik évében lakomát szerze minden fejedelmének és szolgáinak: Persia és Média hatalmasai, fõemberei és a tartományok fejedelmei elõtte valának.
\par 4 Midõn mutogatá országa dicsõségének gazdagságát és az õ méltóságának fényes díszét sok napon át, száznyolczvan napig.
\par 5 És mikor elmultak ezek a napok, tõn a király az egész népnek, mely Susán várában találtaték, kicsinytõl nagyig, hét napig tartó lakomát, a király palotája kertjének udvarán.
\par 6 Fehér gyapjú és kék bíbor szõnyegek valának megerõsítve a fehér gyapjúból és bíborból való köteleken, ezüst karikákon és fehér márványból való oszlopokon. Aranyból és ezüstbõl való kerevetek voltak az alabástromból, fehér márványból, gyöngykõbõl és vörös márványból való padozaton.
\par 7 Ittak pedig arany poharakból és különb-különbféle poharakból, és királyi bor bõven vala, királyi módon.
\par 8 És az ivás, rendelet folytán, kényszer nélkül történt; mert így hagyta meg a király háza minden gondviselõjének, hogy kinek-kinek akarata szerint cselekedjenek.
\par 9 Vásti királyné is lakomát szerze a nõknek, a királyi házban, mely Ahasvérus királyé vala.
\par 10 Hetedik napon, mikor megvídámult a király szíve a bortól, mondá Méhumánnak, Biztának, Hárbonának, Bigtának és Abagtának, Zetárnak és Karkásnak, a hét udvarmesternek, a kik szolgálának Ahasvérus király elõtt,
\par 11 Hogy hozzák el Vástit, a királynét a király elé, királyi koronával, hogy megmutassa a népeknek és fejedelmeknek az õ szépségét; mert szép arczú vala.
\par 12 De Vásti királyné nem akara menni a király szavára, mely vala az udvarmesterek által. Erre igen megharagudott a király, és az õ haragja felgerjede benne.
\par 13 És monda a király az idõket mérõ bölcseknek, (mert a királynak összes dolgai a jog- és törvénytudók elé tartoznak.
\par 14 Legközelebb valának pedig hozzá: Karséna, Sétár, Admata, Társis, Méres, Marséna, Mémukán, Persiának és Médiának hét fejedelme, a kik láták a király arczát, és elõl ülének az országban):
\par 15 Törvény szerint mit kellene cselekedni Vásti királynéval, mivelhogy nem teljesíté Ahasvérus királynak az udvarmesterek által üzent beszédét?
\par 16 Akkor monda Mémukán a király és a fejedelmek elõtt: Nemcsak a király ellen vétkezett Vásti királyné; de valamennyi fejedelem és minden nép ellen, Ahasvérus király minden tartományában;
\par 17 Mert a királyné cselekedetének híre fut minden asszonyhoz, úgy, hogy megvetik férjeiket szemeik elõtt, és azt mondják: Ahasvérus király megparancsolá, hogy Vásti királyné hozzá menjen, és nem ment.
\par 18 És e napon ugyanazt fogják mondani Persia és Média fejedelemasszonyai, a kik meghallják a királyné cselekedetét, a király minden fejedelmének, és lesz elég megvetés és harag.
\par 19 Ha azért a királynak tetszik, bocsásson ki királyi parancsot a maga részérõl és irattassék a Persa-Méd törvények közé visszavonhatatlanul, hogy ne jõjjön többé Vásti Ahasvérus király színe elé, és hogy az õ királyságát adja a király másnak, a ki jobb õ nálánál.
\par 20 És hallják meg a király rendeletét, a melyet teend, egész országában, mert igen nagy az, hogy minden asszony adja meg a tiszteletet az õ férjének, a legnagyobbtól a legkisebbig.
\par 21 És tetszett a szó a királynak és a fejedelmeknek, és a király Mémukán beszéde szerint cselekedék.
\par 22 És leveleket külde a királynak minden tartományába, tartományról tartományra, annak saját írása szerint, és néprõl-népre, annak nyelve szerint, hogy legyen minden férfi úr az õ házában és beszéljen az õ népének nyelvén.

\chapter{2}

\par 1 Ezek után, a mint megszünt Ahasvérus király haragja, megemlékezék Vástiról, és arról a mit cselekedett, és arról a mit végeztek felõle.
\par 2 És mondának a király apródai, a kik néki szolgálának: Keressenek a királynak szép ábrázatú szûz leányokat.
\par 3 És rendeljen a király tiszteket országának minden tartományába, a kik gyûjtsenek össze minden szép ábrázatú szûz leányt Susán várába, az asszonyok házába, Hégainak, a király kamarásának, az asszonyok õrzõjének keze alá, és adjanak nékik szépítõ szereket.
\par 4 És az a leány, a ki tetszik a király szemeinek, uralkodjék Vásti helyett. És tetszett a dolog a királynak, és úgy cselekedék.
\par 5 Vala egy zsidó férfiú Susán várában, a kinek neve Márdokeus, Jáirnak fia, a ki Simei fia, a ki Kis fia, Benjámin nemzetségébõl,
\par 6 A ki Jeruzsálembõl fogva vitetett el a számûzött néppel, a mely elvitetett Jekóniással, Júda királyával együtt, a kit rabságba vitt el Nabukodonozor, babilóniai király.
\par 7 És ez gondviselõje volt Hadassának, azaz Eszternek az õ nagybátyja leányának; mert sem atyja, sem anyja nem volt, és a leány szép alakú és szép ábrázatú vala, s mikor meghalt az õ atyja és anyja, Márdokeus leánya gyanánt magához fogadá.
\par 8 Lõn pedig, mikor a király parancsa és rendelete kihirdettetett és sok leány gyûjtetett Susán várába, Hégai keze alá, akkor felvétetett Eszter is a király házába, Hégainak, az asszonyok õrének keze alá.
\par 9 És tetszett a leány az õ szemeinek és kegyébe fogadá, és siete néki szépítõ szereket adni, és kiadni az õ részét és hét leányzót, a kit néki a király házából kelle kiszemelni, és vivé õt és az õ szolgálóleányait az asszonyok legjobb házába.
\par 10 Meg nem mondá Eszter az õ nemzetségét és származását; mert Márdokeus meghagyta néki, hogy meg ne mondja.
\par 11 Márdokeus pedig minden nap járt az asszonyok házának pitvara elõtt, hogy tudakozódjék Eszter hogyléte felõl, és hogy mi történik vele.
\par 12 Mikor pedig eljött az ideje minden leánynak, hogy bemenne Ahasvérus királyhoz, miután asszonyok törvénye szerint tizenkét hónapig bántak vele (mert ennyi idõbe telik az õ szépítésök, hat hónapig mirtusolajjal és hat hónapig illatos szerekkel és asszonyi szépítõ szerekkel):
\par 13 Akkor így ment a leány a királyhoz: Mindent, a mit kivánt, megadtak néki, hogy menjen azzal az asszonyok házából a király házáig;
\par 14 Este bement és reggel visszatért az asszonyok második házába, Sahásgáznak, a király udvarmesterének, a hálótársak õrzõjének kezei alá; nem jött többé a királyhoz, csak ha kivánta õt a király, és nevén hívták.
\par 15 Mikor azért eljött az ideje Eszternek, a Márdokeus nagybátyja, Abihail leányának, a kit leánya gyanánt fogadott, hogy bemenne a királyhoz, nem kivánt mást, mint a mit Hégai, a király udvarmestere, az asszonyok õrzõje mondott néki, és kedves vala Eszter mindenki szemei elõtt, a ki õt látá.
\par 16 És felviteték Eszter Ahasvérus királyhoz, az õ királyi házába, a tizedik hónapban, ez a Tébet hónapja, országlásának hetedik évében.
\par 17 És a király Esztert minden asszonynál inkább szereté, és minden leánynál nagyobb kedvet és kegyelmet nyert õ elõtte, és tette a királyi koronát az õ fejére, és királynévá tette Vásti helyett.
\par 18 És nagy lakomát adott a király minden õ fejedelmének és szolgáinak, az Eszter lakomáját, és a tartományoknak nyugalmat adott, és ajándékokat osztogata király módjára.
\par 19 És mikor összegyûjték a szûzeket másodszor, Márdokeus üle a király kapujában.
\par 20 És Eszter nem mondá meg sem származását, sem nemzetségét, miképen megparancsolá néki Márdokeus; mert Márdokeus szavát Eszter úgy fogadá, mint mikor gyámja volt.
\par 21 Azokban a napokban pedig, mikor Márdokeus üle a király kapujában, megharagudott Bigtán és Téres, a király két udvarmestere, a kapu õrei, és azon voltak, hogy rávetik kezeiket Ahasvérus királyra.
\par 22 Tudtára esett e dolog Márdokeusnak, és megmondá Eszter királynénak, és Eszter elmondá a királynak Márdokeus nevében.
\par 23 És megvizsgálták a dolgot és úgy találták, és mind a kettõt fára akasztották. És megírattaték a krónikák könyvébe a király elõtt.

\chapter{3}

\par 1 Ezek után nagy méltóságra emelé Ahasvérus király Hámánt, a Hammedáta fiát, az Agágibelit, és felmagasztalá õt, és feljebb helyezteté székét minden fejedelménél, a kik vele valának.
\par 2 És a király minden szolgái, a kik a király kapujában valának, térdet hajtottak és leborultak Hámán elõtt; mert úgy parancsolta meg nékik a király; de Márdokeus nem hajtott térdet és nem borult le.
\par 3 Mondának azért a király szolgái, a kik a király kapujában valának, Márdokeusnak: Miért szeged meg a király parancsát?
\par 4 Lõn pedig, mikor így szólnának néki minden nap és nem hallgata rájok, feljelenték Hámánnak, hogy lássák, megállnak-é Márdokeus dolgai, mert azt jelenté nékik, hogy õ zsidó.
\par 5 És látván Hámán, hogy Márdokeus térdet nem hajt és nem borul le elõtte, megtelék Hámán haraggal.
\par 6 De kevés volt elõtte, hogy csakis Márdokeusra magára vesse rá kezét, (mert megmondták néki Márdokeus nemzetségét) azért igyekezett Hámán elveszteni minden zsidót, a ki Ahasvérus egész országában vala, a Márdokeus nemzetét.
\par 7 Az elsõ hónapban, ez Nisán hónapja, Ahasvérus királyságának tizenkettedik évében, Púrt, azaz sorsot vetének Hámán elõtt napról-napra és hónapról-hónapra a tizenkettedikig, s ez Adár hónapja.
\par 8 És monda Hámán Ahasvérus királynak: Van egy nép, elszórva és elkülönítve a népek között, országod minden tartományában, és az õ törvényei különböznek minden nemzetségtõl, és a király törvényeit nem teljesíti; a királynak bizony nem illik úgy hagyni õket.
\par 9 Ha a királynak tetszik, írja meg, hogy õk elvesztessenek, és én tízezer tálentom ezüstöt mérek a hivatalnokok kezeibe, hogy a király kincstárába vigyék.
\par 10 Akkor lehúzá a király az õ gyûrûjét a maga kezérõl, és adá azt az Agágibeli Hámánnak, Hammedáta fiának, a zsidók ellenségének.
\par 11 És monda a király Hámánnak: Az ezüst tied legyen s a nép is, hogy azt cselekedjed vele, a mi néked tetszik.
\par 12 Elõhivatának azért a király irnokai az elsõ hónap tizenharmadik napján, és megiraték minden úgy, a miként Hámán parancsolá, a király fejedelmeinek és a kormányzóknak, a kik az egyes tartományokban valának, és minden egyes nép fejeinek; minden tartománynak annak írása szerint, és minden egyes népnek az õ nyelve szerint, Ahasvérus király nevében iratott és megpecsételtetett a király gyûrûjével.
\par 13 És elküldettek a levelek futárok által a király minden tartományába, hogy kipusztítsák, megöljék és megsemmisítsék mind a zsidókat, ifjútól a vénig, gyermekeket és asszonyokat egy napon, tizenharmadik napján a tizenkettedik hónapnak, (ez Adár hónapja) és hogy javaikat elragadják.
\par 14 Az írásnak mássa, hogy tétessék törvény minden egyes tartományban, meghirdettetett minden népnek, hogy legyenek készen azon a napon.
\par 15 A futárok kimenének gyorsan a király parancsával. És a törvény Susán várában is kiadatott; a király pedig és Hámán leültek, hogy igyanak; de Susán városa felháborodott.

\chapter{4}

\par 1 Márdokeus pedig megtudta mindazt, a mi történt; és Márdokeus megszaggatá ruháit, zsákba és hamuba öltözék, és a város közepére méne és kiálta nagy és keserves kiáltással.
\par 2 És eljött a királynak kapuja elé; mert nem volt szabad bemenni a király kapuján zsákruhában.
\par 3 És minden egyes tartományban, azon a helyen, a hová a király parancsa és törvénye eljutott, nagy gyásza volt a zsidóknak, bõjt és siralom és jajgatás; zsák és hamu vala terítve sokak alá.
\par 4 Eljöttek azért Eszternek leányai és udvarmesterei és megmondák néki, és megszomorodék a királyasszony nagyon, és külde ruhákat, hogy felöltöztetnék Márdokeust, és hogy vesse le a zsákot magáról; de nem fogadá el.
\par 5 Akkor elõhivatá Eszter Hatákot, a király udvarmesterei közül a ki az õ szolgálatára volt rendelve, és kiküldé õt Márdokeushoz, hogy megtudja, mi az és miért van az?
\par 6 Kiméne tehát Haták Márdokeushoz a város utczájára, mely a király kapuja elõtt vala.
\par 7 És elmondá néki Márdokeus mindazt, a mi érte õt, és az ezüst összegét, a melyet Hámán mondott, hogy juttat a király kincséhez a zsidókért, hogy elvesztessenek.
\par 8 És az írott rendeletnek mássát is, melyet Susánban adtak ki eltöröltetésökre, átadá néki, hogy mutassa meg Eszternek, és jelentse és hagyja meg néki, hogy menjen a királyhoz, könyörögni néki és esedezni elõtte az õ nemzetségéért.
\par 9 Elméne azért Haták, és elmondá Eszternek Márdokeus szavait.
\par 10 És monda Eszter Hatáknak, és meghagyá néki, hogy tudassa Márdokeussal:
\par 11 A király minden szolgája és a király tartományainak népe tudja, hogy minden férfinak és asszonynak, a ki bemegy a királyhoz a belsõ udvarba hivatlanul, egy a törvénye, hogy megölettessék, kivévén, a kire a király aranypálczáját kinyujtja, az él; én pedig nem hívattam, hogy a királyhoz bemenjek, már harmincz napja.
\par 12 És megmondák Márdekousnak Eszter szavait.
\par 13 És monda Márdokeus visszaüzenve Eszternek: Ne gondold magadban, hogy te a király házában megmenekülhetsz a többi zsidó közül.
\par 14 Mert ha e mostani idõben te hallgatsz, másunnan lészen könnyebbségök és szabadulások a zsidóknak; te pedig és atyád háza elvesztek. És ki tudja, talán e mostani idõért jutottál királyságra?
\par 15 És monda Eszter visszaüzenve Márdokeusnak:
\par 16 Menj el és gyûjts egybe minden zsidót, a ki Susánban találtatik, és bõjtöljetek érettem és ne egyetek és ne igyatok három napig se éjjel se nappal, én is és leányaim így bõjtölünk és ekképen megyek be a királyhoz, noha törvény ellenére; ha azután elveszek, hát elveszek.
\par 17 Elméne azért Márdokeus, és úgy cselekedett mindent, a mint néki Eszter parancsolá.

\chapter{5}

\par 1 Történt pedig harmadnapon, hogy Eszter felöltözött királyiasan, és megállt a király házának belsõ udvarában, a király háza ellenében, és a király üle királyiszékében, a királyi házban, a ház ajtajának átellenében.
\par 2 És lõn, a mint meglátá a király Eszter királynét, hogy áll az udvarban, kegyet talála szemei elõtt, és kinyujtá a király Eszterre az arany pálczát, a mely kezében vala, akkor oda méne Eszter, és megilleté az arany pálcza végét.
\par 3 És monda néki a király: Mi kell néked Eszter királyné? És mi a kérésed? Ha az ország fele is, megadatik néked.
\par 4 És felele Eszter: Ha a királynak tetszik, jõjjön a király és Hámán ma a lakomára, melyet néki készítettem.
\par 5 És monda a király: Gyorsan elõ Hámánt, hogy teljesítse Eszter kívánságát. És elment a király és Hámán a lakomára, a melyet Eszter készített vala.
\par 6 Monda pedig a király Eszternek borivás közben: Mi a te kivánságod? Megadatik néked. És mi a te kérésed? Ha az ország fele is, meglészen!
\par 7 És felele Eszter, és monda: Az én kivánságom és kérésem ez:
\par 8 Ha kegyet találtam a király szemei elõtt, és ha a királynak tetszik megadni kivánságomat és teljesíteni kérésemet: jõjjön el a király és Hámán a lakomára, a melyet készítek nékik, és holnap a király beszéde szerint cselekszem.
\par 9 És kiment Hámán azon a napon vígan és jó kedvvel. De a mint meglátá Hámán Márdokeust a király kapujában, és ez fel nem kelt és nem mozdult meg elõtte, megtelék Hámán haraggal Márdokeus ellen.
\par 10 De megtartóztatá magát Hámán, és haza ment, és elküldvén, magához hivatá barátait és Zérest, az õ feleségét.
\par 11 És elbeszélé nékik Hámán gazdagságának dicsõségét, fiainak sokaságát és mindazt, a mivel felmagasztalta õt a király, és hogy feljebb emelte õt a fejedelmeknél és a király szolgáinál;
\par 12 És monda Hámán: Nem is hívott Eszter királyné mást a királylyal a lakomára, a melyet készített, hanem csak engem, és még holnapra is meghivattattam hozzá a királylyal.
\par 13 De mindez semmit sem ér nékem, valamíg látom a zsidó Márdokeust ülni a király kapujában.
\par 14 És monda néki Zéres, az õ felesége és minden barátja: Csináljanak ötven könyökni magas fát, és reggel mondd meg a királynak, hogy Márdokeust akaszszák reá, és akkor menj a királylyal a lakomára vígan. És tetszett a dolog Hámánnak, és megcsináltatta a fát.

\chapter{6}

\par 1 Azon éjjel kerülte az álom a királyt, és megparancsolá, hogy hozzák elõ a történetek emlékkönyvét, és ezek olvastattak a király elõtt.
\par 2 És írva találák, a mint Márdokeus feljelenté Bigtánát és Térest, a király két udvarmesterét, a küszöb õrzõit, a kik azon voltak, hog rávetik kezöket Ahasvérus királyra.
\par 3 És monda a király: Micsoda tisztességet és méltóságot adtak azért Márdokeusnak? És felelének a király apródjai, az õ szolgái: Semmit sem juttattak néki azért.
\par 4 Akkor monda a király: Ki van az udvarban? (Mert Hámán jöve a királyi ház külsõ udvarába, hogy megmondja a királynak, hogy akasztassa fel Márdokeust a fára, a melyet készittetett néki.)
\par 5 És felelének néki a király apródjai: Ímé Hámán áll az udvarban. És mondja a király: Jõjjön be!
\par 6 És beméne Hámán, és monda néki a király: Mit kell cselekedni azzal a férfiúval, a kinek a király tisztességet kiván? (Hámán pedig gondolá az õ szívében: Kinek akarna a király nagyobb tisztességet tenni, mint én nékem?)
\par 7 És monda Hámán a királynak: A férfiúnak, a kinek a király tisztességet kiván,
\par 8 Hozzanak királyi ruhát a melyben a király jár, és lovat, a melyen a király szokott ülni  és a melynek fejére királyi koronát szoktak tenni.
\par 9 És adják azt a ruhát és lovat a király egyik legelsõ fejedelmének a kezébe, és öltöztessék fel azt a férfiút, a kinek a király tisztességet kiván, és hordozzák õt a lovon a város utczáin, és kiáltsák elõtte: Így cselekesznek a férfiúval, a kinek a király tisztességet kiván.
\par 10 Akkor monda a király Hámánnak: Siess, hozd elõ azt a ruhát és azt a lovat, a mint mondád, és cselekedjél úgy a zsidó Márdokeussal, a ki a király kapujában ül, és semmit el ne hagyj mindabból, a mit szóltál.
\par 11 Elõhozá azért Hámán a ruhát és a lovat, és felöltözteté Márdokeust, és lovon hordozá õt a város utczáján, és kiáltá elõtte: Így cselekesznek a férfiúval, a kinek a király tisztességet kiván.
\par 12 És Márdokeus megtére a király kapujához. Hámán pedig siete házába, búsulva és fejét betakarva.
\par 13 És elbeszélé Hámán Zéresnek, az õ feleségének és minden barátjának mindazt, a mi vele történt. És mondának néki az õ bölcsei és Zéres, az õ felesége: Ha Márdokeus, a ki elõtt kezdtél hanyatlani, a zsidók magvából való: nem bírsz vele, hanem bizony elesel elõtte.
\par 14 És a míg így beszélének vele, eljövének a király udvarmesterei, és siettek Hámánt a lakomára vinni, a melyet készített vala Eszter

\chapter{7}

\par 1 És elméne a király és Hámán a lakomára, Eszter királynéhoz.
\par 2 És monda a király Eszternek másodnapon is, borivás közben: Mi a te kivánságod Eszter királyné? Megadatik. És micsoda a te kérésed? Ha az országnak fele is, meglészen.
\par 3 És felele Eszter királyné, és monda: Ha kegyet találtam szemeid elõtt, oh király! és ha a királynak tetszik, add meg nékem életemet kivánságomra, és nemzetségemet kérésemre.
\par 4 Mert eladattunk, én és az én nemzetségem, hogy kipusztítsanak, megöljenek és megsemmisítsenek minket; ha csak szolgákul, vagy szolgálókul adattunk volna el, akkor hallgatnék, jóllehet az ellenség nem adna kárpótlást a király veszteségéért.
\par 5 És szóla Ahasvérus király, és monda Eszternek, a királynénak: Ki az és hol van az, a kit az õ szíve erre vitt, hogy azt cselekedné?
\par 6 És monda Eszter: Az ellenség és gyûlölõ, ez a gonosz Hámán! Akkor Hámán megrettene a király és királyné elõtt.
\par 7 A király pedig felkele haragjában a borivástól és méne a palotakertbe; Hámán pedig ott maradt, hogy életéért könyörögjön Eszter királynénál; mert látta, hogy a király részérõl elvégeztetett az õ veszte.
\par 8 És mikor a király visszatért a palota kertjébõl a borivás házába, Hámán a kerevetre esék, amelyen Eszter vala. Akkor monda a király: Erõszakot is akar elkövetni a királynén én nálam a házban?! A mint e szó kijött a király szájából, Hámán arczát befedék.
\par 9 És monda Harbona, az udvarmesterek egyike a király elõtt: Ímé a fa is, a melyet készített Hámán Márdokeusnak, a ki a király javára  szólott vala, ott áll Hámán házában, ötven könyöknyi magas. És monda a király: Akasszátok õt magát reá!
\par 10 Felakaszták azért Hámánt a fára, a melyet készített Márdokeusnak, és megszünék a király haragja.

\chapter{8}

\par 1 Azon a napon adá Ahasvérus király Eszter királynénak Hámánnak, a zsidók ellenségének házát, és Márdokeus beméne a király elébe, mert elmondotta Eszter, micsodája õ néki.
\par 2 Lehúzá azért a király az õ gyûrûjét, a melyet Hámántól elvett, és adá azt Márdokeusnak. Eszter pedig Márdokeust tette Hámán házának fejévé.
\par 3 És tovább szóla Eszter a király elõtt, és leborult annak lábaihoz és sírt és könyörgött, hogy semmisítse meg az agági Hámán gonoszságát és tervét, a melyet kigondolt a zsidók ellen.
\par 4 És a király kinyujtá Eszterre az arany pálczát, és Eszter felkelt és megálla a király elõtt,
\par 5 És monda: Ha a királynak tetszik, és ha kegyet találtam õ elõtte, és ha helyes a dolog a király elõtt, és én kedves vagyok a szemei elõtt: írassék meg, hogy vonassanak vissza az agági Hámánnak, Hammedáta fiának tervérõl szóló levelek, a melyeket írt, hogy elveszessék a zsidókat, a kik a király minden tartományában vannak.
\par 6 Mert hogyan tudnám nézni azt a nyomorúságot, mely érné az én nemzetségemet, és hogyan tudnám nézni az én atyámfiainak veszedelmét?
\par 7 Monda pedig Ahasvérus király Eszter királynénak és a zsidó Márdokeusnak: Ímé Hámán házát Eszternek adtam, és õt  felakasztották a fára azért, mert a zsidókra bocsátotta kezét.
\par 8 Ti pedig írjatok a zsidóknak, a mint tetszik néktek a király nevében és pecsételjétek meg a király gyûrûjével; mert az írás, a mely a király nevében van írva és a királynak gyûrûjével van megpecsételve,  vissza nem vonható.
\par 9 Hivattatának azért a király irnokai azonnal, a harmadik hónapban (ez a Siván hónap), annak huszonharmadik napján, és megírák úgy, a miként Márdokeus parancsolá a zsidóknak és a fejedelmeknek, a kormányzóknak és a tartományok fejeinek, Indiától fogva Szerecsenországig  százhuszonhét tartományba; minden tartománynak annak írása szerint, és minden nemzetségnek az õ nyelvén, és a zsidóknak az õ írások és az õ nyelvök szerint.
\par 10 És megírá Ahasvérus király nevében, és megpecsételé a király gyûrûjével. És külde leveleket lovas futárok által, a kik lovagolnak a királyi ménes gyors paripáin,
\par 11 Hogy megengedte a király a zsidóknak, a kik bármely városban vannak, hogy egybegyûljenek és keljenek fel életökért, és hogy kipusztítsák, megöljék és megsemmisítsék minden népnek és tartománynak seregét, a mely õket nyomorgatja, kicsinyeket és nõket, és hogy javaikat elragadják.
\par 12 Egy napon, Ahasvérus királynak minden tartományában, tizenharmadik napján a tizenkettedik hónapnak (ez Adár hónapja).
\par 13 Az írásnak mássa, hogy adassék törvény minden egyes tartományban, meghirdettetett minden népnek, hogy a zsidók készen legyenek azon a napon, hogy megbosszulják magokat ellenségeiken.
\par 14 A futárok a királyi gyors paripákon kimenének sietve és sürgõsen a király parancsolata szerint; és kiadatott a törvény Susán várában is.
\par 15 Márdokeus pedig kiméne a király elõl kék bíbor és fehér királyi ruhában, és nagy arany koronával, és bíborbársony palástban, és Susán városa vígada és örvendeze.
\par 16 A zsidóknak világosság támada, öröm, vigasság és tisztesség.
\par 17 És minden tartományban és minden városban, a hová a király szava és rendelete eljuta, öröme és vigalma lõn a zsidóknak, lakoma és ünnep, és sokan a föld népei közül zsidókká lettek; mert a zsidóktól való  félelem szállta meg õket.

\chapter{9}

\par 1 A tizenkettedik hónapban azért (ez az Adár hónap), annak tizenharmadik napján, a mikor eljött az ideje a király szavának és végzésének, hogy akkor teljesíttessék, azon a napon, a melyen a zsidók ellenségei remélték, hogy hatalmat vesznek rajtok, ellenkezõleg fordult; mert magok a zsidók võnek hatalmat azokon, a kik õket gyûlölék.
\par 2 Egybegyûlének a zsidók városaikban, Ahasvérus király minden tartományában, hogy rávessék kezöket azokra, a kik vesztöket keresték, és senki sem állhatott meg elõttük; mert miattok való félelem szállott minden népre.
\par 3 És a tartományok minden feje, fejedelmek, kormányzók és a király hivatalnokai magasztalták a zsidókat; mert a Márdokeustól való félelem szállott rájok.
\par 4 Mert nagy volt Márdokeus a király házában, és híre elment minden tartományba; mert ez a férfiú, Márdokeus, emelkedett és nagyobbodott.
\par 5 És leverték a zsidók minden ellenségeiket fegyverrel, megölvén és megsemmisítvén azokat, és akaratok szerint cselekedének gyûlölõikkel.
\par 6 És Susán várában megölének és megsemmisítének a zsidók ötszáz férfiút.
\par 7 Parsandátát, Dalpont és Aspatát.
\par 8 Porátát, Adáliát és Aridátát.
\par 9 Parmástát, Arisait, Aridait és Vajzátát.
\par 10 Tíz fiát Hámánnak, Hammedáta fiának, a zsidók ellenségének megölték; de zsákmányra nem tették rá kezöket.
\par 11 Azon a napon megtudá a király a Susán várában megöletteknek számát.
\par 12 És monda a király Eszternek királynénak: Susán várában megöltek a zsidók és megsemmisítettek ötszáz férfiút és Hámán tíz fiát; a király többi tartományaiban mit cselekesznek? Mi a kivánságod? és megadatik néked. És mi még a te kérésed? és meglészen.
\par 13 És monda Eszter: Ha a királynak tetszik, engedje meg holnap is a zsidóknak, a kik Susánban laknak, hogy cselekedjenek a mai parancsolat szerint, és Hámán tíz fiát akaszszák fel a fára.
\par 14 És szóla a király, hogy úgy tegyenek. És kiadaték a parancs Susánban, és Hámán tíz fiát felakasztották.
\par 15 És összegyûltek a zsidók, a kik Susánban valának, Adár havának tizennegyedik napján is, és megölének Susánban háromszáz férfiút; de zsákmányra nem tették rá kezöket.
\par 16 És a többi zsidók is, a kik a király tartományaiban voltak, összegyûlének és feltámadtak életökért, és békében maradtak ellenségeiktõl; megölének pedig gyûlölõikbõl hetvenötezeret; de zsákmányra nem tették rá kezöket.
\par 17 Történt ez az Adár hónapjának tizenharmadik napján, és megnyugovának a tizennegyedik napon, és tették azt vigalom és öröm napjává.
\par 18 De a zsidók, a kik Susánban valának, egybegyûlének azon hónap tizenharmadik és tizennegyedik napján, és megnyugovának azon hónap tizenötödik napján, és azt tették vigalom és öröm napjává.
\par 19 Azért a vidéki zsidók, a kik kerítetlen falvakban laktak, Adár hónapjának tizennegyedik napját tették öröm és vigalom napjává és ünneppé, a melyen ajándékokat küldözgetnek egymásnak.
\par 20 És megírá Márdokeus e dolgokat, és leveleket külde minden zsidónak, a ki Ahasvérus király minden tartományában, közel és távol, vala.
\par 21 Meghagyva nékik, hogy tartsák meg az Adár hónapnak tizennegyedik napját és annak tizenötödik napját évrõl-évre.
\par 22 Mint olyan napokat, a melyeken megnyugovának a zsidók ellenségeiktõl, és mint olyan hónapot, a melyben keserûségök örömre és siralmuk ünnepre fordult; hogy tartsák meg azokat vigalom és öröm napjaiul, és küldjenek adjándékot egymásnak és adományokat a szegényeknek.
\par 23 És felfogadák a zsidók, hogy megcselekszik, a mit kezdettek és a mit Márdokeus íra nékik.
\par 24 Mert az agági Hámán, Hammedátának fia, minden zsidónak ellensége, szándékozott a zsidókat elveszteni, és Púrt, azaz sorsot vetett, hogy õket megrontsa és megsemmisítse;
\par 25 De mikor a király tudomására jutott, megparancsolá levélben, hogy gonosz szándéka, a melyet gondolt a zsidók ellen, forduljon az õ fejére, és felakaszták õt és fiait a fára.
\par 26 Annakokáért elnevezék e napokat Púrimnak a Púr nevétõl. És azért ezen levél minden szava szerint, és a mit láttak erre nézve, és a mi érte õket:
\par 27 Elhatározák és elfogadák a zsidók mind magokra, mind ivadékokra és mindazokra, a kik hozzájok csatlakoznak, örök idõkre, hogy megtartják e két napot,  írásuk és határozatuk szerint minden esztendõben.
\par 28 És ezen napok emlékezetben lesznek és megülik azokat nemzedékrõl-nemzedékre, családról-családra, tartományról-tartományra és városról-városra. És ezek a Púrim napjai el nem múlnak a zsidók között, és emlékök ki nem vész ivadékaik közül.
\par 29 Ira pedig Eszter királyné, Abihail leánya és a zsidó Márdokeus egész hatalommal, hogy megerõsítsék e második levelet a Púrimról.
\par 30 És külde leveleket minden zsidónak Ahasvérus országának százhuszonhét tartományába, békességes és igazságos szavakkal;
\par 31 Hogy megerõsíttessenek ezek a Púrim napjai, meghatározott idejökben, a miképen meghagyta nékik a zsidó Márdokeus és Eszter királyné, és a mint elrendelték  magokra és ivadékaikra a bõjtölés és jajkiáltás dolgát.
\par 32 És Eszter beszéde megerõsíté ezt a Púrim történetét, és könyvbe iraték.

\chapter{10}

\par 1 És Ahasvérus király adót vettete a földre és a tenger szigeteire.
\par 2 Az õ erejének és hatalmának minden cselekedete pedig és Márdokeus nagy méltóságának története, a mellyel felmagasztalá õt a király, avagy nincsenek-é megírva Média és Persia királyainak évkönyveiben,
\par 3 Mert a zsidó Márdokeus második volt Ahasvérus király után, nagy a zsidók között, és kedves az õ atyjafiai sokasága elõtt, a ki javát keresé népének, és békességet szerze minden ivadékának.


\end{document}