\begin{document}

\title{Példabeszédek könyve}


\chapter{1}

\par 1 Salamonnak, Dávid fiának, Izráel királyának példabeszédei,
\par 2 Bölcseség és erkölcsnek tanulására, értelmes beszédek megértésére;
\par 3 Okos fenyítéknek, igazságnak és ítéletnek és becsületességnek megnyerésére;
\par 4 Együgyûeknek eszesség, gyermeknek tudomány és meggondolás adására.
\par 5 Hallja a bölcs és öregbítse az õ tanulságát, és az értelmes szerezzen érett tanácsokat.
\par 6 Példabeszédnek és példázatnak, bölcsek beszédeinek és találós meséinek megértésére.
\par 7 Az Úrnak félelme feje a bölcseségnek; a bölcseséget és erkölcsi tanítást a bolondok megútálják.
\par 8 Hallgasd, fiam, a te atyádnak erkölcsi tanítását, és a te anyádnak oktatását el ne hagyd.
\par 9 Mert kedves ékesség lesz a te fejednek, és aranyláncz a te nyakadra.
\par 10 Fiam, ha a bûnösök el akarnak csábítani téged: ne fogadd beszédöket.
\par 11 Ha azt mondják: jere mi velünk, leselkedjünk vér után, rejtezzünk el az ártatlan ellen ok nélkül;
\par 12 Nyeljük el azokat, mint a sír elevenen, és egészen, mint a kik mélységbe szállottak;
\par 13 Minden drága marhát nyerünk, megtöltjük a mi házainkat zsákmánnyal:
\par 14 Sorsodat vesd közénk; egy erszényünk legyen mindnyájunknak:
\par 15 Fiam, ne járj egy úton ezekkel, tartóztasd meg lábaidat ösvényüktõl;
\par 16 Mert lábaik a gonoszra futnak, és sietnek a vérnek ontására.
\par 17 Mert hiába vetik ki a hálót minden szárnyas állat szemei elõtt:
\par 18 Ezek mégis vérök árán is ólálkodnak, lelkök árán is leselkednek;
\par 19 Ilyen az útja minden kapzsi embernek: gazdájának életét veszi el.
\par 20 A bölcseség künn szerül-szerte kiált; az utczákon zengedezteti az õ szavát.
\par 21 Lármás utczafõkön kiált a kapuk bemenetelin, a városban szólja az õ beszédit.
\par 22 Míglen szeretitek, oh ti együgyûek az együgyûséget, és gyönyörködnek a csúfolók csúfolásban, és gyûlölik a balgatagok a tudományt?!
\par 23 Térjetek az én dorgálásomhoz; ímé közlöm veletek az én lelkemet, tudtotokra adom az én beszédimet néktek.
\par 24 Mivelhogy hívtalak titeket, és vonakodtatok, kiterjesztém az én kezemet, és senki eszébe nem vette;
\par 25 És elhagytátok minden én tanácsomat, és az én feddésemmel nem gondoltatok:
\par 26 Én is a ti nyomorúságtokon nevetek, megcsúfollak, mikor eljõ az, a mitõl féltek.
\par 27 Mikor eljõ, mint a vihar, az, a mitõl féltek, és a ti nyomorúságtok, mint a forgószél elközelget: mikor eljõ ti reátok a nyomorgatás és a szorongatás.
\par 28 Akkor segítségül hívnak engem, de nem hallgatom meg: keresnek engem, de meg nem találnak.
\par 29 Azért hogy gyûlölték a bölcseséget, és az Úrnak félelmét nem választották.
\par 30 Nem engedtek az én tanácsomnak; megvetették minden én feddésemet.
\par 31 Esznek azért az õ útjoknak gyümölcsébõl, és az õ tanácsokból megelégednek.
\par 32 Mert az együgyûeknek pártossága megöli õket, és a balgatagoknak szerencséje elveszti õket.
\par 33 A ki pedig hallgat engem, lakozik bátorságosan, és csendes lesz a gonosznak félelmétõl.

\chapter{2}

\par 1 Fiam! ha beveszed az én beszédimet, és az én parancsolatimat elrejted magadnál,
\par 2 Ha figyelmeztetvén a bölcsességre a te füleidet, hajtod a te elmédet az értelemre,
\par 3 Igen, ha a bölcseségért kiáltasz, és az értelemért a te szódat felemeled,
\par 4 Ha keresed azt, mint az ezüstöt, és mint a kincseket kutatod azt:
\par 5 Akkor megérted az Úrnak félelmét, és az Istennek ismeretére jutsz.
\par 6 Mert az Úr ád bölcseséget, az õ szájából tudomány és értelem származik.
\par 7 Az igazaknak valóságos jót rejteget, paizst a tökéletesen járóknak,
\par 8 Hogy megõrizze az igazságnak útait, és kegyeseinek útját megtartja.
\par 9 Akkor megérted az igazságot, és törvényt és becsületességet, és minden jó útat.
\par 10 Mert bölcseség megy a te elmédbe, és a tudomány a te lelkedben gyönyörûséges lesz.
\par 11 Meggondolás õrködik feletted, értelem õriz téged,
\par 12 Hogy megszabadítson téged a gonosznak útától, és a gonoszságszóló férfiútól;
\par 13 A kik elhagyják az igazságnak útát, hogy járjanak a setétségnek útain.
\par 14 A kik örülnek gonoszt cselekedvén, vígadnak a gonosz álnokságokon.
\par 15 A kiknek ösvényeik görbék, és a kik az õ útaikban gonoszok.
\par 16 Hogy megszabadítson téged a nem hozzád tartozó asszonytól, az idegentõl, a ki az õ beszédével  hizelkedik,
\par 17 A ki elhagyja az õ ifjúságának férjét, és az õ Istenének szövetségérõl elfelejtkezik;
\par 18 Mert a halálra hanyatlik az õ háza, és az õ ösvényei az élet nélkül valókhoz.
\par 19 Valakik mennek ahhoz, nem térnek meg, sem meg nem nyerhetik az életnek útait.
\par 20 Hogy járj a jóknak útjokon, és az igazaknak ösvényeit kövessed.
\par 21 Mert az igazak lakják a földet, és a tökéletesek maradnak meg rajta.
\par 22 A gonoszok pedig a földrõl kivágattatnak, és a hitetlenül cselekedõk kiszaggattatnak abból.

\chapter{3}

\par 1 Fiam! az én tanításomról el ne felejtkezzél, és az én parancsolatimat megõrizze a te elméd;
\par 2 Mert napoknak hosszú voltát, és sok esztendõs életet, és békességet hoznak néked bõven.
\par 3 Az irgalmasság és igazság ne hagyjanak el téged: kösd azokat  a te nyakadra, írd be azokat a te szívednek táblájára;
\par 4 Így nyersz kedvességet és jó értelmet Istennek és embernek szemei elõtt.
\par 5 Bizodalmad legyen az Úrban teljes elmédbõl; a magad értelmére pedig ne támaszkodjál.
\par 6 Minden te útaidban megismered õt; akkor õ igazgatja a te útaidat.
\par 7 Ne légy bölcs a te magad  ítélete szerint; féld az Urat, és távozzál el a gonosztól.
\par 8 Egészség lesz ez a te testednek, és megújulás a te csontaidnak.
\par 9 Tiszteld az Urat a te marhádból, a te egész jövedelmed zsengéjébõl.
\par 10 Eképen megtelnek a te csûreid elégséggel, és musttal áradnak el sajtód válúi.
\par 11 Az Úrnak fenyítését fiam, ne útáld meg, se meg ne únd az õ dorgálását.
\par 12 Mert a kit szeret az Úr, megdorgálja, és pedig mint az atya az õ fiát a kit kedvel.
\par 13 Boldog ember, a ki megnyerte a bölcseséget, és az ember, a ki értelmet szerez.
\par 14 Mert jobb ennek megszerzése az ezüstnek megszerzésénél, és a kiásott aranynál ennek jövedelme.
\par 15 Drágább a fényes kárbunkulusoknál, és minden te gyönyörûségeid nem hasonlíthatók hozzá.
\par 16 Napoknak hosszúsága van jobbjában, baljában gazdagság és tisztesség.
\par 17 Az õ útai gyönyörûséges útak, és minden ösvényei: békesség.
\par 18 Életnek fája ez azoknak, a kik megragadják, és a kik  megtartják boldogok!
\par 19 Az Úr bölcseséggel fundálta a földet, erõsítette az eget értelemmel.
\par 20 Az õ tudománya által fakadtak ki a mélységbõl a vizek, és a felhõk csepegnek harmatot,
\par 21 Fiam, ne távozzanak el a te szemeidtõl, õrizd meg az igaz bölcseséget, és a meggondolást!
\par 22 És lesznek ezek élet a te lelkednek, és kedvesség a te nyakadnak.
\par 23 Akkor bátorsággal járod a te útadat, és a te lábadat meg nem ütöd.
\par 24 Mikor lefekszel, nem rettegsz; hanem lefekszel és gyönyörûséges lesz a te álmod.
\par 25 Ne félj a hirtelen való félelemtõl, és a gonoszok pusztításától, ha eljõ;
\par 26 Mert az Úr lesz a te bizodalmad és megõrzi a te lábadat a fogságtól.
\par 27 Ne fogd meg a jótéteményt azoktól, a kiket illet, ha hatalmadban van annak megcselekedése.
\par 28 Ne mondd a te felebarátodnak: menj el, azután térj meg, és holnap adok; holott nálad van, a mit kér.
\par 29 Ne forralj a te felebarátod ellen gonoszt, holott õ együtt ül bátorságosan te veled.
\par 30 Ne háborogj egy emberrel is ok nélkül, ha nem illetett gonoszszal téged.
\par 31 Ne irígykedjél az erõszakos emberre, és néki semmi útát ne válaszd.
\par 32 Mert útálja az Úr az engedetlent; és az igazakkal van az õ titka.
\par 33 Az Úrnak átka van a gonosznak házán; de az igazaknak lakhelyét megáldja.
\par 34 Ha kik csúfolók, õ megcsúfolja azokat; a szelídeknek pedig ád kedvességet.
\par 35 A bölcsek tisztességet örökölnek; a bolondok pedig gyalázatot aratnak.

\chapter{4}

\par 1 Halljátok meg, fiaim, atyátok erkölcsi tanítását, és figyelmezzetek az értelemnek megtudására.
\par 2 Mert jó tanulságot adok néktek; az én tudományomat el ne hagyjátok.
\par 3 Mert én atyámnak fia voltam, gyenge és egyetlenegy az én anyám elõtt.
\par 4 Tehát tanított engem, és mondá nékem: tartsa meg az én beszédemet a te elméd, hogy megtartván az én parancsolatimat, élj;
\par 5 Szerezz bölcseséget, szerezz eszességet; ne felejtkezzél el, se el ne hajolj az én számnak beszéditõl.
\par 6 Ne hagyd el azt, és megtart téged; szeresd azt, és megõriz téged.
\par 7 A bölcseség kezdete ez: szerezz bölcseséget, és minden keresményedbõl szerezz értelmet.
\par 8 Magasztald fel azt, és felmagasztal téged; tiszteltté tesz téged, ha hozzád öleled azt.
\par 9 Ád a te fejednek kedvességnek koszorúját; igen szép ékes koronát ád néked.
\par 10 Hallgasd, fiam, és vedd be az én beszédimet; így sokasulnak meg néked a te életednek esztendei.
\par 11 Bölcseségnek útára tanítottalak téged, vezettelek téged az igazságnak ösvényin.
\par 12 Mikor jársz, semmi nem szorítja meg a te járásodat; és ha futsz, nem ütközöl meg.
\par 13 Ragaszkodjál az erkölcsi tanításhoz, ne hagyd el; õrizd meg azt, mert az a te életed.
\par 14 A hitetleneknek útjára ne menj, se ne járj a gonoszok ösvényén.
\par 15 Hagyd el, át ne menj rajta; térj el tõle, és menj tovább.
\par 16 Mert nem alhatnak azok, ha gonoszt nem cselekesznek: és kimegy szemükbõl az álom, ha mást romlásra nem juttatnak.
\par 17 Mert az istentelenségnek étkét eszik, és az erõszaktételnek borát iszszák.
\par 18 Az igazak ösvénye pedig olyan, mint a hajnal világossága, mely minél tovább halad, annál világosabb lesz, a teljes délig.
\par 19 Az istentelenek útja pedig olyan, mint a homály, nem tudják miben ütköznek meg.
\par 20 Fiam, az én szavaimra figyelmezz, az én beszédimre hajtsad füledet.
\par 21 Ne távozzanak el a te szemeidtõl, tartsd meg ezeket a te elmédben.
\par 22 Mert életök ezek azoknak, a kik megnyerik, és egész testöknek egészség.
\par 23 Minden féltett dolognál jobban õrizd meg szívedet, mert abból indul ki minden élet.
\par 24 Vesd el tõled a száj hamisságát, és az ajkak álnokságát távoztasd el magadtól.
\par 25 A te szemeid elõre nézzenek, és szemöldökid egyenest magad elé irányuljanak.
\par 26 Egyengesd el lábaid ösvényit, s minden te útaid állhatatosak legyenek.
\par 27 Ne térj jobbra, se balra, fordítsd el a te lábadat a gonosztól.

\chapter{5}

\par 1 Fiam! az én bölcseségemre figyelmezz, az én értelmemre hajtsd a te füledet,
\par 2 Hogy megtartsd a meggondolást, és a tudományt a te ajakid megõrizzék.
\par 3 Mert színmézet csepeg az idegen asszony ajka, és símább az olajnál az õ ínye.
\par 4 De annak vége keserû, mint az üröm, éles, mint a kétélû tõr.
\par 5 Az õ lábai a halálra mennek, az õ léptei a sírba törekszenek.
\par 6 Az életnek útját hogy ne követhesse, ösvényei változókká lettek, a nélkül, hogy õ eszébe venné.
\par 7 Most azért, fiaim, hallgassatok engem, és ne távozzatok el számnak beszéditõl!
\par 8 Távoztasd el attól útadat, és ne közelgess házának ajtajához,
\par 9 Hogy másoknak ne add a te ékességedet, és esztendeidet a kegyetlennek;
\par 10 Hogy ne az idegenek teljenek be a te marháiddal, és a te keresményed más házába ne jusson.
\par 11 Hogy nyögnöd kelljen életed végén, a mikor megemésztetik a te húsod és a te tested,
\par 12 És azt kelljen mondanod: miképen gyûlöltem az erkölcsi tanítást, és a fenyítéket útálta az én elmém,
\par 13 És nem hallgattam az én vezetõim szavát, és az én tanítóimhoz nem hajtottam fülemet!
\par 14 Kevés híja volt, hogy minden gonoszságba nem merültem a gyülekezetnek és községnek közepette!
\par 15 Igyál vizet a te kútadból, és a te forrásod közepibõl folyóvizet.
\par 16 Kifolynak-é a te forrásid, az utczákra a te vized folyásai?
\par 17 Egyedül tied legyenek, és nem az idegenekéi veled.
\par 18 Legyen a te forrásod áldott, és örvendezz a te ifjúságod feleségének.
\par 19 A szerelmes szarvas, és kedves zerge; az õ emlõi elégítsenek meg téged minden idõben, az õ szerelmében gyönyörködjél szüntelen.
\par 20 És miért bujdosnál, fiam, az idegen után, és ölelnéd keblét az idegennek?
\par 21 Mert az Úrnak szemei elõtt vannak mindenkinek útai, és minden ösvényeit õ  rendeli.
\par 22 A maga álnokságai fogják meg az istentelent, és a saját bûnének köteleivel kötöztetik meg.
\par 23 Õ meghal fenyíték híján, és bolondságának sokasága miatt támolyog.

\chapter{6}

\par 1 Fiam! ha kezes lettél a te barátodért, és kezedet adván, kötelezted magadat másért:
\par 2 Szádnak beszédei által estél tõrbe, megfogattattál a te szádnak beszédivel.
\par 3 Ezt míveld azért fiam, és mentsd ki magadat, mert a te felebarátodnak kezébe jutottál; eredj, alázd meg magadat, és kényszerítsd felebarátodat.
\par 4 Még álmot se engedj szemeidnek, se szunnyadást szemöldökidnek,
\par 5 Szabadítsd ki magadat, mint a zerge a vadász kezébõl, és mint a madár a madarásznak kezébõl.
\par 6 Eredj a hangyához, te rest, nézd meg az õ útait, és légy bölcs!
\par 7 A kinek nincs vezére, igazgatója, vagy ura,
\par 8 Nyárban szerzi meg az õ kenyerét, aratáskor gyûjti eledelét.
\par 9 Oh te rest, meddig fekszel? mikor kelsz fel a te álmodból?
\par 10 Még egy kis álom, még egy kis szunnyadás, még egy kis kéz-összefonás, hogy pihenjek;
\par 11 Így jõ el, mint az útonjáró, a te szegénységed, és a te szûkölködésed, mint a paizsos férfiú!
\par 12 Haszontalan ember, hamis férfiú, a ki álnok szájjal jár,
\par 13 A ki hunyorgat szemeivel; lábaival is szól, és ujjaival jelt ád.
\par 14 Álnokság van az õ szívében, gonoszt forral minden idõben, háborúságot indít.
\par 15 Annakokáért hirtelen eljõ az õ nyomorúsága, gyorsan megrontatik, s nem lesz gyógyulása.
\par 16 E hat dolgot gyûlöli az Úr, és hét dolog útálat az õ lelkének:
\par 17 A kevély szemek, a hazug nyelv, és az ártatlan vért ontó kezek,
\par 18 Az álnok gondolatokat forraló elme, a gonoszra sietséggel futó lábak,
\par 19 A hazugságlehelõ hamis tanú, és a ki szerez háborúságokat az atyafiak között!
\par 20 Õrizd meg, fiam, atyád parancsolatját, és anyád tanítását el ne hagyd.
\par 21 Kösd azokat szívedre mindenkor, fûzd a nyakadba.
\par 22 Valahová mégysz, vezérel téged, mikor aluszol, õriz téged, mikor felserkensz, beszélget te veled.
\par 23 Mert szövétnek a parancsolat, és a tudomány világosság és életnek úta a tanító-feddések.
\par 24 Hogy a gonosz asszonytól téged megõrizzenek, az idegen asszony nyelvének hizelkedésétõl.
\par 25 Ne kivánd az õ szépségét szivedben, és meg ne fogjon téged szemöldökeivel;
\par 26 Mert a parázna asszony miatt jut az ember egy darab kenyérre, és más férfi felesége drága életet vadász!
\par 27 Vehet-é valaki tüzet az õ kebelébe, hogy ruhái meg ne égnének?
\par 28 Vagy járhat-é valaki elevenszénen, hogy lábai meg ne égnének?
\par 29 Így van, valaki bemegy felebarátjának feleségéhez, nem marad büntetlen, valaki illeti azt!
\par 30 Nem útálják meg a lopót, ha lop az õ kivánságának betöltésére, mikor éhezik;
\par 31 És ha rajta kapatik, hétannyit kell adnia, az õ házának minden marháját érette adhatja;
\par 32 A ki pedig asszonynyal paráználkodik, bolond; a ki magát el akarja veszteni, az cselekszi ezt!
\par 33 Vereséget és gyalázatot nyer, és az õ gyalázatja el nem töröltetik.
\par 34 Mert a féltékenység a férfiú haragja, és nem cselekszik kegyelmességgel a bosszúállásnak napján.
\par 35 Nem gondol semmi váltsággal, nem nyugszik meg rajta, még ha nagy sok ajándékot adsz is néki.

\chapter{7}

\par 1 Fiam, tartsd meg az én beszédeimet, és az én parancsolataimat rejtsd el magadnál.
\par 2 Az én parancsolatimat tartsd meg, és élsz; és az én tanításomat mint a szemed fényét.
\par 3 Kösd azokat ujjaidra, írd fel azokat szíved táblájára.
\par 4 Mondd ezt a bölcseségnek: Én néném vagy te; és az eszességet ismerõsödnek nevezd,
\par 5 Hogy megõrizzen téged a nem hozzád tartozó asszonytól, és az õ beszédivel hizelkedõ idegentõl.
\par 6 Mert házam ablakán, a rács mögül néztem,
\par 7 És láték a bolondok között, eszembe vevék a fiak között egy bolond ifjat,
\par 8 A ki az utczán jár, annak szeglete mellett, a házához menõ úton lépeget,
\par 9 Alkonyatkor, nap estjén, és setét éjfélben.
\par 10 És ímé, egy asszony eleibe jõ, paráznának öltözetében, álnok az õ elméjében.
\par 11 Mely csélcsap és vakmerõ, a kinek házában nem maradhatnak meg az õ lábai.
\par 12 Néha az utczán, néha a tereken van, és minden szegletnél leselkedik.
\par 13 És megragadá õt és megcsókolá õt, és szemtelenségre vetemedvén, monda néki:
\par 14 Hálaáldozattal tartoztam, ma adtam meg fogadásimat.
\par 15 Azért jövék ki elõdbe, szorgalmatosan keresni a te orczádat, és reád találtam!
\par 16 Paplanokkal megvetettem nyoszolyámat, égyiptomi szövésû szõnyegekkel.
\par 17 Beillatoztam ágyamat mirhával, áloessel és fahéjjal.
\par 18 No foglaljuk magunkat bõségesen mind virradtig a szeretetben; vígadjunk szerelmeskedésekkel.
\par 19 Mert nincs otthon a férjem, elment messze útra.
\par 20 Egy erszény pénzt võn kezéhez; holdtöltére jõ haza.
\par 21 És elhiteté õt az õ mesterkedéseinek sokaságával, ajkainak hizelkedésével elragadá õt.
\par 22 Utána megy; mint az ökör a vágóhídra, és mint a bolond, egyszer csak fenyítõ békóba;
\par 23 Mígnem átjárja a nyíl az õ máját. Miképen siet a madár a tõrre, és nem tudja, hogy az az õ élete ellen van.
\par 24 Annakokáért most, fiaim, hallgassatok engem, és figyelmezzetek az én számnak beszédeire.
\par 25 Ne hajoljon annak útaira a te elméd, és ne tévelyegj annak ösvényin.
\par 26 Mert sok sebesültet elejtett, és sokan vannak, a kik attól megölettek.
\par 27 Sírba vívõ út az õ háza, a mely levisz a halálnak hajlékába.

\chapter{8}

\par 1 Avagy a bölcsesség nem kiált-é, és az értelem nem bocsátja-é ki az õ szavát?
\par 2 A magas helyeknek tetein az úton, sok ösvény összetalálkozásánál áll meg.
\par 3 A kapuk mellett a városnak bemenetelin, az ajtók bemenetelinél zeng.
\par 4 Tinéktek kiáltok, férfiak; és az én szóm az emberek fiaihoz van!
\par 5 Értsétek meg ti együgyûek az eszességet, és ti balgatagok vegyétek eszetekbe az értelmet.
\par 6 Halljátok meg; mert jeles dolgokat szólok és az én számnak felnyitása igazság.
\par 7 Mert igazságot mond ki az én ínyem, és útálat az én ajkaimnak a gonoszság.
\par 8 Igaz én számnak minden beszéde, semmi sincs ezekben hamis, vagy elfordult dolog.
\par 9 Mind egyenesek az értelmesnek, és igazak azoknak, kik megnyerték a tudományt.
\par 10 Vegyétek az én tanításomat, és nem a pénzt; és a tudományt inkább, mint a választott aranyat.
\par 11 Mert jobb a bölcseség a drágagyöngyöknél; és semmi gyönyörûségek ehhez egyenlõk nem lehetnek.
\par 12 Én bölcsesség lakozom az eszességben, és a megfontolás tudományát megnyerem.
\par 13 Az Úrnak félelme a gonosznak gyûlölése; a kevélységet és felfuvalkodást és a gonosz útat, és az álnok szájat gyûlölöm.
\par 14 Enyém a tanács és a valóság, én vagyok az eszesség, enyém az erõ.
\par 15 Én általam uralkodnak a királyok, és az uralkodók végeznek igazságot.
\par 16 Én általam viselnek a fejedelmek fejedelemséget, és a nemesek, a földnek minden birái.
\par 17 Én az engem szeretõket szeretem, és a kik engem szorgalmasan keresnek, megtalálnak.
\par 18 Gazdagság és tisztesség van nálam, megmaradandó jó és igazság.
\par 19 Jobb az én gyümölcsöm a tiszta aranynál és színaranynál, és az én hasznom a válogatott ezüstnél.
\par 20 Az igazságnak útán járok, és az igazság ösvényének közepén.
\par 21 Hogy az engem szeretõknek valami valóságost adjak örökségül, és erszényeiket megtöltsem.
\par 22 Az Úr az õ útának kezdetéül szerzett engem; az õ munkái elõtt régen.
\par 23 Örök idõktõl fogva felkenettem, kezdettõl, a föld kezdetétõl fogva.
\par 24 Még mikor semmi mélységek nem voltak, születtem vala; még mikor semmmi források, vízzel teljesek nem voltak.
\par 25 Minekelõtte a hegyek leülepedtek volna, a halmoknak elõtte születtem.
\par 26 Mikor még nem csinálta vala a földet és a mezõket, és a világ porának kezdetét.
\par 27 Mikor készíté az eget, ott valék; mikor felveté a mélységek színén a kerekséget;
\par 28 Mikor megerõsíté a felhõket ott fenn, mikor erõsekké lõnek a mélységeknek forrásai;
\par 29 Mikor felveté a tengernek határit, hogy a vizek át ne hágják az õ parancsolatját, mikor megállapítá e földnek fundamentomait:
\par 30 Mellette valék mint kézmíves, és gyönyörûsége valék mindennap, játszva õ elõtte minden idõben.
\par 31 Játszva az õ földének kerekségén, és gyönyörûségemet lelve az emberek fiaiban.
\par 32 És most fiaim, hallgassatok engemet, és boldogok, a kik az én útaimat megtartják.
\par 33 Hallgassátok a tudományt és legyetek bölcsek, és magatokat el ne vonjátok!
\par 34 Boldog ember, a ki hallgat engem, az  én ajtóm elõtt virrasztván minden nap, az én ajtóim félfáit õrizvén.
\par 35 Mert a ki megnyer engem, nyert életet, és szerzett az Úrtól jóakaratot.
\par 36 De a ki vétkezik ellenem, erõszakot cselekszik az õ lelkén; minden, valaki engem gyûlöl, szereti a halált!

\chapter{9}

\par 1 Bölcseség megépítette az õ házát, annak hét oszlopát kivágván.
\par 2 Megölte vágnivalóit, kitöltötte borát, asztalát is elkészítette.
\par 3 Elbocsátá az õ leányit, hivogat a város magas helyeinek tetein.
\par 4 Ki tudatlan? térjen ide; az értelem nélkül valónak ezt mondja:
\par 5 Jõjjetek, éljetek az én étkemmel, és igyatok a borból, melyet töltöttem.
\par 6 Hagyjátok el a bolondokat, hogy éljetek, járjatok az eszességnek útán.
\par 7 A ki tanítja a csúfolót, nyer magának szidalmat: és a ki feddi a latrot, szégyenére lesz.
\par 8 Ne fedd meg a csúfolót, hogy ne gyûlöljön téged; fedd meg a bölcset, és szeret téged.
\par 9 Adj a bölcsnek, és még bölcsebb lesz; tanítsd az igazat, és öregbíti a tanulságot.
\par 10 A bölcseségnek kezdete az Úrnak félelme; és a Szentnek ismerete az eszesség.
\par 11 Mert én általam sokasulnak meg a te napjaid, és meghosszabbítják néked életednek esztendeit.
\par 12 Ha bölcs vagy, bölcs vagy te magadnak; ha pedig csúfoló vagy, magad vallod kárát.
\par 13 Balgaság asszony fecsegõ, bolond és semmit nem tud.
\par 14 És leült az õ házának ajtajába, székre a városnak magas helyein,
\par 15 Hogy hívja az útonjárókat, a kik egyenesen mennek útjokon.
\par 16 Ki együgyû? térjen ide, és valaki esztelen, annak ezt mondja:
\par 17 A lopott víz édes, és a titkon való étel gyönyörûséges!
\par 18 És az nem tudja, hogy ott élet nélkül valók vannak; és a pokol mélyébe esnek az õ hivatalosai!

\chapter{10}

\par 1 Salamon bölcs mondásai. A bölcs fiú örvendezteti az õ atyját; a bolond fiú pedig szomorúsága az õ anyjának.
\par 2 Nem használnak a gonoszság kincsei; az igazság pedig  megszabadít a halálból.
\par 3 Az Úr nem hagyja éhezni az igaznak lelkét; az istenteleneknek kivánságát pedig elveti.
\par 4 Szegénynyé lesz, a ki cselekszik rest kézzel; a gyors munkások keze pedig meggazdagít.
\par 5 Gyûjt nyárban az eszes fiú; álomba merül az aratás idején a megszégyenítõ fiú.
\par 6 Áldások vannak az igaznak fején; az istentelenek szája pedig erõszaktételt fed be.
\par 7 Az igaznak emlékezete áldott; a hamisaknak neve pedig megrothad.
\par 8 A bölcs elméjû beveszi a parancsolatokat; a bolond ajkú pedig elveszti magát.
\par 9 A ki tökéletességben jár, bátorsággal jár; a ki pedig elferdíti az õ útát, kiismertetik.
\par 10 A ki szemmel hunyorgat, bántást szerez; és a bolond ajkú elesik.
\par 11 Életnek kútfeje az igaznak szája; az istenteleneknek szája pedig erõszaktételt fedez el.
\par 12 A gyûlölség szerez versengést; minden vétket pedig elfedez a szeretet.
\par 13 Az eszesek ajkain bölcseség találtatik; a vesszõ pedig a bolond hátának való.
\par 14 A bölcsek tudományt rejtegetnek; a bolondnak szája pedig közeli romlás.
\par 15 A gazdagnak marhája az õ megerõsített városa; szûkölködõknek romlása az õ szegénységök.
\par 16 Az igaznak keresménye életre, az istentelennek jövedelme bûnre van.
\par 17 A bölcseség megõrizõnek útja életre van; a fenyítéket elhagyó pedig tévelyeg.
\par 18 A ki elfedezi a gyûlölséget, hazug ajkú az; és a ki szól gyalázatot, bolond az.
\par 19 A sok beszédben elmaradhatatlan a vétek; a ki pedig megtartóztatja ajkait, az értelmes.
\par 20 Választott ezüst az igaznak nyelve; a gonosznak elméje kevés érõ.
\par 21 Az igaznak ajkai sokakat legeltetnek; a bolondok pedig esztelenségökben halnak meg.
\par 22 Az Úrnak áldása, az gazdagít meg, és azzal semmi nem szerez bántást.
\par 23 Miképen játék a bolondnak bûnt cselekedni, azonképen az eszes férfiúnak bölcsen cselekedni.
\par 24 A mitõl retteg az istentelen, az esik õ rajta; a mit pedig kivánnak az igazak, meg lesz.
\par 25 A mint a forgószél ráfuvall, már oda van az istentelen; az igaznak pedig örökké való fundamentoma van.
\par 26 Minémû az eczet a fogaknak és a füst a szemeknek, olyan a rest azoknak, a kik azt elküldötték.
\par 27 Az Úrnak félelme hosszabbítja meg a napokat; az istenteleneknek pedig esztendeik  megrövidülnek.
\par 28 Az igazaknak reménysége öröm; az istenteleneknek várakozása pedig elvész.
\par 29 Erõsség a tökéletesnek az Úrnak úta: de romlás a hamisság cselekedõinek.
\par 30 Az igaz soha meg nem mozdul; de az istentelenek nem lakják a földet.
\par 31 Az igaznak szája bõségesen szól bölcsességet; a gonoszság nyelve pedig kivágatik.
\par 32 Az igaznak ajkai azt tudják, a mi kedves; az istenteleneknek szája pedig a gonoszságot.

\chapter{11}

\par 1 Az álnok font útálatos az Úrnál; az igaz mérték pedig kedves néki.
\par 2 Kevélység jõ: gyalázat jõ; az alázatosoknál pedig bölcseség van.
\par 3 Az igazakat tökéletességök vezeti; de a hitetleneket gonoszságuk elpusztítja.
\par 4 Nem használ a vagyon a haragnak idején; az igazság pedig kiragad a halálból.
\par 5 A tökéletesnek igazsága igazgatja az õ útát; de önnön istentelenségében esik el az istentelen.
\par 6 Az igazaknak igazságok megszabadítja õket; de az õ kivánságokban fogatnak meg  a hitetlenek.
\par 7 Mikor meghal az istentelen ember, elvész az õ reménysége; a bûnösök várakozása is elvész.
\par 8 Az igaz a nyomorúságból megszabadul; az istentelen õ helyette beesik abba.
\par 9 Szájával rontja meg a képmutató felebarátját; de az igazak a tudomány által megszabadulnak.
\par 10 Az igazak javán örül a város; és mikor elvesznek az istentelenek, örvendezés van.
\par 11 Az igazaknak áldása által emelkedik a város; az istentelenek szája által pedig megromol.
\par 12 Megútálja felebarátját a bolond; az eszes férfiú pedig hallgat.
\par 13 A rágalmazó megjelenti a titkot; de a hûséges lelkû elfedezi a dolgot.
\par 14 A hol nincs vezetés, elvész a nép; a megmaradás pedig a sok tanácsos által van.
\par 15 Teljességgel megrontatik, a ki kezes lesz idegenért; a ki pedig gyûlöli a kezességet, bátorságos lesz.
\par 16 A kedves asszony megtartja a tiszteletet, a hatalmaskodók pedig megtartják a gazdagságot.
\par 17 Õ magával tesz jól a kegyes férfiú; a kegyetlen pedig öntestének okoz fájdalmat.
\par 18 Az istentelen munkál álnok keresményt; az igazságszerzõnek pedig jutalma valóságos.
\par 19 A ki õszinte az igazságban, az életére -, a ki pedig a gonoszt követi, az vesztére míveli azt.
\par 20 Útálatosok az Úrnál az álnok szívûek; kedvesek pedig õ nála, a kik az õ útjokban tökéletesek.
\par 21 Kézadással erõsítem, hogy nem marad büntetlen a gonosz; az igazaknak pedig magva megszabadul.
\par 22 Mint a disznó orrában az aranyperecz, olyan a szép asszony, a kinek nincs okossága.
\par 23 Az igazaknak kivánsága csak jó, az istentelenek várakozása pedig harag.
\par 24 Van olyan, a ki bõven adakozik, és annál inkább gazdagodik; és a ki megtartóztatja a járandóságot, de ugyan szûkölködik.
\par 25 A mással jóltevõ ember megkövéredik; és a ki mást felüdít, maga is üdül.
\par 26 A ki búzáját visszatartja, átkozza azt a nép; annak  fején pedig, a ki eladja, áldás van.
\par 27 A ki jóra igyekezik, jóakaratot szerez: a ki pedig gonoszt keres, õ magára jõ az.
\par 28 A ki bízik az õ gazdagságában, elesik; de mint a fa ága, az  igazak kivirágoznak.
\par 29 A ki megháborítja az õ házát, annak öröksége szél lesz; és a bolond szolgája a bölcs elméjûnek.
\par 30 Az igaznak gyümölcse életnek fája; és lelkeket nyer meg a bölcs.
\par 31 Ímé, az igaz e földön megnyeri jutalmát; mennyivel inkább az istentelen és a bûnös!

\chapter{12}

\par 1 A ki szereti a dorgálást, szereti a tudományt; a ki pedig gyûlöli a fenyítéket, oktalan az.
\par 2 A jó ember jóakaratot nyer az Úrtól; de a gonosz embert kárhoztatja õ.
\par 3 Nem erõsül meg ember az istentelenséggel; az igazaknak pedig gyökerök  ki nem mozdul.
\par 4 A derék asszony koronája az õ férjének; de mint az õ csontjaiban való rothadás, olyan a megszégyenítõ.
\par 5 Az igazaknak gondolatjaik igazak; az istentelenek tanácsa csalás.
\par 6 Az istenteleneknek beszédei leselkednek a vér után; az igazaknak pedig szája megszabadítja azokat.
\par 7 Leomlanak az istentelenek, és oda lesznek; az igazak háza pedig megáll.
\par 8 Az õ értelme szerint dicsértetik a férfiú; de az elfordult elméjû útálatos lesz.
\par 9 Jobb, a kit kevésre tartanak, és szolgája van, mint a ki magát felmagasztalja, és szûk kenyerû.
\par 10 Az igaz az õ barmának érzését is ismeri, az istentelenek szíve pedig kegyetlen.
\par 11 A ki míveli az õ földét, megelégedik eledellel; a ki pedig követ hiábavalókat, bolond az.
\par 12 Kivánja az istentelen a gonoszok prédáját; de az igaznak gyökere ád gyümölcsöt.
\par 13 Az ajkaknak vétkében gonosz tõr van, de kimenekedik a nyomorúságból az igaz.
\par 14 Az õ szájának gyümölcsébõl elégedik meg a férfi jóval; és az õ cselekedetének fizetését veszi az ember önmagának.
\par 15 A bolondnak úta helyes az õ szeme elõtt, de a ki tanácscsal él, bölcs az.
\par 16 A bolondnak haragja azon napon megismertetik; elfedezi pedig a szidalmat az eszes ember.
\par 17 A ki igazán szól, megjelenti az igazságot, a hamis bizonyság pedig az álnokságot.
\par 18 Van olyan, a ki beszél hasonlókat a tõrszúrásokhoz; de a bölcseknek nyelve orvosság.
\par 19 Az igazmondó ajak megáll mind örökké; a hazugságnak pedig nyelve egy szempillantásig.
\par 20 Álnokság van a gonosz gondolóknak szívében; a békességnek tanácsosiban pedig vígasság.
\par 21 Nem vettetik az igaz semmi bántásba; az istentelenek pedig teljesek nyavalyával.
\par 22 Útálatosok az Úrnál a csalárd beszédek; a kik pedig cselekesznek hûségesen, kedvesek õ nála.
\par 23 Az eszes ember elfedezi a tudományt; a bolondok elméje pedig kiáltja a bolondságot.
\par 24 A gyorsaknak keze uralkodik; a rest pedig adófizetõ lesz.
\par 25 A férfiúnak elméjében való gyötrelem megalázza azt; a jó szó pedig megvidámítja azt.
\par 26 Útba igazítja az õ felebarátját az igaz; de az istentelenek útja eltévelyíti õket.
\par 27 Nem süti meg a rest, amit vadászásával fogott; de drága marhája az embernek serénysége.
\par 28 Az igazságnak útjában van élet; és az õ ösvényének úta halhatatlanság.

\chapter{13}

\par 1 A bölcs fiú enged atyja intésének; de a csúfoló semmi dorgálásnak helyt nem ád.
\par 2 A férfi az õ szájának gyümölcsébõl él jóval; a hitetlenek lelke pedig bosszúságtétellel.
\par 3 A ki megõrzi az õ száját, megtartja önmagát; a ki felnyitja száját, romlása az annak.
\par 4 Kivánsággal felindul, de hiába, a restnek lelke; a gyorsak lelke pedig megkövéredik.
\par 5 A hamis dolgot gyûlöli az igaz; az istentelen pedig gyûlölségessé tesz és megszégyenít.
\par 6 Az igazság megõrzi azt, a ki útjában tökéletes; az istentelenség pedig elveszíti a bûnöst.
\par 7 Van, a ki hányja gazdagságát, holott semmije sincsen; viszont tetteti magát szegénynek, holott sok marhája van.
\par 8 Az ember életének váltsága lehet az õ gazdagsága; a szegény pedig nem hallja a fenyegetést.
\par 9 Az igazak világossága vígassággal ég; de az istenteleneknek  szövétneke kialszik.
\par 10 Csak háborúság lesz a kevélységbõl: azoknál pedig, a kik a tanácsot beveszik, bölcseség van.
\par 11 A hiábavalóságból keresett marha megkisebbül; a ki pedig kezével gyûjt, megõregbíti azt.
\par 12 A halogatott reménység beteggé teszi a szívet; de a megadatott kivánság életnek fája.
\par 13 Az igének megútálója megrontatik; a ki pedig féli a parancsolatot, jutalmát veszi.
\par 14 A bölcsnek tanítása életnek kútfeje,  a halál tõrének eltávoztatására.
\par 15 Jó értelem ád kedvességet; a hitetleneknek pedig útja kemény.
\par 16 Minden eszes cselekszik bölcseséggel; a bolond pedig kijelenti az õ bolondságát.
\par 17 Az istentelen követ bajba esik; a hívséges követ pedig gyógyulás.
\par 18 Szegénység és gyalázat lesz azon, a ki a fenyítéktõl magát elvonja; a ki pedig megfogadja a dorgálást, tiszteltetik.
\par 19 A megnyert kivánság gyönyörûséges a léleknek, és útálatosság a bolondoknak eltávozniok a gonosztól.
\par 20 A ki jár a bölcsekkel, bölcs lesz; a ki pedig magát társul adja a bolondokhoz, megromol.
\par 21 A bûnösöket követi a gonosz; az igazaknak pedig jóval fizet Isten.
\par 22 A jó örökséget hágy unokáinak; a bûnösnek marhái pedig eltétetnek az igaz számára.
\par 23 Bõ étele lesz a szegényeknek az új törésen; de van olyan, a ki igazságtalansága által vész el.
\par 24 A ki megtartóztatja az õ vesszejét, gyûlöli az õ fiát; a ki pedig szereti azt, megkeresi õt fenyítékkel.
\par 25 Az igaz eszik az õ kivánságának megelégedéséig; az istentelenek hasa pedig szûkölködik.

\chapter{14}

\par 1 A bölcs asszony építi a maga házát; a bolond pedig önkezével rontja el azt.
\par 2 A ki igazán jár, féli az Urat; a ki pedig elfordult az õ útaiban, megútálja õt.
\par 3 A bolondnak szájában van kevélységnek pálczája; a bölcseknek pedig beszéde megtartja õket.
\par 4 Mikor nincsenek ökrök: tiszta a jászol; a gabonának bõsége pedig az ökörnek erejétõl van.
\par 5 A hûséges tanú nem hazud; a hamis tanú pedig hazugságot bocsát szájából.
\par 6 A csúfoló keresi a bölcseséget, és nincs; a tudomány pedig az eszesnek könnyû.
\par 7 Menj el a bolond férfiú elõl; és nem ismerted meg a tudománynak beszédét.
\par 8 Az eszesnek bölcsesége az õ útának megértése; a bolondoknak pedig bolondsága csalás.
\par 9 A bolondokat megcsúfolja a bûnért való áldozat; az igazak között pedig jóakarat van.
\par 10 A szív tudja az õ lelke keserûségét; és az õ örömében az idegen nem részes.
\par 11 Az istenteleneknek háza elvész; de az igazaknak sátora megvirágzik.
\par 12 Van olyan út, mely helyesnek látszik az ember elõtt, és vége a halálra menõ út.
\par 13 Nevetés közben is fáj a szív; és végre az öröm fordul szomorúságra.
\par 14 Az õ útaiból elégszik meg az elfordult elméjû; önmagából pedig jó férfiú.
\par 15 Az együgyû hisz minden dolognak; az eszes pedig a maga járására vigyáz.
\par 16 A bölcs félvén, eltávozik a gonosztól; a bolond pedig dühöngõ és elbizakodott.
\par 17 A hirtelen haragú bolondságot cselekszik, és a cselszövõ férfi gyûlölséges lesz.
\par 18 Bírják az esztelenek a bolondságot örökség szerint; az eszesek pedig fonják a tudománynak koszorúját.
\par 19 Meghajtják magokat a gonoszok a jók elõtt, és a hamisak az igaznak kapujánál.
\par 20 Még az õ felebarátjánál is útálatos a szegény; a gazdagnak pedig sok a barátja.
\par 21 A ki megútálja az õ felebarátját, vétkezik; a ki pedig a szegényekkel kegyelmességet cselekszik,  boldog az!
\par 22 Nemde tévelyegnek, a kik gonoszt szereznek? kegyelmesség pedig és igazság a jó szerzõknek.
\par 23 Minden munkából nyereség lesz; de az ajkaknak beszédébõl csak szûkölködés.
\par 24 A bölcseknek ékességök az õ gazdagságuk; a tudatlanok bolondsága pedig csak bolondság.
\par 25 Lelkeket szabadít meg az igaz bizonyság; hazugságokat szól pedig az álnok.
\par 26 Az Úrnak félelmében erõs a bizodalom, és az õ fiainak lesz menedéke.
\par 27 Az Úrnak félelme az életnek kútfeje, a halál tõrének eltávoztatására.
\par 28 A nép sokasága a király dicsõsége; a nép elfogyása pedig az uralkodó romlása.
\par 29 A haragra késedelmes bõvelkedik értelemmel; a ki pedig elméjében hirtelenkedõ, bolondságot szerez az.
\par 30 A szelíd szív a testnek élete; az irígység pedig a csontoknak rothadása.
\par 31 A ki elnyomja a szegényt, gyalázattal illeti annak teremtõjét; az pedig tiszteli,  a ki könyörül a szûkölködõn.
\par 32 Az õ nyavalyájába ejti magát az istentelen; az igaznak pedig halála idején is reménysége van.
\par 33 Az eszesnek elméjében nyugszik a bölcseség; a mi pedig a tudatlanokban van, magát hamar megismerheti.
\par 34 Az igazság felmagasztalja a nemzetet; a bûn pedig gyalázatára van a népeknek.
\par 35 A királynak jóakaratja van az eszes szolgához; haragja pedig a megszégyenítõhöz.

\chapter{15}

\par 1 Az engedelmes felelet elfordítja a harag felgerjedését; a megbántó beszéd pedig támaszt haragot.
\par 2 A bölcsek nyelve beszél jó tudományt: a tudatlanoknak száján pedig bolondság buzog ki.
\par 3 Minden helyeken vannak az Úrnak szemei, nézvén a jókat és gonoszokat.
\par 4 A nyelv szelídsége életnek fája; az abban való hamisság pedig a léleknek gyötrelme.
\par 5 A bolond megútálja az õ atyjának tanítását; a ki pedig megbecsüli a dorgálást, igen eszes.
\par 6 Az igaznak házában nagy kincs van; az istentelennek jövedelmében pedig háborúság.
\par 7 A bölcseknek ajkaik hintegetnek tudományt; a bolondoknak pedig elméje nem helyes.
\par 8 Az istentelenek áldozatja gyûlölséges az Úrnak; az igazak könyörgése pedig kedves néki.
\par 9 Utálat az Úrnál az istentelennek úta; azt pedig, a ki követi az igazságot, szereti.
\par 10 Gonosz dorgálás jõ arra, a ki útját elhagyja; a ki gyûlöli a fenyítéket, meghal.
\par 11 A sír és a pokol az Úr elõtt vannak; mennyivel inkább az  emberek szíve.
\par 12 Nem szereti a csúfoló a feddést, és a bölcsekhez nem megy.
\par 13 A vidám elme megvidámítja az orczát; de a szívnek bánatja miatt a lélek megszomorodik.
\par 14 Az eszesnek elméje keresi a tudományt; a tudatlanok szája pedig legel bolondságot.
\par 15 Minden napjai a szegénynek nyomorúságosak; a vidám elméjûnek pedig szüntelen lakodalma van.
\par 16 Jobb a kevés az Úrnak félelmével, mint a temérdek kincs, a hol háborúság van.
\par 17 Jobb a paréjnak étele, a hol szeretet van, mint a hízlalt ökör, a hol van gyûlölség.
\par 18 A haragos férfiú szerez háborúságot; a hosszútûrõ pedig lecsendesíti a háborgást.
\par 19 A restnek útja olyan, mint a tövises sövény; az igazaknak pedig útja megegyengetett.
\par 20 A bölcs fiú örvendezteti az atyját; a bolond ember pedig megútálja az  anyját.
\par 21 A bolondság öröme az esztelennek; de az értelmes férfiú igazán jár.
\par 22 Hiábavalók lesznek a gondolatok, mikor nincs tanács; de a tanácsosok sokaságában elõmennek.
\par 23 Öröme van az embernek szája feleletében; és az idejében mondott beszéd, oh mely igen jó!
\par 24 Az életnek úta felfelé van az értelmes ember számára, hogy eltávozzék a pokoltól, mely aláfelé van.
\par 25 A kevélyeknek házát kiszakgatja az Úr; megerõsíti pedig az özvegynek határát.
\par 26 Útálatosak az Úrnak a gonosz gondolatok; de kedvesek a tiszta beszédek.
\par 27 Megháborítja az õ házát, a ki követi a telhetetlenséget; a ki pedig gyûlöli az ajándékokat, él az.
\par 28 Az igaznak elméje meggondolja, mit szóljon; az istenteleneknek pedig szája ontja a gonoszt.
\par 29 Messze van az Úr az istentelenektõl; az igazaknak pedig könyörgését meghallgatja.
\par 30 A szemek világa megvidámítja a szívet; a jó hír megerõsíti a csontokat.
\par 31 A mely fül hallgatja az életnek dorgálását, a bölcsek között lakik.
\par 32 A ki elvonja magát az erkölcsi tanítástól, megútálja az õ lelkét; a ki pedig hallgatja a feddést, értelmet szerez.
\par 33 Az Úrnak félelme a bölcseségnek tudománya, és a tisztességnek elõtte jár az  alázatosság.

\chapter{16}

\par 1 Az embernél vannak az elme gondolatjai; de az Úrtól van a nyelv felelete.
\par 2 Minden útai tiszták az embernek a maga szemei elõtt; de a ki a lelkeket vizsgálja, az Úr az!
\par 3 Bízzad az Úrra a te dolgaidat; és a te gondolatid véghez mennek.
\par 4 Mindent teremtett az Úr az õ maga czéljára; az istentelent is a büntetésnek napjára.
\par 5 Útálatos az Úrnak minden, a ki elméjében felfuvalkodott, kezemet adom rá, hogy nem marad büntetetlen.
\par 6 Könyörületességgel és igazsággal töröltetik el a bûn; és az Úrnak félelme által távozhatunk el a gonosztól.
\par 7 Mikor jóakarattal van az Úr valakinek útaihoz, még annak ellenségeit is jóakaróivá teszi.
\par 8 Jobb a kevés igazsággal, mint a gazdag jövedelem hamissággal.
\par 9 Az embernek elméje gondolja meg az õ útát; de az Úr igazgatja annak járását.
\par 10 Jósige van a király ajkain; az ítéletben ne szóljon hamisságot az õ szája.
\par 11 Az Úré az igaz mérték és mérõserpenyõ, az õ mûve minden mérõkõ.
\par 12 Útálatos legyen a királyoknál istentelenséget cselekedni; mert igazsággal erõsíttetik meg a királyiszék.
\par 13 Kedvesek a királyoknak az igaz beszédek; és az  igazmondót szereti a király.
\par 14 A királynak felgerjedt haragja olyan, mint a halál követe; de a bölcs férfiú leszállítja azt.
\par 15 A királynak vidám orczájában élet van, jóakaratja olyan, mint a tavaszi esõ fellege.
\par 16 Szerzeni bölcseséget, oh menynyivel jobb az aranynál; és szerzeni eszességet, kivánatosb az ezüstnél!
\par 17 Az igazak országútja eltávozás a gonosztól; megtartja magát az, a ki megõrzi az õ útát.
\par 18 A megromlás elõtt kevélység jár, és az eset elõtt felfuvalkodottság.
\par 19 Jobb alázatos lélekkel lenni a szelídekkel, mint zsákmányon osztozni a kevélyekkel.
\par 20 A ki figyelmez az igére, jót nyer; és a ki bízik az Úrban, oh mely boldog az!
\par 21 A ki elméjében bölcs, hívatik értelmesnek; a beszédnek pedig édessége neveli a tudományt.
\par 22 Életnek kútfeje az értelem annak, a ki bírja azt; de a bolondok fenyítéke bolondságuk.
\par 23 A bölcsnek elméje értelmesen igazgatja az õ száját, és az õ ajkain öregbíti a tudományt.
\par 24 Lépesméz a gyönyörûséges beszédek; édesek a léleknek, és meggyógyítói a tetemeknek.
\par 25 Van oly út, mely igaz az ember szeme elõtt, de vége a halálnak úta.
\par 26 A munkálkodó lelke magának munkálkodik; mert az õ szája kényszeríti õt.
\par 27 A haszontalan ember gonoszt ás ki, és az õ ajkain mintegy égõ tûz van;
\par 28 A gonosz ember versengést szerez, és a susárló elválasztja a jó barátokat.
\par 29 Az erõszakos ember elhiteti az õ felebarátját, és nem jó úton viszi õt.
\par 30 A ki behúnyja szemeit, azért teszi, hogy álnokságot gondoljon; a ki összeszorítja ajkait, már véghez vitte a gonoszságot.
\par 31 Igen szép ékes korona a vénség, az igazságnak útában találtatik.
\par 32 Jobb a hosszútûrõ az erõsnél; és a ki uralkodik a maga indulatján, annál, a ki várost vesz meg.
\par 33 Az ember kebelében vetnek sorsot; de az Úrtól van annak minden ítélete.

\chapter{17}

\par 1 Jobb a száraz falat, melylyel van csendesség; mint a levágott barmokkal teljes ház, melyben háborúság van.
\par 2 Az értelmes szolga uralkodik a gyalázatos fiún, és az atyafiak között az örökségnek részét veszi.
\par 3 Az olvasztótégely az ezüst számára van, és a kemencze az aranyéra; a szívek vizsgálója pedig az Úr.
\par 4 A gonosztevõ hallgat az álnok beszédekre, a csalárd hallgat a gonosz nyelvre.
\par 5 A ki megcsúfolja a szegényt, gyalázattal illeti annak Teremtõjét; a ki gyönyörködik másnak nyomorúságában, büntetlen  nem lészen!
\par 6 A véneknek ékessége az unokák, és a fiaknak ékessége az atyák.
\par 7 Nem illik a bolondnak az ékes beszéd, még kevésbbé a tisztességesnek a hazug beszéd.
\par 8 Drága kõ az ajándék elfogadójának szemei elõtt; mindenütt, a hova csak fordul, okosan cselekszik.
\par 9 Elfedezi a vétket, a ki keresi a szeretetet; a ki pedig ismétlen elõhoz egy dolgot, elszakasztja egymástól a barátságosokat is.
\par 10 Foganatosb a dorgálás az eszesnél, mint ha megvernéd a bolondot százszor is.
\par 11 Csak ellenkezést keres a gonosz, végre kegyetlen követ bocsáttatik ellene.
\par 12 Találjon valakire a fiától megfosztott medve, csak ne a bolond az õ bolondságában.
\par 13 A ki fizet gonoszt a jóért, nem távozik el a gonosz annak házától.
\par 14 Mint a ki árvizet szabadít el, olyan a háborúság kezdete; azért minekelõtte kihatna, hagyd el a versengést.
\par 15 A ki igaznak mondja a bûnöst, és kárhoztatja az igazat, útálatos az Úrnak egyaránt mind a kettõ.
\par 16 Miért van a vétel ára a bolondnak kezében a bölcseség megszerzésére, holott nincsen néki elméje?
\par 17 Minden idõben szeret, a ki igaz barát, és testvérül születik  a nyomorúság idejére.
\par 18 Értelmetlen ember az, a ki kezét adja, fogadván kezességet barátja elõtt.
\par 19 Szereti a gonoszt, a ki szereti a háborúságot; a ki magasbítja kapuját, romlást keres.
\par 20 Az elfordult szívû ember nem nyerhet jót, és a ki az õ nyelvével gonosz, esik nyomorúságba.
\par 21 A ki szül bolondot, szüli õ magának bánatra; és nem örvendez a bolondnak atyja.
\par 22 A vidám elme jó orvosságul szolgál; a szomorú lélek pedig megszáraztja a csontokat.
\par 23 A kebelbõl kivett ajándékot az istentelen elveszi, a törvény útának elfordítására.
\par 24 Az eszesnek orczájából kitetszik a bölcseség; a bolondnak pedig szemei országolnak a földnek végéig.
\par 25 Búsulása az õ atyjának a bolond fiú, és az õ szülõjének keserûsége.
\par 26 Még megbirságolni is az igazat nem jó, a tisztességest megverni igazságáért.
\par 27 A ki megtartóztatja beszédét, az tudós ember, és a ki higgadt lelkû, az értelmes férfiú.
\par 28 Még a bolond is, amikor hallgat, bölcsnek ítéltetik; mikor ajkait bezárja, eszesnek.

\chapter{18}

\par 1 A maga kivánsága után megy az agyas ember, minden igaz bölcseség ellen dühösködik.
\par 2 Nem gyönyörködik a bolond az értelemben, hanem abban, hogy az õ elméje nyilvánvalóvá legyen.
\par 3 Mikor eljõ az istentelen, eljõ a megútálás; és a szidalommal a gyalázat.
\par 4 Mély víz az ember szájának beszéde, buzogó patak a bölcseségnek kútfeje.
\par 5 A gonosz személyének kedvezni nem jó, elfordítani az igazat az ítéletben.
\par 6 A bolondnak beszédei szereznek versengést, és az õ szája ütésekért kiált.
\par 7 A bolondnak szája az õ romlása, és az õ beszédei az õ életének tõre.
\par 8 A susárlónak beszédei hizelkedõk; és azok a szív belsejét áthatják.
\par 9 A ki lágyan viseli magát az õ dolgában, testvére annak, a ki tönkre tesz.
\par 10 Erõs torony az Úrnak neve, ahhoz folyamodik az igaz, és bátorságos lészen.
\par 11 A gazdagnak vagyona az õ erõs városa, és mint a magas kõfal, az õ gondolatja szerint.
\par 12 A megromlás elõtt felfuvalkodik az ember elméje; a tisztesség elõtt pedig  alázatosság van.
\par 13 A ki felel valamit, míg meg nem hallja, ez bolondság és gyalázatos rá nézve.
\par 14 A férfiú lelke elviseli a maga erõtlenségét; de a megtört lelket ki viseli el?
\par 15 Az eszesnek elméje tudományt szerez, és a bölcseknek füle tudományt keres.
\par 16 Az embernek ajándéka szabad útat szerez néki, és a nagyoknak orczája elé viszi õt.
\par 17 Igaza van annak, a ki elsõ a perben; mígnem eljõ az õ peresfele, és megvizsgálja õt.
\par 18 A versengéseket megszünteti a sorsvetés, és az erõseket elválasztja.
\par 19 A felingerelt atyafiú erõsb az erõs városnál, és az ilyen versengések olyanok, mint a vár zárja.
\par 20 A férfi szájának hasznával elégedik meg az õ belseje; az õ beszédének jövedelmével lakik jól.
\par 21 Mind a halál, mind az élet a nyelv hatalmában van, és a miképen kiki szeret azzal élni, úgy eszi annak gyümölcsét.
\par 22 Megnyerte a jót, a ki talált feleséget, és vett jóakaratot az Úrtól!
\par 23 Alázatos kérést szól a szegény; a gazdag pedig keményen felel.
\par 24 Az ember, a kinek sok barátja van, széttöretik; de van barát, a ki ragaszkodóbb a testvérnél.

\chapter{19}

\par 1 Jobb a tökéletesen járó szegény a gonosz nyelvûnél, a ki bolond.
\par 2 A lélek sem jó tudomány nélkül; és a ki csak a lábával siet, hibázik.
\par 3 Az embernek bolondsága fordítja el az õ útát, és az Úr ellen haragszik az õ szíve.
\par 4 A gazdagság szaporítja a sok barátot; a szegénytõl pedig az õ barátja elválik.
\par 5 A hamis tanú büntetetlen nem marad, és a hazugságoknak szólója meg nem szabadul.
\par 6 Sokan hizelegnek a nemeslelkû embernek, és minden barát az adakozóé.
\par 7 A szegényt minden atyjafia gyûlöli, még barátai is eltávolodnak tõle; unszolja szavakkal, de õk eltünnek.
\par 8 A ki értelmet szerez, szereti az életét, a ki megõrzi az értelmességet, jót nyer.
\par 9 A hamis bizonyság nem marad büntetlen, és a ki hazugságokat beszél, elvész.
\par 10 Nem illik a bolondhoz a gyönyörködés; sokkal inkább nem illik a szolgának uralkodni a fejedelmeken.
\par 11 Az embernek értelme hosszútûrõvé teszi õt; és ékességére van néki elhallgatni a vétket.
\par 12 Mint az ifjú oroszlánnak ordítása, olyan a királynak haragja; mint a harmat pedig a füvön, az õ  jóakaratja.
\par 13 Romlása az õ atyjának a bolond fiú, és mint a szüntelen csepegés, az asszonynak zsémbelõdése.
\par 14 A ház és marha atyától való örökség; az Úrtól van pedig az értelmes feleség.
\par 15 A restség álomba merít, és a lomha lélek  megéhezik.
\par 16 A ki megtartja a parancsolatot, megtartja õ magát; a ki nem vigyáz útaira, meghal.
\par 17 Kölcsön ád az Úrnak, a ki kegyelmes a szegényhez; és az õ jótéteményét megfizeti néki.
\par 18 Fenyítsd meg a te fiadat, mert még van remény felõle; de annyira, hogy õt megöld, ne vigyen haragod.
\par 19 A nagy haragú ember büntetést szenvedjen, mert ha menteni akarod, még növeled haragját.
\par 20 Engedj a tanácsnak, és vedd be az erkölcsi oktatást, hogy bölcs légy végre.
\par 21 Sok gondolat van az ember elméjében; de csak az Úrnak tanácsa áll meg.
\par 22 A mit leginkább kell embernek kivánni, az irgalmasság az, és jobb a szegény a hazug férfiúnál.
\par 23 Az Úrnak félelme életre visz; és az ilyen megelégedve tölti az éjet, gonoszszal nem illettetik.
\par 24 Bemártja a rest az õ kezét a tálba, de már a szájához nem viszi vissza.
\par 25 Ha a csúfolót megvered, az együgyû lesz okosabb; és ha megdorgálod az eszest, megérti a tudományt.
\par 26 A ki atyjával erõszakoskodik, anyját elûzi: gyalázatos és megszégyenítõ fiú az.
\par 27 Szünjél meg, fiam, hallgatni az olyan tanítást, mely téged arra visz, hogy a bölcseségnek igéjétõl eltévedj.
\par 28 A semmirevaló bizonyság csúfolja a törvényt; az istentelenek szája elnyeli a gonoszságot.
\par 29 A csúfolóknak készíttettek a büntetések, és az ütések a bolondok hátának.

\chapter{20}

\par 1 A bor csúfoló, a részegítõ ital háborgó, és valaki abba beletéved, nem bölcs!
\par 2 Mint a fiatal oroszlán ordítása, olyan a királynak rettentése; a ki azt haragra ingerli, vétkezik a maga élete ellen.
\par 3 Tisztesség az embernek elmaradni a versengéstõl; valaki pedig bolond, patvarkodik.
\par 4 A hideg miatt nem szánt a rest; aratni akar majd, de nincs mit.
\par 5 Mély víz a férfiúnak elméjében a tanács; mindazáltal a bölcs ember kimeríti azt.
\par 6 A legtöbb ember talál valakit, a ki jó hozzá; de hû embert, azt ki találhat?
\par 7 A ki az õ tökéletességében jár, igaz ember; boldogok az õ fiai õ utána!
\par 8 A király, ha az õ ítélõszékiben ül, tekintetével minden gonoszt eltávoztat.
\par 9 Ki mondhatná azt: megtisztítottam szívemet, tiszta vagyok az én bûnömtõl?
\par 10 A kétféle font és a kétféle mérték, útálatos az Úrnál egyaránt mind a kettõ.
\par 11 Az õ cselekedetibõl ismerteti meg magát még a gyermek is, ha tiszta-é, és ha igaz-é az õ cselekedete.
\par 12 A halló fület és a látó szemet, az Úr teremtette egyaránt mindkettõt.
\par 13 Ne szeresd az álmot, hogy ne légy szegény; nyisd fel a te szemeidet, és megelégszel kenyérrel.
\par 14 Hitvány, hitvány, azt mondja a vevõ; de mikor elmegy, akkor dicsekedik.
\par 15 Van arany és drágagyöngyök sokasága; de drága szer a tudománynyal teljes ajak.
\par 16 Vedd el ruháját, mert kezes lett másért, és az idegenért vedd el zálogát.
\par 17 Gyönyörûséges az embernek az álnokságnak kenyere; de annakutána betelik az õ szája kavicsokkal.
\par 18 A gondolatok tanácskozással erõsek; és bölcs vezetéssel folytass  hadakozást.
\par 19 Megjelenti a titkot, a ki rágalmazó; tehát a ki fecsegõ szájú, azzal ne barátkozzál.
\par 20 A ki az õ atyját vagy anyját megátkozza, annak kialszik szövétneke a legnagyobb setétségben.
\par 21 A mely örökséget elõször siettetnek, annak vége meg nem áldatik.
\par 22 Ne mondd: bosszút állok rajta! Várjad az Urat, és megszabadít téged!
\par 23 Útálatos az Úrnál a kétféle súly; és a hamis fontok nem jó dolgok.
\par 24 Az Úrtól vannak a férfi lépései; az ember pedig mit ért az õ útában?
\par 25 Tõr az embernek meggondolatlanul mondani: szent, és a fogadástétel után megfontolni.
\par 26 Szétszórja a gonoszokat a bölcs király, és fordít reájok kereket.
\par 27 Az Úrtól való szövétnek az embernek lelke, a ki megvizsgálja a szívnek minden rejtekét.
\par 28 A kegyelmesség és az igazság megõrzik a királyt, megerõsíti irgalmasság által az õ székét.
\par 29 Az ifjaknak ékessége az õ erejök; és a véneknek dísze az õsz haj.
\par 30 A kékek és a sebek távoztatják el a gonoszt, és a belsõ részekig  ható csapások.

\chapter{21}

\par 1 Mint a vizeknek folyásai, olyan a királynak szíve az Úrnak kezében, valahová akarja, oda hajtja azt!
\par 2 Az embernek minden úta igaz a maga szemei elõtt; de a szívek vizsgálója az Úr.
\par 3 Az igazságnak és igaz ítéletnek gyakorlását inkább szereti az Úr az áldozatnál.
\par 4 A szemnek fenhéjázása és az elmének kevélysége: az istentelenek szántása, bûn.
\par 5 A szorgalmatosnak igyekezete csak gyarapodásra van; valaki pedig hirtelenkedik, csak szükségre jut.
\par 6 A hamisságnak nyelvével gyûjtött kincs elveszett hiábavalósága azoknak, a kik a halált keresik.
\par 7 Az istentelenek pusztítása magával ragadja õket; mert nem akartak igazságot cselekedni.
\par 8 Tekervényes a bûnös embernek úta; a tisztának cselekedete pedig igaz.
\par 9 Jobb a tetõ ormán lakni, mint háborgó asszonynyal, és közös házban.
\par 10 Az istentelennek lelke kiván gonoszt; és az õ szeme elõtt nem talál könyörületre az õ felebarátja.
\par 11 Mikor a csúfolót büntetik, az együgyû lesz bölcs; mikor pedig a bölcset oktatják, õ veszi eszébe a tudományt.
\par 12 Nézi az igaz az istentelennek házát, hogy milyen veszedelembe jutottak az istentelenek.
\par 13 A ki bedugja fülét a szegény kiáltására; õ is kiált, de meg nem hallgattatik.
\par 14 A titkon adott ajándék elfordítja a haragot; és a kebelben való ajándék a kemény búsulást.
\par 15 Vígasság az igaznak igazat cselekedni; de ijedelem a hamisság cselekedõinek.
\par 16 Az embr, a ki eltévelyedik az értelemnek útáról, az élet nélkül valók gyülekezetiben nyugszik.
\par 17 Szûkölködõ ember lesz, a ki szereti az örömet; a ki szereti a bort és az olajat, nem lesz gazdag!
\par 18 Az igazért váltságdíj az istentelen, és az igazak helyett a hitetlen büntettetik meg.
\par 19 Jobb lakozni a pusztának földén, mint a feddõdõ és haragos asszonynyal.
\par 20 Kivánatos kincs és kenet van a bölcsnek házában; a bolond ember pedig eltékozolja azt.
\par 21 A ki követi az igazságot és az irgalmasságot, nyer életet, igazságot és tisztességet.
\par 22 A hõsök városába felmegy a bölcs, és lerontja az õ bizodalmoknak erejét.
\par 23 A ki megõrzi száját és nyelvét, megtartja életét a nyomorúságtól.
\par 24 A kevély dölyfösnek csúfoló a neve, a ki haragjában kevélységet cselekszik.
\par 25 A restnek kivánsága megemészti õt; mert az õ kezei nem akarnak dolgozni.
\par 26 Egész nap kivánságtól gyötretik; az igaz pedig ád, és nem tartóztatja meg adományát.
\par 27 Az istentelenek áldozatja útálatos; kivált mikor gonosz tettért viszi.
\par 28 A hazug bizonyság elvész; a ki pedig jól figyelmez, örökké szól.
\par 29 Megkeményíti az istentelen ember az õ orczáját; az igaz pedig jól rendeli az õ útát.
\par 30 Nincs bölcseség, és nincs értelem, és nincs tanács az Úr ellen.
\par 31 Készen áll a ló az ütközetnek napjára; de az Úré a megtartás!

\chapter{22}

\par 1 Kivánatosb a jó hírnév nagy gazdagságnál; ezüstnél és aranynál a kedvesség jobb.
\par 2 A gazdag és szegény összetalálkoznak, mindkettõt pedig az Úr szerzi.
\par 3 Az eszes meglátja a bajt és elrejti magát; a bolondok pedig neki mennek és kárát vallják.
\par 4 Az alázatosságnak bére az Úr félelme, gazdagság és tisztesség és élet.
\par 5 Tövisek és tõrök vannak a gonosznak útában; a ki megõrzi a maga lelkét, távol jár azoktól.
\par 6 Tanítsd a gyermeket az õ útjának módja szerint; még mikor megvénhedik is, el nem távozik attól.
\par 7 A gazdag a szegényeken uralkodik, és szolgája a kölcsönvevõ a kölcsönadónak.
\par 8 A ki vet álnokságot, arat nyomorúságot; és az õ haragjának vesszeje megtöretik.
\par 9 Az irgalmas szemû ember megáldatik, mert adott az õ kenyerébõl a szegénynek.
\par 10 Ûzd el a csúfolót, és elmegy a háborgás is, és megszünik a patvarkodás és a szidalmazás.
\par 11 A ki szereti a szívnek tisztaságát, beszéde kedvesség: annak barátja a király.
\par 12 Az Úrnak szemei megõrzik a tudományt; a hitetlennek beszédét pedig felforgatja.
\par 13 A rest azt mondja: oroszlán van ottkin, az utczák közepén megölettetném.
\par 14 Mély verem az idegen asszonyoknak szája; a kire haragszik az Úr, oda esik.
\par 15 A gyermek elméjéhez köttetett a bolondság; de a fenyítés vesszeje messze elûzi õ tõle azt.
\par 16 A ki elnyomja a szegényt, hogy szaporítsa az õ marháját; a ki ád a gazdagnak: végre szûkölködésre jut.
\par 17 Hajtsd füledet, és hallgasd a bölcseknek beszédeit; és a te elmédet figyelmeztesd az én tudományomra.
\par 18 Mert gyönyörûséges lesz, ha megtartod azokat szívedben; legyenek együtt állandók a te ajkaidon!
\par 19 Hogy az Úrban legyen a te bizodalmad, arra tanítottalak ma téged, igen, téged.
\par 20 Nem írtam-é néked drága szép tanulságokat, tanácsokban és tudományban?
\par 21 Hogy tudtodra adjam néked az igazság beszédinek bizonyos voltát: hogy igaz beszédet vígy válaszul elküldõidnek.
\par 22 Ne rabold ki a szegényt, mert szegény õ; és meg ne rontsd a nyomorultat a kapuban;
\par 23 Mert az Úr forgatja azoknak ügyét, és az õ kirablóik életét elragadja.
\par 24 Ne tarts barátságot a haragossal, és a dühösködõvel ne menj;
\par 25 Hogy el ne tanuld az õ útait, és tõrt ne keress tennen magadnak.
\par 26 Ne légy azok közt, a kik kézbe csapnak, a kik adósságért kezeskednek.
\par 27 Ha nincs néked mibõl megadnod; miért vegye el a te ágyadat te alólad?
\par 28 Ne bontsd el a régi  határt, melyet csináltak a te eleid.
\par 29 Láttál-é az õ dolgában szorgalmatos embert? A királyok elõtt álland, nem marad meg az alsó rendûek között.

\chapter{23}

\par 1 Mikor leülsz enni az uralkodóval, szorgalmasan reá vigyázz, ki van elõtted.
\par 2 És kést tégy a torkodra, ha mértékletlen vagy.
\par 3 Ne kivánd az õ csemegéit; mert ezek hazug étkek.
\par 4 Ne fáraszd magadat ebben, hogy meggazdagulj; ez ilyen testi eszességedtõl szünjél meg.
\par 5 Avagy a te szemeidet veted-é arra? holott az semmi, mert olyan szárnyakat szerez magának nagy hamar, mint a saskeselyû, és az ég felé elrepül!
\par 6 Ne egyél az irígy szemûnek étkébõl, és ne kivánd az õ csemegéit;
\par 7 Mert mint a ki számítgatja a falatot magában, olyan õ: egyél és igyál, azt mondja te néked; de azért nem jó akarattal van tehozzád.
\par 8 A te falatodat, a melyet megettél, kihányod; és a te ékes beszédidet csak hiába vesztegeted.
\par 9 A bolondnak hallására ne szólj; mert megútálja a te beszédidnek bölcseségét.
\par 10 Ne mozdítsd meg a régi  határt, és az árváknak mezeibe ne kapj;
\par 11 Mert az õ megváltójuk erõs, az forgatja az õ ügyöket ellened!
\par 12 Add a te elmédet az erkölcsi tanításra, és a te füleidet a bölcs beszédekre.
\par 13 Ne vond el a gyermektõl a fenyítéket; ha megvered õt vesszõvel, meg nem hal.
\par 14 Te vesszõvel vered meg õt: és az õ lelkét a pokolból ragadod ki.
\par 15 Szerelmes fiam, ha bölcs lesz a te elméd, örvendez a lelkem nékem is.
\par 16 És vígadoznak az én veséim, a te ajkaidnak igazmondásán.
\par 17 Ne irígykedjék a te szíved a bûnösökre; hanem az Úr félelmében légy egész napon;
\par 18 Mert ennek bizonyos vége van; a te várakozásod meg nem csalatkozik.
\par 19 Hallgass te, fiam, engem, hogy légy bölcs, és jártasd ez úton szívedet.
\par 20 Ne légy azok közül való, a kik borral dõzsölnek; azok közül, a kik hússal dobzódnak.
\par 21 Mert a részeges és dobzódó szegény lesz, és rongyokba öltöztet az aluvás.
\par 22 Hallgasd a te atyádat, a ki nemzett téged; és meg ne útáld a te anyádat, mikor megvénhedik.
\par 23 Szerezz igazságot, és el ne adj; bölcseséget és erkölcsöt és eszességet.
\par 24 Igen örül az igaznak atyja, és a bölcsnek szülõje annak vígadoz.
\par 25 Vígadjon a te atyád és a te anyád, és örvendezzen a te szülõd.
\par 26 Adjad, fiam, a te szívedet nékem, és a te szemeid az én útaimat megõrizzék.
\par 27 Mert mély verem a tisztátalan asszony, és szoros kút az idegen asszony.
\par 28 És az, mint a tolvaj leselkedik, és az emberek közt a hitetleneket szaporítja.
\par 29 Kinek jaj? kinek oh jaj? kinek versengések? kinek panasz? kinek ok nélkül való sebek? kinek szemeknek veressége?
\par 30 A bornál mulatóknak, a kik mennek a jó bor kutatására.
\par 31 Ne nézd a bort, mily veres színt játszik, mint mutatja a pohárban az õ csillogását; könnyen alá csuszamlik,
\par 32 Végre, mint a kígyó, megmar, és mint a mérges kígyó, megcsíp.
\par 33 A te szemeid nézik az idegen asszonyt, és a te elméd gondol gonoszságot.
\par 34 És olyan leszel, mint a ki fekszik a tenger közepiben, és a ki fekszik az árbóczfának tetején.
\par 35 Ütöttek engem, nékem nem fájt; vertek, nem éreztem! Mikor ébredek fel? Akkor folytatom, ismét megkeresem azt.

\chapter{24}

\par 1 Ne irígykedjél a gonosztevõkre, se ne kivánj azokkal lenni.
\par 2 Mrt pusztítást gondol az õ szívök, és bajt szólnak az õ ajkaik.
\par 3 Bölcseség által építtetik a ház, és értelemmel erõsíttetik meg.
\par 4 És tudomány által telnek meg a kamarák minden drága és gyönyörûséges marhával.
\par 5 A bölcs férfiú erõs, és a tudós ember nagy erejû.
\par 6 Mert az eszes tanácsokkal viselhetsz hadat hasznodra; és a megmaradás a tanácsosok sokasága által van.
\par 7 Magas a bolondnak a bölcseség; a kapuban nem nyitja meg az õ száját.
\par 8 A ki azon gondolkodik, hogy gonoszt cselekedjék, azt cselszövõnek hívják.
\par 9 A balgatag dolognak gondolása bûn; és a rágalmazó az ember elõtt útálatos.
\par 10 Ha lágyan viselted magadat a nyomorúságnak idején: szûk a te erõd.
\par 11 Szabadítsd meg azokat, a kik a halálra vitetnek, és a kik a megöletésre tántorognak, tartóztasd meg!
\par 12 Ha azt mondanád: ímé, nem tudtuk ezt; nemde, a ki vizsgálja az elméket, õ érti, és a ki õrzi a te lelkedet, õ tudja? és kinek-kinek az õ cselekedetei szerint fizet.
\par 13 Egyél, fiam, mézet, mert jó; és a színméz édes a te ínyednek.
\par 14 Ilyennek ismerd a bölcseséget a te lelkedre nézve: ha azt megtalálod, akkor lesz jó véged, és a te reménységed el nem vész!
\par 15 Ne leselkedjél, oh te istentelen, az igaznak háza ellen, ne pusztítsd el az õ ágyasházát!
\par 16 Mert ha hétszer elesik is az igaz, ugyan felkél azért; az istentelenek pedig csak egy nyavalyával is elvesznek.
\par 17 Mikor elesik a te ellenséged: ne örülj; és mikor megütközik: ne vígadjon a te szíved,
\par 18 Hogy az Úr meg ne lássa és gonosz ne legyen szemeiben, és el ne fordítsa arról az õ haragját te reád.
\par 19 Ne gerjedj haragra a gonosztevõk ellen, ne irígykedjél az istentelenekre;
\par 20 Mert a gonosznak nem lesz jó vége, az istentelenek szövétneke kialszik.
\par 21 Féld az Urat, fiam, és a királyt; a pártütõk közé ne elegyedjél.
\par 22 Mert hirtelenséggel feltámad az õ nyomorúságok, és e két rendbeliek büntetését ki tudja?
\par 23 Ezek is a bölcsek szavai. Személyt válogatni az ítéletben  nem jó.
\par 24 A ki azt mondja az istentelennek: igaz vagy, ezt megátkozzák a népek, megútálják a nemzetek.
\par 25 A kik pedig megfeddik a bûnöst, azoknak gyönyörûségökre lesz, és jó áldás száll reájok!
\par 26 Ajkakat csókolgat az, a ki igaz beszédeket felel.
\par 27 Szerezd el kivül a te dolgodat, és készíts elõ a te mezõdben; annakutána építsd a házadat.
\par 28 Ne légy bizonyság ok nélkül a te felebarátod ellen; avagy ámítanál-é valakit a te ajkaiddal?
\par 29 Ne mondd ezt: a miképen cselekedett én vele, úgy cselekszem õ vele; megfizetek az embernek az õ cselekedete szerint.
\par 30 A rest embernek mezejénél elmenék, és az esztelennek szõleje mellett.
\par 31 És ímé, mindenütt felverte a tövis, és színét elfedte a gyom; és kõgyepüje elromlott vala.
\par 32 Melyet én látván gondolkodám, és nézvén, ezt a tanulságot vevém abból:
\par 33 Egy kis álom, egy ki szunynyadás, egy kis kézösszetevés az alvásra,
\par 34 És így jõ el, mint az útonjáró, a te szegénységed, és a te szükséged, mint a paizsos férfiú.

\chapter{25}

\par 1 Még ezek is Salamon példabeszédei, melyeket összeszedegettek Ezékiásnak, a Júda királyának emberei.
\par 2 Az Istennek tisztességére van a dolgot eltitkolni; a királyoknak pedig tisztességére van a dolgot kikutatni.
\par 3 Az ég magasságra, a föld mélységre, és a királyoknak szíve kikutathatatlan.
\par 4 Távolítsd el az ezüstbõl a salakot, és abból edény lesz az ötvösnek:
\par 5 Távolítsd el a bûnöst a király elõl, és megerõsíttetik igazsággal az õ széke.
\par 6 Ne dicsekedjél a király elõtt, és a nagyok helyére ne állj;
\par 7 Mert jobb, ha azt mondják néked: jer ide fel; hogynem mint levettetned néked a tisztességes elõtt, a kit láttak a te szemeid.
\par 8 Ne indulj fel a versengésre hirtelen, hogy azt ne kelljen kérdened, mit cselekedjél az után, mikor gyalázattal illet téged a te felebarátod.
\par 9 A te ügyedet végezd el felebarátoddal; de másnak titkát meg ne jelentsd;
\par 10 Hogy ne gyalázzon téged, a ki hallja; és a te gyalázatod el ne távozzék.
\par 11 Mint az arany alma ezüst tányéron: olyan a helyén mondott ige!
\par 12 Mint az arany függõ és színarany ékesség: olyan a bölcs intõ a szófogadó fülnél.
\par 13 Mint a havas hideg az aratásnak idején: olyan a hív követ azoknak, a kik õt elbocsátják; mert az õ urainak lelkét megvidámítja.
\par 14 Mint a felhõ és szél, melyekben nincs esõ: olyan a férfiú, a ki kérkedik hamis ajándékkal.
\par 15 Tûrés által engeszteltetik meg a fejedelem, és a szelíd beszéd megtöri a csontot.
\par 16 Ha mézet találsz, egyél a mennyi elég néked; de sokat ne egyél, hogy ki ne hányd azt.
\par 17 Ritkán tedd lábadat a te felebarátodnak házába; hogy be ne teljesedjék te veled, és meg ne gyûlöljön téged.
\par 18 Põröly és kard és éles nyíl az olyan ember, a ki hamis bizonyságot szól felebarátja ellen.
\par 19 Mint a romlott fog és kimarjult láb: olyan a hitetlennek bizodalma a nyomorúság idején.
\par 20 Mint a ki leveti ruháját a hidegnek idején, mint az eczet a sziksón: olyan, a ki éneket mond a bánatos szívû ember elõtt.
\par 21 Ha éhezik, a ki téged gyûlöl: adj enni néki kenyeret; és ha szomjúhozik: adj néki inni vizet;
\par 22 Mert elevenszenet gyûjtesz az õ fejére, és az Úr megfizeti néked.
\par 23 Az északi szél esõt szül; és haragos ábrázatot a suttogó nyelv.
\par 24 Jobb lakni a tetõnek ormán, mint a háborgó asszonynyal, és közös házban.
\par 25 Mint a hideg víz a megfáradt embernek, olyan a messze földrõl való jó hírhallás.
\par 26 Mint a megháborított forrás és megromlott kútfõ, olyan az igaz, a ki a gonosz elõtt ingadozik.
\par 27 Igen sok mézet enni nem jó; hát a magunk dicsõségét keresni dicsõség?
\par 28 Mint a megromlott és kerítés nélkül való város, olyan a férfi, a kinek nincsen birodalma az õ lelkén!

\chapter{26}

\par 1 Mint a hó a nyárhoz és az esõ az aratáshoz, úgy nem illik a bolondhoz a tisztesség.
\par 2 Miképen a madár elmegy és a fecske elrepül, azonképen az ok nélkül való átok nem száll az emberre.
\par 3 Ostor a lónak, fék a szamárnak; és vesszõ  a bolondok hátának.
\par 4 Ne felelj meg a bolondnak az õ bolondsága szerint, hogy ne légy te is õ hozzá hasonlatos;
\par 5 Felelj meg a bolondnak az õ bolondsága szerint, hogy ne legyen bölcs a maga szemei elõtt.
\par 6 A ki bolond által izen valamit, lábait vagdalja el magának, és bosszúságot szenved.
\par 7 Mint a sántának lábai lógnak, úgy a bölcsmondás a bolondoknak szájában.
\par 8 Mint a ki követ köt a parittyába, úgy cselekszik, a ki a bolondnak tisztességet tesz.
\par 9 Mint a részeg ember kezébe akad a tövis, úgy akad az eszes mondás a bolondoknak szájába.
\par 10 Mint a lövöldözõ, a ki mindent megsebez, olyan az, a ki bolondot fogad fel, és a ki csavargókat fogad fel.
\par 11 Mint az eb megtér a maga okádására, úgy a bolond megkettõzteti az õ bolondságát.
\par 12 Láttál-é oly embert, aki a maga szemei elõtt bölcs? A bolond felõl jobb reménységed legyen, hogynem mint a felõl!
\par 13 Azt mondja a rest: ordító oroszlán van az úton! oroszlán van az utczákon!
\par 14 Mint az ajtó forog az õ sarkán, úgy a rest az õ ágyában.
\par 15 Ha a rest az õ kezét a tálba nyujtotta, resteli azt csak szájához is vinni.
\par 16 Bölcsebb a rest a maga szemei elõtt, mint hét olyan, a ki okos feleletet ád.
\par 17 Kóbor ebet ragad fülön, a ki felháborodik a perpatvaron, a mely õt nem illeti.
\par 18 Mint a balga, a ki tüzet, nyilakat és halálos szerszámokat lövöldöz,
\par 19 Olyan az, a ki megcsalja az õ felebarátját, és azt mondja: csak tréfáltam!
\par 20 Ha a fa elfogy, kialuszik a tûz; ha nincs súsárló, megszûnik a háborgás.
\par 21 Mint az elevenszénre a holtszén, és a fa a tûzre, olyan a háborúságszerzõ ember a patvarkodásnak felgyujtására.
\par 22 A fondorlónak beszédei hízelkedõk, és azok áthatják a szív belsejét.
\par 23 Mint a meg nem tisztított ezüst, melylyel valami agyagedényt beborítottak, olyanok a gyulasztó ajkak a gonosz szív mellett.
\par 24 Az õ beszédeivel másnak tetteti magát a gyûlölõ, holott az õ szívében gondol álnokságot.
\par 25 Mikor kedvesen szól, ne bízzál õ hozzá; mert hét iszonyatosság van szívében.
\par 26 Elfedeztethetik a gyûlölség csalással; de nyilvánvalóvá lesz az õ gonoszsága a gyülekezetben.
\par 27 A ki vermet ás másnak, abba belé esik; és a ki felhengeríti a követ, arra gurul vissza.
\par 28 A hazug nyelv gyûlöli az általa megrontott embert, és a hízelkedõ száj romlást szerez.

\chapter{27}

\par 1 Ne dicsekedjél a holnapi nappal; mert nem tudod, mit hoz a nap tereád.
\par 2 Dicsérjen meg téged más, és ne a te szájad; az idegen, és ne a te ajkaid.
\par 3 Nehézség van a kõben, és teher a fövényben; de a bolondnak haragja nehezebb mind a kettõnél.
\par 4 A búsulásban kegyetlenség van, és a haragban áradás; de ki állhatna meg az irígység elõtt?
\par 5 Jobb a nyilvánvaló dorgálás a titkos szeretetnél.
\par 6 Jószándékból valók a barátságos embertõl vett sebek; és temérdek a gyûlölõnek csókja.
\par 7 A jóllakott ember még a lépesmézet is magtapodja; de az éhes embernek minden keserû édes.
\par 8 Mint a madárka, ki elbujdosott fészkétõl, olyan az ember, a ki elbujdosott az õ lakóhelyétõl.
\par 9 Mint a kenet és jó illat megvídámítja a szívet: úgy az õ barátjának édes szavai is, melyek lelke tanácsából valók.
\par 10 A te barátodat, és a te atyádnak barátját el ne hagyd, és a te atyádfiának házába be ne menj nyomorúságodnak idején. Jobb a közel való szomszéd a messze való atyafinál.
\par 11 Légy bölcs fiam, és vídámítsd meg az én szívemet; hogy megfelelhessek annak, a ki engem ócsárol.
\par 12 Az eszes meglátja a bajt, elrejti magát; az esztelenek neki mennek, kárát vallják.
\par 13 Vedd el a ruháját, mert kezes lett másért, és az idegenért zálogold meg.
\par 14 A ki nagy hangon áldja az õ barátját, reggel jó idején felkelvén; átokul tulajdoníttatik néki.
\par 15 A sebes záporesõ idején való szüntelen csepegés, és a morgó asszonyember hasonlók.
\par 16 Valaki el akarja azt rejteni, szelet rejt el, és az õ jobbja olajjal találkozik.
\par 17 Miképen egyik vassal a másikat élesítik, a képen az ember élesíti az õ barátjának orczáját.
\par 18 Mint a ki õrzi a fügét, eszik annak gyümölcsébõl, úgy a ki az õ urára vigyáz, tiszteltetik.
\par 19 Mint a vízben egyik orcza a másikat megmutatja, úgy egyik embernek szíve a másikat.
\par 20 Mint a sír és a pokol meg nem elégednek, úgy az embernek szemei meg nem elégednek.
\par 21 Mint az ezüst a tégelyben, és az arany a kemenczében próbáltatik meg, úgy az ember az õ híre-neve szerint.
\par 22 Ha megtörnéd is a bolondot mozsárban mozsártörõvel a megtört gabona között, nem távoznék el õ tõle az õ bolondsága.
\par 23 Szorgalmasan megismerd a te juhaid külsejét, gondolj a nyájakra.
\par 24 Mert nem örökkévaló a gazdagság, és vajjon a korona nemzetségrõl nemzetségre lesz-é?
\par 25 Mikor levágatott a szénafû, és megtetszett a sarjú, és begyûjtettek a hegyekrõl a fûvek:
\par 26 Vannak juhaid a te ruházatodra, és kecskebakok mezõnek árául,
\par 27 És elég kecsketej a te ételedre, a te házadnépének ételére, és szolgálóleányaidnak ételül.

\chapter{28}

\par 1 Minden istentelen fut, ha senki nem üldözi is; az igazak pedig, mint az ifjú oroszlán, bátrak.
\par 2 Az ország bûne miatt sok annak a fejedelme; az eszes és tudós ember által pedig hosszabbodik fennállása.
\par 3 A szegény emberbõl támadott elnyomója a szegényeknek hasonló a pusztító esõhöz, mely nem hágy kenyeret.
\par 4 A kik elhagyják a törvényt, dicsérik a latrokat; de a kik megtartják a törvényt, harczolnak azokkal.
\par 5 A gonoszságban élõ emberek nem értik meg az igazságot; a kik pedig keresik az Urat, mindent megértenek.
\par 6 Jobb a szegény, a ki jár tökéletesen, mint a kétfelé sántáló istentelen, a ki gazdag.
\par 7 A ki megõrzi a törvényt, eszes fiú az; a ki pedig társalkodik a dobzódókkal, gyalázattal illeti atyját.
\par 8 A ki öregbíti az õ marháját kamattal és uzsorával, annak gyûjt, a ki könyörül a szegényeken.
\par 9 Valaki elfordítja az õ fülét a törvénynek hallásától, annak könyörgése is útálatos.
\par 10 A ki elcsábítja az igazakat gonosz útra, vermébe maga esik bele; a tökéletesek pedig örökség szerint bírják a jót.
\par 11 Bölcs az õ maga szemei elõtt a gazdag ember; de az eszes szegény megvizsgálja õt.
\par 12 Mikor örvendeznek az igazak, nagy ékesség az; mikor pedig az istentelenek feltámadnak, keresni kell az embert.
\par 13 A ki elfedezi az õ vétkeit, nem lesz jó dolga; a ki pedig megvallja és elhagyja, irgalmasságot nyer.
\par 14 Boldog ember, a ki szüntelen retteg; a ki pedig megkeményíti az õ szívét, bajba esik.
\par 15 Mint az ordító oroszlán és éhezõ medve, olyan a szegény népen uralkodó istentelen.
\par 16 Az értelemben szûkölködõ fejedelem nagy elnyomó is; de a ki gyûlöli a hamis nyereséget, meghosszabbítja napjait.
\par 17 Az ember, a kit ember-vér terhel, a sírig fut; senki ne támogassa õt.
\par 18 A ki jár tökéletesen, megtartatik; a ki pedig álnokul két úton jár, egyszerre elesik.
\par 19 A ki munkálja az õ földét, megelégedik étellel; a ki pedig hiábavalóságok után futkos, megelégedik szegénységgel.
\par 20 A hivõ ember bõvelkedik áldásokkal; de a ki hirtelen akar gazdagulni, büntetlen nem marad.
\par 21 Személyt válogatni nem jó; mert még egy falat kenyérért is vétkezhetik az ember.
\par 22 Siet a marhakeresésre a gonosz szemû ember; és nem veszi észre, hogy szükség jõ reá.
\par 23 A ki megfeddi az embert, végre is inkább kedvességet talál, mint a sima nyelvû.
\par 24 A ki megrabolja az atyját, és anyját és azt mondja: nem vétek! társa a romboló embernek.
\par 25 A telhetetlen lélek háborúságot szerez; a ki pedig bízik az Úrban, megerõsödik.
\par 26 A ki bízik magában, bolond az; a ki pedig jár bölcsen, megszabadul.
\par 27 A ki ád a szegénynek, nem lesz néki szüksége; a ki pedig elrejti a szemét, megsokasulnak rajta az átkok.
\par 28 Mikor felemeltetnek az istentelenek, elrejti magát az ember; de mikor azok elvesznek,  öregbülnek az igazak.

\chapter{29}

\par 1 A ki a feddésekre is nyakas marad, egyszer csak összetörik, gyógyíthatatlanul.
\par 2 Mikor öregbülnek az igazak, örül a nép; mikor pedig uralkodik az istentelen, sóhajt a nép.
\par 3 A bölcseség-szeretõ ember megvidámítja az õ atyját; a ki pedig a paráznákhoz adja magát, elveszti  a vagyont.
\par 4 A király igazsággal erõsíti meg az országot; a ki pedig ajándékot vesz, elrontja azt.
\par 5 A férfiú, a ki hizelkedik barátjának, hálót vet annak lábai elé.
\par 6 A gonosz ember vétkében tõr van; az igaz pedig énekel és vígad.
\par 7 Megérti az igaz a szegényeknek ügyét; az istentelen  pedig nem tudja megérteni.
\par 8 A csúfoló férfiak fellobbantják a várost; de a bölcsek elfordítják a haragot.
\par 9 Az eszes ember, ha vetekedik a bolonddal, akár felháborodik, akár nevet, nincs nyugodalom.
\par 10 A vérszomjasak gyûlölik a tökéletes embert; az igazak pedig oltalmazzák annak életét.
\par 11 Az õ egész indulatját elõmutatja a bolond; de a bölcs végre megcsendesíti azt.
\par 12 A mely uralkodó a hamisságnak beszédire hallgat, annak minden szolgái latrok.
\par 13 A szegény és az uzsorás ember összetalálkoznak; mind a kettõnek pedig szemeit az Úr világosítja meg.
\par 14 A mely király hûségesen ítéli a szegényeket, annak széke mindörökké megáll.
\par 15 A vesszõ és dorgálás bölcseséget ád; de a szabadjára hagyott gyermek megszégyeníti az õ anyját.
\par 16 Mikor nevekednek az istentelenek, nevekedik a vétek; az igazak pedig azoknak esetét megérik.
\par 17 Fenyítsd meg a te fiadat, és nyugodalmat hoz néked, és szerez gyönyörûséget a te lelkednek.
\par 18 Mikor nincs mennyei látás, a nép elvadul; ha pedig megtartja a törvényt, oh mely igen boldog!
\par 19 Csak beszéddel nem tanul meg a szolga, mert tudna, de még sem felel meg.
\par 20 Láttál-é beszédeiben hirtelenkedõ embert? a bolond felõl több reménység van, hogynem a felõl!
\par 21 A ki lágyan neveli gyermekségétõl fogva az õ szolgáját, végre az lesz a fiú.
\par 22 A haragos háborgást szerez; és a dühösködõnek sok a vétke.
\par 23 Az embernek kevélysége megalázza õt; az alázatos pedig  tisztességet nyer.
\par 24 A ki osztozik a lopóval, gyûlöli az magát; hallja az esküt, de nem vall.
\par 25 Az emberektõl való félelem tõrt vet; de a ki bízik az Úrban, kiemeltetik.
\par 26 Sokan keresik a fejedelemnek orczáját; de az Úrtól van kinek-kinek ítélete.
\par 27 Iszonyat az igazaknak a hamis ember; és iszonyat az istentelennek az igaz úton járó.

\chapter{30}

\par 1 Agurnak, a Jáké fiának beszédei, próféczia, melyet mondott a férfiú Itielnek, Itielnek és Ukálnak.
\par 2 Minden embernél tudatlanabb vagyok én, és nincs emberi értelem én bennem.
\par 3 És nem tanultam a bölcseséget, hogy a Szentnek ismeretét tudnám.
\par 4 Kicsoda ment fel az égbe, hogy onnan leszállott volna? Kicsoda fogta össze a szelet az õ markába? Kicsoda kötötte a vizet az õ köntösébe? Ki állapította meg a földnek minden határit? Kicsoda ennek neve? Avagy kicsoda ennek fiának neve, ha tudod?
\par 5 Az Istennek teljes beszéde igen tiszta, és paizs az ahhoz folyamodóknak.
\par 6 Ne tégy az õ beszédéhez, hogy meg ne feddjen téged, és hazug ne légy.
\par 7 Kettõt kérek tõled; ne tartsd meg én tõlem, mielõtt meghalnék.
\par 8 A hiábavalóságot és a hazugságot messze távoztasd tõlem; szegénységet vagy gazdagságot ne adj nékem; táplálj engem hozzám illendõ eledellel.
\par 9 Hogy megelégedvén, meg ne tagadjalak, és azt ne mondjam: kicsoda az Úr? Se pedig megszegényedvén, ne lopjak, és gonoszul ne éljek az én Istenem nevével!
\par 10 Ne rágalmazd a szolgát az õ uránál, hogy meg ne átkozzon téged, és bûnhõdnöd ne kelljen.
\par 11 Van oly nemzetség, a ki az õ atyját átkozza, és az õ anyját nem áldja.
\par 12 Van nemzetség, a ki a maga szemei elõtt tiszta, pedig az õ rútságából ki nem tisztíttatott.
\par 13 Van kevély szemû nemzetség, és a kinek szemöldökei igen fellátnak!
\par 14 Van olyan nemzetség, a kinek fogai fegyverek, és a kinek zápfogai kések; hogy a szegényeket kiemészszék e földrõl, és az emberek közül a szûkölködõket.
\par 15 A nadálynak két leánya van: addsza, addsza! E három nem elégszik meg; négyen nem mondják: elég;
\par 16 A sír és a meddõ asszony, a föld meg nem elégszik a vízzel, és a tûz nem mondja: elég!
\par 17 A szemet, mely megcsúfolja atyját, vagy megútálja az anyja iránt való engedelmességet, kivágják a völgynek hollói, vagy megeszik a sasfiak.
\par 18 E három megfoghatatlan elõttem, és e négy dolgot nem tudom:
\par 19 A keselyûnek útát az égben, a kígyónak útát a kõsziklán, a hajónak nyomát a mély tengerben, és a férfiúnak útát a leányzóval.
\par 20 Ilyen a paráználkodó asszonynak úta; eszik, azután megtörli száját és azt mondja: nem cselekedtem semmi gonoszt.
\par 21 Három dolog alatt indul meg a föld, és négyet nem szenvedhet el.
\par 22 A szolga alatt, mikor uralkodik, és a bolond alatt, mikor elég kenyere van,
\par 23 A gyûlölt asszony alatt, ha mégis férjhez megy; és a szolgáló alatt, ha örököse lesz az õ asszonyának.
\par 24 E négy apró állata van a földnek, a melyek bölcsek, elmések:
\par 25 A hangyák erõtlen nép, mégis megkeresik nyárban a magok eledelét;
\par 26 A marmoták nem hatalmas nép, mégis kõsziklán csinálják az õ házokat;
\par 27 Királyuk nincs a sáskáknak, mindazáltal mindnyájan szép renddel mennek ki;
\par 28 A pókot kézzel megfoghatod, mégis ott van a királyok palotáiban.
\par 29 Három állat van, a mely szépen jár, sõt négy, a mely jól jár.
\par 30 Az oroszlán, a hõs a vadak között, mely el nem fut senki elõl;
\par 31 A harczra felékesített ló, vagy a kecskebak, és a király, a kinek senki nem mer ellene állani.
\par 32 Ha bolond voltál felfuvalkodásodban, vagy ha meggondoltad: kezedet szájadra vessed.
\par 33 Mert miképen a ki tejet köpül, vajat csinál; és a ki keményen fújja ki az õ orrát, vért hoz ki: úgy a ki a haragot ingerli, háborúságot szerez.

\chapter{31}

\par 1 Lemuel király beszédei, próféczia, melylyel tanította vala õt az anyja.
\par 2 Mit szóljak, fiam? mit, én méhem gyermeke? mit, én fogadásimnak gyermeke?
\par 3 Ne add asszonyoknak a te erõdet, és a te útaidat a királyok eltörlõinek.
\par 4 Távol legyen a királyoktól, oh Lemuel, távol legyen a királyoktól a bornak itala; és az uralkodóktól a részegítõ ital keresése.
\par 5 Hogy mikor iszik, el ne felejtkezzék a törvényrõl, és el ne fordítsa valamely nyomorultnak igazságát.
\par 6 Adjátok a részegítõ italt az elveszendõnek, és a bort a keseredett szívûeknek.
\par 7 Igyék, hogy felejtkezzék az õ szegénységérõl, és az õ nyavalyájáról ne emlékezzék meg többé.
\par 8 Nyisd meg a te szádat a mellett, a ki néma, és azoknak dolgában, a kik adattak veszedelemre.
\par 9 Nyisd meg a te szádat, ítélj igazságot; forgasd ügyét a szegénynek és a szûkölködõnek!
\par 10 Derék asszonyt kicsoda találhat? Mert ennek ára sokkal felülhaladja az igazgyöngyöket.
\par 11 Bízik ahhoz az õ férjének lelke, és annak marhája el nem fogy.
\par 12 Jóval illeti õt és nem gonosszal, az õ életének minden napjaiban.
\par 13 Keres gyapjat vagy lent, és megkészíti azokat kezeivel kedvvel.
\par 14 Hasonló a kereskedõ hajókhoz, nagy messzirõl behozza az õ eledelét.
\par 15 Felkel még éjjel, eledelt ád az õ házának, és rendel ételt az õ szolgálóleányinak.
\par 16 Gondolkodik mezõ felõl, és megveszi azt; az õ kezeinek munkájából szõlõt plántál.
\par 17 Az õ derekát felövezi erõvel, és megerõsíti karjait.
\par 18 Látja, hogy hasznos az õ munkálkodása; éjjel sem alszik el az õ világa.
\par 19 Kezeit veti a fonókerékre, és kezeivel fogja az orsót.
\par 20 Markát megnyitja a szegénynek, és kezeit nyújtja a szûkölködõnek.
\par 21 Nem félti az õ házanépét a hótól; mert egész házanépe karmazsinba öltözött.
\par 22 Szõnyegeket csinál magának; patyolat és bíbor az õ öltözete.
\par 23 Ismerik az õ férjét a kapukban, mikor ül a tartománynak véneivel.
\par 24 Gyolcsot szõ, és eladja; és övet, melyet ád a kereskedõnek.
\par 25 Erõ és ékesség az õ ruhája; és nevet a következõ napnak.
\par 26 Az õ száját bölcsen nyitja meg, és kedves tanítás van nyelvén.
\par 27 Vigyáz a házanépe dolgára, és restségnek étkét nem eszi.
\par 28 Felkelnek az õ fiai, és boldognak mondják õt; az õ férje, és dicséri õt:
\par 29 Sok leány munkálkodott serénységgel; de te meghaladod mindazokat!
\par 30 Csalárd a kedvesség, és hiábavaló a szépség; a mely asszony féli az Urat, az szerez dicséretet magának!
\par 31 Adjatok ennek az õ keze munkájának gyümölcsébõl, és dicsérjék õt a kapukban az õ cselekedetei!


\end{document}