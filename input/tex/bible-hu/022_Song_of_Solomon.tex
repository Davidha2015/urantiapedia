\begin{document}

\title{Énekek éneke}


\chapter{1}

\par 1 Énekek éneke, mely Salamoné.
\par 2 Csókoljon meg engem az õ szájának csókjaival; mert a te szerelmeid jobbak a bornál.
\par 3 A te drága kenetid jók illatozásra; a te neved kiöntött drága kenet; azért szeretnek téged a leányok.
\par 4 Vonj engemet te uánad, hadd fussunk! Bevitt engem a király az õ ágyasházába; örvendezünk  és vígadunk te benned, elõszámláljuk a te szerelmeidet, melyek jobbak a bornál, méltán szeretnek téged.
\par 5 Fekete vagyok, de szép, Jeruzsálem leányai; mint Kédár sátrai és Salamon szõnyegei.
\par 6 Ne nézzetek engem, hogy én fekete vagyok, hogy a nap lesütött engem; az én anyámnak fiai ellenem megharagudtak, a szõlõknek õrizõjévé tettek engem, - a magam szõlõjét nem õriztem.
\par 7 Mondd meg nékem, te, a kit az én lelkem szeret, hol legeltetsz, hol deleltetsz délben; mert miért legyek én olyan, mint a ki elfátyolozza magát, társaid nyájainál?
\par 8 Mivelhogy nem tudod, oh asszonyok között legszebb! jõjj ki a nyájnak nyomdokain, és õrizd a te kecskéidet a pásztoroknak sátorai körül.
\par 9 A Faraó szekereiben való paripákhoz hasonlítlak téged, én mátkám.
\par 10 Szépek a te orczáid a halántékra való lánczokban, a te nyakad a gyöngysorokban.
\par 11 Arany lánczokat csinálunk néked, ezüstbõl csinált gyöngyökkel.
\par 12 Mikor a király az õ asztalánál ül, nárdusnak jóillatja származik én tõlem.
\par 13 Olyan az én szerelmesem nékem, mint egy kötés mirha, mely az én kebeleim között hál.
\par 14 Mint az Engedi szõlõiben a cziprusfürt, olyan nékem az én szerelmesem.
\par 15 Ímé, szép vagy én mátkám, ímé, szép vagy, a te szemeid olyanok, mint a galambok.
\par 16 Ímé, te is szép vagy én szerelmesem, gyönyörûséges, és a mi nyoszolyánk zöldellõ.
\par 17 A mi házainknak gerendái czédrusfák, és a mi mennyezetünk cziprusfa.

\chapter{2}

\par 1 Én Sáronnak rózsája vagyok, és a völgyek lilioma.
\par 2 Mint a liliom a tövisek közt, olyan az én mátkám a leányok közt.
\par 3 Mint az almafa az erdõnek fái közt, olyan az én szerelmesem az ifjak közt. Az õ árnyékában felette igen kivánok ülni; és az õ gyümölcse gyönyörûséges az én ínyemnek.
\par 4 Bevisz engem a borozó házba, és zászló felettem a szerelme.
\par 5 Erõsítsetek engem szõlõvel, üdítsetek fel engem almákkal; mert betege vagyok a szerelemnek.
\par 6 Az õ balkeze az én fejem alatt van, és jobbkezével megölel engem.
\par 7 Kényszerítlek titeket, Jeruzsálemnek leányai, a vadkecskékre és a mezõnek szarvasira: fel ne költsétek és fel ne serkentsétek a szerelmet addig, a míg akarja.
\par 8 Az én szerelmesemnek szavát hallom, ímé, õ jõ, ugrálva a hegyeken, szökellve a halmokon!
\par 9 Hasonlatos az én szerelmesem az õzhöz, vagy a szarvasoknak fiához. Ímé, ott áll a mi falunkon túl, néz az ablakon keresztül, tekintget a rostélyokon keresztül,
\par 10 Szóla az én szerelmesem nékem, és monda: kelj fel én mátkám, és szépem, és jöszte.
\par 11 Mert ímé a tél elmult, az esõ elmult, elment.
\par 12 Virágok láttatnak a földön, az éneklésnek ideje eljött, és a gerliczének szava hallatik a mi földünkön.
\par 13 A fügefa érleli elsõ gyümölcsét, és a szõlõk virágzásban vannak, jóillatot adnak; kelj fel én mátkám, én szépem, és jõjj hozzám!
\par 14 Én galambom, a kõsziklának hasadékiban, a magas kõszálnak rejtekében, mutasd meg nékem a te orczádat, hadd halljam a te szódat; mert a te szód gyönyörûséges, és a te tekinteted ékes!
\par 15 Fogjátok meg nékünk a rókákat, a rókafiakat, a kik a szõlõket elpusztítják; mert a mi szõlõink virágban vannak.
\par 16 Az én szerelmesem enyém, és én az övé, a ki a liliomok közt legeltet.
\par 17 Míglen meghûsül a nap és az árnyékok elmúlnak: térj meg és légy hasonló, én szerelmesem, az õzhöz, vagy a szarvasoknak fiához a Béther hegyein.

\chapter{3}

\par 1 Az én ágyasházamban éjjeleken keresém azt, a kit szeret az én lelkem, keresém õt, és meg nem találtam.
\par 2 Immár felkelek és eljárom a várost, a tereket és az utczákat, keresem azt, a kit szeret az én lelkem; keresém õt, és nem találám.
\par 3 Megtalálának engem az õrizõk, a kik a várost kerülik. Mondék nékik: Láttátok-é azt, a kit az én lelkem szeret?
\par 4 Alig mentem vala el azoktól, mikor megtaláltam azt, a kit az én lelkem szeret. Megragadám õt, el sem bocsátám, mígnem bevivém õt anyám házába, és az én szülõmnek ágyasházába.
\par 5 Kényszerítelek titeket, Jeruzsálemnek leányai, a vadkecskékre, és a mezõknek szarvasira, fel ne költsétek és fel ne serkentsétek a szerelmet, valamíg õ akarja.
\par 6 Kicsoda az, a ki feljõ a pusztából, mint a füstnek oszlopa? mirhától és tömjéntõl illatos, a patikáriusnak minden jó illatú porától.
\par 7 Ímé, ez a Salamon gyaloghintaja, hatvan erõs férfi van körülötte, Izráelnek erõsei közül!
\par 8 Mindnyájan fegyverfogók, hadakozásban bölcsek, kinek-kinek oldalán fegyvere, az éjszakának félelme ellen.
\par 9 Hálóágyat csinált magának Salamon király a Libánus fáiból.
\par 10 Oszlopait ezüstbõl csinálta, oldalát aranyból, ágyát biborból, belsõ része ki van rakva szeretettel, a Jeruzsálemnek leányi által.
\par 11 Jõjjetek ki, és nézzétek, Sionnak leányai, Salamon királyt a koronában, melylyel megkoronázta õt az anyja, az õ eljegyzésének napjára, és az õ szíve vígasságának napjára!

\chapter{4}

\par 1 Ímé szép vagy, én mátkám, ímé szép vagy, a te szemeid galambok a te fátyolod mögött; a te  hajad hasonló a kecskéknek nyájához, melyek a Gileád hegyérõl szállanak alá.
\par 2 A te fogaid hasonlók a megnyirt juhok nyájához, melyek a fördõbõl feljõnek, melyek mind kettõsöket ellenek, és nincsen azok között meddõ.
\par 3 Mint a karmazsin czérna, a te ajkaid, és a te beszéded kedves, mint a pomagránátnak darabja, olyan a te vakszemed a te fátyolod alatt.
\par 4 Hasonló a te nyakad a Dávid tornyához, a mely építtetett fegyveres háznak, a melyben ezer paizs függesztetett fel, mind az erõs vitézek paizsai.
\par 5 A te két emlõd olyan, mint két vadkecske, egy zergének kettõs fia, a melyek a liliomok közt legelnek.
\par 6 Míg meghûsül a nap, és elmulnak az árnyékok, elmegyek a mirhának hegyére, és a tömjénnek halmára.
\par 7 Mindenestõl szép vagy, én mátkám, és semmi szeplõ nincs benned!
\par 8 Én velem a Libánusról, én jegyesem, én velem a Libánusról eljõjj; nézz az Amanának hegyérõl, a Sénirnek és Hermonnak tetejérõl, az oroszlánoknak barlangjokból, a párduczoknak hegyeirõl.
\par 9 Megsebesítetted az én szívemet, én húgom, jegyesem, megsebesítetted az én szívemet a te szemeidnek egy tekintésével, a te nyakadon való egy aranylánczczal!
\par 10 Mely igen szépek a te szerelmeid, én húgom, jegyesem! mely igen jók a te szerelmeid! jobbak a bornál, és a te keneteidnek illatja minden fûszerszámnál!
\par 11 Színmézet csepegnek a te ajkaid, én jegyesem, méz és tej van a te nyelved alatt, és a te ruháidnak illatja, mint a Libánusnak illatja.
\par 12 Olyan, mint a berekesztett kert az én húgom, jegyesem! mint a befoglaltatott forrás, bepecsételt kútfõ!
\par 13 A te csemetéid gránátalmás kert, édes gyümölcsökkel egybe, cziprusok nárdusokkal egybe.
\par 14 Nárdus és sáfrány, jóillatú nád és fahéj, mindenféle temjéntermõ fákkal, mirha és áloes, minden drága fûszerszámokkal.
\par 15 Kerteknek forrása, élõ vizeknek kútfeje, melyek folynak a Libánusról.
\par 16 Serkenj fel északi szél, és jõjj el déli szél, fújj az én kertemre, folyjanak annak drága illatú szerszámai, jõjjön el az én szerelmesem az õ kertébe, és egye annak drágalátos gyümölcsét.

\chapter{5}

\par 1 Bementem az én kertembe, én húgom, jegyesem, szedem az én mirhámat, az én balzsamommal, eszem az én lépesmézemet az én mézemmel, iszom az én boromat az én tejemmel. Egyetek barátim, igyatok, és részegedjetek meg, szerelmesim!
\par 2 Én elaludtam, de lelkemben vigyázok vala, és ímé az én szerelmesemnek szava, a ki zörget, mondván: Nyisd meg nékem, én húgom, én mátkám, én galambom, én tökéletesem; mert az én fejem megrakodott harmattal, az én hajam az éjszakának harmatjával!
\par 3 Felelék én: Levetettem ruhámat, hogy-hogy öltözhetném fel? Megmostam lábaimat, mimódon keverném azokat a porba?
\par 4 Az én szerelmesem kezét benyujtá az ajtónak hasadékán, és az én belsõ részeim megindulának õ rajta.
\par 5 Felkelék én, hogy az én szerelmesemnek megnyissam, és az én kezeimrõl mirha csepeg vala, és az én ujjaimról folyó mirha a závár kilincsére.
\par 6 Megnyitám az én szerelmesemnek; de az én szerelmesem elfordult, elment; az én lelkem megindult az õ beszédén: keresém õt, de nem találám, kiáltám õt, de nem felele nékem!
\par 7 Megtalálának engem az õrizõk, a kik a várost kerülik, megverének engem, megsebesítének engem, elvevék az én felöltõmet tõlem a kõfalnak õrizõi.
\par 8 Kényszerítelek titeket, Jeruzsálemnek leányai, ha megtaláljátok az én szerelmesemet, mit mondotok néki? hogy én a szerelem betege vagyok!
\par 9 Micsoda a te szerelmesed egyéb szerelmesek felett, oh asszonyoknak szépe? Micsoda a te szerelmesed egyéb szerelmesek felett, hogy minket ilyen igen kényszerítesz?
\par 10 Az én szerelmesem fejér és piros, tízezer közül is kitetszik.
\par 11 Az õ feje, mint a választott drága megtisztított arany; fodor haja fekete, mint a hollónak.
\par 12 Az õ szemei mint a vízfolyás mellett való galambok, melyek tejben fürödnek, szép teljesen helyheztettek.
\par 13 Az õ orczája hasonlatos a drága füveknek táblájához, a melyeknek illatos plántákat nevelnek; az õ ajkai liliomok, melyekrõl csepegõ mirha foly.
\par 14 Az õ kezei aranyhengerek; melyek befoglaltattak topázba; az õ teste elefántcsontból való mû, zafirokkal megrakva.
\par 15 Az õ szárai márványoszlopok; melyek tiszta arany talpakra fundáltattak; az õ tekinteti, mint a Libánus; tetszetes mint a czédrusfa.
\par 16 Az õ ínye édességek, és õ mindenestõl fogva kívánatos! Ez az én szerelmesem, és ez az én barátom, oh Jeruzsálemnek leányai!

\chapter{6}

\par 1 Hová ment a te szerelmesed, oh asszonyoknak szépe? hová fordult a te szerelmesed, hogy keressük õt veled együtt?
\par 2 Az én szerelmesem, elment az õ kertébe, a drága füveknek táblái közé, hogy lakozzék a kertekben, és liliomokat szedjen.
\par 3 Én az én szerelmesemé vagyok, és az én szerelmesem enyim, a ki a liliomok közt legeltet.
\par 4 Szép vagy én mátkám, mint Tirsa városa, kedves, mint Jeruzsálem, rettenetes, mint a zászlós tábor.
\par 5 Fordítsd el a te szemeidet én tõlem, mert azok megzavarnak engem. A te hajad olyan, mint a kecskéknek nyája, melyek a Gileádról szállanak alá.
\par 6 A te fogaid hasonlók a juhok nyájához, melyek feljõnek a fördõbõl, melyek mind kettõsöket ellenek, és meddõ azok között nincsen.
\par 7 Mint a pomagránát darabja a te vakszemed, a te fátyolod alatt.
\par 8 Havanan vannak a királynék, és nyolczvanan az ágyasok és számtalan a leányzó.
\par 9 És az én galambom, az én tökéletesem, az õ anyjának egyetlenegye, az õ szülõjének választottja. Látják a leányok, és boldognak mondják õt, a királynéasszonyok és az ágyasok, és dicsérik õt.
\par 10 Kicsoda az, a ki úgy láttatik mintegy hajnal, szép, mint a hold, tiszta, mint a nap, rettenetes, mint a zászlós tábor?
\par 11 A diófás kertekbe mentem vala alá, hogy a völgynek zöld fûveit lássam; hogy megnézzem, ha fakad-é a szõlõ, és a pomagránátfák virágzanak-é?
\par 12 Nem tudtam, hogy az én elmém ültete engem az én nemes népemnek díszhintajába.
\par 13 Térj meg, oh Sulamit! térj meg, térj meg, hogy nézzünk téged! Mit néztek Sulamiton? mintegy Machanaimbeli körtánczot!

\chapter{7}

\par 1 Oh mely szépek a te lépéseid a sarukban, oh fejedelem leánya! A te csípõd hajlásai olyanok, mint a kösöntyûk, mesteri kezeknek míve.
\par 2 A te köldököd, mint a kerekded csésze, nem szûkölködik nedvesség nélkül; a te hasad mint a gabonaasztag, liliomokkal körül kerítve.
\par 3 A te két emlõd, mint két õzike, a vadkecskének kettõs fiai.
\par 4 A te nyakad, mint az elefánttetembõl csinált torony; a te szemeid, mint a Hesbonbeli halastók, a sok népû kapunál; a te orrod hasonló a Libánus tornyához, mely néz Damaskus felé.
\par 5 A te fejed hasonló rajtad a Kármelhez, és a te fejeden hajadnak fonatékja a biborhoz, a király is megköttetnék fürteid által!
\par 6 Mely igen szép vagy és mely kedves, oh szerelem, a gyönyörûségek közt!
\par 7 Ez a te termeted hasonló a pálmafához, és a te emlõid a szõlõgerézdekhez.
\par 8 Azt mondám: felhágok a pálmafára, megfogom annak ágait: és lesznek a te emlõid, mint szõlõnek gerézdei, és a te orrodnak illatja, mint az almának.
\par 9 És a te ínyed, mint a legjobb bor, melyet szerelmesem kedvére szürcsöl, mely szóra nyitja az alvók ajkait.
\par 10 Én az én szerelmesemé vagyok, és engem kiván õ!
\par 11 No, én szerelmesem, menjünk ki a mezõre, háljunk a falukban.
\par 12 Felkelvén menjünk a szõlõkbe, lássuk meg, ha fakad-é a szõlõ, ha kinyílott-é virágja, ha virágzanak-é a gránátalmafák: ott közlöm az én szerelmimet veled.
\par 13 A mandragórák illatoznak, és a mi ajtónk elõtt vannak minden drágalátos gyümölcsök, ók és újak, melyeket oh én szerelmesem, néked megtartottam!

\chapter{8}

\par 1 Vajha lennél nékem én atyámfia, ki az én anyámnak emlõjét szopta, hogy téged kivül találván megcsókolnálak; még sem útálnának meg engem.
\par 2 Elvinnélek, bevinnélek anyámnak házába, te oktatgatnál engem, én meg borral itatnálak, fûszeressel, gránátalma borral.
\par 3 Az õ balkeze az én fejem alatt, és jobbkezével megölel engem.
\par 4 Kényszerítlek titeket, Jeruzsálemnek leányai, miért költenétek és miért serkentenétek fel a szerelmet, mígnem õ akarja?
\par 5 Kicsoda ez a ki feljõ a pusztából, a ki az õ szerelmeséhez támaszkodik? Az almafa alatt költöttelek fel téged, ott szült téged a te anyád, ott szült téged a te szülõd!
\par 6 Tégy engem mintegy pecsétet a te szívedre, mintegy pecsétet a te karodra; mert erõs a szeretet, mint a halál, kemény, mint a sír a buzgó szerelem; lángjai tûznek lángjai, az Úrnak lángjai.
\par 7 Sok vizek el nem olthatnák e szeretetet: a folyóvizek sem boríthatnák azt el: ha az ember minden házabeli marháját adná is e szeretetért, mégis megvetnék azt.
\par 8 Kicsiny húgunk van nékünk, a kinek nincsen még emlõje; mit cselekedjünk a mi húgunk felõl, a napon melyen arról szót tesznek?
\par 9 Ha õ kõfal, építünk azon ezüstbõl palotát; ha pedig ajtó õ, elrekesztjük õt czédrus-deszkával.
\par 10 Mikor én olyan leszek, mint a kõfal, és az én emlõim, mint a tornyok; akkor olyan leszek õ elõtte, mint a ki békességet nyer.
\par 11 Szõlõje volt Salamonnak Baálhamonban, adta az õ szõlejét a pásztoroknak, kiki annak gyümölcséért hoz ezer-ezer ezüst siklust.
\par 12 Az én szõlõmre, mely én reám néz, nékem gondom lesz: az ezer siklus, Salamon, tiéd legyen, a kétszáz annak gyümölcsének õrizõié.
\par 13 Oh te, a ki lakol a kertekben! A te társaid a te szódra figyelmeznek; hadd halljam én is.
\par 14 Fuss én szerelmesem, és légy hasonló a vadkecskéhez, vagy a szarvasnak fiához, a drága füveknek hegyein!


\end{document}