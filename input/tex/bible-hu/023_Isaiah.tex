\begin{document}

\title{Isaiah}


\chapter{1}

\par 1 Ésaiásnak, Ámós fiának látása, melyet látott Júda és Jeruzsálem felõl, Uzziásnak, Jóthámnak, Akháznak és Ezékiásnak, a Júda királyainak napjaiban.
\par 2 Halljátok egek, és vedd füleidbe föld! mert az Úr szól: Fiakat neveltem, s  méltóságra emeltem, és õk elpártolának tõlem.
\par 3 Az ökör ismeri gazdáját, és a szamár az õ urának jászlát; Izráel nem ismeri, az én népem nem figyel reá!
\par 4 Oh gonosz nemzetség, hamissággal megterheltetett nép, gonosz mag, nemtelen fiak! elhagyták  az Urat, megútálták az Izráel Szentjét, és elfordultak tõle.
\par 5 Miért ostorozzalak tovább, holott a bûnt növelitek? Minden fej beteg, és minden szív erõtelen.
\par 6 Tetõtõl talpig nincs e testben épség, csupa seb és dagadás és kelevény, a melyeket ki sem nyomtak, be sem kötöztek, olajjal sem lágyítottak.
\par 7 Országtok pusztaság, városaitokat tûz perzselé föl, földeteket szemetek láttára idegenek emésztik, és pusztaság az, mint a hol idegenek dúltak;
\par 8 És úgy maradt a Sion leánya, mint kunyhó a szõlõben, mint kaliba az ugorkaföldön, mint megostromlott város.
\par 9 Ha a seregeknek Ura valami keveset meg nem hagyott volna bennünk, úgy jártunk volna mint Sodoma, és Gomorához  volnánk hasonlók.
\par 10 Halljátok az Úrnak beszédét, Sodoma fejedelmei, és vedd füleidbe Istenünk tanítását, Gomora népe!
\par 11 Mire való nékem véres áldozataitoknak sokasága? ezt mondja az Úr; megelégeltem a kosok egészen égõáldozatait és a hízlalt barmok kövérét; s a tulkok, bárányok és bakok vérében nem gyönyörködöm;
\par 12 Ha eljöttök, hogy színem elõtt megjelenjetek, ki kivánja azt tõletek, hogy pitvarimat tapossátok?
\par 13 Ne hozzatok többé hazug ételáldozatot, a jó illattétel útálat elõttem; újhold, szombat s ünnepre-felhívás: bûnt és ünneplést el nem szenvedhetek.
\par 14 Újholdaitokat és ünnepeiteket gyûlöli lelkem; terhemre vannak, elfáradtam viselni.
\par 15 És ha kiterjesztitek kezeiteket, elrejtem szemeimet elõletek; sõt ha megsokasítjátok is az imádságot, én meg nem hallgatom: vérrel  rakvák kezeitek.
\par 16 Mosódjatok, tisztuljatok meg, távoztassátok el szemeim elõl cselekedeteitek gonoszságát, szünjetek meg gonoszt  cselekedni;
\par 17 Tanuljatok jót tenni; törekedjetek igazságra, vezessétek jóra az erõszakoskodót, pártoljátok az árvák és özvegyek ügyét.
\par 18 No jertek, törvénykezzünk, azt mondja az Úr! ha bûneitek skárlátpirosak, hófehérek lesznek, és ha vérszínûek, mint a karmazsin, olyanok lesznek, mint a gyapjú.
\par 19 Ha engedelemmel hallgatándotok, e föld javaival éltek;
\par 20 És ha vonakodtok, sõt pártot üttök, fegyver emészt meg; mert az Úr szája szólt!
\par 21 Mint lett paráznává a hív város! teljes vala jogossággal, igazság lakozott benne, és most gyilkosok!
\par 22 Ezüstöd salakká lett, tiszta borod vízzel elegyítve:
\par 23 Fejedelmid megátalkodottak és lopóknak társai; mind szereti az ajándékot és vesztegetést hajhász, árvát nem pártolnak, és az özvegy ügye nem kerül eléjök.
\par 24 Ezért azt mondja az Úr, a seregeknek Ura, Izráel erõs Istene: Jaj! mert vígasztalást veszek háborgatóimon, és bosszút állok ellenségimen!
\par 25 És kezemet ellened fordítom, és kiolvasztom mintegy lúggal salakodat, és eltávolítom minden ólmodat;
\par 26 És adok néked oly birákat, mint régen, és oly tanácsosokat, mint kezdetben, s ekkor azt mondják te néked: ez igaz város, ez hív város.
\par 27 Sion jogosság által váltatik meg, és megtérõi igazság által;
\par 28 De elvesznek a bûnösök és gonoszok egyetemben, s megemésztetnek, a kik az Urat elhagyták.
\par 29 Mert szégyen éri õket a cserfákért, a melyekben gyönyörködétek, és pirulni fogtok a kertek miatt, a melyeket kedveltek;
\par 30 És hasonlatosok lesztek az elhullott levelû terpentinfához, és a víz nélkül való kerthez:
\par 31 És csepüvé lesz az erõs, és munkája szikrává: mindketten égni fognak, és oltójok nem lészen.

\chapter{2}

\par 1 Ésaiásnak, Ámós fiának beszéde, a melyet látott Júda és Jeruzsálem felõl.
\par 2 Lészen az utolsó idõkben, hogy erõsen fog állani az Úr házának hegye, hegyeknek felette, és magasabb lészen a halmoknál, és özönleni fognak hozzá minden pogányok;
\par 3 És eljönnek sok népek, mondván: Jertek menjünk fel az Úr hegyére, Jákób Istenének házához, hogy megtanítson minket az Õ útaira, és mi járjunk az Õ ösvényein, mert tanítás Sionból jõ, és Jeruzsálembõl az Úrnak beszéde;
\par 4 Ki ítéletet tesz a pogányok között, és bíráskodik sok nép felett; és csinálnak fegyvereikbõl kapákat, és dárdáikból metszõkéseket, és nép népre kardot nem emel, és hadakozást többé nem tanul.
\par 5 Jákóbnak háza! jertek járjunk az Úrnak világosságában!
\par 6 Mert elhagytad, Uram, a Te népedet, Jákób házát; mivel telvék napkeleti erkölcsökkel, és szemfényvesztõk, mint a  Filiszteusok, és idegenek fiaival kötnek szövetséget;
\par 7 És betelt földje ezüsttel és aranynyal, és nincs szere-száma kincseinek; betelt földje lovakkal, és nincs szere-száma szekereinek;
\par 8 És betelt földje bálványokkal, és kezeik csinálmányának hajolnak meg, mit ujjaik csináltak.
\par 9 Ezért porba hajtatik a közember, és megaláztatik a fõember, és meg nem bocsátsz nékik.
\par 10 Menj be a kõsziklába, és rejtõzzél el a porba az Úr félelme elõl, és az Õ nagyságának dicsõsége elõtt.
\par 11 A kevély szemû közember megaláztatik, és a fõemberek magassága porba hajtatik, és csak az Úr magasztaltatik fel ama napon.
\par 12 Mert a seregek Urának napja eljõ minden kevély és magas ellen, és minden felemelkedett ellen, és az megaláztatik.
\par 13 És Libánon minden czédrusai ellen, a melyek magasak és felemelkedettek, s Básánnak minden tölgyfái ellen;
\par 14 Minden magas hegyek ellen, és minden felemelkedett halmok ellen;
\par 15 Minden magas torony ellen, és minden erõs kõfal ellen;
\par 16 És Társisnak minden hajói ellen, és minden gyönyörûséges drágaságok ellen.
\par 17 És porba hajtatik a közember kevélysége, és megaláztatik a fõemberek magassága, és egyedül az Úr magasztaltatik fel ama napon.
\par 18 És a bálványokat teljességgel elveszti.
\par 19 És bemennek a sziklák barlangjaiba és a föld hasadékaiba, az Úr félelme elõl és nagyságának dicsõsége elõtt, mikor felkél, hogy megrettentse a földet.
\par 20 Ama napon odadobja az ember ezüst bálványait és arany bálványait, miket magának csinált, hogy azok elõtt meghajoljon, a vakondokoknak és denevéreknek,
\par 21 Hogy elmenjen a sziklák lyukaiba és a hegyek hasadékiba, az Úr félelme elõl és az Õ nagyságának dicsõsége elõtt, mikor felkél, hogy megrettentse a földet.
\par 22 Oh szünjetek meg hát az emberben bízni, a kinek egy lehellet van orrában, mert hát ugyan mire becsülhetõ õ?

\chapter{3}

\par 1 Mert ímé az Úr, a seregeknek Ura elveszi Jeruzsálembõl és Júdából a támaszt és a táplálót, a kenyérnek minden erejét és a víznek minden erejét;
\par 2 Az erõst és hadakozót, birót és prófétát, jóst és öreget;
\par 3 Az ötvenedes hadnagyot, a tisztes embert, a tanácsost, az ügyes mestert és a varázsláshoz értõt;
\par 4 És adok nékik gyerkõczöket fejedelmekül, és gyermekek uralkodnak rajtok.
\par 5 És nyomorba jut a nép; egyik a másik ellen, ki-ki az õ társa ellen támad, a gyermek az öreg ellen, becstelen a tisztes ellen;
\par 6 És ha valaki megragadja atyja nemzetségébõl való rokonát, mondván: Néked ruhád van, légy fejedelmünk és e romlás kezed alatt legyen:
\par 7 Ez ama napon így fog felkiáltani: Nem leszek sebkötözõ, házamban is nincsen kenyér és nincs ruha; ne tegyetek engem e nép fejedelmévé!
\par 8 Mert megromlott Jeruzsálem, és Júda elesett, mert nyelvök és cselekedeteik az Úr ellen vannak, hogy az Õ dicsõséges szemeit ingereljék.
\par 9 Arczuk tekintete tesz ellenök bizonyságot, bûneikkel Sodoma módjára kérkednek, nemhogy eltitkolnák; jaj lelköknek! mert magok szereztek magoknak gonoszt.
\par 10 Mondjátok az igaznak, hogy jól lészen dolga, mert cselekedeteik gyümölcsével élnek.
\par 11 Jaj a gonosznak, gonoszul lesz dolga, mert kezeinek cselekedete szerint fizetnek néki.
\par 12 Népem nyomorgatói gyermekek, és asszonyok uralkodnak rajta; népem! a te vezéreid hitetõk, és ösvényidnek útját elrejtik elõled.
\par 13 Elõálla perelni az Úr, és itt áll ítélni népeket.
\par 14 Az Úr törvénybe megy népe véneivel és fejedelmivel: Hiszen ti lelegeltétek a szõlõt, szegénytõl rablott marha van házaitokban:
\par 15 Mi dolog, hogy népemet összezúzzátok, és a szegények orczáját összetöritek? ezt mondja az Úr, a seregeknek Ura.
\par 16 És szól az Úr: Mivel Sion leányai felfuvalkodtak, és felemelt nyakkal járnak, szemeikkel pillognak, és aprókat lépve járnak, és lábokkal nagy zengést bongást szereznek:
\par 17 Megkopaszítja az Úr Sion leányainak fejtetõjét, és az õ szemérmöket megmezteleníti.
\par 18 Ama napon eltávolítja az Úr az õ lábaik zengõ ékességét, a napocskákat és holdacskákat,
\par 19 A fülönfüggõket, a karpereczeket és a fátyolokat,
\par 20 A pártákat, a láblánczokat, az öveket, a jóillattartókat és az ereklyéket,
\par 21 A gyûrûket és az orrpereczeket,
\par 22 Az ünneplõ ruhákat, a palástokat, a nagy kendõket és az erszényeket,
\par 23 A tükröket, a gyolcsingeket, a fõkötõket és a keczeléket:
\par 24 És lesz a balzsamillat helyén büdösség, az öv helyén kötél, és a felfodrozott haj helyén kopaszság, a szép köpenynek helyén zsákruha, és a szépség helyén homlokra sütött bélyeg.
\par 25 Férfiaid fegyver által hullnak el, és vitézeid harczban:
\par 26 És sírnak és gyászolnak kapui, és õ elpusztíttatván, a földön ül;

\chapter{4}

\par 1 És megragad hét asszony egy férfit ama napon, mondván: A magunk kenyerét eszszük és a mi ruhánkba öltözködünk, csak hadd neveztessünk a te nevedrõl, és vedd le rólunk gyalázatunkat!
\par 2 És ama napon az Úrnak csemetéje ékes és dicsõséges lészen, és a földnek gyümölcse nagyságos és díszes  Izráel maradékának.
\par 3 És lészen, hogy a ki Sionban meghagyatik, és Jeruzsálemben megmarad, szentnek mondatik, minden, valaki Jeruzsálemben az élõk közé beiratott.
\par 4 Ha elmosta az Úr Sion leányainak undokságát, és Jeruzsálem vérét eltisztítá belõle az ítélet lelkével, a megégetés lelkével:
\par 5 Akkor teremteni fog az Úr Sion hegyének minden helye fölé és gyülekezetei fölé nappal felhõt és ködöt, s lángoló tûznek fényességét éjjel; mert ez egész dicsõségen oltalma lészen;
\par 6 És sátor lészen árnyékul nappal a hõség ellen, s oltalom és rejtek szélvész és esõ elõl.

\chapter{5}

\par 1 Hadd énekelek kedvesemrõl, szerelmesemnek énekét az Õ szõlõjérõl! Kedvesemnek szõlõje van nagyon kövér hegyen;
\par 2 Felásta és megtisztítá kövektõl, nemes vesszõt plántált belé, és közepére tornyot építtetett, sõt benne már sajtót is vágatott; és várta, hogy majd jó szõlõt terem, és az vadszõlõt termett!
\par 3 Mostan azért, Jeruzsálem lakosi és Juda férfiai: ítéljetek köztem és szõlõm között!
\par 4 Mit kellett volna még tennem szõlõmmel; mit meg nem tettem vele? Miért vártam, hogy jó szõlõt terem, holott vadat termett?!
\par 5 Azért most tudatom veletek, hogy mit teszek szõlõmmel; elvonszom kerítését, hogy lelegeltessék, elrontom kõfalát, hogy eltapodtassék;
\par 6 És parlaggá teszem; nem metszetik és nem kapáltatik meg, tövis és gaz veri föl, és parancsolok a fellegeknek, hogy rá esõt ne adjanak!
\par 7 A seregek Urának szõlõje pedig Izráel háza, és Júda férfiai az Õ gyönyörûséges ültetése; és várt jogõrzésre, s ím lõn jogorzás; és irgalomra, s ím lõn siralom!
\par 8 Jaj azoknak, a kik a házhoz házat ragasztanak, és mezõt foglalnak a mezõkhöz, míg egy helyecske sem marad, és csak ti magatok laktok itt e földön!
\par 9 Hallásomra megesküdt a seregeknek Ura, hogy sok házak pusztasággá lesznek, a nagyok és szépek lakos nélkül,
\par 10 Mert tíz hold szõlõ egy báth bort ereszt, és egy hómer mag egy efát terem.
\par 11 Jaj azoknak, a kik jó reggelen részegítõ ital után futkosnak, és mulatnak estig, és bor hevíti õket:
\par 12 És czitera, lant, dob, síp és bor van lakodalmokban, de az Úrnak dolgát nem tekintik meg, és nem látják kezeinek cselekedetét.
\par 13 Ezért fogságba megy népem, mivelhogy tudomány nélkül való, és fõemberei éhen halnak, és sokasága szomjú miatt eped meg;
\par 14 Azért a sír kiszélesíti torkát és feltátja száját szertelen, és leszállnak abba népem fõemberei, és zajongó sokasága, és minden örvendezõi;
\par 15 És porba hajtatik a közember, és megaláztatik a fõember, és a nagyok szemei megaláztatnak.
\par 16 És felmagasztaltatik a seregeknek Ura az ítéletben, s a szent Isten megszenteli magát igazságban.
\par 17 És bárányok legelnek ott, mint legelõjükön, s a gazdagoknak romjain idegenek élnek!
\par 18 Jaj azoknak, a kik a vétket hazugságnak kötelein vonszák, és a bûnt mint szekeret köteleken;
\par 19 A kik ezt mondják: siessen és tegye hamar munkáját, hogy lássuk, s közelítsen és jõjjön el Izráel Szentjének tanácsa, hogy tudjuk meg.
\par 20 Jaj azoknak, a kik a gonoszt jónak mondják és a jót gonosznak; a kik a sötétséget világossággá s a világosságot sötétséggé teszik, és teszik a keserût édessé, s az édest keserûvé!
\par 21 Jaj azoknak, a kik magoknak bölcseknek látszanak, és eszesek önnön magok elõtt!
\par 22 Jaj azoknak, a kik hõsök borivásban és híresek részegítõ ital vegyítésében;
\par 23 A kik a gonoszt ajándékért igaznak mondják, és az igazak igazságát elfordítják tõlünk:
\par 24 Ezért, mint a polyvát megemészti a tûznek nyelve, és az égõ széna összeomlik: gyökerök megrothad, virágjok mint a por elszáll, mert a seregek Urának törvényét megvetették, és Izráel szentjének beszédét megútálták.
\par 25 Ezért gerjedt fel az Úrnak haragja népe ellen, és felemelé rá kezét, és megveri, hogy a hegyek megrendülnek, és holttestök szemétként fekszik az utczán. Mindezekkel haragja el nem mult, és keze még felemelve van.
\par 26 És zászlót emel a távoli népeknek, és süvölt a föld határán lakozóknak, és ímé hamarsággal könnyen eljõnek.
\par 27 Nem lesz köztük egy is elfáradott, sem tántorgó; nem szunnyad, és nem aluszik; derekának öve sem oldódik meg, és nem szakad el saruja szíja sem;
\par 28 Nyilai élesek, és minden õ kézívei felvonvák, lovai körme, miként a kova, és kerekei mint a forgószél;
\par 29 Ordítása, mint az oroszláné, és ordít, mint az oroszlán-kölykök, és morog s prédát ragad s elviszi, és nincs, a ki elvegye tõle;
\par 30 És rámordul ama napon, mint tenger mormmolása; és õ a földre néz, de ímé ott sûrû sötétség; és a nap meghomályosodik a ráborult homályban!

\chapter{6}

\par 1 A mely esztendõben meghala Uzziás király, látám az Urat ülni magas és felemeltetett székben, és palástja betölté a templomot;
\par 2 Szeráfok állanak vala felette: mindeniknek hat-hat szárnya vala: kettõvel orczáját fedé be, kettõvel lábait fedé be, és kettõvel lebegett;
\par 3 És kiált vala egy a másiknak, és mondá: Szent, szent, szent a seregeknek Ura, teljes mind a széles föld az õ dicsõségével!
\par 4 És megrendülének az ajtó küszöbei a kiáltónak szavától, és a ház betelt füsttel.
\par 5 Akkor mondék: Jaj nékem, elvesztem, mivel tisztátalan ajkú vagyok és tisztátalan ajkú nép közt lakom: hisz a királyt, a seregeknek Urát láták szemeim!
\par 6 És hozzám repült egy a szeráfok közül, és kezében eleven szén vala, a melyet fogóval vett volt az oltárról;
\par 7 És illeté számat azzal, és mondá: Ímé ez illeté ajkaidat, és hamisságod eltávozott, és bûnöd elfedeztetett.
\par 8 És hallám az Úrnak szavát, a ki ezt mondja vala: Kit küldjek el és ki megyen el nékünk? Én pedig mondék: Ímhol vagyok én, küldj el engemet!
\par 9 És monda: Menj, és mondd ezt e népnek: Hallván halljatok és ne értsetek, s látván lássatok és ne ismerjetek;
\par 10 Kövérítsd meg e nép szívét, és füleit dugd be, és szemeit kend be: ne lásson szemeivel, ne halljon füleivel, ne értsen szívével, hogy meg ne térjen, és meg ne gyógyuljon.
\par 11 És én mondék: Meddig lészen ez Uram?! És monda: Míg a városok pusztán állanak lakos nélkül, és a házak emberek nélkül, s a föld is puszta lészen;
\par 12 És az Úr az embert messze elveti, s nagy pusztaság lészen a földön;
\par 13 És ha megmarad még rajta egy tizedrész, ismétlen elpusztul ez is; de mint a terpentinfának és cserfának törzsük marad kivágatás után: az õ törzsük szent mag lészen!

\chapter{7}

\par 1 És lõn a Júda királyának, Akháznak, az Uzziás fiának, Jóthám fiának napjaiban, eljöve Reczin, Sziriának királya, és Remaljának fia, Pékah, Izráelnek királya Jeruzsálem ellen, hogy azt megostromolja; de nem veheté meg azt.
\par 2 És hírül vivék a Dávid házának, mondván: Sziria Efraimmal egyesült! és megindula szíve s népének szíve, a mint megindulnak az erdõ fái a szél miatt.
\par 3 És mondá az Úr Ésaiásnak: Menj ki, kérlek, Akház eleibe te és Seár-Jásub fiad, a felsõ tó csatornájának végéhez, a ruhafestõk mezejének útján;
\par 4 És mondd néki: Vigyázz és légy nyugodt; ne félj! és meg ne lágyuljon szíved e két füstölgõ üszögdarab miatt, Reczinnek és Sziriának, és Remalja fiának felgerjedett haragja miatt!
\par 5 Mivelhogy gonosz tanácsot tartott ellened Sziria, Efraim és Remalja fia, mondván:
\par 6 Menjünk el Júda ellen, s reszkettessük meg, és kapcsoljuk magunkhoz, és tegyük királylyá benne Tábeal fiát,
\par 7 Így szól az Úr Isten: Nem áll meg és nem lészen ez!
\par 8 Mert Sziria feje Damaskus, és Damaskusnak feje Reczin, és még hatvanöt esztendõ, s megromol Efraim és nép nem lészen.
\par 9 Efraim feje pedig Samaria, és Samaria feje a Remalja fia: ha nem hisztek, bizony meg nem maradtok!
\par 10 És szóla ismét az Úr Akházhoz, mondván:
\par 11 Kérj jelt magadnak az Úrtól, a te Istenedtõl, kérj a mélységben vagy fent a magasban!
\par 12 És mikor szóla Akház: Nem kérek s nem kisértem az Urat!
\par 13 Akkor monda a próféta: Halld meg hát, Dávid háza! hát nem elég embereket bosszantanotok, hogy még az én Istenemet is bosszantjátok?
\par 14 Ezért ád jelt néktek az Úr maga: Ímé, a szûz fogan méhében, és szül fiat, s nevezi azt Immánuelnek,
\par 15 Ki vajat és mézet eszik, míg megtanulja a gonoszt megvetni, és a jót választani;
\par 16 Mert mielõtt e gyermek megtanulná megvetni a gonoszt, és a jót választani, elpusztul a föld, melynek két királyától te reszketsz.
\par 17 És hozni fog az Úr reád és népedre és atyád házára oly napokat, milyenek még nem jöttenek, mióta Efraim elszakadott Júdától: Assiria királyát.
\par 18 És lesz ama napon: süvölt az Úr az Égyiptom folyóvize mellett való legyeknek, s Assiria földje méheinek,
\par 19 S mind eljönnek, és letelepszenek a meredek völgyekben és sziklák hasadékaiban, minden töviseken és minden legelõkön.
\par 20 Ama napon leberetválja az Úr a folyón túl bérlett beretvával, Assiria királya által a fõt s a lábak szõrét, a mely a szakált is leveszi;
\par 21 És lesz ama napon, hogy kiki egy fejõstehenet tart és két juhot,
\par 22 És lészen, hogy a tej bõsége miatt vajat eszik; mert vajat és mézet eszik, valaki csak megmaradott e földön.
\par 23 S lészen ama napon, hogy minden helyet, a hol ezer szõlõtõ ezer siklust érõ vala, tövis és gaz ver fel.
\par 24 Nyilakkal és kézívvel mehet oda az ember, mivel tövis és gaz verte fel mind az egész földet;
\par 25 Sõt a hegyekre, a melyeket kapával kapáltanak, sem fogsz felmenni, félvén a tövistõl és gaztól; hanem barmok legelõjévé lesznek és juhok járásává.

\chapter{8}

\par 1 És monda nékem az Úr: Végy magadnak egy nagy táblát, és írd fel reá közönséges betûkkel: siess zsákmányra és gyorsan prédára;
\par 2 És én hív tanúkul választom magamnak Úriást, a papot és Zakariást, Jeberekiás fiát.
\par 3 És bementem a prófétaasszonyhoz, a ki fogant, és szült fiat; és mondá az Úr nékem: Nevezd nevét: siess zsákmányra és gyorsan prédára.
\par 4 Mert mielõtt e gyermek ezt ki tudja mondani: apám és anyám, Damaskus gazdagságát és Samaria prédáját Assiria királyának szolgája elviszi.
\par 5 Ismét szólott az Úr hozzám, mondván:
\par 6 Mivel megútálta e nép Siloahnak lassan folyó vizét, és Reczinben és Remalja fiában gyönyörködik:
\par 7 Azért ímé rájok hozza az Úr a folyónak erõs és sok vizét, Assiria királyát és minden õ hatalmát, és feljõ minden medre fölé, és foly minden partjai felett.
\par 8 És becsap Júdába, és megáradván átmegy rajta, s torkig ér, és elterjesztett szárnyai ellepik földednek szélességét, oh Immánuel!
\par 9 Fenekedjetek csak népek és romoljatok meg; figyeljetek, valakik messze laktok; készüljetek és megrontattok; készüljetek és megrontattok.
\par 10 Tanácskozzatok, de haszontalan lesz, beszéljetek beszédet, de nem áll meg,  mert Isten van mi velünk!
\par 11 Mert így szólott hozzám az Úr, rajtam lévén erõs keze, hogy tanítson engem, hogy e népnek útján ne járjak, mondván:
\par 12 Ti ne mondjátok összeesküvésnek, valamit e nép összeesküvésnek mond, és félelme szerint ne féljetek és ne rettegjetek;
\par 13 A seregek Urát: Õt szenteljétek meg, Õt féljétek, és Õt rettegjétek!
\par 14 És Õ néktek szenthely lészen; de megütközés köve és botránkozás sziklája Izráel két házának, s tõr és háló Jeruzsálem lakosainak.
\par 15 És megütköznek köztük sokan, s elesnek és összetöretnek; tõrbe esnek és megfogatnak!
\par 16 Kösd be e bizonyságtételt, és pecsételd be e tanítást tanítványaimban!
\par 17 Én pedig várom az Urat, a ki elrejté orczáját Jákób házától, és benne bízom.
\par 18 Ímhol vagyok én és a fiak, kiket adott nékem az Úr jelekül és csodákul Izráelben: a Sion hegyén lakozó sergeknek Urától vagyunk mi!
\par 19 És ha ezt mondják tinéktek: Tudakozzatok a halottidézõktõl és a jövendõmondóktól, a kik sipognak és suttognak: hát nem Istenétõl tudakozik-é a nép?  az élõkért a holtaktól kell-é tudakozni?
\par 20 A tanításra és bizonyságtételre hallgassatok! Ha nem ekként szólnak azok, a kiknek nincs hajnalok:
\par 21 Úgy bolyongani fognak a földön, szorongva és éhezve; és lészen, hogyha megéhezik, felgerjed és megátkozza királyát és Istenét, s néz fölfelé,
\par 22 És azután a földre tekint, és ímé mindenütt nyomor és sötétség, és szorongatásnak éjszakája, õ pedig a sûrû sötétben elhagyatva!

\chapter{9}

\par 1 De nem lesz mindig sötét ott, a hol most szorongatás van; elõször megalázta Zebulon és Nafthali földjét, de azután megdicsõíti a tenger útját, a Jordán túlsó partját és a pogányok határát.
\par 2 A nép, a mely sötétségben jár vala, lát nagy világosságot; a kik lakoznak a halál árnyékának földében, fény ragyog fel fölöttök!
\par 3 Te megsokasítod e népet, nagy örömöt szerzesz néki, és örvendeznek elõtted az aratók örömével, és vígadoznak, mint mikoron zsákmányt osztanak.
\par 4 Mert terhes igáját és háta vesszejét, az õt nyomorgatónak botját összetöröd, mint a Midián napján;
\par 5 Mert a vitézek harczi saruja és a vérbe fertõztetett öltözet megég, és tûznek eledele lészen;
\par 6 Mert egy gyermek születik nékünk, fiú adatik nékünk, és az uralom az õ vállán lészen, és hívják nevét: csodálatosnak,  tanácsosnak, erõs Istennek, örökkévalóság atyjának, békesség fejedelmének!
\par 7 Uralma növekedésének és békéjének nem lesz vége a Dávid trónján és királysága felett, hogy fölemelje és megerõsítse azt jogosság és igazság által mostantól mindörökké. A  seregek Urának buzgó szerelme mívelendi ezt!
\par 8 Beszédet küldött az Úr Jákóbnak, és leesett Izráelben,
\par 9 Hogy megértse az egész nép: Efraim és Samaria lakosa, a kik ezt mondják kevélyen és felfuvalkodva:
\par 10 Téglák omlottak le, és mi faragott kõbõl építünk; fügefák vágattak ki, és mi czédrusokat ültetünk helyökre!
\par 11 De az Úr ellenök hozza Reczin szorongatóit, és ellenségeiket rájok uszítja;
\par 12 A Sziriabeliek elõl, és a Filiszteusok hátul, s falják Izráelt feltátott torokkal. Mindezekkel azonban haragja el nem múlt, és keze még felemelve van.
\par 13 Hiszen e nép nem tért meg az õt verõ Istenhez, és a seregeknek Urát nem keresték;
\par 14 Ezért kivágja az Úr Izráelbõl a fõt és farkat, a pálmaágat és a kákát egy napon.
\par 15 A fõ: a vén és a fõember, a fark pedig a próféta, a ki hazugságot szól.
\par 16 Mert e nép vezérei hitetõkké lettek, és a kiket vezetének, elvesztek.
\par 17 Ezért ifjaiban sem gyönyörködik az Úr, s árváin és özvegyein sem könyörül; mert mindnyájan istentelenek és gonosztevõk, és minden száj bolondságot beszél. Mindezekkel haragja el nem múlt, és keze még felemelve van.
\par 18 Mert a gonoszság felgerjedt, mint a tûz, s tövist és gazt emészt, és meggyújtja a sûrû erdõt, és felgomolyg a füst oszlopában.
\par 19 A seregek Urának haragja miatt kiégett a föld, és a nép a tûznek eledele lõn: Senki atyjafián  nem könyörül,
\par 20 Jobbkézre vág és megéhezik, eszik balkézre és nem elégszik meg; mindnyájan karjoknak húsát eszik,
\par 21 Manassé Efraimot és Efraim Manassét; s mindketten Júda ellen kelnek. Mindezekkel haragja el nem múlt, és keze még felemelve van.

\chapter{10}

\par 1 Jaj a hamis határozatok határozóinak, és a jegyzõknek, a kik gonoszt jegyeznek,
\par 2 Hogy elriaszszák a gyöngéket a törvénykezéstõl, s elrabolják népem szegényeinek igazságát, hogy legyenek az özvegyek az õ prédájok, és az árvákban zsákmányt vessenek.
\par 3 S vajjon mit míveltek a meglátogatásnak és a messzünnen rátok jövõ pusztulásnak napján? Kihez futtok segítségért, és hol hagyjátok dicsõségeteket?
\par 4 Bizony csak: a foglyok alá hanyatlanak és a megölettek alá hullanak; mind ezekkel haragja el nem múlt, és keze még felemelve van.
\par 5 Jaj Assiriának, haragom botjának, mert pálcza az õ kezében az én búsulásom!
\par 6 Istentelen nemzetség ellen küldtem õt, és haragom népe ellen rendelém, hogy prédáljon és zsákmányt vessen, és eltapodja azt, mint az utczák sarát.
\par 7 De õ nem így vélekedik, és szíve nem így gondolkozik, mert õ pusztítani akar, és kigyomlálni sok népeket.
\par 8 Mert így szól: Vajjon vezéreim nem mind királyok-é?
\par 9 Nem úgy megvettem-é Kalnót, mint Kárkemist? És Hamáthot, mint Arphádot? És Samariát, mint Damaskust?
\par 10 Miképen megtalálta volt kezem a bálványok országait, holott pedig több faragott képük volt, mint Jeruzsálemnek és Samariának,
\par 11 Avagy a mint cselekedtem Samariával és az õ bálványaival; nem úgy cselekedhetem-é Jeruzsálemmel és bálványképeivel?
\par 12 Ezért ha majd elvégezi az Úr minden dolgát Sion hegyén és Jeruzsálemben, meglátogatom az Assiriabeli király nagyakaró szíve gyümölcsét és nagyralátó szeme kevélységét.
\par 13 Mert ezt mondá: Kezem erejével míveltem ezt és bölcseségemmel, mivel okos vagyok; elvetettem sok népeknek határit, és kincseikben zsákmányt vetettem, és leszállítám, mint erõs, a magasan ülõket;
\par 14 És kezem úgy találta mintegy fészket a népek gazdagságát; és a mily könnyen elszedik az elhagyott tojásokat, azonképen foglaltam én el az egész földet; és nem volt senki, a ki szárnyát mozdította, vagy száját megnyitotta, vagy csak sipogott volna is!
\par 15 Avagy dicsekszik-é a fejsze azzal szemben, a ki vele vág? vagy a fûrész felemeli-é magát az ellen, a ki vonsza azt? Mintha a bot forgatná azt, a ki õt felemelé, és a pálcza felemelné azt, a mi nem fa!
\par 16 Azért az Úr, a seregek Ura kövéreire ösztövérséget bocsát, és az õ dicsõsége alatt égés ég, miként a tûz égése;
\par 17 És lészen Izráel világossága tûz gyanánt, és annak Szentje láng gyanánt, és ég és megemészti gazzát és tövisét egy napon;
\par 18 Az õ erdejének és kertjének ékességét pedig lelkétõl mind testéig megemészti, és lesorvad, mint a sorvadozó;
\par 19 Erdõje fáinak maradéka kicsiny lészen, és azokat egy gyermek is felírhatja!
\par 20 És lészen ama napon, hogy többé nem támaszkodik Izráel maradéka és a ki megmaradt Jákób házából, az õt nyomorgatóhoz, hanem támaszkodik az Úrhoz, Izráel Szentjéhez hûségesen;
\par 21 A maradék megtér, a Jákób maradéka az erõs Istenhez.
\par 22 Mert ha néped Izráel számszerint annyi lenne is, mint a tenger fövenye, csak maradéka tér meg; az elvégezett pusztulás elárad igazsággal!
\par 23 Mert elvégezett pusztítást cselekszik az Úr, a seregek Ura, mind az egész földön.
\par 24 Azért így szól az Úr, a seregeknek Ura: Ne félj népem, Sionnak lakosa, az Assiriabeli királytól! bár botjával megver tégedet és pálczáját felemeli rád, miként Égyiptom egykoron;
\par 25 Mert még csak egy kevés idõ van: és elfogy búsulásom és haragom az õ megemésztésökre lészen!
\par 26 És a seregeknek Ura ostort emel õ ellenök, miként a Midián levágatása idején az Oreb kõszikláján, és pálczáját a tenger fölé emeli, mint Égyiptomban egykoron.
\par 27 És lesz ama napon: eltávozik az õ terhe válladról és igája nyakadról, és megromol az iga a kövérségnek miatta.
\par 28 Ajjáthba jõ, Migronba átsiet, Mikhmásnál rakja le hadakozó szerszámait;
\par 29 Átmennek a szoroson, Gébában lesz hálóhelyünk; megrémül Ráma, Gibea, Saulnak városa elfut.
\par 30 Kiálts Gallim leánya, és vedd füleidbe Laisa, és szegény Anathóth!
\par 31 Madména megindul, Gébim lakosai mentik övéiket;
\par 32 Még Nóbba megyen ma, hogy ott megálljon, és emeli már kezét Sion leányának hegye és Jeruzsálemnek halma ellen!
\par 33 De ímé az Úr, a seregeknek Ura: levagdalja az ágakat rémítõ hatalommal, és a magas termetûek kivágattatnak, és a magasságosak megaláztatnak;
\par 34 Az erdõ sûrû ágait levágja vassal, és megdõl a Libánon egy hatalmas által.

\chapter{11}

\par 1 És származik egy vesszõszál Isai törzsökébõl, s gyökereibõl egy virágszál nevekedik.
\par 2 A kin az Úrnak lelke megnyugoszik: bölcseségnek és értelemnek lelke, tanácsnak és hatalomnak lelke, az Úr ismeretének és félelmének lelke.
\par 3 És gyönyörködik az Úrnak félelmében, és nem szemeinek látása szerint ítél, és nem füleinek hallása szerint bíráskodik:
\par 4 Igazságban ítéli a gyöngéket, és tökéletességben bíráskodik a föld szegényei felett; megveri a földet szájának vesszejével, és ajkai lehével megöli a hitetlent.
\par 5 Derekának övedzõje az igazság lészen, és veséinek övedzõje a hûség.
\par 6 És lakozik a farkas a báránynyal, és a párducz a kecskefiúval fekszik, a borjú és az oroszlán-kölyök és a kövér barom együtt lesznek, és egy kis gyermek õrzi azokat;
\par 7 A tehén és medve legelnek, és együtt feküsznek fiaik, az oroszlán, mint az ökör, szalmát eszik;
\par 8 És gyönyörködik a csecsszopó a viperák lyukánál, és a csecstõl elválasztott a baziliskus lyuka felett terjengeti kezét:
\par 9 Nem ártanak és nem pusztítnak sehol szentségemnek hegyén, mert teljes lészen a föld az Úr  ismeretével, mint a vizek a tengert beborítják.
\par 10 És lesz ama napon, hogy Isai gyökeréhez, a ki a népek zászlója lészen, eljõnek a pogányok, és az õ nyugodalma dicsõség lészen.
\par 11 És lesz ama napon: az Úr másodszor nyujtja ki kezét, hogy népe maradékát megvegye, a mely megmaradt Assiriától, Égyiptomtól, Pathrosztól, Szerecsenországtól, Elámtól, Sinártól, Hamáthtól és a tenger szigeteitõl.
\par 12 És zászlót emel a pogányok elõtt, és összegyûjti Izráel elszéledt fiait, és Júdának szétszórt leányait egybegyûjti a földnek négy szárnyairól.
\par 13 Megszünik Efraimnak irígysége, és Júdából a gyûlölködõk kivágattatnak; Efraim nem irígykedik Júdára, és Júda sem támad többé Efraimra.
\par 14 És repülnek a Filiszteusoknak hátára napnyugot felé, és kelet fiaiban együtt vetnek zsákmányt, és kezet vetnek Edomra és Moábra, és az Ammoniták engednek nékik.
\par 15 És az Úr megátkozza Égyiptom tengerének nyelvét, és kezét felemeli az Eufráth fölé erõs száraztó szélben, és hét patakra csapja azt, és népét sarus lábbal vezeti át,
\par 16 És csinált út lészen népe maradékának, a mely megmaradt Assiriától, a mint volt Izráelnek, mikor kijött Égyiptomnak földébõl.

\chapter{12}

\par 1 És így szólsz ama napon: Hálákat adok néked, oh Uram! mert jóllehet haragudtál reám, de elfordult haragod, és megvígasztaltál engemet!
\par 2 Ímé, az Isten az én szabadítóm! bízom és nem félek; mert erõsségem és énekem az Úr, az Úr, és lõn nékem szabadítóm!
\par 3 S örömmel merítetek vizet a szabadító kútfejébõl,
\par 4 És így szólotok ama napon: Adjatok hálát az Úrnak, magasztaljátok az Õ nevét, hirdessétek a népek közt nagyságos dolgait, mondjátok, hogy nagy az Õ neve.
\par 5 Mondjatok éneket az Úrnak, mert nagy dolgot cselekedett; adjátok tudtára ezt az egész földnek!
\par 6 Kiálts és örvendj, Sionnak lakosa, mert nagy közötted Izráelnek Szentje!

\chapter{13}

\par 1 Jövendölés Babilonia ellen, a melyet látott Ésaiás, Ámós fia.
\par 2 Emeljetek zászlót kopasz hegyen, kiáltsatok nékik, kézzel intsetek, hogy bevonuljanak a fejedelmek kapuin!
\par 3 Én parancsoltam felszentelt vitézeimnek, és elhívtam erõsimet haragomnak véghezvitelére, a kik én bennem büszkén örvendenek.
\par 4 Hah! zsibongás a hegyeken, mint nagy néptömegé; hah! összegyûlt népek országainak zúgása; a seregek Ura harczi sereget számlál.
\par 5 Jõnek messze földrõl, az égnek végérõl, az Úr, és haragjának eszközei, elpusztítani mind az egész földet.
\par 6 Jajgassatok, mert közel van az Úrnak napja, mint pusztító hatalom jõ a Mindenhatótól.
\par 7 Ezért megerõtlenülnek minden kezek, és elolvad minden embernek szíve;
\par 8 És megrémülnek, kínok és fájdalmak fogják el õket, és szenvednek, mint a szülõasszony; egyik a másikon csodálkozik, és arczuk lángba borul.
\par 9 Ímé az Úrnak napja jõ kegyetlen búsulással és felgerjedt haraggal, hogy a földet pusztasággá tegye, és annak bûnöseit elveszesse arról.
\par 10 Mert az ég csillagai és csillagzatai nem ragyogtatják fényöket, sötét lesz a nap támadásakor, és a hold fényét nem tündökölteti.
\par 11 És meglátogatom a földön a bûnt, és a gonoszokon vétküket, és megszüntetem az istentelenek kevélységét, és az erõszakoskodóknak gõgjét megalázom.
\par 12 Drágábbá teszem az embert a színaranynál, és a férfit Ofir kincsaranyánál.
\par 13 Ezért az egeket megrendítem, és megindul helyérõl a föld is, a seregek Urának búsulása miatt, és felgerjedett haragjának napján,
\par 14 És mint az ûzött zerge, és mint a pásztor nélkül való nyáj, kiki népéhez tér meg, és kiki az õ földére fut;
\par 15 Valaki ott találtatik, átveretik, és valaki megfogatik, fegyver miatt hull el,
\par 16 Kisdedeiket szemök elõtt zúzzák szét, házaikat elzsákmányolják, és feleségeiket megszeplõsítik.
\par 17 Ímé, én feltámasztom ellenök a Médiabelieket, a kik ezüsttel nem gondolnak, és aranyban nem gyönyörködnek;
\par 18 Kézíveik szétzúznak ifjakat, és nem könyörülnek a méh gyümölcsén, a fiaknak nem irgalmaz szemök:
\par 19 És olyan lesz Babilon, a királyságok ékessége, a Khaldeusok dicsekvésének dísze, mint a hogyan elpusztítá Isten Sodomát és Gomorát;
\par 20 Nem ülik meg soha, és nem lakják nemzetségrõl nemzetségre, nem von sátort ott az arábiai, és pásztorok sem tanyáznak ott;
\par 21 Hanem vadak tanyáznak ott, és baglyok töltik be házaikat, és struczok laknak ott, és bakok szökdelnek ott;
\par 22 És vad ebek üvöltenek palotáikban, és mulató házaikban sakálok; és ideje nem sokára eljõ, és napjai nem késnek.

\chapter{14}

\par 1 Mert könyörül az Úr Jákóbon, és ismét elválasztja Izráelt, és megnyugotja õket földjükön; és a jövevény hozzájok adja magát, és a Jákób házához csatlakoznak;
\par 2 És felveszik õket a népek, és elviszik õket lakhelyökre, és Izráel háza bírni fogja õket az Úr földén, szolgák és szolgálók gyanánt; és foglyaik lesznek foglyul vivõik, és uralkodnak nyomorgatóikon.
\par 3 És majd ama napon, a melyen nyugalmat ád néked az Úr fáradságodtól és nyomorúságodtól és ama kemény szolgálattól, a melylyel szolgálnod kellett,
\par 4 E gúnydalt mondod Babilon királya felett, és szólsz: Miként lõn vége a nyomorgatónak, a szolgaság házának vége lõn!
\par 5 Eltörte az Úr a gonoszok pálczáját, az uralkodóknak vesszejét.
\par 6 A ki népeket vert dühében szüntelen való veréssel, leigázott népségeket haraggal, kergettetik feltartózhatlanul.
\par 7 Nyugszik, csöndes az egész föld. Ujjongva énekelnek.
\par 8 Még a cziprusok is örvendenek rajtad: a Libánon czédrusai ezt mondják: Mióta te megdõltél, nem jõ favágó ellenünk.
\par 9 Alant a sír megindul te miattad megérkezésedkor, miattad felriasztja árnyait, a föld minden hatalmasit, felkölti székeikrõl a népek minden királyait;
\par 10 Mind megszólalnak, és ezt mondják néked: Erõtlenné lettél te is, miként mi; hozzánk hasonlatos levél!
\par 11 Kevélységed és lantjaid zengése a sírba szállt; fekvõ ágyad férgek, és takaró lepled pondrók!
\par 12 Miként estél alá az égrõl fényes csillag, hajnal fia!? Levágattál a földre, a ki népeken tapostál!
\par 13 Holott te ezt mondád szívedben; Az égbe megyek fel, az Isten csillagai fölé helyezem ülõszékemet, és lakom a gyülekezet hegyén messze északon.
\par 14 Felibök hágok a magas felhõknek, és hasonló leszek a Magasságoshoz.
\par 15 Pedig a sírba szállsz alá, sírgödör mélységébe!
\par 16 A kik látnak, rád tekintenek, és elgondolják: Ez-é a föld ama háborgatója, a ki királyságokat rendített meg?
\par 17 A ki a föld kerekségét pusztasággá tette, városait lerontá, és foglyait nem bocsátá haza!
\par 18 A népeknek minden királyai dicsõségben nyugosznak, kiki az õ sírjában,
\par 19 Te pedig messze vettetel sírodtól, mint valami hitvány gally, takarva megölettekkel, fegyverrel átverettekkel, sziklasírba leszállókkal, mint valami eltapodott holttest!
\par 20 Nem egyesülsz velök a sírban, mert elpusztítád földedet, megölted népedet: nem él sokáig a gonoszok magva sem!
\par 21 Készítsetek öldöklést az õ fiainak atyáik vétkéért, hogy fel ne keljenek, és örökségül bírják e földet, és városokkal töltsék be e földnek színét!
\par 22 És felkelnek õ ellenök, szóla a seregeknek Ura, és kivágom Babilon nevét és maradékát, fiait és unokáit, szól az Úr;
\par 23 És a sündisznónak örökségévé és álló vizek tavává teszem azt, és elsöpröm azt a pusztítás seprõjével, szól a seregeknek Ura.
\par 24 Megesküdött a seregeknek Ura, mondván: Úgy lészen, mint elgondolám, úgy megy véghez, mint elvégezém:
\par 25 Megrontom Assiriát földemen, és megtapodom hegyeimen, és eltávozik róluk igája, és terhe válláról eltávozik.
\par 26 Ez az elvégezett tanács az egész föld felõl, és ez ama felemelt kéz minden népek fölött.
\par 27 Mert a seregek Ura végezte, és ki teszi azt erõtelenné? Az õ keze fel van emelve; ki fordítja el azt?
\par 28 A mely esztendõben meghalt Akház király, akkor lõn e jövendölés.
\par 29 Ne örvendj oly nagyon Filisztea, hogy eltört a téged verõnek vesszeje, mert a kígyó magvából baziliskus jõ ki, a melytõl szárnyas sárkány származik.
\par 30 És legelnek a szegényeknek elsõszülöttei, és a szûkölködõk bátorsággal nyugosznak; és én kivesztem gyökeredet éhséggel, és maradékodat megölik.
\par 31 Jajgass kapu, kiálts város! remeg egész Filisztea, mivel füst jõ észak felõl; és nem marad el seregeibõl egy sem!
\par 32 És mit felelnek a népek követei? Azt, hogy az Úr veté meg a Sion alapját, és abban bíznak népe szegényei.

\chapter{15}

\par 1 Jövendölés Moáb ellen. Igen, a pusztulás éjjelén megsemmisült Ar-Moáb; igen, a pusztulás éjjelén megsemmisült Kir-Moáb!
\par 2 Felmennek a templomba, és Dibon a magaslatokra megy sírni, Nebón és Médebán jajgat Moáb, minden fejen kopaszság, minden szakál lenyírva!
\par 3 Utczáin gyászruhába öltöznek, házfedelein és piaczain jajgat minden, és könyekben olvad el!
\par 4 És kiált Hesbon és Eleálé, szavok Jáháczig hallatik; ezért ordítanak Moáb vitézei, és lelke reszket.
\par 5 Szívem kiált Moábért, a melynek végvárai Zoárig, a három éves üszõig nyúltanak; mert Luhith hágóján sírással mennek fel, és mert Horonaim útján romlásuknak kiáltását hallatják.
\par 6 Mert Nimrim vizei pusztasággá lesznek, mert elszáradt a pázsit, elfogyott a fû, és semmi zöld nem lészen.
\par 7 Ezért a mit megmenthettek és jószágukat a fûzfáknak patakja mellé hordják.
\par 8 Mert kiáltás zengé körül Moáb határát, jajgatása Eglaimig, és Beér-Élimig hallatik jajgatása!
\par 9 Mert Dimon vizei megteltek vérrel, mert új romlást hozok Dimonra: Moáb menekültjeire oroszlánt, és a földnek maradékira.

\chapter{16}

\par 1 Küldjétek a föld Urának bárányát Szelából a pusztán át Sion leányának hegyére.
\par 2 Mert mint a bujdosó madár szétszórt fészek körül, olyanok lõnek Moáb leányai az Arnon gázlóin:
\par 3 Adj tanácsot, tarts ítéletet; tegyed árnyékodat délben olyanná, mint az éjszaka, rejtsd el a kiûzötteket, és a bujdosót ne add ki!
\par 4 Lakozzanak benned menekültjeim, és Moábnak te légy oltalom a pusztító ellen! Mert vége a nyomorgatónak, megszünt a pusztítás, és elfogytak a földrõl a tapodók.
\par 5 És Isten kegyelme megerõsített egy ülõszéket, és ül azon igazsággal Dávid sátorában egy bíró, jogosság keresõje, igazság ismerõje.
\par 6 Hallottuk volt Moáb kevélységét, a felettébb kevélyét, gõgjét, kevélységét, dühét, és üres kérkedését.
\par 7 Ezért jajgatni fog Moáb Moábért, minden jajgatni fog, és nyögtök Kir-Háresethnek romjain egészen megtörve.
\par 8 Mert Hesbon földei elhervadának, és Sibma szõlõjének drága vesszõit a népek fejedelmei levágták. Jáézerig értek azok, a pusztát bejárták, kacsai szétterjedtek, és a tengeren túlnyúltak.
\par 9 Ezért siratom Jáézer siralmával Sibma szõlõjét, megnedvesítlek könyeimmel Hesbon és Eleálé, mert szüretedrõl és aratásodról a víg éneklés elmaradt.
\par 10 Elvétetett a vígság és öröm a kertbõl, és a szõlõkben nem vígadnak és nem kiáltanak, bort sajtókban nem nyom a bornyomó, véget vetettem a víg éneklésnek.
\par 11 Ezért bensõm Moábért, mint a czitera sír, és szívem Kir-Heresért!
\par 12 És lesz, hogy meg fog tetszeni, hogy Moáb a magaslaton elfáradt, és hogy templomába megy imádkozni, de nem mehet!
\par 13 Ez a beszéd, a melyet szólott az Úr Moáb felõl már régen.
\par 14 És most szól az Úr, mondván: Három esztendõ alatt, melyek, mint napszámos esztendõi, megaláztatik Moáb dicsõsége egész nagy népével együtt, és maradéka kicsiny, kevés és erõtelen lészen.

\chapter{17}

\par 1 Jövendölés Damaskus ellen. Ímé Damaskus városa elpusztíttatik, és romok halmává lesz;
\par 2 Aróer városai elhagyatnak, és barmokéi lesznek, a melyek ott fognak nyugodni háborítatlanul.
\par 3 És vége lesz Efraim erõsségének, és Damaskus királyságának; Siria maradéka úgy jár, mint az Izráeliták dicsõsége, ezt mondja a seregeknek Ura.
\par 4 És lesz ama napon: megvékonyul a Jákób dicsõsége, és húsa kövérsége megösztövéredik.
\par 5 És lészen, mint mikor az arató összefogja a gabonát és a kalászokat kezével learatja, és lészen, mint mikor valaki kalászokat szed össze a Refáim völgyében;
\par 6 És csak mezgérlés marad belõlök, mint az olajfa megrázásakor két-három bogyó az ágak hegyén, négy-öt a gyümölcsfának lombjai közt, így szól az Úr, Izráel Istene.
\par 7 Ama napon Teremtõjére tekint az ember, és szemei Izráel Szentjére néznek;
\par 8 És nem tekint az oltárokra, kezeinek alkotmányára, és a miket ujjai csináltak, nem nézi azokat, a berkeket és a nap-oszlopokat.
\par 9 Erõs városai olyanok lesznek ama napon, mint elhagyott erdõ és hegytetõ, a melyeket elhagytak volt Izráel fiai elõtt, és pusztasággá lesznek.
\par 10 Mert elfelejtkeztél megszabadító Istenedrõl, és nem emlékeztél meg erõs kõszáladról, ez okért ültetél gyönyörûséges ültetéseket, és idegen vesszõt plántáltál beléjök;
\par 11 A mely napon elültetted, körül sövényezéd és reggelre magodat felvirágoztatád: de nem lészen aratás sebeidnek és nagy fájdalmadnak napján!
\par 12 Jaj a sok nép zúgásának, a kik úgy zúgnak, mint a tenger zúgása, és a népségek háborgásának, a kik háborognak, mint erõs vizek!
\par 13 A népségek, mint sok vizek háborgása, úgy háborognak, de megdorgálja azt és elfut messzire, és elragadtatik, mint a hegyek polyvája szél elõtt és mint pozdorja a forgószél elõtt;
\par 14 Estvének idején rémülés száll reájok és minekelõtte megvirrad, nem lesznek: ez jutalmok pusztítóinknak és sorsok rablóinknak!

\chapter{18}

\par 1 Jaj a szárnysuhogás országának, a mely Szerecsenország folyóin túl van;
\par 2 A mely követeket küld a tengeren gyékénybõl csinált tutajokon, vizek színén. Menjetek el gyors hírnökök, a magas és derék néphez, a néphez, a mely rettenetes mióta van és ezután is; a hatalmas és hódító néphez, a melynek földét folyók hasítják át.
\par 3 Föld kerekségének minden lakói és földnek lakosai meglássátok, mikor a hegyeken zászló emelkedik, és halgassatok, ha kürt szól!
\par 4 Mert ezt mondá az Úr nékem: Nyugodtan nézem sátoromból verõfényes melegnél, harmatozó felhõnél aratás hévségében.
\par 5 Mert aratás elõtt, ha a virágzás véget ér, és a virág érõ fürtté lészen, levágja a vesszõket metszõkésekkel, és a kacsokat eltávolítja és lemetszi.
\par 6 Mind ott hagyatnak a hegyek madarainak és a föld állatainak, és a madarak rajtok nyaralnak, és a föld minden állatai rajtok telelnek.
\par 7 Abban az idõben ajándékot viszen a seregek Urának a magas és derék nép; a nép, a mely rettenetes, mióta van és ezután is, a hatalmas és hódító nép, a melynek földét folyók hasítják át; a seregek Ura nevének helyére, Sion hegyére.

\chapter{19}

\par 1 Jövendölés Égyiptom ellen. Ímé az Úr könnyû felhõre ül, és bemegy Égyiptomba, és megháborodnak elõtte Égyiptomnak  bálványai, és az égyiptomiak szíve megolvad õ bennök.
\par 2 És összeveszítem az égyiptomiakat az égyiptomiakkal, és egyik hadakozik a másik ellen, kiki felebarátja ellen, város város ellen, és ország ország ellen.
\par 3 És elfogyatkozik az égyiptomiak lelke õ bennök, és az õ tanácsát elnyelem, és tudakoznak a bálványoktól, szemfényvesztõktõl, halottidézõktõl és jövendõmondóktól.
\par 4 És adom az égyiptomiakat kemény úrnak kezébe, és kegyetlen király uralkodik rajtok, szól az Úr, a seregeknek Ura.
\par 5 És elfolgyatkoznak a tenger részei, és a folyam kiszárad és elapad.
\par 6 És büdösséget árasztanak a folyamok, elfogynak és kiszáradnak Égyiptomnak patakjai; nád és sás ellankadnak.
\par 7 A pázsit a folyó mellett, a folyónak partján, és a folyónak minden veteménye megszárad, elporlik és nem lészen.
\par 8 És keseregnek a halászok, és gyászolnak mind, a kik a folyóba horgot vetni szoktak, és a kik a vizek színén hálót vetnek ki, búsulnak.
\par 9 És megszégyenülnek a fésült len készítõi és a gyolcs szövõi.
\par 10 Hatalmasaik megtöretnek, és minden napszámosaik bánkódnak lelkökben.
\par 11 Bizony bolondok Zoán fejedelmei, a Faraó bölcs tanácsosai egy eszét veszített tanács; mimódon mondjátok a Faraónak: bölcsek fia vagyok, régi királyoknak fia?
\par 12 Ugyan hol vannak bölcseid? hogy megjelentsék néked és tudják, hogy mit végzett a seregeknek Ura Égyiptom felõl?
\par 13 Megbolondultak Zoán fejedelmei, és csalódtak Nóf fejedelmei, és elámíták az Égyiptomiakat törzseik szegletkövei.
\par 14 Az Úr beléje önté a szédelgés lelkét, hogy Égyiptomot elhitessék minden dolgaiban; miként a részeg tántorog az õ okádása felett.
\par 15 És nem lesz Égyiptomnak semmi mmunkája, a melyet cselekednék a fõ és fark, a pálmaág és a káka.
\par 16 Ama napon olyan lesz Égyiptom, mint az asszonyok, retteg és fél a seregek Ura kezének felemelésétõl, a melyet fölemel ellene.
\par 17 És Júda földe félelmére lesz Égyiptomnak, valaki csak említi azt elõtte, már fél, s a seregek Urának tanácsáért, a melyet Õ végzett felõle.
\par 18 Ama napon lesz öt város Égyiptomnak földén, a melyek Kanaán nyelvén szólanak, és esküsznek a seregek Urára; az egyik "pusztulás városának" neveztetik.
\par 19 Ama napon oltára lesz az Úrnak Égyiptom földének közepette és határán egy oszlop az Úrnak;
\par 20 És lesz jegyül és bizonyságul a seregek Urának Égyiptom földén, hogy ha kiáltanak az Úrhoz a nyomorgatók elõtt, hogy küldjön nékik megtartót és fejedelmet, és õket megszabadítsa.
\par 21 És megismerteti magát az Úr Égyiptommal, és megismeri Égyiptom az Urat ama napon, és szolgálják véres áldozattal és ételáldozattal, és fogadnak fogadást az Úrnak, és teljesítik.
\par 22 De ha megveri az Úr Égyiptomot, megvervén, meggyógyítja, és megtérnek az Úrhoz, és Õ meghallgatja és meggyógyítja õket.
\par 23 Ama napon út lesz Égyiptomból Assiriába, és Assiria megy Égyiptomba, Égyiptom meg Assiriába, és Égyiptom Assiriával az Urat tiszteli.
\par 24 Ama napon Izráel harmadik lesz Égyiptom és Assiria mellett; áldás a földnek közepette;
\par 25 Melyet megáld a seregeknek Ura, mondván: Áldott népem, Égyiptom! és kezem munkája, Assiria! és örökségem Izráel!

\chapter{20}

\par 1 Amely esztendõben Asdódba ment a Tartán, Szargon Assiria királya küldvén õt, és vívá Asdódot és azt elfoglalá:
\par 2 Ez idõben szólott az Úr Ésaiás, Ámos fia által, mondván: Menj és oldd le a gyászruhát derekadról, és sarudat vond le lábadról! És úgy cselekedett, járván ruha és saru nélkül.
\par 3 És mondá az Úr: A mint szolgám Ésaiás ruha és saru nélkül jár három esztendeig jegyül és jelenségül Égyiptomra és Szerecsenországra nézve:
\par 4 Úgy viszi el Assiria királya Égyiptom foglyait és Szerecsenország rabjait, ifjakat és véneket, ruha és saru nélkül és mezítelen alfellel, Égyiptomnak gyalázatára;
\par 5 S megrettennek és megszégyenülnek Szerecsenország miatt, a melyben reménykedtek, és Égyiptom miatt, a melyben dicsekedtek.
\par 6 És szól e partvidék lakosa ama napon: Ímé így járt reménységünk, hova segítségért futottunk, hogy Assiria királyától megszabaduljunk; hát mi miként menekülhetnénk meg?

\chapter{21}

\par 1 Jövendölés a tenger pusztasága ellen. Mint szélvészek, délen tombolók, úgy jõ a pusztából, rettenetes földrõl.
\par 2 Kemény látás jelentetett meg nékem: a csalárd csal, a pusztító pusztít. Jõjj fel Élám, szálld meg Madai, minden õ fohászkodásának véget vetek.
\par 3 Ezért ágyékim telvék fájdalommal, és kínok fogtak el, mint a szûlõ asszony kínjai: gyötrõdöm hallása miatt, és megrémültem látása miatt.
\par 4 Reszket szívem, iszonyúság rettent, a kedves éjszakát remegéssé tevé nékem.
\par 5 Teríts asztalt, vigyázzon a vigyázó, egyetek, igyatok; föl fejedelmek, kenjétek a paizst!
\par 6 Mert így szólott hozzám az Úr: Menj és állass õrállót, a mit lát, mondja meg.
\par 7 És látott lovas csapatot, páros lovagokat, szamaras csapatot, tevés csapatot, és nagy figyelmesen hallgatott.
\par 8 És kiálta, mint oroszlán: Uram az õrtoronyban állok szüntelen napestig, és õrhelyemen állok egész éjszakákon.
\par 9 És ímé, lovas csapat jött, páros lovagok, és szólott és mondá: Elesett, elesett Babilon, s isteneinek minden faragott képeit a földre zúzták le.
\par 10 Oh én cséplésem és szérûmnek fia, a mit hallottam a seregek Urától, Izráel Istenétõl, azt jelentém meg néktek!
\par 11 Jövendölés Dúma ellen: Seirbõl így kiáltnak hozzám: Vigyázó! meddig még az éjszaka, meddig még ez éj?
\par 12 Szólt a vigyázó: Eljött a reggel, az éjszaka is; ha kérdeni akartok, kérdjetek, forduljatok vissza és jertek el!
\par 13 Jövendölés Arábia ellen: Az erdõben háltok Arábiában, Dédán utazó seregei.
\par 14 A szomjazó elé hozzatok vizet! Témá följének lakosi kenyerökkel jönnek a bujdosó elébe.
\par 15 Mert az ellenség fegyver elõtt bujdosnak, a kivont fegyver elõtt és a felvont kézívek elõtt és a nehéz harcz elõtt.
\par 16 Mert így szólott hozzám az Úr: Még egy esztendõ, mely mint a béresnek esztendeje, és elvész Kédárnak minden dicsõsége;
\par 17 És Kédár vitéz fiainak kézíve számának maradéka megkevesedik; mert az Úr, Izráel Istene mondá.

\chapter{22}

\par 1 Jövendölés a látás völgye ellen. Mi lelt most, hogy mindenestõl felmenél a házak tetejére?
\par 2 Te lármával teljes, zajos város, örvendezõ város, megöletteid nem fegyverrel ölettek, meg, és nem harczon hullottak el!
\par 3 Hadnagyaid mind elszaladtak a kézív elõtt, de megkötöztettek; a kik benned találtattak, mind megkötöztettek, midõn futni akartak messzire.
\par 4 Ezért mondom: Ne nézzetek reám, hadd kesergessem magamat sírással; ne siessetek vígasztalni engem, népem leánya romlása felett.
\par 5 Mert rémülésnek és eltapodtatásnak és zavarnak napja jõ Istentõl, a seregeknek Urától a látás völgyére, a mely ledönti a kõfalat, és a hegyeken kiáltás hallatik.
\par 6 És Élám fölvette a tegezt, és jõ szekeren emberekkel és lovagokkal, és Kir paizst meztelenít.
\par 7 És lészen, hogy szép völgyeid betelnek szekerekkel, és a lovagok nyomulnak a kapu felé.
\par 8 És fölfedi Júdának fátyolát, és tekintesz ama napon az erdõ házának fegyverzetére,
\par 9 És meglátjátok, hogy Dávid városán sok repedés van, és összegyûjtitek az alsó tó vizét.
\par 10 És megszámláljátok Jeruzsálem házait, és némely házat lerontotok, hogy a kõfalat megerõsíthessétek;
\par 11 És árkot csináltok a két kõfal között a régi tó vizének; és nem tekintetek arra, a ki ezt cselekvé, és a ki régen elvégzé ezt, Õt nem látjátok!
\par 12 És fölhív az Úr, a seregek Ura ama napon sírásra és gyászolásra, ti magatok megkopaszítására és gyászruha-öltésre,
\par 13 És ímé öröm és vígasság; barmok ölése, juhok levágása, húsevés és borivás; együnk, igyunk, mert holnap meghalunk!
\par 14 És megjelenté magát füleimben a seregek Ura: Meg nem bocsáttatik e bûn tinéktek, míg meg nem haltok; szól az Úr, a seregek Ura.
\par 15 Így szólt az Úr, a seregek Ura: No, menj el a kormányzóhoz, Sébnához, a királyi ház fõemberéhez, és mondd meg néki:
\par 16 Mi dolgod itt, és ki lesz itten tied, hogy sírt vágatsz itten magadnak? mint a ki sírját magas helyen vágatja, és sziklába véset hajlékot magának!
\par 17 Ímé az Úr elhajít téged erõs hajítással, és megragadván megragad,
\par 18 Hempelygetvén hempelyget, mint gombolyagot, mint labdát, nagy messze földre, ott halsz meg, oda mennek dicsõséged szekerei, te, urad házának gyalázata!
\par 19 És kivetlek állásodból és lerántlak helyedrõl.
\par 20 És lesz ama napon, hogy elhívom szolgámat, Eliákimot, a Hilkiás fiát,
\par 21 S felöltöztetem õt öltözetedbe, és öveddel megerõsítem, és uralmadat kezébe adom, és õ lesz atyjok Jeruzsálem lakosainak és a Júda házának;
\par 22 S az õ vállára adom a Dávid házának kulcsát, és a mit megnyit, senki be nem zárja, és a mit bezár, nem nyitja meg senki;
\par 23 S beverem õt, mint szeget erõs helyre, és lészen dicsõséges székül az õ atyja házának;
\par 24 S reá függesztik atyja házának minden dicsõségét: fiakat és unokákat, minden kicsiny edényt, a csészeedényektõl a tömlõknek minden edényeiig.
\par 25 Ama napon, azt mondja a seregeknek Ura, kiesik a szeg, mely erõs helyre veretett, és levágatik és leesik és összetörik a teher, mely rajta volt, mert az Úr mondá.

\chapter{23}

\par 1 Jövendölés Tírus ellen. Jajgassatok Tarsis hajói, mert elpusztíttatott, úgy hogy nincs benne ház és abba bemenet! A Kitteusok földérõl jelentetett meg nékik.
\par 2 Némuljatok meg lakosi e partvidéknek, a melyet Sidon kalmárai, a kik tengeren járnak, töltöttek be egykor.
\par 3 Melynek sok vizeken át a Sihór veteménye és a Nilus aratása vala jövedelme, úgy hogy népek vására volt!
\par 4 Pirulj Sidon, mert szól a tenger és a tenger erõssége, mondván: Nem vajudtam, nem is szültem, és nem tápláltam ifjakat, és nem neveltem szûzeket.
\par 5 Mihelyt e hír Égyiptomba eljut, Tírus e híre miatt szenvednek ott is.
\par 6 Menjetek át Tarsisba, és jajgassatok ti partvidék lakói!
\par 7 Ez-é a ti örvendezõ várostok? melynek eredete õsidõkbõl való; és most lábai viszik õt, bujdosni messzire!
\par 8 Ki végezé ezt a koronás Tírus felõl? melynek kereskedõi fejedelmek, és kalmárai a földnek tiszteletesei.
\par 9 A seregeknek Ura végezé ezt, hogy meggyalázza minden dicsõségnek kevélységét, és hogy megalázza a föld minden tiszteleteseit.
\par 10 Terülj el földeden, mint a folyóvíz, Tarsis leánya, nincs többé megszorító öv!
\par 11 Kezét kinyujtá a tenger fölé, országokat rettentett meg, az Úr parancsolt Kanaán felõl, hogy elpusztítsák erõsségeit;
\par 12 És szólt: Nem fogsz többé örvendezni, te megszeplõsített szûz, Sidon leánya; kelj és menj át Kittimbe, de ott sem lészen nyugodalmad!
\par 13 Ímé, a Káldeusok földe; a nép, mely eddig nem vala; Assiria adá azt a puszta lakosainak; felállítá õrtornyait, lerombolá Tírus palotáit, rommá tevé azt.
\par 14 Jajgassatok Tarsis hajói, mert erõsségtek elpusztíttatott!
\par 15 És lesz ama napon, hogy Tírus elfelejtetik hetven esztendeig egy király napjai szerint; hetven esztendõ multán Tírus sorsa a parázna nõ éneke szerint lészen:
\par 16 Végy cziterát, járd be a várost, elfeledett parázna nõ; pengesd szépen, dalolj sokat, hogy így emlékezetbe jõjj!
\par 17 És lesz hetven esztendõ multán, meglátogatja az Úr Tírust, és az ismét megkapja a maga keresetét, és paráználkodik a föld minden országaival a földnek színén!
\par 18 S lészen az õ nyeresége és keresete szent az Úrnak, mely nem halmoztatik fel, sem el nem rejtetik, mert az Úr elõtt lakozóké lészen az õ nyeresége, hogy egyenek eleget, és szép ruházatuk legyen.

\chapter{24}

\par 1 Ímé az Úr megüresíti a földet és elpusztítja azt, és elfordítja színét és elszéleszti lakóit!
\par 2 S olyan lesz a nép, mint a pap; a szolga, mint az õ ura; a szolgáló, mint asszonya; a vevõ, mint az eladó; a kölcsönadó, mint a kölcsönkérõ; a hitelezõ, mint az, a kinek hitelez;
\par 3 Megüresíttetvén megüresíttetik a föld, és elpusztíttatván elpusztíttatik; mert az Úr szólá e beszédet.
\par 4 Gyászol és megromol a föld, elhervad és megromol a földnek kereksége, elhervadnak a föld népének nagyjai.
\par 5 A föld megfertõztetett lakosai alatt, mert áthágták a törvényeket, a rendelést megszegték, megtörték az örök szövetséget.
\par 6 Ezért átok emészti meg a földet, és lakolnak a rajta lakók; ezért megégnek a földnek lakói, és kevés ember marad meg.
\par 7 Gyászol a must, elhervad a szõlõ, és sóhajtnak minden vidám szívûek.
\par 8 Megszünt a dobok vidámsága, elcsöndesült az örvendõk zajgása, a cziterának vídámsága megszünt.
\par 9 Énekléssel nem isznak bort; keserû a részegítõ ital az ívónak;
\par 10 Rommá lõn az álnokság városa, bezároltatott minden ház, senki be nem mehet!
\par 11 Az utczákon panaszkodás hallik a bor miatt; minden öröm alkonyra szállt, a föld vígassága elköltözött.
\par 12 A városban csak pusztaság maradt és rommá zúzatott a kapu.
\par 13 Mert így lesz a föld közepette, a népek között, mint az olajfa megrázásakor, mint mezgérléskor, midõn a szüre elmult.
\par 14 Õk felemelik szavokat, ujjongnak, az Úr nagyságáért rivalgnak a tenger felõl.
\par 15 Ezért dícsérjétek az Urat keleten, a tenger szigetein az Úrnak, Izráel Istenének nevét.
\par 16 A föld szélérõl énekeket hallánk: dicsõség az igaznak! S én mondék: végem van, végem van, jaj nékem! A hitetlenek hitetlenül cselekesznek és hitetlenséggel a hitetlenek hitetlenséget cselekesznek.
\par 17 Rettegés, verem és tõr vár rád földnek lakója!
\par 18 És lesz, hogy a ki fut a rettegésnek szavától, verembe esik, és a ki kijõ a verembõl, megfogatik a tõrben, mert az egek csatornái megnyílnak, és megrendülnek a föld oszlopai.
\par 19 Romlással megromol a föld, töréssel összetörik a föld, rengéssel megrendül a föld;
\par 20 Inogva meging a föld, miként a részeg, és meglódul, mint a kaliba, és reá nehezedik bûne és elesik; és nem kél fel többé!
\par 21 És lesz ama napon: meglátogatja az Úr a magasság seregét a magasságban, és a föld királyait a földön:
\par 22 És összegyûjtve összegyûjtetnek gödörbe, mint a foglyok, és bezáratnak tömlöczbe, és sok napok után meglátogattatnak.
\par 23 És elpirul a hold, és megszégyenül a nap, mikor a seregek Ura uralkodik Sion hegyén és Jeruzsálemben: s vénei elõtt dicsõség lészen.

\chapter{25}

\par 1 Uram, te vagy Istenem, magasztallak, dícsérem nevedet! Mivel csodát cselekvél, örök tanácsaid hûség és igazság.
\par 2 Mert a városból kõrakást csináltál, az erõs városból romot, az idegenek palotáját olyanná tetted, hogy nincs többé város; soha örökké meg nem épül.
\par 3 Ezért dicsõítnek téged erõs népek, erõszakos pogányok városai félnek téged!
\par 4 Mert erõssége voltál a gyöngének, erõssége a szegénynek szorongásában; a szélvész ellen oltalom, árnyék a hévség ellen, mikor az erõszakosok haragja olyan volt, mint kõfalrontó szélvész.
\par 5 Mint hévséget száraz földön, megalázod az idegeneknek háborgását, mint a hévség sûrû felleg miatt, megszünt a pogányoknak éneke.
\par 6 És szerez a seregek Ura minden népeknek e hegyen lakodalmat kövér eledelekbõl, lakodalmat erõs borból; velõs, kövér eledelekbõl, megtisztult erõs borból;
\par 7 S elveszi e hegyen a fátyolt, mely beboríta minden népeket, és a takarót, mely befödött vala minden népségeket;
\par 8 Elveszti a halált örökre, és letörli az Úr Isten a könyhullatást minden orczáról: és népe gyalázatát eltávolítja az egész földrõl; mert az Úr szólott.
\par 9 És szólnak ama napon: Ímé Istenünk, a kit mi vártunk és a ki megtart minket; ez az Úr, a kit mi vártunk, örüljünk és örvendezzünk szabadításában!
\par 10 Mert az Úr keze nyugszik e hegyen, és eltapostatik Moáb az õ földében, mint eltapostatik a szalma a ganaj levében.
\par 11 És kiterjeszti kezeit abban, mint kiterjeszti az úszó, hogy ússzék; de az Úr megalázza kevélységét és kezei csalárdságát;
\par 12 És magas falaid erõsségét lerontja, megalázza, a földre, porba dobja.

\chapter{26}

\par 1 Ama napon ez éneket énekelik Júda földén: Erõs városunk van nékünk, szabadítását adta kõfal és bástya gyanánt!
\par 2 Nyissátok fel a kapukat, hogy bevonuljon az igaz nép, a hûség megõrzõje.
\par 3 Kinek szíve reád támaszkodik, megõrzöd azt teljes békében, mivel Te benned bízik;
\par 4 Bízzatok az Úrban örökké, mert az Úrban, Jehovában, örök kõszálunk van.
\par 5 Mert meghorgasztá a magasban lakozókat: a magas várost, megalázá azt, megalázá azt a földig, és a porba dobta azt;
\par 6 Láb tapodja azt, a szegény lábai, a gyöngék talpai!
\par 7 Az igaznak ösvénye egyenes, egyenesen készíted az igaznak útját.
\par 8 Mi is vártunk Téged, ítéleted ösvényén, oh Uram! Te neved és emlékezeted után vágyott a lélek!
\par 9 Szívem utánad vágyott éjszaka, az én lelkem is bensõmben Téged keresett, mivel ha ítéleteid megjelennek a földön, igazságot tanulnak a földnek lakosai.
\par 10 Ha kegyelmet nyer a gonosz, nem tanul igazságot, az igaz földön is hamisságot cselekszik, és nem nézi az Úr méltóságát.
\par 11 Uram! magas a Te kezed, de nem látják! de látni fogják, és megszégyenülnek, néped iránt való buzgó szerelmedet; és tûz emészti meg ellenségeidet.
\par 12 Uram! Te adsz nékünk békességet, hisz minden dolgainkat megcselekedted értünk.
\par 13 Uram! mi Istenünk! urak parancsoltak nékünk kívüled; de általad dicsõítjük neved!
\par 14 A meghaltak nem élnek, az árnyak nem kelnek föl: ezért látogatád meg és vesztéd el õket, és eltörléd emlékezetöket.
\par 15 Megszaporítád e népet, Uram! megszaporítád e népet, magadat megdicsõítéd, kiterjesztetted a földnek minden határait.
\par 16 Oh Uram! a szorongásban kerestek Téged, és halk imádságot mondtanak, mikor rajtuk volt ostorod.
\par 17 Miként a terhes asszony, a ki közel a szûléshez, vajudik, felkiált fájdalmiban, elõtted olyanok voltunk, Uram!
\par 18 Mint terhesek vajudtunk, de csak szelet szûltünk: szabadulása nem lõn e földnek, és nem hulltak el a föld kerekségének lakói.
\par 19 Megelevenednek halottaid és holttesteim fölkelnek: serkenjetek föl és énekeljetek, a kik a porban lakoztok, mert harmatod az élet harmata, és visszaadja a föld az árnyakat!
\par 20 Menj be népem, menj be szobáidba, és zárd be ajtóidat utánad, és rejtsd el magad rövid szempillantásig, míg elmúlik a bús harag!
\par 21 Mert ímé az Úr kijõ helyérõl, hogy meglátogassa a föld lakóinak álnokságát, s felmutatja a föld a vért, és el nem fedi megöletteit többé!

\chapter{27}

\par 1 Ama napon meglátogatja az Úr kemény, nagy és erõs kardjával Leviatánt, a futó kígyót, Leviatánt, a keringõ kígyót, és megöli a sárkányt, mely a tengerben van.
\par 2 Ama napon a színború szõlõrõl énekeljetek:
\par 3 Én, az Úr, õrízem azt, minden szempillantásban öntözöm, hogy senki meg ne látogassa, éjjel-nappal megõrzöm azt;
\par 4 Nincsen haragom, de ha tövis és gaz jõ elém, csatára megyek ellene, és meggyújtom azt mind együtt;
\par 5 Avagy fogja meg erõsségemet, kössön békét velem, békét kössön velem!
\par 6 Jövendõben Jákób meggyökerezik, virágzik és virul Izráel, és betöltik a földnek színét gyümölcscsel.
\par 7 Avagy az õt verõnek verése szerint verte õt? vagy ellenségei megöletteinek megölése szerint öletett-é meg?
\par 8 Mérték szerint ítélted õt, midõn elvetetted! Reájok fútt kemény lehelletével a keleti szél napján.
\par 9 Ezért így tisztíttatik el a Jákób hamissága, és épen ez a gyümölcse bûne elvételének, hogy olyanná teszi az oltárnak minden kövét, minõk a széttört mészkövek: nem kelnek fel többé a berkek és naposzlopok!
\par 10 Mert az erõs város magánosan álland, üres és elhagyott hely, miként a puszta, ott legel a borjú és ott hever, és megemészti annak ágait.
\par 11 És ha megszáradnak gallyai, összetöretnek, az asszonyok elõjövén, megégetik azokat; mert értelem nélkül való e nép, ezért nem könyörül meg rajta Teremtõje, és nem kegyelmez néki alkotója.
\par 12 És lesz ama napon: cséplést tart az Úr az Eufrátes folyóvizétõl Égyiptom patakjáig, és ti egyenként összeszedettek, Izráel fiai.
\par 13 És lesz ama napon: megfújják a nagy kürtöt, és eljõnek, a kik elvesztek Assiria földében és a kik kiûzettek Égyiptom földébe, és leborulnak az Úr elõtt a szent hegyen, Jeruzsálemben.

\chapter{28}

\par 1 Jaj Efraim részegei kevély koronájának és dicsõséges ékessége hervadó virágának, mely a bortól megverettek kövér völgye fején van.
\par 2 Ímé, egy erõs és hatalmas jõ az Úrtól, mint jégfergeteg, veszedelmes szélvész, mint özönlõ erõs vizeknek áradása: földhöz veri kezével azt!
\par 3 Lábbal tapodtatik meg Efraim részegeinek kevély koronája,
\par 4 És úgy jár dicsõséges ékességének hervadó virága, a mely a kövér völgynek fején van, mint a korai füge a gyümölcsszedés elõtt, a melyet mihelyt valaki meglát, alig veszi kezébe, lenyeli.
\par 5 Ama napon a seregek Ura lesz ékes koronája és dicsõséges koszorúja népe maradékának;
\par 6 És ítéletnek lelke annak, a ki az ítélõszékben ül, és erõsség azoknak, a kik visszanyomják a kapuig az ellenséget.
\par 7 De ezek is tántorognak a bor miatt, és szédülnek a részegítõ italtól: pap és próféta tántorog részegítõ ital miatt, a bor elnyelte õket, szédülnek a részegítõ ital miatt, tántorognak a jövendõlátásban, és inognak az ítéletmondásban;
\par 8 Mert minden asztal telve undok okádással, úgy hogy hely sincs a nélkül.
\par 9 Kit tanít tudományra? A tanítást kivel érteti meg? A tejtõl elszakasztottakkal-é és a csecstõl elválasztottakkal-é?
\par 10 Mivel parancsra új parancs, parancsra új parancs, szabályra új szabály, szabályra új szabály; itt egy kicsi, ott egy kicsi.
\par 11 Ezért dadogó ajakkal és idegen nyelven fog szólni e néphez,
\par 12 Õ, a ki ezt mondá nékik: Ez a nyugalom, hogy nyugtassátok meg a megfáradottat, és ez a pihenés! És nem akarták hallani!
\par 13 És lõn nékik az Úr beszéde parancsra új parancs, parancsra új parancs, szabályra új szabály, szabályra új szabály; itt egy kicsi, ott egy kicsi; hogy járjanak és hátra essenek és összetöressenek és tõrbe jussanak és megfogassanak!
\par 14 Ezért halljátok az Úrnak beszédét, csúfoló férfiak, a kik uralkodtok e népen, a mely Jeruzsálemben lakik.
\par 15 Mert így szóltok: Frigyet kötöttünk a halállal, a sírral meg szövetséget csináltunk; az ostorozó áradat ha jõ, nem ér el minket; mert a hazugságot választók oltalmunkul, és csalásba rejtezénk el!
\par 16 Ezért így szól az Úr Isten: Ímé, Sionban egy követ tettem le, egy próbakövet, drága szegletkövet, erõs alappal, a ki benne hisz, az nem fut!
\par 17 És a jogosságot mérõkötéllé tevém, és az igazságot színlelõvé, és jég söpri el a hazugság oltalmát, és vizek ragadják el a rejteket.
\par 18 És eltöröltetik a halállal való frigyetek, és a sírral való szövetségtek meg nem áll; az ostorozó áradat ha eljõ, eltapod titeket,
\par 19 S a hányszor eljõ, elragad titeket; mert minden reggel eljõ, nappal és éjszaka; borzalom megértetni e tanítást;
\par 20 Mert rövid lesz az ágy, hogy benne kinyujtózhassék, és a takaró szûk lesz az elrejtõzéskor.
\par 21 Mert mint a Perázim hegyén, felkel az Úr, és mint Gibeon völgyében, megharagszik, hogy megtegye munkáját, a mely szokatlan lesz, és hogy cselekedje dolgát a mely hallatlan lesz.
\par 22 És most ne csúfolódjatok, hogy köteleitek szorosabbak ne legyenek; mert elvégzett pusztítást hallottam az Úrtól, a seregek Urától, az egész föld felett.
\par 23 Vegyétek füleitekbe és halljátok szavam', figyeljetek és hallgassátok beszédem'!
\par 24 Hát mindig szánt-é a szántó, hogy vessen, és barázdálja és boronálja-é földét?
\par 25 Nemde, mikor elegyengette színét, hint fekete köményt, és szór illatos köményt, s vet sorban búzát és árpát a kijelölt földbe, és tönkölyt a szélére?
\par 26 Így szoktatá õt rendre és tanítá Istene.
\par 27 Mert nem cséplõ szánkával csépelik a fekete köményt, és nem szekér kerekével tapodják az illatos köményt; a fekete köményt bottal verik ki, és az illatost pálczával;
\par 28 A búzát csépelik; de nem örökre csépli azt, és bár hajtja rajta szekere kerekét és lovait, de szét nem töreti.
\par 29 Ez is a seregek Urától származott: Õ ád csodás tanácsot és nagyságos bölcseséget!

\chapter{29}

\par 1 Jaj Árielnek, Árielnek, a városnak, a hol Dávid lakott! Esztendõt esztendõhöz adjatok, és forogjanak az ünnepek!
\par 2 És én megszorítom Árielt, és lesz fájdalom és siralom, és lesz nékem, mint Áriel.
\par 3 Körülveszlek táborral, és bezárlak tornyokkal, és erõsségeket állatok ellened.
\par 4 És megaláztatván, a földbõl szólsz és porból morog beszéded, szavad olyan lesz, mint halottidézõé, a földbõl, és porból sipog beszéded.
\par 5 Ellenségidnek sokasága olyan lesz, mint az apró por, és mint a repülõ polyva az erõszakosok sokasága, és lesz hamar és hirtelen.
\par 6 A seregek Urától látogattatik meg, mennydörgéssel, földindulással, nagy zúgással, forgószéllel, viharral és emésztõ tûzi lánggal;
\par 7 Mint éjjeli álomlátás, olyan lesz minden pogányoknak sokasága, a kik hadakoznak Áriel ellen, s a kik hadakoznak ellene és vára ellen, és õt megszorítják.
\par 8 És lesz, mint mikor álmodik az éhezõ, és ímé eszik, és midõn fölserken, üres a hasa, és mint mikor álmodik a szomjazó, és ímé iszik, és midõn fölserken, ímé szomjas és lelke eped: így lesz minden pogányoknak sokasága, a kik hadakoznak Sion hegye ellen.
\par 9 Ámuljatok és bámuljatok, vakítsátok magatokat és megvakultok! részegek, de nem bortól, tántorognak, de nem részegítõ italtól.
\par 10 Mert rátok önté az Úr a mély álomnak lelkét, és bezárta szemeiteket, a prófétákat, és fejeiteket, a nézõket befedezte;
\par 11 És lesz mind e látás néktek, mintegy bepecsételtetett írás beszédei, a melyet oda adnak egy írástudónak, mondván: Olvasd, kérlek, és õ szól: Nem tudom, mert bepecsételtetett.
\par 12 És ha e levelet annak adják, a ki nem tud írást, mondván: Olvasd el, kérlek! õ így szól: Nem tudok írást.
\par 13 És szólt az Úr: Mivel e nép szájjal közelget hozzám, és csak ajkaival tisztel engem, szíve pedig távol van tõlem, úgy  hogy irántam való félelmök betanított emberi parancsolat lõn:
\par 14 Ezért én is csodásan cselekszem ismét e néppel, nagyon csodálatosan, és bölcseinek bölcsesége elvész, és értelmeseinek értelme eltûnik.
\par 15 Jaj azoknak, a kik az Úrtól mélységesen elrejtik tanácsukat, és a kik a sötétségben szoktak cselekedni, mondván: Ki lát  minket és ki ismer minket?
\par 16 Mily együgyûek vagytok! Avagy a fazekas olyan, mint az agyag, hogy így szóljon a csinálmány csinálójának: Nem csinált engem! és az alkotmány ezt mondja alkotójának: Értelmetlen!
\par 17 Nemde kevés idõ multán a Libánon termõfölddé lesz, és a termõföld erdõnek tartatik?
\par 18 És meghallják ama napon a siketek az írás beszédeit, és a homályból és sötétbõl a vakoknak szemei látni fognak.
\par 19 És nagy örömük lesz a szenvedõknek az Úrban, és a szegény emberek vígadnak Izráel Szentjében.
\par 20 Mert a kegyetlen elveszett, és a csúfoló elpusztult, és kivágattak a hamisságnak minden õrei,
\par 21 Kik az embert elítélik egy szóért, és tõrt vetnek annak, a ki õket a kapuban megfeddi, és elejtik csalárdul az igazat.
\par 22 Ezért így szól az Úr Jákób házáról, Õ, a ki megváltá Ábrahámot: Nem szégyenül meg többé Jákób, és nem sáppad meg többé az õ orczája.
\par 23 Ha látni fogják gyermekei kezeim munkáját közöttük, megszentelik nevemet, megszentelik Jákób Szentjét, és félik Izráel Istenét.
\par 24 És megismerik a tévelygõ lelkûek az értelmet, és a kik zúgolódnak, tanulságot tanulnak.

\chapter{30}

\par 1 Jaj a pártos fiaknak, így szól az Úr, a kik tervet visznek véghez nélkülem, és szövetséget kötnek, de nem lelkem által, hogy bûnre bûnt halmozzanak!
\par 2 A kik Égyiptomba szállanak alá, és számat nem kérdezik meg, hogy meneküljenek a Faraónak oltalmába, és Égyiptom árnyékába rejtõzködjenek.
\par 3 És lesz néktek a Faraó oltalma szégyentekre, és az elrejtõzködés Égyiptom árnyékában gyalázatotokra.
\par 4 Mert már Zoánban voltak fejedelmei, és követei Hánesig érkezének.
\par 5 Megszégyenülnek mind a nép miatt, a mely nem használ nékik, a mely nem segít és nem használ, sõt szégyenökre és gyalázatukra lesz!
\par 6 Jövendölés dél barma ellen: A nyomor és szorongatás földén keresztül, a honnan nõstény és hím oroszlán és viperák és szárnyas sárkányok jõnek ki; viszik szamárcsikók hátán gazdagságukat, és a tevéknek púpján kincsöket a népnek, a mely pedig nem használ.
\par 7 Hitvány és üres Égyiptom segítsége, ezért nevezem õt nagyszájúnak, a ki veszteg ül,
\par 8 Most menj, írd ezt táblára nálok, és jegyezd föl könyvbe ezt, hogy megmaradjon az utolsó napra bizonyságul örökre.
\par 9 Mert pártütõ nép ez, apát megtagadó fiak, fiak, kik nem akarják hallani az Úr törvényét;
\par 10 Kik ezt mondják a látóknak: Ne lássatok; és a prófétáknak: Ne prófétáljatok nékünk igazat, beszéljetek kedvünk szerint valókat, prófétáljatok csalárdságokat!
\par 11 Hagyjátok el az útat, térjetek le már ez ösvényrõl, vigyétek el elõlünk Izráelnek Szentjét!
\par 12 Azért így szól Izráel Szentje: Mivel megútáltátok e beszédet, és bíztok a nyomorgatásban és a hamisságban, és ezekre támaszkodtok:
\par 13 Azért e bûn olyan lesz tinéktek, mint a leesendõ falhasadék, a mely már kiáll a magas kõfalon, a melynek aztán nagy hirtelen jõ el romlása;
\par 14 És romlása olyan lesz, mint a fazekasok edényének romlása, a mely kimélés nélkül eltöretik, és nem találni töredéki közt oly cserepet, a melyen tûzhelyrõl tüzet lehetne vinni, avagy vizet meríteni a tócsából.
\par 15 Mert így szól az Úr Isten, Izráelnek Szentje: Megtérve és megnyugodva megmaradhattatok volna; csöndességben és reménységben erõsségtek lett volna: de ti nem akarátok;
\par 16 Hanem ezt mondátok: Nem, sõt lóra ülvén, futunk; ezért futnotok kell; és gyors paripán elvágtatunk; ezért gyorsak lesznek üldözõitek.
\par 17 Ezer fut egynek riasztására, ötnek riasztására mind elfuttok, mígnem úgy maradtok, mint egy szál fenyõ a hegytetõn, mint egy zászló a halmon.
\par 18 S azért vár az Úr, hogy könyörüljön rajtatok, és azért felséges õ, hogy megkegyelmezzen néktek, mert az ítélet Istene az Úr; boldogok mindazok, a kik Õt szolgálják.
\par 19 Mert te nép, mely Sionon Jeruzsálemben lakol, nem fogsz te sírni többé, bizton könyörül rajtad Õ kiáltásod szavára, mihelyt meghallja megfelel néked;
\par 20 És ad néktek az Úr kenyeret a keserûségben, és a nyomorban vizet, és nem kell többé elrejtõzködniök tanítóidnak, hanem szemeid tanítóidra néznek!
\par 21 És füleid meghallják a kiáltó szót mögötted: ez az út, ezen járjatok; ha jobbra és ha balra elhajoltok.
\par 22 És megútáljátok megezüstözött bálványaitokat és megaranyozott képeiteket; kiszórod õket, mint az undokságot: ki innen! szólsz nékik.
\par 23 És ad esõt a magra, a melylyel a földet beveted, és a kenyér, a föld termése, bõ és tápláló lesz, és széles mezõn legelnek nyájaid ama napon.
\par 24 A barmok és a szamarak, a melyek a földet szántják, sózott abrakkal élnek, a melyet megszórtak lapáttal és villával.
\par 25 És lesznek minden magas hegyen és emelkedettebb halmon patakok, folyamok a nagy öldöklés napján, mikor a tornyok leomlanak.
\par 26 És a holdnak fénye olyan lesz, mint a napnak fénye, és a napnak fénye hétszer nagyobb lesz, olyan, mint hét napnak napfénye; ama napon, a melyen az Úr beköti népe romlását, és vereségének sebét meggyógyítja!
\par 27 Ímé, az Úr neve jõ messzirõl, haragja ég és sötét gomolygó füstje; ajkai rakvák haraggal, és nyelve, mint emésztõ tûz!
\par 28 Lehellete, mint megáradott patak, a mely torkig ér, hogy megrostálja a népeket pusztulás rostájában, és tévelygés zabláját vesse a népségek szájába.
\par 29 És fölzendül éneketek, mint a szent ünnepnek éjszakáján, és örvendez szívetek, mint azé, a ki sípolva megy az Úr hegyére, Izráel kõszálához.
\par 30 És megzendíti az Úr dicsõséges szavát, karjának lesujtását megmutatja megbúsult haragjában és emésztõ tûz lángjában, vízáradással, zivatarral és jégesõ kövével.
\par 31 Mert az Úrnak szavától megretten Assiria, veri azt akkor vesszõvel;
\par 32 És a büntetõ vesszõ minden sujtását, a melyet az Úr mérend reá, dobokkal és cziterákkal fogjátok kisérni, és Õ kezét fel-felemelvén, harczol ellene.
\par 33 Mert készen van a szörnyû tûzhely régen, készen áll az már a királynak is, mélyen és szélesen csinálta azt, máglyájában tûz és fa bõven; az Úr fuvallata gyújtja meg azt, mint kénköves patak.

\chapter{31}

\par 1 Jaj nékik, a kik Égyiptomba mennek segítségért, lovakra támaszkodnak, és a szekerek sokaságában bíznak, és a nagyon erõs lovagokban; és nem néznek Izráelnek Szentjére, és az Urat nem keresik.
\par 2 De Õ is bölcs, és hoz veszedelmet, és beszédeit nem változtatja meg; hanem fölkel a gonoszoknak háza ellen és a bûnt cselekvõk segítsége ellen.
\par 3 Hiszen Égyiptom ember és nem Isten, és lovai hús és nem lélek, és ha az Úr kinyújtja kezét, megtántorodik a segítõ, és elesik a megsegített, és együtt mind elvesznek.
\par 4 Mert így szólott az Úr hozzám: A mint mormol az oroszlán és az oroszlánkölyök zsákmánya mellett, a mely ellen összehívatnak a pásztorok sereggel, és szavoktól õ meg nem retten, és meg nem ijed sokaságuktól: így száll alá a seregek Ura, hogy hadakozzék a Sion hegyén és halmán!
\par 5 Mint repesõ madarak, úgy oltalmazza a seregek Ura Jeruzsálemet, oltalmazván megszabadítja, kimélvén megmenti.
\par 6 Térjetek vissza hát hozzá, a kitõl oly nagyon elpártolátok, Izráelnek fiai!
\par 7 Mivel ama napon megveti mindenki ezüst bálványait és arany bálványait, a melyeket kezeitek csináltak néktek bûnre.
\par 8 És elesik Assiria, nem férfiú kardjától, és nem ember kardja emészti meg azt; és fut egy kard elõtt, és ifjai adófizetõk lesznek;
\par 9 És kõszála félelem miatt menekül, és a zászlótól fejedelmei elfutnak, szól az Úr, a kinek tüze Sionban van, és kemenczéje Jeruzsálemben.

\chapter{32}

\par 1 Ímé, igazság szerint uralkodik a király, és a fejedelmek fõk lesznek az ítélettételben;
\par 2 Olyan lesz mindegyik, mint a rejtek szél ellen, mint oltalom zivatar ellen, mint patakok száraz vidéken, mint nagy kõszál árnyéka a szomjúhozó földön.
\par 3 És nem lesznek zárva többé a látóknak szemei, és a hallók fülei figyelmeznek;
\par 4 A hebehurgyák szíve ismerni tanul, és a dadogóknak nyelve gyorsan és világosan szól.
\par 5 Nem nevezik a bolondot többé nemesnek, és a csalárdot sem hívják nagylelkûnek.
\par 6 Mert a bolond csak bolondot beszél, és az õ szíve hamisságot forral, hogy istentelenséget cselekedjék, és szóljon az Úr ellen tévelygést, hogy az éhezõ lelkét éhen hagyja, és a szomjazó italát elvegye.
\par 7 A csalárdnak eszközei csalárdok, õ álnokságot tervel, hogy elveszesse az alázatosokat hazug beszéddel, ha a szegény igazat szólna is.
\par 8 De a nemes nemes dolgokat tervel, és a nemes dolgokban meg is marad.
\par 9 Ti gondtalan asszonyok, keljetek fel, halljátok szavam', és ti elbizakodott leányzók, vegyétek füleitekbe beszédem'!
\par 10 Kevés idõ multán megháborodtok ti elbizakodottak, mert elvész a szüret, és gyümölcsszedés sem lesz.
\par 11 Reszkessetek ti gondtalanok, rettenjetek meg, ti elbizakodottak; vetkezzetek mezítelenre és övezzétek fel ágyékaitokat gyászruhával.
\par 12 Gyászolják az emlõket, a szép mezõket, a termõ szõlõtõket.
\par 13 Népemnek földét tüske, tövis verte fel, és az örvendõ városnak minden öröm-házait;
\par 14 A paloták elhagyatvák, a város zaja elnémult, torony és bástya barlangokká lettek örökre, hol a vadszamár kedvére él, a nyájak meg legelnek.
\par 15 Míglen kiöntetik reánk a lélek a magasból, és lészen a puszta termõfölddé, és a termõföld erdõnek tartatik;
\par 16 És lakozik a pusztában jogosság, és igazság fog ülni a termõföldön;
\par 17 És lesz az igazság mûve békesség, és az igazság gyümölcse nyugalom és biztonság mindörökké.
\par 18 Népem békesség hajlékában lakozik biztonság sátraiban, gondtalan nyugalomban.
\par 19 De jégesõ hull és megdõl az erdõ, és a város elsülyedve elsülyed!
\par 20 Oh boldogok ti, a kik minden vizek mellett vettek, és szabadon eresztitek a barmok és szamarak lábait!

\chapter{33}

\par 1 Jaj néked pusztító és el nem pusztított, te csalárd, a kit még meg nem csaltak! Ha bevégzed a pusztítást, el fogsz pusztíttatni; ha készen leszel csalárdságoddal, téged fognak megcsalni.
\par 2 Uram, könyörülj rajtunk! Téged várunk; légy karjuk reggelenként, és szabadítónk a szorongatásnak idején!
\par 3 Egy zendülõ szózattól elfutnak a népek; ha te felemelkedel, elszélednek a népségek.
\par 4 És elpusztítják zsákmánytokat sáska pusztításával, szöcske-ugrással ugrálnak reá.
\par 5 Felmagasztaltatott az Úr, mert magasságban lakozik, betölté Siont ítélettel és igazsággal.
\par 6 És békés lesz a te idõd, gazdag boldogságban, bölcseségben és tudományban; az Úr félelme lesz kincse.
\par 7 Ímé, erõseik ott künn kiáltanak, a békesség követei keservesen sírnak.
\par 8 Puszták az ösvények, megszünt az úton járó; megszegte a frigyet, lenézte a városokat, nem gondolt az emberrel!
\par 9 Sírt, meghervadt a föld, a Libánon megszégyenült, ellankadt; olyan lett Sáron, mint egy puszta, és lombtalan Básán és Karmel.
\par 10 Most fölkelek, így szól az Úr, most föltámadok, most fölemelkedem!
\par 11 Fogantok szalmát, szültök polyvát, dühötök tûz, megemészt titeket.
\par 12 A népek égetett mészszé lesznek, levágott tövisekké; tûzben hamvadnak el.
\par 13 Halljátok meg távol valók, a mit cselekedtem, és tudjátok meg közel valók az én hatalmamat!
\par 14 Megrettentek a bûnösök Sionban, félelem fogja el a gazokat: ki lakhatik közülünk megemésztõ tûzzel, ki lakhatik közülünk örök hõséggel?
\par 15 A ki igazságban jár és egyenesen beszél, a ki megveti a zsarolt nyereséget, a ki kezeit rázván, nem vesz ajándékot, a ki fülét bedugja, hogy véres tervet ne halljon, és szemeit befogja, hogy gonoszt ne lásson:
\par 16 Az magasságban lakozik, kõszálak csúcsa a bástyája, kenyerét megkapja, vize el nem fogy.
\par 17 A királyt ékességében látják szemeid; látnak széles országot.
\par 18 Szíved elgondolja a mult félelmét: hol az író, hol a mérlegelõ, hol a tornyok összeírója?
\par 19 A gõgös népet nem látod, a homályos, érthetetlen ajkú népet, dadogó nyelve meg nem érthetõ.
\par 20 Lássad Siont, ünnepeinknek városát, szemeid nézzék Jeruzsálemet, mint nyugalom hajlékát, mint sátort, mely nem vándorol, melynek szegei soha ki nem húzatnak, s kötelei soha el nem szakadnak;
\par 21 Sõt az Úr, a dicsõséges lesz ott nékünk folyók és széles vizek gyanánt, a melyekbe nem jõ evezõs hajó, és nehéz gálya rajtok át nem megy.
\par 22 Mert az Úr a mi bíránk, az Úr a mi vezérünk, az Úr a mi királyunk, Õ tart meg minket!
\par 23 Megtágultak köteleid, árbóczfájok alapját nem tartják erõsen, vitorlát nem feszítenek: akkor sok rablott prédát osztanak, még a sánták is zsákmányt vetnek.
\par 24 És nem mondja a lakos: beteg vagyok! a nép, a mely benne lakozik, bûnbocsánatot nyer.

\chapter{34}

\par 1 Jõjjetek népek és halljátok, nemzetségek figyeljetek, hallja a föld és teljessége, e föld kereksége és minden szülöttei.
\par 2 Mert haragszik az Úr minden népekre, és megbúsult minden õ seregökre; megátkozá, halálra adta õket.
\par 3 Megöltjeik temetetlen maradnak, hulláik bûze felszáll, és hegyek olvadnak meg vérök miatt.
\par 4 Elporhad az ég minden serege, és az ég mint írás egybehajtatik, és minden serege lehull, miként lehull a szõlõ levele és a fügefáról a hervadó lomb.
\par 5 Mert megrészegült fegyverem az égben, és ímé leszáll Edomra, átkom népére, ítéletre.
\par 6 Az Úr fegyvere telve vérrel, megrakva kövérrel, bárányoknak és bakoknak vérével, a kosoknak vesekövérével; mert áldozatja lesz az Úrnak Boczrában, és nagy öldöklés Edom földén.
\par 7 Elhullnak a bivalyok is velök, és a tulkok a bikákkal, és megrészegedik földük vértõl, és poruk borítva lesz kövérrel.
\par 8 Mert bosszúállás napja ez az Úrnak, a megfizetés esztendeje Sionnak ügyéért.
\par 9 És változnak patakjai szurokká, és pora kénkõvé, és lészen földe égõ szurokká.
\par 10 Éjjel és nappal el nem alszik, örökre fölgomolyog füstje, nemzetségrõl nemzetségre pusztán marad, soha örökké senki át nem megy rajta;
\par 11 És örökségül bírándja azt ökörbika, sündisznó; és gém és holló lakja azt, és fölvonják rá a pusztaság mérõkötelét és a semmiségnek köveit.
\par 12 Nemesei nem választanak többé királyt, és minden fejedelmei semmivé lesznek.
\par 13 És fölveri palotáit tövis, csalán és bogács a bástyáit, és lesz sakálok hajléka és struczok udvara.
\par 14 És találkozik vadmacska a vadebbel, és a kisértet társára talál, csak ott nyugszik meg az éji boszorkány és ott lel nyughelyet magának.
\par 15 Oda rak fészket a bagoly és tojik és ül tojáson és költ árnyékában, csak ott gyûlnek együvé a sasok!
\par 16 Keressétek meg majd az Úr könyvében, és olvassátok: ezeknek egy hijjok sem lesz, egyik a másiktól el nem marad; mert az Õ szája parancsolta, és az Õ lelke gyûjté össze õket!
\par 17 Õ vetett sorsot köztök, és keze osztá ki azt nékik mérõkötéllel; örökre bírni fogják azt, nemzetségrõl nemzetségre lakoznak abban.

\chapter{35}

\par 1 Örvend a puszta és a kietlen hely, örül a pusztaság és virul mint õszike.
\par 2 Virulva virul és örvend ujjongva, a Libánon dicsõsége adatott néki, Karmel és Sáron ékessége; meglátják õk az Úrnak dicsõségét, Istenünk ékességét.
\par 3 Erõsítsétek a lankadt kezeket, és szilárdítsátok a tántorgó térdeket.
\par 4 Mondjátok a remegõ szívûeknek: legyetek erõsek, ne féljetek! Ímé, Istenetek bosszúra jõ, az Isten, a ki megfizet, Õ jõ, és megszabadít titeket!
\par 5 Akkor a vakok szemei megnyílnak, és a süketek fülei megnyittatnak,
\par 6 Akkor ugrándoz, mint szarvas a sánta, és ujjong a néma nyelve, mert a pusztában víz fakad, és patakok a kietlenben.
\par 7 És tóvá lesz a délibáb, és a szomjú föld vizek forrásivá; a sakálok lakhelyén, a hol feküsznek, fû, nád és káka terem.
\par 8 És lesz ott ösvény és út, és szentség útának hívatik: tisztátalan nem megy át rajta; hisz csak az övék az; a ki ez úton jár, még a bolond se téved el;
\par 9 Nem lesz ott oroszlán, és a kegyetlen vad nem jõ fel reá, nem is található ott, hanem a megváltottak járnak rajta!
\par 10 Hisz az Úr megváltottai megtérnek, és ujjongás között Sionba jönnek; és örök öröm fejökön, vígasságot és örömöt találnak: és eltûnik fájdalom és sóhaj.

\chapter{36}

\par 1 És lõn Ezékiás király tizennegyedik esztendejében, feljöve Szanhérib assir király Júdának minden erõs városai ellen, és azokat megvevé.
\par 2 És elküldé Assiria királya Rabsakét Lákisból Jeruzsálembe, Ezékiás királyhoz nagy seregggel, ki is megállt a felsõ tó folyásánál, a ruhamosók mezejének útján.
\par 3 És kijöve hozzá Eljákim a Hilkiás fia az udvarnagy, és Sebna az íródeák, és Jóák Asáfnak fia az emlékíró.
\par 4 És mondá nékik Rabsaké: Mondjátok meg kérlek Ezékiásnak, így szól a nagy király, Assiria királya: Micsoda bizodalom ez, melyre támaszkodol?
\par 5 Azt mondom, hogy csak szóbeszéd az, hogy ész és erõ van nálatok a háborúhoz; no hát kiben bízol, hogy ellenem feltámadál?
\par 6 Ímé te e megtört nádszálban bízol, Égyiptomban, melyre a ki támaszkodik, tenyerébe megy és átfúrja azt; ilyen a Faraó, Égyiptomnak királya, minden benne bízóknak.
\par 7 És ha azt mondod nékem: az Úrban, a mi Istenünkben bízunk: vajjon nem Õ-é az, a kinek magaslatait és oltárait elrontotta Ezékiás, és ezt mondá Júdának és Jeruzsálemnek: Ez elõtt az oltár elõtt hajoljatok meg.
\par 8 Most azért harczolj meg, kérlek, az én urammal, az assir királylyal, és adok néked két ezer lovat, ha ugyan tudsz lovagokat ültetni reájok.
\par 9 Miképen állasz ellene egyetlen helytartónak is, a ki legkisebb az uram szolgái közt? De hisz Égyiptomban van bizalmad, a szekerekért és lovagokért.
\par 10 És most talán az Úr nélkül jöttem én e földre, hogy elpusztítsam azt? Az Úr mondá nékem: Menj a földre, és pusztítsd el azt!
\par 11 És mondá Eljákim és Sebna és Jóák Rabsakénak: Szólj, kérünk, szolgáidhoz arám nyelven, mert értjük azt, és ne zsidóul szólj hozzánk, e kõfalon levõ nép füle hallatára.
\par 12 És mondá Rabsaké: Avagy a te uradhoz, vagy te hozzád küldött engem az én uram, hogy ezeket elmondjam? és nem az emberekhez-é, a kik ülnek a kõfalon, hogy egyék ganéjokat és igyák vizeletöket veletek együtt?!
\par 13 És odaálla Rabsaké, és kiálta felszóval zsidóul, és mondá: Halljátok a nagy királynak, Assiria királyának beszédit!
\par 14 Ezt mondja a király: Meg ne csaljon benneteket Ezékiás, mert nem szabadíthat meg titeket.
\par 15 És ne biztasson titeket Ezékiás az Úrral, mondván: Kétségtelen megszabadít az Úr minket, nem adatik e város az assiriai király kezébe!
\par 16 Ezékiásra ne hallgassatok, mert azt mondja Assiria királya: Tegyetek velem szövetséget és jõjjetek ki hozzám, és akkor kiki ehetik szõlõjébõl és az õ olajfájáról, és ihatja kútjának vizét.
\par 17 Míg eljövök és elviszlek titeket oly földre, minõ a ti földetek, gabona és must földére, kenyér és szõlõ földére.
\par 18 Rá ne szedjen titeket Ezékiás, mondván: Az Úr megszabadít minket! Avagy megszabadították-é a népek istenei, kiki az õ földét Assiria királyának kezébõl?
\par 19 Hol vannak Hamáth és Arphádnak istenei? hol Sefarvaimnak istenei? talán bizony megmentették Samariát kezembõl?
\par 20 Kicsoda e földek minden istenei között, a ki megszabadította volna földét kezembõl, hogy az Úr megszabadítsa Jeruzsálemet az én kezembõl?
\par 21 Õk pedig hallgatának, és egy szót sem feleltek, mert a király parancsolá így, mondván: Ne feleljetek néki!
\par 22 Akkor elmenének Eljákim a Hilkiás fia az udvarnagy, és Sebna az íródeák, és Jóák Asáf fia az emlékíró, Ezékiáshoz meghasogatott ruhákban, és hírül adák néki a Rabsaké beszédeit.

\chapter{37}

\par 1 És lõn, hogy meghallá Ezékiás, a király, meghasogatá ruháit, gyászba öltözött, és bement az Úr házába.
\par 2 És elküldé Eljákimot az udvarnagyot, és Sebnát az íródeákot, és a papoknak gyászruhába öltözött véneit Ésaiáshoz, Ámós fiához, a prófétához.
\par 3 Kik mondának néki: Így szól Ezékiás: nyomornak, büntetésnek és káromlásnak napja e nap, mert szülésig jutottak a fiak, és erõ nincs a szüléshez!
\par 4 Talán meghallja az Úr, a te Istened, a Rabsaké beszédeit, a kit elküldött az õ ura, az assiriai király, hogy káromolja az élõ Istent, és szidalmazza azon beszédekkel, a melyeket hallott az Úr, a te Istened; és te könyörögj a maradékért, a mely megvan!
\par 5 Így menének el Ezékiás király szolgái Ésaiáshoz.
\par 6 És monda nékik Ésaiás: Így szóljatok uratoknak: ezt mondja az Úr: Ne félj a beszédektõl, a melyeket hallottál, a melyekkel szidalmaztak engem az assiriai király szolgái.
\par 7 Ímé, én oly lelket adok beléje, hogy hírt hallván, térjen vissza földére, és elejtem õt fegyver által az õ földében!
\par 8 Rabsaké pedig visszatérvén, találá az assiriai királyt Libna ellen harczolni, mivel hallotta volt, hogy Lákistól elindult.
\par 9 És meghallván Tirháka, Kús királya felõl e hírt: eljött, hogy ellened harczoljon; ezt meghallván, követeket külde Ezékiáshoz, mondván:
\par 10 Így szóljatok Ezékiáshoz, Júda királyához: Meg ne csaljon Istened, a kiben bízol, mondván: nem adatik Jeruzsálem az assiriai király kezébe!
\par 11 Hiszen te hallottad, mit mûveltek Assiria királyai minden országokkal, eltörölvén azokat, és te megszabadulnál?
\par 12 Hát megszabadíták-é azokat a népek istenei, a melyeket eleim elpusztítának? Gózánt, Háránt, Resefet és Telassárban Eden fiait?
\par 13 Hol van Hamáth királya és Arphádnak királya, Sefarvaim városának királya, Héna és Ivva?
\par 14 És elvevé Ezékiás a levelet a követek kezébõl, és olvasá azt, és felmenvén az Úr házába, kiterjeszté Ezékiás azt az Úr elõtt.
\par 15 És könyörge Ezékiás az Úrhoz mondván:
\par 16 Seregeknek Ura, Izráel Istene, ki a Kerubokon ülsz! Te, csak Te vagy a föld minden országainak Istene, Te teremtéd a mennyet és a földet.
\par 17 Hajtsd ide Uram füledet és halljad, nyisd meg Uram szemeidet és lássad! Halld meg Szanhéribnek minden beszédit, a melyeket izent az élõ Isten káromlására!
\par 18 Bizony, Uram, Assiria királyai elpusztítottak minden országokat és azoknak földjét,
\par 19 És isteneiket tûzbe veték, mert nem istenek voltak, hanem emberi kéznek csinálmánya, fa és kõ; így veszthették el azokat.
\par 20 És most Uram, mi Istenünk! szabadíts meg minket az õ kezébõl, hogy megtudják a föld minden országai, hogy Te vagy, Uram, Isten egyedül.
\par 21 És külde Ésaiás, Ámós fia Ezékiáshoz, mondván: Így szól az Úr, Izráel Istene: mivel hozzám könyörögtél Szanhérib miatt, Assiria királya miatt:
\par 22 Ez a beszéd, a melyet az Úr felõle szól: Megútál téged, gúnyt ûz belõled Sionnak szûz leánya, fejét rázza utánad Jeruzsálem leánya;
\par 23 Kit káromoltál és szidalmazál, és ki ellen emeltél szót, hogy oly magasra látsz? Izráel Szentje ellen!
\par 24 Szolgáid által megkáromlád az Urat, és ezt mondád: Temérdek szekeremmel fölmentem e hegyek magaslatára, a Libánon csúcsára, és kivágom magas czédrusait, válogatott cziprusait, és behatolok végsõ magaslatába és kertjének erdejébe!
\par 25 Én ástam és vizet ittam, és kiszáraztom lábam talpával Égyiptom minden folyóit.
\par 26 Hát nem hallottad-é, hogy régtõl fogva én tevém ezt, az õskor napjaiban elvégzém ezt, és most elõhozám, hogy puszta kõrakássá tégy erõs városokat?!
\par 27 És lakóik elájultak és megrendültek és megszégyenültek, és lõnek mint a mezõ füve és gyönge zöldség, és mint a virág a háztetõn, mint szárba nem indult vetés!
\par 28 Ülésedet, kimentedet, bejöttödet tudom, és ellenem való haragodat.
\par 29 Ellenem való haragodért, és mert kevélységed fülembe jutott, vetem orrodba horgomat és szádba zabolámat, és visszaviszlek az úton, a melyen jövél!
\par 30 S ez legyen jelül néked: ez évben ugartermést esztek és a másik évben sarjut, és a harmadik évben vessetek, arassatok és szõlõt ültessetek, és egyétek gyümölcsét.
\par 31 Júda házának maradványa pedig, a mely megszabadult, ismét gyökeret ver alul, és gyümölcsöt terem felül.
\par 32 Mert Jeruzsálembõl megy ki a maradék és a maradvány Sion hegyérõl; a seregek Urának buzgó szerelme mûvelendi ezt!
\par 33 Azért így szól az Úr Assiria királyáról: Nem jõ be e városba, nyilat sem lõ reá, és nem szállja meg paizszsal azt, és töltést sem készít ellene.
\par 34 Az úton, a melyen jött, visszatér, de e városba be nem jõ, azt mondja az Úr.
\par 35 És megoltalmazom e várost, hogy megtartsam azt én magamért és szolgámért, Dávidért!
\par 36 Akkor kijött az Úrnak angyala, és levágott az assir táborban száznyolczvanötezeret, és midõn reggel az emberek felköltek, ímé azok mindnyájan holt hullák valának!
\par 37 Elindula azért és ment és visszatért Szanhérib, az assiriai király, és lakozék Ninivében.
\par 38 És lõn, hogy mikor imádkozék, Nisróknak, az õ istenének templomában, fiai: Adramélek és Saréser levágák õt karddal; és ezek Ararát földére menekülvén, fia, Esárhaddon uralkodék helyette.

\chapter{38}

\par 1 Azon napokban halálos betegségbe esék Ezékiás, és eljött hozzá Ésaiás Ámós fia, a próféta, és mondá néki: Ezt mondja az Úr: rendeld el házadat, mert meghalsz és meg nem gyógyulsz!
\par 2 És Ezékiás arczczal a falnak fordulván, könyörge az Úrnak,
\par 3 És monda: Oh Uram, emlékezzél meg arról, hogy én elõtted jártam, igazságban és egész szívvel, és hogy a mi jó elõtted, azt mûveltem! és sírt Ezékiás keservesen.
\par 4 És lõn az Úr beszéde Ésaiáshoz, mondván:
\par 5 Menj el, és mondd Ezékiásnak: így szól az Úr, Dávidnak, atyádnak Istene: Hallottam imádságodat, láttam könyeidet, ímé, még napjaidhoz tizenöt esztendõt adok.
\par 6 És az assiriai király kezébõl megszabadítlak téged és e várost; megoltalmazom e várost!
\par 7 Ez legyen jel néked az Úrtól, hogy teljesíti azt az Úr, a mit mondott:
\par 8 Ímé, visszatérítem az árnyékot, azokon a fokokon, a melyeken az Akház napóráján a nap már átvonult, tíz fokkal; és visszatért az árnyék tíz fokkal azokon a fokokon, a melyeken már átvonult.
\par 9 Ezékiásnak, Júda királyának följegyzése, mikor megbetegedett, és betegségébõl fölgyógyult.
\par 10 Én azt mondám: hát napjaimnak nyugalmában kell alászállanom a sír kapuihoz, megfosztva többi éveimtõl!
\par 11 Mondám: nem látom az Urat, az Urat az élõk földében, nem szemlélek embert többé a nyugalom lakói közt.
\par 12 Porsátorom lerontatik, és elmegy tõlem, mint a pásztor hajléka! Összehajtám, mint a takács, életemet; hiszen levágott a fonalról engem; reggeltõl estig végzesz velem!
\par 13 Reggelig nyugton vártam; mint oroszlán, úgy törte össze minden csontjaimat; reggeltõl estig végzesz velem!
\par 14 Mint a fecske és a daru, sipogtam, nyögtem mint a galamb, szemeim a magasságba meredtek: Uram! erõszak rajtam, szabadíts meg!
\par 15 Mit mondjak? hogy szólott nékem és Õ azt meg is cselekedé! Nyugton élem le éveimet lelkem keserûsége után!
\par 16 Oh Uram! ezek által él minden! és ezekben van teljességgel lelkem élete. Te meggyógyítasz és éltetsz engemet!
\par 17 Ímé, áldásul volt nékem a nagy keserûség, és Te szeretettel kivontad lelkemet a pusztulásnak vermébõl, mert hátad mögé vetetted minden bûneimet!
\par 18 Mert nem a sír dicsõít Téged, és nem a halál magasztal Téged, hûségedre nem a sírverembe szállók várnak!
\par 19 Ki él, ki él, csak az dicsõít Téged, mint ma én! Az atya a fiaknak hirdeti hûségedet!
\par 20 Az Úr szabadított meg engemet; azért énekeljük énekimet éltünk minden napjaiban az Úrnak házában!
\par 21 Akkor mondá Ésaiás, hogy vegyenek egy fügekalácsot, és dörzsöljék rá a fekélyre, hogy meggyógyuljon.
\par 22 És mondá Ezékiás: Mi lesz a jele, hogy fölmegyek az Úr házába?

\chapter{39}

\par 1 Abban az idõben levelet és ajándékot küldött a babilóniai király, Meródák Baladán, Baladán fia, Ezékiás királyhoz, mert hallotta, hogy beteg volt és meggyógyult.
\par 2 És örvende rajtok Ezékiás, és megmutatá nékik tárházát, az ezüstöt, az aranyat, a fûszereket, a drága kenetet, s egész fegyvertárát, és mindent, a mi kincsei közt található volt. Semmi nem volt, a mit meg nem mutatott volna nékik Ezékiás házában és egész birodalmában.
\par 3 És eljött Ésaiás, a próféta, Ezékiás királyhoz, és monda néki: Mit szólának ez emberek, és honnan jöttek te hozzád? És monda Ezékiás: Messze földrõl jöttek hozzám, Bábelbõl.
\par 4 És monda: Mit látának házadban? És monda Ezékiás: Mindent láttak, a mi csak házamban van, semmi nincs, a mit meg nem mutattam volna nékik kincseim közül.
\par 5 És monda Ésaiás Ezékiásnak: Halld a seregek Urának beszédét:
\par 6 Ímé napok jõnek, és elvitetik, valami házadban van, és a mit csak e mai napig gyûjtöttek eleid, Bábelbe; nem marad semmi meg, ezt mondja az Úr!
\par 7 És fiaid közül, a kik tõled származnak, a kiket te nemzesz, el fognak hurczolni, és lesznek komornyikok Bábel királyának palotájában.
\par 8 Akkor monda Ezékiás Ésaiásnak: Jóságos az Úrnak beszéde, a melyet te szóltál! és monda: Csak napjaimban legyen béke és állandóság!

\chapter{40}

\par 1 Vígasztaljátok, vígasztaljátok népemet, így szól Istenetek!
\par 2 Szóljatok Jeruzsálem szívéhez, és hirdessétek néki, hogy vége van nyomorúságának, hogy bûne megbocsáttatott; hiszen kétszeresen sujtotta õt az Úr keze minden bûneiért.
\par 3 Egy szó kiált: A pusztában készítsétek az Úrnak útát, ösvényt egyengessetek a kietlenben a mi Istenünknek!
\par 4 Minden völgy fölemelkedjék, minden hegy és halom alászálljon, és legyen az egyenetlen egyenessé és a bérczek rónává.
\par 5 És megjelenik az Úr dicsõsége, és minden test látni fogja azt; mert az Úr szája szólt.
\par 6 Szózat szól: Kiálts! és monda: Mit kiáltsak? Minden test fû, és minden szépsége, mint a mezõ virága!
\par 7 Megszáradt a fû, elhullt a virág, ha az Úrnak szele fuvallt reá; bizony fû a nép.
\par 8 Megszáradt a fû, elhullt a virág; de Istenünk beszéde mindörökre megmarad!
\par 9 Magas hegyre mej fel, örömmondó Sion! emeld föl szódat magasan, örömmondó Jeruzsálem! emeld föl, ne félj! mondjad Júda városinak: Ímhol Istenetek!
\par 10 Ímé, az Úr Isten jõ hatalommal, és karja uralkodik! Ímé, jutalma vele jõ, és megfizetése Õ elõtte.
\par 11 Mint pásztor, nyáját úgy legelteti, karjára gyûjti a bárányokat és ölében hordozza, a szoptatósokat szelíden vezeti.
\par 12 Ki mérte meg markával a vizeket, és ki mértéklé az egeket arasszal, a föld porát ki foglalá mérczébe, és a hegyeket ki tette körtefontra, és a halmokat a mérlegserpenyõbe?
\par 13 Kicsoda igazgatta az Úr lelkét, és ki oktatta Õt, mint tanácsosa?
\par 14 Kivel tanácskozott, hogy felvilágosítsa Õt, és tanítsa Õt igazság ösvényére, és tanítsa ismeretre, és oktassa Õt az értelem útára?
\par 15 Ím a népek, mint egy csöpp a vederben, és mint egy porszem a mérlegserpenyõben, olyanoknak tekintetnek; ímé a szigeteket mint kis port emeli föl!
\par 16 És a Libánon nem elég a tûzre, és vada sem elég az áldozatra.
\par 17 Minden népek semmik Õ elõtte, a semmiségnél és ürességnél alábbvalónak tartja.
\par 18 És kihez hasonlítjátok az Istent, és minõ képet készítetek Õ róla?
\par 19 A bálványt a mester megönti, és az ötvös megaranyozza azt, és olvaszt ezüst lánczot reá;
\par 20 A ki szegény ily áldozatra, oly fát választ, a mely meg nem rothad; okos mestert keres, hogy oly bálványt állítson, a mely nem ingadoz.
\par 21 Hát nem tudjátok és nem hallottátok-é, hát nem hirdettetett néktek eleitõl fogva, hát nem értettétek-é meg a föld fundamentomait?
\par 22 Ki ül a föld kereksége fölött, a melynek lakói mint sáskák elõtte, ki az egeket kiterjeszti mint egy kárpitot, és kifeszíti, mint a sátort, lakásra;
\par 23 Ki a fejedelmeket semmivé teszi, és a föld biráit hiábavalókká változtatja;
\par 24 Még alig plántáltattak, még alig vettetének el, alig vert gyökeret a földben törzsük, és Õ csak rájok fuvall, és kiszáradnak és õket, mint polyvát, forgószél ragadja el:
\par 25 Kihez hasonlíttok hát engem, hogy hasonló volnék? szól a Szent.
\par 26 Emeljétek föl a magasba szemeiteket, és lássátok meg, ki teremté azokat? Õ, a ki kihozza seregöket szám szerint, mindnyáját nevén szólítja; nagy hatalma és erõssége miatt egyetlen híjok sincsen.
\par 27 Miért mondod Jákób és szólsz ekként Izráel: Elrejtetett az én útam az Úrtól, és ügyemmel nem gondol Istenem?!
\par 28 Hát nem tudod-é és nem hallottad-é, hogy örökkévaló Isten az Úr, a ki teremté a föld határait? nem fárad és nem lankad el; végére mehetetlen bölcsesége!
\par 29 Erõt ad a megfáradottnak, és az erõtlen erejét megsokasítja.
\par 30 Elfáradnak az ifjak és meglankadnak, megtántorodnak a legkülönbek is;
\par 31 De a kik az Úrban bíznak, erejök megújul, szárnyra kelnek, mint a saskeselyûk, futnak és nem lankadnak meg, járnak és nem fáradnak el!

\chapter{41}

\par 1 Hallgassatok reám, ti szigetek, és a népek vegyenek új erõt, közelgjenek, majd szóljanak, együtt hadd szálljunk perbe!
\par 2 Ki támasztá fel azt keletrõl, a kit igazságban hív az õ lábához? A népeket kezébe adja és királyok felett uralkodóvá teszi, kardjával mint port szórja szét, mint repülõ polyvát kézíve által!
\par 3 Kergeti õket, békességgel vonul az úton, a melyen lábaival nem járt.
\par 4 Ki tette és vitte végbe ezt? A ki elhívja eleitõl fogva a nemzetségeket: én, az Úr, az elsõ és utolsókkal is az vagyok én!
\par 5 Látták a szigetek és megrémülének, a földnek végei reszkettek, közelegtek és egybegyûltek.
\par 6 Kiki társát segíti, és barátjának ezt mondja: Légy erõs!
\par 7 És bátorítja a mester az ötvöst, és a kalapácscsal simító azt, a ki az ülõt veri; így szól a forrasztásról: jó az, és megerõsíti szegekkel, hogy meg ne mozduljon.
\par 8 De te Izráel, én szolgám, Jákób, a kit én elválasztottam, Ábrahámnak, az én barátomnak magva;
\par 9 Te, a kit én a föld utolsó részérõl hoztalak és véghatárairól elhívtalak, és ezt mondám néked: Szolgám vagy te, elválasztottalak és meg nem útállak:
\par 10 Ne félj, mert én veled vagyok; ne csüggedj, mert én vagyok Istened; megerõsítelek, sõt megsegítlek, és igazságom jobbjával támogatlak.
\par 11 Ímé, megszégyenülnek és meggyaláztatnak, a kik fölgerjednek ellened, semmivé lesznek és elvesznek, a kik veled perlekednek.
\par 12 Keresed õket és meg nem találod a veled versengõket; megsemmisülnek teljesen, a kik téged háborgatnak.
\par 13 Mivel én vagyok Urad, Istened, a ki jobbkezedet fogom, és a ki ezt mondom néked: Ne félj, én megsegítelek!
\par 14 Ne félj, férgecske Jákób, maroknyi Izráel, én megsegítlek, szól az Úr, a te megváltód, Izráelnek Szentje!
\par 15 Ímé, én teszlek éles, új cséplõhengerré, a melynek két éle van; hegyeket csépelj és zúzz össze, és a halmokat pozdorjává tegyed.
\par 16 Te szórd, és a szél vigye el õket, és elszéleszsze õket a forgószél, és te örülsz az Úrban, és dicsekedel Izráelnek Szentjében.
\par 17 A nyomorultak és szegények keresnek vizet, de nincs, nyelvök a szomjúságban elepedt: én, az Úr meghallgatom õket, én, Izráel Istene, nem hagyom el õket.
\par 18 Kopasz hegyeken folyókat nyitok és a rónák közepén forrásokat; a pusztát vizek tavává teszem és az aszú földet vizeknek forrásivá.
\par 19 A pusztában czédrust, akáczot nevelek és mirtust és olajfát, plántálok a kietlenben cziprust, platánt, sudarczédrussal együtt,
\par 20 Hogy lássák, megtudják, eszökbe vegyék és megértsék mindnyájan, hogy az Úrnak keze mívelte ezt, és Izráel Szentje teremtette ezt!
\par 21 Hozzátok ide ügyeteket, szól az Úr, adjátok elõ erõsségeiteket, így szól Jákób királya.
\par 22 Adják elõ és jelentsék meg nékünk, a mik történni fognak; a mik elõször lesznek, jelentsétek meg, hogy eszünkbe vegyük és megtudjuk végöket, vagy a jövendõket tudassátok velünk!
\par 23 Jelentsétek meg, mik lesznek ezután, hogy megtudjuk, hogy ti Istenek vagytok; vagy hát míveljetek jót vagy gonoszt, hogy mérkõzzünk és majd lássuk együtt!
\par 24 Ímé, ti semmibõl valók vagytok, és dolgotok is semmibõl való; útálat az, a ki titeket szeret.
\par 25 Feltámasztám északról, és eljött napkelet felõl; hirdeti nevemet, és tapodja a fejedelmeket, mint az agyagot, és mint a fazekas a sarat tapossa.
\par 26 Ki jelenté meg ezt eleitõl fogva, hogy megtudnók? vagy régen, hogy ezt mondanók: Igaz? De nem jelenté meg, de nem tudatá senki, nem hallá szavatokat senki sem!
\par 27 Sionnak elõször én hirdetém, ímé itt vannak a tanúk, és örömmondót adtam Jeruzsálemnek.
\par 28 És néztem és nem volt senki, ezek közül nem volt tanácsadó, hogy megkérdezzem õket és feleljenek nékem.
\par 29 Ímé, mindnyájan semmik õk, semmiség cselekedetök, szél és hiábavalóság képeik.

\chapter{42}

\par 1 Ímé az én szolgám, a kit gyámolítok, az én választottam, a kit szívem kedvel, lelkemet adtam õ belé, törvényt beszél a népeknek.
\par 2 Nem kiált és nem lármáz, és nem hallatja szavát az utczán.
\par 3 Megrepedt nádat nem tör el, a pislogó gyertya belet nem oltja ki, a törvényt igazán jelenti meg.
\par 4 Nem pislog és meg nem reped, míg a földön törvényt tanít, és a szigetek várnak tanítására.
\par 5 Így szól az Úr Isten, a ki az egeket teremté és kifeszíté, és kiterjeszté termésivel a földet, a ki lelket ád a rajta lakó népnek, és leheletet a rajta járóknak:
\par 6 Én, az Úr, hívtalak el igazságban, és fogom kezedet, és megõrizlek és népnek szövetségévé teszlek, pogányoknak világosságává.
\par 7 Hogy megnyisd a vakoknak szemeit, hogy a foglyot a tömlöczbõl kihozzad, és a fogházból a sötétben ülõket.
\par 8 Én vagyok az Úr, ez a nevem, és dicsõségemet másnak nem adom, sem dicséretemet a bálványoknak.
\par 9 A régiek ímé beteltek, és most újakat hirdetek, mielõtt meglennének, tudatom veletek.
\par 10 Énekeljetek az Úrnak új éneket, és dicséretét a földnek határairól, ti, a tenger hajósai és teljessége, a szigetek és azoknak lakói.
\par 11 Emeljék fel szavokat a puszta és annak városai, a faluk, a melyekben Kédár lakik, ujjongjanak a kõsziklák lakói, a hegyeknek tetejérõl kiáltsanak.
\par 12 Adják az Úrnak a dicsõséget, és dicséretét hirdessék a szigetekben.
\par 13 Az Úr, mint egy hõs kijõ, és mint hadakozó felkölti haragját, kiált, sõt rivalg és ellenségein erõt vesz.
\par 14 Régtõl fogva hallgattam, néma voltam, magamat megtartóztatám: most mint a szülõ nõ nyögök, lihegek és fúvok!
\par 15 Elpusztítok hegyeket és halmokat, és megszáraztom minden fûvöket, szigetekké teszek folyamokat, és tavakat kiszáraztok.
\par 16 A vakokat oly úton vezetem, a melyet nem ismernek, járatom õket oly ösvényeken, a melyeket nem tudnak; elõttök a sötétséget világossággá teszem, és az egyenetlen földet egyenessé; ezeket cselekszem velök, és õket el nem hagyom.
\par 17 Meghátrálnak és mélyen megszégyenülnek, a kik a bálványban bíznak, a kik ezt mondják az öntött képnek: Ti vagytok a mi isteneink!
\par 18 Oh, ti süketek, halljatok, és ti vakok, lássatok!
\par 19 Kicsoda vak, ha nem az én szolgám? és olyan süket, mint az én követem, a kit elbocsátok? Ki olyan vak, mint a békességgel megajándékozott, és olyan vak, mint az Úr szolgája?
\par 20 Sokat láttál, de nem vetted eszedbe; fülei nyitvák, de nem hall.
\par 21 Az Úr igazságáért azt akarta, hogy a törvényt nagygyá teszi és dicsõségessé.
\par 22 De e nép kiraboltatott és eltapodtatott, bilincsbe verve tömlöczben mindnyájan, és fogházakban rejtettek el, prédává lettek és nincs szabadító; ragadománynyá lettek és nincsen, a ki mondaná: add vissza!
\par 23 Ki veszi ezt közületek fülébe? a ki figyelne és hallgatna ezután!
\par 24 Ki adta ragadományul Jákóbot és Izráelt a prédálóknak? Avagy nem az Úr-é, a ki ellen vétkezénk, és nem akartak járni útain és nem hallgattak az Õ törvényére?
\par 25 Ezért ontá ki reá búsulásának haragját és a had erejét; körülte lángolt az, de õ nem értett; és égett benne, de nem tért eszére!

\chapter{43}

\par 1 És most, oh Jákób, így szól az Úr, a te Teremtõd, és a te alkotód, Izráel: Ne félj, mert megváltottalak, neveden hívtalak téged, enyém vagy!
\par 2 Mikor vizen mégy át, én veled vagyok, és ha folyókon, azok el nem borítnak, ha tûzben jársz, nem égsz meg, és a láng meg nem perzsel téged.
\par 3 Mert én vagyok az Úr, a te Istened, Izráelnek Szentje, a te megtartód, adtam váltságodba Égyiptomot, Kúst és Sebát helyetted.
\par 4 Mivel kedves vagy az én szemeimben, becses vagy és én szeretlek: embereket adok helyetted, és népeket a te életedért:
\par 5 Ne félj, mert én veled vagyok, napkeletrõl meghozom magodat, és napnyugotról egybegyûjtelek.
\par 6 Mondom északnak: add meg; és délnek: ne tartsd vissza, hozd meg az én fiaimat messzünnen, és leányimat a földnek végérõl,
\par 7 Mindent, a ki csak az én nevemrõl neveztetik, a kit dicsõségemre teremtettem, a kit alkottam és készítettem!
\par 8 Hozd ki a vak népet, a melynek szemei vannak, és a süketeket, a kiknek füleik vannak!
\par 9 Minden népek gyûljenek egybe, és seregeljenek össze a népségek: ki hirdethet közülök ilyet? Vagy a régieket tudassák velünk, állítsák elõ tanuikat, hogy igazuk legyen, hogy ezek hallván, ezt mondják: Igaz.
\par 10 Ti vagytok az én tanuim, így szól az Úr; és szolgám a kit elválasztottam, hogy megtudjátok és higyjetek nékem és megértsétek, hogy én vagyok az, elõttem Isten nem alkottatott, és utánam nem lesz!
\par 11 Én, én vagyok az Úr, és rajtam kivül nincsen szabadító!
\par 12 Én hirdettem, és megtartottam, és megjelentettem, és nem volt idegen isten köztetek, és ti vagytok az én tanuim, így szól az Úr, hogy én Isten vagyok.
\par 13 Mostantól fogva is én az leszek, és nincs, a ki az én kezembõl kimentsen; cselekszem, és ki változtatja azt meg?
\par 14 Így szól az Úr, a ti megváltótok, Izráel Szentje. Ti értetek küldöttem el Bábelbe, és leszállítom mindnyájokat, mint menekülõket a Káldeusokkal együtt vídámságuk hajóiba.
\par 15 Én az Úr vagyok, szent Istenetek, Izráelnek teremtõje, királyotok.
\par 16 Így szól az Úr, a ki a tengeren utat csinál, és a hatalmas vizeken ösvényt,
\par 17 A ki kihozott szekeret és lovat, sereget és vitézt; együtt hevernek ottan, nem kelnek föl, kialudtak, mint gyertyabél elhamvadának!
\par 18 Ne emlékezzetek a régiekrõl, és az elõbbiekrõl ne gondolkodjatok!
\par 19 Ímé, újat cselekszem; most készül, avagy nem tudjátok még? Igen, a pusztában utat szerzek, és a kietlenben folyóvizeket.
\par 20 Dicsõítni fog engem a mezõ vada, a sakálok és struczok, hogy vizet szereztem a pusztában; a kietlenben folyóvizeket, hogy választott népemnek inni adjak.
\par 21 A nép, a melyet magamnak alkoték, hirdesse dicséretemet!
\par 22 És még sem engem hívtál segítségül Jákób, hanem megfáradtál én bennem Izráel!
\par 23 Nem adtad nékem égõáldozatul bárányaidat, és áldozataiddal nem dicsõítettél engem; nem terheltelek ételáldozattal, és tömjénnel nem fárasztottalak.
\par 24 Nem vettél pénzen nékem jóillatú nádat, és áldozataid kövérével jól nem tartottál, csak bûneiddel terhelél, vétkeiddel fárasztál engemet.
\par 25 Én, én vagyok, a ki eltörlöm álnokságaidat enmagamért, és bûneidrõl nem emlékezem meg!
\par 26 Juttasd eszembe, no pereljünk együtt, beszéld el, hogy igaznak találtassál!
\par 27 Az elsõ atyád vétkezett, és tanítóid elpártoltak tõlem!
\par 28 Ezért én is megfertõztettem a szent fejedelmeket, és veszedelemre adám Jákóbot, és gyalázatra Izráelt!

\chapter{44}

\par 1 És most hallgass Jákób, én szolgám, és Izráel, a kit én elválasztottam.
\par 2 Így szól az Úr, teremtõd és alkotód anyád méhétõl fogva, a ki megsegít: Ne félj, én szolgám Jákób, és te igaz nép, a kit elválasztottam!
\par 3 Mert vizet öntök a szomjúhozóra, és folyóvizeket a szárazra; kiöntöm lelkemet a te magodra, és áldásomat a te csemetéidre.
\par 4 És nevekednek mint fû között, és mint a fûzfák vizek folyásinál.
\par 5 Ez azt mondja: én az Úré vagyok, amaz Jákób nevét emlegeti, és a másik önkezével írja: az Úré vagyok, és hízelkedve Izráel nevét említi.
\par 6 Így szól az Úr, Izráelnek királya és megváltója, a seregeknek Ura: Én vagyok az elsõ, én az utolsó, és rajtam kivül nincsen Isten.
\par 7 És ki hirdetett hozzám hasonlóan? jelentse meg és hozza azt elém, mióta e világ népét teremtém; és jelentsék meg a közeli és távoli jövõt.
\par 8 Ne féljetek és ne rettegjetek! Hát nem mondtam-é meg és nem jelentém elõre? Ti vagytok tanuim! Hát van-é rajtam kivül Isten? Nincs kõszál, nem tudok!
\par 9 A bálványok csinálói mind hiábavalók, és kedvenczeik mit sem használnak, és tanuik nem látnak és nem tudnak, hogy megszégyenüljenek.
\par 10 Ki alkotott istent, és bálványt ki öntött? a mely semmit sem használ!
\par 11 Ímé, minden barátaik megszégyenülnek, és a mesterek magok is emberek; gyûljenek össze mind és álljanak elõ; féljenek, szégyenüljenek meg együtt!
\par 12 A kovács fejszét készít, és munkálkodik a szénnél, és alakítja azt põrölylyel, és munkálja azt erõs karjával, és megéhezik és ereje nincsen, és vizet sem iszik és elfárad.
\par 13 Az ács mérõzsinórt von, és lefesti azt íróvesszõvel, és meggyalulja azt, a czirkalommal alakítja, és csinálja azt férfiú formájára, ember ékességére, hogy házában lakjék.
\par 14 Czédrusfát vág magának, tölgy- és cserfát hoz, és válogat az erdõ fáiban, fenyõt plántál, a melyet az esõ fölnevel.
\par 15 Azokból az ember tüzet gerjeszt, vesz belõlök és melegszik, meggyújtja és kenyeret süt; sõt istent is csinál abból és imádja, bálványt készít és elõtte leborul;
\par 16 Felét tûzben megégeti, felénél húst eszik: pecsenyét süt és megelégszik, és aztán melengeti magát és szól: Bezzeg melegem van, tûznél valék!
\par 17 Maradékából istent készít, bálványát; leborulva imádja azt és könyörög hozzá, és így szól: Szabadíts meg, mert te vagy istenem!
\par 18 Nem tudnak és nem értenek, mert bekenvék szemeik és nem látnak, és szívök nem eszmél.
\par 19 És nem veszi eszébe, nincs ismerete és értelme, hogy mondaná: Felét tûzben megégetém és kenyeret sütöttem annak szenénél, sütöttem húst és megettem; és maradékából útálatosságot csináljak-é, és leboruljak a fa-galy elõtt?
\par 20 Ki hamuban gyönyörködik, megcsalt szíve vezette félre azt, hogy meg ne szabadítsa lelkét és ezt mondja: Hát nem hazugság van-é jobbkezemben?
\par 21 Oh emlékezzél meg Jákób ezekrõl és Izráel, mert az én szolgám vagy te, én alkottalak téged, én szolgám vagy, Izráel! nem feledlek el.
\par 22 Eltöröltem álnokságaidat, mint felleget, és mint felhõt bûneidet; térj én hozzám, mert megváltottalak.
\par 23 Örüljetek egek, mert az Úr végbe vitte, kiáltsatok földnek mélységei, ujjongva énekeljetek hegyek, erdõ és benne minden fa; mert megváltá az Úr Jákóbot, és Izráelben megdicsõíti magát.
\par 24 Így szól az Úr, megváltód és alkotód anyád méhétõl fogva: Én vagyok az Úr, a ki mindent cselekszem, a ki az egeket egyedül kifeszítem, és kiszélesítem a földet magamtól;
\par 25 Ki a hazugok jeleit megrontja, és a varázslókat megbolondítja, a bölcseket megszégyeníti, és tudományukat bolondsággá teszi.
\par 26 A ki szolgája beszédét beteljesíti, és véghez viszi követei tanácsát, a ki így szól Jeruzsálemnek: Lakjanak benne! és Júda városainak: Megépíttessenek! és romjait felállatom!
\par 27 Ki ezt mondja a mélységnek. Száradj ki! és kiapasztom folyóvizeidet!
\par 28 Ki Czírusnak ezt mondja: Pásztorom! ki véghez viszi minden akaratomat, és ezt mondja Jeruzsálemnek: Megépíttessék! és a templomnak: Alapja vettessék!

\chapter{45}

\par 1 Így szól az Úr felkentjéhez, Czírushoz, kinek jobbkezét megfogám, hogy meghódoltassak elõtte népeket, és a királyok derekának övét megoldjam, õ elõtte megnyissam az ajtókat, és a kapuk be ne zároltassanak;
\par 2 Én menéndek elõtted, és az egyenetleneket megegyenesítem, az érczajtókat összetöröm, és leütöm a vaszárakat.
\par 3 Néked adom a sötétségnek kincseit és a rejtekhelyek gazdagságait, hogy megtudjad, hogy én vagyok az Úr, a ki téged neveden hívtalak, Izráel Istene.
\par 4 Az én szolgámért, Jákóbért, és elválasztott Izráelemért neveden hívtalak el, szeretettel szólítálak, noha nem ismerél.
\par 5 Én vagyok az Úr és több nincs, rajtam kivül nincs Isten! felöveztelek téged, bár nem ismerél.
\par 6 Hogy megtudják napkelettõl és napnyugattól fogva, hogy nincsen több rajtam kivül; én vagyok az Úr és több nincsen!
\par 7 Ki a világosságot alkotom és a sötétséget teremtem, ki békességet szerzek és gonoszt teremtek; én vagyok az Úr, a ki mindezt cselekszem!
\par 8 Egek harmatozzatok onnan felül, és a felhõk folyjanak igazsággal, nyiljék meg a föld és viruljon fel a szabadulás, és igazság sarjadjon fel vele együtt; én az Úr teremtettem azt!
\par 9 Jaj annak, a ki alkotójával perbe száll, holott cserép a föld többi cserepeivel! Vajjon mondja-é az agyag alkotójának: Mit csinálsz? és csinálmányod ezt: Nincsenek kezei?
\par 10 Jaj annak, a ki atyjának mondja: Miért nemzesz? és az asszonynak: Miért szülsz?
\par 11 Így szól az Úr, Izráelnek Szentje és Teremtõje: Kérdezzétek meg a jövendõt tõlem, fiaimat és kezeim munkáját csak bízzátok reám!
\par 12 Én alkotám a földet, és az embert rajta én teremtém, én terjesztém ki kezeimmel az egeket, és minden seregöket én állatám elõ.
\par 13 Én támasztottam õt fel igazságban, és minden útait egyengetem, õ építi meg városomat, és foglyaimat elbocsátja, nem pénzért, sem ajándékért, szóla a seregek Ura!
\par 14 Így szól az Úr: Égyiptom gyûjtött kincse és Kús nyeresége és a nagy termetû Szabeusok hozzád mennek és tieid lesznek, téged követnek, békókban járnak, elõtted leborulnak és hozzád könyörögnek: Csak közted van az Isten és nincsen több Isten!
\par 15 Bizony Te elrejtõzködõ Isten vagy, Izráelnek Istene, szabadító!
\par 16 Szégyent vallanak és gyalázatot mind, egyetemben gyalázatban járnak a bálványok faragói;
\par 17 És Izráel megszabadul az Úr által örök szabadulással, nem vallotok szégyent és gyalázatot soha örökké;
\par 18 Mert így szól az Úr, a ki az egeket teremté; Õ az Isten, a ki alkotá a földet és teremté azt és megerõsíté; nem hiába teremté azt, hanem lakásul alkotá: Én vagyok az Úr és több nincsen!
\par 19 Nem titkon szóltam, a sötétség földének helyén; nem mondtam Jákób magvának: hiába keressetek engem! én, az Úr, igazságot szólok, és megjelentem, a mik igazak.
\par 20 Gyûljetek egybe és jõjjetek elõ, közelegjetek mind, a kik a népek közül megszabadultatok; nem tudnak semmit, a kik bálványuk fáját hordják, és könyörögnek oly istenhez, a ki meg nem tart!
\par 21 Jelentsétek meg és hozzátok elõ, sõt egyetemben tanácskozzanak: ki mondta meg ezt régtõl fogva és jelenté meg elõre? Vajjon nem én, az Úr? És nincs több Isten nálam, igaz Isten és megtartó nincs kívülem.
\par 22 Térjetek én hozzám, hogy megtartassatok földnek minden határai, mert én vagyok az Isten, és nincsen több!
\par 23 Magamra esküdtem és igazság jött ki számból, egy szó, mely vissza nem tér: hogy minden térd nékem hajol meg, rám esküszik minden nyelv!
\par 24 Csak az Úrban van, így szólnak felõlem, minden igazság és erõ, Õ hozzá mennek, és megszégyenülnek mindazok, a kik reá haragusznak.
\par 25 Az Úrban igazul meg és dicsekszik Izráelnek egész magva!

\chapter{46}

\par 1 Ledõl Bél, elesik Nebó, oktalan barmokra kerülnek szobraik, és miket ti hordoztatok, felrakatnak, terhéül a megfáradott állatnak;
\par 2 Elesnek, összerogynak együtt, nem menthetik meg a terhet; és õk magok fogságba mennek.
\par 3 Hallgassatok rám, Jákób háza és Izráel házának minden maradéka, a kiket magamra raktam anyátok méhétõl fogva, és hordoztalak születésetek óta;
\par 4 Vénségtekig én vagyok az, és megõszüléstekig én visellek; én teremtettem és én hordozom, én viselem és megszabadítom.
\par 5 Kihez hasonlíttok engem, és kivel tesztek egyenlõvé? És kivel vettek egybe, hogy hasonlók volnánk?
\par 6 Kitöltik az aranyat az erszénybõl, és ezüstöt mérnek a mértékkel, és ötvöst fogadnak, hogy abból istent csináljon; meghajolnak, leborulnak elõtte.
\par 7 Vállukra veszik azt és hordozzák, majd állványára helyezik és veszteg áll, helyérõl meg nem mozdul, ha kiáltasz is hozzá, nem felel, nyomorúságodból nem szabadít meg.
\par 8 Emlékezzetek meg errõl, és legyetek erõsek, vegyétek eszetekbe, pártütõk!
\par 9 Emlékezzetek meg a messze régi dolgokról, hogy én vagyok Isten és nincsen több; Isten vagyok, és nincs hozzám hasonlatos.
\par 10 Ki megjelentem kezdettõl fogva a véget, és elõre azokat, a mik még meg nem történtek, mondván: tanácsom megáll, és véghez viszem minden akaratomat;
\par 11 Ki elhívom napkeletrõl a sast, meszsze földrõl tanácsom férfiát; nem csak szóltam, ki is viszem, elvégezem, meg is cselekszem!
\par 12 Hallgassatok reám, kemény szívûek, a kik távol vagytok az igazságtól.
\par 13 Elhoztam igazságomat, nincs messze, és az én szabadításom nem késik, Sionban lesz szabadításom, és Izráelen dicsõségem.

\chapter{47}

\par 1 Szállj le és ülj a porba, Babilon szûz leánya, ülj a földre királyi szék nélkül, te a Káldeusok leánya, mert nem hívnak többé téged gyöngének és elkényeztetettnek!
\par 2 Vedd a malmot és õrölj lisztet, född fel fátyolodat, emeld föl a hosszú ruhát, född fel czombodat és menj át a folyókon.
\par 3 Födöztessék föl meztelenséged és láttassék meg szemérmed; bosszút állok és embert nem kímélek!
\par 4 Igy szól a mi Megváltónk, seregeknek Ura az Õ neve, Izráel Szentje!
\par 5 Ülj némán és menj a sötétre, te a Káldeusok leánya, mert nem hívnak többé téged országok úrnõjének!
\par 6 Fölgerjedtem volt népem ellen; megfertõztettem örökségemet és kezedbe adtam azt: te nem cselekedtél velök irgalmasságot, az öregre nehéz igát vetettél!
\par 7 És ezt mondád: Örökre úrnõ leszek! úgy hogy ezekre nem is gondolál, és nem emlékeztél meg annak végérõl.
\par 8 És most halld meg ezt, bujálkodó, a ki bátorságban ülsz, a ki ezt mondja szívében: Én vagyok és nincs senki több, nem ülök özvegységben, és a gyermektelenséget nem ismerem!
\par 9 És mind e kettõ eljõ reád nagy hamar egy napon: gyermektelenség és özvegység, teljességökben jõnek el reád, noha gazdag vagy a varázslásban, és sok nagyon igézõ szózatod.
\par 10 Gonoszságodban bíztál, és ezt mondád: Nem lát senki engem! Bölcseséged és tudományod csalt meg téged, és ezt mondád szívedben: Én vagyok és nincs senki több.
\par 11 Azért jõ te reád a gonosz, a melynek keletkezését nem tudod, és romlás sújt le rád, a melyet meg nem engesztelhetsz, és hirtelen jõ pusztulás reád, nem is tudod!
\par 12 No állj elõ hát igézõ szózatiddal és varázslásodnak sokaságával, a melyekkel ifjúságodtól fogva veszõdtél, talán segíthetsz valamit, talán visszariaszthatod a veszedelmet.
\par 13 Tanácsaid sokaságában megfáradtál; no álljanak elõ és tartsanak meg az égnek vizsgálói, a kik a csillagokat nézik, a kik megjelentik az újholdak napján, hogy mi jövend reád.
\par 14 Ímé, olyanok lettek, mint a polyva, tûz emészté meg õket, nem mentik meg életöket a lángból, még szén sem marad belõlök melegülésre, sem körülülhetõ parázs!
\par 15 Így járnak azok, a kikkel veszõdtél; és a kik kereskedõ társaid voltak ifjúságodtól fogva, futnak, kiki a maga útján; senki nem segít néked!

\chapter{48}

\par 1 Halljátok ezt Jákób háza, a kik Izráel nevérõl neveztettek, és Júda forrásából származának, a kik az Úr nevére esküsznek, és Izráel Istenét emlegetik; de nem híven és nem igazán.
\par 2 Mert a szent várostól nevezik magokat, és Izráel Istenéhez támaszkodnak, a kinek neve seregeknek Ura!
\par 3 Mik eddig történtek, elõre megjelentém, szám hirdeté és tudatá azokat, gyorsan véghez vivém, és bekövetkezének,
\par 4 Mert tudtam, hogy te kemény vagy, és vasinakból van nyakad és homlokod ércz:
\par 5 Tehát elõre megjelentém néked, mielõtt bekövetkezett, tudtodra adtam, hogy ezt ne mondd: Faragott képem mívelé ezeket, bálványom és öntött képem parancsolá ezeket.
\par 6 Hallottad volt, és most lásd mindezt; avagy ti nem tesztek-é errõl bizonyságot? Mostantól fogva újakat tudatok veled: titkoltakat, a melyeket nem tudtál;
\par 7 Mostan rendeltettek el és nem régen, és ezelõtt nem hallottál felõlök, hogy ezt ne mondd: Ímé, tudtam én azokat.
\par 8 Nem is hallottad, nem is tudtad, füled sem vala nyitva régen, mivel tudtam, hogy nagyon hûtelen vagy, és pártütõnek hívatál anyád méhétõl fogva.
\par 9 Nevemért elhalasztom haragomat, és dicséretemért fékezem magamat veled szemben, hogy ki ne vágjalak.
\par 10 Ímé, megtisztítottalak, de nem úgy, mint ezüstöt, megpróbáltalak a nyomor kemenczéjében.
\par 11 Enmagamért, enmagamért cselekszem; mert hogyan szentségteleníttetnék meg nevem?! És dicsõségemet másnak nem adom.
\par 12 Hallgass rám Jákób és Izráel, én elhívottam, én vagyok az elsõ és én az utolsó.
\par 13 Hiszen kezem veté e föld alapját, és jobbom terjeszté ki az egeket, ha én szólítom õket, mind itt állnak.
\par 14 Gyûljetek egybe mind, és halljátok meg: ki jelenté meg közülök ezeket? az, kit az Úr szeret, elvégzi akaratját Bábelen, és karja lészen a Káldeusokon.
\par 15 Én, én szóltam, és õt el is hívtam, elhoztam õt, és szerencsés lesz útja.
\par 16 Közelgjetek hozzám, halljátok ezt, nem szóltam eleitõl fogva titkon! mióta ez történik, ott vagyok! És most az Úr Isten engem küldött, és az õ lelkét.
\par 17 Így szól az Úr, Megváltód, Izráelnek Szentje: Én vagyok az Úr, Istened, ki tanítlak hasznosra, és vezetlek oly úton, a melyen járnod kell.
\par 18 Vajha figyelmeztél volna parancsolataimra! olyan volna békességed, mint a folyóvíz, és igazságod, mint a tenger habjai;
\par 19 És olyan volna magod, mint a tengernek fövénye, és méhednek gyümölcsei, mint annak kövecskéi; nem vágatnék ki és nem pusztulna el neve orczám elõl.
\par 20 Menjetek ki Bábelbõl, fussatok el Káldeából ujjongásnak szavával! jelentsétek meg, tudassátok ezt, terjesszétek a föld végsõ határáig; mondjátok: Megváltotta  az Úr szolgáját, Jákóbot.
\par 21 Nem szomjaznak, bárha pusztaságon vezeti is õket: kõsziklából vizet fakaszt nékik, és meghasítja a sziklát és víz ömöl belõle.
\par 22 Nincs békesség, így szól az Úr, az istenteleneknek!

\chapter{49}

\par 1 Hallgassatok reám, ti szigetek, és figyeljetek távol való népek: anyám méhétõl hívott el az Úr, anyámnak szíve alatt már emlékezett nevemrõl.
\par 2 Hasonlóvá tevé számat az éles kardhoz, keze árnyékában rejtett el engem, és fényes nyíllá tett engemet, és tegzébe zárt be engem.
\par 3 És mondá nékem: Szolgám vagy te, Izráel, a kiben és megdicsõülök.
\par 4 És én mondám: Hiába fáradoztam, semmire és haszontalan költöttem el erõmet; de az Úrnál van ítéletem, és jutalmam Istenemnél.
\par 5 És most így szól az Úr, a ki engem anyám méhétõl szolgájává alkotott, hogy Jákóbot Õ hozzá megtérítsem és hogy Izráel hozzá gyûjtessék; hiszen tisztelt vagyok az Úr szemeiben és erõsségem az én Istenem!
\par 6 Így szól: Kevés az, hoggy nékem szolgám légy, a Jákób nemzetséginek megépítésére és Izráel megszabadultjainak visszahozására: sõt a népeknek is világosságul adtalak, hogy üdvöm a föld végéig terjedjen!
\par 7 Így szól az Úr, Izráel megváltója, Szentje, a megvetett lelkûhöz, a nép undorához, a zsarnokok szolgájához: Látjátok királyok! és majd fölkelnek, fejedelmek, és leborulnak az Úrért, a ki hû, Izráel Szentjéért, a ki téged elválasztott.
\par 8 Így szól az Úr: Jókedvem idején én meghallgattalak, és a szabadulás napján segítettelek; megtartlak és nép szövetségévé teszlek, hogy megépítsd a földet, és kioszd az elpusztult örökségeket;
\par 9 Így szólván a foglyoknak: Jõjjetek ki! és azoknak, a kik sötétben ülnek: Lépjetek elõ! Az utakon legelnek, és minden halmokon legelõjük lesz:
\par 10 Nem éheznek, nem szomjúhoznak, nem bántja õket délibáb és a nap; mert a ki rajtok könyörült, vezeti õket, és õket vizek forrásaihoz viszi.
\par 11 És teszem minden hegyemet úttá, és ösvényeim magasak lesznek.
\par 12 Ímé, ezek messzirõl jönnek ímé, amazok észak és a tenger felõl, és amazok Sinnek földérõl!
\par 13 Ujjongjatok egek, és föld örvendezz, ujjongva énekeljetek hegyek; mert megvígasztalá népét az Úr, és könyörül szegényein!
\par 14 És szól Sion: Elhagyott az Úr engem, és rólam elfeledkezett az Úr!
\par 15 Hát elfeledkezhetik-é az anya gyermekérõl, hogy ne könyörüljön méhe fián? És ha elfeledkeznének is ezek: én te rólad el nem feledkezem.
\par 16 Ímé, az én markaimba metszettelek fel téged, kõfalaid elõttem vannak szüntelen.
\par 17 Elõsietnek fiaid, rombolóid és pusztítóid eltávoznak belõled.
\par 18 Emeld fel köröskörül szemeidet, és lássad, mindnyájan egybegyûlnek, hozzád jönnek. Élek én, így szól az Úr, hogy fölrakod mindnyájokat, mint ékszert, és felkötöd, mint menyasszony.
\par 19 Mert romjaid és pusztaságaid és elpusztult földed, mindez szûk lesz most a lakosságnak, és messze távoznak elpusztítóid.
\par 20 Gyermektelenségednek fiai még ezt mondják majd füled hallatára: Szoros e hely nékem, menj el, hogy itt lakhassam!
\par 21 És te így szólsz szívedben: Ki szûlte nékem ezeket? hisz én gyermektelen és terméketlen voltam, fogoly és számkivetett; és ezeket ki nevelte föl? Ímé, én egyedül maradtam meg; ezek hol voltak?
\par 22 Így szól az Úr Isten: Ímé, fölemelem kezemet a népekhez, és elõttök zászlómat felállatom, és elhozzák fiaidat ölükben, és leányaid vállukon hordoztatnak.
\par 23 És királyok lesznek dajkálóid, fejedelmi asszonyaik dajkáid, arczczal a földre borulnak elõtted és lábaid porát nyalják: és megtudod, hogy én vagyok az Úr, kit a kik várnak, meg nem szégyenülnek.
\par 24 Elvétethetik-é a préda az erõstõl, és megszabadulhatnak-é az igazak foglyai?
\par 25 Igen, így szól az Úr, az erõstõl elvétetnek a foglyok is, és megszabadul a kegyetlen zsákmánya, és háborgatóidat és háborítom meg, és én tartom meg fiaidat.
\par 26 És etetem nyomorgatóiddal az önnön húsokat, és mint a musttól, véröktõl megrészegednek, és megtudja minden test, hogy én vagyok  az Úr; megtartód és megváltód, Jákóbnak erõs Istene!

\chapter{50}

\par 1 Így szól az Úr: Hol van anyátok elválólevele, a melylyel õt elbocsátám? vagy hol van egy kölcsönadóim közül, a kinek titeket eladtalak? Ímé, a ti vétkeitekért adattatok el, és bûneitekért bocsáttatott el anyátok!
\par 2 Miért jöttem, holott nem volt ott senki? hívtam és nem felelt senki sem! vagy olyan rövid már a kezem, hogy meg nem válthat? vagy nincs bennem megszabadításra való erõ? Ímé, én dorgálásommal kiszáraztom a tengert, a folyókat pusztává teszem, megbûzhödnek halaik, hogy nincsen víz, és szomjúságtól meghalnak.
\par 3 Felöltöztetem az egeket sötétségbe, és gyászruhával födöm be azokat.
\par 4 Az Úr Isten bölcs nyelvet adott én nékem, hogy tudja erõsítni a megfáradtat beszéddel, fölserkenti minden reggel, fölserkenti fülemet, hogy hallgassak, miként a tanítványok.
\par 5 Az Úr Isten megnyitotta fülemet, és én nem voltam engedetlen, hátra nem fordultam.
\par 6 Hátamat odaadám a verõknek, és orczámat a szaggatóknak, képemet nem födöztem be a gyalázás és köpdösés elõtt.
\par 7 És az Úr Isten megsegít engemet, azért nem szégyenülök meg, ezért olyanná tettem képemet, mint a kova, és tudtam, hogy szégyent nem vallok.
\par 8 Közel van, a ki engem megigazít, ki perel én velem? Álljunk együtt elõ! Kicsoda peresem? közelegjen hozzám!
\par 9 Ímé, az Úr Isten megsegít engem, kicsoda kárhoztatna engem? Ímé, mindnyájan, mint a ruha megavulnak, moly emészti meg õket!
\par 10 Ki féli közületek az Urat? és ki hallgat az õ szolgája szavára? õ, a ki sötétségben jár és nincs fényesség néki, bízzék az Úr nevében, és támaszkodjék Istenhez!
\par 11 Ímé, ti mind, a kik tüzet gyujtotok, felövezvén magatokat tüzes nyilakkal, vettessetek tüzeteknek lángjába és a tüzes nyilakba, a melyeket meggyújtottatok! Kezembõl jõ ez rátok; fájdalomban fogtok feküdni!

\chapter{51}

\par 1 Hallgassatok reám, kik az igazságot követitek, kik az Urat keresitek; tekintsetek a kõszálra, a melybõl kivágattattatok, és a kútfõ nyílására, a melybõl kiásattatok!
\par 2 Tekintsetek Ábrahámra, atyátokra, és Sárára, a ki titeket szûlt, hogy egymagát hívtam el õt, és megáldám és megszaporítám õt.
\par 3 Mert megvígasztalja az Úr Siont, megvígasztalja minden romjait, és pusztáját olyanná teszi, mint az Éden, és kietlenjét olyanná, mint az Úrnak kertje, öröm és vígasság találtatik abban, hálaadás és dicséret szava!
\par 4 Figyeljetek reám, én népem, és reám hallgassatok, én nemzetem! mert tanítás megy ki tõlem, és törvényemet a népek megvilágosítására megalapítom.
\par 5 Közel igazságom, kijõ szabadításom, és karjaim népeket ítélnek: engem várnak a szigetek, és karomba vetik reménységüket.
\par 6 Emeljétek az égre szemeiteket, és nézzetek a földre ide alá, mert az egek mint a füst elfogynak, és a föld, mint a ruha megavul, és lakosai hasonlókép elvesznek; de szabadításom örökre megmarad, és igazságom meg nem romol.
\par 7 Hallgassatok rám, kik tudjátok az igazságot, te nép, kinek szívében van törvényem! ne féljetek az emberek gyalázatától, és szidalmaik miatt kétségbe ne essetek!
\par 8 Mert mint a ruhát, moly emészti meg õket, és mint a gyapjat, féreg eszi meg õket, és igazságom örökre megmarad, és szabadításom nemzetségrõl nemzetségre!
\par 9 Kelj föl, kelj föl, öltözd fel az erõt, oh Úrnak karja! kelj föl, mint a régi idõben, a messze hajdanban! Avagy nem te vagy-é, a ki Ráhábot kivágta, és a sárkányt átdöfte?
\par 10 Nem te vagy-é, a ki a tengert megszáraztotta, a nagy mélység vizeit; a ki a tenger fenekét úttá változtatta, hogy átmenjenek a megváltottak?!
\par 11 Így térnek meg az Úrnak megváltottai, és ujjongás között Sionba jõnek, és örökös öröm fejökön; vígasságot és örömöt találnak, eltünik a fájdalom és sóhaj!
\par 12 Én, én vagyok megvigasztalótok! Ki vagy te, hogy félsz halandó embertõl? ember fiától, a ki olyan lesz, mint a fû?!
\par 13 Hogy elfeledkeztél az Úrról, Teremtõdrõl, a ki az eget kiterjeszté és a földet megalapítá, és hogy félsz szüntelen minden napon nyomorgatódnak haragjától, a ki igyekszik elveszteni? De hol van a nyomorgató haragja?
\par 14 Hirtelen megszabadul a fogoly, és nem hal meg a veremben, kenyere sem fogy el:
\par 15 Hiszen én vagyok az Úr, a te Istened, a ki megreszkettetem a tengert és zúgnak habjai; seregeknek Ura az Én nevem?
\par 16 És adtam beszédemet a te szádba, és kezem árnyékával födöztelek be, hogy újonnan plántáljam az egeket, és megalapítsam a földet, és ezt mondjam Sionnak: Én népem vagy te!
\par 17 Serkenj föl, serkenj föl, kelj föl Jeruzsálem, ki megittad az Úr kezébõl haragja poharát; a tántorgás öblös kelyhét megittad, kiürítéd!
\par 18 Nem vala vezetõje minden fiai közül, a kiket szûlt, nem fogta senki õt kézen minden fiai közül, a kiket fölnevelt.
\par 19 E kettõ esett rajtad! Kicsoda szánt meg téged? A pusztulás, a romlás, az éhség és a fegyver; miként vígasztaljalak téged?
\par 20 Fiaid elájultan ott feküdtek minden utczáknak fejeinél, mint a hálóba esett zerge, megrészegedve az Úr haragjától, Istenednek feddésétõl.
\par 21 Ezért halld meg ezt, szenvedõ, ki részeg vagy, de nem bortól!
\par 22 Így szól Urad, az Úr, és Istened, a ki népéért bosszút áll: Ímé, kiveszem kezedbõl a tántorgás poharát, haragom öblös kelyhét, nem iszod többé azt meg!
\par 23 És adom azt nyomorgatóid kezébe, a kik azt mondották lelkednek: Hajolj meg, hogy átmenjünk te rajtad, és a te hátadat olyanná tetted, mint a föld, és mint a minõ az utcza a járóknak!

\chapter{52}

\par 1 Serkenj föl, serkenj föl, öltözd fel erõsségedet, Sion, öltözzél fel ékességed ruháiba, Jeruzsálem, szent város, mert nem lép te beléd többé körülmetéletlen, tisztátalan!
\par 2 Rázd ki magad a porból, kelj fel, ülj fel Jeruzsálem, oldd ki magadat nyakad bilincseibõl, Sion fogoly leánya!
\par 3 Mert így szól az Úr Isten: Ingyen adattatok el, és nem pénzen váltattok meg!
\par 4 Mert így szól az Úr Isten: Égyiptomba ment alá népem elõször, hogy ott bujdossék, és azután Assiria nyomorgatá õt ok nélkül.
\par 5 És most mit tegyek itt? szól az Úr, hiszen népem elvitetett ok nélkül, és a rajta uralkodók ujjonganak, szól az Úr, és nevem szüntelen mindennap gúnyoltatik.
\par 6 Ezért hadd ismerje meg népem az én nevemet, ezért ama napon! Hogy én vagyok, a ki mondom: Ímé, itt vagyok!
\par 7 Mily szépek a hegyeken az örömmondónak lábai, a ki békességet hirdet, jót mond, szabadulást hirdet, a ki ezt mondja Sionnak: Uralkodik a te Istened!
\par 8 Halld õrállóidat! felemelik szavokat, ujjonganak egyetemben, mert szemtõl-szembe látják, hogy mint hozza vissza Siont az Úr!
\par 9 Ujjongva énekeljetek mindnyájan, Jeruzsálem romjai, mert megvígasztalá az Úr népét, megváltá Jeruzsálemet.
\par 10 Feltûrte az Úr szent karját minden népeknek szemei elõtt, hogy lássák a föld minden határai Istenünk szabadítását!
\par 11 Távozzatok, távozzatok, jertek ki, onnan, tisztátalant ne illessetek, jertek kik közülök, tisztítsátok meg magatokat, a kik az Úr edényeit hordozzátok.
\par 12 Mert ne sietséggel jertek ki, és ne futással menjetek; mert elõttetek megy az Úr, és követni fog Izráel Istene!
\par 13 Ímé, jó szerencsés lesz szolgám, magasságos, felséges és dicsõ lesz nagyon.
\par 14 Miképen eliszonyodtak tõled sokan, oly rút, nem emberi volt ábrázatja, és alakja sem ember fiaié volt:
\par 15 Akképen ejt ámulatba sok népeket; fölötte a királyok befogják szájokat, mert a mit nékik nem beszéltek volt, azt látják, és mit nem hallottak volt, arra figyelnek.

\chapter{53}

\par 1 Ki hitt a mi tanításunknak, és az Úr karja kinek jelentetett meg?
\par 2 Felnõtt, mint egy vesszõszál Õ elõtte, és mint gyökér a száraz földbõl, nem volt néki alakja és ékessége, és néztünk reá, de nem vala ábrázata kivánatos!
\par 3 Útált és az emberektõl elhagyott volt, fájdalmak férfia és betegség ismerõje! mint a ki elõl orczánkat elrejtjük, útált volt; és nem gondoltunk vele.
\par 4 Pedig betegséginket õ viselte, és fájdalmainkat hordozá, és mi azt hittük, hogy ostoroztatik, verettetik és kínoztatik Istentõl!
\par 5 És õ megsebesíttetett bûneinkért, megrontatott a mi vétkeinkért, békességünknek büntetése rajta van, és az õ sebeivel gyógyulánk meg.
\par 6 Mindnyájan, mint juhok eltévelyedtünk, kiki az õ útára tértünk; de az Úr mindnyájunk vétkét õ reá veté.
\par 7 Kínoztatott, pedig alázatos volt, és száját nem nyitotta meg, mint bárány, mely mészárszékre vitetik, és mint juh, mely megnémul az õt nyírõk elõtt; és száját nem nyitotta meg!
\par 8 A fogságból és ítéletbõl ragadtatott el, és kortársainál ki gondolt arra, hogy kivágatott az élõk földébõl, hogy népem bûnéért lõn rajta vereség?!
\par 9 És a gonoszok közt adtak sírt néki, és a gazdagok mellé jutott kínos halál után: pedig nem cselekedett hamisságot, és álnokság sem találtatott szájában.
\par 10 És az Úr akarta õt megrontani betegség által; hogyha önlelkét áldozatul adja, magot lát, és napjait meghosszabbítja, és az Úr akarata az õ keze által jó szerencsés lesz.
\par 11 Mert lelke szenvedése folytán látni fog, és megelégszik, ismeretével igaz szolgám sokakat megigazít, és vétkeiket õ viseli.
\par 12 Azért részt osztok néki a nagyokkal, és zsákmányt a hatalmasokkal oszt, mivelhogy életét halálra adta, és a bûnösök közé számláltatott; pedig õ sokak bûnét hordozá, és a bûnösökért imádkozott!

\chapter{54}

\par 1 Ujjongj te meddõ, ki nem szûltél, ujjongva énekelj és kiálts, ki nem vajudtál, mert többek az elhagyottnak fiai a férjnél való fiainál, azt mondja az Úr.
\par 2 Szélesítsd ki a sátorod helyét, és hajlékidnak kárpitjait terjeszszék ki; ne tiltsd meg; nyújtsd meg köteleidet, és erõsítsd meg szegeidet.
\par 3 Mert mind jobb-, mind balkézre kiterjedsz, és magod népeket vesz örökségbe, és puszta városokat megnépesítnek.
\par 4 Ne félj, mert meg nem szégyenülsz, és ne pirulj, mert meg nem gyaláztatol, mert ifjúságod szégyenérõl elfeledkezel, és özvegységednek gyalázatáról többé meg nem emlékezel.
\par 5 Mert férjed a te Teremtõd, seregeknek Ura az Õ neve, és megváltód Izráelnek Szentje, az egész föld Istenének hívattatik.
\par 6 Mert mint elhagyott és fájó lelkû asszonyt hív téged az Úr, és mint megvetett ifjú asszonyt; ezt mondja Istened:
\par 7 Egy rövid szempillantásig elhagytalak, és nagy irgalmassággal egybegyûjtlek;
\par 8 Búsulásom felbuzdultában elrejtém orczámat egy pillantásig elõled, és örök irgalmassággal könyörülök rajtad; ezt mondja megváltó Urad.
\par 9 Mert úgy lesz ez nékem, mint a Noé özönvize; miként megesküvém, hogy nem megy át többé Noé özönvize e földön, úgy esküszöm meg, hogy rád többé nem haragszom, és téged meg nem feddelek.
\par 10 Mert a hegyek eltávoznak, és a halmok megrendülnek; de az én irgalmasságom tõled el nem távozik, és békességem szövetsége meg nem rendül, így szól könyörülõ Urad.
\par 11 Oh te szegény, szélvésztõl hányt, vígasztalás nélkül való! Ímé, ólomporba rakom köveidet, és zafirokra alapítalak.
\par 12 Rubinból csinálom falad párkányzatát, és kapuidat gránátkövekbõl, és egész határodat drágakövekbõl;
\par 13 És minden fiaid az Úr tanítványai lesznek, és nagy lesz fiaid békessége.
\par 14 Igazság által leszel erõs, ne gondolj a nyomorral, mert nincsen mit félned, és a rettegéssel, mert nem közelg hozzád.
\par 15 És ha összegyûlvén összegyûlnek, ez nem én tõlem lesz! A ki ellened összegyûl, elesik általad.
\par 16 Ímé, én teremtettem a kovácsot, a ki a parazsat éleszti és fegyvert készít mestersége szerint: és én teremtém a pusztítót is a vesztésre!
\par 17 Egy ellened készült fegyver sem lesz jó szerencsés, és minden nyelvet, mely ellened perbe száll, kárhoztatsz: ez az Úr szolgáinak öröksége, és az õ igazságuk, mely tõlem van, így szól az Úr.

\chapter{55}

\par 1 Oh mindnyájan, kik szomjúhoztok, jertek e vizekre, ti is, kiknek nincs pénzetek, jertek, vegyetek és egyetek, jertek, vegyetek pénz nélkül és ingyen, bort és tejet.
\par 2 Miért adtok pénzt azért, a mi nem kenyér, és gyûjtött kincseteket azért, a mi meg nem elégíthet? Hallgassatok, hallgassatok reám, hogy jót egyetek, és gyönyörködjék lelketek kövérségben.
\par 3 Hajtsátok ide füleiteket és jertek hozzám; hallgassatok, hogy éljen lelketek, és szerzek veletek örök szövetséget, Dávid iránt való változhatatlan kegyelmességem szerint.
\par 4 Ímé, bizonyságul adtam õt a népeknek, fejedelmül és parancsolóul népeknek.
\par 5 Ímé, nem ismert népet hívsz elõ, és a nép, a mely téged nem ismert, hozzád siet az Úrért, Istenedért és Izráel Szentjéért, hogy téged megdicsõített.
\par 6 Keressétek az Urat, a míg megtalálható, hívjátok õt segítségül, a míg közel van.
\par 7 Hagyja el a gonosz az õ útát, és a bûnös férfiú gondolatait, és térjen az Úrhoz, és könyörül rajta, és a mi Istenünkhöz, mert bõvelkedik a megbocsátásban.
\par 8 Mert nem az én gondolataim a ti gondolataitok, és nem a ti útaitok az én útaim, így szól az Úr!
\par 9 Mert a mint magasabbak az egek a földnél, akképen magasabbak az én útaim útaitoknál, és gondolataim gondolataitoknál!
\par 10 Mert mint leszáll az esõ és a hó az égbõl, és oda vissza nem tér, hanem megöntözi a földet, és termõvé, gyümölcsözõvé teszi azt, és magot ád a magvetõnek és kenyeret az éhezõnek:
\par 11 Így lesz az én beszédem, a mely számból kimegy, nem tér hozzám üresen, hanem megcselekszi, a mit akarok, és szerencsés lesz ott, a hová küldöttem.
\par 12 Mert örömmel jöttök ki, és békességben vezéreltettek; a hegyek és halmok ujjongva énekelnek ti elõttetek, és a mezõ minden fái tapsolnak.
\par 13 A tövis helyén cziprus nevekedik, és bogács helyett mirtus nevekedik, és lesz ez az Úrnak dicsõségül és örök jegyül, a mely el nem töröltetik.

\chapter{56}

\par 1 Így szól az Úr: Õrizzétek meg a jogosságot, és cselekedjetek igazságot, mert közel van szabadításom, hogy eljõjjön, és igazságom, hogy megjelenjék:
\par 2 Bolond ember, a ki ezt cselekszi, és az ember fia, a ki ahhoz ragaszkodik! a ki megõrzi a szombatot, hogy meg ne fertõztesse azt, és megõrzi kezét, hogy semmi gonoszt ne tegyen.
\par 3 És ne mondja ezt az idegen, a ki az Úrhoz adá magát: Bizony elszakaszt az Úr engem az Õ népétõl! ne mondja a herélt sem: Ímé, én megszáradt fa vagyok!
\par 4 Mert így szól az Úr a herélteknek: A kik megõrzik szombatimat és szeretik azt, a miben gyönyörködöm, és ragaszkodnak az én szövetségemhez:
\par 5 Adok nékik házamban és falaimon belül helyet, és oly nevet, a mely jobb, mint a fiakban és lányokban élõ név; örök nevet adok nékik, a mely soha el nem vész;
\par 6 És az idegeneket, a kik az Úrhoz adák magukat, hogy néki szolgáljanak és hogy szeressék az Úr nevét, hogy Õ néki szolgái legyenek; mindenkit, a ki megõrzi a szombatot, hogy meg ne fertõztesse azt, és a szövetségemhez ragaszkodókat:
\par 7 Szent hegyemre viszem föl ezeket, és megvídámítom õket imádságom házában; egészen égõ és véres áldozataik kedvesek lesznek oltáromon; mert házam imádság házának  hivatik minden népek számára!
\par 8 Így szól az Úr Isten, a ki összegyûjti Izráel elszéledt fiait: Még gyûjtök õ hozzá, az õ egybegyûjtötteihez!
\par 9 Mezõnek minden vadai! jertek el enni, erdõnek minden vadai!
\par 10 Õrállói vakok mindnyájan, mitsem tudnak, mindnyájan néma ebek, nem tudnak ugatni; álmodók, heverõk, szunnyadni szeretõk!
\par 11 És ez ebek telhetetlenek, nem tudnak megelégedni, pásztorok õk, a kik nem tudnak vigyázni; mindnyájan a magok útára tértek, kiki nyeresége után, mind együtt!
\par 12 Jertek, hadd hozzak bort, és igyunk részegítõ italt! és legyen a holnap olyan, mint a ma, nagy és dicsõ felettébb!

\chapter{57}

\par 1 Az igaz elvész és nem veszi eszébe senki, és az irgalmasságtevõk elragadtatnak és senki nem gondolja fel, hogy a veszedelem elõl ragadtatik el az igaz;
\par 2 Bemegy békességbe, nyugosznak ágyaikon, a kik egyenes útaikon járának.
\par 3 És ti közelgjetek ide, szemfényvesztõ fiai, paráznának magva, a ki paráználkodol.
\par 4 Ki felett örvendeztek? Ki ellen tátjátok fel szátokat és öltitek ki nyelveteket? nem ti vagytok-é a bûn gyermekei, a hazugságnak magva?
\par 5 A kik lángoltok a bálványokért minden zöld fa alatt,  megöltök gyermekeket a völgyekben, a hegyek hasadékai alatt.
\par 6 A folyónak sima köveiben van örökséged; azok, azok a te részed, töltöttél nékik italáldozatot is, vivél ételáldozatot és én jó néven vegyem-é ezeket?
\par 7 Magas és felemelkedett hegyen helyezted ágyadat, fel is menél oda áldozni áldozatot.
\par 8 Az ajtó és ajtófél mögé tetted bálványjeleidet, és tõlem eltávozván, fölfedted ágyadat, fölmentél rá, és megszélesítéd, és szövetséget szerzél velök, szeretted ágyukat, a merre csak láttad.
\par 9 És menél a királyhoz olajjal, és megsokasítád keneteidet, és elküldéd követeidet messze földre, és megaláztad magadat a sírig.
\par 10 Nagy útadon megfáradál, és még sem mondád: mind hasztalan! erõd megújulását érezéd, így nem levél beteg!
\par 11 Kitõl féltél és rettegtél, hogy hazudtál és rólam meg nem emlékezél, szívedre sem vevéd? vagy azért nem félsz engem, hogy hallgatok már régtõl fogva?
\par 12 Én jelentem meg igazságodat, és csinálmányaid nem használnak néked.
\par 13 Ha kiáltasz: szabadítson meg téged bálványid raja; mindnyájokat szél viszi el, lehelet kapja fel, és a ki bennem bízik, örökségül bírja a földet, és örökli szent hegyemet.
\par 14 És szól egy szó: Töltsétek, töltsétek, készítsétek az útat, vegyetek el minden botránkozást népem útáról.
\par 15 Mert így szól a magasságos és felséges, a ki örökké lakozik, és a kinek neve szent: Magasságban és szentségben lakom, de a megrontottal és alázatos szívûvel is, hogy megelevenítsem az alázatosok lelkét, és megelevenítsem a megtörtek szívét.
\par 16 Mert nem örökké perlek, és nem mindenha haragszom, mert a lélek elõttem megepedne, és a leheletek, a kiket én teremtettem.
\par 17 Mert a telhetetlenségnek vétkéért haragudtam meg, és megvertem õt, elrejtém magamat és megharagudtam; és õ elfordulva, szíve útjában járt.
\par 18 Útait láttam, és meggyógyítom õt; vezetem õt, és vígasztalást nyujtok néki és gyászolóinak,
\par 19 Megteremtem ajkaikon a hálának gyümölcsét. Békesség, békesség a messze és közel valóknak, így szól az Úr; én meggyógyítom õt!
\par 20 És a hitetlenek olyanok, mint egy háborgó tenger, a mely nem nyughatik, és a melynek vize iszapot és sárt hány ki.
\par 21 Nincs békesség, szól Istenem, a hitetleneknek!

\chapter{58}

\par 1 Kiálts teljes torokkal, ne kiméld; mint trombita emeld fel hangodat, és hirdesd népemnek bûneiket, és Jákób házának vétkeit.
\par 2 Holott õk engem mindennap keresnek, és tudni kivánják útaimat, mint oly nép, a mely igazságot cselekedett és Istene törvényét el nem hagyta; kérik tõlem az igazságnak ítéleteit, és Istennek elközelgését kivánják.
\par 3 Mért bõjtölünk és Te nem nézed, gyötörjük lelkünket és Te nem tudod? Ímé, bõjtöléstek napján kedvtelésteket ûzitek, és minden robotosaitokat szorongatjátok.
\par 4 Ímé perrel és versengéssel bõjtöltök, és sújtotok a gazságnak öklével; nem úgy bõjtöltök mostan, hogy meghallassék szavatok a magasságban.
\par 5 Hát ilyen a bõjt, a melyet én kedvelek, és olyan a nap, a melyen az ember lelkét gyötri? Avagy ha mint káka lehajtja fejét, és zsákot és hamvat terít maga alá: ezt nevezed-é bõjtnek és az Úr elõtt kedves napnak?
\par 6 Hát nem ez-é a bõjt, a mit én kedvelek: hogy megnyisd a gonoszságnak bilincseit, az igának köteleit megoldjad, és szabadon bocsásd az elnyomottakat, és hogy minden igát széttépjetek?
\par 7 Nem az-é, hogy az éhezõnek megszegd kenyeredet, és a szegény bujdosókat házadba bevigyed, ha meztelent látsz, felruházzad, és tested elõtt el ne rejtsd magadat?
\par 8 Akkor felhasad, mint hajnal a te világosságod, és meggyógyulásod gyorsan kivirágzik, és igazságod elõtted jár; az Úr dicsõsége követ.
\par 9 Akkor kiáltasz, és az Úr meghallgat, jajgatsz, és õ azt mondja: Ímé, itt vagyok. Ha elvetended közüled az igát, és megszünsz ujjal mutogatni és hamisságot beszélni;
\par 10 Ha odaadod utolsó falatodat az éhezõnek, és az elepedt lelkût megelégíted: feltámad a setétségben világosságod, és homályosságod olyan lesz, mint a dél.
\par 11 És vezérel téged az Úr szüntelen, megelégíti lelkedet nagy szárazságban is, és csontjaidat megerõsíti, és olyan leszel, mint a megöntözött kert, és mint a vízforrás, a melynek vize el nem fogy.
\par 12 És megépítik fiaid a régi romokat, az emberöltõk alapzatait felrakod, és neveztetel romlás építõjének, ösvények megújítójának, hogy ott lakhassanak.
\par 13 Ha megtartóztatod szombaton lábadat, és nem ûzöd kedvtelésedet szent napomon, és a szombatot gyönyörûségnek hívod, az Úr szent és dicsõséges napjának, és megszenteled azt, dolgaidat nem tevén, foglalkozást sem találván, hamis beszédet sem szólván:
\par 14 Akkor gyönyörûséged lesz az Úrban; és én hordozlak a föld magaslatain, és azt mívelem, hogy Jákóbnak, atyádnak örökségével élj; mert az Úr szája szólt!

\chapter{59}

\par 1 Ímé, nem oly rövid az Úr keze, hogy meg ne szabadíthatna, és nem oly süket az õ füle, hogy meg nem hallgathatna;
\par 2 Hanem a ti vétkeitek választanak el titeket Istenetektõl, és bûneitek fedezték el orczáját ti elõttetek, hogy meg nem hallgatott.
\par 3 Mert kezeitek bemocskolvák vérrel, és ujjaitok vétekkel, ajkaitok hazugságot szólnak, nyelvetek gonoszt suttog.
\par 4 Nincsen, a ki az igazság mellett szólna, és nincsen, a ki igazságosan perelne, haszontalanban bíznak és hazugságot beszélnek, gonoszt fogadnak és vétket szûlnek.
\par 5 Vipera tojásait költik ki, és pókhálót szõnek; a ki tojásaikból eszik, meghal, és ha egyre rátapodsz, vipera kél ki.
\par 6 Pókhálójukból nem lesz ruha, csinálmányok nem felvehetõ; cselekedeteik hamisságnak cselekedetei, és erõszak tette van kezeikben.
\par 7 Lábaik a gonoszra futnak; és sietnek, hogy ártatlan vért ontsanak; gondolataik hamisságnak gondolatai, pusztítás és romlás ösvényeiken.
\par 8 A békesség útját nem ismerik, és nincsen jogosság kerékvágásukban, ösvényeiket elgörbítik, a ki azon jár, nem ismeri a békességet.
\par 9 Ezért van távol tõlünk az ítélet, és nem ér el minket az igazság, várunk világosságra, és ímé, sötétség, és fényességre, és ímé, homályban járunk!
\par 10 Tapogatjuk, mint vakok a falat, és tapogatunk, mint a kiknek szemök nincs, megütközünk délben, mint alkonyatkor, és olyanok vagyunk, mint a halottak az egészségesek közt.
\par 11 Morgunk, mint a medvék mindnyájan, és nyögvén nyögünk, mint a galambok, várjuk az ítéletet és nem jõ, a szabadulást és távol van tõlünk.
\par 12 Mert sokak elõtted gonoszságaink, és bûneink bizonyságot tesznek mi ellenünk, mert gonoszságaink velünk vannak, és vétkeinket ismerjük:
\par 13 Elpártoltunk és megtagadtuk az Urat, és eltávozánk a mi Istenünktõl, szóltunk nyomorgatásról és elszakadásról, gondoltunk és szóltunk szívünkbõl hazug beszédeket.
\par 14 És eltávozott a jogosság, és az igazság messze áll, mivel elesett a hûség az utczán, és az egyenesség nem juthat be.
\par 15 És a hûség hiányzik, és a ki a gonoszt kerüli, prédává lesz. És látta ezt az Úr és nem tetszék szemeinek, hogy jogosság nincsen.
\par 16 És látá, hogy nincsen senki, és álmélkodott, hogy nincsen közbenjáró; ezért karja segít néki, és igazsága gyámolítja õt.
\par 17 És felölté az igazságot, mint pánczélt, és a szabadítás sisakja van fején; felölté a bosszúállás ruháit, mint köpenyt, és búsulással vevé magát körül, mint egy palásttal.
\par 18 A cselekedetek szerint fog megfizetni: haraggal ellenségeinek, büntetéssel szorongatóinak, büntetéssel fizet a szigeteknek.
\par 19 És félik napnyugottól fogva az Úrnak nevét, és naptámadattól az õ dicsõségét, mikor eljõ, mint egy sebes folyóvíz, a melyet az Úr szele hajt.
\par 20 És eljõ Sionnak a megváltó, és azoknak, a kik Jákóbban megtérnek hamisságokból, szól az Úr.
\par 21 És én õ velök ily szövetséget szerzek, szól az Úr: lelkem, a mely rajtad nyugoszik, és beszédeim, a melyeket szádba adtam, el nem távoznak szádból, és magodnak szájából, és magod magvának szájából, így szól az Úr, mostantól mind örökké!

\chapter{60}

\par 1 Kelj fel, világosodjál, mert eljött világosságod, és az Úr dicsõsége rajtad feltámadt.
\par 2 Mert ímé, sötétség borítja a földet, és éjszaka a népeket, de rajtad feltámad az Úr, és dicsõsége rajtad megláttatik.
\par 3 És népek jönnek világosságodhoz, és királyok a néked feltámadott fényességhez.
\par 4 Emeld fel köröskörül szemeidet és lásd meg: mindnyájan egybegyûlnek, hozzád jönnek, fiaid messzirõl jönnek, és leányaid ölben hozatnak el.
\par 5 Akkor meglátod és ragyogsz örömtõl, és remeg és kiterjed szíved, mivel hozzád fordul a tenger kincsözöne, és hozzád jõ a népeknek gazdagsága.
\par 6 A tevék sokasága elborít, Midján és Éfa tevecsikói, mind Sebából jönnek, aranyat és tömjént hoznak, és az Úr dicséreteit hirdetik.
\par 7 Kédár minden juhai hozzád gyûlnek, Nebajóth kosai néked szolgálnak, felmennek kedvem szerint oltáromra, és dicsõségem házát megdicsõítem.
\par 8 Kik ezek, kik repülnek, mint a felleg, s mint a galambok dúczaikhoz?
\par 9 Igen, engem várnak a szigetek; és elõl jönnek Társis hajói, hogy elhozzák fiaidat messzirõl, és ezüstjöket és aranyokat azokkal együtt, a te Urad és Istened nevének és Izráel Szentjének, hogy téged megdicsõített.
\par 10 Az idegenek megépítik kõfalaidat, és királyaik szolgálnak néked; mivel haragomban megvertelek, és kegyelmemben megkönyörültem rajtad.
\par 11 És nyitva lesznek kapuid szüntelen, éjjel és nappal be nem zároltatnak, hogy behozzák hozzád a népek gazdagságát, és királyaik is bevitetnek.
\par 12 Mert a nép és az ország, a mely néked nem szolgáland, elvész, és a népek mindenestõl elpusztulnak.
\par 13 A Libánon ékessége hozzád jõ, cziprus, platán, sudar czédrus, mind együtt szenthelyemnek megékesítésére, hogy lábaim helyét megdicsõítsem.
\par 14 És meghajolva hozzád mennek a téged nyomorgatók fiai, és leborulnak lábad talpainál minden megútálóid, és neveznek téged az Úr városának, Izráel Szentje Sionának.
\par 15 A helyett, hogy elhagyott és gyûlölt valál és senki rajtad át nem ment, örökkévaló ékességgé teszlek, és gyönyörûséggé nemzetségrõl nemzetségre.
\par 16 És szopod a népek tejét, és a királyok emlõjét szopod, és megtudod, hogy én vagyok az Úr, megtartód és megváltód, Jákóbnak erõs Istene.
\par 17 Réz helyett aranyat hozok, vas helyett ezüstöt hozok, és a fák helyett rezet, és a kövek helyett vasat, és teszem fejedelmeiddé a békességet, és elõljáróiddá az igazságot.
\par 18 Nem hallatik többé erõszaktétel földeden, pusztítás és romlás határaidban, és a szabadulást hívod kõfalaidnak, és kapuidnak a dicsõséget.
\par 19 Nem a nap lesz néked többé nappali világosságod, és fényességül nem a hold világol néked, hanem az Úr lesz néked örök világosságod, és Istened lesz ékességed,
\par 20 Napod nem megy többé alá, és holdad sem fogy el, mert az Úr lesz néked örök világosságod, és gyászod napjainak vége szakad.
\par 21 És néped mind igaz lesz, és a földet mindörökké bírják, plántálásom vesszõszála õk, kezeim munkája dicsõségemre.
\par 22 A legkisebb ezerre nõ, és a legkevesebb hatalmas néppé. Én az Úr, idején, hamar megteszem ezt.

\chapter{61}

\par 1 Az Úr Isten lelke van én rajtam azért, mert fölkent engem az Úr, hogy a szegényeknek örömöt mondjak; elküldött, hogy bekössem a megtört szívûeket, hogy hirdessek a foglyoknak szabadulást, és a megkötözötteknek  megoldást;
\par 2 Hogy hirdessem az Úr jókedvének esztendejét, és Istenünk bosszúállása napját; megvígasztaljak minden gyászolót;
\par 3 Hogy tegyek Sion gyászolóira, adjak nékik ékességet a hamu helyett, örömnek kenetét a gyász helyett, dicsõségnek palástját a csüggedt lélek helyett, hogy igazság fáinak neveztessenek, az Úr plántáinak, az Õ dicsõségére!
\par 4 És megépítik a régi romokat, az õsi pusztaságokat helyreállítják, és a puszta városokat megújítják, és a régi nemzetségek pusztaságait.
\par 5 És ott állnak az idegenek, és legeltetik juhaitokat, és a jövevények szántóitok és vinczelléreitek lesznek.
\par 6 Ti pedig az Úr papjainak hívattattok, Istenünk szolgáinak neveztettek; a népek gazdagságát eszitek, és azok dicsõségével dicsekedtek.
\par 7 Gyalázatotokért kettõs jutalmat vesztek, és a szidalom helyett örvendenek örökségükben; ekként két részt öröklenek földükben, örökös örömük lesz.
\par 8 Mert én, az Úr, a jogosságot szeretem, gyûlölöm a gazsággal szerzett ragadományt; és megadom híven jutalmukat, és örök szövetséget szerzek velök.
\par 9 És ismeretes lesz magvok a népek közt, és ivadékaik a népségek között, valakik látják õket, megismerik õket, hogy õk az Úrtól megáldott magok.
\par 10 Örvendezvén örvendezek az Úrban, örüljön lelkem az én Istenemben; mert az üdvnek ruháival öltöztetett fel engem, az igazság palástjával vett engemet körül, mint võlegény, a ki pap módon ékíti fel magát, és mint menyasszony, a ki felrakja ékességeit.
\par 11 Mert mint a föld megtermi csemetéjét, és mint a kert kisarjasztja veteményeit, akként sarjasztja ki az Úr Isten az igazságot s a dicsõséget minden nép elõtt.

\chapter{62}

\par 1 Sionért nem hallgatok és Jeruzsálemért nem nyugszom, míg földerül, mint fényesség az Õ igazsága, és szabadulása, mint a fáklya tündököl.
\par 2 És meglátják a népek igazságodat, és minden királyok dicsõségedet, és új nevet adnak néked, a melyet az Úr szája határoz meg.
\par 3 És leszel dicsõség koronája az Úr kezében, és királyi fejdísz Istened kezében.
\par 4 Nem neveznek többé elhagyatottnak, és földedet sem nevezik többé pusztának, hanem így hívnak: én gyönyörûségem, és földedet így: férjhez adott; mert az Úr gyönyörködik benned, és földed férjhez adatik.
\par 5 Mert mint elveszi a legény a szûzet, akként vesznek feleségül téged fiaid, és a mint örül a võlegény a menyasszonynak, akként fog néked Istened örülni.
\par 6 Kõfalaidra, Jeruzsálem, õrizõket állattam, egész nap és egész éjjel szüntelen nem hallgatnak; ti, kik az Urat emlékeztetitek, ne nyugodjatok!
\par 7 És ne hagyjatok nyugtot néki, míg megújítja és dicsõségessé teszi Jeruzsálemet e földön.
\par 8 Megesküdt az Úr jobbjára s erõssége karjára: nem adom többé gabonádat eleségül ellenségeidnek, s idegenek nem iszszák mustodat, a melyért munkálódtál,
\par 9 Hanem a betakarók egyék azt meg, és dicsérjék az Urat, s a beszûrõk igyák meg azt szentségem pitvaraiban.
\par 10 Menjetek át, menjetek át a kapukon, készítsétek a népnek útát, töltsétek, töltsétek az ösvényt, hányjátok el a köveket, emeljetek zászlót a népek fölé.
\par 11 Ímé, az Úr hirdette mind a föld határáig: Mondjátok meg Sion leányának, ímé, eljött szabadulásod, ímé, jutalma vele jõ, és megfizetése õ elõtte!
\par 12 És hívják õket szent népnek, az Úr megváltottainak, és téged hívnak: keresett és nem elhagyott városnak.

\chapter{63}

\par 1 Ki ez, ki jõ Edomból, veres ruhákban Boczrából, a ki ékes öltözetében, ereje sokaságában büszke? Én, a ki igazságban szólok, elégséges vagyok a megtartásra.
\par 2 Miért veres öltözeted, és ruháid, mint a bornyomó ruhái?
\par 3 A sajtót egyedül tapostam, és a népek közül nem volt velem senki, és megtapodtam õket búsulásomban, és széttapostam õket haragomban: így fecscsent vérök ruháimra, és egész öltözetemet bekevertem.
\par 4 Mert bosszúállás napja volt szívemben, és megváltottaim esztendeje eljött.
\par 5 Körültekinték és nem vala segítõ, s álmélkodám és nem vala gyámolító, és segített nékem karom, és haragom gyámolított engem!
\par 6 És megtapodtam népeket búsulásomban, és megrészegítem õket haragomban, és ontám a földre véröket!
\par 7 Az Úrnak kegyelmességeirõl emlékezem, az Úr dicséreteirõl mind a szerint, a mit az Úr velünk cselekedett; az Izráel házához való sok jóságáról, a melyet velök cselekedett irgalma és kegyelmének sokasága szerint.
\par 8 És õ mondá: Bizony az én népem õk, fiak, a kik nem hazudnak: és lõn nékik megtartójok.
\par 9 Minden szenvedésöket Õ is szenvedte, és orczájának angyala megszabadítá õket, szerelmében és kegyelmében váltotta Õ meg õket, fölvette és hordozá õket a régi idõk minden napjaiban.
\par 10 Õk pedig engedetlenek voltak és megszomoríták szentségének lelkét, és õ ellenségükké lõn, hadakozott ellenök.
\par 11 S megemlékezék népe a Mózes régi napjairól: hol van, a ki õket kihozá a tengerbõl nyájának pásztorával? hol van, a ki belé adá az õ szentséges lelkét?
\par 12 Ki Mózes jobbján járatá dicsõségének karját, a ki a vizeket ketté választá elõttök, hogy magának örök nevet szerezzen?
\par 13 Ki járatá õket mélységekben, mint a lovat a síkon, és meg nem botlottanak!
\par 14 Mint a barmot, a mely völgybe száll alá, nyugodalomba vitte õket az Úr lelke: így vezérletted népedet, hogy magadnak dicsõ nevet szerezz!
\par 15 Tekints alá az égbõl és nézz le szentséged és dicsõséged hajlékából! Hol van buzgó szerelmed és hatalmad? Szívednek dobogása és irgalmad megtartóztatják magokat én tõlem!
\par 16 Hiszen Te vagy Atyánk, hiszen Ábrahám nem tud minket, és Izráel nem ismer minket, Te, Uram, vagy a mi Atyánk, megváltónk, ez neved öröktõl fogva.
\par 17 Miért engedél eltévelyedni minket útaidról, oh Uram! miért keményítéd meg szívünket, hogy ne féljünk tégedet? Térj meg szolgáidért, örökséged nemzetségeiért!
\par 18 Kevés ideig bírta szentségednek népe földét, ellenségink megtapodták szent helyedet.
\par 19 Olyanok lettünk, mint a kiken eleitõl fogva nem uralkodtál, a kik felett nem neveztetett neved.

\chapter{64}

\par 1 Oh, vajha megszakasztanád az egeket és leszállnál, elõtted a hegyek elolvadnának; mint a tûz meggyújtja a rõzsét, a vizet a tûz felforralja; hogy nevedet ellenségidnek megjelentsd, hogy elõtted reszkessenek a népek;
\par 2 Hogy cselekednél rettenetes dolgokat, a miket nem vártunk; leszállnál és elõtted a hegyek elolvadnának.
\par 3 Hiszen öröktõl fogva nem hallottak és fülökbe sem jutott, szem nem látott más Istent te kívüled, a ki így cselekszik azzal, a ki Õt várja.
\par 4 Elébe mégy annak, a ki örvend és igazságot cselekszik, a kik útaidban rólad emlékeznek! Ímé, Te felgerjedtél, és mi vétkezénk; régóta így vagyunk; megtartatunk-é?
\par 5 És mi mindnyájan olyanok voltunk, mint a tisztátalan, és mint megfertéztetett ruha minden mi igazságaink, és elhervadánk, mint a falomb mindnyájan, és álnokságaink, mint a szél, hordának el bennünket!
\par 6 S nem volt, a ki segítségül hívta volna nevedet, a ki felserkenne és beléd fogóznék, mert orczádat elrejtéd tõlünk, és álnokságainkban minket megolvasztál.
\par 7 Most pedig, Uram, Atyánk vagy Te, mi sár vagyunk és Te a mi alkotónk, és kezed munkája vagyunk mi mindnyájan.
\par 8 Oh ne haragudjál Uram felettébb, és ne mindörökké emlékezzél meg álnokságinkról; ímé lásd, kérünk, mindnyájan a Te néped vagyunk.
\par 9 Szentségednek városai pusztává lettenek, Sion pusztává lõn, Jeruzsálem kietlenné.
\par 10 Szentségünk és ékességünk házát, hol téged atyáink dicsértenek, tûz perzselé föl, és minden a miben gyönyörködénk, elpusztult.
\par 11 Hát megtartóztatod-é magad mind e mellett is, Uram; hallgatsz-é és gyötörsz minket felettébb?

\chapter{65}

\par 1 Megkeresni hagytam magamat azoktól, a kik nem is kérdeztenek; megtaláltattam magamat azokkal, a kik nem is kerestenek. Ezt mondám: Ímhol vagyok, ímhol vagyok, a népnek, a mely nem nevemrõl neveztetett.
\par 2 Kiterjesztém kezeimet egész napon a pártos nép után, a mely nem jó úton járt gondolatainak nyomán;
\par 3 A nép után, mely ingerel engem szemtõl szembe, szünetlenül, kertekben áldozik, és téglákon szerez jóillatot,
\par 4 Mely a sírokhoz ül, és a barlangokban hál, a disznóhúst eszi, és fertelmes leves van tálaiban,
\par 5 Mely ezt mondja: Maradj otthon, ne jõjj hozzám, mert szent vagyok néked; e nép füst az orromban és szüntelen égõ tûz.
\par 6 Ímé, feliratott elõttem: nem hallgatok, csak ha elõbb megfizetek, megfizetek keblökben:
\par 7 Vétkeitekért és atyáitok vétkeiért mind együtt, szól az Úr, a kik hegyeken tettek jóillatot és halmokon csúfoltak engemet meg, és visszamérem elõször jutalmokat keblökre.
\par 8 Így szól az Úr: Mint a mikor mustot lelnek a fürtben, ezt mondják: ne veszesd el, mert áldás van benne, ekként cselekszem szolgáimért, és nem vesztek mindent el!
\par 9 És nevelek Jákóbból magot, és Júdából, a ki hegyeimnek örököse legyen, és bírják azt választottaim, és szolgáim lakjanak ott!
\par 10 És lesz Sáron nyájak legelõjévé, és Ákhor völgye barmok fekvõhelyévé népem számára, a mely engem keresett.
\par 11 Ti pedig, a kik az Urat elhagyátok, a kik szent hegyemrõl elfeledkezétek, ti, kik Gádnak asztalt készítetek, és Meninek italáldozatot töltötök,
\par 12 Titeket én a kard alá számlállak, és mindnyájan leborultok megöletésre: mert hívtalak és nem feleltetek, szóltam és nem hallottátok: a gonoszt cselekedtétek szemeim elõtt, és a mit nem szerettem, azt választottátok.
\par 13 Azért így szól az Úr Isten: Ímé, szolgáim esznek, ti pedig éheztek, ímé, szolgáim isznak, ti pedig szomjúhoztok, ímé szolgáim örvendnek, de ti megszégyenültök!
\par 14 Ímé, szolgáim vígadnak szívök boldogságában, és ti kiáltani fogtok szívetek fájdalmában, és megtört lélekkel jajgatni fogtok;
\par 15 És átok gyanánt hagjátok itt neveteket az én választottaimnak, és megöl titeket az Úr Isten, és szolgáit más névvel nevezi,
\par 16 Hogy a ki magát áldja e földön, áldja magát az igaz Istenben, és ki esküszik e földön, esküdjék az igaz Istenre, mert elfeledvék a régi nyomorúságok, és mert elrejtvék szemeim elõl.
\par 17 Mert ímé, új egeket és új földet teremtek, és a régiek ingyen sem emlittetnek, még csak észbe sem jutnak;
\par 18 Hanem örüljetek és örvendjetek azoknak mindörökké, a melyeket én teremtek; mert ímé, Jeruzsálemet vígassággá teremtem, és az õ népét örömmé.
\par 19 És vígadok Jeruzsálem fölött, és örvendek népem fölött, és nem hallatik többé abban siralomnak és kiáltásnak szava!
\par 20 Nem lesz ott többé csupán néhány napot ért gyermek, sem vén ember, a ki napjait be nem töltötte volna, mert az ijfú száz esztendõs korában hal meg és a bûnös száz esztendõs korában átkoztatik meg.
\par 21 Házakat építnek és bennök lakoznak, és szõlõket plántálnak és eszik azok gyümölcsét.
\par 22 Nem úgy építnek, hogy más lakjék benne; nem úgy plántálnak, hogy más egye a gyümölcsöt, mert mint a fáké, oly hosszú lesz népem élete, és kezeik munkáját elhasználják választottaim.
\par 23 Nem fáradnak hiába, nem nemzenek a korai halálnak, mivel az Úr áldottainak magva õk, és ivadékaik velök megmaradnak.
\par 24 És mielõtt kiáltanának, én felelek, õk még beszélnek, és én már meghallgattam.
\par 25 A farkas és bárány együtt legelnek, az oroszlán, mint az ökör, szalmát eszik, és a kígyónak por lesz az õ kenyere. Nem ártanak és nem pusztítnak sehol szentségemnek hegyén; így szól az Úr.

\chapter{66}

\par 1 Így szól az Úr: Az egek nékem ülõszékem, és a föld lábaimnak zsámolya: minõ ház az, a melyet nékem építeni akartok, és minõ az én nyugalmamnak helye?
\par 2 Hiszen mindezeket kezem csinálta, így álltak elõ mindezek; így szól az Úr. Hanem erre tekintek én, a ki szegény és megtörött lelkû, és a ki beszédemet rettegi.
\par 3 A ki bikát öl, embert üt agyon; a ki juhval áldozik, az ebet öl; a ki ételáldozattal jõ, disznóvért hoz elém; a ki tömjént gyújt, bálványt imád! Miként õk így választák útaikat, és lelkök útálatosságaikban gyönyörködött:
\par 4 Akképen választom én is az õ megcsúfolásukat, és rájok hozom, a mitõl félnek; mivel hívtam és senki nem felelt, szóltam és nem hallották, és a gonoszt cselekedték szemeim elõtt, s a mit nem szerettem, azt választák.
\par 5 Halljátok az Úrnak beszédét, a kik rettegtek az õ beszédére: így szólnak testvéreitek, a kik titeket gyûlölnek, nevemért eltaszítanak: Jelenjék meg az Úrnak dicsõsége, hogy lássuk örömötöket; de õk megszégyenülnek.
\par 6 Halld! zúgás a város felõl, hah! a templom felõl, hah! az Úr, a ki megfizet ellenségeinek.
\par 7 Mielõtt vajudott volna, szült, mielõtt fájdalom jött rá, fiút hozott világra.
\par 8 Ki hallott olyat, mint ez, ki látott hasonló dolgokat? Hát egy ország egy nap jön-é világra és egy nép egyszerre születik-é? mert vajudott és meg is szûlé Sion az õ fiait!
\par 9 Hát én csak megindítsam és ne vigyem véghez a szûlést? szól az Úr, vagy én, a ki szûletek, bezárjam-é a méhet? így szól Istened.
\par 10 Örüljetek Jeruzsálemmel, és örvendjetek fölötte mind, a kik õt szeretitek; vígadjatok vele örvendezéssel mindnyájan, a kik gyászoltatok miatta!
\par 11 Hogy szopjatok és megelégedjetek megvígasztaltatásának emlõjén, hogy igyatok és örvendjetek dicsõségének bõségén.
\par 12 Mivel így szól az Úr: Ímé kiterjesztem rá a békességet, mint egy folyóvizet, és mint kiáradott patakot a népek dicsõségét, hogy szopjátok; ölben fognak hordozni, és térdeiken czirógatnak titeket.
\par 13 Mint férfit, a kit anyja vígasztal, akként vígasztallak titeket én, és Jeruzsálemben vesztek vígasztalást!
\par 14 Meglátjátok és örül szívetek, csontjaitok, mint a zöld fû, virágoznak, és megösmerik az Úr kezét az Õ szolgáin, és haragját ellenségei fölött.
\par 15 Mert ímé, az Úr eljõ tûzben, s mint forgószél az õ szekerei, hogy megfizesse búsulásában az Õ haragját, és megfeddését sebesen égõ lánggal.
\par 16 Mert az Úr tûzzel ítél és kardjával minden testet, és sokan lesznek az Úrtól megöltek.
\par 17 A kik magokat megszentelik és mossák a bálvány kertekért, egy pap megett, a középen; a kik disznóhúst esznek és férget és egeret, együtt pusztulnak el mind, mondja az Úr.
\par 18 És én cselekedeteiteket és gondolataitokat megbüntetem! Eljõ az idõ, hogy minden népeket és nyelveket egybegyûjtsek, hogy eljövén, meglássák az én dicsõségemet.
\par 19 És teszek köztök jelt, és küldök közülök megszabadultakat a népekhez, Tarsisba, Pulba és Ludba, az íjjászokhoz, Tubálhoz és Jávánhoz; a messze szigetekbe, a melyek rólam nem hallottak, és nem látták dicsõségemet, és hirdetik dicsõségemet a népek között.
\par 20 És elhozzák minden testvéreiteket minden népek közül ajándékul az Úrnak, lovakon, szekereken, hintókban, öszvéreken és tevéken szentségemnek hegyére Jeruzsálembe, így szól az Úr, a mint hozzák Izráelnek fiai az ajándékot tiszta edényben az Úrnak házába.
\par 21 És ezek közül is választok a papok közé, a Léviták közé, így szól az Úr.
\par 22 Mert mint az új egek és az új föld, a melyeket én teremtek, megállnak én elõttem, szól az Úr, azonképen megáll a ti magvatok és nevetek;
\par 23 És lesz, hogy hónapról-hónapra és szombatról-szombatra eljõ minden test engem imádni, szól az Úr.
\par 24 És kimenvén, látni fogják azoknak holttesteit, a kik ellenem vétkeztek, mert az õ férgök meg  nem hal és tüzök el nem aluszik, és minden test elõtt borzadásul lesznek.


\end{document}