\begin{document}

\title{Jeremiah}


\chapter{1}

\par 1 Jeremiásnak, Hilkiás fiának beszédei, a ki az Anatótban, Benjámin földén lakó papok közül vala;
\par 2 A kihez szóla az Úr Jósiásnak, az Ammon fiának, Júda királyának napjaiban, az õ uralkodásának tizenharmadik esztendejében;
\par 3 Továbbá Jójakimnak, a Jósiás fiának, Júda királyának  napjaiban, Sedékiásnak, a Jósiás fiának, Júda királyának egészen tizenegyedik esztendejéig, Jeruzsálem fogságba viteléig, a mi az ötödik hónapban vala;
\par 4 Szóla pedig az Úr nékem mondván:
\par 5 Mielõtt az anyaméhben megalkottalak, már ismertelek, és mielõtt az anyaméhbõl kijövél, megszenteltelek; prófétának rendeltelek a népek közé.
\par 6 És mondék: Ah, ah Uram Istenem! Ímé, én nem tudok beszélni; hiszen ifjú vagyok én!
\par 7 Az Úr pedig monda nékem: Ne mondd ezt: Ifjú vagyok én; hanem menj mind azokhoz, a kikhez küldelek téged, és beszéld mindazt, a mit parancsolok néked.
\par 8 Ne félj tõlök, mert én veled vagyok, hogy megszabadítsalak téged! mond az Úr.
\par 9 És kinyújtá az Úr az õ kezét, és megilleté számat, és monda nékem az Úr: Ímé, az én igéimet adom a te szádba!
\par 10 Lásd, én e mai napon népek fölé és országok fölé rendellek téged, hogy gyomlálj, irts, pusztíts, rombolj, építs és plántálj!
\par 11 Szóla továbbá nékem az Úr, mondván: Mit látsz te, Jeremiás? És mondék: Mandulavesszõt látok én.
\par 12 És monda nékem az Úr: Jól láttál, mert gondom van az én igémre, hogy beteljesítsem azt.
\par 13 És másodszor is szóla hozzám az Úr, mondván: Mit látsz te? És felelék: Forró fazekat látok én, és pedig a szája észak felõl van.
\par 14 És monda nékem az Úr: Észak felõl támad a veszedelem e földnek minden lakosára.
\par 15 Mert ímé, elõhívom én az északi országok minden nemzetségét, mondja az Úr, és eljõnek, és kiki felállítja az õ királyi székét Jeruzsálem kapui elõtt, és köröskörül minden kerítése ellen, és Júdának minden városa ellen.
\par 16 És kimondom ítéleteimet felettök az õ mindenféle gonoszságukért, minthogy elszakadtak tõlem, és idegen isteneknek áldoztak, és tulajdon kezeik munkáit imádták.
\par 17 Te azért övezd fel derekadat, és kelj fel, és mondd meg nékik mindazt, a mit én parancsolok néked; meg ne riadj tõlök, különben én riasztalak el téged elõlök!
\par 18 Mert ímé én erõsített várossá, vasoszloppá és érczbástyává teszlek ma téged mind ez egész földön, Júda királyai, fejedelmei és papjai ellen és a föld népe ellen.
\par 19 Viaskodni fognak ugyan ellened, de nem gyõznek meg téged, mert én veled vagyok, azt mondja az Úr, hogy megszabadítsalak téged.

\chapter{2}

\par 1 Majd szóla az Úr nékem, mondván:
\par 2 Menj el, és kiálts Jeruzsálem füleibe, mondván: Ezt mondja az Úr: Emlékezem reád gyermekkorod ragaszkodására, mátkaságod szeretetére, a mikor követtél engem a pusztában, a még be nem vetett földön.
\par 3 Szent volt Izráel az Úrnak, az õ termésének zsengéje; a kik emésztik vala õt, mind vétkeznek vala, veszedelem támada rájok, azt mondja az Úr.
\par 4 Halljátok meg az Úr szavát, Jákób háza és Izráel házának minden nemzetsége.
\par 5 Így szól az Úr: Micsoda hamisságot találtak bennem a ti atyáitok, hogy elidegenedtek tõlem, és hiábavalóság után jártak, és hiábavalókká lettek?
\par 6 Még csak azt sem mondták: Hol van az Úr, a ki felhozott minket Égyiptom földérõl, a ki vezérelt minket a pusztában, a kietlen és járatlan földön, a szomjúságnak és a halál árnyékának földén, a melyen nem vonult át ember, és a hol halandó nem lakott?
\par 7 És bevittelek titeket a bõség földébe, hogy annak gyümölcseivel és javaival éljetek; és bementetek, és megfertõztettétek az én földemet, és az én örökségemet útálatossá tevétek.
\par 8 A papok nem mondták: Hol van az Úr? A törvény magyarázói nem ismertek engem, és a pásztorok hûtelenekké lettek hozzám, a próféták pedig a Baál által prófétáltak, és azok után jártak, a kik tehetetlenek.
\par 9 Azért még perbe szállok veletek, mondja az Úr, és perelni fogok a ti fiaitok fiaival is!
\par 10 Menjetek csak át a Kittim szigeteire, és lássátok meg; és küldjetek Kédárba is és szorgalmasan vizsgálódjatok, és lássátok meg, ha volt-é ott ehhez hasonló?
\par 11 Ha cserélt-é valamely nemzet isteneket? noha azok nem istenek. Az én népem pedig felcserélte az õ dicsõségét tehetetlenséggel!
\par 12 Álmélkodjatok ezen, oh egek, és borzadjatok és rémüljetek meg igen! azt mondja az Úr.
\par 13 Mert kettõs gonoszságot követett el az én népem: Elhagytak engem, az élõ vizek forrását, hogy kútakat ássanak magoknak; és repedezett kútakat ástak, a melyek nem tartják a vizet.
\par 14 Szolga-é Izráel, a vagy otthon szülött-é õ? Miért lett prédává?
\par 15 Oroszlánok ordítanak reá, megeresztik hangjokat, és sivataggá teszik földjét; városai szétromboltatnak, lakatlanok.
\par 16 Nóf és Tahpan fiai is betörik koponyádat.
\par 17 Vajjon nem te szerezted-é ezt magadnak? Elhagytad az Urat, a te Istenedet, a mikor vezérelt téged az úton!
\par 18 Most is mi dolgod van néked Égyiptom útjával? Hogy Nilus-vizet igyál? Vagy mi dolgod van néked Assúr útjával? Hogy a folyam vizét igyad?
\par 19 A magad gonoszsága fenyít meg téged, és a te elpártolásod büntet meg téged. Tudd meg hát és lásd meg: mily gonosz és keserves dolog, hogy elhagytad az Urat, a te Istenedet, és hogy nem félsz engemet, ezt mondja az Úr, a Seregeknek Ura.
\par 20 Bizony régóta széttörtem a te igádat, és letéptem köteleidet, és azt mondtad: Nem leszek rabszolga; mindamellett minden magas halmon és minden lombos fa alatt bujkálsz vala te, mint egy parázna.
\par 21 Pedig én úgy plántállak vala el téged, mint nemes szõlõvesszõt, mindenestõl hûséges magot: mimódon változtál hát nékem idegen  szõlõtõnek fattyú hajtásává?
\par 22 Még ha lúgban mosakodnál is, vagy szappanodat megsokasítanád is, feljegyezve marad a te álnokságod elõttem, mondja az Úr Isten.
\par 23 Mimódon mondhatnád: Nem undokíttattam meg, nem jártam a Baálok után?! Lásd meg a te útadat a völgyben, ismerd meg csak: mit cselekedtél! Gyorslábú kancza-teve, a mely ide-oda futkos útain.
\par 24 Pusztához szokott nõstény vadszamár, a mely érzékiségének kívánságában levegõ után kapkod. Gerjedelmét kicsoda csillapíthatja le? Senki ne fáraszsza magát, a ki ezt keresi, megtalálja ezt a maga hónapjában.
\par 25 Tartóztasd meg lábadat a mezítelenségtõl, és torkodat a szomjúságtól! És te azt mondod: Hiába, nem lehet! mert idegeneket szeretek, és utánok járok.
\par 26 A miképen megszégyenül a tolvaj, ha rajtakapják, akképen szégyenül meg Izráel háza: õ maga, az õ királyai, fejedelmei, papjai és prófétái,
\par 27 A kik azt mondják a fának: Te vag az én atyám! a kõnek pedig: Te szûltél engem! Bizony háttal fordulnak felém és nem arczczal, de nyomorúságuk idején azt mondják majd: Kelj föl és szabadíts meg minket!
\par 28 De hol vannak a te isteneid, a melyeket magadnak készítél? Keljenek fel, ha megszabadíthatnak téged a te nyomorúságod idején; hiszen annyi istened volt, oh Júda, a hány városod!
\par 29 Miért perlekedtek velem? Mindnyájan hûtelenekké lettetek hozzá, mondja az Úr.
\par 30 Hiába ostoroztam fiaitokat, a fenyítés nem fogott rajtok; fegyveretek úgy emésztette prófétáitokat, mint pusztító oroszlán.
\par 31 Oh te nemzetség! Lásd meg az Úr dolgát! A puszta voltam-é én Izráelnek, avagy a setétség földje? Miért mondotta az én népem: Szabadok vagyunk, nem megyünk többé hozzád!
\par 32 Vajjon elfelejtkezik-é a lány az õ ékszereirõl; a menyasszony az õ nyaklánczairól? Az én népem pedig számtalan napokon elfelejtett engem!
\par 33 Mit szépíted a te útadat, hogy te szeretetet keresel; holott még a gonoszokat is tanítod a te útaidra!
\par 34 Még ruháid szélén is található szegény, ártatlan emberek vére, nem azért, hogy betörésen kaptad õket, hanem mindamazokért!
\par 35 És azt mondod: Bizonyára ártatlan vagyok, hiszen elfordult tõlem az õ haragja! Ímé, én törvénybe szállok veled, mivelhogy azt mondod: Nem vétkeztem!
\par 36 Mit futkosol oly igen, változtatván útadat? Égyiptom miatt is megszégyenülsz, a mint megszégyenültél Assúr miatt.
\par 37 Még ettõl is elszakadsz s kezeidet fejedre kulcsolod, mert útálja az Úr a te bizodalmasaidat, és nem leszel velök szerencsés.

\chapter{3}

\par 1 Nos hát! Ha elbocsátja valaki az õ feleségét, és ez eltávozik tõle és más férfiúé lesz: vajjon visszamehet-é még a férfi õ hozzá? Nem lenne-é az a föld fertelmesen megfertõztetve? Te pedig sok szeretõvel paráználkodtál, és visszajönnél hozzám? mondja az Úr.
\par 2 Emeld fel szemeidet a magaslatokra, és lásd meg hol nem szeplõsítettek meg téged? Az útakon vártál reájok, mint az arab a pusztában, és megfertõztetted a földet paráznaságaiddal és gonoszságoddal.
\par 3 Noha megvonattak a korai záporok, és késõi esõ sem volt: mégis parázna asszony homlokúvá lettél, szégyenkezni nem akartál.
\par 4 Nemde mostantól így kiáltasz nékem: Atyám! Ifjúságomnak te nagy vezére.
\par 5 Mindörökké haragszik-é, mindvégig bosszankodik-é? Ímé, ezt mondod, de cselekszed a gonoszságokat, a mint tõled telik.
\par 6 És monda az Úr nékem Jósiás király napjaiban: Láttad-é, a mit az elpártolt Izráel cselekedett? Elment õ minden magas hegyre és minden zöldelõ fa alá, és ott paráználkodott.
\par 7 És mondám, miután mindezt megcselekedte: Térj vissza hozzám! de nem tért vissza. És látta ezt az õ hitszegõ húga, a Júda.
\par 8 És láttam, hogy mindamellett is, hogy elbocsátottam az elpártolt parázna Izráelt, és adtam néki elválásról való levelet: nem félt a hitszegõ Júda, az õ húga, hanem elment, és õ is paráználkodott.
\par 9 És lõn, hogy az õ paráznaságának hírével megfertõztette a földet; mert kõvel és fával paráználkodott.
\par 10 És mindez után sem tért vissza hozzám az õ hitszegõ húga, a Júda, az õ szíve teljességével, hanem csak képmutatásssal, azt mondja az Úr.
\par 11 És monda nékem az Úr: Igazabb lelkû az elpártolt Izráel, mint a hitszegõ Júda.
\par 12 Menj el, és kiáltsd e szókat észak felé, és mondjad: Térj vissza, elpártolt Izráel, ezt mondja az Úr, és nem bocsátom reátok haragomat, mert kegyelmes vagyok én, ezt mondja az Úr, nem haragszom mindörökké.
\par 13 Csakhogy ismerd el a te hamisságodat, hogy hûtelenné lettél az Úrhoz, a te Istenedhez, és szertefutottál útaidon az idegenekhez mindenféle zöldelõ fa alá, és az én szómra nem hallgattatok, ezt mondja az Úr.
\par 14 Térjetek meg, szófogadatlan fiak, azt mondja az Úr, mert én férjetekké lettem néktek, és  magamhoz veszlek titeket, egyet egy városból, kettõt egy nemzetségbõl, és beviszlek titeket Sionba.
\par 15 És adok néktek szívem szerint való pásztorokat, és legeltetnek tudománynyal és értelemmel.
\par 16 És ha majd megsokasodtok és megszaporodtok a földön azokban a napokban, azt mondja az Úr, nem mondják többé: Az Úr szövetségének ládája! Szívére se veszi senki, rá se gondolnak, meg sem látogatják, és nem készítik meg újra.
\par 17 Abban az idõben Jeruzsálemet hívják majd az Úr királyiszékének és minden nemzet oda gyülekezik az Úr nevéért Jeruzsálembe, és nem követik többé gonosz szívöknek makacsságát.
\par 18 Azokban a napokban a Júda háza Izráel házával fog járni, és együtt mennek be az északi földrõl abba a földbe, a melyet örökségül adtam a ti atyáitoknak.
\par 19 Azt mondám ugyanis magamban: Miképen tegyelek téged a fiak közé, és adjam néked a kivánatos földet, a pogányok seregének drága örökségét? És azt végezém: Hívj engem atyámnak, és ne pártolj el tõlem!
\par 20 Mindamellett a mint hitszegõvé lesz az asszony az õ társa iránt: úgy lettetek hitszegõkké irántam, Izráel háza, azt mondja az Úr.
\par 21 Szó hallatszik a magaslatokon, Izráel fiainak esdeklõ sírása, mert elfordították útjokat, elfelejtkeztek az Úrról, az õ Istenökrõl.
\par 22 Térjetek vissza, szófogadatlan fiak, és meggyógyítom a ti elpártolástokat! Ímé, mi hozzád járulunk, mert te vagy az Úr, a mi Istenünk!
\par 23 Bizony hiábavaló a halmokról, a hegyekrõl való zajongás; bizony az Úrban, a mi Istenünkben van Izráel megmaradása!
\par 24 És ez a gyalázatosság emésztette a mi atyáink szerzeményét gyermekségünk óta: juhaikat, szarvasmarháikat, fiaikat és leányaikat!
\par 25 Gyalázatunkban heverünk, és elborít minket a mi szégyenünk; mert vétkeztünk az Úr ellen, a mi Istenünk ellen: mi és a mi atyáink gyermekségünk óta mind e mai napig, és nem hallgattunk az Úrnak, a mi Istenünknek szavára.

\chapter{4}

\par 1 Ha visszatérsz, Izráel, ezt mondja az Úr, hozzám térj vissza, és ha eltávolítod a te útálatosságaidat elõlem és nem ingadozol;
\par 2 És így esküszöl: Él az Úr! hûségben, egyenességben és igazságban: akkor õ benne áldják majd magokat a nemzetek, és benne dicsekesznek.
\par 3 Mert ezt mondja az Úr Júda és Jeruzsálem férfiainak: Szántsatok magatoknak új ugart, és ne vessetek tövisek közé!
\par 4 Metéljétek magatokat körül az Úrnak, és távolítsátok el  szívetek elõbõreit, Júda férfiai és Jeruzsálem lakosai, hogy fel ne gyúladjon az én haragom, mint a tûz, és olthatatlanul ne égjen a ti cselekedeteitek gonoszsága miatt.
\par 5 Adjátok hírül Júdában, és hallassátok Jeruzsálemben és mondjátok, és kürtöljétek az országban; kiáltsátok teljes erõvel, és mondjátok: Gyûljetek össsze és menjünk be az erõsített városokba!
\par 6 Emeljetek zászlót a Sion felé; fussatok, meg ne álljatok, mert veszedelmet hozok észak felõl és nagy romlást!
\par 7 Felkelt az oroszlán az õ tanyájából, és a népek pusztítója elindult; kijött helyébõl, hogy elpusztítsa a te földedet; városaid lerontatnak, lakatlanokká lesznek.
\par 8 Öltözzetek azért gyászba, sírjatok és jajgassatok, mert az Úr haragjának tüze nem távozott el rólunk!
\par 9 Azon a napon pedig, ezt mondja az Úr, elvész a király bátorsága és a fõemberek bátorsága, és a papok elálmélkodnak, és a próféták elcsodálkoznak.
\par 10 Én pedig ezt mondom: Oh Uram Isten! Bizony igen megcsaltad ezt a népet, és Jeruzsálemet, ezt mondván: Békességtek lesz! holott a szablya lelkünkig hatott.
\par 11 Abban az idõben azt mondják e népnek és Jeruzsálemnek: Száraztó szél jõ a magas helyekrõl a pusztában, az én népem leányának útján; nem szóráshoz és nem tisztításhoz való!
\par 12 Erõs szél jõ onnan reám is; azért hát ítéletet mondok ellenök.
\par 13 Ímé! úgy jõ fel, mint a felleg, és szekerei olyanok, mint a szélvész, lovai gyorsabbak a saskeselyûknél. Jaj nékünk, mert elvesztünk!
\par 14 Jeruzsálem, tisztítsd meg szívedet a gonoszságtól, hogy megtartassál! Meddig maradnak még te benned a te bûnös gondolataid?
\par 15 Mert hírmondó szava hangzik Dán felõl, és Efraim hegyérõl vészhirdetõ.
\par 16 Mondjátok meg a nemzeteknek; nosza, hirdessétek Jeruzsálem ellen: Õrizõk jönnek messze földrõl, és kiáltoznak Júda városai ellen.
\par 17 Mint a mezõnek õrizõi, úgy lesznek ellene köröskörül, mert daczoskodott velem! mondja az Úr.
\par 18 A te magad viselete és a te cselekedeteid szerezték ezeket néked; ez a gonoszságod bizony keserû, bizony egész a szívedig hatott!
\par 19 Oh én belsõm, oh én belsõm! Aléldozom, oh én szívemnek rekeszei! Háborog a szívem, nem hallgathatok! Hiszen hallottad én lelkem a kürt szavát, a harczi riadót!
\par 20 Vészre vészt jelentenek; bizony elpusztul az egész föld, nagy hamarsággal elpusztulnak az én sátraim, kárpitjaim egy pillantás alatt!
\par 21 Meddig látok még hadi zászlót, és hallom a kürt szavát?
\par 22 Bizony bolond az én népem: engem nem ismernek, balgatag fiak õk, és nem értelmesek! Bölcsek õk a gonoszra, jót cselekedni pedig tudatlanok!
\par 23 Nézek a földre, de ímé kietlen és puszta; és az égre, de nincsen világossága!
\par 24 Nézek a hegyekre is, ímé reszketnek; és a halmokra, de mind ingadoznak!
\par 25 Nézek és ímé egy ember sincsen; és az ég madarai is mind elmenekültek.
\par 26 Nézek, és ímé a bõ termõ föld pusztává lõn; és minden városa összeomlott az Úr elõtt, az õ haragjának tüze elõtt!
\par 27 Bizony ezt mondja az Úr: Pusztasággá lesz az egész ország, de nem vetem  végét!
\par 28 Azért gyászol a föld, és homályosodik el oda fenn az ég, mert szólottam, határoztam, és meg nem bánom és el nem térek attól.
\par 29 A lovasoknak és a kézíveseknek kiáltozása elõl elfut minden város, elrejtõznek a sûrûségekbe, és felmásznak a sziklákra: minden város elhagyottá lett, és egyetlen ember sem lakik azokban.
\par 30 És te, elpusztult, mit cselekszel akkor? Ha bíborba öltözöl, ha arany kösöntyûkkel ékesíted magadat, és ha festékkel mázolod is ki szemeidet: hiába szépítgeted magadat! Megvetnek téged  a szeretõid, életedre törnek.
\par 31 Mert mintha vajudó asszony szavát hallanám, mintha az elõször szûlõnek sikoltozását: olyan Sion leányának hangja; nyög, csapkodja kezeit: Jaj nékem! mert roskadozik lelkem a gyilkosok elõtt.

\chapter{5}

\par 1 Járjátok el Jeruzsálem utczáit, és nézzétek csak és tudjátok meg és tudakozzátok meg annak piaczain, ha találtok-é egy embert; ha van-é valaki, a ki igazán cselekszik, hûségre törekszik, és én megbocsátok néki!
\par 2 Még ha azt mondják is: Él az Úr! bizony hamisan esküsznek!
\par 3 Uram! a te szemeid avagy nem a hûségre néznek-é? Megverted õket, de nem bánkódtak; tönkre tetted õket, de nem akarják felvenni a dorgálást; orczáik keményebbekké lettek a kõsziklánál; nem akartak megtérni.
\par 4 Én pedig ezt mondom vala: Bizony szerencsétlenek ezek; bolondok, mert nem ismerik az Úrnak útját, Istenöknek ítéletét!
\par 5 Elmegyek azért a fõemberekhez, és beszélek velök; hiszen õk ismerik az Úrnak útját, Istenöknek ítéletét! Ámde õk törték össze egyetemlegesen az igát, és tépték le a köteleket!
\par 6 Azért széttépi õket az oroszlán az erdõbõl, elpusztítja õket a puszták farkasa, párducz ólálkodik az õ városaik körül; a ki kijön azokból, mind szétszaggattatik; mert megsokasodtak az õ bûneik, és elhatalmasodtak az õ hitszegéseik.
\par 7 Oh, miért bocsátanék meg néked? Fiaid elhagytak engem, és azokra esküsznek, a kik nem istenek; noha jól tartottam õket, mégis paráználkodtak és tolongtak a parázna házába.
\par 8 Mint a hizlalt lovak, viczkándozókká lettek; kiki az õ felebarátjának feleségére nyerít.
\par 9 Avagy ne büntessem-é meg az ilyeneket, mondja az Úr, és az e féle népen, mint ez, ne álljon-é bosszút az én lelkem?
\par 10 Menjetek fel az õ kerítéseire és rontsátok, de ne tegyétek semmivé egészen! Távolítsátok el az ágait, mert nem az  Úréi azok!
\par 11 Mert igen hûtelenné lett hozzám az Izráel háza és a Júda háza, azt mondja az Úr!
\par 12 Megtagadták az Urat, és azt mondták: Nincsen õ, és nem jöhet reánk veszedelem: sem fegyvert, sem éhséget nem látunk!
\par 13 A próféták is széllé lesznek, és nem lesz, a ki beszéljen bennök: így esik az õ dolguk!
\par 14 Azért ezt mondja az Úr, a Seregeknek Istene: Miután ti ilyen szóval szóltatok: ímé tûzzé teszem az én igémet a te szádban, ezt a népet pedig fákká, hogy megemészsze õket!
\par 15 Ímé, én hozok reátok messzünnen való nemzetet, oh Izráel háza! ezt mondja az Úr. Kemény nemzet ez, õs idõbõl való nemzet ez; nemzet, a melynek nyelvét  nem tudod, és nem érted, mit beszél!
\par 16 Tegze olyan, mint a nyitott sír; mindnyájan vitézek.
\par 17 És megemészti aratásodat és kenyeredet; megemészti fiaidat és leányaidat; megemészti juhaidat és ökreidet; megemészti szõlõdet és fügédet; erõsített városaidat, a melyekben te bizakodol, fegyverrel rontja le.
\par 18 De még ezekben a napokban sem teszlek benneteket egészen semmivé, azt mondja az Úr.
\par 19 Ha pedig az lenne, hogy kérdeznétek: Miért cselekedte mindezt velünk az Úr, a mi Istenünk? akkor azt mondjad nékik: A miképen elhagytatok engem és idegen isteneknek szolgáltatok a ti földeteken: azonképen  idegeneknek fogtok szolgálni olyan földön, a mely nem a tiétek.
\par 20 Hirdessétek ezt a Jákób házában, és hallassátok Júdában, mondván:
\par 21 Halljátok csak, te bolond és esztelen nép, a kiknek  szemeik vannak, de nem látnak; füleik vannak, de nem hallanak!
\par 22 Nem féltek-é engem? ezt mondja az Úr. Az én orczám elõtt nem remegtek-é? Hiszen én rendeltem a fövenyet a tenger határának örök korlátul, a melyet át nem hághat, és ha megrázkódtatják is habjai, de nem bírnak vele, és ha megháborodnak, sem hághatnak át rajta.
\par 23 De ennek a népnek szilaj és daczos szíve van; elhajlottak és elmentek;
\par 24 És még szívökben sem mondják: Oh, csak félnõk az Urat, a mi Istenünket, a ki esõt, korai és késõi záport ad annak idejében; az aratásnak rendelt heteit megõrzi számunkra!
\par 25 A ti bûneitek fordították el ezeket tõletek, és a ti vétkeitek fosztottak meg titeket e jótól!
\par 26 Mert istentelenek vannak az én népem között; guggolva fülelnek, mint a madarászok; tõrt hánynak, embereket fogdosnak.
\par 27 Mint a madárral teli kalitka, úgy vannak teli az õ házaik álnoksággal; ezért lettek nagyokká és gazdagokká!
\par 28 Meghíztak, megfényesedtek; eláradtak a gonosz beszédben; az árvának ügyét nem ítélik  igaz ítélettel, hogy boldoguljanak; sem a szegényeknek nem szolgáltatnak igazságot.
\par 29 Hát ezeket ne büntessem-é meg, ezt mondja az Úr; az ilyen nemzeten, mint ez, avagy ne álljon-é bosszút az én lelkem?
\par 30 Borzadalmas és rettenetes dolgok történnek e földön:
\par 31 A próféták hamisan prófétálnak, és a papok tetszésök szerint hatalmaskodnak, és az én népem így szereti! De mit cselekesznek majd utoljára?!

\chapter{6}

\par 1 Benjámin fiai, fussatok ki Jeruzsálembõl, és fújjatok kürtöt Thekoában, és tûzzetek ki zászlót Beth-Hakkeremben; mert veszedelem fenyeget észak felõl és nagy romlás!
\par 2 A szép és elkényeztetett asszonyhoz tettem hasonlóvá Sion leányát.
\par 3 Pásztorok jõnek el hozzá nyájaikkal együtt; felvonják mellette a sátrakat köröskörül; kiki legelteti, a mi keze ügyébe esik.
\par 4 Készüljetek hadba ellene; keljetek fel, és menjünk fel délben! Jaj nékünk, mert hanyatlik már a nap, mert hosszabbodnak az esteli árnyékok!
\par 5 Keljetek fel és menjünk fel éjjel, és rontsuk le az õ palotáit!
\par 6 Mert ezt mondja a Seregeknek Ura: Vágjatok fákat és hányjatok töltést Jeruzsálem ellen; a büntetés városa ez, csupa nyomorgatás van benne!
\par 7 Mint a kút hidegen tartja meg a vizét, úgy tartja meg az õ gonoszságát: erõszakosság és önkény hallatszik benne, és betegség és vereség van elõttem szüntelen.
\par 8 Térj eszedre, oh Jeruzsálem, hogy el ne szakadjon tõled a lelkem; hogy pusztává ne tegyelek téged, lakhatatlan földdé!
\par 9 Ezt mondja a Seregek Ura: Teljesen megszedik Izráel maradékát, mint a szõlõt. Fordítsd kezedet reájok, mint a szõlõszedõ a kosarakra!
\par 10 Kinek szóljak és kiket kérjek, hogy hallják? Ímé, az õ fülök körülmetéletlen és nem figyelhetnek! Ímé, az Úr szava útálatossággá lett elõttök; nem gyönyörködnek abban:
\par 11 Azért telve vagyok az Úr haragjával, elfáradtam azt visszatartani! Öntsd ki a gyermekekre az utczán, és az ifjak gyülekezetére is egyszersmind; sõt még a férj a feleséggel, az öreg az aggastyánnal szintén fogattassanak el;
\par 12 És házaik idegenekre szálljanak, mezõik és feleségeik is egyszersmind; mert kinyújtom kezemet e föld lakosaira, azt mondja az Úr.
\par 13 Mert kicsinyeiktõl fogva nagyjaikig mindnyájan telhetetlenségnek adták magokat; a prófétától fogva a papig mindnyájan csalárdságot ûznek.
\par 14 És hazugsággal gyógyítgatják az én népem leányának romlását, mondván: Békesség, békesség, és nincs békesség!
\par 15 Szégyenkezniök kellene, hogy útálatosságot cselekedtek, de szégyenkezni nem szégyenkeznek, még pirulni sem tudnak; ezért elesnek majd az elesendõkkel; az õ megfenyíttetésök idején elhullanak, azt mondja az Úr.
\par 16 Így szólt az Úr: Álljatok az utakra, és nézzetek szét, és kérdezõsködjetek és régi ösvények felõl, melyik a jó út, és azon járjatok, hogy nyugodalmat találjatok a ti lelketeknek! És azt mondták: Nem megyünk!
\par 17 Õrállókat is rendeltem föléjök, mondván: Figyeljetek a kürtnek szavára! És azt mondták: Nem figyelünk!
\par 18 Azért halljátok meg, ti nemzetek, és tudd meg, te gyülekezet azt, a mi következik reájok.
\par 19 Halld meg, oh föld! Ímé, én veszedelmet hozok erre a népre: az õ gondolatainak gyümölcsét; mert nem figyeltek az én beszédeimre, és az én törvényemet megvetették.
\par 20 Minek nékem ez a tömjén, a mi Sébából kerül, és a messze földrõl való jóillatú fahéj? A ti égõáldozataitok nincsenek kedvemre, sem a ti véres áldozataitok nem tetszenek nékem.
\par 21 Azért ezt mondja az Úr: Ímé, én akadályokat szerzek e népnek, és megbotlanak bennök az atyák és fiak együttesen, a szomszéd és az õ barátja elvesznek.
\par 22 Így szól az Úr: Ímé, nép jön el az északi földrõl, és nagy nemzet serken fel a földnek végérõl!
\par 23 Kézívet és kopját ragad, kegyetlen az és nem könyörül; szavok zúg, mint a tenger, és lovakon nyargalnak, fejenként viadalra készen te ellened, oh Sion leánya!
\par 24 Halljuk a hírét: kezeink elesnek; szorongás vesz erõt rajtunk, reszketés, mint a vajudó asszonyon.
\par 25 Ki ne menjetek a mezõre, és az úton se járjatok; mert ellenség fegyvere, rémület fenyeget köröskörül.
\par 26 Népem leánya! Ölts gyászt, és heverj a porban, sírj,  mint az egyszülöttet siratják, zokogj keservesen; mert reánk tör a pusztító hamar!
\par 27 Próbálóvá tettelek téged az én népem között; õrállóvá, hogy megismerd és megpróbáld az õ útjokat.
\par 28 Mindnyájan igen vakmerõk, rágalmazva járnak, réz és vas; mindnyájan elvetemültek õk.
\par 29 Megégett a fúvó a tûztõl, elfogyott az on, hiába olvaszt az olvasztó, mert a gonoszok, meg nem tisztíthatók.
\par 30 Megvetett ezüstnek hívjátok õket, mert az Úr megvetette õket!

\chapter{7}

\par 1 Az a beszéd, a melyet az Úr szóla Jeremiásnak, mondván:
\par 2 Állj az Úr házának ajtajába, és kiáltsd ott e beszédet, és mondjad: Halljátok az Úr beszédét mind, ti Júdabeliek, a kik bementek ezeken az ajtókon, hogy imádjátok az Urat!
\par 3 Így szól a Seregek Ura, Izráel Istene: Jobbítsátok meg a ti útaitokat és cselekedeteiteket, és veletek lakozom e helyen!
\par 4 Ne bízzatok hazug beszédekben, mondván: Az Úr temploma, az Úr temploma, az Úr temploma ez!
\par 5 Mert csak ha valóban megjobbítjátok a ti útaitokat és cselekedeteiteket; ha igazán ítéltek az ember között és felebarátja között;
\par 6 Ha jövevényt, árvát és özvegyet meg nem nyomorgattok, és ezen a helyen ártatlan vért ki nem ontotok, és idegen istenek után sem jártok a magatok veszedelmére:
\par 7 Akkor lakozom veletek ezen a helyen, a földön, a melyet a ti atyáitoknak adtam, öröktõl fogva mindörökké.
\par 8 Ímé, ti hisztek a hazug beszédeknek, haszon nélkül!
\par 9 Nemde loptok, öltök és paráználkodtok, hamisan esküsztök, a Baálnak áldoztok és idegen istenek után jártok, a kiket nem ismertek:
\par 10 És eljõtök, és megállotok elõttem e házban, a mely az én nevemrõl neveztetik, és ezt mondjátok: Megszabadultunk; hogy ugyanazokat az útálatosságokat cselekedhessétek!
\par 11 Vajjon latrok barlangjává lett-é ez a ház ti elõttetek, a mely az én nevemrõl neveztetik? Ímé, én is látok, azt mondja az Úr.
\par 12 Mert menjetek csak el az én helyemre, a mely Silóban van, a hol elõször lakoztam az én nevemmel, és lássátok meg, hogy mit cselekedtem azzal az én népemnek, Izráelnek gonoszságáért!
\par 13 Most pedig, mivelhogy mindezeket a cselekedeteket megcselekszitek, azt mondja az Úr, és mivelhogy szüntelen szóltam, és szóltam ti néktek, de nem hallottátok, és kiáltottam néktek, de nem feleltetek:
\par 14 Azért úgy cselekszem e házzal, a mely az én nevemrõl neveztetik, a melyben ti bizakodtok, és e hellyel, a melyet néktek és a ti atyáitoknak adtam, a mint Silóval  cselekedtem.
\par 15 És elvetlek titeket színem elõl, a mint elvetettem mind a ti atyátokfiait, Efraimnak minden magvát.
\par 16 És te ne imádkozzál e népért, se jajszót, se könyörgést ne emelj érettök, és nálam közben ne járj; mert én meg nem hallgatlak téged!
\par 17 Nem látod-é te: mit cselekesznek õk Júda városaiban és Jeruzsálem utczáin?
\par 18 A fiak fát szedegetnek, az atyák gyujtják a tüzet, az asszonyok pedig dagasztanak, hogy az ég királynõjének béleseket készítsenek, és az idegen isteneknek italáldozatokkal áldoznak, hogy engem felingereljenek.
\par 19 Avagy engem ingerelnek-é õk, azt mondja az Úr, és nem magokat-é, hogy gyalázat borítsa arczukat?
\par 20 Azért ezt mondja az Úr Isten: Ímé az én haragom és búsulásom kiömlik e helyre, az emberekre és a barmokra, a mezõnek fáira és a földnek gyümölcseire, és égni fog, és el nem aluszik.
\par 21 Ezt mondja a Seregek Ura, Izráel Istene: Égõáldozataitokat rakjátok a ti véres áldozataitokhoz, és egyetek húst!
\par 22 Mert nem szóltam a ti atyáitokkal, és nem rendelkeztem velök, a mikor kihoztam õket Égyiptom földérõl, az égõáldozat és véres áldozat felõl;
\par 23 Hanem ezekkel a szavakkal utasítottam õket, mondván: Hallgassatok az én szómra, és én  Istenetekké leszek, ti pedig az én népemmé lesztek, és mind csak azon az úton járjatok, a melyre utasítottalak titeket, hogy jól legyen dolgotok!
\par 24 De nem hallgattak, és fülöket sem hajtották felém, hanem az õ gonosz szívök fásultságában a magok tanácsa szerint jártak, és háttal valának felém, és nem arczczal.
\par 25 Attól a naptól fogva, a melyen kijöttek a ti atyáitok Égyiptom földébõl e mai napig küldtem hozzátok minden én szolgámat, a prófétákat napról-napra, szüntelen küldöttem;
\par 26 De nem hallgattak reám, és fülöket sem hajtották felém; hanem megkeményítették nyakukat, és gonoszabbul cselekedtek, mint az õ atyáik!
\par 27 Ha elmondod nékik mind e szavakat, és nem hallgatnak reád, ha kiáltasz nékik, és nem felelnek néked:
\par 28 Akkor ezt mondd nékik: Az a nép ez, a mely nem hallgat az Úrnak, az õ Istenének szavára, sem fenyítését fel nem veszi; elveszett a hûség; kiszaggattatott az õ szájokból.
\par 29 Nyírd le hajadat és vesd el, és kezdj a hegyeken gyászéneket, mert útálja az Úr és elhagyja a nemzetséget, a melyre haragszik.
\par 30 Mert gonoszt mûveltek elõttem Júdának fiai, azt mondja az Úr; útálatosságaikat bevitték abba a házba, a mely az én nevemrõl neveztetik, hogy megfertõztessék azt.
\par 31 És felépítették a Tófet magaslatait, a mely a Ben-Hinnom völgyében van, hogy megégessék fiaikat és leányaikat a tûzben, a mit nem parancsoltam, és a mi gondolatomban sem volt.
\par 32 Azért ímé eljõnek a napok, azt mondja az Úr, a mikor nem beszélnek többé a Tófetrõl, sem a Ben-Hinnom völgyérõl, hanem az öldöklés völgyérõl; és temetkezni fognak Tófetben, és hely sem lesz elég.
\par 33 És e nép holtteste az ég madarainak és a mezei barmoknak lesz eledelökké, és nem lesz, a ki elriassza azokat!
\par 34 És megszüntetem Júda városaiban és Jeruzsálem utczáin az örömnek szavát és a vígasságnak szavát, a võlegény szavát és a menyasszony szavát; mert elpusztul a föld!

\chapter{8}

\par 1 Abban az idõben, azt mondja az Úr, kihányják majd Júda királyainak csontjait és az õ fejedelmeinek csontait, a papok csontjait és a próféták csontjait és Jeruzsálem lakosainak csontjait az õ sírjaikból;
\par 2 És kiterítik azokat a napra és a holdra és az égnek minden serege elé, amelyeket szerettek, és a melyeknek szolgáltak, és a melyek után jártak, és a melyeket kerestek, és a melyek elõtt leborultak; nem szedetnek össze, el sem temettetnek, ganéjjá lesznek a föld színén!
\par 3 És inkább választja a halált, mint az életet az egész maradék, mindazok, a kik megmaradtak a gonosz nemzetségbõl, mindazokon a helyeken; a hol megmaradtak, a hová kiûztem õket; azt mondja a Seregek Ura!
\par 4 Ezt is mondjad nékik: Így szól az Úr: Úgy esnek-é el, hogy fel nem kelhetnek? Ha elfordulnak, nem fordulhatnak-é vissza?
\par 5 Miért fordult el ez a nép, Jeruzsálem népe, örök elfordulással? A csalárdságnak adták magokat, visszafordulni nem akarnak.
\par 6 Figyeltem és hallottam: nem igazán beszélnek, senki sincs, a ki megbánja az õ gonoszságát, ezt mondván: Mit cselekedtem? Mindnyájan az õ futó-pályájokra térnek, mint a harczra rohanó ló.
\par 7 Még az eszterág is tudja a maga rendelt idejét az égben, és a gerlicze, a fecske és daru is megtartják, hogy mikor kell elmenniök, de az én népem nem tudja az Úr ítéletét!
\par 8 Hogyan mondhatjátok: Bölcsek vagyunk, és az Úr törvénye nálunk van? Bizony ímé hazugságra munkál az írástudók hazug tolla!
\par 9 Megszégyenülnek a bölcsek, megrémülnek és megfogattatnak! Ímé, megvetették az Úr szavát; micsoda bölcsességök van tehát nékik?
\par 10 Azért az õ feleségeiket idegeneknek adom, mezõiket hódítóknak; mert kicsinytõl fogva nagyig mindnyájan nyereség után nyargalnak; a prófétától fogva a papig mindnyájan hamisságot ûznek.
\par 11 És hazugsággal gyógyítgatják az én népem leányának romlását, mondván: Békesség, békesség, és nincs békesség!
\par 12 Szégyenkezniök kellene, hogy útálatosságot cselekedtek; de szégyenkezni nem szégyenkeznek, még pirulni sem tudnak: ezért elesnek majd az elesendõkkel; az õ megfenyíttetésök idején elhullanak, azt mondja az Úr.
\par 13 Végképen véget vetek nékik, azt mondja az Úr! Nincs gerezd a szõlõtõkén, nincs füge a fügefán, a levele is elhervadt; azt teszem azért, hogy tovavigyék õket.
\par 14 Miért ülünk még? Gyûljetek össze, és menjünk be az erõsített városokba, és hallgassunk ott; mert az Úr, a mi Istenünk hallgatásra juttatott minket, és mérges vizet adott innunk; mert vétkeztünk az Úr ellen!
\par 15 Miért békességre várni, holott nincs semmi jó; gyógyító idõre, holott ímé itt van az ijedelem.
\par 16 Dántól fogva hallatszik az õ lovainak tüsszögése; méneinek nyerítõ hangjától reng az egész föld; és eljõnek és megemésztik e földet és mindenét, a mije van, a várost és annak lakóit.
\par 17 Mert ímé, én vipera-kígyókat bocsátok reátok, a melyek ellen nincsen varázslás, és megmarnak titeket, azt mondja az Úr!
\par 18 Megnyugodhatom-é a nyomorúság felett? Szívem eleped én bennem!
\par 19 Ímé, az én népem leányának kiáltó szava a messzi földrõl: Nincsen-é ott az Úr a Sionon? Nincsen-é azon az õ királya? Miért ingereltek fel engem az õ faragott bálványaikkal, az idegen semmiségekkel?
\par 20 Elmult az aratás, elvégzõdött a nyár, és mi nem szabadultunk meg!
\par 21 Az én népem leányának romlása miatt megromlottam, szomorkodom, iszonyat fogott el engem!
\par 22 Nincsen-é balzsamolaj Gileádban? Nincsen-é ott orvos? Miért nem gyógyíttatott meg az én népem leánya?

\chapter{9}

\par 1 Bárcsak a fejem vízzé változnék, a szemem pedig könyhullatásnak kútfejévé, hogy éjjel-nappal sirathatnám az én népem leányának megöltjeit!
\par 2 Bárcsak pusztába vinne engem valaki, utasok szállóhelyére, hogy elhagyhatnám az én népemet, és eltávozhatnám tõlök; mert mindnyájan paráznák, hitszegõk gyülekezete!
\par 3 Felvonták nyelvöket, mint kézívöket, hazugsággal és nem igazsággal hatalmasok e földön; mert gonoszságból gonoszságba futnak, engem pedig nem ismernek, azt mondja az Úr.
\par 4 Mindenki õrizkedjék a barátjától, és egyetlen atyátokfiának se higyjetek, mert minden atyafi cselbe csal, és minden barát rágalmazva jár.
\par 5 Egyik a másik ellen csúfoskodik, és nem szólnak igazat; nyelvöket hazug beszédre tanítják, elfáradtak a gonosztevésben.
\par 6 A te lakásod az álnokság közepében van; az álnokság miatt nem akarnak tudni felõlem, ezt mondja az Úr.
\par 7 Azért ezt mondja a Seregek Ura: Ímé, én megolvasztom õket, és megpróbálom õket, mert mit cselekedjem egyebet az én népem leánya miatt?
\par 8 Sebzõ nyíl a nyelvök, álnokságot beszél; szájával békességesen szól barátjához, a szívében pedig lest hány.
\par 9 Avagy ne fenyítsem-é meg õket ezekért? azt mondja az Úr; vajjon az efféle nemzetségen ne álljon-é bosszút a lelkem?
\par 10 A hegyeken sírást és zokogást támasztok, és a pusztai ligetekben gyászéneket; mert kiégnek úgy, hogy senki se megy keresztül rajtok, és nem hallják a nyájak bégetését; az ég madaraitól fogva a barmokig minden elköltözik és elmenekül.
\par 11 Jeruzsálemet pedig kõhalommá teszem, sakálok tanyájává, és Júda városait sivataggá változtatom, hogy lakatlan legyen.
\par 12 Kicsoda olyan bölcs férfiú, hogy értse ezeket, és a kihez az Úr szája szólt, hogy hirdesse azt, hogy miért vész el a föld, és ég ki, mint a puszta, a melyen senki se menjen keresztül?
\par 13 Monda továbbá az Úr: Mivelhogy elhagyták az én törvényemet, a melyet eléjök adtam, és nem hallgattak az én szómra, és nem jártak a szerint;
\par 14 Hanem jártak az õ kevély szívök után és a Baálok után, a mikre atyáik tanították õket:
\par 15 Azért, ezt mondja a Seregek Ura, Izráel Istene: Ímé, én megétetem õket, ezt a népet, ürömmel, és megitatom õket mérges vízzel.
\par 16 És szétszórom õket a nemzetek között, a melyeket nem ismertek sem õk, sem atyáik, és utánok küldöm a fegyvert, a míg megsemmisítem õket.
\par 17 Ezt mondja a Seregek Ura: Figyelmezzetek reá, és hívjátok a sirató asszonyokat, hogy jõjjenek el, és a bölcs asszonyokhoz is küldjetek, hogy jõjjenek el,
\par 18 És siessenek és fogjanak síráshoz miattunk, és a mi szemeink is hullassanak könyeket, és szempilláink vizet ömleszszenek!
\par 19 Mert a Sionról siralomnak szava hallatszik: Oh, hogy elpusztultunk! Igen megszégyenültünk, mert elhagyjuk e földet, mert széthányták lakhelyeinket!
\par 20 Bizony halljátok meg, ti asszonyok, az Úrnak szavát, és vegye be a ti fületek az õ szájának beszédét, és tanítsátok meg lányaitokat a sírásra, és egyik asszony a másikat a jajgatásra;
\par 21 Mert feljött a halál a mi ablakainkra, bejött a mi palotáinkba, hogy kipusztítsa a gyermekeket az útakról, az ifjakat az utczákról.
\par 22 Szólj! ezt mondja az Úr, és az emberek holtteste hever, mint a ganéj a mezõn, és mint a kéve az arató után, és nincs, a ki összegyûjtse.
\par 23 Ezt mondja az Úr: Ne dicsekedjék a bölcs az õ bölcseségével, az erõs se dicsekedjék az erejével, a gazdag se dicsekedjék  gazdagságával;
\par 24 Hanem azzal dicsekedjék, a ki dicsekedik, hogy értelmes és ismer engem, hogy én vagyok az Úr, a ki kegyelmet, ítéletet és igazságot gyakorlok e földön; mert ezekben telik kedvem, azt mondja az Úr.
\par 25 Ímé, eljõnek a napok, azt mondja az Úr, és megfenyítek minden körülmetélkedettet a körülmetéletlenekkel együtt:
\par 26 Égyiptomot és Júdát, Edomot és az Ammon fiait, Moábot és mindazokat, a kik nyírott üstökûek és a pusztában laknak; mert mindez a nemzet körülmetéletlen, és Izráelnek egész háza is körülmetéletlen  szívû.

\chapter{10}

\par 1 Halljátok meg a szót, a mit az Úr szól néktek, Izráel háza!
\par 2 Ezt mondja az Úr: A pogányok útját el ne tanuljátok, és az égi jelektõl ne féljetek, mert a pogányok félnek azoktól!
\par 3 Mert a népek bálványai csupa hiábavalóság, hiszen az erdõ fájából vágják azt; ács-mester kezei készítik bárddal.
\par 4 Ezüsttel és aranynyal megékesíti azt, szegekkel és põrölyökkel megerõsítik, hogy le ne essék.
\par 5 Olyanok, mint az egyenes pálmafa, és nem beszélnek; viszik-hordják õket, mert mozdulni nem tudnak. Ne féljetek tõlök, mert nem tehetnek rosszat; de jót tenni se képesek!
\par 6 Nincs hozzád hasonló, Uram! Nagy vagy és nagy a te neved a te hatalmadért!
\par 7 Ki ne félne tõled, nemzetek királya? Bizony tiéd a tisztelet, mert a nemzetek minden bölcse közt és azok minden országában sincs hozzád hasonló!
\par 8 Mind egyig balgatagok és bolondok; hiábavalóságokra tanít; fa az.
\par 9 Társisból hozott lapított ezüst és Ofirból való arany; az ácsnak és az ötvös kezének munkája; öltözetök kék és piros bíbor; mersterek munkája valahány.
\par 10 De az Úr igaz Isten, élõ Isten õ, és örökkévaló király; az õ haragja elõtt reszket  a föld, és a nemzetek nem szenvedhetik el az õ felindulását.
\par 11 (Mondjátok meg hát nékik: Az istenek, a kik az eget és földet nem alkották, el fognak veszni e földrõl és az ég alól!)
\par 12 Õ teremtette a földet az õ erejével, õ alkotta a világot az õ bölcseségével, és õ terjesztette ki az egeket az õ értelmével.
\par 13 Szavára víz-zúgás támad az égben, és felhõk emelkednek fel a föld határairól; villámlásokat készít az esõnek, és kihozza a szelet az õ rejtekhelyébõl.
\par 14 Minden ember bolonddá lett, tudomány nélkül, minden ötvös megszégyenül az õ öntött képével, mert hazugság az õ öntése, és nincsen azokban lélek.
\par 15 Hiábavalók azok, nevetségre való munka, elvesznek az õ megfenyíttetésök idején!
\par 16 Nem ilyen a Jákób része, mint ezek; mert a mindenség alkotója õ, és Izráel az õ örökségének pálczája; Seregek Ura az õ neve!
\par 17 Gyûjtsd össze a földrõl a te árúidat, a ki erõsített városban lakozol!
\par 18 Mert ezt mondja az Úr: Ímé, én elvetem ezúttal e föld lakosait, és megsanyargatom õket, hogy megtaláljanak.
\par 19 Jaj nékem az én romlásom miatt, gyógyíthatatlan az én sebem! De azt mondom mégis: Bizony ilyen az én vereségem, és szenvedem azt!
\par 20 Sátorom elpusztíttatott, köteleim mind elszakadoztak, fiaim elszakadtak tõlem és oda vannak õk; nincs többé, a ki kifeszítse sátoromat, és felvonja kárpitjaimat!
\par 21 Mert oktalanok voltak a pásztorok, és nem keresték az Urat; ezért nem lettek szerencsésekké, és minden nyájuk szétszóratott.
\par 22 A hír hangja ímé megjött, és nagy zúgás kél észak földe felõl, hogy pusztává tegyék Júdának városait, és sakálok  tanyájává.
\par 23 Tudom Uram, hogy az embernek nincs hatalmában az õ útja, és egyetlen járókelõ sem teheti, hogy irányozza a maga lépését!
\par 24 Fenyíts meg engem, Uram, de mértékkel, nem haragodban, hogy szét ne morzsolj engem!
\par 25 Öntsd ki haragodat ama nemzetekre, a melyek nem ismernek téged, és ama nemzetségekre, a melyek nem hívják segítségül a te nevedet; mert megették Jákóbot, bizony megették õt, és elemésztették õt, és lakóhelyét elpusztították!

\chapter{11}

\par 1 Az a beszéd, a melyet az Úr beszélt Jeremiásnak, mondván:
\par 2 Halljátok meg e szövetség igéit, és beszéljétek el Júda férfiainak és Jeruzsálem lakosainak!
\par 3 Ezt mondjad azért nékik: Így szól az Úr, Izráelnek Istene: Átkozott mindenki, a ki meg nem hallja e szövetségnek igéit,
\par 4 A melyet akkor parancsoltam a ti atyáitoknak, a mikor kihoztam õket Égyiptom földérõl, a vaskemenczébõl, mondván: Halljátok meg az én szómat, és cselekedjétek mindazokat, a miket én parancsolok néktek, és népemmé lesztek, és pedig Istenetekké leszek néktek;
\par 5 Hogy beteljesítsem az esküvést, a melylyel megesküdtem a ti atyáitoknak, hogy nékik adom a téjjel és mézzel folyó földet, a mint most van ez! És felelék, és mondám: Ámen, Uram!
\par 6 És monda az Úr nékem: Kiáltsd mindez igéket Júda városaiban és Jeruzsálem utczáin, mondván: Halljátok e szövetség igéit, és cselekedjétek azokat!
\par 7 Mert kérve kértem a ti atyáitokat, a mikor felhoztam õket Égyiptom földérõl, mind e napig, szünetlenül kérvén és mondván: Halljátok meg az én szómat!
\par 8 De nem hallották, még fülöket sem hajtották arra, hanem ment kiki az õ gonosz szívének hamissága után, és rájok szabtam e szövetségnek minden igéjét, a melyeket azért parancsoltam, hogy megcselekedjék, de nem cselekedték.
\par 9 És monda az Úr nékem: Pártütés van Júda a férfiai és Jeruzsálem lakosai között.
\par 10 Visszatértek az õ atyáiknak elébbi bûneire, a kik nem akarták hallani az én igéimet, és õk magok is idegen istenek után járnak, hogy azoknak szolgáljanak. Izráel háza és Júda háza megszegte az én szövetségemet, a melyet az õ atyáikkal kötöttem.
\par 11 Azért ezt mondja az Úr: Ímé, én veszedelmet hozok reájok, a melybõl ki nem menekülhetnek, és kiáltanak majd én hozzám, de nem hallgatom meg õket.
\par 12 És elmennek Júda városai és Jeruzsálem lakosai, és kiáltanak az istenekhez, a kiknek õk áldozni szoktak, de azok nem oltalmazzák meg õket az õ nyomorúságok idején.
\par 13 Mert városaidnak száma szerint voltak néked isteneid, oh Júda, és Jeruzsálem utczáinak száma szerint készítettétek a gyalázatnak oltárait, az oltárokat, hogy áldozzatok a Baálnak!
\par 14 Te azért ne esdekelj e népért, és egy kiáltó és esdeklõ szót se ejts érettök, mert én meg nem hallgatom õket, mikor kiáltanak majd hozzám az õ nyomorúságuk miatt.
\par 15 Mi köze az én kedveltemnek az én házamhoz? Temérdek istentelenséget cselekedtél; a szent húst abbahagytad; mikor gonoszságban vagy, akkor örvendezel.
\par 16 Lombos, szép formás gyümölcsû olajfa nevet adott néked az Úr. Nagy vihar morajánál tüzet gyújta rajta, és leromlottak az ágai.
\par 17 A Seregek Ura, a ki plántált téged, rosszat végzett felõled Izráel házának és Júda házának rosszasága miatt, a mit elkövettek magokban, hogy engemet haragra ingereljenek, áldozván a Baálnak.
\par 18 Az Úr tudtul adta nékem, és én tudtam; te láttattad meg velem az õ cselekedeteiket is!
\par 19 Én pedig olyan valék, mint a mészárszékre hurczolt szelíd bárány, és nem tudtam, hogy terveket szõttek ellenem, mondván: Pusztítsuk el e fát gyümölcsével együtt; irtsuk ki ezt az élõk földébõl, hogy még a nevét se emlegessék többé!
\par 20 És oh Seregek Ura, igaz biró, veséknek és szívnek vizsgálója: hadd lássam a te bosszúállásodat rajtok, mert néked jelentettem meg az én ügyemet!
\par 21 Azért ezt mondja az Úr az Anatóthbeli embereknek, a kik életedre törnek és ezt mondják: Ne  prófétálj az Úr nevében, hogy meg ne halj a mi kezünk által!
\par 22 Azért ezt mondja a Seregek Ura: Ímé, én megfenyítem õket; az ifjak fegyver által halnak meg, fiaik és leányaik pedig meghalnak éhen.
\par 23 És senki sem marad meg közülök, hogyha veszedelmet hozok az Anatóthbeli emberekre, az õ büntetésök esztendejét.

\chapter{12}

\par 1 Igaz vagy Uram, hogyha perlek is veled; éppen azért hadd beszélhessek veled peres kérdésekrõl! Miért szerencsés az istentelenek útja? Miért vannak békességben mindnyájan a hûtlenkedõk?
\par 2 Beplántálod õket, meg is gyökereznek; felnevekednek, gyümölcsöt is teremnek; közel vagy te az õ szájokhoz, de távol vagy az õ szívöktõl!
\par 3 Engem pedig ismersz te, Uram! látsz engem, és megvizsgáltad irántad való érzésemet: szakítsd külön õket, mint a mészárszékre való juhokat, és készítsd õket a  megölésnek napjára!
\par 4 Meddig gyászoljon a föld, és meddig száradjon el minden fû a mezõn? A benne lakók gonoszsága miatt pusztul el barom és  madár; mert azt mondják: Nem látja meg a mi végünket!
\par 5 Hogyha gyalogokkal futsz, és elfárasztanak téged: mimódon versenyezhetnél a lovakkal? És ha csak békességes földön vagy bátorságban: ugyan mit cselekednél a Jordán hullámai között?
\par 6 Bizony még a te atyádfiai és a te atyádnak háznépe is: õk is hûtlenül bántak veled; õk is tele torokkal kiabáltak utánad! Ne higyj nékik, még ha szépen beszélgetnek is veled!
\par 7 Elhagytam házamat; ellöktem örökségemet, ellenségének kezébe adtam azt, a kit lelkem szeret.
\par 8 Az én örökségem olyanná lett hozzám, mint az oroszlán az erdõben; ordítva támadt ellenem; ezért gyûlölöm õt.
\par 9 Tarka madár-é az én örökségem nékem? Nem gyûlnek-é ellene madarak mindenfelõl? Jõjjetek, seregeljetek össze mind ti mezei vadak; siessetek az evésre!
\par 10 Sok pásztor pusztította az én szõlõmet, taposta az én osztályrészemet; az én drága örökségemet sivatag pusztává tették!
\par 11 Pusztává tették, felém sír, mint puszta! Elpusztíttatik az egész föld, mert nincs senki, a ki eszére térjen.
\par 12 A pusztában lévõ minden magaslatra pusztítók érkeznek, mert az Úr fegyvere emészt a föld egyik szélétõl a föld másik széléig; senkinek sem lesz békessége.
\par 13 Búzát vetettek és tövist aratnak; fáradnak, de nem boldogulnak, és szégyent vallotok a ti jövedelmetekkel az Úr haragjának búsulása miatt.
\par 14 Ezt mondja az Úr minden én gonosz szomszédom felõl, a kik hozzányúlnak az én örökségemhez, a melyet örökségül adtam az én népemnek, Izráelnek: Ímé, én kigyomlálom õket az õ földükbõl, és Júda házát is kigyomlálom közülök!
\par 15 És azután, ha majd kigyomlálom õket, ismét könyörülök rajtok, és visszahozom õket, kit-kit az õ örökségébe, és kit-kit az õ földére.
\par 16 És ha megtanulják az én népemnek utait, és az én nevemre esküsznek ilyen módon: Él az Úr! a mint megtanították népemet megesküdni a Baálra: akkor felépülnek majd népem között.
\par 17 Ha pedig nem hallgatnak meg, akkor bizony kigyomlálom azt a népet, és elvesztem, azt mondja az Úr.

\chapter{13}

\par 1 Ezt mondá az Úr nékem: Menj el, és vásárolj magadnak lenövet, és illeszd azt a derekadra, de vízbe ne vidd be azt!
\par 2 És megvásároltam az övet, a mint az Úr rendelte vala, és derekamra illesztém.
\par 3 És másodszor szól vala az Úr hozzám, mondván:
\par 4 Vedd az övet, a melyet vásároltál, a mely a derekadon van, és kelj fel, és menj az Eufrateshez, és rejtsd el azt ott a kõszikla hasadékában.
\par 5 És elmenék, és elrejtém azt az Eufratesnél, a mint megparancsolta nékem az Úr.
\par 6 Sok nap mulva pedig újra monda nékem az Úr: Kelj fel, menj az Eufrateshez, és vedd el onnan az övet, a mely felõl parancsoltam, hogy ott rejtsd el.
\par 7 És elmenék az Eufrateshez, és kiásám, és kivevém arról a helyrõl az övet, a hová elrejtettem azt, és ímé, az öv rothadt vala, egészen hasznavehetetlen.
\par 8 És szóla az Úr nékem, mondván:
\par 9 Ezt mondja az Úr: Így rothasztom meg a Júda kevélységét, és a nagy Jeruzsálem kevélységét.
\par 10 Ez a gonosz nép, a mely nem akar hallgatni  az én beszédeimre, a mely a maga szívének hamisságában jár, és jár idegen istenek után, hogy azoknak szolgáljon, és azokat imádja: olyanná lesz majd, mint ez az öv, a mely egészen hasznavehetetlen.
\par 11 Mert a miként derekára kapcsolja a férfi az övet; akként kapcsoltam magamhoz Izráelnek egész házát és Júdának egész házát, azt mondja az Úr, hogy legyenek az én népemmé, az én nevemre, dicséretemre és tisztességemre, de nem engedelmeskedtek.
\par 12 Ilyen szókkal szólj azért nékik: Ezt mondja az Úr, Izráelnek Istene: Minden tömlõt meg kell tölteni borral! És azt mondják néked: Avagy nem tudjuk-é jól, hogy minden tömlõt borral kell megtölteni?
\par 13 Te pedig azt mondd nékik: Ezt mondja az Úr: Ímé betöltöm e földnek minden lakosát, és a királyokat, a kik Dávid trónján ülnek, és a papokat, a prófétákat, és Jeruzsálemnek minden lakosát részegséggel;
\par 14 És összeütöm õket, egyiket a másikhoz, az atyákat és a fiakat együtt, azt mondja az Úr; nem kegyelmezek meg, és nem kedvezek, és nem leszek irgalmas, hogy el ne veszítsem õket!
\par 15 Hallgassatok és figyelmezzetek; ne fuvalkodjatok fel, mert az Úr szólott!
\par 16 Dicsõítsétek az Urat, a ti Isteneteket, mielõtt setétséget szerezne, és mielõtt megütnétek lábaitokat a setét hegyekben; mert  világosságot vártok, és halál árnyékává változtatja azt, és sûrû homálylyá fordítja!
\par 17 Ha pedig nem hallgatjátok ezt: sírni fog az én lelkem a rejtekhelyeken a ti kevélységtek miatt, és zokogva zokog; a szemem pedig könyeket hullat, mert az Úr népe fogságba  vitetik.
\par 18 Mondd meg a királynak és a királynéasszonynak: Alázzátok meg magatokat, és üljetek veszteg, mert leesik fejetekrõl a ti dicsõségtek koronája!
\par 19 A dél felõl való városok bezároltatnak, és nem lesz, a ki megnyissa azokat; fogságra vitetik Júda egészen, fogságra vitetik mindenestül!
\par 20 Emeljétek fel szemeiteket, és nézzétek azokat, a kik északról jõnek! Hol van a nyáj, a mely néked adatott, a te dicsõségednek juhnyája?
\par 21 Mit mondasz, hogy ha megfenyít téged? Hiszen te oktattad õket magad ellen, fejedelmekké tetted fejeden! A fájdalmak nem környékeznek-é még téged, mint a szülõasszonyt?
\par 22 És ha azt mondod a te szívedben: Miért következnének ezek reám? A te hamisságod sokaságáért takartatik fel a te ruhád, és lesznek mezítelenekké a te sarkaid.
\par 23 Elváltoztathatja-é bõrét a szerecsen, és a párducz az õ foltosságát? Úgy ti is cselekedhettek jót, a kik megszoktátok a gonoszt.
\par 24 Azért szétszórom õket, a mint hányja-veti a pozdorját a pusztának szele.
\par 25 Ez a te sorsod, a te kimért részed én tõlem, ezt mondja az Úr, a ki elfelejtkeztél én rólam, és hittél a hazugságnak.
\par 26 Azért én is arczodra borítom fel a te ruhádat, hogy látható legyen a te gyalázatod!
\par 27 A te paráznaságaidat és nyihogásaidat bujálkodásodnak undokságát: a  halmokon, a mezõn láttam a te útálatosságaidat. Jaj néked Jeruzsálem! Nem  leszel tiszta ezután se? Meddig még?

\chapter{14}

\par 1 Az Úrnak szava, a mit Jeremiásnak szólott a szárazság felõl:
\par 2 Gyászol Júda, és kapui roskadoznak; szomorkodnak a földön, és Jeruzsálem kiáltása felszáll.
\par 3 Fejedelmeik is kiküldik gyermekeiket vízért: elmennek a kútakig, nem találnak vizet; visszatérnek üres edényekkel; szégyenkeznek és pironkodnak, és  befedik fejöket.
\par 4 A föld miatt, a mely retteg, mert nem esett esõ a földön, szégyenkeznek a szántóvetõk, és befedik fejöket.
\par 5 Még a szarvas-üszõ is megellik a mezõn, és ott hagyja fiát, mert nincsen fû.
\par 6 A vadszamarak pedig a sziklához állanak, levegõ után kapkodnak, mint a tengeri szörnyek, szemeik eltikkadnak, mert nincs fû.
\par 7 Ha bûneink ellenünk tanúskodnak: cselekedjél Uram a te nevedért, mert temérdek  a mi törvényszegésünk; vétkeztünk ellened!
\par 8 Izráel reménysége, megszabadítója a nyomorúság idején! Miért vagy e földön úgy, mint valami jövevény és mint valami utas, a ki éjjeli szállásra tér be?
\par 9 Miért vagy olyan, mint a megriasztott férfi; mint a vitéz, a ki nem tud segíteni? Hiszen te közöttünk vagy, Uram, és mi a te nevedrõl neveztetünk; ne hagyj el minket!
\par 10 Ezt mondja az Úr e népnek: Úgy szerettek ide-oda futkározni, lábaikat meg nem tartóztatták! Azért az Úr nem kedvelte õket. Most megemlékezik az  õ bûnökrõl, és vétkeikért megfenyíti õket.
\par 11 És mondá nékem az Úr: Ne könyörögj e népért, az õ javára.
\par 12 Mikor bõjtölnek, én meg nem hallgatom kiáltozásukat, és ha égõáldozatot vagy ételáldozatot készítenek, nem lesznek kedvesek elõttem; sõt fegyverrel, éhséggel és döghalállal irtom ki õket.
\par 13 És mondék: Ah, Uram Isten! Hiszen a próféták mondják vala nékik: Fegyvert nem láttok, éhség sem lesz rajtatok, sõt állandó békességet adok néktek ezen a helyen.
\par 14 És monda az Úr nékem: Hazugságot prófétálnak a próféták az én nevemben; nem küldtem õket, nem parancsoltam nékik, nem is beszéltem velök; hazug látomást, varázslást, hiábavalóságot és szívbeli csalárdságot jövendölnek néktek.
\par 15 Azért ezt mondja az Úr a próféták felõl, a kik az én nevemben prófétálnak, holott én nem küldtem õket és mégis azt mondják: fegyver és éhség nem lesz e földön: Fegyver és éhség miatt vesznek el azok a próféták!
\par 16 A nép pedig, a melynek õk prófétálnak, ott hever majd Jeruzsálem utczáin az éhség és a fegyver miatt, és nem lesz, a ki eltemesse õket, õket és feleségeiket, fiaikat és leányaikat; így zúdítom rájok gonoszságukat!
\par 17 Azért e szavakat mondjad nékik: Szemeim könyeket hullatnak éjjel és nappal, és nem szünnek meg; mert nagy rontással rontatott meg a szûz, az én népemnek leánya, igen fájdalmas vereséggel.
\par 18 Ha kimegyek a mezõre, hát ímé fegyver által levágottak; ha bemegyek a városba, hát ímé éhség miatt elepedtek vannak ott! Bizony próféta is, pap is olyan földre költöznek, a melyet nem ismernek!
\par 19 Egészen elvetetted-é Júdát, avagy a Siont útálja-é lelked? miért vertél úgy meg, hogy semmi orvosságunk se legyen? Békességet  vártunk, de nincs semmi jó; és gyógyulásnak idejét, de ímé, itt van a rettegés!
\par 20 Ismerjük Uram a mi gonoszságainkat, atyáink bûnét; bizony vétkeztünk ellened!
\par 21 Ne vesd meg a te nevedért; ne gyalázd meg a te dicsõségednek  székét! Emlékezzél; ne rontsd meg a te velünk való szövetségedet!
\par 22 Vannak-é a pogányok bálványai között, a kik esõt adhatnak? És ad-é záporokat az ég? Avagy nem te vagy-é a mi Urunk Istenünk, és nem benned kell-é bíznunk, hiszen te cselekedted mindezt!

\chapter{15}

\par 1 És monda az Úr nékem: Ha Mózes és  Sámuel állanának is elõttem, nem hajolna lelkem e néphez; küldd ki az orczám elõl, hadd menjenek!
\par 2 Ha pedig ezt mondják néked: Hová menjünk? ezt mondjad nékik: Így szól az Úr: A ki halálra való, halálra; a ki fegyverre való, fegyverre; a ki éhségre való, éhségre, és a ki fogságra való, fogságra.
\par 3 Mert négyfélével támadok reájok, ezt mondja az Úr: Fegyverrel, hogy gyilkoljon, kutyákkal, hogy tépjenek, az ég madaraival és a mezei vadakkal, hogy egyenek és pusztítsanak.
\par 4 Bújdosókká teszem õket e földnek minden országában Manasséért, Ezékiásnak, Júda királyának fiáért, azok miatt, a miket õ Jeruzsálemben cselekedett.
\par 5 Mert ki könyörül meg rajtad Jeruzsálem, és ki vígasztal meg téged, és ki mozdul meg, hogy kérdezze: jól vagy-é?
\par 6 Te elhagytál engem, azt mondja az Úr, másfelé jártál; azért kinyújtom kezemet ellened, és elvesztelek téged; belefáradtam a  szánakozásba!
\par 7 Elszórom õket szórólapáttal e földnek kapuiban; gyermektelenné teszem, elvesztem az én népemet; nem tértek vissza útaikról.
\par 8 Özvegyei számosabbak lesznek a tenger fövenyénél; pusztítót viszek reájok, az ifjúság anyjára délben; bocsátok reája nagy hirtelen észvesztést és rettentéseket.
\par 9 Elsenyved, a ki hét fiút szûl; kileheli lelkét; lehanyatlik az õ napja, mikor még nappal volna; megszégyenül és pironkodik; a maradékaikat pedig fegyverre vetem az õ ellenségeik elõtt, azt mondja az Úr!
\par 10 Jaj nékem, anyám, mert versengés férfiává és az egész föld ellen perlekedõ férfiúvá szültél engemet! Nem adtam kölcsönt és nékem sem adtak kölcsönt: mégis mindnyájan szidalmaznak engem!
\par 11 Monda az Úr: Avagy nem jóra tartalak-é meg téged? Avagy nem azt teszem-é, hogy ellenséged a baj idején és nyomorúság idején kérni fog téged?
\par 12 Vajjon eltöri-é a vas az északi vasat és rezet?
\par 13 Vagyonodat és kincseidet rablónak adom, nem pénzért, hanem a te mindenféle vétkedért, minden határodban.
\par 14 És elvitetlek ellenségeiddel olyan földre, a melyet nem ismersz, mert haragomnak tüze felgerjedt, lángra gyúlt ellenetek!
\par 15 Te tudod, Uram! Emlékezzél meg rólam és tekints reám, és állj bosszút értem üldözõimen; a te haragodnak halogatásaival ne ejts el engem; tudd meg, hogy éretted szenvedek gyalázatot!
\par 16 Ha szavaidat hallattad, én élveztem azokat; a te szavaid örömömre váltak nékem és szívemnek vígasságára; mert a te nevedrõl neveztetem oh Uram, Seregeknek Istene!
\par 17 Nem ültem a nevetgélõk gyülekezetében, és nem ujjongtam velök; a te hatalmad miatt egyedül ültem, mert bosszúsággal töltöttél el engem.
\par 18 Miért lett szünetlenné az én fájdalmam, és halálossá, gyógyíthatatlanná az én sebem? Olyanná lettél nékem, mint a bizonytalan vizû, csalóka patak!
\par 19 Azért ezt mondja az Úr: Ha megtérsz, én is visszatérítelek téged, elõttem állasz; és ha elválasztod a jót a hitványtól, olyanná leszel, mint az én szájam. Õk térjenek meg te hozzád, de te ne térj õ hozzájok!
\par 20 És e nép ellen erõs érczbástyává teszlek téged, és viaskodnak ellened, de nem gyõzhetnek meg téged, mert én veled vagyok, hogy megvédjelek és megszabadítsalak téged, azt mondja az Úr!
\par 21 És megszabadítlak téged a gonoszok kezeibõl, és kimentelek téged a hatalmaskodók markából.

\chapter{16}

\par 1 Majd szóla az Úr nékem, mondván:
\par 2 Ne végy magadnak feleséget, és ne legyenek néked fiaid és leányaid ezen a helyen!
\par 3 Mert ezt mondja az Úr a fiak felõl és leányok felõl, a kik ezen a helyen születnek, és anyjaik felõl, a kik szülik õket, és atyjaik felõl, a kik nemzették õket e földön:
\par 4 Keserves halállal halnak meg, nem sirattatnak el és el sem temettetnek; ganéjjá lesznek a föld színén, s fegyver és éhség miatt pusztulnak el, és az õ holttestök az ég madarainak és a mezei vadaknak lesznek eledelül.
\par 5 Mert ezt mondja az Úr: Ne menj be gyászoló házba, se sírni ne menj, se ne vígasztald õket; mert megvontam e néptõl az én békességemet, azt mondja az Úr: az irgalmasságot és kegyelmet.
\par 6 És meghalnak nagyok és kicsinyek e földön; el sem temetik, meg sem siratják õket, sem össze nem metélik magokat, sem hajokat ki nem tépik érettök.
\par 7 Kenyeret sem törnek nékik a gyászoláskor, hogy vígasztalják õket a meghaltért; a vígasztalás poharával sem itatják õket az õ atyjokért és anyjokért.
\par 8 A lakodalmas házba se menj be, hogy leülj velök enni és inni.
\par 9 Mert azt mondja a Seregek Ura, Izráelnek Istene: Ímé, én megszüntetem e helyen, a ti szemeitek elõtt, a ti napjaitokban a vigasságnak szavát és az örömnek szavát, a võlegénynek szavát és a menyasszonynak szavát.
\par 10 És hogyha tudtára adod e népnek mind e határozatokat, és ezt mondják néked: Miért határozta az Úr ellenünk mindezt a nagy gonoszt; és micsoda a mi bûnünk és micsoda a mi vétkünk, a melylyel vétkeztünk az Úr ellen, a mi Istenünk ellen?
\par 11 Akkor ezt mondd nékik: Azért, mert elhagytak engem a ti atyáitok, azt mondja az Úr, és idegen istenek után jártak, és azoknak szolgáltak és azokat imádták, engem pedig elhagytak, és az én törvényemet meg nem tartották.
\par 12 És ti gonoszabbul cselekedtetek, mint atyáitok; mert ímé, ti mindnyájan a ti gonosz szívetek hamisságát követitek, nem hallgatva reám.
\par 13 Azért kivetlek titeket e földbõl arra a földre, a melyet sem ti nem ismertek, sem a ti atyáitok, és ott szolgáltok majd idegen isteneknek nappal és éjjel; mivelhogy nem könyörülök rajtatok.
\par 14 Azért ímé, eljõnek a napok, ezt mondja az Úr, a mikor nem mondják többé; Él az Úr, a ki felhozta Izráel fiait Égyiptom földérõl;
\par 15 Hanem ezt: Él az Úr, a ki felhozta Izráel fiait északnak földérõl és mindama földekrõl, a melyekbe elszórta õket! Mert visszaviszem õket az  õ földjükre, a melyet az õ atyáiknak adtam.
\par 16 Ímé én sok halász után küldök, ezt mondja az Úr, hogy halászszák ki õket; azután pedig elküldök sok vadász után, hogy vadászszák ki õket minden hegybõl, minden halomból és a sziklák hasadékaiból is.
\par 17 Mert szemmel tartom minden útjokat; nem rejtõzhettek el orczám elõl, és nincsen elfedve az õ bûnök szemeim elõl.
\par 18 Elõbb azonban megfizetek nékik az õ bûneikért és vétkeikért kétszeresen, mert megszentségtelenítették az én földemet az õ útálatosságaiknak holttestével, és betöltötték az én örökségemet az õ fertelmességeikkel.
\par 19 Oh Uram, én erõsségem, én bástyám és én menedékem a nyomorúság idején! Hozzád jõnek majd a nemzetek a föld határairól, és ezt mondják: Bizony hamis isteneket bírtak a mi atyáink és hiábavalókat, mert nincs köztük segíteni tudó.
\par 20 Csinálhat-é az ember magának isteneket? Hiszen azok nem istenek!
\par 21 Azért ímé, megismertetem velök ez úttal, megismertetem velök az én kezemet és hatalmamat, és megtudják, hogy az Úr az én nevem.

\chapter{17}

\par 1 A Júda vétke vas tollal, gyémánt hegygyel van felírva; fel van vésve szívök táblájára és oltáraik szarvaira,
\par 2 Mivelhogy megemlékeznek fiaik az õ oltáraikról, Aséra bálványaikról a zöld fák mellett és a magas halmokon.
\par 3 Oh én hegyem a síkon! Vagyonodat, minden kincsedet prédává teszem a te magaslataidért, minden határodban való vétkeidért.
\par 4 És elszakíttatol, és pedig magad által, a te örökségedtõl, a melyet én adtam néked, és szolgáltatni fogom veled a te ellenségeidet olyan földön, a melyet nem ismersz; mert tûzre lobbantottátok haragomat, örökké égni fog.
\par 5 Ezt mondja az Úr: Átkozott az a férfi, a ki emberben bízik és testbe helyezi erejét, az Úrtól pedig eltávozott az õ szíve!
\par 6 Mert olyanná lesz, mint a hangafa a pusztában, és nem látja, hogy jó következik, hanem szárazságban lakik a sivatagban, a sovány és lakhatatlan földön.
\par 7 Áldott az a férfi, a ki az Úrban bízik, és a kinek bizodalma az Úr;
\par 8 Mert olyanná lesz, mint a víz mellé ültetett fa, a mely a folyó felé bocsátja gyökereit, és nem fél, ha hõség következik és a levele zöld marad; és a száraz esztendõben nem retteg, sem a gyümölcsözéstõl meg nem szûnik.
\par 9 Csalárdabb a szív mindennél, és gonosz az; kicsoda ismerhetné azt?
\par 10 Én, az Úr vagyok az, a ki a szívet fürkészem és a veséket vizsgálom, hogy megfizessek kinek-kinek az õ útai szerint és cselekedeteinek gyümölcse szerint.
\par 11 Mint a fogoly madár, mely fiakat gyûjt, melyeket nem õ költött, olyan, a ki gazdagságot gyûjt, de nem igazán; az õ napjainak felén elhagyja azt, a halálakor pedig bolonddá lesz.
\par 12 Óh dicsõség trónja, kezdettõl fogva magasságos, szentségünknek helye.
\par 13 Izráelnek reménység, oh Uram! A kik elhagynak téged, mind megszégyenülnek! A kik elpártolnak tõlem, a porba iratnak be, mert elhagyták az élõ vizeknek kútfejét, az Urat!
\par 14 Gyógyíts meg engem Uram, hogy meggyógyuljak, szabadíts meg engem, hogy megszabaduljak, mert te vagy az én dicsekvedésem!
\par 15 Ímé, õk azt mondják nékem: Hol van az Úr szózata? Most jõjjön el.
\par 16 De én nem siettem elhagyni a te útaidnak követését, sem gonosz napot nem kívántam, te tudod; a mi ajkaimon jött ki, nyilvánvaló volt elõtted.
\par 17 Ne légy nékem rettentésemre: reménységem vagy te a háborúság napján!
\par 18 Szégyenüljenek meg, a kik üldöznek engem, de ne én szégyenüljek meg; õk rettegjenek és ne én rettegjek; hozz reájok háborúság napját, és kétszeres zúzással zúzd össze õket!
\par 19 Ezt mondá nékem az Úr: Menj és állj meg a nép fiainak kapujában, a melyen bemennek és a melyen kijönnek Júdának királyai, és Jeruzsálemnek is minden kapujában!
\par 20 És ezt mondd nékik: Halljátok meg az Úrnak szavát Júdának királyai és egész Júda és Jeruzsálemnek minden lakosa, a kik bejártok e kapukon!
\par 21 Ezt mondja az Úr: Vigyázzatok a ti lelketekre, és ne hordjatok terhet szombat-napon, se Jeruzsálem kapuin be ne vigyetek!
\par 22 Házaitokból se vigyetek ki terhet szombat-napon, és semmi munkát ne végezzetek, hanem szenteljétek meg a szombat-napot, úgy a mint atyáitoknak megparancsoltam!
\par 23 De õk nem hallgattak, és fülöket sem hajtották rá, hanem megkeményítették nyakokat, hogy ne halljanak, és az oktatást be ne vehessék.
\par 24 Pedig ha szívesen hallgattok reám, ezt mondja az Úr, és nem visztek be terhet e város kapuin szombat-napon, és megszentelitek a szombat-napot, úgy hogy semmi dolgot nem végeztek azon:
\par 25 Akkor e város kapuin királyok és fejedelmek fognak bevonulni, a kik a Dávid székén ülnek, szekereken és lovakon járnak, mind magok, mind fejedelmeik, Júdának férfiai és Jeruzsálemnek lakosai, és e városban lakni fognak mindörökké.
\par 26 És bejõnek Júda városaiból, Jeruzsálem környékérõl, Benjámin földérõl, a lapályról, a hegyrõl és dél felõl, hozván égõáldozatot, véres áldozatot, ételáldozatot és temjént, és hozván hálaáldozatot az Úrnak házába.
\par 27 Ha pedig nem hallgattok reám, hogy megszenteljétek a szombat-napot, és hogy ne hordjatok terhet és ne jõjjetek be Jeruzsálem kapuin szombat-napon: tüzet gerjesztek az õ kapuiban, és megemészti Jeruzsálem palotáit, és nem lesz eloltható.

\chapter{18}

\par 1 Az a beszéd, a melyet az Úr beszélt Jeremiásnak, mondván:
\par 2 Kelj fel és menj le a fazekasnak házába, és ott közlöm veled az én beszédeimet!
\par 3 Lemenék azért a fazekas házába, és ímé õ edényt készít vala a korongon.
\par 4 És elromla az edény, a melyet õ készít vala és a mely mint agyag volt a fazekas kezében, és azonnal más edényt készíte belõle, a mint a fazekas jobbnak látta megkészíteni.
\par 5 És szóla az Úr nékem, mondván:
\par 6 Vajjon nem cselekedtem-é veletek úgy, mint ez a fazekas, oh Izráel háza? ezt mondja az Úr. Ímé, mint az agyag a fazekas kezében, olyanok vagytok ti az én kezemben, oh Izráel háza!
\par 7 Hogyha szólok egy nép ellen és ország ellen, hogy kigyomlálom, megrontom és elvesztem:
\par 8 De megtér az a nép az õ gonoszságából, a mely ellen szólottam: én is megbánom a gonoszt, a melyet rajta véghezvinni gondoltam.
\par 9 És hogyha szólok a nép felõl és ország felõl, hogy felépítem, beültetem;
\par 10 De a gonoszt cselekszi elõttem, és nem hallgat az én szómra: akkor megbánom a jót, a melylyel vele jót tenni akartam.
\par 11 Most azért beszélj csak a Júda férfiaival és Jeruzsálem lakosaival, mondván: Ezt mondja az Úr: Ímé, én veszedelmet készítek ellenetek és tervet tervezek ellenetek! Nosza, térjetek meg, kiki a maga gonosz útáról, és jobbítsátok meg útaitokat és cselekedeteiteket!
\par 12 Õk pedig azt mondják: Hagyd el! Mert mi a magunk gondolatai után megyünk, és mindnyájan a mi gonosz szívünk hamisságát cselekesszük.
\par 13 Azért ezt mondja az Úr: Kérdezzétek csak meg a népeket: kicsoda hallott vala ilyeneket? Igen útálatosan cselekedett Izráel leánya!
\par 14 Elhagyja-é a mezõség szikláját a Libanon hava? Vajjon kiszáríthatók-é a felfakadó, csörgedezõ, hullámzó vizek?
\par 15 Ám az én népem elfeledkezett rólam; a hiábavalónak áldozik; elcsábították õket az õ útaikról, az õsrégi nyomról, hogy ösvényeken, járatlan úton járjanak;
\par 16 Hogy pusztasággá tegyem földjüket, örökös csúfsággá, hogy a ki átmegy rajta, elálmélkodjék és fejét csóválja.
\par 17 Mint keleti szél szórom szét õket az ellenség elõtt, háttal és nem arczczal nézek reájok az õ pusztulásuk napján.
\par 18 Õk pedig mondák: Jertek és tervezzünk terveket Jeremiás ellen, mert nem vész el a törvény a paptól, sem a tanács a  bölcstõl, sem az ige a prófétától! Jertek el és verjük meg õt nyelvvel, és ne hallgassunk egy szavára sem!
\par 19 Figyelmezz reám Uram, és az én pereseimnek szavát is halld meg!
\par 20 Hát roszszal fizetnek-é a jóért, hogy õk vermet ásnak nékem? Emlékezzél! Elõtted álltam, hogy javokra beszéljek, hogy elfordítsam rólok haragodat.
\par 21 Azért juttasd fiaikat éhségre és hányd õket fegyver hegyére, hogy legyenek az õ asszonyaik magtalanokká és özvegyekké; férjeik pedig legyenek halál martalékává, ifjaikat fegyver verje le a harczon.
\par 22 Kiáltás hallattassék házaikból, mikor sereggel törsz reájok hirtelen, mert vermet ástak, hogy elfogjanak engem, és tõrt vetettek lábaimnak.
\par 23 Te pedig Uram, tudod minden ellenem való gyilkos szándékukat; ne kegyelmezz meg bûneik miatt, és ne töröld ki vétkeiket orczád elõl, hanem veszni valók legyenek elõtted; a te haragod idején bánj el velök!

\chapter{19}

\par 1 Ezt mondja az Úr: Menj el és végy egy cserépkorsót a fazekastól, és némelyekkel a nép vénei közül és a papok vénei közül.
\par 2 Menj el a Ben-Hinnom völgyébe, a mely a fazekasok kapujának bejáratánál van, és kiáltsd ott azokat a szavakat, a melyeket én szólok néked;
\par 3 És ezt mondjad: Halljátok meg az Úr szavát Júda királyai és Jeruzsálem lakosai: Ezt mondja a Seregek Ura, Izráel Istene: Ímé, én veszedelmet hozok e helyre, úgy hogy a ki csak hallja, megcsendül bele a füle.
\par 4 Azért, mert elhagytak engem, és idegenné tették e helyet, és idegen isteneknek áldoztak benne, a kiket sem õk nem ismertek, sem atyáik, sem Júda királyai, és eltöltötték e helyet  ártatlan vérrel;
\par 5 És magaslatokat építének a Baálnak, hogy megégessék fiaikat a tûzben, égõáldozatul a Baálnak, a mit én nem parancsoltam, sem nem rendeltem, és a mire nem is gondoltam.
\par 6 Azért ímé eljõnek a napok, azt mondja az Úr, és e hely nem neveztetik többé Tófetnek, sem Ben-Hinnom völgyének, hanem öldöklés völgyének.
\par 7 És eszét vesztem e helyen Júdának és Jeruzsálemnek, és fegyverrel ejtem el õket ellenségeik elõtt, és az életökre törõknek kezével; holttesteiket pedig az ég madarainak és a mezei vadaknak adom eledelül.
\par 8 És e várost csudává teszem és nevetséggé, aki csak átmegy rajta álmélkodik, és szörnyûködik az õ nagy romlásán.
\par 9 És megétetem velök az õ fiaik húsát és leányaik húsát és megeszi kiki az õ barátjának húsát, a megszállás alatt és a veszedelem alatt, a melylyel megszorongatják õket ellenségeik és a kik keresik az õ lelköket.
\par 10 Azután törd el a korsót azok szeme láttára, a kik elmennek veled.
\par 11 És ezt mondd nékik: Ezt mondja a Seregek Ura: Így töröm össze e népet és e várost, a mint összetörhetõ e cserépedény, a mely többé meg sem építhetõ; és Tófetben temettetnek el, mert nem lesz más hely a temetkezésre.
\par 12 Így cselekszem e helylyel és ennek lakosaival, azt mondja az Úr, és olyanná teszem e várost, a milyen Tófet.
\par 13 És Jeruzsálem házai és Júda királyainak házai undokokká lesznek, mint a Tófet helye; mindazok a házak, a melyeknek  tetején az ég egész seregének áldoztak és italáldozatot vittek az idegen isteneknek.
\par 14 Azután hazajöve Jeremiás Tófetbõl, a hová az Úr küldte vala õt prófétálni; megálla az Úr házának pitvarában, és szóla az egész népnek:
\par 15 Ezt mondja a Seregek Ura, Izráel Istene: Ímé én ráhozom e városra és ennek minden városára mindama veszedelmet, a melyrõl szóltam õ ellene; mert megkeményítették nyakukat, hogy ne hallják az én beszédeimet.

\chapter{20}

\par 1 És hallá Passúr a pap, az Immár fia (õ pedig fejedelem vala az Úr házában), Jeremiást, a mint e szókat prófétálja vala:
\par 2 És megcsapdosá Passúr Jeremiást a prófétát, és beveté õt a tömlöczbe, a mely a Benjámin felsõ kapujában vala, az Úr háza mellett.
\par 3 És lõn másnap, hogy kivevé Passúr Jeremiást a tömlöczbõl, és monda néki Jeremiás: Nem Passúrnak nevezett téged az Úr, hanem Mágor Missábibnak;
\par 4 Mert ezt mondja az Úr: Ímé, én félelembe ejtelek téged és minden barátodat, és elhullanak az õ ellenségeik fegyvere által a szemeid láttára; az egész Júdát pedig odaadom a babiloni király kezébe, és elviszi õket Babilonba, és fegyverrel vágja le õket:
\par 5 És odaadom e városnak minden vagyonát és minden keresményét, és minden drágaságát, és Júda királyainak minden kincsét odaadom az õ ellenségeik kezébe, és elrabolják, elhurczolják és Babilonba viszik azokat.
\par 6 Te pedig Passúr és a te házadnak minden lakosa, rabságba mentek, és Babilonba jutsz és ott halsz meg és ott temettetel el, te és minden barátod, a kiknek hamisan prófétáltál.
\par 7 Rávettél Uram engem és rávétettem, megragadtál engem és legyõztél! Nevetségessé lettem minden idõre, mindenki csúfol engemet;
\par 8 Mert a hányszor csak szólok, kiáltozom, így kiáltok: erõszak és romlás! Mert az Úr szava mindenkori gyalázatomra és csúfságomra lett nékem.
\par 9 Azért azt mondom: Nem emlékezem róla, sem az õ nevében többé nem szólok; de mintha égõ tûz volna szívemben, az én csontjaimba rekesztetve, és erõlködöm, hogy elviseljem azt, de nem tehetem.
\par 10 Mert hallom sokak rágalmazását, a mindenfelõl való fenyegetést: Jelentsétek fel és ezt mi is feljelentjük: mindazok is, a kik barátaim, az én tántorodásomra figyelmeznek, mondván: Talán megbotlik és megfoghatjuk õt, és bosszút állhatunk rajta;
\par 11 De az Úr velem van, mint hatalmas hõs, azért elesnek az én kergetõim és nem bírnak velem, igen megszégyenülnek, mert nem okosan cselekesznek: örökkévaló  gyalázat lesz rajtok és felejthetetlen.
\par 12 Azért, oh Seregeknek Ura, a ki megpróbálod az igazat, látod a veséket és a szíveket, hadd lássam a te büntetésedet õ rajtok: mert néked jelentettem meg az én ügyemet!
\par 13 Énekeljetek az Úrnak, dicsérjétek az Urat, mert a szegénynek lelkét megszabadítja a gonoszok kezébõl.
\par 14 Átkozott az a nap, a melyen születtem; az a nap, a melyen anyám szûlt engem, ne legyen áldott!
\par 15 Átkozott ember az, a ki örömhírt vitt az én atyámnak, mondván: Fiúmagzatod született néked, igen megörvendeztetvén õt.
\par 16 És legyen az az ember olyan, mint azok a városok, a melyeket elvesztett az Úr és meg nem bánta; és halljon reggel kiáltozást, és harczi riadót délben.
\par 17 Hogy nem ölt meg engem az én anyám méhében, hogy az én anyám nékem koporsóm lett volna, és méhe soha sem szûlt volna!
\par 18 Miért is jöttem ki az én anyámnak méhébõl, hogy nyomorúságot lássak és bánatot, és hogy napjaim gyalázatban végzõdjenek?

\chapter{21}

\par 1 Ez a beszéd, a melyet szóla az Úr Jeremiásnak, mikor elküldé hozzá Sedékiás király Passúrt Melkiásnak fiát, és Sofóniást a Maásiás pap fiát, mondván:
\par 2 Kérdezd meg most érettünk az Urat, mert Nabukodonozor, a babiloni király viaskodik ellenünk, ha cselekszik-é az Úr velünk minden õ csodái szerint, hogy elhagyjon minket?
\par 3 És monda nékik Jeremiás: Ezt mondjátok Sedékiásnak.
\par 4 Így szól az Úr Izráel Istene: Ímé, én elfordítok minden hadi szerszámot, a melyek a ti kezeitekben vannak, a melyekkel ti a babiloni király ellen és a Kaldeusok ellen viaskodtok, a kik kivül a kõfalon ostromolnak titeket, és begyûjtöm õket e városnak közepébe;
\par 5 És én kinyujtott kézzel vívok ellenetek és nagy erõs karral és haraggal, búsulással és nagy felindulással.
\par 6 És megverem e városnak lakosait, mind az embert, mind a barmot; nagy döghalállal halnak meg.
\par 7 És azután, azt mondja az Úr, Sedékiást a Júda királyát, és az õ szolgáit, és a népet, és a kik megmaradnak e városban a döghaláltól, a fegyvertõl és az éhségtõl: odaadom Nabukodonozornak, a babiloni királynak kezébe és az õ ellenségeiknek kezébe és azoknak kezébe, a kik keresik az õ lelköket, és megöli õket éles fegyverrel: nem kedvez nékik, nem enged és nem könyörül rajtok.
\par 8 Azután ezt mondjad e népnek: Ezt mondja az Úr: Ímé, én elõtökbe adom néktek az élet útját és a halál útját.
\par 9 A ki e városban lakik, fegyver, éhség és döghalál miatt kell meghalnia; a ki pedig kimegy belõle és a Kaldeusokhoz megy, a kik megostromolnak titeket, él, és az õ lelkét zsákmányul nyeri;
\par 10 Mert orczámat e város veszedelmére fordítottam és nem megszabadulására, azt mondja az Úr: A babiloni király kezébe adatik, és tûzzel égeti meg azt!
\par 11 Júda királya házának mondd meg: Halljátok meg az Úr szavát!
\par 12 Dávidnak háza, ezt mondja az Úr: Hamarsággal tegyetek igaz ítéletet, a nyomorultat mentsétek meg a nyomorgatónak kezébõl, különben az én haragom kitör, mint a tûz és felgerjed, és nem lesz, a ki megolthassa, az õ cselekedeteiknek gonoszsága miatt.
\par 13 Ímé, én reátok megyek, te völgy lakója és síkságnak szirtje; azt mondja az Úr, a kik azt mondjátok: Kicsoda jön le mi ellenünk, és kicsoda jön be a mi házainkba?
\par 14 És a ti cselekedeteiteknek gyümölcse szerint fenyítlek meg titeket, azt mondja az Úr, és tüzet gyújtok az õ erdejében, és köröskörül az mindent megemészt!

\chapter{22}

\par 1 Ezt mondja az Úr: Menj alá a Júda királyának házába, és beszéld el ezeket:
\par 2 És ezt mondd: Halld meg az Úr szavát, Júda királya, a ki a Dávid királyi székében ülsz, te és a te szolgáid, és a te néped, a kik bejártok e kapukon!
\par 3 Ezt mondja az Úr: Tegyetek ítéletet és igazságot, és mentsétek meg a nyomorultat a nyomorgató kezébõl! A jövevényt, árvát és özvegyet pedig ne nyomorgassátok és rajta ne erõszakoskodjatok, és ártatlan vért e helyen ki ne ontsatok.
\par 4 Mert ha ezt cselekszitek, akkor királyok mennek be e háznak kapuin, a kik a Dávid székébe ülnek, szekereken és lovakon menvén õ, az õ szolgái és az õ népe.
\par 5 Ha pedig nem hallgattok e szókra, én magamra esküszöm, azt mondja az Úr, hogy elpusztul e ház.
\par 6 Mert így szól az Úr a Júda királyának házához: Ha Gileád volnál nékem és a Libánon feje: mégis elpusztítlak téged, mint a városokat, a melyekben nem laknak.
\par 7 És felkészítem ellened a rablókat, mindeniket az õ fegyverével, és kivágják a te válogatott czédrusaidat, és a tûzre vetik.
\par 8 És sok nép megy át e városon, és ezt mondják egymásnak: Miért mívelte ez Úr ezt e nagy várossal?
\par 9 És ezt mondják: Azért, mert elhagyták az Úrnak, az õ Istenöknek frigyét, és idegen istenek elõtt borultak le és azoknak szolgáltak.
\par 10 Ne sirassátok a halottat és ne bánkódjatok érte, hanem azt sirassátok, a ki elment, mert nem jõ vissza többé, és az õ szülõföldjét nem látja meg.
\par 11 Mert ezt mondja az Úr Sallum felõl, Jósiásnak, a Júda királyának fia felõl, a ki uralkodik az õ atyja, Jósiás helyett: A ki kimegy e helybõl, nem tér többé ide vissza.
\par 12 Hanem a helyen, a hová rabságra vitték, ott hal meg, és e földet nem látja többé.
\par 13 Jaj annak, a ki hamisan építi házát, felházait pedig álnokul; a ki az õ felebarátjával ingyen szolgáltat, és munkájának bérét néki meg nem adja.
\par 14 A ki ezt mondja: Nagy házat építek magamnak és tágas felházakat, és ablakait kiszélesíti és czédrusfával béleli meg és megfesti czinóberrel.
\par 15 Király vagy-é azért, hogy czédrus után kivánkozol? A te atyád nem evett és ivott-é? De igazságot és méltányosságot cselekedett, azért jó dolga volt.
\par 16 Igazságosan ítélte a szegényt és a nyomorultat, azért jó volt dolga. Nem ez-é az igaz ismeret felõlem? azt mondja az Úr!
\par 17 De a te szemeid és szíved csak a te nyereségedre vágynak, és az ártatlan vérére, és ragadozásra és erõszak elkövetésére.
\par 18 Azért ezt mondja az Úr Joákimnak, Jósiás, Júda királya fiának: Nem siratják õt: Jaj atyám! Jaj húgom! nem siratják õt: Jaj uram! Jaj az õ dicsõségének.
\par 19 Úgy temetik el, mint a szamarat, kivonják és elvetik Jeruzsálem kapuin kivül!
\par 20 Menj fel a Libánonra és kiálts, és a Básánon emeld fel szódat és kiálts az Abarimról, mert szeretõid mind tönkre jutottak.
\par 21 Szóltam néked, mikor jól volt dolgod, de ezt mondottad: Nem hallom. Ifjúságodtól fogva ez a te szokásod, hogy nem hallgattad az én szómat!
\par 22 Minden pásztorodat szél emészti meg, és a szeretõid rabságra mennek; akkor szégyent vallasz majd és pironkodol minden gonoszságodért.
\par 23 Te, a ki a Libánonon lakozol, a czédrusfákon fészkelsz: hogy fogsz majd nyögni, mikor fájdalmak lepnek meg, a gyermekszülõ vajudása?
\par 24 Élek én, azt mondja az Úr, hogy ha Kóniás; Joákimnak, a Júda királyának fia pecsétgyûrû  volna is az én jobbkezemben: mindazáltal onnan lerántanálak.
\par 25 És odaadlak téged a te lelked keresõinek kezébe, és azoknak kezébe, a kiknek tekintetétõl félsz, és Nabukodonozornak, a babiloni királynak kezébe és a Káldeusoknak kezébe.
\par 26 És elvetlek téged és a te anyádat, a ki szûlt téged, idegen földre, a melyben nem születtetek, és ott haltok meg.
\par 27 És nem jõnek vissza arra a földre, a melyre az õ lelkök visszajõni kivánkozik.
\par 28 Avagy útálatos és elromlott edény-é ez a férfiú, ez a Kóniás, avagy oly edény-é, amelyben semmi gyönyörûség nincsen? Miért lökték el õt és az õ magvát, és dobták olyan földre, a melyet õk nem ismernek?
\par 29 Föld, föld, föld! halld meg az Úrnak szavát!
\par 30 Ezt mondja az Úr: Írjátok fel ezt a férfiút, mint gyermektelent, mint olyan embert, a kinek nem lesz jó elõmenetele az õ idejében; mert senkinek, a ki az õ magvából a Dávid székében ül, nem lesz jó állapotja, és nem uralkodik többé Júdában.

\chapter{23}

\par 1 Jaj a pásztoroknak, a kik elvesztik és elszélesztik az én mezõmnek juhait, azt mondja az Úr.
\par 2 Azért ezt mondja az Úr, Izráel Istene a pásztoroknak, a kik legeltetik az én népemet: Ti szélesztettétek el az én juhaimat és ûztétek el õket; és nem néztetek utánok; ímé, én megbüntetem a ti cselekedeteiteknek gonoszságát, azt mondja az Úr.
\par 3 Juhaimnak maradékát pedig összegyûjtöm minden földrõl, a melyekre elûztem õket, és visszahozom õket az õ legelõikre, és szaporodnak és megsokasodnak.
\par 4 És pásztorokat rendelek melléjök, hogy legeltessék õket, és többé nem félnek és nem rettegnek, sem meg nem fogyatkoznak, azt mondja az Úr.
\par 5 Ímé, eljõnek a napok, azt mondja az Úr, és támasztok Dávidnak igaz magvat, és uralkodik mint király, és bölcsen cselekszik és méltányosságot és igazságot cselekszik e földön.
\par 6 Az õ idejében megszabadul Júda, és Izráel bátorságosan lakozik, és ez lesz az õ neve, a melylyel nevezik õt: Az Úr a mi igazságunk!
\par 7 Azért ímé elközelgetnek a napok, azt mondja az Úr, a melyekben nem mondják többé: Él az Úr, a ki kihozta Izráel fiait Égyiptom földébõl.
\par 8 Hanem inkább ezt mondják: Él az Úr, a ki kihozta és a ki haza vezérlette Izráel házának magvát az északi földrõl és mindama földekrõl, a melyekre kiûztem vala õket, és lakoznak az õ földjökön.
\par 9 A próféták miatt megkeseredett az én szívem én bennem, minden csontom reszket; olyan vagyok, mint a részeg férfi és mint a bortól elázott ember, az Úrért és az õ szent igéjéért.
\par 10 Mert betelt a föld paráznákkal, mert a hamis esküvés miatt gyászol a föld, a pusztának legelõje kiszáradt, és az õ futásuk gonosz, és az õ hatalmassuk hamis.
\par 11 Mert mind a próféta, mind a pap istentelenek, még házamban is megtaláltam az õ gonoszságukat, azt mondja az Úr.
\par 12 Azért az õ útjok olyan lesz, mint a sikamlós útak a setétben, megbotlanak és elesnek: mert veszedelmet hozok reájok, az õ büntetésöknek esztendejét, azt mondja az Úr.
\par 13 A Samariabeli prófétákban is bolondságot láttam: A Baál nevében prófétáltak, és elcsalták az én népemet, az Izráelt.
\par 14 De a jeruzsálemi prófétákban is rútságot láttam: paráználkodnak és hazugságban járnak; sõt pártját fogják a gonoszoknak, annyira; hogy senki sem tér meg az õ gonoszságából; olyanok elõttem mindnyájan, mint Sodoma, és a benne lakók, mint Gomora.
\par 15 Azért ezt mondja a Seregek Ura a próféták felõl: Ímé, én ürmöt adok enniök és mérget adok inniok, mert a jeruzsálemi prófétáktól ment ki az istentelenség minden földre.
\par 16 Ezt mondja a Seregek Ura: Ne hallgassátok azoknak a prófétáknak szavait, a kik néktek prófétálnak, elbolondítanak titeket: az õ szívöknek látását szólják, nem az Úr szájából valót.
\par 17 Szüntelen ezt mondják azoknak, a kik megvetnek engem: Azt mondta az Úr: Békességetek lesz néktek és mindenkinek, a ki az õ szívének keménysége szerint jár; ezt mondák: Nem jõ ti reátok veszedelem!
\par 18 Mert ki állott az Úr tanácsában, és ki látta és hallotta az õ igéjét? Ki figyelmezett az õ igéjére és hallotta azt?
\par 19 Ímé, az Úrnak szélvésze nagy haraggal kitör, és a hitetlenek fejére forgószél zúdul.
\par 20 Nem szünik meg az Úrnak haragja, míg meg nem valósítja és míg be nem teljesíti szívének gondolatait; az utolsó napokban értitek meg e dolog értelmét.
\par 21 Nem küldöttem e prófétákat, de õk futottak, nem szólottam nékik, mégis prófétáltak.
\par 22 Ha tanácsomban állottak volna, akkor az én igéimet hirdették volna az én népemnek, és eltérítették volna õket az õ gonosz útaikról, és az õ cselekedetöknek gonoszságától.
\par 23 Csak a közelben vagyok-é én Isten? azt mondja az Úr, és nem vagyok-é Isten a messzeségben is?
\par 24 Vajjon elrejtõzhetik-é valaki a rejtekhelyeken, hogy én ne lássam õt? azt mondja az Úr, vajjon nem töltöm-é én be a mennyet és a földet? azt mondja az Úr.
\par 25 Hallottam, a mit a hazug próféták mondanak, a kik hazugságot prófétálnak az én nevemben, mondván: Álmot láttam, álmot láttam.
\par 26 Meddig lesz ez a hazugságot prófétáló próféták szívében, a kik az õ szívök csalárdságát prófétálják?
\par 27 A kik el akarják az én népemmel felejtetni az én nevemet az õ álmaikkal, a melyeket egymásnak beszélnek, miképen az õ atyáik elfeledkeztek az én nevemrõl a Baálért?
\par 28 A próféta, a ki álomlátó, beszéljen álmot; a kinél pedig az én igém van, beszélje az én igémet igazán. Mi köze van a polyvának a búzával? azt mondja az Úr.
\par 29 Nem olyan-é az én igém, mint a tûz? azt mondja az Úr, mint a sziklazúzó põröly?
\par 30 Azért ímé én a prófétákra támadok, azt mondja az Úr, a kik az én beszédeimet ellopják egyik a másikától.
\par 31 Ímé, én a prófétákra támadok, azt mondja az Úr, a kik felemelik nyelvöket és azt mondják: az Úr mondja!
\par 32 Ímé, én a prófétákra támadok, a kik hazug álmokat prófétálnak, azt mondja az Úr, és beszélik azokat, és megcsalják az én népemet az õ hazugságaikkal és hízelkedéseikkel, holott én nem küldtem õket, sem nem parancsoltam nékik, és használni sem használtak e népnek, azt mondja az Úr.
\par 33 Mikor pedig megkérdez téged e nép, vagy a próféta, vagy a pap, mondván: Micsoda az Úrnak terhe? akkor mondd meg nékik azt: mi a teher? Az, hogy elvetlek titeket, azt mondja az Úr.
\par 34 A mely próféta vagy pap, vagy község azt mondja: Ez az Úrnak terhe, meglátogatom azt az embert és annak házát.
\par 35 Kiki ezt mondja az õ barátjának és kiki az õ atyjafiának: Mit felel az Úr? és mit szólt az Úr?
\par 36 És az Úrnak terhét ne emlegessétek többé, mert mindenkinek terhes lesz az õ szava, ha elforgatjátok az élõ Istennek, a Seregek Urának, a mi Istenünknek, beszédét.
\par 37 Ezt mondjad a prófétának: Mit felelt néked az Úr és mit szólt az Úr?
\par 38 Hogyha az Úrnak terhét említitek, tehát ezt mondja az Úr: Mivelhogy ti e szót mondottátok: ez az Úrnak terhe, jóllehet küldék ti hozzátok, a kik ezt mondják: Ne mondjátok: ez az Úrnak terhe;
\par 39 Ezért ímé én elfeledlek titeket, és kigyomlállak titeket, és a várost, a melyet néktek és a ti atyáitoknak adtam, elvetem az én orczám elõl.
\par 40 És örökkévaló szégyent és örökkévaló gyalázatot hozok ti reátok, a mely felejthetetlen.

\chapter{24}

\par 1 Látomást mutata nékem az Úr, és ímé, két kosár füge volt letéve az Úr temploma elõtt, miután Nabukodonozor babiloniai király Jékóniást, Joákimnak, a Júda királyának fiát és Júdának fejedelmeit és az ácsokat és kovácsokat fogságra hurczolá Jeruzsálembõl, és elvivé õket Babilonba.
\par 2 Az egyik kosárban igen jó fügék valának, a milyenek az elõször érõ fügék; a másik kosárban pedig igen rossz fügék valának, sõt ehetetlenek a rosszaság miatt.
\par 3 És monda az Úr nékem: Mit látsz te Jeremiás? És mondék: Fügéket. A jó fügék igen jók, de a rosszak igen rosszak, sõt rosszaságok miatt ehetetlenek.
\par 4 És szóla az Úr nékem, mondván:
\par 5 Ezt mondja az Úr, Izráel Istene: Mint ezeket a jó fügéket, úgy megkülönböztetem a Júda foglyait, a kiket e helyrõl a Kaldeusok földjére vitettem, az õ javokra.
\par 6 És õket szemmel tartom az õ javokra, és visszahozom e földre, és megépítem és el nem rontom, és beplántálom és ki nem szaggatom.
\par 7 És szívet adok nékik, hogy megismerjenek engemet, hogy én vagyok az Úr, és õk  én népemmé lesznek, én pedig Istenökké leszek, mert teljes szívökbõl megtérnek hozzám.
\par 8 És a milyenek a rossz függék, a melyek ehetetlenek a rosszaság miatt, azt mondja az Úr, olyanná teszem Sedékiást, a Júda királyát, és az õ maradékát, és az õ fejedelmeit, és Jeruzsálem maradékát, a kik itt maradnak e földön, és azokat, a kik Égyiptom földén laknak.
\par 9 És kiteszem õket rettegésnek, veszedelemnek a föld minden országában; gyalázatnak, példabeszédnek, gúnynak és szidalomnak minden helyen, a hová kiûzöm õket.
\par 10 És fegyvert, éhséget és döghalált bocsátok reájok mindaddig, a míg elfogynak a földrõl, a melyet nékik adtam és az õ atyáiknak.

\chapter{25}

\par 1 Az a beszéd, a mely lõn Jeremiáshoz az egész Júda népe felõl, Joákim negyedik esztendejében; a ki fia vala Jósiásnak, a Júda királyának, az elsõ esztendejében Nabukodonozornak a babiloni királynak;
\par 2 A melyet szóla Jeremiás próféta az egész Júda népéhez és Jeruzsálem minden lakosához, mondván:
\par 3 Jósiásnak tizenharmadik esztendejétõl fogva, a ki fia vala Amonnak, a Júda királyának, e napig (vagyis huszonhárom esztendõ óta) szóla az Úr nékem, és szólottam én néktek, jó reggelt szóltam, de nem hallgattátok.
\par 4 És elküldte az Úr ti hozzátok minden õ szolgáját, a prófétákat, jó reggel elküldte, de nem hallgattátok, és fületeket sem hajtottátok a hallásra.
\par 5 Ezt mondták: Térjetek meg már mindnyájan a ti gonosz útaitokról és a ti gonosz cselekedeteitekbõl, hogy lakhassatok a földön, a melyet az Úr adott néktek és a ti atyáitoknak öröktõl fogva örökké.
\par 6 És ne járjatok idegen istenek után, hogy szolgáljatok nékik és imádjátok õket, és ne ingereljetek fel engem a ti kezeitek munkájával, hogy veszedelmet ne hozzak rátok.
\par 7 De nem hallgattatok reám, azt mondja az Úr, hogy felingereljetek engem a ti kezeitek munkájával a ti veszedelmetekre.
\par 8 Azért ezt mondja a Seregeknek Ura: Mivelhogy nem hallgattatok az én beszédemre:
\par 9 Ímé, kiküldök én és felveszem északnak minden nemzetségét, azt mondja az Úr, és Nabukodonozort, a babiloni királyt, az én szolgámat, és behozom õket e földre és ennek lakóira és mind e körül való nemzetekre, és elveszem õket és csudává és szörnyûséggé teszem õket és  örökkévaló pusztasággá.
\par 10 És elveszem tõlök az öröm szavát, a vígasság szavát, a võlegény szavát és a menyasszony szavát, a malmok zörgését és a szövétnek világosságát.
\par 11 És ez egész föld pusztasággá és csudává lészen, és e nemzetek a babiloni királynak szolgálnak hetven esztendeig.
\par 12 És mikor eltelik a hetven esztendõ, meglátogatom a babiloni királyon és az õ népén az õ álnokságukat, azt mondja az Úr, és a káldeai földön, és örökkévaló pusztasággá teszem azt.
\par 13 És végbeviszem azon a földön mindazokat, a miket szóltam felõle, mindazt, a mi megvan írva e könyvben, a melyet prófétált Jeremiás az összes nemzetek felõl.
\par 14 Mert õ rajtok is uralkodnak majd sok nemzetek és nagy királyok, és megfizetek nékik az õ cselekedeteik szerint és az õ kezeiknek munkája szerint.
\par 15 Mert ezt mondotta az Úr, Izráelnek Istene nékem: Vedd el kezembõl e harag borának poharát, és itasd meg vele mindama nemzeteket, a kikhez én küldelek téged.
\par 16 Hogy igyanak, részegüljenek meg és bolondoskodjanak a fegyver miatt, a melyet én közéjök bocsátok.
\par 17 És elvevém a pohárt az Úr kezébõl, és megitatám mindama nemzeteket, a kikhez külde engem az Úr:
\par 18 Jeruzsálemet és Júda városait, az õ királyait és fejedelmeit, hogy pusztasággá, csudává, szörnyûséggé és átokká tegyem õket, a mint e mai napon van:
\par 19 A Faraót, Égyiptom királyát, az õ szolgáit és fejedelmeit és minden õ népét,
\par 20 És minden egyveleg népet, és az Úz földének minden királyát, és a Filiszteusok földének minden királyát, és Askalont, Gázát, Akkaront és Azótusnak maradékait,
\par 21 Edomot, Moábot és az Ammon fiait,
\par 22 És Tírusnak minden királyát és Szidonnak minden királyát és a szigeteknek minden királyát, a kik túl vannak a tengeren,
\par 23 Dedánt és Témánt és Búzt és mindazokat, a kik lenyírott üstökûek,
\par 24 És Arábiának minden királyát és az egyveleg nép minden királyát, a kik a pusztában laknak,
\par 25 És Zimrinek minden királyát és Elámnak minden királyát és a Médusok minden királyát,
\par 26 És északnak minden királyát, mind a közelvalókat, mind a távolvalókat, egyiket a másikra, és a földnek minden országát, a melyek e föld színén vannak; Sésák királya pedig ezek után iszik.
\par 27 Azért ezt mondd nékik: Ezt mondja a Seregek Ura, Izráelnek Istene: Igyatok és részegüljetek meg, okádjatok és hulljatok el, és fel ne keljetek a fegyver elõtt, a melyet én küldök közétek.
\par 28 És ha majd a te kezedbõl nem akarják elvenni a pohárt, hogy igyanak belõle: ezt mondd nékik: Így szól a Seregek Ura: Meg kell innotok.
\par 29 Mert ímé, a város ellen, a mely az én nevemrõl neveztetik, az ellen veszedelmet indítok, ti pedig egészen megmenekültök-é? Nem menekesztek, mert én fegyvert hozok e földnek minden lakosára, azt mondja a Seregek Ura,
\par 30 Te pedig prófétáld meg nékik mind e szókat, és ezt mondd nékik: Az Úr a magasságból harsog, és az õ szent lakhelyébõl dörög, harsanva harsog az õ házára, riogatva kiált, mint a szõlõtaposók, e föld minden lakosa ellen.
\par 31 Elhat e harsogás a földnek végére, mert pere van az Úrnak a pogányokkal, õ minden testnek ítélõ birája, a hitetleneket fegyverre veti, azt mondja  az Úr.
\par 32 Így szól a Seregek Ura: Ímé, veszedelem indul egyik nemzettõl a másik nemzetre, és nagy szélvész támad a föld széleitõl.
\par 33 És azon a napon az Úrtól levágatnak a föld egyik végétõl fogva a föld másik végéig; nem sirattatnak meg, és össze sem hordatnak, és el sem temettetnek, olyanok lesznek a földnek színén, mint a ganéj.
\par 34 Jajgassatok pásztorok, és kiáltsatok, és heverjetek a porban, ti vezérei a nyájnak, mert eljön a ti megöletéseteknek és szétszóratástoknak ideje, és elhullotok, noha drága edények vagytok.
\par 35 És nincs hová futniok a pásztoroknak, és menekülniök a nyáj vezéreinek.
\par 36 Hallatszik a pásztorok kiáltozása és a nyáj vezéreinek jajgatása, mert elpusztította az Úr az õ legeltetõ helyöket.
\par 37 Elpusztultak a békességes legeltetõ helyek az Úr felgerjedt haragja miatt.
\par 38 Elhagyta azokat, mint az oroszlán az õ barlangját, mert pusztasággát lett az õ földjök a zsarnoknak dühe miatt és az õ felgerjedt haragja miatt.

\chapter{26}

\par 1 A Jojákim uralkodásának kezdetén, a ki fia vala Jósiásnak, a Júda királyának, e szót szólá az Úr, mondván:
\par 2 Ezt mondja az Úr: Állj az Úr házának pitvarába, és mondd el Júdának minden városából azoknak, a kik imádkozni jõnek az Úr házába, mindazokat az igéket, a melyeket parancsoltam néked, hogy mond el nékik. Egy szót se hagyj  el!
\par 3 Hátha szót fogadnak, és mindenki megtér az õ gonosz útjától, akkor megbánom a veszedelmet, a melyet nékik okozni gondoltam az õ cselekedeteik gonoszsága miatt.
\par 4 És mondd meg nékik: Így szól az Úr: Ha nem hallgattok reám, hogy az én törvényeim szerint járjatok, a melyet elõtökbe adtam,
\par 5 Hogy hallgassatok a prófétáknak, az én szolgáimnak beszédére, a kiket én küldék hozzátok, és pedig jó reggel küldém, de nem hallgattatok rájok:
\par 6 Akkor e házat hasonlóvá teszem Silóhoz, e várost pedig a föld minden nemzetének átkává.
\par 7 Hallák pedig a papok és próféták és az egész község Jeremiást, a mint ez igéket szólá az Úr házában.
\par 8 És lõn, mikor elvégezé Jeremiás a beszédet, a melyet az Úr parancsolt vala mind az egész néphez szólania, megragadák õt a papok és a próféták és az egész nép, mondván: Halállal kell lakolnod!
\par 9 Miért prófétáltál az Úr nevében, mondván: Olyan lesz e ház, mint Siló, és e város elpusztul, lakatlanná lesz? És összegyûle az egész nép Jeremiás ellen az Úrnak házában.
\par 10 De meghallák ezeket a Júda fejedelmei, és felmenének a király házából az Úrnak házába, és leülének az Úr új kapujának ajtajában.
\par 11 És szólának a papok és a próféták a fejedelmeknek és az egész népnek, mondván: Halálra méltó ez az ember, mert e város ellen prófétált, a mit füleitekkel hallottátok.
\par 12 És szóla Jeremás mindazoknak a fejedelmeknek és az egész népnek, mondván: Az Úr küldött engem, hogy prófétáljak e ház ellen és e város ellen mind e szókkal, a melyeket hallottatok.
\par 13 Most azért jobbítsátok meg a ti útaitokat és cselekedeteiteket, és hallgassatok az Úrnak, a ti Isteneteknek szavára, és az Úr megbánja a veszedelmet, a melylyel fenyegetett titeket.
\par 14 Én magam pedig ímé kezetekben vagyok, cselekedjetek velem, a mint jónak és a mint helyesnek tetszik néktek.
\par 15 De jól tudjátok meg, hogy ha megöltök engem, ártatlan vér száll ti reátok és e városra és ennek lakosaira, mert bizony az Úr küldött el engem hozzátok, hogy a ti füleitekbe mondjam e szókat.
\par 16 És mondák azok a fejedelmek és az egész község a papoknak és prófétáknak: Nem méltó ez az ember a halálra: mert az Úrnak, a mi Istenünknek nevében szólt nékünk!
\par 17 Felkelének az ország vénei közül is némely férfiak, és szólának az egész összegyûlt népnek, mondván:
\par 18 A móreseti Mikeás prófétált Ezékiásnak, a Júda királyának idejében, és szóla az egész Júda népének, mondván: Ezt mondja a Seregeknek Ura: A Siont megszántják, mint szántóföldet, és Jeruzsálem elpusztul, és e háznak hegyét erdõ növi be;
\par 19 Megölte-é azért õt Ezékiás, Júdának királya és az egész Júda? Nem félte-é az Urat? Nem könyörgött-é az Úrnak? És nem megbánta-é az Úr a veszedelmet, a melylyel fenyegette õket? És mi olyan nagy veszedelmet vonnánk a mi lelkünkre?
\par 20 Volt egy másik ember is, a ki az Úr nevében prófétált, Uriás, Semájának fia, Kirját-Jeárimból, és prófétált e város ellen és a föld ellen, egészen a Jeremiás beszéde szerint.
\par 21 És meghallá Jojákim király és minden vitéze és minden fejedelme az õ beszédeit, és halálra kerestette õt a király; de meghallá Uriás és megijedt, és elfutott és átment Égyiptomba.
\par 22 És Jojákim király embereket küldött Égyiptomba, Elnátánt, az Akbor fiát, és más férfiakat vele együtt Égyiptomba.
\par 23 És kihozák Uriást Égyiptomból és vivék Jojákim király elé, a ki megöleté õt fegyverrel, holttestét pedig a köznép temetõjébe vetteté.
\par 24 De Ahikámnak, a Sáfán fiának keze Jeremiással lõn, hogy ne adják õt a nép kezébe  halálra.

\chapter{27}

\par 1 A Jojákim uralkodásának kezdetén, a ki fia vala Jósiásnak, a Júda királyának, ilyen szavakat szóla az Úr Jeremiáshoz, mondván:
\par 2 Így szól az Úr nékem: Csinálj magadnak köteleket, és jármot és vedd azokat a nyakadba.
\par 3 És küldd azokat Edom királyához és Moáb királyához, az Ammon fiainak királyához, Tírus királyához és Sidon királyához a követek által, a kik eljõnek Jeruzsálembe Sedékiáshoz, Júdának királyához,
\par 4 És parancsold meg nékik, hogy mondják meg az õ uraiknak: Ezt mondja a Seregek Ura, Izráelnek Istene, ezt mondjátok a ti uraitoknak:
\par 5 Én teremtettem a földet, az embert és a barmot, a melyek e föld színén vannak, az én nagy erõmmel és az én kinyujtott karommal, és annak adom azt, a ki kedves az én szemeim elõtt:
\par 6 És most én odaadom mind e földeket Nabukodonozornak, a babiloni királynak, az én szolgámnak kezébe; sõt a mezei állatokat is néki adom, hogy néki szolgáljanak.
\par 7 És néki és az õ fiának és unokájának szolgál minden nemzet mindaddig, míg el nem jõ az õ földének is ideje, és szolgálnak néki sok nemzetek és nagy királyok.
\par 8 Azt a nemzetet és azt az országot pedig, a mely nem szolgál néki, Nabukodonozornak, a babiloni királynak, és a ki nem teszi nyakát a babiloni király jármába: fegyverrel és éhséggel és döghalállal verem meg azt a nemzetet, azt mondja az Úr, míglen kiirtom õket az õ kezével.
\par 9 Ti azért ne hallgassatok a ti prófétáitokra, se jövendõmondóitokra, se álommagyarázóitokra, se varázslóitokra, se szemfényvesztõitekre, a kik ezt mondják néktek: Ne szolgáljatok a babiloni királynak.
\par 10 Mert õk hazugságot prófétálnak néktek, hogy messze vigyelek titeket a ti földetekbõl és kiûzzelek titeket, és elveszszetek!
\par 11 Azt a nemzetet pedig, a mely nyakára veszi a babiloni király jármát és szolgál néki, az õ földében hagyom, azt mondja az Úr, és míveli azt és lakozik benne.
\par 12 Sõt Sedékiásnak, a Júda királyának is mind e beszédek szerint szólottam, mondván: Vegyétek nyakatokra a babiloni királynak jármát, és szolgáljatok néki és az õ népének, és éltek!
\par 13 Miért halsz meg te és a te néped fegyver miatt, éhség és döghalál miatt, a mint szólott az Úr az olyan néprõl, a mely nem szolgál a babiloni királynak?
\par 14 Ne hallgassatok hát a próféták szavaira, a kik így szólnak néktek: Ne szolgáljatok a babiloni királynak; mert õk hazugságot prófétálnak néktek.
\par 15 Mert nem küldöttem õket, azt mondja az Úr, hanem õk az én nevemben hazugságot prófétálnak, hogy kiûzzelek titeket, és elvesszetek ti és a próféták, a kik prófétálnak néktek.
\par 16 A papoknak és ez egész népnek is így szóltam: Ezt mondja az Úr: Ne hallgassatok a ti prófétáitok szavaira, a kik prófétálnak néktek, mondván: Ímé az Úr házának edényei visszahozatnak Babilonból most mindjárt, mert hazugságot prófétálnak õk néktek:
\par 17 Ne hallgassatok rájok; szolgáljatok a babiloni királynak, és éltek; miért legyen e város pusztasággá?
\par 18 Ha õk próféták és ha nálok van az Úrnak igéje: imádkozzanak most a Seregek Urának, hogy az edények, a melyek még az Úr házában és a Júda királyának házában és Jeruzsálemben maradtak, ne jussanak Babilonba;
\par 19 Mert ezt mondja a Seregek Ura az oszlopok felõl, a tenger felõl, az állványok felõl és az edények maradéka felõl, a melyek még e városban maradtak,
\par 20 A melyeket el nem vitt Nabukodonozor, a babiloni király, mikor fogságba vivé Jékóniást, Jojákimnak, a Júda királyának fiát Jeruzsálembõl Babilonba, és Júdának és Jeruzsálemnek minden fõ népét;
\par 21 Bizony ezt mondja a Seregek Ura, Izráelnek Istene az edények felõl, a melyek megmaradtak az Úrnak házában és a Júda királyának házában és Jeruzsálemben:
\par 22 Babilonba vitetnek, és ott lesznek mindama napig, a melyen  meglátogatom õket, azt mondja az Úr; és felhozom azokat, és visszahozom azokat e helyre.

\chapter{28}

\par 1 És abban az esztendõben, Sedékiás, Júda királya uralkodásának kezdetén, a negyedik esztendõben, az ötödik hónapban monda nékem Hanániás (Azúrnak fia, a próféta, a ki Gibeonból való vala) az Úrnak házában, a papok és az egész nép szemei elõtt, mondván:
\par 2 Ezt mondja a Seregek Ura, Izráel Istene, mondván: Eltöröm a babiloni királynak jármát.
\par 3 Teljes két esztendõ mulva visszahozom e helyre az Úr házának mindamaz edényeit, a melyeket elvitt innen Nabukodonozor, a babiloni király, és bevitt Babilonba.
\par 4 És Jékóniást, Jojákimnak, a Júda királyának fiát és mindama júdabeli foglyokat, a kik elvitettek Babilonba, visszahozom én e helyre, azt mondja az Úr; mert eltöröm a babiloni király jármát.
\par 5 Akkor monda Jeremiás próféta Hanániás prófétának a papok és az egész nép szemei elõtt, a mely ott áll vala az Úrnak házában;
\par 6 Mondá pedig Jeremiás próféta: Úgy legyen, úgy cselekedjék az Úr: teljesítse az Úr a te beszédeidet, a melyekkel prófétálád, hogy az Úr házának edényei és a foglyok is mind visszahozatnak Babilonból e helyre,
\par 7 Mindazáltal halld csak e beszédet, a melyet én szólok néked és az egész népnek:
\par 8 A próféták, a kik elõttem és elõtted eleitõl fogva voltak, sok ország ellen és nagy királyságok ellen, hadról, veszedelemrõl és döghalálról prófétáltak.
\par 9 A mely próféta a békességrõl prófétál, mikor beteljesedik a próféta beszéde, akkor ismertetik meg a próféta, ha az Úr küldötte-é azt valóban?
\par 10 És vevé Hanániás próféta a jármot a Jeremiás próféta nyakáról, és széttöré azt.
\par 11 És szóla Hanániás az egész nép elõtt, mondván: Ezt mondja az Úr: Így töröm le Nabukodonozornak, a babiloni királynak jármát két esztendei idõ mulva minden nemzet nyakáról: és elméne Jeremiás próféta a maga útjára.
\par 12 És szóla az Úr Jeremiásnak, miután letöré Hanániás próféta a Jeremiás próféta nyakáról a jármot, mondván:
\par 13 Menj el, és beszélj Hanániással, mondván: Ezeket mondja az Úr: A fajármot eltörted, de csináltál helyébe vasjármokat.
\par 14 Mert ezt mondja a Seregek Ura, Iráel Istene: Vasjármot vetettem mind e nemzetek nyakára, hogy szolgáljanak Nabukodonozornak, a babiloni királynak, és szolgálnak néki, sõt a mezei állatokat is néki adom.
\par 15 És monda Jeremiás próféta Hanániás prófétának: Halld csak, Hanániás! nem az Úr küldött téged, és te hazugsággal biztatod e népet.
\par 16 Azért így szól az Úr: Ímé, én elküldelek téged a föld színérõl, meghalsz ez esztendõben: mert pártütõleg szóltál az Úr ellen.
\par 17 És meghala Hanániás próféta abban az esztendõben, a hetedik hónapban.

\chapter{29}

\par 1 Ezek ama levél szavai, a melyet Jeremiás próféta külde Jeruzsálembõl a fogságban való vének maradékainak, és a papoknak és a prófétáknak, és az egész népnek, a melyet fogva vitt vala el Nabukodonozor Jeruzsálembõl Babilonba.
\par 2 (Miután Jékóniás király és a királyné és az udvari szolgák, Júda és Jeruzsálem fejedelmei és az ács és a kovácsmester kijövének Jeruzsálembõl.)
\par 3 Elasának a Sáfán fiának, és Gamariának a Hilkiás fiának keze által, a kiket Sedékiás, Júda királya küldött Nabukodonozorhoz, a babiloni királyhoz Babilonba, mondván:
\par 4 Ezt mondja a Seregek Ura, Izráel Istene mindama foglyoknak, a kiket Jeruzsálembõl Babilonba vitettem:
\par 5 Építsetek házakat és lakjatok azokban, plántáljatok kerteket és egyétek azoknak gyümölcseit.
\par 6 Vegyetek magatoknak feleségeket és szûljetek fiakat és leányokat, és a fiaitokat is házasítsátok meg, a leányaitokat pedig adjátok férjhez, és szûljenek fiakat és leányokat, és szaporodjatok meg ott, és meg ne kevesedjetek.
\par 7 És igyekezzetek a városnak jólétén, a melybe fogságra küldöttelek titeket, és könyörögjetek érette az Úrnak; mert annak jóléte lesz a ti jólétetek.
\par 8 Mert ezt mondja a Seregek Ura, Izráel Istene: Ne hitessenek el titeket a ti prófétáitok, a kik közöttetek vannak, se a ti jövendõmondóitok, és ne figyelmezzetek a ti álmaitokra, a melyeket álmodoztok.
\par 9 Mert õk hamisan prófétálnak néktek az én nevemben: Nem küldöttem õket, azt mondja az Úr.
\par 10 Mert ezt mondja az Úr: Mihelyt eltelik Babilonban a hetven esztendõ, meglátogatlak titeket; s betöltöm rajtatok az én jó szómat, hogy visszahozzalak titeket e helyre.
\par 11 Mert én tudom az én gondolatimat, a melyeket én felõletek gondolok, azt mondja az Úr; békességnek és nem háborúságnak gondolata, hogy kivánatos véget adjak néktek.
\par 12 Akkor segítségre hívtok engem, és elmentek és imádtok engem, és mghallgatlak titeket.
\par 13 És kerestek engem és megtaláltok, mert teljes szívetekbõl kerestek engem.
\par 14 És megtaláltok engem, azt mondja az Úr, és visszahozlak a fogságból, és összegyûjtlek titeket minden nemzet közül és mindama helyekrõl, a hová kiûztelek titeket, azt mondja az Úr, és visszahozlak e helyre, a honnan számkivetettelek titeket.
\par 15 Mert ezt mondjátok: Támasztott nékünk az Úr prófétákat Babilonban is.
\par 16 Mert ezt mondja az Úr a királynak, a ki Dávid székében ül, és az egész népnek, a mely e városban lakik, tudniillik a ti atyátok fiainak, a kik nem mentek el veletek a fogságra:
\par 17 Ezt mondja a Seregek Ura: Ímé, én reájok küldöm a fegyvert, az éhséget és a döghalált, és olyanokká teszem õket, a milyenek a keserû fügék, a melyek a rosszaság miatt ehetetlenek.
\par 18 És üldözöm õket fegyverrel, éhséggel és döghalállal, és rettenetessé teszem õket a föld minden országára, átokká és csudává és szörnyûséggé és mindama nemzetnek gúnyjává, a melyek közé kivetettem õket.
\par 19 Azért mert nem hallgattak az én beszédeimre, azt mondja az Úr, a melyeket én üzentem nékik az én szolgáim, a próféták által, jó reggel elküldvén, de nem hallgattátok, azt mondja az Úr.
\par 20 Ti azért, ti foglyok, halljátok meg az Úrnak szavát mindnyájan, a kiket Jeruzsálembõl Babilonba küldtem.
\par 21 Ezt mondja a Seregek Ura, az Izráel Istene Akhábra a Kolája fiára, és Sedékiásra a Mahásiás fiára, a kik hamisan prófétálnak néktek az én nevemben: Ímé, én odaadom õket Nabukodonozornak, Babilon királyának kezébe, és szemeitek láttára megöli õket.
\par 22 És mind azokról vesznek átokformát a Júdából való foglyok, a kik Babilonban vannak, mondván: Tegyen téged az Úr olyanná, mint Sedékiást és mint Akhábot, a kiket tûzzel égetett meg a babiloni király.
\par 23 Mert istentelenséget cselekedtek Izráelben, és paráználkodtak az õ felebarátjaik feleségével, és az én nevemben hazugságot szóltak, a mit nem parancsoltam nékik. Én pedig tudom azt, és bizonyság vagyok, azt mondja az Úr!
\par 24 A nehelámiti Semájának is szólj, ezt mondván:
\par 25 Így szól a Seregek Ura, az Izráel Istene, mondván: Mivelhogy te levelet küldöttél a magad nevével az egész néphez, a mely Jeruzsálemben van, és Sofóniás paphoz, a Mahásiás fiához; és az összes papokhoz, mondván:
\par 26 Az Úr tett téged pappá Jojáda pap helyébe, hogy felügyelõk legyenek az Úr házában minden bolond férfiún, és prófétálni akarón, hogy vessed ezt a tömlöczbe és a kalodába.
\par 27 Most azért, miért nem dorgáltad meg az Anatóthbeli Jeremiást, a ki néktek prófétál?
\par 28 Sõt még hozzánk küldött Babilonba, ezt mondván: Hosszú lesz az! Építsetek házakat és lakjatok bennök, plántáljatok kerteket és éljetek azoknak gyümölcseivel.
\par 29 És elolvasá Sofóniás pap e levelet a Jeremiás próféta hallására.
\par 30 És szóla az Úr Jeremiásnak, mondván:
\par 31 Küldj el mind a foglyokhoz, mondván: Ezt mondja az Úr a nehelámiti Semája felõl: Mivelhogy prófétált néktek Semája, holott én nem küldtem õt, és hazugsággal biztatott titeket;
\par 32 Azért ezt mondja az Úr: Ímé, én megfenyítem a nehelámiti Semáját és az õ magvát; nem lesz néki embere, a ki lakjék e nép között, és nem látja a jót, melyet én az én népemmel cselekszem, azt mondja az Úr; mert az Úr ellen való szót szólott.

\chapter{30}

\par 1 Az a szó, a melyet szólott az Úr Jeremiásnak, mondván:
\par 2 Ezt mondja az Úr, Izráel Istene, mondván: Mindama szókat, a melyeket mondottam néked, írd meg magadnak könyvben;
\par 3 Mert ímé, eljõnek a napok, azt mondja az Úr, és visszahozom az én népemet, az Izráelt és Júdát, azt mondja az Úr, és visszahozom õket arra a földre, a melyet az õ atyáiknak adtam, és bírni fogják azt.
\par 4 Ezek pedig azok a szók, a melyeket az Úr Izráel és Júda felõl szólott.
\par 5 Ezt mondta ugyanis az Úr: A félelemnek, rettentésnek szavát hallottuk, és nincs békesség.
\par 6 Kérdjétek meg csak és lássátok, ha szûl-é a férfi? Miért látom minden férfi kezét az ágyékán, mintegy gyermekszûlõét, és miért változtak orczáik fakósárgává?
\par 7 Jaj! mert nagy az a nap annyira, hogy nincs hozzá hasonló; és háborúság ideje az Jákóbon; de megszabadul abból!
\par 8 És azon a napon, azt mondja a Seregek Ura, letöröm majd az õ igáját a te nyakadról, köteleidet leszaggatom, és nem szolgálnak többé idegeneknek.
\par 9 Hanem szolgálnak az Úrnak az õ Istenöknek, és Dávidnak az õ királyoknak, a kit feltámasztok nékik.
\par 10 Te azért ne félj, oh én szolgám Jákób, azt mondja az Úr, se ne rettegj Izráel; mert ímhol vagyok én, a ki megszabadítlak téged a messze földrõl, és a te magodat az õ fogságának földébõl; és visszatér Jákób és megnyugoszik, és bátorságban lesz, és nem lesz, a ki megháborítsa.
\par 11 Mert veled vagyok én, azt mondja az Úr, hogy megtartsalak téged: mert véget vetek minden nemzetnek, a kik közé kiûztelek téged, csak néked nem vetek véget, hanem megfenyítelek téged ítélettel, mert nem hagylak egészen büntetés nélkül.
\par 12 Mert ezt mondja az Úr: Veszedelmes a te sebed, gyógyíthatatlan a te sérülésed.
\par 13 Senki sincsen, a ki megítélje a te ügyedet, hogy bekösse sebedet, orvosságok és balzsam nincsenek számodra.
\par 14 Elfeledkezett rólad minden szeretõd, és nem keres téged: mert megvertelek ellenséges veréssel, kegyetlen ostorozással, a te bûnödnek sokaságáért, és hogy eláradtak a te vétkeid.
\par 15 Miért kiáltozol a te sebed miatt? veszedelmes a te sérülésed? a te bûnöd sokaságáért és vétkeidnek eláradásáért cselekedtem ezeket veled.
\par 16 Azért mindazok, a kik benyelnek téged, elnyeletnek, és valamennyi ellenséged mind fogságra jut: és a te fosztogatóidat kifosztottakká, és minden te zsákmánylóidat zsákmánynyá teszem.
\par 17 Mert orvosságot adok néked, és kigyógyítlak a te sérülésedbõl, azt mondja az Úr. Mert számkivetettnek hívtak téged, Sion; nincs, a ki tudakozódjék felõle.
\par 18 Ezt mondja az Úr: Ímé, visszahozom Jákób sátorának foglyait, és könyörülök az õ hajlékain, és a város felépíttetik az õ magas helyén, és a palota a maga helyén marad.
\par 19 És hálaadás és öröm szava jõ ki belõlök, és megsokasítom õket és meg nem kevesednek, megöregbítem õket és meg nem kisebbednek.
\par 20 És az õ fiai olyanok lesznek, mint eleintén, és az õ gyülekezete erõsen megáll az én orczám elõtt, de megbüntetem mindazokat, a kik nyomorgatták õt.
\par 21 És az õ fejedelme õ belõle támad, és az õ uralkodója belõle jõ ki, és magamhoz bocsátom õt, hogy közeledjék hozzám: mert kicsoda az, a ki szívét arra hajtaná, hogy hozzám jõjjön, azt mondja az Úr.
\par 22 És népemmé lesztek, én pedig a ti Istenetek leszek.
\par 23 Ímé, az Úrnak szélvésze, haragja tör elõ, és a rohanó szélvész a hitetlenek fejére zúdul.
\par 24 Nem szûnik meg az Úr felgerjedt haragja, míg végbe nem viszi, és míg meg nem valósítja az õ szívének gondolatait. Az utolsó napokban értitek meg e dolgot.

\chapter{31}

\par 1 Az idõben, monda az Úr, Izráel minden nemzetségének Istene leszek, és õk az én népemmé lesznek.
\par 2 Ezt mondja az Úr: Kegyelmet talált a pusztában a fegyvertõl megmenekedett nép, az Isten õ elõtte menvén, hogy megnyugtassa õt, az Izráelt.
\par 3 Messzünnen is megjelent nékem az Úr, mert örökkévaló szeretettel szerettelek téged, azért terjesztettem reád az én irgalmasságomat.
\par 4 Újra felépítlek téged, és felépülsz, oh Izráel leánya. Újra felékesíted magadat, dobokkal és vígadók seregében jösz ki.
\par 5 Még szõlõket plántálsz Samariának hegyein; a kik plántálják a plántákat, élnek is azok gyümölcsével.
\par 6 Mert lészen egy nap, mikor a pásztorok kiáltnak az Efraim hegyén: Keljetek fel, és menjünk fel Sionba az Úrhoz, a mi Istenünkhöz.
\par 7 Mert ezt mondja az Úr: Énekeljetek Jákóbnak vígassággal, és ujjongjatok a nemzetek fejének. Hirdessétek dicséretét, és mondjátok: Tartsd meg Uram a te népedet, az Izráel maradékát.
\par 8 Ímé, én elhozom õket észak földébõl, és összegyûjtöm õket a földnek széleirõl, közöttök lesz vak, sánta, viselõs és gyermek-szûlõ is lesz velök, mint nagy sereg jõnek ide vissza.
\par 9 Siralommal jõnek és imádkozva hozom õket, vezetem õket a vizek folyásai mellett egyenes úton, hol el nem esnek, mert atyja leszek az Izráelnek, és az Efraim  nékem elsõszülöttem.
\par 10 Halljátok meg az Úrnak szavát, ti pogányok, és hirdessétek a messzevaló szigeteknek, és ezt mondjátok: A ki elszórta az Izráelt, az gyûjti õt össze, és megõrzi, mint a pásztor a maga nyáját.
\par 11 Mert megváltotta az Úr Jákóbot, és kimentette a nálánál erõsebbnek kezébõl.
\par 12 És eljõnek és énekelnek a Sion ormán, és futnak az Úrnak javaihoz, búza, bor, olaj, juhok és barmok nyája felé, és az õ lelkök olyan lesz, mint a megöntözött kert, és nem bánkódnak többé.
\par 13 Akkor vígadoz a szûz a seregben, és az ifjak és a vének együttesen, és az õ siralmokat örömre fordítom, és megvígasztalom és felvidámítom õket az õ bánatukból.
\par 14 És a papok lelkét megelégítem kövérséggel, és az én népem eltelik javaimmal, azt mondja az Úr.
\par 15 Ezt mondja az Úr: Szó hallatszott Rámában, sírás és keserves jajgatás; Rákhel siratta az õ fiait, nem akart megvígasztaltatni az õ fiai felõl, mert nincsenek.
\par 16 Ezt mondja az Úr: Tartsd vissza szódat a sírástól és szemeidet a könyhullatástól, mert meglesz a te cselekedetednek jutalma, azt mondja az Úr, hiszen az ellenség földébõl térnek vissza.
\par 17 Jövendõdnek is jó reménysége lészen, azt mondja az Úr, mert fiaid visszajõnek az õ határaikra.
\par 18 Jól hallottam, hogy panaszolkodott Efraim: Megvertél engem és megverettetém, mint a tanulatlan tulok; téríts meg engem és megtérek, mert te vagy az Úr, az én Istenem.
\par 19 Mert azután, hogy megtérítettél engem, megbántam bûnömet, és miután megismertem magamat, czombomat vertem; szégyenkezem és pirulok, mert viselem az én ijfúságomnak gyalázatát.
\par 20 Avagy nem kedves fiam-é nékem Efraim? Avagy nem kényeztetett gyermek-é? Hiszen valahányszor ellene szóltam, újra megemlékeztem õ róla, azért az én belsõ részeim megindultak õ rajta, bizony könyörülök rajta, azt mondja az Úr!
\par 21 Rendelj magadnak útjelzõket, rakj útmutató oszlopokat, vigyázz az ösvényre, az útra, a melyen mentél, jõjj vissza Izráelnek leánya, jõjj vissza ide a te városodba!
\par 22 Meddig bújdosol, oh szófogadatlan leány? Mert az Úr új rendet teremt e földön. Asszony  környékezi a férfit.
\par 23 Ezt mondja a Seregek Ura, Izráel Istene: Újra e szókat mondják majd a Júda földén és az õ városaiban, mikor visszahozom az õ foglyaikat: Áldjon meg téged az Úr, oh igazságnak háza, oh szent hegy!
\par 24 És ott lakoznak majd Júda és minden õ városa, a szántóvetõk és baromtartók együttesen.
\par 25 Mert megitatom a szomjú lelket, és minden éhezõ lelket megelégítek.
\par 26 Ezért vagyok ébren és vigyázok, és az én álmom édes nékem.
\par 27 Ímé, eljõnek a napok, azt mondja az Úr, és bevetem az Izráel házát és a Júda házát embernek magvával és baromnak magvával.
\par 28 És a miképen gondom volt arra, hogy kigyomláljam és elrontsam, letörjem és pusztítsam és veszedelembe sodorjam õket, azonképen vigyázok arra, hogy megépítsem és beplántáljam õket, azt mondja az Úr!
\par 29 Ama napokban nem mondják többé: Az atyák ették meg az egrest, és a fiak foga vásott el bele.
\par 30 Sõt inkább kiki a maga gonoszságáért hal meg; minden embernek, ki megeszi az egrest, tulajdon foga vásik el bele.
\par 31 Ímé, eljõnek a napok, azt mondja az Úr; és új szövetséget kötök az Izráel házával és Júda házával.
\par 32 Nem ama szövetség szerint, a melyet az õ atyáikkal kötöttem az napon, a melyen kézen fogtam õket, hogy kihozzam õket Égyiptom földébõl, de a kik megrontották  az én szövetségemet noha én férjök maradtam, azt mondja az Úr.
\par 33 Hanem ezt lesz a szövetség, a melyet e napok után az Izráel házával kötök, azt mondja az Úr: Törvényemet az õ belsejökbe helyezem, és az õ szívökbe írom be, és Istenökké lesznek, õk pedig népemmé lesznek.
\par 34 És nem tanítja többé senki az õ felebarátját, és senki az õ atyjafiát, mondván: Ismerjétek meg az Urat, mert õk mindnyájan megismernek engem, kicsinytõl fogva nagyig, azt mondja az Úr, mert megbocsátom  az õ bûneiket, és vétkeikrõl többé meg nem emlékezem.
\par 35 Ezt mondja az Úr, a ki adta a napot, hogy világítson nappal, a ki törvényt szabott a holdnak és a csillagoknak, hogy világítsanak éjjel, a ki felháborítja a tengert és annak habjai zúgnak, Seregek Ura az õ neve:
\par 36 Ha eltünnek e törvények elõlem, azt mondja az Úr, az Izráelnek magva is megszakad, hogy soha én elõttem nép ne legyen.
\par 37 Ezt mondja az Úr: Ha megmérhetik az egeket ott fenn, és itt alant kifürkészhetik a föld fundamentomait: én is megútálom Izráelnek minden magvát, mindazokért, a miket cselekedtek, azt mondja az Úr!
\par 38 Ímé eljõnek a napok, azt mondja az Úr, és felépíttetik a város az Úrnak a Hanániel tornyától fogva a szeglet kapujáig.
\par 39 És kijjebb megy még a mérõkötél azzal átellenben a Garéb hegyéig, és lefordul Góhat felé.
\par 40 És a holttesteknek és a hamunak egész völgye, és az egész mezõ a Kidron patakáig, a lovak kapujának szegletéig kelet felé az Úr szent helye lesz, nem rontatik el, sem el nem pusztíttatik soha örökké.

\chapter{32}

\par 1 Az a szó, a melyet szóla az Úr Jeremiásnak, Sedékiásnak, a Júda királyának tizedik esztendejében: Ez az esztendõ a Nabukodonozor tizennyolczadik esztendeje.
\par 2 És akkor megszállotta vala a babiloni király serege Jeruzsálemet, és Jeremiás próféta elzárva vala a tömlöcznek pitvarában, a mely a Júda királyának házában vala.
\par 3 Mert Sedékiás, a Júda királya záratta be õt, mondván: Miért prófétálsz te, ezt mondván: Ezt mondta az Úr: Ímé, én e várost a babiloni király kezébe adom és beveszi azt?
\par 4 És Sedékiás, a Júda királya, meg nem menekszik a Káldeusok kezébõl, hanem a babiloni király kezébe adatik, és ennek szája szól amannak szájával, és ennek szemei látják amannak szemeit.
\par 5 És Babilonba viszi Sedékiást és ott lesz mindaddig, míg meg nem látogatom õt, azt mondja az Úr, hogyha hadakoztok a Káldeusok ellen, nem lesz jó dolgotok.
\par 6 És monda Jeremiás: Szólott az Úr nékem, ezt mondván:
\par 7 Ímé, Hanameél, Sallumnak, a te nagybátyádnak fia hozzád megy, mondván: Vedd meg az én mezõmet, a mely Anatótban van, mert téged illet vér szerint, hogy megvegyed.
\par 8 Eljöve azért hozzám Hanameél, az én nagybátyámnak fia, az Úr beszéde szerint a tömlöcz pitvarához, és monda nékem: Kérlek, vedd meg az én mezõmet, a mely Anatótban, a Benjámin földén van, mert téged illet, mint örököst, és te reád néz vér szerint is, vedd meg hát magadnak. Akkor észrevevém, hogy az Úr szava ez.
\par 9 Azért megvevém Hanameéltõl, az én nagybátyámnak fiától a mezõt, a mely Anatótban van, és kifizettem néki a pénzt, tizenhét ezüst siklust.
\par 10 És beírám levélbe és megpecsétlém, és tanúkat is állíték, és megmérém a pénzt mérlegen.
\par 11 Ezután kezembe vevém a vétel felõl való levelet, a mely meg vala pecsételve a parancsolat és törvények szerint, és a közönséges levelet is.
\par 12 És a vétel felõl való levelet odaadám Báruknak, a Néria fiának, a ki Mahásiás fia vala, Hanameélnek, az én nagybátyám fiának szemei elõtt, és a tanúk szemei elõtt, a kik be valának írva a vétel felõl való levélbe, mindama júdaiak szemei elõtt, a kik ülnek vala a tömlöcz pitvarában.
\par 13 És parancsolék Báruknak azok szemei elõtt, mondván:
\par 14 Ezt mondja a Seregek Ura, az Izráel Istene: Vedd fel e leveleket, ezt a vételi levelet, mind a bepecsételtetett, mind a közönséges levelet, és tedd azokat cserépedénybe, hogy sok ideig elálljanak.
\par 15 Mert ezt mondja a Seregek Ura, Izráel Istene: Még házakat, mezõket és szõlõket fognak venni e földön.
\par 16 És könyörgék az Úrnak, miután odaadám a vétel felõl való levelet Báruknak, a Néria fiának, mondván:
\par 17 Ah, ah, Uram Isten! Ímé te teremtetted a mennyet és földet a te nagy hatalmaddal és a te kiterjesztett karoddal, és semmi sincs lehetetlen elõtted!
\par 18 A ki irgalmasságot cselekszel ezeríziglen, és a ki az atyák bûnéért az õ fiaik keblében fizetsz meg õ utánok, te nagy Isten, te hatalmas, a kinek neve Seregeknek Ura!
\par 19 Nagy tanácsú és hatalmas cselekedetû, a kinek szemei jól látják az emberek fiainak minden  útait, hogy kinek-kinek megfizess az õ útai szerint, és az õ cselekedeteinek gyümölcse szerint;
\par 20 A ki jeleket és csudákat tettél Égyiptom földén és mind e napiglan mind Izráel földén, mind az embereken, és nevet szerzettél magadnak, a mint ez mai nap is megvan.
\par 21 És kihoztad a te népedet, az Izráelt Égyiptom földébõl jelekkel és csudákkal, és hatalmas kézzel, és kinyújtott karral, és nagy rettegtetéssel.
\par 22 És nékik adtad e földet, a mely felõl megesküdtél az õ atyáiknak, hogy adsz nékik tejjel és mézzel folyó földet.
\par 23 És bementek és birtokolták azt, mindazáltal nem hallgattak a te szódra, és nem jártak a te törvényeidben, a miket parancsoltál nékik, hogy megcselekedjék, azokból semmit sem cselekedtek, azért mind e gonoszt rájok borítottad.
\par 24 Ímé, a sánczok a városhoz érnek, hogy bevegyék azt, és e város odaadatik a Káldeusok kezébe, a kik megostromolják ezt fegyverrel, éhséggel és döghalállal. És a mit szólottál, meglett, ímé látod is.
\par 25 És mégis azt mondottad, Uram Isten, nékem: Végy magadnak mezõt pénzen, és legyenek tanúid felõle, holott a város a Káldeusok kezébe adatik.
\par 26 És szóla az Úr Jeremiásnak, mondván:
\par 27 Ímé, én az Úr, Istene vagyok minden testnek, vajjon van-é valami lehetetlen nékem?
\par 28 Azért ezt mondja az Úr: Ímé, én odaadom e várost a Káldeusok kezébe, és Nabukodonozornak, a babiloni királynak kezébe, hogy bevegye azt.
\par 29 És bemennek a Káldeusok, a kik ostromolják e várost, és e várost felgyújtják tûzzel és felégetik azt, és a házakat is, a melyeknek tetején füstöltek a Baálnak és áldoztak idegen isteneknek, hogy engem haragra ingereljenek.
\par 30 Mert Izráel fiai és Júda fiai ifjúságoktól fogva csak azt cselekedték, a mi gonosz az én szemeim elõtt, és az Izráel fiai csak haragra gerjesztettek engem az õ kezeik cselekedeteivel, azt mondja az Úr.
\par 31 Mert csak bosszúságomra és búsulásomra volt e város, attól a naptól fogva, a melyen építették azt, mind ez ideig, úgy hogy el kell azt törlenem az én színem elõl.
\par 32 Az Izráel fiainak és a Júda fiainak minden bûnéért, a melyet cselekedtek, hogy felháborítsanak engem, õk magok, az õ királyaik, fejedelmeik, papjaik és prófétáik és Júda vitézei és Jeruzsálem polgárai.
\par 33 És háttal fordultak felém és nem arczczal, és tanítottam õket jó reggel, és noha tanítottam õket,  nem voltak készek az intés befogadására.
\par 34 Sõt az õ útálatosságaikat behelyezték a házba, a mely az én nevemrõl neveztetik, hogy megfertõztessék azt.
\par 35 És magaslatokat emeltek a Baálnak, a Ben-Hinnom völgyében, hogy megáldozzák fiaikat és leányaikat a Moloknak, a mit nem parancsoltam nékik, és még csak nem is gondoltam, hogy ez útálatosságot megcselekedjék, hogy Júdát vétekre vigyék.
\par 36 És most azért azt mondja az Úr, Izráel Istene e városnak, a mely felõl ti mondjátok: Odaadatik a babiloni király kezébe, fegyver, éhség és döghalál miatt:
\par 37 Ímé én összegyûjtöm õket mindama földekrõl, a melyekre kiûztem õket haragomban, felgerjedésemben és nagy bosszankodásomban, és visszahozom õket e helyre, és lakni hagyom õket bátorságban.
\par 38 És népemmé lesznek nékem, én pedig nékik Istenök leszek.
\par 39 És adok nékik egy szívet és egy útat, hogy mindenkor engem féljenek, hogy jól legyen dolguk, nékik és az õ fiaiknak õ utánok.
\par 40 És örökkévaló szövetséget kötök velök, hogy nem fordulok el tõlök és a velök való jótéteménytõl, és az én félelmemet adom az õ szívökbe, hogy el ne távozzanak tõlem.
\par 41 És örvendezek bennök, ha jót cselekedhetem velök és biztosan beplántálhatom  õket e földbe, teljes szívvel és teljes lélekkel.
\par 42 Mert ezt mondja az Úr: A miképen ráhoztam e népre mind e nagy veszedelmet, azonképen hozom rájok mind azt a jót, a miket én õ felõlök mondok.
\par 43 És vesznek még mezõt e földön, a mely felõl ti ezt mondjátok: Pusztaság ez emberek nélkül, barmok nélkül, és odaadatik a Káldeusok kezébe.
\par 44 Pénzen vesznek mezõket, és beírják a levélbe, és megpecsételik, és tanukat állítanak a Benjámin földén, Jeruzsálem környékén és Júda városaiban, a hegyi városokban és a síkföldi városokban és a dél felé való városokban, mert visszahozom az õ foglyaikat, azt mondja az Úr.

\chapter{33}

\par 1 Másodszor is szóla az Úr Jeremiásnak, mikor õ még fogva vala a tömlöcz pitvarában, mondván:
\par 2 Ezt mondja az Úr, a ki megteszi azt, az Úr, a ki megvalósítja azt, hogy megerõsítse azt, Úr az õ neve.
\par 3 Kiálts hozzám és megfelelek, és nagy dolgokat mondok néked, és megfoghatatlanokat, a melyeket nem tudsz.
\par 4 Mert ezt mondja az Úr, az Izráel Istene, e városnak házai és a Júda királyának házai felõl, a melyek lerontattak kosokkal és fegyverrel.
\par 5 Mikor elmentek, hogy vívjanak a Káldeusokkal, és hogy megtöltsék azokat emberek holttesteivel, a kiket én haragomban és bosszúállásomban megöltem, mivelhogy elrejtettem az én orczámat a várostól az õ sok gonoszságukért:
\par 6 Ímé, én hozok néki kötést és orvosságot, és meggyógyítom õket, és megmutatom nékik a békesség és hûség kincseit.
\par 7 És visszahozom Júdát és Izráelt a fogságból, és felépítem õket, mint azelõtt.
\par 8 És megtisztítom õket minden bûneiktõl, a melyekkel vétkeztek ellenem, és megbocsátom  minden bûneiket, a melyekkel vétkeztek ellenem, és a melyekkel gonoszul cselekedtek ellenem.
\par 9 És ez a város lészen nékem híremre, nevemre, örömömre, tisztességemre és dicséretemre e földnek minden nemzetsége elõtt, a kik hallják mindama jót, a melyet én cselekszem velök, és félni és rettegni fognak mindama jóért és mindama békességért, a melyet én szerzek nékik!
\par 10 Ezt mondja az Úr: Hallatszani fog még e helyen (a mely felõl ti ezt mondjátok: Pusztaság ez, emberek nélkül és barom nélkül való), a Júda városaiban és Jeruzsálem utczáiban, a melyek elpusztíttattak és ember nélkül és lakó nélkül és oktalan állat nélkül vannak,
\par 11 Örömnek szava és vígasság szava, võlegény szava és menyasszony szava, és azoknak szava, kik ezt mondják: Dícsérjétek a Seregek Urát, mert jó az Úr, mert örökkévaló az õ kegyelme; a kik hálaáldozatot hoznak az Úr házába, mert visszahozom e föld népét a fogságból, mint annakelõtte, azt mondja az Úr.
\par 12 Ezt mondja a Seregek Ura: E puszta helyen, a melyen nincs ember és barom, és ennek minden városában pásztorok fognak még lakozni, a kik az õ juhaikat terelgetik.
\par 13 A hegyi városokban, a síkföldi városokban, a dél felõl való városokban, a Benjámin földén, Jeruzsálem környékén és Júda városaiban még juhnyájak fognak átmenni a számlálónak keze alatt, azt mondja az Úr.
\par 14 Ímé, eljõnek a napok, azt mondja az Úr, és megbizonyítom az én jó szómat, a melyet az Izráel házának és a Júda házának szóltam.
\par 15 Azokban a napokban és abban az idõben sarjasztok Dávidnak igaz sarjadékot, és jogot és igazságot szerez e földön.
\par 16 Azokban a napokban megszabadul a Júda, és bátorságban lakozik Jeruzsálem, és így hívják majd õt: Az Úr a mi igazságunk.
\par 17 Mert ezt mondja az Úr: Nem vész ki a Dávid férfi sarjadéka, a ki az Izráel házának székébe üljön.
\par 18 És a lévita papok férfi sarjadéka sem vész ki elõlem, a ki égõáldozatot áldozzon, és ételáldozatot égessen, és véres áldozatot készítsen mindenkor.
\par 19 És szóla az Úr Jeremiásnak, mondván:
\par 20 Ezt mondja az Úr: Ha felbonthatjátok az én szövetségemet a nappal, és az én szövetségemet az éjszakával, hogy se nap, se éjszaka ne legyen az õ idejében:
\par 21 Az én szolgámmal, Dáviddal való szövetségem is felbomlik, hogy ne legyen fia, a ki uralkodjék az õ székében, és a lévita papokkal, az én szolgáimmal.
\par 22 Mint az ég serege meg nem számlálható, és a tenger fövenye meg nem mérhetõ, úgy megsokasítom az én szolgámnak, Dávidnak magvát, és a Lévitákat, a kik nékem szolgálnak.
\par 23 És szóla az Úr Jeremiásnak, mondván:
\par 24 Nem vetted észre, mit szóla e nép? mondván: A két nemzetséget, a melyet az Úr kiválasztott vala, elveté, és az én népemet megútálták úgy, hogy többé õ elõttök nem nemzet az.
\par 25 Ezt mondja az Úr: Ha szövetségem nem lesz a nappal és az éjszakával, és ha nem szabtam törvényeket az égnek és a földnek,
\par 26 Jákóbnak és az én szolgámnak, Dávidnak magvát is elvetem, úgy hogy az õ magvából senkit fel ne vegyenek, a ki uralkodjék Ábrahámnak, Izsáknak és Jákóbnak magván: mert visszahozom õket a fogságból, és megkegyelmezek nékik.

\chapter{34}

\par 1 Az a beszéd, a melyet szólott az Úr Jeremiásnak, (mikor Nabukodonozor a babiloni király és az egész serege és a föld minden országa, a melyek az õ hatalma alatt valának, és a népek mind vívták vala Jeruzsálemet és minden városát) mondván:
\par 2 Ezt mondja az Úr, az Izráel Istene: Menj el, és mondd meg Sedékiásnak, a Júda királyának, és így szólj néki: Ezt mondja az Úr: Ímé, én odaadom e várost a babiloni király kezébe, és felgyújtja ezt tûzzel.
\par 3 És te el nem szaladsz az õ kezébõl, hanem bizonynyal megfognak és kezébe adnak, és a te szemeid meglátják a babiloni királynak szemeit, és az õ szája a te száddal szól, és bemégy Babilonba.
\par 4 Mindazáltal halld meg az Úrnak szavát Sedékiás, Júda királya; ezt mondja az Úr te felõled: Nem halsz meg fegyver által.
\par 5 Békességben halsz meg, és a mint füstöltek a te atyáidnak, az elõbbi királyoknak, a kik te elõtted voltak, úgy füstölnek néked is, és így siratnak téged: Jaj uram! Mert én szóltam e szót, azt mondja az Úr.
\par 6 És megmondá Jeremiás próféta Sedékiásnak, a Júda királyának, mind e szavakat Jeruzsálemben.
\par 7 A babiloni király serege pedig vívja vala Jeruzsálemet és Júdának minden városát, a melyek megmaradtak vala, tudniillik Lákist és Azekát, mert a Júda városai közül csak ezek maradtak vala meg, mint erõsített városok.
\par 8 Ez a beszéd, melyet szóla az Úr Jeremiásnak, minekutána Sedékiás király szövetséget köte az egész Jeruzsálembeli néppel, szabadságot hirdetvén köztük.
\par 9 Hogy kiki bocsássa szabadon szolgáját és kiki az õ szolgálóleányát a héber férfit és a héber leányt, hogy senki ne szolgáltasson közöttük az õ Júdabeli atyjafiával.
\par 10 És engedelmeskedtek mindnyájan a fejedelmek és az egész nép, a kik szövetséget kötöttek, hogy kiki szabadon bocsássa az õ szolgáját és kiki az õ szolgálóleányát, hogy senki azokkal ne szolgáltasson többé; és engedelmeskedtek és elbocsáták azokat.
\par 11 De azután elváltozának, és visszahozák a szolgákat és szolgálóleányokat, a kiket szabadon bocsátottak vala, és õket szolgákká és szolgálóleányokká tevék.
\par 12 És lõn az Úrnak szava Jeremiáshoz az Úrtól, mondván:
\par 13 Ezt mondja az Úr, Izráel Istene: Én szövetséget kötöttem a ti atyáitokkal azon a napon, a melyen kihoztam õket Égyiptom földérõl, a szolgálatnak házából, mondván:
\par 14 Mikor a hét esztendõ eltelik, kiki bocsássa el az õ héber atyjafiát, a ki néked eladatott vala és hat esztendeig szolgált téged; bocsássad õt magadtól szabadon. De nem hallgatának a ti atyáitok engemet, és fülöket sem hajtották erre.
\par 15 És ti ma megtértetek vala, és igazat cselekedtetek vala én elõttem, kiki szabadságot hirdetvén az õ atyjafiának, és én elõttem szövetséget kötöttetek abban a házban, a mely az én nevemrõl neveztetett.
\par 16 De elváltoztatok, és beszenynyeztétek az én nevemet, és kiki visszahozta az õ szolgáját és kiki az õ szolgálóleányát, kiket egészen szabadon bocsátottatok vala, és igába vetettétek õket, hogy néktek szolgáitok és szolgáló leányaitok legyenek.
\par 17 Azért ezt mondja az Úr: Ti nem hallgattatok reám, hogy kiki szabadságot hirdessen az õ atyjafiának és kiki az õ felebarátjának. Ímé, én hirdetek néktek szabadságot, azt mondja az Úr, a fegyverre, a döghalálra és az éhségre, és odaadlak titeket e föld minden országainak útálatára.
\par 18 És odaadom a férfiakat, a kik megszegték az én szövetségemet, a kik nem teljesítették a szövetség pontjait, a melyet elõttem kötöttek vala tulokkal, a melyet ketté vágának és átmenének annak részei között,
\par 19 Júdának fejedelmeit és Jeruzsálem fejedelmeit, az udvari szolgákat és a papokat és a földnek minden népét, a kik átmentek a tulok részei között:
\par 20 Odaadom õket az õ ellenségeik kezébe, és az õ lelköket keresõk kezébe, és az õ holttestök ez égi madaraknak és a föld vadainak lesznek eledelévé.
\par 21 Sedékiást, a Júda királyát és az õ fejedelmeit is odaadom az õ ellenségeiknek kezébe, és az õ lelköket keresõk kezébe, és a babiloni király seregének kezébe, a mely  eltávozik tõletek.
\par 22 Ímé, én parancsolok, azt mondja az Úr, és visszahozom õket e városra, és vívják azt, és beveszik és felgyújtják tûzzel, és pusztasággá teszem Júda városait, lakhatatlanokká.

\chapter{35}

\par 1 Az a beszéd, a melyet szóla az Úr Jeremiásnak, Jojákimnak, Jósiás, Júda királya fiának idejében, mondván:
\par 2 Menj el a Rékábiták házához, és szólj velök, és vidd be õket az Úr házába, a kamarák egyikébe, és adj nékik bort inni.
\par 3 És mellém vevém Jaazániát, Jeremiásnak fiát, ki fia vala Habasániának, és az õ rokonait és minden fiait és a Rékábiták egész háznépét.
\par 4 És bevivém õket az Úr házába, a Hanán fiainak a kamarájába ki Igdaliásnak, az Isten emberének fia vala, a mely a fejedelmek kamarája mellett vala, Mahásiásnak, Sallum fiának kamaráján felül, a ki az ajtónak õrizõje vala.
\par 5 És a Rékábiták háza népének fiai elé borral telt kancsókat és poharakat tevék, és ezt mondám nékik: Igyatok bort!
\par 6 És felelének: Nem iszunk bort, mert Jónadáb, Rékábnak fia, a mi atyánk parancsolta nékünk, mondván: Ne igyatok ti bort soha, se a ti fiaitok.
\par 7 Se házat ne építsetek, se vetést ne vessetek, se szõlõt ne ültessetek, se ne tartsatok; hanem sátorokban lakjatok teljes életetekben, hogy sok ideig éljetek e földnek színén, a melyben ti jövevények vagytok.
\par 8 És hallgattunk Jónadábnak a Rékáb fiának, a mi atyánknak szavára mindabban, a miket parancsolt nékünk, hogy teljes életünkben bort ne igyunk mi, a mi feleségeink, a mi fiaink és a mi leányaink.
\par 9 Se házakat ne építsünk, hogy azokban lakjunk, se szõlõnk, se mezõnk, se vetésünk ne legyen nékünk.
\par 10 Hanem lakozzunk sátorokban. Hallgattunk azért, és a szerint cselekedtünk, a mint nékünk Jónadáb, a mi atyánk megparancsolta vala.
\par 11 Mikor pedig feljöve Nabukodonozor, a babiloni király a földre, akkor ezt mondánk: Jertek el, menjünk be Jeruzsálembe a káldeai sereg elõtt és a Siriabeli sereg elõtt; és Jeruzsálemben lakoztunk.
\par 12 És szóla az Úr Jeremiásnak, mondván:
\par 13 Így szól a Seregek Ura, az Izráel Istene: Menj el, mondd meg a Júda fériainak és Jeruzsálem lakosainak: Nem veszitek-é fel az intést, hogy hallgassatok az én beszédeimre? azt mondja az Úr.
\par 14 Jónadábnak, a Rékáb fiának intései teljesedtek, a melyekkel megparancsolta az õ fiainak, hogy bort ne igyanak, és mindez ideig sem ittak bort; mert hallgattak az õ atyjok parancsolatjára; én is szóltam néktek; szóltam pedig jó reggel, de nem engedtetek nékem.
\par 15 És elküldtem hozzátok minden én szolgámat, a prófétákat, és pedig jó reggel küldém, mondván: Kérlek, kiki térjen meg az õ gonosz útjáról, jobbítsátok meg cselekedeteiteket, és idegen istenek után ne járjatok, hogy nékik szolgáljatok, és lakoztok a földön, a melyet néktek és a ti atyáitoknak adtam, de fületeket sem hajtátok reá, és nem hallgattatok reám.
\par 16 Mivelhogy Jónadábnak, a Rékáb fiának fiai teljesítik az õ atyjoknak parancsolatját, melyet parancsolt vala nékik, e nép pedig nem hallgata reám;
\par 17 Azért ezt mondja az Úr, a Seregek Istene, az Izráel Istene: Ímé, én rábocsátom Júdára és Jeruzsálemnek minden lakóira mindama veszedelmet, a melyrõl szóltam nékik; azért mert szóltam nékik, de nem hallották, kiáltottam nékik, de nem feleltek.
\par 18 A Rékábiták házának pedig monda Jeremiás: Ezt mondja a Seregek Ura, az Izráel Istene: Mivelhogy hallgattatok Jónadábnak, a ti atyátoknak parancsolatjára, és megtartottátok minden parancsolatját, és úgy cselekedtetek, a mint meghagyta volt néktek:
\par 19 Azért ezt mondja a Seregek Ura, az Izráel Istene: Nem fogyatkozik el Jónadábnak, a Rékáb fiának maradéka, a ki elõttem álljon mindenkor.

\chapter{36}

\par 1 És Jojákimnak, a Jósiás fiának, Júda királyának negyedik esztendejében is szóla az Úr Jeremiásnak, mondván:
\par 2 Végy elõ egy könyvet, és mind írd belé a szókat, a miket én szóltam néked az Izráel és a Júda ellen és minden nemzet ellen, a naptól fogva, a melyen szóltam néked a Jósiás ideje óta mind e napig.
\par 3 Hátha meghallja a Júda háza mindazokat a veszedelmeket, a melyeket én néki szerezni szándékozom, hogy kiki megtérjen az õ gonosz útáról, és megbocsássam az õ bûnöket és vétköket.
\par 4 És elõhívá Jeremiás Bárukot, a Néria fiát, és megírá Báruk a Jeremiás szája után a könyvbe mindama szókat, a melyeket az Úr szólott vala néki.
\par 5 És parancsola Jeremiás Báruknak, mondván: Én fogoly vagyok, nem mehetek be az Úr házába.
\par 6 Azért te menj be, és a könyvbõl, a melyet az én számból írtál, olvasd el az Úrnak szavait a népnek hallatára az Úrnak házában bõjti napon, és az egész Júdának hallatára, a kik felgyülnek az õ városaikból, olvasd el nékik.
\par 7 Hátha az Úr elé száll könyörgésük, és mindenki megtér az õ gonosz útáról, mert nagy az Úr haragja és felháborodása, a melylyel szólott az Úr e nép ellen!
\par 8 Báruk pedig, a Néria fia, mind a szerint cselekedék, a mint néki Jeremiás próféta megparancsolta vala, elolvasván a könyvbõl az Úrnak szavát, az Úrnak házában.
\par 9 És Jojákimnak a Jósiás fiának, Júda királyának ötödik esztendejében, a kilenczedik hónapban bõjtre hívták fel az Úr elé az egész Jeruzsálem népét és az egész népet, a mely Júda városaiból jött vala fel Jeruzsálembe.
\par 10 És elolvasá Báruk a könyvbõl Jeremiás beszédeit az Úr házában Gamáriának, az írástudó Sáfán fiának szobájában a felsõ pitvarban, az Úr háza új kapujának nyílásában az egész nép hallatára.
\par 11 És hallá Mikeás, Gamáriának, a Sáfán fiának fia az Úrnak minden beszédét a könyvbõl.
\par 12 És leméne a király házába az írástudó szobájába, és ímé, ott ülnek vala mind a fõemberek, Elisáma, az írástudó, és Dalajás, Semájának fia, és Elnátán, Akbórnak fia, és Gamária, Sáfánnak fia, és Sedékiás, Hanániásnak fia, és a többi fõember is.
\par 13 És elbeszélé nékik Mikeás mindazokat, a melyeket hallott vala a könyvbõl, mikor olvasá Báruk a nép hallatára.
\par 14 Azért elküldék mind a fõemberek Jéhudit, a ki Natániának fia vala, a ki Selémiának fia vala, a ki Kusinak fia vala, Bárukhoz, mondván: A könyvet, a melybõl olvastál a nép hallatára, vedd kezedbe, és jövel. És kezébe vevé Báruk, a Néria fia a könyvet, és elméne hozzájok.
\par 15 És mondának néki: Ülj le csak és olvasd azt a mi fülünk hallatára, és elolvasá Báruk fülök hallatára.
\par 16 És mikor meghallák mind e szókat, megrettenve tekintettek egymásra, és mondának Báruknak: Bizony megjelentjük mind e szókat a királynak.
\par 17 Bárukot pedig kérdezék, mondván: Jelentsd meg csak nékünk, mimódon írtad mind e szókat az õ szájából?
\par 18 És monda nékik Báruk: Szájával mondotta nékem mind e szókat, én pedig beírtam e könyvbe tintával.
\par 19 És mondának a fõemberek Báruknak: Menj el, rejtõzzél el te és Jeremiás, és ne tudja senki, hol vagytok.
\par 20 És elmenének a királyhoz a pitvarba (és letevék a könyvet az írástudó Elisámának szobájában) és elmondák mind e szókat a király hallatára.
\par 21 És elküldé a király Jéhudit, hogy hozza el a könyvet; azért elhozá azt az írástudó Elisámának szobájából, és elolvasá azt Jéhudi a király hallatára és mindama fõemberek hallatára, a kik a király elõtt állnak vala.
\par 22 A király pedig a téli házban ül vala a kilenczedik hónapban, és a tûz ég vala elõtte.
\par 23 És mikor Jéhudi három vagy négy levelet elolvasott vala, elmetélé azt az írástudónak késével, és a tûzbe hajítá, a mely a tûzhelyen vala, mígnem az egész könyv megége a tûzben, a mely a tûzhelyen vala.
\par 24 Nem rettentek meg, sem ruhájokat nem szaggatták meg a király és valamennyi szolgája, a kik hallják vala mind e szókat.
\par 25 Sõt még Elnátán is és Delája és Gamária kérék a királyt, hogy a könyvet ne égesse meg, de nem hallgata rájok.
\par 26 Hanem meghagyá a király Jerákmeélnek, a Hammélek fiának, és Sérajának, az Azriel fiának, és Selémiának, az Abdéel fiának, hogy fogják el Bárukot, az írástudót, és Jeremiás prófétát, de az Úr elrejté õket.
\par 27 Szóla pedig az Úr Jeremiásnak, miután a király megégette vala a könyvet és beszédeket, a melyeket Báruk a Jeremiás szájából írt vala, mondván:
\par 28 Térj vissza, végy magadnak más könyvet, és írd bele mind az elébbi szókat, a melyek az elébbi könyvben valának, a melyet Jojákim, a Júda királya megégetett.
\par 29 És Jojákimnak, a Júda királyának mondd meg: Ezt mondja az Úr: Te égetted meg a könyvet, mondván: Miért írtál ilyen szókat bele: Bizonyosan eljõ a babiloni király, és elveszti e földet, és kipusztít belõle embert és állatot?
\par 30 Azért ezt mondja az Úr Jojákim felõl, a Júda királya felõl: Nem lesz néki, a ki a Dávid székébe üljön, és az õ holtteste elvettetik nappal a hévre, éjszaka pedig a dérre.
\par 31 És megbüntetem õt, és az õ magvát, és az õ szolgáit az õ bûneikért, és rájok bocsátom és a Jeruzsálembeli polgárokra és Júdának férfiaira mind azt a veszedelmet, a melyrõl szólottam nékik, de nem hallgattak meg.
\par 32 Azért más könyvet võn Jeremiás, és adá azt Báruknak, az írástudó Néria fiának, és beírá abba a Jeremiásnak szájából minden szavát annak a könyvnek, a melyet Jojákim, a Júda királya megégetett vala a tûzben. És több dolgot is írának bele, hasonlókat az elébbiekhez.

\chapter{37}

\par 1 Uralkodott pedig Sedékiás király, Jósiásnak fia, Kónia helyett, a ki Jojákimnak fia vala, kit Nabukodonozor, a babiloni király királylyá tett vala Júdának földében.
\par 2 De nem hallgatá sem õ, sem az õ szolgái, sem a föld népe az Úrnak szavát, a melyet szólott vala Jeremiás próféta által.
\par 3 És elküldé Sedékiás király Júkált, Selémiának fiát, és Sofóniást, Mahásiás papnak fiát Jeremiás prófétához, mondván: Kérlek, könyörögj mi érettünk az Úrnak, a mi Istenünknek.
\par 4 Jeremiás pedig be- és kimegy vala a nép között, mert még nem vetették vala õt be a tömlöczbe.
\par 5 A Faraó serege pedig kijött Égyiptomból, és a Káldeusok, a kik megszállották Jeruzsálemet, meghallották e hírt õ felõlök, és elhagyták Jeruzsálemet.
\par 6 És szóla az Úr Jeremiás prófétának, mondván:
\par 7 Ezt mondja az Úr, Izráel Istene: Ezt mondjátok a Júda királyának, a ki elküldött titeket én hozzám, hogy megkérdezzetek engem: Ímé, a Faraó serege, a mely kijött a ti segítségetekre, visszamegy Égyiptom földébe.
\par 8 És visszatérnek a Káldeusok, és ostromolják e várost, és beveszik azt, és megégetik tûzzel.
\par 9 Ezt mondja az Úr: Ne csaljátok meg magatokat, mondván: Bizonyoson elmennek mi rólunk a Káldeusok; mert nem mennek el.
\par 10 Mert ha a Káldeusoknak egész seregét megveritek is, a kik ostromolnak titeket, és közülök csak néhány megsebesült marad is: valamennyi felkél az õ sátorából, és megégetik e várost tûzzel.
\par 11 És mikor a Káldeusok serege elvonula Jeruzsálem alól a Faraó serege miatt,
\par 12 Jeremiás is kiindula Jeruzsálembõl, menvén a Benjámin földére, hogy osztályrészt kapjon onnan a nép között.
\par 13 Mikor pedig õ a Benjámin kapujához ére és ott vala egy õr, a kinek neve Jériás vala, Selémiásnak fia, a ki Hanániásnak fia vala, megfogá az Jeremiás prófétát, és monda: Te a Káldeusokhoz szököl!
\par 14 És monda Jeremiás: Hazugság! Nem szököm a Káldeusokhoz; de nem hallgata rá. És Jériás megfogá Jeremiást, és vivé õt a fejedelmekhez.
\par 15 És a fejedelmek megharaguvának Jeremiásra, és megvereték õt, és veték a fogházba az írástudó Jónatán házába: mert azt rendelték vala fogháznak.
\par 16 Mikor Jeremiás a tömlöczbe és a börtönbe juta, és sok napig vala ott Jeremiás;
\par 17 Akkor elkülde Sedékiás király és elõhozatá õt, és megkérdezé a király a maga házánál titkon, és monda: Van-é kijelentésed az Úrtól? Akkor monda Jeremiás: Van! És azt is mondá: A babiloni király kezébe adatol.
\par 18 Azután monda Jeremiás Sedékiás királynak: Mit vétettem ellened és a te szolgáid ellen és e nép ellen, hogy a fogházba vetettetek engem?
\par 19 És hol vannak a ti prófétáitok, a kik prófétáltak néktek, mondván: Nem jõ el a babiloni király ti ellenetek és e föld ellen?
\par 20 Most halld csak, uram király, és hallgasd meg az én könyörgésemet, és ne küldj engem vissza az írástudó Jónatán házába, hogy ott ne haljak meg.
\par 21 Parancsolta azért Sedékiás király, hogy vessék Jeremiást a tömlöcz pitvarába, és adjanak néki naponként egy-egy darab kenyeret a sütõk utczájából, a míg minden kenyér elfogy a városból. És ott marada Jeremiás a tömlöcz pitvarában.

\chapter{38}

\par 1 De meghallá Safátiás Mattánnak fia, és Gedáliás Passúrnak fia, és Jukál a Selémiás fia, és Passúr a Melékiás fia a szókat, a melyeket szólott vala Jeremiás az egész népnek, mondván:
\par 2 Ezt mondja az Úr: A ki megmarad e városban, meghal fegyver miatt, éhség miatt és döghalál miatt, a ki pedig kimegy a Káldeusokhoz, él, és az õ élete nyereség lesz néki és él.
\par 3 Ezt mondja az Úr: Bizonynyal a babiloni király seregének kezébe adatik e város, és beveszi azt.
\par 4 És mondának a fejedelmek a királynak: Kérünk, ölettesd meg ezt az embert, mert megerõtleníti a vitézek kezeit, a kik megmaradtak a városban, és az egész nép kezeit, hogy efféle szókat szól nékik, mert ez az ember nem a nép megmaradására igyekszik, hanem veszedelmére.
\par 5 És monda Sedékiás király: Ám a ti kezetekben van, mert a király semmit sem tehet ellenetekre.
\par 6 Azért elvivék Jeremiást, hogy bevessék Melkiásnak, a Hammélek fiának vermébe, a mely a tömlöcz pitvarában vala; és lebocsáták Jeremiást köteleken; a veremben pedig nem víz vala, hanem sár, és beesék Jeremiás a sárba.
\par 7 És meghallotta Ebed-Melek, a szerecsen, a ki udvari szolga vala (õ pedig a király házában vala), hogy Jeremiást a verembe vetették, a király pedig a Benjámin-kapuban ül vala.
\par 8 Kiméne azért Ebed-Melek a király házából, és szóla a királynak, mondván:
\par 9 Uram, király! gonoszul cselekedtek azok az emberek mindazzal, a mit Jeremiás prófétával cselekedtek, a kik õt a verembe vetették; mert meghal ott éhen, mert nem lesz ezután semmi kenyér e városban.
\par 10 Parancsola azért a király Ebed-Meleknek, a szerecsennek, mondván: Végy magadhoz innét harmincz embert, és vedd fel Jeremiás prófétát a verembõl, mielõtt meghalna.
\par 11 Võn azért Ebed-Melek magához harmincz embert, és beméne a király házába, a kincstartó ház alá, és hoza onnét régi ruhadarabokat és elszakadozott posztókat, és alábocsátá azokat Jeremiásnak köteleken a verembe.
\par 12 És monda Ebed-Melek, a szerecsen, Jeremiásnak: Tedd a régi és elszakadozott ruhadarabokat hónod alá, a kötelek alá; és úgy cselekedék Jeremiás.
\par 13 Kivonták azért Jeremiást köteleken és kihozák õt a verembõl, és lakék Jeremiás a tömlöcz pitvarában.
\par 14 Elkülde pedig Sedékiás király, és magához hozatá Jeremiás prófétát a harmadik ajtóig, mely vala az Úrnak házában, és monda a király Jeremiásnak: Téged valamirõl kérdelek, semmi tagadást benne ne tégy!
\par 15 Monda pedig Jeremiás Sedékiásnak: Ha megjelentem néked, avagy nem bizonyosan megölsz-é engem? és ha tanácsot adok, nem hallgatsz rám.
\par 16 És megesküvék Sedékiás király Jeremiásnak titkon, mondván: Él az Úr, a ki teremtette nékünk e lelket, hogy nem öllek meg és nem adlak azoknak az embereknek kezébe, a kik keresik a te lelkedet!
\par 17 Akkor monda Jeremiás Sedékiásnak: Ezt mondja az Úr, a Seregek Istene, az Izráel Istene: Ha kimégy a babiloni király fejedelmeihez, él a te lelked, és e város nem égettetik meg tûzzel, hanem élsz te és a te házad népe.
\par 18 Ha pedig nem mégy ki a babiloni király fejedelmeihez, akkor e város a Káldeusok kezébe adatik, felégetik ezt tûzzel, te sem szaladsz el kezökbõl.
\par 19 Monda Sedékiás király Jeremiásnak: Félek én a Júdabeliektõl, a kik átszöktek a Káldeusokhoz, hátha azok kezébe adnak engem, és csúfoskodnak rajtam!
\par 20 És monda Jeremiás: Nem adnak; kérlek, halld meg az Úrnak szavát, a melyet én mondok néked, és jó dolgod lesz, és él a te lelked.
\par 21 Ha pedig te kimenni nem akarsz: ez a szó, a melyet megjelentett nékem az Úr:
\par 22 Ímé, minden asszony, a ki megmaradt vala a Júda királyának házában, kivitetik a babiloni király fejedelmeihez; és ezt mondják azok, hogy megcsaltak téged és erõt vettek rajtad a te jóakaró embereid, a te lábaid most beragadtak a sárba, õk pedig visszafordultak.
\par 23 Azért minden feleségedet és gyermekedet kiviszik a Káldeusoknak, te sem menekedel meg kezökbõl, hanem megfogatol a babiloni király kezével, és e várost felégeti tûzzel.
\par 24 Monda pedig Sedékiás Jeremiásnak: Senki se tudjon e szókról, és nem halsz meg!
\par 25 Ha meghallják a fejedelmek, hogy beszéltem veled, és eljõnek hozzád és ezt mondják néked: Mondd meg csak nékünk, mit beszéltél a királynak, ne tagadj el abból tõlünk és nem ölünk meg téged, és mit monda néked a király?
\par 26 Ezt mondd nékik: Alázatosan könyörgék a királynak, hogy ne vitessen vissza Jónatán házába, hogy meg ne haljak ott.
\par 27 És a fejedelmek mind elmenének Jeremiáshoz és megkérdék õt, és egészen úgy felele nékik, a mint a király parancsolta vala: és hallgatással elmenének tõle, mert nem hallották vala a beszédet.
\par 28 És ott marada Jeremás a tömlöcz pitvarában mind a napig, a melyen bevevék Jeruzsálemet, és ott vala, mikor bevevék Jeruzsálemet.

\chapter{39}

\par 1 Sedékiásnak, a Júda királyának kilenczedik esztendejében, a tizedik hónapban eljöve Nabukodonozor, a babiloni király és egész serege Jeruzsálem ellen, és megszállák azt.
\par 2 Sedékiás tizenegyedik esztendejében, a negyedik hónapban, a hónap kilenczedikén ledûle a város kõfala.
\par 3 És bemenének a babiloni király fejedelmei mind és leülének a középsõ kapuban: Nergál-Sarézer, Samegár-Nebó, Sársekim, Rabsáris, Nergál-Sarézer, Rabmág és mind a többi fejedelmei a babiloni királynak.
\par 4 És mikor meglátta vala õket Sedékiás, a Júda királya és mind a vitézlõ férfiak, elfutamodának és kimenének éjjel a városból a király kertjén át az ajtón, a két kõfal között, és kimenének a pusztába vivõ úton.
\par 5 És ûzék õket a káldeai seregek, és elfogák Sedékiást Jerikhó pusztájában, és elhozák õt és elvivék Nabukodonozornak, a babiloni királynak Riblába, Hamát földére, és ítéletet monda rája.
\par 6 És megölé a babiloni király Sedékiásnak fiait szeme láttára Riblában, és Júdának minden nemeseit is megölé a babiloni király.
\par 7 A Sedékiás szemeit pedig kitolatá, és vasba vereté õt, hogy elvigye õt Babilonba.
\par 8 A király házát pedig és a nép házait felgyújták a Káldeusok tûzzel, és Jeruzsálem kõfalait leronták.
\par 9 A nép többi részét pedig, a mely a városban maradt vala: és a szökevényeket, a kik hozzá szöktek vala, a nép többi részét, a még megmaradottakat elvivé Nabuzáradán, a poroszlók feje, Babilonba.
\par 10 A nép szegényeit pedig, a kiknek semmijök sem vala, ott hagyá Nabuzáradán, a poroszlók feje, Júda földében, és ada nékik szõlõket és szántóföldeket azon a napon.
\par 11 Jeremiás felõl pedig parancsot ada Nabukodonozor, a babiloni király Nabuzáradánnak, a poroszlók fejének, mondván:
\par 12 Vedd õt magadhoz, és viselj gondot reá, és semmi bajt ne okozz néki, hanem azt cselekedd vele, a mit õ akar.
\par 13 És elkülde Nabuzáradán, a poroszlók feje, és Nebusázban, Rabsáris és Nergál-Sarézer, Rabmág és a babiloni királynak több fõembere.
\par 14 Elküldének, mondom, és elhozák Jeremiást a tömlöcz pitvarából, és rábízák õt Gedáliásra, Ahikámnak, a Sáfán fiának fiára, hogy haza vigye õt, és lakozzék a nép között.
\par 15 Az Úr pedig szóla Jeremiáshoz, mikor õ még a tömlöcz pitvarában fogva vala, mondván:
\par 16 Menj el, és szólj Ebed-Melekkel, a szerecsennel, mondván: Ezt mondja a Seregek Ura, Izráel Istene: Ímé, én beteljesítem e város kárára és nem javára mondott beszédeimet, és azon a napon szemeid elõtt lesznek azok.
\par 17 És azon a napon megszabadítlak téged, azt mondja az Úr, és nem adatol amaz emberek kezébe, a kiktõl félsz.
\par 18 Hanem bizonyára megszabadítlak téged, nem esel el fegyver miatt, és a lelked zsákmányul lesz néked, mert reménységed volt bennem, azt mondja az Úr.

\chapter{40}

\par 1 Az a szózat, a melyet az Úr szóla Jeremiásnak, miután Nabuzáradán, a poroszlók feje elbocsátá õt Rámából. Mikor elvitte vala õt, õ is lánczokkal vala megkötözve a jeruzsálemi és júdai mindenféle foglyok között, a kik Babilonba vitetnek vala.
\par 2 És a poroszlók feje elvivé Jeremiást, és monda néki: Az Úr, a te Istened rendelte ezt a büntetést e hely ellen.
\par 3 És ráhozta és megcselekedte az Úr, a mint megmondotta vala. Mert vétkeztetek az Úr ellen, és nem hallgattatok az õ szavára, azért teljesedett be ti rajtatok e dolog.
\par 4 Mostan azért ímé, én téged ma megszabadítalak a láncztól, a mely a te kezeiden van; ha tetszik néked Babilonba jõnöd, jõjj velem, nékem pedig gondom lesz reád, ha pedig nem tetszik néked, hogy eljõjj velem Babilonba, maradj itt. Ímé, az egész föld elõtted van, a hova jobbnak és helyesebbnek látszik menned, menj oda.
\par 5 (De õ még nem tér vala vissza.) Vagy menj vissza Gedáliáshoz, Ahikámnak, Sáfán fiának fiához, a kit a babiloni király tiszttartóvá tett Júda városaiban, és lakjál vele a nép között, vagy akárhová tetszik menned, oda menj. És ada a poroszlók feje néki étket és ajándékot, és elbocsátá õt.
\par 6 És elméne Jeremiás Gedáliáshoz, Ahikámnak fiához Mispába, és ott lakék vele a nép között, a kik megmaradtak a földön.
\par 7 És mikor meghallotta a seregek minden vezetõje, a kik a mezõn valának, õk magok és az õ embereik, hogy a babiloni király Gedáliást, Ahikámnak fiát tette tiszttartóvá az országban, és hogy reá bízta a férfiakat és az asszonyokat, a kisdedeket és a föld szegényeit azok közül, a kik Babilonba el nem vitettek vala;
\par 8 Elmenének Gedáliáshoz Mispába, tudniillik Ismáel, Natániának fia, és Jóhanán és Jónatán, Káreának fiai, és Serája, Tankumetnek fia, és a netofáti Efainak fiai, és Jazánia, Maakátnak fia, õk és az õ embereik.
\par 9 És Gedáliás, Ahikámnak, Sáfán fiának fia megesküvék nékik és az õ embereiknek, mondván: Ne féljetek a Káldeusoknak szolgálni, hanem lakozzatok e földön és szolgáljatok a babiloni királynak, és jó dolgotok lesz.
\par 10 És én ímé Mispában lakom, hogy szolgálatukra álljak a Káldeusoknak, a kik eljõnek hozzánk, ti pedig szedjetek össze bort és gyümölcsöt, és olajt is szerezzetek edényeitekbe, és lakjatok a ti városaitokban, a melyeket elfoglaltatok.
\par 11 És mindazok a Júdabeliek is, a kik a Moábitáknál, az Ammon fiainál, az Edomitáknál, és a kik akármely tartományban valának, meghallák, hogy a babiloni király Júdából maradékot hagyott, és hogy Gedáliást, Ahikámnak, Sáfán fiának fiát tette elõttök tiszttartóvá.
\par 12 Azért haza jövének mindnyájan a Júdabeliek mindenünnen, a hova elfutottak vala, és eljövének a Júda földébe Gedáliáshoz Mispába, és bort és igen sok gyümölcsöt szerzének össze.
\par 13 És Jóhanán, Káreának fia, és a seregek minden vezetõje, a kik a mezõn valának, szintén elmenének Gedáliáshoz Mispába.
\par 14 És mondának néki: Nem tudod-é, hogy Baálisz, az Ammoniták királya elküldötte Ismáelt, Natániának fiát, hogy téged megöljön? De nem hive azoknak Gedáliás, Ahikámnak fia.
\par 15 Jóhanán pedig, Káreának fia, titkolózva monda Gedáliásnak Mispában, mondván: Elmegyek és megölöm Ismáelt, Natániának fiát úgy, hogy senki se tudja. Miért oltaná el a te életedet? Hiszen a Júdabeliek, a kik te hozzád gyülekeztek vala, mind eloszlanak, és Júdának maradéka is elvész.
\par 16 És monda Gedáliás, Ahikámnak fia Jóhanánnak, a Kárea fiának: Ne cselekedd e dolgot, mert te hazugságot szólasz Ismáel felõl!

\chapter{41}

\par 1 És a hetedik hónapban csakugyan eljöve Ismáel, Natániának, Elisáma fiának fia, a ki királyi nembõl és a király fõemberei közül való vala, és tíz férfiú vele Gedáliáshoz, Ahikámnak fiához Mispába, és együtt étkezének Mispában.
\par 2 És felkele Ismáel, Natániának fia és a tíz férfiú, a kik vele valának, és megölék Gedáliást, Ahikámnak, a Sáfán fiának fiát szablyával, és megölé õ azt, a kit a babiloni király tiszttartóvá tett vala a földön!
\par 3 És mindazokat a Júdabelieket is, a kik Gedáliással valának Mispában, és a Káldeusokat is, a kik ott találtatának, tudniillik a vitézlõ embereket megölé Ismáel.
\par 4 Másnap pedig a Gedáliás megölése után, még mikor senki sem tudta azt:
\par 5 Férfiak jövének Sikembõl, Silóból és Samariából: nyolczvan férfiú levágott szakállal, megszaggatott ruhában és bevagdalt testtel, kezökben ételáldozat és tömjén, hogy áldozzanak az Úr házában.
\par 6 És kiméne eléjök Ismáel, Natániának fia Mispából, menés közben is sírván. És mikor eléjök ért vala, monda néki: Jertek el Gedáliáshoz, Ahikámnak fiához.
\par 7 És mikor bementek vala a városba, megölé õket Ismáel, Natániának fia, és hányá õket az árokba, õ és a férfiak, a kik vele valának.
\par 8 De tíz ember találtaték õ közöttük, a kik ezt mondák Ismáelnek: Ne ölj meg minket, mert kincsünk van nékünk a mezõn búza és árpa, olaj és méz és megtartóztatá magát, és nem ölé meg õket az õ atyjokfiaival.
\par 9 Az árok pedig, a melybe behányá Ismáel mindamaz emberek holttestét, a kiket megöle Gedáliással együtt, az, a melyet Asa király csinált vala Baása ellen, az Izráel királya ellen. Ismáel, Natániának fia megtölté azt a megölettekkel.
\par 10 És fogságra vivé el Ismáel a nép egész maradékát, a mely Mispában vala, a király leányait és az egész népet, a mely Mispában hagyatott vala, kiket Nabuzáradán, a poroszlók feje Gedáliásra, Ahikámnak fiára bízott vala; foglyul ejté azért õket Ismáel, Natániának fia, és elindula, hogy az Ammon fiaihoz menjen.
\par 11 És meghallá Jóhanán, Káreának fia és a seregek minden vezetõje, a ki õ vele vala, azt az egész gonoszságot, a melyet Ismáel, Natániának fia cselekedett vala.
\par 12 És magokhoz vevék mind az embereket, és elmenének, hogy megvívjanak Ismáellel, Natániának fiával, és elérték õt a nagy víznél, a mely Gibeonnál vala.
\par 13 És mikor meglátta vala az egész nép, a mely Ismáellel vala, Jóhanánt, Káreának fiát és a seregnek minden vezetõjét, a kik vele valának, megörüle.
\par 14 És visszatére az egész nép, a melyet Ismáel elvitt vala Mispából, és visszatére és elméne Jóhanánhoz, Káreának fiához.
\par 15 Ismáel pedig, Natániának fia nyolczadmagával szalada el Jóhanán elõl, és az Ammon fiaihoz méne.
\par 16 És magához vevén Jóhanán, Káreának fia, és a seregnek minden vezetõje, a kik õ vele valának, a nép minden maradékát, a melyet visszahoztak vala Ismáeltõl, Natániának fiától Mispából, miután ez megölé Gedáliást, Ahikámnak fiát, az erõs férfiakat, vitézeket és asszonynépeket és gyermekeket és udvari szolgákat, a kiket visszahozott vala Gibeonból:
\par 17 Elindulának és megállapodának Gérut Kimhámnál, a mely közel vala Betlehemhez, hogy elmenjenek és bemenjenek Égyiptomba,
\par 18 A Káldeusok miatt, mert félnek vala tõlök, mert Ismáel, Natániának fia megölte vala Gedáliást, Ahikámnak fiát, a kit a babiloni király tiszttartóvá tett vala a földön.

\chapter{42}

\par 1 Eljöve pedig a csapatoknak minden tisztje, és Jóhanán, Káreának fia, és Jazánia, Ozániának fia, és az egész nép kicsinytõl fogva nagyig,
\par 2 És mondának Jeremiás prófétának: Hallgasd meg alázatos kérésünket, és könyörögj érettünk az Úrnak, a te Istenednek mind e maradékért; mert kevesen maradtunk meg a sokaságból, mint a te szemeid jól látnak minket;
\par 3 És jelentse meg nékünk az Úr, a te Istened az útat, a melyen járjunk, és a dolgot, a mit cselekedjünk.
\par 4 És monda nékik Jeremiás próféta: Meghallottam. Ímé, én könyörgök az Úrnak, a ti Isteneteknek a ti beszédeitek szerint, és mindazt, a mit felel az Úr néktek, megjelentem néktek, el nem titkolok semmit tõletek.
\par 5 Õk pedig mondának Jeremiásnak: Az Úr legyen ellenünk tökéletes és igaz tanúbizonyság, ha nem mind ama beszéd szerint cselekszünk, a melylyel elküld téged mi hozzánk az Úr, a te Istened.
\par 6 Ha jó, ha rossz, hallgatni fogunk az Úrnak, a mi Istenünknek szavára, a melyért mi téged õ hozzá küldünk; hogy jó dolgunk legyen, mert mi hallgatunk az Úrnak, a mi Istenünknek szavára.
\par 7 Lõn pedig tíz nap múlva, hogy szóla az Úr Jeremiásnak.
\par 8 És odahívá Jóhanánt, Káreának fiát, és a seregeknek minden tisztjét, a kik vele valának, és az egész népet kicsinytõl fogva nagyig.
\par 9 És monda nékik: Ezt mondja az Úr, Izráelnek Istene, a kihez küldöttetek engem, hogy megjelentsem elõtte a ti könyörgésteket:
\par 10 Ha állandóan megmaradtok e földön, felépítlek titeket és el nem rontlak, és elplántállak titeket és ki nem gyomlállak: mert megbántam a gonoszt, a mit cselekedtem veletek.
\par 11 Ne féljetek a babiloni királytól, a kitõl most féltek, ne féljetek tõle, azt mondja az Úr; mert veletek vagyok, hogy megtartsalak és megszabadítsalak titeket az õ kezébõl.
\par 12 És irgalmasságot cselekszem veletek, hogy irgalmas legyen hozzátok, és lakni hagyjon titeket a ti földeteken.
\par 13 De ha ti ezt mondjátok: Nem lakunk e földön, nem hallgatván az Úrnak, a ti Isteneteknek szavára,
\par 14 Mondván: Nem! hanem Égyiptom földére megyünk be, a hol nem látunk harczot és nem hallunk trombitaszót, és kenyeret nem éhezünk, és ott lakozunk:
\par 15 Most azért halljátok meg az Úrnak szavát, Júdának maradékai! Ezt mondja a Seregek Ura, az Izráel Istene: Ha ti csak orczáitokat fordítjátok is úgy, hogy bemenjetek Égyiptomba, és bementek, hogy ott tartózkodjatok:
\par 16 Akkor a fegyver, a melytõl ti féltek, legott utólér benneteket Égyiptom földén, és az éhség is, a mitõl rettegtek, körülvesz titeket Égyiptomban, és ott haltok meg.
\par 17 Mert az lesz, hogy mindazok a férfiak, a kik úgy fordítják orczájokat, hogy bemenjenek Égyiptomba, hogy ott tartózkodjanak, meghalnak fegyver, éhség és döghalál miatt, és egy sem marad meg közülök, és nem szabadul meg a veszedelemtõl, a melyet én bocsátok reájok.
\par 18 Mert ezt mondja a Seregek Ura, az Izráel Istene: Miképen kiömlik az én haragom és búsulásom Jeruzsálem lakosaira, azonképen kiöntöm az én haragomat ti reátok, ha bementek Égyiptomba; és lesztek ott átok, csuda, szidalom, gyalázat, és többé nem látjátok e helyet!
\par 19 Az Úr szólott hozzátok, oh Júdának maradékai! Ne menjetek Égyiptomba; jól tudjátok meg, hogy bizonyságot tettem ma ellenetek.
\par 20 Mert magatokat csaljátok meg a ti szívetekben: mert ti küldöttetek engem az Úrhoz, a ti Istenetekhez, mondván: Könyörögj az Úrnak, a mi Istenünknek, és a mint  szól az Úr, a mi Istenünk, úgy jelentsd meg nékünk, és akképen cselekszünk.
\par 21 Mikor pedig ma megjelentem néktek, nem hallgattok az Úrnak, a ti Isteneteknek szavára és egyáltalán azokra, a melyekért engem ti hozzátok küldött.
\par 22 Most azért tudjátok meg jól, hogy fegyverrel, éhséggel és döghalállal haltok meg a helyen, a hová kivánkoztok menni, hogy ott tartózkodjatok.

\chapter{43}

\par 1 És a mint Jeremiás teljesen elmondta az egész népnek az Úr, az õ Istenök minden rendelését, melyekért az Úr, az õ Istenök küldötte vala õ hozzájok, mindazokat a rendeléseket, mondom,
\par 2 Akkor monda Azariás, Hosájának fia, Jóhanán, Káreának fia, és valamennyi kevély férfiak, mondván Jeremiásnak: Hazugságot szólsz te, nem küldött téged az Úr, a mi Istenünk, hogy ezt mondjad: Ne menjetek be Égyiptomba, hogy ott tartózkodjatok.
\par 3 Hanem Báruk, Nériának fia izgat téged mi ellenünk, hogy minket a Káldeusok kezébe adjon, hogy megöljenek minket, és vitessenek Babilonba.
\par 4 És nem hallgata Jóhanán, Káreának fia, sem a seregnek valamennyi tisztje, sem az egész nép az Úr szavára, hogy a Júda földén maradjanak.
\par 5 Hanem elvivé Jóhanán, Káreának fia, és a seregnek minden tisztje Júdának egész maradékát, a kik visszajöttek vala mindama nemzetek közül, a hová kiûzettek vala, hogy lakozzanak Júdának földében.
\par 6 A férfiakat és az asszonyokat, a gyermekeket és a király leányait és minden lelket, a melyet Nabuzáradán, a poroszlók feje hagyott vala Gedáliással, Ahikámnak, Sáfán fiának fiával, és Jeremiás prófétát és Bárukot, Nériának fiát.
\par 7 És elmenének Égyiptomnak földébe, mert nem hallgattak az Úr szavára, és bemenének Táfnesig.
\par 8 És szóla az Úr Jeremiásnak Táfnesben, mondván:
\par 9 Végy a kezedbe nagy köveket, és rejtsd el azokat a sárba, a téglaégetõkemenczébe, a mely a Faraó házának ajtajában van Táfnesben, a Júdabeli emberek szeme láttára.
\par 10 És ezt mondd nékik: Ezt mondja a Seregek Ura, az Izráel Istene: Ímé, én elküldök és felhozom  Nabukodonozort, a babiloni királyt, az én szolgámat, és az õ székét e kövekre teszem, melyeket elrejtettem, és azokra vonja fel az õ sátorát.
\par 11 És betör, és megveri Égyiptom földét, a ki halálra való, halálra, és a ki rabságra, rabságra, és a ki fegyverre, fegyverre jut.
\par 12 És tüzet gyújtok Égyiptom isteneinek házaiban, és felégeti azokat, és foglyokká teszi õket, és magára ölti Égyiptom földét, miképen a pásztor magára ölti ruháját, és kimegy onnan békességgel.
\par 13 És Bethsemesnek, a mely Égyiptom földében van, faragott képeit lerontja: és Égyiptom isteneinek házait tûzzel felégeti.

\chapter{44}

\par 1 Az a szó, a mely lõn Jeremiáshoz, minden Júdabeliek felõl, a kik laknak vala Égyiptom földében, a kik laknak vala Migdolban, Táfnesben, Nófban és Pátrosz földében, mondván:
\par 2 Ezt mondja a Seregek Ura, az Izráel Istene: Ti láttátok mindazt a veszedelmet, melyet ráhoztam volt Jeruzsálemre és Júdának minden városaira, és ímé, azok most pusztává lettek, és senki sem lakozik bennök.
\par 3 Az õ gonoszságokért, a melyet cselekedtek, hogy felingereljenek engem, elmenvén, hogy áldozatot vigyenek és szolgáljanak az idegen isteneknek, a kiket õk nem ismernek vala, sem ti, sem a ti atyáitok.
\par 4 És elküldöttem hozzátok minden szolgámat, a prófétákat, és pedig jó reggel küldém el, mondván: Kérlek, ne cselekedjétek ez útálatos dolgot, a mit gyûlölök.
\par 5 De nem hallgattak, és a fülöket sem hajtották arra, hogy megtérjenek az õ gonoszságokból, és idegen isteneknek ne áldozzanak.
\par 6 Azért kiömlött az én bosszúm és az én haragom, és felgerjedt Júda városaiban és Jeruzsálem utczáin, és pusztasággá és sivataggá lõnek mind e napig.
\par 7 Most azért ezt mondja az Úr, a Seregek Istene, az Izráel Istene: Miért szereztek nagy veszedelmet a ti lelketek ellen, hogy kipusztítsatok közületek férfit és asszonyt, gyermeket és csecsemõt Júda kebelébõl, hogy magatoknak még csak maradékot se hagyjatok;
\par 8 Ingerelvén engem a ti kezeitek alkotásaival, áldozván az idegen isteneknek Égyiptom földén, a melyre ti tartózkodni jöttetek be, hogy veszedelmet szerezzetek megatoknak, és hogy átokban és gyalázatban legyetek e földnek minden nemzeténél?
\par 9 Vajjon elfelejtkeztetek-é a ti atyáitok gonoszságairól és a Júda királyainak gonoszságairól és az õ feleségeiknek gonoszságairól és a ti gonoszságaitokról, a ti feleségeiteknek gonoszságairól, a melyeket Júdának földén cselekedtetek és Jeruzsálemnek utczáin?
\par 10 Nem alázták meg magokat mind e mai napig se, és nem féltek, sem az én törvényem szerint nem jártak, sem az én parancsolataim szerint, a melyeket elõtökbe és a ti atyáitok elébe adtam.
\par 11 Azért ezt mondja a Seregek Ura, az Izráel Istene: Ímé, én ellenetek fordítom orczámat veszedelemre, és hogy az egész Júdát kipusztítsam.
\par 12 És felveszem Júdának maradékát, a kik magok elé tûzték, hogy bemennek Égyiptom földére, hogy ott lakozzanak, és mindnyájan megemésztetnek Égyiptom földén, elesnek fegyver miatt, megemésztetnek éhség miatt, kicsinytõl fogva nagyig: fegyver és éhség miatt halnak meg, és  átokká, csudává, szidalommá és gyalázattá lesznek.
\par 13 És megfenyítem azokat, a kik Égyiptom földében lakoznak, miképen megfenyítettem a Jeruzsálembelieket fegyverrel, éhséggel és döghalállal.
\par 14 És a Júda maradékai közül, a kik ide jöttek, hogy Égyiptom földében tartózkodjanak, senki sem menekül és szabadul meg, hogy visszatérjen Júdának földébe, a hova lelkök hajtja õket, hogy oda visszatérjenek és ott lakozzanak; mert nem térnek vissza, hanem csak a kik menekülnek.
\par 15 És felelének Jeremiásnak mindama férfiak, a kik tudják vala, hogy az õ feleségeik az idegen isteneknek áldozának, és mindazok az asszonyok, a kik ott állnak vala nagy tömegben, és az egész nép, a mely Égyiptom földén, Pátroszban lakozik vala, mondván:
\par 16 Abban a dologban, a mi végett szóltál nékünk az Úr nevében, nem hallgatunk reád;
\par 17 Hanem csak azt cselekeszszük, a mit mi a mi szánkkal fogadtunk, hogy füstölõ áldozatot viszünk az ég királynéjának, és néki italáldozattal áldozunk, miképen cselekedtünk mi és a mi atyáink és a mi királyaink és a mi fejedelmeink Júda városaiban  és Jeruzsálemnek utczáin, mert akkor beteltünk kenyérrel, és jó dolgunk volt, és semmi rosszat nem láttunk.
\par 18 De a mióta nem áldozunk többé az ég királynéjának füstöléssel, és nem viszünk néki italáldozatot: mindenben szûkölködünk, és fegyver és éhség miatt emésztetünk.
\par 19 És hogyha mi az ég királynéjának füstölve áldozunk és néki italáldozatot viszünk: vajjon a mi férjeink híre nélkül csinálunk-é néki béleseket, hogy õt tiszteljük, és néki itali áldozatot vigyünk?
\par 20 Szóla azért Jeremiás az egész népnek, a férfiaknak és az asszonyoknak és az egész népnek, a kik e szót felelték néki, mondván:
\par 21 Avagy a jó illatról, a melyet Júda városaiban és Jeruzsálem utczáin füstöltetek ti és a ti atyáitok, a ti királyaitok és a ti fejedelmeitek és a föld népe: nem arról emlékezett-é meg az Úr, és nem az jutott-é néki eszébe?
\par 22 És nem szenvedhette tovább az Úr a ti cselekedeteitek gonoszságát, az útálatosságok miatt, a melyeket cselekedtetek, és pusztasággá lett a ti földetek és csudává és átokká, annyira, hogy senki sem lakja mind e napig,
\par 23 A miatt, hogy füstölve áldoztatok, és vétkeztetek az Úr ellen, és nem hallgattatok az Úr szavára, és az õ törvénye és az õ parancsolatai és az õ tanúbizonyságai szerint nem jártatok, azért következett ti reátok ez a veszedelem mind e napig.
\par 24 Monda továbbá Jeremiás az egész népnek és az összes asszonyoknak: Halljátok meg az Úr szavát mind, ti Júdabeliek, kik Égyiptom földén vagytok.
\par 25 Ezt mondja a Seregek Ura, az Izráel Istene, mondván: Ti és a ti feleségeitek szóltatok a ti szájatokkal, és végbevittétek a ti kezeitekkel, mondván: Bizonyára teljesítjük a mi fogadásainkat, a melyeket fogadtunk az ég királynéjának, hogy füstölve áldozzunk, és néki italáldozatot vigyünk. Megerõsítvén megerõsítettétek a ti fogadásaitokat, és megcselekedvén megcselekedtétek a ti fogadásaitokat.
\par 26 Azért halljátok meg az Úr szavát mind, ti Júdabeliek, akik Égyiptom földében lakoztok: Ímé, én az én nagy nevemre megesküdtem, azt mondja az Úr, hogy egyetlen Júdabeli férfiú szája sem fogja az én nevemet kiejteni, mondván: Él az Úr Isten, egész Égyiptom földén!
\par 27 Ímé, én vigyázok reájok az õ kárukra és nem javukra, és megemésztetik Júdának minden férfia, a kik Égyiptom földén vannak, fegyver miatt, éhség miatt, mígnem mind elfogynak.
\par 28 De a kik fegyvertõl megszabadulnak, visszatérnek Égyiptom földérõl Júdának földére, szám szerint kevesen, és mind megtudják Júdának maradékai, a kik bementek Égyiptom földébe, hogy ott tartózkodjanak: melyik szó teljesedik be, az enyém-é vagy az övék?
\par 29 És ez lesz néktek a jel, azt mondja az Úr, hogy én meglátogatlak titeket ezen a helyen, hogy megtudjátok, hogy bizonyára megállanak az én beszédeim a reátok következendõ veszedelem felõl.
\par 30 Ezt mondja az Úr: Ímé, én odaadom Faraó Ofrát, Égyiptom királyát az õ ellenségeinek kezébe és az õ lelkét keresõk kezébe, miképen odaadtam Sedékiást, a Júda királyát Nabukodonozornak, a babiloni királynak, az õ ellenségének és az õ lelkét keresõnek kezébe.

\chapter{45}

\par 1 Az a szó, a melyet Jeremiás próféta szóla Báruknak, Néria fiának, mikor õ könyvbe írá e szókat Jeremiás szájából, Jójákimnak, Jósiás, Júdabeli király fiának negyedik esztendejében, mondván:
\par 2 Ezt mondja az Úr, Izráel Istene, te néked, Báruk:
\par 3 Ezt mondottad: Jaj mostan nékem, mert az Úr az én bánatomra fájdalmat adott, elfáradtam az én fohászkodásomban, és nyugodalmat nem találtam.
\par 4 Ezt mondd néki: Ezt mondja az Úr: Ímé, a kiket én felépítettem, elrontom, és a kiket én beplántáltam, kiszaggatom, és pedig az egész földön.
\par 5 És te kivánsz-é magadnak nagyokat? Ne kivánj; mert ímé én veszedelmet bocsátok minden testre, ezt mondja az Úr, és a te lelkedet zsákmányul adom néked, minden helyen, a hová elmégy.

\chapter{46}

\par 1 Ez a szó, a melyet az Úr szólott Jeremiás prófétának a pogányok felõl;
\par 2 Égyiptom felõl, a Nékó Faraónak, Égyiptom királyának serege felõl, a mely az Eufrátes folyó mellett, Kárkémisben vala, a melyet megvere Nabukodonozor, a babiloni király, Jójákimnak, Jósiás fiának, Júda királyának negyedik esztendejében:
\par 3 Készítsetek vértet és paizst, és induljatok a harczra.
\par 4 Nyergeljétek a lovakat, és üljetek fel ti lovasok, és legyetek sisakokban. Tisztítsátok a kopjákat, öltsétek fel a pánczélokat.
\par 5 Mit látok? Õk megriadva hátrálni kezdenek, vitézeik leveretnek és futásnak erednek, és vissza sem tekintenek! Mindenfelõl félelem, azt mondja az Úr.
\par 6 Nem futhat el a gyors, és az erõs sem menekülhet el; észak felé, az Eufrátes folyó mellett legyõzetnek és elhullanak.
\par 7 Kicsoda ez, a ki növekedik, mint a folyóvíz, és olyan, mint a megháborodott vizû folyamok?
\par 8 Égyiptom növekedik úgy, mint a folyóvíz és mint a megháborodott vizû folyamok, mert ezt mondja: Felmegyek, ellepem a földet, elvesztem a várost és a benne lakókat.
\par 9 Jõjjetek fel lovak, és zörögjetek szekerek, jõjjenek ki a vitézek, a szerecsenek és a Libiabeliek, a kik paizst viselnek, és a Lidiabeliek, a kik kézívet viselnek!
\par 10 Az a nap pedig az Úrnak, a Seregek Urának büntetõ napja, hogy bosszút álljon ellenségein. És a fegyver felemészt és jól lakik és megrészegül az õ vérökkel, mert áldozatja lesz az Úrnak, a Seregek Urának észak földén, az Eufrátes folyó mellett.
\par 11 Menj föl Gileádba, és végy balzsamot te szûz, Égyiptomnak leánya: hiába sokasítod az orvosságokat, nincs gyógyír számodra!
\par 12 Hallották a pogányok a te gyalázatodat, és a te kiáltásoddal betelt a föld, mert vitéz vitézbe ütközött, és mind a ketten együtt estek el.
\par 13 A szó, a melyet az Úr Jeremiás prófétához szólott, Nabukodonozornak, a babiloni királynak eljövetele felõl az Égyiptom földének megverésére:
\par 14 Hirdessétek Égyiptomban és híreszteljétek Migdólban, híreszteljétek Nófban és Táfnesben, és ezt mondjátok: Állj elõ, és készítsd fel magadat, mert fegyver emészti meg a te kerületeidet.
\par 15 Miért verettek le a te erõseid? Nem állhattak meg, mert az Úr rettentette el õket.
\par 16 Megsokasította a tántorgót, egyik a másikra hullott, és ezt mondották: Kelj fel, és menjünk vissza a mi népünkhöz és a mi szülõföldünkre az erõszakoskodó fegyver elõl.
\par 17 Ezt kiáltják akkor: A Faraó, Égyiptom királya, a háborúságnak királya, elhaladta a rendelt idõt.
\par 18 Élek én, azt mondja a király, a kinek a neve Seregek Ura, hogy mint a Táborhegy áll a hegyek között, és mint a Kármel a tenger között, úgy jõ el.
\par 19 Készíts magadnak elköltözésre való edényeket, Égyiptom leányának lakosa, mert Nóf elpusztul és megég, lakatlanná lesz.
\par 20 Szép üszõtinó Égyiptom, de pusztulás tör reá észak felõl.
\par 21 Még zsoldosai is olyanok õ közöttök, mint a hízlalt borjúk, de õk is meghátrálnak, egyetemlegesen elfutnak, meg nem állanak, mert romlásuk napja jön reájok, az õ megfenyíttetésök ideje.
\par 22 Az õ szava mint a csúszómászó kígyóé, mert nagy sereggel indulnak, és szekerczékkel jõnek ellene, mint a favágók.
\par 23 Kivágják az õ erdejét, azt mondja az Úr, mert beláthatatlanok, mert többen lesznek mint a sáskák, és megszámlálhatatlanok.
\par 24 Megszégyenül Égyiptom leánya, északi nép kezébe jut.
\par 25 Ezt mondja a Seregek Ura, az Izráel Istene: Ímé, én megfenyítem Nó-Ammont és a Faraót és Égyiptomot és az õ isteneit és királyait, mind a Faraót, mind azokat, a kik bíznak benne.
\par 26 Odaadom õket az õ lelkök keresõinek kezébe, és Nabukodonozornak, a babiloni királynak kezébe, és az õ szolgáinak kezébe; de azután úgy lakoznak abban, mint azelõtt, azt mondja az Úr.
\par 27 És te ne félj, oh én szolgám Jákób, és ne rettegj Izráel, mert ímé, én megszabadítlak téged messzirõl, és a te magodat is az õ fogságuk földérõl, és visszatér Jákób és megnyugszik és békességben lesz, és nem lesz, a ki megijeszsze.
\par 28 Ne félj te, oh én szolgám Jákób, azt mondja az Úr, mert én veled vagyok, mert véget vetek minden nemzetnek, a kik közé kivetettelek téged, néked pedig nem vetek véget, hanem megverlek téged ítélettel, mert nem hagyhatlak teljesen büntetés nélkül.

\chapter{47}

\par 1 Az a szó, a melyet az Úr szóla Jeremiás prófétának a Filiszteusok felõl, mielõtt megverte a Faraó Gázát.
\par 2 Ezt mondja az Úr: Ímé, víz indul meg északról, és olyan lesz, mint a kiáradott folyó, és elárasztja a földet és annak mindenét, a várost és annak lakosait, és kiáltanak az emberek, és a föld minden lakosa ordít.
\par 3 Ménei patáinak csattogó hangjától, szekereinek zörgésétõl, kerekeinek zúgásától nem gondolnak az atyák a fiakra erejök ellankadáa miatt.
\par 4 És a nap miatt, mely eljött, hogy elpusztítsa egész Filiszteát, kivágja Tírust és Szidont és segítségének minden megmaradt töredékét, mert az Úr elrontja Filiszteát, a Káftor szigetének maradékát.
\par 5 Kopaszság lepte meg Gázát, Askalon elnémult, maradéka az õ völgyüknek: meddig vagdalod magadat?
\par 6 Oh szablyája az Úrnak, meddig nem nyugszol meg? Rejtsd el magadat a te hüvelyedbe, nyugodjál meg és hallgass!
\par 7 Miképen nyughatik meg, holott az Úr parancsolt néki? Askalonra és a tenger partjára oda rendelte õt.

\chapter{48}

\par 1 Moáb felõl ezt mondja a Seregek Ura, az Izráel Istene: Jaj Nébónak, mert elpusztíttatott; megszégyenült, bevétetett Kirjátaim, Misgáb megszégyenült és elrémült.
\par 2 Nincs már dicsõsége Moábnak Hesbonban, gonoszt gondoltak õ ellene, mondván: Jertek el és veszessük el õt, ne legyen nemzetség! Madmen te is elnémulsz, fegyver jár nyomodban!
\par 3 Nagy kiáltás hallatszik Horonáimból: pusztulás és nagy romlás!
\par 4 Elnyomorodott Moáb, kicsinyei sikoltva kiáltanak.
\par 5 Mert a Luhit hágóján siralmat siralom ér, mert Horonáim lejtõin az ellenség hallatja vészkiáltását.
\par 6 Fussatok, mentsétek meg lelketeket, és legyetek mint a hangafa a pusztában!
\par 7 Mivelhogy a te bizodalmad marháidban és kincseidben volt, te is bevétetel, és Kámós fogságra megy papjaival, fejedelmeivel együtt.
\par 8 És rátör a pusztító minden városra, egy város sem menekedik meg, és elvész a völgy, és feldúlatik a síkság, a mint megmondta az Úr.
\par 9 Adjatok szárnyat Moábnak, hogy repülvén elrepülhessen, mert az õ városai elpusztulnak, és senki sem lakik azokban.
\par 10 Átkozott, a ki az Úrnak dolgát csalárdul cselekszi, és átkozott, a ki fegyverét kiméli a vértõl!
\par 11 Nyugodtan élt Moáb gyermekségétõl fogva, és pihent az õ seprejében, és edénybõl-edénybe nem öntötték és fogságra sem ment, azért maradt meg az íze rajta, és nem változott el az õ szaga.
\par 12 De ímé eljõnek a napok, azt mondja az Úr, és rablókat bocsátok reá, a kik megrabolják õt, és megüresítsék edényeit, palaczkjait pedig összetörjék.
\par 13 És megszégyenül Moáb Kámós miatt, a mint megszégyenült Izráel háza Béthel  miatt, a melyben bizodalma volt.
\par 14 Mimódon mondjátok: Hõsök vagyunk és vitéz férfiak a harczra?
\par 15 Elpusztul Moáb, városai fellobbannak, válogatott ifjai pedig mészárszékre jutnak, azt mondja a király, a kinek neve Seregek Ura!
\par 16 Közel van Moáb veszedelme, ihol jõ és igen siet az õ veszedelme!
\par 17 Bánkódjatok miatta mindnyájan, a kik körülte vagytok, és mindnyájan, a kik ismeritek az õ nevét; mondjátok: Hogy eltört az erõs vesszõ, a dicsõ pálcza!
\par 18 Szállj le a dicsõségbõl és ülj szomjan Dibonnak megmaradt leánya, mert a Moáb pusztítója feljött ellened, elrontja a te erõsségeidet.
\par 19 Állj meg az úton, és nézz ide-oda Aroér lakosa, kérdezd meg a futót és a menekülõt, és ezt mondd: Mi történt?
\par 20 Megszégyenült Moáb, mert megtört. Ordítsatok és kiáltsatok! Hirdessétek Arnonban, hogy elpusztíttatott Moáb!
\par 21 Mert rájött az ítélet a sík földre; Hólonra, Jására és Mefátra.
\par 22 És Dibonra, Nébóra és Beth-Diblátaimra.
\par 23 És Kirjáthaimra, Beth-Gámulra és Beth-Meonra.
\par 24 És Kirjátra, Boczrára és Moáb földének minden messze és közel való városaira.
\par 25 Letöretett a Moáb szarva, és karja levágatott, azt mondja az Úr.
\par 26 Részegítsétek le õt, mert hõsködött az Úr ellen, és heverjen Moáb az õ okádásában, és legyen csúfság õ is.
\par 27 Vajjon nem csúfod volt-é néked az Izráel? Avagy a lopók között találtatott, hogy mikor szólottál felõle, kevélyen hánytad magadat?
\par 28 Hagyjátok el a városokat, és lakhatok a kõsziklákban, Moáb lakosai, és legyetek mint a galamb, a mely az odu száján belõl rak fészket.
\par 29 Hallottuk a Moáb kevélységét: igen kevély; az õ felfuvalkodását és kevélységét, kérkedését, és az õ szívének elbizakodottságát.
\par 30 Én ismerem, azt mondja az Úr, az õ szertelenkedését, és az õ fecsegése nem igaz, és nem igaz a cselekedete sem.
\par 31 Azért jajgatok Moábon, és az egész Moábért kiáltok, a Kir-Heres férfiaiért nyög az én lelkem.
\par 32 Jobban siratlak téged, mint siratták Jaézert, a ki Sibmának szõlõje! A te hajtásaid túlhatoltak a tengeren, a Jaézer tengeréig értek; a te nyári gyümölcseidre és a te szüretedre pusztító rontott.
\par 33 És eltünik az öröm és vígasság Kármelbõl és a Moáb földérõl, és a kádakból kifogyasztom a bort, nem sajtolnak örömzajjal; az éneklõ nem énekel.
\par 34 Hesbon kiáltása miatt Elealéig és Jáhásig felhat az õ szavok: Soártól fogva Horonáimig és Eglath-Selisájjáig, mert a Nimrim vize is elapad.
\par 35 És kifogyasztom Moábból, azt mondja az Úr, a ki a magaslaton áldozik és füstöt gerjeszt az õ isteneinek.
\par 36 Ezért zokog a szívem Moábért, mint a síp, és zokog szívem a Kir-Héres férfiaiért, mint a síp, mivelhogy a kincsek elvesztek, a miket gyûjtött.
\par 37 Mert minden fõ kopasz, és minden szakál elnyiratott, minden kézen metélések, és minden derékon gyászruha.
\par 38 Moábnak minden házpadján és utczáján mindenütt siralom, mert összetörtem Moábot, mint az edényt, mely semmirekellõ, azt mondja az Úr.
\par 39 Jajgatnak, ezt mondván: Hogy összezúzatott! Hogy fordult háttal Moáb megszégyenülve! És csúffá lett Moáb, és rettentésére mindazoknak, a kik körülte vannak.
\par 40 Mert ezt mondja az Úr: Ímé, mint a saskeselyû repül reá, és kiterjeszti Moábra szárnyait.
\par 41 Bevétettek a városok, és az erõsségek elfoglaltattak, és a Moáb vitézeinek szíve olyan volt e napon, mint a vajudó asszonynak szíve.
\par 42 És Moáb elpusztul, úgy hogy nem lesz nép többé, mert az Úr ellen felemelkedett.
\par 43 Félelem, verem és tõr jõ te ellened, Moáb lakosa, azt mondja az Úr.
\par 44 A ki elfut a félelem elõl, a verembe esik, és a ki kijõ a verembõl, a tõrben fogatik meg, mert rábocsátom Moábra az õ megfenyítésének esztendejét, azt mondja az Úr.
\par 45 A Hesbon árnyékában állanak meg a hatalom elõl futók; de tûz jõ ki Hesbonból és láng Szihonnak közepébõl, és elemészti Moábnak üstökét és a háborgó fiaknak koponyáját.
\par 46 Jaj néked Moáb! Elveszett Kámósnak népe, mert fiaid fogságra vitettek, és leányaid is fogságra.
\par 47 De visszahozom Moábot a fogságból sok idõ mulva, azt mondja az Úr. Eddig van Moáb ítélete.

\chapter{49}

\par 1 Az Ammon fiai felõl ezt mondja az Úr: Nincsenek-é Izráelnek fiai? Nincsen-é örököse néki? Miért birtokolja Milkom Gádot, és az õ népe miért lakik annak városaiban?
\par 2 Azért ímé eljõnek a napok, azt mondja az Úr, és harczi riadót hallatok Rabbában az Ammon fiaival, és romhalommá lesz, és leányai tûzzel égettetnek meg, és Izráel birtokolja azokat, a kik most õt birtokolják, azt mondja az Úr.
\par 3 Ordíts Hesbon, mert elpusztíttatott Hái! Kiáltsatok Rabbáh leányai, öltözzetek gyászba, sírjatok és futkossatok a szorosokon, mert Milkom a fogságba megy, papjai és fejedelmei is vele együtt.
\par 4 Mit dicsekedel völgyeddel, völgyed bõségével, engedetlen leány, a ki az õ kincseiben bízik és ezt mondja: Kicsoda támad ellenem?
\par 5 Ímé én félelmet bocsátok reád, azt mondja az Úr, a Seregek Ura, minden szomszédod felõl, és szétrebbentek egymástól, és nem lesz, a ki összegyûjtse az elszéledteket.
\par 6 De azután visszahozom majd a fogságból az Ammon fiait, azt mondja az Úr.
\par 7 Ezt mondja a Seregek Ura Edom felõl. Nincs bölcsesség többé Témánban? elveszett-é a tanács az értelmesektõl? hiába valóvá lett-é az õ bölcsességök?
\par 8 Fussatok, forduljatok, rejtõzzetek el mélyen Dédán lakosai, mert Ézsau veszedelmét hozom õ reá az õ megfenyíttetésének idején.
\par 9 Ha szõlõszedõk törnek reád, nem hagynak gerezdeket, ha éjjeli tolvajok: pusztítanak, a míg nékik tetszik.
\par 10 Bizony én mezítelenné teszem Ézsaut, titkait kijelentem, és el nem rejtõzhetik, magva elpusztul, és atyjafiai és szomszédai sem lesznek.
\par 11 Hagyd el a te árváidat, én eltartom özvegyeidet is; bennem vessék reménységüket.
\par 12 Mert azt mondja az Úr: Ímé, a kiknek nem kell vala meginniok a pohárt, ugyancsak megiszszák; te pedig teljesen büntetlenül maradnál-é? Nem maradsz büntetlenül, mert bizonyára megiszod.
\par 13 Mert én magamra esküdtem meg, azt mondja az Úr, hogy útálattá és gyalázattá, pusztasággá és  átokká lesz Boczra, és minden városa örökkévaló pusztasággá lesz.
\par 14 Hírt hallottam az Úrtól, és követ küldetett a nemzetekhez, a ki ezt mondja: Gyûljetek össze, induljatok ellene, és keljetek fel a harczra,
\par 15 Mert ímé, kicsinynyé teszlek téged a nemzetek között, és az emberek között útálatossá.
\par 16 A te könnyelmûséged csalt meg téged és a te szíved kevélysége, a ki a sziklák hasadékaiban lakol, és elfoglaltad a halmok tetejét. Ha olyan magas helyen rakod is fészkedet, mint a saskeselyû, onnét is lerántalak téged, azt mondja az Úr.
\par 17 És pusztasággá lesz Edom. a ki csak átmegy rajta elálmélkodik, és sziszeget egész veresége felett.
\par 18 A mint Sodomának és Gomorának és az õ szomszédainak elsüllyedésekor volt, azt mondja az Úr, ott sem lakik több ember, és benne emberek fia nem tartózkodik.
\par 19 Ímé, mint oroszlán jön fel a Jordán erdõségébõl az örökzöld legelõre, de hamarsággal kiûzöm õt onnan, és a kiválasztottat teszem azon fejedelemmé, mert kicsoda hozzám hasonló, és ki szab nékem törvényt, és kicsoda az a pásztor, a ki megállhat ellenem?
\par 20 Halljátok meg azért az Úr tervét, a melyet tervezett Edom felõl, és az õ gondolatait, a melyeket gondolt Témán polgárai felõl. Bizony elhurczolják õket, a juhnyáj kicsinyeit, bizony szörnyûködik rajtok a saját legelõjök.
\par 21 Az õ romlásuk zajától megrendült a föld, az õ kiáltásuk szava elhallatszik a veres tengerig.
\par 22 Ímé, feljõ mint saskeselyû, és repül és szárnyait szétterjeszti Boczrán: és Edom vitézeinek szíve olyan lesz az napon, mint a vajudó asszony szíve.
\par 23 Damaskus felõl: Megszégyenült Emát és Arphád, mert gonosz hírt hallottak, és remegnek, mint a háborgó tenger, a mely nem nyughatik.
\par 24 Megrendült Damaskus, futáshoz készül és reszketés fogja el, szorongás és fájdalmak szállják meg õt, mint a szûlõ asszonyt.
\par 25 Miért is nem marad ki a dicsõséges város, az én örömömnek városa?
\par 26 De elhullanak az õ ifjai is az õ utczájokon, és minden harczoló ember levágatik azon a napon, azt mondja a Seregek Ura.
\par 27 És tüzet gyújtok Damaskus kõfalán, és megemészi a Ben-Hadád palotáit.
\par 28 Kédárnak és Házornak országai felõl, a melyeket megvert Nabukodonozor, a babiloni király, ezt mondja az Úr: Keljetek fel, menjetek fel Kédárra, és pusztítsátok keletnek fiait.
\par 29 Sátoraikat és nyájokat elveszik, és kárpitjaikat és minden edényöket és tevéiket elviszik, és ezt kiáltják feléjök: Rettegés köröskörül!
\par 30 Fussatok el, igen siessetek, rejtõzzetek el mélyen, Házornak lakói, azt mondja az Úr, mert tervet tervezett ellenetek Nabukodonozor, a babiloni király, és ellenetek gondolatot gondolt.
\par 31 Keljetek fel, menjetek a békességes nemzet közé, azok közé, a kik bátorságban lakoznak, azt mondja az Úr, sem kapujok, sem zárjok nincsen, egyedül laknak!
\par 32 Tevéik prédává lesznek, és az õ sok barmaik zsákmánynyá, és elszórom õket, e nyirott üstökûeket minden szél felé, és minden oldal felõl veszedelmet hozok reájok, azt mondja az Úr.
\par 33 És Házor sakálok lakhelyévé lesz, örökkévaló pusztasággá, senki nem lakik ott, és embernek fia nem tartózkodik azon.
\par 34 Az Úr szava, a melyet szóla Jeremiás prófétának Elám felõl, Sedékiásnak, a Júda királyának országlása kezdetén, mondván:
\par 35 Ezt mondja a Seregek Ura: Ímé, én eltöröm az Elám kézívét, erejének zsengéjét.
\par 36 És négy szelet hozok Elám ellen, az égnek négy határáról, és elszórom õket mindenik szél felé, és nem lesz nemzet, a kihez nem futnak az Elám szökevényei.
\par 37 És megrettentem Elámot az õ ellenségei elõtt és az õ lelköknek keresõi elõtt, és veszedelmet hozok reájok, az én felgerjedt haragomat, azt mondja az Úr, és utánok bocsátom a fegyvert mindaddig, míg meg nem emésztem õket.
\par 38 És az én székemet Elámba helyezem, és kivesztem onnét a királyt és a fejedelmeket, azt mondja az Úr:
\par 39 De végzetre visszahozom Elámot a fogságból, azt mondja az Úr.

\chapter{50}

\par 1 Az a szó, a melyet szóla az Úr Babilon felõl és a Káldeusok földje felõl, Jeremiás próféta által.
\par 2 Hirdessétek a nemzetek között és hallassátok, emeljétek fel a zászlót: hallassátok és el ne titkoljátok; ezt mondjátok: Bevétetett Babilon, megszégyenült Bél, letöretett Merodák, megszégyenültek az õ faragott képei, letörettek az õ bálványai.
\par 3 Mert északról nép jön fel ellene, pusztává teszi ez az õ földét, és nem lesz, a ki lakozzék benne; embertõl fogva a baromig elfutnak, elmennek.
\par 4 Azokban a napokban, és abban az idõben, azt mondja az Úr, eljõnek az Izráel fiai, õk és a Júda fiai együtt, sírva jönnek és mennek és keresik az Urat, az õ Istenöket.
\par 5 A Sion felõl kérdezõsködnek, arrafelé fordítják orczájokat. Eljõnek és oda adják magokat az Úrnak örök szövetségre, a mely feledhetetlen.
\par 6 Elveszett juhnyáj volt az én népem, pásztorai félrevezették õket, a hegyekben bujdostatták õket, hegyrõl halomra jártak, elfelejtkeztek az õ tanyájukról.
\par 7 A ki csak reájok talált, emésztette õket, és az õ elnyomóik ezt mondták: Nem vétkeztünk, mert vétettek az Úr ellen, pedig igazság otthona, atyáiknak reménysége volt az Úr.
\par 8 Fussatok ki Babilonból és jõjjetek ki Káldea földébõl, és olyanok legyetek, mint a kecskebakok a nyáj elõtt;
\par 9 Mert ímé, én nagy nemzetek gyülekezetét támasztom és hozom fel Babilonra északnak földérõl, és sorakoznak ellene, legott bevétetik. Nyilai olyanok, mint a legyõzhetetlen vitézé, a ki nem tér vissza sikertelenül.
\par 10 És Káldea prédává lesz, a kik prédára vetik õt, mind betelnek vele, azt mondja az Úr.
\par 11 Csak örüljetek, csak tomboljatok örökségem elpusztítói: csak ugrándozzatok, mint a nyomtató tinó, és nyerítsetek, mint a ménlovak.
\par 12 Megszégyenül a ti anyátok, a ti szûlõtök igen csúffá lesz: Ímé, a nemzetek seprejévé, pusztává, szárazfölddé, sivataggá lesz.
\par 13 Az Úr haragja miatt nem lakoznak rajta, hanem egészen pusztasággá lesz, a ki csak átmegy Babilonon, álmélkodik és sziszeget egész veresége felett.
\par 14 Sorakozzatok köröskörül Babilon ellen, mind ti ijjászok, lõjjetek reá, ne kiméljétek a nyilat; mert az Úr ellen vétkezett!
\par 15 Kiáltsatok reá köröskörül, kezét adta, lehullottak az õ szegletkövei, leromlottak az õ kõfalai: bizony az Úr büntetése ez; büntessétek meg õt, és a mint cselekedett, úgy cselekedjetek vele.
\par 16 Vágjátok ki Babilonból a magvetõt és a ki sarlót fog aratás idején; a gyilkos fegyver elõl kiki az õ népéhez szalad, kiki az õ földe felé fut.
\par 17 Elszéledt juhnyáj az Izráel, oroszlánok kergették szét; elõször benyelte õt Assiria királya, végre pedig ez a Nabukodonozor,  a babiloni király megtörte az õ csontjait.
\par 18 Azért ezt mondja a Seregek Ura, az Izráel Istene: Ímé, én megfenyítem a babiloni királyt és az õ földét, miként megfenyítém az assiriai királyt.
\par 19 És visszaviszem az Izráelt az õ lakhelyére, és Básánban legel és a Kármelen, és az Efraim hegyén és Gileádban megelégszik az õ lelke.
\par 20 Azokban a napokban és abban az idõben, azt mondja az Úr, kerestetik az Izráel bûne, de nem lesz; a Júda vétkei, de nem találtatnak: mert kegyelmes leszek azokhoz, a kiket meghagyok.
\par 21 A kétszer pártütõk földére mej fel, és a meglátogattatás lakóit irtsd ki, öljed és irtsad õket, azt mondja az Úr, és mind a szerint cselekedjél, a mint parancsoltam néked.
\par 22 Harczi zaj a földön és nagy romlás.
\par 23 Hogy elmúlott és összetört az egész föld põrölye! milyen útálatossá lett Babilon a nemzetek között.
\par 24 Tõrbe ejtettelek téged, és meg is fogattál Babilon, de nem tudtad, utól érettél és megragadtattál, mert pörlekedtél az Úrral.
\par 25 Felnyitotta az Úr az õ tárházát, és elõhozta az õ haragjának szereit: mert e cselekedet az Úré, a Seregek Uráé Káldea földén.
\par 26 Törjetek reá a szélekrõl, nyissátok fel az õ magtárait, tapodjátok õt, mint a kévét, és irtsátok ki, hogy ne legyen maradéka.
\par 27 Döfjétek le minden tulkát, le velök a vágóhídra! Oh jaj nékik; mert eljött az õ napjok, az õ megfenyíttetésök ideje!
\par 28 A futók és a Babilon földébõl menekülõk zaja megjelentik majd a Sionon az Úrnak a mi Istenünknek bosszúállását, az õ templomáért való bosszúállását.
\par 29 Gyûjtsetek össze Babilon ellen igen sokat, mindenkit a ki kézívet feszít, köröskörül járjatok ellene tábort, hogy senki el ne szaladhasson: fizessetek meg néki az õ cselekedete szerint, a mint õ cselekedett, úgy cselekedjetek vele; mert az Úr ellen kevélykedett, az Izráelnek Szentje ellen!
\par 30 Azért elhullanak az õ ifjai az õ utczáiban, és minden vitéze elvész azon a napon, azt mondja az Úr.
\par 31 Ímé, én ellened vagyok, te kevély, azt mondja az Úr, a Seregek Ura, mert eljött a te napod, a te megfenyítésed napja.
\par 32 És megbotlik a kevély és elesik, és senki nem lesz, a ki felköltse õt, és tüzet gyújtok az õ városaiban, hogy megemészsze azokat, a kik körülte vannak.
\par 33 Ezt mondja a Seregek Ura: Megnyomoríttattak az Izráel fiai és Júda fiai együtt és mindnyájan, a kik fogságra vitték õket, beléjök ragadnak, nem akarják õket elbocsátani.
\par 34 De az õ megváltójok erõs, Seregek Ura az õ neve, bizonynyal felveszi az õ peröket, hogy megnyugtassa e földet, és Babilon lakóit megrettentse.
\par 35 Fegyver lesz a Káldeusokon, azt mondja az Úr, és Babilon lakóin és az õ fejedelmein és az õ bölcsein.
\par 36 Fegyver lesz az õ varázslóin, és megbolondulnak; fegyver lesz az õ vitézein, és elijednek.
\par 37 Fegyver lesz az õ lovain és szekerein és az egész egyveleg népen, a mely õ benne van, és hasonlók lesznek az asszonyokhoz. Fegyver lesz az õ kincsein, és elprédáltatnak.
\par 38 Szárazság lesz az õ vizein, és kiszáradnak, mert bálványok földe az, és faragott képekkel dicsekednek.
\par 39 Azért sakálok lakoznak ott baglyokkal, és struczmadárnak fiai lakoznak benne, és soha többé nem lakják azt, és nem lesznek lakosai nemzedékrõl nemzedékre.
\par 40 A mint felforgatta Isten Sodomát és Gomorát és az õ szomszéd városait, azt mondja az Úr, ép úgy nem lakik ott egy ember sem, és embernek fia sem lakja azt.
\par 41 Ímé, nép jött északról, és nagy nemzet és sok király támad fel a föld határaiból.
\par 42 Ívet és paizst ragadnak, kegyetlenek azok, és semmi irgalmasság nem lesz bennök, szavok mint a tenger zúgása, és lovakon jõnek, mind viadalra készek te ellened, te Babilon leánya!
\par 43 Hallja a babiloni király az õ híröket, és kezei elesnek, szorongás fogja el õt, fájdalom, mint a gyermekszûlõt.
\par 44 Ímé, mint a Jordán erdõségébõl való oroszlán, úgy jön fel az örökzöld ligetre, de hamar kiûzöm õt arról, és a ki arra választatott, azt teszem azon fejedelemmé, mert kicsoda hasonlatos hozzám? és ki szab nékem törvényt, és ki az a pásztor, a ki ellenem álljon?
\par 45 Azért halljátok meg az Úr tervét, a melyet Babilon ellen tervezett, és az õ gondolatait, a melyeket Káldea ellen gondolt. Bizony elhajtják õket, a nyáj kicsinyeit, és álmélkodik felettök a legelõ.
\par 46 Babilon bevételének zajától megindul a föld, és kiáltása hallatszik a nemzetek között!

\chapter{51}

\par 1 Ezt mondja az Úr: Ímé, én pusztító szelet támasztok Babilon ellen és azok ellen, a kik az én ellenségem szívében lakoznak.
\par 2 És szórókat küldök Babilon ellen, és felszórják õt, és földét kiüresítik, mert mindenfelõl ellene lesznek a veszedelem napján.
\par 3 A kézívesre kézíves vonja fel íjját, és arra, a ki pánczéljába öltözik. Ne kedvezzetek ifjainak, öldössétek le egész seregét:
\par 4 És essenek el megöletve a Káldeusok földén, és átverve az õ utczáin.
\par 5 Mert nem hagyatott el Izráel és Júda az õ Istenétõl, a Seregek Urától, noha az õ földök rakva vétekkel az Izráelnek Szentje ellen.
\par 6 Fussatok ki Babilonból, és kiki mentse meg az õ lelkét, ne veszszetek el az õ gonoszságáért, mert az Úr bosszúállásának ideje ez, megfizet néki érdem szerint.
\par 7 Arany pohár volt Babilon az Úr kezében, a mely megrészegíté ez egész földet; nemzetek ittak az õ borából, azért bolondultak meg a nemzetek.
\par 8 Hamar elesett Babilon és összeomlott, jajgassatok felette, kössétek be balzsammal  az õ sebét, hátha meggyógyul!
\par 9 Gyógyítottuk Babilont, de nem gyógyult meg. Hagyjátok el õt, és menjünk kiki a maga földére, mert az égig hatott az õ ítélete, és felemelkedett a felhõkig.
\par 10 Kihozta az Úr a mi igazságainkat, jertek és beszéljük meg Sionban az Úrnak, a mi Istenünknek dolgát.
\par 11 Élesítsétek a nyilakat, töltsétek meg a tegzeket; felindította az Úr a Médiabeli királyok lelkét, mert Babilon ellen van az õ gondolatja, hogy elveszesse azt, mert az Úr bosszúállása ez, az õ templomáért való bosszúállása.
\par 12 Babilonnak kõfalain tûzzétek ki a zászlót, erõsítsétek meg az õrséget, szerezzetek vigyázókat, rendeljétek el a leseket: mert az Úr meggondolta és meg is cselekszi azokat, a miket szólott Babilon lakói ellen.
\par 13 Oh te, a ki lakozol a nagy vizek mellett, a kinek kincsed temérdek, eljött a te véged és a te rablásod határa!
\par 14 Megesküdt a Seregek Ura az õ lelkére, mondván: Bizony betöltelek téged emberekkel, mint sáskákkal, és diadalmas éneket énekelnek felõled.
\par 15 Az, a ki teremtette a földet az õ erejével, a ki megalapította a világot az õ bölcsességével, és kiterjesztette az egeket az õ értelmével.
\par 16 Egy szavával vizek zúgását szerez az égben, és felhõket visz fel a föld határairól, villámokat készít az esõhöz, és kihozza a szelet az õ tárházaiból.
\par 17 Minden ember bolonddá lett tudomány nélkül, minden ötvös megszégyenül a maga bálványa miatt, mert hazugság az õ öntése és nincs benne lélek.
\par 18 Hiábavalóságok ezek, nevetségre való mûvek, az õ megfenyíttetésök idején elvesznek.
\par 19 Nem ilyen a Jákób osztályrésze mert mindennek teremtõje és az õ örökségének pálczája, Seregek Ura az õ neve!
\par 20 Põrölyöm vagy te nékem, hadi fegyverem, és nemzeteket zúztam össze veled, és országokat vesztettem el veled.
\par 21 És általad zúztam össze a lovakat és lovagjaikat, és általad zúztam össze a szekeret és a benne ülõt.
\par 22 És összezúztam általad férfit és asszonyt, és összezúztam általad a vénet és a gyermeket, és összezúztam általad az ifjat és a szûzet,
\par 23 És összezúztam általad a pásztort és nyáját, és összezúztam általad a szántóvetõt és az õ igamarháját, és összezúztam általad a hadnagyokat és a fõembereket.
\par 24 És megfizetek Babilonnak és Káldea minden lakosának mindazokért az õ gonoszságaikért, a melyeket Sionban cselekedtek a ti szemeitek láttára, azt mondja az Úr.
\par 25 Ímé, én ellened fordulok, te romlásnak hegye, azt mondja az Úr, a ki az egész földet megrontottad, és kinyújtom reád kezemet, és levetlek téged a kõszikláról, és kiégett hegygyé teszlek téged.
\par 26 És belõled nem visznek követ a szegletre és a fundamentomra, mert örökkévaló pusztaság leszel, azt mondja az Úr.
\par 27 Tûzzétek ki a zászlót az országban, fújjátok meg a trombitát a nemzetek között, avassátok fel ellene a nemzeteket, gyûjtsétek össze ellene az Ararátnak, Menninek és Askenáznak országait, válaszszatok õ ellene hadvezért, hozzátok ki a lovakat mint rettenetes sáskasereget.
\par 28 Avassátok fel ellene a nemzeteket Médiának királyait, az õ hadnagyait és minden fõemberét, és az õ birodalmának egész földét.
\par 29 És megrendül a föld és rázkódik, mert az Úrnak gondolatai beteljesednek Babilon ellen, hogy Babilon földét pusztasággá, lakatlanná tegye.
\par 30 Babilon vitézei felhagytak a viadallal, erõsségeikben ülnek, elfogyott a vitézségök, asszonyokká lettek, felgyújtották lakhelyeit, zárait letörték.
\par 31 Futár futár elé fut, és hírmondó a hírmondó elé, hogy megjelentse a babiloni királynak, hogy bevétetett az õ városa mindenfelülrõl.
\par 32 És a révhelyek elfoglaltattak, és az álló tavak tûzzel kiszáríttattak, és a vitézek elrettentek.
\par 33 Mert ezt mondja a seregek Ura, az Izráel Istene: Babilon leánya olyan, mint a szérû, itt van az õ tapostatásának ideje, egy kis híja még, és eljõ az õ aratásának ideje.
\par 34 Benyelt engem, megemésztett engem Nabukodonozor, a babiloni király, üres edénynyé tett engem, benyelt engem mint a sárkány, betöltötte a hasát az én csemegéimmel, és kivetett engemet.
\par 35 Az én rajtam esett erõszak és az én testem Babilonra térjen, azt mondja a Sionnak lakója, és az én vérem Káldeának lakosaira, azt mondja Jeruzsálem!
\par 36 Azért ezt mondja az Úr: Ímé, én megítélem a te ügyedet, és bosszút állok éretted, és kiszáraztom az õ tengerét, és kiapasztom az õ forrását.
\par 37 És Babilon kõrakássá lesz, sárkányok lakhelyévé, csudává, csúfsággá és lakatlanná lesz.
\par 38 Együtt ordítanak, mint az oroszlánok, harsognak, mint az oroszlánkölykök.
\par 39 Az õ kedvöknek idején készítek nékik lakomát, és megrészegítem õket, hogy vígadjanak, és örökkévaló álmot aludjanak, és fel ne serkenjenek, azt mondja az Úr.
\par 40 Elõhozom õket, mint a bárányokat a megmetszésre, mint a kosokat a bakokkal egyetemben.
\par 41 Mint bevétetett Sésák, és elfoglaltatott az egész földnek dicséreti! Milyen útálattá lett Babilon a nemzetek között!
\par 42 Feljött Babilonra a tenger, habjainak özönével elboríttatott.
\par 43 Városai pusztává, sivataggá és kopár földdé lesznek, a melyen senki sem lakik, sem embernek fia át nem megy rajta.
\par 44 Megfenyítem Bélt is Babilonban, és kivonom szájából, a mit benyelt, és többé nem futnak hozzá a nemzetek, Babilonnak kõfala is ledõl.
\par 45 Jõjjetek ki belõle, oh én népem, és kiki szabadítsa meg lelkét az Úr haragjának tüzétõl.
\par 46 És el ne olvadjon a ti szívetek és ne féljetek a hírtõl, a mely hallatszik e földön, mikor egyik esztendõben hír jõ, és a másik esztendõben is a hír, hogy erõszakosság van a földön, uralkodó tör uralkodóra!
\par 47 Azért ímé, eljõnek a napok, és meglátogatom Babilon faragott képeit, és egész földe megszégyenül, és minden õ megöltjei elhullanak õ közötte.
\par 48 És örvendeznek Babilon felett az ég és a föld és minden benne valók, mert észak felõl eljõnek reá a pusztítók, azt mondja az Úr.
\par 49 Babilon is elesik, Izráel megölöttjei, a mint Babilonban is elhullottak az egész föld megölöttjei.
\par 50 Menjetek el, a kik megszabadultatok a fegyvertõl, meg ne álljatok; emlékezzetek meg a távolból az Úrról, és jusson eszetekbe Jeruzsálem.
\par 51 Megszégyenültünk, mert hallottuk a gyalázkodást, orczáinkat szégyen borította, mert idegenek jöttek az Úr házának szentségébe.
\par 52 Azért ímé, eljõnek a napok, azt mondja az Úr, és meglátogatom az õ faragott képeit, és egész földén sebesültek nyögnek.
\par 53 Ha az égbe hág is fel Babilon, és ha megerõsíti is az õ erõs magaslatát, pusztítók törnek reá tõlem, azt mondja az Úr.
\par 54 Kiáltás hallatszik Babilonból, és a Káldeusok földébõl nagy romlás.
\par 55 Mert elpusztítja az Úr Babilont, és kiveszíti belõle a nagy zajt, és zúgnak az õ habjai, mint a nagy vizek, hallatszik az õ szavok harsogása.
\par 56 Mert pusztító tör reá, Babilonra, és elfogatnak vitézei, eltörik az õ kézívök, mert a megfizetésnek Istene, az Úr, bizonynyal megfizet.
\par 57 És megrészegítem az õ fejedelmeit, bölcseit, hadnagyait, tiszttartóit és vitézeit, és örök álmot alusznak, és nem serkennek fel, azt mondja a király, a kinek neve a Seregek Ura!
\par 58 Ezt mondja a Seregek Ura: Babilon széles kõfala földig lerontatik, és az õ büszke kapuit tûz égeti meg, és a népek hiába munkálkodnak, és a nemzetek a tûznek, és kifáradnak.
\par 59 Ez a szó, a melylyel Jeremiás próféta utasította Seráját, Néria fiát, a ki a Mahásiás fia volt, mikor õ Babilonba méne Sedékiással, a Júda királyával, királyságának negyedik esztendejében; Serája pedig szállásmester vala.
\par 60 És megírá Jeremiás egy könyvben mindazt a veszedelmet, a mely Babilont fogja érni, mindezeket a beszédeket, a melyek megirattak Babilon felõl.
\par 61 És monda Jeremiás Serájának: Mikor Babilonba jutsz, és látod és elolvasod mind e szókat,
\par 62 Ezt mondjad: Uram, te szólottál e hely ellen, hogy elveszítsed ezt annyira, hogy lakó ne legyen benne embertõl baromig, hanem örökkévaló pusztaság legyen.
\par 63 És mikor e könyv olvasását elvégzed, köss reá követ, és hajítsd be az Eufrátes közepébe.
\par 64 És ezt mondd: Így merül el Babilon, és meg nem menekedik a veszedelemtõl, a melyet én hozok reá, akármint fáradjanak. Eddig vannak a Jeremiás beszédei.

\chapter{52}

\par 1 Huszonegy esztendõs volt Sedékiás, mikor uralkodni kezde, és tizenegy esztendeig uralkodék Jeruzsálemben, és az õ anyjának neve Hammutál vala, a Libnából való Jeremiás leánya.
\par 2 És gonoszt cselekedék az Úr szemei elõtt, mint Jojákim cselekedett vala.
\par 3 Mert az Úr haragjáért vala ez Jeruzsálemen és Júdán, míglen elveté õket színe elõl. Sedékiás ugyanis engedetlen lõn a babiloni királynak.
\par 4 És az õ uralkodásának kilenczedik esztendejében, a tizedik hónapban, a hónapnak tizedikén eljöve Nabukodonozor, a babiloni király, õ maga és egész serege Jeruzsálemre, és tábort járának ellene, és mindenfelõl sánczot vetének fel ellene.
\par 5 És megszállva lõn a város Sedékiásnak tizenegyedik esztendejéig.
\par 6 A negyedik hónapban, a hónapnak kilenczedikén nagy éhség támada a városban, annyira, hogy a föld népének kenyere sem vala.
\par 7 És bevéteték a város, és a harczosok mind elfutának, és éjszaka kimenének a városból a kapu felé, a két kõfal között, a melyek a király kertjénél valának (a Káldeusok pedig a város mellett valának köröskörül) és menének az úton, a mely sík földre viszen.
\par 8 A Káldeusok serege pedig ûzé a királyt, és utólérék Sedékiást a jerikói síkon, mert egész serege elfuta mellõle.
\par 9 Megfogák azért a királyt, és vivék õt a babiloni királyhoz Riblába, a Hamát földére, a ki törvényt monda reá.
\par 10 És leöleté a babiloni király a Sedékiás fiait szemei láttára, és Júda minden fejedelmét is leöleté Riblában.
\par 11 Sedékiás királynak pedig szemeit tolatá ki, és lánczra vereté és viteté õt a babiloni király Babilonba, és tömlöczbe veté õt halála napjáig.
\par 12 Az ötödik hónapban pedig, a hónapnak tizedikén (ez az esztendõ pedig a tizenkilenczedik esztendeje vala Nabukodonozornak, a babiloni királynak) Nabuzáradán, a vitézek feje, a ki áll vala a babiloni király elõtt, eljöve Jeruzsálembe.
\par 13 És felégeté az Úr házát és a király házát, és Jeruzsálemnek minden házait és minden nagy házat felégete tûzzel.
\par 14 És Jeruzsálem egész kõfalát köröskörül lerontá a Káldeusok mindenféle serege, a kik a vitézek fejével valának.
\par 15 A községnek szegényeibõl pedig és a többi nép közül, a kik a városban megmaradtak vala, és a szökevények közül, a kik elszöktek vala a babiloni királyhoz, és a sokaságnak maradékából foglyokat vive Nabuzáradán, a vitézek feje.
\par 16 De a föld szegényei közül ott hagyá Nabuzáradán, a vitézek feje, a szõlõmûveseket és szántóvetõket.
\par 17 A rézoszlopokat pedig, a melyek az Úr házában valának és a talpakat és a réztengert, a mely az Úr házában vala, összetörék a Káldeusok, és azoknak minden rezét elvivék Babilonba.
\par 18 A fazekakat is és a lapátokat, és késeket és a medenczéket, a tömjénezõket és a rézedényeket, a melyekkel szolgálnak vala, mind elvivék.
\par 19 És a csészéket, a serpenyõket, a medenczéket és a fazekakat, a gyertyatartókat, a tömjénezõket és a serlegeket, a mi arany, aranyban, a mi ezüst, ezüstben mind elvivé a vitézek feje.
\par 20 Két oszlop, egy réztenger és tizenkét rézökör vala, a melyek a talpak alatt valának, a melyeket Salamon király csináltatott vala az Úr házába. Mindezeknek az edényeknek reze megmérhetetlen vala.
\par 21 És az oszlopok küzül az egyik oszlopnak tizennyolcz sing vala a magassága, és tizenkét sing zsinór éri vala át köröskörül, vastagsága pedig négy ujjnyi, és belõl üres vala.
\par 22 Rézgömb vala rajta, és az egyik gömb magassága öt singnyi vala, és a hálók és gránátalmák a gömbön köröskörül mind rézbõl valának, és ilyen a második oszlop, és ilyenek a gránátalmák is.
\par 23 A gránátalma pedig kilenczvenhat vala kifelé, összesen száz gránátalma vala a hálón felül köröskörül.
\par 24 Elvivé a vitézek feje Seráját is, a fõpapot, és Sofóniást a második papot, és az ajtónak három õrizõjét.
\par 25 És a városból elvive egy fõembert, a ki felügyelõjök vala a harczosoknak, és hetet ama férfiak közül, a kik állanak vala a király elõtt, a kik a városban találtatának, és a seregek fõíródeákját, a ki a föld népét besorozta vala, és hatvan férfiút a föld népe közül, a kik a városban találtatának.
\par 26 És felvevé ezeket Nabuzáradán, a vitézek feje, és elvivé õket a babiloni királyhoz Riblába.
\par 27 És levágatá õket a babiloni király és megöleté õket Riblában, a Hamát földén, és fogságra viteték Júda az õ földérõl.
\par 28 Ennyi az a nép, a melyet fogságra vitete Nabukodonozor a hetedik esztendõben, háromezer és huszonhárom Júdabeli.
\par 29 A Nabukodonozor tizennyolczadik esztendejében Jeruzsálembõl nyolczszáz és harminczkét lélek.
\par 30 Nabukodonozor huszonharmadik esztendejében fogságba vitete Nabuzáradán, a vitézek feje hétszáznegyvenöt Júdabelit. Összesen négyezer és hatszáz lélek.
\par 31 Lõn pedig Jojákin júdabeli király fogságának harminczhetedik esztendejében, a tizenkettedik hónapban, a hónak huszonötödikén, felemelé Evil-Merodák, a babiloni király az õ uralkodásának elsõ esztendejében Jojákinnak, a Júda királyának fejét, és kivevé õt a tömlöczbõl.
\par 32 És szépen szóla vele, és királyi székét ama királyoknak széke fölé emelte, a kik vele valának Babilonban.
\par 33 És felcserélé tömlöczbeli ruháit, és mindenkor vele eszik vala ételt, életének minden napjában.
\par 34 És költségére állandó költség adatik vala néki a babiloni királytól minden napra, a halála napjáig, életének minden napjában.


\end{document}