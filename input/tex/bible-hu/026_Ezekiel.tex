\begin{document}

\title{Ezekiel}


\chapter{1}

\par 1 És lõn a harminczadik esztendõben, a negyedik hónapban, a hónap ötödikén, mikor én a foglyok közt a Kébár folyó mellett voltam: megnyilatkozának az egek, és láték esteni látásokat.
\par 2 A hónap ötödikén (ez az ötödik esztendeje Jojákin király fogságba vitelének).
\par 3 Valójában lõn az Úrnak beszéde Ezékiel paphoz, a Búzi fiához a Káldeusok földén, a Kébár folyó mellett, és lõn ott rajta az Úrnak keze.
\par 4 És látám, és ímé forgószél jött északról, nagy felhõ egymást érõ villámlással, a mely körül fényesség vala, közepébõl pedig mintha izzó ércz látszott volna ki, tudniillik a villámlás közepébõl.
\par 5 És belõle négy lelkes állat formája tetszék ki, és ez vala ábrázatjok: emberi formájok vala,
\par 6 És mindeniknek négy orczája vala, és négy szárnya mindenikõjöknek;
\par 7 És lábaik egyenes lábak, és lábaik talpa mint a borjú lábának talpa, és szikráznak vala, mint a simított ércz színe.
\par 8 Továbbá emberi kezek valának szárnyaik alatt négy oldalukon. Mind a négyõjöknek orczái és szárnyai.
\par 9 Szárnyaik egymás mellé lévén szerkesztve, nem fordultak meg jártukban, mindenik az õ orczája irányában megy vala.
\par 10 És orczájok formája vala emberi orcza, továbbá oroszlán-orcza mind a négynek jobbfelõl, és bikaorcza mind a négynek balfelõl, és sasorcza mind a négynek hátul;
\par 11 És ezek az õ orczáik. És szárnyaik felül kiterjesztve valának, mindeniknél két szárny összeér vala, kettõ pedig fedezé testöket.
\par 12 És mindenik az õ orczája irányában megy vala, a hová a lélek vala menendõ, oda mennek vala, meg nem fordulván jártukban.
\par 13 És a lelkes állatok közt látszék, mint egy égõ üszög, a mely lángolt, mint a fáklyák, ide s tova futkározva a lelkes állatok közt; és a tûznek fényessége vala, és a tûzbõl villámlás jöve ki.
\par 14 És a lelkes állatok ide s tova mozognak vala, mint a villámlás czikázása.
\par 15 És mikor ránéztem a lelkes állatokra, ímé egy-egy kerék vala a földön az állatok mellett mind az õ négy orczájok felõl.
\par 16 A kerekek mintha tarsiskõbõl készültek volna, és mind a négyüknek ugyanazon egy formája vala, és úgy látszának egybeszerkesztve, mintha egyik kerék a másik kerék közepében volna;
\par 17 Jártukban négy oldaluk felé mentek vala; meg nem fordulnak vala jártukban;
\par 18 És talpaik magasak valának és félelmesek, és e talpak rakva valának szemekkel köröskörül mind a négynél.
\par 19 És mikor járnak vala a lelkes állatok, járnak vala a kerekek is mellettök, és mikor fölemelkednek vala az állatok a földrõl, fölemelkednek vala a kerekek is.
\par 20 A hová a lélek vala menendõ, mennek vala, a hová tudniillik a lélek vala menendõ, és a kerekek fölemelkednek vala mellettök, mert a lelkes állatok lelke vala a kerekekben.
\par 21 Ha azok mentek, ezek is mennek vala, és ha azok álltak, ezek is állnak vala, és ha fölemelkedtek a földrõl, fölemelkednek vala a kerekek is mellettök, mert a lelkes állatok lelke vala a kerekekben.
\par 22 És vala mintegy mennyezet az állatok feje fölött, olyan mint a csodálatos kristály, kiterjesztve felül fejök felett.
\par 23 És a mennyezet alatt szárnyaik egyenesen valának, egyik a másikkal összeérvén; mindegyiknek kettõ vala, a melyek befedik vala innen, és mindegyiknek kettõ vala, a melyek befedik vala amonnan az õ testöket.
\par 24 És hallám szárnyaik zúgását, mint sok vizeknek zúgását, úgy mint a Mindenhatónak hangját, mikor járnak vala, zúgás hangját, mint valami tábornak zúgását; mikor állának, leeresztik vala szárnyaikat.
\par 25 És lõn kiáltás a mennyezeten felül, a mely vala fejök felett, és õk megállván, leeresztik vala szárnyaikat.
\par 26 És a mennyezeten felül, a mely fejök felett vala, látszék mint valami zafirkõ, királyi széknek formája, és a királyi széknek formáján látszék mint egy ember formája azon felül;
\par 27 És látám izzó érczként ragyogni, a melyet, mintha tûz vett volna körül derekának alakjától fogva és fölfelé; és derekának alakjától fogva és lefelé látám, mintha tûz volna. És fényesség vala körülötte,
\par 28 Mint a milyen a szivárvány, mely a felhõben szokott lenni esõs idõben, olyan vala a fényesség köröskörül. Ilyen vala az Úr dicsõségének formája, és látám, és orczámra esém, és hallám egy szólónak szavát.

\chapter{2}

\par 1 És monda nékem: Embernek fia! állj lábaidra, és szólok veled.
\par 2 És lélek jöve én belém, a mint szóla, és állata engem lábaimra, és hallám azt, a ki szól vala nékem.
\par 3 És mondá nékem: Embernek fia! küldelek én téged Izráel fiaihoz, a pártos nemzetségekhez, a kik pártot ütöttek ellenem; õk és atyáik vétkeztek ellenem mind e mai napig.
\par 4 A kemény orczájú fiakhoz és makacs szívûekhez küldelek téged, és ezt mondjad nékik: Így szól az Úr Isten!
\par 5 Õk pedig vagy hallják, vagy nem hallják, mivelhogy pártos ház, hadd tudják meg, hogy próféta volt köztük.
\par 6 Te pedig, embernek fia, ne félj tõlök, és az õ beszédöktõl se félj; ha bogácsok és tövisek vannak is veled, és ha skorpiókkal lakol is együtt; beszédöktõl ne félj, orczájoktól ne rettegj, mert õk pártos ház.
\par 7 És szóljad az én beszédimet nékik, vagy hallják, vagy nem, mert pártos ház.
\par 8 Te pedig, embernek fia, halld meg a mit én néked szólok. Ne légy pártos mint ez a pártos ház, nyisd föl szádat, és egyed, a mit én adok néked.
\par 9 És látám, és ímé egy kéz nyúlt felém, és ímé benne egy könyv türete vala.
\par 10 És kiterjeszté azt elõttem, és ímé be vala írva elõl és hátul, és írva valának reá gyászénekek és nyögések és jajszók.

\chapter{3}

\par 1 És mondá nékem: Embernek fia! a mi elõtted van, edd meg; edd meg ezt a türetet, és menj, szólj az Izráel házának.
\par 2 Felnyitám azért számat, és megéteté velem azt a türetet.
\par 3 És mondá nékem: Embernek fia! hasadat tartsd jól és belsõ részeidet töltsd meg ezzel a türettel, a melyet adok néked. És megevém azt, és lõn az én számban, mint az édes méz.
\par 4 És mondá nékem: Embernek fia! eredj, menj el az Izráel házához, és szólj az én szavaimmal nékik.
\par 5 Mert nem valami homályos ajkú és nehéz nyelvû néphez küldetel te, hanem az Izráel házához.
\par 6 Nem sok népekhez, a kik homályos ajkúak és nehéz nyelvûek, kiknek nem érthetnéd beszédöket; bizony, ha õ hozzájok küldöttelek volna, õk hallgatnának reád.
\par 7 De az Izráel háza nem akar téged hallgatni, mert nem akarnak engem hallgatni, mert az egész Izráel háza kemény homlokú és megátalkodott szívû.
\par 8 Ímé keménynyé tettem orczádat, a milyen az õ orczájok, és keménynyé homlokodat, a milyen az õ homlokuk.
\par 9 Olyanná, mint a gyémánt, a mely keményebb a tûzkõnél, tettem a te homlokodat; ne félj tõlök, és meg ne rettenj tekintetöktõl, mert pártos ház.
\par 10 És mondá nékem: Embernek fia! minden beszédimet, a melyeket szólok néked, vedd szívedbe, és füleiddel halld meg.
\par 11 És eredj, menj el a foglyokhoz, a te néped fiaihoz, és szólj és mondjad nékik: Így szól az Úr Isten, vagy hallják vagy nem.
\par 12 És fölemele engem a lélek, és hallék mögöttem nagy dörgés szavát: Áldott az Úrnak dicsõsége az õ helyérõl.
\par 13 És amaz állatok szárnyainak zúgását, a melyek egymást érik vala, és mellettök a kerekek csikorgását, és nagy dörgés szavát.
\par 14 És a lélek fölemele és elragada engem, és elmenék, elkeseredvén haragjában az én lelkem, az Úrnak keze pedig rajtam erõs vala.
\par 15 És eljuték Tél-Ábibba a foglyokhoz, a kik lakoznak vala a Kébár folyó mellett, és leülék, õk is ott ülvén; és ott ülék hét nap némán õ közöttük.
\par 16 És lõn hét nap mulva az Úr szava hozzám, mondván:
\par 17 Embernek fia! õrállóul adtalak én téged Izráel házának, hogy ha szót hallasz számból, intsd meg õket az én nevemben.
\par 18 Ha ezt mondom a hitetlennek: Halálnak halálával halsz meg, és te õt meg nem inted és nem szólasz, hogy visszatérítsd a hitetlent az õ gonosz útjáról, hogy éljen: az a gonosztevõ az õ vétke miatt hal meg, de vérét a te kezedbõl kívánom meg.
\par 19 De ha te megintetted a hitetlent, és õ meg nem tért hitetlenségébõl és gonosz útjáról: õ az õ vétke miatt meghal, de te megmentetted a te lelkedet.
\par 20 És ha elfordul az igaz az õ igazságától, és cselekszik álnokságot, és én vetek eléje botránkozást: õ meg fog halni; ha meg nem intetted õt, vétke miatt hal meg és elfelejtetnek igazságai, a melyeket cselekedett; de vérét a te kezedbõl kívánom meg.
\par 21 Ha pedig te megintetted azt az igazat, hogy az igaz ne vétkezzék, és õ nem vétkezik többé: élvén él, mert engedett az intésnek, és te a te lelkedet megmentetted.
\par 22 És lõn ott az Úrnak keze rajtam, és monda nékem: Kelj fel, menj ki a völgybe, és ott szólok veled.
\par 23 Fölkelék azért és kimenék a völgybe és ímé ott áll vala az Úrnak dicsõsége, hasonlatos ahhoz a dicsõséghez, a melyet a Kébár folyó mellett láttam, és orczámra esém.
\par 24 És jöve belém a lélek, és állata engem lábaimra, és szóla hozzám és monda nékem: Menj be és zárd be magadat a te házadban.
\par 25 És te, oh embernek fia, ímé köteleket vetnek reád és azokkal megkötöznek téged, és ki nem mehetsz közikbe;
\par 26 Nyelvedet pedig én ragasztom ínyedhez, és néma leszel, hogy ne légy közöttök feddõzõ férfiú, mert õk pártos ház.
\par 27 Mikor pedig szólok veled, megnyitom a te szádat, és mondjad nékik: Így szól az Úr Isten; a ki hallja hallja, a ki nem akarja, nem hallja, mert õk pártos ház.

\chapter{4}

\par 1 És te, embernek fia, végy magadnak egy téglát, tedd azt elõdbe, és véss reá egy várost, Jeruzsálemet,
\par 2 És indíts ellene ostromot, és építs ellene tornyot, tölts ellene sánczot, és indíts ellene táborokat, és állass ellene faltörõ kosokat köröskörül.
\par 3 És végy magadnak egy vasserpenyõt, és állasd fel azt vasfal gyanánt te közted és a város között, és irányozd tekintetedet erõsen reá, és legyen ostrom alatt, és te ostromold. Jel ez az Izráel házának.
\par 4 Te pedig feküdj baloldaladra és vesd az Izráel háza vétkét arra; a napok száma szerint, a mennyin azon fekszel, viseljed vétköket.
\par 5 Én pedig meghatároztam néked az õ vétkök éveit napok száma szerint, háromszázkilenczven napban; eddig viseljed az Izráel házának vétkét.
\par 6 És ha ezeket kitöltötted, feküdj a jobboldaladra másodszor, és viseld a Júda házának vétkét negyven napig; egy-egy napot egy-egy esztendõül számítottam néked.
\par 7 És Jeruzsálem ostromára irányozd erõsen tekintetedet, és karod feltûrve legyen, és prófétálj õ ellene.
\par 8 S ímé köteleket vetettem reád, hogy meg ne fordulhass egyik oldaladról a másikra. míg betöltöd ostromodnak napjait.
\par 9 És végy magadnak búzát és árpát és babot és lencsét és kölest és tönkölyt, és tedd ezeket egy edénybe, és ezekbõl csinálj magadnak kenyeret; a napok száma szerint, a míg oldaladon fekszel, háromszázkilenczven napon egyed azt.
\par 10 A te ételed pedig, a melylyel élsz, legyen súly szerint húsz siklus egy napra; idõrõl-idõre egyed azt.
\par 11 És vizet mérték szerint igyál, a hinnek hatodrészét igyad idõrõl-idõre.
\par 12 És ételedet árpa-lepény formájában egyed, és emberi ganéj tõzegénél süssed azt szemök láttára.
\par 13 És mondá az Úr: Így eszik az Izráel fiai tisztátalan kenyeröket a pogányok közt, a kik közé õket kiûzöm.
\par 14 És mondék: Ah, ah, Uram Isten! ímé az én lelkem soha meg nem fertéztetett, és dögöt és vadtól szaggatottat nem ettem ifjúságomtól fogva ez ideig, és számon be nem ment tisztátalan hús.
\par 15 És mondá nékem: Nézd, marhaganéjt engedek néked emberi tõzeg helyett, hogy annál süsd meg a te kenyeredet.
\par 16 És mondá nékem: Embernek fia! ímé én eltöröm a kenyérnek botját Jeruzsálemben, és eszik kenyeröket mértékkel és rettegéssel, és vizöket mértékkel és ájulással iszszák.
\par 17 Azért, hogy kenyér és víz nélkül szûkölködjenek, és elborzadjanak mindnyájan, és megrothadjanak az õ  vétkökben.

\chapter{5}

\par 1 És te, embernek fia, végy magadnak éles kardot, borbélyok beretvájául vedd azt magadnak, és vond el azt a te fejeden és szakálladon, és végy magadnak mérõ serpenyõket, és oszd el szõrüket.
\par 2 Harmadrészét tûzben égesd meg a város közepette, midõn betelnek a megszállásnak napjai; azután vedd a harmadrészét, vagdald apróra a karddal a város körül, és harmadrészét szórd oda a szélnek, és én kardot vonszok utánok.
\par 3 És végy ki innét szám szerint keveset, és kösd be azokat ruhád csücskébe;
\par 4 És ezekbõl ismét végy ki, és vesd azokat a tûz közepébe és égesd meg a tûzben; ebbõl megyen tûz Izráel egész házára.
\par 5 Így szól az Úr Isten: Ez Jeruzsálem, a pogányok közibe helyheztettem õt, és körülte a tartományokat.
\par 6 De pártos volt törvényeim iránt, gonoszabbul, mint a pogányok, és rendeléseim iránt inkább, mint a tartományok, a melyek körülte vannak, mert törvényeimet megútálták, és rendeléseimben nem jártak.
\par 7 Azért így szól az Úr Isten: A miért ti pártosabbak valátok, mint a pogányok, a kik körültetek vannak, rendeléseimben nem jártatok és törvényeimet nem cselekedtétek, sõt csak a pogányok törvényei szerint is, a kik körültetek vannak, nem cselekedtetek:
\par 8 Ezokért így szól az Úr Isten: Ímé én is ellened leszek, és teszek közötted ítéletet a pogányok szeme láttára;
\par 9 És cselekszem rajtad azt, mit soha nem cselekedtem és minémût nem cselekszem többé, - minden te útálatosságidért.
\par 10 Azért az apák egyék meg fiaikat te közötted, és a fiak egyék meg apáikat, és cselekszem rajtad ítéletet, és szétszórom minden maradékodat a szél minden irányában.
\par 11 Ezért, élek én! szól az Úr Isten, bizonyára, mivelhogy szenthelyemet megfertéztetted minden undokságaiddal és minden útálatosságaiddal, azért én is elfordítom rólad irgalom nélkül szememet, s én sem könyörülök rajtad.
\par 12 Harmadrészed döghalállal hal meg és éhség miatt pusztul el közötted, és harmadrészed fegyver miatt hull el körülötted, és harmadrészedet szétszórom a szél minden irányában, és kardot vonszok utánok.
\par 13 És teljessé lesz haragom s nyugtatom rajtok búsulásomat s vígasztalást veszek; és megértik, hogy én, az Úr szóltam buzgó szerelmemben, mikor betöltöm búsulásomat rajtok.
\par 14 És teszlek pusztasággá és gyalázattá a pogányok között, a kik körülted vannak, minden melletted elmenõ szeme láttára.
\par 15 És leszel gyalázat és csúfság, példa és eliszonyodás a pogányoknak, a kik körülted vannak, mikor ítéletet tartok fölötted haraggal és búsulással és búsult feddésekkel, én, az Úr mondottam;
\par 16 Mikor bocsátom az éhség gonosz nyilait rájok, hogy pusztítsanak, a melyeket a ti pusztítástokra fogok bocsátani; és éhséget halmozok fölétek, és eltöröm köztetek a kenyér botját.
\par 17 És bocsátok reátok éhséget és gonosz vadállatokat, hogy gyermektelenné tegyenek, és döghalál és vérontás megy át rajtad, és fegyvert hozok reád, én, az Úr mondottam.

\chapter{6}

\par 1 És lõn az Úrnak szava én hozzám, mondván:
\par 2 Embernek fia, vesd tekintetedet Izráel hegyeire, és prófétálj ellenök:
\par 3 És mondjad: Izráel hegyei, halljátok meg az Úr Isten beszédét! Ezt mondja az Úr Isten a hegyeknek és a halmoknak, a mélységeknek és a völgyeknek: Ímé én fegyvert hozok reátok, és elvesztem a ti magaslataitokat.
\par 4 És elpusztulnak oltáraitok, és összetörnek naposzlopaitok, és elhullatom sebesültjeiteket bálványaitok elõtt.
\par 5 És vetem az Izráel fiainak holttesteit bálványaik elé, és szétszórom csontjaitokat oltáraitok körül.
\par 6 Minden lakóhelyeteken a városok elpusztuljanak, és a magaslatok elveszszenek, hogy elpusztuljanak és rommá legyenek oltáraitok, és törjenek össze és legyenek semmivé bálványaitok, és kivágattassanak naposzlopaitok, és eltöröltessenek csinálmányaitok.
\par 7 És elhulljon a sebesült közöttetek, hogy megtudjátok, hogy én vagyok az Úr.
\par 8 De maradékot hagyok hogy legyenek közületek, a kik megmenekedtek a fegyvertõl a pogányok közt, mikor szétszórattok a tartományokban.
\par 9 Akkor megemlékeznek én rólam menekültjeitek a pogányok közt, kik közé fogságba vitetének, mert megtörtem parázna szívöket, mely tõlem elszakadt, és bálványaikkal paráználkodó szemeiket, és megundorodnak önmagok elõtt a gonoszságokért, melyeket cselekedtek minden útálatosságuk szerint;
\par 10 És megismerik, hogy én vagyok az Úr: nem hiába mondottam, hogy megcselekszem velök e gonoszt.
\par 11 Így szólt az Úr Isten: Csapj tenyeredbe, toppants lábaddal és mondd: Jaj az Izráel háza minden gonosz útálatosságáért, mert fegyver, éhség és döghalál miatt hullanak el;
\par 12 A ki messze van, döghalál miatt hal meg, a ki közel van, fegyver miatt esik el, és a ki megmaradt biztonságban, éhség miatt hal meg; és teljessé teszem búsulásomat rajtok.
\par 13 És megtudjátok, hogy én vagyok az Úr, mikor sebesültjeik ott lesznek bálványaik között az õ oltáraik körül, minden magas  halmon, minden hegyeknek tetein, minden zöld fa alatt és minden lombos terpentinfa alatt, a hol csak kedves illatot adának minden bálványaiknak.
\par 14 Kinyújtom azért kezemet reájok, és teszem a földet kietlen pusztasággá, a pusztától fogva Dibláig minden lakóhelyökön, hadd tudják meg, hogy én vagyok az Úr!

\chapter{7}

\par 1 És lõn az Úr beszéde hozzám, mondván:
\par 2 És te, embernek fia, így szól az Úr Isten Izráel földjének: Vége! eljött a vég a föld négy szárnyára!
\par 3 Immár itt a vég rajtad; s bocsátom haragomat reád, és megítéllek útaid szerint, és vetem reád minden útálatosságodat.
\par 4 És nem kedvez az én szemem néked, sem meg nem szánlak; hanem a te útaidat vetem reád, és útálatosságaid közötted lesznek és megtudjátok, hogy én vagyok az Úr.
\par 5 Így szól az Úr Isten: Ímé veszedelem, egyetlen veszedelem; ímé eljött.
\par 6 Vég jött, eljött a vég, fölserkent ellened, ímé eljött!
\par 7 Eljött a végzet reád, földnek lakosa! eljött az idõ, közel a nap, rémülés és nem víg éneklés a hegyeken.
\par 8 Most rövid idõn kiöntöm búsulásomat reád, és teljessé teszem haragomat rajtad, és megítéllek útaid szerint, és rád vetem minden útálatosságodat.
\par 9 És nem kedvez az én szemem, sem meg nem szánlak; útaid szerint fizetek tenéked, és a te útálatosságaid közötted lesznek; és megtudjátok, hogy én vagyok az Úr, a ki ver.
\par 10 Ímé a nap, ímé eljött, kisarjadt a végzet, kivirágzott a vesszõ, kivirult a kevélység.
\par 11 Az erõszakosság a gonoszság veszszejévé nõtt fel, nincs semmi meg belõlök, sem sokaságukból, sem tömegökbõl, s nincs egy jaj is miattok!
\par 12 Eljött az idõ, elközelgett a nap; a vevõ ne örüljön, az eladó ne szomorkodjék, mert harag jön minden sokaságára.
\par 13 Mert az eladó eladott jószágához nem térhet vissza többé, még ha élve az élõk közt maradna is, mert a jövendelés az õ egész sokasága ellen vissza nem tér, és vétke miatt senki sem lehet hosszú életû.
\par 14 Kürtöljetek a kürttel és készítsetek el mindent; ám nincsen, a ki harczra menjen, mert haragom minden õ sokasága ellen.
\par 15 A fegyver kivül, a döghalál és éhség belül; a ki a mezõn van, fegyver miatt hal meg, és a ki a városban, azt éhség és döghalál emészti meg.
\par 16 És menekülnek menekültjeik, és lesznek a hegyeken, mint a völgyek galambjai: mindnyájan nyögvén, kiki vétke miatt.
\par 17 Minden kéz elerõtlenül, és minden térd elolvad, mint a víz.
\par 18 Felövezkednek zsákkal, és befedi õket rettegés, és minden orczán szégyen, és mindnyájok fején kopaszság.
\par 19 Ezüstjöket az utczákra vetik, és aranyuk szenny lesz elõttök; ezüstjök s aranyuk meg nem szabadíthatja õket az Úr búsulásának napján; lelköket azzal jól nem lakatják, s hasokat meg nem tölthetik; mert csábítójok volt az a vétekre.
\par 20 És a belõle készült drága ékességeket kevélykedésre használják, és útálatosságuk képeit, undokságaikat abból csinálták, azért tettem elõttök azt szenynyé;
\par 21 És adom azt az idegenek kezébe zsákmányul, és a föld hitetleninek prédául, hadd fertéztessék meg.
\par 22 És elfordítom tõlük arczomat, hadd fertéztessék meg szent helyemet; s betörjenek belé a rontók és megfertéztessék.
\par 23 Készítsd a lánczot; mert a föld tele van véres ítélettel,  és a város tele van erõszakossággal.
\par 24 És elhozom a pogányok leggonoszabbjait, hadd foglalják el házaikat; s véget vetek a hatalmasok kevélységének, s fertézett lesz templomuk.
\par 25 Rettegés jött el, s keresnek békét és nincs.
\par 26 Egy romlás a másikra jõ, és egy hír után más támad, s kérnek látást a prófétától, ám törvény nem lesz a papnál, sem tanács a véneknél.
\par 27 A király szomorkodik, a fejedelem irtózatba öltözik; s a föld népének kezei megdermednek. Útjok szerint cselekszem velök, ítéletök szerint ítélem meg õket, hadd tudják meg, hogy én vagyok az Úr.

\chapter{8}

\par 1 És lõn a hatodik esztendõben, a hatodik hónapban, a hónap ötödik napján: én ülök vala házamban, és Júda vénei ülnek vala elõttem és esék reám ott az Úr Istennek keze.
\par 2 És látám, és ímé vala mintegy tûznek formája, derekának alakjától fogva lefelé tûz vala, és derekától fogva fölfelé vala mint a fényesség, mint az izzó ércz.
\par 3 És kinyújta egy kézformát, és megragada engem fejem üstökénél fogva, és fölemelve vitt engem a lélek a föld és az ég között, és bevive engem Jeruzsálembe isteni látásokban a belsõ kapu bejáratához, a mely északra néz, a hol vala helye a bosszúság bálványának, a mely bosszúságra  ingerel vala;
\par 4 És ímé ott vala Izráel Istenének dicsõsége, a látás szerint, a melyet láttam a völgyben.
\par 5 És monda nékem: Embernek fia! emeld föl csak szemeidet észak felé. Fölemelém azért szemeimet észak felé: és ímé északra az oltár kapujától áll vala a bosszúság ama bálványa a bejáratnál.
\par 6 És mondá nékem: Embernek fia! látod-e mit cselekesznek? A nagy útálatosságokat, melyeket Izráel háza itt cselekszik, hogy eltávozzam az én szenthelyemtõl. De még egyéb nagy útálatosságokat is fogsz látni.
\par 7 És vive engem a pitvar bejáratához, és látám, és ímé egy lyuk vala a falban.
\par 8 És mondá nékem: Embernek fia! ronts csak át a falon! és átronték a falon és ímé egy ajtó vala ott.
\par 9 És mondá nékem: Menj be és lásd meg a gonosz útálatosságokat, a melyeket ezek ott cselekesznek.
\par 10 Bemenék azért és látám, és ímé az útálatos csúszó-mászó állatoknak és barmoknak mindenféle képei és Izráel házának minden bálványai vannak bevésve a falon köröskörül.
\par 11 És hetven férfiú Izráel házának vénei közül (ezek közepette Jaazanjáhu, a Sáfán fia) áll vala elõttök, mindenik a maga tömjénezõjével kezében, s a füstölõszer felhõjének illata száll vala fel.
\par 12 És mondá nékem: Láttad-é, embernek fia, Izráel házának vénei mit cselekesznek a sötétben, kiki az õ képes házában? mert azt mondják: Nem lát minket az Úr, elhagyta az Úr ezt a földet.
\par 13 És mondá nékem: Még egyéb nagy útálatosságokat is fogsz látni, miket ezek cselekesznek.
\par 14 És vive engem az Úr háza kapujának bejáratához, a mely északra van, és ímé ott asszonyok ülnek vala, siratván a Tammúzt.
\par 15 És mondá nékem: Láttad-é, embernek fia? még egyéb, ezeknél nagyobb útálatosságokat is fogsz látni.
\par 16 És bevive engem az Úr házának belsõ pitvarába, és ímé az Úr templomának bejáratánál, a tornácz és az oltár között vala mintegy huszonöt férfiú, kik hátokkal az Úr templomára és orczájokkal keletre fordultak, s ezek kelet felé leborulva imádták a napot.
\par 17 És mondá nékem: Láttad-é, embernek fia? avagy kevés-é Júda házának ily útálatosságokat cselekedni, a milyeneket itt cselekedtek? hogy még a földet is betöltik erõszakossággal, és engem megint haragra ingerelnek, ímé, hogy tartják a venyigét orrukhoz!
\par 18 Én is búsulásom szerint cselekszem, nem fog kedvezni szemem, sem meg nem szánom õket; s ha kiáltanak füleimbe nagy felszóval, nem hallgatom meg  õket.

\chapter{9}

\par 1 És kiáltá füleimbe nagy felszóval, mondván: Hozzátok el a városra a meglátogatásokat, kinek-kinek a kezében legyen vesztõ eszköze.
\par 2 És ímé hat férfi jõ vala a felsõ kapu útjáról, a mely északra néz vala, mindeniknek kezében zúzó eszköze, egy férfi pedig köztük gyolcsba vala öltözve, és íróeszköz vala derekán. És bemenének és állának az érczoltár mellé.
\par 3 És Izráel Istenének dicsõsége elvonula a Kérubról, a mely fölött vala, a ház küszöbéhez, és kiálta a gyolcsba öltözött férfiúnak, a kinek derekán  íróeszköz vala.
\par 4 És monda az Úr néki: Menj át a város közepén, Jeruzsálem közepén, és jegyezz egy jegyet a férfiak homlokára, a kik sóhajtanak és nyögnek mindazokért az útálatosságokért, a melyeket cselekedtek annak közepében.
\par 5 És amazoknak mondá az én hallásomra: Menjetek át a városon õ utána, és vágjátok; ne kedvezzen a ti szemetek, és ne szánakozzatok:
\par 6 Vénet, ifjat, szûzet, gyermeket és asszonyokat öljetek meg mind egy lábig, de azokhoz a férfiakhoz, a kiken a jegy van, ne közelítsetek, és az én templomomon kezdjétek el. Elkezdék azért a vén férfiakon,  a kik a ház elõtt valának.
\par 7 És mondá nékik: Fertõztessétek meg a házat, és töltsétek meg a pitvarokat megölettekkel. Menjetek ki. És kimenének és öldöklének a városban.
\par 8 És lõn, hogy levágák õket, és én megmaradtam és esém az én orczámra és kiálték és mondék: Ah, ah, Uram Isten, avagy ki akarod-é írtani Izráel egész maradékát, mikor kiöntöd búsulásodat Jeruzsálemre?
\par 9 És mondá nékem: Izráel és Júda házának vétke felette nagy, mivelhogy tele a föld  vérontással, és a város tele van igazságtalansággal, mert azt mondották: Elhagyta az Úr ezt a földet, és az Úr nem lát.
\par 10 Azért én is (nem kedvez szemem, sem meg nem szánom õket) útjokat fejökhöz verem!
\par 11 És ímé a gyolcsba öltözött férfi, kinek íróeszköz vala a derekán, választ hozott, mondván: Úgy cselekedtem, a mint parancsolád.

\chapter{10}

\par 1 És látám, és ímé a mennyezeten, a mely a Kérubok feje fölött vala, látszék felettök, mint valami zafirkõ, olyan, mint egy királyi széknek formája.
\par 2 És szóla a gyolcsba öltözött férfiúnak, és mondá: Menj be a forgókerekek közé a Kérubok alá, és töltsd meg tenyereidet égõ üszöggel onnét a Kérubok közül, és szórd a városra. És beméne szemem láttára.
\par 3 A Kérubok pedig állanak a háztól jobbra, mikor a férfi beméne, és a felhõ betölté a belsõ pitvart.
\par 4 És eltávozék az Úr dicsõsége a Kérubról, a ház küszöbére, és megtelék a ház a felhõvel, és a pitvar betelék az Úr dicsõségének fényességével.
\par 5 És a Kérubok szárnyainak csattogása meghallaték a külsõ pitvarig, mint az erõs Isten hangja, mikor beszél.
\par 6 És lõn, mikor parancsolt ama gyolcsba öltözött férfinak, mondván: Végy tüzet a forgókerekek közül, a Kérubok közül, az beméne, és álla a kerék mellé.
\par 7 És kinyújtá egy Kérub a kezét a Kérubok közül a tûzhöz, mely vala a Kérubok között, és võn és tevé a gyolcsba öltözöttnek markába, ki elvevé és kiméne.
\par 8 És látszék a Kérubokon emberi kéznek formája a szárnyaik alatt.
\par 9 És látám, és ímé, négy kerék vala a Kérubok mellett, egyik kerék vala az egyik Kérub mellett és a másik kerék a másik Kérub mellett, és olyanok valának a kerekek, mintha tarsiskõbõl volnának.
\par 10 És mintha volna mind a négyöknek ugyanazon egy formája, mintha egyik kerék a másik kerék közepében volna.
\par 11 Jártokban mind a négy oldaluk felé mennek vala, meg nem fordulnak vala jártokban, hanem arra, a merre a fõ fordult, mennek vala utána, meg nem fordulnak vala mentökben.
\par 12 És egész testök és hátok és kezeik és szárnyaik és a kerekek rakva valának szemekkel köröskörül mind a négy kerekükön.
\par 13 Hallám, hogy a kerekeket forgókerekeknek nevezték fülem hallatára.
\par 14 És négy orczája vala mindeniknek, az elsõ orcza vala Kérub-orcza, és a második orcza ember-orcza, a harmadik oroszlán-orcza és a negyedik sas-orcza.
\par 15 És fölemelkedének a Kérubok. Ez ama lelkes állat, a melyet láttam a Kébár folyó mellett.
\par 16 És mikor járnak vala a Kérubok, járnak vala a kerekek is mellettök, mikor pedig felemelék a Kérubok szárnyaikat, hogy fölemelkedjenek a földrõl, nem fordulának el a kerekek sem az õ oldaluktól.
\par 17 Ha azok állanak vala, ezek is állának, és ha azok felemelkednek vala, fölemelkedének ezek is velök, mert a lelkes állat lelke vala bennök.
\par 18 És elvonula az Úrnak dicsõsége a ház küszöbétõl, és álla a Kérubok fölé.
\par 19 És fölemelék a Kérubok szárnyaikat, és fölemelkedének a földrõl szemem láttára kimentökben és a kerekek is mellettök; és megállának az Úr háza keleti kapujának bejáratánál, és Izráel Istenének dicsõsége vala felül õ rajtok.
\par 20 Ez ama lelkes állat, a melyet láttam az Izráel Istene alatt a Kébár folyó mellett, és megismerém, hogy Kérubok valának.
\par 21 Négy orczája vala mindeniknek és négy szárnya mindeniknek és emberi kezek formája vala szárnyaik alatt.
\par 22 És orczáik formája: ugyanazok az orczák valának, a melyeket a Kébár folyó mellett láttam, tudniillik formájokat és õket magokat. Mindenik a maga orczája felé megy vala.

\chapter{11}

\par 1 És fölemelt engem a lélek, és bevive az Úr házának keleti kapujához, a mely néz keletnek, és ímé, a kapu bejáratánál huszonöt férfi vala, kik között látám Jaazanját, Azzur fiát, és Pelatjáhut, Benájáhu fiát, a nép fejedelmeit.
\par 2 És mondá nékem: Embernek fia! ezek a férfiak, a kik gonoszt eszelnek ki és rossz tanácsot adnak ebben a városban,
\par 3 Mondván: Nem egyhamar fogunk házakat építeni; ez a város a fazék, mi pedig a hús.
\par 4 Azért prófétálj ellenök; prófétálj, embernek fia!
\par 5 És esék reám az Úr lelke, és mondá nékem: Mondjad, így szól az Úr: Így szólottatok, Izráel háza! és a mi lelketekben készül, én tudom.
\par 6 Sokakat öltetek meg ebben a városban, és utczáit megtöltöttétek megölettekkel.
\par 7 Ezért, így szól az Úr Isten: Megöletteitek, kiket a város közepére vetettetek, ezek a hús, a város pedig a fazék, és titeket kivisznek belõle.
\par 8 Fegyvertõl féltetek, fegyvert hozok reátok azt mondja az Úr Isten.
\par 9 És kiviszlek titeket belõle, és adlak titeket idegenek kezébe, és tartok ítéletet fölöttetek.
\par 10 Fegyver miatt hulljatok el, Izráel határán ítéllek meg titeket, és megtudjátok, hogy én vagyok az Úr.
\par 11 E város ne legyen fazekatok, hogy ti benne hús legyetek, Izráel határán ítéllek el titeket.
\par 12 És megtudjátok, hogy én vagyok az Úr, mert az én végzéseimben nem jártatok, és rendeléseimet nem cselekedtétek, hanem a pogányok módja szerint cselekedtetek, a kik körültetek vannak.
\par 13 És lõn, mikor prófétáltam, Pelatjáhu, Benájáhu fia meghala, én pedig orczámra esém és kiálték nagy felszóval, és mondék: Ah, ah, Uram Isten, te véget vetsz Izráel maradékának!
\par 14 És lõn az Úrnak szava én hozzám, mondván:
\par 15 Embernek fia! a te atyádfiai, atyádfiai, a te rokonaid és Izráel egész háza együtt azok, a kikrõl Jeruzsálem lakói ezt mondják: távozzatok el az Úrtól, nékünk adatott ez a föld örökségül.
\par 16 Ezokáért mondjad: így szól az Úr Isten: Mivelhogy távol vetettem õket a pogányok közé, és szétszórtam õket a tartományokba, tehát én leszek nékik templomul rövid idõre a tartományokban, a melyekbe mentek.
\par 17 Ennekokáért mondjad: Így szól az Úr Isten: Egybegyûjtelek titeket a népek közül és együvé hozlak titeket a tartományokból, a melyekben szétszórattatok, és adom néktek Izráel földjét.
\par 18 És bemennek oda és eltávolítják minden õ fertelmességeit és minden útálatosságait õ belõle.
\par 19 És adok nékik egy szívet, és új lelket  adok belétek, és eltávolítom a kõszívet az õ testökböl, és adok nékik hússzívet;
\par 20 Hogy az én végzéseimben járjanak és rendeléseimet megõrizzék és cselekedjék azokat, és legyenek nékem népem és én leszek nékik istenök.
\par 21 De a kiknek szívök az õ fertelmességeik és útálatosságaik szíve szerint jár, azoknak útját fejökhöz verem, mondja az Úr Isten.
\par 22 És fölemelék a Kérubok szárnyaikat s a kerekek mellettök, és Izráel Istenének dicsõsége rajtok felül vala.
\par 23 És felszálla az Úrnak dicsõsége a város közepébõl, és álla a hegyre, mely a várostól keletre van.
\par 24 A lélek pedig felvõn engem, és vive Káldeába a foglyokhoz látásban az Isten lelke által, és felszálla elõlem a látás, a melyet láttam.
\par 25 És elbeszélém a foglyoknak az Úrnak minden beszédét, a melyet nékem megjelentetett.

\chapter{12}

\par 1 És lõn az Úrnak szava énhozzám, mondván:
\par 2 Embernek fia! pártos ház közepette lakol, kiknek szemeik vannak a látásra, de nem látnak, füleik vannak a hallásra, de nem hallanak, mert õk pártos ház.
\par 3 És te, embernek fia, készíts magadnak vándorútra való eszközöket, és vándorolj ki nappal szemeik elõtt, és vándorolj ki helyedrõl más helyre szemök láttára; talán meglátják! mert õk pártos ház.
\par 4 És vidd ki eszközeidet, úgy, mint vándorútra való eszközöket, nappal szemök láttára, te pedig menj ki estve szemök láttára, úgy a hogy a vándorok szoktak.
\par 5 Szemök láttára lyukaszd át a falat, és azon át vidd ki.
\par 6 Szemök láttára emeld válladra, a sötétben vidd ki, orczádat fedd be, hogy lásd a földet, mert csodajelül rendeltelek az Izráel házának.
\par 7 Úgy cselekedtem azért, a mint parancsolva vala nékem; eszközeimet kihordám nappal, mint vándorútra való eszközöket, és este átlyukasztám a falat kezemmel; a sötétben kivivém, vállamra emelém szemök láttára.
\par 8 És lõn az Úr beszéde én hozzám reggel, mondván:
\par 9 Embernek fia! Nem mondta-é néked Izráel háza, ez a pártos ház: mit cselekszel?
\par 10 Mondjad nékik: Így szól az Úr Isten: a fejedelemnek szól ez a próféczia, ki Jeruzsálemben van, és Izráel egész házának, a mely ott lakozik.
\par 11 Mondjad: Én csodajeletek vagyok; a mint én cselekedtem, úgy történik velök: fogságba, rabságra mennek.
\par 12 És a fejedelem, ki közöttök van, vállát megrakván a sötétben, kimegyen; a falat átlyukasztják, hogy így vigyék ki õt, orczáit befedi, hogy ne lássa szemeivel épen õ a földet.
\par 13 És kiterjesztem hálómat ellene, és megfogatik varsámban, és elviszem õt Bábelbe a Káldeusok földére, de azt nem fogja látni és ott fog meghalni.
\par 14 És mindeneket, kik körülte vannak az õ segítségére, és minden seregeit szélnek szórom mindenfelé, és kardot vonok utánok.
\par 15 És megtudják, hogy én vagyok az Úr, mikor eloszlatom õket a pogányok közé, és szétszórom õket a tartományokba.
\par 16 De meghagyok közülök kevés férfiakat a fegyvertõl, éhségtõl s döghaláltól, hogy elbeszéljék minden útálatosságukat a pogányok közt, a kik közé mennek, s hogy megtudják, hogy én vagyok az Úr.
\par 17 És lõn az Úr beszéde hozzám, mondván:
\par 18 Embernek fia! kenyeredet rettegéssel egyed, és vizedet reszketéssel és félelemmel igyad.
\par 19 És szólj a föld népének: Ezt mondja az Úr Isten Jeruzsálem lakóiról, Izráel földjérõl: kenyeröket félelemmel eszik és vizöket ájulással iszszák, hogy pusztaságra vetkõzzék földje bõségébõl minden lakói álnoksága miatt.
\par 20 És a lakott városok elpusztulnak s a föld pusztaság lesz és megtudjátok, hogy én vagyok az Úr.
\par 21 És lõn az Úr beszéde hozzám, mondván:
\par 22 Embernek fia! micsoda közmondástok van néktek Izráel földjén? hogy azt mondják: a napok csak haladnak, ám semmivé lesz minden látás.
\par 23 Ezokért mondd nékik: Ezt mondja az Úr Isten: Megszüntetem e közmondást és nem mondogatják azt többé Izráelben, sõt inkább mondd nékik: elközelgettek a napok, és minden látás teljesül.
\par 24 Mert nem lesz többé semmi hiábavaló látás és hizelgõ jövendölgetés Izráel házának közepette.
\par 25 Mert én szólok, az Úr; s a mely szót szólok, meglészen, nem halad tovább. Mert a ti napjaitokban, pártos ház, szólok egy szót és megcselekszem, ezt mondja az Úr Isten!
\par 26 És lõn az Úr beszéde hozzám, mondván:
\par 27 Embernek fia! ímé, Izráel háza ezt mondja: A látás, melyet ez lát, sok napra való, és messze idõkre prófétál õ.
\par 28 Ezokért mondjad nékik: Így szól az Úr Isten: Nem halad tovább semmi én beszédem; a mit szólok, az a szó meglészen, ezt mondja az Úr Isten.

\chapter{13}

\par 1 És lõn az Úr beszéde hozzám, mondván:
\par 2 Embernek fia! prófétálj Izráel prófétái ellen, a kik prófétálnak, és mondjad azoknak, a kik önnön szívökbõl próféták: Halljátok meg az Úr beszédét!
\par 3 Így szól az Úr Isten: Jaj a bolond prófétáknak, a kik az önnön lelkök után mennek, mert semmit  sem láttak.
\par 4 Mint rókák a romok közt, olyanokká lettek prófétáid, oh Izráel!
\par 5 Nem hágtatok fel a törésekhez, nem falaztatok falat Izráel háza körül, hogy megállhassatok a harczban az Úrnak napján.
\par 6 Hívságot láttak s hazug jövendölgetést, kik ezt mondják: Monda az Úr! holott az Úr nem bocsátotta õket, és még várják, hogy betelik beszédök.
\par 7 Avagy nem hiábavaló látást láttatok-é, és nem hazug jövendölgetést szóltatok-é? midõn ezt mondjátok vala: Monda az Úr! holott én nem szólottam!
\par 8 Ennekokáért így szól az Úr Isten: Mivelhogy hívságot szólottatok és hazugságot láttatok, azért ímé én ellenetek leszek, ezt mondja az Úr Isten.
\par 9 És lészen kezem a próféták ellen, kik hívságot látnak és hazugságot jövendölgetnek; az én népem gyülekezetében nem lesznek, Izráel házának könyvébe nem irattatnak, és Izráel földjére be nem mennek, és megtudjátok, hogy én vagyok az Úr Isten.
\par 10 Azért, mert eláltatták az én népemet, mondván: béke! holott nincsen béke; és ha a nép falat épít, ímé õk bemázolják azt mázzal.
\par 11 Mondjad a mázzal mázolóknak, hogy leomlik: ömlõ záporesõ lészen, és ti jégesõ kövei hulljatok, szélvihar hasítsd!
\par 12 És ímé leomlik a fal. Avagy nem mondják-é majd néktek: hol a mázolás, a melylyel mázolátok?
\par 13 Ezokáért így szól az Úr Isten: És meghasogatom szélviharral búsulásomban, és ömlõ zápor lészen haragomban, és lésznek jégesõ kövei búsulásomban annak elrontására.
\par 14 És ledöntöm a falat, melyet mázzal mázoltatok és levetem a földre, és meztelen lesz fundamentoma; és leomlik, és ti veszszetek el közepette, és tudjátok meg, hogy én vagyok az Úr.
\par 15 És teljessé teszem búsulásomat a falon és azokon, a kik mázolák azt mázzal, és mondom néktek: nincs a fal és nincsenek, a kik azt mázolák,
\par 16 Tudniillik Izráel prófétái, kik prófétálnak Jeruzsálemnek és látnak néki békességnek látását, holott nincsen békesség, ezt mondja az Úr Isten.
\par 17 És te, embernek fia, fordítsd orczádat néped leányaira, a kik önnön szívükbõl prófétálnak, és prófétálj ellenök.
\par 18 És mondjad: Így szól az Úr Isten: Jaj azoknak, a kik kötéseket varrogatnak minden kézcsuklóra, és takarókat készítenek akármilyen termetûek fejére, hogy lelkeket vadászszanak! Némely lelkeket elvadásztok népemtõl, másokat elevenen megtartotok a magatok hasznára.
\par 19 És megszentségtelenítetek engem népem elõtt néhány marok árpáért és falat kenyérért, hogy lelkeket öljetek, melyeknek nem kellene meghalniok, és hogy lelkeket elevenen megtartsatok, melyeknek nem kellene élniök, hazudozván népemnek, a mely hazugságra hallgat.
\par 20 Ezokáért így szól az Úr Isten: Ímé én kötéseitek ellen leszek, a melyekkel ti a lelkeket vadászszátok, mint madarakat; és leszaggatom azokat karjaitokról, s a lelkeket, melyeket ti vadásztok, szabadon bocsátom, mint madarakat.
\par 21 És elszaggatom takaróitokat, és magszabadítom népemet kezetekbõl, és nem lesznek többé zsákmány a ti kezetekben, s megtudjátok, hogy én vagyok az Úr.
\par 22 Mivelhogy megszomorítjátok az igaznak szívét hazugsággal, holott én õt bántani nem akartam, s megerõsítitek a hitetlen kezeit, hogy meg ne térjen gonosz útjáról, hogy õt életben megtartsam.
\par 23 Ezokáért hívságot nem láttok és jövendõt nem jövendölgettek többé, s megszabadítom népemet a ti kezetekbõl, és megtudjátok, hogy én vagyok az Úr.

\chapter{14}

\par 1 És jövének hozzám férfiak Izráel vénei közül, és leülének én elõttem.
\par 2 És lõn az Úrnak beszéde hozzám, mondván:
\par 3 Embernek fia! Ezek a férfiak fölvették bálványaikat szívökbe, és vétkeik botránkozását orczáik elé állították. Vajjon engedjem-é, hogy megkérdezzenek engem?
\par 4 Ezokáért szólj velök, és mondjad nékik: Ezt mondja az Úr Isten: Valaki az Izráel házából bálványait szívébe fölveszi, és vétkének botránkozását teszi orczái elé, és megy a prófétához: én, az Úr felelek meg annak énmagam által bálványainak sokasága miatt;
\par 5 Hogy megragadjam Izráel házát az õ szívökben, a kik elfordultak tõlem bálványaik miatt mindnyájan.
\par 6 Ezokáért mondjad Izráel házának: Ezt mondja az Úr Isten: Térjetek meg, és forduljatok el bálványaitoktól és minden útálatosságtoktól fordítsátok el orczátokat.
\par 7 Mert valaki az Izráel házából és a jövevények közül, a kik Izráelben laknak, elhajlik tõlem, és az õ bálványait veszi föl szívébe, és vétkének botránkozását teszi orczája elé, és megyen a prófétához, hogy ez tanácsot kérjen õ néki én tõlem: én, az Úr felelek meg annak énmagam által;
\par 8 És ellene fordítom orczámat annak a férfiúnak, és vetem õt jegyül és közbeszédül,  és kiirtom népem közül, és megtudjátok, hogy én vagyok az Úr!
\par 9 Ha pedig a próféta megtéveszteni engedi magát, hogy kijelentést adjon: én, az Úr tévesztettem meg azt a prófétát; és kinyújtom kezemet ellene, és kiveszem õt  az én népem, Izráel közül.
\par 10 És viselik vétköket; a milyen a kérdezõ vétke, olyan legyen a próféta vétke is.
\par 11 Azért, hogy el ne tévelyedjék többé Izráel háza én tõlem, és többé meg ne fertéztessék magokat minden õ elszakadásukkal, hanem legyenek az én népem és én legyek Istenök, ezt mondja az Úr Isten.
\par 12 És lõn az Úr beszéde hozzám, mondván:
\par 13 Embernek fia! ha valamely ország vétkeznék ellenem, elpártolván tõlem, és én kinyújtván kezemet ellene, eltörném néki a kenyérnek botját, és bocsátanék reá éhséget, és kiirtanék belõle embert és barmot;
\par 14 És ha volna ez a három férfiú benne: Noé, Dániel és Jób: akkor õk az igazságukkal a magok lelkét megszabadítanák, azt mondja az Úr Isten.
\par 15 Ha gonosz vadállatokat bocsátanék át az országon, hogy azt gyermektelenné tegyék, és az pusztává lenne, a melyen senki át nem menne a vadállatok miatt:
\par 16 Benne ama három férfiú (élek én, az Úr Isten mondja) sem fiakat, sem leányokat meg nem szabadítana, csak magokat szabadítanák meg, az ország pedig pusztává lenne.
\par 17 Avagy ha fegyvert hoznék amaz országra, és mondanám: fegyver, menj át ez országon! és kiirtanék belõle embert és barmot,
\par 18 És ama három férfiú benne volna, élek én, az Úr Isten mondja, nem szabadítana meg sem fiakat, sem leányokat, hanem csak magokat szabadítanák meg.
\par 19 Avagy ha döghalált bocsátanék arra az országra, és kiönteném búsulásomat reá vérben, hogy kiirtsak belõle embert és barmot,
\par 20 S Noé, Dániel és Jób benne volna: élek én, az Úr Isten mondja, nem szabadítanának meg sem fiat, sem leányt; õk igazságukkal csak a magok lelkét szabadítanák meg.
\par 21 Mert így szól az Úr Isten: Mennyivel inkább, ha e négy nehéz ítéletemet: a fegyvert, éhséget, vadállatot és döghalált bocsátom Jeruzsálemre, hogy kiirtsak belõle embert és barmot!
\par 22 Ímé, megmaradnak benne némely menekültek, a kiket kivezetnek, fiak és leányok; ímé õk kimennek hozzátok, hogy lássátok útjokat és cselekedeteiket, és vígasztalást vegyetek a veszedelembõl, melyet Jeruzsálemre hoztam, mindarra nézve, a mit hoztam reá.
\par 23 És megvígasztalnak titeket, ha látjátok útjokat és cselekedeteiket megismeritek, hogy nem hiába cselekedtem mindazt, a mit cselekedtem vele, ezt mondja az Úr Isten.

\chapter{15}

\par 1 És lõn az Úr beszéde hozzám, mondván:
\par 2 Embernek fia! mire való a szõlõtõke fája egyéb fa között, a venyige, mely az erdõ fái között van?
\par 3 Avagy vesznek-é abból fát, hogy valami eszközt csináljanak belõle? avagy vesznek-é belõle szeget, hogy mindenféle edényt akaszszanak reá?
\par 4 Ímé a tûznek adatott, hogy megemészsze; két végét megemésztette már a tûz, és közepe megpörkölõdött, vajjon való-é valami eszközre?
\par 5 Ímé, míg ép vala, semmi eszközre nem vala jó; menynyivel kevésbbé csinálhatnak belõle valamit most, mikor a tûz megemésztette és megpörkölõdött!
\par 6 Azért így szól az Úr Isten: A mint a szõlõtõke fáját az erdõ fái közül a tûznek adtam megemésztésre, úgy adtam oda Jeruzsálem lakóit,
\par 7 És ellenök fordítom arczomat. A tûzbõl jöttek ki és a tûz emészsze meg õket, és megtudjátok, hogy én vagyok az Úr, mikor arczomat ellenök fordítom.
\par 8 És teszem a földet pusztasággá, mivelhogy elpártoltak tõlem, ezt mondja az Úr Isten.

\chapter{16}

\par 1 És lõn az Úr beszéde hozzám, mondván:
\par 2 Embernek fia! add tudtára Jeruzsálemnek az õ útálatosságait,
\par 3 És mondjad: Így szól az Úr Isten Jeruzsálemnek: A te származásod és születésed Kanaán földjérõl való; atyád az Emoreus és anyád Hitteus asszony.
\par 4 Születésed pedig ilyen volt: a mely napon születtél, el nem metszették a köldöködet, és vízzel meg nem mostak, hogy tiszta lennél, sóval sem töröltek meg, sem be nem póláltak.
\par 5 Szem meg nem szánt téged, hogy ezekbõl valamit veled cselekedett volna, könyörülvén rajtad; hanem kivetettek a nyilt mezõre, mert útáltak, a mely napon születtél.
\par 6 Ekkor elmenék melletted és látálak véredben eltapodva, és mondék néked: A te véredben élj! mondék ismét néked: A te véredben élj!
\par 7 Sok ezerekre szaporítottalak, mint a mezei füvet, és megszaporodál és fölnevekedél és jutál nagy szépségre; emlõid duzzadának s szõröd kinõtt vala, de te mezítelen és befedezetlen valál.
\par 8 Ekkor elmenék melletted, és látálak, és ímé a te korod a szerelem kora vala, és kiterjesztém fölötted szárnyamat s befödözém mezítelenségedet, és megesküvém néked  s frigyre léptem veled; azt mondja az Úr Isten, és lõl az enyém.
\par 9 És megmosálak vízzel és elmosám rólad véredet, és megkenélek olajjal.
\par 10 És felöltöztetélek hímes ruhába, és felsaruztalak borjúfóka bõrrel, s övezélek fehér gyolcscsal s befedélek selyemmel.
\par 11 És felékesítélek ékeségekkel, s adtam karpereczeket kezeidre és lánczot nyakadra.
\par 12 És adtam orrpereczet orrodra és függõket füleidre és ékes koronát fejedre.
\par 13 És felékesítéd magadat aranynyal és ezüsttel, és öltözeted vala fehér gyolcs és selyem és hímes ruha; lánglisztet, mézet és olajat ettél, és megszépülél felette igen, s királyságra jutál.
\par 14 És kiméne híred a pogányok közé a te szépségedért; mert tökéletes vala az ékességeim által, a melyeket reád tettem, azt mondja az Úr Isten.
\par 15 De elbízád magadat szépségedben és paráznává lõn híred szerint, elárasztál paráznaságaiddal minden melletted elmenõt:  legyen kedve szerint!
\par 16 És vevél a te ruháidból s csinálál magadnak magaslatokat különbözõ szinnel borítva, s paráználkodál azokon; ilyen még nem volt és nem is lesz.
\par 17 És vevéd a te ékességidet az én aranyomból és ezüstömbõl, melyeket néked adtam, és csináltál magadnak férfiú képeket, és azokkal paráználkodál.
\par 18 És vevéd hímes ruháidat és befedezéd azokat, és olajomat és füstölõszeremet veted eléjök.
\par 19 És az én eledelemet, melyet néked adtam, - lángliszttel és olajjal és mézzel etettelek vala, - õ eléjök rakád kedves illatul; így lõn, ezt mondja az Úr Isten.
\par 20 És vevéd a te fiaidat és leányaidat, kiket nékem szültél vala, és megáldozád õket azoknak eledelül. Avagy nem volt-é már elég paráznaságodból,
\par 21 Hogy megölted fiaimat is, és oda adád õket, mindõn tûzben nékik áldozád?
\par 22 És minden útálatosságaidban és paráznaságaidban meg nem emlékeztél a te ifjúságod napjairól mikor mezítelen és befedezetlen valál, véredben eltapodva voltál.
\par 23 És lõn minden gonoszságod után, (Jaj, jaj néked! azt mondja az Úr Isten,)
\par 24 Építél magadnak tetõt, és csináltál magaslatot minden utczán.
\par 25 És minden keresztútnál megépítéd magaslatodat, s útálatossá tevéd szépségedet, és kétfelé vetéd lábaidat minden melletted elmenõnek, és sokasítád paráznaságodat.
\par 26 És paraználkodál Égyiptom fiaival, szomszédiddal, a nagytestûekkel, és sokasítád paráznaságodat, hogy engem ingerelj.
\par 27 És ímé kinyújtottam kezemet ellened, s megkisebbítém rendelt részedet, és adálak téged a te gyûlölõidnek, a Filiszteusok leányainak csúfolásukra, a kik átallák fajtalan útadat.
\par 28 S Assiria fiaival is paráználkodál, mert meg nem elégedél; paráználkodál velök, és még sem elégedél meg.
\par 29 És sokasítád paráználkodásodat a kalmárok földe, Káldea felé, de még ezzel sem elégedél meg.
\par 30 Mily gyenge a szíved, azt mondja az Úr Isten, hogy mindezeket cselekedted, egy rakonczátlan rima cselekedeteit!
\par 31 Hogy állítál magadnak fedelet minden keresztúton, és magaslatokat csinálál minden utczán, de nem voltál olyan, mint a rima, kicsibe véve a bért.
\par 32 Te házasságtörõ asszony! férje helyett idegeneket fogad el!
\par 33 Minden rimának bért adnak, te pedig magad adtad ajándékidat minden szeretõdnek, így megvásárlád õket, hogy bemenjenek hozzád mindenfelõl paráznaságidért.
\par 34 És lõn különbség közted és más asszonyok közt paráznaságaidban; mert utánad nem jártak a paráznák; te adtál bért nékik és bért õk nem adának néked, így lõn különbséged.
\par 35 Azért te rima, halld meg az Úr beszédét!
\par 36 Így szól az Úr Isten: A miatt, hogy eláradt gyalázatod és föl van takarva mezítelenséged a te szeretõiddel való paráznaságaidban; és minden útálatos bálványaid miatt és fiaid vére miatt, kiket azoknak adtál:
\par 37 Ezokáért ímé egybegyûjtöm minden szeretõdet, kiknek kedves valál, és mindazokat, a kiket szerettél, együtt azokkal, a kiket gyûlöltél, és egybegyûjtöm õket ellened mindenfelõl, és föltakarom mezítelenségedet elõttök hogy lássák minden te mezítelenségedet.
\par 38 És megítéllek téged a házasságtörõ és vért ontó asszonyok ítéletével, és véredet kiontatom búsulásomban és féltõ szerelmemben.
\par 39 És adlak téged kezökbe, és leszakítják tetõdet és lerontják magaslataidat és lehúzzák rólad ruháidat és elveszik ékességeidet, és hagynak mezítelenül  s ruhátalanul.
\par 40 És összehoznak gyûlést ellened, és megköveznek és összevagdalnak fegyvereikkel.
\par 41 És megégetik házaidat tûzzel, és ítéletet cselekesznek rajtad sok asszony szeme láttára, és megszûntetem paráznaságodat, és bért sem adsz többé.
\par 42 És megnyugotom búsulásomat rajtad, hogy eltávozzék féltõ szerelmem te tõled, s megnyugoszom és többé nem haragszom.
\par 43 Mivelhogy meg nem emlékeztél ifjúságod napjairól, és ingerlettél engem mindezekkel, azért ímé én is fejedhez verem útadat, ezt mondja az Úr Isten, és többé nem cselekszed a fajtalanságot minden útálatosságod mellett.
\par 44 Ímé, valaki közmondással él, rólad veszi azt, mondván: A minémû az anya, olyan a leánya is.
\par 45 Anyád leánya vagy te, a ki megútálta férjét s fiait, és öcséidnek nénje vagy, a kik megútálták férjeiket és fiaikat; anyátok Hitteus asszony és atyátok Emoreus.
\par 46 És a te nénéd Samaria vala, õ és leányai, ki balkezed felõl lakik vala; és öcséd,  a ki jobbkezed felõl lakik vala, Sodoma és leányai.
\par 47 És nem az õ útaikon jártál, és nem az õ útálatosságaik szerint cselekedtél, de csak kevés ideig; de aztán gonoszabb valál azoknál minden útadban.
\par 48 Élek én! azt mondja az Úr Isten, így nem cselekedett Sodoma, a te öcséd, õ és leányai, a mint cselekedtél te és a te leányaid.
\par 49 Ímé, ez volt a vétke Sodomának, a te öcsédnek: kevélység, eledel bõsége és gondtalan békesség volt nála és leányainál, de a szûkölködõnek és szegénynek kezét nem fogta meg.
\par 50 És felfuvalkodának s cselekedének útálatosságot elõttem, és elveszítém õket, mikor ezt megláttam.
\par 51 És Samaria félannyit sem vétkezett, mint te, mert többek a te útálatosságaid, mint az övék; és így nõtestvéreidet nálad igazabbaknak bizonyítád minden útálatosságiddal, a melyeket cselekvél.
\par 52 Te is azért viseld gyalázatodat, a melyre pedig nénédet ítélted; a te bûneid miatt, melyekben nálok útálatosabban cselekvél, igazabbak õk nálad. Szégyenülj meg hát te is s viseld gyalázatodat, hogy nõtestvéreidet nálad igazabbaknak bizonyítád.
\par 53 És visszahozom foglyaikat, Sodomának s leányainak foglyait és Samariának s leányainak foglyait; s visszahozom a te foglyaidat is amazok közepette;
\par 54 Azért, hogy viseljed gyalázatodat és megszégyenülj mindazokért, miket cselekedtél, mikor azoknak vígasztalásukra leszel.
\par 55 És nõtestvéreid, Sodoma és leányai visszatérnek elõbbi állapotjokba, és Samaria s leányai visszatérnek elõbbi állapotjokba, és te is és leányaid visszatértek elõbbi állapototokba.
\par 56 És nem vala-é öcséd, Sodoma, szóbeszéd a te szádban kevélykedésed napján,
\par 57 Minekelõtte kitudódott volna gonoszságod; a miképen te most gyalázatuk vagy Siria leányainak s minden körülötted valóknak, a Filiszteusok leányainak, kik útálnak téged köröskörül?
\par 58 Fajtalanságodat és útálatosságaidat magad viseled, azt mondja az Úr.
\par 59 Mert ezt mondja az Úr Isten: És ha úgy cselekedtem veled, mint te cselekedtél, midõn megvetetted az esküt, hogy megtörd a frigyet:
\par 60 Én megemlékezem frigyemrõl, a melyet veled ifjúságod napjaiban kötöttem, és örök frigyet vetek veled.
\par 61 És te megemlékezel útaidról és megszégyenled magadat, mikor hozzád veszed nõtestvéreidet, a kik nagyobbak nálad, együtt azokkal, a kik kisebbek, s adom õket néked leányaidul, de nem a te frigyedbõl.
\par 62 És én megerõsítem frigyemet veled, s megismered, hogy én vagyok az Úr.
\par 63 Hogy megemlékezzél és pirulj, és meg ne nyissad többé szádat szégyenletedben, mikor megkegyelmezek néked mindenekben, valamit cselekedtél, azt mondja az Úr Isten.

\chapter{17}

\par 1 És lõn az Úr beszéde hozzám, mondván:
\par 2 Embernek fia, adj találós mesét, és mondj példabeszédet Izráel házának.
\par 3 És mondjad: Így szól az Úr Isten: A nagyszárnyú nagy saskeselyû, melynek hosszú csapótollai valának, s mely rakva vala különféle színû tollakkal, jöve a Libanonra, és elfoglalá a czédrusfa tetejét.
\par 4 Gyönge ágainak hegyét letépte, és vivé azt a kalmárok földére, az árusok városában tevé le.
\par 5 És võn annak a földnek magvából, és elveté azt termékeny mezõbe; sok vizek mellé vivé; mint fûzfát ülteté el.
\par 6 És felsarjadt, és lõn elterülõ, alacsony szõlõtõvé, hogy vesszõit amahhoz fordítsa s gyökerei amaz alatt legyenek, és szõlõtõvé lõn, és vesszõket terme, és ágacskákat bocsáta ki.
\par 7 És vala más nagy szárnyú, soktollú nagy saskeselyû, és ímé, ez a szõlõtõ feléje terjeszté gyökereit s vesszeit hozzá nyújtá ültetésének ágyaiból, hogy öntözze õt;
\par 8 Pedig jó földbe, sok víz mellé ültették vala el, hogy ágakat hajtson és gyümölcsöt teremjen, hogy legyen jeles szõlõtõ.
\par 9 Mondjad: Így szól az Úr Isten: Vajjon jó szerencsés lesz-é? Vajjon gyökereit nem szaggatja-é ki, és gyümölcsét nem vágja-é le, hogy elszáradjon, hajtásának minden ága elszáradjon, és pedig nem erõs karral és sok néppel támad reá, hogy kitépje azt gyökerestõl.
\par 10 És ímé elplántáltatott, jó szerencsés lesz-é? Avagy ha a napkeleti szél illeti õt, nem szárad-é el teljesen, ültetésének ágyában nem szárad-é el?
\par 11 És lõn az Úr szava hozzám, mondván:
\par 12 No, mondjad a pártos háznak: Avagy nem értettétek-é mi ez? Mondjad: Ímé, eljött a babiloni király Jeruzsálembe, és fogá királyát és fejedelmeit és elvivé õket magához Babilonba.
\par 13 És võn a királyi magból, s frigyet szerze vele, s megesketé õt, de földnek erõseit elvivé,
\par 14 Hogy alacsony királyság legyen, hogy fel ne emelkedjék, hanem frigyét megõrizze, hogy ez megálljon.
\par 15 De pártot üte ellene, bocsátván követeit Égyiptomba, hogy adjon néki lovakat és sok népet. Vajjon jó szerencsés lesz-é? Vajjon megszabadítja-é magát, a ki ezeket cselekszi? a ki megszegte a szövetséget, megszabadul-é?
\par 16 Élek én, ezt mondja az Úr Isten, hogy annak a királynak lakóhelyén, a ki õt királylyá tette,  a kinek tett esküjét megvetette, s a kivel tett frigyét megszegte, ott nála, Babilonban hal meg.
\par 17 És a Faraó nagy haddal és nagy sokasággal vele semmit nem tesz a háborúban, mikor sánczot töltenek és tornyot építenek sok lélek kiirtására.
\par 18 S ha megvetette az esküt, hogy megszegje a frigyet, pedig ímé, kezet adott rá, s mégis megcselekedte mindezeket; nem fog megszabadulni!
\par 19 Azért így szól az Úr Isten: Élek én, hogy eskümet, melyet megvetett, és frigyemet, melyet megszegett, fejéhez verem.
\par 20 És kiterjesztem hálómat ellene, és megfogatik varsámban, és elviszem õt Babilonba s ott törvénykezem vele gonoszságáért, melylyel ellenem járt.
\par 21 És minden menekültje minden seregébõl fegyver miatt hull el, és a megmaradottak szélnek szélednek mindenfelé, és megtudjátok, hogy én, az Úr beszéltem.
\par 22 Így szól az Úr Isten: És veszek én ama magas czédrus tetejébõl, és elültetem; felsõ ágaiból egy gyönge ágat szegek le, s elplántálom én magas és fölemelt hegyen.
\par 23 Izráel magasságos hegyén plántálom õt, és ágat nevel és gyümölcsöt terem s nagyságos czédrussá nevekedik, hogy lakjanak alatta, mindenféle szárnyas madarak ágainak árnyékában fognak lakozni.
\par 24 És megismeri a mezõ minden fája, hogy én, az Úr tettem a magas fát alacsonynyá, az alacsony fát magassá; megszáraztottam a zöldelõ fát, és zölddé tettem az asszú fát. Én, az Úr szólottam és megcselekedtem.

\chapter{18}

\par 1 És lõn az Úr szava hozzám, mondván:
\par 2 Mi dolog, hogy ezt a közbeszédet szoktátok mondani Izráel földjén, mondván: Az atyák ettek egrest, és a fiak foga vásott meg bele?
\par 3 Élek én, ezt mondja az Úr Isten, nem lesz többé helye köztetek ennek a közbeszédnek Izráelben.
\par 4 Ímé, minden lélek enyém, úgy az atyának lelke, mint a fiúnak lelke enyém; a mely lélek vétkezik, annak kell meghalni!
\par 5 És ha valaki igaz lesz, és törvény szerint igazságot cselekszik;
\par 6 Ha a hegyeken nem eszik, és szemei föl nem emeli Izráel házának bálványaira, és felebarátja feleségét meg nem fertézteti,  és aszzonyhoz tisztátalanságában nem közeledik;
\par 7 És senkit nem nyomorgat, az adósnak a zálogot visszaadja,  ragadományt nem ragadoz, az éhezõnek kenyerét adja és a  mezítelent ruhával befödi;
\par 8 Uzsorára nem ád, kamatot nem vesz, megvonja kezét az álnokságtól, igaz ítéletet tesz a felek közt;
\par 9 Az én parancsolatimban jár és törvényeimet megõrzi, hogy igazságot cselekedjék; ez az igaz, õ élvén él, ezt mondja az Úr Isten.
\par 10 És ha erõszakos fiat nemz, a ki vért ont, és csak egyet is cselekszik amazokból;
\par 11 Mindezeket pedig nem cselekedte; hanem a hegyeken evett, és felebarátjának feleségét megfertéztette;
\par 12 A szûkölködõt és szegényt nyomorgatta, ragadományokat ragadozott, zálogot vissza nem adott, és a bálványokra emelte szemeit, útálatosságot cselekedett;
\par 13 Uzsorára adott és kamatot vett: és az ilyen éljen? Nem él! Mindezeket az útálatosságokat cselekedte, halállal haljon meg, az õ vére legyen õ rajta!
\par 14 Ímé, ha fiat nemz, és ez látja atyjának minden vétkét, melyeket cselekszik; látja, de nem cselekszik azok szerint:
\par 15 A hegyeken nem eszik, és szemeit nem emeli fel Izráel házának bálványaira, felebarátjának feleségét meg nem fertézteti,
\par 16 És senkit sem nyomorgat, zálogot nem vesz, ragadományt nem ragadoz, kenyerét az éhezõnek adja és a mezítelent ruhával befödi;
\par 17 A szegényre nem veti rá kezét, uzsorát és kamatot nem vesz, törvényeim szerint cselekszik, parancsolataimban jár: az ilyen ne haljon meg atyja vétkéért, hanem élvén éljen.
\par 18 Atyja, mert nyomorgatást követett el, ragadományt ragadozott atyjafiától, és a mi nem jó, azt cselekedte népe között: ímé meghal a maga vétkéért.
\par 19 És ti ezt mondjátok: Miért ne viselje a fiú az apa vétkét? Ám a fiú, törvény szerint és igazságot cselekedett, minden parancsolatimat megtartotta s cselekedte azokat: élvén éljen.
\par 20 A mely lélek vétkezik, annak kell meghalni; a fiú ne viselje az apa vétkét, se az apa ne viselje a fiú vétkét; az igazon legyen az õ igazsága; és a gonoszon az õ gonoszsága.
\par 21 És ha a gonosztevõ megtér minden vétkébõl, melyeket cselekedett, és megtartja minden parancsolatimat és törvény szerint és igaszságot cselekszik: élvén éljen, és meg ne haljon.
\par 22 Semmi gonoszságáról. melyet cselekedett, emlékezés nem lészen; az õ igazságáért, melyet cselekedett, élni fog.
\par 23 Hát kivánva kivánom én a gonosznak halálát? ezt mondja az Úr Isten! nem inkább azt, hogy megtérjen útjáról és éljen?
\par 24 És ha az igaz elhajol az õ igazságától, és gonoszságot cselekszik, minden útálatosság szerint, melyeket a hitetlen cselekedett, cselekeszik, nemde éljen-é? Semmi igazságairól, a melyeket cselekedett, emlékezés nem lészen: gonoszságáért, melyet cselekedett, és az õ vétkéért, melylyel vétkezett, ezekért meg kell halnia.
\par 25 És azt mondjátok: Nem igazságos az Úrnak útja! Oh, halljátok meg, Izráel háza: az én útam nem igazságos-é? nem inkább a ti útaitok nem igazságosak-é?
\par 26 Ha elhajol az igaz az õ igazságától, és gonoszságot cselekszik, és a miatt meghal: gonoszsága miatt hal meg, melyet cselekedett.
\par 27 És ha a gonosztevõ megtér az õ gonoszságától, melyet cselekedett, és törvény szerint és igazságot cselekszik: ez az õ lelkét megtartja életben.
\par 28 Mert belátta és megtért minden gonoszságától, melyeket cselekedett: élvén éljen, ne haljon meg.
\par 29 És azt mondja az Izráel háza: Nem igazságos az Úrnak útja! Az én útaim nem igazságosak-é, Izráel háza? nem inkább a ti útaitok nem igazságosak-é?
\par 30 Ennekokáért mindeniteket az õ útai szerint ítélem, Izráel háza, ezt mondja az Úr Isten. Térjetek meg és forduljatok el minden vétkeitektõl, hogy romlástokra ne legyen gonoszságotok.
\par 31 Vessétek el magatoktól minden vétkeiteket, melyekkel vétkeztetek, és szerezzetek magatoknak új szívet és új lelket; Miért halnátok meg, oh Izráel háza?
\par 32 Mert nem gyönyörködöm a meghaló halálában, ezt mondja az Úr Isten. Térjetek meg azért és éljetek!

\chapter{19}

\par 1 Te pedig kezdj gyászéneket Izráel fejedelmeirõl.
\par 2 És mondjad: Mi volt anyád? Nõstény oroszlán, oroszlánok közt fekszik vala, fiatal oroszlánok között nevelé fel kölykeit.
\par 3 És fölnevele egyet kölykei közül; fiatal oroszlánná lõn, és megtanula ragadományt ragadozni, embereket evett.
\par 4 És meghallák ezt róla a népek; vermökben megfogaték, és elvivék õt horgokon Égyiptom földjére.
\par 5 És mikor látta, hogy késik, hogy oda veszett reménysége, võn egyet kölykei közül, ezt tevé fiatal oroszlánná.
\par 6 És ez jár vala az oroszlánok között, fiatal oroszlánná lõn; és megtanula ragadományt ragadozni, embereket evett.
\par 7 És ismeré az õ özvegyeiket s városaikat elpusztítá, és megrettene a föld és annak teljessége, ordítása hangjától.
\par 8 És veték ellene a pogányok köröskörül a tartományokból és kiteríték reá hálójukat, vermökben megfogaték;
\par 9 És veték õt ketreczbe horgokon, s elvivék õt Babilon királyához, elvivék õt várakba, hogy többé ne hallassék szava Izráel hegyein.
\par 10 Anyád a te nyugalmadban olyan vala, mint a víz mellé ültetett szõlõtõ, gyümölcscsel és vesszõvel bõvelkedik vala a sok víztõl.
\par 11 És lõnek neki erõs vesszei, uralkodók pálczáinak valók, és nagy vala magassága a sûrûség között és felett, és kitetszék magasságával és vesszeinek sokaságával.
\par 12 De kiszaggattaték Isten haragja által, a földre vetteték, és a napkeleti szél elszárasztá gyümölcsét; letöretének és elszáradának erõs ágai; tûz emészté meg.
\par 13 És most a pusztában plántáltatott, száraz és szomjú földön.
\par 14 És tûz jött ki ágainak egyik vesszejébõl, gyümölcsét megemészté, és nincs többé rajta erõs vesszõ, nincs pálcza az uralkodásra! Gyászének ez és gyászének lesz.

\chapter{20}

\par 1 És lõn a hetedik esztendõben, az ötödik hónap tizedikén: jövének férfiak Izráel vénei közül megkérdezni az Urat, és leülének elõttem.
\par 2 És lõn az Úr beszéde hozzám, mondván:
\par 3 Embernek fia! beszélj Izráel véneivel, és mondjad nékik: Így szól az Úr Isten: Nemde engem megkérdezni jöttetek-é? Élek én, hogy nem engedem, hogy megkérdezzetek engem, ezt mondja az Úr Isten.
\par 4 Ítélni akarod õket, ítélni akarsz, embernek fia? add tudtokra atyáik útálatosságait!
\par 5 És mondjad nékik: Így szól az Úr Isten: Azon a napon, melyen elválasztám Izráelt, és fölemelém kezemet a Jákób háza magvának, és megismertetém magamat velök Égyiptom földjén, és kezemet fölemelém nékik, mondván: Én vagyok az Úr, a ti Istentek!
\par 6 Azon a napon felemeltem kezemet nékik, hogy kihozzam õket Égyiptom földjérõl a földre, a melyet kinéztem vala nékik, a mely téjjel és mézzel folyó, ékessége az minden tartománynak;
\par 7 És mondék nékik: Kiki az õ szemei útálatosságait elvesse, és Égyiptom bálványaival meg ne fertéztessétek magatokat; én vagyok az Úr, a ti Istentek!
\par 8 De pártot ütének ellenem, s nem akarának hallgatni reám. Senki az õ szeme útálatosságait el nem veté és Égyiptom bálványait el nem hagyá. Mondám azért, hogy kiöntöm búsulásomat rájok, teljessé teszem haragomat rajtok Égyiptom földjének közepette.
\par 9 De cselekedtem az én nevemért, hogy ez meg ne gyaláztassék a pogányok szemei elõtt, a kik közt õk valának, a kiknek szemei elõtt megismertettem magamat velök, hogy kihozom õket Égyiptom földjérõl.
\par 10 És kihozám õket Égyiptom földjérõl, s vivém õket a pusztába.
\par 11 És adám nékik parancsolatimat, és törvényeimet kijelentém nékik, melyeket az ember ha cselekszik, él azok által.
\par 12 És adám nékik szombataimat is, hogy legyenek jegyül köztem és õ közöttök; hogy megtudják, hogy én vagyok az Úr, az õ megszentelõjök.
\par 13 De pártot üte ellenem Izráel háza a pusztában, az én parancsolatimban nem jártak és törvényeimet megveték, a melyeket az ember ha cselekszik, él azok által; és az én szombataimat megfertéztették felette igen. Mondám azért, hogy kiöntöm búsulásomat rájok a pusztában, hogy elveszessem õket.
\par 14 De cselekedtem az én nevemért, hogy meg ne gyaláztassék a pogányok szemei elõtt, a kiknek szeme láttára kihoztam vala õket.
\par 15 És föl is emelém én kezemet nékik a pusztában, hogy be nem viszem õket a földre, melyet adtam nékik, mely téjjel és mézzel folyó, ékessége az minden tartománynak;
\par 16 Mivelhogy törvényeimet megvetették és parancsolataimban nem jártak, és szombataimat megfertéztették, mert bálványaik után járt vala szívök:
\par 17 Mindazáltal kedvezett szemem nékik, hogy el ne veszessem õket, és nem vetettem nékik véget a pusztában.
\par 18 És mondék fiaiknak a pusztában: A ti atyáitok parancsolataiban ne járjatok, és az õ törvényeit meg ne tartsátok, s bálványaikkal magatokat meg ne fertéztessétek.
\par 19 Én vagyok a ti Uratok, Istentek: az én parancsolatimban járjatok, az én törvényimet tartsátok meg, és azokat cselekedjétek.
\par 20 És az én szombatimat megszenteljétek, hogy legyenek jegyül én köztem és tiköztetek, hogy megtudjátok, hogy én vagyok az Úr, a ti Istenetek.
\par 21 De pártot ütének a fiak ellenem, parancsolataimban nem jártak, s törvényeimet meg nem tartották, hogy azokat cselekedjék, a melyeket az ember ha cselekszik, él azok által; szombataimat megfertéztették; mondám azért, hogy kiöntöm búsulásomat rájok, teljessé teszem haragomat rajtok a pusztában.
\par 22 De visszavontam kezemet, s cselekedtem az én nevemért, hogy meg ne gyaláztassék a pogányok elõtt, a kiknek szeme láttára kihoztam õket.
\par 23 Föl is emelém én kezemet nékik a pusztában, hogy elszélesztem õket a pogányok közé, és szétszórom õket a tartományokba;
\par 24 Mivelhogy törvényeimet nem cselekedték, s parancsolataimat megvetették, és szombataimat megfertéztették, s atyáik bálványai után voltak szemeik.
\par 25 És én is adtam nékik nem jó parancsolatokat, s törvényeket, a melyek által ne éljenek.
\par 26 S megfertéztetém õket ajándékaikkal, mikor tûzön vittek át minden elsõszülöttet,  hogy elpusztítsam õket, hogy megtudják, hogy én vagyok az Úr.
\par 27 Azért szólj az Izráel házának, embernek fia, és mondjad nékik: Így szól az Úr Isten: Még ebben is gyalázattal illettek engem a ti atyáitok, hogy hûtlenül elszakadának tõlem:
\par 28 Mikor bevittem õket a földre, a melyért fölemeltem kezemet, hogy azt nékik adom, megtekintének minden magas halmot és minden sûrû ágú fát, és ott áldozzák vala az õ áldozatjaikat, és ott adják vala haragra ingerlõ  ajándékaikat; és oda teszik vala kedvelt illatjokat, és oda öntik vala italáldozataikat.
\par 29 És mondék nékik: Micsoda e magaslat, a hova ti gyülekeztek? És nevezik nevét magaslatnak mind e mai napig.
\par 30 Ennekokáért mondjad Izráel házának: Így szól az Úr Isten: Nemde a ti atyáitok módjára fertéztetitek-é meg magatokat, és az õ útálatosságaik szerint paráználkodtok-é?
\par 31 És ajándékaitok vitelével, mikor átviszitek fiaitokat a tûzön, fertéztetitek meg magatokat minden bálványaitok elõtt mind e mai napig, és én engedjem, hogy megkérdezzetek engem, Izráel háza? Élek én, ezt mondja az Úr Isten, hogy nem engedem, hogy megkérdezzetek engem!
\par 32 És a mi lelketekben támadt, semmiképpen sem lesz meg, hogy azt mondjátok: Leszünk olyanok, mint a pogányok, mint a tartományok nemzetségei, szolgálván fának és kõnek.
\par 33 Élek én, ezt mondja az Úr Isten, hogy erõs kézzel és kinyújtott karral és kiontott búsulással uralkodom rajtatok.
\par 34 És kiviszlek titeket a népek közül, és egybegyûjtelek titeket a tartományokból, melyekbe elszéledtetek, erõs kézzel és kinyújtott karral és kiontott búsulással;
\par 35 És vezetlek titeket a népek pusztájára, hol szemtõl-szembe törvénykezem veletek.
\par 36 A mint törvénykeztem a ti atyáitokkal Égyiptom földének pusztájában; úgy törvénykezem veletek, ezt mondja az Úr Isten:
\par 37 És átviszlek titeket a vesszõ alatt, és hozlak titeket a frigynek kötelébe.
\par 38 És kitisztítom közületek a pártosokat, és az ellenem támadókat, és a földrõl, melyen jövevények voltak, kihozom õket, de Izráel földjére nem fognak bemenni, hogy megtudjátok, hogy én vagyok az Úr!
\par 39 Ti pedig, Izráel háza, azt mondja az Úr Isten, mindenitek járjon az õ bálványai után és szolgáljon azoknak; de azután bizony, hallgatni fogtok rám, és az én szent nevemet többé meg nem fertéztetitek ajándékitokkal és bálványaitokkal.
\par 40 Mert az én szent hegyemen, Izráelnek magas hegyén, ezt mondja az Úr Isten, ott fog szolgálni nékem Izráel egész háza együtt azon a földön; ott kedvelem õket, ott kivánom meg a ti áldozataitokat és ajándékitoknak elsõ zsengéjét mindenben, mit nékem szenteltek.
\par 41 Kedves illatban kedvellek titeket, mikor kihozlak titeket a népek közül és egybegyûjtelek titeket a tartományokból, melyeken elszéledtetek, és megszenteltetem ti bennetek a pogányok szemei elõtt.
\par 42 És megtudjátok, hogy én vagyok az Úr, mikor beviszlek titeket Izráel földjére, arra a földre, melyért fölemeltem kezemet, hogy adom azt a ti atyáitoknak.
\par 43 És ott megemlékeztek útaitokról, s minden cselekedeteitekrõl, melyekkel magatokat megfertéztettétek, s megútáljátok ti magatokat minden gonoszságtokért, melyeket cselekedtetek.
\par 44 És megtudjátok, hogy én vagyok az Úr, mikor cselekszem veletek az én nevemért, és nem a ti gonosz útaitok és romlott cselekedeteitek szerint, oh Izráel háza, ezt mondja az Úr Isten!
\par 45 És lõn az Úrnak beszéde hozzám, mondván:
\par 46 Embernek fia! fordítsd orczádat délre, és szólj dél felé és prófétálj a déli mezõ erdeje ellen!
\par 47 És mondjad a dél erdejének: Halld meg az Úr beszédét, így szól az Úr Isten: Ímé, én tüzet gyújtok benned, hogy megemészszen te benned minden zöldelõ és aszú fát. Meg nem aluszik a lángoló láng, és megég a miatt minden orcza déltõl északig.
\par 48 És meglátja minden test, hogy én, az Úr gyújtottam meg azt, mert meg nem aluszik.
\par 49 És mondék: Ah, ah Uram Isten! ezek azt mondják nékem: Hát nem példabeszédekben beszél ez?

\chapter{21}

\par 1 És lõn az Úrnak beszéde hozzám, mondván:
\par 2 Embernek fia! fordítsd arczodat Jeruzsálem felé, és szólj a szent helyek ellen és prófétálj Izráel földje ellen!
\par 3 És mondjad Izráel földjének: Így szól az Úr: Ímé, én reád megyek, és kivonszom kardomat hüvelyébõl, és kivágok belõled igazat és gonoszt.
\par 4 Azért, hogy kivágjak belõled igazat és gonoszt, azért megyen ki kardom hüvelyébõl minden test ellen délrõl északig.
\par 5 És megérti minden test, hogy én, az Úr vontam ki kardomat hüvelyébõl, mert belé többé vissza nem tér.
\par 6 És te, embernek fia, nyögj! Derekad fájdalmában keserûséggel nyögj szemök láttára.
\par 7 És lészen, mikor mondják néked: Miért nyögsz te? ezt mondjad: A hírért, mert beteljesedett: és elolvad minden szív és elerõtlenedik minden kéz és elcsügged minden lélek és minden térd elolvad mint a víz. Ímé beteljesedett és meglett, ezt mondja az Úr Isten.
\par 8 És lõn az Úr beszéde hozzám, mondván:
\par 9 Embernek fia! prófétálj és mondjad: Ezt mondta az Úr: Mondjad: fegyver, fegyver! megélesített és meg is fényesített!
\par 10 Hogy öldököljön, megélesíttetett, hogy legyen villámlása, megfényesíttetett. Avagy örüljünk-é? Fiam e veszszeje megvet minden fát!
\par 11 És adta azt megfényesítésre, hogy marokba fogják; megélesíttetett az a kard s megfényesíttetett, hogy adják a megölõnek kezébe.
\par 12 Kiálts és jajgass, embernek fia! mert ez az én népemen lészen és Izráel minden fejedelmin: e kardra jutnak népemmel együtt, ezokért üss czombodra.
\par 13 Már megpróbáltatott. De hát ha a vesszõ maga is vonakodnék? Nem úgy lészen, ezt mondja az Úr Isten!
\par 14 Te pedig, embernek fia, prófétálj, és csapd össze tenyeredet, mert kettõs lesz a kard, most harmadszor; öldöklõ kard az, a nagy öldöklõ kard körüljárja õket.
\par 15 Hogy elolvadjon a szív és sokan elhulljanak: minden kapujokban rájok vetem a kard villámlását, hisz villámlásra készült, öldöklésre kifényesíttetett!
\par 16 Szedd össze magadat; tarts jobbra, tarts elõre, tarts balra, a merre éled irányozva van.
\par 17 Én is összecsapom tenyeremet, és megnyugotom haragomat.  Én, az Úr szólottam.
\par 18 És lõn az Úr beszéde hozzám, mondván:
\par 19 És te, embernek fia, csinálj magadnak két útat, a melyen jõjjön a babiloni király kardja; egy földrõl jõjjön ki mind a kettõ; s egy kezet véss föl, a városba vezetõ út fejénél vésd föl.
\par 20 Útat csinálj, hogy jõjjön a kard az Ammon fiainak Rabbájára, aztán Júdára, a körülkerített Jeruzsálemre.
\par 21 Mert megáll a babiloni király az útak kezdetén, a két út fejénél, hogy jövendõt láttasson; megrázza a nyilakat, megkérdezi a Teráfimot, megnézi a májat.
\par 22 Jobbjába adta a jövendölés Jeruzsálemet, hogy állasson faltörõ kosokat, hogy nyissa száját ordításra, hogy üssön zajt trombitával, hogy állasson faltörõ kosokat a kapuk ellen, hogy töltsön sánczot, építtessen tornyot.
\par 23 De ez nékik hamis jövendölésnek látszik; szent esküik vannak nékik Istentõl; ám Õ emlékezetbe hozza vétköket, hogy megfogassanak.
\par 24 Ezokáért ezt mondja az Úr Isten: Mivelhogy emlékezetbe hozzátok vétketeket, midõn nyilvánvalókká lesznek gonoszságaitok, hogy megláttassanak bûneitek minden cselekedeteitekben; mivelhogy eszembe juttok, kézzel megfogattok.
\par 25 És te elvetemedett, te gonosztevõ, Izráel fejedelme, a kinek napja eljött az utolsó vétek idején:
\par 26 Így szól az Úr Isten! El a süveggel, le a koronával! Ez nem lészen ez: az alacsony legyen magas, és a magas alacsony!
\par 27 Rommá, rommá, rommá teszem azt; ez sem lesz állandó, míg el nem jõ az, a kié az uralkodás, és néki adom azt!
\par 28 És te, embernek fia, prófétálj, és mondjad: Ezt mondja az Úr Isten Ammon fiairól és gyalázkodásukról, és mondjad: Fegyver, fegyver, öldöklésre kivont, megfényesíttetett, hogy ragyogjon, azért, hogy villámljék;
\par 29 De mivel felõled hiábavalóságot látnak, hazugságot jövendölnek, hogy odatesznek téged a megölt gonoszok nyakára, a kiknek napjok eljött az utolsó vétek idején:
\par 30 Tedd vissza hüvelyébe azt; a helyen, melyen teremtettél, származásod földjén ítéllek meg téged.
\par 31 És kiontom reád haragomat, búsulásom tüzét fúvom reád, és adlak goromba férfiak kezébe, kik mesterek, téged elveszteni.
\par 32 A tûz eledele leszel; véredet benyeli a föld; emlékezetbe nem jösz többé, mert én, az Úr szóltam.

\chapter{22}

\par 1 És lõn az Úr beszéde hozzám, mondván:
\par 2 És te, embernek fia, ítélni akarsz? meg akarod-é ítélni a vérontó várost? add tudtára minden útálatosságait.
\par 3 És mondjad: Így szólt az Úr Isten: Te város, ki közepében vért ontott, hogy eljõjjön ideje, és bálványokat csinált magának önmaga megfertéztetésére;
\par 4 Véred miatt, melyet ontottál, lettél bûnös, és bálványaiddal, melyeket csináltál, fertéztetted meg magadat, s közelebb hoztad napjaidat s eljutottál esztendeidig; azért adlak gyalázatul a pogányoknak, és csúfolásul minden tartománynak.
\par 5 A kik közel s távol vannak tõled, megcsúfolnak téged, te fertézett nevû, sok háborúságú!
\par 6 Ímé, Izráel fejedelmei, kiki az õ tehetsége szerint azon volt benned, hogy vért ontsanak.
\par 7 Apát és anyát megútáltak te benned, a jövevényen nyomorgatást cselekedtek te közepetted, árvát és özvegyet sanyargattak benned.
\par 8 A mi nékem szenteltetett, megútáltad, s szombatimat megfertéztetted.
\par 9 Rágalmazók voltak benned, hogy vért ontsanak, s a hegyeken ettek benned, fajtalanságot cselekedtek közepetted.
\par 10 Az atya szemérmét föltakarták benned, a havivér miatt tisztátalant erõszakolták benned.
\par 11 Egyik felebarátjának feleségével cselekedett útálatosságot,  a másik meg menyét fertéztette meg fajtalanságban, s volt, a ki húgát, atyjának leányát erõszakolta benned.
\par 12 Ajándékokat vettek fel benned a vérontásra, uzsorát és kamatot szedtél, s nyerekedtél felebarátaidon csalárdsággal, s én rólam elfelejtkeztél, ezt mondja az Úr Isten.
\par 13 És ímé, összecsapom tenyeremet nyereségeden, a melyet csináltál, és a vérontásokon, melyek lõnek te benned.
\par 14 Vajjon megállhat-é szíved, avagy erõsek lesznek-é kezeid azokban a napokban, mikor én számolok veled? Én, az Úr, szólottam és meg is cselekszem.
\par 15 És eloszlatlak téged a pogányok közé, és szétszórlak a tartományokba, s véget vetek tisztátalanságodnak.
\par 16 S örökségül bírlak téged a pogányok szeme láttára, és megtudod, hogy én vagyok az Úr.
\par 17 És lõn az Úr beszéde hozzám, mondván:
\par 18 Embernek fia! Izráel háza salakká lett nékem; egészen; réz és ón és vas és ólom a kemencze közepette; ezüstsalakká lettek:
\par 19 Ennekokáért így szól az Úr Isten. Mivelhogy mindnyájan salakká lettetek, azért ímé, egybegyûjtelek titeket Jeruzsálem közepébe.
\par 20 A mint egybe szoktak gyûjteni ezüstöt és rezet és vasat és ólmot és ónt a kemencze közepébe, hogy tüzet gerjeszszenek rá a megolvasztásra; így gyûjtelek egybe búsulásomban és haragomban, és bevetlek s megolvasztlak titeket.
\par 21 És egybegyûjtelek titeket, és rátok fúvom búsulásom tüzét, hogy benne megolvadjatok.
\par 22 A mint megolvadt az ezüst a kemencze közepében, úgy olvadtok meg õ benne, és megtudjátok, hogy én, az Úr öntöttem ki haragomat reátok.
\par 23 És lõn az Úr beszéde hozzám, mondván:
\par 24 Embernek fia! mondjad néki: Te vagy a föld, mely meg nem tisztult; esõt nem kapott a haragnak napján.
\par 25 Pártosok az õ prófétái õ közepette; olyanok, mint az ordító oroszlán, mely ragadományt ragad: lelkeket ettek, kincset és drágaságot elvesznek, özvegyeit megsokasítják õbenne.
\par 26 Papjai erõszakot tettek törvényemen, s megfertéztették, a mi nékem szenteltetett! különbséget nem tettek a között, a mi szent és a mi köz,  s a tisztátalan és tiszta között különbséget nem tanítottak, s szombataimtól elrejtették szemeiket, úgyhogy megszentségtelenítettek engem.
\par 27 Elõljárói õ közepette mint a ragadományt ragadozó farkasok: vért ontani, a lelkeket elveszteni, hogy nyerekedhessenek nyereséggel.
\par 28 És prófétái mázolnak nékik mázzal: hiábavalóságot látnak s jövendölnek hazugságot nékik, mondván: Így szól az Úr Isten! holott az Úr nem beszélt.
\par 29 A föld népe nyomorgatást cselekszik és ragadományt ragadoz, a szûkölködõt és szegényt sanyargatja, s a jövevényt törvénytelen nyomorgatja.
\par 30 És keresék közülök valakit, a ki falat falazna, és állana a törésen én elõmbe az országért, hogy el ne pusztítsam azt; de senkit nem találék.
\par 31 Ennekokáért kiontám haragomat reájok, megemésztém õket búsulásom tüzével, útjokat fejökhöz verém, azt mondja az Úr Isten.

\chapter{23}

\par 1 És lõn az Úr beszéde hozzám, mondván:
\par 2 Embernek fia! Volt két asszony, egy anyának leányai.
\par 3 És paráználkodának Égyiptomban, ifjúságukban paráználkodtak; ott szorongatták emlõjüket, ott nyomogatták szûzi keblöket.
\par 4 És nevök: Oholá a nagyobbik, és húga Oholibá; és lõnek enyimekké, és szülének fiakat és leányokat. A mi pedig a nevöket illeti: Samaria az Oholá és Jeruzsálem az Oholibá.
\par 5 És paráználkodék Oholá oldalamon, és fölgerjede szeretõihez, a közeli Assiriabeliekhez.
\par 6 Kik kék bíborba öltözöttek, helytartók és fejedelmek, kívánatos ifjak mindnyájan, lovagok, lovakon ülõk.
\par 7 És nékik adá magát paráznaságaiban Assiria válogatott ifjainak; és mindazoknál, kikhez felgerjede, minden õ bálványaikkal megfertézteté magát.
\par 8 De az Égyiptombeliektõl való paráznaságait is el nem hagyá, mert vele háltak ifjúságában, s õk nyomogatták szûzi kebelét, és kiöntötték õ reá paráznaságukat.
\par 9 Ennekokáért adtam õt szeretõinek kezébe, Assiria fiainak kezébe, kikhez fölgerjedett.
\par 10 Azok feltakarák szemérmét, fiait és leányit elvivék s magát fegyverrel ölék meg, úgy hogy híre-neve lõn az asszonyoknál, s ítéletet cselekedének rajta.
\par 11 És látá húga, Oholibá, és még gonoszabbul folytatá bujálkodását amannál, és paráznaságait nénje paráználkodásainál.
\par 12 Assiria fiaihoz fölgerjedt, közeli helytartókhoz s fejedelmekhez, teljes szépségben öltözõkhöz, lovagokhoz, lovakon ülõkhöz, kik mindnyájan kivánatos ifjak.
\par 13 És látám, hogy megfertéztette magát: egy az útjok kettõjöknek.
\par 14 És még szaporítá paráznaságait, és láta férfiakat bevésve a falon, a Káldeusok képeit, bevésve vörös festékkel,
\par 15 Kik övet viseltek derekukon, csomós süvegeket fejükön, olyanok mind, mint a szekérrõl harczolók, hasonlók Bábel fiaihoz, kiknek szülõföldje Káldea;
\par 16 És fölgerjedt hozzájok szemei nézésében, s bocsáta követeket hozzájok Káldeába.
\par 17 És eljövének õ hozzá Bábel fiai a szerelem ágyasházába, s megfertézteték õt paráznaságukkal, úgyhogy tisztátalan lett miattok; s ekkor lelke eltávozék tõlök.
\par 18 És mikor feltakarta paráznaságait és feltakarta szemérmét, eltávozék az én lelkem õ tõle, a mint az õ nénjétõl lelkem eltávozott vala.
\par 19 És megsokasítá paráznaságait, megemlékezvén ifjúságának napjairól, mikor Égyiptom földjén paráználkodott;
\par 20 És fölgerjede azok bujálkodóihoz, kiknek teste olyan, mint a szamarak teste, és folyásuk, mint lovak folyása.
\par 21 És megemlékezél ifjúságod fajtalankodására, mikor õk, az égyiptomiak, nyomogatták kebledet, hogy szorongassák ifjúságod emlõit.
\par 22 Ennekokáért Oholibá, így szól az Úr Isten: Ímé, én feltámasztom a te szeretõidet ellened, kiktõl pedig eltávozott lelked, s reád hozom õket mindenfelõl.
\par 23 Babilon fiait és minden Káldeabelit, Pekódot és Soát és Koát, Assiria minden fiát õ velük, kívánatos ifjakat, helytartókat s fejedelmeket, mindnyájokat, szekérrõl harczolókat s elõkelõket, és lovakon ülõket, mindnyájokat.
\par 24 És jõnek reád szekereknek és kerekeknek tömegével s népek sokaságával, nagy és kis paizszsal és sisakkal körülvesznek téged mindenfelõl, s adok nékik hatalmat az ítéletre, s megítélnek téged az õ ítéletök szerint.
\par 25 És megmutatom rajtad féltõ szerelmemet, s cselekszenek veled kegyetlenül; orrodat s füleidet elmetélik, s maradékod fegyver miatt hull el; õk fiaidat és leányaidat elviszik, s maradékodat tûz emészti meg.
\par 26 S megfosztanak ruháidtól, és elveszik ékességeidet.
\par 27 És véget vetek fajtalanságodnak s Égyiptom földjérõl való paráznaságodnak, s nem emeled föl szemeidet rájok, s Égyiptomra nem emlékezel többé.
\par 28 Mert így szól az Úr Isten: Ímé, én adlak téged azoknak kezébe, a kiket gyûlölsz, azoknak kezébe, a kiktõl eltávozott lelked.
\par 29 És gyûlölséggel cselekesznek veled, és mindent, mit kerestél, elvesznek tõled, és mezítelen s ruhátalan hagynak, hogy feltakartassék paráznaságaid szemérme. És fajtalánságod és paráználkodásaid
\par 30 Hozták ezeket reád; mivelhogy paráználkodtál a pogányok után, mert megfertéztetted magad azok bálávnyaival.
\par 31 Nénéd útján jártál, azért az õ poharát adom kezedbe.
\par 32 Így szól az Úr Isten: Nénéd poharát megiszod, mely mély és széles; leszen nevetségedre s csúfoltatásodra, hogy sok fér bele.
\par 33 Részegséggel és bánattal megtelsz; pusztaság és elpusztulás pohara a te nénéd, Samaria pohara!
\par 34 Meg kell innod azt s fenékig hajtanod; és cserepein rágódni fogsz, emlõidet megszaggatod azokon, mert én szólottam, ezt mondja az Úr Isten.
\par 35 Ennekokáért ezt mondja az Úr Isten: Mivelhogy elfelejtkeztél én rólam s hátad mögé vetettél engemet, te is hordozd fajtalanságodat és paráznaságaidat.
\par 36 És monda az Úr nékem: Embernek fia! avagy nem ítéled-é Oholát és Oholibát? Hirdesd nékik útálatosságaikat.
\par 37 Mert házasságot törtek, és vér van kezeiken, és bálványaikkal törtek házasságot, és fiaikat is, kiket szültek vala nékem, tûzben nékik áldozák azok eledeléül.
\par 38 Sõt ezt is cselekedték velem: megfertézteték az én szent helyemet azon a napon, és szombatimat megszentségteleníték.
\par 39 És mikor megölték fiaikat az õ bálványaiknak, bemenének az én szenthelyembe azon a napon, hogy megszentségtelenítsék, és ímé, így cselekedtek az én házamban.
\par 40 Sõt elküldöttek messzünnen jövõ emberekhez, kikhez követség küldetett, és ímé eljövének, a kiknek kedvéért megmosódál, kendõzéd szemeidet, és fölékesítéd magad ékességgel;
\par 41 És ültél pompás kerevetre, s terített asztal vala az elõtt, és az én füstölõ szeremet és olajomat arra tevéd;
\par 42 És lõn ott örvendezõ sokaságnak zaja. És küldöttek az emberek sokaságából való férfiakhoz, hozatának ivótársakat a pusztából; és ezek adának karpereczeket az õ kezeikre és ékes koronát fejökre.
\par 43 És mondék: Még az elaggott is házasságot tör? most már paráznaságod fog paráználkodni, és úgy lõn.
\par 44 És bemenének hozzá, mint a hogy a parázna asszonyhoz bemennek; így mentek be Oholához és Oholibához, e fajtalan asszonyokhoz.
\par 45 És az igaz férfiak, ezek ítélik meg õket a házasságtörõk, és vérontók ítéletével, mert házasságtörõk és  vér van kezeiken.
\par 46 Mert így szól az Úr Isten: Hozzanak rájok gyülekezetet, és adják õket bántalmazásra és ragadományra.
\par 47 És kövezze meg õket a gyülekezet, és vagdalják össze õket fegyvereikkel; fiaikat és leányaikat öljék meg, és házaikat tûzzel égessék meg.
\par 48 És megszüntetem a fajtalanságot a földrõl, és tanul minden asszony, és nem cselekesznek a ti fajtalanságotok szerint.
\par 49 És reátok vetik fajtalanságotokat, s bálványaitok vétkeit viselitek, és megtudjátok, hogy én vagyok az Úr Isten.

\chapter{24}

\par 1 És lõn az Úr beszéde hozzám a kilenczedik esztendõben, a tizedik hónapban, a hónapnak tizedikén, mondván:
\par 2 Embernek fia! írd fel magadnak e nap nevét, épen ezen napét: Babilon királya épen ezen a napon jött Jeruzsálemre.
\par 3 És mondj példabeszédet a pártos házra, és mondjad nékik: Ezt mondja az Úr Isten: Tedd föl a fazekat, tedd föl, és tölts vizet is bele.
\par 4 Gyûjtsd össze a bele való darabokat, minden jó darabot, czombot, lapoczkát; válogatott csontokkal töltsd meg.
\par 5 Végy válogatott juhokat, és tégy máglyát a csontoknak is a fazék alá; forrald erõsen, még csontjai is fõjjenek benne.
\par 6 Ezokáért így szól az Úr Isten: Jaj a vérontó városnak, a fazéknak, a melynek rozsdája benne van, és rozsdája nem ment le róla! Darabról darabra szedd ki, a mi benne van; nem esett sors reá.
\par 7 Mert vére ott van közepében, kopasz sziklára ontotta, nem a földre öntötte, hogy por fedje be.
\par 8 Hogy haragomat felindítsam és bosszút álljak, kopasz sziklára ontottam vérét, hogy be ne fedeztessék.
\par 9 Azért így szól az Úr Isten: Jaj a vérontó városnak! én is nagy máglyát rakok!
\par 10 Bõven rakd a fát, gyújtsd meg a tüzet, fõzd meg jól a húst, forrald e levet, és csontok szétfõjjenek.
\par 11 És állítsd üresen az õ szenére, hogy meghevüljön s megtüzesedjék ércze, és megolvadjon benne tisztátalansága, megemésztessék rozsdája.
\par 12 A fáradozásokat kifárasztotta, és nem ment le róla az õ sok rozsdája, tûzbe hát rozsdájával!
\par 13 A te tisztátalanságodban fajtalanság van, mivelhogy tisztogattalak, de meg nem tisztultál, azért tisztátalanságodból többé meg nem tisztulsz, míg meg nem nyugotom haragomat rajtad.
\par 14 Én, az Úr szólottam; jõni fog és megcselekszem, el nem engedem s nem kedvezek és nem könyörülök: a te útaid és cselekedeteid szerint ítélnek meg téged, ezt mondja az Úr Isten.
\par 15 És lõn az Úr beszéde hozzám, mondván:
\par 16 Embernek fia! ímé, én elveszem tõled szemeidnek gyönyörûségét hirtelen halállal, és ne sírj és ne jajgass, se könyed ne hulljon.
\par 17 Fohászkodjál csöndesen, halottakért való sírást ne tégy, fejékességedet kösd fel, és saruidat vedd lábaidra, s ne fedezd be bajuszodat, és az emberek kenyerét ne egyed.
\par 18 És szólék reggel a néphez, és estére meghala feleségem, és úgy cselekedém reggel, a mint meg vala hagyva nékem.
\par 19 És mondá nékem a nép: Avagy nem jelented-é meg nékünk, mit jelentenek ezek nékünk, hogy te így cselekszel?
\par 20 És mondék nékik: Az Úr beszéde volt én hozzám, mondván:
\par 21 Mondd meg Izráel házának: Ezt mondja az Úr Isten: Ímé, én megfertéztetem szenthelyemet, a ti erõsségteknek kevélységét, szemeitek gyönyörûségét, lelketek kívánságát, és fiaiatok és leányaitok, kiket hátrahagytatok, fegyver miatt hullnak el.
\par 22 És úgy cselekesztek, a mint én cselekedtem: bajuszotokat be nem fedezitek, s az emberek kenyerét nem eszitek,
\par 23 S fejékességtek fejeteken, és saruitok lábaitokon lesznek; nem sírtok s nem jajgattok, hanem megrothadtok vétkeitekben, s nyögve fohászkodtok egymáshoz.
\par 24 És lészen néktek Ezékiel csodajelül: a mint õ cselekedett, egészen úgy cselekesztek ti is, mikor ez eljõ; és megtudjátok, hogy én vagyok az Úr Isten.
\par 25 És te, embernek fia, bizonyára azon a napon, mikor elveszem tõlök erõsségöket, dicsõségök örömét, szemeik gyönyörûségét és lelkök kívánságát, fiaikat és leányaikat:
\par 26 Azon a napon, a ki megmenekült, eljõ hozzád, hogy hírt mondjon néked.
\par 27 Azon a napon megnyílik szád ott a megmenekült elõtt, és szólasz és tovább nem maradsz néma; s leszel nékik csodajelül, és megtudják, hogy én vagyok az Úr.

\chapter{25}

\par 1 És lõn az Úr beszéde hozzám, mondván:
\par 2 Embernek fia! vesd tekintetedet Ammon fiaira, és prófétálj ellenök.
\par 3 És mondjad Ammon fiainak: Halljátok az Úr Isten beszédét! így szól az Úr Isten: Mivelhogy ezt mondod: Haha! az én  szenthelyemre, hogy megfertéztetett; és Izráel földjére, hogy elpusztíttatott; és Júda házára, hogy fogságba mentek:
\par 4 Ezokáért íme adlak téged Kelet fiainak örökségül, hogy felüssék sátraikat benned, s felállassák benned lakóhelyeiket, õk eszik meg gyümölcsödet, s õk iszszák meg tejedet.
\par 5 És teszem Rabbát tevék legelõjévé, és Ammon fiait nyájak fekvõhelyévé, s megtudjátok, hogy én vagyok az Úr!
\par 6 Mert így szól az Úr Isten: Mivelhogy tapsolsz kezeddel és tombolsz lábaddal és örülsz teljes megvetéssel lelkedben Izráel földje felett:
\par 7 Ennekokáért ímé kinyújtom kezemet reád, s adlak ragadományul a népeknek, és kiváglak a népségek közül, és elvesztlek a tartományok közül, eltöröllek, hogy megtudjad, hogy én vagyok az Úr.
\par 8 Azt mondja az Úr Isten: Mivelhogy Moáb és Seir ezt mondta: Ímé, Júda háza is olyan,  mint a többi népek:
\par 9 Ezokáért ímé megnyitom Moáb oldalát a városok felõl, az õ városai felõl mindenfelõl, az ország ékességét Beth-Hajesimóthot, Baál-Meont és Kirjátaim felé,
\par 10 Megnyitom Kelet fiainak, Ammon fiaival együtt adom örökségül, hogy emlékezetben se legyenek többé Ammon fiai a népek között;
\par 11 Moáb fölött is ítéletet tartok: hogy megtudják, hogy én vagyok az Úr.
\par 12 Így szól az Úr Isten: Mivelhogy Edom kegyetlen bosszúállást cselekedett Júda házán, és vétkezve vétkezett, hogy bosszút állott rajta:
\par 13 Azért, így szól az Úr Isten, kinyújtom az én kezemet Edomra, és kivágok belõle embert és barmot, és teszem õt pusztasággá Temántól fogva, és Dedánig fegyver miatt hulljanak el.
\par 14 És bosszúmat állom Edomon népemnek, Izráelnek keze által és cselekszenek Edommal az én búsulásom és haragom szerint, s megismerik bosszúállásomat, ezt mondja az Úr Isten.
\par 15 Így szól az Úr Isten: Mivelhogy a Filiszteusok bosszúból cselekedtek s kegyetlen bosszút álltak lelkökben megvetéssel, hogy elveszessék Izráelt örök gyûlölséggel:
\par 16 Ezokáért ezt mondja az Úr Isten: Ímé, én kinyújtom kezemet a Filiszteusokra, és kivágom a Keréteusokat, és elvesztem a tenger partján lakók maradékát;
\par 17 És cselekszem rajtok nagy busszúállásokat fenyítõ haragomban, hogy megtudják, hogy én vagyok az Úr, ha bosszúmat állom rajtok.

\chapter{26}

\par 1 És lõn a tizenegyedik esztendõben, a hónap elsején, lõn az Úrnak beszéde hozzám, mondván:
\par 2 Embernek fia! mivelhogy ezt mondá Tírus Jeruzsálemre: Haha! eltört a népek kapuja,  felém fordulva nyitva van; én megtelek, ha õ elpusztul;
\par 3 Ezokáért így szól az Úr Isten: Ímé én, Tírus, te reád megyek, és hozok fel ellened sok nemzetet, miként a tenger fölhozza hullámit.
\par 4 És elhányják Tírus kõfalait és lerontják tornyait, s levonszom még porát is róla, s kopasz sziklává teszem õt.
\par 5 Hálók kivetõ helye lesz a tenger közepén, mert én szólottam, ezt mondja az Úr Isten, s legyen a nemzetek ragadománya.
\par 6 És leányai, kik a mezõségen vannak, fegyverrel ölettessenek meg, hogy megtudják, hogy én vagyok az Úr.
\par 7 Mert így szól az Úr Isten: Ímé, én hozom Tírus ellen Nabukodonozort, Babilon királyát északról, a királyok királyát; lovakkal, szekerekkel, lovagokkal, sereggel és sok néppel.
\par 8 Leányaidat ott a mezõségen fegyverrel öli meg, és állat ellened tornyot, és tölt ellened sánczot, és emel ellened paizs-fedelet.
\par 9 És faltörõ kosával ütteti kõfalaidat, s tornyaidat lerontja fegyvereivel.
\par 10 Lovainak sokasága miatt belep téged poruk; a lovagoknak, kerekeknek és szekereknek robogása miatt megrendülnek kõfalaid, mikor bemegy kapuidon, a mint be szoktak menni a megtörött városba.
\par 11 Lovainak körmeivel tapodja meg minden utczádat, népedet fegyverrel öli meg, s erõsséged oszlopai a földre dõlnek.
\par 12 És prédára hányják gazdagságodat, és elragadozzák árúidat, és letörik kõfalaidat, s gyönyörûséges házaidat lerontják, és köveidet és fáidat, s még porodat is a víz közepére hányják.
\par 13 És megszüntetem éneklésed hangosságát, és cziteráid zengése nem hallatik többé.
\par 14 És kopasz sziklává teszlek; hálók kivetõ helye leszel, hogy többé meg ne építsenek; mert én, az Úr szóltam, ezt mondja az Úr Isten.
\par 15 Így szól az Úr Isten Tírusnak: Bizonyára a te romlásod zuhanásától, mikor a sebesültek nyögnek, mikor a te benned valók öltön-ölettetnek, megrendülnek a szigetek!
\par 16 És leszáll királyi székérõl a tenger minden fejedelme, és elvetik köntöseiket, s hímes ruháikat levetik: rettegésekbe öltöznek, a földön ülnek és rettegnek minden szempillantásban, s elborzadnak miattad.
\par 17 És gyászéneket kezdenek rólad, és ezt mondják néked: Mimódon veszél el te, a kit laknak a tengerekrõl, te híres-neves város, mely hatalmas vala a tengeren, õ és lakosai, a kik félelmökre valának minden mellettök lakozóknak!
\par 18 Ímé, rettegnek a szigetek zuhanásod napján, és megfélemlenek a tengerben való szigetek ilyen véged miatt.
\par 19 Mert azt mondja az Úr Isten: Mikor én téged elpusztult várossá teszlek, mint a mely városokat nem laknak; mikor a mélység árját felhozom reád, hogy beborítsanak a sok vizek:
\par 20 Akkor levonszlak téged azokkal, kik sírgödörbe szállanak, a hajdan népéhez; és lakatlak téged a mélységek országában, a hajdan pusztaságaiban, együtt azokkal, a kik sírgödörbe szálltak, hogy többé senki ne lakjék benned. És ha megmutattam dicsõségemet az élõknek földén:
\par 21 Rémségesen cselekszem veled és nem leszel; s keresni fognak, de többé örökké meg nem találnak, ezt mondja az Úr Isten.

\chapter{27}

\par 1 És lõn az Úr beszéde hozzám, mondván:
\par 2 Te pedig, embernek fia, kezdj gyászéneket Tírusról.
\par 3 És mondjad Tírusnak: Ki lakol a tenger révhelyein, ki a népek között kereskedik sok sziget felé, így szól az Úr Isten: Oh Tírus, te ezt mondtad: Én tökéletes szépségben vagyok.
\par 4 A tengerek szívében vannak határaid, építõid tökéletessé tették szépségedet.
\par 5 Senir cziprusaiból építették mindkét oldaladat, czédrust hoztak a Libánonról, hogy árbóczfát csináljanak néked.
\par 6 Básán cserfáiból csinálták evezõidet; evezõpadjaidat a cziprusiak szigeteirõl való sudar czédrusba foglalt elefántcsontból csinálták.
\par 7 Égyiptomi hímes fehér gyolcs vala vitorlád, hogy legyen zászlód; Elisának szigeteirõl való kék és piros bíbor volt sátrad borítékja.
\par 8 Sidon és Arvad lakói voltak evezõid; a te bölcseid, oh Tírus, kik benned valának, azok voltak kormányosaid.
\par 9 Gebal vénei és bölcsei voltak benned, kik hasadékaidat javítgaták; a tenger minden hajója és hajósa volt benned, hogy kicseréljék árúidat.
\par 10 Perzsák és lidiaiak és libiaiak voltak seregedben hadakozó férfiaid, paizst és sisakot függesztettek fel benned, a mik ékessé tevének.
\par 11 Arvad fiai és sereged voltak kõfalaidon köröskörül, és kemény vitézek voltak tornyaidban; paizsaikat felfüggesztették kõfalaidon köröskörül, mik tökéletessé tették szépségedet.
\par 12 Tarsis volt a te kereskedõtársad, sok különféle gazdagsága miatt: ezüstöt, vasat, ónt és ólmot adtak õk árúidért.
\par 13 Jáván, Tubál és Mések, ezek a te kalmáraid; rabszolgákat és rézedényeket adtak cserébe árúidért.
\par 14 Tógarma házából lovakat és lovagokat és öszvéreket adtak néked árúkul.
\par 15 Dedán fiai a te kalmáraid, sok sziget kereskedése van hatalmad alatt, elefántcsont-szarukat és ébenfát hoztak néked adóba.
\par 16 Arám a te kereskedõtársad mestermûveid sokasága miatt; gránátot, bíbort, hímes ruhákat, fehér gyolcsot, korálokat és rubint adtak õk árúidért.
\par 17 Júda és Izráel földe ezek a te kalmáraid; búzát Minnithbõl és édes süteményt és mézet és olajat és balzsamot adtak csereárúidért.
\par 18 Damaskus a te kereskedõtársad mestermûveid sokaságában, sok különféle gazdagságod miatt, - Helbon borával és hófehér gyapjúval.
\par 19 Vedán és Jáván Uzálból adtak árúidért: kovácsolt vasat, kásiát és jó illatú nádat, ezeket csereárúidért.
\par 20 Dedán a te kalmárod, lovagláshoz való nyeregtakarókkal.
\par 21 Arábai és Kédár fejedelmei, õk mindnyájan hatalmad alatt kereskedének; bárányokkal, kosokkal és bakokkal kereskedének veled.
\par 22 Séba és Raema kalmárai a te kalmáraid; mindenféle drága fûszerszámokat és mindenféle drágaköveket és aranyat adtak õk árúidért.
\par 23 Hárán és Kanne és Eden, Seba kalmárai, Assiria és Kilmad mind a te kereskedõtársad.
\par 24 Ezek a te kalmáraid tökéletes árúkkal, bíbor és hímes köntösökkel és drága ruháknak kötelekkel egybekötött és czédrusfából csinált szekrényeivel, a te sokadalmadban.
\par 25 Tarsis hajói hordozták árúidat, és megtelél, és felette igen híressé levél a tengerek szívében.
\par 26 Nagy vizekre vivének téged a te evezõid, a keleti szél összetört téged a tengerek  szívében.
\par 27 Gazdagságod és árúid, csereárúid, hajósaid és kormányosaid, hasadékaid javítgatói és csereárúid árúsai és minden hadakozó férfiaid, a kik benned vannak, és minden benned való sokaság, beleesnek a tenger szívébe zuhanásod napján.
\par 28 Kormányosaid kiáltásának hangjára megrendülnek a mezõségek.
\par 29 És kiszállnak hajóikból mindnyájan, kik az evezõt fogják, hajósok és a tengernek minden kormányosi a szárazföldre lépnek.
\par 30 És hallatják fölötted hangjokat, s keservesen kiáltanak, s port hintenek fejökre, hamuban fetrengenek.
\par 31 És kopaszra nyiratkoznak miattad, és zsákba övezkednek, és sírnak feletted lelki keserûségben keserves sírással.
\par 32 És kezdenek fölötted fájdalmukban gyászéneket, és így énekelnek rólad: Ki volt olyan, mint Tírus? mely most mint temetõ a tenger közepette!
\par 33 Mikor kimennek vala árúid a tengerekbõl, sok népet megelégítél; gazdagságod és csereárúid sokaságával meggazdagítád a földnek királyait.
\par 34 Most összeomlottál a tengerekrõl le a vizek mélységébe, csereárúid és egész sokaságod benned elsülyedt.
\par 35 A szigetek minden lakosai elborzadnak miattad, s királyaik iszonyodva megiszonyodnak, arczaik rángatóznak.
\par 36 A népek közt való kereskedõk fütyölnek feletted. Rémségessé lettél, s többé örökké  nem leszel!

\chapter{28}

\par 1 És lõn az Úr beszéde hozzám, mondván:
\par 2 Embernek fia! mondjad Tírus fejedelmének: Ezt mondja az Úr Isten: Mivelhogy felfuvalkodott szíved és ezt mondtad: Isten vagyok én, Isten székében ülök a tengerek szívében, holott csak ember vagy és nem Isten, mégis olylyá tevéd szíved, minõ az Isten szíve,
\par 3 Lám, hát bölcsebb volnál te Dánielnél! Semmi elrejtett dolog nem homályos néked!
\par 4 Bölcseségeddel és értelmeddel gyûjtöttél magadnak gazdagságot, gyûjtöttél aranyat s ezüstöt kincses házaidba.
\par 5 Bölcseségednek nagy voltával kereskedésed közben megsokasítád gazdagságodat, és felfuvalkodott szíved gazdagságod miatt.
\par 6 Ezokáért így szól az Úr Isten: Mivel olylyá tevéd szíved, minõ az Isten szíve:
\par 7 Azért ímé, hozok reád idegeneket, a nemzetek legkegyetlenebbjeit, és kivonszák fegyvereiket bölcseséged szépsége ellen, és megfertéztetik fényességedet.
\par 8 A sírgödörbe szállítnak alá, s meghalsz a megölettek halálával a tengerek szívében.
\par 9 Vajjon mondván mondod-é megölõd elõtt: Isten vagyok én? holott ember vagy és nem Isten a téged átütõnek kezében!
\par 10 Körülmetéletlenek halálával halsz meg, idegeneknek keze által. Mert én szóltam, ezt mondja az Úr Isten.
\par 11 És lõn az Úr beszéde hozzám, mondván:
\par 12 Embernek fia! kezdj gyászéneket Tírus királyáról, és mondd néki: Így szól az Úr Isten: Te valál az arányosság pecsétgyûrûje, teljes bölcseséggel, tökéletes szépségben.
\par 13 Édenben, Isten kertjében voltál; rakva valál mindenféle drágakövekkel: karniollal, topázzal és jáspissal, társiskõvel és onixxal, berillussal, zafirral, gránáttal és smaragddal; és karikáid mesterkézzel és mélyedéseid aranyból készültek ama napon, melyen teremtetél.
\par 14 Valál felkent oltalmazó Kérub; és úgy állattalak téged, hogy Isten szent hegyén valál, tüzes kövek közt jártál.
\par 15 Feddhetetlen valál útaidban attól a naptól fogva, melyen teremtetél, míg gonoszság nem találtaték benned.
\par 16 Kereskedésed bõsége miatt belsõd erõszakossággal telt meg és vétkezél; azért levetélek téged az Isten hegyérõl, és elvesztélek, te oltalmazó Kérub, a tüzes kövek közül.
\par 17 Szíved felfuvalkodott szépséged miatt; megrontottad bölcseségedet fényességedben;  a földre vetettelek királyok elõtt, adtalak szemök gyönyörûségére.
\par 18 Vétkeid sokaságával kereskedésed hamisságában megfertéztetted szenthelyeidet; azért tüzet hoztam ki belsõdbõl, ez emésztett meg téged; és tevélek hamuvá a földön mindenek láttára, a kik reád néznek.
\par 19 Mindnyájan, a kik ismertek a népek közt, elborzadnak miattad; rémségessé lettél, s többé örökké nem leszel!
\par 20 És lõn az Úr beszéde hozzám, mondván:
\par 21 Embernek fia! vesd tekintetedet Sidonra, és prófétálj ellene.
\par 22 És mondjad: Így szól az Úr Isten: Ímé, én ellened megyek, Sidon, és megdicsõítem magamat közepetted, hogy megtudják, hogy én vagyok az Úr, mikor ítéleteket cselekszem benne, és megszentelem magamat benne.
\par 23 És bocsáték reá döghalált és vért utczáira, és sebesültek hullanak el benne fegyver miatt, mely reá jõ mindenfelõl; hogy megtudják, hogy én vagyok az Úr.
\par 24 És ne legyen többé Izráel házának szúró tövise és fájdalomszerzõ tüskéje mindazok között, kik körülöttök vannak, kik õket megvetik, és tudják meg, hogy én vagyok az Úr.
\par 25 Így szól az Úr Isten: Mikor egybegyûjtöm Izráel házát a népek közül, kik közé szétszórattak, akkor megszentelem magamat rajtok s pogányok szeme láttára, és laknak az õ földjökön, melyet adtam Jákóbnak, az én szolgámnak.
\par 26 És laknak azon bátorsággal, és házakat építenek s szõlõket plántálnak, és laknak bátorsággal, mikor ítéleteket cselekedtem mindazokon, kik õket megvetik vala õ körülöttök, hogy megtudják, hogy én vagyok az Úr, az õ Istenök.

\chapter{29}

\par 1 A tizedik esztendõben, a tizedik hónapban, a hónap tizenkettedikén lõn az Úr beszéde hozzám, mondván:
\par 2 Embernek fia! vesd tekintetedet a Faraóra, Égyiptom királyára, és prófétálj ellene és egész Égyiptom ellen.
\par 3 Szólj és mondjad: Így szól az Úr Isten: Ímé, én ellened megyek, Faraó, Égyiptom királya, te nagy krokodil, a ki fekszik folyói közepette, a ki ezt mondja: Enyém az én folyóm, és én teremtettem magamnak.
\par 4 És horgokat vetek szádba, és azt cselekszem, hogy folyóid halai odaragadjanak pikkelyeidhez, és kivonszlak folyóid közepébõl, és folyóid minden halait, melyek odaragadnak pikkelyeidhez.
\par 5 És kivetlek a pusztába téged és folyóid minden halát; a mezõ színére esel; egybe nem szedetel s nem gyûjtetel; a földi vadaknak s az égi madaraknak adlak eledelül.
\par 6 És megtudják mindnyájan Égyiptom lakói, hogy én vagyok az Úr. Mivelhogy õk nádszál-bot valának Izráel házának;
\par 7 Melyet ha megfognak kezökkel, összetörsz, s felhasítod egész vállokat, s ha reád támaszkodnak, összeroppansz, s megrázod egész derekokat:
\par 8 Ezokáért így szól az Úr Isten: Ímé, hozok ellened fegyvert, és kivágok belõled embert és barmot.
\par 9 És legyen Égyiptom földje pusztasággá és sivataggá, és megtudják, hogy én vagyok az Úr. Mivelhogy azt mondja: A folyó enyém, és én teremtettem:
\par 10 Ezokáért ímé, én ellened megyek, és folyóid ellen, és teszem Égyiptom földjét nagy pusztaságok pusztaságává Migdoltól fogva Siénéig, és Szerecsenország határáig.
\par 11 Ne menjen át azon emberi láb, se baromi láb ne menjen át rajta, és ne lakják negyven esztendeig.
\par 12 És adom Égyiptom földjét pusztaságra az elpusztult földek között, s városai lesznek az elpusztult városok közt pusztaságban negyven esztendeig, és eloszlatom az égyiptomiakat a nemzetek közé, s szétszórom õket a tartományokba.
\par 13 Mert így szól az Úr Isten: Negyven esztendõ múlva egybegyûjtöm az égyiptomiakat a népek közül, kik közé szétszórattak.
\par 14 És visszahozom Égyiptom foglyait s visszaviszem õket Pathrós földjére, az õ eredetök földjére, és ott lesznek alacsony királyság.
\par 15 A többi királyságokhoz képest alacsony lészen s többé nem emeli magát a nemzetek fölé; és megkevesbítem õket, hogy el ne tapodják a nemzeteket.
\par 16 És nem lesz többé Izráel házának bizodalma, mely vétekre emlékeztessen engem, midõn feléje hajlanak, és megtudják, hogy én vagyok az Úr Isten.
\par 17 És lõn a huszonhetedik esztendõben, az elsõ hónapban, a hónap elsején, lõn az Úr beszéde hozzám, mondván:
\par 18 Embernek fia! Nabukodonozor Babilon királya nagy fáradsággal fárasztotta seregét Tírus ellen: minden fõ megkopaszult és minden váll feltört, de jutalma nem lõn néki és seregének Tírusból a fáradságért, a melylyel miatta fáradott.
\par 19 Ennekokáért így szól az Úr Isten: Ímé, én Nabukodonozornak a babiloni királynak adom Égyiptom földjét, és elviszi gazdagságát, s elragadja ragadományát, s elprédálja prédáját, és ez lesz jutalma seregének.
\par 20 Fizetésül, a melyért fáradott, adom néki Égyiptom földjét, mert értem cselekedtek, ezt mondja az Úr Isten.
\par 21 Azon a napon szarvat sarjasztok Izráel házának, és a te szádat megnyitom közöttök, és megtudják, hogy én vagyok az Úr.

\chapter{30}

\par 1 És lõn az Úr beszéde hozzám, mondván:
\par 2 Embernek fia! prófétálj és mondd: Így szól az Úr Isten: Jajgassatok! Jaj annak a napnak!
\par 3 Mert közel egy nap és közel az Úrnak napja, felhõnek napja, a pogányok ideje lesz az.
\par 4 És bemegy a fegyver Égyiptomba, és lesz reszketés Szerecsenországban, mikor hullanak a sebesültek Égyiptomban, és elviszik gazdagságát, s elrontatnak fundamentomai.
\par 5 A szerecsenek s a libiaiak s a lidiaiak és az egész gyülevész és Kúb és a frigy földének fiai velök együtt fegyver miatt hullnak el.
\par 6 Így szól az Úr: És elessenek, a kik Égyiptomot támogatják, és alászálljon erõsségének kevélysége: Migdoltól fogva Siénéig fegyver miatt hullanak el benne, ezt mondja az Úr Isten.
\par 7 És elpusztulnak elpusztult tartományok között, és városai elpusztult városoknak közepette lesznek.
\par 8 És megtudjátok, hogy én vagyok az Úr, mikor tüzet vetek Égyiptomra, és összetörik minden segítõje.
\par 9 Azon a napon követek mennek ki elõttem hajókon a bátorságban lakó Szerecsenország elrémítésére, s lészen rettegés bennök Égyiptom ama napján; mert ímé jön!
\par 10 Ezt mondja az Úr Isten: És megszüntetem Égyiptom lármáját Nabukodonozor, babiloni király keze által.
\par 11 Õ és az õ népe õ vele együtt, a nemzetek legkegyetlenebbjei, elhozatnak a föld elvesztésére, és kivonszák fegyveröket Égyiptom ellen, és betöltik a földet megölettekkel.
\par 12 És a folyóvizeket kiszárasztom, és eladom a földet gonoszok kezébe, és elpusztítom a földet s teljességét idegenek keze által, én, az Úr szólottam.
\par 13 Így szól az Úr Isten: És elvesztem a bálványokat és megszüntetem a bálványképeket Nófból s a fejedelmet Égyiptom földjérõl, nem lészen többé, és bocsátok félelmet Égyiptom földjére.
\par 14 És elpusztítom Pathróst, s vetek tüzet Zoánra, és teszek ítéletet Nóban.
\par 15 És kiöntöm haragomat Sinre, Égyiptom erõsségére, és kivágom a sokaságot Nóból.
\par 16 És vetek tüzet Égyiptomra, kínok közt vergõdik Sin, és Nó megtörik és Nófba betörnek fényes nappal.
\par 17 Aven és Fibéseth ifjai fegyver miatt hullanak el, és õk magok fogságra mennek.
\par 18 És Tehafneheszben megsötétedik a nap, mikor ott összetöröm Égyiptom pálczáját, és megszünik benne erejének kevélysége, õt magát felhõ takarja el, és leányai fogságra mennek.
\par 19 És cselekszem ítéleteket Égyiptomban, és megtudják, hogy én vagyok az Úr.
\par 20 És lõn a tizenegyedik esztendõben, az elsõ hónapban, a hónap hetedikén, lõn az Úr beszéde hozzám, mondván:
\par 21 Embernek fia! A Faraónak, Égyiptom királyának karját eltörtem, és ímé be nem kötötték, hogy megorvosolják, hogy tegyenek körülkötést reá, hogy megerõsödjék fegyverfogásra.
\par 22 Ennekokáért így szól az Úr Isten: Ímé, a Faraó ellen megyek, Égyiptomnak királya ellen, és eltöröm mind a két karját, azt, amely még erõs, és azt, amely már eltörött, s elejtetem vele kezébõl a fegyvert.
\par 23 És eloszlatom az Égyiptombelieket a nemzetek közé, és szétszórom õket a tartományokba.
\par 24 És megerõsítem Babilon királyának karjait, és adom az én fegyveremet kezébe, és eltöröm a Faraó karjait, és nyögni fog elõtte, mint a megsebesültek szoktak.
\par 25 Megerõsítem azért Babilon királyának karjait, a Faraó karjai pedig leessenek, hogy megtudják, hogy én vagyok az Úr, mikor adom fegyveremet Babilon királyának kezébe, hogy azt kinyújtsa Égyiptomnak földje ellen.
\par 26 És eloszlatom az égyiptomiakat a nemzetek közé, és szétszórom õket a tartományokba, hogy megtudják, hogy én vagyok az Úr.

\chapter{31}

\par 1 És lõn a tizenegyedik esztendõben, a harmadik hónapban, a hónap elsején, lõn az Úr szava hozzám, mondván:
\par 2 Embernek fia! szólj a Faraónak, Égyiptom királyának és az õ sokaságának: Kihez vagy hasonló a te nagyságodban?
\par 3 Ímé, Assúr czédrus vala a Libánonon,  szép ágakkal és sûrû galyaival árnyékot tartó s magas növésû, melynek felhõkig ért a teteje.
\par 4 Víz nevelte nagygyá, a mélység vizei tették magassá, folyóikkal körüljárták ültetése földjét, s csak folyásaikat bocsáták a mezõ egyéb fáihoz.
\par 5 Ennekokáért lõn magasb növése a mezõ minden fájánál: s megsokasulának ágai, s hosszúra nevekedének galyai a sok víztõl, midõn kiterjeszté azokat.
\par 6 Az õ ágain csinál vala fészket minden égi madár, és galyai alatt fiadzott a mezõ minden vada, s árnyékában lakik vala sokmindenféle nép.
\par 7 És széppé lõn magasságával, hosszan kiterjedt ágaival, mert gyökere sok víz felé nyúlik vala.
\par 8 A czédrusok el nem takarák õt az Isten kertjében, a cziprusok nem valának hasonlók ágaihoz, s a platánoknak nem valának olyan galyai, mint néki; Isten kertjében egy fa sem vala hasonló hozzá az õ szépségében.
\par 9 Széppé tõm õt az õ sok ágaival, úgy hogy irígykedett rá Éden minden fája az Isten kertjében.
\par 10 Ennekokáért így szól az Úr Isten: Mivelhogy magasra nevekedtél, és fölemelé tetejét a felhõk közé, és szíve felfuvalkodott az õ magasságában:
\par 11 Azért adom õt a nemzetek urának kezébe, bánjék el vele, gonoszságáért kiûztem õt.
\par 12 És kivágták õt az idegenek, a nemzetek legkegyetlenebbjei és leterítették. A hegyekre és minden völgyekbe hullottak ágai, és összetörtek galyai a föld minden mélységében, és leszállt árnyékából a föld minden népe, és ott hagyták õt.
\par 13 Ledõlt törzsökén lakik vala minden égi madár, és ágaihoz gyül vala a mezõ minden vada;
\par 14 Azért, hogy magasra ne nevekedjék egy víz mellett való fa se, és föl ne emelje tetejét a felhõk közé; hogy ne bízzék önmagában kevélyen senki a vízivók közül; mert mindnyájan halálra adattak a mélység országába az emberek fiai közt azokhoz, a kik sírgödörbe szálltak.
\par 15 Így szól az Úr Isten: Azon a napon, a melyen sírba aláméne, gyászba öltöztetém miatta a mélység vizeit, és megtartóztatám folyóikat, úgy hogy a sok víz elzáraték, s meggyászoltatám õt a Libánonnal, és a mezõ minden fája elepede miatta.
\par 16 Zuhanásának hangja miatt megreszkettetém a nemzeteket, mikor leszállítám õt a sírba együtt velök, kik sírgödörbe szállnak; és vígasztalást võn a mélység országában Éden minden fája, a Libánon  szépsége és java, minden vízivó.
\par 17 Ezek is alászállottak vele a sírba azokhoz, a kik fegyverrel ölettek meg, s a kik mint segítõtársai árnyékában ülének a nemzetek között.
\par 18 Kihez vagy hát hasonló dicsõségben és nagyságban Éden fái közt!? Hiszen le fogsz szállíttatni Éden fáival a mélység országába; körülmetéletlenek közt feküszöl együtt a fegyverrel megölettekkel. A Faraó ez és minden sokasága, ezt mondja az Úr Isten.

\chapter{32}

\par 1 És lõn a tizenkettedik esztendõben, a tizenkettedik hónapban, a hónap elsején, lõn az Úr beszéde hozzám, mondván:
\par 2 Embernek fia! kezdj gyászéneket a Faraóról, Égyiptom királyáról, és mondjad néki: Nemzetek fiatal oroszlánja, elvesztél; hiszen te olyan valál, mint a krokodil a tengerekben, mert kijövél folyóidon és megháborítád lábaiddal a vizeket és felzavartad folyóikat.
\par 3 Így szól az Úr Isten: Ezért kivetem rád hálómat sok nép serege által, és kihúznak téged varsámban.
\par 4 És leterítlek a földre, a mezõség színére vetlek, és reád szállítom az ég minden madarát, s megelégítem belõled az egész föld vadait.
\par 5 S vetem húsodat a hegyekre, és betöltöm a völgyeket rothadásoddal.
\par 6 És megitatom áradásod földjét véredbõl mind a hegyekig, s mélységek megtelnek belõled.
\par 7 És mikor eloltalak, beborítom az eget, s csillagait besötétítem, a napot felhõbe borítom, és a hold nem fényeskedik fényével.
\par 8 Minden világító testet megsötétítek miattad az égen, és bocsátok sötétséget földedre, ezt mondja az Úr Isten.
\par 9 És megszomorítom sok nép szívét, ha elhíresztelem romlásodat a nemzetek között, a földeken, a melyeket nem ismersz.
\par 10 És cselekszem, hogy elborzadjanak miattad a sok népek, s királyaik rémüljenek el rémüléssel miattad, mikor villogtatom fegyveremet elõttök, s remegni fognak minden szempillantásban, kiki az õ életéért, zuhanásod napján.
\par 11 Mert így szól az Úr Isten: A babiloni király fegyvere jön reád.
\par 12 Vitézeknek fegyvereivel hullatom el sokaságodat, õk a népek legkegyetlenebbjei mindnyájan; és elpusztítják Égyiptom kevélységét, és elvész minden sokasága.
\par 13 És elvesztem minden barmát a sok vizek mellõl, és nem zavarja fel azokat többé ember lába, sem barmok körme azokat föl nem zavarja.
\par 14 Akkor meghiggasztom vizeiket, és folyóikat, olajként folyatom, ezt mondja az Úr Isten,
\par 15 Mikor Égyiptom földjét pusztasággá teszem, és kipusztul a föld teljességébõl, mert megverek minden rajta lakót, hogy megtudják, hogy én vagyok az Úr.
\par 16 Gyászének ez és sírva énekeljék; a népek leányai énekeljék sírva; Égyiptomról és minden sokaságáról énekeljék sírva, ezt mondja az Úr Isten.
\par 17 És lõn a tizenkettedik esztendõben, a hónap tizenötödikén, lõn az Úr beszéde hozzám, mondván:
\par 18 Embernek fia! sírj Égyiptom sokaságán, és szállítsd alá õt te és a hatalmas nemzetek leányai a mélységek országába azokhoz, kik sírgödörbe szálltak.
\par 19 Kinél nem voltál volna kedvesebb? Szállj alá, hadd fektessenek a körülmetéletlenek mellé!
\par 20 A fegyverrel megölettek közt essenek el! A fegyver átadatott; húzzátok lefelé õt és minden sokaságát!
\par 21 Szólni fognak felõle a vitézek hatalmasai a sír közepébõl együtt az õ segítõivel: alászállottak, hogy itt feküdjenek a körülmetéletlenek, mint fegyverrel megöltek!
\par 22 Itt van Assúr és egész sokasága, körülte sírjai, mindnyájan megölettek, kik fegyver miatt hulltak el,
\par 23 A kinek sírjai a sírgödör legmélyén vannak és sokasága az õ sírjai körül van, mindnyájan megölettek, fegyver miatt elestek, kik félelmére valának az élõk földjének.
\par 24 Itt van Elám és egész sokasága az õ sírja körül, mindnyájan megölettek, kik fegyver miatt hulltak el, kik alászálltak körülmetéletlenül a mélységek országába, kik félelmére valának az élõk földjének, és viselik gyalázatukat azok mellett, kik sírgödörbe szálltak.
\par 25 A megölettek között vetettek néki ágyat minden sokaságával, körülte vannak ennek sírjai, mindnyájan körülmetéletlenek, fegyverrel megölettek, mert félelmére valának az élõk földjének, és viselik gyalázatukat azok mellett, kik sírgödörbe szálltak; a megölettek közé vetették õket!
\par 26 Itt van Mések-Tubál és minden sokasága, körülte vannak ennek sírjai, mindnyájan körülmetéletlenek, fegyverrel megölettek, mert félelmére valának az élõk földjének.
\par 27 És nem feküsznek együtt az erõsekkel, kik elestek a körülmetéletlenek közül, kik hadiszerszámaikkal szálltak alá a sírba s kiknek az õ fegyvereiket fejük alá tették; mert lõn az õ vétkök csontjaikon, mivelhogy félelmére valának a vitézeknek az élõk földjén.
\par 28 Te is a körülmetéletlenek közt rontatol meg, és fekszel a fegyverrel megölettek mellett.
\par 29 Itt van Edom királyaival és minden fejedelmével, kik vettettek erõsségökben a fegyverrel megöltekhez, õk a körülmetéletlenekkel feküsznek s azokkal, a kik sírgödörbe szálltak.
\par 30 Ott vannak észak uralkodói mindnyájan, és minden sidoni, a kik alászálltak a megölettekkel, rettenetes voltukban, erõsségökben megszégyenülve; és feküsznek körülmetéletlenül a fegyverrel megölettek mellett, és viselik gyalázatukat a sírgödörbe szálltak között.
\par 31 Ezeket látni fogja a Faraó, és megvígasztalódik minden sokasága felõl; fegyverrel ölettek meg a Faraó és minden õ serege, ezt mondja az Úr Isten,
\par 32 Mert félelmére adtam õt az élõk földjének; ezért ott vetnek ágyat néki a körülmetéletlenek között, a fegyverrel megölettek mellett, a Faraónak és minden sokaságának, ezt mondja az Úr Isten.

\chapter{33}

\par 1 És lõn az Úr beszéde hozzám, mondván:
\par 2 Embernek fia! szólj néped fiaihoz és mondjad nékik: Mikor hozándok fegyvert valamely földre, és a föld népe választ egy férfit õ magok közül, és õt õrállójokká teszik:
\par 3 És õ látja jõni a fegyvert a földre, és megfújja a trombitát, és meginti a népet;
\par 4 Ha valaki hallándja a trombitaszót, de nem hajt az intésre, s aztán a fegyver eljõ és õt utóléri; az õ vére az õ fején lesz.
\par 5 Hallotta a trombitaszót, de nem hajtott az intésre, tehát az õ vére õ rajta lesz; és ha hajtott az intésre, megmentette lelkét.
\par 6 Ha pedig az õrálló látja a fegyvert jõni, s nem fújja meg a trombitát, és a nép nem kap intést, és eljõ a fegyver s utólér közülök valamely lelket, ez a maga vétke miatt éretett utól, de vérét az õrálló kezébõl kérem elõ.
\par 7 És te, embernek fia, õrállóul adtalak téged Izráel házának, hogy ha szót hallasz a számból, megintsed õket az én nevemben.
\par 8 Ha ezt mondom a hitetlennek: Hitetlen, halálnak halálával halsz meg; és te nem szólándasz, hogy visszatérítsd a hitetlent az õ útjáról: az a hitetlen vétke miatt hal meg, de vérét a te kezedbõl kívánom meg.
\par 9 De ha te megintetted a hitetlent az õ útja felõl, hogy térjen meg róla, de nem tért meg útjáról, õ vétke miatt meghal, de te megmentetted a te lelkedet.
\par 10 Te pedig, embernek fia, mondjad Izráel házának: Ezt mondjátok, mondván: Bizony a mi bûneink és vétkeink rajtunk vannak, és bennök mi megrothadunk, mimódon éljünk azért?
\par 11 Mondjad nékik: Élek én, ezt mondja az Úr Isten, hogy nem gyönyörködöm a hitetlen halálában, hanem hogy a hitetlen megtérjen útjáról és éljen. Térjetek meg, térjetek meg gonosz útaitokról! hiszen miért  halnátok meg, oh Izráel háza!?
\par 12 Te pedig, embernek fia, szólj néped fiaihoz: Az igaznak igazsága meg nem menti õt a napon, a melyen vétkezendik, és a hitetlen hitetlensége által el nem esik a napon, melyen megtérend hitetlenségébõl, és az igaz nem élhet az õ igazsága által a napon, melyen vétkezendik.
\par 13 Mikor ezt mondom az igazról: Élvén éljen; és õ bízván igazságában, gonoszságot cselekszik: semmi igazsága emlékezetbe nem jõ, és gonoszsága miatt, melyet cselekedett, meghal.
\par 14 S ha mondom a hitetlennek: halállal halsz meg; és õ megtér bûnébõl s törvény szerint s igazságot cselekszik;
\par 15 Zálogot visszaad a hitetlen, rablottat megtérít, az életnek parancsolatiban jár, többé nem cselekedvén gonoszságot: élvén él, és meg nem hal.
\par 16 Semmi õ vétke, melylyel vétkezett, emlékezetbe nem jön néki; törvény szerint és igazságot cselekedett, élvén él.
\par 17 És ezt mondják néped fiai: Nem igazságos az Úrnak útja; holott az õ útjok nem igazságos.
\par 18 Mikor az igaz elfordul az õ igazságától s gonoszságot cselekszik, meghal a miatt.
\par 19 És mikor a hitetlen elfordul az õ hitetlenségétõl és törvény szerint s igazságot cselekszik, él a miatt.
\par 20 És azt mondjátok: Nem igazságos az Úrnak útja: mindeniteket az õ útja szerint ítélem meg, Izráel háza.
\par 21 És lõn fogságunknak tizenkettedik esztendejében, a tizedik hónapban, a hónap ötödikén, jöve hozzám egy menekült Jeruzsálembõl, mondván: Megvétetett a város!
\par 22 És az Úrnak keze lõn én rajtam este a menekült eljötte elõtt, és megnyitá számat, mikorra az hozzám jöve reggel, és megnyilatkozék szám, s nem valék néma többé.
\par 23 És lõn az Úr beszéde hozzám, mondván:
\par 24 Embernek fia! azok, kik amaz elpusztított helyeket lakják Izráel földjén, ezt mondják, mondván: Egyetlen egy volt Ábrahám, mikor örökségül kapta a földet, mi pedig sokan vagyunk, nékünk a föld örökségül adatott.
\par 25 Ezokáért mondjad nékik: Ezt mondja az Úr Isten: Véreset esztek, és szemeiteket bálványaitokra emelitek, és vért ontotok; és a földet örökségül kapnátok?
\par 26 Fegyveretekben bíztatok, útálatosságot cselekedtetek, és mindenitek az õ felebarátja feleségét megfertéztette; és a földet örökségül kapnátok?
\par 27 Ezt mondjad nékik: Így szól az Úr Isten: Élek én, hogy a kik az elpusztult helyeken vannak, fegyver miatt hullnak el; és a ki a mezõnek színén, azt a vadaknak adtam eledelül, és a kik az erõsségekben és a barlangokban vannak, döghalállal halnak meg.
\par 28 És teszem a földet pusztaságok pusztaságává, és megszünik erõsségének kevélysége, és puszták lesznek Izráel hegyei, mert nem lesz, a ki átmenjen rajtok.
\par 29 És megtudják, hogy én vagyok az Úr, mikor a földet pusztaságok pusztaságává teszem mindenféle útálatosságukért, melyeket cselekedtek.
\par 30 És te, embernek fia, néped fiai beszélgetnek felõled a falak mellett s a házaknak ajtaiban; és egyik a másikkal szól, kiki az õ atyjafiával, mondván: Jertek, kérlek, és halljátok: micsoda beszéd az, amely az Úrtól jõ ki?
\par 31 És eljõnek hozzád, a hogy a nép össze szokott jõni, s oda ülnek elõdbe, mint az én népem, és hallgatják beszédidet, de nem cselekeszik, hanem szerelmeskedõ énekként veszik azokat ajkokra,  szívök pedig nyereség után jár.
\par 32 És ímé, te olyan vagy nékik, mint valamely szerelmeskedõ ének, szép hangú, s mint valamely jó hegedûs; csak hallják beszédidet, de nem cselekszik azokat.
\par 33 De beteljesednek, mert ímé beteljesednek, megtudják, hogy próféta volt közöttök.

\chapter{34}

\par 1 És lõn az Úrnak beszéde hozzám, mondván:
\par 2 Embernek fia! prófétálj Izráel pásztorai felõl, prófétálj és mondjad nékik, a pásztoroknak: Így szól az Úr Isten: Jaj Izráel pásztorainak, a kik önmagokat legeltették! Avagy nem a nyájat kell-é legeltetni a pásztoroknak?
\par 3 A tejet megettétek, és a gyapjúval ruházkodtatok, a hízottat megöltétek; a nyájat nem legeltettétek.
\par 4 A gyöngéket nem erõsítettétek, és a beteget nem gyógyítottátok, s a megtöröttet nem kötözgettétek, s az elûzöttet vissza nem hoztátok és az elveszettet meg nem kerestétek, hanem keményen és kegyetlenül uralkodtatok rajtok;
\par 5 Szétszóródtak hát pásztor nélkül, és lõnek mindenféle mezei vadak eledelévé, és szétszóródtak;
\par 6 Tévelygett nyájam minden hegyen s minden magas halmon, és az egész föld színén szétszóródott az én nyájam, s nem volt, a ki keresné, sem a ki tudakozódnék utána.
\par 7 Annakokáért, ti pásztorok, halljátok meg az Úr beszédét:
\par 8 Élek én, ezt mondja az Úr Isten, mivelhogy az én nyájam ragadománynyá lõn, és lõn az én nyájam mindenféle mezei vadak eledelévé, pásztor hiányában, és nem keresték az én pásztoraim az én nyájamat, hanem legeltették a pásztorok önmagokat, és az én nyájamat nem legeltették;
\par 9 Ennekokáért, ti pásztorok, halljátok meg az Úr beszédét:
\par 10 Így szól az Úr Isten: Ímé, megyek a pásztorok ellen, és elõkérem a nyájamat az õ kezökbõl, s megszüntetem õket a nyáj legeltetésétõl, és nem legeltetik többé a pásztorok önmagokat, s kiragadom juhaimat szájokból, hogy ne legyenek nékik ételül.
\par 11 Mert így szól az Úr Isten: Ímé, én magam keresem meg nyájamat, és magam tudakozódom utána.
\par 12 Miképen a pásztor tudakozódik nyája után, a mely napon ott áll elszéledt juhai között; így tudakozódom nyájam után, és kiszabadítom õket minden helyrõl, a hova szétszóródtak a felhõnek s borúnak napján.
\par 13 És kihozom õket a népek közül s egybegyûjtöm a földekrõl, és beviszem õket az õ földjökre, és legeltetem õket Izráel hegyein, a mélységekben s a föld minden lakóhelyén.
\par 14 Jó legelõn legeltetem õket, és Izráel magasságos hegyein leszen akluk, ott feküsznek jó akolban, s kövér legelõn legelnek Izráel hegyein.
\par 15 Én magam legeltetem nyájamat, s én nyugosztom meg õket, ezt mondja az Úr Isten;
\par 16 Az elveszettet megkeresem, s az elûzöttet visszahozom, s a megtöröttet kötözgetem, s a beteget erõsítem; és a kövéret s erõset elvesztem, és legeltetem õket úgy, mint illik.
\par 17 Ti pedig, én juhaim, így szól az Úr Isten, ímé én ítéletet teszek juh és juh között, a kosok és bakok közt.
\par 18 Avagy kevés-é néktek, hogy a jó legelõt legelitek, hogy még legelõitek maradékát lábaitokkal eltapodjátok? és hogy a víz tisztáját iszszátok, hogy még a maradékát lábaitokkal felzavarjátok?
\par 19 És az én juhaim a ti lábaitok tapodását legelik, s lábaitok zavarását iszszák!
\par 20 Annakokáért így szól az Úr Isten hozzájok: Ímé én, én teszek ítéletet kövér és ösztövér juh között,
\par 21 Mivelhogy oldallal és vállal eltaszíttok és szarvaitokkal elökleltek minden erõtelent, míg szétszórván, azokat kiûzitek;
\par 22 És megtartom az én juhaimat, hogy többé ne legyenek zsákmányul, és ítéletet teszek juh és juh között.
\par 23 És állatok föléjök egyetlenegy pásztort, hogy legeltesse õket: az én szolgámat, Dávidot,  õ legelteti õket s õ lesz nékik pásztoruk.
\par 24 Én pedig, az Úr, leszek nékik Istenök, és az én szolgám, Dávid, fejedelem közöttök. Én, az Úr mondottam.
\par 25 És szerzek õ velök békességnek frigyét, és megszüntetem a gonosz vadakat  a földrõl, hogy bátorságosan lakhassanak a pusztában és alhassanak az erdõkben.
\par 26 És adok reájok és az én magaslatom környékére áldást, és bocsátom az esõt idejében; áldott esõk lesznek.
\par 27 A mezõ fája megadja gyümölcsét s a föld megadja termését, és lesznek földjökön bátorságosan, és megtudják, hogy én vagyok az Úr, mikor eltöröm jármok keresztfáit, és kimentem õket azok kezébõl, kik õket szolgáltatják.
\par 28 És nem lesznek többé prédául a pogányoknak, s a föld vadai nem eszik meg õket; és laknak bátorságosan, s nem lesz, a ki felijeszsze õket.
\par 29 És támasztok nékik drága plántaföldet, hogy többé meg ne emésztessenek  éhség miatt a földön, s ne viseljék többé a pogányok gyalázatát;
\par 30 És megismerik, hogy én, az Úr, az õ Istenök, velök vagyok, és õk népem, Izráel háza, ezt mondja az Úr Isten;
\par 31 Ti pedig az én juhaim, legelõm nyája vagytok, emberek vagytok, én pedig Istenetek, ezt mondja az Úr Isten.

\chapter{35}

\par 1 És lõn az Úr beszéde hozzám, mondván:
\par 2 Embernek fia! vesd tekintetedet Seir hegyére, és prófétálj ellene.
\par 3 És mondjad néki: Ezt mondja az Úr Isten: Ímé, én ellened megyek, Seir hegye, és kinyújtom kezemet reád, s teszlek pusztaságok pusztaságává.
\par 4 Városaidat pusztaságra vetem, és te pusztulásra jutsz, és megtudod, hogy én vagyok az Úr.
\par 5 Mivelhogy örökkévaló gyûlölséget tartasz, és odaadád Izráel fiait a fegyver kezébe veszedelmök idején, az utolsó vétek  idején;
\par 6 Ennekokáért élek én, ezt mondja az Úr Isten, hogy vérré teszlek téged, és vér kergessen téged. Te nem gyûlölted a vért, a vér kergessen hát téged.
\par 7 És teszem Seir hegyét pusztaságok pusztaságává, és kiirtom belõle az átmenõt és visszatérõt.
\par 8 És betöltöm hegyeit megölöttjeivel; halmaid és völgyeid és minden mélységed: fegyverrel megölöttek hullanak beléjök.
\par 9 Örök pusztasággá teszlek, és városaidat ne lakják, és megtudjátok, hogy én vagyok az Úr.
\par 10 Mivelhogy ezt mondod: Az a két nemzet és az a két föld enyém lesz és örökségül bírjuk, holott az Úr ott volt;
\par 11 Annakokáért élek én, ezt mondja az Úr Isten, hogy a te haragod és fölgerjedésed szerint cselekszem veled, a melylyel te cselekedtél irántok való gyûlölségedbõl; és megjelentem magamat közöttök, mikor megítéllek.
\par 12 És megtudod, hogy én, az Úr, meghallottam minden szidalmadat, melyeket Izráel hegyei ellen mondtál, mondván: Elpusztultak, nékünk adattak ételül;
\par 13 Így kérkedtetek ellenem szátokkal, s szórtátok ellenem beszédeiteket:  én meghallottam!
\par 14 Így szól az Úr Isten: Az egész föld örömére pusztulást hozok rád.
\par 15 A mint te örültél Izráel háza örökségén azért, hogy elpusztult, úgy cselekszem veled; pusztává légy Seir hegye és mind egészen Edom, és megtudják, hogy én vagyok az Úr!

\chapter{36}

\par 1 És te, embernek fia, prófétálj Izráel hegyei felõl, és mondjad: Izráel hegyei, halljátok meg az Úr beszédét,
\par 2 Így szól az Úr Isten: Mivelhogy ezt mondja az ellenség reátok: Haha! és: Az õsi magasságok a mi örökségünk lõnek.
\par 3 Ennekokáért prófétálj, és mondjad: Ezt mondja az Úr Isten: Azért, mert pusztítanak, és kívánnak titeket mindenfelõl, hogy legyetek öröksége a pogányok maradékának, és az emberek rágalmazó ajkára-nyelvére kerültetek;
\par 4 Ezokáért, Izráel hegyei, halljátok meg az Úr Isten beszédét! Így szól az Úr Isten a hegyeknek és halmoknak, a mélységeknek és völgyeknek, és az elpusztult romoknak és az elhagyott városoknak, melyek ragadományra és csúfolásra lõnek a pogányok maradékának köröskörül.
\par 5 Azért ezt mondja az Úr Isten: Bizony, féltõ szerelmem tüzében beszéltem a pogányok maradékai és egész Edom ellen, kik magoknak vették az én földemet örökségül teljes szívöknek örömével és lelkök megvetésével, hogy azt néptelenül zsákmányokká tegyék;
\par 6 Ezokáért prófétálj Izráel földjérõl, és mondjad a hegyeknek és halmoknak, a mélységeknek és a völgyeknek: Így szól az Úr Isten: Ímé, féltõ szerelmemben és búsulásomban beszélek, mivelhogy a pogányok gyalázását viseltétek;
\par 7 Ennekokáért ezt mondja az Úr Isten: Én fölemelem kezemet! Bizonyára a pogányok, kik körülöttetek vannak, õk viseljék gyalázatukat.
\par 8 Ti pedig, Izráel hegyei, neveljétek ágaitokat és hozzátok gyümölcsötöket az én népemnek, Izráelnek, mert közel vannak, hogy hazajõjjenek.
\par 9 Mert ímé, én hozzátok hajlok és hozzátok fordulok, és megmívelnek és bevetnek titeket.
\par 10 És megsokasítom rajtatok az embereket, Izráel házát egészen, és lakják a városokat, és a romokat megépítik.
\par 11 És megsokasítom rajtatok az embereket és barmokat, hogy sokasodjanak és szaporodjanak, és lakatom õket rajtatok, mint régi idõtökben, és több jóval lészek hozzátok, mint elsõ napjaitokban, s megtudjátok, hogy én vagyok az Úr.
\par 12 És járatok rajtatok embereket, az én népemet, Izráelt, és bírjanak téged, s te légy nékik örökségül, s többé nem teszed õket gyermektelenné.
\par 13 Így szól az Úr Isten: Mivelhogy mondják néktek: emberevõ vagy és gyermektelenné teszed népedet:
\par 14 Ennekokáért embert nem eszel többé, és népedet gyermektelenné nem teszed többé, ezt mondja az Úr Isten.
\par 15 És többé nem hallatom ellened a pogányok gyalázását, és a népek szidalmát többé nem viseled, és nemzetedet többé gyermektelenné nem teszed, ezt mondja az Úr Isten.
\par 16 És lõn az Úr beszéde hozzám, mondván:
\par 17 Embernek fia! mikor Izráel háza a maga földén lakott, megfertéztette azt életével és cselekedeteivel; mint a havi betegség  tisztátalansága, olyan vala élete elõttem.
\par 18 És kiöntém haragomat reájok a vérért, a melyet ontottak a földre, és bálványaikkal megfertéztették azt.
\par 19 És eloszlatám õket a pogányok közé, és szétszóratának a tartományokba, életök és cselekedeteik szerint ítéltem meg õket.
\par 20 És bemenvén a pogányokhoz, a kikhez menének, megfertéztették az én szent nevemet, mikor ezt mondták rólok: az Úr népe ezek, és az õ földérõl jöttek ki!
\par 21 Ekkor könyörültem szent nevemért, melyet megfertéztetett Izráel háza a pogányok közt, a kikhez menének.
\par 22 Ennekokáért mondjad Izráel házának: Ezt mondja az Úr Isten: Nem ti érettetek cselekszem, Izráel háza, hanem az én szent nevemért, melyet ti megfertéztettetek a pogányok között, a kik közé menétek.
\par 23 És megszentelem az én nagy nevemet, mely megfertéztetett a pogányok között, melyet ti fertéztettetek meg köztök; és megtudják a pogányok, hogy én vagyok az Úr, ezt mondja az Úr Isten, mikor megszentelem magamat rajtatok az õ szemök láttára.
\par 24 És fölveszlek titeket a pogányok közül, s egybegyûjtelek titeket minden tartományból, és beviszlek titeket a ti földetekre.
\par 25 És hintek reátok tiszta vizet, hogy megtisztuljatok, minden tisztátalanságtoktól és minden bálványaitoktól megtisztítlak titeket.
\par 26 És adok néktek új szívet, és új lelket adok belétek, és elveszem a kõszívet testetekbõl, és adok néktek hússzívet.
\par 27 És az én lelkemet adom belétek, és azt cselekszem, hogy az én parancsolataimban járjatok és az én törvényeimet megõrizzétek és betöltsétek.
\par 28 És laktok azon a földön, melyet adtam atyáitoknak, és lesztek nékem népem s én leszek néktek Istenetek.
\par 29 És megtartalak titeket minden tisztátalanságotoktól, és elõhívom a gabonát és megsokasítom azt, és nem adok reátok éhséget.
\par 30 És megsokasítom a fa gyümölcsét és a mezõ termését, hogy többé ne viseljétek az éhségnek gyalázatját a pogányok között.
\par 31 És megemlékeztek a ti gonosz útaitokról és cselekedeteitekrõl, melyek nem voltak jók, és megútáljátok ti magatokat vétkeitek és útálatosságaitok miatt.
\par 32 Nem ti érettetek cselekszem, ezt mondja az Úr Isten, tudtotokra legyen! Piruljatok és szégyenüljetek meg útaitok miatt, Izráel háza!
\par 33 Ezt mondja az Úr Isten: Azon a napon, melyen megtisztítlak titeket minden vétketektõl, azt cselekszem, hogy lakják a városokat, és a romok megépíttetnek.
\par 34 És az elpusztult földet mívelik, a helyett, hogy pusztán hevert minden átmenõ szeme láttára;
\par 35 És ezek mondják: Ez a föld, ez az elpusztult, olyanná lett, mint az Éden kertje, és a rommá lett s elpusztult s lerontott városokat megerõsítve lakják.
\par 36 És megtudják a pogányok, a kik körülöttetek megmaradtak, hogy én, az Úr építettem meg a lerontottakat s plántáltam be a pusztaságot. Én, az Úr mondtam és megcselekedtem.
\par 37 Így szól az Úr Isten: Még arra nézve is kérni hagyom magamat Izráel házának; hogy cselekedjem õ velök: Megsokasítom õket, mint a nyájat, emberekkel.
\par 38 Mint az áldozatra szentelt juhok, mint a Jeruzsálem juhai az õ ünnepein, így lesznek a rommá lett városok teljesek emberek nyájával; és megtudják, hogy én vagyok az Úr.

\chapter{37}

\par 1 Lõn én rajtam az Úrnak keze, és kivitt engem az Úr lélek által, és letõn engem a völgynek közepette, mely csontokkal rakva vala.
\par 2 És átvitt engem azok mellett köröskörül, és ímé, felette sok vala a völgy színén, és ímé, igen megszáradtak vala.
\par 3 És monda nékem: Embernek fia! vajjon megélednek-é ezek a tetemek? és mondék: Uram Isten, te tudod!
\par 4 És monda nékem: prófétálj e tetemek felõl és mondjad nékik: Ti megszáradt tetemek, halljátok meg az Úr beszédét!
\par 5 Így szól az Úr Isten ezeknek a tetemeknek: Ímé, én bocsátok ti belétek lelket, hogy megéledjetek.
\par 6 És adok reátok inakat, és hozok reátok húst, és bõrrel beborítlak titeket, és adok belétek lelket, hogy megéledjetek, és megtudjátok, hogy én vagyok az Úr.
\par 7 És én prófétálék, a mint parancsolva vala nékem. És mikor prófétálnék, lõn zúgás és ímé zörgés, és egybemenének a tetemek, mindenik tetem az õ teteméhez.
\par 8 És látám, és ímé inak valának rajtok, és hús nevekedett, és felül bõr borította be õket; de lélek nem vala még bennök.
\par 9 És monda nékem: Prófétálj a léleknek, prófétálj embernek fia, és mondjad a léleknek: Ezt mondja az Úr Isten: A négy szelek felõl jõjj elõ lélek, és lehelj ezekbe a megölettekbe, hogy megéledjenek!
\par 10 És prófétálék a mint parancsolá. És beléjök méne a lélek s megéledének, s állának lábaikra, felette igen nagy sereg.
\par 11 És monda nékem: Embernek fia! ezek a tetemek az Izráel egész háza. Ímé, ezt mondják: Elszáradtak a mi csontjaink és elveszett a mi reménységünk; kivágattunk!
\par 12 Annakokáért prófétálj, és mondjad nékik: Így szól az Úr Isten: Ímé, én megnyitom a ti sírjaitokat és kihozlak titeket sírjaitokból, én népem! s beviszlek titeket Izráel földjére.
\par 13 És megtudjátok, hogy én vagyok az Úr, mikor megnyitándom sírjaitokat és kihozlak titeket sírjaitokból, én népem!
\par 14 És adom az én lelkemet belétek, hogy megéledjetek, és leteszlek titeket a ti földetekre, és megtudjátok, hogy én, az Úr, szóltam és megcselekedtem, ezt mondja az Úr Isten.
\par 15 És lõn az Úr beszéde hozzám, mondván:
\par 16 És te, embernek fia, végy magadnak fát, és írd ezt reá: Júdáé és Izráel fiaié, az õ társaié; és végy egy másik fát, és írd ezt reá: Józsefé, Efraim fája és az egész Izráel házáé, az õ társaié.
\par 17 És tedd együvé azokat, egyiket a másikhoz egy fává, hogy egygyé legyenek kezedben.
\par 18 És ha mondják néked a te néped fiai, mondván: Avagy nem jelented-é meg nékünk, mit akarsz ezekkel?
\par 19 Szólj nékik: Ezt mondja az Úr Isten: Ímé, én fölveszem a József fáját, mely Efraim kezében van, és Izráel nemzetségeit, az õ társait, és teszem õket õ hozzá, a Júda fájához, és összeteszem õket egy fává, hogy egygyé legyenek az én kezemben.
\par 20 És ha e fák, a melyekre írsz, kezedben lesznek szemök láttára:
\par 21 Szólj nékik: Ezt mondja az Úr Isten: Ímé, én fölveszem Izráel fiait a pogányok közül, a kik közé mentek, és egybegyûjtöm õket mindenfelõl, és beviszem õket az õ földjökre.
\par 22 És egy néppé teszem õket azon a földön, Izráelnek hegyein, és egyetlenegy király lesz  mindnyájok királya, és nem lesznek többé két néppé, és ezután nem oszolnak többé két királyságra.
\par 23 És többé meg nem fertéztetik magokat bálványaikkal és útálatosságaikkal és minden bûneikkel; és megtartom õket minden oly lakóhelyöktõl, a melyekben vétkeztek, és megtisztítom õket, és lesznek nékem népem és én leszek nékik istenök.
\par 24 És az én szolgám, Dávid lesz a király õ rajtok, s egy pásztora lesz mindnyájoknak; és az én törvényeim szerint járnak, s parancsolataimat megõrzik és cselekszik.
\par 25 És laknak a földön, melyet adtam vala az én szolgámnak, Jákóbnak, a melyen laktak a ti atyáitok; és laknak azon õk és fiaik és fiaiknak fiai mindörökké, és az én szolgám, Dávid az õ fejedelmök örökké.
\par 26 És szerzek velök békességnek frigyét, örökkévaló frigy lesz ez velök; és elültetem õket és megsokasítom, és helyheztetem az én szenthelyemet közéjök örökké.
\par 27 És lesz az én lakhelyem felettök, és leszek nékik Istenök és õk nékem népem.
\par 28 És megtudják a pogányok, hogy én vagyok az Úr, ki megszentelem Izráelt, mikor szenthelyem közöttök lesz mindörökké.

\chapter{38}

\par 1 És lõn az Úr beszéde hozzám, mondván:
\par 2 Embernek fia! vesd tekintetedet Góg ellen, és Mágóg földjén, Rós, Mések és Tubál fejedelme ellen, és prófétálj felõle.
\par 3 És mondjad: Ezt mondja az Úr Isten: Ímé, én ellened megyek Góg, Rós, Mések és Tubál fejedelme.
\par 4 És elcsalogatlak, és horgokat vetek szádba, és kivezetlek téged és egész seregedet, lovakat és lovagokat, kik mindnyájan teljes fegyverzetbe öltözvék, nagy sokaságot, nagy és kis paizszsal, fegyvert viselõk mindnyájan;
\par 5 Perzsák, szerecsenek, libiaiak vannak velök, mindnyájan paizszsal és sisakkal;
\par 6 Továbbá Gómer és minden serege, Tógarma háza messze északról minden seregével, sok nép van veled.
\par 7 Készülj hozzá és készítsd elõ magadat te és minden sokaságod, kik te hozzád gyûltek, és légy nekik vezérök.
\par 8 Sok idõ mulva kirendeltetel: esztendõk végével bejösz a földre, mely a fegyvertõl már megnyugodott, melynek lakói sok nép közül gyûjtettek egybe Izráel hegyeire, melyek szüntelen való pusztulásban voltak; és e nemzetség a népek közül hozatott ki, s aztán lakozék bátorságosan mindnyája;
\par 9 És feljösz, bemégy mint a szélvész, és leszel mint a felleg, hogy beborítsd a földet, te és minden sereged s a sok nép veled.
\par 10 Így szól az Úr Isten: És lészen abban az idõben, hogy tanácsok támadnak szívedben, és gondolsz gondolatot.
\par 11 És mondasz: Felmegyek a nyilt földre, jövök azokra, kik nyugalomban vannak s bátorságosan laknak; kik laknak mindnyájan kõfal-kerítés nélkül, sem zárjok, sem kapujok nincs nékik;
\par 12 Hogy zsákmányt vess és prédát prédálj, hogy fordítsd kezedet a már népes pusztaságok ellen s a nép ellen, a mely a pogányok közül gyûjtetett egybe, mely jószágot és gazdagságot szerez s lakozik a földnek köldökén.
\par 13 Séba és Dedán és Társis kalmárai és minden fiatal oroszlánja ezt mondják néked: Nemde te zsákmányt vetni jöttél? Nemde prédát prédálni gyûjtötted egybe sokaságodat? hogy elvigy ezüstöt és aranyat, magadhoz végy jószágot s gazdagságot, hogy nagy zsákmányt vess.
\par 14 Annakokáért prófétálj, embernek fia, és mondd ezt Gógnak: Így szól az Úr Isten: Avagy abban az idõben, mikor az én népem, Izráel bátorságosan lakik, nem tudod-é meg?
\par 15 És eljössz helyedrõl, a messze északról te és sok nép veled, lovon ülõk mindnyájan, nagy sokaság és hatalmas sereg.
\par 16 És feljössz az én népem, Izráel ellen, mint a felleg, hogy beborítsd a földet, az utolsó idõkben lészen ez, és hozlak téged az én földemre, hogy a pogányok megismerjenek engem, mikor megszentelem magamat rajtad az õ szemök láttára, Góg!
\par 17 Így szól az Úr Isten: Te vagy-é hát, a kirõl szólottam a régi napokban az én szolgáim, Izráel prófétái által, kik prófétáltak azokban a napokban esztendõkön át, hogy téged õ reájok hozlak?
\par 18 És lészen azon a napon, a mely napon Góg eljõ Izráel földje ellen, ezt mondja az Úr Isten, felszáll haragom orromban.
\par 19 És féltõ szerelmemben, búsulásom tüzében szólok: Bizony azon a napon nagy földindulás lesz Izráel földjén.
\par 20 És megremegnek elõttem a tenger halai és az ég madarai és a mezõ vadai és a földön csúszó-mászó mindenféle állatok és minden ember a föld színén; és leszakadnak a hegyek,  és leesnek a meredek kõsziklák, és minden fal a földre hull.
\par 21 És elõhívom ellene minden hegyem felõl a fegyvert, ezt mondja az Úr Isten; egyiknek fegyvere a másik ellen lészen.
\par 22 És törvénykezem vele döghalállal és vérrel; és ömlõ záporesõt, jégesõ köveit, tüzet és kénkövet, mint esõt bocsátok reá és seregére s a sok népre, a mely vele lesz.
\par 23 És nagynak s szentnek mutatom, és megjelentem magamat a pogányok sokasága elõtt, hogy megtudják, hogy én vagyok az Úr!

\chapter{39}

\par 1 És te, embernek fia, prófétálj Góg ellen, és mondjad: Így szól az Úr Isten: Ímé, én ellened megyek Góg, Rós, Mések és Tubál fejedelme!
\par 2 És elcsalogatlak és vezetgetlek, és felhozlak messze északról, és beviszlek Izráel hegyeire.
\par 3 És kiütöm kézívedet balkezedbõl, és nyilaidat jobbkezedbõl kiejtem.
\par 4 Izráel hegyein esel el te és minden sereged és a népek, melyek veled lesznek; a ragadozó madaraknak, minden szárnyas állatnak és a mezei vadaknak adtalak eledelül.
\par 5 A mezõ színén esel el, mert én szóltam, ezt mondja az Úr Isten.
\par 6 És bocsátok tüzet Mágógra és azokra, a kik a szigeteken bátorságosan laknak, hogy megtudják, hogy én vagyok az Úr.
\par 7 És az én szent nevemet megismertetem az én népem, Izráel között, s többé nem hagyom megfertéztetni szent nevemet; és megtudják a pogányok, hogy én, az Úr, szent vagyok Izráelben!
\par 8 Ímé, eljött és meglett, ezt mondja az Úr Isten; ez az a nap, a melyrõl szólottam.
\par 9 És kimennek Izráel városainak lakói, és feltüzelik és felégetik a fegyvereket s a kis és nagy paizsokat, a kézívet és a nyilakat és a kézbeli pálczákat és dárdákat, és tüzelnek velök hét esztendeig.
\par 10 És fát nem hordanak a mezõrõl, sem az erdõkrõl nem vágnak, hanem a fegyverekbõl tüzelnek, és zsákmányt vetnek zsákmányolóikban, s prédálóikat elprédálják, ezt mondja az Úr Isten.
\par 11 És lészen azon a napon, hogy adok Gógnak helyet, hol temetõje legyen Izráelben, a vándorok völgyét keletre a tengertõl, ez fogja bezárni e vándor népséget; és ott temetik el Gógot és minden gyülevészét, és nevezik Góg gyülevésze völgyének.
\par 12 És eltemeti õket Izráel háza, hogy a földet megtisztítsa, hét álló hónapig.
\par 13 És temetni fog az ország egész népe, és lészen ez nékik dicsõségökre a napon, melyen megdicsõítem magamat, ezt mondja az Úr Isten.
\par 14 És választanak embereket, kik a földet szüntelen bejárják, temetgetvén ama vándor népséget, azokat, kik még a föld színén maradtak, hogy azt megtisztítsák. Hét hónap mulva indulnak keresni.
\par 15 És bejárják e járók a földet, és ha ki embertetemet lát, jelt állít melléje, míg a temetgetõk eltemetik azt a Góg gyülevésze völgyében.
\par 16 És egy városnak is Hamóna lesz a neve, és megtisztítják a földet.
\par 17 És te, embernek fia, így szól az Isten, mondjad a madaraknak, minden szárnyas állatnak és minden mezei vadnak: Gyûljetek egybe és jõjjetek el, seregeljetek egybe mindenfelõl az én áldozatomra, mert én nagy áldozatot szerzek néktek Izráel hegyein, és egyetek húst és igyatok vért!
\par 18 Vitézek húsát egyétek, s a föld fejedelmeinek vérét igyátok, kosok, bárányok és bakok, bikák, Básánban hízottak mindnyájan;
\par 19 S egyetek kövérséget jóllakásig, és igyatok vért megrészegedésig az én áldozatomból, a melyet szerzek néktek;
\par 20 És lakjatok jól az én asztalomnál lovakból és paripákból, vitézekbõl és minden hadakozó férfiakból, ezt mondja az Úr Isten.
\par 21 És megmutatom az én dicsõségemet a pogányok között, és meglátják mindazok a pogányok az én ítéletemet, melyet cselekedtem, és az én kezemet, melyet rájok vetettem.
\par 22 És megtudja Izráel háza, hogy én vagyok az Úr, az õ Istenök, attól a naptól fogva és azután.
\par 23 És megtudják a pogányok, hogy az õ vétke miatt vitetett fogságra Izráel háza, mivelhogy elpártolt tõlem, s én elrejtettem orczámat tõle; azért adtam õt ellenségei kezébe, és hullottak el fegyver miatt mindnyájan.
\par 24 Az õ tisztátalanságuk s bûneik szerint cselekedtem velök, és elrejtettem orczámat tõlök.
\par 25 Azért így szól az Úr Isten: Most már haza hozom Jákób foglyait, és megkegyelmezek Izráel egész házának, s féltõ szerelemre gyulladok szent nevemért.
\par 26 És elfelejtik gyalázatukat és minden vétköket, melylyel vétkeztek ellenem, mikor laknak földjökön bátorságosan, és õket senki sem rettegteti.
\par 27 Mikor visszahozom õket a népek közül, és egybegyûjtöm õket ellenségeik földjeirõl, akkor megszentelem magamat bennök sok nép szeme láttára.
\par 28 És megtudják, hogy én vagyok az Úr, az õ Istenök, ki fogságra vittem õket a pogányok közé, majd egybegyûjtém õket földjökre, és senkit közülök többé ott nem hagyok.
\par 29 És többé el nem rejtem orczámat tõlök, mivelhogy kiöntöttem lelkemet Izráel házára, ezt mondja az Úr Isten.

\chapter{40}

\par 1 Huszonötödik esztendejében fogságunknak, az esztendõnek kezdetén, a hónap tizedikén, a tizennegyedik esztendõben az után, hogy a város megveretett, épen ezen a napon lõn az Úr keze én rajtam, és elvitt engemet oda.
\par 2 Isteni látásokban vitt engem Izráel földjére, és letõn engem egy igen magas hegyre, s azon vala mint egy város épülete dél felõl.
\par 3 És oda vitt engem, és ímé egy férfiú vala ott, tekintete min az ércznek tekintete, és len-zsinór vala kezében és mérõpálcza; és a kapuban áll vala.
\par 4 És szóla nékem az a férfiú: Embernek fia! láss szemeiddel és füleiddel hallj, és figyelmetes légy mindarra, a mit mutatok néked; mert hogy ezeket megmutassam néked, azért hozattál ide: hirdesd mindazokat, a miket látsz, Izráel házának.
\par 5 És ímé, kõfal vala a házon kivül köröskörül; és a férfiú kezében a mérõpálcza vala hat singnyi (a közsingben s egy tenyérben); és méré az épület szélességét egy pálczányira s magasságát egy pálczányira.
\par 6 És méne egy kapuhoz, mely napkeletre néz vala, és felméne grádicsain, és méré a kapu küszöbét egy pálczányi szélességre, a másik küszöböt is egy pálczányi szélességre;
\par 7 És az õrkamarát egy pálczányi hosszúságra és egy pálczányi szélességre, és az õrkamarák közét öt singnyire, és a kapu küzsöbét, a kapu tornácza mellett belõl, egy pálczányira.
\par 8 És méré a kapu tornáczát belöl egy pálzánzira.
\par 9 És méré a kapu tornáczát nyolcz singnyire; és gyámoszlopait két singnyire; vala pedig a kapu tornácza belõl.
\par 10 És a napkeleti kapunak mind egyfelõl, mind másfelõl három-három õrkamarája vala, egy mértéke mind a háromnak, és egy mértékök a gyámoszlopoknak is mind egyfelõl, mind másfelõl.
\par 11 És méré a kapu nyílásának szélességét tíz singre, és a kapu hosszúságát tizenhárom singre;
\par 12 És az õrkamarák elõtt való korlátot egy singnyire, és egy singnyire vala e korlát másfelõl is; mindenik õrkamara pedig hat singnyi vala egyfelõl és hat singnyi másfelõl.
\par 13 És méré a kaput az egyik õrkamara tetejétõl a másik tetejéig; huszonöt singnyi szélességre ott, hol ajtó ajtóval vala szemben.
\par 14 És tevé a gyámoszlopokat hatvan singre, és gyámoszlopokhoz nyúlik vala a pitvar a kapunál körös-körül.
\par 15 És a bejárat kapujának elejétõl a belsõ kapu tornáczának elejéig vala ötven sing.
\par 16 És az õrkamarákon szoros ablakok valának, és gyámoszlopaikon is belül a kapuban köröskörül, hasonlóképen a tornáczokon; és valának ablakok köröskörül belõl, és a gyámoszlopokon pálmafaragások.
\par 17 És vitt engem a külsõ pitvarba, és ímé, ott kamarák és kõbõl rakott pádimentom vala készítve a pitvaron köröskörül; harmincz kamara vala a kõbõl rakott pádimentomon.
\par 18 És a kõbõl rakott pádimentom a kapuk mellett vala a kapuk hosszúsága szerint: az alsó kõbõl rakott pádimentom vala ez.
\par 19 És méré a szélességet az alsó kapu elejétõl fogva a belsõ pitvar külsõ elejéig száz singnyire, a keleti és északi oldalon.
\par 20 És a kapunak is, mely néz vala északra, a külsõ pitvaron, megméré hosszúságát és szélességét;
\par 21 És õrkamarái valának: három egyfelõl és három másfelõl, és gyámoszlopai és tornácza egy mértékben valának az elsõ kapuval: ötven sing a hosszúsága, és szélessége huszonöt sing.
\par 22 És ablakai és tornácza és pálmafaragásai annak a kapunak mértéke szerint valának, mely néz napkeletre, és hét grádicson mennek vala fel hozzá; és tornáczai, e grádicsok elõtt valának.
\par 23 És a belsõ pitvarnak kapuja vala az északi és napkeleti kapu ellenében, és mére kaputól kapuig száz singet.
\par 24 És vitt engem a déli útra, és ímé, egy kapu vala ottdél felé, és megméré gyámoszlopait és tornáczát ugyanama mérték szerint.
\par 25 És ablakai valának és tornáczának is köröskörül, olyanok mint amaz ablakok; hosszúsága ötven sing, és szélessége huszonöt sing.
\par 26 És hét grádicsa vala feljáratának, és tornácza azok elõtt vala; és pálmafaragásai valának egyik egyfelõl, a másik másfelõl gyámoszlopain.
\par 27 És kapuja vala a belsõ pitvarnak dél felé, és mére kaputól kapuig dél felé száz singet.
\par 28 És bevitt engemet a déli kapun át a belsõ pitvarba, és megméré a déli kaput ugyanama mértékek szerint;
\par 29 És õrkamaráit és gyámoszlopait és tornáczát ugyanama mértékek szerint, és ablakai valának és tornáczának is köröskörül, hosszúsága ötven sing, és szélessége huszonöt sing.
\par 30 És tornáczok valának köröskörül, hosszúságuk huszonöt sing, és szélességök öt sing.
\par 31 És tornácza a külsõ pitvar felé vala, és pálmafaragások valának gyámoszlopain, és nyolcz grádicsa vala feljáratának.
\par 32 Vitt továbbá engem a belsõ pitvarba kelet felé, és megméré a kaput ugyanama mértékek szerint;
\par 33 És õrkamaráit és gyámoszlopait és tornáczát ugyanama mértékek szerint, és ablakai valának és tornáczának is köröskörül, hosszúsága ötven sing, és szélessége huszonöt sing.
\par 34 És tornácza vala a külsõ pitvar felé, és gyámoszlopain pálmafaragások valának mind egy-, mind másfelõl, és nyolcz grádicsa vala feljáratának.
\par 35 És vitt engem az északi kapuhoz, és megméré ugyanama mértékek szerint;
\par 36 Õrkamaráit, gyámoszlopait és tornáczát, és ablakai valának köröskörül, hosszúsága ötven sing, és szélessége huszonöt sing.
\par 37 És tornácza a külsõ pitvar felé vala, és pálmafaragások valának gyámoszlopain mind egyfelõl, mind másfelõl, és nyolcz grádicsa vala feljáratának.
\par 38 És egy kamara vala, és annak ajtaja a kapuk gyámoszlopainál; ott mossák vala meg az egészen égõáldozatot.
\par 39 És a kapu tornáczában két asztal vala egyfelõl, és másfelõl is két asztal vala, hogy azokon öljék meg az egészen égõáldozatot és a bûnért való áldozatot és a vétekért való áldozatot.
\par 40 És oldalt kivül, északra onnét, hol felmennek a kapu ajtajához, két asztal vala, és a kapu tornáczának másik oldalán is két asztal.
\par 41 Négy asztal egyfelõl, és másfelõl is négy asztal a kapu oldalán; nyolcz asztal, ezeken ölik vala az áldozatot.
\par 42 És négy asztal vala az égõáldozatra faragott kõbõl, másfél sing hosszú és másfél sing széles és egy sing magas; ezekre teszik a szerszámokat, melyekkel az égõáldozatot és egyéb áldozatot ölnek.
\par 43 És a szegek valának egy tenyérnyiek, oda erõsítve belõl köröskörül, s az asztalokra jöve az áldozat húsa.
\par 44 A belsõ kapun kivül pedig vala két kamara az énekesek számára a belsõ pitvarban, egyik oldalt az északi kaputól, melynek eleje dél felé vala, s a másik oldalt a déli kaputól, melynek eleje észak felé vala.
\par 45 És szóla nékem: Ez a kamara, mely délre néz, a papoké, kik a házhoz való szolgálatban foglalatosak:
\par 46 Az a kamara pedig, mely északra néz, azoké a papoké, kik az oltárhoz való szolgálatban foglalatosak, ezek a Sádók fiai, kik az Úrhoz járulnak a Lévi fiai közül, hogy szolgáljanak néki.
\par 47 És megméré a pitvart; hosszúsága száz sing és szélessége is száz sing, négyszögre; és az oltár vala a ház elõtt.
\par 48 És vitt engem a ház tornáczába, és a tornácz gyámoszlopát megméré egyfelõl is öt singre, másfelõl is öt singre; a kapu szélességét pedig három singre egyfelõl s másfelõl is három singre.
\par 49 A tornácz hosszúsága húsz sing vala és a szélessége tizenegy sing, és tíz grádicson mennek vala föl hozzá; és oszlopok valának a gyámoszlopoknál, egyik egyfelõl, másik másfelõl.

\chapter{41}

\par 1 És bevitt engem a templomba, és megméré a gyámoszlopokat, hat sing széles vala egyfelõl és hat sing széles másfelõl a gyámoszlopok szélessége.
\par 2 És az ajtó szélessége tíz sing vala, és az ajtó oldalfalai öt sing egyfelõl és öt sing másfelõl; és megméré hosszúságát is negyven singre, szélességét pedig húsz singre.
\par 3 És beméne belsejébe, és megméré az ajtó gyámoszlopait két singre, és az ajtót hat singre, és az ajtó oldalfalait hét singre.
\par 4 És megméré hosszúságát húsz singre, és szélességét húsz singre a templom felé, és mondá nékem: Ez a szentek  szentje.
\par 5 És megméré a ház falát hat singre, és az oldalépület szélességét négy singre, a ház körül mindenfelõl.
\par 6 És az oldalkamarák, egyik kamara a másikon, harminczan valának három sorban, és a háznak azon falához nyúlának, mely az oldalkamarák felõl vala, köröskörül, hogy azon erõsen álljanak, de nem magába a falba valának erõsítve.
\par 7 És vala kiszélesedése és megnagyobbodása az oldalkamaráknak a magasságban fölfelé, mivel megnagyobbodása vala a háznak fölfelé a magasságban köröskörül a házon, ezért lõn kiszélesedése az épületnek fölfelé. És az alsó sorról menének föl a felsõhöz és a középsõhöz.
\par 8 És láték a háznál valami magasságot köröskörül, fundamentoma ez az oldalkamaráknak, egy teljes pálczányi, hat singnyire a fal tövének párkányáig.
\par 9 Az oldalkamara külsõ falának szélessége pedig öt singnyi vala. És a mi üresen maradt a ház oldalkamrái között.
\par 10 És a kamarák közt húsz singnyi szélesség vala a ház körül köröskörül.
\par 11 És az oldalépületnek ajtaja az üresen maradt hely felé vala, egyik ajtó északra és a másik ajtó délre, és a szabadon hagyott hely szélessége öt singnyi vala köröskörül.
\par 12 És az épület, mely az elkülönített hely elõtt, nyugot felé vala, hetven singnyi széles vala, s az épület fala öt singnyi széles köröskörül, hosszúsága pedig kilenczven sing.
\par 13 És méré a háznak hosszát száz singre, és az elkülönített helyet s az épületet és falait száz sing hosszúságra;
\par 14 És a ház és az elkülönített hely elejének szélességét kelet felé száz singre.
\par 15 És méré az épület hosszúságát az elkülönített hely felé, mely ennek háta mögött vala, és ennek folyósóit mind egyfelõl, mind másfelõl száz singre. És magában a templomban és a belsõ helyen és a külsõ tornáczban.
\par 16 A küszöbök és a szoros ablakok és a folyosók valának mind hármuk körül, szemben a küszöbbel, simított fadeszkákból köröskörül. A földtõl az ablakokig, az ablakok pedig befedve valának.
\par 17 Az ajtónak felsõ részéig és a belsõ házig és kifelé és az egész falon köröskörül a belsõ és külsõ helyen minden mérték szerint.
\par 18 Készítve valának továbbá Kérubok és pálmafaragások, és mindenik pálmafaragás vala két Kérub között, és mindenik Kérubnak két orczája vala.
\par 19 És emberorcza néz vala a pálmafaragásról egyfelõl, és oroszlánorcza a másik pálmára másfelõl. Így vala készítve az egész házon köröskörül.
\par 20 A földtõl fogva az ajtó felsõ részéig Kérubok és pálmafaragások valának készítve a falon.
\par 21 A szenthelynek ajtófélfái négyszegûek valának, és a szentek szentje elõtt vala mint egy oltárnak a formája.
\par 22 Fából, három sing magas és két sing hosszú, szegletei valának és talpa, és falai fából valának. És mondá nékem: Ez az asztal, mely az Úr elõtt áll.
\par 23 És kettõs ajtói valának a szenthelynek és a szentek szentjének,
\par 24 És kettõs ajtói valának az ajtószárnyaknak, két forgó ajtóik, kettõ az egyik ajtószárnyban és két ajtó a másikban.
\par 25 És készítve valának rajtok, a szenthely ajtóin, Kérubok és pálmák, mint valának készítve a falakon, és faküszöb vala a tornácz elõtt kivül;
\par 26 És szoros ablakok és pálmák valának mind egy- s mind másfelõl a tornácz oldalfalain. Így voltak a ház oldalkamarái és a gerendák.

\chapter{42}

\par 1 És kivitt engem a külsõ pitvarba az északi úton és vitt engem a kamarák épületéhez, mely az elkülönített hely ellenében és az épület ellenében észak felé vala;
\par 2 A száz sing hosszú oldal elé az északi oldalra, és vala ötven sing a szélessége.
\par 3 A húsz singnek ellenében, mely a belsõ pitvarhoz tartozék, és a kõbõl rakott pádimentom ellenében, mely a külsõ udvarhoz tartozék, folyosó vala folyosó ellenében három sorban.
\par 4 És a kamarák elõtt tíz sing széles út vala befelé, hosszúsága száz sing; és ajtaik észak felé valának.
\par 5 És a felsõ kamarák rövidebbek valának az épület alsó és középsõ kamaráinál, mert a folyosók elvettek belõlök.
\par 6 Mivelhogy három sorban valának és nem valának oszlopaik, mint a pitvaroknak; ezért lõn az alsókhoz és a középsõkhöz képest elvéve a helybõl.
\par 7 És egy fal, mely kivül vala, párhuzamosan a kamarákkal a külsõ pitvar felé, a kamarák elõtt, ötven sing hosszú vala.
\par 8 Mert a kamarák hosszúsága, melyek a külsõ pitvar felé valának, ötven sing vala, és ímé, a szenthely ellenében száz sing vala.
\par 9 És e kamarák alatt vala a bejáró hely napkelet felõl, ha beléjök a külsõ pitvarból méne valaki.
\par 10 A pitvar falának szélességében dél felé az elkülönített hely és az épület elõtt is kamarák valának.
\par 11 És egy út vala elõttök, melyek hasonlatosak valának a kamarákhoz, melyek északra valának, olyan hosszúságúak és olyan szélességûek, mint ezek, és minden kijárásuk és elrendezésök olyan vala, mint ezeké. És valamint amazoknak ajtói,
\par 12 Akképen valának azoknak a kamaráknak ajtói is, melyek dél felé valának: egy ajtó kezdetén az útnak, annak az útnak, mely a megfelelõ fal elõtt kelet felé vala, ha valaki hozzájok méne.
\par 13 És mondá nékem: Az északi és a déli kamarák, melyek az elkülönített hely elõtt vannak, azok a szent kamarák, melyekben a papok, kik az Úrhoz közelednek, eszik az igen szentséges áldozatokat, ott rakják le az igen szentséges áldozatokat, az eledeli, a bûnért való és a vétekért való áldozatot; mert szent az a hely.
\par 14 Mikor bemennek oda a papok, ki ne jõjjenek a szent helyrõl a külsõ pitvarba, hanem ott rakják le ruháikat, melyekben szolgáltak; mert szentek ezek, és más ruhába öltözzenek, úgy közeledjenek a nép helyéhez.
\par 15 És mikor a belsõ háznak méréseit véghezvitte, kivitt engem a napkelet felõl való kapu útján, és megméré azt köröskörül.
\par 16 Méré a keleti oldalt a mérõpálczával ötszáz singnyire, a mérõpálczával. És megfordula, és
\par 17 Méré az északi oldalt ötszáz singnyire a mérõpálczával. És megfordula,
\par 18 A déli oldal felé, és mére ötszáz singet a mérõpálczával.
\par 19 Fordula aztán a nyugoti oldal felé, és mére ötszáz singet mérõpálczával.
\par 20 Mind a négy szél irányában méré azt. Kõfala vala köröskörül, hosszúsága ötszáz, szélessége is ötszáz sing, hogy elválaszsza a szentet a köztõl.

\chapter{43}

\par 1 És vitt engem a kapuhoz, ahhoz a kapuhoz, a mely napkelet felé néz vala.
\par 2 És ímé, Izráel Istenének dicsõsége jõ vala napkelet felõl, és zúgása, mint nagy víz zugása és a föld világos vala az õ dicsõségétõl.
\par 3 És a jelenség, melyet láttam, olyan vala, mint az a jelenség, melyet akkor láttam, mikor menék a várost elveszteni; és olyan látások valának, mint az a jelenség, melyet a Kébár folyó mellett láttam. És orczámra esém.
\par 4 És az Úr dicsõsége beméne a házba a kapunak útján, mely néz napkeletre.
\par 5 És fölemele engem a lélek, és bevitt engem a belsõ pitvarba, és ímé, az Úr dicsõsége betölté a házat.
\par 6 És hallám, hogy valaki beszélget hozzám a házból, holott ama férfiú mellettem áll vala.
\par 7 És mondá nékem: Embernek fia! ezt az én királyi székemnek helyét és lábaim talpainak helyét, a hol lakom Izráel fiai között örökké, és itt az én szent nevemet többé meg ne fertéztesse  Izráel háza, õk és királyaik paráznaságukkal és királyaik holttesteivel, magaslataikkal,
\par 8 Mikor küszöbüket az én küszöböm mellé és ajtófélfáikat az én ajtófélfáim mellé tették, és csak a fal vala köztem és közöttök; és megfertéztették az én szent nevemet útálatosságaikkal, melyeket cselekedtek, ezokáért elvesztém õket haragomban.
\par 9 Most már eltávoztatják paráznaságukat és királyaik holttesteit tõlem, és õ köztök lakozom mindörökké.
\par 10 Te, embernek fia, hirdessed Izráel házának ezt a házat, hogy piruljanak vétkeik miatt. És mérjék utána arányosságát;
\par 11 És ha pirulni fognak mind a miatt, a mit cselekedtek: e ház formáját és berendezését, kijáratait és bejáratait és minden formáit és minden rendeléseit és minden formáit és minden törvényeit jelentsd meg nékik és írd meg szemeik elõtt, hogy megtartsák minden formáját és minden rendeléseit, s azokat cselekedjék.
\par 12 Ez a ház törvénye: A hegy tetején egész határa köröskörül igen szentséges. Ímé, ez a ház törvénye.
\par 13 És ezek az oltár mértékei singek szerint (egy sing tesz egy közsinget és egy tenyért): talpa vala egy sing magas és egy sing széles, korlátja pedig köröskörül egy arasznyi. És ez az oltár magassága:
\par 14 A földön való talptól az alsó bekerítésig két sing s szélessége egy sing, és a kisebb bekerítéstõl a nagyobb bekerítésig négy sing és szélessége egy sing;
\par 15 És az Istenhegye négy sing, és az Isten-tûzhelyétõl fölfelé a szarvak egy sing.
\par 16 És az Isten-tûzhelyének hosszúsága tizenkét sing vala tizenkét sing szélesség mellett, négyszögben négy oldala szerint.
\par 17 És a bekerítés vala tizennégy sing hosszú, tizennégy sing szélesség mellett, négy oldala szerint; és a korlát köröskörül rajta fél sing. És talpa vala egy sing köröskörül. És grádicsai napkeletre néztek.
\par 18 És mondá nékem: Embernek fia, ezt mondja az Úr Isten: Ezek az oltárnak rendtartásai: Azon a napon, mikor elkészül, hogy égõáldozattal áldozzanak rajta és vért hintsenek reá,
\par 19 Adj a papoknak, a Lévitáknak, kik a Sádók magvából valók, a kik járulhatnak én hozzám, ezt mondja az Úr Isten, hogy szolgáljanak nékem, egy fiatal bikát bûnért  való áldozatul.
\par 20 És végy vérébõl, és tedd az oltár négy szarvára és a bekerítés négy szegletére és a korlátra köröskörül, és tisztísd meg azt, és szenteld meg.
\par 21 És vegyed a bikát, mely a bûnért való áldozat, és égesd meg azt a háznak erre rendelt helyén, a szenthelyen kivül.
\par 22 Másodnap pedig vígy egy ép kecskebakot bûnért való áldozatul, és tisztítsák meg vele az oltárt, mint a fiatal bikával megtisztították.
\par 23 Mikor elvégezed az oltár megtisztítását, vígy áldozatul egy ép, fiatal bikát és egy ép kost a nyájból.
\par 24 És áldozzál velök az Úr elõtt, és a papok hintsenek sót reájok, és vigyék azokat égõáldozatul az Úrnak.
\par 25 Hét napig mindennap egy-egy bakot hozz bûnért való áldozatul, és egy fiatal bikát és egy kost a nyájból, mint épeket hozzák.
\par 26 Hét napon át szenteljék és tisztítsák meg az oltárt és töltsék meg kezét.
\par 27 És betöltsék e napokat; a nyolczadik napon és azután pedig áldozzák a papok az oltáron égõáldozataitokat és hálaadó áldozataitokat, és én kegyelmesen fogadlak titeket, ezt mondja az Úr Isten.

\chapter{44}

\par 1 És visszavitt engem a szenthely külsõ kapujának útjára, mely keletre néz vala; és az zárva volt.
\par 2 És monda nékem az Úr: Ez a kapu be lesz zárva, meg nem nyílik és senki rajta be nem megy, mert az Úr, az Izráel Istene ment be rajta, azért be lesz zárva.
\par 3 A fejedelem pedig, õ a fejedelem benne üljön, hogy kenyeret egyék az Úr elõtt; e kapu tornáczának útján menjen be és ennek útján menjen ki.
\par 4 És bevitt engem az északi kapu útján a ház eleibe, és látám, és ímé, az Úr háza betelék az Úr dicsõségével, és orczámra esém.
\par 5 És monda nékem az Úr: Embernek fia, figyelmetes légy, és lásd szemeiddel, és füleiddel halld meg, valamiket én szólok te veled az Úr házának minden rendtartásairól és minden törvényérõl, és légy figyelmetes a háznak bejáratára a szenthelynek minden kijáratánál.
\par 6 És mondjad a pártosnak, Izráel házának: Így szól az Úr Isten: Elég legyen immár néktek minden útálatosságtok, Izráel háza!
\par 7 Mikor idegeneket, körülmetéletlen szívûeket és körülmetéletlen testûeket engedtetek bemenni, hogy szenthelyemen legyenek, hogy megfertéztessék azt, az én házamat; mikor áldozatra fölvittétek kenyeremet, a kövérséget és vért: ekkor megtörtétek az én frigyemet minden útálatosságtokon felül;
\par 8 És nem voltatok foglalatosak a szentségeimhez való szolgálatban, hanem azokat tettétek foglalatosakká szolgálatomban az én szenthelyemen, magatok helyett.
\par 9 Így szól az Úr Isten: Senki idegen körülmetéletlen szívû és körülmetéletlen testû be ne menjen az én szenthelyemre, mindazon idegenek közül, a kik Izráel fiai között vannak.
\par 10 Sõt még a Léviták is, kik eltávoztak én tõlem, mikor Izráel eltévelyedék (a kik eltévelyedtek én tõlem bálványaik után), viseljék vétköket;
\par 11 És legyenek szenthelyemben szolgák a ház kapuinak õrizetére, és legyenek szolgák a házban: õk öljék le az égõáldozatot és egyéb áldozatot a népnek, és õk álljanak elõttök, hogy szolgáljanak nékik;
\par 12 Mivelhogy szolgáltak nékik bálványaik elõtt, és lõnek Izráel házának vétekre csábítói: ezért emeltem föl kezemer ellenök, ezt mondja az Úr Isten, hogy viseljék vétköket.
\par 13 Ne közeledjenek hozzám, hogy nékem, mint papok szolgáljanak, és hogy minden szentségemhez járuljanak, a szentségek szentségeihez; hanem viseljék gyalázatukat és útálatosságaikat, melyeket cselekedtek.
\par 14 És teszem õket foglalatosakká a házhoz való szolgálatban, annak minden szolgálatánál, és mindannál, a mit abban cselekedni kell.
\par 15 De azok a lévita-papok, a Sádók fiai, a kik a szenthelyemhez való szolgálatomban foglalatosak voltak, mikor Izráel fiai eltévelyedtek tõlem, õk járuljanak én hozzám, hogy szolgáljanak nékem és álljanak én elõttem, hogy áldozzanak nékem kövérséggel és vérrel, ezt mondja az Úr Isten.
\par 16 Õk járjanak be az én szenthelyembe, és járuljanak az én asztalomhoz, hogy szolgáljanak nekem, és legyenek foglalatosak az én szolgálatomban.
\par 17 És mikor a belsõ pitvar kapuihoz be akarnak menni, len-ruhába öltözzenek, és gyapjú ne legyen rajtok, ha a belsõ pitvar kapuiban és a házban szolgálnak.
\par 18 Len-süveg legyen fejökön, és len alsó-nadrág derekukon; ne övezkedjenek izzadásig.
\par 19 Mikor pedig kimennek a külsõ pitvarba, a külsõ udvarba a néphez, vessék le ruháikat, a melyekben szolgáltak, és tegyék le azokat a szent kamarákban, és öltözzenek más ruhákba, hogy meg ne szenteljék a népet ruháikkal.
\par 20 És fejöket kopaszra ne nyírják, de nagy hajat se neveljenek, hanem hajokat megegyengessék.
\par 21 És senki a papok közül bort ne igyék, mikor a belsõ udvarba bemennek.
\par 22 És özvegyet vagy eltaszítottat feleségül magoknak ne vegyenek, hanem szûzeket Izráel házának magvából; de olyan özvegyet, ki papnak özvegye, elvehetnek.
\par 23 És az én népemet tanítsák, hogy mi a különbség szent és köz között, és a tisztátalan és tiszta között való különbséget ismertessék meg velök.
\par 24 És peres ügyben õk áljanak elõ ítélni, az én törvényeim szerint ítéljék meg azt; és tanításaimat s rendeléseimet megtartsák minden ünnepemen, és az én szombataimat megszenteljék.
\par 25 És holt emberhez be ne menjen, hogy magát megfertéztesse, csak atyjáért és anyjáért és fiáért és leányáért és fiútestvéréért és leánytestvéréért, a kinek még férje nem volt, fertéztetheti meg magát.
\par 26 És megtisztulása után számláljanak néki hét napot.
\par 27 És a mely napon bemegy a szenthelyre, a belsõ pitvarba, hogy szolgáljon a szenthelyen, mutassa be bûnért való áldozatát, ezt mondja az Úr Isten.
\par 28 És lészen nékik örökségök: én vagyok az õ örökségök, és birtokot ne adjatok nékik Izráelben: én vagyok az õ birtokuk.
\par 29 Az ételáldozat és a bûnért és vétekért való áldozat, ez legyen élésök, és valami Izráelben Istennek szenteltetik, minden az övék legyen.
\par 30 És minden elsõ termés zsengéje mindenbõl, és minden, mit áldozatra visztek mindenbõl, tudniillik minden áldozatotokból, legyen a papoké, és lisztjeitek zsengéjét adjátok a papnak, hogy áldás nyugodjék házadon.
\par 31 Semmi döglöttet és vadtól szaggatottat, akár madár, akár egyéb állat legyen, ne egyenek a papok.

\chapter{45}

\par 1 És mikor a földet sorsvetéssel elosztjátok örökségül, adjatok áldozatot az Úrnak, szent részt a földbõl; hosszúsága legyen huszonötezer és szélessége tízezer sing, szent legyen az egész határában köröskörül.
\par 2 Ebbõl legyen a szenthelyé ötszáz sing, ötszáz singgel mérve egy négyszög minden oldalát, és mellette ötven sing tágasság legyen minden oldalon.
\par 3 És ebbõl a megmért helybõl mérj ki huszonötezer sing hosszúságot és tízezer sing szélességet, és ebben lesz a szenthely, mint igen szent hely.
\par 4 Szent rész lesz ez a földbõl; legyen a papoké, a szenthely szolgáié, a kik ide járulnak szolgálni az Urat; és legyen ez nékik házaik helye, és szent hely a szenthely számára.
\par 5 És huszonötezer sing hosszában és tízezer sing széltében legyen a Lévitáké, a ház szolgáié tulajdonukul, húszkamarául,
\par 6 És a város tulajdonául adjatok ötezer sing szélességet és huszonötezer sing hosszúságot a szent áldozat mentén; az egész Izráel házáért legyen ez.
\par 7 És a fejedelem tulajdona lészen a szent áldozatnak és a város tulajdonának mind a két oldalán a szent áldozat elõtt és a város tulajdona elõtt a napnyugoti oldalon nyugotra és a napkeleti oldalon keletre, és hosszúsága olyan lesz, mint a részek közül egyé, a napnyugoti határtól a napkeleti határig.
\par 8 Földje legyen az néki, tulajdona Izráelben; és többé ne sanyargassák fejedelmeim az én népemet, hanem adják át a többi földet Izráel házának az õ nemzetségei szerint.
\par 9 Ezt mondja az Úr Isten: Legyen elég már néktek, Izráel fejedelmei! A törvénytelenséget és erõszaktételt távoztassátok el, és cselekedjetek törvény szerint és igazságot. Vegyétek le népemrõl sarczolástokat, ezt mondja az Úr Isten.
\par 10 Igaz mérõserpenyõitek legyenek és igaz éfátok és báthotok:
\par 11 Az éfa és a báth egy mértékûek legyenek, úgy hogy a hómernek tizedét fogadja be a báth, és a hómernek tizede legyen az éfa, a hómerhez kell mértéköket igazítani.
\par 12 És a sekelnek húsz gérája legyen; húsz sekel, huszonöt sekel, tizenöt sekel legyen a mina nálatok.
\par 13 Ez az áldozat, melyet fel kell vinnetek: egy hómer búzából egy hatodrész éfa, és egy hómer árpából is az éfa hatodrészét adjátok.
\par 14 És az olajból rendelt rész egy báth olajból: a báth tizede a kórból, tíz báthból, egy hómérbõl, mert tíz báth egy hómer.
\par 15 És egy darab a nyájból, kétszáz darab után Izráel bõvizû földjérõl ételáldozatra és égõáldozatra és hálaadó áldozatokra, az õ megszentelésökre, ezt mondja az Úr Isten.
\par 16 A földnek egész népe köteles legyen erre az áldozatra a fejedelem részére Izráelben.
\par 17 A fejedelem tiszte pedig lészen, hogy vigyen egészen égõáldozatokat és ételáldozatot és italáldozatot az ünnepeken és az újholdak napján és a szombatokon, Izráel házának minden ünnepein: õ teljesítse a bûnért való áldozatot és az ételáldozatot és az egészen égõáldozatot és a hálaadó áldozatokat, hogy megszentelje Izráel házát.
\par 18 Ezt mondja az Úr Isten: Az elsõ hónapban, a hónap elsején végy egy fiatal, hibátlan bikát és tisztítsd meg a szenthelyet.
\par 19 És vegyen a pap a bûnért való áldozat vérébõl, és hintse a ház ajtófélfáira és az oltár bekerítésének négy szegletére és a belsõ pitvar kapufélfáira.
\par 20 És így cselekedjél a hónap hetedik napján a tudatlanságból vétkezõért és a vigyázatlanságért, és így tisztítsátok meg a házat.
\par 21 Az elsõ hónapban, a hónap tizennegyedik napján legyen a ti páskátok; a hét napos ünnepen kovásztalan kenyeret egyetek.
\par 22 És a fejedelem azon a napon áldozzék õ magáért és a föld egész népéért egy bikával, bûnért való áldozatul.
\par 23 És az ünnep hét hapján tegyen egészen égõáldozatot az Úrnak hét bikával és hét kossal, épekkel, naponként hét napon át, és bûnért való áldozatot egy kecskebakkal naponként;
\par 24 És ételáldozatul tegyen egy éfát a bika mellé és egy éfát a kos mellé, és az olajból egy hínt az éfához.
\par 25 A hetedik hónapban, a hónap tizenötödik napján, az ünnepen hasonlóképen cselekedjék hét napon át: a bûnért való áldozattal, az egészen égõáldozattal, az ételáldozattal és az olajjal.

\chapter{46}

\par 1 Ezt mondja az Úr Isten: A belsõ pitvar kapuja, mely keletre néz, zárva legyen a dologtevõ hat napon, szombaton pedig nyissák ki, és újhold napján is nyissák ki.
\par 2 És a fejedelem jõjjön be a kapu tornáczának útján kívülrõl, és álljon a kapu félfája mellé, és mikor a papok megáldozzák az õ égõáldozatát és hálaadóáldozatait, õ leborulva imádkozzék a kapu küszöbén, azután menjen ki, a kaput pedig ne zárják be estvéig.
\par 3 És leborulva imádkozzék az ország népe ugyanannak a kapunak bejáratánál a szombatokon és az újholdnak napjain az Úr elõtt.
\par 4 Az égõáldozat pedig, melyet a fejedelem vigyen az Úrnak szombatnapon, hat ép bárány és egy ép kos legyen;
\par 5 És az ételáldozat; egy éfa a kos mellé; és a bárányok mellé ételáldozatul, a mit keze adhat, s az olajból egy hín az éfához.
\par 6 Az újhold napján pedig egy ép, fiatal bika és hat bárány és egy kos, mind épek legyenek.
\par 7 És a bika mellé egy éfát és a kos mellé egy éfát tegyen ételáldozatul; és a bárányok mellé azt, a mi kezétõl telik, s az olajból egy hínt az éfához.
\par 8 És mikor bemegy a fejedelem, a kapu tornáczának útján menjen be, és ezen az úton menjen ki.
\par 9 Mikor pedig a föld népe megyen be az Úr eleibe az ünnepeken, a ki az északi kapu útján ment be, hogy leborulva imádkozzék, a déli kapu útján menjen ki; a ki pedig a déli kapu útján ment be, az az északi kapu útján menjen ki; ne térjen vissza azon kapu útjához, a melyen bement, hanem az annak ellenében valón menjen ki.
\par 10 A fejedelem pedig, mikor bemennek, közöttök menjen be, és mikor kimennek, együtt menjen ki velök.
\par 11 És az ünnepeken és a szent egybegyûléseken legyen az ételáldozat egy éfa egy bika mellé és egy éfa a kos mellé, és a bárányok mellé, a mit keze adhat, s az olajból egy hín az éfához.
\par 12 Továbbá, mikor a fejedelem szabad akaratból tesz égõáldozatot, vagy hálaadó áldozatokat, szabad akaratból az Úrnak, nyissák ki néki a kaput, mely napkeletre néz, és õ vigye égõáldozatát és hálaadó áldozatait, mint a hogy  szombat napon szokta tenni, és azután menjen ki, és zárják be a kaput kimenése után.
\par 13 És egy esztendõs ép bárányt áldozz égõáldozatul naponként az Úrnak; minden reggel áldozz azzal.
\par 14 És ételáldozatot tégy hozzá minden reggel: egy hatodrész éfát, és az olajból a hín harmadrészét a liszt megnedvesítésére, ételáldozatul az Úrnak; örökre állandó rendelések ezek.
\par 15 Hozzátok azért a bárányt és az ételáldozatot és az olajat minden reggel állandó égõáldozatul.
\par 16 Ezt mondja az Úr Isten: Ha a fejedelem ajándékot ad valamelyik fiának a maga örökségébõl, az az õ fiaié legyen tulajdonul örökségeképen.
\par 17 De ha örökségébõl valamelyik szolgájának ad ajándékot, az a szabadság esztendejéig lesz azé, azután visszaszáll a fejedelemre; csak az õ öröksége lesz az õ fiaié.
\par 18 És a fejedelem el ne vegyen a nép örökségébõl, hogy nyomorgatással kivesse õket tulajdonukból; a maga tulajdonából adjon örökséget fiainak, hogy az én népem közül senki el ne széledjen a maga tulajdonából.
\par 19 És bevitt engem ahhoz a bejárathoz, mely a kapu mellett oldalaslag vala, a kamarákhoz, a papok szenthelyéhez, melyek északra néznek, és ímé, ott egy hely vala leghátul nyugotra.
\par 20 És monda nékem: Ez a hely, a hol a papok fõzzék a vétekért és a bûnért való áldozatot, és a hol süssék az ételáldozatot, hogy ne kelljen kivinniök a külsõ pitvarba a nép megszentelésére.
\par 21 És kivitt engem a külsõ pitvarba, és elhordoza engem a pitvar négy szegletén, és ímé,a pitvar mindenik szegletében egy-egy pitvar vala.
\par 22 A pitvarnak négy szegletében zárt pitvarok valának, negyven sing hosszúságúak és harmincz sing szélesek; egy mértéke vala a négy szegleten való pitvaroknak.
\par 23 És falazások valának bennök köröskörül mind a négy körül, és a falazások alatt konyhák valának csinálva köröskörül.
\par 24 És monda nékem: Ez a fõzõház, a hol fõzzék a háznak szolgái a nép véres áldozatát.

\chapter{47}

\par 1 És visszatéríte engem a ház ajtajához, és ímé, víz jõ vala ki a ház küszöbe alól napkelet felé, mert a ház eleje kelet felé vala, és a víz aláfoly vala a ház jobb oldala alól az oltártól délre.
\par 2 És kivitt engem az északi kapu útján, és elhordoza engem a kivül való úton a külsõ kapuhoz, mely napkeletre néz, és ímé, a víz ott forr vala ki a jobb oldal alól.
\par 3 Mikor kiméne az a férfiú napkelet felé, mérõzsinórral a kezében, mére ezer singet; és átvitt engem a vizen, a víz bokáig ér vala.
\par 4 És mére ismét ezeret, és átvitt engem a vizen, a víz pedig térdig ér vala. És mére ismét ezeret és átvitt engem s a víz derékig ér vala.
\par 5 És mére még ezeret, s vala olyan folyó, hogy át nem meheték rajta, mert magas vala a víz, megúszni való víz, folyó, mely meg nem lábolható.
\par 6 És monda nékem: Láttad-é, embernek fia? És visszavezete engem a folyó partján.
\par 7 És mikor visszatértem, ímé, a folyó partján igen sok fa vala mindkét felõl.
\par 8 És mondá nékem: Ez a víz a keleti tájékra foly ki, és a lapáczra megyen alá, és a tengerbe megyen be, a tengerbe szakad, és meggyógyul a víz.
\par 9 És lészen, hogy minden élõ állat, a mely nyüzsög, valahova e folyam bemegyen, élni fog; és a halaknak nagy bõségök lészen, mert ez a víz bement oda, és azok meggyógyulnak, és él minden, valahova e folyó bement.
\par 10 És lészen, hogy halászok állanak rajta Éngeditõl Énegláimig: varsák kivetõ helye lészen; nemök szerint lesznek benne a halak, mint a nagy tenger halai, nagy bõséggel.
\par 11 Mocsarai és tócsái pedig nem gyógyulnak meg, só helyei lesznek.
\par 12 És a folyó mellett, mind a két partján mindenféle ennivaló gyümölcs fája nevekedik fel; leveleik el nem hervadnak és gyümölcseik el nem fogynak; havonként új meg új gyümölcsöt teremnek, mert vizök onnét a szenthelybõl  foly ki; és gyümölcsük eledelre és leveleik orvosságra valók.
\par 13 És monda az Úr Isten: Ez a határ, a mely szerint örökségül osszátok el magatok közt a földet Izráel tizenkét törzsöke szerint, Józsefnek két örökséget.
\par 14 És vegyétek azt örökségül mindnyájan egyenlõen, mivelhogy fölemeltem kezemet, hogy atyáitoknak adom azt, tehát essék néktek az a föld örökségül.
\par 15 És ez legyen a föld határa, észak felé a nagy-tengertõl, Hetlónon át Sedádnak menve:
\par 16 Hamát, Berótha, Szibraim, mely Damaskus határa és Hamát határa közt van, a középsõ Háser, mely Havrán határán van.
\par 17 Így legyen a határ a tengertõl Haczar-Énónig Damaskus határán és azontúl északra, és Hamát határán: ez az északi oldal.
\par 18 És a keleti oldalon: Havrán és Damaskus és Gileád között és Izráel földje között a Jordán legyen; ama határtól a keleti tengerig mérjétek: ez a keleti oldal.
\par 19 És a déli oldalon dél felé: Támártól a versengések vizéig Kádesben, azután a folyó felé a nagy-tengerig: ez a déli oldal dél felé.
\par 20 És a nyugoti oldal: a nagy tenger, ama határtól fogva addig, a hol egyenesen Hamátba menni: ez a nyugoti oldal.
\par 21 És oszszátok el ezt a földet magatok közt Izráel nemzetségei szerint.
\par 22 És legyen, hogy sorsvetéssel oszszátok el azt magatok közt és a jövevények közt, a kik közöttetek lakoznak, a kik közöttetek fiakat nemzettek, és úgy tartsátok õket, mint a ki ott született az Izráel fiai között: veletek együtt sorsvetéssel legyen örökségök Izráel nemzetségei között.
\par 23 És úgy legyen, hogy a mely nemzetséggel lakozik a jövevény, ott adjétok ki az õ örökségét. Ezt mondja az Úr Isten.

\chapter{48}

\par 1 Ezek a nemzetségek nevei: az északi határon a Hetlóntól Hamáthig vezetõ út mentén Haczar-Énonig Damaskus határán, észak felé, Hamát mentén, és pedig legyen az övé a keleti és nyugoti oldal: Dán, egy rész;
\par 2 És Dán határa mellett a keleti oldaltól a nyugoti oldalig: Áser, egy rész;
\par 3 És Áser határa mellett a keleti oldaltól a nyugoti oldalig: Nafthali, egy rész;
\par 4 És Nafthali határa mellett a keleti oldaltól a nyugoti oldalig: Manasse, egy rész;
\par 5 És Manesse határa mellett a keleti oldaltól a nyugoti oldalig: Efraim, egy rész;
\par 6 És Efraim határa mellett a keleti oldaltól a nyugoti oldalig: Rúben, egy rész;
\par 7 És Rúben határa mellett a keleti oldaltól a nyugoti oldalig: Júda, egy rész;
\par 8 És Júda határa mellett a keleti oldaltól a nyugoti oldalig legyen a szent áldozat, melyet az Úrnak szenteltek: huszonötezer sing széles és olyan hosszú, mint egy-egy rész a keleti oldaltól a nyugoti oldalig; és a szenthely annak közepette legyen;
\par 9 A szent áldozat, melyet az Úrnak szenteltek, huszonötezer sing hosszú és tízezer sing széles legyen;
\par 10 És ezeké legyen ez a szent áldozat: a papoké északra huszonötezer sing és nyugotra tízezer sing szélesség és keletre tízezer sing szélesség és délre huszonötezer sing hosszúság, és az Úr szenthelye annak közepette legyen;
\par 11 A papoké, a kik megszenteltettek a Sádók fiai közül, a kik szolgálatomban foglalatosak voltak, a kik nem tévelyedtek el, mikor Izráel fiai eltévelyedtek, mint a hogy eltévelyedtek volt a Léviták.
\par 12 Övék legyen ez, mint egy áldozati rész a föld áldozatából, mint igen szentséges, a Léviták határán.
\par 13 A Lévitáké pedig legyen a papok határa mentén huszonötezer sing hosszúság és tízezer szélesség; az egész hosszúság legyen huszonötezer és a szélesség tízezer.
\par 14 És semmit abból el ne adjanak, se el ne cseréljék, se másra át ne szálljon a földnek e zsengéje, mert az Úrnak szenteltetett.
\par 15 És az ötezer sing, mely a szélességben megmaradt a huszonötezernek mentén, közhely a város számára, mint lakóhely és tágasság, és legyen a város annak a közepében.
\par 16 És ezek legyenek annak méretei: az északi oldalon négyezerötszáz sing és a déli oldalon négyezerötszáz és a keleti oldalon négyezerötszáz és a nyugoti oldalon négyezerötszáz.
\par 17 És a város alatt legyen tágasság észak felé kétszázötven sing és dél felé kétszázötven és keletre kétszázötven és nyugotra kétszázötven.
\par 18 A mi pedig megmaradt a hosszaságban a szent áldozat mentén, tízezer keletre és tízezer nyugotra, az ott maradjon a szent áldozat mentén, és legyen annak termése a város szántóvetõinek eledele.
\par 19 És a ki szántóvetõ a városban, mívelje azt Izráelnek minden nemzetségébõl.
\par 20 Az egész huszonötezer sing szent áldozatot huszonötezer sing négyszögben adjátok szent ajándékul a város tulajdonával együtt.
\par 21 A mi pedig megmaradt, a fejedelemé lészen; a szent áldozatnak és a város tulajdonának mind a két oldalán, szemben a huszonötezer sing szent áldozattal a keleti határig és nyugot felé szemben a huszonötezer singgel egészen a nyugoti határig lészen, a kiosztottrészek mentén, a fejedelemé; és legyen a szent áldozat és a ház szenthelye annak közepette.
\par 22 És a Léviták tulajdonától és a város tulajdonától fogva, mely közepében van annak, a mi a fejedelemé, a Júda határa és Benjámin határa között a fejedelemé legyen.
\par 23 És a többi nemzetségek: a keleti oldaltól a nyugoti oldalig: Benjámin, egy rész.
\par 24 És Benjámin határán a keleti oldaltól a nyugoti oldaig: Simeon, egy rész.
\par 25 És Simeon határán a keleti oldaltól a nyugoti oldalig: Issakhár, egy rész.
\par 26 És Issakhár határán a keleti oldaltól a nyugori oldalig: Zebulon, egy rész.
\par 27 És Zebulon határán a keleti oldaltól a nyugori oldalig: Gád, egy rész.
\par 28 És Gád határán a déli oldalon dél felé, legyen a határ Támártól a versengések vizéig Kádesben, a patak felé a nagy-tengerig.
\par 29 Ez a föld, melyet sorsvetéssel örökségül eloszszatok Izráel nemzetségei közt, és ezek azoknak részei, ezt mondja az Úr Isten.
\par 30 És ezek a város külsõ részei: Az északi oldalon négyezerötszáz sing mérték.
\par 31 És a város kapui, Izráel nemzetségeinek nevei szerint, három kapu északra: Rúben kapuja egy, Júda kapuja egy, Lévi kapuja egy.
\par 32 És a keleti oldalon négyezerötszáz sing és három kapu: József kapuja egy, Benjámin kapuja egy, Dán kapuja egy.
\par 33 És a déli oldalon is négyezerötszáz sing mérték és három kapu: Simeon kapuja egy, Issakhár kapuja egy, Zebulon kapuja egy.
\par 34 A nyugoti oldalon négyezerötszáz sing három kapuval: Gád kapuja egy, Áser kapuja egy, Nafthali kapuja egy:
\par 35 Köröskörül tizennyolcezer sing; és a város neve ama naptól fogva: Ott lakik az Úr!


\end{document}