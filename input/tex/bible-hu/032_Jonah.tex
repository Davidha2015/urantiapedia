\begin{document}

\title{Jónás}


\chapter{1}

\par 1 És lõn az Úrnak szava Jónáshoz, az Amittai fiához, mondván:
\par 2 Kelj fel, menj Ninivébe, a nagy városba, és kiálts ellene, mert gonoszságuk felhatolt elémbe!
\par 3 És felkele Jónás, hogy Tarsisba szaladna az Úr elõl. Leméne azért Jáfóba, és talála ott egy hajót, a mely méne Tarsisba, és megadván a hajóbért, beszálla abba, hogy Tarsisba menne velök az Úr színe elõl.
\par 4 Az Úr pedig nagy szelet bocsáta a tengerre, és nagy vihar lõn a tengeren, és a hajó már-már töredezik vala.
\par 5 Megfélemlének azért a hajósok és kiáltának, kiki az õ istenéhez, és a hajóban lévõ holmit a tengerbe hányák, hogy könnyítsenek magukon. Jónás pedig leméne a hajó aljába, és lefeküdt és elaludt.
\par 6 De hozzáméne a kormányos mester, és mondá néki: Mi lelt, te nagy alvó? Kelj fel, kiálts a te Istenedhez; hát ha gondol velünk az Isten, és nem veszünk el!
\par 7 Egymásnak pedig ezt mondák: Jertek, vessünk sorsot, hogy megtudhassuk: mi miatt van rajtunk e veszedelem? És sorsot vetének, és a sors Jónásra esék.
\par 8 Mondák azért néki: Kérünk, beszéld el nékünk: mi miatt van rajtunk e veszedelem? Mi a te foglalkozásod és honnan jösz? Melyik a te hazád és miféle népbõl való vagy te?
\par 9 És monda nékik: Héber vagyok én, és az Urat, az egek Istenét félem én, a ki a tengert és a szárazt teremtette.
\par 10 És megfélemlének az emberek nagy félelemmel, és mondák néki: Mit cselekedtél? Mert megtudták azok az emberek, hogy az Úr színe elõl fut, mivelhogy elbeszélé nékik.
\par 11 Mondák azután néki: Mit cselekedjünk veled, hogy a tenger megcsendesedjék ellenünk? Mert a tenger háborgása növekedék.
\par 12 Õ pedig monda nékik: Fogjatok meg és vessetek engem a tengerbe, és megcsendesedik a tenger ellenetek; mert tudom én, hogy miattam van ez a nagy vihar rajtatok.
\par 13 És erõlködtek azok az emberek, hogy visszajussanak a szárazra; de nem tudtak, mert a tenger háborgása növekedék ellenök.
\par 14 Kiáltának azért az Úrhoz, és mondák: Kérünk Uram, kérünk, ne veszszünk el ez ember lelkéért, és ne háríts reánk ártatlan vért; mert te, Uram, úgy cselekedtél, a mint akartad!
\par 15 És felragadák Jónást és beveték õt a tengerbe, és megszûnék a tenger az õ háborgásától.
\par 16 Azok az emberek pedig nagy félelemmel félék az Urat, és áldozattal áldozának az Úrnak, és fogadásokat fogadának.
\par 17 Az Úr pedig egy nagy halat rendelt, hogy benyelje Jónást. És lõn Jónás a halnak gyomrában három nap és három éjjel.

\chapter{2}

\par 1 És könyörge Jónás az Úrnak, az õ Istenének a halnak gyomrából.
\par 2 És mondá: Nyomorúságomban az Úrhoz kiálték és meghallgata engem; a Seol torkából sikolték és meghallád az én szómat.
\par 3 Mert mélységbe vetettél engem, tenger közepébe, és körülfogott engem a víz; örvényeid és habjaid mind átmentek rajtam!
\par 4 És én mondám: Elvettettem a te szemeid elõl; vajha láthatnám még szentséged templomát!
\par 5 Körülvettek engem a vizek lelkemig, mély ár kerített be engem, hinár szövõdött fejemre.
\par 6 A hegyek alapjáig sülyedtem alá; bezáródtak a föld závárjai felettem örökre! Mindazáltal kiemelted éltemet a mulásból, oh Uram, Istenem!
\par 7 Mikor elcsüggedt bennem az én lelkem, megemlékeztem az Úrról, és bejutott az én könyörgésem te hozzád, a te szentséged templomába.
\par 8 A kik hiú bálványokra ügyelnek, elhagyják boldogságukat;
\par 9 De én hálaadó szóval áldozom néked; megadom, a mit fogadtam. Az Úré a szabadítás.
\par 10 És szóla az Úr a halnak, és kiveté Jónást a szárazra.

\chapter{3}

\par 1 És lõn az Úrnak szava Jónáshoz másodszor is, mondván:
\par 2 Kelj fel, menj Ninivébe, a nagy városba, és hirdesd néki azt a beszédet, a mit én parancsolok néked.
\par 3 És felkele Jónás, és elméne Ninivébe az Úr szava szerint. Ninive pedig nagy városa vala Istennek, három napi járó föld.
\par 4 És kezde Jónás bemenni a városba egy napi járóra, és kiálta és monda: Még negyven nap, és elpusztul Ninive!
\par 5 A niniveiek pedig hivének Istenben, és bõjtöt hirdetének, és nagyjaiktól fogva kicsinyeikig zsákba öltözének.
\par 6 És eljuta a beszéd Ninive királyához, és felkele királyi székébõl, és leveté magáról az õ királyi ruháját, és zsákba borítkozék, és üle a porba.
\par 7 És kiáltának és szólának Ninivében, a királynak és fõembereinek akaratából, mondván: Emberek és barmok, ökrök és juhok: semmit meg ne kóstoljanak, ne legeljenek és vizet se igyanak!
\par 8 Hanem öltözzenek zsákba az emberek és barmok, és kiáltsanak az Istenhez erõsen, és térjen meg kiki az õ gonosz útáról és az erõszakosságból, a mely az õ kezökben van!
\par 9 Ki tudja? talán visszatér és megengesztelõdik az Isten és elfordul haragjának búsulásától, és nem veszünk el!
\par 10 És látá Isten az õ cselekedeteiket, hogy megtértek az õ gonosz útjokról: és megbáná az Isten azt a gonoszt, a melyrõl mondá, hogy végrehajtja rajtok, és nem hajtá végre.

\chapter{4}

\par 1 És igen rossznak látszék ez Jónás elõtt, és megharaguvék.
\par 2 Könyörge azért az Úrhoz, és mondá: Kérlek, Uram! Avagy nem ez vala-é az én mondásom, mikor még az én hazámban valék? azért siettem, hogy Tarsisba futnék, mert tudtam, hogy te irgalmas és kegyelmes Isten vagy, nagy türelmû és nagy irgalmasságú és a gonosz miatt is bánkódó.
\par 3 Most azért Uram, vedd el, kérlek, az én lelkemet én tõlem, mert jobb meghalnom, mintsem élnem!
\par 4 Az Úr pedig mondá: Avagy méltán haragszol-é?
\par 5 Majd kiméne Jónás a városból, és üle a város keleti része felõl, és csinála ott magának hajlékot, és üle az alatt az árnyékban, a míg megláthatná, mi lészen a városból?
\par 6 Az Úr Isten pedig egy tököt rendele, és felnöve az Jónás fölé, hogy árnyékot tartson feje fölött és megoltalmazza õt a hévség bántásától. És nagy örömmel örvendezék Jónás a tök miatt.
\par 7 De másnapra férget rendele az Isten hajnal-költekor, és megszúrá az a tököt, és elszárada.
\par 8 És lõn napköltekor, hogy tikkasztó keleti szelet rendele Isten, és a nap rátûzött a Jónás fejére, és õ elbágyada. Kiváná azért magának a halált, és monda: Jobb halnom, mint élnem!
\par 9 És monda az Isten Jónásnak: Avagy méltán haragszol-é a tök miatt? És monda: Méltán haragszom, mind halálig!
\par 10 Az Úr pedig monda: Te szánod a tököt, a melyért nem fáradtál és a melyet nem neveltél, a mely egy éjjel támadt és más éjjel elveszett:
\par 11 Én pedig ne szánjam Ninivét, a nagy várost, a melyben több van tizenkétszer tízezer embernél, a kik nem tudnak különbséget tenni jobb- és balkezük között, és barom is sok van?!


\end{document}