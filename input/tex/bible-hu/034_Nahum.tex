\begin{document}

\title{Nahum}


\chapter{1}

\par 1 Ninive terhe; az elkosi Náhum látásának könyve.
\par 2 Buzgón szeretõ és bosszúálló Isten az Úr, bosszúálló az Úr, és telve haraggal; bosszút áll az Úr az õ ellenségein, és haragot tartó az õ gyûlölõi ellen.
\par 3 Hosszútûrõ az Úr és nagyhatalmú, és nem hagy büntetlenül. Szélvészben és viharban van az Úrnak útja, és lábainak pora a felhõ.
\par 4 Megfeddi a tengert és kiapasztja azt, és minden folyamot kiszáraszt. Elfonnyad a Básán és a Kármel, és Libánon virága elfonnyad.
\par 5 A hegyek reszketnek elõtte, és a halmok szétmállanak. Tekintetétõl megrendül a föld, és a világ, és minden, a mi rajta él.
\par 6 Ki állhatna meg haragja elõtt, és ki birhatná ki búsulásának tüzét? Heve szétfoly, mint a láng, és a kõszálak is szétporlanak tõle.
\par 7 Jó az Úr, erõsség a szorongatás idején,  és õ ismeri a benne bízókat.
\par 8 De gáttörõ árvízként pusztít annak helyén, és gyûlölõit setétség üldözi.
\par 9 Mit koholtok az Úr ellen? Elpusztít õ, nem lészen kétszer veszedelem.
\par 10 Ha annyira összefonódnak is, mint a tüskebokrok, és olyan ázottak is, mint az italuk: megemésztetnek, mint a teljesen megszáradt tarló.
\par 11 Belõled származott, a ki gonoszt koholt az Úr ellen, a ki álnokságot tanácsolt.
\par 12 Így szól az Úr: Ha teljes erõben és sokan vannak is, mégis levágatnak és elenyésznek. Megaláztalak téged, de nem foglak többé megalázni.
\par 13 Most már leveszem rólad az õ igáját, és bilincseidet leszaggatom.
\par 14 Felõled pedig azt rendeli az Úr: Nevednek ne támadjon többé magva; isteneid házából kivesztem a faragott és öntött képeket; megásom sírodat, mert becstelen vagy.
\par 15 Ímé a hegyeken örömhírhozónak lábai! Békességet hirdet. Ünnepeld Júda ünnepeidet, fizesd le fogadásaidat; mert nem vonul át rajtad többé a  semmirekellõ; mindenestõl kiirtatott.

\chapter{2}

\par 1 Pusztító jön fel ellened; õrizd a várat, nézzed az útat, erõsítsd derekadat, keményítsd meg erõdet igen.
\par 2 Mert helyreállítja az Úr Jákób büszkeségét, mint Izráel büszkeségét; mert rablók rabolták ki õket, csemetéiket pedig kivágták.
\par 3 Vitézeinek pajzsa veres, katonáinak ruházata bíborszínû, hadiszekere aczéltûzben ragyog fegyverkezése napján, és a dárdákat rengetik.
\par 4 Az utczákon robognak a szekerek, összeütköznek a piaczokon; tekintetök mint a fáklyák, futkosnak mint a villámok.
\par 5 Emlegeti vitézlõ hõseit; ingadoznak lépéseikben, sietnek a kõfalra, és felállíttatik a védõsáncz.
\par 6 A folyóvizek kapui megnyílnak, és a palota megrendül.
\par 7 De elvégeztetett: felfedik, elvitetik, és szolgálói keseregnek, mint nyögõ galambok, mellöket vervén.
\par 8 Pedig Ninive olyan, mint a bõvizû tó, eleitõl fogva: mégis futnak õk. Álljatok meg, álljatok meg! De vissza se tekint senki.
\par 9 Raboljatok ezüstöt, raboljatok aranyat; száma sincs a rejtett kincseknek, gazdag minden drága edényben.
\par 10 Feldúlva, széthányva, kifosztatva! Szíve megolvadt, a térdek reszketnek, fájdalom van egész derekában, és mindnyájok arcza elvesztette pirosságát.
\par 11 Hol van az oroszlánok tanyája, és az oroszlán-kölyköknek ama legelõje, a hová járt a nõstény- és hímoroszlán, az oroszlán-kölyök, és nem volt a ki elriaszsza?
\par 12 Az oroszlán torkig valót ragadozott kölykeinek, és fojtogatott nõstényeinek, és megtöltötte barlangjait zsákmánynyal, tanyáit pedig prédával.
\par 13 Ímé rád török, azt mondja a Seregek Ura, és füstté égetem szekereit. Oroszlán-kölykeidet kard emészti meg, és kiirtom e földrõl zsákmányodat, és nem hallatszik többé követeidnek szava.

\chapter{3}

\par 1 Jaj a vérszopó városnak! Mindenestõl hazug és erõszakkal telve, és nem szûnik meg rabolni.
\par 2 Ostor-csattogás, kerék-zörgés zaja; dobogó ló, robogó szekér;
\par 3 Törtetõ lovag, kardok villogása, dárda villanása, sebesült tömegek, holtak sokasága, nincs számok az elesetteknek; megbotlanak hulláikban.
\par 4 A szép parázna sok paráznaságáért, a hitetésnek mesternõje miatt, a ki népeket ejtett meg paráznaságával, és nemzetségeket bûbájaival:
\par 5 Ímé, rád török, azt mondja a Seregek Ura, és orczádra fordítom ruhádnak alját, és népeknek mutatom meg meztelenségedet, és országoknak gyalázatodat.
\par 6 Rútságot hányatok rád, és gyalázattal illetlek téged, és olyanná teszlek, mint a kit csudálnak.
\par 7 És mind, a ki meglát, elmenekül tõled, és ezt mondja: Epusztult Ninive! Ki bánkódik rajta? Hol keressek néked vígasztalókat?
\par 8 Avagy jobb vagy-é Nó-Amonnál, a mely a folyamoknál fekszik, vizek veszik körül; a melynek tenger a sáncza, tenger a kõfala?
\par 9 Kús volt erõssége meg Égyiptom, és száma sem volt annak. Puth és Libia is segítõid voltak;
\par 10 De ez is számkivetésbe, fogságba jutott; kisdedeik is falhoz verettek minden utcza sarkán; fõembereire sorsot vetettek, és nagyjait mind bilincsekbe verték.
\par 11 Te is megrészegedel, elfeledtté leszel; te is keresel majd menedéket a gyûlölködõ elõl.
\par 12 Minden erõsséged olyan, mint a zsenge gyümölcsû fügefa; ha megrázatnak, az evõ szájába hullnak.
\par 13 Ímé, a te néped asszonynép te benned, földednek kapui tárva kitárulnak gyûlölõidnek, tûz emészti meg záraidat!
\par 14 Meríts magadnak ostromhoz való vizet, javítsd erõsségeidet; menj be a sárba, taposd az agyagot, javítsd a tégla-vetõt!
\par 15 Legott tûz emészt meg téged, fegyver irt ki téged, megemészt, mint a szöcske; szaporodjál bár, mint a szöcske, szaporodjál bár mint a sáska!
\par 16 Többen voltak kalmáraid, mint az égnek csillagai: a szöcske csapong és elrepül!
\par 17 Fejedelmeid mint a sáska, vezéreid mint a tücsök-raj; hideg idõkben gyepûkben tanyáz, napkeletkor pedig elrepül, és a helye sem tudható meg, hol volt.
\par 18 Szunnyadoznak pásztoraid, Assiria királya, feküsznek vitézlõ hõseid; néped a hegyeken széledez, és nincsen, a ki összegyûjtse.
\par 19 Nincs enyhítés a te sebedre, gyógyíthatatlan a te nyavalyád. A kik híredet hallják, mind tapsolnak feletted, mert kire nem hatott volna ki a te gonoszságod soha?


\end{document}