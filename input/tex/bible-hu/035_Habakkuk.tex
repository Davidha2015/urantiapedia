\begin{document}

\title{Habakuk}


\chapter{1}

\par 1 A teher, a melyet Habakuk próféta látott.
\par 2 Meddig kiáltok még oh Uram, és nem hallgatsz meg! Kiáltozom hozzád az erõszak miatt, és nem szabadítasz meg!
\par 3 Miért láttatsz velem hamisságot, és szemléltetsz nyomorgatást? Pusztítás és erõszak van elõttem, per keletkezik és versengés támad!
\par 4 Azért inog a törvény, és nem érvényesül az igaz ítélet; mert gonosz hálózza be az igazat, azért származik hamis ítélet!
\par 5 Nézzetek szét a népek között, vizsgálódjatok és csodálkozással csodálkozzatok, mert oly dolgot cselekszem a ti napjaitokban, mit el sem hinnétek, ha beszélnék!
\par 6 Mert ímé, feltámasztom a Káldeusokat, a kegyetlen és vakmerõ nemzetet, a mely eljárja a földet széltében, hogy hajlékokat foglaljon el, a mely nem az övéi.
\par 7 Rettenetes és iszonyatos ez, maga szerzi törvényét és hatalmát.
\par 8 És lovai serényebbek a párduczoknál, és gyorsabbak az estveli farkasoknál, és elõtörtetnek az õ lovasai; és az õ lovasai messzirõl jõnek, repülnek, mint a zsákmányra sietõ keselyû.
\par 9 Mindnyája ragadományért jön, arczuk elõre néz, és annyi foglyot gyûjt, mint a föveny.
\par 10 Kaczag ez a királyokon, és a fejedelmek néki nevetség, minden erõsséget csak nevet, tültést emel és megostromolja azt.
\par 11 Majd tovaszáll viharként és elvonul és bûnbe esik; õ kinek istene az õ hatalma.
\par 12 Avagy nem te vagy-é Uram, öröktõl fogva az én Istenem, Szentem? Nem veszünk el! Ítéletre rendelted õt, oh Uram, fenyítõül választottad õt, én erõsségem!
\par 13 Tisztábbak szemeid, hogysem nézhetnéd a gonoszt, és a nyomorgatást nem szemlélheted: miért szemléled hát a hitszegõket? és hallgatsz, mikor a gonosz elnyeli a nálánál igazabbat?!
\par 14 Olyanokká teszed az embert, mint a tenger halai, és mint a csúszómászó állatok, a melyeknek nincsen vezérök?
\par 15 Mindnyáját kivonsza horoggal, gyalomjába keríti, és hálójába takarítja be õket; ezért örül és vígad.
\par 16 Ezért áldozik gyalomjának, és füstöl az õ hálójának, mert ezekkel kövér az õ része, és zsíros az õ eledele.
\par 17 Vajjon azért ürítheti-é gyalmát, és szüntelen ölheti-é a nemzeteket kímélet nélkül?!

\chapter{2}

\par 1 Õrhelyemre állok, és megállok a bástyán, és vigyázok, hogy lássam, mit szól hozzám, és mit feleljek én panaszom dolgában.
\par 2 És felele nékem az Úr, és mondá: Írd fel e látomást, és vésd táblákra, hogy könnyen olvasható legyen.
\par 3 Mert e látomás bizonyos idõre szól, de vége felé siet és meg nem csal; ha késik is, bízzál benne; mert eljön, el fog jõni, nem marad el!
\par 4 Ímé, felfuvalkodott, nem igaz õ benne az õ lelke; az igaz pedig az õ hite által él.
\par 5 És bizony, a bor is megcsal: felgerjed a férfiú és nincsen nyugalma; õ, a ki tátja száját, mint a Seol, és olyan õ, mint a halál: telhetetlen, és magához ragad minden népet és magához csatol minden nemzetet.
\par 6 Avagy nem költenek-é ezek mindnyájan példabeszédet róla, és találós mesét reá, mondván:
\par 7 Avagy nem támadnak-é hirtelen, a kik téged mardossanak, és nem serkennek-é fel, a kik háborgassanak téged? És zsákmányul esel nékik.
\par 8 Mivelhogy kifosztogattál sok nemzetet, kifosztanak téged mind a többi népek az emberek véréért, és az országok, városok és minden bennök lakozók ellen való erõszaktételért.
\par 9 Jaj annak, a ki bûnös szerzeményt szerez házának, hogy magasra rakhassa fészkét, hogy megszabadulhasson a gonosz hatalma elõl.
\par 10 Házadnak gyalázatára tervezted a sok nép kiirtását, és bûnössé tetted lelkedet.
\par 11 Mert a kõ is ellened kiált a falból, és a gerenda a fa-alkotmányból visszhangoz néki.
\par 12 Jaj annak, a ki várost épít vérengzéssel, és a ki várat emel álnoksággal.
\par 13 Avagy ímé, nem a Seregek Urától van-é ez, hogy a népek tûznek építenek, és a nemzetek a hiábavalóságnak fáradoznak?
\par 14 Mert az Úr dicsõségének ismeretével betelik a föld, a miképen a folyamok megtöltik a tengert.
\par 15 Jaj annak, a ki megitatja felebarátját, epédet keverve belé, hogy megrészegítsed õt, hogy láthassad az õ szemérmöket!
\par 16 Gyalázattal telsz meg dicsõség helyett; igyál te is és láttassék szemérmed; reád fordul az Úr jobbjának pohara, és gyalázat borítja el dicsõségedet.
\par 17 Bizony, a libanoni erõszakoskodás gyászba borít téged, és a vadak pusztítása, a mely rettegteté õket, az emberek véréért és az országon, a városon és annak minden lakosán ûzött erõszakosságért.
\par 18 Mit használ a faragott kép, hogy a faragója kifaragta azt? vagy az öntött kép és a mely hazugságot tanít, hogy a képnek faragója bízik abban, csinálván néma bálványokat?
\par 19 Jaj annak, aki fának mondja: Serkenj fel! néma kõnek: Ébredj fel! Taníthat-é ez? Ímé, borítva van aranynyal és ezüsttel, lélek pedig nincs benne semmi!
\par 20 Ellenkezõleg az Úr az õ szent templomában, hallgasson elõtte az egész föld!

\chapter{3}

\par 1 Habakuk próféta könyörgése a sigjónóth szerint.
\par 2 Uram, hallám, a mit hirdettél, és megrettenék! Uram! Évek közepette keltsd életre a te munkádat, évek közepette jelentsd meg azt! Haragban emlékezzél meg kegyelmességrõl!
\par 3 Isten a Témán felõl jön, és a Szent a Párán hegyérõl. Szela. Dicsõsége elborítja az egeket, és dícséretével megtelik a föld.
\par 4 Ragyogása, mint a napé, sugarak támadnak mellõle; és ott van az õ hatalmának rejteke.
\par 5 Elõtte döghalál jár, és nyomaiban forró láz támad.
\par 6 Megáll és méregeti a földet, pillant és megrendíti a népeket, az örökkévaló hegyek szétporlanak, elsüllyednek az örökkévaló halmok; az õ ösvényei örökkévalók!
\par 7 Bomlani látom Khusán sátrait, reszketnek a Midián-föld kárpitjai!
\par 8 A folyók ellen gerjedt-é fel az Úr? Vajjon a folyókra haragszol-é, vagy a tengerre bõszültél-é fel, hogy lovaidon és diadal-szekereiden robogsz?
\par 9 Csupasz, meztelen a te kézíved, a törzseknek esküvéssel tett igéret szerint! Szela. A föld folyókat ömleszt.
\par 10 Látnak téged és megrendülnek a hegyek, gátat tör a víz-ár, harsog a hullám, és magasra emeli karjait.
\par 11 A nap és hold megállnak helyökön czikázó nyilaid fényétõl és ragyogó kopjád villanásától.
\par 12 Haragodban eltaposod a földet, búsultodban szétmorzsolod a nemzeteket.
\par 13 Kiszállsz néped szabadítására, fölkented segítségére; szétzúzod a fõt a gonosznak házában; nyakig feltakarod az alapjait. Szela!
\par 14 Saját dárdájával vered át az õ vezéreinek fejét, a kik berohannak, hogy szétszórjanak engem; ujjonganak, hogy rejtekében emészthetik meg a szegényt.
\par 15 Lovaiddal megtaposod a tengert, a nagy vizek hullámait.
\par 16 Hallám és reszket a bensõm, a szózatra remegnek ajkaim; porladni kezdenek csontjaim, reszketnek lábaim: hogy nyugton legyek a nyomorúság napján, a mely feljön a népre, mely megsanyargatja azt.
\par 17 Mert a fügefa nem fog virágozni, a szõlõkben nem lészen gyümölcs, megcsal az olajfa termése, a szántóföldek sem teremnek eleséget, kivész a juh az akolból, és nem lesz ökör az istállóban.
\par 18 De én örvendezni fogok az Úrban, és vígadok az én szabadító Istenemben.
\par 19 Az Úr Isten az én erõsségem, hasonlókká teszi lábaimat a nõstény szarvasokéihoz, és az én magas helyeimen jártat engemet!


\end{document}