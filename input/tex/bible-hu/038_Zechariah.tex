\begin{document}

\title{Zakariás}


\chapter{1}

\par 1 Dárius második esztendejében, a nyolczadik hónapban szóla az Úr Zakariáshoz, a Berekiás fiához, a ki Iddó próféta fia, mondván:
\par 2 Igen megharagudott az Úr a ti atyáitokra.
\par 3 Mondjad azért nékik: Ezt mondja a Seregeknek Ura: Térjetek hozzám, szól a Seregeknek Ura, és hozzátok térek, mond a Seregeknek Ura.
\par 4 Ne legyetek olyanok, mint atyáitok, a kikhez az elébbi próféták kiáltottak, mondván: Ezt mondja a Seregeknek Ura: Térjetek meg kérlek a ti gonosz útaitokról, és a ti gonosz cselekedeteitekbõl, de nem hallgattak meg, és nem figyelmeztek reám, szól az Úr.
\par 5 Atyáitok? Hol vannak õk? És a próféták örökké élnek-é?
\par 6 De az én beszédeim és végzéseim, a melyeket szolgáim, a próféták által hirdettem: nem beteljesedtek-é a ti atyáitokon? És megtértek és azt mondták: A mint elhatározta vala a Seregeknek Ura, hogy a mi útaink és cselekedeteink szerint bánik velünk: úgy bánt velünk.
\par 7 A tizenegyedik hónapnak, azaz a Sebat hónapnak huszonnegyedik napján, Dáriusnak második esztendejében, szóla az Úr Zakariáshoz, a Berekiás fiához, a ki Iddó próféta fia, mondván:
\par 8 Látám éjszaka, hogy ímé, egy férfiú veres lovon ül vala, és áll vala a mirtus-fák között, a melyek egy árnyas völgyben valának; háta megett pedig veres, tarka és fehér lovak.
\par 9 És mondám: Mik ezek Uram? És mondá nékem az angyal, aki beszél vala nékem: Én megmutatom néked: mik ezek.
\par 10 Akkor felele az a férfiú, a ki a mirtus-fák között áll vala, és mondá: Azok ezek, a kiket az Úr küldött, hogy járják be e földet;
\par 11 És felelének az Úr angyalának, a ki a mirtus-fák között áll vala, és mondák: Bejártuk a földet, és ímé. az egész föld vesztegel és nyugodt.
\par 12 Az Úr angyala pedig felele, és mondá: Seregeknek Ura! Meddig nem könyörülsz még Jeruzsálemen és Júdának városain, a melyekre haragszol immár hetven esztendõ óta?
\par 13 És nyájas szavakkal, vígasztaló szókkal felele az Úr az angyalnak, a ki beszél vala velem.
\par 14 És mondá nékem az angyal, a ki beszél vala velem: Kiálts, ezt mondván: Ezt mondja a Seregeknek Ura: szeretem Jeruzsálemet és a Siont nagy szeretettel!
\par 15 De nagy haraggal haragszom és a hivalkodó népekre, a kik kevéssé haragudtam ugyan, de õk gonoszra törtek.
\par 16 Azt mondja azért az Úr: könyörületességgel fordulok Jeruzsálemhez; benne építtetik meg az én házam, szól a Seregeknek Ura, és mérõzsinór nyujtatik ki Jeruzsálem felett.
\par 17 Mégis kiálts, mondván: Ezt mondja a Seregeknek Ura: Bõvelkedni fognak még városaim a jóban, mert megvígasztalja még az Úr a Siont, és magáévá fogadja még Jeruzsálemet!
\par 18 Majd felemelém szemeimet, és ímé, négy szarvat láték.
\par 19 És mondám az angyalnak, a ki beszél vala velem: Mik ezek? És monda nékem: Ezek azok a szarvak, a melyek szétszórták Júdát, Izráelt és Jeruzsálemet.
\par 20 Azután mutata nékem az Úr négy mesterembert.
\par 21 És mondám: Mit jöttek ezek cselekedni? Õ pedig szóla, mondván: Ezek azok a szarvak, a melyek szétszórták Júdát annyira, hogy senki sem emelheti vala fel fejét: de eljöttek, ezek, hogy elrettentsék õket, hogy letörjék a pogányok szarvait, a ki szarvakkal támadtak vala Júda földe ellen, hogy szétszórják azt.

\chapter{2}

\par 1 Felemelém ismét szemeimet, és ímé, láték egy férfiút és a kezében mérõ-kötelet.
\par 2 És mondám: Hová megy te? És mondá nékem: Megmérni Jeruzsálemet, hogy lássam: mennyi a széle és mennyi a hossza?
\par 3 És ímé, az angyal, a ki beszél vala velem, kijöve, és más angyal is kijöve eléje.
\par 4 És monda annak: Fuss, és szólj e gyermekhez, mondván: Kerítetlenül fogják lakni Jeruzsálemet a benne lévõ emberek és barmok sokasága miatt.
\par 5 Én pedig, szól az Úr, tûz-fal leszek körülötte és megdicsõítem magamat õ benne!
\par 6 Jaj, jaj! Fussatok ki az északi földrõl, így szól az Úr, mert az ég négy szele felé szórtalak szét titeket, szól az Úr.
\par 7 Jaj Sion! Szabadíts ki magadat, ki Babilon leányánál lakozol.
\par 8 Mert így szól a Seregeknek Ura: Dicsõség után küldött engem a pogányokhoz, a kik fosztogatnak titeket, mert a ki titeket bánt, az õ szemefényét bántja.
\par 9 Mert ímé én felemelem kezemet ellenök, és saját szolgáik prédájává lesznek, és megtudjátok, hogy a Seregeknek Ura küldött engem.
\par 10 Örülj és örvendezz, Sionnak leánya, mert ímé elmegyek és közötted lakozom! így szól az Úr.
\par 11 És sok pogány csatlakozik azon a napon az Úrhoz, és népemmé lesznek, és közötted lakozom, és megtudod, hogy a Seregeknek Ura küldött hozzád engem.
\par 12 És birtokba veszi az Úr Júdát, mint az õ osztályrészét a szent földön, és újra magáévá fogadja Jeruzsálemet.
\par 13 Hallgasson minden test az Úr elõtt: mert felkelt az õ szentséges helyérõl.

\chapter{3}

\par 1 Azután megmutatá nékem Jósuát, a fõpapot, a ki az Úr angyala elõtt álla, és a Sátánt, a ki jobb keze felõl álla, hogy vádolja õt.
\par 2 És mondá az Úr a Sátánnak: Dorgáljon meg téged az Úr, te Sátán; dorgáljon meg az Úr, a ki magáévá fogadja Jeruzsálemet. Avagy nem tûzbõl kikapott üszög-é ez?
\par 3 Jósua pedig szennyes ruhába vala öltöztetve, és áll vala az angyal elõtt.
\par 4 És szóla és monda az elõtte állóknak, mondván: Vegyétek le róla a szennyes ruhákat! És monda néki: Lásd! Levettem rólad a te álnokságodat, és ünnepi ruhákba öltöztetlek téged!
\par 5 Azután mondám: Tegyenek fejére tiszta süveget! Feltevék azért fejére a tiszta süveget, és ruhákba öltözteték õt, az Úrnak angyala padig ott áll vala.
\par 6 És bizonyságot tõn az Úrnak angyala Jósuának, mondván:
\par 7 Ezt mondja a Seregeknek Ura: Ha az én útaimban jársz, és ha parancsolataimat magtartod: te is ítélõje leszel az én házamnak, sõt õrizni fogod az én pitvaraimat, és ki- s bejárást engedek néked ez itt állók között.
\par 8 Halld meg Jósua, te fõpap; te és barátaid, a kik elõtted ülnek, mert jelképes férfiak ezek: Ímé, bizony elõhozom az én szolgámat, Csemetét!
\par 9 Mert ímé e kõ az, a melyet Jósua elé helyezek; egy kövön hét szem; ímé, én faragom annak faragványait, így szól a Seregeknek Ura, és eltörlöm e földnek álnokságait egy napon.
\par 10 Azon a napon, így szól a Seregeknek Ura, kiki hívja majd a maga felebarátját a szõlõtõ alá és a fügefa alá.

\chapter{4}

\par 1 Majd visszatére az angyal, a ki beszél vala velem, és felkölte engem, mint mikor valaki álmából költetik fel.
\par 2 És mondá nékem: Mit látsz te? És mondám: Látok ímé egy merõ arany gyertyatartót, tetején az olajtartója, rajta pedig annak hét szövétneke, és hét csõ a szövétnekekhez, a melyek a tetején vannak;
\par 3 És mellette két olajfa: egyik az olajtartó jobb oldalán, a másik pedig annak bal oldalán.
\par 4 És felelék, és mondám az angyalnak, a ki beszél vala velem, mondván: Mik ezek, Uram?
\par 5 És felele az angyal, a ki beszél vala velem, és mondá nékem: Hát nem tudod-é, mik ezek? És mondám: Nem, Uram!
\par 6 És felele, és szóla nékem, mondván: Az Úrnak beszéde ez Zorobábelhez, mondván: Nem erõvel, sem hatalommal, hanem az én lelkemmel! azt mondja a Seregeknek Ura.
\par 7 Ki vagy te, te nagy hegy? Lapálylyá leszel Zorobábel elõtt, és felviszi a csúcs-követ, és ilyen kiáltás támad: Áldás, áldás reá!
\par 8 És szóla hozzám az Úr, mondván:
\par 9 A Zorobábel kezei veték meg e ház alapját, és az õ kezei végzik el azt, és megtudod, hogy a Seregeknek Ura küldött el engem hozzátok.
\par 10 Mert a kik csúfolták a kicsiny kezdetet, örülni fognak, ha meglátják Zorobábel kezében az ónkövet. Hét van ilyen,  az Úrnak szemei ezek, a melyek átpillantják az egész földet.
\par 11 És felelék, és mondám néki: Mi ez a két olajfa a gyartyatartó jobb és bal oldalán?
\par 12 És másodszor is felelék, és mondám néki: Micsoda az olajfának az a két ága, a melyek a két arany csõ mellett vannak, és öntik magukból az aranyat?
\par 13 És szóla nékem, mondván: Hát nem tudod-é, mik ezek? És mondám: Nem, Uram!
\par 14 És mondá nékem: Ezek ketten az olajjal felkenettek, a kik az egész föld Ura mellett állnak.

\chapter{5}

\par 1 Azután ismét felemelém szemeimet, és látám, hogy ímé, egy könyv repül vala.
\par 2 És monda nékem: Mit látsz te? És én mondám: Látok egy repülõ könyvet, húsz sing a hossza és tíz sing a széle.
\par 3 És monda nékem: Ez az átok, a mely kihat az egész föld színére; mert mindaz, a ki lop, ehhez képest, fog innen kiirtatni, és mindaz, a ki hamisan esküszik, ehhez képest fog innen kiirtatni.
\par 4 Kibocsátom ezt, szól a Seregeknek Ura, és bemegy a lopónak házába, és annak házába, a ki hamisan esküszik az én nevemre, és ott marad annak házában, és megemészti azt, s annak fáit és köveit.
\par 5 Majd kijöve az angyal, a ki beszél vala velem, és monda nékem: Emeld csak fel szemeidet, és lásd meg: micsoda az, a mi kijön?
\par 6 És mondám: Micsoda ez? Õ pedig mondá: Az, a mi kijön, mérõedény. És mondá: Ilyen a formájok az egész földön.
\par 7 És ímé, egy kerek ón-darab repül vala, és ül vala egy asszony a mérõ-edény közepében.
\par 8 És mondá: Ez az istentelenség. És veté ezt a mérõ-edény közepébe, a darab ónt pedig veté annak szájára.
\par 9 És felemelém szemeimet, és látám, hogy ímé, két asszony jöve elõ, és szél vala szárnyaikban, és szárnyaik olyanok, mint az eszterágnak szárnyai, és felemelék a mérõ-edényt a föld és az ég közé.
\par 10 És mondám az angyalnak, a ki beszél vala velem: Hová viszik ezek a mérõ-edényt?
\par 11 És mondá nékem: Hogy házat építsenek annak a Sineár földén, és oda erõsítsék, és ott hagyják azt a maga helyén.

\chapter{6}

\par 1 Majd megfordulék és felemelém szemeimet, és látám, hogy ímé négy szekér jön vala ki két hegy közül, és azok a hegyek ércz-hegyek.
\par 2 Az elsõ szekérben veres lovak, a második szekérben fekete lovak;
\par 3 És a harmadik szekérben fehér lovak; a negyedik szekérben pedig tarka lovak, erõsek.
\par 4 És megszólalék, és mondám az angyalnak, a ki beszél vala velem: Mik ezek, Uram?
\par 5 Az angyal felele, és mondá nekem: Ezek az égnek négy szele, jõnek az egész föld Ura mellett való szolgálatukból.
\par 6 A melyben a fekete lovak vannak, észak földére mennek, és a fehérek mennek utánok, a tarkák pedig a dél földére mennek;
\par 7 Az erõsek is mennek és kivánják eljárni a földet. És mondá: Menjetek, járjátok el a földet; és eljárák a földet.
\par 8 Majd híva engem, és beszéle velem, mondván: Lásd! Az észak földére menõk lecsendesíték lelkemet északnak földén.
\par 9 És szóla az Úr hozzám, mondván:
\par 10 Végy a számkivetésbõl valóktól: Heldaitól, Tóbiástól és Jedajától; és menj be azon a napon, menj be Jósiának, a Sefániás fiának házába, a kik Babilonból jöttenek,
\par 11 Végy ugyanis ezüstöt és aranyat, csinálj koronákat, és tedd Jósuának, a Jehosadák fiának, a fõpapnak fejére!
\par 12 És szólj néki, mondván: Ezt mondja a Seregeknek Ura, mondván: Ímé, egy férfiú, a neve Csemete, mert csemete támad belõle, és megépíti az Úrnak templomát!
\par 13 Mert õ fogja megépíteni az Úrnak templomát, és nagy lesz az õ dicsõsége, és ülni és uralkodni fog az õ székében, és pap is lesz az õ székében, és békesség tanécsa lesz kettejük között.
\par 14 És a koronák legyenek Hélemnek, Tóbiásnak, Jedajának és Hénnek, a Sefániás fiának emlékjelei az Úr templomában.
\par 15 És a messzelakók eljõnek és építenek az Úr templomában, és megtudjátok, hogy a Seregeknek Ura küldött el engem hozzátok. Így lesz, ha hallgattok az Úrnak, a ti Isteneteknek szavára!

\chapter{7}

\par 1 És lõn a Dárius király negyedik esztendejében, hogy szóla az Úr Zakariáshoz a kilenczedik hónapnak, a Kiszlévnek negyedikén,
\par 2 Mikor elküldék az Isten házába Saréczert és Régem-Méleket és társait, hogy esedezzenek az Úr színe elõtt,
\par 3 És hogy megkérdezzék a papokat, a kik a Seregeknek Urának házában vannak, és megkérdezzék a prófétákat is: Sírjak-é az ötödik hónapban és bõjtöljek-é, a mint cselekedtem azt néhány esztendõ óta?
\par 4 Szóla ekkor a Seregeknek Ura nékem, mondván:
\par 5 Szólj az ország minden népének és a papoknak, mondván: Mikor bõjtöltetek és gyászoltatok az ötödik és hetedik hónapban, és pedig hetven esztendeig: avagy bõjtölvén, nékem bõjtöltetek-é?
\par 6 És mikor ettetek, és mikor ittatok: avagy nem magatoknak ettetek és magatoknak ittatok-é?
\par 7 Avagy nem ezek a beszédek-é azok, a melyeket szólott vala az Úr az elõbbi próféták által, mikor még Jeruzsálem népes és gazdag vala a körülte levõ városokkal együtt, és mind a déli táj, mind a lapály-föld népes vala?
\par 8 És szóla az Úr Zakariásnak, mondván:
\par 9 Így szólt a Seregeknek Ura, mondván: Igaz ítélettel ítéljetek, és irgalmasságot és könyörületességet gyakoroljon kiki az õ felebarátjával!
\par 10 Özvegyet és árvát, jövevényt és szegényt meg ne sarczoljatok, és egymás ellen még szívetekben se gondoljatok gonoszt.
\par 11 De nem akarák meghallani, sõt vállaikat vonogaták, és bedugák füleiket, hogy ne halljanak.
\par 12 Szívöket is megkeményíték, hogy ne hallják a törvényt és az igéket, a melyeket a Seregeknek Ura küldött vala az õ lelke által, az elébbi próféták által. És igen felgerjedt vala a Seregeknek Ura.
\par 13 És lõn, hogy a mint én kiáltottam és nem hallották meg: úgy kiáltottak, de nem hallottam meg, azt mondja a Seregeknek Ura;
\par 14 Hanem szétszórtam õket mindenféle nemzetek közé, a kik nem ismerték õket, és puszta lõn utánok a föld, hogy senki azon sem át nem megy, sem meg nem tér. Így tevék pusztává a kívánatos földet.

\chapter{8}

\par 1 Majd szóla a Seregeknek Ura, mondván:
\par 2 Ezt mondja a Seregeknek Ura: Nagy gerjedezéssel gerjedeztem a Sionért, és nagy haragra gerjedtem ellene.
\par 3 Ezt mondja az Úr: Megtértem a Sionhoz, és Jeruzsálem közepette lakozom, és Jeruzsálem igazság városának neveztetik, a Seregeknek Urának hegye pedig szent hegynek.
\par 4 Ezt mondja a Seregeknek Ura: Agg férfiak és agg nõk ülnek majd Jeruzsálem utczáin, és kinek-kinek pálcza lesz kezében a napok sokasága miatt.
\par 5 És megtelnek a város utczái fiúkkal és leányokkal, a kik játszadoznak annak utczáin.
\par 6 Ezt mondja a Seregeknek Ura: Ha ez csoda lészen e nép maradékának szemei elõtt azokban a napokban, vajjon az én szemeim elõtt is csoda lészen-é? így szól a Seregeknek Ura.
\par 7 Ezt mondja a Seregeknek Ura: Ímé, én megszabadítom az én népemet a nap keltének földérõl és a nap nyugtának földérõl.
\par 8 És elhozom õket és Jeruzsálemben lakoznak, és népemmé lesznek, én pedig Istenökké leszek hûséggel és igazsággal.
\par 9 Ezt mondja a Seregeknek Ura: Erõsödjenek meg kezeitek, a kik hallottátok e napokban a beszédeket a próféták szájából, a kik szóltak, mikor megvetteték a Seregek Ura házának alapja, hogy megépíttessék  a templom.
\par 10 Mert e napok elõtt nem volt az embernek bére, és a baromnak sem volt bére, sem a kimenõnek, sem a bejövõnek nem volt békessége a háborúság miatt, mert minden embert felindítottam: kit-kit az õ felebarátja ellen.
\par 11 De most nem olyan leszek e nép maradékához, mint az elébbi napokban voltam, így szól a Seregeknek Ura.
\par 12 Mert a vetés békességes lészen, a szõlõtõ megadja gyümölcsét, a föld is megadja termését, az egek is megadják harmatjokat, és örökössé teszem mindezeken e nép maradékát.
\par 13 És lészen, hogy a miképen átok valátok a pogányok között, oh Júda háza és Izráel háza: azonképen megszabadítlak titeket, és áldássá lesztek. Ne féljetek! Erõsödjenek meg kezeitek!
\par 14 Mert ezt mondja a Seregeknek Ura: A miképen elgondoltam vala, hogy veszedelmet hozok reátok, mikor atyáitok megharagítottak vala engem, így szól a Seregeknek Ura, és nem könyörültem:
\par 15 Azonképen megtértem és elgondoltam e napokban, hogy jót teszek Jeruzsálemmel és Júda házával; ne féljetek!
\par 16 Ezek azok a dolgok, a melyeket cselekedjetek: Igazságot szóljon ki-ki az õ felebarátjával: igazságos és békességes ítélettel ítéljetek a ti kapuitokban.
\par 17 És senki ne gondoljon az õ szívében gonoszt az õ felebarátja ellen; s a hamis esküvést se szeressétek, mert ezek azok, a miket én mind gyûlölök, így szól az Úr.
\par 18 Majd szóla hozzám a Seregeknek Ura, mondván:
\par 19 Ezt mondja a Seregeknek Ura: A negyedik hónapnak bõjtje, az ötödiknek bõjtje, a hetediknek bõjtje és a tizediknek bõjtje vígalommá, örvendezéssé és kedves ünnepekké lesznek Júda házában. Csak a hûséget és a békességet szeressétek.
\par 20 Ezt mondja a Seregeknek Ura: Még lesz idõ, a mikor népek jõnek el, és sok városoknak lakói.
\par 21 És egyiknek lakói a másikhoz mennek, mondván: Menten menjünk el az Úr orczájának engesztelésére, és a Seregek Urának keresésére; én is elmegyek!
\par 22 És eljõnek sok népek és erõs nemzetek Jeruzsálembe a Seregeknek Urának keresésére; és az Úr orczájának engesztelésére.
\par 23 Ezt mondja a Seregeknek Ura: E napokban lesz az, hogy a minden nyelvû pogányok közül tíz ember ragad egy zsidó férfiúba s ragad annak ruhája szélébe, mondván: Hadd menjünk veletek, mert hallottuk, hogy veletek van az Isten!

\chapter{9}

\par 1 Az Úr igéjének terhe a Kadrák földe ellen; Damaskus lesz pedig annak nyugvóhelye, (mert az Úr szemmel tartja az embereket és Izráelnek minden törzsét);
\par 2 És Hámát is, a mely szomszédos vele; Tírus és Sídon, noha igen okosak!
\par 3 Várat épített magának Tírus, és annyi az ezüstje rakáson, mint a por, és az aranya, mint az utczák sara.
\par 4 Ímé szegénynyé teszi õt az Úr, és megrontja hatalmát a tengeren, magát pedig tûz emészti meg.
\par 5 Meglátja ezt Askalon és megretten; Gáza is igen bánkódik; Ekron is, mert megszégyenült reménységében. Mert kivész a király Gázából, és Askalon lakatlan marad.
\par 6 Asdódban pedig idegenek laknak, és a Filiszteusok kevélységét megtöröm.
\par 7 Kivonszom a vért szájukból és útálatosságaikat fogaik közül, és õ is a mi Istenünké marad; és olyan lesz, mint egy fejedelem Júdában, Ekron pedig, mint a Jebuzeus.
\par 8 És tábort járok házam körül, mint a sereg ellen, az ide-oda kóborlók ellen, és nem megy át többé rajtok a sarczoló, mert most szemmel tartom õt.
\par 9 Örülj nagyon, Sionnak leánya, örvendezz, Jeruzsálem leánya! Ímé, jön néked a te királyod; igaz és szabadító õ; szegény és szamárháton ülõ, azaz nõstényszamárnak vemhén.
\par 10 És kivesztem a szekeret Efraimból és a lovat Jeruzsálembõl, kivesztem a harczi kézívet is, és békességet hirdet a pogányoknak; és uralkodik tengertõl tengerig, és a folyamtól a föld határáig.
\par 11 Sõt a veled való szövetségnek véréért a te foglyaidat is kibocsátom a kútból, a melyben nincs víz.
\par 12 Térjetek vissza az erõsséghez, reménységnek foglyai! Ma is azt hirdetem néktek: kétszeresen megfizetek néked!
\par 13 Mert kifeszítem Júdát magamnak mintegy kézívet és megtöltöm Efraimot; és felindítom fiaidat, oh Sion, a te fiaid ellen, oh Jáván, és olyanná teszlek, mint a hõs fegyvere.
\par 14 És megjelen felettök az Úr, és nyila repül mint a villámlás; az Úr Isten kürtöt fuvall, és déli szelekben nyomul elõ.
\par 15 A Seregeknek Ura megoltalmazza õket; megemésztik és letapossák a parittya-köveket, és isznak és zajongnak, mint a bortól, és megtelnek, mint a csészék és mint az oltár szegletei.
\par 16 És megsegíti õket az Úr, az õ Istenök ama napon, mint az õ népének nyáját, és mint korona-kövek ragyognak az õ földén.

\chapter{10}

\par 1 Kérjetek esõt az Úrtól a késõi esõ idején! Az Úr villámlást szerez, és záporesõt ad nékik, és kinek-kinek füvet a mezõn.
\par 2 Mert a bálványok hazugságot szólnak, a varázslók pedig hamisságot látnak és üres álmokat beszélnek, hiábavalósággal vígasztalnak; azért elszélednek, mint a juhnyáj, a mely sanyarog, mert nincs pásztora.
\par 3 Haragra gerjedtem a pásztorok ellen, és megfenyítem a bakokat. Bizony megfenyíti a Seregeknek Ura az õ nyáját, a Júda házát, és olyanokká teszi õket, a milyen a harczra felékesített ló.
\par 4 Közülök támad a szegletkõ, közülök a szeg, közülök a harczi ív, közülök egyszersmind minden sarczoló.
\par 5 És olyanok lesznek, mint a hõsök, a kik az utczák sarát tapodják a harczban, és harczolnak, mert velök van az Úr, és megszégyenítik a lovon ülõket.
\par 6 És megerõsítem a Júda házát, és a József házát megsegítem, és visszahozom õket, mert szánom õket, és olyanokká lesznek, mintha el sem vetettem volna õket; mert én vagyok az Úr, az õ Istenök, és meghallgatom õket.
\par 7 Efraim is olyan lesz, mint egy hõs, és örvendeznek majd mintegy bortól ittasodva, és látják fiaik és örvendeznek; örvend az õ szívök az Úrban.
\par 8 Süvöltök nékik és egybegyûjtöm õket, mert megszabadítom õket, és megsokasulnak, a mint megsokasultak volna.
\par 9 És széthintem õket a népek között, hogy a messze földeken is emlegessenek engem, és fiakat neveljenek és visszatérjenek.
\par 10 Mert visszatérítem õket, Égyiptom földérõl, Assiriából is összegyûjtöm õket, és behozom õket Gileád és Libánon földjére, elég sem lesz nékik.
\par 11 És átvonulnak a nyomor tengerén, és megveri a tenger hullámait, és kiszáradnak a folyam örvényei, letöretik Assiriának kevélysége, és Égyiptom királyi pálczája elvész.
\par 12 És megerõsítem õket az Úrban, és az õ nevében járnak, így szól az Úr.

\chapter{11}

\par 1 Nyisd meg kapuidat, oh Libánon, hogy tûz emészszen czédrusaid közt!
\par 2 Jajgass te cziprus, mert esik a czédrus, leomlott, a mi legjava! Jajgassatok ti Básán tölgyei, mert pusztul a rengeteg erdõ.
\par 3 Hangzik a pásztorok jajja, mert elpusztult az õ büszkeségök! Hangzik az oroszlán ordítása, mert elpusztult a Jordán kevélysége!
\par 4 Ezt mondja az Úr, az én Istenem: Legeltesd a leölésre szánt juhokat,
\par 5 A melyeket leölnek az õ tulajdonosaik, a nélkül, hogy bûnnek tartanák, eladóik pedig ezt mondják: Áldott az Úr, mert meggazdagodtam! és pásztoraik sem kimélik õket.
\par 6 Bizony nem kimélem többé e föld lakosait, ezt mondja az Úr; sõt ímé odaadok minden embert a felebarátja kezébe és az õ királya kezébe, és megrontják e földet, és nem szabadítom ki kezökbõl!
\par 7 Legeltetém hát a leölésre szánt juhokat, azaz a megnyomorgatott juhokat, és választék magamnak két pálczát, az egyiket nevezém szépségnek, a másikat nevezém egyességnek; így legeltetém a juhokat.
\par 8 És három pásztort vertem el egy hónap alatt, mert elkeseredék a lelkem miattok, és az õ lelkök is megútála engem.
\par 9 És mondám: Nem õrizlek én titeket, haljon meg a halálra való és vágattassék ki a kivágni való, a megmaradottak pedig egyék meg egymásnak húsát.
\par 10 És vevém egyik pálczámat, a szépséget, és eltörém azt, hogy felbontsam az én szövetségemet, a melyet az összes népekkel kötöttem.
\par 11 És felbomla az azon a napon, és így tudták meg az elsanyargatott juhok, a kik ragaszkodnak vala hozzám, hogy az Úr dolga ez.
\par 12 És mondám nékik: Ha jónak tetszik néktek, adjátok meg az én béremet; ha pedig nem: hagyjátok abba! És harmincz ezüst pénzt fizettek béremül.
\par 13 És monda az Úr nékem: Vesd a fazekas elé! Nagy jutalom, a melyre becsültek engem. Vevém azért a harmincz ezüst pénzt, és vetém azt az Úrnak házába, a fazekas elé.
\par 14 Majd eltörém a másik pálczámat is, az egyességet, hogy felbontsam a testvérséget Júda között és Izráel között.
\par 15 És mondá az Úr nékem: Most már szerezz magadnak bolond pásztornak való szerszámot.
\par 16 Mert ímé, én pásztort állítok e földre, a ki elveszetteket meg nem keresi, a gyöngével nem törõdik, a megtépettet meg nem gyógyítja, a jó karban levõt nem táplálja, a kövérinek húsát megeszi, és körmeiket széttördeli.
\par 17 Jaj a mihaszna pásztornak, a ki elhagyja a juhokat! Fegyver a karjára és jobb szemére. Karja szárazra száradjon és jobb szeme sötétre sötétedjék.

\chapter{12}

\par 1 Az Úr igéjének terhe Izráel ellen. Így szól az Úr, a ki az egeket kiterjesztette, a földet fundálta, és az ember keblébe lelket alkotott.
\par 2 Ímé, én részegítõ pohárrá teszem Jeruzsálemet minden körülte való népnek; Júdának is az lesz, mikor ostromolják Jeruzsálemet.
\par 3 És azon a napon lesz, hogy nyomtatókõvé teszem Jeruzsálemet minden népnek; a ki emelni akarja azt, mind szakadva-szakad meg, noha összegyül ellene a föld minden pogánya.
\par 4 Azon a napon, így szól az Úr, megverek minden lovat rettegéssel, a lovagját pedig õrültséggel, de a Júda házát nyitott szemmel nézem, a népeknek pedig minden lovát vaksággal verem meg.
\par 5 És azt mondják szívökben Júda fejedelmei: Az én erõsségem Jeruzsálemnek lakói, az õ Istenökkel, a Seregek Urával.
\par 6 Azon a napon olyanokká teszem Júda fejedelmeit, mint a milyen a tözes serpenyõ a fák között, és amilyen a tözes fáklya a kévék között: megemésztenek jobb és bal felõl minden körülvaló népet; de Jeruzsálem tovább is a helyén marad Jeruzsálemben!
\par 7 És megoltalmazza az Úr Júdának sátrait, mint azelõtt, hogy ne legyen nagyobb Dávid házának dicsõsége és Jeruzsálem lakosának dicsõsége, mint a Júdáé.
\par 8 Azon a napon oltalma lészen az Úr Jeruzsálem lakosának, és azon a napon olyan lesz köztök a legalábbvaló, mint Dávid, a Dávid háza pedig, mint az Isten, mint az Úrnak angyala õ elõttök.
\par 9 És azon a napon lesz, hogy kész leszek elveszteni minden pogányt, a kik Jeruzsálemre támadnak;
\par 10 A Dávid házára és Jeruzsálem lakosaira pedig kiöntöm a kegyelemnek és könyörületességnek lelkét, és reám tekintenek, a kit átszegeztek, és siratják õt, a mint siratják az egyetlen fiút, és keseregnek utána, a mint keseregnek az elsõszülött  után.
\par 11 Azon a napon nagy siralom lesz Jeruzsálemben, a milyen volt a hadadrimmoni siralom a Megiddo völgyében.
\par 12 És sír a föld: nemzetségek és nemzetségek külön; külön a Dávid házának nemzetsége, feleségeik is külön; külön a Nátán házának nemzetsége, és feleségeik is külön;
\par 13 Külön a Lévi házának nemzetsége, és feleségeik is külön; külön a Sémei nemzetsége, feleségeik is külön.
\par 14 A többi nemzetségek mind; nemzetségek és nemzetségek külön, feleségeik is külön.

\chapter{13}

\par 1 Azon a napon kútfõ fakad a Dávid házának és Jeruzsálem lakosainak a bûn és tisztátalanság ellen.
\par 2 És lészen azon a napon, így szól a Seregeknek Ura: Kivesztem a bálványok neveit e földrõl, és emlegetni sem fogják többé; sõt a prófétákat és a fertelmes lelket is kiszaggatom e földrõl.
\par 3 És úgy lesz, ha prófétálni fog még valaki, azt mondják annak az õ apja és anyja, az õ szülõi: Ne élj, mert hazugságot szóltál az Úr nevében! És általverik õt az õ apja és anyja, az õ szülõi, az õ prófétálása közben.
\par 4 És azon a napon megszégyenülnek a próféták, kiki az õ látása miatt az õ prófétálásaik közben, és nem öltözködnek szõrös ruhába, hogy hazudjanak.
\par 5 Hanem ezt mondja kiki: Nem vagyok én próféta, szántóvetõ ember vagyok én, sõt más szolgájává lettem én gyermekségem óta.
\par 6 És ha mondja néki valaki: Micsoda ütések ezek a kezeiden? azt mondja: A miket az én barátaim házában ütöttek rajtam.
\par 7 Fegyver, serkenj fel az én pásztorom ellen és a férfiú ellen, a ki nékem társam! így szól a Seregeknek Ura. Verd meg a pásztort és elszélednek a juhok, én pedig a kicsinyek ellen fordítom kezemet.
\par 8 És lészen az egész földön, így szól az Úr: a két rész kivágattatik azon és meghal, de a harmadik megmarad rajta.
\par 9 És beviszem a harmadrészt a tûzbe, és negtisztítom õket, a mint tisztítják az ezüstöt és megpróbálom õket, a mint próbálják az aranyat; õ segítségül hívja az én nevemet és én felelni fogok néki; ezt mondom: Népem õ! Õ pedig ezt mondja: Az Úr az én Istenem!

\chapter{14}

\par 1 Ímé, eljön az Úrnak napja, és a te prédádat felosztják benned.
\par 2 Mert minden népet ütközetre gyûjtök Jeruzsálemhez, és megszállják a várost, és kirabolják a házakat, megszeplõsítik az asszonyokat; és a város fele számkivetésbe megy, de a nép maradéka nem gyomláltatik ki a városból.
\par 3 Mert eljön az Úr, és harczol azok ellen a népek ellen, a mint harczolt vala ama napon, a harcznak napján.
\par 4 És azon a napon az Olajfák hegyére veti lábait, a mely szemben van Jeruzsálemmel napkelet felõl, és az Olajfák hegye közepén ketté válik, kelet felé és nyugot felé, igen nagy völgygyé, és a hegynek fele észak felé, fele pedig dél felé szakad.
\par 5 És az én hegyem völgyébe futtok, mert a hegyközi völgy Azálig nyúlik, és úgy futtok, a mint futottak a földindulás elõl Uzziásnak, Júda királyának napjaiban. Bizony eljõ az Úr, az én Istenem, és minden szent vele.
\par 6 És úgy lesz azon a napon: Nem lesz világosság, a ragyogó testek összezsugorodnak.
\par 7 De lesz egy nap, a melyet az Úr tud, se nappal, se éjszaka, és világosság lesz az estvének idején.
\par 8 És e napon lesz, hogy élõ vizek jõnek ki Jeruzsálembõl, felerészök a napkeleti tenger felé, felerészök pedig a nyugoti tenger felé, és nyárban és télben is úgy lesz.
\par 9 És az Úr lesz az egész földnek királya, e napon egy Úr lészen, és a neve is egy.
\par 10 Az egész föld síksággá változik Gebától kezdve Rimmonig, déli irányban Jeruzsálem felé, és felmagasztaltatik és a maga helyén marad a Benjámin kapujától az elsõ kapu helyéig, a szegletkapuig, és a Hananéel tornyától a király sajtójáig.
\par 11 És lakni fognak benne, és nem éri többé pusztulás, és barátságban lakoznak Jeruzsálemben.
\par 12 És ez lesz a csapás, a melylyel megcsapkod az Úr minden népet, a melyek Jeruzsálem ellen gyülekeznek: Megsenyved a húsok és pedig a míg lábaikon állnak, szemeik is megsenyvednek gödreikben, myelvök is megsenyved szájokban.
\par 13 És azon a napon lesz, hogy az Úr nagy háborúságot támaszt közöttök, úgy, hogy kiki a maga társának kezét ragadja meg, és a maga társának keze ellen emeli fel kezét.
\par 14 Sõt még Júda is harczolni fog Jeruzsálem ellen, és összegyûjtetik a köröskörül lakó népek minden gazdagsága: arany, ezüst és igen sok ruha.
\par 15 És éppen olyan csapás lesz a lovakon, öszvéreken, tevéken, szamarakon és mindenféle barmokon, a melyek e táborban lesznek, a milyen ez a csapás.
\par 16 És lészen, hogy a kik megmaradnak mindama népek közül, a melyek Jeruzsálem ellen jõnek: esztendõrõl esztendõre mind felmennek, hogy hódoljanak a királynak, a Seregeknek Urának, és megünnepeljék a sátorok ünnepét.
\par 17 És lészen, hogy a ki nem megy fel e föld nemzetségei közül Jeruzsálembe, hogy hódoljon a királynak, a Seregek Urának: nem lészen azokra esõ.
\par 18 És ha nem megy fel, vagy nem jön fel az égyiptomi nemzetség, õ rájok sem lészen; de lészen az a csapás, a melylyel megcsapkodja az Úr a népeket, a kik nem mennek fel a sátorok ünnepét megünnepelni.
\par 19 Ez lészen Égyiptomnak büntetése, és mindama népek büntetése, a kik nem mennek fel a sátorok ünnepét megünnepelni.
\par 20 Azon a napon a lovak csengettyûin is ez lesz: Az Úrnak szenteltetett. És a fazekak az Úrnak házában olyanokká lesznek, mint az oltár elõtt való medenczék.
\par 21 És Jeruzsálemben és Júdában minden fazék a Seregek Urának szenteltetik, és eljõnek mind, a kik áldozni akarnak, és választanak közülök és fõznek azokban; és nem lészen többé Kananeus a Seregek Urának házában e napon.


\end{document}