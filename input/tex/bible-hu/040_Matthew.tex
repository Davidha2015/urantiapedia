\begin{document}

\title{Máté evangéliuma}


\chapter{1}

\par 1 Jézus Krisztusnak, Dávid fiának, Ábrahám fiának nemzetségérõl való könyv.
\par 2 Ábrahám nemzé Izsákot; Izsák nemzé Jákóbot; Jákób nemzé Júdát és testvéreit;
\par 3 Júda nemzé Fárest és Zárát Támártól; Fáres nemzé Esromot; Esrom  nemzé Arámot;
\par 4 Arám nemzé Aminádábot; Aminádáb nemzé Naássont; Naásson nemzé Sálmónt;
\par 5 Sálmón nemzé Boázt Ráhábtól; Boáz nemzé Obedet Ruthtól; Obed nemzé Isait;
\par 6 Isai nemzé Dávid királyt; Dávid király nemzé Salamont az  Uriás feleségétõl;
\par 7 Salamon nemzé Roboámot; Roboám nemzé Abiját; Abija nemzé Asát;
\par 8 Asa nemzé Josafátot; Josafát nemzé Jórámot; Jórám nemzé Uzziást;
\par 9 Uzziás nemzé Jóathámot; Joathám nemzé Ákházt; Ákház  nemzé Ezékiást;
\par 10 Ezékiás nemzé Manassét; Manassé nemzé  Ámont; Ámon nemzé Jósiást;
\par 11 Jósiás nemzé Jekoniást és testvéreit a babilóni fogságra vitelkor.
\par 12 A babilóni fogságravitel után pedig Jekoniás nemzé Saláthielt; Saláthiel nemzé  Zorobábelt;
\par 13 Zorobábel nemzé Abiudot; Abiud nemzé Eliákimot; Eliákim nemzé Azort;
\par 14 Azor nemzé Sádokot; Sádok nemzé Akimot; Akim nemzé Eliudot;
\par 15 Eliud nemzé Eleázárt; Eleázár nemzé Matthánt; Matthán nemzé Jákóbot;
\par 16 Jákób nemzé Józsefet, férjét Máriának, a kitõl született Jézus, a ki Krisztusnak neveztetik.
\par 17 Az összes nemzetség tehát Ábrahámtól Dávidig tizennégy nemzetség, és Dávidtól a babilóni fogságravitelig tizennégy nemzetség, és a babilóni fogságraviteltõl Krisztusig tizennégy nemzetség.
\par 18 A Jézus Krisztus születése pedig így vala: Mária, az õ anyja, eljegyeztetvén Józsefnek, mielõtt egybekeltek volna, viselõsnek találtaték a Szent Lélektõl.
\par 19 József pedig, az õ férje, mivelhogy igaz ember vala és nem akará õt gyalázatba keverni, el akarta õt titkon bocsátani.
\par 20 Mikor pedig ezeket magában elgondolta: ímé az Úrnak angyala álomban megjelenék néki, mondván: József, Dávidnak fia, ne félj magadhoz venni Máriát, a te feleségedet, mert a mi benne fogantatott, a Szent Lélektõl van az.
\par 21 Szûl pedig fiat, és nevezd annak nevét Jézusnak, mert õ szabadítja meg az õ népét  annak bûneibõl.
\par 22 Mindez pedig azért lõn, hogy beteljesedjék, a mit az Úr mondott volt a próféta által, a ki így szól:
\par 23 Ímé a szûz fogan méhében és szûl fiat, és annak nevét Immanuelnek nevezik, a mi azt jelenti: Velünk az Isten.
\par 24 József pedig az álomból felserkenvén, úgy tõn, a mint az Úr angyala parancsolta vala néki, és feleségét magához vevé.
\par 25 És nem ismeré õt, míg meg nem szülé az õ elsõszülött fiát; és nevezé annak nevét Jézusnak.

\chapter{2}

\par 1 A mikor pedig megszületik vala Jézus a júdeai Bethlehemben, Heródes király idejében, ímé napkeletrõl bölcsek jövének Jeruzsálembe, ezt mondván:
\par 2 Hol van a zsidók királya, a ki megszületett? Mert láttuk az õ csillagát napkeleten, és azért jövénk, hogy tisztességet tegyünk néki.
\par 3 Heródes király pedig ezt hallván, megháborodék, és vele együtt az egész Jeruzsálem.
\par 4 És egybegyûjtve minden fõpapot és a nép írástudóit, tudakozódik vala tõlük, hol kell a Krisztusnak megszületnie?
\par 5 Azok pedig mondának néki: A júdeai Bethlehemben; mert így írta vala meg a próféta:
\par 6 És te Bethlehem, Júdának földje, semmiképen sem vagy legkisebb Júda fejedelmi városai között: mert belõled származik a fejedelem, a ki legeltetni fogja az én népemet, az Izráelt.
\par 7 Ekkor Heródes titkon hivatván a bölcseket, szorgalmatosan megtudakolá tõlük a csillag megjelenésének idejét.
\par 8 És elküldvén õket Bethlehembe, mondta nékik: Elmenvén, szorgalmatosan kérdezõsködjetek a gyermek felõl, mihelyt pedig megtaláljátok, adjátok tudtomra, hogy én is elmenjek és tisztességet tegyek néki.
\par 9 Õk pedig a király beszédét meghallván, elindulának. És ímé a csillag, a melyet napkeleten láttak, elõttük megy vala mind addig, a míg odaérvén, megálla a hely fölött, a hol a gyermek vala.
\par 10 És mikor meglátták a csillagot, igen nagy örömmel örvendezének.
\par 11 És bemenvén a házba, ott találák a gyermeket anyjával, Máriával; és leborulván, tisztességet tõnek néki; és kincseiket kitárván, ajándékokat adának néki: aranyat, tömjént és mirhát.
\par 12 És mivel álomban meginttettek, hogy Heródeshez vissza ne menjenek, más úton térének vissza hazájokba.
\par 13 Mikor pedig azok visszatérnek vala, ímé megjelenék az Úrnak angyala Józsefnek álomban, és monda: Kelj fel, vedd a gyermeket és annak anyját, és fuss Égyiptomba, és maradj ott, a míg én mondom néked; mert Heródes halálra fogja keresni a gyermeket.
\par 14 Õ pedig fölkelvén, vevé a gyermeket és annak anyját éjjel, és Égyiptomba távozék.
\par 15 És ott vala egész a Heródes haláláig, hogy beteljesedjék, a mit az Úr mondott a próféta által, a ki így szólt: Égyiptomból hívtam ki az én fiamat.
\par 16 Ekkor Heródes látván, hogy a bölcsek megcsúfolták õt, szerfölött felháborodék, és kiküldvén, megölete Bethlehemben és annak egész környékén minden gyermeket, két esztendõstõl és azon alól, az idõ szerint, a melyet szorgalmasan tudakolt a böcsektõl.
\par 17 Ekkor teljesedék be, a mit Jeremiás próféta mondott, a midõn így szólt:
\par 18 Szó hallatszott Rámában: Sírás-rívás és sok keserves jajgatás. Rákhel siratta az õ fiait és nem akart megvigasztaltatni, mert nincsenek.
\par 19 Mikor pedig Heródes meghalt, ímé az Úrnak angyala megjelenék álomban Józsefnek Égyiptomban.
\par 20 Mondván: Kelj fel, vedd a gyermeket és annak anyját, és eredj az Izráel földére; mert meghaltak, a kik a gyermeket halálra keresik vala.
\par 21 Õ pedig felkelvén, magához vevé a gyermeket és annak anyját, és elméne Izráel földére.
\par 22 Mikor pedig hallá, hogy Júdeában Arkhelaus uralkodik, Heródesnek, az õ atyjának helyén, nem mert vala oda menni, hanem minthogy álomban meginteték, Galilea vidékeire tére.
\par 23 És oda jutván, lakozék Názáret nevû városban, hogy beteljesedjék, a mit a próféták mondottak,  hogy názáretinek fog neveztetni.

\chapter{3}

\par 1 Azokban a napokban pedig eljöve Keresztelõ János, a ki prédikál vala Júdea pusztájában.
\par 2 És ezt mondja vala: Térjetek meg, mert elközelített a mennyeknek országa.
\par 3 Mert ez az, a kirõl Ésaiás próféta szólott, ezt mondván: Kiáltó szó a pusztában: Készítsétek az Úrnak útját, és egyengessétek meg az õ ösvényeit.
\par 4 Ennek a Jánosnak a ruhája pedig teveszõrbõl vala, és bõröv vala a dereka körül, elesége pedig sáska és erdei méz.
\par 5 Ekkor kiméne õ hozzá Jeruzsálem és az egész Júdea és a Jordánnak egész környéke.
\par 6 És megkeresztelkednek vala õ általa a Jordán vizében, vallást tevén az õ bûneikrõl.
\par 7 Mikor pedig látá, hogy a farizeusok és sadduceusok közül sokan mennek õ hozzá, hogy megkeresztelkedjenek, monda nékik: Mérges kígyóknak fajzatai! Kicsoda intett meg titeket, hogy az Istennek elkövetkezendõ haragjától megmeneküljetek?
\par 8 Teremjetek hát megtéréshez illõ gyümölcsöket.
\par 9 És ne gondoljátok, hogy így szólhattok magatokban: Ábrahám a mi atyánk! Mert mondom néktek, hogy Isten eme kövekbõl is támaszthat fiakat Ábrahámnak.
\par 10 A fejsze pedig immár a fák gyökerére vettetett. Azért minden fa, a mely jó gyümölcsöt nem terem, kivágattatik és tûzre vettetik.
\par 11 Én ugyan vízzel keresztellek titeket megtérésre, de a ki utánam jõ, erõsebb nálamnál, a kinek saruját hordozni sem vagyok méltó;  õ Szent Lélekkel és tûzzel keresztel majd titeket.
\par 12 A kinek szóró lapát van az õ kezében, és megtisztítja az õ szérûjét; és az õ gabonáját csûrbe takarítja, a polyvát pedig megégeti olthatatlan tûzzel.
\par 13 Akkor eljöve Jézus Galileából a Jordán mellé Jánoshoz, hogy megkeresztelkedjék õ általa.
\par 14 János azonban visszatartja vala õt, mondván: Nékem kell általad megkeresztelkednem, és te jõsz én hozzám?
\par 15 Jézus pedig felelvén, monda néki: Engedj most, mert így illik nékünk minden igazságot betöltenünk. Ekkor engede néki.
\par 16 És Jézus megkeresztelkedvén, azonnal kijöve a vízbõl; és ímé az egek megnyilatkozának néki, és õ látá az Istennek Lelkét alájõni mintegy galambot és õ reá szállani.
\par 17 És ímé egy égi hang ezt mondja vala: Ez amaz én szerelmes fiam, a kiben  én gyönyörködöm.

\chapter{4}

\par 1 Akkor Jézus viteték a Lélektõl a pusztába, hogy megkisértessék az ördögtõl.
\par 2 És mikor negyven nap és negyven éjjel bõjtölt vala, végre megéhezék.
\par 3 És hozzámenvén a kisértõ, monda néki: Ha Isten fia vagy, mondd, hogy e kövek változzanak kenyerekké.
\par 4 Õ pedig felelvén, monda: Meg van írva: Nemcsak kenyérrel él az ember, hanem minden ígével, a mely Istennek szájából származik.
\par 5 Ekkor vivé õt az ördög a szent városba, és odahelyezé a templom tetejére.
\par 6 És monda néki: Ha Isten fia vagy, vesd alá magadat; mert meg van írva: Az õ angyalainak parancsol felõled, és kézen hordoznak téged, hogy meg ne üsd lábadat a kõbe.
\par 7 Monda néki Jézus: Viszont meg van írva: Ne kisértsd az Urat, a te Istenedet.
\par 8 Ismét vivé õt az ördög egy igen magas hegyre, és megmutatá néki a világ minden országát és azok dicsõségét,
\par 9 És monda néki: Mindezeket néked adom, ha leborulva imádsz engem.
\par 10 Ekkor monda néki Jézus: Eredj el Sátán, mert meg van írva: Az Urat, a te Istenedet imádd, és csak néki szolgálj.
\par 11 Ekkor elhagyá õt az ördög. És ímé angyalok jövének hozzá és szolgálnak vala néki.
\par 12 Mikor pedig meghallotta Jézus, hogy János börtönbe vettetett, visszatére Galileába;
\par 13 És odahagyva Názáretet, elméne és lakozék a tengerparti Kapernaumban, a Zebulon és Naftali határain;
\par 14 Hogy beteljesedjék, a mit Ésaiás próféta mondott, így szólván:
\par 15 (Zebulonnak földje és Naftalinak földje, a tenger felé, a Jordánon túl, a pogányok Galileája,
\par 16 A nép, a mely sötétségben ül vala, láta nagy világosságot, és a kik a halálnak földében és árnyékában ülnek vala, azoknak világosság támada.
\par 17 Ettõl fogva kezde Jézus prédikálni, és ezt mondani: Térjetek meg, mert elközelgetett a mennyeknek országa.
\par 18 Mikor pedig a galileai tenger mellett jár vala Jézus, láta két testvért, Simont, a kit Péternek neveznek, és Andrást az õ testvérét, a mint a tengerbe hálót vetnek vala; mert halászok valának.
\par 19 És monda nékik: Kövessetek engem, és azt mívelem, hogy embereket halásszatok.
\par 20 Azok pedig azonnal otthagyván a hálókat, követék õt.
\par 21 És onnan tovább menve, láta más két testvért, Jakabot a Zebedeus fiát, és Jánost amannak testvérét, a mint a hajóban atyjukkal Zebedeussal a hálóikat kötözgetik vala; és hívá õket.
\par 22 Azok pedig azonnal otthagyván a hajót és atyjukat, követék õt.
\par 23 És bejárá Jézus az egész Galileát, tanítva azok zsinagógáiban, és hirdetve az Isten országának evangyéliomát, és gyógyítva a nép között minden betegséget és minden erõtlenséget.
\par 24 És elterjede az õ híre egész Siriában: és hozzávivék mindazokat, a kik rosszul valának, a különféle betegségekben és kínokban sínlõdõket, ördöngösöket, holdkórosokat és gutaütötteket; és meggyógyítja vala õket.
\par 25 És nagy sokaság követé õt Galileából és a Tízvárosból és Jeruzsálembõl és Júdeából és a Jordánon túlról.

\chapter{5}

\par 1 Mikor pedig látta Jézus a sokaságot, felméne a hegyre, és a mint leül vala, hozzámenének az õ tanítványai.
\par 2 És megnyitván száját, tanítja vala õket, mondván:
\par 3 Boldogok a lelki szegények: mert övék a mennyeknek országa.
\par 4 Boldogok, a kik sírnak: mert õk megvígasztaltatnak.
\par 5 Boldogok a szelídek: mert õk örökségül bírják a földet.
\par 6 Boldogok, a kik éhezik és szomjúhozzák az igazságot: mert õk megelégíttetnek.
\par 7 Boldogok, az irgalmasok: mert õk irgalmasságot nyernek.
\par 8 Boldogok, a kiknek szívök tiszta: mert õk az Istent meglátják.
\par 9 Boldogok a békességre igyekezõk: mert õk az Isten fiainak mondatnak.
\par 10 Boldogok, a kik háborúságot szenvednek az igazságért: mert övék a mennyeknek országa.
\par 11 Boldogok vagytok, ha szidalmaznak és háborgatnak titeket és minden gonosz hazugságot mondanak ellenetek én érettem.
\par 12 Örüljetek és örvendezzetek, mert a ti jutalmatok bõséges a mennyekben: mert így háborgatták a prófétákat is, a kik elõttetek voltak.
\par 13 Ti vagytok a földnek savai; ha pedig a só megízetlenül, mivel sózzák meg? nem jó azután semmire, hanem hogy kidobják és eltapossák az emberek.
\par 14 Ti vagytok a világ világossága. Nem rejtethetik el a hegyen épített város.
\par 15 Gyertyát sem azért gyújtanak, hogy a véka alá, hanem hogy a gyertyatartóba tegyék és fényljék mindazoknak, a kik a házban vannak.
\par 16 Úgy fényljék a ti világosságtok az emberek elõtt, hogy lássák a ti jó cselekedeteiteket, és dicsõítsék a ti mennyei Atyátokat.
\par 17 Ne gondoljátok, hogy jöttem a törvénynek vagy a prófétának eltörlésére. Nem jöttem, hogy eltöröljem, hanem inkább, hogy betöltsem.
\par 18 Mert bizony mondom néktek, míg az ég és a föld elmúlik, a törvénybõl egy jóta vagy egyetlen pontocska el nem múlik, a míg minden be nem teljesedik.
\par 19 Valaki azért csak egyet is megront e legkisebb parancsolatok közül és úgy tanítja az embereket, a mennyeknek országában a legkisebb lészen; valaki pedig cselekszi és úgy tanít, az a mennyeknek országában nagy lészen.
\par 20 Mert mondom néktek, hogy ha a ti igazságotok nem több az írástudók és farizeusok igazságánál, semmiképen sem mehettek be a mennyeknek országába.
\par 21 Hallottátok, hogy megmondatott a régieknek: Ne ölj, mert a ki öl, méltó az ítéletre.
\par 22 Én pedig azt mondom néktek, hogy mindaz, a ki haragszik az õ atyjafiára ok nélkül, méltó az ítéletre: a ki pedig azt mondja az õ atyjafiának: Ráka, méltó a fõtörvényszékre: a ki pedig ezt mondja: Bolond, méltó a gyehenna tüzére.
\par 23 Azért, ha a te ajándékodat az oltárra viszed és ott megemlékezel arról, hogy a te atyádfiának valami panasza van ellened:
\par 24 Hagyd ott az oltár elõtt a te ajándékodat, és menj el, elébb békélj meg a te atyádfiával, és azután eljövén, vidd fel a te ajándékodat.
\par 25 Légy jóakarója a te ellenségednek hamar, a míg az úton vagy vele, hogy ellenséged valamiképen a bíró kezébe ne adjon, és a bíró oda ne adjon a poroszló kezébe, és tömlöczbe ne vessen téged.
\par 26 Bizony mondom néked: ki nem jõsz onnét, mígnem megfizetsz az utolsó fillérig.
\par 27 Hallottátok, hogy megmondatott a régieknek: Ne paráználkodjál!
\par 28 Én pedig azt mondom néktek, hogy valaki asszonyra tekint gonosz kivánságnak okáért, immár paráználkodott azzal az õ szívében.
\par 29 Ha pedig a te jobb szemed megbotránkoztat téged, vájd ki azt és vesd el magadtól; mert jobb néked, hogy egy vesszen el a te tagjaid közül, semhogy egész tested a gyehennára vettessék.
\par 30 És ha a te jobb kezed botránkoztat meg téged, vágd le azt és vesd el magadtól; mert jobb néked, hogy egy vesszen el a te tagjaid közül, semhogy egész tested a gyehennára vettessék.
\par 31 Megmondatott továbbá: Valaki elbocsátja feleségét, adjon néki elválásról való levelet.
\par 32 Én pedig azt mondom néktek: Valaki elbocsátja feleségét, paráznaság okán kívül, paráznává teszi azt; és a ki elbocsátott asszonyt veszen el, paráználkodik.
\par 33 Ismét hallottátok, hogy megmondatott a régieknek: Hamisan ne esküdjél, hanem teljesítsd az Úrnak tett esküidet.
\par 34 Én pedig azt mondom néktek: Teljességgel ne esküdjetek; se az égre, mert az  az Istennek királyi széke;
\par 35 Se a földre, mert az az õ lábainak zsámolya; se Jeruzsálemre, mert a nagy Királynak városa;
\par 36 Se a te fejedre ne esküdjél, mert egyetlen hajszálat sem tehetsz fehérré vagy feketévé;
\par 37 Hanem legyen a ti beszédetek: Úgy úgy; nem nem; a mi pedig ezeken felül vagyon, a gonosztól vagyon.
\par 38 Hallottátok, hogy megmondatott: Szemet szemért és fogat fogért.
\par 39 Én pedig azt mondom néktek: Ne álljatok ellene a gonosznak, hanem a ki arczul üt téged jobb felõl, fordítsd felé a másik orczádat is.
\par 40 És a ki törvénykezni akar veled és elvenni a te alsó ruhádat, engedd oda néki a felsõt is.
\par 41 És a ki téged egy mértföldútra kényszerít, menj el vele kettõre.
\par 42 A ki tõled kér, adj néki; és a ki tõled kölcsön akar kérni, el ne fordulj attól.
\par 43 Hallottátok, hogy megmondatott: Szeresd felebarátodat és gyûlöld ellenségedet.
\par 44 Én pedig azt mondom néktek: Szeressétek ellenségeiteket, áldjátok azokat, a kik titeket átkoznak, jót tegyetek azokkal, a kik titeket gyûlölnek,  és imádkozzatok azokért, a kik háborgatnak és kergetnek titeket.
\par 45 Hogy legyetek a ti mennyei Atyátoknak fiai, a ki felhozza az õ napját mind a gonoszokra, mind a jókra, és esõt ád mind az igazaknak, mind a hamisaknak.
\par 46 Mert ha azokat szeretitek, a kik titeket szeretnek, micsoda jutalmát veszitek? Avagy a vámszedõk is nem ugyanazt cselekeszik-é?
\par 47 És ha csak a ti atyátokfiait köszöntitek, mit cselekesztek másoknál többet? Nemde a vámszedõk is nem azonképen cselekesznek-é?
\par 48 Legyetek azért ti tökéletesek, miként a ti mennyei Atyátok tökéletes.

\chapter{6}

\par 1 Vigyázzatok, hogy alamizsnátokat ne osztogassátok az emberek elõtt, hogy lássanak titeket; mert különben nem lesz jutalmatok a ti mennyei Atyátoknál.
\par 2 Azért mikor alamizsnát osztogatsz, ne kürtöltess magad elõtt, a hogy a képmutatók tesznek a zsinagógákban és az utczákon, hogy az emberektõl dícséretet nyerjenek. Bizony mondom néktek: elvették jutalmukat.
\par 3 Te pedig a mikor alamizsnát osztogatsz, ne tudja a te bal kezed, mit cselekszik a te jobb kezed;
\par 4 Hogy a te alamizsnád titkon legyen; és a te Atyád, a ki titkon néz, megfizet néked nyilván.
\par 5 És mikor imádkozol, ne légy olyan, mint a képmutatók, a kik a gyülekezetekben és az utczák szegeletein fenállva szeretnek imádkozni, hogy lássák õket az emberek. Bizony mondom néktek: elvették jutalmukat.
\par 6 Te pedig a mikor imádkozol, menj be a te belsõ szobádba, és ajtódat bezárva, imádkozzál a te Atyádhoz, a ki titkon van; és a te Atyád, a ki titkon néz, megfizet néked nyilván.
\par 7 És mikor imádkoztok, ne legyetek sok beszédûek, mint a pogányok, a kik azt gondolják, hogy az õ sok beszédükért hallgattatnak meg.
\par 8 Ne legyetek hát ezekhez hasonlók; mert jól tudja a ti Atyátok, mire van szükségetek, mielõtt kérnétek tõle.
\par 9 Ti azért így imádkozzatok: Mi Atyánk, ki vagy a mennyekben, szenteltessék meg a te neved;
\par 10 Jõjjön el a te országod; legyen meg a te akaratod, mint a mennyben, úgy a földön is.
\par 11 A mi mindennapi kenyerünket add meg nékünk ma.
\par 12 És bocsásd meg a mi vétkeinket, miképen mi is megbocsátunk azoknak, a kik ellenünk vétkeztek;
\par 13 És ne vígy minket kísértetbe, de szabadíts meg minket a gonosztól. Mert tiéd az ország és a hatalom és a dicsõség mind örökké. Ámen!
\par 14 Mert ha megbocsátjátok az embereknek az õ vétkeiket, megbocsát néktek is a ti mennyei Atyátok.
\par 15 Ha pedig meg nem bocsátjátok az embereknek az õ vétkeiket, a ti mennyei Atyátok sem bocsátja meg a ti vétkeiteket.
\par 16 Mikor pedig bõjtöltök, ne legyen komor a nézéstek, mint a képmutatóké, a kik eltorzítják arczukat, hogy lássák az emberek, hogy õk bõjtölnek. Bizony mondon néktek, elvették jutalmukat.
\par 17 Te pedig mikor bõjtölsz, kend meg a te fejedet, és a te orczádat mosd meg;
\par 18 Hogy ne az emberek lássák bõjtölésedet, hanem a te Atyád, a ki titkon van; és a te Atyád, a ki titkon néz, megfizet néked nyilván.
\par 19 Ne gyûjtsetek magatoknak kincseket a földön, hol a rozsda és a moly megemészti, és a hol a tolvajok kiássák és ellopják;
\par 20 Hanem gyûjtsetek magatoknak kincseket mennyben, a hol sem a rozsda, sem a moly meg nem emészti, és a hol a tolvajok ki nem ássák, sem el nem lopják.
\par 21 Mert a hol van a ti kincsetek, ott van a ti szívetek is.
\par 22 A test lámpása a szem. Ha azért a te szemed tiszta, a te egész tested világos lesz.
\par 23 Ha pedig a te szemed gonosz, a te egész tested sötét lesz. Ha azért a benned lévõ világosság sötétség: mekkora akkor a sötétség?!
\par 24 Senki sem szolgálhat két úrnak. Mert vagy az egyiket gyûlöli és a másikat szereti; vagy egyikhez ragaszkodik és a másikat megveti. Nem szolgálhattok Istennek és a Mammonnak.
\par 25 Azért azt mondom néktek: Ne aggodalmaskodjatok a ti éltetek felõl, mit egyetek és mit igyatok; sem a ti testetek felõl, mibe öltözködjetek. Avagy nem több-é az élet hogynem az eledel, és a test hogynem az öltözet?
\par 26 Tekintsetek az égi madarakra, hogy nem vetnek, nem aratnak, sem csûrbe nem takarnak; és a ti mennyei Atyátok eltartja azokat. Nem sokkal különbek vagytok-é azoknál?
\par 27 Kicsoda pedig az közületek, a ki aggodalmaskodásával megnövelheti termetét egy araszszal?
\par 28 Az öltözet felõl is mit aggodalmaskodtok? Vegyétek eszetekbe a mezõ liliomait, mi módon növekednek: nem munkálkodnak és nem fonnak;
\par 29 De mondom néktek, hogy Salamon minden dicsõségében sem öltözködött úgy, mint ezek közül egy.
\par 30 Ha pedig a mezõnek füvét, a mely ma van, és holnap kemenczébe vettetik, így ruházza az Isten; nem sokkal inkább-é titeket, ti kicsinyhitûek?
\par 31 Ne aggodalmaskodjatok tehát, és ne mondjátok: Mit együnk? vagy: Mit igyunk? vagy: Mivel ruházkodjunk?
\par 32 Mert mind ezeket a pogányok kérdezik. Mert jól tudja a ti mennyei Atyátok, hogy mind ezekre szükségetek van.
\par 33 Hanem keressétek elõször Istennek országát, és az õ igazságát;  és ezek mind megadatnak néktek.
\par 34 Ne aggodalmaskodjatok tehát a holnap felõl; mert a holnap majd aggodalmaskodik a maga dolgai felõl. Elég minden napnak a maga baja.

\chapter{7}

\par 1 Ne ítéljetek, hogy ne ítéltessetek.
\par 2 Mert a milyen ítélettel ítéltek, olyannal ítéltettek, és a milyen mértékkel mértek, olyannal mérnek néktek.
\par 3 Miért nézed pedig a szálkát, a mely a te atyádfia szemében van, a gerendát pedig, a mely a te szemedben van, nem veszed észre?
\par 4 Avagy mi módon mondhatod a te atyádfiának: Hadd vessem ki a szálkát a te szemedbõl; holott ímé, a te szemedben gerenda van?
\par 5 Képmutató, vesd ki elõbb a gerendát a te szemedbõl és akkor gondolj arra, hogy kivessed a szálkát a te atyádfiának szemébõl!
\par 6 Ne adjátok azt, a mi szent, az ebeknek, se gyöngyeiteket ne hányjátok a disznók elé, hogy meg ne tapossák azokat lábaikkal, és néktek fordulván, meg ne szaggassanak titeket.
\par 7 Kérjetek és adatik néktek; keressetek és találtok; zörgessetek és megnyittatik néktek.
\par 8 Mert a ki kér, mind kap; és a ki keres, talál; és a zörgetõnek megnyittatik.
\par 9 Avagy ki az az ember közületek, a ki, ha az õ fia kenyeret kér tõle, követ ád néki?
\par 10 És ha halat kér, vajjon kígyót ád-e néki?
\par 11 Ha azért ti gonosz létetekre tudtok a ti fiaitoknak jó ajándékokat adni, mennyivel inkább ád a ti mennyei Atyátok jókat azoknak, a kik kérnek tõle?!
\par 12 A mit akartok azért, hogy az emberek ti veletek cselekedjenek, mindazt ti is úgy cselekedjétek azokkal; mert ez a törvény és a próféták.
\par 13 Menjetek be a szoros kapun. Mert tágas az a kapu és széles az az út, a mely a veszedelemre visz, és sokan vannak, a kik azon járnak.
\par 14 Mert szoros az a kapu és keskeny az az út, a mely az életre visz, és kevesen vannak, a kik megtalálják azt.
\par 15 Õrizkedjetek pedig a hamis prófétáktól, a kik juhoknak ruhájában jõnek hozzátok, de  belõl ragadozó farkasok.
\par 16 Gyümölcseikrõl ismeritek meg õket. Vajjon a tövisrõl szednek-é szõlõt, vagy a bojtorjánról fügét?
\par 17 Ekképen minden jó fa jó gyümölcsöt terem; a romlott fa pedig rossz gyümölcsöt terem.
\par 18 Nem teremhet jó fa rossz gyümölcsöt; romlott fa sem teremhet jó gyümölcsöt.
\par 19 Minden fa, a mely nem terem jó gyümölcsöt, kivágattatik és tûzre vettetik.
\par 20 Azért az õ gyümölcseikrõl ismeritek meg õket.
\par 21 Nem minden, a ki ezt mondja nékem: Uram! Uram! megyen be a mennyek országába; hanem a ki cselekszi az én mennyei Atyám akaratát.
\par 22 Sokan mondják majd nékem ama napon: Uram! Uram! nem a te nevedben prófétáltunk-é, és nem a te nevedben ûztünk-é ördögöket, és nem cselekedtünk-é sok hatalmas dolgot a te nevedben?
\par 23 És akkor vallást teszek majd nékik: Sohasem ismertelek titeket; távozzatok tõlem, ti gonosztevõk.
\par 24 Valaki azért hallja én tõlem e beszédeket, és megcselekszi azokat, hasonlítom azt a bölcs emberhez, a ki a kõsziklára építette az õ házát:
\par 25 És ömlött az esõ, és eljött az árvíz, és fújtak a szelek, és beleütköztek abba a házba; de nem dõlt össze: mert a kõsziklára építtetett.
\par 26 És valaki hallja én tõlem e beszédeket, és nem cselekszi meg azokat, hasonlatos lesz a bolond emberhez, a ki a fövényre építette házát:
\par 27 És ömlött az esõ, és eljött az árvíz, és fújtak a szelek, és beleütköztek abba a házba; és összeomlott: és nagy lett annak romlása.
\par 28 És lõn, mikor elvégezte Jézus e beszédeket, álmélkodik vala a sokaság az õ tanításán:
\par 29 Mert úgy tanítja vala õket, mint a kinek hatalma van, és nem úgy, mint az írástudók.

\chapter{8}

\par 1 Mikor leszállott vala a hegyrõl, nagy sokaság követé õt.
\par 2 És ímé eljövén egy bélpoklos, leborula elõtte, mondván: Uram, ha akarod, megtisztíthatsz engem.
\par 3 És kinyújtván kezét, megilleté õt Jézus, mondván: Akarom, tisztulj meg. És azonnal eltisztult annak poklossága.
\par 4 És monda néki Jézus: Meglásd, senkinek se szólj. Hanem eredj,  mutasd meg magadat a papnak, és vidd fel az ajándékot, a melyet Mózes rendelt, bizonyságul nékik.
\par 5 Mikor pedig beméne Jézus Kapernaumba, egy százados méne hozzá, kérvén õt,
\par 6 És ezt mondván: Uram, az én szolgám otthon gutaütötten fekszik, és nagy kínokat szenved.
\par 7 És monda néki Jézus: Elmegyek és meggyógyítom õt.
\par 8 És felelvén a százados, monda: Uram, nem vagyok méltó, hogy az én hajlékomban jõjj; hanem csak szólj egy szót, és meggyógyul az én szolgám.
\par 9 Mert én is hatalmasság alá vetett ember vagyok, és vannak alattam vitézek; és mondom egyiknek: Eredj el, és elmegy; és a másiknak: Jöszte, és eljõ; és az én szolgámnak: Tedd ezt, és megteszi.
\par 10 Jézus pedig, a mikor ezt hallá, elcsodálkozék, és monda az õt követõknek: Bizony mondom néktek, még az Izráelben sem találtam ilyen nagy hitet.
\par 11 De mondom néktek, hogy sokan eljõnek napkeletrõl és napnyugatról, és letelepednek  Ábrahámmal, Izsákkal és Jákóbbal a mennyek országában:
\par 12 Ez ország fiai pedig kivettetnek a külsõ sötétségre; holott lészen sírás és fogaknak csikorgatása.
\par 13 És monda Jézus a századosnak: Eredj el, és legyen néked a te hited szerint. És meggyógyult annak szolgája abban az órában.
\par 14 És bemenvén Jézus a Péter házába, látá, hogy annak napa fekszik és lázas.
\par 15 És illeté annak kezét, és elhagyta õt a láz; és fökele, és szolgála nékik.
\par 16 Az est beálltával pedig vivének hozzá sok ördöngõst, és egy szóval kiûzé a tisztátalan lelkeket, és meggyógyít vala minden beteget;
\par 17 Hogy beteljesedjék, a mit Ésaiás próféta mondott, így szólván: Õ vette el a mi erõtlenségünket, és õ hordozta a mi betegségünket.
\par 18 Látván pedig Jézus a nagy sokaságot maga körül, parancsolá, hogy menjenek a túlsó partra.
\par 19 És hozzámenvén egy írástudó, monda néki: Mester, követlek téged, akárhova mégy.
\par 20 És monda néki Jézus: A rókáknak vagyon barlangjok és az égi madaraknak fészkük; de az ember Fiának nincs hová fejét lehajtani.
\par 21 Egy másik pedig az õ tanítványai közül monda néki: Uram, engedd meg nékem, hogy elõbb elmenjek és eltemessem az én atyámat.
\par 22 Jézus pedig monda néki: Kövess engem, és hagyd, hogy a halottak temessék el az õ halottaikat.
\par 23 És mikor a hajóra szállt vala, követék õt az õ tanítványai.
\par 24 És ímé nagy háborgás lõn a tengeren, annyira, hogy a hajót elborítják vala a hullámok; õ pedig aluszik vala.
\par 25 És az õ tanítványai hozzámenvén, felkölték õt, mondván: Uram, ments meg minket; mert elveszünk.
\par 26 És monda nékik: Mit féltek, óh kicsinyhitûek? Ekkor fölkelvén, megdorgálá a szeleket és a tengert, és lõn nagy csendesség.
\par 27 Az emberek pedig elcsodálkozának, mondván: Kicsoda ez, hogy mind a szelek, mind a tenger engednek néki.
\par 28 És a mikor eljutott vala a túlsó partra, a Gadarénusok tartományába, két ördöngõs ment eléje, a sírboltokból kijövén, igen kegyetlenek, annyira, hogy senki sem mer vala elmenni azon az úton.
\par 29 És ímé kiáltának mondván: Mi közünk te veled Jézus, Istennek fia? Azért jöttél ide, hogy idõ elõtt  meggyötörj minket?
\par 30 Tõlük távol pedig egy nagy disznónyáj legelészik vala.
\par 31 Az ördögök pedig kérik vala õt mondván: Ha kiûzesz minket, engedd meg nékünk, hogy ama disznónyájba mehessünk!
\par 32 És monda nékik: Menjetek. Azok pedig kimenvén, menének a disznónyájba: és ímé az egész disznónyáj a meredekrõl a tengerbe rohana, és oda vesze a vízben.
\par 33 A pásztorok pedig elfutának, és bemenvén a városba hírré adának mindent, azokat is, a mik az ördöngõsökkel történtek vala.
\par 34 És ímé az egész város kiméne Jézus elébe; és mihelyt meglátták, kérék õt, hogy távozzék az õ határukból.

\chapter{9}

\par 1 És hajóra szállva átkele, és méne a maga városába.
\par 2 És ímé hoznak vala hozzá egy ágyban fekvõ gutaütött embert. És látva Jézus azoknak hitét, monda a gutaütöttnek: Bízzál fiam! Megbocsáttattak néked a te bûneid.
\par 3 És ímé némelyek az írástudók közül mondának magukban: Ez káromlást szól.
\par 4 És Jézus, látva az õ gondolataikat, monda: Miért gondoltok gonoszt a ti szívetekben?
\par 5 Mert mi könnyebb, ezt mondani-é: Megbocsáttattak néked a te bûneid; vagy ezt mondani: Kelj föl és járj?
\par 6 Hogy pedig megtudjátok, hogy az ember Fiának van hatalma a földön a bûnöket megbocsátani (ekkor monda a gutaütöttnek): Kelj föl, vedd a te ágyadat, és eredj haza.
\par 7 És az felkelvén, haza méne.
\par 8 A sokaság pedig ezt látván, elálmélkodék, és dicsõíté az Istent, hogy ilyen hatalmat adott az embereknek.
\par 9 És mikor Jézus onnét tovább méne, láta egy embert ülni a vámszedõ helyen a kinek Máté volt a neve, és monda néki: Kövess engem! És az felkelvén, követé õt.
\par 10 És lõn, a mikor õ letelepedék a házban, ímé sok vámszedõ és bûnös jött oda és letelepedtek Jézussal és az õ tanítványaival az asztalhoz.
\par 11 És látva ezt a farizeusok, mondának az õ tanítványainak: Miért eszik ez a ti Mesteretek a vámszedõkkel és bûnösökkel együtt?
\par 12 Jézus pedig ezt hallván, monda nékik: Nem az egészségeseknek van szüksége orvosra, hanem a betegeknek.
\par 13 Elmenvén pedig tanuljátok meg, mi az: Irgalmasságot akarok és nem áldozatot. Mert nem az igazakat hivogatni jöttem,  hanem a bûnösöket a megtérésre.
\par 14 Akkor a János tanítványai jövének hozzá, mondván: Miért hogy mi és a farizeusok sokat bõjtölünk, a te tanítványaid pedig nem bõjtölnek?
\par 15 És monda nékik Jézus: Vajjon szomorkodhatik-é a násznép a míg velök van a võlegény? de eljõnek a napok, a mikor elvétetik tõlök a võlegény, és akkor bõjtölni fognak.
\par 16 Senki sem vet pedig új posztóból foltot az ócska ruhára; mert a mi azt kitoldaná, még elszakít a ruhából és nagyobb szakadás lesz.
\par 17 Új bort sem töltenek ó tömlõkbe; máskülönben a tömlõk szétszakadoznak, és a bor kiömöl, a tömlõk is elvesznek; hanem az új bort új tömlõkbe töltik, és mindkettõ megmarad.
\par 18 Mikor ezeket mondá nékik, ímé egy fõember eljövén leborula elõtte, mondván: Az én leányom épen most halt meg; de jer, vesd reá kezedet, és megelevenedik.
\par 19 És felkelvén Jézus követé õt tanítványaival együtt.
\par 20 És ímé, egy asszony, a ki tizenkét év óta vérfolyásban szenved vala, hozzájárulván hátulról, illeté az õ ruhájának szegélyét.
\par 21 Mert ezt mondja vala magában: Ha csak ruháját illetem is, meggyógyulok.
\par 22 Jézus pedig megfordulván és reá tekintvén, monda: Bízzál leányom; a te hited megtartott téged. És meggyógyult az asszony abban az órában.
\par 23 És Jézus a fõember házához érvén, látván a sípolókat és a tolongó sokaságot,
\par 24 Monda nékik: Menjetek el innen, mert a leányzó nem halt meg, hanem aluszik. És kinevették õt.
\par 25 Mikor pedig a sokaság eltávolíttaték, bemenvén, megfogá annak kezét, és a leányzó felkelt.
\par 26 És elterjede ez a hír abban az egész tartományban.
\par 27 És mikor Jézus tovább ment onnét, két vak követé õt, kiáltozva és ezt mondva: Könyörülj rajtunk, Dávidnak fia!
\par 28 Mikor pedig beméne a házba, oda menének hozzá a vakok, és monda nékik Jézus: Hiszitek-é, hogy én azt megcselekedhetem? Mondának néki: Igen, Uram.
\par 29 Akkor illeté az õ szemeiket, mondván: Legyen néktek a ti hitetek szerint.
\par 30 És megnyilatkozának azoknak szemei; és rájok parancsola Jézus, mondván: Meglássátok, senki meg ne tudja!
\par 31 Azok pedig kimenvén, elterjeszték az õ hírét abban az egész tartományban.
\par 32 Mikor pedig azok elmentek vala, ímé egy ördöngõs néma embert hozának néki.
\par 33 És az ördögöt kiûzvén, megszólalt a néma; és a sokaság csudálkozik vala, mondván: Soha nem láttak ilyet Izráelben!
\par 34 A farizeusok pedig ezt mondják vala: Az ördögök fejedelme által ûzi ki az ördögöket.
\par 35 És körüljárja vala Jézus a városokat mind, és a falvakat, tanítván azoknak zsinagógáiban, és hirdetvén az Isten országának evangyéliomát, és gyógyítván mindenféle betegséget és mindenféle erõtelenséget a nép között.
\par 36 Mikor pedig látta vala a sokaságot, könyörületességre indula rajtok, mert el voltak gyötörve és szétszórva, mint a pásztor nélkül  való juhok.
\par 37 Akkor monda az õ tanítványainak: Az aratni való sok, de a munkás kevés.
\par 38 Kérjétek azért az aratásnak Urát, hogy küldjön munkásokat az õ aratásába.

\chapter{10}

\par 1 És elõszólítván tizenkét tanítványát, hatalmat ada nékik a tisztátalan lelkek felett, hogy kiûzzék azokat, és gyógyítsanak minden betegséget és minden erõtelenséget.
\par 2 A tizenkét apostol nevei pedig ezek: Elsõ Simon, a kit Péternek hívnak, és András, az õ testvére; Jakab, a Zebedeus fia, és János az õ testvére;
\par 3 Filep és Bertalan; Tamás és Máté, a vámszedõ; Jakab, az Alfeus fia, és Lebbeus, a kit Taddeusnak hívtak;
\par 4 Simon a kananita, és Judás, az Iskariotes, a ki el is árulta õt.
\par 5 Ezt a tizenkettõt küldé ki Jézus, és megparancsolá nékik, mondván: Pogányok útjára ne menjetek, és Samaritánusok városába ne menjetek be;
\par 6 Hanem menjetek inkább Izráel házának eltévelyedett juhaihoz.
\par 7 Elmenvén pedig prédikáljatok, mondván: Elközelített a mennyeknek országa.
\par 8 Betegeket gyógyítsatok, poklosokat tisztítsatok, halottakat támasszatok, ördögöket ûzzetek. Ingyen vettétek, ingyen adjátok.
\par 9 Ne szerezzetek aranyat, se ezüstöt, se réz-pénzt a ti erszényetekbe,
\par 10 Se útitáskát, se két ruhát, se sarut, se pálczát; mert  méltó a munkás az õ táplálékára.
\par 11 A mely városba vagy faluba pedig bementek, tudakozzátok meg, ki abban méltó; és ott maradjatok, a míg tovább mehettek.
\par 12 Ha pedig bementek a házba, köszöntsétek azt.
\par 13 És ha méltó a ház, szálljon a ti békességtek reá; ha pedig nem méltó, a ti békességtek rátok térjen vissza.
\par 14 És ha valaki nem fogad be titeket, és nem hallgatja a ti beszédeteket, mikor kimentek abból a házból, vagy városból, lábaitok porát is verjétek le.
\par 15 Bizony mondom néktek: Az ítélet napján könnyebb lesz a Sodoma és Gomora földjének dolga, mint annak a városnak.
\par 16 Ímé, én elbocsátlak titeket, mint juhokat a farkasok közé;  legyetek azért okosak mint a kígyók, és szelidek mint a galambok.
\par 17 De óvakodjatok az emberektõl; mert törvényszékekre adnak titeket és az õ gyülekezeteikben megostoroznak titeket;
\par 18 És helytartók és királyok elé visznek titeket érettem, bizonyságul õ magoknak és a pogányoknak.
\par 19 De mikor átadnak titeket, ne aggodalmaskodjatok, mi módon vagy mit szóljatok; mert megadatik néktek abban az órában, mit mondjatok.
\par 20 Mert nem ti vagytok, a kik szóltok, hanem a ti Atyátoknak Lelke az, a ki szól ti bennetek.
\par 21 Halálra adja pedig testvér testvérét, atya gyermekét; támadnak magzatok szüleik ellen, és megöletik õket.
\par 22 És gyûlöletesek lesztek, mindenki elõtt az én nevemért; de a ki mindvégig megáll, az megtartatik.
\par 23 Mikor pedig abban a városban üldöznek titeket, szaladjatok a másikba. Mert bizony mondom néktek: be sem járjátok Izráel városait, míg az embernek Fia eljövend.
\par 24 Nem fölebbvaló a tanítvány a tanítónál, sem a szolga az õ uránál.
\par 25 Elég a tanítványnak, ha olyan mint a mestere, és a szolga mint az õ Ura. Ha a házigazdát Belzebubnak hívták, mennyivel inkább az õ házanépét?!
\par 26 Azért ne féljetek tõlök. Mert nincs oly rejtett dolog, a mi napfényre ne jõne; és oly titok, a mi ki ne tudódnék.
\par 27 A mit néktek a sötétben mondok, a világosságban mondjátok; és a mit fülbe súgva hallotok, a háztetõkrõl hirdessétek.
\par 28 És ne féljetek azoktól, a kik a testet ölik meg, a lelket pedig meg nem ölhetik; hanem attól féljetek inkább, a ki mind a  lelket, mind a testet elvesztheti a gyehennában.
\par 29 Nemde, két verebecskét meg lehet venni egy kis fillérért? És egy sem esik azok közül a földre a ti Atyátok akarata nélkül!
\par 30 Néktek pedig még a fejetek hajszálai is mind számon vannak.
\par 31 Ne féljetek azért; ti sok verebecskénél drágábbak vagytok.
\par 32 Valaki azért vallást tesz én rólam az emberek elõtt, én is vallást teszek arról az én mennyei Atyám elõtt;
\par 33 A ki pedig megtagad engem az emberek elõtt, én is megtagadom azt az én mennyei Atyám elõtt.
\par 34 Ne gondoljátok, hogy azért jöttem, hogy békességet bocsássak e földre; nem azért jöttem, hogy békességet bocsássak, hanem hogy fegyvert.
\par 35 Mert azért jöttem, hogy meghasonlást támaszszak az ember és az õ atyja, a leány és az õ anyja, a meny és az õ napa közt;
\par 36 És hogy az embernek ellensége legyen az õ házanépe.
\par 37 A ki inkább szereti atyját és anyját, hogynem engemet, nem méltó én hozzám; és a ki inkább szereti fiát és leányát, hogynem engemet, nem méltó én hozzám.
\par 38 És a ki föl nem veszi az õ keresztjét és úgy nem követ engem, nem méltó én hozzám.
\par 39 A ki megtalálja az õ életét, elveszti azt; és a ki elveszti az õ életét én érettem, megtalálja azt.
\par 40 A ki titeket befogad, engem fogad be; és a ki engem befogad, azt fogadja be, a ki engem küldött.
\par 41 A ki befogadja a prófétát próféta nevében, prófétának jutalmát veszi; és aki befogadja az igazat igaznak nevében, igaznak jutalmát veszi;
\par 42 És a ki inni ád egynek e kicsinyek közül, csak egy pohár hideg vizet tanítvány nevében, bizony mondom néktek, el nem vesztheti jutalmát.

\chapter{11}

\par 1 És lõn, mikor elvégezé Jézus a tizenkét tanítványnak adott utasítást, elméne onnan, hogy tanítson és prédikáljon azoknak városaiban.
\par 2 János pedig, mikor meghallotta a fogságban a Krisztus cselekedeteit, elküldvén kettõt az õ  tanítványai közül,
\par 3 Monda néki: Te vagy-é az, a ki eljövendõ, vagy mást várjunk?
\par 4 És felelvén Jézus, monda nékik: Menjetek el és jelentsétek Jánosnak, a miket hallotok és láttok:
\par 5 A vakok látnak, és a sánták járnak; a poklosok megtisztulnak és a siketek hallanak; a halottak föltámadnak, és a szegényeknek evangyéliom hirdettetik;
\par 6 És boldog, a ki én bennem meg nem botránkozik.
\par 7 Mikor pedig azok elmentek vala, szólni kezde Jézus a sokaságnak Jánosról: Mit látni mentetek ki a pusztába? Nádszálat-é, a mit a szél hajtogat?
\par 8 Hát mit látni mentetek ki? Puha ruhába öltözött embert-é? Ímé a kik puha ruhákat viselnek, a királyok palotáiban vannak.
\par 9 Hát mit látni mentetek ki? Prófétát-é? Bizony, mondom néktek, prófétánál is nagyobbat!
\par 10 Mert õ az, a kirõl meg van írva: Ímé én elküldöm az én követemet a te orczád elõtt, a ki megkészíti elõtted a te útadat.
\par 11 Bizony mondom néktek: az asszonyoktól szülöttek között nem támadott nagyobb Keresztelõ Jánosnál; de a ki legkisebb a mennyeknek országában, nagyobb nálánál.
\par 12 A Keresztelõ János idejétõl fogva pedig mind mostanig erõszakoskodnak a mennyek országáért, és az erõszakoskodók ragadják el azt.
\par 13 Mert a próféták mindnyájan és a törvény Jánosig prófétáltak vala.
\par 14 És, ha be akarjátok venni, Illés õ, a ki eljövendõ vala.
\par 15 A kinek van füle a hallásra, hallja.
\par 16 De kihez hasonlítsam ezt a nemzetséget? Hasonlatos a gyermekekhez, a kik a piaczon ülnek, és kiáltoznak az õ társaiknak,
\par 17 És ezt mondják: Sípoltunk néktek, és nem tánczoltatok; siralmas énekeket énekeltünk néktek, és nem sírtatok.
\par 18 Mert eljött János, a ki sem eszik, sem iszik, és azt mondják: Ördög van benne.
\par 19 Eljött az embernek Fia, a ki eszik és iszik, és ezt mondják: Ímé a nagy étü és részeges ember, a vámszedõk és bûnösök barátja! És igazoltaték a bölcseség az õ fiaitól.
\par 20 Ekkor elkezdé szemökre hányni ama városoknak, a melyekben legtöbb csodái lõnek, hogy nem tértek vala meg:
\par 21 Jaj néked Korazin! Jaj néked Bethsaida! Mert ha Tirusban és Sidonban történnek vala azok a csodák, a melyek bennetek lõnek,  rég megtértek volna gyászruhában és hamuban.
\par 22 De mondom néktek: Tirusnak és Sidonnak könnyebb dolga lesz az ítélet napján, hogynem néktek.
\par 23 Te is Kapernaum, a ki az égig felmagasztaltattál, a pokolig fogsz megaláztatni; mert ha Sodomában töténnek vala azok a csodák, a melyek te benned lõnek, mind e mai napig megmaradt volna.
\par 24 De mondom néktek, hogy Sodoma földének könnyebb dolga lesz az ítélet napján, hogynem néked.
\par 25 Abban az idõben szólván Jézus, monda: Hálákat adok néked, Atyám, mennynek és földnek Ura, hogy elrejtetted ezeket a bölcsek és az értelmesek elõl, és a kisdedeknek megjelentetted.
\par 26 Igen, Atyám, mert így volt kedves te elõtted.
\par 27 Mindent nékem adott át az én Atyám, és senki sem ismeri a Fiút, csak az Atya; az Atyát sem ismeri senki, csak a Fiú, és a kinek a Fiú akarja megjelenteni.
\par 28 Jõjjetek én hozzám mindnyájan, a kik megfáradtatok és megterheltettetek, és én megnyugosztlak titeket.
\par 29 Vegyétek föl magatokra az én igámat, és tanuljátok meg tõlem, hogy én szelid és alázatos szívû vagyok: és nyugalmat találtok a ti lelkeiteknek.
\par 30 Mert az én igám gyönyörûséges, és az én terhem könnyû.

\chapter{12}

\par 1 Abban az idõben a vetéseken át haladt Jézus szombatnapon; tanítványai pedig megéheztek, és kezdték a kalászokat tépni és enni.
\par 2 Látván pedig ezt a farizeusok, mondának néki: Ímé a te tanítványaid azt cselekszik, a mit nem szabad szombatnapon cselekedni.
\par 3 Õ pedig monda nékik: Nem olvastátok-é, mit cselekedett Dávid, mikor megéhezett vala õ és a kik vele valának?
\par 4 Hogyan ment be az Isten házába, és ette meg a szentelt kenyereket, a melyeket nem vala szabad megennie néki, sem azoknak, a kik õ vele valának, hanem csak  a papoknak?
\par 5 Vagy nem olvastátok-é a törvényben, hogy szombatnapon megtörik a papok a szombatot a templomban és nem vétkeznek?
\par 6 Mondom pedig néktek, hogy a templomnál nagyobb van itt.
\par 7 Ha pedig tudnátok, mi ez: Irgalmasságot akarok és nem áldozatot, nem kárhoztattátok volna az ártatlanokat.
\par 8 Mert a szombatnak is Ura az embernek Fia.
\par 9 És távozván onnan, méne az õ zsinagógájukba.
\par 10 És ímé, vala ott egy elszáradt kezû ember. És megkérdék õt, mondván: Ha szabad-é szombatnapon gyógyítani? hogy vádolhassák õt.
\par 11 Õ pedig monda nékik: Kicsoda közületek az az ember, a kinek van egy juha, és ha az szombatnapon a verembe esik, meg nem ragadja és ki nem vonja azt?
\par 12 Mennyivel drágább pedig az ember a juhnál! Szabad tehát szombatnapon jót cselekedni.
\par 13 Akkor monda annak az embernek: Nyújtsd ki a kezedet. És kinyújtá, és olyan éppé lõn, mint a másik.
\par 14 A farizeusok pedig kimenvén, tanácsot tartának ellene, hogyan veszíthetnék el õt.
\par 15 Jézus pedig észrevévén ezt, eltávozék onnan. És követé õt nagy sokaság, és õ meggyógyítja vala mindnyájokat;
\par 16 És megfenyegeté õket, hogy õt ismertté ne tegyék;
\par 17 Hogy beteljesedjék Ésaiás próféta mondása, a ki így szólt:
\par 18 Ímé az én szolgám, a kit választottam; az én szerelmesem, a kiben az én lelkem kedvét lelé; lelkemet adom õ belé, és ítéletet hirdet a pogányoknak.
\par 19 Nem verseng, és nem kiált; az utczákon senki nem hallja szavát.
\par 20 A megrepedezett nádat nem töri el, és a pislogó gyertyabelet nem oltja ki, mígnem diadalomra viszi az ítéletet.
\par 21 És az õ nevében reménykednek majd a pogányok.
\par 22 Akkor egy vak és néma ördöngõst hoztak õ eléje; és meggyógyítá azt, annyira, hogy a vak és néma mind beszél, mind lát vala.
\par 23 És elálmélkodék az egész sokaság, és monda: Vajjon nem ez-é Dávidnak ama Fia?
\par 24 A farizeusok pedig ezt hallván, mondának: Ez nem ûzi ki az ördögöket, hanemha Belzebubbal, az ördögök fejedelmével.
\par 25 Jézus pedig, tudva az õ gondolataikat, monda nékik: Minden ország a mely magával meghasonlik, elpusztul; és egy város vagy háznép sem állhat meg, a mely meghasonlik magával.
\par 26 Ha pedig a Sátán a Sátánt ûzi ki, önmagával hasonlott meg; mimódon állhat meg tehát az õ országa?
\par 27 És ha én Belzebub által ûzöm ki az ördögöket, a ti fiaitok ki által ûzik ki? Azért õk magok lesznek a ti bíráitok.
\par 28 Ha pedig én Istennek Lelke által ûzöm ki az ördögöket, akkor kétség nélkül elérkezett hozzátok az Isten országa.
\par 29 Avagy mi módon mehet be valaki a hatalmasnak házába és rabolhatja el annak kincseit, hanemha megkötözi elõbb a hatalmast és akkor rabolja ki annak házát?
\par 30 A ki velem nincsen, ellenem van; és a ki velem nem gyûjt, tékozol.
\par 31 Azt mondom azért néktek: Minden bûn és káromlás megbocsáttatik az embereknek; de a Lélek káromlása nem bocsáttatik meg az embereknek.
\par 32 Még a ki az ember Fia ellen szól, annak is megbocsáttatik; de a ki a Szent Lélek ellen szól, annak sem ezen, sem a más világon meg nem bocsáttatik.
\par 33 Vagy legyetek jó fák, és teremjetek jó gyümölcsöt, vagy legyetek romlott fák, és teremjetek romlott gyümölcsöt; mert gyümölcsérõl ismerik meg a fát.
\par 34 Mérges kígyóknak fajzatai, mi módon szólhattok jókat, holott gonoszak vagytok? Mert a szívnek teljességébõl szól a száj.
\par 35 A jó ember az õ szívének jó kincseibõl hozza elõ a jókat; és a gonosz ember az õ szívének gonosz kincseibõl hozza elõ a gonoszokat.
\par 36 De mondom néktek: Minden hivalkodó beszédért, a mit beszélnek az emberek, számot adnak majd az ítélet napján.
\par 37 Mert a te beszédidbõl ismertetel igaznak, és a te beszédidbõl ismertetel hamisnak.
\par 38 Ekkor felelének néki némelyek az írástudók és farizeusok közül, mondván: Mester, jelt akarnánk látni te tõled.
\par 39 Õ pedig felelvén, monda nékik: E gonosz és parázna nemzetség jelt kiván; és nem adatik jel néki, hanemha Jónás prófétának jele.
\par 40 Mert a miképen Jónás három éjjel és három nap volt a czethal gyomrában, azonképen az embernek Fia is három nap és három éjjel lesz a föld gyomrában.
\par 41 Ninive férfiai az ítéletkor együtt támadnak majd fel ezzel a nemzetséggel, és kárhoztatják ezt: mivelhogy õk megtértek a Jónás prédikálására; és ímé nagyobb van itt Jónásnál.
\par 42 Délnek királyné asszonya felkél majd az ítéletkor e nemzetséggel együtt, és kárhoztatja ezt: mert õ eljött a földnek szélérõl, hogy hallhassa a Salamon böcseségét; és ímé, nagyobb van itt Salamonnál.
\par 43 Mikor pedig a tisztátalan lélek kimegy az emberbõl, víz nélkül való helyeken jár, nyugalmat keresve, és nem talál:
\par 44 Akkor ezt mondja: Visszatérek az én házamba, a honnét kijöttem. És oda menvén, üresen, kisöpörve és fölékesítve találja azt.
\par 45 Akkor elmegy és vesz maga mellé más hét lelket, gonoszabbakat õ magánál, és bemenvén, ott lakoznak; és ennek az embernek utolsó állapotja gonoszabb lesz az elsõnél. Így lesz ezzel a gonosz nemzetséggel is.
\par 46 Mikor pedig még szóla a sokaságnak, ímé az õ anyja és az õ testvérei állanak vala odakünn, akarván õ vele szólni.
\par 47 És monda néki valaki: Ímé a te anyád és testvéreid odakünn állanak, és szólni akarnak veled.
\par 48 Õ pedig felelvén, monda a hozzá szólónak: Kicsoda az én anyám; és kik az én testvéreim?
\par 49 És kinyujtván kezét az õ tanítványaira, monda: Ímé az én anyám és az én testvéreim!
\par 50 Mert a ki cselekszi az én mennyei Atyám akaratát, az nékem fitestvérem, nõtestvérem és anyám.

\chapter{13}

\par 1 Azon a napon kimenvén Jézus a házból, leüle a tenger mellett.
\par 2 És nagy sokaság gyülekezék õ hozzá, annyira, hogy õ a hajóba méne leülni; az egész sokaság pedig a parton áll vala.
\par 3 És sokat beszéle nékik példázatokban, mondván: Ímé kiméne a magvetõ vetni,
\par 4 És a mikor õ vet vala, némely mag az útfélre esék; és eljövén a madarak, elkapdosák azt.
\par 5 Némely pedig a köves helyre esék, a hol nem sok földje vala; és hamar kikele, mivelhogy nem vala mélyen a földben.
\par 6 De mikor a nap felkelt, elsüle; és mivelhogy gyökere nem vala, elszáradott.
\par 7 Némely pedig a tövisek közé esék, és a tövisek felnevekedvén, megfojták azt.
\par 8 Némely pedig a jó földbe esék, és gyümölcsöt terme, némely száz annyit, némely hatvan annyit, némely pedig harmincz annyit.
\par 9 A kinek van füle a hallásra, hallja.
\par 10 A tanítványok pedig hozzámenvén, mondának néki: Miért szólasz nékik példázatokban?
\par 11 Õ pedig felelvén, monda nékik: Mert néktek megadatott, hogy érthessétek a mennyek országának titkait, ezeknek pedig nem adatott meg.
\par 12 Mert a kinek van, annak adatik, és bõvölködik; de a kinek nincs, az is elvétetik tõle, a mije van.
\par 13 Azért szólok velök példázatokban, mert látván nem látnak, és hallván nem hallanak, sem nem értenek.
\par 14 És beteljesedék rajtok Ésaiás jövendölése, a mely ezt mondaj: Hallván halljatok, és ne értsetek; és látván lássatok, és ne ismerjetek:
\par 15 Mert megkövéredett e népnek szíve, és füleikkel nehezen hallottak, és szemeiket behunyták; hogy valami módon ne lássanak szemeikkel, és ne halljanak füleikkel, és ne értsenek szívükkel, és meg ne térjenek, és meg ne gyógyítsam õket.
\par 16 A ti szemeitek pedig boldogok, hogy látnak; és a ti füleitek, hogy hallanak.
\par 17 Mert bizony mondom néktek, hogy sok próféta és igaz kívánta látni, a miket ti láttok, és nem látták; és hallani, a miket ti hallotok, és nem hallották.
\par 18 Ti halljátok meg azért a magvetõ példázatát.
\par 19 Ha valaki hallja az ígét a mennyeknek országáról és nem érti, eljõ a gonosz és elkapja azt, a mi annak szívébe vettetett vala. Ez az, a mely az útfélre esett.
\par 20 A mely pedig a köves helyre esett, ez az, a ki hallja az ígét, és mindjárt örömmel fogadja;
\par 21 De nincs gyökere benne, hanem csak ideig való; mihelyt pedig nyomorgatás vagy üldözés támad az íge miatt, azonnal megbotránkozik.
\par 22 A mely pedig a tövisek közé esett, ez az, a ki hallja az ígét, de e világnak gondja és a gazdagságnak csalárdsága elfojtja az ígét, és gyümölcsöt nem terem.
\par 23 A mely pedig a jó földbe esett, ez az, a ki hallja és érti az igét; a ki gyümöcsöt is terem, és terem némely száz annyit, némely hatvan annyit, némely pedig harmincz annyit.
\par 24 Más példázatot is adott eléjök, mondván: Hasonlatos a mennyeknek országa az emberhez, a ki az õ földébe jó magot vetett;
\par 25 De mikor az emberek alusznak vala, eljöve az õ ellensége és konkolyt vete a búza közé, és elméne.
\par 26 Mikor pedig felnevekedék a vetés, és gyümölcsöt terme, akkor meglátszék a konkoly is.
\par 27 A gazda szolgái pedig elõállván, mondának néki: Uram, avagy nem tiszta magot vetettél-e a te földedbe? honnan van azért benne a konkoly?
\par 28 Õ pedig monda nékik: Valamely ellenség cselekedte azt. A szolgák pedig mondának néki: Akarod-é tehát, hogy elmenvén, összeszedjük azokat?
\par 29 Õ pedig monda: Nem. Mert a mikor összeszeditek a konkolyt, azzal együtt netalán a búzát is kiszaggatjátok.
\par 30 Hagyjátok, hogy együtt nõjjön mind a kettõ az aratásig, és az aratás idején azt mondom majd az aratóknak: Szedjétek össze elõször a konkolyt, és kössétek kévékbe, hogy megégessétek; a búzát pedig takarítsátok az én csûrömbe.
\par 31 Más példázatot is adott eléjök, mondván: Hasonlatos a mennyeknek országa a mustármaghoz, a melyet vévén az ember, elvete az õ mezejében;
\par 32 A mely kisebb ugyan minden magnál; de a mikor felnõ, nagyobb a veteményeknél, és fává lesz, annyira, hogy reá szállanak az égi madarak, és fészket raknak ágain.
\par 33 Más példázatot is mondott nékik: Hasonlatos a mennyeknek országa a kovászhoz, a melyet vévén az asszony, három mércze lisztbe elegyíte, mígnem az egész megkele.
\par 34 Mind ezeket példázatokban mondá Jézus a sokaságnak, és példázat nélkül semmit sem szóla nékik,
\par 35 Hogy beteljék a mit a próféta szólott, mondván: Megnyitom az én számat példázatokra; és kitárom, amik e világ alapítása óta rejtve valának.
\par 36 Ekkor elbocsátván a sokaságot, beméne Jézus a házba. És az õ tanítványai hozzámenének, mondván: Magyarázd meg nékünk a szántóföld konkolyáról való példázatot.
\par 37 Õ pedig felelvén monda nékik: A ki a jó magot veti, az az embernek Fia;
\par 38 A szántóföld pedig a világ; a jó mag az Isten országának fiai; a konkoly pedig a gonosznak fiai.
\par 39 Az ellenség pedig, a ki a konkolyt vetette, az ördög; az aratás pedig a világ vége; az aratók pedig az angyalok.
\par 40 A miképen azért összegyûjtik a konkolyt és megégetik: akképen lesz a világnak végén.
\par 41 Az embernek Fia elküldi az õ angyalait, és az õ országából összegyûjtik a botránkozásokat mind, és azokat is, a kik gonoszságot cselekesznek,
\par 42 És bevetik õket a tüzes kemenczébe: ott lészen sírás és fogcsikorgatás.
\par 43 Akkor az igazak fénylenek, mint a nap, az õ Atyjoknak országában. A kinek van füle a hallásra, hallja.
\par 44 Ismét hasonlatos a mennyeknek országa a szántóföldben elrejtett kincshez, a melyet megtalálván az ember, elrejté azt; és a felett való örömében elmegy és eladván mindenét a mije van, megveszi azt a szántóföldet.
\par 45 Ismét hasonlatos a mennyeknek országa a kereskedõhöz, a ki igazgyöngyöket keres;
\par 46 A ki találván egy drágagyöngyre, elméne, és mindenét eladván a mije volt, megvevé azt.
\par 47 Szintén hasonlatos a mennyeknek országa a tengerbe vetett gyalomhoz, a mely mindenféle fajtát összefogott;
\par 48 Melyet, minekutána megtelt, a partra vontak a halászok, és leülvén, a jókat edényekbe gyûjtötték, a hitványakat pedig kihányták.
\par 49 Így lesz a világ végén is: Eljõnek majd az angyalok, és kiválasztják a gonoszokat az igazak közül.
\par 50 És a tüzes kemenczébe vetik õket; ott lészen sírás és fogcsikorgatás.
\par 51 Monda nékik Jézus: Megértették-é mindezeket? Mondának néki: Megértettük Uram.
\par 52 Õ pedig monda nékik: Annakokáért minden írástudó, a ki a mennyeknek országa felõl megtaníttatott, hasonlatos az olyan gazdához, a ki ót és újat hoz elõ az õ éléstárából.
\par 53 És lõn, a mikor elvégzé Jézus ezeket a példázatokat, elméne onnan.
\par 54 És hazájába érve, tanítja vala õket az õ zsinagógájukban, annyira, hogy álmélkodnak és ezt mondják vala: Honnét van ebben ez a bölcseség és az erõk?
\par 55 Nem ez-é amaz ácsmesternek fia? Nem az õ anyját hívják-é Máriának, és az õ testvéreit Jakabnak, Józsénak, Simonnak és Júdásnak?
\par 56 És az õ nõtestvérei is nem mind minálunk vannak-é? Honnét vannak tehát ennél mindezek?
\par 57 És megbotránkoznak vala õ benne. Jézus pedig monda nékik: Nincsen próféta tisztesség nélkül, hanem csak az õ hazájában és házában.
\par 58 Nem is tõn ott sok csodát, az õ hitetlenségök miatt.

\chapter{14}

\par 1 Abban az idõben hírét hallá Heródes negyedes fejedelem Jézusnak,
\par 2 És monda szolgáinak: Ez ama Keresztelõ János; õ támadt fel a halálból, és azért mûködnek benne az erõk.
\par 3 Mert Heródes elfogatta vala Jánost, és megkötöztetvén, tömlöczbe vetette vala Heródiásért, az õ testvérének, Fülöpnek feleségéért.
\par 4 Mert ezt mondja vala néki János: Nem szabad néked õvele élned.
\par 5 De mikor meg akarta öletni, félt a sokaságtól, mert mint egy prófétát úgy tartják vala õt.
\par 6 Hanem mikor a Heródes születése napját ünnepelték, tánczola a Heródiás leánya õ elõttük, és megtetszék Heródesnek;
\par 7 Azért esküvéssel fogadá, hogy a mit kér, megadja néki.
\par 8 A leány pedig, anyja rábeszélésére, monda: Add ide nékem egy tálban a Keresztelõ János fejét.
\par 9 És megszomorodék a király, de esküjéért és a vendégek miatt parancsolá, hogy adják oda.
\par 10 És elküldvén, fejét véteté Jánosnak a tömlöczben.
\par 11 És elõhozák az õ fejét egy tálban, és adák a leánynak; az pedig vivé az õ anyjának.
\par 12 És elõjövén az õ tanítványai, elvivék a testet, és eltemeték azt; és elmenvén, megjelenték Jézusnak.
\par 13 És mikor ezt meghallotta Jézus, elméne onnét hajón egy puszta helyre egyedül. A sokaság pedig ezt hallva, gyalog követé õt a városokból.
\par 14 És kimenvén Jézus, láta nagy sokaságot, és megszáná õket, és azoknak betegeit meggyógyítá.
\par 15 Mikor pedig estveledék, hozzá menének az õ tanítványai, mondván: Puszta hely ez, és az idõ már elmúlt; bocsásd el a sokaságot, hogy menjenek el a falvakba és vegyenek magoknak eleséget.
\par 16 Jézus pedig monda nékik: Nem szükség elmenniök; adjatok nékik ti enniök.
\par 17 Azok pedig mondának néki: Nincsen itt, csupán öt kenyerünk és két halunk.
\par 18 Õ pedig monda: Hozzátok azokat ide hozzám.
\par 19 És mikor megparancsolá a sokaságnak, hogy üljenek le a fûre, vevé az öt kenyeret és két halat, és szemeit az égre emelvén, hálákat ada; és megszegvén a kenyereket, adá a tanítványoknak, a tanítványok pedig a sokaságnak.
\par 20 És mindnyájan evének, és megelégedének; és felszedék a maradék darabokat, tizenkét teli kosárral.
\par 21 A kik pedig ettek vala, mintegy ötezeren valának férfiak, asszonyokon és gyermekeken kívül.
\par 22 És mindjárt kényszeríté Jézus az õ tanítványait, hogy szálljanak a hajóba és menjenek át elõre a túlsó partra, míg õ elbocsátja a sokaságot.
\par 23 És a mint elbocsátá a sokaságot, felméne a hegyre, magánosan imádkozni. Mikor pedig beestveledék, egyedül vala ott.
\par 24 A hajó pedig immár a tenger közepén vala, a haboktól háborgattatva; mivelhogy a szél szembe fújt vala.
\par 25 Az éjszaka negyedik részében pedig hozzájuk méne Jézus, a tengeren járván.
\par 26 És mikor látták a tanítványok, hogy õ a tengeren jára, megrémülének, mondván: Ez kísértet; és a félelem miatt kiáltozának.
\par 27 De Jézus azonnal szóla hozzájuk, mondván: Bízzatok; én vagyok, ne féljetek!
\par 28 Péter pedig felelvén néki, monda: Uram, ha te vagy, parancsolj, hogy hozzád mehessek a vizeken.
\par 29 Õ pedig monda: Jövel! És Péter kiszállván a hajóból, jár vala a vizeken, hogy Jézushoz menjen.
\par 30 De látva a nagy szelet, megrémüle; és a mikor kezd vala merülni, kiálta, mondván: Uram, tarts meg engem!
\par 31 Jézus pedig azonnal kinyújtván kezét, megragadá õt, és monda néki: Kicsinyhitû, miért kételkedél?
\par 32 És a mikor beléptek a hajóba, elállt a szél.
\par 33 A hajóban levõk pedig hozzámenvén: leborulának elõtte, mondván: Bizony, Isten Fia vagy!
\par 34 És általkelvén, eljutának Genezáret földére.
\par 35 És mikor megismerték õt annak a helynek lakosai, szétküldének abba az egész környékbe, és minden beteget hozzá hozának;
\par 36 És kérik vala õt, hogy csak az õ ruhájának peremét illethessék. És a kik illeték vala, mindnyájan meggyógyulának.

\chapter{15}

\par 1 Akkor írástudók és farizeusok jõnek vala Jézushoz, Jeruzsálembõl, mondván:
\par 2 Miért hágják át a te tanítványaid a vének rendeléseit? Mert nem  mossák meg a kezeiket, mikor enni akarnak.
\par 3 Õ pedig felelvén monda nékik: Ti meg miért hágjátok át az Isten parancsolatját a ti rendeléseitek által?
\par 4 Mert Isten parancsolta ezt, mondván: Tiszteld atyádat és anyádat, és: A ki atyját vagy anyját szidalmazza, halállal lakoljon.
\par 5 Ti pedig ezt mondjátok: A ki atyjának vagy anyjának ezt mondja: Templomi ajándék az, a mivel megsegíthetlek, az olyan akár ne is tisztelje az õ atyját vagy anyját.
\par 6 És erõtelenné tettétek az Isten parancsolatját a ti rendeléseitek által.
\par 7 Képmutatók, igazán prófétált felõletek Ésaiás, mondván:
\par 8 Ez a nép szájával közelget hozzám, és ajkával tisztel engemet; szíve pedig távol van tõlem.
\par 9 Pedig hiába tisztelnek engem, ha oly tudományokat tanítanak, a melyek embereknek parancsolatai.
\par 10 És elõszólítván a sokaságot, monda nékik: Halljátok és értsétek meg:
\par 11 Nem az fertõzteti meg az embert, a mi a szájon bemegy, hanem a mi kijön  a szájból, az fertõzteti meg az embert.
\par 12 Akkor hozzájárulván az õ tanítványai, mondának néki: Tudod-é, hogy a farizeusok e beszédet hallván, megbotránkoztak?
\par 13 Õ pedig felelvén monda: Minden plánta, a melyet nem az én mennyei Atyám plántált, kitépetik.
\par 14 Hagyjátok õket; vakoknak vak vezetõi õk: ha pedig vak vezeti a vakot, mind a ketten a verembe esnek.
\par 15 Péter pedig felelvén, monda néki: Magyarázd meg nékünk ezt a példázatot.
\par 16 Jézus pedig monda: Ti is értelem nélkül vagytok-é még?
\par 17 Mégsem értitek-é, hogy minden, a mi a szájon bemegy, a gyomorba jut, és az árnyékszékbe vettetik?
\par 18 A mik pedig a szájból jõnek ki, a szívbõl származnak, és azok fertõztetik meg az embert.
\par 19 Mert a szívbõl származnak a gonosz gondolatok, gyilkosságok, házasságtörések, paráznaságok, lopások, hamis tanubizonyságok, káromlások.
\par 20 Ezek fertõztetik meg az embert; de a mosdatlan kézzel való evés nem fertõzteti meg az embert.
\par 21 És elmenvén onnét Jézus, Tirus és Sidon vidékeire tére.
\par 22 És ímé egy kananeus asszony jövén ki abból a tartományból, kiált vala néki: Uram, Dávidnak fia, könyörülj rajtam! az én leányom az ördögtõl gonoszul gyötörtetik.
\par 23 Õ pedig egy szót sem felele néki. És az õ tanítványai hozzá menvén, kérik vala õt, mondván: Bocsásd el õt, mert utánunk kiált.
\par 24 Õ pedig felelvén, monda: Nem küldettem, csak az Izráel házának elveszett juhaihoz.
\par 25 Az asszony pedig odaérvén, leborula elõtte, mondván: Uram, légy segítségül nékem!
\par 26 Õ pedig felelvén, monda: Nem jó a fiak kenyerét elvenni, és az ebeknek vetni.
\par 27 Az pedig monda: Úgy van, Uram; de hiszen az ebek is esznek a morzsalékokból, a mik az õ uroknak asztaláról aláhullanak.
\par 28 Ekkor felelvén Jézus, monda néki: Óh asszony, nagy a te hited! Legyen néked a te akaratod szerint. És meggyógyula az õ leánya attól a pillanattól fogva.
\par 29 És onnét távozva, méne Jézus a Galilea tengere mellé; és felmenvén a hegyre, ott leüle.
\par 30 És nagy sokaság megy vala hozzá, vivén magokkal sántákat, vakokat, némákat, csonkákat és sok egyebeket, és odahelyezék õket a Jézus lábai elé; és meggyógyítá õket,
\par 31 Úgy hogy a sokaság álmélkodik vala, látván, hogy a némák beszélnek, a csonkák megépülnek, a sánták járnak, a vakok látnak: és dicsõíték Izráel Istenét.
\par 32 Jézus pedig elõszólítván az õ tanítványait, monda: Szánakozom e sokaságon, mert három napja immár, hogy velem vannak, és nincs mit enniök. Éhen pedig nem akarom õket elbocsátani, hogy valamiképen ki ne dõljenek az úton.
\par 33 És mondának néki az õ tanítványai: Honnét volna e pusztában annyi kenyerünk, hogy megelégítsünk ily nagy sokaságot?
\par 34 És monda nékik Jézus: Hány kenyeretek van? Õk pedig mondának: Hét, és néhány halunk.
\par 35 És parancsolá a sokaságnak, hogy telepedjenek le a földön.
\par 36 És vevén a hét kenyeret és a halakat, és hálákat adván, megtöré, és adá az õ tanítványainak, a tanítványok pedig a sokaságnak.
\par 37 És mindnyájan evének, és megelégedének; és fölszedék a maradék darabokat hét teli kosárral.
\par 38 A kik pedig ettek vala, négyezeren valának férfiak, asszonyokon és gyermekeken kívül.
\par 39 És elbocsátván a sokaságot, beszálla a hajóba, és elméne Magdala határaiba.

\chapter{16}

\par 1 És hozzá menvén a farizeusok és sadduczeusok, kisértvén, kérék õt, hogy mutasson nékik mennyei jelt.
\par 2 Õ pedig felelvén, monda nékik: Mikor estveledik, azt mondjátok: Szép idõ lesz; mert veres az ég.
\par 3 Reggel pedig: Ma zivatar lesz; mert az ég borús és veres. Képmutatók, az ég ábrázatját meg tudjátok ítélni, az idõk jeleit pedig nem tudjátok?
\par 4 E gonosz és parázna nemzetség jelt kíván; és nem adatik néki jel, hanemha a Jónás prófétának jele. És ott hagyván õket, elméne.
\par 5 És az õ tanítványai a tulsó partra menvén, elfelejtettek kenyeret vinni magukkal.
\par 6 Jézus pedig monda nékik: Vigyázzatok és õrizkedjetek a farizeusok és sadduczeusok kovászától.
\par 7 Õk pedig tanakodnak vala maguk között, mondván: Nem hoztunk kenyeret magunkkal.
\par 8 Jézus pedig megértvén ezt, monda nékik: Mit tanakodtok magatok között óh kicsinyhitûek, hogy kenyeret nem hoztatok magatokkal?!
\par 9 Mégsem értitek-é, nem is emlékeztek-é az ötezernek öt kenyerére, és hogy hány kosárt töltöttetek meg?
\par 10 Sem a négyezernek hét kenyerére, és hogy hány kosárt töltöttetek meg?
\par 11 Hogyan nem értitek, hogy nem kenyérrõl mondtam néktek, hogy õrizkedjetek a farizeusok és sadduczeusok kovászától?
\par 12 Ekkor értették meg, hogy nem arról szólott, hogy a kenyér kovászától, hanem hogy a farizeusok és sadduczeusok tudományától õrizkedjenek.
\par 13 Mikor pedig Jézus Czézárea Filippi környékére méne, megkérdé tanítványait, mondván: Engemet, embernek Fiát, kinek mondanak az emberek?
\par 14 Õk pedig mondának: Némelyek Keresztelõ Jánosnak, mások  Illésnek; némelyek pedig Jeremiásnak, vagy egynek a próféták közül.
\par 15 Monda nékik: Ti pedig kinek mondotok engem?
\par 16 Simon Péter pedig felelvén, monda: Te vagy a Krisztus, az élõ Istennek Fia.
\par 17 És felelvén Jézus, monda néki: Boldog vagy Simon, Jónának fia, mert nem test és vér jelentette ezt meg néked, hanem az én mennyei Atyám.
\par 18 De én is mondom néked, hogy te Péter vagy, és ezen a kõsziklán építem fel az én anyaszentegyházamat, és a pokol kapui sem vesznek rajta diadalmat.
\par 19 És néked adom a mennyek országának kulcsait; és a mit megkötsz a földön, a mennyekben is kötve lészen; és a mit megoldasz a földön, a mennyekben is oldva lészen.
\par 20 Akkor megparancsolá tanítványainak, hogy senkinek se mondják, hogy õ a Jézus Krisztus.
\par 21 Ettõl fogva kezdé Jézus jelenteni az õ tanítványainak, hogy néki Jeruzsálembe kell menni, és sokat szenvedni a vénektõl és a fõpapoktól és az írástudóktól, és megöletni, és harmadnapon  föltámadni.
\par 22 És Péter elõfogván õt, kezdé feddeni, mondván: Mentsen Isten, Uram! Nem eshetik ez meg te véled.
\par 23 Õ pedig megfordulván, monda Péternek: Távozz tõlem Sátán; bántásomra vagy nékem; mert nem gondolsz az Isten dolgaira, hanem az emberi dolgokra.
\par 24 Ekkor monda Jézus az õ tanítványainak: Ha valaki jõni akar én utánam, tagadja meg magát és vegye fel az õ keresztjét,  és kövessen engem.
\par 25 Mert a ki meg akarja tartani az õ életét, elveszti azt; a ki pedig elveszti az õ életét én érettem, megtalálja azt.
\par 26 Mert mit használ az embernek, ha az egész világot megnyeri is, de az õ lelkében kárt vall? Avagy micsoda váltságot adhat az ember az õ lelkéért?
\par 27 Mert az embernek Fia eljõ az õ Atyjának dicsõségében, az õ angyalaival; és akkor megfizet mindenkinek az õ cselekedete szerint.
\par 28 Bizony mondom néktek: Azok között, a kik itt állanak, vannak némelyek, a kik nem kóstolják meg a halált a míg meg nem látják az embernek Fiát eljõni  az õ országában.

\chapter{17}

\par 1 És hat nap mulva magához vevé Jézus Pétert, Jakabot és ennek testvérét Jánost, és felvivé õket magokban egy magas hegyre.
\par 2 És elváltozék elõttök, és az õ orczája ragyog vala, mint a nap, ruhája pedig fehér lõn, mint a fényesség.
\par 3 És ímé megjelenék õ nékik Mózes és Illés, a kik beszélnek vala õ vele.
\par 4 Péter pedig megszólalván, monda Jézusnak: Uram, jó nékünk itt lennünk. Ha akarod, építsünk itt három hajlékot, néked egyet, Mózesnek is egyet, Illésnek is egyet.
\par 5 Mikor õ még beszél vala, ímé, fényes felhõ borítá be õket; és ímé szózat lõn a felhõbõl, mondván: Ez az én szerelmes Fiam, a kiben én gyönyörködöm: õt hallgassátok.
\par 6 És a tanítványok a mint ezt hallák, arczra esének és igen megrémülének.
\par 7 Jézus pedig hozzájok menvén, illeté õket, és monda: Keljetek fel és ne féljetek!
\par 8 Mikor pedig szemeiket fölemelék, senkit sem látának, hanem csak Jézust egyedül.
\par 9 És mikor a hegyrõl alájövének, megparancsolá nékik Jézus, mondván: Senkinek se mondjátok el a mit láttatok, míg fel nem támadt az embernek Fia a halálból.
\par 10 És megkérdezék õt az õ tanítványai, mondván: Miért mondják tehát az írástudók, hogy elõbb Illésnek kell eljõnie?
\par 11 Jézus pedig felelvén, monda nékik: Illés bizony eljõ elõbb, és mindent helyreállít;
\par 12 De mondom néktek, hogy Illés immár eljött, és nem ismerék meg õt, hanem azt mívelék vele, a mit akarának. Ezenképen az ember Fiának is szenvednie kell majd õ tõlük.
\par 13 Ekkor megértették a tanítványok, hogy Keresztelõ Jánosról szóla nékik.
\par 14 És mikor a sokasághoz értek, egy ember jöve hozzá, térdre esvén õ elõtte,
\par 15 És mondván: Uram, könyörülj az én fiamon, mert holdkóros és kegyetlenül szenved; mivelhogy gyakorta esik a tûzbe, és gyakorta a vízbe.
\par 16 És elvittem õt a te tanítványaidhoz, és nem tudták õt meggyógyítani.
\par 17 Jézus pedig felelvén, monda: Óh hitetlen és elfajult nemzetség! vajjon meddig leszek veletek? vajjon meddig szenvedlek titeket? Hozzátok õt ide nékem.
\par 18 És megdorgálá õt Jézus, és kiméne belõle az ördög; és meggyógyula a gyermek azon órától fogva.
\par 19 Ekkor a tanítványok magukban Jézushoz menvén, mondának néki: Mi miért nem tudtuk azt kiûzni?
\par 20 Jézus pedig monda nékik: A ti hitetlenségetek miatt. Mert bizony mondom néktek: Ha akkora hitetek volna, mint a mustármag, azt mondanátok ennek a hegynek: Menj innen amoda, és elmenne; és semmi sem volna lehetetlen néktek.
\par 21 Ez a fajzat pedig ki nem megy, hanemha könyörgés és bõjtölés által.
\par 22 Mikor pedig Galileában jártak vala, monda nékik Jézus: Az ember Fia emberek kezébe adatik;
\par 23 És megölik õt, de harmadnapon föltámad. És felettébb megszomorodának.
\par 24 Mikor pedig eljutottak vala Kapernaumba, a kétdrakma-szedõk Péterhez menének és mondának néki: A ti mesteretek nem fizeti-é a kétdrakmát?
\par 25 Monda: Igen. És mikor beméne a házba, megelõzé õt Jézus, mondván: Mit gondolsz Simon? A föld királyai kiktõl szednek vámot vagy adót? a fiaiktól-é, vagy az idegenektõl?
\par 26 Monda néki Péter: Az idegenektõl. Monda néki Jézus: Tehát a fiak szabadok.
\par 27 De hogy õket meg ne botránkoztassuk, menj a tengerre, vesd be a horgot, és vond ki az elsõ halat, a mely rá akad: és felnyitván a száját, egy státert találsz benne: azt kivévén, add oda nékik én érettem és te éretted.

\chapter{18}

\par 1 Abban az órában menének a tanítványok Jézushoz, mondván: Vajjon ki nagyobb a mennyeknek országában?
\par 2 És elõhíván Jézus egy kis gyermeket, közéjök állítja vala azt,
\par 3 És monda: Bizony mondom néktek, ha meg nem tértek és olyanok nem lesztek mint a kis gyermekek, semmiképen nem mentek be a mennyeknek országába.
\par 4 A ki azért megalázza magát, mint ez a kis gyermek, az a nagyobb a mennyeknek országában.
\par 5 És a ki egy ilyen kis gyermeket befogad az én nevemben, engem fogad be.
\par 6 A ki pedig megbotránkoztat egyet e kicsinyek közül, a kik én bennem hisznek, jobb annak, hogy malomkövet kössenek a nyakára, és a tenger mélységébe vessék.
\par 7 Jaj a világnak a botránkozások miatt! Mert szükség, hogy botránkozások essenek; de jaj annak az embernek, a ki által a botránkozás esik.
\par 8 Ha pedig a te kezed vagy a te lábad megbotránkoztat téged, vágd le azokat és vesd el magadtól; jobb néked az életre sántán vagy csonkán bemenned, hogynem két kézzel vagy két lábbal vettetned az örök tûzre.
\par 9 És ha a te szemed botránkoztat meg téged, vájd ki azt és vesd el magadtól; jobb néked félszemmel bemenned az életre, hogynem két szemmel vettetned a gyehenna tüzére.
\par 10 Meglássátok, hogy eme kicsinyek közül egyet is meg ne utáljatok; mert mondom néktek, hogy az õ angyalaik a mennyekben mindenkor látják az én mennyei Atyám orczáját.
\par 11 Mert az embernek Fia azért jött, hogy megtartsa, a mi elveszett vala.
\par 12 Mit gondoltok? Ha valamely embernek száz juha van, és egy azok közül eltévelyedik: vajjon a kilenczvenkilenczet nem hagyja-é ott, és a hegyekre menvén, nem keresi-é azt, a melyik eltévelyedett?
\par 13 És ha történetesen megtalálja azt, bizony mondom néktek, inkább örvend azon, mint a kilenczvenkilenczen, a mely el nem tévelyedett.
\par 14 Ekképen a ti mennyei Atyátok sem akarja, hogy egy is elveszszen e kicsinyek közül.
\par 15 Ha pedig a te atyádfia vétkezik ellened, menj el és dorgáld meg õt négy szem között: ha hallgat rád, megnyerted a te atyádfiát;
\par 16 Ha pedig nem hallgat rád, végy magad mellé még egyet vagy kettõt, hogy két vagy három tanú vallomásával erõsíttessék minden szó.
\par 17 Ha azokra nem hallgat, mondd meg a gyülekezetnek; ha a gyülekezetre sem hallgat, legyen elõtted olyan, mint a pogány és a vámszedõ.
\par 18 Bizony mondom néktek: A mit megköttök a földön, a mennyben is kötve lészen; és a mit megoldotok a földön, a mennyben is oldva lészen.
\par 19 Ismét, mondom néktek, hogy ha ketten közületek egy akaraton lesznek a földön minden dolog felõl, a mit csak kérnek, megadja nékik az én mennyei Atyám.
\par 20 Mert a hol ketten vagy hárman egybegyûlnek az én nevemben, ott vagyok közöttük.
\par 21 Ekkor hozzámenvén Péter, monda: Uram, hányszor lehet az én atyámfiának ellenem vétkezni, és néki megbocsátanom? még hétszer is?
\par 22 Monda néki Jézus: Nem mondom néked, hogy még hétszer is, hanem még hetvenhétszer is.
\par 23 Annakokáért hasonlatos a mennyeknek országa a királyhoz, a ki számot akar vala vetni az õ szolgáival.
\par 24 Mikor pedig számot kezde vetni, hozának eléje egyet, a ki tízezer tálentommal vala adós.
\par 25 Nem tudván pedig fizetni, parancsolá annak ura, hogy adják el azt, és a feleségét és gyermekeit, és mindenét, a mije vala, és fizessenek.
\par 26 Leborulván azért a szolga elõtte, könyörög vala néki, mondván: Uram, légy türelemmel hozzám, és mindent megfizetek néked.
\par 27 Az úr pedig megszánván azt a szolgát, elbocsátá õt, és az adósságot is elengedé néki.
\par 28 Kimenvén pedig az a szolga, találkozék egygyel az õ szolgatársai közül, a ki száz dénárral vala néki adós; és megragadván azt, fojtogatja vala, mondván: Fizesd meg nékem, a mivel tartozol.
\par 29 Leborulván azért az õ szolgatársa az õ lábai elé, könyörög vala néki, mondván: Légy türelemmel hozzám, és mindent megfizetek néked.
\par 30 De õ nem akará; hanem elmenvén, börtönbe veté õt, mígnem megfizeti, a mivel tartozik.
\par 31 Látván pedig az õ szolgatársai, a mik történtek vala, felettébb megszomorodának; és elmenvén, mindent megjelentének az õ uroknak, a mik történtek vala.
\par 32 Akkor elõhivatván õt az õ ura, monda néki: Gonosz szolga, minden adósságodat elengedtem néked, mivelhogy könyörögtél nékem:
\par 33 Nem kellett volna-é néked is könyörülnöd a te szolgatársadon, a miképen én is könyörültem te rajtad?
\par 34 És megharagudván az õ ura, átadta õt a hóhérok kezébe, mígnem megfizeti mind, a mivel tartozik.
\par 35 Ekképen cselekszik az én mennyei Atyám is veletek, ha szivetekbõl meg nem bocsátjátok, kiki az õ atyjafiának, az õ vétkeiket.

\chapter{19}

\par 1 És lõn, mikor elvégezte Jézus e beszédeket, elméne Galileából, és méne Júdeának határaiba a Jordánon túl;
\par 2 És követé õt nagy sokaság, és meggyógyítá ott õket.
\par 3 És hozzá menének a farizeusok, kisértvén õt és mondván: Szabad-é az embernek az õ feleségét akármi okért elbocsátani?
\par 4 Õ pedig felelvén, monda: Nem olvastátok-é, hogy a teremtõ kezdettõl fogva férfiúvá és asszonynyá teremté õket,
\par 5 És ezt mondá: Annak okáért elhagyja a férfiú atyját és anyját; és ragaszkodik feleségéhez, és lesznek ketten egy testté.
\par 6 Úgy hogy többé nem kettõ, hanem egy test. A mit azért az Isten egybeszerkesztett, ember el ne válaszsza.
\par 7 Mondának néki: Miért rendelte tehát Mózes, hogy a válólevelet kell adni, és úgy bocsátani el az asszonyt?
\par 8 Monda nékik: Mózes a ti szívetek keménysége miatt engedte volt meg néktek, hogy feleségeiteket elbocsássátok; de kezdettõl fogva nem így volt.
\par 9 Mondom pedig néktek, hogy a ki elbocsátja feleségét, hanemha paráznaság miatt, és mást vesz el, házasságtörõ; és a ki elbocsátottat vesz el, az is házasságtörõ.
\par 10 Mondának néki tanítványai: Ha így van a férfi dolga az asszonynyal, nem jó megházasodni.
\par 11 Õ pedig monda nékik: Nem mindenki veszi be ezt a beszédet, hanem a kinek adatott.
\par 12 Mert vannak heréltek, a kik anyjuk méhébõl születtek így; és vannak heréltek, a kiket az emberek heréltek ki; és vannak heréltek, a kik maguk herélték ki magukat a mennyeknek országáért. A ki beveheti, vegye be.
\par 13 Ekkor kis gyermekeket hozának hozzá hogy kezeit vesse azokra, és imádkozzék; a tanítványok pedig dorgálják vala azokat.
\par 14 Jézus pedig monda: Hagyjatok békét e kis gyermekeknek, és ne tiltsátok meg nekik, hogy hozzám jõjjenek; mert ilyeneké a mennyeknek országa.
\par 15 És kezeit reájuk vetvén, eltávozék onnét.
\par 16 És ímé hozzá jövén egy ember, monda néki: Jó mester, mi jót cselekedjem, hogy örök életet nyerjek?
\par 17 Õ pedig monda néki: Miért mondasz engem jónak? Senki sem jó, csak egy, az Isten. Ha pedig be akarsz menni az életre, tartsd meg a parancsolatokat.
\par 18 Monda néki: Melyeket? Jézus pedig monda: Ezeket: Ne ölj; ne paráználkodjál; ne lopj; hamis tanubizonyságot ne tégy;
\par 19 Tiszteld atyádat és anyádat; és: Szeresd felebarátodat, mint temagadat.
\par 20 Monda néki az ifjú: Mindezeket megtartottam ifjúságomtól fogva; mi fogyatkozás van még bennem?
\par 21 Monda néki Jézus: Ha tökéletes akarsz lenni, eredj, add el vagyonodat, és oszd ki a szegényeknek; és kincsed lesz mennyben; és jer és kövess engem.
\par 22 Az ifjú pedig e beszédet hallván, elméne megszomorodva; mert sok jószága vala.
\par 23 Jézus pedig monda az õ tanítványainak: Bizony mondom néktek, hogy a gazdag nehezen megy be a mennyeknek országába.
\par 24 Ismét mondom pedig néktek: Könnyebb a tevének a tû fokán átmenni, hogynem a gazdagnak az Isten országába bejutni.
\par 25 A tanítványok pedig ezeket hallván, felettébb álmélkodnak vala, mondván: Kicsoda üdvözülhet tehát?
\par 26 Jézus pedig rájuk tekintvén, monda nékik: Embereknél ez lehetetlen, de Istennél minden lehetséges.
\par 27 Akkor felelvén Péter, monda néki: Ímé, mi elhagytunk mindent és követtünk téged: mink lesz hát minékünk?
\par 28 Jézus pedig monda nékik: Bizony mondom néktek, hogy ti, a kik követtetek engem, az újjá születéskor, a mikor az embernek Fia beül az õ dicsõségének királyi székébe, ti is beültök majd tizenkét királyi székbe, és ítélitek az Izráel tizenkét nemzetségét.
\par 29 És a ki elhagyta házait, vagy fitestvéreit, vagy nõtestvéreit, vagy atyját, vagy anyját, vagy feleségét, vagy gyermekeit, vagy szántóföldjeit az én nevemért, mindaz száz annyit vészen, és örökség szerint nyer örök életet.
\par 30 Sok elsõk pedig lesznek utolsók, és sok utolsók elsõk.

\chapter{20}

\par 1 Mert hasonlatos a mennyeknek országa a gazdaemberhez, a ki jó reggel kiméne, hogy munkásokat fogadjon az õ szõlejébe.
\par 2 Megszerzõdvén pedig a munkásokkal napi tíz pénzben, elküldé õket az õ szõlejébe.
\par 3 És kimenvén három óra tájban, láta másokat, a kik hivalkodván a piaczon álltak vala.
\par 4 És monda nékik: Menjetek el ti is a szõlõbe, és a mi igazságos, megadom néktek.
\par 5 Azok pedig elmenének. Hat és kilencz óra tájban ismét kimenvén, ugyanazon képen cselekedék.
\par 6 Tizenegy óra tájban is kimenvén, talála másokat, a kik hivalkodva állottak vala, és monda nékik: Miért álltok itt egész napon át, hivalkodván?
\par 7 Mondának néki: Mert senki sem fogadott meg minket. Monda nékik: Menjetek el ti is a szõlõbe, és a mi igazságos, megkapjátok.
\par 8 Mikor pedig beestveledék, monda a szõlõnek ura az õ vinczellérjének: Hívd elõ a munkásokat, és add ki nékik a bért, az utolsóktól kezdve mind az elsõkig.
\par 9 És jövén a tizenegyórásak, fejenként tíz-tíz pénzt võnek.
\par 10 Jövén azután az elsõk, azt gondolják vala, hogy õk többet kapnak: de õk is tíz-tíz pénzt võnek fejenként.
\par 11 A mint pedig fölvevék, zúgolódnak vala a házigazda ellen,
\par 12 Mondván: Azok az utolsók egyetlen óráig munkálkodtak és egyenlõkké tetted azokat velünk, a kik a napnak terhét és hõségét szenvedtük.
\par 13 Õ pedig felelvén, monda azok közül egynek: Barátom, nem cselekszem igazságtalanul veled; avagy nem tíz pénzben szerzõdtél-é meg velem?
\par 14 Vedd, a mi a tiéd, és menj el. Én pedig ennek az utolsónak is annyit akarok adni, mint néked.
\par 15 Avagy nem szabad-é nékem a magaméval azt tennem, amit akarok? avagy a te szemed azért gonosz, mert én jó vagyok?
\par 16 Ekképen lesznek az utolsók elsõk és az elsõk utolsók; mert sokan vannak a hivatalosok,  de kevesen a választottak.
\par 17 És mikor felmegy vala Jézus Jeruzsálembe, útközben csupán a tizenkét tanítványt vévén magához, monda nékik:
\par 18 Ímé, felmegyünk Jeruzsálembe, és az embernek Fia átadatik a fõpapoknak és írástudóknak; és halálra kárhoztatják õt,
\par 19 És a pogányok kezébe adják õt, hogy megcsúfolják és megostorozzák és keresztre feszítsék; de harmadnap feltámad.
\par 20 Ekkor hozzá méne a Zebedeus fiainak anyja az õ fiaival együtt, leborulván és kérvén õ tõle valamit.
\par 21 Õ pedig monda néki: Mit akarsz? Monda néki: Mondd, hogy ez az én két fiam üljön a te országodban egyik jobb kezed felõl, a másik bal kezed felõl.
\par 22 Jézus pedig felelvén, monda: Nem tudjátok, mit kértek. Megihatjátok-é a pohárt, a melyet én megiszom? és megkeresztelkedhettek-é azzal  a keresztséggel, a melylyel én megkeresztelkedem? Mondának néki: Meg.
\par 23 És monda nékik: Az én poharamat megiszszátok ugyan, és a keresztséggel, a melylyel én megkeresztelkedem, megkeresztelkedtek; de az én jobb és balkezem felõl való ülést nem az én dolgom megadni, hanem azoké lesz az, a kiknek az én Atyám elkészítette.
\par 24 És hallva ezt a tíz, megboszankodék a két testvérre.
\par 25 Jézus pedig elõszólítván õket, monda: Tudjátok, hogy a pogányok fejedelmei uralkodnak azokon, és a nagyok hatalmaskodnak rajtok.
\par 26 De ne így legyen közöttetek; hanem a ki közöttetek nagy akar lenni, legyen a ti szolgátok;
\par 27 És a ki közöttetek elsõ akar lenni, legyen a ti szolgátok.
\par 28 Valamint az embernek Fia nem azért jött, hogy néki szolgáljanak, hanem hogy õ szolgáljon, és adja az õ életét  váltságul sokakért.
\par 29 És mikor Jerikóból távozának, nagy sokaság követé õt.
\par 30 És ímé, két vak, a ki az út mellett ül vala, meghallván, hogy Jézus arra megy el, kiált vala, mondván: Uram, Dávidnak Fia, könyörülj rajtunk!
\par 31 A sokaság pedig dorgálja vala õket, hogy hallgassanak; de azok annál jobban kiáltnak vala, mondván: Uram Dávidnak Fia, könyörülj rajtunk!
\par 32 És megállván Jézus, megszólítá õket és monda: Mit akartok, hogy cselekedjem veletek?
\par 33 Mondának néki: Azt, Uram, hogy megnyíljanak a mi szemeink.
\par 34 Jézus pedig megszánván õket, illeté az õ szemeiket; és szemeik azonnal megnyíltak; és követék õt.

\chapter{21}

\par 1 És mikor közeledtek Jeruzsálemhez, és Bethfagéba, az olajfák hegyéhez jutottak vala, akkor elkülde Jézus két tanítványt,
\par 2 És monda nékik: Menjetek ebbe a faluba, a mely elõttetek van, és legott találtok egy megkötött szamarat és vele együtt az õ vemhét; oldjátok el és hozzátok ide nékem.
\par 3 És ha valaki valamit szól néktek, mondjátok, hogy az Úrnak van szüksége rájuk: és legott el fogja bocsátani õket.
\par 4 Mindez pedig azért lett, hogy beteljesedjék a próféta mondása, a ki így szólott:
\par 5 Mondjátok meg Sion leányának: Ímhol jõ néked a te királyod, alázatosan és szamáron ülve, és teherhordozó szamárnak vemhén.
\par 6 A tanítványok pedig elmenvén és úgy cselekedvén, a mint Jézus parancsolta vala nékik,
\par 7 Elhozák a szamarat és annak vemhét, és felsõ ruháikat rájuk teríték, és ráüle azokra.
\par 8 A sokaság legnagyobb része pedig felsõ ruháit az útra teríté; mások pedig a fákról galyakat vagdalnak és hintenek vala az útra.
\par 9 Az elõtte és utána menõ sokaság pedig kiált vala, mondván: Hozsánna a Dávid fiának! Áldott, a ki jõ az Úrnak nevében! Hozsánna a magasságban!
\par 10 És a mikor bemegy vala Jeruzsálembe, felháborodék az egész város, mondván: Kicsoda ez?
\par 11 A sokaság pedig monda: Ez Jézus, a galileai Názáretbõl való próféta.
\par 12 És beméne Jézus az Isten templomába, és kiûzé mindazokat, a kik árulnak és vásárolnak vala a templomban; és a pénzváltók asztalait és a galambárusok székeit felforgatá.
\par 13 És monda nékik: Meg van írva: Az én házam imádság házának mondatik. Ti pedig azt latroknak barlangjává  tettétek.
\par 14 És menének hozzá vakok és sánták a templomban; és meggyógyítá õket.
\par 15 A fõpapok és írástudók pedig, látván a csodákat, a melyeket cselekedett vala, és a gyermekeket, a kik kiáltottak vala a templomban, és ezt mondták vala: Hozsánna a Dávid fiának; haragra gerjedének,
\par 16 És mondának néki: Hallod, mit mondanak ezek? Jézus pedig monda nékik: Hallom. Sohasem olvastátok-é: A gyermekek és csecsemõk szája által szereztél dicsõséget?
\par 17 És ott hagyván õket, kiméne a városból Bethániába, és ott marada éjjel.
\par 18 Reggel pedig, a városba visszajövet, megéhezék.
\par 19 És meglátva egy fügefát az út mellett, oda méne hozzá, és nem talála azon semmit, hanem csak levelet; és monda annak: Gyümölcs te rajtad ezután soha örökké ne teremjen. És a fügefa azonnal elszárada.
\par 20 És látván ezt a tanítványok, elcsodálkozának, mondván: Hogyan száradt el a fügefa oly hirtelen?
\par 21 Jézus pedig felelvén, monda nékik: Bizony mondom néktek, ha van hitetek és nem kételkedtek, nemcsak azt cselekszitek, a mi e fügefán esett, hanem ha azt mondjátok e hegynek: Kelj fel és zuhanj a tengerbe, az is meglészen;
\par 22 És a mit könyörgéstekben kértek, mindazt meg is kapjátok, ha hisztek.
\par 23 És mikor bement vala a templomba, hozzámenének a fõpapok és a nép vénei, a mint tanít vala, mondván: Micsoda hatalommal cselekszed ezeket? és ki adta néked ezt a hatalmat?
\par 24 Jézus pedig felelvén, monda nékik: Én is kérdek egy dolgot tõletek, a mire ha megfeleltek nékem, én is megmondom néktek, micsoda hatalommal cselekszem ezeket.
\par 25 A János keresztsége honnan vala? Mennybõl-é, vagy emberektõl? Azok pedig tanakodnak vala magukban, mondván: Ha azt mondjuk: mennybõl, azt mondja majd nékünk: Miért nem hittetek tehát néki?
\par 26 Ha pedig azt mondjuk: emberektõl; félünk a sokaságtól; mert Jánost mindnyájan prófétának tartják.
\par 27 És felelvén Jézusnak, mondának: Nem tudjuk. Monda nékik õ is: Én sem mondom meg néktek, micsoda hatalommal cselekszem ezeket.
\par 28 De mit gondoltok ti? Vala egy embernek két fia, és odamenvén az elsõhöz, monda: Eredj fiam, munkálkodjál ma az én szõlõmben.
\par 29 Az pedig felelvén, monda: Nem megyek; de azután meggondolván magát, elméne.
\par 30 A másikhoz is odamenvén, hasonlóképen szóla. Az pedig felelvén, monda: Én elmegyek, uram; de nem méne el.
\par 31 E kettõ közül melyik teljesítette az atya akaratát? Mondának néki: Az elsõ. Monda nékik Jézus: Bizony mondom néktek: A vámszedõk és a parázna nõk megelõznek titeket az Isten országában.
\par 32 Mert eljött hozzátok János, az igazság útján, és nem hittetek néki, a vámszedõk és a parázna nõk pedig hittek néki; ti pedig, a kik ezt láttátok, azután sem tértetek meg, hogy hittetek volna néki.
\par 33 Más példázatot halljatok: Vala egy házigazda, a ki szõlõt plántála, és azt gyepûvel körülvevé, sajtót ása le benne, és tornyot építe, és kiadá azt munkásoknak, és  elutazék.
\par 34 Mikor pedig a gyümölcs ideje elérkezett vala, elküldé szolgáit a munkásokhoz, hogy vegyék át az õ gyümölcsét.
\par 35 És a munkások megfogván az õ szolgáit, az egyiket megverék, a másikat megölék, a harmadikat pedig megkövezék.
\par 36 Ismét külde más szolgákat, többet mint elõbb; és azokkal is úgy cselekedének.
\par 37 Utoljára pedig elküldé azokhoz a maga fiát, ezt mondván: A fiamat meg fogják becsülni.
\par 38 De a munkások, meglátván a fiút, mondának magok közt: Ez az örökös; jertek, öljük meg õt, és foglaljuk el az õ örökségét.
\par 39 És megfogván õt, kiveték a szõlõn kívül és megölék.
\par 40 Mikor azért megjõ a szõlõnek ura, mit cselekszik ezekkel a munkásokkal?
\par 41 Mondának néki: Mint gonoszokat gonoszul elveszti õket; a szõlõt pedig kiadja más munkásoknak, a kik beadják majd néki a gyümölcsöt annak idejében.
\par 42 Monda nékik Jézus: Sohasem olvastátok-é az írásokban: A mely követ az építõk megvetettek, az lett a szegletnek feje; az Úrtól lett ez, és csodálatos a mi szemeink elõtt.
\par 43 Annakokáért mondom néktek, hogy elvétetik tõletek az Istennek országa, és oly népnek adatik, a mely megtermi annak gyümölcsét.
\par 44 És a ki e kõre esik, szétzúzatik; a kire pedig ez esik reá, szétmorzsolja azt.
\par 45 És a fõpapok és farizeusok hallván az õ példázatait, megértették, hogy róluk szól.
\par 46 És mikor meg akarák õt fogni, megfélemlének a sokaságtól, mivelhogy úgy tartják vala õt mint prófétát.

\chapter{22}

\par 1 És megszólalván Jézus, ismét példázatokban beszél vala nékik, mondván:
\par 2 Hasonlatos a mennyeknek országa a királyhoz, a ki az õ fiának menyegzõt szerze.
\par 3 És elküldé szolgáit, hogy meghívják azokat, a kik a menyegzõre hivatalosak valának; de nem akarnak vala eljõni.
\par 4 Ismét külde más szolgákat, mondván: Mondjátok meg a hivatalosoknak: Ímé, ebédemet elkészítettem, tulkaim és hízlalt állataim levágva vannak, és kész minden; jertek el a menyegzõre.
\par 5 De azok nem törõdvén vele, elmenének, az egyik a maga szántóföldjére, a másik a maga kereskedésébe;
\par 6 A többiek pedig megfogván az õ szolgáit, bántalmazák és megölék õket.
\par 7 Meghallván pedig ezt a király, megharaguvék, és elküldvén hadait, azokat a gyilkosokat elveszté, és azoknak városait fölégeté.
\par 8 Akkor monda az õ szolgáinak: A menyegzõ ugyan készen van, de a hivatalosok nem valának méltók.
\par 9 Menjetek azért a keresztútakra, és a kiket csak találtok, hívjátok be a menyegzõbe.
\par 10 És kimenvén azok a szolgák az útakra, begyûjték mind a kiket csak találtak vala, jókat és gonoszokat egyaránt. És megtelék a menyegzõ vendégekkel.
\par 11 Bemenvén pedig a király, hogy megtekintse a vendégeket, láta ott egy embert, a kinek nem vala menyegzõi ruhája.
\par 12 És monda néki: Barátom, mi módon jöttél ide, holott nincsen menyegzõi ruhád? Az pedig hallgata.
\par 13 Akkor monda a király a szolgáknak: Kötözzétek meg a lábait és kezeit, és vigyétek és vessétek õt a külsõ sötétségre; ott lészen sírás és fogcsikorgatás.
\par 14 Mert sokan vannak a hivatalosok, de kevesen a választottak.
\par 15 Ekkor a farizeusok elmenvén, tanácsot tartának, hogy szóval ejtsék õt tõrbe.
\par 16 És elküldék hozzá tanítványaikat a Heródes pártiakkal, a kik ezt mondják vala: Mester, tudjuk, hogy igaz vagy és az Isten útját igazán tanítod, és nem törõdöl senkivel, mert embereknek személyére nem nézel.
\par 17 Mondd meg azért nékünk, mit gondolsz: Szabad-é a császárnak adót fizetnünk, vagy nem?
\par 18 Jézus pedig ismervén az õ álnokságukat, monda: Mit kisértgettek engem, képmutatók?
\par 19 Mutassátok nékem az adópénzt. Azok pedig oda vivének néki egy dénárt.
\par 20 És monda nékik: Kié ez a kép, és a felírás?
\par 21 Mondának néki: A császáré. Akkor monda nékik: Adjátok meg azért a mi a császáré a császárnak; és a mi az Istené, az Istennek.
\par 22 És ezt hallván, elcsodálkozának; és ott hagyván õt, elmenének.
\par 23 Ugyanazon a napon menének hozzá a sadduczeusok, a kik a feltámadást tagadják,  és megkérdezék õt,
\par 24 Mondván: Mester, Mózes azt mondotta: Ha valaki magzatok nélkül hal meg, annak testvére vegye el annak feleségét, és támaszszon magot testvérének.
\par 25 Vala pedig minálunk hét testvér: és az elsõ feleséget vevén, meghala; és mivelhogy nem vala magzata, feleségét a testvérére hagyá;
\par 26 Hasonlóképen a második is, a harmadik is, mind hetediglen.
\par 27 Legutoljára pedig az asszony is meghala.
\par 28 A feltámadáskor azért a hét közül melyiké lesz az asszony? Mert mindeniké vala.
\par 29 Jézus pedig felelvén, monda nékik: Tévelyegtek, mivelhogy nem ismeritek sem az írásokat, sem az Istennek hatalmát.
\par 30 Mert a feltámadáskor sem nem házasodnak, sem férjhez nem mennek, hanem olyanok lesznek, mint az Isten angyalai a mennyben.
\par 31 A halottak feltámadása felõl pedig nem olvastátok-é, a mit az Isten mondott néktek, így szólván:
\par 32 Én vagyok az Ábrahám Istene, és az Izsák Istene, és a Jákób Istene; az Isten nem holtaknak, hanem élõknek Istene.
\par 33 És a sokaság ezt hallván, csodálkozék az õ tudományán.
\par 34 A farizeusok pedig, hallván, hogy a sadduczeusokat elnémította vala, egybegyûlének;
\par 35 És megkérdé õt közülök egy törvénytudó, kisértvén õt, és mondván:
\par 36 Mester, melyik a nagy parancsolat a törvényben?
\par 37 Jézus pedig monda néki: Szeresd az Urat, a te Istenedet teljes szívedbõl, teljes lelkedbõl és teljes elmédbõl.
\par 38 Ez az elsõ és nagy parancsolat.
\par 39 A második pedig hasonlatos ehhez: Szeresd felebarátodat, mint magadat.
\par 40 E két parancsolattól függ az egész törvény és a próféták.
\par 41 Mikor pedig a farizeusok összegyülekezének, kérdezé õket Jézus,
\par 42 Mondván: Miképen vélekedtek ti a Krisztus felõl? kinek a fia? Mondának néki: A Dávidé.
\par 43 Monda nékik: Miképen hívja tehát õt Dávid lélekben Urának, ezt mondván:
\par 44 Monda az Úr az én Uramnak: Ülj az én jobb kezem felõl, míglen vetem a te ellenségeidet a te lábaid alá zsámolyul.
\par 45 Ha tehát Dávid Urának hívja õt, mi módon fia?
\par 46 És senki egy szót sem felelhet vala néki; sem pedig nem meri vala õt e naptól fogva többé senki megkérdezni.

\chapter{23}

\par 1 Akkor szóla Jézus a sokaságnak és az õ tanítványainak,
\par 2 Mondván: az írástudók és a farizeusok a Mózes székében ülnek:
\par 3 Annakokáért a mit parancsolnak néktek, mindazt megtartsátok és megcselekedjétek; de az õ cselekedeteik szerint ne cselekedjetek. Mert õk mondják, de nem cselekszik.
\par 4 Mert õk nehéz és elhordozhatatlan terheket kötöznek egybe, és az emberek vállaira vetik; de õk az ujjokkal sem akarják azokat illetni.
\par 5 Minden õ dolgaikat pedig csak azért cselekszik, hogy lássák õket az emberek: mert megszélesítik az õ homlokszíjjaikat; és megnagyobbítják az õ köntöseik peremét;
\par 6 És szeretik a lakomákon a fõhelyet, és a gyülekezetekben az elölûlést.
\par 7 És a piaczokon való köszöntéseket, és hogy az emberek így hívják õket: Mester, Mester!
\par 8 Ti pedig ne hivassátok magatokat Mesternek, mert egy a ti Mesteretek, a Krisztus; ti pedig mindnyájan testvérek vagytok.
\par 9 Atyátoknak se hívjatok senkit e földön; mert egy a ti Atyátok, a ki a mennyben van.
\par 10 Doktoroknak se hivassátok magatokat, mert egy a ti Doktorotok, a Krisztus.
\par 11 Hanem a ki a nagyobb közöttetek, legyen a ti szolgátok.
\par 12 Mert a ki magát felmagasztalja, megaláztatik; és a ki magát megalázza, felmagasztaltatik.
\par 13 De jaj néktek képmutató írástudók és farizeusok, mert a mennyeknek országát bezárjátok az emberek elõtt; mivelhogy ti nem mentek be, a kik be akarnának menni, azokat sem bocsátjátok be.
\par 14 Jaj néktek képmutató írástudók és farizeusok, mert felemésztitek az özvegyek házát, és színbõl hosszan imádkoztok; ezért annál súlyosabb lesz a ti büntetéstek.
\par 15 Jaj néktek képmutató írástudók és farizeusok! mert megkerülitek a tengert és a földet, hogy egy pogányt zsidóvá tegyetek; és ha azzá lett, a gyehenna fiává teszitek õt, kétszerte inkább magatoknál.
\par 16 Jaj néktek vak vezérek, a kik ezt mondjátok: Ha valaki a templomra esküszik, semmi az; de ha valaki a templom aranyára esküszik, tartozik az.
\par 17 Bolondok és vakok: mert melyik nagyobb, az arany-é, vagy a templom, a mely szentté teszi az aranyat?
\par 18 És: Ha valaki az oltárra esküszik, semmi az; de ha valaki a rajta levõ ajándékra esküszik, tartozik az.
\par 19 Bolondok és vakok: mert melyik nagyobb, az ajándék-é vagy az oltár, a mely szentté teszi az ajándékot?
\par 20 A ki azért az oltárra esküszik, esküszik arra és mindazokra, a mik azon vannak.
\par 21 És a ki a templomra esküszik, esküszik arra és Arra, a ki abban lakozik.
\par 22 És a ki az égre esküszik, esküszik az Isten királyiszékére és arra, ki abban ül.
\par 23 Jaj néktek képmutató írástudók és farizeusok! mert megdézsmáljátok a mentát, a kaprot és a köményt, és elhagyjátok a mik nehezebbek  a törvényben, az ítéletet, az irgalmasságot és a hívséget: pedig ezeket kellene cselekedni, és amazokat sem elhagyni.
\par 24 Vak vezérek, a kik megszûritek a szúnyogot, a tevét pedig elnyelitek.
\par 25 Jaj néktek képmutató írástudók és farizeusok! mert megtisztítjátok a pohárnak és tálnak külsejét, belõl pedig rakvák azok ragadománynyal és mértékletlenséggel.
\par 26 Vak farizeus, tisztítsd meg elõbb a pohár és tál belsejét, hogy külsejük is tiszta legyen.
\par 27 Jak néktek képmutató írástudók és farizeusok, mert hasonlatosak vagytok a meszelt sírokhoz, a melyek kívülrõl szépeknek tetszenek, belõl pedig holtaknak csontjaival és minden undoksággal rakvák.
\par 28 Épen így ti is, kívülrõl igazaknak látszotok ugyan az emberek elõtt, de belõl rakva vagytok képmutatással és törvénytelenséggel.
\par 29 Jaj néktek képmutató írástudók és farizeusok! mert építitek a próféták sírjait! és ékesgetitek az igazak síremlékeit.
\par 30 És ezt mondjátok: Ha mi atyáink korában éltünk volna, nem lettünk volna az õ bûntársaik a próféták vérében.
\par 31 Így hát magatok ellen tesztek bizonyságot, hogy fiai vagytok azoknak, a kik megölték a prófétákat.
\par 32 Töltsétek be ti is a ti atyáitoknak mértékét!
\par 33 Kígyók, mérges kígyóknak fajzatai, miképen kerülitek ki a gyehennának büntetését?
\par 34 Annakokáért ímé prófétákat, bölcseket és írástudókat küldök én hozzátok: és azok közül némelyeket megöltök, és megfeszítetek, másokat azok közül a ti zsinagógáitokban megostoroztok és városról-városra üldöztök.
\par 35 Hogy reátok szálljon minden igaz vér, a mely kiömlött a földön, az igaz Ábelnek  vérétõl Zakariásnak, a Barakiás fiának véréig, a kit a templom és az oltár között megöltetek.
\par 36 Bizony mondom néktek, mindezek reá következnek erre a nemzetségre.
\par 37 Jeruzsálem, Jeruzsálem! Ki megölöd a prófétákat és megkövezed azokat, a kik te hozzád küldettek, hányszor akartam egybegyûjteni a te fiaidat, miképen a tyúk egybegyûjti kis csirkéit szárnya alá; és te nem akartad.
\par 38 Ímé, pusztán hagyatik néktek a ti házatok.
\par 39 Mert mondom néktek: Mostantól fogva nem láttok engem mindaddig, mígnem ezt mondjátok: Áldott,  a ki jõ az Úrnak nevében!

\chapter{24}

\par 1 És kijõvén Jézus a templomból, tovább méne; és hozzámenének az õ tanítványai, hogy mutogassák néki a templom épületeit.
\par 2 Jézus pedig monda nékik: Nem látjátok-é mind ezeket? Bizony mondom néktek: Nem marad itt kõ kövön, mely le nem romboltatik.
\par 3 Mikor pedig az olajfák hegyén ül vala, hozzá menének a tanítványok magukban mondván: Mondd meg nékünk, mikor lesznek meg ezek? és micsoda jele lesz a te eljövetelednek, és a világ végének?
\par 4 És Jézus felelvén, monda nékik: Meglássátok, hogy valaki el ne hitessen titeket,
\par 5 Mert sokan jõnek majd az én nevemben, a kik ezt mondják: Én vagyok a Krisztus; és sokakat elhitetnek.
\par 6 Hallanotok kell majd háborúkról és háborúk híreirõl: meglássátok, hogy meg ne rémüljetek; mert mindezeknek meg kell lenniök. De még ez nem itt a vég.
\par 7 Mert nemzet támad nemzet ellen, és ország ország ellen; és lésznek éhségek és döghalálok, és földindulások mindenfelé.
\par 8 Mind ez pedig a sok nyomorúságnak kezdete.
\par 9 Akkor nyomorúságra adnak majd benneteket, és megölnek titeket; és gyûlöletesek lesztek minden nép elõtt az én nevemért.
\par 10 És akkor sokan megbotránkoznak, és elárulják egymást, és gyûlölik egymást.
\par 11 És sok hamis próféta támad, a kik sokakat elhitetnek.
\par 12 És mivelhogy a gonoszság megsokasodik, a szeretet sokakban meghidegül.
\par 13 De a ki mindvégig állhatatos marad, az idvezül.
\par 14 És az Isten országának ez az evangyélioma hirdettetik majd az egész világon,  bizonyságul minden népnek; és akkor jõ el a vég.
\par 15 Mikor azért látjátok majd, hogy az a pusztító utálatosság, a melyrõl Dániel próféta szólott, ott áll a szent helyen (a ki olvassa, értse meg):
\par 16 Akkor, a kik Júdeában lesznek, fussanak a hegyekre;
\par 17 A ház tetején levõ ne szálljon alá, hogy házából valamit kivigyen;
\par 18 És a mezõn levõ ne térjen vissza, hogy az õ ruháját elvigye.
\par 19 Jaj pedig a terhes és szoptató asszonyoknak azokon a napokon.
\par 20 Imádkozzatok pedig, hogy a ti futástok ne télen legyen, se szombatnapon:
\par 21 Mert akkor nagy nyomorúság lesz, a milyen nem volt a világ kezdete óta mind ez ideig, és nem lesz soha.
\par 22 És ha azok a napok meg nem rövidíttetnének, egyetlen ember sem menekülhetne meg; de a választottakért megrövidíttetnek majd azok a napok.
\par 23 Ha valaki ezt mondja akkor néktek: Ímé, itt a Krisztus, vagy amott; ne higyjétek.
\par 24 Mert hamis Krisztusok és hamis próféták támadnak, és nagy jeleket és csodákat tesznek, annyira, hogy elhitessék, ha lehet, a választottakat is.
\par 25 Ímé eleve megmondottam néktek.
\par 26 Azért ha azt mondják majd néktek: Ímé a pusztában van; ne menjetek ki. Ímé a belsõ szobákban; ne higyjétek.
\par 27 Mert a miképen a villámlás napkeletrõl támad és ellátszik egész napnyugtáig, úgy lesz az ember Fiának eljövetele is.
\par 28 Mert a hol a dög, oda gyûlnek a keselyûk.
\par 29 Mindjárt pedig ama napok nyomorúságai után a nap elsötétedik, és a hold nem fénylik, és a csillagok az égrõl lehullanak, és az egeknek erõsségei megrendülnek.
\par 30 És akkor feltetszik az ember Fiának jele az égen. És akkor sír a föld minden nemzetsége, és meglátják az embernek Fiát eljõni az ég  felhõiben nagy hatalommal és dicsõséggel.
\par 31 És elküldi az õ angyalait nagy trombitaszóval, és egybegyûjtik az õ választottait a négy szelek felõl, az ég egyik végétõl a másik végéig.
\par 32 A fügefáról vegyétek pedig a példát: mikor az ága már zsendül, és levelet hajt, tudjátok, hogy közel van a nyár:
\par 33 Azonképen ti is, mikor mindezeket látjátok, tudjátok meg, hogy közel van, az ajtó elõtt.
\par 34 Bizony mondom néktek, el nem múlik ez a nemzetség, mígnem mindezek meglesznek.
\par 35 Az ég és a föld elmúlnak, de az én beszédeim semmiképen el nem múlnak.
\par 36 Arról a napról és óráról pedig senki sem tud, az ég angyalai sem, hanem csak az én Atyám egyedül.
\par 37 A miképen pedig a Noé napjaiban vala, akképen lesz az ember Fiának eljövetele is.
\par 38 Mert a miképen az özönvíz elõtt való napokban esznek és isznak vala, házasodnak és férjhez mennek vala, mind ama napig, a melyen Noé a bárkába méne.
\par 39 És nem vesznek vala észre semmit, mígnem eljöve az özönvíz és mindnyájukat elragadá: akképen lesz az ember Fiának eljövetele is.
\par 40 Akkor ketten lesznek a mezõn; az egyik felvétetik, a másik ott hagyatik.
\par 41 Két asszony õröl a malomban; az egyik felvétetik, a másik ott hagyatik.
\par 42 Vigyázzatok azért, mert nem tudjátok, mely órában jõ el a ti Uratok.
\par 43 Azt pedig jegyezzétek meg, hogy ha tudná a ház ura, hogy az éjszakának melyik szakában jõ el a tolvaj: vigyázna, és nem engedné, hogy házába törjön.
\par 44 Azért legyetek készen ti is; mert a mely órában nem gondoljátok, abban jõ el az embernek Fia.
\par 45 Kicsoda hát a hû és bölcs szolga, a kit az õ ura gondviselõvé tõn az õ házanépén, hogy a maga idejében adjon azoknak eledelt?
\par 46 Boldog az a szolga, a kit az õ ura, mikor haza jõ, ily munkában talál.
\par 47 Bizony mondom néktek, hogy minden jószága fölött gondviselõvé teszi õt.
\par 48 Ha pedig ama gonosz szolga így szólna az õ szívében: Halogatja még az én uram a hazajövetelt;
\par 49 És az õ szolgatársait verni kezdené, a részegesekkel pedig enni és inni kezdene:
\par 50 Megjõ annak a szolgának az ura, a mely napon nem várja és a mely órában nem gondolja,
\par 51 És ketté vágatja õt, és a képmutatók sorsára juttatja; ott lészen sírás és fogcsikorgatás.

\chapter{25}

\par 1 Akkor hasonlatos lesz a mennyeknek országa ama tíz szûzhöz, a kik elõvevén az õ lámpásaikat, kimenének a võlegény elé.
\par 2 Öt pedig közülök eszes vala, és öt bolond.
\par 3 A kik bolondok valának, mikor lámpásaikat elõvevék, nem vivének magukkal olajat;
\par 4 Az eszesek pedig lámpásaikkal együtt olajat vivének az õ edényeikben.
\par 5 Késvén pedig a võlegény, mindannyian elszunnyadának és aluvának.
\par 6 Éjfélkor pedig kiáltás lõn: Ímhol jõ a võlegény! Jõjjetek elébe!
\par 7 Akkor felkelének mind azok a szûzek, és elkészíték az õ lámpásaikat.
\par 8 A bolondok pedig mondának az eszeseknek: Adjatok nékünk a ti olajotokból, mert a mi lámpásaink kialusznak.
\par 9 Az eszesek pedig felelének, mondván: Netalán nem lenne elegendõ nékünk és néktek; menjetek inkább az árúsokhoz, és vegyetek magatoknak.
\par 10 Mikor pedig venni járnak vala, megérkezék a võlegény; és a kik készen valának, bemenének õ vele a menyegzõbe, és bezáraték az ajtó.
\par 11 Késõbb pedig a többi szûzek is megjövének, mondván: Uram! Uram! nyisd meg mi nékünk.
\par 12 Õ pedig felelvén, monda: Bizony mondom néktek, nem ismerlek titeket.
\par 13 Vigyázzatok azért, mert sem a napot, sem az órát nem tudjátok, a melyen az embernek Fia eljõ.
\par 14 Mert épen úgy van ez, mint az az ember, a ki útra akarván kelni, eléhívatá az õ szolgáit, és a mije volt, átadá nékik.
\par 15 És ada az egyiknek öt tálentomot, a másiknak kettõt, a harmadiknak pedig egyet, kinek-kinek az õ erejéhez képest; és azonnal útra kele.
\par 16 Elmenvén pedig a ki az öt tálentomot kapta vala, kereskedék azokkal, és szerze más öt tálentomot.
\par 17 Azonképen a kié a kettõ vala, az is más kettõt nyere.
\par 18 A ki pedig az egyet kapta vala, elmenvén, elásá azt a földbe, és elrejté az õ urának pénzét.
\par 19 Sok idõ múlva pedig megjöve ama szolgáknak ura, és számot vete velök.
\par 20 És elõjövén a ki az öt tálentomot kapta vala, hoza más öt tálentomot, mondván: Uram, öt tálentomot adtál vala nékem; ímé más öt tálentomot nyertem azokon.
\par 21 Az õ ura pedig monda néki: Jól vagyon jó és hû szolgám, kevesen voltál hû, sokra bízlak ezután; menj be a te uradnak örömébe.
\par 22 Elõjövén pedig az is, a ki a két tálentomot kapta vala, monda: Uram, két tálentomot adtál volt nékem; ímé más két tálentomot nyertem azokon.
\par 23 Monda néki az õ ura: Jól vagyon jó és hû szolgám, kevesen voltál hû, sokra bízlak ezután; menj be a te uradnak örömébe.
\par 24 Elõjövén pedig az is, a ki az egy tálentomot kapta vala, monda: Uram, tudtam, hogy te kegyetlen ember vagy, a ki ott is aratsz, a hol nem vetettél, és ott is takarsz, a hol nem vetettél;
\par 25 Azért félvén, elmentem és elástam a te tálentomodat a földbe; ímé megvan a mi a tied.
\par 26 Az õ ura pedig felelvén, monda néki: Gonosz és rest szolga, tudtad, hogy ott is aratok, a hol nem vetettem, és ott is takarok, a hol nem vetettem;
\par 27 El kellett volna tehát helyezned az én pénzemet a pénzváltóknál; és én, megjövén, nyereséggel kaptam volna meg a magamét.
\par 28 Vegyétek el azért tõle a tálentomot, és adjátok annak, a kinek tíz tálentoma van.
\par 29 Mert mindenkinek, a kinek van, adatik, és megszaporíttatik; a kinek pedig nincsen, attól az is elvétetik, a mije van.
\par 30 És a haszontalan szolgát vessétek a külsõ sötétségre; ott lészen sírás és fogcsikorgatás.
\par 31 Mikor pedig eljõ az embernek Fia az õ dicsõségében, és õ vele mind a szent angyalok, akkor beül majd az õ dicsõségének királyiszékébe.
\par 32 És elébe gyûjtetnek mind a népek, és elválasztja õket egymástól, miként a pásztor elválasztja a juhokat a kecskéktõl.
\par 33 És a juhokat jobb keze felõl, a kecskéket pedig bal keze felõl állítja.
\par 34 Akkor ezt mondja a király a jobb keze felõl állóknak: Jertek, én Atyámnak áldottai, örököljétek ez országot, a mely számotokra készíttetett a világ megalapítása óta.
\par 35 Mert éheztem, és ennem adtatok; szomjúhoztam, és innom adtatok; jövevény voltam, és befogadtatok engem;
\par 36 Mezítelen voltam, és megruháztatok; beteg voltam, és meglátogattatok; fogoly voltam, és eljöttetek hozzám.
\par 37 Akkor felelnek majd néki az igazak, mondván: Uram, mikor láttuk, hogy éheztél, és tápláltunk volna? vagy szomjúhoztál, és innod adtunk volna?
\par 38 És mikor láttuk, hogy jövevény voltál, és befogadtunk volna? vagy mezítelen voltál, és felruháztunk volna?
\par 39 Mikor láttuk, hogy beteg vagy fogoly voltál, és hozzád mentünk volna?
\par 40 És felelvén a király, azt mondja majd nékik: Bizony mondom néktek, a mennyiben megcselekedtétek egygyel az én legkisebb atyámfiai közül, én velem cselekedtétek meg.
\par 41 Akkor szól majd az õ bal keze felõl állókhoz is: Távozzatok tõlem, ti átkozottak, az örök tûzre, a mely az ördögöknek és az õ angyalainak készíttetett.
\par 42 Mert éheztem, és nem adtatok ennem; szomjúhoztam, és nem adtatok innom;
\par 43 Jövevény voltam, és nem fogadtatok be engem; mezítelen voltam, és nem ruháztatok meg engem; beteg és fogoly voltam, és nem látogattatok meg engem.
\par 44 Akkor ezek is felelnek majd néki, mondván: Uram, mikor láttuk, hogy éheztél, vagy szomjúhoztál, vagy hogy jövevény, vagy mezítelen, vagy beteg, vagy fogoly voltál, és nem szolgáltunk volna néked?
\par 45 Akkor felel majd nékik, mondván: Bizony mondom néktek, a mennyiben nem cselekedtétek meg egygyel eme legkisebbek közül, én velem sem cselekedtétek meg.
\par 46 És ezek elmennek majd az örök gyötrelemre; az igazak pedig az örök életre.

\chapter{26}

\par 1 És mikor mindezeket a beszédeket elvégezte vala Jézus, monda az õ tanítványainak:
\par 2 Tudjátok, hogy két nap mulva a husvétnak ünnepe lészen, és az embernek Fia  elárultatik, hogy megfeszíttessék.
\par 3 Akkor egybegyûlének a fõpapok, az írástudók és a nép vénei a fõpap házába, a kit Kajafásnak hívtak,
\par 4 És tanácsot tartának, hogy Jézust álnoksággal megfogják és megöljék.
\par 5 De azt mondják vala: Ne az ünnepen: hogy zendülés ne legyen a nép között.
\par 6 És mikor Jézus Bethániában, a poklos Simon házánál vala,
\par 7 Méne õ hozzá egy asszony, a kinél vala drága kenetnek alabástrom szelenczéje, és az õ fejére tölté, a mint az asztalnál ül vala.
\par 8 Látván pedig ezt az õ tanítványai, bosszankodának, mondván: Mire való ez a tékozlás?
\par 9 Mert eladhatták volna ezt a kenetet nagy áron, és adhatták volna a szegényeknek.
\par 10 Mikor pedig ezt eszébe vette Jézus, monda nékik: Miért bántjátok ezt az asszonyt? hiszen jó dolgot cselekedett én velem.
\par 11 Mert a szegények mindenkor veletek lesznek, de én nem leszek mindenkor veletek.
\par 12 Mert hogy õ ezt a kenetet testemre töltötte, az én temetésemre nézve cselekedte azt.
\par 13 Bizony mondom néktek: Valahol az egész világon prédikáltatik az evangyéliom, a mit ez én velem cselekedék, az is hirdettetik az õ emlékezetére.
\par 14 Akkor a tizenkettõ közül egy, a kit Iskariótes Júdásnak hívtak, a fõpapokhoz menvén,
\par 15 Monda: Mit akartok nékem adni, és én kezetekbe adom õt?  Azok pedig rendelének néki harmincz ezüst pénzt.
\par 16 És ettõl fogva alkalmat keres vala, hogy elárulja õt.
\par 17 A kovásztalan kenyerek elsõ napján pedig Jézushoz menének a tanítványok, mondván: Hol akarod, hogy megkészítsük néked ételedre  husvéti bárányt?
\par 18 Õ pedig monda: Menjetek el a városba ama bizonyos emberhez, és ezt mondjátok néki: A Mester üzeni: Az én idõm közel van; nálad tartom meg a husvétot tanítványaimmal.
\par 19 És úgy cselekedének a tanítványok, a mint Jézus parancsolta vala nékik; és elkészíték a husvéti bárányt.
\par 20 Mikor pedig beestveledék, letelepszik vala a tizenkettõvel,
\par 21 És a mikor esznek vala, monda: Bizony mondom néktek, ti közületek egy elárul engem.
\par 22 És felettébb megszomorodva, kezdék mindannyian mondani néki: Én vagyok-é az, Uram?
\par 23 Õ pedig felelvén, monda: A ki velem együtt mártja kezét a tálba, az árul el engem.
\par 24 Az embernek Fia jóllehet elmegyen, a mint meg van írva felõle, de jaj annak az embernek, a ki az embernek Fiát elárulja; jobb volna annak az embernek, ha nem született volna.
\par 25 Megszólalván Júdás is, a ki elárulja vala õt, monda: Én vagyok-é az, Mester? Monda néki: Te mondád.
\par 26 Mikor pedig evének, vevé Jézus a kenyeret és hálákat adván, megtöré és adá a tanítványoknak, és monda: Vegyétek, egyétek; ez az én testem.
\par 27 És vevén a poharat és hálákat adván, adá azoknak, ezt mondván: Igyatok ebbõl mindnyájan;
\par 28 Mert ez az én vérem, az új szövetségnek vére, a mely sokakért kiontatik bûnöknek bocsánatára.
\par 29 Mondom pedig néktek, hogy: Mostantól fogva nem iszom a szõlõtõkének ebbõl a termésébõl mind ama napig, a mikor újan iszom azt veletek az én Atyámnak országában.
\par 30 És dícséretet énekelvén,  kimenének az olajfák hegyére.
\par 31 Akkor monda nékik Jézus: Mindnyájan ezen az éjszakán megbotránkoztok én bennem. Mert meg van írva: Megverem a pásztort, és elszélednek a nyájnak juhai.
\par 32 De föltámadásom után elõttetek megyek majd Galileába.
\par 33 Péter pedig felelvén, monda néki: Ha mindnyájan megbotránkoznak is te benned, én soha meg nem botránkozom.
\par 34 Monda néki Jézus: Bizony mondom néked, ezen az éjszakán, mielõtt megszólal a kakas, háromszor megtagadsz engem.
\par 35 Monda néki Péter: Ha meg kell is veled halnom, meg nem tagadlak téged. Hasonlóképen szólnak vala a többi tanítványok is.
\par 36 Akkor elméne Jézus velök egy helyre, a melyet Gecsemánénak hívtak, és monda a tanítványoknak: Üljetek le itt, míg elmegyek és amott imádkozom.
\par 37 És maga mellé vévén Pétert és Zebedeusnak két fiát, kezde szomorkodni és gyötrõdni.
\par 38 Ekkor monda nékik: Felette igen szomorú az én lelkem mind halálig! maradjatok itt és vigyázzatok én velem.
\par 39 És egy kissé elõre menve, arczra borula, könyörögvén és mondván: Atyám! ha lehetséges, múljék el tõlem e  pohár; mindazáltal ne úgy legyen a mint én akarom, hanem a mint te.
\par 40 Akkor méne a tanítványokhoz és aluva találá õket, és monda Péternek: Így nem birtatok vigyázni velem egy óráig sem!?
\par 41 Vigyázzatok és imádkozzatok, hogy kísértetbe ne essetek; mert jóllehet a lélek kész, de a test erõtelen.
\par 42 Ismét elméne másodszor is, és könyörge, mondván: Atyám! ha el nem múlhatik tõlem e pohár, hogy ki ne igyam, legyen meg a te akaratod.
\par 43 És mikor visszatér vala, ismét aluva találá õket; mert megnehezedtek vala az õ szemeik.
\par 44 És ott hagyva õket, ismét elméne és imádkozék harmadszor, ugyanazon beszéddel szólván.
\par 45 Ekkor méne az õ tanítványaihoz, és monda nékik: Aludjatok immár és nyugodjatok. Ímé, elközelgett az óra, és az embernek Fia a bûnösök kezébe adatik.
\par 46 Keljetek fel, menjünk! Ímé elközelgett, a ki engem elárul.
\par 47 És még mikor beszél vala, ímé Júdás, egy a tizenkettõ közül, eljöve és vele együtt sok nép fegyverekkel és fustélyokkal, a fõpapoktól és a nép véneitõl.
\par 48 A ki pedig õt elárulja vala, jelt ada nékik, mondván: A kit én majd megcsókolok, õ az, fogjátok meg õt.
\par 49 És mindjárt Jézushoz lépvén, monda: Üdvöz légy Mester! és megcsókolá õt.
\par 50 Jézus pedig monda néki: Barátom, miért jöttél? Akkor hozzámenvén, kezeiket Jézusra veték és megfogák õt.
\par 51 És ímé egyik azok közül, a kik a Jézussal valának, kinyújtván kezét, szablyáját kirántá, és a fõpap szolgáját megcsapván, levágá annak egyik fülét.
\par 52 Akkor monda néki Jézus: Tedd helyére szablyádat; mert a kik fegyvert fognak, fegyverrel kell veszniök.
\par 53 Avagy azt gondolod-é, hogy nem kérhetném most az én Atyámat, hogy adjon ide mellém többet tizenkét sereg angyalnál?
\par 54 De mi módon teljesednének be az írások, hogy így kell lenni?
\par 55 Ugyanekkor monda Jézus a sokaságnak: Mint valami latorra, úgy jöttetek fegyverekkel és fustélyokkal, hogy megfogjatok engem? Naponként nálatok ültem, tanítván a templomban, és nem fogtatok meg engem.
\par 56 Mindez pedig azért lõn, hogy beteljesedjenek a próféták írásai. Ekkor elhagyák õt a tanítványok mind, és elfutának.
\par 57 Amazok pedig megfogván Jézust, vivék Kajafáshoz, a fõpaphoz, a hol az írástudók és a vének egybegyûltek vala.
\par 58 Péter pedig követi vala õt távolról egész a fõpap pitvaráig; és bemenvén, ott ül vala a szolgákkal, hogy lássa a végét.
\par 59 A fõpapok pedig és a vének és az egész tanács hamis bizonyságot keresnek vala Jézus ellen, hogy megölhessék õt;
\par 60 És nem találának. És noha sok hamis tanú jött vala elõ, még sem találának. Utoljára pedig elõjövén két hamis tanú,
\par 61 Monda: Ez azt mondta: Leronthatom az Isten templomát, és három nap alatt felépíthetem azt.
\par 62 És fölkelvén a fõpap, monda néki: Semmit sem felelsz-é? Micsoda tanúbizonyságot tesznek ezek ellened?
\par 63 Jézus pedig hallgat vala. És felelvén a fõpap, monda néki: Az élõ Istenre kényszerítelek téged, hogy mondd meg nékünk, ha te vagy-é a Krisztus, az Istennek Fia?
\par 64 Monda néki Jézus: Te mondád. Sõt mondom néktek: Mostantól fogva meglátjátok az embernek Fiát ülni az Istennek hatalmas jobbján, és eljõni az égnek felhõiben.
\par 65 Ekkor a fõpap megszaggatá a maga ruháit, és monda: Káromlást szólott. Mi szükségünk van még bizonyságokra? Ímé most hallottátok az õ káromlását.
\par 66 Mit gondoltok? Azok pedig felelvén mondának: Méltó a halálra.
\par 67 Akkor szemébe köpdösének és arczul csapdosák õt, némelyek pedig botokkal verék,
\par 68 Mondván: Prófétáld meg nékünk Krisztus, kicsoda az, a ki üt téged?
\par 69 Péter pedig künn ül vala az udvaron, és hozzá menvén egy szolgálóleány, monda: Te is a Galileabeli Jézussal valál.
\par 70 Õ pedig mindenkinek hallatára megtagadá, mondván: Nem tudom, mit beszélsz.
\par 71 Mikor pedig kiméne a tornáczra meglátá õt egy másik szolgálóleány, és monda az ott levõknek: Ez is a názáreti Jézussal vala.
\par 72 És ismét megtagadá esküvéssel, hogy: Nem is ismerem ezt az embert.
\par 73 Kevés idõ múlva pedig az ott álldogálók menének hozzá, és mondának Péternek: Bizony te is közülök való vagy; hiszen a te beszéded is elárul téged.
\par 74 Ekkor átkozódni és esküdözni kezde, hogy: Nem ismerem ezt az embert. És a kakas azonnal megszólala.
\par 75 És megemlékezék Péter a Jézus beszédérõl, ki ezt mondotta vala néki: Mielõtt a kakas szólana, háromszor megtagadsz engem; és kimenvén onnan, keservesen síra.

\chapter{27}

\par 1 Mikor pedig reggel lõn, tanácsot tartának mind a fõpapok és a nép vénei Jézus ellen, hogy õt megöljék.
\par 2 És megkötözvén õt, elvivék, és átadák õt Ponczius Pilátusnak a helytartónak.
\par 3 Akkor látván Júdás, a ki õt elárulá, hogy elítélték õt, megbánta dolgát, és visszavivé a harmincz ezüst pénzt a fõpapoknak és a véneknek,
\par 4 Mondván: Vétkeztem, hogy elárultam az ártatlan vért. Azok pedig mondának: Mi közünk hozzá? Te lássad.
\par 5 Õ pedig eldobván az ezüst pénzeket a templomban, eltávozék; és elmenvén felakasztá magát.
\par 6 A fõpapok pedig felszedvén az ezüst pénzeket, mondának: Nem szabad ezeket a templom kincsei közé tennünk, mert vérnek ára.
\par 7 Tanácsot ülvén pedig, megvásárlák azon a fazekasnak mezejét idegenek számára való temetõnek.
\par 8 Ezért hívják ezt a mezõt vérmezejének mind e mai napig.
\par 9 Ekkor teljesedék be a Jeremiás próféta mondása, a ki így szólott: És vevék a harmincz ezüst pénzt, a megbecsültnek árát, a kit Izráel fiai részérõl megbecsültek,
\par 10 És adák azt a fazekas mezejéért, a mint az Úr rendelte volt nékem.
\par 11 Jézus pedig ott álla a helytartó elõtt; és kérdezé õt a helytartó, mondván: Te vagy-é a zsidók királya? Jézus pedig monda néki: Te mondod.
\par 12 És mikor vádolák õt a fõpapok és a vének, semmit sem felele.
\par 13 Akkor monda néki Pilátus: Nem hallod-é, mily sok bizonyságot tesznek ellened?
\par 14 És nem felele néki egyetlen szóra sem, úgy hogy a helytartó igen elcsodálkozék.
\par 15 Ünnepenként pedig egy foglyot szokott szabadon bocsátani a helytartó a sokaság kedvéért, a kit akarának.
\par 16 Vala pedig akkor egy nevezetes foglyuk, a kit Barabbásnak hívtak.
\par 17 Mikor azért egybegyülekezének, monda nékik Pilátus: Melyiket akarjátok hogy elbocsássam néktek: Barabbást-é, vagy Jézust, a kit Krisztusnak hívnak?
\par 18 Mert jól tudja vala, hogy irigységbõl adák õt kézbe.
\par 19 A mint pedig õ az ítélõszékben ül vala, külde õ hozzá a felesége, ezt üzenvén: Ne avatkozzál amaz igaz ember dolgába; mert sokat szenvedtem ma álmomban õ miatta.
\par 20 A fõpapok és vének pedig reá beszélék a sokaságot, hogy Barabbást kérjék ki, Jézust pedig veszítsék el.
\par 21 Felelvén pedig a helytartó, monda nékik: A kettõ közül melyiket akarjátok, hogy elbocsássam néktek? Azok pedig mondának: Barabbást.
\par 22 Monda nékik Pilátus: Mit cselekedjem hát Jézussal, a kit Krisztusnak hívnak? Mindnyájan mondának: Feszíttessék meg!
\par 23 A helytartó pedig monda: Mert mi rosszat cselekedett? Azok pedig még inkább kiáltoznak vala, mondván: Feszíttessék meg!
\par 24 Pilátus pedig látván, hogy semmi sem használ, hanem még nagyobb háborúság támad, vizet vévén, megmosá kezeit a sokaság elõtt, mondván: Ártatlan vagyok ez igaz embernek vérétõl; ti lássátok!
\par 25 És felelvén az egész nép, monda: Az õ vére mi rajtunk és a mi magzatainkon.
\par 26 Akkor elbocsátá nékik Barabbást; Jézust pedig megostoroztatván, kezökbe adá, hogy megfeszíttessék.
\par 27 Akkor a helytartó vitézei elvivék Jézust az õrházba, és oda gyûjték hozzá az egész csapatot.
\par 28 És levetkeztetvén õt, bíbor palástot adának reá.
\par 29 És tövisbõl fonott koronát tõnek a fejére, és nádszálat a jobb kezébe; és térdet hajtva elõtte, csúfolják vala õt, mondván: Üdvöz légy zsidóknak királya!
\par 30 És mikor megköpdösék õt, elvevék a nádszálat, és a fejéhez verdesik vala.
\par 31 És miután megcsúfolták, levevék róla a palástot és az õ maga ruháiba öltözteték; és elvivék, hogy megfeszítsék õt.
\par 32 Kifelé menve pedig találkozának egy czirénei emberrel, a kit Simonnak hívnak vala; ezt kényszeríték, hogy vigye az õ keresztjét.
\par 33 És mikor eljutának arra a helyre, a melyet Golgothának, azaz koponya helyének neveznek,
\par 34 Méreggel megelegyített eczetet adának néki inni; és megízlelvén, nem akara inni.
\par 35 Minek utána pedig megfeszíték õt, eloszták az õ ruháit, sorsot vetvén; hogy beteljék a próféta mondása: Megosztozának az én ruháimon, és az én köntösömre sorsot vetének.
\par 36 És leülvén, ott õrzik vala õt.
\par 37 És feje fölé illeszték az õ kárhoztatásának okát, oda írván: Ez Jézus, a zsidók királya.
\par 38 Akkor megfeszítének vele együtt két latrot, egyiket jobbkéz velõl, és a másikat balkéz felõl.
\par 39 Az arramenõk pedig szidalmazzák vala õt, fejüket hajtogatván.
\par 40 És ezt mondván: Te, ki lerontod a templomot és harmadnapra fölépíted, szabadítsd meg magadat; ha Isten Fia vagy, szállj le a keresztrõl!
\par 41 Hasonlóképen a fõpapok is csúfolódván az írástudókkal és a vénekkel egyetemben, ezt mondják vala:
\par 42 Másokat megtartott, magát nem tudja megtartani. Ha Izráel királya, szálljon le most a keresztrõl, és majd hiszünk néki.
\par 43 Bízott az Istenben; mentse meg most õt, ha akarja; mert azt mondta: Isten Fia vagyok.
\par 44 A kiket vele együtt feszítének meg, a latrok is ugyanazt hányják vala szemére.
\par 45 Hat órától kezdve pedig sötétség lõn mind az egész földön, kilencz óráig.
\par 46 Kilencz óra körül pedig nagy fenszóval kiálta Jézus, mondván: ELI, ELI! LAMA SABAKTÁNI? azaz: Én Istenem, én Istenem! miért hagyál el engemet?
\par 47 Némelyek pedig az ott állók közül, a mint ezt hallák, mondának: Illést hívja ez.
\par 48 És egy közülök azonnal oda futamodván, egy szivacsot võn, és megtöltvén eczettel és egy nádszálra tûzvén, inni ád vala néki.
\par 49 A többiek pedig ezt mondják vala: Hagyd el, lássuk eljõ-é Illés, hogy megszabadítsa õt?
\par 50 Jézus pedig ismét nagy fenszóval kiáltván, kiadá lelkét.
\par 51 És ímé a templom kárpítja fölétõl aljáig ketté hasada; és a föld megindula, és a kõsziklák megrepedezének;
\par 52 És a sírok megnyílának, és sok elhúnyt szentnek teste föltámada.
\par 53 És kijövén a sírokból, a Jézus föltámadása után bementek a szent városba, és sokaknak megjelenének.
\par 54 A százados pedig és a kik õ vele õrizték vala Jézust, látván a földindulást és a mik történtek vala, igen megrémülének, mondván: Bizony, Istennek Fia vala ez!
\par 55 Sok asszony vala pedig ott, a kik távolról szemlélõdnek vala, a kik Galileából követték Jézust, szolgálván néki;
\par 56 Ezek közt volt Mária Magdaléna, és Mária a Jakab és Józsé anyja, és a Zebedeus fiainak anyja.
\par 57 Mikor pedig beesteledék, eljöve egy gazdag ember Arimathiából, név szerint  József, a ki maga is tanítványa volt Jézusnak;
\par 58 Ez Pilátushoz menvén, kéri vala a Jézus testét. Akkor parancsolá Pilátus, hogy adják át a testet.
\par 59 És magához vévén József a testet, begöngyölé azt tiszta gyolcsba,
\par 60 És elhelyezé azt a maga új sírjába, a melyet a sziklába vágatott: és a sír szájára egy nagy követ hengerítvén, elméne.
\par 61 Ott vala pedig Mária Magdaléna és a másik Mária, a kik a sír átellenében ülnek vala.
\par 62 Másnap pedig, a mely péntek után következik, egybegyûlének a fõpapok és a farizeusok Pilátushoz,
\par 63 Ezt mondván: Uram, emlékezünk, hogy az a hitetõ még életében azt mondotta volt: Harmadnapra föltámadok.
\par 64 Parancsold meg azért, hogy õrizzék a sírt harmadnapig, ne hogy az õ tanítványai odamenvén éjjel, ellopják õt és azt mondják a népnek: Feltámadott a halálból; és az utolsó hitetés gonoszabb legyen az elsõnél.
\par 65 Pilátus pedig monda nékik: Van õrségetek; menjetek, õríztessétek, a mint tudjátok.
\par 66 Õk pedig elmenvén, a sírt õrizet alá helyezék, lepecsételvén a követ, az õrséggel.

\chapter{28}

\par 1 A szombat végén pedig, a hét elsõ napjára virradólag, kiméne Mária Magdaléna és a másik Mária, hogy megnézzék a sírt.
\par 2 És ímé nagy földindulás lõn; mert az Úrnak angyala leszállván a mennybõl, és oda menvén, elhengeríté a követ a sír szájáról, és reá üle arra.
\par 3 A tekintete pedig olyan volt, mint a villámlás, és a ruhája fehér, mint a hó.
\par 4 Az õrizõk pedig tõle való féltökben megrettenének, és olyanokká lõnek mint a holtak.
\par 5 Az angyal pedig megszólalván, monda az asszonyoknak: Ti ne féljetek; mert tudom, hogy a megfeszített Jézust keresitek.
\par 6 Nincsen itt, mert feltámadott, a mint megmondotta volt. Jertek, lássátok a helyet, a hol feküdt vala  az Úr.
\par 7 És menjetek gyorsan és mondjátok meg az õ tanítványainak, hogy feltámadott a halálból; és ímé elõttetek megy Galileába; ott meglátjátok õt, ímé megmondottam néktek.
\par 8 És gyorsan eltávozván a sírtól félelemmel és nagy örömmel, futnak vala, hogy megmondják az õ tanítványainak.
\par 9 Mikor pedig mennek vala, hogy megmondják az õ tanítványainak, ímé szembe jöve õ velök Jézus, mondván: Legyetek üdvözölve! Azok pedig hozzá járulván, megragadák az õ lábait, és leborulának elõtte.
\par 10 Akkor monda nékik Jézus: Ne féljetek; menjetek el, mondjátok meg az én atyámfiainak, hogy menjenek Galileába, és ott meglátnak engem.
\par 11 A mialatt pedig õk mennek vala, ímé az õrségbõl némelyek bemenvén a városba, megjelentének a fõpapoknak mindent a mi történt.
\par 12 És egybegyülekezvén a vénekkel együtt, és tanácsot tartván, sok pénzt adának a vitézeknek,
\par 13 Ezt mondván: Mondjátok, hogy: Az õ tanítványai odajövén éjjel, ellopták õt, mikor mi aluvánk.
\par 14 És ha ez a helytartó fülébe jut, mi elhitetjük õt, és kimentünk titeket a bajból.
\par 15 Azok pedig fölvevén a pénzt, úgy cselekedének, a mint megtanították õket. És elterjedt ez a hír a zsidók között mind e mai napig.
\par 16 A tizenegy tanítvány pedig elméne Galileába, a hegyre, a hová Jézus rendelte vala õket.
\par 17 És mikor megláták õt, leborulának elõtte; némelyek pedig kételkedének.
\par 18 És hozzájuk menvén Jézus, szóla nékik, mondván: Nékem adatott minden hatalom mennyen és földön.
\par 19 Elmenvén azért, tegyetek tanítványokká minden népeket, megkeresztelvén õket az Atyának, a Fiúnak és a Szent Léleknek nevében,
\par 20 Tanítván õket, hogy megtartsák mindazt, a mit én parancsoltam néktek: és ímé én ti veletek vagyok minden napon a világ végezetéig. Ámen!


\end{document}