\begin{document}

\title{Márk evangéliuma}


\chapter{1}

\par 1 A Jézus Krisztus, az Isten Fia evangyéliomának kezdete,
\par 2 A mint meg van írva a prófétáknál: Ímé én elküldöm az én követemet a te orczád elõtt, a ki megkészíti a te útadat elõtted;
\par 3 Kiáltónak szava a pusztában: Készítsétek meg az Úrnak útját, egyengessétek meg az õ ösvényeit:
\par 4 Elõáll vala János, keresztelvén a pusztában és prédikálván a megtérésnek keresztségét a bûnöknek bocsánatára.
\par 5 És kiméne hozzá Júdeának egész tartománya és a Jeruzsálembeliek is, és megkeresztelkedének mindnyájan õ általa a Jordán vizében, bûneikrõl vallást tévén.
\par 6 János pedig teveszõrruhát és dereka körül bõrövet viselt vala, és sáskát és erdei mézet eszik vala.
\par 7 És prédikála, mondván: Utánam jõ, a ki erõsebb nálam, a kinek nem vagyok méltó, hogy lehajolván, sarujának szíjját megoldjam.
\par 8 Én vízzel kereszteltelek titeket, de õ Szent Lélekkel keresztel titeket.
\par 9 És lõn azokban a napokban, eljöve Jézus a galileai Názáretbõl, és megkeresztelteték János által a Jordánban.
\par 10 És azonnal feljõvén a vízbõl, látá az egeket megnyilatkozni, és a Lelket mint egy galambot õ reá leszállani;
\par 11 És szózat lõn az égbõl: Te vagy az én szerelmes fiam, a kiben én gyönyörködöm.
\par 12 És a lélek azonnal elragadá õt a pusztába.
\par 13 És ott volt a pusztában negyven napig kísértetve a Sátántól, és a vad állatokkal vala együtt; és az angyalok szolgálnak vala néki.
\par 14 Minekutána pedig János tömlöczbe vettetett, elméne Jézus Galileába, prédikálván az Isten országának evangyéliomát,
\par 15 És mondván: Bétölt az idõ, és elközelített az Istennek országa; térjetek meg, és higyjetek az evangyéliomban.
\par 16 Mikor pedig Galilea tengere mellett járt, látá Simont és Andrást, annak testvérét, a mint a tengerbe hálót vetének; mert halászok valának.
\par 17 És monda nékik Jézus: Kövessetek engem, és én azt mívelem, hogy embereket halászszatok.
\par 18 És azonnal elhagyván az õ hálóikat, követék õt.
\par 19 És onnan egy kevéssé elébb menve, látá Jakabot, a Zebedeus fiát és annak testvérét, Jánost, a mint a hajóban azok is a hálókat kötözgetik vala.
\par 20 És azonnal hívá õket. És õk atyjukat, Zebedeust a napszámosokkal a hajóban hagyva, utána menének.
\par 21 És bemenének Kapernaumba; és mindjárt szombatnapon bemenvén a zsinagógába, tanít vala.
\par 22 És elálmélkodának az õ tanításán; mert úgy tanítja vala õket, mint a kinek hatalma van, és nem úgy mint az írástudók.
\par 23 Vala pedig azok zsinagógájában egy ember, a kiben tisztátalan lélek volt, és felkiálta,
\par 24 És monda: Ah! mi dolgunk van nékünk veled, Názáreti Jézus? Azért jöttél-é, hogy elveszíts minket? Tudom, hogy ki vagy te:  az Istennek Szentje.
\par 25 És megdorgálá õt Jézus, mondván: Némulj meg, és menj ki belõle.
\par 26 És a tisztátalan lélek megszaggatá õt, és fenszóval kiáltva, kiméne belõle.
\par 27 És mindnyájan elálmélkodának, annyira, hogy egymás között kérdezgeték, mondván: Mi ez? Micsoda új tudomány ez, hogy hatalommal parancsol a tisztátalan lelkeknek is, és engedelmeskednek néki?
\par 28 És azonnal elméne az õ híre Galilea egész környékére.
\par 29 És a zsinagógából azonnal kimenvén, a Simon és András házához menének Jakabbal és Jánossal együtt.
\par 30 A Simon napa pedig hideglelésben fekszik vala, és azonnal szólának néki felõle.
\par 31 És õ odamenvén, fölemelé azt, annak kezét fogván; és elhagyá azt a hideglelés azonnal, és szolgál vala nékik.
\par 32 Estefelé pedig, a mikor leszállt a nap, mind õ hozzá vivék a betegeseket és az ördöngõsöket;
\par 33 És az egész város oda gyûlt vala az ajtó elé.
\par 34 És meggyógyíta sokakat, a kik különféle betegségekben sínlõdnek vala; és sok ördögöt kiûze, és nem hagyja vala szólni az ördögöket, mivelhogy õt ismerék.
\par 35 Kora reggel pedig, még szürkületkor, fölkelvén, kiméne, és elméne egy puszta helyre és ott imádkozék.
\par 36 Simon pedig és a vele lévõk utána sietének;
\par 37 És a mikor megtalálák õt, mondának néki: Mindenki téged keres.
\par 38 És õ monda nékik: Menjünk a közel való városokba, hogy ott is prédikáljak, mert azért jöttem.
\par 39 És prédikál vala azoknak zsinagógáiban, egész Galileában, és ördögöket ûz vala.
\par 40 És jöve hozzá egy bélpoklos, kérvén õt és leborulván elõtte és mondván néki: Ha akarod, megtisztíthatsz engem.
\par 41 Jézus pedig könyörületességre indulván, kezét kinyújtva megérinté õt, és monda néki: Akarom, tisztulj meg.
\par 42 És a mint ezt mondja vala, azonnal eltávozék tõle a poklosság és megtisztula.
\par 43 És erõsen megfenyegetvén, azonnal elküldé õt,
\par 44 És monda néki: Meglásd, hogy senkinek semmit ne szólj; hanem eredj el, mutasd meg magadat a papnak, és vidd fel a te tisztulásodért, a mit Mózes parancsolt, bizonyságul nékik.
\par 45 Az pedig kimenvén, kezde sokat beszélni és terjeszteni a dolgot, annyira, hogy nyilvánosan immár be sem mehetett Jézus a városba, hanem künn puszta helyeken vala, és mennek vala hozzá mindenfelõl.

\chapter{2}

\par 1 Napok mulva pedig ismét beméne Kapernaumba, és meghallák, hogy otthon van.
\par 2 És azonnal sokan összegyülekezének, annyira, hogy még az ajtó elébe sem fértek; és hirdeté nékik az igét.
\par 3 És jövének hozzá egy gutaütöttet hozva, a kit négyen emelnek vala.
\par 4 És mivel a sokaság miatt nem férkõzhettek azzal õ hozzá, megbonták ama háznak fedelét, a hol Õ vala, és rést törvén, leereszték a nyoszolyát, a melyben a gutaütött feküdt.
\par 5 Jézus pedig azoknak hitét látván, monda a gutaütöttnek: Fiam, megbocsáttattak néked a te bûneid.
\par 6 Valának pedig ott némely írástudók, a kik ott ülnek vala, szívökben így okoskodván:
\par 7 Mi dolog, hogy ez ilyen káromlásokat szól? ki bocsáthatja meg a bûnöket, hanemha egyedül az Isten?
\par 8 És Jézus azonnal észrevevé az õ lelkével, hogy azok magukban így okoskodnak, és monda nékik: Miért gondoljátok ezeket a ti szívetekben?
\par 9 Mi könnyebb, azt mondanom-é a gutaütöttnek: Megbocsáttattak néked a te bûneid, vagy ezt mondanom: Kelj fel, vedd fel a te nyoszolyádat, és járj?
\par 10 Hogy pedig megtudjátok, hogy az ember Fiának van hatalma e földön a bûnöket megbocsátani, monda a gutaütöttnek:
\par 11 Mondom néked, kelj föl, vedd fel a te nyoszolyádat, és eredj haza.
\par 12 Az pedig azonnal fölkele és felvévén nyoszolyáját, kiméne mindenkinek láttára; úgy hogy mindenki elálmélkodék, és dicsõíté az Istent, ezt mondván: Soha sem láttunk ilyet!
\par 13 És ismét kiméne a tenger mellé; és az egész sokaság megy vala õ hozzá, és õ tanítja vala õket.
\par 14 És a mikor tovaméne, meglátá Lévit, az Alfeus fiát, a ki a vámszedõ helyen ül vala, és monda néki: Kövess engemet. És felkelvén, követi vala õt.
\par 15 És lõn, a mikor õ ennek házában asztalhoz üle, a vámszedõk és bûnösök is sokan odaülnek vala Jézussal és az õ tanítványaival; mert sokan valának, és követék õt.
\par 16 És a mikor látták az írástudók és a farizeusok, hogy együtt eszik a vámszedõkkel és bûnösökkel, mondának az õ tanítványainak: Mi dolog, hogy a vámszedõkkel és a bûnösökkel eszik és iszik?
\par 17 És a mikor ezt hallja vala Jézus, monda nékik: Nem az egészségeseknek van szükségök orvosra, hanem a betegeknek, nem azért jöttem, hogy igazakat, hanem hogy bûnösöket hívjak megtérésre.
\par 18 A János és a farizeusok tanítványai pedig bõjtölnek vala. Odamenének azért és mondának néki: Mi az oka, hogy Jánosnak és a farizeusoknak tanítványai bõjtölnek, a te tanítványaid pedig nem bõjtölnek?
\par 19 Jézus pedig monda nékik: Avagy bõjtölhet-é a võlegény násznépe, a míg velök van a võlegény? A meddig a võlegény velök van, nem bõjtölhetnek.
\par 20 De jõnek majd napok, a mikor elvétetik tõlük a võlegény, és akkor bõjtölni fognak azokon a napokon.
\par 21 Senki sem varr pedig új posztóból foltot ó ruhára; máskülönben a mi azt kitoldaná, még kiszakít belõle, az új a régibõl, és nagyobb szakadás lesz.
\par 22 És senki sem tölt új bort régi tömlõkbe; különben az új bor a tömlõket szétszakítja, a bor is kiömlik, a tömlõk is elpusztulnak; hanem az új bort új tömlõkbe kell tölteni.
\par 23 És lõn, hogy szombatnapon a vetések közt megy vala által, és az õ tanítványai mentökben a kalászokat kezdék vala szaggatni.
\par 24 Ekkor a farizeusok mondának néki: Ímé, miért mûvelik azt szombatnapon, a mit nem szabad?
\par 25 Õ pedig monda nékik: Soha sem olvastátok-é, mit mívelt Dávid, mikor megszûkült és megéhezett vala társaival egybe?
\par 26 Mi módon ment be az Isten házába az Abiátár fõpap idejében és ette meg a szent kenyereket, a melyeket nem  szabad megenni csak a papoknak; és adott a társainak is?
\par 27 És monda nékik: A szombat lõn az emberért, nem az ember a szombatért.
\par 28 Annak okáért az embernek Fia a szombatnak is ura.

\chapter{3}

\par 1 És ismét beméne a zsinagógába, és vala ott egy megszáradt kezû ember.
\par 2 És lesik vala õt, hogy meggyógyítja-é szombatnapon; hogy vádolhassák õt.
\par 3 Akkor monda a megszáradt kezû embernek: Állj elõ a középre.
\par 4 Azoknak pedig monda: Szabad-é szombatnapon jót vagy rosszat tenni? lelket menteni, vagy kioltani? De azok hallgatnak vala.
\par 5 Õ pedig elnézvén õket haraggal, bánkódván szívök keménysége miatt, monda az embernek: Nyújtsd ki a kezedet. És kinyújtá, és meggyógyult a keze és éppé lõn, mint a másik.
\par 6 Akkor a farizeusok kimenvén, a Heródes pártiakkal mindjárt tanácsot tartának ellene, hogy elveszítsék õt.
\par 7 Jézus pedig elméne tanítványaival a tenger mellé; és nagy sokaság követé õt Galileából és Júdeából,
\par 8 És Jeruzsálembõl és Idumeából és a Jordánon túlról; és a Tirus és a Sidon környékiek is, a mikor hallották, hogy miket mível vala, nagy sokasággal jövének õ hozzá.
\par 9 És megmondá tanítványainak, hogy egy kis hajót tartsanak néki készen, a sokaság miatt, hogy ne szorongassák õt.
\par 10 Mert sokakat meggyógyított, úgy hogy a kiknek valami bajuk volt, reá rohanának, hogy illethessék õt.
\par 11 A tisztátalan lelkek is, mikor meglátták vala õt, leborulának elõtte, és kiáltának, mondván: Te vagy az Istennek a Fia.
\par 12 Õ pedig erõsen fenyegeti vala õket, hogy õt ki ne jelentsék.
\par 13 Azután felméne a hegyre, és magához szólítá, a kiket akar vala; és hozzá menének.
\par 14 És választa tizenkettõt, hogy vele legyenek, és hogy kiküldje õket prédikálni,
\par 15 És hatalmuk legyen a betegeket gyógyítani és az ördögöket kiûzni:
\par 16 Simont, a kinek Péter nevet ada;
\par 17 És Jakabot a Zebedeus fiát és Jánost a Jakab testvérét; és Boanerges nevet ada nékik, a mely azt teszi: mennydörgés fiai;
\par 18 És Andrást és Filepet, Bertalant és Mátét, Tamást és Jakabot az Alfeus fiát, Taddeust és a kananeai Simont,
\par 19 És Iskáriótes Júdást, a ki el is árulta õt.
\par 20 Azután haza térének. És ismét egybegyûle a sokaság, annyira, hogy még nem is ehetének.
\par 21 A mint az övéi ezt meghallák, eljövének, hogy megfogják õt; mert azt mondják vala, hogy magán kívül van.
\par 22 Az írástudók pedig, a kik Jeruzsálembõl jöttek vala le, azt mondák, hogy: Belzebúb van vele, és: Az ördögök fejedelme által ûzi ki az ördögöket.
\par 23 Õ pedig magához híván azokat, példázatokban monda nékik: Sátán miként tud Sátánt kiûzni?
\par 24 És ha egy ország önmagában meghasonlik, meg nem maradhat az az ország.
\par 25 És ha egy ház önmagában meghasonlik, meg nem maradhat az a ház.
\par 26 És ha a Sátán önmaga ellen támadt és maghasonlott, nem maradhat meg, hanem vége van.
\par 27 Nem rabolhatja el senki az erõsnek kincseit, bemenvén annak házába, hanemha elébb az erõset megkötözi és azután rabolja ki annak házát.
\par 28 Bizony mondom néktek, hogy minden bûn megbocsáttatik az emberek fiainak, még a káromlások is mind, a melyekkel káromlanak:
\par 29 De a ki a Szent Lélek ellen szól káromlást, nem nyer bocsánatot soha, hanem örök kárhozatra méltó;
\par 30 Mivelhogy ezt mondják vala: Tisztátalan lélek van benne.
\par 31 És megérkezének az õ testvérei és az õ anyja, és kívül megállva, beküldének hozzá, hivatván õt.
\par 32 Körülötte pedig sokaság ül vala; és mondának néki: Ímé a te anyád és a te testvéreid ott künn keresnek téged.
\par 33 Õ pedig felele nékik, mondván: Ki az én anyám vagy kik az én testvéreim?
\par 34 Azután elnézvén köröskörül a körülötte ülõkön, monda: Ímé az én anyám és az én testvéreim.
\par 35 Mert a ki az Isten akaratát cselekszi, az az én fitestvérem és nõtestvérem és az én anyám.

\chapter{4}

\par 1 És ismét kezde tanítani a tenger mellett. És nagy sokaság gyûle õ hozzá, úgy hogy õ a hajóba lépvén, a tengeren ül vala, az egész sokaság pedig a tenger mellett a földön vala.
\par 2 És sokat tanítja vala õket példázatokban, és ezt mondja vala nékik tanításában:
\par 3 Halljátok: Ímé, a magvetõ kiméne vetni.
\par 4 És lõn vetés közben, hogy némely az út mellé esék, és eljövének az égi madarak és megevék azt.
\par 5 Némely pedig a köves helyre esék, a hol nem sok földje vala, és hamar kikele, mivel nem vala mélyen a földben.
\par 6 Mikor pedig fölkelt a nap, elsûle, és mivelhogy nem volt gyökere, elszárada.
\par 7 Némely pedig a tövisek közé esék, és felnevekedének a tövisek és megfojták azt, és nem ada gyümölcsöt.
\par 8 Némely pedig a jó földbe esék; és ád vala nevekedõ és bõvölködõ gyümölcsöt, és némely hoz vala harmincz annyit, némely hatvan annyit, némely pedig száz annyit.
\par 9 És monda nékik: A kinek van füle a hallásra, hallja.
\par 10 Mikor pedig egyedül vala, megkérdezék õt a körülötte lévõk a tizenkettõvel együtt a példázat felõl.
\par 11 Õ pedig monda nékik: Néktek adatott, hogy az Isten országának titkát tudjátok, ama kívül levõknek pedig példázatokban adatnak mindenek,
\par 12 Hogy nézvén nézzenek és ne lássanak; és hallván halljanak és ne értsenek, hogy soha meg ne térjenek és bûneik meg ne bocsáttassanak.
\par 13 És monda nékik: Nem értitek ezt a példázatot? Akkor mimódon értitek meg majd a többi példázatot?
\par 14 A magvetõ az ígét hinti.
\par 15 Az útfélen valók pedig azok, a kiknek hintik az ígét, de mihelyest hallják, azonnal eljõ a Sátán és elragadja a szívökbe vetett ígét.
\par 16 És hasonlóképen a köves helyre vetettek azok, a kik mihelyst hallják az ígét, mindjárt örömmel fogadják,
\par 17 De nincsen õ bennük gyökere, hanem ideig valók; azután ha nyomorúság vagy háborúság támad az íge miatt, azonnal megbotránkoznak.
\par 18 A tövisek közé vetettek pedig azok, a kik az ígét hallják,
\par 19 De a világi gondok és a gazdagság csalárdsága és egyéb dolgok kívánsága közbejõvén, elfojtják az ígét, és gyümölcstelen lesz.
\par 20 A jó földbe vetettek pedig azok, a kik hallják az ígét és beveszik, és gyümölcsöt teremnek, némely harmincz annyit, némely hatvan annyit, némely száz annyit.
\par 21 És monda nékik: Avagy azért hozzák-é elõ a gyertyát, hogy véka alá tegyék, vagy az ágy alá? És nem azért-é, hogy a gyertyatartóba tegyék?
\par 22 Mert nincs semmi rejtett dolog, a mi meg ne jelentetnék; és semmi sem volt eltitkolva, hanem hogy nyilvánosságra jusson.
\par 23 Ha valakinek van füle a hallásra, hallja.
\par 24 És monda nékik: Megjegyezzétek, a mit hallotok: A milyen mértékkel mértek, olyannal mérnek néktek, sõt ráadást adnak néktek, a kik halljátok.
\par 25 Mert a kinek van, annak adatik; és a kinek nincs, attól az is elvétetik, a mije van.
\par 26 És monda: Úgy van az Isten országa, mint mikor az ember beveti a magot a földbe.
\par 27 És alszik és fölkel éjjel és nappal; a mag pedig kihajt és felnõ, õ maga sem tudja miképen.
\par 28 Mert magától terem a föld, elõször füvet, azután kalászt, azután teljes buzát a kalászban.
\par 29 Mihelyt pedig a gabona arra való, azonnal sarlót ereszt reá, mert az aratás elérkezett.
\par 30 És monda: Mihez hasonlítsuk az Isten országát? Avagy milyen példában példázzuk azt?
\par 31 A mustármaghoz, a mely mikor a földbe vettetik, minden földi magnál kisebb,
\par 32 És mikor elvettetik, felnõ, és minden veteménynél nagyobb lesz és nagy ágakat hajt, úgy hogy árnyéka alatt fészket rakhatnak az égi madarak.
\par 33 És sok ilyen példázatban hirdeti vala nékik az ígét, úgy a mint megérthetik vala.
\par 34 Példázat nélkül pedig nem szól vala nékik; maguk közt azonban a tanítványnak mindent megmagyaráz vala.
\par 35 Azután monda nékik azon a napon, a mint este lõn: Menjünk át a túlsó partra.
\par 36 Elbocsátván azért a sokaságot, elvivék õt, úgy a mint a hajóban vala; de más hajók is valának vele.
\par 37 Akkor nagy szélvihar támada, a hullámok pedig becsapnak vala a hajóba, annyira, hogy már-már megtelék.
\par 38 Õ pedig a hajó hátulsó részében a fejaljon aluszik vala. És fölkelték õt és mondának néki: Mester, nem törõdöl vele, hogy elveszünk?
\par 39 És felkelvén megdorgálá a szelet, és monda a tengernek: Hallgass, némulj el! És elállt a szél, és lõn nagy csendesség.
\par 40 És monda nékik: Miért vagytok ily félénkek? Hogy van, hogy nincs hitetek?
\par 41 És megfélemlének nagy félelemmel, és ezt mondják vala egymásnak: Kicsoda hát ez, hogy mind a szél, mind a tenger engednek néki?

\chapter{5}

\par 1 És menének a tenger túlsó partjára, a Gadarenusok földére.
\par 2 És a mint a hajóból kiméne, azonnal elébe méne egy ember a sírboltokból, a kiben tisztátalan lélek volt,
\par 3 A kinek lakása a sírboltokban vala; és már lánczokkal sem bírta õt senki sem lekötni.
\par 4 Mert sokszor megkötözték õt békókkal és lánczokkal, de õ a lánczokat szétszaggatta, és a békókat összetörte, és senki sem tudta õt megfékezni.
\par 5 És éjjel és nappal mindig a hegyeken és a sírboltokban volt, kiáltozva és magát kövekkel vagdosva.
\par 6 Mikor pedig Jézust távolról meglátta, oda futamodék, és elébe borula,
\par 7 És fennhangon kiáltva monda: Mi közöm nékem te veled, Jézus, a magasságos Istennek Fia? Az Istenre kényszerítelek, ne kínozz engem.
\par 8 (Mert ezt mondja vala néki: Eredj ki, tisztátalan lélek, ez emberbõl.)
\par 9 És kérdezé tõle: Mi a neved? És felele, mondván: Légió a nevem, mert sokan vagyunk.
\par 10 És igen kéré õt, hogy ne küldje el õket arról a vidékrõl.
\par 11 Vala pedig ott a hegynél egy nagy disznónyáj, a mely legel vala.
\par 12 És az ördögök kérik vala õt mindnyájan, mondván: Küldj minket a disznókba, hogy azokba menjünk be.
\par 13 És Jézus azonnal megengedé nékik. A tisztátalan lelkek pedig kijövén, bemenének a disznókba; és a nyáj a meredekrõl a tengerbe rohana. Valának pedig mintegy kétezeren; és belefúlának a tengerbe.
\par 14 A kik pedig õrzik vala a disznókat, elfutának, és hírt vivének a városba és a falvakba. És kimenének, hogy lássák, mi az, a mi történt.
\par 15 És menének Jézushoz, és láták, hogy az ördöngõs ott ül, fel van öltözködve és eszénél van, az, a kiben a légió volt; és megfélemlének.
\par 16 A kik pedig látták, elbeszélék nékik, hogy mi történt vala az ördöngõssel, és a disznókkal.
\par 17 És kezdék kérni õt, hogy távozzék el az õ határukból.
\par 18 Mikor pedig a hajóba beszállott vala, a volt ördöngõs kéré õt, hogy vele lehessen.
\par 19 De Jézus nem engedé meg néki, hanem monda néki: Eredj haza a tiéidhez, és jelentsd meg nékik, mely nagy dolgot cselekedett veled az Úr, és mint könyörült rajtad.
\par 20 El is méne, és kezdé hirdetni a Tízvárosban, mely nagy dolgot cselekedett vele Jézus; és mindnyájan elcsodálkozának.
\par 21 És mikor ismét általment Jézus a hajón a tulsó partra, nagy sokaság gyûle õ hozzá; és vala a tenger mellett.
\par 22 És ímé, eljöve a zsinagóga fõk egyike, névszerint Jairus, és meglátván õt, lábaihoz esék,
\par 23 És igen kéré õt, mondván: Az én leánykám halálán van; jer, vesd reá kezedet, hogy meggyógyuljon és éljen.
\par 24 El is méne vele, és követé õt nagy sokaság, és összeszorítják vala õt.
\par 25 És egy asszony, a ki tizenkét év óta vérfolyásos vala,
\par 26 És sok orvostól sokat szenvedett, és minden vagyonát magára költötte, és semmit sem javult, sõt inkább még rosszabbul lett,
\par 27 Mikor Jézus felõl hallott vala, a sokaságban hátulról kerülve, illeté annak ruháját.
\par 28 Mert ezt mondja vala: Ha csak ruháit illethetem is, meggyógyulok.
\par 29 És vérének forrása azonnal kiszárada és megérzé testében, hogy kigyógyult bajából.
\par 30 Jézus pedig azonnal észrevevén magán, hogy isteni erõ áradott ki belõle, megfordult a sokaságban, és monda: Kicsoda illeté az én ruháimat?
\par 31 És mondának néki az õ tanítványai: Látod, hogy a sokaság szorít össze téged, és azt kérdezed: Kicsoda illetett engem?
\par 32 És körülnéze, hogy lássa azt, a ki ezt cselekedte.
\par 33 Az asszony pedig tudva, hogy mi történt vele, félve és remegve megy vala oda és elébe borula, és elmonda néki mindent igazán.
\par 34 Õ pedig monda néki: Leányom, a te hited megtartott téged. Eredj el békével, és gyógyulj meg a te bajodból.
\par 35 Mikor még beszél vala, odajövének a zsinagóga fejétõl, mondván: Leányod meghalt; mit fárasztod tovább a Mestert?
\par 36 Jézus pedig, a mint hallá a beszédet, a mit mondanak vala, azonnal monda a zsinagóga fejének: Ne félj, csak higyj.
\par 37 És senkinek sem engedé, hogy vele menjen, csak Péternek és Jakabnak és Jánosnak, a Jakab testvérének.
\par 38 És méne a zsinagóga fejének házához, és látá a zûrzavart, a sok síránkozót és jajgatót.
\par 39 És bemenvén, monda nékik: Mit zavarogtok és sírtok? A gyermek nem halt meg, hanem alszik.
\par 40 És nevetik vala õt. Õ pedig kiküldvén valamennyit, maga mellé vevé a gyermeknek atyját és anyját és a vele levõket, és beméne oda, a hol a gyermek fekszik vala.
\par 41 És megfogván a gyermeknek kezét, monda néki: Talitha, kúmi; a mi megmagyarázva azt teszi: Leányka, néked mondom, kelj föl.
\par 42 És a leányka azonnal fölkele és jár vala. Mert tizenkét esztendõs vala. És nagy csodálkozással csodálkozának.
\par 43 Õ pedig erõsen megparancsolá nékik, hogy ezt senki meg ne tudja. És mondá, hogy adjanak annak enni.

\chapter{6}

\par 1 És kiméne onnét, és méne az õ hazájába, és követék õt az õ tanítványai.
\par 2 És a mint eljött vala a szombat, tanítani kezde a zsinagógában; és sokan, a kik õt hallák, elálmélkodának vala, mondván: Honnét vannak ennél ezek? És mely bölcsesség az, a mi néki adatott, hogy ily csodadolgok is történnek általa?
\par 3 Avagy nem ez-é az az ácsmester, Máriának a fia, Jakabnak, Józsénak, Júdásnak és Simonnak pedig testvére? És nincsenek-é itt közöttünk az õ nõtestvérei is? És megbotránkoznak vala õ benne.
\par 4 Jézus pedig monda nékik: Nincs próféta tisztesség nélkül csak a maga hazájában, és a rokonai között és a maga házában.
\par 5 Nem is tehet vala ott semmi csodát, csak nehány beteget gyógyíta meg, rájok vetvén kezeit.
\par 6 És csodálkozik vala azoknak hitetlenségén. Aztán köröskörül járja vala a falvakat, tanítván.
\par 7 Majd magához szólítá a tizenkettõt, és kezdé õket kiküldeni  kettõnként, és ada nékik hatalmat a tisztátalan lelkeken.
\par 8 És megparancsolá nékik, hogy az útra semmit ne vigyenek egy pálczán kívül; se táskát, se kenyeret, se pénzt az övükben;
\par 9 Hanem kössenek sarut, de két ruhát ne öltsenek.
\par 10 És monda nékik: A hol valamely házba bementek, ott maradjatok mindaddig, a míg tovább mentek onnét.
\par 11 A kik pedig nem fogadnak titeket, sem nem hallgatnak rátok, onnét kimenvén, verjétek le a port lábaitokról, bizonyságul õ ellenök. Bizony mondom néktek: Sodomának vagy Gomorának tûrhetõbb lesz a dolga az ítélet napján, mint annak a városnak.
\par 12 Kimenvén azért, prédikálják vala, hogy térjenek meg.
\par 13 És sok ördögöt ûznek vala ki, és olajjal sok beteget megkennek és meggyógyítnak vala.
\par 14 És meghallá ezeket Heródes király (mert nyilvánvalóvá lõn az õ neve) és monda: Keresztelõ János támadt fel a halálból és azért mûködnek benne ez erõk.
\par 15 Némelyek azt mondják vala, hogy Illés õ; mások meg azt mondják vala, hogy Próféta, vagy olyan, mint egy a próféták közül.
\par 16 Heródes pedig ezeket hallván, monda: A kinek én fejét vétetém, az a János ez; õ támadt fel a halálból.
\par 17 Mert maga Heródes fogatta el és vettette vala börtönbe Jánost, Heródiás miatt, Fülöpnek, az õ testvérének felesége miatt, mivelhogy azt vette vala feleségül.
\par 18 Mert János azt mondá Heródesnek: Nem szabad néked a testvéred feleségével élned.
\par 19 Heródiás pedig ólálkodik vala utána, és meg akarja vala õt ölni; de nem teheté.
\par 20 Mert Heródes fél vala Jánostól, igaz és szent embernek ismervén õt, és oltalmazá õt; és ráhallgatván, sok dologban követi, és örömest hallgatja vala õt.
\par 21 De egy alkalmatos nap jöttével, mikor Heródes a maga születése ünnepén nagyjainak, vezéreinek és Galilea elõkelõ embereinek lakomát ad vala.
\par 22 És ennek a Heródiásnak a leánya beméne és tánczola, és megtetszék Heródesnek és a vendégeknek, monda a király a leánynak: Kérj tõlem, a mit akarsz, és megadom néked.
\par 23 És megesküvék néki, hogy: Bármit kérsz tõlem, megadom néked, még ha országom felét is.
\par 24 Az pedig kimenvén, monda az õ anyjának: Mit kérjek? Ez pedig mondja: A Keresztelõ János fejét.
\par 25 És a királyhoz nagy sietve azonnal bemenvén, kéré õt mondván: Akarom, hogy mindjárt add ide nékem a Keresztelõ János fejét egy tálban.
\par 26 A király pedig noha igen megszomorodék, eskûje és a vendégek miatt nem akará õt elutasítani.
\par 27 És azonnal hóhért küldvén a király, megparancsolá, hogy hozzák el annak fejét.
\par 28 Ez pedig elmenvén, fejét vevé annak a börtönben, és elõhozá a fejét egy tálban és adá a leánynak; a leány pedig az anyjának adá azt.
\par 29 A tanítványai pedig, a mikor ezt meghallották vala, eljövének, és elvivék a testét, és sírba tevék.
\par 30 És az apostolok összegyûlekezének Jézushoz, és elbeszélének néki mindent, azt is, a miket cselekedtek, azt is, a miket tanítottak vala.
\par 31 Õ pedig monda nékik: Jertek el csupán ti magatok valamely puszta helyre és pihenjetek meg egy kevéssé. Mert sokan valának a járó-kelõk, és még evésre sem volt alkalmas idejök.
\par 32 És elmenének hajón egy puszta helyre csupán õ magok.
\par 33 A sokaság pedig meglátá õket, a mint mennek vala, és sokan megismerék õt; és minden városból egybefutának oda gyalog, és megelõzék õket, és hozzá gyülekezének.
\par 34 És kimenvén Jézus nagy sokaságot láta, és megszáná õket, mert olyanok valának, mint a pásztor nélkül való juhok. És kezdé õket sokra tanítani.
\par 35 Mikor pedig immár nagy idõ vala, hozzámenvén az õ tanítványai mondának: Puszta ez a hely, és immár nagy idõ van:
\par 36 Bocsásd el õket, hogy elmenvén a körülfekvõ majorokba és falvakba, vegyenek magoknak kenyeret; mert nincs mit enniök.
\par 37 Õ pedig felelvén, monda nékik: Adjatok nékik ti enniök. És mondának néki: Elmenvén, vegyünk-é kétszáz pénz árú kenyeret, hogy enni adjunk nékik?
\par 38 Õ pedig monda nékik: Hány kenyeretek van? Menjetek és nézzétek meg. És megtudván, mondának: Öt, és két halunk.
\par 39 És parancsolá nékik, hogy ültessenek le mindenkit csoportonként a zöld pázsitra.
\par 40 Letelepedének azért szakaszonként, százával és ötvenével.
\par 41 Õ pedig vette vala az öt kenyeret és a két halat, és az égre tekintvén, hálákat ada; és megszegé a kenyereket és adá tanítványainak, hogy tegyék azok elé; és a két halat is elosztá mindnyájok között.
\par 42 Evének azért mindnyájan, és megelégedének;
\par 43 És maradékot is szedének fel tizenkét tele kosárral, és a halakból is.
\par 44 A kik pedig a kenyerekbõl ettek, mintegy ötezeren valának férfiak.
\par 45 És azonnal kényszeríté tanítványait, hogy hajóba szálljanak, és menjenek át elõre a túlsó partra Bethsaida felé, a míg õ a sokaságot elbocsátja.
\par 46 Minekutána pedig elbocsátotta õket, fölméne a hegyre imádkozni.
\par 47 És mikor beesteledék, a hajó a tenger közepén vala, õ pedig egymaga a szárazon.
\par 48 És látá õket, a mint veszõdnek az evezéssel; mert a szél szembe fú vala velök; és az éj negyedik szakában hozzájuk méne a tengeren járva; és el akar vala haladni mellettük.
\par 49 Azok pedig látván õt a tengeren járni, kisértetnek vélték, és felkiáltának;
\par 50 Mert mindnyájan látják vala õt és megrémülének. De õ azonnal megszólítá õket, és monda nékik: Bízzatok; én vagyok, ne féljetek.
\par 51 Ekkor beméne hozzájuk a hajóba, és elállt a szél; õk pedig magukban szerfölött álmélkodnak és csodálkoznak vala.
\par 52 Mert nem okultak a kenyereken, mivelhogy a szívök meg vala keményedve.
\par 53 És átkelve, eljutának a Genezáret földére, és kikötének.
\par 54 De mihelyt kiszálltak a hajóból, azonnal megismerék õt,
\par 55 És azt az egész környéket befutván, kezdék a betegeket a nyoszolyákon ide-oda hordozni, a merre hallják vala, hogy õ ott van.
\par 56 És a hová bemegy vala a falvakba vagy városokba vagy majorokba, a betegeket letevék a piaczokon, és kérik vala õt, hogy legalább a ruhája szegélyét illethessék. És valahányan csak illeték, meggyógyulának.

\chapter{7}

\par 1 És hozzá gyûlének a farizeusok és némelyek az írástudók közül, a kik Jeruzsálembõl jöttek vala.
\par 2 És látván, hogy az õ tanítványai közül némelyek közönséges, azaz mosdatlan kézzel esznek kenyeret, panaszkodának.
\par 3 Mert a farizeusok és a zsidók mind, a régiek rendelését követve, nem esznek, hanemha kezöket erõsen megmossák;
\par 4 És piaczról jövén sem esznek, ha meg nem mosakodnak; és sok egyéb is van, a minek megtartását átvették, poharaknak, korsóknak, rézedényeknek és nyoszolyáknak megmosását.
\par 5 Azután megkérdék õt a farizeusok és az írástudók: Mi az oka, hogy a te tanítványaid nem járnak a régiek rendelése szerint, hanem mosdatlan kézzel esznek kenyeret?
\par 6 Õ pedig felelvén, monda nékik: Igazán jövendölt felõletek, képmutatók felõl Ésaiás próféta, a mint meg van írva: Ez a nép ajkaival tisztel engem, a szívök pedig távol van tõlem.
\par 7 Pedig hiába tisztelnek engem, ha oly tudományokat tanítanak, a melyek embereknek parancsolatai.
\par 8 Mert az Isten parancsolatját elhagyva, az emberek rendelését tartjátok meg, korsóknak és poharaknak mosását; és sok egyéb efféléket is cselekesztek.
\par 9 És monda nékik: Az Isten parancsolatját szépen félre teszitek, azért, hogy a magatok rendelését tartsátok meg.
\par 10 Mert Mózes azt mondotta: Tiszteld atyádat és anyádat. És: A ki atyját vagy anyját szidalmazza, halállal haljon meg.
\par 11 Ti pedig azt mondjátok: Ha valaki ezt mondja atyjának vagy anyjának: Korbán (azaz: templomi ajándék) az, a mivel megsegíthetnélek:
\par 12 Úgy már nem engeditek, hogy az atyjával vagy anyjával valami jót tegyen,
\par 13 Eltörölvén az Isten beszédét a ti rendelésetekkel, a melyet rendeltetek; és sok effélét is cselekesztek.
\par 14 És elõszólítván az egész sokaságot, monda nékik: Hallgassatok reám mindnyájan és értsétek meg:
\par 15 Nincs semmi az emberen kívülvaló, a mi bemenvén õ belé, megfertõztethetné õt; hanem a mik belõle jõnek ki, azok fertõztetik meg az embert.
\par 16 Ha valakinek van füle a hallásra, hallja.
\par 17 És mikor házba ment vala be a sokaság közül, megkérdezék õt tanítványai a példázat felõl.
\par 18 És monda nékik: Ti is ennyire tudatlanok vagytok-é? Nem értitek-é, hogy a mi kívülrõl megy az emberbe, semmi sem fertõztetheti meg õt?
\par 19 Mert nem a szívébe megy be, hanem a gyomrába; és az árnyékszékbe kerül, a mely minden eledelt megtisztít.
\par 20 Monda továbbá: A mi az emberbõl jõ ki, az fertõzteti meg az embert.
\par 21 Mert onnan belõlrõl, az emberek szívébõl származnak a gonosz gondolatok, házasságtörések, paráznaságok, gyilkosságok,
\par 22 Lopások, telhetetlenségek, gonoszságok, álnokság, szemérmetlenség, gonosz szem, káromlás, kevélység, bolondság:
\par 23 Mind ezek a gonoszságok belõlrõl jõnek ki, és megfertõztetik az embert.
\par 24 És onnét fölkelvén, elméne Tírus és Sídon határaiba; és házba menvén, nem akará, hogy valaki észrevegye, de nem titkolhatá el magát.
\par 25 Mert hallván felõle egy asszony, a kinek leányában tisztátalan lélek vala, eljõve és lábaihoz borula.
\par 26 Ez az asszony pedig pogány vala síro-fenicziai származású. És kéré õt, hogy ûzze ki az õ leányából az ördögöt.
\par 27 Jézus pedig monda néki: Engedd, hogy elõször a fiak elégíttessenek meg. Mert nem jó a fiak kenyerét elvenni, és az ebeknek vetni.
\par 28 Az pedig felele és monda néki: Úgy van Uram; de hiszen az ebek is esznek az asztal alatt a gyermekek morzsalékaiból.
\par 29 Erre monda néki: E beszédért, eredj el; az ördög kiment a te leányodból.
\par 30 És haza menvén, úgy találá, hogy az ördög kiment, a leány pedig az ágyon feküvék.
\par 31 Aztán ismét kimenvén Tírus és Sídon határaiból, a galileai tengerhez méne, a Tízváros határain át.
\par 32 És hozának néki egy nehezen szóló siketet, és kérik vala õt, hogy vesse reá kezét.
\par 33 Õ pedig, mikor kivitte vala azt a sokaság közül egy magát, az újjait annak fülébe bocsátá, és köpvén illeté annak nyelvét,
\par 34 És föltekintvén az égre, fohászkodék, és monda néki: Effata, azaz: nyilatkozzál meg.
\par 35 És azonnal megnyilatkozának annak fülei: és nyelvének kötele megoldódék, és helyesen beszél vala.
\par 36 És megparancsolá nékik, hogy senkinek se mondják el; de mennél inkább tiltja vala, annál inkább híresztelék.
\par 37 És szerfelett álmélkodnak vala, ezt mondván: Mindent jól cselekedett; a siketeket is hallókká teszi, a némákat is beszélõkké.

\chapter{8}

\par 1 Azokban a napokban, mivelhogy fölötte nagy volt a sokaság, és nem volt mit enniök, magához szólította Jézus az õ tanítványait, és monda nékik:
\par 2 Szánakozom e sokaságon, mert immár harmad napja hogy velem vannak, és nincs mit enniök;
\par 3 És ha éhen bocsátom haza õket, kidõlnek az úton; mert némelyek õ közülök messzünnen jöttek.
\par 4 Az õ tanítványai pedig felelének néki: Honnan elégíthetné meg ezeket valaki kenyérrel itt e pusztában?
\par 5 És megkérdé õket: Hány kenyeretek van? Azok pedig mondának: Hét.
\par 6 Akkor megparancsolá a sokaságnak, hogy telepedjenek le a földre. És vevén a hét kenyeret, és hálákat adván, megszegé, és adá az õ tanítványainak, hogy eléjök tegyék. És a sokaság elé tevék.
\par 7 Volt egy kevés haluk is. És hálákat adván mondá, hogy tegyék eléjök azokat is.
\par 8 Evének azért, és megelégedének; és fölszedék a maradék darabokat, hét kosárral.
\par 9 Valának pedig a kik ettek mintegy négyezeren; és elbocsátá õket.
\par 10 És azonnal a hajóba szálla tanítványaival, és méne Dalmánuta vidékére.
\par 11 És kijövének a farizeusok, és kezdék õt faggatni, mennyei jelt kívánván tõle, hogy kísértsék õt.
\par 12 Õ pedig lelkében felfohászkodván, monda: Miért kíván jelt ez a nemzetség? Bizony mondom néktek: Nem adatik jel ennek a nemzetségnek.
\par 13 És ott hagyván õket, ismét hajóba szálla, és a túlsó partra méne.
\par 14 De elfelejtének kenyeret vinni, és egy kenyérnél nem vala velök több a hajóban.
\par 15 És õ inti vala õket, mondván: Vigyázzatok, õrizkedjetek a farizeusok kovászától és a  Heródes kovászától!
\par 16 Ekkor egymás között tanakodván, mondának: Nincs kenyerünk.
\par 17 Jézus pedig észrevévén ezt, monda nékik: Mit tanakodtok, hogy nincsen kenyeretek? Még sem látjátok-é be és nem értitek-é? Mégis kemény-é a szívetek?
\par 18 Szemeitek lévén, nem láttok-é? és füleitek lévén, nem hallotok-é? és nem emlékeztek-é?
\par 19 Mikor az öt kenyeret megszegtem az ötezernek, hány kosarat hoztatok el darabokkal tele? Mondának néki: Tizenkettõt.
\par 20 Mikor pedig a hetet a négyezernek, hány kosarat hoztatok el darabokkal tele? Azok pedig mondának: Hetet.
\par 21 És monda nékik: Hogy nem értitek hát?
\par 22 Azután Bethsaidába méne; és egy vakot vivének hozzá és kérik vala õt, hogy illesse azt.
\par 23 Õ pedig megfogván a vaknak kezét, kivezeté õt a falun kívül; és a szemeibe köpvén és kezeit reá tévén, megkérdé õt, ha lát-é valamit?
\par 24 Az pedig föltekintvén, monda: Látom az embereket, mint valami járkáló fákat.
\par 25 Azután kezeit ismét ráveté annak szemeire, és feltekintete véle. És megépüle, és látá messze és világosan mindent.
\par 26 És haza küldé, mondván: Se a faluba be ne menj, se senkinek el ne mondd a faluban.
\par 27 És elméne Jézus és az õ tanítványai Czézárea Filippi falvaiba; és útközben megkérdé az õ tanítványait, mondván nékik: Kinek mondanak engem az emberek?
\par 28 Õk pedig felelének: Keresztelõ Jánosnak; és némelyek Illésnek; némelyek pedig egynek a próféták közül.
\par 29 És õ monda nékik: Ti pedig kinek mondotok engem? Felelvén pedig Péter, monda néki: Te vagy a Krisztus.
\par 30 És rájok parancsola, hogy senkinek se szóljanak felõle.
\par 31 És kezdé õket tanítani, hogy az ember Fiának sokat kell szenvedni, és megvettetni a vénektõl és a fõpapoktól és írástudóktól, és megöletni, és harmadnapra feltámadni.
\par 32 És ezt nyilván mondja vala. Péter pedig magához vonván õt, kezdé dorgálni.
\par 33 És õ megfordulván és az õ tanítványaira tekintvén, megfeddé Pétert, mondván: Távozz tõlem Sátán, mert nem gondolsz az Isten dolgaira, hanem az emberi dolgokra.
\par 34 A sokaságot pedig az õ tanítványaival együtt magához szólítván, monda nékik: Ha valaki én utánam akar jõni, tagadja meg magát, és vegye fel az õ keresztjét, és kövessen engem.
\par 35 Mert valaki meg akarja tartani az õ életét, elveszti azt; valaki pedig elveszti az õ életét én érettem és az evangyéliomért, az megtalálja azt.
\par 36 Mert mit használ az embernek, ha az egész világot megnyeri, lelkében pedig kárt vall?
\par 37 Avagy mit adhat az ember váltságul az õ lelkéért?
\par 38 Mert valaki szégyel engem és az én beszédeimet e parázna és bûnös nemzetség között, az embernek Fia is szégyelni fogja azt, mikor eljõ az õ Atyja dicsõségében a szent angyalokkal.

\chapter{9}

\par 1 Azután monda nékik: Bizony mondom néktek, hogy vannak némelyek az itt állók között, a kik nem kóstolnak addig halált, a míg meg nem látják, hogy az Isten országa eljött hatalommal.
\par 2 És hat nap múlva magához vevé Jézus Pétert és Jakabot és Jánost, és felvivé õket csupán magukban egy magas hegyre. És elváltozék elõttük;
\par 3 És a ruhája fényes lõn, igen fehér, mint a hó, mihez hasonlót a ruhafestõ e földön nem fehéríthet.
\par 4 És megjelenék nékik Mózes Illéssel együtt, és beszélnek vala Jézussal.
\par 5 Péter pedig megszólalván, monda Jézusnak: Mester, jó nékünk itt lenni: csináljunk azért három hajlékot, néked egyet, Mózesnek is egyet, Illésnek is egyet.
\par 6 De nem tudja vala mit beszél, mivelhogy megrémülének.
\par 7 És felhõ támada, mely õket befogá, és a felhõbõl szózat jöve, mondván: Ez az én szerelmes Fiam; õt hallgassátok.
\par 8 És mikor nagyhirtelen körültekintének, senkit sem látának többé maguk körül, egyedül a Jézust.
\par 9 Mikor pedig a hegyrõl leszállának, megparancsolá nékik, hogy senkinek se beszéljék el, a mit láttak vala, csak a mikor az embernek Fia a halálból feltámad.
\par 10 És ezt a szót megtarták magukban, tudakozván egymás között, mit tesz a halálból feltámadni?
\par 11 És megkérdezék õt, mondván: Miért mondják az írástudók, hogy elõbb Illésnek kell eljõnie?
\par 12 Õ pedig felelvén, monda nékik: Illés ugyan elõbb eljövén helyre állít mindent; de hogyan van az embernek Fiáról megírva, hogy sokat kell szenvednie és megvettetnie?
\par 13 De mondom néktek, hogy Illés is eljött, és azt cselekedték vele, a mit akartak, a mint meg van  írva õ felõle.
\par 14 És mikor a tanítványokhoz ment vala, nagy sokaságot láta körülöttök, és írástudókat, a kik azokkal versengenek vala.
\par 15 És az egész sokaság meglátván õt, azonnal elálmélkodék, és hozzásietvén köszönté õt.
\par 16 Õ pedig megkérdezé az írástudókat: Mit versengetek ezekkel?
\par 17 És felelvén egy a sokaságból, monda: Mester, ide hoztam hozzád az én fiamat, a kiben néma lélek van.
\par 18 És a hol csak elõfogja, szaggatja õt; õ pedig tajtékot túr, a fogát csikorgatja, és elfonnyad. Mondám hát tanítványaidnak, hogy ûzzék ki azt, de nem tudták.
\par 19 Õ pedig felelvén néki, monda: Óh hitetlen nemzetség, meddig leszek még veletek? Meddig szenvedlek még titeket? Hozzátok õt hozzám.
\par 20 És hozzá vivék azt; és mihelyt õ meglátta azt, a lélek azonnal szaggatá azt; és leesvén a földre, tajtékot túrván fetreng vala.
\par 21 És megkérdezé az atyját: Mennyi ideje, hogy ez esett rajta? Az pedig monda: Gyermeksége óta.
\par 22 És gyakorta veté õt tûzbe is, vízbe is, hogy elveszítse õt; de ha valamit tehetsz, légy segítségül nékünk, könyörülvén rajtunk.
\par 23 Jézus pedig monda néki: Ha hiheted azt, minden lehetséges a hívõnek.
\par 24 A gyermek atyja pedig azonnal kiáltván, könnyhullatással monda: Hiszek Uram! Légy segítségül az én hitetlenségemnek.
\par 25 Jézus pedig mikor látta vala, hogy a sokaság még inkább összetódul, megdorgálá a tisztátalan lelket, mondván néki: Te néma és siket lélek, én parancsolom néked, menj ki belõle, és többé belé ne menj!
\par 26 És kiáltás és erõs szaggatás között kiméne; az pedig olyan lõn, mint egy halott, annyira, hogy sokan azt mondják vala, hogy meghalt.
\par 27 Jézus pedig megfogván kezét, fölemelé; és az fölkele.
\par 28 Mikor pedig bement vala a házba, tanítványai megkérdezék õt külön: Mi miért nem ûzhettük ki azt?
\par 29 Õ pedig monda nékik: Ez a faj semmivel sem ûzhetõ ki, csupán könyörgéssel és bõjtöléssel.
\par 30 És onnét kimenvén, Galileán mennek vala át; és nem akará, hogy valaki megtudja.
\par 31 Mert tanítja vala tanítványait, és ezt mondja vala nékik: Az embernek Fia az emberek kezébe adatik, és megölik õt; de ha megölték, harmadnapra föltámad.
\par 32 De õk nem értik vala e mondást, és féltek õt megkérdezni.
\par 33 És elméne Kapernaumba. És odahaza megkérdezé õket: Mi felett vetekedtetek egymással az úton?
\par 34 De õk hallgatának, mert egymás között a felett vetekedtek vala az úton, ki a nagyobb?
\par 35 És leülvén, odaszólítá a tizenkettõt, és monda nékik: Ha valaki elsõ akar lenni, legyen mindenek között utolsó és mindeneknek szolgája.
\par 36 És elõfogván egy gyermeket, közéjök állatá azt; és ölébe vévén azt, monda nékik:
\par 37 A ki az ilyen gyermekek közül egyet befogad az én nevemben, engem fogad be; és a ki engem befogad, nem engem fogad be, hanem azt, a ki engem elbocsátott.
\par 38 János pedig felele néki, mondván: Mester, látánk valakit, a ki a te neveddel ördögöket ûz, a ki nem követ minket; és eltiltók õt, mivelhogy nem követ minket.
\par 39 Jézus pedig monda: Ne tiltsátok el õt; mert senki sincs, a ki csodát tesz az én nevemben és mindjárt gonoszul szólhatna felõlem.
\par 40 Mert a ki nincs ellenünk, mellettünk van.
\par 41 Mert a ki innotok ád egy pohár vizet az én nevemben, mivelhogy a Krisztuséi vagytok, bizony mondom néktek, el nem veszti az õ jutalmát.
\par 42 A ki pedig megbotránkoztat egyet ama kicsinyek közül, a kik én bennem hisznek, jobb annak, ha malomkövet kötnek a nyakára, és a tengerbe vetik.
\par 43 És ha megbotránkoztat téged a te kezed, vágd le azt: jobb néked csonkán bemenned az életre, mint két kézzel menned a gyehennára, a megolthatatlan tûzre.
\par 44 A hol az õ férgök meg nem hal, és tüzök el nem aluszik.
\par 45 És ha a te lábad botránkoztat meg téged, vágd le azt: jobb néked sántán bemenned az életre, mint két lábbal vettetned a gyehennára, a megolthatatlan tûzre.
\par 46 A hol az õ férgök meg nem hal, és tüzök el nem aluszik.
\par 47 És ha a te szemed botránkoztat meg téged, vájd ki azt: jobb néked félszemmel bemenned az Isten országába, mint két szemmel vettetned a tüzes gyehennára.
\par 48 A hol az õ férgök meg nem hal, és tüzök el nem aluszik.
\par 49 Mert mindenki tûzzel sózatik meg, és minden áldozat sóval sózatik meg.
\par 50 Jó a só: de ha a só ízét veszti, mivel adtok ízt néki? Legyen bennetek só, és legyetek békében egymással.

\chapter{10}

\par 1 Onnan pedig felkelvén Judea határaiba méne, a Jordánon túl való részen által; és ismét sokaság gyûl vala hozzá; õ pedig szokása szerint ismét tanítja vala õket.
\par 2 És a farizeusok hozzámenvén megkérdezék tõle, ha szabad-é férjnek feleségét elbocsátani, kísértvén õt.
\par 3 Õ pedig felelvén, monda nékik: Mit parancsolt néktek Mózes?
\par 4 Õk pedig mondának: Mózes megengedte, hogy válólevelet írjunk, és elváljunk.
\par 5 És Jézus felelvén, monda nékik: A ti szívetek keménysége miatt írta néktek ezt a parancsolatot;
\par 6 De a teremtés kezdete óta férfiúvá és asszonnyá teremté õket az Isten.
\par 7 Annakokáért elhagyja az ember az õ atyját és anyját; és ragaszkodik a feleségéhez,
\par 8 És lesznek ketten egy testté! Azért többé nem két, hanem egy test.
\par 9 Annakokáért a mit az Isten egybe szerkesztett, ember el ne válaszsza.
\par 10 És odahaza az õ tanítványai ismét megkérdezék õt e dolog felõl.
\par 11 Õ pedig monda nékik: A ki elbocsátja feleségét és mást vesz el, házasságtörést követ el az ellen.
\par 12 Ha pedig a feleség hagyja el a férjét és mással kel egybe, házasságtörést követ el.
\par 13 Ekkor gyermekeket hozának hozzá, hogy illesse meg õket; a tanítványok pedig feddik vala azokat, a kik hozák.
\par 14 Jézus pedig ezt látván, haragra gerjede és monda nékik: Engedjétek hozzám jõni a gyermekeket és ne tiltsátok el õket; mert ilyeneké az Istennek országa.
\par 15 Bizony mondom néktek: A ki nem úgy fogadja az Isten országát, mint gyermek, semmiképen sem megy be abba.
\par 16 Aztán ölébe vevé azokat, és kezét rájok vetvén, megáldá õket.
\par 17 És mikor útnak indult vala, hozzá futván egy ember és letérdelvén elõtte, kérdezi vala õt: Jó Mester, mit cselekedjem, hogy az örökéletet elnyerhessem?
\par 18 Jézus pedig monda néki: Miért mondasz engem jónak? Senki sem jó, csak egy, az Isten.
\par 19 A parancsolatokat tudod: Ne paráználkodjál; ne ölj; ne lopj; hamis tanubizonyságot ne tégy, kárt ne tégy; tiszteljed atyádat és anyádat.
\par 20 Az pedig felelvén, monda néki: Mester, mindezeket megtartottam ifjúságomtól fogva.
\par 21 Jézus pedig rátekintvén, megkedvelé õt, és monda néki: Egy fogyatkozásod van; eredj el, add el minden vagyonodat, és add a szegényeknek, és kincsed lesz mennyben; és jer, kövess engem, felvévén a keresztet.
\par 22 Az pedig elszomorodván e beszéden, elméne búsan; mert sok jószága vala.
\par 23 Jézus pedig körültekintvén, monda tanítványainak: Mily nehezen mennek be az Isten országába, a kiknek gazdagságuk van!
\par 24 A tanítványok pedig álmélkodának az õ beszédén; de Jézus ismét felelvén, monda nékik: Gyermekeim, mily nehéz azoknak, a kik a gazdagságban bíznak, az Isten országába bemenni!
\par 25 Könnyebb a tevének a tû fokán átmenni, hogynem a gazdagnak az Isten országába bejutni.
\par 26 Azok pedig még inkább álmélkodnak vala, mondván magok között: Kicsoda idvezülhet tehát?
\par 27 Jézus pedig rájuk tekintvén, monda: Az embereknél lehetetlen, de nem az Istennél; mert az Istennél minden lehetséges.
\par 28 És Péter kezdé mondani néki: Ímé, mi elhagytunk mindent, és követtünk téged.
\par 29 Jézus pedig felelvén, monda: Bizony mondom néktek, senki sincs, a ki elhagyta házát, vagy fitestvéreit, vagy nõtestvéreit, vagy atyját, vagy anyját, vagy feleségét, vagy gyermekeit, vagy szántóföldeit én érettem és az evangyéliomért,
\par 30 A ki százannyit ne kapna most ebben az idõben, házakat, fitestvéreket, nõtestvéreket, anyákat, gyermekeket és szántóföldeket, üldözésekkel együtt; a jövendõ világon pedig örök életet.
\par 31 Sok elsõk pedig lesznek utolsók, és sok utolsók elsõk.
\par 32 Útban valának pedig Jeruzsálembe menve fel; és elõttök megy vala Jézus, õk pedig álmélkodának, és követvén õt, félnek vala. És õ a tizenkettõt ismét maga mellé vévén, kezde nékik szólni azokról a dolgokról, a mik majd vele történnek,
\par 33 Mondván: Ímé, felmegyünk Jeruzsálembe, és az embernek Fia átadatik a fõpapoknak és az írástudóknak, és halálra kárhoztatják õt, és a pogányok kezébe adják õt;
\par 34 És megcsúfolják õt, és megostorozzák õt, és megköpdösik õt, és megölik õt; de harmadnapon feltámad.
\par 35 És hozzájárulának Jakab és János, a Zebedeus fiai, ezt mondván: Mester, szeretnõk, hogy a mire kérünk, tedd meg nékünk.
\par 36 Õ pedig monda nékik: Mit kívántok, hogy tegyek veletek?
\par 37 Azok pedig mondának néki: Add meg nékünk, hogy egyikünk jobb kezed felõl, másikunk pedig bal kezed felõl üljön a te dicsõségedben.
\par 38 Jézus pedig monda nékik: Nem tudjátok, mit kértek. Megihatjátok-é a pohárt, a melyet én megiszom; és megkeresztelkedhettek-é  azzal a keresztséggel, a melylyel én megkeresztelkedem?
\par 39 Azok pedig mondának néki: Megtehetjük. Jézus pedig monda nékik: A pohárt ugyan, a melyet én megiszom, megiszszátok, és a keresztséggel, a melylyel én megkeresztelkedem, megkeresztelkedtek;
\par 40 De az én jobb és bal kezem felõl való ülést nem az én dolgom megadni, hanem azoké lesz az, a kiknek elkészíttetett.
\par 41 És hallván ezt a tíz tanítvány, haragudni kezdének Jakabra és Jánosra.
\par 42 Jézus pedig magához szólítván õket, monda nékik: Tudjátok, hogy azok, a kik a pogányok között fejedelmeknek tartanak, uralkodnak felettök, és az õ nagyjaik hatalmaskodnak rajtok.
\par 43 De nem így lesz közöttetek; hanem, a ki nagy akar lenni közöttetek, az legyen a ti szolgátok;
\par 44 És a ki közületek elsõ akar lenni, mindenkinek szolgája legyen:
\par 45 Mert az embernek Fia sem azért jött, hogy néki szolgáljanak, hanem hogy õ szolgáljon, és adja az õ életét váltságul sokakért.
\par 46 És Jerikóba érkezének: és mikor õ és az õ tanítványai és nagy sokaság Jerikóból kimennek vala, a Timeus fia, a vak Bartimeus, ott üle az úton, koldulván.
\par 47 És a mikor meghallá, hogy ez a Názáreti Jézus, kezde kiáltani, mondván: Jézus, Dávidnak Fia, könyörülj rajtam!
\par 48 És sokan feddik vala õt, hogy hallgasson; de õ annál jobban kiáltja vala: Dávidnak Fia, könyörülj rajtam!
\par 49 Akkor Jézus megállván, mondá, hogy hívják elõ. És elõhívják vala a vakot, mondván néki: Bízzál; kelj föl, hív tégedet.
\par 50 Az pedig felsõ ruháját ledobván, és felkelvén, Jézushoz méne.
\par 51 És felelvén Jézus, monda néki: Mit akarsz, hogy cselekedjem veled? A vak pedig monda néki: Mester, hogy lássak.
\par 52 Jézus pedig monda néki: Eredj el, a te hited megtartott téged. És azonnal megjött a szemevilága, és követi vala Jézust az úton.

\chapter{11}

\par 1 És mikor Jeruzsálemhez közeledének, Bethfagé és Bethánia felé, az olajfák hegyénél, elkülde kettõt tanítványai közül,
\par 2 És monda nékik: Eredjetek abba a faluba, a mely elõttetek van; és a mikor abba bejuttok, azonnal találtok egy megkötött vemhet, a melyen ember nem ült még soha; azt oldjátok el és hozzátok ide.
\par 3 És ha valaki azt mondja néktek: Miért teszitek ezt? mondjátok: Az Úrnak van szüksége reá. És azonnal elbocsátja azt ide.
\par 4 Elmenének azért és megtalálák a megkötött vemhet, az ajtónál kívül a kettõs útnál, és eloldák azt.
\par 5 Az ott állók közül pedig némelyek mondának nékik: Mit míveltek, hogy eloldjátok a vemhet?
\par 6 Õk pedig felelének nékik, úgy, a mint Jézus megparancsolta vala. És elbocsáták õket.
\par 7 És oda vivék a vemhet Jézushoz, és ráveték felsõ ruháikat; õ pedig felüle reá.
\par 8 Sokan pedig felsõ ruháikat az útra teríték, mások pedig ágakat szegdelnek vala a fákról és az útra hányják vala.
\par 9 A kik pedig elõtte menének, és a kik követék, kiáltának, mondván: Hozsánna! Áldott, a ki jõ az Úrnak nevében!
\par 10 Áldott a mi Atyánknak, Dávidnak országa, a mely jõ az Úrnak nevében! Hozsánna a magasságban!
\par 11 És beméne Jézus Jeruzsálembe, és a templomba; és mindent körülnézvén, mivelhogy az idõ már késõ vala, kiméne Bethániába a tizenkettõvel.
\par 12 És másnap, mikor Bethániából kimentek vala, megéhezék.
\par 13 És meglátván messzirõl egy fügefát, a mely leveles vala, odaméne, ha talán találna valamit rajta: de odaérvén ahhoz, levélnél egyebet semmit sem talála; mert nem vala fügeérésnek ideje.
\par 14 Akkor felelvén Jézus, monda a fügefának: Soha örökké ne egyék rólad gyümölcsöt senki. És hallák az õ tanítványai.
\par 15 És Jeruzsálembe érkezének. És Jézus bemenvén a templomba, kezdé kiûzni azokat, a kik a templomban árulnak és vásárolnak vala; a pénzváltók asztalait, és a galambárúsok székeit pedig felforgatá;
\par 16 És nem engedi vala, hogy valaki edényt vigyen a templomon keresztül.
\par 17 És tanít vala, mondván nékik: Nincsen-é megírva: Az én házam imádság házának neveztetik minden nép között? Ti pedig  rablók barlangjává tettétek azt.
\par 18 És meghallák az írástudók és a fõpapok, és tanakodnak vala, hogy mi módon veszíthetnék el õt. Mert félnek vala tõle, mivelhogy az egész sokaság álmélkodik vala az õ tanításán.
\par 19 És mikor beestveledék, kiméne a városból.
\par 20 Reggel pedig, a mikor mellette menének el, látják vala, hogy a fügefa gyökerestõl kiszáradott.
\par 21 És Péter visszaemlékezvén, monda néki: Mester nézd, a fügefa, a melyet megátkoztál, kiszáradott.
\par 22 És Jézus felelvén, monda nékik: Legyen hitetek Istenben.
\par 23 Mert bizony mondom néktek, ha valaki azt mondja ennek a hegynek: Kelj fel és ugorjál a tengerbe! és szívében nem kételkedik, hanem hiszi, hogy a mit mond, megtörténik, meg lesz néki, a mit mondott.
\par 24 Azért mondom néktek: A mit könyörgéstekben kértek, higyjétek, hogy mindazt megnyeritek, és meglészen néktek.
\par 25 És mikor imádkozva megállotok, bocsássátok meg, ha valaki ellen valami panaszotok van; hogy a ti mennyei Atyátok is megbocsássa néktek a ti vétkeiteket.
\par 26 Ha pedig ti meg nem bocsátotok, a ti mennyei Atyátok sem bocsátja meg a ti vétkeiteket.
\par 27 És ismét Jeruzsálembe menének. Mikor pedig õ a templomban körüljára, hozzámennek vala a fõpapok és az írástudók és a vének.
\par 28 És mondának néki: Micsoda hatalommal cselekszed ezeket? és ki adta néked a hatalmat, hogy ezeket cselekedd?
\par 29 Jézus pedig felelvén, monda nékik: Én is kérdek egy dolgot tõletek, és feleljetek meg nékem, akkor megmondom néktek, hogy miféle hatalomnál fogva cselekszem ezeket:
\par 30 A János keresztsége mennybõl vala-é, vagy emberektõl? feleljetek nékem.
\par 31 Azok pedig tanakodnak vala maguk között, mondván: Ha azt mondjuk: Mennybõl, azt fogja mondani: Miért nem hittetek tehát néki?
\par 32 Ha pedig azt mondjuk: Emberektõl, - félnek vala a néptõl. Mert mindenki azt tartja vala Jánosról, hogy valóban próféta vala.
\par 33 Felelvén tehát, mondának Jézusnak: Nem tudjuk. Jézus is felelvén, monda nékik: Én sem mondom meg néktek, miféle hatalomnál fogva cselekszem ezeket.

\chapter{12}

\par 1 És kezde nékik példázatokban beszélni: Egy ember  szõlõt ültetett, és körülvevé gyepûvel, és sajtót ása, és tornyot építe, és kiadá azt munkásoknak, és elutazék.
\par 2 És a maga idejében szolgát külde a munkásokhoz, hogy kapjon a munkásoktól a szõlõ gyümölcsébõl.
\par 3 Azok pedig megfogván azt, megverék, és üresen küldék vissza.
\par 4 És ismét külde hozzájuk egy másik szolgát; azt pedig kõvel dobálván meg, fejét betörék, és gyalázattal illetve, visszaküldék.
\par 5 És ismét másikat külde; ezt pedig megölék: és sok másokat; némelyeket megvervén, némelyeket pedig megölvén.
\par 6 Minthogy pedig még egy egyetlen szerelmes fia is vala, utoljára azt is elküldé hozzájok, ezt mondván: A fiamat meg fogják becsülni.
\par 7 Azok a munkások azonban ezt mondák magok között: Ez az örökös; jertek, öljük meg õt, és a miénk lesz az örökség.
\par 8 És megfogván azt, megölék, és a szõlõn kívül veték.
\par 9 Mit cselekszik hát a szõlõnek ura? Eljõ és elveszti a munkásokat, és a szõlõt másoknak adja.
\par 10 Ezt az írást sem olvastátok-é? A mely követ az építõk megvetettek, az lett a szeglet fejévé.
\par 11 Az Úrtól lett ez, és csodálatos a mi szemeink elõtt.
\par 12 És igyekeznek vala õt megfogni, de féltek a sokaságtól. Mert tudták, hogy a példázatot ellenük mondotta. Azért elhagyván õt, tovább menének.
\par 13 És küldének hozzá némelyeket a farizeusok és a Heródes pártiak közül, hogy megfogják õt a beszédben.
\par 14 Azok pedig odamenvén, mondának néki: Mester, tudjuk hogy igaz vagy és nem gondolsz senkivel; mert nem tekintesz emberek személyére, hanem igazság szerint tanítod az Istennek útját. Szabad-é a császárnak adót fizetni vagy nem? Fizessünk-é vagy ne fizessünk?
\par 15 Õ pedig ismervén az õ képmutatásukat, monda nékik: Mit kísértetek engem? Hozzatok nekem egy pénzt, hogy lássam.
\par 16 Azok pedig hozának. És monda nékik: Kié ez a kép és a felírás? Azok pedig mondának néki: A Császáré.
\par 17 És felelvén Jézus, monda nékik: Adjátok meg a mi a Császáré, a Császárnak, és a mi az Istené, az Istennek. És elálmélkodának õ rajta.
\par 18 És jövének hozzá Sadduczeusok, a kik azt mondják, hogy nincsen feltámadás. És megkérdezék õt, mondván:
\par 19 Mester, Mózes azt írta nékünk, hogy ha valakinek fitestvére meghalt, és feleséget hagyott hátra, gyermekeket pedig nem hagyott, akkor az õ feleségét vegye el az õ fitestvére, és támasszon magot a fitestvérének.
\par 20 Heten valának tehát fitestvérek. És az elsõ feleséget võn, de meghalván, magot nem hagya.
\par 21 És a második elvevé az asszonyt, de meghala, és magot õ sem hagya: a harmadik is hasonlóképen;
\par 22 És mind a hét elvevé azt, és magot nem hagyának. Legutoljára meghalt az asszony is.
\par 23 A feltámadáskor tehát, mikor feltámadnak, melyiköknek lesz a felesége? Mert mind a hétnek a felesége volt.
\par 24 Jézus pedig felelvén, monda nékik: Avagy nem azért tévelyegtek-é, mert nem ismeritek az írásokat, sem az Istennek hatalmát?
\par 25 Mert mikor a halálból feltámadnak, sem nem házasodnak, sem férjhez nem mennek, hanem olyanok lesznek, mint az angyalok a mennyekben.
\par 26 A halottakról pedig, hogy feltámadnak, nem olvastátok-é a Mózes könyvében, a csipkebokornál, hogy mi módon szólott néki az Isten, mondván: Én vagyok Ábrahám Istene, és Izsák Istene, és Jákób Istene.
\par 27 Az Isten nem holtaknak, hanem élõknek Istene. Ti tehát igen tévelyegtek.
\par 28 Akkor hozzá menvén egy az írástudók közül, a ki az õ vetekedésöket hallotta vala, és tudván, hogy jól megfelele nékik, megkérdezé tõle: Melyik az elsõ minden parancsolatok között?
\par 29 Jézus pedig felele néki: Minden parancsolatok között az elsõ: Halljad Izráel: Az Úr, a mi Istenünk egy Úr.
\par 30 Szeressed azért az Urat, a te Istenedet teljes szívedbõl, teljes lelkedbõl, és teljes elmédbõl és teljes erõdbõl. Ez az elsõ parancsolat.
\par 31 A második pedig hasonlatos ehhez: Szeresd felabarátodat, mint magadat. Nincs más ezeknél nagyobb parancsolat.
\par 32 Akkor monda néki az írástudó: Jól van, Mester, igazán mondád, hogy egy Isten van, és nincsen kívüle más.
\par 33 És szeretni õt teljes szívbõl, teljes elmébõl, teljes lélekbõl és teljes erõbõl, és szeretni embernek felebarátját, mint önmagát, többet ér minden égõáldozatnál és véres áldozatnál.
\par 34 Jézus pedig látván, hogy bölcsen felelt vala, monda néki: Nem messze vagy az Isten országától. És többé senki sem meri vala õt megkérdezni.
\par 35 És felele Jézus és monda, a mint a templomban tanít vala: Mi módon mondják az írástudók, hogy a Krisztus Dávidnak Fia?
\par 36 Hiszen Dávid maga mondotta a  Szent Lélek által: Monda az Úr az én uramnak: ülj az én jobb kezem felõl, míglen vetem a te ellenségeidet lábaid alá zsámolyul.
\par 37 Tehát maga Dávid nevezi õt Urának, mimódon fia hát néki? És a nagy sokaság örömest hallgatja vala õt.
\par 38 Õ pedig monda nékik az õ tanításában: Õrizkedjetek az írástudóktól, a kik örömest járnak hosszú köntösökben és szeretik a piaczokon való köszöntéseket.
\par 39 És a gyülekezetekben az elõlüléseket, és a lakomákon a fõhelyeket:
\par 40 A kik az özvegyeknek házát fölemésztik és színbõl hosszan imádkoznak: ezek súlyosabb ítélet alá esnek.
\par 41 És leülvén Jézus a templomperselynek átellenében, nézi vala, hogy a sokaság miként vet pénzt a perselybe. Sok gazdag pedig sokat vet vala abba.
\par 42 És egy szegény özvegy asszony is odajövén, két fillért, azaz egy negyed pénzt vete bele.
\par 43 Akkor elõszólítván tanítványait, monda nékik: Bizony mondom néktek, hogy ez a szegény özvegy asszony többet vetett, hogynem mind a többi, a kik a perselybe vetettek vala.
\par 44 Mert azok mindnyájan az õ fölöslegükbõl vetének; ez pedig az õ szegénységébõl, a mije csak volt, mind beveté, az õ egész vagyonát.

\chapter{13}

\par 1 Mikor pedig a templomból kiméne, monda néki egy az õ tanítványai közül: Mester: nézd, milyen kövek és milyen épületek!
\par 2 Jézus pedig felelvén, monda néki: Látod ezeket a nagy épületeket? Nem marad kõ kövön, a mely le nem romboltatik.
\par 3 Mikor pedig az olajfák hegyén ül vala, a templom átellenében, megkérdezék õt magukban Péter, Jakab, János és András:
\par 4 Mondd meg nékünk, mikor történnek meg ezek; és mi lesz a jel, a mikor mindezek beteljesednek?
\par 5 Jézus pedig felelvén nékik, kezdé mondani: Meglássátok, hogy valaki el ne hitessen titeket.
\par 6 Mert sokan jõnek majd az én nevemben, a kik azt mondják: Én vagyok: és sokakat elhitetnek.
\par 7 Mikor pedig hallani fogtok háborúkról és háborúk híreirõl, meg ne rémüljetek, mert meg kell lenniök; de ez még nem a vég.
\par 8 Mert nemzet nemzet ellen, és ország ország ellen támad; és lesznek földindulások mindenfelé, és lesznek éhségek és háborúságok.
\par 9 Nyomorúságoknak kezdetei ezek. Ti pedig vigyázzatok magatokra: mert törvényszékeknek adnak át titeket, és gyülekezetekben vernek meg titeket, és helytartók és királyok elé állítanak és érettem, bizonyságul õ nékik.
\par 10 De elõbb hirdettetnie kell az evangyéliomnak minden pogányok között.
\par 11 Mikor pedig fogva visznek, hogy átadjanak titeket, ne aggodalmaskodjatok elõre, hogy mit szóljatok, és ne gondolkodjatok, hanem a mi adatik néktek abban az órában, azt szóljátok; mert nem ti vagytok, a kik szólotok, hanem a Szent Lélek.
\par 12 Halálra fogja pedig adni testvér testvérét, atya gyermekét; és magzatok támadnak szülõk ellen, és megöletik õket.
\par 13 És lesztek gyûlöletesek mindenki elõtt az én nevemért; de a ki mindvégig megmarad, az megtartatik.
\par 14 Mikor pedig látjátok a pusztító utálatosságot, a melyrõl Dániel próféta szólott, ott állani, a hol nem kellene (a ki  olvassa, értse meg), akkor a kik Júdeában lesznek, fussanak a hegyekre;
\par 15 A háztetõn levõ pedig le ne szálljon a házba, se be ne menjen, hogy házából valamit kivigyen;
\par 16 És a mezõn levõ haza ne térjen, hogy ruháját elvigye.
\par 17 Jaj pedig a terhes és a szoptató asszonyoknak azokban a napokban.
\par 18 Imádkozzatok pedig, hogy a ti futástok ne télen legyen.
\par 19 Mert azok a napok olyan nyomorúságosak lesznek, a milyenek a világ kezdete óta, a melyet Isten teremtett, mind ez ideig nem voltak, és nem is lesznek.
\par 20 És ha az Úr meg nem rövidítette volna azokat a napokat, egyetlen test sem menekülne meg; de a választottakért, a kiket kiválasztott, megrövidítette azokat a napokat.
\par 21 Ha pedig akkor ezt mondja néktek valaki: Ímé itt a Krisztus, vagy: Ímé amott, ne higyjétek.
\par 22 Mert hamis Krisztusok és hamis próféták támadnak, és jeleket és csodákat tesznek, hogy elhitessék, ha lehet, még a választottakat is.
\par 23 Ti pedig vigyázzatok; ímé elõre megmondottam néktek mindent.
\par 24 De azokban a napokban, azután a nyomorúság után, a nap elsötétedik, és a hold nem fénylik,
\par 25 És az ég csillagai lehullanak, és az egekben levõ hatalmasságok megrendülnek.
\par 26 És akkor meglátják az embernek Fiát eljõni felhõkben nagy hatalommal és dicsõséggel.
\par 27 És akkor elküldi az az õ angyalait, és egybegyûjti az õ választottait a négy szelek felõl, a föld végsõ határától az ég végsõ határáig.
\par 28 A fügefáról vegyétek pedig a példát. A mikor ága már zsendül, és levelet hajt, tudjátok, hogy közel van a nyár.
\par 29 Azonképen ti is, mikor látjátok, hogy ezek meglesznek, tudjátok meg, hogy közel van, az ajtó elõtt.
\par 30 Bizony mondom néktek, hogy el nem múlik ez a nemzetség, a míg meg nem lesznek mindezek.
\par 31 Az ég és a föld elmúlnak, de az én beszédeim soha el nem múlnak.
\par 32 Arról a napról és óráról pedig senki semmit sem tud, sem az égben az angyalok, sem a Fiú, hanem csak az Atya.
\par 33 Figyeljetek, vigyázzatok és imádkozzatok; mert nem tudjátok, mikor jõ el az az idõ.
\par 34 Úgy mint az az ember, a ki messze útra kelve, házát elhagyván, és szolgáit felhatalmazván, és kinek-kinek a maga dolgát megszabván, az ajtónállónak is megparancsolta, hogy vigyázzon.
\par 35 Vigyázzatok azért, mert nem tudjátok, mikor érkezik meg a háznak ura, este-é vagy éjfélkor, vagy kakasszókor, vagy reggel?
\par 36 Hogy, ha hirtelen megérkezik, ne találjon titeket aludva.
\par 37 A miket pedig néktek mondok, mindenkinek mondom: Vigyázzatok!

\chapter{14}

\par 1 Két nap mulva pedig húsvét vala és a kovásztalan kenyerek ünnepe. És a fõpapok és az írástudók tanakodnak vala, hogy csalárdsággal mimódon fogják meg és öljék meg õt.
\par 2 Mert azt mondják vala: Ne az ünnepen, hogy a nép fel ne zendüljön.
\par 3 Mikor pedig Bethániában a poklos Simon házánál vala, a mint asztalhoz üle, egy asszony méne oda, a kinél alabástrom edény vala valódi és igen drága nárdus olajjal: és eltörvén az alabástrom edényt kitölté azt az õ fejére.
\par 4 Némelyek pedig háborognak vala magok között és mondának: Mire való volt az olajnak ez a tékozlása?
\par 5 Mert el lehetett volna azt adni háromszáz pénznél is többért, és odaadni a szegényeknek. És zúgolódnak vala ellene.
\par 6 Jézus pedig monda: Hagyjatok békét néki; miért bántjátok õt? jó dolgot cselekedett én velem.
\par 7 Mert a szegények mindenkor veletek lesznek, és a mikor csak akarjátok, jót tehettek velök; de én nem leszek mindenkor veletek.
\par 8 Õ a mi tõle telt, azt tevé: elõre megkente az én testemet a temetésre.
\par 9 Bizony mondom néktek: Valahol csak prédikálják ezt az evangyéliomot az egész világon, a mit ez az asszony cselekedett, azt is hirdetni fogják az õ emlékezetére.
\par 10 Akkor Júdás, az Iskariotes, egy a tizenkettõ közül, elméne a fõpapokhoz, hogy õt azoknak elárulja.
\par 11 Azok pedig, a mint meghallák, örvendezének, és igérék, hogy pénzt adnak néki. Õ pedig keresi vala, mimódon árulhatná el õt jó alkalommal.
\par 12 És a kovásztalan kenyerek ünnepének elsõ napján, mikor a húsvéti bárányt vágják vala, mondának néki az õ tanítványai: Hol akarod, hogy elmenvén megkészítsük, hogy megehesd a húsvéti bárányt?
\par 13 Akkor elkülde kettõt az õ tanítványai közül, és monda nékik: Menjetek el a városba, és egy ember jõ elõtökbe, a ki egy korsó vizet visz; kövessétek õt,
\par 14 És a hová bemegy, mondjátok a házi gazdának: A Mester kérdi: hol van az a szállás, a hol megeszem az én tanítványaimmal a húsvéti bárányt?
\par 15 És õ mutat néktek egy nagy vacsoráló házat berendezve, készen: ott készítsétek el nékünk.
\par 16 Elmenének azért az õ tanítványai, és jutának a városba, és úgy találák, a mint nékik megmondotta, és elkészíték a húsvéti bárányt.
\par 17 Mikor pedig este lõn, oda méne a tizenkettõvel.
\par 18 És a mikor leülnek és esznek vala monda Jézus: Bizony mondom néktek, egy közületek elárul engem, a ki velem eszik.
\par 19 Õk pedig kezdének szomorkodni és néki egyenként mondani: Csak nem én? A másik is: Csak nem én?
\par 20 Õ pedig felelvén, monda nékik: Egy a tizenkettõ közül, a ki velem együtt márt a tálba.
\par 21 Az embernek Fia jóllehet elmegy, a mint meg van írva felõle; de jaj annak az embernek, a ki az embernek Fiát elárulja; jobb lenne annak az embernek, ha nem született volna.
\par 22 És mikor õk evének, vévén Jézus a kenyeret, és hálákat adván, megtöré és adá nékik, mondván: Vegyétek, egyétek; ez az én testem.
\par 23 És vévén a poharat, és hálákat adván, adá nékik; és ivának abból mindnyájan;
\par 24 És monda nékik: Ez az én vérem, az új szövetség vére, a mely sokakért kiontatik.
\par 25 Bizony mondom néktek, nem iszom többé a szõlõtõnek gyümölcsébõl mind ama napig, a mikor mint újat iszom azt az Isten országában.
\par 26 És dicséretet énekelve kimenének az olajfák hegyére.
\par 27 És monda nékik Jézus: Ezen az éjszakán mindnyájan megbotránkoztok bennem; mert meg van írva: Megverem  a pásztort, és elszélednek a juhok.
\par 28 De feltámadásom után elõttetek fogok felmenni Galileába.
\par 29 Péter pedig monda néki: Ha mindnyájan megbotránkoznak is, de én nem.
\par 30 És monda néki Jézus: Bizony mondom néked, hogy ma, ezen az éjszakán, mielõtt a kakas  kétszer szólana, háromszor tagadsz meg engem.
\par 31 Õ pedig annál inkább erõsíti vala: Ha veled együtt kell is meghalnom, semmiképen meg nem tagadlak téged. Hasonlóképen szólanak vala a többiek is.
\par 32 És menének ama helyre, a melynek Gecsemáné a neve; és monda az õ tanítványainak: Üljetek le itt, a míg imádkozom.
\par 33 És maga mellé vevé Pétert és Jakabot és Jánost, és kezde rettegni és gyötrõdni;
\par 34 És monda nékik: Szomorú az én lelkem mind halálig; maradjatok itt, és vigyázzatok.
\par 35 És egy kevéssé elõre menvén, a földre esék, és imádkozék, hogy, ha lehetséges, múljék el tõle ez az óra;
\par 36 És monda: Abba, Atyám! Minden lehetséges néked. Vidd el tõlem ezt a poharat; mindazáltal ne az én akaratom legyen meg, hanem a tied.
\par 37 Azután visszatére és aluva találá õket, és monda Péternek: Simon, alszol? Nem bírtál egy óráig vigyázni?
\par 38 Vigyázzatok és imádkozzatok, hogy kísértetbe ne jussatok; a lélek ugyan kész, de a test erõtelen.
\par 39 És ismét elmenvén, imádkozék, ugyanazon szavakkal szólván.
\par 40 A mikor pedig visszatére, ismét aluva találá õket; mert a szemeik megnehezedtek vala, és nem tudták mit feleljenek néki.
\par 41 Harmadszor is jöve, és monda nékik: Aludjatok immár és nyugodjatok. Elég; eljött az óra; ímé az embernek Fia a bûnösök kezébe adatik.
\par 42 Keljetek föl, menjünk: ímé elközelgett, a ki engem elárul.
\par 43 És mindjárt még mikor õ szól vala, eljöve Júdás, egy a tizenkettõ közül, és vele együtt nagy sokaság, fegyverekkel és botokkal, a fõpapoktól, az írástudóktól és a vénektõl.
\par 44 Az õ elárulója pedig jelt ada nékik, mondván: A kit megcsókolok majd, õ az; fogjátok meg azt, és vigyétek el biztonsággal.
\par 45 És odajutván, azonnal hozzáméne, és monda: Mester! Mester! és megcsókolá õt.
\par 46 Azok pedig ráveték kezeiket, és megfogák õt.
\par 47 De egy az ott állók közül az õ szablyáját kivonván, a fõpap szolgájához csapa, és levágá annak fülét.
\par 48 Jézus pedig felelvén, monda nékik: Mint egy rablóra, úgy jöttetek-é reám fegyverekkel és botokkal, hogy megfogjatok engem?!
\par 49 Naponta nálatok valék, a templomban tanítva, és nem fogtatok meg engem; de szükség, hogy az írások beteljesedjenek.
\par 50 Akkor elhagyván õt, mindnyájan elfutának.
\par 51 Egy ifjú pedig követé õt, a kinek testét csak egy gyolcs ing takarta; és megfogák õt az ifjak.
\par 52 De õ ott hagyva az ingét, meztelenül elszalada tõlük.
\par 53 És vivék Jézust a fõpaphoz. És oda gyûlének mindnyájan a fõpapok, a vének és az írástudók.
\par 54 Péter pedig távolról követé õt, be egészen a fõpap udvaráig: és ott üle a szolgákkal, és melegszik vala a tûznél.
\par 55 A fõpapok pedig és az egész tanács bizonyságot keresnek vala Jézus ellen, hogy megölhessék õt; de nem találnak vala.
\par 56 Mert sokan tesznek vala ugyan hamis tanúbizonyságot ellene, de a bizonyságtételek nem valának megegyezõk.
\par 57 És némelyek fölkelének és hamis tanúbizonyságot tõnek ellene, mondván:
\par 58 Mi hallottuk, mikor ezt mondá: Én lerontom ezt a kézzel csinált templomot, és három nap alatt mást építek, a mely nem kézzel csináltatott.
\par 59 De még így sem vala egyezõ az õ bizonyságtételük.
\par 60 Akkor a fõpap odaállván a középre, megkérdé Jézust, mondván: Semmit sem felelsz-é? Miféle bizonyságot tesznek ezek te ellened?
\par 61 Õ pedig hallgat vala, és semmit sem felele. Ismét megkérdezé õt a fõpap, és monda néki: Te vagy-é a Krisztus, az áldott Isten Fia?
\par 62 Jézus pedig monda: Én vagyok. És meglátjátok majd az embernek Fiát ülni a hatalomnak jobbján, és eljõni az ég felhõivel.
\par 63 A fõpap pedig megszaggatván ruháit, monda: Mi szükségünk van még tanúkra?
\par 64 Hallátok a káromlást. Mi tetszik néktek? Azok pedig halálra méltónak ítélték õt mindnyájan.
\par 65 És kezdék õt némelyek köpdösni, és az õ orczáját elfedni, és õt öklözni, és mondani néki: Prófétálj! A szolgák pedig arczul csapdossák vala õt.
\par 66 A mint pedig Péter lent vala az udvarban, odajöve egy a fõpap szolgálói közül;
\par 67 És meglátván Pétert, a mint melegszik vala, rátekintvén, monda: Te is a Názáreti Jézussal valál!
\par 68 Õ pedig megtagadá, mondván: Nem ismerem, s nem is értem, mit mondasz. És kiméne a tornáczra; és a kakas megszólala.
\par 69 A szolgáló pedig meglátva õt, kezdé ismét mondani az ott állóknak: Ez közülök való.
\par 70 Õ pedig ismét megtagadá. De kevés idõ múlva az ott állók ismét mondják vala Péternek: Bizony közülök való vagy; mert Galileabeli is vagy, és a beszéded is hasonló.
\par 71 Õ pedig kezde átkozódni és esküdözni, hogy: Nem ismerem azt az embert, a kirõl beszéltek.
\par 72 És másodszor szóla a kakas. És Péternek eszébe juta a beszéd, a melyet néki Jézus mondott vala, hogy mielõtt a kakas kétszer szólana, háromszor megtagadsz engem. És sírva fakada.

\chapter{15}

\par 1 És mindjárt reggel tanácsot tartván a fõpapok a vénekkel és írástudókkal, és az egész tanács, megkötözvén Jézust, elvivék és átadák Pilátusnak.
\par 2 És megkérdé õt Pilátus: Te vagy-é a zsidók királya? Õ pedig felelvén, monda néki: Te mondod.
\par 3 És erõsen vádolják vala õt a fõpapok.
\par 4 Pilátus pedig ismét megkérdé õt, mondván: Semmit sem felelsz-é? Ímé, mennyi tanúbizonyságot szólnak ellened!
\par 5 Jézus pedig semmit sem felele, annyira hogy Pilátus elcsudálkozék.
\par 6 Ünnepenként pedig egy foglyot szokott vala elbocsátani nékik, a kit épen óhajtának.
\par 7 Vala pedig egy Barabbás nevû, megkötöztetve ama lázadókkal együtt, a kik a lázadás alkalmával gyilkosságot követtek vala el.
\par 8 És a sokaság kiáltván, kezdé kérni Pilátust arra, a mit mindenkor megtesz vala nékik.
\par 9 Pilátus pedig felele nékik, mondván: Akarjátok-é, hogy elbocsássam néktek a zsidók királyát?
\par 10 Mert tudja vala, hogy írigységbõl adták õt kézbe a fõpapok.
\par 11 A fõpapok azonban felindíták a sokaságot, hogy inkább Barabbást bocsássa el nékik.
\par 12 Pilátus pedig felelvén, ismét monda nékik: Mit akartok tehát, hogy cselekedjem ezzel, a kit a zsidók királyának mondotok?
\par 13 És azok ismét kiáltának: Feszítsd meg õt!
\par 14 Pilátus pedig monda nékik: Mert mi rosszat cselekedett? Azok pedig annál jobban kiáltanak vala: Feszítsd meg õt!
\par 15 Pilátus pedig eleget akarván tenni a sokaságnak, elbocsátá nékik Barabbást, Jézust pedig megostoroztatván, kezökbe adá, hogy megfeszítsék.
\par 16 A vitézek pedig elvivék õt az udvar belsõ részébe, a mi az õrház; és összehívák az egész csapatot.
\par 17 És bíborba öltözteték õt, és tövisbõl font koszorút tevének a fejére,
\par 18 És elkezdék õt köszönteni: Üdvöz légy, zsidók királya!
\par 19 És verik vala a fejét nádszállal, és köpdösik vala õt, és térdet hajtva tisztelik vala õt.
\par 20 Mikor pedig kicsúfolták õt, leveték róla a bíbor ruhát, és a maga ruháiba öltözteték; és kivivék õt, hogy megfeszítsék.
\par 21 És kényszerítének egy mellettök elmenõt, bizonyos czirénei Simont, a ki a mezõrõl jõ vala, Alekszándernek és  Rufusnak az atyját, hogy vigye az õ keresztjét.
\par 22 És vivék õt a Golgotha nevû helyre, a mely megmagyarázva annyi, mint: koponya helye.
\par 23 És mirhás bort adnak vala néki inni; de õ nem fogadá el.
\par 24 És megfeszítvén õt, eloszták az õ ruháit, sorsot vetvén azokra, ki mit kapjon.
\par 25 Vala pedig három óra, mikor megfeszíték õt.
\par 26 Az õ kárhoztatásának oka pedig így vala fölébe felírva: A zsidók királya.
\par 27 Két rablót is megfeszítének vele, egyet jobb és egyet bal keze felõl.
\par 28 És beteljesedék az írás, a mely azt mondja: És a bûnösök közé számláltaték.
\par 29 Az arra menõk pedig szidalmazzák vala õt, fejüket hajtogatván és mondván: Hah! a ki lerontod a templomot, és három nap alatt fölépíted;
\par 30 Szabadítsd meg magadat, és szállj le a keresztrõl!
\par 31 Hasonlóképen pedig a fõpapok is, csúfolodván egymás között, az írástudókkal együtt mondják vala: Másokat megtartott, magát nem bírja megtartani.
\par 32 A Krisztus, az Izráel királya, szálljon le most a keresztrõl, hogy lássuk és higyjünk. A kiket vele feszítettek meg, azok is szidalmazzák vala õt.
\par 33 Mikor pedig hat óra lõn, sötétség támada az egész földön kilencz óráig.
\par 34 És kilencz órakor fennszóval kiálta Jézus mondván: Elói, Elói! Lamma Sabaktáni? a mi megmagyarázva annyi, mint: Én Istenem, én Istenem! miért hagyál el engemet?
\par 35 Némelyek pedig meghallván ezt az ott állók közül, mondának: Ímé Illést hívja.
\par 36 Egy ember pedig odafutamodék és egy szivacsot megtöltvén eczettel és azt nádszálra tûzvén, inni ada néki, mondván: Hagyjátok el, lássuk, ha eljõ-é Illés, hogy levegye õt.
\par 37 Jézus pedig nagy fennszóval kiáltván kibocsátá lelkét.
\par 38 És a templom kárpítja fölétõl aljáig ketté hasada.
\par 39 Látván pedig a százados, a ki vele átellenben áll vala, hogy ekként kiáltva bocsátá ki lelkét, monda: Bizony, ez az ember Isten Fia vala!
\par 40 Valának pedig asszonyok is, a kik távolról nézik vala, a kik között vala Mária Magdaléna,  és Mária, a kis Jakabnak és Józsénak anyja, és Salomé,
\par 41 A kik, mikor Galileában vala, akkor is követték vala õt, és szolgálnak vala néki; és sok más asszony, a kik vele mentek vala fel Jeruzsálembe.
\par 42 És mikor immár este lõn, mivelhogy péntek vala, azaz szombat elõtt való nap,
\par 43 Eljöve az arimathiai József, egy tisztességes tanácsbeli, a ki maga is várja vala az Isten országát; beméne bátran Pilátushoz, és kéré Jézusnak testét.
\par 44 Pilátus pedig csodálkozék, hogy immár meghalt volna; és magához hivatva a századost, megkérdé tõle, ha régen halt-é meg?
\par 45 És megtudván a századostól, odaajándékozá a testet Józsefnek.
\par 46 Õ pedig gyolcsot vásárolván, és levévén õt, begöngyölé a gyolcsba, és elhelyezé egy sírboltba, a mely kõsziklából vala kivágva; és követ hengeríte a sírbolt szájára.
\par 47 Mária Magdaléna pedig és Mária, a Józsé anyja, nézik vala, hová helyezék.

\chapter{16}

\par 1 Mikor pedig elmult a szombat, Mária Magdaléna, és Mária a Jakab anyja, és Salomé, drága keneteket vásárlának, hogy elmenvén, megkenjék õt.
\par 2 És korán reggel, a hétnek elsõ napján a sírbolthoz menének napfelköltekor.
\par 3 És mondják vala maguk között: Kicsoda hengeríti el nékünk a követ a sírbolt szájáról?
\par 4 És odatekintvén, láták, hogy a kõ el van hengerítve; mert felette nagy vala.
\par 5 És bemenvén a sírboltba, látának egy ifjút ülni jobb felõl, fehér ruhába öltözve; és megfélemlének.
\par 6 Az pedig monda nékik: Ne féljetek. A Názáreti Jézust keresitek, a ki megfeszíttetett; föltámadott, nincsen itt; ímé a hely, a hová õt helyezék.
\par 7 De menjetek el, mondjátok meg az õ tanítványainak és Péternek, hogy elõttetek megyen Galileába; ott meglátjátok õt, a mint megmondotta néktek.
\par 8 És nagyhamar kijövén, elfutának a sírbolttól, mert félelem és álmélkodás fogta vala el õket; és senkinek semmit sem szólának, mert félnek vala.
\par 9 Mikor pedig reggel, a hétnek elsõ napján föltámadott vala, megjelenék elõször Mária Magdalénának, a kibõl hét ördögöt ûzött vala ki.
\par 10 Ez elmenvén, megjelenté azoknak, a kik vele valának és keseregnek és sírnak vala.
\par 11 Azok pedig mikor hallották, hogy él és õ látta vala, nem hivék.
\par 12 Ezután pedig közülök kettõnek jelenék meg más alakban, útközben, mikor a mezõre mennek vala.
\par 13 Ezek is elmenvén, megjelenték a többieknek; ezeknek sem hivének.
\par 14 Azután, mikor asztalnál ülnek vala megjelenék magának a tizenegynek, és szemükre hányá az õ hitetlenségöket és keményszívûségöket, hogy azoknak, a kik õt feltámadva látták vala, nem hivének,
\par 15 És monda nékik: Elmenvén e széles világra, hirdessétek az evangyéliomot minden teremtésnek.
\par 16 A ki hiszen és megkeresztelkedik, idvezül; a ki pedig nem hiszen, elkárhozik.
\par 17 Azok pedig, a kik hisznek, ilyen jelek követik: az én nevemben ördögöket ûznek;  új nyelveken szólnak.
\par 18 Kígyókat vesznek föl; és ha valami halálost isznak, meg nem árt nékik: betegekre  vetik kezeiket, és meggyógyulnak.
\par 19 Az Úr azért, minekutána szólott vala nékik, felviteték a mennybe, és üle az Istennek jobbjára.
\par 20 Azok pedig kimenvén, prédikálának mindenütt, az Úr együtt munkálván velök, és megerõsítvén az ígét a jelek által, a melyek követik vala. Ámen!


\end{document}