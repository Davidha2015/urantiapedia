\begin{document}

\title{John}


\chapter{1}

\par 1 Kezdetben vala az Íge, és az Íge vala az Istennél, és Isten vala az Íge.
\par 2 Ez kezdetben az Istennél vala.
\par 3 Minden õ általa lett és nála nélkül semmi sem lett, a mi lett.
\par 4 Õ benne vala az élet, és az élet vala az emberek világossága;
\par 5 Ez a világosság a sötétségben fénylik, de a sötétség nem fogadta be azt.
\par 6 Vala egy Istentõl küldött ember, kinek neve János.
\par 7 Ez jött tanúbizonyságul, hogy bizonyságot tegyen a világosságról, hogy mindenki higyjen õ általa.
\par 8 Nem õ vala a világosság, hanem jött, hogy bizonyságot tegyen a világosságról.
\par 9 Az igazi világosság eljött volt már a világba, a mely megvilágosít minden embert.
\par 10 A világban volt és a világ általa lett, de a világ nem ismerte meg õt.
\par 11 Az övéi közé jöve, és az övéi nem fogadák be õt.
\par 12 Valakik pedig befogadák õt, hatalmat ada azoknak, hogy Isten fiaivá legyenek, azoknak, a kik az õ nevében hisznek;
\par 13 A ki nem vérbõl, sem a testnek akaratából, sem a férfiúnak indulatjából, hanem Istentõl születtek.
\par 14 És az Íge testté lett és lakozék mi közöttünk (és láttuk az õ dicsõségét, mint az Atya egyszülöttjének dicsõségét), a ki teljes vala kegyelemmel és igazsággal.
\par 15 János bizonyságot tett õ róla, és kiáltott, mondván: Ez vala, a kirõl mondám: A ki utánam jõ, elõttem lett, mert elõbb volt nálamnál.
\par 16 És az õ teljességébõl vettünk mindnyájan kegyelmet is kegyelemért.
\par 17 Mert a törvény Mózes által adatott, a kegyelem pedig és az igazság Jézus Krisztus által lett.
\par 18 Az Istent soha senki nem látta; az egyszülött Fiú, a ki az Atya kebelében van, az jelentette ki õt.
\par 19 És ez a János bizonyságtétele, a mikor a zsidók papokat és Lévitákat küldöttek Jeruzsálembõl, hogy megkérdezzék õt: Kicsoda vagy te?
\par 20 És megvallá és nem tagadá; és megvallá, hogy: Nem én vagyok a Krisztus.
\par 21 És kérdezék õt: Kicsoda tehát? Illés vagy-é te? És monda: Nem vagyok. A próféta vagy-é te? És õ felele: Nem.
\par 22 Mondának azért néki: Kicsoda vagy? Hogy megfelelhessünk azoknak, a kik minket elküldöttek: Mit mondasz magad felõl?
\par 23 Monda: Én kiáltó szó vagyok a pusztában. Egyengessétek az Úrnak útját, a mint megmondotta Ésaiás próféta.
\par 24 És a küldöttek a farizeusok közül valók voltak:
\par 25 És megkérdék õt és mondának néki: Miért keresztelsz tehát, ha te nem vagy a Krisztus, sem Illés, sem a próféta?
\par 26 Felele nékik János, mondván: Én vízzel keresztelek; de köztetek van, a kit ti nem ismertek.
\par 27 Õ az, a ki utánam jõ, a ki elõttem lett, a kinek én nem  vagyok méltó, hogy saruja szíjját megoldjam.
\par 28 Ezek Béthabarában lettek, a Jordánon túl, a hol János keresztel vala.
\par 29 Másnap látá János Jézust õ hozzá menni, és monda: Ímé az Istennek ama báránya, a ki  elveszi a világ bûneit!
\par 30 Ez az, a kirõl én ezt mondám: Én utánam jön egy férfiú, a ki elõttem lett, mert elõbb volt nálamnál.
\par 31 És én nem ismertem õt; de hogy megjelentessék Izráelnek, azért jöttem én, a ki vízzel keresztelek.
\par 32 És bizonyságot tõn János, mondván: Láttam a Lelket leszállani az égbõl, mint egy galambot; és megnyugovék õ rajta.
\par 33 És én nem ismertem õt; de a ki elkülde engem, hogy vízzel kereszteljek, az mondá nékem: A kire látod a Lelket leszállani és rajta megnyugodni, az az, a ki keresztel Szent Lélekkel.
\par 34 És én láttam, és bizonyságot tettem, hogy ez az Isten Fia.
\par 35 Másnap ismét ott állt vala János és kettõ az õ tanítványai közül;
\par 36 És ránézvén Jézusra, a mint ott jár vala, monda: Ímé az Isten Báránya!
\par 37 És hallá õt a két tanítvány, a mint szól vala, és követék Jézust.
\par 38 Jézus pedig hátrafordulván és látván, hogy követik azok, monda nékik:
\par 39 Mit kerestek? Azok pedig mondának néki: Rabbi, (a mi megmagyarázva azt teszi: Mester) hol lakol?
\par 40 Monda nékik: Jõjjetek és lássátok meg. Elmenének és megláták, hol lakik; és nála maradának azon a napon: vala pedig körülbelül tíz óra.
\par 41 A kettõ közül, a kik Jánostól ezt hallották és õt követték vala, András volt az egyik, a Simon Péter testvére.
\par 42 Találkozék ez elõször a maga testvérével, Simonnal, és monda néki: Megtaláltuk a Messiást (a mi megmagyarázva azt teszi: Krisztus);
\par 43 És vezeté õt Jézushoz. Jézus pedig reá tekintvén, monda: Te Simon vagy, a Jóna fia; te Kéfásnak fogsz hivatni (a mi megmagyarázva: Kõszikla).
\par 44 A következõ napon Galileába akart menni Jézus; és találkozék Fileppel, és monda néki: Kövess engem!
\par 45 Filep pedig Bethsaidából, az András és Péter városából való volt.
\par 46 Találkozék Filep Nátánaellel, és monda néki: A ki felõl írt Mózes a törvényben, és a próféták, megtaláltuk a názáreti Jézust, Józsefnek fiát.
\par 47 És monda néki Nátánael: Názáretbõl támadhat-é valami jó? Monda néki Filep: Jer és lásd meg!
\par 48 Látá Jézus Nátánaelt õ hozzá menni, és monda õ felõle: Ímé egy igazán Izráelita, a kiben hamisság nincsen.
\par 49 Monda néki Nátánael: Honnan ismersz engem? Felele Jézus és monda néki: Mielõtt hítt téged Filep, láttalak téged, a mint a fügefa alatt voltál.
\par 50 Felele Nátánael és monda néki: Rabbi, te vagy az Isten Fia, te vagy az Izráel Királya!
\par 51 Felele Jézus és monda néki: Hogy azt mondám néked: láttalak a fügefa alatt, hiszel? Nagyobbakat látsz majd ezeknél.
\par 52 És monda néki: Bizony, bizony mondom néktek: Mostantól fogva meglátjátok a megnyilt eget, és az Isten angyalait, a mint felszállnak és leszállnak az ember Fiára.

\chapter{2}

\par 1 És harmadnapon menyegzõ lõn a galileai Kánában; és ott volt Jézus anyja;
\par 2 És Jézus is meghivaték az õ tanítványaival együtt a menyegzõbe.
\par 3 És elfogyván a bor, a Jézus anyja monda néki: Nincs boruk.
\par 4 Monda néki Jézus: Mi közöm nékem te hozzád, oh asszony? Nem jött még el az én órám.
\par 5 Mond az õ anyja a szolgáknak: Valamit mond néktek, megtegyétek.
\par 6 Vala pedig ott hat kõveder elhelyezve a zsidók tisztálkodási módja szerint, melyek közül egybe-egybe két-három métréta fér vala.
\par 7 Monda nékik Jézus: Töltsétek meg a vedreket vízzel. És megtölték azokat színig.
\par 8 És monda nékik: Most merítsetek, és vigyetek a násznagynak. És vittek.
\par 9 A mint pedig megízlelé a násznagy a borrá lett vizet, és nem tudja vala, honnét van, (de a szolgák tudták, a kik a vizet merítik vala), szólítá a násznagy a võlegényt,
\par 10 És monda néki: Minden ember a jó bort adja fel elõször, és mikor megittasodtak, akkor az alábbvalót: te a jó bort ekkorra tartottad.
\par 11 Ezt az elsõ jelt a galileai Kánában tevé Jézus, és megmutatá az õ dicsõségét; és hivének benne az õ tanítványai.
\par 12 Azután leméne Kapernaumba, õ és az õ anyja és a testvérei és tanítványai; és ott maradának néhány napig,
\par 13 Mert közel vala a zsidók husvétja, és felméne Jézus Jeruzsálembe.
\par 14 És ott találá a templomban az ökrök, juhok és galambok árúsait és a pénzváltókat, a mint ülnek vala:
\par 15 És kötélbõl ostort csinálván, kiûzé mindnyájokat a templomból, az ökröket is a juhokat is; és a pénzváltók pénzét kitölté, az asztalokat pedig feldönté;
\par 16 És a galambárúsoknak monda: Hordjátok el ezeket innen; ne tegyétek az én Atyámnak házát kalmárság házává.
\par 17 Megemlékezének pedig az õ tanítványai, hogy meg van írva: A te házadhoz való féltõ szeretet emészt engem.
\par 18 Felelének azért a zsidók és mondának néki: Micsoda jelt mutatsz nékünk, hogy ezeket cselekszed?
\par 19 Felele Jézus és monda nékik: Rontsátok le a templomot, és három nap alatt megépítem azt.
\par 20 Mondának azért a zsidók: Negyvenhat esztendeig épült ez a templom, és te három nap alatt megépíted azt?
\par 21 Õ pedig az õ testének templomáról szól vala.
\par 22 Mikor azért feltámadt a halálból, megemlékezének az õ tanítványai, hogy ezt mondta; és hivének az írásnak, és a beszédnek, a melyet Jézus mondott vala.
\par 23 A mint pedig Jeruzsálemben vala husvétkor az ünnepen, sokan hivének az õ nevében, látván az õ jeleit, a melyeket cselekszik vala.
\par 24 Maga azonban Jézus nem bízza vala magát reájok, a miatt, hogy õ ismeré mindnyájokat.
\par 25 És mivelhogy nem szorult rá, hogy valaki bizonyságot tegyen az emberrõl; mert magától is tudta, mi volt az emberben.

\chapter{3}

\par 1 Vala pedig a farizeusok közt egy ember, a neve Nikodémus, a zsidók fõembere:
\par 2 Ez jöve Jézushoz éjjel, és monda néki: Mester, tudjuk, hogy Istentõl jöttél tanítóul; mert senki sem teheti e jeleket, a melyeket te teszel, hanem ha az Isten van  vele.
\par 3 Felele Jézus és monda néki: Bizony, bizony mondom néked: ha valaki újonnan nem születik, nem láthatja az Isten országát.
\par 4 Mondta néki Nikodémus: Mimódon születhetik az ember, ha vén? Vajjon bemehet-é az õ anyjának méhébe másodszor, és születhetik-é?
\par 5 Felele Jézus: Bizony, bizony mondom néked: Ha valaki nem születik víztõl és Lélektõl, nem mehet be az Isten országába.
\par 6 A mi testtõl született, test az; és a mi Lélektõl született, lélek az.
\par 7 Ne csodáld, hogy azt mondám néked: Szükség néktek újonnan születnetek.
\par 8 A szél fú, a hová akar, és annak zúgását hallod, de nem tudod honnan jõ és hová megy: így van mindenki, a ki Lélektõl született.
\par 9 Felele Nikodémus és monda néki: Mimódon lehetnek ezek?
\par 10 Felele Jézus és monda néki: Te Izráel tanítója vagy, és nem tudod ezeket?
\par 11 Bizony, bizony mondom néked, a mit tudunk, azt mondjuk, és a mit látunk, arról teszünk bizonyságot; és a mi bizonyságtételünket el nem fogadjátok.
\par 12 Ha a földiekrõl szóltam néktek és nem hisztek, mimódon hisztek, ha a mennyeiekrõl szólok néktek?
\par 13 És senki sem ment fel a mennybe, hanemha az, a ki a mennybõl szállott alá, az embernek Fia, a ki a mennyben van.
\par 14 És a miképen felemelte Mózes a kígyót a pusztában, akképen kell az ember Fiának  felemeltetnie.
\par 15 Hogy valaki hiszen õ benne, el ne veszszen, hanem örök élete legyen.
\par 16 Mert úgy szerette Isten e világot, hogy az õ egyszülött Fiát adta, hogy valaki hiszen õ benne, el ne vesszen, hanem örök élete legyen.
\par 17 Mert nem azért küldte az Isten az õ Fiát a világra, hogy kárhoztassa a világot, hanem hogy megtartassék a világ általa.
\par 18 A ki hiszen õ benne, el nem kárhozik; a ki pedig nem hisz, immár elkárhozott, mivelhogy nem hitt az Isten egyszülött Fiának nevében.
\par 19 Ez pedig a kárhoztatás, hogy a világosság e világra jött, és az emberek inkább szerették a sötétséget, mint a világosságot; mert az õ cselekedeteik gonoszak valának.
\par 20 Mert minden, a ki hamisan cselekszik, gyûlöli a világosságot és nem megy a világosságra, hogy az õ cselekedeteit fel ne fedessenek;
\par 21 A ki pedig az igazságot cselekszi, az a világosságra megy, hogy az õ cselekedetei nyilvánvalókká legyenek, hogy Isten szerint való cselekedetek.
\par 22 Ezután elméne Jézus az õ tanítványaival a Júdea földére; és ott idõzék velök, és keresztele.
\par 23 János pedig szintén keresztel vala Énonban, Sálemhez közel, mert ott sok volt a víz. És oda járulának és megkeresztelkedének.
\par 24 Mert János még nem vetteték a tömlöczbe.
\par 25 Vetekedés támada azért a János tanítványai és a judeaiak között a mosakodás felõl.
\par 26 És menének Jánoshoz és mondának néki: Mester! A ki veled vala a Jordánon túl, a kirõl te bizonyságot tettél, ímé az keresztel, és hozzá megy mindenki.
\par 27 Felele János és monda: Az ember semmit sem vehet, hanem ha a mennybõl adatott néki.
\par 28 Ti magatok vagytok a bizonyságaim, hogy megmondtam: Nem vagyok én a Krisztus, hanem hogy õ  elõtte küldettem el.
\par 29 A kinek jegyese van, võlegény az; a võlegény barátja pedig, a ki ott áll és hallja õt, örvendezve örül a võlegény szavának. Ez az én örömem immár betelt.
\par 30 Annak növekednie kell, nékem pedig alább szállanom.
\par 31 A ki felülrõl jött, feljebb való mindenkinél. A ki a földrõl való, földi az és földieket szól; a ki a mennybõl  jött, feljebb való mindenkinél.
\par 32 És arról tesz bizonyságot, a mit látott és hallott; és az õ bizonyságtételét senki sem fogadja be.
\par 33 A ki az õ bizonyságtételét befogadja, az megpecsételte, hogy az  Isten igaz.
\par 34 Mert a kit az Isten küldött, az Isten beszédeit szólja; mivelhogy az Isten nem mérték szerint adja a Lelket.
\par 35 Az Atya szereti a Fiút, és az õ kezébe adott mindent.
\par 36 A ki hisz a Fiúban, örök élete van; a ki pedig nem enged a Fiúnak, nem lát életet, hanem az Isten haragja marad rajta.

\chapter{4}

\par 1 Amint azért megtudta az Úr, hogy a farizeusok meghallották, hogy Jézus több tanítványt szerez és keresztel, mint János,
\par 2 (Jóllehet Jézus maga nem keresztelt, hanem a tanítványai,)
\par 3 Elhagyá Júdeát és elméne ismét Galileába.
\par 4 Samárián kell vala pedig általmennie.
\par 5 Megy vala azért Samáriának Sikár nevû városába, annak a teleknek szomszédjába, a melyet Jákób adott vala az õ fiának, Józsefnek.
\par 6 Ott vala pedig a Jákób forrása. Jézus azért, az utazástól elfáradva, azonmód leült a forráshoz. Mintegy hat óra vala.
\par 7 Jöve egy samáriabeli asszony vizet meríteni; monda néki Jézus: Adj innom!
\par 8 Az õ tanítványai ugyanis elmentek a városba, hogy ennivalót vegyenek.
\par 9 Monda azért néki a samáriai asszony: Hogy kérhetsz inni zsidó létedre én tõlem, a ki samáriai asszony vagyok?! Mert a zsidók nem barátkoznak a samáriaiakkal.
\par 10 Felele Jézus és monda néki: Ha ismernéd az Isten ajándékát, és hogy ki az, a ki ezt mondja néked: Adj innom!; te kérted volna õt, és adott volna néked élõ vizet.
\par 11 Monda néki az asszony: Uram, nincs mivel merítened, és a kút mély: hol vennéd tehát az élõ vizet?
\par 12 Avagy nagyobb vagy-é te a mi atyánknál, Jákóbnál, a ki nékünk adta ezt a kutat, és ebbõl ivott õ is, a fiai is és jószága is?
\par 13 Felele Jézus és monda néki: Mindaz, a ki ebbõl a vízbõl iszik, ismét megszomjúhozik:
\par 14 Valaki pedig abból a vízbõl iszik, a melyet én adok néki, soha örökké meg nem szomjúhozik; hanem az a víz, a melyet én adok néki, örök életre buzgó víznek  kútfeje lesz õ benne.
\par 15 Monda néki az asszony: Uram, add nékem azt a vizet, hogy meg ne szomjúhozzam, és ne jõjjek ide meríteni!
\par 16 Monda néki Jézus: Menj el, hívd a férjedet, és jõjj ide!
\par 17 Felele az asszony és monda: Nincs férjem. Monda néki Jézus: Jól mondád, hogy: Nincs férjem;
\par 18 Mert öt férjed volt, és a mostani nem férjed: ezt igazán mondtad.
\par 19 Monda néki az asszony: Uram, látom, hogy te próféta vagy.
\par 20 A mi atyáink ezen a hegyen imádkoztak; és ti azt mondjátok, hogy Jeruzsálemben van az a hely, a hol imádkozni kell.
\par 21 Monda néki Jézus: Asszony, hidd el nékem, hogy eljõ az óra, a mikor sem nem ezen a hegyen, sem nem Jeruzsálemben imádjátok az Atyát.
\par 22 Ti azt imádjátok, a mit nem ismertek; mi azt imádjuk, amit ismerünk: mert az idvesség  a zsidók közül támadt.
\par 23 De eljõ az óra, és az most vagyon, amikor az igazi imádók lélekben, és igazságban imádják az Atyát: mert az Atya is ilyeneket keres, az õ imádóiul.
\par 24 Az Isten lélek: és a kik õt imádják, szükség, hogy lélekben és igazságban imádják.
\par 25 Monda néki az asszony: Tudom, hogy Messiás jõ (a ki Krisztusnak mondatik); mikor az eljõ, megjelent nékünk mindent.
\par 26 Monda néki Jézus: Én vagyok az, a ki veled beszélek.
\par 27 Eközben megjövének az õ tanítványai; és csodálkozának, hogy asszonnyal beszélt; mindazáltal egyik sem mondá: Mit keresel? vagy: Mit beszélsz vele?
\par 28 Ott hagyá azért az asszony a vedrét, és elméne a városba, és monda az embereknek:
\par 29 Jertek, lássatok egy embert, a ki megmonda nékem mindent, a mit cselekedtem. Nem ez-é a Krisztus?
\par 30 Kimenének azért a városból, és hozzá menének.
\par 31 Aközben pedig kérék õt a tanítványok, mondván: Mester, egyél!
\par 32 Õ pedig monda nékik: Van nékem eledelem, a mit egyem, a mit ti nem tudtok.
\par 33 Mondának azért a tanítványok egymásnak: Hozott-é néki valaki enni?
\par 34 Monda nékik Jézus: Az én eledelem az, hogy annak akaratját cselekedjem, a ki elküldött engem, és az õ dolgát elvégezzem.
\par 35 Ti nem azt mondjátok-é, hogy még négy hónap és eljõ az aratás? Ímé, mondom néktek: Emeljétek fel szemeiteket, és lássátok meg a tájékokat, hogy már fehérek az aratásra.
\par 36 És a ki arat, jutalmat nyer, és az örök életre gyümölcsöt gyûjt; hogy mind a vetõ, mind az arató együtt örvendezzen.
\par 37 Mert ebben az a mondás igaz, hogy más a vetõ, más az arató.
\par 38 Én annak az aratására küldtelek titeket, a mit nem ti munkáltatok; mások munkálták, és ti a mások munkájába állottatok.
\par 39 Abból a városból pedig sokan hivének benne a Samaritánusok közül annak az asszonynak beszédéért, a ki bizonyságot tett vala, hogy: Mindent megmondott nékem, a mit cselekedtem.
\par 40 A mint azért oda mentek hozzá a Samaritánusok, kérék õt, hogy maradjon náluk; és ott marada két napig.
\par 41 És sokkal többen hivének a maga beszédéért,
\par 42 És azt mondják vala az asszonynak, hogy: Nem a te beszédedért hiszünk immár: mert magunk hallottuk, és tudjuk, hogy bizonnyal ez a világ idvezítõje, a Krisztus.
\par 43 Két nap mulva pedig kiméne onnét, és elméne Galileába.
\par 44 Mert Jézus maga tett bizonyságot arról, hogy a prófétának nincs tisztessége a maga hazájában.
\par 45 Mikor azért beméne Galileába, befogadták õt a Galileabeliek, mivelhogy látták vala mindazt, a mit Jeruzsálemben cselekedett az ünnepen; mert õk is elmentek vala az ünnepre.
\par 46 Ismét a galileai Kánába méne azért Jézus, a hol a vizet borrá változtatta. És volt Kapernaumban egy királyi ember, a kinek a fia beteg vala.
\par 47 Mikor ez meghallá, hogy Jézus Júdeából Galileába érkezett, hozzá méne és kéré õt, hogy menjen el és gyógyítsa meg az õ fiát; mert halálán vala.
\par 48 Monda azért néki Jézus: Ha jeleket és csodákat nem láttok, nem hisztek.
\par 49 Monda néki a királyi ember: Uram, jõjj, mielõtt a gyermekem meghal.
\par 50 Monda néki Jézus: Menj el, a te fiad él. És hitt az ember a szónak, a mit Jézus mondott néki, és elment.
\par 51 A mint pedig már megy vala, elébe jövének az õ szolgái, és hírt hozának néki, mondván, hogy: A te fiad él.
\par 52 Megtudakozá azért tõlük az órát, a melyben megkönnyebbedett vala; és mondának néki: Tegnap hét órakor hagyta el õt a láz;
\par 53 Megérté azért az atya, hogy abban az órában, amelyben azt mondá néki a Jézus, hogy: a te fiad él. És hitt õ, és az õ egész háza népe.
\par 54 Ezt ismét második jel gyanánt tevé Jézus, mikor Júdeából Galileába ment.

\chapter{5}

\par 1 Ezek után ünnepök vala a zsidóknak, és felméne Jézus Jeruzsálembe.
\par 2 Van pedig Jeruzsálemben a Juhkapunál egy tó, a melyet héberül Bethesdának neveznek. Öt tornácza van.
\par 3 Ezekben feküvék a betegek, vakok, sánták, aszkórosok nagy sokasága, várva a víznek megmozdulását.
\par 4 Mert idõnként angyal szálla a tóra, és felzavará a vizet: a ki tehát elõször lépett bele a víz felzavarása után, meggyógyult, akárminémû betegségben volt.
\par 5 Vala pedig ott egy ember, a ki harmincnyolcz esztendõt töltött betegségében.
\par 6 Ezt a mint látta Jézus, hogy ott fekszik, és megtudta, hogy már sok idõ óta úgy van; monda néki: Akarsz-é meggyógyulni?
\par 7 Felele néki a beteg: Uram, nincs emberem, hogy a mikor a víz felzavarodik, bevigyen engem a tóba; és mire én oda érek, más lép be elõttem.
\par 8 Monda néki Jézus: Kelj fel, vedd fel a te nyoszolyádat, és járj!
\par 9 És azonnal meggyógyula az ember, és felvevé nyoszolyáját, és jár vala. Aznap pedig szombat vala.
\par 10 Mondának azért a zsidók a meggyógyultnak: Szombat van, nem szabad néked a nyoszolyádat hordanod!
\par 11 Felele nékik: A ki meggyógyított engem, azt mondá nékem: Vedd fel a nyoszolyádat, és járj.
\par 12 Megkérdék azért õt: Ki az az ember, a ki mondá néked: Vedd fel a nyoszolyádat, és járj?
\par 13 A meggyógyult pedig nem tudja vala, hogy ki az; mert Jézus félre vonult, sokaság lévén azon a helyen.
\par 14 Ezek után találkozék vele Jézus a templomban, és monda néki: Ímé meggyógyultál; többé ne vétkezzél, hogy  rosszabbul ne legyen dolgod!
\par 15 Elméne az az ember, és hírül adá a zsidóknak, hogy Jézus az, a ki õt meggyógyította.
\par 16 És e miatt üldözõbe vevék a zsidók Jézust, és meg akarták õt ölni, hogy ezeket mûvelte szombaton.
\par 17 Jézus pedig felele nékik: Az én Atyám mind ez ideig munkálkodik, én is munkálkodom.
\par 18 E miatt aztán még inkább meg akarák õt ölni a zsidók, mivel nem csak a szombatot rontotta meg, hanem az Istent is saját Atyjának mondotta, egyenlõvé tévén magát az Istennel.
\par 19 Felele azért Jézus, és monda nékik: Bizony, bizony mondom néktek: a Fiú semmit sem tehet önmagától, hanem ha látja  cselekedni az Atyát, mert a miket az cselekszik, ugyanazokat hasonlatosképen a Fiú is cselekszi.
\par 20 Mert az Atya szereti a Fiút, és mindent megmutat néki, a miket õ maga cselekszik; és ezeknél nagyobb dolgokat is mutat majd néki, hogy ti csudálkozzatok.
\par 21 Mert a mint az Atya feltámasztja a halottakat és megeleveníti, úgy a Fiú is a kiket akar, megelevenít.
\par 22 Mert az Atya nem ítél senkit, hanem az ítéletet egészen a Fiúnak adta;
\par 23 Hogy mindenki úgy tisztelje a Fiút, miként tisztelik az Atyát. A ki nem tiszteli a Fiút, nem tiszteli az Atyát, a ki elküldte õt.
\par 24 Bizony, bizony mondom néktek, hogy a ki az én beszédemet hallja és hisz annak, a ki engem elbocsátott, örök élete van; és nem megy a kárhozatra, hanem  általment a halálból az életre.
\par 25 Bizony, bizony mondom néktek, hogy eljõ az idõ, és az most vagyon, mikor a halottak hallják az Isten Fiának szavát, és a kik hallják, élnek.
\par 26 Mert a miként az Atyának élete van önmagában, akként adta a Fiúnak is, hogy élete legyen önmagában:
\par 27 És hatalmat ada néki az ítélettételre is, mivelhogy embernek fia.
\par 28 Ne csodálkozzatok ezen: mert eljõ az óra, a melyben mindazok, a kik a koporsókban vannak, meghallják az õ szavát,
\par 29 És kijõnek; a kik a jót cselekedték, az élet feltámadására; a kik pedig a gonoszt mûvelték, a kárhozat feltámadására.
\par 30 Én semmit sem cselekedhetem magamtól; a mint hallok, úgy ítélek, és az én ítéletem igazságos; mert nem a magam akaratát keresem, hanem annak  akaratát, a ki elküldött engem, az Atyáét.
\par 31 Ha én teszek bizonyságot magamról, az én bizonyságtételem nem igaz.
\par 32 Más az, a ki bizonyságot tesz rólam; és tudom, hogy igaz az a bizonyságtétel, a melylyel bizonyságot tesz rólam.
\par 33 Ti elküldtetek Jánoshoz, és bizonyságot tett az igazságról.
\par 34 De én nem embertõl nyerem a bizonyságtételt; hanem ezeket azért mondom, hogy ti megtartassatok.
\par 35 Õ az égõ és fénylõ szövétnek vala, ti pedig csak egy ideig akartatok örvendezni az õ világosságában.
\par 36 De nékem nagyobb bizonyságom van a Jánosénál: mert azok a dolgok, a melyeket rám bízott az Atya, hogy elvégezzem azokat, azok a dolgok, a melyeket én cselekszem, tesznek bizonyságot rólam, hogy az Atya küldött engem.
\par 37 A ki elküldött engem, maga az Atya is bizonyságot tett rólam. Sem hangját nem hallottátok soha, sem ábrázatát  nem láttátok.
\par 38 Az õ ígéje sincs maradandóan bennetek: mert a kit õ elküldött, ti annak nem hisztek.
\par 39 Tudakozzátok az írásokat, mert azt hiszitek, hogy azokban van a ti örök életetek; és ezek azok, a melyek bizonyságot  tesznek rólam;
\par 40 És nem akartok hozzám jõni, hogy életetek legyen!
\par 41 Dicsõséget emberektõl nem nyerek.
\par 42 De ismerlek benneteket, hogy az Istennek szeretete nincs bennetek:
\par 43 Én az én Atyám nevében jöttem, és nem fogadtatok be engem; ha más jõne a maga nevében, azt befogadnátok.
\par 44 Mimódon hihettek ti, a kik egymástól nyertek dicsõséget, és azt a dicsõséget, a mely az egy Istentõl van, nem keresitek?
\par 45 Ne állítsátok, hogy én vádollak majd benneteket az Atyánál; van a ki vádol titeket, Mózes, a kiben ti reménykedtetek.
\par 46 Mert ha hinnétek Mózesnek, nékem is hinnétek; mert én rólam írt õ.
\par 47 Ha pedig az õ írásainak nem hisztek, mimódon hisztek az én beszédeimnek?

\chapter{6}

\par 1 Ezek után elméne Jézus a galileai tengeren, a Tiberiáson túl.
\par 2 És nagy sokaság követé õt, mivelhogy látják vala az õ csodatételeit, a melyeket cselekszik vala a betegeken.
\par 3 Felméne pedig Jézus a hegyre, és leüle ott a tanítványaival.
\par 4 Közel vala pedig husvét, a zsidók ünnepe.
\par 5 Mikor azért felemelé Jézus a szemeit, és látá, hogy nagy sokaság jõ hozzá, monda Filepnek: Honnan vegyünk kenyeret, hogy ehessenek ezek?
\par 6 Ezt pedig azért mondá, hogy próbára tegye õt; mert õ maga tudta, mit akar vala cselekedni.
\par 7 Felele néki Filep: Kétszáz dénár árú kenyér nem elég ezeknek, hogy mindenikök kapjon valami keveset.
\par 8 Monda néki egy az õ tanítványai közül, András, a Simon Péter testvére:
\par 9 Van itt egy gyermek, a kinek van öt árpa kenyere és két hala; de mi az ennyinek?
\par 10 Jézus pedig monda: Ültessétek le az embereket. Nagy fû vala pedig azon a helyen. Leülének azért a férfiak, számszerint mintegy ötezeren.
\par 11 Jézus pedig vevé a kenyereket, és hálát adván, adta a tanítványoknak, a tanítványok pedig a leülteknek; hasonlóképen a halakból is, a mennyit akarnak vala.
\par 12 A mint pedig betelének, monda az õ tanítványainak: Szedjétek össze a megmaradt darabokat, hogy semmi el ne veszszen.
\par 13 Összeszedék azért, és megtöltének tizenkét kosarat az öt árpa kenyérbõl való darabokkal, a melyek megmaradtak vala az evõk után.
\par 14 Az emberek azért látva a jelt, a melyet Jézus tõn, mondának: Bizonnyal ez ama próféta, a ki eljövendõ vala a világra.
\par 15 Jézus azért, a mint észrevevé, hogy jõni akarnak és õt elragadni, hogy királylyá tegyék, ismét elvonula egymaga a hegyre.
\par 16 Mikor pedig estveledék, lemenének az õ tanítványai a tengerhez,
\par 17 És beszállva a hajóba, mennek vala a tengeren túl Kapernaumba. És már sötétség volt, és még nem ment vala hozzájuk Jézus.
\par 18 És a tenger a nagy szél fúvása miatt háborog vala.
\par 19 Mikor azért huszonöt, vagy harmincz futamatnyira beevezének, megláták Jézust, a mint jár vala a tengeren és a hajóhoz közeledik vala: és megrémülének.
\par 20 Õ pedig monda nékik: Én vagyok, ne féljetek!
\par 21 Be akarák azért õt venni a hajóba: és a hajó azonnal ama földnél vala, a melyre menének.
\par 22 Másnap a sokaság, a mely a tengeren túl állott vala, látva, hogy nem vala ott más hajó, csak az az egy, a melybe a Jézus tanítványai szállottak, és hogy Jézus nem ment be az õ tanítványaival a hajóba, hanem csak az õ tanítványai mentek el,
\par 23 De jöttek más hajók Tiberiásból közel ahhoz a helyhez, a hol a kenyeret ették, miután hálákat adott az Úr:
\par 24 Mikor azért látta a sokaság, hogy sem Jézus, sem a tanítványai nincsenek ott, beszállának õk is a hajókba, és elmenének Kapernaumba, keresvén Jézust.
\par 25 És megtalálván õt a tengeren túl, mondának néki: Mester, mikor jöttél ide?
\par 26 Felele nékik Jézus és monda: Bizony, bizony mondom néktek: nem azért kerestek engem, hogy jeleket láttatok, hanem azért, mert ettetek ama kenyerekbõl, és jóllaktatok.
\par 27 Munkálkodjatok ne az eledelért, a mely elvész, hanem az eledelért, a mely megmarad az örök életre, a melyet az embernek Fia ád majd néktek; mert õt az Atya pecsételte el,  az Isten.
\par 28 Mondának azért néki: Mit csináljunk, hogy az Isten dolgait cselekedjük?
\par 29 Felele Jázus és monda nékik: Az az Isten dolga, hogy higyjetek abban, a kit õ küldött.
\par 30 Mondának azért néki: Micsoda jelt mutatsz tehát te, hogy lássuk és higyjünk néked? Mit mûvelsz?
\par 31 A mi atyáink a mannát ették a pusztában; a mint meg van írva: Mennyei kenyeret adott vala enniök.
\par 32 Monda azért nékik Jézus: Bizony, bizony mondom néktek: nem Mózes adta néktek a mennyei kenyeret, hanem az én Atyám adja majd néktek az igazi mennyei kenyeret.
\par 33 Mert az az Istennek kenyere, a mely mennybõl száll alá, és életet ád a világnak.
\par 34 Mondának azért néki: Uram, mindenkor add nékünk ezt a kenyeret!
\par 35 Jézus pedig monda nékik: Én vagyok az életnek ama kenyere; a ki hozzám jõ, semmiképen meg nem éhezik, és a ki hisz bennem, meg nem szomjúhozik soha.
\par 36 De mondám néktek, hogy noha láttatok is engem, még sem hisztek.
\par 37 Minden, a mit nékem ád az Atya, én hozzám jõ; és azt, a ki hozzám jõ, semmiképen ki nem vetem.
\par 38 Mert azért szállottam le a mennybõl, hogy ne a magam akaratát cselekedjem, hanem annak  akaratát, a ki elküldött engem.
\par 39 Az pedig az Atyának akarata, a ki elküldött engem, hogy a mit nékem adott, abból semmit el ne veszítsek, hanem feltámaszszam azt az utolsó napon.
\par 40 Az pedig annak az akarata, a ki elküldött engem, hogy mindaz, a ki látja a Fiút és hisz õ benne, örök élete legyen; és én feltámaszszam azt azt utolsó napon.
\par 41 Zúgolódának azért a zsidók õ ellene, hogy azt mondá: Én vagyok az a kenyér, a mely a mennybõl szállott alá.
\par 42 És mondának: Nem ez-é Jézus, a József fia, a kinek mi ismerjük atyját és anyját? mimódon mondja hát ez, hogy: A mennybõl szállottam alá?
\par 43 Felele azért Jézus és monda nékik: Ne zúgolódjatok egymás között!
\par 44 Senki sem jöhet én hozzám, hanemha az Atya vonja azt, a ki elküldött engem; én pedig feltámasztom azt az utolsó napon.
\par 45 Meg van írva a prófétáknál: És mindnyájan Istentõl tanítottak lesznek. Valaki azért az Atyától hallott, és tanult, én hozzám jõ.
\par 46 Nem hogy az Atyát valaki látta, csak az, a ki Istentõl van, az látta az Atyát.
\par 47 Bizony, bizony mondom néktek: A ki én bennem hisz, örök élete van annak.
\par 48 Én vagyok az életnek kenyere.
\par 49 A ti atyáitok a mannát ették a pusztában, és meghaltak.
\par 50 Ez az a kenyér, a mely a mennybõl szállott alá, hogy kiki egyék belõle és meg ne haljon.
\par 51 Én vagyok amaz élõ kenyér, a mely a mennybõl szállott alá; ha valaki eszik e kenyérbõl, él örökké. És az a kenyér pedig, a melyet én adok, az én testem, a melyet én adok a világ életéért.
\par 52 Tusakodának azért a zsidók egymás között, mondván: Mimódon adhatja ez nekünk a testét, hogy azt együk?
\par 53 Monda azért nékik Jézus: Bizony, bizony mondom néktek: Ha nem eszitek az ember Fiának testét és nem iszszátok az õ vérét, nincs élet bennetek.
\par 54 A ki eszi az én testemet és iszsza az én véremet, örök élete van annak, és én feltámasztom azt az utolsó napon.
\par 55 Mert az én testem bizony étel és az én vérem bizony ital.
\par 56 A ki eszi az én testemet és iszsza az én véremet, az én bennem lakozik és én is abban.
\par 57 A miként elküldött engem amaz élõ Atya, és én az Atya által élek: akként az is, a ki engem eszik, él én általam.
\par 58 Ez az a kenyér, a mely a mennybõl szállott alá; nem úgy, a mint a ti atyáitok evék a mannát és meghalának: a ki ezt a kenyeret eszi, él örökké.
\par 59 Ezeket mondá a zsinagógában, a mikor tanít vala Kapernaumban.
\par 60 Sokan azért, a kik hallák ezeket az õ tanítványai közül, mondának: Kemény beszéd ez; ki hallgathatja õt?
\par 61 Tudván pedig Jézus õ magában, hogy e miatt zúgolódnak az õ tanítványai, monda nékik: Titeket ez megbotránkoztat?
\par 62 Hát ha meglátjátok az embernek Fiát felszállani oda, a hol elébb vala?!
\par 63 A lélek az, a mi megelevenít, a test nem használ semmit: a beszédek, a melyeket én szólok néktek, lélek és élet.
\par 64 De vannak némelyek közöttetek, a kik nem hisznek. Mert eleitõl fogva tudta Jézus, kik azok, a kik nem hisznek, és ki az,  a ki elárulja õt.
\par 65 És monda: Azért mondtam néktek, hogy senki sem jöhet én hozzám, hanemha az én Atyámtól van megadva néki.
\par 66 Ettõl fogva sokan visszavonulának az õ tanítványai közül és nem járnak vala többé õ vele.
\par 67 Monda azért Jézus a tizenkettõnek: Vajjon ti is el akartok-é menni?
\par 68 Felele néki Simon Péter: Uram, kihez mehetnénk? Örök életnek beszéde van te nálad.
\par 69 És mi elhittük és megismertük, hogy te vagy a Krisztus, az élõ Istennek Fia.
\par 70 Felele nékik Jézus: Nem én választottalak-é ki titeket, a tizenkettõt? és egy közületek ördög.
\par 71 Érette pedig Júdás Iskáriótest, Simon fiát, mert ez akarta õt elárulni, noha egy volt a tizenkettõ közül.

\chapter{7}

\par 1 És ezek után Galileában jár vala Jézus; mert nem akar vala Júdeában járni, mivelhogy azon igyekezének a Júdeabeliek, hogy õt megöljék.
\par 2 Közel vala pedig a zsidók ünnepe, a sátoros ünnep.
\par 3 Mondának azért néki az õ atyjafiai: Menj el innen, és térj Júdeába, hogy a te tanítványaid is lássák a te dolgaidat, a melyeket cselekszel.
\par 4 Mert senki sem cselekszik titkon semmit, a ki maga ismeretessé akar lenni. Ha ilyeneket cselekszel, mutasd meg magadat a világnak.
\par 5 Mert az õ atyjafiai sem hivének benne.
\par 6 Monda azért nékik Jézus: Az én idõm még nincs itt; a ti idõtök pedig mindig készen van.
\par 7 Titeket nem gyûlölhet a világ, de engem gyûlöl; mert én bizonyságot teszek felõle, hogy az õ  cselekedetei gonoszak.
\par 8 Ti menjetek fel erre az ünnepre: én még nem megyek fel erre az ünnepre; mert az én idõm még nem tölt be.
\par 9 Ezeket mondván pedig nékik, marada Galileában.
\par 10 A mint pedig felmenének az õ atyjafiai, akkor õ is felméne az ünnepre, nem nyilvánosan, hanem mintegy titkon.
\par 11 A zsidók azért keresik vala õt az ünnepen, és mondának: Hol van õ?
\par 12 És a sokaságban nagy zúgás vala õ miatta. Némelyek azt mondják vala, hogy jó ember; mások pedig azt mondják vala: Nem, hanem a népnek hitetõje.
\par 13 Mindamellett senki sem beszélt vala nyiltan õ felõle a zsidóktól való félelem miatt.
\par 14 Már-már az ünnep közepén azonban felméne Jézus a templomba, és tanít vala.
\par 15 És csodálkozának a zsidók, mondván: Mimódon tudja ez az írásokat, holott nem tanulta?!
\par 16 Felele nékik Jézus és monda: Az én tudományom nem az enyém, hanem azé, a ki küldött engem.
\par 17 Ha valaki cselekedni akarja az õ akaratát, megismerheti e tudományról, vajjon Istentõl van-é, vagy én magamtól szólok?
\par 18 A ki magától szól, a maga dicsõségét keresi; a ki pedig annak dicsõségét keresi, a ki küldte õt, igaz az, és nincs abban hamisság.
\par 19 Nem Mózes adta-é néktek a törvényt? és senki sem teljesíti közületek a törvényt. Miért akartok engem megölni?
\par 20 Felele a sokaság és monda: Ördög van benned. Ki akar téged megölni?
\par 21 Felele Jézus és monda nékik: Egy dolgot cselekvém, és mindnyájan csodáljátok.
\par 22 Azért Mózes adta néktek a körülmetélkedést (nem mintha Mózestõl való volna, hanem az atyáktól): és szombaton körülmetélitek az embert.
\par 23 Ha körülmetélhetõ az ember szombaton, hogy a Mózes törvénye meg ne romoljon; én rám haragusztok-é, hogy egy embert egészen meggyógyítottam szombaton?
\par 24 Ne ítéljetek a látszat után, hanem igaz ítélettel ítéljetek!
\par 25 Mondának azért némelyek a jeruzsálemiek közül: Nem ez-é az, a kit meg akarnak ölni?
\par 26 És ímé nyiltan szól, és semmit sem szólnak néki. Talán bizony megismerték a fõemberek, hogy bizony ez a Krisztus?
\par 27 De jól tudjuk, honnan való ez; mikor pedig eljõ a Krisztus, senki sem tudja, honnan való.
\par 28 Kiálta azért Jézus a templomban, tanítván és mondván: Mind engem ismertek, minda azt tudjátok, honnan való vagyok; és én magamtól nem jöttem, de igaz  az, a ki engem elküldött, a kit ti nem ismertek.
\par 29 Én azonban ismerem õt, mert õ tõle vagyok, és õ küldött engem.
\par 30 Akarják vala azért õt megfogni; de senki sem veté reá a kezét, mert nem jött még el az õ órája.
\par 31 A sokaság közül pedig sokan hivének õ benne; és azt mondják vala, hogy: A Krisztus mikor eljõ, tehet-é majd több csodát azoknál, a melyeket ez tett?
\par 32 Meghallák a farizeusok, a mint a sokaság ezeket suttogja vala felõle; és szolgákat küldének a farizeusok és a fõpapok, hogy fogják meg õt.
\par 33 Monda azért nékik Jézus: Egy kevés ideig még veletek vagyok, és majd ahhoz megyek, a ki elküldött engem.
\par 34 Kerestek majd engem, és nem találtok meg, és a hol én vagyok, ti nem jöhettek oda.
\par 35 Mondának azért a zsidók magok között: Hová akar ez menni, hogy mi majd nem találjuk meg õt? Vajjon a görögök közé szóródottakhoz akar-é menni, és a görögöket tanítani?
\par 36 Micsoda beszéd ez, a melyet monda: Kerestek majd engem, és nem találtok meg; és a hol én vagyok, ti nem jöhettek oda?
\par 37 Az ünnep utolsó nagy napján pedig felálla Jézus és kiálta, mondván: Ha valaki  szomjúhozik, jõjjön én hozzám, és igyék.
\par 38 A ki hisz én bennem, a mint az írás mondotta, élõ  víznek folyamai ömlenek annak belsejébõl.
\par 39 Ezt pedig mondja vala a Lélekrõl, a melyet veendõk valának az õ benne hívõk: mert még nem vala Szent Lélek; mivelhogy Jézus még nem dicsõítteték meg.
\par 40 Sokan azért a sokaság közül, a mint hallák e beszédet, ezt mondják vala: Bizonynyal ez ama Próféta.
\par 41 Némelyek mondának: Ez a Krisztus. Mások pedig mondának: Csak nem Galileából jön el a Krisztus?
\par 42 Nem az írás mondta-é, hogy a Dávid magvából, és Bethlehembõl, ama városból jön el a Krisztus, a hol Dávid vala?
\par 43 Hasonlás lõn azért õ miatta a sokaságban.
\par 44 Némelyek pedig közûlök akarják vala õt megfogni, de senki sem veté reá a kezét.
\par 45 Elmenének azért a szolgák a fõpaphoz és farizeusokhoz; és mondának azok õ nékik: Miért nem hoztátok el õt?
\par 46 Felelének a szolgák: Soha ember úgy nem szólott, mint ez az ember!
\par 47 Felelének azért nékik a farizeusok: Vajjon ti is el vagytok-é hitetve?
\par 48 Vajjon a fõemberek vagy a farizeusok közül hitt-é benne valaki?
\par 49 De ez a sokaság, a mely nem ismeri a törvényt, átkozott!
\par 50 Monda nékik Nikodémus, a ki éjjel ment vala õ hozzá, a ki egy vala azok közül:
\par 51 Vajjon a mi törvényünk kárhoztatja-é az embert, ha elõbb ki nem hallgatja és nem tudja, hogy mit cselekszik?
\par 52 Felelének és mondának néki: Vajjon te is Galileus vagy-é? Tudakozódjál és lásd meg, hogy Galileából nem támadt próféta.
\par 53 És mindnyájan haza menének.

\chapter{8}

\par 1 Jézus pedig elméne az Olajfák hegyére.
\par 2 Jó reggel azonban ismét ott vala a templomban, és az egész nép hozzá méne; és leülvén, tanítja vala õket.
\par 3 Az írástudók és a farizeusok pedig egy asszonyt vivének hozzá, a kit házasságtörésen kaptak vala, és a középre állítván azt,
\par 4 Mondának néki: Mester, ez az asszony tetten kapatott, mint házasságtörõ.
\par 5 A törvényben pedig megparancsolta nékünk Mózes, hogy az ilyenek köveztessenek meg: te azért mit mondasz?
\par 6 Ezt pedig azért mondák, hogy megkísértsék õt, hogy legyen õt mivel vádolniok. Jézus pedig lehajolván, az ujjával ír vala a földre.
\par 7 De mikor szorgalmazva kérdezék õt, felegyenesedve monda nékik: A ki közületek nem bûnös, az vesse rá elõször a követ.
\par 8 És újra lehajolván, írt vala a földre.
\par 9 Azok pedig ezt hallván és a lelkiismeret által vádoltatván, egymásután kimenének a vénektõl kezdve mind az utolsóig; és egyedül Jézus maradt vala és az asszony a középen állva.
\par 10 Mikor pedig Jézus felegyenesedék és senkit sem láta az asszonyon kívül, monda néki: Asszony, hol vannak azok a te vádlóid? Senki sem kárhoztatott-é téged?
\par 11 Az pedig monda: Senki, Uram! Jézus pedig monda néki: Én sem kárhoztatlak: eredj el és többé ne vétkezzél!
\par 12 Ismét szóla azért hozzájok Jézus, mondván: Én vagyok a világ világossága: a ki engem követ, nem járhat a sötétségben, hanem övé lesz az életnek világossága.
\par 13 Mondának azért néki a farizeusok: Te magadról teszel bizonyságot; a te bizonyságtételed nem igaz.
\par 14 Felele Jézus és monda nékik: Ha magam teszek is bizonyságot magamról, az én bizonyságtételem igaz; mert tudom honnan jöttem és hová megyek; ti pedig nem tudjátok honnan jövök  és hová megyek.
\par 15 Ti test szerint ítéltek, én nem ítélek senkit.
\par 16 De ha ítélek is én, az én ítéletem igaz; mert én nem egyedül vagyok, hanem én és az Atya, a ki küldött engem.
\par 17 A ti törvényetekben is meg van pedig írva, hogy két ember bizonyságtétele igaz.
\par 18 Én vagyok a ki bizonyságot teszek magamról, és bizonyságot tesz rólam az Atya, a ki küldött engem.
\par 19 Mondának azért néki: Hol van a te Atyád? Felele Jézus: Sem engem nem ismertek, sem az én Atyámat; ha engem ismernétek, az én Atyámat is ismernétek.
\par 20 Ezeket a beszédeket mondá Jézus a kincstartó helyen, a mikor tanít vala a templomban; és senki sem fogta meg õt, mert még nem jött el az õ órája.
\par 21 Ismét monda azért nékik Jézus: Én elmegyek, és kerestek majd engem, és a ti bûneitekben fogtok meghalni: a hová én megyek, ti nem jöhettek oda.
\par 22 Mondának azért a zsidók: Avagy megöli-é magát, hogy azt mondja: A hová én megyek, ti nem jöhettek oda?
\par 23 És monda nékik: Ti innét alól valók vagytok, én önnét felül való vagyok; ti e világból valók vagytok, én nem vagyok e világból való.
\par 24 Azért mondám néktek, hogy a ti bûneitekben haltok meg: mert ha nem hiszitek, hogy én vagyok, meghaltok a ti bûneitekben.
\par 25 Mondának azért néki: Ki vagy te? És monda nékik Jézus: A mit eleitõl fogva mondok is néktek.
\par 26 Sok beszélni és ítélni valóm van felõletek: de igaz az, a ki küldött engem; és én azokat beszélem a világnak, a miket tõle hallottam vala.
\par 27 Nem vevék észre, hogy az Atyáról szól vala nékik.
\par 28 Monda azért nékik Jézus: Mikor felemelitek az embernek Fiát, akkor megismeritek, hogy én vagyok és semmit sem cselekszem magamtól, hanem a mint az Atya tanított engem, úgy szólok.
\par 29 És a ki küldött engem, én velem van. Nem hagyott engem az Atya egyedül, mert én mindenkor azokat cselekszem, a melyek néki kedvesek.
\par 30 A mikor ezeket mondá, sokan hivének õ benne.
\par 31 Monda azért Jézus a benne hívõ zsidóknak: Ha ti megmaradtok az én beszédemben, bizonynyal az én tanítványaim vagytok;
\par 32 És megismeritek az igazságot, és az igazság szabadokká tesz titeket.
\par 33 Felelének néki: Ábrahám magva vagyunk, és nem szolgáltunk soha senkinek: mimódon mondod te, hogy szabadokká lesztek?
\par 34 Felele nékik Jézus: Bizony, bizony mondom néktek, hogy mindaz, a ki bûnt cselekszik, szolgája a bûnnek.
\par 35 A szolga pedig nem marad mindörökké a házban: a Fiú marad ott mindörökké.
\par 36 Azért ha a Fiú megszabadít titeket, valósággal szabadok lesztek.
\par 37 Tudom, hogy Ábrahám magva vagytok; de meg akartok engem ölni, mert az én beszédemnek nincs helye nálatok.
\par 38 Én azt beszélem, a mit az én Atyámnál láttam; ti  is azt cselekszitek azért, a mit a ti atyátoknál láttatok.
\par 39 Felelének és mondának néki: A mi atyánk Ábrahám. Monda nékik Jézus: Ha Ábrahám gyermekei volnátok, az Ábrahám dolgait cselekednétek.
\par 40 Ámde meg akartok engem ölni, olyan embert, a ki az igazságot beszéltem néktek, a melyet az Istentõl hallottam. Ábrahám ezt nem cselekedte.
\par 41 Ti a ti atyátok dolgait cselekszitek. Mondának azért néki: Mi nem paráznaságból születtünk; egy atyánk van, az Isten.
\par 42 Monda azért nékik Jézus: Ha az Isten volna a ti atyátok, szeretnétek engem: mert én az Istentõl származtam és jöttem; mert nem is magamtól jöttem, hanem õ küldött engem.
\par 43 Miért nem értitek az én beszédemet? Mert nem hallgatjátok az én szómat.
\par 44 Ti az ördög atyától valók vagytok, és a ti atyátok kívánságait akarjátok teljesíteni. Az emberölõ volt kezdettõl fogva, és nem állott  meg az igazságban, mert nincsen õ benne igazság. Mikor hazugságot szól, a sajátjából szól; mert hazug és hazugság atyja.
\par 45 Mivelhogy pedig én igazságot szólok, nem hisztek nékem.
\par 46 Ki vádol engem közületek bûnnel? Ha pedig igazságot szólok: miért nem hisztek ti nékem?
\par 47 A ki az Istentõl van, hallgatja az Isten beszédeit; azért nem hallgatjátok ti, mert nem vagytok az Istentõl valók.
\par 48 Felelének azért a zsidók és mondának néki: Nem jól mondjuk-é mi, hogy te Samaritánus vagy, és ördög van benned?
\par 49 Felele Jézus: Nincs én bennem ördög; hanem tisztelem az én Atyámat, és ti gyaláztok engem.
\par 50 Pedig én nem keresem az én dicsõségemet: van a ki keresi és megítéli.
\par 51 Bizony, bizony mondom néktek, ha valaki megtartja az én beszédemet, nem lát halált soha örökké.
\par 52 Mondának azért néki a zsidók: Most értettük meg, hogy ördög van benned. Ábrahám meghalt, a próféták is; és te azt mondod: Ha valaki megtartja az én beszédemet, nem kóstol halált örökké.
\par 53 Avagy nagyobb vagy-é te a mi atyánknál Ábrahámnál, a ki meghalt? A próféták is meghaltak: kinek állítod te magadat?
\par 54 Felele Jézus: Ha én dicsõítem magamat, az én dicsõségem semmi: az én Atyám az, a ki dicsõít engem, a kirõl ti azt mondjátok, hogy a ti Istenetek,
\par 55 És nem ismeritek õt: de én ismerem õt; és ha azt mondom, hogy nem ismerem õt, hozzátok hasonlóvá, hazuggá leszek: de ismerem õt, és az õ beszédét megtartom.
\par 56 Ábrahám a ti atyátok örvendezett, hogy meglátja az én napomat; látta is, és örült.
\par 57 Mondának azért néki a zsidók: Még ötven esztendõs nem vagy, és Ábrahámot láttad?
\par 58 Monda nékik Jézus: Bizony, bizony mondom néktek: Mielõtt Ábrahám lett, én vagyok.
\par 59 Köveket ragadának azért, hogy reá hajigálják; Jézus pedig elrejtõzködék, és kiméne a templomból, átmenvén közöttük; és ilyen módon eltávozék.

\chapter{9}

\par 1 És a mint eltávozék, láta egy embert, a ki születésétõl kezdve vak vala.
\par 2 És kérdezék õt a tanítványai, mondván: Mester, ki vétkezett, ez-é vagy ennek szülei, hogy vakon született?
\par 3 Felele Jézus: Sem ez nem vétkezett, sem ennek szülei; hanem, hogy nyilvánvalókká legyenek benne az Isten dolgai.
\par 4 Nékem cselekednem kell annak dolgait, a ki elküldött engem, a míg nappal van: eljõ az éjszaka, mikor senki sem munkálkodhatik.
\par 5 Míg e világon vagyok, e világ világossága vagyok.
\par 6 Ezeket mondván, a földre köpe, és az õ nyálából sárt csinála, és rákené a sarat a vak szemeire,
\par 7 És monda néki: Menj el, mosakodjál meg a Siloám tavában (a mi azt jelenti: Küldött). Elméne azért és megmosakodék, és megjöve látva.
\par 8 A szomszédok azért, és a kik az elõtt látták azt, hogy vak vala, mondának: Nem ez-é az, a ki itt szokott ülni és koldulni?
\par 9 Némelyek azt mondák, hogy: Ez az; mások pedig, hogy: Hasonlít hozzá. Õ azt mondá, hogy: Én vagyok az.
\par 10 Mondának azért néki: Mimódon nyiltak meg a te szemeid?
\par 11 Felele az és monda: Egy ember, a kit Jézusnak mondanak, sarat készíte és rákené a szemeimre, és monda nékem: Menj el a Siloám tavára és mosódjál meg; miután pedig elmenék és megmosakodám, megjöve látásom.
\par 12 Mondának azért néki: Hol van az? Monda: Nem tudom.
\par 13 Vivék õt, a ki elõbb még vak volt, a farizeusokhoz.
\par 14 Mikor pedig Jézus a sarat csinálá és felnyitá ennek szemeit, szombat vala.
\par 15 Szintén a farizeusok is megkérdezék azért õt, mimódon jött meg a látása? Õ pedig monda nékik: Sarat tõn szemeimre, és megmosakodám, és látok.
\par 16 Mondának azért némelyek a farizeusok közül: Ez az ember nincsen Istentõl, mert nem tartja meg a szombatot. Mások mondának: Mimódon tehet bûnös ember ilyen jeleket? És hasonlás lõn közöttük.
\par 17 Újra mondának a vaknak: Te mit szólsz õ róla, hogy megnyitá a szemeidet? Õ pedig monda: Hogy próféta.
\par 18 Nem hivék azért a zsidók róla, hogy vak vala és megjöve a látása, mígnem elõhívák annak szüleit, a kinek megjöve a látása,
\par 19 És megkérdezék azokat, mondván: Ez a ti fiatok, a kirõl azt mondjátok, hogy vakon született? mimódon lát hát most?
\par 20 Felelének nékik annak szülei és mondának: Tudjuk, hogy ez a mi fiunk, és hogy vakon született:
\par 21 De mimódon lát most, nem tudjuk; vagy ki nyitotta meg a szemeit, mi nem tudjuk: elég idõs már õ; õt kérdezzétek; õ beszéljen magáról.
\par 22 Ezeket mondák annak szülei, mivelhogy félnek vala a zsidóktól: mert megegyeztek már a zsidók, hogy ha valaki Krisztusnak vallja õt, rekesztessék ki a gyülekezetbõl.
\par 23 Ezért mondák annak szülei, hogy: Elég idõs, õt kérdezzétek.
\par 24 Másodszor is szólíták azért az embert, a ki vak vala, és mondának néki: Adj dicsõséget az Istennek; mi tudjuk, hogy ez az ember bûnös.
\par 25 Felele azért az és monda: Ha bûnös-é, nem tudom: egyet tudok, hogy noha vak voltam, most látok.
\par 26 Újra mondák pedig néki: Mit csinált veled? Mimódon nyitotta meg a szemeidet?
\par 27 Felele nékik: Már mondám néktek és nem hallátok: miért akarjátok újra hallani? avagy ti is az õ tanítványai akartok lenni?
\par 28 Szidalmazák azért õt és mondának: Te vagy annak a tanítványa; mi pedig a Mózes tanítványai vagyunk.
\par 29 Mi tudjuk, hogy Mózessel beszélt az Isten: errõl pedig azt sem tudjuk, honnan való.
\par 30 Felele az ember és monda nékik: Bizony csodálatos az, hogy ti nem tudjátok honnan való, és az én szemeimet megnyitotta.
\par 31 Pedig tudjuk, hogy az Isten nem hallgatja meg a bûnösöket; hanem ha valaki istenfélõ, és az õ akaratát cselekszi, azt hallgatja meg.
\par 32 Öröktõl fogva nem hallaték, hogy vakon szülöttnek szemeit valaki megnyitotta volna.
\par 33 Ha ez nem Istentõl volna, semmit sem cselekedhetnék.
\par 34 Felelének és mondának néki: Te mindenestõl bûnben születtél, és te tanítasz minket? És kiveték õt.
\par 35 Meghallá Jézus, hogy kiveték azt; és találkozván vele, monda néki: Hiszel-é te az Isten Fiában?
\par 36 Felele az és monda: Ki az, Uram, hogy higyjek benne?
\par 37 Monda pedig néki Jézus: Láttad is õt, és a ki beszél veled, az az.
\par 38 Az pedig monda: Hiszek, Uram. És imádá Õt.
\par 39 És monda Jézus: Ítélet végett jöttem én e világra, hogy a kik nem látnak, lássanak; és a kik látnak, vakok legyenek.
\par 40 És hallák ezeket némely farizeusok, a kik vele valának, és mondának néki: Avagy mi is vakok vagyunk-é?
\par 41 Mondá nékik Jézus: Ha vakok volnátok, nem volna bûnötök; ámde azt mondjátok, hogy látunk: azért a ti bûnötök megmarad.

\chapter{10}

\par 1 Bizony, bizony mondom néktek: A ki nem az ajtón megy be a juhok aklába, hanem másunnan hág be, tolvaj az és rabló.
\par 2 A ki pedig az ajtón megy be, a juhok pásztora az.
\par 3 Ennek az ajtónálló ajtót nyit; és a juhok hallgatnak annak szavára; és a maga juhait nevökön szólítja, és kivezeti õket.
\par 4 És mikor kiereszti az õ juhait, elõttök megy; és a juhok követik õt, mert ismerik az õ hangját.
\par 5 Idegent pedig nem követnek, hanem elfutnak attól: mert nem ismerik az idegenek hangját.
\par 6 Ezt a példázatot mondá nékik Jézus; de õk nem értették, mi az, a mit szól vala nékik.
\par 7 Újra monda azért nékik Jézus: Bizony, bizony mondom néktek, hogy én vagyok a juhoknak ajtaja.
\par 8 Mindazok, a kik elõttem jöttek, tolvajok és rablók: de nem hallgattak rájok a juhok.
\par 9 Én vagyok az ajtó: ha valaki én rajtam megy be, megtartatik és bejár és kijár majd, és legelõt talál.
\par 10 A tolvaj nem egyébért jõ, hanem hogy lopjon és öljön és pusztítson; én azért jöttem, hogy életök legyen, és bõvölködjenek.
\par 11 Én vagyok a jó pásztor: a jó pásztor  életét adja a juhokért.
\par 12 A béres pedig és a ki nem pásztor, a kinek a juhok nem tulajdonai, látja a farkast jõni, és elhagyja a juhokat, és elfut: és a farkas elragadozza azokat, és elszéleszti a juhokat.
\par 13 A béres pedig azért fut el, mert béres, és nincs gondja a juhokra.
\par 14 Én vagyok a jó pásztor; és ismerem az enyéimet, és engem is ismernek az enyéim,
\par 15 A miként ismer engem az Atya, és én is ismerem az Atyát; és életemet adom a juhokért.
\par 16 Más juhaim is vannak nékem, a melyek nem ebbõl az akolból valók: azokat is elõ kell hoznom, és hallgatnak majd az én szómra; és lészen egy akol és egy pásztor.
\par 17 Azért szeret engem az Atya, mert én leteszem az én életemet, hogy újra felvegyem azt.
\par 18 Senki sem veszi azt el én tõlem, hanem én teszem le azt én magamtól. Van hatalmam letenni azt, és van hatalmam ismét felvenni azt. Ezt a parancsolatot vettem az én Atyámtól.
\par 19 Újra hasonlás lõn a zsidók között e beszédek miatt.
\par 20 És sokan mondják vala közülök: Ördög van benne és bolondozik, mit hallgattok reá?
\par 21 Mások mondának: Ezek nem ördöngõsnek beszédei. Vajjon az ördög megnyithatja-é a vakok szemeit?
\par 22 Lõn pedig Jeruzsálemben a templomszentelés ünnepe: és tél vala;
\par 23 És Jézus a templomban, a Salamon tornáczában jár vala.
\par 24 Körülvevék azért õt a zsidók, és mondának néki: Meddig tartasz még bizonytalanságban bennünket? Ha te vagy a Krisztus, mondd meg nékünk nyilván!
\par 25 Felele nékik Jézus: Megmondtam néktek, és nem hiszitek: a cselekedetek, a melyeket én cselekszem az én Atyám nevében, azok tesznek bizonyságot rólam.
\par 26 De ti nem hisztek, mert ti nem az én juhaim közül vagytok. A mint megmondtam néktek:
\par 27 Az én juhaim hallják az én szómat, és én ismerem õket, és követnek engem:
\par 28 És én örök életet adok nékik; és soha örökké el nem vesznek, és senki ki nem ragadja  õket az én kezembõl.
\par 29 Az én Atyám, a ki azokat adta nékem, nagyobb mindeneknél; és senki sem ragadhatja ki azokat az én Atyámnak kezébõl.
\par 30 Én és az Atya egy vagyunk.
\par 31 Ismét köveket ragadának azért a zsidók, hogy megkövezzék õt.
\par 32 Felele nékik Jézus: Sok jó dolgot mutattam néktek az én Atyámtól; azok közül melyik dologért köveztek meg engem?
\par 33 Felelének néki a zsidók, mondván: Jó dologért nem kövezünk meg téged, hanem káromlásért, tudniillik, hogy te ember létedre Istenné teszed magadat.
\par 34 Felele nékik Jézus: Nincs-é megírva a ti törvényetekben: Én mondám: Istenek vagytok?
\par 35 Ha azokat isteneknek mondá, a kikhez az Isten beszéde lõn (és az írás fel nem bontható),
\par 36 Arról mondjátok-é ti, a kit az Atya megszentelt és elküldött e világra: Káromlást szólsz; mivelhogy azt mondám: Az Isten Fia vagyok?!
\par 37 Ha az én Atyám dolgait nem cselekszem, ne higyjetek nékem;
\par 38 Ha pedig azokat cselekszem, ha nékem nem hisztek is, higyjetek a cselekedeteknek: hogy megtudjátok és elhigyjétek, hogy az Atya én bennem van,  és én õ benne vagyok.
\par 39 Ismét meg akarák azért õt fogni; de kiméne az õ kezökbõl.
\par 40 És újra elméne túl a Jordánon, arra a helyre, a hol János elõször keresztelt vala; és ott marada.
\par 41 És sokan menének õ hozzá és mondják vala, hogy: János nem tett ugyan semmi csodát; de mindaz, a mit János e felõl mondott, igaz vala.
\par 42 És sokan hivének ott õ benne.

\chapter{11}

\par 1 Vala pedig egy beteg, Lázár, Bethániából, Máriának és az õ testvérének, Márthának falujából.
\par 2 Az a Mária volt pedig az, a kinek a testvére Lázár beteg vala, a ki megkente vala az Urat kenettel és a hajával törlé meg annak lábait.
\par 3 Küldének azért a testvérek õ hozzá, mondván: Uram, ímé, a kit szeretsz, beteg.
\par 4 Jézus pedig, a mikor ezt hallotta, monda: Ez a betegség nem halálos, hanem az Isten dicsõségére való, hogy dicsõíttessék általa az Istennek Fia.
\par 5 Szereti vala pedig Jézus Márthát és annak nõtestvérét, és Lázárt.
\par 6 Mikor azért meghallá, hogy beteg, akkor két napig marada azon a helyen, a hol vala.
\par 7 Ez után aztán monda tanítványainak: Menjünk ismét Júdeába.
\par 8 Mondának néki a tanítványok: Mester, most akarnak vala téged megkövezni a Júdabeliek, és újra oda mégy?
\par 9 Felele Jézus: Avagy nem tizenkét órája van-é a napnak? Ha valaki nappal jár, nem botlik meg, mert látja e világnak világosságát.
\par 10 De a ki éjjel jár, megbotlik, mert nincsen abban világosság.
\par 11 Ezeket mondá; és ezután monda nékik: Lázár, a mi barátunk, elaludt; de elmegyek, hogy felköltsem õt.
\par 12 Mondának azért az õ tanítványai: Uram, ha elaludt, meggyógyul.
\par 13 Pedig Jézus annak haláláról beszélt; de õk azt hitték, hogy álomnak alvásáról szól.
\par 14 Ekkor azért nyilván monda nékik Jézus: Lázár megholt.
\par 15 És örülök, hogy nem voltam ott, ti érettetek, hogy higyjetek. De menjünk el õ hozzá!
\par 16 Monda azért Tamás, a ki Kettõsnek mondatik, az tanítványtársainak: Menjünk el mi is, hogy meghaljunk vele.
\par 17 Elmenvén azért Jézus, úgy találá, hogy az már négy napja vala sírban.
\par 18 Bethánia pedig közel vala Jeruzsálemhez, mintegy tizenöt futamatnyira;
\par 19 És a zsidók közül sokan mentek vala Márthához és Máriához, hogy vigasztalják õket az õ testvérök felõl.
\par 20 Mártha azért, a mint hallja vala, hogy Jézus jõ, elébe méne; Mária pedig otthon ül vala.
\par 21 Monda azért Mártha Jézusnak: Uram, ha itt lettél volna, nem halt volna meg a testvérem.
\par 22 De most is tudom, hogy a mit csak kérsz az Istentõl, megadja néked az Isten.
\par 23 Monda néki Jézus: Feltámad a te testvéred.
\par 24 Monda néki Mártha: Tudom, hogy feltámad a feltámadáskor az utolsó napon.
\par 25 Monda néki Jézus: Én vagyok a feltámadás és az élet: a ki hisz én bennem, ha meghal is, él;
\par 26 És a ki csak él és hisz én bennem, soha meg nem hal. Hiszed-é ezt?
\par 27 Monda néki: Igen Uram, én hiszem, hogy te vagy a Krisztus, az Istennek Fia, a ki e világra jövendõ vala.
\par 28 És a mint ezeket mondotta vala, elméne, és titkon szólítá az õ testvérét Máriát, mondván: A Mester itt van és hív téged.
\par 29 Mihelyt ez hallá, felkele hamar és hozzá méne.
\par 30 Jézus pedig nem ment vala még be a faluba, hanem azon a helyen vala, a hová Mártha elébe ment vala.
\par 31 A zsidók azért, a kik õ vele otthon valának és vigasztalák õt, látván, hogy Mária hamar felkél és kimegy vala, utána menének, ezt mondván: A sírhoz megy, hogy ott sírjon.
\par 32 Mária azért, a mint oda ére, a hol Jézus vala, meglátván õt, az õ lábaihoz esék, mondván néki: Uram, ha itt voltál volna, nem halt volna meg az én testvérem.
\par 33 Jézus azért, a mint látja vala, hogy az sír és sírnak a vele jött zsidók is, elbúsula lelkében és igen megrendüle.
\par 34 És monda: Hová helyeztétek õt? Mondának néki: Uram, jer és lásd meg!
\par 35 Könnyekre fakadt Jézus.
\par 36 Mondának azért a zsidók: Ímé, mennyire szerette õt!
\par 37 Némelyek pedig mondának közülök: Nem megtehette volna-é ez, a ki a vaknak szemét felnyitotta, hogy ez ne haljon meg?
\par 38 Jézus pedig újra felindulva magában, oda megy vala a sírhoz. Az pedig egy üreg vala, és kõ feküvék rajta.
\par 39 Monda Jézus: Vegyétek el a követ. Monda néki a megholtnak nõtestvére, Mártha: Uram, immár szaga van, hiszen negyednapos.
\par 40 Monda néki Jézus: Nem mondtam-é néked, hogy ha hiszel, meglátod majd az Istennek dicsõségét?
\par 41 Elvevék azért a követ onnan, a hol a megholt feküszék vala. Jézus pedig felemelé szemeit az égre, és monda: Atyám, hálát adok néked, hogy meghallgattál engem.
\par 42 Tudtam is én, hogy te mindenkor meghallgatsz engem; csak a körülálló sokaságért mondtam, hogy elhigyjék, hogy te küldtél engem.
\par 43 És mikor ezeket mondá, fenszóval kiálta: Lázár, jõjj ki!
\par 44 És kijöve a megholt, lábain és kezein kötelékekkel megkötözve, és az orczája kendõvel vala leborítva. Monda nékik Jézus: Oldozzátok meg õt, és hagyjátok menni.
\par 45 Sokan hivének azért õ benne ama zsidók közül, a kik Máriához mentek vala, és láták, a miket cselekedett vala.
\par 46 De némelyek azok közül elmenének a farizeusokhoz, és elbeszélék nékik, a miket Jézus cselekedett vala.
\par 47 Egybegyûjték azért a papifejedelmek és a farizeusok a fõtanácsot, és mondának: Mit cselekedjünk? mert ez az ember sok csodát mível.
\par 48 Ha ekképen hagyjuk õt, mindenki hinni fog õ benne: és eljõnek majd a rómaiak és elveszik tõlünk mind e helyet, mind e népet.
\par 49 Egy pedig õ közülök, Kajafás, a ki fõpap vala abban az esztendõben, monda nékik: Ti semmit sem tudtok.
\par 50 Meg sem gondoljátok, hogy jobb nékünk, hogy egy ember haljon meg a népért, és az egész nép el ne vesszen.
\par 51 Ezt pedig nem magától mondta: hanem mivelhogy abban az esztendõben fõpap vala, jövendõt monda, hogy Jézus meg fog halni a népért;
\par 52 És nemcsak a népért, hanem azért is, hogy az Istennek elszéledt gyermekeit egybegyûjtse.
\par 53 Ama naptól azért azon tanakodának, hogy õt megöljék.
\par 54 Jézus azért nem jár vala többé nyilvánosan a zsidók között, hanem elméne onnan a vidékre, a pusztához közel, egy Efraim nevû városba; és ott tartózkodék az õ tanítványaival.
\par 55 Közel vala pedig a zsidók husvétja: és sokan menének fel Jeruzsálembe a vidékrõl husvét elõtt, hogy megtisztuljanak.
\par 56 Keresék azért Jézust, és szólnak vala egymással a templomban állva: Mit gondoltok, hogy nem jön-é fel az ünnepre?
\par 57 A papi fejedelmek pedig és a farizeusok is parancsolatot adának, hogy ha valaki megtudja, hogy hol van, jelentse meg, hogy õt megfogják.

\chapter{12}

\par 1 Jézus azért hat nappal a husvét elõtt méne Bethániába, a hol a megholt Lázár vala, a kit feltámasztott a halálból.
\par 2 Vacsorát készítének azért ott néki, és Mártha szolgál vala fel; Lázár pedig egy vala azok közül, a kik együt ülnek vala õ vele.
\par 3 Mária azért elõvévén egy font igazi, drága nárdusból való kenetet, megkené a Jézus lábait, és megtörlé annak lábait a saját hajával; a ház pedig megtelék a kenet illatával.
\par 4 Monda azért egy az õ tanítványai közül, Iskáriótes Júdás, Simonnak fia, a ki õt elárulandó vala:
\par 5 Miért nem adták el ezt a kenetet háromszáz dénáron, és miért nem adták a szegényeknek?
\par 6 Ezt pedig nem azért mondá, mintha néki a szegényekre volna gondja, hanem mivelhogy tolvaj vala, és nála vala az erszény, és amit abba tesznek vala, elcsené.
\par 7 Monda azért Jézus: Hagyj békét néki; az én temetésem idejére tartogatta õ ezt.
\par 8 Mert szegények mindenkor vannak veletek, én pedig nem mindenkor vagyok.
\par 9 A zsidók közül azért nagy sokaság értesült vala arról, hogy õ ott van: és oda menének nemcsak Jézusért, hanem hogy Lázárt is lássák, a kit feltámasztott a halálból.
\par 10 A papifejedelmek pedig tanácskozának, hogy Lázárt is megöljék;
\par 11 Mivelhogy a zsidók közül sokan õ miatta menének oda és hivének a Jézusban.
\par 12 Másnap a nagy sokaság, a mely az ünnepre jött vala, hallván, hogy Jézus Jeruzsálembe jõ,
\par 13 Pálmaágakat võn, és kiméne elébe, és kiált vala: Hozsánna: Áldott, a ki jõ az Úrnak nevében, az Izráelnek ama királya!
\par 14 Találván pedig Jézus egy szamarat, felüle arra, a mint meg van írva:
\par 15 Ne félj Sionnak leánya: Ímé a te királyod jõ, szamárnak vemhén ülve.
\par 16 Ezeket pedig nem értették eleinte az õ tanítványai: hanem mikor megdicsõítteték Jézus, akkor emlékezének vissza, hogy ezek õ felõle vannak megírva, és hogy ezeket mívelték õ vele.
\par 17 A sokaság azért, a mely õ vele vala, mikor kihívta Lázárt a koporsóból és feltámasztotta õt a halálból, bizonyságot tõn.
\par 18 Azért is méne õ elébe a sokaság, mivel hallá, hogy ezt a csodát mívelte vala.
\par 19 Mondának azért a farizeusok egymás között: Látjátok-é, hogy semmit sem értek? Ímé, mind e világ õ utána megy.
\par 20 Néhány görög is vala azok között, a kik felmenének, hogy imádkozzanak az ünnepen:
\par 21 Ezek azért a galileai Bethsaidából való Filephez menének, és kérék õt, mondván: Uram, látni akarjuk a Jézust.
\par 22 Megy vala Filep és szóla Andrásnak, és viszont András és Filep szóla Jézusnak.
\par 23 Jézus pedig felele nékik, mondván: Eljött az óra, hogy megdicsõíttessék az embernek Fia.
\par 24 Bizony, bizony mondom néktek: Ha a földbe esett gabonamag el nem hal, csak egymaga marad; ha pedig elhal, sok gyümölcsöt terem.
\par 25 Aki szereti a maga életét, elveszti azt; és a ki gyûlöli a maga életét e világon, örök életre tartja meg azt.
\par 26 A ki nékem szolgál, engem kövessen; és a hol én vagyok, ott lesz az én szolgám is: és a ki nékem szolgál, megbecsüli azt az Atya.
\par 27 Most az én lelkem háborog; és mit mondjak? Atyám, ments meg engem ettõl az órától. De azért jutottam ez órára.
\par 28 Atyám, dicsõítsd meg a te nevedet! Szózat jöve azért az égbõl: Meg is dicsõítettem, és újra megdicsõítem.
\par 29 A sokaság azért, a mely ott állt és hallotta vala, azt mondá, hogy mennydörgött; mások mondának: Angyal szólt néki.
\par 30 Felele Jézus és monda: Nem én érettem lõn e szó, hanem ti érettetek.
\par 31 Most van e világ kárhoztatása; most vettetik ki e világ fejedelme:
\par 32 És én, ha felemeltetem e földrõl, mindeneket magamhoz vonszok.
\par 33 Ezt pedig azért mondá, hogy megjelentse, milyen halállal kell meghalnia.
\par 34 Felele néki a sokaság: Mi azt hallottuk a törvénybõl, hogy a Krisztus örökké megmarad: hogyan mondod hát te, hogy az ember Fiának fel kell emeltetnie? Kicsoda ez az ember Fia?
\par 35 Monda azért nékik Jézus: Még egy kevés ideig veletek van a világosság. Járjatok, a míg világosságotok van, hogy sötétség ne lepjen meg titeket: és a ki a sötétségben jár, nem tudja,  hová megy.
\par 36 Míg a világosságotok megvan, higyjetek a világosságban, hogy a világosság fiai legyetek. Ezeket mondá Jézus, és elmenvén, elrejtõzködék elõlük.
\par 37 És noha õ ennyi jelt tett vala elõttük, mégsem hivének õ benne:
\par 38 Hogy beteljesedjék az Ésaiás próféta beszéde, a melyet monda: Uram, ki hitt a mi tanításunknak? és az Úr karja kinek jelentetett meg?
\par 39 Azért nem hihetnek vala, mert ismét monda Ésaiás:
\par 40 Megvakította az õ szemeiket, és megkeményítette az õ szívöket; hogy szemeikkel ne lássanak és szívökkel ne értsenek, és meg ne térjenek, és meg ne gyógyítsam õket.
\par 41 Ezeket mondá Ésaiás, a mikor látta az õ dicsõségét; és beszéle õ felõle.
\par 42 Mindazáltal a fõemberek közül is sokan hivének õ benne: de a farizeusok miatt nem vallák be, hogy ki ne rekesztessenek a gyülekezetbõl:
\par 43 Mert inkább szerették az emberek dicséretét, mintsem az Istennek dicséretét.
\par 44 Jézus pedig kiálta és monda: A ki hisz én bennem, nem én bennem hisz, hanem abban, a ki elküldött engem.
\par 45 És a ki engem lát, azt látja, a ki küldött engem.
\par 46 És világosságul jöttem e világra, hogy senki ne maradjon a sötétségben, a ki én bennem hisz.
\par 47 És ha valaki hallja az én beszédeimet és nem hisz, én nem kárhoztatom azt: mert nem azért jöttem, hogy kárhoztassam a világot, hanem hogy megtartsam a világot.
\par 48 A ki megvet engem és nem veszi be az én beszédeimet, van annak, a ki õt kárhoztassa: a beszéd, a melyet szólottam, az kárhoztatja azt az utolsó napon.
\par 49 Mert én nem magamtól szóltam; hanem az Atya, a ki küldött engem, õ parancsolta nékem, hogy mit mondjak és mit beszéljek.
\par 50 És tudom, hogy az õ parancsolata örök élet. A miket azért én beszélek, úgy beszélem, a mint az Atya mondotta vala nékem.

\chapter{13}

\par 1 A husvét ünnepe elõtt pedig, tudván Jézus, hogy eljött az õórája, hogy átmenjen e világból az Atyához, mivelhogy szerette az övéit e világon, mindvégig szerette õket.
\par 2 És vacsora közben, a mikor az ördög belesugalta már Iskáriótes Júdásnak, a Simon fiának szívébe, hogy árulja el õt,
\par 3 Tudván Jézus, hogy az Atya mindent hatalmába adott néki, és hogy õ az Istentõl jött és az Istenhez megy,
\par 4 Felkele a vacsorától, leveté a felsõ ruháját; és egy kendõt vévén, körülköté magát.
\par 5 Azután vizet tölte a medenczébe, és kezdé mosni a tanítványok lábait, és megtörleni a kendõvel, a melylyel körül van kötve.
\par 6 Méne azért Simon Péterhez; és az monda néki: Uram, te mosod-é meg az én lábaimat?
\par 7 Felele Jézus és monda néki: A mit én cselekszem, te azt most nem érted, de ezután majd megérted.
\par 8 Monda néki Péter: Az én lábaimat nem mosod meg soha! Felele néki Jézus: Ha meg nem moslak téged, semmi közöd sincs én hozzám.
\par 9 Monda néki Simon Péter: Uram, ne csak lábaimat, hanem kezeimet és fejemet is!
\par 10 Mondá néki Jézus: A ki megfürödött, nincs másra szüksége, mint a lábait megmosni, különben egészen tiszta; ti is tiszták vagytok, de nem mindnyájan.
\par 11 Tudta ugyanis, hogy ki árulja el õt; azért mondá: Nem vagytok mindnyájan tiszták!
\par 12 Mikor azért megmosta azoknak lábait, és a felsõ ruháját felvette, újra leülvén, monda nékik: Értitek-é, hogy mit cselekedtem veletek?
\par 13 Ti engem így hívtok: Mester, és Uram  És jól mondjátok, mert az vagyok.
\par 14 Azért, ha én az Úr és a Mester megmostam a ti lábaitokat, néktek is meg kell mosnotok egymás lábait.
\par 15 Mert példát adtam néktek, hogy a miképen én cselekedtem veletek, ti is akképen cselekedjetek.
\par 16 Bizony, bizony mondom néktek: A szolga nem nagyobb az õ Uránál; sem a követ nem nagyobb annál, a ki azt küldte.
\par 17 Ha tudjátok ezeket, boldogok lesztek, ha cselekszitek ezeket.
\par 18 Nem mindnyájatokról szólok; tudom én kiket választottam el; hanem hogy beteljesedjék az írás: A ki velem ette a kenyeret, a sarkát emelte fel ellenem.
\par 19 Most megmondom néktek, mielõtt meglenne, hogy mikor meglesz, higyjétek majd, hogy én vagyok.
\par 20 Bizony, bizony mondom néktek: A ki befogadja, ha valakit elküldök, engem fogad be; a ki pedig engem befogad, azt fogadja be, a ki engem küldött.
\par 21 Mikor ezeket mondja vala Jézus, igen nyugtalankodék lelkében, s bizonyságot tõn, és monda: Bizony, bizony mondom néktek, hogy egy ti közületek elárul engem.
\par 22 A tanítványok ekkor egymásra tekintének bizonytalankodva, hogy kirõl szól.
\par 23 Egy pedig az õ tanítványai közül a Jézus kebelén nyugszik vala, a kit szeret vala Jézus.
\par 24 Int azért ennek Simon Péter, hogy tudakozza meg, ki az, a kirõl szól?
\par 25 Az pedig a Jézus kebelére hajolván, monda néki: Uram, ki az?
\par 26 Felele Jézus: Az, a kinek én a bemártott falatot adom. És bemártván a falatot, adá Iskáriótes Júdásnak, a Simon fiának.
\par 27 És a falat után akkor beméne abba a Sátán. Monda azért néki Jézus: A mit cselekszel, hamar cselekedjed.
\par 28 Ezt pedig senki sem érté a leültek közül, miért mondta néki?
\par 29 Némelyek ugyanis állíták, mivelhogy az erszény Júdásnál vala, hogy azt mondá néki Jézus: Vedd meg, a mikre szükségünk van az ünnepre; vagy, hogy adjon valamit a szegényeknek.
\par 30 Az pedig, mihelyt a falatot elvevé, azonnal kiméne: vala pedig éjszaka.
\par 31 Mikor azért kiment vala, monda Jézus: Most dicsõítteték meg az embernek Fia, az Isten is megdicsõítteték õ benne.
\par 32 Ha megdicsõítteték õ benne az Isten, az Isten is megdicsõíti õt õ magában, és ezennel megdicsõíti õt.
\par 33 Fiaim, egy kevés ideig még veletek vagyok. Kerestek  majd engem; de a miként a zsidóknak mondám, hogy: A hová én megyek, ti nem jöhettek; most néktek is mondom.
\par 34 Új parancsolatot adok néktek, hogy egymást szeressétek; a mint én szerettelek titeket, úgy szeressétek ti is egymást.
\par 35 Errõl ismeri meg mindenki, hogy az én tanítványaim vagytok, ha egymást szeretni fogjátok.
\par 36 Monda néki Simon Péter: Uram, hová mégy? Felele néki Jézus: A hová én megyek, most én utánam nem jöhetsz; utóbb azonban utánam jössz.
\par 37 Monda néki Péter: Uram, miért nem mehetek most utánad? Az életemet adom éretted!
\par 38 Felele néki Jézus: Az életedet adod érettem? Bizony, bizony mondom néked, nem szól addig a kakas, mígnem háromszor megtagadsz engem.

\chapter{14}

\par 1 Ne nyugtalankodjék a ti szívetek: higyjetek Istenben, és higyjetek én bennem.
\par 2 Az én Atyámnak házában sok lakóhely van; ha pedig nem volna, megmondtam volna néktek. Elmegyek, hogy helyet készítsek nektek.
\par 3 És ha majd elmegyek és helyet készítek néktek, ismét eljövök és magamhoz veszlek titeket; hogy a hol én vagyok, ti is ott legyetek.
\par 4 És hogy hová megyek én, tudjátok; az útat is tudjátok.
\par 5 Monda néki Tamás: Uram, nem tudjuk hová mégy; mimódon tudhatjuk azért az útat?
\par 6 Monda néki Jézus: Én vagyok az út, az igazság és az élet; senki sem mehet az Atyához, hanemha én általam.
\par 7 Ha megismertetek volna engem, megismertétek volna az én Atyámat is; és mostantól fogva ismeritek õt, és láttátok õt.
\par 8 Monda néki Filep: Uram, mutasd meg nékünk az Atyát, és elég nékünk!
\par 9 Monda néki Jézus: Annyi idõ óta veletek vagyok, és még se ismertél meg engem, Filep? a ki engem látott, látta az Atyát; mimódon mondod azért te: Mutasd meg nékünk az Atyát?
\par 10 Nem hiszed-é, hogy én az Atyában vagyok, és az Atya én bennem van? A beszédeket, a melyeket én mondok néktek, nem magamtól mondom;  hanem az Atya, a ki én bennem lakik, õ cselekszi e dolgokat.
\par 11 Higyjetek nékem, hogy én az Atyában vagyok, és az Atya én bennem van; ha pedig nem, magokért a cselekedetekért higyjetek nékem.
\par 12 Bizony, bizony mondom néktek: A ki hisz én bennem, az is cselekszi majd azokat a cselekedeteket, a melyeket én cselekeszem; és nagyobbakat is cselekszik azoknál; mert én az én Atyámhoz megyek.
\par 13 És akármit kértek majd az én nevemben, megcselekszem azt, hogy dicsõíttessék az Atya a Fiúban.
\par 14 Ha valamit kértek az én nevemben, én megcselekszem azt.
\par 15 Ha engem szerettek, az én parancsolataimat megtartsátok.
\par 16 És én kérem az Atyát, és más vígasztalót ád néktek, hogy veletek maradjon mindörökké.
\par 17 Az igazságnak ama Lelkét: a kit a világ be nem fogadhat, mert nem látja õt és nem ismeri õt; de ti ismeritek õt, mert nálatok lakik, és bennetek marad.
\par 18 Nem hagylak titeket árvákul; eljövök ti hozzátok.
\par 19 Még egy kevés idõ és a világ nem lát engem többé; de ti megláttok engem: mert én élek, ti is élni fogtok.
\par 20 Azon a napon megtudjátok majd ti, hogy én az én Atyámban vagyok, és ti én bennem, és én ti bennetek.
\par 21 A ki ismeri az én parancsolataimat és megtartja azokat, az szeret engem; a ki pedig engem szeret, azt szereti az én Atyám, és én is szeretem azt, és kijelentem magamat annak.
\par 22 Monda néki Júdás (nem az Iskáriótes): Uram, mi dolog, hogy nékünk jelented ki magadat, és nem a világnak?
\par 23 Felele Jézus és monda néki: Ha valaki szeret engem, megtartja az én beszédemet: és az én Atyám szereti azt, és ahhoz megyünk, és annál lakozunk.
\par 24 A ki nem szeret engem, nem tartja meg az én beszédeimet: és az a beszéd, a melyet hallotok, nem az enyém, hanem az Atyáé, a ki küldött engem.
\par 25 Ezeket beszéltem néktek, a míg veletek valék.
\par 26 Ama vígasztaló pedig, a Szent Lélek, a kit az én nevemben küld az Atya, az mindenre megtanít majd titeket, és eszetekbe juttatja mindazokat, a miket mondottam néktek.
\par 27 Békességet hagyok néktek; az én békességemet adom néktek: nem úgy adom én néktek, a mint a világ adja. Ne nyugtalankodjék a ti szívetek, se ne féljen!
\par 28 Hallottátok, hogy én azt mondtam néktek: Elmegyek, és eljövök hozzátok. Ha szeretnétek engem, örvendeznétek, hogy azt mondtam: Elmegyek az Atyához; mert az én Atyám nagyobb nálamnál.
\par 29 És most mondtam meg néktek, mielõtt meglenne: hogy a mikor majd meglesz, higyjetek.
\par 30 Nem sokat beszélek már veletek, mert jön a világ fejedelme: és én bennem nincsen semmije;
\par 31 De, hogy megtudja a világ, hogy szeretem az Atyát és úgy cselekszem, a mint az én Atyám parancsolta nékem: keljetek fel, menjünk el innen.

\chapter{15}

\par 1 Én vagyok az igazi szõlõtõ, és az én Atyám a szõlõmûves.
\par 2 Minden szõlõvesszõt, a mely én bennem gyümölcsöt nem terem, lemetsz; mindazt pedig, a mely gyümölcsöt terem, megtisztítja, hogy több gyümölcsöt teremjen.
\par 3 Ti már tiszták vagytok ama beszéd által, a melyet szóltam néktek.
\par 4 Maradjatok én bennem és én is ti bennetek. Miképen a szõlõvesszõ nem teremhet gyümölcsöt magától, hanemha a szõlõtõkén marad; akképen ti sem, hanemha én bennem maradtok.
\par 5 Én vagyok a szõlõtõ, ti a szõlõvesszõk: A ki én bennem marad, én pedig õ benne, az terem sok gyümölcsöt: mert nálam nélkül semmit sem cselekedhettek.
\par 6 Ha valaki nem marad én bennem, kivettetik, mint a szõlõvesszõ, és megszárad; és egybe gyûjtik ezeket és a tûzre vetik, és megégnek.
\par 7 Ha én bennem maradtok, és az én beszédeim bennetek maradnak, kérjetek, a mit csak akartok, és meglesz az néktek.
\par 8 Abban dicsõíttetik meg az én Atyám, hogy sok gyümölcsöt teremjetek; és legyetek nékem tanítványaim.
\par 9 A miképen az Atya szeretett engem, én is úgy szerettelek titeket: maradjatok meg ebben az én szeretetemben.
\par 10 Ha az én parancsolataimat megtartjátok, megmaradtok az én szeretetemben; a miképen én megtartottam az én Atyámnak parancsolatait, és megmaradok az õ szeretetében.
\par 11 Ezeket beszéltem néktek, hogy megmaradjon ti bennetek az én örömem és a ti örömetek beteljék.
\par 12 Ez az én parancsolatom, hogy szeressétek  egymást, a miképen én szerettelek titeket.
\par 13 Nincsen senkiben nagyobb szeretet annál, mintha valaki életét adja az õ barátaiért.
\par 14 Ti az én barátaim vagytok, ha azokat cselekszitek, a miket én parancsolok néktek.
\par 15 Nem mondalak többé titeket szolgáknak; mert a szolga nem tudja, mit cselekszik az õ ura; titeket pedig barátaimnak mondottalak; mert mindazt, a mit az én Atyámtól hallottam, tudtul adtam néktek.
\par 16 Nem ti választottatok engem, hanem én választottalak titeket, és én rendeltelek titeket, hogy ti elmenjetek és gyümölcsöt teremjetek, és a ti gyümölcsötök megmaradjon; hogy akármit kértek az Atyától  az én nevemben, megadja néktek.
\par 17 Ezeket parancsolom néktek, hogy egymást szeressétek.
\par 18 Ha gyûlöl titeket a világ, tudjátok meg, hogy engem elébb gyûlölt ti nálatoknál.
\par 19 Ha e világból volnátok, a világ szeretné azt, a mi az övé; de mivelhogy nem vagytok e világból, hanem én választottalak ki magamnak titeket e világból, azért gyûlöl titeket a világ.
\par 20 Emlékezzetek meg ama beszédekrõl, a melyeket én mondtam néktek: Nem nagyobb a szolga az õ uránál. Ha engem üldöztek, titeket is üldöznek majd; ha az én beszédemet megtartották, a tiéteket is megtartják majd.
\par 21 De mindezt az én nevemért cselekszik veletek, mivelhogy nem ismerik azt, a ki küldött engem.
\par 22 Ha nem jöttem volna és nem beszéltem volna nékik, nem volna bûnük: de most nincs mivel menteniök az õ bûnöket.
\par 23 A ki engem gyûlöl, gyûlöli az én Atyámat is.
\par 24 Ha ama cselekedeteket nem cselekedtem volna közöttük, a melyeket senki más nem cselekedett, nem volna bûnük; de most láttak is, gyûlöltek is, mind engem, mind az én Atyámat.
\par 25 De azért lõn így, hogy beteljesedjék a mondás, a mely megiratott az õ törvényökben: Ok nélkül gyûlöltek engem.
\par 26 Mikor pedig eljõ majd a Vígasztaló, a kit én küldök néktek az Atyától, az igazságnak Lelke, a ki az Atyától származik, az tesz majd én rólam bizonyságot.
\par 27 De ti is bizonyságot tesztek; mert kezdettõl fogva én velem vagytok.

\chapter{16}

\par 1 Ezeket beszéltem néktek, hogy meg ne botránkozzatok.
\par 2 A gyülekezetekbõl kirekesztenek titeket; sõt jön idõ, hogy a ki öldököl titeket, mind azt hiszi, hogy isteni tiszteletet cselekszik.
\par 3 És ezeket azért cselekszik veletek, mert nem ismerték meg az Atyát, sem engem.
\par 4 Ezeket pedig azért beszéltem néktek, hogy a mikor eljõ az az idõ, megemlékezzetek róluk, hogy én mondtam néktek. De ezeket kezdettõl fogva nem mondottam néktek, mivelhogy veletek valék.
\par 5 Most pedig elmegyek ahhoz, a ki küldött engem; és senki sem kérdezi tõlem közületek: Hová mégy?
\par 6 Hanem, mivelhogy ezeket beszéltem néktek, a szomorúság eltöltötte a szíveteket.
\par 7 De én az igazat mondom néktek: Jobb néktek, hogy én elmenjek: mert ha el nem megyek, nem jõ el hozzátok a Vígasztaló: ha pedig elmegyek, elküldöm azt ti hozzátok.
\par 8 És az, mikor eljõ, megfeddi a világot bûn, igazság és ítélet tekintetében:
\par 9 Bûn tekintetében, hogy nem hisznek én bennem;
\par 10 És igazság tekintetében, hogy én az én Atyámhoz megyek, és többé nem láttok engem;
\par 11 Ítélet tekintetében pedig, hogy e világnak fejedelme megítéltetett.
\par 12 Még sok mondani valóm van hozzátok, de most el nem hordozhatjátok.
\par 13 De mikor eljõ amaz, az igazságnak Lelke, elvezérel majd titeket minden igazságra. Mert nem õ magától szól, hanem azokat szólja, a miket hall, és a bekövetkezendõket megjelenti néktek.
\par 14 Az engem dicsõít majd, mert az enyémbõl vesz, és megjelenti néktek.
\par 15 Mindaz, a mi az Atyáé, az enyém: azért mondám, hogy az enyémbõl vesz, és megjelenti néktek.
\par 16 Egy kevés idõ, és nem láttok engem; és ismét egy kevés idõ, és megláttok majd engem: mert én az Atyához megyek.
\par 17 Mondának azért az õ tanítványai közül egymásnak: Mi az, a mit nékünk mond: Egy kevés idõ, és nem láttok engem; és ismét egy kevés idõ, és megláttok majd engem; és: mert én az Atyához megyek?
\par 18 Mondának azért: Mi az a kevés idõ, a mirõl szól? Nem tudjuk, mit mond.
\par 19 Megérté azért Jézus, hogy õt akarnák megkérdezni, és monda nékik: Arról tudakozzátok-é egymást, hogy azt mondám: Egy kevés idõ, és nem láttok engem; és ismét egy kevés idõ, és megláttok majd engem?
\par 20 Bizony, bizony mondom néktek, hogy sírtok és jajgattok ti, a világ pedig örül: ti szomorkodtok, hanem a ti szomorúságtok örömre fordul.
\par 21 Az asszony mikor szûl, szomorúságban van, mert eljött az õ órája: de mikor megszûli az õ gyermekét, nem emlékezik többé a kínra az öröm miatt, hogy ember született e világra.
\par 22 Ti is azért most ugyan szomorúságban vagytok, de ismét meglátlak majd titeket, és örülni fog a ti szívetek, és senki el nem veszi tõletek a ti örömeteket.
\par 23 És azon a napon nem kérdeztek majd engem semmirõl. Bizony, bizony mondom néktek, hogy a mit csak kérni fogtok az Atyától az én nevemben, megadja néktek.
\par 24 Mostanáig semmit sem kértetek az Atyától az én nevemben: kérjetek és megkapjátok, hogy a ti örömetek teljes legyen.
\par 25 Ezeket példázatokban mondottam néktek; de eljõ az idõ , mikor nem példázatokban beszélek majd néktek, hanem nyiltan beszélek néktek az Atyáról.
\par 26 Azon a napon az én nevemben kértek majd: és nem mondom néktek, hogy én kérni fogom az Atyát ti érettetek;
\par 27 Mert maga az Atya szeret titeket, mivelhogy ti szerettetek engem, és elhittétek, hogy én az Istentõl jöttem ki.
\par 28 Kijöttem az Atyától, és jöttem e világba: ismét elhagyom e világot, és elmegyek az Atyához.
\par 29 Mondának néki az õ tanítványai: Ímé, most nyiltan beszélsz és semmi példázatot nem mondasz.
\par 30 Most tudjuk, hogy te mindent tudsz, és nincs szükséged arra, hogy valaki téged megkérdezzen: errõl hiszszük, hogy az Istentõl jöttél ki.
\par 31 Felele nékik Jézus: Most hiszitek?
\par 32 Ímé eljõ az óra, és immár eljött, hogy szétoszoljatok kiki az övéihez, és engem egyedül hagyjatok; de nem vagyok egyedül, mert az Atya velem van.
\par 33 Azért beszéltem ezeket néktek, hogy békességetek legyen én bennem. E világon nyomorúságtok lészen; de bízzatok: és meggyõztem a világot.

\chapter{17}

\par 1 Ezeket beszélte Jézus; és felemelve szemeit az égre, és monda: Atyám, eljött az óra; dicsõítsd meg a te Fiadat, hogy a te Fiad is dicsõítsen téged;
\par 2 A miként te hatalmat adtál néki minden testen, hogy örök életet  adjon mindennek, a mit néki adtál.
\par 3 Az pedig az örök élet, hogy megismerjenek téged, az egyedül igaz Istent, és a kit elküldtél, a Jézus Krisztust.
\par 4 Én dicsõítettelek téged e földön: elvégeztem a munkát, a melyet reám bíztál, hogy végezzem azt.
\par 5 És most te dicsõíts meg engem, Atyám, te magadnál azzal a dicsõséggel, a melylyel bírtam te nálad a világ létele elõtt.
\par 6 Megjelentettem a te nevedet az embereknek, a kiket e világból nékem adtál: tiéid valának, és nékem adtad azokat, és a te beszédedet megtartották.
\par 7 Most tudták meg, hogy mindaz te tõled van, a mit nékem adtál:
\par 8 Mert ama beszédeket, a melyeket nékem adtál, õ nékik adtam; és õk befogadták, és igazán megismerték, hogy én tõled jöttem ki, és elhitték, hogy te küldtél engem.
\par 9 Én ezekért könyörgök: nem a világért könyörgök, hanem azokért, a kiket nékem adtál, mert a tiéid.
\par 10 És az enyémek mind a tiéid, és a tiéid az enyémek: és megdicsõíttetem õ bennök.
\par 11 És nem vagyok többé e világon, de õk a világon vannak, és pedig te hozzád megyek. Szent Atyám, tartsd meg õket a te nevedben, a kiket nékem adtál, hogy egyek legyenek, mint  mi!
\par 12 Mikor velök valék a világon, én megtartám õket a te nevedben; a kiket nékem adtál, megõrizém, és senki el nem veszett közülök, csak a veszedelemnek fia, hogy az írás  beteljesüljön.
\par 13 Most pedig te hozzád megyek; és ezeket beszélem a világon, hogy õk az én örömemet teljesen bírják õ magokban.
\par 14 Én a te ígédet nékik adtam; és a világ gyûlölte õket, mivelhogy nem e világból valók, a mint hogy én sem e világból vagyok.
\par 15 Nem azt kérem, hogy vedd ki õket e világból, hanem hogy õrizd meg õket a gonosztól.
\par 16 Nem e világból valók, a mint hogy én sem e világból vagyok.
\par 17 Szenteld meg õket a te igazságoddal: A te ígéd igazság.
\par 18 A miképen te küldtél engem e világra, úgy küldtem én is õket e világra;
\par 19 És én érettök oda szentelem magamat, hogy õk is megszenteltekké legyenek az igazságban.
\par 20 De nemcsak õ érettök könyörgök, hanem azokért is, a kik az õ beszédökre hisznek majd én bennem;
\par 21 Hogy mindnyájan egyek legyenek; a mint te én bennem, Atyám, és én te benned, hogy õk is egyek legyenek mi  bennünk: hogy elhigyje a világ, hogy te küldtél engem.
\par 22 És én azt a dicsõséget, a melyet nékem adtál, õ nékik adtam, hogy egyek legyenek, a miképen mi egy vagyunk:
\par 23 Én õ bennök, és te én bennem: hogy tökéletesen egygyé legyenek, és hogy megismerje a világ, hogy te küldtél engem, és szeretted õket, a miként engem szerettél.
\par 24 Atyám, a kiket nékem adtál, akarom, hogy a hol én vagyok, azok is én velem legyenek; hogy megláthassák az én dicsõségemet, a melyet nékem adtál: mert szerettél engem e világ alapjának felvettetése elõtt.
\par 25 Igazságos Atyám! És e világ nem ismert téged, de én ismertelek téged; és ezek megismerik, hogy te küldtél engem;
\par 26 És megismertettem õ velök a te nevedet, és megismertetem; hogy az a szeretet legyen õ bennök, a mellyel engem szerettél, és én is õ bennök legyek.

\chapter{18}

\par 1 Mikor ezeket mondta vala Jézus, kiméne az õ tanítványaival együtt túl a Kedron patakán, a hol egy kert vala, a melybe bemenének  õ és az õ tanítványai.
\par 2 Ismeré pedig azt a helyet Júdás is, a ki õt elárulja vala; mivelhogy gyakorta ott gyûlt egybe Jézus az õ tanítványaival.
\par 3 Júdás azért magához vevén a katonai csapatot, és a papi fejedelmektõl és a farizeusoktól szolgákat, oda méne fáklyákkal, lámpásokkal és fegyverekkel.
\par 4 Jézus azért tudván mindazt, a mi reá következendõ vala, elõre méne, és monda azoknak: Kit kerestek?
\par 5 Felelének néki: A názáreti Jézust. Monda nékik Jézus: Én vagyok. Ott állt pedig õ velök Júdás is, a ki elárulta õt.
\par 6 Mikor azért azt mondá nékik, hogy: Én vagyok; hátra vonulának és földre esének.
\par 7 Ismét megkérdezé azért õket: Kit kerestek? És azok mondának: A názáreti Jézust.
\par 8 Felele Jézus: Mondtam néktek, hogy én vagyok az. Azért, ha engem kerestek, ezeket bocsássátok el;
\par 9 Hogy beteljesüljön a beszéd, a melyet mondott: Azok közül, a kiket nékem adtál, senkit sem vesztettem el.
\par 10 Simon Péter pedig, a kinek szablyája vala, kirántá azt, és megüté a fõpap szolgáját, és levágá annak jobb fülét. A szolga neve pedig Málkus vala.
\par 11 Monda azért Jézus Péternek: Tedd hüvelyébe a te szablyádat; avagy nem kell-é kiinnom a pohárt, a melyet az Atya adott nékem?
\par 12 A csapat azért és az ezredes és a zsidók szolgái megfogák Jézust, és megkötözék õt,
\par 13 És vivék õt elõször Annáshoz; mert ipa vala ez Kajafásnak, a ki abban az esztendõben fõpap vala.
\par 14 Kajafás pedig az vala, a ki tanácsolta vala a zsidóknak, hogy jobb, hogy egy ember veszszen el a népért.
\par 15 Simon Péter pedig, és egy másik tanítvány követi vala Jézust. Ez a tanítvány pedig ismerõs vala a fõpappal, és beméne Jézussal együtt a fõpap udvarába,
\par 16 Péter pedig kívül áll vala az ajtónál. Kiméne azért ama másik tanítvány, a ki a fõpappal ismerõs vala, és szóla az ajtóõrzõnek, és bevivé Pétert.
\par 17 Szóla azért Péterhez az ajtóõrzõ leány: Nemde, te is ez ember tanítványai közül való vagy? Monda õ: Nem vagyok.
\par 18 A szolgák pedig és a poroszlók ott állnak vala, szítván a tüzet, mivelhogy hûvös vala, és melegszenek vala. Ott áll vala pedig Péter is õ velök együtt, és melegszik vala.
\par 19 A fõpap azért kérdezé Jézust az õ tanítványai felõl, és az õ tudománya felõl.
\par 20 Felele néki Jézus: Én nyilván szólottam a világnak, én mindenkor tanítottam a zsinagógában és a templomban, a hol a zsidók mindenünnen összegyülekeznek; és titkon semmit sem szólottam.
\par 21 Mit kérdesz engem? Kérdezd azokat, a kik hallották, mit szóltam nékik: ímé õk tudják, a miket nékik szólottam.
\par 22 Mikor pedig õ ezeket mondja vala, egy a poroszlók közül, a ki ott áll vala, arczul üté Jézust, mondván: így felelsz-é a fõpapnak?
\par 23 Felele néki Jézus: Ha gonoszul szóltam, tégy bizonyságot a gonoszságról; ha pedig jól, miért versz engem.
\par 24 Elküldé õt Annás megkötözve Kajafáshoz, a fõpaphoz.
\par 25 Simon Péter pedig ott áll vala és melegszik vala. Mondának azért néki: Nemde, te is ennek a tanítványai közül való vagy? Megtagadá õ, és monda: Nem vagyok.
\par 26 Monda egy a fõpap szolgái közül, rokona annak, a kinek a fülét Péter levágta: Nem láttalak-e én téged õ vele együtt a kertben?
\par 27 Ismét megtagadá azért Péter; és a kakas azonnal megszólala.
\par 28 Vivék azért Jézust Kajafástól a törvényházba. Vala pedig reggel. És õk nem menének be a törvényházba, hogy meg ne fertõztessenek,  hanem hogy megehessék a husvétibárányt.
\par 29 Kiméne azért Pilátus õ hozzájok, és monda: Micsoda vádat hoztok fel ez ember ellen?
\par 30 Felelének és mondának néki: Ha gonosztevõ nem volna ez, nem adtuk volna õt a te kezedbe.
\par 31 Monda azért nékik Pilátus: Vigyétek el õt ti, és ítéljétek meg õt a ti törvényeitek szerint. Mondának azért néki a zsidók: Nékünk senkit sem szabad megölnünk;
\par 32 Hogy beteljesedjék a Jézus szava, a melyet monda, a mikor jelenti vala, hogy milyen halállal kell majd meghalnia.
\par 33 Ismét beméne azért Pilátus a törvényházba, és szólítja vala Jézust, és monda néki: Te vagy a Zsidók királya?
\par 34 Felele néki Jézus: Magadtól mondod-é te ezt, vagy mások beszélték néked én felõlem?
\par 35 Felele Pilátus: Avagy zsidó vagyok-e én? A te néped és a papifejedelmek adtak téged az én kezembe: mit cselekedtél?
\par 36 Felele Jázus: Az én országom nem e világból való. Ha e világból való volna az én országom, az én szolgáim vitézkednének, hogy át ne adassam a zsidóknak. Ámde az én országom nem innen való.
\par 37 Monda azért néki Pilátus: Király vagy-é hát te csakugyan? Felele Jézus: Te mondod, hogy én király vagyok. Én azért születtem, és azért jöttem e világra, hogy bizonyságot tegyek az igazságról. Mindaz, a ki az igazságból való, hallgat az én szómra.
\par 38 Monda néki Pilátus: Micsoda az igazság? És a mint ezt mondá, újra kiméne a zsidókhoz, és monda nékik: Én nem találok benne semmi bûnt.
\par 39 Szokás pedig az nálatok, hogy elbocsássak néktek egyet a husvétünnepen: akarjátok-é azért, hogy elbocsássam néktek a zsidók királyát?
\par 40 Kiáltának azért viszont mindnyájan, mondván: Nem ezt, hanem Barabbást. Ez a Barabbás pedig tolvaj vala.

\chapter{19}

\par 1 Akkor azért elõfogá Pilátus Jézust, és megostoroztatá.
\par 2 És a vitézek tövisbõl koronát fonván, a fejére tevék, és bíbor köntöst adának reá,
\par 3 És mondának: Üdvöz légy zsidók királya! És arczul csapdossák vala õt.
\par 4 Majd ismét kiméne Pilátus, és monda nékik: Ímé kihozom õt néktek, hogy értsétek meg, hogy nem találok benne semmi bûnt.
\par 5 Kiméne azért Jézus a töviskoronát és a bíbor köntöst viselve. És monda nékik Pilátus: Ímhol az ember!
\par 6 Mikor azért látják vala õt a papifejedelmek és a szolgák, kiáltozának, mondván: Feszítsd meg, feszítsd meg! Monda nékik Pilátus: Vigyétek el õt ti és feszítsétek meg, mert én nem találok bûnt õ benne.
\par 7 Felelének néki a zsidók: Nékünk törvényünk van, és a mi törvényünk szerint meg kell halnia, mivelhogy Isten Fiává tette magát.
\par 8 Mikor pedig ezt a beszédet hallotta Pilátus, még inkább megrémül vala;
\par 9 És ismét beméne a törvényházba, és szóla Jézusnak: Honnét való vagy te? De Jézus nem felelt néki.
\par 10 Monda azért néki Pilátus: Nékem nem szólsz-é? Nem tudod-é hogy hatalmam van arra, hogy megfeszítselek, és hatalmam van arra, hogy szabadon bocsássalak?
\par 11 Felele Jézus: Semmi hatalmad sem volna rajtam, ha felülrõl nem adatott volna néked: nagyobb bûne van azért annak, a ki a te kezedbe adott engem.
\par 12 Ettõl fogva igyekszik vala Pilátus õt szabadon bocsátani; de a zsidók kiáltozának, mondván: Ha ezt szabadon bocsátod, nem vagy a császár barátja; valaki magát királylyá teszi, ellene mond a császárnak!
\par 13 Pilátus azért, a mikor hallja vala e beszédet, kihozá Jézust, és ûle a törvénytevõ székbe azon a helyen, a melyet Kõpadolatnak hívtak, zsidóul pedig Gabbathának.
\par 14 Vala pedig a husvét péntekje; és mintegy hat óra. És monda a zsidóknak: Ímhol a ti királyotok!
\par 15 Azok pedig kiáltoznak vala: Vidd el, vidd el, feszítsd meg õt! Monda nékik Pilátus: A ti királyotokat feszítsem meg? Felelének a papifejedelmek: Nem királyunk van, hanem császárunk!
\par 16 Akkor azért nékik adá õt, hogy megfeszíttessék. Átvevék azért Jézust és elvivék.
\par 17 És emelvén az õ keresztfáját, méne az úgynevezett Koponya helyére, a melyet héberül Golgothának hívnak:
\par 18 A hol megfeszíték õt, és õ vele más kettõt, egyfelõl, és másfelõl, középen pedig Jézust.
\par 19 Pilátus pedig czímet is íra, és feltevé a keresztfára. Ez vala pedig az írás: A NÁZÁRETI JÉZUS, A ZSIDÓK KIRÁLYA.
\par 20 Sokan olvasák azért e czímet a zsidók közül; mivelhogy közel vala a városhoz az a hely, a hol Jézus megfeszíttetett vala: és héberül, görögül és latinul vala az írva.
\par 21 Mondának azért Pilátusnak a zsidók papifejedelmei: Ne írd: A zsidók királya: hanem hogy õ mondotta: A zsidók királya vagyok.
\par 22 Felele Pilátus: A mit megírtam, megírtam.
\par 23 A vitézek azért, mikor megfeszítették Jézust, vevék az õ ruháit, és négy részre oszták, egy részt mindenik vitéznek, és a köntösét. A köntös pedig varrástalan vala, felülrõl mindvégig szövött.
\par 24 Mondának azért egymásnak: Ezt ne hasogassuk el, hanem vessünk sorsot reá, kié legyen. Hogy beteljesedjék az írás, a mely ezt mondja: Megosztoztak ruháimon, és a köntösömre sorsot vetettek. A vitézek tehát ezeket mûvelék.
\par 25 A Jézus keresztje alatt pedig ott állottak vala az õ anyja, és az õ anyjának nõtestvére; Mária, a Kleopás felesége, és Mária Magdaléna.
\par 26 Jézus azért, mikor látja vala, hogy ott áll az õ anyja és az a tanítvány, a kit szeret vala, monda az õ anyjának: Asszony, ímhol a te fiad!
\par 27 Azután monda a tanítványnak: Ímhol a te anyád! És ettõl az órától magához fogadá azt az a tanítvány.
\par 28 Ezután tudván Jézus, hogy immár minden elvégeztetett, hogy beteljesedjék az írás, monda: Szomjúhozom.
\par 29 Vala pedig ott egy eczettel teli edény. Azok azért szivacsot töltvén meg eczettel, és izsópra tévén azt, oda vivék az õ szájához.
\par 30 Mikor azért elvette Jézus az eczetet, monda: Elvégeztetett! És lehajtván fejét, kibocsátá lelkét.
\par 31 A zsidók pedig, hogy a testek szombaton át a keresztfán ne maradjanak, miután péntek vala, (mert annak a szombatnak napja nagy nap vala) kérék Pilátust, hogy törjék meg azoknak lábszárait és vegyék le õket.
\par 32 Eljövének azért a vitézek, és megtörék az elsõnek lábszárait és a másikét is, a ki õ vele együtt feszíttetett meg;
\par 33 Mikor pedig Jézushoz érének és Látják vala, hogy õ már halott, nem törék meg az õ lábszárait;
\par 34 Hanem egy a vitézek közül dárdával döfé meg az õ oldalát, és azonnal vér és víz jöve ki abból.
\par 35 És a ki látta, bizonyságot tett, és igaz az õ tanúbizonysága; és az tudja, hogy õ igazat mond, hogy ti is higyjetek.
\par 36 Mert azért lettek ezek, hogy beteljesedjék az írás: Az õ csontja meg ne törettessék.
\par 37 Másutt ismét így szól az írás: Néznek majd arra, a kit általszegeztek.
\par 38 Ezek után pedig kéré Pilátust az arimathiai József (a ki a Jézus tanítványa vala, de csak titokban, a zsidóktól való félelem miatt), hogy levehesse  a Jézus testét. És megengedé Pilátus. Elméne azért és levevé a Jézus testét.
\par 39 Eljöve pedig Nikodémus is (a ki éjszaka ment vala elõször Jézushoz), hozván mirhából és áloéból való kenetet, mintegy száz fontot.
\par 40 Vevék azért a Jézus testét, és begöngyölgeték az lepedõkbe illatos szerekkel együtt, a mint a zsidóknál szokás temetni.
\par 41 Azon a helyen pedig, a hol megfeszítteték, vala egy kert, és a kertben egy új sír, a melybe még senki sem helyeztetett vala.
\par 42 A zsidók péntekje miatt azért, mivelhogy az a sír közel vala, abba helyhezteték Jézust.

\chapter{20}

\par 1 A hétnek elsõ napján pedig jó reggel, a mikor még sötétes vala, oda
\par 2 Futa azért és méne Simon Péterhez és ama másik tanítványhoz, a kit Jézus szeret vala, és monda nékik: Elvitték az Urat a sírból, és nem tudjuk, hová tették õt.
\par 3 Kiméne azért Péter és a másik tanítvány, és menének a sírhoz.
\par 4 Egyut futnak vala pedig mindketten: de ama másik tanítvány hamar megelõzé Pétert, és elõbb juta a sírhoz;
\par 5 És lehajolván, látá, hogy ott vannak a lepedõk; mindazáltal nem megy vala be.
\par 6 Megjöve azután Simon Péter is nyomban utána, és beméne a sírba: és látá, hogy a lepedõk ott vannak.
\par 7 És a keszkenõ, a mely az õ fején volt, nem együtt van a lepedõkkel, hanem külön összegöngyölítve egy helyen.
\par 8 Akkor aztán beméne a másik tanítvány is, a ki elõször jutott a sírhoz, és lát és hisz vala.
\par 9 Mert nem tudják vala még az írást, hogy fel kell támadnia a halálból.
\par 10 Visszamenének azért a tanítványok az övéikhez.
\par 11 Mária pedig künn áll vala a sírnál sírva. A míg azonban siránkozék, behajol vala a sírba;
\par 12 És láta két angyalt fehér ruhában ülni, egyiket fejtõl, másikat lábtól, a hol a Jézus teste feküdt vala.
\par 13 És mondának azok néki: Asszony mit sírsz? Monda nékik: Mert elvitték az én Uramat, és nem tudom, hova tették õt.
\par 14 És mikor ezeket mondotta, hátra fordula, és látá Jézust ott állani, és nem tudja vala, hogy Jézus az.
\par 15 Monda néki Jézus: Asszony, mit sírsz? kit keressz? Az pedig azt gondolván, hogy a kertész az, monda néki: Uram, ha te vitted el õt, mondd meg nékem, hová tetted õt, és én elviszem õt.
\par 16 Monda néki Jézus: Mária! Az megfordulván, monda néki: Rabbóni! a mi azt teszi: Mester!
\par 17 Monda néki Jézus: Ne illess engem; mert nem mentem még fel az én Atyámhoz; hanem menj az én atyámfiaihoz és mondd nékik: Felmegyek az én Atyámhoz és a ti Atyátokhoz, és az én Istenemhez, és a ti Istenetekhez.
\par 18 Elméne Mária Magdaléna, hirdetvén a tanítványoknak, hogy látta az Urat, és hogy ezeket mondotta néki.
\par 19 Mikor azért estve vala, azon a napon, a hétnek elsõ napján, és mikor az ajtók zárva valának, a hol egybegyûltek vala a tanítványok, a zsidóktól való félelem miatt, eljöve Jézus és megálla középen, és monda nékik: Békesség néktek!
\par 20 És ezt mondván, megmutatá nékik a kezeit és az oldalát. Örvendezének azért a tanítványok, hogy látják vala az Urat.
\par 21 Ismét monda azért nékik Jézus: Békesség néktek! A miként engem küldött vala az Atya, én is akképen küldelek titeket.
\par 22 És mikor ezt mondta, rájuk lehelle, és monda nékik: Vegyetek Szent Lelket:
\par 23 A kiknek bûneit megbocsátjátok, megbocsáttatnak azoknak; a kikéit megtartjátok, megtartatnak.
\par 24 Tamás pedig, egy a tizenkettõ közül, a kit Kettõsnek hívtak, nem vala õ velök, a mikor eljött vala Jézus.
\par 25 Mondának azért néki a többi tanítványok: Láttuk az Urat. Õ pedig monda nékik: Ha nem látom az õ kezein a szegek helyeit, és be nem bocsátom ujjaimat a szegek helyébe, és az én kezemet be nem bocsátom az õ oldalába, semmiképen el nem hiszem.
\par 26 És nyolcz nap múlva ismét benn valának az õ tanítványai, Tamás is õ velök. Noha az ajtó zárva vala, beméne Jézus, és megálla a középen és monda: Békesség néktek!
\par 27 Azután monda Tamásnak: Hozd ide a te ujjadat és nézd meg az én kezeimet; és hozd ide a te kezedet, és bocsássad az én oldalamba: és ne légy hitetlen, hanem hívõ.
\par 28 És felele Tamás és monda néki: Én Uram és én Istenem!
\par 29 Monda néki Jézus: Mivelhogy láttál engem, Tamás, hittél: boldogok, a kik nem látnak és hisznek.
\par 30 Sok más jelt is mívelt ugyan Jézus az õ tanítványai elõtt, a melyek nincsenek megírva ebben a könyvben;
\par 31 Ezek pedig azért irattak meg, hogy higyjétek, hogy Jézus a Krisztus, az Istennek Fia, és hogy ezt hívén, életetek legyen az õ nevében.

\chapter{21}

\par 1 Ezek után ismét megjelentette magát Jézus a tanítványoknak a Tibériás tengerénél; megjelentette pedig ekképen:
\par 2 Együtt valának Simon Péter, és Tamás, a kit Kettõsnek hívtak, és Nátánáel, a galileai Kánából való, és a Zebedeus fiai, és más kettõ is az õ tanítványai közül.
\par 3 Monda nékik Simon Péter: Elmegyek halászni. Mondának néki: Elmegyünk mi is te veled. Elmenének és azonnal a hajóba szállának; és azon az éjszakán nem fogtak semmit.
\par 4 Mikor pedig immár reggeledék, megálla Jézus a parton; a tanítványok azonban nem ismerék meg, hogy Jézus van ott.
\par 5 Monda azért nékik Jézus: Fiaim! Van-é valami ennivalótok? Felelének néki: Nincsen!
\par 6 Õ pedig mondá nékik: Vessétek a hálót a hajónak jobb oldala felõl, és találtok. Oda veték azért, és kivonni már nem bírták azt a halaknak sokasága miatt.
\par 7 Szóla azért az a tanítvány, a kit Jézus szeret vala, Péternek: Az Úr van ott! Simon Péter azért, a mikor hallja vala, hogy ott van az Úr, magára vevé az ingét (mert mezítelen vala), és beveté magát a tengerbe.
\par 8 A többi tanítványok pedig a hajón menének (mert nem messze valának a parttól, hanem mintegy kétszáz singnyire), és vonszszák vala a hálót a halakkal.
\par 9 Mikor azért a partra szállának, látják, hogy parázs van ott, és azon felül hal és kenyér.
\par 10 Monda nékik Jézus: Hozzatok a halakból, a melyeket most fogtatok.
\par 11 Felszálla Simon Péter, és kivoná a hálót a partra, a mely tele volt nagy halakkal, százötvenhárommal; és noha ennyi vala, nem szakadozik vala a háló.
\par 12 Monda nékik Jézus: Jertek, ebédeljetek. A tanítványok közül pedig senki sem meri vala tõle megkérdezni: Ki vagy te? Mivelhogy tudják vala, hogy az Úr õ.
\par 13 Oda méne azért Jézus, és vevé a kenyeret és adá nékik, és hasonlóképen a halat is.
\par 14 Ezzel már harmadszor jelent meg Jézus az õ tanítványainak, minekutána feltámadt a halálból.
\par 15 Mikor aztán megebédelének, monda Jézus Simon Péternek: Simon, Jónának fia: jobban szeretsz-é engem ezeknél? Monda néki: Igen, Uram; te tudod, hogy szeretlek téged! Monda néki: Legeltesd az én bárányaimat!
\par 16 Monda néki ismét másodszor is Simon, Jónának fia, szeretsz-é engem? Monda néki: Igen, Uram; te tudod, hogy én szeretlek téged. Monda néki: Õrizd az én juhaimat!
\par 17 Monda néki harmadszor is: Simon, Jónának fia, szeretsz-é engem? Megszomorodék Péter, hogy harmadszor is mondotta vala néki: Szeretsz-é engem? És monda néki: Uram, te mindent tudsz; te tudod, hogy én szeretlek téged. Monda néki Jézus: Legeltesd az én juhaimat!
\par 18 Bizony, bizony mondom néked, a mikor ifjabb valál, felövezéd magadat, és oda mégy vala, a hova akarád; mikor pedig megöregszel, kinyújtod a te kezedet és más övez fel téged, és oda visz, a hová nem akarod.
\par 19 Ezt pedig azért mondá, hogy jelentse, milyen halállal dicsõíti majd meg az Istent. És ezt mondván, szóla néki: Kövess engem!
\par 20 Péter pedig megfordulván, látja, hogy követi az a tanítvány, a kit szeret vala Jézus, a ki nyugodott is ama vacsora közben az õ kebelén és mondá: Uram! ki az, a ki elárul téged?
\par 21 Ezt látván Péter, monda Jézusnak: Uram, ez pedig mint lészen?
\par 22 Monda néki Jézus: Ha akarom, hogy õ megmaradjon, a míg eljövök, mi közöd hozzá? Te kövess engem!
\par 23 Kiméne azért e beszéd az atyafiak közé, hogy az a tanítvány nem hal meg: pedig Jézus nem mondta néki, hogy nem hal meg; hanem: Ha akarom, hogy ez megmaradjon, a míg eljövök, mi közöd hozzá?
\par 24 Ez az a tanítvány, a ki bizonyságot tesz ezekrõl, és a ki megírta ezeket, és tudjuk, hogy az õ bizonyságtétele igaz.
\par 25 De van sok egyéb is, a miket Jézus cselekedett vala, a melyek, ha egyenként megiratnának, azt vélem, hogy maga a világ sem foghatná be a könyveket, a melyeket írnának. Ámen.


\end{document}