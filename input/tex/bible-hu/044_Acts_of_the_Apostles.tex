\begin{document}

\title{Apostolok Cselekedetei}


\chapter{1}

\par 1 Elsõ könyvemet írtam, Theofilus, mindazokról, a miket kezdett Jézus cselekedni és tanítani,
\par 2 Mind a napig, melyen fölviteték, minekutána parancsolatokat adott a Szent Lélek által az apostoloknak, kiket választott vala magának.
\par 3 Kiknek az õ szenvedése után sok jel által meg is mutatta, hogy õ él, negyven napon át megjelenvén nékik, és szólván az Isten országára tartozó dolgokról.
\par 4 És velök összejövén, meghagyá nékik, hogy el ne menjenek Jeruzsálembõl, hanem várják be az Atyának ígéretét,  melyet úgymond, hallottatok tõlem:
\par 5 Hogy János ugyan vízzel keresztelt, ti azonban Szent Lélekkel fogtok megkereszteltetni nem sok nap mulva.
\par 6 Mikor azért azok egybegyûltek, megkérdék õt, mondván: Uram, avagy nem ez idõben állítod-é helyre az országot Izráelnek?
\par 7 Monda pedig nékik: Nem a ti dolgotok tudni az idõket vagy alkalmakat, melyeket az Atya a maga hatalmába helyheztetett.
\par 8 Hanem vesztek erõt, minekutána a Szent Lélek eljõ reátok: és lesztek nékem tanúim úgy Jeruzsálemben, mint az egész Júdeában és Samariában és a földnek mind végsõ határáig.
\par 9 És mikor ezeket mondotta, az õ láttokra felemelteték, és felhõ fogá el õt szemeik elõl.
\par 10 És a mint szemeiket az égre függesztették, mikor õ elméne, ímé két férfiú állott meg mellettük fehér ruhában,
\par 11 Kik szóltak is: Galileabeli férfiak, mit állotok nézve a mennybe? Ez a Jézus, a ki felviteték tõletek a mennybe, akképen jõ el, a miképen láttátok õt felmenni a mennybe.
\par 12 Akkor megtérének Jeruzsálembe a hegyrõl, mely hívatik Olajfák hegyének, mely Jeruzsálem mellett van, egy szombatnapi járóföldre.
\par 13 És mikor bementek, felmenének a felsõházba, a hol szállva valának: Péter és Jakab, János és András, Filep és Tamás, Bertalan és Máté, Jakab, az Alfeus fia, és Simon, a zelóta, és Júdás, a Jakab fia.
\par 14 Ezek mindnyájan egy szívvel-lélekkel foglalatosak valának az imádkozásban és a könyörgésben, az asszonyokkal és Máriával, Jézusnak anyjával, és az atyjafiaival együtt.
\par 15 És azokban a napokban felkelvén Péter a tanítványok között, monda (vala pedig ott együtt mintegy százhúsz fõnyi sokaság):
\par 16 Atyámfiai, férfiak, szükség volt betelni annak az írásnak, melyet megjövendölt a Szent Lélek Dávid szája által Júdás felõl, ki vezetõjük  lõn azoknak, a kik megfogták Jézust.
\par 17 Mert mi közénk számláltatott, és elnyerte ennek a szolgálatnak az osztályrészét.
\par 18 (Ez hát mezõt szerze hamisságának bérébõl; és alá zuhanván, elhasadt középen, és minden belsõ része kiomlott.
\par 19 És ez tudtokra lõn mindazoknak, kik Jeruzsálemben lakoznak; úgy hogy az a mezõ tulajdon nyelvökön Akeldamának, azaz Vérmezõnek neveztetett el.)
\par 20 Mert meg van írva a Zsoltárok könyvében: Legyen az õ lakóhelye puszta, és ne legyen lakó abban. És: Az õ püspökségét más  vegye el.
\par 21 Szükség azért, hogy azok közül a férfiak közül, a kik velünk együtt jártak minden idõben, míg az Úr Jézus közöttünk járt-kelt,
\par 22 A János keresztségétõl kezdve mind a napig, melyen fölviteték tõlünk, azok közül egy az õ feltámadásának bizonysága legyen mivelünk egyetemben.
\par 23 Állatának azért elõ kettõt, Józsefet, ki hivatik Barsabásnak, kinek mellékneve Justus vala, és Mátyást.
\par 24 És imádkozván, mondának: Te, Uram, ki mindeneknek szívét ismered, mutasd meg a kettõ közül egyiket, a kit kiválasztottál,
\par 25 Hogy elnyerje az osztályrészét e szolgálatnak és apostolságnak, melytõl eltévelyedék Júdás, hogy az õ saját helyére jusson.
\par 26 Sorsot vetének azért reájok, és esék a sors Mátyásra, és a tizenegy apostól közé számláltaték.

\chapter{2}

\par 1 És mikor a pünkösd napja eljött, mindnyájan egyakarattal együtt valának.
\par 2 És lõn nagy hirtelenséggel az égbõl mintegy sebesen zúgó szélnek zendülése, és eltelé az egész házat, a hol ülnek vala.
\par 3 És megjelentek elõttük kettõs tüzes nyelvek és üle mindenikre azok közül.
\par 4 És megtelének mindnyájan Szent Lélekkel, és kezdének szólni más nyelveken, a mint a Lélek adta nékik szólniok.
\par 5 Lakoznak vala pedig Jeruzsálemben zsidók, istenfélõ férfiak, minden nép közül, melyek az ég alatt vannak.
\par 6 Minekutána pedig ez a zúgás lõn, egybegyûle a sokaság és megzavarodék, mivelhogy mindegyik a maga nyelvén hallá õket szólni.
\par 7 Álmélkodnak pedig mindnyájan és csodálkoznak vala, mondván egymásnak: Nemde nem Galileusok-é ezek mindnyájan, a kik szólnak?
\par 8 Mimódon halljuk hát õket, kiki közülünk a saját nyelvén, a melyben születtünk?
\par 9 Párthusok és médek és elámiták, és kik lakozunk Mesopotámiában, Júdeában és Kappadócziában, Pontusban és  Ázsiában,
\par 10 Frigiában és Pamfiliában, Égyiptomban és Libiának tartományiban, mely Cziréne mellett van, és a római jövevények, mind zsidók, mind prozelitusok,
\par 11 Krétaiak és arabok, halljuk a mint szólják a mi nyelvünkön az Istennek nagyságos dolgait.
\par 12 Álmélkodnak vala pedig mindnyájan és zavarban valának, egymásnak ezt mondván: Vajjon mi akar ez lennie?
\par 13 Mások pedig csúfolódva mondának: Édes bortól részegedtek meg.
\par 14 Péter azonban elõállván a tizenegygyel, felemelé szavát, és szóla nékik: Zsidó férfiak és mindnyájan, kik lakoztok Jeruzsálemben, legyen ez néktek tudtotokra, és vegyétek füleitekbe az én beszédimet!
\par 15 Mert nem részegek ezek, a mint ti állítjátok; hiszen a napnak harmadik órája van;
\par 16 Hanem ez az, a mi megmondatott Jóel prófétától:
\par 17 És lészen az utolsó napokban, ezt mondja az Isten, kitöltök az én Lelkembõl minden testre: és prófétálnak a ti fiaitok és leányaitok, és a ti ifjaitok látásokat látnak, és a ti véneitek álmokat álmodnak.
\par 18 És épen az én szolgáimra és az én szolgálóleányaimra is kitöltök azokban a napokban az én Lelkembõl, és prófétálnak.
\par 19 És tészek csudákat az égben odafenn, és jeleket a földön idelenn, vért, tüzet és füstnek gõzölgését.
\par 20 A nap sötétséggé változik, és a hold vérré, minekelõtte eljõ az Úrnak ama nagy és fényes napja.
\par 21 És lészen, hogy mindaz, a ki az Úrnak nevét segítségül hívja, megtartatik.
\par 22 Izráelita férfiak, halljátok meg e beszédeket: A názáreti Jézust, azt a férfiút, a ki Istentõl bizonyságot nyert elõttetek erõk, csudatételek és jelek által, melyeket õ általa cselekedett Isten ti köztetek, a mint magatok is tudjátok.
\par 23 Azt, a ki Istennek elvégezett tanácsából és rendelésébõl adatott halálra, megragadván, gonosz kezeitekkel keresztfára feszítve megölétek:
\par 24 Kit az Isten feltámasztott, a halál fájdalmait megoldván; mivelhogy lehetetlen volt néki attól fogvatartatnia.
\par 25 Mert Dávid ezt mondja õ róla: Magam elõtt láttam az Urat mindenkor, mert õ nékem jobb kezem felõl van, hogy meg ne tántorodjam.
\par 26 Annakokáért örvendezett az én szívem, és vígadott az én nyelvem; annakfelette az én testem is reménységben nyugszik.
\par 27 Mert nem hagyod az én lelkemet a sírban, és nem engeded, hogy a te szented rothadást lásson.
\par 28 Megjelentetted nékem az életnek útait; betöltesz engem örömmel a te orczád elõtt.
\par 29 Atyámfiai férfiak, szabad nyilván szólanom ti elõttetek Dávid pátriárkáról, hogy õ megholt és eltemettetett, és az õ sírja mind e mai napig minálunk van.
\par 30 Próféta lévén azért, és tudván, hogy az Isten néki esküvéssel megesküdött, hogy majd az õ ágyékának gyümölcsébõl támasztja a Krisztust test szerint, hogy helyheztesse az õ királyi székibe,
\par 31 Elõre látván ezt, szólott a Krisztus feltámadásáról, hogy az õ lelke nem hagyatott a sírban, sem az õ teste rothadást nem látott.
\par 32 Ezt a Jézust feltámasztotta az Isten, minek mi mindnyájan tanúbizonyságai vagyunk.
\par 33 Annakokáért az Istennek jobbja által felmagasztaltatván, és a megígért Szent Lelket megnyervén az Atyától, kitöltötte ezt, a mit ti most láttok és hallotok.
\par 34 Mert nem Dávid ment fel a mennyországba; hiszen õ maga mondja: Monda az Úr az én Uramnak: Ülj az én jobbkezem felõl,
\par 35 Míglen vetem a te ellenségeidet lábaid alá zsámolyul.
\par 36 Bizonynyal tudja meg azért Izráelnek egész háza, hogy Úrrá és Krisztussá tette õt az Isten, azt a Jézust, a kit ti megfeszítettetek.
\par 37 Ezeket pedig mikor hallották, szívökben megkeseredének, és mondának Péternek és a többi apostoloknak: Mit cselekedjünk, atyámfiai, férfiak?
\par 38 Péter pedig monda nékik: Térjetek meg és keresztelkedjetek meg mindnyájan a Jézus Krisztusnak nevében a bûnöknek bocsánatjára; és veszitek a Szent Lélek ajándékát.
\par 39 Mert néktek lett az ígéret és a ti gyermekeiteknek, és mindazoknak, kik messze vannak, valakiket  csak elhív magának az Úr, a mi Istenünk.
\par 40 Sok egyéb beszéddel is buzgón kéri és inti vala õket, mondván: Szakaszszátok el magatokat e gonosz nemzetségtõl!
\par 41 A kik azért örömest vevék az õ beszédét, megkeresztelkedének; és hozzájuk csatlakozék azon a napon mintegy háromezer lélek.
\par 42 És foglalatosok valának az apostolok tudományában és a közösségben, a kenyérnek megtörésében és a könyörgésekben.
\par 43 Támada pedig minden lélekben félelem, és az apostolok sok csudát és jelt tesznek vala.
\par 44 Mindnyájan pedig, a kik hivének, együtt valának, és mindenük köz vala;
\par 45 És jószágukat és marháikat eladogaták, és szétosztogaták azokat mindenkinek, a mint kinek-kinek szüksége vala.
\par 46 És minden nap egyakarattal kitartva a templomban, és megtörve házanként a kenyeret, részesednek vala eledelben örömmel és tiszta szívvel.
\par 47 Dícsérve az Istent, és az egész nép elõtt kedvességet találva. Az Úr pedig minden napon szaporítja vala a gyülekezetet az idvezülõkkel.

\chapter{3}

\par 1 Péter és János pedig együtt mennek vala fel a templomba az imádkozásnak órájára, kilenczre.
\par 2 És hoznak vala egy embert, ki az õ anyjának méhétõl fogva sánata vala, kit minden nap le szoktak tenni a templom kapujánál, melyet Ékesnek neveznek, hogy kérjen alamizsnát azoktól, a kik bemennek a temlomba.
\par 3 Ez mikor látta, hogy Péter és János a templomba akarnak bemenni, kére õ tõlük alamizsnát.
\par 4 Péter pedig mikor szemeit reá vetette Jánossal egyben, monda: Nézz mi reánk!
\par 5 Az annakokáért figyelmez vala reájok, remélvén, hogy valamit kap tõlük.
\par 6 Péter pedig monda: Ezüstöm és aranyam nincsen nékem; hanem a mim van, azt adom néked: a názáreti Jézus Krisztus nevében, kelj fel és járj!
\par 7 És õt jobbkezénél fogva felemelé, és azonnal megerõsödének az õ lábai és bokái.
\par 8 És felszökvén, megálla és jár vala és beméne õ velök a templomba, járkálva és szökdelve és dícsérve az Istent.
\par 9 És látá õt az egész nép, hogy jár és dícséri az Istent:
\par 10 És megismerék õt, hogy õ volt az, a ki alamizsnáért ült a templomnak Ékeskapujában; és megtelének csodálkozással és azon való álmélkodással, a mi történt vala õ vele.
\par 11 Mikor pedig ragaszkodék Péterhez és Jánoshoz az a sánta, a ki meggyógyult, az egész nép álmélkodva összefuta õ hozzájok a tornáczba, mely Salamonénak neveztetik.
\par 12 Mikor pedig ezt látta Péter, monda a népnek: Izráel férfiai, mit csodálkoztok ezen, vagy mit néztek mi reánk, mintha tulajdon erõnkkel vagy jámborságunkkal míveltük volna azt, hogy az járjon?
\par 13 Az Ábrahámnak, Izsáknak és Jákóbnak Istene, a mi atyáinknak Istene megdicsõítette az õ Fiát, Jézust, kit ti elárulátok, és megtagadátok Pilátus elõtt, noha õ úgy ítélt, hogy elbocsátja.
\par 14 Ti pedig azt a szentet és igazat megtagadátok és kívánátok, hogy a gyilkos ember bocsáttassék el néktek,
\par 15 Az életnek fejedelmét pedig megölétek; kit az Isten feltámasztott a halálból, minek mi vagyunk bizonyságai.
\par 16 És az õ nevében való hit által erõsítette meg az õ neve ezt, a kit láttok és ismertek; és a hit, mely õ általa van, adta néki ezt az épséget mindnyájan a ti szemetek láttára.
\par 17 De most, atyámfiai, tudom, hogy tudatlanságból cselekedtetek, miképen a ti fejedelmeitek is.
\par 18 Az Isten pedig, a mikrõl eleve megmondotta minden õ prófétájának szája által, hogy a Krisztus elszenvedi, ekképen töltötte be.
\par 19 Bánjátok meg azért és térjetek meg, hogy eltöröltessenek a ti bûneitek, hogy így eljõjjenek a felüdülés  idei az Úrnak színétõl.
\par 20 És elküldje a Jézus Krisztust, a ki néktek elõre hirdettetett.
\par 21 Kit az égnek kell magába fogadnia mind az idõkig, míglen újjá teremtetnek mindenek, a mikrõl szólott az Isten minden õ szent prófétájának szája által eleitõl fogva.
\par 22 Mert Mózes ezt mondotta az atyáknak: Prófétát támaszt néktek az Úr, a ti Istenetek a ti atyátokfiai közül, mint engem; azt hallgassátok mindenben, a mit csak szólánd néktek.
\par 23 Lészen pedig, hogy minden lélek, valamely nem hallgatánd arra a prófétára, ki fog irtatni a nép közül.
\par 24 De a próféták is mindnyájan Sámueltõl és a következõktõl fogva, a kik csak szóltak, e napokról jövendöltek.
\par 25 Ti vagytok a prófétáknak és a szövetségnek fiai, melyet Isten szerzett a mi atyáinkkal, mondván Ábrahámnak: És a te magodban megáldatnak a földnek nemzetségei mindnyájan.
\par 26 Az Isten az õ Fiát, Jézust elsõ sorban néktek támasztván, elküldé õt, hogy megáldjon  titeket, mindegyikõtöket megtérítvén bûneitekbõl.

\chapter{4}

\par 1 Míg õk azonban a néphez szólottak, oda léptek hozzájok a papok és a templom felügyelõje és a  sadduczeusok,
\par 2 Neheztelve a miatt, hogy õk a népet tanítják, és hirdetik a Jézusban a halálból való feltámadást;
\par 3 És rájuk veték kezüket, és veték õket õrizet alá másnapig, mert már este vala.
\par 4 Sokan pedig azok közül, kik hallgaták az ígét, hivének; és lõn a férfiak száma mintegy ötezer.
\par 5 Lõn pedig, hogy másnapra egybegyûlének azoknak fejei, vénei és írástudói Jeruzsálembe.
\par 6 És Annás, a fõpap, és Kajafás és János és Sándor, és a kik csak fõpapi nemzetségbeliek valának.
\par 7 És mikor õket a középre állaták, tudakozzák vala: Micsoda hatalommal, vagy micsoda név által cselekedtétek ti ezt?
\par 8 Akkor Péter, Szent Lélekkel megtelve, monda nékik: Népnek fejedelmei és Izráelnek vénei!
\par 9 Ha e mai napon mi egy nyavalyás emberrel való jótétemény felõl hallgattatunk ki, mi által gyógyult meg ez:
\par 10 Legyen tudtotokra mindnyájatoknak és a Izráel egész népének, hogy a názáretbeli Jézus Krisztusnak neve által, a kit ti megfeszítettetek, kit Isten feltámasztott halottaiból, az által áll ez ti elõttetek épségben.
\par 11 Ez ama kõ, melyet ti építõk megvetettetek, mely lett a szegeletnek fejévé.
\par 12 És nincsen senkiben másban idvesség: mert nem is adatott emberek között az ég alatt más név, mely által kellene nékünk megtartatnunk.
\par 13 Mikor pedig látták Péternek és Jánosnak a szólásában való bátorságukat, és megértették, hogy írástudatlan és közönséges emberek, csodálkoznak vala; meg is ismerék õket, hogy Jézussal voltak vala.
\par 14 Mikor azonban látták, hogy a mely ember meggyógyult vala, õ velök együtt ott áll, semmit nem bírtak ellenök szólni.
\par 15 Mikor pedig õket a gyûlésbõl kiküldötték, tanácskoztak maguk közt, mondván:
\par 16 Mit cselekedjünk ez emberekkel? Mert hogy nyilvánvaló csoda lõn általok, mindazoknak, kik Jeruzsálemben laknak, tudtokra van, és el nem tagadhatjuk.
\par 17 De hogy tovább ne terjedjen a nép között, fenyegetéssel fenyegessük meg õket, hogy többé egy embernek se szóljanak ebben a névben.
\par 18 Azért beszólítván õket, megparancsolák nékik, hogy teljességgel ne szóljanak és ne tanítsanak a Jézus nevében.
\par 19 Péter és János pedig felelvén, mondának nékik: Vajjon igaz dolog-é Isten elõtt, rátok hallgatnunk inkább, hogynem Istenre, ítéljétek meg!
\par 20 Mert nem tehetjük, hogy a miket láttunk és hallottunk, azokat ne szóljuk.
\par 21 Amazok pedig nem találván semmi módot, hogyan büntessék meg õket, még megfenyegetvén, elbocsáták õket a nép miatt, mert mindnyájan dicsõítik vala az Istent azért, a mi történt.
\par 22 Mert több vala negyven esztendõsnél az az ember, kin a gyógyításnak ez a csodája lett vala.
\par 23 Mikor pedig elbocsáttattak, menének az övéikhez, és elbeszélék, a miket a fõpapok és a vének mondottak nékik.
\par 24 Ezek pedig mikor hallották, egy szívvel-lélekkel felemelék szavokat az Istenhez, és mondának: Urunk, te vagy az Isten, ki teremtetted a mennyet és a földet, a tengert és minden azokban levõ dolgot.
\par 25 Ki Dávidnak, a te szolgádnak szája által ezt mondottad: Miért zúgolódtak a pogányok, és gondoltak a népek hiábavalókat?
\par 26 Felállottak a földnek királyai, és a fejedelmek egybegyûltek az Úr ellen és az õ Krisztusa ellen.
\par 27 Mert bizony egybegyûltek a te szent Fiad, a Jézus ellen, a kit felkentél, Heródes és Ponczius Pilátus a pogányokkal és Izráel népével,
\par 28 Hogy véghezvigyék, a mikrõl a te kezed és a te tanácsod eleve elvégezte volt, hogy megtörténjenek.
\par 29 Most azért, Urunk, tekints az õ fenyegetéseikre: és adjad a te szolgáidnak, hogy teljes bátorsággal szólják a te beszédedet,
\par 30 A te kezedet kinyújtván gyógyításra; és hogy jelek és csodák történjenek a te szent Fiadnak, a Jézusnak neve által.
\par 31 És minekutána könyörögtek, megmozdula a hely, a hol egybegyûltek; és betelének mindnyájan Szent Lélekkel, és az Isten beszédét bátorsággal szólják vala.
\par 32 A hívõk sokaságának pedig szíve-lelke egy vala; és senki semmi marháját nem mondá magáénak, hanem nékik mindenök köz vala.
\par 33 És az apostolok nagy erõvel tesznek vala bizonyságot az Úr Jézus feltámadásáról; és nagy kegyelem vala mindnyájukon.
\par 34 Mert szûkölködõ sem vala õ közöttük senki; mert valakik földek vagy házak birtokosai voltak, eladván, elhozák az eladottak árát,
\par 35 És leveték az apostolok lábainál: aztán elosztatott az egyesek közt, a mint kinek-kinek szüksége vala.
\par 36 József is, ki az apostoloktól Barnabásnak neveztetett el (a mi megmagyarázva annyi, mint Vígasztalás Fia), Lévita, származása szerint ciprusi.
\par 37 Mivelhogy néki mezeje vala, eladván, a pénz elhozá, és az apostolok lábainál letevé.

\chapter{5}

\par 1 Egy ember azonban, névszerint Anániás, Safirával, az õ feleségével, eladá birtokát.
\par 2 És félre tõn az árából, feleségének is tudtával, és valami részét elvivén, az apostoloknak lábai elé letevé.
\par 3 Monda pedig Péter: Anániás, miért foglalta el a Sátán a te szívedet, hogy megcsald a Szent Lelket, és a mezõnek árából félre tégy?
\par 4 Nemde megmaradva néked maradt volna meg, és eladva a te hatalmadban volt? Miért hogy ezt a dolgot cselekedted szívedben? Nem embereknek hazudtál, hanem Istennek.
\par 5 Hallván pedig Anániás e szavakat, lerogyott és meghala; és mindenekben nagy félelem támada, kik ezeket hallják vala.
\par 6 Az ifjak pedig felkelvén, begöngyölék õt, és kivivén eltemeték.
\par 7 Történt aztán mintegy három órai szünet múlva, hogy az õ felesége, nem tudva, mi történt, beméne.
\par 8 Monda pedig néki Péter: Mondd meg nékem, vajjon ennyiért adtátok-é el a földet? Õ pedig monda: Igen, ennyiért.
\par 9 Péter pedig monda néki: Miért hogy megegyeztetek, hogy az Úrnak lelkét megkísértsétek? Ímé a küszöbön vannak azoknak lábaik, a kik eltemették férjedet, és kivisznek téged.
\par 10 És azonnal összerogyott lábainál, és meghala; bemenvén pedig az ifjak, halálva találák õt, és kivivén eltemeték férje mellé.
\par 11 És támada nagy félelem az egész gyülekezetben és mindazokban, kik ezeket hallják vala.
\par 12 Az apostolok kezei által pedig sok jel és csoda lõn a nép között; és egyakarattal mindnyájan a Salamon tornáczában valának.
\par 13 Egyebek közül pedig senki sem mert közéjük elegyedni: hanem a nép magasztalá õket;
\par 14 Hívõk pedig mindinkább csatlakoztak az Úrhoz, úgy férfiaknak, mint asszonyoknak sokasága.
\par 15 Úgyannyira, hogy az utczákra hozák ki a betegeket, és letevék ágyakon és nyoszolyákon, hogy az arra menõ Péternek csak árnyéka is érje valamelyiket közülök,
\par 16 És a szomszéd városok sokasága is Jeruzsálembe gyûlt, hozva betegeket és tisztátalan lelkektõl gyötretteket: kik mind meggyógyulának.
\par 17 De felkelvén a fõpap és mind a kik vele valának, azaz a sadduczeusok felekezete, betelének irigységgel,
\par 18 És ráveték kezöket az apostolokra, és a közönséges tömlöczbe tevék õket.
\par 19 Hanem az Úrnak angyala éjszaka megnyitá a tömlöcz ajtaját, és kihozván õket, monda:
\par 20 Menjetek el, és felállván, hirdessétek a templomban a népnek ez életnek minden beszédit!
\par 21 Azok pedig ezt hallván, bemenének jó reggel a templomba, és tanítának. A fõpap pedig elmenvén és a vele levõk, egybehívák a gyûlést, és Izráel fiainak egész tanácsát, és küldének a tömlöczbe, hogy azokat elõhozzák.
\par 22 Mikor azonban a poroszlók oda mentek, nem találák õket a tömlöczben; visszatérvén tehát, megjelenték,
\par 23 Mondván: A tömlöczöt ugyan nagy erõsen bezárva találtuk, és az õröket kívül az ajtó elõtt állva; mikor azonban kinyitottuk, ott benn senkit sem találánk.
\par 24 A mint pedig hallották e szavakat a pap és a templom felügyelõje és a fõpapok, zavarban  voltak azok miatt, mi lehet ez?
\par 25 Eljövén pedig valaki, hírül adá nékik, mondván: Ímé, ama férfiak, kiket a tömlöczbe vetettetek, a templomban állanak és tanítják a népet.
\par 26 Akkor elmenvén a felügyelõ a poroszlókkal, elõhozá õket erõszak nélkül; féltek ugyanis a néptõl, hogy megkövezi õket.
\par 27 Elõhozván pedig õket, állaták a tanács elé; és megkérdé õket a fõpap,
\par 28 Mondván: Nem megparancsoltuk-é néktek parancsolattal, hogy ne tanítsatok ebben a névben? És ímé betöltöttétek Jeruzsálemet tudományotokkal, és mi reánk akarjátok hárítani annak az embernek  vérét.
\par 29 Felelvén pedig Péter és az apostolok, mondának: Istennek kell inkább engedni, hogynem az embereknek.
\par 30 A mi atyáinknak Istene feltámasztotta Jézust, kit ti fára függesztve megölétek.
\par 31 Ezt az Isten fejedelemmé és megtartóvá emelte jobbjával, hogy adjon az Izráelnek bûnbánatot  és bûnöknek bocsánatát.
\par 32 És mi vagyunk néki bizonyságai ezen beszédek felõl, és a Szent Lélek is, kit  Isten adott azoknak, a kik néki engednek.
\par 33 Azok pedig ezeket hallván, fogukat csikorgaták, és arról tanácskozának, hogy megölik õket.
\par 34 Felkelvén azonban a tanácsban egy farizeus, névszerint Gamáliel, az egész nép elõtt tisztelt törvénytudó, parancsolá, hogy egy kis idõre vezessék ki az apostolokat.
\par 35 És monda azoknak: Izráel férfiai, vigyázzatok magatokra ez emberekkel szemben, mit akartok cselekedni!
\par 36 Mert ez idõnek elõtte felkelt Theudás, azt mondván, hogy õ valaki, kihez mintegy négyszáz embernyi tömeg csatlakozott; õ megöletett, és mindnyájan, a kik csak követték õt, eloszlottak és semmivé lettek.
\par 37 Ezután felkelt ama Galileus Júdás az összeírás idején, és sok népet maga után csábított: ez is elveszett; és mindazok, a kik õt követték, szétszórattak.
\par 38 Mostanra nézve is mondom néktek, álljatok el ez emberektõl, és hagyjatok békét nékik: mert ha emberektõl van e tanács, vagy e dolog, semmivé lesz;
\par 39 Ha pedig Istentõl van, ti fel nem bonthatjátok azt; nehogy esetleg Isten ellen harczolóknak is találtassatok.
\par 40 Engedének azért néki; és miután elõszólították az apostolokat, megveretvén, megparancsolák, hogy a Jézus nevében  ne szóljanak, és elbocsáták õket.
\par 41 Õk annakokáért örömmel menének el a tanács elõl, hogy méltókká tétettek arra, hogy az õ nevéért gyalázattal illettessenek.
\par 42 És mindennap a templomban és házanként nem szûnnek vala meg tanítani és hirdetni Jézust, a Krisztust.

\chapter{6}

\par 1 Azokban a napokban pedig, mikor a tanítványok szaporodának, támada a görög zsidók közt panaszolkodás a héberek ellen, hogy az õ közülük való özvegyasszonyok mellõztetnek a mindennapi szolgálatban.
\par 2 Annakokáért a tizenkettõ egybegyûjtvén a tanítványok sokaságát, mondának: Nem helyes, hogy mi az Isten ígéjét elhagyjuk és az asztalok körül szolgáljunk.
\par 3 Válaszszatok azért, atyámfiai, ti közületek hét férfiút, kiknek jó bizonyságuk van, kik Szent Lélekkel és bölcseséggel teljesek, kiket erre a foglalatosságra beállítsunk.
\par 4 Mi pedig foglalatosok maradunk a könyörgésben és az ígehirdetés szolgálatában.
\par 5 És tetszék e beszéd az egész sokaságnak: és kiválaszták Istvánt, ki hittel és Szent Lélekkel teljes férfiú vala, Filepet, Prokhórust, Nikánórt, Timónt, Párménást és Nikolaust, ki Antiókhiából való prozelitus vala;
\par 6 Kiket állatának az apostolok elébe; és miután imádkoztak, kezeiket reájok veték.
\par 7 És az Isten ígéje növekedék; és sokasodék nagyon a tanítványok száma Jeruzsálemben; és a papok közül is nagy sokan követék a hitet.
\par 8 István pedig teljes lévén hittel és erõvel, nagy csodákat és jeleket cselekszik vala a nép között.
\par 9 Elõállának azonban némelyek ahhoz a zsinagógához tartozók közül, mely a szabadosokénak, Czirénebeliekének, Alexandriabeliekének és a Czilicziából és Ázsiából valókénak neveztetett, kik Istvánnal vetekednek vala.
\par 10 De nem állhattak ellene a bölcseségnek és a Léleknek, mely által szól vala.
\par 11 Akkor felbujtottak valami embereket, kik mondának: Hallottuk õt káromló beszédeket szólni Mózes ellen és az Isten ellen.
\par 12 És felzendíték a népet, a véneket és az írástudókat; és reá rohanván, magukkal ragadák õt, és vivék a tanács elé;
\par 13 És állatának hamis tanúkat, kik mondának: Ez az ember nem szûnik meg káromló beszédeket szólni e szent hely ellen és a törvény ellen:
\par 14 Mert hallottuk, amint azt mondá, hogy az a názáreti Jézus ezt a helyet elrontja, és megváltoztatja a czerimóniákat, melyeket adott nékünk Mózes.
\par 15 És szemeiket reá vetvén a tanácsban ûlõk mindnyájan, olyannak láták az õ orczáját, mint egy angyalnak orczáját.

\chapter{7}

\par 1 Monda pedig a fõpap: Vajjon így vannak-é hát ezek?
\par 2 Õ pedig monda: Férfiak, atyámfiai és atyák, halljátok meg! A dicsõségnek Istene megjelenék a mi atyánknak, Ábrahámnak, mikor Mezopotámiában vala, minekelõtte Háránban lakott,
\par 3 És monda néki: Eredj ki a te földedbõl és a te nemzetséged közül, és jer arra a földre, a melyet mutatok néked.
\par 4 Akkor kimenvén a Káldeusok földébõl, lakozék Háránban: és onnét, minekutána megholt az õ atyja, kihozta õt e földre, a melyen ti most laktok:
\par 5 És nem adott néki abban örökséget csak egy lábnyomnyit is: és azt ígérte, hogy néki adja azt birtokul és az õ magvának õ utána, holott nem vala néki gyermeke.
\par 6 Szólt pedig az Isten akképen, hogy az õ magva zsellér lészen idegen földön, és szolgálat alá vetik azt, és nyomorgatják, négyszáz esztendeig.
\par 7 De azt a népet, melynek szolgálnak, én megítélem, monda az Isten: és ezek után kijõnek, és szolgálnak nékem e helyen.
\par 8 És adta néki a körülmetélés szövetségét: és így nemzé Izsákot,  és körülmetélé õt nyolczadnapon; és Izsák Jákóbot, és Jákób a  tizenkét pátriárkhát.
\par 9 A pátriárkhák pedig irígységbõl eladák Józsefet Égyiptomba; de Isten vele vala,
\par 10 És megszabadítá õt minden nyomorúságából, és ada néki kedvességet és bölcseséget a Faraó elõtt, Égyiptom királya elõtt, ki õt Égyiptom fölé és az õ egész háza fölé kormányzóul állatá.
\par 11 Következék pedig éhség Égyiptom és Kanaán egész földére, és nagy nyomorúság; és nem találnak vala eledelt a mi atyáink.
\par 12 Mikor pedig meghallotta Jákób, hogy Égyiptomban van gabona, elküldé elõször a mi atyáinkat.
\par 13 És második alkalommal fölismerék Józsefet testvérei, és a Faraó megtudá a József nemzetségét.
\par 14 És József elküldvén, magához hívatá az õ atyját, Jákóbot, és egész hetvenöt lélekbõl álló nemzetségét.
\par 15 Leméne azért Jákób Égyiptomba, és maghala  õ és a mi atyáink;
\par 16 És elvitetének Sikembe, és helyheztetének a sírba, melyet Ábrahám vett vala ezüstpénzen, Emmórnak, a Sikem atyjának fiaitól.
\par 17 Mikor pedig elközelgetett az ígéretnek ideje, melyet Isten esküvel ígért Ábrahámnak, megnevekedék a nép és  megsokasodék Égyiptomban,
\par 18 Mindaddig, mígnem más király támada, ki nem ismeri vala Józsefet.
\par 19 Ez a mi nemzetségünkkel álnokul bánva nyomorgatta a mi atyáinkat, hogy magzataikat kitétesse, hogy életben ne maradjanak.
\par 20 Akkor születék Mózes, és ékes vala az Isten elõtt. Ez három hónapig atyja házában tartaték.
\par 21 Mikor pedig kitétetett, a Faraó leánya felvevé, és felnevelé õt a saját fia gyanánt.
\par 22 És Mózes taníttaték az Égyiptombeliek minden bölcseségére; és hatalmas vala beszédben és cselekedetben.
\par 23 Mikor pedig negyvenéves kora betölt, eszébe jutott, hogy meglátogassa atyjafiait, az Izráel fiait.
\par 24 És mikor látta, hogy egyik bántalommal illettetik, megoltalmazá, és az égyiptomi embert megölvén, bosszút álla azért, a ki bosszúsággal illettetett.
\par 25 És azt gondolá, hogy az õ atyjafiai megértik, hogy az Isten az õ keze által ád nékik szabadulást; de azok nem értették meg.
\par 26 Másnap meg olyankor jelent meg köztük, mikor összevesztek, és inté õket békességre, mondván: Férfiak, testvérek vagytok ti; miért illetitek egymást bosszúsággal?
\par 27 De az, a ki felebarátját bántalmazta, elutasítá õt magától, mondván: Kicsoda tett téged fejedelemmé és bíróvá mi rajtunk?
\par 28 Csak nem akarsz engem is megölni, miképen megöléd tegnap az égyiptomit?
\par 29 E beszédre aztán Mózes elfuta és lõn jövevény a Midián földén, a hol két fia születék.
\par 30 És negyven esztendõ elteltével megjelenék néki a Sínai hegy pusztájában az Úrnak angyala csipkebokornak tüzes lángjában.
\par 31 Mózes pedig mikor meglátta, elcsodálkozék a látáson. Mikor pedig oda méne, hogy megszemlélje, lõn az Úrnak szava õ hozzá:
\par 32 Én vagyok a te atyáidnak Istene, Ábrahámnak Istene, és Izsáknak Istene, és Jákóbnak Istene. Mózes pedig megrémülvén, nem meré megnézni.
\par 33 Az Úr pedig monda néki: Oldozd le sarudat lábaidról; mert a hely, a melyen állasz, szent föld.
\par 34 Látván láttam az én népemnek nyomorúságát, mely Égyiptomban van, és az õ fohászkodásukat meghallgattam, és azért szállottam le, hogy õket megszabadítsam; most azért jõjj, elküldelek téged Égyiptomba.
\par 35 Ezt a Mózest, a kit megtagadának, mondván: Ki tett téged fejedelemmé és bíróvá? ezt az Isten fejedelmül és szabadítóul küldé angyal keze által, a ki megjelent néki a  csipkebokorban.
\par 36 Ez hozta ki õket, csodákat és jeleket tévén Égyiptomnak földében és a Verestengeren  és a pusztában negyven esztendeig.
\par 37 Ez ama Mózes, ki az Izráel fiainak ezt mondotta: Prófétát támaszt néktek az Úr, a ti Istentek, a ti atyátokfiai közül, mint engem: azt hallgassátok.
\par 38 Ez az, aki ott volt a gyülekezetben a pusztában a Sinai hegyen vele beszélõ angyallal és a mi atyáinkkal: ki élõ  igéket võn, hogy nékünk adja;
\par 39 A kinek nem akartak engedni a mi atyáink, hanem eltaszíták maguktól, és szívökben Égyiptom felé fordulának,
\par 40 Ezt mondván Áronnak: Csinálj nékünk isteneket, kik elõttünk járjanak: mert ez a Mózes, ki minket Égyiptom földébõl kihozott, nem tudjuk, mi történt õ vele.
\par 41 És borjúképet csinálának azokban a napokban, és áldozatot vivének a bálványnak, és gyönyörködének az õ kezeik csinálmányaiban.
\par 42 Az Isten pedig elfordula, és adá õket, hogy szolgáljanak az ég seregének; a mint meg van írva  a próféták könyvében: Vajjon áldozati barmokat és áldozatokat hoztatok-é nékem negyven esztendeig a pusztában, Izráelnek háza?
\par 43 Sõt inkább hordoztátok a Molok sátorát, és a ti istenteknek, Remfánnak csillagát, a képeket, melyeket csináltatok, hogy azokat imádjátok: elviszlek azért titeket Babilónon túl.
\par 44 A bizonyságnak sátora a mi atyáinknál volt a pusztában, a mint parancsolta az, a ki mondotta Mózesnek, hogy azt arra a mintára csinálja, melyet látott vala.
\par 45 Melyet a mi atyáink átvévén, be is hoztak Józsuéval, mikor birodalmukba vették a pogányokat, kiket kiûzött az Isten a mi atyáink színe elõl, mind a Dávidnak napjaiig;
\par 46 Ki kegyelmet talált az Isten elõtt, és könyörgött, hogy hajlékot  találhasson a Jákób Istenének.
\par 47 Salamon építe pedig néki házat.
\par 48 De ama Magasságos nem kézzel csinált templomokban lakik, mint a próféta mondja:
\par 49 A menny nékem ülõszékem, a föld pedig az én lábaimnak zsámolya; micsoda házat építhettek nékem? azt mondja az Úr, vagy melyik az én nyugodalmamnak helye?
\par 50 Nem az én kezem csinálta-é mindezeket?
\par 51 Kemény nyakú és körülmetéletlen szívû és fülû emberek, ti mindenkor a Szent Léleknek ellene igyekeztek, mint atyáitok, ti azonképen.
\par 52 A próféták közül kit nem üldöztek a ti atyáitok? és megölték azokat, a kik eleve hirdették amaz Igaznak eljövetelét: kinek ti most árulóivá és gyilkosaivá  lettetek;
\par 53 Kik a törvényt angyalok rendelésére vettétek, és nem tartottátok meg.
\par 54 Mikor pedig ezeket hallották, szívükben dühösködnek és fogaikat csikorgatják vala õ ellene.
\par 55 Mivel pedig teljes vala Szent Lélekkel, a mennybe függesztvén szemeit, látá Istennek dicsõségét, és Jézust állani az Istennek jobbja  felõl,
\par 56 És monda: Ímé látom az egeket megnyilni, és az embernek Fiát az Isten jobbja felõl állani.
\par 57 Felkiáltván pedig nagy fenszóval, füleiket bédugák, és egyakarattal reá rohanának;
\par 58 És kiûzvén a városon kívül, megkövezék: a tanúbizonyságok  pedig felsõruháikat egy Saulus nevezetû ifjú lábaihoz rakták le.
\par 59 Megkövezék azért Istvánt, ki imádkozik és ezt mondja vala: Uram Jézus, vedd magadhoz az én lelkemet!
\par 60 Térdre esvén pedig, nagy fenszóval kiálta: Uram, ne tulajdonítsd nékik e bûnt! És ezt mondván, elaluvék.

\chapter{8}

\par 1 Saulus pedig szintén javallta az õ megöletését. És támada azon a napon nagy üldözés a jeruzsálemi gyülekezet ellen, és mindnyájan eloszlának  Júdeának és Samáriának tájaira, az apostolokat kivéve.
\par 2 Istvánt pedig eltakaríták kegyes férfiak, és nagy sírást tõnek õ rajta.
\par 3 Saulus pedig pusztítá az anyaszentegyházat, házról-házra járva, és férfiakat és asszonyokat elõvonszolva, tömlöczbe veti vala.
\par 4 Amazok annakokáért eloszolván, széjjeljártak, hirdetve az ígét.
\par 5 És Filep lemenvén Samária városába, prédikálja vala nékik a Krisztust.
\par 6 A sokaság pedig egy szívvel-lélekkel figyelmeze azokra, a miket Filep mondott, hallván és látván a jeleket, melyeket cselekedék.
\par 7 Mert sokakból, kikben tisztátalan lelkek voltak, nagy hangon kiáltva kimenének; sok gutaütött és sánta pedig meggyógyula.
\par 8 És lõn nagy öröm abban a városban.
\par 9 Egy Simon nevû ember pedig már elõbb gyakorolta abban a városban az ördögi tudományt és elámította Samária népét, magát valami nagynak állítván:
\par 10 Kire mindnyájan figyeltek, kicsinytõl nagyig, mondván: Ez az Istennek ama nagy ereje!
\par 11 Azért figyeltek pedig rá, mert sok idõn át az ördögi mesterségekkel elámította õket.
\par 12 De miután hittek Filepnek, a ki az Isten országára és a Jézus Krisztus nevére tartozó örvendetes dolgokat hirdeti vala, megkeresztelkedének mind férfiak, mind asszonyok.
\par 13 És Simon maga is hûn, és megkeresztelkedvén, Fileppel tarta; és látván, hogy jelek és nagy erõk lesznek, álmélkodik vala.
\par 14 Mikor pedig meghallották a jeruzsálemi apostolok, hogy Samária bevette az Isten ígéjét, elküldék azokhoz Pétert és Jánost;
\par 15 Kik mikor lementek, könyörögtek érettük, hogy vegyenek Szent Lelket:
\par 16 Mert még senkire azok közül nem szállott rá, csak meg voltak keresztelve az Úr Jézus nevére.
\par 17 Akkor kezeiket reájuk veték, és võnek Szent Lelket.
\par 18 Mikor pedig látta Simon, hogy az apostolok kézrátétele által adatik a Szent Lélek, megkínálá õket pénzzel,
\par 19 Mondván: Adjátok nékem is ezt a hatalmat, hogy valakire vetem kezeimet, Szent Lelket vegyen.
\par 20 De Péter monda néki: A te pénzed veled együtt veszszen el, mivel azt gondoltad, hogy az Istennek ajándéka pénzen megvehetõ.
\par 21 Nincsen néked részed, sem örökséged e dologban, mert a te szíved nem igaz az Isten elõtt.
\par 22 Térj meg azért ezen gonoszságodból, és kérjed az Istent, ha talán megbocsáttatik néked szívednek gondolatja.
\par 23 Mert látom, hogy te keserûséges méregben és álnokságnak kötelékében leledzel.
\par 24 Felelvén pedig Simon, monda: Könyörögjetek ti énérettem az Úrnak, hogy semmi azokból, a miket mondtatok, reám ne jõjjön.
\par 25 Azok annakokáért, minekutána bizonyságot tettek, és hirdették az Úrnak ígéjét, megtérének Jeruzsálembe, és a Samaritánusoknak sok falujában prédikálák az evangyéliomot.
\par 26 Az Úrnak angyala pedig szóla Filepnek, mondván: Kelj fel és menj el dél felé, arra az útra, mely Jeruzsálembõl Gázába megy alá. Járatlan ez.
\par 27 És felkelvén, elméne. És ímé egy szerecsen férfiú, Kandakénak, a szerecsenek királyasszonyának hatalmas komornyikja, ki az õ egész kincstárának felügyelõje vala, ki feljött imádkozni Jeruzsálembe;
\par 28 És visszatérõben volt és az õ szekerén ül vala, és olvasá Ésaiás prófétát.
\par 29 Monda pedig a Lélek Filepnek: Járulj oda és csatlakozzál ehhez a szekérhez!
\par 30 Filep azért oda futamodván, hallá, a mint az Ésaiás prófétát olvassa vala. És monda: Vajjon érted-é, a mit olvasol?
\par 31 Õ pedig monda: Mimódon érthetném, ha csak valaki meg nem magyarázza nékem? És kéré Filepet, hogy felhágván, üljön mellé.
\par 32 Az írásnak helye pedig, melyet olvasott, ez vala: Mint juh viteték mészárszékre, és mint a bárány az õ nyírõje elõtt néma, azonképen nem nyitotta fel az õ száját.
\par 33 Az õ megaláztatásában az õ ítélete elvétetett, az õ nemzetségét pedig kicsoda sorolja el? mert elvétetik a földrõl az õ élete.
\par 34 Felelvén pedig a komornyik Filepnek, monda: Kérlek téged, kirõl mondja ezt a próféta? Magáról-é, vagy más valakirõl?
\par 35 Filep pedig száját megnyitván, és elkezdvén ezen az íráson, hirdeté néki a Jézust.
\par 36 Mikor pedig menének az úton, jutának egy vízhez; és monda a komornyik: Ímhol a víz: mi gátol, hogy megkeresztelkedjem?
\par 37 Filep pedig monda: Ha teljes szívbõl hiszel, meglehet. Az pedig felelvén, monda: Hiszem, hogy a Jézus Krisztus az Isten Fia.
\par 38 És megállítá a szekeret; és leszállának mindketten a vízbe, Filep és a komornyik; és megkeresztelé õt.
\par 39 Mikor pedig a vízbõl feljöttek, az Úrnak Lelke elragadá Filepet; és többé nem látta õt a komornyik, mert tovább méne az õ útján örömmel.
\par 40 Filep pedig találtaték Azótusban; és széjjeljárva hirdeté az evangyéliomot minden városnak, míglen Czézáreába juta.

\chapter{9}

\par 1 Saulus pedig még fenyegetéstõl és öldökléstõl lihegve az Úrnak tanítványai ellen, elmenvén a fõpaphoz,
\par 2 Kére õ tõle leveleket Damaskusba a zsinagógákhoz, hogy ha talál némelyeket, kik ez útnak követõi, akár férfiakat, akár asszonyokat, fogva vigye Jeruzsálembe.
\par 3 És a mint méne, lõn, hogy közelgete Damaskushoz, és nagy hirtelenséggel fény sugárzá õt körül a mennybõl:
\par 4 És õ leesvén a földre, halla szózatot, mely ezt mondja vala néki: Saul, Saul, mit kergetsz engem?
\par 5 És monda: Kicsoda vagy, Uram? Az Úr pedig monda: én vagyok Jézus, a kit te kergetsz: nehéz néked az ösztön ellen rúgódoznod.
\par 6 Remegve és ámulva monda: Uram, mit akarsz, hogy cselekedjem? Az Úr pedig monda néki: Kelj fel és menj be a városba, és majd megmondják néked, mit kell cselekedned.
\par 7 A vele utazó férfiak pedig némán álltak, hallva ugyan a szót, de senkit sem látva.
\par 8 Felkele azonban Saulus a földrõl; de mikor felnyitá szemeit, senkit sem láta, azért kézenfogva vezeték be õt Damaskusba.
\par 9 És három napig nem látott, és nem evett és nem ivott.
\par 10 Vala pedig egy tanítvány Damaskusban, névszerint Ananiás, és monda annak az Úr látásban: Ananiás! Az pedig monda: Ímhol vagyok Uram!
\par 11 Az Úr pedig monda néki: Kelj fel és menj el az úgynevezett Egyenes utczába, és keress föl a Júdás házában egy Saulus nevû tárzusi embert, mert ímé imádkozik.
\par 12 És látá Saulus látásban, hogy egy Ananiás nevû férfiú beméne hozzá és kezét reá veté, hogy lásson.
\par 13 Felele pedig Ananiás: Uram, sok embertõl hallottam e férfiú felõl, mily sok bosszúsággal illeté a te szenteidet Jeruzsálemben:
\par 14 És itt is hatalma van a fõpapoktól, hogy mindazokat megkötözze, kik a te nevedet segítségül hívják.
\par 15 Monda pedig néki az Úr: Eredj el, mert õ nékem választott edényem, hogy hordozza az én nevemet a pogányok  és királyok, és Izráel  fiai elõtt.
\par 16 Mert én megmutatom néki, mennyit kell néki az én nevemért szenvedni.
\par 17 Elméne azért Ananiás és beméne a házba, és kezeit reá vetvén, monda: Saul atyámfia, az Úr küldött engem, Jézus, a ki megjelent néked az úton, melyen jöttél, hogy szemeid megnyiljanak és beteljesedjél Szent Lélekkel.
\par 18 És azonnal mintegy pikkelyek estek le szemeirõl, és mindjárt visszanyeré látását; és felkelvén, megkeresztelkedék;
\par 19 És miután evett, megerõsödék. Vala pedig Saulus a damaskusi tanítványokkal néhány napig.
\par 20 És azonnal prédikálá a zsinagógákban a Krisztust, hogy õ az Isten Fia.
\par 21 Álmélkodnak vala pedig mindnyájan, a kik hallák, és mondának: Nem ez-é az, a ki pusztította Jeruzsálemben azokat, a kik ezt a nevet hívják segítségül, és ide is azért jött, hogy õket fogva vigye a fõpapokhoz?
\par 22 Saulus pedig annál inkább erõt võn, és zavarba hozta a Damaskusban lakó zsidókat, bebizonyítván, hogy ez a Krisztus.
\par 23 Több nap elteltével azonban a zsidók tanácsot tartának, hogy õt megöljék:
\par 24 De tudtára esék Saulusnak az õ leselkedésök. És õrizék a kapukat mind nappal, mind éjjel, hogy õt megöljék;
\par 25 A tanítványok azért vevén õt éjjel, a kõfalon bocsáták alá, leeresztve egy kosárban.
\par 26 Mikor pedig Saulus Jeruzsálembe ment, a tanítványokhoz próbált csatlakozni; de mindnyájan féltek tõle, nem hivén, hogy õ tanítvány.
\par 27 Barnabás azonban maga mellé vevén õt, vivé az apostolokhoz, és elbeszélé nékik, mint látta az úton az Urat, és hogy beszélt vele, és mint tanított Damaskusban nagy bátorsággal a Jézus nevében.
\par 28 És ki- és bejáratos vala köztük Jeruzsálemben:
\par 29 És nagy bátorsággal tanítván az Úr Jézusnak nevében, beszél, sõt vetekedik vala a görög zsidókkal; azok pedig igyekeznek vala õt megölni.
\par 30 Megtudván azonban az atyafiak, levivék õt Czézáreába, és elküldék õt Tárzusba.
\par 31 A gyülekezeteknek tehát egész Júdeában, Galileában és Samariában békességök vala; épülvén és járván az Úrnak félelmében és a Szent Léleknek vígasztalásában, sokasodnak vala.
\par 32 Lõn pedig, hogy Péter, mikor mindnyájukat bejárá, leméne a Liddában lakozó szentekhez is.
\par 33 Talála pedig ott egy Éneás nevû embert, ki nyolcz esztendõ óta ágyban fekszik vala, ki gutaütött vala.
\par 34 És monda néki Péter: Éneás, gyógyítson meg téged a Jézus Krisztus: kelj föl, vesd meg magad az ágyadat! És azonnal felkele.
\par 35 És láták õt mindnyájan, kik laknak vala Liddában és Sáronban, kik megtérének az Úrhoz.
\par 36 Joppéban pedig vala egy nõtanítvány, névszerint Tábitha, mely megmagyarázva Dorkásnak, azaz zergének mondatik: ez gazdag vala jó cselekedetekben és alamizsnákban, melyeket osztogatott.
\par 37 Lõn pedig azokban a napokban, hogy megbetegedvén, meghala: és miután megmosták õt, kiteríték a felházban.
\par 38 Mivelhogy pedig Lidda Joppéhoz közel vala, a tanítványok meghallván, hogy Péter ott van, küldének két férfiút õ hozzá, kérve, hogy késedelem nélkül menjen át hozzájuk.
\par 39 Felkelvén azért Péter, elméne azokkal. Mihelyt oda ére, felvezeték õt a felházba: és elébe állának néki az özvegyasszonyok és mindnyájan sírva és mutogatva a ruhákat és öltözeteket, melyeket Dorkás csinált, míg velük együtt volt.
\par 40 Péter pedig mindenkit kiküldvén, térdre esve imádkozék; és a holt testhez fordulván, monda: Tábitha, kelj  fel! Az pedig felnyitá szemeit; és meglátván Pétert, felüle.
\par 41 És az kezét nyújtva néki, felemelé õt; és beszólítván a szenteket és az özvegyasszonyokat, eleikbe állatá õt elevenen.
\par 42 És tudtára lõn az egész Joppénak; és sokan hivének az Úrban.
\par 43 És lõn, hogy õ több napig marada Joppéban egy Simon nevû tímárnál.

\chapter{10}

\par 1 Vala pedig Czézáreában egy Kornélius nevû férfiú, százados az úgynevezett itáliai seregbõl.
\par 2 Jámbor és istenfélõ egész házanépével egybe, ki sok alamizsnát osztogat vala a népnek, és szüntelen könyörög vala Istennek.
\par 3 Ez látá látásban világosan, a napnak mintegy kilenczedik órája körül, hogy az Istennek angyala beméne õhozzá, és monda néki: Kornélius!
\par 4 Õ pedig szemeit reá függesztve és megrémülve monda: Mi az, Uram? Az pedig monda néki: A te könyörgéseid és alamizsnáid felmentek Isten elébe emlékezetnek okáért.
\par 5 Most azért küldj Joppéba embereket, és hivasd magadhoz Simont, ki neveztetik Péternek;
\par 6 Õ egy Simon nevû tímárnál van szálláson, kinek háza a tenger mellett van. Õ megmondja néked, mit kell cselekedned.
\par 7 A mint pedig elment az angyal, a ki Kornéliussal beszélt, szólíta kettõt az õ szolgái közül, és egy kegyes vitézt azok közül, kik rendelkezésére állnak vala.
\par 8 És elmondván nékik mindent, elküldé õket Joppéba.
\par 9 Másnap pedig, míg azok menének és közelgetének a városhoz, felméne Péter a háznak felsõ részére imádkozni hat óra tájban.
\par 10 Megéhezék azonban, és akara enni: míg azonban azok ételt készítének, szálla õ reá elragadtatás;
\par 11 És látá, hogy az ég megnyilt és leszálla õ hozzá valami edény, mint egy nagy lepedõ, négy sarkánál fogva felkötve, és leeresztve a földre:
\par 12 Melyben valának mindenféle földi négylábú állatok, vadak, csúszómászó állatok és égi madarak.
\par 13 És szózat lõn õ hozzá: Kelj fel Péter, öljed és egyed.
\par 14 Péter pedig monda: Semmiképen sem, Uram; mert sohasem ettem semmi közönségest, vagy tisztátalant.
\par 15 És is szózat lõn õ hozzá másodszor is: A miket az Isten megtisztított, te ne mondd tisztátalanoknak.
\par 16 Ez pedig három ízben történt; és ismét felviteték az edény az égbe.
\par 17 A mint pedig Péter magában tünõdék, mi lehet az a látás, a melyet látott, ímé az férfiak, kiket Kornélius küldött, megtudakozván a Simon házát, odaérkezének a kapuhoz,
\par 18 És bekiáltván megtudakozák, vajjon Simon, ki neveztetik Péternek, ott van-é szálláson?
\par 19 És a míg Péter a látás felõl gondolkodék, monda néki a Lélek: Ímé három férfiú keres téged:
\par 20 Nosza kelj fel, eredj alá, és minden kételkedés nélkül menj el õ velök: mert én küldöttem õket.
\par 21 Alámenvén azért Péter a férfiakhoz, kiket Kornélius küldött õ hozzá, monda: Ímé, én vagyok, a kit kerestek: mi dolog az, a miért jöttetek?
\par 22 Õk pedig mondának: Kornélius százados, igaz és istenfélõ férfiú, ki mellett a zsidók egész népe jó bizonyságot tesz, szent angyal által megintetett, hogy hívasson téged házához, és halljon tõled valami dolgokról.
\par 23 Behíván azért õket, szállására fogadá. Másnap pedig elméne Péter õ velök, és a Joppébeli atyafiak közül is némelyek együtt menének õ vele.
\par 24 És másnap eljutának Czézáreába. Kornélius pedig várja vala õket, egybegyûjtvén rokonait és jó barátait.
\par 25 És lõn, hogy a mint Péter beméne, Kornélius elébe menvén, lábaihoz borulva imádá õt.
\par 26 Péter azonban felemelé õt, mondván: Kelj fel; én magam is ember vagyok.
\par 27 És beszélgetve vele, belépett, és talála sokakat egybegyûlve;
\par 28 És monda nékik: Ti tudjátok, hogy tilalmas zsidó embernek más nemzetbelivel barátkozni, vagy hozzámenni; de nékem az Isten megmutatá, hogy senkit se mondjak közönséges, vagy tisztátalan embernek:
\par 29 Annak okáért ellenmondás nélkül el is jöttem, miután meghívattam. Azt kérdem azért, mi okból hivattatok engem?
\par 30 És Kornélius monda: Negyednaptól fogva mind ez óráig bõjtöltem, és kilencz órakor imádkozám az én házamban; és ímé egy férfiú álla meg elõttem fényes ruhában,
\par 31 És monda: Kornélius, meghallgattatott a te imádságod, és a te alamizsnáid emlékezetbe jutottak Isten elõtt.
\par 32 Küldj el azért Joppéba, és hívasd magadhoz simont, ki Péternek neveztetik; ez Simon tímár házában van szálláson a tenger mellett: õ, minekutána eljõ, szól néked.
\par 33 Azonnal azért küldöttem hozzád; és te jól tetted, hogy eljöttél. Most azért mi mindnyájan az Isten elõtt állunk, hogy meghallgassuk mindazokat, a miket Isten néked parancsolt.
\par 34 Péter pedig megnyitván száját, monda: Bizonynyal látom, hogy nem személyválogató az Isten;
\par 35 Hanem minden nemzetben kedves õ elõtte, a ki õt féli és igazságot cselekszik.
\par 36 Azt az ígét, melyet elkülde az Izráel fiainak, hirdetvén békességet a Jézus Krisztus által (õ mindeneknek Ura).
\par 37 Ti ismeritek azt a dolgot, mely lõn az egész Júdeában, Galileától kezdve, az után a keresztség után, melyet János prédikált,
\par 38 A názáreti Jézust, mint kené fel õt az Isten Szent Lélekkel és hatalommal, ki széjjeljárt jót tévén és meggyógyítván mindeneket, kik az ördög hatalma alatt voltak; mert az Isten vala õ vele.
\par 39 És mi vagyunk bizonyságai mindazoknak, a miket mind a zsidóknak tartományában, mind Jeruzsálemben cselekedett; a kit megölének, fára feszítvén.
\par 40 Ezt az Isten feltámasztá harmadnapon, és megadá, hogy õ megjelenjék nyilván,
\par 41 Nem az egész népnek, hanem az Istentõl eleve választott bizonyságoknak, nékünk, kik együtt ettünk és együtt ittunk õ vele, minekutána feltámadott halottaiból.
\par 42 És megparancsolta nékünk, hogy hirdessük a népnek, és tegyünk bizonyságot, hogy õ az Istentõl rendelt bírája  élõknek és holtaknak.
\par 43 Errõl a próféták mind bizonyságot tesznek, hogy bûneinek bocsánatját veszi az õ neve által mindenki, a ki hiszen õ benne.
\par 44 Mikor még szólá Péter ez ígéket, leszálla a Szent Lélek mindazokra, a kik hallgatják vala e beszédet.
\par 45 És elálmélkodának a zsidóságból való hívek, mindazok, a kik Péterrel együtt mentek, hogy a pogányokra is kitöltetett a Szent Lélek ajándéka.
\par 46 Mert hallják vala, hogy õk nyelveken szólnak és magasztalják az Istent. Akkor felele Péter:
\par 47 Vajjon eltilthatja-é valaki a vizet, hogy ezek meg ne keresztelkedjenek, kik vették a Szent Lelket  miképen mi is?
\par 48 És parancsolá, hogy keresztelkedjenek meg az Úrnak nevében. Akkor kérék õt, hogy maradjon náluk néhány napig.

\chapter{11}

\par 1 Meghallák azonban az apostolok és a Júdeában levõ atyafiak, hogy a pogányok is bevették az Istennek beszédét.
\par 2 Mikor azért felment Péter Jeruzsálembe, vetekedének õ vele a zsidóságból valók,
\par 3 Mondván: Körülmetéletlen emberekhez mentél be, és együtt ettél velük.
\par 4 Elkezdvén pedig Péter, megmagyarázta nékik rendre, mondván:
\par 5 Én Joppé városában imádkozám; és láték elragadtatásban egy látást, valami alászálló edényt, mint egy nagy lepedõt, négy sarkánál fogva leeresztve az égbõl; és egészen hozzám szálla:
\par 6 Melyre szememet rávetve megnézém, és látám a földi négylábú állatokat, a vadakat és a csúszómászókat és az égi madarakat.
\par 7 Hallék pedig szót is, mely ezt mondja vala nékem: Kelj fel Péter, öljed és egyél!
\par 8 Mondék azonban: Semmiképen sem, Uram; mert soha semmi közönséges vagy tisztátalan nem ment be az én számba.
\par 9 Felele pedig nékem a szózat másodszor az égbõl: A miket az Isten megtisztított, te ne mondd tisztátalanoknak.
\par 10 Ez pedig három ízben történt; és ismét felvonaték az egész az égbe.
\par 11 És ímé, azonnal három férfiú érkezék a házhoz, melyben valék, kik Czézáreából küldettek én hozzám.
\par 12 Mondá pedig nékem a Lélek, hogy menjek el velök minden kételkedés nélkül. Eljöve pedig velem ez a hat atyafi is; és bemenénk annak az embernek a házába:
\par 13 És elbeszélé nékünk, mimódon látta, a mint az angyal megálla az õ házában és ezt mondá néki: Küldj embereket Joppéba, és hívasd magadhoz Simont, ki Péternek neveztetik;
\par 14 Õ szólni fog hozzád olyan ígéket, melyek által megtartatol te és a te egész házadnépe.
\par 15 Mikor pedig én elkezdtem szólni, leszálla a Szent Lélek õ reájok, miképen mi reánk is kezdetben.
\par 16 Megemlékezém pedig az Úrnak ama mondásáról, a mint mondá: János ugyan vízzel keresztelt, ti azonban Szent Lélekkel fogtok megkereszteltetni.
\par 17 Ha tehát az Isten hasonló ajándékát adta nékik, mint nékünk is, kik hittünk az Úr Jézus Krisztusban, kicsoda voltam én, hogy az Istent eltilthattam volna?
\par 18 Ezeknek hallatára aztán megnyugovának, és dicsõíték az Istent, mondván: Eszerint hát a pogányoknak is adott az Isten megtérést az életre!
\par 19 Azok tehát, a kik eloszlottak az üldözés miatt, mely Istvánért támadott, eljutának Fenicziáig, Cziprusig és Antiókhiáig, senkinek nem prédikálván az ígét, hanem csak a zsidóknak.
\par 20 Voltak azonban közöttük némely cziprusi és czirénei férfiak, kik mikor Antiókhiába bementek, szólának a görögöknek, hirdetve az Úr Jézust.
\par 21 És az Úrnak keze vala velök; és nagy sokaság tére meg az Úrhoz, hívõvé lévén.
\par 22 Elhatott pedig a hír õ felõlük a jeruzsálemi gyülekezet fülébe; és kiküldék Barnabást, hogy menjen el egész Antiókhiáig.
\par 23 Ki mikor oda jutott és látta az Isten kegyelmét, örvendeze; és inté mindnyájukat, hogy állhatatos szívvel maradjanak meg az Úrban.
\par 24 Mert jámbor és Szent Lélekkel és hittel teljes férfiú vala õ. És nagy sokaság csatlakozék az Úrhoz.
\par 25 Elméne pedig Barnabás Tárzusba, hogy felkeresse Saulust, és rátalálván, elvivé õt Antiókhiába.
\par 26 És lõn, hogy õk egy egész esztendeig forgolódtak a gyülekezetben, és tanítottak nagy sokaságot; és a tanítványokat elõször Antiókhiában nevezték keresztyéneknek.
\par 27 Ez idõtájban pedig menének Jeruzsálembõl Antiókhiába próféták.
\par 28 Felkelvén pedig egy azok közül, névszerint Agabus, megjelenté a Lélek által, hogy az egész föld kerekségén nagy éhség lesz; a mely meg is lõn Klaudius császár idejében.
\par 29 A tanítványok pedig elhatározták, hogy a szerint, a mint kinek-kinek közöttük módjában áll, küldenek valamit segítségül a Júdeában lakozó atyafiaknak:
\par 30 A mit meg is cselekedének, elküldvén a vénekhez Barnabás és Saulus keze  által.

\chapter{12}

\par 1 Abban az idõben pedig Heródes király elkezde kegyetlenkedni némelyekkel, a gyülekezetbõl valók közül.
\par 2 Megöleté pedig Jakabot, Jánosnak testvérét, fegyverrel.
\par 3 És látván, hogy ez tetszik a zsidóknak, föltette magában, hogy elfogatja Pétert is. (Valának pedig a kovásztalan kenyerek napjai.)
\par 4 Kit el is fogatván, tömlöczbe veté, átadván négy négyes katonai szakasznak, hogy õrizzék õt; husvét után akarván õt a nép elé vezettetni.
\par 5 Péter azért õrizteték a fogságban; a gyülekezet pedig szüntelen könyörög vala az Istennek õ érette.
\par 6 Mikor pedig Heródes õt elõ akará vezettetni, azon az éjszakán aluszik vala Péter két vitéz között, megkötözve két lánczczal; és õrök õrizék az ajtó elõtt a tömlöczöt.
\par 7 És ímé az Úrnak angyala eljöve, és világosság fénylék a tömlöczben: és meglökvén Péter oldalát, felkölté õt, mondván: Kelj föl hamar! És leesének a lánczok kezeirõl.
\par 8 És monda néki az angyal: Övezd fel magadat, és kösd fel saruidat. És úgy cselekedék. És monda néki: Vedd rád felsõruhádat és kövess engem!
\par 9 És kimenvén, követé õt; és nem tudta, hogy valóság az, a mi történik az angyal által, hanem azt hitte, hogy látást lát.
\par 10 Mikor pedig általmentek az elsõ õrsön és a másodikon, jutának a vaskapuhoz, mely a városba visz; mely magától megnyílék elõttük: és kimenvén, egy utczán elõremenének; és azonnal eltávozék az angyal õ tõle.
\par 11 És Péter magához térve monda: Most tudom igazán, hogy az Úr elbocsátotta az õ angyalát, és megszabadított engem Heródes kezébõl és a zsidók népének egész várakozásától.
\par 12 És miután ezt megértette, elméne Máriának, a János anyjának házához, ki Márknak neveztetik; hol sokan valának egybegyûlve és könyörögnek vala.
\par 13 És mikor Péter zörgetett a tornácz ajtaján, egy Rhodé nevû szolgálóleány méne oda, hogy hallgatózzék:
\par 14 És megismervén a Péter szavát, örömében nem nyitá meg a kaput, hanem befutván, hírül adá, hogy Péter áll a kapu elõtt.
\par 15 Azok pedig mondának néki: Elment az eszed. Õ azonban erõsíté, hogy úgy van. Azok pedig mondának: Az õ angyala az.
\par 16 Péter pedig szüntelen zörget vala: mikor azért felnyitották, megláták õt és elálmélkodának.
\par 17 Miután pedig kezével hallgatást intett nékik, elbeszélé nékik, mimódon hozta ki õt az Úr a tömlöczbõl. És monda: Adjátok tudtára ezeket Jakabnak és az atyafiaknak. És kimenvén elméne más helyre.
\par 18 Mikor pedig megvirradt, nem csekély háborúság támada a vitézek között, mi történt hát Péterrel.
\par 19 Heródes pedig mikor elõkérte õt és nem találta, kivallatván az õröket, parancsolá, hogy kivégeztessenek. És lemenvén Júdeából Czézáreába, ott idõzött.
\par 20 Heródes pedig ellenséges indulattal vala a tirusiak és sidoniak iránt; de azok egyakarattal eljövének õ hozzá, és Blástust, a király kamarását megnyervén, békességet kérének, mivelhogy az õ tartományuk a királyéból élelmeztetik vala.
\par 21 Egy kitûzött napon pedig Heródes királyi ruhájába felöltözve és székibe ülve nyilvánosan szóla hozzájuk.
\par 22 A nép pedig felkiálta: Isten szava ez és nem emberé.
\par 23 És azonnal megveré õt az Úrnak angyala, azért, hogy nem az Istennek adá a dicsõséget;  és a férgektõl megemésztetvén, meghala.
\par 24 Az Istennek ígéje pedig növekedik és terjed vala,
\par 25 Barnabás és Saulus pedig visszatérének Jeruzsálembõl, betöltvén szolgálatukat, maguk mellé véve Jánost is, kinek mellékneve Márk vala.

\chapter{13}

\par 1 Valának pedig Antiókhiában az ott levõ gyülekezetben némely próféták és tanítók: Barnabás és Simeon, ki hivattatik vala Nigernek, és a Czirénei Luczius és Manaen, ki Heródessel, a negyedes fejedelemmel együtt neveltetett vala, és Saulus.
\par 2 Mikor azért azok szolgálának az Úrnak és bõjtölének, monda a Szent Lélek: Válaszszátok el nékem Barnabást és Saulust a munkára, a melyre én õket elhívtam.
\par 3 Akkor, miután bõjtöltek és imádkoztak, és kezeiket reájok vetették, elbocsáták õket.
\par 4 Õk annakokáért, miután kibocsáttattak a Szent Lélektõl, lemenének Szeleucziába; és onnét elevezének Cziprusba.
\par 5 És mikor Salamisba jutottak, hirdeték az Isten beszédét a zsidóknak zsinagógáiban: és János is velük vala, mint segítõtárs.
\par 6 És eljárván a szigetet mind Páfusig, találkozának egy ördöngõs hamispróféta zsidóra, kinek neve vala Barjézus;
\par 7 Ki Sergius Paulus tiszttartóval, ez okos emberrel vala. Ez magához hivatván Barnabást és Saulust, kíváná hallani az Isten beszédét.
\par 8 Elimás, az ördöngõs azonban (mert így magyaráztatik az õ neve) ellenkezik vala velök, igyekezvén a tiszttartót elfordítani a hittõl.
\par 9 De Saulus, ki Pál is, megtelvén Szent Lélekkel, szemeit reá vetve,
\par 10 Monda: Ó minden álnoksággal és minden gonoszsággal teljes ördögfi, minden igazságnak ellensége, nem szûnöl-é meg az Úrnak igaz útait elfordítani?
\par 11 Most azért ímé az Úrnak keze van ellened, és vak leszel és nem látod a napot egy ideig. És azonnal homály és sötétség szálla reá; és kerengve keres vala vezetõket.
\par 12 Akkor a tiszttartó, mikor látta a történt dolgot, hûn, elálmélkodván az Úrnak tudományán.
\par 13 Elhajózván pedig Páfusból Pál és kisérõi, Pergába, Pámfiliának városába menének. János azonban elválván tõlük, megtére Jeruzsálembe.
\par 14 Õk pedig Pergából tovább menve, eljutának Antiókhiába, Pisidiának városába, és bemenvén szombatnapon a zsinagógába; leülének.
\par 15 És a törvénynek és a prófétáknak felolvasása után küldének a zsinagógának elõljárói õ hozzájok, mondván: Atyánkfiai, férfiak, ha van valami intõbeszédetek a néphez, szóljatok.
\par 16 Pál azért felkelvén és kezével intvén, monda: Izráelnek férfiai, és ti, kik félitek az Istent, halljátok meg.
\par 17 Ennek a népnek, Izráelnek Istene kiválasztotta a mi atyáinkat, és e népet fölemelte, mikor Égyiptomnak földében jövevények valának, és onnét kihozá õket  hatalmas karja által.
\par 18 És közel negyven esztendõnek idejéig tûrte az õ erkölcsöket a pusztában.
\par 19 És minekutána eltörölt hét népet a Kanaán földén, azoknak földöket sorsvetés által elosztá  nékik.
\par 20 És azután mintegy négyszázötven esztendeig adott birákat mind Sámuel prófétáig;
\par 21 Annakutána pedig királyt kérének maguknak, és adá nékik az Isten  Sault, a Kis fiát, a Benjamin nemzetségébõl való férfiút negyven esztendeig.
\par 22 És mikor õt elveté, támasztá nékik Dávidot királyul; kirõl bizonyságot is tõn és monda: Találtam szívem szerint való férfiút, Dávidot, a Jesse fiát, ki minden akaratomat véghez viszi.
\par 23 Ennek magvából támasztott Isten, ígérete szerint, Izráelnek  szabadítót, Jézust;
\par 24 Minekutána elõbb János az õ eljövetele elõtt a megtérésnek keresztségét prédikálta Izráel egész népének.
\par 25 És mikor be akará végezni János az õ tisztét, monda: Kinek gondoltok engem? Nem én vagyok az, hanem ímé én utánam jõ, kinek nem vagyok méltó megoldani lábainak saruját.
\par 26 Atyámfiai, férfiak, Ábrahám nemzetének fiai, és kik ti köztetek félik az Istent, ez idvességnek beszéde néktek küldetett.
\par 27 Mert a kik lakoznak Jeruzsálemben és azoknak fejei, mivelhogy õt fel nem ismerék, a prófétáknak szavait is, melyeket minden szombaton felolvasnak, ítéletükkel betöltötték.
\par 28 És bár semmi halálra való okot nem találtak, kérék Pilátustól, hogy ölettessék meg.
\par 29 És mikor mindazokat elvégezték, a mik õ felõle megirattak, a fáról levéve sírba helyhezteték.
\par 30 De az Isten feltámasztá õt halottaiból:
\par 31 És õ megjelent több napon át azoknak, kik együtt jöttek fel õ vele Galileából Jeruzsálembe, kik néki bizonyságai  a nép elõtt.
\par 32 És mi hirdetjük néktek az atyáknak tett ígéretet, hogy azt az Isten betöltötte nékünk, az õ fiaiknak feltámasztván Jézust:
\par 33 Mint a második zsoltárban is meg van írva: Én Fiam vagy te; ma nemzettelek én téged.
\par 34 Hogy pedig feltámasztotta õt halottaiból, úgy hogy nem is fog többé az enyészetbe visszatérni, azt így mondotta: Néktek adom a Dávid biztos  szent javait.
\par 35 Azért mondja másutt is: Nem engeded, hogy a te Szented rothadást lásson.
\par 36 Mert Dávid, minekutána a saját idejében szolgált az Isten akaratának, elaludt, és helyhezteték az õ atyáihoz, és rothadást látott.
\par 37 De a kit Isten feltámasztott, az nem látott rothadást.
\par 38 Azért legyen néktek tudtotokra, atyámfiai, férfiak, hogy ez által hirdettetik néktek a bûnöknek bocsánata:
\par 39 És mindenekbõl, a mikbõl a Mózes törvénye által meg nem igazíttathattatok, ez által mindenki, a ki hisz, megigazul.
\par 40 Meglássátok azért, hogy rajtatok ne essék, a mit a próféták megmondottak:
\par 41 Lássátok meg, ti megvetõk, és csodálkozzatok és semmisüljetek meg; mert én oly dolgot cselekszem a ti idõtökben, oly dolgot, melyet nem hinnétek, ha valaki elmondaná néktek.
\par 42 Mikor pedig kimentek a zsidók zsinagógájából, kérék a pogányok, hogy a következõ szombaton prédikálják nékik ezen beszédeket.
\par 43 Mikor pedig eloszlott a gyülekezet, sokan a zsidók közül és az istenfélõ prozelitusok közül követék Pált és Barnabást; a kik szólván hozzájuk, biztaták õket, hogy maradjanak meg az Isten  kegyelmében.
\par 44 A következõ szombaton aztán majdnem az egész város egybegyûle az Isten ígéjének hallgatására,
\par 45 Mikor pedig látták a zsidók a sokaságot, betelének irigységgel, és ellene mondának azoknak, miket Pál mond vala, ellenkezve és káromlást szólva.
\par 46 Akkor Pál és Barnabás nagy bátorsággal szólva mondának: Szükséges volt, hogy elõször néktek hirdettessék az Isten ígéje; de mivelhogy ti megvetitek azt, és nem tartjátok méltóknak magatokat az örök életre, ímé a pogányokhoz  fordulunk.
\par 47 Mert így parancsolta nékünk az Úr: Rendeltelek téged világosságul a pogányoknak, hogy légy üdvösségükre a földnek széléig.
\par 48 A pogányok pedig ezeket hallván, örvendezének, és magasztalják vala az Úrnak ígéjét; és a kik csak örök életre választattak vala, hivének.
\par 49 Terjede pedig az Úrnak ígéje az egész tartományban.
\par 50 A zsidók azonban felindíták az istenfélõ és tisztességbeli asszonyokat és a városnak eleit, és üldözést támasztának Pál és Barnabás ellen, és kiûzék õket határukból.
\par 51 Azok pedig lábuknak porát lerázván ellenük, elmenének Ikóniumba.
\par 52 A tanítványok pedig betelnek vala örömmel és Szent Lélekkel.

\chapter{14}

\par 1 Lõn pedig Ikóniumban, hogy õk együtt menének be a zsidók zsinagógájába, és prédikálának, úgyannyira, hogy mind zsidóknak, mind görögöknek nagy sokasága lõn hívõvé.
\par 2 A kik azonban a zsidók közül nem hivének, felindíták és megharagíták a pogányoknak lelkét az atyafiak ellen.
\par 3 Azért sok idõt töltöttek ott, bátran prédikálva az Úrban, ki bizonyságot tesz vala az õ kegyelmének beszéde mellett, és adja vala, hogy jelek és csodák történjenek az õ kezeik által.
\par 4 De a városnak sokasága meghasonlék; és némelyek a zsidók mellett, mások pedig az apostolok mellett valának.
\par 5 És mikor a pogányok és zsidók az õ fõembereikkel egybe támadást indítának, hogy bosszúsággal illessék és megkövezzék õket,
\par 6 Õk megtudták, és elfutának Likaóniának városaiba, Listrába és Derbébe, és a körülvaló tartományba,
\par 7 És ott prédikálják vala az evangyéliomot.
\par 8 És Listrában ül vala egy lábaival tehetetlen ember, ki az õ anyjának méhétõl fogva sánta volt, és soha nem járt.
\par 9 Ez hallá Pált beszélni: a ki szemeit reá függesztvén, és látván, hogy van hite, hogy meggyógyul,
\par 10 Monda nagy fenszóval: Állj fel lábaidra egyenesen! És felszökött és járt.
\par 11 A sokaság pedig mikor látta, a mit Pál cselekedett, felkiálta, likaóniai nyelven mondván: Az istenek jöttek le mihozzánk emberi ábrázatban!
\par 12 És hívják vala Barnabást Jupiternek, Pált pedig Merkúriusnak, minthogy õ volt a szóvivõ.
\par 13 Jupiter papja pedig, a kinek temploma az õ városuk elõtt vala, felkoszorúzott bikákat hajtva a kapukhoz, a sokasággal együtt áldozni akar vala.
\par 14 Mikor azonban ezt meghallották az apostolok, Barnabás és Pál, köntösüket megszaggatván, a sokaság közé futamodának, kiáltván
\par 15 És ezt mondván: Férfiak, miért mívelitek ezeket? Mi is hozzátok hasonló természetû emberek vagyunk, és azt az örvendetes izenetet hirdetjük néktek, hogy e hiábavalóktól az élõ Istenhez  térjetek, ki teremtette a mennyet, a földet, a tengert és minden azokban valókat:
\par 16 Ki az elmúlt idõkben hagyta a pogányokat mind a maguk útján haladni:
\par 17 Jóllehet nem hagyta magát tanúbizonyság nélkül, mert jóltevõnk volt, adván mennybõl esõket és termõ idõket nékünk, és betöltvén eledellel és örömmel a mi szívünket.
\par 18 És ezeket mondván, nagynehezen lecsendesíték a sokaságot, hogy nékik ne áldozzék.
\par 19 Jövének azonban Antiókhiából és Ikóniumból zsidók, és a sokaságot eláltatván, megkövezék Pált, és kivonszolák a városból, azt gondolván, hogy meghalt.
\par 20 De mikor körülvették õt a tanítványok, felkelvén, beméne a városba; és másnap Barnabással elméne Derbébe.
\par 21 És miután hirdették az evangyéliomot annak a városnak, és sokakat tanítványokká tettek, megtérének Listrába, Ikoniumba és Antiókhiába.
\par 22 Erõsítve a tanítványok lelkét, intvén, hogy maradjanak meg a hitben, és hogy sok háborúságon által kell  nékünk az Isten országába bemennünk.
\par 23 Miután pedig választottak nékik gyülekezetenként véneket, imádkozván bõjtölésekkel  egybe, ajánlák õket az Úrnak, kiben hittek vala.
\par 24 És Pisidián általmenvén, menének Pamfiliába.
\par 25 És miután Pergában hirdették az ígét, lemenének Attáliába;
\par 26 És onnét elhajózának Antiókhiába, a honnét az Isten kegyelmére bízták volt õket arra a munkára, melyet elvégeztek.
\par 27 Mikor pedig megérkeztek és a gyülekezetet egybehívták, elbeszélék, mily nagy dolgokat cselekedett az Isten õ velök, és hogy a pogányoknak kaput nyitott a hitre.
\par 28 Ott aztán nem kevés idõt töltöttek a tanítványokkal.

\chapter{15}

\par 1 Némelyek pedig, kik Júdeából jöttek alá, így tanítják vala az atyafiakat: Ha körül nem metélkedtek Mózes rendtartása  szerint, nem idvezülhettek.
\par 2 Mikor azért Pálnak és Barnabásnak nagy háborúsága és vetekedése lõn azok ellen, azt végezék, hogy Pál és Barnabás és némely mások õ közülök menjenek fel az apostolokhoz és a vénekhez Jeruzsálembe e kérdés ügyében.
\par 3 Õk tehát kikísértetvén a gyülekezettõl, általmentek Fenicián és Samárián, elbeszélve a pogányok megtérését; és nagy örömet szerzének az összes atyafiaknak.
\par 4 Mikor pedig megérkeztek Jeruzsálembe, a gyülekezet és az apostolok és a vének fogadák õket, és õk elbeszélék, mily nagy dolgokat cselekedék az Isten õ velök.
\par 5 Elõállának azonban némely hivõk a farizeusok szerzetébõl valók közül, mondván, hogy körül kell metélni õket, és megparancsolni, hogy a Mózes törvényét megtartsák.
\par 6 Egybegyülének azért az apostolok és a vének, hogy e dolog felõl végezzenek.
\par 7 És mikor nagy vetekedés támadt, felkelvén Péter, monda nékik: Atyámfiai, férfiak, ti tudjátok, hogy az Isten régebbi idõ óta kiválasztott engem mi közülünk, hogy a pogányok az én számból hallják az evangyéliomnak beszédét, és higyjenek.
\par 8 És a szíveket ismerõ Isten bizonyságot tett mellettük, mert adta nékik a Szent Lelket, miként nékünk is;
\par 9 És semmi különbséget sem tett mi köztünk és azok között, a hit által tisztítván meg azoknak szívét.
\par 10 Most azért mit kísértitek az Istent, hogy a tanítványok nyakába oly igát tegyetek, melyet sem a mi atyáink, sem mi el nem hordozhattunk?
\par 11 Sõt inkább az Úr Jézus Krisztus kegyelme által hiszszük, hogy megtartatunk, miképen azok is.
\par 12 Elhallgatott azért az egész sokaság; és hallgatják vala Barnabást és Pált, a mint elbeszélék, mennyi jelt és csudát tett az Isten õ általok a pogányok között.
\par 13 Miután pedig õk elhallgattak, felele Jakab, mondván: Atyámfiai, férfiak, hallgassatok meg engem!
\par 14 Simeon elbeszélé, mimódon gondoskodott elõször az Isten, hogy a pogányok közül vegyen népet az õ nevének,
\par 15 És ezzel egyeznek a próféták mondásai, mint meg van írva:
\par 16 Ezek után megtérek és felépítem a Dávidnak leomlott sátorát; és annak omladékait helyreállítom, és ismét felállatom azt:
\par 17 Hogy megkeresse az embereknek többi része az Urat, és a pogányok mindnyájan, a kik az én nevemrõl neveztetnek. Ezt mondja az Úr, ki mindezeket megcselekszi.
\par 18 Tudja az Isten öröktõl fogva minden õ cselekedeteit.
\par 19 Azokáért én azt mondom, hogy nem kell háborgatni azokat, kik a pogányok közül térnek meg az Istenhez;
\par 20 Hanem írjuk meg nékik, hogy tartózkodjanak a a bálványok fertelmességeitõl, a paráznaságtól, a  fúlvaholt állattól és a  vértõl.
\par 21 Mert Mózesnek régi nemzedékek óta városonként megvannak a hirdetõi, mivelhogy a zsinagógákban minden szombaton olvassák.
\par 22 Akkor tetszék az apostoloknak és a véneknek az egész gyülekezettel egybe, hogy férfiakat válaszszanak ki magok közül és elküldjék Antiókhiába Pállal és Barnabással, Júdást, kinek mellékneve Barsabás, és Silást, kik az atyafiak között fõemberek valának.
\par 23 Megírván azok keze által ezeket: Az apostolok, a vének, és az atyafiak az Antiókhiában, Siriában és Czilicziában levõ, a pogányok közül való atyafiaknak üdvözletüket!
\par 24 Mivelhogy meghallottuk, hogy némelyek mi közülünk kimenvén, megháborítottak titeket beszédeikkel, feldúlva a ti lelketeket, azt mondván, hogy körülmetélkedjetek és a törvényt megtartsátok; kiknek mi parancsot nem adtunk:
\par 25 Tetszék nékünk, miután egyértelemre jutottunk, hogy férfiakat válaszszunk ki és elküldjük ti hozzátok a mi szeretteinkkel, Barnabással és Pállal,
\par 26 Oly emberekkel, kik életüket tették koczkára a mi Urunk Jézus Krisztus nevéért.
\par 27 Küldöttük azért Júdást és Silást, kik élõszóval szintén tudtotokra adják ugyanezeket.
\par 28 Mert tetszék a Szent Léleknek és nékünk, hogy semmi több teher ne vettessék ti reátok ezeken a szükséges dolgokon kívül,
\par 29 Hogy tartózkodjatok a bálványoknak áldozott dolgoktól, a vértõl, a fúlvaholt állattól, és a paráznaságtól; melyektõl ha megóvjátok magatokat, jól lesz dolgotok. Legyetek egészségben!
\par 30 Azok annakokáért elbocsáttatván, elmenének Antiókhiába; és egybegyûjtvén a sokaságot, átadák a levelet.
\par 31 És mikor elolvasták, örvendezének az intésen.
\par 32 Júdás és Silás pedig maguk is próféták lévén, sok beszéddel inték az atyafiakat, és megerõsíték.
\par 33 Miután pedig bizonyos idõt eltöltöttek, elbocsáták õket az atyafiak békességgel az apostolokhoz.
\par 34 De Silásnak tetszék ott maradni.
\par 35 Pál és Barnabás is Antiókhiában idõzének, tanítva és prédikálva másokkal is többekkel az Úrnak ígéjét.
\par 36 Egynéhány nap mulva pedig monda Pál Barnabásnak: Visszatérve most, látogassuk meg a mi atyánkfiait minden városban, melyben hírdettük az Úrnak ígéjét, hogyan vannak.
\par 37 És Barnabás azt tanácsolá, hogy vegyék maguk mellé Jánost, ki Márknak hívatik.
\par 38 Pál azonban azt tartá méltónak, hogy a ki elszakadt tõlük Pamfiliától fogva, és nem ment velök a munkára, ne vegyék maguk mellé azt.
\par 39 Meghasonlás támada azért, úgyhogy elszakadának egymástól, és Barnabás maga mellé véve Márkot, elhajózék Cziprusba;
\par 40 Pál pedig Silást választván maga mellé, elméne, az Isten kegyelmére bízatván az atyafiaktól.
\par 41 És eljárá Siriát és Czilicziát, erõsítve a gyülekezeteket.

\chapter{16}

\par 1 Juta pedig Derbébe és Listrába: És ímé vala ott egy Timótheus nevû tanítvány, egy hívõ zsidó asszonynak, de görög atyának fia;
\par 2 Kirõl jó bizonyságot tesznek vala a Listrában és Ikóniumban levõ atyafiak.
\par 3 Ezt Pál magával akará vinni; és vévén, körülmetélé õt a zsidókért, kik azokon a helyeken valának: mert ismerték mindnyájan az õ atyját, hogy görög volt.
\par 4 És a mint általmentek a városokon, meghagyák nékik, hogy tartsák meg a rendeléseket, melyeket végeztek a Jeruzsálemben levõ apostolok és vének.
\par 5 A gyülekezetek azért erõsödének a hitben, és gyarapodának számban naponként.
\par 6 Eljárván pedig Frigiát és Galácia tartományát, mivelhogy eltiltatának a Szent Lélektõl, hogy az ígét Ázsiában hirdessék,
\par 7 Misia felé menvén, igyekeznek vala Bithiniába jutni; de nem ereszté õket a Lélek.
\par 8 Áthaladván azért Misián, lemenének Tróásba.
\par 9 És azon az éjszakán látás jelenék meg Pálnak: egy macedón férfiú állt elõtte, kérve õt és ezt mondva: Jer által Macedóniába, és légy segítségül nékünk!
\par 10 Mihelyt pedig a látást látta, azonnal igyekezénk elmenni Macedóniába, megértvén, hogy oda hívott minket az Úr, hogy azoknak prédikáljuk az evangyéliomot.
\par 11 Elhajózván azért Tróásból, egyenesen Sámothrákéba mentünk, és másnap Neápolisba;
\par 12 Onnét pedig Filippibe, mely Macedónia azon részének elsõ gyarmatvárosa. És ebben a városban töltöttünk néhány napot.
\par 13 És szombatnapon kimenénk a városon kívül egy folyóvíz mellé, hol az imádkozás szokott lenni; és leülvén, beszélgeténk az egybegyûlt asszonyokkal.
\par 14 És egy Lidia nevû, Thiatira városbeli bíborárús asszony, ki féli vala az Istent, hallgata reánk. Ennek az Úr megnyitá szívét, hogy figyelmezzen azokra, a miket Pál mond vala.
\par 15 Mikor pedig megkeresztelkedék mind házanépével egybe, kére minket, mondván: Ha az Úr hívének ítéltetek engem, jertek az én házamhoz, és maradjatok ott. És unszola minket.
\par 16 Lõn pedig, hogy mikor mentünk a könyörgésre, egy szolgálóleányka jöve elõnkbe, kiben jövendõmondásnak lelke vala, ki az õ urainak nagy hasznot hajta jövendõmondásával.
\par 17 Ez követvén Pált és minket, kiált vala, mondván: Ezek az emberek a magasságos Istennek szolgái, kik néktek az idvességnek útját hirdetik.
\par 18 Ezt pedig több napon át mívelte. Pál azonban megbosszankodván, és hátrafordulván, mondá a léleknek: Parancsolom néked a Jézus Krisztus nevében, hogy menj ki belõle. És kiméne abban az órában.
\par 19 Látván pedig annak az urai, hogy keresetüknek a reménysége elveszett, megfogva Pált és Silást, vonák a piaczra a hatóságok elé.
\par 20 És odavezetvén õket a bírákhoz, mondának: Ezek az emberek zsidó létükre megháborítják a mi városunkat,
\par 21 És olyan szertartásokat hirdetnek, melyeket nem szabad nékünk bevennünk, sem cselekednünk, mivelhogy rómaiak vagyunk.
\par 22 És velök egyben feltámada a sokaság õ ellenök. A bírák pedig letépetvén ruháikat, megvesszõzteték õket.
\par 23 És miután sok ütést mértek rájok, tömlöczbe veték õket, megparancsolva a tömlöcztartónak, hogy gondosan õrizze õket.
\par 24 Ki ilyen parancsolatot vévén, veté õket a belsõ tömlöczbe, és lábaikat kalodába szorítá.
\par 25 Éjféltájban pedig Pál és Silás imádkozván, énekkel dicsõíték az Istent. A foglyok pedig hallgatják vala õket.
\par 26 És hirtelen nagy földindulás lõn, úgyannyira, hogy megrendülének a tömlöcz fundamentomai; és azonnal megnyílának az ajtók mind, és mindnyájoknak a bilincsei feloldódnak.
\par 27 Fölserkenvén pedig a tömlöcztartó, és látván, hogy nyitva vannak a tömlöcznek ajtai, kivonva fegyverét, meg akará magát ölni, azt gondolván, hogy elszöktek a foglyok.
\par 28 Pál azonban nagy fenszóval kiáltá, mondván: Semmi kárt ne tégy magadban; mert mindnyájan itt vagyunk!
\par 29 Az pedig világot kérve beugrott, és remegve borult Pál és Silás elé,
\par 30 És kihozván õket, monda: Uraim, mit kell nékem cselekednem, hogy idvezüljek?
\par 31 Azok pedig mondának: Higyj az Úr Jézus Krisztusban, és idvezülsz mind te, mind a te házadnépe!
\par 32 És hirdeték néki az Úrnak ígéjét, és mindazoknak, kik az õ házánál valának.
\par 33 És az magához vévén õket az éjszakának azon órájában, megmosá az ütésektõl: és megkeresztelkedék azonnal õ és az övéimindnyájan.
\par 34 És bevivén õket házába, asztalt teríte nékik, és egész háznépével egyben örvendeze, hogy hitt az Istennek.
\par 35 Mikor pedig megvirradt, a bírák elküldék a poroszlókat, mondván: Bocsásd el azokat az embereket.
\par 36 A tömlöcztartó pedig tudtára adá e szavakat Pálnak: A bírák ide küldöttek, hogy bocsássalak el titeket: most azért kimenvén, menjetek el békességgel!
\par 37 Pál pedig monda nékik: Megvesszõztek minket nyilvánosan, ítélet nélkül, holott római emberek vagyunk, és tömlöczbe vetettek: és most alattomban akarnak bennünket kiküldeni? Nem úgy; hanem jõjjenek õk maguk és vezessenek ki minket.
\par 38 A poroszlók pedig megmondák a bíráknak e beszédeket: és azok megfélemlének, mikor meghallották, hogy rómaiak,
\par 39 És odamenvén, megkérlelék õket: és kivezetvén, kérék, hogy menjenek ki a városból.
\par 40 Kijövén pedig a tömlöczbõl, bemenének Lidiához; és mikor látták az atyafiakat, vígasztalák õket, és eltávozának.

\chapter{17}

\par 1 Miután pedig általmentek Ámfipolison és Apollónián, Thessalonikába érkeztek, a hol volt a zsidóknak zsinagógájok.
\par 2 Pál pedig, a mint szokása vala, beméne hozzájok, és három szombaton át vetekedék velök az írásokból,
\par 3 Megmagyarázva és kimutatva, hogy a Krisztusnak szükség volt szenvedni és feltámadni a halálból; és hogy ez a Jézus a Krisztus,  a kit én hirdetek néktek.
\par 4 És némelyek azok közül hivének és csatlakozának Pálhoz és Siláshoz; úgyszintén az istenfélõ görögök közül nagy sokaság, és az elõkelõ asszonyok közül nem kevesen.
\par 5 De a zsidók, kik nem hisznek vala, írigységtõl felindíttatván, és magok mellé vévén a piaczi népségbõl némely gonosz férfiakat, és csõdületet támasztván, felháboríták a várost; és a Jáson házát megostromolván, igyekeztek õket kihozni a nép közé.
\par 6 Mikor pedig õket nem találák, Jásont és némely atyafiakat vonszolák a város elõljárói elé, kiáltozva, hogy ezek az országháborítók itt is megjelentek;
\par 7 Kiket Jáson házába fogadott: pedig ezek mindnyájan a császár parancsolatai ellen cselekesznek, mivelhogy mást tartanak királynak, Jézust.
\par 8 Fel is indíták a sokaságot és a város elõljáróit, kik hallják vala ezeket.
\par 9 De mikor kezességet nyertek Jáson és a többiek részérõl, elbocsáták õket.
\par 10 Az atyafiak pedig azonnal, azon éjszakán elküldék Pált Silással egyetemben Béreába; kik mikor odamentek, elmenének a zsidóknak zsinagógájába.
\par 11 Ezek pedig nemesb lelkûek valának a Thessalonikabelieknél, úgymint kik bevevék az ígét teljes készséggel, naponként tudakozva az írásokat, ha úgy vannak-é ezek.
\par 12 Sokan hivének azért õ közülök; sõt az elõkelõ görög asszonyok és férfiak közül is nem kevesen.
\par 13 Mikor azonban tudtokra esett a Thessalonikából való zsidóknak, hogy Béreában is prédikálta Pál az Istennek ígéjét, elmenének, és a sokaságot ott is felháboríták.
\par 14 De akkor mindjárt kibocsáták az atyafiak Pált, hogy utazzék a tenger felé; Silás és Timótheus azonban ott maradának.
\par 15 A kik pedig elkísérték Pált, elvivék õt egész Athénig; és parancsát vévén Siláshoz és Timótheushoz, hogy minél hamarább menjenek õ hozzá, elmenének.
\par 16 Athénben pedig, mikor azokat várá Pál, lelke háborog vala õ benne, látván, hogy a város bálványokkal van tele.
\par 17 Vetekedik vala azért a zsinagógában a zsidókkal és az istenfélõ emberekkel, és a piaczon mindennap azokkal, a kiket elõtalált.
\par 18 Némelyek pedig az epikureus és stoikus filozófusok közül összeakadtak õ vele. És némelyek mondának: Mit akarhat ez a csacsogó mondani? Mások meg: Idegen istenségek hirdetõjének látszik. Mivelhogy a Jézust és a feltámadást hirdeti vala nékik.
\par 19 És megragadván õt, az Areopágusra vivék, ezt mondván: Vajjon megérthetjük-é mi az az új tudomány, melyet te hirdetsz?
\par 20 Mert valami idegen dolgokat beszélsz a mi füleinknek: meg akarjuk azért érteni, mik lehetnek ezek.
\par 21 Az athéniek pedig mindnyájan és az ott lakó jövevények semmi másban nem valának foglalatosok, mint valami újságnak beszélésében és hallgatásában.
\par 22 Elõállván pedig Pál az Areopágusnak közepette, monda: Athéni férfiak, minden tekintetben nagyon istenfélõknek látlak titeket.
\par 23 Mert mikor bejárám és szemlélém a ti szentélyeiteket, találkozám egy oltárral is, melyre ez vala ráírva: Ismeretlen Istennek. A kit azért ti nem ismerve tiszteltek, azt hirdetem én néktek.
\par 24 Az Isten, a ki teremtette a világot és mindazt, a mi abban van, mivelhogy õ mennynek és földnek ura, kézzel csinált templomokban  nem lakik.
\par 25 Sem embereknek kezeitõl nem tiszteltetik, mintha valami nélkül szûkölködnék, holott õ ád mindeneknek életet leheletet és  mindent;
\par 26 És az egész emberi nemzetséget egy vérbõl teremtette, hogy lakozzanak a földnek egész színén, meghatározván eleve rendelt idejöket és  lakásuknak határait;
\par 27 Hogy keressék az Urat, ha talán kitapogathatnák õt és megtalálhatnák, jóllehet bizony nincs messze egyikõnktõl sem:
\par 28 Mert õ benne élünk, mozgunk és vagyunk; miképen a költõitek közül is mondották némelyek: Mert az õ nemzetsége is vagyunk.
\par 29 Mivelhogy azért az Istennek nemzetsége vagyunk, nem kell azt gondolnunk, hogy aranyhoz, vagy ezüsthöz, vagy kõhöz, emberi mesterség és kitalálás faragványához hasonlatos az istenség.
\par 30 E tudatlanságnak idejét azért elnézvén az Isten, mostan parancsolja az embereknek, mindenkinek mindenütt, hogy  megtérjenek:
\par 31 Mivelhogy rendelt egy napot, melyen megítéli majd a föld kerekségét igazságban egy férfiú által, kit arra rendelt; bizonyságot tévén mindenkinek, az által, hogy feltámasztá õt  halottaiból.
\par 32 Mikor pedig a halottak feltámadásáról hallottak, némelyek gúnyolódtak; mások pedig mondának: Majd még meghallgatunk téged e felõl.
\par 33 És ilyen módon Pál kiméne azok közül.
\par 34 Némely férfiak azonban csatlakozván õ hozzá, hivének; ezek között az areopágita Dienes is, és egy Damaris nevû asszony, és mások õ velök.

\chapter{18}

\par 1 Ezekután Pál Athénbõl eltávozván, méne Korinthusba.
\par 2 És mikor egy Akvila nevû, pontusi származású zsidóra talált, ki nem régen jött Itáliából, és feleségére Prisczillára (mivelhogy Klaudius megparancsolta vala, hogy a zsidók mind távozzanak Rómából): hozzájuk csatlakozék.
\par 3 És mivelhogy azonféle míves vala, náluk marada és dolgozik vala. Mesterségökre nézve ugyanis sátorcsinálók valának.
\par 4 Vetekedék pedig minden szombaton a zsinagógában, és igyekezék mind zsidókat, mind görögöket meggyõzni.
\par 5 Mikor pedig megérkeztek Maczedóniából Silás és Timótheus, szorongatá a lélek Pált, és bizonyságot tõn a zsidóknak, hogy Jézus a Krisztus.
\par 6 Mikor pedig azok ellenszegülének és káromlásokat szólának, ruháit megrázva monda nékik: Véretek a fejetekre;  én tiszta vagyok: mostantól fogva a pogányokhoz  megyek.
\par 7 És általmenvén onnét, méne egy Justus nevû, istenfélõ ember házához, kinek háza szomszédos vala a zsinagógával.
\par 8 Krispus pedig, a zsinagógának feje hûn az Úrban egész házanépével egybe; a Korinthusbeliek közül is sokan hallván, hisznek vala, és megkeresztelkednek vala.
\par 9 Monda pedig az Úr látás által éjszaka Pálnak: Ne félj, hanem szólj és ne hallgass:
\par 10 Mert én veled vagyok és senki sem támad reád, hogy néked ártson; mert nékem sok népem van ebben a városban.
\par 11 És ott lakozék egy esztendeig és hat hónapig, tanítva köztük az Isten ígéjét.
\par 12 Mikor pedig Gallió volt Akhája tiszttartója, reátámadának a zsidók egyakarattal Pálra, és vivék õt a törvényszék elébe,
\par 13 Mondván: Ez a törvény ellen való istentiszteletre csábítja az embereket.
\par 14 Mikor pedig Pál meg akará nyitni száját, monda Gallió a zsidóknak: Ha valóban valami bosszútételrõl, vagy gonosz cselekedetrõl volna szó, zsidók, igazság szerint meghallgatnálak benneteket:
\par 15 De ha tanításról, nevekrõl és a ti törvénytekrõl van kérdés, ti magatok lássátok; mert én ezekben bíró nem akarok lenni.
\par 16 És elûzé õket a törvényszék elõl.
\par 17 A görögök pedig mindnyájan Sosthenest, a zsinagógának fejét megragadván, verik vala a törvényszék elõtt; de Gallió velök semmit sem gondola.
\par 18 Pál pedig, miután még több napig ott marada, az atyafiaktól elbúcsúzván, Siriába hajózék, és vele együtt Prisczilla és Akvila, minekutána fejét megnyírta Kenkreában; mert fogadása vala.
\par 19 Juta pedig Efézusba, és azokat ott hagyá; õ maga pedig bemenvén a zsinagógába, vetekedék a zsidókkal.
\par 20 Mikor pedig azok kérék, hogy több ideig maradjon nálok, nem álla reá;
\par 21 Hanem búcsút võn tõlük, mondván: Mindenesetre Jeruzsálemben kell nékem a következõ ünnepet töltenem; de ismét megjövök hozzátok, ha Isten akarja. És elhajózék Efézusból.
\par 22 És miután Czézáreába érkezék, felmenvén Jeruzsálembe és köszöntvén a gyülekezetet, leméne Antiókhiába.
\par 23 És miután ott bizonyos idõt eltöltött, elméne, eljárván renddel Galácia tartományát és Frigiát, erõsítve a tanítványokat mind.
\par 24 Érkezék pedig Efézusba egy Apollós nevû zsidó, alekszandriai származású, ékesenszóló férfiú, ki az írásokban tudós vala.
\par 25 Ez meg volt tanítva az Úrnak útára; és lélekben buzgó lévén, szólja és tanítja vala nagy szorgalmatosan az Úrra tartozó dolgokat, jóllehet csak a János keresztségét tudja vala.
\par 26 És ez kezde nagy bátorsággal szólni a zsinagógában. Mikor pedig meghallgatta õt Akvila és Prisczilla, magok mellé vevék õt, és nyilvábban kifejtették elõtte az Istennek útát.
\par 27 Mikor pedig Akhájába akara átmenni, buzdítván õt az atyafiak, írának a tanítványoknak, hogy fogadják be õt. Ki mikor odajutott, sokat használa azoknak, kik hittek vala a kegyelem által:
\par 28 Mert hatalmasan meggyõzi vala a zsidókat nyilvánosan, bebizonyítva az írásokból, hogy Jézus a Krisztus.

\chapter{19}

\par 1 Lõn pedig azonközben, míg Apollós Korinthusban volt, hogy Pál, eljárván a felsõbb tartományokat, Efézusba érkezék: és mikor némely tanítványokra talált,
\par 2 Monda nékik: Vajjon vettetek-é Szent Lelket, minekutána hivõkké lettetek? Azok pedig mondának néki: Sõt inkább azt sem hallottuk, hogy ha vagyon-é Szent Lélek.
\par 3 És monda nékik: Mire keresztelkedtetek meg tehát? Azok pedig mondának: A János keresztségére.
\par 4 Monda pedig Pál: János megtérésnek keresztségével keresztelt, azt mondván a népnek, hogy a ki õ utána jövendõ, abban higyjenek, tudniillik a Krisztus Jézusban.
\par 5 Mikor pedig ezt hallák, megkeresztelkedének az Úr Jézusnak nevére.
\par 6 És mikor Pál reájok vetette kezét, szálla a Szent Lélek õ reájok; és szólnak vala nyelveken,  és prófétálnak vala.
\par 7 Valának pedig a férfiak összesen mintegy tizenketten.
\par 8 Bemenvén pedig a zsinagógába, bátorsággal szól vala, három hónapon át vetekedvén és igyekezvén meggyõzni az Isten országára tartozó dolgokról.
\par 9 Mikor pedig némelyek megkeményíték magokat és nem hivének, gonoszul szólván az Úrnak útáról a sokaság elõtt, azoktól eltávozván, elszakasztá a tanítványokat, mindennap egy bizonyos Tirannus oskolájában prédikálván.
\par 10 Ez pedig lõn két esztendeig; úgyannyira, hogy mindazok, kik lakoznak vala Ázsiában, mind zsidók, mind görögök, hallgaták az Úr Jézusnak ígéjét.
\par 11 És nem közönséges csodákat cselekszik vala az Isten Pál keze által:
\par 12 Annyira, hogy a betegekhez is elvivék az õ testérõl a keszkenõket, vagy kötényeket, és eltávozának azoktól a betegségek, és a gonosz lelkek kimenének belõlök.
\par 13 Elkezdték pedig némelyek a lézengõ zsidó ördögûzõk közül az Úr Jézus nevét hívni azokra, a kikben gonosz lelkek valának, mondván: Kényszerítünk titeket a Jézusra, kit Pál prédikál.
\par 14 Valának pedig némelyek Skévának, egy zsidó fõpapnak fiai heten, a kik ezt mívelik vala.
\par 15 Felelvén pedig a gonosz lélek, monda: A Jézust ismerem, Pálról is tudok;  de ti kicsodák vagytok?
\par 16 És reájok ugorván az az ember, a kiben a gonosz lélek vala, és legyõzvén õket, hatalmat võn rajtuk annyira, hogy mezítelenen és megsebesülve szaladának ki abból a házból.
\par 17 Ez pedig tudtokra lõn mindeneknek, mind zsidóknak, mind görögöknek, kik Efézusban laknak vala, és félelem szálla mindnyájokra, és magasztaltatik vala az Úr Jézusnak neve.
\par 18 És sokan a hívõk közül eljõnek vala, megvallván és megjelentvén cselekedeteiket.
\par 19 Sokan pedig azok közül, kik ördögi mesterségeket gyakoroltak, könyveiket összehordva, mindeneknek láttára megégetik vala. És összeszámlálák azoknak árát, és találák ötvenezer ezüstpénznek.
\par 20 Ekképen az Úrnak ígéje erõsen nevekedik és hatalmat vesz vala.
\par 21 Midõn pedig ezek elteltek, elvégezé Pál magában, hogy Maczedóniát és Akháját eljárván,  Jeruzsálembe megy, mondván: Ott létem után Rómát is meg kell nékem látnom.
\par 22 Elküldvén pedig Maczedóniába kettõt azok közül, kik néki szolgálnak vala, Timótheust és Erástust, õ maga egy idáig Ázsiában marada.
\par 23 Támada azonban azon idõtájban nem csekély háborúság az Úrnak útárt.
\par 24 Mert egy Demeter nevû ötvös, ezüstbõl Diána templomokat csinálván, a mesterembereknek nem csekély nyereséget ád vala;
\par 25 Kiket egybegyûjtvén az ilyenfélékkel foglalkozó mívesekkel egybe, monda: Férfiak, tudjátok, hogy ebbõl a mesterségbõl van a mi jóllétünk.
\par 26 Látjátok pedig és halljátok, hogy ez a Pál nemcsak Efézusnak, hanem közel az egész Ázsiának sok népét eláltatván, elfordította, mivelhogy azt mondja, hogy nem istenek azok, a melyek kézzel csináltatnak.
\par 27 Nemcsak az a veszély fenyeget pedig bennünket, hogy ez a mesterség tönkre jut, hanem hogy a nagy istenasszonynak, Diánának temploma is semmibe vétetik, és el is vész az õ nagysága, kit az egész Ázsia és a világ tisztel.
\par 28 Mikor pedig ezeket hallották és haraggal megtelének, kiáltnak vala, mondván: Nagy az efézusi Diána!
\par 29 És betelék az egész város háborúsággal; és egyakarattal a színházba rohanának, megfogván Gájust és Aristárkhust, kik Maczedóniából valók és Pálnak útitársai valának.
\par 30 Pál pedig mikor a nép közé akara menni, nem ereszték õt a tanítványok.
\par 31 És az ázsiai fõpapok közül is némelyek, kik barátai valának néki, küldvén õ hozzá, kérék, hogy ne menjen a színházba.
\par 32 Már most ki egyet, ki mást kiáltoz vala, mert a népgyûlés összezavarodott volt, és a többség nem tudta, miért gyûltek össze.
\par 33 A sokaság közül pedig elõállaták Alekszándert, minthogy elõre tuszkolták õt a zsidók. Alekszánder pedig kezével intvén, védekezni akara a nép elõtt.
\par 34 Megismervén azonban, hogy zsidó, egy kiáltás tört ki mindnyájokból, mintegy két óra hosszáig kiáltozván: Nagy az efézusi Diána!
\par 35 Miután pedig a városi jegyzõ lecsendesítette a sokaságot, monda: Efézusbeli férfiak, ugyan kicsoda az az ember, a ki ne tudná, hogy Efézus városa a nagy Diána istenasszonynak és a Jupitertõl esett képnek templomõrzõje?
\par 36 Mivelhogy azért ezeknek senki ellene nem szólhat, szükség, hogy megcsendesedjetek, és semmi vakmerõ dolgot ne cselekedjetek.
\par 37 Mert ide hoztátok az embereket, kik sem nem szentségrontók, sem a ti istenasszonyotok ellen káromlást nem szóltak.
\par 38 Ha tehát Demeternek és a hozzátartozó mesterembereknek valaki ellen panaszuk van, törvényszékek vannak és tisztartók vannak: pereljenek egymással.
\par 39 Ha pedig egyéb dolgok felõl van valami panasztok, a törvényes népgyûlésen majd elintéztetik.
\par 40 Mert félõ, hogy lázadással vádoltatunk a mai napért, mivelhogy semmi ok sincs, a melylyel számot tudnánk adni ezért a csõdületért. És ezeket mondván, feloszlatá a gyûlést.

\chapter{20}

\par 1 Minekutána pedig megszûnt a háborúság, magához híván Pál a tanítványokat és tõlük búcsút vévén, elindula, hogy Maczedóniába menjen.
\par 2 Miután pedig azokat a tartományokat eljárta, és intette õket bõ beszéddel, Görögországba méne.
\par 3 És ott töltött három hónapot. És mivelhogy a zsidók lest hánytak néki, a mint Siriába készült hajózni, úgy végezé, hogy Maczedónián át tér vissza.
\par 4 Kíséré pedig õt Ázsiáig a béreai Sopater,  a Thessalonikabeliek közül pedig Aristárkhus és Sekundus, és a derbei Gájus és Timótheus; Ázsiabeliek pedig Tikhikus  és Trofimus.
\par 5 Ezek elõremenvén, megvárának minket Troásban.
\par 6 Mi pedig a kovásztalan kenyerek napjai után kievezénk Fillippibõl, és menénk õ hozzájok Troásba öt nap alatt; hol hét napot tölténk.
\par 7 A hétnek elsõ napján pedig a tanítványok egybegyûlvén a kenyér megszegésére, Pál prédikál vala nékik, mivelhogy másnap el akara menni; és a tanítást megnyújtá éjfélig.
\par 8 Vala pedig elegendõ szövétnek abban a felházban, a hol egybe valának gyülekezve.
\par 9 Egy Eutikhus nevû ifjú pedig ül vala az ablakban, mély álomba merülve: és mivelhogy Pál sok ideig orédikála, elnyomatván az álom által, aláesék a harmadik rend házból, és halva véteték föl.
\par 10 Pál pedig alámenvén, reá borula, és magához ölelve monda: Ne háborogjatok; mert a lelke benne van.
\par 11 Azután fölméne, és megszegé a kenyeret és evék, és sokáig, mind virradatig beszélgetvén, úgy indula el.
\par 12 Felhozák pedig az ifjat elevenen, és felette igen megvigasztalódának.
\par 13 Mi pedig elõremenvén a hajóra, Assusba evezénk, ott akarván fölvenni Pált; mert így rendelkezett, õ maga gyalog akarván jõni.
\par 14 Mikor pedig Assusban összetalálkozott velünk, felvévén õt, menénk Mitilénébe.
\par 15 És onnét elevezvén, másnap eljutánk Khius ellenébe; a következõn pedig áthajózánk Sámusba; és Trogilliumban megszállván, másnap mentünk Milétusba.
\par 16 Mert elvégezé Pál, hogy Efézus mellett elhajózik, hogy ne kelljen néki idõt múlatni Ázsiában; mert siet vala, hogy ha lehetne néki, pünkösd napjára Jeruzsálemben legyen.
\par 17 Milétusból azonban küldvén Efézusba, magához hívatá a gyülekezet véneit.
\par 18 Mikor pedig hozzá mentek, monda nékik: Ti tudjátok, hogy az elsõ naptól fogva, melyen Ázsiába jöttem, mint viseltem magamat ti köztetek az egész idõ alatt,
\par 19 Szolgálván az Úrnak teljes alázatossággal és sok könnyhullatás és kisértetek közt, melyek én rajtam a zsidóknak utánam való  leselkedése miatt estek;
\par 20 Hogy semmitõl sem vonogattam magamat, a mi hasznos, hogy hirdessem néktek, és tanítsalak titeket nyilvánosan és házanként,
\par 21 Bizonyságot tévén mind zsidóknak, mind görögöknek az Istenhez való megtérés, és a mi Urunk Jézus Krisztusban való hit  felõl.
\par 22 És most ímé én a Lélektõl kényszerítve megyek Jeruzsálembe, nem tudván, mik következnek ott én reám.
\par 23 Kivéve, hogy a Szent Lélek városonként bizonyságot tesz, mondván, hogy én reám fogság és nyomorúság következik.
\par 24 De semmivel sem gondolok, még az én életem sem drága nékem, csakhogy elvégezhessem az én futásomat örömmel, és azt a szolgálatot,  melyet vettem az Úr Jézustól, hogy bizonyságot tegyek az Isten kegyelmének evangyéliomáról.
\par 25 És most íme én tudom, hogy nem látjátok többé az én orczámat ti mindnyájan, kik között általmentem, prédikálván az Istennek országát.
\par 26 Azért bizonyságot teszek elõttetek a mai napon, hogy én mindeneknek vérétõl tiszta vagyok.
\par 27 Mert nem vonogattam magamat, hogy hirdessem néktek az Istennek teljes akaratát.
\par 28 Viseljetek gondot azért magatokra és az egész nyájra,  melyben a Szent Lélek titeket vigyázókká tett, az Isten anyaszentegyházának legeltetésére, melyet tulajdon vérével szerzett.
\par 29 Mert én tudom azt, hogy az én eltávozásom után jõnek ti közétek gonosz farkasok, kik nem kedveznek a nyájnak.
\par 30 Sõt ti magatok közül is támadnak férfiak, kik fonák dolgokat beszélnek, hogy a tanítványokat magok után vonják.
\par 31 Azért vigyázzatok, megemlékezvén arról, hogy én három esztendeig éjjel és nappal meg nem szüntem könnyhullatással inteni mindenkit.
\par 32 És most, atyámfiai, ajánlak titeket az Istennek és az õ kegyelmessége ígéjének, a ki felépíthet és adhat néktek örökséget minden megszenteltek közt.
\par 33 Senkinek ezüstjét, vagy aranyát, vagy ruháját nem kívántam:
\par 34 Sõt magatok tudjátok, hogy a magam szükségeirõl és a velem valókról ezek a kezek gondoskodtak.
\par 35 Mindenestõl megmutattam néktek, hogy ily módon munkálkodva kell az erõtlenekrõl gondot viselni, és megemlékezni az Úr Jézus szavairól, mert õ mondá: Jobb adni, mint venni.
\par 36 És mikor ezeket mondotta, térdre esve imádkozék mindazokkal egybe.
\par 37 Mindnyájan pedig nagy sírásra fakadtak; és Pálnak nyakába borulva csókolgaták õt.
\par 38 Keseregve kiváltképen azon a szaván, a melyet mondott, hogy többé az õ orczáját nem fogják látni. Aztán elkísérték õt a hajóra.

\chapter{21}

\par 1 A mint pedig, õ tõlük elszakadván, elindultunk, egyenesen haladva Kóusba érkezénk, másnap pedig Rhodusba, és onnét Patarába.
\par 2 És mikor találtunk egy hajót, mely Fenicziába méne által, abba beülvén, elhajózánk.
\par 3 És miután megláttuk Cziprust és elhagytuk azt balkézre, evezénk Siriába, és Tirusban köténk ki: mert a hajó ott rakja vala ki a terhét.
\par 4 És ott maradánk hét napig, miután feltaláltuk a tanítványokat, kik Pálnak mondják vala a Lélek által, hogy ne menjen fel  Jeruzsálembe.
\par 5 Mikor pedig eltöltöttük azokat a napokat, kimenvén, elutazánk; kikísérvén bennünket mindnyájan feleségestõl, gyermekestõl egészen a városon kívülre. És a tenger partján térdre esve imádkozánk.
\par 6 És egymástól elbúcsúzván, beülénk a hajóba, azok pedig megtérének az övéikhez.
\par 7 Mi pedig a hajózást bevégezvén, Tirusból eljutánk Ptolemaisba; és köszöntvén az atyafiakat, nálok maradánk egy napig.
\par 8 Másnap pedig elmenvén Pál és mi, kik õ vele valánk, érkezénk Czezáreába; és bemenvén a Filep evangyélista házába, ki ama hét közül való vala, õ nála maradánk.
\par 9 Ennek pedig vala négy szûz leánya, a kik prófétálnak vala.
\par 10 Mialatt pedig mi több napig ott maradánk, alájöve egy Júdeából való próféta, névszerint Agabus.
\par 11 És mikor hozzánk jött, vevé Pálnak az övét, és megkötözvén a maga kezeit és lábait, monda: Ezt mondja a Szent Lélek: A férfiút, a kié ez az öv, ekképen kötözik meg a zsidók Jeruzsálemben, és adják a pogányoknak kezébe.
\par 12 Mikor pedig ezeket hallottuk, kérõk, mind mi, mind az oda valók, hogy ne menjen fel Jeruzsálembe.
\par 13 De Pál felele: Mit míveltek sírván és az én szívemet kesergetvén? mert én nemcsak megkötöztetni, hanem meghalni is kész vagyok Jeruzsálemben az Úr Jézusnak nevéért.
\par 14 Mikor azért nem engedett, megnyugodtunk, mondván: Legyen meg az Úrnak akaratja.
\par 15 E napok után pedig felkészülõdvén, felmenénk Jeruzsálembe.
\par 16 Jövének pedig mi velünk együtt a tanítványok közül is Czézáreából, kik elvezetének bizonyos cziprusi Mnásonhoz, egy régi tanítványhoz, hogy ott legyen szállásunk.
\par 17 Mikor azért Jeruzsálembe jutottunk, örömmel fogadának minket az atyafiak.
\par 18 Másnap pedig beméne Pál velünk együtt Jakabhoz; és a vének mindnyájan ott valának.
\par 19 És köszöntvén õket, elbeszélé egyenként, a miket az Isten a pogányok között az õ szolgálata által cselekedett.
\par 20 Azok pedig ezt hallván, dicsõíték az Urat; és mondának néki: Látod, atyámfia, mely sok ezeren vannak zsidók, kik hívõkké lettek; és mindnyájan buzognak a törvény mellett:
\par 21 Felõled pedig azt hallották, hogy te mindazokat a zsidókat, kik a pogányok között vannak, Mózestõl való elszakadásra tanítod, azt mondván, hogy ne metéljék körül fiaikat, se a zsidó szokások szerint ne járjanak.
\par 22 Micsoda annakokáért? Mindenesetre össze kell gyülekezni a sokaságnak; mert meghallják, hogy ide jöttél.
\par 23 Ezt míveld azért, a mit néked mondunk: Van mi köztünk négy férfiú, kik fogadalmat vettek magokra;
\par 24 Ezeket magad mellé vévén, tisztulj meg velök, és költs rájok, hogy megnyíressék fejöket: és megtudják mindenek, hogy semmi sincs azokban, a miket te felõled hallottak; hanem te magad is úgy jársz, hogy a törvényt megtartod.
\par 25 A pogányokból lett hívõk felõl pedig mi írtunk, azt végezvén, hogy õk semmi ilyenfélét ne tartsanak meg, hanem csak oltalmazzák meg magokat mind a bálványoknak áldozott hústól, mind a vértõl, mind a fúlvaholt állattól, mind a paráznaságtól.
\par 26 Akkor Pál maga mellé véve azokat a férfiakat, másnap õ velök megtisztulván, beméne a templomba, bejelentvén a tisztulás napjainak eltelését, a míg mindegyikökért elvégeztetik az áldozat.
\par 27 Mikor pedig a hét nap immár eltelõben volt, az Ázsiából való zsidók, meglátván õt a templomban, felindíták az egész sokaságot, és reá veték kezöket,
\par 28 Kiáltván: Izraelita férfiak, legyetek segítségül: ez az az ember, ki e nép ellen, a törvény ellen és e hely ellen tanít mindenkit mindenütt; ezen felül még görögöket is hozott be a templomba, és megfertéztette ezt a szent helyet.
\par 29 Mert látták vala annakelõtte az efézusi Trofimust õ vele a városban, kirõl azt vélék, hogy Pál bevitte a templomba.
\par 30 Megmozdula azért az egész város, és a nép összecsõdüle: és Pált megragadván, vonszolják vala ki õt a templomból: és mindjárt bezáratának az ajtók.
\par 31 Mikor pedig meg akará õt ölni, feljuta a hír a sereg ezredeséhez, hogy az egész Jeruzsálem felzendült.
\par 32 Ki azonnal vitézeket és századosokat vévén maga mellé, lefutott hozzájok. Azok pedig mikor meglátták az ezredest és a vitézeket, megszûnének Pált verni.
\par 33 Akkor odaérvén az ezredes, elfogatá õt, és parancsolá, hogy kötözzék meg két lánczczal; és tudakozá, hogy kicsoda és mit cselekedett.
\par 34 De ki egyet, ki mást kiált vala a sokaság között; és mikor nem értheté meg a bizonyos valóságot a zajongás miatt, parancsolá, hogy vigyék el õt a várba.
\par 35 Mikor pedig a lépcsõkhöz jutott, lõn, hogy úgy vivék õt a vitézek a néptömeg erõszaktétele miatt;
\par 36 Mert követi vala a népnek sokasága, kiáltozva: Öld meg õt!
\par 37 És mikor immár a várba akarák bevinni Pált, monda az ezredesnek: Vajjon szabad-e nékem valamit szólanom te hozzád? Az pedig monda: Tudsz görögül?
\par 38 Hát nem te vagy az az egyiptomi, ki e napoknak elõtte fellázította és kivitte a pusztába azt a négyezer orgyilkos férfiút?
\par 39 Monda pedig Pál: Én ugyan tárzusi zsidó ember vagyok, Cziliczia nem ismeretlen városának polgára; de kérlek téged, engedd meg nékem, hogy szóljak a néphez.
\par 40 Mikor aztán az megengedte, Pál a lépcsõkön állva intett kezével a népnek: és mikor nagy csendesség lõn, megszólala zsidó nyelven, mondván:

\chapter{22}

\par 1 Atyámfiai, férfiak és atyák, hallgassátok meg az én beszédemet, a melylyel most magamat elõttetek mentem.
\par 2 Mikor pedig hallották, hogy zsidó nyelven szól hozzájok, még inkább nyugalmat tanusítottak. És monda:
\par 3 Én zsidó ember vagyok, születtem a czilicziai Tárzusban, fölneveltettem pedig ebben a városban a Gamáliel lábainál, taníttattam az atyák törvényének  pontossága szerint, buzgó lévén az Istenhez, miként ti mindnyájan vagytok ma:
\par 4 És ezt a tudományt üldöztem mind halálig, megkötözvén és tömlöczbe vetvén mind férfiakat, mind asszonyokat.
\par 5 Miképen a fõpap is bizonyságom nékem, és a véneknek egész tanácsa; kiktõl leveleket is vévén az atyafiakhoz, Damaskusba menék, hogy az odavalókat is fogva hozzam Jeruzsálembe, hogy bûnhõdjenek.
\par 6 Lõn pedig, hogy a mint menék és közelgeték Damaskushoz, déltájban nagy hirtelenséggel az égbõl nagy világosság sugárzott körül engem.
\par 7 És leesém a földre, és hallék szót, mely monda nékem: Saul, Saul, mit kergetsz engem?
\par 8 Én pedig felelék: Kicsoda vagy, Uram? És monda nékem: Én vagyok a názáreti Jézus, a kit te kergetsz.
\par 9 A kik pedig velem valának, a világosságot ugyan látták, és megrémültek; de annak szavát, a ki velem szól vala, nem hallották.
\par 10 Én pedig mondék: Mit cselekedjem, Uram? Az Úr pedig monda nékem: Kelj fel és menj el Damaskusba; és ott megmondják néked mindazokat, a mik elrendelvék néked, hogy véghez vigyed.
\par 11 Mikor pedig nem láték annak a világosságnak dicsõsége miatt, a velem valóktól kézenfogva vezettetve menék Damaskusba.
\par 12 Egy bizonyos Ananiás pedig, ki a törvény szerint istenfélõ férfiú, kirõl az ott lakó zsidók mind jó bizonyságot tesznek,
\par 13 Hozzám jöve és mellém állva monda nékem: Saul atyámfia, nyerd vissza szemed világát. És én azon szempillantásban reá tekintettem.
\par 14 Õ pedig monda: A mi atyáinknak Istene választott téged, hogy megismerd az õ akaratát, és meglásd amaz Igazat, és szót hallj az õ szájából.
\par 15 Mert leszel néki tanúbizonysága minden embernél azok felõl, a miket láttál és hallottál.
\par 16 Most annakokáért mit késedelmezel? Kelj fel és keresztelkedjél meg és mosd le a te bûneidet, segítségül híván az Úrnak nevét.
\par 17 Lõn pedig, hogy mikor Jeruzsálembe megtértem és imádkozám a templomban, elragadtatám lelkemben,
\par 18 És látám õt, ki ezt mondá nékem: Siess és menj ki hamar Jeruzsálembõl: mert nem veszik be a te tanúbizonyságtételedet én felõlem.
\par 19 És én mondék: Uram, õk magok tudják, hogy én tömlöczbe vetettem és vertem zsinagógánként azokat, a kik hisznek vala te benned:
\par 20 És mikor ama te mártírodnak, Istvánnak vére kiontaték, én is ott állék és helyeslém az õ megöletését, és õrizém azoknak köntösét, a kik õt megölték.
\par 21 És monda nékem: Eredj el, mert én téged messze küldelek a pogányok közé.
\par 22 Hallgatják vala pedig õt e szóig; de most felemelék szavokat, mondván: Töröld el a földszínérõl az ilyent, mert nem illik néki élni.
\par 23 Mikor pedig azok kiabáltak, és köntösüket elhányák, és port szórának a levegõbe,
\par 24 Parancsolá az ezredes, hogy vigyék õt a várba, mondván, hogy korbácsütésekkel vallassák ki õt, hogy megtudhassa, mi okért kiabáltak úgy reá.
\par 25 A mint azonban lekötötték õt a szíjakkal, monda Pál az ott álló századosnak: Vajjon szabad-é néktek római embert, kit el nem ítéltek, megostorozni?
\par 26 Miután pedig ezt meghallá a százados, elmenvén, megjelenté az ezredesnek, mondván: Meglásd, mit akarsz cselekedni; mert ez az ember római.
\par 27 Hozzámenvén azért az ezredes, monda néki: Mondd meg nékem, te római vagy-é? Õ pedig monda: Az.
\par 28 És felele az ezredes: Én nagy összegért vettem meg ezt a polgárjogot. Pál pedig monda: Én pedig benne is születtem.
\par 29 Mindjárt eltávozának azért õ tõle, a kik õt vallatni akarák. Sõt az ezredes is megijede, mikor megértette, hogy római, és hogy õt megkötöztette.
\par 30 Másnap pedig meg akarván tudni a bizonyos valóságot, miben vádoltatik a zsidóktól, feloldatá õt bilincseibõl, és megparancsolá, hogy a fõpapok az õ egész tanácsokkal egyben hozzá menjenek; és levezetvén Pált, eleikbe állatá.

\chapter{23}

\par 1 Mikor pedig a tanácsra vetette szemét Pál, monda: Atyámfiai, férfiak, én teljes jó lelkiismerettel szolgáltam az Istennek mind e mai napig.
\par 2 Ananiás fõpap pedig megparancsolá azoknak, kik õ mellette állanak vala, hogy üssék õt szájon.
\par 3 Akkor Pál monda néki: Megver az Isten téged, te kimeszelt fal! És te leülsz engem a törvény  szerint megítélni, és törvényellenesen cselekedve parancsolod, hogy engem verjenek?
\par 4 Az ott állók pedig mondának: Az Istennek fõpapját szidalmazod-é?
\par 5 Pál pedig monda: Nem tudtam, atyámfiai, hogy fõpap. Mert meg van írva: A te néped fejedelmét ne átkozd!
\par 6 Mikor pedig Pál eszébe vette, hogy az egyik részök a sadduczeusok, a másik pedig a farizeusok közül való, felkiálta a tanács elõtt: Atyámfiai, férfiak, én farizeus vagyok, farizeus fia, a halottak reménysége és feltámadása miatt vádoltatom én.
\par 7 A mint pedig õ ezt mondta, meghasonlás támada a farizeusok és a sadduczeusok között, és a sokaság megoszlott.
\par 8 Mert a sadduczeusok azt mondják, hogy nincs feltámadás, sem angyal, sem lélek; a farizeusok pedig mind a kettõt vallják.
\par 9 Támada azért nagy kiáltozás: és felkelvén az írástudók a farizeusok pártjából, tusakodnak vala, mondván: Semmi rosszat sem találunk ez emberben; ha pedig lélek szólott néki, vagy angyal, ne tusakodjunk Isten ellen.
\par 10 Mikor pedig nagy hasonlás támadt, félvén az ezredes, hogy Pál szétszaggattatik azoktól, parancsolá, hogy a sereg alájõvén, ragadja ki õt közülök, és vigye el a várba.
\par 11 A következõ éjszakán pedig melléállván néki az Úr, monda: Bízzál Pál! Mert miképen bizonyságot tettél az én felõlem való dolgokról Jeruzsálemben, azonképen kell néked Rómában  is bizonyságot tenned.
\par 12 Midõn pedig nappal lõn, a zsidók közül némelyek összeszövetkezvén, átok alatt kötelezék el magokat, mondván, hogy sem nem esznek, sem nem isznak addig, míg meg nem ölik Pált.
\par 13 Többen valának pedig negyvennél, kik ezt az összeesküvést szõtték.
\par 14 Ezek elmenvén a fõpapokhoz és a vénekhez, mondának: Átok alatt megesküdtünk, hogy semmit nem ízlelünk addig, míg meg nem öljük Pált.
\par 15 Most azért ti jelentsétek be az ezredesnek a tanácscsal egybe, hogy holnap hozza le õt ti hozzátok, mintha az õ dolgának tüzetesebben végére akarnátok járni. Mi pedig, minekelõtte õ ide érne, készek vagyunk õt megölni.
\par 16 Meghallván azonban a Pál nõtestvérének fia e cselvetést, megjelenvén és bemenvén a várba, tudtára adá Pálnak.
\par 17 Pál pedig egyet a századosok közül magához hívatván, monda: Ezt az ifjat vezesd az ezredeshez; mert valamit akar néki jelenteni.
\par 18 Az annakokáért maga mellé vévén õt, vivé az ezredeshez, és monda: A fogoly Pál magához hivatván engem, kéré, hogy ez ifjat hozzád vezessem, mert valamit akar néked mondani.
\par 19 Az ezredes pedig õt kézen fogván, és félrevonulván külön, tudakolá: Micsoda az, a mit nékem jelenteni akarsz?
\par 20 Az pedig monda: A zsidók elvégezték, hogy megkérnek téged, hogy Pált holnap vidd le a tanács elé, mintha valamit tüzetesebben meg akarnának tudakozni õ felõle.
\par 21 Te azért ne engedj nékik: mert közülök negyvennél több férfiú leselkedik õ utána, kik átok alatt kötelezték el magukat, hogy sem nem esznek, sem nem isznak addig, míg meg nem ölik õt. És immár készen vannak, várakozva a te izenetedre.
\par 22 Az ezredes tehát elbocsátá az ifjat, meghagyván néki, hogy el ne mondd senkinek, hogy ezeket megjelentetted nékem.
\par 23 És magához hivatván kettõt a századosok közül, monda: Készítsetek föl kétszáz vitézt, hogy induljanak Czézáreába, és hetven lovast, és kétszáz parittyást, az éjszakának harmadik órájától fogva.
\par 24 És hogy barmokat is adjanak melléjük, hogy Pált felültetve békességgel vigyék Félix tiszttartóhoz.
\par 25 Levelet is írt, melynek tartalma ez vala:
\par 26 Klaudius Lisiás a nemes Félix tiszttartónak üdvöt!
\par 27 Ezt a férfiút, kit a zsidók megfogtak és meg akartak ölni, oda menvén a sereggel, kiszabadítám, megértvén, hogy római.
\par 28 Meg akarván pedig tudni az okát, miért vádolják õt, levivém õt az õ tanácsuk elébe.
\par 29 És úgy találtam, hogy õ az õ törvényüknek kérdései felõl vádoltatik, de semmi halálra vagy fogságra méltó vétke nincs.
\par 30 Minthogy pedig nékem megjelentették, hogy a zsidók e férfiú után ólálkodni akarnak, azonnal hozzád küldém, meghagyva a vádolóinak is, hogy a mi dolguk van õ ellene, te elõtted mondják meg. Légy jó egészségben!
\par 31 A vitézek tehát, a mint nékik megparancsolták, Pált felvévén, elvivék azon éjszakán Antipatrisba.
\par 32 Másnap pedig hagyván a lovagokat tovább menni õ vele, visszatérének a várba.
\par 33 Azok pedig eljutván Czézáreába, és átadván a levelet a tiszttartónak, Pált is elébe állaták.
\par 34 Mikor pedig elolvasta a tiszttartó, és megkérdezte, melyik tartományból való, és megértette, hogy Czilicziából,
\par 35 Monda: Majd kihallgatlak, mikor vádlóid is eljõnek. És parancsolá, hogy a Héródes palotájában õrizzék õt.

\chapter{24}

\par 1 Öt nap mulva aztán aláméne Ananiás fõpap a vénekkel és egy Tertullus nevû prókátorral, kik panaszt tettek a tiszttartónál Pál ellen.
\par 2 Mikor pedig õ elõszólíttatott, Tertullus vádolni kezdé, mondván:
\par 3 Nagyságos Félix, teljes háládatossággal ismerjük el, hogy te általad nagy békességet nyerünk, és a te gondoskodásod folytán igen jó intézkedések történnek e népre nézve, minden tekintetben és mindenütt.
\par 4 De hogy téged sok ideig ne tartóztassalak, kérlek hallgass meg minket röviden a te kegyelmességed szerint.
\par 5 Mi ugyanis úgy találtuk, hogy ez veszedelmes ember, és hasonlást támaszt a föld kerekségén levõ valamennyi zsidó közt, és a nazarénusok felekezetének feje,
\par 6 Ki a templomot is meg akarta fertõztetni. Meg is fogtuk õt, és a mi törvényünk szerint akartuk megítélni.
\par 7 De Lisiás, az ezredes, nagy karhatalommal oda jövén, kivevé õt kezünkbõl.
\par 8 És azt parancsolá, hogy az õ vádolói hozzád jõjjenek. Tõle te magad, ha kihallgatod, értesülhetsz mindezekrõl, melyekkel mi õt vádoljuk.
\par 9 Helybenhagyák pedig a zsidók is, mondogatván, hogy úgy vannak ezek.
\par 10 Felele pedig Pál, miután intett néki a tiszttartó a szólásra: Mivelhogy tudom, hogy te sok esztendõ óta vagy e népnek bírája, bátorságosabban védekezem a magam ügyében,
\par 11 Mert megtudhatod, hogy nincsen tizenkét napjánál több, mióta feljöttem imádkozni Jeruzsálembe.
\par 12 És a templomban sem találtak engem, hogy valakivel vetekedtem volna, vagy hogy a népet egybezendítettem volna, sem a zsinagógákban, sem a városban.
\par 13 Rám sem bizonyíthatják azokat, a mikkel most engem vádolnak.
\par 14 Errõl pedig vallást teszek néked, hogy én a szerint az út szerint, melyet felekezetnek mondanak, úgy szolgálok az én atyáim Istenének, mint a ki hiszek mindazokban, a mik a törvényben és a prófétákban meg vannak írva.
\par 15 Reménységem lévén az Istenben, hogy a mit ezek maguk is várnak, lesz feltámadásuk a halottaknak, mind igazaknak, mind hamisaknak.
\par 16 Ebben gyakorlom pedig magamat, hogy botránkozás nélkül való lelkiismeretem legyen az Isten és emberek elõtt mindenkor.
\par 17 Sok esztendõ múlva pedig eljövék, hogy az én népemnek alamizsnát hozzak és áldozatokat.
\par 18 Ezek közben találának engem megtisztulva a templomban, nem sokasággal, sem pedig háborúságtámasztásban, némely Ázsiából való zsidók,
\par 19 Kiknek ide kellett volna te elõdbe jõni és vádolni, ha valami panaszuk volna ellenem.
\par 20 Avagy ezek magok mondják meg, vajjon találtak-é bennem valami hamis cselekedetet, mikor én a tanács elõtt álltam;
\par 21 Hacsak ez egy szó tekintetében nem, melyet közöttük állva kiáltottam, hogy: A halottak feltámadása felõl vádoltatom én tõletek e mai napon.
\par 22 Mikor pedig ezeket hallotta Félix, elhalasztá dolgukat, mivelhogy tüzetesebb tudomása volt e szerzet dolgai felõl, és monda: Mikor Lisias ezredes alájõ, dönteni fogok ügyetekben.
\par 23 És megparancsolá a századosnak, hogy Pált õrizzék, de enyhébb fogságban legyen, és senkit ne tiltsanak el az övéi közül attól, hogy szolgáljon néki, vagy hozzá menjen.
\par 24 Egynéhány nap mulva pedig Félix megjelenvén feleségével Drusillával egybe, ki zsidó asszony vala, maga elé hívatá Pált, és hallgatá õt a Krisztusban való hit felõl.
\par 25 Mikor pedig õ igazságról, önmegtartóztatásról és az eljövendõ ítéletrõl szólt, megrémülve monda Félix: Mostan eredj el; de mikor alkalmatosságom lesz, magamhoz hivatlak téged.
\par 26 Egyszersmind pedig azt is reményli vala, hogy Pál pénzt ad néki, hogy õt szabadon bocsássa: ezért gyakrabban is magához hivatván õt, beszélget vala véle.
\par 27 Mikor pedig két esztendõ elmúlt, Félix utóda Porcius Festus lõn; és a zsidóknak kedveskedni akarván Félix, Pált fogságban hagyá.

\chapter{25}

\par 1 Festus tehát, miután bement a tartományba, három nap mulva felméne Jeruzsálembe Czézáreából.
\par 2 Panaszt tõnek pedig néki a fõpap és a zsidók fõemberei Pál ellen, és kérék õt,
\par 3 Kérvén magok számára jóindulatát õ ellene, hogy hozassa át õt Jeruzsálembe, lest vetvén, hogy megölhessék õt az úton.
\par 4 Festus azonban azt felelé, hogy Pált Czézáreában õrzik, õ maga pedig csakhamar ki fog menni:
\par 5 A kik azért köztetek, úgymond, fõemberek, velem alájõvén, ha valami gonoszság van abban a férfiúban, emeljenek vádat ellene.
\par 6 Miután pedig tíz napnál tovább idõzött közöttük, lemenvén Czézáreába, másnap ítélõszékibe üle, és Pált elõhozatá.
\par 7 Mikor pedig az megjelent, körülállák a zsidók, kik alámentek vala Jeruzsálembõl, sok és súlyos vádat hozván fel Pál ellen, melyeket nem bírtak bebizonyítani;
\par 8 Mivelhogy õ a maga mentségére ezt feleli vala: Sem a zsidók törvénye ellen, sem a templom ellen, sem a császár ellen semmit sem vétettem.
\par 9 Festus pedig a zsidóknak kedveskedni akarván, felelvén Pálnak, monda: Akarsz-é Jeruzsálembe felmenni és ott ítéltetni meg ezekrõl én elõttem?
\par 10 Pál azonban monda: A császár ítélõszéke elõtt állok, itt kell nékem megítéltetnem. A zsidóknak semmit sem vétettem, miként te is jól tudod.
\par 11 Mert ha vétkes vagyok és valami halálra méltót cselekedtem, nem vonakodom a haláltól; ha azonban semmi sincs azokban, a mikkel ezek vádolnak engem, senki sem ajándékozhat oda engem azoknak. A császárra appellálok!
\par 12 Akkor Festus tanácsával értekezvén, felele: A császárra appelláltál, a császár elé fogsz menni!
\par 13 Néhány nap elmúltával pedig Agrippa király és Bernicé érkezék Czézáreába, hogy köszöntsék Festust.
\par 14 Mikor pedig több napig idõztek ott, Festus elébe adá a királynak a Pál dolgát, mondván: Van itt egy férfiú, kit Félix hagyott fogva.
\par 15 Ki felõl, mikor Jeruzsálembe mentem, jelentést tõnek a fõpapok és a zsidóknak vénei, kérve õ ellene ítéletet.
\par 16 Kiknek azt felelém, hogy nem szokásuk a rómaiaknak, hogy valamely embert halálra adjanak, mielõtt a vádlott szembe nem állíttatik vádlóival, és alkalmat nem nyer a vád felõl való mentségére.
\par 17 Mikor azért õk ide gyûltek, minden haladék nélkül másnap az ítélõszékbe ülvén, elõhozatám azt a férfiút,
\par 18 Ki ellen, mikor vádlói elõálltak, semmi bûnt nem hoztak fel azok közül, a miket én sejtettem:
\par 19 Hanem valami vitás kérdéseik valának õ vele az õ tulajdon babonaságuk felõl, és bizonyos megholt Jézus felõl, kirõl Pál azt állítja vala, hogy él.
\par 20 Én pedig bizonytalanságban lévén az e felõl való vitára nézve, kérdém, vajjon akar-é Jeruzsálembe menni, és ott ítéltetni meg ezek felõl.
\par 21 Pál azonban appellálván, hogy õ Augustus döntésére tartassék fenn, parancsolám, hogy tartassék fogva, míg õt a császárhoz nem küldhetem.
\par 22 Agrippa pedig monda Festusnak: Szeretném magam is azt az embert hallani. Õ pedig monda: Holnap meg fogod õt hallani.
\par 23 Másnap tehát eljõvén Agrippa és Bernicé nagy pompával, és bemenvén a kihallgatási terembe az ezredesekkel és a város fõfõpolgáraival együtt, Festus parancsolatjára elõhozák Pált.
\par 24 És monda Festus: Agrippa király, és ti férfiak mindnyájan, kik velünk egybe itt vagytok! Látjátok õt, ki felõl a zsidóknak egész sokasága megkeresett engem, mind Jeruzsálemben, mind itt, azt kiáltva, hogy nem kell néki tovább élnie.
\par 25 Én pedig, ámbár megértém, hogy semmi halálra méltó dolgot sem cselekedett, de mivel éppen õ maga appellált Augustusra, elvégeztem, hogy elküldöm õt.
\par 26 Ki felõl nem tudok valami bizonyost írni az én uramnak. Ezért hoztam õt elõtökbe, és kiváltképen te elõdbe, Agrippa király, hogy a kihallgatás megtörténtével tudjak mit írni.
\par 27 Mert esztelen dolognak látszik nékem, hogy a ki foglyot küld, az ellene való vádakat is meg ne jelentse.

\chapter{26}

\par 1 Agrippa pedig monda Pálnak: Megengedtetik néked, hogy szólj a magad mentségére. Akkor Pál kinyújtván kezét, védõbeszédet tartott:
\par 2 Agrippa király! Boldognak tartom magamat, hogy mindazok felõl, a mikkel a zsidóktól vádoltatom, te elõtted fogok védekezni e mai napon;
\par 3 Mivel te nagyon jól ismered a zsidók minden szokását és vitás kérdését. Azért kérlek, hallgass meg engem türelmesen!
\par 4 Az én ifjúságomtól fogva való életemet tehát, mely kezdetétõl az én népem közt Jeruzsálemben folyt le, tudják a zsidók mindnyájan.
\par 5 Kik tudják rólam eleitõl fogva (ha bizonyságot akarnak tenni), hogy én a mi vallásunknak legszigorúbb felekezete szerint éltem, mint farizeus.
\par 6 Most is az Istentõl a mi atyáinknak tett ígéret reménységéért állok itt ítélet alatt:
\par 7 Melyre a mi tizenkét nemzetségünk, éjjel és nappal buzgón szolgálva reményli, hogy eljut; mely reménységért vádoltatom, Agrippa király, a zsidóktól.
\par 8 Micsoda? Hihetetlen dolognak tetszik néktek, hogy Isten halottakat támaszt fel?
\par 9 Én bizonyára elvégeztem vala magamban, hogy ama názáreti Jézus neve ellen sok ellenséges dolgot kell cselekednem.
\par 10 Mit meg is cselekedtem Jeruzsálemben: és a szentek közül én sokat börtönbe vettettem, a fõpapoktól való felhatalmazást megnyervén. Sõt mikor megölettetének,  szavazatommal hozzájárultam.
\par 11 És minden zsinagógában gyakorta büntetvén õket, káromlásra kényszerítettem; és felettébb dühösködvén ellenök, kergettem mind az idegen városokig is.
\par 12 E dologban épen útban lévén Damaskus felé a fõpapoktól vett felhatalmazással és engedelemmel,
\par 13 Délben látám az úton király, hogy mennybõl a napnak fényességét meghaladó világosság sugárzott körül engem és azokat, kik velem együtt haladnak vala.
\par 14 Mikor pedig mi mindnyájan leestünk a földre, hallék szózatot, mely én hozzám szól és ezt mondja vala zsidó nyelven: Saul, Saul, mit kergetsz engem? Nehéz néked az ösztön ellen rúgódoznod.
\par 15 Én pedig mondék: Kicsoda vagy, Uram? És az monda: Én vagyok Jézus, a kit te kergetsz.
\par 16 De kelj fel, és állj lábaidra: mert azért jelentem meg néked, hogy téged szolgává és bizonysággá rendeljelek úgy azokban, a miket láttál, mint azokban, a mikre nézve meg fogok néked jelenni;
\par 17 Megszabadítván téged e néptõl és a pogányoktól, kik közé most küldelek,
\par 18 Hogy megnyissad szemeiket, hogy setétségbõl világosságra és a Sátánnak hatalmából az Istenhez térjenek, hogy bûneiknek bocsánatát és a megszenteltettek között  osztályrészt nyerjenek az én bennem való hit által.
\par 19 Azért, Agrippa király, nem levék engedetlen a mennyei látás iránt;
\par 20 Hanem elõször a Damaskusbelieknek és Jeruzsálembelieknek, majd Júdeának egész tartományában és a pogányoknak hirdettem, hogy bánják meg bûneiket és térjenek meg az  Istenhez, a megtéréshez méltó cselekedeteket cselekedvén.
\par 21 Ezekért akartak engem megölni a zsidók, megfogván a templomban.
\par 22 De Istentõl segítséget vévén, mind e mai napig állok, bizonyságot tévén mind kicsinynek, mind nagynak, semmit sem mondván azokon kívül, a mikrõl mind a próféták megmondották, mind Mózes, hogy be fognak teljesedni:
\par 23 Hogy a Krisztusnak szenvedni kell, hogy mint a halottak feltámadásából elsõ, világosságot  fog hirdetni e népnek és a pogányoknak.
\par 24 Mikor pedig õ ezeket mondá a maga mentségére, Festus nagy fenszóval monda: Bolond vagy te, Pál! A sok tudomány téged õrültségbe visz.
\par 25 Õ pedig monda: Nem vagyok bolond, nemes Festus, hanem igaz és józan beszédeket szólok.
\par 26 Mert tud ezekrõl a király, kihez bátorságosan is szólok: mert épen nem gondolom, hogy ezek közül õ elõtte bármi is ismeretlen volna; mert nem valami zugolyában lett dolog ez.
\par 27 Hiszel-e, Agrippa király, a prófétáknak? Tudom, hogy hiszel.
\par 28 Agrippa pedig monda Pálnak: Majdnem ráveszel engem, hogy keresztyénné legyek.
\par 29 Pál pedig monda: Kívánnám Istentõl, hogy necsak majdnem, hanem nagyon is, ne csak te, hanem mindazok is, kik ma engem hallgatnak, lennétek olyanok, a minõ én is vagyok, e bilincsektõl megválva.
\par 30 És mikor õ ezeket mondá, felkele a király és a tiszttartó és Bernicé és a kik velük együtt ültek;
\par 31 És visszavonultokban beszélgetnek vala egymással, mondván: Semmi halálra, vagy fogságra méltó dolgot nem cselekszik ez az ember.
\par 32 Agrippa pedig monda Festusnak: Ezt az embert szabadon lehetett volna bocsátani, ha a császárra nem appellált volna.

\chapter{27}

\par 1 Midõn pedig elvégeztetett, hogy mi Itáliába hajózzunk, átadák mind Pált, mind némely egyéb foglyokat egy Július nevû századosnak a császári seregbõl.
\par 2 Beülvén azért egy Adramittiumból való hajóba, az Ázsia mentében fekvõ helyeket akarván behajózni, elindulánk, velünk lévén a maczedóniai Aristárkhus, ki Thessalonikából való.
\par 3 És másnap megérkezénk Sidonba. És Július emberséggel bánván Pállal, megengedé, hogy barátaihoz elmenve gondoskodásukban részesüljön.
\par 4 És onnan elindulván, Ciprus alatt evezénk el, mivelhogy a szelek ellenkezõk valának.
\par 5 És a Cziliczia és Pámfilia mellett levõ tengeren átevezvén, eljutánk a licziai Mirába.
\par 6 És mivel ott a százados egy Itáliába menõ alexandriai hajót talált, abba szállított be minket.
\par 7 Több napon át azonban lassan hajózván és nehezen érkezvén Knidushoz, mivel nem enged vala bennünket odajutni a szél, elhajózánk Kréta alatt, Salmóné mellett,
\par 8 És nagy ügygyel-bajjal elhajózván mellette, jutánk egy helyre, melyet Szépkikötõknek neveznek, melyhez közel vala Lásea városa.
\par 9 Mivel pedig sok idõ mult el, és a hajózás más veszedelmes vala, mivelhogy a bõjt is elmult immár, inti vala Pál õket,
\par 10 Ezt mondván nékik: Férfiak, látom, hogy nemcsak a teréhnek és a hajónak, hanem a mi életünknek is bántódásával és nagy kárával fog történni e hajózás.
\par 11 De a százados inkább hisz vala a kormányosmesternek és a hajótulajdonosnak, hogynem annak, a mit Pál mond vala.
\par 12 Mivel pedig az a kikötõ telelésre nem volt alkalmas, a többség azt határozá, hogy hajózzanak el onnan is, ha valami módon eljutva Fénixbe, Kréta kikötõjébe, mely délnyugot és északnyugot felé néz, kitelelhetnének.
\par 13 Mivel pedig déli szél kezdett lassan fúni, azt gondolván, hogy feltett szándékuknak uraivá lettek, elindulván, közelebb hajóztak el Kréta mellett.
\par 14 Nemsokára azonban viharos szélvész csapott le oda, mely Észak-keleti szélnek neveztetik.
\par 15 Mikor pedig a hajó elragadtatott, és nem bírt a széllel szembe menni, nekieresztvén azt, vitetünk vala tova.
\par 16 Mikor pedig egy kis sziget alá futottunk, mely Klaudának hívattatik, csak alig bírtuk hatalmunkba keríteni a csolnakot.
\par 17 Melyet miután felvontak, védõeszközöket alkalmaznak vala, alól megövedzvén a hajót; és mivel félnek vala, hogy zátonyra bukkannak, lebocsátván a vitorlát, úgy vitetnek vala.
\par 18 Mikor pedig a szélvésztõl nagyon hányattatánk, másnap a hajóterhet kihányák;
\par 19 És harmadnap tulajdon kezeinkkel hányók ki a hajó felszerelését.
\par 20 Mikor pedig több napon át sem nap, sem csillagok nem látszottak, és nem kis vihar szorongatott, továbbra minden reménységünk elvétetett életben maradásunk felõl.
\par 21 Mikor pedig hosszas volt már az étlenség, akkor Pál felállván õ közöttük, monda: Jóllehet szükséges lett volna, óh férfiak, hogy engedelmeskedve nékem, ne indultunk volna el Krétából, és elkerültük volna ezt a bajt és kárt:
\par 22 Mindazáltal mostanra nézve is intelek benneteket, hogy jó reménységben legyetek; mert egy lélek sem vész el közületek, hanem csak a hajó.
\par 23 Mert ez éjjel mellém álla egy angyala az Istennek, a kié vagyok, a kinek szolgálok is,
\par 24 Ezt mondván: Ne félj Pál! A császár elé kell néked állnod. És ímé az Isten ajándékba adta néked mindazokat, kik te veled hajóznak.
\par 25 Annakokáért jó reménységben legyetek, férfiak! Mert hiszek az Istnnek, hogy úgy lesz, a mint nékem megmondatott.
\par 26 Egy szigetre kell pedig nékünk kivetõdnünk.
\par 27 Mikor pedig a tizennegyedik éjszaka eljött, a mint ide s tova hányatánk az Ádrián, éjféltájban észrevevék a hajósok hogy valami szárazföld közelget hozzájok.
\par 28 És lebocsátván a vízmérõ ónt, húsz ölnyinek találák, majd egy kevéssé tovább menvén és ismét lebocsátván a vízmérõ ónt, találák tizenöt ölnyinek.
\par 29 És mivel féltek, hogy szirtes helyekre vetõdhetnek, a hajónak hátulsó részébõl négy vasmacskát vetvén ki, kívánják vala, hogy nappal legyen.
\par 30 A hajósok pedig mikor el akarának menekülni a hajóból, és a csolnakot lebocsáták a tengerre, annak színe alatt, mintha a hajó orrából vasmacskákat akarnának vetni,
\par 31 Monda Pál a századosnak és a vitézeknek: Ha ezek a hajóban nem maradnak, ti meg nem szabadulhattok.
\par 32 Akkor a vitézek elvágták a csolnak köteleit, és ki hagyák esni azt.
\par 33 Addig pedig, míg nappal lenne, inti vala Pál mindnyájokat, hogy egyenek, mondván: Ma tizennegyedik napja, mióta folyton étlen várakoztok, semmit sem véve magatokhoz.
\par 34 Azért intelek benneteket, hogy egyetek, mert ez a ti javatokra szolgál. Mert küzületek senkinek sem esik le egy hajszál a fejérõl.
\par 35 Mikor pedig ezeket mondá, és kenyeret võn kezébe, hálákat ada Istennek mindnyájok elõtt, és megtörvén, kezde enni.
\par 36 Felbátorodván pedig mindnyájan, szintén vevének magukhoz táplálékot.
\par 37 Valánk pedig a hajóban lélekszám szerint összesen kétszázhetvenhatan.
\par 38 Miután pedig megelégedtek eledellel, a hajót könnyebbítik vala, a gabonát kihányván a tengerbe.
\par 39 Mikor pedig megvirradt, a szárazföldet nem ismerik vala fel; hanem egy tengeröblöt sajdítanak vala, melynek síma partja van, melyre végezék, hogy kihajtják a hajót, ha bírják.
\par 40 A vasmacskákat azért körös-körül elvagdalván, a tengerben hagyák, egyszersmind eloldván a kormányrudak köteleit és felvonván a nagy vitorlát a szélfúvásnak, igyekeznek vala a part felé haladni.
\par 41 De mikor egy zátonyos helyre találtak, ráhajtották a hajót. És az elsõ része ugyan megakadván, mozdíthatatlanul marad vala, a hátulsó része azonban szakadoz vala a haboknak ereje miatt.
\par 42 A vitézeknek pedig az lõn tanácsa, hogy a foglyokat vágják le, hogy senki el ne szaladhasson, minekutána kiúszott.
\par 43 De a százados meg akarván tartani Pált, eltiltá õket e szándéktól, és megparancsolá, hogy a kik úszni tudnak, elõször azok szökdössenek a tengerbe és meneküljenek ki a szárazföldre.
\par 44 A többiek pedig ki deszkákon, ki a hajó egyéb darabjain. És így lõn, hogy mindnyájan szerencsésen kimenekültek a szárazföldre.

\chapter{28}

\par 1 És miután szerencsésen megmenekültek, akkor megtudták, hogy Melitának neveztetik az a sziget.
\par 2 A barbárok pedig nem közönséges emberséget cselekesznek vala mi velünk: mert tüzet gerjesztvén, befogadának mindnyájónkat a rajtunk való záporért és a hidegért.
\par 3 Mikor pedig Pál nagy sok venyigét szedett és a tûzre tette, egy vipera a melegbõl kimászva, az õ kezére ragada.
\par 4 Mikor pedig látták a barbárok az õ kezérõl függeni a mérges kígyót, mondják vala egymásnak: Nyilván gyilkos ez az ember, kit nem hagya élni a bosszúállás, noha a tengerbõl megszabadult.
\par 5 De néki, minekutána a kígyót lerázta a tûzbe, semmi baja sem lõn.
\par 6 Azok pedig azt várják vala, hogy õ meg fog dagadni, vagy nagyhirtelenséggel halva rogyik le. Mikor azonban sok ideig várták, és látták, hogy semmi baja nem lesz, megváltoztatva értelmöket, istennek mondják vala õt.
\par 7 Annak a helynek környékén valának pedig a sziget fõemberének, névszerint Publiusnak mezei jószágai, ki befogadván minket, három napig nagy emberségesen vendégül látott.
\par 8 Lõn pedig, hogy a Publius atyja hideglelésben és vérhasban betegen feküvék. Kihez Pál beméne, és minekutána  könyörgött, kezeit reá vetve meggyógyítá õt.
\par 9 Minekutána azért ez megtörtént, egyebek is, kik betegek valának a szigeten, õ hozzá jövének és meggyógyulának.
\par 10 Kik nékünk nagy tisztességet is tõnek, és mikor elindulánk, a szükséges dolgokkal ellátának.
\par 11 Három hónap mulva tudniillik egy Alexandriába való hajón elindulánk, mely a szigeten telelt, melynek czímere Kásztor és Pollux vala.
\par 12 És Szirakúzába eljutván, ott maradánk három napig.
\par 13 Onnét körülkerülvén, eljutánk Régiumba, és egy nap mulva déli szél támadván, másnap megérkezénk Puteóliba.
\par 14 Hol mikor atyafiakat találtunk, kérének minket, hogy nálok maradjunk hét napig; és úgy menénk Rómába.
\par 15 Onnét is az atyafiak, mikor a mi dolgainkat meghallották, nékünk elõnkbe jövének Appii Forumig és Tres Tabernaeig. És mikor Pál meglátta õket, hálákat adván az Istennek, bátorságot võn.
\par 16 Mikor pedig Rómába jutottunk, a százados átadá a foglyokat a testõrsereg fõvezérének. Pálnak azonban megengedteték, hogy külön lakjék az õt õrizõ vitézzel.
\par 17 Lõn pedig, hogy három nap mulva magához hívatá Pál a zsidók között való fõembereket. Mikor pedig egybegyûltek, monda nékik: Atyámfiai,férfiak, én jóllehet semmit sem vétkeztem a nép ellen, vagy az õsi szokások ellen, mindazáltal foglyul adattam át Jeruzsálembõl a rómaiak kezébe.
\par 18 Kik miután kihallgattak, el akarának engem bocsátani, mivelhogy én bennem semmi halálra méltó vétek nincsen.
\par 19 De mivel a zsidók ellene mondtak, kényszeríttettem a császárra appellálni, nem mintha az én népem ellen volna valami vádam.
\par 20 Ennekokáért hívattalak tehát titeket, hogy lássalak benneteket és szóljak veletek; mert az Izráelnek reménységéért vétettem körül e lánczczal.
\par 21 Azok pedig mondának néki: Mi te felõled sem levelet nem vettünk Júdeából, sem pedig az atyafiak közül ise jõve valaki, nem jelentett, vagy szólott te felõled valami rosszat.
\par 22 Akarnók azért tõled hallani, micsoda értelemben vagy. Mert e felekezet felõl tudva van elõttünk, hogy mindenütt  ellene mondanak.
\par 23 Kitûzvén tehát néki egy napot, eljövének hozzá a szállására többen; kiknek nagy bizonyságtétellel szól vala az Istennek országa felõl, igyekezvén elhitetni õ velök a Jézus felõl való dolgokat, úgy a Mózes törvényébõl, mint a prófétákból, reggeltõl fogva mind estvéig.
\par 24 És némelyek hivének az õ beszédének, mások nem hivének.
\par 25 Mivel pedig nem egyezének meg egymással, eloszlának, miután Pál ez egy szót mondá: Jól szólott a Szent Lélek Ésaiás próféta által a mi atyáinknak, mondván:
\par 26 Eredj el a néphez és mondd: Hallván halljátok, és ne értsetek; és nézvén nézzetek, és ne lássatok!
\par 27 Mert megkövéredett e népnek szíve, és füleikkel nehezen hallanak, és szemeiket behunyják; hogy szemeikkel ne lássanak, füleikkel ne halljanak, szívükkel ne értsenek és meg ne térjenek, és meg ne gyógyítsam õket.
\par 28 Legyen azért néktek tudtotokra, hogy a pogány népeknek küldetett az Istennek ez idvezítése, és õk meg is hallgatják.
\par 29 És mikor ezeket mondotta, elmenének a zsidók, magok között sokat vetekedve.
\par 30 Marada pedig Pál két egész esztendeig az õ tulajdon bérelt szállásán, és mindazokat befogadja vala, ki õ hozzá menének.
\par 31 Prédikálván az Istennek országát és tanítván az Úr Jézus Krisztus felõl való dolgokat teljes bátorsággal, minden tiltás nélkül.


\end{document}