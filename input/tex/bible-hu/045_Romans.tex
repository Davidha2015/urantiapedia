\begin{document}

\title{Romans}


\chapter{1}

\par 1 Pál, Jézus Krisztusnak szolgája, elhívott apostol, elválasztva Isten evangyéliomának hirdetésére.
\par 2 Melyet eleve megígért az õ prófétái által a szentírásokban,
\par 3 Az õ Fia felõl, a ki Dávid magvából lett test szerint,
\par 4 A ki megbizonyíttatott hatalmasan Isten Fiának a szentség Lelke szerint, a halálból való feltámadás által, a mi Urunk Jézus Krisztus felõl,
\par 5 A ki által vettük a kegyelmet és az apostolságot a hitben való engedelmességnek okáért, minden pogányok között, az õ nevéért;
\par 6 Kik között vagytok ti is, Jézus Krisztusnak hivatalosai:
\par 7 Mindeneknek, a kik Rómában vagytok, Isten szerelmeseinek, hivatalos szenteknek: Kegyelem néktek és békesség Istentõl, a mi Atyánktól és az Úr Jézus Krisztustól.
\par 8 Elõször hálát adok az én Istenemnek a Jézus Krisztus által mindnyájatokért, hogy a ti hiteteknek az egész világon híre van;
\par 9 Mert bizonyságom nékem az Isten, kinek lelkem szerint szolgálok az õ Fiának evangyéliomában, hogy szüntelen emlékezem felõletek,
\par 10 Imádkozásaimban mindenkor könyörögvén, vajha egyszer már jó szerencsés út adódnék nékem Istennek akaratából, hogy hozzátok mehessek;
\par 11 Mert kívánlak titeket látni, hogy valami lelki ajándékot közölhessek veletek a ti megerõsítésetekre,
\par 12 Azaz, hogy együtt felbuzduljunk ti nálatok egymás hite által, a tiétek meg az enyém által.
\par 13 Nem akarom pedig, hogy ne tudjátok atyámfiai, hogy én gyakran elvégeztem magamban, hogy elmegyek hozzátok (de mindez ideig megakadályoztattam), hogy köztetek is nyerjek valami lelki gyümölcsöt, mint a többi pogányok közt is.
\par 14 Mind a görögöknek, mind a barbároknak, mind a bölcseknek, mind a tudatlanoknak köteles vagyok.
\par 15 Azért a mi rajtam áll, kész vagyok néktek is, a kik Rómában vagytok, az evangyéliomot hirdetni.
\par 16 Mert nem szégyenlem a Krisztus evangyéliomát; mert Istennek hatalma az  minden hívõnek idvességére, zsidónak elõször meg görögnek.
\par 17 Mert az Istennek igazsága jelentetik ki abban hitbõl hitbe, miképen meg van írva: Az igaz ember pedig hitbõl él.
\par 18 Mert nyilván van az Istennek haragja mennybõl, az embereknek minden hitetlensége és hamissága ellen, kik az igazságot hamissággal  feltartóztatják.
\par 19 Mert a mi az Isten felõl tudható nyilván van õ bennök; mert az Isten megjelentette nékik:
\par 20 Mert a mi Istenben láthatatlan, tudniilik az õ örökké való hatalma és istensége, a világ teremtésétõl fogva az õ alkotásaiból megértetvén megláttatik; úgy, hogy õk menthetetlenek.
\par 21 Mert bár az Istent megismerték, mindazáltal nem mint Istent dicsõítették õt, sem néki hálákat nem adtak; hanem az õ okoskodásaikban hiábavalókká lettek, és az õ balgatag szívök megsötétedett.
\par 22 Magaokat bölcseknek vallván, balgatagokká lettek;
\par 23 És az örökkévaló Istennek dicsõségét felcserélték a mulandó embereknek és madaraknak és négylábú állatoknak és csúszó-mászó állatoknak képmásával.
\par 24 Annakokáért adta is õket az Isten szívök kivánságaiban tisztátalanságra, hogy egymás testét megszeplõsítsék;
\par 25 Mint a kik az Isten igazságát hazugsággá változtatták, és a teremtett dolgokat tisztelték és szolgálták a teremtõ helyett, a ki mind örökké áldott. Ámen.
\par 26 Annakokáért adta õket az Isten tisztátalan indulatokra; mert az õ asszonynépeik is elváltoztatták a természet folyását természetellenesre:
\par 27 Hasonlóképen a férfiak is elhagyván az asszonynéppel való természetes élést, egymásra gerjedtek bujaságukban, férfiak férfiakkal fertelmeskedvén, és az õ tévelygésöknek méltó jutalmát elvevén önmagokban.
\par 28 És a miképen nem méltatták az Istent arra, hogy ismeretökben megtartsák, azonképen oda adták õket az Isten méltatlan gondolkozásra, hogy illetlen dolgokat cselekedjenek;
\par 29 A kik teljesek minden hamissággal, paráznasággal, gonoszsággal, kapzsisággal, rosszasággal; rakvák, írigységgel, gyilkossággal, versengéssel, álnoksággal, rossz erkölcscsel;
\par 30 Súsárlók, rágalmazók, istengyûlölõk, dölyfösek, kevélyek, dicsekedõk, rosszban mesterkedõk, szüleiknek engedetlenek,
\par 31 Balgatagok, összeférhetetlenek, szeretet nélkül valók, engesztelhetetlenek, irgalmatlanok.
\par 32 Kik jóllehet az Isten végzését ismerik, hogy a kik ilyeneket cselekesznek, méltók a halálra, mégis nemcsak cselekszik azokat, hanem az akképen cselekvõkkel egyet is értenek.

\chapter{2}

\par 1 Annakokáért menthetetlen vagy óh ember, bárki légy, a ki ítélsz: mert a miben mást megítélsz, önmagadat kárhoztatod; mivel ugyanazokat míveled te, a ki ítélsz.
\par 2 Tudjuk pedig, hogy az Istennek ítélete igazság szerint van azokon, a kik ilyeneket cselekesznek.
\par 3 Vagy azt gondolod, óh ember, a ki megítéled azokat, a kik ilyeneket cselekesznek, és te is azokat cselekszed, hogy te elkerülöd az Istennek ítéletét?
\par 4 Avagy megveted az õ jóságának, elnézésének és hosszútûrésének gazdagságát, nem tudván, hogy az Istennek jósága téged megtérésre indít?
\par 5 De a te keménységed és meg nem tért szíved szerint gyûjtesz magadnak haragot a haragnak és az Isten igaz ítélete kijelentésének napjára.
\par 6 A ki megfizet mindenkinek az õ cselekedetei szerint:
\par 7 Azoknak, a kik a jó cselekedetben való állhatatossággal dicsõséget, tisztességet és halhatatlanságot keresnek, örök élettel;
\par 8 Azoknak pedig, a kik versengõk és a kik nem engednek az igazságnak, hanem engednek a hamisságnak, búsulással és haraggal.
\par 9 Nyomorúság és ínség minden gonoszt cselekedõ ember lelkének, zsidónak elõször meg görögnek;
\par 10 Dicsõség pedig, tisztesség és békesség minden jót cselekedõnek, zsidónak elõször meg görögnek:
\par 11 Mert nincsen Isten elõtt személyválogatás.
\par 12 Mert a kik törvény nélkül vétkeztek, a törvény nélkül vesznek is el: és a kik a törvény alatt vétkeztek, törvény által ítéltetnek meg,
\par 13 (Mert nem azok igazak Isten elõtt, a kik a törvényt hallgatják, hanem azok fognak megigazulni, a kik a törvényt betöltik.
\par 14 Mert mikor a pogányok, a kiknek törvényök nincsen, természettõl a törvény dolgait cselekszik, akkor õk, törvényök nem lévén önmagoknak törvényök:
\par 15 Mint a kik megmutatják, hogy a törvény cselekedete be van írva az õ szívökbe, egyetemben bizonyságot tévén arról az õ lelkiismeretök és gondolataik, a melyek egymást kölcsönösen vádolják vagy mentegetik.)
\par 16 Azon a napon, melyen az Isten megítéli az emberek titkait az én evangyéliomom szerint a Jézus Krisztus által.
\par 17 Ímé, te zsidónak neveztetel, és a törvényre támaszkodol, és Istennel dicsekedel.
\par 18 És ismered az õ akaratát, és választást tudsz tenni azok között, a melyek különböznek attól,  mivelhogy a törvénybõl megtaníttattál;
\par 19 És azt hiszed magadról, hogy te a vakoknak vezetõje, a sötétségben levõknek világossága,
\par 20 A balgatagok tanítója, a kiskorúak mestere vagy, bírván a törvényben az ismeret és igazság formáját.
\par 21 A ki azért mást tanítasz, magadat nem tanítod-é? a ki azt hirdeted, hogy ne lopj, lopsz-é?
\par 22 A ki azt mondod, ne paráználkodjál, paráználkodol-é? a ki útálod a bálványokat, szentségtörõ vagy-é?
\par 23 Ki a törvényben dicsekszel, a törvénynek megrontása által az Istent gyalázod-é?
\par 24 Mert az Istennek neve miattatok káromoltatik a pogányok között, a mint meg van írva.
\par 25 Mert használ ugyan a körülmetélkedés, ha a törvényt megtartod; de ha a törvényt áthágod, a te körülmetélkedésed körülmetéletlenséggé lett.
\par 26 Ha tehát a körülmetéletlen pogány megtartja a törvénynek parancsolatait, az õ körülmetéletlensége nem körülmetélkedésül tulajdoníttatik-é néki?
\par 27 És a természettõl fogva körülmetéletlen ember, ha a törvényt megtartja, megítél téged, a ki a betû és körülmetélkedés mellett is a törvényeknek megrontója vagy.
\par 28 Mert nem az a zsidó, a ki külsõképen az; sem nem az a körülmetélés, a mi a testen külsõképen van:
\par 29 Hanem az a zsidó, a ki belsõképen az; és a szívnek lélekben, nem betû szerint való körülmetélése az igazi körülmetélkedés; a melynek dícsérete nem emberektõl, hanem Istentõl van.

\chapter{3}

\par 1 Mi tekintetben különb hát a zsidó? vagy micsoda haszna van a körülmetélkedésnek?
\par 2 Minden tekintetben sok. Mindenek elõtt, hogy az Isten reájok bízta az õ beszédeit.
\par 3 De hát hogy ha némelyek nem hittek? Vajjon azoknak hitetlensége nem teszi-é hiábavalóvá az Istennek hûségét?
\par 4 Távol legyen. Sõt inkább az Isten legyen igaz, minden ember pedig hazug, a mint meg van írva: Hogy igaznak ítéltessél a te beszédeidben, és  gyõzedelmes légy, mikor vádolnak téged.
\par 5 Ha pedig a mi igazságtalanságunk az Istennek igazságát mutatja meg, mit mondjunk? Vajjon igazságtalan-é az Isten, hogy minket büntet? Emberi módon szólok.
\par 6 Távol legyen! Mert akkor mi módon ítéli meg az Isten a világot?
\par 7 Mert ha az Istennek igazsága az én hazugságom által öregbült az õ dicsõségére, miért kárhoztattatom még én is, mint bûnös?
\par 8 Sõt inkább ne cselekedjük-é a rosszat, hogy abból jó származzék? - a mint minket rágalmaznak, és a mint némelyek mondogatják, hogy mi így beszélünk, a kiknek kárhoztatása igazságos.
\par 9 Micsoda tehát? Különbek vagyunk-é? Semmiképen nem. Mert az elébb megmutattuk nyilván, hogy zsidók és görögök mindnyájan bûn alatt vannak;
\par 10 A mint meg van írva, hogy nincsen csak egy igaz is;
\par 11 Nincs, a ki megértse, nincs, a ki keresse az Istent.
\par 12 Mindnyájan elhajlottak, egyetemben haszontalanokká lettek; nincs, a ki jót cselekedjék, nincsen csak egy is.
\par 13 Nyitott sír az õ torkuk; nyelvökkel álnokságot szólnak; áspis kígyó mérge van ajkaik alatt.
\par 14 Szájok telve átkozódással és keserûséggel.
\par 15 Lábaik gyorsak a vérontásra.
\par 16 Útjaikon romlás és nyomorúság van.
\par 17 És a békességnek útját nem ismerik.
\par 18 Nincs isteni félelem az õ szemök elõtt.
\par 19 Tudjuk pedig, hogy a mit a törvény mond, azoknak mondja, a kik a törvény alatt vannak; hogy minden száj bedugassék, és az egész világ Isten ítélete alá essék.
\par 20 Annakokáért a törvénynek cselekedeteibõl egy test sem igazul meg õ elõtte: mert a bûn ismerete  a törvény által vagyon.
\par 21 Most pedig törvény nélkül jelent meg az Istennek igazsága, a melyrõl tanúbizonyságot tesznek a törvény és a próféták;
\par 22 Istennek igazsága pedig a Jézus Krisztusban való hit által mindazokhoz és mindazoknak, a kik hisznek. Mert nincs különbség,
\par 23 Mert mindnyájan vétkeztek, és szûkölködnek az Isten dicsõsége nélkül.
\par 24 Megigazulván ingyen az õ kegyelmébõl a Krisztus Jézusban való váltság által,
\par 25 Kit az Isten eleve rendelt engesztelõ áldozatul, hit által, az õ vérében, hogy megmutassa az õ igazságát az elõbb elkövetett bûnöknek elnézése miatt,
\par 26 Az Isten hosszútûrésénél fogva, az õ igazságának megbizonyítására, a mostani idõben, hogy igaz legyen Õ és megigazítsa azt, a ki a Jézus hitébõl való.
\par 27 Hol van tehát a dicsekedés? Kirekesztetett. Mely törvény által? A cselekedeteké által? Nem; hanem a hit törvénye által.
\par 28 Azt tartjuk tehát, hogy az ember hit által igazul meg, a törvény cselekedetei nélkül.
\par 29 Avagy Isten csak a zsidóké-e? Avagy nem a pogányoké is? Bizony a pogányoké is.
\par 30 Mivelhogy egy az Isten, a ki megtisztítja a zsidót hitbõl és a pogányt hit által.
\par 31 A törvényt tehát hiábavalóvá tesszük-é a hit által? Távol legyen! Sõt inkább a törvényt megerõsítjük.

\chapter{4}

\par 1 Mit mondunk tehát, hogy Ábrahám a mi atyánk nyert volna test szerint?
\par 2 Mert ha Ábrahám cselekedetekbõl igazult meg, van mivel dicsekedjék, de nem az Isten elõtt.
\par 3 Mert mit mond az írás: Hitt pedig Ábrahám az Istennek, és tulajdoníttaték az õ néki igazságul.
\par 4 Annak pedig, a ki munkálkodik, a jutalom nem tulajdoníttatik kegyelembõl, hanem tartozás szerint;
\par 5 Ellenben annak, a ki nem munkálkodik, hanem hisz abban, a ki az istentelent megigazítja, az õ hite tulajdoníttatik igazságul.
\par 6 A mint Dávid is boldognak mondja azt az embert, a kinek az Isten igazságot tulajdonít cselekedetek nélkül.
\par 7 Boldogok, a kiknek megbocsáttattak az õ hamisságaik, és a kikenek elfedeztettek az õ bûneik.
\par 8 Boldog ember az, a kinek az Úr bûnt nem tulajdonít.
\par 9 Ez a boldogság tehát a zsidónak, vagy a pogánynak is tulajdoníttatik-é? Mert azt mondjuk, hogy Ábrahámnak a hit tulajdoníttaték igazságul.
\par 10 Miképen tulajdoníttaték tehát? Körülmetélt vagy körülmetéletlen állapotában? Nem körülmetélt, hanem körülmetéletlen állapotában.
\par 11 És a körülmetélkedés jegyét körülmetéletlenségben tanusított hite igazságának pecsétjéül nyerte: hogy atyja legyen mindazoknak, a kik körülmetéletlen létökre  hisznek, hogy azoknak is tulajdoníttassék az igazság;
\par 12 És hogy atyja legyen a körülmetélteknek is, azoknak, a kik nemcsak körülmetélkednek, hanem követik is a mi atyánknak Ábrahámnak körülmetéletlenségében tanusított hitének nyomdokait.
\par 13 Mert nem a törvény által adatott az ígéret Ábrahámnak, vagy az õ magvának, hogy e világnak örököse lesz, hanem a hitnek igazsága által.
\par 14 Mert ha azok az örökösök, kik a törvénybõl valók, hiábavalóvá lett a hit, és haszontalanná az ígéret:
\par 15 Mert a törvény haragot nemz: a hol pedig nincsen törvény, ott törvény ellen való cselekedet sincsen.
\par 16 Azért hitbõl, hogy kegyelembõl legyen; hogy erõs legyen az ígéret az egész magnak; nemcsak a törvénybõl valónak, hanem az Ábrahám hitébõl valónak is, a ki mindnyájunknak atyánk
\par 17 (A mint meg van írva, hogy sok nép atyjává tettelek téged) az elõtt, az Isten elõtt, a kiben hitt, a ki a holtakat megeleveníti, és azokat, a melyek nincsenek, elõszólítja mint meglevõket.
\par 18 A ki reménység ellenére reménykedve hitte, hogy sok népnek atyjává lesz, a szerint, a mint megmondatott: Így lészen a te magod.
\par 19 És hitében erõs lévén, nem gondolt az õ már elhalt testére, mintegy százesztendõs lévén, sem Sárának elhalt méhére;
\par 20 Az Istennek ígéretében sem kételkedett hitetlenséggel, hanem erõs volt a hitben, dicsõséget adván az Istennek,
\par 21 És teljesen elhitte, hogy a mit õ ígért, meg is cselekedheti.
\par 22 Azért is tulajdoníttaték néki igazságul.
\par 23 De nemcsak õ érette iratott meg, hogy tulajdoníttaték néki igazságul,
\par 24 Hanem mi érettünk is, a kiknek majd tulajdoníttatik, azoknak tudniillik, a kik hisznek Abban, a ki feltámasztotta a mi Urunkat a Jézust a halálból,
\par 25 Ki a mi bûneinkért halálra adatott, és feltámasztatott  a mi megigazulásunkért.

\chapter{5}

\par 1 Megigazulván azért hit által, békességünk van Istennel, a mi Urunk Jézus Krisztus által,
\par 2 A ki által van a menetelünk is hitben ahhoz a kegyelemhez, a melyben állunk; és dicsekedünk az Isten dicsõségének reménységében.
\par 3 Nemcsak pedig, hanem dicsekedünk a háborúságokban is, tudván, hogy a háborúság békességes tûrést nemz,
\par 4 A békességes tûrés pedig próbatételt, a próbatétel pedig reménységet,
\par 5 A reménység pedig nem szégyenít meg; mert az Istennek szerelme kitöltetett a mi szívünkbe a Szent Lélek által, ki adatott nékünk.
\par 6 Mert Krisztus, mikor még erõtelenek valánk, a maga idejében meghalt a gonoszokért.
\par 7 Bizonyára igazért is alig hal meg valaki; ám a jóért talán csak meg merne halni valaki.
\par 8 Az Isten pedig a mi hozzánk való szerelmét abban mutatta meg, hogy mikor még bûnösök voltunk, Krisztus érettünk meghalt.
\par 9 Minekutána azért most megigazultunk az õ vére által, sokkal inkább megtartatunk a harag ellen õ általa.
\par 10 Mert ha, mikor ellenségei voltunk, megbékéltünk Istennel az õ Fiának halála által, sokkal inkább megtartatunk az õ élete által minekutána megbékéltünk vele.
\par 11 Nemcsak pedig, hanem dicsekedünk is az Istenben a mi Urunk Jézus Krisztus által, a ki által most a megbékélést nyertük.
\par 12 Annakokáért, miképen egy ember által jött be a világra a bûn és a bûn által a halál, és akképen a halál minden emberre elhatott, mivelhogy mindenek vétkeztek;
\par 13 Mert a törvényig vala bûn a világon; a bûn azonban nem számíttatik be, ha nincsen törvény.
\par 14 Úgyde a halál uralkodott Ádámtól Mózesig azokon is, a kik nem az Ádám esetének hasonlatossága szerint vétkeztek, a ki ama következendõnek kiábrázolása vala.
\par 15 De a kegyelmi ajándék nem úgy van, mint a bûneset; mert ha amaz egynek esete miatt sokan haltak meg, az Isten kegyelme és a kegyelembõl való ajándék, mely az egy ember Jézus Krisztusé, sokkal inkább elhatott sokakra.
\par 16 És az ajándék sem úgy van, mint egy vétkezõ által; mert az ítélet egybõl lett kárhozottá, az ajándék pedig sok bûnbõl van igazulásra.
\par 17 Mert ha az egynek bûnesete miatt uralkodott a halál az egy által: sokkal inkább az életben uralkodnak az egy Jézus Krisztus által azok, kik a kegyelemnek és az igazság ajándékának bõvölködésében részesültek.
\par 18 Bizonyára azért, miképen egynek bûnesete által minden emberre elhatott a kárhozat: azonképen egynek igazsága által minden emberre elhatott az életnek megigazulása.
\par 19 Mert miképen egy embernek engedetlensége által sokan bûnösökké lettek: azonképen egynek engedelmessége által sokan igazakká lesznek.
\par 20 A törvény pedig bejött, hogy a bûn megnövekedjék; de a hol megnövekedik a bûn, ott a kegyelem sokkal inkább bõvölködik:
\par 21 Hogy miképen uralkodott a bûn a halálra, azonképen a kegyelem is uralkodjék igazság által az örök életre a mi Urunk Jézus Krisztus által.

\chapter{6}

\par 1 Mit mondunk tehát? Megmaradjunk-é a bûnben, hogy a kegyelem annál nagyobb legyen?
\par 2 Távol legyen: a kik meghaltunk a bûnnek, mimódon élnénk még abban?
\par 3 Avagy nem tudjátok-é, hogy a kik megkeresztelkedtünk Krisztus Jézusba, az õ halálába keresztelkedtünk meg?
\par 4 Eltemettettünk azért õ vele együtt a keresztség által a halálba: hogy miképen feltámasztatott Krisztus a halálból az Atyának dicsõsége által, azonképen mi is új életben  járjunk.
\par 5 Mert ha az õ halálának hasonlatossága szerint vele egygyé lettünk, bizonyára feltámadásáé szerint is azok leszünk.
\par 6 Tudván azt, hogy a mi ó emberünk õ vele megfeszíttetett, hogy megerõtelenüljön a bûnnek teste, hogy ezután ne szolgáljunk a bûnnek:
\par 7 Mert a ki meghalt, felszabadult a bûn alól.
\par 8 Hogyha pedig meghaltunk Krisztussal, hiszszük, hogy élünk is õ vele.
\par 9 Tudván, hogy Krisztus, a ki feltámadott a halálból, többé meg nem hal; a halál többé rajta nem uralkodik,
\par 10 Mert hogy meghalt, a bûnnek halt meg egyszer; hogy pedig él, az Istennek él.
\par 11 Ezenképen gondoljátok ti is, hogy meghaltatok a bûnnek, de éltek az Istennek a mi Urunk Jézus Krisztusban.
\par 12 Ne uralkodjék tehát a bûn a ti halandó testetekben, hogy engedjetek néki az õ kívánságaiban:
\par 13 Se ne szánjátok oda a ti tagjaitokat hamisságnak fegyvereiül a bûnnek; hanem szánjátok oda magatokat az Istennek, mint a kik a halálból életre keltetek, és a ti tagjaitokat igazságnak fegyvereiül az Istennek.
\par 14 Mert a bûn ti rajtatok nem uralkodik; mert nem vagytok törvény alatt, hanem kegyelem alatt.
\par 15 Mit is tehát? Vétkezzünk-é mivelhogy nem vagyunk törvény alatt, hanem kegyelem alatt? Távol legyen.
\par 16 Avagy nem tudjátok, hogy a kinek oda szánjátok magatokat szolgákul az engedelmességre, annak vagytok szolgái, a kinek engedelmeskedtek: vagy a bûnnek halálra, vagy az engedelmességnek igazságra?
\par 17 De hála az Istennek, hogy jóllehet a bûn szolgái voltatok, de szívetek szerint engedelmeskedtek a tudomány azon alakjának, a melyre adattatok.
\par 18 Felszabadulván pedig a bûn alól, az igazságnak szolgáivá lettetek.
\par 19 Emberi módon szólok a ti testeteknek erõtlensége miatt. Mert a miképen oda szántátok a ti tagjaitokat a tisztátalanságnak és a hamisságnak szolgáiul a hamisságra: azonképen szánjátok oda most a ti tagjaitokat szolgáiul az igazságnak a megszenteltetésre.
\par 20 Mert mikor a bûn szolgái valátok, az igazságtól szabadok valátok.
\par 21 Micsoda gyümölcsét vettétek azért akkor azoknak, a miket most szégyenletek? mert azoknak a vége halál.
\par 22 Most pedig, minekutána felszabadultatok a bûn alól, szolgáivá lettetek Pedig az Istennek: megvan a gyümölcsötök a megszenteltetésre, a vége pedig örök élet.
\par 23 Mert a bûn zsoldja halál; az Isten kegyelmi ajándéka pedig örök élet a mi Urunk Krisztus Jézusban.

\chapter{7}

\par 1 Avagy nem tudjátok-é atyámfiai, mert törvényismerõkhöz szólok, hogy a törvény uralkodik az emberen, a míg él?
\par 2 Mert a férjes asszony, míg él a férj, ehhez van kötve törvény szerint, de ha meghal a férj, felszabadul az asszony a férj törvénye alól.
\par 3 Azért tehát az õ férjének életében paráznának mondatik, ha más férfihoz megy; ha azonban meghal a férje, szabaddá lesz a törvénytõl, úgy hogy nem lesz parázna, ha más férfihoz megy.
\par 4 Azért atyámfiai, ti is meghaltatok a törvénynek a Krisztus teste által, hogy legyetek máséi, azéi, a ki a halálból feltámasztatott, hogy gyümölcsöt teremjünk Istennek.
\par 5 Mert mikor a testben voltunk, a bûnök indulatai a törvény által dolgoztak a mi tagjainkban, hogy gyümölcsözzenek a halálnak;
\par 6 Most pedig megszabadultunk a törvénytõl, minekutána meghaltunk arra nézve, a mely által lekötve tartattunk; hogy szolgáljunk a léleknek újságában és nem a betû óságában.
\par 7 Mit mondunk tehát? A törvény bûn-é? Távol legyen: sõt inkább a bûnt nem ismertem, hanem csak a törvény által; mert a kívánságról sem tudtam volna, ha a törvény nem mondaná: Ne kívánjad.
\par 8 De a bûn alkalmat vévén, a parancsolat által nemzett bennem minden kívánságot; mert törvény nélkül holt a bûn.
\par 9 Én pedig éltem régen a törvény nélkül: de ama parancsolatnak eljövetelével felelevenedék a bûn,
\par 10 Én pedig meghalék; és úgy találtaték, hogy az a parancsolat, mely életre való, nékem halálomra van.
\par 11 Mert a bûn alkalmat vévén, ama parancsolat által megcsalt engem, és megölt általa.
\par 12 Azért ám a törvény szent, és a parancsolat szent és igaz és jó.
\par 13 Tehát a jó nékem halálom lett-é? Távol legyen: sõt inkább a bûn az, hogy megtessék a bûn, mely a jó által nékem halált szerez, hogy felette igen bûnös legyen a bûn a parancsolat által.
\par 14 Mert tudjuk, hogy a törvény lelki; de én testi vagyok, a bûn alá rekesztve.
\par 15 Mert a mit cselekeszem, nem ismerem: mert nem azt mívelem, a mit akarok, hanem a mit gyûlölök, azt cselekeszem.
\par 16 Ha pedig azt cselekszem, a mit nem akarok, megegyezem a törvénnyel, hogy jó.
\par 17 Most azért már nem én cselekszem azt, hanem a bennem lakozó bûn.
\par 18 Mert tudom, hogy nem lakik én bennem, azaz a testemben jó; mert az akarás megvan bennem, de a jó véghezvitelét nem találom.
\par 19 Mert nem a jót cselekeszem, melyet akarok; hanem a gonoszt cselekeszem, melyet nem akarok.
\par 20 Ha pedig én azt cselekeszem, a mit nem akarok, nem én mívelem már azt, hanem a bennem lakozó bûn.
\par 21 Megtalálom azért magamban, ki a jót akarom cselekedni, ezt a törvényt, hogy a bûn megvan bennem.
\par 22 Mert gyönyörködöm az Isten törvényében a belsõ ember szerint;
\par 23 De látok egy másik törvényt az én tagjaimban, mely ellenkezik az elmém törvényével, és engem rabul ád a bûn törvényének, mely van az én tagjaimban.
\par 24 Óh én nyomorult ember! Kicsoda szabadít meg engem e halálnak testébõl?
\par 25 Hálát adok Istennek a mi Urunk Jézus Krisztus által. Azért jóllehet én az elmémmel az Isten törvényének, de testemmel a bûn törvényének szolgálok.

\chapter{8}

\par 1 Nincsen azért immár semmi kárhoztatásuk azoknak, a kik Krisztus Jézusban vannak, kik nem test szerint járnak, hanem  Lélek szerint.
\par 2 Mert a Jézus Krisztusban való élet lelkének törvénye megszabadított engem a bûn és a halál törvényétõl.
\par 3 Mert a mi a törvénynek lehetetlen vala, mivelhogy erõtlen vala a test miatt, az Isten az õ Fiát elbocsátván bûn testének hasonlatosságában és a bûnért, kárhoztatá a bûnt a testben.
\par 4 Hogy a törvénynek igazsága beteljesüljön bennünk, kik nem test szerint járunk, hanem Lélek szerint.
\par 5 Mert a test szerint valók a test dolgaira gondolnak; a Lélek szerint valók pedig a Lélek dolgaira.
\par 6 Mert a testnek gondolata halál; a Lélek gondolata pedig élet és békesség.
\par 7 Mert a test gondolata ellenségeskedés Isten ellen; minthogy az Isten törvényének nem engedelmeskedik, mert nem is teheti.
\par 8 A kik pedig testben vannak, nem lehetnek kedvesek Isten elõtt.
\par 9 De ti nem vagytok testben, hanem lélekben, ha ugyan az Isten Lelke lakik bennetek. A kiben pedig nincs a Krisztus Lelke, az nem az övé.
\par 10 Hogyha pedig Krisztus ti bennetek van, jóllehet a test holt a bûn miatt, a lélek ellenben élet az igazságért.
\par 11 De ha Annak a Lelke lakik bennetek, a ki feltámasztotta Jézust a halálból, ugyanaz, a ki feltámasztotta a Krisztus Jézust a halálból, megeleveníti a ti halandó testeiteket  is az õ ti bennetek lakozó Lelke által.
\par 12 Annakokáért atyámfiai, nem vagyunk adósok a testnek, hogy test szerint éljünk:
\par 13 Mert, ha test szerint éltek, meghaltok; de ha a test cselekedeteit a lélekkel megöldökölitek, éltek.
\par 14 Mert a kiket Isten Lelke vezérel, azok Istennek fiai.
\par 15 Mert nem kaptátok szolgaság lelkét ismét a félelemre, hanem a fiúságnak Lelkét kaptátok, a ki  által kiáltjuk: Abbá, Atyám!
\par 16 Ez a Lélek bizonyságot tesz a mi lelkünkkel együtt, hogy Isten gyermekei vagyunk.
\par 17 Ha pedig gyermekek, örökösök is; örökösei Istennek, örököstársai pedig Krisztusnak; ha ugyan vele együtt szenvedünk,  hogy vele együtt is dicsõüljünk meg.
\par 18 Mert azt tartom, hogy a miket most szenvedünk, nem hasonlíthatók ahhoz a dicsõséghez, mely nékünk megjelentetik.
\par 19 Mert a teremtett világ sóvárogva várja az Isten fiainak megjelenését.
\par 20 Mert a teremtett világ hiábavalóság alá vettetett, nem önként, hanem azért, a ki az alá vetette.
\par 21 Azzal a reménységgel, hogy maga a teremtett világ is megszabadul a rothadandóság rabságától az Isten fiai dicsõségének szabadságára.
\par 22 Mert tudjuk, hogy az egész teremtett világ egyetemben fohászkodik és nyög mind idáig.
\par 23 Nemcsak ez pedig, hanem magok a Lélek zsengéjének birtokosai, mi magunk is fohászkodunk magunkban, várván a fiúságot, a mi testünknek megváltását.
\par 24 Mert reménységben tartattunk meg; a reménység pedig, ha láttatik, nem reménység; mert a mit lát valaki, miért reményli is azt?
\par 25 Ha pedig, a mit nem látunk, azt reméljük, békességes tûréssel várjuk.
\par 26 Hasonlatosképen pedig a Lélek is segítségére van a mi erõtelenségünknek. Mert azt, a mit kérnünk kell, a mint kellene, nem tudjuk; de maga a Lélek esedezik mi érettünk kimondhatatlan fohászkodásokkal.
\par 27 A ki pedig a szíveket vizsgálja, tudja, mi a Lélek gondolata, mert Isten szerint esedezik a szentekért.
\par 28 Tudjuk pedig, hogy azoknak, a kik Istent szeretik, minden javokra van, mint a kik az õ végzése szerint hivatalosak.
\par 29 Mert a kiket eleve ismert, eleve el is rendelte, hogy azok az õ Fia ábrázatához hasonlatosak legyenek, hogy õ legyen elsõszülött  sok atyafi között.
\par 30 A kiket pedig eleve elrendelt, azokat el is hívta; és a kiket elhívott, azokat meg is igazította; a kiket pedig megigazított, azokat meg is dicsõítette.
\par 31 Mit mondunk azért ezekre? Ha az Isten velünk, kicsoda ellenünk?
\par 32 A ki az õ tulajdon Fiának nem kedvezett, hanem õt mindnyájunkért odaadta, mimódon ne ajándékozna vele együtt mindent minékünk?
\par 33 Kicsoda vádolja az Isten választottait? Isten az, a ki megigazít;
\par 34 Kicsoda az, a ki kárhoztat? Krisztus az, a ki meghalt, sõt a ki fel is támadott, a ki az Isten jobbján van, a ki esedezik is  érettünk:
\par 35 Kicsoda szakaszt el minket a Krisztus szerelmétõl? nyomorúság vagy szorongattatás, vagy üldözés, vagy éhség, vagy meztelenség, vagy veszedelem, vagy fegyver-é?
\par 36 A mint megvan írva, hogy: Te éretted gyilkoltatunk minden napon; olybá tekintenek mint vágó juhokat.
\par 37 De mindezekben felettébb diadalmaskodunk, Az által, a ki minket szeretett,
\par 38 Mert meg vagyok gyõzõdve, hogy sem halál, sem élet, sem angyalok, sem fejedelemségek, sem hatalmasságok, sem jelenvalók, sem következendõk,
\par 39 Sem magasság, sem mélység, sem semmi más teremtmény nem szakaszthat el minket az Istennek szerelmétõl, mely vagyon a mi Urunk Jézus Krisztusban.

\chapter{9}

\par 1 Igazságot szólok Krisztusban, nem hazudok, lelkiismeretem velem együtt tesz bizonyságot a Szent Lélek által,
\par 2 Hogy nagy az én szomorúságom és szüntelen való az én szívemnek fájdalma;
\par 3 Mert kívánnám, hogy én magam átok legyek, elszakasztva a Krisztustól az én atyámfiaiért, a kik rokonaim test szerint;
\par 4 A kik izráeliták, a kiké a fiúság és a dicsõség és a szövetségek, meg a  törvényadás és az isteni tisztelet és az ígéretek;
\par 5 A kiké az atyák, és a kik közül való test szerint a Krisztus, a ki mindeneknek felette örökké áldandó  Isten. Ámen.
\par 6 Nem lehet pedig, hogy meghiúsult legyen az Isten beszéde. Mert nem mindnyájan izráeliták azok, kik Izráeltõl valók;
\par 7 Sem nem mindnyájan fiak, kik az Ábrahám magvából valók; hanem: Izsákban neveztetik néked a te magod.
\par 8 Azaz, nem a testnek fiai az Isten fiai; hanem az ígéret fiait tekinti magul.
\par 9 Mert ígéretnek beszéde ez: Ez idõ tájban eljövök, és Sárának fia lesz.
\par 10 Nemcsak pedig, hanem Rebeka is, ki egytõl fogant méhében, Izsáktól a mi atyánktól:
\par 11 Mert mikor még meg sem születtek, sem semmi jót vagy gonoszt nem cselekedtek, hogy az Istennek kiválasztás szerint való végzése megmaradjon, nem cselekedetekbõl, hanem az elhívótól,
\par 12 Megmondatott néki, hogy: A nagyobbik szolgál a kisebbiknek.
\par 13 Miképen meg van írva: Jákóbot szerettem, Ézsaut pedig gyûlöltem.
\par 14 Mit mondunk tehát: Vajjon nem igazságtalanság-é ez az Istentõl? Távol legyen!
\par 15 Mert Mózesnek ezt mondja: Könyörülök azon, a kin könyörülök, és kegyelmezek annak, a kinek kegyelmezek.
\par 16 Annakokáért tehát nem azé, a ki akarja, sem nem azé, a ki fut, hanem a könyörülõ Istené.
\par 17 Mert azt mondja az írás a Faraónak, hogy: Azért támasztottalak téged, hogy megmutassam benned az én hatalmamat, és hogy hirdessék az én nevemet az egész földön.
\par 18 Annakokáért a kin akar könyörül, a kit pedig akar, megkeményít.
\par 19 Mondod azért nékem: Miért fedd hát engem? Hiszen az õ akaratának kicsoda áll ellene?
\par 20 Sõt inkább kicsoda vagy te óh ember, hogy versengsz az Istennel? Avagy mondja-é a készítmény a készítõnek: Miért csináltál engem így?
\par 21 Avagy nincsen-é a fazekasnak hatalma az agyagon, hogy ugyanazon gyuradékból némely edényt tisztességre, némelyt pedig becstelenségre csináljon?
\par 22 Ha pedig az Isten az õ haragját megmutatni és hatalmát megismertetni kívánván, nagy békességes tûréssel elszenvedte a harag edényeit, melyek veszedelemre készíttettek,
\par 23 És hogy megismertesse az õ dicsõségének gazdagságát az irgalom edényein, melyeket eleve elkészített a dicsõségre, mit szólhatsz ellene?
\par 24 A kikül el is hívott minket nemcsak a zsidók, hanem a pogányok közül is,
\par 25 A mint Hóseásnál is mondja: Hívom a nem én népemet én népemnek; és a nem szerettet szeretettnek.
\par 26 És lészen, hogy azon a helyen, a hol ez mondatott nékik: Ti nem vagytok az én népem, ott az élõ Isten fiainak fognak hívatni.
\par 27 Ésaiás pedig ezt kiáltja Izráel felõl: Ha Izráel fiainak száma annyi volna is, mint a tenger fövenye, a maradék tartatik meg.
\par 28 Mert a dolgot bevégezi és rövidre metszi igazságban; mivel rövidesen végez az Úr a földön.
\par 29 És a mint Ésaiás megmondotta: Ha a Seregeknek Ura nem hagyott volna nékünk magot, olyanokká lettünk volna, mint Sodoma, és Gomorához volnánk hasonlók.
\par 30 Mit mondunk hát? Azt, hogy a pogányok, a kik az igazságot nem követték, az igazságot elnyerték, még pedig a hitbõl való igazságot;
\par 31 Izráel ellenben, mely az igazság törvényét követte, nem jutott el az igazság törvényére.
\par 32 Miért? Azért, mert nem hitbõl keresték, hanem mintha a törvény cselekedeteibõl volna. Mert beleütköztek a beleütközés kövébe,
\par 33 A mint meg van írva: Ímé beleütközés kövét és megbotránkozás szikláját teszem Sionba; és a ki hisz benne, nem szégyenül meg.

\chapter{10}

\par 1 Atyámfiai, szívem szerint kívánom és Istentõl könyörgöm az Izráel idvességét.
\par 2 Mert bizonyságot teszek felõlök, hogy Isten iránt való buzgóság van bennök, de  nem megismerés szerint.
\par 3 Mert az Isten igazságát nem ismervén, és az õ tulajdon igazságukat igyekezvén érvényesíteni, az Isten igazságának nem engedelmeskedtek.
\par 4 Mert a törvény vége Krisztus minden hívõnek igazságára.
\par 5 Mert Mózes a törvénybõl való igazságról azt írja, hogy a ki azokat cselekeszi, él azok által.
\par 6 A hitbõl való igazság pedig így szól: Ne mondd a te szívedben: Kicsoda megy föl a mennybe? (azaz, hogy Krisztus aláhozza;)
\par 7 Avagy: Kicsoda száll le a mélységbe? (azaz, hogy Krisztust a halálból felhozza.)
\par 8 De mit mond? Közel hozzád a beszéd, a szádban és a szívedben van: azaz a hit beszéde, a melyet mi hirdetünk.
\par 9 Mert ha a te száddal vallást teszel az Úr Jézusról, és szívedben hiszed, hogy az Isten feltámasztotta õt a halálból, megtartatol.
\par 10 Mert szívvel hiszünk az igazságra, szájjal teszünk pedig vallást az idvességre.
\par 11 Mert azt mondja az írás: Valaki hisz õ benne, meg nem szégyenül.
\par 12 Mert nincs különbség zsidó meg görög között; mert ugyanaz az Ura mindeneknek, a ki kegyelemben gazdag mindenekhez, a kik õt segítségül hívják.
\par 13 Mert minden, a ki segítségül hívja az Úr nevét, megtartatik.
\par 14 Mimódon hívják azért segítségül azt, a kiben nem hisznek? Mimódon hisznek pedig abban, a ki felõl nem hallottak? Mimódon hallanának pedig prédikáló nélkül?
\par 15 Mimódon prédikálnak pedig, ha el nem küldetnek? A miképen meg van írva: Mely szépek a békesség hirdetõknek lábai, a kik jókat hirdetnek!
\par 16 De nem mindenek engedelmeskedtek az evangyéliomnak. Mert Ésaiás azt mondja: Uram! Kicsoda hitt a mi beszédünknek?
\par 17 Azért a hit hallásból van, a hallás pedig Isten ígéje által.
\par 18 De mondom: Avagy nem hallották-é? Sõt inkább az egész földre elhatott az õ hangjok, és a lakóföld véghatáráig az õ beszédök.
\par 19 De mondom: Avagy nem ismerte-é Izráel? Elõször Mózes mondja: Én titeket felingerellek egy nem néppel, értelmetlen néppel haragítalak meg titeket.
\par 20 Ésaiás pedig bátorságosan ezt mondja: Megtaláltak azok, a kik engem nem keresnek; nyilvánvaló lettem azoknak, a kik felõlem nem kérdezõsködtek.
\par 21 Az Izráelrõl pedig ezt mondja: Egész napon kiterjesztettem kezeimet az engedetlenkedõ és ellenmondó néphez.

\chapter{11}

\par 1 Mondom tehát: Avagy elvetette-é Isten az õ népét? Távol legyen; mert én is izráelita vagyok, az Ábrahám magvából, Benjámin nemzetségébõl való.
\par 2 Nem vetette el Isten az õ népét, melyet eleve ismert. Avagy nem tudjátok-é mit mond az írás Illésrõl? a mint könyörög Istenhez Izráel ellen, mondván:
\par 3 Uram, a te prófétáidat megölték, és a tel oltáraidat lerombolták; és csak én egyedül maradtam, és engem is halálra keresnek.
\par 4 De mit mond néki az isteni felelet? Meghagytam magamnak hétezer embert, a kik nem hajtottak térdet a Baálnak.
\par 5 Ekképen azért most is van maradék a kegyelembõl való választás szerint.
\par 6 Hogyha pedig kegyelembõl, akkor nem cselekedetekbõl: különben a kegyelem nem volna többé kegyelem. Hogyha pedig cselekedetekbõl, akkor nem kegyelembõl: különben a cselekedet nem volna többé cselekedet.
\par 7 Micsoda tehát? A mit Izráel keres, azt nem nyerte meg: a választottak ellenben megnyerték, a többiek pedig megkeményíttettek:
\par 8 A mint meg van írva: Az Isten kábultság lelkét adta nékik; szemeket, hogy ne lássanak, füleket, hogy ne halljanak, mind e mai napig.
\par 9 Dávid is ezt mondja: Legyen az õ asztaluk tõrré, hálóvá, botránkozássá és megtorlássá.
\par 10 Sötétüljenek meg az õ szemeik, hogy ne lássanak, és az õ hátokat mindenkorra görbítsd meg.
\par 11 Annakokáért mondom: Avagy azért botlottak-é meg, hogy elessenek? Távol legyen; hanem az õ esetük folytán lett az idvesség a pogányoké, hogy  õk felingereltessenek.
\par 12 Ha pedig az õ esetök világnak gazdagsága, és az õ veszteségök pogányok gazdagsága, mennyivel inkább az õ teljességök?
\par 13 Mert néktek mondom a pogányoknak, a mennyiben hát én pogányok apostola vagyok, a szolgálatomat dicsõítem,
\par 14 Ha ugyan felingerelhetném az én atyámfiait, és megtarthatnék közülök némelyeket.
\par 15 Mert ha az õ elvettetésök a világnak megbékélése, micsoda lesz a felvételök hanemha élet a halálból?
\par 16 Ha pedig a zsenge szent, akkor a tészta is; és ha a gyökér szent, az ágai is azok.
\par 17 Ha pedig némely ágak kitörettek, te pedig vadolajfa létedre beoltattál azok közé, és részese lettél az olajfa gyökerének és zsírjának;
\par 18 Ne kevélykedjél az ágak ellenében: ha pedig kevélykedel, nem te hordozod a gyökeret, hanem a gyökér téged.
\par 19 Azt mondod azért: Kitörettek az ágak, hogy én oltassam be.
\par 20 Úgy van; hitetlenség miatt törettek ki, te pedig hit által állasz; fel ne fuvalkodjál, hanem félj;
\par 21 Mert ha az Isten a természet szerint való ágaknak nem kedvezett, majd néked sem kedvez.
\par 22 Tekintsd meg azért az Istennek kegyességét és keménységét: azok iránt a kik elestek, keménységét; irántad pedig a kegyességét, ha megmaradsz a kegyességben; különben te is kivágatol.
\par 23 Sõt azok is, ha meg nem maradnak a hitetlenségben, beoltatnak; mert az Isten ismét beolthatja õket.
\par 24 Mert ha te a természet szerint való vadolajfából kivágattál, és természet ellenére beoltattál a szelid olajfába: mennyivel inkább beoltatnak ezek a természet szerint valók az õ saját olajfájokba.
\par 25 Mert nem akarom, hogy ne tudjátok atyámfiai ezt a titkot, hogy magatokat el ne higyjétek, hogy a megkeményedés Izráelre nézve csak részben  történt, a meddig a pogányok teljessége bemegyen.
\par 26 És így az egész Izráel megtartatik, a mint meg van írva: Eljõ Sionból a Szabadító, és elfordítja Jákóbtól a gonoszságokat:
\par 27 És ez nékik az én szövetségem, midõn eltörlöm az õ bûneiket.
\par 28 Az evangyéliomra nézve ugyan ellenségek ti érettetek; de a választásra nézve szerelmetesek az atyákért.
\par 29 Mert megbánhatatlanok az Istennek ajándékai és az õ elhívása.
\par 30 Mert miképen ti egykor engedetlenkedtetek az Istennek, most pedig irgalmasságot nyertetek az õ engedetlenségök miatt:
\par 31 Azonképen õk is most engedetlenkedtek, hogy a ti irgalmasságba jutásotok folytán õk is irgalmasságot nyerjenek;
\par 32 Mert az Isten mindeneket engedetlenség alá rekesztett, hogy mindeneken könyörüljön.
\par 33 Óh Isten gazdagságának, bölcseségének és tudományának mélysége! Mely igen kikutathatatlanok az õ ítéletei s kinyomozhatatlanok az õ útai!
\par 34 Mert kicsoda ismerte meg az Úr értelmét? vagy kicsoda volt néki tanácsosa?
\par 35 Avagy kicsoda adott elõbb néki, hogy annak visszafizesse azt?
\par 36 Mert õ tõle, õ általa és õ reá nézve vannak mindenek. Övé a dicsõség mindörökké. Ámen.

\chapter{12}

\par 1 Kérlek azért titeket atyámfiai az Istennek irgalmasságára, hogy szánjátok oda a ti testeiteket élõ, szent és Istennek kedves áldozatul, mint a ti okos tiszteleteteket.
\par 2 És ne szabjátok magatokat e világhoz, hanem változzatok el a ti elméteknek megújulása által, hogy megvizsgáljátok, mi az Istennek jó, kedves és tökéletes akarata.
\par 3 Mert a nékem adott kegyelem által mondom mindenkinek közöttetek, hogy feljebb ne bölcselkedjék, mint a hogy kell bölcselkedni; hanem józanon bölcselkedjék, a mint az Isten adta kinek-kinek a  hit mértékét.
\par 4 Mert miképen egy testben sok tagunk van, minden tagnak pedig nem ugyanazon cselekedete van:
\par 5 Azonképen sokan egy test vagyunk a Krisztusban, egyenként pedig egymásnak tagjai vagyunk.
\par 6 Minthogy azért külön-külön ajándékaink vannak a nékünk adott kegyelem szerint, akár írásmagyarázás, a hitnek szabálya szerint teljesítsük;
\par 7 Akár szolgálat, a szolgálatban; akár tanító, a tanításban;
\par 8 Akár intõ, az intésben; az adakozó szelídségben; az elõljáró szorgalmatossággal; a könyörülõ vídámsággal mívelje.
\par 9 A szeretet képmutatás nélkül való legyen. Iszonyodjatok a gonosztól, ragaszkodjatok a jóhoz.
\par 10 Atyafiúi szeretettel egymás iránt gyöngédek; a tiszteletadásban egymást megelõzõk legyetek.
\par 11 Az igyekezetben ne legyetek restek; lélekben buzgók legyetek; az Úrnak szolgáljatok.
\par 12 A reménységben örvendezõk; a háborúságban tûrõk;  a könyörgésben állhatatosak;
\par 13 A szentek szükségeire adakozók legyetek; a vendégszeretetet  gyakoroljátok.
\par 14 Áldjátok azokat, a kik titeket kergetnek; áldjátok és ne átkozzátok.
\par 15 Örüljetek az örülõkkel, és sírjatok a sírókkal.
\par 16 Egymás iránt ugyanazon indulattal legyetek; ne kevélykedjetek, hanem az alázatosakhoz szabjátok magatokat. Ne legyetek bölcsek  timagatokban.
\par 17 Senkinek gonoszért gonoszszal ne fizessetek. A tisztességre gondotok legyen minden ember  elõtt.
\par 18 Ha lehetséges, a mennyire rajtatok áll, minden emberrel békességesen éljetek.
\par 19 Magatokért bosszút ne álljatok szerelmeseim, hanem adjatok helyet ama haragnak; mert meg van írva: Enyém a bosszúállás, én  megfizetek, ezt mondja az Úr.
\par 20 Azért, ha éhezik a te ellenséged, adj ennie; ha szomjuhozik, adj innia; mert ha ezt míveled, eleven szenet gyûjtesz az õ fejére.
\par 21 Ne gyõzettessél meg a gonosztól, hanem a gonoszt jóval gyõzd meg.

\chapter{13}

\par 1 Minden lélek engedelmeskedjék a felsõ hatalmasságoknak; mert nincsen hatalmasság, hanem csak Istentõl: és a mely hatalmasságok vannak, az Istentõl rendeltettek.
\par 2 Azért, a ki ellene támad a hatalmasságnak, az Isten rendelésének támad ellene; a kik pedig ellene támadnak, önmagoknak ítéletet szereznek.
\par 3 Mert a fejedelmek nem a jó, hanem a rossz cselekedetnek rettegésére vannak. Akarod-é pedig, hogy ne félj a hatalmasságtól? Cselekedjed a jót, és dícséreted lesz attól.
\par 4 Mert Isten szolgája õ a te javadra. Ha pedig a gonoszt cselekszed, félj: mert nem ok nélkül viseli a fegyvert: mert Isten szolgája, bosszúálló a haragra annak, a ki gonoszt cselekszik.
\par 5 Annakokáért szükség engedelmeskedni, nem csak a haragért, hanem a lelkiismeretért is.
\par 6 Mert azért fizettek adót is; mivelhogy Istennek szolgái, kik ugyanabban foglalatoskodnak.
\par 7 Adjátok meg azért mindenkinek, a mivel tartoztok: a kinek az adóval, az adót; a kinek a vámmal, a vámot; a kinek a félelemmel, a félelmet; a kinek a tisztességgel, a tisztességet.
\par 8 Senkinek semmivel ne tartozzatok, hanem csak azzal, hogy egymást szeressétek; mert a ki szereti a felebarátját, a törvényt betöltötte.
\par 9 Mert ez: Ne paráználkodjál, ne ölj, ne orozz, hamis tanubizonyságot ne szólj, ne kívánj, és ha valamely más parancsolat van, ebben az ígében foglaltatik egybe: Szeressed felebarátodat mint temagadat.
\par 10 A szeretet nem illeti gonoszszal a felebarátot. Annakokáért a törvénynek betöltése a szeretet.
\par 11 Ezt pedig cselekedjétek, tudván az idõt, hogy ideje már, hogy az álomból felserkenjünk; mert most közelebb van hozzánk az idvesség, mint a mikor hívõkké lettünk.
\par 12 Az éjszaka elmúlt, a nap pedig elközelgett; vessük el azért a sötétségnek cselekedeteit,  és öltözzük fel a világosság fegyvereit.
\par 13 Mint nappal, ékesen járjunk, nem dobzódásokban és részegségekben, nem bujálkodásokban és feslettségekben, nem versengésben és írigységben:
\par 14 Hanem öltözzétek fel az Úr Jézus Krisztust, és a testet ne tápláljátok a kívánságokra.

\chapter{14}

\par 1 A hitben erõtelent fogadjátok be, nem ítélgetvén vélekedéseit.
\par 2 Némely ember azt hiszi, hogy mindent megehetik; a hitben erõtelen pedig zöldséget eszik.
\par 3 A ki eszik, ne vesse meg azt, a ki nem eszik; és a ki nem eszik, ne kárhoztassa azt, a ki eszik. Mert az Isten befogadta õt.
\par 4 Te kicsoda vagy, hogy kárhoztatod a más szolgáját? Az õ tulajdon urának áll vagy esik. De meg fog állani, mert az Úr által képes, hogy megálljon.
\par 5 Emez az egyik napot különbnek tartja a másiknál: amaz pedig minden napot egyformának tart. Ki-ki a maga értelme felõl legyen meggyõzõdve.
\par 6 A ki ügyel a napra, az Úrért ügyel: és a ki nem ügyel a napra, az Úrért nem ügyel. A ki eszik, az Úrért eszik, mert hálákat ád az Istennek: és a ki nem eszik, az Úrért nem eszik, és hálákat ád az Istennek.
\par 7 Mert közülünk senki sem él önmagának, és senki sem hal önmagának:
\par 8 Mert ha élünk, az Úrnak élünk; ha meghalunk, az Úrnak halunk meg. Azért akár éljünk, akár haljunk, az Úréi vagyunk.
\par 9 Mert azért halt meg és támadott fel és elevenedett meg Krisztus, hogy mind holtakon mind élõkön uralkodjék.
\par 10 Te pedig miért kárhoztatod a te atyádfiát? avagy te is miért veted meg a te atyádfiát? Hiszen mindnyájan oda állunk majd a Krisztus ítélõszéke elé.
\par 11 Mert meg van írva: Élek én, mond az Úr, mert nékem hajol meg minden térd, és minden nyelv Istent magasztalja.
\par 12 Azért hát mindenikünk maga ad számot magáról az Istennek.
\par 13 Annakokáért egymást többé ne kárhoztassuk: hanem inkább azt tartsátok, hogy a ti atyátokfiának ne szerezzetek megütközést vagy megbotránkozást.
\par 14 Tudom és meg vagyok gyõzõdve az Úr Jézusban, hogy semmi sem tisztátalan önmagában: hanem bármi annak tisztátalan, a ki tisztátalannak tartja.
\par 15 De ha a te atyádfia az ételért megszomorodik, akkor te nem szeretet szerint cselekszel. Ne veszítsd el azt a te ételeddel, a kiért Krisztus meghalt.
\par 16 Ne káromoltassék azért a ti javatok.
\par 17 Mert az Isten országa nem evés, nem ivás, hanem igazság, békesség és Szent Lélek által való öröm.
\par 18 Mert a ki ezekben szolgál a Krisztusnak, kedves Istennek, és az emberek elõtt megpróbált.
\par 19 Azért tehát törekedjünk azokra, a mik a békességre és az egymás épülésére valók.
\par 20 Ne rontsd le az ételért az Isten munkáját. Minden tiszta ugyan, de rossz annak az embernek, a ki botránkozással eszi.
\par 21 Jó nem enni húst és nem inni bort, sem semmit nem tenni, a miben a te atyádfia megütközik vagy megbotránkozik, vagy erõtelen.
\par 22 Te néked hited van: tartsd meg magadban Isten elõtt. Boldog, a ki nem kárhoztatja magát abban, a mit helyesel.
\par 23 A ki pedig kételkedik, ha eszik, kárhoztatva van, mert nem hitbõl eszik. A mi pedig hitbõl nincs, bûn az.

\chapter{15}

\par 1 Tartozunk pedig mi az erõsek, hogy az erõtelenek erõtlenségeit hordozzuk, és ne magunknak kedveskedjünk.
\par 2 Mindenikünk tudniillik az õ felebarátjának kedveskedjék annak javára, épülésére.
\par 3 Mert Krisztus sem önmagának kedveskedett, hanem a mint meg van írva: A te gyalázóidnak gyalázásai hullottak reám.
\par 4 Mert a melyek régen megirattak, a mi tanulságunkra irattak meg: hogy békességes tûrés által és az írásoknak vígasztalása által reménységünk legyen.
\par 5 A békességes tûrésnek és vígasztalásnak Istene pedig adja néktek, hogy ugyanazon indulat legyen bennetek egymás iránt Krisztus Jézus szerint:
\par 6 Hogy egy szívvel, egy szájjal dicsõítsétek az Istent és a mi Urunk Jézus Krisztusnak Atyját.
\par 7 Azért fogadjátok be egymást, miképen Krisztus is befogadott minket az Isten dicsõségére.
\par 8 Mondom pedig, hogy Jézus Krisztus szolgája lett a körülmetélkedésnek az Isten igazságáért, hogy megerõsítse az atyák  ígéreteit;
\par 9 A pogányok pedig irgalmasságáért dicsõítik Istent, a mint meg van írva: Annakokáért vallást teszek rólad a pogányok között, és dícséretet éneklek a te nevednek.
\par 10 És ismét azt mondja: Örüljetek pogányok az õ népével együtt.
\par 11 És ismét: Dícsérjétek az Urat minden pogányok, és magasztaljátok õt minden népek.
\par 12 És viszont Ésaiás így szól: Lészen a Jessének gyökere, és a ki felkel, hogy uralkodjék a pogányokon; õ benne reménykednek a pogányok.
\par 13 A reménységnek Istene pedig töltsön be titeket minden örömmel és békességgel a hivésben, hogy bõvölködjetek a reménységben a Szent Lélek ereje által.
\par 14 Meg vagyok pedig gyõzõdve atyámfiai én magam is ti felõletek, hogy teljesek vagytok minden jósággal, betöltve minden ismerettel, képesek lévén egymást is inteni.
\par 15 Bátorságosabban írtam pedig néktek atyámfiai, részben, mintegy emlékeztetvén titeket az Istentõl nékem adott kegyelem által,
\par 16 Hogy legyetek a Jézus Krisztus szolgája a pogányok között, munkálkodván az Isten evangyéliomában, hogy legyen a pogányoknak áldozata kedves és a Szent Lélek által megszentelt.
\par 17 Van azért mivel dicsekedjem a Jézus Krisztusban, az Istenre tartozó dolgokban.
\par 18 Mert nem merek szólni semmirõl, a mit nem Krisztus cselekedett volna általam a pogányoknak engedelmességére, szóval és tettel.
\par 19 Jelek és csodák ereje által, az Isten Lelkének ereje által; úgyannyira, hogy én Jeruzsálemtõl és környékétõl fogva Illyriáig betöltöttem a Krisztus evangyéliomát.
\par 20 Ekképen pedig tisztességbeli dolog, hogy ne ott hirdessem az evangyéliomot, ahol neveztetett Krisztus, hogy ne más alapra építsek:
\par 21 Hanem a mint meg van írva: A kiknek nem hirdettetett õ felõle, azok meglátják; és a kik nem hallották, megértik.
\par 22 Annakokáért meg is akadályoztattam gyakran a hozzátok való menetelben.
\par 23 Most pedig, mivelhogy nincs már helyem e tartományokban, vágyódván pedig sok esztendõ óta, hogy elmenjek hozzátok:
\par 24 Ha Hispániába megyek, elmegyek ti hozzátok. Mert remélem, hogy átutazóban meglátlak titeket, és ti elkísértek oda, ha elõbb részben beteljesedem veletek.
\par 25 Most pedig megyek Jeruzsálembe, szolgálván a szenteknek.
\par 26 Mert tetszett Macedóniának és Akhájának, hogy a Jeruzsálembeli szentek szegényei részére némileg adakozzanak.
\par 27 Mert tetszett nékik, és tartoznak is vele. Mert ha a pogányok azoknak a lelki javaiban részesültek, tartoznak nékik viszont szolgálni a  testiekben.
\par 28 Ezt azért ha majd elvégezem, és nékik e gyümölcsöt átadom, elmegyek közöttetek által Hispániába.
\par 29 Tudom pedig, hogy mikor hozzátok megyek, a Krisztus evangyélioma áldásának teljességével megyek.
\par 30 Kérlek pedig titeket atyámfiai a mi Urunk Jézus Krisztusra és a Lélek szerelmére, tusakodjatok velem együtt az imádkozásokban, én érettem Isten elõtt,
\par 31 Hogy szabaduljak meg azoktól, a kik engedetlenek Júdeában, és hogy az én Jeruzsálemben való szolgálatom legyen kedves a szentek elõtt;
\par 32 Hogy örömmel menjek hozzátok az Isten akaratából, és veletek együtt megújuljak.
\par 33 A békességnek Istene pedig legyen mindnyájan ti veletek! Ámen.

\chapter{16}

\par 1 Ajánlom pedig néktek Fébét, a mi nénénket, ki a Kenkhréabeli gyülekezetnek szolgálója:
\par 2 Hogy fogadjátok õt az Úrban szentekhez illendõen, és legyetek mellette, ha valami dologban rátok szorul. Mert õ is sokaknak pártfogója volt, nékem magamnak is.
\par 3 Köszöntsétek Priscillát és Akvilát, kik nékem munkatársaim Krisztus Jézusban
\par 4 A kik az én életemért a saját nyakukat tették le; a kiknek nemcsak én mondok köszönetet, hanem a pogányok minden gyülekezete is.
\par 5 És köszöntsétek azt a gyülekezetet, mely az õ házokban van. Köszöntsétek az én szerelmetes Epenétusomat, a ki Akhája zsengéje a Krisztusban.
\par 6 Köszöntsétek Máriát, ki sokat munkálkodott körülöttünk.
\par 7 Köszöntsétek Andronikust és Juniát az én rokonaimat és az én fogolytársaimat, a kik híresek az apostolok között, a kik nálamnál is elõbb voltak a Krisztusban.
\par 8 Köszöntsétek Ampliást, ki nékem szerelmesem az Úrban.
\par 9 Köszöntsétek Orbánt, a mi munkatársunkat a Krisztusban, és Stakhist az én szerelmesemet.
\par 10 Köszöntsétek Apellest, ki a Krisztusban megpróbáltatott. Köszöntsétek az Aristóbulus háznépébõl valókat.
\par 11 Köszöntsétek Heródiont az én rokonomat. Köszöntsétek a Nárcissus háznépébõl azokat, a kik az Úrban vannak.
\par 12 Köszöntsétek Trifénát és Trifósát, kik munkálódnak az Úrban. Köszöntsétek a szerelmetes Persist, ki sokat munkálódott az Úrban.
\par 13 Köszöntsétek Rufust, ki kiválasztott az Úrban, és az õ anyját, a ki az enyém is.
\par 14 Köszöntsétek Ásinkritust, Flégont, Hermást, Pátrobást, Merkuriust, és az atyafiakat, kik velök vannak.
\par 15 Köszöntsétek Filológust és Juliát, Néreust és az õ nénjét, és Olimpást és minden szenteket, kik velök vannak.
\par 16 Köszöntsétek egymást szent csókolással. Köszöntenek titeket a Krisztus gyülekezetei.
\par 17 Kérlek pedig titeket atyámfiai, vigyázzatok azokra, a kik szakadásokat és botránkozásokat okoznak a tudomány körül, melyet tanultatok; és azoktól hajoljatok el.
\par 18 Mert az ilyenek a mi Urunk Jézus Krisztusnak nem szolgálnak, hanem az õ hasuknak; és nyájas beszéddel, meg hizelkedéssel megcsalják az ártatlanoknak szívét.
\par 19 Mert a ti engedelmességetek mindenekhez eljutott. Örülök azért rajtatok; de akarom, hogy bölcsek legyetek a jóban, ártatlanok pedig a rosszban.
\par 20 A békességnek Istene megrontja a Sátánt a ti lábaitok alatt hamar. A mi Urunk Jézus Krisztus kegyelme veletek. Ámen.
\par 21 Köszöntenek titeket Timótheus, az én munkatársam, és Luczius, Jáson, és Sosipáter az én rokonaim.
\par 22 Köszöntelek titeket az Úrban és Tertius, ki e levelet írtam.
\par 23 Köszönt titeket Gájus, a ki nékem és az egész gyülekezetnek gazdája. Köszönt tieket Erástus a városnak kincstartója, és Kvártus atyafi.
\par 24 A mi Urunk Jézus Krisztus kegyelme mindnyájan ti veletek. Ámen.
\par 25 Annak pedig, a ki titeket megerõsíthet az én evangyéliomom és a Jézus Krisztus hirdetése szerint, ama titoknak kijelentése folytán, mely örök idõtõl fogva el volt  hallgatva,
\par 26 Most pedig megjelentetett a prófétai írások által, az örök Isten parancsolata szerint, a hitben való engedelmesség végett minden pogányoknak tudomására adatván,
\par 27 Az egyedül bölcs Istennek a Jézus Krisztus által dicsõség mindörökké. Ámen.


\end{document}