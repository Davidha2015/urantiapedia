\begin{document}

\title{Filippieknek írt levél}


\chapter{1}

\par 1 Pál és Timótheus, Jézus Krisztus szolgái, minden szenteknek a Krisztus Jézusban, a kik Filippiben vannak, a püspökökkel és diakónusokkal egyetemben:
\par 2 Kegyelem néktek és békesség Istentõl, a mi Atyánktól és az Úr Jézus Krisztustól.
\par 3 Hálát adok az én Istenemnek, minden ti rólatok való emlékezésemben,
\par 4 Mindenkor minden én könyörgésemben mindenitekért nagy örömmel könyörögvén,
\par 5 Mivelhogy résztvettetek az evangyéliom ügyében az elsõ naptól fogva mind ez ideig;
\par 6 Meg lévén gyõzõdve arról, hogy a ki elkezdette bennetek a jó dolgot, elvégezi a Krisztus Jézusnak napjáig:
\par 7 A mint hogy méltó, hogy én ilyen értelemben legyek mindenitek felõl, azért, mert én szívemben hordalak titeket, mint a kik mind az én fogságomban, mind az evangyéliomnak oltalmazásában és megbizonyításában mindnyájan részestársaim vagytok a nékem adott kegyelemben.
\par 8 Mert bizonyságom az Isten, mely igen vágyakozom mindnyájatok után a Krisztus Jézus szerelmében.
\par 9 És azért imádkozom, hogy a ti szeretetetek még jobban-jobban bõvölködjék ismeretben és minden értelmességben;
\par 10 Hogy megítélhessétek, hogy mi a rossz és mi a jó; hogy legyetek tiszták és botlás nélkül valók a Krisztusnak napjára;
\par 11 Teljesek lévén az igazságnak gyümölcsével, melyet Jézus Krisztus teremt az Isten dicsõségére és magasztalására.
\par 12 Tudtotokra akarom pedig adni, atyámfiai, hogy az én dolgaim inkább elõmenetelére lõnek az evangyéliomnak;
\par 13 Annyira, hogy a Krisztusban híressé lett az én fogságom a testõrség egész házában és minden mások elõtt;
\par 14 És többen az Úrban való atyafiak közül bízván az én fogságomban, nagyobb bátorsággal merik szólani az ígét.
\par 15 Némelyek ugyanis irígységbõl és versengésból is, de mások jóakaratból is hirdetik a Krisztust.
\par 16 Némelyek versengésbõl prédikálják a Krisztust, nem tiszta lélekkel, azt hivén, hogy fogságom nyomorúságait így megnevelik;
\par 17 De mások szeretetbõl, tudván, hogy én az evangyéliomnak oltalmazására rendeltettem.
\par 18 Mert mit mondjak? csakhogy minden módon, akár színbõl, akár szívbõl, a Krisztus prédikáltatik: és én ennek örülök, sõt örülni is fogok.
\par 19 Mert tudom, hogy ez nékem idvességemre lesz a ti könyörgéstek által és a Jézus Krisztus Lelkének segedelme által,
\par 20 Amaz én esengõ várásom és reménységem szerint, hogy semmiben meg nem szégyenülök, hanem mint mindenkor, úgy most is nagy bátorsággal fog magasztaltatni Krisztus az én testemben, akár életem, akár halálom által.
\par 21 Mert nékem az élet Krisztus, és a meghalás nyereség.
\par 22 De ha e testben való életem munkámat gyümölcsözteti: hogy melyiket válaszszam, meg sem mondhatom.
\par 23 Mert szorongattatom e kettõ között, kívánván elköltözni és a Krisztussal lenni; mert ez sokkal inkább jobb;
\par 24 De e testben megmaradnom szükségesebb ti érettetek.
\par 25 És ebben bízva, tudom, hogy megmaradok és együtt maradok mindnyájatokkal a ti hitben való gyarapodástokra és örömötökre;
\par 26 Hogy bõven dicsekedhessetek velem Krisztus Jézusban az én ti nálatok való újabb megjelenésem által.
\par 27 Csakhogy a Krisztus evangyéliomához méltóan viseljétek magatokat, hogy akár oda menvén és látván titeket, akár távol lévén, azt halljam dolgaitok felõl, hogy megállotok egy lélekben, egy érzéssel viaskodván az evangyéliom hitéért;
\par 28 És meg nem félemlvén semmiben az ellenségek elõtt: a mi azoknak a veszedelem jele, néktek pedig az üdvösségé, és ez az Istentõl van;
\par 29 Mert néktek adatott az a kegyelem a Krisztusért, nemcsak hogy higyjetek Õ benne, hanem hogy szenvedjetek is Õ érette:
\par 30 Ugyanolyan tusakodástok lévén, a milyent láttatok én nálam, és most hallotok és felõlem.

\chapter{2}

\par 1 Ha annakokáért helye van Krisztusban az intésnek, ha helye van a szeretet vígasztalásának, ha helye van a Lélekben való közösségnek, ha helye van a szívnek és könyörületességnek,
\par 2 Teljesítsétek be az én örömömet, hogy egyenlõ indulattal legyetek, ugyanazon szeretettel viseltetvén, egy érzésben, egyugyanazon indulattal lévén;
\par 3 Semmit nem cselekedvén versengésbõl, sem hiábavaló dicsõségbõl, hanem alázatosan egymást különbeknek tartván ti magatoknál.
\par 4 Ne nézze kiki a maga hasznát, hanem mindenki a másokét is.
\par 5 Annakokáért az az indulat legyen bennetek, mely volt a Krisztus Jézusban is,
\par 6 A ki, mikor Istennek formájában vala, nem tekintette zsákmánynak azt, hogy õ az Istennel egyenlõ.
\par 7 Hanem önmagát megüresíté, szolgai formát vévén föl, emberekhez hasonlóvá lévén;
\par 8 És mikor olyan állapotban találtatott mint ember, megalázta magát, engedelmes lévén halálig, még pedig a keresztfának haláláig.
\par 9 Annakokáért az Isten is felmagasztalá õt, és ajándékoza néki oly nevet, a mely minden név fölött való;
\par 10 Hogy a Jézus nevére minden térd meghajoljon, mennyeieké, földieké és föld alatt valóké.
\par 11 És minden nyelv vallja, hogy Jézus Krisztus Úr az Atya Isten dicsõségére.
\par 12 Annakokáért, szerelmeseim, a miképen mindenkor engedelmeskedtetek, nem úgy, mint az én jelenlétemben csak, hanem most sokkal inkább az én távollétemben, félelemmel és rettegéssel vigyétek véghez a ti idvességteket;
\par 13 Mert Isten az, a ki munkálja bennetek mind az akarást, mind a munkálást jó kedvébõl.
\par 14 Mindeneket zúgolódások és versengések nélkül cselekedjetek;
\par 15 Hogy legyetek feddhetetlenek és tiszták, Istennek szeplõtlen gyermekei az elfordult és elvetemedett nemzetség közepette, kik között fényletek, mint csillagok e világon.
\par 16 Életnek beszédét tartván elébök; hogy dicsekedhessem majd a Krisztus napján, hogy nem futottam hiába, sem nem fáradtam hiába.
\par 17 De ha kiontatom is italáldozatként a ti hitetek áldozatánál és papiszolgálatánál, mégis örülök, és együtt örülök maindnyájatokkal;
\par 18 Azonképen ti is örüljetek, és örüljetek együtt velem.
\par 19 Reménylem pedig az Úr Jézusban, hogy Timótheust rövid nap elküldöm tihozzátok, hogy én is megviduljak, megértvén a ti dolgaitokat.
\par 20 Mert nincsen velem senki hozzá hasonló indulatú, a ki igazán szívén viselné dolgaitokat.
\par 21 Mert mindenki a maga hasznát keresi, nem a Krisztus Jézusét.
\par 22 Az õ kipróbált voltát pedig ismeritek, hogy miképen atyjával a gyermek, együtt szolgált velem az evangyéliom ügyében.
\par 23 Õt azért reménylem, hogy elküldöm, mihelyt meglátom az én dolgaimat, tüstént;
\par 24 Bízom pedig az Úrban, hogy magam is csakhamar el fogok menni.
\par 25 De szükségesnek tartám, hogy Epafróditust, az én atyámfiát és munkatársamat és bajtársamat, néktek pedig követeteket és szükségemben áldozatot hozó szolgátokat hazaküldjem hozzátok;
\par 26 Mivelhogy vágyva vágyott mindnyájatok után, és gyötrõdött a miatt, hogy meghallottátok, hogy õ beteg volt.
\par 27 Mert bizony beteg volt, halálhoz közel; de az Isten könyörült rajta, nem csak õ rajta pedig, hanem én rajtam is, hogy szomorúság ne jõjjön szomorúságomra.
\par 28 Annakokáért hamarabb küldtem õt haza, hogy meglátván õt, ismét örüljetek, és nékem is kisebb legyen a szomorúságom.
\par 29 Fogadjátok azért õt az Úrban teljes örömmel; és az ilyeneket megbecsüljétek:
\par 30 Mert a Krisztus dolgáért jutott majdnem halálra, koczkára tévén életét, hogy kárpótoljon engem azért, hogy nékem tett szolgálatotoknál ti nem voltatok jelen.

\chapter{3}

\par 1 Továbbá atyámfiai, örüljetek az Úrban. Ugyanazokat írni néktek én nem restellem, tinéktek pedig bátorságos.
\par 2 Õrizkedjetek az ebektõl, õrizkedjetek a gonosz munkásoktól, õrizkedjetek a megmetélkedéstõl:
\par 3 Mert mi vagyunk a körülmetélkedés, a kik lélekben szolgálunk az Istennek, és a Krisztus Jézusban dicsekedünk, és nem a testben bizakodunk:
\par 4 Jóllehet énnékem van bizakodásom test szerint is. Ha bárki más mer testben bizakodni, én sokkal inkább;
\par 5 Körülmetéltettem nyolczadnapon, Izráel nemzetségébõl, Benjámin törzsébõl való vagyok, zsidókból való zsidó, törvény tekintetében farizeus,
\par 6 Buzgóság tekintetében az egyházat üldözõ, a törvénybeli igazság tekintetében feddhetetlen voltam.
\par 7 De a melyek nékem egykor nyereségek valának, azokat a Krisztusért kárnak ítéltem.
\par 8 Sõt annakfelette most is kárnak ítélek mindent az én Uram, Jézus Krisztus ismeretének gazdagsága miatt: a kiért mindent kárba veszni hagytam és szemétnek ítélek, hogy a Krisztust megnyerjem,
\par 9 És találtassam Õ benne, mint a kinek nincsen saját igazságom a törvénybõl, hanem van igazságom a Krisztusban való hit által, Istentõl való igazságom a hit alapján:
\par 10 Hogy megismerjem Õt, és az Õ feltámadásának erejét, és az Õ szenvedéseiben való részesülésemet, hasonlóvá lévén az õ halálához;
\par 11 Ha valami módon eljuthatnék a halottak feltámadására.
\par 12 Nem mondom, hogy már elértem, vagy hogy már tökéletes volnék; hanem igyekezem, hogy el is érjem, a miért meg is ragadott engem a Krisztus Jézus.
\par 13 Atyámfiai, én enmagamról nem gondolom, hogy már elértem volna:
\par 14 De egyet cselekszem, azokat, a melyek hátam megett vannak, elfelejtvén, azoknak pedig, a melyek elõttem vannak, nékik dõlvén, czélegyenest igyekszem az Istennek a Krisztus Jézusban onnét felülrõl való elhívása jutalmára.
\par 15 Valakik annakokáért tökéletesek vagyunk, ilyen értelemben legyünk: és ha valamiben másképen értetek, az Isten azt is ki fogja jelenteni néktek:
\par 16 Csakhogy a mire eljutottunk, ugyanabban egy szabály szerint járjunk, ugyanazon értelemben legyünk.
\par 17 Legyetek én követõim, atyámfiai, és figyeljetek azokra, a kik úgy járnak, a miképen mi néktek példátok vagyunk.
\par 18 Mert sokan járnak másképen, kik felõl sokszor mondtam néktek, most pedig sírva is mondom, hogy a Krisztus keresztjének ellenségei;
\par 19 Kiknek végök veszedelem, kiknek Istenök az õ hasok, és a kiknek dicsõségök az õ gyalázatukban van, kik mindig a földiekkel törõdnek.
\par 20 Mert a mi országunk mennyekben van, honnét a megtartó Úr Jézus Krisztust is várjuk;
\par 21 Ki elváltoztatja a mi nyomorúságos testünket, hogy hasonló legyen az Õ dicsõséges testéhez, amaz Õ hatalmas munkája szerint, mely által maga alá is vethet mindeneket.

\chapter{4}

\par 1 Annakokáért szerelmes atyámfiai, a kik után úgy vágyakozom, ti én örömöm és én koronám, ekképen álljatok meg az Úrban, én szerelmeseim!
\par 2 Evódiát intem, Sintikhét is intem, hogy egyenlõ indulattal legyenek az Úrban.
\par 3 Igen, kérlek téged is, igaz szolgatársam, légy segítségül ezeknek, mint a kik az evangyéliom dolgában együtt viaskodtak velem, Kelemennel is, és ama többi munkatársaimmal, kiknek neveik fölírvák az életnek könyvében.
\par 4 Örüljetek az Úrban mindenkor; ismét mondom, örüljetek!
\par 5 A ti szelídlelkûségetek ismert legyen minden ember elõtt. Az Úr közel!
\par 6 Semmi felõl ne aggódjatok, hanem imádságotokban és könyörgéstekben minden alkalommal hálaadással tárjátok fel kívánságaitokat az Isten elõtt.
\par 7 És az Istennek békessége, mely minden értelmet felül halad, meg fogja õrizni szíveiteket és gondolataitokat a Krisztus Jézusban.
\par 8 Továbbá, Atyámfiai, a mik csak igazak, a mik csak tisztességesek, a mik csak igazságosak, a mik csak tiszták, a mik csak kedvesek, a mik csak jó hírûek; ha van valami erény és ha van valami dícséret, ezekrõl gondolkodjatok.
\par 9 A miket tanultatok is, el is fogadtatok, hallottatok is, láttatok is én tõlem, azokat cselekedjétek; és a békességnek Istene veletek lesz.
\par 10 Felette igen örültem pedig az Úrban, hogy immár valahára megújultatok az én felõlem való gondviseléstekben; mely dologban gondoskodtatok is, de nem volt alkalmatok.
\par 11 Nem hogy az én szûkölködésemre nézve szólnék; mert én megtanultam, hogy azokban, a melyekben vagyok, megelégedett legyek.
\par 12 Tudok megaláztatni is, tudok bõvölködni is; mindenben és mindenekben ismerõs vagyok a jóllakással is, az éhezéssel is, a bõvölködéssel is, a szûkölködéssel is.
\par 13 Mindenre van erõm a Krisztusban, a ki engem megerõsít.
\par 14 Mindazáltal jól tettétek, hogy nyomorúságomban részesekké lettetek.
\par 15 Tudjátok pedig ti is, Filippibeliek, hogy az evangyéliom hirdetésének kezdetén, mikor Macedóniából kimentem, egyetlen egyház sem volt részes velem a kölcsönös adásban és vevésben, csak ti egyedül:
\par 16 Mert már Thessalónikában is, egyszer is, másszor is, küldtetek nékem szükségemre.
\par 17 Nem mintha kívánnám az ajándékot: hanem kívánom azt a gyümöcsöt, mely sokasodik a ti hasznotokra.
\par 18 Megkaptam pedig mindent, és bõvölködöm; beteltem, vévén Epafróditustól, a mit küldöttetek, mint kedves jó illatot, kellemes, tetszõ áldozatot az Istennek.
\par 19 Az én Istenem pedig be fogja tölteni minden szükségeteket az Õ gazdagsága szerint dicsõségesen a Krisztus Jézusban.
\par 20 Az Istennek pedig és a mi Atyánknak dicsõség mind örökkön örökké. Ámen.
\par 21 Köszöntsetek minden szentet a Krisztus Jézusban. Köszöntenek titeket az atyafiak, a kik velem vannak.
\par 22 Köszöntenek titeket minden szentek, mindeneknek felette pedig a császár udvarából valók.
\par 23 A mi Urunk Jézus Krisztusnak kegyelme legyen mindnyájatokkal! Ámen.


\end{document}