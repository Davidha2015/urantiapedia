\begin{document}

\title{Kolosszeieknek írt levél}


\chapter{1}

\par 1 Pál, Jézus Krisztusnak apostola Isten akaratjából, és Timótheus atyafi.
\par 2 A Kolosséban levõ szenteknek és hívõ atyafiaknak a Krisztusban: kegyelem néktek és békesség Istentõl, a mi Atyánktól, és az Úr Jézus Krisztustól.
\par 3 Hálát adunk az Istennek és a mi Urunk Jézus Krisztus Atyjának, mindenkor ti értetek könyörögvén,
\par 4 Mivelhogy hallottuk a ti hiteteket a Krisztus Jézusban, és a szeretetet, a melylyel minden szentekhez vagytok,
\par 5 A mennyekben néktek eltett reménységért, a melyet már elõbb hallottatok az igazság beszédében, mely az evangyéliom:
\par 6 Mely eljutott hozzátok, miképen az egész világra is, és gyümölcsöt terem, úgy mint nálatok is a naptól fogva, melyen hallottátok és megismertétek az Isten kegyelmét igazán:
\par 7 Úgy a hogy tanultátok is Epafrástól, a mi szerelmes szolgatársunktól, a ki hív szolgája ti érettetek a Krisztusnak,
\par 8 A ki meg is jelentette nékünk a ti Lélekben való szereteteteket.
\par 9 Azért mi is, a mely naptól fogva ezeket hallottuk, nem szûnünk meg érettetek imádkozni, és kérni, hogy betöltessetek az Isten akaratának megismerésével minden lelki bölcseségben és értelemben,
\par 10 Hogy járjatok méltóan az Úrhoz, teljes tetszésére, minden jó cselekedettel gyümölcsöt teremvén és nevekedvén az Isten megismerésében;
\par 11 Minden erõvel megerõsíttetvén az Õ dicsõségének hatalma szerint minden kitartásra és hosszútûrésre örömmel;
\par 12 Hálákat adván az Atyának, ki alkalmasakká tett minket a szentek örökségében való részvételre a világosságban;
\par 13 A ki megszabadított minket a sötétség hatalmából, és általvitt az Õ szerelmes Fiának országába;
\par 14 Kiben van a mi váltságunk az Õ vére által, bûneinknek bocsánata;
\par 15 A ki képe a láthatatlan Istennek, minden teremtménynek elõtte született;
\par 16 Mert Õ benne teremtetett minden, a mi van a mennyekben és a földön, láthatók és láthatatlanok, akár királyi székek, akár uraságok, akár fejedelemségek, akár hatalmasságok; mindenek Õ általa és Õ reá nézve teremttettek;
\par 17 És Õ elõbb volt mindennél, és minden Õ benne áll fenn.
\par 18 És Õ a feje a testnek, az egyháznak: a ki a kezdet, elsõszülött a halottak közül; hogy mindenekben Õ legyen az elsõ;
\par 19 Mert tetszett az Atyának, hogy Õ benne lakozzék az egész teljesség;
\par 20 És hogy Õ általa békéltessen meg mindent Magával, békességet szerezvén az Õ keresztjének vére által; Õ általa mindent, a mi csak van, akár a földön, akár a mennyekben.
\par 21 Titeket is, kik hajdan elidegenültek és ellenségek valátok gonosz cselekedetekben gyönyörködõ értelmetek miatt, most mégis megbékéltetett.
\par 22 Az Õ emberi testében a halál által, hogy mint szenteket, tisztákat és feddhetetleneket állasson titeket Õ maga elé:
\par 23 Ha ugyan megmaradtok a hitben alaposan és erõsen, és el nem távoztok az evangyéliom reménységétõl, a melyet hallottatok, a mely hirdettetett minden teremtménynek az ég alatt; a melynek lettem én, Pál, szolgájává.
\par 24 Most örülök a ti érettetek való szenvedéseimnek, és a magam részérõl betöltöm a mi híja van a Krisztus szenvedéseinek az én testemben az Õ testéért, a mi az egyház;
\par 25 A melynek lettem én szolgájává az Isten sáfársága szerint, a melyet nékem adott ti rátok nézve, hogy betöltsem az Isten ígéjét,
\par 26 Tudniillik ama titkot, mely el vala rejtve õsidõk óta és nemzetségek óta, most pedig megjelentetett az Õ szenteinek,
\par 27 A kikkel az Isten meg akarta ismertetni azt, hogy milyen nagy a pogányok között eme titok dicsõségének gazdagsága, az tudniillik, hogy a Krisztus ti köztetek van, a dicsõségnek ama reménysége:
\par 28 A kit mi prédikálunk, intvén minden embert, és tanítván minden embert minden bölcseséggel, hogy minden embert tökéletesnek állassunk a Krisztus Jézusban;
\par 29 A mire igyekezem is, tusakodván az Õ ereje szerint, mely én bennem hatalmasan munkálkodik.

\chapter{2}

\par 1 Mert akarom, hogy tudtotokra legyen, hogy milyen nagy tusakodásom van ti érettetek, és azokért kik Laodiczeában vannak, és mindazokért, a kik nem láttak engem személy szerint e testben;
\par 2 Hogy vígasztalást vegyen az õ szívök, egybeköttetvén a szeretetben, és hogy eljussanak az értelem meggyõzõdésének teljes gazdagságára, az Isten és az Atya és a Krisztus ama titkának megismerésére,
\par 3 A melyben van a bölcsességnek és ismeretnek minden kincse elrejtve.
\par 4 Ezt pedig azért mondom, hogy valaki titeket rá ne szedjen hitetõ beszéddel.
\par 5 Mert jóllehet testben távol vagyok tõletek, mindazáltal lélekben veletek vagyok, örülvén és látván ti köztetek a jó rendet és Krisztusba vetett hiteteknek erõsségét.
\par 6 Azért, a miképen vettétek a Krisztus Jézust, az Urat, akképen járjatok Õ benne,
\par 7 Meggyökerezvén és tovább épülvén Õ benne, és megerõsödvén a hitben, a miképen arra taníttattatok, bõvölködvén abban hálaadással.
\par 8 Meglássátok, hogy senki ne legyen, a ki bennetek zsákmányt vet a bölcselkedés és üres csalás által, mely emberek rendelése szerint, a világ elemi tanításai szerint, és nem a Krisztus szerint való:
\par 9 Mert Õ benne lakozik az istenségnek egész teljessége testileg,
\par 10 És ti Õ benne vagytok bételjesedve, a ki feje minden fejedelemségnek és hatalmasságnak;
\par 11 A kiben körül is metéltettetek kéz nélkül való körülmetéléssel, levetkezvén az érzéki bûnök testét a Krisztus körülmetélésében;
\par 12 Eltemettetvén Õ vele együtt a keresztségben, a kiben egyetemben fel is támasztattatok az Isten erejébe vetett hit által, a ki feltámasztá Õt a halálból.
\par 13 És titeket, kik holtak valátok a bûnökben és a ti testeteknek körülmetéletlenségében, megelevenített együtt Õ vele, megbocsátván minden bûnötöket,
\par 14 Az által, hogy eltörölte a parancsolatokban ellenünk szóló kézírást, a mely ellenünkre volt nekünk, és azt eltette az útból, odaszegezvén azt a keresztfára;
\par 15 Lefegyverezvén a fejedelemségeket és a hatalmasságokat, õket bátran mutogatta, diadalt vévén rajtok abban.
\par 16 Senki azért titeket meg ne ítéljen evésért, vagy ivásért, avagy ünnep, vagy újhold, vagy szombat dolgában:
\par 17 Melyek csak árnyékai a következendõ dolgoknak, de a valóság a Krisztusé.
\par 18 Senki tõletek a pálmát el ne vegye, kedvét találván alázatoskodásban és az angyalok tisztelésében, a melyeket nem látott, olyakat tudakozván, ok nélkül felfuvalkodván az õ testének értelmével.
\par 19 És nem ragaszkodván a Fõhöz, a Kibõl az egész test, a kapcsok és kötelek által segedelmet vévén és egybeszerkesztetvén, nevekedik az Isten szerint való nevekedéssel.
\par 20 Ha meghalván a Krisztussal, megszabadultatok e világ elemi tanításaitól, miért terheltetitek magatokat, mintha e világban élõk volnátok, efféle rendelésekkel:
\par 21 Ne fogd meg, meg se kóstold, még csak ne is illesd.
\par 22 (A mik mind a velök való élés által elfogyasztásra vannak rendelve), az emberek parancsolatai és tanításai szerint?
\par 23 A melyek bölcsességnek látszanak ugyan a magaválasztotta istentiszteletben és alázatoskodásban és a test gyötrésében; de nincs bennök semmi becsülni való, mivelhogy a test hízlalására valók.

\chapter{3}

\par 1 Annakokáért ha feltámadtatok a Krisztussal, az odafelvalókat keressétek, a hol a Krisztus van, az Istennek jobbján ülvén,
\par 2 Az odafelvalókkal törõdjetek, nem a földiekkel.
\par 3 Mert meghaltatok, és a ti éltetek el van rejtve együtt a Krisztussal az Istenben.
\par 4 Mikor a Krisztus, a mi életünk, megjelen, akkor majd ti is, Õ vele együtt, megjelentek dicsõségben.
\par 5 Öldököljétek meg azért a ti földi tagjaitokat, paráznaságot, tisztátalanságot, bujaságot, gonosz kívánságot és a fösvénységet, a mi bálványimádás;
\par 6 Melyek miatt jõ az Isten haragja az engedetlenség fiaira;
\par 7 Melyekben ti is jártatok régenten, mikor éltetek azokban.
\par 8 Most pedig vessétek el magatoktól ti is mindazokat; haragot, fölgerjedést, gonoszságot és szátokból a káromkodást és gyalázatos beszédet.
\par 9 Ne hazudjatok egymás ellen, mivelhogy levetkeztétek amaz ó embert, az õ cselekedeteivel együtt.
\par 10 És felöltöztétek amaz új embert, melynek újulása van Annak ábrázatja szerint való ismeretre, a ki teremtette azt:
\par 11 A hol nincs többé görög és zsidó: körülmetélkedés és körülmetélkedetlenség, idegen, scithiai, szolga, szabad, hanem minden és mindenekben Krisztus.
\par 12 Öltözzetek föl azért mint az Istennek választottai, szentek és szeretettek, könyörületes szívet, jóságosságot, alázatosságot, szelídséget, hosszútûrést;
\par 13 Elszenvedvén egymást és megbocsátván köcsönösen egymásnak, ha valakinek valaki ellen panasza volna; miképen a Krisztus is megbocsátott néktek, akképen ti is;
\par 14 Mindezeknek fölébe pedig öltözzétek föl a szeretetet, mint a mely a tökéletességnek kötele.
\par 15 És az Istennek békessége uralkodjék a ti szívetekben, amelyre el is hívattatok egy testben; és háládatosak legyetek.
\par 16 A Krisztusnak beszéde lakozzék ti bennetek gazdagon, minden bölcsességben; tanítván és intvén egymást zsoltárokkal, dícséretekkel, lelki énekekkel, hálával zengedezvén a ti szívetekben az Úrnak.
\par 17 És mindent, a mit csak cselekesztek szóval vagy tettel, mindent az Úr Jézusnak nevében cselekedjetek, hálát adván az Istennek és Atyának Õ általa.
\par 18 Ti asszonyok, engedelmeskedjetek a ti férjeteknek, a miképen illik az Úrban.
\par 19 Ti férfiak, szeressétek a ti feleségeteket, és ne legyetek irántok keserû kedvûek.
\par 20 Ti gyermekek, szót fogadjatok a ti szüleiteknek mindenben; mert ez kedves az Úrnak.
\par 21 Ti atyák, ne bosszantsátok a ti gyermekeiteket, hogy kétségbe ne essenek.
\par 22 Ti szolgák, szót fogadjatok mindenben a ti test szerint való uraitoknak, nem a szemnek szolgálván, mint a kik embereknek akarnak tetszeni, hanem szíveteknek egyenességében, félvén az Istent.
\par 23 És valamit tesztek, lélekbõl cselekedjétek, mint az Úrnak és nem embereknek;
\par 24 Tudván, hogy ti az Úrtól veszitek az örökségnek jutalmát: mert az Úr Krisztusnak szolgáltok.
\par 25 A ki pedig igazságtalanságot cselekszik, jutalmát veszi igazságtalanságának; és nincsen személyválogatás.

\chapter{4}

\par 1 Ti urak, a mi igazságos és méltányos, a ti szolgáitoknak megadjátok, tudván, hogy néktek is van Uratok mennyekben.
\par 2 Az imádságban álhatatosak legyetek, vigyázván abban hálaadással;
\par 3 Imádkozván egyszersmind mi érettünk is, hogy az Isten nyissa meg elõttünk az íge ajtaját, hogy szólhassuk a Krisztus titkát, a melyért fogoly is vagyok;
\par 4 Hogy nyilvánvalóvá tegyem azt úgy, a mint nékem szólnom kell.
\par 5 Bölcsen viseljétek magatokat a kívül valók irányában, a jó alkalmatosságot áron is megváltván.
\par 6 A ti beszédetek mindenkor kellemetes legyen, sóval fûszerezett; hogy tudjátok, hogy mimódon kell néktek kinek-kinek megfelelnetek.
\par 7 Minden én dolgaimat megismerteti veletek Tikhikus, a szeretett atyafi és hív szolga és szolgatárs az Úrban;
\par 8 A kit épen a végett küldtem hozzátok, hogy megismerje a ti dolgaitokat és megvígasztalja a ti szíveteket,
\par 9 Onézimussal együtt, a hû és szeretett atyafival, ki ti közületek való; minden itt való dolgot megismertetnek õk veletek.
\par 10 Köszönt titeket Aristárkhus, az én fogolytársam, és Márk, a Barnabás unokatestvére, (ki felõl parancsolatokat vettetek: ha hozzátok megy, fogadjátok õt szívesen),
\par 11 És Jézus, kit Justusnak is hívnak, kik a zsidók közül valók: csak ezek azok a munkatársaim az Isten országában, a kik nékem vígasztalásomra voltak.
\par 12 Köszönt titeket Epafrás, ki ti közületek való, Krisztusnak szolgája, mindenkor tusakodván ti érettetek imádságaiban, hogy megállhassatok tökéletesen és teljes meggyõzõdéssel az Istennek minden akaratjában.
\par 13 Mert bizonyságot teszek õ felõle, hogy sokat fárad érettetek és azokért, kik Laodiczeában és Jerápolisban vannak.
\par 14 Köszönt titeket Lukács, ama szeretett orvos, és Démás.
\par 15 Köszöntsétek az atyafiakat, kik Laodiczeában vannak, és Nimfást és a gyülekezetet, mely az õ házánál van.
\par 16 És mikor felolvastatik nálatok e levél, cselekedjétek meg, hogy a laodiczeaiak gyülekezetében is felolvastassék, és hogy a Laodiczeából átveendõ levelet ti is felolvassátok.
\par 17 És mondjátok meg Arkhippusnak: Vigyázz a szolgálatra, melyre vállalkoztál az Úrban, hogy azt betöltsed!
\par 18 A köszöntés a saját kezemmel, a Páléval van írva. Emlékezzetek meg az én bilincseimrõl! A kegyelem veletek. Ámen.


\end{document}