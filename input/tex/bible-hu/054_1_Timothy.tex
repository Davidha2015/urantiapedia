\begin{document}

\title{1 Timothy}


\chapter{1}

\par 1 Pál, Jézus Krisztus apostala a mi megtartó Istenünknek, és Jézus Krisztusnak a mi reménységünknek rendelése szerint.
\par 2 Timótheusnak, az én igaz fiamnak a hitben: Kegyelem, irgalmasság és békesség Istentõl, a mi Atyánktól. és Krisztus Jézustól, a mi Urunktól.
\par 3 A miképpen Maczedóniába menetelemkor kértelek téged, hogy maradj Efézusban, hogy megmondjad némelyeknek, ne tanítsanak más tudományt,
\par 4 Se mesékkel és végehossza nélkül való nemzetségi táblázatokkal ne foglalkozzanak, a melyek inkább versengéseket támasztanak,  mint Istenben való épülést a hit által.
\par 5 A parancsolatnak vége pedig a tiszta szívbõl, jó lelkiismeretbõl és igaz hitbõl való szeretet:
\par 6 Melyektõl némelyek eltévelyedvén, hiábavaló beszédre hajlottak:
\par 7 Kik törvénytanítók akarván lenni, nem értik, sem a miket beszélnek, sem a miket erõsítgetnek.
\par 8 Tudjuk pedig, hogy a törvény jó, ha valaki törvényszerûen él vele,
\par 9 Tudván azt, hogy a törvény nem az igazért van, hanem a törvénytaposókért és engedetlenekért, az istentelenekért és bûnösökért, a latrokért és fertelmesekért, az atya- és anyagyilkosokért, emberölõkért.
\par 10 Paráznákért, férfifertõztetõkért, emberrablókért, hazugokért, hamisan esküvõkért, és a mi egyéb csak az egészséges tudománynyal ellenkezik,
\par 11 A boldog Isten dicsõségének evangyélioma szerint, mely reám bízatott.
\par 12 És hálát adok annak, a ki engem megerõsített, a Krisztus Jézusnak, a mi Urunknak, hogy engem hívnek ítélt, rendelvén a szolgálatra,
\par 13 Ki elõbb istenkáromló, üldözõ és erõszakoskodó valék: de könyörült rajtam, mert tudatlanul cselekedtem hitetlenségben;
\par 14 Szerfelett megsokasodott pedig a mi Urunknak kegyelme a Krisztus Jézusban való hittel és szeretettel.
\par 15 Igaz beszéd ez és teljes elfogadásra méltó, hogy Krisztus Jézus azért jött e világra, hogy megtartsa a bûnösöket, a kik közül elsõ vagyok én.
\par 16 De azért könyörült rajtam, hogy Jézus Krisztus bennem mutassa meg legelõbb a teljes hosszútûrését, példa gyanánt azokra, akik hiendenek õ benne az örök életre.
\par 17 Az örökkévaló királynak pedig, a halhatatlan, láthatatlan, egyedül bölcs Istennek tisztesség és dicsõség örökkön örökké! Ámen.
\par 18 Ezt a parancsolatot adom néked fiam Timótheus, a rólad való korábbi jövendölések szerint, hogy vitézkedjél azokban ama jó vitézséggel,
\par 19 Megtartván a hitet és jó lelkiismeretet, melyet némelyek elvetvén, a hit dolgában hajótörést szenvedtek;
\par 20 Kik közül való Himenéus és Alexander, kiket átadtam a sátánnak, hogy megtanulják, hogy ne káromkodjanak.

\chapter{2}

\par 1 Intelek azért mindenek elõtt, hogy tartassanak könyörgések, imádságok, esedezések, hálaadások, minden emberekért,
\par 2 Királyokért és minden méltóságban levõkért, hogy csendes és nyugodalmas életet éljünk, teljes istenfélelemmel és tisztességgel.
\par 3 Mert az jó és kedves dolog a mi megtartó Istenünk elõtt,
\par 4 A ki azt akarja, hogy minden ember idvezüljön és az igazság ismeretére eljusson.
\par 5 Mert egy az Isten, egy a közbenjáró is Isten és emberek között, az ember Krisztus Jézus,
\par 6 A ki adta önmagát váltságul mindenekért, mint tanúbizonyság a maga idejében,
\par 7 A mi végett rendeltettem én hirdetõvé és apostollá (igazságot szólok a Krisztusban, nem hazudok), pogányok tanítójává hitben és igazságban.
\par 8 Akarom azért, hogy imádkozzanak a férfiak minden helyen, tiszta kezeket emelvén föl harag és versengés nélkül.
\par 9 Hasonlatosképen az aszonyok tisztességes öltözetben, szemérmetességgel és mértékletességgel ékesítsék magokat; nem hajfonatokkal és aranynyal vagy gyöngyökkel, vagy drága öltözékkel,
\par 10 Hanem, a mint illik az istenfélelmet valló asszonyokhoz, jó cselekedetekkel.
\par 11 Az asszony csendességben tanuljon teljes engedelmességgel.
\par 12 A tanítást pedig nem engedem meg az asszonynak, sem hogy a férfin uralkodjék,  hanem legyen csendességben.
\par 13 Mert Ádám teremtetett elsõnek, azután Éva.
\par 14 És Ádám nem csalattatott meg, hanem az asszony megcsalattatván, bûnbe esett:
\par 15 Mindazonáltal megtartatik a gyermekszüléskor, ha megmaradnak a hitben és szeretetben, és a szent életben mértékletességgel.

\chapter{3}

\par 1 Igaz ez a beszéd: Ha valaki püspökséget kiván, jó dolgot kíván.
\par 2 Szükséges annakokáért, hogy a püspök feddhetetlen legyen, egy feleségû férfiú, józan, mértékletes illedelmes, vendégszeretõ, a  tanításra alkalmatos;
\par 3 Nem borozó, nem verekedõ, nem rút nyereségre vágyó; hanem szelíd, versengéstõl ment, nem pénzsóvárgó;
\par 4 Ki a maga házát jól eligazgatja, gyermekeit engedelmességben tartja, minden tisztességgel;
\par 5 (Mert ha valaki az õ tulajdon házát nem tudja igazgatni, mimódon visel gondot az Isten egyházára?)
\par 6 Ne legyen új ember, nehogy felfuvalkodván, az ördög kárhozatába essék.
\par 7 Szükséges pedig, hogy jó bizonysága is legyen a kívülvalóktól; hogy gyalázatba és az ördög tõribe ne essék.
\par 8 Hasonlóképpen a diakónusok tisztességesek legyenek, nem kétnyelvûek, nem sok borivásba merültek, nem rút nyereségre vágyók;
\par 9 Kiknél megvan a hit titka tiszta lelkiismerettel.
\par 10 És Ezek is elõször megpróbáltassanak, azután szolgáljanak, ha feddhetetlenek.
\par 11 Feleségeik hasonlóképen tisztességesek, nem patvarkodók, józanok, mindenben hívek legyenek.
\par 12 A diakónusok egy feleségû férfiak legyenek, a kik gyermekeiket és tulajdon házaikat jól igazgatják.
\par 13 Mert a kik jól szolgálnak, szép tisztességet szereznek magoknak, és sok bizodalmat a Jézus Krisztusban való hitben.
\par 14 Ezeket írom néked, remélvén, hogy nem sokára hozzád megyek;
\par 15 De ha késném, hogy tudd meg, mimódon kell forgolódni az Isten házában, mely az élõ Istennek egyháza, és az igazságnak oszlopa és erõssége.
\par 16 És minden versengés nélkül nagy a kegyességnek eme titka: Isten megjelent testben, megigazíttatott lélekben, megláttatott az angyaloktól, hirdettetett a pogányok közt, hittek benne a világon, felvitetett dicsõségbe.

\chapter{4}

\par 1 A Lélek pedig nyilván mondja, hogy az utolsó idõben némelyek elszakadnak a hittõl, hitetõ lelkekre és gonosz lelkek tanításaira figyelmezvén.
\par 2 Hazug beszédûeknek képmutatása által, kik meg vannak bélyegezve a saját lelkiismeretökben.
\par 3 A kik tiltják a házasságot, sürgetik az eledelektõl való tartózkodást, melyeket Isten teremtett hálaadással való élvezésre a hívõknek és azoknak, a kik megismerték az igazságot.
\par 4 Mert Istennek minden teremtett állata jó, és semmi sem megvetendõ, ha hálaadással élnek azzal;
\par 5 Mert megszenteltetik Istennek ígéje és könyörgés által.
\par 6 Ezeket ha eleikbe adod az atyafiaknak, Krisztus Jézusnak jó szolgája leszel, táplálkozván a hitnek és jó tudománynak beszédeivel, a melyet követtél;
\par 7 A szentségtelen és vénasszonyos meséket pedig eltávoztasd. Hanem gyakorold magadat a kegyességre:
\par 8 Mert a test gyakorlásának kevés haszna van; de a kegyesség mindenre hasznos, meglévén benne a jelenvaló és a jövõ életnek ígérete.
\par 9 Igaz ez a beszéd, és méltó, hogy mindenképpen elfogadjuk.
\par 10 Mert azért fáradunk és szenvedünk szidalmakat, mivelhogy reménységünket vetettük az élõ Istenben, a ki minden embernek megtartója, kiváltképen a hívõknek.
\par 11 Ezeket hirdessed és tanítsad.
\par 12 Senki a te ifjúságodat meg ne vesse, hanem légy példa a hívõknek a beszédben, a magaviseletben, a szeretetben, a lélekben, a hitben, a tisztaságban.
\par 13 A míg oda megyek, legyen gondod a felolvasásra, az intésre és a tanításra.
\par 14 Meg ne vesd a kegyelemnek benned való ajándékát, a mely adatott néked prófétálás által, a presbitérium kezeinek reád tevésével.
\par 15 Ezekrõl gondoskodjál, ezeken légy, hogy elõhaladásod nyilvánvaló legyen mindenek elõtt.
\par 16 Gondot viselj magadról és a tudományról; maradj meg azokban; mert ezt cselekedvén, mind magadat megtartod, mind a te hallgatóidat.

\chapter{5}

\par 1 Az idõsb embert ne dorgáld meg, hanem intsed mint atyádat; az ifjabbakat mint atyádfiait;
\par 2 A idõsb asszonyokat mint anyákat; az ifjabbakat mint nõtestvéreidet, teljes tisztasággal.
\par 3 Az özvegyasszonyokat, a kik valóban özvegyek, tiszteld.
\par 4 Ha pedig valamely özvegyasszonynak gyermekei vagy unokái vannak, tanulják meg, hogy elsõ sorban a maguk háza iránt legyenek istenfélõk, és adják meg szüleiknek a viszont tartozást; mert ez szép és kedves dolog Isten elõtt.
\par 5 A valóban özvegy és magára hagyatott asszony pedig reménységét az Istenben veti, és foglalatos a könyörgésekben és imádságokban éjjel és nappal.
\par 6 A bujálkodó pedig élvén megholt.
\par 7 És ezeket hirdesd, hogy feddhetetlenek legyenek.
\par 8 Ha pedig valaki az övéirõl és az õ házanépérõl gondot nem visel: a hitet megtagadta, és rosszabb a hitetlennél.
\par 9 Özvegyasszonyul hatvan éven alul levõ meg ne választassék; egy férj felesége lett légyen,
\par 10 A kinek jó cselekedetekrõl bizonysága van; ha gyermeket nevelt, ha vendéglátó volt, ha a szentek lábait mosta, ha a nyomorultakon segített, ha minden jó cselekedetben foglalatos vala.
\par 11 A fiatalabb özvegyasszonyokat pedig mellõzd; mert ha gerjedeznek Krisztus ellenére, férjhez akarnak menni:
\par 12 Ezeknek ítélete megvan, mivelhogy az elsõ hitet megvetették.
\par 13 Egyszersmind pedig dologkerülõk lévén, megtanulják a házról házra való járogatást; sõt nemcsak dologkerülõk, hanem fecsegõk is és más dolgába avatkozók, olyanokat szólván, a miket nem kellene.
\par 14 Akarom tehát, hogy a fiatalabbak férjhez menjenek, gyermekeket szûljenek, háztartást vigyenek, és semminemû alkalmat se adjanak az ellenségnek a szidalmazásra.
\par 15 Mert némelyek már elhajlottak a Sátánhoz.
\par 16 Ha valamely hívõ férfinak vagy nõnek vannak özvegyei, segítse azokat, és ne terheltessék meg a gyülekezet; hogy azokat segíthesse, a kik valóban özvegyek.
\par 17 A jól forgolódó presbiterek kettõs tisztességre méltattassanak, fõképen a kik a beszédben és tanításban fáradoznak.
\par 18 Mert azt mondja az Írás: A nyomtató ökörnek ne kösd be a száját; és: Méltó a munkás  a maga jutalmára.
\par 19 Presbiter ellen vádat ne fogadj el, hanem csak két vagy három tanúbizonyságra.
\par 20 A vétkeseket mindenek elõtt fedd meg, hogy a többiek is megfélemljenek.
\par 21 Kérve kérlek az Istenre és a Krisztus Jézusra és a kiválasztott angyalokra, hogy ezeket tartsd meg elõítélet nélkül, semmit sem cselekedvén részrehajlásból.
\par 22 A kézrátevést el ne hirtelenkedd, se ne légy részes a más bûneiben; tenmagadat tisztán tartsd.
\par 23 Ne légy tovább vízivó, hanem élj egy kevés borral, gyomrodra és gyakori gyengélkedésedre való tekintetbõl.
\par 24 Némely embereknek a bûnei nyilvánvalóak, elõttök mennek az ítéletre; némelyeket pedig hátul követnek is.
\par 25 Hasonlóképen a jó cselekedetek is nyilvánvalók; és a melyek másképen vannak, azok sem titkolhatók el.

\chapter{6}

\par 1 A kik iga alatt vannak mint szolgák, az õ uraikat minden tisztességre méltóknak tekintsék, hogy Isten neve és a tudomány ne káromoltassék.
\par 2 A kiknek pedig hívõ uraik vannak, azokat meg ne vessék, mivelhogy atyafiak; hanem annál inkább szolgáljanak, mivelhogy hívõk és szeretettek, kik a jótevésben buzgólkodnak. Ezekre taníts és ints.
\par 3 Ha valaki másképen tanít, és nem követi a mi Urunk Jézus Krisztus egészséges beszédeit és a kegyesség szerint való tudományt,
\par 4 Az felfuvalkodott, a ki semmit sem ért, hanem vitatkozásokban és szóharczokban szenved, a melyekbõl származik irígység, viszálykodás, káromlások, rosszakaratú gyanúsítások,
\par 5 Megbomlott elméjû és az igazságtól megfosztott embereknek hiábavaló torzsalkodásai a kik az istenfélelmet nyerekedésnek tekintik. Azoktól, akik ilyenek, eltávozzál.
\par 6 De valóban nagy nyereség az Istenfélelem, megelégedéssel;
\par 7 Mert semmit sem hoztunk a világra, világos, hogy ki sem vihetünk semmit;
\par 8 De ha van élelmünk és ruházatunk, elégedjünk meg vele.
\par 9 A kik pedig meg akarnak gazdagodni, kísértetbe meg tõrbe és sok esztelen és káros kívánságba esnek, melyek az embereket veszedelembe és romlásba merítik.
\par 10 Mert minden rossznak gyökere a pénz szerelme: mely után sóvárogván némelyek eltévelyedtek a hittõl, és magokat általszegezték sok fájdalommal.
\par 11 De te, óh Istennek embere, ezeket kerüld; hanem kövessed az igazságot, az istenfélelmet, a hitet, a szeretetet, s békességes tûrést, a szelídséget.
\par 12 Harczold meg a hitnek szép harczát, nyerd el az örök életet, a melyre hívattattál, és szép vallástétellel vallást tettél sok bizonyság elõtt.
\par 13 Meghagyom néked Isten elõtt, a ki megelevenít mindeneket, és Krisztus Jézus elõtt, a ki bizonyságot tett Ponczius Pilátus alatt ama szép vallástétellel,
\par 14 Hogy tartsd meg a parancsolatot mocsoktalanul, feddhetetlenül a mi Urunk Jézus Krisztus megjelenéséig,
\par 15 A mit a maga idejében megmutat ama boldog és egyedül hatalmas, a királyoknak Királya és az uraknak Ura,
\par 16 Kié egyedül a halhatatlanság, a ki hozzáférhetetlen világosságban lakozik; a kit az emberek közül senki nem látott, sem nem láthat: a kinek tisztesség és örökké való hatalom. Ámen.
\par 17 Azoknak, a kik gazdagok e világon, mondd meg, hogy ne fuvalkodjanak fel, se ne reménykedjenek a bizonytalan gazdagságban, hanem az élõ Istenben, a ki bõségesen megad nékünk mindent a mi tápláltatásunkra;
\par 18 Hogy jót tegyenek, legyenek gazdagok a jó cselekedetekben, legyenek szíves adakozók, közlõk,
\par 19 Kincset gyûjtvén magoknak jó alapul a jövõre, hogy elnyerjék az örök  életet.
\par 20 Óh Timótheus, õrizd meg a mi rád van bízva, elfordulván a szentségtelen üres beszédektõl és a hamis nevû ismeretnek ellenvetéseitõl;
\par 21 A melylyel némelyek kevélykedvén, a hit mellõl eltévelyedtek. Kegyelem veled! Ámen.


\end{document}