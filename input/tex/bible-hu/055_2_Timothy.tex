\begin{document}

\title{Timóteusnak írt II. levél}


\chapter{1}

\par 1 Pál, Jézus Krisztusnak apostola, Isten akaratából, a Krisztus Jézusban való életnek ígérete szerint,
\par 2 Timótheusnak, az én szeretett fiamnak: kegyelem, irgalmasság, békesség az Atya Istentõl és Krisztus Jézustól, a mi Urunktól.
\par 3 Hálát adok az Istennek, a kinek szolgálok õseimtõl fogva tiszta lelkiismerettel, hogy szüntelen gondolok reád könyörgéseimben éjjel és nappal,
\par 4 Kívánván téged látni, megemlékezvén a te könyhullatásaidról, hogy örömmel teljesedjem be;
\par 5 Eszembe jutván a benned levõ, képmutatás nélkül való hit, a mely lakozott elõször a te nagyanyádban Loisban és anyádban Eunikában; meg vagyok azonban gyõzõdve, hogy benned is.
\par 6 Minekokáért emlékeztetlek téged, hogy gerjeszd fel az Isten kegyelmi ajándékát, a mely benned van az én kezeimnek rád tétele által.
\par 7 Mert nem félelemnek lelkét adott nékünk az Isten; hanem erõnek és szeretetnek és józanságnak lelkét.
\par 8 Ne szégyeneld hát a mi Urunk bizonyságtételét, se engem az õ foglyát; hanem együtt szenvedj az evangyéliomért Istennek hatalma szerint.
\par 9 A ki megtartott minket és hívott szent hívással, nem a mi cselekedeteink szerint, hanem az õ saját végezése és kegyelme szerint, mely adatott nékünk Krisztus Jézusban örök idõknek elõtte,
\par 10 Megjelentetett pedig most a mi Megtartónknak, Jézus Krisztusnak megjelenése által, a ki eltörölte a halált,  világosságra hozta pedig az életet és halhatatlanságot az evangyéliom által,
\par 11 Amelyre nézve tétettem én hirdetõvé és apostollá és pogányok tanítójává.
\par 12 Amiért szenvedem ezeket is: de nem szégyenlem; mert tudom, kinek hittem, és bizonyos vagyok benne, hogy õ az én nála letett kincsemet meg tudja õrizni ama napra.
\par 13 Az egészséges beszédeknek példáját megtartsd, a miket én tõlem hallottál, a Krisztus Jézusban való hitben és szeretetben.
\par 14 A rád bízott drága kincset õrizd meg a bennünk lakozó Szent Lélek által.
\par 15 Azt tudod, hogy elfordultak tõlem az Ázsiabeliek mind, kik közül való Figellus és Hermogénes.
\par 16 Az Úr legyen irgalmas az Onesiforus házanépének: mert gyakorta megvidámított engem, és az én bilincsemet nem szégyenlette;
\par 17 Sõt mikor Rómában volt, buzgón keresett engem, meg is talált.
\par 18 Az Úr engedje meg néki, hogy találjon irgalmasságot az Úrnál ama napon. És hogy mily nagy szolgálatot tett Efézusban, te jobban tudod.

\chapter{2}

\par 1 Te annakokáért, én fiam, erõsödjél meg a Krisztus Jézusban való kegyelemben;
\par 2 És a miket tõlem hallottál sok bizonyság által, azokat bízzad hív emberekre, a kik másoknak a tanítására is alkalmasak lesznek.
\par 3 Te azért a munkának terhét hordozzad, mint a Jézus Krisztus jó vitéze.
\par 4 Egy harczos sem elegyedik bele az élet dolgaiba; hogy tessék annak, a ki õt harczossá avatta.
\par 5 Ha pedig küzd is valaki, nem koronáztatik meg, ha nem szabályszerûen küzd.
\par 6 A ki munkálkodik, a földmívelõnek kell a gyümölcsökben elõször részesülnie.
\par 7 Értsd meg a mit mondok; adjon azért az Úr néked belátást mindenekben.
\par 8 Emlékezzél meg, hogy Jézus Krisztus feltámadott a halálból, ki a Dávid magvából való az én evangyéliomom szerint:
\par 9 A melyért, mint egy gonosztevõ, szenvedek mind a fogságig; de az Istennek beszéde nincs bilincsbe verve.
\par 10 Annakokáért mindent elszenvedek a választottakért, hogy õk is elnyerjék a Krisztus Jézusban való idvességet örök dicsõséggel egyben.
\par 11 Igaz beszéd ez. Mert ha vele együtt meghaltunk, vele együtt fogunk élni is.
\par 12 Ha tûrünk, vele együtt fogunk uralkodni is: ha megtagadjuk, õ is megtagad minket;
\par 13 Ha hitetlenkedünk, õ hû marad: õ magát meg nem tagadhatja.
\par 14 Ezekre emlékeztesd, kérvén kérve õket az Úr színe elõtt, hogy ne vitatkozzanak haszontalanul, a hallgatóknak romlására.
\par 15 Igyekezzél, hogy Isten elõtt becsületesen megállj, mint oly munkás, a ki szégyent nem vall, a ki helyesen hasogatja az igazságnak beszédét.
\par 16 A szentségtelen üres lármákat pedig kerüld, mert mind nagyobb istentelenségre növekednek.
\par 17 És az õ beszédök mint a rákfekély terjed; közülök való Himenéus és Filétus.
\par 18 A kik az igazság mellõl eltévelyedtek, azt mondván, hogy a feltámadás már megtörtént, és feldúlják némelyeknek a hitét.
\par 19 Mindazáltal megáll az Istennek erõs fundamentoma, melynek pecséte ez: Ismeri az Úr az övéit; és: Álljon el a hamisságtól minden, a ki Krisztus nevét vallja.
\par 20 Nagy házban pedig nemcsak arany- és ezüstedények vannak, hanem fából és cserépbõl valók is; és azok közül némelyek tisztességre, némelyek pedig gyalázatra valók.
\par 21 Ha tehát valaki magát ezektõl tisztán tartja, tisztességre való edény lesz, megszentelt, és hasznos a gazdájának, minden jó cselekedetre alkalmas.
\par 22 Az ifjúkori kivánságokat pedig kerüld; hanem kövessed az igazságot, a hitet, a szeretetet, a békességet azokkal egyetembe, a kik segítségül hívják az Urat tiszta szívbõl.
\par 23 A botor és gyermekes vitatkozásokat pedig kerüld, tudván, hogy azok háborúságokat szülnek.
\par 24 Az Úr szolgájának pedig nem kell torzsalkodni, hanem legyen mindenkihez nyájas, tanításra alkalmas, türelmes.
\par 25 A ki szelíden fenyíti az ellenszegülõket; ha talán adna nékik az Isten megtérést az igazság megismerésére,
\par 26 És felocsudnának az ördög tõrébõl, foglyokká tétetvén az Úr szolgája által az Isten akaratára.

\chapter{3}

\par 1 Azt pedig tudd meg, hogy az utolsó napokban nehéz idõk állanak be.
\par 2 Mert lesznek az emberek magukat szeretõk, pénzsóvárgók, kérkedõk, kevélyek, káromkodók, szüleik iránt engedetlenek, háládatlanok, tisztátalanok,
\par 3 Szeretet nélkül valók, kérlelhetetlenek, rágalmazók, mértékletlenek, kegyetlenek, a jónak nem kedvelõi.
\par 4 Árulók, vakmerõk, felfuvalkodottak, inkább a gyönyörnek, mint Istennek szeretõi.
\par 5 Kiknél megvan a kegyességnek látszata, de megtagadják annak erejét. És ezeket kerüld.
\par 6 Mert ezek közül valók azok, a kik betolakodnak a házakba, és foglyul ejtik a bûnökkel megterhelt és sokféle kívánságoktól ûzött asszonykákat,
\par 7 Kik mindenkor tanulnak, de az igazság megismerésére soha el nem juthatnak.
\par 8 Miképen pedig Jánnes és Jámbres ellene állottak Mózesnek, akképen ezek is ellene állanak az igazságnak; megromlott elméjû, a hitre nézve nem becsületes emberek.
\par 9 De többre nem mennek: mert esztelenségök nyilvánvaló lesz mindenek elõtt, a miképen amazoké is az lett.
\par 10 Te pedig követted az én tanításomat, életmódomat, szándékomat, hitemet, hosszútûrésemet, szeretetemet, türelmemet,
\par 11 Üldöztetéseimet, szenvedéseimet, a melyek rajtam estek Antiókhiában, Ikóniumban, Listrában: minémû üldöztetéseket szenvedtem! de mindezekbõl megszabadított engem az Úr.
\par 12 De mindazok is, a kik kegyesen akarnak élni Krisztus Jézusban, üldöztetni fognak.
\par 13 A gonosz emberek pedig és az ámítók nevekednek a rosszaságban, eltévelyítvén és eltévelyedvén.
\par 14 De te maradj meg azokban, a miket tanultál és a mik reád bízattak, tudván kitõl tanultad,
\par 15 És hogy gyermekségedtõl fogva tudod a szent írásokat, melyek téged bölcscsé tehetnek az idvességre a Krisztus Jézusban való hit által.
\par 16 A teljes írás Istentõl ihletett és hasznos a tanításra, a feddésre, a megjobbításra, az igazságban való nevelésre,
\par 17 Hogy tökéletes legyen az Isten embere, minden jó cselekedetre felkészített.

\chapter{4}

\par 1 Kérlek azért az Isten és Krisztus Jézus színe elõtt, a ki ítélni fog élõket és holtakat az õ eljövetelekor és az õ országában.
\par 2 Hirdesd az ígét, állj elõ vele alkalmatos, alkalmatlan idõben, ints, feddj, buzdíts teljes béketûréssel és tanítással.
\par 3 Mert lesz idõ, mikor az egészséges tudományt el nem szenvedik, hanem a saját kívánságaik szerint gyûjtenek magoknak tanítókat, mert viszket a fülök;
\par 4 És az igazságtól elfordítják az õ fülöket, de a mesékhez oda fordulnak.
\par 5 De te józan légy mindenekben, szenvedj, az evangyélista munkáját cselekedd, szolgálatodat teljesen betöltsd.
\par 6 Mert én immár megáldoztatom, és az én elköltözésem ideje beállott.
\par 7 Ama nemes harczot megharczoltam, futásomat elvégeztem, a hitet megtartottam:
\par 8 Végezetre eltétetett nékem az igazság koronája, melyet megád nékem az Úr ama napon, az igaz Bíró; nemcsak  nékem pedig, hanem mindazoknak is, a kik vágyva várják az õ megjelenését.
\par 9 Igyekezzél hozzám jõni hamar.
\par 10 Mert Démás engem elhagyott, e jelen való világhoz ragaszkodván, és elment Thessalónikába: Kresczens Galátziába, Titus Dalmátziába.
\par 11 Egyedül Lukács van velem. Márkust magadhoz vévén, hozd magaddal: mert nekem alkalmas a szolgálatra.
\par 12 Tikhikust pedig Efézusba küldöttem.
\par 13 A felsõruhámat, melyet Troásban Kárpusnál hagytam, jöttödben hozd el, a könyveket is, kiváltképen a hártyákat.
\par 14 Az érczmíves Sándor sok bajt szerzett nékem: fizessen meg az Úr néki cselekedetei szerint.
\par 15 Tõle te is õrizkedjél, mert szerfelett ellenállott a mi beszédinknek.
\par 16 Elsõ védekezésem alkalmával senki sem volt mellettem, sõt mindnyájan elhagytak; ne számíttassék be nékik.
\par 17 De az Úr mellettem állott, és megerõsített engem; hogy teljesen bevégezzem az igehirdetést, és hallják meg azt az összes pogányok: és megszabadultam az oroszlán szájából.
\par 18 És megszabadít engem az Úr minden gonosz cselekedettõl, és megtart az õ mennyei országára; a kinek dicsõség örökkön örökké! Ámen.
\par 19 Köszöntsed Priszkát és Akvilát, és az Onesifórus háznépét.
\par 20 Erástus Korinthusban maradt; Trófimust pedig Milétumban hagytam betegen.
\par 21 Igyekezzél tél elõtt eljõni. Köszönt téged Eubulus és Pudens és Linus és Klaudia, és mind az atyafiak.
\par 22 Az Úr Jézus Krisztus a te lelkeddel. Kegyelem veletek! Ámen.


\end{document}