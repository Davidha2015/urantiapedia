\begin{document}

\title{Titus}


\chapter{1}

\par 1 Pál, Istennek szolgája, Jézus Krisztusnak pedig apostola, az Isten választottainak hite és a kegyesség szerint való igazságnak megismerése szerint,
\par 2 Az örök élet reménységére, melyet megígért az igazmondó Isten örök idõknek elõtte,
\par 3 Megjelentette pedig a maga idejében az õ beszédét a prédikálás által, a mely reám bízatott  a mi megtartó Istenünknek parancsolata szerint; Titusnak, a közös hit szerint való igaz fiamnak:
\par 4 Kegyelem, irgalmasság és békesség az Atya Istentõl és az Úr Jézus Krisztustól, a mi Megtartónktól.
\par 5 A végett hagytalak téged Krétában, hogy a hátramaradt dolgokat hozd rendbe, és rendelj városonként presbitereket, a miképen én néked meghagytam;
\par 6 Ha van feddhetetlen, egy feleségû férfiú, a kinek hívõ, nem kicsapongással vádolt avagy engedetlen gyermekei vannak.
\par 7 Mert szükséges, hogy a püspök feddhetetlen legyen, mint Isten sáfára; nem akaratos, nem haragos, nem részeges, nem verekedõ, nem rút nyerészkedõ;
\par 8 Hanem vendégszeretõ, jónak kedvelõje, mértékletes, igaz, tiszta, maga tûrtetõ;
\par 9 A ki a tudomány szerint való igaz beszédhez tartja magát, hogy inthessen az egészséges tudománnyal és meggyõzhesse az ellenkezõket.
\par 10 Mert van sok engedetlen, hiába való beszédû és csaló, kiváltképen a körülmetélkedésbõl valók,
\par 11 A kiknek be kell dugni a szájokat; a kik egész házakat feldúlnak, tanítván rút nyereség okáért, a miket nem kellene.
\par 12 Azt mondta valaki közülök, az õ saját prófétájok: A krétaiak mindig hazugok, gonosz vadak, rest hasak.
\par 13 E bizonyság igaz: annakokáért fedd õket kímélés nélkül, hogy a hitben épek legyenek,
\par 14 Nem ügyelvén zsidó mesékre, és az igazságot megvetõ emberek parancsolataira.
\par 15 Minden tiszta a tisztáknak: de a megfertõztetetteknek és hitetleneknek semmi sem tiszta; hanem megfertõztetett azoknak mind elméjök, mind lelkiismeretök.
\par 16 Vallják, hogy Istent ismerik, de cselekedeteikkel tagadják, mivelhogy útálatosak és hitetlenek és minden jó cselekedetre méltatlanok.

\chapter{2}

\par 1 Te pedig azokat szóljad, a mik az egészséges tudományhoz illenek.
\par 2 Hogy a vén emberek józanok legyenek, tisztességesek, mértékletesek; a hitben, szeretetben, tûrésben épek.
\par 3 Hasonlóképen a vén asszonyok szentekhez illõ magaviseletûek legyenek, nem patvarkodók, sem sok borivás rabjai, jóra oktatók;
\par 4 Hogy megokosítsák az ifjú asszonyokat, hogy férjöket és magzataikat szeressék,
\par 5 Legyenek mértékletesek, tiszták, háziasak, jók, férjöknek engedelmesek, hogy az Isten beszéde ne  káromoltassék.
\par 6 Az ifjakat hasonlóképen intsed, hogy legyenek mértékletesek:
\par 7 Mindenben tenmagadat adván példaképül a jó cselekedetekben; a tudományban romlatlanságot, méltóságot mutatván,
\par 8 Egészséges, feddhetetlen beszédet; hogy az ellenfél megszégyenüljön, semmi gonoszt sem tudván rólatok mondani.
\par 9 A szolgákat intsed, hogy az õ uraiknak engedelmeskedjenek, mindenben kedvöket keressék, ne ellenkezzenek,
\par 10 Ne tolvajkodjanak, hanem teljes jó hûséget tanusítsanak; hogy a mi megtartó Istenünknek tudományát ékesítsék mindenben.
\par 11 Mert megjelent az Isten idvezítõ kegyelme minden embernek,
\par 12 A mely arra tanít minket, hogy megtagadván a hitetlenséget és a világi kivánságokat, mértékletesen, igazán és szentül éljünk a jelenvaló világon:
\par 13 Várván ama boldog reménységet és a nagy Istennek és megtartó Jézus Krisztusnak dicsõsége megjelenését;
\par 14 A ki önmagát adta mi érettünk, hogy megváltson minket minden hamisságtól, és tisztítson önmagának kiváltképen való népet, jó cselekedetekre igyekezõt.
\par 15 Ezeket szóljad, és ints és feddj teljes méltósággal; senki téged meg ne vessen.

\chapter{3}

\par 1 Emlékeztessed õket, hogy a fejedelemségeknek és hatalmasságoknak engedelmeskedjenek, hódoljanak, minden jó cselekedetre készek legyenek,
\par 2 Senkit ne szidalmazzanak, ne veszekedjenek, gyöngédek legyenek, teljes szelídséget tanusítván minden ember iránt.
\par 3 Mert régenten mi is esztelenek, engedetlenek, tévelygõk, különbözõ kívánságoknak és gyönyöröknek szolgái, gonoszságban és irígységben élõk, gyûlölségesek, egymást gyûlölõk valánk.
\par 4 De mikor a mi megtartó Istenünknek jóvolta és az emberekhez való szeretete megjelent,
\par 5 Nem az igazságnak cselekedeteibõl, a melyeket mi cselekedtünk, hanem az õ irgalmasságából tartott meg minket az újjászületésnek fürdõje és a Szent Lélek megújítása által,
\par 6 Melyet kitöltött reánk bõséggel a mi megtartó Jézus Krisztusunk által;
\par 7 Hogy az õ kegyelmébõl megigazulván, örökösök legyünk az örök élet reménysége szerint.
\par 8 Igaz ez a beszéd, és akarom, hogy ezeket erõsítsed, hogy igyekezzenek jó cselekedetekkel elõljárni azok, a kik Istenben hívõkké lettek. Ezek jók és hasznosak ez embereknek;
\par 9 A balgatag vitatkozásokat azonban és a nemzetségekrõl való tudakozásokat, és a civakodást és a törvény felõl való harczokat kerüld; mert haszontalanok és hiábavalók.
\par 10 Az eretnek embert egy vagy két intés után kerüld;
\par 11 Tudván, hogy az ilyen romlott, és vétkezik, önmaga is kárhoztatván magát.
\par 12 Mikor Ártemást vagy Tikhikust hozzád küldöm, siess hozzám jõni Nikápolyba; mert elhatároztam, hogy ott töltöm a telet.
\par 13 A törvénytudó Zénást és Apollóst gondosan indítsd útnak, hogy semmiben se legyen fogyatkozásuk.
\par 14 Tanulják meg pedig a mieink is, hogy jó cselekedetekkel járjanak elõl a szükséges hasznokra, hogy ne legyenek gyümölcstelenek.
\par 15 Köszöntenek téged a velem levõk mindnyájan. Köszöntsd azokat, a kik szeretnek minket hitben. Kegyelem mindnyájatokkal! Ámen.


\end{document}