\begin{document}

\title{Hebrews}


\chapter{1}

\par 1 Minekutána az Isten sok rendben és sokféleképen szólott hajdan az atyáknak a próféták által, ez utolsó idõkben szólott nékünk Fia által,
\par 2 A kit tett mindennek örökösévé, a ki által a világot is teremtette,
\par 3 A ki az õ dicsõségének visszatükrözõdése, és az õ valóságának képmása, a ki hatalma szavával fentartja a mindenséget, a ki minket bûneinktõl megtisztítván,  üle a Felségnek jobbjára a magasságban,
\par 4 Annyival kiválóbb lévén az angyaloknál, a mennyivel különb nevet örökölt azoknál.
\par 5 Mert kinek mondotta valaha az angyalok közül: Én Fiam vagy te, én ma szûltelek téged? és ismét: Én leszek  néki Atyja és õ lesz nékem Fiam?
\par 6 Viszont mikor behozza az õ elsõszülöttét a világba, így szól: És imádják õt az Istennek minden angyalai.
\par 7 És bár az angyalokról így szól: Ki az õ angyalait szelekké teszi és az õ szolgáit tûz lángjává,
\par 8 Ámde a Fiúról így: A te királyi széked óh Isten örökkön örökké. Igazságnak pálczája a te országodnak pálczája.
\par 9 Szeretted az igazságot és gyûlölted a hamisságot: annakokáért felkent téged az Isten, a te Istened, örömnek olajával a te társaid felett.
\par 10 És: Te Uram kezdetben alapítottad a földet és a te kezeidnek mûvei az egek;
\par 11 Azok elvesznek, de te megmaradsz, és mindazok, mint a ruha megavulnak.
\par 12 És palástként összehajtod azokat és elváltoznak, te pedig ugyanaz vagy és a te esztendeid el nem fogynak.
\par 13 Melyik angyalnak mondotta pedig valaha: Ülj az én jobbkezem felõl, míglen ellenségeidet lábaidnak zsámolyává teszem?
\par 14 Avagy nem szolgáló lelkek-é mindazok, elküldve szolgálatra azokért, a kik örökölni fogják az idvességet?

\chapter{2}

\par 1 Annakokáért annál is inkább szükséges nékünk a hallottakra figyelmeznünk, hogy valaha el ne sodortassunk.
\par 2 Mert ha az angyaloktól hirdetett beszéd erõs volt és minden bûn és engedetlenség elvette  igazságos büntetését:
\par 3 Mimódon menekedünk meg mi, hogyha nem törõdünk ily nagy idvességgel? a melyet, miután kezdetben hirdetett az Úr, azok, a kik hallották, biztosítottak számunkra,
\par 4 Velök együtt bizonyságot tevén arról az Isten, jelekkel meg csodákkal és sokféle erõkkel s a Szent Léleknek közléseivel  az õ akarata szerint.
\par 5 Mert nem angyaloknak vetette alá a jövendõ világot, a melyrõl szólunk.
\par 6 Sõt bizonyságot tett valahol valaki, mondván: Micsoda az ember, hogy megemlékezel õ róla, avagy az embernek fia, hogy gondod van reá?
\par 7 Kisebbé tetted õt rövid idõre az angyaloknál, dicsõséggel és tisztességgel megkoronáztad õt és úrrá tetted kezeid munkáin,
\par 8 Mindent lábai alá vetettél. Mert azzal, hogy néki mindent alávetett, semmit sem hagyott alávetetlenül: de most még nem látjuk, hogy néki minden alávettetett.
\par 9 Azt azonban látjuk, hogy Jézus, a ki egy kevés idõre kisebbé tétetett az angyaloknál, a halál elszenvedéséért dicsõséggel és tisztességgel koronáztatott meg, hogy az Isten kegyelmébõl mindenkiért megízlelje a  halált.
\par 10 Mert illendõ vala, hogy a kiért minden és a ki által minden, sok fiakat vezérelvén dicsõségre, az õ idvességök fejedelmét szenvedések által tegye tökéletessé.
\par 11 Mert a megszentelõ és a megszenteltek egytõl valók mindnyájan, a mely oknál fogva nem szégyenli õket atyjafiainak hívni,
\par 12 Mondván: Hirdetem a te nevedet az én atyámfiainak, az anyaszentegyháznak közepette dícséretet mondok néked.
\par 13 És ismét: Én õ benne bízom; és ismét:  Ímhol vagyok én és a gyermekek, a kiket az Isten nékem adott.
\par 14 Mivel tehát a gyermekek testbõl és vérbõl valók, õ is hasonlatosképen részese lett azoknak, hogy a  halál által megsemmisítse azt, a kinek hatalma van a halálon, tudniillik az ördögöt,
\par 15 És megszabadítsa azokat, a kik a haláltól való félelem miatt teljes életökben rabok valának.
\par 16 Mert nyilván nem angyalokat karolt fel, hanem az Ábrahám magvát karolta fel.
\par 17 Annakokáért mindenestõl fogva hasonlatosnak kellett lennie az atyafiakhoz, hogy könyörülõ legyen és hív fõpap az Isten elõtt való dolgokban, hogy engesztelést szerezzen a nép bûneiért.
\par 18 Mert a mennyiben szenvedett, õ maga is megkísértetvén, segíthet azokon, a kik megkísértetnek.

\chapter{3}

\par 1 Annakokáért szent atyafiak, mennyei elhívásnak részesei, figyelmezzetek, a mi vallásunknak apostolára és fõpapjára, Krisztus Jézusra,
\par 2 A ki hû ahhoz, a ki õt rendelte, valamint  Mózes is az õ egész házában.
\par 3 Mert ez nagyobb dicsõségre méltattatott, mint Mózes, a mennyiben a ház építõjének nagyobb a tisztessége, mint a háznak.
\par 4 Mert minden háznak van építõje, a ki pedig mindent elkészített, az Isten az.
\par 5 Mózes is hû volt ugyan az õ egész házában, mint szolga, a hirdetendõknek bizonyságára,
\par 6 Krisztus ellenben mint Fiú a maga háza felett, a kinek háza mi vagyunk, ha a bizodalmat és a reménységnek dicsekedését mind végig erõsen megtartjuk.
\par 7 Annakokáéért a mint a Szent Lélek mondja: Ma, ha az õ szavát halljátok,
\par 8 Meg ne keményítsétek a ti szíveteket, mint az elkeseredéskor, a kísértés ama napján a pusztában,
\par 9 A hol a ti atyáitok próbára tevéssel megkísértének engem és látták az én cselekedeteimet negyven esztendeig.
\par 10 Azért megharagudtam arra a nemzetségre és mondám: mindig tévelyegnek szivökben; õk pedig nem ismerték meg az én útaimat.
\par 11 Úgy hogy megesküdtem haragomban, hogy nem fognak bemenni az én nyugodalmamba.
\par 12 Vigyázzatok atyámfiai, hogy valaha ne legyen bármelyikõtöknek hitetlen gonosz szíve, hogy az élõ Istentõl elszakadjon;
\par 13 Hanem intsétek egymást minden napon, míg tart a ma, hogy egyikõtök se keményíttessék meg a bûnnek csalárdsága által:
\par 14 Mert részeseivé lettünk Krisztusnak, ha ugyan az elkezdett bizodalmat mindvégig erõsen megtartjuk.
\par 15 E mondás szerint: Ma, ha az õ szavát halljátok, meg ne keményítsétek a ti szíveiteket, mint az elkeseredéskor.
\par 16 Mert kik keseredtek el, mikor ezt hallák? Nemde mindazok, a kik kijövének Égyiptomból Mózes által?
\par 17 Kikre haragudott vala pedig meg negyven esztendeig? avagy nem azokra-é, a kik vétkeztek, a kiknek testei elhullottak a pusztában?
\par 18 Kiknek esküdött pedig meg, hogy nem mennek be az õ nyugodalmába, hanemha az engedetleneknek?
\par 19 Látjuk is, hogy nem mehettek be hitetlenség miatt.

\chapter{4}

\par 1 Óvakodjunk tehát, hogy mivel megvan az õ nyugodalmába való bemenetel ígérete, valaki közületek fogyatkozásban levõnek ne láttassék.
\par 2 Mert nékünk is hirdettetett az evangyéliom, miképen azoknak: de nem használt nékik a hallott beszéd, mivel nem párosították hittel azok, a kik hallották.
\par 3 Mert mi, hívõk, bemegyünk a nyugodalomba, miképen megmondotta: A mint megesküdtem az én haragomban, nem fognak bemenni az én nyugodalmamba; jóllehet munkáit a világ megalapításától kezdve bevégezte.
\par 4 Mert valahol a hetedik napról ekképen szólott: És megnyugovék Isten a hetedik napon minden õ cselekedeteitõl.
\par 5 És ugyanabban ismét: Nem mennek be az én nyugodalmamba.
\par 6 Mivelhogy annakokáért áll az, hogy némelyek bemennek abba, és a kiknek elõször hirdettetett az evangyéliom, nem mentek be engedetlenség miatt:
\par 7 Ismét határoz egy napot: Ma, szólván Dávid által annnyi idõ multán, a mint elõbb mondva volt. Ma, ha az õ szavát halljátok, meg ne keményítsétek a ti szíveiteket.
\par 8 Mert ha õket Józsué nyugodalomba helyezte volna, nem szólana azok után más napról.
\par 9 Annakokáért megvan a szombatja az Isten népének.
\par 10 Mert a ki bement az õ nyugodalmába, az maga is megnyugodott cselekedeteitõl, a miképen Isten is a magáéitól,
\par 11 Igyekezzünk tehát bemenni abba a nyugodalomba, hogy valaki a hitetlenségnek ugyanazon példájába ne essék.
\par 12 Mert az Istennek beszéde élõ és ható, és élesebb minden kétélû fegyvernél, és elhat a szívnek és léleknek, az ízeknek és a velõknek megoszlásáig, és megítéli a gondolatokat és a szívnek indulatait.
\par 13 És nincsen oly teretmény, a mely nyilvánvaló nem volna elõtte, sõt mindenek meztelenek és leplezetlenek annak szemei elõtt, a kirõl mi beszélünk.
\par 14 Lévén annakokáért nagy fõpapunk, a ki áthatolt az egeken, Jézus, az Istennek Fia,  ragaszkodjunk vallásunkhoz.
\par 15 Mert nem oly fõpapunk van, a ki nem tudna megindulni gyarlóságainkon, hanem a ki megkísértetett mindenekben, hozzánk hasonlóan, kivéve a  bûnt.
\par 16 Járuljunk azért bizodalommal a kegyelem királyi székéhez, hogy irgalmasságot nyerjünk és kegyelmet találjunk, alkalmas idõben való segítségül.

\chapter{5}

\par 1 Mert minden fõpap emberek közül választatván, emberekért rendeltetik az Isten elõtt való dolgokban, hogy ajándékokat és áldozatokat vigyen a bûnökért,
\par 2 A ki képes együttérezni a tudatlanokkal és tévelygõkkel, mivelhogy maga is körül van véve gyarlósággal.
\par 3 És ezért köteles, miképen a népért, azonképen önmagáért is áldozni a bûnökért.
\par 4 És senki sem veszi magának e tisztességet, hanem a kit Isten hív el, miként Áront is.
\par 5 Hasonnlóképen Krisztus sem maga dicsõítette meg magát azzal, hogy fõpap lett, hanem az, a ki így szólott hozzá: Én Fiam vagy te, ma szûltelek téged.
\par 6 Miképen másutt is mondja: Te örökké való pap vagy, Melkisédek rendje szerint.
\par 7 Ki az õ testének napjaiban könyörgésekkel és esedezésekkel, erõs kiáltás és könyhullatás közben járult ahhoz, a ki képes megszabadítani õt a halálból, és meghallgattatott  az õ istenfélelméért,
\par 8 Ámbár Fiú, megtanulta azokból, a miket szenvedett, az engedelmességet;
\par 9 És tökéletességre jutván, örök idvesség szerzõje lett mindazokra nézve, a kik neki engedelmeskednek,
\par 10 Neveztetvén az Istentõl Melkisédek rendje szerint való fõpapnak.
\par 11 A kirõl nékünk sok és nehezen megmagyarázható mondani valónk van, mivel restek lettetek a hallásra.
\par 12 Mert noha ez idõ szerint tanítóknak kellene lennetek, ismét arra van szükségetek, hogy az Isten beszédeinek kezdõ elemeire tanítson valaki titeket; és olyanok lettetek, a kiknek tejre van szükségetek és nem kemény eledelre.
\par 13 Mert mindaz, a ki tejjel él, járatlan az igazságnak beszédiben, mivelhogy kiskorú:
\par 14 Az érettkorúaknak pedig kemény eledel való, mint a kiknek mivoltuknál fogva gyakorlottak az érzékeik a jó és rossz között való különbségtételre.

\chapter{6}

\par 1 Annakokáért elhagyván a Krisztusról való kezdetleges beszédet, törekedjünk tökéletességre, nem rakosgatván le újra alapját a holt cselekedetekbõl való megtérésnek és az Istenben való hitnek,
\par 2 A mosakodásoknak, tanításnak, kezek rátevésének, holtak feltámadásának és az örök ítéletnek.
\par 3 És ezt megcselekeszszük, ha az Isten megengedi.
\par 4 Mert lehetetlen dolog, hogy a kik egyszer megvilágosíttattak, megízlelvén a mennyei ajándékot, és részeseivé lettek a Szent Léleknek,
\par 5 És megízlelték az Istennek jó beszédét és a jövendõ világnak erõit,
\par 6 És elestek, ismét megújuljanak a megtérésre, mint a kik önmagoknak feszítik meg az Istennek ama Fiát, és meggyalázzák õt.
\par 7 Mert a föld, a mely beiszsza a gyakorta reá hulló esõt és hasznos füvet terem azoknak, a kikért mûveltetik, áldást nyer Istentõl;
\par 8 A mely pedig töviseket és bojtorjánokat terem, megvetett és közel van az átokhoz, annak vége megégetés.
\par 9 De ti felõletek szerelmeseim, ezeknél jobb és idvességesebb dolgokról vagyunk meggyõzõdve, ha így szólunk is.
\par 10 Mert nem igazságtalan az Isten, hogy elfelejtkezzék a ti cselekedeteitekrõl és a szeretetrõl, melyet tanúsítottatok az õ neve iránt, mint a kik szolgáltatok és szolgáltok a szenteknek.
\par 11 Kívánjuk pedig, hogy közületek kiki ugyanazon buzgóságot tanusítsa a reménységnek bizonyossága iránt mindvégiglen.
\par 12 Hogy ne legyetek restek, hanem követõi azoknak, a kik hit és békességes tûrés által öröklik az ígéreteket.
\par 13 Mert az Isten, mikor ígéretet tett Ábrahámnak, mivelhogy nem esküdhetett nagyobbra, önmagára esküdött.
\par 14 Mondván: Bizony megáldván megáldalak téged, és megsokasítván megsokasítalak téged.
\par 15 És ekképen, békességestûrõ lévén, megnyerte az ígéretet.
\par 16 Mert az emberek nagyobbra esküsznek, és nálok minden versengésnek vége megerõsítésül az eskü;
\par 17 Miért is az Isten, kiválóbban megakarván mutatni az ígéret örököseinek az õ végzése változhatatlan voltát, esküvéssel lépett közbe,
\par 18 Hogy két változhatatlan tény által, melyekre nézve lehetetlen, hogy az Isten hazudjon, erõs vígasztalásunk legyen minékünk, mint a kik oda menekültünk, hogy megragadjuk az elõttünk levõ reménységet,
\par 19 Mely lelkünknek mintegy bátorságos és erõs horgonya és beljebb hatol a kárpítnál,
\par 20 A hová útnyitóul bement érettünk Jézus, a ki örökké való fõpap lett Melkisédek rendje szerint.

\chapter{7}

\par 1 Mert a Melkisédek Sálem királya, a felséges Isten papja, a ki a királyok leverésébõl visszatérõ Ábrahámmal találkozván, õt megáldotta,
\par 2 A kinek tizedet is adott Ábrahám mindenbõl: a ki elsõben is magyarázat szerint igazság királya, azután pedig Sálem királya is, azaz békesség királya,
\par 3 Apa nélkül, anya nélkül, nemzetség nélkül való; sem napjainak kezdete, sem életének vége nincs, de hasonlóvá tétetvén az Isten Fiához, pap marad örökké.
\par 4 Nézzétek meg pedig, mily nagy ez, a kinek a zsákmányból tizedet is adott Ábrahám, a pátriárka;
\par 5 És bár azoknak, kik a Lévi fiai közül nyerik el a papságot, parancsolatjok van, hogy törvény szerint tizedet szedjenek a néptõl, azaz az õ atyafiaiktól, jóllehet õk is az Ábrahám ágyékából származtak;
\par 6 De az, a kinek nemzetsége nem azok közül való, tizedet vett Ábrahámtól, és az ígéretek birtokosát megáldotta,
\par 7 Pedig minden ellenmondás nélkül való, hogy a nagyobb áldja meg a kisebbet.
\par 8 És itt halandó emberek szednek tizedet, ott ellenben az, a ki bizonyság szerint él:
\par 9 És hogy úgy szóljak, Ábrahámnál fogva tized vétetett Lévitõl is, a tizedszedõtõl,
\par 10 Mert õ még az atyja ágyékában vala, a mikor annak elébe ment Melkisédek.
\par 11 Ha tehát a lévitai papság által volna a tökéletesség (mert a nép ez alatt nyerte a törvényt): mi szükség tovább is mondogatni, hogy más pap támadjon a Melkisédek rendje szerint és ne az Áron rendje szerint?
\par 12 Mert a papság megváltozásával szükségképen megváltozik a törvény is.
\par 13 Mert a kirõl ezek mondatnak, az más nemzetségbõl származott, a melybõl senki sem szolgált az oltár körül;
\par 14 Mert nyilvánvaló, hogy a mi Urunk Júdából támadott, a mely nemzetségre nézve semmit sem szólott Mózes a papságról.
\par 15 És még inkább nyilvánvaló az, ha a Melkisédek hasonlatossága szerint áll elõ más pap,
\par 16 A ki nem testi parancsolatnak törvénye szerint, hanem enyészhetetlen életnek ereje szerint lett.
\par 17 Mert ez a bizonyságtétel: Te pap vagy örökké, Melkisédek rendje szerint.
\par 18 Mert az elõbbi parancsolat eltöröltetik, mivelhogy erõtelen és haszontalan,
\par 19 Minthogy a törvény semmiben sem szerzett tökéletességet; de beáll a jobb reménység,  a mely által közeledünk az Istenhez.
\par 20 És a mennyiben nem esküvés nélkül való, mert amazok esküvés nélkül lettek papokká,
\par 21 De az esküvéssel, az által, a ki azt mondá néki: Megesküdött az Úr, és nem bánja meg, te pap vagy örökké, Melkisédek rendje szerint:
\par 22 Annyiban jobb szövetségnek lett kezesévé Jézus.
\par 23 És amazok jóllehet többen lettek papokká, mert a halál miatt meg nem maradhattak:
\par 24 De ennek, minthogy örökké megmarad, változhatatlan a papsága.
\par 25 Ennekokáért õ mindenképen idvezítheti is azokat, a kik õ általa járulnak Istenhez, mert mindenha él, hogy esedezzék érettök.
\par 26 Mert ilyen fõpap illet vala minket, szent, ártatlan, szeplõtelen, a bûnösöktõl elválasztott, és a ki az egeknél magasságosabb lõn,
\par 27 A kinek nincs szüksége, mint a fõpapoknak, hogy napról-napra elõbb a saját bûneiért vigyen áldozatot, azután a népéiért, mert ezt egyszer megcselekedte, maga-magát megáldozván.
\par 28 Mert a törvény gyarló embereket rendel fõpapokká, de a törvény után való esküvés beszéde örök tökéletes Fiút.

\chapter{8}

\par 1 Fõdolog pedig azokra nézve, a miket mondunk, az, hogy olyan fõpapunk van, a ki mennyei Felség királyi székének jobbjára üle,
\par 2 Mint a szent helynek és amaz igazi sátornak szolgája, a melyet az Úr és nem ember épített.
\par 3 Mert minden fõpap ajándékok meg áldozatok vitelére rendeltetik, a miért szükséges, hogy legyen valamije ennek is, a mit áldozatul vigyen.
\par 4 Ha tehát a földön volna, még csak pap sem volna, lévén olyan papok, a kik a törvény szerint áldoznak ajándékokkal,
\par 5 A kik a mennyei dolgok ábrázolatának és árnyékának szolgálnak, a mint Isten mondotta Mózesnek, mikor be akarta végezni a  sátort: Meglásd, úgymond, hogy mindeneket azon minta szerint készíts, a mely a hegyen mutattatott néked.
\par 6 Most azonban annyival kiválóbb szolgálatot nyert, a mennyivel jobb szövetségnek közbenjárója, a mely jobb ígéretek alapján köttetett.
\par 7 Mert ha az az elsõ kifogástalan volt volna, nem kerestetett volna hely a másodiknak.
\par 8 Mert dorgálván õket, így szól: Ímé napok jõnek, ezt mondja az Úr, és az Izráel házával és Júdának házával új szövetséget kötök.
\par 9 Nem azon szövetség szerint, a melyet kötöttem az õ atyáikkal ama napon, mikor kézen fogtam õket, hogy kivezessem Égyiptomból, mert õk  nem maradtak meg abban az én szövetségemben, azért én sem gondoltam velök, mondja az Úr.
\par 10 Mert ez az a szövetség, melyet kötök az Izráel házával, ama napok multán, mond az Úr: Adom az én törvényemet az õ elméjökbe, és az õ szívökbe írom azokat, és leszek nekik Istenök és õk lesznek nekem népem.
\par 11 És nem tanítja kiki az õ felebarátját és kiki az õ atyafiát, mondván: Ismerd meg az Urat; mert mindnyájan megismernek engem a kicsinytõl nagyig.
\par 12 Mert megkegyelmezek álnokságaiknak, és az õ bûneikrõl és gonoszságaikról meg nem emlékezem.
\par 13 Mikor újról beszél, óvá tette az elsõt; a mi pedig megavul és megvénhedik, közel van az enyészethez.

\chapter{9}

\par 1 Annakokáért voltak ugyan az elsõ szövetségnek is istentiszteletei rendtartásai, mint szintén világi szenthelye.
\par 2 Mert sátor építtetett, az elsõ, a melyben vala a gyertyatartó, meg az asztal és a kenyerek felrakása; ezt nevezték szenthelynek.
\par 3 A második kárpiton túl pedig az a sátor, melyet neveztek szentek szentének,
\par 4 Melyben vala az arany füstölõ oltár és a szövetség ládája beborítva minden felõl aranynyal, ebben a mannás aranykorsó és Áron kihajtott vesszeje meg  meg a szövetség táblái,
\par 5 Fölötte pedig a dicsõség kérubjai, beárnyékolva a fedelet, a mikrõl most nem szükséges külön szólani.
\par 6 Ezek pedig ekképen levén elrendezve; az elsõ sátorba ugyan mindenkor bejárnak a papok az istentisztelet elvégzésére,
\par 7 A másodikba azonban egy-egy évben egyszer csak maga a fõpap, vérrel, melyet magáért és a nép bûneiért áldoz.
\par 8 Azt jelentvén ki ezzel a Szent Lélek, hogy még nem nyilt meg a szentély útja, fennállván még az elsõ sátor.
\par 9 A mi példázat a jelenkori idõre, mikor áldoznak oly ajándékokkal és áldozatokkal, melyek nem képesek lelkiismeret szerint tökéletessé tenni a szolgálattevõt,
\par 10 Csakis ételekkel meg italokkal és különbözõ mosakodásokkal - melyek testi rendszabályok - a megjobbulás idejéig kötelezõk.
\par 11 Krisztus pedig megjelenvén, mint a jövendõ javaknak fõpapja, a nagyobb és tökéletesebb, nem kézzel csinált, azaz nem e világból való sátoron keresztül,
\par 12 És nem bakok és tulkok vére által, hanem az õ tulajdon vére által, ment be egyszer s mindenkorra a szentélybe, örök  váltságot szerezve.
\par 13 Mert ha a bakoknak és bikáknak a vére, meg a tehén hamva,  a tisztátalanokra hintetvén, megszentel a testnek tisztaságára:
\par 14 Mennyivel inkább Krisztusnak a vére, a ki örökké való Lélek által önmagát áldozta fel ártatlanul Istennek: megtisztítja a ti lelkiismereteteket a holt cselekedetektõl, hogy szolgáljatok az élõ Istennek.
\par 15 És ezért új szövetségnek a közbenjárója õ, hogy meghalván az elsõ szövetségbeli bûnök váltságáért, a hivatottak elnyerjék az  örökkévaló örökségnek ígéretét.
\par 16 Mert a hol végrendelet van, szükséges, hogy a végrendelkezõ halála bekövetkezzék.
\par 17 Mivel a végrendelet holtak után jogerõs, különben pedig, ha él a végrendelkezõ, épen nem érvényes.
\par 18 Innét van, hogy az elsõ sem szenteltetett meg vér nélkül.
\par 19 Mert mikor Mózes a törvény szerint minden parancsolatot elmondott az egész népnek, vevén a borjúknak és a bakoknak vérét, vízzel és vörös gyapjúval meg izsóppal együtt, magát a könyvet is és az egész népet meghintette,
\par 20 Mondván: Ez azon szövetség vére, a melyet Isten számotokra rendelt.
\par 21 Majd a sátort is és az istentiszteletre való összes edényeket hasonlóképen meghintette vérrel.
\par 22 És csaknem minden vérrel tisztíttatik meg a törvény szerint, és vérontás nélkül nincsen bûnbocsánat.
\par 23 Annakokáért szükséges, hogy a mennyei dolgoknak ábrázolatai effélékkel tisztíttassanak meg, magok a mennyei dolgok azonban ezeknél különb áldozatokkal.
\par 24 Mert nem kézzel csinált szentélybe, az igazinak csak másolatába ment be Krisztus, hanem magába a mennybe, hogy most Isten színe elõtt megjelenjék érettünk.
\par 25 Nem is, hogy sokszor adja magát áldozatul, mint a hogy a fõpap évenként bemegy a szentélybe idegen vérrel;
\par 26 Mert különben sokszor kellett volna szenvednie a világ teremtése óta; így pedig csak egyszer jelent meg az idõknek végén, hogy áldozatával eltörölje a bûnt.
\par 27 És miképen elvégezett dolog, hogy az emberek egyszer meghaljanak, azután az ítélet:
\par 28 Azonképen Krisztus is egyszer megáldoztatván sokak bûneinek eltörlése végett, másodszor bûn nélkül jelen meg azoknak, a kik õt várják  idvességökre.

\chapter{10}

\par 1 Minthogy a törvényben a jövendõ jóknak árnyéka, nem maga a dolgok képe van meg, ennélfogva azokkal az áldozatokkal, a melyeket esztendõnként szzünetlenül visznek, sohasem képes tökéletességre juttatni az odajárulókat;
\par 2 Különben megszûnt volna az áldozás, mivelhogy az egyszer megtisztult áldozók többé semminemû bûntudattal nem bírtak volna.
\par 3 De azok esztendõnként bûnre emlékeztetnek.
\par 4 Mert lehetetlen, hogy a bikák és bakok vére eltörölje a bûnöket.
\par 5 Azért a világba bejövetelekor így szól: Áldozatot és ajándékot nem akartál, de testet alkottál nékem,
\par 6 Égõ és bûnért való áldozatokat nem kedveltél.
\par 7 Akkor mondám: Ímé itt vagyok, (a könyv fejezetében írva vagyon rólam), hogy cselekedjem óh Isten a te akaratodat.
\par 8 Fentebb mondván, hogy áldozatot és ajándékot és égõ, meg bûnért való áldozatokat nem akartál, sem nem kedveltél a melyeket a törvény szerint visznek,
\par 9 Ekkor ezt mondotta: Ímé itt vagyok, hogy cselekedjem a te akaratodat. Eltörli az elsõt, hogy meghagyja a másodikat,
\par 10 A mely akarattal szenteltettünk meg egyszer s mindenkorra, a Jézus Krisztus testének megáldozása által.
\par 11 És minden pap naponként szolgálatban áll és gyakorta viszi ugyanazokat az áldozatokat, a melyek sohasem képesek eltörölni a bûnöket.
\par 12 Õ azonban, egy áldozattal áldozván a bûnökért, mindörökre ûle az Istennek jobbjára,
\par 13 Várván ímmár, míg lábainak zsámolyául vettetnek az õ ellenségei.
\par 14 Mert egyetlenegy áldozatával örökre tökéletesekké tette a megszentelteket.
\par 15 Bizonyságot tesz pedig errõl mi nékünk a Szent Lélekk is, mert minekutána elõre mondotta:
\par 16 Ez az a szövetség, melyet kötök velök ama napok után, mondja az Úr: Adom az én törvényemet az õ szíveikbe, és az õ elméjökbe írom be azokat,
\par 17 Azután így szól: És az õ bûneikrõl és álnokságaikról többé meg nem emlékezem.
\par 18 A hol pedig bûnök bocsánata vagyon, ott nincs többé bûnért való áldozat.
\par 19 Mivelhogy azért atyámfiai bizodalmunk van a szentélybe való bemenetelre a Jézus vére által,
\par 20 Azon az úton, a melyet õ szentelt nékünk új és élõ út gyanánt, a kárpit, azaz az õ teste által,
\par 21 És lévén nagy papunk az Isten háza felett:
\par 22 Járuljunk hozzá igaz szívvel, hitnek teljességével, mint a kiknek szívök tiszta a gonosz lelkiismerettõl,
\par 23 És testök meg van mosva tiszta vízzel; tartsuk meg a reménységnek vallását tántoríthatatlanul, mert hû az, a ki ígéretet tett,
\par 24 És ügyeljünk egymásra, a szeretetre és jó cselekedetekre való felbuzdulás végett,
\par 25 El nem hagyván a magunk gyülekezetét, a miképen szokásuk némelyeknek, hanem intvén egymást annyival inkább, mivel látjátok, hogy ama nap közelget.
\par 26 Mert ha szándékosan vétkezünk, az igazság megismerésére való eljutás után, akkor többé nincs bûnökért való áldozat,
\par 27 Hanem az ítéletnek valami rettenetes várása és a a tûznek lángja, a mely megemészti az ellenszegülõket.
\par 28 A ki megveti a Mózes törvényét, két vagy három tanubizonyságra  irgalom nélkül meghal;
\par 29 Gondoljátok meg, mennyivel súlyosabb büntetésre méltónak ítéltetik az, a ki az Isten Fiát megtapodja, és a szövetségnek vérét, melylyel megszenteltetett, tisztátalannak tartja, és a kegyelemnek Lelkét bántalmazza?
\par 30 Mert ismerjük azt, a ki így szólt: Enyém a bosszúállás, én megfizetek, ezt mondja az Úr. És ismét: Az Úr megítéli az õ népét.
\par 31 Rettenetes dolog az élõ Istennek kezébe esni.
\par 32 Emlékezzetek pedig vissza a régebbi napokra, a melyekben, minekutána megvilágosíttattatok, sok szenvedésteljes küzdelmet állottatok ki,
\par 33 Midõn egyfelõl gyalázásokkal és nyomorgattatásokkal nyilvánosság elé hurczoltak titeket, másfelõl társai lettetek azoknak, a kik így jártak.
\par 34 Mert a foglyokkal is együtt szenvedtetek, és vagyonotok elrablását örömmel fogadtátok, tudván, hogy néktek jobb és maradandó vagyonotok van a mennyekben.
\par 35 Ne dobjátok el hát bizodalmatokat, melynek nagy jutalma van.
\par 36 Mert békességes tûrésre van szükségetek, hogy az Isten akaratát cselekedvén, elnyerjétek az ígéretet.
\par 37 Mert még vajmi kevés idõ, és a ki eljövendõ, eljõ és nem késik.
\par 38 Az igaz pedig hitbõl él. És a ki meghátrál,  abban nem gyönyörködik a lelkem.
\par 39 De mi nem vagyunk meghátrálás emberei, hogy elvesszünk, hanem hitéi, hogy életet nyerjünk.

\chapter{11}

\par 1 A hit pedig a reménylett dolgoknak valósága, és a nem látott dolgokról való meggyõzõdés.
\par 2 Mert ezzel szereztek jó bizonyságot a régebbiek.
\par 3 Hit által értjük meg, hogy a világ Isten beszéde által teremtetett, hogy a mi látható, a láthatatlanból állott elõ.
\par 4 Hit által vitt Ábel becsesebb áldozatot Istennek, mint Kain, a mi által bizonyságot nyert a felõl, hogy igaz, bizonyságot tevén az õ ajándékairól Isten, és az által még holta után is beszél.
\par 5 Hit által vitetett fel Énokh, hogy ne lásson halált, és nem találták meg, mert az Isten felvitte õt. Mert felvitetése elõtt bizonyságot nyert a felõl, hogy kedves volt Istennek.
\par 6 Hit nélkül pedig lehetetlen Istennek tetszeni; mert a ki Isten elé járul, hinnie kell, hogy õ létezik és megjutalmazza azokat, a kik õt keresik.
\par 7 Hit által tisztelte Istent Noé, mikor megintetvén a még nem látott dolgok felõl, házanépe megtartására bárkát készített, a mely által kárhoztatá e világot és a hitbõl való igazságnak örökösévé lett.
\par 8 Hit által engedelmeskedett Ábrahám, mikor elhívatott, hogy menjen ki arra a helyre, a melyet örökölendõ vala, és kiméne, nem tudván, hová megy.
\par 9 Hit által lakott az ígéret földjén, mint idegenben, sátorokban lakván Izsákkal és Jákóbbal, ugyanazon ígéretnek örökös társaival.
\par 10 Mert várja vala az alapokkal bíró várost, melynek építõje és alkotója az Isten.
\par 11 Hit által nyert erõt Sára is az õ méhében való foganásra, és életkora ellenére szûlt, minthogy hûnek tartotta azt, a ki az ígéretet tette.
\par 12 Azért is egytõl, még pedig mintegy kihalttól annyian származtak, mint az égnek csillagai sokaságra nézve, és mint a tenger partja mellett a fövény, mely megszámlálhatatlan.
\par 13 Hitben haltak meg mindezek, nem nyerve meg az ígéreteket, hanem csak távolról látva és üdvözölve azokat, és vallást tevén arról,  hogy idegenek és vándorok a földön.
\par 14 Mert a kik így szólnak, nyilván jelentik, hogy hazát keresnek.
\par 15 És hogyha eszökbe jutott volna az, a melybõl kijöttek, volt volna idejök a visszatérésre.
\par 16 Így azonban jobb után vágyódnak, tudniillik mennyei után; azért nem szégyenli õket az Isten, hogy Istenöknek neveztessék, mert készített nékik várost.
\par 17 Hit által áldozta meg Ábrahám Izsákot, próbára tétetvén, és az egyszülöttet vitte áldozatul, õ, ki az ígéreteket nyerte,
\par 18 A kinek meg volt mondva: Izsákban neveztetik néked mag;
\par 19 Úgy gondolkozván, hogy az Isten a halálból is képes feltámasztani, miért is õt példaképen visszanyerte.
\par 20 Hit által áldá meg a jövendõkre nézve Izsák Jákóbot és Ézsaut.
\par 21 Hit által áldá meg a haldokló Jákób a József fiainak mindenikét, és botja végére hajolva imádkozott.
\par 22 Hit által emlékezett meg élete végén József az Izráel fiainak kimenetelérõl, és az õ tetemeirõl rendelkezett.
\par 23 Hit által rejtegették Mózest az õ szülei születése után három hónapig, mivel látták, hogy kellemes a gyermek, és nem féltek a király parancsától.
\par 24 Hit által tiltakozott Mózes, midõn felnövekedett, hogy a Faraó leánya fiának mondják,
\par 25 Inkább választván az Isten népével való együttnyomorgást, mint a bûnnek ideig-óráig való gyönyörûségét;
\par 26 Égyiptom kincseinél nagyobb gazdagságnak tartván Krisztus gyalázatát, mert a megjutalmazásra tekintett.
\par 27 Hit által hagyta oda Égyiptomot, nem félvén a király haragjától; mert erõs szívû volt, mintha látta volna a láthatatlant.
\par 28 Hit által rendelte a páskát és a vérnek kiontását, hogy az öldöklõ ne illesse az õ elsõszülötteiket.
\par 29 Hit által keltek át a veres tengeren, mint valami szárazföldön, a mit megpróbálván az égyiptomiak, elnyeletlek.
\par 30 Hit által omlottak le Jérikónak kõfalai, midõn hét napig köröskörül járták.
\par 31 Hit által nem veszett el Ráháb, a parázna nõ az engedetlenekkel együtt, befogadván a kémeket békességgel.
\par 32 És mit mondjak még? Hiszen kifogynék az idõbõl, ha szólnék Gedeonról, Bárákról, Sámsonról, Jeftérõl, Dávidról, Sámuelrõl és a prófétákról;
\par 33 A kik hit által országokat gyõztek le, igazságokat cselekedtek, az ígéreteket elnyerték, az oroszlánok száját betömték.
\par 34 Megoltották a tûznek erejét, megmenekedtek a kard élitõl, felerõsödtek a betegségbõl, erõsek lettek a háborúban, megszalasztották az idegenek  táborait.
\par 35 Asszonyok feltámadás útján visszanyerték halottjaikat; mások kínpadra vonattak, visszautasítván a szabadulást, hogy becsesebb feltámadásban részesüljenek.
\par 36 Mások pedig megcsúfoltatások és megostoroztatások próbáját állották ki, sõt még bilincseket és börtönt is;
\par 37 Megköveztettek, kínpróbát szenvedtek, szétfûrészeltettek, kardra hányattak, juhoknak és kecskéknek bõrében bujdostak, nélkülözve, nyomorgattatva, gyötörtetve,
\par 38 A kikre nem volt méltó e világ, bujdosva pusztákon és hegyeken, meg barlangokban és a földnek hasadékaiban.
\par 39 És mindezek, noha hit által jó bizonyságot nyertek, nem kapták meg az ígéretet.
\par 40 Mivel Isten mi felõlünk valami jobbról gondoskodott, hogy nálunk nélkül tökéletességre ne jussanak.

\chapter{12}

\par 1 Annakokáért mi is, kiket a bizonyságoknak ily nagy fellege vesz körül, félretéve minden akadályt és a megkörnyékezõ bûnt, kitartással fussuk meg az elõttünk levõ küzdõ tért.
\par 2 Nézvén a hitnek fejedelmére és bevégezõjére Jézusra, a ki az elõtte levõ öröm helyett, megvetve a gyalázatot, keresztet szenvedett, s az Isten királyi székének jobbjára ült.
\par 3 Gondoljátok meg azért, hogy õ ily ellene való támadást szenvedett el a bûnösöktõl, hogy el ne csüggedjetek lelkeitekben elalélván.
\par 4 Mert még végig nem állottatok ellent, tusakodván a bûn ellen.
\par 5 És elfeledkeztetek-é az intésrõl, a mely néktek mint fiaknak szól: Fiam, ne vesd meg az Úrnak fenyítését, se meg ne lankadj, ha õ dorgál téged;
\par 6 Mert a kit szeret az Úr, megdorgálja, megostoroz pedig mindent, a kit fiává fogad.
\par 7 Ha a fenyítést elszenveditek, akkor veletek úgy bánik az Isten, mint fiaival; mert melyik fiú az, a kit meg nem fenyít az apa?
\par 8 Ha pedig fenyítés nélkül valók vagytok, melyben mindenek részesültek, korcsok vagytok és nem fiak.
\par 9 Aztán, a mi testi apáink fenyítettek minket és becsültük õket; avagy nem sokkal inkább engedelmeskedünk-é a lelkek Atyjának, és élünk!
\par 10 Mert ám azok kevés ideig, tetszésök szerint fenyítettek; õ pedig javunkra, hogy szentségében részesüljünk.
\par 11 Bármely fenyítés ugyan jelenleg nem látszik örvendetesnek, hanem keservesnek, ámde utóbb az igazságnak békességes gyümölcsével fizet azoknak, a kik általa gyakoroltatnak.
\par 12 Annakokáért a lecsüggesztett kezeket és az ellankadt térdeket egyenesítsétek föl,
\par 13 És lábaitokkal egyenesen járjatok, hogy a sánta el ne hajoljon, sõt inkább meggyógyuljon.
\par 14 Kövessétek mindenki irányában a békességet és a szentséget, a mely nélkül senki sem látja meg az Urat:
\par 15 Vigyázván arra, hogy az Isten kegyelmétõl senki el ne szakadjon; nehogy a keserûségnek bármely gyökere, fölnevekedvén, megzavarjon, és ez által sokan megfertõztettessenek.
\par 16 Ne legyen senki parázna vagy istentelen, mint Ézsau, a ki egy ételért eladta elsõ szülöttségi jogát.
\par 17 Mert tudjátok, hogy azután is, mikor akarta örökölni az áldást, megvettetett; mert nem találta meg a megbánás helyét, noha könyhullatással kereste azt az áldást.
\par 18 Mert nem járultatok megtapintható hegyhez, és lángoló tûzhöz, és sûrû homályhoz, és sötétséghez, és szélvészhez,
\par 19 És trombita harsogásához, és a mondásoknak szavához, melyet a kik hallottak, kérték, hogy ne intéztessék hozzájok szó;
\par 20 Mert nem bírták ki, a mi parancsolva volt: Még ha oktalan állat ér is a hegyhez, megköveztessék, vagy nyillal lövettessék le;
\par 21 És oly rettenetes volt a látomány, hogy Mózes is mondá: Megijedtem és remegek:
\par 22 Hanem járultatok Sion hegyéhez, és az élõ Istennek városához, a mennyei Jeruzsálemhez, és az angyalok ezreihez,
\par 23 Az elsõszülöttek seregéhez és egyházához, a kik be vannak írva a mennyekben, és mindenek bírájához Istenhez, és a tökéletes igazak lelkeihez.
\par 24 És az újszövetség közbenjárójához Jézushoz, és a meghintésnek véréhez, mely jobbat beszél, mint az Ábel vére.
\par 25 Vigyázzatok, meg ne vessétek azt, a ki szól; mert ha azok meg nem menekültek, a kik a földön szólót megvetették, sokkal kevésbbé mi, ha elfordulunk attól, a ki a mennyekbõl vagyon,
\par 26 Kinek szava akkor megrendítette a földet, most pedig ígéretet tesz, mondván: Még egyszer megrázom nemcsak a földet, hanem az eget is.
\par 27 Az a "még egyszer" pedig jelenti az állhatatlan dolgoknak mint teremtményeknek megváltozását, hogy a rendíthetetlen dolgok maradjanak meg.
\par 28 Annakokáért mozdíthatatlan országot nyervén, legyünk háládatosak, melynél fogva szolgáljunk az Istennek tetszõ módon kegyességgel és félelemmel.
\par 29 Mert a mi Istenünk megemésztõ tûz.

\chapter{13}

\par 1 Az atyafiúi szeretet maradjon meg.
\par 2 A vendégszeretetrõl el ne felejtkezzetek, mert ez által némelyek, tudtokon kívül, angyalokat vendégeltek meg.
\par 3 Emlékezzetek meg a foglyokról, mint fogolytársak, a gyötrõdõkrõl, mint a kik magatok is testben vagytok.
\par 4 Tisztességes minden tekintetben a házasság és a szeplõtelen házaságy; a paráznákat pedig és a házasságrontókat megítéli az Isten.
\par 5 Fösvénység nélkül való legyen a magatok viselete; elégedjetek meg azzal, a mitek van; mert õ mondotta: Nem hagylak el,  sem el nem távozom tõled;
\par 6 Így hogy bízvást mondjuk: Az Úr az én segítségem, nem félek; ember mit árthat nékem?
\par 7 Emlékezzetek meg a ti elõljáróitokról, a kik szólották néktek az Isten beszédét, és figyelmezvén az õ életök végére, kövessétek hitöket.
\par 8 Jézus Krisztus tegnap és ma és örökké ugyanaz.
\par 9 Különbözõ és idegen tudományok által ne hagyjátok magatokat félrevezettetni; mert jó dolog, hogy kegyelemmel erõsíttessék meg a szív, nem ennivalókkal, a melyeknek semmi hasznát sem veszik azok, a kik azok körül járnak.
\par 10 Van oltárunk, a melyrõl nincs joguk enni azoknak, a kik sátornak szolgálnak.
\par 11 Mert a mely állatok vérét a fõpap beviszi a szentélybe a bûnért, azoknak testét megégetik a táboron kívül.
\par 12 Annakokáért Jézus is, hogy megszentelje az õ tulajdon vére által a népet, a kapun kívül szenvedett.
\par 13 Menjünk ki tehát õ hozzá a táboron kívül, az õ gyalázatát hordozván.
\par 14 Mert nincsen itt maradandó városunk, hanem a jövendõt keressük.
\par 15 Annakokáért õ általa vigyünk dícséretnek áldozatát mindenkor Isten elé, azaz az õ nevérõl vallást tevõ ajkaknak gyümölcsét.
\par 16 A jótékonyságról pedig és az adakozásról el ne felejtkezzetek, mert ilyen áldozatokban gyönyörködik az Isten.
\par 17 Engedelmeskedjetek elõljáróitoknak és fogadjatok szót, mert õk vigyáznak lelkeitekre,  mint számadók; hogy ezt örömmel míveljék és nem bánkódva, mert ez néktek nem használ.
\par 18 Imádkozzatok érettünk. Mert úgy vagyunk meggyõzõdve, hogy jó lelkiismeretünk van, igyekezvén mindenekben tisztességesen forgolódni.
\par 19 Kiváltképen pedig arra kérlek, hogy ezt cselekedjétek, hogy mihamarább visszaadhassam néktek.
\par 20 A békesség Istene pedig, a ki kihozta a halálból a juhoknak nagy pásztorát, örök szövetség vére által, a mi Urunkat Jézust,
\par 21 Tegyen készségesekké titeket minden jóra, hogy cselekedjétek az õ akaratát, azt munkálván ti bennetek, a mi kedves õ elõtte a Jézus Krisztus által, a kinek dicsõség örökkön örökké. Ámen.
\par 22 Kérlek pedig titeket atyámfiai, szívleljétek meg ez intõ beszédet, hiszen rövideden is írtam néktek.
\par 23 Legyen tudtotokra, hogy a mi atyánkfia Timótheus kiszabadult, a kivel ha csakhamar eljõ, meglátogatlak titeket.
\par 24 Köszöntsétek minden elõljárótokat és a szenteket mind. Köszöntenek titeket az Olaszországból valók.
\par 25 Kegyelem mindnyájotokkal! Ámen!


\end{document}