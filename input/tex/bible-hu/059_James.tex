\begin{document}

\title{James}


\chapter{1}

\par 1 Jakab, Istennek és az Úr Jézus Krisztusnak szolgája, az elszórtan levõ tizenkét nemzetségnek; üdvözletemet.
\par 2 Teljes örömnek tartsátok, atyámfiai, mikor különféle kísértésekbe estek,
\par 3 Tudván, hogy a ti hiteteknek megpróbáltatása kitartást szerez.
\par 4 A kitartásban pedig tökéletes cselekedet legyen, hogy tökéletesek és épek legyetek minden fogyatkozás nélkül.
\par 5 Ha pedig valakinek közületek nincsen bölcsessége, kérje  Istentõl, a ki mindenkinek készségesen és szemrehányás nélkül adja; és megadatik néki.
\par 6 De kérje hittel, semmit sem kételkedvén: mert a ki kételkedik, hasonlatos a tenger habjához, a melyet a szél hajt és ide s tova hány.
\par 7 Mert ne vélje az ilyen ember, hogy kaphat valamit az Úrtól;
\par 8 A kétszívû, a minden útjában állhatatlan ember.
\par 9 Dicsekedjék pedig az alacsony sorsú atyafi az õ nagyságával;
\par 10 A gazdag pedig az õ alacsonyságával: mert elmúlik, mint a fûnek virága.
\par 11 Mert felkél a nap az õ hévségével, és megszárítja a füvet; és annak virága elhull, és ábrázatának kedvessége elvész: így hervad el a  gazdag is az õ útaiban.
\par 12 Boldog ember az, a ki a kísértésben kitart; mert minekutána megpróbáltatott, elveszi az életnek koronáját, a mit az Úr ígért az õt szeretõknek.
\par 13 Senki se mondja, mikor kísértetik: Az Istentõl kísértetem: mert az Isten gonoszsággal nem kísérthetõ, õ maga pedig senkit sem kísért.
\par 14 Hanem mindenki kísértetik, a mikor vonja és édesgeti a tulajdon kívánsága.
\par 15 Azután a kívánság megfoganván, bûnt szûl; a bûn pedig teljességre jutván halált nemz.
\par 16 Ne tévelyegjetek szeretett atyámfiai!
\par 17 Minden jó adomány és minden tökéletes ajándék felülrõl való, és a világosságok Atyjától száll alá, a kinél nincs változás, vagy változásnak árnyéka.
\par 18 Az õ akarata szült minket az igazságnak ígéje által, hogy az õ teremtményeinek valami zsengéje legyünk.
\par 19 Azért, szeretett atyámfiai, legyen minden ember gyors a hallásra, késedelmes a szólásra, késedelmes a haragra.
\par 20 Mert ember haragja Isten igazságát nem munkálja.
\par 21 Elvetvén azért minden undokságot és a gonoszságnak sokaságát, szelídséggel fogadjátok a beoltott ígét, a mely  megtarthatja a ti lelkeiteket.
\par 22 Az ígének pedig megtartói legyetek és ne csak hallgatói, megcsalván magatokat.
\par 23 Mert ha valaki hallgatója az ígének és nem megtartója, az ilyen hasonlatos ahhoz az emberhez, a ki tükörben nézi az õ természet szerinti ábrázatát:
\par 24 Mert megnézte magát és elment, és azonnal elfelejtette, milyen volt.
\par 25 De a ki belenéz a szabadság tökéletes törvényébe és megmarad a mellett, az nem feledékeny hallgató, sõt cselekedet követõje lévén, az  boldog lesz az õ cselekedetében.
\par 26 Ha valaki istentisztelõnek látszik köztetek, de nem zabolázza meg nyelvét, sõt megcsalja a maga szívét, annak az istentisztelete hiábavaló.
\par 27 Tiszta és szeplõ nélkül való istentisztelet az Isten és az Atya elõtt ez: meglátogatni az árvákat és özvegyeket az õ nyomorúságukban, és szeplõ nélkül megtartani magát e világtól.

\chapter{2}

\par 1 Atyámfiai ne legyen személyválogatás a ti hitetekben, a mely van a dicsõség Urában, a mi Jézus Krisztusunkban.
\par 2 Mert ha a ti gyülekezetetekbe bemegy egy aranygyûrûs férfiú fényes ruhában, bemegy pedig egy szegény is szennyes ruhában;
\par 3 És rátekinttek arra, a kin a fényes ruha van, és azt mondjátok néki: Te ülj ide szépen; és a szegénynek ezt mondjátok: Te állj ott, vagy ülj ide az én zsámolyom mellé:
\par 4 Nem mondtatok-é ellent magatoknak, és nem lettetek-é gonosz gondolkozású birákká?
\par 5 Halljátok meg szeretett atyámfiai, avagy nem az Isten választotta-é ki e világ szegényeit, hogy  gazdagok legyenek hitben, és örökösei az országnak, a melyet azoknak ígért, a kik õt szeretik?
\par 6 Ti pedig meggyaláztátok a szegényt. Avagy nem a gazdagok hatalmaskodnak-é rajtatok, és nem õk hurczolnak-é titeket a törvény elé?
\par 7 Nem õk káromolják-é azt a szép nevet, a melyrõl neveztettek?
\par 8 Ha ellenben megtartjátok a királyi törvényt az Írás szerint: Szeressed felebarátodat, mint tenmagadat, jól cselekesztek.
\par 9 De ha személyválogatók vagytok, vétkeztek, elmarasztaltatva a törvény által, mint annak megrontói.
\par 10 Mert ha valaki az egész törvényt megtartja is, de vét egy ellen, az egésznek megrontásában bûnös.
\par 11 Mert a ki ezt mondotta: Ne paráználkodjál, ezt is mondotta: Ne ölj. És ha nem paráználkodol, de ölsz, törvényszegõvé lettél.
\par 12 Úgy szóljatok és úgy cselekedjetek, mint a kiket a szabadság törvénye fog megítélni.
\par 13 Mert az ítélet irgalmatlan az iránt, a ki nem cselekszik irgalmasságot; és dicsekedik az irgalmasság az ítélet ellen.
\par 14 Mi a haszna, atyámfiai, ha valaki azt mondja, hogy hite van, cselekedetei pedig nincsenek? Avagy megtarthatja-é õt a hit?
\par 15 Ha pedig az atyafiak, férfiak vagy nõk, mezítelenek, és szûkölködnek mindennapi eledel nélkül,
\par 16 És azt mondja nékik valaki ti közületek: Menjetek el békességgel, melegedjetek meg és lakjatok jól; de nem adjátok meg nékik, a mikre szüksége van a testnek; mi annak a haszna?
\par 17 Azonképen a hit is, ha cselekedetei nincsenek, megholt õ magában.
\par 18 De mondhatja valaki: Néked hited van, nékem pedig cselekedeteim vannak. Mutasd meg nékem a te hitedet a te cselekedeteidbõl, és én meg fogom néked mutatni az én cselekedeteimbõl az én hitemet.
\par 19 Te hiszed, hogy az Isten egy. Jól teszed. Az ördögök is hiszik,  és rettegnek.
\par 20 Akarod-é pedig tudni, te hiábavaló ember, hogy a hit cselekedetek nélkül megholt?
\par 21 Avagy Ábrahám, a mi atyánk, nem cselekedetekbõl igazíttatott-é meg, felvivén Izsákot, az õ fiát az oltárra?
\par 22 Látod, hogy a hit együtt munkálkodott az õ cselekedeteivel, és a cselekedetekbõl lett teljessé a hit;
\par 23 És beteljesedett az Írás, a mely ezt mondja: Hitt pedig Ábrahám az Istennek, és tulajdoníttatott néki igazságul, és Isten  barátjának neveztetett.
\par 24 Látjátok tehát, hogy cselekedetekbõl igazul meg az ember, és nem csupán hitbõl.
\par 25 Hasonlatosképen pedig a tisztátalan Ráháb is, avagy nem cselekedetekbõl igazíttatott é meg, a mikor a követeket házába fogadta, és más úton bocsátotta ki?
\par 26 Mert a miképen holt a test lélek nélkül, akképen holt a hit is cselekedetek nélkül.

\chapter{3}

\par 1 Atyámfiai, ne legyetek sokan tanítók, tudván azt, hogy súlyosabb ítéletünk lészen.
\par 2 Mert mindnyájan sokképen vétkezünk. Ha valaki beszédben nem vétkezik, az tökéletes ember, képes az egész testét is megzabolázni.
\par 3 Ímé a lovaknak szájába zabolát vetünk, hogy engedelmeskedjenek nékünk, és az õ egész testöket igazgatjuk.
\par 4 Ímé a hajók is, noha mily nagyok, és erõs szelektõl hajtatnak, mindazáltal igen kis kormánytól oda fordíttatnak, a hová a kormányos szándéka akarja.
\par 5 Ezenképen a nyelv is kicsiny tag és nagy dolgokkal hányja magát. Ímé csekély tûz mily nagy erdõt felgyújt!
\par 6 A nyelv is tûz, a gonoszságnak összessége. Úgy van a nyelv a mi tagjaink között, hogy megszeplõsíti az egész testet, és lángba borítja életünk folyását, maga is lángba boríttatván a gyehennától.
\par 7 Mert minden természet, vadállatoké, madaraké, csúszómászóké és vízieké megszelídíthetõ és megszelidíttetett az emberi természet által:
\par 8 De a nyelvet az emberek közül senki sem szelidítheti meg; fékezhetetlen gonosz az, halálos méreggel teljes.
\par 9 Ezzel áldjuk az Istent és Atyát, és ezzel átkozzuk az embereket, a kik az Isten hasonlatosságára  teremttettek:
\par 10 Ugyanabból a szájból jõ ki áldás és átok. Atyámfiai, nem kellene ezeknek így lenni!
\par 11 Vajjon a forrás ugyan abból a nyílásból csörgedeztet-é édest és keserût?
\par 12 A vagy atyámfiai, teremhet-é a fügefa olajmagvakat, vagy a szõlõtõ fügét? Azonképen egy forrás sem adhat sós és édes vizet.
\par 13 Kicsoda köztetek bölcs és okos? Mutassa meg az õ jó életébõl az õ cselekedeteit bölcsességnek szelídségével.
\par 14 Ha pedig keserû irígység és czivódás van a ti szívetekben, ne dicsekedjetek és ne hazudjatok az igazság ellen.
\par 15 Ez nem az a bölcsesség, a mely felülrõl jõ, hanem földi, testi és ördögi.
\par 16 Mert a hol irígység és czivakodás van, ott háborúság és minden gonosz cselekedet is van.
\par 17 A felülrõl való bölcsesség pedig elõször is tiszta, azután békeszeretõ, méltányos, engedelmes, irgalmassággal és jó gyümölcsökkel teljes, nem kételkedõ és nem képmutató.
\par 18 Az igazság gyümölcse pedig békességben vettetik azoknak, a  kik békességesen munkálkodnak.

\chapter{4}

\par 1 Honnét vannak háborúk és harczok közöttetek? Nem onnan-é a ti gerjedelmeitekbõl, a melyek a ti tagjaitokban vitézkednek?
\par 2 Kívántok valamit, én nincs néktek: gyilkoltok és irígykedtek, és nem nyerhetitek meg; harczoltok és háborúskodtok; és nincsen semmitek, mert nem kéritek.
\par 3 Kéritek, de nem kapjátok, mert nem jól kéritek, hogy gerjedelmeitekre költsétek azt.
\par 4 Parázna férfiak és asszonyok, nem tudjátok-é, hogy a világ barátsága ellenségeskedés az Istennel? A ki azért e világ barátja akar lenni, az Isten ellenségévé lesz.
\par 5 Vagy azt gondoljátok, hogy az Írás hiába mondja: Irígységre kívánkozik a lélek, a mely bennünk lakozik?
\par 6 De majd nagyobb kegyelmet ád; ezért mondja: Az Isten a kevélyeknek ellene áll, az alázatosoknak pedig kegyelmet ád.
\par 7 Engedelmeskedjetek azért az Istennek; álljatok ellene az ördögnek, és elfut tõletek.
\par 8 Közeledjetek az Istenhez, és közeledni fog  hozzátok. Tisztítsátok meg kezeiteket, ti bûnösök, és szenteljétek meg szíveiteket ti kétszívûek.
\par 9 Nyomorkodjatok és gyászoljatok és sírjatok; a ti nevetéstek gyászra forduljon, és örömötök szomorúságra.
\par 10 Alázzátok meg magatokat az Úr elõtt, és felmagasztal titeket.
\par 11 Ne szóljátok meg egymást atyámfiai. A ki megszólja atyjafiát, és a ki kárhoztatja atyjafiát, az a törvény ellen szól, és a törvényt kárhoztatja. Ha pedig a törvényt kárhoztatod, nem megtartója, hanem bírája vagy a törvénynek.
\par 12 Egy a törvényhozó, a ki hatalmas megtartani és elveszíteni: kicsoda vagy te, hogy kárhoztatod a másikat?
\par 13 Nosza immár ti, a kik azt mondjátok: Ma vagy holnap elmegyünk ama városba, és ott töltünk egy esztendõt, és kalmárkodunk, és nyerünk;
\par 14 A kik nem tudjátok mit hoz a holnap: mert micsoda a ti életetek? Bizony pára az, a mely rövid ideig látszik, azután pedig eltûnik.
\par 15 Holott ezt kellene mondanotok: Ha az Úr akarja és élünk, ím ezt, vagy amazt fogjuk cselekedni.
\par 16 Ti ellenben elbizakodottságtokban dicsekedtek: Minden ilyen dicsekedés gonosz.
\par 17 A ki azért tudna jót cselekedni, és nem cselekszik, bûne az annak.

\chapter{5}

\par 1 Nosza immár ti gazdagok, sírjatok, jajgatván a ti nyomorúságaitok miatt, a melyek elkövetkeznek reátok.
\par 2 Gazdagságotok megrothadt, és a ruháitok moly ette meg;
\par 3 Aranyotokat és ezüstötöket rozsda fogta meg, és azok rozsdája bizonyság ellenetek, és megemészti a ti testeteket, mint a tûz. Kincset gyûjtöttetek az utolsó napokban!
\par 4 Ímé a ti mezõiteket learató munkások bére, a mit ti elfogtatok, kiált. És az aratók kiáltásai eljutottak a Seregek Urának füleihez.
\par 5 Dõzsöltetek e földön és dobzódtatok; szívetek legeltettétek mint  áldozás napján.
\par 6 Elkárhoztattátok, megöltétek az igazat; nem áll ellent néktek.
\par 7 Legyetek azért, atyámfiai, béketûrõk az Úrnak eljöveteléig. Ímé a szántóvetõ várja a földnek drága gyümölcsét, béketûréssel várja, míg reggeli és estveli esõt kap.
\par 8 Legyetek ti is béketûrõk, és erõsítsétek meg szíveteket, mert az Úrnak eljövetele közel van.
\par 9 Ne sóhajtozzatok egymás ellen, atyámfiai, hogy el ne ítéltessetek: ímé a Bíró az ajtó elõtt áll.
\par 10 Például vegyétek, atyámfiai, a szenvedésben és béketûrésben a prófétákat, a kik az Úr nevében szólottak.
\par 11 Ímé, boldogoknak mondjuk a tûrni tudókat. Jóbnak tûrését hallottátok, és az Úrtól való végét láttátok, hogy igen irgalmas az Úr és könyörületes.
\par 12 Mindeneknek elõtte pedig ne esküdjetek, atyámfiai, se az égre, se a földre, se semmi más esküvéssel. Hanem legyen a ti igenetek igen, és a nem nem; hogy kárhoztatás alá ne essetek
\par 13 Szenved-é valaki köztetek? Imádkozzék. Öröme van-é valakinek?  Dícséretet énekeljen.
\par 14 Beteg-é valaki köztetek? Hívja magához a gyülekezet véneit, és imádkozzanak felette, megkenvén õt olajjal az Úrnak nevében.
\par 15 És a hitbõl való imádság megtartja a beteget, és az Úr felsegíti õt. És ha bûnt követett is el, megbocsáttatik néki.
\par 16 Valljátok meg bûneiteket egymásnak és imádkozzatok egymásért, hogy meggyógyuljatok: mert igen hasznos az  igaznak buzgóságos könyörgése.
\par 17 Illés ember volt, hozzánk hasonló természetû; és imádsággal kéré, hogy ne legyen esõ, és nem volt esõ a földön három esztendeig és hat hónapig:
\par 18 És ismét imádkozott, és az ég esõt adott, és a föld megtermé az õ gyümölcsét.
\par 19 Atyámfiai, hogyha valaki ti köztetek eltévelyedik az igazságtól, és megtéríti õt valaki,
\par 20 Tudja meg, hogy a ki bûnöst térít meg az õ tévelygõ útjáról, lelket ment meg a haláltól és  sok bûnt elfedez.


\end{document}