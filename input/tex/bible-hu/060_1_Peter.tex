\begin{document}

\title{1 Peter}


\chapter{1}

\par 1 Péter, Jézus Krisztus apostola, a Pontusban, Galátziában, Kappadóciában, Ázsiában és Bithiniában elszéledt jövevényeknek,
\par 2 A kik ki vannak választva az Atya Isten eleve rendelése szerint, a Lélek megszentelésében, engedelmességre és Jézus Krisztus vérével való meghintésre: kegyelem és békesség adassék néktek bõségesen.
\par 3 Áldott az Isten és a mi Urunk Jézus Krisztusnak Atyja, a ki az õ nagy irgalmassága szerint újonnan szûlt minket élõ reménységre Jézus Krisztusnak a halálból való feltámadása által,
\par 4 Romolhatatlan, szeplõtelen és hervadhatatlan örökségre, a mely a mennyekben van fenntartva számunkra,
\par 5 A kiket Isten hatalma õriz hit által az idvességre, a mely készen van, hogy az utolsó idõben nyilvánvalóvá legyen.
\par 6 A melyben örvendeztek, noha most kissé, ha meg kell lenni, szomorkodtok különféle kísértések között,
\par 7 Hogy a ti kipróbált hitetek, a mi sokkal becsesebb a veszendõ, de tûz által kipróbált aranynál, dícséretre, tisztességre és dicsõségre méltónak találtassék a Jézus Krisztus megjelenésekor;
\par 8 A kit, noha nem láttatok, szerettek; a kiben, noha most nem látjátok, de hisztek benne, kibeszélhetetlen és dicsõült örömmel örvendeztek:
\par 9 Elérvén hitetek czélját, a lélek idvességét.
\par 10 A mely idvesség felõl tudakozódtak és nyomozódtak a próféták, a kik az irántatok való kegyelem felõl jövendöltek:
\par 11 Nyomozódván, hogy mely vagy milyen idõre jelenté azt ki a Krisztusnak õ bennök levõ Lelke, a ki eleve bizonyságot tett a Krisztus szenvedéseirõl és az azok után való dicsõségrõl.
\par 12 A kiknek megjelentetett, hogy nem magoknak, hanem nékünk szolgáltak azokkal, a melyeket most hirdetnek néktek azok, a kik prédikálták néktek az evangyéliomot az egekbõl küldött Szent Lélek által; a mikbe angyalok vágyakoznak betekinteni.
\par 13 Annakokáért felövezvén elmétek derekait, mint józanok, tökéletesen reménykedjetek abban a kegyelemben, a melyet a Jézus Krisztus hoz néktek, mikor megjelen.
\par 14 Mint engedelmes gyermekek ne szabjátok magatokat a ti elõbbi kívánságaitokhoz, a melyek tudatlanságotok alatt voltak bennetek;
\par 15 Hanem a miképen szent az, a ki elhívott titeket, ti is szentek legyetek teljes életetekben;
\par 16 Mert meg van írva: Szentek legyetek, mert én szent vagyok.
\par 17 És ha Atyának hívjátok õt, a ki személyválogatás nélkül ítél, kinek-kinek cselekedete szerint, félelemmel töltsétek a ti jövevénységtek idejét:
\par 18 Tudván, hogy nem veszendõ holmin, ezüstön vagy aranyon váltattatok meg a ti atyáitoktól örökölt hiábavaló életetekbõl;
\par 19 Hanem drága véren, mint hibátlan és szeplõtlen bárányén, a Krisztusén:
\par 20 A ki eleve el volt ugyan rendelve, a világ megalapítása elõtt, megjelent pedig az idõk végén ti érettetek,
\par 21 A kik õ általa hisztek Istenben, a ki feltámasztotta õt a halálból és dicsõséget adott néki; hogy a ti hitetek reménység is legyen Istenben.
\par 22 Lelketeket az igazság iránt való engedelmességben képmutatás nélkül való atyafiúi szeretetre tisztítván meg a Lélek által, egymást tiszta szívbõl buzgón szeressétek;
\par 23 Mint a kik újonnan születtetek nem romlandó magból, de romolhatatlanból, Istennek ígéje által, a mely él és megmarad örökké.
\par 24 Mert minden test olyan, mint a fû, és az embernek minden dicsõsége olyan, mint a fû virága. Megszárad a fû, és virága elhull:
\par 25 De az Úr beszéde megmarad örökké. Ez pedig az a beszéd, a mely néktek hirdettetett.

\chapter{2}

\par 1 Levetvén azért minden gonoszságot, minden álnokságot, képmutatást, irígykedést, és minden rágalmazást.
\par 2 Mint most született csecsemõk, a tiszta, hamisítatlan tej után vágyakozzatok, hogy azon növekedjetek;
\par 3 Mivelhogy ízleltétek, hogy jóságos az Úr.
\par 4 A kihez járulván, mint élõ, az emberektõl ugyan megvetett, de Istennél választott, becses kõhöz,
\par 5 Ti magatok is mint élõ kövek épüljetek fel lelki házzá, szent papsággá, hogy lelki áldozatokkal áldozzatok, a melyek kedvesek Istennek a Jézus Krisztus által.
\par 6 Azért van meg az Írásban: Ímé szegeletkövet teszek Sionban, a mely kiválasztott, becses; és a ki hisz abban, meg nem szégyenül.
\par 7 Tisztesség azért néktek, a kik hisztek; az engedetleneknek pedig: A kõ, a melyet az építõk megvetettek, az lett a szegletnek fejévé és megütközésnek kövévé s botránkozásnak sziklájává;
\par 8 A kik engedetlenek lévén, megütköznek az ígében, a mire rendeltettek is.
\par 9 Ti pedig választott nemzetség, királyi papság, szent nemzet, megtartásra való nép vagytok, hogy hirdessétek Annak hatalmas dolgait, a ki a sötétségbõl az õ csodálatos világosságára hívott el titeket;
\par 10 A kik hajdan nem nép voltatok, most pedig Isten népe vagytok; a kik nem kegyelmezettek voltatok, most pedig kegyelmezettek vagytok.
\par 11 Szeretteim, kérlek titeket, mint jövevényeket és idegeneket, tartóztassátok meg magatokat a testi kívánságoktól, a melyek a lélek ellen vitézkednek;
\par 12 Magatokat a pogányok közt jól viselvén, hogy a miben rágalmaznak titeket mint gonosztévõket, a jó cselekedetekbõl, ha látják azokat, dicsõítsék Istent a meglátogatás napján.
\par 13 Engedelmeskedjetek azért minden emberi rendelésnek az Úrért: akár királynak, mint felebbvalónak;
\par 14 Akár helytartóknak, mint a kiket õ küld a gonosztevõk megbüntetésére, a jól cselekvõknek pedig dícsérésére.
\par 15 Mert úgy van az Isten akaratja, hogy jót cselekedvén, elnémítsátok a balgatag emberek tudatlanságát;
\par 16 Mint szabadok, és nem mint a kiknél a szabadság a gonoszság palástja, hanem mint Istennek szolgái.
\par 17 Mindenkit tiszteljetek, az atyafiúságot szeressétek; az Istent féljétek; a királyt tiszteljétek.
\par 18 A cselédek teljes félelemmel engedelmeskedjenek az uraknak; nem csak a jóknak és kíméleteseknek, de a szívteleneknek is.
\par 19 Mert az kedves dolog, ha valaki Istenrõl való meggyõzõdéséért tûr keserûségeket, méltatlanul szenvedvén.
\par 20 Mert micsoda dicsõség az, ha vétkezve és arczul veretve tûrtök? de ha jót cselekedve és mégis szenvedve tûrtök, ez kedves dolog Istennél.
\par 21 Mert arra hívattatok el; hiszen Krisztus is szenvedett érettetek, néktek példát hagyván, hogy az õ nyomdokait kövessétek:
\par 22 A ki bûnt nem cselekedett, sem a szájában álnokság nem találtatott:
\par 23 A ki szidalmaztatván, viszont nem szidalmazott, szenvedvén nem fenyegetõzött; hanem hagyta az igazságosan ítélõre:
\par 24 A ki a mi bûneinket maga vitte fel testében a fára, hogy a bûnöknek meghalván, az igazságnak éljünk: a kinek sebeivel gyógyultatok meg.
\par 25 Mert olyanok valátok, mint tévelygõ juhok; de most megtértetek lelketek pásztorához és felvigyázójához.

\chapter{3}

\par 1 Hasonlóképen az asszonyok engedelmeskedjenek az õ férjöknek, hogy ha némelyek nem engedelmeskednének is az ígének, feleségük magaviselete által íge nélkül is megnyeressenek;
\par 2 Szemlélvén a ti félelemben való feddhetetlen életeteket.
\par 3 A kinek ékessége ne legyen külsõ, hajuknak fonogatásából és aranynak felrakásából vagy öltözékek felvevésébõl való;
\par 4 Hanem a szívnek elrejtett embere, a szelíd és csendes lélek romolhatatlanságával, a mi igen becses az Isten elõtt.
\par 5 Mert így ékesítették magokat hajdan ama szent asszonyok is, a kik Istenben reménykedtek, engedelmeskedvén az õ férjöknek.
\par 6 Miként Sára engedelmeskedett Ábrahámnak, urának nevezvén õt, a kinek gyermekei lettek, ha jót cselekesztek, és semmi félelemtõl nem rettegtek.
\par 7 A férfiak hasonlóképen, együtt lakjanak értelmes módon feleségükkel,az asszonyi nemnek, mint gyöngébb edénynek, tisztességet tévén, mint a kik örökös társaik az élet kegyelmében; hogy a ti imádságaitok meg ne hiúsuljanak.
\par 8 Végezetre mindnyájan legyetek egyértelmûek, rokonérzelmûek, atyafiszeretõk, irgalmasak, kegyesek:
\par 9 Nem fizetvén gonoszszal a gonoszért, avagy szidalommal a szidalomért; sõt ellenkezõleg áldást mondván, tudva, hogy arra hivattatok el, hogy áldást örököljetek,
\par 10 Mert a ki akarja az életet szeretni, jó napokat látni, tiltsa meg nyelvét a gonosztól, és ajkait, hogy ne szóljanak álnokságot:
\par 11 Forduljon el a gonosztól, és cselekedjék jót; keresse a békességet, és kövesse azt.
\par 12 Mert az Úr szemei az igazakon vannak, és az õ fülei azoknak könyörgésein; az Úr orczája pedig a gonoszt cselekvõkön.
\par 13 És kicsoda az, a ki bántalmaz titeket, ha a jónak követõi lesztek?
\par 14 De ha szenvedtek is az igazságért, boldogok vagytok,azoktól való félelembõl pedig ne féljetek, se zavarba ne essetek;
\par 15 Az Úr Istent pedig szenteljétek meg a ti szívetekben. Mindig készek legyetek megfelelni mindenkinek, a ki számot kér tõletek a bennetek levõ reménységrõl, szelídséggel és félelemmel:
\par 16 Jó lelkiismeretetek lévén; hogy a miben rágalmaznak titeket, mint gonosztevõket, megszégyenüljenek a kik gyalázzák a ti Krisztusban való jó élteteket.
\par 17 Mert jobb ha jót cselekedve szenvedtek, ha így akarja az Isten akarata, hogynem gonoszt cselekedve.
\par 18 Mert Krisztus is szenvedett egyszer a bûnökért, mint igaz a nem igazakakért, hogy minket Istenhez vezéreljen; megölettetvén ugyan test szerint, de megeleveníttetvén lélek szerint;
\par 19 A melyben elmenvén, a tömlöczben lévõ lelkeknek is prédikált,
\par 20 A melyek engedetlenek voltak egykor, mikor egyszer várt az Isten béketûrésre a Noé napjaiban, a bárka készítésekor, a melyben kevés, azaz nyolcz lélek tartatott meg víz által;
\par 21 A mi minket is megtart most képmás gyanánt, mint keresztség, a mi nem a test szenyjének lemosása, hanem jó lelkiismeret keresése Isten iránt, a Jézus Krisztus feltámadása által;
\par 22 A ki Istennek jobbján van, felmenvén a mennybe; a kinek alávettettek az angyalok, hatalmasságok és erõk.

\chapter{4}

\par 1 Minthogy azért Krisztus testileg szenvedett, fegyverkezzetek fel ti is azzal a gondolattal, hogy a ki testileg szenved, megszûnik a bûntõl,
\par 2 Hogy többé ne embereknek kívánságai, hanem Isten akarata szerint éljétek a testben hátralevõ idõt.
\par 3 Mert elég nékünk, hogy életünk elfolyt idejében pogányok akaratát cselekedtük, járván feslettségekben, kívánságokban, részegségekben, dobzódásokban, ivásokban és undok bálványimádásokban.
\par 4 A mi miatt csudálkoznak, hogy nem futtok velök együtt a kicsapongásnak ugyanabba az áradatába, szitkozódván.
\par 5 A kik számot adnak majd annak, a ki készen van megítélni élõket és holtakat.
\par 6 Mert azért hirdettetett az evangyéliom a holtaknak is, hogy megítéltessenek emberek szerint testben, de éljenek Isten szerint lélekben.
\par 7 A vége pedig mindennek közel van. Annakokáért legyetek mértékletesek és józanok, hogy imádkozhassatok.
\par 8 Mindenek elõtt pedig legyetek hajlandók az egymás iránti szeretetre; mert a szeretet sok vétket elfedez.
\par 9 Legyetek egymáshoz vendégszeretõk, zúgolódás nélkül.
\par 10 Kiki amint kegyelmi ajándékot kapott, úgy sáfárkodjatok azzal egymásnak, mint Isten sokféle kegyelmének jó sáfárai;
\par 11 Ha valaki szól, mintegy Isten ígéit szólja: ha valaki szolgál, mintegy azzal az erõvel szolgáljon, a melyet Isten ád: hogy mindenben dícsõíttessék a Jézus Krisztus által, a kinek dicsõség és hatalom örökkön-örökké. Ámen.
\par 12 Szeretteim, ne rémüljetek meg attól a tûztõl, a mely próbáltatás végett támadt köztetek, mintha valami rémületes dolog történnék veletek;
\par 13 Sõt, a mennyiben részetek van a Krisztus szenvedéseiben, örüljetek, hogy az õ dicsõségének megjelenésekor is vígadozva örvendezhessetek.
\par 14 Boldogok vagytok, ha Krisztus nevéért gyaláznak titeket; mert megnyugszik rajtatok a dicsõségnek és az Istennek Lelke, a mit amazok káromolnak ugyan, de ti dicsõítitek azt.
\par 15 Mert senki se szenvedjen közületek mint gyilkos, vagy tolvaj, vagy gonosztévõ, vagy mint más dolgába avatkozó:
\par 16 Ha pedig mint keresztyén szenved, ne szégyelje, sõt dicsõítse azért az Istent.
\par 17 Mert itt az ideje, hogy elkezdõdjék az ítélet az Istennek házán: ha pedig elõször mi rajtunk kezdõdik, mi lesz azoknak a végök, a kik nem engedelmeskednek az Isten evangyéliomának?
\par 18 És ha az igaz is alig tartatik meg, hová lesz az istentelen és bûnös?
\par 19 Annakokáért a kik az Isten akaratából szenvednek is, ajánlják néki lelköket mint hû teremtõnek, jót cselekedvén.

\chapter{5}

\par 1 A köztetek lévõ presbitereket kérem én, a presbitertárs, és a Krisztus szenvedésének tanuja, és a megjelenendõ dicsõségnek részese;
\par 2 Legeltessétek az Istennek köztetek lévõ nyáját, gondot viselvén arra nem kényszerítésbõl, hanem örömest; sem nem rút nyerészkedésbõl, hanem jóindulattal;
\par 3 Sem nem úgy, hogy uralkodjatok a gyülekezeteken, hanem mint példányképei a nyájnak.
\par 4 És mikor megjelenik a fõpásztor, elnyeritek a dicsõségnek hervadatlan koronáját.
\par 5 Hasonlatosképen ti ifjabbak engedelmeskedjetek a véneknek: mindnyájan pedig, egymásnak engedelmeskedvén, az alázatosságot öltsétek fel, mert az Isten a kevélyeknek ellene áll, az alázatosaknak pedig kegyelmet ád.
\par 6 Alázzátok meg tehát magatokat Istennek hatalmas keze alatt, hogy felmagasztaljon titeket annak idején.
\par 7 Minden gondotokat õ reá vessétek, mert néki gondja van reátok.
\par 8 Józanok legyetek, vigyázzatok; mert a ti ellenségetek, az ördög, mint ordító oroszlán szerte jár, keresvén, kit elnyeljen:
\par 9 A kinek álljatok ellen, erõsek lévén a hitben, tudva, hogy a világban lévõ atyafiságotokon ugyanazok a szenvedések telnek be.
\par 10 A minden kegyelemnek Istene pedig, a ki az õ örök dicsõségére hívott el minket a a Krisztus Jézusban, titeket, a kik rövid ideig szenvedtetek, õ maga tegyen tökéletesekké, erõsekké, szilárdakká és állhatatosokká,
\par 11 Övé a dicsõség és a hatalom örökkön-örökké. Ámen.
\par 12 Silvánus által, a ki, a mint gondolom, hû atyátokfia, röviden írtam, intve és bizonyságot téve, hogy az az Istennek igaz kegyelme, a melyben állotok.
\par 13 Köszönt titeket a veletek együtt választott babiloni gyülekezet és Márk, az én fiam.
\par 14 Köszöntsétek egymást szeretet csókjával. Békesség mindnyájatoknak, a kik Krisztusban vagytok.


\end{document}