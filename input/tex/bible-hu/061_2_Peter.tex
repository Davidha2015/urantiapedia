\begin{document}

\title{2 Peter}


\chapter{1}

\par 1 Simon Péter, Jézus Krisztus szolgája és apostola, azoknak, a  kik velünk egyenlõ drága hitet nyertek a mi Istenünknek és megtartónknak Jézus Krisztusnak igazságában:
\par 2 Kegyelem és békesség adassék néktek bõségesen az Istennek és Jézusnak a mi Urunknak megismerésében.
\par 3 Mivelhogy az õ isteni ereje mindennel megajándékozott minket, a mi az életre és a kegyességre való, Annak megismerése által, a ki minket a saját dicsõségével és hatalmával elhívott;
\par 4 A melyek által igen nagy és becses ígéretekkel ajándékozott meg bennünket; hogyazok által isteni természet részeseivé legyetek, kikerülvén a  romlottságot, a mely a kívánságban van e világon.
\par 5 Ugyanerre pedig teljes igyekezetet is fordítván, a ti hitetek mellé ragasszatok jó cselekedetet, a jó cselekedet mellé tudományt,
\par 6 A tudomány mellé pedig mértékletességet, a mértékletesség mellé pedig tûrést, a tûrés mellé pedig kegyességet,
\par 7 A kegyesség mellé pedig atyafiakhoz való hajlandóságot, az atyafiakhoz való hajlandóság mellé pedig szeretetet.
\par 8 Mert ha ezek megvannak és gyarapodnak bennetek, nem tesznek titeket hivalkodókká, sem gyümölcstelenekké a mi Urunk Jézus Krisztus megismerésére nézve.
\par 9 Mert a kiben ezek nincsenek meg, az vak, rövidlátó, elfelejtkezvén a régi bûneibõl való  megtisztulásáról.
\par 10 Annakokáért, atyámfiai, igyekezzetek inkább a ti elhívatástokat és kiválasztatásotokat erõssé tenni; mert ha ezeket cselekszitek, nem ütköztök meg soha.
\par 11 Mert ekképen gazdagon adatik majd néktek a mi Urunknak és megtartónknak, a Jézus Krisztusnak örök országába való bemenetel.
\par 12 Annakokáért nem mulasztom el, hogy mindenkor emlékeztesselek titeket ezekre, hogy tudjátok ezeket, és erõsek vagytok a jelenvaló igazságban.
\par 13 Méltónak vélem pedig, a míg ebben a sátorban vagyok, hogy emlékeztetés által ébresztgesselek titeket;
\par 14 Mint a ki tudom, hogy hamar leteszem sátoromat, a miképen a mi Urunk Jézus Krisztus is megjelentette nékem.
\par 15 De igyekezni fogok azon, hogy ti az én halálom után is mindenkor megemlékezhessetek ezekrõl.
\par 16 Mert nem mesterkélt meséket követve ismertettük meg veletek a mi Urunk Jézus Krisztus hatalmát és eljövetelét; hanem mint a kik szemlélõi voltunk az  õ nagyságának.
\par 17 Mert a mikor az Atya Istentõl azt a tisztességet és dicsõséget nyerte, hogy hozzá a felséges dicsõség ilyen szózata jutott: Ez az én szeretett Fiam, a kiben én gyönyörködöm:
\par 18 Ezt az égbõl jövõ szózatot mi hallottuk, együtt lévén vele a szent hegyen.
\par 19 És igen biztos nálunk a prófétai beszéd is, a melyre jól teszitek, ha figyelmeztek, mint sötét  helyen világító szövétnekre, míg nappal virrad, és hajnalcsillag kél fel szívetekben;
\par 20 Tudván elõször azt, hogy az írásban egy prófétai szó sem támad saját magyarázatból.
\par 21 Mert sohasem ember akaratából származott a prófétai szó; hanem a Szent Lélektõl indíttatva szólottak az Istennek szent emberei.

\chapter{2}

\par 1 Valának pedig hamis próféták is a nép között, a miképen ti köztetek is lesznek hamis tanítók, a kik veszedelmes eretnekséget fognak becsempészni, és az Urat, a ki megváltotta õket, megtagadván, önmagokra hirtelen való veszedelmet hoznak.
\par 2 És sokan fogják követni azoknak romlottságát; a kik miatt az igazság útja káromoltatni fog.
\par 3 És a telhetetlenség miatt költött beszédekkel vásárt ûznek belõletek; kiknek kárhoztatásuk régtõl fogva nem szünetel, és romlásuk nem szunnyad.
\par 4 Mert ha nem kedvezett az Isten a bûnbe esett angyaloknak, hanem mélységbe taszítván, a sötétség lánczaira adta oda õket, hogy fenntartassanak az ítéletre;
\par 5 És ha a régi világnak sem kedvezett, de Nóét az igazság hirdetõjét, nyolczad magával megõrizte, özönvízzel borítván el az istentelenek világát;
\par 6 És ha Sodoma és Gomora városait elhamvasztotta, végromlásra kárhoztatta, például tévén azokra nézve, a kik istentelenkedni fognak;
\par 7 És ha megszabadította az igaz Lótot, a ki az istenteleneknek fajtalanságban való forgolódása miatt elfáradt;
\par 8 (Mert amaz igaz, azok között lakván, a gonosz cselekedeteket látva és hallva, napról-napra gyötri vala az õ igaz lelkét):
\par 9 Meg tudja szabadítani az Úr a kegyeseket a kísértésekbõl, a gonoszokat pedig az ítélet napjára büntetésre fenntartani.
\par 10 Fõképen pedig azokat, a kik a testet követvén, tisztátalan kívánságban járnak, és a hatalmasságot megvetik. Vakmerõk, magoknak kedveskedõk, akik a méltóságokat káromolni nem  rettegnek:
\par 11 Holott az angyalok, a kik erõre és hatalomra nézve nagyobbak, nem szólnak azok ellen az Úr elõtt káromló ítéletet.
\par 12 De ezek, mint oktalan természeti állatok, a melyek megfogatásra és elpusztításra valók, azokat, a miket nem ismernek, káromolván, azoknak pusztulásával fognak el is pusztulni,
\par 13 Megkapván gonoszságuk díját, mint a kik gyönyörûségnek tartják a naponkénti dobzódást; undokságok és fertelmek, a kik kéjelegnek az õ csalárdságukban, mikor együtt lakmároznak veletek;
\par 14 A kiknek szemei paráznasággal telvék, bûnnel telhetetlenek; elhitetik az állhatatlan lelkeket, szívök gyakorlott a telhetetlenségben, átok gyermekei;
\par 15 A kik elhagyván az egyenes útat, eltévelyedtek, követvén Bálámnak, Bosor fiának útját, a ki a gonoszság díját kedvelte.
\par 16 De megfeddetett az õ törvénytelenségéért: egy igavonó néma állat emberi szóval szólván, megakadályozta a próféta esztelenségét.
\par 17 Ezek víztelen kútfõk, széltõl hányatott fellegek, a kiknek a sötétség homálya van fenntartva örökre.
\par 18 Mert hiábavalóság kevély szavait szólván, testi kívánsággal, bujálkodással elhitetik azokat, a kik valóban elszakadtak a tévelygésben élõktõl,
\par 19 Szabadságot ígérvén azoknak, holott õk magok a romlottság szolgái; mert a kit valaki legyõzött, az annak szolgájává lett.
\par 20 Mert ha az Úrnak, a megtartó Jézus Krisztusnak megismerése által a világ fertelmeit elkerülték, de ezekbe ismét belekeveredve legyõzetnek, az õ utolsó állapotjuk  gonoszabbá lett az elsõnél.
\par 21 Mert jobb volna rájok nézve, ha meg sem ismerték volna az igazság útját, mint hogy megismervén, elpártoljanak a nekik adott szent parancsolattól.
\par 22 De betelt rajtok az igaz példabeszéd szava: Az eb visszatért a saját okádására, és a megmosódott disznó a sárnak fertõjébe.

\chapter{3}

\par 1 Ez immár második levélírásom néktek, szeretteim, amelylyel a ti tiszta gondolkozástokat emlékeztetés által serkentgetem;
\par 2 Hogy megemlékezzetek a szent prófétáktól ezelõtt mondott beszédekrõl, és az Úrnak és  Megtartónak általunk, az apostolok által közölt parancsolatjáról:
\par 3 Tudván elõször azt, hogy az utolsó idõben csúfolkodók támadnak, a kik saját kívánságaik szerint járnak,
\par 4 És ezt mondják: Hol van az õ eljövetelének ígérete? Mert a mióta az atyák elhunytak, minden azonképen marad a teremtés kezdetétõl fogva.
\par 5 Mert kész-akarva nem tudják azt, hogy egek régtõl fogva voltak, és föld, mely vízbõl és víz által állott elõ az Isten szavára;
\par 6 A melyek által az akkori világ vízzel elboríttatván elveszett:
\par 7 A mostani egek pedig és a föld, ugyanazon szó által megkíméltettek, tûznek tartatván fenn, az ítéletnek és az istentelen emberek romlásának  napjára.
\par 8 Ez az egy azonban ne legyen elrejtve elõttetek, szeretteim, hogy egy nap az Úrnál olyan, mint ezer esztendõ, és ezer esztendõ mint egy nap.
\par 9 Nem késik el az ígérettel az Úr, mint némelyek késedelemnek tartják; hanem hosszan tûr érettünk, nem akarván, hogy némelyek elveszszenek, hanem hogy mindenki  megtérésre jusson.
\par 10 Az Úr napja pedig úgy jõ majd el, mint éjjeli tolvaj, a mikor az egek ropogva elmúlnak, az elemek pedig megégve felbomlanak, és  a föld és a rajta lévõ dolgok is megégnek.
\par 11 Mivelhogy azért mindezek felbomlanak, milyeneknek kell lennetek néktek szent életben és kegyességben,
\par 12 A kik várjátok és sóvárogjátok az Isten napjának eljövetelét, a melyért az egek tûzbe borulva felbomlanak, és az elemek égve megolvadnak!
\par 13 De új eget és új földet várunk az õ ígérete szerint, a melyekben igazság lakozik.
\par 14 Annakokáért szeretteim, ezeket várván , igyekezzetek, hogy  szeplõ nélkül és hiba nélkül valóknak találjon titeket békességben.
\par 15 És a mi Urunknak hosszútûrését idvességnek tartsátok; a mi képen a mi szeretett atyánkfia Pál is írt néktek a néki adott bölcsesség szerint.
\par 16 Szinte minden levélben is, a mikor ezekrõl beszél azokban; a melyekben vannak némely nehezen érthetõ dolgok, a miket a tudatlanok és állhatatlanok elcsûrnek-csavarnak, mint egyéb írásokat is, a magok vesztére.
\par 17 Ti azért szeretteim elõre tudván ezt, õrizkedjetek, hogy az istentelenek tévelygéseitõl elragadtatva, a saját erõsségetekbõl  ki ne essetek;
\par 18 Hanem növekedjetek a kegyelemben és a mi Urunknak és megtartó Jézus Krisztusunknak ismeretében. Néki legyen dicsõség, mind most, mind örökkön-örökké. Ámen.


\end{document}