\begin{document}

\title{Szent János III. levele}


\chapter{1}

\par 1 A presbiter a szeretett Gájusnak, a kit én igazán szeretek.
\par 2 Szeretett barátom, kívánom, hogy mindenben jól legyen dolgod, és légy egészséges, a mint jó dolga van a lelkednek.
\par 3 Mert felettébb örültem, a mikor atyafiak jöttek és bizonyságot tettek a te igazságodról, úgy, a mint te az igazságban jársz.
\par 4 Nincs annál nagyobb örömem, mintha hallom, hogy az én gyermekeim az igazságban járnak.
\par 5 Szeretett barátom, híven cselekszel mindenben, a mit az atyafiakért, és pedig az idegenekért teszel,
\par 6 A kik bizonyságot tettek a te szeretetedrõl a gyülekezet elõtt; a kiket jól teszed, ha Istenhez méltóan bocsátasz útjokra.
\par 7 Mert az õ nevéért mentek ki, semmit sem fogadván el a pogányoktól;
\par 8 Nékünk azért be kell fogadnunk az ilyeneket, hogy munkatársaikká lehessünk az igazságban.
\par 9 Írtam a gyülekezetnek; de Diotrefesz, a ki elsõséget kíván közöttük, nem fogad el minket.
\par 10 Ezért, ha odamegyek, felemlítem az õ dolgait, a melyeket cselekszik, gonosz szavakkal csácsogván ellenünk; sõt nem elégedvén meg ezzel, maga sem fogadja be az atyafiakat, és a kik ezt akarnák, azokat is akadályozza, és az egyházból kiveti.
\par 11 Szeretett barátom, ne a rosszat kövesd, hanem a jót. A ki jót cselekszik, az Istentõl  van; a ki pedig rosszat cselekszik, nem látta az Istent.
\par 12 Demeter mellett mindenki bizonyságot tett, maga az igazság is; de mi is bizonyságot teszünk, és tudjátok, hogy a mi bizonyságtételünk igaz.
\par 13 Sok írni valóm volna, de nem akarok tintával és tollal írni néked:
\par 14 Hanem reménylem, hogy csakhamar meglátlak téged és szemtõl szembe beszélhetünk.
\par 15 Békesség néked! Köszöntenek téged a te barátaid. Köszöntsd a barátainkat név szerint.


\end{document}