\begin{document}

\title{Kejadian}


\chapter{1}

\par 1 Pada mulanya, waktu Allah mulai menciptakan alam semesta,
\par 2 bumi belum berbentuk, dan masih kacau-balau. Samudra yang bergelora, yang menutupi segala sesuatu, diliputi oleh gelap gulita, tetapi kuasa Allah bergerak di atas permukaan air.
\par 3 Allah berkata, "Jadilah terang!" Lalu ada terang.
\par 4 Allah senang melihat hal itu. Lalu dipisahkan-Nya terang itu dari gelap,
\par 5 dan dinamakan-Nya terang itu "Siang" dan gelap itu "Malam". Malam lewat, dan jadilah pagi. Itulah hari yang pertama.
\par 6 Kemudian Allah berkata, "Jadilah sebuah kubah untuk membagi air itu menjadi dua, dan menahannya dalam dua tempat yang terpisah." Lalu hal itu terjadi. Demikianlah Allah membuat kubah yang memisahkan air yang ada di bawah kubah itu dari air yang ada di atasnya.
\par 8 Kubah itu dinamakan-Nya "Langit". Malam lewat dan jadilah pagi. Itulah hari yang kedua.
\par 9 Kemudian Allah berkata, "Hendaklah air yang ada di bawah langit mengalir ke satu tempat, sehingga tanah akan kelihatan." Lalu hal itu terjadi.
\par 10 Allah menamakan tanah itu "Darat", dan kumpulan air itu dinamakan-Nya "Laut". Dan Allah senang melihat hal itu.
\par 11 Allah berkata lagi, "Hendaklah tanah mengeluarkan segala macam tumbuh-tumbuhan, yaitu jenis yang menghasilkan biji-bijian dan jenis yang menghasilkan buah-buahan." Lalu hal itu terjadi.
\par 12 Demikianlah tanah mengeluarkan segala macam tumbuh-tumbuhan. Dan Allah senang melihat hal itu.
\par 13 Malam lewat dan jadilah pagi. Itulah hari yang ketiga.
\par 14 Kemudian Allah berkata, "Hendaklah ada benda-benda terang di langit untuk menerangi bumi, untuk memisahkan siang dari malam, dan untuk menunjukkan saat mulainya hari, tahun, dan hari raya agama." Maka hal itu terjadi.
\par 16 Demikianlah Allah membuat kedua benda terang yang besar, yaitu matahari untuk menguasai siang, dan bulan untuk menguasai malam; selain itu dibuat-Nya juga bintang-bintang.
\par 17 Allah menempatkan benda-benda terang itu di langit untuk menerangi bumi,
\par 18 untuk menguasai siang dan malam, dan untuk memisahkan terang dari gelap. Dan Allah senang melihat hal itu.
\par 19 Malam lewat, dan jadilah pagi. Itulah hari yang keempat.
\par 20 Kemudian Allah berkata, "Hendaklah di dalam air berkeriapan banyak macam makhluk hidup, dan di udara beterbangan banyak burung-burung."
\par 21 Maka Allah menciptakan binatang-binatang raksasa laut, dan segala jenis makhluk yang hidup di dalam air, serta segala jenis burung. Dan Allah senang melihat hal itu.
\par 22 Allah memberkati semuanya itu dengan memberi perintah kepada makhluk yang hidup di dalam air supaya berkembang biak dan memenuhi laut, dan kepada burung-burung supaya bertambah banyak.
\par 23 Malam lewat dan jadilah pagi. Itulah hari yang kelima.
\par 24 Kemudian Allah berkata, "Hendaklah bumi mengeluarkan segala jenis binatang darat, yang jinak dan yang liar, besar maupun kecil." Lalu hal itu terjadi.
\par 25 Demikianlah Allah membuat semuanya itu dan ia senang melihat hal itu.
\par 26 Kemudian Allah berkata, "Sekarang Kita akan membuat manusia yang akan menjadi seperti Kita dan menyerupai Kita. Mereka akan berkuasa atas ikan-ikan, burung-burung, dan segala binatang lain, baik jinak maupun liar, baik besar maupun kecil."
\par 27 Demikianlah Allah menciptakan manusia, dan dijadikannya mereka seperti diri-Nya sendiri. Diciptakan-Nya mereka laki-laki dan perempuan.
\par 28 Kemudian diberkati-Nya mereka dengan ucapan "Beranakcuculah yang banyak, supaya keturunanmu mendiami seluruh muka bumi serta menguasainya. Kamu Kutugaskan mengurus ikan-ikan, burung-burung, dan semua binatang lain yang liar.
\par 29 Untuk makananmu Kuberikan kepadamu segala jenis tumbuhan yang menghasilkan biji-bijian dan buah-buahan.
\par 30 Tetapi kepada segala burung dan binatang liar lainnya, Kuberikan rumput dan tanaman berdaun sebagai makanannya." Maka hal itu terjadi.
\par 31 Allah memandang segala sesuatu yang telah dibuat-Nya itu, dan Ia sangat senang. Malam lewat dan jadilah pagi. Itulah hari yang keenam.

\chapter{2}

\par 1 Maka selesailah penciptaan seluruh alam semesta.
\par 2 Pada hari yang ketujuh Allah telah menyelesaikan pekerjaan-Nya itu, lalu Ia beristirahat.
\par 3 Maka diberkati-Nya hari yang ketujuh itu dan dijadikan-Nya hari yang khusus, karena pada hari itu Allah beristirahat setelah menyelesaikan pekerjaan-Nya.
\par 4 Itulah riwayat penciptaan alam semesta. Ketika TUHAN Allah membuat alam semesta,
\par 5 belum ada benih yang bertunas dan belum ada tanam-tanaman di bumi, karena TUHAN belum menurunkan hujan dan belum ada orang untuk mengerjakan tanah itu.
\par 6 Tetapi air mulai merembes dari bawah dan membasahi permukaan bumi.
\par 7 Kemudian TUHAN Allah mengambil sedikit tanah, membentuknya menjadi seorang manusia, lalu menghembuskan napas yang memberi hidup ke dalam lubang hidungnya; maka hiduplah manusia itu.
\par 8 Selanjutnya TUHAN Allah membuat taman di Eden, di sebelah timur, dan ditempatkan-Nya di situ manusia yang sudah dibentuk-Nya itu.
\par 9 TUHAN Allah menumbuhkan segala macam pohon yang indah, yang menghasilkan buah-buahan yang baik. Di tengah-tengah taman tumbuhlah pohon yang memberi hidup, dan pohon yang memberi pengetahuan tentang yang baik dan yang jahat.
\par 10 Sebuah sungai mengalir dari Eden, membasahi taman itu; dan di luar Eden sungai itu terbagi menjadi empat cabang.
\par 11 Yang pertama bernama Pison; sungai itu mengalir mengelilingi tanah Hawila.
\par 12 Di situ terdapat emas murni dan juga wangi-wangian yang sulit diperoleh, serta batu-batu permata.
\par 13 Sungai yang kedua bernama Gihon; airnya mengalir mengelilingi tanah Kus.
\par 14 Sungai yang ketiga bernama Tigris dan mengalir di sebelah timur Asyur. Sungai yang keempat bernama Efrat.
\par 15 Kemudian TUHAN Allah menempatkan manusia itu di taman Eden untuk mengerjakan dan memelihara taman itu.
\par 16 TUHAN berkata kepada manusia itu, "Engkau boleh makan buah-buahan dari semua pohon di taman ini,
\par 17 kecuali dari pohon yang memberi pengetahuan tentang yang baik dan yang jahat. Buahnya tidak boleh engkau makan; jika engkau memakannya, engkau pasti akan mati pada hari itu juga."
\par 18 Lalu TUHAN Allah berkata, "Tidak baik manusia hidup sendirian. Aku akan membuat teman yang cocok untuk membantunya."
\par 19 Maka Ia mengambil sedikit tanah dan membentuk segala macam binatang darat dan binatang udara. Semuanya dibawa Allah kepada manusia itu untuk melihat nama apa yang akan diberikannya kepada binatang-binatang itu. Itulah asal mulanya binatang di darat dan di udara mendapat namanya masing-masing.
\par 20 Demikianlah manusia itu memberi nama kepada semua binatang di darat dan di udara. Tetapi tidak satu pun di antaranya bisa menjadi teman yang cocok untuk membantunya.
\par 21 Lalu TUHAN Allah membuat manusia tidur nyenyak, dan selagi ia tidur, TUHAN Allah mengeluarkan salah satu rusuk dari tubuh manusia itu, lalu menutup bekasnya dengan daging.
\par 22 Dari rusuk itu TUHAN membentuk seorang perempuan, lalu membawanya kepada manusia itu.
\par 23 Maka berkatalah manusia itu, "Ini dia, orang yang sama dengan aku--tulang dari tulangku, dan daging dari dagingku. Kunamakan dia perempuan, karena ia diambil dari laki-laki."
\par 24 Itulah sebabnya orang laki-laki meninggalkan ayah dan ibunya, dan bersatu dengan istrinya, lalu keduanya menjadi satu.
\par 25 Laki-laki dan perempuan itu telanjang, tetapi mereka tidak merasa malu.

\chapter{3}

\par 1 Ular adalah binatang yang paling licik dari segala binatang yang dibuat oleh TUHAN Allah. Ular itu bertanya kepada perempuan itu, "Apakah Allah benar-benar melarang kalian makan buah-buahan dari segala pohon di taman ini?"
\par 2 "Kami boleh makan buah-buahan dari setiap pohon di dalam taman ini," jawab perempuan itu,
\par 3 "kecuali dari pohon yang ada di tengah-tengah taman. Allah melarang kami makan buah dari pohon itu ataupun menyentuhnya; jika kami melakukannya, kami akan mati."
\par 4 Ular itu menjawab, "Itu tidak benar; kalian tidak akan mati.
\par 5 Allah mengatakan itu karena dia tahu jika kalian makan buah itu, pikiran kalian akan terbuka; kalian akan menjadi seperti Allah dan mengetahui apa yang baik dan apa yang jahat."
\par 6 Perempuan itu melihat bahwa pohon itu indah, dan buahnya nampaknya enak untuk dimakan. Dan ia berpikir alangkah baiknya jika dia menjadi arif. Sebab itu ia memetik buah pohon itu, lalu memakannya, dan memberi juga kepada suaminya, dan suaminya pun memakannya.
\par 7 Segera sesudah makan buah itu, pikiran mereka terbuka dan mereka sadar bahwa mereka telanjang. Sebab itu mereka menutupi tubuh mereka dengan daun ara yang mereka rangkaikan.
\par 8 Petang itu mereka mendengar TUHAN Allah berjalan di dalam taman, lalu mereka berdua bersembunyi di antara pohon-pohon supaya tidak dilihat oleh TUHAN.
\par 9 Tetapi TUHAN Allah berseru kepada laki-laki itu, "Di manakah engkau?"
\par 10 Laki-laki itu menjawab, "Saya mendengar Engkau di taman; saya takut, jadi saya bersembunyi karena telanjang."
\par 11 "Siapa yang mengatakan kepadamu bahwa engkau telanjang?" Allah bertanya. "Apakah engkau makan buah yang Kularang engkau makan itu?"
\par 12 Laki-laki itu menjawab, "Perempuan yang Engkau berikan untuk menemani saya, telah memberi buah itu kepada saya, lalu saya memakannya."
\par 13 TUHAN Allah bertanya kepada perempuan itu, "Mengapa kaulakukan itu?" Jawabnya, "Saya ditipu ular, sehingga saya makan buah itu."
\par 14 Sesudah itu TUHAN Allah berkata kepada ular itu, "Engkau akan dihukum karena perbuatanmu itu; dari segala binatang hanya engkau saja yang harus menanggung kutukan ini: Mulai sekarang engkau akan menjalar dengan perutmu, dan makan debu seumur hidupmu.
\par 15 Engkau dan perempuan itu akan saling membenci, keturunannya dan keturunanmu akan selalu bermusuhan. Keturunannya akan meremukkan kepalamu, dan engkau akan menggigit tumit mereka."
\par 16 Lalu kata TUHAN kepada perempuan itu, "Aku akan menambah kesakitanmu selagi engkau hamil dan pada waktu engkau melahirkan. Tetapi meskipun demikian, engkau masih tetap berahi kepada suamimu, namun engkau akan tunduk kepadanya."
\par 17 Lalu kata TUHAN kepada laki-laki itu, "Engkau mendengarkan kata-kata istrimu lalu makan buah yang telah Kularang engkau makan. Karena perbuatanmu itu, terkutuklah tanah. Engkau harus bekerja keras seumur hidupmu agar tanah ini bisa menghasilkan cukup makanan bagimu.
\par 18 Semak dan duri akan dihasilkan tanah ini bagimu, dan tumbuh-tumbuhan liar akan menjadi makananmu.
\par 19 Engkau akan bekerja dengan susah payah dan berkeringat untuk membuat tanah ini menghasilkan sesuatu, sampai engkau kembali kepada tanah, sebab dari tanahlah engkau dibentuk. Engkau dijadikan dari tanah, dan akan kembali ke tanah."
\par 20 Adam menamakan istrinya Hawa, karena perempuan itu menjadi ibu seluruh umat manusia.
\par 21 Maka TUHAN Allah membuat pakaian dari kulit binatang untuk Adam dan istrinya, lalu mengenakan-Nya kepada mereka.
\par 22 TUHAN Allah berkata, "Sekarang manusia telah menjadi seperti Kita dan mempunyai pengetahuan tentang yang baik dan yang jahat. Jadi perlu dicegah dia makan buah pohon yang memberi hidup, supaya dia jangan hidup untuk selama-lamanya."
\par 23 Maka TUHAN Allah mengusir manusia dari taman Eden dan menyuruhnya mengusahakan tanah yang menjadi asalnya itu.
\par 24 Kemudian, di sebelah timur taman itu di depan pintu masuk, TUHAN Allah menempatkan kerub-kerub, dan sebilah pedang berapi yang berputar ke segala arah, untuk menjaga jalan ke pohon yang memberi hidup itu. Dengan demikian tak seorang pun dapat masuk dan mendekati pohon itu.

\chapter{4}

\par 1 Kemudian Adam bersetubuh dengan Hawa, istrinya, dan hamillah wanita itu. Ia melahirkan seorang anak laki-laki dan berkata, "Dengan pertolongan TUHAN aku telah mendapat seorang anak laki-laki." Maka dinamakannya anak itu Kain.
\par 2 Lalu Hawa melahirkan seorang anak laki-laki lagi, namanya Habel. Habel menjadi gembala domba, tetapi Kain menjadi petani.
\par 3 Beberapa waktu kemudian Kain mengambil sebagian dari panenannya lalu mempersembahkannya kepada TUHAN.
\par 4 Lalu Habel mengambil anak domba yang sulung dari salah seekor dombanya, menyembelihnya, lalu mempersembahkan bagian yang paling baik kepada TUHAN. TUHAN senang kepada Habel dan persembahannya,
\par 5 tetapi menolak Kain dan persembahannya. Kain menjadi marah sekali, dan mukanya geram.
\par 6 Maka berkatalah TUHAN kepada Kain, "Mengapa engkau marah? Mengapa mukamu geram?
\par 7 Jika engkau berbuat baik, pasti engkau tersenyum; tetapi jika engkau berbuat jahat, maka dosa menunggu untuk masuk ke dalam hatimu. Dosa hendak menguasai dirimu, tetapi engkau harus mengalahkannya."
\par 8 Lalu kata Kain kepada Habel, adiknya, "Mari kita pergi ke ladang." Ketika mereka sampai di situ, Kain menyerang dan membunuh Habel adiknya.
\par 9 TUHAN bertanya kepada Kain, "Di mana Habel, adikmu?" Kain menjawab, "Saya tak tahu. Haruskah saya menjaga adik saya?"
\par 10 Lalu TUHAN berkata, "Mengapa engkau melakukan hal yang mengerikan itu? Darah adikmu berseru kepada-Ku dari tanah, seperti suara yang berteriak minta pembalasan.
\par 11 Engkau terkutuk sehingga tak bisa lagi mengusahakan tanah. Tanah itu telah menyerap darah adikmu, seolah-olah dibukanya mulutnya untuk menerima darah adikmu itu ketika engkau membunuhnya.
\par 12 Jika engkau bercocok tanam, tanah tidak akan menghasilkan apa-apa; engkau akan menjadi pengembara yang tidak punya tempat tinggal di bumi."
\par 13 Maka kata Kain kepada TUHAN, "Hukuman itu terlalu berat, saya tak dapat menanggungnya.
\par 14 Engkau mengusir saya dari tanah ini, jauh dari kehadiran-Mu. Saya akan menjadi pengembara yang tidak punya tempat tinggal di bumi, dan saya akan dibunuh oleh siapa saja yang menemukan saya."
\par 15 Tetapi TUHAN berkata, "Tidak. Kalau engkau dibunuh, maka sebagai pembalasan, tujuh orang termasuk pembunuhmu itu akan dibunuh juga." Kemudian TUHAN menaruh tanda pada Kain supaya siapa saja yang bertemu dengan dia jangan membunuhnya.
\par 16 Lalu pergilah Kain dari hadapan TUHAN dan tinggal di tanah yang bernama "Pengembaraan" di sebelah timur Eden.
\par 17 Kain dan istrinya mendapat anak laki-laki, yang diberi nama Henokh. Kemudian Kain mendirikan sebuah kota dan dinamakannya kota itu menurut nama anaknya itu.
\par 18 Henokh ayah Irad. Irad ayah Mehuyael. Mehuyael ayah Metusael, dan Metusael adalah ayah Lamekh.
\par 19 Lamekh mempunyai dua orang istri, Ada dan Zila.
\par 20 Ada melahirkan Yabal, dan keturunan Yabal itulah bangsa yang memelihara ternak dan tinggal dalam kemah.
\par 21 Adiknya bernama Yubal, dan keturunan Yubal adalah pemain musik kecapi dan seruling.
\par 22 Zila melahirkan Tubal-Kain, dan keturunannya membuat segala macam perkakas dari tembaga dan besi. Adik perempuan Tubal-Kain bernama Naama.
\par 23 Berkatalah Lamekh kepada kedua istrinya, "Ada dan Zila, dengarkan! Seorang pemuda kubunuh karena telah menghantam aku.
\par 24 Kalau tujuh orang dibunuh untuk membalas pembunuhan Kain, maka tujuh puluh tujuh orang akan dibunuh kalau aku dibunuh."
\par 25 Adam dan istrinya mendapat seorang anak laki-laki lagi. Kata Hawa, "Allah telah memberi aku anak laki-laki sebagai ganti Habel, yang telah dibunuh oleh Kain." Sebab itu Hawa menamakan anak itu Set.
\par 26 Set mempunyai anak laki-laki yang diberi nama Enos. Pada zaman itulah orang mulai menyebut nama TUHAN bila menyembah.

\chapter{5}

\par 1 Inilah daftar keturunan Adam. (Pada waktu Allah menciptakan manusia, dijadikan-Nya mereka seperti Allah sendiri.
\par 2 Diciptakan-Nya mereka laki-laki dan perempuan. Diberkati-Nya mereka dan dinamakan-Nya mereka "Manusia".)
\par 3 Ketika Adam berumur 130 tahun, ia mendapat anak laki-laki yang mirip dengan dirinya, lalu diberinya nama Set.
\par 4 Setelah itu Adam masih hidup 800 tahun lagi dan mendapat anak-anak lain.
\par 5 Ia meninggal pada usia 930 tahun.
\par 6 Pada waktu Set berumur 105 tahun, ia mendapat anak laki-laki, namanya Enos.
\par 7 Setelah itu Set masih hidup 807 tahun lagi dan mendapat anak-anak lain.
\par 8 Ia meninggal pada usia 912 tahun.
\par 9 Pada waktu Enos berumur 90 tahun, ia mendapat anak laki-laki, namanya Kenan.
\par 10 Setelah itu Enos masih hidup 815 tahun lagi dan mendapat anak-anak lain.
\par 11 Ia meninggal pada usia 905 tahun.
\par 12 Pada waktu Kenan berumur 70 tahun, ia mendapat anak laki-laki, namanya Mahalaleel.
\par 13 Kenan masih hidup 840 tahun lagi dan mendapat anak-anak lain.
\par 14 Ia meninggal pada usia 910 tahun.
\par 15 Pada waktu Mahalaleel berumur 65 tahun, ia mendapat anak laki-laki, namanya Yared.
\par 16 Setelah itu Mahalaleel masih hidup 830 tahun lagi dan mendapat anak-anak lain.
\par 17 Ia meninggal pada usia 895 tahun.
\par 18 Pada waktu Yared berumur 162 tahun, ia mendapat anak laki-laki, namanya Henokh.
\par 19 Setelah itu Yared masih hidup 800 tahun lagi dan mendapat anak-anak lain.
\par 20 Ia meninggal pada usia 962 tahun.
\par 21 Pada waktu Henokh berumur 65 tahun, ia mendapat anak laki-laki namanya Metusalah.
\par 22 Setelah itu Henokh hidup dalam persekutuan dengan Allah selama 300 tahun. Ia mendapat anak-anak lain
\par 23 dan mencapai umur 365 tahun.
\par 24 Karena Henokh selalu hidup akrab dengan Allah, ia menghilang karena diambil oleh Allah.
\par 25 Pada waktu Metusalah berumur 187 tahun, ia mendapat anak laki-laki, namanya Lamekh.
\par 26 Setelah itu Metusalah masih hidup 782 tahun lagi dan mendapat anak-anak lain.
\par 27 Ia meninggal pada usia 969 tahun.
\par 28 Pada waktu Lamekh berumur 182 tahun, ia mendapat anak laki-laki.
\par 29 Lamekh berkata, "Anak ini akan memberi keringanan pada waktu kita bekerja keras mengolah tanah yang dikutuk TUHAN." Karena itu Lamekh menamakan anak itu Nuh.
\par 30 Setelah itu Lamekh masih hidup 595 tahun lagi. Ia mendapat anak-anak lain,
\par 31 dan meninggal pada usia 777 tahun.
\par 32 Setelah Nuh berumur 500 tahun, ia mendapat tiga anak laki-laki, yaitu Sem, Yafet dan Ham.

\chapter{6}

\par 1 Setelah manusia bertambah banyak dan tersebar di seluruh dunia, dan anak-anak perempuan dilahirkan,
\par 2 makhluk-makhluk ilahi melihat bahwa gadis-gadis itu cantik-cantik. Lalu mereka mengawini gadis-gadis yang mereka sukai.
\par 3 Maka berkatalah TUHAN, "Aku tidak memperkenankan manusia hidup selama-lamanya; mereka makhluk fana, yang harus mati. Mulai sekarang umur mereka tidak akan melebihi 120 tahun."
\par 4 Pada zaman itu, dan juga sesudahnya, ada orang-orang raksasa di bumi. Mereka keturunan gadis-gadis manusia yang kawin dengan makhluk-makhluk ilahi. Orang-orang raksasa itu adalah pahlawan-pahlawan besar dan orang-orang termasyhur di zaman purbakala.
\par 5 TUHAN melihat betapa jahatnya orang-orang di bumi; semua pikiran mereka selalu jahat.
\par 6 Ia pun menyesal telah menjadikan mereka dan menempatkan mereka di bumi. Ia begitu kecewa,
\par 7 sehingga berkata, "Akan Kubinasakan manusia yang telah Kuciptakan itu, dan juga segala burung dan binatang lainnya, sebab Aku menyesal telah menciptakan mereka."
\par 8 Tetapi Nuh menyenangkan hati TUHAN.
\par 9 Inilah riwayat Nuh. Ia mempunyai tiga anak laki-laki, yaitu Sem, Yafet dan Ham. Nuh tidak berbuat salah, dan dia satu-satunya orang yang baik pada zamannya. Ia hidup akrab dengan Allah.
\par 11 Tetapi semua orang lainnya jahat dalam pandangan Allah, dan kekejaman terdapat di mana-mana.
\par 12 Allah memandang dunia itu dan hanya melihat kejahatan saja, sebab semua manusia jahat hidupnya.
\par 13 Lalu berkatalah Allah kepada Nuh, "Aku telah memutuskan untuk mengakhiri hidup segala makhluk. Aku akan memusnahkan mereka beserta bumi, karena bumi telah penuh dengan kekejaman mereka.
\par 14 Buatlah sebuah kapal untukmu dari kayu yang kuat; buatlah bilik-bilik di dalamnya, dan lapisilah dengan ter dari dalam dan dari luar.
\par 15 Kapal itu harus 133 meter panjangnya, 22 meter lebarnya, dan 13 meter tingginya.
\par 16 Buatlah atap pada kapal itu, dan berilah jarak sebesar 44 sentimeter di antara atap dan dinding-dindingnya. Buatlah kapal itu bertingkat tiga, dan pasanglah sebuah pintu di sisinya.
\par 17 Aku akan mendatangkan banjir untuk membinasakan setiap makhluk yang hidup di bumi. Segala sesuatu di bumi akan mati,
\par 18 tetapi dengan engkau Aku hendak membuat perjanjian. Masuklah ke dalam kapal itu bersama-sama dengan istrimu, dan anak-anakmu serta istri-istri mereka.
\par 19 Bawalah ke dalam kapal itu seekor jantan dan seekor betina dari setiap jenis burung dan binatang lainnya, supaya mereka tidak turut binasa.
\par 21 Bawalah juga segala macam makanan untukmu dan untuk binatang-binatang itu."
\par 22 Nuh melakukan segala sesuatu yang diperintahkan Allah kepadanya.

\chapter{7}

\par 1 Lalu berkatalah TUHAN kepada Nuh, "Aku melihat bahwa engkau satu-satunya orang yang melakukan kehendak-Ku. Jadi, masuklah ke dalam kapal itu bersama-sama dengan seluruh keluargamu.
\par 2 Bawalah juga tujuh pasang dari setiap jenis burung dan binatang lainnya yang halal, sedangkan dari yang haram hanya satu pasang saja dari setiap jenis. Lakukanlah itu supaya dari setiap jenis binatang ada yang luput dari kebinasaan dan bisa berkembang biak lagi di bumi.
\par 4 Tujuh hari lagi Aku akan menurunkan hujan yang tidak akan reda selama empat puluh hari empat puluh malam, supaya makhluk hidup yang telah Kuciptakan itu binasa."
\par 5 Lalu Nuh melakukan segala yang diperintahkan TUHAN kepadanya.
\par 6 Nuh berumur 600 tahun ketika terjadi banjir di bumi.
\par 7 Nuh dan istrinya, dan anak-anaknya beserta istri-istri mereka, masuk ke dalam kapal itu untuk menyelamatkan diri dari banjir.
\par 8 Seekor jantan dan seekor betina dari setiap jenis burung dan binatang lainnya--baik yang halal maupun yang haram--
\par 9 masuk ke dalam kapal itu bersama-sama dengan Nuh, sesuai dengan perintah Allah.
\par 10 Tujuh hari kemudian banjir datang melanda bumi.
\par 11 Pada waktu Nuh berumur 600 tahun, pada tanggal tujuh belas bulan dua, pecahlah segala mata air di bawah bumi. Segala pintu air di langit terbuka,
\par 12 dan hujan turun selama empat puluh hari empat puluh malam.
\par 13 Pada hari itu juga, Nuh dan istrinya masuk ke dalam kapal itu bersama ketiga anaknya, yaitu Sem, Yafet dan Ham beserta istri-istri mereka.
\par 14 Bersama-sama dengan mereka masuk pula setiap jenis burung dan binatang lainnya, baik yang jinak maupun yang liar, yang besar maupun yang kecil.
\par 15 Seekor jantan dan seekor betina dari setiap jenis makhluk hidup masuk ke dalam kapal itu bersama-sama dengan Nuh,
\par 16 sesuai dengan perintah Allah kepadanya. Setelah semuanya masuk, TUHAN menutup pintu kapal.
\par 17 Banjir itu terus melanda selama empat puluh hari, dan air menjadi cukup tinggi sehingga kapal itu dapat mengapung.
\par 18 Air semakin tinggi, dan kapal itu terapung-apung pada permukaan air.
\par 19 Air itu terus bertambah tinggi, sehingga tergenanglah gunung-gunung yang paling tinggi.
\par 20 Air terus naik sampai mencapai ketinggian tujuh meter di atas puncak-puncak gunung.
\par 21 TUHAN membinasakan segala makhluk yang hidup di bumi ini: Manusia, burung dan binatang darat baik kecil maupun besar. Yang tidak binasa hanyalah Nuh dan semua yang ada bersama-sama dengan dia di dalam kapal itu.
\par 24 Air itu tidak kunjung surut selama 150 hari.

\chapter{8}

\par 1 Allah tidak melupakan Nuh dan segala binatang yang ada bersamanya di dalam kapal itu. Allah membuat angin bertiup, sehingga air itu mulai surut.
\par 2 Semua mata air di bawah bumi dan semua pintu air di langit ditutupnya. Hujan berhenti,
\par 3 dan air semakin surut. Sesudah 150 hari air tidak begitu tinggi lagi.
\par 4 Pada tanggal tujuh belas bulan tujuh, kapal itu kandas di sebuah puncak di pegunungan Ararat.
\par 5 Air terus surut dan pada tanggal satu bulan sepuluh, puncak-puncak gunung mulai tampak.
\par 6 Setelah empat puluh hari, Nuh membuka sebuah jendela kapal,
\par 7 dan melepaskan seekor burung gagak. Burung itu tidak kembali ke kapal melainkan terus terbang kian kemari sampai air banjir sudah surut sama sekali.
\par 8 Sementara itu, Nuh melepaskan seekor burung merpati untuk mengetahui apakah air itu memang telah surut.
\par 9 Tetapi karena air masih menutupi seluruh muka bumi, burung merpati itu tidak menemukan tempat untuk bertengger. Maka kembalilah ia ke kapal; Nuh mengulurkan tangannya lalu membawanya masuk.
\par 10 Nuh menunggu tujuh hari lagi, lalu melepaskan lagi burung merpati itu.
\par 11 Pada petang hari burung itu kembali kepadanya membawa sehelai daun zaitun yang segar pada paruhnya. Sekarang Nuh tahu bahwa air telah surut.
\par 12 Setelah menunggu tujuh hari lagi, ia melepaskan merpati itu sekali lagi; dan kali itu burung itu tidak kembali kepadanya.
\par 13 Pada waktu Nuh berumur 601 tahun, pada tanggal satu bulan satu, air sudah surut sama sekali. Nuh membuka atap kapal itu, dan melihat ke sekelilingnya. Ia melihat bahwa permukaan tanah sudah kering.
\par 14 Pada tanggal dua puluh tujuh bulan dua, bumi sudah kering.
\par 15 Lalu berkatalah Allah kepada Nuh,
\par 16 "Keluarlah dari kapal itu bersama-sama dengan istrimu, anak-anakmu dan istri-istri mereka.
\par 17 Bawalah keluar semua burung dan binatang lainnya, besar maupun kecil, supaya mereka bisa berkembang biak dan menyebar ke seluruh bumi."
\par 18 Lalu keluarlah Nuh dari kapal itu bersama-sama dengan istrinya, anak-anaknya dan istri-istri mereka.
\par 19 Semua burung dan binatang darat keluar dari kapal itu, masing-masing bersama kelompok sejenisnya.
\par 20 Nuh mendirikan sebuah mezbah untuk TUHAN. Diambilnya seekor dari setiap jenis burung dan binatang lainnya yang halal, lalu dipersembahkannya sebagai kurban bakaran di atas mezbah itu.
\par 21 Bau harum kurban persembahan itu menyenangkan hati TUHAN, dan Ia berkata di dalam hati, "Aku tak akan lagi mengutuk dunia ini karena perbuatan manusia; Aku tahu bahwa sejak masa mudanya, pikiran manusia itu jahat. Aku tak akan pernah lagi membinasakan segala makhluk yang hidup seperti yang baru Kulakukan ini.
\par 22 Selama dunia ini ada, selalu akan ada masa menanam dan masa menuai, musim dingin dan musim panas, musim kemarau dan musim hujan, siang dan malam."

\chapter{9}

\par 1 Allah memberkati Nuh dan anak-anaknya serta berkata, "Beranakcuculah yang banyak, supaya keturunanmu mendiami seluruh bumi.
\par 2 Segala burung dan ikan serta binatang yang lain akan takut kepadamu. Mereka semua ada dalam kekuasaanmu.
\par 3 Semuanya itu boleh menjadi makananmu, seperti juga tumbuh-tumbuhan hijau; semuanya itu Kuberikan kepadamu untuk menjadi makananmu.
\par 4 Satu-satunya yang tidak boleh kamu makan ialah daging yang masih ada darahnya, sebab nyawa itu ada di dalam darah.
\par 5 Setiap orang dan binatang yang membunuh manusia akan Kuhukum mati.
\par 6 Manusia diciptakan seperti Aku; sebab itu, barangsiapa membunuh manusia, dia sendiri akan dibunuh juga oleh manusia.
\par 7 Ya, kamu harus beranak cucu yang banyak, supaya keturunanmu mendiami seluruh bumi."
\par 8 Allah berkata kepada Nuh dan anak-anaknya,
\par 9 "Sekarang Aku membuat perjanjian-Ku dengan kamu dan dengan keturunanmu,
\par 10 dan dengan segala makhluk yang hidup, yaitu burung-burung dan semua binatang darat, ya semuanya yang keluar dari kapal itu bersama-sama dengan kamu.
\par 11 Inilah perjanjian-Ku dengan kamu: Aku berjanji bahwa segala makhluk yang hidup tidak akan lagi dibinasakan oleh banjir. Tidak akan lagi ada banjir yang membinasakan bumi ini.
\par 12 Sebagai tanda perjanjian kekal, yang Kubuat dengan kamu dan dengan segala makhluk yang hidup,
\par 13 maka Kutaruh pelangi-Ku di awan sebagai tanda perjanjian-Ku dengan dunia.
\par 14 Setiap kali, jika Aku menutupi langit dengan awan, lalu pelangi itu tampak,
\par 15 Aku akan mengingat janji-Ku kepadamu dan kepada segala makhluk hidup, yaitu bahwa banjir tidak akan lagi membinasakan segala yang hidup.
\par 16 Bilamana pelangi tampak di awan, Aku akan melihatnya dan mengingat perjanjian yang kekal itu antara Aku dengan segala makhluk yang hidup di bumi.
\par 17 Itulah tanda janji-Ku yang Kuberikan kepada segala makhluk yang hidup di bumi."
\par 18 Anak-anak Nuh yang keluar dari kapal itu ialah Sem, Yafet dan Ham. (Ham adalah ayah Kanaan.)
\par 19 Ketiga anak Nuh itu adalah nenek moyang semua orang di dunia.
\par 20 Nuh seorang petani, dan dialah yang pertama-tama membuat kebun anggur.
\par 21 Setelah Nuh minum anggurnya, ia menjadi mabuk. Dilepaskannya segala pakaiannya lalu tidurlah ia telanjang di dalam kemahnya.
\par 22 Ketika Ham, yaitu ayah Kanaan, melihat bahwa ayahnya telanjang, ia keluar dan memberitahukan hal itu kepada kedua saudaranya.
\par 23 Kemudian Sem dan Yafet mengambil sehelai jubah dan membentangkannya pada bahu mereka. Mereka berjalan mundur memasuki kemah itu dan menyelimuti ayah mereka dengan jubah itu. Mereka memalingkan muka supaya tidak melihat ayah mereka yang telanjang itu.
\par 24 Setelah Nuh sadar dari mabuknya dan mengetahui apa yang diperbuat anak bungsunya terhadap dirinya,
\par 25 ia berkata, "Terkutuklah Kanaan! Dia akan menjadi budak terhina bagi saudara-saudaranya.
\par 26 Pujilah TUHAN, Allah Sem! Kanaan akan menjadi budak Sem.
\par 27 Semoga Allah menambahkan berkat kepada Yafet dengan meluaskan tempat kediamannya. Semoga keturunannya tinggal bersama-sama dengan keturunan Sem. Kanaan akan menjadi budak Yafet."
\par 28 Sesudah banjir itu, Nuh masih hidup 350 tahun lagi.
\par 29 Ia meninggal pada usia 950 tahun.

\chapter{10}

\par 1 Inilah keturunan anak-anak Nuh, yaitu Sem, Yafet dan Ham. Sesudah banjir, ketiganya mendapat anak-anak lelaki.
\par 2 Anak-anak Yafet ialah Gomer, Magog, Madai, Yawan, Tubal, Mesekh dan Tiras.
\par 3 Keturunan Gomer adalah penduduk Askenas, Rifat dan Togarma.
\par 4 Keturunan Yawan adalah penduduk Elisa, Spanyol, Siprus dan Rodes.
\par 5 Mereka leluhur bangsa-bangsa yang tinggal di sepanjang pantai dan di pulau-pulau. Itulah semua keturunan Yafet. Setiap bangsa dan suku tinggal di negerinya masing-masing, dan mempunyai bahasanya sendiri.
\par 6 Anak-anak Ham ialah Kus, Mesir, Libia dan Kanaan.
\par 7 Keturunan Kus ialah penduduk Seba, Hawila, Sabta, Raema dan Sabtekha. Keturunan Raema ialah penduduk Syeba dan Dedan.
\par 8 Kus mempunyai anak laki-laki bernama Nimrod yang menjadi orang perkasa pertama di dunia.
\par 9 Dengan pertolongan TUHAN dia menjadi pemburu yang ulung. Sebab itu orang biasa berkata, "Semoga TUHAN menjadikan engkau seorang pemburu yang ulung seperti Nimrod."
\par 10 Mula-mula kerajaannya meliputi Babel, Erekh, dan Akad, ketiga-tiganya di Babilonia.
\par 11 Dari sana ia pergi ke Asyur, lalu mendirikan kota Niniwe, Rehobot-Ir, Kalah,
\par 12 dan Resen yang terletak di antara Niniwe dan Kalah. Semuanya itu adalah kota-kota besar.
\par 13 Keturunan Mesir adalah penduduk Lidia, Anamim, Lehabim, Naftuhim,
\par 14 Patrusim, Kasluhim, dan Kaftorim. Bangsa Kasluhim itu leluhur bangsa Filistin.
\par 15 Anak-anak Kanaan, ialah Sidon, yang sulung, dan Het.
\par 16 Kanaan juga leluhur bangsa Yebusi, Amori, Girgasi,
\par 17 orang Hewi, Arki, Sini,
\par 18 Arwadi, Semari dan Hamati. Suku-suku bangsa Kanaan itu tersebar jauh,
\par 19 sehingga batas-batas negeri mereka meluas dari Sidon ke selatan menuju Gerar, sampai dekat Gaza, dan ke timur menuju Sodom, Gomora, Adma, dan Zeboim, sampai dekat Lasa.
\par 20 Itulah semua keturunan Ham. Setiap bangsa dan suku tinggal di negerinya masing-masing dan mempunyai bahasanya sendiri.
\par 21 Sem, abang Yafet, adalah leluhur semua bangsa Ibrani.
\par 22 Anak-anak Sem ialah Elam, Asyur, Arpakhsad, Lud dan Aram.
\par 23 Keturunan Aram ialah penduduk Us, Hul, Geter dan Mas.
\par 24 Arpakhsad ayah Selah. Dan Selah ayah Eber.
\par 25 Eber mempunyai dua anak laki-laki; yang pertama bernama Peleg, karena pada zamannya bangsa-bangsa di dunia menjadi terbagi-bagi; yang kedua bernama Yoktan.
\par 26 Keturunan Yoktan adalah penduduk Almodad, Selef, Hazar-Mawet dan Yerah,
\par 27 Hadoram, Uzal, Dikla,
\par 28 Obal, Abimael, Syeba,
\par 29 Ofir, Hawila dan Yobab. Semuanya itu keturunan Yoktan.
\par 30 Daerah tempat tinggal mereka meluas dari Mesa ke Sefar, yaitu daerah pegunungan di sebelah timur.
\par 31 Itulah semua keturunan Sem. Setiap bangsa dan suku tinggal di negerinya masing-masing dan mempunyai bahasanya sendiri.
\par 32 Semuanya itu keturunan Nuh, bangsa demi bangsa, menurut garis keturunan mereka masing-masing. Merekalah yang menurunkan bangsa-bangsa di dunia ini sesudah banjir besar itu.

\chapter{11}

\par 1 Semula, bangsa-bangsa di seluruh dunia hanya mempunyai satu bahasa dan mereka memakai kata-kata yang sama.
\par 2 Ketika mereka mengembara ke sebelah timur, sampailah mereka di sebuah dataran di Babilonia, lalu menetap di sana.
\par 3 Mereka berkata seorang kepada yang lain, "Ayo kita membuat batu bata dan membakarnya sampai keras." Demikianlah mereka mempunyai batu bata untuk batu rumah dan ter untuk bahan perekatnya.
\par 4 Kata mereka, "Mari kita mendirikan kota dengan sebuah menara yang puncaknya sampai ke langit, supaya kita termasyhur dan tidak tercerai berai di seluruh bumi."
\par 5 Maka turunlah TUHAN untuk melihat kota dan menara yang didirikan oleh manusia.
\par 6 Lalu Ia berkata, "Mereka ini satu bangsa dengan satu bahasa, dan ini baru permulaan dari rencana-rencana mereka. Tak lama lagi mereka akan sanggup melakukan apa saja yang mereka kehendaki.
\par 7 Sebaiknya Kita turun dan mengacaukan bahasa mereka supaya mereka tidak mengerti lagi satu sama lain."
\par 8 Demikianlah TUHAN menceraiberaikan mereka ke seluruh bumi. Lalu berhentilah mereka mendirikan kota itu.
\par 9 Sebab itu kota itu diberi nama Babel, karena di situ TUHAN mengacaukan bahasa semua bangsa, dan dari situ mereka diceraiberaikan oleh TUHAN ke seluruh bumi.
\par 10 Inilah keturunan Sem. Dua tahun sesudah banjir besar, ketika Sem berumur 100 tahun, ia mendapat seorang anak laki-laki yang bernama Arpakhsad.
\par 11 Setelah itu ia masih hidup 500 tahun lagi, dan mendapat anak-anak lain.
\par 12 Pada waktu Arpakhsad berumur 35 tahun, ia mendapat anak laki-laki, namanya Selah.
\par 13 Setelah itu Arpakhsad masih hidup 403 tahun lagi, dan mendapat anak-anak lain.
\par 14 Pada waktu Selah berumur 30 tahun, ia mendapat anak laki-laki, namanya Eber.
\par 15 Setelah itu Selah masih hidup 403 tahun lagi, dan mendapat anak-anak lain.
\par 16 Pada waktu Eber berumur 34 tahun, ia mendapat anak laki-laki, namanya Peleg.
\par 17 Setelah itu Eber masih hidup 430 tahun lagi, dan mendapat anak-anak lain.
\par 18 Pada waktu Peleg berumur 30 tahun, ia mendapat anak laki-laki, namanya Rehu.
\par 19 Setelah itu Peleg masih hidup 209 tahun lagi, dan mendapat anak-anak lain.
\par 20 Pada waktu Rehu berumur 32 tahun, ia mendapat anak laki-laki, namanya Serug.
\par 21 Setelah itu Rehu masih hidup 207 tahun lagi dan mendapat anak-anak lain.
\par 22 Pada waktu Serug berumur 30 tahun, ia mendapat anak laki-laki, namanya Nahor.
\par 23 Setelah itu Serug masih hidup 200 tahun lagi, dan mendapat anak-anak lain.
\par 24 Pada waktu Nahor berumur 29 tahun, ia mendapat anak laki-laki, namanya Terah.
\par 25 Setelah itu Nahor masih hidup 119 tahun lagi, dan mendapat anak-anak lain.
\par 26 Setelah Terah berumur 70 tahun, ia mendapat tiga anak laki-laki, yaitu Abram, Nahor dan Haran.
\par 27 Inilah keturunan Terah: Terah ayah Abram, Nahor dan Haran. Haran mempunyai anak laki-laki, namanya Lot.
\par 28 Haran meninggal di kampung halamannya, yaitu Ur di Babilonia, pada waktu ayahnya masih hidup.
\par 29 Abram kawin dengan Sarai, dan Nahor kawin dengan Milka anak perempuan Haran. Haran masih mempunyai anak laki-laki lain namanya Yiska.
\par 30 Adapun Sarai mandul.
\par 31 Terah meninggalkan kota Ur di Babilonia bersama-sama dengan Abram anaknya, Lot cucunya, dan Sarai menantunya, yaitu istri Abram. Dia bermaksud hendak pergi ke negeri Kanaan. Tetapi setibanya di Haran, mereka menetap di sana.
\par 32 Dan di tempat itu pula Terah meninggal pada usia 205 tahun.

\chapter{12}

\par 1 TUHAN berkata kepada Abram, "Tinggalkanlah negerimu, kaum keluargamu dan rumah ayahmu, lalu pergilah ke negeri yang akan Kutunjukkan kepadamu.
\par 2 Aku akan memberikan kepadamu keturunan yang banyak dan mereka akan menjadi bangsa yang besar. Aku akan memberkati engkau dan membuat namamu masyhur, sehingga engkau akan menjadi berkat.
\par 3 Aku akan memberkati orang-orang yang memberkati engkau dan mengutuk orang-orang yang mengutuk engkau. Dan karena engkau Aku akan memberkati semua bangsa di bumi."
\par 4 Abram berusia tujuh puluh lima tahun ketika ia meninggalkan Haran, sesuai dengan perintah TUHAN kepadanya. Abram berangkat ke tanah Kanaan bersama-sama dengan Sarai istrinya, dan Lot kemanakannya, dan segala harta benda serta hamba-hamba yang mereka peroleh di Haran. Setelah mereka tiba di Kanaan,
\par 6 Abram menjelajahi tanah itu sampai ia tiba di pohon keramat di More, yaitu tempat ibadat di Sikhem. (Pada masa itu orang Kanaan masih mendiami tanah itu.)
\par 7 TUHAN menampakkan diri kepada Abram dan berkata kepadanya, "Inilah negeri yang akan Kuberikan kepada keturunanmu." Lalu Abram mendirikan sebuah mezbah di situ untuk TUHAN yang telah menampakkan diri kepadanya.
\par 8 Setelah itu ia meneruskan perjalanannya ke selatan, ke daerah berbukit di sebelah timur kota Betel, dan berkemah di antara Betel dan kota Ai: Betel di sebelah barat dan Ai di sebelah timur. Juga di situ ia mendirikan mezbah dan menyembah TUHAN.
\par 9 Kemudian ia meneruskan lagi perjalanannya dari satu tempat ke tempat berikutnya, menuju ke bagian selatan tanah Kanaan.
\par 10 Tetapi di Kanaan sedang ada kelaparan yang sangat hebat, sehingga Abram pergi lebih jauh lagi ke selatan, ke negeri Mesir untuk tinggal di sana sampai bencana itu lewat.
\par 11 Pada waktu ia hendak melintasi perbatasan dan masuk ke negeri Mesir, berkatalah ia kepada Sarai istrinya, "Engkau cantik, istriku.
\par 12 Kalau orang Mesir melihatmu, mereka akan menduga bahwa engkau istriku lalu saya pasti dibunuh dan engkau dibiarkan hidup.
\par 13 Jadi sebaiknya kaukatakan saja bahwa engkau adik saya supaya saya dibiarkan hidup dan diperlakukan dengan baik karena engkau."
\par 14 Dan benarlah, setelah Abram melintasi perbatasan dan sampai di negeri Mesir, orang Mesir melihat bahwa Sarai cantik sekali.
\par 15 Beberapa orang pegawai istana melihat dia dan memberitahukan kepada raja betapa cantiknya wanita itu; sebab itu dia dibawa ke istana raja.
\par 16 Demi Sarai, raja memperlakukan Abram dengan baik dan memberikan kepadanya hamba-hamba, kawanan domba dan kambing, sapi, keledai dan unta.
\par 17 Tetapi karena raja mengambil Sarai, TUHAN mendatangkan penyakit-penyakit yang mengerikan atas raja dan orang-orang di dalam istananya.
\par 18 Lalu raja memanggil Abram dan bertanya kepadanya, "Apa yang telah kaulakukan terhadap aku ini? Mengapa tidak kauberitahukan bahwa ia istrimu?
\par 19 Mengapa kaukatakan bahwa dia adikmu dan membiarkan aku mengambilnya menjadi istriku? Ini dia, istrimu; ambillah dan pergilah!"
\par 20 Raja memberi perintah kepada beberapa pegawainya, dan mereka mengeluarkan Abram dari negeri itu bersama-sama dengan istrinya dan segala miliknya.

\chapter{13}

\par 1 Abram meninggalkan Mesir dan pergi ke arah utara, menuju ke bagian selatan Kanaan dengan istrinya serta segala miliknya, dan Lot ikut juga.
\par 2 Abram kaya raya; ia memiliki domba, kambing, sapi, juga perak dan emas.
\par 3 Maka pergilah ia dari satu tempat ke tempat yang lain, menuju ke Betel. Ia sampai ke daerah di antara Betel dan Ai, di tempat keramat di mana ia dahulu berkemah
\par 4 dan mendirikan mezbah. Di situ ia menyembah TUHAN.
\par 5 Lot juga mempunyai keluarga dan hamba-hamba serta domba, kambing dan sapi.
\par 6 Karena itu tanah di situ tidak cukup padang rumputnya untuk didiami mereka berdua, sebab ternak mereka terlalu banyak.
\par 7 Lalu terjadilah pertengkaran antara para gembala Abram dan para gembala Lot. (Pada masa itu orang Kanaan dan orang Feris masih mendiami tanah itu).
\par 8 Abram berkata kepada Lot, "Kita ini bersaudara, tidak baik jika orang-orangmu dan orang-orangku saling bertengkar.
\par 9 Sebab itu sebaiknya kita berpisah. Pilihlah bagian mana dari tanah ini yang kausukai. Jika engkau pergi ke arah ini, saya akan pergi ke arah yang lain."
\par 10 Lot memandang ke sekitarnya dan dilihatnya bahwa seluruh Lembah Yordan, sampai ke kota Zoar banyak airnya, seperti Taman TUHAN atau seperti tanah Mesir. (Begitulah keadaannya sebelum TUHAN memusnahkan kota-kota Sodom dan Gomora.)
\par 11 Akhirnya Lot memilih seluruh Lembah Yordan itu, lalu berangkat ke timur. Demikianlah kedua orang itu berpisah.
\par 12 Abram tetap tinggal di tanah Kanaan, sedangkan Lot berkemah di kota-kota di lembah sampai dekat Sodom.
\par 13 Kota Sodom itu didiami oleh orang-orang yang sangat jahat dan berdosa terhadap TUHAN.
\par 14 Setelah Lot pergi, TUHAN berkata kepada Abram, "Dari tempat engkau berdiri itu, pandanglah baik-baik ke segala arah;
\par 15 Aku akan memberikan kepadamu dan kepada keturunanmu seluruh tanah yang engkau lihat itu supaya menjadi milikmu selama-lamanya.
\par 16 Aku akan memberikan kepadamu keturunan yang sangat banyak, sehingga tak seorang pun sanggup menghitung mereka. Sebagaimana orang tak dapat menghitung debu di tanah, demikian juga keturunanmu tidak akan dapat dihitung.
\par 17 Sekarang, jelajahilah seluruh tanah ini, sebab Aku akan memberikannya kepadamu."
\par 18 Setelah itu Abram memindahkan perkemahannya lalu menetap di dekat pohon-pohon keramat tempat ibadat di Mamre dekat Hebron, dan di situ ia mendirikan mezbah bagi TUHAN.

\chapter{14}

\par 1 Empat raja, yaitu: Amrafel dari Babilonia, Ariokh dari Elasar, Kedorlaomer dari Elam, dan Tideal dari Goyim
\par 2 berperang melawan lima raja lain, yaitu: Bera dari Sodom, Birsya dari Gomora, Syinab dari Adma, Semeber dari Zeboim, dan raja dari Bela yang disebut juga Zoar.
\par 3 Kelima raja itu bersekutu dan mengumpulkan tentara mereka di Lembah Sidim, yang sekarang disebut Laut Mati.
\par 4 Selama dua belas tahun mereka dikuasai oleh Raja Kedorlaomer, tetapi dalam tahun yang ketiga belas, mereka memberontak melawan dia.
\par 5 Dalam tahun keempat belas, Kedorlaomer dan sekutu-sekutunya datang dengan tentara mereka dan mengalahkan orang Refaim di Asyterot-Karnaim, orang Zuzim di Ham, orang Emim di Syawe-Kiryataim,
\par 6 dan orang Hori di pegunungan Edom, lalu mengejar mereka sejauh El-Paran di pinggir padang gurun.
\par 7 Setelah itu para pengejar balik lagi dan kembali ke En-Mispat yang sekarang disebut Kades. Mereka menaklukkan seluruh tanah Amalek serta mengalahkan orang Amori yang tinggal di Hazezon-Tamar.
\par 8 Kemudian raja-raja dari Sodom, Gomora, Adma, Zeboim dan Bela mengatur barisan tentara mereka di Lembah Sidim lalu berperang
\par 9 melawan raja-raja dari Elam, Goyim, Babilonia, dan Elasar; jadi, lima raja melawan empat.
\par 10 Lembah itu penuh dengan sumur aspal, dan ketika raja-raja dari Sodom dan Gomora berusaha melarikan diri dari peperangan, mereka jatuh ke dalam sumur-sumur itu. Ketiga raja lainnya berhasil lolos dan lari ke pegunungan.
\par 11 Keempat raja yang menang, merampas segala-galanya di Sodom dan Gomora, termasuk bahan makanan, lalu pergi.
\par 12 Lot kemanakan Abram juga bertempat tinggal di Sodom, karena itu ia pun dibawa oleh musuh beserta segala harta bendanya.
\par 13 Tetapi seorang yang berhasil lolos melaporkan semua kejadian itu kepada Abram, orang Ibrani itu. Abram tinggal dekat pohon-pohon keramat tempat ibadat milik Mamre, seorang Amori. Mamre dan saudara-saudaranya, yaitu Eskol dan Aner adalah teman-teman sekutu Abram.
\par 14 Ketika Abram mendengar bahwa kemanakannya tertawan, dikumpulkannya semua orang di dalam perkemahannya, yakni yang pandai berperang, sebanyak 318 orang, lalu mengejar keempat raja itu sampai ke kota Dan.
\par 15 Di situ Abram membagi orang-orangnya atas beberapa kelompok, lalu menyerang musuh pada waktu malam dan mengalahkan mereka. Ia mengejar mereka sampai ke Hoba di sebelah utara kota Damsyik.
\par 16 Abram berhasil merebut kembali barang-barang yang telah dirampas, juga Lot dengan segala harta bendanya, dan para wanita dan tawanan lainnya.
\par 17 Sesudah mengalahkan Raja Kedorlaomer dengan raja-raja sekutunya, Abram kembali. Lalu raja Sodom pergi menyongsong dia di Lembah Syawe yang disebut juga Lembah Raja.
\par 18 Melkisedek, raja Salem yang juga menjabat imam Allah Yang Mahatinggi, membawa roti dan anggur untuk Abram,
\par 19 lalu memberkati Abram, katanya, "Semoga Allah Yang Mahatinggi, yang telah menciptakan langit dan bumi, memberkati Abram!
\par 20 Terpujilah Allah Yang Mahatinggi, yang telah memberikan kepadamu kemenangan atas musuhmu." Setelah itu Abram memberikan kepada Melkisedek sepersepuluh dari segala barang rampasan yang telah dibawanya kembali.
\par 21 Lalu berkatalah raja Sodom kepada Abram, "Ambillah barang rampasan itu, tetapi sudilah mengembalikan kepadaku semua orang-orangku."
\par 22 Tetapi Abram menjawab, "Saya bersumpah di hadapan TUHAN, Allah Yang Mahatinggi, Pencipta langit dan bumi,
\par 23 bahwa saya tak akan mengambil apa-apa dari milikmu, bahkan sehelai benang atau sepotong tali sandal pun tidak. Dengan begitu engkau tidak akan dapat berkata, 'Sayalah yang membuat Abram menjadi kaya.'
\par 24 Saya tidak mau mengambil apa-apa untuk saya sendiri. Saya hanya mau menerima apa yang telah dipakai oleh orang-orang saya. Tetapi biarlah teman-teman sekutu saya, yakni Aner, Eskol dan Mamre, mengambil bagian mereka."

\chapter{15}

\par 1 Setelah itu, Abram menerima penglihatan dan mendengar TUHAN berkata kepadanya, "Jangan takut, Abram, Aku akan melindungi engkau dari bahaya, dan memberikan kepadamu upah yang besar."
\par 2 Tetapi Abram berkata, "TUHAN Yang Mahatinggi, TUHAN tidak memberikan anak kepada saya. Orang yang akan mewarisi harta saya hanyalah Eliezer, hamba saya dari Damsyik. Jadi apa gunanya TUHAN memberi upah kepada saya?"
\par 4 Kemudian Abram mendengar TUHAN berkata lagi kepadanya, "Bukan hambamu yang akan menjadi ahli warismu, melainkan anak laki-lakimu sendiri."
\par 5 TUHAN membawa Abram ke luar lalu berkata kepadanya, "Pandanglah langit, dan cobalah menghitung bintang-bintang; engkau akan mempunyai keturunan sebanyak bintang-bintang itu."
\par 6 Abram percaya kepada TUHAN, dan karena itu TUHAN menerima dia sebagai orang yang menyenangkan hati-Nya.
\par 7 Kemudian TUHAN berkata lagi kepadanya, "Akulah TUHAN, yang telah memimpin engkau keluar dari negeri Ur di Babilonia, untuk memberikan tanah ini kepadamu menjadi milikmu."
\par 8 Tetapi Abram bertanya, "Ya, TUHAN Yang Mahatinggi, bagaimana saya dapat mengetahui bahwa tanah itu akan menjadi milik saya?"
\par 9 Jawab TUHAN, "Ambillah untuk-Ku seekor sapi betina, seekor kambing betina dan seekor kambing jantan, masing-masing berumur tiga tahun, dan juga seekor burung tekukur dan seekor burung merpati."
\par 10 Lalu Abram mengambil binatang-binatang itu bagi Allah. Sapi, kambing dan domba itu masing-masing dibelahnya menjadi dua, dan belahan-belahan itu diletakkannya saling berhadapan dalam dua deret. Tetapi burung-burung itu tidak dibelahnya.
\par 11 Daging itu dihinggapi burung-burung pemakan bangkai, tetapi Abram mengusirnya.
\par 12 Ketika matahari mulai terbenam, Abram tertidur nyenyak. Tiba-tiba ia diliputi rasa takut yang amat sangat.
\par 13 TUHAN berkata kepadanya, "Ingatlah, keturunanmu akan menjadi orang asing di negeri lain; mereka akan menjadi hamba di sana dan akan ditindas empat ratus tahun lamanya.
\par 14 Tetapi bangsa yang memperbudak keturunanmu itu akan Kuhukum dan pada waktu keturunanmu meninggalkan negeri itu, mereka akan membawa banyak harta benda.
\par 15 Engkau sendiri akan mencapai usia yang tinggi; engkau akan mati dengan tenang, lalu dikuburkan.
\par 16 Sesudah empat keturunan, anak cucumu akan kembali ke sini, karena Aku tidak akan mengusir orang Amori sebelum mereka menjadi begitu jahatnya sehingga perlu dihukum."
\par 17 Ketika hari sudah malam, tiba-tiba muncul sebuah anglo yang berasap dan obor yang menyala, lalu lewat di antara potongan-potongan daging itu.
\par 18 Pada waktu itu juga TUHAN mengadakan perjanjian dengan Abram. Kata TUHAN, "Aku berjanji akan memberikan kepada keturunanmu seluruh tanah ini, mulai dari batas negeri Mesir sampai ke Sungai Efrat,
\par 19 termasuk juga tanah orang Keni, Kenas, Kadmon,
\par 20 Het, Feris, Refaim,
\par 21 Amori, Kanaan, Girgasi dan Yebus."

\chapter{16}

\par 1 Sarai, istri Abram, belum juga mendapat anak. Tetapi ia mempunyai seorang hamba dari Mesir, seorang gadis yang bernama Hagar.
\par 2 Sarai berkata kepada Abram, "TUHAN tidak memungkinkan saya melahirkan anak. Sebab itu, sebaiknya engkau tidur dengan hamba saya ini. Barangkali dia dapat melahirkan anak untuk saya." Abram setuju dengan usul Sarai.
\par 3 Demikianlah Sarai memberikan Hagar kepada Abram untuk dijadikan selir. Pada waktu itu Abram sudah sepuluh tahun di Kanaan.
\par 4 Abram tidur dengan Hagar, lalu mengandunglah wanita itu. Tetapi ketika Hagar tahu bahwa ia hamil, ia menjadi sombong dan meremehkan Sarai.
\par 5 Lalu Sarai berkata kepada Abram, "Saya sudah memberikan Hagar hamba saya kepadamu, dan sejak ia tahu bahwa ia mengandung, ia meremehkan saya. Itu salahmu. Semoga TUHAN memutuskan perkara ini antara engkau dan saya."
\par 6 Jawab Abram, "Baiklah, dia hambamu dan engkau berkuasa atas dia; perlakukanlah dia semaumu." Lalu Sarai memperlakukan Hagar dengan sangat kejam, sehingga ia melarikan diri.
\par 7 Malaikat TUHAN menjumpai Hagar di dekat mata air di padang gurun, di jalan yang menuju ke padang gurun Sur.
\par 8 Kata malaikat itu, "Hagar, hamba Sarai, engkau dari mana dan mau ke mana?" Jawab Hagar, "Saya lari dari Sarai nyonya saya."
\par 9 Malaikat itu berkata, "Kembalilah kepadanya dan layanilah dia."
\par 10 Lalu kata malaikat itu lagi, "Aku akan memberikan kepadamu begitu banyak anak cucu, sehingga tidak seorang pun dapat menghitung mereka.
\par 11 Tidak lama lagi engkau akan melahirkan anak laki-laki; namakanlah dia Ismael, karena TUHAN telah mendengar tangismu.
\par 12 Tetapi anakmu itu akan hidup seperti keledai liar; ia akan melawan setiap orang, dan setiap orang akan melawan dia. Ia akan hidup terpisah dari semua sanak saudaranya."
\par 13 Hagar bertanya dalam hatinya, "Benarkah saya telah melihat Allah yang memperhatikan saya?" Maka ia menyebut TUHAN, yang telah berkata-kata kepadanya, "Allah Yang Memperhatikan".
\par 14 Itulah sebabnya orang menyebut sumur di antara Kades dan Bered itu, "Sumur Dia Yang Hidup Yang Memperhatikan Aku".
\par 15 Lalu Hagar melahirkan anak laki-laki, dan Abram ayahnya, menamakan anak itu Ismael.
\par 16 Pada waktu itu Abram berumur delapan puluh enam tahun.

\chapter{17}

\par 1 Ketika Abram berumur sembilan puluh sembilan tahun, TUHAN menampakkan diri kepadanya dan berkata, "Akulah Allah Yang Mahakuasa. Taatilah Aku dan lakukanlah kehendak-Ku selalu.
\par 2 Aku akan mengikat perjanjian denganmu dan memberikan kepadamu keturunan yang banyak."
\par 3 Lalu sujudlah Abram, kemudian Allah berkata,
\par 4 "Inilah perjanjian yang Kubuat dengan engkau: Aku berjanji bahwa engkau akan menjadi bapak leluhur banyak bangsa. Oleh karena itu namamu bukan lagi Abram, melainkan Abraham.
\par 6 Aku akan memberikan kepadamu banyak anak cucu, dan di antara mereka akan ada yang menjadi raja-raja. Keturunanmu akan begitu banyak, sehingga mereka akan menjadi bangsa-bangsa.
\par 7 Aku akan memenuhi janji-Ku kepadamu dan kepada keturunanmu, turun-temurun, dan perjanjian itu kekal. Aku akan menjadi Allahmu dan Allah keturunanmu.
\par 8 Aku akan memberikan kepadamu dan kepada keturunanmu, tanah ini, yang sekarang engkau diami sebagai orang asing. Seluruh tanah Kanaan akan menjadi milik anak cucumu untuk selama-lamanya dan Aku akan menjadi Allah mereka."
\par 9 Allah berkata lagi kepada Abraham, "Engkau pun harus setia kepada perjanjian ini, baik engkau maupun keturunanmu turun-temurun.
\par 10 Engkau dan semua keturunanmu yang laki-laki harus disunat.
\par 11 Mulai dari sekarang engkau harus menyunatkan setiap bayi laki-laki yang berumur delapan hari, termasuk para hamba yang lahir di rumahmu atau yang kaubeli. Sunat itu akan menjadi tanda dari perjanjian antara Aku dan kamu.
\par 13 Setiap orang harus disunat, dan itu akan menjadi tanda lahiriah yang menunjukkan bahwa perjanjian-Ku denganmu itu kekal.
\par 14 Setiap laki-laki yang tidak disunat tidak lagi dianggap anggota umat-Ku karena ia tidak berpegang pada perjanjian itu."
\par 15 Kemudian Allah berkata kepada Abraham, "Engkau jangan lagi memanggil istrimu Sarai; mulai sekarang namanya Sara.
\par 16 Aku akan memberkatinya dan ia akan melahirkan seorang anak laki-laki yang akan Kuberikan kepadamu. Ya, Aku akan memberkati Sara, dan ia akan menjadi ibu leluhur bangsa-bangsa. Di antara keturunannya akan ada raja-raja."
\par 17 Lalu sujudlah Abraham, tetapi ia tertawa ketika berpikir, "Mana mungkin seorang laki-laki yang sudah berumur seratus tahun mendapat anak? Mana mungkin Sara melahirkan pada usia sembilan puluh tahun?"
\par 18 Lalu berkatalah ia kepada Allah, "Sebaiknya Ismael saja yang menjadi ahli waris saya."
\par 19 Tetapi Allah berkata, "Tidak. Sara istrimu akan melahirkan anak laki-laki dan engkau akan menamakannya Ishak. Aku akan setia kepada perjanjian-Ku dengan anak itu dan dengan keturunannya untuk selama-lamanya. Perjanjian itu kekal.
\par 20 Tetapi Aku mengabulkan juga permohonanmu mengenai Ismael. Karena itu dia akan Kuberkati dan Kuberi keturunan yang banyak. Ia akan menjadi leluhur dua belas kepala suku, dan keturunannya akan Kujadikan suatu bangsa yang besar.
\par 21 Tetapi perjanjian-Ku akan Kuikat dengan Ishak, anakmu yang akan dilahirkan oleh Sara, tahun depan kira-kira pada waktu seperti ini."
\par 22 Setelah selesai berkata begitu, Allah meninggalkan Abraham.
\par 23 Pada hari itu juga, Abraham mentaati Allah dan menyunatkan Ismael anaknya, dan semua orang laki-laki dalam rumahnya, termasuk para hamba yang lahir dalam rumahnya maupun yang dibelinya.
\par 24 Abraham berumur sembilan puluh sembilan tahun ketika ia disunat
\par 25 dan Ismael anaknya, berumur tiga belas tahun.
\par 26 Mereka disunat pada hari yang sama,
\par 27 bersama-sama dengan semua hamba Abraham.

\chapter{18}

\par 1 TUHAN menampakkan diri kepada Abraham dekat pohon-pohon keramat tempat ibadat di Mamre. Pada waktu itu hari sangat panas, dan Abraham sedang duduk di pintu kemahnya.
\par 2 Ketika ia mengangkat kepalanya, ia melihat tiga orang berdiri di depannya. Abraham segera berlari hendak menyambut mereka. Dengan bersujud
\par 3 ia berkata, "Tuan-tuan, janganlah melewati kemah saya tanpa singgah dahulu; perkenankanlah saya melayani Tuan-tuan.
\par 4 Biarlah saya mengambil air untuk membasuh kaki Tuan-tuan. Silakan Tuan-tuan melepaskan lelah di bawah pohon ini.
\par 5 Saya akan menghidangkan makanan sekedarnya, supaya Tuan-tuan mendapat kekuatan baru untuk meneruskan perjalanan. Tuan-tuan telah menghormati saya dengan singgah ke rumah hambamu ini." Jawab mereka, "Terima kasih atas kebaikanmu. Kami akan singgah."
\par 6 Lekas-lekas Abraham masuk ke dalam kemah dan berkata kepada Sara, "Cepatlah, ambil sekarung tepung yang paling baik, dan buatlah roti bundar."
\par 7 Kemudian ia lari ke tempat kawanan ternaknya, memilih anak sapi yang gemuk serta empuk dagingnya, lalu memberikannya kepada pelayannya, yang segera menyiapkannya.
\par 8 Setelah itu Abraham mengambil susu, kepala susu, dan daging yang sudah dimasak itu, lalu menghidangkannya kepada tamu-tamunya. Sementara mereka makan, Abraham sendiri melayani mereka di bawah pohon itu.
\par 9 Kemudian mereka bertanya, "Di mana Sara, istrimu?" "Di sana, di dalam kemah," jawab Abraham.
\par 10 Seorang dari mereka berkata, "Sembilan bulan lagi Aku akan kembali. Dan pada waktu itu Sara istrimu akan mendapat anak laki-laki." Pada saat itu Sara sedang mendengarkan di pintu kemah, di belakang tamu itu.
\par 11 Adapun Abraham dan Sara sudah sangat tua, dan Sara sudah mati haid.
\par 12 Sebab itu Sara tertawa dalam hatinya dan berkata, "Aku yang sudah tua dan layu begini, mana mungkin masih ingin campur dengan suamiku? --Lagipula suamiku sudah tua juga."
\par 13 Lalu TUHAN bertanya kepada Abraham, "Mengapa Sara tertawa dan meragukan apakah ia masih bisa melahirkan anak pada masa tuanya?
\par 14 Adakah sesuatu yang mustahil bagi TUHAN? Seperti telah Kukatakan tadi, sembilan bulan lagi Aku akan kembali ke sini. Dan pada waktu itu Sara akan melahirkan anak laki-laki."
\par 15 Karena Sara takut, ia menyangkal, katanya, "Saya tidak tertawa." Tetapi TUHAN menjawab, "Engkau memang tertawa tadi."
\par 16 Setelah itu berangkatlah ketiga tamu itu diantar oleh Abraham. Lalu sampailah mereka ke suatu tempat dari mana mereka dapat memandang Sodom.
\par 17 TUHAN berkata dalam hati, "Aku tak mau merahasiakan kepada Abraham apa yang akan Kulakukan.
\par 18 Bukankah keturunannya akan menjadi bangsa yang besar dan berkuasa? Bukankah melalui dia, semua bangsa di bumi akan Kuberkati?
\par 19 Dia telah Kupilih supaya memerintah anak-anaknya dan keturunannya untuk mentaati Aku dan melakukan apa yang baik dan adil. Jika hal itu mereka lakukan, segala janji-Ku kepada Abraham akan Kupenuhi."
\par 20 Selanjutnya TUHAN berkata kepada Abraham, "Ada tuduhan yang berat terhadap Sodom dan Gomora, dan memang dosa mereka itu sangat besar.
\par 21 Sebab itu Aku hendak turun ke sana untuk memeriksa apakah semua tuduhan yang Kudengar itu benar atau tidak."
\par 22 Lalu dua di antara tamu-tamu itu berangkat menuju ke Sodom, tetapi TUHAN masih tinggal dengan Abraham.
\par 23 Abraham mendekati TUHAN dan bertanya, "Benarkah TUHAN hendak membinasakan orang yang tidak bersalah bersama-sama dengan orang yang bersalah?
\par 24 Seandainya ada lima puluh orang yang tidak bersalah di dalam kota itu, apakah seluruh kota itu akan TUHAN binasakan? Tidakkah TUHAN mau mengampuninya supaya kelima puluh orang itu selamat?
\par 25 Janganlah orang-orang yang tidak bersalah itu dibunuh bersama-sama dengan yang bersalah. Jangan TUHAN! Sebab jika TUHAN melakukannya, orang yang tidak bersalah pasti akan dihukum bersama-sama dengan yang bersalah. Mana mungkin hakim alam semesta bertindak tidak adil!"
\par 26 TUHAN berkata, "Seandainya Kudapati lima puluh orang yang tidak bersalah di Sodom, akan Kuampuni seluruh kota itu demi mereka."
\par 27 Abraham berkata lagi, "Ampunilah keberanian saya menyambung pembicaraan ini, Tuhan. Saya ini hanya manusia, dan tidak berhak untuk mengatakan sesuatu.
\par 28 Tetapi barangkali hanya ada empat puluh lima orang yang tidak bersalah, dan bukan lima puluh. Apakah Tuhan akan membinasakan seluruh kota itu hanya karena selisih lima orang saja?" TUHAN berkata, "Kota itu tidak akan Kubinasakan seandainya di sana Kudapati empat puluh lima orang yang tidak bersalah."
\par 29 Abraham berkata lagi, "Bagaimana kalau hanya ada empat puluh orang?" TUHAN menjawab, "Demi empat puluh orang itu, akan Kubatalkan niat-Ku."
\par 30 Abraham berkata, "Jangan marah, Tuhan, jika saya berbicara lagi. Bagaimana seandainya hanya ada tiga puluh orang saja?" TUHAN berkata, "Jika Kudapati tiga puluh orang, akan Kuurungkan niat-Ku."
\par 31 Abraham berkata, "Ampunilah keberanian saya meneruskan pembicaraan ini, Tuhan. Seandainya hanya ada dua puluh orang saja?" TUHAN menjawab, "Jika Kudapati dua puluh orang, kota itu tak akan Kubinasakan."
\par 32 Akhirnya Abraham berkata, "Janganlah marah, Tuhan, saya hanya akan berbicara sekali lagi. Bagaimana jika hanya terdapat sepuluh orang saja?" TUHAN berkata, "Jika ada sepuluh orang yang tidak bersalah, Aku tidak akan membinasakan kota itu."
\par 33 Setelah selesai berbicara dengan Abraham, TUHAN pergi, dan Abraham pulang ke rumahnya.

\chapter{19}

\par 1 Sesudah bertamu pada Abraham, kedua malaikat itu pergi ke Sodom dan tiba di sana pada waktu malam. Lot sedang duduk di pintu gerbang kota, dan setelah melihat mereka, ia bangkit untuk menyambut mereka. Lalu sujudlah ia di hadapan mereka,
\par 2 dan berkata, "Tuan-tuan, silakan singgah di rumah saya. Tuan-tuan dapat membasuh kaki dan bermalam di rumah saya. Besok kalau mau, Tuan-tuan dapat bangun pagi-pagi dan meneruskan perjalanan." Tetapi mereka menjawab, "Terima kasih, biar kami bermalam di sini saja, di lapangan kota."
\par 3 Lot memohon dengan sangat, dan akhirnya mereka masuk bersama dia ke dalam rumahnya. Lot menyediakan hidangan lezat dan memanggang roti secukupnya, lalu makanlah mereka.
\par 4 Tetapi sebelum tamu-tamu itu pergi tidur, orang-orang Sodom mengepung rumah itu. Semua orang laki-laki di kota itu, baik yang tua maupun yang muda, ada di situ.
\par 5 Mereka berseru kepada Lot, dan bertanya, "Di mana orang-orang yang datang bermalam di rumahmu? Serahkan mereka, supaya kami dapat bercampur dengan mereka!"
\par 6 Lot keluar dari rumahnya, dan sesudah menutup pintu,
\par 7 ia berkata kepada orang-orang Sodom itu, "Saudara-saudara, saya minta dengan sangat, janganlah melakukan hal yang sejahat itu!
\par 8 Coba dengar, saya punya dua anak perawan. Biar saya serahkan mereka kepada kalian dan kalian boleh melakukan apa saja dengan mereka. Tetapi jangan apa-apakan tamu-tamu saya ini; sebab saya wajib melindungi mereka."
\par 9 Tetapi kata orang-orang Sodom itu kepada Lot, "Pergi! Engkau orang asing mau mengatur kami? Ayo, pergi! Kalau tidak, engkau akan kami hajar lebih berat daripada kedua orang itu." Lalu mereka mendorong Lot dan menyerbu hendak mendobrak pintu.
\par 10 Tetapi kedua tamu itu mengulurkan tangan mereka dan menarik Lot masuk ke dalam rumah, lalu menutup pintu.
\par 11 Mereka membutakan semua orang yang ada di luar rumah itu, sehingga orang-orang itu tidak dapat menemukan pintu itu lagi.
\par 12 Kedua tamu itu berkata kepada Lot, "Jika engkau mempunyai anak laki-laki, anak perempuan, menantu atau sanak saudara lainnya yang tinggal di dalam kota ini, bawalah mereka keluar dari sini,
\par 13 sebab kota ini akan kami musnahkan. TUHAN telah mendengar tuduhan-tuduhan berat terhadap penduduk di sini, dan kami telah diutusnya untuk menumpas kota Sodom."
\par 14 Lalu pergilah Lot menemui dua orang tunangan kedua anaknya, dan berkata, "Cepatlah keluar dari tempat ini, sebab TUHAN akan memusnahkannya." Tetapi mereka mengira Lot bergurau saja.
\par 15 Pada waktu subuh, kedua malaikat itu mendesak Lot supaya lekas berangkat. "Cepatlah," kata mereka. "Pergilah dengan istrimu dan kedua anak gadismu dari sini, supaya kalian jangan mati apabila kota ini dimusnahkan."
\par 16 Lot bimbang. Tetapi TUHAN merasa kasihan kepadanya; karena itu kedua tamunya menuntun Lot dan istrinya serta kedua anaknya ke luar kota.
\par 17 Sesudah itu seorang dari malaikat itu berkata, "Larilah, selamatkan nyawamu! Jangan menoleh ke belakang dan jangan berhenti di lembah. Larilah ke pegunungan, supaya kalian jangan mati!"
\par 18 Tetapi Lot menjawab, "Aduh, Tuan jangan suruh kami lari ke pegunungan.
\par 19 Memang, Tuan telah sangat menolong saya dan menyelamatkan nyawa saya. Tetapi pegunungan itu begitu jauh; jangan-jangan saya tersusul bencana itu lalu mati sebelum sampai di sana.
\par 20 Lihat, di depan itu ada kota kecil yang tidak begitu jauh. Izinkan saya lari ke sana supaya selamat."
\par 21 Malaikat itu menjawab, "Baiklah, kota kecil itu tak akan kutumpas.
\par 22 Cepat, larilah! Sebab aku tidak bisa berbuat apa-apa sebelum engkau sampai di sana." Karena Lot menyebut kota itu kecil, maka kota itu dinamakan Zoar.
\par 23 Matahari sedang terbit ketika Lot sampai di Zoar.
\par 24 Tiba-tiba TUHAN menurunkan hujan belerang yang berapi atas Sodom dan Gomora.
\par 25 Kedua kota itu dihancurkan, juga seluruh lembah dan semua tumbuh-tumbuhan serta semua penduduk di situ.
\par 26 Tetapi istri Lot menoleh ke belakang, lalu dia berubah menjadi tiang garam.
\par 27 Keesokan harinya, pagi-pagi, Abraham cepat-cepat pergi ke tempat ia berdiri di hadapan TUHAN sehari sebelumnya.
\par 28 Ia memandang ke arah Sodom dan Gomora dan ke seluruh lembah, dan melihat asap mengepul dari tanah itu, seperti asap dari tungku raksasa.
\par 29 Demikianlah, ketika Allah membinasakan kota-kota itu di lembah di mana Lot tinggal, Allah ingat kepada Abraham dan menolong Lot melarikan diri.
\par 30 Karena Lot takut menetap di Zoar, maka pergilah ia ke pegunungan bersama-sama dengan kedua anaknya perempuan, lalu tinggal di dalam sebuah gua.
\par 31 Anak perempuan yang sulung berkata kepada adiknya, "Ayah sudah tua, dan di seluruh negeri ini tak ada orang laki-laki yang dapat mengawini kita supaya kita mendapat anak.
\par 32 Mari, kita buat ayah mabuk, lalu kita tidur dengan dia supaya kita mendapat anak."
\par 33 Pada malam itu mereka memberi ayah mereka minum anggur, lalu anak yang sulung tidur dengan ayahnya; tetapi ayahnya begitu mabuk sehingga tidak tahu apa yang terjadi.
\par 34 Keesokan harinya, anak yang sulung berkata kepada adiknya, "Tadi malam saya sudah tidur dengan ayah! Nanti malam kita buat dia mabuk lagi. Lalu tidurlah kau dengan dia. Nanti kita masing-masing mendapat anak."
\par 35 Demikianlah pada malam itu mereka membuat Lot mabuk, dan anaknya yang kedua tidur dengan dia. Dan Lot terlalu mabuk lagi sehingga tidak tahu apa yang terjadi.
\par 36 Lalu mengandunglah kedua anak Lot itu karena ayah mereka sendiri.
\par 37 Anak yang sulung melahirkan anak laki-laki yang dinamakannya Moab. Dia menjadi leluhur orang Moab yang sekarang.
\par 38 Anak yang kedua melahirkan anak laki-laki juga yang dinamakannya Ben-Ami. Dia menjadi leluhur bangsa Amon yang sekarang.

\chapter{20}

\par 1 Abraham pindah dari Mamre ke bagian selatan tanah Kanaan, dan menetap di antara Kades dan Syur. Beberapa waktu kemudian, ketika ia tinggal di Gerar,
\par 2 ia mengatakan bahwa Sara istrinya adalah adiknya. Karena itu Raja Abimelekh dari Gerar menyuruh utusannya membawa Sara kepadanya.
\par 3 Pada suatu malam Allah menampakkan diri kepada raja itu di dalam mimpi, dan berkata, "Engkau akan mati, karena telah kauambil wanita itu; sebab ia sudah bersuami."
\par 4 Tetapi Abimelekh belum sampai menjamah Sara, maka kata raja itu, "Tuhan, saya tidak bersalah! Apakah Tuhan akan membinasakan saya dan bangsaku?
\par 5 Abraham sendiri mengatakan bahwa wanita itu adiknya, dan wanita itu berkata demikian juga. Saya telah melakukan hal itu dengan hati nurani yang bersih, jadi aku tidak bersalah."
\par 6 Maka Allah menjawab dalam mimpi itu, "Memang, Aku tahu bahwa engkau melakukannya dengan hati nurani yang bersih. Karena itu Aku telah mencegah engkau berbuat dosa terhadap Aku, dan tidak Kubiarkan engkau menjamah wanita itu.
\par 7 Tetapi sekarang, kembalikanlah dia kepada suaminya. Suaminya seorang nabi, dan ia akan berdoa untukmu supaya engkau tidak mati. Tetapi jika engkau tidak mengembalikannya, ingat, engkau dan seluruh rakyatmu akan mati!"
\par 8 Keesokan harinya, pagi-pagi, Abimelekh memanggil semua pegawainya dan memberi tahu kepada mereka segala kejadian itu. Mereka menjadi sangat ketakutan.
\par 9 Setelah itu Abimelekh memanggil Abraham dan bertanya, "Apa yang telah engkau perbuat terhadap kami? Apa salahku terhadapmu, sehingga engkau mendatangkan musibah ini atas diriku dan kerajaanku? Sungguh tidak patut engkau berbuat seperti itu kepadaku.
\par 10 Mengapa engkau berbuat begitu?"
\par 11 Abraham menjawab, "Pikir saya, di sini tak ada orang yang takut kepada Allah, sehingga saya akan dibunuh oleh orang yang menginginkan istri saya.
\par 12 Lagipula dia itu memang adik saya. Kami satu ayah, tetapi berlainan ibu, dan kemudian kami kawin.
\par 13 Maka ketika Allah menyuruh saya meninggalkan rumah ayah saya dan mengembara ke negeri-negeri asing, saya berkata kepada istri saya, 'Nyatakanlah kesetiaanmu kepada saya dengan mengatakan kepada setiap orang bahwa saya ini abangmu.'"
\par 14 Setelah itu Abimelekh mengembalikan Sara kepada Abraham, dan pada waktu itu ia memberikan juga hamba-hamba, domba dan sapi.
\par 15 Kata raja itu kepada Abraham, "Inilah seluruh negeriku; menetaplah di mana engkau suka."
\par 16 Lalu katanya kepada Sara, "Abangmu sudah kuberi seribu uang perak sebagai bukti kepada semua orang bahwa engkau benar-benar tidak bersalah."
\par 17 TUHAN telah membuat mandul semua wanita di istana Abimelekh karena kejadian dengan Sara, istri Abraham itu. Lalu Abraham berdoa kepada Allah, dan Allah menyembuhkan Abimelekh, istrinya dan hamba-hambanya perempuan, sehingga wanita-wanita itu dapat melahirkan lagi.

\chapter{21}

\par 1 TUHAN memberkati Sara, seperti yang telah dijanjikan-Nya.
\par 2 Pada waktu yang telah ditentukan Allah, ketika Abraham sudah tua, mengandunglah Sara lalu melahirkan seorang anak laki-laki.
\par 3 Abraham menamakan anak itu Ishak;
\par 4 dan ketika Ishak berumur delapan hari, Abraham menyunatnya, sesuai dengan perintah Allah.
\par 5 Abraham berusia seratus tahun ketika Ishak lahir.
\par 6 Sara berkata, "Allah telah membuat saya tertawa karena gembira. Setiap orang yang mendengar hal ini akan tertawa gembira bersama saya."
\par 7 Kemudian ditambahkannya, "Siapa tadinya dapat mengatakan kepada suami saya bahwa saya akan menyusui anak? Namun saya telah melahirkan juga walaupun suami saya sudah tua sekali."
\par 8 Anak itu bertambah besar, dan pada hari ia mulai disapih, Abraham mengadakan pesta meriah.
\par 9 Pada suatu hari, Ismael, anak Abraham dan Hagar wanita Mesir itu, sedang bermain-main dengan Ishak, anak Sara.
\par 10 Sara melihat hal itu lalu berkata kepada Abraham, "Usirlah hamba wanita itu bersama anaknya. Tidak boleh anak wanita itu menerima bagian dari kekayaanmu yang akan diwarisi anak saya Ishak."
\par 11 Abraham sama sekali tidak senang dengan usul itu, karena Ismael adalah anaknya juga.
\par 12 Tetapi Allah berkata kepada Abraham, "Janganlah engkau khawatir mengenai hambamu Hagar dan anaknya itu. Turutilah kemauan Sara, karena melalui Ishaklah engkau akan mendapat keturunan yang Kujanjikan itu.
\par 13 Begitu juga kepada anak Hagar akan Kuberikan banyak anak cucu supaya mereka menjadi suatu bangsa. Sebab ia anakmu juga."
\par 14 Keesokan harinya pagi-pagi, Abraham memberi kepada Hagar makanan dan sebuah kantong kulit berisi air untuk bekal di jalan. Ia meletakkan anak itu pada punggung Hagar, dan menyuruh wanita itu pergi. Lalu berangkatlah Hagar dan mengembara di padang gurun Bersyeba.
\par 15 Ketika air bekalnya habis, Hagar meletakkan anaknya di bawah semak,
\par 16 lalu duduk kira-kira seratus meter dari tempat itu. Katanya dalam hati, "Saya tidak tahan melihat anak saya mati." Lalu menangislah ia.
\par 17 Allah mendengar suara Ismael, dan dari langit malaikat Allah berbicara kepada Hagar, katanya, "Apa yang engkau susahkan, Hagar? Janganlah takut. Allah telah mendengar suara anakmu.
\par 18 Pergilah kepada anakmu, angkat dan tenangkanlah dia. Aku akan menjadikan keturunannya suatu bangsa yang besar."
\par 19 Lalu Allah membuat Hagar melihat dengan jelas, sehingga tampak olehnya sebuah sumur. Maka pergilah ia lalu mengisi kantong kulit itu dengan air, kemudian diberinya anaknya minum.
\par 20 Allah menyertai Ismael. Anak itu bertambah besar; ia menetap di padang gurun Paran, dan menjadi pemburu yang mahir.
\par 21 Ibunya mengawinkan dia dengan seorang wanita Mesir.
\par 22 Pada waktu itu Raja Abimelekh dan Pikhol, panglima tentaranya, datang kepada Abraham dan berkata, "Allah menolong engkau dalam segala sesuatu yang engkau lakukan.
\par 23 Sebab itu, bersumpahlah di sini di hadapan Allah, bahwa engkau tidak akan berbuat curang terhadap aku, maupun terhadap anak-anakku, atau keturunanku. Seperti aku telah berbaik hati kepadamu, berbuatlah begitu juga kepadaku dan kepada negeri yang kaudiami ini."
\par 24 Kata Abraham, "Aku bersumpah."
\par 25 Tetapi kemudian Abraham menyesali Abimelekh tentang sebuah sumur yang telah dirampas oleh hamba-hamba raja itu.
\par 26 Abimelekh berkata, "Aku tidak tahu siapa yang telah melakukannya. Belum pernah engkau mengatakannya kepadaku, dan inilah pertama kalinya aku mendengar tentang hal itu."
\par 27 Lalu Abraham memberikan sejumlah domba dan sapi kepada Abimelekh, dan kedua orang itu mengadakan perjanjian.
\par 28 Abraham memisahkan tujuh anak domba dari kawanan dombanya,
\par 29 dan Abimelekh bertanya kepadanya, "Mengapa engkau lakukan hal itu?"
\par 30 Abraham menjawab, "Terimalah ketujuh anak domba ini sebagai tanda Tuan mengakui bahwa akulah yang telah menggali sumur ini."
\par 31 Demikianlah tempat itu dinamakan Bersyeba, karena di situlah kedua orang itu mengucapkan sumpah.
\par 32 Setelah mereka mengadakan perjanjian di Bersyeba, Abimelekh dan Pikhol kembali ke negeri orang Filistin.
\par 33 Lalu Abraham menanam sebatang pohon tamariska di Bersyeba dan sejak itu tempat itu dipakainya untuk menyembah TUHAN, Allah yang kekal.
\par 34 Sesudah itu Abraham tinggal agak lama di negeri orang Filistin.

\chapter{22}

\par 1 Beberapa waktu kemudian Allah menguji kesetiaan Abraham. Allah memanggil, "Abraham!" Lalu Abraham menjawab, "Ya, Tuhan."
\par 2 Kata Allah, "Pergilah ke tanah Moria dengan Ishak, anakmu yang tunggal, yang sangat kaukasihi. Di situ, di sebuah gunung yang akan Kutunjukkan kepadamu, persembahkanlah anakmu sebagai kurban bakaran kepada-Ku."
\par 3 Keesokan harinya pagi-pagi, Abraham membelah-belah kayu untuk kurban bakaran dan mengikat kayu itu di atas keledainya. Ia berangkat dengan Ishak dan dua orang hambanya ke tempat yang dikatakan Allah kepadanya.
\par 4 Pada hari yang ketiga tampaklah oleh Abraham tempat itu di kejauhan.
\par 5 Lalu ia berkata kepada kedua hambanya itu, "Tinggallah kamu di sini dengan keledai ini. Saya dan anak saya akan pergi ke sana untuk menyembah TUHAN, nanti kami kembali kepadamu."
\par 6 Abraham meletakkan kayu untuk kurban bakaran itu pada pundak Ishak, sedang ia sendiri membawa pisau dan bara api untuk membakar kayu. Ketika mereka berjalan bersama-sama,
\par 7 Ishak berkata, "Ayah!" Abraham menjawab, "Ada apa, anakku?" Ishak bertanya, "Kita sudah membawa api dan kayu, tetapi di manakah anak domba untuk kurban bakaran itu?"
\par 8 Abraham menjawab, "Allah sendiri akan menyediakan anak domba itu." Lalu keduanya berjalan terus.
\par 9 Ketika mereka sampai di tempat yang dikatakan Allah kepada Abraham, ia mendirikan sebuah mezbah dan menyusun kayu bakar itu di atasnya. Lalu diikatnya anaknya dan dibaringkannya di mezbah, di atas kayu bakar itu.
\par 10 Setelah itu, diambilnya pisaunya hendak membunuh anaknya.
\par 11 Tetapi malaikat TUHAN berseru kepadanya dari langit, "Abraham, Abraham!" Jawab Abraham, "Ya, Tuhan!"
\par 12 "Jangan kausakiti anak itu atau kauapa-apakan dia," kata TUHAN melalui malaikat itu. "Sekarang Aku tahu bahwa engkau hormat dan taat kepada-Ku, karena engkau tidak menolak untuk menyerahkan anakmu yang tunggal itu kepada-Ku."
\par 13 Lalu Abraham memandang ke sekitarnya dan melihat seekor domba jantan yang tanduknya tersangkut dalam semak-semak. Abraham mengambil domba itu lalu mempersembahkannya kepada TUHAN sebagai kurban bakaran pengganti anaknya.
\par 14 Abraham menamakan tempat itu "TUHAN menyediakan yang diperlukan". Dan sampai sekarang pun orang mengatakan "Di atas gunung-Nya TUHAN menyediakan yang diperlukan".
\par 15 Sekali lagi dari langit malaikat TUHAN berseru kepada Abraham,
\par 16 "TUHAN berkata: Aku bersumpah demi nama-Ku sendiri, karena engkau telah melakukan hal ini dan tidak menolak untuk menyerahkan anakmu yang tunggal itu kepada-Ku,
\par 17 Aku akan memberkati engkau dengan berlimpah-limpah dan membuat keturunanmu sebanyak bintang di langit dan sebanyak pasir di tepi laut. Anak cucumu akan mengalahkan musuh-musuh mereka.
\par 18 Semua bangsa di bumi akan memohon kepada-Ku supaya Aku memberkati mereka sebagaimana telah Kuberkati keturunanmu--karena engkau telah mentaati perintah-Ku."
\par 19 Setelah itu kembalilah Abraham kepada kedua hambanya, lalu mereka bersama-sama pergi ke Bersyeba, dan Abraham menetap di sana.
\par 20 Abraham mempunyai abang yang bernama Nahor. Sesudah kejadian di Gunung Moria itu, Abraham mendengar bahwa Milka, istri Nahor telah mempunyai delapan anak laki-laki, yaitu:
\par 21 Us, yang sulung; Bus, adiknya; Kemuel, yang kemudian menjadi ayah Aram;
\par 22 Kesed, Hazo, Pildas, Yidlaf, dan Betuel,
\par 23 yang kemudian menjadi ayah Ribka. Itulah kedelapan anak yang dilahirkan Milka bagi Nahor, saudara Abraham.
\par 24 Nahor mempunyai selir yang bernama Reuma dan selir itu melahirkan Tebah, Gaham, Tahas dan Maakha.

\chapter{23}

\par 1 Sesudah Sara mencapai usia 127 tahun,
\par 2 ia meninggal di Hebron di tanah Kanaan. Abraham sedih dan meratapi kematian istrinya itu.
\par 3 Setelah itu Abraham meninggalkan jenazah istrinya dan pergi kepada orang-orang Het, yang mendiami negeri itu. Ia berkata,
\par 4 "Saya ini orang asing yang tinggal di tengah-tengah Saudara-saudara; izinkanlah saya membeli sebidang tanah supaya saya dapat menguburkan istri saya."
\par 5 Mereka menjawab,
\par 6 "Kami menghormati Tuan sebagai pemimpin yang kuat di antara kami, jadi kuburkanlah istri Tuan dalam kuburan yang paling baik yang kami miliki. Kami semua merasa senang menyediakan kuburan untuk istri Tuan."
\par 7 Kemudian sujudlah Abraham di hadapan mereka,
\par 8 dan berkata, "Jika Saudara-saudara rela saya menguburkan istri saya di sini, tolonglah mintakan kepada Efron, anak laki-laki Zohar,
\par 9 supaya ia menjual kepadaku Gua Makhpela yang terletak di pinggir ladangnya. Baiklah ia menjual tanah itu di sini, dengan disaksikan Saudara-saudara. Harganya akan kubayar penuh, supaya tanah itu jadi milik saya."
\par 10 Pada waktu itu Efron sendiri sedang duduk dengan orang-orang Het itu pada tempat pertemuan di pintu gerbang kota. Lalu dengan disaksikan oleh semua orang yang ada di situ, ia menjawab,
\par 11 "Bangsa saya sendiri menjadi saksi bahwa ladang itu beserta guanya saya berikan supaya Tuan dapat menguburkan istri Tuan."
\par 12 Maka Abraham sujud di hadapan orang-orang Het itu,
\par 13 dan dengan didengar oleh semua orang, ia berkata kepada Efron, "Mohon dengarkan saya. Saya ingin membeli seluruh ladang itu. Terimalah uangnya, supaya saya menguburkan istri saya di situ."
\par 14 Efron menjawab,
\par 15 "Sebidang tanah yang hanya berharga empat ratus uang perak--apa artinya itu bagi kita berdua? Kuburkanlah istri Tuan di situ."
\par 16 Abraham setuju dengan harga itu. Ia menghitung uang perak sejumlah yang disebut oleh Efron, yaitu empat ratus uang perak, menurut timbangan yang dipakai oleh para saudagar. Lalu ia membayarnya kepada Efron, dengan disaksikan oleh semua orang Het itu.
\par 17 Demikianlah ladang Efron yang letaknya di Makhpela di sebelah timur Mamre, menjadi milik Abraham. Tanah itu terdiri dari ladang dan guanya serta semua pohon-pohonnya termasuk pohon-pohon yang ada di batas-batasnya.
\par 18 Tanah itu diakui sebagai milik Abraham oleh semua orang Het yang hadir dalam pertemuan di pintu gerbang kota.
\par 19 Sesudah itu Abraham menguburkan Sara istrinya di dalam gua Makhpela di tanah Kanaan.
\par 20 Demikianlah ladang beserta guanya yang tadinya milik orang Het, menjadi milik Abraham untuk tanah pekuburan.

\chapter{24}

\par 1 Abraham sudah tua sekali, dan ia diberkati TUHAN dalam segala hal.
\par 2 Pada suatu hari berkatalah Abraham kepada hambanya yang paling tua, yang mengurus segala harta bendanya, "Letakkanlah tanganmu di antara pangkal pahaku.
\par 3 Bersumpahlah demi TUHAN, Allah yang menguasai langit dan bumi, bahwa engkau tidak akan mencari istri bagi Ishak anakku dari antara orang-orang di tanah Kanaan ini.
\par 4 Engkau harus kembali ke negeri kelahiranku untuk mencari istri baginya dari antara sanak saudaraku."
\par 5 Lalu bertanyalah hamba itu kepadanya, "Bagaimana jika gadis itu tidak mau meninggalkan rumahnya dan tak mau mengikuti saya ke negeri ini? Haruskah saya membawa anak Tuan itu ke negeri asal Tuan?"
\par 6 Tetapi Abraham menjawab, "Awas! Jangan sekali-kali kaubawa anakku itu ke sana!
\par 7 TUHAN, Allah yang menguasai langit, telah membawa aku dari rumah ayahku dan dari negeri sanak saudaraku, dan Ia telah bersumpah kepadaku bahwa Ia akan memberikan negeri ini kepada keturunanku. Dia juga akan mengutus malaikat-Nya untuk menolongmu, supaya engkau dapat menemukan seorang istri dari sana untuk anakku.
\par 8 Tetapi jika gadis itu tidak mau ikut dengan engkau, maka engkau bebas dari sumpah ini. Bagaimanapun juga, jangan kaubawa anakku itu ke sana."
\par 9 Kemudian hamba itu meletakkan tangannya di antara pangkal paha Abraham tuannya, dan bersumpah akan melakukan apa yang diminta oleh Abraham itu.
\par 10 Hamba itu mengambil sepuluh ekor unta serta bermacam-macam barang berharga milik tuannya dan berangkat ke kota tempat tinggal Nahor semasa hidupnya; letaknya di sebelah utara Mesopotamia.
\par 11 Setelah ia tiba di sana, dihentikannya unta-unta itu supaya beristirahat dekat sebuah sumur di luar kota. Hari sudah sore, dan itulah waktunya para wanita datang ke sumur untuk menimba air.
\par 12 Lalu hamba itu berdoa, "TUHAN, Allah tuan saya Abraham, tolonglah, supaya tugas saya berhasil pada hari ini; berbaik hatilah kepada tuan saya Abraham.
\par 13 Lihatlah, saya berdiri di dekat sumur ini. Sebentar lagi gadis-gadis dari kota akan datang menimba air.
\par 14 Saya akan berkata kepada salah seorang dari mereka, 'Tolong turunkan buyungmu dan berilah saya minum.' Jika ia menjawab, 'Minumlah, dan unta-unta Bapak juga akan saya beri minum!' dialah yang TUHAN pilih untuk hambamu Ishak. Jika hal itu terjadi, saya akan tahu bahwa TUHAN telah berbaik hati kepada tuan saya Abraham."
\par 15 Ketika ia sedang berdoa itu, Ribka, seorang gadis yang sangat cantik dan masih perawan, tiba di sumur itu membawa buyung di atas bahunya. Ia anak Betuel, dan orang tua Betuel adalah Nahor dan Milka; Nahor itu abang Abraham. Ribka menuju ke sumur itu dan mengisi buyungnya, lalu berjalan kembali.
\par 17 Hamba Abraham lari mendekatinya dan berkata, "Tolong, Nak, berilah saya minum dari buyungmu itu."
\par 18 Gadis itu menjawab, "Minumlah, Pak," lalu dengan cepat diturunkannya buyungnya dari bahunya, dan sambil memegang buyung itu diberinya hamba itu minum.
\par 19 Setelah hamba itu selesai minum; berkatalah gadis itu, "Saya akan menimba air untuk unta-unta Bapak juga, supaya semua binatang itu dapat minum sepuasnya."
\par 20 Kemudian dengan cepat air di dalam buyung itu dituangkannya ke dalam tempat minum unta, lalu berlarilah ia kembali ke sumur untuk menimba air lebih banyak lagi sampai semua binatang itu puas minum.
\par 21 Sementara itu hamba Abraham terus memperhatikannya tanpa berkata sepatah pun. Ia ingin tahu apakah TUHAN membuat ia berhasil melakukan tugasnya atau tidak.
\par 22 Setelah semua unta minum sepuas-puasnya, hamba Abraham mengambil perhiasan emas yang berharga, lalu mengenakannya pada hidung gadis itu. Pada lengannya diberi sepasang gelang emas.
\par 23 Lalu hamba itu berkata, "Coba katakan siapa ayahmu. Adakah tempat bermalam di rumahnya untuk saya dan orang-orangku?"
\par 24 "Ayah saya Betuel, anak Nahor dan Milka," jawab gadis itu.
\par 25 "Di rumah kami ada tempat bermalam untuk Bapak dan juga banyak jerami dan makanan ternak."
\par 26 Maka sujudlah hamba itu menyembah TUHAN.
\par 27 Katanya, "Pujilah TUHAN, Allah tuan saya Abraham. TUHAN ternyata setia dan berbaik hati kepada tuan saya Abraham. Ia memimpin saya langsung kepada sanak saudaranya."
\par 28 Kemudian gadis itu lari ke rumah ibunya dan menceritakan segala kejadian itu.
\par 29 Ribka mempunyai seorang abang yang bernama Laban. Ia mendengar cerita adiknya tentang apa yang dikatakan hamba itu. Ia melihat juga perhiasan di hidung dan gelang pada lengan adiknya itu. Maka berlarilah Laban keluar rumah untuk menemui hamba Abraham itu yang masih berdiri dengan unta-untanya di dekat sumur.
\par 31 Lalu kata Laban, "Bapak orang yang diberkati TUHAN! Jangan tinggal di luar sini. Mari ke rumah kami. Kami telah menyediakan kamar untuk Bapak dan juga tempat bagi unta-unta Bapak."
\par 32 Lalu ikutlah hamba itu dan masuk ke dalam rumah. Laban menurunkan beban unta-unta, dan memberi jerami dan makanan kepada binatang-binatang itu. Setelah itu dibawanya air pembasuh kaki untuk hamba Abraham dan orang-orangnya.
\par 33 Ketika makanan dihidangkan, hamba itu berkata, "Saya tidak akan makan sebelum menyampaikan pesan yang saya bawa." Laban berkata, "Silakan bicara."
\par 34 Lalu mulailah hamba Abraham berbicara, katanya, "Saya ini hamba Abraham.
\par 35 TUHAN sangat memberkati tuan saya itu dan menjadikan dia kaya raya. TUHAN telah memberikan kepadanya kawanan domba, kambing, sapi, unta, keledai, dan juga perak, emas, serta hamba laki-laki dan perempuan.
\par 36 Pada usia yang lanjut, Sara, istri tuan saya itu telah melahirkan anak laki-laki, dan tuan saya telah mewariskan segala harta bendanya kepada anaknya itu.
\par 37 Dia menyuruh saya berjanji dengan sumpah supaya mentaati perintahnya. Katanya, 'Janganlah engkau memilih istri bagi anakku dari antara gadis-gadis di tanah Kanaan ini,
\par 38 melainkan dari antara sanak saudara kaum ayahku. Jadi pergilah ke sana dengan segera.'
\par 39 Lalu saya bertanya kepadanya, 'Bagaimana jika gadis itu tidak mau ikut dengan saya?'
\par 40 Lalu jawabnya, 'TUHAN yang selalu kutaati, akan mengutus malaikat-Nya untuk menolong supaya tugasmu berhasil. Engkau akan menemukan istri bagi anakku dari antara kaumku sendiri, yaitu dari sanak saudara ayahku.
\par 41 Hanya apabila mereka menolakmu, barulah engkau bebas dari sumpahmu.'
\par 42 Dan ketika saya sampai di sumur tadi, saya berdoa di dalam hati, 'TUHAN, Allah tuan saya Abraham, hendaknya TUHAN membuat tugas saya berhasil.
\par 43 Saya akan tetap berdiri di dekat sumur ini. Apabila seorang gadis datang untuk menimba air, saya akan minta kepadanya supaya diberi minum.
\par 44 Jika dia mau melakukannya, dan menawarkan juga untuk memberi minum kepada unta-unta saya, maka dialah kiranya yang TUHAN pilih menjadi istri anak tuan saya itu.'
\par 45 Belum lagi saya selesai mengucapkan doa itu di dalam hati, datanglah Ribka membawa buyung di atas bahunya lalu mengambil air dari sumur. Kemudian saya berkata kepadanya, 'Tolong berilah saya minum.'
\par 46 Dengan segera ia menurunkan buyung itu dari bahunya dan berkata, 'Minumlah, dan unta-unta Bapak juga akan saya beri minum.' Lalu saya minum dan setelah itu unta-unta saya itu juga diberinya minum.
\par 47 Sesudah itu saya bertanya, 'Siapakah ayahmu?' Dan ia menjawab, 'Ayah saya Betuel, anak Nahor dan Milka.' Lalu saya mengenakan perhiasan emas pada hidungnya dan sepasang gelang pada lengannya.
\par 48 Saya sujud menyembah dan memuji TUHAN, Allah tuan saya Abraham. Ia telah memimpin saya langsung ke sini, kepada sanak saudara tuan saya, sehingga saya dapat menemukan gadis ini bagi anaknya.
\par 49 Sekarang, jika kalian bersedia memperlakukan tuan saya dengan baik dan memenuhi tanggung jawab kalian terhadapnya sebagai saudara, sudilah mengatakannya kepada saya, tetapi jika tidak, katakanlah juga, supaya saya tahu harus berbuat apa."
\par 50 Laban dan Betuel menjawab, "Karena apa yang terjadi ini berasal dari TUHAN, kami tidak patut memberi keputusan.
\par 51 Ini Ribka; biarlah ia ikut dengan Bapak dan menjadi istri anak tuan Bapak, seperti yang dikatakan TUHAN sendiri."
\par 52 Ketika hamba Abraham itu mendengar perkataan mereka, ia sujud menyembah TUHAN.
\par 53 Lalu ia mengeluarkan pakaian, perhiasan perak dan emas, dan memberikan semua itu kepada Ribka. Juga abangnya dan ibunya diberinya hadiah yang mahal-mahal.
\par 54 Setelah itu hamba Abraham dan orang-orangnya makan dan minum, lalu bermalam di situ. Sesudah mereka bangun keesokan harinya, hamba itu berkata, "Izinkanlah saya pulang kepada tuan saya."
\par 55 Tetapi abang serta ibu Ribka berkata, "Biarlah Ribka tinggal bersama kami kira-kira seminggu atau sepuluh hari lagi, dan sesudah itu bolehlah dia pergi."
\par 56 Tetapi hamba itu berkata, "Janganlah menahan saya. TUHAN membuat perjalanan saya berhasil; izinkanlah saya pulang kepada tuan saya."
\par 57 Jawab mereka, "Mari kita panggil Ribka dan menanyakan pendapatnya."
\par 58 Lalu mereka memanggil Ribka dan bertanya, "Maukah engkau ikut orang ini?" Jawab gadis itu, "Ya, saya mau."
\par 59 Maka mereka mengizinkan Ribka dan inang pengasuhnya pergi dengan hamba Abraham dan orang-orangnya. Lalu mereka memberkati Ribka dengan kata-kata ini, "Semoga engkau, Ribka menjadi ibu jutaan orang! Semoga keturunanmu menaklukkan kota-kota musuhnya!"
\par 61 Kemudian berkemaslah Ribka beserta wanita-wanita muda yang menjadi hambanya, lalu naik unta dan berangkat mengikuti hamba Abraham itu.
\par 62 Sementara itu Ishak telah datang dari padang gurun, lewat sumur yang disebut "Dia Yang Memperhatikan Aku" dan tinggal di bagian selatan Kanaan.
\par 63 Pada suatu sore ia keluar kemahnya hendak berjalan-jalan di ladang, lalu dilihatnya unta-unta datang.
\par 64 Ketika Ribka melihat Ishak, ia turun dari untanya,
\par 65 dan bertanya kepada hamba Abraham itu, "Siapa orang laki-laki di ladang itu yang datang ke arah kita?" "Dia tuan saya," jawab hamba itu. Lalu Ribka mengambil selendangnya dan menutupi wajahnya.
\par 66 Kemudian hamba itu menceritakan kepada Ishak segala yang telah dilakukannya.
\par 67 Setelah itu Ishak membawa Ribka masuk ke dalam kemah Sara ibunya, dan ia memperistri Ribka. Ishak mencintai Ribka; maka terhiburlah hati Ishak yang sedih karena kehilangan ibunya.

\chapter{25}

\par 1 Abraham menikah lagi dengan seorang wanita yang bernama Ketura.
\par 2 Istrinya itu melahirkan anak-anak yang bernama: Zimran, Yoksan, Medan, Midian, Isybak, dan Suah.
\par 3 Yoksan ayah Syeba dan Dedan, dan keturunan Dedan ialah orang Asyur, orang Letus, dan orang Leum.
\par 4 Anak-anak Midian ialah: Efa, Efer, Henokh, Abida, dan Eldaa. Mereka semuanya keturunan Ketura.
\par 5 Walaupun Abraham mewariskan segala harta bendanya kepada Ishak,
\par 6 semasa hidupnya ia memberikan banyak hadiah kepada anak-anak yang diperolehnya dari istri-istrinya yang lain. Kemudian disuruhnya anak-anak itu pindah ke daerah di sebelah timur Kanaan, supaya menjauhi Ishak anaknya.
\par 7 Abraham meninggal pada usia yang lanjut, yaitu 175 tahun.
\par 9 Ia dikuburkan oleh anak-anaknya, yaitu Ishak dan Ismael, di Gua Makhpela yang terletak di ladang sebelah timur Mamre. Dahulu ladang itu milik Efron anak Zohar, orang Het,
\par 10 tetapi sudah dibeli oleh Abraham dari orang Het itu. Sara, istri Abraham, juga dikuburkan di sana.
\par 11 Setelah Abraham meninggal, Allah memberkati Ishak, anak Abraham. Pada waktu itu Ishak tinggal di dekat sumur yang disebut "Dia Yang Memperhatikan Aku".
\par 12 Ismael adalah anak Abraham dan Hagar, wanita Mesir, hamba Sara itu.
\par 13 Ismael mempunyai dua belas anak yang disebutkan di sini menurut urutan lahirnya: Nebayot, Kedar, Adbeel, Mibsam, Misyma, Duma, Masa, Hadad, Tema, Yetur, Nafis dan Kedma. Anak-anak itu menjadi bapak leluhur dua belas suku bangsa, dan desa dan perkemahan mereka disebut menurut nama-nama mereka.
\par 17 Ismael berumur 137 tahun ketika ia meninggal.
\par 18 Keturunannya tinggal di daerah antara Hawila dan Syur di sebelah timur Mesir ke arah Asyur. Mereka hidup terpisah dari keturunan Abraham yang lain.
\par 19 Inilah riwayat Ishak, anak Abraham.
\par 20 Ishak berumur empat puluh tahun ketika ia menikah dengan Ribka, adik Laban, anak Betuel, seorang Aram dari Mesopotamia.
\par 21 Karena Ribka tidak mendapat anak, Ishak berdoa kepada TUHAN untuk istrinya. Dan TUHAN mengabulkan doa Ishak, sehingga Ribka mengandung.
\par 22 Ia mengandung anak kembar, dan sebelum anak-anak itu lahir, mereka bergelut di dalam rahimnya. Kata Ribka, "Mengapa hal ini harus terjadi pada diriku?" Lalu ia memohon petunjuk kepada TUHAN.
\par 23 TUHAN berkata kepadanya, "Dua bangsa ada di dalam rahimmu; kau akan melahirkan dua bangsa yang berseteru; adiknya lebih kuat dari kakaknya yang sulung melayani yang bungsu."
\par 24 Tibalah saatnya bagi Ribka untuk bersalin, dan ia melahirkan dua anak laki-laki kembar.
\par 25 Yang sulung warnanya kemerah-merahan, dan kulitnya seperti jubah yang berbulu. Sebab itu ia dinamakan Esau.
\par 26 Waktu anak yang kedua dilahirkan, tangannya memegang tumit Esau. Sebab itu ia dinamakan Yakub. Ishak berumur enam puluh tahun pada waktu mereka lahir.
\par 27 Kedua anak itu bertambah besar. Esau menjadi pemburu yang cakap dan suka tinggal di padang, tetapi Yakub bersifat tenang dan suka tinggal di rumah.
\par 28 Ishak lebih sayang kepada Esau, sebab Ishak suka makan daging buruan. Tetapi Ribka lebih sayang kepada Yakub.
\par 29 Pada suatu hari ketika Yakub sedang memasak sayur kacang merah, datanglah Esau dari perburuannya. Ia lapar.
\par 30 Katanya kepada Yakub, "Saya lapar sekali. Minta sedikit kacang merah itu." (Itulah sebabnya ia disebut Edom.)
\par 31 Jawab Yakub, "Baik, asal kauberikan lebih dahulu hakmu sebagai anak sulung."
\par 32 Kata Esau, "Peduli apa hak itu bagi saya. Saya lapar setengah mati!"
\par 33 Kata Yakub pula, "Bersumpahlah dulu bahwa kauberikan hakmu kepada saya." Esau bersumpah dan memberi haknya kepada Yakub.
\par 34 Setelah itu Yakub memberi kepadanya roti dan sebagian dari sayur kacang merah itu. Esau makan dan minum lalu pergi. Demikianlah Esau meremehkan haknya sebagai anak sulung.

\chapter{26}

\par 1 Di negeri itu timbul lagi kelaparan seperti yang telah terjadi pada zaman Abraham. Ishak menghadap Abimelekh, raja orang Filistin, di Gerar dan minta izin untuk tinggal di sana.
\par 2 Hal itu dilakukannya karena TUHAN telah menampakkan diri kepadanya dan berkata, "Jangan pergi ke Mesir; tinggallah di negeri ini, seperti yang telah Kuperintahkan kepadamu.
\par 3 Aku akan melindungi dan memberkatimu. Seluruh wilayah ini akan Kuberikan kepadamu dan kepada keturunanmu. Aku akan menepati janji yang telah Kuberikan kepada Abraham ayahmu.
\par 4 Aku akan memberikan kepadamu anak cucu sebanyak bintang di langit, dan seluruh wilayah ini akan Kuberikan kepada mereka. Semua bangsa di bumi akan mohon kepada-Ku supaya Aku memberkati mereka sebagaimana telah Kuberkati keturunanmu.
\par 5 Aku akan memberkati kamu, karena Abraham telah mentaati Aku dan memelihara segala hukum dan perintah-Ku."
\par 6 Jadi tinggallah Ishak di Gerar.
\par 7 Ketika orang-orang di situ bertanya tentang Ribka, Ishak menjawab bahwa dia itu adiknya, dan bukan istrinya. Ishak takut orang-orang itu akan membunuhnya untuk mendapat Ribka yang amat cantik itu.
\par 8 Setelah Ishak tinggal di sana beberapa lama, pada suatu hari Raja Abimelekh memandang dari jendelanya, dan dilihatnya Ishak sedang bercumbu-cumbu dengan Ribka.
\par 9 Maka dipanggilnya Ishak dan ia berkata, "Bukankah wanita itu istrimu? Mengapa engkau mengatakan bahwa dia adikmu?" Jawab Ishak, "Saya pikir, saya akan dibunuh jika mengatakan bahwa dia istri saya."
\par 10 "Apa yang telah kaulakukan terhadap kami?" kata Abimelekh. "Salah seorang dari rakyatku bisa saja tidur dengan istrimu, dan kalau hal itu terjadi, engkaulah yang bertanggung jawab atas kesalahan kami itu."
\par 11 Kemudian Abimelekh memperingatkan seluruh rakyatnya demikian, "Barangsiapa mengganggu orang ini atau istrinya, akan dihukum mati."
\par 12 Ishak menaburkan bibit untuk bercocok tanam di negeri itu, dan tahun itu mendapat hasil seratus kali lipat dari yang ditaburkannya, karena TUHAN memberkati dia.
\par 13 Kekayaannya semakin bertambah dan ia menjadi kaya raya.
\par 14 Orang Filistin cemburu kepadanya karena ia mempunyai banyak hamba, kawanan domba dan sapi.
\par 15 Lalu mereka menimbuni dengan tanah semua sumur yang telah digali oleh hamba-hamba Abraham semasa ia masih hidup.
\par 16 Kemudian Abimelekh berkata kepada Ishak, "Pergilah dari negeri kami. Engkau sudah lebih berkuasa dari kami."
\par 17 Jadi pergilah Ishak dan berkemah di Lembah Gerar. Di situ ia tinggal beberapa lamanya.
\par 18 Dia menggali kembali sumur-sumur yang telah digali pada zaman Abraham dan yang telah ditimbuni dengan tanah oleh orang Filistin sepeninggal Abraham. Sumur-sumur itu diberi nama yang sama seperti yang diberikan ayahnya.
\par 19 Hamba-hamba Ishak menggali sumur di Lembah Gerar itu dan menemukan air berlimpah-limpah.
\par 20 Tetapi para gembala di Gerar bertengkar dengan para gembala Ishak. Kata mereka, "Air ini milik kami." Karena itu Ishak menamakan sumur itu "Pertengkaran".
\par 21 Hamba-hamba Ishak menggali sumur yang lain lagi, maka terjadilah pertengkaran mengenai sumur itu juga. Karena itu, Ishak menamakan sumur itu "Permusuhan".
\par 22 Kemudian ia pindah dari situ dan menggali sumur yang ketiga. Kali ini tidak terjadi pertengkaran. Karena itu, Ishak menamakan sumur itu "Kebebasan". Katanya, "Sekarang TUHAN telah memberi kita kebebasan untuk tinggal di negeri ini, dan kita akan menjadi makmur di sini."
\par 23 Kemudian Ishak berangkat dan tiba di Bersyeba.
\par 24 Pada malam itu TUHAN menampakkan diri kepadanya dan berkata, "Akulah Allah ayahmu Abraham. Jangan takut; Aku melindungimu. Aku akan memberkatimu dan memberi kepadamu keturunan yang banyak karena janji-Ku kepada hamba-Ku Abraham."
\par 25 Lalu Ishak mendirikan mezbah dan menyembah TUHAN di tempat itu. Ia berkemah di situ dan hamba-hambanya menggali sumur yang lain.
\par 26 Pada suatu hari datanglah Abimelekh dari Gerar, dengan Ahuzat, penasihatnya, dan Pikhol, panglima tentaranya, untuk menemui Ishak.
\par 27 Maka bertanyalah Ishak, "Mengapa kalian datang menemui saya sekarang, padahal dahulu kalian memusuhi dan mengusir saya dari negeri kalian?"
\par 28 Lalu mereka menjawab, "Sekarang kami tahu bahwa TUHAN melindungi engkau dan pada hemat kami sebaiknya kita mengadakan perjanjian. Bersumpahlah
\par 29 bahwa engkau tidak akan berbuat jahat terhadap kami, sebagaimana kami pun tidak pernah berbuat jahat terhadapmu. Kami telah bersikap baik terhadapmu dan membiarkan engkau pergi dengan damai. Sudah jelas bagi kami bahwa TUHAN memberkati engkau."
\par 30 Kemudian Ishak menjamu mereka, dan mereka makan dan minum.
\par 31 Keesokan harinya pagi-pagi masing-masing mengucapkan janji yang disahkan dengan sumpah. Lalu Ishak mengucapkan selamat jalan kepada mereka dan berpisahlah mereka sebagai sahabat.
\par 32 Pada hari itu juga hamba-hamba Ishak datang dan melaporkan kepadanya tentang sumur yang telah mereka gali. Kata mereka, "Kami telah mendapatkan air."
\par 33 Lalu Ishak menamakan sumur itu "Sumpah". Itulah asal mulanya kota itu dinamakan Bersyeba.
\par 34 Ketika Esau berumur empat puluh tahun, ia menikah dengan dua gadis dari suku bangsa Het, yaitu Yudit anak Beeri, dan Basmat anak Elon.
\par 35 Kedua wanita itu menyusahkan hidup Ishak dan Ribka.

\chapter{27}

\par 1 Pada suatu hari, ketika Ishak sudah tua dan buta pula, dipanggilnya Esau, anaknya yang sulung, lalu berkata kepadanya, "Anakku!" "Ya, Ayah," jawab Esau.
\par 2 Ishak berkata, "Engkau tahu bahwa saya sudah tua dan mungkin tidak akan hidup lama lagi.
\par 3 Jadi ambillah busur dan panah-panahmu, pergilah memburu seekor binatang di padang.
\par 4 Masaklah yang enak seperti yang saya sukai, lalu bawalah kepada saya. Setelah saya memakannya, akan saya berikan berkat saya kepadamu sebelum saya mati."
\par 5 Percakapan Ishak dengan Esau itu didengar oleh Ribka. Maka setelah Esau berangkat untuk berburu,
\par 6 berkatalah Ribka kepada Yakub, "Baru saja saya dengar ayahmu mengatakan kepada Esau begini,
\par 7 'Burulah seekor binatang dan masaklah yang enak untukku. Setelah aku memakannya, akan kuberikan berkatku kepadamu di hadapan TUHAN, sebelum aku mati.'
\par 8 Nah, anakku," kata Ribka lagi, "dengarkanlah dan lakukanlah apa yang saya katakan ini.
\par 9 Pergilah ke tempat domba kita, dan pilihlah dua anak kambing yang gemuk-gemuk, supaya saya masak menjadi makanan kesukaan ayahmu.
\par 10 Kemudian bawalah kepadanya supaya dimakannya, dan setelah itu ia akan memberikan berkatnya kepadamu sebelum ia meninggal."
\par 11 Tetapi Yakub berkata kepada ibunya, "Ibu, bukankah badan Esau berbulu, sedangkan badan saya tidak?
\par 12 Jangan-jangan ayah meraba badan saya dan mengetahui bahwa saya menipunya; nanti ia bukannya memberikan berkat, malahan mengutuki saya."
\par 13 Ibunya menjawab, "Jangan khawatir, Nak. Biar saya yang menanggung segala kutuknya. Lakukanlah saja apa yang saya katakan, pergilah mengambil kambing-kambing itu."
\par 14 Maka pergilah Yakub mengambil kambing-kambing itu dan membawanya kepada ibunya, lalu Ribka memasak makanan kesukaan Ishak.
\par 15 Kemudian Ribka mengambil pakaian Esau yang paling bagus, yang disimpannya di rumah, lalu dikenakannya pada Yakub.
\par 16 Ia membalutkan juga kulit anak kambing pada lengan dan leher Yakub yang tidak berbulu itu.
\par 17 Lalu diberikannya kepada Yakub masakan yang enak itu dengan roti yang telah dibuatnya.
\par 18 Setelah itu pergilah Yakub kepada ayahnya dan berkata, "Ayah!" "Ya," jawab Ishak, "siapa engkau, Esau atau Yakub?"
\par 19 Jawab Yakub, "Esau, anak ayah yang sulung; pesan ayah sudah saya lakukan. Duduklah dan makanlah daging buruan yang saya bawakan ini, supaya ayah dapat memberkati saya."
\par 20 Ishak berkata, "Cepat sekali engkau mendapatnya, Nak." Jawab Yakub, "Karena TUHAN Allah yang disembah ayah telah menolong saya."
\par 21 Lalu kata Ishak kepada Yakub, "Marilah dekat-dekat supaya saya dapat merabamu. Benarkah engkau Esau?"
\par 22 Yakub mendekati ayahnya, dan ayahnya itu merabanya serta berkata, "Suaramu seperti suara Yakub, tetapi lenganmu seperti lengan Esau."
\par 23 Ishak tidak mengenali Yakub karena lengannya berbulu seperti lengan Esau. Tetapi pada saat Yakub hendak diberkatinya,
\par 24 ia masih bertanya sekali lagi, "Benarkah engkau Esau?" "Benar," jawab Yakub.
\par 25 Lalu berkatalah Ishak, "Berilah saya daging itu. Setelah saya makan akan saya berikan berkat saya kepadamu." Yakub memberikan daging itu kepadanya dan juga sedikit anggur untuk diminum.
\par 26 Lalu berkatalah ayahnya kepadanya, "Marilah lebih dekat lagi, Nak, dan ciumlah saya."
\par 27 Ketika Yakub mendekat untuk mengecupnya, Ishak mencium bau pakaian Esau, lalu diberkatinya dia. Kata Ishak, "Bau sedap anak saya seperti bau padang yang telah diberkati TUHAN.
\par 28 Semoga Allah memberikan kepadamu embun dari langit, dan membuat ladang-ladangmu subur! Semoga Dia memberikan kepadamu gandum dan anggur berlimpah-limpah!
\par 29 Semoga bangsa-bangsa menjadi hambamu, dan suku-suku bangsa takluk kepadamu. Semoga engkau menguasai semua sanak saudaramu, dan keturunan ibumu sujud di hadapanmu. Semoga terkutuklah semua orang yang mengutuk engkau dan diberkatilah semua orang yang memberkati engkau."
\par 30 Segera sesudah Ishak memberikan berkatnya dan Yakub pergi, Esau, abangnya, pulang dari berburu.
\par 31 Dia juga memasak makanan yang enak lalu membawanya kepada ayahnya, katanya, "Duduklah, Ayah, dan makanlah daging yang saya bawa untuk Ayah, supaya Ayah dapat memberkati saya."
\par 32 "Siapa engkau?" tanya Ishak. "Esau anak Ayah yang sulung," jawabnya.
\par 33 Ishak mulai gemetar seluruh tubuhnya, dan dia bertanya, "Jika begitu, siapa yang telah memburu binatang dan membawanya kepada saya tadi? Saya telah memakannya sebelum engkau tiba. Lalu saya telah berikan berkat saya yang terakhir kepadanya, dan kini berkat itu menjadi miliknya selama-lamanya."
\par 34 Setelah Esau mendengar itu, dia menangis dengan nyaring dan penuh kepedihan, lalu katanya, "Berkatilah saya juga, Ayah!"
\par 35 Ishak berkata, "Adikmu telah datang kemari dan menipu saya. Dia telah mengambil berkat yang sebetulnya akan saya berikan kepadamu."
\par 36 Esau berkata, "Inilah kedua kalinya dia menipu saya. Pantas namanya Yakub. Dia telah mengambil hak saya sebagai anak sulung, dan sekarang ia mengambil pula berkat yang untuk saya. Apakah Ayah tidak mempunyai berkat lain bagi saya?"
\par 37 Ishak menjawab, "Saya telah menjadikan dia tuanmu, dan semua sanak saudaranya saya jadikan hambanya. Saya telah memberikan kepadanya gandum dan anggur. Sekarang tidak ada apa-apa lagi yang dapat saya lakukan untukmu, Nak!"
\par 38 Esau tidak mau berhenti memohon kepada ayahnya, "Apakah Ayah hanya mempunyai satu berkat saja? Berkatilah saya juga, Ayah!" Lalu mulailah dia menangis lagi.
\par 39 Kemudian Ishak berkata kepadanya, "Tidak akan ada embun dari langit bagimu, tidak akan ada ladang yang subur untukmu.
\par 40 Engkau akan hidup dari pedangmu, namun menjadi hamba adikmu, tetapi bila engkau memberontak, engkau akan lepas dari kuasanya."
\par 41 Maka Esau membenci Yakub karena ayahnya telah memberikan berkatnya kepada adiknya itu. Pikirnya, "Tidak lama lagi ayah meninggal dan sehabis kita berkabung, Yakub akan saya bunuh!"
\par 42 Ketika Ribka mendengar tentang rencana Esau itu, ia menyuruh memanggil Yakub dan berkata, "Dengarkan, abangmu Esau bermaksud membalas dendam dan membunuh engkau.
\par 43 Nah, lakukanlah apa yang saya katakan ini. Pergilah dengan segera kepada abang saya Laban di Haran,
\par 44 dan tinggallah bersama dia untuk beberapa waktu lamanya, sampai kemarahan abangmu reda
\par 45 dan ia melupakan apa yang telah engkau lakukan terhadapnya. Kemudian saya akan menyuruh orang membawamu pulang kemari. Tidak mau saya kehilangan kedua anak saya pada hari yang sama."
\par 46 Maka berkatalah Ribka kepada Ishak, "Saya jemu dan bosan melihat istri-istri Esau dari suku bangsa asing itu. Jika Yakub juga kawin dengan gadis Het, lebih baik saya mati saja."

\chapter{28}

\par 1 Kemudian Ishak memanggil Yakub, lalu memberkati dia serta berkata, "Janganlah engkau kawin dengan gadis Kanaan.
\par 2 Pergilah ke Mesopotamia, kepada keluarga Betuel, kakekmu, dan kawinlah dengan salah seorang anak pamanmu Laban.
\par 3 Semoga Allah Yang Mahakuasa memberkati perkawinanmu dan memberikan kepadamu anak cucu yang banyak, sehingga engkau menjadi leluhur banyak bangsa!
\par 4 Semoga TUHAN memberkati engkau dan keturunanmu sebagaimana telah diberkatinya Abraham, dan semoga engkau memiliki tanah yang kaudiami ini, yang telah diberikan Allah kepada Abraham!"
\par 5 Demikianlah Ishak melepas Yakub pergi ke Mesopotamia, kepada Laban, anak Betuel orang Aram itu. Laban adalah abang Ribka ibu Yakub dan Esau.
\par 6 Esau mendengar bahwa Ishak telah memberkati Yakub dan melepas dia pergi ke Mesopotamia untuk mencari istri di situ. Juga bahwa pada waktu Ishak memberkati Yakub, ia melarang Yakub kawin dengan gadis Kanaan.
\par 7 Selain itu Esau tahu bahwa Yakub mentaati perkataan ayah ibunya dan telah berangkat ke Mesopotamia.
\par 8 Maka mengertilah Esau bahwa Ishak, ayahnya, tidak suka pada perempuan Kanaan,
\par 9 sebab itu pergilah Esau kepada Ismael, anak Abraham, dan kawin dengan anak Ismael yang bernama Mahalat, adik Nebayot.
\par 10 Yakub meninggalkan Bersyeba dan berangkat ke Haran.
\par 11 Pada waktu matahari terbenam ia sampai di suatu tempat, lalu bermalam di situ. Ia berbaring hendak tidur; kepalanya berbantalkan sebuah batu.
\par 12 Kemudian bermimpilah ia bahwa ia melihat sebuah tangga yang berdiri di bumi dan ujungnya mencapai langit dan malaikat-malaikat turun naik di tangga itu.
\par 13 Dan di sampingnya berdirilah TUHAN dan berkata, "Akulah TUHAN, Allah yang dipuja Abraham dan Ishak. Aku akan memberikan kepadamu dan kepada keturunanmu tanah tempat engkau berbaring ini.
\par 14 Keturunanmu akan sebanyak debu di bumi. Mereka akan memperluas wilayah mereka ke segala arah, dan melalui engkau dan keturunanmu, Aku akan memberkati semua bangsa di bumi.
\par 15 Ingatlah, Aku akan menolong dan melindungimu, ke mana pun engkau pergi, dan Aku akan membawamu kembali ke negeri ini. Aku tak akan meninggalkan engkau sampai telah Kulakukan segala apa yang Kujanjikan kepadamu."
\par 16 Lalu Yakub bangun dari tidurnya dan berkata, "TUHAN ada di sini, di tempat ini, tetapi baru sekarang aku mengetahuinya!"
\par 17 Ia menjadi takut lalu berkata, "Alangkah seramnya tempat ini! Pastilah ini rumah Allah, pintu gerbang surga."
\par 18 Keesokan harinya pagi-pagi, Yakub bangkit, lalu mengambil batu yang dipakainya sebagai bantal dan menegakkannya menjadi batu peringatan. Dituangkannya minyak zaitun di atas batu itu untuk dikhususkan bagi Allah.
\par 19 Tempat itu dinamakannya Betel. (Dahulu namanya Lus).
\par 20 Lalu bersumpahlah Yakub, "Jika TUHAN menolong dan melindungi saya pada perjalanan ini serta memberikan kepada saya makanan dan pakaian
\par 21 sehingga saya kembali ke rumah ayah saya dengan selamat, TUHAN akan menjadi Allah saya.
\par 22 Dan saya akan memberikan kepada TUHAN sepersepuluh dari segala sesuatu yang TUHAN berikan kepada saya. Batu peringatan yang saya tegakkan ini akan menjadi tempat pemujaan bagi TUHAN."

\chapter{29}

\par 1 Yakub meneruskan perjalanannya dan pergi ke negeri di sebelah timur Kanaan.
\par 2 Tiba-tiba dilihatnya sebuah sumur di padang, dan di dekatnya berbaringlah tiga kawanan kambing domba yang biasanya diberi minum di situ. Sumur itu tertutup batu besar.
\par 3 Kalau semua kawanan kambing domba sudah berkumpul di tempat itu, para gembala menggulingkan batu itu ke samping, lalu kambing domba itu diberi minum. Setelah itu para gembala menutup sumur itu kembali.
\par 4 Lalu bertanyalah Yakub kepada para gembala yang ada di situ, "Hai kawan, dari mana kalian ini?" "Dari Haran," jawab mereka.
\par 5 Yakub bertanya lagi, "Kenalkah kalian kepada Laban anak Nahor?" "Kenal," jawab mereka.
\par 6 "Bagaimana keadaannya?" tanya Yakub lagi. "Baik-baik saja," jawab mereka. "Lihatlah, itu anaknya perempuan yang bernama Rahel. Ia datang membawa kawanan kambing domba ayahnya."
\par 7 Kata Yakub, "Hari masih siang dan belum tiba waktunya kambing domba dimasukkan ke dalam kandang. Mengapa binatang-binatang itu tidak kalian beri minum, lalu kalian biarkan makan rumput lagi?"
\par 8 Jawab mereka, "Kami baru dapat memberi minum domba-domba itu kalau semua kawanan domba sudah berkumpul di sini dan batu penutup sudah digulingkan."
\par 9 Selagi Yakub berbicara dengan mereka, Rahel datang dengan kawanan kambing dombanya.
\par 10 Ketika Yakub melihat Rahel bersama kawanannya itu, pergilah Yakub ke sumur itu, lalu menggulingkan batu penutupnya ke samping dan memberi minum kawanan itu.
\par 11 Kemudian ia mencium Rahel, dan menangis karena terharu.
\par 12 Katanya kepada Rahel, "Saya anak Ribka, saudara ayahmu." Lalu berlarilah Rahel pulang ke rumahnya untuk menceritakan hal itu kepada ayahnya.
\par 13 Ketika Laban mendengar berita tentang Yakub, kemanakannya itu, berlarilah ia hendak menemuinya. Laban memeluk Yakub dan menciumnya, lalu membawanya masuk ke dalam rumah. Setelah Yakub menceritakan segala hal ihwalnya kepada Laban,
\par 14 Laban berkata, "Memang benar, engkau keluarga dekat dengan saya." Yakub tinggal di rumah itu sebulan penuh.
\par 15 Pada suatu hari berkatalah Laban kepada Yakub, "Tidak patut engkau bekerja padaku dengan cuma-cuma, hanya karena engkau sanak saudaraku. Berapa upah yang kauinginkan?"
\par 16 Adapun Laban mempunyai dua anak perempuan; yang sulung bernama Lea, dan adiknya bernama Rahel.
\par 17 Lea matanya indah, sedangkan Rahel bertubuh molek dan berwajah cantik.
\par 18 Yakub telah jatuh cinta kepada Rahel, sebab itu ia menjawab kepada Laban, "Saya mau bekerja selama tujuh tahun jika Paman mengizinkan saya kawin dengan Rahel."
\par 19 Kata Laban, "Lebih baik saya berikan dia kepadamu daripada kepada orang lain; jadi tinggallah di sini."
\par 20 Maka bekerjalah Yakub selama tujuh tahun untuk mendapat Rahel, dan baginya tahun-tahun itu berlalu seperti beberapa hari saja, karena cintanya kepada gadis itu.
\par 21 Sesudah itu Yakub berkata kepada Laban, "Tujuh tahun sudah lewat; izinkanlah saya kawin dengan anak Paman."
\par 22 Lalu Laban mengadakan pesta perkawinan dan mengundang semua orang di kota itu.
\par 23 Tetapi pada malam itu, bukan Rahel, melainkan Lea yang diantarkan Laban kepada Yakub, dan Yakub bersetubuh dengan dia.
\par 24 (Laban memberikan hambanya perempuan yang bernama Zilpa kepada Lea, untuk menjadi hambanya.)
\par 25 Baru keesokan harinya Yakub mengetahui bahwa Lea yang dikawininya. Lalu pergilah Yakub kepada Laban dan berkata, "Mengapa Paman perlakukan saya begini? Bukankah saya bekerja untuk mendapat Rahel? Mengapa Paman menipu saya?"
\par 26 Laban menjawab, "Menurut adat di sini, adik tidak boleh kawin lebih dahulu daripada kakak.
\par 27 Tunggulah sampai selesai pesta tujuh hari perkawinan, nanti saya berikan Rahel kepadamu juga, asal engkau mau bekerja tujuh tahun lagi."
\par 28 Yakub setuju, dan setelah pesta itu lewat, Laban memberikan kepadanya Rahel menjadi istrinya.
\par 29 (Laban memberikan hambanya perempuan yang bernama Bilha kepada Rahel, untuk menjadi hambanya.)
\par 30 Yakub bersetubuh dengan Rahel juga, dan ia lebih mencintai Rahel daripada Lea. Kemudian ia bekerja tujuh tahun lagi.
\par 31 Ketika TUHAN melihat bahwa Lea tidak begitu dicintai seperti Rahel, TUHAN mengizinkan Lea melahirkan anak, tetapi Rahel tetap tidak mendapat anak.
\par 32 Lea mengandung lalu melahirkan seorang anak laki-laki. Katanya, "TUHAN telah melihat kesusahan saya, dan sekarang suami saya akan mencintai saya." Lalu dinamakannya anaknya itu Ruben.
\par 33 Lea mengandung lagi lalu melahirkan seorang anak laki-laki. Kata Lea, "TUHAN telah memberikan kepada saya anak ini pula karena didengarnya bahwa saya tidak dicintai." Lalu dinamakannya anaknya itu Simeon.
\par 34 Lea mengandung lagi lalu melahirkan seorang anak laki-laki. Katanya, "Sekarang suami saya akan lebih terikat kepada saya karena saya telah melahirkan tiga anak laki-laki." Lalu dinamakannya anaknya itu Lewi.
\par 35 Kemudian Lea hamil lagi dan melahirkan seorang anak laki-laki. Katanya, "Kali ini saya akan memuji TUHAN." Lalu dinamakannya anaknya itu Yehuda. Sesudah itu Lea tidak melahirkan lagi.

\chapter{30}

\par 1 Tetapi Rahel belum juga mendapat anak. Sebab itu dia menjadi cemburu kepada kakaknya, lalu berkata kepada Yakub, "Berikanlah anak kepada saya, kalau tidak, saya akan mati."
\par 2 Yakub marah kepada Rahel dan berkata, "Saya ini bukan Allah. Dialah yang membuat engkau tidak mendapat anak."
\par 3 Lalu kata Rahel, "Ini hamba saya Bilha; tidurlah dengan dia supaya ia melahirkan. Dengan demikian saya bisa menjadi ibu melalui dia."
\par 4 Lalu diberikannya Bilha kepada suaminya, dan Yakub bersetubuh dengan hamba itu.
\par 5 Bilha mengandung lalu melahirkan seorang anak laki-laki.
\par 6 Rahel berkata, "Allah telah mengadili untuk kepentingan saya. Didengarnya doaku dan diberikannya seorang anak laki-laki kepada saya." Lalu dinamakannya anak itu Dan.
\par 7 Bilha mengandung lagi dan melahirkan seorang anak laki-laki.
\par 8 Kata Rahel, "Saya telah berjuang mati-matian melawan kakak saya, dan saya menang," karena itu dinamakannya anak itu Naftali.
\par 9 Ketika Lea menyadari bahwa ia tidak bisa mendapat anak lagi, diberikannya hambanya, yaitu Zilpa, kepada Yakub untuk menjadi istri.
\par 10 Zilpa melahirkan seorang anak laki-laki.
\par 11 Lea berkata, "Saya beruntung," lalu dinamakannya anak itu Gad.
\par 12 Zilpa melahirkan seorang anak laki-laki lagi,
\par 13 dan Lea berkata, "Alangkah berbahagia saya! Sekarang semua wanita akan mengatakan saya berbahagia." Karena itu dinamakannya anak itu Asyer.
\par 14 Pada musim panen gandum, Ruben pergi ke padang dan ditemukannya di sana sejenis tanaman obat, lalu dibawanya kepada Lea, ibunya. Rahel berkata kepada Lea, "Berikanlah kepada saya sedikit dari tanaman obat yang ditemukan anakmu itu."
\par 15 Jawab Lea, "Belum cukupkah engkau mengambil suami saya? Sekarang engkau malahan mencoba pula mengambil tanaman obat yang ditemukan anak saya." Kata Rahel, "Jika engkau mau memberikan tanaman obat anakmu itu kepada saya, engkau boleh tidur dengan Yakub malam ini."
\par 16 Ketika Yakub pulang ke rumah dari padang sore itu, Lea menyambutnya sambil berkata, "Tidurlah bersama saya malam ini karena saya telah membayar untuk itu dengan tanaman obat dari anak saya." Lalu Yakub bersetubuh dengan dia pada malam itu.
\par 17 Allah mengabulkan doa Lea, dan ia mengandung lalu melahirkan anak yang kelima.
\par 18 Kata Lea, "Allah telah memberi saya upah sebab saya telah menyerahkan Zilpa kepada suami saya," karena itu dinamakannya anaknya itu Isakhar.
\par 19 Lea mengandung lagi dan melahirkan anaknya yang keenam.
\par 20 Katanya, "Allah telah memberikan kepada saya hadiah yang indah sekali. Sekarang suami saya akan menghargai saya, sebab saya telah melahirkan enam anak laki-laki," karena itu dinamakannya anaknya itu Zebulon.
\par 21 Sesudah itu Lea melahirkan seorang anak perempuan yang dinamakannya Dina.
\par 22 Lalu ingatlah Allah akan Rahel; Allah mengabulkan doanya dan memungkinkan dia melahirkan anak.
\par 23 Maka Rahel mengandung dan melahirkan seorang anak laki-laki. Kata Rahel, "Allah telah menghilangkan kehinaan saya.
\par 24 Semoga TUHAN memberikan kepada saya anak laki-laki seorang lagi." Karena itu dinamakannya anaknya itu Yusuf.
\par 25 Sesudah Rahel melahirkan Yusuf, Yakub berkata kepada Laban, "Izinkanlah saya pulang ke negeri saya.
\par 26 Biarkan saya pergi membawa kedua istri serta anak-anak saya. Mereka telah saya peroleh dengan bekerja pada Paman. Paman tahu, saya sudah bekerja keras sekali."
\par 27 Laban berkata kepadanya, "Dengar dulu: Saya mendapat ramalan bahwa TUHAN telah memberkati saya karena engkau.
\par 28 Katakan saja berapa yang kauminta."
\par 29 Jawab Yakub, "Paman sendiri tahu saya sudah bekerja keras sekali, dan ternak Paman bertambah banyak karena pemeliharaan saya.
\par 30 Sebelum saya datang, harta milik Paman tidak seberapa, tetapi sekarang Paman sudah kaya. TUHAN telah memberkati Paman karena saya. Sekarang sudah waktunya saya bekerja untuk keluarga saya sendiri."
\par 31 Kata Laban, "Jadi apa yang harus kuberikan kepadamu?" Yakub menjawab, "Tak usah Paman berikan apa-apa. Saya akan tetap memelihara ternak Paman, asal Paman setuju dengan usul saya ini:
\par 32 Izinkan saya pada hari ini berjalan di tengah-tengah segala kawanan kambing domba Paman, dan mengambil setiap anak domba yang hitam, dan juga setiap kambing muda yang belang atau berbintik-bintik. Itu saja upah yang saya inginkan.
\par 33 Di kemudian hari dengan mudah dapat Paman ketahui apakah saya telah berlaku jujur. Bilamana Paman datang untuk memeriksa upah saya, dan melihat ada kambing yang tidak berbintik-bintik atau belang, atau domba yang tidak hitam, Paman akan tahu bahwa binatang itu telah saya curi."
\par 34 Laban menjawab, "Setuju! Kita akan melakukan apa yang kauusulkan itu!"
\par 35 Tetapi pada hari itu Laban memisahkan semua kambing jantan yang loreng atau belang dan semua kambing betina yang belang, berbintik-bintik, atau yang ada noda putihnya; ia juga memisahkan semua domba yang hitam. Kemudian disuruhnya anak-anaknya mengurus binatang-binatang itu,
\par 36 lalu pergilah mereka dengan binatang-binatang itu menjauhi Yakub, jaraknya tiga hari perjalanan. Sisa dari kawanan kambing domba milik Laban dipelihara oleh Yakub.
\par 37 Kemudian Yakub mengambil dahan-dahan hijau dari pohon hawar, pohon badam dan pohon berangan, dan mengupas sebagian dari kulitnya sehingga menjadi belang-belang.
\par 38 Diletakkannya dahan-dahan itu di depan kawanan kambing itu, di dalam tempat minumnya, karena binatang-binatang itu suka kawin waktu hendak minum.
\par 39 Jadi jika kambing-kambing itu kawin di depan dahan-dahan itu, anaknya loreng atau belang atau berbintik-bintik.
\par 40 Kemudian Yakub mengambil domba-domba lalu menghadapkannya ke arah kambing-kambing yang loreng dan hitam. Dengan cara itu ia memperoleh kawanan binatangnya sendiri dan memisahkannya dari kawanan binatang Laban.
\par 41 Apabila binatang-binatang yang kuat sedang kawin, Yakub meletakkan dahan-dahan yang dikupasnya itu di depan mereka, di dalam tempat minumnya.
\par 42 Tetapi dahan-dahan itu tidak diletakkannya di depan binatang-binatang yang lemah. Jadi, semua binatang yang putih itu lemah, dan itulah milik Laban, sedangkan semua binatang yang hitam, belang dan berbintik-bintik itu kuat, dan itulah milik Yakub.
\par 43 Dengan demikian Yakub menjadi kaya raya. Dia mempunyai banyak kawanan kambing domba, unta, keledai dan hamba-hamba.

\chapter{31}

\par 1 Yakub mendengar anak-anak Laban berkata begini, "Yakub telah mengambil segala harta milik ayah kita. Semua kekayaannya berasal dari harta ayah kita."
\par 2 Yakub melihat juga bahwa Laban tidak lagi ramah seperti dahulu kepadanya.
\par 3 Lalu TUHAN berkata kepada Yakub, "Kembalilah ke negeri bapakmu dan kepada sanak saudaramu. Aku akan melindungi engkau."
\par 4 Sesudah itu Yakub menyuruh orang memberitahukan kepada Rahel dan Lea supaya bertemu dengan dia di padang, di tempat kawanan kambing dombanya.
\par 5 Lalu kata Yakub kepada mereka, "Saya melihat bahwa ayahmu tidak ramah lagi seperti biasanya kepada saya; tetapi Allah yang dipuja ayah saya melindungi saya.
\par 6 Kamu berdua tahu juga bahwa saya telah bekerja pada ayahmu sekuat tenaga saya.
\par 7 Meskipun begitu saya telah ditipunya dan diubahnya upah saya sampai sepuluh kali. Tetapi Allah tidak membiarkan dia berbuat jahat kepada saya.
\par 8 Sewaktu ayahmu berkata, 'Kambing yang berbintik-bintik akan menjadi upahmu,' maka segala kawanan melahirkan anak yang berbintik-bintik. Sewaktu ia berkata, 'Kambing yang belang akan menjadi upahmu,' maka segala kawanan itu melahirkan anak yang belang.
\par 9 Demikianlah Allah telah mengambil kawanan binatang milik ayahmu lalu memberikannya kepada saya.
\par 10 Dan pada musim kawin binatang-binatang itu, saya bermimpi melihat semua kambing jantan yang sedang kawin, loreng, berbintik-bintik dan belang.
\par 11 Dan Allah dengan rupa malaikat berbicara dengan saya dalam mimpi itu, kata-Nya, 'Yakub!' Saya pun menjawab, 'Ya'.
\par 12 Malaikat itu berkata lagi, 'Lihat, semua kambing jantan yang sedang kawin itu loreng, dan berbintik-bintik dan belang. Aku sengaja mengatur itu bagimu karena telah Kulihat semua yang dilakukan oleh Laban terhadapmu.
\par 13 Akulah Allah yang telah menampakkan diri kepadamu di Betel. Di sana engkau telah mempersembahkan sebuah batu peringatan, dengan menuangkan minyak zaitun diatasnya. Di sana pula engkau bersumpah kepada-Ku. Sekarang bersiaplah dan kembalilah ke negeri kelahiranmu.'"
\par 14 Lalu Lea dan Rahel menjawab, "Tidak ada bagian lagi dari kekayaan ayah kami yang akan kami warisi.
\par 15 Kami ini dianggapnya sebagai orang asing. Dia telah menjual kami dan apa yang telah diterimanya untuk itu, telah dihabiskannya juga.
\par 16 Semua kekayaan yang telah diambil Allah dari ayah kami sebenarnya milik kami dan milik anak-anak kami. Lakukanlah apa yang diperintahkan Allah kepadamu."
\par 17 Lalu Yakub berkemas-kemas untuk pulang kepada ayahnya di negeri Kanaan. Dinaikkannya istri-istri dan anak-anaknya ke atas unta, dan digiringnya semua kawanan kambing dombanya dan segala apa yang telah diperolehnya di Mesopotamia.
\par 19 Pada waktu itu Laban ada di padang sedang menggunting bulu domba-dombanya, dan pada waktu itulah Rahel mencuri patung pelindung keluarga kepunyaan ayahnya.
\par 20 Yakub mengakali Laban dengan tidak memberitahukan keberangkatannya.
\par 21 Ia mengambil segala harta miliknya dan pergi dengan tergesa-gesa. Ia menyeberangi Sungai Efrat lalu menuju daerah pegunungan Gilead.
\par 22 Baru tiga hari kemudian Laban diberitahukan bahwa Yakub telah lari.
\par 23 Lalu Laban mengumpulkan orang-orangnya dan bersama-sama mereka mengejar Yakub selama tujuh hari sampai menyusulnya di daerah pegunungan Gilead.
\par 24 Malam hari Allah datang kepada Laban dalam mimpinya, dan berkata, "Awas, jangan sekali-kali engkau mengancam Yakub dengan sepatah kata pun."
\par 25 Pada waktu itu Yakub sudah memasang kemahnya di sebuah bukit. Kemudian di sebuah bukit yang lain di daerah itu juga Laban dan orang-orangnya memasang kemahnya.
\par 26 Setelah itu Laban berkata kepada Yakub, "Mengapa engkau mengakali saya dan membawa lari anak-anak saya, seakan-akan mereka tawanan perang?
\par 27 Mengapa engkau mengakali saya dan pergi dengan diam-diam, tanpa minta diri? Seandainya engkau minta diri, pastilah saya melepasmu pergi dengan senang diiringi nyanyian musik rebana dan kecapi!
\par 28 Engkau bahkan tidak memberi saya kesempatan untuk mencium anak cucu saya sebagai perpisahan. Bodoh benar tindakanmu!
\par 29 Saya cukup kuat untuk berbuat jahat kepadamu, tetapi tadi malam Allah yang dipuja ayahmu memperingatkan saya supaya sekali-kali jangan mengancam engkau.
\par 30 Saya tahu bahwa engkau berangkat itu karena ingin sekali pulang. Tetapi mengapa engkau mencuri patung pelindung keluarga saya?"
\par 31 Lalu Yakub menjawab, "Saya pergi dengan diam-diam karena takut Paman akan menahan anak-anak Paman.
\par 32 Mengenai patung itu, jika Paman menemukan seorang di sini yang telah mencurinya, ia akan dihukum mati. Periksalah, apakah ada barang-barang milik Paman di sini. Kalau ada, silakan ambil. Dan orang-orang kita boleh menyaksikannya." Yakub tidak tahu bahwa Rahel yang mencuri patung itu.
\par 33 Lalu Laban pergi ke kemah Yakub dan menggeledahnya; sesudah itu ia ke kemah Lea dan ke kemah kedua hamba wanita itu, tetapi patung itu tidak ditemukannya. Akhirnya ia masuk ke dalam kemah Rahel.
\par 34 Sementara itu Rahel telah mengambil patung itu dan memasukkannya ke dalam kantong pelana unta, sedangkan ia sendiri duduk di atas pelana itu. Laban menggeledah seluruh kemah Rahel, tetapi tidak berhasil menemukan patung itu.
\par 35 Lalu Rahel berkata kepada ayahnya, "Janganlah Ayah marah, saya tidak dapat berdiri menyambut Ayah, sebab saya sedang datang bulan." Laban mencari dengan teliti, tetapi tidak menemukan patung itu.
\par 36 Maka marahlah Yakub, "Kejahatan apakah yang telah saya lakukan?" tanyanya geram. "Apa salah saya sehingga Paman mengejar-ngejar saya?
\par 37 Sekarang, sesudah menggeledah segala barang saya, apakah Paman menemukan barang milik Paman? Keluarkanlah barang itu dan letakkan di sini, supaya orang-orang saya dan orang-orang Paman dapat melihatnya dan memutuskan siapa di antara kita yang benar.
\par 38 Dua puluh tahun saya sudah tinggal pada Paman. Kawanan domba dan kambing Paman telah berhasil berkembang biak, dan belum pernah saya makan seekor kambing jantan pun dari ternak Paman.
\par 39 Yang mati diterkam binatang buas tidak pernah saya bawa kepada Paman, tetapi saya sendirilah yang menggantinya. Yang dicuri pada waktu siang maupun malam selalu Paman tuntut supaya saya yang menggantinya.
\par 40 Seringkali saya kepanasan pada waktu siang dan kedinginan pada waktu malam. Dan kerap kali saya tidak dapat tidur pula.
\par 41 Begitulah keadaan saya ketika bekerja pada Paman selama dua puluh tahun penuh. Empat belas tahun lamanya saya bekerja untuk mendapat kedua anak Paman itu--dan enam tahun lagi untuk mendapat kawanan kambing domba itu. Dan selama itu pun Paman mengubah upah saya sampai sepuluh kali.
\par 42 Seandainya saya tidak dilindungi oleh Allah leluhur saya, yaitu Allah yang dipuja oleh Abraham dan Ishak, maka Paman pasti sudah menyuruh saya pergi dengan tangan kosong. Tetapi Allah telah memperhatikan kesusahan saya dan pekerjaan yang telah saya lakukan, dan tadi malam Ia telah memberikan keputusan-Nya."
\par 43 Lalu Laban menjawab Yakub, "Kedua wanita itu anak saya; anak mereka adalah milik saya, dan kawanan domba itu kepunyaan saya pula. Sesungguhnya, segala sesuatu yang kaulihat di sini sayalah yang punya. Tetapi saya tidak dapat berbuat apa-apa untuk mempertahankan anak cucu saya.
\par 44 Karena itu marilah kita membuat perjanjian dan menyusun batu-batu sebagai tanda perjanjian itu."
\par 45 Kemudian Yakub mengambil sebuah batu dan ditegakkannya menjadi batu peringatan.
\par 46 Lalu ia menyuruh orang-orangnya mengumpulkan batu dan membuat timbunan. Setelah itu mereka makan bersama di dekat timbunan batu itu.
\par 47 Laban menamakan timbunan batu itu Yegar-Sahaduta, sedangkan Yakub menamakannya Galed.
\par 48 Kata Laban kepada Yakub, "Timbunan batu ini akan menjadi peringatan bagi kita berdua." Itulah sebabnya tempat itu dinamakan Galed.
\par 49 Laban berkata juga, "Semoga TUHAN mengawasi kita selama kita hidup berjauhan." Karena itu tempat itu dinamakan juga Mizpa.
\par 50 Laban berkata lagi, "Jika engkau berbuat jahat kepada anak-anak saya, atau mengawini wanita-wanita lain, walaupun tidak ada yang mengetahuinya, ingatlah, Allah mengawasi kita.
\par 51 Inilah batu-batu yang telah saya timbun, dan ini batu peringatan.
\par 52 Baik timbunan ini, maupun batu peringatan itu mengingatkan kita bahwa saya tidak boleh melewati timbunan ini untuk menyerang engkau, dan engkau pun tidak boleh melewati timbunan ini dan batu peringatan ini untuk menyerang saya.
\par 53 Allah yang dipuja Abraham dan dipuja ayah saya Nahor akan menjadi hakim antara kita." Lalu Yakub bersumpah demi Allah yang disembah oleh Ishak, ayahnya, untuk menepati janjinya.
\par 54 Ia menyembelih seekor binatang dan mempersembahkannya di atas bukit itu. Lalu diundangnya orang-orangnya untuk makan bersama. Setelah selesai makan, mereka bermalam di bukit itu.
\par 55 Keesokan harinya pagi-pagi, Laban memberi cium perpisahan kepada anak cucunya, memberkati mereka, lalu pulang ke tempat tinggalnya.

\chapter{32}

\par 1 Yakub meneruskan perjalanannya. Lalu beberapa malaikat datang menemui dia.
\par 2 Ketika Yakub melihat mereka, ia berkata, "Ini perkemahan Allah," karena itu dinamakannya tempat itu Mahanaim.
\par 3 Setelah itu Yakub menyuruh beberapa utusan berjalan mendahului dia ke negeri Edom untuk menemui Esau, abangnya.
\par 4 Yakub berkata kepada mereka, "Katakanlah kepada abang saya Esau begini, 'Saya Yakub, hambamu. Selama ini saya telah tinggal pada Laban.
\par 5 Sekarang saya mempunyai sapi, keledai, domba, kambing dan hamba-hamba. Saya mengirimkan kabar kepada tuan saya, dengan harapan tuan saya akan senang.'"
\par 6 Ketika para utusan kembali lagi kepada Yakub, mereka berkata, "Kami telah pergi kepada Esau, abang Tuan dan sekarang ia dalam perjalanan kemari untuk menemui Tuan. Ia datang dengan empat ratus orang."
\par 7 Mendengar itu Yakub sangat ketakutan dan khawatir. Lalu orang-orang yang ada bersamanya dibaginya menjadi dua kelompok, demikian pula dombanya, kambingnya, sapi dan untanya.
\par 8 Pikirnya, "Jika Esau datang dan menyerang kelompok yang pertama, maka kelompok kedua mungkin sempat menyelamatkan diri."
\par 9 Kemudian Yakub berdoa, "Ya Allah, yang dipuja oleh kakek saya Abraham, dan ayah saya Ishak, dengarkanlah saya! TUHAN telah menyuruh saya pulang kembali ke negeri saya dan sanak saudara saya. Dan TUHAN berjanji akan mengatur segala sesuatu dengan baik bagi saya.
\par 10 Saya tidak layak menerima segala kasih dan kesetiaan yang sudah TUHAN tunjukkan kepada saya. Ketika saya berangkat menyeberangi Sungai Yordan, saya tidak membawa apa-apa selain tongkat. Sekarang saya datang kembali dengan dua kelompok ini.
\par 11 Saya mohon dengan sangat, selamatkanlah saya dari abang saya Esau. Saya takut jangan-jangan ia datang menyerang dan membinasakan kami semua, juga wanita dan anak-anak.
\par 12 Ingatlah bahwa TUHAN telah berjanji akan mengatur segala sesuatu dengan baik bagi saya dan memberi kepada saya banyak keturunan. Tidak seorang pun dapat menghitungnya karena banyaknya seperti pasir di pantai laut."
\par 13 Setelah bermalam di situ, Yakub mengambil sebagian dari binatang-binatang untuk dihadiahkan kepada Esau. Hadiah itu berupa: dua ratus kambing betina, dua puluh kambing jantan, dua ratus domba betina, dua puluh domba jantan, tiga puluh unta yang sedang menyusui, bersama anaknya, empat puluh sapi betina, sepuluh sapi jantan, dua puluh keledai betina dan sepuluh keledai jantan.
\par 16 Dibagi-baginya binatang-binatang itu menjadi beberapa kelompok lalu masing-masing diserahkannya kepada seorang hambanya. Katanya kepada mereka, "Berjalanlah mendahului aku, dan jagalah supaya ada jarak antara kelompok yang satu dengan kelompok yang berikutnya."
\par 17 Yakub memerintahkan kepada hamba yang pertama, "Jika abangku Esau bertemu denganmu, dan bertanya, 'Siapa tuanmu? Mau ke mana? Siapakah pemilik binatang-binatang yang kamu giring itu?'
\par 18 Maka jawablah, 'Milik hambamu Yakub. Ia mengirim ini sebagai persembahan kepada Esau, tuannya. Hambamu Yakub sendiri sedang menyusul kami.'"
\par 19 Kepada hambanya yang kedua, ketiga dan semua orang yang ditugaskan mengurus kawanan-kawanan itu, diberinya perintah yang sama. Sebab Yakub berpikir ia dapat membujuk Esau dengan persembahan itu, sehingga kalau bertemu nanti, Esau mau memaafkan dia.
\par 21 Jadi ia mengirimkan lebih dahulu persembahan itu, sedangkan ia sendiri bermalam di perkemahan.
\par 22 Pada malam itu juga Yakub bangun lalu membawa kedua istrinya, kedua selirnya dan kesebelas anaknya, menyeberang Sungai Yabok.
\par 23 Setelah menyeberangkan mereka, ia kembali dan mengirim segala miliknya ke seberang.
\par 24 Tetapi ia tinggal seorang diri. Maka datanglah seorang laki-laki bergumul dengan Yakub sampai menjelang pagi.
\par 25 Ketika orang itu merasa bahwa ia tidak akan menang dalam pergumulan itu, dipukulnya Yakub pada pinggulnya, sampai sendi pinggul itu terkilir.
\par 26 Lalu kata orang itu, "Lepaskan aku; sebentar lagi matahari terbit." Jawab Yakub, "Saya tidak akan melepaskan Tuan, kecuali jika Tuan memberkati saya."
\par 27 "Siapa namamu?" tanya orang itu. "Yakub," jawabnya.
\par 28 Orang itu berkata, "Namamu bukan Yakub lagi. Engkau telah bergumul dengan Allah dan dengan manusia, dan engkau menang; karena itu namamu menjadi Israel."
\par 29 Yakub berkata, "Katakanlah nama Tuan." Tetapi orang itu menjawab, "Tidak perlu engkau bertanya siapa namaku!" Lalu diberkatinya Yakub.
\par 30 Kemudian Yakub berkata, "Saya telah bertemu muka dengan Allah, dan saya masih hidup." Karena itu dinamakannya tempat itu Pniel.
\par 31 Pada waktu matahari terbit, Yakub meninggalkan Pniel, dan ia pincang karena sendi pinggulnya terkilir.
\par 32 Sampai sekarang pun keturunan Israel tidak makan daging yang menutupi sendi pinggul binatang, karena Yakub telah kena pukulan pada pinggulnya.

\chapter{33}

\par 1 Yakub melihat Esau datang dengan empat ratus orangnya. Karena itu dibaginya anak-anaknya di antara Lea, Rahel dan kedua selirnya.
\par 2 Ia menempatkan kedua selirnya bersama anak-anak mereka di depan, kemudian Lea bersama anak-anaknya, lalu Rahel dan Yusuf di belakang sekali.
\par 3 Yakub berjalan di depan mereka semua, dan sambil mendekati abangnya, ia sujud sampai tujuh kali.
\par 4 Tetapi Esau berlari mendapatkan Yakub, lalu memeluknya dan menciumnya. Dan kedua orang itu bertangis-tangisan.
\par 5 Ketika Esau melihat semua wanita dan anak-anak itu, ia bertanya, "Siapa mereka yang ada bersamamu ini?" "Mereka anak-anak saya," jawab Yakub, "Allah telah mengaruniakan anak-anak itu kepada saya."
\par 6 Setelah itu para selir maju dengan anak-anak mereka, lalu sujud.
\par 7 Kemudian datanglah Lea beserta anak-anaknya, dan yang terakhir Yusuf dan Rahel mendekat lalu mereka semua sujud.
\par 8 Esau bertanya, "Apa maksudmu dengan rombongan-rombongan yang saya jumpai tadi?" Yakub menjawab, "Untuk menyenangkan hati Abang."
\par 9 Tetapi Esau berkata, "Sudah cukup harta saya, tak perlu kau memberi hadiah kepada saya."
\par 10 Kata Yakub, "Jangan! Jika saya telah menyenangkan hati Abang, terimalah persembahan saya ini. Bagiku, melihat wajah Abang sama dengan melihat wajah Allah, karena Abang begitu ramah kepada saya.
\par 11 Saya mohon, terimalah persembahan yang telah saya bawa ini. Allah telah baik hati kepada saya dan memberikan segala sesuatu yang saya perlukan." Yakub terus mendesak sampai akhirnya Esau menerima pemberiannya itu.
\par 12 Lalu Esau berkata, "Marilah kita bersiap-siap untuk berangkat, saya akan berjalan mendahuluimu."
\par 13 Yakub menjawab, "Abang tahu bahwa anak-anak ini lemah, dan saya harus hati-hati dengan semua ternak yang sedang menyusui. Jika digiring cepat-cepat, walaupun hanya satu hari saja, maka seluruh kawanan binatang itu akan mati.
\par 14 Berjalanlah lebih dahulu, saya ikut dengan perlahan-lahan. Saya akan berjalan secepat mungkin dengan ternak dan anak-anakku ini sampai dapat menyusul Abang di Edom."
\par 15 Kata Esau, "Kalau begitu, baiklah saya tinggalkan padamu beberapa anak buah saya." Tetapi Yakub menjawab, "Tak usahlah bersusah-susah, saya hanya ingin menyenangkan hati Abang."
\par 16 Demikianlah pada hari itu Esau berangkat kembali ke Edom.
\par 17 Tetapi Yakub pergi ke Sukot, lalu mendirikan rumah bagi dirinya dan tempat berteduh bagi ternaknya. Itulah sebabnya tempat itu dinamakan Sukot.
\par 18 Dalam perjalanan dari Mesopotamia, sampailah Yakub dengan selamat di kota Sikhem di negeri Kanaan, lalu memasang kemahnya di sebuah padang dekat kota.
\par 19 Padang tempat perkemahannya itu dibelinya dari orang keturunan Hemor, ayah Sikhem, dengan harga seratus uang perak.
\par 20 Kemudian ia mendirikan mezbah di situ dan menamakannya "El, Allah Israel".

\chapter{34}

\par 1 Pada suatu hari, Dina anak perempuan Yakub dan Lea, mengunjungi beberapa wanita Kanaan.
\par 2 Ketika Sikhem, anak Hemor orang Hewi yang menjadi raja di wilayah itu melihat Dina, dilarikannya gadis itu, lalu diperkosanya.
\par 3 Tetapi ia sangat tertarik kepada Dina, sehingga jatuh cinta kepadanya dan berusaha supaya gadis itu mencintainya pula.
\par 4 Kata Sikhem kepada ayahnya, "Saya mohon Ayah berusaha mendapat gadis itu bagi saya. Saya ingin mengawininya."
\par 5 Yakub mendengar bahwa anaknya telah dinodai kehormatannya, tetapi karena anak-anaknya sedang menjaga ternaknya di padang, ia tidak dapat mengambil tindakan apa pun sebelum mereka pulang.
\par 6 Hemor, ayah Sikhem, datang kepada Yakub hendak berunding dengan dia.
\par 7 Tepat pada waktu itu anak-anak Yakub pulang dari padang. Mendengar peristiwa itu, mereka terkejut dan sangat marah. Sebab dengan memperkosa anak Yakub itu, Sikhem telah menghina semua orang Israel.
\par 8 Kata Hemor kepada Yakub, "Anak saya Sikhem telah jatuh cinta kepada anak Saudara; saya mohon supaya Saudara mengizinkan anak saya kawin dengan dia.
\par 9 Marilah kita membuat persetujuan bahwa bangsa Saudara dan bangsa kami kawin campur.
\par 10 Maka Saudara-saudara boleh tinggal bersama kami di negeri ini; di mana Saudara suka. Saudara-saudara boleh bebas berdagang dan memiliki harta benda."
\par 11 Kemudian Sikhem berkata kepada ayah Dina serta abang-abangnya, "Penuhilah permintaan saya ini, maka saya akan memberikan apa saja yang kalian mau.
\par 12 Katakanlah hadiah apa yang kalian inginkan, dan tentukanlah emas kawinnya. Saya akan memberikan apa yang kalian minta, asalkan kalian mengizinkan saya mengawini gadis itu."
\par 13 Karena Sikhem telah menodai kehormatan adik mereka Dina, anak-anak Yakub berbohong kepada Sikhem dan Hemor ayahnya.
\par 14 Kata mereka kepada kedua orang itu, "Kami malu mengizinkan adik kami kawin dengan orang yang tidak bersunat.
\par 15 Kami hanya dapat menyetujui permintaanmu itu dengan satu syarat, yaitu: Kalian harus menjadi seperti kami, artinya semua laki-laki di antara kalian harus disunat.
\par 16 Setelah itu kami akan menyetujui usulmu tentang kawin campur itu. Kami akan menetap di tengah-tengahmu dan menjadi satu bangsa denganmu.
\par 17 Tetapi jika kamu tidak mau menerima syarat-syarat kami, dan tidak mau disunat, kami akan mengambil kembali gadis itu dan pergi."
\par 18 Hemor dan Sikhem merasa syarat-syarat itu pantas,
\par 19 dan Sikhem bersedia segera memenuhi syarat itu karena ia mencintai Dina. Di antara seluruh kaum keluarganya, Sikhemlah yang paling berpengaruh.
\par 20 Kemudian Hemor dan Sikhem pergi ke tempat pertemuan di pintu gerbang kota, dan berbicara kepada warga kota mereka, begini,
\par 21 "Orang-orang Israel itu sahabat kita. Jadi biarkanlah mereka tinggal di negeri ini bersama kita serta bebas pergi ke mana mereka suka. Negeri ini cukup luas bagi mereka juga. Kita bisa kawin dengan gadis-gadis mereka, dan mereka pun bisa kawin dengan gadis-gadis kita.
\par 22 Tetapi orang-orang itu hanya mau tinggal bersama kita dan menjadi satu bangsa dengan kita dengan satu syarat, yaitu: semua laki-laki di antara kita harus disunat seperti mereka.
\par 23 Nanti semua kawanan binatang mereka dan segala harta mereka menjadi milik kita. Jadi marilah kita menyetujui permintaan mereka supaya mereka tinggal bersama dengan kita."
\par 24 Semua warga kota yang berkumpul itu menyetujui usul Hemor dan Sikhem, lalu semua orang laki-laki dewasa di kota itu disunat.
\par 25 Tiga hari kemudian, ketika orang-orang lelaki itu masih kesakitan karena disunat, dua anak Yakub, yaitu abang Dina yang bernama Simeon dan Lewi, mengambil pedang mereka, lalu diam-diam masuk ke dalam kota. Kemudian mereka membunuh semua orang laki-laki di situ,
\par 26 termasuk juga Hemor dan Sikhem. Lalu mereka mengambil Dina dari rumah Sikhem dan pergi.
\par 27 Setelah pembantaian itu, anak-anak Yakub yang lain merampok kota itu sebagai balas dendam karena adik mereka telah dinodai kehormatannya.
\par 28 Mereka mengambil kawanan kambing domba, sapi, keledai, dan segala yang ada di dalam kota dan di padang.
\par 29 Mereka mengambil semua barang yang berharga, menawan semua wanita dan anak-anak, dan merampas segala isi rumah-rumah di kota itu.
\par 30 Yakub berkata kepada Simeon dan Lewi, "Kalian menyusahkan saya. Sekarang orang Kanaan dan orang Feris dan semua penduduk di negeri ini akan membenci saya. Orang-orang kita tidak banyak; jika penduduk itu semua bersekutu melawan dan menyerang saya, seluruh keluarga kita akan dibinasakan."
\par 31 Tetapi mereka menjawab, "Kami tidak dapat membiarkan adik kami diperlakukan sebagai pelacur."

\chapter{35}

\par 1 Pada suatu hari Allah berkata kepada Yakub, "Akulah Allah yang menampakkan diri kepadamu ketika engkau lari dari abangmu Esau. Pergilah segera ke Betel. Tinggallah di situ dan dirikanlah sebuah mezbah bagi-Ku."
\par 2 Lalu Yakub berkata kepada keluarganya dan kepada semua orang yang ada bersama-sama dengan dia, "Kita akan berangkat dari sini dan pergi ke Betel. Di situ saya akan mendirikan mezbah bagi Allah yang telah menolong saya pada masa kesukaran saya dan yang telah melindungi saya ke mana saja saya pergi. Jadi buanglah patung dewa-dewa asing yang ada padamu; lakukanlah upacara pembersihan diri dan kenakanlah pakaian yang bersih."
\par 4 Lalu mereka menyerahkan kepada Yakub semua patung dewa asing yang ada pada mereka dan juga anting-anting mereka. Semua benda itu ditanam Yakub di bawah pohon besar di dekat kota Sikhem.
\par 5 Pada waktu Yakub dan rombongan keluarganya berangkat, Allah meliputi penduduk kota-kota sekitar dengan rasa takut, sehingga mereka tidak mengejar Yakub dan rombongannya.
\par 6 Lalu sampailah mereka di tanah Kanaan di kota Betel yang dahulu bernama Lus.
\par 7 Di situ Yakub mendirikan mezbah dan menamakan tempat itu "El-Betel", karena Allah telah menampakkan diri kepadanya di situ ketika ia lari dari abangnya.
\par 8 Di sana juga Debora, inang pengasuh Ribka, meninggal lalu dikuburkan di bawah pohon besar di sebelah selatan kota itu. Itulah sebabnya pohon itu dinamakan "Pohon Tangis".
\par 9 Setelah Yakub kembali dari Mesopotamia, Allah menampakkan diri lagi kepadanya dan memberkatinya.
\par 10 Kata Allah kepadanya, "Engkau tidak akan disebut lagi Yakub, tetapi Israel." Demikianlah Allah menamakan dia Israel.
\par 11 Lalu kata Allah pula, "Akulah Allah Yang Mahakuasa. Hendaklah engkau beranak cucu yang banyak! Engkau akan menurunkan bangsa-bangsa dan menjadi bapak leluhur raja-raja.
\par 12 Negeri yang telah Kuberikan kepada Abraham dan Ishak, akan Kuberikan kepadamu dan keturunanmu."
\par 13 Setelah itu Allah meninggalkan Yakub.
\par 14 Di tempat Allah berbicara kepada Yakub, ia menegakkan sebuah batu peringatan. Lalu batu itu dipersembahkannya kepada Allah dengan menuangkan anggur dan minyak zaitun di atasnya.
\par 15 Tempat itu dinamakannya "Betel".
\par 16 Sesudah itu Yakub dan rombongannya meninggalkan Betel. Ketika mereka masih agak jauh dari Efrata, tibalah saatnya bagi Rahel untuk melahirkan. Tetapi anaknya sukar lahir.
\par 17 Ketika sakit bersalinnya memuncak, berkatalah bidan kepadanya, "Besarkan hatimu, Rahel, anakmu laki-laki lagi."
\par 18 Tetapi Rahel sudah mendekati ajalnya, dan ketika ia hendak menghembuskan napasnya yang penghabisan, dinamakannya anak itu Ben-Oni tetapi ayahnya menamakannya Benyamin.
\par 19 Setelah Rahel meninggal, ia dikuburkan di sisi jalan yang menuju ke Efrata, yang sekarang bernama Betlehem.
\par 20 Yakub menegakkan sebuah batu peringatan di situ, dan sampai hari ini pun kuburan Rahel masih ditandai oleh batu itu.
\par 21 Sesudah itu Yakub meneruskan perjalanannya, lalu memasang kemahnya di suatu tempat lewat menara Eder.
\par 22 Sementara mereka ada di tempat itu Ruben tidur dengan Bilha, selir ayahnya. Hal itu didengar Yakub, dan ia menjadi sangat marah. Yakub mempunyai dua belas anak laki-laki.
\par 23 Anak-anak Lea ialah Ruben (anak sulung Yakub), Simeon, Lewi, Yehuda, Isakhar dan Zebulon.
\par 24 Anak-anak Rahel ialah Yusuf dan Benyamin.
\par 25 Anak-anak Bilha, hamba Rahel ialah Dan serta Naftali.
\par 26 Anak-anak Zilpa, hamba Lea ialah Gad dan Asyer. Anak-anak itu lahir di Mesopotamia.
\par 27 Lalu sampailah Yakub kepada Ishak ayahnya, di tempat tinggalnya di Mamre dekat Hebron. (Dulu Abraham kakeknya, juga tinggal di sana).
\par 28 Pada usia amat tua, yaitu seratus delapan puluh tahun, Ishak meninggal lalu dikuburkan oleh Esau dan Yakub, anak-anaknya.

\chapter{36}

\par 1 Inilah keturunan Esau yang disebut juga Edom.
\par 2 Esau menikah dengan tiga wanita Kanaan, yaitu: Ada, anak Elon orang Het; Oholibama, anak Ana dan cucu Zibeon orang Hewi;
\par 3 Basmat, anak Ismael dan adik Nebayot.
\par 4 Ada melahirkan Elifas, Basmat melahirkan Rehuel,
\par 5 dan Oholibama melahirkan Yeus, Yaelam dan Korah. Semua anak itu lahir di negeri Kanaan.
\par 6 Kemudian Esau meninggalkan Yakub dan pergi ke negeri lain membawa semua istrinya, anaknya dan semua orang yang ada di rumahnya, bersama dengan segala ternak dan harta benda yang telah diperolehnya di Kanaan.
\par 7 Esau berpisah dari Yakub karena harta mereka terlalu banyak sehingga mereka tak dapat hidup bersama. Lagipula di negeri yang mereka diami itu tidak ada cukup makanan untuk ternak mereka yang sangat banyak itu.
\par 8 Maka Esau yang juga dinamakan Edom, menetap di daerah pegunungan Seir.
\par 9 Inilah keturunan Esau, leluhur orang Edom.
\par 10 Istri Esau yang bernama Ada melahirkan seorang anak laki-laki, namanya Elifas. Dan Elifas mempunyai lima anak laki-laki, yaitu: Teman, Omar, Zefo, Gaetam dan Kenas. Elifas mempunyai selir, namanya Timna. Dia melahirkan anak laki-laki yang bernama Amalek. Istri Esau yang bernama Basmat melahirkan seorang anak laki-laki yang bernama Rehuel. Dan Rehuel mempunyai empat anak laki-laki, yaitu: Nahat, Zerah, Syama dan Miza.
\par 14 Istri Esau yang bernama Oholibama, yaitu anak Ana dan cucu Zibeon, melahirkan tiga anak laki-laki, yaitu: Yeus, Yaelam dan Korah.
\par 15 Inilah kepala suku-suku keturunan Esau: Elifas, anak sulung Esau, adalah leluhur suku-suku yang berikut ini: Teman, Omar, Zefo, Kenas,
\par 16 Korah, Gaetam dan Amalek. Mereka semua keturunan Ada, istri Esau.
\par 17 Rehuel, anak Esau, adalah leluhur suku-suku yang berikut ini: Nahat, Zerah, Syama dan Miza. Mereka semua keturunan Basmat, istri Esau.
\par 18 Suku-suku yang berikut ini adalah keturunan Esau dari istrinya yang bernama Oholibama anak Ana, yaitu: Yeus, Yaelam dan Korah.
\par 19 Semua suku itu keturunan Esau.
\par 20 Penduduk asli tanah Edom dibagi atas suku-suku keturunan Seir, orang Hori. Suku-suku itu ialah: Lotan, Syobal, Zibeon, Ana, Disyon, Ezer dan Disyan.
\par 22 Lotan adalah leluhur marga Hori dan Heman. (Lotan mempunyai saudara perempuan, yaitu Timna).
\par 23 Syobal adalah leluhur marga: Alwan, Manahat, Ebal, Syefo dan Onam.
\par 24 Zibeon mempunyai dua anak laki-laki, yaitu Aya dan Ana. (Ana inilah yang menemukan sumber-sumber air panas di padang gurun, ketika ia sedang menggembalakan keledai-keledai ayahnya).
\par 25 Ana ialah ayah Disyon, dan Disyon leluhur marga: Hemdan, Esyban, Yitran dan Keran. Ana juga mempunyai seorang anak perempuan, namanya Oholibama.
\par 27 Ezer leluhur marga: Bilhan, Zaawan, dan Akan.
\par 28 Disyan leluhur marga Us dan Aran.
\par 29 Inilah suku-suku Hori di negeri Edom: Lotan, Syobal, Zibeon, Ana, Disyon, Ezer dan Disyan.
\par 31 Sebelum ada raja yang memerintah di Israel, raja-raja yang berikut ini memerintah di tanah Edom secara berturut-turut: Bela anak Beor dari Dinhaba. Yobab anak Zera dari Bozra. Husyam dari daerah Teman. Hadad anak Bedad dari Awit (dialah yang mengalahkan orang Midian dalam peperangan di daerah Moab). Samla dari Masreka. Saul dari Rehobot di pinggir sungai. Baal-Hanan anak Akhbor. Hadar dari Pahu (istrinya bernama Mehetabeel, anak Matred dan cucu Mezahab).
\par 40 Esau adalah leluhur suku-suku Edom yang berikut ini: Timna, Alwa, Yetet, Oholibama, Ela, Pinon, Kenas, Teman, Mibzar, Magdiel, dan Iram. Setiap suku memberikan namanya kepada daerah tempat tinggalnya.

\chapter{37}

\par 1 Yakub menetap di negeri Kanaan, tempat tinggal ayahnya,
\par 2 dan inilah riwayat keluarga Yakub: Pada waktu Yusuf, anak Yakub berumur tujuh belas tahun, ia mengurus kawanan kambing domba bersama-sama dengan abang-abangnya, yaitu anak-anak Bilha dan Zilpa, kedua selir ayahnya. Ia melaporkan kepada ayahnya perbuatan-perbuatan jahat yang dilakukan oleh abang-abangnya.
\par 3 Yakub lebih sayang kepada Yusuf dari semua anaknya yang lain, karena Yusuf dilahirkan ketika ayahnya sudah tua. Pada suatu hari dibuatnya untuk Yusuf sebuah jubah yang sangat bagus.
\par 4 Setelah abang-abang Yusuf melihat bahwa ayah mereka lebih sayang kepada Yusuf daripada kepada mereka, bencilah mereka kepada Yusuf, sehingga tidak mau lagi bicara baik-baik dengan dia.
\par 5 Pada suatu malam Yusuf bermimpi, dan ketika ia menceritakan mimpinya itu kepada abang-abangnya, mereka bertambah benci kepadanya.
\par 6 Inilah yang dikatakan Yusuf kepada mereka, "Coba dengar!
\par 7 Saya bermimpi kita semua sedang di ladang mengikat gandum, lalu gandum saya berdiri tegak. Gabung-gabung kalian mengelilingi gabung saya lalu sujud kepadanya."
\par 8 "Kaukira engkau akan menjadi raja dan berkuasa atas kami?" tegur abang-abangnya. Lalu makin bencilah mereka kepadanya karena mimpi-mimpinya dan karena apa yang dikatakannya.
\par 9 Kemudian Yusuf bermimpi lagi, dan ia mengatakan kepada abang-abangnya, "Saya bermimpi lagi, saya lihat matahari, bulan dan sebelas bintang sujud kepada saya."
\par 10 Mimpi itu diceritakannya pula kepada ayahnya, dan ayahnya menegur dia, katanya, "Mimpi apa itu? Kaupikir saya, ibumu dan saudara-saudaramu akan datang dan sujud menyembah kepadamu?"
\par 11 Abang-abang Yusuf iri hati kepadanya, tetapi ayahnya tetap memikirkan mimpi itu.
\par 12 Pada suatu hari ketika abang-abang Yusuf pergi ke Sikhem untuk menggembalakan kawanan kambing domba ayah mereka,
\par 13 berkatalah Yakub kepada Yusuf, "Pergilah ke Sikhem, ke tempat abang-abangmu menggembalakan kawanan kambing domba kita." Jawab Yusuf, "Baik, Ayah."
\par 14 Kata ayahnya, "Lihatlah bagaimana keadaan abang-abangmu dan kawanan kambing domba, lalu kembalilah untuk melapor kepada saya." Maka dilepaskannya Yusuf pergi meninggalkan Lembah Hebron. Tak lama kemudian sampailah Yusuf di Sikhem.
\par 15 Sementara ia berjalan ke sana kemari di padang, bertemulah ia dengan seorang laki-laki yang bertanya kepadanya, "Apa yang kaucari?"
\par 16 "Abang-abang saya. Mereka sedang menggembalakan kawanan kambing domba," jawab Yusuf, "Tahukah Bapak di mana mereka berada?"
\par 17 Kata orang itu, "Mereka telah berangkat dari sini. Saya dengar mereka berkata bahwa mereka hendak pergi ke Dotan." Lalu berangkatlah Yusuf menyusul abang-abangnya dan ditemukannya mereka di Dotan.
\par 18 Dari jauh mereka telah melihat Yusuf, dan sebelum ia sampai kepada mereka, mereka sepakat untuk membunuh dia.
\par 19 Kata mereka seorang kepada yang lain, "Lihat, si tukang mimpi itu datang.
\par 20 Ayo kita bunuh dia dan lemparkan mayatnya ke dalam sumur yang kering. Kita katakan nanti bahwa dia diterkam binatang buas. Kita lihat nanti bagaimana jadinya mimpi-mimpinya itu."
\par 21 Ruben mendengar rencana mereka itu, lalu ia berusaha untuk menyelamatkan Yusuf. "Jangan bunuh dia," katanya.
\par 22 "Kita lemparkan saja ke dalam sumur di padang gurun ini, tetapi jangan kita pukul atau lukai dia." Hal itu dikatakannya karena ia bermaksud menyelamatkan Yusuf dan menyuruh dia pulang ke rumah.
\par 23 Ketika Yusuf sampai kepada abang-abangnya, dengan kasar mereka menanggalkan jubah Yusuf yang sangat bagus itu.
\par 24 Lalu mereka menyeret dia dan melemparkannya ke dalam sumur yang kering.
\par 25 Ketika mereka sedang makan, tiba-tiba terlihat oleh mereka suatu kafilah orang Ismael yang sedang dalam perjalanan dari Gilead ke Mesir. Unta-unta mereka bermuatan rempah-rempah dan kismis.
\par 26 Lalu kata Yehuda kepada saudara-saudaranya, "Apa gunanya membunuh adik kita dan merahasiakan pembunuhan itu?
\par 27 Mari kita jual dia kepada orang Ismael itu. Tak usah kita sakiti dia. Bagaimanapun juga dia adalah adik kita sendiri." Saudara-saudaranya setuju,
\par 28 dan ketika beberapa pedagang Midian lewat, Yusuf dikeluarkan oleh abang-abangnya dari dalam sumur itu lalu dijual kepada orang Ismael itu dengan harga dua puluh keping perak. Kemudian ia dibawa oleh pedagang-pedagang itu ke Mesir.
\par 29 Ketika Ruben kembali ke sumur itu dan tidak menemui Yusuf di situ, dikoyakkannya pakaiannya karena sedih.
\par 30 Ia kembali kepada saudara-saudaranya dan berkata, "Anak itu tidak ada lagi di situ! Apa yang harus saya lakukan sekarang?"
\par 31 Lalu mereka menyembelih seekor kambing dan mencelupkan jubah Yusuf ke dalam darah kambing itu.
\par 32 Kemudian jubah itu mereka suruh antarkan kepada ayah mereka dengan pesan, "Jubah ini kami temukan. Milik anak Ayahkah ini?"
\par 33 Yakub mengenali jubah itu, lalu berkata, "Betul, ini jubah anakku! Pasti dia sudah diterkam binatang buas. Aduh, anak saya Yusuf sudah mati dikoyak-koyak binatang itu!"
\par 34 Yakub merobek pakaiannya karena sedih dan memakai pakaian kabung. Berhari-hari lamanya ia meratapi anaknya itu.
\par 35 Semua anak-anaknya, baik laki-laki maupun perempuan datang menghiburnya, tetapi ia tak mau dihibur. Katanya, "Sampai mati saya akan terus meratapi anak saya." Demikianlah ia terus berkabung karena Yusuf anaknya.
\par 36 Sementara itu, di Mesir, orang-orang Midian telah menjual Yusuf kepada Potifar, seorang perwira raja yang menjabat kepala pengawal istana.

\chapter{38}

\par 1 Kira-kira pada waktu itu Yehuda meninggalkan saudara-saudaranya dan tinggal bersama Hira, seorang laki-laki yang berasal dari kota Adulam.
\par 2 Di situ Yehuda berkenalan dengan seorang gadis Kanaan, anak Sua, lalu mereka kawin.
\par 3 Anak mereka yang pertama laki-laki, diberi nama Er oleh ayahnya.
\par 4 Anak kedua, juga laki-laki, dinamakannya Onan.
\par 5 Anak ketiga pun laki-laki, dan dinamakannya Syela. Waktu Syela lahir Yehuda sedang ada di Kezib.
\par 6 Yehuda mengawinkan Er, anaknya yang sulung dengan Tamar.
\par 7 Kelakuan Er jahat sekali, sehingga TUHAN marah kepadanya dan membunuhnya.
\par 8 Lalu Yehuda berkata kepada Onan adik Er, "Pergilah kepada janda abangmu, dan tidurlah dengan dia. Penuhilah kewajibanmu terhadap dia, sebab engkau adik suaminya; dengan demikian abangmu bisa mendapat keturunan."
\par 9 Tetapi Onan tahu bahwa anak-anaknya nanti tidak akan menjadi miliknya. Jadi, setiap kali ia bersetubuh dengan janda abangnya itu, dibiarkannya maninya tumpah di luar supaya abangnya tidak akan mendapat keturunan.
\par 10 Perbuatannya itu membuat TUHAN marah, dan TUHAN membunuh dia juga.
\par 11 Kemudian berkatalah Yehuda kepada Tamar menantunya itu, "Kembalilah ke rumah orang tuamu dan tinggallah di situ sebagai janda sampai anakku Syela menjadi besar." Ia berkata demikian karena takut jangan-jangan Syela akan dibunuh TUHAN juga, seperti kedua abangnya. Maka pulanglah Tamar ke rumah orang tuanya.
\par 12 Beberapa waktu kemudian istri Yehuda meninggal. Setelah habis masa berkabung, Yehuda mengajak Hira, temannya dari Adulam itu, pergi ke Timna, tempat domba-dombanya digunting bulunya.
\par 13 Tamar mendapat kabar bahwa Yehuda mertuanya akan datang ke Timna untuk menggunting bulu domba-dombanya.
\par 14 Maka ia mengganti pakaian jandanya dengan pakaian lain. Mukanya ditutupnya dengan selubung, lalu duduklah ia di pintu gerbang kota Enaim, yang terletak di jalan menuju ke Timna. Tamar tahu betul bahwa Syela, anak Yehuda yang bungsu, sudah besar, tetapi ia belum juga dikawinkan dengan pemuda itu.
\par 15 Ketika Yehuda melihat Tamar, disangkanya wanita itu seorang pelacur, karena wajahnya terselubung.
\par 16 Lalu Yehuda mendekatinya di pinggir jalan itu, dan berkata, "Berapa yang kauminta?" Ia tidak tahu bahwa wanita itu menantunya sendiri. Tamar menjawab, "Terserah pada Tuan."
\par 17 Yehuda berkata lagi, "Saya akan memberikan kepadamu seekor kambing muda." Jawab Tamar, "Boleh, asal ada jaminan, sampai Tuan mengirimkan kambing itu."
\par 18 "Jaminan apa?" tanya Yehuda. Jawab Tamar, "Berilah kepada saya stempel Tuan dengan talinya dan juga tongkat yang ada pada Tuan itu." Yehuda memberikan benda-benda itu kepadanya, lalu mereka bersetubuh, dan Tamar menjadi hamil.
\par 19 Tamar pulang ke rumahnya dan membuka kain penutup mukanya, lalu mengenakan pakaian jandanya lagi.
\par 20 Beberapa waktu kemudian Yehuda mengutus Hira, temannya, untuk mengantarkan kambing itu dan meminta kembali benda-benda yang telah diberikannya sebagai jaminan, tetapi Hira tidak dapat menemukan wanita itu.
\par 21 Hira bertanya-tanya kepada orang-orang di Enaim, "Di mana pelacur yang biasanya menunggu di pinggir jalan ini?" "Tidak pernah ada pelacur di sini," jawab mereka.
\par 22 Lalu kembalilah Hira kepada Yehuda dan berkata, "Saya tidak dapat menemukan wanita itu. Menurut orang-orang di sana tak pernah ada pelacur di situ."
\par 23 Kata Yehuda, "Biarlah benda-benda itu untuk dia, asal saja kita tidak mendapat malu. Saya sudah berusaha untuk membayarnya, tetapi engkau tak dapat menemukan dia."
\par 24 Kira-kira tiga bulan kemudian Yehuda mendapat kabar bahwa Tamar menantunya telah bertindak sebagai pelacur dan sudah hamil. Lalu Yehuda memerintahkan, "Ambillah dia dan bakarlah sampai mati!"
\par 25 Sementara Tamar dibawa keluar, ia mengirimkan pesan kepada ayah mertuanya, katanya, "Aku telah dihamili oleh orang yang memiliki benda-benda ini. Periksalah siapa pemilik stempel dengan talinya dan tongkat ini."
\par 26 Yehuda mengenali benda-benda itu dan berkata, "Wanita itu tidak bersalah. Saya tidak memenuhi kewajiban saya terhadap dia; seharusnya saya kawinkan dia dengan anak saya Syela." Yehuda tidak pernah lagi bersetubuh dengan Tamar.
\par 27 Ketika sudah waktunya Tamar bersalin, ternyata ia akan melahirkan anak kembar.
\par 28 Sedang ia bersalin, salah satu bayi kembar itu mengeluarkan tangannya, lalu bidan memegang tangan itu dan mengikatnya dengan benang merah. Katanya, "Anak ini lahir lebih dahulu."
\par 29 Tetapi bayi itu menarik tangannya kembali, dan bayi yang satu lagi lahir lebih dahulu. Kemudian bidan berkata, "Jadi begitulah caramu mendesak keluar!" Maka anak yang lahir lebih dahulu itu dinamakannya Peres.
\par 30 Setelah itu barulah lahir adiknya yang tangannya ada benang merahnya. Dia diberi nama Zerah.

\chapter{39}

\par 1 Yusuf telah dibawa ke Mesir oleh orang-orang Ismael itu, dan dijual kepada Potifar, seorang perwira raja yang menjabat kepala pengawal istana.
\par 2 TUHAN menolong Yusuf sehingga ia selalu berhasil dalam semua pekerjaannya. Ia tinggal di rumah tuannya, orang Mesir itu.
\par 3 Tuannya melihat bahwa TUHAN menolong Yusuf dan karena itu Yusuf berhasil baik dalam segala yang dikerjakannya.
\par 4 Potifar senang kepada Yusuf dan mengangkatnya menjadi pelayan pribadinya; lalu ditugaskannya Yusuf mengurus rumah tangganya dan segala miliknya.
\par 5 Sejak saat itu, demi Yusuf, TUHAN memberkati rumah tangga orang Mesir itu dan segala apa yang dimilikinya, baik yang di rumah maupun yang di ladang.
\par 6 Segala sesuatu yang dimiliki Potifar dipercayakannya kepada Yusuf. Dengan demikian Potifar sama sekali tidak mau tahu tentang urusan rumahnya, kecuali hal makanannya. Yusuf gagah dan tampan.
\par 7 Selang beberapa waktu, istri Potifar mulai berahi kepada Yusuf, lalu pemuda itu diajaknya tidur bersama.
\par 8 Yusuf tidak mau dan berkata kepadanya, "Maaf, Nyonya, tuan Potifar telah mempercayakan segala miliknya kepada saya. Ia tidak perlu memikirkan apa-apa lagi di rumah ini.
\par 9 Di sini kuasa saya sama besar dengan kuasanya. Tidak ada satu pun yang tidak dipercayakannya kepada saya kecuali Nyonya. Bagaimana mungkin saya melakukan perbuatan sejahat itu dan berdosa terhadap Allah?"
\par 10 Meskipun istri Potifar membujuk Yusuf setiap hari, pemuda itu tetap tidak mau tidur bersamanya.
\par 11 Pada suatu hari ketika Yusuf masuk ke dalam rumah untuk melakukan pekerjaannya, tidak ada seorang pun di situ.
\par 12 Istri Potifar menarik Yusuf pada jubahnya dan berkata, "Mari kita tidur bersama." Yusuf meronta dan dapat lepas, lalu lari ke luar, tetapi jubahnya tertinggal di tangan wanita itu.
\par 13 Ketika istri Potifar melihat bahwa Yusuf telah lari dan jubahnya tertinggal,
\par 14 ia memanggil pelayan-pelayannya dan berkata, "Coba lihat! Orang Ibrani yang dibawa suami saya ke rumah ini, menghina kita. Dia masuk ke dalam kamar saya dan mau memperkosa saya, tetapi saya berteriak keras-keras.
\par 15 Waktu mendengar teriakan saya, ia lari ke luar dan jubahnya tertinggal."
\par 16 Lalu istri Potifar menyimpan jubah itu sampai suaminya pulang.
\par 17 Sekembalinya suaminya, ia segera menuturkan cerita itu kepadanya, katanya, "Orang Ibrani yang kaubawa ke mari itu, masuk ke dalam kamar untuk menghina saya.
\par 18 Tetapi ketika saya berteriak, ia lari ke luar dan jubahnya tertinggal."
\par 19 Potifar menjadi sangat marah.
\par 20 Dan ia memerintahkan supaya Yusuf segera ditangkap dan dimasukkan ke dalam penjara, tempat tahanan-tahanan raja dikurung.
\par 21 Tetapi TUHAN menolong Yusuf dan terus mengasihinya, sehingga kepala penjara suka kepadanya.
\par 22 Ia mempercayakan tahanan-tahanan lainnya kepada Yusuf, dan Yusuflah yang diserahi tanggung jawab atas segala pekerjaan yang dilakukan di dalam penjara itu.
\par 23 Kepala penjara itu tidak lagi mengawasi segala yang dipercayakannya kepada Yusuf, karena TUHAN menolongnya sehingga dia berhasil dalam segala pekerjaannya.

\chapter{40}

\par 1 Beberapa waktu kemudian dua pelayan raja Mesir, yaitu pengurus minuman dan pengurus rotinya, membuat kesalahan terhadap raja.
\par 2 Maka marahlah raja kepada kedua pelayannya itu,
\par 3 lalu mereka dimasukkannya ke dalam penjara di rumah kepala pengawal istana, di tempat Yusuf ditahan.
\par 4 Lama juga mereka di penjara itu, dan kepala pengawal istana menugaskan Yusuf untuk melayani mereka.
\par 5 Pada suatu malam pengurus minuman dan pengurus roti itu masing-masing bermimpi. Arti mimpi mereka itu tidak sama.
\par 6 Ketika Yusuf datang kepada mereka keesokan paginya, mereka kelihatan sedih.
\par 7 Lalu ia bertanya, "Mengapa Saudara-saudara begitu sedih hari ini?"
\par 8 Jawab mereka, "Tadi malam kami mimpi, dan tidak ada yang tahu artinya." Kata Yusuf, "Cuma Allah yang memungkinkan orang menerangkan arti mimpi. Coba ceritakan mimpimu itu."
\par 9 Pengurus minuman itu berkata, "Dalam mimpi itu saya melihat ada pohon anggur di depan saya.
\par 10 Pohon itu bercabang tiga. Baru saja cabang-cabangnya mulai berdaun, segera bunga-bunganya berkembang, lalu buahnya menjadi masak.
\par 11 Pada waktu itu saya sedang memegang gelas minuman raja, jadi saya ambil buah anggur itu dan saya peras ke dalam gelas raja, lalu saya hidangkan kepadanya."
\par 12 Yusuf berkata, "Inilah keterangannya: Tiga cabang itu artinya tiga hari.
\par 13 Dalam tiga hari ini raja akan membebaskan engkau, ia akan mengampuni engkau dan mengembalikan engkau kepada jabatanmu yang dahulu. Engkau akan menghidangkan gelas minuman kepada raja seperti dahulu.
\par 14 Tetapi ingatlah kepada saya apabila keadaanmu sudah baik. Tolong sampaikan persoalan saya kepada raja, supaya saya dibebaskan dari penjara ini.
\par 15 Sebab, sebetulnya dahulu saya diculik dari negeri orang Ibrani dan di sini pun, di Mesir ini, tidak pernah saya melakukan sesuatu kejahatan sampai harus dimasukkan ke dalam penjara."
\par 16 Setelah pengurus roti itu tahu bahwa arti mimpi pengurus minuman itu baik, maka dia pun berkata kepada Yusuf, "Saya bermimpi juga, saya menjunjung tiga buah keranjang roti di atas kepala.
\par 17 Dalam keranjang yang paling atas terdapat bermacam-macam kue-kue untuk raja, tetapi burung-burung datang memakan kue-kue itu."
\par 18 Jawab Yusuf, "Inilah keterangan mimpi itu: Tiga keranjang itu artinya tiga hari.
\par 19 Dalam tiga hari ini raja akan menyuruh orang memenggal kepalamu lalu menggantungkan mayatmu pada sebuah tiang, dan burung-burung akan makan dagingmu."
\par 20 Tiga hari kemudian adalah hari ulang tahun raja, dan ia mengadakan pesta besar bagi semua pegawai. Ia memerintahkan supaya pengurus minuman dan pengurus roti dikeluarkan dari penjara dan dibawa ke hadapan semua pegawai istana.
\par 21 Pengurus minuman dikembalikannya kepada jabatan yang dahulu.
\par 22 Tetapi pengurus roti dihukum mati. Semuanya itu terjadi sesuai dengan apa yang dikatakan Yusuf.
\par 23 Tetapi pengurus minuman itu tidak ingat lagi kepada Yusuf. Ia sama sekali lupa padanya.

\chapter{41}

\par 1 Setelah lewat dua tahun, raja Mesir bermimpi, bahwa ia sedang berdiri di tepi Sungai Nil.
\par 2 Tiba-tiba tujuh ekor sapi yang gemuk-gemuk dan berkulit mengkilat, keluar dari sungai itu lalu mulai makan rumput di tepi sungai itu.
\par 3 Kemudian tujuh sapi yang lain muncul pula; binatang-binatang itu kurus dan tinggal kulit pembalut tulang. Sapi-sapi yang kurus itu berdiri di samping sapi-sapi yang gemuk, di tepi sungai itu.
\par 4 Kemudian sapi-sapi yang kurus memakan sapi-sapi yang gemuk. Setelah itu raja bangun dari tidurnya.
\par 5 Kemudian ia tertidur dan bermimpi lagi. Dalam mimpinya ia melihat tujuh bulir gandum yang berisi dan masak-masak tumbuh pada satu tangkai.
\par 6 Kemudian tumbuh pula tujuh bulir gandum yang lain, yang kurus-kurus dan kerut kering oleh angin gurun.
\par 7 Lalu bulir gandum yang kurus itu menelan ketujuh bulir yang berisi tadi. Setelah itu raja terbangun dan sadar bahwa ia telah bermimpi.
\par 8 Paginya raja merasa gelisah, karena itu disuruhnya memanggil semua tukang sihir dan orang berilmu di Mesir. Lalu diceritakannya mimpinya kepada mereka, tetapi tak seorang pun dapat menerangkan artinya.
\par 9 Kemudian pengurus minuman berkata kepada raja, "Hari ini hamba harus mengaku kesalahan hamba.
\par 10 Dahulu Baginda marah kepada pengurus roti dan kepada hamba, lalu kami dimasukkan ke dalam penjara, di rumah kepala pengawal istana.
\par 11 Pada suatu malam kami berdua bermimpi, dan mimpi kami itu tidak sama artinya.
\par 12 Seorang pemuda Ibrani ada di sana dengan kami. Dia pelayan kepala pengawal istana itu. Kami menceritakan mimpi kami kepadanya, lalu diterangkannya arti mimpi itu.
\par 13 Ternyata semuanya tepat terjadi seperti dikatakannya, yaitu: Baginda mengembalikan hamba kepada jabatan hamba semula, tetapi menghukum mati pengurus roti itu."
\par 14 Maka raja menyuruh mengambil Yusuf, dan dengan segera ia dikeluarkan dari penjara. Setelah Yusuf bercukur dan berganti pakaian, ia menghadap raja.
\par 15 Kata raja kepadanya, "Aku telah bermimpi, dan tak seorang pun dapat menerangkan artinya. Ada yang mengabarkan kepadaku bahwa engkau dapat menerangkan mimpi."
\par 16 Yusuf menjawab, "Bukan hamba, Tuanku, melainkan Allah yang akan memberikan penjelasan yang tepat."
\par 17 Lalu berkatalah raja, "Aku bermimpi bahwa aku sedang berdiri di tepi Sungai Nil.
\par 18 Lalu keluarlah dari sungai itu tujuh sapi yang gemuk-gemuk dan berkulit mengkilap, lalu mulai makan rumput di tepi sungai itu.
\par 19 Kemudian muncullah pula tujuh sapi yang lain, yang kurus-kurus dan tinggal kulit pembalut tulang. Belum pernah aku melihat sapi yang begitu jelek di seluruh Mesir.
\par 20 Sapi-sapi yang kurus itu memakan habis ketujuh sapi yang gemuk tadi.
\par 21 Tetapi setelah itu sapi-sapi yang kurus itu masih tetap kurus. Lalu terbangunlah aku dari tidurku.
\par 22 Kemudian aku tertidur dan bermimpi lagi, bahwa aku melihat tujuh bulir gandum yang berisi dan masak-masak, tumbuh pada satu tangkai.
\par 23 Lalu tumbuh pula tujuh bulir gandum yang kurus-kurus dan kerut kering oleh angin gurun.
\par 24 Bulir gandum yang kurus itu menelan bulir yang berisi tadi. Telah kuceritakan kedua mimpiku itu kepada para tukang sihir, tetapi tak seorang pun dapat menerangkan artinya."
\par 25 Lalu kata Yusuf kepada raja, "Kedua mimpi itu sama artinya; Allah telah memberitahukan kepada Tuanku apa yang akan dilakukannya.
\par 26 Tujuh sapi yang gemuk itu ialah tujuh tahun, dan tujuh bulir gandum yang berisi itu ialah tujuh tahun juga; keduanya sama artinya.
\par 27 Tujuh sapi yang kurus, yang muncul kemudian, serta tujuh bulir gandum yang kurus dan kerut kering oleh angin gurun itu ialah masa kelaparan selama tujuh tahun.
\par 28 Sebagaimana telah hamba katakan kepada Tuanku, Allah telah memperlihatkan kepada Tuanku apa yang akan dilakukannya.
\par 29 Nanti akan datang tujuh tahun masa penuh kemakmuran di seluruh negeri Mesir.
\par 30 Setelah itu akan datang tujuh tahun kelaparan, dan masa penuh kemakmuran itu akan dilupakan sama sekali, karena masa kelaparan itu akan hebat sekali sehingga negeri ini menjadi tandus.
\par 32 Mimpi Tuanku terjadi dua kali, itu berarti bahwa hal itu telah ditetapkan oleh Allah dan bahwa Allah akan melaksanakannya dengan segera.
\par 33 Karena itu, sebaiknya Tuanku memilih seorang yang cerdas dan bijaksana dan memberinya kuasa untuk mengatur negeri ini.
\par 34 Tuanku harus pula mengangkat pegawai-pegawai lainnya, dan memberi mereka kuasa untuk memungut seperlima dari semua panen gandum selama tujuh tahun masa penuh kemakmuran itu, lalu menimbunnya di kota-kota serta menjaganya.
\par 36 Gandum itu akan menjadi persediaan makanan selama tujuh tahun masa kelaparan yang akan datang di Mesir. Dengan demikian rakyat tidak akan mati kelaparan."
\par 37 Raja dan para pegawainya menyetujui rencana Yusuf itu.
\par 38 Lalu raja berkata kepada mereka, "Tak mungkin kita mendapatkan orang lain yang lebih cocok daripada Yusuf, sebab ia dipimpin oleh Roh Allah."
\par 39 Maka raja berkata kepada Yusuf, "Allah telah memberitahukan semua ini kepadamu, jadi jelaslah bahwa engkau lebih cerdas dan bijaksana dari siapa pun juga.
\par 40 Engkau akan kuangkat menjadi gubernur, dan seluruh rakyatku akan mentaati perintahmu. Hanya aku sajalah yang lebih berkuasa daripadamu."
\par 41 Setelah itu raja menanggalkan dari jarinya cincin yang berukiran stempel kerajaan, lalu memasukkannya ke jari Yusuf sambil berkata, "Dengan ini engkau kuangkat menjadi gubernur seluruh Mesir." Kemudian dikenakannya pada Yusuf sebuah jubah linen yang halus, dan dikalungkannya pada lehernya sebuah rantai emas.
\par 43 Lalu diberikannya kepada Yusuf kereta kerajaan yang kedua untuk kendaraannya, dan pengawal kehormatan raja berjalan di depan kereta itu sambil berseru-seru, "Awas! Beri jalan! Beri jalan!" Demikianlah Yusuf diangkat menjadi gubernur seluruh Mesir.
\par 44 Kata raja kepadanya, "Akulah raja--dan aku mengumumkan bahwa tanpa izinmu tidak seorang pun di seluruh Mesir boleh melakukan apa-apa."
\par 45 Lalu raja memberikan sebuah nama Mesir kepada Yusuf, yaitu Zafnat-Paaneah. Diberikannya juga seorang istri yang bernama Asnat, anak Potifera yang menjabat imam di kota Heliopolis. Yusuf berumur tiga puluh tahun ketika ia mulai bekerja untuk raja Mesir. Maka berangkatlah Yusuf dari istana raja dan pergi mengelilingi seluruh negeri.
\par 47 Dalam masa tujuh tahun penuh kemakmuran itu, tanah menghasilkan panen yang berlimpah-limpah.
\par 48 Gandum itu dikumpulkan oleh Yusuf lalu disimpannya di kota-kota. Dalam setiap kota ia menyimpan gandum hasil ladang-ladang di sekitar kota itu.
\par 49 Gandum yang dikumpulkannya itu begitu banyak sehingga Yusuf berhenti menakarnya, karena banyaknya seperti pasir di tepi laut.
\par 50 Sebelum masa kelaparan itu tiba, Asnat istri Yusuf melahirkan dua anak laki-laki.
\par 51 Kata Yusuf, "Allah telah membuat saya lupa kepada segala penderitaan saya dan kepada kaum keluarga ayah saya." Karena itu dinamakannya anaknya yang pertama "Manasye".
\par 52 Dia berkata pula, "Allah telah memberikan anak-anak kepada saya dalam masa kesukaran saya," lalu dinamakannya anaknya yang kedua "Efraim".
\par 53 Tujuh tahun masa penuh kemakmuran yang telah dinikmati negeri Mesir itu berakhir.
\par 54 Maka datanglah tujuh tahun masa kelaparan, tepat seperti yang telah dikatakan Yusuf. Di seluruh dunia terjadi kelaparan, tetapi di seluruh Mesir ada persediaan makanan.
\par 55 Ketika rakyat Mesir mulai menderita lapar, mereka meminta makanan kepada raja. Lalu raja menyuruh mereka pergi kepada Yusuf dan mentaati segala apa yang akan diperintahkan Yusuf kepada mereka.
\par 56 Ketika kelaparan itu menjadi makin hebat dan menyebar di seluruh negeri, Yusuf membuka semua gudang dan menjual gandum kepada orang Mesir.
\par 57 Dari seluruh dunia orang-orang datang ke Mesir untuk membeli gandum dari Yusuf, karena kelaparan itu sungguh dahsyat di mana-mana.

\chapter{42}

\par 1 Ketika Yakub mendengar bahwa ada gandum di Mesir, berkatalah ia kepada anak-anaknya, "Mengapa kamu tenang-tenang saja?
\par 2 Telah kudengar bahwa ada gandum di Mesir; pergilah ke sana dan belilah gandum supaya kita jangan mati kelaparan."
\par 3 Lalu pergilah kesepuluh abang Yusuf itu membeli gandum di Mesir.
\par 4 Tetapi Yakub tidak mengizinkan Benyamin, adik kandung Yusuf, pergi bersama mereka, karena ia takut jangan-jangan terjadi kecelakaan dengan anaknya itu.
\par 5 Karena kelaparan di negeri Kanaan, anak-anak Yakub bersama banyak orang lain datang membeli gandum di Mesir.
\par 6 Yang menjual gandum kepada orang-orang dari seluruh dunia ialah Yusuf, sebagai gubernur Mesir. Sebab itu abang-abang Yusuf datang dan sujud di hadapannya.
\par 7 Ketika Yusuf melihat abang-abangnya, ia mengenali mereka, tetapi ia pura-pura tidak kenal. Dengan kasar ia bertanya kepada mereka, "Kamu orang mana?" "Kami orang Kanaan. Kami datang untuk membeli makanan," jawab mereka.
\par 8 Meskipun Yusuf mengenali abang-abangnya, mereka tidak mengenali dia.
\par 9 Lalu ia teringat kepada mimpinya tentang mereka. Dan berkatalah ia, "Kamu ini mata-mata; kamu datang untuk menyelidiki di mana kelemahan negeri kami."
\par 10 "Tidak, Tuanku," jawab mereka. "Kami, hamba-hamba Tuan datang hanya untuk membeli makanan.
\par 11 Kami ini bersaudara, Tuanku. Kami ini orang baik-baik, bukan mata-mata."
\par 12 Yusuf berkata kepada mereka, "Bohong! Kamu datang kemari untuk menyelidiki di mana kelemahan negeri ini."
\par 13 Jawab mereka, "Kami ini dua belas bersaudara, Tuanku, kami anak dari satu ayah di negeri Kanaan. Seorang dari kami telah meninggal dan yang bungsu sekarang ada bersama ayah kami."
\par 14 "Memang benar seperti kataku," jawab Yusuf, "kamu ini mata-mata.
\par 15 Aku mau menguji kamu: Aku bersumpah demi nama raja bahwa kamu tidak akan meninggalkan negeri ini jika adikmu yang bungsu itu tidak datang kemari.
\par 16 Seorang dari kamu harus pulang mengambilnya. Yang lain akan ditahan sampai perkataanmu terbukti. Jika tidak, demi nama raja, kamu ini mata-mata musuh!"
\par 17 Kemudian mereka dimasukkan ke dalam penjara tiga hari lamanya.
\par 18 Pada hari yang ketiga Yusuf berkata kepada mereka, "Aku orang yang takut dan taat kepada Allah. Kamu akan kuselamatkan dengan satu syarat.
\par 19 Untuk membuktikan bahwa kamu ini jujur, seorang dari kamu akan ditahan dalam penjara; yang lain boleh pulang dan membawa gandum yang kamu beli untuk keluargamu yang sedang menderita lapar.
\par 20 Setelah itu kamu harus membawa adikmu yang bungsu kepadaku. Itulah buktinya nanti bahwa perkataanmu itu benar, dan kamu tidak akan kuhukum mati." Mereka setuju dengan keputusan gubernur itu.
\par 21 Lalu berkatalah mereka seorang kepada yang lain, "Nah, kita sekarang dihukum akibat kesalahan kita terhadap adik kita dahulu; ia minta tolong tetapi kita tidak mau peduli, walaupun kita melihat bahwa ia sangat menderita. Itulah sebabnya kita sekarang mengalami penderitaan ini."
\par 22 Kata Ruben, "Bukankah dahulu sudah saya katakan kepada kalian supaya anak itu jangan diapa-apakan. Tetapi kalian tidak mau mendengarkan. Dan sekarang kematiannya dibalaskan kepada kita."
\par 23 Yusuf mengerti bahasa mereka, tetapi mereka tidak mengetahui hal itu, karena mereka berbicara dengan Yusuf dengan perantaraan seorang juru bahasa.
\par 24 Yusuf meninggalkan mereka, lalu menangis. Ketika sudah dapat berkata-kata lagi, ia kembali kepada mereka, lalu mengambil Simeon, dan menyuruh mengikat dia di depan semua saudaranya.
\par 25 Yusuf memerintahkan supaya karung-karung yang dibawa abang-abangnya diisi dengan gandum, dan uang mereka masing-masing dimasukkan ke dalam karung-karung itu. Juga supaya mereka diberi makanan untuk bekal di jalan. Perintahnya itu dilaksanakan.
\par 26 Setelah itu abang-abang Yusuf membebani keledai mereka dengan gandum yang telah mereka beli itu, lalu berangkatlah mereka dari situ.
\par 27 Di tempat mereka bermalam, seorang dari mereka membuka karung gandumnya untuk memberi makan keledainya. Ditemukannya uangnya di atas gandum itu.
\par 28 "Uang saya dikembalikan," serunya kepada saudara-saudaranya. "Lihat, ada di dalam karung saya!" Hati mereka menjadi kecut, dan dengan ketakutan mereka saling bertanya, "Apa yang dilakukan Allah kepada kita?"
\par 29 Waktu sampai di Kanaan, mereka menceritakan kepada ayah mereka segala sesuatu yang telah mereka alami. Kata mereka,
\par 30 "Gubernur Mesir bicara dengan kasar kepada kami dan menuduh kami memata-matai negerinya.
\par 31 Kami menjawab, 'Kami orang baik-baik, bukan mata-mata, kami orang jujur.
\par 32 Kami semua dua belas bersaudara, anak dari satu ayah, dan seorang telah meninggal, sedangkan yang bungsu ada bersama ayah di Kanaan.'
\par 33 Gubernur itu mengatakan, 'Aku mau menguji kamu untuk mengetahui apakah kamu orang jujur: Seorang dari kamu harus tinggal; yang lain boleh pulang membawa gandum kepada keluargamu yang sedang menderita lapar.
\par 34 Setelah kamu kembali membawa adikmu yang bungsu, aku akan tahu bahwa kamu bukan mata-mata, melainkan orang jujur. Maka saudaramu yang kutahan itu akan kukembalikan kepadamu dan kamu boleh tinggal di negeri ini dan bebas berdagang.'"
\par 35 Kemudian, pada waktu mereka mengosongkan karung-karung mereka, mereka menemukan dompet masing-masing yang berisi uang; lalu mereka sangat ketakutan, juga ayah mereka.
\par 36 Lalu berkatalah ayah mereka, "Kalian menyebabkan aku kehilangan semua anakku. Yusuf tidak ada lagi; Simeon tidak ada lagi; dan sekarang kalian akan mengambil Benyamin juga. Bukan main penderitaanku!"
\par 37 Lalu kata Ruben kepada ayahnya, "Serahkanlah Benyamin kepada saya, Ayah; nanti akan saya bawa kembali, jika tidak, Ayah boleh membunuh kedua anak laki-laki saya."
\par 38 Tetapi Yakub berkata, "Tidak! Benyamin tidak boleh kalian bawa; abangnya telah mati dan kini hanya dialah yang masih tinggal. Jangan-jangan ia mendapat celaka dalam perjalanan itu. Aku ini sudah tua, dan kesedihan yang akan kalian datangkan kepadaku itu akan mengakibatkan kematianku."

\chapter{43}

\par 1 Kelaparan di Kanaan makin hebat, dan setelah keluarga Yakub menghabiskan semua gandum yang dibawa dari Mesir, berkatalah Yakub kepada anak-anaknya, "Pergilah lagi ke sana dan belilah gandum untuk kita."
\par 3 Lalu Yehuda berkata kepadanya, "Gubernur Mesir telah memberi peringatan keras bahwa kami tidak boleh menghadap dia jika kami tidak membawa adik kami itu.
\par 4 Kalau Ayah mengizinkan adik kami ikut, kami mau pergi ke sana membeli makanan.
\par 5 Tetapi kalau Ayah tidak mengizinkan Benyamin ikut, kami tidak mau pergi, sebab gubernur itu telah berkata bahwa kami tidak boleh menghadap jika adik kami tidak ikut."
\par 6 Kata Yakub, "Mengapa kalian menyusahkan aku dengan memberitahukan kepada orang itu bahwa kalian masih mempunyai adik?"
\par 7 Jawab mereka, "Orang itu terus-menerus bertanya tentang kami dan tentang keluarga kita, katanya, 'Masih hidupkah ayahmu? Apakah kamu masih punya adik laki-laki lain?' Kami terpaksa menjawab segala pertanyaannya. Bagaimana kami dapat menduga bahwa dia akan menyuruh kami membawa adik kami itu?"
\par 8 Lalu kata Yehuda kepada ayahnya, "Izinkanlah anak itu ikut, dan saya yang bertanggung jawab atas dia. Kami akan berangkat dengan segera supaya tidak seorang pun dari kita mati kelaparan.
\par 9 Sayalah jaminannya, dan Ayah boleh menuntut dia dari saya. Kalau dia tidak saya bawa kembali dengan selamat, saya tanggung hukumannya seumur hidup.
\par 10 Seandainya kita tidak menunggu begitu lama, pasti kami sekarang sudah pulang pergi dua kali."
\par 11 Lalu ayah mereka berkata, "Jika memang harus begitu, bawalah hasil yang paling baik dari negeri ini dalam karung-karungmu sebagai hadiah untuk gubernur itu: Kismis, madu, rempah-rempah, buah kemiri dan buah badam.
\par 12 Bawalah juga uang dua kali lipat, karena uang yang ditemukan di dalam karungmu itu harus kalian kembalikan. Barangkali itu suatu kekeliruan.
\par 13 Bawalah adikmu itu, dan kembalilah dengan segera.
\par 14 Semoga Allah Yang Mahakuasa membuat gubernur itu merasa kasihan kepada kalian, sehingga ia mau mengembalikan Benyamin dan Simeon kepadamu. Mengenai aku ini, jika aku memang harus kehilangan anak-anakku, apa boleh buat."
\par 15 Kemudian saudara-saudara Yusuf membawa hadiah-hadiah itu dan uang dua kali lipat, lalu berangkat ke Mesir bersama Benyamin. Setelah sampai di sana, mereka menghadap Yusuf.
\par 16 Ketika Yusuf melihat Benyamin dan abang-abangnya, berkatalah ia kepada pelayannya yang mengepalai rumah tangganya, "Bawalah orang-orang itu ke rumahku. Mereka akan makan bersama aku siang ini, sebab itu sembelihlah seekor binatang ternak lalu siapkanlah itu."
\par 17 Pelayan itu melaksanakan perintah itu dan membawa saudara-saudara Yusuf ke dalam rumah gubernur.
\par 18 Ketika mereka dibawa ke dalam rumah Yusuf, mereka ketakutan dan berpikir, "Kita dibawa ke sini, karena uang yang kita temukan dalam karung kita pada waktu kita datang kemari dahulu. Mereka akan menangkap kita dengan tiba-tiba, lalu mengambil keledai kita dan menjadikan kita hamba mereka."
\par 19 Karena itu, di dekat pintu rumah Yusuf, mereka berkata kepada kepala rumah tangga,
\par 20 "Maaf, Tuan, kami sudah pernah datang kemari untuk membeli makanan.
\par 21 Pada perjalanan pulang, ketika kami hendak bermalam, kami membuka karung kami. Tahu-tahu seluruh uang pembayaran gandum kami ada di atas gandum. Kami tidak tahu siapa yang memasukkan uang itu. Sekarang kami membawanya kembali kepada Tuan. Selain itu kami masih membawa uang juga untuk membeli makanan lagi."
\par 23 Pelayan itu berkata, "Jangan takut. Jangan khawatir. Allahmu, yaitu Allah yang dipuja ayahmu, Dialah yang memasukkan uang itu ke dalam karungmu. Saya telah menerima pembayaranmu untuk gandum itu." Kemudian dibawanya Simeon kepada mereka,
\par 24 lalu diantarkannya mereka ke dalam rumah. Ia memberi mereka air untuk membasuh kaki, dan juga makanan untuk keledai mereka.
\par 25 Setelah itu mereka menyiapkan hadiah untuk diberikan kepada Yusuf apabila ia datang pada waktu tengah hari, karena mereka sudah diberitahukan bahwa mereka akan makan bersama-sama dengan dia.
\par 26 Ketika Yusuf sampai di rumah, mereka memberikan hadiah-hadiah itu kepada Yusuf sambil sujud kepadanya.
\par 27 Yusuf bertanya tentang keadaan mereka, lalu berkata, "Kamu telah menceritakan kepadaku tentang ayahmu yang sudah tua itu. Bagaimana keadaannya? Baik-baikkah?"
\par 28 Jawab mereka, "Hamba Tuan, ayah kami, baik keadaannya." Lalu mereka sujud kepadanya.
\par 29 Ketika Yusuf melihat Benyamin, adiknya, berkatalah ia, "Jadi, inikah adikmu yang bungsu itu, yang telah kamu sebut-sebut kepadaku? Semoga Allah memberkatimu, anakku."
\par 30 Hati Yusuf meluap karena rasa rindu dan sayang kepada adiknya. Ia hampir tak dapat menguasai dirinya, karena itu pergilah ia dari situ dengan tiba-tiba lalu masuk ke kamarnya dan menangis.
\par 31 Sesudah itu ia membasuh mukanya, lalu keluar lagi, dan dengan menahan hatinya, ia menyuruh menghidangkan makanan.
\par 32 Yusuf makan pada sebuah meja tersendiri, dan saudara-saudaranya pada meja yang lain. Orang-orang Mesir yang ada di situ makan pada meja yang lain lagi, karena mereka merasa hina bila makan bersama-sama dengan orang Ibrani.
\par 33 Saudara-saudara Yusuf ditempatkan pada meja yang berhadapan dengan Yusuf, menurut urutan umurnya masing-masing, mulai dari yang sulung sampai yang bungsu. Ketika mereka melihat cara penempatan itu, mereka saling memandang dengan heran.
\par 34 Mereka diberi makanan dari meja Yusuf, dan Benyamin dilayani lima kali lebih banyak daripada abang-abangnya. Lalu makan dan minumlah mereka bersama-sama dengan Yusuf sampai puas.

\chapter{44}

\par 1 Sementara itu Yusuf memerintahkan kepada kepala rumah tangganya, "Isilah karung orang-orang itu dengan gandum, sebanyak yang dapat mereka bawa, dan masukkan uang masing-masing ke dalam karungnya, di atas gandum itu.
\par 2 Masukkan juga piala perakku ke dalam karung adik mereka yang bungsu, bersama-sama dengan uang pembayaran gandumnya." Pelayan itu melaksanakan perintah itu.
\par 3 Keesokan harinya, pagi-pagi sekali, Yusuf berpisah dengan saudara-saudaranya, lalu mereka berangkat dengan keledai mereka.
\par 4 Mereka belum jauh dari kota itu, waktu Yusuf berkata kepada kepala rumah tangganya, "Cepatlah kejar orang-orang itu. Jika sudah tersusul, katakan kepada mereka, 'Mengapa kamu membalas kebaikan dengan kejahatan?
\par 5 Mengapa kamu mencuri piala perak tuanku? Piala itu dipakainya untuk minum dan sebagai alat menujum. Kamu telah melakukan kejahatan besar!'"
\par 6 Ketika pelayan itu sampai kepada saudara-saudara Yusuf, dikatakannya apa yang diperintahkan Yusuf.
\par 7 Jawab mereka kepadanya, "Apa maksud Tuan? Kami bersumpah bahwa kami tidak berbuat begitu!
\par 8 Tuan sendiri tahu bahwa uang yang kami temukan di dalam karung-karung kami di atas gandum itu, telah kami kembalikan kepada Tuan. Jadi, tak mungkin kami mencuri perak atau emas dari rumah gubernur!
\par 9 Tuan, andaikata benda itu kedapatan pada salah seorang dari kami, biarlah dia dihukum mati, dan kami menjadi hamba Tuan."
\par 10 Pelayan itu berkata, "Baiklah; tetapi hanya dia pada siapa piala itu kedapatan, dialah yang akan menjadi hambaku; yang lain boleh pergi."
\par 11 Lalu dengan cepat mereka masing-masing menurunkan dan membuka karungnya.
\par 12 Pelayan Yusuf itu memeriksa karung-karung itu dengan teliti, mulai dari karung kepunyaan yang sulung sampai kepada karung kepunyaan yang bungsu, dan piala itu ditemukan di dalam karung Benyamin.
\par 13 Abang-abangnya sangat sedih sehingga mengoyak-ngoyakkan pakaian mereka. Mereka membebani keledai mereka dan kembali ke kota.
\par 14 Ketika Yehuda dan saudara-saudaranya sampai di rumah Yusuf, ia masih ada di situ, dan mereka sujud di hadapannya.
\par 15 Maka berkatalah Yusuf, "Apa yang kamu lakukan itu? Tak tahukah kamu bahwa orang yang seperti aku ini dapat mengetahui perbuatanmu yang jahat dengan ilmu gaib?"
\par 16 "Apa yang dapat kami katakan, Tuanku?" jawab Yehuda. "Bagaimana kami dapat membantah dan membenarkan diri kami? Allah telah menyingkapkan kesalahan kami. Sekarang kami semua hamba Tuan, bukan hanya dia pada siapa kedapatan piala itu."
\par 17 Kata Yusuf, "Tidak! Aku tidak mau berbuat begitu! Hanya dia pada siapa kedapatan piala itu akan menjadi hambaku. Yang lain boleh pulang dengan bebas kepada ayahmu."
\par 18 Yehuda maju mendekati Yusuf dan berkata, "Maaf, Tuanku, izinkanlah hamba berbicara lagi dengan Tuanku. Jangan marah kepada hamba; Tuanku seperti raja Mesir sendiri.
\par 19 Tuanku telah bertanya kepada kami ini, 'Apakah kamu masih mempunyai ayah atau saudara yang lain?'
\par 20 Kami menjawab, 'Ayah kami sudah tua dan adik kami lahir ketika ayah sudah lanjut usia. Abang seibu dari adik kami itu sudah meninggal, jadi sekarang hanya dia sendirilah yang masih hidup dari mereka berdua, dan ayah sangat sayang kepadanya.'
\par 21 Tuanku menyuruh kami membawa dia kemari, supaya Tuanku dapat melihatnya,
\par 22 lalu kami menjawab bahwa anak itu tidak dapat berpisah dari ayahnya; jika ia berpisah dari ayahnya, ayah akan meninggal.
\par 23 Kemudian Tuanku berkata, 'Kamu tidak boleh menghadap aku lagi jika tidak membawa adikmu itu.'
\par 24 Ketika kami kembali kepada ayah kami, kami sampaikan kepadanya perkataan Tuanku itu.
\par 25 Kemudian ayah kami menyuruh kami datang lagi kemari untuk membeli makanan.
\par 26 Kami menjawab, 'Kami tidak dapat pergi ke sana, sebab kami tak boleh menghadap gubernur jika adik kami yang bungsu tidak ikut. Kami hanya dapat pergi ke sana kalau dia pergi juga.'
\par 27 Kemudian ayah kami berkata, 'Kalian tahu bahwa Rahel, istriku hanya punya dua anak.
\par 28 Yang pertama telah meninggalkan aku. Dia pasti sudah diterkam binatang buas, karena sampai sekarang aku tidak melihatnya lagi.
\par 29 Jika kalian mengambil anak yang bungsu ini daripadaku, dan terjadi apa-apa dengan dia, kesedihan yang kalian datangkan kepadaku itu akan mengakibatkan kematianku, karena aku ini sudah tua.'"
\par 30 "Karena itu, Tuanku," kata Yehuda kepada Yusuf, "jika hamba kembali kepada ayah kami tanpa adik kami itu, pasti ayah kami akan meninggal. Nyawanya bergantung kepada anak itu, dan ia sudah begitu tua sehingga kesedihan yang kami datangkan kepadanya itu akan mengakibatkan kematiannya.
\par 32 Lagipula, hamba telah berjanji kepada ayah hamba bahwa hamba menjadi jaminan anak itu. Hamba berkata kepadanya, bahwa jika hamba tidak membawa anak itu kembali kepadanya, hambalah yang akan menanggung hukuman seumur hidup.
\par 33 Jadi, hamba mohon, Tuanku, izinkanlah hamba tinggal di sini menjadi hamba Tuanku menggantikan adik kami ini; biarlah ia pulang bersama-sama dengan abang-abangnya.
\par 34 Bagaimana hamba dapat kembali kepada ayah kami jika anak itu tidak ikut? Hamba tidak tahan nanti melihat musibah yang akan menimpa ayah kami itu."

\chapter{45}

\par 1 Yusuf tidak sanggup lagi menahan perasaannya di hadapan pegawai-pegawainya Karena itu disuruhnya mereka meninggalkan ruangan itu supaya ia dapat menyatakan kepada saudara-saudaranya siapa dia sebenarnya.
\par 2 Setelah semua pegawainya meninggalkan ruangan, menangislah Yusuf keras-keras, sehingga orang-orang Mesir di luar ruangan mendengarnya, dan sampailah kabar itu ke istana raja.
\par 3 Yusuf berkata kepada saudara-saudaranya, "Saya ini Yusuf. Masih hidupkah ayah?" Melihat itu saudara-saudaranya takut sekali sehingga tidak dapat menjawab.
\par 4 Lalu kata Yusuf kepada mereka, "Marilah ke sini." Mereka mendekat, dan dia berkata lagi, "Saya Yusuf, yang telah kalian jual ke Mesir.
\par 5 Jangan takut atau menyesali dirimu karena kalian telah menjual saya. Sebenarnya Allah sendiri yang membawa saya ke sini mendahului kalian untuk menyelamatkan banyak orang.
\par 6 Sekarang baru tahun kedua dari masa kelaparan, dan selama lima tahun lagi orang tidak akan membajak atau panen.
\par 7 Allah telah membawa saya mendahului kalian untuk menyelamatkan kalian dengan cara yang mengherankan ini, dan untuk menjamin keselamatanmu dan kelanjutan keturunanmu.
\par 8 Jadi, sebetulnya bukan kalian yang menyebabkan saya ada di sini, melainkan Allah. Dia telah menjadikan saya pegawai tertinggi raja Mesir. Saya diserahi kuasa atas seluruh rumah tangganya dan seluruh Mesir.
\par 9 Sekarang, cepatlah kalian kembali kepada ayah dan katakanlah kepadanya bahwa Yusuf, anaknya, berkata begini, 'Allah telah menjadikan saya penguasa atas seluruh Mesir; datanglah selekas mungkin.
\par 10 Ayah dapat tinggal di daerah Gosyen, dekat dengan saya--ayah, dengan anak cucu, domba, kambing, sapi dan segala milik ayah.
\par 11 Jika ayah ada di Gosyen, saya dapat memelihara ayah. Masa kelaparan masih berlangsung lima tahun lagi dan akan saya usahakan supaya ayah, keluarga dan ternak ayah jangan kekurangan apa-apa.'"
\par 12 Kata Yusuf lagi, "Sekarang kalian lihat sendiri, juga engkau Benyamin, bahwa saya benar-benar Yusuf.
\par 13 Katakanlah kepada ayah kita betapa besar kuasa saya di sini, di Mesir, dan ceritakanlah kepadanya segala yang sudah kalian lihat. Lalu cepat-cepatlah bawa dia kemari."
\par 14 Sesudah itu ia memeluk Benyamin, adiknya itu, lalu menangis; Benyamin juga menangis sambil memeluk Yusuf pula.
\par 15 Kemudian, dengan masih menangis, Yusuf memeluk semua saudaranya dan mencium mereka. Setelah itu mereka mulai bercakap-cakap dengan dia.
\par 16 Ketika di istana raja terdengar kabar bahwa saudara-saudara Yusuf datang, raja dan pegawai-pegawainya ikut senang.
\par 17 Lalu berkatalah raja kepada Yusuf, "Suruhlah saudara-saudaramu itu membebani keledai mereka dengan gandum dan pulang ke Kanaan
\par 18 untuk menjemput ayah dan keluarga mereka lalu pindah ke mari. Aku akan memberikan kepada mereka tanah yang paling baik di Mesir, dan mereka akan dapat hidup dengan berkecukupan dari hasil tanah itu.
\par 19 Suruhlah mereka juga membawa dari sini beberapa kereta untuk istri dan anak-anak mereka yang masih kecil, dan untuk menjemput ayah mereka.
\par 20 Mereka tak perlu memikirkan barang-barang yang terpaksa mereka tinggalkan, karena yang paling baik di seluruh Mesir akan menjadi milik mereka."
\par 21 Anak-anak Yakub melakukan perintah raja itu. Yusuf memberikan kepada mereka beberapa kereta, sesuai dengan perintah raja, dan juga bekal untuk perjalanan.
\par 22 Diberikannya juga kepada mereka masing-masing satu setel pakaian baru, tetapi kepada Benyamin diberinya tiga ratus uang perak dan lima setel pakaian baru.
\par 23 Ia mengirimkan kepada ayahnya sepuluh ekor keledai yang dibebani dengan barang-barang yang terbaik dari Mesir, dan sepuluh keledai lagi yang dibebani dengan gandum, roti dan makanan lain untuk dimakan dalam perjalanannya ke Mesir.
\par 24 Kemudian ia memberangkatkan saudara-saudaranya, sambil berkata kepada mereka, "Jangan bertengkar di jalan."
\par 25 Mereka meninggalkan Mesir dan pulang ke Kanaan kepada Yakub, ayah mereka.
\par 26 "Yusuf masih hidup!" kata mereka kepadanya. "Dia penguasa atas seluruh Mesir!" Mendengar berita itu, Yakub termangu-mangu, dan ia tidak percaya kepada mereka.
\par 27 Tetapi setelah mereka menceritakan semua yang dikatakan Yusuf kepada mereka, dan ketika ia melihat kereta yang dikirim Yusuf untuk menjemputnya dan membawanya ke Mesir, sadarlah ia dari lamunannya.
\par 28 "Anakku Yusuf masih hidup!" katanya. "Hanya itulah yang kuinginkan! Aku harus pergi dan melihatnya sebelum aku mati."

\chapter{46}

\par 1 Yakub mengemasi segala miliknya lalu berangkat. Sampai di Bersyeba ia mempersembahkan kurban kepada Allah yang dipuja oleh Ishak ayahnya.
\par 2 Dalam suatu penglihatan pada waktu malam, Allah berkata kepadanya, "Yakub, Yakub!" "Ya, Tuhan," jawabnya.
\par 3 "Aku Allah, Allah yang dipuja ayahmu," katanya. "Jangan takut untuk pergi ke Mesir. Aku akan menjadikan keturunanmu bangsa yang besar di sana.
\par 4 Aku akan menyertai engkau ke Mesir, dan membawa keturunanmu kembali ke negeri ini. Yusuf akan ada di sampingmu apabila ajalmu tiba."
\par 5 Lalu berangkatlah Yakub dari Bersyeba. Anak-anaknya menaikkan ayah mereka serta anak-istri mereka ke atas kereta yang dikirimkan raja Mesir.
\par 6 Mereka juga membawa ternak dan segala harta benda yang telah mereka peroleh di Kanaan, lalu berangkat ke Mesir. Yakub membawa seluruh keturunannya, yaitu
\par 7 semua anak cucunya laki-laki dan perempuan.
\par 8 Anggota-anggota keluarga Yakub yang turut ke Mesir ialah:
\par 9 Ruben, anak sulung; anak-anaknya: Henokh, Palu, Hezron dan Karmi.
\par 10 Simeon; anak-anaknya: Yemuel, Yamin, Ohad, Yakhin, Zohar dan Saul, anak dari istrinya seorang wanita Kanaan.
\par 11 Lewi; anak-anaknya: Gerson, Kehat dan Merari.
\par 12 Yehuda; anak-anaknya: Syela, Peres dan Zerah, sedangkan anak Yehuda yang lain, yaitu Er dan Onan meninggal di Kanaan. Anak-anak Peres: Hezron dan Hamul.
\par 13 Isakhar; anak-anaknya: Tola, Pua, Ayub dan Simron.
\par 14 Zebulon; anak-anaknya: Sered, Elon dan Yahleel.
\par 15 Mereka semua adalah anak-anak yang dilahirkan oleh Lea bagi Yakub di Mesopotamia, selain itu juga anaknya perempuan yang bernama Dina. Jadi seluruh keturunan Yakub dari istrinya Lea berjumlah tiga puluh tiga orang.
\par 16 Gad; anak-anaknya: Zifyon, Hagi, Syuni, Ezbon, Eri, Arodi dan Areli.
\par 17 Asyer; anak-anaknya: Yimna, Yiswa, Yiswi dan Beria serta Serah, saudara perempuan mereka. Anak-anak Beria ialah Heber dan Malkiel.
\par 18 Keenam belas orang itu adalah keturunan Yakub dari Zilpa, yaitu hamba wanita yang diberikan Laban kepada Lea, anaknya perempuan.
\par 19 Rahel istri Yakub mempunyai dua anak, yaitu Yusuf dan Benyamin.
\par 20 Di Mesir Yusuf memperoleh dua anak, yaitu Manasye dan Efraim yang dilahirkan oleh istrinya Asnat, anak Potifera, yang menjabat imam di Heliopolis.
\par 21 Benyamin; anak-anaknya: Bela, Bekher, Asybel, Gera, Naaman, Ehi, Ros, Mupim, Hupim dan Ared.
\par 22 Keempat belas orang itu adalah keturunan Yakub dari istrinya Rahel.
\par 23 Berikutnya Dan; anaknya Husim.
\par 24 Naftali; anak-anaknya: Yahzeel, Guni, Yezer dan Syilem.
\par 25 Ketujuh orang itu adalah keturunan Yakub dari Bilha, hamba perempuan yang diberikan Laban kepada Rahel anaknya.
\par 26 Keturunan Yakub yang pergi ke Mesir semuanya berjumlah enam puluh enam orang, tidak termasuk menantu-menantunya.
\par 27 Anak-anak Yusuf yang lahir di Mesir ada dua orang, sehingga keluarga Yakub yang tiba di Mesir seluruhnya berjumlah tujuh puluh orang.
\par 28 Yakub menyuruh Yehuda berjalan mendahuluinya untuk memanggil Yusuf supaya menemui ayahnya di Gosyen. Ketika mereka sampai di Gosyen,
\par 29 Yusuf naik keretanya untuk bertemu dengan ayahnya di situ. Waktu mereka berjumpa, Yusuf memeluk ayahnya dan lama menangis.
\par 30 Kata Yakub kepada Yusuf, "Sekarang aku rela mati, karena sudah melihat engkau dan tahu bahwa engkau masih hidup."
\par 31 Lalu kata Yusuf kepada saudara-saudaranya dan sanak saudaranya yang lain, "Saya harus pergi dan memberitahukan kepada raja bahwa saudara-saudara saya dan seluruh sanak saudara saya yang tinggal di Kanaan telah datang.
\par 32 Saya akan mengatakan bahwa kalian gembala-gembala domba dan sapi, dan telah membawa ternakmu dan segala milikmu.
\par 33 Jika raja memanggil kalian dan menanyakan pekerjaanmu,
\par 34 katakanlah kepadanya bahwa kalian ini pemelihara ternak sejak kecil, sama seperti leluhurmu. Dengan demikian ia akan menyuruh kalian tinggal di daerah Gosyen." Yusuf mengatakan hal itu karena orang Mesir merasa hina untuk bergaul dengan gembala-gembala.

\chapter{47}

\par 1 Kemudian Yusuf membawa lima orang dari saudara-saudaranya lalu menghadap raja. Katanya, "Ayah dan saudara-saudara hamba telah datang dari Kanaan dengan kawanan kambing domba, sapi dan segala milik mereka. Sekarang mereka ada di daerah Gosyen." Lalu diperkenalkannya saudara-saudaranya itu kepada raja.
\par 3 Raja bertanya kepada mereka, "Apa pekerjaanmu?" Mereka menjawab, "Kami ini gembala seperti leluhur kami.
\par 4 Kami datang untuk tinggal di negeri ini, karena di tanah Kanaan kelaparan sangat hebatnya, sehingga tidak ada rumput lagi untuk kawanan kambing domba kami. Izinkanlah kami tinggal di daerah Gosyen."
\par 5 Lalu berkatalah raja kepada Yusuf, "Sekarang ayah dan saudara-saudaramu sudah ada di sini.
\par 6 Anggaplah negeri Mesir sebagai negerimu sendiri. Biarlah mereka menetap di daerah Gosyen, daerah yang paling baik di negeri ini. Dan tugaskanlah kepada orang yang cakap bekerja untuk mengurus ternakku."
\par 7 Yusuf juga memperkenalkan ayahnya kepada raja. Yakub memberkati raja,
\par 8 dan raja bertanya kepadanya, "Berapa umur Bapak?"
\par 9 Jawab Yakub, "Hamba sudah hidup seratus tiga puluh tahun sebagai pengembara. Hidup hamba itu penuh kesukaran dan pendek apabila dibandingkan dengan umur leluhur hamba sebagai pengembara."
\par 10 Setelah itu Yakub minta diri dan memberi berkat perpisahan.
\par 11 Yusuf membantu ayah dan saudara-saudaranya menetap di Mesir. Diberikannya kepada mereka tanah sebagai hak milik di bagian yang paling baik di negeri itu, di dekat kota Rameses, sesuai dengan perintah raja.
\par 12 Dan Yusuf melengkapi persediaan makanan bagi ayahnya, saudara-saudaranya, dan seluruh sanak saudaranya, sampai kepada yang paling muda.
\par 13 Kelaparan itu begitu hebat, sehingga di mana-mana tidak ada makanan lagi. Rakyat Mesir dan rakyat Kanaan lemah dan tidak berdaya karena lapar.
\par 14 Setiap kali mereka membeli gandum, Yusuf mengumpulkan uang pembayar gandum itu, lalu disimpannya di istana raja.
\par 15 Setelah habis uang di Mesir dan di Kanaan, orang-orang Mesir datang kepada Yusuf dan berkata, "Berilah kami makanan! Jangan biarkan kami mati. Uang kami sudah habis."
\par 16 Yusuf menjawab, "Jika uangmu sudah habis, berilah ternakmu; aku akan memberi makanan kepadamu."
\par 17 Lalu mereka memberikan ternaknya kepada Yusuf, dan ia memberi makanan kepada mereka sebagai ganti kuda, domba, kambing, sapi dan keledai mereka. Pada tahun itu Yusuf memberi makanan kepada mereka dan mereka membayar dengan ternak.
\par 18 Pada tahun berikutnya mereka datang lagi kepadanya dan berkata, "Kami berterus terang kepada Tuan bahwa uang kami sudah habis dan ternak kami sudah menjadi milik Tuan. Kami tidak punya apa-apa lagi yang dapat kami berikan kepada Tuan, selain diri kami sendiri dan tanah kami.
\par 19 Jangan biarkan kami mati! Jangan biarkan ladang-ladang kami menjadi tandus. Belilah kami dan tanah kami dengan gandum sebagai pembayarannya. Kami dan tanah kami akan menjadi milik raja. Berilah kami gandum untuk menyambung hidup kami dan juga benih untuk ditabur di ladang kami!"
\par 20 Lalu Yusuf membeli semua tanah di mesir untuk raja. Setiap orang Mesir terpaksa menjual tanahnya, karena masa kelaparan itu sangat dahsyat; lalu semua tanah di Mesir menjadi milik raja.
\par 21 Seluruh rakyat Mesir dijadikan hamba oleh Yusuf.
\par 22 Satu-satunya tanah yang tidak dibelinya ialah tanah para imam. Mereka tidak perlu menjual tanah mereka karena diberi tunjangan tetap oleh raja untuk keperluan hidup mereka.
\par 23 Yusuf berkata kepada rakyat itu, "Lihatlah, kamu dan tanahmu sudah kubeli untuk raja. Inilah benih yang dapat kamu tabur di ladangmu.
\par 24 Pada waktu panen, kamu harus memberikan seperlima bagian hasilnya kepada raja. Selebihnya boleh kamu pakai untuk benih dan untuk makanan bagimu dan bagi keluargamu."
\par 25 Jawab mereka, "Tuan telah menyelamatkan kami dan kami berterima kasih. Kami rela menjadi hamba raja."
\par 26 Kemudian Yusuf menjadikan hal itu undang-undang di negeri Mesir, yaitu bahwa seperlima dari hasil panen harus menjadi milik raja. Sampai sekarang undang-undang itu masih berlaku. Hanya tanah imam-imamlah yang tidak menjadi milik raja.
\par 27 Demikianlah orang-orang Israel menetap di Mesir, di daerah Gosyen. Mereka menjadi kaya dan beranak cucu banyak di situ.
\par 28 Yakub masih hidup tujuh belas tahun di Mesir sampai umurnya menjadi seratus empat puluh tujuh tahun.
\par 29 Ketika sudah dekat saatnya akan meninggal, dipanggilnya Yusuf, anaknya itu, lalu berkata kepadanya, "Letakkanlah tanganmu di antara pangkal pahaku dan bersumpah bahwa engkau tidak akan menguburkan aku di Mesir ini.
\par 30 Aku mau dikuburkan di tempat para leluhurku; bawalah mayatku dari negeri ini dan kuburkan dalam makam mereka." Lalu jawab Yusuf, "Saya akan melakukan apa yang Ayah katakan itu."
\par 31 Kata Yakub, "Bersumpahlah bahwa engkau akan melakukannya." Yusuf bersumpah, dan Yakub mengucapkan syukur di tempat tidurnya.

\chapter{48}

\par 1 Beberapa waktu kemudian dikabarkan kepada Yusuf bahwa ayahnya sakit keras. Jadi pergilah ia dengan kedua anaknya, yaitu Manasye dan Efraim mengunjungi Yakub.
\par 2 Ketika Yakub mendengar bahwa Yusuf datang, ia mengumpulkan seluruh tenaganya, lalu duduk di tempat tidur.
\par 3 Kata Yakub kepada Yusuf, "Allah Yang Mahakuasa telah menampakkan diri kepadaku di Lus, di tanah Kanaan, dan telah memberkati aku."
\par 4 Dia berkata kepadaku, "Aku akan memberikan kepadamu banyak anak cucu, supaya keturunanmu menjadi banyak bangsa; Aku akan memberikan negeri ini kepada keturunanmu sebagai milik mereka untuk selama-lamanya."
\par 5 Yakub meneruskan, "Yusuf, kedua anakmu Efraim dan Manasye yang lahir di Mesir ini sebelum aku tiba di sini, kuanggap anakku, sama seperti Ruben dan Simeon.
\par 6 Jika engkau mendapat anak-anak lagi, mereka tidak akan kuanggap sebagai anakku melainkan tetap anakmu; warisan untuk mereka akan mereka terima dari Efraim dan Manasye.
\par 7 Aku melakukan hal itu untuk Rahel ibumu. Dalam perjalananku kembali dari Mesopotamia, ibumu meninggal di tanah Kanaan, tidak jauh dari Efrata, dan aku sangat sedih. Aku menguburkannya di sana, di sisi jalan ke Efrata." (Sekarang Efrata dikenal sebagai Betlehem.)
\par 8 Ketika Yakub melihat anak-anak Yusuf itu, ia bertanya, "Siapa anak-anak ini?"
\par 9 Jawab Yusuf, "Inilah anak-anak saya yang diberikan Allah kepada saya di sini, di Mesir." Yakub berkata, "Dekatkanlah mereka kepadaku supaya kuberkati."
\par 10 Yakub tidak dapat melihat dengan terang karena sudah tua dan matanya sudah kabur. Yusuf mendekatkan anak-anak itu kepada Yakub, dan ia memeluk serta mencium mereka.
\par 11 Lalu berkatalah Yakub kepada Yusuf, "Sama sekali tidak kusangka akan bertemu lagi dengan engkau, tetapi sekarang Allah bahkan mengizinkan aku melihat anak-anakmu juga."
\par 12 Kemudian Yusuf mengambil mereka dari pangkuan Yakub, lalu ia sendiri sujud di hadapan ayahnya.
\par 13 Setelah itu Yusuf menempatkan kedua anaknya di dekat Yakub, Efraim yang bungsu itu di sebelah kiri dan Manasye yang sulung di sebelah kanan. Tetapi Yakub menyilangkan lengannya dan meletakkan tangan kanannya di atas kepala Efraim, sedangkan tangan kirinya diletakkannya di atas kepala Manasye.
\par 15 Sesudah itu diberkatinya Yusuf, katanya, "Allah, Dialah Tuhan, pujaan Abraham dan Ishak, Dialah yang membimbing aku sampai sekarang;
\par 16 Dialah malaikat yang telah melepaskan aku dari segala bahaya. Semoga Allah itu memberkati anak-anak ini. Semoga terus hidup namaku dan nama Abraham dan Ishak oleh karena anak-anak ini! Semoga mereka beranak cucu, dan berketurunan banyak di bumi."
\par 17 Yusuf tidak senang ketika melihat ayahnya meletakkan tangan kanannya di atas kepala Efraim; lalu dipegangnya tangan ayahnya untuk memindahkannya dari atas kepala Efraim ke atas kepala Manasye.
\par 18 Katanya kepada Ayahnya, "Jangan begitu, Ayah. Inilah anak yang sulung; letakkanlah tangan kanan Ayah ke atas kepalanya."
\par 19 Tetapi ayahnya menolak, katanya, "Aku tahu, anakku, aku tahu. Manasye akan besar kuasanya dan keturunannya pun akan menjadi bangsa yang besar. Tetapi adiknya akan lebih besar kuasanya daripada dia, dan keturunan adiknya itu akan menjadi bangsa-bangsa yang besar."
\par 20 Lalu diberkatinya lagi anak-anak itu pada hari itu, katanya, "Orang Israel akan menyebut namamu bilamana mereka memberkati orang. Mereka akan berkata, 'Semoga Allah membuat engkau seperti Efraim dan Manasye.'" Dengan demikian Yakub mendahulukan Efraim daripada Manasye.
\par 21 Lalu berkatalah Yakub kepada Yusuf, "Seperti engkau lihat, ajalku sudah dekat, tetapi Allah akan menolong kalian dan membawa kalian kembali ke negeri leluhurmu.
\par 22 Kepadamulah, dan bukannya kepada saudara-saudaramu, aku berikan Sikhem, daerah subur itu yang telah kurebut dari orang Amori dengan pedang dan panahku."

\chapter{49}

\par 1 Kemudian Yakub memanggil anak-anaknya dan berkata, "Berkumpullah di sini, di dekatku. Akan kuberitahukan kepadamu apa yang akan kalian alami di kemudian hari:
\par 2 Berkumpullah, hai anak-anak Yakub, dengarlah Israel, ayahmu.
\par 3 Ruben, anakku yang sulung, engkaulah kekuatanku, buah pertama keperkasaanku. Engkau tergagah dan terkuat di antara semua anakku.
\par 4 Engkau mudah tergoncang seperti air, sehingga bukan yang utama; sebab engkau tidur dengan selirku, kaunodai tempat tidur ayahmu.
\par 5 Simeon dan Lewi bersaudara; senjata mereka alat kekerasan.
\par 6 Aku takkan ikut dalam permupakatan mereka, juga tidak di dalam perkumpulan mereka. Sebab mereka membunuh dalam kemarahan, melumpuhkan banteng demi kesenangan.
\par 7 Terkutuklah kemarahan mereka, karena dahsyatnya. Terkutuklah keberangan mereka, karena kejamnya. Aku akan menceraiberaikan mereka di seluruh Israel. Aku akan menghamburkan mereka di antara seluruh bangsa.
\par 8 Yehuda, engkau akan menerkam tengkuk musuhmu; saudara-saudaramu akan memujimu dan sujud di hadapanmu.
\par 9 Yehuda bagaikan singa muda; ia membunuh mangsanya, lalu kembali ke sarangnya; ia menggeliat lalu berbaring, dan tak seorang pun berani mengusiknya.
\par 10 Yehuda akan memegang tongkat kerajaan, keturunannya akan memerintah selama-lamanya. Bangsa-bangsa akan membawa upeti, dan sujud dengan takluk di hadapannya.
\par 11 Anak keledainya diikatnya pada pohon, pohon anggur yang paling baik. Dia mencuci pakaiannya dengan anggur, anggur semerah darah.
\par 12 Merah matanya karena minum anggur, putih giginya karena minum susu.
\par 13 Zebulon akan diam di dekat laut; kapal-kapal akan berlabuh di pantainya; sampai Sidon batas wilayahnya.
\par 14 Isakhar itu bagaikan keledai kuat yang berbaring di antara keranjang bebannya.
\par 15 Dilihatnya betapa baiknya tempat itu dan betapa indahnya negeri itu, maka membungkuklah ia rela dibebani, dan dipaksa bekerja sebagai hamba.
\par 16 Engkau Dan, akan mengadili bangsamu, engkau menjadi suku seperti suku lain di Israel.
\par 17 Engkau seperti ular di pinggir jalan, ular berbisa di tepi lorong, yang memagut tumit kuda sampai terlempar penunggangnya.
\par 18 Ya TUHAN, keselamatan yang Kauberikan, itulah yang kunanti-nantikan.
\par 19 Gad, engkau akan diserang perampok, namun engkau akan balik merampok mereka.
\par 20 Asyer, makananmu limpah mewah, kau akan menyediakan makanan bagi raja-raja.
\par 21 Naftali, laksana rusa terlepas, sangat cantik anak-anaknya.
\par 22 Yusuf bagai keledai muda, keledai liar dekat mata air, berlari-lari di lereng gunung.
\par 23 Musuh menyerang dengan sengit, mengejarnya dengan busur dan panah.
\par 24 Namun Yusuf tetap kukuh lengannya, tetap kuat panahnya, karena kuasa Sang Gembala Mahakuat, Allah Yakub, Pelindung Israel.
\par 25 Allah ayahmu menolong engkau; Yang Mahakuasa memberkati engkau dengan hujan dari langit dan air dari bawah tanah. Dengan anak dan sapi,
\par 26 gandum dan bunga. Berkat Allah bapakmu melebihi kekayaan gunung-gunung yang sangat tua. Semoga turunlah semua berkat itu ke atas kepala Yusuf, ke atas dahinya. Dialah yang teristimewa di antara saudara-saudaranya.
\par 27 Benyamin adalah serigala ganas. Di pagi hari ia menerkam mangsanya. Di malam hari ia membagi-bagikannya."
\par 28 Itulah kedua belas suku Israel, dan kata-kata yang diucapkan ayah mereka ketika memberi salam perpisahan kepada mereka masing-masing.
\par 29 Kemudian Yakub berpesan kepada anak-anaknya, "Sebentar lagi aku akan berpulang seperti leluhurku. Kuburkanlah aku nanti di samping mereka dalam gua di Makhpela, sebelah timur Mamre di tanah Kanaan. Abraham telah membeli gua dan ladang di sekitarnya dari Efron orang Het untuk dijadikan pekuburan.
\par 31 Di situ dikuburkan Abraham dan Sara, Ishak dan Ribka; dan di situ juga aku menguburkan Lea.
\par 32 Ladang dan gua yang ada di situ telah dibeli dari orang Het. Kuburkanlah aku di situ."
\par 33 Setelah Yakub selesai berpesan kepada anak-anaknya, berbaringlah ia lalu meninggal.

\chapter{50}

\par 1 Yusuf merebahkan dirinya pada ayahnya sambil menangis dan mencium wajahnya.
\par 2 Lalu Yusuf memerintahkan tukang pengawet mayat untuk merempah-rempahi jenazah Yakub.
\par 3 Pekerjaan itu memakan waktu empat puluh hari, yaitu waktu yang biasanya diperlukan untuk merempah-rempahi mayat. Orang Mesir berkabung bagi Yakub, tujuh puluh hari lamanya.
\par 4 Setelah habis masa berkabung, berkatalah Yusuf kepada pegawai-pegawai raja, "Sampaikanlah pesan ini kepada raja,
\par 5 'Waktu ayah hamba sudah dekat ajalnya, disuruhnya hamba berjanji kepadanya, bahwa hamba akan menguburkan dia di dalam kuburan yang telah disiapkannya di tanah Kanaan. Sebab itu, izinkanlah hamba pergi untuk menguburkan ayah hamba. Setelah itu hamba akan kembali.'"
\par 6 Lalu berkatalah raja, "Pergilah menguburkan ayahmu seperti yang telah kaujanjikan kepadanya."
\par 7 Lalu pergilah Yusuf untuk menguburkan ayahnya. Semua pegawai raja, para tua-tua istana dan semua pembesar Mesir, pergi bersama-sama dengan Yusuf.
\par 8 Keluarganya, saudara-saudaranya, dan sanak saudaranya yang lain pergi juga. Hanya anak-anak kecil dan ternak mereka ditinggalkan di daerah Gosyen.
\par 9 Pasukan yang berkereta dan berkuda juga ikut, sehingga iring-iringan itu panjang sekali.
\par 10 Ketika mereka sampai di tempat orang memukul gandum di Goren-Haatad, di sebelah timur Sungai Yordan, mereka meratap dengan sedih dan suara nyaring. Yusuf mengadakan upacara berkabung, tujuh hari lamanya.
\par 11 Ketika penduduk Kanaan melihat perkabungan di Goren-Haatad itu, berkatalah mereka, "Alangkah pilunya upacara perkabungan orang Mesir itu!" Itulah sebabnya tempat itu dinamakan Abel-Mizraim.
\par 12 Demikianlah anak-anak Yakub melakukan apa yang dipesankan Yakub kepada mereka.
\par 13 Mereka mengangkut jenazahnya ke Kanaan dan menguburkannya dalam gua Makhpela, di sebelah timur Mamre, di ladang yang telah dibeli oleh Abraham dari Efron orang Het itu, untuk dijadikan pekuburan.
\par 14 Setelah Yusuf menguburkan ayahnya, ia kembali ke Mesir bersama-sama dengan saudara-saudaranya dan semua orang yang mengikuti dia untuk penguburan ayahnya itu.
\par 15 Setelah ayah mereka meninggal, abang-abang Yusuf berkata, "Bagaimana seandainya Yusuf dendam dan mau membalas kejahatan kita dahulu?"
\par 16 Sebab itu mereka mengirim pesan ini kepada Yusuf, "Sebelum ayah kita meninggal,
\par 17 ia menyuruh kami mengatakan kepadamu begini, 'Ampunilah kesalahan yang dahulu dilakukan abang-abangmu terhadapmu.' Jadi, sebagai hamba-hamba Allah yang dipuja ayah kita, kami mohon, ampunilah kesalahan yang telah kami lakukan." Yusuf menangis pada waktu menerima pesan itu.
\par 18 Lalu saudara-saudaranya itu sendiri datang dan sujud di hadapannya serta berkata, "Kami ini hambamu."
\par 19 Tetapi Yusuf berkata kepada mereka, "Jangan takut; sebab saya tidak bisa bertindak sebagai Allah.
\par 20 Kalian telah bermupakat untuk berbuat jahat kepada saya, tetapi Allah mengubah kejahatan itu menjadi kebaikan, supaya dengan yang terjadi dahulu itu banyak orang yang hidup sekarang dapat diselamatkan.
\par 21 Jangan khawatir. Saya akan mencukupi kebutuhan kalian dan anak-anak kalian." Demikianlah ia menentramkan hati mereka dengan kata-kata yang ramah, sehingga mereka terharu.
\par 22 Yusuf tetap tinggal di Mesir dengan sanak saudaranya; ia hidup sampai berumur seratus sepuluh tahun.
\par 23 Ia masih sempat melihat anak cucu Efraim. Ia sempat juga mengasuh anak-anak Makhir, yaitu cucu Manasye, sebagai anaknya sendiri.
\par 24 Katanya kepada saudara-saudaranya, "Ajal saya sudah dekat, tetapi Allah pasti akan memelihara kalian dan memimpin kalian keluar dari negeri ini, ke negeri yang telah dijanjikan-Nya dengan sumpah kepada Abraham, Ishak dan Yakub."
\par 25 Lalu Yusuf menyuruh semua sanak saudaranya bersumpah, katanya, "Berjanjilah kepada saya, bahwa apabila Allah memimpin kalian ke negeri itu, kalian akan membawa juga jenazah saya."
\par 26 Kemudian meninggallah Yusuf di Mesir pada usia seratus sepuluh tahun. Mereka merempah-rempahi jenazahnya dan menaruhnya dalam peti mati.


\end{document}