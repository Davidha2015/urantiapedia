\begin{document}

\title{1 Raja-Raja}


\chapter{1}

\par 1 Kini Raja Daud sudah tua sekali. Meskipun ia diselimuti dengan kain tebal, ia tetap kedinginan.
\par 2 Oleh karena itu para pegawainya berkata, "Yang Mulia, sebaiknya kami mencarikan seorang gadis untuk tinggal dengan Baginda dan merawat Baginda. Ia akan tidur bersama Baginda supaya Baginda merasa hangat."
\par 3 Lalu dicarilah di seluruh Israel seorang gadis yang cantik. Maka di Sunem ditemukan seorang gadis yang cantik sekali bernama Abisag. Ia dibawa kepada raja, lalu ia menunggui raja dan merawatnya. Tetapi raja tidak bersetubuh dengan dia.
\par 5 Karena Absalom sudah meninggal, maka Adonia, putra Daud yang kedua menjadi yang tertua. Ibunya bernama Hagit. Adonia adalah orang yang tampan. Ayahnya tidak pernah memarahinya kalau ia berbuat salah. Adonia ingin sekali menjadi raja, maka ia menyediakan untuk dirinya sejumlah kereta perang dan tentara berkuda serta lima puluh pengawal pribadi.
\par 7 Lalu ia pergi berunding dengan Yoab dan Imam Abyatar. Ibu Yoab bernama Zeruya. Yoab dan Imam Abyatar setuju untuk mendukung usaha Adonia.
\par 8 Tetapi Imam Zadok, Benaya anak Yoyada, Nabi Natan, Simei, Rei dan pengawal pribadi Raja Daud tidak memihak kepada Adonia.
\par 9 Pada suatu hari di tempat yang bernama Batu Ular dekat mata air En-Rogel, Adonia mempersembahkan kurban domba, sapi jantan dan anak sapi yang gemuk-gemuk. Ia mengundang putra-putra Raja Daud yang lain, dan pegawai-pegawai istana dari suku Yehuda ke pesta kurban itu.
\par 10 Tetapi Salomo adiknya, dan Nabi Natan serta Benaya dan pengawal pribadi Raja Daud tidak diundangnya.
\par 11 Maka Nabi Natan pergi menemui Batsyeba, ibu Salomo, lalu berkata kepadanya, "Apakah Sri Ratu belum mendengar bahwa Adonia, putra Hagit, sudah mengangkat dirinya menjadi raja, sedangkan Raja Daud tidak mengetahui apa-apa tentang hal itu?
\par 12 Kalau Sri Ratu ingin supaya Salomo dan Sri Ratu sendiri selamat, saya nasihatkan
\par 13 segeralah menghadap Raja Daud, dan mengatakan begini kepadanya: 'Bukankah Baginda sendiri sudah bersumpah kepadaku bahwa putraku Salomo akan menjadi raja menggantikan Baginda? Sekarang, mengapa Adonia yang menjadi raja?'"
\par 14 Kata Natan selanjutnya, "Nanti, sementara Sri Ratu berbicara dengan raja, saya akan masuk untuk menguatkan apa yang telah dikatakan oleh Sri Ratu."
\par 15 Maka pergilah Batsyeba menghadap raja di dalam kamarnya. Raja pada waktu itu sudah sangat tua, dan Abisag, gadis dari Sunem itu, sedang melayaninya.
\par 16 Batsyeba sujud di depan raja, lalu raja bertanya, "Kau ingin apa?"
\par 17 Batsyeba menjawab, "Paduka Yang Mulia, Baginda telah bersumpah kepadaku demi nama TUHAN, Allah Baginda, bahwa putraku Salomo akan menjadi raja menggantikan Baginda.
\par 18 Tetapi sekarang dengan tidak diketahui Baginda, Adonia sudah menjadi raja.
\par 19 Ia sudah mempersembahkan banyak domba, sapi jantan dan anak sapi yang gemuk-gemuk. Ia sudah pula mengundang ke pesta itu putra-putra Baginda dan Imam Abyatar serta Yoab, panglima angkatan perang Baginda. Tetapi, ia tidak mengundang Salomo.
\par 20 Paduka Yang Mulia, seluruh rakyat Israel sekarang sedang menanti-nantikan keputusan dari Baginda tentang siapa yang akan menggantikan Baginda menjadi raja.
\par 21 Kalau Baginda tidak memberikan keputusan itu, pasti segera sesudah Baginda meninggal, aku dan putraku akan diperlakukan sebagai pengkhianat."
\par 22 Sementara Batsyeba masih berbicara, Nabi Natan tiba di istana.
\par 23 Lalu raja diberitahukan tentang kedatangan Natan. Batsyeba keluar, dan Natan masuk, lalu sujud di depan raja.
\par 24 Natan berkata, "Paduka Yang Mulia, apakah Baginda telah mengumumkan bahwa Adonia akan menggantikan Baginda menjadi raja?
\par 25 Hari ini juga ia telah mempersembahkan banyak domba, sapi jantan dan anak sapi yang gemuk-gemuk. Semua putra Baginda telah diundangnya. Juga para panglima angkatan bersenjata Baginda, dan Imam Abyatar. Mereka sekarang sedang berpesta dengan dia dan bersorak-sorak, 'Hidup Raja Adonia!'
\par 26 Tetapi, ia tidak mengundang saya. Imam Zadok dan Benaya serta Salomo pun tidak diundangnya.
\par 27 Apakah Baginda yang menyuruh dia melakukan semuanya ini, tanpa memberitahukan kepada para pegawai Baginda siapa yang akan menggantikan Baginda?"
\par 28 Raja Daud berkata, "Panggil Batsyeba." Maka Nabi Natan keluar dan Batsyeba kembali menghadap raja.
\par 29 Lalu kata Baginda kepadanya, "Memang aku telah berjanji kepadamu, demi nama TUHAN, Allah Israel, bahwa Salomo putramu akan menggantikan aku menjadi raja. Nah, sekarang aku berjanji kepadamu demi TUHAN yang hidup, yang telah melepaskan aku dari segala kesukaranku, bahwa pada hari ini juga aku akan menepati janjiku kepadamu."
\par 31 Maka sujudlah Batsyeba sambil berkata, "Hiduplah Paduka Raja untuk selama-lamanya!"
\par 32 Lalu Raja Daud menyuruh memanggil Zadok, Natan dan Benaya. Setelah mereka datang,
\par 33 berkatalah raja kepada mereka, "Panggillah para perwiraku dan pergilah dengan mereka kepada Salomo putraku. Naikkanlah dia ke atas bagalku sendiri, dan bawalah dia ke mata air Gihon.
\par 34 Di sana Zadok dan Natan harus melantiknya menjadi raja Israel. Setelah itu kalian harus membunyikan trompet dan bersorak, 'Hidup Raja Salomo!'
\par 35 Kemudian iringilah dia kembali ke sini untuk menduduki tahtaku, karena dialah yang telah kupilih menjadi raja menggantikan aku untuk memerintah Israel dan Yehuda."
\par 36 "Baik, Yang Mulia," sahut Benaya, "semoga TUHAN, Allah Baginda, menguatkan perintah Baginda itu.
\par 37 Sebagaimana TUHAN telah menyertai Baginda, semoga Ia pun menyertai Salomo juga. Semoga TUHAN membuat pemerintahannya lebih jaya daripada pemerintahan Baginda."
\par 38 Maka Zadok, Natan, Benaya dan pengawal pribadi raja mempersilakan Salomo naik ke atas bagal raja, lalu mereka mengiringinya ke mata air Gihon.
\par 39 Kemudian Zadok mengambil tempat minyak zaitun yang telah dibawanya dari Kemah TUHAN, lalu ia melantik Salomo dengan memakai minyak itu. Trompet pun dibunyikan dan semua yang hadir di situ bersorak, "Hidup Raja Salomo!"
\par 40 Kemudian mereka semuanya mengiringi dia kembali sambil bersorak-sorak dan membunyikan seruling, sehingga tanah seolah-olah akan terbelah karena keramaian itu.
\par 41 Adonia dan semua tamunya baru saja selesai berpesta, ketika mereka mendengar keramaian itu. Pada waktu Yoab mendengar bunyi trompet, ia bertanya, "Apa yang terjadi di kota sehingga ramai sekali?"
\par 42 Yoab belum lagi selesai berbicara, tiba-tiba datang Yonatan, anak Imam Abyatar, "Mari masuk," kata Adonia. "Engkau orang baik, pasti yang kaubawa, berita yang baik pula."
\par 43 "Maaf, bukan berita baik," jawab Yonatan. "Raja Daud telah mengangkat Salomo menjadi raja!
\par 44 Zadok, Natan, Benaya dan pengawal pribadi raja sudah disuruh mengiringi Salomo. Mereka telah menaikkannya ke atas bagal raja,
\par 45 dan Zadok serta Natan telah melantiknya di mata air Gihon. Kini mereka telah kembali ke kota sambil bersorak-sorak dengan ramai, dan seluruh kota gempar. Itulah keramaian yang kalian dengar tadi.
\par 46 Salomo sekarang sudah menjadi raja.
\par 47 Bahkan para perwira raja telah pergi mengucapkan selamat kepada Raja Daud. Mereka berkata, 'Semoga Allah baginda menjadikan Salomo lebih termasyhur daripada baginda; semoga pemerintahan Salomo lebih jaya daripada pemerintahan baginda.' Kemudian di tempat tidurnya, Raja Daud sujud menyembah Allah,
\par 48 dan berdoa, 'Terpujilah Engkau, ya TUHAN, Allah yang disembah umat Israel. Hari ini seorang dari keturunanku telah Kauangkat menjadi raja menggantikan aku. Dan Engkau telah mengizinkan aku hidup untuk menyaksikannya!'"
\par 49 Para tamu Adonia menjadi takut, sehingga mereka semuanya berdiri lalu pergi, masing-masing mengambil jalannya sendiri.
\par 50 Maka sangatlah takut Adonia kepada Salomo, sehingga ia lari ke Kemah TUHAN dan memegang ujung-ujung mezbah di situ.
\par 51 Orang memberitahukan hal itu kepada Raja Salomo. Mereka memberitahukan bahwa karena Adonia sangat takut kepada Salomo, maka ia telah pergi ke mezbah dan memegang ujung-ujung mezbah itu serta berkata, "Saya tidak akan pergi dari sini sebelum Salomo bersumpah kepada saya bahwa ia tidak akan membunuh saya."
\par 52 Salomo menjawab, "Jika ia berlaku baik, ia tidak akan dihukum sedikit pun; tetapi jika ia berbuat jahat, ia harus dibunuh."
\par 53 Lalu raja menyuruh orang pergi mengambil Adonia dari mezbah itu. Maka datanglah Adonia menghadap raja, dan sujud di depannya. Lalu raja berkata, "Kau boleh pulang!"

\chapter{2}

\par 1 Ketika telah dekat ajalnya, Daud memanggil Salomo, putranya, dan memberikan pesan-pesannya yang terakhir. Daud berkata,
\par 2 "Sudah saatnya aku meninggal dunia. Hendaklah engkau yakin dan berani.
\par 3 Lakukanlah apa yang diperintahkan TUHAN Allahmu kepadamu. Taatilah semua hukum-hukum dan perintah-perintah-Nya yang tertulis dalam Buku Musa, supaya ke mana pun engkau pergi engkau akan berhasil dalam segala usahamu.
\par 4 Kalau engkau taat kepada TUHAN, Ia akan menepati janji-Nya bahwa keturunanku akan memerintah Israel, asal mereka dengan sepenuh hati dan segenap jiwa selalu setia mentaati perintah-perintah-Nya.
\par 5 Ada lagi satu hal! Kau pasti ingat apa yang telah dilakukan Yoab kepadaku dengan membunuh kedua perwira Israel, yaitu Abner anak Ner dan Amasa anak Yeter. Ia membunuh mereka pada masa damai untuk membalas pembunuhan yang mereka lakukan pada masa perang. Ia membunuh orang yang tidak bersalah tapi sekarang akulah yang harus menanggung perbuatannya dan akulah yang menderita.
\par 6 Nah, anakku, kau harus bertindak terhadap dia menurut apa yang kauanggap baik; hanya jangan biarkan dia meninggal secara wajar.
\par 7 Tetapi terhadap anak-anak lelaki Barzilai dari Gilead hendaklah engkau berlaku baik. Kau harus menjamin kebutuhan mereka sehari-hari, sebab mereka telah berbuat baik kepadaku ketika aku melarikan diri dari abangmu, Absalom.
\par 8 Mengenai Simei anak Gera dari kota Bahurim di wilayah Benyamin, ada pula pesanku kepadamu. Dalam perjalananku ke Mahanaim, Simei itu dahulu mengutuki aku dengan kejam. Tetapi kemudian, ketika ia datang menjemput aku di Sungai Yordan, aku bersumpah kepadanya demi TUHAN bahwa aku tidak akan membunuhnya.
\par 9 Tetapi inilah pesanku kepadamu: jangan biarkan dia bebas dari hukuman. Aku tahu engkau bijaksana; jadi, meskipun ia sudah tua, usahakanlah supaya ia dihukum mati."
\par 10 Kemudian Daud meninggal dan dimakamkan di Kota Daud.
\par 11 Ia menjadi raja Israel selama empat puluh tahun: tujuh tahun ia berkedudukan di Hebron, dan tiga puluh tiga tahun di Yerusalem.
\par 12 Kemudian Salomo menjadi raja menggantikan Daud, ayahnya. Kekuasaan Salomo sebagai raja sangat kukuh.
\par 13 Pada suatu hari, Adonia pergi kepada Batsyeba. Bertanyalah Batsyeba kepada Adonia, "Apakah kunjunganmu ini dengan maksud baik?" "Ya," jawab Adonia.
\par 14 Lalu ia menambahkan, "Ada sesuatu yang saya ingin minta dari ibu." "Apa?" tanya Batsyeba.
\par 15 Adonia menjawab, "Ibu tahu bahwa sayalah yang seharusnya menjadi raja; itu sudah diharapkan oleh seluruh Israel. Tetapi yang terjadi adalah yang sebaliknya; adikku yang menjadi raja, karena TUHAN menghendaki demikian.
\par 16 Sekarang ada satu permintaan saya, kiranya Ibu jangan menolaknya."
\par 17 "Apa permintaanmu?" tanya Batsyeba. Adonia menjawab, "Saya mohon Ibu mau meminta kepada Raja Salomo untuk mengizinkan saya mengawini Abisag, gadis dari Sunem itu. Saya yakin Salomo tidak akan menolak permintaan Ibu."
\par 18 "Ya, baiklah," jawab Batsyeba. "Saya akan membicarakan hal itu dengan raja untuk engkau."
\par 19 Maka pergilah Batsyeba kepada raja dengan maksud membicarakan hal itu untuk Adonia. Raja berdiri menyambut ibunya dan sujud kepadanya, lalu duduk lagi di atas tahtanya. Kemudian ia menyuruh orang menyediakan kursi di sebelah kanannya untuk ibunya.
\par 20 Setelah duduk, berkatalah Batsyeba, "Ibu ingin meminta sesuatu yang kecil saja daripadamu. Ibu harap kau tidak akan menolaknya." "Apa itu, Ibu?" tanya Salomo. "Saya tidak akan menolak."
\par 21 Batsyeba menjawab, "Izinkanlah Adonia, abangmu itu mengawini Abisag."
\par 22 "Mengapa Ibu meminta hal itu kepada saya?" tanya Salomo. "Itu sama saja dengan meminta supaya saya memberikan tahta kerajaan ini kepadanya. Bukankah dia abang saya? Lagi pula Imam Abyatar dan Yoab memihak dia!"
\par 23 Setelah berkata begitu bersumpahlah Salomo demi TUHAN, "Biarlah saya terkena kutukan TUHAN, kalau saya tidak menghukum mati Adonia karena permintaannya itu!
\par 24 TUHAN telah mengukuhkan kedudukan saya pada tahta ayah saya, Daud. TUHAN telah menepati janji-Nya dan telah memberikan kerajaan ini kepada saya dan keturunan saya. Saya bersumpah demi TUHAN yang hidup bahwa Adonia akan mati hari ini juga!"
\par 25 Maka Raja Salomo memberi perintah kepada Benaya, lalu Benaya pergi membunuh Adonia.
\par 26 Kemudian berkatalah Raja Salomo kepada Imam Abyatar, "Pergilah ke kampung halamanmu di Anatot. Kau patut dihukum mati, tapi hari ini aku tak akan membunuhmu, karena ketika kau masih bersama ayahku, kau ikut merasakan segala penderitaannya. Dan kau juga yang mengurus Peti Perjanjian TUHAN."
\par 27 Lalu Salomo memecat Abyatar sebagai imam. Dengan perbuatan Salomo itu apa yang telah dikatakan TUHAN di Silo tentang keturunan Imam Eli telah menjadi kenyataan.
\par 28 Yoab mendengar tentang semua yang telah terjadi itu. Ia menjadi takut karena ia memihak kepada Adonia. Tapi, dahulu ia tidak memihak kepada Absalom. Maka ia lari ke Kemah TUHAN dan memegang ujung-ujung mezbah.
\par 29 Berita ini disampaikan kepada Raja Salomo. Maka raja menyuruh Benaya membunuh Yoab.
\par 30 Benaya pergi ke Kemah TUHAN dan berkata kepada Yoab, "Raja menyuruh engkau keluar dari situ!" "Tidak," jawab Yoab. "Saya mau mati di sini." Benaya kembali dan melaporkan hal itu kepada raja.
\par 31 Mendengar itu raja berkata, "Pergilah dan laksanakanlah apa yang dikatakannya itu. Bunuh dia dan kuburkan dia, supaya aku dan seluruh keturunan ayahku tidak bertanggung jawab lagi atas pembunuhan yang dilakukan Yoab terhadap orang yang tidak bersalah.
\par 32 TUHAN akan menghukum dia karena pembunuhan yang telah dilakukannya tanpa setahu ayahku. Abner perwira Israel, dan Amasa perwira Yehuda, kedua-duanya tidak bersalah. Mereka lebih baik daripada Yoab sendiri, tetapi ia membunuh mereka.
\par 33 Hukuman karena kematian kedua perwira itu akan ditanggung oleh Yoab dan keturunannya untuk selama-lamanya. Tetapi seluruh keturunan ayahku yang menjadi raja, selalu akan diberkati TUHAN--mereka akan beruntung."
\par 34 Maka pergilah Benaya ke Kemah TUHAN dan membunuh Yoab di situ. Lalu Yoab dikuburkan di rumahnya di luar kota.
\par 35 Kemudian raja mengangkat Benaya menjadi panglima angkatan bersenjata menggantikan Yoab, dan Zadok diangkat menjadi imam menggantikan Abyatar.
\par 36 Sesudah itu raja memanggil Simei lalu berkata kepadanya, "Buatlah rumah untuk dirimu di Yerusalem ini. Tinggallah di situ dan jangan keluar dari kota.
\par 37 Kalau kau berani keluar dan pergi melewati anak Sungai Kidron, kau harus dibunuh; dan kau sendirilah yang menanggung kesalahan itu."
\par 38 "Baik, Paduka Yang Mulia," jawab Simei. "Saya akan menuruti perintah Baginda." Maka tinggallah Simei di Yerusalem beberapa waktu lamanya.
\par 39 Setelah tiga tahun terjadilah hal berikut ini: Dua orang hamba Simei lari ke Gat, kepada Raja Akhis putra Maakha. Ketika Simei diberitahu tentang hal itu, ia
\par 40 memasang pelana pada keledainya lalu pergi kepada Raja Akhis di Gat untuk mencari kedua hambanya itu. Ia menemukan mereka lalu membawa mereka pulang.
\par 41 Ketika Salomo mendengar apa yang telah dilakukan oleh Simei,
\par 42 ia menyuruh memanggil Simei lalu berkata, "Aku sudah membuat engkau berjanji demi TUHAN bahwa engkau tidak akan meninggalkan Yerusalem. Dan aku juga sudah memperingatkan engkau bahwa apabila engkau keluar dari Yerusalem, engkau harus mati. Nah, engkau sudah berjanji dan setuju untuk mentaati perintahku itu, bukan?
\par 43 Tetapi sekarang mengapa kauingkari janjimu itu dan kaulanggar perintahku?
\par 44 Kau tahu betul semua kejahatanmu terhadap ayahku. Untuk itu TUHAN akan menghukummu,
\par 45 tetapi aku akan diberkati-Nya dan kerajaan ayahku akan dikukuhkan-Nya sampai selama-lamanya."
\par 46 Setelah berkata begitu, raja memberi perintah kepada Benaya lalu pergilah Benaya membunuh Simei. Kini Salomo berkuasa penuh di dalam kerajaannya.

\chapter{3}

\par 1 Raja Salomo menjadikan Mesir sekutunya dengan mengawini putri raja Mesir. Putri itu dibawa oleh Salomo ke kota Daud untuk tinggal di situ sampai istana raja dan Rumah TUHAN serta benteng sekeliling Yerusalem selesai dibangun.
\par 2 Karena Rumah TUHAN belum dibangun pada waktu itu, rakyat masih mempersembahkan kurban di berbagai-bagai tempat.
\par 3 Salomo mengasihi TUHAN dan mengikuti petunjuk-petunjuk dari ayahnya, tetapi ia masih menyembelih dan mempersembahkan binatang di berbagai tempat.
\par 4 Tempat ibadat yang paling terkenal terdapat di Gibeon. Salomo sering mempersembahkan kurban di sana. Pada suatu hari, seperti biasa, ia pergi ke sana lagi.
\par 5 Malam itu TUHAN menampakkan diri kepadanya dalam mimpi dan berkata, "Salomo, mintalah apa yang kauinginkan, itu akan Kuberikan kepadamu!"
\par 6 Salomo menjawab, "TUHAN, Engkau sudah menunjukkan bahwa Engkau sangat mengasihi ayahku Daud, hamba-Mu itu. Ia baik, jujur serta setia kepada-Mu. Dan sebagai jaminan bahwa untuk seterusnya Engkau tetap mengasihi dia, Engkau memberikan kepadanya seorang putra yang sekarang memerintah menggantikan dia.
\par 7 Ya TUHAN, Allahku, Engkau mengangkat aku menjadi raja menggantikan ayahku, meskipun aku masih sangat muda dan tak tahu bagaimana caranya memerintah.
\par 8 Aku berada di tengah-tengah bangsa yang telah Kaupilih menjadi umat-Mu sendiri. Mereka suatu bangsa besar yang tak terhitung jumlahnya.
\par 9 Sebab itu, TUHAN, berikanlah kiranya kepadaku kebijaksanaan yang kuperlukan untuk memerintah umat-Mu ini dengan adil dan untuk dapat membedakan mana yang baik dan mana yang jahat. Kalau tidak demikian, mana mungkin aku dapat memerintah umat-Mu yang besar ini?"
\par 10 Tuhan senang dengan permintaan Salomo itu
\par 11 dan berkata kepadanya, "Sebab engkau meminta kebijaksanaan untuk memerintah dengan adil, dan bukan umur panjang atau kekayaan untuk dirimu sendiri ataupun kematian musuh-musuhmu,
\par 12 maka permintaanmu itu akan Kupenuhi. Aku akan menjadikan engkau lebih bijaksana dan lebih berpengetahuan daripada siapa pun juga, baik pada masa lalu maupun pada masa yang akan datang.
\par 13 Bahkan apa yang tidak kauminta pun akan Kuberikan juga kepadamu: seumur hidupmu kau akan kaya dan dihormati melebihi raja lain yang mana pun juga.
\par 14 Dan kalau kau, seperti ayahmu Daud, mentaati Aku serta mengikuti hukum-hukum-Ku, Aku akan memberikan kepadamu umur yang panjang."
\par 15 Setelah bangun, Salomo menyadari bahwa Allah telah berbicara kepadanya di dalam mimpi. Maka kembalilah ia ke Yerusalem lalu berdiri di depan Peti Perjanjian TUHAN dan mempersembahkan kurban bakaran dan kurban perdamaian. Kemudian ia mengadakan pesta untuk semua pegawainya.
\par 16 Pada suatu hari, dua orang wanita pelacur datang menghadap Raja Salomo.
\par 17 Salah seorang dari mereka berkata, "Paduka Yang Mulia, saya dengan wanita ini tinggal serumah. Tidak ada orang lain yang tinggal bersama kami. Beberapa waktu yang lalu, saya melahirkan seorang anak laki-laki. Dua hari kemudian wanita ini pun melahirkan seorang anak laki-laki pula.
\par 19 Pada suatu malam ketika kami sedang tidur, wanita ini tanpa sengaja menindih bayinya sampai mati.
\par 20 Tengah malam sementara saya tidur, ia bangun lalu mengambil bayi saya dari sisi saya. Bayi saya itu ditaruhnya di tempat tidurnya, dan bayinya yang sudah mati itu ditaruhnya di tempat tidur saya.
\par 21 Besok paginya, ketika saya bangun dan hendak menyusui bayi saya itu, saya dapati ia telah mati. Setelah saya mengamat-amatinya, nyatalah ia bukan bayi saya."
\par 22 "Bohong!" kata wanita yang lain itu, "Yang hidup ini bayiku, yang mati itu bayimu!" "Tidak!" jawab wanita yang pertama itu. "Yang mati ini bayimu, yang hidup itu bayiku!" Begitulah mereka bertengkar di depan raja.
\par 23 Lalu kata Raja Salomo, "Kamu masing-masing mengaku bahwa bayi yang hidup ini anakmu dan bukan yang mati itu."
\par 24 Setelah berkata demikian raja menyuruh orang mengambil pedang.
\par 25 Kemudian raja memerintahkan, "Potonglah bayi yang hidup itu menjadi dua dan berikanlah satu potong kepada masing-masing wanita itu."
\par 26 Mendengar perintah itu, ibu yang sebenarnya dari anak itu berkata, "Ampun, Baginda, jangan bunuh anak itu. Berikan saja kepada dia." Ibu itu berkata begitu karena ia sangat mencintai anaknya. Tetapi wanita yang lain itu berkata, "Ya, potong saja, biar tak seorang pun dari kami yang mendapatnya!"
\par 27 Maka berkatalah Salomo, "Jangan bunuh bayi itu! Serahkan kepada wanita yang pertama itu--dialah ibunya."
\par 28 Ketika bangsa Israel mendengar tentang keputusan Salomo dalam perkara tersebut, mereka merasa kagum dan hormat kepadanya. Sebab, nyatalah bahwa Allah telah memberikan kepadanya hikmat untuk berlaku adil.

\chapter{4}

\par 1 Salomo telah menjadi raja atas seluruh negeri Israel.
\par 2 Inilah pejabat-pejabat tinggi yang diangkat oleh Salomo: Imam-imam: Zadok, Azarya anak Zadok, Abyatar. Sekretaris negara: Elihoref dan Ahia, yaitu anak-anak Sisia. Bendahara negara: Yosafat anak Ahilud. Panglima angkatan bersenjata: Benaya, anak Yoyada. Pengawas para bupati: Azarya, anak Natan. Penasehat raja: Imam Zabud, anak Natan. Kepala rumah tangga istana: Ahisar. Kepala pekerja rodi: Adoniram, anak Abda.
\par 7 Salomo mengangkat dua belas bupati di seluruh negeri Israel. Tugas mereka ialah mengumpulkan dari wilayah mereka bahan makanan untuk raja dan seisi istananya. Setiap bupati secara bergilir bertanggung jawab atas hal itu selama satu bulan setiap tahun.
\par 8 Kedua belas bupati itu dikepalai oleh seorang gubernur. Inilah nama para bupati itu dan wilayah kekuasaan mereka: Ben-Hur: daerah pegunungan Efraim. Ben-Deker: kota Makas, Saalbim, Bet-Semes, Elon, Bet-Hanan. Ben-Hesed: kota Arubot, Sokho, dan seluruh daerah Hefer. Ben-Abinadab, suami Tafat putri Salomo: seluruh daerah Dor. Baana anak Ahilud: kota Taanakh, Megido, dan seluruh daerah dekat Bet-Sean, dekat kota Sartan di sebelah selatan kota Yizreel sampai sejauh kota Abel-Mehola dan kota Yokmeam. Ben-Geber: kota Ramot di Gilead, kampung-kampung di Gilead milik kaum Yair keturunan Manasye, daerah Argob di Basan; seluruhnya 60 kota besar yang diperkuat dengan benteng dan palang-palang perunggu pada gerbang-gerbangnya. Ahinadab anak Ido: daerah Mahanaim. Ahimaas, suami Basmat putri Salomo: daerah Naftali. Baana anak Husai: daerah Asyer, kota Alot. Yosafat anak Paruah: daerah Isakhar. Simei anak Ela: daerah Benyamin. Geber anak Uri: daerah Gilead, yang dahulu dikuasai Sihon raja Amori, dan Og raja Basan.
\par 20 Rakyat Yehuda dan Israel banyaknya seperti pasir di tepi laut. Mereka mempunyai cukup makanan dan minuman serta hidup senang.
\par 21 Setiap keluarga mempunyai kebun anggur dan kebun ara. Seumur hidup Salomo rakyat di seluruh Yehuda dan Israel dari Dan sampai ke Bersyeba hidup aman dan tentram. Sebab, semua raja di sebelah barat Efrat takluk kepadanya, dan semua bangsa dari negara-negara di sekitar kerajaannya bersahabat dengan dia. Tanah di sebelah barat Sungai Efrat dari Tifsah di tepi Sungai Efrat terus ke barat ke kota Gaza di Filistin sampai ke perbatasan Mesir, seluruhnya dikuasai oleh Salomo. Bangsa-bangsa di negeri itu takluk kepadanya dan membayar upeti. Bahan-bahan makanan yang diperlukan oleh Salomo setiap hari adalah 5.000 liter tepung halus, 10.000 liter tepung kasar, 10 sapi kandang, 20 sapi padang, dan 100 domba. Selain itu, juga rusa, kijang, menjangan dan unggas.
\par 26 Salomo mempunyai 12.000 tentara berkuda dan 40.000 kandang untuk kuda-kuda yang menarik kereta-keretanya.
\par 27 Secara bergilir, selama sebulan, setiap bupati menyerahkan bahan makanan untuk Raja Salomo dan untuk semua orang yang mendapat makanan dari istana, sehingga mereka tidak kekurangan sesuatu pun.
\par 28 Setiap bupati itu memberikan juga gandum dan jerami untuk semua kuda Salomo, baik kuda tunggangan maupun kuda kereta. Makanan hewan itu dibawa ke kandang-kandang kuda sebanyak yang dibutuhkan.
\par 29 Allah memberikan kepada Salomo hikmat, dan pengetahuan yang luar biasa, serta pengertian yang amat dalam.
\par 30 Di negeri-negeri timur dan di Mesir tak ada orang yang lebih pandai dari dia.
\par 31 Ia benar-benar terpandai dari semua orang. Ia melebihi Etan orang Ezrahi, Heman, Kalkol dan Darda anak-anak Mahol. Nama Salomo terkenal di mana-mana di negeri-negeri sekeliling kerajaannya.
\par 32 Ada 3.000 pepatah dan lebih dari seribu nyanyian yang dikarangnya.
\par 33 Ia sanggup berbicara tentang pohon-pohon dan tanaman-tanaman--dari pohon cemara Libanon sampai kepada rerumput hisop yang tumbuh pada dinding-dinding batu. Ia berbicara juga mengenai binatang-binatang, seperti misalnya unggas, binatang melata dan ikan.
\par 34 Raja-raja di seluruh dunia mendengar tentang kepandaian Salomo sehingga mereka mengutus orang kepadanya untuk mendengarkan dia.

\chapter{5}

\par 1 Hiram, raja Tirus, selalu bersahabat dengan Raja Daud. Ketika Hiram mendengar bahwa Salomo telah menjadi raja menggantikan Daud, ayahnya, ia mengirim utusan kepada Salomo.
\par 2 Lalu Salomo pun mengirim pesan ini kepada Hiram,
\par 3 "Raja Hiram Yang Mulia! Tentu Tuan mengetahui bahwa ayahku Daud selalu terpaksa berperang karena diserang musuh dari negeri-negeri di sekitar kami. Dan karena TUHAN belum memberikan kemenangan kepadanya atas semua musuhnya, ia tidak sempat mendirikan gedung tempat ibadat kepada TUHAN Allahnya.
\par 4 Tetapi sekarang TUHAN Allahku telah memberikan ketentraman di seluruh wilayah kekuasaanku. Aku tak mempunyai musuh, dan tak ada pula musibah yang menimpa negeriku.
\par 5 TUHAN telah berjanji begini kepada ayahku Daud, 'Putramu yang akan Kuangkat menjadi raja menggantikan engkau akan membangun gedung tempat ibadat kepada-Ku.' Karena itu, Tuan Hiram Yang Mulia, aku, Salomo, sudah memutuskan untuk membangun gedung itu.
\par 6 Tuan telah maklum bahwa di antara rakyatku tak ada yang pandai menebang pohon seperti rakyat Tuan. Sebab itu, aku mohon sudilah Tuan menyuruh orang-orang Tuan menebang pohon-pohon cemara Libanon untuk aku. Rakyatku akan kusuruh membantu mereka, dan berapa saja upah yang Tuan tentukan untuk orang-orang Tuan itu, aku akan membayarnya."
\par 7 Hiram senang sekali ketika menerima berita itu. Ia berkata, "Pujilah TUHAN hari ini, karena Ia memberikan kepada Daud seorang putra yang sangat bijaksana untuk menjadi raja atas bangsa yang besar itu."
\par 8 Lalu Hiram mengirim jawaban ini kepada Salomo, "Aku sudah menerima berita permintaan Tuan, dan aku siap memenuhi permintaan Tuan. Aku akan menyediakan kayu cemara Libanon dan kayu cemara biasa sesuai permintaan Tuan.
\par 9 Orang-orangku akan membawa kayu itu turun dari Libanon ke laut. Di situ batang-batang kayu itu akan mereka ikat menjadi rakit dan akan mereka hanyutkan menyusur pantai ke tempat yang Tuan tentukan. Di sana rakit-rakit itu akan dibongkar dan diserahkan kepada orang-orang Tuan untuk diurus selanjutnya. Dari pihak Tuan, aku minta agar Tuan menyediakan makanan untuk orang-orang seisi istanaku."
\par 10 Maka Hiram memberikan kepada Salomo semua kayu cemara Libanon dan kayu cemara biasa yang diperlukannya.
\par 11 Dan untuk memberi makan orang-orang seisi istana Hiram itu, setiap tahun Salomo menyerahkan kepada Hiram 322.500 kilogram gandum dan 4.400 liter minyak zaitun asli.
\par 12 TUHAN menepati janji-Nya dan memberikan hikmat kepada Salomo. Antara Hiram dan Salomo ada hubungan yang baik, dan mereka berdua telah membuat ikatan perjanjian.
\par 13 Raja Salomo mengerahkan 30.000 orang laki-laki dari seluruh Israel untuk pekerjaan rodi,
\par 14 dan mengangkat Adoniram menjadi pengawas mereka. Salomo membagi mereka dalam tiga kelompok, masing-masing terdiri dari sepuluh ribu orang. Ketiga kelompok itu secara bergilir tinggal sebulan di Libanon dan dua bulan di rumah.
\par 15 Salomo mengerahkan juga 80.000 orang untuk memahat batu dari gunung dan 70.000 orang untuk memikul batu-batu itu.
\par 16 Ia mengangkat 3.300 mandur untuk mengawasi pekerjaan itu.
\par 17 Atas perintahnya juga mereka memahat batu-batu besar yang bagus-bagus untuk pondasi Rumah TUHAN.
\par 18 Demikianlah para pekerja Salomo dan para pekerja Hiram serta orang-orang dari kota Gebal menyiapkan batu dan kayu untuk pembangunan Rumah TUHAN itu.

\chapter{6}

\par 1 Salomo mulai membangun Rumah TUHAN pada bulan Ziw, yaitu bulan kedua, dalam tahun keempat pemerintahannya, 480 tahun setelah bangsa Israel meninggalkan Mesir.
\par 2 Ukuran Rumah TUHAN, bagian dalam adalah sebagai berikut: panjang 27 meter, lebar 9 meter, dan tinggi 13,5 meter.
\par 3 Di bagian depan Rumah TUHAN itu ada balai yang panjangnya 4,5 meter dan lebarnya 9 meter, selebar rumah itu juga.
\par 4 Pada tembok rumah itu terdapat lubang-lubang jendela yang melebar ke dalam.
\par 5 Sekeliling bagian belakang rumah itu, pada tembok sebelah luarnya dibangun kamar-kamar bertingkat. Ada tiga tingkatnya, masing-masing setinggi 2,2 meter.
\par 6 Pada tingkat pertama, lebar setiap kamarnya 2,2 meter, pada tingkat kedua 2,7 meter dan pada tingkat ketiga 3,1 meter. Tembok tingkat bawah lebih tebal dari tembok tingkat yang di atasnya, sehingga balok-balok kamar tambahan itu dapat dihubungkan dengan tembok Rumah TUHAN itu tanpa melubangi tembok itu.
\par 7 Untuk balok-balok itu dipakai kayu cemara Libanon. Pintu ke kamar-kamar tambahan di tingkat bawah berada di sebelah selatan, dan untuk ke tingkat dua dan tiga ada tangga pilin. Kemudian Salomo memasang langit-langit rumah itu. Untuk itu ia memakai kasau-kasau dan papan dari kayu cemara Libanon. Selama Rumah TUHAN itu dibangun, sama sekali tidak terdengar bunyi palu, kapak atau perkakas besi, sebab batu-batu yang dipakai untuk pembangunan itu telah disiapkan terlebih dahulu ketika masih di tambang-tambangnya. Demikianlah Salomo membangun Rumah TUHAN itu sampai selesai.
\par 11 Maka berkatalah TUHAN kepada Salomo,
\par 12 "Kalau engkau mentaati semua hukum dan perintah-Ku, Aku akan melakukan untukmu apa yang telah Kujanjikan kepada ayahmu Daud.
\par 13 Aku akan tinggal di dalam rumah yang kaubangun ini di tengah-tengah umat-Ku Israel. Aku tidak akan meninggalkan mereka."
\par 14 Beginilah Salomo menyelesaikan pembangunan Rumah TUHAN itu:
\par 15 Seluruh tembok bagian dalam dilapisinya dengan kayu cemara Libanon, mulai dari lantai sampai ke langit-langit. Lantainya dibuat dari kayu cemara biasa.
\par 16 Di dalam rumah itu, di bagian belakang dibuat kamar dengan memasang dinding pemisah dari kayu cemara, mulai dari lantai sampai ke langit-langit. Kamar itu dinamakan Ruang Mahasuci; panjangnya sembilan meter.
\par 17 Ruang di depan Ruang Mahasuci itu disebut Ruang Besar, dan panjangnya 18 meter.
\par 18 Tembok bagian dalam seluruh rumah itu dilapisi dengan kayu cemara Libanon, sehingga batu-batu temboknya tidak kelihatan sama sekali. Lapisan itu dihiasi dengan ukiran buah labu dan bunga-bunga mekar.
\par 19 Ruang Mahasuci yang dibuat di dalam Rumah TUHAN itu disediakan untuk Peti Perjanjian TUHAN.
\par 20 Ruangan itu panjangnya, lebarnya dan tingginya sama, yaitu sembilan meter. Mezbah di depan Ruang Mahasuci itu dibuat dari kayu cemara Libanon. Pada pintu masuk ke ruangan itu direntangkan rantai emas. Seluruh bagian dalam Rumah TUHAN itu termasuk Ruang Mahasuci dan mezbahnya dilapisi dengan emas murni.
\par 23 Di dalam Ruang Mahasuci itu ditaruh juga dua patung kerub yang dibuat dari kayu zaitun. Kedua patung itu masing-masing tingginya 4,4 meter;
\par 24 ukuran dan bentuknya sama. Masing-masing mempunyai dua sayap, setiap sayap itu 2,2 meter panjangnya. Jarak antara kedua ujung sayap dari masing-masing patung 4,4 meter.
\par 27 Kedua patung itu diletakkan berdampingan di dalam Ruang Mahasuci sehingga satu sayap dari setiap patung itu menyentuh dinding, dan sayap yang lainnya saling menyentuh di tengah-tengah ruangan itu.
\par 28 Kedua patung kerub itu dilapisi dengan emas.
\par 29 Seluruh dinding bagian dalam Rumah TUHAN dihiasi dengan ukiran kerub-kerub, pohon-pohon palem, dan bunga-bunga mekar.
\par 30 Lantai di dalam Rumah TUHAN itu seluruhnya dilapisi dengan emas.
\par 31 Pada pintu masuk ke dalam Ruang Mahasuci itu dipasang dua daun pintu dari kayu zaitun. Ambang pintu itu bagian atas berbentuk lengkungan yang tajam di tengah-tengahnya.
\par 32 Kedua daun pintu itu dihiasi dengan ukiran kerub, pohon palem dan bunga-bunga mekar. Daun pintu, patung kerub dan ukiran pohon-pohon palem, semuanya dilapisi dengan emas.
\par 33 Pada pintu masuk ke dalam ruangan muka, yaitu ruang besar, dipasang bingkai pintu dari kayu zaitun, berbentuk empat persegi panjang.
\par 34 Juga dibuat dua daun pintu dari kayu cemara biasa, yang masing-masing dapat dilipat.
\par 35 Daun-daun pintu itu dihiasi dengan ukiran kerub, pohon palem dan bunga mekar. Semua ukiran itu berlapis emas.
\par 36 Di depan Rumah TUHAN itu dibuat pelataran dalam. Pelataran itu dikelilingi tembok yang berlapis-lapis: di atas setiap tiga lapis batu, ada selapis kayu cemara Libanon.
\par 37 Pondasi Rumah TUHAN itu dipasang pada bulan Ziw, yaitu bulan kedua, dalam tahun keempat pemerintahan Salomo.
\par 38 Dan pada bulan Bul, yaitu bulan kedelapan, dalam tahun kesebelas pemerintahannya, selesailah Rumah TUHAN itu dibangun oleh Salomo, tepat seperti yang sudah direncanakan. Pembangunan rumah itu makan waktu tujuh tahun.

\chapter{7}

\par 1 Salomo membangun juga istana untuk dirinya. Pembangunan istana itu makan waktu tiga belas tahun.
\par 2 Di istana itu Salomo membuat ruangan yang disebut "Balai Hutan Libanon". Ukurannya: panjang 44 meter, lebar 22 meter, dan tinggi 13,5 meter. Balai itu mempunyai tiga jajar tiang dari kayu cemara Libanon; tiap jajar terdiri dari lima belas tiang dan di atasnya dipasang kasau-kasau dari kayu cemara Libanon. Langit-langitnya terbuat dari kayu yang sama dan menutupi kasau-kasau itu.
\par 4 Pada dinding kiri dan kanan balai itu terdapat jendela-jendela: tiga deretan pada dinding yang satu berhadapan dengan tiga deretan pada dinding yang lainnya. Bingkai pintu-pintu dan jendela-jendelanya berbentuk empat persegi panjang.
\par 6 Salomo juga membangun "Balai Pilar". Panjang balai itu 22 meter dan lebarnya 13,5 meter. Di depan balai itu ada serambi yang berpilar dan bertangga.
\par 7 Balai yang lain ialah "Balai Tahta" yang disebut juga Balai Pengadilan. Balai ini dari lantai sampai ke kasau-kasaunya dilapisi dengan kayu cemara Libanon. Di balai inilah Salomo memutuskan perkara-perkara pengadilan.
\par 8 Tempat tinggal Salomo dibuat seperti gedung-gedung lainnya dan terdapat di pelataran lain di belakang Balai Pengadilan. Untuk istrinya, yaitu putri raja Mesir, Salomo membangun juga gedung yang serupa.
\par 9 Semua bangunan itu, termasuk pelataran besarnya, dari pondasinya sampai ke atapnya, dibuat dari batu yang bagus-bagus. Batu-batu itu disiapkan di tambang batu, dan dipotong menurut ukuran; bagian dalam dan bagian luarnya diratakan dengan gergaji.
\par 10 Pondasi gedung-gedung itu dibuat dari batu-batu besar yang berharga. Batu-batu itu ada yang 3,5 meter panjangnya, ada pula yang empat meter.
\par 11 Di atasnya dipasang batu-batu lain yang berharga dan dipotong menurut ukuran, lalu ditutup dengan kayu cemara Libanon.
\par 12 Tembok pelataran istana dan tembok pelataran dalam di Rumah TUHAN serta tembok balai di bagian depan Rumah TUHAN, masing-masing terdiri dari satu lapisan kayu cemara Libanon di atas tiap tiga lapisan batu.
\par 13 Di kota Tirus ada seorang laki-laki bernama Huram. Ia benar-benar ahli dan banyak berpengalaman dalam membuat segala macam barang perunggu. Almarhum ayahnya adalah orang Tirus dan juga ahli dalam barang-barang perunggu. Ibunya dari suku Naftali. Raja Salomo telah menyuruh orang pergi menjemput Huram untuk datang dan mengepalai seluruh pekerjaan pembuatan barang perunggu. Maka datanglah Huram dan mengerjakan apa yang diminta oleh Salomo.
\par 15 Huram membuat dua tiang perunggu masing-masing tingginya 8 meter, lingkarnya 5,3 meter, dan tebal perunggunya 7,4 sentimeter. Tiang-tiang itu kosong di dalamnya.
\par 16 Untuk tiang-tiang itu ia membuat dua kepala tiang dari perunggu, masing-masing tingginya 2,2 meter
\par 17 berbentuk bunga bakung setinggi 1,8 meter. Setiap kepala tiang itu dihias dengan dua jajar buah delima dari perunggu. Semuanya ada dua ratus buah delima pada setiap kepala tiang. Lalu Huram membuat anyaman rantai dari perunggu dan menaruhnya di atas tiang-tiang itu. Ada tujuh anyaman rantai di atas setiap tiang, dan di atas anyaman rantai itu pula ada sebuah alas bundar. Di atas alas bundar inilah
\par 21 kepala-kepala tiang yang berbentuk bunga bakung itu dipasang pada tiang-tiang itu. Tiang-tiang itu ditempatkan di depan pintu masuk ke Rumah TUHAN--satu di sebelah selatan, yang lainnya di sebelah utara. Yang di sebelah selatan dinamakan Yakhin, dan yang di sebelah utara dinamakan Boas. Demikianlah Huram menyelesaikan pembuatan tiang-tiang itu.
\par 23 Kemudian Huram membuat sebuah bejana perunggu yang bundar, dengan ukuran sebagai berikut: dalamnya 2,2 meter, garis tengahnya 4,4 meter dan lingkarnya 13,2 meter.
\par 24 Sekeliling tepi luarnya dihias dengan dua jajar buah labu, yang dicor bersama-sama dengan bejana itu.
\par 25 Bejana itu ditempatkan di atas punggung dua belas sapi perunggu--tiga menghadap ke utara, tiga ke selatan, tiga ke barat, dan tiga ke timur.
\par 26 Tebal bejana itu 75 milimeter. Tepinya serupa tepi cangkir yang melengkung keluar mirip bunga bakung yang mekar. Bejana itu dapat memuat kira-kira 40.000 liter.
\par 27 Huram juga membuat sepuluh kereta perunggu, masing-masing panjangnya 1,8 meter, lebarnya 1,8 meter, tingginya 1,3 meter.
\par 28 Kereta-kereta itu terdiri dari papan-papan persegi yang dipasang dalam bingkai.
\par 29 Papan-papan itu dihias dengan ukiran singa, sapi dan kerub. Pada bingkai-bingkainya--di atas dan di bawah gambar-gambar singa dan sapi itu--tergantung ukiran rangkaian bunga.
\par 30 Setiap kereta itu mempunyai empat roda perunggu dengan poros roda dari perunggu juga. Pada keempat sudut kereta itu ada tiang penahan baskom dari perunggu dengan hiasan rangkaian bunga di sebelah luarnya.
\par 31 Bagian atas keempat tiang penahan itu dihubungkan oleh sebuah bingkai bundar untuk penahan baskomnya. Tiang-tiang penahan itu menonjol setinggi 44 sentimeter di atas papan persegi itu, dan 22 sentimeter di bawahnya. Pada mulut bingkai itu ada ukiran.
\par 32 Tinggi roda-rodanya 66 sentimeter; letaknya di bawah papan-papan. Poros rodanya dengan keretanya merupakan satu bagian.
\par 33 Roda-rodanya itu serupa dengan roda kereta perang; lingkar roda, jari-jari, poros dan pusat jari-jari rodanya semuanya terbuat dari perunggu.
\par 34 Ada empat tiang penahan pada setiap sudut bagian bawah kereta itu. Tiang-tiang itu merupakan satu bagian dengan keretanya.
\par 35 Di sekeliling bagian atas kereta itu ada pinggiran selebar 22 sentimeter. Tiang penahannya dan papan-papannya semuanya merupakan satu bagian dengan keretanya.
\par 36 Tiang-tiang penahan dan papan-papan itu dihiasi dengan gambar kerub, singa, dan pohon palem di mana saja ada tempat yang kosong untuk itu, disertai rangkaian bunga sekelilingnya.
\par 37 Demikianlah caranya kereta-kereta itu dibuat. Semuanya serupa--bentuk dan ukurannya.
\par 38 Huram membuat juga sepuluh baskom besar, satu untuk setiap kereta. Tiap baskom itu mempunyai garis tengah 1,8 meter dan dapat memuat 800 liter air.
\par 39 Huram menaruh lima kereta di sebelah selatan Rumah TUHAN, dan lima di sebelah utara. Bejana perunggu diletakkannya di sudut sebelah tenggara.
\par 40 Huram membuat juga kuali-kuali, sekop-sekop dan mangkuk-mangkuk. Sesudah itu selesailah seluruh pekerjaannya bagi Raja Salomo untuk Rumah TUHAN. Inilah perlengkapan yang telah dibuatnya: Dua tiang besar; Dua kepala tiang berbentuk mangkuk yang ditempatkan di atas kedua tiang itu; Anyaman rantai pada setiap kepala tiang; Empat ratus delima perunggu yang disusun dalam dua jajar sekeliling anyaman rantai pada setiap kepala tiang; Sepuluh kereta; Sepuluh baskom besar; Bejana perunggu; Dua belas sapi perunggu yang menopang bejana itu; Kuali-kuali, sekop-sekop dan mangkuk-mangkuk. Semua perlengkapan untuk Rumah TUHAN itu, yang dibuat Huram untuk Raja Salomo, dibuat dari perunggu, dan digosok sampai berkilap.
\par 46 Raja menyuruh membuat barang-barang itu di pengecoran logam antara Sukot dan Sartan di Lembah Yordan.
\par 47 Salomo tidak menyuruh menimbang barang-barang perunggu itu, sebab jumlahnya terlalu banyak. Jadi, berat barang-barang itu tidak diketahui.
\par 48 Raja Salomo juga menyuruh membuat perlengkapan-perlengkapan lain dari emas untuk Rumah TUHAN. Inilah perlengkapan-perlengkapan yang dibuatnya itu: Mezbah; Meja untuk roti yang dipersembahkan kepada Allah; Sepuluh kaki pelita yang ditempatkan di depan Ruang Mahasuci yaitu lima di sebelah selatan dan lima di sebelah utara; Bunga-bunga; Pelita-pelita; Sepit-sepit; Cangkir-cangkir; Alat pemadam pelita; Mangkuk-mangkuk; Piring-piring untuk dupa; Piring-piring untuk bara; Engsel-engsel untuk pintu Ruang Mahasuci dan pintu-pintu luar Rumah TUHAN.
\par 51 Setelah Raja Salomo selesai membangun Rumah TUHAN, ia menyimpan di dalam gudang-gudang rumah itu semua perak, emas, dan barang-barang lain yang telah dipersembahkan oleh Daud ayahnya kepada TUHAN.

\chapter{8}

\par 1 Setelah itu Raja Salomo menyuruh semua pemimpin suku dan kaum dari bangsa Israel datang kepadanya di Yerusalem untuk memindahkan Peti Perjanjian TUHAN dari Sion, Kota Daud, ke Rumah TUHAN.
\par 2 Maka datanglah mereka semua pada Hari Raya Pondok Daun dalam bulan Etanim.
\par 3 Setelah semua pemimpin itu berkumpul, para imam mengangkat Peti Perjanjian itu,
\par 4 dan membawanya ke Rumah TUHAN. Kemah TUHAN serta semua perlengkapannya dipindahkan juga ke Rumah TUHAN oleh para imam dan orang Lewi.
\par 5 Raja Salomo dan seluruh rakyat Israel berkumpul di depan Peti Perjanjian itu lalu mempersembahkan kepada TUHAN domba dan sapi yang tak terhitung banyaknya.
\par 6 Setelah itu para imam membawa Peti Perjanjian itu ke dalam Rumah TUHAN dan meletakkannya di bawah patung-patung kerub di dalam Ruang Mahasuci.
\par 7 Sayap patung-patung itu terbentang menutupi peti itu dan kayu-kayu pengusungnya.
\par 8 Kayu pengusung itu panjang sekali, sehingga ujung-ujungnya terlihat dari depan Ruang Mahasuci. (Sampai hari ini kayu-kayu itu masih di situ.)
\par 9 Di dalam Peti Perjanjian itu tidak ada sesuatu pun kecuali dua batu. Batu-batu itu telah dimasukkan ke situ oleh Musa di Gunung Sinai, ketika TUHAN membuat perjanjian dengan bangsa Israel pada waktu mereka dalam perjalanan dari Mesir ke negeri Kanaan.
\par 10 Pada saat para imam keluar dari Rumah TUHAN itu tiba-tiba rumah itu dipenuhi awan.
\par 11 Awan itu berkilauan oleh keagungan kehadiran TUHAN sehingga para imam itu tak dapat masuk kembali untuk melaksanakan tugas mereka.
\par 12 Maka berdoalah Salomo, katanya, "Engkaulah yang menempatkan surya di langit, ya TUHAN. Namun Engkau lebih suka tinggal dalam kegelapan awan.
\par 13 Kini kubangun untuk-Mu gedung yang megah, untuk tempat tinggal-Mu selama-lamanya."
\par 14 Setelah berdoa, Raja Salomo berpaling kepada seluruh rakyat yang sedang berdiri di situ, lalu memohonkan berkat Allah bagi mereka.
\par 15 Ia berkata, "Dahulu TUHAN telah berjanji kepada ayahku Daud begini, 'Sejak Aku membawa umat-Ku keluar dari Mesir, di seluruh negeri Israel tidak ada satu kota pun yang Kupilih menjadi tempat di mana harus dibangun rumah untuk tempat ibadat kepada-Ku. Tetapi engkau, Daud, Kupilih untuk memerintah umat-Ku.' Terpujilah TUHAN Allah Israel yang sudah menepati janji-Nya itu!"
\par 17 Selanjutnya Salomo berkata, "Ayahku Daud telah merencanakan untuk membangun rumah tempat ibadat kepada TUHAN Allah Israel.
\par 18 Tetapi TUHAN berkata kepadanya, 'Maksudmu itu baik.
\par 19 Tetapi, bukan engkau, melainkan anakmulah yang akan membangun rumah-Ku itu.'
\par 20 Sekarang TUHAN telah menepati janji-Nya. Aku telah menjadi raja Israel menggantikan ayahku, dan aku telah pula membangun rumah untuk tempat ibadat kepada TUHAN, Allah Israel.
\par 21 Di dalam Rumah TUHAN itu telah kusediakan tempat untuk Peti Perjanjian yang berisi batu perjanjian antara TUHAN dengan leluhur kita ketika Ia membawa mereka keluar dari Mesir."
\par 22 Setelah itu, di hadapan rakyat yang hadir, Salomo berdiri menghadap mezbah lalu mengangkat kedua tangannya
\par 23 dan berdoa, "TUHAN, Allah Israel! Di langit atau pun di bumi tak ada yang seperti Engkau! Engkau menepati janji-Mu dan menunjukkan kasih-Mu kepada umat-Mu yang setia dan taat dengan sepenuh hati kepada-Mu.
\par 24 Engkau telah menepati janji-Mu kepada ayahku Daud. Apa yang Kaujanjikan itu sudah Kaulaksanakan hari ini.
\par 25 Engkau juga telah berjanji kepada ayahku bahwa kalau keturunannya sungguh-sungguh taat kepada-Mu seperti dia, maka selalu akan ada seorang dari keturunannya yang menjadi raja atas Israel. Sekarang, ya TUHAN Allah Israel, tepatilah juga semua yang telah Kaujanjikan kepada ayahku Daud, hamba-Mu itu.
\par 27 Tetapi, ya Allah, sungguhkah Engkau sudi tinggal di bumi ini? Langit seluruhnya pun tak cukup luas untuk-Mu, apalagi rumah ibadat yang kubangun ini!
\par 28 Ya TUHAN, Allahku, aku hamba-Mu! Namun dengarkanlah kiranya doaku, dan kabulkanlah permohonanku hari ini.
\par 29 Semoga siang malam Engkau melindungi rumah ini yang telah Kaupilih sebagai tempat ibadat kepada-Mu. Semoga dari tempat kediaman-Mu di surga Engkau mendengar dan mengampuni aku serta umat-Mu apabila kami menghadap ke rumah ini dan berdoa kepada-Mu.
\par 31 Apabila seseorang dituduh bersalah terhadap orang lain dan dibawa ke mezbah-Mu di dalam rumah ibadat ini untuk bersumpah bahwa ia tidak bersalah,
\par 32 ya TUHAN, hendaklah Engkau mendengarkan di surga dan memutuskan perkara hamba-hamba-Mu ini. Hukumlah yang bersalah setimpal perbuatannya dan bebaskanlah yang tidak bersalah.
\par 33 Apabila umat-Mu Israel dikalahkan oleh musuh-musuhnya karena mereka berdosa, lalu mereka kembali kepada-Mu dan menghormati Engkau sebagai TUHAN, kemudian datang ke rumah ibadat ini serta berdoa mohon ampun kepada-Mu,
\par 34 semoga Engkau mendengarkan mereka di surga. Semoga Engkau mengampuni dosa umat-Mu ini, dan membawa mereka kembali ke negeri yang telah Kauberikan kepada leluhur mereka.
\par 35 Apabila umat-Mu berdosa kepada-Mu dan Engkau menghukum mereka dengan tidak menurunkan hujan lalu mereka bertobat dari dosa mereka dan menghormati Engkau sebagai TUHAN, kemudian menghadap ke rumah ibadat ini serta berdoa kepada-Mu,
\par 36 kiranya dari surga Engkau mendengarkan mereka. Dan ampunilah dosa umat-Mu Israel dan raja mereka. Ajarlah mereka melakukan apa yang benar. Setelah itu, ya TUHAN, turunkanlah hujan ke negeri-Mu ini, negeri yang Kauberikan kepada umat-Mu untuk menjadi miliknya selama-lamanya.
\par 37 Apabila negeri ini dilanda kelaparan atau wabah, dan tanaman-tanaman dirusak oleh angin panas, hama atau serangan belalang, atau apabila umat-Mu diserang musuh, atau diserang penyakit,
\par 38 semoga Engkau mendengarkan doa mereka. Kalau dari antara umat-Mu Israel ada yang dengan bersedih hati berdoa kepada-Mu sambil menengadahkan tangannya ke arah rumah ibadat ini,
\par 39 kiranya Engkau di dalam kediaman-Mu di surga mendengar serta mengampuni dan menolong mereka. Hanya Engkaulah yang mengenal isi hati manusia. Sebab itu perlakukanlah setiap orang setimpal perbuatan-perbuatan
\par 40 supaya umat-Mu taat kepada-Mu selalu selama mereka tinggal di negeri yang Kauberikan kepada leluhur kami.
\par 41 Apabila seorang asing di negeri yang jauh mendengar tentang kebesaran-Mu dan tentang keajaiban-keajaiban yang telah Kaulakukan untuk umat-Mu, lalu ia datang di rumah ibadat ini untuk menghormati Engkau dan berdoa kepada-Mu,
\par 43 semoga dari kediaman-Mu di surga Engkau mendengarkan doanya dan mengabulkan permintaannya. Dengan demikian segala bangsa di seluruh dunia mengenal Engkau dan taat kepada-Mu seperti umat-Mu Israel. Mereka akan mengetahui bahwa rumah yang kubangun ini adalah tempat untuk beribadat kepada-Mu.
\par 44 Apabila Engkau memerintahkan umat-Mu untuk pergi berperang melawan musuh, lalu di mana pun mereka berada, mereka menghadap ke kota pilihan-Mu ini dan berdoa ke arah rumah ini yang telah kubangun untuk-Mu,
\par 45 semoga di surga Engkau mendengarkan doa mereka itu, dan memberikan kemenangan kepada mereka.
\par 46 Apabila umat-Mu berdosa kepada-Mu--sesungguhnya tidak ada seorang pun yang tidak berdosa--lalu karena kemarahan-Mu Kaubiarkan mereka dikalahkan oleh musuh, dan dibawa sebagai tawanan ke suatu negeri yang jauh atau dekat,
\par 47 semoga Engkau mendengarkan doa mereka. Jikalau di negeri itu mereka meninggalkan dosa-dosa mereka, dan berdoa kepada-Mu sambil mengakui bahwa mereka telah berdosa dan berbuat jahat, dengarkanlah doa mereka, ya TUHAN.
\par 48 Jikalau di negeri itu mereka dengan ikhlas dan sungguh-sungguh meninggalkan dosa-dosa mereka, dan berdoa kepada-Mu sambil menghadap ke negeri ini yang Kauberikan kepada leluhur kami, ke arah kota pilihan-Mu dan rumah ibadat yang telah kubangun untuk-Mu ini,
\par 49 maka dari kediaman-Mu di surga hendaklah Engkau mendengar doa mereka dan mengasihani mereka.
\par 50 Ampunilah semua dosa dan kedurhakaan mereka terhadap-Mu, dan buatlah musuh-musuh mereka memperlakukan mereka dengan baik.
\par 51 Mereka adalah umat-Mu sendiri yang telah Kaubawa keluar dari Mesir negeri di mana mereka disiksa.
\par 52 Semoga Engkau, ya TUHAN Allah, selalu memperhatikan umat-Mu Israel beserta rajanya. Dan sudilah Engkau mendengarkan doa mereka setiap kali mereka berseru minta tolong kepada-Mu.
\par 53 Engkau, ya TUHAN Allah, sudah memilih mereka dari antara segala bangsa untuk menjadi umat-Mu sendiri, seperti yang telah Kaukatakan kepada mereka melalui hamba-Mu Musa, ketika Engkau menuntun leluhur kami keluar dari Mesir."
\par 54 Demikianlah Salomo berdoa sambil berlutut di depan mezbah dan dengan tangan ditengadahkan ke atas. Sesudah itu ia berdiri
\par 55 dan dengan suara keras memohonkan berkat Allah untuk semua orang yang berkumpul di situ, katanya,
\par 56 "Terpujilah TUHAN yang telah memberikan ketentraman kepada umat-Nya seperti yang dijanjikan-Nya melalui Musa hamba-Nya. Semua yang baik yang dijanjikan-Nya telah ditepati-Nya.
\par 57 Semoga TUHAN Allah kita menyertai kita sebagaimana Ia menyertai leluhur kita. Semoga Ia tidak meninggalkan kita.
\par 58 Semoga Ia menjadikan kita orang-orang yang setia kepada-Nya supaya kita selalu hidup menurut kehendak-Nya dan taat kepada semua hukum dan perintah yang telah diberikan-Nya kepada leluhur kita.
\par 59 Semoga TUHAN, Allah kita, selalu ingat akan doa dan permohonan yang telah kusampaikan ini kepada-Nya. Semoga Ia selalu menunjukkan kemurahan-Nya kepada umat Israel dan rajanya dengan memberikan kepada mereka apa yang mereka perlukan setiap hari.
\par 60 Dengan demikian segala bangsa di dunia akan tahu bahwa hanya TUHAN sendirilah Allah, tak ada yang lain.
\par 61 Hendaklah kalian, umat-Nya selalu setia kepada TUHAN, Allah kita dan taat kepada hukum-hukum dan perintah-perintah-Nya seperti yang kalian lakukan pada hari ini."
\par 62 Setelah itu Raja Salomo dan semua orang yang berkumpul di situ mempersembahkan kurban kepada TUHAN.
\par 63 Untuk kurban perdamaian Salomo mempersembahkan 22.000 sapi dan 120.000 domba. Demikianlah raja dan seluruh rakyat mengadakan upacara peresmian Rumah TUHAN.
\par 64 Hari itu juga Salomo meresmikan bagian tengah pelataran yang di depan Rumah TUHAN itu. Kemudian di pelataran itu ia mempersembahkan kurban bakaran, kurban gandum, dan lemak binatang untuk kurban perdamaian. Semua itu dipersembahkannya di situ karena mezbah perunggu terlalu kecil untuk semua persembahan itu.
\par 65 Di Rumah TUHAN itu Salomo dan seluruh umat Israel mengadakan perayaan Pondok Daun selama tujuh hari. Sangat besar jumlah orang yang berkumpul di situ. Mereka datang dari daerah sejauh Jalan Hamat di utara sampai ke perbatasan Mesir di selatan.
\par 66 Pada hari kedelapan Salomo menyuruh orang-orang itu pulang. Mereka semuanya memuji dia lalu kembali ke rumah mereka dengan hati yang gembira, karena semua berkat yang telah diberikan TUHAN kepada hamba-Nya Daud dan kepada bangsa Israel umat-Nya.

\chapter{9}

\par 1 Setelah Salomo selesai mendirikan Rumah TUHAN dan istana raja serta semua yang direncanakannya,
\par 2 TUHAN menampakkan diri lagi kepadanya seperti yang terjadi di Gibeon.
\par 3 TUHAN berkata kepadanya, "Doamu sudah Kudengar, dan dengan ini rumah yang telah kaudirikan ini Kunyatakan menjadi tempat khusus untuk beribadat kepada-Ku selama-lamanya. Aku akan selalu memperhatikan dan menjaga tempat ini.
\par 4 Kalau engkau mengabdi kepada-Ku dengan tulus hati dan jujur seperti ayahmu Daud, dan engkau mentaati hukum-hukum dan perintah-perintah-Ku,
\par 5 maka Aku akan menepati janji-Ku kepada ayahmu Daud bahwa anak cucunya turun-temurun akan selalu memerintah Israel.
\par 6 Tetapi kalau engkau atau keturunanmu membelakangi Aku dan tidak taat kepada hukum-hukum dan perintah-perintah-Ku serta engkau menyembah ilah-ilah lain,
\par 7 maka Aku akan mengusir umat-Ku Israel dari negeri yang telah Kuberikan kepada mereka. Aku juga akan meninggalkan rumah ini yang telah Kutetapkan menjadi tempat ibadat kepada-Ku. Di mana-mana orang Israel akan dihina dan ditertawakan.
\par 8 Rumah ibadat ini akan menjadi suatu timbunan puing sehingga setiap orang yang lewat di situ akan terkejut dan ngeri. Mereka akan berkata, 'Mengapa TUHAN berbuat begitu terhadap negeri dan rumah ini?'
\par 9 Lalu orang akan menjawab, 'Karena mereka meninggalkan TUHAN, Allah mereka, yang telah mengantar leluhur mereka keluar dari Mesir. Mereka menyembah ilah-ilah lain. Itulah sebabnya TUHAN mendatangkan bencana ini ke atas mereka.'"
\par 10 Salomo membangun Rumah TUHAN dan istana raja dalam waktu dua puluh tahun.
\par 11 Semua kayu cemara Libanon dan kayu cemara biasa serta semua emas yang diperlukan Salomo untuk pembangunan itu telah diberikan Raja Hiram kepadanya. Setelah pekerjaan itu selesai, Raja Salomo memberikan kepada Hiram dua puluh buah kota di wilayah Galilea.
\par 12 Maka pergilah Hiram melihatnya, tetapi ia tidak senang dengan pemberian itu.
\par 13 Ia berkata kepada Salomo, "Saudaraku, beginikah macamnya kota-kota yang kauberikan kepadaku?" Itulah sebabnya daerah itu masih disebut Kabul.
\par 14 Lebih dari 4.000 kilogram emas telah dikirim Hiram kepada Salomo.
\par 15 Raja Salomo memakai cara kerja paksa untuk membangun Rumah TUHAN, istana raja, tembok Yerusalem dan untuk menimbun tanah di sebelah selatan Rumah TUHAN. Ia memakai cara yang sama untuk membangun kembali kota Hazor, Megido dan Gezer.
\par 16 (Kota Gezer adalah kota yang dahulu diserang dan dikalahkan oleh raja Mesir, lalu penduduknya dibunuh dan kotanya dibakar. Kota itu kemudian diberikan oleh raja Mesir kepada putrinya sebagai hadiah pada hari pernikahannya dengan Salomo.
\par 17 Dan Salomolah yang membangun kembali kota itu.) Dengan kerja paksa juga, Salomo membangun kembali Bet-Horon-Hilir,
\par 18 Baalat, Tamar di padang gurun Yehuda,
\par 19 kota-kota perbekalan, serta kota-kota untuk pangkalan kereta perang dan kudanya. Semua rencana pembangunan di Yerusalem, Libanon, dan di seluruh wilayah kekuasaannya telah dilaksanakannya.
\par 20 Untuk kerja paksa itu Salomo mengerahkan orang-orang keturunan orang Amori, Het, Feris, Hewi dan Yebus. Mereka adalah orang-orang keturunan bangsa Kanaan yang tidak dapat dibunuh habis oleh orang Israel ketika mereka menduduki negeri itu. Sampai sekarang keturunan mereka masih menjadi hamba.
\par 22 Orang Israel tidak dijadikan hamba oleh Salomo; mereka ditugaskan sebagai prajurit, perwira, panglima, komandan kereta perang, dan tentara pasukan berkuda.
\par 23 Lima ratus lima puluh pegawai diserahi tanggung jawab atas orang-orang yang melakukan kerja paksa dalam berbagai proyek pembangunan Salomo.
\par 24 Tanah di sebelah selatan Rumah TUHAN ditimbun oleh Salomo setelah istrinya, yaitu putri raja Mesir, pindah dari Kota Daud ke istana yang dibangun Salomo untuk dia.
\par 25 Tiga kali setahun Salomo mempersembahkan kurban bakaran dan kurban perdamaian di atas mezbah yang telah didirikannya untuk TUHAN. Ia membakar juga dupa untuk TUHAN. Demikianlah Salomo menyelesaikan pembangunan Rumah TUHAN.
\par 26 Untuk armadanya, Raja Salomo membuat kapal-kapal di Ezion-Geber, dekat Elot di pantai Teluk Akaba, wilayah Edom.
\par 27 Raja Hiram mengirim awak-awak kapalnya yang berpengalaman untuk berlayar bersama awak-awak kapal Salomo.
\par 28 Pernah mereka berlayar ke negeri Ofir untuk mengambil 14.000 kilogram emas dan membawanya kepada Salomo.

\chapter{10}

\par 1 Ratu negeri Syeba mendengar tentang kemasyhuran Salomo. Maka ia datang ke Yerusalem untuk menguji Salomo dengan pertanyaan yang sulit-sulit.
\par 2 Ia datang disertai sejumlah besar pengiring dan unta yang sarat bermuatan rempah-rempah, batu permata, dan banyak sekali emas. Pada waktu bertemu dengan Salomo, ratu itu mengajukan segala macam pertanyaan yang dapat dipikirkannya.
\par 3 Semua pertanyaan itu dapat dijawab oleh Salomo, tidak satu pun yang terlalu sukar baginya.
\par 4 Ratu itu menyaksikan sendiri betapa bijaksananya Salomo. Ia melihat istana yang dibangun Salomo,
\par 5 tata kerja pegawai-pegawai istananya, dan pakaian seragam serta perumahan mereka. Ia melihat makanan dan minuman yang dihidangkan, cara para pelayan melayani pesta, dan kurban-kurban yang dipersembahkan Salomo di Rumah TUHAN. Semuanya itu membuat ratu negeri Syeba itu kagum dan terpesona.
\par 6 Maka berkatalah ia kepada Raja Salomo, "Segala yang saya dengar di tanah air saya tentang Tuan dan kebijaksanaan Tuan, memang benar!
\par 7 Dahulu saya tidak dapat percaya, tapi setelah saya datang dan menyaksikan semuanya dengan mata kepala sendiri, barulah saya yakin. Sesungguhnya segala yang saya dengar itu belum setengah dari yang saya lihat sekarang. Nyatanya kebijaksanaan dan kekayaan Tuan jauh lebih besar dari yang diberitakan kepada saya.
\par 8 Alangkah untungnya istri-istri Tuan! Dan alangkah mujurnya pegawai-pegawai yang selalu bekerja untuk Tuan sehingga dapat mendengar dari Tuan sendiri segala ajaran yang bijaksana!
\par 9 Terpujilah TUHAN, Allah Tuan! Dengan mengangkat Tuan menjadi raja Israel, TUHAN menunjukkan betapa senangnya Ia terhadap Tuan. Tidak habis-habisnya Ia mengasihi bangsa Israel; itu sebabnya Ia mengangkat Tuan menjadi raja mereka, supaya Tuan dapat menegakkan hukum dan keadilan."
\par 10 Kemudian ratu negeri Syeba itu menyerahkan kepada Salomo hadiah-hadiah yang dibawanya, yaitu lebih dari 4.000 kilogram emas dan sejumlah besar batu permata serta rempah-rempah. Tidak pernah lagi Salomo menerima rempah-rempah yang begitu banyak seperti yang diberikan ratu negeri Syeba itu kepadanya.
\par 11 (Kapal-kapal Hiram yang membawa emas dari Ofir untuk Salomo, juga membawa sejumlah besar batu permata dan kayu cendana.
\par 12 Kayu itu dipakai Salomo untuk membuat terali-terali di Rumah TUHAN dan di istana raja; juga untuk membuat kecapi dan gambus bagi para penyanyi. Kayu-kayu itu adalah kayu cendana terbaik yang pernah didatangkan ke Israel; tidak pernah lagi orang melihat kayu sebaik itu.)
\par 13 Selain hadiah-hadiah yang biasanya diberikan oleh Salomo, ia juga memberikan kepada ratu dari negeri Syeba itu segala yang dimintanya. Kemudian pulanglah ratu itu ke negerinya bersama semua pengiringnya.
\par 14 Setiap tahun Raja Salomo menerima hampir 23.000 kilogram emas,
\par 15 belum terhitung keuntungan-keuntungan dari perdagangan, dan pajak-pajak dari para saudagar, raja-raja Arab serta gubernur-gubernur Israel.
\par 16 Salomo membuat 200 perisai besar dari emas tempaan. Emas yang dipakai untuk setiap perisai itu ada kurang lebih tujuh kilogram.
\par 17 Ia juga membuat 300 perisai emas yang lebih kecil. Emas yang dipakai untuk setiap perisai kecil itu ada kurang lebih dua kilogram, semuanya emas tempaan. Semua perisai itu ditaruhnya di balai yang bernama Balai Hutan Libanon.
\par 18 Ia juga membuat sebuah kursi kerajaan yang besar yang dilapis dengan gading dan dihias dengan emas murni.
\par 19 Kursi itu berlengan, dan di sebelah menyebelahnya ada patung singa. Pada belakang kursi itu ada patung kepala sapi. Kursi itu juga mempunyai enam anak tangga.
\par 20 Pada ujung kiri dan ujung kanan setiap anak tangga itu ada patung singa--seluruhnya dua belas buah. Tidak pernah ada kursi kerajaan seperti itu di kerajaan mana pun juga.
\par 21 Semua perkakas minum Salomo dibuat dari emas, dan semua perkakas di Balai Hutan Libanon pun dibuat dari emas murni. Tidak ada yang dibuat dari perak, sebab pada zaman Salomo orang menganggap perak tidak berharga.
\par 22 Salomo mempunyai banyak kapal besar yang berlayar bersama kapal-kapal Raja Hiram. Tiga tahun sekali kapal-kapal itu kembali membawa emas, perak, gading, kera dan burung merak.
\par 23 Raja Salomo lebih kaya dan lebih bijaksana dari raja mana pun di dunia.
\par 24 Semua orang di seluruh dunia berusaha menemui Salomo untuk mendengar ajaran bijaksana yang diberikan Allah kepadanya.
\par 25 Mereka masing-masing datang dengan membawa hadiah untuk Salomo. Mereka memberikan barang-barang perak, emas, pakaian, senjata, rempah-rempah, kuda dan bagal. Begitulah tahun demi tahun.
\par 26 Salomo membentuk angkatan perang yang terdiri dari 1.400 kereta perang dan 12.000 tentara berkuda. Sebagian ditempatkannya di Yerusalem dan selebihnya di berbagai kota lain.
\par 27 Dalam zaman pemerintahan Salomo, di Yerusalem perak merupakan barang biasa sama seperti batu, dan kayu cemara Libanon berlimpah-limpah seperti kayu ara biasa.
\par 28 Melalui pengusaha-pengusaha raja, kuda didatangkan dari Mesir dan Kilikia,
\par 29 dan kereta-kereta perang didatangkan dari Mesir. Harga satu kuda 150 uang perak, dan harga satu kereta perang 600 uang perak. Kemudian melalui pengusaha-pengusaha itu pula kuda-kuda dan kereta-kereta perang itu dijual lagi kepada raja-raja Het dan raja-raja Siria.

\chapter{11}

\par 1 Raja Salomo mencintai banyak wanita asing. Selain putri raja Mesir, Salomo menikah juga dengan wanita-wanita Het, Moab, Amon, Edom dan Sidon.
\par 2 TUHAN sudah memerintahkan bahwa orang Israel tak boleh kawin campur dengan bangsa-bangsa itu, sebab mereka nanti menyebabkan orang Israel menyembah ilah-ilah lain. Walaupun demikian, Salomo menikahi juga wanita-wanita asing.
\par 3 Ada 700 putri bangsawan yang dinikahi Salomo, dan ada pula 300 selirnya. Istri-istri itulah yang menyebabkan Salomo meninggalkan Allah,
\par 4 sehingga pada waktu ia sudah tua mereka berhasil membujuknya menyembah ilah-ilah lain. Daud, ayah Salomo, setia kepada TUHAN, Allahnya, tetapi Salomo tidak seperti ayahnya.
\par 5 Salomo menyembah Asytoret, dewi bangsa Sidon, dan juga Molokh dewa yang memuakkan yang disembah bangsa Amon.
\par 6 Demikianlah Salomo berdosa terhadap TUHAN. Daud, ayah Salomo mengikuti TUHAN dengan sepenuh hati, tetapi Salomo tidak seperti itu.
\par 7 Di atas gunung sebelah timur Yerusalem, ia mendirikan tempat penyembahan untuk Kamos, dewa orang Moab, dan untuk Molokh, dewa orang Amon--kedua-duanya dewa yang memuakkan.
\par 8 Salomo mendirikan juga tempat-tempat di mana istri-istrinya yang berbangsa asing dapat membakar dupa, dan mempersembahkan kurban untuk dewa-dewa mereka sendiri.
\par 9 Sudah dua kali TUHAN, Allah Israel, menampakkan diri kepada Salomo dan mengingatkan dia bahwa ia tidak boleh menyembah ilah-ilah lain, namun Salomo tidak mentaati TUHAN. Ia malah meninggalkan TUHAN, sehingga TUHAN marah kepadanya.
\par 11 TUHAN berkata, "Karena dengan sadar engkau melanggar perjanjianmu dengan Aku dan tidak mentaati perintah-perintah-Ku, maka Aku akan merenggut kerajaan itu daripadamu, dan memberikannya kepada seorang pegawaimu.
\par 12 Meskipun begitu, demi ayahmu Daud, Aku tidak akan melakukan hal itu semasa hidupmu. Pada masa pemerintahan putramu, barulah Aku melakukannya.
\par 13 Tetapi tidak seluruh kerajaan itu akan Kurenggut daripadanya. Demi hamba-Ku Daud, dan demi Yerusalem yang telah Kujadikan kota-Ku itu, Aku akan menyisakan satu suku untuk putramu."
\par 14 Maka TUHAN membuat Hadad, seorang keturunan raja Edom, memusuhi Raja Salomo.
\par 15 Lama sebelum itu, sesudah Raja Daud menaklukkan Edom, Yoab, panglima angkatan bersenjata Raja Daud, pergi ke Edom untuk menguburkan orang-orangnya yang tewas. Enam bulan lamanya ia dan anak buahnya tinggal di Edom. Dan di dalam waktu itu mereka membunuh semua orang laki-laki di sana.
\par 17 Tetapi, Hadad yang pada waktu itu masih muda, telah lari ke Mesir bersama beberapa orang Edom, pelayan-pelayan ayahnya.
\par 18 Mereka berangkat dari Midian, lalu pergi ke Paran. Di sana beberapa orang lainnya bergabung dengan mereka, lalu bersama-sama pergi ke Mesir menemui raja Mesir. Raja itu memberikan kepada Hadad rumah dan tanah untuk tempat tinggalnya serta jaminan hidup.
\par 19 Maka Hadad pun menjadi kawan baik raja, sehingga raja memberikan iparnya sendiri, yaitu adik permaisuri Tahpenes, kepada Hadad untuk menjadi istrinya.
\par 20 Kemudian istri Hadad itu melahirkan seorang anak laki-laki yang diberi nama Genubat. Anak ini dibesarkan oleh permaisuri sendiri dan tinggal di istana bersama putra-putra raja.
\par 21 Ketika Hadad menerima berita bahwa Raja Daud sudah meninggal, dan Yoab, panglima angkatan bersenjata Israel juga sudah meninggal, berkatalah Hadad kepada raja Mesir, "Perkenankanlah aku kembali ke tanah airku."
\par 22 "Kenapa?" tanya raja. "Apakah engkau kekurangan di sini, sehingga engkau mau kembali ke tanah airmu?" "Aku tidak kekurangan apa-apa," sahut Hadad, "tetapi biarkanlah aku pergi." Maka kembalilah ia ke tanah airnya.
\par 23 Allah membuat juga Rezon, anak Elyada, memusuhi Salomo. Rezon adalah hamba yang melarikan diri dari tuannya yaitu Raja Hadadezer dari Zoba.
\par 24 Raja Daud pernah mengalahkan Hadadezer ini dan membunuh kawan-kawan Hadadezer dari Siria. Pada waktu itu Rezon melarikan diri dari Hadadezer dan mengumpulkan orang-orang lalu menjadi kepala gerombolan. Ia dan anak buahnya pergi ke Damsyik lalu tinggal di kota itu. Di sana anak buahnya mengangkat dia menjadi raja Siria.
\par 25 Seperti Hadad, Raja Rezon pun memusuhi Israel semasa hidup Salomo, dan mengacau di negeri itu.
\par 26 Orang lain yang memberontak melawan Salomo ialah salah seorang pegawainya yang bernama Yerobeam. Ia anak Nebat dari Zereda di Efraim. Ibunya sudah janda, bernama Zerua.
\par 27 Berikut ini kisah tentang pemberontakan Yerobeam. Ketika tanah di sebelah selatan Rumah TUHAN sedang ditimbun, dan tembok kota diperbaiki,
\par 28 Yerobeam juga bekerja di situ. Ketika Salomo melihat bahwa Yerobeam trampil dan rajin, ia memberikan kepadanya tanggung jawab untuk mengawasi kerja paksa di wilayah suku Manasye dan Efraim.
\par 29 Pada suatu hari Yerobeam keluar dari Yerusalem. Di tengah jalan di padang ia didatangi Nabi Ahia dari Silo yang pada waktu itu memakai jubah yang baru.
\par 30 Lalu Ahia melepaskan jubahnya, dan merobek-robeknya menjadi dua belas potong.
\par 31 Kemudian berkatalah ia kepada Yerobeam, "Ambillah sepuluh potong untuk dirimu, sebab TUHAN, Allah Israel, berkata begini kepadamu, 'Aku akan merenggut kerajaan ini dari Salomo, dan memberikan sepuluh suku kepadamu.
\par 32 Satu suku akan tetap pada Salomo, demi hamba-Ku Daud, dan demi Yerusalem, yang telah Kupilih dari seluruh negeri Israel menjadi kota-Ku.
\par 33 Aku akan melakukan hal itu karena Salomo telah meninggalkan Aku dan menyembah dewa-dewa, yaitu: Asytoret, dewi bangsa Sidon; Kamos, dewa bangsa Moab; dan Milkom, dewa bangsa Amon. Salomo tidak menuruti kehendak-Ku. Ia tidak seperti ayahnya Daud yang taat kepada-Ku dan melaksanakan perintah-perintah-Ku.
\par 34 Meskipun begitu Aku tidak akan mengambil seluruh kerajaan itu dari dia. Ia akan Kubiarkan berkuasa selama hidupnya. Hal itu Kulakukan demi hamba-Ku Daud, yang telah Kupilih dan yang mentaati hukum-hukum-Ku dan perintah-perintah-Ku.
\par 35 Kerajaan ini akan Kuambil dari putra Salomo dan sepuluh suku akan Kuberikan kepadamu.
\par 36 Namun satu suku akan Kubiarkan padanya, supaya selalu ada seorang dari keturunan hamba-Ku Daud yang memerintah untuk Aku di Yerusalem, kota yang telah Kupilih sebagai tempat ibadat kepada-Ku.
\par 37 Dan kau, Yerobeam, akan Kujadikan raja Israel dan Kuperkenankan memerintah seluruh daerah yang engkau kehendaki.
\par 38 Jikalau engkau, seperti hamba-Ku Daud, taat sepenuhnya kepada-Ku serta hidup menurut hukum-hukum-Ku, dan menyenangkan hati-Ku dengan melakukan semua yang Kuperintahkan kepadamu, maka Aku akan menyertaimu selalu. Aku akan menjadikan engkau raja di Israel. Aku akan menjamin bahwa keturunanmu akan menjadi raja Israel menggantikan engkau, sama seperti yang telah Kubuat untuk Daud.
\par 39 Karena Salomo berdosa, Aku akan menghukum keturunan Daud, tetapi tidak untuk selama-lamanya.'"
\par 40 Ketika berita itu sampai kepada Salomo, ia berusaha membunuh Yerobeam. Tetapi Yerobeam lari kepada Raja Sisak di Mesir, dan tinggal di situ sampai Salomo meninggal.
\par 41 Kisah lainnya mengenai Salomo, mengenai semua yang telah dikerjakannya dan mengenai pengetahuannya, sudah dicatat dalam buku Riwayat Hidup Salomo.
\par 42 Empat puluh tahun lamanya Salomo memerintah di Yerusalem atas seluruh Israel.
\par 43 Kemudian ia meninggal dan dimakamkan di Kota Daud. Lalu Rehabeam putranya menjadi raja menggantikan dia.

\chapter{12}

\par 1 Rehabeam pergi ke Sikhem, karena seluruh rakyat Israel bagian utara telah berkumpul di sana untuk melantik dia menjadi raja.
\par 2 Pada waktu itu Yerobeam anak Nebat masih ada di Mesir karena melarikan diri dari Raja Salomo. Ketika ia mendengar tentang Rehabeam, ia kembali ke Israel.
\par 3 Rakyat bagian utara mengundang Yerobeam, lalu mereka bersama dia pergi menghadap Rehabeam. Kata mereka,
\par 4 "Salomo ayah Baginda telah memberi beban yang berat kepada kami. Sekarang kalau Baginda meringankan beban dan mengurangi derita kami, kami akan mengabdi kepada Baginda."
\par 5 Rehabeam menyahut, "Aku harus berpikir dahulu. Datanglah kembali tiga hari lagi." Maka pulanglah orang-orang itu.
\par 6 Lalu Rehabeam meminta nasihat kepada orang-orang tua yang pernah menjadi penasihat Salomo, ayah Rehabeam. "Bagaimana aku harus menjawab orang-orang itu?" tanya Rehabeam, "Apa nasihat kalian?"
\par 7 Mereka menjawab, "Kalau Baginda sungguh-sungguh ingin mengabdi kepada rakyat, kabulkanlah permohonan mereka, maka mereka akan mengabdi kepada Baginda selama-lamanya."
\par 8 Tetapi nasihat orang-orang tua itu tidak dihiraukan oleh Rehabeam. Sebaliknya, ia pergi kepada orang-orang muda yang sebaya dengan dia dan yang sekarang membantu dia.
\par 9 Kepada mereka ia bertanya, "Apa nasihat kalian kepadaku untuk menjawab rakyat yang meminta supaya aku meringankan beban mereka?"
\par 10 Mereka menjawab, "Beginilah hendaknya Baginda katakan kepada mereka, 'Selemah-lemahnya aku, aku masih lebih kuat dari ayahku!'
\par 11 Ayahku memberikan beban yang berat kepadamu, tetapi aku akan memberikan yang lebih berat lagi. Ia menyebat kalian dengan cemeti, tetapi aku akan memecut kalian dengan cemeti berduri besi!"
\par 12 Tiga hari kemudian Yerobeam dan semua orang itu kembali menghadap Raja Rehabeam, seperti yang telah diperintahkannya.
\par 13 Berlawanan dengan nasihat orang-orang tua itu, rakyat yang datang kepadanya itu disapanya dengan kasar.
\par 14 Sesuai dengan nasihat orang-orang muda, ia berkata, "Ayahku memberikan kepadamu beban yang berat, tetapi aku akan membuat beban itu lebih berat lagi. Ia menyebat kalian dengan cemeti, tetapi aku akan memecut kalian dengan cemeti berduri besi!"
\par 15 Permintaan rakyat tidak dihiraukannya. Tetapi itu memang kehendak TUHAN. Apa yang dikatakan TUHAN melalui Nabi Ahia dari Silo mengenai Yerobeam anak Nebat harus terjadi.
\par 16 Ketika rakyat melihat bahwa raja tidak mau mendengarkan mereka, mereka berteriak, "Perduli amat dengan Daud dan keturunannya! Mereka tidak pernah berbuat apa-apa untuk kita. Rakyat Israel, mari kita pulang! Biarkan Rehabeam itu mengurus dirinya sendiri!" Maka pulanglah rakyat Israel,
\par 17 dan Rehabeam menjadi raja hanya atas wilayah Yehuda.
\par 18 Meskipun begitu Raja Rehabeam masih juga menyuruh Adoniram, seorang kepala pekerja rodi, menenangkan rakyat. Tetapi mereka melempari dia dengan batu sampai mati. Melihat hal itu Rehabeam cepat-cepat naik ke keretanya, lalu melarikan diri ke Yerusalem.
\par 19 Mulai saat itu rakyat di bagian utara kerajaan Israel selalu memberontak terhadap raja-raja keturunan Daud.
\par 20 Pada waktu seluruh rakyat Israel mendengar bahwa Yerobeam sudah kembali dari Mesir, mereka mengundang dia ke suatu pertemuan lalu mengangkat dia menjadi raja Israel. Hanya rakyat suku Yehuda saja yang tetap setia kepada keturunan Daud.
\par 21 Ketika Raja Rehabeam tiba di Yerusalem, ia mengumpulkan 180.000 prajuritnya yang terbaik dari suku Yehuda dan Benyamin untuk memerangi orang Israel dan memulihkan kekuasaannya atas suku-suku di bagian utara Israel.
\par 22 Tetapi Allah menyuruh Nabi Semaya
\par 23 menyampaikan berita ini kepada Rehabeam, dan kepada seluruh suku Yehuda dan Benyamin,
\par 24 "Janganlah memerangi saudara-saudaramu orang Israel. Pulanglah! Apa yang telah terjadi adalah kehendak-Ku." Maka mereka semuanya menuruti perintah TUHAN, lalu pulang ke rumah masing-masing.
\par 25 Setelah menjadi raja Israel, Yerobeam memperkuat kota Sikhem di daerah pegunungan Efraim, lalu tinggal di sana untuk sementara waktu. Kemudian ia pergi lagi dari situ dan memperkuat kota Pnuel.
\par 26 Ia berpikir begini dalam hatinya, "Jikalau rakyat tetap pergi ke Yerusalem untuk mempersembahkan kurban di Rumah TUHAN seperti yang mereka lakukan sekarang ini, mereka akan kembali mendukung Rehabeam, raja Yehuda, lalu membunuh aku."
\par 28 Sesudah mempertimbangkan hal itu, ia membuat dua patung sapi emas lalu berkata kepada rakyat, "Bangsa Israel! Sudah cukup lama kalian pergi ke Yerusalem untuk beribadat. Sekarang, inilah ilah-ilahmu yang telah membawa kalian keluar dari Mesir!"
\par 29 Lalu Yerobeam menempatkan satu dari sapi emas itu di Betel dan yang lainnya di Dan.
\par 30 Maka rakyat pun berbuat dosa, karena mereka pergi ke Betel dan ke Dan untuk beribadat.
\par 31 Yerobeam juga mendirikan tempat-tempat ibadat lain di puncak-puncak gunung, dan ia mengangkat imam-imam dari antara keluarga-keluarga yang bukan suku Lewi.
\par 32 Kemudian Yerobeam menempatkan imam-imam itu di Betel untuk bertugas di sana dan ia sendiri mempersembahkan kurban kepada sapi-sapi emas yang dibuatnya itu. Lalu ia menentukan tanggal lima belas bulan delapan sebagai hari raya untuk Israel, sama seperti hari raya di Yehuda. Pada hari itu Yerobeam pergi ke Betel dan mempersembahkan kurban untuk merayakan hari raya yang telah ditentukannya itu.

\chapter{13}

\par 1 Pada waktu itu ada seorang nabi di Yehuda. Ia disuruh oleh TUHAN untuk pergi ke Betel. Nabi itu tiba di sana ketika Yerobeam sedang berdiri di depan mezbah untuk mempersembahkan kurban.
\par 2 Atas perintah TUHAN berserulah nabi itu kepada mezbah itu, "Hai mezbah, hai mezbah! Beginilah kata TUHAN, 'Seorang anak yang akan dinamakan Yosia akan lahir di dalam keluarga Daud. Di atasmu, hai mezbah, ia akan membunuh imam-imam yang bertugas di tempat-tempat ibadat kepada dewa, dan yang telah pula mempersembahkan kurban di atasmu. Dan di atasmu juga ia akan membakar tulang-tulang manusia.'"
\par 3 Kata nabi itu selanjutnya, "Mezbah ini akan runtuh, dan abu yang di atasnya akan tumpah. Itulah buktinya bahwa saya disuruh TUHAN."
\par 4 Pada saat Yerobeam mendengar kata-kata itu, ia menunjuk kepada nabi itu lalu berteriak, "Tangkap orang itu!" Tetapi langsung lengan Yerobeam menjadi kaku sehingga ia tak dapat menariknya kembali.
\par 5 Mezbah itu tiba-tiba runtuh dan abu di atasnya tumpah ke tanah, seperti yang telah dikatakan oleh nabi itu atas perintah TUHAN.
\par 6 Lalu berkatalah Raja Yerobeam kepada nabi itu, "Tolonglah berdoa kepada TUHAN Allahmu untukku. Mintalah belas kasihan-Nya untuk menyembuhkan lenganku!" Nabi itu pun berdoa kepada TUHAN, lalu sembuhlah lengan raja.
\par 7 Kemudian raja berkata kepada nabi itu, "Mari makan di rumahku. Aku mau memberi hadiah kepadamu."
\par 8 Nabi itu menjawab, "Sekalipun Baginda memberi saya separuh dari kekayaan Baginda, saya tak akan ikut untuk makan atau minum dengan Baginda.
\par 9 Saya mendapat perintah dari TUHAN untuk tidak makan atau minum, dan tidak pulang mengikuti jalan yang saya lalui ketika datang."
\par 10 Demikianlah nabi itu pergi lewat jalan lain.
\par 11 Pada waktu itu ada seorang nabi tua yang tinggal di Betel. Anak-anaknya memberitahukan kepadanya tentang nabi dari Yehuda itu. Mereka menceritakan tentang apa yang dilakukannya di Betel pada hari itu, dan apa yang dikatakannya kepada Raja Yerobeam.
\par 12 Lalu bertanyalah nabi tua itu, "Nabi dari Yehuda itu mengambil jalan yang mana ketika ia pulang?" Maka mereka menunjukkan jalan itu kepadanya.
\par 13 Lalu ia menyuruh mereka memasangkan pelana pada keledainya. Sesudah itu ia naik keledainya dan pergi
\par 14 mengejar nabi dari Yehuda itu. Ia menemukan nabi itu sedang duduk di bawah sebuah pohon besar. "Apakah Anda nabi dari Yehuda?" tanya nabi tua itu. "Benar," jawab orang itu.
\par 15 "Mari makan di rumah saya," kata nabi tua itu.
\par 16 Tetapi nabi dari Yehuda itu menjawab, "Maaf! Saya tidak bisa menerima undangan Bapak untuk ikut ke rumah Bapak. Saya juga tak akan makan atau minum di sini dengan Bapak.
\par 17 TUHAN telah memerintahkan saya untuk tidak makan atau minum dan tidak pulang mengikuti jalan yang sama ketika datang."
\par 18 Maka nabi tua dari Betel itu pun berdusta kepadanya, dengan mengatakan, "Saya juga nabi seperti Anda. Atas perintah TUHAN pun seorang malaikat sudah menyuruh saya menjemput dan menjamu Anda di rumah saya."
\par 19 Maka nabi dari Yehuda itu ikut ke rumah nabi tua itu dan makan dengan dia.
\par 20 Sementara mereka duduk makan, TUHAN berbicara kepada nabi tua itu,
\par 21 sehingga ia menegur nabi dari Yehuda itu, katanya, "TUHAN berkata bahwa Anda tidak taat kepada-Nya. Anda tidak melakukan apa yang diperintahkan-Nya kepada Anda.
\par 22 Anda telah kembali dan makan minum, padahal Anda telah dilarang untuk melakukan hal itu. Karena itu Anda akan dibunuh, dan mayat Anda tidak akan dikubur di kuburan keluarga Anda."
\par 23 Setelah mereka selesai makan, nabi tua itu memasangkan pelana pada keledai nabi dari Yehuda itu,
\par 24 lalu berangkatlah nabi itu. Di tengah jalan ia diserang seekor singa, lalu mati diterkam. Mayatnya kemudian terkapar di jalan, dan keledainya berdiri di dekatnya. Singa itu pun berdiri di situ.
\par 25 Orang-orang yang lewat, melihat mayat dan singa itu di situ. Lalu mereka pergi ke Betel dan memberitahukan apa yang mereka lihat.
\par 26 Ketika nabi tua itu mendengar tentang hal itu, ia berkata, "Pasti itu nabi yang tidak taat kepada perintah TUHAN! Itu sebabnya TUHAN mengirim singa itu untuk menerkam dan membunuh dia seperti yang dikatakan TUHAN sebelumnya."
\par 27 Lalu berkatalah nabi tua itu kepada anak-anaknya, "Pasangkan pelana pada keledai saya." Setelah naik ke atas keledainya, ia pun berangkat.
\par 28 Ia menemukan mayat nabi dari Yehuda itu masih terletak di jalan dengan keledai dan singa di dekatnya. Singa itu tidak menerkam keledai itu, dan tidak pula makan mayat itu.
\par 29 Ia mengangkat mayat itu dan menaruhnya di atas keledainya, lalu membawanya kembali ke Betel untuk diratapi dan dikuburkan.
\par 30 Nabi tua itu sendiri menguburkan mayat itu di kuburan keluarganya, lalu ia dan anak-anaknya meratapi mayat itu. Mereka berkata, "Kasihan kau, saudaraku, kasihan!"
\par 31 Kemudian kata nabi tua itu kepada anak-anaknya, "Apabila aku meninggal, kuburlah aku di kuburan ini di samping nabi itu.
\par 32 Apa yang telah diucapkannya atas perintah TUHAN terhadap mezbah di Betel itu dan terhadap semua tempat ibadat di kota-kota Samaria pasti akan terjadi."
\par 33 Raja Yerobeam tetap tidak mau meninggalkan perbuatannya yang jahat. Untuk mengurus tempat-tempat ibadat yang telah didirikannya itu, ia terus saja memilih imam dari keluarga-keluarga biasa. Siapa saja yang mau, diangkatnya menjadi imam.
\par 34 Perbuatannya yang jahat itu menyebabkan keturunannya celaka dan hancur sama sekali.

\chapter{14}

\par 1 Pada waktu itu anak laki-laki Yerobeam yang bernama Abia jatuh sakit.
\par 2 Berkatalah Yerobeam kepada istrinya, "Bersiap-siaplah untuk pergi ke Silo ke rumah Nabi Ahia yang dahulu mengatakan bahwa saya akan menjadi raja Israel. Tetapi menyamarlah supaya orang tidak tahu siapa engkau.
\par 3 Bawalah untuk dia sepuluh roti, sedikit kue dan madu sebotol. Tanyalah kepadanya apa yang akan terjadi dengan Abia, anak kita. Nabi itu akan memberitahukannya kepadamu."
\par 4 Maka pergilah istri Yerobeam ke rumah Nabi Ahia di Silo. Nabi itu tidak bisa melihat lagi karena ia sudah tua sekali.
\par 5 Tetapi TUHAN memberitahukan kepadanya bahwa istri Yerobeam sedang datang untuk menanyakan tentang anaknya yang sakit. Dan TUHAN juga memberitahukan kepada Nabi Ahia apa yang harus dikatakannya nanti. Ketika istri Yerobeam tiba di sana, ia berbuat seolah-olah ia orang lain.
\par 6 Tetapi begitu Nabi Ahia mendengar bunyi langkahnya di pintu, ia berkata, "Mari masuk, saya tahu engkau istri Yerobeam. Tak usah berpura-pura. Ada berita buruk untukmu.
\par 7 Pulanglah dan beritahukan kepada Yerobeam bahwa inilah yang dikatakan TUHAN, Allah Israel, 'Engkau telah Kupilih dari antara rakyat dan Kujadikan penguasa atas umat-Ku Israel.
\par 8 Engkau telah Kuberikan kerajaan yang Kurenggut dari keturunan Daud. Tetapi engkau tidak seperti hamba-Ku Daud. Ia setia betul kepada-Ku dan taat kepada perintah-Ku. Ia melakukan hanya yang Kusetujui.
\par 9 Tetapi engkau melakukan kejahatan yang lebih besar daripada yang dilakukan oleh mereka yang memerintah sebelum engkau. Engkau meninggalkan Aku dan membangkitkan amarah-Ku dengan membuat berhala serta patung-patung dari logam untuk disembah.
\par 10 Karena itu Aku akan mendatangkan celaka ke atas keluargamu. Semua keturunanmu yang laki-laki, apa pun kedudukannya, akan dibunuh. Aku akan melenyapkan keluargamu, sama seperti orang menyapu bersih kotoran.
\par 11 Anggota keluargamu yang mati di kota akan dimakan anjing, dan yang mati di luar kota akan dimakan burung. Ingat, Akulah TUHAN yang berbicara!'"
\par 12 Selanjutnya Nabi Ahia berkata kepada istri Yerobeam, "Nah, pulanglah sekarang! Pada saat engkau memasuki kota, anakmu itu akan meninggal.
\par 13 Seluruh rakyat Israel akan berkabung dan menguburkannya. Dialah satu-satunya anggota keluarga Yerobeam yang akan dikubur secara terhormat, sebab hanya dia yang berkenan di hati TUHAN, Allah Israel.
\par 14 Israel akan diberi TUHAN seorang raja yang akan mengakhiri kedudukan keturunan Yerobeam sebagai raja. Harinya sudah tiba! Mulai sekarang
\par 15 TUHAN akan menghukum Israel sehingga tergoyang-goyang seperti gelagah di dalam air. TUHAN akan mencabut rakyat Israel dari negeri yang baik ini yang diberikan-Nya kepada leluhur mereka. Ia akan menceraiberaikan mereka ke seberang Sungai Efrat, karena mereka telah membangkitkan kemarahan-Nya dengan membuat patung-patung dewi Asyera.
\par 16 TUHAN akan meninggalkan Israel karena Yerobeam telah berdosa dan menyebabkan orang Israel pun berdosa juga."
\par 17 Setelah itu berangkatlah istri Yerobeam ke Tirza. Pada saat ia masuk ke rumahnya, anaknya meninggal.
\par 18 Rakyat Israel berkabung lalu menguburkannya seperti yang telah dikatakan TUHAN melalui Nabi Ahia, hamba-Nya.
\par 19 Kisah lainnya mengenai Raja Yerobeam, yaitu pemerintahannya dan peperangan yang dipimpinnya, sudah dicatat dalam buku Sejarah Raja-raja Israel.
\par 20 Dua puluh dua tahun lamanya Yerobeam memerintah sebagai raja, kemudian ia meninggal dan dikubur. Anaknya yang bernama Nadab menggantikan dia sebagai raja.
\par 21 Ketika Rehabeam, anak Salomo, menjadi raja Yehuda, ia berumur empat puluh satu tahun. Ia memerintah tujuh belas tahun lamanya di Yerusalem. Dari seluruh Israel kota itu telah dikhususkan TUHAN sebagai tempat ibadat kepada-Nya. Ibu Rehabeam ialah Naama dari bangsa Amon.
\par 22 Bangsa Yehuda melakukan yang jahat pada pemandangan TUHAN. Perbuatan mereka lebih membangkitkan kemarahan TUHAN daripada segala perbuatan leluhur mereka.
\par 23 Di atas bukit-bukit dan di bawah pohon-pohon besar, mereka membangun tempat-tempat penyembahan berhala, dan mendirikan tugu-tugu serta patung-patung dewi Asyera untuk disembah.
\par 24 Di tempat-tempat itu mereka mengadakan pelacuran sebagai penyembahan kepada dewa. Bahkan laki-laki pun bertugas sebagai pelacur. Orang Yehuda melakukan semua perbuatan keji yang dilakukan oleh bangsa-bangsa yang telah diusir TUHAN pada waktu orang Israel memasuki negeri itu.
\par 25 Pada tahun kelima pemerintahan Rehabeam, Sisak, raja Mesir menyerang Yerusalem.
\par 26 Ia merampas semua barang berharga yang terdapat di Rumah TUHAN dan di istana raja, juga perisai-perisai emas yang dibuat oleh Salomo.
\par 27 Maka sebagai gantinya Raja Rehabeam membuat perisai-perisai perunggu, dan mempercayakannya kepada para perwira yang mengawal pintu gerbang istana raja.
\par 28 Setiap kali raja ke Rumah TUHAN, para pengawal membawa perisai-perisai itu, kemudian mengembalikannya ke kamar jaga.
\par 29 Kisah lainnya mengenai Raja Rehabeam sudah dicatat dalam buku Sejarah Raja-raja Yehuda.
\par 30 Semasa pemerintahan Rehabeam, selalu ada perang antara Rehabeam dan Yerobeam.
\par 31 Kemudian Rehabeam meninggal dan dikubur di makam raja-raja di Kota Daud. Abiam, putranya, menggantikan dia sebagai raja.

\chapter{15}

\par 1 Pada tahun kedelapan belas pemerintahan Raja Yerobeam atas Israel, Abiam menjadi raja Yehuda
\par 2 dan memerintah selama tiga tahun di Yerusalem. Ibunya ialah Maakha, anak Abisalom.
\par 3 Abiam melakukan dosa-dosa seperti ayahnya. Ia tidak sepenuh hati setia kepada TUHAN Allahnya, berlainan dengan Daud, kakek buyutnya.
\par 4 Tetapi, demi Raja Daud, maka TUHAN, Allah yang disembah oleh Daud, telah memberikan kepada Abiam seorang anak untuk menggantikan dia memerintah di Yerusalem dan mempertahankan kota itu.
\par 5 TUHAN melakukan hal itu oleh sebab Daud telah menyenangkan hati-Nya. Daud tidak pernah mengabaikan perintah-perintah TUHAN, kecuali dalam perkara Uria orang Het itu.
\par 6 Perang antara Rehabeam dan Yerobeam yang dimulai pada masa Rehabeam masih berlangsung selama pemerintahan Abiam.
\par 7 Kisah lainnya mengenai Abiam sudah dicatat dalam buku Sejarah Raja-raja Yehuda.
\par 8 Abiam pun meninggal dan dimakamkan di Kota Daud. Asa anaknya menjadi raja menggantikan dia.
\par 9 Pada tahun kedua puluh pemerintahan Raja Yerobeam atas Israel, Asa menjadi raja Yehuda.
\par 10 Ia memerintah empat puluh satu tahun lamanya di Yerusalem. Neneknya ialah Maakha, anak Abisalom.
\par 11 Seperti Daud, nenek moyangnya, Asa menyenangkan hati TUHAN.
\par 12 Semua pelacur yang bertugas di tempat-tempat penyembahan dewa disingkirkannya, dan semua berhala yang didirikan oleh raja-raja yang memerintah sebelum dia dihapuskannya dari seluruh negeri.
\par 13 Neneknya, Maakha, dipecatnya dari kedudukannya sebagai Ibu suri, sebab ia telah membuat patung berhala yang cabul untuk Asyera, dewi kesuburan. Asa merobohkan berhala itu dan membakarnya di Lembah Kidron.
\par 14 Meskipun tidak semua tempat penyembahan berhala dihancurkan oleh Asa, namun ia tetap setia kepada TUHAN sepanjang hidupnya.
\par 15 Semua emas dan perak yang telah dipersembahkan ayahnya kepada TUHAN, dibawanya ke Rumah TUHAN, begitu juga emas dan perak, persembahannya sendiri.
\par 16 Semasa pemerintahan Raja Asa dari Yehuda dan Raja Baesa dari Israel selalu ada perang antara kedua raja itu.
\par 17 Baesa menyerang Yehuda dan memperkuat kota Rama untuk menutup jalan keluar masuk Yehuda.
\par 18 Karena itu Raja Asa mengambil semua emas dan perak yang masih ada di Rumah TUHAN dan istana raja lalu mengirimnya ke Damsyik kepada Benhadad raja Siria, anak Tabrimon, yaitu cucu Hezion. Barang-barang itu dikirim dengan perantaraan beberapa pejabat istana, disertai pesan berikut ini,
\par 19 "Marilah kita mengikat persahabatan seperti yang sudah dilakukan oleh orang tua kita. Bersama ini saya mengirim emas dan perak sebagai hadiah, dan mengajak tuan memutuskan hubungan dengan Baesa raja Israel, supaya ia menarik kembali pasukannya dari wilayah saya."
\par 20 Benhadad setuju dengan tawaran itu lalu menyuruh perwira-perwiranya bersama pasukan mereka menyerang kota-kota Israel. Mereka mengalahkan Iyon, Dan, Abel-Bet-Maakha dan Kinerot di dekat Danau Galilea serta seluruh wilayah Naftali.
\par 21 Ketika Baesa mendengar hal itu, ia berhenti memperkuat Rama. Ia pergi ke Tirza dan untuk sementara waktu tidak berperang.
\par 22 Setelah itu, semua orang di seluruh Yehuda tanpa kecuali diperintahkan oleh Raja Asa untuk mengangkut batu dan kayu yang dipakai Baesa di Rama untuk memperkuat kota itu. Bahan-bahan itu dipakai Asa untuk memperkuat kota Mizpa dan Geba, sebuah kota di wilayah Benyamin.
\par 23 Kisah lainnya mengenai Raja Asa, yaitu kepahlawanannya dan kota-kota yang diperkuatnya, sudah dicatat dalam buku Sejarah Raja-raja Yehuda. Pada masa tuanya Asa menderita penyakit pada kedua kakinya.
\par 24 Ia meninggal dan dikubur di makam raja-raja di Kota Daud. Yosafat, putranya menjadi raja menggantikan dia.
\par 25 Pada tahun kedua pemerintahan Raja Asa atas Yehuda, Nadab anak Yerobeam menjadi raja di Israel dan memerintah dua tahun lamanya.
\par 26 Sama seperti ayahnya yang memerintah sebelumnya, ia pun melakukan yang jahat pada pemandangan TUHAN sehingga menyebabkan orang Israel berbuat dosa.
\par 27 Baesa, anak Ahia, dari suku Isakhar berkomplot melawan Nadab, lalu membunuh dia. Pada waktu itu Nadab dengan pasukannya sedang mengepung kota Gibeton di wilayah Filistin.
\par 28 Itu terjadi dalam tahun ketiga pemerintahan Raja Asa atas Yehuda. Demikianlah Baesa menjadi raja Israel menggantikan Nadab.
\par 29 Segera setelah Baesa menjadi raja, ia membunuh seluruh keluarga Yerobeam. Sesuai dengan apa yang telah dikatakan TUHAN melalui hamba-Nya, Nabi Ahia dari Silo, maka seluruh keluarga Yerobeam terbunuh, tidak seorang pun yang luput.
\par 30 Yerobeam berdosa dan menyebabkan orang Israel pun berdosa. Dengan demikian Yerobeam membangkitkan kemarahan TUHAN, Allah yang disembah orang Israel.
\par 31 Kisah lainnya mengenai Nadab sudah dicatat dalam buku Sejarah Raja-raja Israel.
\par 32 Selama masa pemerintahan Raja Baesa dari Israel, perang selalu terjadi antara Baesa dan Raja Asa dari Yehuda.
\par 33 Pada tahun ketiga pemerintahan Raja Asa atas Yehuda, Baesa anak Ahia menjadi raja atas seluruh Israel. Ia memerintah di Tirza dua puluh empat tahun lamanya.
\par 34 Sama seperti Raja Yerobeam yang memerintah sebelumnya, Raja Baesa pun berdosa kepada TUHAN dan menyebabkan orang Israel juga berbuat dosa.

\chapter{16}

\par 1 Maka berbicaralah TUHAN kepada Nabi Yehu anak Hanani, kata-Nya, "Sampaikanlah kepada Baesa pesan-Ku ini,
\par 2 'Dahulu engkau bukan apa-apa, tetapi engkau sudah Kuangkat menjadi pemimpin atas umat-Ku Israel. Sekarang kau berdosa seperti Yerobeam dan menyebabkan umat-Ku Israel berdosa, sehingga membangkitkan kemarahan-Ku.
\par 3 Karena itu Aku akan melenyapkan engkau dan keluargamu sama seperti yang telah Kulakukan terhadap Yerobeam.
\par 4 Anggota keluargamu yang mati di kota akan dimakan anjing, dan mereka yang mati di luar kota akan dimakan burung.'"
\par 5 Pesan TUHAN kepada Baesa dan keluarganya itu disampaikan oleh Nabi Yehu karena Baesa telah melakukan yang jahat pada pemandangan TUHAN, Baesa membangkitkan kemarahan TUHAN, bukan hanya karena ia telah membunuh seluruh keluarga Yerobeam. Kisah lainnya mengenai Baesa, termasuk kepahlawanannya sudah dicatat dalam buku Sejarah Raja-raja Israel. Baesa meninggal lalu dimakamkan di Tirza. Ela, putranya, menggantikan dia menjadi raja.
\par 8 Pada tahun kedua puluh enam pemerintahan Raja Asa atas Yehuda, Ela anak Baesa menjadi raja atas Israel. Ia memerintah di Tirza dua tahun lamanya.
\par 9 Zimri, seorang perwira yang mengepalai separuh dari pasukan berkereta, berkomplot melawan Ela. Pada suatu hari di Tirza, Ela minum-minum sampai mabuk di rumah Arza, kepala rumah tangga istana raja.
\par 10 Lalu Zimri masuk dan membunuh Ela, kemudian menjadi raja menggantikan Ela. Itu terjadi pada tahun kedua puluh tujuh pemerintahan Raja Asa atas Yehuda.
\par 11 Segera sesudah Zimri menjadi raja, ia membunuh seluruh keluarga Raja Baesa. Bahkan semua sanak saudara Baesa yang laki-laki serta kawan-kawan keluarganya pun tidak ada yang luput.
\par 12 Sesuai dengan apa yang dikatakan TUHAN terhadap Baesa melalui Nabi Yehu, seluruh keluarga Baesa terbunuh.
\par 13 Baesa dan Ela, anaknya, menyembah dewa-dewa dan menyebabkan orang Israel berbuat dosa. Karena itu, mereka membangkitkan kemarahan TUHAN, Allah yang harus disembah oleh orang Israel.
\par 14 Kisah lainnya mengenai Ela sudah dicatat dalam buku Sejarah Raja-raja Israel.
\par 15 Pada tahun kedua puluh tujuh pemerintahan Raja Asa atas Yehuda, Zimri menjadi raja Israel dan memerintah di Tirza selama tujuh hari. Pada waktu itu tentara Israel sedang mengepung Gibeton, kota orang Filistin.
\par 16 Ketika mereka mendengar bahwa Zimri telah berkomplot melawan raja dan telah membunuhnya, pada hari itu juga di perkemahan mereka itu, mereka semua mengangkat Omri, panglima mereka, sebagai raja Israel.
\par 17 Kemudian Omri dan pasukannya meninggalkan Gibeton lalu pergi mengepung Tirza.
\par 18 Pada saat Zimri melihat bahwa kota itu sudah direbut, ia masuk ke dalam istana dan membakar istana itu bersama-sama dengan dirinya, maka matilah ia.
\par 19 Itu terjadi karena ia melakukan yang jahat pada pemandangan TUHAN. Sama seperti Raja Yerobeam, Zimri pun berdosa kepada TUHAN dan menyebabkan umat Israel berdosa pula.
\par 20 Kisah lainnya mengenai Zimri, termasuk komplotan pengkhianatan yang direncanakannya, sudah dicatat di dalam buku Sejarah Raja-raja Israel.
\par 21 Terjadilah perpecahan di dalam bangsa Israel karena ada yang ingin supaya Tibni anak Ginat menjadi raja, tetapi ada pula yang menghendaki Omri menjadi raja.
\par 22 Akhirnya mereka yang berpihak kepada Omri menang. Tibni mati dan Omri menjadi raja.
\par 23 Itu terjadi pada tahun ketiga puluh satu pemerintahan Raja Asa atas Yehuda. Omri memerintah sebagai raja atas Israel dua belas tahun lamanya. Selama enam tahun yang pertama ia memerintah di Tirza.
\par 24 Tetapi kemudian ia membeli sebuah gunung seharga 6.000 uang perak dari seseorang bernama Semer. Lalu ia membangun sebuah kota di gunung itu dan menamakannya Samaria, menurut nama Semer, pemilik lama gunung itu.
\par 25 Omri melakukan dosa-dosa yang lebih besar daripada yang dilakukan oleh raja-raja yang memerintah sebelum dia.
\par 26 Seperti Yerobeam, Omri berdosa serta menyebabkan orang Israel pun berdosa dan menyembah dewa. Oleh karena itu mereka membangkitkan kemarahan TUHAN, Allah yang harus disembah oleh orang Israel.
\par 27 Kisah lainnya mengenai Raja Omri, dan semua yang telah dilakukannya sudah dicatat dalam buku Sejarah Raja-raja Israel.
\par 28 Setelah Omri meninggal dan dimakamkan di Samaria, Ahab anaknya menjadi raja menggantikan dia.
\par 29 Pada tahun ketiga puluh delapan pemerintahan Raja Asa atas Yehuda, Ahab anak Omri menjadi raja atas Israel dan memerintah di Samaria dua puluh dua tahun lamanya.
\par 30 Ia melakukan dosa-dosa yang lebih besar pada pemandangan TUHAN daripada raja-raja sebelumnya.
\par 31 Belum cukup juga baginya hidup berdosa seperti Raja Yerobeam, ia malah mengawini Izebel anak Raja Etbaal dari Sidon, dan ia beribadat kepada Baal.
\par 32 Ia mendirikan rumah ibadat di Samaria untuk Baal, dan di dalam rumah ibadat itu ia membuat sebuah mezbah untuk dewa itu.
\par 33 Ia membuat juga patung Dewi Asyera. Ia melakukan lebih banyak dosa daripada yang dilakukan oleh semua raja Israel yang memerintah sebelum dia. Karena itu ia membangkitkan kemarahan TUHAN, Allah yang harus disembah oleh orang Israel.
\par 34 Dalam masa pemerintahan Ahab, kota Yerikho didirikan kembali oleh Hiel orang Betel. Tetapi seperti yang telah dikatakan sebelumnya oleh TUHAN melalui Yosua anak Nun, Hiel kehilangan dua anaknya: Abiram yang sulung meninggal ketika Hiel meletakkan dasar kota, dan Segub yang bungsu meninggal ketika Hiel memasang pintu-pintu gerbang kota itu.

\chapter{17}

\par 1 Adalah seorang nabi bernama Elia. Ia berasal dari Tisbe, suatu kota di wilayah Gilead. Pada suatu hari ia datang kepada Raja Ahab dan berkata, "Demi TUHAN yang hidup, Allah yang disembah orang Israel dan yang saya layani, saya memberitahukan kepada Tuan bahwa selama dua atau tiga tahun yang berikut ini tidak akan ada embun atau hujan sedikit pun, kecuali kalau saya mengatakannya."
\par 2 Setelah itu TUHAN berkata kepada Elia,
\par 3 "Tinggalkanlah tempat ini, pergilah ke timur ke seberang Sungai Yordan dan bersembunyilah di sana dekat anak Sungai Kerit.
\par 4 Engkau dapat minum dari anak sungai itu, dan burung gagak akan Kusuruh membawa makanan untukmu."
\par 5 Elia menuruti perintah TUHAN itu, dan pergi ke anak sungai Kerit lalu tinggal di situ.
\par 6 Ia minum dari anak sungai itu, dan makan roti dan daging yang dibawa oleh burung gagak setiap pagi dan setiap sore.
\par 7 Setelah beberapa waktu lamanya, anak sungai itu pun kering karena tidak ada hujan.
\par 8 Kemudian TUHAN berkata kepada Elia,
\par 9 "Sekarang kau harus pergi ke kota Sarfat, dekat Sidon dan tinggal di sana. Aku sudah menyuruh seorang janda di sana supaya ia memberi makan kepadamu."
\par 10 Maka pergilah Elia ke Sarfat. Ketika tiba di pintu kota itu, ia melihat seorang janda sedang mengumpulkan kayu api. Lalu kata Elia kepada janda itu, "Ibu, tolong ambilkan sedikit air minum untuk saya."
\par 11 Ketika janda itu sedang berjalan untuk mengambil air itu, Elia berseru, "Ibu, bawakanlah juga sedikit roti."
\par 12 Janda itu menjawab, "Demi TUHAN yang hidup, Allah yang disembah oleh Bapak, saya bersumpah bahwa saya tak punya roti. Saya hanya mempunyai segenggam tepung terigu di dalam mangkuk, dan sedikit minyak zaitun di dalam botol. Saya sedang mengumpulkan kayu api untuk memasak bahan yang sedikit itu, supaya saya dan anak saya bisa makan. Itulah makanan kami yang terakhir; sesudah itu kami akan menunggu mati saja."
\par 13 "Jangan khawatir, Ibu!" kata Elia kepadanya, "Silakan Ibu membuat makanan untuk Ibu dan anak Ibu. Tapi sebelum itu buatlah dahulu satu roti kecil dari tepung dan minyak itu, dan bawalah ke mari.
\par 14 Sebab TUHAN, Allah yang disembah orang Israel, mengatakan bahwa mangkuk itu selalu akan berisi tepung, dan botol itu selalu akan berisi minyak sampai pada saat TUHAN mengirim hujan ke bumi."
\par 15 Lalu pergilah janda itu melakukan apa yang dikatakan Elia. Maka mereka bertiga mempunyai persediaan makanan yang cukup untuk berhari-hari lamanya.
\par 16 Dan seperti yang sudah dikatakan TUHAN melalui Elia, mangkuk itu selalu saja berisi tepung, dan botol itu pun selalu berisi minyak.
\par 17 Beberapa waktu kemudian anak janda itu jatuh sakit. Makin lama makin parah sakitnya sehingga ia meninggal.
\par 18 Lalu kata janda itu kepada Elia, "Hamba Allah, mengapa Bapak melakukan hal ini terhadap saya? Apakah Bapak datang untuk menyebabkan Allah ingat akan dosa saya, sehingga anak saya harus meninggal?"
\par 19 Elia menjawab, "Bawa anak itu ke mari." Lalu Elia mengambil anak laki-laki itu dari ibunya, dan membawanya ke ruang atas, ke kamar yang ditumpangi Elia. Elia membaringkan anak itu di atas tempat tidur,
\par 20 lalu berdoa dengan suara yang keras, "Ya TUHAN, Allahku, mengapa Engkau mendatangkan celaka ini ke atas janda ini? Ia sudah memberi tumpangan kepadaku dan sekarang Engkau membunuh anaknya!"
\par 21 Setelah itu tiga kali Elia menelungkupkan badannya di atas anak itu, sambil berdoa, "Ya TUHAN, Allahku, hidupkanlah kiranya anak ini!"
\par 22 TUHAN mendengarkan doa Elia; anak itu mulai bernapas dan hidup kembali.
\par 23 Lalu Elia turun dari kamarnya sambil membawa anak itu kepada ibunya dan berkata, "Bu, ini anak Ibu! Ia sudah hidup kembali!"
\par 24 Janda itu menjawab, "Sekarang saya tahu Bapak adalah hamba Allah dan perkataan Bapak memang benar dari TUHAN!"

\chapter{18}

\par 1 Setelah beberapa waktu lamanya, pada tahun ketiga musim kering itu, TUHAN berkata kepada Elia, "Pergilah menghadap Raja Ahab, tak lama lagi Aku akan menurunkan hujan."
\par 2 Maka berangkatlah Elia. Pada waktu itu bencana kelaparan sudah sangat parah di Samaria.
\par 3 Karena itu Ahab memanggil Obaja untuk datang menghadap. (Obaja adalah kepala rumah tangga istana. Ia seorang yang sungguh-sungguh beribadat kepada TUHAN.
\par 4 Pada waktu Izebel membunuh nabi-nabi TUHAN, Obaja ini menolong 100 orang nabi dengan menyembunyikan mereka di dalam gua--50 orang dalam satu gua--serta memberi mereka makan dan minum.)
\par 5 Ahab berkata kepada Obaja, "Mari kita menjelajahi seluruh negeri ini dan mencari mata air atau sungai. Mungkin kita akan menemukan rumput untuk menyelamatkan kuda dan bagal kita, supaya tidak perlu kita membunuh seekor pun dari ternak kita."
\par 6 Setelah membagi daerah yang akan dijelajahi itu, mereka pun berangkat. Ahab ke satu jurusan dan Obaja ke jurusan yang lain.
\par 7 Di tengah jalan Obaja bertemu dengan Elia. Setelah mengenali Elia, Obaja sujud menghormatinya lalu bertanya, "Betulkah ini Pak Elia?"
\par 8 "Betul, saya Elia," jawab Elia. "Sekarang pergilah beritahukan rajamu bahwa saya ada di sini."
\par 9 Obaja menjawab, "Apa salah saya sehingga Bapak mau menyerahkan saya kepada Ahab untuk dibunuh?
\par 10 Demi TUHAN yang hidup, Allah yang disembah oleh Bapak, saya menyatakan bahwa raja telah menyuruh saya mencari Bapak di mana-mana. Apabila di suatu negeri penguasanya berkata bahwa Bapak tidak ada di sana, Ahab menuntut agar penguasa negeri itu bersumpah bahwa Bapak memang tidak ditemukan di situ.
\par 11 Dan sekarang Bapak mau supaya saya pergi kepadanya dan berkata bahwa Bapak ada di sini?
\par 12 Apabila saya pergi dari sini, Roh TUHAN mungkin akan membawa Bapak ke tempat yang tidak saya ketahui! Dan kalau saya memberitahukan kepada Ahab bahwa Bapak ada di sini, kemudian ia tidak menemukan Bapak, tentu ia akan membunuh saya. Ketahuilah, Pak, saya ini sungguh-sungguh beribadat kepada TUHAN sejak kecil!
\par 13 Apakah Bapak belum mendengar bahwa ketika Izebel membunuh nabi-nabi TUHAN, saya menyembunyikan 100 dari mereka di dalam gua--50 orang dalam satu gua--dan memberi mereka makan dan minum?
\par 14 Sekarang, mengapa Bapak menyuruh saya pergi memberitahukan kepada raja bahwa Bapak ada di sini? Ia pasti akan membunuh saya!"
\par 15 Elia menjawab, "Demi TUHAN Yang Mahakuasa yang saya layani, saya berjanji akan datang sendiri menghadap raja hari ini juga."
\par 16 Maka pergilah Obaja melaporkan kepada raja, dan Ahab pun berangkat untuk menemui Elia.
\par 17 Pada waktu Ahab melihat Elia, Ahab berseru, "Ini dia si pengacau di Israel!"
\par 18 "Saya bukan pengacau," sahut Elia, "tetapi Baginda sendiri. Baginda dan keluarga Bagindalah yang pengacau. Dengan menyembah berhala-berhala Baal, Baginda melanggar perintah-perintah TUHAN.
\par 19 Sekarang, perintahkanlah seluruh rakyat Israel untuk bertemu dengan saya di Gunung Karmel. Bawa juga keempat ratus lima puluh nabi Baal dan keempat ratus nabi Dewi Asyera itu yang dibiayai oleh permaisuri Izebel!"
\par 20 Maka Ahab mengerahkan seluruh rakyat dan nabi-nabi Baal itu ke Gunung Karmel.
\par 21 Lalu Elia mendekati rakyat itu dan berkata, "Sampai kapan kalian mau tetap mendua hati? Kalau TUHAN itu Allah, sembahlah TUHAN! Kalau Baal itu Allah, sembahlah Baal!" Rakyat yang berkumpul di situ diam saja.
\par 22 Kemudian Elia berkata, "Di antara nabi-nabi TUHAN hanya sayalah yang tertinggal, padahal di sini ada 450 nabi Baal.
\par 23 Sekarang bawalah ke mari dua ekor sapi jantan. Suruh nabi-nabi Baal itu mengambil seekor dan menyembelihnya, kemudian memotong-motongnya lalu meletakkannya di atas kayu api. Tetapi mereka tidak boleh menyalakan api di situ. Sapi yang seekor lagi akan saya persiapkan begitu juga. Saya akan menyembelihnya dan memotong-motongnya serta meletakkannya di atas kayu api. Tapi saya pun tak akan menyalakan api di situ.
\par 24 Biarlah nabi-nabi Baal itu berdoa kepada dewa mereka, dan saya pun akan berdoa kepada TUHAN. Yang menjawab dengan mengirim api, Dialah Allah." Seluruh rakyat menyahut dengan suara yang keras, "Setuju!"
\par 25 Lalu kata Elia kepada nabi-nabi Baal itu, "Karena kalian banyak, silakan kalian memilih dulu sapinya dan menyiapkannya, tetapi jangan menyalakan api pada kayunya. Setelah itu berdoalah kepada ilahmu."
\par 26 Maka mereka memilih seekor sapi dan menyiapkannya. Setelah itu mereka berdoa kepada Baal dari pagi sampai tengah hari sambil berteriak-teriak, "Jawablah kami, Baal!" Mereka melakukan itu sambil terus menari-nari sekeliling mezbah mereka. Tetapi tidak ada jawaban sama sekali.
\par 27 Pada tengah hari mulailah Elia memperolok-olok mereka. "Berdoalah lebih keras! Ia ilah, bukan? Mungkin ia sedang melamun, atau ke kamar kecil. Boleh jadi juga ia sedang bepergian! Atau barangkali ia sedang tidur, dan kalian harus membangunkan dia!"
\par 28 Nabi-nabi itu berdoa lebih keras lagi. Dan seperti yang biasanya mereka lakukan, mereka menggores-goresi badan mereka dengan pedang dan tombak sampai darah bercucuran.
\par 29 Begitulah mereka terus-menerus sampai petang hari seperti orang kesurupan. Tetapi tidak ada yang menjawab, tidak ada yang memperhatikan.
\par 30 Lalu kata Elia kepada seluruh rakyat itu, "Mari mendekat!" Mereka semua berkumpul di sekelilingnya, kemudian ia mulai memperbaiki mezbah TUHAN yang telah runtuh.
\par 31 Ia mengambil dua belas batu, setiap batu mewakili salah satu dari kedua belas suku keturunan Yakub, yakni orang yang telah diberi nama Israel oleh TUHAN.
\par 32 Dengan batu-batu itu Elia membangun kembali mezbah tempat ibadat kepada TUHAN. Di sekeliling mezbah itu ia menggali parit yang cukup besar sehingga dapat menampung kurang lebih lima belas liter air.
\par 33 Ia menyusun kayu api di atas mezbah, lalu sapi yang seekor itu dipotong-potongnya dan ditaruhnya di atas kayu itu. Kemudian ia berkata, "Isilah empat tempayan dengan air sampai penuh, lalu tuangkan air itu ke atas persembahan kurban dan kayunya." Setelah mereka melakukan hal itu,
\par 34 ia berkata, "Sekali lagi," lalu mereka melakukannya. "Satu kali lagi," kata Elia, dan mereka melakukannya pula.
\par 35 Maka mengalirlah air di sekeliling mezbah itu sehingga paritnya pun penuh air.
\par 36 Ketika tiba saat mempersembahkan kurban petang, Nabi Elia mendekati mezbah itu lalu berdoa, "Ya TUHAN, Allah yang disembah oleh Abraham, Ishak dan Yakub, nyatakanlah sekarang ini bahwa Engkaulah Allah di Israel, dan saya hamba-Mu. Nyatakanlah juga bahwa segala yang saya lakukan ini adalah atas perintah-Mu.
\par 37 Jawablah, TUHAN! Jawablah saya supaya rakyat ini tahu bahwa Engkau, ya TUHAN, adalah Allah, dan bahwa Engkaulah yang membuat mereka kembali kepada-Mu."
\par 38 Lalu TUHAN mengirim api dari langit dan membakar hangus kurban itu bersama kayu apinya, batu-batunya dan tanahnya serta menjilat habis air yang terdapat di dalam parit itu.
\par 39 Pada saat rakyat melihat hal itu mereka tersungkur ke tanah sambil berkata, "TUHAN itu Allah! Sungguh TUHAN itu Allah!"
\par 40 Maka berkatalah Elia, "Tangkap nabi-nabi Baal itu! Jangan biarkan seorang pun lolos!" Lalu orang-orang menangkap nabi-nabi Baal itu, kemudian Elia membawa mereka ke Sungai Kison dan di sana ia membunuh mereka semuanya.
\par 41 Setelah itu berkatalah Elia kepada Raja Ahab, "Sekarang baiklah Baginda pergi makan! Sebentar lagi akan hujan, sebab derunya sudah terdengar."
\par 42 Lalu Ahab pergi makan, dan Elia naik ke atas Gunung Karmel. Di sana ia sujud dengan mukanya ke tanah di antara kedua lututnya.
\par 43 Kepada pelayannya ia berkata, "Naiklah ke puncak dan lihatlah ke arah laut." Hamba itu naik, lalu kembali dan berkata, "Saya tidak melihat apa-apa." Sampai tujuh kali Elia menyuruh hambanya naik turun untuk melihat.
\par 44 Pada yang ketujuh kalinya hamba itu kembali dan berkata, "Saya melihat awan sekecil telapak tangan datang dari arah laut." Maka Elia berkata kepada hambanya, "Pergilah kepada Raja Ahab, suruh dia naik ke keretanya dan pulang sebelum ia terhalang oleh hujan."
\par 45 Dalam sekejap saja langit menjadi mendung, dan angin kencang bertiup serta hujan lebat pun mulai turun. Ahab naik ke keretanya lalu pulang ke Yizreel.
\par 46 Saat itu Elia mendapat kekuatan dari TUHAN. Ia melipat jubahnya ke atas dan mengikatnya pada pinggangnya lalu berlari mendahului Ahab sepanjang jalan sampai ke pintu gerbang kota Yizreel.

\chapter{19}

\par 1 Raja Ahab menceritakan kepada Izebel istrinya, semua yang telah dilakukan oleh Elia, juga bagaimana Elia membunuh semua nabi Baal.
\par 2 Maka Izebel mengirim berita ini kepada Elia, "Semoga para dewa menghukum, malah membunuh saya, kalau besok pada saat seperti ini saya tidak memperlakukan engkau seperti engkau memperlakukan nabi-nabi Baal itu."
\par 3 Elia menjadi takut, lalu lari supaya tidak dibunuh. Ia pergi dengan pelayannya ke Bersyeba di Yehuda. Di sana ia meninggalkan pelayannya itu,
\par 4 lalu berjalan kaki ke padang gurun selama sehari dan berhenti di bawah sebuah pohon yang rindang. Di situ ia duduk dan ingin supaya mati saja. "Saya tidak tahan lagi, TUHAN," katanya kepada TUHAN. "Ambillah nyawa saya. Saya tidak lebih baik dari leluhur saya!"
\par 5 Lalu ia berbaring di bawah pohon itu dan tertidur. Tiba-tiba seorang malaikat menyentuhnya dan berkata, "Bangun Elia, makanlah!"
\par 6 Elia melihat sekelilingnya, lalu tampak di dekat kepalanya sepotong roti bakar dan kendi berisi air. Ia pun makan dan minum, lalu berbaring lagi.
\par 7 Untuk kedua kalinya malaikat TUHAN datang menyentuhnya dan berkata, "Bangun, Elia, makanlah, supaya kau dapat tahan mengadakan perjalanan jauh."
\par 8 Elia bangun, lalu makan dan minum. Maka ia menjadi kuat dan dapat berjalan empat puluh hari lamanya ke Gunung Sinai, tempat Allah menyatakan diri.
\par 9 Di sana Elia bermalam di dalam gua. Lalu TUHAN berkata kepadanya, "Elia, sedang apa kau di sini?"
\par 10 Elia menjawab, "Ya TUHAN, Allah Yang Mahakuasa, saya selalu bekerja hanya untuk Engkau sendiri. Tetapi umat Israel melanggar perjanjian mereka dengan Engkau. Mereka membongkar mezbah-mezbah-Mu dan membunuh nabi-nabi-Mu. Hanya saya sendirilah yang tinggal, dan sekarang mereka mau membunuh saya!"
\par 11 "Keluarlah dari gua itu," kata TUHAN kepadanya, "dan berdirilah menghadap Aku di atas gunung." Lalu TUHAN lewat di situ, didahului oleh angin yang bertiup kencang sekali sehingga bukit-bukit terbelah dan gunung-gunung batu pecah. Tetapi TUHAN tidak menyatakan diri di dalam angin itu. Sesudah angin itu reda, terjadilah gempa bumi, tetapi di dalam gempa itu pun TUHAN tidak menyatakan diri.
\par 12 Kemudian datanglah api, tetapi TUHAN pun tidak berada di dalam api itu. Sesudah itu suasana menjadi senyap, lalu terdengar suatu suara yang kecil lembut.
\par 13 Ketika Elia mendengar suara itu, ia menutup mukanya dengan jubahnya, lalu keluar dan berdiri di mulut gua itu. Maka terdengarlah suara yang berkata, "Elia, sedang apa kau di sini?"
\par 14 Elia menjawabnya, "Ya TUHAN, Allah Yang Mahakuasa, saya selalu bekerja hanya untuk Engkau sendiri. Tetapi umat Israel melanggar perjanjian mereka dengan Engkau. Mereka membongkar mezbah-mezbah-Mu dan membunuh nabi-nabi-Mu. Hanya saya sendirilah yang tinggal, dan sekarang mereka mau membunuh saya!"
\par 15 TUHAN berkata, "Kembalilah ke padang gurun dekat Damsyik! Lalu pergilah ke kota dan tuangkanlah minyak ke atas kepala Hazael sebagai tanda pengangkatannya menjadi raja Siria.
\par 16 Berbuatlah begitu juga kepada Yehu anak Nimsi, supaya dia menjadi raja Israel, dan kepada Elisa anak Safat dari Abel-Mehola, supaya dia menjadi nabi menggantikan engkau.
\par 17 Siapa saja yang tidak dibunuh oleh Hazael, akan dibunuh oleh Yehu, dan mereka yang lolos dari Yehu akan dibunuh oleh Elisa.
\par 18 Tetapi 7.000 orang di Israel akan Kuselamatkan, yaitu orang-orang yang tetap setia kepada-Ku dan tak pernah sujud menyembah patung Baal atau menciumnya."
\par 19 Maka berangkatlah Elia dan mendapatkan Elisa yang sedang membajak. Ada dua belas pasang sapi di depannya dan Elisa sendiri membajak dengan pasangan sapi yang terakhir. Ketika Elia lewat dekat Elisa, Elia melepaskan jubahnya dan menghamparkannya ke bahu Elisa.
\par 20 Maka Elisa meninggalkan sapi-sapinya dan berlari mengejar Elia, serta berkata, "Izinkanlah saya minta diri dulu kepada orang tua saya, kemudian saya akan mengikuti Bapak." Sahut Elia, "Silakan. Saya tidak melarang engkau."
\par 21 Elisa kembali ke ladangnya lalu menyembelih sapi-sapinya. Kayu bajaknya dipakainya sebagai kayu api untuk memasak daging sapi-sapi itu. Kemudian ia membagi-bagikan daging itu kepada anak buahnya dan mereka memakannya. Setelah itu Elisa pergi mengikuti Elia dan menjadi pembantunya.

\chapter{20}

\par 1 Pada suatu hari Benhadad, raja Siria, mengumpulkan seluruh tentaranya, dan bersama 32 raja lain dengan kereta perang, dan kuda mereka, ia mengepung dan menyerang kota Samaria.
\par 2 Lalu ia mengutus orang kepada Ahab, raja Israel, di kota itu dengan pesan ini,
\par 3 "Raja Benhadad menuntut supaya engkau menyerahkan kepadanya emas, perak, istri-istrimu, dan anak-anakmu yang kuat-kuat."
\par 4 Ahab menjawab, "Katakan kepada baginda Raja Benhadad bahwa aku setuju. Ia boleh memiliki aku dan segala kepunyaanku."
\par 5 Kemudian para utusan itu kembali lagi kepada Ahab untuk menyampaikan tuntutan yang lain dari Benhadad. Beginilah bunyi tuntutan itu, "Dulu raja kami menuntut supaya engkau menyerahkan kepadanya emas dan perak serta istri-istrimu dan anak-anakmu.
\par 6 Tetapi sekarang ia akan mengutus perwira-perwira untuk menggeledah istanamu dan rumah para pegawaimu, serta mengambil apa saja yang mereka anggap berharga. Mereka akan datang ke sini besok sekitar saat ini."
\par 7 Maka Raja Ahab memanggil semua pemimpin Israel dan berkata, "Kalian saksikan sendiri bahwa orang ini ingin mencelakakan kita. Ia menuntut supaya aku menyerahkan istri-istriku dan anak-anakku serta emas perakku, dan aku tidak menolak tuntutannya itu."
\par 8 Para pemimpin dan rakyat berkata, "Jangan hiraukan dia, Baginda! Jangan menuruti kemauannya."
\par 9 Karena itu Ahab menjawab utusan-utusan Benhadad itu begini, "Beritahukan kepada Baginda, rajamu itu bahwa aku menyetujui tuntutannya yang pertama, tetapi tuntutannya yang kedua ini tidak dapat kusetujui." Maka pergilah utusan-utusan itu, kemudian kembali lagi kepada Ahab dan berkata,
\par 10 "Raja kami akan mengirim begitu banyak tentara untuk menghancurkan kotamu, sehingga debu dari puing-puing kotamu itu dapat mereka angkut habis hanya dengan tangan mereka. Ia bersumpah supaya para dewa menghukum, bahkan membunuh dia, kalau ia tidak melaksanakan hal itu!"
\par 11 Raja Ahab menjawab, "Katakan kepada Raja Benhadad: Prajurit yang sejati akan merasa bangga sesudah bertempur, bukan sebelumnya."
\par 12 Jawaban Ahab itu disampaikan kepada Benhadad pada waktu ia bersama raja-raja lain yang menjadi sekutunya sedang minum-minum di dalam perkemahan mereka. Mendengar itu Benhadad memerintahkan para perwiranya untuk bersiap-siap menyerang kota Samaria. Maka mereka pun bersiap-siap.
\par 13 Sementara itu, seorang nabi pergi kepada Raja Ahab dan berkata, "TUHAN berkata: 'Kaulihat tentara yang banyak itu? Aku akan memberi kemenangan kepadamu atas mereka hari ini. Engkau akan tahu bahwa Akulah TUHAN.'"
\par 14 "Siapa yang harus menyerang lebih dahulu?" tanya Ahab. Nabi itu menjawab, "Prajurit-prajurit muda, pengiring para kepala daerah. Begitulah kata TUHAN." "Dan siapa yang harus memimpin penyerbuan itu?" tanya raja lagi. "Baginda sendiri," jawab nabi itu.
\par 15 Maka raja pun mengerahkan prajurit-prajurit muda pengiring para kepala daerah. Semuanya ada 232 prajurit. Kemudian ia melakukan yang sama dengan tentara Israel. Seluruhnya ada 7.000 orang.
\par 16 Penyerbuan dimulai siang hari pada waktu Benhadad dan ketiga puluh dua raja yang menjadi sekutunya itu sedang minum-minum sampai mabuk di dalam perkemahan mereka.
\par 17 Yang maju lebih dahulu adalah prajurit-prajurit muda itu. Orang-orang yang disuruh oleh Benhadad untuk mengintai musuh, melaporkan kepada Benhadad bahwa sekelompok tentara sedang mendatangi mereka dari arah Samaria.
\par 18 Benhadad memerintahkan supaya tentara musuh itu ditangkap hidup-hidup, tidak peduli apakah mereka datang untuk berperang atau untuk berdamai.
\par 19 Prajurit-prajurit muda yang dikerahkan oleh Raja Ahab itu mulai bergerak dari kota Samaria, diikuti oleh tentara Israel.
\par 20 Mereka masing-masing membunuh musuh yang dihadapinya. Orang-orang Siria lari, dikejar oleh orang Israel. Tetapi Benhadad lolos. Ia melarikan diri dengan kuda disertai beberapa dari tentara berkudanya.
\par 21 Raja Ahab juga terjun ke medan pertempuran lalu menghancurkan kereta-kereta perang dan membunuh kuda-kudanya. Pada hari itu orang Siria mengalami kekalahan besar.
\par 22 Kemudian nabi itu datang kepada Ahab dan berkata, "Sekarang, sebaiknya tuan pulang dan memperkuat diri serta membuat rencana yang matang, karena raja Siria akan datang menyerang lagi pada musim semi yang berikut."
\par 23 Perwira-perwira Raja Benhadad berkata kepada Benhadad, "Ilah-ilah yang disembah bangsa Israel berkuasa di pegunungan, itu sebabnya kita dapat dikalahkan oleh bangsa itu. Tapi kita pasti dapat mengalahkan mereka jika kita berperang di dataran rendah.
\par 24 Baiklah sekarang Baginda menggantikan ketiga puluh dua raja itu dengan perwira-perwira Baginda sendiri.
\par 25 Kemudian hendaklah Baginda mengerahkan pasukan sebesar pasukan Baginda yang telah dikalahkan itu, disertai kuda dan kereta perang sebanyak yang dahulu juga. Kita akan memerangi bangsa Israel itu di dataran rendah, dan kali ini kita pasti menang." Raja Benhadad setuju, lalu mengikuti nasihat perwira-perwiranya itu.
\par 26 Pada musim semi berikutnya ia mengerahkan tentaranya dan bergerak ke kota Afek untuk menyerang orang Israel.
\par 27 Orang Israel pun memperlengkapi tentaranya dan mengerahkan mereka. Lalu tentara Israel itu bergerak maju dalam dua pasukan dan berkemah berhadapan dengan orang-orang Siria itu. Dibandingkan dengan tentara Siria yang tersebar luas memenuhi daerah itu, pasukan Israel kelihatan seperti dua kawanan kambing saja.
\par 28 Lalu datanglah seorang nabi kepada Raja Ahab dan berkata, "Inilah yang dikatakan TUHAN: 'Karena orang-orang Siria itu menduga bahwa Aku hanya berkuasa di pegunungan dan tidak di dataran rendah, maka Aku akan memberi kemenangan kepadamu atas tentara mereka yang luar biasa besarnya itu. Engkau dan rakyatmu akan tahu bahwa Akulah TUHAN.'"
\par 29 Selama tujuh hari orang Siria dan orang Israel tetap di dalam perkemahan mereka, berhadap-hadapan. Pada hari ketujuh mulailah mereka melancarkan serangan. Dalam pertempuran itu orang Israel menewaskan 100.000 orang Siria.
\par 30 Orang-orang yang luput, lari ke kota Afek, tetapi tembok kota itu runtuh menimpa 27.000 orang di antara mereka. Benhadad juga melarikan diri ke kota itu dan bersembunyi di sebuah rumah di kamar bagian dalam.
\par 31 Para perwiranya datang kepadanya dan berkata, "Kata orang, raja-raja Israel adalah raja-raja yang berbelaskasihan. Karena itu, izinkanlah kami menghadap raja Israel dengan memakai kain karung sebagai tanda penyesalan dan tali di leher sebagai tanda takluk. Mudah-mudahan ia tidak membunuh Baginda."
\par 32 Maka mereka pun memakai kain karung dan tali di leher mereka, lalu pergi menghadap Ahab dan berkata, "Benhadad adalah hamba Baginda. Ia mohon supaya Baginda jangan membunuh dia." Ahab menjawab, "Masih hidupkah dia, saudara saya itu?"
\par 33 Memang para perwira Benhadad itu sedang menunggu tanda yang baik dari Ahab. Jadi, ketika Ahab berkata "saudara", mereka langsung menjawab, "Betul, Benhadad adalah saudara Baginda." Lalu kata Ahab, "Antarkan dia ke mari!" Ketika Benhadad tiba, Ahab mengajak dia naik ke kereta.
\par 34 Lalu kata Benhadad kepadanya, "Saya akan mengembalikan kepada Tuan kota-kota yang telah direbut ayah saya dari ayah Tuan. Tuan boleh juga mendirikan pusat perdagangan di kota Damsyik seperti yang dilakukan ayah saya di Samaria." Sahut Ahab, "Atas dasar janji Tuan itu, saya akan membebaskan Tuan." Setelah itu Ahab membuat perjanjian dengan Benhadad lalu melepaskan dia.
\par 35 Atas perintah TUHAN, seorang nabi dari antara sekelompok nabi menyuruh rekannya memukul dia. Tetapi rekannya itu tidak mau.
\par 36 Lalu nabi itu berkata, "Karena engkau tidak mentaati perintah TUHAN, engkau akan diterkam singa segera sesudah pulang dari sini." Benar. Segera sesudah ia pergi dari situ, ia diterkam oleh seekor singa.
\par 37 Kemudian nabi itu pergi kepada orang lain dan berkata, "Pukullah saya!" Maka orang itu memukulnya dengan keras sekali sampai luka.
\par 38 Setelah itu nabi itu membalut mukanya dengan sepotong kain, supaya tidak dikenali orang, lalu ia pergi berdiri di pinggir jalan untuk menunggu raja Israel lewat.
\par 39 Ketika raja lewat, nabi itu berseru kepadanya, "Paduka Yang Mulia, pada waktu saya sedang bertempur di medan perang, seorang prajurit membawa kepada saya seorang musuh yang tertangkap. Prajurit itu berkata, 'Jagalah orang ini baik-baik. Kalau ia lari, kau harus dibunuh menggantikan dia atau membayar 3.000 uang perak.'
\par 40 Tetapi kemudian ketika saya sedang sibuk, orang itu melarikan diri." Raja menjawab, "Engkau sudah menjatuhkan hukumanmu sendiri. Kau harus menanggungnya."
\par 41 Nabi itu membuka kain pembalut dari mukanya, lalu raja mengenali dia sebagai seorang nabi.
\par 42 Maka berkatalah nabi itu kepada raja, "Inilah yang dikatakan TUHAN: 'Orang yang Kuperintahkan supaya dibunuh telah kaubiarkan lolos. Karena itu kau harus mati sebagai gantinya. Tentaramu akan hancur karena telah membiarkan tentara musuh melarikan diri.'"
\par 43 Raja pulang ke Samaria dengan hati yang kesal dan marah.

\chapter{21}

\par 1 Dekat istana Raja Ahab di Yizreel ada sebidang kebun anggur kepunyaan orang yang bernama Nabot.
\par 2 Suatu hari Ahab berkata kepada Nabot, "Berikanlah kebun anggurmu itu kepadaku, sebab kebun itu dekat dengan istanaku. Aku ingin menanam sayur-sayuran di situ. Kau akan mendapat kebun anggur yang lebih baik sebagai gantinya, atau kalau kau mau, aku akan membelinya dengan harga yang layak."
\par 3 Nabot menjawab, "Kebun anggur ini pusaka nenek moyang saya. Demi Allah, saya tidak boleh memberikannya kepada Tuan!"
\par 4 Dengan hati yang kesal dan marah karena mendengar apa yang dikatakan Nabot kepadanya, Ahab pulang lalu berbaring di tempat tidurnya dengan memalingkan mukanya dan tak mau makan.
\par 5 Izebel istrinya pergi kepadanya dan bertanya, "Mengapa engkau kesal? Mengapa tak mau makan?"
\par 6 Ahab menjawab, "Saya tidak senang dengan Nabot. Saya minta kepadanya supaya ia menjual kebun anggurnya kepada saya, atau saya menukarnya dengan kebun anggur yang lain, jika itulah yang dikehendakinya, tetapi ia tidak mau!"
\par 7 "Kau kan raja di Israel!" sahut Izebel. "Bangunlah sekarang dan makanlah. Senangkanlah hatimu, sebab saya akan memberikan kebun anggur Nabot itu kepadamu!"
\par 8 Maka Izebel menulis surat atas nama Ahab dan membubuhi segel raja pada surat itu. Surat itu dikirimnya kepada pemimpin-pemimpin dan tokoh-tokoh masyarakat di Yizreel.
\par 9 Inilah isi surat itu, "Umumkanlah supaya rakyat berpuasa dan suruhlah mereka berkumpul! Dalam pertemuan itu berikanlah kepada Nabot tempat duduk yang terhormat.
\par 10 Suruhlah dua orang jahat duduk menghadapnya dan menuduh dia bahwa ia telah mengutuk Allah dan raja. Lalu bawalah dia ke luar kota, dan lemparilah dia dengan batu sampai mati!"
\par 11 Perintah Izebel itu dilaksanakan oleh para pemimpin dan tokoh-tokoh masyarakat Yizreel.
\par 12 Mereka mengumumkan supaya rakyat berpuasa. Kemudian mereka menyuruh orang berkumpul dan Nabot diberi tempat yang terhormat.
\par 13 Dua penjahat duduk berhadapan dengan Nabot, dan di depan umum mereka menuduh dia bahwa ia telah mengutuki Allah dan raja. Karena itu ia dibawa ke luar kota lalu dilempari dengan batu sampai mati.
\par 14 Berita tentang pelaksanaan pembunuhan Nabot disampaikan kepada Izebel.
\par 15 Segera setelah Izebel menerima berita itu berkatalah ia kepada Ahab, "Nabot sudah mati. Sekarang pergilah ambil kebun anggur itu yang tidak mau dijualnya kepadamu."
\par 16 Ahab segera pergi dan mengambil kebun anggur itu menjadi miliknya.
\par 17 Lalu kata TUHAN kepada Elia, nabi dari Tisbe itu,
\par 18 "Ahab, raja Samaria itu sekarang ada di kebun anggur Nabot hendak mengambil kebun itu menjadi miliknya. Jadi, pergilah ke sana,
\par 19 dan sampaikan kepadanya bahwa Aku, TUHAN, berkata begini, 'Sudah membunuh, merampas lagi! Karena itu darahmu akan dijilat anjing di tempat anjing-anjing menjilat darah Nabot!'"
\par 20 Ketika Ahab melihat Elia, ia berkata, "Hai musuh, akhirnya kau mendapat aku!" "Benar," sahut Elia. "Karena Tuan dengan tekad melakukan yang jahat pada pemandangan TUHAN,
\par 21 maka inilah yang dikatakan TUHAN kepada Tuan, 'Aku akan mendatangkan bencana ke atasmu. Kau akan Kusingkirkan, dan setiap orang laki-laki dalam keluargamu, tua dan muda, akan Kulenyapkan.
\par 22 Keluargamu akan menjadi seperti keluarga Raja Yerobeam anak Nebat dan seperti keluarga Raja Baesa anak Ahia, karena engkau sudah menyebabkan orang Israel berbuat dosa sehingga membangkitkan kemarahan-Ku.'
\par 23 Dan mengenai Izebel, TUHAN berkata bahwa badannya akan dimakan anjing di dalam kota Yizreel.
\par 24 Siapa saja dari anggota keluargamu yang mati di kota akan dimakan anjing, dan yang mati di luar kota akan dimakan burung."
\par 25 Tidak pernah ada orang yang dengan tekad melakukan yang jahat pada pemandangan TUHAN seperti yang dilakukan oleh Ahab. Semua kejahatan itu dilakukannya atas dorongan Izebel, istrinya.
\par 26 Ahab melakukan dosa-dosa yang sangat hina: ia menyembah berhala seperti yang dilakukan orang Amori, yaitu orang-orang yang telah diusir TUHAN keluar dari negeri Kanaan pada waktu orang Israel memasuki negeri itu.
\par 27 Setelah Elia selesai berbicara, Ahab menyobek pakaiannya dan memakai kain karung sebagai tanda penyesalan, lalu berpuasa. Pada waktu tidur pun ia memakai kain karung, dan kalau berjalan, mukanya murung terus.
\par 28 TUHAN berkata kepada Nabi Elia,
\par 29 "Sudahkah kaulihat bagaimana Ahab merendahkan diri di hadapan-Ku? Karena itu, Aku tidak akan mendatangkan bencana selama ia masih hidup. Pada masa anaknya barulah Aku mendatangkan bencana itu ke atas keluarganya."

\chapter{22}

\par 1 Selama dua tahun lebih tidak ada perang antara Israel dan Siria.
\par 2 Tetapi pada tahun ketiga, Yosafat raja Yehuda pergi mengunjungi Ahab raja Israel.
\par 3 Sebelum itu Ahab sudah berkata kepada para perwiranya, "Kalian mengetahui bahwa kota Ramot di Gilead itu milik kita! Mengapa kita tidak merebutnya kembali dari raja Siria?"
\par 4 Maka ketika Yosafat datang, Ahab bertanya, "Maukah Anda pergi bersama aku menyerang Ramot?" Yosafat menjawab, "Baik! Kita pergi bersama-sama. Tentara dan pasukan berkudaku akan bergabung dengan tentara dan pasukan berkuda Anda.
\par 5 Tetapi sebaiknya kita tanyakan dulu kepada TUHAN."
\par 6 Maka Ahab mengumpulkan kira-kira 400 nabi lalu bertanya kepada mereka, "Bolehkah aku pergi menyerang Ramot atau tidak?" "Boleh!" jawab mereka. "TUHAN akan menyerahkan kota itu kepada Baginda."
\par 7 Tetapi Yosafat bertanya lagi, "Apakah di sini tidak ada nabi lain yang dapat bertanya kepada TUHAN untuk kita?"
\par 8 Ahab menjawab, "Masih ada satu, Mikha anak Yimla. Tapi aku benci kepadanya, sebab tidak pernah ia meramalkan sesuatu yang baik untuk aku; selalu yang tidak baik." "Ah, jangan berkata begitu!" sahut Yosafat.
\par 9 Maka Ahab memanggil seorang pegawai istana lalu menyuruh dia segera pergi menjemput Mikha.
\par 10 Pada waktu itu Ahab dan Yosafat, dengan pakaian kebesaran, duduk di kursi kerajaan di tempat pengirikan gandum depan pintu gerbang Samaria, sementara para nabi datang menghadap dan menyampaikan ramalan mereka.
\par 11 Salah seorang dari nabi-nabi itu, yang bernama Zedekia anak Kenaana, membuat tanduk-tanduk besi lalu berkata kepada Ahab, "Inilah yang dikatakan TUHAN, 'Dengan tanduk-tanduk seperti ini Baginda akan menghantam Siria dan menghancurkan mereka.'"
\par 12 Semua nabi yang lain setuju dan berkata, "Serbulah Ramot, Baginda akan berhasil. TUHAN akan memberi kemenangan kepada Baginda."
\par 13 Sementara itu utusan yang menjemput Mikha, berkata kepada Mikha, "Semua nabi yang lain meramalkan kemenangan untuk raja. Kiranya Bapak juga meramalkan yang baik seperti mereka."
\par 14 Tetapi Mikha menjawab, "Demi TUHAN yang hidup, aku hanya akan mengatakan apa yang dikatakan TUHAN kepadaku!"
\par 15 Setelah Mikha tiba di depan Raja Ahab, raja bertanya, "Bolehkah aku dan Raja Yosafat pergi menyerang Ramot, atau tidak?" "Seranglah!" sahut Mikha. "Tentu Baginda akan berhasil. TUHAN akan memberi kemenangan kepada Baginda."
\par 16 Ahab menjawab, "Kalau kau berbicara kepadaku demi nama TUHAN, katakanlah yang benar. Berapa kali engkau harus kuperingatkan tentang hal itu?"
\par 17 Mikha membalas, "Aku melihat tentara Israel kucar-kacir di gunung-gunung. Mereka seperti domba tanpa gembala, dan TUHAN berkata tentang mereka, 'Orang-orang ini tidak mempunyai pemimpin. Biarlah mereka pulang dengan selamat.'"
\par 18 Lalu kata Ahab kepada Yosafat, "Benar kataku, bukan? Tidak pernah ia meramalkan yang baik untuk aku! Selalu yang jelek saja!"
\par 19 Mikha berkata lagi, "Sekarang dengarkan apa yang dikatakan TUHAN! Aku melihat TUHAN duduk di tahta-Nya di surga, dan semua malaikat-Nya berdiri di dekat-Nya.
\par 20 TUHAN bertanya, 'Siapa akan membujuk Ahab supaya ia mau pergi berperang dan tewas di Ramot di Gilead?' Jawaban malaikat-malaikat itu berbeda-beda.
\par 21 Akhirnya tampillah suatu roh. Ia mendekati TUHAN dan berkata, 'Akulah yang akan membujuk dia.'
\par 22 'Bagaimana caranya?' tanya TUHAN. Roh itu menjawab, 'Aku akan pergi dan membuat semua nabi Ahab membohong.' TUHAN berkata, 'Baik, lakukanlah itu, engkau akan berhasil membujuk dia.'"
\par 23 Selanjutnya Mikha berkata, "Nah, itulah yang terjadi! TUHAN telah membuat nabi-nabi Baginda berdusta kepada Baginda sebab TUHAN sudah menentukan untuk menimpakan bencana kepada Baginda!"
\par 24 Maka majulah Nabi Zedekia mendekati Mikha lalu menampar mukanya dan berkata, "Mana mungkin Roh TUHAN meninggalkan aku dan berbicara kepadamu?"
\par 25 Mikha menjawab, "Nanti kaulihat buktinya pada waktu engkau masuk ke sebuah kamar untuk bersembunyi."
\par 26 "Tangkap dia!" perintah Raja Ahab, "dan bawa dia kepada Amon, wali kota, dan kepada Pangeran Yoas.
\par 27 Suruh mereka memasukkan dia ke dalam penjara, dan memberi dia makan dan minum sedikit saja sampai aku kembali dengan selamat."
\par 28 Kata Mikha, "Kalau Baginda kembali dengan selamat, berarti TUHAN tidak berbicara melalui saya! Semua yang hadir di sini menjadi saksi."
\par 29 Kemudian Ahab raja Israel, dan Yosafat raja Yehuda pergi menyerang kota Ramot di Gilead.
\par 30 Ahab berkata kepada Yosafat, "Aku akan menyamar dan ikut bertempur, tetapi Anda hendaklah memakai pakaian kebesaranmu." Demikianlah raja Israel menyamar ketika pergi bertempur.
\par 31 Pada waktu itu ketiga puluh dua panglima pasukan kereta perang Siria telah diperintahkan oleh rajanya untuk menyerang hanya raja Israel.
\par 32 Jadi, ketika mereka melihat Raja Yosafat, mereka semua menyangka ia raja Israel. Karena itu mereka menyerang dia. Tetapi Yosafat berteriak,
\par 33 maka mereka pun menyadari bahwa ia bukan raja Israel. Lalu mereka berhenti menyerang dia.
\par 34 Secara kebetulan seorang prajurit Siria melepaskan anak panahnya, tanpa mengarahkannya ke sasaran tertentu. Tetapi anak panah itu mengenai Ahab dan menembus baju perangnya pada bagian sambungannya. "Aku kena!" seru Ahab kepada pengemudi keretanya. "Putar dan bawalah aku keluar dari pertempuran!"
\par 35 Tapi karena pertempuran masih berkobar, Raja Ahab tetap berdiri sambil ditopang dalam keretanya menghadap tentara Siria. Darahnya mengalir dari lukanya, menggenangi lantai kereta. Petang harinya ia meninggal.
\par 36 Menjelang matahari terbenam seluruh pasukan Israel diperintahkan untuk pulang ke kota dan ke daerahnya masing-masing,
\par 37 karena raja sudah meninggal. Lalu pulanglah mereka dan menguburkan jenazah Ahab di Samaria.
\par 38 Ketika keretanya dibersihkan di kolam Samaria, wanita-wanita pelacur sedang mandi di situ, dan anjing menjilat darah di kereta itu, tepat seperti yang telah dikatakan TUHAN.
\par 39 Kisah lainnya mengenai Raja Ahab, mengenai istana gading dan semua kota yang didirikannya, sudah dicatat dalam buku Sejarah Raja-raja Israel.
\par 40 Setelah Ahab meninggal, Ahazia anaknya menjadi raja menggantikan dia.
\par 41 Pada tahun keempat pemerintahan Ahab raja Israel, Yosafat anak Asa menjadi raja atas Yehuda.
\par 42 Pada waktu itu ia berumur tiga puluh lima tahun. Ia memerintah di Yerusalem dua puluh lima tahun lamanya. Ibunya ialah Azuba anak Silhi.
\par 51 Seperti Asa, ayahnya, Yosafat melakukan apa yang baik pada pemandangan TUHAN. Ia melenyapkan dari kerajaannya semua pelacur laki-laki dan perempuan yang bertugas di tempat-tempat penyembahan berhala yang masih tertinggal dari zaman Asa, ayahnya. Tapi tempat-tempat penyembahan itu sendiri tidak dihancurkannya. Rakyat masih saja mempersembahkan kurban dan kemenyan di sana. Yosafat membuat kapal-kapal Tarsis untuk mengambil emas dari Ofir, tetapi kapal-kapal itu tidak jadi berlayar karena rusak di Ezion-Geber. Raja Ahazia dari Israel menawarkan supaya awak kapalnya berlayar bersama-sama dengan awak kapal Yosafat, tetapi Yosafat menolak meskipun ia mempunyai hubungan yang baik dengan raja Israel. Kemudian Yosafat meninggal dan dimakamkan di pekuburan raja-raja di kota Daud. Yoram anaknya menjadi raja menggantikan dia. Kisah lainnya mengenai Yosafat, mengenai kepahlawanannya dan pertempuran-pertempurannya dicatat dalam buku Sejarah Raja-raja Yehuda. Pada zaman itu negeri Edom tidak mempunyai raja. Yang memerintah di sana adalah seorang kepala daerah.
\par 52 Pada tahun ketujuh belas pemerintahan Raja Yosafat dari Yehuda, Ahazia anak Ahab menjadi raja Israel lalu memerintah Israel selama dua tahun di Samaria.
\par 53 Ia berdosa kepada TUHAN dengan menuruti jejak ayahnya dan ibunya serta jejak Raja Yerobeam yang menyebabkan orang Israel berbuat dosa.
\par 54 Ia menyembah Baal dan mengabdi kepadanya. Dan seperti ayahnya yang memerintah sebelumnya, ia pun membangkitkan kemarahan TUHAN, Allah yang disembah oleh orang Israel.


\end{document}