\begin{document}

\title{2 Chronicles}

2Ch 1:1  Salomo putra Raja Daud telah menjadi raja yang kuat kedudukannya di dalam kerajaan Israel. TUHAN Allahnya memberkati dia dan menjadikan dia sangat berkuasa.
2Ch 1:2  Semua kepala pasukan 1.000 dan pasukan 100, semua pejabat pemerintah, dan semua kepala keluarga Israel diperintahkan oleh Raja Salomo
2Ch 1:3  untuk pergi dengan dia ke tempat ibadat di Gibeon. Mereka pergi ke sana, karena di situlah terdapat Kemah TUHAN yang dibuat oleh Musa hamba TUHAN sewaktu di padang gurun.
2Ch 1:4  (Tetapi Peti Perjanjian TUHAN berada di Yerusalem di dalam kemah yang dipasang oleh Raja Daud untuk Peti itu ketika ia memindahkannya dari Kiryat-Yearim.)
2Ch 1:5  Mezbah perunggu yang dibuat oleh Bezaleel anak Uri, cucu Hur, juga terdapat di Gibeon di depan Kemah TUHAN. Raja Salomo dan seluruh rakyat beribadat kepada TUHAN di sana.
2Ch 1:6  Di depan Kemah itu raja mempersembahkan di atas mezbah perunggu 1.000 binatang sebagai kurban bakaran bagi TUHAN.
2Ch 1:7  Malam itu Allah menampakkan diri kepada Salomo dan berkata, "Salomo, mintalah apa yang kauinginkan, itu akan Kuberikan kepadamu!"
2Ch 1:8  Salomo menjawab, "Ya Allah, Engkau sudah menunjukkan bahwa Engkau sangat mengasihi ayahku Daud, dan Engkau telah mengangkat aku menjadi raja menggantikan dia.
2Ch 1:9  Sekarang, ya TUHAN Allah, semoga Engkau menepati janji-Mu kepada ayahku. Engkau telah mengangkat aku menjadi raja atas bangsa besar ini yang tak terhitung jumlahnya.
2Ch 1:10  Karena itu berikanlah kiranya kepadaku kebijaksanaan dan pengetahuan yang kuperlukan untuk memimpin mereka. Kalau tidak demikian, bagaimana mungkin aku dapat memerintah umat-Mu yang besar ini?"
2Ch 1:11  Jawab Allah kepada Salomo, "Permintaanmu itu baik. Engkau tidak minta kekayaan atau harta atau kemasyhuran atau kematian musuh-musuhmu atau umur panjang untuk dirimu sendiri, melainkan kebijaksanaan dan pengetahuan untuk memerintah umat-Ku yang Kupercayakan kepadamu.
2Ch 1:12  Sebab itu kebijaksanaan dan pengetahuan akan Kuberikan kepadamu. Selain itu Aku akan menjadikan engkau lebih makmur, lebih kaya dan lebih masyhur daripada raja yang mana pun juga baik pada masa lalu maupun pada masa yang akan datang."
2Ch 1:13  Kemudian Salomo meninggalkan tempat ibadah di Gibeon di mana terdapat Kemah TUHAN, lalu kembali ke Yerusalem. Dari situ ia menjalankan pemerintahan atas Israel.
2Ch 1:14  Ia melengkapi angkatan perangnya dengan 1.400 kereta perang dan 12.000 kuda perang. Sebagian ditempatkannya di Yerusalem, dan selebihnya di berbagai kota lain.
2Ch 1:15  Dalam zaman pemerintahan Salomo, emas dan perak merupakan barang biasa di Yerusalem, sama seperti batu; dan kayu cemara Libanon banyak sekali seperti kayu ara biasa.
2Ch 1:16  Melalui pedagang-pedagang yang membantu raja, kuda diimpor dari Mesir dan Kewe,
2Ch 1:17  dan kereta-kereta perang diimpor dari Mesir, kemudian diekspor lagi kepada raja-raja Het dan raja-raja Siria dengan harga 150 uang perak untuk satu ekor kuda, dan 600 uang perak untuk satu kereta perang.
2Ch 2:1  Raja Salomo telah memutuskan untuk membangun Rumah TUHAN, dan sebuah istana bagi dirinya.
2Ch 2:2  Untuk itu ia mengerahkan 70.000 orang untuk mengangkut bahan bangunan, 80.000 orang untuk memahat batu di pegunungan dan 3.600 mandur untuk mengawasi pekerjaan itu.
2Ch 2:3  Salomo mengirim berita ini kepada Hiram raja Tirus, "Hendaknya Tuan mengadakan hubungan dagang dengan aku seperti dengan Daud ayahku. Tuan telah menjual kepadanya kayu cemara Libanon untuk pembangunan istananya.
2Ch 2:4  Sekarang aku mau membangun sebuah rumah ibadat yang khusus untuk menghormati TUHAN Allahku. Di rumah itu aku dan rakyatku akan beribadat kepada-Nya dengan membakar dupa wangi-wangian, serta mempersembahkan roti sajian bagi-Nya secara tetap. Di rumah itu juga kami akan mempersembahkan kurban bakaran setiap pagi dan malam, setiap hari Sabat, hari raya Bulan Baru dan hari-hari khusus lainnya untuk menghormati TUHAN Allah kami. Sebab, Ia telah memerintahkan agar hal itu dilaksanakan untuk selama-lamanya oleh orang Israel.
2Ch 2:5  Aku ingin agar Rumah TUHAN yang akan kubangun itu merupakan rumah yang agung, sebab Allah kami lebih agung daripada ilah yang mana pun juga.
2Ch 2:6  Tapi sebenarnya tidak seorang pun mampu mendirikan rumah bagi Allah, sebab seluruh angkasa raya tidak cukup besar untuk menjadi tempat tinggal-Nya. Jadi, Rumah TUHAN yang dapat kudirikan itu paling-paling hanya tempat untuk membakar dupa bagi Allah!
2Ch 2:7  Karena itu, hendaknya Tuan mengirim kepada kami seorang yang pandai membuat ukiran dan dapat mengerjakan emas, perak, perunggu, besi, juga pandai mengerjakan kain biru, ungu dan merah. Ia akan bekerja sama dengan para pengrajin di Yehuda dan Yerusalem yang telah dipilih oleh ayahku Daud.
2Ch 2:8  Aku tahu bahwa para penebang kayu di negeri Tuan pandai menebang pohon, sebab itu hendaknya Tuan mengirim kepadaku bermacam-macam kayu cemara dan kayu cendana dari Libanon. Aku akan mengerahkan orang-orangku untuk membantu orang-orang Tuan,
2Ch 2:9  supaya mereka menyiapkan sejumlah besar kayu. Sebab Rumah TUHAN yang hendak kubangun itu haruslah besar dan hebat.
2Ch 2:10  Untuk perbekalan orang-orang Tuan itu akan kukirim kepada Tuan dua jenis gandum, masing-masing 2.000 ton, anggur sebanyak 400.000 liter dan minyak zaitun 400.000 liter."
2Ch 2:11  Sebagai balasan atas surat Raja Salomo itu, Raja Hiram mengirim surat yang berbunyi begini, "Karena TUHAN mengasihi umat-Nya, Ia telah mengangkat Tuan menjadi raja mereka.
2Ch 2:12  Terpujilah TUHAN Allah Israel, Pencipta langit dan bumi! Ia memberikan kepada Daud seorang putra yang sangat bijaksana lagi cerdas, dan berpengetahuan, yang kini sedang berusaha mendirikan sebuah rumah bagi TUHAN dan sebuah istana bagi dirinya.
2Ch 2:13  Bersama ini aku mengirim kepada Tuan seorang tenaga ahli yang pandai dan trampil, bernama Huram.
2Ch 2:14  Ibunya dari suku Dan, ayahnya orang Tirus. Ia pandai membuat barang-barang dari emas, perak, perunggu, besi, batu dan kayu. Ia bisa mengerjakan kain biru, ungu, merah, dan juga kain lenan. Segala macam pekerjaan ukiran dapat dibuatnya menurut model apa saja yang diminta. Baiklah ia ditugaskan untuk bekerja sama dengan para pengrajin Tuan, dan dengan orang-orang yang telah bekerja untuk Raja Daud ayah Tuan.
2Ch 2:15  Kami menantikan kiriman kedua jenis gandum, anggur dan minyak zaitun yang Tuan janjikan itu.
2Ch 2:16  Kami akan menebang kayu cemara dari pegunungan Libanon sebanyak yang Tuan perlukan, lalu mengikatnya menjadi rakit, dan menghanyutkannya melalui laut sampai ke Yope. Dari sana Tuan dapat mengangkutnya ke Yerusalem."
2Ch 2:17  Raja Salomo mengadakan sensus terhadap semua orang asing yang tinggal di Israel, seperti yang pernah diadakan oleh Raja Daud ayahnya. Ternyata ada 153.600 orang asing.
2Ch 2:18  Tujuh puluh ribu dari mereka ditugaskannya untuk mengangkut bahan-bahan bangunan, 80.000 untuk memahat batu di pegunungan, dan 3.600 untuk mengawasi dan bertanggung jawab atas pekerjaan itu.
2Ch 3:1  Raja Daud sudah menentukan bahwa Rumah TUHAN harus dibangun di Yerusalem di Gunung Moria di tempat pengirikan gandum milik Arauna orang Yebus. Pembangunan Rumah TUHAN itu dimulai oleh Raja Salomo
2Ch 3:2  pada bulan kedua tahun keempat pemerintahannya.
2Ch 3:3  Panjang Rumah itu 27 meter dan lebarnya 9 meter.
2Ch 3:4  Panjang ruang depan sama dengan lebar Rumah itu, yaitu 9 meter. Tingginya 54 meter. Bagian dalam ruangan itu dilapisi dengan emas murni.
2Ch 3:5  Ruangan utama dipapani dengan kayu cemara yang terbaik dan dilapisi dengan emas tua. Pada lapisan emas itu ada ukiran gambar pohon palem dan gambar rantai.
2Ch 3:6  Raja menghias Rumah itu dengan batu permata yang bagus-bagus dan dengan emas yang didatangkan dari negeri Parwaim.
2Ch 3:7  Ia memakai emas itu juga untuk melapisi tembok-tembok Rumah TUHAN itu, balok-baloknya, ambang-ambang pintu dan pintu-pintunya. Tembok-temboknya diukir dengan gambar kerub.
2Ch 3:8  Ruang dalam yang dinamakan Ruang Mahasuci panjangnya 9 meter dan lebarnya 9 meter, sama dengan lebar Rumah TUHAN itu. Dua puluh ton emas dipakai untuk melapisi tembok-tembok Ruang Mahasuci itu
2Ch 3:9  dan 570 gram emas untuk membuat paku. Tembok ruangan-ruangan tingkat atas pun dilapisi dengan emas.
2Ch 3:10  Raja juga menyuruh para pekerjanya itu membuat dua patung kerub dari logam. Patung-patung itu dilapisinya dengan emas dan dipasang berdampingan di Ruang Mahasuci menghadap pintu. Setiap kerub itu mempunyai dua sayap, masing-masing panjangnya 2,2 meter. Sayap-sayap itu mengembang sedemikian rupa sehingga ujung yang satu saling bersentuhan di tengah ruangan dan ujung yang lain mencapai tembok di kiri kanan. Jadi, sayap-sayap itu terbentang sepanjang 9 meter yaitu selebar ruangan itu.
2Ch 3:14  Untuk Ruang Mahasuci itu dibuat pula gorden dari kain lenan dan kain lain yang berwarna biru, ungu dan merah. Gorden itu dihias dengan gambar-gambar kerub.
2Ch 3:15  Raja membuat dua tiang yang masing-masing tingginya 15,5 meter, lalu menempatkannya di depan Rumah TUHAN. Setiap tiang itu mempunyai kepala tiang setinggi 2,2 meter.
2Ch 3:16  Bagian atas tiang-tiang itu diperindah dengan hiasan dari perunggu, berupa anyaman rantai dengan 100 buah delima.
2Ch 3:17  Tiang-tiang itu didirikan di samping kiri dan kanan pintu masuk Rumah TUHAN; yang di sebelah selatan dinamakan Yakhin, dan yang di sebelah utara dinamakan Boas.
2Ch 4:1  Raja Salomo menyuruh membuat sebuah mezbah perunggu, yang panjang dan lebarnya masing-masing 9 meter, dan tingginya 4,5 meter.
2Ch 4:2  Ia juga membuat sebuah bejana perunggu yang bundar, dengan ukuran sebagai berikut: dalamnya 2,2 meter, garis tengahnya 4,4 meter, dan lingkarannya 13,2 meter.
2Ch 4:3  Sekeliling tepi luarnya dihias dengan dua jajar gambar sapi, yang dicor bersama-sama dengan bejana itu.
2Ch 4:4  Bejana itu ditempatkan di atas punggung dua belas sapi perunggu--tiga menghadap ke utara, tiga ke selatan, tiga ke barat, dan tiga ke timur.
2Ch 4:5  Tebal bejana itu 75 milimeter. Tepinya serupa tepi cangkir yang melengkung keluar mirip bunga bakung yang mekar. Bejana itu dapat memuat kira-kira 60.000 liter.
2Ch 4:6  Raja Salomo juga membuat 10 baskom besar, lima untuk ditaruh di sebelah selatan Rumah TUHAN, dan lima lagi untuk di sebelah utaranya. Baskom-baskom itu gunanya untuk mencuci potongan-potongan binatang yang akan dibakar sebagai kurban. Air di dalam bejana besar itu disediakan bagi para imam untuk membasuh diri.
2Ch 4:7  Dibuatnya juga 10 kaki pelita dari emas menurut model yang telah direncanakan, dan 10 meja. Semua kaki pelita dan meja itu ditaruh di dalam ruang utama di Rumah TUHAN--masing-masing lima di sebelah kiri, dan lima di sebelah kanannya. Dibuatnya juga 100 mangkuk emas.
2Ch 4:9  Raja Salomo juga membuat sebuah pelataran dalam untuk para imam, dan sebuah pelataran luar. Daun pintu pada gerbang-gerbang antara kedua pelataran itu dilapisi dengan perunggu.
2Ch 4:10  Bejana ditempatkan dekat sudut tenggara Rumah TUHAN.
2Ch 4:11  Huram, kepala para pengrajin itu, membuat kuali-kuali, sekop-sekop dan mangkuk-mangkuk. Maka selesailah ia membuat segala perlengkapan Rumah TUHAN sesuai dengan janjinya kepada Raja Salomo. Inilah perlengkapan yang telah dibuatnya itu: Dua tiang besar; Dua kepala tiang berbentuk mangkuk yang ditempatkan di atas kedua tiang itu; Anyaman rantai pada setiap kepala tiang; Empat ratus delima perunggu yang disusun dalam dua jajar sekeliling anyaman rantai pada setiap kepala tiang; Sepuluh kereta; Sepuluh baskom besar; Bejana perunggu; Dua belas sapi perunggu yang menopang bejana itu; Kuali-kuali, sekop-sekop dan garpu-garpu. Semua perlengkapan itu yang dibuat oleh Huram untuk Rumah TUHAN, dibuat dari perunggu dan digosok sampai berkilap. Huram membuat semuanya itu sesuai dengan perintah Raja Salomo.
2Ch 4:17  Raja menyuruh membuat barang-barang itu di pengecoran logam antara Sukot dan Zereda di Lembah Yordan.
2Ch 4:18  Barang-barang itu begitu banyak, sehingga tak seorang pun yang tahu berapa kilogram perunggu yang dipakai.
2Ch 4:19  Raja Salomo juga menyuruh membuat perkakas-perkakas lain dari emas murni untuk Rumah TUHAN. Inilah perkakas-perkakas yang dibuatnya itu: Mezbah; Meja untuk roti sajian; Semua kaki pelita dengan pelitanya dari emas tua untuk dinyalakan di depan Ruang Mahasuci, sesuai dengan rencana; Hiasan-hiasan berbentuk bunga; Pelita-pelita; Sepit-sepit; Alat untuk membersihkan sumbu pelita; Mangkuk-mangkuk; Piring-piring untuk dupa; Piring-piring untuk bara; Semua pintu masuk Rumah TUHAN dan pintu Ruang Mahasuci dilapis dengan emas.
2Ch 5:1  Setelah Raja Salomo selesai membangun Rumah TUHAN itu, ia menyimpan di dalam gudang-gudang Rumah itu semua perak, emas, dan barang-barang lain yang telah dipersembahkan oleh Daud ayahnya kepada TUHAN.
2Ch 5:2  Setelah itu Raja Salomo menyuruh semua pemimpin suku dan kaum dari bangsa Israel berkumpul di Yerusalem untuk memindahkan Peti Perjanjian TUHAN dari Sion, Kota Daud ke Rumah TUHAN.
2Ch 5:3  Maka datanglah mereka semua pada hari raya Pondok Daun.
2Ch 5:4  Setelah semua pemimpin itu berkumpul, orang-orang Lewi mengangkat Peti Perjanjian itu,
2Ch 5:5  dan membawanya ke Rumah TUHAN. Kemah TUHAN serta semua perlengkapannya dipindahkan juga ke Rumah TUHAN oleh para imam dan orang Lewi.
2Ch 5:6  Raja Salomo dan seluruh rakyat Israel berkumpul di depan Peti Perjanjian itu lalu mempersembahkan kepada TUHAN domba dan sapi yang tak terhitung banyaknya.
2Ch 5:7  Setelah itu para imam memasukkan Peti Perjanjian itu ke dalam Rumah TUHAN dan meletakkannya di bawah patung kerub di dalam Ruang Mahasuci.
2Ch 5:8  Sayap patung-patung itu terbentang menutupi Peti itu dan kayu-kayu pengusungnya.
2Ch 5:9  Kayu pengusung itu panjang sekali, sehingga ujung-ujungnya terlihat dari depan Ruang Mahasuci, tapi tidak terlihat dari luar. (Sampai hari ini kayu-kayu itu masih di situ.)
2Ch 5:10  Di dalam Peti Perjanjian itu tidak ada sesuatu pun kecuali dua lempeng batu. Batu-batu itu telah dimasukkan ke situ oleh Musa di Gunung Sinai, ketika TUHAN membuat perjanjian dengan bangsa Israel pada waktu mereka dalam perjalanan dari Mesir ke negeri Kanaan.
2Ch 5:11  Semua imam yang ada di situ, dari semua kelompok, telah mempersiapkan diri untuk upacara itu. Semua penyanyi dan pemain musik dalam suku Lewi, yaitu Asaf, Heman, Yedutun dan seluruh anggota kaum mereka, memakai pakaian dari kain lenan. Orang-orang Lewi itu berdiri di sebelah timur mezbah dengan memegang simbal dan kecapi. Bersama mereka ada 120 orang imam yang meniup trompet. Nyanyian para biduan berpadu dengan iringan musik trompet, simbal dan alat musik lain. Inilah kata-kata pujian yang mereka nyanyikan untuk TUHAN: "Pujilah TUHAN, sebab Ia baik. Kasih-Nya kekal abadi." Pada waktu para imam keluar dari Rumah TUHAN itu, tiba-tiba Rumah itu penuh dengan cahaya kehadiran TUHAN, sehingga mereka tidak dapat meneruskan upacara ibadah.
2Ch 6:1  Raja Salomo berdoa, "Ya TUHAN, Engkau lebih suka tinggal dalam kegelapan awan.
2Ch 6:2  Kini kubangun bagi-Mu gedung yang megah, untuk tempat tinggal-Mu selama-lamanya."
2Ch 6:3  Setelah berdoa, raja berpaling kepada seluruh rakyat yang sedang berdiri di situ, lalu ia memohonkan berkat Allah bagi mereka.
2Ch 6:4  Ia berkata, "Dahulu TUHAN telah berjanji kepada ayahku Daud begini, 'Sejak Aku membawa umat-Ku ke luar dari Mesir sampai pada hari ini, di negeri Israel tidak ada satu kota pun yang Kupilih menjadi tempat di mana harus dibangun rumah untuk tempat ibadat kepada-Ku, dan tidak seorang pun yang Kupilih untuk memimpin umat-Ku Israel. Tetapi sekarang Aku memilih Yerusalem sebagai tempat ibadah kepada-Ku. Dan engkau, Daud, Kupilih untuk memerintah umat-Ku.' Terpujilah TUHAN Allah Israel yang sudah menepati janji-Nya itu!"
2Ch 6:7  Selanjutnya Salomo berkata, "Ayahku Daud telah merencanakan untuk membangun rumah tempat ibadat kepada TUHAN Allah Israel.
2Ch 6:8  Tetapi TUHAN berkata kepadanya, 'Maksudmu itu baik.
2Ch 6:9  Tetapi, bukan engkau, melainkan anakmulah yang akan membangun rumah-Ku itu.'
2Ch 6:10  Sekarang TUHAN telah menepati janji-Nya. Aku telah menjadi raja Israel menggantikan ayahku, dan aku telah pula membangun rumah untuk tempat ibadat kepada TUHAN, Allah Israel.
2Ch 6:11  Di dalam Rumah TUHAN itu telah kusediakan tempat untuk Peti Perjanjian yang berisi batu perjanjian antara TUHAN dengan umat Israel."
2Ch 6:12  Salomo telah membuat panggung perunggu di tengah-tengah halaman Rumah TUHAN. Panjangnya dan lebarnya 2,2 meter, dan tingginya 1,3 meter. Di hadapan rakyat yang hadir, Salomo menaiki panggung itu, dan berdiri menghadap mezbah. Semua orang dapat melihatnya. Kemudian ia berlutut, lalu mengangkat kedua tangannya serta menadahkannya ke langit dan berdoa,
2Ch 6:14  "TUHAN, Allah Israel! Di langit ataupun di bumi tak ada yang seperti Engkau! Engkau menepati janji-Mu dan menunjukkan kasih-Mu kepada umat-Mu yang setia dan taat dengan sepenuh hati kepada-Mu.
2Ch 6:15  Engkau telah menepati janji-Mu kepada ayahku Daud. Apa yang Kaujanjikan itu sudah Kaulaksanakan hari ini.
2Ch 6:16  Engkau juga telah berjanji kepada ayahku bahwa kalau keturunannya sungguh-sungguh taat kepada hukum-Mu seperti dia, maka untuk selama-lamanya seorang dari keturunannya akan memerintah sebagai raja di Israel. Sekarang, ya TUHAN Allah Israel, tepatilah juga semua yang telah Kaujanjikan kepada ayahku Daud, hamba-Mu itu.
2Ch 6:18  Tetapi, ya Allah, sungguhkah Engkau sudi tinggal di bumi ini di antara manusia? Langit seluruhnya pun tak cukup luas untuk-Mu, apalagi rumah ibadat yang kubangun ini!
2Ch 6:19  Ya TUHAN, Allahku, aku hamba-Mu! Namun dengarkanlah kiranya doaku, dan kabulkanlah permohonanku.
2Ch 6:20  Semoga siang malam Engkau melindungi rumah ini yang telah Kaupilih sebagai tempat ibadat kepada-Mu. Semoga dari tempat kediaman-Mu di surga Engkau mendengar dan mengampuni aku serta umat Israel, umat-Mu itu, apabila kami menghadap rumah ini dan berdoa kepada-Mu.
2Ch 6:22  Apabila seseorang dituduh bersalah terhadap orang lain dan dibawa ke mezbah-Mu di dalam Rumah-Mu ini untuk bersumpah bahwa ia tidak bersalah,
2Ch 6:23  ya TUHAN, hendaklah Engkau mendengarkan di surga dan memutuskan perkara hamba-hamba-Mu ini. Hukumlah yang bersalah dan bebaskanlah yang tidak bersalah.
2Ch 6:24  Apabila umat-Mu Israel dikalahkan oleh musuh-musuhnya karena mereka berdosa, lalu menyesal dan menghormati Engkau lagi sebagai TUHAN, kemudian datang ke Rumah-Mu ini serta berdoa mohon ampun kepada-Mu,
2Ch 6:25  semoga Engkau mendengarkan mereka di surga. Semoga Engkau mengampuni dosa umat-Mu ini, dan membawa mereka kembali ke negeri yang telah Kauberikan kepada mereka dan kepada leluhur mereka.
2Ch 6:26  Apabila umat-Mu berdosa kepada-Mu dan Engkau menghukum mereka dengan tidak menurunkan hujan, lalu mereka bertobat dari dosa mereka dan menghormati Engkau sebagai TUHAN, kemudian menghadap ke Rumah-Mu ini serta berdoa kepada-Mu,
2Ch 6:27  ya TUHAN di surga, dengarkanlah mereka. Dan ampunilah dosa hamba-hamba-Mu umat Israel. Ajarlah mereka melakukan apa yang benar. Setelah itu, ya TUHAN, turunkanlah hujan ke negeri-Mu ini, negeri yang Kauberikan kepada umat-Mu untuk menjadi miliknya selama-lamanya.
2Ch 6:28  Apabila negeri ini dilanda kelaparan atau wabah, atau tanaman-tanaman dirusak oleh angin panas, hama atau serangga belalang, atau apabila umat-Mu diserang musuh, atau diserang penyakit,
2Ch 6:29  semoga Engkau mendengarkan doa mereka. Kalau dari antara umat-Mu Israel ada yang dengan bersedih hati berdoa kepada-Mu sambil menengadahkan tangannya ke arah Rumah-Mu ini,
2Ch 6:30  kiranya Engkau di dalam kediaman-Mu di surga mendengar serta mengampuni mereka. Hanya Engkaulah yang mengenal isi hati manusia. Sebab itu perlakukanlah setiap orang setimpal perbuatan-perbuatannya,
2Ch 6:31  supaya umat-Mu takut dan taat kepada-Mu selalu selama mereka tinggal di negeri yang Kauberikan kepada leluhur kami.
2Ch 6:32  Apabila seorang asing di negeri yang jauh mendengar tentang keagungan dan kekuatan-Mu, dan bahwa Engkau selalu siap untuk menolong, lalu ia datang di Rumah-Mu ini untuk berdoa kepada-Mu,
2Ch 6:33  semoga dari kediaman-Mu di surga Engkau mendengarkan doanya dan mengabulkan permintaannya. Dengan demikian segala bangsa di seluruh dunia mengenal Engkau dan taat kepada-Mu seperti umat-Mu Israel. Mereka akan mengetahui bahwa rumah yang kubangun inilah tempat untuk beribadat kepada-Mu.
2Ch 6:34  Apabila Engkau memerintahkan umat-Mu untuk pergi berperang melawan musuh, lalu di mana pun mereka berada, mereka menghadap ke kota pilihan-Mu ini dan berdoa ke arah rumah ini yang telah kubangun untuk-Mu,
2Ch 6:35  semoga di surga Engkau mendengarkan doa mereka itu, dan memberikan kemenangan kepada mereka.
2Ch 6:36  Apabila umat-Mu berdosa kepada-Mu--sesungguhnya tidak ada seorang pun yang tidak berdosa--lalu karena kemarahan-Mu Kaubiarkan mereka dikalahkan oleh musuh dan dibawa sebagai tawanan ke suatu negeri yang jauh atau dekat,
2Ch 6:37  semoga Engkau mendengarkan doa mereka. Jikalau di negeri itu mereka meninggalkan dosa-dosa mereka, dan berdoa kepada-Mu sambil mengakui bahwa mereka telah berdosa dan berbuat jahat, dengarkanlah doa mereka, ya TUHAN.
2Ch 6:38  Jikalau di negeri itu mereka dengan ikhlas dan sungguh-sungguh meninggalkan dosa-dosa mereka, dan berdoa kepada-Mu sambil menghadap ke negeri ini yang Kauberikan kepada leluhur kami, ke arah kota pilihan-Mu dan rumah ibadat yang telah kubangun untuk-Mu ini,
2Ch 6:39  maka dari kediaman-Mu di surga hendaklah Engkau mendengar doa mereka dan mengasihani mereka. Ampunilah semua dosa umat-Mu.
2Ch 6:40  Kini, ya Allahku, perhatikanlah kami dan dengarkanlah doa-doa yang disampaikan kepada-Mu dari tempat ini.
2Ch 6:41  Bangkitlah sekarang, ya TUHAN Allah! Dan bersama dengan Peti Perjanjian yang melambangkan kuasa-Mu itu hendaklah Engkau memasuki rumah-Mu ini dan tinggal di sini untuk selama-lamanya. Berkatilah segala pekerjaan imam-imam-Mu, dan semoga seluruh umat-Mu berbahagia karena Engkau baik kepada mereka.
2Ch 6:42  TUHAN Allah, janganlah Kautolak raja yang telah Kaupilih ini. Ingatlah akan kasih-Mu kepada Daud hamba-Mu."
2Ch 7:1  Setelah Raja Salomo selesai berdoa, api turun dari langit membakar kurban-kurban yang dipersembahkan kepada TUHAN. Lalu cahaya kehadiran TUHAN memenuhi Rumah TUHAN itu,
2Ch 7:2  sehingga para imam tak dapat memasukinya.
2Ch 7:3  Ketika rakyat Israel melihat api turun dari langit dan cahaya kehadiran TUHAN memenuhi Rumah TUHAN, mereka tersungkur di lantai lalu menyembah serta memuji Allah, karena Ia baik dan kasih-Nya kekal abadi.
2Ch 7:4  Setelah itu Raja Salomo dan semua orang yang berkumpul di situ mempersembahkan kurban kepada TUHAN.
2Ch 7:5  Untuk kurban perdamaian Salomo mempersembahkan 22.000 sapi dan 120.000 domba. Demikianlah raja dan seluruh rakyat mengadakan upacara penyerahan rumah ibadat itu kepada TUHAN.
2Ch 7:6  Para imam berdiri pada tempat yang ditetapkan bagi mereka; berhadapan dengan orang Lewi yang memuji TUHAN dengan alat-alat musik yang dibuat oleh Raja Daud, sambil menyanyikan nyanyian "Kasih-Nya Kekal Abadi!" sesuai dengan anjuran Daud. Para imam meniup trompet sementara seluruh rakyat berdiri.
2Ch 7:7  Lalu bagian tengah dari pelataran yang di depan rumah ibadat itu diserahkan Salomo kepada TUHAN. Kemudian ia mempersembahkan di situ kurban bakaran, kurban gandum dan lemak binatang untuk kurban perdamaian. Semua itu dipersembahkannya di situ karena mezbah perunggu yang telah dibuatnya terlalu kecil untuk semua persembahan itu.
2Ch 7:8  Salomo dan seluruh umat Israel mengadakan perayaan Pondok Daun selama tujuh hari. Sangat besar jumlah orang yang berkumpul untuk perayaan itu. Mereka datang dari daerah sejauh Jalan Hamat di utara sampai ke perbatasan Mesir di selatan.
2Ch 7:9  Tujuh hari lamanya mereka berkumpul untuk upacara peresmian mezbah TUHAN, kemudian tujuh hari lagi untuk perayaan Pondok Daun. Pada hari terakhir diadakan perayaan penutupan,
2Ch 7:10  dan keesokan harinya, pada tanggal 23 bulan tujuh, Salomo menyuruh orang-orang itu pulang. Mereka gembira atas semua yang baik yang dilakukan TUHAN kepada bangsa Israel umat-Nya, kepada Daud dan kepada Salomo.
2Ch 7:11  Setelah Salomo berhasil menyelesaikan pembangunan Rumah TUHAN dan istana raja sesuai dengan rencananya,
2Ch 7:12  TUHAN menampakkan diri lagi kepadanya pada waktu malam. TUHAN berkata kepadanya, "Doamu sudah Kudengar dan Aku menerima rumah ini sebagai tempat yang khusus untuk mempersembahkan kurban bagi-Ku.
2Ch 7:13  Apabila Aku tidak menurunkan hujan atau Aku mengirim belalang untuk menghabiskan hasil bumi atau mendatangkan wabah penyakit ke atas umat-Ku,
2Ch 7:14  lalu umat-Ku yang memakai nama-Ku itu merendahkan diri, dan berdoa serta datang kepada-Ku dan meninggalkan perbuatan mereka yang jahat, maka dari surga Aku akan mendengar doa mereka. Aku akan mengampuni dosa-dosa mereka dan menjadikan negeri mereka makmur kembali.
2Ch 7:15  Aku akan terus memperhatikan Rumah-Ku ini dan siap mendengar semua doa yang dipanjatkan dari tempat ini.
2Ch 7:16  Semuanya itu akan Kulakukan karena tempat ini telah Kupilih dan Kukhususkan menjadi tempat ibadat kepada-Ku untuk selama-lamanya. Aku akan memperhatikannya dan melindunginya sepanjang masa.
2Ch 7:17  Jikalau engkau, seperti Daud ayahmu, hidup menurut kehendak-Ku dan taat kepada hukum-hukum-Ku serta melakukan semua yang Kuperintahkan kepadamu,
2Ch 7:18  maka Aku akan menepati janji-Ku kepada ayahmu Daud bahwa selalu akan ada seorang dari keturunannya yang menjadi raja di Israel.
2Ch 7:19  Tetapi kalau engkau dan rakyatmu melanggar hukum-hukum dan perintah-perintah-Ku, serta menyembah ilah-ilah lain,
2Ch 7:20  maka Aku akan mengeluarkan kamu dari negeri yang telah Kuberikan kepadamu. Aku juga akan meninggalkan rumah ini yang telah Kutetapkan menjadi tempat ibadat kepada-Ku. Di mana-mana rumah ini akan dihina dan ditertawakan.
2Ch 7:21  Rumah ibadat ini sekarang sangat dihormati, tapi pada waktu itu setiap orang yang lewat di situ akan heran melihatnya. Mereka akan berkata, 'Mengapa TUHAN berbuat begitu terhadap negeri dan rumah ini?'
2Ch 7:22  Lalu orang akan menjawab, 'Karena mereka meninggalkan TUHAN, Allah mereka, yang telah mengantar leluhur mereka keluar dari Mesir. Mereka menyembah dan berjanji setia kepada ilah-ilah lain. Itulah sebabnya TUHAN mendatangkan bencana ini ke atas mereka.'"
2Ch 8:1  Salomo membangun Rumah TUHAN dan istana raja dalam waktu 20 tahun.
2Ch 8:2  Kota-kota yang diberikan Raja Hiram kepadanya dibangunnya kembali, lalu menyuruh orang Israel tinggal di situ.
2Ch 8:3  Ia merebut daerah Hamat dan Zoba
2Ch 8:4  serta memperkuat kota Tadmor di padang gurun, lalu membangun kembali semua pusat perbekalan di Hamat.
2Ch 8:5  Kota Bet-Horon-Atas dan Bet-Horon-Bawah dibangunnya kembali menjadi kota berbenteng dengan gerbang yang berpalang pintu.
2Ch 8:6  Juga dibangunnya kembali kota Baalat, dan semua kota perbekalan, serta kota-kota pangkalan kereta perang dan kudanya. Semua rencana pembangunannya di Yerusalem, Libanon dan di seluruh wilayah kekuasaannya telah dilaksanakannya.
2Ch 8:7  Semua keturunan bangsa Kanaan yang tidak dapat dibunuh habis oleh orang Israel ketika mereka menduduki negeri itu, dikerahkan oleh Salomo untuk kerja paksa. Mereka adalah orang Het, Amori, Feris, Hewi dan Yebus. Sampai sekarang keturunan mereka masih menjadi hamba.
2Ch 8:9  Orang Israel tidak disuruh kerja paksa; mereka ditugaskan sebagai prajurit, perwira, komandan kereta perang, dan tentara pasukan berkuda.
2Ch 8:10  Dua ratus lima puluh pegawai diserahi tanggung jawab atas orang-orang yang melakukan kerja paksa dalam berbagai proyek pembangunan.
2Ch 8:11  Salomo memindahkan istrinya, yaitu putri raja Mesir, dari Kota Daud ke rumah yang didirikan baginya. Sebab Salomo berpikir, "Istriku tidak boleh tinggal di istana Daud, raja Israel, karena semua tempat di mana Peti Perjanjian pernah diletakkan adalah tempat yang suci."
2Ch 8:12  Salomo mempersembahkan kurban bakaran kepada TUHAN di atas mezbah yang dibangunnya di depan Rumah TUHAN.
2Ch 8:13  Kurban-kurban itu dipersembahkannya menurut Hukum Musa untuk setiap hari raya: hari Sabat, hari raya Bulan Baru, dan ketiga hari raya tahunan, yaitu hari raya Roti Tidak Beragi, hari raya Panen, dan hari raya Pondok Daun.
2Ch 8:14  Sesuai dengan peraturan-peraturan Daud ayahnya, Salomo mengatur tugas-tugas harian untuk para imam dan untuk orang Lewi yang harus menyanyikan puji-pujian kepada TUHAN dan membantu para imam dalam melaksanakan upacara-upacara ibadah. Juga para pengawal Rumah TUHAN dibaginya dalam regu-regu untuk setiap pintu gerbang, sesuai dengan petunjuk Daud hamba Allah.
2Ch 8:15  Semua petunjuk Daud kepada para imam dan orang Lewi mengenai gudang-gudang dan hal-hal lain, diikutinya dengan cermat.
2Ch 8:16  Maka selesailah proyek pembangunan yang direncanakan Salomo. Seluruh pekerjaan, mulai dari perletakan pondasi sampai selesainya Rumah TUHAN, telah dikerjakan dengan baik.
2Ch 8:17  Setelah itu Salomo pergi ke Ezion-Geber dan ke Elot, kota-kota pelabuhan di pantai Teluk Akaba di negeri Edom.
2Ch 8:18  Raja Hiram mengirim kepadanya kapal-kapal yang dijalankan oleh anak buah Hiram sendiri bersama awak kapal yang berpengalaman. Mereka berlayar ke negeri Ofir bersama anak buah Salomo dan mengambil dari sana 15.000 kilogram emas lalu membawanya kepada Salomo.
2Ch 9:1  Ratu negeri Syeba mendengar tentang kemasyhuran Salomo. Maka ia datang ke Yerusalem untuk menguji Salomo dengan pertanyaan yang sulit-sulit. Ia datang disertai sejumlah besar pengiring dan unta yang sarat bermuatan rempah-rempah, batu permata, dan banyak sekali emas. Pada waktu bertemu dengan Salomo, ratu itu mengajukan segala macam pertanyaan yang dapat dipikirkannya.
2Ch 9:2  Semua pertanyaan itu dapat dijawab oleh Salomo, tidak satu pun yang terlalu sukar baginya.
2Ch 9:3  Ratu itu menyaksikan sendiri betapa bijaksananya Salomo. Ia melihat istana yang dibangun Salomo,
2Ch 9:4  tata kerja pegawai-pegawai istananya, dan pakaian seragam serta perumahan mereka. Ia melihat makanan yang dihidangkan, dan pakaian para pelayan yang melayani pada pesta; ia melihat juga kurban-kurban yang dipersembahkan Salomo di Rumah TUHAN. Semuanya itu membuat ratu negeri Syeba itu kagum dan terpesona.
2Ch 9:5  Maka berkatalah ia kepada Raja Salomo, "Segala yang kudengar di tanah airku tentang Tuan dan kebijaksanaan Tuan, memang benar!
2Ch 9:6  Dahulu aku tidak dapat percaya, tetapi setelah aku datang dan menyaksikan semuanya dengan mata kepala sendiri, barulah aku yakin. Sesungguhnya segala yang kudengar itu belum setengah dari yang kulihat sekarang. Nyatanya kebijaksanaan dan kekayaan Tuan jauh lebih besar dari yang diberitakan kepadaku.
2Ch 9:7  Alangkah mujurnya pegawai-pegawai yang melayani Tuan dan selalu bekerja untuk Tuan sehingga dapat mendengar dari Tuan sendiri segala ajaran yang bijaksana!
2Ch 9:8  Terpujilah TUHAN, Allah Tuan! Dengan mengangkat Tuan menjadi raja untuk memerintah atas nama-Nya, Ia menunjukkan betapa senangnya Ia terhadap Tuan. Karena Ia mengasihi umat Israel dan menghendaki supaya mereka tetap ada, maka Ia telah mengangkat Tuan menjadi raja mereka, supaya Tuan dapat menegakkan hukum dan keadilan."
2Ch 9:9  Kemudian ratu negeri Syeba itu menyerahkan kepada Salomo hadiah-hadiah yang dibawanya, yaitu lebih dari 4.000 kilogram emas dan sejumlah besar batu permata serta rempah-rempah. Tidak pernah lagi Salomo menerima rempah-rempah yang begitu baik mutunya seperti yang diberikan oleh ratu negeri Syeba itu kepadanya.
2Ch 9:10  (Anak buah Raja Hiram dan anak buah Raja Salomo yang membawa emas dari Ofir untuk Salomo, juga membawa batu permata dan kayu cendana.
2Ch 9:11  Kayu itu dipakai Salomo untuk membuat tangga di Rumah TUHAN dan di istananya, juga untuk membuat kecapi dan gambus bagi para penyanyi. Belum pernah ada yang seperti itu di negeri Yehuda.)
2Ch 9:12  Selain hadiah-hadiah balasan yang biasanya diberikan oleh Salomo, Salomo juga memberikan kepada ratu dari negeri Syeba itu segala yang dimintanya. Kemudian pulanglah ratu itu ke negerinya bersama semua pengiringnya.
2Ch 9:13  Setiap tahun Raja Salomo menerima hampir 23.000 kilogram emas,
2Ch 9:14  belum terhitung pajak-pajak dari para saudagar dan pedagang. Raja-raja Arab serta gubernur-gubernur Israel juga memberikan emas dan perak kepadanya.
2Ch 9:15  Salomo membuat 200 perisai besar dari emas tempaan. Emas yang dipakai untuk setiap perisai itu ada kurang lebih 7 kilogram.
2Ch 9:16  Ia juga membuat 300 perisai emas yang lebih kecil. Emas yang dipakai untuk setiap perisai kecil itu ada kurang lebih 3 kilogram, semuanya dari emas tempaan. Perisai-perisai itu ditaruhnya di balai yang bernama Balai Hutan Libanon.
2Ch 9:17  Raja juga membuat sebuah kursi kerajaan yang besar yang dibuat dari gading dan dihias dengan emas murni.
2Ch 9:18  Kursi itu berlengan dan di sebelah-menyebelahnya ada patung singa. Kursi itu juga mempunyai enam anak tangga, dan sebuah tempat kaki yang dilapisi dengan emas. Tempat kaki itu dijadikan satu dengan kursinya.
2Ch 9:19  Pada ujung kiri dan ujung kanan setiap anak tangga kursi itu ada patung singa--seluruhnya dua belas buah. Tidak pernah ada kursi seperti itu di kerajaan mana pun juga.
2Ch 9:20  Semua perkakas minum Salomo dibuat dari emas, dan semua perkakas di Balai Hutan Libanon pun dibuat dari emas murni. Pada zaman Salomo, perak dianggap tidak berharga.
2Ch 9:21  Salomo mempunyai banyak kapal besar yang berlayar di samudra raya bersama kapal-kapal Raja Hiram. Tiga tahun sekali kapal-kapal itu kembali membawa emas, perak, gading, kera dan burung merak.
2Ch 9:22  Raja Salomo lebih kaya dan lebih bijaksana dari raja mana pun di dunia.
2Ch 9:23  Semua raja berusaha menemui Salomo untuk mendengar ajaran bijaksana yang diberikan Allah kepadanya.
2Ch 9:24  Mereka masing-masing datang dengan membawa hadiah untuk Salomo. Mereka memberikan barang-barang perak, emas, pakaian, senjata, rempah-rempah, kuda dan bagal. Begitulah tahun demi tahun.
2Ch 9:25  Raja Salomo juga mempunyai 4.000 kandang untuk kuda dan keretanya. Kuda perangnya ada 12.000 ekor. Sebagian ditempatkannya di Yerusalem, dan selebihnya di berbagai kota pangkalan untuk kereta-kereta perangnya.
2Ch 9:26  Ia berkuasa atas semua raja di wilayah yang terbentang dari Sungai Efrat sampai ke negeri Filistin dan perbatasan Mesir.
2Ch 9:27  Dalam zaman pemerintahan Salomo, perak merupakan barang biasa di Yerusalem, sama seperti batu, dan kayu cemara Libanon banyak sekali seperti kayu ara biasa.
2Ch 9:28  Salomo mengimpor kuda dari Mesir dan dari setiap negeri lain.
2Ch 9:29  Kisah lain mengenai Salomo dari mula sampai akhir sudah dicatat dalam buku Riwayat Nabi Natan dan dalam buku Nubuatan Ahia orang Silo, dan dalam buku Wahyu Nabi Ido, yang juga memuat kisah pemerintahan Yerobeam raja Israel.
2Ch 9:30  Empat puluh tahun lamanya Salomo memerintah di Yerusalem atas seluruh Israel.
2Ch 9:31  Kemudian ia meninggal dan dimakamkan di Kota Daud. Lalu Rehabeam putranya menjadi raja menggantikan dia.
2Ch 10:1  Rehabeam pergi ke Sikhem karena seluruh rakyat Israel bagian utara telah berkumpul di sana untuk melantik dia menjadi raja.
2Ch 10:2  Pada waktu itu Yerobeam anak Nebat masih ada di Mesir karena melarikan diri dari Raja Salomo. Ketika ia mendengar tentang Rehabeam, ia kembali ke Israel.
2Ch 10:3  Rakyat bagian utara mengundang Yerobeam, lalu mereka bersama dia pergi menghadap Rehabeam. Kata mereka,
2Ch 10:4  "Salomo ayah Baginda telah memberi beban yang berat kepada kami. Kalau Baginda meringankan beban dan mengurangi penderitaan kami, kami akan tunduk kepada Baginda."
2Ch 10:5  Rehabeam menyahut, "Aku harus berpikir dahulu. Datanglah kembali tiga hari lagi." Maka pulanglah orang-orang itu.
2Ch 10:6  Lalu Rehabeam meminta nasihat kepada orang-orang tua yang pernah menjadi penasihat Salomo ayah Rehabeam, "Bagaimana aku harus menjawab orang-orang itu?" tanya Rehabeam. "Apa nasihat kalian?"
2Ch 10:7  Mereka menjawab, "Kalau Baginda mau berbuat baik kepada rakyat, dan menyenangkan hati mereka, berikanlah jawaban yang baik kepada mereka, maka mereka akan setia kepada Baginda selama-lamanya."
2Ch 10:8  Tetapi nasihat orang-orang tua itu tidak dihiraukan oleh Rehabeam. Sebaliknya, ia pergi kepada orang-orang muda yang sebaya dengan dia dan yang sekarang membantu dia.
2Ch 10:9  Kepada mereka ia bertanya, "Apa nasihat kalian kepadaku untuk menjawab rakyat yang meminta supaya aku meringankan beban mereka?"
2Ch 10:10  Mereka menjawab, "Beginilah hendaknya Baginda katakan kepada mereka, 'Selemah-lemahnya aku, aku masih lebih kuat dari ayahku.
2Ch 10:11  Ayahku memberikan beban yang berat kepadamu, tetapi aku akan memberikan yang lebih berat lagi. Ia menyebat kalian dengan cemeti, tetapi aku akan memecut kalian dengan cemeti berduri besi!'"
2Ch 10:12  Tiga hari kemudian Yerobeam dan semua orang itu kembali menghadap Raja Rehabeam, seperti yang telah diperintahkannya.
2Ch 10:13  Berlawanan dengan nasihat orang-orang tua, rakyat yang datang kepadanya itu disapanya dengan kasar.
2Ch 10:14  Sesuai dengan nasihat orang-orang muda, ia berkata, "Ayahku memberikan kepadamu beban yang berat, tetapi aku akan membuat beban itu lebih berat lagi. Ia menyebat kalian dengan cemeti, tetapi aku akan memecut kalian dengan cemeti berduri besi!"
2Ch 10:15  Permintaan rakyat tidak dihiraukannya. Tetapi itu memang kehendak TUHAN Allah. Apa yang dikatakan TUHAN melalui Nabi Ahia dari Silo mengenai Yerobeam anak Nebat harus terjadi.
2Ch 10:16  Ketika rakyat Israel melihat bahwa raja tidak mau mendengarkan mereka, mereka berteriak, "Peduli amat dengan Daud dan keturunannya! Mereka tidak pernah berbuat apa-apa untuk kita. Rakyat Israel, mari kita pulang! Biarkan Rehabeam itu mengurus dirinya sendiri!" Maka pulanglah rakyat Israel,
2Ch 10:17  dan Rehabeam menjadi raja hanya atas wilayah Yehuda.
2Ch 10:18  Meskipun begitu Raja Rehabeam masih juga menyuruh Adoniram, seorang kepala pekerja rodi, untuk menenangkan rakyat. Tetapi mereka melempari dia dengan batu sampai mati. Rehabeam sendiri berhasil lolos dan cepat-cepat naik ke keretanya, serta melarikan diri ke Yerusalem.
2Ch 10:19  Mulai saat itu rakyat di bagian utara kerajaan Israel selalu memberontak terhadap raja-raja keturunan Daud.
2Ch 11:1  Ketika Raja Rehabeam tiba di Yerusalem, ia mengumpulkan 180.000 prajuritnya yang terbaik dari suku Yehuda dan Benyamin untuk memerangi orang Israel dan memulihkan kekuasaannya atas suku-suku di bagian utara Israel.
2Ch 11:2  Tetapi TUHAN menyuruh Nabi Semaya
2Ch 11:3  menyampaikan pesan ini kepada Rehabeam dan kepada semua orang dalam suku Yehuda dan Benyamin,
2Ch 11:4  "Janganlah memerangi saudara-saudaramu orang Israel. Pulanglah! Apa yang telah terjadi adalah kehendak-Ku." Maka mereka semuanya menuruti perintah TUHAN dan tidak jadi pergi memerangi Yerobeam.
2Ch 11:5  Rehabeam tinggal di Yerusalem dan menyuruh orang membangun benteng-benteng untuk kota-kota di wilayah Yehuda dan Benyamin, yaitu:
2Ch 11:6  kota Betlehem, Etam, Tekoa,
2Ch 11:7  Bet-Zur, Sokho, Adulam,
2Ch 11:8  Gat, Mares, Zif,
2Ch 11:9  Adoraim, Lakhis, Azeka,
2Ch 11:10  Zora, Ayalon dan Hebron.
2Ch 11:11  Ia memperkuat kota-kota berbenteng itu dan menempatkan seorang komandan pasukan di setiap kota itu. Masing-masing kota itu dilengkapinya dengan persediaan makanan, minyak zaitun dan anggur
2Ch 11:12  serta perisai dan tombak. Dengan demikian wilayah Yehuda dan Benyamin tetap di dalam kekuasaannya.
2Ch 11:13  Imam-imam dan orang Lewi dari semua wilayah Israel pergi bergabung dengan Rehabeam.
2Ch 11:14  Orang-orang Lewi itu meninggalkan padang-padang rumput dan tanah mereka dan pindah ke Yehuda dan Yerusalem, sebab Yerobeam dan raja-raja yang menggantikannya tidak mengizinkan mereka bekerja sebagai imam TUHAN.
2Ch 11:15  Yerobeam mengangkat imam-imamnya sendiri untuk melayani di tempat-tempat penyembahan berhala dan untuk menyembah jin-jin serta patung sapi yang dibuatnya sendiri.
2Ch 11:16  Orang-orang dari semua suku Israel yang sungguh-sungguh ingin menyembah TUHAN, Allah Israel, pindah ke Yerusalem, mengikuti orang-orang Lewi itu supaya dapat mempersembahkan kurban kepada TUHAN, Allah yang disembah leluhur mereka.
2Ch 11:17  Dengan demikian kerajaan Yehuda menjadi lebih kuat. Selama tiga tahun mereka mendukung pemerintahan Rehabeam putra Salomo itu, dan hidup seperti pada zaman pemerintahan Raja Daud dan Raja Salomo.
2Ch 11:18  Rehabeam kawin dengan Mahalat; ayah Mahalat ialah Yerimot putra Daud, dan ibunya ialah Abihail anak Eliab, cucu Isai.
2Ch 11:19  Mereka mempunyai tiga anak laki-laki: Yeus, Semarya dan Zaham.
2Ch 11:20  Kemudian Rehabeam kawin dengan Maakha anak Absalom. Mereka mendapat empat anak laki-laki: Abia, Atai, Ziza dan Selomit.
2Ch 11:21  Rehabeam mempunyai 18 istri dan 60 selir, 28 anak laki-laki dan 60 anak perempuan. Dari semua istri dan selirnya itu, yang paling dicintainya adalah Maakha.
2Ch 11:22  Itu sebabnya Abia anak Maakha lebih disukainya daripada semua anaknya yang lain, sehingga Abialah juga yang dipilihnya untuk menggantikan dia menjadi raja.
2Ch 11:23  Dengan bijaksana Rehabeam memberikan tugas-tugas kepada putra-putranya dan menempatkan mereka di kota-kota berbenteng di seluruh wilayah Yehuda dan Benyamin. Ia memberi mereka makanan berlimpah-limpah, dan juga menyediakan banyak istri untuk mereka.
2Ch 12:1  Setelah kedudukan Rehabeam sebagai raja telah menjadi kuat, ia dan seluruh rakyat Israel mengabaikan hukum TUHAN,
2Ch 12:2  dan tidak setia kepada-Nya. Karena itu TUHAN menghukum mereka. Pada tahun kelima pemerintahan Rehabeam, Sisak raja Mesir menyerang Yerusalem
2Ch 12:3  dengan 1.200 kereta perang, 60.000 tentara berkuda, dan sejumlah prajurit yang tak terhitung banyaknya, termasuk pasukan Libia, Suki dan Sudan.
2Ch 12:4  Ia merebut kota-kota berbenteng di Yehuda lalu maju sampai ke Yerusalem.
2Ch 12:5  Pada waktu itu Raja Rehabeam dan para pemimpin Yehuda telah berkumpul di Yerusalem karena melarikan diri dari Sisak. Nabi Semaya datang kepada mereka dan berkata, "Begini kata TUHAN, 'Kamu meninggalkan Aku, jadi sekarang Aku pun meninggalkan kamu sehingga kamu dikuasai oleh Sisak.'"
2Ch 12:6  Raja dan para pemimpin mengakui bahwa mereka sudah berdosa. Mereka berkata, "Apa yang dilakukan TUHAN itu adil."
2Ch 12:7  Ketika TUHAN melihat bahwa mereka telah merendahkan diri, Ia berbicara lagi kepada Semaya, kata-Nya, "Karena mereka mengakui dosa mereka, maka Aku tak akan membinasakan mereka. Aku akan segera meluputkan mereka; Yerusalem tidak akan hancur. Meskipun begitu, Sisak
2Ch 12:8  akan menguasai mereka, supaya mereka tahu perbedaan antara mengabdi kepada-Ku dan mengabdi kepada penguasa-penguasa dunia."
2Ch 12:9  Maka datanglah Raja Sisak menyerang kota Yerusalem dan merampas barang-barang berharga yang terdapat di Rumah TUHAN dan di istana raja, juga perisai-perisai emas yang dibuat oleh Raja Salomo.
2Ch 12:10  Sebagai gantinya Rehabeam membuat perisai-perisai perunggu, dan mempercayakannya kepada para perwira yang mengawal pintu gerbang istana.
2Ch 12:11  Setiap kali raja pergi ke Rumah TUHAN, para pengawal membawa perisai-perisai itu, kemudian mengembalikannya ke kamar jaga.
2Ch 12:12  Karena raja merendahkan diri di hadapan TUHAN, dan juga karena masih ada hal-hal yang baik di Yehuda maka redalah kemarahan TUHAN kepada Rehabeam dan tidak membinasakan dia sama sekali.
2Ch 12:13  Rehabeam memerintah di Yerusalem dan kekuasaannya semakin besar. Ia berumur 41 tahun ketika menjadi raja, dan ia memerintah 17 tahun lamanya di Yerusalem, kota yang dari seluruh wilayah Israel dipilih TUHAN untuk menjadi tempat ibadat kepada-Nya. Ibu Rehabeam adalah Naama dari negeri Amon.
2Ch 12:14  Rehabeam berbuat jahat karena tidak berusaha melakukan kehendak TUHAN.
2Ch 12:15  Kisah tentang Rehabeam dari mula sampai akhir dan silsilahnya dicatat dalam buku Sejarah Nabi Semaya dan dalam buku Sejarah Nabi Ido. Antara Rehabeam dan Yerobeam selalu saja terjadi peperangan.
2Ch 12:16  Rehabeam meninggal dan dikuburkan di makam raja-raja di Kota Daud. Abia putranya menjadi raja menggantikan dia.
2Ch 13:1  Pada tahun kedelapan belas pemerintahan Raja Yerobeam atas Israel, Abia menjadi raja Yehuda,
2Ch 13:2  dan memerintah tiga tahun lamanya di Yerusalem. Ibunya ialah Mikhaya anak Uriel dari kota Gibea. Kemudian pecahlah perang antara Abia dan Yerobeam.
2Ch 13:3  Abia mengerahkan 400.000 prajurit, sedangkan Yerobeam mengerahkan 800.000 prajurit.
2Ch 13:4  Kedua pasukan itu berhadap-hadapan di daerah pegunungan Efraim. Abia naik ke atas Gunung Zemaraim lalu berseru kepada Yerobeam dan orang Israel, "Dengar!
2Ch 13:5  Kamu tahu bahwa TUHAN, Allah Israel, telah mengikat janji dengan Daud bahwa ia dan keturunannya akan memerintah Israel untuk selama-lamanya. Janji itu tidak dapat dibatalkan.
2Ch 13:6  Meskipun begitu, Yerobeam anak Nebat itu memberontak terhadap Salomo, rajanya.
2Ch 13:7  Ia mengumpulkan segerombolan orang-orang kurang ajar dan memaksakan kehendak mereka kepada Rehabeam putra Salomo, yang pada waktu itu terlalu muda dan tak berpengalaman untuk menentang.
2Ch 13:8  Sekarang kamu menyangka dapat melawan kekuasaan yang diberikan TUHAN kepada keturunan Daud, karena pasukanmu besar dan kamu mempunyai sapi-sapi emas yang oleh Yerobeam dijadikan dewamu.
2Ch 13:9  Bukankah imam-imam TUHAN keturunan Harun dan orang-orang Lewi telah kamu usir semuanya? Sebagai gantinya siapa saja yang ingin menjadi imam, asal dia membawa seekor sapi atau tujuh ekor domba, kamu jadikan imam untuk yang bukan Allah, sebab memang begitulah cara bangsa-bangsa lain mengangkat imam mereka. Tetapi kami tidak berbuat seperti itu.
2Ch 13:10  Kami tetap mengabdi kepada TUHAN Allah kami dan tidak meninggalkan Dia. Imam-imam kami tetap keturunan Harun, dan orang-orang Lewi tetap membantu mereka.
2Ch 13:11  Setiap pagi dan sore mereka membakar dupa dan mempersembahkan kurban bakaran untuk TUHAN, serta menyediakan roti sajian. Setiap malam mereka menyalakan pelita-pelita pada kaki pelita emas. Kami tetap mentaati perintah-perintah TUHAN, tetapi kamu sudah meninggalkan Dia.
2Ch 13:12  Pemimpin kami adalah Allah sendiri dan para imam-Nya ada di sini. Mereka sudah siap untuk meniup trompet sebagai tanda bagi kami untuk menyerang kamu. Hai orang Israel, janganlah melawan TUHAN, Allah leluhurmu! Kamu takkan bisa menang!"
2Ch 13:13  Sementara itu Yerobeam mengirim sebagian dari pasukannya untuk menghadang pasukan Yehuda dari belakang, sedangkan sebagian lagi menghadapi mereka dari depan.
2Ch 13:14  Ketika pasukan Yehuda melihat bahwa mereka dikepung, mereka berteriak minta tolong kepada TUHAN. Para imam meniup trompet,
2Ch 13:15  lalu pasukan Yehuda menyerukan pekik perang dan maju menyerang, dipimpin oleh Abia. Sementara mereka berseru, Allah mengalahkan Yerobeam dan pasukan Israel
2Ch 13:16  sehingga mereka lari. Allah membuat orang Yehuda menang atas mereka.
2Ch 13:17  Besarlah kekalahan yang didatangkan Abia dan pasukannya ke atas orang Israel; 500.000 prajurit mereka yang terbaik tewas.
2Ch 13:18  Orang Yehuda menang, karena mereka mengandalkan TUHAN, Allah leluhur mereka.
2Ch 13:19  Ketika Abia mengejar pasukan Yerobeam, ia merebut beberapa kota dengan desa-desa di sekitarnya. Kota-kota itu ialah Betel, Yesana dan Efron.
2Ch 13:20  Selama pemerintahan Abia, Yerobeam tidak dapat memulihkan kekuasaannya. Akhirnya ia dibunuh oleh TUHAN.
2Ch 13:21  Sebaliknya, Abia semakin berkuasa. Ia mempunyai 14 orang istri, 22 anak laki-laki dan 16 anak perempuan.
2Ch 13:22  Kisah lainnya mengenai Abia, dan mengenai semua perkataan dan perbuatannya telah dicatat dalam buku Sejarah Nabi Ido.
2Ch 14:1  Setelah Raja Abia meninggal dan dikuburkan di makam raja-raja di Kota Daud, Asa putranya menjadi raja menggantikan dia. Di bawah pemerintahan Asa, negeri Yehuda tenteram selama sepuluh tahun.
2Ch 14:2  Asa menyenangkan hati TUHAN Allahnya dengan melakukan yang adil dan baik.
2Ch 14:3  Ia menyingkirkan mezbah-mezbah bangsa lain dan tempat-tempat penyembahan berhala. Tugu-tugu dewa dihancurkannya, dan tiang-tiang patung Dewi Asyera ditumbangkannya.
2Ch 14:4  Rakyat Yehuda disuruhnya hidup menurut kehendak TUHAN, Allah leluhur mereka, dan mentaati ajaran-ajaran dan perintah-perintah-Nya.
2Ch 14:5  Semua tempat penyembahan berhala dan tempat membakar dupa disingkirkannya dari kota-kota Yehuda. Maka tenanglah kerajaan itu di bawah pemerintahannya,
2Ch 14:6  sehingga ia dapat membangun benteng-benteng untuk kota-kota Yehuda. Beberapa tahun lamanya tidak terjadi perang karena TUHAN memberikan keadaan damai kepadanya.
2Ch 14:7  Kata Asa kepada rakyat Yehuda, "Baiklah kita memperkuat kota-kota dengan membangun tembok sekelilingnya dan menara-menara serta pintu gerbang dengan palang-palangnya. Kita telah menguasai negeri ini karena kita sudah menuruti kehendak TUHAN Allah kita. Ia memelihara kita dan menjaga keamanan di seluruh negeri kita." Maka mulailah mereka membangun dan menyelesaikan pekerjaan itu dengan baik.
2Ch 14:8  Tentara Raja Asa terdiri dari 300.000 orang Yehuda yang bersenjatakan perisai dan tombak, dan 280.000 orang Benyamin yang bersenjatakan perisai, busur dan panah. Mereka semuanya prajurit yang berani-berani dan terlatih.
2Ch 14:9  Pada suatu waktu seorang Sudan bernama Zerah menyerang Yehuda dengan pasukan yang terdiri dari 1.000.000 prajurit dan 300 kereta perang. Mereka maju sejauh Maresa.
2Ch 14:10  Maka keluarlah Asa menghadapi Zerah, dan keduanya mengatur barisan masing-masing di Lembah Zefata dekat Maresa.
2Ch 14:11  Lalu Asa berdoa kepada TUHAN Allahnya, "TUHAN, hanya Engkaulah yang dapat membantu yang lemah terhadap yang kuat. Tolonglah kami sekarang, ya TUHAN Allah kami! Pada-Mulah kami bersandar dan atas nama-Mu kami keluar berperang melawan pasukan yang sangat besar ini. TUHAN, Engkaulah Allah kami. Jangan biarkan diri-Mu dikalahkan oleh seorang manusia."
2Ch 14:12  Lalu Asa dan tentara Yehuda menyerang tentara Sudan. TUHAN mengalahkan tentara Sudan sehingga mereka melarikan diri.
2Ch 14:13  Asa dan tentaranya mengejar mereka sampai sejauh Gerar. Di pihak Sudan banyak sekali yang tewas sehingga tentara mereka tidak berdaya karena dihancurkan oleh TUHAN dan tentara Yehuda yang juga merampas banyak sekali dari barang-barang mereka. Kemudian tentara Yehuda menghancurkan kota-kota di wilayah sekitar Gerar, karena TUHAN membuat penduduknya menjadi ketakutan. Tentara Yehuda merampasi semua kota itu dan mendapat banyak sekali barang.
2Ch 14:15  Mereka juga menyerang perkemahan para peternak, dan mengangkut banyak domba dan unta. Setelah itu mereka kembali ke Yerusalem.
2Ch 15:1  Roh Allah menguasai Nabi Azarya anak Oded,
2Ch 15:2  sehingga ia pergi menemui Raja Asa, dan berkata, "Hai Raja Asa, dan seluruh penduduk Yehuda dan Benyamin! Dengar! TUHAN dekat padamu kalau kamu dekat pada-Nya. Jika kamu minta petunjuk TUHAN, Ia akan memberikannya. Tetapi kalau kamu tidak menghiraukan Dia, Ia pun tidak menghiraukan kamu.
2Ch 15:3  Lama sekali Israel tidak mengabdi kepada Allah yang benar, tidak mempunyai hukum-hukum yang harus ditaati, dan tidak mempunyai imam-imam yang mengajar mereka.
2Ch 15:4  Tetapi ketika mereka dalam kesukaran, mereka minta tolong kepada TUHAN, Allah Israel. Mereka minta petunjuk-Nya dan Ia memberikannya.
2Ch 15:5  Pada masa itu orang tidak dapat bepergian dengan aman, sebab di setiap negeri ada kerusuhan besar.
2Ch 15:6  Bangsa yang satu menindas bangsa yang lain, dan kota yang satu menindas kota yang lain, karena Allah mendatangkan kerusuhan besar atas mereka.
2Ch 15:7  Tetapi kamu harus tabah dan berani. Usahamu akan berhasil."
2Ch 15:8  Ketika Asa mendengar perkataan Azarya anak Oded itu, berkobarlah semangatnya. Ia menyingkirkan semua berhala di negeri Yehuda dan Benyamin, dan di kota-kota yang telah dikalahkannya di pegunungan Efraim, lalu ia memperbaiki mezbah TUHAN yang terdapat di pelataran Rumah TUHAN.
2Ch 15:9  Banyak orang Efraim, Manasye, dan Simeon bergabung dengan Asa, dan tinggal di wilayah kerajaannya, karena mereka melihat bahwa TUHAN menolong dia. Asa menyuruh semua orang itu bersama dengan rakyat Yehuda dan Benyamin
2Ch 15:10  berkumpul di Yerusalem. Maka berkumpullah mereka pada bulan tiga dalam tahun kelima belas pemerintahan Asa.
2Ch 15:11  Pada hari itu mereka mempersembahkan kepada TUHAN 700 sapi dan 7.000 domba, hasil rampasan yang mereka bawa pulang dari peperangan.
2Ch 15:12  Mereka berjanji kepada TUHAN bahwa dengan sepenuh hati dan jiwa mereka akan hidup menurut kehendak TUHAN, Allah leluhur mereka.
2Ch 15:13  Siapa saja, tua atau muda, laki-laki atau perempuan, yang tidak hidup menurut kehendak TUHAN pasti dihukum mati.
2Ch 15:14  Dengan suara yang keras mereka bersumpah kepada TUHAN bahwa mereka akan memegang janji itu. Kemudian mereka bersorak-sorak dan meniup trompet.
2Ch 15:15  Seluruh rakyat Yehuda gembira karena mereka telah membuat sumpah itu dengan sepenuh hati. Mereka senang beribadat kepada TUHAN, dan TUHAN pun berkenan menerima mereka. Ia memberikan kepada mereka keadaan tenteram di wilayah mereka.
2Ch 15:16  Selanjutnya Raja Asa memecat neneknya Maakha, dari kedudukannya sebagai Ibu Suri sebab ia telah membuat patung berhala yang cabul untuk Asyera, dewi kesuburan. Asa merobohkan dan menghancurluluhkan patung itu lalu membakarnya di Lembah Kidron.
2Ch 15:17  Meskipun tidak semua tempat penyembahan berhala di negeri itu dihancurkan oleh Asa, namun ia tetap setia kepada TUHAN sepanjang hidupnya.
2Ch 15:18  Semua emas dan perak serta barang-barang lain hasil rampasan perang milik ayahnya dan dirinya sendiri diletakkannya di Rumah TUHAN untuk menjadi milik TUHAN.
2Ch 15:19  Sampai pada tahun ketiga puluh lima pemerintahan Asa tak pernah timbul peperangan lagi di negeri itu.
2Ch 16:1  Pada tahun ketiga puluh enam pemerintahan Raja Asa atas Yehuda, Baesa raja Israel menyerang Yehuda dan memperkuat kota Rama untuk menutup jalan keluar masuk Yehuda.
2Ch 16:2  Maka Raja Asa mengambil emas dan perak dari perbendaharaan Rumah TUHAN dan istana raja, lalu mengirimnya ke Damsyik kepada Benhadad raja Siria dengan pesan ini,
2Ch 16:3  "Marilah kita mengikat persahabatan seperti yang sudah dilakukan oleh orang tua kita. Bersama ini aku mengirimkan emas dan perak sebagai hadiah, dan mengajak Tuan memutuskan hubungan dengan Baesa raja Israel, supaya ia menarik kembali pasukannya dari wilayahku."
2Ch 16:4  Benhadad setuju dengan tawaran itu lalu menyuruh perwira-perwiranya bersama pasukan mereka menyerang kota-kota Israel. Mereka mengalahkan Iyon, Dan, Abel-Bet-Maakha dan semua kota-kota perbekalan di Naftali.
2Ch 16:5  Ketika Raja Baesa mendengar hal itu, ia berhenti memperkuat Rama dan meninggalkan kota itu.
2Ch 16:6  Lalu Raja Asa mengerahkan orang-orang dari seluruh Yehuda untuk mengangkut batu dan kayu yang dipakai Baesa di Rama untuk memperkuat kota itu. Bahan-bahan itu dipakai Asa untuk memperkuat kota Geba dan Mizpa.
2Ch 16:7  Pada waktu itu Nabi Hanani datang kepada Raja Asa dan berkata, "Karena engkau bersandar kepada raja Siria, dan bukan kepada TUHAN Allahmu, maka hilanglah kesempatanmu untuk mengalahkan tentara raja Israel.
2Ch 16:8  Bukankah Sudan dan Libia mempunyai angkatan bersenjata yang besar dengan banyak kereta perang dan tentara berkuda? Meskipun begitu TUHAN memberikan kemenangan kepadamu atas mereka, karena engkau bersandar kepada-Nya.
2Ch 16:9  TUHAN mengawasi seluruh bumi untuk memberikan kekuatan-Nya bagi orang-orang yang setia kepada-Nya. Engkau telah bertindak bodoh, karena itu mulai sekarang engkau selalu akan mengalami peperangan."
2Ch 16:10  Perkataan Nabi Hanani itu membuat Asa begitu marah sehingga ia memasukkan nabi itu ke dalam penjara. Pada waktu itu juga Asa menindas beberapa orang dari rakyatnya.
2Ch 16:11  Semua kejadian dalam pemerintahan Asa dari mula sampai akhir sudah dicatat dalam buku Sejarah Raja-raja Yehuda dan Israel.
2Ch 16:12  Pada tahun ketiga puluh sembilan pemerintahannya Asa menderita penyakit yang parah pada kedua kakinya. Sekalipun demikian ia tidak mencari pertolongan pada TUHAN, melainkan pada dokter.
2Ch 16:13  Dua tahun kemudian ia meninggal.
2Ch 16:14  Jenazahnya diberi ramuan yang dibuat dari rempah-rempah dan wangi-wangian, lalu dimakamkan di makam raja-raja di Kota Daud. Dan untuk memberi penghormatan kepadanya rakyat menyalakan api unggun yang sangat besar.
2Ch 17:1  Yosafat menjadi raja Yehuda menggantikan Asa ayahnya, dan ia memperkuat diri untuk melawan Israel.
2Ch 17:2  Ia menempatkan tentara di semua kota berbenteng di Yehuda, dan di daerah luar kota di Yehuda, serta di kota-kota yang telah direbut ayahnya di wilayah Efraim.
2Ch 17:3  TUHAN memberkati Yosafat karena ia mengikuti jejak ayahnya; ia tidak menyembah Baal,
2Ch 17:4  melainkan beribadat kepada Allah. Ia taat kepada perintah-perintah Allah, dan hidupnya tidak seperti raja-raja Israel.
2Ch 17:5  TUHAN memberi kepada Yosafat kedudukan yang kuat sebagai raja Yehuda. Seluruh rakyat memberikan hadiah-hadiah kepadanya dan ia menjadi kaya serta sangat dihormati.
2Ch 17:6  Dengan kemauan yang keras ia mentaati perintah TUHAN dan menghancurkan semua tempat penyembahan berhala serta patung-patung Dewi Asyera di Yehuda.
2Ch 17:7  Pada tahun ketiga pemerintahannya ia mengutus pejabat-pejabat tinggi untuk mengajarkan hukum-hukum TUHAN di kota-kota Yehuda. Mereka adalah: Benhail, Obaja, Zakharia, Netaneel dan Mikha.
2Ch 17:8  Mereka ditemani oleh sembilan orang Lewi dan dua imam. Orang-orang Lewi itu ialah Semaya, Netanya, Zebaja, Asael, Semiramot, Yonatan, Adonia, Tobia, dan Tob-Adonia. Kedua imam itu ialah Elisama dan Yoram.
2Ch 17:9  Dengan membawa buku Hukum-hukum TUHAN mereka pergi ke mana-mana di semua kota Yehuda dan mengajar rakyat di sana.
2Ch 17:10  TUHAN membuat semua kerajaan di sekitar Yehuda takut berperang melawan Raja Yosafat.
2Ch 17:11  Dari orang Filistin ia menerima banyak perak dan hadiah-hadiah lain, dan dari orang Arab ia menerima 7.700 domba dan 7.700 kambing.
2Ch 17:12  Demikianlah Yosafat makin lama makin berkuasa. Di seluruh Yehuda ia membangun benteng-benteng dan kota-kota perbekalan.
2Ch 17:13  Ia menimbun perbekalan yang banyak di kota-kota itu. Di Yerusalem ia menempatkan perwira-perwira yang perkasa,
2Ch 17:14  masing-masing dengan kesatuannya menurut kaumnya. Adna adalah panglima kesatuan dari kaum-kaum Yehuda. Ia memimpin 300.000 prajurit.
2Ch 17:15  Orang kedua adalah Yohanan; ia mengepalai 280.000 prajurit.
2Ch 17:16  Orang ketiga ialah Amasia anak Zikhri; ia mengepalai 200.000 prajurit. (Amasia adalah seorang sukarelawan yang mengabdi kepada TUHAN.)
2Ch 17:17  Panglima kesatuan dari kaum-kaum dalam suku Benyamin adalah Elyada, seorang pejuang yang perkasa. Ia memimpin 200.000 prajurit lengkap dengan panah dan perisai.
2Ch 17:18  Orang kedua dalam kesatuan ini ialah Yosabad; ia memimpin 180.000 prajurit lengkap dengan senjata mereka.
2Ch 17:19  Merekalah yang bertugas di Yerusalem. Di samping itu, raja menempatkan kesatuan-kesatuan lain di benteng-benteng di wilayah Yehuda.
2Ch 18:1  Ketika Yosafat raja Yehuda sudah kaya dan termasyhur, ia mengawinkan seorang anggota keluarganya dengan seorang anggota keluarga Ahab raja Israel.
2Ch 18:2  Beberapa tahun kemudian Yosafat pergi mengunjungi Ahab di kota Samaria. Untuk menghormati Yosafat dan rombongannya, Ahab mengadakan pesta dan menyuruh menyembelih banyak domba dan sapi. Pada kesempatan itu ia mengajak Yosafat untuk menyerang kota Ramot di Gilead.
2Ch 18:3  Ia bertanya, "Maukah Anda pergi bersama aku menyerang Ramot?" Yosafat menjawab, "Baik, tentaraku akan bergabung dengan tentara Anda. Kita bertempur bersama-sama.
2Ch 18:4  Tapi sebaiknya kita tanyakan dulu kepada TUHAN."
2Ch 18:5  Maka Ahab mengumpulkan kira-kira 400 nabi lalu bertanya kepada mereka, "Bolehkah aku pergi menyerang Ramot atau tidak?" "Boleh!" jawab mereka. "Allah akan menyerahkan kota itu kepada Baginda."
2Ch 18:6  Tetapi Yosafat bertanya lagi, "Apakah di sini tidak ada nabi lain yang dapat bertanya kepada TUHAN untuk kita?"
2Ch 18:7  Ahab menjawab, "Masih ada satu, Mikha anak Yimla. Tapi aku benci kepadanya, sebab tidak pernah ia meramalkan sesuatu yang baik untuk aku; selalu yang tidak baik." "Ah, jangan berkata begitu!" sahut Yosafat.
2Ch 18:8  Maka Ahab memanggil seorang pegawai istana lalu menyuruh dia cepat-cepat pergi menjemput Mikha.
2Ch 18:9  Pada waktu itu Ahab dan Yosafat, dengan pakaian kebesaran, duduk di kursi kerajaan di tempat pengirikan gandum depan pintu gerbang Samaria, sementara para nabi datang menghadap dan menyampaikan ramalan mereka.
2Ch 18:10  Salah seorang dari nabi-nabi itu, yang bernama Zedekia anak Kenaana, membuat tanduk-tanduk besi lalu berkata kepada Ahab, "Inilah yang dikatakan TUHAN, 'Dengan tanduk-tanduk seperti ini Baginda akan menghantam Siria dan menghancurkan mereka.'"
2Ch 18:11  Semua nabi yang lain setuju dan berkata, "Serbulah Ramot, Baginda akan berhasil. TUHAN akan memberi kemenangan kepada Baginda."
2Ch 18:12  Sementara itu utusan yang menjemput Mikha, berkata kepada Mikha, "Semua nabi yang lain meramalkan kemenangan untuk raja. Kiranya Bapak juga meramalkan yang baik seperti mereka."
2Ch 18:13  Tetapi Mikha menjawab, "Demi TUHAN yang hidup, aku hanya akan mengatakan apa yang dikatakan Allah kepadaku!"
2Ch 18:14  Setelah Mikha tiba di depan Raja Ahab, raja bertanya, "Bolehkah aku dan Raja Yosafat pergi menyerang Ramot atau tidak?" "Seranglah!" sahut Mikha. "Tentu Baginda akan berhasil. TUHAN akan memberi kemenangan kepada Baginda."
2Ch 18:15  Ahab menjawab, "Kalau kau berbicara kepadaku demi nama TUHAN, katakanlah yang benar. Berapa kali engkau harus kuperingatkan tentang hal itu?"
2Ch 18:16  Mikha membalas, "Aku melihat tentara Israel kucar-kacir di gunung-gunung. Mereka seperti domba tanpa gembala, dan TUHAN berkata tentang mereka, 'Orang-orang ini tidak mempunyai pemimpin. Biarlah mereka pulang dengan selamat.'"
2Ch 18:17  Lalu kata Ahab kepada Yosafat, "Benar kataku, bukan? Tidak pernah ia meramalkan yang baik untuk aku! Selalu yang jelek saja!"
2Ch 18:18  Mikha berkata lagi, "Sekarang dengarkan apa yang dikatakan TUHAN! Aku melihat TUHAN duduk di atas takhta-Nya di surga, dan semua malaikat-Nya berdiri di dekat-Nya.
2Ch 18:19  TUHAN bertanya, 'Siapa akan membujuk Ahab supaya ia mau pergi berperang dan tewas di Ramot di Gilead?' Jawaban malaikat-malaikat itu berbeda-beda.
2Ch 18:20  Akhirnya tampillah suatu roh. Ia mendekati TUHAN dan berkata, 'Akulah yang akan membujuk dia.' 'Bagaimana caranya?' tanya TUHAN.
2Ch 18:21  Roh itu menjawab, 'Aku akan pergi dan membuat semua nabi Ahab membohong.' TUHAN berkata, 'Baik, lakukanlah itu, engkau akan berhasil membujuk dia.'"
2Ch 18:22  Selanjutnya Mikha berkata, "Nah, itulah yang terjadi! TUHAN telah membuat nabi-nabi Baginda berdusta kepada Baginda sebab TUHAN sudah menentukan untuk menimpakan bencana kepada Baginda!"
2Ch 18:23  Maka majulah Nabi Zedekia mendekati Mikha lalu menampar mukanya dan berkata, "Mana mungkin Roh TUHAN meninggalkan aku dan berbicara kepadamu?"
2Ch 18:24  Mikha menjawab, "Nanti kaulihat buktinya pada waktu engkau masuk ke sebuah kamar untuk bersembunyi!"
2Ch 18:25  "Tangkap dia!" perintah raja Ahab, "dan bawa dia kepada Amon, walikota, dan kepada Pangeran Yoas.
2Ch 18:26  Suruh mereka memasukkan dia ke dalam penjara, dan memberi dia makan dan minum sedikit saja sampai aku kembali dengan selamat."
2Ch 18:27  Kata Mikha, "Kalau Baginda kembali dengan selamat, berarti TUHAN tidak berbicara melalui aku. Semua yang hadir di sini menjadi saksi."
2Ch 18:28  Kemudian Ahab raja Israel, dan Yosafat raja Yehuda pergi menyerang kota Ramot di Gilead.
2Ch 18:29  Ahab berkata kepada Yosafat, "Aku akan menyamar dan ikut bertempur, tetapi Anda hendaklah memakai pakaian kebesaranmu." Demikianlah raja Israel menyamar ketika pergi bertempur.
2Ch 18:30  Pada waktu itu para panglima pasukan kereta perang Siria telah diperintahkan oleh rajanya untuk menyerang hanya raja Israel.
2Ch 18:31  Jadi, ketika mereka melihat Raja Yosafat, mereka semua menyangka ia raja Israel. Karena itu mereka menyerang dia. Tetapi Yosafat berteriak, lalu TUHAN Allah menolong dia, dan membelokkan serangan itu ke arah lain.
2Ch 18:32  Ketika para panglima pasukan kereta perang Siria menyadari bahwa itu bukan raja Israel, mereka berhenti menyerang dia.
2Ch 18:33  Secara kebetulan seorang prajurit Siria melepaskan anak panahnya tanpa mengarahkannya ke sasaran tertentu. Tetapi anak panah itu mengenai Ahab dan menembus baju perangnya pada bagian sambungannya. "Aku kena!" seru Ahab kepada pengemudi keretanya, "Putar dan bawalah aku ke luar dari pertempuran!"
2Ch 18:34  Tapi karena pertempuran masih berkobar, Raja Ahab tetap berdiri sambil ditopang dalam keretanya menghadap tentara Siria. Pada waktu matahari terbenam ia meninggal.
2Ch 19:1  Setelah Yosafat raja Yehuda kembali dengan selamat di istananya di Yerusalem,
2Ch 19:2  seorang nabi bernama Yehu anak Hanani datang menghadap raja. Kata nabi itu kepadanya, "Patutkah Baginda membantu orang jahat dan memihak pada orang yang membenci TUHAN? Perbuatan Baginda itu menyebabkan TUHAN marah kepada Baginda.
2Ch 19:3  Meskipun begitu ada juga yang baik pada Baginda. Baginda telah menyingkirkan semua patung Dewi Asyera yang disembah rakyat dan Baginda telah pula berusaha untuk hidup menurut kemauan Allah."
2Ch 19:4  Raja Yosafat bertempat tinggal di Yerusalem. Tetapi ia sering mengunjungi rakyat di seluruh negeri untuk menganjurkan mereka kembali kepada TUHAN, Allah leluhur mereka.
2Ch 19:5  Di setiap kota berbenteng di Yehuda ia mengangkat hakim-hakim,
2Ch 19:6  dan memberikan perintah ini kepada mereka, "Bertindaklah bijaksana, karena kamu menghakimi bukan atas nama manusia, melainkan atas nama TUHAN. Ia mengawasi kamu pada waktu kamu memutuskan perkara.
2Ch 19:7  Kamu harus takut kepada TUHAN, dan bertindak hati-hati, sebab TUHAN Allah kita membenci orang yang berbuat curang, yang bertindak berat sebelah atau yang menerima suap."
2Ch 19:8  Juga di Yerusalem, Yosafat mengangkat hakim-hakim dari kalangan orang-orang Lewi, imam-imam, dan tokoh-tokoh rakyat. Mereka harus memutuskan perkara-perkara yang menyangkut pelanggaran Hukum TUHAN dan perkara-perkara lainnya.
2Ch 19:9  Kepada hakim-hakim itu Yosafat memberi perintah ini, "Jalankanlah tugasmu dengan takwa kepada TUHAN. Turutilah perintah-Nya dalam segala sesuatu yang kamu lakukan.
2Ch 19:10  Apabila rekan-rekanmu dari kota lain mengajukan kepadamu suatu perkara pembunuhan, pelanggaran hukum atau pelanggaran undang-undang, hendaklah kamu memberi petunjuk kepada mereka. Dengan begitu mereka tidak bersalah terhadap TUHAN, dan TUHAN tidak akan marah kepadamu maupun kepada mereka. Lakukanlah perintah ini, maka kamu tidak akan bersalah.
2Ch 19:11  Imam Agung Amarya kuberi kekuasaan tertinggi atas segala perkara yang menyangkut hukum-hukum agama. Zebaja anak Ismael, yaitu gubernur Yehuda, kuberi kekuasaan tertinggi atas segala perkara lain. Orang-orang Lewi bertanggung jawab menjaga agar keputusan-keputusan pengadilan dilaksanakan. Ikutilah petunjuk-petunjuk ini dan bertindaklah dengan tegas. Semoga TUHAN mendampingi orang yang berlaku adil!"
2Ch 20:1  Beberapa waktu kemudian tentara Moab dan Amon, bersama-sama dengan sekutu mereka tentara Meunim, menyerang Yehuda.
2Ch 20:2  Maka datanglah orang membawa berita ini kepada Raja Yosafat, "Banyak sekali tentara Edom telah datang dari seberang Laut Mati untuk menyerbu Baginda. Mereka sudah merebut Hazezon-Tamar." (Hazezon-Tamar adalah nama lain untuk En-Gedi.)
2Ch 20:3  Yosafat menjadi takut lalu ia berdoa minta petunjuk dari TUHAN. Kemudian ia memerintahkan supaya seluruh rakyat berpuasa.
2Ch 20:4  Dari setiap kota di Yehuda orang pergi ke Yerusalem untuk minta tolong kepada TUHAN.
2Ch 20:5  Mereka bersama-sama dengan penduduk Yerusalem berkumpul di pelataran yang baru di Rumah TUHAN. Raja Yosafat berdiri di depan mereka,
2Ch 20:6  dan berdoa dengan suara keras, "Ya TUHAN, Allah leluhur kami! Engkau Allah di surga dan memerintah atas semua bangsa di bumi. Engkau berkuasa dan kuat; tak seorang pun sanggup melawan Engkau.
2Ch 20:7  Engkaulah Allah kami. Ketika umat-Mu, bangsa Israel, datang ke negeri ini, Engkau mengusir orang-orang yang pada waktu itu tinggal di sini. Lalu Engkau memberikan negeri ini kepada keturunan Abraham, sahabat-Mu, untuk menjadi tanah milik mereka selama-lamanya.
2Ch 20:8  Mereka telah tinggal di sini dan mendirikan sebuah rumah tempat menyembah Engkau. Mereka tahu
2Ch 20:9  bahwa kalau mereka mendapat hukuman karena dosa-dosa mereka, dan mereka ditimpa peperangan, wabah penyakit, atau kelaparan, mereka dapat datang kepada-Mu dan berdiri di depan rumah ibadat ini. Di dalam kesukaran mereka, mereka dapat berdoa kepada-Mu, dan Engkau akan mendengarkan dan menolong mereka.
2Ch 20:10  Sekarang bangsa Amon, Moab dan Edom menyerang kami. Ketika leluhur kami datang dari Mesir, Engkau tidak mengizinkan mereka menduduki negeri bangsa-bangsa itu. Maka leluhur kami mengambil jalan keliling dan tidak menghancurkan bangsa Amon, Moab dan Edom itu.
2Ch 20:11  Sekarang, beginilah tindakan bangsa-bangsa itu kepada kami. Mereka mau mengusir kami dari negeri yang Kauberikan kepada kami.
2Ch 20:12  Engkaulah Allah kami! Hukumlah mereka, karena kami ini tak berdaya menghadapi tentara yang banyak itu, yang mau menyerang kami. Kami tidak tahu harus berbuat apa, sebab itu kami datang kepada-Mu minta tolong."
2Ch 20:13  Sementara itu semua orang laki-laki Yehuda bersama dengan istri dan anak-anak mereka berdiri di Rumah TUHAN.
2Ch 20:14  Di tengah-tengah orang banyak itu ada seorang Lewi bernama Yahaziel. Ia anak Zakharia dari kaum Asaf dalam garis keturunan Benaya, Yeiel, Matanya. Pada saat itu Yahaziel dikuasai Roh TUHAN,
2Ch 20:15  dan ia berkata, "Paduka Yang Mulia serta Saudara-saudara sekalian. TUHAN berkata bahwa kalian tidak boleh putus asa atau takut menghadapi tentara yang banyak itu. Sebab yang bertempur bukan kalian, melainkan Allah.
2Ch 20:16  Besok pada waktu mereka naik melalui jalan menanjak di Zis, seranglah mereka. Kalian akan berhadapan dengan mereka pada ujung lembah di depan padang gurun dekat Yeruel.
2Ch 20:17  Kalian tidak perlu bertempur dalam perang ini. Pergilah ke tempat yang telah ditentukan bagimu dan tunggu saja di situ. Kalian akan melihat TUHAN memberikan kemenangan kepada kalian. Hai penduduk Yehuda dan Yerusalem, janganlah cemas atau takut. Masukilah saja pertempuran itu, TUHAN akan menolong kalian!"
2Ch 20:18  Maka sujudlah Raja Yosafat. Seluruh rakyat turut juga dengan dia sujud menyembah TUHAN.
2Ch 20:19  Lalu orang-orang Lewi dari kaum Kehat dan Korah berdiri, dan bersorak memuji TUHAN, Allah Israel.
2Ch 20:20  Keesokan harinya pagi-pagi benar, rakyat keluar dan pergi ke daerah padang gurun dekat Tekoa. Ketika mereka hendak berangkat, Yosafat memberikan pesan ini kepada mereka, "Rakyat Yehuda dan Yerusalem! Percayalah kepada TUHAN Allahmu, maka kamu akan sanggup bertahan. Terimalah nasihat dari nabi-nabimu, maka kamu akan berhasil!"
2Ch 20:21  Setelah berunding dengan rakyat, raja menyuruh paduan suara memakai pakaian seragam yang biasanya dipakai pada upacara-upacara khusus dalam agama. Paduan suara itu harus berbaris dengan semarak di depan barisan tentara sambil menyanyi, "Pujilah TUHAN! Kasih-Nya kekal abadi!"
2Ch 20:22  Pada waktu paduan suara itu mulai bersorak menyanyi, TUHAN mengadakan kekacauan di tengah-tengah tentara musuh yang sedang menyerang.
2Ch 20:23  Tentara Amon dan Moab menyerang tentara Edom dan menghancurkannya sama sekali. Kemudian mereka balik menyerang satu sama lain dengan sengit.
2Ch 20:24  Ketika tentara Yehuda sampai pada menara jaga di padang gurun dan mengamati tempat musuh, mereka melihat mayat-mayat musuh berserakan di tanah. Tidak seorang pun dari tentara musuh itu yang hidup.
2Ch 20:25  Maka Yosafat bersama anak buahnya pergi merampas barang-barang musuh. Mereka menemukan banyak ternak, perlengkapan, pakaian, dan barang-barang berharga lainnya. Sesudah mengumpulkan barang-barang itu tiga hari lamanya, masih banyak barang-barang lain, sehingga terpaksa ditinggalkan.
2Ch 20:26  Pada hari keempat mereka semua berkumpul di Lembah Pujian dan memuji TUHAN atas semua yang telah dilakukan-Nya. Itu sebabnya lembah itu disebut "Pujian".
2Ch 20:27  Lalu Yosafat memimpin seluruh pasukan pulang ke Yerusalem dengan sukacita karena TUHAN sudah mengalahkan musuh mereka.
2Ch 20:28  Ketika tiba di kota, mereka berbaris ke Rumah TUHAN diiringi kecapi dan trompet.
2Ch 20:29  Semua negara lain menjadi takut ketika mendengar bagaimana TUHAN mengalahkan musuh-musuh Israel.
2Ch 20:30  Itu sebabnya Yosafat memerintah dengan tenteram, dan TUHAN memberikan kepadanya keadaan damai di seluruh negeri.
2Ch 20:31  Yosafat menjadi raja Yehuda pada usia 35 tahun, dan ia memerintah di Yerusalem 25 tahun lamanya. Ibunya ialah Azuba anak Silhi.
2Ch 20:32  Seperti Asa ayahnya, Yosafat melakukan yang baik pada pemandangan TUHAN.
2Ch 20:33  Tetapi tempat-tempat penyembahan berhala tidak dihancurkannya. Rakyat belum juga dengan sepenuh hati beribadat kepada Allah leluhur mereka.
2Ch 20:34  Kisah lainnya mengenai Yosafat, dari permulaan sampai akhir pemerintahannya, dicatat dalam Riwayat Yehu Anak Hanani, yang tercantum dalam buku Sejarah Raja-raja Israel.
2Ch 20:35  Pada suatu waktu Yosafat raja Yehuda mengadakan hubungan persahabatan dengan Ahazia raja Israel, yang melakukan banyak kejahatan.
2Ch 20:36  Di pelabuhan Ezion-Geber mereka membuat kapal-kapal untuk berlayar ke Tarsis.
2Ch 20:37  Tetapi Nabi Eliezer anak Dodawa dari kota Maresa memberi peringatan kepada Yosafat, katanya, "Karena Baginda mengadakan hubungan persahabatan dengan Ahazia, TUHAN akan menghancurkan pekerjaan Baginda." Maka hancurlah kapal-kapal yang telah dibuatnya sehingga tak dapat berlayar sama sekali.
2Ch 21:1  Yosafat meninggal dan dimakamkan di makam raja-raja di Kota Daud. Yehoram putranya menjadi raja menggantikan dia.
2Ch 21:2  Yehoram putra Yosafat, raja Yehuda, mempunyai enam saudara laki-laki: Azarya, Yehiel, Zakharia, Azariahu, Mikhail, dan Sefaca.
2Ch 21:3  Mereka mendapat banyak emas, perak dan barang-barang berharga lain dari ayah mereka, dan masing-masing diberi juga satu kota berbenteng di Yehuda. Tetapi karena Yehoram putra yang sulung, dialah yang ditentukan oleh Yosafat untuk menjadi raja.
2Ch 21:4  Ketika Yehoram merasa kedudukannya sebagai raja sudah kuat, ia menyuruh membunuh semua saudara-saudaranya, dan juga beberapa pejabat tinggi.
2Ch 21:5  Yehoram menjadi raja pada usia 32 tahun, dan ia memerintah di Yerusalem selama 8 tahun.
2Ch 21:6  Ia mengikuti cara hidup Raja Ahab dan raja-raja Israel lainnya, karena ia kawin dengan salah seorang anak perempuan Ahab. Ia berdosa kepada TUHAN,
2Ch 21:7  tetapi TUHAN tidak mau membinasakan raja dari keturunan Daud karena Ia sudah berjanji bahwa keturunan Daud akan tetap memerintah.
2Ch 21:8  Dalam pemerintahan Yehoram, Edom memberontak terhadap Yehuda dan membentuk kerajaan sendiri.
2Ch 21:9  Karena itu Yehoram dan para panglimanya keluar dengan kereta-kereta perangnya lalu menyerang Edom. Ia dikepung pasukan Edom, tapi malamnya mereka menerobos kepungan musuh dan melarikan diri.
2Ch 21:10  Sejak itu Edom tidak tunduk lagi kepada Yehuda. Pada masa itu juga kota Libna pun memberontak. Semua itu terjadi karena Yehoram telah meninggalkan TUHAN, Allah yang disembah leluhurnya.
2Ch 21:11  Ia bahkan mendirikan tempat-tempat penyembahan berhala di daerah pegunungan Yehuda, dan menyebabkan rakyat Yehuda dan Yerusalem berdosa kepada TUHAN.
2Ch 21:12  Lalu Yehoram mendapat sepucuk surat yang ditulis oleh Nabi Elia ketika ia masih hidup. Begini bunyi surat itu, "TUHAN, Allah yang disembah Daud, menghukum Baginda sebab Baginda tidak mengikuti jejak ayah dan kakek Baginda.
2Ch 21:13  Sebaliknya, Baginda hidup seperti raja-raja Israel, dan menyebabkan rakyat Yehuda dan Yerusalem tidak setia kepada Allah. Baginda berbuat seperti Raja Ahab dan pengganti-penggantinya. Baginda bahkan membunuh saudara-saudara Baginda, padahal mereka lebih baik dari Baginda.
2Ch 21:14  Hukuman yang berat dari TUHAN akan menimpa rakyat, anak-anak serta istri-istri Baginda, dan membinasakan semua harta milik Baginda.
2Ch 21:15  Baginda sendiri pun akan menderita penyakit usus yang makin hari makin parah!"
2Ch 21:16  Pada waktu itu ada orang Filistin dan orang Arab yang tinggal di dekat pemukiman orang Sudan di daerah pantai. TUHAN menggerakkan hati orang-orang itu untuk memerangi Yehoram.
2Ch 21:17  Mereka menyerang Yehuda, menjarahi istana raja, lalu membawa pergi semua anak istri Yehoram sebagai tawanan, kecuali Ahazia putranya yang bungsu.
2Ch 21:18  Setelah semua kejadian itu, TUHAN menghukum Yehoram dengan penyakit usus yang tak dapat sembuh.
2Ch 21:19  Selama hampir dua tahun penyakit itu semakin parah sampai akhirnya dengan sangat menderita ia meninggal. Rakyatnya tidak menyalakan api unggun sebagai pernyataan belasungkawa seperti yang mereka lakukan untuk nenek moyangnya.
2Ch 21:20  Tidak seorang pun menyesali kematiannya. Ia dikuburkan di Kota Daud, tetapi tidak di dalam makam raja-raja. Ketika menjadi raja, Yehoram berumur 32 tahun, dan ia memerintah di Yerusalem 8 tahun lamanya.
2Ch 22:1  Dalam suatu penyerbuan oleh segerombolan orang Arab, semua putra Yehoram, kecuali Ahazia yang bungsu terbunuh. Karena itu penduduk Yerusalem mengangkat Ahazia menjadi raja menggantikan ayahnya.
2Ch 22:2  Ahazia menjadi raja pada usia 22 tahun, dan ia memerintah di Yerusalem selama satu tahun. Ahazia juga hidup seperti keluarga Raja Ahab, karena Atalya ibunya menasihatkan dia untuk melakukan yang jahat. Atalya adalah putri Raja Ahab, cucu Omri raja Israel.
2Ch 22:4  Ahazia berdosa kepada TUHAN seperti keluarga Ahab, karena merekalah yang menjadi penasihatnya setelah ayahnya meninggal. Penasihat-penasihat itulah yang mencelakakan dia.
2Ch 22:5  Atas nasihat mereka, ia bergabung dengan Yoram raja Israel untuk berperang melawan Hazael raja Siria. Mereka bertempur di Ramot, daerah Gilead. Yoram terluka dalam pertempuran itu,
2Ch 22:6  lalu ia kembali ke kota Yizreel untuk dirawat, dan Ahazia pergi menengok dia di sana.
2Ch 22:7  Allah telah menentukan bahwa Ahazia akan menemui ajalnya ketika ia mengunjungi Yoram. Pada kunjungan itu Ahazia dan Yoram mendapat perlawanan dari seorang laki-laki bernama Yehu anak Nimsi, yang telah dipilih oleh TUHAN untuk membinasakan keluarga Raja Ahab.
2Ch 22:8  Sementara Yehu melaksanakan hukuman TUHAN atas keluarga itu, ia bertemu dengan segerombolan pembesar-pembesar Yehuda dan kemanakan-kemanakan Ahazia yang ikut dalam perkunjungan itu. Maka Yehu membunuh mereka semua.
2Ch 22:9  Kemudian ia menyuruh orang mencari Ahazia yang pada waktu itu sedang bersembunyi di Samaria. Setelah Ahazia ditemukan, ia dibawa kepada Yehu lalu dibunuh. Meskipun begitu ia dikuburkan juga oleh mereka karena mereka menghormati Raja Yosafat, kakeknya yang mengabdi kepada TUHAN dengan sepenuh hati. Dari keluarga Ahazia tidak ada yang sanggup mengambil alih pemerintahan.
2Ch 22:10  Ketika Atalya mengetahui bahwa Raja Ahazia anaknya sudah mati, ia memerintahkan supaya seluruh keluarga raja di Yehuda dibunuh.
2Ch 22:11  Ahazia mempunyai seorang saudara perempuan bernama Yoseba yang kawin dengan seorang imam bernama Yoyada. Dengan diam-diam Yoseba mengambil seorang putra Ahazia yang bernama Yoas dari antara anak-anak raja yang hendak dibunuh Atalya. Anak itu bersama pengasuhnya disembunyikannya di dalam sebuah kamar di Rumah TUHAN. Demikianlah Yoseba menyelamatkan anak itu sehingga tidak dibunuh oleh Atalya.
2Ch 22:12  Enam tahun lamanya Yoas disembunyikan di Rumah TUHAN sementara Atalya memerintah sebagai ratu.
2Ch 23:1  Pada tahun ketujuh Imam Yoyada memutuskan untuk bertindak. Ia mengadakan persepakatan dengan lima perwira, yaitu Azarya anak Yeroham, Ismail anak Yohanan, Azarya anak Obed, Maaseya anak Adaya, dan Elisafat anak Zikhri.
2Ch 23:2  Mereka pergi ke semua kota di Yehuda, dan mengumpulkan orang-orang Lewi dan semua kepala kaum, lalu membawa mereka ke Yerusalem.
2Ch 23:3  Mereka berkumpul di Rumah TUHAN, dan membuat perjanjian dengan Yoas, putra raja itu. Yoyada berkata kepada mereka, "Inilah putra mendiang Raja Ahazia! Ia sekarang harus naik takhta, sebab TUHAN sudah berjanji bahwa keturunan Raja Daud akan tetap memerintah.
2Ch 23:4  Inilah yang harus kita lakukan: Apabila para imam dan orang Lewi selesai bertugas pada hari Sabat, mereka tidak boleh pulang. Sepertiga dari mereka harus ikut menjaga di pintu-pintu gerbang Rumah TUHAN,
2Ch 23:5  sepertiga lagi di istana raja, dan selebihnya di Pintu Gerbang Dasar. Seluruh rakyat harus berkumpul di halaman Rumah TUHAN.
2Ch 23:6  Tidak seorang pun boleh memasuki gedung Rumah TUHAN kecuali para imam dan orang Lewi yang sedang bertugas. Mereka boleh masuk sebab mereka sudah dikhususkan untuk TUHAN. Orang-orang lain harus taat kepada peraturan TUHAN dan tinggal di luar.
2Ch 23:7  Orang-orang Lewi harus berdiri di sekeliling raja dengan pedang terhunus, dan mengawal dia ke mana pun ia pergi. Siapa berusaha memasuki Rumah TUHAN, harus dibunuh."
2Ch 23:8  Perintah Yoyada itu dilaksanakan oleh orang-orang Lewi dan rakyat Yehuda. Orang-orang yang lepas tugas pada hari Sabat tidak dibolehkan pulang. Jadi para perwira dapat mengerahkan baik orang-orang yang lepas tugas maupun yang baru datang untuk bertugas.
2Ch 23:9  Kepada para perwira itu Yoyada membagikan tombak dan perisai milik Raja Daud yang disimpan di Rumah TUHAN.
2Ch 23:10  Yoyada juga menempatkan orang-orang dengan pedang terhunus di seputar bagian depan Rumah TUHAN untuk melindungi raja.
2Ch 23:11  Sesudah itu Yoyada membawa Yoas keluar, lalu meletakkan mahkota di kepalanya, dan menyerahkan kepadanya buku Hukum Allah. Demikianlah Yoas diangkat menjadi raja. Kemudian Imam Yoyada dengan anak-anaknya menuangkan minyak upacara penobatan pada Yoas, lalu semua orang berseru, "Hidup raja!"
2Ch 23:12  Atalya mendengar rakyat bersorak mengelu-elukan raja. Cepat-cepat ia pergi ke Rumah TUHAN di mana orang banyak berkerumun.
2Ch 23:13  Ia melihat raja yang baru itu berdiri di pintu masuk Rumah TUHAN, di mana ia dinobatkan, dikelilingi oleh perwira-perwira dan peniup trompet. Seluruh rakyat bersorak gembira sambil trompet dibunyikan, dan para pemain musik di Rumah TUHAN memeriahkan perayaan itu dengan alat-alat musik mereka. Dengan cemas Atalya merobek pakaiannya dan berteriak, "Khianat! Khianat!"
2Ch 23:14  Yoyada tidak mengizinkan Atalya dibunuh di sekitar Rumah TUHAN, jadi ia berkata kepada para perwira, "Bawalah dia ke luar melalui barisan pengawal, dan bunuh siapa saja yang berusaha menyelamatkan dia."
2Ch 23:15  Mereka menangkap dia, lalu membawanya ke istana. Di sana ia dibunuh di Pintu Gerbang Kuda.
2Ch 23:16  Imam Yoyada mengajak Raja Yoas dan rakyat bersama-sama dengan dia membuat perjanjian bahwa mereka akan menjadi umat TUHAN.
2Ch 23:17  Setelah itu mereka semua menyerbu dan meruntuhkan kuil Baal. Mezbah serta patung-patungnya dihancurkan, dan Matan, Imam Baal dibunuh di depan mezbah-mezbah itu.
2Ch 23:18  Lalu Yoyada menugaskan para imam dan orang Lewi untuk mengurus pekerjaan di Rumah TUHAN menurut pembagian kerja yang telah ditentukan oleh Raja Daud. Mereka harus mempersembahkan kurban bakaran kepada TUHAN--sesuai dengan hukum-hukum Musa--dengan perayaan dan nyanyian-nyanyian seperti yang telah diatur oleh Daud.
2Ch 23:19  Di pintu-pintu gerbang Rumah TUHAN, Yoyada menempatkan pengawal untuk menjaga supaya orang yang tidak patut menyembah TUHAN jangan masuk.
2Ch 23:20  Para perwira, para tokoh masyarakat, para pejabat pemerintah, dan seluruh rakyat mengikuti Yoyada mengarak Raja Yoas dari Rumah TUHAN ke istana. Mereka masuk dari pintu gerbang utama lalu raja duduk di atas singgasana.
2Ch 23:21  Seluruh rakyat gembira, kota pun tenteram dan Atalya sudah dibunuh.
2Ch 24:1  Yoas berumur 7 tahun pada waktu ia menjadi raja Yehuda, dan ia memerintah di Yerusalem 40 tahun lamanya. Ibunya bernama Zibya dari kota Bersyeba.
2Ch 24:2  Selama Imam Yoyada masih hidup, Yoas melakukan hal-hal yang menyenangkan hati TUHAN.
2Ch 24:3  Yoyada memilih dua orang wanita untuk istri Yoas, dan dari mereka Yoas mendapat anak-anak laki-laki dan perempuan.
2Ch 24:4  Setelah memerintah beberapa waktu lamanya, Yoas memutuskan untuk memperbaiki Rumah TUHAN.
2Ch 24:5  Para imam dan orang Lewi diperintahkannya untuk pergi ke penduduk kota-kota di Yehuda dan mengumpulkan uang dari seluruh rakyat untuk perbaikan gedung Rumah TUHAN setiap tahun. Ia menyuruh mereka segera melaksanakan tugas itu, tetapi orang-orang Lewi berlambat-lambat.
2Ch 24:6  Karena itu ia memanggil Imam Yoyada, pemimpin mereka, lalu berkata, "Bukankah Musa hamba TUHAN telah menyuruh rakyat membayar pajak untuk membiayai Kemah TUHAN? Sekarang, mengapa orang-orang Lewi itu tidak Bapak desak untuk mengumpulkan pajak dari rakyat di Yehuda dan Yerusalem?"
2Ch 24:7  (Pengikut-pengikut Atalya, wanita jahat itu, sudah merusak Rumah TUHAN, dan banyak dari barang-barang yang dikhususkan untuk Allah telah mereka pakai untuk menyembah Baal.)
2Ch 24:8  Maka raja memerintahkan supaya orang Lewi membuat sebuah kotak uang dan meletakkannya di depan pintu gerbang Rumah TUHAN.
2Ch 24:9  Di seluruh Yerusalem dan Yehuda diumumkan bahwa semua orang harus membawa pajak untuk TUHAN, seperti yang diwajibkan oleh Musa.
2Ch 24:10  Pengumuman itu menyenangkan hati rakyat dan pemimpin-pemimpin mereka. Beramai-ramai mereka memasukkan uang pajak mereka ke dalam kotak itu.
2Ch 24:11  Setiap hari kotak itu dibawa oleh orang-orang Lewi kepada pegawai raja untuk diperiksa. Apabila kotak itu telah penuh sekretaris negara dan wakil Imam Agung mengeluarkan uangnya lalu kotak itu dikembalikan ke tempatnya yang semula. Demikianlah mereka lakukan setiap hari, dan banyak sekali uang yang terkumpul.
2Ch 24:12  Kemudian raja dan Yoyada menyerahkan uang itu kepada orang-orang yang mengepalai perbaikan Rumah TUHAN itu. Lalu orang-orang itu mengupah tukang batu, tukang kayu, dan tukang logam untuk mengerjakan pekerjaan itu.
2Ch 24:13  Maka mulailah tukang-tukang itu memperbaiki Rumah TUHAN itu. Mereka semua bekerja begitu sehingga akhirnya Rumah itu menjadi kuat dan bagus seperti semula.
2Ch 24:14  Setelah pekerjaan itu selesai, emas dan perak yang selebihnya diserahkan kepada raja dan Yoyada. Lalu emas dan perak itu dipakai untuk membuat perkakas-perkakas bagi Rumah TUHAN, antara lain: alat-alat untuk upacara ibadat, untuk kurban bakaran, piring-piring untuk dupa, dan mangkuk-mangkuk. Selama Imam Yoyada masih hidup, kurban bakaran tetap dipersembahkan di Rumah TUHAN.
2Ch 24:15  Yoyada mencapai usia yang lanjut dan meninggal ketika berumur 130 tahun.
2Ch 24:16  Ia dikuburkan di makam raja-raja di Kota Daud untuk menghargai jasa-jasanya kepada Allah, kepada Rumah TUHAN dan kepada umat Israel.
2Ch 24:17  Setelah Yoyada meninggal, Yoas dibujuk dan dipengaruhi oleh pemimpin-pemimpin Yehuda sehingga ia tidak lagi mengindahkan ajaran Yoyada.
2Ch 24:18  Maka rakyat tidak lagi beribadat di Rumah TUHAN, Allah leluhur mereka, tetapi menyembah berhala dan patung-patung Dewi Asyera. Dosa mereka itu menyebabkan TUHAN marah kepada Yehuda dan Yerusalem.
2Ch 24:19  TUHAN mengutus nabi-nabi untuk menegur mereka supaya kembali mengabdi kepada TUHAN, tetapi rakyat tidak mau mendengarkan.
2Ch 24:20  Lalu Roh Allah menguasai Zakharia anak Imam Yoyada. Zakharia tampil di depan rakyat, lalu berseru, "TUHAN Allah berkata, 'Mengapa kamu tidak taat kepada perintah-perintah-Ku dan mencelakakan dirimu sendiri. Kamu tidak peduli kepada-Ku, maka Aku pun tidak peduli kepadamu!'"
2Ch 24:21  Tetapi rakyat berkomplot menentang Zakharia, dan Raja Yoas ikut dalam komplotan itu. Ia tidak ingat akan jasa-jasa dan kesetiaan Yoyada, ayah Zakharia. Atas perintah raja, Zakharia dilempari batu sampai mati di pelataran Rumah TUHAN. Pada saat akan meninggal, Zakharia berseru, "Semoga TUHAN melihat perbuatanmu ini dan membalasnya!"
2Ch 24:23  Tahun itu juga, pada pergantian musim, tentara Siria menyerang Yehuda dan Yerusalem. Mereka membunuh semua pemimpin dan mengangkut banyak barang rampasan ke Damsyik.
2Ch 24:24  Tentara Siria sedikit saja jumlahnya, tetapi TUHAN membiarkan mereka mengalahkan tentara Yehuda yang jauh lebih besar, karena orang Yehuda tidak menghiraukan TUHAN Allah leluhur mereka. Dengan cara itu TUHAN menghukum Raja Yoas.
2Ch 24:25  Yoas luka berat. Setelah tentara musuh pergi, dua orang pegawainya berkomplot. Mereka membunuh dia di tempat tidurnya sebagai balasan terhadap pembunuhan anak Imam Yoyada. Raja Yoas dikuburkan di Kota Daud, tetapi tidak di makam raja-raja.
2Ch 24:26  (Orang-orang yang berkomplot melawan dia adalah Zabad anak Simeat wanita Amon, dan Yozabad anak Simrit wanita Moab.)
2Ch 24:27  Kisah mengenai anak-anak Yoas, mengenai ramalan-ramalan yang tidak baik tentang dia, dan mengenai bagaimana ia memperbaiki Rumah TUHAN, semuanya tercantum dalam buku Penjelasan Kitab Raja-raja. Amazia, putra Yoas, menjadi raja menggantikan ayahnya.
2Ch 25:1  Amazia menjadi raja ketika berumur 25 tahun, dan ia memerintah di Yerusalem 29 tahun lamanya. Ibunya ialah Yoadan dari Yerusalem.
2Ch 25:2  Amazia melakukan yang menyenangkan hati TUHAN, tetapi dengan setengah hati.
2Ch 25:3  Segera setelah kuat kedudukannya, ia menghukum mati pegawai-pegawai yang membunuh ayahnya.
2Ch 25:4  Tetapi anak-anak pegawai-pegawai itu tidak dibunuhnya karena ia menuruti perintah TUHAN dalam Buku Musa yang berbunyi, "Jangan menghukum mati orang tua karena kejahatan yang dilakukan oleh anak-anak mereka, dan jangan menghukum mati anak-anak karena kejahatan yang dilakukan oleh orang tua mereka. Setiap orang hanya boleh dihukum mati karena kejahatan yang dilakukannya sendiri."
2Ch 25:5  Raja Amazia mengumpulkan semua orang suku Yehuda dan Benyamin lalu menyuruh mereka berdiri menurut kaum masing-masing, dipimpin oleh komandan pasukan seribu dan komandan pasukan seratus. Kemudian ia memilih orang-orang yang berumur 20 tahun ke atas yang mampu bertempur dan pandai menggunakan tombak dan perisai. Jumlah mereka ada 300.000 orang.
2Ch 25:6  Selain itu ia menyewa 100.000 prajurit dari Israel dengan bayaran kira-kira 3.400 kilogram perak.
2Ch 25:7  Tetapi seorang nabi datang kepada raja dan berkata, "Jangan pakai prajurit-prajurit Israel itu, sebab TUHAN tidak menolong umat Israel yang dari Kerajaan Utara itu.
2Ch 25:8  Tetapi kalau Baginda memakai mereka juga, Allah akan membiarkan Baginda dikalahkan oleh musuh meskipun tentara Baginda kuat dan berani. Sebab hanya Allah saja berkuasa memberi kemenangan atau kekalahan."
2Ch 25:9  "Tapi bagaimana dengan semua perak yang telah kuberikan kepada mereka?" tanya Amazia. Nabi itu menjawab, "TUHAN bisa mengembalikan kepada Baginda lebih dari itu!"
2Ch 25:10  Karena itu Amazia membebaskan prajurit-prajurit sewaan itu dan menyuruh mereka pulang ke negerinya. Mereka pun pulang, tetapi dengan marah sekali.
2Ch 25:11  Amazia memberanikan diri dan membawa tentaranya ke Lembah Asin. Di sana mereka bertempur serta menewaskan 10.000 prajurit Edom.
2Ch 25:12  Sepuluh ribu yang lain ditawan. Mereka dibawa ke puncak gunung batu di kota Sela lalu dilemparkan ke bawah sehingga mati.
2Ch 25:13  Sementara itu prajurit-prajurit Israel yang telah disuruh pulang oleh Amazia, dan tidak diizinkan turut bertempur, telah menyerang kota-kota Yehuda antara Samaria dan Bet-Horon. Mereka membunuh 3.000 orang serta merampas banyak barang.
2Ch 25:14  Setelah mengalahkan tentara Edom, Amazia pulang membawa patung-patung berhala orang Edom dan menempatkannya di Yehuda. Lalu ia membakar dupa untuk patung-patung itu dan menyembahnya.
2Ch 25:15  Perbuatannya itu membuat TUHAN menjadi marah. Ia mengutus seorang nabi kepada Amazia. Nabi itu berkata, "Mengapa Baginda menyembah dewa bangsa lain yang tidak dapat menyelamatkan bangsanya sendiri dari kekuasaan Baginda?"
2Ch 25:16  "Diam," kata Amazia, "Kalau tidak, kami bunuh kau! Tidak pernah kami mengangkat kau menjadi penasihat raja!" Maka diamlah nabi itu, tetapi sebelumnya ia sempat berkata, "Sekarang saya tahu bahwa Allah sudah memutuskan untuk membinasakan Baginda karena perbuatan Baginda itu, dan karena Baginda tidak menghiraukan nasihat saya."
2Ch 25:17  Amazia raja Yehuda dengan para penasihatnya berkomplot melawan Israel. Ia mengirim utusan kepada Yoas raja Israel, anak Yoahas, cucu Yehu, untuk menantang dia berperang.
2Ch 25:18  Tetapi Raja Yoas mengirim jawaban ini, "Suatu waktu semak berduri di pegunungan Libanon mengirim tuntutan ini kepada pohon cemara, 'Hai pohon cemara, berikanlah anak gadismu kepada anakku untuk menjadi istrinya.' Tetapi kemudian lewatlah di situ seekor binatang hutan yang menginjak-injak semak berduri itu.
2Ch 25:19  Engkau, Amazia, membual bahwa kau telah mengalahkan orang Edom, tapi lebih baik kau tinggal saja di rumah. Untuk apa mencari-cari persoalan yang hanya akan mencelakakan dirimu dan rakyatmu?"
2Ch 25:20  Tetapi Amazia, raja Yehuda tidak menghiraukan kata-kata Yoas raja Israel. Dan Allah memang menghendaki supaya Amazia dikalahkan karena ia telah menyembah patung berhala orang Edom.
2Ch 25:21  Karena itu Yoas berangkat dengan anak buahnya ke Bet-Semes dan berperang dengan Amazia di sana.
2Ch 25:22  Tentara Yehuda dikalahkan dan semua prajuritnya lari pulang ke rumahnya masing-masing.
2Ch 25:23  Yoas menangkap Amazia dan membawanya ke Yerusalem, lalu meruntuhkan tembok kota itu sepanjang kurang lebih 200 meter, mulai dari Pintu Gerbang Efraim sampai ke Pintu Gerbang Sudut.
2Ch 25:24  Semua emas dan perak serta semua perkakas yang ditemukannya di Rumah TUHAN dan yang berada di bawah pengawasan Obed-Edom, juga semua harta benda istana, bersama beberapa orang sandera diangkut oleh Yoas ke Samaria.
2Ch 25:25  Setelah Yoas raja Israel meninggal, Amazia raja Yehuda masih hidup 15 tahun lagi.
2Ch 25:26  Sejak Amazia meninggalkan TUHAN, orang berkomplot melawan dia di Yerusalem. Karena itu, ia lari ke kota Lakhis, tetapi musuh-musuhnya mengejar dia ke sana dan membunuhnya. Jenazahnya diangkut dengan kuda ke Yerusalem, lalu dikuburkan di makam raja-raja di Kota Daud. Kisah lainnya mengenai Raja Amazia dari mula sampai akhir pemerintahannya dicatat dalam buku Sejarah Raja-raja Yehuda dan Israel.
2Ch 26:1  Sesudah Amazia meninggal, seluruh rakyat Yehuda memilih Uzia putranya menjadi raja. Pada waktu itu ia berumur 16 tahun. Ibunya bernama Yekholya, wanita Yerusalem. Uzia memerintah di Yerusalem 52 tahun lamanya. Dialah yang merebut dan membangun kembali kota Elot.
2Ch 26:4  Seperti ayahnya, Uzia pun melakukan yang menyenangkan hati TUHAN.
2Ch 26:5  Zakharia adalah penasihat rohaninya. Selama Zakharia masih hidup, Uzia dengan setia beribadat kepada TUHAN sehingga TUHAN memberkatinya.
2Ch 26:6  Uzia bertempur melawan orang Filistin. Ia meruntuhkan tembok-tembok kota Gat, Yabne, dan Asdod, lalu mendirikan kota-kota berbenteng dekat Asdod dan di daerah lain di Filistin.
2Ch 26:7  Allah menolong dia mengalahkan orang Filistin, dan orang Arab yang tinggal di Gur-Baal serta orang Meunim.
2Ch 26:8  Orang Amon membayar upeti kepada Uzia. Ia makin berkuasa sehingga namanya tersohor sampai ke Mesir.
2Ch 26:9  Untuk memperkuat kota Yerusalem, Uzia mendirikan menara di Pintu Gerbang Sudut, di Pintu Gerbang Lembah dan di tikungan tembok.
2Ch 26:10  Ia juga mendirikan menara-menara berbenteng di daerah padang, dan menggali banyak sumur karena ternaknya banyak sekali, baik di dataran tinggi maupun di dataran rendah di sebelah barat. Karena ia menaruh perhatian pada pertanian, maka ia mempekerjakan banyak orang untuk membuka kebun-kebun anggur di daerah pegunungan, dan untuk menggarap tanah di daerah-daerah yang subur.
2Ch 26:11  Uzia mempunyai tentara besar yang selalu siap untuk bertempur. Mereka terdiri atas kesatuan-kesatuan yang jumlahnya dicatat oleh kedua sekretaris raja, yaitu Yeiel dan Maaseya di bawah pimpinan Hananya, salah seorang anggota staf raja.
2Ch 26:12  Seluruh angkatan bersenjata itu terdiri dari 307.500 prajurit dipimpin oleh 2.600 perwira. Mereka semua mempunyai kemampuan yang sangat besar untuk bertempur dan membantu raja mengalahkan musuh.
2Ch 26:14  Uzia memperlengkapi tentaranya itu dengan perisai tombak, topi baja, baju besi, busur dan panah, serta batu pengumban.
2Ch 26:15  Di Yerusalem para ahli senjatanya telah menciptakan alat yang dapat membidikkan anak panah dan batu-batu besar dari dalam menara dan sudut-sudut tembok kota. Uzia termasyhur di mana-mana, dan ia menjadi sangat berkuasa karena Allah sangat menolong dia.
2Ch 26:16  Setelah Raja Uzia kuat, ia menjadi sombong, dan itu menyebabkan kehancurannya. Ia melanggar perintah TUHAN Allahnya karena memasuki Rumah TUHAN untuk membakar dupa di atas mezbah dupa.
2Ch 26:17  Tetapi Imam Azarya bersama 80 imam lain yang kuat-kuat dan berani mengikuti dia
2Ch 26:18  untuk menentang perbuatannya itu. Mereka berkata, "Paduka Yang Mulia! Baginda tidak berhak membakar dupa untuk TUHAN. Itu tugas imam-imam keturunan Harun, sebab merekalah yang dikhususkan untuk itu. Hendaklah Baginda meninggalkan tempat yang suci ini! Baginda telah melawan TUHAN Allah, dan tidak lagi mendapat restu-Nya."
2Ch 26:19  Pada waktu itu Uzia sedang berdiri di dekat mezbah dupa di Rumah TUHAN dengan alat pembakar dupa di tangannya. Ia marah kepada imam-imam itu, dan tiba-tiba timbul penyakit kulit yang mengerikan pada dahinya.
2Ch 26:20  Azarya dan imam-imam lain melihat kepadanya dengan sangat terkejut, lalu mendesak supaya ia meninggalkan Rumah TUHAN. Cepat-cepat ia ke luar karena TUHAN telah menghukumnya.
2Ch 26:21  Sampai Raja Uzia meninggal penyakitnya itu tidak sembuh-sembuh. Ia diasingkan di sebuah rumah dan tidak diizinkan lagi memasuki Rumah TUHAN. Ia dibebaskan dari tugas-tugasnya, dan Yotam putranya memerintah rakyat sebagai wakilnya.
2Ch 26:22  Kisah lain mengenai pemerintahan Raja Uzia, sudah dicatat oleh Nabi Yesaya anak Amos.
2Ch 26:23  Ketika meninggal, Uzia dikuburkan di tanah dekat makam raja-raja, tetapi tidak di makam itu sendiri, sebab ia berpenyakit kulit yang mengerikan. Yotam putranya menjadi raja menggantikan dia.
2Ch 27:1  Yotam menjadi raja pada usia 25 tahun, dan ia memerintah di Yerusalem 16 tahun lamanya. Ibunya ialah Yerusa anak Zadok.
2Ch 27:2  Yotam menyenangkan hati TUHAN, seperti Uzia ayahnya. Hanya ia tidak membakar dupa di Rumah TUHAN, seperti yang dilakukan ayahnya. Sekalipun demikian, rakyat terus saja berbuat dosa.
2Ch 27:3  Yotamlah yang membangun Pintu Gerbang Utara di Rumah TUHAN, dan memperkuat tembok Yerusalem di daerah yang disebut Ofel.
2Ch 27:4  Ia mendirikan kota-kota di pegunungan Yehuda dan benteng-benteng serta menara-menara di hutan-hutan.
2Ch 27:5  Ia memerangi dan mengalahkan raja Amon bersama tentaranya, lalu memaksa orang Amon membayar upeti. Setiap tahun selama tiga tahun berturut-turut mereka harus membayar 3.400 kilogram perak, dua jenis gandum, masing-masing 1.000 ton.
2Ch 27:6  Yotam menjadi semakin berkuasa karena ia setia dan taat kepada TUHAN Allahnya.
2Ch 27:7  Peristiwa-peristiwa lain dalam pemerintahan Yotam, peperangan-peperangan dan siasat-siasatnya, semuanya sudah dicatat dalam buku Sejarah Raja-raja Israel dan Yehuda.
2Ch 27:8  Yotam berumur 25 tahun ketika ia menjadi raja, dan ia memerintah di Yerusalem selama 16 tahun.
2Ch 27:9  Ia meninggal dan dikuburkan di Kota Daud, lalu Ahas putranya menjadi raja menggantikan dia.
2Ch 28:1  Ahas menjadi raja pada usia 20 tahun, dan ia memerintah di Yerusalem 16 tahun lamanya. Ia tidak mengikuti teladan yang baik dari Raja Daud, leluhurnya, melainkan melakukan yang tidak menyenangkan hati TUHAN.
2Ch 28:2  Ia hidup seperti raja-raja Israel dan menyuruh orang membuat patung-patung Baal dari logam,
2Ch 28:3  serta membakar dupa di Lembah Hinom. Bahkan putranya sendiri dipersembahkannya sebagai kurban kepada berhala, menurut kebiasaan buruk orang-orang yang telah diusir TUHAN dari negeri Kanaan ketika orang Israel memasuki negeri itu.
2Ch 28:4  Ahas mempersembahkan kurban dan membakar dupa di tempat-tempat penyembahan dewa di gunung-gunung dan di bawah pohon-pohon yang rindang.
2Ch 28:5  Karena Raja Ahas berdosa, TUHAN Allahnya membiarkan dia dikalahkan oleh raja Siria yang juga mengangkut sejumlah besar orang Yehuda sebagai tawanan ke Damsyik. TUHAN juga membiarkan dia dikalahkan oleh Raja Pekah anak Remalya dari Israel, yang dalam satu hari membunuh 120.000 prajurit Yehuda yang berani-berani. TUHAN, Allah leluhur mereka membiarkan semuanya itu terjadi, karena rakyat Yehuda tidak lagi menghiraukan TUHAN.
2Ch 28:7  Putra Ahas, yang bernama Maaseya dibunuh oleh Zikhri seorang perwira Israel. Zikhri juga membunuh Azrikam pengurus istana raja, dan Elkana, tangan kanan raja.
2Ch 28:8  Meskipun orang-orang Yehuda adalah sesama bangsa Israel, namun tentara Israel menawan 200.000 wanita dan anak-anak, lalu mengangkut mereka ke Samaria bersama-sama dengan banyak sekali barang rampasan.
2Ch 28:9  Seorang nabi TUHAN bernama Oded tinggal di kota Samaria. Pada waktu tentara Israel kembali membawa tawanan dari Yehuda, Oded menemui mereka ketika mereka akan memasuki kota. Ia berkata, "Kalian telah mengalahkan orang Yehuda karena TUHAN Allah leluhurmu marah kepada mereka. Tetapi TUHAN mendengar bagaimana kejamnya kalian membunuh tentara Yehuda itu.
2Ch 28:10  Dan sekarang orang-orang Yerusalem dan Yehuda hendak kalian jadikan hamba-hambamu! Tidakkah kalian tahu bahwa kalian pun telah berdosa terhadap TUHAN Allahmu?
2Ch 28:11  Dengar! Para tawanan ini adalah saudara-saudaramu. Jadi, suruhlah mereka pulang. Kalau tidak, TUHAN akan marah dan menghukum kalian."
2Ch 28:12  Empat tokoh dari Israel, yaitu Azarya anak Yohanam, Berekhya anak Mesilemot, Yehizkia anak Salum, dan Amasa anak Hadlai, juga menentang tindakan tentara yang pulang itu.
2Ch 28:13  Mereka berkata, "Jangan bawa tawanan-tawanan itu ke mari! Kita sudah berdosa terhadap TUHAN, dan dosa-dosa kita sudah cukup besar untuk dihukum. Sekarang kalian mau membuat sesuatu yang hanya akan menambah kesalahan kita."
2Ch 28:14  Maka prajurit-prajurit itu menyerahkan para tawanan dan barang-barang rampasan itu kepada rakyat dan pemimpin-pemimpin mereka.
2Ch 28:15  Lalu keempat tokoh tersebut bangkit untuk mengurus para tawanan itu. Mereka membagikan pakaian dari barang rampasan itu kepada para tawanan yang memerlukannya. Mereka juga membagikan sepatu, makanan dan minuman. Para tawanan yang luka-luka diobati dengan minyak zaitun. Yang terlalu lemah untuk berjalan, dinaikkan ke atas keledai, lalu semua tawanan itu dibawa ke Yerikho, kota pohon-pohon kurma, supaya dapat pulang ke kampung halaman mereka di Yehuda. Setelah itu orang Israel pulang ke Samaria.
2Ch 28:16  Orang-orang Edom mulai lagi menyerbu Yehuda dan menawan banyak orang. Karena itu Raja Ahas minta kepada Tiglat-Pileser, raja Asyur, supaya mengirim bantuan.
2Ch 28:18  Pada waktu itu juga orang-orang Filistin sedang menyerbu kota-kota di daerah kaki gunung di sebelah barat, dan di daerah sebelah selatan Yehuda. Mereka merebut kota Bet-Semes, Ayalon, dan Gederot, serta kota Sokho, Timna, dan Gomzo serta desa-desa di sekitar ketiga kota itu, lalu menetap di situ.
2Ch 28:19  Karena Ahas raja Yehuda meninggalkan TUHAN dan tidak memerintah rakyatnya sebagaimana mestinya, maka TUHAN mendatangkan kesusahan di Yehuda.
2Ch 28:20  Raja Asyur itu datang bukan untuk membantu Ahas, melainkan untuk menyusahkannya.
2Ch 28:21  Terpaksa Ahas mengambil emas dari Rumah TUHAN, dari istana, dan dari rumah-rumah para pemimpin rakyat, lalu menyerahkannya kepada raja Asyur. Tetapi itu pun tidak berguna baginya.
2Ch 28:22  Dalam keadaan yang sangat sulit itu, Ahas malah semakin berdosa terhadap TUHAN.
2Ch 28:23  Ia mencelakakan dirinya dan rakyatnya karena mempersembahkan kurban kepada dewa-dewa orang Siria yang sudah mengalahkan dia. Ia berpikir, "Dewa-dewa Siria telah membantu raja-raja Siria, jadi kalau saya mempersembahkan kurban kepada mereka, barangkali saya akan ditolongnya juga."
2Ch 28:24  Kemudian ia mengambil semua perkakas Rumah TUHAN dan menghancurkannya. Lalu ditutupnya Rumah TUHAN, dan didirikannya mezbah-mezbah di seluruh Yerusalem.
2Ch 28:25  Di setiap kota dan desa di Yehuda ia membangun tempat-tempat penyembahan berhala untuk membakar dupa bagi dewa-dewa bangsa lain. Dengan demikian ia membuat TUHAN, Allah para leluhurnya marah kepadanya.
2Ch 28:26  Kisah lain mengenai pemerintahannya dan semua perbuatannya dari mula sampai akhir dicatat di dalam buku Sejarah Raja-raja Yehuda dan Israel.
2Ch 28:27  Ahas meninggal; dan dikuburkan di Yerusalem, tetapi tidak di makam raja-raja. Hizkia putranya menjadi raja menggantikan dia.
2Ch 29:1  Hizkia menjadi raja Yehuda pada usia 25 tahun, dan ia memerintah di Yerusalem 29 tahun lamanya. Ibunya bernama Abia anak Zakharia.
2Ch 29:2  Hizkia melakukan yang menyenangkan hati TUHAN seperti Raja Daud leluhurnya.
2Ch 29:3  Pada bulan satu dalam tahun pertama pemerintahannya, Hizkia membuka pintu-pintu Rumah TUHAN dan menyuruh orang memperbaikinya.
2Ch 29:4  Imam-imam dan orang Lewi disuruhnya berkumpul di pelataran bagian timur Rumah TUHAN,
2Ch 29:5  dan ia berbicara dengan mereka di sana. Ia berkata, "Kamu, orang-orang Lewi, harus mengkhususkan diri untuk TUHAN, dan menyucikan Rumah TUHAN, Allah leluhurmu. Semua yang menajiskan rumah itu harus disingkirkan.
2Ch 29:6  Leluhur kita tidak setia kepada TUHAN Allah kita dan mereka melakukan yang tidak menyenangkan hati-Nya. Mereka tidak menghiraukan Dia dan tidak mempedulikan tempat kediaman-Nya.
2Ch 29:7  Mereka menutup pintu-pintu Rumah TUHAN dan memadamkan pelita-pelitanya. Mereka tidak membakar dupa di Rumah TUHAN, dan tidak mempersembahkan kurban bakaran kepada Allah yang disembah oleh orang Israel.
2Ch 29:8  Lalu TUHAN dalam kemarahan-Nya menghukum orang Yehuda dan Yerusalem. Dan kamu semua tahu betapa tindakan TUHAN itu telah membuat semua orang terkejut dan ngeri.
2Ch 29:9  Bapak-bapak kita tewas dalam pertempuran, dan anak istri kita diangkut sebagai tawanan.
2Ch 29:10  Sekarang aku mau membuat perjanjian dengan TUHAN, Allah Israel, supaya Ia tidak marah lagi kepada kita.
2Ch 29:11  Anak-anakku, janganlah menunda-nunda lagi, sebab kamulah yang dipilih TUHAN untuk beribadat dan mengabdi kepada-Nya serta membakar dupa dan memimpin rakyat untuk beribadat."
2Ch 29:12  Orang-orang Lewi yang hadir pada waktu itu adalah: Dari kaum Kehat: Mahat anak Amasai, dan Yoel anak Azarya. Dari kaum Merari: Kish anak Abdi, dan Azarya anak Yehaleleel. Dari kaum Gerson: Yoah anak Zima, dan Eden anak Yoah. Dari kaum Elisafan: Simri dan Yeiel. Dari kaum Asaf: Zakharia dan Matanya. Dari kaum Heman: Yehiel dan Simei. Dari kaum Yedutun: Semaya dan Uziel.
2Ch 29:15  Kemudian mereka mengumpulkan orang-orang Lewi lainnya, dan mereka semua menyucikan diri. Setelah itu, sesuai dengan perintah raja, para imam masuk ke dalam Rumah TUHAN dan mulai menyucikannya menurut hukum-hukum TUHAN. Semua yang najis dibawa keluar ke halaman Rumah TUHAN. Dari situ orang Lewi membawanya ke Lembah Kidron di luar kota.
2Ch 29:17  Pekerjaan itu dimulai pada tanggal satu bulan satu. Pada tanggal delapan mereka selesai menyucikan seluruh pelataran Rumah TUHAN sampai balai depannya. Kemudian selama 8 hari lagi mereka menyucikan gedung itu seluruhnya sampai selesai pada tanggal 16 bulan itu.
2Ch 29:18  Lalu orang-orang Lewi itu melaporkan kepada Raja Hizkia, "Kami sudah selesai menyucikan Rumah TUHAN seluruhnya, termasuk mezbah kurban bakaran, meja untuk roti sajian, dan semua perkakasnya.
2Ch 29:19  Kami juga sudah mengambil kembali perkakas-perkakas yang disingkirkan Raja Ahas pada waktu ia tidak menghiraukan TUHAN dalam masa pemerintahannya. Semua perkakas itu telah kami khususkan lagi untuk TUHAN, dan sekarang ada di depan mezbah."
2Ch 29:20  Tanpa menunggu lama-lama Raja Hizkia mengumpulkan tokoh-tokoh masyarakat, lalu pergi bersama mereka ke Rumah TUHAN.
2Ch 29:21  Untuk kurban pengampunan dosa bagi keluarga raja dan rakyat Yehuda, dan untuk menyucikan Rumah TUHAN, mereka membawa 7 sapi jantan, 7 domba jantan, 7 anak domba, dan 7 kambing jantan. Raja menyuruh imam-imam keturunan Harun mempersembahkan binatang-binatang itu di atas mezbah sebagai kurban bakaran.
2Ch 29:22  Sapi-sapi itu disembelih lalu imam-imam menyiramkan darahnya pada mezbah. Kemudian mereka berbuat begitu juga dengan ketujuh domba dan ketujuh anak domba itu.
2Ch 29:23  Akhirnya mereka mengambil kambing-kambing itu dan membawanya kepada raja dan rakyat yang sedang beribadat di situ, lalu mereka semua meletakkan tangan mereka ke atas kambing-kambing itu.
2Ch 29:24  Setelah itu kambing-kambing itu disembelih oleh imam-imam dan darahnya dituangkan pada mezbah sebagai kurban pengampunan dosa seluruh rakyat. Hal itu mereka lakukan karena raja sudah memerintahkan supaya diadakan kurban bakaran dan kurban pengampunan dosa untuk seluruh rakyat Israel.
2Ch 29:25  Sesuai dengan petunjuk-petunjuk yang diberikan TUHAN kepada Daud melalui Nabi Gad, yang bekerja bagi raja dan melalui Nabi Natan, raja menempatkan orang-orang Lewi di Rumah TUHAN untuk bermain kecapi dan gambus,
2Ch 29:26  seperti yang dipakai oleh Daud. Juga para imam ditempatkannya di situ dengan trompet.
2Ch 29:27  Hizkia memberi tanda supaya upacara persembahan kurban dimulai. Sementara kurban dipersembahkan, orang-orang mulai menyanyikan pujian kepada TUHAN, dan para pemain musik membunyikan trompet serta alat-alat musik lain.
2Ch 29:28  Raja Hizkia dan semua orang yang ada di situ sujud menyembah TUHAN sementara para penyanyi terus menyanyi dan para pemain musik terus memainkan musiknya sampai semua kurban itu terbakar habis.
2Ch 29:30  Raja dan tokoh-tokoh masyarakat menyuruh orang-orang Lewi menyanyikan untuk TUHAN puji-pujian yang dikarang oleh Daud dan Nabi Asaf. Semua orang menyanyi dengan sangat gembira sambil berlutut dan menyembah Allah.
2Ch 29:31  Setelah itu Hizkia berkata, "Karena kalian sekarang sudah dikhususkan untuk TUHAN, maka sebagai persembahan syukurmu, kalian harus membawa kurban ke Rumah TUHAN." Lalu semua orang yang hadir di situ membawa persembahan-persembahan itu, bahkan ada pula yang dengan sukarela membawa binatang-binatang untuk dipersembahkan sebagai kurban bakaran.
2Ch 29:32  Mereka membawa 70 sapi jantan, 100 domba, dan 200 anak domba serta mempersembahkannya untuk kurban bakaran kepada TUHAN.
2Ch 29:33  Mereka juga membawa 600 sapi jantan dan 3.000 domba untuk dipersembahkan dan dimakan bersama oleh semua yang hadir.
2Ch 29:34  Imam-imam yang sudah menyucikan diri terlalu sedikit jumlahnya untuk menguliti semua binatang itu. Sebab itu mereka dibantu oleh orang-orang Lewi, karena orang Lewi lebih setia pada peraturan penyucian. Mereka tetap membantu sampai ada cukup imam yang sudah menyucikan diri.
2Ch 29:35  Selain kurban bakaran, para imam juga mempersembahkan lemak dari kurban syukur yang dimakan oleh rakyat. Mereka juga mempersembahkan anggur yang dituang bersama kurban bakaran itu. Demikianlah ibadat di Rumah TUHAN diadakan lagi.
2Ch 29:36  Maka senanglah hati Raja Hizkia dan rakyatnya karena dalam waktu singkat Allah telah membantu mereka melakukan semuanya itu.
2Ch 30:1  Perayaan Paskah belum dapat diadakan oleh rakyat pada waktunya dalam bulan satu, karena imam-imam yang sudah menyucikan diri terlalu sedikit jumlahnya dan belum banyak orang yang berkumpul di Yerusalem. Karena itu Raja Hizkia dan pegawai-pegawainya serta penduduk Yerusalem sepakat untuk merayakan Paskah itu dalam bulan dua, dan mereka semua senang dengan rencana itu. Lalu raja mengirim surat kepada semua orang di seluruh Israel dan Yehuda. Juga kepada orang Efraim dan Manasye. Seluruh bangsa Israel dari wilayah suku Dan di sebelah utara sampai ke Bersyeba di selatan, diundang ke Rumah TUHAN di Yerusalem untuk merayakan Paskah dan menghormati TUHAN Allah Israel seperti yang ditentukan dalam hukum-hukum Musa. Belum pernah Paskah dirayakan seperti itu.
2Ch 30:6  Atas perintah raja dan pegawai-pegawainya dikirimlah utusan-utusan ke seluruh Yehuda dan Israel untuk membawa surat-surat itu yang berbunyi sebagai berikut, "Rakyat Israel! Kamu sudah terluput dari kekuasaan orang Asyur. Sekarang hendaklah kamu kembali mengabdi kepada TUHAN, Allah yang disembah oleh Abraham, Ishak, dan Yakub, supaya Ia pun kembali menolong kamu.
2Ch 30:7  Janganlah seperti leluhurmu dan orang-orang Israel lain yang tidak setia kepada TUHAN Allah mereka. Kamu sudah melihat sendiri bagaimana beratnya Ia menghukum mereka.
2Ch 30:8  Janganlah keras kepala seperti mereka, tetapi taatilah TUHAN. Datanglah ke Yerusalem, ke Rumah TUHAN yang sudah dikhususkan oleh TUHAN Allahmu bagi diri-Nya untuk selama-lamanya. Marilah beribadat kepada-Nya supaya redalah kemarahan-Nya terhadap kamu.
2Ch 30:9  Kalau kamu kembali mengabdi kepada TUHAN, maka orang-orang yang mengangkut sanak saudaramu sebagai tawanan, akan berbelaskasihan dan mengizinkan mereka kembali. TUHAN Allahmu adalah Allah yang baik dan berbelaskasihan; Ia akan menerima kamu kalau kamu kembali kepada-Nya."
2Ch 30:10  Ketika utusan-utusan itu pergi dari kota ke kota di wilayah suku Efraim dan Manasye sampai ke utara sejauh wilayah suku Zebulon, mereka ditertawakan dan dihina.
2Ch 30:11  Tapi, ada juga orang-orang dari suku Asyer, Manasye, dan Zebulon yang mau datang ke Yerusalem.
2Ch 30:12  Di Yehuda pun Allah bekerja dan menyatukan hati rakyat untuk sungguh-sungguh taat kepada kemauan Allah dengan menuruti perintah raja dan pegawai-pegawainya.
2Ch 30:13  Pada bulan dua sejumlah besar rakyat berkumpul di Yerusalem untuk merayakan Hari Raya Roti Tidak Beragi.
2Ch 30:14  Mezbah-mezbah di Yerusalem yang dipakai untuk mempersembahkan kurban dan membakar dupa kepada dewa-dewa, semuanya diambil dan dibuang ke dalam Lembah Kidron.
2Ch 30:15  Pada tanggal 14 bulan itu mereka menyembelih domba-domba untuk dipersembahkan sebagai kurban Paskah. Para imam dan orang-orang Lewi yang masih najis, menjadi sangat malu. Mereka menyucikan diri sehingga dapat mempersembahkan kurban bakaran di Rumah TUHAN.
2Ch 30:16  Mereka masing-masing mengambil tempat di Rumah TUHAN menurut ketentuan-ketentuan yang tercantum dalam buku Hukum-hukum Musa. Darah dari kurban-kurban itu diserahkan oleh orang Lewi kepada para imam, lalu mereka menuangkannya pada mezbah.
2Ch 30:17  Banyak dari antara orang yang hadir di situ belum menyucikan diri, jadi mereka tidak boleh menyembelih domba untuk kurban Paskah. Oleh karena itu orang-orang Lewilah yang melakukan hal itu untuk mereka dan mempersembahkannya kepada TUHAN.
2Ch 30:18  Selain itu, di antara orang-orang Efraim, Manasye, Isakhar, dan Zebulon, banyak yang datang tanpa lebih dahulu menyucikan diri. Jadi, mereka merayakan Paskah itu dengan tidak memenuhi syarat. Karena itu Raja Hizkia berdoa untuk mereka, katanya,
2Ch 30:19  "TUHAN, Allah, pujaan leluhur kami, semoga dari kebaikan hati-Mu Engkau mengampuni mereka yang sedang beribadat kepada-Mu dengan sepenuh hati, sekalipun mereka belum menyucikan diri."
2Ch 30:20  TUHAN mendengar doa Hizkia. Ia mengampuni orang-orang itu, dan tidak menghukum mereka.
2Ch 30:21  Tujuh hari lamanya, dengan sangat gembira orang-orang yang berkumpul di Yerusalem itu mengadakan perayaan Roti Tidak Beragi. Setiap hari orang-orang Lewi dan para imam memuji-muji TUHAN dengan sungguh-sungguh.
2Ch 30:22  Orang-orang Lewi itu begitu pandai memimpin ibadat kepada TUHAN, sehingga Hizkia memuji mereka. Setelah 7 hari lamanya mereka menikmati makanan dari perayaan itu serta mempersembahkan kurban untuk mengucap terima kasih kepada TUHAN, Allah leluhur mereka,
2Ch 30:23  mereka semua memutuskan untuk meneruskan perayaan itu 7 hari lagi. Lalu mereka merayakannya lagi dengan gembira.
2Ch 30:24  Raja Hizkia menyumbangkan 1.000 sapi jantan dan 7.000 domba. Pegawai-pegawai Hizkia menyumbang 1.000 sapi jantan dan 10.000 domba. Semua binatang itu diberikan kepada rakyat untuk disembelih dan dimakan. Banyak sekali imam menyucikan diri.
2Ch 30:25  Rakyat Yehuda, para imam, orang-orang Lewi, dan orang-orang yang datang dari utara, serta orang-orang asing yang telah menetap di Israel dan Yehuda, semuanya merasa bahagia.
2Ch 30:26  Kota Yerusalem meriah sekali karena sejak masa Salomo putra Daud, belum pernah terjadi peristiwa seperti itu.
2Ch 30:27  Imam-imam dan orang Lewi memintakan berkat TUHAN bagi rakyat, dan dari tempat tinggal-Nya di surga Allah mendengar doa mereka serta menerima permintaan mereka.
2Ch 31:1  Setelah perayaan selesai, seluruh rakyat pergi ke setiap kota di Yehuda, lalu menghancurkan tugu-tugu berhala, menumbangkan patung-patung yang didirikan untuk Dewi Asyera, dan memusnahkan mezbah-mezbah serta tempat-tempat penyembahan berhala. Mereka melakukan begitu juga di tempat-tempat lain di Yehuda, serta di wilayah Benyamin, Efraim, dan Manasye. Sesudah itu mereka semua pulang ke tempat mereka masing-masing.
2Ch 31:2  Raja Hizkia menetapkan kembali regu-regu para imam dan orang Lewi menurut pembagian tugas mereka. Tugas-tugas itu meliputi hal-hal berikut: mempersembahkan kurban bakaran dan kurban perdamaian, melayani upacara ibadat di Rumah TUHAN, dan memuji serta mengucap terima kasih kepada TUHAN di pelbagai tempat di dalam Rumah TUHAN.
2Ch 31:3  Dari ternaknya sendiri raja menyumbang binatang-binatang yang diperlukan untuk kurban bakaran setiap pagi dan setiap sore, dan untuk kurban yang dipersembahkan pada hari Sabat, pada perayaan Bulan Baru, dan pada perayaan-perayaan lain seperti yang ditentukan dalam hukum-hukum TUHAN.
2Ch 31:4  Selain itu raja menyuruh penduduk Yerusalem membawa persembahan yang ditentukan bagi para imam dan orang Lewi, supaya mereka dapat mencurahkan seluruh perhatian mereka kepada pelayanan di Rumah TUHAN seperti yang tercantum dalam hukum-hukum TUHAN.
2Ch 31:5  Setelah perintah itu diumumkan, orang Israel membawa ke Rumah TUHAN pemberian-pemberian dari hasil bumi mereka yang pertama, yaitu terigu, anggur, minyak zaitun, madu, dan hasil bumi lainnya. Mereka juga membawa sepersepuluh bagian dari milik mereka.
2Ch 31:6  Semua orang yang tinggal di kota-kota Yehuda membawa sepersepuluh bagian dari ternak sapi dan domba mereka, dan juga sejumlah besar pemberian-pemberian lain, lalu mempersembahkan semuanya itu kepada TUHAN Allah mereka.
2Ch 31:7  Pemberian-pemberian itu mulai berdatangan pada bulan tiga, dan terus mengalir sampai bulan tujuh.
2Ch 31:8  Melihat sumbangan yang sebanyak itu, Raja Hizkia dan pegawai-pegawainya memuliakan TUHAN dan memuji rakyat Israel, umat-Nya.
2Ch 31:9  Ketika raja berbicara dengan para imam dan orang Lewi tentang pemberian-pemberian itu,
2Ch 31:10  Imam Agung Azarya keturunan Zadok berkata, "Sejak rakyat mengantarkan persembahan-persembahan mereka ke Rumah TUHAN, kami tidak kekurangan makanan, malah kelebihan, sehingga masih banyak persediaan makanan kami. Semua itu kami terima karena TUHAN telah memberkati umat-Nya."
2Ch 31:11  Atas perintah raja, mereka menyiapkan gudang-gudang di sekitar Rumah TUHAN,
2Ch 31:12  lalu menyimpan semua pemberian dan sepersepuluhan itu di dalam gudang-gudang itu. Seorang Lewi yang bernama Konanya, ditugaskan menjadi pengawas, dan Simei adiknya dijadikan pembantunya.
2Ch 31:13  Di bawah pimpinan mereka ada 10 orang Lewi lain yang ditugaskan, yaitu: Yehiel, Azazya, Nahat, Asael, Yerimot, Yozabad, Eliel, Yismakhya, Mahat, dan Benanya. Semua hal itu ditetapkan atas petunjuk Raja Hizkia dan Imam Agung Azarya.
2Ch 31:14  Kore anak Yimna, seorang Lewi yang menjadi pengawal kepala pada Pintu Gerbang Timur di Rumah TUHAN, diserahi tanggung jawab untuk menerima pemberian-pemberian yang dipersembahkan kepada TUHAN, lalu membagi-bagikannya.
2Ch 31:15  Di kota-kota lain tempat kediaman imam-imam, Kore dibantu oleh orang-orang Lewi lain yaitu: Eden, Minyamin, Yesua, Semaya, Amarya, dan Sekhanya. Bahan-bahan makanan itu dibagi sama rata di antara sesama suku Lewi baik yang tua maupun yang muda menurut tugas mereka.
2Ch 31:16  Semua orang laki-laki yang berumur tiga puluh tahun ke atas, yang mendapat giliran tugas setiap hari di Rumah TUHAN, juga diberi bagiannya masing-masing menurut tugasnya dan menurut regunya.
2Ch 31:17  Para imam didaftarkan menurut kaumnya, dan orang Lewi yang berumur 20 tahun ke atas didaftarkan menurut tugasnya dan regunya.
2Ch 31:18  Mereka semua didaftarkan bersama-sama dengan anak istri, dan anggota-anggota lain dalam rumah tangganya, sebab mereka diharuskan bersiap sedia setiap saat untuk melaksanakan tugas-tugas mereka yang khusus untuk TUHAN.
2Ch 31:19  Di antara imam-imam yang tinggal di kota-kota yang ditentukan untuk keturunan Harun, atau di padang-padang di sekitar kota-kota itu, ditunjuk petugas-petugas untuk membagi-bagikan bahan makanan kepada semua orang laki-laki dalam keluarga imam, dan kepada setiap orang yang terdaftar dalam suku Lewi.
2Ch 31:20  Di seluruh Yehuda, Raja Hizkia menjalankan keadilan, berlaku jujur, dan menyenangkan hati TUHAN Allahnya.
2Ch 31:21  Ia berhasil, karena segala yang dibuatnya untuk Rumah TUHAN atau untuk mentaati hukum-hukum TUHAN, dijalankannya dengan sepenuh hati dan dengan cinta kepada TUHAN Allahnya.
2Ch 32:1  Setelah Raja Hizkia melakukan hal-hal yang menunjukkan kesetiaannya kepada TUHAN, Sanherib raja Asyur menyerang Yehuda. Ia mengepung kota-kota berbenteng dan memerintahkan pasukannya supaya merebutnya.
2Ch 32:2  Ketika Hizkia mengetahui bahwa Sanherib berniat menyerbu Yerusalem,
2Ch 32:3  ia bersama para pegawai dan perwira-perwiranya memutuskan untuk menutup tempat-tempat air di luar kota dengan maksud supaya orang-orang Asyur tidak mendapat air apabila mereka sampai di dekat Yerusalem. Banyak sekali orang yang dikerahkan untuk menutup semua sumber air yang mengalir ke luar kota.
2Ch 32:5  Lalu raja memperkuat pertahanan kota dengan memperbaiki temboknya, mendirikan menara-menara di atasnya dan membangun tembok lain di luar. Selain itu ia memperbaiki benteng yang didirikan di atas tanah timbunan di sebelah timur bagian lama kota Yerusalem. Ia juga menyuruh membuat banyak sekali tombak dan perisai.
2Ch 32:6  Ia mengangkat perwira-perwira untuk mengepalai semua orang laki-laki dalam kota, dan menyuruh mereka berkumpul di lapangan depan pintu gerbang kota. Lalu ia berkata kepada mereka,
2Ch 32:7  "Tabahlah dan yakinlah; janganlah takut kepada raja Asyur atau kepada pasukannya. Kekuatan di pihak kita lebih besar daripada yang di pihak dia.
2Ch 32:8  Ia mempunyai kekuatan manusia, tetapi kita mempunyai TUHAN Allah kita untuk membantu dan bertempur bagi kita." Mendengar kata-kata raja mereka itu, rakyat menjadi berani.
2Ch 32:9  Beberapa waktu kemudian ketika Sanherib bersama pasukannya masih di Lakhis, ia mengirim utusan kepada Hizkia dan rakyat Yehuda di Yerusalem untuk menyampaikan pesan ini,
2Ch 32:10  "Aku Sanherib raja Asyur, ingin bertanya apa sebabnya kamu begitu berani untuk tetap tinggal di Yerusalem yang sedang dikepung itu.
2Ch 32:11  Apakah Hizkia berkata bahwa TUHAN Allahmu akan melepaskan kamu dari kekuasaan kami? Jangan percaya, sebab ia hanya menipu kamu. Ia akan membiarkan kamu mati kelaparan dan kehausan.
2Ch 32:12  Dialah yang menghancurkan tempat-tempat ibadat dan mezbah-mezbah untuk TUHAN, lalu menyuruh orang Yehuda dan Yerusalem beribadat serta membakar dupa di satu mezbah saja.
2Ch 32:13  Apakah kamu tidak tahu apa yang telah dilakukan oleh aku dan leluhurku terhadap bangsa-bangsa lain? Belum pernah dewa-dewa bangsa lain melepaskan bangsanya dari kekuasaan raja Asyur!
2Ch 32:14  Siapa dari mereka pernah melakukan hal itu? Tak mungkin dewamu dapat menyelamatkan kamu!
2Ch 32:15  Jangan biarkan rajamu itu menipu dan membodohi kamu! Jangan percaya kepadanya! Sebab tidak pernah dewa bangsa mana pun sanggup melepaskan bangsanya dari kekuasaan raja Asyur. Pastilah dewamu itu juga tak akan dapat melepaskan kamu!"
2Ch 32:16  Masih banyak kata-kata lain yang diucapkan oleh utusan-utusan Asyur itu untuk menghina TUHAN Allah dan Hizkia, hamba TUHAN itu.
2Ch 32:17  Raja Asyur juga menulis surat yang menentang TUHAN, Allah yang disembah oleh orang Israel. Begini bunyi surat itu, "Dewa-dewa segala bangsa lain tidak dapat melepaskan bangsanya dari kekuasaanku, begitu juga dewa Hizkia itu tidak dapat melepaskan bangsanya dari kekuasaanku."
2Ch 32:18  Utusan-utusan Asyur itu meneriakkan semuanya itu dalam bahasa Ibrani untuk menakut-nakuti dan mematahkan semangat orang-orang Yerusalem yang berada di atas tembok kota, supaya mereka dapat merebut kota itu dengan mudah.
2Ch 32:19  Mereka berbicara tentang Allah yang disembah di Yerusalem sama seperti tentang dewa-dewa bangsa lain, yaitu patung-patung buatan manusia.
2Ch 32:20  Maka berdoalah Raja Hizkia dan Nabi Yesaya anak Amos. Mereka berseru minta pertolongan kepada Allah.
2Ch 32:21  Lalu TUHAN mengirim seorang malaikat yang membunuh prajurit-prajurit dan perwira-perwira Asyur itu. Dengan sangat malu kembalilah raja Asyur ke negerinya. Pada suatu hari ketika ia sedang di dalam rumah dewanya, beberapa dari anak-anaknya sendiri datang membunuh dia dengan pedang.
2Ch 32:22  Demikianlah caranya TUHAN melepaskan Raja Hizkia dan penduduk Yerusalem dari kekuasaan Sanherib raja Asyur, dan dari musuh-musuh mereka yang lain. TUHAN memberikan kepada mereka kehidupan yang aman tanpa gangguan dari negara-negara tetangga mereka.
2Ch 32:23  Banyak orang datang ke Yerusalem membawa persembahan untuk TUHAN dan hadiah-hadiah untuk Hizkia, sehingga sejak itu ia dihormati oleh segala bangsa.
2Ch 32:24  Sekitar waktu itu Raja Hizkia jatuh sakit. Penyakitnya itu begitu parah sehingga ia hampir meninggal. Ia berdoa, lalu TUHAN memberikan tanda kepadanya bahwa ia akan sembuh.
2Ch 32:25  Tetapi karena ia sombong, ia tidak berterima kasih atas kesembuhan yang diberikan TUHAN kepadanya. Oleh karena itu TUHAN marah kepada Yehuda dan Yerusalem.
2Ch 32:26  Tetapi akhirnya Hizkia dan orang-orang Yerusalem merendahkan diri, maka selama Hizkia masih hidup hukuman itu tidak dijalankan.
2Ch 32:27  Raja Hizkia menjadi sangat kaya, dan dihormati semua orang. Ia membuat kamar-kamar untuk menyimpan hartanya yaitu: emas, perak, batu-batu permata, rempah-rempah, perisai dan barang-barang berharga lainnya.
2Ch 32:28  Ia mempunyai gudang-gudang untuk barang-barang hasil buminya seperti: gandum, anggur, dan minyak zaitun; juga kandang-kandang untuk sapi-sapi dan domba-dombanya yang sangat banyak.
2Ch 32:29  Selain itu ia mendirikan juga banyak kota. Kekayaan yang diberikan Allah kepadanya berlimpah-limpah.
2Ch 32:30  Raja Hizkia itulah yang membendung aliran mata air Gihon, dan mengalirkan airnya melalui saluran di bawah tanah ke dalam kota Yerusalem. Segala yang dilakukan Hizkia berhasil.
2Ch 32:31  Dan pada waktu utusan-utusan dari Babel datang untuk menanyakan kepadanya tentang kesembuhannya yang ajaib, TUHAN membiarkan Hizkia bertindak sendiri, supaya dapat menguji hatinya.
2Ch 32:32  Kisah lainnya mengenai Raja Hizkia, dan mengenai cintanya kepada TUHAN dicatat di dalam buku Wahyu dari Allah Kepada Nabi Yesaya Anak Amos, dan di dalam buku Sejarah Raja-raja Yehuda dan Israel.
2Ch 32:33  Hizkia meninggal, lalu dimakamkan pada bagian tanjakan di tanah pekuburan raja-raja. Seluruh rakyat Yehuda dan Yerusalem memberikan penghormatan yang besar kepadanya pada waktu ia meninggal. Manasye putranya menjadi raja menggantikan dia.
2Ch 33:1  Manasye berumur 12 tahun ketika ia menjadi raja Yehuda dan ia memerintah di Yerusalem 55 tahun lamanya.
2Ch 33:2  Ia berdosa kepada TUHAN, karena mengikuti kebiasaan-kebiasaan jahat bangsa-bangsa yang diusir TUHAN dari Kanaan pada waktu orang Israel memasuki negeri itu.
2Ch 33:3  Tempat-tempat penyembahan berhala yang telah dimusnahkan Hizkia, ayahnya, dibangunnya kembali. Ia membangun mezbah-mezbah untuk beribadat kepada Baal, dan ia membuat patung-patung Dewi Asyera, serta menyembah bintang-bintang.
2Ch 33:4  TUHAN telah berkata bahwa untuk selama-lamanya Yerusalem adalah tempat untuk beribadat kepada-Nya, tetapi di sana di Rumah TUHAN itu, Manasye telah mendirikan mezbah-mezbah untuk dewa-dewa.
2Ch 33:5  Di kedua halaman Rumah TUHAN itu ia mendirikan mezbah-mezbah untuk penyembahan kepada bintang-bintang.
2Ch 33:6  Anak-anaknya dipersembahkannya sebagai kurban bakaran di Lembah Hinom. Ia juga melakukan praktek-praktek pedukunan, penujuman, ilmu gaib, dan meminta petunjuk kepada roh-roh. Ia sangat berdosa kepada TUHAN sehingga membangkitkan kemarahan TUHAN.
2Ch 33:7  Patung berhala ditaruhnya di dalam Rumah TUHAN, padahal mengenai tempat itu Allah telah berkata kepada Daud dan Salomo putranya, "Rumah-Ku di Yerusalem ini, yang telah Kupilih dari antara kedua belas wilayah suku Israel, adalah tempat yang Kutentukan sebagai tempat ibadat kepada-Ku untuk selama-lamanya.
2Ch 33:8  Kalau umat Israel mentaati semua perintah-Ku dan menuruti hukum-hukum yang diberikan Musa hamba-Ku kepada mereka, maka Aku tidak akan membiarkan mereka diusir dari negeri yang telah Kuberikan kepada leluhur mereka."
2Ch 33:9  Manasye menyebabkan rakyat Yehuda melakukan dosa-dosa yang lebih jahat dari dosa yang dilakukan oleh bangsa-bangsa yang diusir TUHAN dari Kanaan pada waktu umat TUHAN memasuki negeri itu.
2Ch 33:10  TUHAN menegur Manasye dan rakyatnya, tetapi mereka tidak mau mendengar.
2Ch 33:11  Karena itu TUHAN mendatangkan para panglima tentara Asyur untuk menyerang Yehuda. Manasye ditangkap dengan kaitan, lalu dibelenggu dan diangkut ke Babel.
2Ch 33:12  Dalam penderitaannya itu ia merendahkan diri terhadap TUHAN Allahnya, dan berdoa mohon pertolongan-Nya.
2Ch 33:13  Allah mengabulkan doa Manasye dan mengembalikan dia ke Yerusalem untuk memerintah lagi. Karena kejadian itu Manasye menjadi yakin bahwa TUHAN adalah Allah.
2Ch 33:14  Kemudian Manasye mempertinggi tembok luar di bagian timur Kota Daud. Ia mulai dari satu tempat di lembah dekat mata air Gihon lalu terus ke utara sampai ke Pintu Gerbang Ikan dan bagian kota yang disebut Ofel. Di setiap kota berbenteng di Yehuda, ia menempatkan juga seorang panglima.
2Ch 33:15  Ia menyingkirkan patung-patung dewa bangsa lain serta berhala-berhala yang telah diletakkannya di Rumah TUHAN. Ia membongkar mezbah-mezbah berhala yang berada di bukit Rumah TUHAN, dan di tempat-tempat lain di Yerusalem. Semuanya itu dibawanya ke luar kota lalu dibuang.
2Ch 33:16  Kemudian ia memperbaiki mezbah tempat ibadat kepada TUHAN, lalu mempersembahkan di situ kurban perdamaian dan kurban syukur. Semua orang Yehuda disuruhnya beribadat kepada TUHAN Allah Israel.
2Ch 33:17  Rakyat masih mempersembahkan kurban di tempat-tempat penyembahan lain, tetapi persembahan itu adalah untuk TUHAN.
2Ch 33:18  Kisah lainnya mengenai Manasye, dan mengenai doanya kepada Allah, serta pesan-pesan Allah yang disampaikan kepadanya oleh nabi-nabi atas nama TUHAN, Allah Israel, semuanya sudah dicatat di dalam buku Sejarah Raja-raja Israel.
2Ch 33:19  Dalam buku Sejarah Nabi-nabi sudah dicatat juga doa Manasye dan jawaban Allah atas doa itu, serta kisah tentang dosa-dosanya sebelum ia bertobat, tempat-tempat penyembahan berhala dan patung-patung Dewi Asyera yang dibuatnya, dan berhala-berhala yang disembahnya.
2Ch 33:20  Manasye meninggal dan dimakamkan di istana. Amon putranya menjadi raja menggantikan dia.
2Ch 33:21  Amon berumur 22 tahun ketika ia menjadi raja Yehuda, dan ia memerintah di Yerusalem dua tahun lamanya.
2Ch 33:22  Seperti ayahnya ia pun berdosa kepada TUHAN. Ia menyembah berhala-berhala yang disembah ayahnya dan mempersembahkan kurban kepada berhala-berhala itu.
2Ch 33:23  Tetapi, berbeda dengan ayahnya, Amon tidak merendahkan diri dan tidak kembali kepada TUHAN; ia malahan lebih berdosa dari ayahnya.
2Ch 33:24  Pegawai-pegawai Raja Amon berkomplot melawan dia dan membunuhnya di istana.
2Ch 33:25  Tetapi rakyat Yehuda membunuh para pembunuh Raja Amon itu, lalu mengangkat Yosia anaknya menjadi raja.
2Ch 34:1  Yosia berumur 8 tahun ketika ia menjadi raja Yehuda, dan ia memerintah di Yerusalem 31 tahun lamanya.
2Ch 34:2  Ia melakukan yang menyenangkan hati TUHAN; ia mengikuti teladan Raja Daud leluhurnya, dan mentaati seluruh hukum Allah dengan sepenuhnya.
2Ch 34:3  Setelah 8 tahun menjadi raja, pada usia sangat muda, Yosia mulai beribadat kepada Allah yang disembah oleh Raja Daud leluhurnya. Empat tahun kemudian ia mulai menghancurkan tempat-tempat penyembahan berhala, patung-patung Dewi Asyera, dan berhala-berhala lainnya.
2Ch 34:4  Patung-patung dan berhala-berhala itu dihancurluluhkan oleh anak buah Yosia lalu serbuknya mereka hamburkan ke atas kuburan orang-orang yang mempersembahkan kurban kepada berhala-berhala itu. Mezbah-mezbah Baal dan mezbah-mezbah dupa yang terdapat di situ mereka robohkan.
2Ch 34:5  Kemudian Yosia membakar di atas mezbah itu tulang-tulang para imam yang melayani berhala-berhala itu. Dengan demikian ia membuat Yehuda dan Yerusalem layak lagi beribadat kepada TUHAN.
2Ch 34:6  Hal yang sama dilakukannya juga di kota-kota dan di daerah-daerah yang telah hancur di wilayah Manasye, Efraim, dan Simeon, sampai ke Naftali di utara.
2Ch 34:7  Mezbah-mezbah dan patung-patung Asyera di seluruh Kerajaan Utara dirobohkan oleh Yosia, lalu patung-patungnya dan semua mezbah dupa dihancurkannya. Kemudian pulanglah Yosia ke Yerusalem.
2Ch 34:8  Pada tahun kedelapan belas pemerintahan Yosia, setelah ia menyapu bersih penyembahan berhala dari Rumah TUHAN dan seluruh negeri, ia mengirim 3 orang untuk memperbaiki Rumah TUHAN Allah. Ketiga orang itu ialah Safan anak Azalya, Maaseya gubernur Yerusalem, dan Yoah anak Yoahas, seorang pegawai tinggi.
2Ch 34:9  Orang-orang Lewi yang menjaga Pintu Rumah TUHAN, menyerahkan kepada Imam Agung Hilkia, uang yang telah mereka kumpulkan. Uang itu dari orang Efraim dan Manasye, dan dari rakyat lainnya di Kerajaan Utara, serta dari rakyat Yehuda, Benyamin dan Yerusalem.
2Ch 34:10  Uang itu kemudian diserahkan oleh Imam Agung Hilkia kepada ketiga orang yang ditugaskan untuk mengawasi perbaikan Rumah TUHAN itu. Selanjutnya mereka memberikan uang itu
2Ch 34:11  kepada para tukang kayu dan tukang bangunan yang harus membeli batu dan kayu untuk memperbaiki gedung-gedung yang telah dibiarkan rusak oleh raja-raja Yehuda.
2Ch 34:12  Orang-orang yang mengerjakan perbaikan itu sungguh-sungguh jujur. Mereka dikepalai oleh 4 orang Lewi: Yahat dan Obaja dari kaum Merari, serta Zakharia dan Mesulam dari kaum Kehat. Orang-orang Lewi, yang semuanya pandai main musik,
2Ch 34:13  diberi tugas mengawasi pengangkutan bahan-bahan, mengepalai buruh pada berbagai macam pekerjaan, mengerjakan administrasi, dan menjaga pintu.
2Ch 34:14  Pada waktu uang sumbangan itu dikeluarkan dari tempat penyimpanannya, Hilkia menemukan buku Hukum TUHAN, yaitu hukum-hukum yang diberikan Allah kepada Musa.
2Ch 34:15  Berkatalah Hilkia kepada Safan, "Saya menemukan buku Hukum TUHAN di Rumah TUHAN," lalu diberikannya buku itu kepada Safan.
2Ch 34:16  Safan menerima buku itu, lalu pergi kepada raja dan berkata, "Semua yang Baginda perintahkan sudah kami lakukan.
2Ch 34:17  Kami sudah mengambil uang yang disimpan di Rumah TUHAN, dan menyerahkannya kepada para pekerja dan pengawas-pengawasnya."
2Ch 34:18  Kemudian ia tambahkan, "Hilkia memberi buku ini kepada saya." Lalu Safan membacakan buku itu kepada raja.
2Ch 34:19  Mendengar isi buku itu, raja merobek-robek pakaiannya karena sedih.
2Ch 34:20  Lalu raja memberi perintah ini kepada Hilkia, Ahikam anak Safan, Abdon anak Mikha, Safan sekretaris negara, dan Asaya ajudan raja,
2Ch 34:21  "Pergilah bertanya kepada TUHAN untuk aku dan untuk orang-orang yang masih ada di Israel dan Yehuda mengenai isi buku ini. TUHAN marah kepada kita karena leluhur kita tidak menjalankan perintah-perintah yang tertulis di dalamnya."
2Ch 34:22  Maka pergilah Hilkia, Ahikam, Abdon, Safan dan Asaya meminta petunjuk kepada seorang wanita bernama Hulda. Ia seorang nabi yang tinggal di perkampungan baru di Yerusalem. Suaminya bernama Salum anak Tikwa, cucu Harhas. Ia pengurus pakaian ibadat di Rumah TUHAN. Setelah Hulda mendengar keterangan mereka,
2Ch 34:23  ia menyuruh mereka kembali kepada raja dan menyampaikan pesan ini, "TUHAN, Allah yang disembah oleh orang Israel berkata,
2Ch 34:24  'Aku akan menghukum Yerusalem dan seluruh penduduknya dengan kutukan-kutukan yang tertulis di dalam buku itu yang dibacakan kepada raja.
2Ch 34:25  Mereka telah meninggalkan Aku dan mempersembahkan kurban kepada ilah-ilah lain. Semua yang mereka lakukan membangkitkan kemarahan-Ku. Aku marah kepada Yerusalem, dan kemarahan-Ku tidak bisa diredakan.
2Ch 34:26  Tetapi mengenai Raja Yosia, inilah pesan-Ku, TUHAN Allah Israel: Setelah engkau mendengar apa yang tertulis dalam buku itu,
2Ch 34:27  engkau menyesal dan merendahkan diri di hadapan-Ku. Aku telah mengancam untuk menghukum Yerusalem dan penduduknya, tapi ketika engkau mendengar tentang ancaman-Ku itu engkau menangis dan merobek pakaianmu tanda sedih. Aku telah mendengar doamu,
2Ch 34:28  karena itu hukuman atas Yerusalem tidak akan Kujatuhkan selama engkau masih hidup. Engkau akan Kuizinkan meninggal dengan damai.'" Maka kembalilah utusan-utusan itu kepada raja dan menyampaikan pesan itu.
2Ch 34:29  Raja Yosia memanggil semua pemimpin Yehuda dan Yerusalem,
2Ch 34:30  lalu pergi dengan mereka ke Rumah TUHAN. Bersama mereka ikut juga para imam, orang Lewi dan seluruh rakyat, baik yang miskin maupun yang kaya. Di depan mereka semua, di dekat pilar yang khusus untuk raja, raja berdiri dan membacakan dengan suara keras seluruh buku perjanjian yang telah ditemukan di dalam Rumah TUHAN. Kemudian raja membuat perjanjian dengan TUHAN untuk taat kepada-Nya dan menjalankan dengan sepenuh hati dan segenap jiwa semua perintah dan hukum-hukum-Nya, serta memenuhi syarat perjanjian antara TUHAN dengan umat-Nya yang tercantum dalam buku itu.
2Ch 34:32  Semua orang Benyamin dan semua orang lain yang ada di Yerusalem disuruhnya berjanji untuk mentaati perjanjian itu. Maka orang-orang Yerusalem mentaati peraturan-peraturan dalam perjanjian itu, yaitu perjanjian antara mereka dengan Allah leluhur mereka.
2Ch 34:33  Berhala-berhala yang memuakkan, yang terdapat di tempat-tempat yang termasuk wilayah orang Israel, semuanya dihancurkan oleh Raja Yosia. Lalu ia mengharuskan rakyatnya beribadat kepada TUHAN, Allah leluhur mereka. Maka selama Yosia hidup, rakyat tetap setia kepada TUHAN.
2Ch 35:1  Kemudian Yosia merayakan Paskah di Yerusalem untuk menghormati TUHAN. Domba untuk perayaan itu disembelih pada tanggal 14 bulan satu.
2Ch 35:2  Yosia membagikan kepada para imam tugas-tugas yang harus mereka kerjakan di Rumah TUHAN, dan ia menganjurkan supaya mereka mengerjakannya dengan baik.
2Ch 35:3  Orang-orang Lewi yang telah dikhususkan untuk TUHAN, dan yang menjadi guru-guru Israel mendapat perintah ini dari Yosia, "Taruhlah Peti Perjanjian TUHAN yang suci itu di dalam Rumah TUHAN yang didirikan oleh Raja Salomo putra Daud. Tidak usah kalian memindah-mindahkannya lagi. Layanilah TUHAN Allahmu dan bangsa Israel umat-Nya.
2Ch 35:4  Bersiaplah di Rumah TUHAN menurut regu-regumu, sesuai dengan pembagian tugas yang ditetapkan kepadamu oleh Raja Daud dan Raja Salomo putranya.
2Ch 35:5  Aturlah regu-regumu itu sedemikian rupa sehingga selalu ada dari kalian yang dapat menolong setiap kelompok orang Israel di Rumah TUHAN.
2Ch 35:6  Kalian harus memotong domba untuk perayaan Paskah. Oleh karena itu siapkanlah dirimu agar patut menyembah TUHAN, dan siapkanlah juga kurban-kurban yang harus dipersembahkan kepada TUHAN, supaya dengan demikian orang-orang Israel yang lainnya dapat menjalankan apa yang diperintahkan oleh TUHAN melalui Musa."
2Ch 35:7  Supaya orang-orang yang hadir di perayaan Paskah itu dapat mempersembahkan kurban Paskah, Raja Yosia menyumbang dari ternaknya sendiri 30.000 ekor domba, anak domba, dan kambing muda, serta 3.000 ekor sapi jantan.
2Ch 35:8  Para pegawai Yosia pun memberi sumbangan untuk imam-imam, dan orang Lewi dari seluruh rakyat. Pegawai-pegawai yang diserahi tanggung jawab untuk Rumah TUHAN, yaitu Imam Agung Hilkia, dan Zakharia, serta Yehiel, ketiga-tiganya memberikan kepada para imam 2.600 ekor anak domba dan kambing muda serta 300 ekor sapi jantan untuk persembahan dalam perayaan itu.
2Ch 35:9  Para pemimpin orang Lewi, yaitu: Konanya, Semaya dan Netaneel saudaranya, Hasabya, Yeiel, dan Yozabad, memberikan kepada orang Lewi 5.000 ekor anak domba dan kambing muda serta 500 ekor sapi jantan untuk persembahan dalam perayaan itu.
2Ch 35:10  Setelah semuanya siap untuk perayaan Paskah, imam-imam dan regu-regu orang Lewi mengambil tempatnya masing-masing seperti yang diperintahkan oleh raja.
2Ch 35:11  Lalu mereka menyembelih binatang untuk perayaan Paskah; orang Lewi mengulitinya dan para imam memercikkan darah binatang itu ke atas mezbah.
2Ch 35:12  Setiap kelompok keluarga yang hadir di situ mendapat seekor dari binatang-binatang yang akan dipersembahkan sebagai kurban bakaran. Dengan demikian mereka dapat mempersembahkan kurban menurut peraturan yang tercantum dalam buku Hukum Musa.
2Ch 35:13  Lalu orang-orang Lewi memanggang kurban Paskah itu di atas api, sesuai dengan peraturan, dan merebus kurban-kurban khusus di dalam kuali, periuk dan belanga, kemudian dagingnya cepat-cepat dibagikan kepada orang-orang.
2Ch 35:14  Sesudah itu barulah orang-orang Lewi menyiapkan bagian untuk mereka sendiri dan untuk para imam keturunan Harun. Sebab, sampai malam para imam itu bekerja terus mempersembahkan binatang-binatang sebagai kurban bakaran, dan membakar lemak dari kurban-kurban itu.
2Ch 35:15  Para ahli musik dari kaum Asaf dalam suku Lewi berada pada tempat tugas yang sudah ditentukan oleh Raja Daud bagi mereka. Inilah nama-nama mereka: Asaf, Heman, dan Yedutun, seorang nabi yang bekerja untuk raja. Para pengawal pintu-pintu gerbang tidak perlu meninggalkan pos mereka, karena mereka dapat merayakan Paskah di tempat tugas mereka dengan bantuan orang-orang Lewi lainnya.
2Ch 35:16  Pada hari itu mereka melakukan segala sesuatu yang diperintahkan oleh Raja Yosia untuk perayaan Paskah, termasuk persembahan-persembahan pada mezbah.
2Ch 35:17  Tujuh hari lamanya seluruh rakyat Israel yang hadir di situ merayakan Paskah dan Pesta Roti Tidak Beragi.
2Ch 35:18  Paskah seperti itu belum pernah dirayakan lagi sejak masa Nabi Samuel. Tidak seorang pun dari raja-raja sebelumnya pernah merayakan Paskah seperti yang dirayakan
2Ch 35:19  dalam tahun kedelapan belas pemerintahan Yosia itu, oleh raja itu sendiri bersama imam-imam, orang Lewi, orang Yehuda dan Israel, serta penduduk Yerusalem.
2Ch 35:20  Setelah Raja Yosia melakukan semuanya itu untuk Rumah TUHAN; Nekho raja Mesir memimpin pasukannya untuk berperang di Karkemis dekat Sungai Efrat. Yosia berusaha mencegah dia,
2Ch 35:21  tetapi Nekho mengirim berita ini kepada Yosia, "Perang ini bukan menyangkut engkau, raja Yehuda! Aku datang bukan untuk melawan engkau, melainkan untuk melawan musuh-musuhku, dan Allah menyuruh aku bertindak cepat. Allah ada di pihakku, karena itu janganlah menghalangi Dia, nanti engkau dihancurkan-Nya."
2Ch 35:22  Tetapi Yosia berkeras untuk melawan, dan tidak mau menuruti apa yang dikatakan Allah melalui Raja Nekho. Yosia menyamar lalu memasuki pertempuran di dataran rendah Megido.
2Ch 35:23  Dalam pertempuran itu Raja Yosia terkena panah. Lalu ia memerintahkan hamba-hambanya supaya membawa dia keluar dari pertempuran karena ia luka berat.
2Ch 35:24  Mereka mengangkat dia dari keretanya dan memindahkannya ke kereta lain yang ada di situ, kemudian membawanya ke Yerusalem. Di sana ia meninggal dan dikuburkan di makam raja-raja. Seluruh rakyat Yehuda dan Yerusalem berkabung atas kematiannya.
2Ch 35:25  Nabi Yeremia menggubah sebuah nyanyian ratapan untuk Raja Yosia. Sampai sekarang nyanyian itu dipakai di Israel oleh para penyanyi, baik pria maupun wanita, dalam peringatan perkabungan untuk Yosia. Nyanyian itu telah dimasukkan ke dalam kumpulan nyanyian-nyanyian ratapan.
2Ch 35:26  Kisah lain mengenai Yosia, mengenai cintanya kepada TUHAN, dan ketaatannya kepada hukum-hukum TUHAN,
2Ch 35:27  singkatnya, mengenai seluruh riwayat hidupnya dari mula sampai akhir, sudah dicatat dalam buku Sejarah Raja-raja Israel dan Yehuda.
2Ch 36:1  Rakyat Yehuda memilih dan mengangkat Yoahas menjadi raja di Yerusalem menggantikan Yosia, ayahnya.
2Ch 36:2  Yoahas berumur 23 tahun ketika ia menjadi raja Yehuda, dan ia memerintah di Yerusalem 3 bulan lamanya.
2Ch 36:3  Nekho raja Mesir menawan dan mengangkut Yoahas ke Mesir, lalu ia menyuruh Yehuda membayar upeti sebesar 3.400 kilogram perak dan 34 kilogram emas. Kemudian ia mengangkat Elyakim saudara Yoahas menjadi raja, dan mengubah nama Elyakim menjadi Yoyakim. Kemudian Yoahas diangkutnya ke Mesir.
2Ch 36:5  Yoyakim berumur 25 tahun ketika ia menjadi raja Yehuda, dan ia memerintah di Yerusalem 11 tahun lamanya. Ia berdosa kepada TUHAN Allahnya.
2Ch 36:6  Ketika Nebukadnezar raja Babel merebut Yehuda, ia menangkap Yoyakim, dan membawanya terbelenggu ke Babel.
2Ch 36:7  Sebagian barang-barang berharga di Rumah TUHAN diangkut oleh Nebukadnezar dan ditaruh di dalam istananya di Babel.
2Ch 36:8  Kisah lain mengenai Yoyakim, termasuk perbuatan-perbuatannya yang keji dan kejahatan-kejahatannya, sudah dicatat dalam buku Sejarah Raja-raja Israel dan Yehuda. Yoyakhin putranya menjadi raja menggantikan dia.
2Ch 36:9  Yoyakhin berumur 18 tahun ketika ia menjadi raja Yehuda, dan ia memerintah di Yerusalem 3 bulan 10 hari lamanya. Ia juga berdosa kepada TUHAN.
2Ch 36:10  Pada pergantian tahun, Raja Nebukadnezar mengangkut Yoyakhin ke Babel sebagai tawanan, dan barang-barang berharga di Rumah TUHAN dibawanya juga. Kemudian Nebukadnezar mengangkat Zedekia, paman Yoyakhin, menjadi raja atas Yehuda dan Yerusalem.
2Ch 36:11  Zedekia berumur 21 tahun ketika ia menjadi raja Yehuda. Ia memerintah di Yerusalem 11 tahun lamanya.
2Ch 36:12  Ia juga berdosa kepada TUHAN Allahnya. Ia tidak mau merendahkan diri dan tidak menuruti Nabi Yeremia yang menyampaikan pesan dari TUHAN kepadanya.
2Ch 36:13  Raja Nebukadnezar telah memaksa Zedekia untuk bersumpah demi Allah bahwa ia akan setia kepada Nebukadnezar. Tetapi kemudian Zedekia memberontak terhadap Nebukadnezar. Zedekia berkeras kepala terhadap TUHAN, Allah Israel. Ia tidak mau meninggalkan dosa-dosanya dan tidak mau kembali kepada TUHAN.
2Ch 36:14  Selain itu, Rumah TUHAN yang telah dikhususkan oleh TUHAN sendiri bagi diri-Nya, dinajiskan oleh pemimpin-pemimpin Yehuda, imam-imam, dan rakyat, karena mereka turut menyembah berhala dan mengikuti cara hidup yang berdosa dari bangsa-bangsa di sekitar mereka.
2Ch 36:15  TUHAN, Allah leluhur mereka, telah berulang-ulang mengirim utusan-utusan-Nya untuk memperingatkan umat-Nya, karena Ia sayang kepada mereka dan kepada rumah-Nya.
2Ch 36:16  Tetapi mereka mempermainkan utusan-utusan Allah itu, dan menertawakan nabi-nabi-Nya serta tidak menghiraukan kata-kata-Nya. Akhirnya TUHAN begitu marah kepada mereka sehingga tidak ada lagi jalan keluar bagi mereka untuk luput dari hukuman-Nya.
2Ch 36:17  TUHAN menggerakkan hati raja Babel supaya menyerang mereka. Raja itu membunuh orang Yehuda yang muda-muda, termasuk mereka yang berada di dalam Rumah TUHAN. Ia tidak mengasihani seorang pun, tua atau muda, laki-laki atau perempuan, sakit atau sehat. Mereka semua diserahkan Allah kepada raja Babel.
2Ch 36:18  Perkakas-perkakas Rumah TUHAN, dan barang-barangnya yang berharga, juga harta benda raja dan para pegawainya, semuanya dirampas oleh raja Babel itu lalu dibawa ke Babel.
2Ch 36:19  Ia meruntuhkan tembok kota Yerusalem, dan membakar kota itu, termasuk Rumah TUHAN dan semua istana bersama barang-barang berharga yang masih ada di dalam istana-istana itu.
2Ch 36:20  Rakyat yang luput diangkutnya ke Babel. Di sana mereka menjadi hambanya dan hamba keturunannya sampai pada masa kerajaan Persia berkuasa.
2Ch 36:21  Demikianlah terjadi apa yang dikatakan TUHAN melalui Yeremia bahwa negeri itu akan tandus 70 tahun lamanya sebagai ganti tahun-tahun Sabat yang tidak diindahkan.
2Ch 36:22  Pada tahun pertama pemerintahan Kores, raja Persia, TUHAN melaksanakan apa yang telah diucapkan-Nya melalui Nabi Yeremia. TUHAN menggerakkan hati Kores untuk mengeluarkan sebuah perintah yang dibacakan di seluruh kerajaannya. Bunyinya demikian:
2Ch 36:23  "Inilah perintah Kores, raja Persia! TUHAN, Allah penguasa di surga telah menjadikan aku raja atas seluruh dunia, dan menugaskan aku untuk membangun rumah bagi-Nya di Yerusalem di daerah Yehuda. Jadi, kamu semua yang menjadi umat Allah harus kembali ke sana! Semoga TUHAN Allahmu melindungi kamu."


\end{document}