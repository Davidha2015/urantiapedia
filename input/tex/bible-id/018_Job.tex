\begin{document}

\title{Ayub}


\chapter{1}

\par 1 Di tanah Us tinggallah seorang laki-laki yang bernama Ayub. Ia menyembah Allah dan setia kepada-Nya. Ia orang yang baik budi dan tidak berbuat kejahatan sedikit pun.
\par 2 Ia mempunyai tujuh orang anak laki-laki dan tiga anak perempuan.
\par 3 Di samping itu ia mempunyai banyak budak-budak, 7.000 ekor domba, 3.000 ekor unta, 1.000 ekor sapi, dan 500 ekor keledai. Pendek kata, dia adalah orang yang paling kaya di antara penduduk daerah Timur.
\par 4 Ketujuh anak laki-laki Ayub mempunyai kebiasaan untuk mengadakan pesta di rumah masing-masing secara bergilir. Pada pesta itu ketiga anak perempuan Ayub juga diundang, lalu mereka semua makan dan minum bersama-sama.
\par 5 Sehabis setiap pesta, Ayub selalu bangun pagi-pagi dan mempersembahkan kurban untuk tiap-tiap anaknya supaya mereka diampuni TUHAN. Sebab Ayub berpikir, boleh jadi anak-anaknya itu sudah berdosa dan menghina Allah tanpa sengaja.
\par 6 Pada suatu hari makhluk-makhluk surgawi menghadap TUHAN, dan si Penggoda ada di antara mereka juga.
\par 7 TUHAN bertanya kepadanya, "Dari mana engkau?" Jawab Si Penggoda, "Hamba baru saja mengembara di sana sini dan menjelajahi seluruh bumi."
\par 8 Lalu TUHAN bertanya, "Apakah telah kauperhatikan hamba-Ku Ayub? Di seluruh bumi tak ada orang yang begitu setia dan baik hati seperti dia. Ia menyembah Aku dan sama sekali tidak berbuat kejahatan."
\par 9 Tetapi Si Penggoda menjawab, "Tentu saja Ayub menyembah Engkau sebab ia menerima imbalan.
\par 10 Dia, keluarganya dan segala kekayaannya selalu Kaulindungi. Pekerjaannya Kauberkati dan Kauberi dia banyak ternak, cukup untuk memenuhi seluruh negeri.
\par 11 Tetapi seandainya segala kekayaannya itu Kauambil, pasti dia akan langsung mengutuki Engkau!"
\par 12 Maka kata TUHAN kepada Si Penggoda, "Baiklah, lakukanlah apa saja dengan seluruh kekayaan Ayub, asal jangan kausakiti dia!" Lalu pergilah Si Penggoda dari hadapan TUHAN.
\par 13 Beberapa waktu kemudian anak-anak Ayub sedang mengadakan pesta di rumah abang mereka yang tertua.
\par 14 Tiba-tiba seorang pesuruh datang berlari-lari ke rumah Ayub dan melaporkan, "Tuan, orang Syeba telah datang menyerang kami ketika sapi-sapi sedang membajak ladang dan keledai-keledai sedang merumput di dekatnya. Mereka telah merampas binatang-binatang itu, dan membunuh hamba-hamba Tuan yang ada di situ. Hanya hamba saja yang luput sehingga dapat melapor kepada Tuan."
\par 16 Ketika ia masih berbicara, datanglah hamba kedua yang berkata, "Domba-domba Tuan dan para gembala telah disambar petir hingga tewas semua. Hanya hamba sendiri yang luput sehingga dapat melapor kepada Tuan."
\par 17 Ia belum selesai berbicara, ketika hamba yang ketiga datang memberitakan, "Tiga pasukan perampok Kasdim telah merampas unta-unta Tuan dan membunuh hamba-hamba Tuan. Hanya hamba saja yang luput sehingga dapat melapor kepada Tuan."
\par 18 Ketika ia masih berbicara, datanglah hamba lain yang membawa kabar, "Anak-anak Tuan sedang mengadakan pesta di rumah anak Tuan yang sulung.
\par 19 Tiba-tiba angin ribut bertiup dari arah padang pasir dan melanda rumah itu hingga roboh dan menewaskan semua anak Tuan. Hanya hambalah yang luput sehingga dapat melapor kepada Tuan."
\par 20 Lalu berdirilah Ayub dan merobek pakaiannya tanda berdukacita. Ia mencukur kepalanya, lalu sujud
\par 21 dan berkata, "Aku dilahirkan tanpa apa-apa, dan aku akan mati tanpa apa-apa juga. TUHAN telah memberikan dan TUHAN pula telah mengambil. Terpujilah nama-Nya!"
\par 22 Jadi, meskipun Ayub mengalami segala musibah itu, ia tidak berbuat dosa dan tidak mempersalahkan Allah.

\chapter{2}

\par 1 Makhluk-makhluk surgawi kembali menghadap TUHAN, dan Si Penggoda ada pula di antara mereka.
\par 2 Maka TUHAN bertanya kepadanya, "Dari mana engkau?" Si Penggoda menjawab, "Dari perjalanan mengembara ke sana sini dan menjelajahi bumi."
\par 3 TUHAN bertanya, "Apakah sudah kauperhatikan hamba-Ku Ayub? Di seluruh bumi tak ada orang yang begitu setia dan baik hati seperti dia. Ia menyembah Aku dan sama sekali tidak melakukan kejahatan. Dia masih tetap setia kepada-Ku, walaupun engkau telah membujuk Aku mencelakakan dia tanpa alasan."
\par 4 Tetapi Si Penggoda menjawab, "Nyawa dan kesehatan lebih berharga daripada harta. Manusia rela mengurbankan segala miliknya asal ia dapat tetap hidup.
\par 5 Seandainya tubuhnya Kausakiti, pasti ia akan langsung mengutuki Engkau!"
\par 6 Maka berkatalah TUHAN kepada Si Penggoda, "Baiklah, lakukanlah apa saja dengan dia, asal jangan kaubunuh dia."
\par 7 Maka Si Penggoda pergi dari hadapan TUHAN, dan menimbulkan borok pada seluruh tubuh Ayub dari telapak kaki sampai ujung kepalanya.
\par 8 Lalu Ayub duduk di dekat timbunan sampah, dan mengambil beling untuk menggaruk-garuk badannya.
\par 9 Istrinya berkata kepadanya, "Mana bisa engkau masih tetap setia kepada Allah? Ayo, kutukilah Dia, lalu matilah!"
\par 10 Jawab Ayub, "Kaubicara seperti orang dungu! Masakan kita hanya mau menerima apa yang baik dari Allah, sedangkan yang tidak baik kita tolak?" Jadi, meskipun Ayub mengalami segala musibah itu, ia tidak mengucapkan kata-kata yang melawan Allah.
\par 11 Musibah yang menimpa Ayub telah didengar oleh tiga orang temannya, yaitu Elifas dari kota Teman, Bildad orang Suah dan Zofar orang Naamah. Mereka bersepakat hendak menjenguk dan menghiburnya.
\par 12 Tetapi ketika dari kejauhan mereka memandangnya, mereka hampir tak dapat mengenalinya lagi. Mereka menangis dengan nyaring dan merobek pakaian mereka serta menaburkan debu ke udara dan di kepala.
\par 13 Tujuh hari tujuh malam mereka duduk di tanah, di samping Ayub, tanpa mengucapkan sepatah kata pun, karena mereka melihat betapa berat penderitaannya.

\chapter{3}

\par 1 Kemudian Ayub mulai berbicara dan mengutuki hari kelahirannya, katanya,
\par 2 "Ya Allah, kutukilah hari kelahiranku, dan malam aku mulai dikandung ibuku!
\par 4 Ya Allah, jadikanlah hari itu gelap, hapuskan dari ingatan-Mu hingga lenyap; janganlah Engkau biarkan pula cahaya cerah menyinarinya.
\par 5 Jadikanlah hari itu hitam kelam, gelap gulita, kabur dan suram; liputilah dengan awan dan mega, tudungilah dari sinar sang surya.
\par 6 Hendaknya malam itu dihilangkan dari hitungan tahun dan bulan; jangan lagi dikenang, jangan pula dibilang.
\par 7 Biarlah malam itu penuh kegelapan tiada kemesraan, tiada kegembiraan.
\par 8 Hai orang perdukunan dan pengendali Lewiatan, timpalah hari itu dengan sumpah dan kutukan;
\par 9 jangan sampai bintang kejora bersinar, jangan biarkan sinar fajar memancar! Biarlah malam itu percuma menunggu datangnya hari dan harapan yang baru.
\par 10 Terkutuklah malam celaka ketika aku dilahirkan bunda, dan dibiarkan menanggung sengsara.
\par 11 Mengapa aku tidak mati dalam rahim ibu, atau putus nyawa pada saat kelahiranku?
\par 12 Mengapa aku dipeluk ibuku dan dipangkunya, serta disusuinya pada buah dadanya?
\par 13 Sekiranya pada saat itu aku berpulang, maka aku tidur dan mengaso dengan tenang,
\par 14 seperti para raja dan penguasa dahulu kala, yang membangun kembali istana zaman purba.
\par 15 Aku tertidur seperti putra raja, yang mengisi rumahnya dengan perak kencana.
\par 16 Mengapa aku tidak lahir tanpa nyawa supaya tidurku lelap dan terlena?
\par 17 Di sana, di dalam kuburan, penjahat tidak melakukan kejahatan, dan buruh yang habis tenaga dapat melepaskan lelahnya.
\par 18 Juga tawanan merasa lega, bebas dari hardik para penjaga.
\par 19 Di sana semua orang sama: yang tenar dan yang tidak ternama. Dan para budak bebas akhirnya.
\par 20 Mengapa manusia dibiarkan terus hidup sengsara? Mengapa terang diberi kepada yang duka?
\par 21 Mereka lebih suka kuburan daripada harta, menanti maut, tapi tak kunjung tiba.
\par 22 Kebahagiaan baru dapat dirasakan bila mereka mati dan dikuburkan.
\par 23 Masa depan mereka diselubungi oleh Allah, mereka dikepung olehnya dari segala arah.
\par 24 Gantinya makan aku mengeluh, tiada hentinya aku mengaduh.
\par 25 Segala yang kucemaskan, menimpa aku, segala yang kutakuti, melanda aku.
\par 26 Bagiku tiada ketentraman, aku menderita tanpa kesudahan."

\chapter{4}

\par 1 Lalu berbicaralah Elifas, katanya: "Ayub, kesalkah engkau bila aku bicara? Tak sanggup aku berdiam diri lebih lama.
\par 3 Banyak orang telah kauberi pelajaran, dan mereka yang lemah telah kaukuatkan.
\par 4 Kata-katamu yang memberi semangat, membangunkan orang yang tersandung, lemas dan penat.
\par 5 Tetapi kini engkau sendiri ditimpa duka; kau terkejut, dan menjadi putus asa.
\par 6 Bukankah engkau setia kepada Allah; bukankah hidupmu tiada cela? Jika begitu, sepantasnyalah engkau yakin dan tak putus asa.
\par 7 Pikirlah, pernahkah orang yang tak bersalah ditimpa celaka dan musibah?
\par 8 Aku tahu dari pengamatan bahwa orang yang membajak ladang kejahatan, dan menabur benih bencana bagai biji tanaman, akan menuai celaka dan kesusahan!
\par 9 Bagai badai, begitulah murka Allah membinasakan mereka hingga punah!
\par 10 Orang jahat mengaum dan meraung, bagaikan singa mereka menggerung. Tetapi Allah membungkam mereka serta mematahkan gigi-giginya.
\par 11 Mereka mati seperti singa kekurangan makanan, sehingga anak-anaknya bercerai-berai tak keruan.
\par 12 Pernah suatu wahyu sampai kepadaku, bagaikan bisikan halus dalam telingaku.
\par 13 Tidurku terganggu dan terusik olehnya, seperti mendapat mimpi yang buruk di malam buta.
\par 14 Aku pun terkejut dan menggigil gentar; seluruh tubuhku bergundang, gemetar.
\par 15 Angin sepoi menyentuh wajahku, maka tegaklah bulu romaku.
\par 16 Suatu sosok berdiri di hadapanku; kutatap, tetapi ia asing bagiku. Lalu kudengar bunyi suara memecah heningnya suasana,
\par 17 'Mungkinkah manusia tanpa salah di hadapan Allah? Mungkinkah ia tidak bercela di mata Penciptanya?
\par 18 Bahkan hamba-hamba Allah di surga, tak dapat dipercayai oleh-Nya. Bahkan pada malaikat-malaikat-Nya didapati-Nya kesalahan dan cela.
\par 19 Apalagi makhluk dari tanah liat makhluk debu yang dapat dipencet seperti ngengat!
\par 20 Makhluk itu hidup di waktu pagi, lalu mati terlantar di senja hari, dan untuk selamanya ia tak diingat lagi.
\par 21 Maka hilanglah segala yang dimiliki; ia pun mati tanpa beroleh akal budi.'

\chapter{5}

\par 1 Berserulah, hai Ayub, adakah jawaban? Malaikat mana yang kaumintai bantuan?
\par 2 Hanyalah orang yang bodoh saja yang mati sebab sakit hatinya.
\par 3 Pernah kulihat orang bodoh; hidupnya tampak amat kokoh. Tetapi tiba-tiba saja kukutuki tempat tinggalnya.
\par 4 Maka anak-anaknya tak pernah mendapat perlindungan; tak ada yang mau membela mereka di pengadilan.
\par 5 Apa yang dituai mereka di ladangnya, habis dimakan orang yang kosong perutnya. Bahkan gandum yang di tengah belukar, habis dilalap orang yang lapar. Milik dan kekayaan mereka menjadi incaran orang yang haus harta.
\par 6 Bukan dari bumi kejahatan muncul; bukan dari tanah kesusahan timbul.
\par 7 Bukan! Melainkan manusia sendirilah yang mendatangkan celaka atas dirinya, seperti percikan bunga api timbul dari apinya sendiri.
\par 8 Kalau aku dalam keadaanmu, pada Allah aku akan berseru. Kepada-Nya aku akan mengadu, serta menyerahkan perkaraku.
\par 9 Karya-karya Allah luar biasa; tidak sanggup kita memahaminya. Mujizat-mujizat yang dibuat-Nya, tak terbilang dan tiada habisnya.
\par 10 Ia menurunkan hujan ke atas bumi, hingga ladang dan padang basah tersirami.
\par 11 Dialah Allah yang meninggikan orang rendah, dan membahagiakan orang yang susah.
\par 12 Digagalkan-Nya rencana orang licik, dijebak dan ditangkap-Nya orang cerdik, sehingga semua usaha mereka tak jadi, dan mereka tertipu oleh akalnya sendiri.
\par 14 Pada siang hari mereka tertimpa kelam, meraba-raba seperti di waktu malam.
\par 15 Tetapi Allah menyelamatkan: orang miskin dari kematian, orang rendah dari penindasan.
\par 16 Kini ada harapan pula bagi orang yang tidak punya. Dan orang jahat diam seribu bahasa, karena Allah telah membungkam mulutnya.
\par 17 Mujurlah engkau bila Allah menegurmu! Jangan kecewa bila Ia mencelamu.
\par 18 Sebab Allah yang menyakiti, Ia pula yang mengobati. Dan tangan-Nya yang memukuli, juga memulihkan kembali.
\par 19 Jikalau bahaya mengancam, selalu engkau diselamatkan.
\par 20 Apabila paceklik mencekam, kau tak akan mati kelaparan. Jika perang mengganas seram, kau aman dari kematian.
\par 21 Kau dijaga dari fitnah dan hasut; kauhadapi musibah tanpa takut.
\par 22 Bila kekejaman dan kelaparan melanda, kau akan tabah dan bahkan tertawa. Kau tidak merasa takut atau cemas, menghadapi binatang buas yang ganas.
\par 23 Ladang yang kaubajak tak akan berbatu; binatang yang liar tak akan menyerangmu.
\par 24 Kau akan aman tinggal di kemahmu; dan jika kauperiksa domba-dombamu, semuanya selamat, tak kurang suatu.
\par 25 Anak cucumu tidak akan terbilang, mereka sebanyak rumput di padang.
\par 26 Bagai gandum masak, dituai pada waktunya, engkau pun akan hidup segar sampai masa tua.
\par 27 Ayub, semuanya itu telah kami periksa. Perkataan itu benar, jadi terimalah saja."

\chapter{6}

\par 1 Lalu Ayub menjawab, "Andaikata duka nestapaku ditimbang beratnya,
\par 3 pasti lebih berat daripada pasir samudra. Jadi, jangan heran jika kata-kataku kurang hati-hati serta terburu-buru.
\par 4 Panah dari Yang Mahakuasa menembus tubuhku; racunnya menyebar ke seluruh jiwa ragaku. Kedahsyatan Allah sangat mengerikan, dan menyerang aku bagai pasukan lawan.
\par 5 Keledai akan puas jika diberi rumput muda, begitu pula lembu jika diberi makanannya.
\par 6 Tetapi makanan hambar, siapa suka? Mana boleh putih telur ada rasanya?
\par 7 Tidak sudi aku menyentuhnya; muak aku jika memakannya.
\par 8 Mengapa Allah enggan mendengar doaku? Mengapa tak diperhatikan-Nya seruanku?
\par 9 Kiranya Allah berkenan meremukkan aku! Kiranya Ia bertindak dan membunuh aku!
\par 10 Bagiku hal itu akan merupakan hiburan; aku bakal menari di tengah penderitaan. Segala perintah Allah Yang Mahakudus, telah kutaati dan kuperhatikan terus.
\par 11 Apa kekuatanku sehingga aku masih ada? Apa harapanku untuk ingin hidup lebih lama?
\par 12 Sekuat batukah badanku ini? Dari tembagakah tubuhku ini?
\par 13 Habislah tenagaku mencari bantuan; bagiku tak ada lagi pertolongan.
\par 14 Dalam derita seperti ini, kudambakan sahabat sejati. Entah aku masih tetap setia atau sudah melalaikan Yang Mahakuasa.
\par 15 Tetapi kamu, hai kawan-kawan, tak dapat dipercaya dan diandalkan. Kamu seperti kali yang habis airnya, di kala hujan tak kunjung tiba.
\par 16 Kamu seperti sungai yang diam dan kaku, karena tertutup salju dan air beku.
\par 17 Segera bila tiba musim panas, salju dan es itu hilang tanpa bekas. Dasar sungai menjadi gersang, tidak berair dan kering kerontang.
\par 18 Kafilah-kafilah sesat ketika mencari air; mereka mengembara dan mati di padang pasir.
\par 19 Kafilah dari Syeba dan dari Tema mencari air itu dan mengharapkannya.
\par 20 Tetapi harapan mereka sia-sia di tepi kali yang tiada airnya.
\par 21 Seperti sungai itulah kamu, kawanku; kaumundur dan takut melihat deritaku.
\par 22 Kenapa? Apakah kuminta sesuatu darimu? Atau menyuruhmu menyogok orang untuk kepentinganku?
\par 23 Apakah aku minta diselamatkan dan ditebus dari musuh yang tak berbelaskasihan?
\par 24 Nah, ajarilah aku, tunjukkanlah kesalahanku! Aku akan diam dan mendengarkan perkataanmu.
\par 25 Kata-kata yang tulus menyejukkan hati, tetapi bicaramu kosong, tiada arti!
\par 26 Segala perkataanku kamu anggap angin saja; percuma kamu jawab aku yang sudah putus asa.
\par 27 Bahkan anak yatim piatu kamu undikan nasibnya, teman karibmu kamu curangi untuk menjadi kaya.
\par 28 Coba, perhatikanlah aku; masakan aku ini berdusta kepadamu?
\par 29 Jangan bertindak tak adil, sadarlah! Jangan mencela aku, aku sungguh tak salah.
\par 30 Apakah pada sangkamu aku berdusta, tak bisa membedakan yang baik dan yang tercela?

\chapter{7}

\par 1 Manusia itu seperti dipaksa berjuang; hidupnya berat seperti hidup seorang upahan;
\par 2 seperti budak yang merindukan naungan; seperti buruh yang menantikan imbalan.
\par 3 Bulan demi bulan hidupku tanpa tujuan; malam demi malam hatiku penuh kesedihan.
\par 4 Bila aku pergi tidur, malam merentang panjang; kurindukan fajar, tak dapat kuberbaring tenang.
\par 5 Tubuhku penuh cacing dan kerak darah; kulitku luka dan mengeluarkan nanah.
\par 6 Hidupku yang tanpa harap itu melaju menuju akhirnya, lebih laju daripada penenun menjalankan sekocinya.
\par 7 Ingatlah, ya Allah, hidupku hanya hembusan napas; kebahagiaanku hilang, tak meninggalkan bekas.
\par 8 Kini Engkau melihat aku--tetapi itu tidak lama. Jika nanti aku Kaucari, maka sudah tiada.
\par 9 Seperti awan yang meredup lalu menghilang, manusia pun mati, tak akan kembali pulang. Semua orang yang pernah mengenal dia, lupa kepadanya dan tak lagi mengingatnya.
\par 11 Sebab itu aku tak dapat tinggal diam! Rasa pedih dan pahitku tak dapat kupendam. Aku harus membuka mulutku, dan mencurahkan isi hatiku.
\par 12 Mengapa aku ini terus Kauawasi dan Kaujaga? Apakah aku ini naga laut yang berbahaya?
\par 13 Aku berbaring dan mencoba melepaskan lelah; aku mencari keringanan bagi hatiku yang gundah.
\par 14 Tetapi Kautakuti aku dengan impian; Kaudatangkan mimpi buruk dan khayalan.
\par 15 Sehingga aku lebih suka dicekik lalu mati daripada hidup dalam tubuh penuh derita ini.
\par 16 Aku lelah dan jemu hidup; aku ingin mati! Biarkan aku, sebab hidupku tidak berarti.
\par 17 Mengapa manusia begitu penting bagi-Mu? Mengapa tindakannya Kauperhatikan selalu?
\par 18 Kauselidiki dia setiap pagi, dan setiap saat dia Kauuji.
\par 19 Kapankah Engkau berpaling daripadaku, sehingga sempat aku menelan ludahku?
\par 20 Hai Penjagaku, rugikah Engkau karena dosaku? Mengapa Kaupakai aku sebagai sasaran panah-Mu? Begitu beratkah aku membebani diri-Mu?
\par 21 Tidak dapatkah Engkau mengampuni dosaku? Tidak mungkinkah Engkau menghapuskan salahku? Sebentar lagi aku terbaring dalam kuburan, dan bila Kaucari aku, tak akan Kaudapatkan."

\chapter{8}

\par 1 Lalu berkatalah Bildad, "Berapa lama lagi kaubicara begitu? Kata-katamu seperti angin yang menderu.
\par 3 Allah tidak pernah membengkokkan keadilan; tidak pernah gagal menegakkan kebenaran.
\par 4 Mungkin anak-anakmu berdosa terhadap Dia, maka sepantasnyalah mereka dihukum oleh-Nya.
\par 5 Tetapi jika kepada-Nya engkau bernaung, meminta belas kasihan dan memohon ampun,
\par 6 jika hatimu jujur, tanpa cela, Allah akan menolongmu dengan segera; dan sebagai imbalan, rumah tanggamu akan dipulihkan.
\par 7 Kekayaanmu yang hilang itu tidak berarti dibandingkan dengan apa yang kaudapat nanti.
\par 8 Orang arif di zaman dahulu hendaknya kauperhatikan, dan kaurenungkan pengalaman para nenek moyang.
\par 9 Hidup kita pendek, kita tak tahu apa-apa; hari-hari kita seperti bayangan belaka.
\par 10 Dengarkan perkataan orang arif itu dahulu, mereka memberi pelajaran ini kepadamu,
\par 11 'Di tempat berair saja tumbuh gelagah; pandan hanya terdapat di tanah bencah.
\par 12 Jika airnya kering, gelagah itu merana, lebih cepat daripada tumbuhan lainnya. Padahal masih segar dan belum saatnya, ia dipotong dan diambil manfaatnya.
\par 13 Begitulah orang yang tidak bertuhan. Ia lupa pada Allah, maka hilanglah harapan.
\par 14 Seutas benang yang lembut menjadi andalannya; sarang laba-laba menjadi kepercayaannya.
\par 15 Kuatkah sarang itu jika dijadikan sandaran? Tahankah benang itu jika dijadikan pegangan?'
\par 16 Seperti ilalang, segarlah orang yang tidak bertuhan; jika disinari surya, ia tumbuh subur dan memenuhi taman.
\par 17 Akarnya membelit batu-batu di tanah; melilit kuat, ia tak mudah goyah.
\par 18 Tetapi, coba, cabutlah sekarang ilalang itu, maka seolah-olah tak pernah ia ada di situ.
\par 19 Ya, kesenangan orang jahat cuma itu saja; orang lain datang dan menggantikan dia.
\par 20 Tapi Allah tak pernah meninggalkan orang setia, dan tak pernah pula Ia menolong orang durhaka.
\par 21 Mulutmu akan dibuat-Nya tertawa, bibirmu akan bersorak-sorak ria.
\par 22 Pembencimu akan malu dan merasa rendah, dan rumah penjahat akan dirusak hingga musnah."

\chapter{9}

\par 1 Tapi Ayub menjawab, "Memang, aku tahu, kata-katamu itu tak salah. Tapi, mana mungkin manusia berperkara melawan Allah dan mengalahkan-Nya?
\par 3 Dari seribu pertanyaan yang diajukan Allah, satu pun tak dapat dijawab oleh manusia.
\par 4 Allah itu sangat arif dan berkuasa; siapa dapat tahan melawan Dia?
\par 5 Allah memindahkan gunung tanpa diketahui orang, lalu menjungkirbalikkannya dengan murka dan berang.
\par 6 Allah membuat gempa sampai bumi berguncang, dan tiang penyangga dunia bergoyang-goyang,
\par 7 Allah dapat melarang matahari terbit di waktu pagi, dan mencegah bintang-bintang bersinar di malam hari.
\par 8 Dibentangkan-Nya angkasa tanpa bantuan; diinjak-injak-Nya punggung naga lautan.
\par 9 Dipasang-Nya gugusan bintang selatan di cakrawala, juga bintang Biduk, bintang Belantik dan bintang Kartika.
\par 10 Tak dapat kita memahami segala karya-Nya, tak sanggup kita menghitung mujizat-mujizat-Nya.
\par 11 Ia lewat di mukaku, tapi tidak tampak olehku; Ia lalu disampingku, tapi tidak nyata bagiku.
\par 12 Jika Ia merampas, siapa berani melarang-Nya? atau berani bertanya pada-Nya, 'Hai, Kau sedang apa?'
\par 13 Allah tidak menahan marah dan panas hati-Nya; para pembantu Rahab pun takluk kepada-Nya.
\par 14 Jadi, bagaimana aku dapat membantah Dia? Dengan kata-kata apa aku akan menjawab-Nya?
\par 15 Walaupun aku tidak bersalah, apa dayaku, kecuali mohon belas kasihan dari Allah hakimku?
\par 16 Seandainya Ia menjawab bila aku berseru, aku ragu, benarkah Ia sudi mendengar suaraku?
\par 17 Dia meremukkan aku dalam angin topan, dan menambah deritaku tanpa alasan.
\par 18 Tak dibiarkan-Nya aku menghela napas barang sesaat; bahkan dilimpahi-Nya aku dengan kepahitan yang hebat.
\par 19 Haruskah aku adu tenaga dengan Dia? Tetapi lihat, betapa hebat kekuatan-Nya! Seandainya terhadap Dia aku mengajukan gugatan, siapa berani memanggil-Nya ke pengadilan?
\par 20 Aku setia dan tak berbuat dosa, tetapi mulutku seakan berkata sebaliknya; setiap kata yang dibentuk bibirku, seolah-olah mempersalahkan diriku.
\par 21 Aku tak bersalah, tapi aku tak perduli. Aku jemu hidup. Bagiku tak ada yang penting lagi; yang bersalah dan yang benar, sama saja nasibnya: Allah akan membinasakan kedua-duanya.
\par 23 Bila orang yang tak bersalah mati tiba-tiba, Allah hanya tertawa saja.
\par 24 Bumi diserahkan Allah kepada pendurhaka, dan hakim-hakim telah dibuat-Nya buta semua. Allah melakukan semua ini, kalau bukan Dia, siapa lagi?
\par 25 Hari-hariku berlalu dengan kencang, lalu menghilang tanpa merasa senang.
\par 26 Seperti perahu laju, hidupku lewat dengan segera, secepat burung elang menukik, menyambar mangsanya.
\par 27 Jika aku mau tersenyum dan tertawa gembira, jika kucoba melupakan segala derita, maka dukaku datang kembali, memburu aku; sebab kutahu, Allah tetap mempersalahkan aku.
\par 29 Nah, jika begitu, peduli apa aku?
\par 30 Tak ada sabun yang dapat menghilangkan dosaku!
\par 31 Allah membenamkan aku dalam kotoran, sampai pakaianku pun menganggap aku menjijikkan.
\par 32 Seandainya Allah itu manusia, aku akan dapat menjawab Dia; lalu kami akan menghadap ke pengadilan untuk menyelesaikan segala pertengkaran.
\par 33 Tapi di antara kami tak ada jaksa yang dapat mengadili kami berdua.
\par 34 Semoga Allah berhenti menghukum aku dan menjauhkan kedahsyatan-Nya daripadaku!
\par 35 Aku tidak takut kepada-Nya dan akan bicara kini, sebab aku mengenal hatiku sendiri.

\chapter{10}

\par 1 Aku bosan dan muak dengan hidupku, maka kucurahkan kepahitan jiwaku.
\par 2 Ya Allah, janganlah aku Kaupersalahkan; jelaskanlah mengapa aku Kaulawan.
\par 3 Apa untungnya jika Engkau menindas begini, dan membuang hasil karya-Mu sendiri? Apa untungnya jika Engkau mendukung pendapat dan rencana para penjahat?
\par 4 Pandangan-Mu tak sama dengan pandangan manusia
\par 5 dan usia-Mu tidak sependek umurnya.
\par 6 Kalau begitu, mengapa Kauusut segala dosaku? mengapa Kauburu setiap kesalahanku?
\par 7 Sebenarnya Engkau tahu dan sadar, bahwa aku tak salah, tetapi benar. Kau tahu bahwa seorang pun tidak mampu menyelamatkan aku dari tangan-Mu.
\par 8 Aku ini dibentuk oleh tangan-Mu, masakan kini hendak Kaubinasakan aku?
\par 9 Ingatlah bahwa dari tanah liat Kauciptakan aku! Masakan Kaubuat aku kembali menjadi debu?
\par 10 Kaumungkinkan ayahku menjadikan aku dan Kaubesarkan aku dalam rahim ibu.
\par 11 Tubuhku Kaubentuk dengan kerangka dan urat; tulangku Kauberi daging dan kulit pembebat.
\par 12 Kauberi aku hidup; Engkau mengasihi aku, nyawaku Kaujaga dengan pemeliharaan-Mu.
\par 13 Tetapi sekarang kutahu bahwa selama itu, diam-diam telah Kaurancangkan celakaku.
\par 14 Kauawasi aku kalau-kalau berbuat kesalahan agar dapat Kautolak memberi pengampunan.
\par 15 Jikalau aku berbuat dosa, maka nasibku sungguh celaka! Tapi jika perbuatanku tak tercela, tetaplah aku dianggap berbuat dosa! Tak berani aku mengangkat kepala, sebab merasa sedih dan terhina.
\par 16 Jika kuberhasil, walau tak seberapa, Engkau memburu aku seperti singa. Dan Kautunjukkan kembali kuasa-Mu, hanyalah untuk menakutkan aku.
\par 17 Selalu Kauajukan saksi melawan aku; dan semakin besarlah murka-Mu kepadaku. Kaukerahkan pasukan-pasukan baru untuk menyerang dan memerangi aku.
\par 18 Mengapa Kaubiarkan aku lahir ke dunia? Lebih baik aku mati saja sebelum dilihat manusia!
\par 19 Maka seolah-olah aku tidak pernah dilahirkan, sebab dari rahim langsung dikuburkan.
\par 20 Ah, tak lama lagi aku akan mati, maka biarkanlah aku sendiri, agar dapat aku menikmati masaku yang masih sisa ini.
\par 21 Tak lama lagi aku pergi dan tak kembali, menuju negeri yang gelap dan suram sekali,
\par 22 negeri yang kelam, penuh bayangan dan kekacauan, di mana terang serupa dengan kegelapan."

\chapter{11}

\par 1 Kemudian Zofar berkata, "Tidakkah omong kosong itu diberi jawaban? Haruskah orang yang banyak mulut itu dibenarkan?
\par 3 Ayub, kaukira kami tak mampu menjawabmu? Kausangka kami bungkam karena ejekanmu?
\par 4 Menurut anggapanmu kata-katamu itu tak salah; menurut pendapatmu engkau bersih di hadapan Allah.
\par 5 Tapi, semoga Allah sendiri berbicara!
\par 6 Dan semoga engkau diberitahu oleh-Nya, bahwa hikmat itu banyak seginya, dan tak dapat dimengerti manusia. Maka sadarlah engkau bahwa deritamu tak berapa, dibandingkan dengan hukuman yang layak kauterima.
\par 7 Masakan hakekat Allah dapat kauselami? Masakan mampu kuasa-Nya engkau fahami?
\par 8 Kuasa-Nya lebih tinggi daripada angkasa; tak dapat engkau menjangkau dan meraihnya. Kuasa-Nya lebih dalam dari dunia orang mati, tak dapat kaumengerti sama sekali.
\par 9 Kuasa Allah lebih luas daripada buana, dan lebih lebar dari samudra raya.
\par 10 Jika Ia menangkap dan membawamu ke mahkamah-Nya, maka siapa berani menghalangi tindakan-Nya?
\par 11 Allah mengenali orang yang suka berdusta; kejahatan mereka tak luput dari mata-Nya.
\par 12 Kalau induk keledai liar melahirkan keledai jinak, barulah orang bodoh menjadi bijak.
\par 13 Ayub, bersihkanlah hatimu, menyesallah! Berdoalah kepada Allah!
\par 14 Hilangkanlah dosa dari hatimu dan jauhkanlah kejahatan dari rumahmu!
\par 15 Maka kau boleh menghadapi hidup dengan tabah dan gagah; kau akan berdiri teguh dan tak perlu merasa gelisah,
\par 16 bahkan deritamu tidak lagi kaukenang; bagai banjir yang surut, dilupakan orang.
\par 17 Hidupmu akan menjadi lebih terang dari siang hari, dan saat-saat gelap dalam hidupmu secerah sinar pagi.
\par 18 Kau akan teguh dan penuh harapan; Allah akan melindungimu sehingga kau aman.
\par 19 Engkau tidak akan takut kepada seteru; banyak orang akan minta tolong kepadamu.
\par 20 Tapi orang jahat akan memandang kebingungan, sebab bagi mereka tak ada pertolongan. Satu-satunya harapan mereka ialah agar ajal segera tiba."

\chapter{12}

\par 1 Ayub menjawab, "Memang, kamu ini mewakili umat manusia. Jika kamu mati, hikmat akan mati juga.
\par 3 Aku pun manusia yang berakal budi; semua yang kamu katakan itu sudah kumengerti. Lagipula, siapa yang tak tahu semua itu? Jadi, jangan sangka kamu melebihi aku!
\par 4 Aku ditertawakan teman dan sahabat, padahal aku ini benar dan tanpa cacat. Dahulu Allah menjawab doaku, bilamana aku berseru minta dibantu.
\par 5 Kamu menghina orang celaka, sedang hidupmu aman; orang yang hampir jatuh kamu beri pukulan.
\par 6 Tetapi hidup perampok tidak terancam; orang yang berani menentang Allah, hidup tentram. Padahal dewa yang menjadi andalan mereka, hanyalah kekuatan sendiri saja.
\par 7 Bertanyalah kepada burung dan binatang lainnya maka kamu akan diberi pengajaran oleh mereka.
\par 8 Mintalah keterangan kepada makhluk di bumi dan di lautan, maka kamu akan menerima penjelasan.
\par 9 Siapa di antara semua itu tidak menyadari, bahwa pencipta mereka adalah Allah sendiri?
\par 10 Dia mengatur hidup segala makhluk yang ada; Dia berkuasa atas nyawa setiap manusia.
\par 11 Seperti lidahku suka mengecap makanan yang nyaman, begitulah telingaku suka mendengar perkataan.
\par 12 Kabarnya, hikmat ada pada orang yang tinggi umurnya, tapi hikmat dan kekuatan ada pada Allah saja. Konon pengertian ada pada orang yang lanjut usia, namun pengertian dan wewenang ada pada Allah jua!
\par 14 Apa yang dibongkar-Nya tak dapat dibangun lagi siapa yang ditawan-Nya tak dapat bebas kembali.
\par 15 Ia membendung air, maka kering dan tandus semuanya; Ia melepaskannya mengalir, maka banjir melanda!
\par 16 Allah itu kuat, tangguh dan jaya! Si penipu dan yang tertipu ada di bawah kekuasaan-Nya.
\par 17 Dicabut-Nya hikmat para penguasa pemerintahan, dan para pemimpin dijadikan-Nya bahan tertawaan.
\par 18 Digulingkan-Nya raja dan ditawan-Nya mereka;
\par 19 direndahkan-Nya imam-imam dan para pemuka.
\par 20 Dibungkam-Nya orang yang fasih bicara, dicabut-Nya hikmat orang-orang tua.
\par 21 Dipermalukan-Nya orang yang terkemuka; diambil-Nya kedaulatan mereka yang berkuasa.
\par 22 Diterangi-Nya tempat-tempat yang suram; disinari-Nya bayangan-bayangan hitam kelam.
\par 23 Bangsa-bangsa dibuat-Nya berkembang dan makmur, lalu dibinasakan-Nya mereka sampai hancur.
\par 24 Dijadikan-Nya para pemimpin kebingungan, sehingga mereka mengembara tanpa tujuan.
\par 25 Mereka meraba-raba di dalam kegelapan, dan terhuyung-huyung bagai orang mabuk minuman.

\chapter{13}

\par 1 Semuanya itu pernah kulihat dan kudengar sendiri, dan sungguh-sungguh kufahami.
\par 2 Apa yang kamu tahu, aku pun tahu; jangan sangka aku kalah denganmu.
\par 3 Tetapi aku hendak bicara dengan Yang Mahakuasa; aku ingin membela perkaraku di hadapan Allah.
\par 4 Sebab kamu menutupi kebodohanmu dengan tipu, tak ubahnya seperti dukun-dukun palsu.
\par 5 Seandainya kamu tidak bicara, mungkin kamu dianggap bijaksana.
\par 6 Maka dengarkanlah bantahanku; perhatikanlah pembelaanku.
\par 7 Bolehkah demi Allah, kamu berdusta? Bolehkah kamu berbohong untuk kepentingan-Nya?
\par 8 Bolehkah kamu memihak Allah, dan membela-Nya sebagai pengacara?
\par 9 Jika Allah memeriksamu, akan baikkah hasilnya? Dapatkah kamu menipu-Nya seperti menipu manusia?
\par 10 Bila kamu memihak, walaupun dengan diam-diam, kamu akan dihukum Allah dengan kejam.
\par 11 Apakah keagungan Allah tidak mengejutkan jiwamu? Apakah dahsyat-Nya tidak mengecutkan hatimu?
\par 12 Segala nasihatmu seperti debu yang tak berfaedah; pembelaanmu seperti tanah lempung yang mudah pecah.
\par 13 Sebab itu diamlah, biarlah aku bicara! Aku tak perduli bagaimana pun akibatnya!
\par 14 Aku siap mempertaruhkan nyawa!
\par 15 Aku nekad sebab sudah putus asa! Jika Allah hendak membunuhku, aku berserah saja, namun akan kubela kelakuanku di hadapan-Nya.
\par 16 Mungkin karena keberanianku itu aku selamat, sebab orang jahat tak akan berani menghadap Allah!
\par 17 Sekarang dengarlah baik-baik perkataanku; perhatikanlah keteranganku.
\par 18 Perkaraku sudah siap kukemukakan; aku yakin, tak dapat aku dipersalahkan!
\par 19 TUHAN, jika Engkau datang dan menuduh aku, aku akan diam dan menunggu ajalku.
\par 20 Tapi kabulkanlah dua permohonanku ini, supaya aku berani menghadap-Mu lagi:
\par 21 berhentilah menyiksa aku, dan janganlah Kautimpa aku dengan kedahsyatan-Mu!
\par 22 Bicaralah, maka akan kuberi jawaban. Atau biarlah aku bicara, lalu berilah balasan!
\par 23 Berapa banyak salah dan dosa yang kulakukan? Segala pelanggaranku hendaknya Kausebutkan!
\par 24 Mengapa Kau menghindar dan menyingkir daripadaku? Apa sebabnya Kauanggap aku sebagai musuh-Mu?
\par 25 Aku hanya daun yang gugur kena angin lalu, masakan Kaugentarkan aku! Aku hanya jerami yang kering dan layu, mana boleh Kaukejar-kejar aku!
\par 26 Kautentukan nasib yang pahit bagiku; Kaubalaskan kesalahanku di masa mudaku.
\par 27 Kakiku Kauikat dengan rantai besi; segala gerak-gerikku Kauawasi. Bahkan telapak kakiku, tak lepas dari pandangan-Mu.
\par 28 Maka aku menjadi rapuh seperti kayu yang busuk; seperti kain dimakan ngengat aku menjadi lapuk.

\chapter{14}

\par 1 Sejak lahir manusia itu lemah, tidak berdaya; hidupnya singkat serta penuh derita.
\par 2 Ia bersemi dan layu seperti kembang; lenyap seperti bayangan, terus menghilang.
\par 3 Ya Allah, masakan Engkau mau memandangku, dan menghadapkan aku ke pengadilan-Mu!
\par 4 Dapatkah manusia yang berdosa mendatangkan hal yang sempurna?
\par 5 Jumlah umur manusia sudah Kautentukan; jumlah bulannya sudah Kaupastikan. Kautetapkan pula batas-batas hidupnya; tidak mungkin ia melangkahinya.
\par 6 Biarkanlah ia beristirahat, jangan ganggu dia; supaya ia dapat menikmati hidupnya sampai selesai tugasnya.
\par 7 Masih ada harapan bagi pohon yang ditebang; ia akan bertunas lagi, lalu bercabang.
\par 8 Meskipun di dalam tanah akarnya menjadi lapuk, dan tanggulnya mati karena busuk,
\par 9 tetapi bila disentuh air, ia tumbuh lagi; seperti tanaman muda, tunas-tunasnya muncul kembali.
\par 10 Tapi bila manusia mati, habis riwayatnya; ia meninggal dunia, lalu ke mana perginya?
\par 11 Seperti air menguap dari dalam telaga, seperti sungai surut sampai habis airnya,
\par 12 begitu pula manusia yang telah mati: ia tidak akan dapat bangkit kembali. Ia tak akan terjaga selama langit masih ada, tak pernah lagi bangun dari tidurnya.
\par 13 Sembunyikanlah aku di dalam dunia orang mati; lindungilah aku sampai Kau tidak marah lagi. Tapi tentukanlah waktu untuk mengingat diriku.
\par 14 Sebab, apabila manusia mati, dapatkah ia hidup kembali? Hari demi hari aku menunggu sampai masa pahitku ini lalu.
\par 15 Maka Engkau akan memanggil aku, dan aku pun akan memberi jawaban; Engkau akan sayang lagi kepadaku, makhluk yang Kauciptakan.
\par 16 Lalu akan Kauawasi setiap langkahku, tapi tidak lagi Kauperhatikan dosaku.
\par 17 Dosaku akan Kauampuni dan Kausingkirkan; salahku waktu dulu akan Kauhapuskan.
\par 18 Kelak gunung-gunung akan runtuh dan porak poranda, dan gunung batu yang kokoh bergeser dari tempatnya.
\par 19 Batu-batu akan dikikis oleh air yang mengalir kuat; tanah akan dihanyutkan oleh hujan yang lebat. Demikianlah Kauhancurkan harapan manusia.
\par 20 Kaukalahkan dia untuk selama-lamanya; Kausuruh dia pergi dan Kauubah wajahnya.
\par 21 Anak-anaknya menjadi orang mulia, tetapi ia tidak mengetahuinya. Dan apabila mereka menjadi hina, tak ada yang memberitahukan kepadanya.
\par 22 Hanya nyeri tubuhnya yang dirasakannya; hanya pilu hatinya yang dideritanya."

\chapter{15}

\par 1 Maka Elifas menjawab, "Omong kosong, Ayub, cakapmu sungguh tiada arti! Tak ada orang arif yang menjawab seperti kau ini, tak akan ia membela dirinya dengan kata-kata yang tak ada maknanya.
\par 4 Seandainya omonganmu itu dituruti, tak seorang pun takut atau berdoa kepada Allah lagi.
\par 5 Kata-katamu membuktikan bahwa engkau bersalah, tapi kejahatanmu kaututupi dengan bersilat lidah.
\par 6 Tak perlu engkau kutuduh dan persalahkan, sebab oleh kata-katamu sendiri kau diadukan.
\par 7 Kaukira engkau manusia pertama yang dilahirkan? Hadirkah engkau ketika gunung-gunung diciptakan?
\par 8 Apakah kau mendengar Allah membuat rencana-Nya? Apakah hanya engkau yang mempunyai hikmat manusia?
\par 9 Segala yang kauketahui, kami pun ketahui; segala yang kaufahami, jelas pula bagi kami.
\par 10 Hikmat ini kami terima dari orang yang beruban; mereka sudah ada sebelum ayahmu dilahirkan!
\par 11 Mengapa penghiburan Allah enggan kauterima? Kami bicara dengan sabar dan lembut atas nama-Nya.
\par 12 Tetapi kau naik pitam, matamu menyala-nyala;
\par 13 kau marah kepada Allah dan membantah-Nya.
\par 14 Mungkinkah manusia sama sekali tak salah? Dapatkah ia dibenarkan di hadapan Allah?
\par 15 Bahkan kepada malaikat pun Allah tidak percaya; mereka tidak suci pada pemandangan-Nya.
\par 16 Apalagi manusia yang bejat dan ternoda, yang meneguk kejahatan seperti air saja.
\par 17 Dengar Ayub, kau akan kuterangkan sesuatu,
\par 18 yang diajarkan orang arif kepadaku. Ajaran itu diterimanya dari leluhurnya, dan diteruskan dengan lengkap kepada keturunannya.
\par 19 Waktu itu tak ada orang asing di negeri mereka; tak ada yang menyesatkan mereka dari Allah.
\par 20 Orang jahat yang menindas sesamanya, akan merasa cemas sepanjang hidupnya.
\par 21 Bunyi-bunyi dahsyat memekakkan telinganya; di saat yang aman perampok datang menyerangnya.
\par 22 Tak ada harapan baginya mengelak kegelapan, sebab pedang pembunuh mengejarnya pada setiap kesempatan.
\par 23 Burung-burung nasar menunggu saat kematiannya, mereka hendak melahap mayatnya. Maka sadarlah ia bahwa suramlah hari depannya.
\par 24 Bencana bagaikan raja perkasa, sudah siap hendak menyergapnya.
\par 25 Begitulah nasib orang yang menantang Allah, dan berani melawan Yang Mahakuasa.
\par 26 Dengan sombong ia menyerbu dan melawan Allah; diangkatnya perisainya, ia pantang mengalah.
\par 28 Ia menetap di kota-kota yang porak-poranda, di rumah-rumah yang tak ada penghuninya. Kota-kota itu sudah ditentukan untuk tetap menjadi reruntuhan.
\par 29 Kekayaan orang itu akan hilang tanpa bekas; harta bendanya akan habis tandas.
\par 30 Ia tak akan luput dari gelap gulita; api akan menghanguskan tunas-tunasnya. Ia akan musnah oleh hembusan mulut Allah.
\par 31 Jika ia percaya kepada yang tak berguna, akan tertipulah ia; dan imbalan yang akan diterimanya, tidak berguna juga.
\par 32 Sebelum tiba masanya, ia akan mati, seperti dahan layu yang tak dapat hijau lagi.
\par 33 Ia seperti pohon anggur yang gugur buahnya, seperti pohon zaitun yang rontok bunganya.
\par 34 Orang yang jahat tak akan berketurunan; habis terbakarlah rumah yang dibangunnya dari hasil suapan.
\par 35 Itulah mereka yang merancangkan kejahatan dan melaksanakannya; tipu muslihat selalu terkandung dalam hatinya."

\chapter{16}

\par 1 Tetapi Ayub menjawab, "Seringkali kudengar pendapat demikian; penghiburanmu hanyalah siksaan.
\par 3 Kapankah omong kosong itu kamu hentikan? Apa yang merangsang kamu untuk memberi jawaban?
\par 4 Seandainya kamu ini aku, dan aku kamu, aku pun dapat bicara sama seperti itu. Kubanjiri kamu dengan penuturan; kepalaku akan kugeleng-gelengkan.
\par 5 Hatimu akan kukuatkan dengan berbagai anjuran; kata-kataku akan memberi penghiburan.
\par 6 Kalau aku bicara, deritaku tidak reda; jika aku berdiam diri, apa pula gunanya?
\par 7 Allah, membuat aku kepayahan; seluruh keluargaku telah dibinasakan.
\par 8 Dia menentang dan menangkap aku. Sekarang kurus keringlah tubuhku, dan bagi banyak orang itulah buktinya bahwa aku telah berdosa.
\par 9 Dengan geram Allah merobek-robek tubuhku; dengan sangat benci Ia memandang aku.
\par 10 Orang-orang mengejek aku dengan mulut terbuka lebar; aku dikeroyok dan pipiku ditampar.
\par 11 Allah menyerahkan aku kepada orang durhaka; aku dijatuhkan-Nya ke tangan orang durjana.
\par 12 Tadinya hidupku aman dan sentosa, tapi Allah menyerang aku dengan tiba-tiba. Tengkukku dicengkeram-Nya dan aku dicampakkan; dijadikan-Nya aku sasaran untuk latihan.
\par 13 Tanpa rasa iba Ia terus memanah aku, sehingga terburailah isi perutku.
\par 14 Ia menyerbu seperti seorang pejuang, dan melukai aku dengan berulang-ulang.
\par 15 Aku memakai karung tanda kesedihan, dan duduk dalam debu karena dikalahkan.
\par 16 Wajahku merah karena tangisku; kelopak mataku bengkak dan biru.
\par 17 Tapi aku tidak melakukan kekerasan; nyata tuluslah doaku kepada TUHAN.
\par 18 Hai bumi, kejahatan terhadapku jangan sembunyikan; jangan diamkan teriakku minta keadilan.
\par 19 Aku tahu bahwa Pembelaku ada di surga; Ia memberi kesaksian bahwa aku tak berdosa.
\par 20 Aku diejek teman-temanku dan ditertawakan; sambil menangis aku menghadap Allah minta bantuan.
\par 21 Ah, kiranya Allah sendiri membela aku di hadapan-Nya, seperti seorang yang rela membela sahabatnya.
\par 22 Tahun-tahunku yang sisa tak banyak lagi; sebentar lagi aku pergi dan tak akan kembali.

\chapter{17}

\par 1 Ajalku sudah dekat, hampir putuslah napasku; hanyalah kuburan yang tinggal bagiku.
\par 2 Orang menjadikan aku bahan ejekan; kulihat betapa mereka melontarkan sindiran.
\par 3 Aku ini jujur, ya Allah. Percayalah padaku! Siapa lagi yang dapat menyokong perkataanku?
\par 4 Kaututup hati mereka sehingga tak mengerti; jangan sampai mereka menundukkan aku kini.
\par 5 Menurut pepatah, siapa mengadukan teman demi keuntungan, anak-anaknya sendiri akan menerima pembalasan.
\par 6 Kini aku disindir dengan pepatah itu; mereka datang untuk meludahi mukaku.
\par 7 Mataku kabur karena dukacita; seluruh tubuhku kurus merana.
\par 8 Orang yang saleh, terkejut dan heran; orang yang tak bersalah, menganggap aku tidak bertuhan.
\par 9 Orang yang baik dan yang tidak bersalah, makin yakin cara hidupnya berkenan kepada Allah.
\par 10 Tapi seandainya kamu semua datang ke mari, tak seorang bijaksana pun yang akan kudapati.
\par 11 Hari-hariku telah lalu, gagallah segala rencanaku; hilang pula semua cita-cita hatiku.
\par 12 Tetapi sahabat-sahabatku berkata, 'Malam itu siang dan terang hampir tiba.' Namun aku tahu dalam hatiku bahwa tetap gelaplah keadaanku.
\par 13 Hanya dunia mautlah yang kuharapkan, di sanalah aku akan tidur dalam kegelapan.
\par 14 Kuburku kunamakan "Ayahku", dan cacing-cacing pemakan tubuhku kusebut "Ibu" dan "Saudara perempuanku".
\par 15 Di manakah harapan bagiku; siapa melihat adanya bahagia untukku?
\par 16 Apabila aku turun ke dunia orang mati, aku tidak mempunyai harapan lagi."

\chapter{18}

\par 1 Maka jawab Bildad, "Hai Ayub, kapankah kau habis bicara? Diamlah, dan dengarkanlah kini kami mau berkata-kata.
\par 3 Mengapa kauanggap kami dungu, dan kausamakan kami dengan lembu?
\par 4 Kemarahanmu hanya menyakiti dirimu. Haruskah untuk kepentinganmu bumi kehilangan penduduknya, dan gunung-gunung dipindahkan dari tempatnya?
\par 5 Pelita orang jahat pasti dipadamkan; apinya tak akan pernah lagi dinyalakan.
\par 6 Terang dalam kemahnya menjadi pudar; pelita penerangnya tidak lagi bersinar.
\par 7 Langkahnya yang mantap menjadi terhuyung-huyung; rancangannya sendiri menyebabkan ia tersandung.
\par 8 Ia berjalan ke dalam jaring, maka tersangkutlah kakinya.
\par 9 Tumitnya terjerat oleh perangkap, sehingga tertangkaplah ia.
\par 10 Di tanah, tersembunyi tali jerat; di jalan, terpasang jebak dan pikat.
\par 11 Orang jahat dikejutkan oleh kengerian dari segala arah; ketakutan mengikutinya langkah demi langkah.
\par 12 Dahulu ia kuat, kini ia merana; bencana menemaninya di mana-mana.
\par 13 Kulitnya dimakan penyakit parah; lengan dan kakinya busuk bernanah.
\par 14 Ia direnggut dari kemahnya, tempat ia merasa aman, lalu diseret untuk menghadap kematian.
\par 15 Kini siapa saja boleh tinggal dalam kemahnya, dan di situ ditaburkan belerang, pembasmi penyakitnya.
\par 16 Akar-akarnya gersang dan berkerut; ranting-rantingnya kering dan kisut.
\par 17 Ia tak dikenal lagi di dalam maupun di luar kota; tak ada seorang pun yang masih ingat namanya.
\par 18 Dari terang ia diusir ke dalam kegelapan; dari dunia orang hidup ia dienyahkan.
\par 19 Anak dan keturunan ia tak punya; di kampung halamannya seorang pun tak tersisa.
\par 20 Mendengar nasibnya penduduk di barat terkejut, sedang penduduk di timur gemetar karena takut.
\par 21 Begitulah nasib orang durhaka, mereka yang tidak mengindahkan Allah."

\chapter{19}

\par 1 Tetapi Ayub menjawab, "Mengapa aku terus kamu kecam, dan kamu siksa dengan perkataan?
\par 3 Berkali-kali kamu menghina aku, dan kamu aniaya aku tanpa rasa malu.
\par 4 Seandainya salah perbuatanku, itu tidak merugikan kamu.
\par 5 Kamu pikir dirimu lebih baik daripadaku; susahku kamu anggap bukti kesalahanku.
\par 6 Ketahuilah bahwa aku sedang disiksa Allah, dan ditangkap dalam perangkap-Nya.
\par 7 Aku meronta karena kekejaman-Nya itu, tetapi tidak seorang pun yang memperhatikan aku. "Di mana keadilan," teriakku, tetapi tak ada yang mendengar aku.
\par 8 Allah menutup jalanku, aku tak dapat lewat, lorong-lorongku dibuat-Nya gelap pekat.
\par 9 Ia merampas hartaku semua, dan nama baikku dirusakkan-Nya.
\par 10 Ia menghantam aku dari segala jurusan, seperti orang mencabut akar dari tanaman, lalu membiarkannya merana dan layu, begitulah direnggut-Nya segala harapanku.
\par 11 Murka Allah kepadaku menyala-nyala; aku dianggap-Nya sebagai musuh-Nya.
\par 12 Pasukan-Nya menyerbu tanpa dapat dibendung; jalanku dihalangi, dan kemahku dikepung.
\par 13 Sanak saudaraku dijauhkan-Nya daripadaku; aku menjadi orang asing bagi semua kenalanku.
\par 14 Kaum kerabatku semua menjauhkan diri; teman-temanku tak ingat kepadaku lagi.
\par 15 Hamba perempuanku lupa siapa aku, tuan mereka; dianggapnya aku orang yang belum dikenalnya.
\par 16 Kupanggil hambaku, tapi ia tak menyahut, meskipun kubujuk dia dengan lembut.
\par 17 Istriku muak mencium bau napasku, saudara kandungku tak sudi mendekatiku.
\par 18 Aku dihina oleh anak-anak di jalan; jika aku berdiri, aku ditertawakan.
\par 19 Melihat aku, teman karibku merasa ngeri; aku ditinggalkan mereka yang kukasihi.
\par 20 Tubuhku tinggal kulit pembalut tulang; hampir saja aku mati dan nyawaku melayang.
\par 21 Hai kawan-kawanku, kasihanilah aku, sebab tangan Allah memukul aku.
\par 22 Allah terus menekan aku; mengapa kamu tiru Dia? Belum puaskah kamu menyiksa?
\par 23 Ah, kiranya kata-kataku dicatat, sehingga akan selalu diingat;
\par 24 kiranya dengan besi dipahat pada batu, supaya bertahan sepanjang waktu.
\par 25 Aku tahu bahwa di surga ada Pembelaku; akhirnya Ia akan datang menolong aku.
\par 26 Meskipun kulitku luka-luka dan pecah, tapi selama aku bertubuh, akan kupandang Allah.
\par 27 Dengan mataku sendiri Dia akan kulihat, dan bagiku Dia menjadi sahabat. Hatiku hancur sebab kamu berkata,
\par 28 'Bagaimana caranya kita mendakwanya?' Kamu mencari alasan untuk membuat perkara.
\par 29 Tetapi, kini takutlah kepada pedang! Sebab Allah murka dan menghukum orang berdosa; maka tahulah kamu, bahwa ada Allah yang mengadili manusia."

\chapter{20}

\par 1 Lalu Zofar menjawab, "Hai Ayub, aku merasa tersinggung olehmu, kini aku ingin segera memberi jawabanku.
\par 3 Kata-katamu itu sungguh menghina, tetapi aku tahu bagaimana menjawabnya.
\par 4 Tetapi tahukah engkau bahwa dari zaman purba, sejak manusia mula-mula ditempatkan di dunia,
\par 5 kegembiraan orang jahat hanya sebentar saja, dan kesenangan orang durhaka sekejap mata?
\par 6 Walaupun kebesarannya sampai ke angkasa, sehingga kepalanya menyentuh mega,
\par 7 namun ia akan lenyap selama-lamanya, menghilang dari dunia dengan cara yang terhina. Orang-orang yang pernah mengenal dia, akan bertanya, "Hai, ke mana perginya?"
\par 8 Ia akan hilang seperti bayangan mimpi, lenyap seperti penglihatan di malam hari.
\par 9 Ia tak tampak lagi oleh mata; ia tak ada lagi di tempat tinggalnya.
\par 10 Yang dulu dicurinya dari orang tak punya harus diganti oleh anak-anaknya.
\par 11 Walaupun ia muda dan perkasa, tapi sebentar lagi ia menjadi debu belaka.
\par 12 Alangkah manis kejahatan dalam mulutnya! Rasanya sayang untuk segera menelannya; sebab itu disimpannya di bawah lidahnya, supaya lama ia menikmatinya.
\par 14 Tapi makanan itu berubah di dalam perut, menjadi racun pahit pembawa maut.
\par 15 Harta curian yang ditelannya, terpaksa dimuntahkannya; Allah mengeluarkannya dari dalam perutnya.
\par 16 Penjahat akan minum racun pembawa bencana, ia akan mati olehnya seperti digigit ular berbisa.
\par 17 Tak akan ia menikmati minyak zaitun yang berlimpah, ataupun susu dan madu yang bertumpah ruah.
\par 18 Segala labanya harus dikembalikannya; hasil usahanya tak akan dinikmatinya.
\par 19 Sebab ia menindas dan menterlantarkan orang yang tak punya; ia merampas rumah-rumah yang tidak dibangunnya.
\par 20 Karena serakahnya tak mengenal batas, maka ia tak akan menjadi puas.
\par 21 Jika ia makan semuanya dihabiskan, sebab itu kemakmurannya tidak bertahan.
\par 22 Ketika memuncak kemakmurannya, derita dan duka datang menimpanya.
\par 23 Ketika ia sibuk mengisi perutnya, Allah menjadi sangat murka dan menghukumnya.
\par 24 Jika ia lari menghindar dari pedang baja, ia akan dilukai panah tembaga.
\par 25 Ia kena panah, sehingga luka; ujung panah yang berkilat menembus tubuhnya, maka ketakutan meliputi hatinya.
\par 26 Hancurlah segala harta simpanannya; dia beserta seluruh keluarganya dimakan api yang tidak dinyalakan manusia.
\par 27 Langit menyingkapkan kejahatannya; bumi bangkit melawan dia.
\par 28 Segala kekayaannya akan musnah, karena luapan amarah Allah.
\par 29 Itulah nasib orang yang durjana, nasib yang ditentukan Allah baginya."

\chapter{21}

\par 1 Ayub menjawab, "Dengarkan apa yang akan kukatakan; hanya itu yang kuminta sebagai penghiburan.
\par 3 Izinkanlah aku ganti bicara, setelah itu boleh lagi kamu menghina!
\par 4 Bukan dengan manusia aku bertengkar, jadi tak mengapalah jika aku kurang sabar.
\par 5 Pandanglah aku, maka kamu akan tercengang, kamu terpaku dan mulutmu bungkam.
\par 6 Jika kupikirkan keadaanku ini gemetarlah aku karena ngeri.
\par 7 Mengapa orang jahat diberi umur panjang oleh Allah, dan harta mereka terus bertambah?
\par 8 Mereka hidup cukup lama sehingga melihat anak cucu mereka menjadi dewasa.
\par 9 Rumah tangga mereka aman sentosa; Allah tidak mendatangkan bencana atas mereka.
\par 10 Ternak mereka berkembang biak, dan tanpa kesulitan, beranak.
\par 11 Anak-anak orang jahat berlompatan dengan gembira, seperti domba muda yang bersukaria.
\par 12 Diiringi bunyi rebana, seruling dan kecapi, mereka ramai bernyanyi dan menari-nari.
\par 13 Hari-harinya dihabiskan dalam kebahagiaan, dan mereka meninggal penuh kedamaian.
\par 14 Padahal mereka telah berkata kepada Allah, "Jauhilah kami dan pergilah! Kami tak peduli dan tak ingin mengerti maksud dan kehendak-Mu bagi hidup kami."
\par 15 Pikir mereka, "Melayani Allah tak ada gunanya, dan berdoa kepada-Nya tiada manfaatnya.
\par 16 Bukankah karena kekuatan kita saja, tercapailah segala maksud dan tujuan kita?" Akan tetapi aku sama sekali tidak setuju dengan jalan pikiran dan pendapat begitu.
\par 17 Pernahkah pelita orang jahat dipadamkan, dan mereka ditimpa bencana dan kemalangan? Pernahkah Allah marah kepada mereka, sehingga mereka dihukum-Nya?
\par 18 Pernahkah mereka seperti jerami dan debu yang ditiup oleh badai dan oleh angin lalu?
\par 19 Kamu berkata, "Anak dihukum Allah karena dosa ayahnya." Tapi kataku: Orang berdosa itulah yang harus dihukum Allah, agar mereka sadar bahwa karena dosa mereka, maka Allah mengirimkan hukuman-Nya.
\par 20 Biarlah orang berdosa menanggung dosanya sendiri, biarlah dirasakannya murka Allah Yang Mahatinggi.
\par 21 Jika manusia habis masanya di dunia, masih pedulikah ia entah keluarganya bahagia?
\par 22 Dapatkah manusia mengajar Allah, sedangkan Allah sendiri yang menghakimi makhluk di surga?
\par 23 Ada orang yang sehat selama hidupnya; ia meninggal dengan puas dan lega. Matinya tenang dengan rasa bahagia, sedang tubuhnya masih penuh tenaga.
\par 25 Tapi ada pula yang mati penuh kepahitan, tanpa pernah mengenyam kebahagiaan.
\par 26 Namun di kuburan, mereka sama-sama terbaring; dikerumuni oleh ulat dan cacing.
\par 27 Memang aku tahu apa yang kamu pikirkan dan segala kejahatan yang kamu rancangkan.
\par 28 Tanyamu, "Di mana rumah penguasa yang melakukan perbuatan durhaka?"
\par 29 Belumkah kamu menanyai orang yang banyak bepergian? Tidak percayakah kamu berita yang mereka laporkan?
\par 30 Kata mereka, "Pada hari Allah memberi hukuman, para penjahat akan diselamatkan."
\par 31 Tak ada yang menggugat kelakuannya; tak ada yang membalas kejahatannya.
\par 32 Ia dibawa ke kuburan, dan dimasukkan ke dalam liang lahat; makamnya dijaga dan dirawat.
\par 33 Ribuan orang berjalan mengiringi jenazahnya; dengan lembut tanah pun menimbuninya.
\par 34 Jadi, penghiburanmu itu kosong dan segala jawabanmu bohong!"

\chapter{22}

\par 1 Lalu berkatalah Elifas, "Di antara umat manusia, tidak seorang pun berguna bagi Allah. Orang yang sangat berakal budi, hanya berguna bagi dirinya sendiri.
\par 3 Apakah ada faedahnya bagi Allah, jika engkau melakukan kehendak-Nya? Apakah ada untung bagi-Nya, jika hidupmu sempurna?
\par 4 Bukan karena takutmu kepada Allah, engkau dituduh dan dianggap bersalah,
\par 5 melainkan karena sangat banyak dosamu, dan amat jahat tindakan dan kelakuanmu.
\par 6 Jika saudaramu tak dapat membayar hutangnya, kaurampas semua pakaiannya.
\par 7 Orang yang lelah tidak kauberi minuman, yang lapar tidak kautawari makanan.
\par 8 Kaupakai jabatan dan kuasa untuk menyita tanah seluruhnya.
\par 9 Bukan saja kau tidak menolong para janda, tetapi yatim piatu kautindas pula.
\par 10 Karena itu di sekitarmu, kini penuh jebakan, dan dengan tiba-tiba hatimu diliputi ketakutan.
\par 11 Hari semakin gelap, tak dapat engkau melihat; engkau tenggelam dilanda banjir yang dahsyat.
\par 12 Bukankah Allah mendiami langit yang tertinggi, dan memandang ke bawah, ke bintang-bintang yang tinggi sekali?
\par 13 Namun engkau bertanya, "Tahu apa Dia? Ia ada di balik awan dan tak dapat mengadili kita."
\par 14 Engkau menyangka bahwa pandangan-Nya tertutup awan dan bahwa hanya pada batas antara langit dan bumi Ia berjalan?
\par 15 Apakah engkau tetap hendak lewat di jalan yang dipilih orang-orang jahat?
\par 16 Mereka direnggut sebelum tiba saat kematiannya, dan dihanyutkan oleh banjir yang melanda.
\par 17 Mereka itulah yang berani menolak Yang Mahakuasa, dan mengira Ia tak dapat berbuat apa-apa kepada mereka.
\par 18 Padahal Allah yang telah menjadikan mereka kaya! Sungguh aku tak mengerti pikiran orang durjana!
\par 19 Orang yang baik, tertawa penuh kegembiraan, bila melihat orang jahat mendapat hukuman.
\par 20 Segala milik orang jahat telah hancur binasa, dan api membakar habis apa yang masih tersisa.
\par 21 Nah, Ayub, berdamailah dengan TUHAN, supaya engkau mendapat ketentraman. Kalau itu kaulakukan, pasti engkau mendapat keuntungan.
\par 22 Terimalah apa yang diajarkan TUHAN kepadamu; simpanlah itu semua di dalam hatimu.
\par 23 Kembalilah kepada TUHAN dengan rendah hati kejahatan di rumahmu hendaknya kauakhiri.
\par 24 Buanglah emasmu yang paling murni; lemparlah ke dasar sungai yang tidak berair lagi.
\par 25 Biarlah Yang Mahakuasa menjadi emasmu, dan perakmu yang sangat bermutu.
\par 26 Maka kau boleh percaya kepada Allah selalu, dan mengetahui bahwa Dia sumber bahagiamu.
\par 27 Bila engkau berdoa, Ia akan menjawabmu, dan engkau dapat menepati segala janjimu.
\par 28 Usahamu akan berhasil selalu, dan terang akan menyinari hidupmu.
\par 29 Orang yang sombong direndahkan TUHAN, tetapi yang rendah hati diselamatkan.
\par 30 Allah akan menolongmu jika kau tidak bersalah, dan jika kau melakukan kehendak-Nya."

\chapter{23}

\par 1 Tetapi Ayub menjawab, "Aku meronta dan mengeluh terhadap Allah; tak dapat aku menahan keluh kesah.
\par 3 Ah, kiranya kuketahui tempat Ia berada, supaya aku dapat pergi dan bertemu dengan Dia.
\par 4 Maka kepada-Nya perkaraku ini kuhadapkan, dari mulutku berderai kata-kata pembelaan.
\par 5 Aku ingin tahu apa yang akan Ia katakan, dan bagaimana Ia memberi jawaban.
\par 6 Apakah Ia akan melancarkan kuasa-Nya kepadaku? Tidak! Ia pasti akan mendengarkan kata-kataku.
\par 7 Aku tak bersalah dan dapat membela diri di hadapan-Nya, maka aku akan dinyatakan bebas untuk selama-lamanya.
\par 8 Kucari Allah di timur, barat, selatan, utara, tetapi di mana-mana Allah tak ada; dan aku tak dapat menemukan Dia.
\par 10 Namun Dia tahu segala jalanku juga setiap langkahku. Kalau seperti emas aku diuji, akan terbukti bahwa hatiku murni.
\par 11 Aku taat kepada-Nya dengan setia; tak pernah aku menyimpang dari jalan yang ditentukan-Nya.
\par 12 Perintah-perintah Allah selalu kutaati, kehendak-Nya kuikuti, dan bukan keinginanku sendiri.
\par 13 Allah itu tak berubah; tak ada yang dapat melawan Dia. Ia melakukan apa yang dikehendaki-Nya.
\par 14 Ia akan menjalankan rencana-Nya bagiku, dan masih banyak lagi rencana-Nya selain itu.
\par 15 Karena takut kepada-Nya, gemetarlah aku; semakin semua itu kupikirkan, semakin takutlah aku.
\par 16 Yang Mahakuasa menghancurkan segala keberanianku. Aku takut karena Allah, dan bukan karena gelap gulita, meskipun kegelapan itu mengelilingi aku, dan menutupi wajahku.

\chapter{24}

\par 1 Mengapa Allah tak menetapkan hari penghakiman, supaya orang-orang yang mengenal-Nya mendapat keadilan?
\par 2 Ada orang-orang yang menggeser tanda batas, supaya tanahnya menjadi luas. Mereka mencuri kawanan domba, dan mengandangnya di kandang domba mereka.
\par 3 Keledai milik yatim piatu, mereka larikan; sapi seorang janda, mereka sita sebagai jaminan.
\par 4 Mereka menghalangi orang miskin mendapat haknya, maka terpaksa bersembunyilah orang yang papa.
\par 5 Jadi, si miskin itu seperti keledai liar yang mencari makan di padang belukar; hanya di tempat-tempat itu saja ada makanan untuk anak-anaknya.
\par 6 Ladang orang lain terpaksa ia kerjakan; buah anggur orang jahat harus ia kumpulkan.
\par 7 Ia tak punya selimut dan pakaian penghangat tubuh di waktu malam.
\par 8 Ia basah oleh hujan lebat di gunung, lalu merapat pada gunung batu untuk berlindung.
\par 9 Orang jahat memperbudak anak yang tak beribu bapa, dan mengambil bayi orang miskin yang berhutang kepadanya.
\par 10 Orang miskin pergi tanpa sandang; ia lapar selagi ia menuai gandum di ladang.
\par 11 Dari zaitun ia membuat minyak dan dari buah anggur, minuman, tetapi ia sendiri sangat kehausan.
\par 12 Di kota-kota terdengar rintihan orang sekarat, orang-orang luka berseru minta dirawat, tetapi Allah tak mendengarkan doa mereka; Ia tak mau mengindahkannya.
\par 13 Ada orang-orang yang menolak terang dan memusuhinya, mereka tak mengenal dan tak mengikuti jalannya.
\par 14 Di waktu subuh si pembunuh bangun dari tidurnya, lalu keluar membunuh orang yang papa, dan selagi hari belum pagi, ia mengendap-endap seperti pencuri.
\par 15 Si pezinah menunggu datangnya senja; dipakainya tudung muka agar orang tak mengenalnya.
\par 16 Pencuri membongkar rumah pada malam hari; di waktu siang ia menghindari terang dan bersembunyi.
\par 17 Baginya, pagi sangat menakutkan, tapi gelap yang dahsyat, menyenangkan."
\par 18 Lalu kata Zofar, "Orang jahat hanyut oleh air bah, tanah miliknya terkutuk oleh Allah; kebun anggurnya kini sepi; tak ada yang bekerja di situ lagi.
\par 19 Seperti salju lenyap kena kemarau dan matahari, demikianlah orang berdosa ditelan ke dalam dunia orang mati.
\par 20 Ibunya sendiri melupakan dia, dan cacing-cacing makan tubuhnya. Namanya tak akan lagi dikenang; ia dimusnahkan seperti pohon yang tumbang.
\par 21 Semua itu terjadi karena ia menindas para janda, dan berlaku kejam kepada ibu yang tak berputra.
\par 22 Tapi Allah, dengan kuasa-Nya, menghalau orang perkasa. Allah bertindak, maka matilah orang durhaka.
\par 23 Allah memberi dia hidup sentosa, tetapi mengawasinya tak henti-hentinya.
\par 24 Hanya sebentar ia hidup bahagia, tapi kemudian pergi untuk selama-lamanya. Ia layu seperti rumput yang tak berguna; seperti bulir padi yang dipotong dari batangnya.
\par 25 Siapakah dapat menyangkal kenyataan itu, atau menyanggah kebenaran perkataanku?"

\chapter{25}

\par 1 Lalu Bildad menjawab, "Allah itu sangat berkuasa; semua orang harus gentar di hadapan-Nya; Dialah yang memelihara kedamaian dalam kerajaan-Nya di surga.
\par 3 Dapatkah dihitung malaikat yang melayani-Nya? Adakah tempat yang tidak disinari oleh terang-Nya?
\par 4 Mungkinkah manusia suci di mata Allah? Mungkinkah ia murni pada pemandangan-Nya?
\par 5 Bagi Allah, bahkan bulan pun tidak terang, dan bintang dianggapnya suram.
\par 6 Apalagi manusia, si cacing, si serangga! Di mata Allah, ia sungguh tak berharga."

\chapter{26}

\par 1 Tetapi Ayub berkata, "Alangkah mahirnya kauberi pertolongan kepadaku orang yang lemah dan kepayahan!
\par 3 Alangkah baiknya nasihat dan ajaran itu yang telah kauberikan kepadaku, orang yang dungu!
\par 4 Kepada siapakah tuturmu itu tertuju? Siapa mengilhamimu untuk bicara seperti itu?"
\par 5 Jawab Bildad, "Orang-orang di alam maut gemetar; air dan penghuninya bergeletar.
\par 6 Di hadapan Allah, dunia orang mati terbuka, tak bertutup sehingga kelihatan oleh-Nya.
\par 7 Allah membentangkan langit, di atas samudra, dan menggantungkan bumi pada ruang hampa.
\par 8 Dimuati-Nya awan dengan air berlimpah-limpah, namun awan itu tidak robek karena beratnya.
\par 9 Disembunyikan-Nya wajah bulan purnama di balik awan yang telah dibentangkan-Nya.
\par 10 Digambar-Nya lingkaran pada muka lautan untuk memisahkan terang dari kegelapan.
\par 11 Bila Ia menghardik dengan suara menggelegar, tiang-tiang penyangga langit gemetar.
\par 12 Samudra ditaklukkan oleh kuasa-Nya dan Rahab pun dihajar oleh kemahiran-Nya.
\par 13 Napas-Nya menyapu langit hingga cerah sekali; tangan-Nya membunuh naga yang nyaris lari.
\par 14 Tetapi semua itu hanya pertanda kuasa-Nya; hanya bisikan yang sampai di telinga kita. Betapa sedikit pengertian kita tentang Allah dan hebatnya kuasa-Nya!"

\chapter{27}

\par 1 Ayub meneruskan uraiannya, katanya, "Demi Allah yang hidup, yang tak memberi keadilan kepadaku,
\par 3 aku bersumpah: Selama Allah masih memberi napas kepadaku, selama nyawa masih ada dalam badanku,
\par 4 bibirku tak akan menyebut kata dusta, lidahku tak akan mengucapkan tipu daya.
\par 5 Jadi, tak mau aku mengatakan bahwa kamu benar; sampai mati pun kupertahankan bahwa aku tak cemar.
\par 6 Aku tetap berpegang kepada kepatuhanku, dan hati nuraniku pun bersih selalu.
\par 7 Semoga musuhku dihukum sebagai pendurhaka, dan lawanku dihajar sebagai orang durjana.
\par 8 Adakah harapan bagi orang dursila pada saat Allah menuntut jiwanya?
\par 9 Apakah Allah akan mendengar tangisnya bilamana kesulitan menimpa dia?
\par 10 Seharusnya ia merindukan kesenangan dari Allah, dan berdoa kepada-Nya tanpa merasa lelah.
\par 11 Kamu akan kuajari tentang besarnya kuasa Allah, kuberitahukan kepadamu rencana Yang Mahakuasa.
\par 12 Tetapi, kamu semua telah melihatnya sendiri. Jadi, mengapa kamu berikan nasihat yang tak berarti?"
\par 13 Maka berkatalah Zofar, "Beginilah caranya Allah Yang Mahakuasa menghukum orang yang lalim dan durhaka.
\par 14 Jika anaknya banyak, mereka akan mati dalam perang dan anak cucunya akan hidup berkekurangan.
\par 15 Sanaknya yang masih ada, mati karena wabah, dan janda-jandanya tidak menangisi mereka.
\par 16 Boleh saja peraknya bertimbun-timbun dan pakaiannya bersusun-susun,
\par 17 tetapi perak dan pakaian itu semua akan menjadi milik orang yang tulus hatinya.
\par 18 Rumah orang jahat rapuh seperti sarang laba-laba, hanya rumah sementara seperti gubug seorang penjaga.
\par 19 Ia membaringkan diri sebagai orang kaya, tetapi ia tak dapat mengulanginya, ketika ia bangun dari tidurnya, sudah hilang lenyaplah kekayaannya.
\par 20 Kedahsyatan menimpa seperti air bah yang datang tiba-tiba. Angin ribut di malam hari meniup dan menyeret dia pergi.
\par 21 Angin timur mengangkat dia, dan menyapunya dari rumahnya.
\par 22 Ia dilanda tanpa kasihan, dan terpaksa lari mencari perlindungan.
\par 23 Jatuhnya disambut orang dengan tepuk tangan; di mana-mana ia mendapat penghinaan."

\chapter{28}

\par 1 Ada pertambangan di mana perak ditemukan; ada tempat di mana emas dimurnikan.
\par 2 Besi digali dari dalam tanah; dari batu dilelehkan tembaga.
\par 3 Gelap yang pekat ditembusi, tempat yang paling dalam diselidiki. Di situ, di dalam kegelapan, orang mencari batu-batuan.
\par 4 Jauh di tempat yang tak ada penghuni, yang belum pernah diinjak dan dilalui, orang bekerja sambil bergantungan pada tali di dalam terowongan yang sunyi sepi.
\par 5 Tanah menghasilkan pangan bagi manusia, tapi di bawah tanah itu juga, semua dibongkarbalikkan sehingga isi bumi berantakan.
\par 6 Batu di dalam tanah mengandung nilakandi, dan debunya berisikan emas murni.
\par 7 Burung elang tak kenal jalan ke sana, dan burung nasar pun belum pernah terbang di atasnya.
\par 8 Belum pernah singa maupun binatang buas lainnya melalui jalan sepi yang menuju ke sana.
\par 9 Orang menggali dalam batu yang betapa pun kerasnya, dibongkarnya gunung sampai pada akarnya.
\par 10 Ketika ia membuat tembusan di dalam gunung batu, didapatinya permata yang sangat bermutu.
\par 11 Sampai kepada sumber sungai-sungai ia menggali, lalu menyingkapkan apa yang tersembunyi.
\par 12 Tetapi di manakah hikmat dapat dicari? Di manakah kita dapat belajar agar mengerti?
\par 13 Hikmat tidak ada di tengah-tengah manusia; tak ada yang tahu nilainya yang sesungguhnya.
\par 14 Dasar-dasar laut dan samudra berkata bahwa hikmat tidak ada padanya.
\par 15 Hikmat tak dapat ditukar walau dengan emas murni, dan dengan perak pun tak dapat dibeli.
\par 16 Emas dan permata yang paling berharga tidak dapat mengimbangi nilainya.
\par 17 Emas atau kaca halus tak dapat berbanding dengannya, tak dapat dibayar dengan jambangan kencana.
\par 18 Hikmat jauh lebih tinggi nilainya daripada merjan, kristal, atau mutiara.
\par 19 Batu topas yang asli dan emas yang murni, kurang nilainya dari akal budi.
\par 20 Di manakah sumbernya kebijaksanaan? Di mana kita mendapat pengertian?
\par 21 Tak ada makhluk hidup yang pernah melihatnya, bahkan burung di udara tak menampaknya.
\par 22 Maut dan kebinasaan pun berkata, mereka hanya mendengar desas-desus belaka.
\par 23 Hanya Allah tahu tempat hikmat berada, hanya Dia mengetahui jalan ke sana,
\par 24 karena Ia melihat ujung-ujung bumi; segala sesuatu di bawah langit Ia amati.
\par 25 Ketika angin diberi-Nya kekuatan, dan ditetapkan-Nya batas-batas lautan;
\par 26 ketika ditentukan-Nya tempat hujan jatuh, dan jalan yang dilalui kilat dan guruh;
\par 27 pada waktu itulah hikmat dilihat-Nya, diuji-Nya nilainya, lalu diberikan-Nya restu-Nya.
\par 28 Allah berkata kepada manusia, "Untuk mendapat hikmat, Allah harus kamu hormati. Untuk dapat mengerti, kejahatan harus kamu jauhi."

\chapter{29}

\par 1 Ayub melanjutkan uraiannya, katanya,
\par 2 "Kiranya hidupku dapat lagi seperti dahulu, waktu Allah melindungi aku.
\par 3 Aku selalu diberi-Nya pertolongan, diterangi-Nya waktu berjalan dalam kegelapan.
\par 4 Itulah hari-hari kejayaanku, ketika keakraban Allah menaungi rumahku.
\par 5 Waktu itu, Yang Mahakuasa masih mendampingi aku, dan anak-anakku ada di sekelilingku.
\par 6 Ternakku menghasilkan banyak sekali susu. Banyak minyak dihasilkan oleh pohon-pohon zaitunku, meskipun ditanam di tanah berbatu.
\par 7 Jika para tua-tua kota duduk bersama, dan kuambil tempatku di antara mereka,
\par 8 minggirlah orang-orang muda, segera setelah aku dilihat mereka. Juga orang-orang tua bangkit dengan khidmat; untuk memberi hormat.
\par 9 Bahkan para pembesar berhenti berkata-kata,
\par 10 dan orang penting pun tidak berbicara.
\par 11 Siapa pun kagum jika mendengar tentang aku; siapa yang melihat aku, memuji jasaku.
\par 12 Sebab, kutolong orang miskin yang minta bantuan; kusokong yatim piatu yang tak punya penunjang.
\par 13 Aku dipuji oleh orang yang sangat kesusahan, kutolong para janda sehingga mereka tentram.
\par 14 Tindakanku jujur tanpa cela; kutegakkan keadilan senantiasa.
\par 15 Bagi orang buta, aku menjadi mata; bagi orang lumpuh, aku adalah kakinya.
\par 16 Bagi orang miskin, aku menjadi ayah; bagi orang asing, aku menjadi pembela.
\par 17 Tapi kuasa orang kejam, kupatahkan, dan kurban mereka kuselamatkan.
\par 18 Harapanku ialah mencapai umur yang tinggi, dan mati dengan tenang di rumahku sendiri.
\par 19 Aku seperti pohon yang subur tumbuhnya, akarnya cukup air dan embun membasahi dahannya.
\par 20 Aku selalu dipuji semua orang, dan tak pernah kekuatanku berkurang.
\par 21 Orang-orang diam, jika aku memberi nasihat; segala perkataanku mereka dengarkan dengan cermat.
\par 22 Sehabis aku bicara, tak ada lagi yang perlu ditambahkan; perkataan meresap seperti tetesan air hujan.
\par 23 Semua orang menyambut kata-kataku dengan gembira, seperti petani menyambut hujan di musim bunga.
\par 24 Kutersenyum kepada mereka ketika mereka putus asa; air mukaku yang bahagia menambah semangat mereka.
\par 25 Akulah yang memegang pimpinan, dan mengambil segala keputusan. Kupimpin mereka seperti raja di tengah pasukannya, dan kuhibur mereka dalam kesedihannya.

\chapter{30}

\par 1 Tetapi kini aku diejek oleh orang yang lebih muda. Dahulu ayah mereka kupandang terlalu hina untuk menjaga dombaku bersama anjing gembala.
\par 2 Bagiku mereka tidak berguna karena sudah kehabisan tenaga.
\par 3 Mereka lapar dan menderita sekali, sehingga makan akar kering di gurun yang sunyi.
\par 4 Mereka mencabut belukar di padang belantara lalu memakan baik daun maupun akarnya.
\par 5 Mereka diusir dengan tengking seperti orang mengusir maling.
\par 6 Mereka tinggal di dalam gua-gua; lubang-lubang di dinding gunung menjadi rumah mereka.
\par 7 Di rimba mereka meraung-raung seperti binatang, berkelompok di bawah semak belukar di hutan.
\par 8 Mereka tak bernama dan tak berharga, orang-orang yang sudah dihalau dari negerinya.
\par 9 Sekarang mereka datang dan aku ditertawakannya; bagi mereka, aku ini lelucon belaka.
\par 10 Aku dipandang oleh mereka hina dan keji, bahkan mukaku mereka ludahi.
\par 11 Karena Allah membuat aku lemah tidak berdaya, mereka melampiaskan amukan mereka.
\par 12 Gerombolan itu menyerang aku dari depan, dan kejatuhanku mereka rencanakan.
\par 13 Mereka memotong jalanku untuk membinasakan aku; tak seorang pun menghalangi ketika mereka menyerbu.
\par 14 Bagaikan banjir mereka dobrak tembok pertahananku; beramai-ramai mereka datang menindih tubuhku.
\par 15 Kedahsyatan meliputi diriku; bagaikan hembusan angin, harga diriku berlalu; bagaikan awan lewat, hilanglah kebahagiaanku.
\par 16 Sekarang hampir matilah aku; tak ada keringanan bagi deritaku.
\par 17 Pada waktu malam semua tulangku nyeri; rasa sakit yang menusuk tak kunjung berhenti.
\par 18 Allah mencengkeram aku pada leher bajuku sehingga pakaianku menggelambir pada tubuhku.
\par 19 Ke dalam lumpur aku dihempaskan-Nya, aku menjadi seperti sampah saja!
\par 20 Aku berseru kepada-Mu, ya Allah, Kau tak memberi jawaban; bila aku berdoa, Kau tak memperhatikan.
\par 21 Engkau berlaku kejam terhadapku, Kautindas aku dengan seluruh kekuatan-Mu.
\par 22 Engkau membiarkan angin melayangkan aku; dalam angin ribut Kauombang-ambingkan diriku.
\par 23 Aku tahu, Kaubawa aku kepada alam kematian, tempat semua yang hidup dikumpulkan.
\par 24 Mengapa Kau menyerang orang yang celaka, yang tak dapat berbuat apa pun kecuali mohon iba?
\par 25 Bukankah aku menangis bersama orang yang kesusahan, dan mengasihani orang yang berkekurangan?
\par 26 Aku mengharapkan bahagia dan terang, tapi kesukaran dan kegelapanlah yang datang.
\par 27 Aku terkoyak oleh duka dan nestapa; hari demi hari makin banyak yang kuderita.
\par 28 Di dalam kelam, tanpa cahaya, aku berkeliaran; aku berdiri di muka umum, minta pertolongan.
\par 29 Suaraku sedih penuh iba seperti tangis serigala dan burung unta.
\par 30 Kulitku menjadi hitam; tubuhku terbakar oleh demam.
\par 31 Dahulu kudengar musik gembira, kini hanya ratapan tangis belaka.

\chapter{31}

\par 1 Dengan sumpah aku telah berjanji gadis muda tak akan kupandang dengan berahi.
\par 2 Apakah yang dilakukan Allah terhadap kita? Bagaimanakah dibalas-Nya perbuatan manusia?
\par 3 Celaka dan kemalangan pasti Ia datangkan kepada orang yang melakukan kejahatan!
\par 4 Allah pasti mengetahui segala perbuatanku; dilihat-Nya segala langkahku.
\par 5 Aku bersumpah bahwa belum pernah aku bertindak curang; belum pernah pula aku menipu orang.
\par 6 Biarlah Allah menimbang aku di atas neraca yang sah, maka Ia akan tahu bahwa aku tidak bersalah.
\par 7 Andaikata aku telah menyimpang dari jalan yang benar, atau hatiku tertarik oleh hal yang cemar, jika tanganku ternoda oleh dosa,
\par 8 maka biarlah orang lain makan apa yang kutabur, dan seluruh hasil bumiku hancur.
\par 9 Seandainya pernah aku tertarik kepada istri tetanggaku, dan dengan sembunyi, kuintip dia di balik pintu,
\par 10 maka biarlah istriku memasak untuk orang lain; biarlah di ranjang lelaki lain ia berbaring.
\par 11 Jika dosa yang keji itu memang kulakukan, aku patut menerima hukuman.
\par 12 Dosa itu membinasakan seperti api neraka, segala yang kumiliki habis dibakarnya.
\par 13 Ketika hambaku mengeluh karena haknya kusalahi, kudengarkan dia dan kuperlakukan dengan tulus hati.
\par 14 Jika tidak, bagaimana harus kuhadapi Allahku? Apa jawabku pada waktu Ia datang menghakimi aku?
\par 15 Bukankah Allah yang menciptakan aku, menciptakan juga hamba-hambaku itu?
\par 16 Belum pernah aku tak mau menolong orang yang papa, atau membiarkan para janda hidup berputus asa.
\par 17 Belum pernah kubiarkan yatim piatu kelaparan, sedangkan aku sendiri cukup makanan.
\par 18 Sejak kecil mereka kupelihara; seumur hidupku kubimbing mereka.
\par 19 Jika kulihat orang yang berkekurangan, terlalu miskin untuk membeli pakaian,
\par 20 kuhangatkan dia dengan kain wol dari dombaku sendiri, maka ia akan memuji aku dengan segenap hati.
\par 21 Sekiranya pernah aku menindas yatim piatu, sebab yakin akan menang perkaraku,
\par 22 maka biarlah patah kedua lenganku sehingga terpisah dari bahuku.
\par 23 Tak akan aku berbuat begitu, sebab hukuman Allah sangat mengecutkan hatiku.
\par 24 Tidak pernah aku mengandalkan hartaku,
\par 25 atau membanggakan kekayaanku.
\par 26 Tak pernah kusembah mentari yang bersinar cerah ataupun bulan yang bercahaya indah.
\par 27 Tak pernah aku terpikat olehnya, atau kukecup tanganku untuk menghormatinya.
\par 28 Dosa semacam itu patut mendapat hukuman mati; karena Allah Yang Mahakuasa telah diingkari.
\par 29 Belum pernah aku bersenang karena musuhku menderita, atau bersukacita karena ia mendapat celaka.
\par 30 Aku tidak berdoa untuk kematian musuhku; tak pernah aku berbuat dosa semacam itu.
\par 31 Orang-orang yang bekerja padaku tahu, bahwa siapa saja kujamu di rumahku.
\par 32 Rumahku terbuka bagi orang yang bepergian; tak pernah kubiarkan mereka bermalam di jalan.
\par 33 Orang lain menyembunyikan dosanya, tetapi aku tak pernah berbuat seperti mereka.
\par 34 Pendapat umum tidak kutakuti, dan penghinaan orang, aku tak perduli. Tak pernah aku tinggal di rumah atau diam saja, hanya karena takut akan dihina.
\par 35 Tiadakah orang yang mau mendengarkan kata-kataku? Ku bersumpah bahwa benarlah semuanya itu. Kiranya Yang Mahakuasa menjawab aku. Seandainya tuduhan musuh terhadap aku ditulis semua sehingga terlihat olehku,
\par 36 maka dengan bangga akan kupasang pada bahu, dan sebagai mahkota kulekatkan di kepalaku.
\par 37 Akan kuberitahukan kepada Allah segala yang kubuat; akan kuhadapi Dia dengan bangga dan kepala terangkat.
\par 38 Seandainya tanah yang kubajak telah kucuri, dan kurampas dari pemiliknya yang sejati,
\par 39 seandainya hasilnya habis kumakan, dan petani yang menanamnya kubiarkan kelaparan,
\par 40 biarlah bukan jelai dan gandum yang tumbuh di ladang, melainkan semak berduri dan rumput ilalang." Sekianlah kata-kata Ayub.

\chapter{32}

\par 1 Karena Ayub yakin sekali akan kebenaran dirinya, maka ketiga sahabatnya itu pun tak mau menjawab dia lagi.
\par 2 Tetapi di situ ada seorang yang bernama Elihu anak Barakheel, seorang keturunan Bus dari kaum Ram. Ia tidak dapat menahan marahnya, karena Ayub membenarkan dirinya sendiri dan mempersalahkan Allah.
\par 3 Ia juga marah kepada ketiga sahabat Ayub itu karena mereka tidak dapat membantah kata-kata Ayub, meskipun mereka mempersalahkannya.
\par 4 Elihu orang yang paling muda di antara mereka, sebab itu ia menunggu sampai semuanya selesai berbicara.
\par 5 Setelah melihat bahwa ketiga orang itu tidak dapat menjawab, ia menjadi marah,
\par 6 dan berkata demikian, "Aku masih muda, sedangkan kamu sudah tua, sebab itu aku takut dan ragu mengemukakan pendapatku.
\par 7 Pikirku, kamulah yang harus berbicara, yang lebih tua harus membagikan hikmatnya.
\par 8 Tetapi yang memberi hikmat kepada manusia, hanyalah Roh Allah Yang Mahakuasa.
\par 9 Orang menjadi bijak, bukan karena lanjut umurnya; orang mengerti yang benar, bukan karena tinggi usianya.
\par 10 Sebab itu, dengarkanlah aku; izinkanlah aku mengatakan pendapatku.
\par 11 Dengan sabar aku mendengarkan ketika kamu berbicara, dan menanti ketika kamu mencari kata-kata yang bijaksana.
\par 12 Kuperhatikan dengan saksama; kudengar kamu menemui kegagalan. Kesalahan dalam kata-kata Ayub tak dapat kamu buktikan.
\par 13 Bagaimana dapat kamu katakan bahwa hikmat telah kamu temukan? Karena kamu terpaksa menyerah. Yang bisa menjawab Ayub hanyalah Allah.
\par 14 Kepadamulah Ayub berbicara, dan bukan kepadaku, tetapi aku tak akan memberi jawaban seperti kamu.
\par 15 Ayub, mereka bingung dan tak dapat memberi jawaban; tak ada yang dapat mereka katakan.
\par 16 Mereka berdiri saja, tak dapat berbicara lagi. Haruskah aku menunggu meskipun mereka berdiam diri?
\par 17 Tidak, sekarang akan kuberi jawaban; pendapatku akan kusampaikan.
\par 18 Tak sabar lagi aku menunggu. Tak dapat lagi kutahan kata-kataku.
\par 19 Jika aku diam saja, akan pecahlah aku, seperti kantong yang penuh dengan anggur baru.
\par 20 Aku harus berbicara, supaya hatiku tenang; aku harus membuka mulutku dan memberi jawaban.
\par 21 Tak akan kubela siapa pun dalam sengketa ini dan tak seorang pun akan kupuji-puji.
\par 22 Cara menyanjung-nyanjung pun, aku tidak tahu, dan seandainya aku melakukan itu, Allah akan segera menghukum aku.

\chapter{33}

\par 1 Sekarang, hai Ayub, dengarkanlah dengan teliti kata-kata yang hendak kusampaikan ini.
\par 2 Aku sudah siap sedia hendak berkata-kata.
\par 3 Dengan tulus hati aku berbicara; yang kukatakan adalah yang sebenarnya.
\par 4 Roh Allah telah menciptakan aku dan memberikan hidup kepadaku.
\par 5 Jadi, jika dapat, jawablah aku. Siapkanlah pembelaanmu.
\par 6 Bagi Allah, kau dan aku tidak berbeda dari tanah liat kita dibentuk-Nya.
\par 7 Jadi, tak usah kau takut kepadaku; aku tidak bermaksud mengalahkanmu.
\par 8 Nah, telah kudengar apa yang kaukatakan, dan aku mengerti apa yang kaumaksudkan.
\par 9 Kau berkata, 'Aku bersih, tak melakukan pelanggaran. Aku tak bercela dan tak berbuat kesalahan.
\par 10 Tetapi Allah mencari-cari alasan melawan aku, dan diperlakukan-Nya aku sebagai seteru.
\par 11 Ia mengenakan rantai pada kakiku; dan mengawasi segala gerak-gerikku.'
\par 12 Hai Ayub, pendapatmu salah belaka! Sebab Allah lebih besar daripada manusia.
\par 13 Mengapa engkau menuduh Allah bahwa Ia tak mengindahkan keluhan manusia?
\par 14 Allah berbicara dengan berbagai cara, namun tak seorang pun memperhatikan perkataan-Nya.
\par 15 Sedang orang tidur nyenyak di waktu malam, dalam mimpi dan penglihatan, Allah berbicara.
\par 16 Allah menyuruh mereka mendengarkan; dikejutkan-Nya mereka dengan teguran-teguran.
\par 17 Maksud-Nya supaya mereka berhenti berdosa dan meninggalkan kesombongan mereka.
\par 18 Tidak dibiarkan-Nya mereka mengalami kehancuran; dilindungi-Nya mereka dari kematian.
\par 19 Allah menegur orang dengan mendatangkan penyakit sehingga tubuhnya penuh rasa sakit.
\par 20 Si sakit kehilangan nafsu makan, makanan yang paling lezat pun memuakkan.
\par 21 Tubuhnya menjadi kurus merana, tulang-tulangnya kelihatan semua.
\par 22 Ia sudah hampir pulang ke alam baka dunia orang mati telah dekat kepadanya.
\par 23 Mungkin satu di antara seribu malaikat Allah yang mengingatkan manusia akan tugasnya, akan datang menolong dia.
\par 24 Dengan iba malaikat itu akan berkata, 'Lepaskanlah dia, tak boleh ia turun ke dunia orang mati. Inilah uang tebusan, agar ia bebas lagi.'
\par 25 Tubuhnya akan menjadi kuat perkasa segar seperti orang muda.
\par 26 Bila ia berdoa, Allah akan mengasihaninya, maka ia akan memuji Allah dengan gembira dan Allah akan memulihkan keadaannya.
\par 27 Maka di depan umum ia akan berkata, 'Yang jahat kuanggap baik, besarlah dosaku, namun Allah tidak menghukum aku.'
\par 28 Allah mencegah aku pergi ke dunia orang mati, sehingga aku masih hidup kini.
\par 29 Dengan berulang kali, Allah telah melakukan semua ini,
\par 30 supaya Ia dapat menyelamatkan manusia dan memberi kebahagiaan dalam hidupnya.
\par 31 Maka dengarlah Ayub, pasanglah telinga diamlah, kini akulah yang berbicara.
\par 32 Tetapi jika ada yang hendak kaukatakan, silakan bicara; dan jika engkau benar, aku akan rela mengakuinya.
\par 33 Tetapi jika tidak, diamlah dan dengarkanlah aku, aku hendak mengajarkan hikmat kepadamu.

\chapter{34}

\par 1 Bukankah kamu pandai dan berakal budi? Nah, dengarkanlah segala perkataanku ini.
\par 3 Orang tahu makanan enak bila mengecapnya, dan kata-kata bijak bila mendengarnya.
\par 4 Persoalan ini harus kita periksa lalu kita pecahkan bersama-sama.
\par 5 Kata Ayub, 'Tak ada salah padaku, tetapi Allah tak mau memberi apa yang adil kepadaku.
\par 6 Aku dianggap berdusta, karena mengatakan aku tak berdosa. Kini aku luka parah, meskipun aku tak bersalah.'
\par 7 Pernahkah kamu melihat orang seperti Ayub ini? Ia mencemooh Allah berkali-kali.
\par 8 Ia suka berkawan dengan orang-orang durhaka serta bergaul dengan orang-orang durjana.
\par 9 Ia berkata, bahwa sia-sia sajalah jika ia berusaha melakukan kehendak Allah.
\par 10 Hai orang-orang yang bijak, dengarlah! Masakan Allah Yang Mahakuasa berbuat salah?
\par 11 Ia mengganjar manusia setimpal perbuatannya, memperlakukan dia sesuai kelakuannya.
\par 12 Allah Yang Mahakuasa tidak melakukan kejahatan; semua orang diberi-Nya keadilan.
\par 13 Dari siapakah Allah menerima kuasa-Nya? Siapakah mempercayakan bumi ini kepada-Nya?
\par 14 Seandainya Allah mencabut nyawa manusia, dan mengambil kembali napas hidupnya,
\par 15 maka matilah semua makhluk yang bernyawa, dan manusia menjadi debu seperti semula.
\par 16 Nah, jika engkau arif, perhatikanlah, pasanglah telingamu dan dengarkanlah.
\par 17 Apakah Allah yang adil dan perkasa itu kaupersalahkan? Apakah pada sangkamu Allah membenci keadilan?
\par 18 Allah menghukum raja dan penguasa bila mereka jahat dan durhaka.
\par 19 Ia tidak memihak kepada para raja, atau mengutamakan orang kaya daripada orang papa. Karena mereka semua adalah ciptaan-Nya.
\par 20 Di tengah malam, mereka dapat mati dengan tiba-tiba. Allah menghukum penguasa dan mereka pun binasa; dengan mudah dibunuh-Nya orang perkasa.
\par 21 Ia mengawasi hidup manusia, dilihat-Nya segala langkahnya.
\par 22 Tak ada kegelapan, betapa pun pekatnya, yang dapat menyembunyikan orang berdosa.
\par 23 Tidak perlu Allah menentukan saatnya, manusia datang untuk diadili oleh-Nya.
\par 24 Tanpa menyelidiki dan memeriksa Ia memecat penguasa dan mengangkat penggantinya.
\par 25 Sebab, Ia tahu apa yang mereka lakukan; maka Ia menggulingkan dan menghancurkan mereka di waktu malam.
\par 26 Mereka ditampar-Nya karena dosa mereka, di depan umum mereka dihukum-Nya.
\par 27 Sebab, mereka tidak mentaati kehendak-Nya dan tak mengindahkan segala perintah-Nya.
\par 28 Mereka menyebabkan orang miskin berkeluh kesah tangisan mereka didengar Allah.
\par 29 Apabila Allah memutuskan untuk diam saja, tak seorang pun akan berani mengecam-Nya. Apabila Ia menyembunyikan wajah-Nya, tak seorang pun dapat menemukan-Nya.
\par 30 Maka bangsa-bangsa tidak berdaya untuk mencegah penindasan jahat berkuasa.
\par 31 Ayub, apakah dosamu kepada Allah telah kauakui? sudahkah kau berjanji tak akan berdosa lagi?
\par 32 Sudahkah kauminta agar ditunjukkan-Nya kesalahanmu? Sudahkah kau bersumpah menghentikan perbuatan itu?
\par 33 Masakan Allah akan mengabulkan apa yang kauingini, setelah perbuatan Allah tidak kausetujui? Kini, engkau yang harus memutuskan, bukan aku. Katakanlah apa pendapatmu!
\par 34 Semua orang arif pastilah akan menyetujuinya; orang bijak yang mendengar aku akan berkata
\par 35 bahwa Ayub berbicara tanpa mengerti, dan bahwa perkataannya tidak mengandung arti.
\par 36 Jika kata-kata Ayub ditinjau dengan cermat, akan nyatalah bahwa bicaranya seperti orang jahat.
\par 37 Kesalahannya ditambah lagi dengan sebuah dosa; ia memberontak terhadap Allah, dan menghina-Nya di hadapan kita.

\chapter{35}

\par 1 Ayub, engkau keliru jika menyangka bahwa di mata Allah engkau tak bersalah.
\par 3 Engkau khilaf jika bertanya, apa pengaruh dosamu terhadap-Nya; dan keuntungan apa yang kauterima kalau engkau tidak berbuat dosa.
\par 4 Akulah yang akan memberi jawaban kepadamu dan juga kepada teman-temanmu itu.
\par 5 Lihatlah ke langit dan perhatikan betapa tingginya awan-awan.
\par 6 Jika engkau berdosa, Allah tidak akan rugi. Jika salahmu banyak, Ia tak terpengaruhi.
\par 7 Dengan berbuat baik, Allah tidak kaubantu. Sungguh, Ia tak memerlukan apa pun darimu.
\par 8 Hanya sesamamu yang dirugikan oleh dosa-dosamu. Dia juga yang beruntung oleh kebaikanmu.
\par 9 Orang-orang yang ditindas, akan mengerang; mereka berteriak minta pertolongan.
\par 10 Tetapi bukan kepada Allah, Penciptanya yang memberi harapan di kala mereka berduka,
\par 11 dan yang memberi manusia akal budi melebihi burung dan hewan di bumi.
\par 12 Orang-orang tertindas itu berseru, tetapi Allah diam karena mereka jahat dan penuh kesombongan.
\par 13 Sia-sialah mereka berteriak sekuat tenaga; Yang Mahakuasa tidak melihat atau mendengar mereka.
\par 14 Ayub, lebih-lebih lagi kalau engkau berkata bahwa engkau tidak melihat Dia, bahwa perkaramu sudah ada di hadapan-Nya dan engkau menanti-nantikan Dia!
\par 15 Sangkamu Allah tidak memberi hukuman dan tidak memperhatikan pelanggaran.
\par 16 Sia-sialah pembicaraanmu engkau teruskan; jelaslah, engkau tak tahu apa yang kaukatakan.

\chapter{36}

\par 1 Dengarkanlah sebentar lagi, dan bersabarlah, masih ada yang hendak kukatakan demi Allah.
\par 3 Pengetahuanku luas; akan kugunakan itu untuk membuktikan bahwa adillah Penciptaku.
\par 4 Perkataanku tidak ada yang palsu; orang yang sungguh arif ada di depanmu.
\par 5 Allah itu perkasa! Segala sesuatu difahami-Nya. Tak seorang pun dipandang-Nya hina.
\par 6 Orang yang berdosa tak dibiarkan-Nya hidup lama, Ia memberi keadilan kepada orang yang menderita.
\par 7 Orang-orang jujur diperhatikan-Nya, dibuat-Nya mereka berkuasa seperti raja-raja, sehingga mereka dihormati selama-lamanya.
\par 8 Tetapi bila orang dibelenggu dengan rantai besi, menderita akibat perbuatannya sendiri,
\par 9 maka dosa dan kesombongan mereka akan disingkapkan oleh Allah.
\par 10 Disuruhnya mereka mendengarkan peringatan-Nya dan meninggalkan kejahatan mereka.
\par 11 Jika mereka menurut kepada Allah dan berbakti kepada-Nya, mereka hidup damai dan makmur sampai akhir hayatnya.
\par 12 Tetapi jika mereka tidak mendengarkan, mereka akan mati dalam kebodohan.
\par 13 Orang yang tak bertuhan menyimpan kemarahan; biar dihukum TUHAN, tak mau mereka minta bantuan.
\par 14 Mereka mati kepayahan di masa mudanya, karena hidupnya penuh hina.
\par 15 Allah mengajar manusia melalui derita, Ia memakai kesusahan untuk menyadarkannya.
\par 16 Allah telah membebaskan engkau dari kesukaran, sehingga kau dapat menikmati ketentraman, dan meja hidanganmu penuh makanan.
\par 17 Tetapi kini sesuai dengan kejahatanmu, engkau menerima hukumanmu.
\par 18 Waspadalah, jangan kau tertipu oleh uang sogokan; jangan kau disesatkan karena kekayaan.
\par 19 Sia-sia saja kau berseru minta dibantu, percuma segala tenaga dan kekuatanmu.
\par 20 Jangan kaurindukan malam gelap, saatnya bangsa-bangsa musnah dan lenyap.
\par 21 Waspadalah, jangan berpaling kepada kedurhakaan. Deritamu dimaksudkan agar kautinggalkan kejahatan.
\par 22 Ingatlah, Allah itu sungguh besar kuasa-Nya. Adakah guru sehebat Dia?
\par 23 Siapakah dapat menentukan jalan bagi-Nya atau berani menuduh-Nya berbuat salah?
\par 24 Selalu Ia dipuji karena karya-Nya, dan engkau pun patut menjunjung-Nya.
\par 25 Semua orang melihat perbuatan-Nya; tetapi tak seorang pun benar-benar memahami-Nya.
\par 26 Allah sungguh mulia, tak dapat kita menyelami-Nya ataupun menghitung jumlah tahun-Nya.
\par 27 Allah yang menarik air dari bumi menjadi awan lalu mengubahnya menjadi tetesan air hujan.
\par 28 Ia mencurahkan hujan dari mega; disiramkan-Nya ke atas umat manusia.
\par 29 Tak seorang pun mengerti gerak awan-awan serta bunyi guruh di langit tempat Allah berdiam.
\par 30 Ia menerangi seluruh langit dengan kilat, tetapi dasar laut tetap gelap pekat.
\par 31 Itulah caranya Ia menghidupi bangsa-bangsa dan memberinya makanan yang berlimpah ruah.
\par 32 Ia menangkap kilat dengan tangan-Nya dan menyuruhnya menyambar sasaran-Nya.
\par 33 Bunyi guruh menandakan bahwa badai akan melanda, ternak pun tahu angin ribut segera tiba.

\chapter{37}

\par 1 Badai membuat hatiku gentar dan jantungku berdebar-debar.
\par 2 Dengarlah suara Allah, hai kamu semua; dengarlah guruh yang keluar dari mulut-Nya.
\par 3 Ke seluruh langit, dilepaskannya kilat-Nya; dikirim-Nya petir-Nya ke ujung-ujung dunia.
\par 4 Kemudian, terdengar suara-Nya menderu, bunyi megah guntur dan guruh; dan di tengah suara yang menggelegar, petir berkilat sambar-menyambar.
\par 5 Karena perintah Allah, maka mujizat terjadi, hal-hal ajaib yang tak dapat kita fahami.
\par 6 Salju jatuh ke bumi atas perintah-Nya; hujan lebat turun atas suruhan-Nya.
\par 7 Dihentikan-Nya pekerjaan manusia, supaya mereka tahu bahwa Ia sedang bekerja.
\par 8 Juga binatang liar masuk ke dalam lubang dan gua, dan berlindung di dalam sarangnya.
\par 9 Dari selatan keluar angin taufan dan dari utara hawa dingin yang mencekam.
\par 10 Napas Allah membekukan permukaan air yang luas, mengubahnya menjadi es yang keras.
\par 11 Mega dimuati-Nya dengan air, dan awan bercahaya diterangi petir.
\par 12 Awan-awan melayang ke seluruh dunia, atas perintah Allah mereka bergerak ke mana-mana.
\par 13 Allah memberi hujan untuk membasahi tanah, atau juga untuk menghukum umat manusia; mungkin pula untuk memperlihatkan kepada mereka, betapa besar kasih-Nya yang tetap untuk selamanya.
\par 14 Diamlah sebentar, hai Ayub, dan dengarkanlah; perhatikanlah keajaiban-keajaiban Allah.
\par 15 Tahukah engkau bagaimana Allah memberi aba-aba, sehingga kilat memancar dari awan dan mega?
\par 16 Tahukah engkau bagaimana awan-awan melayang hasil keahlian Allah yang mengagumkan?
\par 17 Tidak, engkau hanya dapat mengeluh kepanasan apabila bumi dilanda oleh angin selatan.
\par 18 Dapatkah engkau seperti Allah membentangkan cakrawala dan mengeraskannya seperti logam tuangan atau kaca?
\par 19 Ajarlah kami apa yang harus kami katakan kepada Allah; tak ada yang dapat kami jelaskan, pikiran kami hampa.
\par 20 Tak mau aku memohon bicara kepada Allah jangan-jangan Ia mendapat alasan membuat aku celaka.
\par 21 Kini cahaya langit sangat terang sehingga menyilaukan mata; angin membersihkan cuaca dengan hembusannya.
\par 22 Sinar keemasan muncul di utara, dan kemuliaan Allah mengagumkan hati kita.
\par 23 Sungguh besar kuasa Allah kita; tak sanggup kita menghampiri-Nya, Ia jujur dan adil senantiasa; tak pernah Ia menindas manusia.
\par 24 Itulah sebabnya Ia patut dihormati oleh siapa saja, dan orang yang mengaku arif, tak dihiraukan-Nya."

\chapter{38}

\par 1 Kemudian dari dalam badai TUHAN berbicara kepada Ayub, demikian:
\par 2 "Siapa engkau, sehingga berani meragukan hikmat-Ku dengan kata-katamu yang bodoh dan kosong itu?
\par 3 Sekarang, hadapilah Aku sebagai laki-laki, dan jawablah pertanyaan-pertany ini.
\par 4 Sudah adakah engkau ketika bumi Kujadikan? Jika memang luas pengetahuanmu, beritahukan!
\par 5 Siapakah menentukan luasnya dunia? Siapakah membentangkan tali ukuran padanya? Tahukah engkau jawabannya?
\par 6 Bagaimanakah tiang-tiang penyangga bumi berdiri teguh? Siapa meletakkan batu penjuru dunia dengan kukuh,
\par 7 pada waktu bintang-bintang pagi bernyanyi bersama dan makhluk-makhluk surga bersorak-sorak gembira.
\par 8 Siapakah menutup pintu untuk membendung samudra ketika dari rahim bumi membual keluar airnya?
\par 9 Akulah yang menudungi laut dengan awan dan membungkusnya dengan kegelapan.
\par 10 Aku menentukan batas bagi samudra dan dengan pintu terpalang Aku membendungnya.
\par 11 Kata-Ku kepadanya, 'Inilah batasnya. Jangan kaulewati! Di sinilah ombak-ombakmu yang kuat harus berhenti.'
\par 12 Hai Ayub, pernahkah engkau barang sekali, menyuruh datang dinihari?
\par 13 Pernahkah engkau menyuruh fajar memegang bumi dan mengebaskan orang jahat dari tempat mereka bersembunyi?
\par 14 Terang siang menampakkan dengan jelas gunung dan lembah seperti cap pada tanah liat dan lipatan pada sebuah jubah.
\par 15 Terang siang terlalu cerah bagi orang tak bertuhan, dan menahan mereka melakukan kekerasan.
\par 16 Pernahkah engkau turun ke sumber laut, jauh di dasarnya? Pernahkah engkau berjalan-jalan di lantai samudra raya?
\par 17 Pernahkah orang menunjukkan kepadamu gapura di depan alam maut yang gelap gulita?
\par 18 Dapatkah engkau menduga luasnya dunia? Jawablah jika engkau mengetahuinya.
\par 19 Tahukah engkau dari mana datangnya terang, dan di mana sebenarnya sumber kegelapan?
\par 20 Dapatkah engkau menentukan batas antara gelap dan terang? atau menyuruh mereka pulang setelah datang?
\par 21 Tentu engkau dapat, karena engkau telah tua, dan ketika dunia diciptakan, engkau sudah ada!
\par 22 Pernahkah engkau mengunjungi gudang-gudang-Ku tempat salju dan hujan batu,
\par 23 yang Kusimpan untuk masa kesukaran, untuk waktu perang dan hari-hari pertempuran?
\par 24 Tahukah engkau tempat matahari berpangkal? atau dari mana angin timur berasal?
\par 25 Siapakah yang menggali saluran bagi hujan lebat dan memberi jalan bagi guruh dan kilat?
\par 26 Siapakah menurunkan hujan ke atas padang belantara, dan ke atas tanah yang tak dihuni manusia?
\par 27 Siapakah menyirami bumi yang kering dan merana sehingga rumput bertunas semua?
\par 28 Apakah hujan mempunyai ayah kandung? Siapa bapa titik-titik air embun?
\par 29 Apakah air es mempunyai ibu? Siapa melahirkan embun beku?
\par 30 Siapa mengubah air menjadi batu, dan membuat permukaan laut menjadi kaku?
\par 31 Dapatkah ikatan bintang Kartika kauberkas? atau belenggu bintang Belantik kaulepas?
\par 32 Dapatkah kaubimbing bintang-bintang setiap musimnya, dan bintang Biduk besar dan kecil, kautentukan jalannya?
\par 33 Tahukah engkau hukum-hukum di cakrawala? dan dapatkah engkau menerapkannya di dunia?
\par 34 Dapatkah engkau meneriakkan perintah kepada awan, dan menyuruhnya membanjirimu dengan hujan?
\par 35 Dan jika engkau menyuruh petir-petir bersambaran, apakah mereka datang dan berkata, "Saya, Tuan!"?
\par 36 Dari siapa burung ibis tahu kapan Sungai Nil akan menggenang? Siapa memberitahu ayam jantan bahwa hujan akan datang?
\par 37 Siapakah cukup arif untuk menghitung awan dan membalikkannya sehingga turun hujan,
\par 38 hujan yang mengeraskan debu menjadi gumpalan dan tanah menjadi berlekat-lekatan?
\par 39 Dapatkah engkau memburu mangsa untuk singa-singa dan mengenyangkan perut anak-anaknya,
\par 40 bilamana mereka berlindung di dalam gua atau mengendap di dalam sarangnya?
\par 41 Siapa membantu burung gagak yang kelaparan dan berkeliaran ke sana sini mencari pangan? Siapakah pula memberi pertolongan apabila anak-anaknya berseru kepada-Ku minta makanan?

\chapter{39}

\par 1 Tahukah engkau kapan kambing gunung dilahirkan induknya? Pernahkah kauamati rusa liar melahirkan anaknya?
\par 2 Tahukah engkau berapa lama mereka mengandung? Dapatkah saatnya beranak engkau hitung?
\par 3 Tahukah engkau kapan mereka meringkukkan tubuhnya, lalu melahirkan anak-anaknya?
\par 4 Anak-anaknya bertambah besar dan kuat di padang belantara; mereka pergi dan tak kembali kepada induknya.
\par 5 Siapa melepaskan keledai liar di hutan? Siapa membuka talinya dan membiarkan dia berkeliaran?
\par 6 Kepadanya Kuberikan gurun sebagai rumahnya dan padang-padang garam untuk tempat tinggalnya.
\par 7 Ia menjauhi kota-kota dan keramaiannya, tak ada yang dapat menjinakkan dan mempekerjakannya.
\par 8 Padang rumput di gunung tempat makanannya, dicarinya tetumbuhan yang hijau di sana.
\par 9 Apakah lembu liar mau bekerja untukmu? Maukah ia bermalam di dalam kandangmu?
\par 10 Dapatkah kauikat dia dan kaupaksa membajak untukmu? Dapatkah kausuruh dia menggaru ladangmu?
\par 11 Dapatkah kauandalkan tenaganya yang kuat, dan kauserahkan kepadanya kerjamu yang berat?
\par 12 Apakah kaupercayakan dia mengumpulkan panenmu, dan membawanya ke tempat penebahanmu?
\par 13 Betapa cepatnya kibasan sayap burung unta! Tetapi terbang burung bangau tak dapat diimbanginya.
\par 14 Burung unta bertelur di tanah, lalu ditinggalkannya; maka pasirlah yang akan memanaskan telurnya.
\par 15 Ia tak sadar bahwa mungkin orang memijaknya dan binatang-binatang liar menginjaknya.
\par 16 Ia bersikap seolah-olah anaknya bukan miliknya; ia tidak peduli jika usahanya sia-sia saja.
\par 17 Akulah yang membuat dia bodoh dan bebal, tak Kuberikan kepadanya hikmat dan akal.
\par 18 Tetapi apabila ia mulai berlari kencang, ia mengalahkan kuda serta penunggang.
\par 19 Hai Ayub, engkaukah yang memberi tenaga kepada kuda dan surai yang melambai-lambai pada tengkuknya?
\par 20 Engkaukah yang menyuruhnya melompat seperti belalang, dan dengan dengusnya menakut-nakuti orang?
\par 21 Ia menggaruk tanah di lembah dengan gembira, dan penuh semangat ia menyerbu ke medan laga.
\par 22 Ia tak tahu artinya gentar dan takut; tak ada pedang yang membuat dia surut.
\par 23 Senjata yang disandang penunggangnya gemerincing dan gemerlapan kena cahaya.
\par 24 Dengan semangat menyala-nyala kuda itu berlari; bila trompet berbunyi tak dapat ia menahan diri.
\par 25 Ia mendengus setiap kali trompet dibunyikan dari jauh tercium olehnya bau pertempuran. Didengarnya teriak para perwira ketika mereka memberi aba-aba.
\par 26 Apakah burung elang kauajar terbang bila ia menuju ke selatan dengan sayap terkembang?
\par 27 Perintahmukah yang diikuti burung rajawali sebelum ia membuat sarangnya di gunung yang tinggi?
\par 28 Di puncak gunung batu ia membangun rumahnya; ujung batu-batu yang runcing menjadi bentengnya.
\par 29 Dari situ ia mengawasi segala arah matanya mengintai mencari mangsa.
\par 30 Di mana ada yang tewas, di situlah dia, dan darah mangsa itu diminum oleh anak-anaknya.

\chapter{40}

\par 1 Hai Ayub, kautantang Aku, Allah Yang Mahakuasa; maukah engkau mengalah atau maukah engkau membantah?"
\par 3 Maka jawab Ayub kepada TUHAN, "Aku berbicara seperti orang bodoh, ya TUHAN. Jawab apakah yang dapat kuberikan? Tak ada apa-apa lagi yang hendak kukatakan.
\par 5 Sudah terlalu banyaklah yang kututurkan."
\par 6 Lalu, dari dalam badai TUHAN berbicara lagi kepada Ayub.
\par 7 Lalu TUHAN berkata kepada Ayub, "Hadapilah Aku sebagai laki-laki, dan jawablah segala pertanyaan-Ku ini.
\par 8 Apakah hendak kausangkal keadilan-Ku, dan membenarkan dirimu dengan mempersalahkan Aku?
\par 9 Apakah engkau kuat seperti Aku? Dapatkah suaramu mengguntur seperti suara-Ku?
\par 10 Hiasilah dirimu dengan kemegahan dan kebesaran, kenakanlah keagungan dan keluhuran.
\par 11 Pandanglah mereka yang congkak hatinya; luapkanlah marahmu dan rendahkanlah mereka.
\par 12 Ya, pandanglah orang yang sombong, tundukkan dia! remukkanlah orang jahat di tempatnya.
\par 13 Kuburlah mereka semua di dalam debu; kurunglah mereka di dunia orang mati.
\par 14 Maka engkau akan Kupuji karena engkau menang dengan kekuatan sendiri.
\par 15 Perhatikanlah Behemot, si binatang raksasa; seperti engkau, dia pun ciptaan-Ku juga. Rumput-rumput menjadi makanannya, seperti sapi dan lembu biasa.
\par 16 Tetapi amatilah tenaga dalam badannya dan kekuatan pada otot-ototnya!
\par 17 Ia menegakkan ekornya seperti pohon aras, otot-otot pahanya kokoh dan keras.
\par 18 Tulang-tulangnya kuat seperti tembaga, kakinya teguh bagaikan batang-batang baja.
\par 19 Di antara segala makhluk-Ku dialah yang paling menakjubkan; hanya oleh Penciptanya saja ia dapat ditaklukkan!
\par 20 Di bukit-bukit tempat binatang liar bermain-main gembira, tumbuhlah rumput yang menjadi makanannya.
\par 21 Ia berbaring di bawah belukar berduri, di antara gelagah di rawa-rawa ia bersembunyi.
\par 22 Belukar berduri menaungi dia dengan bayang-bayangnya. Pohon gandarusa di pinggir sungai meneduhi dia.
\par 23 Ia tidak gentar biarpun Sungai Yordan sangat kuat arusnya, ia tetap tenang meskipun air melanda mukanya.
\par 24 Siapakah berani membutakan matanya, lalu menangkap dia dengan menjerat moncongnya?

\chapter{41}

\par 1 Dapatkah kautangkap si buaya Lewiatan, hanya dengan sebuah pancing ikan? Dapatkah lidahnya kautambat dengan tali-tali pengikat?
\par 2 Dapatkah engkau memasang tali pada hidungnya ataupun kait besi pada rahangnya?
\par 3 Mungkinkah ia mohon padamu untuk dibebaskan? atau berunding denganmu, minta belas kasihan?
\par 4 Mungkinkah ia membuat persetujuan denganmu, dan berjanji akan selalu melayanimu?
\par 5 Mungkinkah engkau mengikatnya seperti burung peliharaan, yang menyenangkan hamba-hamba perempuan?
\par 6 Mungkinkah ia diperdagangkan oleh nelayan-nelayan dan dibagi-bagikan di antara para pedagang?
\par 7 Dapatkah kautusuk kulitnya dengan tombak bermata tiga atau kaulempari dia dengan lembing yang menembus kepalanya?
\par 8 Sentuhlah dia sekali saja, dan tak akan lagi engkau mengulanginya; pertarungan itu tak akan kaulupakan selama-lamanya.
\par 9 Setiap orang yang melihat Lewiatan, akan menjadi lemah lalu jatuh pingsan.
\par 10 Ia ganas bila dibangunkan dari tidurnya; tak seorang pun berani berdiri di hadapannya.
\par 11 Siapa yang dapat menyerangnya tanpa kena cedera? Di dunia ini tak ada yang sanggup melakukannya.
\par 12 Marilah Kuceritakan tentang anggota badan Lewiatan, tentang kekuatannya dan bentuknya yang tampan.
\par 13 Tak seorang pun dapat mengoyakkan baju luarnya atau menembus baju perang yang dipakainya.
\par 14 Siapa dapat membuka moncongnya yang kuat, berisi gigi-gigi yang dahsyat?
\par 15 Bagai perisai tersusun, itulah punggungnya terlekat rapat, seperti batu kerasnya.
\par 16 Tindih-menindih, terikat erat, sehingga angin pun tak dapat masuk menyelinap.
\par 17 Perisai itu begitu kuat bertautan sehingga tak mungkin diceraikan.
\par 18 Apabila Lewiatan bersin, berpijaran cahaya; matanya berkilau bagai terbitnya sang surya.
\par 19 Lidah api menghambur dari mulutnya; bunga api berpancaran ke mana-mana.
\par 20 Asap mengepul dari dalam hidungnya, seperti asap kayu bakar di bawah belanga.
\par 21 Napasnya menyalakan bara; nyala api keluar dari mulutnya.
\par 22 Tengkuknya demikian kuatnya, sehingga semuanya ketakutan di hadapannya.
\par 23 Tak ada tempat lemah pada kulitnya; tak mungkin pecah karena sekeras baja.
\par 24 Hatinya seteguh batu, tak kenal bimbang kokoh dan keras seperti batu gilingan.
\par 25 Bila ia bangkit, orang terkuat pun kehilangan keberanian, dibuat tak berdaya karena sangat ketakutan.
\par 26 Tak ada pedang yang dapat melukainya; tombak, panah ataupun lembing tak dapat menyakitinya.
\par 27 Besi dianggapnya sehalus rerumputan dan tembaga selunak kayu bercendawan.
\par 28 Tak ada panah yang dapat menghalau dia; batu yang dilemparkan kepadanya seolah-olah jerami saja.
\par 29 Gada dianggapnya sehelai rumput kering; ia tertawa jika orang melemparkan lembing.
\par 30 Sisik di perutnya seperti beling yang runcing ujungnya. Bagai alat penebah ia mengorek lumpur dan membelahnya.
\par 31 Laut dikocoknya sehingga menyerupai air mendidih; seperti panci pemasak minyak yang berbuih-buih.
\par 32 Ia meninggalkan bekas tapak kaki yang bercahaya, laut diubahnya menjadi buih yang putih warnanya.
\par 33 Di atas bumi tak ada tandingannya; makhluk yang tak kenal takut, itulah dia!
\par 34 Binatang yang paling megah pun dipandangnya hina; di antara segala binatang buas, dialah raja."

\chapter{42}

\par 1 Jawab Ayub kepada TUHAN, "Aku tahu, ya TUHAN, bahwa Engkau Mahakuasa; Engkau sanggup melakukan apa saja.
\par 3 Engkau bertanya mengapa aku berani meragukan hikmat-Mu padahal aku sendiri tidak tahu menahu. Aku bicara tentang hal-hal yang tidak kumengerti, tentang hal-hal yang terlalu ajaib bagiku ini.
\par 4 Engkau menyuruh aku mendengarkan ketika Engkau bicara, dan memberi jawaban bila Engkau bertanya.
\par 5 Dahulu, pengetahuanku tentang Engkau hanya kudengar dari orang saja, tetapi sekarang kukenal Engkau dengan berhadapan muka.
\par 6 Oleh sebab itu aku malu mengingat segala perkataanku dan dengan menyesal aku duduk dalam debu dan abu."
\par 7 Setelah selesai berbicara kepada Ayub, berkatalah TUHAN kepada Elifas, "Aku marah kepadamu dan kepada kedua temanmu, karena kamu tidak mengatakan yang sebenarnya tentang diri-Ku. Tetapi hamba-Ku Ayub tidaklah demikian.
\par 8 Nah, ambillah sekarang tujuh ekor sapi jantan dan tujuh ekor domba jantan, dan pergilah kepada Ayub. Kemudian persembahkanlah semuanya itu sebagai kurban bakaran untuk dirimu. Ayub akan berdoa untuk kamu dan doanya akan Kuterima. Kamu tidak akan Kupermalukan, meskipun kamu patut menerima hukuman. Kamu tidak mengatakan yang sebenarnya tentang diri-Ku, sedangkan Ayub mengatakannya."
\par 9 Elifas, Bildad dan Zofar melakukan apa yang disuruh TUHAN, dan TUHAN menerima doa Ayub.
\par 10 Kemudian, setelah Ayub berdoa bagi ketiga temannya, TUHAN membuat dia kaya kembali dan memberikan kepadanya dua kali lipat dari segala kepunyaannya dahulu.
\par 11 Semua abang dan saudara perempuan Ayub, serta semua kenalannya yang lama, datang mengunjunginya lalu berpesta bersama-sama dia di rumahnya. Mereka menyatakan turut berdukacita atas segala kesusahan yang telah didatangkan TUHAN ke atasnya, dan menghibur dia. Masing-masing di antara mereka memberikan sejumlah uang dan sebuah cincin emas kepada Ayub.
\par 12 Ayub diberkati TUHAN dengan lebih berlimpah dalam sisa hidupnya, daripada masa sebelum ia mengalami musibah. Ayub memiliki 14.000 ekor kambing domba, 6.000 ekor unta, 2.000 ekor sapi, dan 1.000 ekor keledai.
\par 13 Ia mendapat tujuh orang anak laki-laki dan tiga orang anak perempuan.
\par 14 Anak perempuannya yang pertama dinamakannya Yemima, yang kedua Kezia, dan yang bungsu Kerenhapukh.
\par 15 Di seluruh negeri tidak ada gadis secantik ketiga anak perempuan Ayub. Mereka diberi bagian dalam warisan ayahnya, sama dengan saudara-saudara mereka yang laki-laki.
\par 16 Sesudah itu Ayub masih hidup seratus empat puluh tahun lamanya, sehingga ia sempat melihat cucu-cucunya serta anak-anak mereka.
\par 17 Lalu meninggallah Ayub dalam usia yang lanjut sekali.


\end{document}