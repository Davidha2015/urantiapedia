\begin{document}

\title{Proverbs}

Pro 1:1  Inilah petuah-petuah dari Salomo putra Daud, raja Israel.
Pro 1:2  Tujuannya ialah untuk menolong orang mengetahui bagaimana menjadi bijaksana, dan tahu tata tertib hidup serta dapat memahami ungkapan-ungkapan yang mengandung arti yang dalam.
Pro 1:3  Petuah-petuah ini mengajar bagaimana orang dapat hidup dengan bijaksana, jujur, adil dan benar.
Pro 1:4  Orang yang tidak berpengalaman diajarnya sehingga mempunyai pikiran yang tajam, dan orang muda diajarnya menjadi orang yang pandai dan dapat berpikir secara dewasa.
Pro 1:5  Dengan petuah-petuah ini orang bijaksana pun akan bertambah pengetahuannya, dan orang yang telah berpendidikan akan mendapat bimbingan.
Pro 1:6  Dengan demikian mereka dapat menyelami arti yang tersembunyi di dalam petuah dan memahami ucapan-ucapan orang bijaksana serta masalah-masalah yang diajukan oleh mereka.
Pro 1:7  Untuk memperoleh ilmu sejati, pertama-tama orang harus mempunyai rasa hormat dan takut kepada TUHAN. Orang bodoh tidak menghargai hikmat dan tidak mau diajar.
Pro 1:8  Perhatikanlah apa yang dikatakan ayah ibumu kepadamu, anakku,
Pro 1:9  sebab ajaran-ajaran mereka menambah budi baikmu seperti hiasan kepala dan kalung memperindah rupamu.
Pro 1:10  Kalau orang berdosa membujuk engkau, anakku, janganlah turuti bujukan mereka.
Pro 1:11  Seandainya mereka berkata, "Ayo, mari kita mencari orang dan mengeroyok dia. Untuk iseng-iseng, mari kita menyerang orang yang tak bersalah.
Pro 1:12  Sekarang mereka hidup senang dan sehat, tapi nanti mereka akan menjadi seperti orang yang setengah mati.
Pro 1:13  Kita ambil barang-barang mereka yang berharga, supaya rumah kita penuh dengan barang rampasan.
Pro 1:14  Mari ikut! Nanti hasil curiannya kita bagi rata!"
Pro 1:15  Janganlah ikut dengan orang-orang yang demikian, anakku! Jauhilah mereka.
Pro 1:16  Mereka tidak dapat tinggal diam kalau belum berbuat jahat. Mereka ingin cepat-cepat membunuh.
Pro 1:17  Sedangkan burung pun tidak akan masuk ke dalam jaring yang dibentangkan di depan matanya,
Pro 1:18  tetapi orang-orang jahat itu malah memasang jerat untuk dirinya sendiri--jerat yang akan mencelakakan mereka.
Pro 1:19  Memang, orang yang mencari nafkah dengan memakai kekerasan akan membayarnya dengan nyawanya sendiri.
Pro 1:20  Dengarlah! Hikmat memanggil di jalan-jalan dan berteriak di lapangan-lapangan!
Pro 1:21  Ia berseru di pintu-pintu gerbang dan di tempat-tempat yang ramai:
Pro 1:22  "Hai orang-orang bebal! Sampai kapan kamu mau tetap demikian? Kapan tiba waktunya kamu berhenti meremehkan pengetahuan dan menolak pelajaran?
Pro 1:23  Dengarkanlah aku apabila aku menegurmu, maka kepadamu akan kunyatakan isi hatiku dan kuajarkan pengetahuanku.
Pro 1:24  Kamu sudah kupanggil, namun kamu menolak dan tak mau menghiraukan.
Pro 1:25  Semua nasihatku kamu abaikan dan teguranku kamu tolak.
Pro 1:26  Karena itu, kalau kamu celaka, aku akan menertawakan kamu. Apabila kamu ketakutan, aku akan mengejek kamu.
Pro 1:27  Ya, aku akan mengejek kamu apabila kamu cemas dan menderita karena ditimpa bencana yang datang seperti badai.
Pro 1:28  Pada waktu itu kamu akan memanggil aku, tetapi aku tak akan menyahut. Kamu akan mencari aku ke mana-mana tetapi tak akan menemukan aku.
Pro 1:29  Kamu seperti orang yang tak pernah suka mendapat pengetahuan, dan enggan mentaati TUHAN.
Pro 1:30  Kamu tidak pernah mau menerima nasihat-nasihatku atau memperhatikan teguran-teguranku.
Pro 1:31  Karena itu, kamu akan merasakan akibat dari perbuatan-perbuatanmu dan rencana-rencanamu yang buruk.
Pro 1:32  Orang yang tak berpengalaman akan mati karena mengabaikan aku, dan orang bodoh akan hancur karena tak menghiraukan aku.
Pro 1:33  Tetapi orang yang mendengarkan aku akan terpelihara. Ia hidup dengan aman dan tak perlu takut."
Pro 2:1  Terimalah ajaran-ajaranku, anakku, dan ingatlah selalu akan nasihat-nasihatku kepadamu.
Pro 2:2  Perhatikanlah apa yang bijaksana dan berusahalah memahaminya.
Pro 2:3  Ya, anakku, berusahalah untuk mempunyai pikiran yang tajam dan mintalah pengertian.
Pro 2:4  Carilah itu seperti mencari emas, dan kejarlah itu seperti mengejar harta yang terpendam.
Pro 2:5  Dengan demikian kau akan tahu apa artinya takut akan TUHAN dan kau akan mendapat pengetahuan tentang Allah.
Pro 2:6  Tuhanlah yang memberikan hikmat; dari Dialah manusia mendapat pengetahuan dan pengertian.
Pro 2:7  Kepada orang yang tulus dan tak bercela, diberikan-Nya pertolongan dan perlindungan.
Pro 2:8  TUHAN menjaga orang-orang yang berlaku adil, dan melindungi mereka yang mencintai Dia.
Pro 2:9  Kalau engkau menuruti aku, anakku, engkau akan tahu apa yang adil, jujur dan baik. Kau akan tahu juga bagaimana caranya kau harus hidup.
Pro 2:10  Kau akan menjadi bijaksana, dan pengetahuanmu akan menyenangkan hatimu.
Pro 2:11  Pengertian dan kecerdasanmu akan melindungimu,
Pro 2:12  serta mencegah engkau mengikuti cara hidup yang tidak baik, dan juga menjauhkan dirimu dari orang-orang yang bermulut jahat.
Pro 2:13  Mereka tak mau mengikuti cara hidup yang baik; mereka mengambil jalan yang gelap dan penuh dosa.
Pro 2:14  Mereka mendapatkan kesenangan dari perbuatan mereka yang jahat.
Pro 2:15  Mereka curang dan cara hidup mereka serong.
Pro 2:16  Engkau, anakku, akan bisa menolak bujukan perempuan nakal yang berusaha memikat engkau dengan kata-kata yang manis.
Pro 2:17  Wanita itu tidak setia kepada suaminya dan telah melupakan janjinya kepada Allah.
Pro 2:18  Kalau engkau ke rumahnya berarti engkau menuju kematian. Pergi ke sana sama saja dengan pergi ke dunia orang mati.
Pro 2:19  Orang yang pergi kepadanya tidak pernah ada yang kembali ke jalan yang menuju kehidupan.
Pro 2:20  Karena itu, anakku, ikutilah teladan orang baik, dan hiduplah menurut kemauan Allah.
Pro 2:21  Sebab, orang yang hidup menurut kemauan Allah, yaitu orang yang tulus hatinya, merekalah yang akan tinggal di negeri yang dijanjikan oleh TUHAN.
Pro 2:22  Tetapi orang jahat dan berdosa akan dilempar keluar oleh Allah dari negeri itu, seperti rumput dicabut dari tanah.
Pro 3:1  Janganlah lupa akan apa yang telah kuajarkan kepadamu, anakku. Ingatlah selalu akan perintahku,
Pro 3:2  supaya panjang umurmu dan sejahtera hidupmu.
Pro 3:3  Hendaklah engkau tetap percaya dan setia kepada Allah dan sesamamu. Ingatlah itu dan simpanlah di dalam hatimu,
Pro 3:4  supaya engkau disenangi dan dihargai oleh Allah dan manusia.
Pro 3:5  Percayalah kepada TUHAN dengan sepenuh hatimu, dan janganlah mengandalkan pengertianmu sendiri.
Pro 3:6  Ingatlah pada TUHAN dalam segala sesuatu yang kaulakukan, maka Ia akan menunjukkan kepadamu cara hidup yang baik.
Pro 3:7  Janganlah menganggap dirimu lebih pandai daripada yang sebenarnya; taatilah TUHAN dan jauhilah yang jahat.
Pro 3:8  Perbuatanmu itu akan menjadi seperti obat bagimu yang menyembuhkan badanmu dan menyegarkan batinmu.
Pro 3:9  Hormatilah TUHAN dengan mempersembahkan kepada-Nya yang terbaik dari segala harta milik dan hasil tanahmu,
Pro 3:10  maka lumbung-lumbungmu akan penuh gandum, dan air anggurmu akan berlimpah-limpah sehingga tidak cukup tempat untuk menyimpannya.
Pro 3:11  Apabila TUHAN menghajar engkau, anakku, terimalah itu sebagai suatu peringatan, dan jangan hatimu kesal terhadap didikan-Nya itu.
Pro 3:12  TUHAN menghajar orang yang dicintai-Nya, sama seperti seorang ayah menghajar anak yang disayanginya.
Pro 3:13  Beruntunglah orang yang menjadi bijaksana dan mendapat pengertian.
Pro 3:14  Keuntungannya lebih besar daripada yang diperoleh dari perak, dan lebih berharga dari emas.
Pro 3:15  Hikmat lebih berharga daripada batu permata; semua yang kauidamkan tak dapat menyamainya.
Pro 3:16  Hikmat memberikan kepadamu umur panjang, kekayaan dan kehormatan.
Pro 3:17  Hikmat membuat hidupmu senang dan sejahtera.
Pro 3:18  Orang yang berpegang teguh pada hikmat akan mengalami hidup yang sejati dan bahagia.
Pro 3:19  Dengan hikmat, TUHAN menciptakan bumi; dengan akal budi-Nya Ia membentangkan langit di tempat-Nya.
Pro 3:20  Dengan pengetahuan-Nya Ia membuat sumber-sumber air di bawah tanah pecah dan mengalirkan airnya serta awan di langit mencurahkan air ke bumi.
Pro 3:21  Sebab itu, berpeganglah pada hikmat dan pada pertimbangan yang matang, anakku! Jangan sekali-kali melepaskannya,
Pro 3:22  maka hidupmu akan terpelihara--indah dan menyenangkan.
Pro 3:23  Engkau akan berjalan dengan aman, dan tidak akan tersandung.
Pro 3:24  Engkau akan pergi tidur tanpa merasa takut, dan engkau tidur nyenyak sepanjang malam.
Pro 3:25  Tak perlu engkau takut akan bencana yang datang tiba-tiba seperti badai, dan melanda orang-orang jahat.
Pro 3:26  TUHAN akan menjaga engkau. Ia tidak akan membiarkan engkau terperosok ke dalam perangkap.
Pro 3:27  Jika kau mempunyai kemampuan untuk berbuat baik kepada orang yang memerlukan kebaikanmu, janganlah menolak untuk melakukan hal itu.
Pro 3:28  Janganlah menyuruh sesamamu menunggu sampai besok, kalau pada saat ini juga engkau dapat menolongnya.
Pro 3:29  Janganlah merencanakan sesuatu yang merugikan sesamamu yang tinggal di dekatmu dan mempercayaimu.
Pro 3:30  Jangan bertengkar tanpa sebab dengan seseorang yang tak pernah berbuat jahat kepadamu.
Pro 3:31  Jangan iri terhadap orang yang menggunakan kekerasan, dan jangan meniru tingkah laku mereka.
Pro 3:32  Sebab, TUHAN membenci orang yang berbuat jahat, tetapi Ia akrab dengan orang yang lurus hidupnya.
Pro 3:33  TUHAN mengutuk rumah orang jahat, tetapi memberkati rumah orang yang taat kepada-Nya.
Pro 3:34  TUHAN membenci orang yang tinggi hati, tetapi memberkati orang yang rendah hati.
Pro 3:35  Orang bijaksana akan bertambah harum namanya, sedangkan orang bodoh semakin tercela.
Pro 4:1  Hai anak-anak, dengarkanlah nasihat ayahmu! Perhatikanlah itu, maka engkau akan menjadi arif.
Pro 4:2  Yang kuajarkan kepadamu ini baik, sebab itu janganlah kau meremehkannya.
Pro 4:3  Ketika aku masih kecil, anak tunggal orang tuaku,
Pro 4:4  aku diajar oleh ayahku. Ia berkata, "Ingatlah akan nasihat-nasihatku, janganlah sekali-kali kau melupakannya. Jalankanlah petunjuk-petunjukku, supaya hidupmu bahagia.
Pro 4:5  Jadilah bijaksana dan cerdas! Ingatlah selalu akan nasihat-nasihatku dan janganlah membuangnya."
Pro 4:6  Hargailah hikmat, maka hikmat akan melindungimu; cintailah dia maka ia akan menjaga engkau agar tetap aman.
Pro 4:7  Hal terpenting yang harus pertama-tama kaulakukan ialah berusaha menjadi bijaksana. Apa pun yang kaukejar, yang terutama ialah berusahalah untuk mendapat pengertian.
Pro 4:8  Junjunglah hikmat, maka engkau akan ditinggikan olehnya. Rangkullah dia, maka ia akan mendatangkan kehormatan kepadamu.
Pro 4:9  Ia akan memberikan kepadamu karangan bunga yang elok untuk menjadi mahkotamu.
Pro 4:10  Dengarkan aku, anakku! Perhatikanlah baik-baik nasihat-nasihatku, maka umurmu akan panjang.
Pro 4:11  Aku sudah mengajarkan hikmat kepadamu dan menunjukkan cara hidup yang benar.
Pro 4:12  Kalau engkau hidup demikian, maka engkau tidak akan terhalang pada waktu berjalan, dan tak akan tersandung pada waktu berlari.
Pro 4:13  Ingatlah selalu akan ajaran yang sudah kauterima daripadaku. Jagalah itu baik-baik, sebab dengan ajaran itu hidupmu akan berhasil.
Pro 4:14  Jangan menuruti cara hidup orang jahat, dan jangan meniru perbuatan mereka.
Pro 4:15  Janganlah menaruh perhatianmu kepada mereka. Jauhilah mereka dan jalanlah terus!
Pro 4:16  Orang jahat tidak dapat tidur sebelum melakukan yang tidak baik. Mereka tidak mengantuk sebelum mencelakakan orang lain.
Pro 4:17  Kejahatan dan kekejaman adalah seperti makanan dan minuman bagi mereka.
Pro 4:18  Jalan orang jahat gelap seperti kelamnya malam. Mereka tersandung dan jatuh tanpa mengetahuinya. Sebaliknya, jalan yang dilalui orang baik adalah seperti terbitnya matahari; makin lama makin terang, sampai akhirnya menjadi terang benderang.
Pro 4:20  Perhatikanlah kata-kataku, anakku! Dengarkan nasihat-nasihatku.
Pro 4:21  Janganlah membuangnya, melainkan simpanlah selalu di dalam hatimu.
Pro 4:22  Orang yang memahaminya akan hidup dan menjadi sehat.
Pro 4:23  Jagalah hatimu baik-baik, sebab hatimu menentukan jalan hidupmu.
Pro 4:24  Janganlah sekali-kali mengucapkan sesuatu yang tidak benar. Jauhkanlah ucapan-ucapan dusta dan kata-kata yang dimaksud untuk menyesatkan orang.
Pro 4:25  Hendaklah wajahmu memancarkan kejujuran hatimu; tak perlu engkau berlaku seolah-olah ada udang di balik batu.
Pro 4:26  Pikirlah baik-baik sebelum berbuat, maka engkau akan berhasil dalam segala usahamu.
Pro 4:27  Jauhilah yang jahat, dan hiduplah dengan jujur. Janganlah sekali-kali menyimpang dari jalan yang benar.
Pro 5:1  Anakku, dengarkanlah aku! Perhatikanlah kebijaksanaanku dan pengertian yang kuajarkan kepadamu,
Pro 5:2  supaya engkau tahu bagaimana engkau harus membawa diri dan berbicara sebagai orang yang berpengetahuan.
Pro 5:3  Perempuan nakal, mulutnya semanis madu dan kata-katanya memikat hati,
Pro 5:4  tetapi apabila semuanya telah berlalu, yang tertinggal hanyalah kepahitan dan penderitaan.
Pro 5:5  Engkau diseretnya turun ke dunia orang mati; jalan yang ditempuhnya menuju kepada maut.
Pro 5:6  Ia tidak tetap pada jalan yang menuju hidup; tanpa diketahuinya ia telah menyimpang dari jalan itu.
Pro 5:7  Sebab itu, anak-anak, dengarkanlah kata-kataku dan janganlah mengabaikannya.
Pro 5:8  Jauhilah wanita yang demikian! Bahkan jangan mendekati rumahnya.
Pro 5:9  Kalau engkau bergaul dengan wanita itu, engkau akan kehilangan kehormatanmu dan nyawamu akan direnggut di masa mudamu oleh orang yang tidak kenal belas kasihan.
Pro 5:10  Kekayaanmu akan habis dimakan orang lain, dan hasil jerih payahmu menjadi milik orang yang tidak kaukenal.
Pro 5:11  Akhirnya engkau akan mengeluh, apabila badanmu habis dimakan penyakit.
Pro 5:12  Lalu engkau akan berkata, "Ah, kenapa aku membenci nasihat? Kenapa aku tidak mau ditegur?
Pro 5:13  Aku tak mau mendengarkan guru-guruku dan tidak mau menuruti mereka.
Pro 5:14  Tahu-tahu aku telah jatuh di mata masyarakat."
Pro 5:15  Sebab itu, setialah kepada istrimu sendiri dan berikanlah cintamu kepada dia saja.
Pro 5:16  Tidak ada gunanya bagimu mencari kenikmatan pada orang yang bukan istrimu.
Pro 5:17  Kenikmatan itu khusus untuk engkau dengan istrimu, tidak dengan orang lain.
Pro 5:18  Sebab itu, hendaklah engkau berbahagia dengan istrimu sendiri; carilah kenikmatan pada gadis yang telah kaunikahi--
Pro 5:19  gadis jelita dan lincah seperti kijang. Biarlah kemolekan tubuhnya selalu membuat engkau tergila-gila dan asmaranya memabukkan engkau.
Pro 5:20  Apa gunanya bernafsu kepada wanita lain, anakku? Untuk apa menggauli perempuan nakal?
Pro 5:21  TUHAN melihat segala-galanya yang dilakukan oleh manusia. Ke mana pun manusia pergi TUHAN mengawasinya.
Pro 5:22  Dosa orang jahat bagaikan perangkap yang menjerat orang itu sendiri.
Pro 5:23  Ia mati karena tidak menerima nasihat. Ketololannya membawa dia kepada kehancuran.
Pro 6:1  Anakku, barangkali kau pernah berjanji kepada seseorang untuk menanggung utangnya.
Pro 6:2  Dan boleh jadi kau telah terjerat oleh kata-katamu dan terjebak oleh janjimu sendiri.
Pro 6:3  Kalau benar begitu, anakku, engkau sudah berada dalam kekuasaan orang itu. Tetapi inilah caranya kau dapat lolos: cepatlah pergi kepada orang itu; mintalah dengan sangat supaya ia mau membebaskan engkau.
Pro 6:4  Janganlah pergi tidur dahulu, dan jangan beristirahat.
Pro 6:5  Lepaskanlah dirimu dari perangkap itu seperti burung atau kijang melepaskan diri dari pemburu.
Pro 6:6  Orang yang malas harus memperhatikan cara hidup semut dan belajar daripadanya.
Pro 6:7  Semut tidak punya pemimpin, tidak punya penguasa atau pengawas,
Pro 6:8  tetapi selama musim menuai mereka mengumpulkan bekal untuk musim paceklik.
Pro 6:9  Sampai kapan si pemalas itu mau tidur? Kapankah ia mau bangun?
Pro 6:10  Ia duduk berpangku tangan untuk beristirahat, dan ia berkata, "Ah, aku tidur sejenak, aku mengantuk."
Pro 6:11  Tetapi sementara ia tidur, ia ditimpa kekurangan dan kemiskinan yang datang seperti perampok bersenjata.
Pro 6:12  Orang jahat dan kurang ajar membohong ke mana-mana.
Pro 6:13  Ia bermain mata dan membuat isyarat untuk menipu.
Pro 6:14  Pikirannya yang busuk penuh dengan rencana jahat dan menimbulkan pertengkaran di mana-mana.
Pro 6:15  Semuanya itu akan menyebabkan kecelakaan menimpa dirinya dengan tiba-tiba, dan ia hancur sama sekali.
Pro 6:16  Ada tujuh perkara yang dibenci TUHAN dan tak dapat dibiarkan-Nya: Sikap yang sombong, mulut yang berbohong, tangan yang membunuh orang tak bersalah, otak yang merencanakan hal-hal jahat, kaki yang bergegas menuju kejahatan, saksi yang terus-terusan berdusta, dan orang yang menimbulkan permusuhan di antara teman.
Pro 6:20  Anakku, lakukanlah apa yang diperintahkan ayahmu, dan jangan lupa akan nasihat-nasihat ibumu.
Pro 6:21  Ingatlah selalu kata-kata mereka dan simpanlah itu di dalam hatimu.
Pro 6:22  Ajaran-ajaran mereka akan membimbing engkau dalam perjalanan, menjaga engkau pada waktu malam, dan memberi petunjuk kepadamu pada waktu engkau bangun.
Pro 6:23  Petunjuk-petunjuk orang tuamu bagaikan lampu yang terang; teguran mereka menunjukkan kepadamu cara hidup yang baik.
Pro 6:24  Dengan demikian engkau dijauhkan dari perempuan-perempuan nakal, dan dari rayuan-rayuan berbisa istri orang lain.
Pro 6:25  Janganlah engkau tergoda oleh kecantikan mereka, jangan terpikat oleh mata mereka yang merayu.
Pro 6:26  Pelacur dapat disewa seharga makanan sepiring, tetapi berzinah dengan istri orang lain harus dibayar dengan nyawa.
Pro 6:27  Dapatkah orang memangku api tanpa terbakar bajunya?
Pro 6:28  Dapatkah orang menginjak bara api tanpa terbakar kakinya?
Pro 6:29  Itu sama bahayanya dengan menggauli istri orang lain. Siapa melakukannya tak akan luput dari hukuman.
Pro 6:30  Kalau orang mencuri makanan, sekalipun karena lapar, ia akan dihina.
Pro 6:31  Dan jika ia tertangkap ia harus membayar kembali tujuh kali lipat; untuk itu ia harus kehilangan semua harta miliknya.
Pro 6:32  Tetapi bagaimanakah dengan orang laki-laki yang berzinah? Ia merusak dirinya sendiri! Bodoh sekali dia!
Pro 6:33  Ia akan disiksa dan dicemooh. Namanya akan menjadi cemar untuk selama-lamanya.
Pro 6:34  Sebab, dengan perbuatannya itu suami wanita itu menjadi cemburu, sehingga kemarahannya meluap-luap. Dan apabila ia membalas dendam, ia tak akan mempunyai belas kasihan.
Pro 6:35  Ia tak bisa dibujuk dengan uang; hadiah sebanyak apa pun tak bisa meredakan kemarahannya.
Pro 7:1  Ingatlah perkataan-perkataanku, anakku, dan jangan lupa akan apa yang kuperintahkan kepadamu.
Pro 7:2  Turutilah nasihat-nasihatku supaya engkau hidup bahagia. Ikutilah ajaran-ajaranku baik-baik seperti engkau menjaga biji matamu sendiri.
Pro 7:3  Ingatlah selalu akan ajaran-ajaranku itu, dan simpanlah di dalam hati sanubarimu.
Pro 7:4  Anggaplah hikmat sebagai saudaramu dan pengetahuan sebagai sahabat karibmu.
Pro 7:5  Hikmat akan menjauhkan engkau dari perempuan nakal, dari wanita yang memikat dengan kata-kata yang manis.
Pro 7:6  Suatu hari aku memandang dari jendela rumahku.
Pro 7:7  Lalu kulihat banyak pemuda yang masih hijau; tetapi khusus kuperhatikan seorang yang bodoh di antaranya.
Pro 7:8  Pada petang hari ketika sudah mulai gelap, pemuda itu berjalan-jalan dekat tikungan di jalan yang menuju tempat tinggal seorang wanita.
Pro 7:10  Wanita itu wanita yang tidak betah tinggal di rumahnya. Sebentar-sebentar ia berada di tepi jalan, kemudian di lapangan, atau berdiri menunggu di persimpangan. Tingkah lakunya berani dan tidak kenal malu. Kulihat ia keluar dengan berpakaian seperti pelacur, dan menemui pemuda itu dengan rencana yang licik.
Pro 7:13  Ia merangkul pemuda itu dan menciumnya. Tanpa malu-malu berkatalah ia,
Pro 7:14  "Hari ini aku harus membayar kaulku, dan untuk itu aku sudah mempersembahkan kurban.
Pro 7:15  Itu sebabnya aku keluar untuk mencari engkau, supaya engkau makan di rumahku. Sekarang aku menemukan engkau!
Pro 7:16  Tempat tidurku telah kututupi dengan seperei beraneka warna dari Mesir,
Pro 7:17  dan sudah kuharumkan dengan wangi-wangian mur, cendana dan kayu manis.
Pro 7:18  Sekarang, mari kita bercumbu-cumbuan dan menikmati asmara sepanjang malam.
Pro 7:19  Suamiku tidak ada di rumah, ia sedang mengadakan perjalanan jauh.
Pro 7:20  Ia membawa banyak uang, dan tak akan pulang dalam dua minggu."
Pro 7:21  Demikianlah wanita itu merayu pemuda itu dengan bujukan-bujukan yang memikat sehingga tergodalah ia.
Pro 7:22  Tanpa pikir ia mengikuti wanita itu seperti sapi digiring ke pejagalan dan seperti orang tahanan yang dibawa untuk menerima hukuman yang disediakan bagi orang bodoh;
Pro 7:23  sebentar lagi anak panah akan menembus hatinya. Seperti burung yang terbang menuju jerat, demikianlah pemuda itu tidak menyadari bahwa nyawanya terancam.
Pro 7:24  Sebab itu, anak-anak, dengarkanlah aku. Perhatikanlah nasihat-nasihatku.
Pro 7:25  Jangan biarkan wanita seperti itu memikat hatimu; jangan pergi mencari dia,
Pro 7:26  sebab ia sudah menghancurkan kehidupan banyak laki-laki. Tidak terhitung banyaknya yang binasa karena dia.
Pro 7:27  Pergi ke rumahnya berarti mengambil jalan pendek ke arah kematian.
Pro 8:1  Dengar! Kebijaksanaan berseru-seru, hikmat mengangkat suara.
Pro 8:2  Ia berdiri di bukit-bukit di sisi jalan, dan di persimpangan-persimpangan.
Pro 8:3  Di pintu gerbang, di jalan masuk ke kota, di situlah terdengar suaranya.
Pro 8:4  "Hai, umat manusia, kepadamu aku berseru; setiap insan di bumi, perhatikanlah himbauanku!
Pro 8:5  Kamu yang belum berpengalaman, belajarlah mempunyai pikiran yang tajam; kamu yang bebal, belajarlah menjadi insaf.
Pro 8:6  Perhatikanlah perkataan-perkataanku, karena semuanya tepat dan bermutu.
Pro 8:7  Yang kukatakan, betul semua, sebab aku benci kepada dusta.
Pro 8:8  Perkataan-perkataanku jujur semua, tak satu pun yang berbelit atau salah.
Pro 8:9  Bagi orang cerdas, perkataanku benar, bagi orang yang arif, perkataanku tepat.
Pro 8:10  Hargailah nasihatku melebihi perak asli, pentingkanlah pengetahuan melebihi emas murni.
Pro 8:11  Akulah hikmat, lebih berharga dari berlian; tak dapat dibandingkan dengan apa pun yang kauidamkan.
Pro 8:12  Akulah hikmat; padaku ada pengertian, kebijaksanaan dan pengetahuan.
Pro 8:13  Menghormati TUHAN berarti membenci kejahatan; aku tidak menyukai kesombongan dan keangkuhan. Aku benci tingkah laku yang jahat dan kata-kata tipu muslihat.
Pro 8:14  Akulah yang memberi ilham. Dan aku juga yang mewujudkannya. Aku cerdas dan kuat pula.
Pro 8:15  Raja-raja kubantu menjalankan pemerintahan, para penguasa kutolong menegakkan keadilan.
Pro 8:16  Karena jasaku, para pembesar dan para bangsawan memerintah dan menjalankan keadilan.
Pro 8:17  Aku mengasihi mereka yang suka kepadaku; yang mencari aku, akan menemukan aku.
Pro 8:18  Padaku tersedia kekayaan juga kehormatan dan kemakmuran.
Pro 8:19  Yang kaudapat dari aku melebihi emas murni; lebih berharga dari perak asli.
Pro 8:20  Aku mengikuti jalan keadilan, aku melangkah di jalan kejujuran.
Pro 8:21  Orang yang mengasihi aku, kujadikan kaya; kupenuhi rumahnya dengan harta benda.
Pro 8:22  Aku diciptakan TUHAN sebagai yang pertama, akulah hasil karya-Nya yang semula pada zaman dahulu kala.
Pro 8:23  Aku dibentuk sejak permulaan zaman, pada mulanya, sebelum bumi diciptakan.
Pro 8:24  Aku lahir sebelum tercipta samudra raya, sebelum muncul sumber-sumber air.
Pro 8:25  Aku lahir sebelum gunung-gunung ditegakkan, sebelum bukit-bukit didirikan,
Pro 8:26  sebelum TUHAN menciptakan bumi dan padang-padangnya, bahkan sebelum diciptakan-Nya gumpalan tanah yang pertama.
Pro 8:27  Aku menyaksikan ketika langit dihamparkan, dan cakrawala direntangkan di atas lautan,
Pro 8:28  ketika TUHAN menempatkan awan di angkasa, dan membuka sumber-sumber samudra,
Pro 8:29  ketika Ia memerintahkan air laut supaya jangan melewati batas-batasnya. Aku pun turut hadir di sana ketika alas bumi diletakkan-Nya.
Pro 8:30  Aku berada di samping-Nya sebagai anak kesayangan-Nya, setiap hari akulah kebahagiaan-Nya; selalu aku bermain-main di hadapan-Nya.
Pro 8:31  Aku bersenang-senang di atas bumi-Nya, dan merasa bahagia di antara manusia.
Pro 8:32  Karena itu, dengarkanlah aku, hai orang muda! Turutilah petunjukku, maka kau akan bahagia.
Pro 8:33  Terimalah petuah dan jadilah bijaksana, janganlah engkau meremehkannya.
Pro 8:34  Bahagialah orang yang mendengarkan aku yang setiap hari duduk menanti di pintu rumahku, dan berjaga-jaga di gerbang kediamanku.
Pro 8:35  Siapa mendapat aku, memperoleh kehidupan, kepadanya TUHAN berkenan.
Pro 8:36  Siapa tidak mendapat aku, merugikan diri sendiri; orang yang membenciku, mencintai maut."
Pro 9:1  Hikmat telah mendirikan rumah, dan menegakkan ketujuh tiangnya.
Pro 9:2  Ia telah memotong ternak untuk pesta, mengolah air anggur dan menyediakan hidangan.
Pro 9:3  Pelayan-pelayan wanita disuruhnya pergi untuk berseru-seru dari tempat-tempat tinggi di kota,
Pro 9:4  "Siapa tak berpengalaman, silakan ke mari!" Kepada yang tidak berakal budi, hikmat berkata,
Pro 9:5  "Mari menikmati makananku dan mengecap anggur yang telah kuolah.
Pro 9:6  Tinggalkanlah kebodohan, supaya engkau hidup bahagia. Turutilah jalan orang arif."
Pro 9:7  Kalau orang yang tak mau diajar kautunjukkan kesalahannya, ia akan menertawakan engkau. Kalau orang jahat kaumarahi, ia akan mencaci makimu.
Pro 9:8  Jangan mencela orang yang tak mau diajar, ia akan membencimu. Tetapi kalau orang bijaksana kautunjukkan kesalahannya, ia akan menghargaimu.
Pro 9:9  Kalau orang bijaksana kaunasihati, ia akan menjadi lebih bijaksana. Dan kalau orang yang taat kepada Allah kauajar, pengetahuannya akan bertambah.
Pro 9:10  Untuk menjadi bijaksana, pertama-tama orang harus mempunyai rasa hormat dan takut kepada TUHAN. Jika engkau mengenal Yang Mahasuci, engkau akan mendapat pengertian.
Pro 9:11  Hikmat akan memberikan kepadamu umur panjang.
Pro 9:12  Apabila hikmat kaumiliki, engkau sendiri yang beruntung. Tetapi jika hikmat kautolak, engkau sendiri pula yang dirugikan.
Pro 9:13  Kebodohan adalah seperti wanita cerewet yang tidak berpengalaman dan tidak tahu malu.
Pro 9:14  Tempatnya ialah di pintu rumahnya atau di pintu gerbang kota.
Pro 9:15  Dari situ ia berseru kepada orang yang lewat. Orang yang tulus hati dibujuknya,
Pro 9:16  "Mari singgah, hai kamu yang belum berpengalaman!" Dan kepada orang yang tak berakal budi ia berkata,
Pro 9:17  "Air curian rasanya manis, dan makan sembunyi-sembunyi lebih enak."
Pro 9:18  Mereka yang menjadi mangsanya tidak tahu bahwa orang yang mengunjungi dia menemui ajalnya di situ; dan mereka yang telah masuk ke dalam rumahnya, sekarang berada di dalam dunia orang mati.
Pro 10:1  Inilah petuah-petuah Salomo: Anak yang bijaksana adalah kebanggaan ayahnya; anak yang bodoh menyusahkan hati ibunya.
Pro 10:2  Kekayaan yang didapat dengan curang tidak memberi keuntungan; sebaliknya, kejujuran akan menyelamatkan.
Pro 10:3  TUHAN tak akan membiarkan orang baik kelaparan; tetapi Ia menghalang-halangi orang jahat supaya orang itu tidak memperoleh yang diinginkannya.
Pro 10:4  Orang malas akan jatuh miskin; orang yang rajin akan menjadi kaya.
Pro 10:5  Orang bijaksana mengumpulkan panen pada musimnya, tapi orang yang tidur saja pada musim panen, mendatangkan malu pada dirinya.
Pro 10:6  Orang baik akan mendapat berkat. Kekejaman tersembunyi di balik kata-kata orang jahat.
Pro 10:7  Kenangan akan orang baik merupakan berkat, tetapi orang jahat segera dilupakan.
Pro 10:8  Orang yang pandai, suka menerima nasihat; orang yang bicaranya bodoh akan sesat.
Pro 10:9  Orang jujur, hidupnya aman; orang yang menipu akan ketahuan.
Pro 10:10  Siapa menyembunyikan kebenaran, menimbulkan kesusahan; siapa yang mengeritik dengan terang-terangan, mengusahakan kesejahteraan.
Pro 10:11  Tutur kata orang baik membuat hidup bahagia, tetapi di balik kata-kata orang jahat tersembunyi hati yang keji.
Pro 10:12  Kebencian menimbulkan pertengkaran; cinta kasih mengampuni semua kesalahan.
Pro 10:13  Orang yang pikirannya tajam mengucapkan kata-kata bijaksana; orang bodoh perlu didera.
Pro 10:14  Orang bijaksana menghimpun pengetahuan; jika orang bodoh berbicara, ia memancing kecelakaan.
Pro 10:15  Kekayaan melindungi si kaya, kemelaratan menghancurkan orang miskin.
Pro 10:16  Kalau berbuat baik, upahnya ialah hidup bahagia; kalau berbuat dosa, akibatnya lebih banyak dosa.
Pro 10:17  Siapa mengindahkan teguran akan hidup sejahtera, siapa enggan mengakui kesalahan berada dalam bahaya.
Pro 10:18  Orang yang menyembunyikan kebencian adalah penipu. Orang yang menyebarkan fitnah adalah dungu.
Pro 10:19  Makin banyak bicara, makin banyak kemungkinan berdosa; orang yang dapat mengendalikan lidahnya adalah bijaksana.
Pro 10:20  Perkataan orang yang baik bagaikan perak asli; buah pikiran orang yang jahat tidak berarti.
Pro 10:21  Perkataan orang yang baik, merupakan berkat bagi banyak orang; kebodohan orang bodoh membunuh dirinya sendiri.
Pro 10:22  Karena berkat TUHAN sajalah orang menjadi kaya; kerja keras tak dapat menambah harta.
Pro 10:23  Orang bodoh senang berbuat salah; orang bijaksana gemar mencari hikmat.
Pro 10:24  Orang tulus mendapat apa yang diinginkannya; orang jahat mendapat apa yang paling ditakutinya.
Pro 10:25  Jika topan melanda, lenyaplah orang jahat; tetapi orang jujur tetap teguh selamanya.
Pro 10:26  Jangan menyuruh orang malas, ia hanya menjengkelkan saja, seperti cuka melinukan gigi atau asap memedihkan mata.
Pro 10:27  Hormatilah TUHAN, maka engkau akan hidup lama; orang jahat mati sebelum waktunya.
Pro 10:28  Harapan orang baik menjadikan dia bahagia; harapan orang jahat kosong belaka.
Pro 10:29  TUHAN melindungi orang jujur, tetapi membinasakan orang yang berbuat jahat.
Pro 10:30  Orang tulus akan hidup aman sejahtera; orang jahat tidak akan tinggal di tanah pusaka.
Pro 10:31  Orang tulus menuturkan kata-kata bijaksana; orang jahat akan dibungkamkan mulutnya.
Pro 10:32  Kata-kata orang tulus menyenangkan hati; kata-kata orang jahat selalu menyakiti.
Pro 11:1  TUHAN membenci orang yang memakai timbangan yang curang tapi Ia senang dengan orang yang memakai timbangan yang tepat.
Pro 11:2  Orang yang sombong akan dihina; orang yang rendah hati adalah bijaksana.
Pro 11:3  Orang baik dituntun oleh kejujurannya; orang yang suka bohong dihancurkan oleh kebohongannya.
Pro 11:4  Apabila menghadapi maut, harta benda tak berarti; hidupmu akan selamat bila engkau tulus hati.
Pro 11:5  Jalan hidup orang baik diratakan oleh kejujuran, tetapi orang jahat membawa diri kepada kehancuran.
Pro 11:6  Orang jujur selamat karena ketulusan hatinya; orang yang tak dapat dipercaya, terperosok oleh keserakahannya.
Pro 11:7  Bila orang jahat meninggal, harapannya pun mati; berharap kepada kekuatan sendiri tidak berarti.
Pro 11:8  Orang saleh terhindar dari kesukaran; orang jahat mendapat rintangan.
Pro 11:9  Percakapan orang jahat membinasakan; hikmat orang baik menyelamatkan.
Pro 11:10  Kota semarak jika orang jujur mendapat rejeki; rakyat bersorak-sorai jika orang jahat mati.
Pro 11:11  Restu orang jujur memperindah kota; perkataan orang jahat merusakkannya.
Pro 11:12  Menghina orang lain adalah perbuatan yang dungu; orang yang berbudi, tidak akan mengatakan sesuatu pun.
Pro 11:13  Penyebar kabar angin tak dapat memegang rahasia, tapi orang yang dapat dipercaya bisa merahasiakan perkara.
Pro 11:14  Bangsa akan hancur jika tak ada pimpinan; semakin banyak penasihat, semakin terjamin keselamatan.
Pro 11:15  Berjanji membayar utang orang lain berarti mendatangkan celaka; lebih baik tidak terlibat dalam hal itu supaya aman.
Pro 11:16  Wanita yang baik budi mendapat kehormatan; orang kejam mengumpulkan kekayaan.
Pro 11:17  Orang yang baik hati menguntungkan dirinya; orang yang kejam merugikan dirinya.
Pro 11:18  Keuntungan orang jahat adalah keuntungan semu, keuntungan orang yang berbuat baik adalah keuntungan yang sejati.
Pro 11:19  Siapa tekun berbuat baik, akan hidup bahagia; siapa berkeras untuk berbuat jahat menuju maut.
Pro 11:20  TUHAN membenci orang yang berhati jahat, tapi Ia mengasihi orang yang hidup tanpa cela.
Pro 11:21  Orang jahat pasti mendapat hukuman; orang baik akan selamat.
Pro 11:22  Kecantikan wanita yang tak berbudi serupa cincin emas di hidung babi.
Pro 11:23  Keinginan orang baik selalu menghasilkan yang baik, tapi yang dapat diharapkan oleh orang jahat hanyalah kemarahan.
Pro 11:24  Ada orang suka memberi, tapi bertambah kaya, ada yang suka menghemat, tapi bertambah miskin papa.
Pro 11:25  Orang yang banyak memberi akan berkelimpahan, orang yang suka menolong akan ditolong juga.
Pro 11:26  Siapa menimbun akan dikutuk orang, tetapi orang yang menjualnya mendapat pujian.
Pro 11:27  Siapa rajin berbuat baik akan disenangi orang; siapa mencari kejahatan, akan ditimpa kesukaran.
Pro 11:28  Siapa mengandalkan harta akan jatuh seperti daun tua; orang yang saleh akan berkembang seperti tunas muda.
Pro 11:29  Orang yang menyusahkan rumah tangganya, akan kehilangan segala-galanya. Orang bodoh akan melayani orang yang bijaksana.
Pro 11:30  Orang yang saleh akan terjamin hidupnya; orang yang bijaksana bertambah pengikutnya.
Pro 11:31  Jikalau di dunia ini orang baik pun akan menerima balasan, sudah pasti orang jahat dan berdosa akan mendapat hukuman.
Pro 12:1  Orang yang cinta kepada pengetahuan senang mendapat teguran; tapi orang yang tidak suka ditegur adalah orang dungu.
Pro 12:2  Orang baik disenangi TUHAN, tapi orang yang merancangkan kejahatan akan menerima hukuman.
Pro 12:3  Tak ada seorang pun yang dapat tetap jaya oleh kejahatan; tapi orang yang jujur tetap kukuh, tak tergoyahkan.
Pro 12:4  Istri yang baik adalah kebanggaan dan kebahagiaan suaminya, istri yang membuat suaminya malu adalah bagaikan penyakit tulang yang menggerogoti.
Pro 12:5  Orang jujur memikirkan hal-hal yang baik; orang jahat merencanakan tipu daya.
Pro 12:6  Kata-kata orang jahat mematikan; kata-kata orang jujur menyelamatkan.
Pro 12:7  Orang jahat akan jatuh dan binasa tanpa bekas; tapi orang baik akan tetap teguh turun-temurun.
Pro 12:8  Orang dipuji sesuai dengan kebijaksanaannya; orang dihina sesuai dengan kedunguannya.
Pro 12:9  Lebih baik menjadi rakyat kecil yang mempunyai pekerjaan, daripada berlagak orang besar padahal kekurangan makanan.
Pro 12:10  Orang baik memperhatikan ternaknya; tapi orang jahat menyiksanya.
Pro 12:11  Petani yang bekerja keras mempunyai banyak makanan, tapi orang yang menghabiskan waktunya untuk hal yang tak berguna adalah orang bodoh.
Pro 12:12  Orang jahat ingin mendapat keuntungan dari orang durhaka; orang baik bagaikan pohon yang berbuah.
Pro 12:13  Orang jahat terjerat oleh kata-kata buruk yang diucapkannya; orang baik luput dari kesukaran.
Pro 12:14  Setiap orang mendapat ganjaran sesuai dengan kata-kata dan perbuatannya; masing-masing diberi upah yang setimpal.
Pro 12:15  Orang dungu merasa dirinya tak pernah salah, tapi orang bijaksana suka mendengarkan nasihat.
Pro 12:16  Kalau orang bodoh tersinggung, saat itu juga ia menyatakan sakit hatinya; tapi orang bijaksana tidak peduli bila dicela.
Pro 12:17  Dengan mengatakan yang benar, orang menegakkan keadilan; dengan berdusta, orang mendatangkan ketidakadilan.
Pro 12:18  Omongan yang sembarangan dapat melukai hati seperti tusukan pedang; kata-kata bijaksana bagaikan obat yang menyembuhkan.
Pro 12:19  Dusta akan terbongkar dalam sekejap mata, tapi kata-kata benar akan tetap sepanjang masa.
Pro 12:20  Orang yang merencanakan kejahatan suka akan ketidakadilan; orang yang mengusahakan kebaikan akan bahagia.
Pro 12:21  Orang baik tak akan kena musibah; orang jahat akan selalu kena susah.
Pro 12:22  TUHAN benci kepada pendusta; tapi Ia senang dengan orang yang jujur.
Pro 12:23  Orang bijaksana tidak menonjolkan pengetahuannya; orang bodoh mengobralkan kebodohannya.
Pro 12:24  Kerja keras membuat orang berkuasa; kemalasan memaksa orang menjadi hamba.
Pro 12:25  Rasa khawatir mematahkan semangat, tetapi kata-kata ramah membesarkan hati.
Pro 12:26  Orang baik lebih beruntung dari tetangganya; orang jahat sesat karena kejahatannya.
Pro 12:27  Dengan bermalas-malas takkan tercapai yang diidamkan; dengan bekerja keras orang mendapat kekayaan.
Pro 12:28  Orang yang mengikuti jalan yang benar akan hidup bahagia; orang yang mengikuti jalan yang jahat menuju kepada maut.
Pro 13:1  Anak yang arif memperhatikan bila ayahnya memberi petuah; orang sombong tak mau menerima teguran.
Pro 13:2  Perkataan orang baik mendatangkan keuntungan; orang yang tak jujur senang pada kekerasan.
Pro 13:3  Orang yang hati-hati dalam tutur katanya akan aman hidupnya; orang yang bicara sembarangan akan ditimpa kemalangan.
Pro 13:4  Si malas banyak keinginan tapi tak satu pun yang dicapainya; orang yang bekerja keras mendapat segala yang diinginkannya.
Pro 13:5  Orang jujur benci akan dusta, tingkah laku orang jahat memalukan dan tercela.
Pro 13:6  Kejujuran melindungi orang yang hidup lurus; kejahatan menghancurkan orang yang berdosa.
Pro 13:7  Ada yang berlagak kaya, padahal tak berharta; ada yang berlagak miskin, padahal kaya raya.
Pro 13:8  Orang kaya harus mengeluarkan uang agar hidupnya aman; orang miskin bebas dari ancaman.
Pro 13:9  Orang saleh bagaikan cahaya cemerlang; orang jahat bagaikan lampu padam.
Pro 13:10  Keangkuhan hanya menghasilkan pertengkaran; orang yang bijaksana mau menerima ajaran.
Pro 13:11  Kekayaan yang didapat dengan mudah akan cepat berkurang pula; tapi harta yang dikumpulkan sedikit demi sedikit akan semakin bertambah.
Pro 13:12  Kalau harapan tidak dipenuhi, batin merana; kalau keinginan terkabul, hati bahagia.
Pro 13:13  Orang yang meremehkan ajaran TUHAN, mencelakakan dirinya; orang yang taat kepada hukum Allah akan mendapat upahnya.
Pro 13:14  Ajaran orang bijaksana bagaikan sumber kebahagiaan; bila ada ancaman, ajaran itu menyelamatkan.
Pro 13:15  Orang berbudi akan disanjung; orang yang tak dapat dipercaya akan hidup susah.
Pro 13:16  Orang bijaksana berpikir dahulu sebelum bertindak; orang bodoh mengobralkan kebodohannya.
Pro 13:17  Utusan yang tak becus mendatangkan celaka; utusan yang dapat dipercaya menjamin keberhasilan.
Pro 13:18  Orang yang tak mau dididik menjadi miskin dan hina; orang yang mengindahkan nasihat akan dihormati.
Pro 13:19  Alangkah baiknya mendapat apa yang diinginkan! Orang bodoh tak mau berpaling dari kejahatan.
Pro 13:20  Orang yang bergaul dengan orang bijaksana, akan menjadi bijaksana; orang yang bergaul dengan orang bodoh, akan celaka.
Pro 13:21  Orang berdosa selalu dikejar-kejar kemalangan orang baik selalu menerima kebahagiaan.
Pro 13:22  Orang baik mewariskan kekayaan kepada anak cucunya; kekayaan orang berdosa disimpan untuk orang yang lurus hidupnya.
Pro 13:23  Tanah kosong dapat menghasilkan banyak makanan untuk orang miskin; tetapi ketidakadilan menyebabkan tanah itu tidak dikerjakan.
Pro 13:24  Tidak memukul anak, berarti tidak cinta kepadanya; kalau cinta, harus berani memukul dia.
Pro 13:25  Orang yang baik selalu berkecukupan, tetapi orang jahat selalu kekurangan.
Pro 14:1  Rumah tangga dibangun oleh kebijaksanaan wanita, tapi diruntuhkan oleh kebodohannya.
Pro 14:2  Orang yang jujur takut dan hormat kepada TUHAN Allah; orang yang hidupnya tidak lurus menghina Dia.
Pro 14:3  Karena pongahnya, orang bodoh suka membesarkan diri; orang bijaksana akan dilindungi oleh kata-katanya sendiri.
Pro 14:4  Tanpa lembu, hasil di ladang tak ada; dengan kekuatan lembu, panen akan berlimpah.
Pro 14:5  Saksi yang jujur selalu mengatakan yang sesungguhnya, saksi yang tak dapat dipercaya selalu berdusta.
Pro 14:6  Orang sombong tak akan menjadi bijaksana, tapi orang cerdas belajar dengan mudah.
Pro 14:7  Janganlah berkawan dengan orang dungu; tak ada yang dapat diajarkannya kepadamu.
Pro 14:8  Orang bijaksana tahu bagaimana harus bertindak, orang bodoh tertipu oleh kebodohannya.
Pro 14:9  Orang bodoh tidak peduli apakah dosanya diampuni atau tidak; orang baik ingin diampuni dosanya.
Pro 14:10  Suka maupun duka tersimpan dalam kalbu; orang lain tak dapat turut merasakannya.
Pro 14:11  Rumah orang baik tetap kokoh; rumah orang jahat akan roboh.
Pro 14:12  Ada jalan yang kelihatannya lurus, tapi akhirnya jalan itu menuju maut.
Pro 14:13  Di balik tawa mungkin ada tangis; kegembiraan dapat berakhir dengan kedukaan.
Pro 14:14  Orang jahat akan memetik buah kejahatannya, orang baik akan memetik buah kebaikannya.
Pro 14:15  Orang bodoh percaya kepada setiap perkataan, orang bijaksana bertindak hati-hati.
Pro 14:16  Orang berbudi selalu waspada dan menjauhi kejahatan, orang bodoh naik pitam, lalu merasa aman.
Pro 14:17  Orang yang suka marah bertindak bodoh; orang bijaksana bersikap sabar.
Pro 14:18  Orang yang tak berpengalaman akan menjadi bodoh; orang bijaksana akan bertambah pengetahuannya.
Pro 14:19  Orang jahat akan tunduk kepada orang yang lurus hati, untuk mohon supaya dikasihani.
Pro 14:20  Orang miskin tidak disenangi bahkan oleh kawan-kawannya; tetapi orang kaya banyak sahabatnya.
Pro 14:21  Siapa menghina orang lain, berbuat dosa; siapa baik hati kepada orang miskin, akan bahagia.
Pro 14:22  Siapa mengusahakan yang baik, akan dipercaya dan mendapat hormat; siapa merencanakan yang jahat ada di jalan yang sesat.
Pro 14:23  Dalam setiap usaha ada keuntungan; obrolan yang kosong membuat orang jatuh miskin.
Pro 14:24  Orang bijaksana dipuji karena kebijaksanaannya, orang bodoh terkenal karena kebodohannya.
Pro 14:25  Kalau seorang saksi berkata benar, ia menyelamatkan nyawa; kalau ia berbohong, ia mengkhianati sesamanya.
Pro 14:26  Orang yang takwa kepada TUHAN menjadi tentram, dan keluarganya akan mempunyai perlindungan.
Pro 14:27  Takwa kepada TUHAN adalah pangkal kebahagiaan; dan jalan untuk menghindari kematian.
Pro 14:28  Kejayaan raja terletak pada jumlah rakyatnya; tanpa rakyat ia tidak dapat berkuasa.
Pro 14:29  Orang bijaksana tidak cepat marah; orang bodoh tidak dapat menahan dirinya.
Pro 14:30  Hati yang tenang menyehatkan badan; iri hati bagaikan penyakit yang mematikan.
Pro 14:31  Siapa berbuat baik kepada orang miskin, menghormati Allahnya; siapa menindas orang lemah, menghina Penciptanya.
Pro 14:32  Orang jahat binasa karena kejahatan, orang baik terlindung oleh ketulusannya.
Pro 14:33  Hikmat selalu ada di dalam pikiran orang berbudi; tapi tertindas dalam pikiran orang bodoh.
Pro 14:34  Keadilan dan kebaikan mengangkat martabat bangsa, tapi dosa membuat bangsa menjadi hina.
Pro 14:35  Raja senang kepada pegawai yang cakap; tapi ia marah kepada pegawai yang tak dapat menjalankan tugas.
Pro 15:1  Dengan jawaban yang ramah, kemarahan menjadi reda; jawaban yang pedas membangkitkan amarah.
Pro 15:2  Kata-kata orang bijak membuat pengetahuan menarik hati; kata-kata orang bodoh hanya berisi kebodohan.
Pro 15:3  TUHAN melihat semua yang terjadi di segala tempat; Ia memperhatikan semua yang baik dan yang jahat.
Pro 15:4  Kata-kata yang baik menambah semangat, kata-kata yang menyakitkan melemahkan hasrat.
Pro 15:5  Orang bodoh meremehkan nasihat ayahnya, orang bijak mengindahkan teguran.
Pro 15:6  Di rumah orang baik ada banyak harta; kekayaan orang jahat lenyap oleh bencana.
Pro 15:7  Kata-kata orang bijaksana penuh dengan pengetahuan, tapi pikiran orang bodoh tidak demikian.
Pro 15:8  TUHAN menerima doa orang baik, tapi Ia membenci persembahan orang jahat.
Pro 15:9  TUHAN membenci cara hidup orang durhaka, tetapi Ia mengasihi orang yang berusaha melakukan kehendak-Nya.
Pro 15:10  Orang yang melakukan yang jahat akan disiksa; orang yang tak mau ditegur akan mati.
Pro 15:11  TUHAN tahu dunia orang mati dan seluk beluknya; mana mungkin pikiran manusia disembunyikan dari Dia?
Pro 15:12  Orang yang suka menghina tidak suka ditegur; ia enggan meminta nasihat dari orang yang bijaksana.
Pro 15:13  Hati yang gembira membuat muka berseri-seri; hati yang sedih mematahkan semangat.
Pro 15:14  Orang bijaksana mencari pengetahuan; orang bodoh sibuk dengan kebodohan.
Pro 15:15  Orang muram harus terus berjuang dalam hidupnya, tetapi orang yang periang selalu bahagia.
Pro 15:16  Lebih baik sedikit harta tapi disertai takut kepada TUHAN, daripada banyak harta tapi disertai kecemasan.
Pro 15:17  Lebih baik makan sayur tapi disertai cinta kasih, daripada makan daging lezat tapi disertai kebencian.
Pro 15:18  Orang yang cepat marah menimbulkan pertengkaran, orang yang sabar membawa perdamaian.
Pro 15:19  Orang malas akan selalu mengalami kesukaran; orang jujur tidak menemui kesulitan.
Pro 15:20  Anak yang bijaksana, menyenangkan hati ayahnya; anak yang bodoh tidak menghargai ibunya.
Pro 15:21  Kebodohan adalah kesenangan orang bebal, orang bijaksana hidup lurus.
Pro 15:22  Rencana gagal, jika tidak disertai pertimbangan; rencana berhasil, jika banyak yang memberi nasihat.
Pro 15:23  Kata-kata yang diucapkan tepat pada waktunya, mendatangkan sukacita.
Pro 15:24  Orang bijaksana mengikuti jalan mendaki yang menuju kehidupan; bukan jalan menurun yang menuju kematian.
Pro 15:25  TUHAN akan merobohkan rumah orang yang tinggi hati, tetapi tanah milik seorang janda Ia lindungi.
Pro 15:26  TUHAN benci pada rencana yang jahat; tapi kata-kata yang baik menyenangkan hati-Nya.
Pro 15:27  Siapa mencari keuntungan yang tidak halal, menghancurkan keluarganya sendiri. Siapa membenci uang sogok, akan hidup bahagia.
Pro 15:28  Orang baik mempertimbangkan kata-katanya; orang jahat mengucapkan hal-hal yang keji.
Pro 15:29  TUHAN mendengarkan apabila orang baik berdoa, tetapi Ia tidak mengindahkan orang yang durhaka.
Pro 15:30  Wajah gembira meriangkan hati, berita yang baik menyegarkan jiwa.
Pro 15:31  Orang yang mengindahkan teguran tergolong orang bijaksana.
Pro 15:32  Orang yang tidak mau dinasihati, tidak menghargai diri sendiri; orang yang mau menerima teguran, menjadi berbudi.
Pro 15:33  Takut akan TUHAN adalah dasar pendidikan yang baik; kehormatan didahului oleh kerendahan hati.
Pro 16:1  Manusia boleh membuat rencana, tapi Allah yang memberi keputusan.
Pro 16:2  Setiap perbuatan orang mungkin baik dalam pandangannya sendiri, tapi Tuhanlah yang menilai maksud hatinya.
Pro 16:3  Percayakanlah kepada TUHAN semua rencanamu, maka kau akan berhasil melaksanakannya.
Pro 16:4  Segala sesuatu yang dibuat oleh TUHAN ada tujuannya; dan tujuan bagi orang jahat adalah kebinasaan.
Pro 16:5  Semua orang sombong dibenci TUHAN; Ia tidak membiarkan mereka luput dari hukuman.
Pro 16:6  Orang yang setia kepada Allah akan mendapat pengampunan; Orang yang takwa akan terhindar dari segala kejahatan.
Pro 16:7  Jika engkau menyenangkan hati TUHAN, musuh-musuhmu dijadikannya kawan.
Pro 16:8  Lebih baik berpenghasilan sedikit dengan kejujuran, daripada berpenghasilan banyak dengan ketidakadilan.
Pro 16:9  Manusia dapat membuat rencana, tetapi Allah yang menentukan jalan hidupnya.
Pro 16:10  Raja menerima kuasa dari Allah, jadi, ia tidak bersalah dalam keputusannya.
Pro 16:11  TUHAN menghendaki orang berlaku jujur dalam perdagangan, juga dalam memakai ukuran dan timbangan.
Pro 16:12  Bagi penguasa, berbuat jahat adalah kekejian, sebab pemerintahannya kukuh karena keadilan.
Pro 16:13  Keterangan yang benar menyenangkan penguasa, ia mengasihi orang yang berbicara dengan jujur.
Pro 16:14  Kemarahan raja adalah bagaikan berita hukuman mati; orang yang bijaksana akan berusaha meredakannya!
Pro 16:15  Kebaikan hati raja mendatangkan hidup sejahtera, seperti awan menurunkan hujan di musim kemarau.
Pro 16:16  Mendapat hikmat jauh lebih baik daripada mendapat emas; mendapat pengetahuan lebih berharga daripada mendapat perak.
Pro 16:17  Orang baik menjauhi yang jahat; orang yang memperhatikan cara hidupnya, melindungi dirinya.
Pro 16:18  Kesombongan mengakibatkan kehancuran; keangkuhan mengakibatkan keruntuhan.
Pro 16:19  Lebih baik rendah hati dan tidak berharta, daripada ikut dengan orang sombong dan menikmati harta rampasan mereka.
Pro 16:20  Perhatikanlah apa yang diajarkan kepadamu, maka kau akan mendapat apa yang baik. Percayalah kepada TUHAN, maka kau akan bahagia.
Pro 16:21  Orang bijaksana dikenal dari pikirannya yang tajam; cara bicaranya yang menarik, membuat kata-katanya makin meyakinkan.
Pro 16:22  Kebijaksanaan adalah sumber kebahagiaan hidup orang berbudi; orang bodoh disiksa oleh kebodohannya sendiri.
Pro 16:23  Pikiran orang berbudi membuat kata-katanya bijaksana, dan ajarannya semakin meyakinkan.
Pro 16:24  Perkataan ramah serupa madu; manis rasanya dan menyehatkan tubuh.
Pro 16:25  Ada jalan yang kelihatannya lurus, tapi akhirnya jalan itu menuju maut.
Pro 16:26  Keinginan untuk makan mendorong orang untuk berusaha; karena perutnya, maka ia terpaksa bekerja.
Pro 16:27  Orang jahat berusaha mencelakakan sesamanya; kata-katanya jahat seperti api membara.
Pro 16:28  Orang yang curang menimbulkan pertengkaran; pemfitnah menceraikan sahabat yang akrab.
Pro 16:29  Orang kejam menipu kawan-kawannya, dan membawa mereka ke dalam bahaya.
Pro 16:30  Waspadalah terhadap orang yang tersenyum dan bermain mata, ia sedang merencanakan kejahatan dalam hatinya.
Pro 16:31  Orang jujur akan dianugerahi umur panjang; ubannya bagaikan mahkota yang gemilang.
Pro 16:32  Tidak cepat marah lebih baik daripada mempunyai kuasa; menguasai diri lebih baik daripada menaklukkan kota.
Pro 16:33  Untuk mengetahui nasib, manusia membuang undi, tetapi yang menentukan jawabannya hanyalah TUHAN sendiri.
Pro 17:1  Lebih baik sesuap nasi disertai ketentraman, daripada makanan lezat berlimpah-limpah disertai pertengkaran.
Pro 17:2  Hamba yang cerdas akan berkuasa atas anak yang membuat malu; dan akan menerima warisan bersama saudara-saudara anak itu.
Pro 17:3  Emas dan perak diuji di perapian, tetapi hati orang diuji oleh TUHAN.
Pro 17:4  Orang jahat menuruti saran-saran yang jahat, pendusta suka mendengarkan kata-kata fitnahan.
Pro 17:5  Siapa mengejek orang miskin papa, menghina Allah penciptanya. Siapa gembira atas kemalangan orang, pasti mendapat hukuman.
Pro 17:6  Kebanggaan orang yang sudah tua adalah anak cucunya; kebanggaan anak-anak adalah orang tuanya.
Pro 17:7  Orang terhormat tidak patut mengucapkan kata-kata dusta; orang bodoh tidak pantas mengucapkan kata-kata berharga.
Pro 17:8  Ada yang menyangka uang sogok dapat membuat keajaiban; dengan uang sogok segalanya dapat terlaksana.
Pro 17:9  Kalau ingin disukai orang, maafkanlah kesalahan yang mereka lakukan. Membangkit-bangkit kesalahan hanya memutuskan persahabatan.
Pro 17:10  Satu teguran lebih berarti bagi orang berbudi daripada seratus cambukan pada orang yang bodoh.
Pro 17:11  Orang jahat selalu menimbulkan keonaran, tapi maut datang kepadanya sebagai utusan yang kejam.
Pro 17:12  Lebih baik berjumpa dengan induk beruang yang kehilangan anaknya, daripada dengan orang bodoh yang sibuk dengan kebodohannya.
Pro 17:13  Siapa membalas kebaikan dengan kejahatan, kejahatan pun tak akan dapat dikeluarkan dari rumahnya.
Pro 17:14  Memulai pertengkaran adalah seperti membuka jalan air; karena itu undurlah sebelum pertengkaran mulai.
Pro 17:15  TUHAN membenci orang yang membenarkan orang durhaka, dan yang menyalahkan orang yang tak bersalah.
Pro 17:16  Percuma orang bodoh menghabiskan uang mencari hikmat, sebab ia tidak mempunyai pikiran yang sehat.
Pro 17:17  Seorang sahabat selalu setia kepada kawan, tapi seorang saudara ikut menanggung kesusahan.
Pro 17:18  Orang yang berjanji untuk menjadi penanggung hutang sesamanya adalah orang yang bodoh.
Pro 17:19  Orang yang suka pada dosa, suka bertengkar. Orang yang bermulut besar, mencari kehancuran.
Pro 17:20  Seorang penipu tidak akan bahagia, orang dengan lidah bercabang akan mendapat celaka.
Pro 17:21  Mendapat anak yang dungu berarti mendapat kesedihan; menjadi ayah anak yang bodoh tidak memberi kegembiraan.
Pro 17:22  Hati yang gembira menyehatkan badan; hati yang murung mematahkan semangat.
Pro 17:23  Hakim yang curang, menerima uang sogok secara rahasia dan tidak menjalankan keadilan.
Pro 17:24  Tujuan orang yang berpengertian ialah untuk mendapat hikmat, tetapi tujuan orang bodoh tidak menentu.
Pro 17:25  Anak yang bodoh menyusahkan ayahnya, dan menyedihkan hati ibunya.
Pro 17:26  Tidak patut mengenakan denda pada orang yang tak bersalah; tidak patut menindas orang yang berbudi luhur.
Pro 17:27  Orang yang tajam pikirannya, tidak banyak bicara. Orang yang bijaksana, selalu tenang.
Pro 17:28  Seorang bodoh pun akan disangka cerdas dan bijaksana kalau ia berdiam diri dan menutup mulutnya.
Pro 18:1  Orang yang memisahkan diri dari orang lain berarti memperhatikan diri sendiri saja; setiap pendapat orang lain ia bantah.
Pro 18:2  Orang bodoh tidak suka diberi pengertian; ia hanya ingin membeberkan isi hatinya.
Pro 18:3  Dosa dan kehinaan berjalan bersama; kalau sudah tercela, pasti pula dinista.
Pro 18:4  Perkataan orang dapat merupakan sumber kebijaksanaan dalam seperti samudra, segar seperti air yang mengalir.
Pro 18:5  Tidak baik berpihak kepada orang durhaka dan menindas orang yang tak bersalah.
Pro 18:6  Jika orang bodoh berbicara, ia menimbulkan pertengkaran dan minta dihajar.
Pro 18:7  Ucapan orang bodoh menghancurkan dirinya; ia terjerat oleh kata-katanya.
Pro 18:8  Fitnah itu enak rasanya; orang suka menelannya.
Pro 18:9  Orang yang melalaikan tugasnya sama buruknya dengan orang yang suka merusak.
Pro 18:10  TUHAN itu seperti menara yang kuat; ke sanalah orang jujur pergi dan mendapat tempat yang aman.
Pro 18:11  Tetapi orang kaya menyangka hartanyalah yang melindungi dia seperti tembok tinggi dan kuat di sekeliling kota.
Pro 18:12  Orang yang angkuh akan jatuh, orang yang rendah hati akan dihormati.
Pro 18:13  Menjawab sebelum mendengar adalah perbuatan yang bodoh dan tercela.
Pro 18:14  Oleh kemauan untuk hidup, orang dapat menanggung penderitaan; hilang kemauan itu, hilang juga segala harapan.
Pro 18:15  Orang berbudi selalu haus akan pengetahuan; orang bijaksana selalu ingin mendapat ajaran.
Pro 18:16  Hadiah membuka jalan dan mengantar orang kepada orang-orang besar.
Pro 18:17  Pembicara pertama dalam sidang pengadilan selalu nampaknya benar, tapi pernyataannya mulai diuji apabila datang lawannya.
Pro 18:18  Dengan undian, pertikaian dapat diakhiri, bahkan pertentangan antara orang berkuasa pun dapat diselesaikan.
Pro 18:19  Saudara yang telah disakiti hatinya lebih sukar didekati daripada kota yang kuat; pertengkaran bagaikan palang gerbang kota yang berbenteng.
Pro 18:20  Bagi kata-kata yang diucapkan ada akibat yang harus dirasakan.
Pro 18:21  Lidah mempunyai kuasa untuk menyelamatkan hidup atau merusaknya; orang harus menanggung akibat ucapannya.
Pro 18:22  Orang yang mendapat istri, mendapat keuntungan; istri adalah karunia dari TUHAN.
Pro 18:23  Orang miskin memohon dengan sopan; orang kaya menjawab dengan bentakan.
Pro 18:24  Ada sahabat yang tidak setia, ada pula yang lebih akrab dari saudara.
Pro 19:1  Lebih baik orang miskin yang lurus hidupnya, daripada orang bodoh yang suka berdusta.
Pro 19:2  Kerajinan tanpa pengetahuan, tidak baik; orang yang tergesa-gesa akan membuat kesalahan.
Pro 19:3  Manusia merugikan diri sendiri oleh kebodohannya, kemudian menyalahkan TUHAN atas hal itu.
Pro 19:4  Orang kaya, kawannya selalu bertambah; orang miskin malah ditinggalkan temannya.
Pro 19:5  Orang yang berdusta di pengadilan, tak akan luput dari hukuman.
Pro 19:6  Setiap orang berusaha mengambil hati orang penting; semua orang mau bersahabat dengan dermawan.
Pro 19:7  Orang miskin diremehkan oleh saudaranya, apalagi oleh temannya. Betapapun ia berusaha, ia tak dapat memikat mereka.
Pro 19:8  Siapa mengejar pengetahuan, mengasihi dirinya; dan siapa mengingat apa yang dipelajarinya, akan bahagia.
Pro 19:9  Orang yang berdusta di pengadilan pasti akan dihukum dan dibinasakan.
Pro 19:10  Tidaklah pantas orang bodoh hidup mewah, dan tidak pula patut hamba memerintah penguasa.
Pro 19:11  Orang bijaksana dapat menahan kemarahannya. Ia terpuji karena tidak menghiraukan kesalahan orang terhadapnya.
Pro 19:12  Murka raja seperti auman singa; kebaikan raja seperti embun yang membasahi tumbuh-tumbuhan.
Pro 19:13  Anak yang bodoh bisa menghancurkan ayahnya. Istri yang suka mengomel bagaikan air menetes tiada hentinya.
Pro 19:14  Rumah dan harta bisa diperoleh dari orang tua, tetapi istri yang bijaksana adalah karunia dari TUHAN.
Pro 19:15  Bermalas-malas membuat orang tertidur lelap, dan akhirnya si pemalas akan kelaparan.
Pro 19:16  Orang yang mentaati perintah-perintah Allah akan selamat, orang yang tidak menghargainya akan mati.
Pro 19:17  Menolong orang miskin sama seperti memberi pinjaman kepada TUHAN; nanti TUHAN juga yang akan membalasnya.
Pro 19:18  Tertibkan anakmu selama masih ada harapan; kalau tidak, berarti kau menginginkan kehancurannya.
Pro 19:19  Biarlah orang yang cepat marah merasakan sendiri akibatnya. Jika engkau menolong dia, engkau hanya menambah kemarahannya.
Pro 19:20  Jika engkau suka belajar dan mendengar nasihat, kelak engkau menjadi orang yang berhikmat.
Pro 19:21  Manusia mempunyai banyak rencana, tetapi hanya keputusan TUHAN yang terlaksana.
Pro 19:22  Sifat yang diharapkan dari seseorang ialah kesetiaannya. Lebih baik miskin daripada menjadi pendusta.
Pro 19:23  Hormatilah TUHAN, maka hidupmu bahagia; engkau akan puas dan bebas dari celaka.
Pro 19:24  Ada orang yang malas bukan kepalang; menyuap makanan ke mulutnya pun ia enggan.
Pro 19:25  Kalau orang sombong dihukum, orang yang tak berpengalaman mendapat pelajaran. Kalau orang berbudi ditegur ia akan bertambah bijaksana.
Pro 19:26  Anak yang menganiaya dan mengusir orang tuanya, adalah anak yang memalukan dan tercela.
Pro 19:27  Anakku, jika engkau tak mau belajar lagi, engkau akan lupa apa yang sudah kaupelajari.
Pro 19:28  Memberi kesaksian palsu, berarti meremehkan hukum; mencelakakan orang, sedap rasanya bagi orang jahat.
Pro 19:29  Orang bodoh yang tinggi hati pasti akan dipukuli.
Pro 20:1  Minuman keras membuat orang kurang ajar dan ribut. Bodohlah orang yang minum sampai mabuk.
Pro 20:2  Murka raja bagaikan auman singa; orang yang membangkitkan kemarahannya berarti mencelakakan diri sendiri.
Pro 20:3  Hanya orang bodohlah yang suka bertengkar; sikap yang terpuji ialah menjauhi pertengkaran.
Pro 20:4  Petani yang malas tidak mengerjakan ladangnya pada waktunya; akhirnya pada musim menuai tanahnya tidak menghasilkan apa-apa.
Pro 20:5  Isi hati orang ibarat air sumur yang dalam; tapi bisa ditimba oleh orang yang punya pengertian.
Pro 20:6  Banyak orang mengaku dirinya adalah kawan, tetapi yang betul-betul setia, sukar ditemukan.
Pro 20:7  Anak-anak beruntung jika mempunyai ayah yang baik dan hidup lurus.
Pro 20:8  Bila raja duduk di kursi pengadilan, ia dapat melihat semua yang jahat.
Pro 20:9  Adakah orang yang bisa berkata, "Hatiku bersih, aku sudah bebas dari dosa?"
Pro 20:10  Timbangan dan ukuran yang curang tidak disenangi TUHAN.
Pro 20:11  Dari perbuatan anak dapat diketahui apakah kelakuannya baik dan lurus.
Pro 20:12  Telinga untuk mendengar dan mata untuk memandang, kedua-duanya Tuhanlah yang menciptakan.
Pro 20:13  Orang yang suka tidur akan jatuh miskin. Orang yang rajin bekerja mempunyai banyak makanan.
Pro 20:14  Pembeli selalu mengeluh tentang mahalnya harga. Tetapi setelah membeli, ia bangga atas harga yang diperolehnya.
Pro 20:15  Kata-kata yang mengandung pengetahuan, lebih berharga daripada emas dan berlian.
Pro 20:16  Siapa mau menanggung utang orang lain, layak diambil miliknya sebagai jaminan janjinya.
Pro 20:17  Harta hasil tipuan, mula-mula lezat rasanya, tetapi kemudian terasa seperti kerikil belaka.
Pro 20:18  Rencana berhasil oleh pertimbangan; sebab itu, janganlah berjuang tanpa membuat rencana yang matang.
Pro 20:19  Orang yang senang membicarakan orang lain, tidak dapat menyimpan rahasia; janganlah bergaul dengan orang yang terlalu banyak bicara.
Pro 20:20  Orang yang mengutuk orang tuanya, hidupnya akan lenyap seperti lampu yang padam di malam yang gelap.
Pro 20:21  Harta yang mula-mula diperoleh dengan cepat, akhirnya ternyata bukan berkat.
Pro 20:22  Janganlah membalas kejahatan dengan kejahatan; percayalah kepada TUHAN, Ialah yang akan menolong.
Pro 20:23  Neraca dan batu timbangan yang curang tidak disenangi TUHAN.
Pro 20:24  TUHAN sudah menentukan jalan hidup manusia di dunia ini; itu sebabnya manusia tak dapat mengerti jalan hidupnya sendiri.
Pro 20:25  Pikir baik-baik sebelum menjanjikan kurban kepada TUHAN. Boleh jadi engkau akan menyesal kemudian.
Pro 20:26  Raja yang bijaksana tahu siapa orang yang jahat; ia akan menghukum mereka tanpa ampun.
Pro 20:27  Hati nurani manusia merupakan terang dari TUHAN yang menyoroti seluruh batin.
Pro 20:28  Kalau raja memerintah dengan kasih, jujur dan adil, maka ia akan tetap berkuasa.
Pro 20:29  Orang muda dikagumi karena kekuatannya, dan orang tua dihormati karena ubannya.
Pro 20:30  Ada kalanya pengalaman pahit menghapuskan kejahatan, dan membersihkan hati manusia.
Pro 21:1  Sama seperti TUHAN mengatur air sungai supaya mengalir menurut kehendak-Nya, begitu juga Ia membimbing pikiran raja.
Pro 21:2  Setiap perbuatan orang mungkin baik dalam pandangannya sendiri, tapi Tuhanlah yang menilai maksud hatinya.
Pro 21:3  Perbuatan yang adil dan benar lebih menyenangkan TUHAN daripada segala persembahan.
Pro 21:4  Orang jahat itu berdosa, karena dikuasai oleh keangkuhan dan kesombongannya.
Pro 21:5  Rencana orang rajin membawa kelimpahan; tindakan tergesa-gesa mengakibatkan kekurangan.
Pro 21:6  Kekayaan yang diperoleh dengan tidak jujur cepat hilang dan membawa orang ke liang kubur.
Pro 21:7  Orang jahat tak mau mengikuti hukum; ia tersiksa oleh kekejamannya sendiri.
Pro 21:8  Orang yang bersalah, berliku-liku jalannya; orang yang baik selalu jujur hidupnya.
Pro 21:9  Tinggal di sudut loteng lebih menyenangkan daripada tinggal serumah dengan istri yang suka bertengkar.
Pro 21:10  Orang jahat selalu ingin melakukan kejahatan; terhadap siapa pun ia tidak punya belas kasihan.
Pro 21:11  Hukuman bagi pencemooh menjadi pelajaran bagi orang yang tak berpengalaman. Kalau orang berbudi ditegur, ia akan bertambah bijaksana.
Pro 21:12  Allah Yang Mahaadil tahu apa yang terjadi di dalam rumah orang durhaka. Ia akan menjerumuskan mereka sehingga mereka binasa.
Pro 21:13  Siapa tidak mau mendengar keluhan orang yang berkekurangan tidak akan diperhatikan bila ia sendiri minta pertolongan.
Pro 21:14  Untuk meredakan marah dan geram, berilah hadiah secara diam-diam.
Pro 21:15  Kalau keadilan dijalankan, maka orang baik merasa senang, tetapi orang jahat merasa terancam.
Pro 21:16  Orang yang tidak mengikuti cara hidup yang berbudi, pasti akan sampai di dunia orang mati.
Pro 21:17  Orang yang gemar bersenang-senang akan tetap berkekurangan; orang yang suka berfoya-foya tidak akan menjadi kaya.
Pro 21:18  Jika masyarakat dihukum TUHAN, bukan orang baik, melainkan orang jahat yang mendapat kesusahan.
Pro 21:19  Lebih baik tinggal di padang belantara, daripada tinggal dengan istri yang suka mengomel dan marah-marah.
Pro 21:20  Orang bijaksana tetap makmur dan kaya; tetapi orang bodoh memboroskan hartanya.
Pro 21:21  Siapa berusaha agar keadilan dan cinta kasih dilaksanakan, akan mendapat kesejahteraan, kehormatan dan umur yang panjang.
Pro 21:22  Orang yang cerdik sanggup merebut kota yang dijaga tentara yang perkasa; ia meruntuhkan benteng-benteng yang mereka andalkan.
Pro 21:23  Untuk menghindari kesukaran, hendaklah berhati-hati dengan ucapan.
Pro 21:24  Orang sombong dan tinggi hati suka mencela dan kurang ajar.
Pro 21:25  Si pemalas yang tak mau bekerja; membunuh dirinya dengan keinginannya.
Pro 21:26  Sepanjang hari ia hanya memikirkan tentang apa yang ia inginkan. Sebaliknya, orang yang lurus hidupnya dapat memberi dengan berlimpah-limpah.
Pro 21:27  TUHAN tidak senang dengan persembahan orang durhaka, lebih-lebih kalau dipersembahkan dengan maksud yang tercela.
Pro 21:28  Kesaksian pendusta tidak akan dipercaya; tapi ucapan orang yang tahu seluk beluk perkara, akan diterima.
Pro 21:29  Orang jujur yakin akan dirinya; orang jahat bermuka tebal.
Pro 21:30  Tidak ada kepintaran, kecerdasan atau kebijaksanaan yang dapat bertahan di hadapan TUHAN.
Pro 21:31  Sekalipun pertempuran diperlengkapi dengan kuda perang, yang menentukan kemenangan adalah TUHAN.
Pro 22:1  Nama baik lebih berharga daripada harta yang banyak; dikasihi orang lebih baik daripada diberi perak dan emas.
Pro 22:2  Orang kaya dan orang miskin mempunyai satu hal yang sama: Tuhanlah yang menciptakan mereka semua.
Pro 22:3  Orang bijaksana menghindar apabila melihat bahaya; orang bodoh berjalan terus lalu tertimpa malapetaka.
Pro 22:4  Orang yang takwa kepada TUHAN dan merendahkan diri, akan bahagia, makmur dan dihormati.
Pro 22:5  Perangkap dan jebakan terdapat pada jalan orang curang; jauhilah semua itu jika engkau ingin selamat.
Pro 22:6  Ajarlah seorang anak cara hidup yang patut baginya, maka sampai masa tuanya ia akan hidup demikian.
Pro 22:7  Orang miskin dikuasai oleh orang kaya; orang yang meminjam dikuasai oleh orang yang meminjamkan.
Pro 22:8  Orang yang menabur kecurangan akan menuai bencana; akhirnya kuasanya untuk bertindak sewenang-wenang akan lenyap.
Pro 22:9  Orang yang baik hati diberkati TUHAN, karena ia membagi rezeki dengan orang yang berkekurangan.
Pro 22:10  Usirlah orang yang suka menghina, maka berhentilah pertikaian, permusuhan dan cela-mencela.
Pro 22:11  Orang yang menyukai ketulusan dan yang manis bicaranya, akan menjadi sahabat raja.
Pro 22:12  Dengan membongkar kebohongan orang yang curang, TUHAN menjaga agar pengetahuan yang benar tetap terpelihara.
Pro 22:13  Si pemalas suka tinggal di rumah; ia berkata, "Ada singa di luar, aku bisa diterkam di tengah jalan."
Pro 22:14  Rayuan wanita pelacur merupakan jebakan; orang yang dimurkai TUHAN terperosok ke dalamnya.
Pro 22:15  Sudah sewajarnya anak-anak berbuat hal-hal yang bodoh, tetapi rotan dapat mengajar mereka mengubah kelakuan.
Pro 22:16  Siapa mau menjadi kaya dengan menindas orang miskin dan memberi hadiah kepada orang berada pada akhirnya akan melarat.
Pro 22:17  Dengarkan, aku akan mengajarkan kepadamu petuah orang arif! Perhatikanlah pengajaranku.
Pro 22:18  Engkau akan senang apabila pengajaranku itu kausimpan dalam hatimu. Karena dengan demikian kau telah siap apabila kau hendak memakainya.
Pro 22:19  Aku mengajarkannya kepadamu sekarang karena aku ingin kau menaruh kepercayaanmu kepada TUHAN.
Pro 22:20  Tiga puluh petuah sudah kutuliskan untukmu--petuah-petuah yang mengandung pengetahuan dan nasihat yang baik;
Pro 22:21  dan yang akan mengajarkan kepadamu apa yang sungguh-sungguh benar. Nanti, jika kau ditanyai, kau dapat memberi jawaban yang tepat.
Pro 22:22  Janganlah merugikan orang miskin, hanya karena ia lemah; dan di sidang pengadilan janganlah memperkosa hak orang yang tak berdaya.
Pro 22:23  Sebab, TUHAN akan membela perkara mereka dan mencabut nyawa orang yang menindas mereka.
Pro 22:24  Janganlah bergaul dengan orang yang suka marah dan cepat naik darah.
Pro 22:25  Nanti engkau akan meniru dia, dan tidak bisa lagi menghilangkan kebiasaan itu.
Pro 22:26  Janganlah berjanji untuk menanggung utang orang lain.
Pro 22:27  Nanti jika engkau tidak sanggup melunasinya, tempat tidurmu pun akan disita.
Pro 22:28  Janganlah sekali-kali memindahkan batas tanah yang sudah ditetapkan oleh nenek moyangmu.
Pro 22:29  Pernahkah engkau melihat orang yang cakap melakukan pekerjaannya? Orang itu akan dipekerjakan di istana raja-raja, bukan di rumah orang biasa.
Pro 23:1  Jika engkau duduk makan dengan seorang pembesar, ingatlah siapa dia.
Pro 23:2  Bila engkau mempunyai nafsu makan yang besar, tahanlah keinginanmu itu.
Pro 23:3  Jangan rakus akan makanan enak yang dihidangkannya, barangkali ia hendak menjebak engkau.
Pro 23:4  Janganlah bersusah payah untuk menjadi kaya. Batalkanlah niatmu itu.
Pro 23:5  Sebab, dalam sekejap saja hartamu bisa lenyap, seolah-olah ia bersayap dan terbang ke angkasa seperti burung rajawali.
Pro 23:6  Jangan makan bersama orang kikir. Jangan pula ingin akan makanannya yang lezat.
Pro 23:7  Sebab, ia akan berkata, "Mari makan; silakan tambah lagi," padahal maksudnya bukan begitu. Ia bermulut manis, tapi hati lain.
Pro 23:8  Nanti apa yang sudah kautelan kaumuntahkan kembali, dan semua kata-katamu yang manis kepadanya tak ada gunanya.
Pro 23:9  Janganlah menasihati orang bodoh; ia tidak akan menghargai nasihatmu itu.
Pro 23:10  Jangan sekali-kali memindahkan batas tanah warisan atau mengambil tanah kepunyaan yatim piatu.
Pro 23:11  Sebab, TUHAN adalah pembela yang kuat dialah yang akan membela mereka melawan engkau.
Pro 23:12  Perhatikanlah ajaran gurumu dan belajarlah sebanyak mungkin.
Pro 23:13  Janganlah segan-segan mendidik anakmu. Jika engkau memukul dia dengan rotan, ia tak akan mati,
Pro 23:14  malah akan selamat.
Pro 23:15  Anakku, aku senang sekali kalau engkau bijaksana.
Pro 23:16  Aku bangga bila mendengar engkau mengucapkan kata-kata yang tepat.
Pro 23:17  Janganlah iri hati kepada orang berdosa. Taatlah selalu kepada Allah
Pro 23:18  supaya masa depanmu terjamin, dan harapanmu tidak hilang.
Pro 23:19  Dengarkan, anakku! Jadilah bijaksana. Perhatikanlah sungguh-sungguh cara hidupmu.
Pro 23:20  Janganlah bergaul dengan pemabuk atau orang rakus,
Pro 23:21  sebab mereka akan menjadi miskin. Jika engkau tidur saja, maka tak lama lagi engkau akan berpakaian compang-camping.
Pro 23:22  Taatilah ayahmu; sebab, tanpa dia engkau tidak ada. Apabila ibumu sudah tua, tunjukkanlah bahwa engkau menghargai dia.
Pro 23:23  Ajaran yang benar, hikmat, didikan, dan pengertian--semuanya itu patut dibeli, tetapi terlalu berharga untuk dijual.
Pro 23:24  Seorang ayah akan gembira kalau anaknya mempunyai budi pekerti yang baik; ia akan bangga kalau anaknya bijaksana.
Pro 23:25  Semoga ayah ibumu bangga terhadapmu; semoga bahagialah wanita yang melahirkan engkau.
Pro 23:26  Perhatikanlah baik-baik dan contohilah hidupku, anakku!
Pro 23:27  Perempuan nakal yang melacur adalah perangkap yang berbahaya.
Pro 23:28  Mereka menghadang seperti perampok dan membuat banyak laki-laki berzinah.
Pro 23:29  Tahukah engkau apa yang terjadi pada orang yang minum anggur terlalu banyak, dan sering mengecap minuman keras? Orang itu sengsara dan menderita. Ia selalu bertengkar dan mengeluh. Matanya merah dan badannya luka-luka, padahal semuanya itu dapat dihindarinya.
Pro 23:31  Janganlah membiarkan anggur menggodamu, sekalipun warnanya sangat menarik dan nampaknya berkilauan dalam gelas serta mengalir masuk dengan nikmat ke dalam tenggorokan.
Pro 23:32  Esok paginya engkau merasa seperti telah dipagut ular berbisa.
Pro 23:33  Matamu berkunang-kunang, pikiranmu kacau dan mulutmu mengoceh.
Pro 23:34  Engkau merasa seperti berada pada ujung tiang kapal di tengah lautan; kepalamu pusing dan engkau terhuyung-huyung.
Pro 23:35  Lalu kau akan berkata, "Rupanya aku dipukul dan ditampar orang, tetapi aku tak dapat mengingat apa yang telah terjadi. Aduh, aku ngantuk sekali! Aku perlu minum lagi!"
Pro 24:1  Janganlah iri kepada orang jahat, dan jangan ingin berkawan dengan mereka.
Pro 24:2  Mereka hanya memikirkan kekejaman dan hanya membicarakan apa yang mencelakakan.
Pro 24:3  Rumah tangga dibangun dengan hikmat dan pengertian.
Pro 24:4  Dan apabila ada pengetahuan, maka kamar-kamarnya akan terisi lengkap dengan barang-barang berharga dan indah.
Pro 24:5  Orang bijaksana lebih berwibawa daripada orang kuat; pengetahuan lebih penting daripada tenaga.
Pro 24:6  Karena sebelum bertempur harus ada rencana yang matang dahulu, dan semakin banyak penasihat, semakin besar kemungkinan akan menang.
Pro 24:7  Orang bodoh tidak dapat menyelami hikmat. Ia tidak dapat berkata apa-apa kalau orang sedang membicarakan hal-hal penting.
Pro 24:8  Orang yang selalu merencanakan kejahatan, akan disebut perusuh.
Pro 24:9  Setiap siasat orang bodoh adalah dosa. Orang yang selalu mencela orang lain, tidak disenangi oleh siapa pun.
Pro 24:10  Jika engkau putus asa dalam keadaan gawat, maka engkau orang yang lemah.
Pro 24:11  Jangan ragu-ragu membebaskan orang yang sudah dijatuhi hukuman mati; selamatkanlah orang yang sedang digiring ke tempat penggantungan.
Pro 24:12  Boleh saja kauberkata, "Itu bukan urusanku." Tetapi Allah mengawasi engkau. Ia mengetahui dan mengadili pikiranmu. Ia membalas manusia menurut perbuatannya.
Pro 24:13  Anakku, makanlah madu, sebab itu baik. Sebagaimana madu dari sarang lebah, manis untuk dimakan,
Pro 24:14  begitu pula hikmat baik untuk jiwamu. Jika engkau bijaksana, cerahlah masa depanmu.
Pro 24:15  Janganlah seperti orang jahat yang bersepakat merampok orang jujur dan merampas rumahnya.
Pro 24:16  Sebab, sekalipun orang jujur jatuh berkali-kali, selalu ia akan bangun kembali. Tetapi sebaliknya, orang jahat akan hancur lebur oleh malapetaka.
Pro 24:17  Janganlah senang kalau musuhmu celaka, dan jangan gembira kalau ia jatuh.
Pro 24:18  Sebab, pasti TUHAN akan melihat perbuatanmu itu dan menilainya jahat, lalu tidak lagi menghukum musuhmu itu.
Pro 24:19  Jangan jengkel atau iri kepada orang jahat.
Pro 24:20  Orang jahat tidak punya masa depan dan tidak punya harapan.
Pro 24:21  Anakku, takutlah kepada TUHAN, dan hormatilah raja. Jangan ikut-ikutan dengan orang yang menentang mereka.
Pro 24:22  Orang semacam itu bisa hancur dalam sekejap, karena bencana yang ditimbulkan Allah atau raja bukanlah perkara kecil.
Pro 24:23  Orang-orang arif pernah berkata begini: Hakim tidak boleh berat sebelah.
Pro 24:24  Jika orang bersalah dinyatakannya tidak bersalah, maka hakim itu akan dikutuk dan diumpat oleh semua orang.
Pro 24:25  Tetapi hakim yang menghukum orang bersalah akan bahagia dan dihormati.
Pro 24:26  Jawaban yang tepat adalah tanda persahabatan sejati.
Pro 24:27  Janganlah mendirikan rumah tangga sebelum kau menyiapkan ladangmu dan mempunyai mata pencaharian.
Pro 24:28  Janganlah menjadi saksi terhadap orang lain tanpa alasan yang patut; janganlah juga berdusta mengenai dia.
Pro 24:29  Janganlah berkata, "Aku akan membalas kepadanya apa yang sudah dilakukannya terhadapku!"
Pro 24:30  Pernah aku melalui ladang dan kebun anggur seorang pemalas yang bodoh.
Pro 24:31  Yang kulihat di situ hanyalah tanaman berduri dan alang-alang. Pagar temboknya pun telah runtuh.
Pro 24:32  Setelah kuperhatikan dan kurenungkan hal itu, kudapati pelajaran ini:
Pro 24:33  Dengan mengantuk dan tidur sebentar, dengan duduk berpangku tangan dan beristirahat sejenak,
Pro 24:34  kekurangan dan kemiskinan datang menyerang seperti perampok bersenjata.
Pro 25:1  Berikut ini ada beberapa petuah lain dari Salomo yang disalin oleh pegawai-pegawai di istana Hizkia, raja Yehuda.
Pro 25:2  Allah diagungkan karena apa yang dirahasiakan-Nya; raja dihormati karena apa yang dapat diterangkannya.
Pro 25:3  Seperti samudra yang dalam dan langit yang tinggi, demikianlah pikiran raja tak dapat diselami.
Pro 25:4  Bersihkanlah dahulu perak dari sanganya, barulah yang indah dapat dibentuk oleh tangan seniman.
Pro 25:5  Singkirkanlah penasihat-penasihat jahat dari istana, barulah pemerintahan kukuh oleh keadilan.
Pro 25:6  Bila menghadap raja hendaklah rendah hati, jangan berlagak orang yang berkedudukan tinggi.
Pro 25:7  Lebih baik dipersilakan naik ke tempat yang lebih terhormat daripada disuruh memberi tempatmu kepada orang yang lebih berpangkat.
Pro 25:8  Janganlah terburu-buru membawa perkara ke pengadilan; sebab, kalau kemudian engkau terbukti salah apa lagi yang dapat kaulakukan?
Pro 25:9  Salah faham dengan kawanmu, selesaikanlah hanya dengan dia, tetapi rahasia orang lain janganlah kaubuka.
Pro 25:10  Sebab, nanti engkau dicap sebagai orang yang bocor mulut dan namamu cemar seumur hidup.
Pro 25:11  Pendapat yang diutarakan dengan tepat pada waktunya seperti buah emas di dalam pinggan perak.
Pro 25:12  Teguran orang arif kepada orang yang mau mendengarnya, seperti cincin emas atau perhiasan kencana.
Pro 25:13  Utusan yang setia, membuat pengutusnya senang, seperti air sejuk bagi penuai di ladang.
Pro 25:14  Janji-janji yang tidak diwujudkan, bagaikan awan dan angin yang tidak menurunkan hujan.
Pro 25:15  Kesabaran disertai kata-kata yang ramah dapat meyakinkan orang yang berkuasa, dan menghancurkan semua perlawanan.
Pro 25:16  Jangan makan madu banyak-banyak; nanti engkau menjadi muak.
Pro 25:17  Janganlah terlalu sering datang ke rumah tetanggamu, nanti ia bosan lalu membencimu.
Pro 25:18  Tuduhan palsu dapat mematikan, seperti pedang, palu atau panah yang tajam.
Pro 25:19  Mempercayai pengkhianat pada masa kesusahan adalah seperti mengunyah dengan gigi yang goyang atau berjalan dengan kaki yang timpang.
Pro 25:20  Bernyanyi untuk orang yang berduka seperti menelanjanginya dalam kedinginan cuaca seperti menuang cuka pada lukanya.
Pro 25:21  Kalau musuhmu lapar, berilah ia makan; dan kalau ia haus, berilah ia minum.
Pro 25:22  Dengan demikian engkau membuat dia menjadi malu dan TUHAN akan memberkatimu.
Pro 25:23  Angin utara pasti mendatangkan hujan; begitu pula pergunjingan pasti menimbulkan kemarahan.
Pro 25:24  Tinggal di sudut loteng lebih menyenangkan daripada tinggal serumah dengan istri yang suka pertengkaran.
Pro 25:25  Menerima berita yang baik dari negeri jauh seperti minum air sejuk ketika haus.
Pro 25:26  Orang baik yang mengalah kepada orang durhaka seperti mata air yang keruh atau sumur yang kotor.
Pro 25:27  Tidak baik makan madu berlebihan, begitu juga tak baik mengucapkan banyak pujian.
Pro 25:28  Orang yang tidak dapat menguasai dirinya seperti kota yang telah runtuh pertahanannya.
Pro 26:1  Seperti hujan di musim kemarau, dan salju di musim panas, begitu juga pujian bagi orang bodoh tidak pantas.
Pro 26:2  Seperti burung terbang dan melayang-layang di udara, begitu juga kutukan tak bisa kena pada orang yang tak bersalah.
Pro 26:3  Keledai harus dikenakan kekang, kuda harus dicambuk, demikian juga orang bodoh harus dipukul.
Pro 26:4  Orang yang menjawab pertanyaan orang dungu, sama bodohnya dengan orang itu.
Pro 26:5  Pertanyaan yang bodoh harus dijawab dengan jawaban yang bodoh pula, supaya si penanya sadar bahwa ia tidak pandai seperti yang disangkanya.
Pro 26:6  Mengutus seorang bodoh untuk mengirim berita, sama dengan mematahkan kaki sendiri dan mencari celaka.
Pro 26:7  Seperti orang lumpuh menggunakan kakinya, begitulah orang bodoh yang mengucapkan petuah.
Pro 26:8  Memuji orang yang tak berpengetahuan, seperti mengikat batu erat-erat pada jepretan.
Pro 26:9  Seperti pemabuk mengeluarkan duri dari tangannya, begitulah orang bodoh yang mengucapkan petuah.
Pro 26:10  Siapa mempekerjakan orang bodoh atau sembarang orang akan merugikan banyak orang.
Pro 26:11  Seperti anjing kembali kepada muntahnya, begitulah orang bodoh yang mengulangi kebodohannya.
Pro 26:12  Orang yang bodoh sekali masih lebih baik daripada orang yang menganggap dirinya pandai.
Pro 26:13  Si pemalas suka tinggal di rumah; ia berkata "Ada singa di luar, aku bisa diterkam di tengah jalan."
Pro 26:14  Seperti pintu berputar pada engselnya, begitulah si pemalas membalik-balikkan badannya di atas tempat tidurnya.
Pro 26:15  Ada orang yang malas bukan kepalang; menyuap makanan ke mulutnya pun ia enggan.
Pro 26:16  Si pemalas menganggap dirinya lebih berhikmat daripada tujuh orang yang memberi jawaban yang tepat.
Pro 26:17  Orang yang ikut campur dalam pertengkaran yang bukan urusannya sama seperti orang yang menangkap anjing liar pada telinganya.
Pro 26:18  Orang yang menipu, lalu berkata, "Aku hanya bergurau saja," sama dengan orang gila yang bermain dengan senjata berbahaya.
Pro 26:20  Jika kayu telah habis, padamlah api; jika si bocor mulut sudah tiada, pertengkaran pun berhenti.
Pro 26:21  Seperti arang dan kayu membuat api tetap menyala; begitulah orang yang suka bertengkar membakar suasana.
Pro 26:22  Fitnah itu enak rasanya; orang suka menelannya.
Pro 26:23  Bagaikan periuk tanah disepuh perak, begitulah orang yang manis di mulut, tapi berhati jahat.
Pro 26:24  Si pembenci manis kata-katanya tapi hatinya penuh tipu daya.
Pro 26:25  Meskipun ia ramah, janganlah percaya; karena kebencian menguasai hatinya.
Pro 26:26  Sekalipun ia menyembunyikan kebenciannya, semua orang akan melihat kejahatannya.
Pro 26:27  Siapa menggali lobang untuk orang lain, akan terperosok ke dalamnya. Siapa menggelindingkan batu supaya menimpa orang lain, akan tertimpa sendiri oleh batu itu.
Pro 26:28  Mendustai orang sama saja dengan membencinya. Mulut manis mendatangkan celaka.
Pro 27:1  Jangan membual tentang hari esok, karena engkau tidak tahu apa yang akan terjadi nanti.
Pro 27:2  Janganlah memuji dirimu sendiri; biarlah orang lain yang melakukan hal itu, bahkan orang yang tidak kaukenal.
Pro 27:3  Batu dan pasir itu masih ringan, bila dibandingkan dengan sakit hati yang ditimbulkan oleh orang bodoh.
Pro 27:4  Kemarahan itu kejam dan menghancurkan, tetapi menghadapi cemburu siapa tahan?
Pro 27:5  Lebih baik teguran yang terang-terangan daripada kasih yang tidak diungkapkan.
Pro 27:6  Kawan memukul dengan cinta, tetapi musuh merangkul dengan bisa.
Pro 27:7  Kalau kenyang, madu pun ditolak; kalau lapar, yang pahit pun terasa enak.
Pro 27:8  Orang yang meninggalkan rumahnya, seperti burung yang meninggalkan sarangnya.
Pro 27:9  Sebagaimana minyak harum dan wangi-wangian menyenangkan hati, demikian juga kebaikan kawan menyegarkan jiwa.
Pro 27:10  Jangan lupa kawanmu atau kawan ayahmu. Dalam kesukaran janganlah minta bantuan saudaramu; tetangga yang dekat lebih berguna daripada saudara yang jauh.
Pro 27:11  Anakku, hendaklah engkau bijaksana, agar aku senang dan dapat menjawab bila dicela.
Pro 27:12  Orang bijaksana menghindar apabila melihat bahaya; orang bodoh berjalan terus lalu tertimpa malapetaka.
Pro 27:13  Siapa mau menanggung utang orang lain, layak diambil miliknya sebagai jaminan janjinya.
Pro 27:14  Siapa pagi-pagi mengucapkan salam kepada kawannya dengan suara yang kuat, dianggap mengucapkan laknat.
Pro 27:15  Istri yang suka pertengkaran seperti bunyi hujan yang turun seharian.
Pro 27:16  Tak mungkin ia disuruh diam, seperti angin tak bisa ditahan dan minyak tak bisa digenggam.
Pro 27:17  Sebagaimana baja mengasah baja, begitu pula manusia belajar dari sesamanya.
Pro 27:18  Siapa memelihara pohon, akan makan buahnya. Pelayan akan dihargai bila memanjakan tuannya.
Pro 27:19  Sebagaimana air memantulkan wajahmu, demikian juga hatimu menunjukkan dirimu.
Pro 27:20  Di dunia orang mati, selalu ada tempat; begitu pula keinginan manusia tidak ada batasnya.
Pro 27:21  Emas dan perak diuji dalam perapian; orang dikenal dari sikapnya terhadap pujian.
Pro 27:22  Sekalipun orang bodoh dipukul sekeras-kerasnya, tak akan lenyap kebodohannya.
Pro 27:23  Peliharalah ternakmu baik-baik,
Pro 27:24  karena kekayaan tidak akan kekal, bahkan kuasa untuk memerintah pun tidak akan tetap selama-lamanya.
Pro 27:25  Rumput di ladang dan di gunung dipotong dan dikumpulkan untuk ternakmu itu, tapi sementara itu tumbuhlah rumput yang baru.
Pro 27:26  Dari bulu domba-dombamu engkau mendapat pakaian, dan dari uang penjualan sebagian kambing-kambingmu engkau dapat membeli tanah yang baru.
Pro 27:27  Dari kambing-kambingmu yang lain engkau mendapat susu untuk dirimu dan keluargamu serta pelayan-pelayanmu.
Pro 28:1  Orang jahat lari tanpa ada yang mengejarnya, tapi orang jujur, berani seperti singa.
Pro 28:2  Kalau bangsa berdosa, penguasanya berganti-ganti; kalau pemimpin bijaksana, bangsa akan bertahan dan tetap jaya.
Pro 28:3  Penguasa yang menindas orang miskin, seperti hujan lebat yang merusak panen.
Pro 28:4  Siapa mengabaikan hukum, memihak orang jahat; siapa mentaati hukum menentang orang bejat.
Pro 28:5  Keadilan tidak difahami orang durhaka, tetapi orang yang menyembah TUHAN, sungguh-sungguh memahaminya.
Pro 28:6  Lebih baik orang miskin yang tulus hatinya daripada orang kaya yang curang.
Pro 28:7  Pemuda yang mentaati hukum adalah orang bijaksana. Tetapi orang yang bergaul dengan orang pemboros memalukan ayahnya.
Pro 28:8  Siapa menjadi kaya karena mengambil rente dan mengeruk keuntungan sebanyak-banyaknya, kekayaannya akan jatuh kepada orang yang berbelaskasihan terhadap orang miskin.
Pro 28:9  Jika engkau tidak mentaati hukum Allah, Allah pun akan merasa muak terhadap doamu.
Pro 28:10  Siapa membujuk orang jujur supaya berbuat jahat, ia sendiri jatuh ke dalam perangkap. Orang yang murni hatinya akan diberkati sehingga bahagia.
Pro 28:11  Orang kaya selalu merasa dirinya bijaksana, tetapi orang miskin yang berbudi tahu siapa dia sebenarnya.
Pro 28:12  Bila orang baik berkuasa, semua orang senang, kalau orang jahat memerintah, rakyat bersembunyi.
Pro 28:13  Siapa menyembunyikan dosanya tidak akan beruntung. Siapa mengakui dan meninggalkannya, akan dikasihani TUHAN.
Pro 28:14  Hormatilah TUHAN selalu, maka engkau akan bahagia. Jika engkau melawan Dia, engkau mendapat celaka.
Pro 28:15  Orang jahat yang berkuasa atas orang yang berkekurangan, seperti singa yang mengaum atau beruang yang menyerang.
Pro 28:16  Penguasa yang menindas orang lain, tidak mempunyai akal yang sehat; penguasa yang membenci kecurangan akan memerintah bertahun-tahun.
Pro 28:17  Seorang pembunuh akan merasa dikejar terus sampai ajalnya; tak seorang pun dapat menahan dia.
Pro 28:18  Orang yang hidup tanpa cela akan selamat. Orang yang berliku-liku hidupnya, akan jatuh.
Pro 28:19  Petani yang bekerja keras akan cukup makanan; orang yang membuang-buang waktu akan berkekurangan.
Pro 28:20  Orang jujur akan diberkati TUHAN, tetapi orang yang ingin cepat kaya, tak luput dari hukuman.
Pro 28:21  Tidak baik membeda-bedakan orang; tetapi ada juga hakim yang mau berlaku curang hanya untuk mendapat sedikit uang.
Pro 28:22  Orang yang loba ingin cepat kaya; ia tak sadar bahwa kemiskinan segera menimpa dirinya.
Pro 28:23  Orang yang memberi teguran akhirnya lebih dihargai daripada orang yang memberi sanjungan.
Pro 28:24  Siapa mencuri dari orang tuanya dan menyangka itu bukan dosa, sama dengan pencuri-pencuri lainnya.
Pro 28:25  Mementingkan diri sendiri menimbulkan pertengkaran; engkau lebih beruntung apabila percaya kepada TUHAN.
Pro 28:26  Siapa mengandalkan pendapatnya sendiri, tidak punya akal yang sehat. Siapa mengikuti ajaran orang bijaksana akan selamat.
Pro 28:27  Siapa memberi kepada orang miskin tidak akan kekurangan. Siapa menutup mata terhadap kebutuhan orang miskin, akan kena kutukan.
Pro 28:28  Kalau orang jahat berkuasa, rakyat bersembunyi; kalau orang jahat binasa, orang yang adil akan memimpin dengan leluasa.
Pro 29:1  Siapa terus membangkang kalau dinasihati, suatu waktu akan hancur dan tak dapat diperbaiki lagi.
Pro 29:2  Apabila orang adil memerintah, rakyat gembira. Tetapi apabila orang jahat berkuasa, rakyat menderita.
Pro 29:3  Siapa suka kepada hikmat, menyenangkan hati ayahnya. Siapa bergaul dengan pelacur memboroskan uangnya.
Pro 29:4  Jika penguasa memperhatikan keadilan, negerinya akan kukuh. Tetapi jika ia mementingkan uang, negerinya akan runtuh.
Pro 29:5  Siapa menyanjung sesamanya, memasang jerat untuk dirinya.
Pro 29:6  Orang jahat terjerat oleh dosanya, orang jujur merasa senang dan bahagia.
Pro 29:7  Orang baik mengetahui hak orang lemah, tetapi orang jahat tidak memahaminya.
Pro 29:8  Orang yang suka mencela dapat mengacaukan kota, tetapi orang bijaksana menentramkan suasana.
Pro 29:9  Kalau orang bijaksana dan orang bodoh berperkara, orang bodoh itu hanya mengamuk dan tertawa sehingga menimbulkan keributan.
Pro 29:10  Orang yang tulus hati dibenci oleh orang yang haus darah, tapi dilindungi oleh orang yang baik.
Pro 29:11  Orang bodoh marah secara terang-terangan, tetapi orang bijaksana bersabar dan menahan kemarahan.
Pro 29:12  Jika penguasa memperhatikan berita dusta, pegawainya akan menjadi jahat semua.
Pro 29:13  Seorang miskin dan penindasnya mempunyai persamaan; kepada kedua-duanya TUHAN memberi mata untuk melihat cahaya kehidupan.
Pro 29:14  Kalau raja membela hak orang lemah, bertahun-tahun ia akan memerintah.
Pro 29:15  Didikan dan teguran menjadikan orang bijaksana. Anak yang selalu dituruti kemauannya akan memalukan ibunya.
Pro 29:16  Bila orang jahat berkuasa, pelanggaran meningkat. Tetapi orang jujur akan sempat melihat kehancuran orang jahat.
Pro 29:17  Didiklah anakmu, maka ia akan memberikan ketentraman kepadamu, dan menjadi hiburan bagimu.
Pro 29:18  Bangsa yang tidak mendapat bimbingan dari TUHAN menjadi bangsa yang penuh kekacauan. Berbahagialah orang yang taat kepada hukum TUHAN.
Pro 29:19  Seorang hamba tidak dapat diajar hanya dengan kata-kata. Ia mengerti, tapi tidak akan memperhatikannya.
Pro 29:20  Lebih banyak harapan bagi orang dungu, daripada bagi orang yang berbicara tanpa berpikir dahulu.
Pro 29:21  Hamba yang dimanjakan sejak muda, akhirnya akan menjadi keras kepala.
Pro 29:22  Orang yang cepat marah membuat banyak orang bertengkar dan berdosa.
Pro 29:23  Orang angkuh akan direndahkan; orang yang rendah hati akan dipuji.
Pro 29:24  Orang yang bekerja sama dengan pencuri, berarti membenci diri sendiri. Kalau ia berterus terang di pengadilan, ia akan dijatuhi hukuman. Tetapi jika ia diam saja, ia akan terkena kutukan Allah.
Pro 29:25  Takut akan pendapat orang, mengakibatkan kesusahan. Percayalah kepada TUHAN, maka engkau akan aman.
Pro 29:26  Banyak orang suka mencari muka pada penguasa, tetapi yang memberi keadilan, hanya TUHAN saja.
Pro 29:27  Orang baik tidak senang dengan orang jahat; orang jahat membenci orang baik.
Pro 30:1  Inilah perkataan-perkataan yang diucapkan oleh Agur anak Yake dari Masa: "Aku lelah, ya Allah, aku lelah! Habislah tenagaku!
Pro 30:2  Aku bodoh seperti hewan; tak punya pengertian.
Pro 30:3  Tidak pernah kupelajari hikmah, sedikit pun tidak kuketahui tentang Allah.
Pro 30:4  Manusia manakah pernah naik turun surga? Pernahkah ia menggenggam angin dalam tangannya, dan membungkus air dalam bajunya, atau menetapkan batas-batas dunia? Siapakah dia, dan siapa anaknya? Pasti engkau mengetahuinya!
Pro 30:5  Allah menepati setiap janji-Nya. Ia bagaikan perisai bagi semua yang datang berlindung pada-Nya.
Pro 30:6  Kalau engkau menambah apa yang dikatakan-Nya, maka Ia akan menegurmu dan membuktikan bahwa engkau pendusta."
Pro 30:7  "Ya Allah, aku mohon sebelum aku mati, berikanlah kepadaku dua hal ini.
Pro 30:8  Jangan sampai aku mengucapkan kata-kata curang dan dusta, dan jangan biarkan aku miskin atau kaya. Berikanlah kepadaku hanya apa yang kuperlukan.
Pro 30:9  Sebab, apabila aku kaya, mungkin aku akan berkata bahwa aku tidak memerlukan Engkau. Dan jika aku miskin, mungkin aku akan mencuri sehingga mencemarkan nama-Mu."
Pro 30:10  Jangan memfitnah seorang hamba pada tuannya, nanti engkau disumpahi dan dianggap bersalah.
Pro 30:11  Ada orang yang menyumpahi ayahnya dan tidak menghargai ibunya.
Pro 30:12  Ada orang yang menganggap dirinya suci padahal ia kotor sekali.
Pro 30:13  Ada orang merasa dirinya baik sekali--bukan main baiknya!
Pro 30:14  Ada orang yang mencari nafkah dengan cara yang kejam; mereka bengis dan memeras orang miskin dan orang lemah.
Pro 30:15  Lintah darat mempunyai dua anak; kedua-duanya bernama "Untuk aku"! Ada empat hal yang tidak pernah puas:
Pro 30:16  dunia orang mati, wanita yang tak pernah melahirkan, tanah kering yang haus akan hujan, dan api yang menjilat ke mana-mana.
Pro 30:17  Orang yang mencemoohkan ayahnya atau ibunya yang sudah tua, pantas dicampakkan ke luar supaya matanya dipatuk burung gagak dan mayatnya dimakan oleh burung rajawali.
Pro 30:18  Ada empat hal yang terlalu sukar bagiku untuk dimengerti:
Pro 30:19  burung rajawali yang terbang di udara, ular yang menjalar di atas batu karang, kapal yang berlayar di tengah lautan, dan sepasang muda mudi yang sedang jatuh cinta.
Pro 30:20  Inilah siasat seorang wanita yang tidak setia kepada suaminya: Sesudah berzinah, ia makan lalu menyeka mulutnya, kemudian berkata, "Aku tidak berbuat apa-apa!"
Pro 30:21  Ada empat hal yang sukar diterima di seluruh dunia:
Pro 30:22  seorang hamba yang menjadi raja, seorang bodoh yang mendapat segala yang diinginkannya,
Pro 30:23  seorang wanita yang berhasil menikah padahal ia tidak disukai orang, seorang hamba perempuan yang merampas kedudukan nyonyanya.
Pro 30:24  Di bumi ini ada empat macam binatang yang kecil tetapi pandai sekali, yaitu:
Pro 30:25  Semut, binatang yang tidak kuat, tetapi menyediakan makanannya pada musim panas.
Pro 30:26  Pelanduk, binatang yang lemah, tetapi membuat rumahnya di bukit batu.
Pro 30:27  Belalang, binatang yang tidak mempunyai raja, tetapi berbaris dengan teratur.
Pro 30:28  Cicak, binatang yang dapat ditangkap dengan tangan, tetapi terdapat di istana raja.
Pro 30:29  Ada empat hal yang mengesankan apabila diperhatikan caranya berjalan, yaitu:
Pro 30:30  Singa, binatang terkuat di antara segala binatang dan tidak gentar terhadap apa saja;
Pro 30:31  kambing jantan, ayam jantan yang berjalan tegak, dan raja di depan rakyatnya.
Pro 30:32  Jika karena kebodohanmu engkau dengan angkuh merencanakan perkara-perkara jahat, awas!
Pro 30:33  Sebab, kalau engkau mengocok susu, kau menghasilkan mentega. Kalau engkau memukul hidung orang, keluarlah darah. Kalau engkau menimbulkan kemarahan, kau terlibat dalam pertengkaran.
Pro 31:1  Inilah perkataan-perkataan yang diucapkan oleh ibunda Lemuel, raja Masa kepada anaknya,
Pro 31:2  "Anakku, buah hatiku, yang kulahirkan sebagai jawaban atas doaku. Apakah yang akan kukatakan kepadamu?
Pro 31:3  Janganlah memboroskan tenagamu atau menghamburkan kekuatanmu kepada wanita. Sudah banyak raja yang hancur karena wanita.
Pro 31:4  Ingatlah, Lemuel! Minum anggur dan ketagihan minuman keras, tidak pantas bagi penguasa.
Pro 31:5  Sebab, apabila raja minum minuman keras, ia lupa akan hukum dan tidak menghiraukan hak orang lemah.
Pro 31:6  Minuman keras adalah untuk mereka yang merana dan bersedih hati.
Pro 31:7  Mereka minum untuk melupakan kemiskinan dan kesusahan mereka.
Pro 31:8  Belalah mereka yang tak dapat membela dirinya sendiri. Lindungilah hak semua orang yang tak berdaya.
Pro 31:9  Berjuanglah untuk mereka, dan jadilah hakim yang adil. Lindungilah hak orang miskin dan orang tertindas."
Pro 31:10  Istri yang cakap sukar ditemukan; ia lebih berharga daripada intan berlian.
Pro 31:11  Suaminya tak akan kekurangan apa-apa, karena menaruh kepercayaan kepadanya.
Pro 31:12  Ia tak pernah berbuat jahat kepada suaminya; sepanjang umurnya ia berbuat baik kepadanya.
Pro 31:13  Ia rajin mengumpulkan rami dan bulu domba lalu sibuk bekerja menenunnya.
Pro 31:14  Dari jauh ia mendatangkan makanan, seperti yang dilakukan oleh kapal-kapal pedagang.
Pro 31:15  Pagi-pagi buta ia bangun untuk menyiapkan makanan bagi keluarganya, dan untuk membagi-bagikan tugas kepada pelayan-pelayannya.
Pro 31:16  Ia mencari sebidang tanah, lalu membelinya; ia mengusahakan sebuah kebun anggur dari pendapatannya.
Pro 31:17  Ia menyiapkan dirinya untuk bekerja sekuat tenaga.
Pro 31:18  Ia tahu bahwa segala sesuatu yang dibuatnya, menguntungkan; ia bekerja sampai jauh malam.
Pro 31:19  Benang dipintalnya dan kain ditenunnya.
Pro 31:20  Ia tidak kikir kepada yang berkekurangan; ia baik hati kepada yang memerlukan pertolongan.
Pro 31:21  Ia tidak khawatir apabila musim dingin tiba, karena baju panas tersedia bagi keluarganya.
Pro 31:22  Ia sendiri yang membuat permadaninya; pakaiannya dari kain lenan ungu yang mewah.
Pro 31:23  Suaminya adalah orang ternama--salah seorang dari antara para pemimpin kota.
Pro 31:24  Ia membuat pakaian dan ikat pinggang lalu menjualnya kepada pedagang.
Pro 31:25  Ia berwibawa dan dihormati; dan tidak khawatir tentang hari nanti.
Pro 31:26  Dengan lemah lembut ia berbicara; kata-katanya bijaksana.
Pro 31:27  Ia selalu rajin bekerja dan memperhatikan urusan rumah tangganya.
Pro 31:28  Ia dihargai oleh anak-anaknya, dan dipuji oleh suaminya.
Pro 31:29  "Ada banyak wanita yang baik," kata suaminya, "tetapi engkau yang paling baik dari mereka semua."
Pro 31:30  Paras yang manis tak dapat dipercaya, dan kecantikan akan hilang; tetapi wanita yang taat kepada TUHAN layak mendapat pujian.
Pro 31:31  Balaslah segala kebaikannya; ia wanita yang patut dihormati di mana-mana!


\end{document}