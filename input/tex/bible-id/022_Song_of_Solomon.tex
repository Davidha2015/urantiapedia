\begin{document}

\title{Kidung Agung}


\chapter{1}

\par 1 Kidung Agung ciptaan Salomo.
\par 2 Ciumilah aku dengan bibirmu; cintamu lebih nikmat dari anggur!
\par 3 Engkau harum semerbak, namamu seperti minyak wangi yang tertumpah; sebab itulah gadis-gadis cinta padamu!
\par 4 Bergegaslah kita, ya rajaku, bawalah aku ke dalam kamarmu. Karena engkau kami semua bersukaria, dan memuji cintamu melebihi anggur; pantaslah gadis-gadis cinta padamu!
\par 5 Biar hitam, aku cantik, hai putri-putri Yerusalem; hitam seperti kemah-kemah Kedar, tapi indah seperti tirai-tirai di istana Salomo!
\par 6 Jangan perhatikan kulitku yang hitam, sebab aku terbakar sinar matahari. Abang-abangku marah kepadaku, dan menyuruh aku bekerja di kebun anggur; aku tiada waktu mengurus diriku sendiri.
\par 7 Katakanlah, hai kekasihku, di mana engkau menggembalakan domba-domba, di mana kaubaringkan mereka di waktu petang? Masakan aku akan seperti pengembara di antara kawanan domba teman-temanmu?
\par 8 Masakan engkau tak tahu tempatnya, hai yang jelita di antara wanita? Ikut saja jejak kawanan domba, dan gembalakanlah kambing-kambingmu di dekat perkemahan para gembala.
\par 9 Kekasihku, engkau laksana kuda betina yang menarik kereta raja Mesir.
\par 10 Pipimu molek di tengah perhiasan, lehermu indah dengan kalung permata.
\par 11 Kami buatkan perhiasan emas bagimu, dengan manik-manik perak.
\par 12 Sementara rajaku di pembaringannya, semerbak wangi narwastuku.
\par 13 Kekasihku seperti mur harumnya, waktu berbaring di dadaku.
\par 14 Kekasihku laksana serumpun bunga pacar di kebun-kebun anggur En-Gedi.
\par 15 Engkau cantik jelita, manisku, sungguh cantik engkau! Matamu bagaikan merpati.
\par 16 Engkau tampan, sayang, sungguh tampan engkau! Petiduran kita di rumput hijau.
\par 17 Pohon aras jadi tiang rumah kita, dan pohon cemara langit-langitnya.

\chapter{2}

\par 1 Aku hanya bunga mawar dari Saron, bunga bakung di lembah-lembah.
\par 2 Seperti bunga bakung di tengah semak berduri, begitulah kekasihku di antara para putri.
\par 3 Seperti pohon apel di tengah pohon-pohon di hutan, begitulah kekasihku di antara kaum pria. Aku senang bernaung di bawahnya, buah-buahnya manis rasanya.
\par 4 Dibawanya aku ke ruang pesta, pandangannya padaku penuh cinta.
\par 5 Kuatkanlah aku dengan manisan buah anggur, segarkanlah aku dengan buah apel, sebab aku sakit asmara.
\par 6 Tangan kirinya menopang kepalaku, tangan kanannya memeluk aku.
\par 7 Berjanjilah, hai putri-putri Yerusalem, demi rusa-rusa dan kijang-kijang di padang, bahwa kamu takkan mengganggu cinta, sampai ia dipuaskan.
\par 8 Dengar! Kekasihku datang! Ia seperti anak rusa atau kijang. Melompat-lompat di gunung-gunung, meloncat-loncat di bukit-bukit. Lihat, di balik tembok ia berdiri, dengan matanya ia mencari; mengintip dari kisi-kisi jendela.
\par 10 Dengar, kekasihku berbicara kepadaku.
\par 11 Lihat, musim dingin sudah lewat, musim hujan sudah berlalu.
\par 12 Di ladang bunga-bunga bermekaran; musim memangkas telah tiba; bunyi tekukur terdengar di tanah kita.
\par 13 Pohon ara mulai berbuah, pohon anggur semerbak bunganya. Datanglah manisku, marilah jelitaku.
\par 14 O merpatiku di celah-celah batu, di persembunyian lereng-lereng yang terjal, biarlah aku melihat wajahmu, dan mendengar suaramu, sebab wajahmu elok, suaramu merdu.
\par 15 Tangkaplah rubah-rubah itu, rubah-rubah kecil yang merusak kebun anggur, sebab kebun anggur kami sedang berkembang.
\par 16 Kekasihku milikku, dan aku miliknya, ia menggembalakan domba-dombanya di antara bunga-bunga bakung
\par 17 sampai bertiup angin pagi yang melenyapkan kegelapan malam. Kembalilah, kekasihku, seperti kijang, seperti anak rusa di pegunungan Beter.

\chapter{3}

\par 1 Malam-malam, di ranjangku, dalam mimpi kucari kekasihku; kucari dia, tapi sia-sia.
\par 2 Aku mau bangun dan keliling kota, menjelajahi semua jalan dan lorongnya. Aku mau mencari jantung hatiku; kucari dia, tapi sia-sia.
\par 3 Aku ditemui peronda kota, dan kutanya, "Apakah kamu melihat jantung hatiku?"
\par 4 Baru saja kutinggalkan mereka, kutemui jantung hatiku. Kupegang dia, dan tidak kulepaskan sampai kubawa ke rumah ibuku, ke bilik orang yang melahirkan aku.
\par 5 Berjanjilah, hai putri-putri Yerusalem, demi rusa-rusa dan kijang-kijang di padang, bahwa kamu takkan mengganggu cinta, sampai ia dipuaskan.
\par 6 Apakah itu yang datang dari padang gurun, menyerupai gumpalan asap? Wanginya seperti mur dan kemenyan, dan bedak-bedak harum dari pedagang.
\par 7 Lihat, itu tandu Salomo, dikelilingi enam puluh orang perkasa, orang-orang perkasa Israel,
\par 8 yang berpengalaman dalam perang. Mereka semua membawa pedang di pinggang siap menghadapi serangan di waktu malam.
\par 9 Salomo membuatkan dirinya sebuah tandu, dari kayu yang paling bermutu.
\par 10 Tiangnya terbuat dari perak, dan sandarannya dari emas. Tempat duduknya berwarna ungu, disulam dengan senang hati oleh putri-putri Yerusalem.
\par 11 Keluarlah, hai putri-putri Sion, tengoklah Raja Salomo dengan mahkota yang dipasang ibunya, pada hari nikahnya, pada hari ia bersukaria.

\chapter{4}

\par 1 Engkau cantik jelita, manisku, sungguh, engkau cantik jelita. Matamu di balik cadarmu bagaikan merpati; rambutmu seperti kawanan kambing yang menuruni bukit-bukit Gilead.
\par 2 Gigimu putih seperti kawanan domba yang baru dicukur dan dimandikan; berpasangan seperti anak-anak domba kembar, satu pun tidak ada yang hilang.
\par 3 Bibirmu bagaikan seutas pita merah, dan eloklah mulutmu. Pipimu seperti belahan buah delima, tersembunyi di balik cadarmu.
\par 4 Lehermu bagaikan menara Daud, yang dibangun untuk menyimpan senjata. Kalungmu serupa seribu perisai, gada para pahlawan semuanya.
\par 5 Buah dadamu laksana dua anak rusa, anak kembar kijang yang tengah merumput di antara bunga-bunga bakung.
\par 6 Ingin aku pergi ke gunung mur, ke bukit kemenyan, sampai berhembus angin pagi, yang melenyapkan kegelapan malam.
\par 7 Engkau cantik sekali, manisku, tiada cacat padamu.
\par 8 Mari kita turun dari Libanon, pengantinku, mari kita turun dari Libanon. Turunlah dari puncak Amana, dari puncak Senir dan Hermon, tempat tinggal macan tutul dan singa.
\par 9 Engkau menawan hatiku, dinda, pengantinku, engkau menawan hatiku dengan pandanganmu, dengan permata indah pada kalungmu.
\par 10 Betapa nikmat cintamu, dinda, pengantinku, jauh lebih nikmat daripada anggur. Minyakmu harum semerbak, melebihi segala macam rempah.
\par 11 Manis kata-katamu, pengantinku, seperti madu murni dan susu. Pakaianmu seharum Gunung Libanon.
\par 12 Kekasihku adalah kebun bertembok, kebun bertembok, mata air terkunci.
\par 13 Di sana tumbuh pohon-pohon delima, dengan buah-buah yang paling lezat. Bunga pacar dan narwastu,
\par 14 narwastu, kunyit, kayu manis dan tebu, dengan macam-macam pohon kemenyan. Mur dan gaharu, dengan macam-macam rempah pilihan.
\par 15 Mata air di kebunku membualkan air hidup yang mengalir dari Libanon!
\par 16 Bangunlah, hai angin utara, datanglah, hai angin selatan! Bertiuplah di kebunku, biar harumnya semerbak. Semoga kekasihku datang ke kebunnya, dan makan buah-buahnya yang lezat.

\chapter{5}

\par 1 Aku datang ke kebunku, dinda, pengantinku, kukumpulkan mur dan rempah-rempahku; kumakan sarang lebah dan maduku, kuminum susu dan air anggurku.
\par 2 Aku tidur, namun hatiku berjaga. Dengarlah, kekasihku mengetuk pintu.
\par 3 Bajuku sudah kulepaskan; apakah akan kupakai lagi? Kakiku sudah kubasuh, apakah akan kukotori lagi?
\par 4 Berdebar-debar hatiku karena kekasihku memegang gagang pintu.
\par 5 Maka bangunlah aku hendak membuka pintu bagi kekasihku. Mur menetes dari tangan dan jari-jariku, membasahi pegangan kancing pintu.
\par 6 Kubukakan pintu bagi kekasihku, tetapi ia telah berbalik dan pergi. Aku sangat merindukan suaranya; kucari dia, tapi sia-sia. Kupanggil namanya, tapi ia tak menyahut.
\par 7 Aku ditemui para peronda kota; mereka memukul dan melukai aku, selendangku mereka ambil dengan paksa.
\par 8 Berjanjilah, hai putri-putri Yerusalem, bila kamu menemukan kekasihku, kabarkanlah kepadanya, bahwa aku sakit asmara.
\par 9 Apakah kekasihmu melebihi kekasih-kekasih lain, hai gadis yang paling jelita? Apakah kekasihmu melebihi kekasih-kekasih lain, sehingga engkau menyuruh kami berjanji?
\par 10 Kekasihku gagah dan tampan, unggul di antara sepuluh ribu orang.
\par 11 Kepalanya seperti emas, emas murni, rambutnya berombak dan hitam, sehitam gagak.
\par 12 Matanya bagaikan merpati pada mata air, merpati bermandi susu, duduk di tepi kolam.
\par 13 Pipinya seperti kebun rempah yang wangi, bibirnya bunga bakung yang meneteskan mur asli.
\par 14 Tangannya elok, bercincin emas dengan permata, tubuhnya bagaikan gading bertatah batu nilam.
\par 15 Kakinya seperti tiang-tiang marmer putih, dengan alas emas murni. Perawakannya segagah gunung-gunung di Libanon, dan seanggun pohon aras.
\par 16 Teramat manis tutur katanya, segala sesuatu padanya menarik. Begitulah kekasih dan sahabatku, hai putri-putri Yerusalem!

\chapter{6}

\par 1 Ke mana kekasihmu pergi, hai gadis yang paling jelita? Ke mana kekasihmu pergi, agar kami ikut mencarinya?
\par 2 Kekasihku pergi ke kebunnya, ke kebun rempah-rempah untuk menggembalakan domba-domba di kebun, dan memetik bunga-bunga bakung.
\par 3 Aku milik kekasihku, dan dia milikku, ia menggembalakan domba-domba di tengah bunga-bunga bakung.
\par 4 Engkau cantik, manisku, secantik kota Yerusalem, seelok kota Tirzah. Engkau sangat mempesonakan, seperti bala tentara dengan panji-panjinya.
\par 5 Palingkanlah wajahmu daripadaku, sebab pandanganmu membingungkan aku. Rambutmu seperti kawanan kambing yang menuruni bukit-bukit Gilead.
\par 6 Gigimu putih seperti kawanan domba yang baru dicukur dan dimandikan; berpasangan seperti anak-anak domba kembar, satu pun tak ada yang hilang.
\par 7 Pipimu seperti belahan buah delima, tersembunyi di balik cadarmu.
\par 8 Biarpun ada enam puluh permaisuri, dan delapan puluh selir, serta gadis-gadis banyak sekali,
\par 9 namun hanya satulah merpatiku, idamanku, anak tunggal kesayangan ibunya. Gadis-gadis melihat dan memuji dia, para permaisuri dan selir kagum padanya.
\par 10 Siapakah dia yang datang laksana fajar merekah, seindah bulan purnama, secerah sang surya? Ia sangat mempesonakan, seperti bala tentara dengan panji-panjinya.
\par 11 Aku datang ke kebun kenari, melihat kuntum-kuntum di lembah; entah pohon anggur sudah ada kuncupnya, dan pohon delima sudah berbunga.
\par 12 Aku sangat merindukan cintamu, seperti kurindukan naik kereta perang bersama para bangsawan.
\par 13 Menarilah, hai gadis Sulam, berputar-putarlah, supaya kami kagumi.

\chapter{7}

\par 1 O, gadis yang anggun, manis benar kakimu dengan sandal itu. Lengkung pahamu seperti perhiasan, karya seorang seniman.
\par 2 Pusarmu seperti cawan bulat yang tak pernah kekurangan anggur campur. Perutmu bagaikan timbunan gandum, dikelilingi bunga-bunga bakung.
\par 3 Buah dadamu laksana dua anak rusa, kijang kembar dua.
\par 4 Lehermu seperti menara gading. Matamu bagaikan kolam-kolam di Hesybon, dekat pintu gerbang Batrabim. Hidungmu seperti menara di Gunung Libanon, yang menghadap ke kota Damsyik.
\par 5 Kepalamu bagaikan bukit Karmel; rambutmu yang dikepang seperti lembayung, mempesonakan bahkan seorang raja.
\par 6 Sungguh cantik jelita engkau, yang tercinta di antara yang disenangi.
\par 7 Tubuhmu seanggun pohon kurma, buah dadamu gugusan buahnya.
\par 8 Ingin aku memanjat pohon kurma itu, dan memperoleh buah-buahnya. Kiranya buah dadamu seperti gugusan buah anggur, napasmu seharum buah apel,
\par 9 mulutmu semanis air anggur!
\par 10 Aku milik kekasihku, ia menginginkan aku.
\par 11 Mari kita ke padang, kekasihku, dan bermalam di ladang di tengah-tengah bunga pacar.
\par 12 Mari kita pagi-pagi ke kebun, dan melihat apakah pohon anggur sudah berkuncup, dan bunganya sudah mekar; apakah pohon delima sudah berbunga. Di sana akan kuberi cintaku kepadamu.
\par 13 Pohon arak harum semerbak baunya, di dekat pintu kita ada buah-buahan lezat, yang sudah lama dipetik dan yang baru; itu kusimpan bagimu, kekasihku.

\chapter{8}

\par 1 Ah, sekiranya engkau saudara kandungku, yang pernah menyusu pada ibuku, pasti kau kucium bila kujumpai di jalan, tak ada yang akan menghina aku.
\par 2 Engkau akan kubawa ke rumah ibuku, agar engkau dapat mengajari aku. Kau akan kuberi minum anggur harum, dan air buah delima.
\par 3 Tangan kirimu menopang aku, tangan kananmu memeluk aku.
\par 4 Berjanjilah, hai putri-putri Yerusalem, bahwa kamu takkan mengganggu cinta, sampai ia dipuaskan.
\par 5 Siapakah itu yang datang dari padang gurun, bersandar pada kekasihnya?
\par 6 Jadikanlah aku buah hatimu, jangan memeluk siapa pun selain aku; karena cinta itu sekuat maut, dan nafsu berkuasa seperti kematian. Nyalanya seperti nyala api yang berkobar dengan dahsyat.
\par 7 Air yang banyak tak mampu memadamkan cinta, dan banjir tak dapat menghanyutkannya. Jika seorang memberi segala hartanya untuk membeli cinta, pasti hanya penghinaan yang didapatnya.
\par 8 Kami mempunyai seorang adik wanita yang masih kecil buah dadanya. Apa yang harus kami buat baginya bila ada yang datang meminangnya?
\par 9 Andaikata ia sebuah tembok, akan kami dirikan menara perak di atasnya. Andaikata ia sebuah gapura, akan kami palangi dengan kayu cemara.
\par 10 Aku ini sebuah tembok, dan buah dadaku menaranya. Kekasihku tahu bahwa pada dia aku bahagia.
\par 11 Salomo mempunyai kebun anggur di Baal-Hamon; kebun itu diserahkannya kepada para penjaga. Masing-masing membayar seribu uang perak untuk hasilnya,
\par 12 dan mendapat dua ratus untuk jerih payahnya. Biarlah uang itu untukmu, Salomo, dan untuk para penjaga kebunmu. Tapi kebun anggurku, milikku sendiri, ada di hadapanku.
\par 13 Hai, penghuni kebun, teman-teman ingin mendengar suaramu; perdengarkanlah itu kepadaku!
\par 14 Kekasihku, datanglah seperti kijang, seperti anak rusa di pegunungan Beter.


\end{document}