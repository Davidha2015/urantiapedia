\begin{document}

\title{Yeremia}


\chapter{1}

\par 1 Buku ini berisi kata-kata Yeremia anak Hilkia, salah seorang imam dari kota Anatot di wilayah Benyamin.
\par 2 TUHAN berbicara kepada Yeremia pada tahun ketiga belas pemerintahan Yosia anak Amon raja Yehuda.
\par 3 Sesudah itu TUHAN berbicara lagi kepadanya berkali-kali, mulai pada masa pemerintahan Yoyakim anak Yosia sampai pada tahun kesebelas pemerintahan Zedekia anak Yosia. Pada bulan kelima tahun itu juga penduduk Yerusalem diangkut ke pembuangan.
\par 4 TUHAN berkata kepadaku,
\par 5 "Sebelum Aku membentuk engkau dalam rahim ibumu, dan sebelum engkau lahir, Aku sudah memilih dan mengangkat engkau untuk menjadi nabi bagi bangsa-bangsa."
\par 6 Aku menjawab, "Ya TUHAN Yang Mahatinggi, aku tidak pandai berbicara karena aku masih terlalu muda."
\par 7 Tetapi TUHAN menjawab, "Jangan katakan engkau masih terlalu muda. Kalau Aku mengutus engkau kepada siapa pun, kau harus pergi, dan semua yang Kusuruh kaukatakan, haruslah kausampaikan kepada mereka.
\par 8 Jangan takut kepada mereka, sebab Aku akan menyertai engkau untuk melindungi engkau. Aku, TUHAN, telah berbicara!"
\par 9 Setelah itu TUHAN mengulurkan tangan-Nya, dan menjamah mulutku lalu berkata, "Dengarlah dan umumkanlah kata-kata yang Kuucapkan ini.
\par 10 Hari ini Aku memberikan kepadamu kuasa atas bangsa-bangsa dan kerajaan-kerajaan--kuasa untuk menumbangkan dan meruntuhkan, untuk menggulingkan dan membinasakan, untuk membangun dan menanam."
\par 11 TUHAN bertanya kepadaku, "Apa yang kaulihat, Yeremia?" Aku menjawab, "Dahan pohon badam, TUHAN."
\par 12 "Benar," kata TUHAN, "Aku menjaga agar semua yang Kukatakan menjadi kenyataan."
\par 13 Lalu TUHAN bertanya lagi kepadaku, "Apa lagi yang kaulihat?" Aku menjawab, "Di sebelah utara kulihat periuk yang isinya sedang mendidih dan hampir terbalik ke arah sini."
\par 14 TUHAN berkata, "Dari utara malapetaka akan meluap dan menimpa semua penduduk negeri ini.
\par 15 Aku akan menyuruh semua bangsa di sebelah utara datang untuk menguasai Yerusalem. Tembok-tembok di sekeliling kota ini serta pintu-pintu gerbangnya dan semua kota di Yehuda akan mereka rebut.
\par 16 Aku akan menghukum umat-Ku, karena mereka berdosa dan membelakangi Aku. Mereka telah mempersembahkan kurban kepada ilah-ilah lain, dan membuat patung-patung serta menyembahnya.
\par 17 Bersiap-siaplah, Yeremia! Pergilah dan sampaikanlah kepada mereka, semua yang Kuperintahkan kepadamu. Jangan takut kepada mereka, supaya engkau jangan Kubuat lebih takut lagi kalau berhadapan dengan mereka.
\par 18 Dengarkan! Engkau akan ditentang oleh penduduk negeri ini, yaitu raja-raja Yehuda, para pejabat pemerintah, para imam, dan rakyat. Tetapi hari ini Aku memberikan kepadamu kekuatan untuk melawan mereka. Engkau akan menjadi seperti kota berbenteng, seperti tiang besi dan tembok tembaga. Mereka tidak akan dapat mengalahkan engkau, sebab Aku akan menyertai dan melindungi engkau. Aku, TUHAN, telah berbicara."

\chapter{2}

\par 1 TUHAN menyuruh aku
\par 2 mengumumkan berita ini kepada semua orang di Yerusalem, "Hai Israel, Kuingat betapa kau setia di kala engkau masih muda; betapa besar cintamu ketika kita berbulan madu. Ke padang gurun Aku kauikuti melalui daerah yang tidak ditanami.
\par 3 Israel, kau milik-Ku pribadi, buah hati-Ku yang Kusayangi. Aku menimpakan malapetaka kepada semua yang membuat kau menderita, Aku, TUHAN, telah berbicara."
\par 4 Dengarkan pesan TUHAN, hai kamu keturunan Yakub, suku-suku Israel!
\par 5 TUHAN berkata, "Kesalahan apa didapati leluhurmu pada-Ku, sehingga mereka mengkhianati Aku? Mereka mengejar berhala yang tak berharga dan akhirnya mereka sendiri menjadi hina.
\par 6 Aku tak dihiraukan; Aku, yang telah menuntun mereka di jalan, dan membawa mereka keluar dari Mesir melewati padang pasir, daerah tandus yang banyak lekak-lekuknya, serta kering lagi berbahaya--padang yang tidak dihuni dan tidak dilalui!
\par 7 Ke negeri yang subur Kuantar mereka untuk menikmati hasil-hasil dan segala yang baik di sana. Tapi di situ tanah-Ku mereka najiskan dan milik-Ku mereka jadikan sesuatu yang menjijikkan.
\par 8 Para imam tak ada yang bertanya di mana Aku, Tuhannya. Para penegak hukum pun tak tahu siapa Aku, para penguasa memberontak terhadap-Ku. Atas nama Baal para nabi berbicara, dan menyembah berhala yang tak berguna."
\par 9 TUHAN berkata, "Sekarang umat-Ku Kuperkarakan lagi, dan keturunannya Kuadili.
\par 10 Susurilah pesisir Siprus sebelah barat, kirimlah orang ke timur ke negeri Kedar, periksalah di sana dengan seksama, apakah pernah terjadi hal yang serupa?
\par 11 Belum pernah suatu bangsa menukarkan ilahnya sekali pun itu bukan Allah sesungguhnya. Tapi, umat-Ku telah menukar Aku, kebanggaan mereka dengan sesuatu yang tak dapat berbuat apa-apa.
\par 12 Karena itu, hai langit, bergoncanglah! Tercengang dan gemetarlah!
\par 13 Umat-Ku melakukan dua macam dosa: mereka membelakangi Aku, sumber air pemberi hidup bagi manusia; mereka membuat bagi dirinya kolam bocor yang tak dapat menahan airnya."
\par 14 "Israel bukan hamba, bukan juga keturunan hamba sahaya. Tapi mengapa ia telah menjadi mangsa lawannya?
\par 15 Musuhnya mengaum kepadanya seperti singa, tanahnya dijadikan tandus dan hampa, kota-kotanya habis dimakan api, dibiarkan terlantar tak berpenghuni.
\par 16 Hai Israel, rambut kepalamu dipangkas oleh orang Memfis dan Tahpanhes.
\par 17 Kau sendiri yang menyebabkan semua yang terjadi pada dirimu, karena ketika kau Kutuntun di perjalanan, kau membelakangi Aku, TUHAN Allahmu.
\par 18 Apa untungnya kau ke Mesir? Hendak minum air Sungai Nil? Apa untungnya kau ke Asyur? Hendak minum air Sungai Efrat?
\par 19 Kejahatanmu sendiri menghukum dirimu, kau tersiksa karena menolak Aku, Allahmu. Sekarang rasakan betapa pahit dan pedih bila Aku kaubelakangi dan tidak kauhormati. Aku, TUHAN Allahmu telah berbicara; Akulah TUHAN Yang Mahatinggi dan Mahakuasa."
\par 20 TUHAN berkata, "Sudah lama engkau tak mau mengabdi kepada-Ku; tak mau mentaati atau menyembah Aku. Di atas setiap bukit yang menjulang, dan di bawah setiap pohon yang rindang engkau menyembah dewa-dewa kesuburan.
\par 21 Engkau Kutanam seperti pohon anggur yang Kupilih dari benih yang unggul. Tapi sekarang engkau berubah, menjadi tanaman liar dan tak berguna.
\par 22 Sekali pun engkau mandi dengan memakai sabun banyak sekali, namun noda-noda kesalahanmu masih terlihat juga oleh-Ku.
\par 23 Berani benar kau berkata bahwa kau tidak keji, dan bahwa Baal tidak pernah kauikuti! Lihat tingkah lakumu di lembah, ingat kejahatanmu dan akuilah! Engkau bagaikan unta betina yang berahi, berlari kian ke mari tak terkendali,
\par 24 lalu masuk ke padang gurun yang sunyi. Ia tak dapat ditahan bila sedang berahi. Yang mengingininya tak perlu mencari-cari, sebab pada musim berjantan ia selalu menyodorkan diri.
\par 25 Israel, janganlah engkau berlari mengejar dewa-dewa, dan janganlah berteriak memanggil mereka, nanti kakimu cedera dan tenggorokanmu kering karena dahaga. 'Tidak, aku tak mungkin kembali,' kau berkata, 'sebab aku cinta kepada dewa-dewa, dan mau mengikuti mereka.'"
\par 26 TUHAN berkata, "Seperti pencuri merasa malu ketika tertangkap, demikianlah kamu semua akan merasa malu, hai orang Israel, raja-raja dan pejabat-pejabatmu, imam-imam dan nabi-nabimu!
\par 27 Kamu akan dipermalukan, sebab sepotong kayu kamu panggil bapak dan sebuah batu kamu panggil ibu. Kamu bukannya datang kepada-Ku, melainkan meninggalkan Aku. Tapi apabila datang kesukaran, Akulah yang kamu panggil untuk datang menyelamatkan kamu.
\par 28 Di manakah berhala-berhalamu yang kamu buat itu? Jika ada kesukaran, suruhlah mereka menyelamatkan kamu, kalau mereka bisa! Hai Yehuda, dewa-dewamu sebanyak kota-kotamu!
\par 29 Apakah yang tidak kamu senangi mengenai Aku sehingga kamu melawan Aku?
\par 30 Percuma saja kamu Kuhukum, sebab kamu tidak mau menerima teguran. Seperti singa yang sedang mengamuk, demikianlah kamu membunuh para nabimu.
\par 31 Hai umat Israel, dengarkan kata-kata-Ku ini. Pernahkah Aku seperti padang gurun bagimu atau seperti tanah yang gelap gulita? Tapi mengapa kamu berkata bahwa kamu mau bebas, dan tak mau lagi kembali kepada-Ku?
\par 32 Apakah ada gadis yang melupakan perhiasannya, atau pengantin wanita yang melupakan baju pengantinnya? Tetapi kamu telah lama sekali melupakan Aku--sejak waktu yang tak terhitung lamanya.
\par 33 Kamu sungguh pandai mengatur siasat untuk memikat hati kekasih-kekasihmu. Pelacur yang paling bejat pun masih dapat belajar daripadamu!
\par 34 Pakaianmu ternoda oleh darah orang miskin dan orang tak bersalah, bukan oleh darah pencuri yang tertangkap. Namun demikian,
\par 35 kamu berkata bahwa kamu tak bersalah, bahwa Aku tidak marah lagi kepadamu. Tetapi Aku, TUHAN, akan menghukum kamu karena kamu tidak mau mengakui bahwa kamu bersalah.
\par 36 Cepat sekali kamu pergi kepada dewa-dewa bangsa lain untuk minta tolong! Kamu pasti akan dikecewakan oleh Mesir, sama seperti kamu dikecewakan oleh Asyur.
\par 37 Kamu akan meninggalkan Mesir dengan kecewa dan malu. Aku, TUHAN, telah menolak mereka yang kamu andalkan; kamu tidak akan mendapat keuntungan apa-apa dari mereka."

\chapter{3}

\par 1 TUHAN berkata, "Apabila seorang istri diceraikan oleh suaminya, lalu wanita itu menjadi istri orang lain, maka bekas suaminya tak boleh mengambil dia kembali sebagai istri, sebab hal itu akan merusak negeri ini. Engkau Israel, sudah mempunyai banyak sekali kekasih, masakan sekarang engkau mau kembali kepada-Ku!
\par 2 Coba lihat ke atas, ke puncak-puncak bukit! Tempat manakah yang belum pernah kaudatangi untuk melacur? Seperti orang Arab di padang gurun, demikian engkau duduk di pinggir jalan menantikan orang yang mau bermain cinta. Engkau menajiskan negeri ini dengan perzinahan dan kejahatanmu.
\par 3 Itu sebabnya air dari langit tertahan, dan hujan di musim semi tidak kunjung datang. Engkau mirip pelacur, dan tak tahu malu.
\par 4 Di waktu kesukaranmu itu engkau berkata kepada-Ku bahwa Aku bapakmu, bahwa Aku mengasihi engkau sejak engkau kecil.
\par 5 Engkau berkata juga bahwa Aku tidak akan terus-menerus marah kepadamu. Itulah yang kaukatakan, hai Israel, tapi sementara itu kaulakukan juga segala kejahatan yang dapat kaulakukan."
\par 6 Pada masa pemerintahan Raja Yosia, TUHAN berkata begini kepadaku, "Sudahkah kaulihat apa yang dilakukan Israel, wanita yang tidak setia itu? Ia meninggalkan Aku dan melakukan pelacuran di atas setiap bukit yang tinggi dan di bawah setiap pohon yang rindang.
\par 7 Pikir-Ku setelah ia melakukan semuanya itu, ia akan kembali kepada-Ku. Tetapi ia tidak kembali, malah pergi melacur. Itu sebabnya Aku menceraikan dan mengusir dia. Yehuda, saudara Israel, yang tidak setia itu melihat semuanya itu, tetapi ia tidak menjadi takut. Ia malah turut menjadi pelacur,
\par 9 dan sama sekali tidak merasa malu. Ia berzinah karena menyembah batu dan pohon, sehingga negeri ini menjadi najis.
\par 10 Setelah melakukan semuanya itu, Yehuda yang tidak setia itu kembali kepada-Ku tapi dengan pura-pura, karena hatinya tidak tulus. Aku, TUHAN, telah berbicara."
\par 11 Kemudian TUHAN memberitahukan kepadaku bahwa meskipun Israel telah membelakangi Dia, namun Israel ternyata lebih baik daripada Yehuda yang tidak setia itu.
\par 12 TUHAN menyuruh aku pergi untuk mengatakan hal ini kepada Israel: "Hai Israel yang tidak setia, kembalilah kepada-Ku! Aku penuh belas kasihan dan tidak akan selamanya marah kepadamu.
\par 13 Akuilah saja bahwa engkau bersalah dan telah memberontak terhadap TUHAN, Allahmu. Akuilah bahwa di bawah setiap pohon yang rindang engkau telah mencurahkan cintamu kepada ilah-ilah bangsa lain, dan tidak mentaati perintah-perintah-Ku. Aku, TUHAN telah berbicara."
\par 14 TUHAN berkata, "Hai umat yang tidak setia, kembalilah kepada-Ku! Kamu adalah milik-Ku. Aku akan mengambil kamu, seorang dari setiap kota, dua orang dari setiap kaum, dan membawa kamu kembali ke Bukit Sion.
\par 15 Aku akan memberikan kepadamu pemimpin-pemimpin yang taat kepada-Ku. Mereka akan memimpin kamu dengan bijaksana dan penuh pengertian.
\par 16 Lalu apabila kamu sudah bertambah banyak di negeri ini, orang tidak akan berbicara lagi tentang Peti Perjanjian-Ku. Mereka tidak akan memikirkan atau mengingatnya lagi, mereka malah tidak akan memerlukannya atau membuat Peti Perjanjian yang lain.
\par 17 Pada waktu itu kota Yerusalem akan disebut 'Takhta TUHAN', dan segala bangsa akan berkumpul di kota itu untuk menyembah Aku. Mereka tidak akan lagi mengikuti kemauan hati mereka yang bebal dan jahat itu.
\par 18 Israel akan bergabung dengan Yehuda, dan bersama-sama datang dari negeri pembuangan di sebelah utara. Mereka akan kembali ke negeri yang telah Kuberikan sebagai tanah pusaka kepada leluhurmu."
\par 19 TUHAN berkata, "Israel, Aku ingin menerima engkau sebagai anak-Ku, dan memberikan kepadamu negeri yang menyenangkan, negeri yang paling bagus di seluruh dunia. Aku ingin kau menyebut Aku bapak, dan tidak lari lagi dari Aku.
\par 20 Tetapi seperti istri yang tidak setia, engkau pun tidak setia kepada-Ku. Aku, TUHAN telah berbicara."
\par 21 Di atas puncak-puncak gunung terdengar suara gaduh; itu suara umat Israel yang menangis dan memohon-mohon, karena mereka telah berdosa, dan melupakan TUHAN Allah mereka.
\par 22 Hai kamu semua yang telah meninggalkan TUHAN, kembalilah! Ia akan menyembuhkan kamu, dan menjadikan kamu setia. Kamu berkata, "Ya, sekarang kami datang kepada TUHAN, sebab Ia Allah kami.
\par 23 Sia-sia saja kami beramai-ramai menyembah berhala di atas puncak-puncak gunung! Pertolongan untuk Israel hanya datang dari TUHAN Allah kami.
\par 24 Penyembahan kepada Baal, berhala yang memalukan itu telah membuat kami kehilangan anak-anak kami serta ternak sapi dan domba, yaitu segalanya yang telah diusahakan oleh leluhur kami sejak dahulu kala.
\par 25 Jadi, biarlah kami menanggung malu, dan biarlah kami menjadi hina. Sebab, kami dan leluhur kami telah berdosa kepada TUHAN Allah kami, dan tak pernah taat kepada perintah-perintah-Nya."

\chapter{4}

\par 1 TUHAN berkata, "Umat Israel, jika kamu mau kembali, kembalilah kepada-Ku. Buanglah dahulu berhala-berhala yang Kubenci itu, dan setialah kepada-Ku.
\par 2 Jikalau kamu bersumpah demi nama-Ku dan kamu hidup jujur adil dan benar, maka segala bangsa akan minta kepada-Ku supaya Kuberkati mereka, dan mereka akan memuji Aku."
\par 3 TUHAN berkata kepada penduduk Yehuda dan Yerusalem, "Kerjakanlah tanahmu yang belum dikerjakan; jangan menabur benih di tempat tanaman berduri tumbuh.
\par 4 Peganglah janjimu dengan Aku, Tuhanmu, dan khususkanlah dirimu untuk Aku, hai penduduk Yehuda dan Yerusalem. Kalau tidak, maka kemarahan-Ku akan meluap dan membakar seperti api yang tidak dapat dipadamkan oleh siapapun juga. Semuanya itu akan menimpa dirimu karena perbuatan-perbuatanmu yang jahat."
\par 5 Tiuplah trompet di seluruh negeri! Berserulah dengan jelas dan nyaring. Suruhlah penduduk Yehuda dan Yerusalem lari ke kota-kota berbenteng.
\par 6 Tunjukkanlah jalan ke Sion, bukit TUHAN, cepatlah mengungsi, jangan tinggal diam. TUHAN sedang mendatangkan malapetaka dan kehancuran besar dari utara.
\par 7 Seperti singa keluar dari persembunyiannya begitu pula telah berangkat pemusnah bangsa-bangsa yang datang hendak membinasakan Yehuda. Kota-kota Yehuda akan menjadi reruntuhan tempat yang tak didiami orang.
\par 8 Sebab itu, menangis dan merataplah, pakailah kain karung tanda duka. Karena amarah TUHAN yang menyala-nyala belum juga surut dari Yehuda.
\par 9 TUHAN berkata, "Pada hari itu raja-raja dan pejabat-pejabat akan patah semangat; imam-imam gentar dan nabi-nabi terkejut."
\par 10 Lalu aku berkata, "TUHAN Yang Mahatinggi, Engkau sudah menipu orang Yehuda dan penduduk Yerusalem! Engkau berkata bahwa akan ada damai, padahal sekarang nyawa kami terancam pedang."
\par 11 Saatnya akan tiba orang Yehuda dan penduduk Yerusalem mendapat berita bahwa angin panas dari padang gurun sedang bertiup ke arah mereka. Angin itu bukan angin biasa yang hanya menerbangkan sekam,
\par 12 melainkan angin kuat yang datang atas perintah TUHAN sendiri untuk menjatuhkan hukuman ke atas umat-Nya.
\par 13 Lihat, musuh muncul seperti awan. Kereta-kereta perangnya seperti angin topan, kuda-kudanya lebih cepat dari burung rajawali. Wah, celaka kita! Kita binasa!
\par 14 Yerusalem, buanglah kejahatan dari hatimu supaya engkau diselamatkan. Sampai kapan pikiran-pikiran jahat itu hendak kausimpan di dalam hatimu?
\par 15 Dari kota Dan, serta dari pegunungan Efraim telah datang utusan-utusan yang membawa kabar buruk.
\par 16 Mereka memperingatkan bangsa-bangsa dan memberitahukan kepada penduduk Yerusalem bahwa musuh sedang datang dari negeri yang jauh. Mereka akan meneriakkan pekik peperangan terhadap kota-kota Yehuda
\par 17 dan mengepung Yerusalem seperti orang menjaga ladang. Semuanya itu terjadi karena bangsa Yehuda telah memberontak terhadap TUHAN. TUHAN telah berbicara.
\par 18 Yehuda, kau sendirilah yang mendatangkan bencana itu ke atas dirimu. Cara hidupmu dan perbuatan-perbuatanmu menyebabkan semua penderitaan itu menusuk hatimu.
\par 19 Betapa hatiku sengsara, tak tahan lagi aku menderita. Aku tak sanggup menenangkan diri jantungku berdebar di dalam dada, sebab kudengar trompet berbunyi, dan pekik perang bergema.
\par 20 Bencana datang bertubi-tubi menghancurkan seluruh negeri. Tiba-tiba kemahku dirusak, kain-kainnya dikoyak-koyak.
\par 21 Sampai kapan harus kusaksikan orang berperang mati-matian? Sampai kapan harus kudengarkan bunyi trompet yang memekakkan?
\par 22 TUHAN berkata, "Bodoh sekali umat-Ku itu, mereka tidak mengenal Aku. Mereka seperti anak-anak bebal belaka tanpa pengertian sedikit pun juga; mahir dalam kejahatan, gagal dalam kebaikan."
\par 23 Ke arah bumi mataku memandang, nampaknya kosong dan gersang, ke arah langit aku menengadah, wahai, tak ada yang bercahaya.
\par 24 Kulihat gunung dan bukit-bukit yang tinggi berguncang-guncang hebat sekali.
\par 25 Kulihat tak ada manusia, burung pun telah terbang semua.
\par 26 Tanah subur menjadi gurun kering, kota-kota hancur berpuing-puing, karena TUHAN sangat marah kepada umat-Nya.
\par 27 (TUHAN mengatakan bahwa seluruh dunia akan menjadi padang gurun, tetapi Ia tidak akan memusnahkannya sama sekali.)
\par 28 Bumi akan berduka, langit gelap gulita. TUHAN telah berbicara dan tak akan merubah rencana-Nya. Ia telah mengambil keputusan dan tak akan mundur setapak juga.
\par 29 Terdengar hiruk-pikuk pasukan berkuda dan pemanah, membuat setiap orang melarikan diri terengah-engah. Hutan-hutan mereka masuki, bukit batu mereka panjati. Semua kota menjadi sunyi, tak ada yang mau tinggal di situ lagi.
\par 30 Hai Yerusalem, engkau telah ditinggalkan dan tak dapat berbuat apa-apa! Untuk apa memakai baju merah dan menghias diri dengan emas perak, serta memalit mata dengan celak? Percuma kau mempersolek dirimu, sebab para kekasihmu tak sudi lagi kepadamu, malah mereka mau mencabut nyawamu.
\par 31 Aku mendengar suara rintihan seperti wanita yang mau melahirkan; seperti wanita muda melahirkan anaknya yang pertama, begitulah tangis Yerusalem yang sesak napasnya. Tangannya menggapai dan ia mengeluh, "Celakalah aku--aku hendak dibunuh musuh!"

\chapter{5}

\par 1 Hai penduduk Yerusalem, susurilah jalan-jalan di kotamu! Carilah di mana-mana, dan saksikanlah sendiri. Periksalah di alun-alun kota apakah ada satu orang jujur yang berusaha setia kepada Allah. Kalau ada, maka TUHAN akan mengampuni Yerusalem.
\par 2 Kamu berkata bahwa kamu menyembah TUHAN, padahal kamu tidak bermaksud demikian.
\par 3 Sesungguhnya yang dituntut oleh TUHAN ialah kesetiaan. TUHAN memukul kamu, tapi kamu tidak peduli. Ia meremukkan kamu, tapi kamu tidak mau diajar. Kamu keras kepala, dan tak mau bertobat dari dosa-dosamu.
\par 4 Aku berpikir, "Ah, mereka hanyalah rakyat jelata yang tidak tahu apa-apa. Mereka bertindak bodoh, karena tidak tahu apa yang diinginkan dan dituntut TUHAN Allah dari mereka.
\par 5 Baiklah aku pergi kepada para pembesar, dan berbicara dengan mereka. Pastilah mereka tahu apa yang diinginkan dan dituntut TUHAN Allah dari mereka." Tapi, mereka semua juga tidak mau diperintah oleh TUHAN; mereka tidak mau taat kepada-Nya.
\par 6 Itu sebabnya mereka akan dibunuh oleh singa dari hutan, dan dikoyak-koyak oleh serigala dari padang gurun. Macan tutul akan berkeliaran mencari mangsa di dekat kota-kota mereka, dan setiap orang yang keluar akan diterkam. Semuanya itu terjadi karena mereka telah banyak berdosa, dan berkali-kali membelakangi Allah.
\par 7 TUHAN berkata, "Mengapa Aku harus mengampuni dosa-dosa umat-Ku? Mereka telah meninggalkan Aku dan menyembah yang bukan Allah. Aku memberi makanan kepada mereka sampai mereka kenyang, tapi mereka berbuat zinah, dan sering mengunjungi pelacur-pelacur.
\par 8 Mereka seperti kuda jantan yang gemuk dan kuat nafsu berahinya; masing-masing menginginkan istri kawannya.
\par 9 Apakah tidak seharusnya Aku menghukum bangsa yang semacam itu dan membalas apa yang telah mereka lakukan?
\par 10 Aku akan mengutus musuh untuk merusak kebun anggur mereka, tapi bukan untuk memusnahkannya. Aku akan menyuruh musuh itu memangkas carang pohon-pohon anggur mereka, sebab carang-carang itu bukan milik-Ku.
\par 11 Umat Israel dan Yehuda sungguh-sungguh telah mengkhianati Aku. Aku, TUHAN, telah berbicara."
\par 12 Umat TUHAN mengatakan yang tidak benar mengenai TUHAN. Mereka berkata, "Ah, Ia tidak akan berbuat apa-apa. Kita tidak akan kena bencana dan tidak pula akan mengalami perang atau kelaparan.
\par 13 Anggaplah nabi-nabi itu angin saja, sebab pesan TUHAN tidak ada pada mereka."
\par 14 Lalu TUHAN Yang Mahakuasa berkata kepadaku, "Yeremia, karena orang-orang itu berbicara begitu, maka mereka akan kena malapetaka yang menurut mereka tidak akan terjadi. Aku akan menjadikan kata-kata-Ku seperti api di dalam mulutmu. Bangsa ini akan seperti kayu bakar, dan api itu akan membakar mereka sampai habis."
\par 15 Hai umat Israel, TUHAN sedang mendatangkan suatu bangsa dari negeri yang jauh untuk menyerang kamu. Bangsa itu sudah ada sejak dulu kala, bahasanya tidak kamu kenal. Mereka kuat sekali,
\par 16 dan pemanah-pemanah mereka gagah berani; mereka membunuh tanpa belas kasihan.
\par 17 Mereka akan menghabiskan hasil tanah dan makananmu serta membunuh anak-anakmu. Ternak sapi dan dombamu akan mereka sembelih, dan kebun-kebun anggur serta pohon-pohon aramu akan mereka musnahkan. Kota-kota berbenteng yang kamu andalkan akan mereka hancurkan.
\par 18 TUHAN berkata, "Sekalipun demikian, pada waktu itu pun umat-Ku tidak akan Kumusnahkan.
\par 19 Tetapi kalau mereka bertanya mengapa Aku melakukan semuanya itu terhadap mereka, katakanlah hai Yeremia, bahwa sebagaimana mereka meninggalkan Aku di negeri mereka sendiri untuk mengabdi kepada dewa-dewa bangsa lain, begitu pula mereka akan menjadi hamba orang-orang asing di negeri lain."
\par 20 TUHAN menyuruh aku menyampaikan pesan-Nya ini kepada keturunan Yakub yang tinggal di Yehuda,
\par 21 "Dengarkan, hai kamu orang bodoh dan tolol! Kamu punya mata tapi tidak melihat, punya telinga tapi tidak mendengar!
\par 22 Akulah TUHAN; mengapa kamu tidak takut kepada-Ku? Seharusnya kamu gemetar terhadap Aku! Akulah yang membuat pantai menjadi perbatasan laut yang tidak dapat dilampauinya untuk selama-lamanya. Laut dapat bergelora dan gelombang dapat mengamuk, tapi tidak dapat melewati perbatasan itu.
\par 23 Tetapi kamu keras kepala dan suka melawan; kamu telah menyeleweng dan meninggalkan Aku.
\par 24 Tidak pernah kamu ingat untuk menghormati Aku, padahal Akulah yang mengirim hujan, baik pada awal maupun pada akhir musim hujan. Aku pula yang memberikan kepadamu musim panen setiap tahun.
\par 25 Tetapi dosa-dosamu itulah yang menghambat semua yang baik itu sehingga kamu tidak menikmatinya.
\par 26 Hai umat-Ku, di antara kamu ada orang-orang jahat. Seperti orang menunggui jerat yang dipasangnya untuk menangkap burung, begitulah mereka memasang jerat untuk menangkap manusia.
\par 27 Seperti pemburu burung mengisi sangkarnya dengan burung, begitu pula mereka mengisi rumah mereka dengan barang-barang hasil tipuan. Demikianlah mereka menjadi berkuasa dan kaya
\par 28 serta gemuk dan makmur. Kejahatan mereka tak ada batasnya. Mereka tidak berlaku adil dalam perkara anak-anak yatim piatu, dan tidak membela hak orang-orang yang tertindas.
\par 29 Tetapi, Aku, TUHAN, akan menghukum mereka karena semuanya itu; Aku akan membalas perbuatan-perbuatan bangsa ini.
\par 30 Sesuatu yang dahsyat dan mengerikan telah terjadi di negeri ini:
\par 31 nabi-nabi berdusta, imam-imam melayani menurut kemauannya sendiri dan kamu umat-Ku suka akan hal itu. Tetapi apa yang akan kamu lakukan bila tiba hari terakhir?"

\chapter{6}

\par 1 Pergilah mengungsi, hai orang-orang Benyamin! Tinggalkanlah Yerusalem! Bunyikan trompet di Tekoa, dan nyalakan api isyarat di Bet-Kerem! Bencana dan malapetaka besar mengancam dari utara.
\par 2 Kota Sion indah sekali, tetapi TUHAN akan menghancurkannya;
\par 3 penguasa-penguasa akan berkemah di sana dengan tentaranya. Di sekeliling kota itu mereka akan memasang tenda-tenda mereka, masing-masing di tempat yang disukainya.
\par 4 Mereka akan berkata, "Ayo, bersiaplah menyerang Yerusalem! Mari kita menyerbunya siang ini!" Tetapi kemudian mereka akan berkata pula, "Wah, terlambat! Hari sudah mulai gelap, bayangan-bayangan sudah mulai memanjang.
\par 5 Baiklah kita menyerbu pada waktu malam, dan menghancurkan benteng-benteng kota itu."
\par 6 TUHAN Yang Mahakuasa telah memerintahkan penguasa-penguasa itu supaya menebang pohon-pohon dan mendirikan bukit-bukit pertahanan untuk mengepung Yerusalem. TUHAN berkata, "Kota ini akan Kuhukum karena di dalamnya telah dilakukan banyak sekali penindasan.
\par 7 Sebagaimana mata air tetap mengeluarkan air segar, begitu pula kota itu tetap menghasilkan kejahatan. Hanya berita kekerasan dan perampokan belaka yang Kudengar dari kota itu; luka dan penyakit Kulihat di mana-mana.
\par 8 Hai penduduk Yerusalem, hendaknya semua kesukaran itu menjadi peringatan kepadamu! Jangan sampai Aku meninggalkan kamu dan Kuubah kotamu menjadi padang gurun yang tidak didiami orang."
\par 9 TUHAN Yang Mahakuasa berkata kepadaku, "Israel akan digunduli sampai tandas seperti kebun anggur yang buahnya sudah dipetik semua. Karena itu, sementara masih ada waktu, selamatkanlah yang sisa."
\par 10 Aku menjawab, "TUHAN, kepada siapa aku harus berbicara, dan siapa yang harus kuperingatkan? Mereka keras kepala dan tidak mau memperhatikan. Mereka menertawakan dan membenci pesan-Mu yang kusampaikan kepada mereka.
\par 11 Kemarahan-Mu kepada mereka juga membara di dalam diriku, TUHAN! Sukar untuk menahannya." Lalu TUHAN berkata, "Lampiaskan kemarahan itu ke atas anak-anak di jalan-jalan raya dan ke atas orang-orang muda yang sedang berkumpul. Laki-laki dan perempuan akan ditangkap, dan orang lanjut usia pun tidak akan luput.
\par 12 Rumah mereka akan diberikan kepada orang lain, begitu juga ladang-ladang dan istri-istri mereka. Penduduk negeri ini akan Kuhukum.
\par 13 Mereka semua, besar kecil, mencari keuntungan dengan jalan yang tidak jujur. Bahkan nabi dan imam pun menipu orang.
\par 14 Luka-luka umat-Ku mereka anggap luka kecil saja. 'Ah, tidak apa-apa,' kata mereka, 'semuanya baik dan aman.' Padahal sama sekali tidak.
\par 15 Seharusnya mereka malu karena telah melakukan semua yang hina itu. Tetapi mereka tebal muka dan tidak tahu malu. Karena itu mereka akan jatuh seperti orang lain; habislah riwayat mereka apabila Aku menghukum mereka. Aku, TUHAN, telah berbicara."
\par 16 TUHAN berkata kepada umat-Nya, "Berdirilah di persimpangan-persimpangan jalan, dan perhatikanlah baik-baik! Tanyakanlah jalan-jalan yang dahulu kala, dan mana yang terbaik di antaranya. Ikutilah jalan itu, supaya kamu hidup tenang." Tetapi mereka berkata, "Tidak, kami tidak mau mengikuti jalan itu!"
\par 17 Lalu TUHAN berkali-kali mengangkat penjaga-penjaga untuk mengingatkan mereka supaya mendengarkan bunyi isyarat trompet. Tetapi umat berkata, "Kami tidak mau mendengarkan."
\par 18 Sebab itu TUHAN berkata, "Dengarkan, hai bangsa-bangsa! Kamu harus tahu apa yang akan terjadi atas umat-Ku.
\par 19 Perhatikan, hai bumi! Umat-Ku tidak mau menerima ajaran-Ku dan tidak taat kepada-Ku. Sebagai akibat dari rancangan mereka Aku mendatangkan malapetaka ke atas mereka.
\par 20 Buat apa kemenyan yang mereka bawa untuk-Ku dari Syeba, atau rempah-rempah dari negeri jauh? Tak sudi Aku menerima persembahan mereka. Tak suka Aku kepada kurban-kurban bakaran mereka.
\par 21 Aku akan menyebabkan orang-orang ini tersandung dan jatuh. Ayah dan anak, kawan dan tetangga semuanya akan binasa."
\par 22 TUHAN berkata, "Suatu bangsa yang kuat sedang bergerak dari negeri yang jauh di utara. Mereka datang untuk berperang.
\par 23 Mereka bersenjatakan panah dan tombak; mereka bengis dan tak kenal ampun. Seperti bunyi laut bergelora begitulah suara derap kuda mereka yang sedang dipacu untuk maju menyerang Yerusalem."
\par 24 Mendengar berita itu, penduduk Yerusalem menjadi tak berdaya. Mereka dicekam perasaan takut, dan menderita seperti wanita yang mau melahirkan.
\par 25 Mereka tak berani turun ke jalan atau pergi ke ladang, sebab di mana-mana musuh yang bersenjata menimbulkan teror.
\par 26 TUHAN berkata kepada umat-Nya, "Pakailah kain karung dan berguling-gulingl dalam abu. Berkabunglah dan merataplah dengan pedih seperti orang menangisi kematian anaknya yang tunggal, sebab pembinasa akan menyerang kamu dengan tiba-tiba."
\par 27 Lalu TUHAN berkata kepada Yeremia, "Ujilah umat-Ku, seperti engkau menguji logam. Selidikilah tabiat mereka!
\par 28 Mereka semua pemberontak yang keras kepala, sekeras perunggu dan besi. Mereka berlaku busuk dan berjalan ke mana-mana untuk menyebarkan berita yang menjelekkan orang lain.
\par 29 Peleburan sudah sangat panas, tapi kotoran logam tak mau juga meleleh dan hilang. Percuma umat-Ku terus-menerus diuji untuk dimurnikan, sebab orang-orangnya yang jahat tidak disingkirkan.
\par 30 Mereka akan dinamakan perak buangan, sebab Aku, TUHAN, telah membuang mereka."

\chapter{7}

\par 1 TUHAN menyuruh aku pergi ke Rumah-Nya dan berdiri di pintu gerbang yang dilalui orang Yehuda yang datang berbakti. Ia menyuruh aku mengumumkan pesan-Nya ini kepada mereka, "Ubahlah kelakuan dan cara hidupmu, maka Aku, TUHAN Yang Mahakuasa, Allah Israel akan mengizinkan kamu tetap tinggal di sini.
\par 4 Janganlah percaya kepada penipu yang berkata, 'Kita aman, sebab ini Rumah TUHAN. Sungguh ini Rumah TUHAN!'
\par 5 Ubahlah hidupmu dan kelakuanmu. Berlakulah adil seorang terhadap yang lain.
\par 6 Jangan lagi menindas orang asing, anak yatim piatu, dan janda-janda. Jangan membunuh orang yang tidak bersalah di negeri ini. Jangan menyembah ilah-ilah lain, sebab hal itu akan membinasakan kamu.
\par 7 Kalau kamu mengubah cara hidupmu dan kelakuanmu serta berhenti melakukan semuanya itu, Aku akan mengizinkan kamu tetap tinggal di negeri ini yang telah Kuberikan sebagai tanah pusaka kepada leluhurmu.
\par 8 Tetapi, sesungguhnya percuma saja kamu percaya kepada kata-kata dusta.
\par 9 Kamu mencuri, membunuh, berzinah, memberi kesaksian dusta, mempersembahkan kurban kepada Baal, dan menyembah dewa-dewa yang tidak kamu kenal.
\par 10 Setelah semua itu kamu lakukan, kamu datang menghadap Aku di sini di Rumah-Ku sendiri, dan berkata, 'Kami aman!' Kemudian kamu melanjutkan perbuatanmu yang keji itu.
\par 11 Apakah kamu menganggap Rumah-Ku ini sarang perampok? Semua perbuatanmu sudah Kulihat.
\par 12 Pergilah ke Silo, kota yang dahulu Kupilih menjadi tempat ibadat kepada-Ku. Lihatlah apa yang telah Kulakukan terhadap tempat itu karena dosa umat-Ku Israel.
\par 13 Sekalipun berkali-kali Aku berbicara kepadamu, kamu tidak mau mendengarkan. Kamu melakukan semua dosa itu, dan tak mau menyahut jika Aku memanggil.
\par 14 Nah, Rumah-Ku yang kamu andalkan, dan tempat yang Kuberikan kepada leluhurmu dan kepadamu ini akan Kuperlakukan seperti Silo dahulu.
\par 15 Kamu akan Kuusir dari hadapan-Ku seperti telah Kuusir sanak keluargamu, umat Israel."
\par 16 TUHAN berkata, "Yeremia, janganlah mendoakan orang-orang ini. Jangan minta tolong atau berdoa untuk mereka; jangan mendesak Aku, sebab Aku tidak mau mendengarkan permohonan itu.
\par 17 Engkau sudah melihat sendiri apa yang mereka lakukan di kota-kota Yehuda dan di jalan-jalan di Yerusalem!
\par 18 Anak-anak mengumpulkan kayu bakar, orang laki-laki menyalakan api, dan orang perempuan mencampur adonan untuk membuat kue bagi dewi yang mereka namakan Ratu Surga. Dan untuk menyakiti hati-Ku mereka juga mempersembahkan air anggur kepada ilah-ilah lain.
\par 19 Tetapi sesungguhnya, bukan Aku yang mereka sakiti, melainkan diri mereka sendiri sehingga mereka malu.
\par 20 Aku, TUHAN Yang Mahatinggi, akan menimpakan murka-Ku yang dahsyat ke atas rumah ibadat ini, atas manusia dan binatang, juga atas pohon-pohon dan hasil bumi. Kemarahan-Ku akan berkobar-kobar seperti api yang tidak dapat dipadamkan."
\par 21 TUHAN berkata, "Hai umat-Ku, kamu tahu bahwa persembahan yang kamu bakar di atas mezbah harus dibakar sampai habis, dan persembahan lainnya boleh kamu makan. Tetapi Aku, TUHAN berkata, silakan makan saja semuanya!
\par 22 Ketika Aku membawa leluhurmu keluar dari Mesir, Aku tidak memberi mereka peraturan mengenai kurban bakaran atau kurban-kurban lainnya.
\par 23 Aku hanya memerintahkan mereka untuk mentaati Aku supaya Aku menjadi Allah mereka, dan mereka menjadi umat-Ku. Aku menyuruh mereka hidup menurut perintah-perintah-Ku supaya mereka bahagia.
\par 24 Tetapi mereka tidak mau mendengarkan dan tidak mau memperhatikan. Mereka malah menuruti kemauan hati mereka sendiri yang keras dan jahat itu. Mereka bukannya menjadi baik melainkan menjadi lebih jahat.
\par 25 Sejak leluhurmu keluar dari Mesir sampai hari ini Aku selalu mengutus hamba-hamba-Ku, para nabi, kepadamu.
\par 26 Tapi tak ada yang mendengarkan atau memperhatikan. Kamu malah menjadi lebih keras kepala dan jahat daripada leluhurmu."
\par 27 Kemudian TUHAN berkata, "Karena itu, hai Yeremia, sekalipun engkau memberitahukan semuanya itu kepada umat-Ku, mereka tidak akan memperhatikan kata-katamu. Sekalipun engkau memanggil mereka, mereka tidak akan menjawab.
\par 28 Jadi, katakanlah kepada mereka bahwa mereka adalah suatu bangsa yang tidak mau taat kepada Aku, TUHAN, Allah mereka. Mereka tidak mau belajar dari hukuman yang telah dirasakannya. Kesetiaan kepada-Ku sudah lenyap, malah tidak dibicarakan lagi."
\par 29 TUHAN berkata, "Wahai penduduk Yerusalem, berkabunglah! Potonglah rambutmu dan buanglah! Nyanyikanlah nyanyian ratapan di atas puncak-puncak di pegunungan, sebab Aku, TUHAN, sedang murka dan telah menolak kamu semua.
\par 30 Orang Yehuda telah melakukan kejahatan. Berhala-berhala mereka yang Kubenci itu telah mereka tempatkan di dalam Rumah-Ku, sehingga menajiskannya.
\par 31 Sebuah mezbah yang bernama Tofet telah mereka dirikan di Lembah Hinom untuk mempersembahkan anak-anak mereka sebagai kurban bakaran. Padahal Aku tak pernah menyuruh mereka melakukan hal itu, bahkan tak pernah hal semacam itu timbul dalam pikiran-Ku.
\par 32 Karena itu, akan tiba waktunya tempat itu tidak lagi disebut Tofet atau Lembah Hinom, melainkan Lembah Pembantaian. Tanah itu akan menjadi tanah pekuburan karena tak ada tempat yang lain.
\par 33 Mayat-mayat mereka akan menjadi makanan burung dan binatang buas, tanpa ada yang mengusiknya.
\par 34 Negeri ini akan menjadi padang tandus. Suara kegembiraan dan suara pesta perkawinan akan Kuhentikan di kota-kota Yehuda dan di jalan-jalan Yerusalem.

\chapter{8}

\par 1 Pada masa itu tulang-tulang raja-raja dan pejabat-pejabat pemerintahan di Yehuda, juga tulang-tulang para imam, nabi, dan penduduk Yerusalem lainnya akan dikeluarkan dari kuburan mereka,
\par 2 bukan untuk dikumpulkan dan dikuburkan lagi, melainkan dibiarkan di tanah sebagai sampah. Tulang-tulang orang-orang itu akan terserak di depan matahari, bulan dan bintang-bintang yang mereka dewakan, cintai, layani, sembah, dan mintai petunjuk.
\par 3 Maka orang-orang yang masih hidup dari bangsa yang jahat ini yang telah Kuceraiberaikan ke mana-mana akan lebih suka mati daripada hidup. Aku, TUHAN Yang Mahakuasa, telah berbicara."
\par 4 TUHAN menyuruh aku mengatakan begini kepada umat-Nya, "Kalau orang jatuh, tentu ia akan bangun lagi, dan kalau orang sesat di jalan, tentu ia akan kembali.
\par 5 Tapi mengapa kamu, umat-Ku, terus-menerus menyeleweng dan meninggalkan Aku? Kamu berpegang teguh kepada berhalamu, dan tak mau kembali kepada-Ku!
\par 6 Aku telah mendengarkan baik-baik, tapi kamu tidak berbicara sebagaimana mestinya. Tidak seorang pun dari kamu menyesali perbuatannya yang jahat. Tak ada yang berkata, 'Apa kesalahanku?' Kamu semua mengikuti jalanmu sendiri seperti kuda menyerbu ke dalam pertempuran.
\par 7 Burung bangau pun tahu waktunya untuk kembali; tekukur, burung layang-layang dan murai juga tahu masanya untuk berpindah tempat. Tetapi kamu, umat-Ku, tidak mengenal hukum-hukum yang Kuberikan kepadamu.
\par 8 Berani benar kamu berkata bahwa kamu bijaksana, dan tahu hukum-hukum-Ku! Coba lihat bagaimana hukum-hukum-Ku itu diubah oleh ahli-ahli agama yang curang!
\par 9 Para cerdik pandai di tengah-tengahmu akan dipermalukan. Mereka akan hilang akal dan terjebak. Mereka tidak mau menerima nasihat-Ku; sekarang, kebijaksanaan apakah yang masih ada pada mereka?
\par 10 Karena itu Aku akan memberikan ladang mereka kepada tuan tanah lain, dan istri mereka kepada orang lain. Mereka semua, besar kecil, mencari keuntungan dengan jalan yang tidak jujur. Bahkan nabi dan imam pun menipu orang.
\par 11 Luka-luka umat-Ku mereka anggap luka kecil saja. 'Ah, tidak apa-apa,' kata mereka, 'semua baik dan aman.' Padahal sama sekali tidak.
\par 12 Hai umat-Ku, seharusnya kamu malu karena telah melakukan semua yang hina itu! Tetapi kamu tebal muka dan tidak tahu malu. Karena itu kamu akan jatuh seperti orang lain yang telah jatuh; habislah riwayatmu apabila Aku menghukum kamu.
\par 13 Ingin sekali Aku mengumpulkan umat-Ku seperti orang mengumpulkan hasil tanahnya. Tetapi, umat-Ku seperti pohon anggur dan pohon ara yang tidak berbuah, bahkan daun-daunnya pun sudah layu. Itu sebabnya Aku telah mengizinkan orang asing datang dan menguasai negeri mereka. Aku, TUHAN, telah berbicara."
\par 14 "Mengapa kita tidak berbuat apa-apa?" kata umat Allah. "Mari kita bersama-sama mengungsi ke kota-kota berbenteng, biar kita mati di sana. TUHAN Allah kita sudah memutuskan bahwa kita harus binasa. Ia sudah memberikan kepada kita racun untuk diminum sebab kita berdosa kepada-Nya.
\par 15 Percuma saja kita mengharapkan kedamaian dan masa penyembuhan, sebab yang ada hanyalah kekerasan.
\par 16 Musuh telah berada di kota Dan; ringkikan kuda mereka sudah terdengar, dan seluruh negeri menjadi takut. Musuh datang untuk menghancurkan seluruh negeri kita, juga kota kita dan semua penduduknya."
\par 17 "Perhatikan!" kata TUHAN. "Sekarang Aku mengirim ular-ular ke tengah-tengahmu untuk memagut kamu, ular berbisa yang tak dapat kena mantra."
\par 18 Hatiku tersayat dan pedih, nestapaku tak terobati.
\par 19 Dengar! Di seluruh negeri menggema jerit tangis bangsaku yang bertanya, "Tiadakah TUHAN di Sion lagi? Di manakah rajanya kini?" Jawab TUHAN, raja mereka, "Mengapa kausujud kepada dewa-dewa yang sama sekali tak berguna? Mengapa kausakiti hati-Ku dengan menyembah berhala-berhala?"
\par 20 Umat Israel berseru, "Musim menuai dan musim kemarau kini sudah lewat, tapi kita belum juga selamat."
\par 21 Hatiku remuk karena bangsaku sakit parah; aku berkabung dan gelisah.
\par 22 Apa gerangan yang menyebabkan sehingga bangsaku sampai sekarang belum juga sembuh? Apakah di seluruh Gilead, tak ada dokter atau obat?

\chapter{9}

\par 1 Alangkah baiknya jika kepalaku seperti sumur yang penuh air, dan mataku pancuran yang terus mengalir! Maka aku dapat menangis siang dan malam, meratapi orang-orang sebangsaku yang dibunuh lawan.
\par 2 Sekiranya di padang gurun dapat kutemukan tempat menginap bagi orang yang dalam perjalanan! Maka aku akan meninggalkan bangsaku, dan lari ke sana untuk menjauh dari mereka. Mereka pengkhianat semuanya, suatu kumpulan orang yang tak setia,
\par 3 dan selalu siap untuk berdusta. Di seluruh negeri, kebenaran tidak berkuasa, bahkan ketidakjujuran merajalela. TUHAN berkata, "Umat-Ku melakukan kejahatan dengan tak henti-hentinya, mereka tidak mengakui Aku sebagai Allah mereka."
\par 4 Terhadap kawan, setiap orang harus waspada dan saudara sekandung pun tak dapat dipercaya. Sebab, seperti Yakub demikianlah mereka, masing-masing menipu saudaranya. Setiap orang berjalan ke mana-mana untuk menyebarkan fitnah dan dusta.
\par 5 Semuanya saling menipu, tak ada yang berbicara jujur. Mulutnya telah terbiasa berdusta, mereka tak sanggup meninggalkan dosanya. Kekejaman demi kekejaman, penipuan demi penipuan terus-menerus mereka lakukan, mereka sama sekali menolak TUHAN.
\par 7 Sebab itu TUHAN Yang Mahakuasa berkata, "Seperti logam, umat-Ku akan Kuuji supaya mereka menjadi murni. Apa lagi yang dapat Kulakukan terhadap umat-Ku yang telah berbuat kejahatan?
\par 8 Kata-kata mereka sangat berbahaya, mampu mematikan seperti panah berbisa. Ucapan-ucapan mereka dusta belaka. Kata-kata mereka manis memikat, tetapi sebenarnya mereka memasang jerat.
\par 9 Masakan ke atas bangsa semacam ini tak Kutimpakan hukuman yang sesuai? Akan Kubalas perbuatan mereka; Aku, TUHAN, telah berbicara!"
\par 10 Aku, Yeremia, berkata, "Aku hendak menangis dan berkabung karena padang rumput di gurun dan di gunung-gunung yang telah kering kerontang serta tak dilalui orang. Suara ternak tak terdengar lagi, unggas dan binatang-binatang lainnya telah lari dan lenyap."
\par 11 TUHAN berkata, "Yerusalem akan Kujadikan reruntuhan tempat bersembunyi anjing hutan. Kota-kota Yehuda akan menjadi sunyi, tempat yang tidak lagi dihuni."
\par 12 Aku bertanya, "TUHAN, mengapa negeri ini dibiarkan tandus dan kering seperti padang gurun sehingga tak ada yang melaluinya? Apakah ada yang cukup pandai untuk mengerti semuanya ini? Kepada siapakah telah Kauberi penjelasan supaya ia dapat meneruskannya kepada orang lain?"
\par 13 TUHAN menjawab, "Itu terjadi karena umat-Ku tidak lagi memperhatikan ajaran-ajaran yang Kuberikan kepada mereka. Mereka tidak taat kepada-Ku, dan tidak mau menuruti perintah-Ku.
\par 14 Mereka keras kepala dan menyembah berhala-berhala Baal seperti yang diajarkan oleh leluhur mereka.
\par 15 Karena itu, dengarkanlah apa yang Aku, TUHAN Yang Mahakuasa, Allah Israel, akan lakukan: umat-Ku akan Kuberi tanaman pahit untuk dimakan dan racun untuk diminum.
\par 16 Mereka akan Kuserakkan ke antara bangsa-bangsa yang tidak dikenal oleh mereka atau leluhur mereka. Aku akan terus mengirim tentara yang menggempur mereka sampai Aku membinasakan mereka sama sekali."
\par 17 TUHAN Yang Mahakuasa berkata kepada umat-Nya, "Perhatikanlah apa yang sedang terjadi di negeri ini, dan segeralah panggil para peratap. Suruhlah mereka datang dan menyanyikan lagu perkabungan."
\par 18 Orang Yehuda berkata, "Suruh mereka segera menyanyikan lagu perkabungan untuk kita sekalian, supaya berlinang-linang air mata membasahi pipi kita."
\par 19 Di Sion terdengar suara tangisan, "Kita binasa dan sangat dipermalukan. Negeri kita harus kita tinggalkan, rumah-rumah kita hancur berantakan."
\par 20 Aku berkata, "Dengarkan kata-kata TUHAN, hai wanita sekalian! Kepada putri-putrimu ajarkan nyanyian ratapan, kepada kawan-kawanmu lagu perkabungan.
\par 21 Maut telah masuk melalui jendela, telah menerobos ke dalam istana; ia menerkam anak-anak di jalan serta orang muda di lapangan.
\par 22 Mayat terserak di mana-mana seperti sampah di atas tanah; seperti gandum yang telah dipotong lalu ditinggalkan dan tidak lagi dikumpulkan. Itulah pesan TUHAN yang harus kuberitakan."
\par 23 TUHAN berkata, "Orang arif tak boleh bangga karena kebijaksanaannya, orang kuat karena kekuatannya, dan orang kaya karena kekayaannya.
\par 24 Siapa mau berbangga tentang sesuatu, haruslah berbangga bahwa ia mengenal dan mengerti Aku; bahwa ia tahu Aku mengasihi untuk selama-lamanya dan Aku menegakkan hukum serta keadilan di dunia. Semuanya itu menyenangkan hati-Ku. Aku, TUHAN, yang mengatakan itu."
\par 25 TUHAN berkata, "Akan tiba masanya Aku menghukum orang Mesir, Yehuda, Edom, Amon, Moab, dan penduduk padang gurun yang memotong pendek rambutnya. Mereka semua disunat sama seperti orang Israel, tapi baik mereka maupun bangsa Israel berlaku seperti orang-orang yang tidak disunat, karena mereka tidak mentaati perjanjian yang dilambangkan oleh sunat itu."

\chapter{10}

\par 1 Umat Israel, dengarkanlah pesan TUHAN kepadamu.
\par 2 Ia berkata, "Jangan ikuti kebiasaan bangsa-bangsa lain, dan jangan gentar melihat tanda-tanda di langit yang ditakuti bangsa-bangsa itu.
\par 3 Adat-istiadat bangsa-bangsa itu sama sekali tak berguna. Mereka menebang pohon di hutan, dan menyuruh tukang kayu mengerjakannya dengan alat menjadi patung berhala.
\par 4 Kemudian patung berhala itu dihiasi dengan emas dan perak, lalu dikuatkan dengan paku supaya jangan jatuh.
\par 5 Patung-patung berhala itu seperti orang-orangan di kebun mentimun. Mereka tak dapat bicara, tak dapat melangkah, dan harus diangkat. Jangan takut kepada mereka, sebab mereka tak dapat berbuat jahat kepadamu; berbuat baik pun mereka tak mampu."
\par 6 TUHAN yang mulia, Engkau tak ada taranya, nama-Mu agung dan penuh kuasa.
\par 7 Siapa gerangan tak mau takwa kepada-Mu, raja segala bangsa? Sungguh, Engkau patut sekali dihormat. Dari semua yang pandai di antara bangsa-bangsa dan dari semua raja-raja mereka, tiada seorang pun dapat disamakan dengan diri-Mu, ya TUHAN.
\par 8 Bebal dan bodoh mereka semua sebab dewanya hanya kayu belaka, tak dapat mengajar yang berguna.
\par 9 Ia dilapis dengan perak dan emas yang dibawa dari Spanyol dan kota Ufas, semuanya buah tangan para seniman dan pandai emas. Pakaiannya dari kain ungu tua dan ungu muda, dikerjakan oleh para ahli tenun dengan seksama.
\par 10 Tapi Engkau, TUHAN, Allah yang benar, Allah pemberi hidup, raja kekal. Bila Engkau murka, bumi berguncang, jika Engkau geram bangsa-bangsa tak tahan.
\par 11 (Kamu, hai umat, harus memberitahukan kepada mereka bahwa dewa-dewa yang tidak menciptakan langit dan bumi itu akan dibinasakan. Mereka akan lenyap dari muka bumi.)
\par 12 TUHAN menciptakan bumi dengan kuasa-Nya, membentuk dunia dengan hikmat-Nya, dan membentangkan langit dengan akal budi-Nya.
\par 13 Hanya dengan memberi perintah, menderulah air di cakrawala. Dari ujung-ujung bumi didatangkan-Nya awan, dan dibuat-Nya kilat memancar di dalam hujan, serta dikirim-Nya angin dari tempat penyimpanan-Nya.
\par 14 Melihat semua itu sadarlah manusia, bahwa ia bodoh dan tak punya pengertian. Para pandai emas kehilangan muka, sebab patung berhala buatannya itu palsu dan tak bernyawa.
\par 15 Berhala-berhala itu tak berharga, patut diejek dan dihina. Apabila tiba waktunya mereka akan binasa.
\par 16 Allah Yakub tidak seperti berhala-berhala itu; Ia pencipta segala sesuatu. Nama-Nya ialah TUHAN Yang Mahakuasa; Ia telah memilih Israel menjadi umat-Nya.
\par 17 Penduduk Yerusalem, kamu dikepung! Karena itu, berkemas-kemaslah!
\par 18 Kali ini TUHAN akan melemparkan seluruh bangsa Yehuda ke pembuangan; kamu semua akan dihancurkan-Nya sama sekali. Itulah pesan dari TUHAN.
\par 19 Penduduk Yerusalem berseru, "Kami luka parah dan tidak dapat sembuh. Dahulu kami menyangka bahwa derita ini dapat kami tanggung, padahal tidak.
\par 20 Kemah kami rusak, dan tali-talinya putus semua. Tak ada lagi yang dapat mendirikan kemah dan memasang kain-kainnya, karena anak-anak telah pergi."
\par 21 Aku menjawab, "Bodoh pemimpin-pemimpin kita itu. Mereka tidak minta petunjuk dari TUHAN, itu sebabnya mereka gagal, dan kita terserak ke mana-mana.
\par 22 Dengarlah! Ada berita bahwa sebuah bangsa di utara menimbulkan kegemparan. Tentaranya akan membuat kota-kota Yehuda menjadi sepi dan tak berpenghuni. Anjing-anjing hutan akan bersembunyi di sana."
\par 23 TUHAN, aku tahu tak seorang pun berkuasa menentukan nasibnya atau mengendalikan jalan hidupnya.
\par 24 Hajarlah kami ya TUHAN, tapi janganlah dengan pukulan yang terlalu menyakitkan. Jangan juga menghukum kami bila Engkau sedang murka, sebab pasti kami akan binasa semua.
\par 25 Lampiaskanlah kemarahan-Mu ke atas bangsa-bangsa yang tidak mengakui Engkau dan yang tak mau berbakti kepada-Mu. Sebab mereka telah membinasakan umat-Mu sama sekali dan menghancurkan negeri kami.

\chapter{11}

\par 1 TUHAN berkata kepadaku,
\par 2 "Perhatikanlah syarat-syarat perjanjian yang Kubuat dengan umat-Ku. Beritahukanlah kepada orang Yehuda dan penduduk Yerusalem bahwa
\par 3 Aku, TUHAN Allah Israel, mengutuk siapa saja yang tidak mentaati syarat-syarat perjanjian-Ku itu.
\par 4 Perjanjian itu telah Kubuat dengan leluhur mereka ketika mereka Kubawa keluar dari Mesir, negeri yang seperti neraka bagi mereka. Aku menyuruh mereka taat kepada-Ku dan melakukan segala yang telah Kuperintahkan. Sekarang sudah Kukatakan juga kepada orang Yehuda dan penduduk Yerusalem bahwa jika mereka taat kepada-Ku, mereka akan menjadi umat-Ku dan Aku Allah mereka.
\par 5 Kalau mereka melakukan semuanya itu, Aku akan menepati janji-Ku kepada leluhur mereka bahwa negeri yang subur dan makmur, yang sekarang mereka tempati itu, akan Kuberikan kepada mereka." Aku menjawab, "Baik, TUHAN."
\par 6 Lalu TUHAN berkata kepadaku, "Pergilah ke kota-kota di Yehuda, dan ke jalan-jalan di Yerusalem. Sampaikanlah pesan-Ku ini kepada orang-orang di sana. Suruh mereka memperhatikan dan mentaati syarat-syarat perjanjian-Ku.
\par 7 Ketika Aku membawa leluhur mereka keluar dari Mesir, dengan tegas Aku memperingatkan mereka supaya taat kepada-Ku. Sampai hari ini pun Aku masih terus memperingatkan umat-Ku mengenai hal itu,
\par 8 dan menyuruh mereka mentaati perjanjian-Ku itu, tetapi mereka tidak mau mendengarkan dan tak mau taat. Malahan mereka tetap keras kepala dan jahat seperti dahulu. Karena itu segala hukuman yang tercantum dalam perjanjian itu telah Kutimpakan ke atas mereka."
\par 9 Kemudian TUHAN berkata lagi kepadaku, "Orang Yehuda dan penduduk Yerusalem telah bersepakat melawan Aku.
\par 10 Mereka telah melakukan dosa yang dilakukan oleh leluhur mereka yang tidak mau taat kepada-Ku, mereka menyembah ilah-ilah lain dan tidak mau menuruti perintah-Ku. Baik Israel maupun Yehuda melanggar perjanjian yang telah Kubuat dengan leluhur mereka.
\par 11 Karena itu, Aku, TUHAN, memperingatkan bahwa Aku akan mendatangkan malapetaka ke atas mereka dan mereka tak dapat menghindarinya. Jika mereka berseru minta tolong kepada-Ku, Aku tidak mau mendengarkan.
\par 12 Lalu orang Yehuda dan penduduk Yerusalem pergi minta tolong kepada ilah-ilah lain yang mereka puja dengan persembahan kurban. Ilah-ilah itu tidak akan mampu menyelamatkan mereka pada waktu malapetaka itu datang.
\par 13 Ilah-ilah orang Yehuda banyak sekali, sebanyak kota-kota mereka. Penduduk Yerusalem mendirikan mezbah-mezbah sebanyak jalan-jalan di kota Yerusalem untuk mempersembahkan kurban kepada Dewa Baal.
\par 14 Karena itu, Yeremia, janganlah berdoa untuk mereka atau minta tolong kepada-Ku untuk mereka! Kalau mereka dalam kesukaran dan minta tolong kepada-Ku, Aku tidak akan mau mendengarkan mereka."
\par 15 TUHAN berkata, "Umat-Ku yang Kukasihi telah melakukan hal-hal yang jahat. Mereka tidak berhak berada di Rumah-Ku. Mereka menyangka bahwa dengan membuat janji-janji, dan dengan mempersembahkan binatang, mereka dapat mencegah malapetaka. Apakah dengan itu mereka bisa senang?
\par 16 Pernah Aku menamakan mereka pohon zaitun rindang, penuh buah-buahan yang bagus. Tetapi, sekarang dengan suara gemuruh yang hebat akan Kubakar daun-daunnya dan Kupatahkan semua rantingnya.
\par 17 Akulah yang mendirikan Israel dan Yehuda, Aku, TUHAN Yang Mahakuasa. Tapi, sekarang Aku sudah merencanakan untuk mendatangkan malapetaka ke atas mereka. Mereka sendirilah yang menyebabkan semuanya itu, karena mereka telah melakukan kejahatan, yaitu mempersembahkan kurban kepada Baal sehingga Aku marah."
\par 18 TUHAN memberitahukan kepadaku tentang rencana jahat musuh-musuhku terhadapku.
\par 19 Aku seperti domba yang tanpa curiga dibawa ke tempat pembantaian. Aku tidak tahu bahwa akulah yang menjadi sasaran rencana-rencana jahat mereka. Mereka berkata, "Mari kita tebang pohon ini sementara ia masih dapat menghasilkan buah; kita musnahkan dia supaya namanya dilupakan orang."
\par 20 Lalu aku berdoa, "Ya TUHAN Yang Mahakuasa, Engkau hakim yang adil; Engkaulah yang menguji pikiran dan hati orang. Perkaraku ini kuserahkan kepada-Mu. Semoga aku dapat menyaksikan Engkau membalas perbuatan orang-orang itu."
\par 21 Orang-orang kota Anatot ingin supaya aku mati. Mereka berkata, "Kalau engkau terus saja menyampaikan pesan-pesan TUHAN kepada kami, engkau akan kami bunuh."
\par 22 Karena itu, TUHAN Yang Mahakuasa berkata, "Aku akan menghukum mereka! Orang-orang muda mereka akan tewas dalam pertempuran, anak-anak mereka akan mati kelaparan.
\par 23 Aku sudah menentukan waktunya untuk mendatangkan malapetaka ke atas orang-orang Anatot, dan jika waktu hukuman mereka itu tiba, tak seorang pun akan luput."

\chapter{12}

\par 1 "Jika aku mengajukan perkaraku di hadapan-Mu, ya TUHAN, tentulah akan terbukti bahwa Engkaulah yang benar. Tapi aku ingin juga menanyakan kepada-Mu soal keadilan: Mengapa orang jahat makmur? Mengapa justru orang tak jujur yang berhasil?
\par 2 Kautanam mereka seperti tumbuh-tumbuhan, lalu mereka berakar, bertumbuh, dan berbuah. Mereka selalu berbicara yang baik-baik tentang diri-Mu, tetapi dalam hati, mereka sebenarnya tidak peduli kepada-Mu.
\par 3 Tetapi Engkau, ya TUHAN, mengenal aku. Engkau melihat apa yang kulakukan, dan Engkau mengetahui bagaimana perasaan hatiku terhadap-Mu. Giringlah orang-orang jahat itu keluar, bawalah mereka seperti domba ke pembantaian. Jagalah mereka sampai tiba waktunya mereka disembelih.
\par 4 Sampai berapa lama lagi negeri ini kering, dan rumputnya layu di setiap ladang? Burung dan binatang lainnya mati karena kejahatan bangsa kami. Mereka berkata, bahwa aku tidak akan melihat nasib mereka."
\par 5 TUHAN berkata, "Yeremia, jika engkau menjadi lelah berlomba dengan manusia, mana mungkin engkau berlomba dengan kuda? Jika di tanah yang aman engkau ketakutan, apakah yang akan kaulakukan apabila kau berada di hutan belukar di tepi Yordan?
\par 6 Engkau sudah dikhianati bahkan oleh sanak saudaramu sendiri; mereka ikut menyerang engkau. Janganlah percaya kepada mereka, meskipun kata-kata mereka manis."
\par 7 TUHAN berkata, "Tempat tinggal-Ku tidak lagi Kupedulikan, negeri yang menjadi milik-Ku telah Kutinggalkan, dan umat-Ku yang Kukasihi telah Kuserahkan ke tangan musuh-musuhnya.
\par 8 Umat pilihan-Ku melawan Aku; seperti singa di rimba demikianlah mereka mengaum terhadap-Ku. Itulah sebabnya Aku membenci mereka.
\par 9 Mereka seperti burung yang diserang oleh elang dari segala pihak. Panggillah segala binatang hutan untuk turut menghabiskan mereka.
\par 10 Banyak penguasa asing telah merusak kebun anggur-Ku dan menginjak-injak ladang-ladang-Ku. Negeri-Ku yang indah mereka ubah menjadi padang gurun yang sepi.
\par 11 Ya, Aku melihat mereka menjadikan negeri-Ku tempat sepi yang menyedihkan. Seluruhnya terlantar, tak ada yang memperhatikan.
\par 12 Dari seberang padang gurun di pegunungan, orang-orang datang untuk merampok. Aku mendatangkan peperangan untuk merusak seluruh negeri sehingga tak seorang pun dapat hidup damai.
\par 13 Umat-Ku menabur gandum, tapi durilah yang dituai. Mereka bersusah payah tapi usahanya tidak berhasil. Aku sangat murka sehingga panenan mereka gagal."
\par 14 TUHAN berkata, "Inilah pesan-Ku tentang negara-negara tetangga Israel yang telah merusak negeri yang Kuberikan kepada umat-Ku Israel. Seperti tanaman yang dicabut dari tanah, demikianlah orang-orang jahat itu akan Kukeluarkan dari negeri-negeri mereka, dan Yehuda akan Kulepaskan dari genggaman mereka.
\par 15 Tetapi setelah mengeluarkan mereka, Aku akan mengasihani mereka lagi; setiap bangsa akan Kubawa kembali ke negerinya sendiri.
\par 16 Jika mereka dengan sepenuh hati mau menerima agama umat-Ku dan mau bersumpah dengan berkata, 'Demi TUHAN yang hidup,' --seperti dahulu mereka mengajar umat-Ku bersumpah demi Baal--maka mereka pun akan tergolong umat-Ku dan menjadi makmur.
\par 17 Tetapi bangsa yang tidak mau menuruti Aku, akan Kucabut seperti tanaman dan Kupunahkan. Aku, TUHAN, telah berbicara."

\chapter{13}

\par 1 TUHAN menyuruh aku pergi membeli sarung pendek dari lenan dan memakainya, tapi aku tidak boleh mencelupkannya ke dalam air.
\par 2 Maka pergilah aku membeli sarung pendek itu, lalu memakainya.
\par 3 Kemudian TUHAN berbicara lagi kepadaku, katanya,
\par 4 "Pergilah ke Sungai Efrat, dan sembunyikan sarung pendek itu di celah-celah batu."
\par 5 Maka sesuai dengan perintah TUHAN pergilah aku dan menyembunyikan sarung pendek itu di dekat Sungai Efrat.
\par 6 Beberapa waktu kemudian TUHAN menyuruh aku pergi lagi ke Sungai Efrat untuk mengambil sarung pendek itu.
\par 7 Maka pergilah aku, dan mengeluarkan sarung pendek itu dari tempat aku menyembunyikannya. Ternyata sarung itu sudah lapuk, dan sama sekali tidak berguna lagi.
\par 8 Lalu TUHAN berbicara lagi kepadaku, kata-Nya,
\par 9 "Begitulah caranya Aku akan bertindak terhadap orang Yehuda dan penduduk Yerusalem. Aku akan menghancurkan keangkuhan mereka yang luar biasa itu.
\par 10 Bangsa yang jahat ini sangat keras kepala dan tak mau taat kepada-Ku. Mereka beribadat dan berbakti kepada ilah-ilah lain. Itu sebabnya mereka akan menjadi seperti sarung pendek yang tak berguna itu.
\par 11 Seperti sarung pendek terikat erat pada pinggang pemakainya, demikianlah Aku bermaksud supaya umat Israel dan Yehuda terikat erat pada-Ku. Dengan demikian mereka menjadi umat-Ku yang membuat nama-Ku terpuji dan dihormati. Tapi, mereka tidak mau menurut kepada-Ku."
\par 12 TUHAN Allah berkata kepadaku, "Yeremia, sampaikanlah pesan-Ku kepada orang Israel bahwa setiap guci tempat anggur harus diisi penuh dengan anggur. Mereka akan menjawab, 'Masakan kami tidak tahu bahwa setiap guci harus dipenuhi dengan anggur?'
\par 13 Maka katakanlah kepada mereka bahwa orang-orang di negeri ini, raja-raja keturunan Daud, para imam, nabi, dan seluruh penduduk Yerusalem akan Kuisi penuh dengan anggur sampai mabuk.
\par 14 Lalu mereka akan Kubenturkan satu sama lain. Tua dan muda akan Kuperlakukan demikian tanpa ampun. Mereka akan Kubinasakan tanpa merasa sayang atau kasihan. Aku, TUHAN, telah berbicara."
\par 15 Umat Israel, TUHAN telah berbicara! Jangan angkuh, dengarkan Dia!
\par 16 TUHAN Allahmu harus kauagungkan sebelum Ia mendatangkan kegelapan, dan kakimu tersandung di pegunungan, sebelum terang yang kauharapkan diubah-Nya menjadi kekelaman.
\par 17 Jika kau enggan mendengarkan, aku akan menangis dengan diam-diam. Dengan rasa pilu kutangisi keangkuhanmu itu. Air mataku akan berlinang karena umat TUHAN diangkut sebagai tawanan.
\par 18 TUHAN berkata kepadaku, "Suruh raja dan ibu suri turun dari takhtanya, sebab mahkotanya yang indah sudah jatuh dari kepalanya.
\par 19 Kota-kota Yehuda di bagian selatan sedang dikepung, dan tak ada yang dapat menerobosi kepungan itu. Semua orang Yehuda sudah diangkut ke pembuangan."
\par 20 TUHAN berkata kepada penduduk Yerusalem, "Perhatikanlah! Musuh-musuhmu sedang datang dari utara! Di manakah orang-orang yang harus kamu lindungi dan yang sangat kamu banggakan itu.
\par 21 Apa yang akan kamu katakan jika orang-orang yang kamu perlakukan sebagai kawan-kawanmu mengalahkan dan menguasai kamu? Pasti kamu akan merasa sakit seperti wanita yang sedang melahirkan.
\par 22 Barangkali kamu bertanya mengapa semuanya itu menimpa dirimu, mengapa pakaianmu dirobek dan kamu diperkosa. Ketahuilah bahwa itu terjadi karena dosamu terlalu besar.
\par 23 Dapatkah orang hitam mengubah warna kulitnya, atau harimau menghilangkan belangnya? Tentu tidak! Begitu juga kamu yang biasa berbuat jahat tidak mungkin berbuat baik.
\par 24 Aku akan menceraiberaikan kamu seperti sekam ditiup angin dari padang gurun.
\par 25 Itulah nasibmu yang telah Kutentukan karena kamu telah melupakan Aku, dan telah percaya kepada ilah-ilah palsu.
\par 26 Aku sendiri akan merobek pakaian dari badanmu supaya kamu menjadi telanjang dan malu.
\par 27 Aku sudah melihat kamu melakukan hal-hal yang Kubenci. Aku telah melihat kamu mengejar ilah-ilah lain di bukit dan di ladang, seperti seorang laki-laki tergila-gila terhadap istri orang lain atau seperti kuda jantan bernafsu terhadap kuda betina. Penduduk Yerusalem, celakalah kamu! Kapan kamu akan bebas dari semua kejahatan?"

\chapter{14}

\par 1 TUHAN berkata begini kepadaku tentang musim kering yang hebat,
\par 2 "Yehuda bersedih hati kota-kotanya hampir mati. Orang-orangnya berbaring di tanah dan mengeluh. Yerusalem menjerit minta dibantu.
\par 3 Para pelayan orang kaya disuruh mencari air di mana-mana; ke sumur-sumur pergilah mereka tapi air telah habis semua. Kosonglah kendi yang mereka bawa kembali. Dengan bimbang dan putus asa mereka malu dan menutup muka.
\par 4 Hujan tak turun dan tanah kering kerontang; petani menutup muka sebab sangat kecewa.
\par 5 Rumput tidak ada lagi, sebab itu induk rusa pergi. Anaknya yang baru lahir ditinggalkannya di padang belantara.
\par 6 Di puncak gunung keledai liar berdiri saja, matanya lesu karena rumput tak ada; ia terengah-engah seperti serigala.
\par 7 Umat-Ku berseru, 'TUHAN, meskipun dosa kami menuduh kami namun tolonglah kami sesuai dengan janji-Mu. Engkau telah kami tinggalkan berkali-kali; sungguh, kami berdosa kepada-Mu.
\par 8 Engkau harapan Israel satu-satunya, yang menyelamatkan kami dari celaka. Tapi mengapa Engkau seperti orang asing di negeri kami, seperti musafir yang hanya singgah semalam lalu pergi?
\par 9 Mengapa Engkau seperti orang yang bingung; seperti pahlawan yang tak sanggup menolong? Tapi, pasti Engkau ada di tengah kami, TUHAN! Kami umat-Mu, janganlah kami Kautinggalkan!'"
\par 10 Tentang bangsa itu TUHAN berkata, "Aku tidak senang dengan orang-orang itu yang suka melarikan diri dari Aku, dan tak mau menguasai diri. Aku tak akan melupakan kesalahan mereka, Aku akan menghukum mereka karena dosa mereka."
\par 11 TUHAN berkata kepadaku, "Jangan minta Aku menolong orang-orang itu.
\par 12 Sekalipun mereka berpuasa, Aku tidak mau memperhatikan seruan mereka minta tolong; dan sekalipun mereka mempersembahkan kepada-Ku kurban bakaran dan kurban gandum, Aku tidak akan senang dengan mereka. Sebaliknya, Aku akan menumpas mereka dengan perang, kelaparan, dan wabah penyakit."
\par 13 Lalu aku berkata, "Ya TUHAN Yang Mahatinggi, Engkau tahu nabi-nabi telah berkata kepada rakyat bahwa tak akan ada peperangan atau kelaparan, karena Engkau telah berjanji bahwa di negeri ini selalu akan ada kedamaian."
\par 14 Tapi TUHAN menjawab, "Nabi-nabi itu membawa berita dusta atas nama-Ku; Aku tidak mengutus mereka, dan tidak juga memberi perintah atau berbicara kepada mereka. Penglihatan-penglihatan yang mereka ceritakan bukan dari Aku; mereka meramalkan hal-hal tak berguna yang mereka karang sendiri.
\par 15 Mereka mengatakan atas nama-Ku bahwa negeri ini tidak akan ditimpa perang atau kelaparan, padahal Aku tidak menyuruh mereka mengatakan itu. Aku, TUHAN, memberitahukan kepadamu, Yeremia, apa yang akan Kulakukan dengan nabi-nabi yang tidak Kuutus itu. Aku akan membunuh mereka dengan perang dan kelaparan.
\par 16 Orang-orang yang telah menerima ramalan-ramalan mereka itu pun akan dibunuh dengan cara yang sama. Mayat mereka akan dilemparkan ke jalan-jalan di Yerusalem tanpa ada yang menguburkannya. Demikianlah nasib mereka dan anak istri mereka. Akan Kubuat mereka merasakan akibat kejahatan mereka."
\par 17 TUHAN menyuruh aku memberitahukan kesedihanku kepada umat-Nya. Aku harus berkata begini, "Biarlah aku menangis tiada hentinya, siang malam bercucuran air mata. Sebab bangsaku luka parah dan sangat menderita.
\par 18 Apabila aku ke padang, kulihat mayat-mayat korban perang. Di dalam kota kulihat penderitaan, orang-orang hampir mati kelaparan. Nabi dan imam menjalankan kewajiban tanpa mengerti apa yang mereka lakukan."
\par 19 TUHAN, sudahkah Yehuda Kautolak sama sekali dan penduduk Sion Kaubenci? Mengapa begitu keras Kaupukul kami sehingga kami tak dapat sembuh lagi? Percuma saja kami mencari kedamaian, tiada gunanya kami mengharap kesembuhan, sebab yang ada hanyalah kekerasan.
\par 20 TUHAN, terhadap Engkau saja kami telah berdosa. Kami mengakui dosa kami sendiri dan dosa leluhur kami.
\par 21 Ingatlah akan janji-Mu, janganlah kami Kauhinakan; perjanjian-Mu dengan kami janganlah Kaubatalkan. Yerusalem tempat takhta-Mu yang mulia, janganlah Kauhina.
\par 22 Dewa-dewa bangsa-bangsa tak dapat menurunkan hujan ke bumi. Langit pun tak bisa juga mencurahkan hujan dengan kuasanya sendiri. Hanya Engkaulah harapan kami. Engkau membuat semua itu terjadi.

\chapter{15}

\par 1 TUHAN berkata kepadaku, "Aku tidak akan mengasihani bangsa ini, sekalipun Musa dan Samuel berdiri di sini untuk memohon belas kasihan-Ku bagi mereka. Suruhlah bangsa ini pergi, Aku tak mau melihat mereka!
\par 2 Jika mereka bertanya kepadamu ke mana mereka harus pergi, jawablah begini, 'TUHAN berkata, yang ditentukan untuk mati karena sakit biarlah ia mati karena penyakit. Yang ditentukan untuk mati dalam pertempuran biarlah ia mati dalam pertempuran. Yang ditentukan untuk mati karena lapar biarlah ia mati kelaparan! Yang ditentukan untuk ditawan, biarlah ia diangkut sebagai tawanan!'
\par 3 Aku, TUHAN, telah menentukan bahwa empat hukuman yang mengerikan akan terjadi pada mereka: mereka akan tewas dalam pertempuran, mayat mereka akan diseret anjing, dimakan burung, dan dihabiskan oleh binatang buas.
\par 4 Aku akan membuat segala bangsa di dunia ngeri melihat mereka. Semua itu Kulakukan karena perbuatan Manasye putra Hizkia di Yerusalem ketika ia menjadi raja Yehuda."
\par 5 TUHAN berkata, "Hai penduduk Yerusalem, siapakah akan mengasihani kamu? Siapakah yang mau bersedih hati dengan kamu? Siapakah mau singgah untuk menanyakan keadaanmu?
\par 6 Kamu telah menolak dan meninggalkan Aku, karena itu Kuacungkan tangan-Ku untuk menghancurkan kamu, sebab Aku sudah jemu menahan kemarahan-Ku.
\par 7 Di setiap kota di negeri ini kamu akan Kubuang seperti jerami. Sekalipun kamu adalah umat-Ku namun Aku akan membinasakan kamu dan anak-anakmu karena kamu tak mau mengubah cara hidupmu.
\par 8 Negerimu akan penuh dengan janda, melebihi pasir di pantai banyaknya. Orang-orang akan Kubunuh pada masa mudanya sehingga bunda mereka menderita. Pada siang hari Kudatangkan kepada mereka seorang pembinasa sehingga mereka merasa takut dan gelisah.
\par 9 Ibu yang kehilangan tujuh putra pingsan karena sesak napasnya. Hari hidupnya telah menjadi suram, ia merasa dihina dan dipermalukan. Musuh-musuhmu Kusuruh membunuh yang masih hidup di antara kamu. Aku, TUHAN, telah berbicara."
\par 10 Sungguh sial aku ini! Untuk apa ibuku melahirkan aku? Dengan setiap orang di negeri ini aku harus bertengkar dan berbantah. Sekalipun aku tidak pernah meminjamkan atau meminjam, namun aku dikutuk semua orang.
\par 11 TUHAN, kalau aku memang tidak melakukan tugasku dengan baik, dan tidak membela musuh-musuh di depan-Mu pada waktu mereka mendapat bencana dan susah, biarlah kutukan orang-orang itu benar-benar terjadi.
\par 12 (Tak seorang pun dapat mematahkan besi, terutama besi dari utara yang sudah dicampur dengan tembaga.)
\par 13 TUHAN berkata kepadaku, "Aku akan mengirim musuh untuk mengangkut kekayaan dan harta umat-Ku sebagai hukuman atas dosa-dosa yang telah mereka lakukan di seluruh negeri.
\par 14 Di negeri asing yang tak mereka kenal, Aku akan menjadikan mereka hamba musuh-musuh mereka, sebab kemarahan-Ku sudah meluap seperti api yang berkobar dan tidak mau padam."
\par 15 Lalu aku berkata, "TUHAN, Engkau mengerti. Ingatlah dan tolonglah aku. Balaslah mereka yang mengejar dan menindas aku. Janganlah terlalu sabar terhadap mereka sehingga mereka nanti berhasil membunuh aku. Ingatlah bahwa karena Engkaulah maka aku dihina.
\par 16 Engkau berbicara kepadaku, dan aku mendengarkan setiap perkataan-Mu. Aku milik-Mu, ya TUHAN Allah Yang Mahakuasa, karena itu perkataan-perkataan-Mu menyenangkan dan membahagiakan aku.
\par 17 Aku tidak turut bersenang-senang dengan orang-orang yang berkumpul untuk bersenda gurau. Karena taat kepada-Mu, maka aku menyendiri dengan perasaan marah.
\par 18 Mengapa aku terus saja menderita? Mengapa lukaku parah dan tak sembuh-sembuh? Apakah Engkau bermaksud mengecewakan aku seperti sungai yang menjadi kering pada musim kemarau?"
\par 19 TUHAN menjawab, "Jika engkau kembali, Aku akan menerimamu. Engkau akan menjadi hamba-Ku lagi. Jika engkau memberitakan hal-hal yang berguna dan bukan omong kosong, maka engkau boleh menjadi nabi-Ku lagi. Orang-orang akan datang lagi kepadamu, dan tak perlu engkau pergi kepada mereka.
\par 20 Engkau akan Kujadikan seperti benteng perunggu terhadap mereka. Mereka akan memerangi engkau, tapi tidak akan dapat mengalahkan engkau. Aku akan menyertaimu untuk melindungi dan menyelamatkan engkau.
\par 21 Aku akan melepaskan engkau dari kekuasaan orang-orang jahat yang kejam. Aku, TUHAN, telah berbicara."

\chapter{16}

\par 1 TUHAN berbicara lagi kepadaku, kata-Nya,
\par 2 "Jangan kawin dan mendapat anak di tempat ini.
\par 3 Akan Kukatakan kepadamu apa yang akan terjadi dengan anak-anak yang lahir di sini dan dengan orang tuanya.
\par 4 Mereka akan mati diserang penyakit yang membawa maut, dan mereka akan tewas dalam pertempuran atau mati kelaparan. Tak seorang pun akan menangisi atau menguburkan mereka. Mayat mereka akan terserak seperti pupuk di tanah, dan dimakan burung serta binatang buas.
\par 5 Janganlah pergi ke rumah orang yang sedang berduka, Yeremia. Jangan berkabung untuk siapa pun. Aku tidak akan memberkati umat-Ku lagi dengan kesejahteraan, dan tidak juga akan menunjukkan rasa sayang atau belas kasihan kepada mereka.
\par 6 Orang kaya dan orang miskin di negeri ini akan mati tanpa ada yang menguburkan mereka. Tak seorang pun akan menyiksa diri atau menggunduli kepalanya sebagai tanda berkabung untuk mereka.
\par 7 Tak ada pula yang mau mengadakan selamatan untuk menghibur orang yang sedang berkabung. Tidak ada yang ikut berduka cita bahkan dengan orang yang kehilangan ayah atau ibunya.
\par 8 Jangan masuk ke rumah orang yang sedang berpesta. Jangan makan minum dengan mereka.
\par 9 Dengarkan apa yang Aku, TUHAN Yang Mahakuasa, Allah Israel, hendak katakan kepadamu. Suara gembira dan sukacita serta keramaian pesta kawin akan Kuhentikan semua. Hal itu akan disaksikan oleh orang-orang yang hidup di zamanmu ini.
\par 10 Apabila engkau menyampaikan semua ini kepada mereka, barangkali mereka akan bertanya mengapa Aku mau menghukum mereka begitu keras. Mereka akan bertanya apa kesalahan dan dosa mereka terhadap Aku, TUHAN Allah mereka.
\par 11 Beritahukanlah kepada mereka bahwa Aku berkata begini, 'Leluhurmu telah meninggalkan Aku untuk menyembah dan beribadat kepada ilah-ilah lain. Mereka mengabaikan Aku dan tidak mau taat kepada peraturan-peraturan-Ku.
\par 12 Tapi kamu melakukan yang lebih jahat lagi. Kamu keras kepala dan hanya mengikuti kemauanmu sendiri yang jahat. Kamu tidak mau taat kepada-Ku.
\par 13 Karena itu, kamu akan Kuusir dari negeri ini dan Kubuang ke sebuah negeri yang sama sekali asing bagimu dan bagi leluhurmu. Di sana kamu akan beribadat siang malam kepada ilah-ilah lain, dan Aku tidak akan berbelaskasihan kepadamu.'"
\par 14 TUHAN berkata, "Akan datang waktunya orang tidak lagi bersumpah demi Aku, Allah yang hidup, yang membawa umat Israel keluar dari Mesir,
\par 15 melainkan demi Aku, Allah yang hidup, yang membawa umat Israel keluar dari sebuah negeri di utara dan dari semua negeri lain di mana mereka telah Kuceraiberaikan. Mereka akan Kubawa kembali ke tanah air mereka, negeri yang telah Kuberikan kepada leluhur mereka."
\par 16 TUHAN berkata, "Aku hendak memanggil banyak nelayan untuk menangkap orang-orang ini. Kemudian Aku akan memanggil banyak pemburu untuk memburu mereka di setiap gunung dan bukit, dan di gua-gua bukit batu.
\par 17 Aku melihat segala yang mereka lakukan. Tak ada yang tersembunyi bagi-Ku; dosa-dosa mereka pun sudah Kulihat.
\par 18 Dosa dan kejahatan mereka akan Kubalas dengan hukuman dua kali lipat, karena mereka sudah menajiskan negeri-Ku dengan banyak berhala. Berhala-berhala itu tak bernyawa seperti mayat."
\par 19 TUHAN, Engkaulah yang melindungi aku dan memberikan kekuatan kepadaku; Engkau menolong aku pada masa kesukaran. Bangsa-bangsa dari ujung bumi akan datang kepada-Mu dan berkata, "Leluhur kami hanya mempunyai ilah palsu yang sama sekali tak berguna.
\par 20 Dapatkah manusia membuat ilahnya sendiri? Tentu tidak! Yang dibuatnya itu pasti bukan Allah."
\par 21 "Karena itu," kata TUHAN, "sekali ini Aku akan membuat bangsa-bangsa merasakan kekuasaan dan kekuatan-Ku supaya untuk selama-lamanya mereka akan tahu bahwa Akulah TUHAN."

\chapter{17}

\par 1 TUHAN berkata, "Hai bangsa Yehuda, dosamu telah terukir pada hatimu dan pada semua sudut mezbah-mezbahmu dengan pena besi yang bermata pena dari intan.
\par 2 Orang-orangmu beribadat pada mezbah-mezbah dan tiang-tiang berhala yang telah didirikan untuk Dewi Asyera di setiap pohon yang rindang, di puncak-puncak bukit,
\par 3 dan di gunung-gunung di luar kota. Karena semua tempat penyembahan berhala yang telah kamu dirikan di seluruh negeri dan karena segala dosa yang telah kamu lakukan, maka Aku akan membuat musuh-musuhmu mengangkut segala harta dan kekayaanmu.
\par 4 Kamu akan terpaksa menyerahkan tanah yang telah Kuberikan kepadamu, dan Aku akan membuat kamu mengabdi kepada musuhmu di negeri yang sama sekali asing bagimu. Aku melakukan semuanya itu karena kemarahan-Ku telah meluap seperti api yang menyala dan tak dapat dipadamkan."
\par 5 TUHAN berkata, "Apabila orang meninggalkan Aku, Tuhannya, dan berharap kepada manusia serta bersandar pada kekuatannya, maka Aku akan menghukum dia.
\par 6 Ia seperti tanaman yang tumbuh di padang, di tanah tandus, sunyi sepi dan bergaram, tak pernah ia mengalami kebaikan.
\par 7 Tapi orang yang berharap kepada-Ku akan Kuberkati selalu.
\par 8 Ia bagaikan pohon di tepi sungai yang mengalir; akarnya merambat sampai ke air. Ia tak takut musim kemarau, daun-daunnya selalu hijau. Sekalipun negeri dilanda kekeringan, ia tak gelisah sebab ia selalu menghasilkan buah.
\par 9 Hati manusia tak dapat diduga, paling licik dari segala-galanya dan terlalu parah penyakitnya.
\par 10 Aku, TUHAN, menyelidiki hati, batin manusia Kuuji. Setiap orang akan Kubalas menurut tingkah lakunya, dan Kuperlakukan sesuai dengan perbuatannya."
\par 11 Bagaikan burung mengerami telur yang bukan miliknya, begitulah orang yang tak jujur mendapat harta. Pada usia setengah baya ia akan kehilangan semua hartanya. Pada akhir hidupnya terbukti ia orang yang bodoh sekali.
\par 12 Rumah TUHAN kita seperti takhta yang mulia, di gunung yang tinggi sejak semula.
\par 13 TUHAN, Engkaulah harapan Israel, umat-Mu; semua yang meninggalkan Engkau akan menjadi malu. Mereka akan lenyap dari dunia ini dan pergi ke dunia orang mati. Sebab, Engkau sumber air kehidupan telah mereka tinggalkan, ya TUHAN.
\par 14 TUHAN, sembuhkanlah aku, maka aku akan menjadi sehat. Selamatkanlah aku, maka aku akan selamat. Engkau saja yang kumuliakan!
\par 15 Orang berkata kepadaku, "Di mana semua ancaman TUHAN itu? Biarlah sekarang dilaksanakan!"
\par 16 Tapi, ya TUHAN, tak pernah aku mendesak supaya Engkau mendatangkan celaka atas mereka. Aku tidak juga menginginkan supaya mereka ditimpa kemalangan. Engkau tahu itu, TUHAN. Engkau tahu apa yang kuucapkan.
\par 17 Janganlah Engkau menjadi sesuatu yang menakutkan hatiku. Engkau tempat pengungsian bagiku bila datang kesukaran.
\par 18 Semoga yang mengejar dan menindas aku, Kaupermalukan dan Kaupenuhi dengan ketakutan. Tetapi janganlah berbuat begitu terhadap aku, ya TUHAN. Datangkanlah malapetaka ke atas mereka sampai mereka hancur binasa.
\par 19 TUHAN berkata kepadaku, "Yeremia, pergilah ke Pintu Gerbang Rakyat yang dilalui raja-raja Yehuda apabila mereka keluar masuk kota. Pergilah juga ke semua pintu gerbang lainnya di Yerusalem.
\par 20 Sampaikanlah pesan-Ku kepada raja-raja dan semua orang Yehuda serta seluruh penduduk Yerusalem yang lewat di pintu-pintu itu.
\par 21 Katakanlah begini: 'Jika kamu tidak ingin mati, janganlah mengangkut barang pada hari Sabat, atau membawanya masuk melalui pintu-pintu gerbang Yerusalem.
\par 22 Jangan bekerja atau membawa barang apa pun keluar dari rumahmu pada hari Sabat. Khususkanlah hari itu untuk Aku seperti yang telah Kuperintahkan kepada leluhurmu.
\par 23 Tapi leluhurmu tidak mau mendengarkan atau memperhatikan apa yang Kukatakan. Mereka keras kepala, dan tidak mau taat atau ditegur.
\par 24 Taatilah semua perintah-Ku. Jangan mengangkut barang apa pun melalui pintu gerbang kota ini pada hari Sabat. Khususkanlah, hari Sabat untuk Aku, dan janganlah mengerjakan apa pun pada hari itu.
\par 25 Jika kamu sungguh-sungguh menjalankan perintah-perintah-Ku itu, maka raja-raja dan pejabat-pejabat yang memerintah seperti Daud akan memasuki gerbang-gerbang kota Yerusalem. Mereka bersama orang Yehuda dan penduduk Yerusalem, akan mengendarai kereta kuda, sehingga kota Yerusalem menjadi ramai sekali.
\par 26 Orang-orang akan datang dari kota-kota Yehuda, dan dari desa-desa di sekitar Yerusalem; mereka akan datang dari wilayah Benyamin, dari daerah kaki bukit, dari daerah pegunungan, dan dari Yehuda bagian selatan. Mereka akan membawa ke Rumah-Ku binatang-binatang kurban, baik yang untuk dibakar maupun yang tidak, juga persembahan gandum dan kemenyan serta persembahan syukur.
\par 27 Semua itu akan terjadi apabila kamu taat kepada-Ku dan mengkhususkan hari Sabat untuk Aku. Jangan mengangkut barang apa pun lewat pintu-pintu gerbang Yerusalem pada hari itu. Kalau kamu melanggar perintah-perintah-Ku itu, pintu-pintu gerbang Yerusalem serta istana-istananya akan Kubakar habis dan api itu tak akan dapat dipadamkan.'"

\chapter{18}

\par 1 TUHAN berkata kepadaku,
\par 2 "Pergilah ke rumah tukang periuk; di sana akan Kusampaikan kepadamu pesan-Ku."
\par 3 Maka pergilah Aku ke rumah tukang periuk dan Kulihat dia sedang bekerja dengan pelarikan.
\par 4 Apabila periuk yang sedang dikerjakannya itu kurang sempurna, ia mengerjakannya kembali menjadi periuk yang lain sesuai dengan yang dikehendakinya.
\par 5 Lalu TUHAN menyuruh aku mengatakan kepada umat Israel,
\par 6 "Hai umat-Ku, masakan Aku tidak dapat berbuat kepadamu seperti yang dilakukan tukang itu dengan tanah liatnya? Seperti tanah liat dalam tangan tukang periuk, demikian juga kamu dalam tangan-Ku.
\par 7 Jika pada suatu waktu Aku memutuskan untuk merenggut, meruntuhkan dan menghancurkan suatu bangsa atau kerajaan,
\par 8 lalu bangsa itu bertobat dari dosanya, maka Aku akan membatalkan niat-Ku itu.
\par 9 Sebaliknya, apabila Aku memutuskan untuk mendirikan atau menguatkan suatu bangsa atau kerajaan,
\par 10 tapi kemudian bangsa itu tidak taat kepada-Ku lalu melakukan yang jahat, maka niat-Ku itu akan Kubatalkan."
\par 11 TUHAN berkata kepadaku, "Sekarang, katakanlah kepada orang-orang Yehuda dan penduduk Yerusalem bahwa Aku sedang membuat rencana untuk melawan mereka, dan menyiapkan hukuman untuk mereka. Suruhlah mereka menghentikan cara hidup mereka yang berdosa itu serta memperbaiki kelakuan dan perbuatan mereka.
\par 12 Tapi mereka akan menjawab, 'Apa gunanya? Kami akan tetap mengikuti kehendak kami sendiri! Kami masing-masing akan melakukan keinginan hati kami yang jahat!'"
\par 13 TUHAN berkata, "Tanyakan kepada setiap bangsa apakah pernah terjadi hal seperti ini. Umat Israel telah melakukan sesuatu yang mengerikan!
\par 14 Tidak pernah salju lenyap dari gunung batu di Libanon. Tidak pernah air sejuk yang mengalir di pegunungan menjadi kering.
\par 15 Tapi umat-Ku telah melupakan Aku, dan membakar dupa untuk berhala. Mereka tersandung di jalan dahulu kala yang harus mereka lalui, lalu menyimpang ke jalan yang belum diratakan.
\par 16 Negeri ini telah mereka jadikan tempat yang mengerikan, dan selamanya menjadi bahan ejekan. Semua yang melewatinya akan menggelengkan kepala karena merasa ngeri.
\par 17 Seperti debu diembus angin timur, begitulah umat-Ku akan Kuserakkan di depan musuh. Apabila datang celaka, Aku akan membelakangi mereka dan tidak menolong mereka!"
\par 18 Orang berkata, "Mari kita berkomplot melawan Yeremia. Bukan hanya dia yang bisa berbicara untuk Allah. Selalu akan ada imam yang dapat memberi pelajaran, ada orang bijaksana yang dapat memberi nasihat; dan ada nabi yang dapat menyampaikan kepada kita pesan-pesan dari Allah. Mari kita mengadukan dia ke pengadilan dan tidak lagi mendengarkan kata-katanya."
\par 19 Maka aku berdoa, "TUHAN, dengarkanlah doaku; perhatikanlah apa yang dikatakan musuhku tentang aku!
\par 20 Pantaskah kebaikan dibalas dengan kejahatan? Tapi mereka telah menggali lubang supaya aku jatuh ke dalamnya. Ingatlah bahwa aku telah membela mereka di hadapan-Mu, supaya Engkau jangan marah kepada mereka.
\par 21 Tapi sekarang, TUHAN, biarlah anak-anak mereka mati kelaparan atau tewas dalam peperangan. Biarlah wanita-wanita mereka kehilangan suami dan anak; biarlah kaum laki-laki mati karena wabah penyakit dan orang muda tewas dalam pertempuran.
\par 22 Suruhlah perampok datang dengan tiba-tiba ke rumah mereka, sehingga mereka menjerit ketakutan. Mereka telah menggali lubang supaya aku jatuh ke dalamnya, dan mereka telah memasang jerat untuk menangkap aku.
\par 23 Tapi Engkau, TUHAN, tahu semua yang mereka rencanakan untuk menghabisi nyawaku. Karena itu, jangan mengampuni mereka; jangan menghapus kesalahan mereka. Pukullah mereka sampai jatuh. Bertindaklah terhadap mereka dalam kemarahan-Mu."

\chapter{19}

\par 1 TUHAN menyuruh aku pergi membeli kendi, dan mengajak beberapa pemimpin bangsa serta beberapa imam yang tua-tua,
\par 2 untuk pergi bersama-sama mereka ke Lembah Hinom melalui Pintu Gerbang Beling. Di sana aku harus mengumumkan apa yang dipesankan-Nya kepadaku.
\par 3 Aku harus berkata begini, "Hai raja-raja Yehuda dan penduduk Yerusalem! Dengarkan apa yang Aku, TUHAN Yang Mahakuasa, Allah Israel, hendak katakan kepadamu. Aku akan mendatangkan malapetaka yang besar ke atas tempat ini sehingga orang yang mendengarnya terkejut.
\par 4 Hal itu Kulakukan karena bangsa ini telah meninggalkan Aku. Mereka menajiskan tempat ini dengan mempersembahkan kurban kepada dewa-dewa yang asing bagi mereka, bagi leluhur mereka dan bagi raja-raja Yehuda. Mereka juga telah mengisi tempat ini dengan darah orang-orang yang tak bersalah.
\par 5 Mereka mendirikan mezbah bagi Baal untuk mempersembahkan anak-anak mereka sebagai kurban bakaran, padahal Aku tak pernah menyuruh mereka melakukan hal itu, bahkan tak pernah hal semacam itu timbul dalam pikiran-Ku.
\par 6 Karena itu, akan tiba waktunya tempat ini tidak lagi disebut Tofet atau Lembah Hinom, melainkan Lembah Pembantaian.
\par 7 Di tempat ini Aku akan menggagalkan semua rencana orang Yehuda dan penduduk Yerusalem. Mereka akan Kubiarkan dikalahkan oleh musuh-musuh mereka dan dibunuh dalam pertempuran. Mayat mereka akan Kubiarkan dimakan burung dan binatang buas.
\par 8 Aku akan menimpakan malapetaka yang dahsyat ke atas kota ini sehingga setiap orang yang lewat di situ akan terkejut dan ngeri.
\par 9 Musuh akan mengepung kota ini, dan berusaha membunuh penduduknya. Pengepungan itu begitu hebat, sehingga orang-orang di dalam kota ini saling memakan temannya, bahkan anak-anak mereka sendiri."
\par 10 Kemudian TUHAN menyuruh aku memecahkan kendi itu di depan orang-orang yang mengikuti aku.
\par 11 Aku harus mengatakan kepada mereka bahwa TUHAN Yang Mahakuasa berkata begini, "Bangsa dan kota ini akan Kuhancurkan sehingga menjadi seperti kendi yang pecah itu yang tak dapat diperbaiki lagi. Orang akan menguburkan mayat-mayat mereka di Tofet, karena tak ada lagi tempat pekuburan lain.
\par 12 Kota ini dengan penduduknya akan Kujadikan seperti Tofet.
\par 13 Semua rumah di Yerusalem dan semua rumah raja-raja Yehuda akan menjadi najis seperti Tofet; karena di atas atap rumah-rumah itu orang telah membakar dupa untuk dewa-dewa langit, dan mempersembahkan anggur untuk dewa-dewa lain."
\par 14 Lalu aku meninggalkan Tofet, tempat aku disuruh menyampaikan pesan TUHAN. Aku pergi ke pelataran Rumah TUHAN dan berdiri di situ, lalu memberitahukan kepada semua orang,
\par 15 bahwa TUHAN Yang Mahakuasa, Allah Israel, berkata begini, "Karena kamu keras kepala dan tak mau mendengarkan, maka Aku akan mendatangkan ke atas kota ini dan ke atas semua desa di sekitarnya segala bencana yang telah Kurencanakan."

\chapter{20}

\par 1 Imam Pasyhur anak Imer adalah kepala pengawas Rumah TUHAN. Ketika ia mendengar aku mengumumkan semua hal itu,
\par 2 ia memerintahkan supaya aku dipukul dan dipasung di Pintu Gerbang Benyamin, gerbang bagian atas di Rumah TUHAN.
\par 3 Pagi berikutnya setelah Pasyhur melepaskan aku dari pasungan, aku berkata kepadanya, "Nama yang akan diberikan TUHAN kepadamu bukan Pasyhur, tetapi 'Teror di mana-mana'.
\par 4 TUHAN sendiri berkata, 'Engkau akan Kujadikan teror bagi dirimu sendiri dan bagi kawan-kawanmu. Kau akan melihat mereka dibunuh oleh musuh mereka. Semua orang Yehuda akan Kubiarkan dikuasai oleh raja Babel; sebagian dari mereka akan diangkutnya sebagai tawanan ke Babel, dan sisanya akan dibunuh.
\par 5 Musuh mereka akan Kubiarkan juga menjarahi dan mengangkut ke Babel semua kekayaan kota ini, serta semua harta miliknya, bahkan barang-barang pusaka raja-raja Yehuda.
\par 6 Dan engkau, Pasyhur, bersama seluruh keluargamu juga akan ditangkap dan diangkut ke Babel. Di sana engkau akan mati dan dikuburkan; begitu pula semua kawanmu yang telah mendengarkan kebohongan-kebohonganmu.'"
\par 7 TUHAN, Engkau membujuk aku dan aku telah terbujuk. Engkau lebih kuat dari aku dan telah menundukkan aku. Aku diolok setiap orang, dihina dari pagi sampai petang.
\par 8 Setiap kali aku berbicara, aku harus berteriak sekuat tenaga, "Kekejaman! Bencana!" TUHAN, aku diejek dan dihina setiap waktu, karena menyampaikan pesan-Mu.
\par 9 Tapi bila dalam hatiku aku berkata, "Biarlah TUHAN kulupakan saja, tak mau lagi aku berbicara atas nama-Nya," maka pesan-Mu bagaikan api yang membara di hati sanubari. Telah kucoba menahannya, tapi ternyata aku tak kuasa.
\par 10 Terdengar orang berbisik di mana-mana, "Ketakutan merajalela! Mari laporkan dia kepada yang berkuasa!" Bahkan semua sahabat karibku menantikan kejatuhanku. Kata mereka, "Barangkali dengan bujukan, ia dapat kita kalahkan, supaya dapatlah kita membalas dendam kepadanya."
\par 11 Tetapi Engkau, ya TUHAN, di pihakku; Engkau sangat kuat lagi perkasa. Mereka yang mengejar dan menindas aku akan jatuh dan tak berdaya. Mereka akan malu selamanya, gagallah semua rencana mereka. Kehinaan mereka itu akan diingat selalu.
\par 12 Tetapi, ya TUHAN Yang Mahakuasa, dengan adil Kauuji manusia; Kau tahu hati dan pikiran mereka. Karena itu perkenankanlah aku melihat Engkau membalas kejahatan musuh sebab kepada-Mu kuserahkan perkaraku.
\par 13 Menyanyilah bagi TUHAN dan pujilah Dia sebab Ia melepaskan orang tertekan dari kuasa orang durhaka.
\par 14 Terkutuklah hari kelahiranku! Biarlah terhina saat aku dilahirkan ibu.
\par 15 Terkutuklah juga pembawa berita yang membuat ayahku sangat gembira, ketika diberitahukan kepadanya, "Engkau mendapat seorang putra!"
\par 16 Biarlah si pembawa berita itu serupa kota-kota yang dihancurkan TUHAN tanpa iba. Biarlah ia mendengar jerit kesakitan di waktu pagi, dan pekik pertempuran di tengah hari.
\par 17 Sebab ia tidak membunuh aku ketika aku masih dalam kandungan, supaya aku tetap dikandung ibuku dan rahimnya menjadi bagiku sebagai kuburan.
\par 18 Mengapa aku harus dilahirkan? Hanyakah untuk derita dan kesukaran? Dan supaya hidupku berlalu semata-mata dalam malu?

\chapter{21}

\par 1 Zedekia raja Yehuda mengutus kepadaku Imam Pasyhur anak Malkia, dan Imam Zefanya anak Maaseya. Ia menyuruh mereka menyampaikan kepadaku permintaan ini,
\par 2 "Mintalah petunjuk dari TUHAN untuk kita, sebab Nebukadnezar raja Babel dengan pasukannya sedang mengepung kota kita. Barangkali TUHAN akan mengadakan keajaiban untuk kita, dan memaksa Nebukadnezar menarik mundur pasukannya."
\par 3 Lalu TUHAN berbicara kepadaku, maka aku menyuruh orang-orang yang diutus itu
\par 4 memberitahukan kepada Zedekia bahwa TUHAN, Allah Israel, berkata begini, "Zedekia, Aku akan mengalahkan pasukanmu yang sedang bertempur melawan raja Babel dan pasukannya. Senjata-senjata tentaramu akan Kutumpuk di tengah-tengah kota.
\par 5 Aku sendiri akan memerangi kamu dengan sekuat tenaga dan dengan kemarahan yang meluap-luap.
\par 6 Penduduk kota ini akan Kubunuh; baik manusia maupun binatang akan mati karena wabah penyakit yang dahsyat.
\par 7 Engkau dan para pegawaimu serta orang-orang yang luput dari peperangan, kelaparan dan wabah penyakit, semuanya akan Kubiarkan ditangkap oleh Raja Nebukadnezar dan oleh musuh-musuh yang mau membunuhmu. Nebukadnezar akan membunuh kamu semua tanpa ampun dan tanpa belas kasihan. Aku, TUHAN, telah berbicara."
\par 8 Lalu TUHAN menyuruh aku berkata begini kepada umat-Nya, "Dengarkan! Aku, TUHAN, memberi kepadamu suatu pilihan: jalan yang menuju kehidupan, atau jalan yang menuju kematian.
\par 9 Setiap orang yang tinggal di dalam kota akan mati karena perang, kelaparan, atau wabah penyakit. Tapi mereka yang keluar untuk menyerah kepada orang Babel yang sedang menyerang kota ini, akan luput dan tidak dibunuh.
\par 10 Aku sudah memutuskan untuk menghancurkan kota ini, dan tidak menyayangkannya. Kota ini akan diserahkan kepada raja Babel; ia akan membakarnya sampai habis sama sekali. Aku, TUHAN, telah berbicara."
\par 11 TUHAN menyuruh aku menyampaikan pesan ini kepada keluarga raja Yehuda, keturunan Daud: "Dengarkan apa yang Aku, TUHAN, katakan. Berusahalah supaya keadilan dijalankan setiap hari. Lepaskanlah orang yang diperas dari kekuasaan orang yang memerasnya. Kalau tidak, maka kejahatan yang kaulakukan akan membuat kemarahan-Ku berkobar seperti api yang tak dapat dipadamkan."
\par 13 Kemudian TUHAN berkata kepada Yerusalem, "Hai Yerusalem! Letakmu lebih tinggi dari lembah-lembah di sekitarmu. Engkau seperti bukit batu menjulang di padang yang rata, tapi Aku akan melawan engkau. Engkau berkata bahwa tak seorang pun dapat menyerangmu atau mendobrak benteng-bentengmu,
\par 14 tapi Aku akan menghukum engkau karena perbuatanmu. Istanamu akan Kubakar, dan segala sesuatu di sekitarnya akan terbakar juga. Aku, TUHAN, telah berbicara."

\chapter{22}

\par 1 TUHAN menyuruh aku pergi ke istana raja Yehuda, keturunan Daud, untuk menyampaikan pesan TUHAN kepada raja, dan kepada para pegawainya, serta penduduk Yerusalem. Inilah kata-kata TUHAN kepada mereka,
\par 3 "Berlakulah adil dan jujur. Lepaskanlah orang yang diperas dari kekuasaan orang yang memerasnya. Janganlah menindas atau berbuat jahat terhadap orang asing, anak yatim piatu, atau janda. Janganlah membunuh orang yang tak bersalah di tempat ini.
\par 4 Jika kamu sungguh-sungguh menjalankan perintah-perintah-Ku itu, maka raja-raja yang memerintah seperti Daud akan memasuki gerbang-gerbang istana ini. Mereka bersama para pegawai dan rakyat mereka akan mengendarai kereta dan kuda.
\par 5 Tapi kalau kamu tidak mentaati perintah-perintah-Ku, maka Aku bersumpah bahwa istana ini akan menjadi puing. Aku, TUHAN, telah berbicara.
\par 6 Bagi-Ku istana raja Yehuda indah sekali, seindah tanah Gilead dan gunung-gunung Libanon; tapi tempat itu akan Kujadikan seperti padang gurun yang tidak didiami orang.
\par 7 Akan Kukirim orang untuk menghancurkannya. Mereka akan membawa kapak, dan merobohkan tiang-tiangnya yang indah yang dibuat dari kayu cemara Libanon. Lalu mereka akan melemparkannya ke dalam api.
\par 8 Setelah itu banyak orang asing akan lewat di situ dan bertanya satu sama lain mengapa Aku, TUHAN, telah bertindak begitu terhadap kota yang indah itu.
\par 9 Lalu orang akan menjawab bahwa semua itu terjadi karena kamu telah mengingkari perjanjianmu dengan Aku, Allahmu, dan telah menyembah dan beribadat kepada ilah-ilah lain."
\par 10 Orang Yehuda, janganlah berduka untuk kematian Raja Yosia. Tangisilah saja Yoahas putranya sebab ia telah dibawa pergi dan tak akan kembali. Tanah tumpah darahnya tak akan dilihatnya lagi.
\par 11 Mengenai Yoahas, putra Yosia, yang menjadi raja Yehuda menggantikan ayahnya, TUHAN berkata begini, "Ia telah pergi dari sini, dan tak akan kembali.
\par 12 Ia akan meninggal di pembuangan, dan tak akan melihat negeri ini lagi."
\par 13 Celakalah orang yang membangun rumahnya dengan ketidakadilan, dan memperluasnya dengan ketidakjujuran. Celakalah dia yang menyuruh orang bekerja tanpa upah.
\par 14 Celakalah juga orang yang berkata, "Akan kudirikan istana yang besar dengan kamar-kamar yang luas di tingkat atas," lalu di istana itu dibuatnya jendela dengan bingkai dari kayu cemara Libanon, dan dicat merah.
\par 15 Dengarlah, hai Raja Yoyakim! Jika engkau membangun istana dengan kayu cemara Libanon yang paling bagus, apakah itu buktinya bahwa engkau raja yang lebih baik dari raja-raja lain? Coba ingat akan ayahmu: Ia menikmati hidup yang bahagia dan berhasil dalam segala usahanya, karena ia selalu lurus dan jujur serta mengadili perkara orang miskin dengan adil. Memang begitulah seharusnya tindakan orang yang mengenal TUHAN.
\par 17 Tapi engkau, Yoyakim, lain. Engkau hanya mencari keuntungan pribadi. Engkau membunuh orang yang tak bersalah, dan menindas rakyatmu dengan kejam.
\par 18 Karena itu mengenai Yoyakim raja Yehuda, putra Yosia, TUHAN berkata begini, "Kematiannya tak akan ditangisi; tak ada yang akan berkata, 'Aduh, kawanku, aduh.' Tak ada yang berduka atau berseru, 'Kasihan sri baginda!'
\par 19 Yoyakim akan diseret dan dilemparkan ke luar pintu gerbang Yerusalem, seperti yang dilakukan orang terhadap bangkai keledai."
\par 20 TUHAN berkata, "Hai penduduk Yerusalem, pergilah ke Libanon dan ke Basan, dan berteriaklah di sana. Menjeritlah di pegunungan Moab, sebab semua sekutumu telah dikalahkan.
\par 21 Aku telah berbicara kepadamu ketika keadaanmu baik, tapi kamu tak mau mendengarkan. Memang begitulah tingkah lakumu dari dulu sampai sekarang; tak pernah kamu mau taat kepada-Ku.
\par 22 Karena itu, para pemimpinmu akan lenyap seperti diembus angin. Sekutu-sekutumu akan diangkut sebagai tawanan, dan kotamu dihina dan dipermalukan karena segala kejahatanmu.
\par 23 Kamu tinggal dengan aman dikelilingi tiang-tiang cemara Libanon, tapi kelak apabila kamu ditimpa kesakitan, kamu akan menderita seperti wanita yang sedang melahirkan."
\par 24 TUHAN berkata kepada Raja Yoyakhin, putra Yoyakim raja Yehuda, "Demi Aku, Allah yang hidup, Aku tegaskan bahwa sekalipun engkau bagi-Ku seperti cincin meterai pada tangan kanan-Ku, namun engkau akan Kucabut
\par 25 dan Kuserahkan kepada orang-orang yang kautakuti dan yang ingin membunuhmu. Engkau akan Kuserahkan kepada Nebukadnezar raja Babel, bersama anak buahnya.
\par 26 Engkau dan ibumu akan Kuusir dan Kubuang ke negeri yang bukan tanah tumpah darahmu, dan di sana kamu akan mati.
\par 27 Kamu akan sangat merindukan negeri ini, tapi kamu tak akan kembali."
\par 28 Aku Yeremia berkata, "Apakah Raja Yoyakhin telah menjadi seperti periuk pecah yang dibuang dan tidak disukai lagi? Itukah sebabnya ia dan anak-anaknya diusir dan dibuang ke negeri yang asing bagi mereka?"
\par 29 Wahai negeriku, dengarkan apa yang dikatakan TUHAN,
\par 30 "Orang itu telah ditentukan untuk kehilangan anak-anaknya dan selalu gagal dalam hidupnya. Dari keturunannya tak ada yang akan berhasil memerintah di Yehuda sebagai raja keturunan Daud. Aku, TUHAN, telah berbicara."

\chapter{23}

\par 1 TUHAN Allah Israel berkata, "Celakalah kamu hai para pemimpin yang membinasakan dan menceraiberaikan umat-Ku! Kamu seharusnya memelihara umat-Ku, tetapi kamu tidak melakukannya; kamu membiarkannya terserak dan lari. Sekarang Aku akan menghukum kamu karena kejahatanmu.
\par 3 Tetapi sisa-sisa umat-Ku akan Kukumpulkan dari negeri-negeri di tempat mereka Kubuang. Mereka akan Kubawa kembali ke tanah air mereka dan keturunan mereka akan bertambah banyak.
\par 4 Aku akan mengangkat pemimpin untuk memelihara mereka. Umat-Ku tidak akan takut lagi atau gentar; tak seorang pun dari mereka akan hilang. Aku, TUHAN, telah berbicara."
\par 5 TUHAN berkata, "Akan tiba waktunya, Aku mengangkat seorang raja yang adil dari keturunan Daud. Raja itu akan memerintah dengan bijaksana, dan melakukan apa yang adil dan benar di seluruh negeri.
\par 6 Apabila ia memerintah, orang Yehuda akan selamat, dan orang Israel akan hidup dengan aman. Raja itu akan disebut 'TUHAN Keselamatan Kita'.
\par 7 Akan tiba waktunya, orang tidak lagi bersumpah demi Aku, Allah yang hidup, yang membawa umat Israel keluar dari Mesir,
\par 8 melainkan demi Aku, Allah yang hidup, yang membawa umat Israel keluar dari sebuah negeri di utara, dan dari semua negeri lain di mana mereka telah Kuceraiberaikan. Mereka akan tinggal di tanah airnya sendiri."
\par 9 Inilah pesanku mengenai nabi-nabi: Hatiku hancur dan aku gemetar. TUHAN dan perkataan-Nya yang suci membuat aku seperti orang mabuk yang terlalu banyak minum anggur.
\par 10 Negeri kita penuh dengan orang yang tidak setia kepada TUHAN. Mereka hidup jahat dan menyalahgunakan kekuasaan mereka. Karena kutukan TUHAN, rumput di padang menjadi kering dan seluruh negeri berdukacita.
\par 11 TUHAN berkata, "Nabi dan imam tak mau taat kepada-Ku. Bahkan di Rumah-Ku Aku telah mendapati mereka berbuat jahat.
\par 12 Karena itu, jalan mereka akan licin dan gelap, sehingga mereka tersandung dan jatuh. Aku akan mendatangkan celaka atas mereka, apabila telah tiba waktunya untuk menghukum mereka. Aku, TUHAN, telah berbicara.
\par 13 Dosa nabi-nabi Samaria sudah Kulihat. Mereka berbicara atas nama Baal, dan menyesatkan Israel, umat-Ku.
\par 14 Tapi di Yerusalem Aku melihat nabi-nabi melakukan yang lebih jahat lagi. Mereka berdusta dan berzinah, serta menyokong orang yang berbuat jahat, sehingga tak seorang pun mau bertobat. Bagi-Ku mereka semuanya jahat seperti penduduk Sodom dan Gomora."
\par 15 TUHAN Yang Mahakuasa berkata, "Akan Kuberikan kepada para nabi di Yerusalem itu tanaman pahit untuk dimakan dan racun untuk diminum, karena mereka telah menyebabkan orang-orang di seluruh negeri tidak menghormati Aku."
\par 16 Kepada penduduk Yerusalem, TUHAN Yang Mahakuasa berkata, "Jangan dengarkan perkataan para nabi yang selalu hanya memberi harapan yang kosong. Mereka hanya menyampaikan khayalan mereka sendiri dan bukan pesan-Ku.
\par 17 Kepada orang-orang yang tak mau mendengarkan kata-kata-Ku, mereka selalu berkata, 'Kamu akan selamat.' Dan kepada setiap orang yang keras kepala, mereka berkata, 'Kau tak akan mendapat celaka.'"
\par 18 Aku berkata, "Tidak seorang pun dari nabi-nabi itu mengetahui apa yang terkandung dalam pikiran TUHAN. Tak ada dari mereka pernah mendengar dan mengerti rencana-rencana-Nya; tak ada juga yang memperhatikan atau mengindahkan kata-kata-Nya.
\par 19 Kemarahan TUHAN bagaikan angin ribut yang mengamuk dan melanda orang-orang yang jahat.
\par 20 Kemarahan-Nya tak akan reda sebelum semua yang direncanakan-Nya terlaksana. Di kemudian hari umat TUHAN akan mengerti hal itu dengan jelas."
\par 21 TUHAN berkata, "Giat sekali nabi-nabi itu, padahal mereka tidak Kusuruh. Atas nama-Ku mereka menyampaikan berita, padahal tak ada pesan yang Kuberikan kepada mereka.
\par 22 Andaikata mereka tahu apa yang terkandung dalam pikiran-Ku, tentulah mereka telah menyampaikan kepada umat-Ku segala yang telah Kuucapkan. Maka umat-Ku akan bertobat dari dosa-dosa dan tingkah lakunya yang jahat.
\par 23 Aku Allah yang berada di mana-mana, bukan di satu tempat saja.
\par 24 Tak seorang pun dapat menyembunyikan dirinya dari pandangan-Ku, sebab Aku ada di mana-mana: di surga dan di bumi.
\par 25 Semua yang dikatakan oleh nabi-nabi itu Aku sudah tahu. Atas nama-Ku mereka berdusta bahwa pesan-Ku telah mereka terima melalui mimpi.
\par 26 Sampai kapan nabi-nabi itu hendak menyesatkan umat-Ku dengan berita karangan mereka sendiri?
\par 27 Mereka pikir bahwa dengan mimpi-mimpi yang mereka ceritakan itu, umat-Ku dapat lupa kepada-Ku seperti leluhur mereka melupakan Aku dan menyembah Baal.
\par 28 Nabi yang bermimpi seharusnya berkata, 'Ini hanya mimpi.' Tapi nabi yang mendapat pesan dari Aku haruslah menyampaikan pesan itu dengan sebenarnya. Gandum tak dapat disamakan dengan jerami.
\par 29 Perkataan-Ku seperti api, dan seperti palu yang menghancurkan batu!
\par 30 Aku akan melawan nabi-nabi yang menyampaikan pesan dari sesama rekannya, lalu mengatakan bahwa itu pesan dari Aku.
\par 31 Aku juga melawan nabi-nabi yang menyampaikan kata-katanya sendiri, lalu mengatakan bahwa itu dari Aku.
\par 32 Aku, TUHAN, akan melawan nabi-nabi itu yang menyampaikan mimpi-mimpinya yang penuh dusta. Mereka menyesatkan umat-Ku dengan dusta dan bualan mereka. Aku tidak menyuruh nabi-nabi itu. Mereka sama sekali tidak berguna untuk umat-Ku. Aku, TUHAN, telah berbicara."
\par 33 TUHAN berkata, "Yeremia, jika salah seorang dari umat-Ku, atau seorang nabi atau imam bertanya kepadamu, 'Apa pesan TUHAN?' Engkau harus menjawab, 'Engkaulah beban TUHAN; sebab itu engkau akan dibuangnya.'
\par 34 Jika salah seorang dari umat-Ku atau nabi atau imam masih berbicara tentang 'beban TUHAN', maka dia dan keluarganya akan Kuhukum.
\par 35 Sebaliknya, setiap orang seharusnya menanyakan kepada kawannya dan kepada keluarganya, 'Apakah jawaban dari TUHAN?' atau 'Apakah yang dikatakan TUHAN?'
\par 36 Mereka tidak boleh lagi menggunakan istilah 'beban TUHAN'. Kalau ada yang masih menggunakan istilah itu, maka Aku akan menjadikan pesan-Ku betul-betul beban baginya. Orang yang mengatakan 'beban TUHAN' hanyalah memutarbalikkan perkataan-Ku, Allah mereka, Allah yang hidup, TUHAN Yang Mahakuasa.
\par 37 Mereka hanya boleh mengatakan kepada nabi-nabi, 'Apakah jawaban TUHAN kepadamu? Apakah yang dikatakan TUHAN?'
\par 38 Kalau ada yang tidak taat kepada perintah-Ku itu, dan masih memakai istilah 'beban TUHAN',
\par 39 maka Aku akan memungut mereka, dan melemparkan mereka jauh-jauh, baik mereka maupun kota yang telah Kuberikan kepada mereka dan leluhur mereka.
\par 40 Aku akan membuat mereka malu dan hina untuk selama-lamanya dan nasib mereka itu tak akan dilupakan orang."

\chapter{24}

\par 1 Suatu waktu setelah Yoyakhin putra Yoyakim raja Yehuda bersama para pejabat pemerintah Yehuda, para pengrajin dan para pekerja ahli diangkut oleh Nebukadnezar raja Babel, dan dibawa sebagai tawanan dari Yerusalem ke Babel, TUHAN memberi suatu penglihatan kepadaku. Aku melihat dua keranjang berisi buah ara terletak di depan Rumah TUHAN.
\par 2 Keranjang yang pertama berisi buah ara yang bagus-bagus, seperti buah ara hasil pertama, sedangkan keranjang yang kedua berisi buah ara yang busuk-busuk, sehingga tak dapat dimakan.
\par 3 Lalu TUHAN berkata kepadaku, "Yeremia, apa yang kaulihat?" Aku menjawab, "Buah ara, TUHAN. Yang bagus-bagus sangat bagus, dan yang busuk-busuk sangat busuk sehingga tak dapat dimakan."
\par 4 Maka TUHAN berkata lagi,
\par 5 "Orang-orang yang telah Kubiarkan diangkut ke Babel itu Kuanggap seperti buah ara yang bagus-bagus ini; karena itu Aku, TUHAN, Allah Israel, akan memperlakukan mereka dengan baik.
\par 6 Aku akan menjaga mereka dan membawa mereka kembali ke negeri ini. Aku akan membangun, bukan meruntuhkan mereka; Aku akan menanam, bukan mencabut mereka.
\par 7 Aku akan memberikan kepada mereka hasrat untuk mengerti bahwa Akulah TUHAN. Maka mereka akan menjadi umat-Ku, dan Aku menjadi Allah mereka, karena mereka akan kembali kepada-Ku dengan sepenuh hati.
\par 8 Tetapi Zedekia raja Yehuda, beserta para pejabat yang dekat dengan dia, dan sisa orang Yerusalem yang tetap tinggal di negeri ini atau pindah ke Mesir, mereka semuanya akan Kuperlakukan seperti buah ara yang busuk-busuk ini, yang tak dapat dimakan. Aku, TUHAN, telah berbicara.
\par 9 Aku akan mencelakakan mereka sedemikian rupa sehingga segala bangsa di seluruh dunia menjadi ketakutan. Di mana saja mereka Kuserakkan, mereka akan diejek, dijadikan bahan tertawaan dan dicemooh. Nama mereka akan dipakai sebagai kutukan.
\par 10 Aku akan mendatangkan ke atas mereka peperangan, kelaparan, dan wabah penyakit sehingga mereka semua mati dan tidak ada yang tertinggal di negeri ini yang telah Kuberikan kepada mereka dan leluhur mereka."

\chapter{25}

\par 1 Pada tahun keempat setelah Yoyakim putra Yosia menjadi raja Yehuda, aku mendapat pesan dari TUHAN mengenai seluruh bangsa Yehuda. Waktu itu tahun pertama setelah Nebukadnezar menjadi raja Babel.
\par 2 Aku berkata kepada seluruh bangsa Yehuda dan penduduk Yerusalem,
\par 3 "Selama dua puluh tiga tahun, sejak tahun ketiga belas pemerintahan Yosia putra Amon atas Yehuda, sampai hari ini TUHAN telah berbicara kepadaku dan tidak pernah aku lalai menyampaikan kepadamu apa yang telah dikatakan-Nya. Tapi kamu tidak peduli.
\par 4 TUHAN juga terus-menerus mengutus kepadamu para nabi hamba-hamba-Nya, tapi kamu tidak mau mendengarkan dan memperhatikan mereka.
\par 5 Mereka menyuruh kamu meninggalkan cara hidupmu yang jahat, dan perbuatan-perbuatanmu yang berdosa, supaya kamu dapat tetap tinggal di negeri ini yang telah diberikan TUHAN kepadamu dan kepada leluhurmu sebagai milik pusaka.
\par 6 Nabi-nabi itu melarang kamu menyembah dan beribadat kepada ilah-ilah lain. Mereka melarang kamu menyembah berhala-berhala buatanmu sendiri, sebab hal itu membuat TUHAN marah. Kalau kamu menuruti perintah TUHAN itu, Ia tidak menghukum kamu.
\par 7 Tapi TUHAN sendiri berkata bahwa kamu tidak mendengarkan Dia. Kamu malah membangkitkan kemarahan-Nya karena berhala-berhalamu itu. Kamu sendirilah yang mendatangkan hukuman-Nya ke atas dirimu.
\par 8 Karena kamu tidak mau mendengarkan kata-kata-Nya, maka TUHAN Yang Mahakuasa berkata,
\par 9 'Aku akan memanggil semua bangsa dari utara, dan juga hamba-Ku, Nebukadnezar raja Babel. Mereka akan Kukerahkan untuk berperang melawan Yehuda dan penduduknya serta segala bangsa di sekitarnya. Bangsa Yehuda bersama bangsa-bangsa tetangganya akan Kubinasakan dan Kubiarkan hancur untuk selamanya. Orang-orang yang melihatnya akan terkejut dan merasa ngeri. Aku, TUHAN, telah berbicara.
\par 10 Suara-suara mereka yang riang gembira dan keramaian pesta perkawinan akan Kuhentikan. Mereka tidak akan mempunyai minyak untuk lampu, dan gandum mereka akan habis.
\par 11 Seluruh negeri ini akan Kubiarkan hancur dan menjadi sunyi. Bangsa-bangsa di sekitarnya akan mengabdi kepada raja Babel tujuh puluh tahun lamanya.
\par 12 Setelah itu Aku akan menghukum bangsa Babel dan rajanya karena dosa mereka. Negeri Babel akan Kuruntuhkan dan Kubiarkan hancur untuk selamanya.
\par 13 Babel akan Kuhukum dengan semua bencana yang telah Kurencanakan untuk bangsa-bangsa dan yang telah Kuberitahukan melalui Yeremia--yaitu semua bencana yang dicatat dalam buku ini.
\par 14 Perbuatan-perbuatan orang Babel akan Kubalas; mereka akan diperbudak oleh raja-raja besar dan oleh banyak bangsa.'"
\par 15 TUHAN, Allah Israel, berkata kepadaku, "Ambillah gelas anggur ini yang telah Kuisi dengan anggur kemarahan-Ku. Bawalah ke segala bangsa ke mana engkau Kuutus, dan suruhlah mereka minum anggur ini.
\par 16 Apabila mereka meminumnya, mereka terhuyung-huyung dan kehilangan akal karena peperangan yang Kudatangkan kepada mereka."
\par 17 Maka aku menerima gelas anggur itu dari tangan TUHAN, dan pergi kepada segala bangsa ke mana TUHAN telah mengutus aku, lalu aku membuat mereka minum anggur itu.
\par 18 Yerusalem dan semua kota di Yehuda, bersama raja-raja dan pejabat-pejabatnya, kuberi minum anggur itu, sehingga mereka menjadi seperti padang yang tandus dan mengerikan. Sampai pada hari ini nama mereka dipakai sebagai kutukan.
\par 19 Inilah daftar nama orang-orang lain yang harus minum dari gelas anggur TUHAN itu: raja Mesir dengan para pegawai dan pejabat-pejabatnya; semua orang Mesir dan orang asing di Mesir; semua raja di negeri Us; semua raja kota-kota Filistin di Askelon, Gaza, Ekron; semua orang yang masih hidup di Asdod; semua orang Edom, Moab dan Amon; semua raja di Tirus dan Sidon; semua raja di daerah pesisir Laut Tengah; kota-kota Dedan, Tema, dan Bus; semua orang yang memangkas pendek rambutnya; semua raja Arab; semua raja suku-suku campuran yang tinggal di padang gurun; semua raja di Zimri, Elam, dan Madai; semua raja di utara, jauh dan dekat, seorang demi seorang. Segala bangsa di seluruh muka bumi harus minum anggur itu. Dan orang terakhir yang harus minum anggur itu adalah raja Babel.
\par 27 Lalu TUHAN berkata kepadaku, "Katakanlah kepada orang-orang itu bahwa Aku, TUHAN Yang Mahakuasa, Allah Israel, menyuruh mereka minum sampai mabuk dan muntah-muntah. Kubuat mereka jatuh dan tak dapat bangun lagi karena peperangan yang Kudatangkan kepada mereka.
\par 28 Dan kalau mereka tidak mau menerima gelas itu dari tanganmu untuk minum anggurnya, katakanlah bahwa TUHAN Yang Mahakuasa telah memerintahkan hal itu.
\par 29 Aku akan memulai penghancuran itu di kota-Ku sendiri. Jangan sampai mereka menyangka bahwa mereka tidak akan dihukum. Mereka pasti akan dihukum karena Aku akan mendatangkan peperangan kepada segala bangsa di seluruh muka bumi. Aku, TUHAN Yang Mahakuasa, telah berbicara.
\par 30 Semua yang Kukatakan kepadamu, hai Yeremia, haruslah kausampaikan. Katakanlah kepada mereka, 'Suara TUHAN menggelegar dari surga; dari kediaman-Nya yang suci Ia menggemuruh melawan umat-Nya. Seluruh penduduk bumi mendengar Ia memekik seperti pekerja yang memeras anggur di tempat pengirik.
\par 31 Sampai ke ujung bumi suara-Nya menggema; TUHAN membuat perkara terhadap bangsa-bangsa. Semua orang akan diadili dan yang jahat akan dihukum mati.'"
\par 32 TUHAN Yang Mahakuasa berkata, "Dari ujung-ujung bumi angin ribut menggemuruh; bencana menimpa bangsa-bangsa satu demi satu."
\par 33 Pada hari itu mayat orang-orang yang ditewaskan oleh TUHAN akan bergelimpangan di seluruh muka bumi. Mereka tidak akan ditangisi, tidak pula diangkat untuk dikuburkan, melainkan dibiarkan saja di tanah seperti timbunan sampah.
\par 34 Hai pemimpin-pemimpin, hai para gembala umat TUHAN! Menangislah keras-keras! Berguling-gulinglah dalam abu, sebab sudah tiba waktunya kamu dibunuh, dan yang luput akan diceraiberaikan. Kamu akan seperti bejana berharga yang jatuh dan pecah.
\par 35 Kamu pasti tak akan lolos.
\par 36 Kamu berkeluh kesah dan menangis dengan sedih karena TUHAN dalam kemarahan-Nya telah membinasakan bangsamu serta menghancurkan negerimu yang aman itu.
\par 38 Seperti singa keluar dari liangnya, demikian pun TUHAN keluar untuk menghukum umat-Nya. Kekejaman perang dan kemarahan TUHAN telah mengubah negeri ini menjadi padang gurun.

\chapter{26}

\par 1 Segera setelah Yoyakim putra Yosia menjadi raja Yehuda,
\par 2 TUHAN berkata kepadaku, "Pergilah berdiri di pelataran Rumah-Ku dan berbicaralah kepada penduduk Yehuda yang datang beribadat di situ. Sampaikanlah kepada mereka semua yang Kuperintahkan kepadamu untuk diberitahukan kepada mereka. Jangan kurangi sedikit pun.
\par 3 Barangkali bangsa itu mau mendengarkan, lalu berhenti berbuat jahat, sehingga Aku mengurungkan niat-Ku untuk mencelakakan mereka sesuai dengan perbuatan mereka yang jahat."
\par 4 TUHAN menyuruh aku mengatakan begini kepada mereka, "Aku, TUHAN, sudah memerintahkan supaya kamu taat kepada-Ku dan menjalankan hukum-hukum yang telah Kuberikan kepadamu.
\par 5 Kamu harus pula mendengarkan apa yang dikatakan oleh hamba-hamba-Ku, para nabi yang terus-menerus Kuutus kepadamu. Tapi kamu tidak mau mendengarkan mereka.
\par 6 Jika kamu tetap membangkang, rumah ibadat ini akan Kuperlakukan seperti Silo, sehingga segala bangsa di dunia memakai nama kota Yerusalem ini sebagai kutukan."
\par 7 Semua yang kukatakan di Rumah TUHAN itu didengar oleh para imam, para nabi dan rakyat.
\par 8 Segera setelah aku selesai mengumumkan semua pesan TUHAN seperti yang diperintahkan-Nya, mereka menangkap aku sambil berteriak, "Kau harus mati!
\par 9 Berani benar engkau mengatakan atas nama TUHAN bahwa Rumah-Nya ini akan menjadi seperti Silo, dan bahwa kota ini akan dimusnahkan dan menjadi tempat yang tidak didiami lagi." Lalu aku dikerumuni orang banyak.
\par 10 Ketika pejabat-pejabat pemerintah Yehuda mendengar apa yang telah terjadi, mereka cepat-cepat keluar dari istana raja lalu pergi ke Rumah TUHAN, dan duduk di Pintu Gerbang Baru.
\par 11 Para imam dan nabi-nabi berkata kepada pejabat-pejabat itu dan kepada rakyat, "Orang ini patut dihukum mati karena ia telah mengeluarkan kata-kata yang menentang kota kita ini. Kamu sendiri sudah mendengarnya."
\par 12 Lalu aku berkata, "Tuhanlah yang menyuruh aku menyampaikan pesan yang menentang Rumah TUHAN dan kota ini seperti yang kamu dengar.
\par 13 Sebab itu ubahlah cara hidupmu dan perbuatan-perbuatanmu; taatilah TUHAN Allahmu, supaya Ia mengurungkan niat-Nya untuk mencelakakan kamu.
\par 14 Mengenai diriku, memang ada dalam kuasamu untuk melakukan apa saja menurut kemauanmu.
\par 15 Tapi ingat, kalau kamu membunuh aku, kamu dan penduduk kota ini akan menanggung kesalahan atas pembunuhan terhadap orang yang tidak bersalah. Sebab, Tuhanlah yang mengutus aku untuk memberi peringatan itu kepadamu."
\par 16 Maka para pejabat pemerintah dan rakyat berkata kepada imam-imam dan nabi-nabi itu, "Orang ini berbicara atas nama TUHAN Allah kita; tidak patut ia dihukum mati."
\par 17 Setelah itu beberapa di antara para pemimpin bangsa tampil ke depan dan berkata kepada orang-orang yang berkumpul di situ,
\par 18 "Ketika Hizkia menjadi raja Yehuda, Nabi Mikha dari Moresyet menyampaikan kepada rakyat perkataan ini dari TUHAN Yang Mahakuasa, 'Sion akan dibajak seperti ladang, Yerusalem akan menjadi timbunan reruntuhan, dan bukit Rumah TUHAN akan menjadi hutan.'
\par 19 Tapi Raja Hizkia dan orang Yehuda tidak membunuh Mikha. Hizkia justru takut kepada TUHAN dan minta dikasihani. Maka TUHAN menarik kembali ancaman-Nya untuk mencelakakan mereka. Dan sekarang, hampir saja kita ini mendatangkan celaka yang besar atas diri kita."
\par 20 Pernah juga seorang lain berbicara menentang kota dan bangsa ini atas nama TUHAN, seperti yang kulakukan. Orang itu bernama Uria anak Semaya dari Kiryat-Yearim.
\par 21 Ketika Raja Yoyakim bersama perwira-perwiranya dan para pejabat pemerintah mendengar ucapan-ucapan Uria, raja berusaha membunuh dia. Tapi Uria mendengar hal itu lalu menjadi takut dan lari ke Mesir.
\par 22 Tapi Raja Yoyakim menyuruh Elnatan anak Akhbor bersama beberapa orang lain pergi ke Mesir dan mengambil Uria.
\par 23 Lalu mereka membawa dia kembali kepada Raja Yoyakim. Kemudian raja menyuruh membunuh dia, dan mayatnya dibuang di pekuburan umum.
\par 24 Tapi sekarang berkat pertolongan Ahikam anak Safan, aku Yeremia tidak diserahkan kepada rakyat untuk dibunuh.

\chapter{27}

\par 1 Tak lama setelah Zedekia putra Yosia menjadi raja Yehuda, TUHAN menyuruh aku
\par 2 membuat tali-tali dari kulit serta gandar dari kayu, dan memasangnya di tengkukku.
\par 3 Ia juga menyuruh aku menyampaikan pesan-Nya kepada raja-raja Edom, Moab, Amon, Tirus dan Sidon, dengan perantaraan utusan-utusan mereka yang telah datang ke Yerusalem untuk mengunjungi Raja Zedekia.
\par 4 Aku harus memberitahukan kepada mereka bahwa TUHAN Yang Mahakuasa, Allah Israel, berkata,
\par 5 "Dengan kuasa-Ku yang besar dan dengan kekuatan-Ku Aku telah menciptakan bumi dan manusia serta segala binatang, dan Aku dapat memberikannya kepada siapa saja yang Kukehendaki.
\par 6 Akulah yang telah menyerahkan segala bangsa ini ke dalam kekuasaan hamba-Ku, Nebukadnezar, raja Babel. Bahkan binatang pun telah Kuserahkan kepadanya untuk dikuasai.
\par 7 Segala bangsa akan mengabdi kepadanya, kepada putranya, dan kepada cucunya sampai tiba saatnya negaranya sendiri jatuh. Pada waktu itu bangsa Babel akan mengabdi kepada bangsa-bangsa yang kuat dan raja-raja yang besar.
\par 8 Bangsa atau kerajaan yang tidak mau takluk kepada Nebukadnezar akan Kuhukum dengan peperangan, kelaparan, dan wabah penyakit, sampai akhirnya bangsa itu takluk kepadanya.
\par 9 Janganlah mendengarkan nabi-nabi atau siapa pun juga yang melarang kamu takluk kepada raja Babel, walaupun mereka mengaku telah menerima petunjuk dari mimpi, roh-roh halus, atau dengan memakai kekuatan gaib.
\par 10 Mereka menipu kamu sehingga kamu dibawa jauh dari negerimu. Aku akan mengusir kamu dari negerimu dan kamu akan binasa.
\par 11 Tapi bangsa yang mau takluk dan mau mengabdi kepada raja Babel akan Kubiarkan tinggal di tanah airnya sendiri, dan bercocok tanam di situ. Aku, TUHAN, telah berbicara."
\par 12 Kepada Zedekia raja Yehuda, aku, Yeremia, menyampaikan hal yang sama. Aku berkata, "Menyerahlah kepada raja Babel. Taatlah kepadanya dan kepada bangsanya, supaya Baginda selamat.
\par 13 TUHAN telah berkata bahwa bangsa yang tidak takluk kepada raja Babel akan mati oleh peperangan, kelaparan, atau wabah penyakit. Mengapa Baginda dan rakyat mau mati secara demikian?
\par 14 Jangan dengarkan nabi-nabi yang melarang Baginda takluk kepada raja Babel, sebab mereka menipu Baginda.
\par 15 Mereka berkata bahwa mereka bicara atas nama TUHAN, tapi mereka berdusta. TUHAN sendiri berkata bahwa Ia tidak mengutus mereka. Sebab itu Ia akan menceraiberaikan dan membinasakan Baginda bersama nabi-nabi yang telah berdusta kepada Baginda."
\par 16 Kemudian aku memberitahukan kepada imam-imam dan kepada seluruh rakyat bahwa TUHAN berkata begini, "Jangan dengarkan nabi-nabi yang berkata bahwa tidak lama lagi barang-barang di Rumah-Ku akan dibawa kembali dari Babel. Mereka bohong.
\par 17 Jangan dengarkan mereka! Menyerahlah kepada raja Babel, supaya kamu selamat. Untuk apa kota ini harus hancur?
\par 18 Kalau mereka memang benar nabi, dan telah menerima pesan dari Aku, pasti mereka sekarang minta kepada-Ku, TUHAN Yang Mahakuasa, supaya barang-barang yang masih ada di Rumah-Ku dan di istana raja di Yerusalem tidak diangkut ke Babel.
\par 19 Sebab Aku, TUHAN Yang Mahakuasa, Allah Israel, telah memutuskan bahwa semua barang itu akan diangkut ke Babel, dan akan tetap di sana sampai Aku memperhatikannya lagi. Pada waktu itu barulah Aku membawanya kembali ke tempat ini. Aku, TUHAN, telah berbicara." Barang-barang yang masih ada di Rumah TUHAN itu ialah: tiang-tiang, bejana perunggu, kereta-kereta, dan beberapa barang lain milik Rumah TUHAN. Barang-barang itu tidak ikut dibawa ke Babel, ketika Yoyakhin putra Yoyakim raja Yehuda serta pemuka-pemuka masyarakat Yehuda dan Yerusalem diangkut ke Babel oleh Raja Nebukadnezar.

\chapter{28}

\par 1 Tahun itu juga pada bulan lima tahun keempat pemerintahan Raja Zedekia, Hananya anak Azur, seorang nabi dari kota Gibeon, berbicara kepadaku di Rumah TUHAN. Di depan imam-imam dan rakyat ia mengatakan kepadaku
\par 2 bahwa TUHAN Yang Mahakuasa, Allah Israel, berkata begini, "Kekuatan raja Babel telah Kulumpuhkan.
\par 3 Dalam waktu dua tahun ini Aku akan mengembalikan barang-barang dari Rumah-Ku, yang telah diangkut ke Babel oleh Raja Nebukadnezar.
\par 4 Aku juga akan membawa kembali Yoyakhin putra Yoyakim raja Yehuda, bersama seluruh rakyat Yehuda yang telah diangkut ke Babel. Sungguh, Aku akan melumpuhkan kekuatan Babel. Aku, TUHAN, telah berbicara."
\par 5 Maka di depan imam-imam dan semua rakyat yang berdiri di Rumah TUHAN itu, aku berkata kepada Hananya,
\par 6 "Bagus! Mudah-mudahan saja ramalanmu itu menjadi kenyataan, dan TUHAN betul-betul membawa kembali dari Babel barang-barang Rumah TUHAN bersama dengan semua orang yang telah dibuang ke sana.
\par 7 Tapi engkau, Hananya, dan kamu semua, hai rakyat, dengarkanlah ini:
\par 8 Nabi-nabi sebelum masa kita telah meramalkan bahwa peperangan, malapetaka, serta wabah penyakit akan menimpa banyak bangsa dan kerajaan yang kuat-kuat.
\par 9 Tapi nabi yang meramalkan keadaan damai, hanya dapat diakui sebagai nabi yang sungguh-sungguh diutus oleh TUHAN, kalau ramalan-ramalannya menjadi kenyataan."
\par 10 Lalu Hananya merenggut gandar dari tengkukku dan mematahkannya.
\par 11 Di depan seluruh rakyat yang hadir di situ, ia berkata, "TUHAN berkata bahwa beginilah caranya Ia akan mematahkan gandar yang dipasang Raja Nebukadnezar pada tengkuk segala bangsa; TUHAN akan melakukan itu dalam waktu dua tahun ini." Setelah itu aku pun pergi.
\par 12 Tidak berapa lama setelah kejadian itu, TUHAN menyuruh aku
\par 13 pergi kepada Hananya untuk mengatakan begini, "TUHAN berkata bahwa engkau dapat saja mematahkan gandar kayu, tapi Ia akan menggantikannya dengan gandar besi.
\par 14 TUHAN Yang Mahakuasa, Allah Israel, berkata bahwa Ia akan memasang gandar besi pada tengkuk segala bangsa ini, dan mereka akan mengabdi kepada Nebukadnezar, raja Babel. TUHAN juga berkata bahwa binatang buas pun akan diserahkannya kepada kekuasaan Nebukadnezar."
\par 15 Semua itu kusampaikan kepada Hananya, lalu aku berkata lagi, "Dengarkan, hai Hananya! TUHAN tidak mengutus engkau; engkau telah membuat bangsa ini percaya kepada perkataan dusta.
\par 16 Karena itu TUHAN sendiri berkata bahwa engkau akan dilenyapkan-Nya dari muka bumi. Tahun ini juga engkau akan mati, sebab engkau telah menyuruh bangsa ini melawan TUHAN."
\par 17 Maka tahun itu juga pada bulan ketujuh, Hananya meninggal.

\chapter{29}

\par 1 Aku menulis surat kepada imam-imam, nabi-nabi, pemimpin-pemimpin bangsa yang masih hidup, dan kepada semua orang lain yang telah diangkut dari Yerusalem ke Babel oleh Nebukadnezar.
\par 2 Surat itu kutulis setelah Raja Yoyakhin, ibunya, pegawai-pegawai istana, pejabat-pejabat pemerintahan Yehuda dan Yerusalem, para pengrajin, dan para pekerja ahli diangkut ke pembuangan.
\par 3 Aku menitipkan surat itu kepada Elasa anak Safan, dan Gemarya anak Hilkia yang diutus Zedekia raja Yehuda kepada Nebukadnezar raja Babel. Surat itu berbunyi,
\par 4 "Beginilah kata TUHAN Yang Mahakuasa, Allah Israel kepada semua orang yang telah dibuangnya dari Yerusalem ke Babel:
\par 5 'Menetaplah di situ. Bangunlah rumah-rumahmu. Bukalah ladang dan nikmatilah hasilnya.
\par 6 Kawinlah supaya kamu mendapat anak, dan biarlah anak-anakmu juga kawin supaya mereka pun mendapat anak. Jumlahmu harus bertambah dan tidak boleh berkurang.
\par 7 Bekerjalah untuk kesejahteraan kota-kota tempat kamu Kubuang. Berdoalah kepada-Ku untuk kepentingan kota-kota itu, sebab kalau kota-kota itu makmur, kamu pun akan makmur.
\par 8 Aku TUHAN, Yang Mahakuasa, Allah Israel, memperingatkan kamu supaya jangan membiarkan dirimu ditipu oleh nabi-nabi yang tinggal di tengah-tengahmu atau oleh siapapun juga yang berkata bahwa mereka dapat meramalkan masa depan. Jangan memperhatikan mimpi-mimpi mereka.
\par 9 Mereka memakai nama-Ku untuk menceritakan kepadamu hal-hal yang tidak benar. Aku tidak menyuruh mereka. Aku, TUHAN, telah berbicara.'
\par 10 TUHAN berkata, 'Apabila telah genap masa tujuh puluh tahun bagi Babel, Aku akan memperhatikan kamu lagi, dan menepati janji-Ku untuk membawa kamu kembali ke tempat ini.
\par 11 Bukankah Aku sendiri tahu rencana-rencana-Ku bagi kamu? Rencana-rencana itu bukan untuk mencelakakan kamu, tetapi untuk kesejahteraanmu dan untuk memberikan kepadamu masa depan yang penuh harapan.
\par 12 Maka kamu akan minta tolong kepada-Ku. Kamu akan datang untuk berdoa kepada-Ku, dan Aku akan menjawab doamu.
\par 13 Kamu akan mencari Aku dan menemukan Aku, sebab kamu mencari dengan sepenuh hati.
\par 14 Sungguh, kamu akan menemukan Aku. Kamu akan Kukumpulkan dari tengah-tengah setiap bangsa dan dari setiap negeri tempat kamu diceraiberaikan dan dibuang. Kamu akan Kubawa kembali ke negeri asalmu ini serta memulihkan keadaanmu. Aku, TUHAN, telah berbicara.'
\par 15 Menurut kamu, TUHAN telah memberikan kepadamu nabi-nabi di Babel.
\par 16 Dengarkan apa yang dikatakan TUHAN tentang raja yang memerintah sekarang ini, yaitu raja keturunan Daud, dan tentang penduduk kota ini, yaitu sanak keluargamu yang tidak diangkut ke pembuangan bersama kamu.
\par 17 TUHAN Yang Mahakuasa berkata, 'Aku akan mendatangkan peperangan, kelaparan, dan wabah penyakit ke atas mereka. Mereka akan Kuperlakukan seperti buah-buah ara yang terlalu busuk sehingga tak dapat dimakan.
\par 18 Mereka akan Kukejar-kejar dengan peperangan, kelaparan, dan wabah penyakit sehingga segala bangsa di seluruh dunia akan merasa ngeri melihat keadaan mereka. Di mana saja Aku menceraiberaikan mereka, orang akan merasa ngeri dan takut melihat apa yang telah terjadi dengan mereka. Mereka akan diejek, dan nama mereka dipakai sebagai kutukan.
\par 19 Semua itu terjadi atas mereka, karena mereka tidak menuruti pesan-Ku yang terus-menerus Kusampaikan kepada mereka melalui para nabi hamba-hamba-Ku. Mereka tidak mau memperhatikan pesan-Ku itu.
\par 20 Dan kamu semua yang telah Kubuang ke Babel, dengarkan apa yang Aku, TUHAN, katakan.'
\par 21 TUHAN Yang Mahakuasa, Allah Israel, telah berbicara mengenai Ahab putra Kolaya, dan mengenai Zedekia putra Maaseya yang memakai nama TUHAN untuk menyampaikan kepadamu hal-hal yang tidak benar. TUHAN berkata bahwa Ia akan menyerahkan mereka kepada kekuasaan Nebukadnezar raja Babel, yang akan membunuh mereka di depan matamu.
\par 22 Maka apabila orang-orang yang telah diangkut dari Yerusalem ke Babel hendak mengutuki seseorang, mereka akan berkata, 'Semoga TUHAN memperlakukan engkau seperti Zedekia dan Ahab, yang telah dibakar hidup-hidup oleh raja Babel.'
\par 23 Begitulah nasib mereka karena telah melakukan dosa-dosa besar di Israel: mereka berzinah, dan memakai nama TUHAN untuk menceritakan hal-hal yang tidak benar. Itu bertentangan dengan kemauan TUHAN. Ia tahu apa yang mereka lakukan, dan ia sendiri menjadi saksi melawan mereka. TUHAN telah berbicara."
\par 24 TUHAN Yang Mahakuasa, Allah Israel memberi pesan kepadaku untuk Semaya, orang Nehelam, karena Semaya telah mengirim surat atas namanya sendiri kepada seluruh penduduk Yerusalem, kepada Imam Zefanya anak Maaseya, dan kepada semua imam yang lain. Dalam surat itu Semaya menulis begini kepada Zefanya:
\par 26 "TUHAN telah mengangkat engkau menjadi imam menggantikan Yoyada, dan sekarang engkaulah pejabat tertinggi di Rumah TUHAN. Tugasmu ialah memerintahkan supaya setiap orang gila, yang mengaku dirinya nabi, dibelenggu dengan rantai besi pada lehernya.
\par 27 Tetapi mengapa engkau tidak bertindak terhadap Yeremia orang Anatot itu yang telah berlaku sebagai nabi di antara bangsa kita?
\par 28 Orang itu harus ditutup mulutnya, sebab ia mengatakan kepada orang-orang kita di Babel bahwa mereka masih lama tinggal di sana. Dan karena itu, katanya, mereka harus membangun rumah serta membuka ladang dan menikmati hasilnya."
\par 29 Surat itu dibacakan oleh Imam Zefanya kepadaku.
\par 30 Sebab itu TUHAN menyuruh aku
\par 31 mengirim berita tentang Semaya kepada semua orang Israel yang dibuang di Babel. Inilah berita itu, "Semaya telah berbicara kepadamu seolah-olah ia nabi; padahal Aku, TUHAN, tidak menyuruh dia. Karena ia telah membuat kamu percaya kepada perkataan dusta,
\par 32 dan karena ia telah menyuruh kamu melawan Aku, maka Aku akan menghukum dia dan keturunannya. Tidak seorang pun dari antara mereka yang akan tinggal di antara kamu untuk melihat hal-hal baik yang akan Kulakukan untuk seluruh umat-Ku. Aku, TUHAN, telah berbicara."

\chapter{30}

\par 1 TUHAN, Allah Israel,
\par 2 berkata kepadaku, "Tulislah dalam sebuah buku semua yang telah Kukatakan kepadamu.
\par 3 Sebab akan tiba waktunya Aku memulihkan keadaan umat-Ku Israel dan Yehuda. Aku akan membawa mereka kembali ke negeri yang telah Kuberikan kepada leluhur mereka. Negeri itu akan mereka miliki kembali. Aku, TUHAN, telah berbicara."
\par 4 TUHAN berkata kepada umat Israel dan Yehuda,
\par 5 "Aku telah mendengar jeritan orang yang gentar, jeritan orang ketakutan yang tidak mempunyai kedamaian.
\par 6 Cobalah pikir dan selidiki! Mungkinkah laki-laki melahirkan bayi? Kalau begitu, mengapa Kulihat setiap laki-laki berwajah pucat pasi dan menahan perutnya dengan tangan seperti wanita yang hendak melahirkan?
\par 7 Hari dahsyat telah tiba, hari yang tak ada taranya. Bagi umat-Ku, itu hari yang mencemaskan; tapi mereka akan Kuselamatkan."
\par 8 TUHAN Yang Mahakuasa berkata lagi kepada umat-Nya, "Apabila tiba hari itu, gandar yang dikenakan pada tengkukmu akan Kupatahkan, dan belenggumu akan Kulepaskan. Kamu tidak akan menjadi budak orang asing lagi.
\par 9 Kamu akan mengabdi kepada-Ku, TUHAN Allahmu, dan kepada seorang keturunan Daud yang akan Kuangkat sebagai raja.
\par 10 Umat-Ku Israel, janganlah takut! Hamba-Ku Yakub, jangan gentar! Kamu dan keturunanmu akan Kuselamatkan dari negeri jauh, tempat kamu ditawan. Kamu akan pulang dan tinggal di negerimu dengan aman. Kamu akan hidup dengan tentram; tak ada yang perlu kamu takutkan.
\par 11 Aku akan datang dan kamu akan Kuselamatkan. Bangsa-bangsa, tempat kamu Kuceraiberaikan semuanya akan Kubinasakan, tapi kamu tidak Kuperlakukan demikian. Memang kamu tidak akan luput dari hukuman tetapi hukuman-Ku kepadamu adalah adil sesuai dengan ketentuan. Aku, TUHAN, telah berbicara."
\par 12 TUHAN berkata kepada umat-Nya: "Penyakitmu sangat parah, bagi lukamu tak ada obatnya.
\par 13 Tak seorang pun sudi memperjuangkan kesembuhanmu, tiada obat untuk bisul-bisulmu, tiada harapan bagimu untuk sembuh.
\par 14 Semua kekasihmu telah lupa padamu; mereka tak mau lagi memikirkan dirimu. Seperti musuh Aku telah menyerang dan menghukum kamu dengan kejam, karena banyaklah dosamu dan besar kesalahanmu.
\par 15 Janganlah lagi mengeluh mengenai luka-lukamu, sebab bagimu tak ada harapan sembuh. Kamu Kuhukum begitu karena banyaklah dosamu dan besar kesalahanmu.
\par 16 Tapi sekarang semua yang menelanmu akan ditelan, dan semua musuhmu akan diangkut ke pembuangan. Aku akan menindas yang menindasmu, Aku akan merampok yang merampokmu.
\par 17 Sekalipun musuhmu berkata, 'Sion telah dibuang, dan tak ada yang memperhatikan,' namun kamu akan Kubuat sehat kembali, dan luka-lukamu akan kuobati."
\par 18 TUHAN berkata lagi: "Kamu, umat-Ku akan Kukasihani dan Kupulihkan keadaanmu di tanah airmu sendiri. Yerusalem akan dibangun lagi, dan istananya diperbaiki.
\par 19 Penduduknya akan memuji Aku dengan nyanyian, dan bersorak-sorai dengan riang. Aku membuat mereka dihormati dan tidak lagi dihina, dan jumlah mereka akan terus bertambah.
\par 20 Bangsamu akan Kujadikan jaya seperti dahulu, dan berdiri lagi sebagai umat-Ku. Semua yang menindasnya akan Kuhukum juga.
\par 21 Raja yang memerintah kamu berasal dari bangsamu. Dengan bebas ia akan mendekati Aku apabila ia Kupanggil menghadap-Ku. Sebab siapakah yang berani menghadap Aku atas kemauannya sendiri? Kamu akan menjadi umat-Ku dan Aku menjadi Allahmu. Aku, TUHAN yang mengatakan itu."
\par 23 Kemarahan TUHAN bagaikan badai, bagaikan angin ribut yang mengamuk menimpa orang-orang jahat, dan tak akan reda sebelum segala rencana-Nya terlaksana. Di kemudian hari umat-Nya akan memahami hal itu.

\chapter{31}

\par 1 TUHAN berkata, "Akan tiba waktunya Aku menjadi Allah semua suku Israel, dan mereka menjadi umat-Ku.
\par 2 Di padang gurun Aku menunjukkan belas kasihan-Ku kepada mereka yang telah luput dari maut. Ketika umat Israel mencari ketenangan,
\par 3 dari jauh Aku menampakkan diri-Ku kepada mereka. Hai umat Israel, sejak dulu Aku selalu mengasihi kamu, dan untuk seterusnya Aku akan tetap menunjukkan bahwa Aku selalu mengasihi kamu.
\par 4 Bangsamu akan Kujadikan jaya seperti dahulu. Sekali lagi kamu akan mengambil rebana, dan menari dengan gembira.
\par 5 Kamu akan membuka kebun anggur di pegunungan Samaria, dan yang menanam akan memetik buahnya pula.
\par 6 Sungguh, akan tiba waktunya para pengawal kota berseru di pegunungan Efraim, 'Mari kita naik ke Sion, kepada TUHAN Allah kita.'"
\par 7 TUHAN berkata, "Bernyanyilah gembira bagi Israel, umat-Ku--bangsa yang terutama dari segala bangsa. Nyanyikanlah kidung pujian, dan beritakanlah: 'TUHAN telah menyelamatkan yang tersisa dari umat-Nya.'
\par 8 Dari utara Kubawa mereka kembali; Kukumpulkan mereka dari ujung-ujung bumi. Orang buta dan orang lumpuh akan ikut dengan mereka, juga wanita yang baru melahirkan dan yang hamil tua. Mereka semua akan datang kembali dalam jumlah yang besar ke negeri ini.
\par 9 Mereka berjalan dengan bercucuran air mata, Kuhibur dan Kubimbing mereka. Kubawa mereka melalui tempat-tempat yang banyak airnya, melalui jalan rata di tempat mereka tak akan tersandung. Bagi Israel, aku bagaikan ayah; Efraim adalah putra-Ku yang sulung."
\par 10 TUHAN berkata, "Dengarlah hai bangsa-bangsa! Sampaikanlah pesan-Ku ke seberang lautan: Umat-Ku yang telah Kuceraiberaikan akan Kukumpulkan dan Kupelihara seperti gembala menjaga dombanya.
\par 11 Umat Israel telah Kuselamatkan, dan Kubebaskan dari bangsa yang kuat.
\par 12 Ke Bukit Sion mereka akan datang dengan muka yang berseri-seri, sambil menyanyi dengan riang karena semua pemberian-Ku kepada mereka, yaitu gandum, minyak, anggur, sapi dan domba. Mereka akan seperti taman yang cukup airnya, dan tidak kekurangan apa-apa.
\par 13 Pada waktu itu gadis-gadis akan menari dengan bersuka hati. Orang tua dan orang muda sama-sama berbahagia, sebab Aku akan menghibur mereka. Duka mereka Kuubah menjadi kesukaan, kesedihan Kuubah menjadi kebahagiaan.
\par 14 Para imam akan Kupuaskan dengan makanan berkelimpahan; dan semua yang umat-Ku perlukan akan Kuberikan. Aku, TUHAN, yang mengatakan."
\par 15 TUHAN berkata, "Di Rama terdengar suara ratapan, dan keluh kesah yang diliputi kepedihan. Rahel meratapi anak-anaknya, ia tak mau dihibur sebab mereka sudah tiada.
\par 16 Hentikan tangismu dan keringkan air matamu! Takkan sia-sia jerih payahmu untuk anak-anakmu. Bagimu akan ada ganjaran, anak-anakmu pulang dari negeri lawan.
\par 17 Ada harapan bagimu di masa depan; anak-anakmu akan kembali ke kampung halaman.
\par 18 Kudengar umat Israel berkata penuh kesedihan, 'Ya TUHAN, kami seperti hewan yang belum dijinakkan tapi Kaulatih kami untuk patuh, dan sekarang kami siap untuk balik kepada-Mu. Jadi, bawalah kami kembali, ya TUHAN, Allah kami.
\par 19 Engkau telah kami belakangi, tapi perbuatan itu kami sesali. Setelah Engkau menghukum kami, kepala kami tertunduk karena sedih. Kami malu dan menjadi hina karena telah berdosa pada masa muda.'
\par 20 Israel, engkau anak kesayangan-Ku, anak tercinta, buah hati-Ku. Setiap kali Aku mengancammu Aku selalu ingat padamu dengan rindu. Engkau akan Kuperlakukan dengan penuh belas kasihan.
\par 21 Tandailah jalan-jalan, pasanglah rambu-rambu carilah jalan yang kautempuh dahulu. Hai, Israel, pulanglah! Kembalilah ke kota-kotamu yang semula.
\par 22 Sampai kapan kau terus ragu-ragu, hai bangsa yang tak setia kepada-Ku? Aku telah menciptakan sesuatu yang baru di dunia, sesuatu yang ganjil--seganjil wanita melindungi pria."
\par 23 TUHAN Yang Mahakuasa, Allah Israel berkata, "Apabila umat-Ku sudah Kujadikan jaya seperti dahulu, maka di negeri Yehuda dan di desa-desanya akan terdengar lagi orang berkata, 'Semoga TUHAN memberkati bukit suci di Yerusalem tempat kediaman yang khusus bagi-Nya.'
\par 24 Orang akan tinggal di Yehuda dan di desa-desanya. Di sana akan ada petani dan peternak dengan sapi dan dombanya.
\par 25 Mereka yang lelah akan Kusegarkan, dan yang lesu karena kelaparan akan Kukenyangkan.
\par 26 Pada waktu itu orang akan berkata, 'Setelah tidur nyenyak, aku merasa segar.'
\par 27 Aku, TUHAN, berkata bahwa akan tiba masanya negeri Israel dan Yehuda Kupenuhi dengan manusia dan hewan.
\par 28 Sebagaimana dahulu Aku telah menjaga mereka supaya kemudian Aku mencabut, meruntuhkan dan menghancurkan mereka, begitu pula Aku akan menjaga mereka supaya Aku meneguhkan dan bangun mereka kembali.
\par 29 Pada waktu itu orang tidak akan berkata lagi, 'Orang tua makan buah yang asam, gigi anaknya yang merasa ngilu,'
\par 30 melainkan siapa makan buah yang asam, dia sendirilah yang akan ngilu giginya. Setiap orang akan mati karena dosanya sendiri."
\par 31 TUHAN berkata, "Akan tiba masanya Aku membuat perjanjian yang baru dengan umat Israel dan Yehuda.
\par 32 Tapi perjanjian itu bukan seperti perjanjian yang Kubuat dengan leluhur mereka ketika Kutuntun mereka keluar dari Mesir. Sekalipun Aku seperti seorang suami bagi mereka, namun mereka mengingkari perjanjian-Ku dengan mereka.
\par 33 Inilah perjanjian baru yang akan Kubuat dengan umat Israel: Hukum-hukum-Ku akan Kutaruh di dalam batin mereka, dan Kutulis pada hati mereka. Aku akan menjadi Allah mereka, dan mereka akan menjadi umat-Ku.
\par 34 Tak perlu lagi seorang pun dari mereka mengajar sesamanya untuk mengenal TUHAN. Sebab mereka semua, besar kecil, akan mengenal Aku. Kesalahan-kesalahan mereka akan Kulupakan, dosa-dosa mereka akan Kuampuni. Aku, TUHAN, telah berbicara."
\par 35 Matahari disediakan TUHAN untuk menerangi siang; bulan dan bintang-bintang untuk menerangi malam. Laut diaduknya sehingga gelombang bergelora, TUHAN Yang Mahakuasa, itulah namanya.
\par 36 TUHAN berjanji, "Selama hukum alam tak berubah Israel pun akan tetap ada sebagai bangsa.
\par 37 Sekiranya suatu waktu langit dapat diukur, dan dasar bumi dapat diselidiki, pada waktu itulah Israel akan Kutolak karena segala perbuatan mereka yang jahat. Aku, TUHAN, telah berbicara."
\par 38 TUHAN berkata, "Akan tiba masanya seluruh Yerusalem dibangun kembali sebagai kota-Ku, dari Menara Hananeel sampai ke Pintu Gerbang Sudut.
\par 39 Dari situ garis batasnya menuju ke Bukit Gareb, lalu membelok ke Goa.
\par 40 Seluruh lembah itu, yang dipakai sebagai kuburan dan tempat pembuangan sampah, dan semua ladang di tepi Sungai Kidron sampai ke Pintu Gerbang Kuda ke arah timur, akan dikhususkan untuk Aku. Kota Yerusalem tidak akan diruntuhkan lagi atau dihancurkan."

\chapter{32}

\par 1 Pada tahun kesepuluh pemerintahan Zedekia raja Yehuda, yaitu tahun kedelapan belas pemerintahan Nebukadnezar raja Babel, TUHAN berbicara kepadaku.
\par 2 Pada waktu itu tentara raja Babel sedang menyerang Yerusalem. Aku ditahan di pelataran istana raja
\par 3 oleh Raja Zedekia karena dituduh telah mengumumkan bahwa TUHAN berkata begini, "Raja Babel akan Kubiarkan menaklukkan dan menduduki kota ini,
\par 4 dan Raja Zedekia tidak akan luput. Ia akan diserahkan kepada raja Babel, dan menghadap raja itu serta berbicara sendiri dengan dia.
\par 5 Kemudian Zedekia akan diangkut ke Babel dan tinggal di sana sampai Aku menghukum dia. Sekalipun ia memerangi orang Babel, ia tidak akan berhasil. Aku, TUHAN, telah berbicara."
\par 6 TUHAN berkata kepadaku
\par 7 bahwa Hanameel, anak laki-laki pamanku Salum, akan datang untuk minta supaya aku membeli tanahnya di Anatot di wilayah Benyamin. Sebab, akulah sanaknya yang terdekat yang berhak membeli tanah itu.
\par 8 Lalu tepat seperti yang telah dikatakan TUHAN, datanglah Hanameel kepadaku di pelataran istana, dan minta supaya aku membeli tanahnya. Karena itu aku yakin TUHAN betul-betul telah berbicara kepadaku.
\par 9 Maka aku membeli tanah itu dari Hanameel dengan harga tujuh belas uang perak. Kupanggil saksi-saksi dan di depan mereka kutandatangani surat pembeliannya, lalu kububuhi segel dan kutimbang uangnya.
\par 11 Kemudian aku mengambil surat pembelian itu yang sudah diberi segel dan berisi syarat-syarat serta ketentuan-ketentuan. Kuambil juga salinan surat yang tidak ada segelnya,
\par 12 dan kedua surat itu kuberikan kepada Barukh anak Neria cucu Mahseya. Aku menyerahkannya di depan Hanameel serta saksi-saksi yang telah menandatangani surat-surat itu, dan di depan orang-orang yang sedang duduk di pelataran itu.
\par 13 Di depan mereka semua aku berkata kepada Barukh,
\par 14 "TUHAN Yang Mahakuasa, Allah Israel, memerintahkan supaya engkau mengambil kedua surat ini, baik yang asli maupun salinannya, dan memasukkannya ke dalam sebuah periuk tembikar supaya tahan lama.
\par 15 TUHAN Yang Mahakuasa, Allah Israel, berkata bahwa rumah, tanah, dan kebun anggur kelak akan diperjualbelikan lagi di negeri ini."
\par 16 Setelah surat-surat pembelian itu kuberikan kepada Barukh, aku berdoa,
\par 17 "TUHAN Yang Mahatinggi, Engkaulah yang menciptakan langit dan bumi dengan kuasa dan kemampuan-Mu yang besar. Tak ada sesuatu pun yang sukar bagi-Mu.
\par 18 Engkau menunjukkan kasih-Mu yang abadi kepada beribu-ribu orang, tapi Engkau juga menghukum orang karena dosa orang tuanya. Engkau Allah yang agung dan perkasa; nama-Mu TUHAN Yang Mahakuasa.
\par 19 Rencana-rencana-Mu hebat, dan perbuatan-perbuatan-Mu ajaib; Engkau melihat segala yang dilakukan manusia, dan membalas mereka sesuai dengan perbuatan mereka.
\par 20 Dahulu kala Engkau melakukan keajaiban dan hal-hal luar biasa di Mesir, dan sampai sekarang pun Engkau masih terus melakukannya, baik di antara orang Israel maupun di antara segala bangsa lain. Karena itu kini Engkau termasyhur di mana-mana.
\par 21 Dengan kuasa dan kekuatan-Mu yang besar Engkau membawa umat Israel keluar dari Mesir. Engkau melakukan mujizat-mujizat dan keajaiban-keajaiban serta kejadian-kejadian yang menggemparkan musuh.
\par 22 Negeri yang baik dan subur ini Kauberikan kepada mereka seperti yang telah Kaujanjikan kepada leluhur mereka.
\par 23 Tapi ketika mereka datang ke negeri ini dan merebutnya, mereka tidak taat kepada perintah-perintah-Mu, dan tidak hidup menurut ajaran-ajaran-Mu. Semua yang Kauperintahkan kepada mereka tidak satu pun yang mereka laksanakan. Sebab itu Engkau mendatangkan segala bencana ini ke atas mereka.
\par 24 Orang-orang Babel telah membangun tembok-tembok pengepungan di sekeliling kota untuk merebutnya, dan sekarang mereka menyerang. Pertempuran, kelaparan, dan wabah penyakit akan menyebabkan kota ini jatuh ke tangan mereka. Engkau melihat bahwa semua yang Kaukatakan telah menjadi kenyataan.
\par 25 Namun demikian, ya TUHAN Yang Mahatinggi, Engkaulah yang menyuruh aku membeli tanah ini di depan saksi-saksi, sekali pun sebentar lagi kota akan direbut oleh orang Babel."
\par 26 Kemudian TUHAN berkata kepadaku,
\par 27 "Akulah TUHAN, Allah semua orang. Tak ada yang sulit bagi-Ku.
\par 28 Kota ini akan Kubiarkan dikuasai oleh Nebukadnezar raja Babel bersama pasukannya. Mereka akan merebut kota ini,
\par 29 dan membakarnya sampai habis bersama-sama dengan rumah-rumah di mana orang menimbulkan kemarahan-Ku, karena di atas atap rumah-rumah itu mereka membakar dupa kepada Baal dan menuangkan anggur untuk persembahan bagi dewa-dewa.
\par 30 Sejak permulaan sejarah mereka, orang Israel dan orang Yehuda telah membuat Aku tidak senang. Mereka membangkitkan kemarahan-Ku dengan perbuatan-perbuatan mereka yang jahat.
\par 31 Sejak kota ini dibangun, penduduknya telah membuat Aku marah sekali. Aku telah memutuskan untuk membinasakannya,
\par 32 karena segala kejahatan yang dilakukan orang Yehuda dan Yerusalem bersama para raja, pejabat pemerintah, imam, dan nabi-nabi mereka.
\par 33 Mereka telah meninggalkan Aku dan sekali pun Aku terus mengajar mereka, mereka tidak mau mendengarkan dan tidak mau insaf.
\par 34 Malah rumah yang dibangun untuk tempat ibadat kepada-Ku telah dinajiskan oleh mereka dengan meletakkan berhala-berhala mereka yang memuakkan di tempat itu.
\par 35 Mereka membangun mezbah-mezbah untuk Baal di Lembah Hinom, dan mempersembahkan anak-anak mereka kepada Dewa Molokh. Padahal, Aku tak pernah menyuruh mereka melakukan hal itu, bahkan tak pernah timbul dalam pikiran-Ku bahwa mereka akan melakukan perbuatan sekeji itu dan membuat orang Yehuda berdosa."
\par 36 TUHAN, Allah Israel, berkata kepadaku, "Yeremia, bangsa Israel berkata bahwa peperangan, kelaparan, dan wabah penyakit akan membuat kota ini jatuh ke tangan raja Babel. Sekarang, dengarkan juga apa yang akan Kukatakan.
\par 37 Aku akan mengumpulkan bangsa ini dari semua negeri tempat mereka Kuceraiberaikan karena kemarahan-Ku dan geram-Ku kepada mereka. Mereka akan Kubawa kembali ke tempat ini dan Kumungkinkan tinggal di sini dengan aman.
\par 38 Mereka akan menjadi umat-Ku, dan Aku menjadi Allah mereka.
\par 39 Aku akan memberi mereka hanya satu tujuan hidup: yaitu, menghormati Aku selama-lamanya; hal itu akan membawa kebaikan bagi mereka sendiri, dan bagi keturunan mereka.
\par 40 Aku akan membuat perjanjian yang kekal dengan mereka. Aku tak akan berhenti berbuat baik kepada mereka. Aku akan membuat mereka takut dan hormat kepada-Ku dengan sepenuh hati supaya mereka tidak menjauhi Aku.
\par 41 Aku akan senang berbuat baik kepada mereka, dan membuat mereka menetap di negeri ini untuk selama-lamanya.
\par 42 Sebagaimana Aku telah mendatangkan bencana ke atas bangsa ini, begitu pula Aku akan memberikan semua yang baik yang telah Kujanjikan kepada mereka.
\par 43 Orang berkata bahwa negeri ini telah menjadi seperti padang gurun yang tidak didiami oleh manusia dan binatang, dan bahwa negeri ini telah jatuh ke tangan orang Babel. Tapi Aku berkata bahwa kelak ladang-ladang akan diperjualbelikan lagi di negeri ini;
\par 44 surat pembeliannya akan ditandatangani, disegel, dan diperkuat oleh saksi-saksi. Itu akan terjadi di wilayah Benyamin, di desa-desa sekitar Yerusalem, di kota-kota Yehuda, di kota-kota di daerah pegunungan, di daerah kaki pegunungan, dan di Yehuda selatan. Bangsa ini akan Kupulihkan keadaannya di negeri ini. Aku, TUHAN, telah berbicara."

\chapter{33}

\par 1 Ketika aku masih ditahan di pelataran, aku mendapat pesan lain dari TUHAN
\par 2 yang menjadikan bumi dan membentuk serta meletakkannya di tempatnya. Ia berkata,
\par 3 "Berserulah kepada-Ku, maka Aku akan menyahut; akan Kuberitahukan kepadamu hal-hal yang indah dan mengagumkan yang belum kauketahui.
\par 4 Istana raja Yehuda dan rumah-rumah di kota Yerusalem telah dibongkar untuk memperkuat pertahanan terhadap kepungan dan serangan musuh. Tapi Aku, TUHAN, Allah Israel, berkata bahwa,
\par 5 percuma saja melawan orang Babel, karena hal itu hanya akan membuat kota ini penuh dengan mayat orang-orang yang Kutewaskan dalam kemarahan dan geram-Ku kepada mereka. Aku tak sudi lagi melihat kota ini, karena orang-orangnya telah melakukan yang jahat.
\par 6 Tapi kelak Aku akan memulihkan keadaan kota ini dan penduduknya; mereka akan kuat lagi. Aku akan memberikan kepada mereka kesejahteraan dan keamanan yang berlimpah.
\par 7 Yehuda dan Israel akan Kujadikan jaya seperti dahulu dan mereka akan berdiri lagi sebagai umat-Ku.
\par 8 Aku akan membersihkan mereka dari dosa-dosa yang telah mereka lakukan terhadap-Ku; dosa-dosa dan kedurhakaan mereka akan Kuampuni.
\par 9 Yerusalem akan menjadi kegembiraan dan kebanggaan-Ku. Segala bangsa di dunia akan menjadi takut dan gentar serta menghormati Aku, apabila mendengar tentang segala kebaikan yang Kulakukan untuk penduduk Yerusalem, dan tentang kemakmuran yang Kuberikan kepada kota itu."
\par 10 TUHAN berkata, "Orang mengatakan bahwa tempat ini seperti padang gurun, tidak ada orang atau binatang yang tinggal di situ. Perkataan mereka memang benar, sebab kota-kota Yehuda dan jalan-jalan di Yerusalem sudah kosong semua; tidak ada orang atau binatang yang tinggal di situ.
\par 11 Tapi di tempat-tempat itu akan terdengar lagi suara kesukaan dan kegembiraan serta suara pesta kawin. Akan terdengar juga nyanyian orang-orang yang membawa kurban syukur ke rumah-Ku. Mereka akan berkata, 'Bersyukurlah kepada TUHAN Yang Mahakuasa, karena Ia baik dan kasih-Nya kekal abadi.' Negeri ini akan Kujadikan makmur seperti semula. Aku, TUHAN, telah berbicara."
\par 12 TUHAN Yang Mahakuasa berkata, "Negeri dan kota-kotanya yang seperti padang gurun ini tidak didiami oleh manusia atau binatang. Namun demikian, di kemudian hari akan ada lagi padang-padang rumput tempat orang menggembalakan domba-dombanya.
\par 13 Di kota-kota di daerah pegunungan, kaki pegunungan, daerah Yehuda selatan, dan di wilayah Benyamin serta di desa-desa sekitar Yerusalem, dan di kota-kota Yehuda, para gembala akan menghitung lagi domba-domba mereka. Aku, TUHAN, telah berbicara."
\par 14 TUHAN berkata, "Akan tiba saatnya Aku menepati janji yang telah Kubuat dengan Israel dan Yehuda.
\par 15 Pada waktu itu seorang yang adil dari keturunan Daud akan Kupilih menjadi raja. Raja itu akan melakukan apa yang adil dan benar di seluruh negeri.
\par 16 Orang Yehuda dan penduduk Yerusalem akan diselamatkan dan hidup aman. Kota Yerusalem akan dinamakan 'TUHAN Penyelamat Kita'.
\par 17 Aku, TUHAN, berjanji bahwa selalu akan ada seorang dari keturunan Daud yang menjadi raja Israel;
\par 18 selalu akan ada imam-imam dari suku Lewi yang melayani Aku dan mempersembahkan kurban binatang, gandum dan persembahan yang dibakar."
\par 19 TUHAN berkata kepadaku,
\par 20 "Aku sudah mengikat perjanjian dengan siang dan malam, supaya siang dan malam selalu datang pada waktunya; perjanjian itu tak bisa ditiadakan.
\par 21 Begitu juga Aku telah mengikat perjanjian dengan hamba-Ku Daud, bahwa dari keturunannya selalu akan ada seorang yang menjadi raja; dan Aku telah mengikat perjanjian dengan imam-imam keturunan Lewi bahwa mereka selalu akan melayani Aku. Perjanjian-perjanjian itu tidak dapat dihapuskan.
\par 22 Aku akan menambah keturunan hamba-Ku Daud dan imam-imam suku Lewi, sehingga jumlahnya menjadi sebanyak pasir di pantai dan bintang di langit yang tak dapat dihitung."
\par 23 TUHAN berkata kepadaku,
\par 24 "Sudahkah kauperhatikan apa yang dikatakan orang terhadap Israel dan Yehuda? Mereka berkata bahwa Aku telah membuang kedua bangsa yang telah Kupilih itu. Itu sebabnya mereka memandang rendah umat-Ku, dan tidak lagi menganggap mereka sebagai suatu bangsa.
\par 25 Tapi sebagaimana Aku, TUHAN, telah membuat suatu ikatan perjanjian dengan siang dan malam, dan membuat hukum yang mengatur langit dan bumi,
\par 26 begitu juga Aku akan tetap memegang perjanjian-Ku dengan keturunan Yakub dan dengan hamba-Ku Daud. Seorang dari keturunan Daud akan Kupilih untuk memerintah keturunan Abraham, Ishak, dan Yakub. Aku akan berbelaskasihan kepada umat-Ku, dan membuat mereka makmur lagi."

\chapter{34}

\par 1 Ketika Nebukadnezar raja Babel dengan angkatan perangnya yang diperkuat oleh pasukan-pasukan dari segala bangsa dan negara yang takluk kepadanya datang menyerang Yerusalem dan kota-kota di sekitarnya, TUHAN berbicara kepadaku.
\par 2 Ia menyuruh aku pergi kepada Zedekia raja Yehuda, dan berkata begini, "Aku, TUHAN Allah Israel, akan menyerahkan kota ini kepada raja Babel dan ia akan membakarnya.
\par 3 Engkau tak akan lolos, melainkan ditangkap dan diserahkan kepadanya. Engkau akan bertemu dan berbicara sendiri dengan dia, lalu engkau akan pergi ke Babel.
\par 4 Tetapi dengarkan perkataan-Ku ini mengenai engkau, Zedekia! Engkau tidak akan tewas dalam pertempuran,
\par 5 melainkan mati dengan tentram. Dan sebagaimana rakyat membakar kemenyan ketika memakamkan leluhurmu yang memerintah sebelum engkau, begitu pula rakyat akan membakar kemenyan untukmu. Mereka akan berkabung dan berkata, 'Aduh! Raja kami sudah tiada!' Aku, TUHAN, telah berbicara."
\par 6 Maka pesan TUHAN itu kusampaikan kepada Raja Zedekia di Yerusalem,
\par 7 ketika pasukan raja Babel sedang menyerang kota itu. Mereka juga menyerang kota Lakhis dan Aseka, dua kota berbenteng yang masih bertahan di Yehuda.
\par 8 Raja Zedekia dan penduduk Yerusalem mengadakan perjanjian untuk membebaskan
\par 9 budak-budak Ibrani mereka, baik laki-laki maupun perempuan supaya tak ada seorang pun dari bangsa Israel yang memperbudak orang sebangsanya.
\par 10 Seluruh rakyat dan para pejabat pemerintah yang telah mengadakan perjanjian untuk membebaskan budak-budak mereka dan tak akan memperbudaknya lagi, betul-betul melepaskan budak-budak mereka itu.
\par 11 Tapi kemudian mereka berubah pikiran, lalu mengambil kembali budak-budak itu, dan memaksanya menjadi budak lagi.
\par 12 Maka TUHAN, Allah Israel, menyuruh aku berkata begini kepada orang-orang itu, "Ketika Aku membawa leluhurmu keluar dari Mesir dan membebaskan mereka dari perbudakan, Aku telah membuat perjanjian dengan mereka.
\par 14 Aku berkata bahwa semua budak Ibrani yang telah melayani enam tahun lamanya, harus dibebaskan pada akhir tahun ketujuh. Tapi leluhurmu tidak mau memperhatikan atau mendengarkan apa yang Kukatakan itu.
\par 15 Beberapa hari yang lalu kamu mengubah kelakuanmu dan melakukan yang menyenangkan hati-Ku. Kamu semua menyetujui untuk membebaskan budak-budak sebangsamu, dan membuat perjanjian dengan Aku di dalam Rumah-Ku.
\par 16 Pejabat-pejabat pemerintah Yehuda dan Yerusalem bersama para pegawai istana, imam-imam dan semua pengawas tanahmu telah mensahkan perjanjian itu dengan menyembelih sapi, memotongnya menjadi dua, dan berjalan di antara kedua potongan itu. Tapi, kemudian kamu berubah pikiran dan menghina Aku. Budak-budak yang sudah kamu bebaskan sesuai dengan keinginan mereka, kamu ambil kembali dan paksakan untuk menjadi budak lagi. Karena itu, Aku, TUHAN, berkata bahwa kamu telah melanggar perintah-Ku, dan tidak mentaati syarat-syaratnya, sebab kamu tidak membebaskan sesamamu bangsa Israel. Nah, sekarang baiklah Aku memberikan kebebasan kepadamu: kebebasan untuk tewas dalam pertempuran atau oleh wabah penyakit atau oleh kelaparan. Aku akan memperlakukan kamu seperti kamu memperlakukan sapi yang kamu sembelih itu. Aku akan membuat segala bangsa di dunia merasa ngeri melihat apa yang Kulakukan terhadap kamu.
\par 20 Aku akan menyerahkan kamu kepada musuh yang ingin membunuh kamu. Mayatmu akan dimakan burung dan binatang buas.
\par 21 Zedekia raja Yehuda dan para pejabat pemerintahnya akan Kuserahkan kepada orang-orang yang ingin membunuh mereka. Aku akan menyerahkan mereka kepada tentara Babel, yang kini telah meninggalkan kamu.
\par 22 Aku akan memberi aba-aba supaya mereka kembali ke kota ini, lalu menyerang, merebut, dan membakarnya. Kota-kota Yehuda akan Kujadikan seperti padang gurun yang tidak didiami orang. Aku, TUHAN, telah berbicara."

\chapter{35}

\par 1 Ketika Yoyakim anak Yosia menjadi raja Yehuda, TUHAN berkata kepadaku,
\par 2 "Pergilah berbicara dengan orang-orang kaum Rekhab. Bawalah mereka ke salah satu kamar di dalam Rumah-Ku, dan sajikanlah anggur kepada mereka."
\par 3 Maka pergilah aku menjemput seluruh kaum Rekhab, yaitu Yaazanya (anak seorang yang bernama Yeremia anak Habazinya) bersama semua saudaranya dan anak-anaknya yang laki-laki.
\par 4 Aku membawa mereka ke Rumah TUHAN ke dalam ruang murid-murid Nabi Hanan anak Yigdalya. Ruangan itu terletak di atas ruangan Imam Maaseya anak Salum, dekat ruangan pejabat-pejabat lain. Maaseya adalah pegawai tinggi yang mengawasi pintu-pintu di Rumah TUHAN.
\par 5 Kemudian aku meletakkan di depan orang-orang Rekhab itu piala-piala penuh dengan anggur, dan gelas-gelas. Lalu aku berkata kepada mereka, "Silakan minum."
\par 6 Tetapi mereka menjawab, "Kami tidak minum anggur. Leluhur kami Yonadab anak Rekhab melarang kami dan keturunan kami untuk minum anggur. Larangan itu berlaku untuk selama-lamanya.
\par 7 Kami tidak boleh bercocok tanam, dan juga tidak boleh membuka atau memiliki kebun anggur. Seumur hidup, kami juga tidak boleh mendirikan rumah. Kami harus tinggal di dalam kemah, supaya dapat menetap di tanah ini sebagai orang asing.
\par 8 Semua yang diperintahkan Yonadab kepada kami telah kami taati. Seumur hidup, kami dan anak istri kami tidak minum anggur.
\par 9 Kami tidak mendirikan rumah untuk tempat kediaman kami, tetapi kami tinggal dalam rumah kemah. Kami juga tidak mempunyai kebun anggur, ladang atau pun benih gandum untuk ditanam. Singkatnya, segala yang diperintahkan oleh leluhur kami Yonadab itu, kami taati sepenuhnya.
\par 11 Tapi ketika Raja Nebukadnezar menyerbu negeri ini, kami memutuskan untuk lari dari tentara Babel dan tentara Siria, dan mengungsi ke Yerusalem. Itu sebabnya kami sekarang tinggal di sini."
\par 12 Lalu TUHAN Yang Mahakuasa, Allah Israel, menyuruh aku pergi kepada orang Yehuda dan Yerusalem, dan menyampaikan pesan TUHAN ini kepada mereka, "Aku, TUHAN, bertanya mengapa kamu tidak mau mendengarkan Aku, dan tak mau menuruti perintah-perintah-Ku.
\par 14 Yonadab melarang keturunannya minum anggur, dan mereka taat kepada larangan itu. Sampai hari ini tak seorang pun dari mereka minum anggur. Tapi Aku terus-menerus berbicara kepada kamu, dan kamu tidak mau mendengarkan.
\par 15 Terus-menerus Aku mengutus kepadamu semua hamba-Ku para nabi, dan mereka telah menasihatkan kamu untuk memperbaiki kelakuanmu dan berhenti berbuat jahat. Mereka memperingatkan kamu untuk tidak menyembah ilah-ilah lain dan tidak mengabdi kepada mereka, supaya kamu dapat terus tinggal di negeri ini yang telah Kuberikan kepadamu dan kepada leluhurmu. Tapi kamu tidak mau memperhatikan dan tidak mau mendengarkan Aku.
\par 16 Keturunan Yonadab taat kepada perintah bapak leluhur mereka, tapi kamu tidak mau taat kepada-Ku.
\par 17 Karena itu, Aku, TUHAN Yang Mahakuasa, Allah Israel, akan mendatangkan ke atas kamu--orang Yehuda dan Yerusalem--segala bencana yang telah Kutentukan untuk kamu, seperti yang sudah Kukatakan. Hal itu Kulakukan karena kamu tidak mau mendengarkan apabila Aku berbicara kepadamu, dan kamu tidak mau menjawab ketika Aku memanggil kamu."
\par 18 Lalu aku berkata kepada orang-orang kaum Rekhab bahwa TUHAN Yang Mahakuasa, Allah Israel, berkata begini, "Kamu telah taat kepada leluhurmu Yonadab; semua perintahnya telah kamu turuti dan jalankan.
\par 19 Karena itu, Aku, TUHAN Yang Mahakuasa, Allah Israel, berjanji bahwa dari keturunan Yonadab anak Rekhab selalu akan ada orang laki-laki yang melayani Aku."

\chapter{36}

\par 1 Pada tahun keempat setelah Yoyakim anak Yosia menjadi raja Yehuda, TUHAN berkata kepadaku,
\par 2 "Ambillah sebuah buku gulungan, dan tulislah di situ semua yang telah Kukatakan kepadamu tentang Israel dan Yehuda, dan tentang segala bangsa. Tulislah semua perkataan-Ku sejak Aku pertama kali berbicara kepadamu pada masa pemerintahan Raja Yosia sampai hari ini.
\par 3 Barangkali orang Yehuda akan berhenti berbuat jahat apabila mereka mendengar tentang semua bencana yang akan Kutimpakan ke atas mereka. Maka Aku akan mengampuni dosa dan kejahatan mereka."
\par 4 Karena itu aku memanggil Barukh anak Neria dan mendiktekan kepadanya semua yang dikatakan TUHAN kepadaku. Maka Barukh menulis semuanya pada buku gulungan itu.
\par 5 Kemudian aku memberi perintah ini kepadanya, "Aku tidak diizinkan lagi memasuki Rumah TUHAN.
\par 6 Karena itu, apabila tiba harinya orang berpuasa, engkau harus pergi ke sana, dan membacakan dengan suara keras semua yang telah kudiktekan kepadamu, supaya mereka mendengar semua yang dikatakan TUHAN kepadaku. Lakukanlah itu di tempat yang memungkinkan engkau dapat didengar oleh setiap orang yang berada di Rumah TUHAN, termasuk orang Yehuda yang datang dari kota-kota mereka.
\par 7 Barangkali karena itu mereka akan berdoa kepada TUHAN, dan bertobat dari perbuatan-perbuatan mereka yang jahat. TUHAN marah sekali kepada mereka dan mengancam mereka."
\par 8 Maka Barukh membacakan perkataan TUHAN di Rumah TUHAN, seperti yang kutugaskan kepadanya.
\par 9 Pada bulan sembilan tahun kelima pemerintahan Yoyakim raja Yehuda, rakyat berpuasa untuk memperoleh belas kasihan dari TUHAN. Penduduk Yerusalem dan orang-orang yang datang dari kota-kota Yehuda, semuanya berpuasa.
\par 10 Lalu Barukh membacakan dari buku gulungan itu segala sesuatu yang telah kukatakan, dan semua orang mendengarkan. Pembacaan itu dilakukan di dalam Rumah TUHAN di kantor Gemarya, anak Safan sekretaris negara. Kantor itu terletak di pelataran sebelah atas, dekat jalan masuk ke Pintu Gerbang Baru di Rumah TUHAN.
\par 11 Mikhaya anak Gemarya dan cucu Safan mendengar Barukh membacakan perkataan TUHAN dari buku gulungan itu.
\par 12 Lalu ia pergi ke istana, ke kantor sekretaris negara. Di situ semua pejabat sedang berapat. Delaya anak Semaya, Elnatan anak Akhbor, Gemarya anak Safan, Zedekia anak Hananya, Elisama seorang sekretaris, dan pejabat-pejabat lainnya ada di situ.
\par 13 Mikhaya memberitahukan kepada mereka, semua yang dibacakan Barukh di Rumah TUHAN.
\par 14 Lalu para pejabat itu mengutus Yehudi (yaitu anak Netanya, cucu Selemya, dan buyut Kusyi) kepada Barukh dengan perintah untuk datang dan membawa kepada mereka buku gulungan yang telah dibacakannya itu. Maka Barukh membawa buku itu.
\par 15 "Silakan duduk," kata mereka, "bacakanlah buku gulungan itu kepada kami." Barukh pun menurut.
\par 16 Setelah ia selesai membaca, mereka saling memandang dengan perasaan cemas lalu berkata kepada Barukh, "Kami harus melaporkan hal ini kepada raja."
\par 17 Kemudian mereka bertanya, "Coba beritahukan kepada kami bagaimana engkau menulis semuanya ini. Apakah Yeremia yang mendiktekannya kepadamu?"
\par 18 "Betul, setiap kata yang tertulis di sini telah didiktekan oleh Yeremia kepadaku, dan aku menulisnya dengan tinta pada buku gulungan ini," jawab Barukh.
\par 19 "Engkau dan Yeremia harus bersembunyi. Jangan sampai orang tahu tempat persembunyianmu itu," kata mereka.
\par 20 Para pejabat itu menaruh buku gulungan itu di kamar Elisama, sekretaris itu, kemudian pergi kepada raja dan melaporkan semuanya.
\par 21 Lalu raja menyuruh Yehudi pergi untuk mengambil buku gulungan itu. Setelah mengambilnya dari kamar Elisama, Yehudi membacakannya kepada raja dan kepada semua pejabat yang sedang berdiri di sekelilingnya.
\par 22 Waktu itu bulan sembilan, dan raja sedang duduk di depan perapian di istana musim dinginnya.
\par 23 Begitu Yehudi selesai membaca tiga atau empat lajur, raja memotong bagian itu dengan sebilah pisau kecil dan melemparkannya ke dalam api. Demikianlah dilakukannya terus sampai seluruh buku gulungan itu terbakar habis.
\par 24 Sekalipun Elnatan, Delaya, dan Gemarya memohon dengan sangat supaya raja jangan membakar buku gulungan itu, namun ia tidak mau mendengarkan mereka. Baik raja maupun para pejabat yang mendengar isi buku itu tidak menjadi takut atau menunjukkan penyesalan.
\par 26 Raja malah memerintahkan putranya, yaitu Yerahmeel, bersama dengan Seraya anak Azriel dan Selemya anak Abdeel, supaya menangkap aku dan Barukh sekretarisku. Tetapi TUHAN menyembunyikan kami.
\par 27 Setelah Raja Yoyakim membakar buku gulungan yang kudiktekan kepada Barukh itu, TUHAN menyuruh aku
\par 28 mengambil buku gulungan yang lain, dan menulis semua yang telah tertulis pada buku gulungan yang pertama itu.
\par 29 TUHAN menyuruh aku mengatakan begini kepada raja, "Engkau sudah membakar buku gulungan itu, dan engkau bertanya kepada Yeremia mengapa ia menulis bahwa raja Babel akan datang dan menghancurkan negeri ini serta membunuh penduduknya bersama binatang-binatangnya.
\par 30 Sebab itu, Aku, TUHAN, berkata kepadamu, hai Raja Yoyakim, bahwa tidak seorang pun dari keturunanmu akan memerintah sebagai raja keturunan Daud. Mayatmu akan dilempar ke luar, tertimpa panas di waktu siang, dan embun dingin di waktu malam.
\par 31 Engkau, keturunanmu, dan para pejabatmu akan Kuhukum karena dosa-dosa yang kamu lakukan. Aku akan mendatangkan ke atas kamu semua bencana yang telah Kuancamkan kepadamu, karena baik engkau maupun penduduk Yerusalem dan Yehuda tidak memperhatikan ancaman-ancaman-Ku."
\par 32 Maka aku mengambil buku gulungan yang baru, dan memberikannya kepada Barukh sekretarisku. Lalu ia menuliskan semua yang kudiktekan kepadanya, yaitu isi buku gulungan yang pertama ditambah beberapa pesan lain semacam itu.

\chapter{37}

\par 1 Pada waktu Nebukadnezar raja Babel memecat Yoyakhin anak Yoyakim, ia mengangkat Zedekia anak Yosia menjadi raja Yehuda.
\par 2 Tetapi baik Zedekia maupun para pejabatnya dan rakyatnya tidak menuruti pesan TUHAN yang diberikan-Nya melalui aku.
\par 3 Raja Zedekia mengutus Yukhal anak Selemya, dan Imam Zefanya anak Maaseya untuk minta supaya aku berdoa kepada TUHAN Allah untuk kepentingan bangsa.
\par 4 Pada waktu itu aku belum dimasukkan ke dalam penjara dan masih bebas bergerak di antara rakyat.
\par 5 Tentara Babel sedang mengepung Yerusalem, tapi ketika mereka mendengar bahwa tentara Mesir telah berangkat dari Mesir untuk berperang, mereka mundur.
\par 6 TUHAN, Allah Israel, menyuruh aku
\par 7 berkata begini kepada utusan Raja Zedekia: "Katakanlah kepada Zedekia, 'Tentara Mesir yang sedang menuju kemari untuk membantumu, akan balik dan pulang.
\par 8 Lalu orang Babel akan kembali dan menyerang kota ini, serta mengalahkan dan membakarnya.
\par 9 Aku, TUHAN, memperingatkan kamu, supaya jangan menipu dirimu dengan berpikir bahwa orang Babel telah pergi dan tidak akan kembali. Sebab mereka pasti akan kembali.
\par 10 Seandainya kamu mengalahkan seluruh tentara Babel sehingga yang tinggal hanyalah orang-orang yang luka parah di dalam kemah-kemah mereka, namun mereka akan bangun dan membakar habis kota ini.'"
\par 11 Tentara Babel mundur dari Yerusalem karena tentara Mesir bergerak menuju kota itu.
\par 12 Maka aku meninggalkan Yerusalem dan pergi ke wilayah Benyamin untuk menerima pembagian warisanku dari tanah milik keluarga.
\par 13 Tetapi ketika aku sampai di Pintu Gerbang Benyamin, komandan tentara yang bertugas di situ menahan aku dan berkata, "Engkau mau lari ke pihak orang Babel!" (Perwira itu bernama Yeria; ia anak Selemya dan cucu Hananya.)
\par 14 Aku membalas, "Tidak benar! Aku tidak bermaksud lari ke pihak musuh." Tapi Yeria tidak mau mendengarkan. Ia menangkap dan membawa aku menghadap para pejabat pemerintah.
\par 15 Mereka marah sekali kepadaku, dan memerintahkan supaya aku dipukuli dan dikurung di rumah Yonatan, sekretaris negara. Rumah itu sudah dijadikan penjara.
\par 16 Aku dimasukkan ke dalam sebuah kolam di bawah tanah, dan lama ditahan di situ.
\par 17 Kemudian Raja Zedekia menyuruh orang membawa aku menghadap dia di istana. Di situ secara rahasia ia bertanya kepadaku apakah ada pesan dari TUHAN. "Ada," jawabku. Lalu kutambahkan, "Baginda akan diserahkan kepada raja Babel."
\par 18 Setelah itu aku bertanya, "Apa kejahatanku terhadap Baginda atau terhadap para pejabat Baginda atau rakyat, sehingga Baginda memasukkan aku ke dalam penjara?
\par 19 Di mana para nabi Baginda itu yang mengatakan bahwa raja Babel tidak akan menyerang Baginda dan negeri ini?
\par 20 Sekarang, Baginda yang mulia, sudilah mendengarkan permohonanku ini. Janganlah Baginda mengirim aku kembali ke penjara di rumah Yonatan itu, nanti aku mati di sana."
\par 21 Maka Raja Zedekia memerintahkan supaya aku dipenjarakan di pelataran istana. Jadi aku tinggal di situ, dan setiap hari diberi roti dari tempat pembuatan roti sampai persediaan roti di kota itu telah habis semuanya.

\chapter{38}

\par 1 TUHAN menyuruh aku menyampaikan pesan ini kepada rakyat, "Kota ini akan Kuberikan kepada tentara Babel, dan mereka akan mengalahkannya. Barangsiapa tinggal di dalam kota akan tewas dalam peperangan, atau karena kelaparan atau wabah penyakit. Tapi barangsiapa keluar dan menyerahkan diri kepada orang Babel, tidak akan dibunuh; nyawanya akan selamat." Sefaca anak Matan, Gedalya anak Pasyhur, Yukhal anak Selemya, dan Pasyhur anak Malkia mendengar semua yang Kukatakan itu.
\par 4 Maka pergilah para pejabat itu kepada raja dan berkata, "Orang ini harus dihukum mati. Sebab, dengan berkata begitu ia melemahkan semangat semua prajurit dan orang lain yang masih tinggal di kota ini. Ia tidak menolong rakyat, melainkan mencelakakan mereka."
\par 5 Raja Zedekia menjawab, "Baik! Perlakukanlah dia sesuai dengan kehendakmu, aku tidak dapat mencegah kamu."
\par 6 Lalu mereka mengambil aku dan menurunkan aku dengan tali ke dalam sumur milik Malkia putra raja, yang terletak di pelataran istana. Tidak ada air di dalam sumur itu, hanya lumpur, dan aku masuk ke dalam lumpur itu.
\par 7 Pada waktu itu ada seorang Sudan yang bekerja di istana raja. Namanya Ebed-Melekh. Ia mendengar bahwa aku telah dimasukkan ke dalam sumur itu. Ketika raja sedang memimpin rapat di Pintu Gerbang Benyamin,
\par 8 Ebed-Melekh pergi ke sana dan berkata kepada raja,
\par 9 "Paduka Yang Mulia, perbuatan orang-orang itu tidak baik. Mereka telah memasukkan Nabi Yeremia ke dalam sumur; ia pasti akan mati kelaparan di situ, sebab makanan sudah habis di kota ini."
\par 10 Lalu raja memerintahkan Ebed-Melekh, "Bawalah tiga orang dan keluarkanlah Yeremia dari sumur itu sebelum ia mati."
\par 11 Maka pergilah Ebed-Melekh dengan ketiga orang itu ke gudang istana, dan mengambil kain-kain tua dari situ, lalu menurunkannya dengan tali kepadaku di dalam sumur.
\par 12 Ebed-Melekh menyuruh aku menaruh kain-kain itu di bawah ketiakku sebagai ganjalan, supaya tali itu tidak menyakiti aku. Maka aku menuruti perintah Ebed-Melekh.
\par 13 Lalu mereka menarik aku ke atas, keluar dari sumur itu. Setelah itu aku ditahan di pelataran istana itu.
\par 14 Pada suatu waktu yang lain, Raja Zedekia menyuruh orang membawa aku menghadap dia di pintu gerbang yang ketiga pada Rumah TUHAN. Ia berkata kepadaku, "Aku mau bertanya kepadamu, dan kau harus menjawab dengan terus terang. Jangan sembunyikan apa-apa."
\par 15 Aku menjawab, "Kalau aku mengatakan yang benar, pasti Baginda akan menjatuhkan hukuman mati kepadaku; dan kalau aku memberi nasihat, Baginda tidak akan mau menuruti nasihat itu."
\par 16 Raja Zedekia dengan diam-diam berjanji kepadaku, katanya, "Demi Allah yang hidup, Allah yang memberi hidup kepada kita, aku bersumpah bahwa aku tidak akan membunuh engkau atau menyerahkan engkau kepada orang-orang yang mau membunuhmu."
\par 17 Lalu aku mengatakan kepada Zedekia bahwa TUHAN Yang Mahakuasa, Allah yang disembah oleh orang Israel, berkata begini, "Jika engkau menyerah kepada pejabat-pejabat raja Babel, engkau tidak akan dibunuh, dan kota ini pun tidak akan dibakar. Engkau dan keluargamu akan selamat.
\par 18 Tapi kalau engkau tidak mau menyerah, kota ini akan jatuh ke dalam tangan orang Babel. Mereka akan membakarnya sampai habis, dan engkau tidak akan luput."
\par 19 Raja menjawab, "Aku takut kepada orang-orang bangsa kita yang sudah lari ke pihak orang Babel. Jangan-jangan aku akan diserahkan kepada mereka dan disiksa."
\par 20 "Tidak," jawabku, "Baginda tidak akan diserahkan kepada mereka. Aku mohon dengan sangat supaya Baginda menuruti pesan TUHAN; nanti semuanya akan beres, dan Baginda tidak akan dibunuh.
\par 21 Tapi apabila Baginda tidak mau menyerahkan diri, TUHAN sudah memperlihatkan kepadaku apa yang akan terjadi.
\par 22 Dalam penglihatan itu aku melihat semua wanita yang masih ada di istana Yehuda digiring keluar kepada pejabat-pejabat raja Babel. Sambil berjalan, mereka berkata, 'Raja telah diperdaya oleh sahabat-sahabat karibnya, dan dikuasai oleh mereka. Kini, setelah kakinya terperosok ke dalam lumpur, mereka semua undur.'"
\par 23 Aku berkata lagi, "Semua anak istri Baginda akan dibawa keluar kepada orang Babel. Baginda sendiri pun tidak akan luput dari mereka. Baginda akan ditangkap oleh raja Babel, dan kota ini dibakar habis."
\par 24 Lalu Zedekia berkata, "Jangan beritahukan kepada siapa pun tentang percakapan kita ini, supaya nyawamu tidak terancam.
\par 25 Apabila para pejabat mendengar bahwa aku telah berbicara dengan engkau, mereka akan datang dan bertanya kepadamu tentang pembicaraan kita. Mereka akan berjanji untuk tidak membunuh engkau, kalau engkau menceritakan semuanya kepada mereka.
\par 26 Katakan saja bahwa engkau membujuk aku supaya aku tidak mengirim engkau kembali ke penjara dan mati di sana."
\par 27 Tak lama kemudian semua pejabat itu datang dan bertanya-tanya kepadaku. Aku memberitahukan kepada mereka tepat seperti yang diperintahkan raja kepadaku. Dengan demikian mereka tak dapat berbuat apa-apa sebab tak ada seorang pun yang mengetahui pembicaraan raja dengan aku.
\par 28 Aku tetap ditahan di pelataran istana sampai pada hari Yerusalem direbut musuh.

\chapter{39}

\par 1 Pada bulan sepuluh tahun kesembilan pemerintahan Zedekia raja Yehuda, Nebukadnezar raja Babel dengan seluruh angkatan perangnya datang dan menyerang Yerusalem.
\par 2 Pada tanggal sembilan bulan empat tahun kesebelas pemerintahan Raja Zedekia, tembok kota didobrak musuh,
\par 3 dan Yerusalem direbut. Maka semua pejabat-pejabat raja Babel datang dan pergi duduk di Pintu Gerbang Tengah. Di antara mereka terdapat Nergal Sarezer, Samgar Nebo, Sarsekim kepala rumah tangga istana, Nergal Sarezer Rabmag dan perwira-perwira lain.
\par 4 Ketika Raja Zedekia dan seluruh tentaranya melihat semua kejadian itu, mereka berusaha melarikan diri menuju Lembah Yordan malam itu juga. Mereka mengambil jalan lewat taman istana, lalu keluar melalui pintu gerbang yang menghubungi kedua tembok di tempat itu.
\par 5 Tapi tentara Babel mengejar mereka dan menangkap Zedekia di dataran Yerikho. Kemudian ia dibawa kepada Raja Nebukadnezar di kota Ribla di daerah Hamat, lalu dijatuhi hukuman mati.
\par 6 Di kota itu juga anak-anaknya dibunuh di depan matanya, dan semua pejabat pemerintah Yehuda pun dibunuh.
\par 7 Setelah itu Zedekia dicungkil matanya, lalu dibelenggu dan dibawa ke Babel.
\par 8 Sementara itu orang-orang Babel membakar istana raja dan rumah-rumah rakyat, serta meruntuhkan tembok-tembok Yerusalem.
\par 9 Akhirnya rakyat yang masih ada di kota, bersama orang-orang yang telah lari ke pihak orang Babel, diangkut ke Babel oleh Nebuzaradan, komandan pasukan Babel.
\par 10 Sebagian dari rakyat yang paling miskin dan tak punya harta, ditinggalkannya di Yehuda serta diberi kebun anggur dan ladang.
\par 11 Raja Nebukadnezar telah menyuruh Nebuzaradan, komandan pasukannya, mengeluarkan perintah berikut ini.
\par 12 "Pergilah mencari Yeremia, dan jagalah dia baik-baik. Jangan apa-apakan dia; lakukanlah apa yang dimintanya."
\par 13 Maka Nebuzaradan bersama-sama dengan kedua perwira tingginya, Nebusyazban dan Nergal-Sarezer serta semua perwira Babel lainnya,
\par 14 mengambil aku dari pelataran istana. Lalu aku diserahkan kepada Gedalya anak Ahikam dan cucu Safan, yang harus mengantarkan aku pulang dengan selamat. Maka tinggallah aku di antara rakyat.
\par 15 Ketika aku masih dipenjarakan di pelataran istana, TUHAN menyuruh aku
\par 16 mengatakan kepada Ebed-Melekh orang Sudan itu bahwa TUHAN Yang Mahakuasa, Allah Israel, berkata begini, "Seperti yang Kukatakan dahulu, Aku akan mendatangkan ke atas kota ini bencana, dan bukan kemakmuran. Engkau akan melihat hal itu terjadi.
\par 17 Tetapi Aku, TUHAN, akan melindungimu. Engkau tidak akan diserahkan kepada orang-orang yang kautakuti.
\par 18 Engkau tidak akan dibunuh, sebab Aku akan menyelamatkan engkau. Nyawamu akan selamat karena engkau bersandar kepada-Ku. Aku, TUHAN, yang berbicara."

\chapter{40}

\par 1 TUHAN berbicara kepadaku setelah Nebuzaradan, komandan pasukan itu, membebaskan aku di Rama. Aku telah dibawa ke sana dalam keadaan terbelenggu bersama semua orang lain dari Yerusalem dan Yehuda yang sedang diangkut sebagai tawanan ke Babel.
\par 2 Komandan pasukan itu memanggil aku tersendiri dan berkata, "TUHAN Allahmu telah mengancam untuk menghancurkan negeri ini.
\par 3 Sekarang hal itu telah dilaksanakan-Nya, karena bangsamu telah berdosa dan tidak taat kepada-Nya.
\par 4 Sekarang kulepaskan belenggu ini dari tanganmu dan kubebaskan engkau. Kalau engkau mau, engkau boleh ikut ke Babel, dan aku akan memelihara engkau. Tapi kalau engkau tidak mau, tidak mengapa. Di seluruh negeri ini engkau bebas pergi ke mana saja kaukehendaki."
\par 5 Sebelum pergi, Nebuzaradan berkata, "Kalau mau, engkau boleh kembali kepada Gedalya anak Ahikam dan cucu Safan, yang telah diangkat oleh raja Babel menjadi gubernur kota-kota Yehuda. Engkau boleh tinggal dengan dia di antara rakyat, atau pergi ke mana saja menurut keinginanmu." Lalu ia memberikan bekal makanan dan hadiah kepadaku, dan melepaskan aku pergi.
\par 6 Aku pergi kepada Gedalya di Mizpa, dan diam dengan dia di tengah-tengah rakyat yang ditinggalkan di negeri Yehuda.
\par 7 Raja Babel telah mengangkat Gedalya menjadi gubernur di Yehuda untuk memerintah atas semua orang yang masih tinggal di Yehuda, yaitu orang-orang yang paling miskin. Sebagian dari para perwira dan prajurit Yehuda yang tidak menyerah kepada Babel, mendengar tentang pengangkatan Gedalya itu.
\par 8 Maka Ismael anak Netanya, Yohanan anak Kareah, Seraya anak Tanhumet, dan anak-anak Efai dari Netofa, serta Yezanya dari Maakha pergi dengan orang-orang mereka kepada Gedalya di Mizpa.
\par 9 Gedalya berkata kepada mereka, "Kalian tidak perlu takut untuk menyerah kepada orang Babel. Tinggallah saja di negeri kita ini, dan jadilah hamba raja Babel. Saya tanggung kalian akan selamat.
\par 10 Tinggallah di kota-kota yang kalian kehendaki, dan silakan mengumpulkan serta menyimpan anggur, buah-buahan dan minyak zaitun. Saya sendiri akan tinggal di Mizpa dan mewakili rakyat apabila orang-orang Babel datang ke mari."
\par 11 Sementara itu semua orang Israel yang berada di negeri Moab, Amon, Edom, dan negeri-negeri lain, mendengar bahwa raja Babel telah mengizinkan sebagian orang Israel tinggal di Yehuda, dan bahwa ia telah mengangkat Gedalya menjadi gubernur.
\par 12 Karena itu mereka meninggalkan tempat-tempat di mana mereka telah diceraiberaikan, lalu kembali ke Yehuda. Mereka datang kepada Gedalya di Mizpa, dan mengumpulkan banyak sekali anggur dan buah-buahan.
\par 13 Setelah itu, Yohanan dan para perwira yang tidak menyerah kepada musuh, datang kepada Gedalya di Mizpa
\par 14 dan berkata, "Tahukah engkau bahwa Baalis raja Amon telah menyuruh Ismael membunuh engkau?" Tetapi Gedalya tidak percaya kepada perkataan mereka.
\par 15 Lalu dengan diam-diam berkatalah Yohanan kepada Gedalya, "Baiklah aku pergi membunuh Ismael; tak seorang pun akan tahu siapa yang melakukannya. Mengapa ia harus dibiarkan membunuh engkau? Itu akan mengakibatkan semua orang Yehuda yang telah berkumpul di sini dengan engkau, tercerai berai lagi, dan hal itu akan menghancurkan seluruh rakyat yang masih tersisa di Yehuda."
\par 16 Tetapi Gedalya menjawab, "Jangan! Yang kaukatakan tentang Ismael itu tidak benar!"

\chapter{41}

\par 1 Pada bulan tujuh tahun itu Ismael anak Netanya dan cucu Elisama pergi bersama sepuluh orang ke Mizpa untuk bertemu dengan Gedalya. (Ismael adalah seorang keturunan raja, salah seorang perwira tinggi kerajaan.) Sementara mereka semua makan bersama-sama,
\par 2 Ismael dan kesepuluh orang itu mencabut pedang mereka dan membunuh Gedalya, gubernur yang diangkat oleh orang Babel itu.
\par 3 Ismael membunuh juga semua orang Yehuda yang ada bersama Gedalya di Mizpa, dan semua tentara Babel yang kebetulan berada di situ.
\par 4 Hari berikutnya, sebelum pembunuhan Gedalya diketahui orang,
\par 5 datanglah delapan puluh orang dari Sikhem, Silo, dan Samaria. Mereka telah mencukur janggut, merobek pakaian, dan menoreh-noreh badan mereka. Mereka membawa gandum dan kemenyan untuk dipersembahkan di Rumah TUHAN.
\par 6 Maka keluarlah Ismael dari Mizpa dan pergi menemui mereka. Sambil menangis, ia berkata, "Mari pergi melihat Gedalya."
\par 7 Segera setelah mereka sampai di tengah-tengah kota, Ismael dan orang-orangnya mulai membunuh mereka. Tapi sepuluh di antara mereka berkata kepada Ismael, "Jangan bunuh kami! Kami masih punya dua macam gandum, juga minyak zaitun, dan madu. Semuanya kami sembunyikan di luar kota." Karena itu Ismael tidak membunuh mereka, tapi yang lain dibunuhnya dan mayat mereka dilempar ke dalam sumur.
\par 9 Sumur itu adalah sumur besar yang dibuat Raja Asa pada waktu ia diserang Baesa raja Israel.
\par 10 Kemudian Ismael menangkap para putri raja dan semua orang yang masih ada di Mizpa yang telah diserahkan oleh Nebuzaradan kepada pengawasan Gedalya. Ismael menawan mereka lalu berangkat dengan mereka ke jurusan wilayah Amon.
\par 11 Yohanan anak Kareah dan semua perwira yang bersama-sama dengan dia mendengar tentang kejahatan Ismael.
\par 12 Maka mereka bersama anak buah mereka pergi mengejar dia, dan mendapati dia di kolam besar Gibeon.
\par 13 Melihat Yohanan bersama perwira-perwira itu datang, para tawanan Ismael sangat gembira,
\par 14 lalu berbalik dan lari mendapatkan rombongan Yohanan.
\par 15 Tapi Ismael dan delapan anak buahnya lolos, dan melarikan diri ke negeri Amon.
\par 16 Kemudian Yohanan dan perwira-perwiranya mengumpulkan para prajurit, wanita, anak-anak, dan para pegawai istana yang ditawan oleh Ismael setelah membunuh Gedalya.
\par 17 Mereka semua menjadi takut kepada orang Babel, karena Ismael telah membunuh Gedalya yang diangkat oleh raja Babel menjadi gubernur negeri itu. Jadi, mereka melarikan diri ke Mesir. Di tengah perjalanan, mereka berhenti di Kimham dekat Betlehem.

\chapter{42}

\par 1 Setelah itu semua orang, besar kecil, bersama perwira-perwira, termasuk Yohanan anak Kareah dan Azarya anak Hosaya datang
\par 2 kepadaku dan berkata, "Pak, seperti yang Bapak lihat sendiri, kami ini dulu banyak, sekarang tinggal sedikit saja. Kami mohon Bapak mau berdoa untuk kami kepada TUHAN Allah yang Bapak sembah. Doakanlah kami yang masih tersisa ini,
\par 3 supaya TUHAN Allah menunjukkan kepada kami ke mana kami harus pergi dan apa yang harus kami lakukan."
\par 4 Aku menjawab, "Baiklah kalau begitu. Aku menerima permintaanmu itu. Aku akan berdoa kepada TUHAN Allah kita, dan apa pun jawaban-Nya, akan kuberitahukan semuanya; satu pun tidak akan kusembunyikan daripadamu."
\par 5 Mereka berkata kepadaku, "Kami minta Bapak berdoa kepada TUHAN Allah kita, karena kami mau taat kepada-Nya, supaya keadaan kami menjadi baik. Jadi, baik atau buruk pesan TUHAN itu, kami akan menurutinya. Biarlah TUHAN menjadi saksi yang jujur dan setia terhadap kami, apabila kami tidak menuruti semua perintah yang TUHAN berikan kepada kami melalui Bapak."
\par 7 Sepuluh hari kemudian TUHAN berbicara kepadaku.
\par 8 Maka aku memanggil Yohanan serta perwira-perwiranya dan semua orang lain yang datang bersama dengan dia.
\par 9 Aku berkata kepada mereka, "Kamu menyuruh aku menyampaikan permohonanmu kepada TUHAN, Allah Israel. Nah, TUHAN berkata,
\par 10 'Kalau kamu mau tetap tinggal di negeri ini, Aku akan mengangkat kamu, bukan menjatuhkan; Aku akan menegakkan kamu, bukan meruntuhkan. Aku sedih sekali karena telah mendatangkan celaka ke atasmu.
\par 11 Jangan lagi takut kepada raja Babel itu. Aku ada di tengah-tengahmu untuk menolong dan melepaskan kamu dari kekuasaannya.
\par 12 Karena Aku berbelaskasihan kepadamu, Aku akan membuat raja Babel mengasihani kamu dan mengizinkan kamu pulang. Aku, TUHAN, telah berbicara.'"
\par 13 Tapi kamu yang masih tinggal di Yehuda janganlah membangkang kepada TUHAN Allahmu dan jangan berkata, "Kami tidak mau tinggal di negeri ini. Kami mau pergi ke Mesir dan tinggal di negeri itu. Di sana kami tidak lagi mengalami peperangan, tidak mendengar bunyi trompet yang memanggil untuk bertempur, dan tidak juga menderita kelaparan." Kalau kamu berkeras hati untuk pergi ke Mesir dan tinggal di sana, inilah pesan TUHAN Yang Mahakuasa, Allah Israel, bagimu:
\par 16 "Peperangan dan kelaparan yang kamu takuti itu akan mengejar kamu sampai ke Mesir, dan kamu akan mati di negeri itu.
\par 17 Semua orang yang berkeras untuk pergi ke Mesir dan tinggal di sana akan mati dalam peperangan, atau mati karena kelaparan atau wabah penyakit. Tidak seorang pun akan luput dari bencana yang akan Kudatangkan ke atas mereka; semuanya akan mati.
\par 18 Sebagaimana Aku, TUHAN Allah Israel menumpahkan kemarahan dan kegeraman-Ku ke atas penduduk Yerusalem, begitu pula akan Kutumpahkan kegeraman-Ku ke atas kamu yang pergi ke Mesir. Orang akan ngeri melihat kamu; kamu akan dihina, dan namamu akan dipakai sebagai kutukan. Kamu tidak akan melihat tanah airmu ini lagi."
\par 19 Lalu aku berkata lagi, "TUHAN sudah memerintahkan supaya kamu yang masih tersisa di Yehuda tidak pergi ke Mesir. Karena itu aku memperingatkan kamu sekarang
\par 20 bahwa kamu telah membuat suatu kesalahan besar. Kamu minta supaya aku berdoa kepada TUHAN Allah kita untuk kamu, dan kamu berjanji untuk melaksanakan semua yang dikatakan oleh TUHAN.
\par 21 Sekarang aku sudah menyampaikan pesan TUHAN Allah kita kepadamu, tapi kamu tidak mau menuruti pesan itu.
\par 22 Sebab itu, ingatlah: di negeri yang kamu ingin datangi untuk menetap itu, kamu akan mati dalam peperangan, atau karena kelaparan atau wabah penyakit."

\chapter{43}

\par 1 Pesan TUHAN, Allah Israel, yang harus kusampaikan kepada rakyat telah kusampaikan seluruhnya.
\par 2 Lalu Azarya anak Hosaya dan Yohanan anak Kareah serta semua orang lain yang sombong itu berkata kepadaku, "Engkau bohong. TUHAN Allah kami tidak menyuruh engkau melarang kami untuk pergi tinggal di Mesir.
\par 3 Barukh anak Neria itulah yang menghasut engkau untuk menentang kami, supaya orang Babel menguasai kami dan dapat membunuh atau mengangkut kami ke Babel."
\par 4 Baik Yohanan maupun semua perwira dan rakyat tak mau menuruti perintah TUHAN untuk tinggal di Yehuda.
\par 5 Yohanan dan semua perwira itu membawa ke Mesir setiap orang yang masih ada di Yehuda, juga semua orang yang telah kembali dari negeri lain tempat mereka tersebar.
\par 6 Pria, wanita, anak-anak, dan putri-putri raja yang diserahkan Nebuzaradan kepada pengawasan Gedalya, termasuk aku dan Barukh, semuanya dibawa ke Mesir.
\par 7 Mereka melawan perintah TUHAN, dan pergi ke Mesir sampai sejauh kota Tahpanhes.
\par 8 Di sana TUHAN berkata kepadaku,
\par 9 "Ambillah batu-batu besar, dan tanamlah di dalam tanah di depan pintu masuk ke gedung pemerintah kota itu. Usahakan supaya orang-orang Yehuda melihat kau melakukan hal itu.
\par 10 Lalu katakan kepada mereka bahwa Aku, TUHAN Yang Mahakuasa, Allah Israel, akan mendatangkan hamba-Ku Nebukadnezar raja Babel, ke tempat itu. Dan di atas batu-batu yang kautanam di dalam tanah itu, ia akan mendirikan takhtanya serta membentangkan kemah kerajaannya.
\par 11 Nebukadnezar akan datang dan mengalahkan Mesir. Dan orang-orang yang telah ditentukan untuk mati karena wabah penyakit akan mati karena wabah penyakit. Mereka yang ditentukan untuk diangkut sebagai tawanan akan diangkut sebagai tawanan, dan yang ditentukan untuk tewas dalam peperangan akan tewas dalam peperangan.
\par 12 Aku akan menyalakan api di kuil-kuil dewa-dewa Mesir, dan raja Babel akan membakar patung-patung berhala di kuil-kuil atau mengangkutnya ke negerinya. Seperti seorang gembala membersihkan pakaiannya dari kutu, begitu juga raja Babel akan menyapu bersih Mesir lalu berangkat lagi dengan tak ada yang mengganggunya.
\par 13 Tugu-tugu berhala di Heliopolis di Mesir, akan dihancurkan, dan kuil-kuil dewa-dewa Mesir akan dibakar habis."

\chapter{44}

\par 1 TUHAN berbicara kepadaku mengenai semua orang Israel yang tinggal di Mesir, yaitu di kota Migdol, Tahpanhes, Memfis, dan di daerah selatan.
\par 2 Inilah pesan TUHAN Yang Mahakuasa, Allah Israel, kepada mereka, "Kamu sendiri telah menyaksikan bencana yang Kutimpakan ke atas Yerusalem dan semua kota lain di Yehuda. Sampai sekarang pun kota-kota itu masih dalam keadaan hancur tanpa penghuni.
\par 3 Hal itu terjadi karena penduduknya berdosa, sehingga membuat Aku marah. Mereka mempersembahkan kurban dan berbakti kepada dewa-dewa yang belum pernah disembah oleh mereka sendiri, oleh kamu atau leluhurmu.
\par 4 Aku terus-menerus mengirim kepadamu hamba-hamba-Ku para nabi, yang melarang mereka melakukan kejahatan yang Kubenci itu.
\par 5 Tapi kamu tidak mau mendengar dan tidak mau memperhatikan. Kamu tidak mau berhenti melakukan yang jahat, yaitu mempersembahkan kurban kepada dewa-dewa.
\par 6 Itu sebabnya Aku menumpahkan kemarahan dan murka-Ku ke atas kota-kota Yehuda dan jalan-jalan di Yerusalem. Semuanya Kubakar sehingga menjadi puing-puing dan tandus seperti yang dapat dilihat sekarang ini.
\par 7 Dan kini Aku, TUHAN Yang Mahakuasa, Allah Israel, bertanya mengapa kamu melakukan hal yang mendatangkan celaka yang besar itu terhadap dirimu. Apakah kamu mau membinasakan semua orang--pria, wanita, anak-anak, dan bayi--sehingga tak seorang pun dari bangsamu yang tertinggal?
\par 8 Mengapa kamu melakukan hal-hal yang membuat Aku marah? Kamu datang dan tinggal di Mesir, negeri yang asing bagimu itu, lalu kamu menyembah berhala dan mempersembahkan kurban kepada dewa-dewa negeri itu! Apakah kamu mau menghancurkan dirimu sendiri supaya kamu dihina oleh segala bangsa di dunia dan namamu dipakai sebagai kutukan?
\par 9 Sudah lupakah kamu akan semua kejahatan yang dilakukan di tanah Yehuda dan di jalan-jalan kota Yerusalem oleh leluhurmu, oleh raja-raja Yehuda dan istri-istri mereka, serta kamu dan istri-istrimu?
\par 10 Tapi sampai pada hari ini kamu tidak merendahkan diri. Kamu tidak menghormati Aku dan tidak hidup menurut hukum yang Kuberikan kepadamu dan kepada leluhurmu.
\par 11 Sebab itu, Aku, TUHAN Yang Mahakuasa, Allah Israel, akan melawan kamu dan menghancurkan seluruh Yehuda.
\par 12 Dan jika dari orang Yehuda yang tersisa ada yang tetap ingin ke Mesir untuk tinggal di sana, mereka akan binasa semua. Mereka semua, besar kecil, akan mati di Mesir dalam peperangan atau karena kelaparan. Orang akan merasa ngeri melihat mereka dan akan menghina mereka. Nama mereka akan dipakai sebagai kutukan.
\par 13 Seperti Aku menghukum Yerusalem, begitu juga akan Kuhukum mereka yang tinggal di Mesir. Mereka akan mati dalam peperangan, karena kelaparan atau wabah penyakit.
\par 14 Dari orang Yehuda yang tersisa, yang telah pergi ke Mesir untuk tinggal di sana, tak seorang pun akan luput atau hidup. Tak seorang pun dari mereka akan kembali ke Yehuda, meskipun hal itu sangat mereka inginkan. Sungguh, tidak ada yang akan kembali kecuali beberapa pengungsi."
\par 15 Lalu datanglah banyak sekali orang kepadaku, yaitu semua laki-laki yang tahu bahwa istrinya mempersembahkan kurban kepada dewa, dan semua wanita yang sedang berdiri di situ, termasuk orang Israel yang tinggal di Patros, bagian selatan Mesir. Mereka berkata kepadaku,
\par 16 "Kami tidak mau mendengar apa yang kaukatakan kepada kami atas nama TUHAN.
\par 17 Segala yang telah kami janjikan, akan kami lakukan juga. Kami akan tetap mempersembahkan kurban kepada Ratu Surga, dan menuang air anggur sebagai persembahan kepadanya seperti yang biasanya dilakukan di kota-kota Yehuda dan di jalan-jalan di Yerusalem oleh leluhur kami, oleh raja-raja dan pejabat-pejabat pemerintah kami serta kami sendiri. Pada waktu itu kami mempunyai cukup makanan, kami makmur, dan tidak mempunyai kesukaran apa-apa.
\par 18 Tapi sejak kami berhenti mempersembahkan kurban kepada Ratu Surga, dan tidak lagi menuang anggur sebagai persembahan kepadanya, kami kekurangan segala-galanya; orang-orang kami mati dalam peperangan atau karena kelaparan."
\par 19 Kemudian wanita-wanita itu berkata, "Suami kami setuju bahwa kami membuat roti berbentuk Ratu Surga dan membakar kurban serta mempersembahkan anggur untuk dewa kami itu."
\par 20 Kepada semua pria dan wanita yang menjawab begitu, aku berkata,
\par 21 "Apakah kamu menyangka TUHAN tidak mengetahui atau sudah melupakan kurban-kurban yang dipersembahkan di kota-kota Yehuda dan di jalan-jalan di Yerusalem oleh leluhurmu, oleh raja-raja dan pejabat-pejabatmu, serta rakyat negeri ini dan oleh kamu sendiri?
\par 22 TUHAN tidak tahan lagi melihat kamu mempersembahkan kurban kepada ilah-ilah lain. Perbuatanmu itu jahat dan hina. Kamu telah berdosa kepada TUHAN dan tidak taat kepada perintah-perintah-Nya. Sebab itu sekarang ini juga negerimu sudah hancur sehingga tak dapat ditempati lagi, dan namanya dipakai sebagai kutukan. Orang yang melihatnya merasa ngeri."
\par 24 Lalu aku memberitahukan kepada semua orang itu, terutama kepada para wanitanya, bahwa TUHAN Yang Mahakuasa berkata begini kepada orang-orang Yehuda yang tinggal di Mesir, "Kamu dan istri-istrimu sudah bersumpah kepada Ratu Surga bahwa kamu akan membakar kurban dan mempersembahkan air anggur kepadanya. Janjimu itu sudah kamu tepati. Jadi, baiklah! Lakukan saja apa yang kamu janjikan itu!
\par 26 Tetapi sekarang dengarkan apa yang Kukatakan kepadamu, hai orang Israel di Mesir: Aku, TUHAN, bersumpah demi nama-Ku yang agung bahwa kamu tidak lagi Kuizinkan memakai nama-Ku untuk membuat sesuatu janji. Kamu tidak boleh berkata, 'Aku bersumpah demi Allah yang hidup!'
\par 27 Aku akan berusaha supaya kamu celaka dan tidak bahagia, sampai kamu mati semua, baik dalam peperangan maupun karena wabah penyakit.
\par 28 Hanya beberapa orang saja yang akan selamat dan kembali ke Yehuda dari Mesir. Maka semua orang Yehuda yang ada di Mesir akan melihat apakah perkataan mereka ataukah perkataan-Ku yang menjadi kenyataan.
\par 29 Aku akan mendatangkan celaka ke atasmu di tempat ini untuk membuktikan kepadamu bahwa hukuman yang Kurencanakan untuk kamu, sungguh-sungguh akan terjadi.
\par 30 Hofra raja Mesir akan Kuserahkan kepada musuh-musuh yang ingin membunuhnya, sama seperti Aku menyerahkan Zedekia raja Yehuda kepada Nebukadnezar raja Babel, musuhnya yang mau membunuhnya itu."

\chapter{45}

\par 1 Pada tahun keempat pemerintahan Yoyakim anak Yosia atas Yehuda, Barukh menuliskan apa yang kudiktekan kepadanya. Lalu aku menyampaikan kepadanya
\par 2 perkataan TUHAN, Allah Israel. TUHAN berkata, "Barukh,
\par 3 engkau mengeluh begini, 'Ah, sungguh sial aku ini! Kesukaranku amat banyak, dan TUHAN menambahnya lagi dengan penderitaan. Aku sudah capek mengeluh, tidak ada ketenangan sama sekali!'
\par 4 Hai Barukh! Aku, TUHAN, sedang meruntuhkan yang telah Kubangun, dan sedang mencabut yang telah Kutanam. Seluruh dunia Kuperlakukan demikian.
\par 5 Mengapa engkau ingin diistimewakan? Jangan begitu, Barukh! Seluruh umat manusia Kuhukum dengan bencana, tapi mengenai engkau, setidak-tidaknya engkau akan selamat. Ke mana pun engkau pergi, engkau akan tetap hidup. Aku, TUHAN telah berbicara."

\chapter{46}

\par 1 TUHAN berbicara kepadaku tentang bangsa-bangsa,
\par 2 mulai dengan Mesir. Ia memberitahukan kepadaku tentang tentara Nekho raja Mesir, yang dikalahkan di Karkemis dekat Sungai Efrat oleh Nebukadnezar raja Babel pada tahun keempat pemerintahan Yoyakim, raja Yehuda. Inilah yang dikatakan TUHAN tentang hal itu,
\par 3 "Perwira-perwira Mesir meneriakkan pekikan siaga kepada pasukannya, 'Tajamkan tombak! Pakailah baju besi dan topi baja! Siapkan perisai dan masuklah dalam barisan! Kamu, pasukan berkuda, pasanglah pelana! Tunggangi kudamu, maju bertempur!'
\par 5 Tapi Aku, TUHAN, melihat mereka mundur ketakutan. Prajurit-prajurit mereka dikalahkan, dan lari ketakutan tanpa menoleh ke belakang.
\par 6 Yang gesit tak mampu melarikan diri, dan yang perkasa tak dapat luput dari musuh. Mereka terantuk dan jatuh di utara, di tepi Sungai Efrat itu.
\par 7 Siapakah dia yang naik seperti air Sungai Nil, seperti sungai yang membanjiri tepi-tepinya? Itulah negeri Mesir. Katanya, 'Aku akan bangkit dan menguasai bumi. Akan kubinasakan kota-kota serta penduduknya.
\par 9 Berpaculah, hai kuda! Bergeraklah, hai kereta! Majulah, hai pahlawan pembawa perisai dari Sudan dan Libia. Juga kamu, hai pemanah-pemanah ahli dari negeri Lidia.'"
\par 10 Tapi aku, Yeremia berkata bahwa inilah hari TUHAN Yang Mahatinggi dan Mahakuasa; pada hari itu Ia mengadakan pembalasan atas semua musuh-musuh-Nya. Pedang-Nya akan makan mereka sampai kenyang, dan minum darah mereka sampai puas. Hari ini TUHAN Yang Mahakuasa, menyembelih kurban-kurban-Nya di utara, di tepi Sungai Efrat itu.
\par 11 Hai penduduk Mesir, pergilah ke Gilead untuk mencari obat! Tapi, sekalipun demikian tak ada obat yang dapat menyembuhkan kamu.
\par 12 Keadaanmu yang memalukan telah diketahui oleh bangsa-bangsa. Suara teriakanmu didengar di seluruh dunia. Prajurit yang satu tersandung pada yang lain; sama-sama mereka jatuh terjerembab ke atas tanah.
\par 13 Ketika Nebukadnezar raja Babel datang menyerang Mesir, TUHAN berbicara kepadaku, katanya,
\par 14 "Yeremia, umumkanlah berita berikut ini di kota-kota Mesir, yakni di Memfis, Migdol dan Tahpanhes, 'Bersiaplah untuk mempertahankan diri sebab semua yang kamu miliki akan dimusnahkan dalam peperangan!
\par 15 Mengapa Apis, dewamu yang kuat itu telah melarikan diri? Karena TUHAN telah mengalahkan dia!'
\par 16 Banyak tentaramu tersandung dan jatuh. Mereka berkata satu sama lain, 'Mari pulang ke bangsa kita agar kita dapat luput dari bahaya pedang musuh.'
\par 17 Berilah nama baru kepada raja Mesir: 'Si tukang ribut yang melewatkan kesempatan.'
\par 18 Aku Allah yang hidup, Aku raja, TUHAN Yang Mahakuasa. Seperti Gunung Tabor menjulang antara gunung-gunung, dan Gunung Karmel menjulang di atas permukaan laut, begitulah kekuatan dia yang datang menyerang engkau.
\par 19 Bersiap-siaplah, penduduk Mesir, untuk ditawan dan dibawa pergi! Memfis akan menjadi puing--tempat yang sepi karena tidak dihuni.
\par 20 Mesir seperti anak sapi gemuk, tapi diganggu lalat besar dari utara.
\par 21 Bahkan prajurit-prajurit sewaannya tidak berdaya seperti anak sapi yang terlalu gemuk. Mereka tak dapat bertahan. Semuanya berbalik dan lari supaya selamat. Hari hukuman bagi mereka telah tiba; sudah waktunya mereka ditimpa bencana.
\par 22 Mesir melarikan diri sambil mendesis seperti ular karena tentara musuh makin mendekat. Mereka menyerang dia dengan kapak seperti orang menebang pohon
\par 23 dan menggunduli hutan lebat yang tak dapat diterobosi. Tentara musuh tak terhitung, lebih banyak dari belalang. Aku, TUHAN berkata.
\par 24 Penduduk Mesir akan dibuat malu, karena dikalahkan oleh bangsa dari utara. Aku, TUHAN, telah berbicara."
\par 25 TUHAN Yang Mahakuasa, Allah Israel, berkata, "Aku akan menghukum Amon, dewa dari Tebe, bersama seluruh Mesir dengan dewa-dewa dan raja-rajanya. Raja Mesir dan semua orang yang mengandalkan dia
\par 26 akan Kuserahkan kepada Nebukadnezar raja Babel dengan tentaranya yang mau membunuh mereka. Tetapi kemudian, Mesir akan didiami seperti sediakala. Aku, TUHAN, telah berbicara."
\par 27 "Umat-Ku, Israel, janganlah takut! Hamba-Ku Yakub, jangan gentar! Kamu dan keturunanmu akan Kuselamatkan dari negeri jauh, di tempat kamu ditawan. Kamu akan pulang dan tinggal di negerimu dengan aman. Kamu akan hidup dengan tentram; tak ada yang perlu kamu takutkan.
\par 28 Jadi janganlah takut, hai umat-Ku, Aku akan datang untuk menyelamatkan kamu. Bangsa-bangsa, tempat kamu Kuceraiberaikan, semuanya akan Kubinasakan, tapi kamu tidak Kuperlakukan demikian. Memang kamu tidak akan luput dari hukuman tapi hukuman-Ku kepadamu adalah adil sesuai dengan ketentuan. Aku, TUHAN, telah berbicara."

\chapter{47}

\par 1 Sebelum raja Mesir menyerang Gaza, TUHAN berbicara kepadaku mengenai negeri Filistin.
\par 2 TUHAN berkata, "Lihat, air mengalir dari utara dan meluap seperti sungai yang sedang banjir melanda negeri serta isinya, kota-kota bersama penduduknya. Orang akan berteriak minta tolong dan seluruh rakyat meratap.
\par 3 Mereka akan mendengar derap kuda, bunyi derak-derik kereta, dan kertak-kertuk roda-rodanya. Orang tua akan menjadi lemah dan tidak lagi memikirkan anaknya.
\par 4 Telah tiba saatnya bangsa Filistin dibinasakan seluruhnya, sehingga Tirus dan Sidon tidak lagi mendapat bantuan. Aku, TUHAN, akan membinasakan semua orang Filistin yang tersisa dari Kreta.
\par 5 Orang Gaza ditimpa kesedihan besar; penduduk Askelon membisu. Sampai kapan orang Filistin yang tersisa terus berduka?
\par 6 Kamu berteriak, 'Sampai kapan engkau menyerang, hai pedang TUHAN? Kembalilah ke sarungmu dan tinggallah di sana dengan tenang!'
\par 7 Tapi dapatkah ia menjadi tenang, apabila Aku sudah memerintahkannya untuk menyerang Askelon dan penduduk tepi pantai?"

\chapter{48}

\par 1 Inilah yang dikatakan TUHAN Yang Mahakuasa, Allah Israel mengenai Moab, "Celakalah penduduk Nebo, kotanya sudah roboh! Kiryataim direbut, dan bentengnya yang kuat dihancurkan, sehingga penduduknya menjadi malu;
\par 2 lenyaplah kebanggaan Moab. Musuh telah merebut Hesybon dan di sana mereka merencanakan untuk melenyapkan seluruh bangsa Moab. Suara-suara di kota Madmen akan dihentikan, karena kota itu akan diserbu tentara.
\par 3 Dengar! Penduduk Horonaim berteriak, 'Kebinasaan! Kehancuran!'
\par 4 Moab sudah musnah! Teriakan anak-anak kecilnya terdengar sampai ke Zoar.
\par 5 Dengarlah suara tangis di jalan naik ke Luhit, dan suara ratapan di jalan turun ke Horonaim.
\par 6 Kata mereka, 'Cepat! Selamatkan nyawamu! Larilah seperti keledai liar di gurun!'
\par 7 Moab! Engkau mengandalkan ketangkasan dan kekayaanmu. Tapi sekarang engkau sendiri akan dikalahkan. Kamos, dewamu, bersama para imam dan para pejabatnya akan diangkut ke pembuangan.
\par 8 Semua kota akan hancur, tak satu pun yang akan luput. Lembah dan dataran rendah seluruhnya akan musnah. Aku, TUHAN, telah berbicara.
\par 9 Pasanglah batu nisan untuk Moab, sebab tak lama lagi negeri itu akan dimusnahkan sama sekali. Kota-kotanya akan runtuh dan tak seorang pun akan tinggal di sana."
\par 10 Terkutuklah orang yang tidak melaksanakan tugas dari TUHAN dengan sungguh-sungguh. Terkutuklah dia yang tak menghunus pedang dan membunuh.
\par 11 TUHAN berkata, "Sejak dulu Moab hidup aman. Belum pernah ia diangkut ke pembuangan. Ia seperti air anggur yang dibiarkan mengendap dan tak pernah dituang ke tempat anggur yang lain. Anggurnya masih harum dan enak.
\par 12 Tetapi, akan tiba waktunya Aku menyuruh agar Moab dituang seperti anggur, dan tempat-tempatnya dikosongkan dan dipecahkan.
\par 13 Moab akan kecewa terhadap Kamos, dewa mereka, sama seperti orang Israel kecewa terhadap Betel yang mereka andalkan itu.
\par 14 Bagaimana dapat laki-laki Moab berani berkata, 'Kami pahlawan, prajurit perkasa yang telah diuji dalam peperangan?'
\par 15 Moab dan kota-kotanya telah dihancurkan, dan orang-orang mudanya yang terbaik telah dibunuh. Aku telah berbicara. Aku raja, TUHAN Yang Mahakuasa.
\par 16 Malapetaka untuk Moab sudah diambang pintu, tak lama lagi ia hancur.
\par 17 Hai kamu negara-negara tetangga Moab yang tahu tentang kemasyhurannya, berkabunglah untuk dia! Katakan, 'Moab tidak lagi berkuasa, kebesaran dan kekuatannya sudah lenyap!'
\par 18 Hai penduduk Dibon, turunlah dari tempatmu yang terhormat; duduklah di tanah dan di atas debu. Karena yang merusak Moab negerimu telah tiba, dan akan menghancurkan semua bentengmu.
\par 19 Kamu yang tinggal di Aroer, berdirilah di tepi jalan dan perhatikan! Tanyakanlah kepada mereka yang melarikan diri, dan lolos, 'Apakah yang telah terjadi?'
\par 20 Mereka akan menjawab, 'Moab sudah jatuh dan menjadi hina, karena itu tangisilah dia. Umumkan di sepanjang Sungai Arnon bahwa Moab sudah hancur.'"
\par 21 "Hukuman telah dijatuhkan ke atas kota-kota di daerah dataran tinggi: atas Holon, Yahas, Mefaat,
\par 22 Dibon, Nebo, Bet-Diblataim,
\par 23 Kiryataim, Bet-Gamul, Bet-Meon,
\par 24 Keriot, dan Bozra. Semua kota Moab, jauh dan dekat, telah dihukum.
\par 25 Kekuatan Moab telah dipatahkan, dan kekuasaannya dihancurkan. Aku, TUHAN, telah berbicara."
\par 26 TUHAN berkata, "Buatlah Moab menjadi mabuk, karena ia menyombongkan diri terhadap Aku. Ia akan berguling-guling dalam muntahnya, dan ditertawakan orang.
\par 27 Bukankah dahulu orang Israel telah ditertawakan oleh Moab dan dipergunjingkan seolah-olah mereka telah tertangkap basah bersama pencuri-pencuri?
\par 28 Kamu penduduk Moab, tinggalkanlah kota-kotamu dan diamlah di bukit batu seperti burung merpati membuat sarangnya di tebing jurang.
\par 29 Kamu angkuh sekali! Aku sudah mendengar betapa sombong, tinggi hati, congkak, dan pongahnya kamu!
\par 30 Aku, TUHAN, tahu keangkuhanmu. Bicaramu kosong belaka, dan perbuatanmu tidak berarti.
\par 31 Sebab itu Aku akan menangis keras-keras untuk kamu semua dan untuk orang-orang di Kir-Heres, ibukotamu.
\par 32 Aku akan lebih menangisi penduduk Sibma daripada penduduk Yaezer. Kota Sibma seperti tanaman anggur yang carang-carangnya merambat sampai ke seberang Laut Mati sejauh Yaezer. Tapi sekarang buah-buah musim panasnya dan buah anggurnya sudah dimusnahkan.
\par 33 Sorak-sorai keriangan sudah lenyap dari negeri Moab yang subur. Aku membuat air anggur berhenti mengalir dari alat pemeras anggur; tak ada lagi orang yang membuat anggur dan bersorak gembira.
\par 34 Orang di Hesybon dan Eleale menangis dan tangisnya terdengar sampai ke Yahas, dan terdengar juga oleh orang-orang dari Zoar sampai ke Horonaim dan Eglat-Selisia. Mereka menangis karena air di anak Sungai Nimrim pun sudah kering.
\par 35 Aku akan membuat orang Moab berhenti membakar kurban pada tempat-tempat penyembahan mereka dan berhenti mempersembahkan kurban kepada dewa-dewa mereka. Aku, TUHAN, telah berbicara.
\par 36 Semua kekayaan Kir-Heres dan seluruh Moab sudah lenyap. Aku sedih melihat keadaan mereka itu; Aku seperti orang yang meniup lagu duka pada seruling.
\par 37 Mereka semua sudah mencukur rambut kepala dan janggut, melukai lengan, dan memakai kain karung tanda sedih.
\par 38 Di atas semua sotoh rumah dan di tanah-tanah lapang hanya terdengar tangisan. Sebab Moab telah Kupecahkan seperti tempayan yang tak disukai.
\par 39 Moab telah hancur lebur! Betapa malunya Moab! Ia menangis karena telah dihina dan menjadi tertawaan bangsa-bangsa di sekelilingnya. Aku, TUHAN, telah berbicara."
\par 40 TUHAN berkata, "Seperti burung rajawali menyambar dengan sayap terkembang, demikianlah sebuah bangsa akan datang dan menyambar Moab.
\par 41 Kota-kota dan benteng-bentengnya akan direbut dan tentaranya akan ketakutan seperti wanita yang mau melahirkan.
\par 42 Moab akan hancur dan tidak lagi merupakan sebuah bangsa, karena mereka menyombongkan diri terhadap Aku.
\par 43 Mereka dihadapkan dengan teror, lubang dan jerat.
\par 44 Orang yang melarikan diri dari teror akan jatuh ke dalam lubang, dan yang keluar dari lubang itu akan tertangkap dalam jerat. Semua itu akan terjadi pada saat yang telah Kutetapkan untuk menghancurkan Moab.
\par 45 Kaum pengungsi yang tak berdaya pergi mencari perlindungan ke Hesybon, tapi kota itu sedang terbakar dan istana Raja Sihon pun sedang dimakan api. Api telah memusnahkan daerah perbatasan dan pegunungan Moab, negeri orang-orang yang suka berperang.
\par 46 Celakalah Moab! Penduduknya yang menyembah Dewa Kamos telah dibinasakan, dan anak-anak mereka diangkut sebagai tawanan.
\par 47 Tapi di kemudian hari Aku akan memulihkan keadaan negeri Moab. Aku, TUHAN, telah berbicara!" Itulah yang akan dilakukan TUHAN terhadap Moab.

\chapter{49}

\par 1 Inilah yang dikatakan TUHAN mengenai Amon, "Mengapa tanah suku Gad direbut dan didiami orang-orang yang menyembah Dewa Milkom? Apakah sudah tak ada lagi orang Israel yang dapat membela daerah itu?
\par 2 Tapi akan tiba saatnya Aku membuat penduduk Raba, ibukota negeri Amon itu, mendengar pekik-pekik pertempuran. Kota itu akan menjadi reruntuhan dan desa-desa di sekitarnya terbakar habis. Lalu Israel akan mengambil kembali tanahnya dari orang-orang yang dahulu merebutnya.
\par 3 Hai orang Hesybon, menangislah sebab kota Ai sudah hancur! Hai wanita-wanita kota Raba, merataplah! Pakailah kain karung tanda sedih. Berlarilah ke sana ke mari penuh kebingungan. Milkom dewamu akan diangkut ke pembuangan bersama imam-imam dan para pejabat pemerintahnya!
\par 4 Hai kamu yang tak setia! Mengapa kamu membanggakan lembah-lembahmu yang subur? Apa sebab kamu mengandalkan kekayaanmu? Bagaimana mungkin kamu dapat berkata bahwa tak seorang pun berani menyerangmu?
\par 5 Kamu akan ditimpa kekejaman yang Kudatangkan dari segala pihak. Kamu akan lari, masing-masing berusaha menyelamatkan dirinya sendiri, dan tak ada yang mengumpulkan kamu lagi.
\par 6 Tapi di kemudian hari, keadaan bangsa Amon akan Kupulihkan. Aku, TUHAN, telah berbicara."
\par 7 Inilah yang dikatakan TUHAN Yang Mahakuasa mengenai Edom, "Apakah orang Edom tidak dapat berpikir secara bijaksana? Apakah para penasihatnya tidak memberi nasihat lagi kepada mereka? Sudah lenyapkah kebijaksanaan mereka!
\par 8 Hai penduduk Dedan, baliklah dan larilah! Pergilah bersembunyi. Aku akan menghancurkan keturunan Esau, karena sudah waktunya Aku menghukum mereka.
\par 9 Kalau orang memetik buah anggur, selalu ada yang disisakannya, dan kalau pencuri datang pada malam hari, ia hanya mengambil apa yang diingininya.
\par 10 Tetapi, Aku menyapu bersih milik keturunan Esau dan membuka rahasia tempat persembunyian mereka; mereka tak dapat menyembunyikan diri lagi. Orang Edom bersama saudara-saudara dan tetangga-tetangga mereka, semuanya sudah dibinasakan, tak seorang pun yang luput.
\par 11 Hai Edom! Tinggalkanlah anak-anak yatimmu kepada-Ku, Aku akan memelihara mereka. Biarlah para jandamu menaruh kepercayaannya kepada-Ku.
\par 12 Orang yang tidak patut dihukum pun, terkena hukuman juga, masakan kamu tidak? Pasti kamu harus dihukum!
\par 13 Aku bersumpah demi nama-Ku sendiri bahwa kota Bozra akan menjadi tempat yang mengerikan yang ditinggalkan orang. Orang akan menertawakannya dan memakai namanya sebagai kutukan. Semua desa di sekitarnya akan menjadi reruntuhan untuk selama-lamanya. Aku, TUHAN, telah berbicara."
\par 14 Aku, Yeremia, berkata, "Hai Edom, aku telah menerima suatu berita dari TUHAN. Ia telah mengirim utusan kepada bangsa-bangsa untuk menyuruh mereka mengumpulkan tentara dan bersiap-siap untuk menyerang engkau.
\par 15 TUHAN akan membuat engkau menjadi lemah dan dihina orang.
\par 16 Engkau tertipu oleh keangkuhanmu. Tak ada yang takut kepadamu seperti sangkamu. Engkau tinggal di celah-celah batu, jauh di puncak gunung. Tapi, meskipun kaubuat rumahmu di tempat yang tinggi sekali, setinggi tempat sarang burung rajawali, TUHAN akan menurunkan engkau dari situ. Tuhanlah yang mengatakan semuanya itu."
\par 17 TUHAN berkata, "Malapetaka yang menimpa Edom akan begitu hebat sehingga setiap orang yang lewat di situ akan terkejut dan takut.
\par 18 Sebagaimana Sodom dan Gomora dimusnahkan bersama desa-desa di sekitarnya, begitu juga Edom. Tak seorang pun akan tinggal lagi di sana. Aku, TUHAN, telah berbicara.
\par 19 Seperti singa muncul dari hutan lebat dekat Sungai Yordan dan mendatangi padang tempat domba merumput, demikianlah Aku akan datang dan membuat orang Edom lari dari negeri mereka dengan tiba-tiba. Lalu Aku akan memilih seorang pemimpin untuk memerintah bangsa itu. Siapakah dapat disamakan dengan Aku? Siapakah berani membuat perkara dengan Aku? Apakah ada pemimpin yang dapat melawan Aku?
\par 20 Karena itu, dengarkanlah apa yang telah Kurencanakan terhadap orang Edom, dan apa yang hendak Kulakukan terhadap penduduk kota Teman. Anak-anak mereka pun akan diseret pergi, dan semua orang akan ketakutan.
\par 21 Apabila Edom jatuh, akan terdengar keributan yang begitu hebat sehingga seluruh dunia goncang; teriakan-teriakan penduduknya akan terdengar sampai ke Laut Gelagah.
\par 22 Seperti burung rajawali menyambar dengan sayap terkembang, begitulah musuh akan datang dan menyambar Bozra. Pada hari itu tentara Edom akan ketakutan seperti wanita yang mau melahirkan."
\par 23 Inilah yang dikatakan TUHAN tentang Damsyik. "Penduduk kota Hamat dan Arpad khawatir dan gelisah karena mendengar berita buruk. Hati mereka risau dan terombang-ambing seperti gelombang laut.
\par 24 Penduduk Damsyik menjadi lemah, dan lari ketakutan. Mereka kesakitan dan menderita seperti wanita yang mau melahirkan.
\par 25 Kota termasyhur dan riang gembira itu kini sepi tanpa penghuni.
\par 26 Pada hari itu pemuda-pemudanya akan tewas di jalan-jalan kotanya, dan seluruh tentaranya dihancurkan.
\par 27 Tembok-tembok Damsyik akan Kubakar, dan istana-istana Raja Benhadad Kuhanguskan. Aku, TUHAN Yang Mahakuasa, telah berbicara."
\par 28 Inilah yang dikatakan TUHAN tentang suku Kedar dan daerah-daerah kekuasaan Hazor, yang telah direbut Nebukadnezar raja Babel, "Hai Nebukadnezar, seranglah suku Kedar dan binasakanlah orang-orang di sebelah timur!
\par 29 Rampaslah ternak domba, unta, serta kemah-kemah mereka dengan segala isinya, dan berteriaklah kepada mereka, 'Teror meliputi kamu!'
\par 30 Hai kamu orang Hazor! Aku, TUHAN, berkata: Cepatlah pergi mengungsi, dan bersembunyi. Nebukadnezar raja Babel telah merencanakan untuk menyerang kamu. Ia berkata,
\par 31 'Ayo maju! Mari kita menyerang orang-orang yang merasa aman dan tentram itu. Kota mereka tidak terlindung karena tak ada pintu gerbangnya dan tak ada palangnya.'"
\par 32 TUHAN berkata lagi kepada Nebukadnezar, "Rampaslah unta-unta dan seluruh ternak mereka! Orang-orang yang rambutnya dipotong pendek itu akan Kuserakkan ke mana-mana. Aku akan mendatangkan malapetaka ke atas mereka dari segala pihak.
\par 33 Hazor akan terlantar untuk selama-lamanya dan menjadi tempat bersembunyi anjing hutan. Tak seorang pun akan tinggal di sana lagi. Aku, TUHAN, telah berbicara."
\par 34 Tak lama setelah Zedekia menjadi raja Yehuda, TUHAN Yang Mahakuasa berbicara kepadaku tentang negeri Elam.
\par 35 TUHAN berkata, "Aku akan membunuh semua ahli memanah yang membuat Elam sangat kuat.
\par 36 Aku akan meniupkan angin ke Elam dari segala jurusan, dan menyerakkan penduduknya ke mana-mana sehingga tidak ada satu negeri pun yang tidak kedatangan pengungsi-pengungsi dari Elam itu.
\par 37 Aku akan membuat orang Elam takut kepada musuh yang mau membunuh mereka. Aku sangat marah kepada mereka dan akan menghancurkan mereka. Akan Kukirim tentara yang memerangi mereka sampai mereka habis sama sekali.
\par 38 Raja-raja dan pejabat-pejabat mereka akan Kuhancurkan, dan di negeri mereka akan Kudirikan takhta-Ku.
\par 39 Tapi di kemudian hari, keadaan Elam akan Kupulihkan. Aku, TUHAN, telah berbicara."

\chapter{50}

\par 1 Inilah pesan TUHAN kepadaku tentang kota Babel dan penduduknya,
\par 2 "Pasanglah tanda dan umumkan kepada bangsa-bangsa bahwa Babel telah jatuh! Jangan rahasiakan hal itu! Merodakh dewanya telah dihancurkan, dan patung-patungnya yang cabul pecah berantakan, serta berhala-berhalanya sangat dihinakan.
\par 3 Suatu bangsa dari utara akan datang menyerang Babel dan membuat negeri itu menjadi padang tandus. Manusia dan binatang akan lari, dan tak ada lagi yang mau tinggal di sana."
\par 4 TUHAN berkata, "Pada waktu itu orang Israel dan Yehuda akan datang bersama-sama dengan menangis, mencari Aku, Allah mereka.
\par 5 Mereka akan menanyakan jalan ke Sion, lalu berjalan ke jurusan itu. Mereka akan membuat perjanjian abadi dengan Aku, dan akan tetap memegangnya.
\par 6 Umat-Ku ibarat domba yang telah dibiarkan tersesat di pegunungan oleh gembala-gembalanya. Mereka mengembara dari satu gunung ke gunung yang lain dan tidak tahu lagi di mana rumah mereka.
\par 7 Mereka diserang dan disiksa oleh semua yang bertemu dengan mereka. Musuh-musuh umat-Ku berkata, 'Apa yang kita lakukan, tidak salah, sebab orang-orang itu telah berdosa kepada TUHAN. Leluhur mereka percaya kepada TUHAN, jadi seharusnya mereka pun tetap setia kepada-Nya.'
\par 8 Hai umat Israel, larilah dari Babel! Kamulah yang mula-mula harus pergi, supaya yang lain menyusul.
\par 9 Aku akan mengerahkan sejumlah bangsa yang kuat-kuat dari utara supaya mereka menyerang Babel. Mereka akan mengatur barisan untuk bertempur melawan Babel dan mengalahkannya. Mereka adalah pemanah-pemanah ahli yang panahnya tak pernah meleset.
\par 10 Babel akan dirampasi, dan perampasnya akan mengambil segalanya dengan sesuka hati. Aku, TUHAN, telah berbicara."
\par 11 TUHAN berkata, "Hai orang Babel, kamu merampok umat-Ku. Kamu melompat-lompat gembira seperti kuda yang meringkik, dan seperti sapi yang sedang merumput.
\par 12 Tetapi kamu akan menjadi bangsa yang paling tak berarti di antara segala bangsa. Kotamu sendiri yang besar itu akan dihina dan dipermalukan. Negerimu akan menjadi padang gurun yang tandus dan kering.
\par 13 Karena Aku marah, maka Babel akan menjadi reruntuhan dan tak berpenghuni. Semua yang lewat di situ akan terkejut dan ngeri.
\par 14 Hai kamu pemanah-pemanah! Aturlah barisanmu untuk mengepung dan menyerang Babel. Bidikkan semua anak panahmu ke arah Babel, karena ia telah berdosa kepada-Ku.
\par 15 Teriakkanlah pekik pertempuran di sekeliling kota itu! Sekarang Babel sudah menyerah. Tembok-temboknya telah didobrak dan diruntuhkan. Aku sedang melaksanakan pembalasan terhadap Babel. Sebab itu perlakukanlah mereka seperti mereka memperlakukan orang-orang lain.
\par 16 Jangan biarkan orang bercocok tanam atau menuai di negeri itu. Semua orang asing yang tinggal di situ akan pulang ke negerinya, karena mereka takut kepada tentara yang menyerang."
\par 17 TUHAN berkata, "Orang Israel seperti kawanan domba yang dikejar dan diceraiberaikan oleh singa. Mula-mula mereka diserang oleh raja Asyur, lalu Nebukadnezar raja Babel menggerogoti tulang-tulang mereka.
\par 18 Karena itu, Aku, TUHAN Yang Mahakuasa, Allah Israel, akan menghukum Raja Nebukadnezar dan negerinya sama seperti Kuhukum raja Asyur.
\par 19 Aku akan mengembalikan orang Israel ke negeri mereka. Mereka akan makan dari hasil tanah Gunung Karmel dan daerah Basan. Mereka akan dikenyangkan oleh hasil tanah daerah Efraim dan Gilead.
\par 20 Pada waktu itu Israel dan Yehuda akan bersih dari dosa, karena Aku akan mengampuni orang-orang yang telah Kuselamatkan. Aku, TUHAN telah berbicara."
\par 21 TUHAN berkata kepada suatu bangsa dari utara, "Seranglah penduduk Merataim dan Pekod. Bunuh dan binasakan mereka. Laksanakanlah perintah-Ku. Aku, TUHAN telah berbicara."
\par 22 Bunyi pertempuran bergemuruh di negeri, dan terjadilah kehancuran besar.
\par 23 Seluruh dunia dipalu oleh Babel sampai hancur, tapi sekarang palu itu sendiri telah patah! Segala bangsa terkejut mendengar apa yang telah terjadi dengan negeri itu.
\par 24 TUHAN berkata, "Babel, engkau melawan Aku, sebab itu kau terjebak di dalam jerat yang Kupasang untukmu, tetapi kau tidak menyadarinya.
\par 25 Gudang senjata-Ku telah Kubuka, dan dengan marah Kukeluarkan senjata-senjata itu, karena Aku TUHAN Yang Mahatinggi dan Mahakuasa harus melakukan suatu tugas di Babel.
\par 26 Seranglah Babel dari segala jurusan, dan dobraklah gudang-gudang gandumnya! Tumpuklah barang-barang rampasan seperti kamu menumpuk gandum! Musnahkan negeri itu! Jangan ada yang disisakan!
\par 27 Bunuhlah semua tentaranya! Tewaskan mereka! Celakalah bangsa Babel! Sudah tiba waktunya mereka dihukum!"
\par 28 (Orang-orang yang lari ke Yerusalem dari Babel menceritakan bagaimana TUHAN Allah kita membalas perbuatan orang Babel terhadap Rumah TUHAN.)
\par 29 TUHAN berkata, "Suruhlah para pemanah menyerang Babel. Kerahkanlah setiap orang yang pandai memanah. Kepunglah kota itu dan jangan biarkan seorang pun lolos. Balaslah semua perbuatannya dan perlakukanlah dia setimpal dengan kelakuannya, sebab ia telah bertindak kurang ajar terhadap Aku, Yang Mahasuci, Allah Israel.
\par 30 Pada hari itu pemuda-pemudanya akan tewas di jalan-jalan kotanya, dan seluruh tentaranya dihancurkan. Aku, TUHAN, telah berbicara.
\par 31 Babel, engkau terlalu tinggi hati! Karena itu Aku, TUHAN Yang Mahatinggi, Allah Yang Mahakuasa, melawan engkau. Sudah waktunya engkau Kuhukum.
\par 32 Bangsamu yang tinggi hati itu akan tersandung dan jatuh. Tak seorang pun akan menolong engkau untuk bangkit. Kota-kotamu akan Kubakar, dan segala yang di sekitarnya akan dimusnahkan."
\par 33 TUHAN Yang Mahakuasa berkata, "Orang Israel dan orang Yehuda tertekan. Semua yang menawan mereka menjaga mereka ketat-ketat dan tak mau melepaskan mereka.
\par 34 Tetapi Aku penyelamat mereka itu kuat, nama-Ku TUHAN Yang Mahakuasa. Aku sendirilah yang akan memperjuangkan perkara mereka dan membawa damai ke atas bumi. Tetapi ke atas orang Babel akan Kudatangkan kerusuhan dan ketakutan."
\par 35 TUHAN berkata, "Binasalah Babel bersama rakyat dan pejabat pemerintah serta kaum cerdik pandai mereka!
\par 36 Binasalah nabi-nabi Babel! Mereka pendusta dan dungu. Binasalah tentara Babel yang perkasa! Betapa takutnya mereka!
\par 37 Hancurkan kuda dan kereta perangnya! Binasalah prajurit-prajurit sewaannya! Betapa lemahnya mereka! Musnahkan kekayaan Babel! Jarahilah harta bendanya!
\par 38 Keringkanlah segala sungai dan ladangnya! Sebab, Babel penuh dengan berhala-berhala yang mengerikan, yang membuat para pemujanya menjadi gila.
\par 39 Karena itu Babel akan menjadi tempat jin-jin dan roh-roh jahat serta burung-burung unta. Untuk selama-lamanya tidak akan ada orang yang mau tinggal lagi di sana.
\par 40 Sebagaimana Aku memusnahkan Sodom dan Gomora bersama desa-desa di sekitarnya, begitu juga Aku akan memusnahkan Babel. Tak seorang pun akan tinggal lagi di sana. Aku, TUHAN, telah berbicara.
\par 41 Suatu bangsa yang kuat sedang bergerak dari negeri yang jauh di utara. Mereka datang bersama banyak raja untuk berperang.
\par 42 Mereka bersenjatakan panah dan tombak; mereka bengis dan tak kenal ampun. Seperti bunyi laut bergelora begitulah suara derap kuda mereka yang sedang dipacu untuk maju menyerang Babel.
\par 43 Mendengar berita itu, raja Babel menjadi tak berdaya. Ia dicekam perasaan takut, dan menderita seperti wanita yang mau melahirkan.
\par 44 Seperti singa muncul dari hutan lebat dekat Sungai Yordan dan mendatangi padang tempat domba merumput, demikianlah Aku, TUHAN, akan datang dan membuat orang Babel lari dari kota mereka dengan tiba-tiba. Lalu Aku akan memilih seorang pemimpin untuk memerintah bangsa itu. Siapakah dapat disamakan dengan Aku? Siapakah berani membuat perkara dengan Aku? Apakah ada pemimpin yang dapat melawan Aku?
\par 45 Karena itu, dengarkanlah apa yang telah Kurencanakan terhadap kota Babel, dan apa yang hendak Kulakukan terhadap penduduknya. Anak-anak mereka pun akan diseret pergi, dan semua orang akan ketakutan.
\par 46 Apabila Babel jatuh, akan terdengar keributan yang begitu hebat sehingga seluruh dunia goncang; teriakan-teriakan penduduknya akan terdengar oleh bangsa-bangsa lain."

\chapter{51}

\par 1 TUHAN berkata, "Aku akan mengirim angin perusak yang bertiup ke arah Babel dan ke arah penduduknya.
\par 2 Akan Kuutus orang-orang asing yang akan menghancurkan Babel seperti angin menerbangkan kulit gandum. Pada hari malapetaka itu mereka akan menyerang negeri itu dari segala pihak dan menyapunya bersih.
\par 3 Janganlah beri kesempatan kepada tentaranya untuk membidikkan anak panahnya atau membanggakan pakaian tempurnya. Anak-anak mudanya jangan dibiarkan hidup. Hancurkanlah seluruh tentaranya!
\par 4 Mereka akan luka parah dan tewas di jalan-jalan kota mereka.
\par 5 Aku, TUHAN, Allah Yang Mahakuasa, Allah Kudus Israel, tidak akan meninggalkan Israel dan Yehuda, sekalipun mereka telah berdosa kepada-Ku.
\par 6 Selamatkanlah dirimu! Larilah dari Babel, jangan sampai kamu ikut terbunuh karena dosanya! Sebab inilah saatnya Aku membalas kejahatan Babel, dan menghukum dia setimpal dengan perbuatannya.
\par 7 Babel tadinya seperti gelas emas yang Kupegang dan yang membuat seluruh dunia mabuk. Bangsa-bangsa minum anggur dari gelas itu, sehingga mereka menjadi gila.
\par 8 Tetapi tiba-tiba Babel jatuh dan pecah. Tangisilah dia, dan carilah obat untuk luka-lukanya, barangkali ia dapat sembuh.
\par 9 Orang asing yang tinggal di sana berkata satu sama lain, 'Kita sudah berusaha menolong Babel, tapi terlambat! Lebih baik kita meninggalkan negeri ini dan pulang ke negeri kita masing-masing. TUHAN telah menghukum Babel dengan keras, dan menghancurkannya sama sekali.'"
\par 10 Umat TUHAN berkata, "TUHAN sudah menunjukkan bahwa kita ada di pihak yang benar. Marilah ke Yerusalem dan menceritakan di sana apa yang telah dilakukan oleh TUHAN Allah kita."
\par 11 TUHAN telah menghasut raja-raja Media, karena Ia hendak menghancurkan Babel. Itulah cara-Nya Ia membalas perbuatan orang-orang yang menghancurkan rumah-Nya. Para perwira pasukan tempur memerintahkan, "Tajamkan anak panah! Siapkan perisai!
\par 12 Berilah tanda untuk menyerang tembok-tembok Babel. Perkuatlah penjagaan. Tempatkan pengawal. Siapkan penyergapan!" TUHAN telah melaksanakan apa yang direncanakan-Nya terhadap orang Babel.
\par 13 Negeri itu kaya sekali dan banyak sungainya, tapi masa hidupnya sudah habis; akhir hidupnya sudah tiba.
\par 14 TUHAN Yang Mahakuasa bersumpah demi diri-Nya sendiri bahwa Ia akan mengirim orang sebanyak belalang untuk menyerang Babel, dan mereka akan meneriakkan sorak kemenangan.
\par 15 TUHAN menciptakan bumi dengan kuasa-Nya, membentuk dunia dengan hikmat-Nya, dan membentangkan langit dengan akal budi-Nya.
\par 16 Hanya dengan memberi perintah, menderulah air di cakrawala. Dari ujung-ujung bumi didatangkan-Nya awan, dan dibuat-Nya kilat memancar di dalam hujan, serta dikirim-Nya angin dari tempat penyimpanannya.
\par 17 Melihat semua itu sadarlah manusia, bahwa ia bodoh dan tak punya pengertian. Para pandai emas kehilangan muka, sebab patung berhala buatannya itu palsu dan tak bernyawa.
\par 18 Berhala-berhala itu tak berharga, patut diejek dan dihina. Apabila tiba waktunya, mereka akan binasa.
\par 19 Allah Yakub tidak seperti berhala-berhala itu; Ia pencipta segala sesuatu. Nama-Nya ialah TUHAN Yang Mahakuasa; Ia telah memilih Israel menjadi umat-Nya.
\par 20 TUHAN berkata, "Babel, kau bagaikan palu, senjata-Ku untuk berperang. Engkau Kupakai untuk menghantam bangsa-bangsa dan kerajaan-kerajaan.
\par 21 Engkau Kupakai untuk menghancurkan kuda dan penunggangnya, serta kereta dan pengendaranya.
\par 22 Engkau Kupakai untuk membunuh pria dan wanita, tua maupun muda, gadis dan jejaka.
\par 23 Engkau Kupakai untuk membinasakan gembala dan ternaknya, petani dan lembu pembajaknya, penguasa dan perwira-perwira tinggi."
\par 24 TUHAN berkata, "Saksikanlah sendiri bagaimana Aku sekarang membalas kepada Babel dan penduduknya segala kejahatan yang mereka lakukan terhadap Yerusalem.
\par 25 Hai Babel, engkau seperti gunung yang menghancurkan seluruh dunia. Tapi Aku, TUHAN, akan melawan engkau. Engkau akan Kutarik dan Kugulingkan serta Kubiarkan terbakar menjadi abu.
\par 26 Tidak ada satu batu pun dari reruntuhanmu yang akan dipakai lagi untuk membangun. Engkau akan menjadi seperti padang gurun untuk selama-lamanya. Aku, TUHAN, telah berbicara."
\par 27 "Hai raja dari utara! Berilah tanda untuk menyerang! Bunyikan trompet supaya bangsa-bangsa mendengar! Kerahkan bangsa-bangsa untuk berperang melawan Babel! Suruhlah kerajaan Ararat, Mini dan Askenas menyerang. Angkatlah seorang panglima untuk memimpin penyerbuan itu. Kerahkanlah kuda sebanyak kumpulan belalang.
\par 28 Siapkan bangsa-bangsa untuk berperang melawan Babel. Panggil raja-raja Media bersama penguasa-penguasa dan pemuka-pemukanya serta tentara dari semua negeri yang mereka kuasai.
\par 29 Bumi bergetar dan berguncang karena Aku sedang melaksanakan rencana-Ku untuk menjadikan Babel tempat tandus yang tidak didiami manusia.
\par 30 Tentara Babel sudah berhenti berperang dan tinggal di benteng-benteng. Mereka hilang keberanian dan tidak berdaya. Pintu-pintu gerbang kota sudah didobrak, dan rumah-rumah dibakar.
\par 31 Utusan-utusan berlari susul-menyusul untuk memberitahukan kepada raja Babel bahwa kotanya sudah diserbu dari segala jurusan.
\par 32 Tempat penyeberangan telah diduduki musuh, dan benteng-benteng pertahanan dibakar. Tentara Babel menjadi panik.
\par 33 Tak lama lagi mereka dikalahkan dan diinjak-injak oleh musuh seperti gandum di tempat pengirikan. Aku, TUHAN Yang Mahakuasa, Allah Israel, telah berbicara."
\par 34 Yerusalem dihancurkan dan dimakan habis oleh raja Babel; ia mengosongkan kota itu seperti orang mengosongkan botol. Seperti seekor binatang raksasa, ia menelan Yerusalem dan mengisi perutnya dengan segala yang baik dari kota itu, lalu membuang sisanya.
\par 35 Penduduk Sion, katakanlah, "Semoga Babel tertimpa kekejaman yang dilakukannya terhadap kita!" Penduduk Yerusalem, katakanlah, "Semoga Babel tertimpa penderitaan yang ditimpakannya kepada kita!"
\par 36 Sebab itu TUHAN berkata kepada penduduk Yerusalem, "Aku akan memperjuangkan perkaramu dan membalas perbuatan musuh-musuhmu kepadamu. Sumber-sumber air dan sungai-sungai Babel akan Kukeringkan.
\par 37 Negeri itu akan menjadi timbunan puing, tempat bersembunyi anjing hutan. Orang merasa ngeri melihatnya, dan tak seorang pun mau tinggal di sana.
\par 38 Orang Babel mengaum seperti singa, dan menggeram seperti anak singa.
\par 39 Sementara nafsu makan mereka memuncak, Aku akan menyiapkan hidangan bagi mereka, dan membuat mereka mabuk dan pusing sampai tertidur dan tidak bangun-bangun lagi.
\par 40 Mereka akan Kubawa untuk disembelih seperti anak domba, kambing, dan domba jantan. Aku, TUHAN, telah berbicara."
\par 41 TUHAN berkata, "Babel yang dipuji di seluruh dunia telah direbut dan diduduki! Alangkah mengerikan negeri itu bagi bangsa-bangsa.
\par 42 Air laut meluap ke Babel; gelombang-gelombangnya menderu melanda negeri itu.
\par 43 Kota-kotanya menjadi tempat yang mengerikan, seperti padang gurun yang gersang. Tak ada orang yang mau tinggal atau lewat di situ.
\par 44 Bel, dewa negeri Babel, akan Kuhukum. Akan Kubuat dia mengembalikan apa yang telah dirampasnya. Bangsa-bangsa tidak akan menyembah dia lagi. Tembok-tembok Babel sudah runtuh.
\par 45 Sebab itu larilah dari sana, hai umat Israel! Selamatkan dirimu dari kemarahan-Ku yang meluap.
\par 46 Janganlah takut atau cemas karena desas-desus yang kamu dengar. Setiap tahun tersiar kabar yang berlainan--kabar tentang kekejaman di dalam negeri dan tentang raja-raja yang memerangi satu sama lain.
\par 47 Percayalah, saatnya akan tiba Aku menghukum berhala-berhala Babel. Seluruh negeri itu akan dihina, dan segenap penduduknya dibunuh.
\par 48 Segala sesuatu di langit dan di bumi akan bersorak gembira apabila Babel jatuh ke tangan bangsa dari utara yang datang untuk menghancurkannya.
\par 49 Banyak orang di seluruh dunia telah mati terbunuh karena Babel, tapi sekarang Babel akan jatuh demi orang Israel yang mati terbunuh. Aku, TUHAN, telah berbicara."
\par 50 TUHAN berkata kepada orang Israel di Babel, "Kamu sudah luput dari pembunuhan! Jadi, pergilah sekarang! Jangan menunggu! Sekalipun kamu jauh dari rumah, ingatlah kepada-Ku, Tuhanmu, dan kepada Yerusalem.
\par 51 Kamu berkata, 'Kami dihina dan dipermalukan; kami kehilangan muka karena orang-orang asing menduduki ruangan-ruangan suci di dalam Rumah TUHAN.'
\par 52 Nah, perhatikanlah perkataan-Ku ini. Akan datang saatnya Aku menghukum berhala-berhala Babel. Di seluruh negeri itu akan terdengar suara orang merintih karena luka parah.
\par 53 Sekalipun Babel dapat naik ke langit dan membangun pertahanan yang tak dapat dicapai, Aku tetap akan mengirim orang untuk merusaknya. Aku, TUHAN, telah berbicara."
\par 54 TUHAN berkata, "Dengarkan! Di Babel orang menjerit dan menangis sedih karena kehancuran yang terjadi di negeri itu.
\par 55 Aku sedang menghancurkan Babel dan menghentikan suara-suara keramaiannya. Musuh datang seperti gelombang menderu, dan tentaranya menyerbu dengan pekik-pekik perang.
\par 56 Mereka datang untuk merusak Babel, menangkap prajurit-prajuritnya dan mematahkan senjata-senjata mereka. Akulah TUHAN yang menghukum kejahatan, dan Babel akan Kubalas setimpal dengan perbuatannya.
\par 57 Aku akan memabukkan para pejabat pemerintah Babel--para cerdik pandai, pemimpin dan tentara. Mereka tidak akan bangun-bangun lagi, tertidur untuk selama-lamanya. Aku telah berbicara; Akulah raja, Aku TUHAN Yang Mahakuasa.
\par 58 Tembok Babel yang kuat-kuat akan Kuhancurkan sehingga menjadi serata dengan tanah. Pintu gerbangnya yang tinggi-tinggi akan Kubakar habis. Percuma saja jerih payah bangsa-bangsa, sebab semua hasil pekerjaan mereka akan dimakan api. Aku, TUHAN Yang Mahakuasa, telah berbicara."
\par 59 Pengawal pribadi raja Zedekia adalah Seraya anak Neria cucu Mahseya. Pada tahun keempat pemerintahan Zedekia raja Yehuda, Seraya mengikuti Zedekia ke Babel. Maka aku memberikan kepadanya suatu tugas.
\par 60 Semua malapetaka yang akan menimpa Babel dan hal-hal lain mengenai negeri itu telah kutulis dalam sebuah buku gulungan.
\par 61 Lalu aku berkata kepada Seraya, "Setelah engkau sampai di Babel, usahakanlah supaya semua yang tertulis dalam buku ini dapat kaubacakan kepada orang-orang di sana.
\par 62 Sesudah itu berdoalah begini, 'TUHAN, Engkau sudah berkata bahwa kota ini akan Kauhancurkan sehingga baik manusia maupun binatang lenyap sama sekali, dan tempat ini menjadi seperti padang gurun untuk selama-lamanya.'
\par 63 Nah, setelah kaubacakan isi buku ini kepada orang-orang itu, ikatkanlah batu pada buku ini, lalu lemparkanlah ke tengah-tengah Sungai Efrat,
\par 64 sambil berkata, 'Beginilah akan terjadi dengan Babel. Negeri ini akan tenggelam dan tidak timbul lagi karena malapetaka yang ditimpakan TUHAN ke atasnya.'" Kata-kata Yeremia berakhir di sini.

\chapter{52}

\par 1 Zedekia berumur dua puluh satu tahun ketika ia menjadi raja Yehuda. Ia memerintah di Yerusalem sebelas tahun lamanya. Ibunya bernama Hamutal, anak Yeremia dari kota Libna.
\par 2 Seperti Raja Yoyakim, Raja Zedekia juga berdosa kepada TUHAN.
\par 3 TUHAN marah sekali kepada penduduk Yerusalem dan orang Yehuda sehingga Ia mengusir mereka dan tidak melindungi mereka lagi. Zedekia memberontak terhadap Nebukadnezar raja Babel,
\par 4 karena itu Nebukadnezar dengan seluruh angkatan perangnya datang dan menyerang Yerusalem pada tanggal sepuluh bulan sepuluh dalam tahun kesembilan pemerintahan Zedekia. Mereka mendirikan markas di luar kota, membangun tembok pengepungan
\par 5 dan terus mengepung kota itu sampai tahun kesebelas pemerintahan Zedekia.
\par 6 Pada tanggal sembilan bulan empat tahun itu juga, ketika bencana kelaparan sudah begitu hebat sehingga rakyat tidak lagi mempunyai makanan sama sekali,
\par 7 tembok kota didobrak musuh. Malam itu, meskipun orang Babel sedang mengepung kota itu, semua tentara Yehuda melarikan diri menuju Lembah Yordan. Mereka mengambil jalan lewat taman istana, lalu keluar melalui pintu gerbang yang menghubungkan kedua tembok di tempat itu.
\par 8 Tetapi tentara Babel mengejar Raja Zedekia, dan menangkapnya di dataran Yerikho. Semua anak buahnya lari meninggalkan dia.
\par 9 Kemudian Raja Zedekia dibawa kepada Raja Nebukadnezar di kota Ribla di daerah Hamat, lalu dijatuhi hukuman mati.
\par 10 Di kota itu juga anak-anaknya dibunuh di depan matanya, dan semua pejabat pemerintah Yehuda pun dibunuh.
\par 11 Setelah itu Zedekia dicungkil matanya, lalu dibelenggu dan dibawa ke Babel. Di sana ia dimasukkan ke dalam penjara, dan tidak pernah keluar dari situ sampai saat ia meninggal.
\par 12 Pada tanggal sepuluh bulan lima dalam tahun kesembilan belas pemerintahan Nebukadnezar raja Babel, datanglah ke Yerusalem seorang yang bernama Nebuzaradan. Ia adalah penasihat dan kepala pengawal pribadi Nebukadnezar.
\par 13 Nebuzaradan membakar Rumah TUHAN, istana raja, dan rumah semua orang terkemuka di Yerusalem.
\par 14 Lalu semua anak buahnya meruntuhkan tembok-tembok kota itu.
\par 15 Kemudian rakyat yang masih ada di kota, yaitu para pekerja ahli yang tersisa, orang-orang miskin, dan orang-orang yang telah lari ke pihak orang Babel, semuanya diangkut ke Babel oleh Nebuzaradan.
\par 16 Tapi sebagian dari rakyat yang paling miskin dan tak punya harta, ditinggalkannya di Yehuda, dan disuruhnya mengerjakan kebun anggur dan ladang-ladang.
\par 17 Tiang-tiang perunggu dan kereta-kereta di Rumah TUHAN bersama bejana perunggu dipecahkan oleh orang Babel, lalu semua perunggunya diangkut ke negeri mereka.
\par 18 Mereka juga mengangkut barang-barang ini: penyodok-penyodok, tempat abu mezbah, perkakas-perkakas pelita, mangkuk-mangkuk untuk menampung darah binatang yang disembelih untuk kurban, mangkuk-mangkuk untuk membakar dupa, dan semua barang perunggu lainnya yang dipakai untuk upacara ibadat.
\par 19 Barang-barang emas dan perak, termasuk mangkuk-mangkuk kecil dan tempat bara, mangkuk-mangkuk untuk tempat darah binatang yang disembelih untuk kurban, tempat abu, kaki pelita, mangkuk-mangkuk untuk dupa, dan mangkuk-mangkuk untuk menuangkan persembahan air anggur, semuanya diambil oleh Nebuzaradan sendiri.
\par 20 Barang-barang perunggu yang dibuat oleh Raja Salomo untuk Rumah TUHAN, yaitu kedua tiang, kereta-kereta, bejana dan kedua belas sapi yang menopang bejana itu, semuanya begitu berat, sehingga tidak tertimbang.
\par 21 Kedua tiang perunggu itu sama: masing-masing tingginya 8 meter, lingkarnya 5,3 meter, dan tengah-tengahnya kosong. Tebal perunggunya 75 milimeter. Di atas setiap tiang itu ada kepala tiang setinggi 2,2 meter. Sekeliling kepala-kepala tiang itu ada anyaman dengan hiasan buah delima, semuanya dari perunggu juga.
\par 23 Buah delima pada anyaman di setiap kepala tiang itu ada seratus, tapi hanya sembilan puluh enam yang kelihatan dari bawah.
\par 24 Selain itu, Nebuzaradan menangkap Imam Agung Seraya, Zefanya wakilnya, tiga orang penjaga pintu Rumah TUHAN,
\par 25 ketujuh penasihat pribadi raja yang masih berada di kota, panglima, wakil panglima yang mengurus administrasi tentara, dan enam puluh orang pembesar lainnya.
\par 26 Semua orang itu dibawanya kepada Raja Nebukadnezar di kota Ribla,
\par 27 di wilayah Hamat. Di sana mereka disiksa, lalu dibunuh. Demikianlah orang-orang Yehuda diangkut dari negeri mereka dan dibawa ke pembuangan;
\par 28 semuanya ada 4.600 orang. Nebukadnezar mengangkut mereka dalam tahap-tahap sebagai berikut: 3.023 orang pada tahun ketujuh pemerintahannya, 832 orang dari Yerusalem pada tahun kedelapan belas pemerintahannya, dan 745 orang diangkut oleh Nebuzaradan pada tahun kedua puluh tiga pemerintahan Nebukadnezar.
\par 31 Pada tanggal dua puluh lima bulan dua belas dalam tahun ketiga puluh tujuh sesudah Yoyakhin diangkut ke pembuangan, Ewil-Merodakh menjadi raja Babel. Pada tahun itu juga ia menunjukkan belas kasihannya kepada Yoyakhin, dan melepaskan dia dari penjara,
\par 32 serta memperlakukannya dengan baik. Yoyakhin diberinya kedudukan yang lebih tinggi daripada raja-raja lain yang juga dibuang ke Babel.
\par 33 Ia diizinkan mengganti pakaian penjaranya dan untuk seterusnya selama hidupnya ia boleh makan di istana, dan setiap hari diberi uang untuk keperluan hidupnya.


\end{document}