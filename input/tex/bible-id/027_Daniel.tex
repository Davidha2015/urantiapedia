\begin{document}

\title{Daniel}

\chapter{1}

\par 1 Tiga tahun setelah Raja Yoyakim memerintah Yehuda, Nebukadnezar raja Babel, menyerbu ke kota Yerusalem dan mengepungnya.
\par 2 TUHAN mengizinkan dia mengalahkan Raja Yoyakim dan merampas sebagian dari barang-barang berharga di Rumah TUHAN. Orang-orang yang ditawannya, dibawanya ke kuil dewanya di Babel, dan barang-barang berharga yang dirampasnya, disimpannya di dalam gudang-gudang kuil itu.
\par 3 Kemudian raja memerintahkan Aspenas, kepala rumah tangga istana, untuk memilih dari antara tawanan-tawanan Israel beberapa pemuda keturunan raja atau berdarah bangsawan.
\par 4 Mereka harus pemuda-pemuda yang tak bercacat. Mereka harus tampan, berpendidikan tinggi, cerdas dan berpengetahuan luas serta patut untuk bertugas di dalam istana. Aspenas harus mengajar mereka membaca dan menulis bahasa Babel.
\par 5 Raja menentukan juga bahwa makanan dan minuman untuk mereka harus sama dengan yang disiapkan bagi anggota keluarga raja. Setelah dilatih selama tiga tahun, mereka harus menghadap raja untuk bertugas di istana.
\par 6 Di antara pemuda-pemuda itu terdapat juga Daniel, Hananya, Misael dan Azarya, semuanya dari suku Yehuda.
\par 7 Kepala rumah tangga istana mengganti nama-nama mereka menjadi: Beltsazar, Sadrakh, Mesakh dan Abednego.
\par 8 Dengan pertolongan Allah, Daniel disayang dan dikasihani oleh Aspenas. Daniel bertekad untuk tidak menajiskan dirinya dengan makanan dan minuman anggur dari istana raja, sebab itu ia minta kepada Aspenas supaya boleh mendapat makanan lain.
\par 10 Karena takut kepada raja, Aspenas berkata, "Raja sendiri telah menetapkan makanan dan minumanmu, jadi jika menurut pendapatnya engkau kelihatan kurang sehat daripada pemuda-pemuda yang lain, pasti aku akan dibunuhnya."
\par 11 Kemudian Daniel merundingkan hal itu dengan pengawal yang ditugaskan oleh Aspenas untuk mengurus Daniel dan ketiga kawannya. Kata Daniel,
\par 12 "Ujilah kami selama sepuluh hari; berilah kami hanya sayuran dan air untuk makanan dan minuman.
\par 13 Setelah sepuluh hari, bandingkanlah rupa kami dengan rupa pemuda-pemuda yang makan makanan yang ditetapkan oleh raja, lalu ambillah keputusan berdasarkan pengamatanmu itu."
\par 14 Pengawal itu setuju dan mengadakan percobaan itu selama sepuluh hari.
\par 15 Setelah waktu itu habis, mereka kelihatan lebih sehat dan kuat daripada semua pemuda yang telah mendapat makanan dari meja raja.
\par 16 Sejak itu pengawal itu tidak lagi menghidangkan kepada mereka makanan dan minuman yang ditetapkan oleh raja, melainkan hanya sayuran dan air saja.
\par 17 Allah memberikan kepada keempat pemuda itu hikmat dan keahlian dalam kesusasteraan dan ilmu. Selain itu kepada Daniel diberikan-Nya juga kepandaian untuk menerangkan penglihatan dan mimpi.
\par 18 Pada akhir tiga tahun masa pendidikan yang ditentukan oleh raja itu, Aspenas membawa semua pemuda itu menghadap Raja Nebukadnezar.
\par 19 Setelah berwawancara dengan mereka semua, raja mendapati bahwa Daniel, Hananya, Misael dan Azarya melebihi yang lain-lainnya dalam segala bidang. Maka mulailah mereka bertugas dalam istana raja.
\par 20 Tiap kali raja mengemukakan persoalan yang memerlukan penerangan dan pertimbangan, ia melihat bahwa nasihat dan tanggapan keempat pemuda itu sepuluh kali lebih baik daripada nasihat dan tanggapan semua peramal serta ahli jampi di seluruh kerajaannya.
\par 21 Daniel tetap bertugas di istana sampai Raja Koresh dari Persia menaklukkan Babel.

\chapter{2}

\par 1 Pada suatu malam, dua tahun setelah Raja Nebukadnezar memerintah, ia bermimpi. Mimpinya itu begitu menggelisahkan hatinya, sehingga ia tidak dapat tidur lagi.
\par 2 Karena itu dipanggilnya para peramal, ahli jampi, dukun dan orang-orang berilmu, untuk menerangkan mimpinya itu.
\par 3 Ketika mereka berdiri di hadapannya, raja berkata, "Aku bermimpi, dan hatiku gelisah karena aku ingin tahu artinya!"
\par 4 Mereka menjawab dalam bahasa Aram, "Hiduplah Tuanku selama-lamanya! Ceritakanlah mimpi itu kepada kami, maka kami akan menerangkan artinya."
\par 5 Raja menjawab, "Aku sudah mengambil keputusan ini: Kamu harus memberitahukan mimpi itu kepadaku, begitu pula artinya. Jika kamu tidak sanggup, kamu akan dipotong-potong dan rumahmu akan dirobohkan.
\par 6 Tetapi jika kamu sanggup, kamu akan kuberikan hadiah yang indah-indah serta kehormatan yang besar. Nah, beritahukanlah sekarang mimpiku itu dengan artinya!"
\par 7 Mereka menjawab lagi, "Hendaknya Tuanku memberitahukan dulu mimpi itu kepada kami, setelah itu kami akan menerangkan artinya."
\par 8 Mendengar itu raja menjawab, "Tepat seperti dugaanku. Kamu hanya mengulur-ulur waktu saja, karena kamu tahu bahwa aku bermaksud
\par 9 untuk menghukum kamu semua dengan cara yang sama jika kamu tidak sanggup memberitahukan mimpi itu kepadaku. Kamu bermufakat untuk terus membohongi dan mempermainkan aku dengan harapan bahwa aku akan mengubah keputusanku. Nah, beritahukan mimpi itu, maka aku akan tahu bahwa kamu dapat juga menerangkan artinya."
\par 10 Para ahli itu menjawab, "Di seluruh dunia tidak ada seorang pun yang dapat memberitahukan apa yang Tuanku kehendaki itu. Dan belum pernah seorang raja, betapa pun besar dan mulianya, menuntut hal seperti itu dari para peramal, ahli jampi, dan orang-orang berilmu di istananya.
\par 11 Tuanku menginginkan sesuatu yang terlalu sulit. Tak seorang pun dapat memenuhi kecuali para dewa, tetapi mereka tidak diam di antara manusia."
\par 12 Mendengar itu raja menjadi marah sekali dan memberi perintah untuk membunuh semua para cerdik pandai yang ada di Babel.
\par 13 Lalu perintah itu diumumkan, dan semua cerdik pandai, termasuk Daniel dan teman-temannya dicari untuk dibunuh.
\par 14 Maka Daniel menemui Ariokh, pemimpin pengawal raja, yang telah siap melaksanakan hukuman mati itu. Dengan hati-hati
\par 15 Daniel bertanya kepadanya, mengapa raja telah mengeluarkan perintah yang keras itu. Ariokh memberitahukan kepada Daniel apa yang telah terjadi.
\par 16 Dengan segera Daniel menghadap raja dan memohon diberi waktu untuk menerangkan arti mimpi itu.
\par 17 Permintaannya dikabulkan, lalu Daniel pulang ke rumah dan memberitahukan segala kejadian itu kepada Hananya, Misael dan Azarya.
\par 18 Disuruhnya mereka berdoa kepada Allah di surga supaya Ia berbelaskasihan kepada mereka dan mengungkapkan rahasia mimpi itu. Dengan demikian mereka tidak akan dihukum mati bersama para cerdik pandai yang lain.
\par 19 Pada malam itu juga Allah memberitahukan rahasia itu kepada Daniel dalam suatu penglihatan. Maka Daniel memuji Allah di surga,
\par 20 katanya, "Allah itu bijaksana dan perkasa, terpujilah Dia selama-lamanya!
\par 21 Dialah yang menetapkan musim dan masa, Dia menggulingkan dan melantik penguasa. Dialah pula yang memberi kebijaksanaan, serta menganugerahkan pengertian.
\par 22 Dialah yang mengungkapkan rahasia yang amat dalam. Dia tahu segala yang terjadi di dalam kelam. Ia diliputi oleh cahaya terang.
\par 23 Ya, Allah pujaan leluhurku, Engkau kupuji dan kumuliakan. Sebab Kauberikan kepadaku hikmat dan kekuatan. Permohonan kami telah Kaukabulkan. Dan jawaban bagi raja telah Kautunjukkan."
\par 24 Setelah itu Daniel pergi kepada Ariokh yang telah ditugaskan untuk membunuh para cerdik pandai di Babel. Kata Daniel kepadanya, "Jangan bunuh para cerdik pandai itu. Bawalah aku menghadap raja, sebab aku mau memberitahukan kepada baginda arti mimpinya itu."
\par 25 Dengan segera Ariokh membawa Daniel menghadap Raja Nebukadnezar lalu berkata, "Hamba telah menemukan di antara para buangan Yahudi seorang yang dapat memberitahukan arti mimpi Tuanku."
\par 26 Maka berkatalah raja kepada Daniel yang juga disebut Beltsazar, "Dapatkah engkau memberitahu kepadaku mimpiku dan juga artinya?"
\par 27 Daniel menjawab, "Tuanku, tak ada orang berilmu, ahli jampi, peramal atau ahli perbintangan yang dapat memberitahukan hal itu kepada Tuanku.
\par 28 Tetapi di surga ada Allah yang menyingkapkan segala rahasia, dan Dia telah memberitahukan kepada Tuanku apa yang akan terjadi di kemudian hari. Sekarang, perkenankanlah hamba menerangkan mimpi dan penglihatan yang Tuanku terima waktu tidur itu.
\par 29 Ketika Tuanku sedang tidur, Tuanku bermimpi tentang zaman yang akan datang. Allah yang menyingkapkan segala rahasia memperlihatkan kepada Tuanku apa yang akan terjadi.
\par 30 Rahasia ini telah dinyatakan kepada hamba bukan karena hamba lebih pandai daripada siapa pun juga, melainkan supaya Tuanku dapat mengetahui arti mimpi Tuanku itu dan mengerti pula tentang isi hati Tuanku.
\par 31 Inilah penglihatan yang Tuanku lihat: Di depan Tuanku ada sebuah patung! Patung itu amat besar, dan berkilau-kilauan tetapi rupanya sangat mengerikan.
\par 32 Kepalanya terbuat dari emas murni, dada dan lengannya dari perak, pinggang dan pinggulnya dari tembaga,
\par 33 pahanya dari besi dan kakinya sebagian dari besi dan sebagian lagi dari tanah liat.
\par 34 Ketika Tuanku sedang menatapnya, sebuah batu besar terlepas dari tebing, tanpa disentuh orang, lalu menimpa patung itu pada kakinya yang dari besi dan tanah liat itu, sehingga remuk.
\par 35 Pada saat itu juga seluruh patung itu roboh dan menjadi timbunan besi, tanah liat, tembaga, perak, dan emas yang telah hancur lebur seperti debu di tempat penebahan gandum di musim panas. Timbunan itu beterbangan ditiup angin, sehingga tak meninggalkan bekas apa pun. Tetapi batu yang merobohkan patung itu menjadi sebesar gunung dan memenuhi seluruh bumi.
\par 36 Itulah mimpi Tuanku, dan sekarang hamba akan menerangkan artinya.
\par 37 Yang mulia Tuanku adalah raja dari segala raja; Tuankulah yang paling berkuasa. Allah di surga telah memberikan kepada Tuanku kerajaan ini, kekuasaan, kekuatan dan keagungan.
\par 38 Tuanku dijadikan-Nya penguasa atas bagian dunia yang berpenduduk dan atas segala burung dan binatang lainnya. Tuanku sendiri adalah kepala emas pada patung itu.
\par 39 Tetapi setelah kerajaan Tuanku berakhir, akan muncul kerajaan lain yang tidak sebesar kerajaan Tuanku. Kemudian muncullah kerajaan ketiga yang dilambangkan oleh bagian dari tembaga pada patung itu, yang akan menguasai seluruh dunia.
\par 40 Berikutnya akan muncul kerajaan keempat, yang sekuat besi. Dan seperti besi yang meremukkan dan menghancurkan apa saja, begitu pula kerajaan yang keempat ini akan meremukkan dan menghancurkan kerajaan-kerajaan yang lain itu.
\par 41 Kaki dan jari-jari kaki patung yang sebagian dari tanah liat dan sebagian besi itu, menunjukkan bahwa di kemudian hari kerajaan itu akan terbagi-bagi. Kerajaan itu ada kekuatannya juga karena tanah liat itu ada campuran besinya.
\par 42 Jari-jari kaki yang sebagian dari besi dan sebagian dari tanah liat berarti bahwa ada bagian kerajaan yang kuat dan ada juga yang lemah.
\par 43 Campuran besi dengan tanah liat itu menandakan juga bahwa kerajaan itu akan berusaha memperkuat diri dengan mengadakan kawin campur, tetapi usaha itu sia-sia, seperti besi pun tidak dapat bersenyawa dengan tanah liat.
\par 44 Pada masa pemerintahan raja-raja itu, Allah di surga akan mendirikan sebuah kerajaan yang akan bertahan selama-lamanya dan yang tak akan dikalahkan oleh bangsa mana pun. Kerajaan itu akan menghancurleburkan segala kerajaan yang lain itu.
\par 45 Bukankah Tuanku telah melihat bahwa tanpa disentuh orang, sebuah batu terlepas dari tebing lalu menimpa dan meremukkan patung dari besi, tembaga, tanah liat, perak dan emas itu? Allah yang besar menyatakan kepada Tuanku apa yang kelak akan terjadi. Mimpi itu dapat dipercaya dan keterangan hamba tepat."
\par 46 Lalu sujudlah Raja Nebukadnezar di hadapan Daniel dan memberi perintah supaya Daniel dihormati dan kepadanya dipersembahkan kurban bakaran serta persembahan-persembahan yang lain.
\par 47 Raja berkata, "Sungguh, Daniel, Allahmu itu paling besar di antara segala Allah. Ia adalah penguasa atas segala raja, dan penyingkap segala rahasia. Aku tahu hal itu sebab engkau telah sanggup menerangkan arti rahasia ini."
\par 48 Kemudian Daniel diberinya kedudukan yang tinggi, dianugerahinya dengan banyak hadiah yang indah-indah, dan diangkatnya menjadi gubernur Babel, serta dijadikannya pemimpin tertinggi atas semua penasihat istana.
\par 49 Tetapi atas permintaan Daniel, raja menyerahkan pemerintahan provinsi Babel itu kepada Sadrakh, Mesakh dan Abednego, sedang Daniel sendiri tinggal di istana raja.

\chapter{3}

\par 1 Pada suatu waktu Raja Nebukadnezar membuat sebuah patung emas yang tingginya 27 meter dan lebarnya hampir 3 meter. Ia mendirikannya di dataran Dura di provinsi Babel.
\par 2 Kemudian raja mengundang semua raja wilayah, para gubernur, bupati, penasihat negara, bendahara, hakim, ahli hukum, dan semua kepala daerah untuk menghadiri upacara peresmian patung yang telah didirikannya itu.
\par 3 Setelah mereka semua datang dan berdiri di depan patung itu,
\par 4 berserulah ajudan raja dengan nyaring, "Saudara-saudara dari segala bangsa, suku bangsa dan bahasa! Dengarlah perintah raja ini:
\par 5 Jika trompet berbunyi, diikuti bunyi seruling, kecapi, rebab, gambus, serdam, dan alat-alat musik lainnya, saudara-saudara harus sujud menyembah patung emas yang telah didirikan oleh Raja Nebukadnezar.
\par 6 Barangsiapa tidak mentaati perintah ini, akan langsung dilemparkan ke dalam perapian yang menyala-nyala."
\par 7 Maka mendengar alat-alat musik itu dibunyikan, orang-orang dari segala bangsa, suku bangsa dan bahasa, sujud dan menyembah patung emas itu.
\par 8 Beberapa orang Babel memakai kesempatan itu untuk mencelakakan orang Yahudi.
\par 9 Mereka berkata kepada Raja Nebukadnezar, "Hiduplah Tuanku selama-lamanya!
\par 10 Tuanku sendiri telah mengeluarkan perintah bahwa segera setelah alat-alat musik dibunyikan, semua orang harus sujud dan menyembah patung emas itu,
\par 11 dan barangsiapa yang tidak mematuhi perintah itu akan dilemparkan ke dalam perapian yang menyala-nyala.
\par 12 Tetapi beberapa orang Yahudi yang telah Tuanku serahi pemerintahan provinsi Babel menganggap sepi perintah Tuanku itu. Mereka ialah Sadrakh, Mesakh dan Abednego. Orang-orang itu tidak mau memuja ilah-ilah Tuanku dan tidak pula menyembah patung emas yang Tuanku dirikan."
\par 13 Mendengar itu raja menjadi marah sekali, lalu memberi perintah supaya ketiga orang itu dibawa menghadap kepadanya.
\par 14 Raja bertanya kepada mereka, "Sadrakh, Mesakh dan Abednego! Betulkah kamu tidak mau menyembah ilah-ilahku dan tidak mau pula sujud kepada patung emas yang telah kudirikan itu?
\par 15 Nah, sekarang, bersediakah kamu untuk sujud dan menyembah patung itu pada waktu musik berbunyi? Jika kamu tidak mau, kamu akan langsung dilemparkan ke dalam perapian yang menyala-nyala. Dan dewa manakah yang akan sanggup menyelamatkan kamu dari kuasaku?"
\par 16 Sadrakh, Mesakh dan Abednego menjawab, "Baginda yang mulia, kami tidak akan mencoba membela diri.
\par 17 Jika Allah yang kami sembah sanggup menyelamatkan kami dari perapian yang menyala-nyala itu dan dari kuasa Tuanku, pasti Ia melakukannya.
\par 18 Tetapi seandainya Ia tidak melakukannya juga, hendaknya Tuanku maklum bahwa kami tidak akan memuja dewa Tuanku dan tidak pula menyembah patung emas yang Tuanku dirikan itu."
\par 19 Maka meluaplah amarah Raja Nebukadnezar terhadap Sadrakh, Mesakh dan Abednego, sehingga wajahnya menjadi merah padam. Ia memerintahkan supaya perapian dibuat tujuh kali lebih panas daripada biasanya.
\par 20 Lalu ia menyuruh beberapa orang yang sangat kuat dari tentaranya untuk mengikat Sadrakh, Mesakh dan Abednego, serta melemparkan mereka ke dalam perapian yang menyala itu.
\par 21 Segera ketiga orang itu pun diikat erat dalam keadaan berpakaian lengkap, yaitu dengan kemeja, jubah, topi dan pakaian lainnya, lalu dilemparkan ke dalam perapian yang menyala-nyala itu.
\par 22 Karena perintah raja itu begitu keras, maka perapian itu telah dipanaskan dengan luar biasa sehingga nyala api membakar mati orang-orang yang mengangkat Sadrakh, Mesakh dan Abednego ke dekat perapian.
\par 23 Demikianlah, ketiga orang yang terikat erat itu jatuh ke dalam perapian yang menyala-nyala itu.
\par 24 Tiba-tiba Raja Nebukadnezar sangat terkejut. Ia bangkit dengan cepat dan berseru kepada para pegawainya, "Bukankah kita tadi mengikat tiga orang dan melemparkan mereka ke dalam api itu?" Mereka menjawab, "Memang benar, Tuanku."
\par 25 Sahut raja, "Tetapi mengapa kulihat empat orang berjalan-jalan di tengah-tengah api itu? Mereka tidak terikat dan sama sekali tidak apa-apa. Dan yang keempat itu rupanya seperti dewa."
\par 26 Lalu Nebukadnezar mendekati pintu perapian itu dan berseru, "Sadrakh, Mesakh dan Abednego, hamba-hamba Allah Yang Mahatinggi! Keluarlah dari perapian itu!" Maka keluarlah mereka.
\par 27 Semua wakil raja, para gubernur, bupati, dan pegawai-pegawai lainnya mengelilingi ketiga orang itu dan melihat bahwa mereka sama sekali tak disentuh oleh api. Rambut mereka tidak hangus, dan pakaian mereka tidak gosong, bahkan bau asap pun tidak ada pada mereka.
\par 28 Lalu berkatalah raja, "Pujilah Allah yang disembah Sadrakh, Mesakh dan Abednego. Dia telah mengutus malaikat-Nya untuk menyelamatkan ketiga hamba-Nya yang percaya kepada-Nya. Mereka telah melanggar perintahku dan lebih suka mati daripada menyembah atau memuja dewa mana pun kecuali Allah mereka sendiri.
\par 29 Sebab itu aku memerintahkan bahwa setiap orang dari bangsa, suku bangsa atau bahasa mana pun, yang mengucapkan penghinaan terhadap Allah yang disembah Sadrakh, Mesakh dan Abednego, akan dipotong-potong dan rumahnya akan dirobohkan dan dijadikan timbunan puing. Sebab tidak ada dewa yang dapat melakukan apa yang telah dilakukan Allah itu."
\par 30 Setelah itu raja menaikkan pangkat Sadrakh, Mesakh dan Abednego, sehingga mereka menjadi pejabat-pejabat tinggi di provinsi Babel.

\chapter{4}

\par 1 Inilah pengumuman Raja Nebukadnezar yang dikirimnya kepada orang-orang dari segala bangsa, suku bangsa dan bahasa di seluruh dunia: "Salam sejahtera!
\par 2 Aku ingin menceritakan tentang keajaiban dan mujizat yang telah dilakukan Allah Yang Mahatinggi kepadaku.
\par 3 Alangkah besar mujizat yang diperbuat-Nya! Alangkah hebat keajaiban yang dilakukan-Nya! Allah akan menjadi raja untuk selamanya. Ia berkuasa sepanjang masa."
\par 4 Aku tinggal dengan sejahtera dalam istanaku dan hidup dengan mewah.
\par 5 Tetapi pada suatu malam ketika aku sedang tidur, aku bermimpi dan mendapat penglihatan yang sangat mencemaskan hatiku.
\par 6 Lalu kupanggil para cerdik pandai yang ada di Babel untuk menerangkan kepadaku arti mimpi itu.
\par 7 Tetapi ketika para peramal, ahli jampi, orang-orang berilmu dan para ahli perbintangan itu datang dan kuceritakan mimpiku itu kepada mereka, tidak seorang pun yang dapat menerangkan artinya.
\par 8 Akhirnya datanglah Daniel yang disebut juga Beltsazar seperti nama dewaku. Daniel dipenuhi oleh roh dewa-dewa yang suci. Kuceritakan mimpiku kepadanya juga, kataku,
\par 9 "Hai, Beltsazar, pemimpin orang-orang berilmu, aku tahu bahwa engkau dipenuhi oleh roh dewa-dewa yang suci, sehingga tak ada rahasia yang tersembunyi bagimu. Dengarlah mimpiku ini, dan terangkanlah artinya.
\par 10 Ketika aku sedang tidur, kulihat sebuah pohon yang tumbuh di tengah-tengah bumi.
\par 11 Pohon itu sangat tinggi; batangnya besar dan kuat. Puncaknya sampai ke langit, sehingga dapat dilihat oleh semua orang di bumi.
\par 12 Daun-daunnya segar dan buahnya lebat sekali, cukup untuk dimakan oleh penghuni seluruh dunia. Binatang-binatang liar berbaring di bawah naungannya dan burung-burung bersarang di dahan-dahannya. Dan segala makhluk hidup dapat makan buah-buahnya.
\par 13 Ketika aku sedang merenungkan penglihatan itu, kulihat seorang malaikat turun dari surga; ia tampak siaga dan waspada.
\par 14 Dengan nyaring ia berkata, 'Tebanglah pohon itu dan potonglah dahan-dahannya, gugurkanlah daun-daunnya dan hamburkanlah buahnya. Usirlah binatang-binatang yang berbaring di bawahnya dan burung-burung yang bersarang di dahan-dahannya.
\par 15 Tetapi biarkanlah tunggulnya tinggal di dalam tanah. Ikatlah dengan rantai besi dan tembaga lalu tinggalkanlah di padang rumput. Biarlah ia dibasahi embun, dan hidup bersama dengan binatang-binatang serta tumbuh-tumbuhan.
\par 16 Biarlah akal manusianya berubah menjadi akal binatang selama tujuh tahun.
\par 17 Itulah keputusan kami malaikat-malaikat yang waspada dan siaga. Maksud kami ialah supaya orang-orang di mana pun mengakui, bahwa Allah Yang Mahatinggi berkuasa atas kerajaan manusia dan kerajaan itu diberikan-Nya kepada siapa saja yang dipilih-Nya, bahkan kepada orang yang paling tidak berarti sekalipun!'
\par 18 Hai Beltsazar, itulah mimpiku. Sekarang katakanlah artinya kepadaku. Tak seorang pun dari para cerdik pandai yang ada di kerajaanku dapat mengatakannya kepadaku, tetapi engkau dapat, sebab engkau dipenuhi oleh roh dewa-dewa yang suci."
\par 19 Mendengar itu, Daniel yang disebut juga Beltsazar, terkejut dan tercengang untuk beberapa saat. Maka aku, Raja Nebukadnezar berkata kepadanya, "Beltsazar, janganlah engkau cemas karena mimpi dan artinya itu." Beltsazar menjawab, "Tuanku, alangkah baiknya seandainya kejadian-kejadian yang diramalkan oleh mimpi itu menimpa musuh-musuh Tuanku dan bukan Tuanku.
\par 20 Pohon yang Tuanku lihat itu begitu tinggi, sehingga puncaknya sampai ke langit, dan dapat dilihat oleh semua orang di bumi.
\par 21 Daun-daunnya segar dan buahnya lebat sehingga cukup untuk memberi makan seluruh penghuni dunia. Binatang-binatang liar berbaring di bawah naungannya, dan burung-burung bersarang di dahan-dahannya.
\par 22 Ya Tuanku, Tuankulah pohon yang tinggi dan kuat itu. Kebesaran Tuanku bertambah sampai ke langit dan kekuasaan Tuanku meluas sampai ke ujung bumi.
\par 23 Tuanku melihat juga seorang malaikat turun dari surga sambil berkata, 'Tebanglah pohon ini dan binasakanlah, tetapi biarkanlah tunggulnya tinggal di dalam tanah. Ikatlah dengan rantai besi dan tembaga lalu tinggalkanlah di padang rumput. Biarlah ia dibasahi embun dan hidup bersama dengan binatang-binatang selama tujuh tahun.'
\par 24 Tuanku, inilah arti penglihatan itu, dan inilah yang diputuskan Allah Yang Mahatinggi mengenai Tuanku sendiri.
\par 25 Tuanku akan diusir dari masyarakat manusia dan hidup di antara binatang-binatang di padang. Selama tujuh tahun Tuanku akan makan rumput seperti sapi dan tidur di lapangan terbuka, sehingga dibasahi embun. Setelah itu Tuanku akan mengakui bahwa Allah Yang Mahatinggi berkuasa atas kerajaan manusia, dan kerajaan itu diberikan-Nya kepada siapa saja yang dipilih-Nya.
\par 26 Para malaikat memerintahkan supaya tunggul pohon itu dibiarkan tinggal di dalam tanah. Itu berarti bahwa Tuanku akan menjadi raja lagi setelah Tuanku mengakui bahwa Allah menguasai seluruh dunia.
\par 27 Sebab itu hendaknya Tuanku menuruti nasihat hamba. Janganlah Tuanku berdosa lagi, tapi lakukanlah apa yang baik dan adil, dan kasihanilah orang miskin. Dengan demikian Tuanku akan tetap berbahagia."
\par 28 Semuanya itu terjadi atas diriku, Raja Nebukadnezar.
\par 29 Dua belas bulan kemudian, pada waktu aku berjalan-jalan di taman di tingkat atas gedung istanaku di Babel,
\par 30 aku berkata, "Lihat, alangkah hebatnya kota Babel! Akulah yang membangunnya menjadi ibukota negara untuk membuktikan kekuasaan dan kekuatanku, keagungan dan kebesaranku!"
\par 31 Aku belum habis bicara ketika terdengar suara dari langit, "Raja Nebukadnezar, dengarlah ini! Kekuasaanmu sebagai raja telah diambil daripadamu.
\par 32 Engkau akan diusir dari masyarakat manusia, dan hidup dengan binatang-binatang di padang. Selama tujuh tahun engkau akan makan rumput seperti sapi. Setelah itu engkau akan mengakui bahwa Allah Yang Mahatinggi berkuasa atas kerajaan manusia dan kerajaan itu diberikan-Nya kepada siapa saja yang dipilih-Nya."
\par 33 Pada saat itu juga, kata-kata itu menjadi kenyataan. Aku, Nebukadnezar diusir dari masyarakat manusia dan makan rumput seperti sapi. Tubuhku dibasahi embun, rambutku menjadi sepanjang bulu elang, dan kukuku sepanjang cakar burung.
\par 34 Ketika masa tujuh tahun itu sudah lewat, aku menengadah ke langit, lalu kembalilah kesadaranku. Maka kupuji Allah Yang Mahatinggi. Kusanjung dan kumuliakan Dia yang hidup kekal itu. "Allah akan menjadi raja untuk selamanya. Ia berkuasa sepanjang masa.
\par 35 Bangsa-bangsa di dunia tidak berarti. Allah menguasai malaikat di surga dan penduduk bumi. Tak seorang pun dapat melawan kehendak-Nya, tak ada yang berani menanyakan apa yang dilakukan-Nya."
\par 36 Ketika kesadaranku kembali kepadaku, maka kebesaran, keagungan dan kemasyhuran kerajaanku dikembalikan kepadaku. Aku disambut oleh para pegawaiku dan para pembesarku, dan aku dikembalikan kepada kerajaanku, bahkan keagunganku menjadi lebih besar daripada yang dahulu.
\par 37 Jadi sekarang aku, Nebukadnezar memuji, meninggikan dan memuliakan Raja Surga. Segala perbuatan-Nya benar dan adil, dan Ia sanggup merendahkan orang yang berlaku sombong.

\chapter{5}

\par 1 Pada suatu hari Raja Belsyazar mengundang seribu orang pembesar untuk menghadiri pestanya yang mewah, dan minum-minum anggur.
\par 2 Sementara mereka sedang minum-minum, Belsyazar memerintahkan untuk mengambil mangkuk-mangkuk emas dan perak, yang telah dirampas oleh Nebukadnezar, ayahnya, dari dalam Rumah TUHAN di Yerusalem. Maksud Raja Belsyazar ialah supaya para pembesar, istri-istrinya, selir-selirnya dan ia sendiri, minum dari mangkuk-mangkuk itu.
\par 3 Segera dibawalah mangkuk-mangkuk itu kepadanya, lalu mereka semua minum anggur dari alat-alat minum itu.
\par 4 Sambil minum mereka memuji-muji dewa-dewa yang terbuat dari emas, perak, tembaga, kayu dan batu.
\par 5 Tiba-tiba tampaklah tangan manusia yang menuliskan sesuatu pada dinding istana di tempat yang paling terang kena sinar lampu, sehingga raja dapat melihatnya dengan jelas.
\par 6 Ia menjadi pucat pasi dan begitu ketakutan sehingga lututnya gemetaran.
\par 7 Dengan berteriak ia meminta supaya para ahli nujum, ahli jampi dan ahli perbintangan dipanggil. Ketika mereka datang, raja berkata, "Barangsiapa dapat membaca tulisan itu dan memberitahukan artinya kepadaku, akan kuberi pakaian kerajaan berupa jubah ungu dan kalung emas tanda kehormatan. Selain itu ia akan kuangkat menjadi penguasa ketiga dalam kerajaanku."
\par 8 Para cerdik pandai melangkah maju, tapi tak seorang pun dari mereka dapat membaca tulisan itu atau memberitahukan artinya kepada raja.
\par 9 Raja Belsyazar menjadi lebih cemas dan pucat lagi; juga para pembesarnya sangat kebingungan.
\par 10 Mendengar teriakan-teriakan raja dan para pembesarnya, ibunda raja masuk ke dalam ruang pesta. Lalu ia berkata, "Hiduplah Tuanku selama-lamanya! Tenangkanlah hati Tuanku dan jangan menjadi pucat.
\par 11 Sebab dalam kerajaan Tuanku ada seorang laki-laki yang dipenuhi oleh roh dewa-dewa yang suci. Pada masa pemerintahan ayah Tuanku, orang itu terbukti mempunyai kecerdasan, pengertian dan hikmat, yang seperti hikmat para dewa. Ayah Tuanku Raja Nebukadnezar telah mengangkat dia menjadi pemimpin para ahli tenung, ahli jampi, orang-orang berilmu dan ahli perbintangan.
\par 12 Ia luar biasa pandai dan bijaksana serta ahli dalam menerangkan mimpi, mengungkapkan rahasia dan memecahkan soal-soal yang sulit. Namanya Daniel, tetapi raja menamakannya Beltsazar. Panggillah dia. Ia akan memberitahukan kepada Tuanku apa artinya tulisan ini."
\par 13 Dengan segera Daniel dibawa menghadap raja. Lalu bertanyalah raja kepadanya, "Engkaukah Daniel, orang Yahudi buangan, yang telah diangkut oleh ayahku dari tanah Yehuda?
\par 14 Kudengar bahwa engkau dipenuhi oleh roh-roh dewa yang suci, dan bahwa engkau mempunyai kecerdasan, pengertian, dan hikmat yang luar biasa.
\par 15 Para cerdik pandai dan ahli-ahli jampi telah mencoba membaca tulisan ini dan memberitahukan artinya kepadaku. Tetapi mereka tidak bisa.
\par 16 Baru saja kudengar bahwa engkau dapat menyingkapkan segala rahasia dan memecahkan soal-soal yang sulit. Jika engkau dapat membaca tulisan ini dan memberitahukan artinya kepadaku, engkau akan kuberi pakaian kerajaan berupa jubah ungu dan kalung emas tanda kehormatan. Selain itu engkau akan kuangkat menjadi penguasa ketiga dalam kerajaanku."
\par 17 Daniel menjawab, "Tuanku tak perlu memberi hadiah-hadiah itu kepadaku, berikan sajalah kepada orang lain. Aku akan membaca tulisan itu untuk Tuanku dan memberitahukan artinya.
\par 18 Allah Yang Mahatinggi membuat Nebukadnezar, ayah Tuanku menjadi raja agung dan memberikannya kebesaran dan kemasyhuran.
\par 19 Ia begitu besar, sehingga orang-orang dari segala bangsa, suku bangsa, dan bahasa, gentar dan takut kepadanya. Jika ia ingin membunuh orang, dibunuhnya saja orang itu. Jika ia mau menyelamatkan seseorang, diselamatkannya orang itu. Ia meninggikan atau merendahkan siapa saja menurut kehendaknya.
\par 20 Tetapi ketika ia menjadi sombong, keras kepala dan kejam, ia digulingkan dari takhta kerajaannya sehingga ia kehilangan keagungannya.
\par 21 Ia diusir dari masyarakat manusia dan akalnya menjadi seperti akal binatang. Ia hidup dengan keledai hutan, makan rumput seperti sapi dan tidur di lapangan terbuka sehingga dibasahi embun. Akhirnya ia mengakui bahwa Allah Yang Mahatinggi berkuasa atas kerajaan manusia dan kerajaan itu diberikan-Nya kepada siapa saja yang dipilih-Nya.
\par 22 Tetapi putranya, yaitu Tuanku sendiri, tidak mau merendahkan diri, meskipun Tuanku tahu semuanya itu.
\par 23 Tuanku meninggikan diri terhadap TUHAN di surga dan berani menyuruh membawa masuk mangkuk-mangkuk yang telah dirampas dari Rumah TUHAN. Lalu Tuanku serta para pembesar, para istri dan para selir Tuanku minum dari mangkuk-mangkuk itu. Tuanku memuji-muji dewa-dewa yang terbuat dari emas, perak, tembaga, besi, kayu dan batu, dewa-dewa yang tidak dapat melihat atau mendengar atau mengerti apa-apa. Tuanku tidak menghormati Allah yang menentukan hidup dan mati Tuanku dan menetapkan jalan hidup Tuanku.
\par 24 Itulah sebabnya Allah mengirim tangan itu untuk menuliskan pesan-Nya.
\par 25 Inilah tulisan itu, 'Dihitung, dihitung, ditimbang, dibagi.'
\par 26 Dan inilah artinya: Dihitung, masa pemerintahan Tuanku telah dihitung oleh Allah dan diakhirinya;
\par 27 ditimbang, Tuanku telah ditimbang dan didapati tidak memuaskan;
\par 28 dibagi; kerajaan Tuanku dibagi dan diberikan kepada orang Media dan Persia."
\par 29 Dengan segera Belsyazar memerintahkan pegawai-pegawainya supaya memakaikan kepada Daniel jubah ungu dan mengalungkan kalung emas pada lehernya. Lalu ia mengangkat Daniel menjadi penguasa ketiga dalam kerajaannya.
\par 30 Pada malam itu juga terbunuhlah Belsyazar raja Babel itu.

\chapter{6}

\par 1 Sesudah Belsyazar terbunuh, Darius orang Media merebut takhta kerajaan. Pada waktu itu ia berumur enam puluh dua tahun.
\par 2 Darius membagi kerajaannya menjadi seratus dua puluh provinsi yang masing-masing diperintah oleh seorang gubernur.
\par 3 Daniel dan dua orang lain diangkatnya untuk mengawasi para gubernur itu supaya raja jangan dirugikan.
\par 4 Segera ternyata bahwa pekerjaan Daniel lebih baik daripada pekerjaan para gubernur dan pengawas-pengawas lainnya. Karena itu, raja ingin mengangkatnya menjadi penguasa seluruh kerajaan.
\par 5 Tetapi para gubernur dan pengawas-pengawas itu berusaha mencari kesalahan-kesalahan Daniel dalam tugas pemerintahan, namun mereka tidak berhasil, karena Daniel setia dan jujur serta tidak melakukan kelalaian atau kesalahan apa pun.
\par 6 Lalu mereka berkata, "Kita hanya dapat menemukan kesalahan Daniel dalam hal yang berhubungan dengan agamanya."
\par 7 Kemudian pergilah mereka serentak menghadap raja dan berkata, "Ya Tuanku Raja Darius, hiduplah Tuanku untuk selama-lamanya!
\par 8 Kami semua yang mengurus kerajaan Tuanku, baik para pengawas, para gubernur, wakil-wakil gubernur dan pejabat-pejabat yang lain, telah mufakat untuk mengusulkan supaya Tuanku mengeluarkan surat perintah yang harus ditaati dengan sungguh-sungguh. Hendaknya Tuanku memerintahkan supaya selama tiga puluh hari tak seorang pun diizinkan menyampaikan permohonan kepada salah satu dewa atau manusia, kecuali kepada Tuanku sendiri. Barangsiapa melanggar perintah itu akan dilemparkan ke dalam gua singa.
\par 9 Kami mohon agar Tuanku menandatangani surat perintah itu supaya menjadi undang-undang Media dan Persia yang tak dapat dicabut kembali."
\par 10 Maka Raja Darius menandatangani surat perintah itu.
\par 11 Ketika Daniel mendengar tentang hal itu, pulanglah ia ke rumahnya. Kamarnya yang di tingkat atas mempunyai jendela-jendela yang menghadap ke arah Yerusalem. Dan seperti biasanya, ia berdoa kepada Allahnya dan memuji-Nya tiga kali sehari dengan berlutut di depan jendela-jendela yang terbuka itu.
\par 12 Ketika musuh-musuh Daniel melihat Daniel sedang berdoa kepada Allahnya,
\par 13 pergilah mereka semua menghadap raja untuk mengadukan Daniel. Mereka mengatakan, "Bukankah Tuanku telah menandatangani surat perintah yang melarang semua orang selama tiga puluh hari ini menyampaikan permohonan kepada salah satu dewa atau manusia kecuali kepada Tuanku saja? Dan juga, bahwa barangsiapa melanggar perintah itu akan dilemparkan ke dalam gua singa?" Raja menjawab, "Memang, dan perintah itu menjadi undang-undang Media dan Persia yang tak dapat dicabut kembali."
\par 14 Lalu kata mereka kepada raja, "Daniel, salah seorang buangan dari Yehuda, tidak menghiraukan Tuanku dan meremehkan perintah Tuanku. Ia berdoa secara teratur tiga kali sehari."
\par 15 Mendengar itu raja menjadi sedih dan khawatir, sehingga ia mencari akal untuk menyelamatkan Daniel. Sampai sore harinya pun raja masih berpikir-pikir.
\par 16 Kemudian orang-orang itu kembali menghadap raja dan berkata, "Tuanku, hendaknya Tuanku ingat bahwa menurut undang-undang Media dan Persia, perintah yang dikeluarkan raja tak dapat diubah-ubah."
\par 17 Maka akhirnya raja memerintahkan supaya Daniel ditangkap dan dilemparkan ke dalam gua singa. Kata raja kepada Daniel, "Semoga Allahmu yang kausembah dengan setia itu menyelamatkan engkau."
\par 18 Setelah itu sebuah batu besar diletakkan pada mulut gua itu, dan raja mencap batu itu dengan cap kerajaan dan cap para pembesar, sehingga tak seorang pun dapat membebaskan Daniel dari singa-singa itu.
\par 19 Kemudian pulanglah raja ke istana. Ia tidak mau makan atau pun dihibur. Dan semalam-malaman itu ia tidak bisa tidur.
\par 20 Pada waktu subuh bangunlah raja dan pergi dengan buru-buru ke gua singa.
\par 21 Sesampainya di sana, berserulah ia dengan suara cemas, "Daniel, hamba Allah yang hidup! Apakah Allahmu yang kausembah dengan setia itu telah sanggup menyelamatkan engkau dari singa-singa itu?"
\par 22 Lalu terdengarlah suara Daniel yang menjawab, "Hiduplah Tuanku untuk selama-lamanya!
\par 23 Allah hamba telah mengutus malaikat-Nya untuk menutup mulut singa-singa itu sehingga mereka tidak mengapa-apakan hamba. Allah menyelamatkan hamba sebab Ia tahu bahwa hamba tidak berbuat kesalahan terhadap-Nya dan terhadap Tuanku."
\par 24 Bukan main senang hati raja dan ia memerintahkan supaya Daniel dikeluarkan dari gua itu. Setelah perintah itu dilaksanakan, ternyata bahwa tidak terdapat luka sedikit pun pada Daniel, karena ia percaya kepada Allahnya.
\par 25 Kemudian raja memerintahkan orang supaya menangkap orang-orang yang telah mengadukan Daniel. Lalu mereka bersama-sama dengan anak-anak dan istri-istri mereka dilemparkan ke dalam gua singa itu. Belum lagi mereka sampai ke dasar gua itu, singa-singa itu telah menerkam mereka dan meremukkan tulang-tulang mereka.
\par 26 Setelah itu Raja Darius mengirim surat kepada orang-orang dari segala bangsa, suku bangsa dan bahasa di seluruh dunia, "Salam sejahtera!
\par 27 Aku perintahkan kepada semua orang yang berada di wilayah kerajaanku supaya takut dan hormat kepada Allah yang disembah oleh Daniel! Ia adalah Allah yang hidup selama-lamanya, sampai akhir zaman Ia memerintah. Kerajaan-Nya tak mungkin binasa. Kekuasaan-Nya tak ada habisnya.
\par 28 Ia menyelamatkan dan membebaskan, melakukan mujizat dan keajaiban di langit maupun di bumi. Daniel telah diselamatkan-Nya, dari terkaman singa-singa."
\par 29 Demikianlah Daniel tetap berkedudukan tinggi selama pemerintahan Darius dan pemerintahan Koresh, orang Persia itu.

\chapter{7}

\par 1 Pada suatu malam, pada tahun pertama Belsyazar menjadi raja di Babel, aku, Daniel bermimpi dan mendapat penglihatan. Mimpi itu kucatat dan laporan mengenai penglihatanku itu adalah sebagai berikut: Angin kencang bertiup dari segala arah dan mengakibatkan adanya badai di samudra raya.
\par 3 Kemudian empat ekor binatang raksasa muncul dari dalam air, yang satu berbeda dengan yang lain.
\par 4 Yang pertama rupanya seperti singa, tetapi ia mempunyai sayap burung garuda. Ketika aku memandangnya, tiba-tiba sayapnya tercabut dan lepas; ia terangkat dari tanah dan ditegakkan pada dua kaki seperti manusia, lalu diberikan akal manusia.
\par 5 Binatang yang kedua rupanya seperti beruang. Ia berdiri pada kaki belakangnya, dan membawa tiga potong tulang rusuk di antara gigi-giginya. Kudengar suara berkata kepadanya, "Ayo, makanlah daging sebanyak kausuka!"
\par 6 Ketika aku sedang mengamat-amatinya, muncullah binatang yang lain. Rupanya seperti macan tutul, tetapi pada punggungnya ada empat sayap burung, dan ia berkepala empat. Ia kelihatan berwibawa.
\par 7 Ketika aku sedang memandangnya, binatang yang keempat muncul. Ia sangat kuat dan dahsyat serta mengerikan. Dengan gigi-giginya yang besar dan sekeras besi, ia meremukkan dan melahap mangsanya, lalu menginjak-injak sisa-sisanya. Berbeda dengan binatang-binatang yang sebelumnya ia bertanduk sepuluh.
\par 8 Sementara aku memperhatikan tanduk-tanduknya, tiba-tiba sebuah tanduk lain tumbuh di antaranya. Tanduk itu kecil, tetapi dapat mendesak dan mencabut tiga buah di antara tanduk-tanduk yang mula-mula itu. Tanduk kecil itu mempunyai mata seperti mata manusia dan mulut yang membual dengan sombong.
\par 9 Sementara aku terus melihat, beberapa takhta sedang diletakkan. Lalu Dia yang hidup kekal duduk di atas salah satu dari takhta-takhta itu. Pakaian-Nya dan rambut-Nya putih bersih seperti kapas. Takhta-Nya dengan roda-rodanya menyala-nyala karena kobaran api,
\par 10 dan aliran api mengalir dari takhta itu. Ribuan orang melayani Dia, dan jutaan orang berdiri di hadapan-Nya. Kemudian dimulailah sidang pengadilan, dan buku-buku di buka.
\par 11 Ketika aku sedang melihatnya, masih saja kudengar tanduk kecil itu membual dengan sombongnya. Kemudian binatang yang keempat itu dibunuh, dan bangkainya dilemparkan ke dalam api hingga musnah.
\par 12 Binatang-binatang yang lain telah dicabut kekuasaannya, tetapi mereka diizinkan hidup sampai waktu yang telah ditentukan.
\par 13 Dalam penglihatanku pada malam itu, kulihat sesuatu yang seperti manusia. Ia datang dengan dikelilingi awan lalu pergi kepada Dia yang hidup kekal dan diperkenalkan kepadanya.
\par 14 Ia diberi kehormatan dan kekuasaan sebagai raja, sehingga orang-orang dari segala bangsa, suku bangsa dan bahasa mengabdi kepadanya. Kekuasaannya akan bertahan selama-lamanya, pemerintahannya tidak akan digulingkan.
\par 15 Penglihatan-penglihatan yang kulihat itu membingungkan dan menggelisahkan hatiku.
\par 16 Lalu kudekati salah seorang yang berdiri di sana dan kuminta keterangan tentang semuanya itu. Maka ia pun memberitahukannya.
\par 17 Ia mengatakan, "Keempat ekor binatang raksasa itu ialah empat kerajaan yang akan muncul di bumi.
\par 18 Tetapi umat Allah Yang Mahatinggi akan menerima hak untuk memerintah dan pemerintahannya akan bertahan selama-lamanya."
\par 19 Kemudian aku ingin tahu arti binatang yang keempat itu, yang berbeda dengan yang lain; yang begitu mengerikan dan yang menghancurkan dan melahap mangsanya dengan kuku tembaganya dan gigi besinya, serta menginjak-injak sisa-sisanya.
\par 20 Aku ingin tahu juga tentang kesepuluh tanduk pada kepalanya dan tentang tanduk kecil yang muncul kemudian lalu mencabut tiga buah dari tanduk-tanduk yang mula-mula itu. Tanduk kecil itu mempunyai mata dan mulut yang membual dengan sombongnya. Ia lebih mengerikan daripada tanduk-tanduk yang lain itu.
\par 21 Ketika aku memperhatikannya, tanduk itu berperang melawan umat Allah dan mengalahkan mereka.
\par 22 Kemudian Dia yang hidup kekal itu datang lalu memberi keputusan yang membenarkan umat Allah Yang Mahatinggi. Masanya telah tiba mereka menerima kuasa untuk memerintah.
\par 23 Inilah keterangan yang kudapat, "Binatang yang keempat itu ialah kerajaan yang keempat yang akan berdiri di bumi dan akan berbeda dengan kerajaan-kerajaan yang lain. Kerajaan itu akan menghancurkan dan melahap seluruh bumi dan menginjak-injak sisanya.
\par 24 Kesepuluh tanduk itu ialah kesepuluh raja yang akan memerintah kerajaan itu. Kemudian seorang raja yang lain akan muncul; ia akan berbeda sekali dengan raja-raja yang mula-mula, dan ia akan mengalahkan tiga orang raja.
\par 25 Ia akan berbicara melawan Allah Yang Mahatinggi dan menindas umat Allah. Ia akan berusaha mengubah hukum-hukum dan pesta-pesta agama umat Allah, dan mereka akan dikuasainya selama tiga setengah tahun.
\par 26 Lalu pengadilan surga akan bersidang dan mencabut kekuasaannya serta menghancurkan dia sampai musnah.
\par 27 Kuasa dan kebesaran segala kerajaan di bumi akan diberikan kepada umat Allah Yang Mahatinggi. Kuasa mereka untuk memerintah tak akan berakhir dan semua penguasa di bumi akan mengabdi mereka dengan taat."
\par 28 Sekianlah laporan ini. Aku sangat gelisah sehingga menjadi pucat, tetapi tak seorang pun kuberitahu tentang semuanya itu.

\chapter{8}

\par 1 Pada tahun ketiga pemerintahan Raja Belsyazar, aku mendapat penglihatan kedua.
\par 2 Dalam penglihatan itu kudapati diriku di provinsi Elam, di Susan, ibukota Persia. Aku berdiri di dekat Sungai Ulai,
\par 3 dan di pinggir sungai itu tampak seekor domba jantan yang mempunyai dua buah tanduk panjang, yang satu lebih panjang dan lebih baru daripada yang lain.
\par 4 Kulihat domba jantan itu menanduk ke arah barat, utara dan selatan. Tak seekor binatang pun yang tahan menghadapinya atau luput dari kuasanya. Ia berbuat sekehendak hatinya dan menjadi sombong.
\par 5 Sementara aku memandangnya, tampak seekor kambing jantan berlari-lari dari sebelah barat melintasi bumi. Begitu kencang larinya sehingga kakinya tidak menyentuh tanah. Di antara kedua matanya ada satu tanduk yang menyolok.
\par 6 Kambing jantan itu mendekati domba jantan yang kulihat di pinggir sungai itu, lalu menyerbu ke arahnya dengan ganas.
\par 7 Kuperhatikan ia menyerang domba jantan itu. Ia begitu ganas sehingga menubruk domba jantan itu dan mematahkan kedua tanduknya. Domba jantan itu tidak berdaya untuk melawan. Ia terlempar ke tanah dan diinjak-injak, dan tak ada seorang pun yang dapat menolongnya.
\par 8 Kambing jantan itu semakin besar. Tetapi ketika ia sampai pada puncak kekuasaannya, patahlah tanduknya yang besar itu, lalu di tempat itu tumbuh empat buah tanduk yang menyolok, masing-masing menunjuk ke arah yang berlainan.
\par 9 Dari salah satu tanduk itu tumbuhlah tanduk kecil, yang menjadi sangat besar dan kekuasaannya meluas ke arah selatan, ke arah timur dan ke arah tanah yang permai.
\par 10 Tanduk itu menjadi semakin besar sampai cukup kuat untuk menyerang tentara surga yaitu bintang-bintang, malahan beberapa di antaranya dilemparkannya ke tanah dan diinjak-injaknya.
\par 11 Bahkan ia menantang panglima tentara surga, menghentikan kurban persembahan harian yang dipersembahkan kepada panglima itu, dan merobohkan rumah ibadat untuk dia.
\par 12 Lalu orang-orang di sana mulai berdosa dan tidak mempersembahkan kurban-kurban yang diwajibkan itu. Maka ibadah yang benar telah dicampakkan ke tanah. Tanduk itu berhasil dalam segala perbuatannya.
\par 13 Kemudian kudengar seorang malaikat berkata kepada yang lain, "Sampai berapa lamakah semua yang tampak dalam penglihatan itu akan berlangsung? Sampai kapan dosa besar itu menggantikan kurban harian? Sampai kapan tentara surga dan rumah ibadat itu diinjak-injak?"
\par 14 Kudengar malaikat yang satu lagi menjawab, "Sampai 1.150 hari lagi. Selama itu kurban petang dan kurban pagi tidak dipersembahkan. Setelah itu barulah rumah ibadat akan dipulihkan."
\par 15 Ketika aku sedang berusaha untuk memahami arti penglihatan itu, tiba-tiba berdiri di depanku sesuatu yang seperti manusia.
\par 16 Kudengar suara yang berseru di seberang Sungai Ulai, katanya, "Gabriel, terangkanlah kepadanya penglihatannya itu."
\par 17 Lalu Gabriel yang berdiri di depanku itu mendekati aku. Aku menjadi begitu takut, sehingga aku rebah. Kata Gabriel kepadaku, "Hai manusia fana, engkau harus tahu bahwa penglihatan itu adalah mengenai akhir zaman."
\par 18 Sementara ia berbicara, aku pingsan. Tetapi ia memegang aku dan menolong aku berdiri kembali. Lalu ia berkata,
\par 19 "Aku akan memberitahukan kepadamu apa yang akan terjadi kelak sebagai akibat kemarahan Allah. Sebab penglihatanmu itu menunjuk kepada akhir zaman.
\par 20 Domba jantan yang kaulihat itu, yang mempunyai dua buah tanduk, melambangkan kerajaan Media dan Persia.
\par 21 Kambing jantan itu melambangkan kerajaan Yunani, dan tanduk yang menyolok di antara kedua matanya itu ialah rajanya yang pertama.
\par 22 Keempat tanduk yang muncul setelah tanduk pertama itu patah, berarti bahwa kerajaan Yunani akan terbagi-bagi menjadi empat kerajaan, tetapi satu pun tidak akan sekuat kerajaan yang mula-mula itu.
\par 23 Menjelang akhir kerajaan-kerajaan itu, apabila kejahatan sudah memuncak, maka akan muncul seorang raja yang keras kepala dan pandai menipu.
\par 24 Ia akan menjadi kuat sekali, tetapi tidak karena kekuatannya sendiri. Ia akan mendatangkan kebinasaan yang mengerikan dan apa saja yang dilakukannya akan berhasil. Ia akan membinasakan orang-orang perkasa dan umat Allah.
\par 25 Karena ia licik, penipuan-penipuannya akan berhasil. Ia akan menyombongkan dirinya, dan tanpa memberi peringatan lebih dahulu ia akan membinasakan banyak orang. Bahkan ia berani melawan Raja Yang Mahabesar. Tetapi ia akan dihancurkan tanpa kekuatan manusia.
\par 26 Penglihatan tentang kurban petang dan pagi yang telah diterangkan kepadamu itu, benar-benar akan terjadi. Tetapi semua itu akan terjadi di masa depan yang masih jauh; jadi jangan ceritakan kepada seorang pun."
\par 27 Kemudian aku merasa lemah lalu jatuh sakit beberapa hari lamanya. Setelah itu bangunlah aku dan kembali melakukan tugas-tugasku untuk raja. Tetapi aku gelisah memikirkan penglihatan-penglihatan itu dan tak dapat memahaminya.

\chapter{9}

\par 1 Pada tahun pertama pemerintahan Darius orang Media, putra Ahasyweros atas kerajaan Babel, aku Daniel, membaca-baca buku-buku agama. Aku merenungkan bahwa menurut perkataan TUHAN kepada Nabi Yeremia, kota Yerusalem akan tetap runtuh selama tujuh puluh tahun.
\par 3 Lalu berdoalah aku dengan sungguh-sungguh kepada TUHAN Allah, dan memohon belas kasihannya. Aku pun berpuasa dan memakai kain karung serta duduk di atas abu.
\par 4 Segala dosa bangsaku kuakui kepada TUHAN Allahku. Kataku, "TUHAN Allah, Engkau besar, dan kami menghormati Engkau. Engkau selalu menepati perjanjian-Mu dan menunjukkan kasih-Mu yang abadi kepada mereka yang mengasihi Engkau dan melakukan apa yang Kauperintahkan.
\par 5 Kami telah berdosa, kami telah berbuat jahat, dan melakukan kesalahan. Kami telah berontak dan melanggar perintah-Mu serta menyimpang dari peraturan-peraturan-Mu.
\par 6 Kami tidak mendengarkan para nabi, hamba-hamba-Mu itu, yang telah berbicara atas nama-Mu kepada raja-raja kami, kepada para pemimpin, para leluhur dan kepada seluruh bangsa kami.
\par 7 Ya TUHAN, tindakan-Mu selalu tepat dan adil, tetapi kami selalu mendatangkan malu kepada diri kami, yaitu kami semua yang tinggal di Yudea dan di Yerusalem dan semua orang sebangsa kami yang telah Kausebar ke negeri-negeri yang jauh dan yang dekat oleh karena mereka tidak setia kepada-Mu.
\par 8 Raja-raja kami, para pemimpin dan nenek moyang kami telah melakukan hal-hal yang memalukan dan telah berdosa kepada-Mu, ya TUHAN.
\par 9 Engkau penuh pengampunan dan belas kasihan, meskipun kami telah melawan Engkau.
\par 10 Kami tidak mendengarkan suara-Mu, ya TUHAN Allah kami, ketika Kausuruh kami hidup menurut hukum-hukum yang telah Kauberikan kepada kami melalui para nabi, hamba-hamba-Mu itu.
\par 11 Seluruh Israel telah melanggar hukum-hukum-Mu dan tidak mau mendengarkan suara-Mu. Kami berdosa terhadap Engkau, dan sebab itu Kautimpakan kepada kami kutuk yang telah tertulis dalam Buku Musa, hamba-Mu.
\par 12 Engkau bertindak tepat seperti yang telah Kaujanjikan kepada kami dan kepada para penguasa kami. Yerusalem telah Kauhukum lebih keras daripada kota mana pun di dunia ini.
\par 13 Segala malapetaka yang tertulis dalam Hukum Musa telah Kautimpakan kepada kami. Tetapi meskipun demikian, ya TUHAN Allah kami, kami masih saja menolak untuk menyenangkan hati-Mu. Kami tetap tak mau meninggalkan dosa-dosa kami dan tidak mau pula memperhatikan ajaran-ajaran-Mu.
\par 14 Engkau, ya TUHAN Allah kami, telah menghukum kami dengan malapetaka yang telah Kaupersiapkan; tindakan-Mu selalu tepat dan adil, tetapi kami tidak mendengarkan suara-Mu.
\par 15 Ya TUHAN, Allah kami, Engkau telah menunjukkan kuasa-Mu dengan membawa umat-Mu keluar dari Mesir, dan kuasa-Mu itu masih tetap dikenang. Kami telah berdosa dan berbuat kesalahan.
\par 16 Engkau telah membela kami di zaman dulu, sebab itu janganlah marah lagi kepada Yerusalem, kota-Mu sendiri, bukit-Mu yang suci. Oleh karena dosa kami dan kejahatan nenek moyang kami, maka semua orang yang di sekeliling kami memandang rendah terhadap Yerusalem dan terhadap umat-Mu.
\par 17 Ya Allah kami, dengarkanlah doa dan permohonanku. Pulihkanlah rumah-Mu yang telah dimusnahkan itu, perbaikilah supaya semua orang tahu bahwa Engkaulah Allah.
\par 18 Dengarkanlah kami, ya Allah, pandanglah kami dan perhatikanlah penderitaan kami serta kerusakan kota yang disebut dengan nama-Mu. Kami berdoa kepada-Mu bukan karena jasa-jasa kami, melainkan karena Engkau penuh belas kasihan.
\par 19 TUHAN, dengarlah kami, ampunilah kami, perhatikanlah kami dan bertindaklah segera! Janganlah lama-lama, supaya semua orang tahu bahwa Engkaulah Allah. Sebab Yerusalem dan umat-Mu adalah milik-Mu yang khas."
\par 20 Aku terus berdoa dan mengaku dosaku dan dosa bangsaku Israel, serta memohon kepada TUHAN Allahku untuk memulihkan rumah-Nya yang suci.
\par 21 Sementara aku berdoa, Gabriel yang telah kulihat dalam penglihatanku yang dahulu itu, terbang ke arahku. Saat itu waktunya orang mempersembahkan kurban petang.
\par 22 Lalu Gabriel menerangkan demikian, "Daniel, aku datang untuk membantu engkau memahami ramalan itu.
\par 23 Ketika engkau mulai memohon kepada Allah, Ia memberi jawaban. Aku datang menyampaikannya sebab Ia mengasihi engkau. Nah, dengarkanlah baik-baik sementara aku menerangkan arti penglihatan itu.
\par 24 Tujuh kali tujuh puluh tahun ialah masa yang ditetapkan Allah untuk membebaskan bangsamu dan kotamu yang suci itu dari dosa dan kejahatan. Segala dosa akan diampuni dan keadilan yang kekal akan ditegakkan, sehingga penglihatan dan ramalan itu akan menjadi kenyataan, dan Rumah TUHAN yang suci itu akan dibaktikan lagi kepada TUHAN.
\par 25 Catatlah dan fahamilah ini: Tujuh kali tujuh tahun akan lewat, mulai dari saat dikeluarkannya perintah untuk membangun kembali Yerusalem, sampai datangnya seorang pemimpin pilihan Allah. Selama tujuh kali enam puluh dua tahun jalan-jalan dan kubu-kubu Yerusalem akan dibangun kembali, tetapi masa itu penuh kesukaran.
\par 26 Pada akhir masa itu pemimpin pilihan Allah itu akan dibunuh, padahal ia tidak bersalah. Maka datanglah tentara seorang raja yang kuat, lalu memusnahkan kota Yerusalem serta Rumah TUHAN. Akhir zaman itu akan datang seperti banjir yang membawa perang dan kehancuran, seperti yang telah ditetapkan oleh Allah.
\par 27 Raja itu akan membuat perjanjian teguh dengan banyak orang selama tujuh tahun. Pada pertengahan masa itu, ia akan menghentikan diadakannya kurban dan persembahan. Kemudian sesuatu yang mengerikan yang disebut Kejahatan yang menghancurkan akan ditempatkan di Rumah TUHAN dan akan tetap ada di sana sampai dia yang menempatkannya di situ tertimpa kebinasaan yang telah ditetapkan oleh Allah baginya."

\chapter{10}

\par 1 Pada tahun ketiga pemerintahan Raja Koresh atas Persia, suatu pesan dinyatakan kepada Daniel alias Beltsazar. Pesan itu benar, tetapi sangat sukar untuk dimengerti. Ketika berusaha memahaminya Daniel menerima keterangan tentang arti pesan itu dalam suatu penglihatan.
\par 2 Pada waktu itu aku sedang bertapa tiga minggu penuh.
\par 3 Selama waktu itu aku tidak makan makanan yang enak atau pun daging; aku tidak minum anggur, dan tidak juga menyisir rambut.
\par 4 Pada tanggal dua puluh empat bulan pertama tahun itu, aku sedang berdiri di tepi Sungai Tigris yang besar itu.
\par 5 Aku menengadah, lalu kulihat seorang yang memakai pakaian dari linen dan ikat pinggang dari emas murni.
\par 6 Tubuhnya bersinar-sinar seperti permata, wajahnya seterang cahaya kilat, dan matanya menyala-nyala seperti api. Lengan dan kakinya berkilau seperti tembaga yang digosok, dan suaranya terdengar seperti suara orang banyak.
\par 7 Hanya aku sendiri yang melihat penglihatan itu. Orang-orang yang bersamaku tidak melihatnya, tetapi mereka ketakutan sehingga lari dan bersembunyi.
\par 8 Aku ditinggalkan di situ seorang diri, sambil memperhatikan penglihatan yang mengagetkan itu. Aku sudah tidak berdaya lagi dan wajahku menjadi pucat pasi.
\par 9 Ketika aku mendengar suaranya, jatuh pingsanlah aku sampai tertelungkup di tanah.
\par 10 Kemudian ada tangan menyentuh aku dan membuat aku bangun sambil bertumpu pada tangan dan lututku.
\par 11 Malaikat itu berkata kepadaku, "Daniel, engkau dikasihi oleh Allah. Bangkitlah dan berdiri lalu dengarkanlah kata-kataku ini baik-baik. Aku ini telah diutus kepadamu." Ketika ia mengatakan hal itu, berdirilah aku dengan gemetar.
\par 12 Lalu katanya kepadaku, "Daniel, jangan takut. Allah telah mendengar doamu sejak hari pertama engkau mengambil keputusan untuk merendahkan dirimu supaya menjadi bijaksana. Aku telah datang sebagai jawaban atas doamu.
\par 13 Malaikat pelindung kerajaan Persia melawan aku selama dua puluh satu hari. Lalu Mikhael, salah seorang dari pemimpin-pemimpin malaikat, datang menolong aku. Kutinggalkan dia di sana, di Persia.
\par 14 Aku datang untuk membuat engkau mengerti apa yang kelak akan terjadi pada bangsamu. Penglihatan ini tentang hari depan."
\par 15 Ketika ia mengatakan hal itu, aku tunduk dan tak dapat berbicara.
\par 16 Kemudian malaikat yang menyerupai manusia itu mengulurkan tangannya dan menyentuh bibirku. Aku berkata kepadanya, "Tuan, penglihatan ini membuat aku begitu tertekan sehingga aku tidak berdaya sama sekali.
\par 17 Aku seperti seorang hamba yang berdiri di hadapan tuannya. Bagaimana mungkin aku bicara kepada Tuan? Tenagaku habis dan napasku sesak."
\par 18 Maka sekali lagi ia menyentuh aku, lalu kurasa tenagaku bertambah.
\par 19 Ia berkata, "Allah mengasihi engkau, sebab itu janganlah cemas atau takut, ayo, jadilah kuat!" Sementara ia mengatakan hal itu, aku merasa lebih kuat lagi dan berkata, "Silakan bicara, Tuan. Tuan telah menambah kekuatanku."
\par 20 Lalu ia menjawab, "Tahukah engkau mengapa aku datang kepadamu? Maksudnya ialah untuk mengatakan kepadamu apa yang tertulis di dalam Buku Kebenaran. Sekarang aku harus kembali untuk berperang melawan malaikat pelindung Persia. Sesudah itu malaikat pelindung Yunani akan datang. Tak ada yang akan menolongku kecuali Mikhael, malaikat pelindung Israel, negaramu.

\chapter{11}

\par 1 Dialah yang ditugaskan untuk membantu dan membela aku pada tahun pertama pemerintahan Darius orang Media itu.
\par 2 Dan apa yang akan kukatakan kepadamu adalah benar." Malaikat itu mengatakan, "Tiga raja lain akan memerintah Persia, diikuti oleh raja yang keempat yang akan lebih kaya daripada yang lain. Pada puncak kekuasaan dan kekayaannya, ia akan beradu kekuatan dengan kerajaan Yunani.
\par 3 Kemudian akan muncul seorang raja yang perkasa. Ia akan memerintah kerajaan yang sangat besar dan akan berbuat semaunya.
\par 4 Tetapi pada puncak kejayaannya, kerajaannya akan pecah dan terbagi-bagi menjadi empat bagian. Raja-raja yang bukan keturunannya akan memerintah sebagai gantinya, dan mereka tidak akan sekuat dia.
\par 5 Raja negeri selatan akan menjadi kuat. Tetapi salah seorang dari panglima-panglimanya akan menjadi lebih kuat daripadanya, dan ia akan memerintah kerajaan yang lebih besar lagi.
\par 6 Beberapa tahun kemudian raja negeri selatan akan mengadakan persekutuan dengan raja negeri utara dan mengawinkan putrinya dengan raja negeri utara itu. Tetapi persekutuan itu tidak bertahan, dan putri itu akan dibunuh, demikian juga suami dan anaknya dan hamba-hamba yang telah mengantarkannya.
\par 7 Tidak lama setelah itu, seorang yang sekeluarga dengan putri itu akan menjadi raja. Ia akan menyerang tentara raja negeri utara, menyerbu bentengnya dan mengalahkannya.
\par 8 Pada waktu ia kembali ke Mesir, ia akan membawa patung dewa musuhnya dan perkakas emas dan perak yang telah dipersembahkan kepada dewa itu. Beberapa tahun lamanya ia tidak akan memerangi raja negeri utara.
\par 9 Kemudian raja negeri utara akan menyerbu negeri selatan, tetapi ia akan dipukul mundur dan pulang ke negerinya sendiri.
\par 10 Lalu putra-putra raja negeri utara akan bersiap-siap untuk berperang dan membentuk tentara yang besar. Salah seorang dari mereka itu akan maju menyerbu seperti banjir. Dalam serbuan yang kedua ia menyerang benteng musuh.
\par 11 Maka marahlah raja negeri selatan, dan ia akan maju ke medan perang melawan raja negeri utara. Ia akan berhasil mengalahkan tentara besar yang dikerahkan raja negeri utara.
\par 12 Kemudian ia akan bangga karena kemenangannya dan karena telah menewaskan banyak musuh, tetapi ia tidak akan terus berkuasa.
\par 13 Untuk kedua kalinya raja negeri utara akan membentuk pasukan yang besar, malahan yang lebih besar daripada yang pertama. Jika waktunya tiba, ia akan bergerak maju dengan tentara yang besar dan perlengkapan yang banyak sekali.
\par 14 Maka banyak orang akan memberontak terhadap raja negeri selatan. Dan beberapa orang kejam dari bangsamu, Daniel, akan memberontak juga karena mereka telah melihat suatu penglihatan. Tetapi mereka akan gagal.
\par 15 Kemudian raja negeri utara akan mengepung sebuah kota berbenteng dan merebutnya. Tentara negeri selatan tidak akan dapat bertahan, dan pasukan-pasukan pilihannya pun tidak dapat lagi mengadakan perlawanan.
\par 16 Raja negeri utara itu akan berbuat semaunya tanpa mendapat perlawanan sedikit pun. Ia pun akan menduduki tanah yang permai dan menguasainya sepenuhnya.
\par 17 Kemudian raja negeri utara akan menyiapkan seluruh tentaranya untuk menyerang negeri selatan. Lalu dengan maksud menguasai seluruh kerajaan musuhnya, ia akan membuat persetujuan dengan dia dan mengawinkan putrinya dengan raja Mesir itu; tetapi rencananya itu tidak akan berhasil.
\par 18 Setelah itu ia akan menyerang bangsa-bangsa di tepi laut, dan banyak yang dikalahkannya. Tetapi seorang pemimpin negeri asing akan menghentikan penghinaan yang dilakukannya itu, bahkan akan membalas penghinaan itu kepadanya.
\par 19 Raja negeri utara itu akan kembali ke benteng-benteng negerinya sendiri, tetapi dia akan dikalahkan dan tak ada berita lagi tentang dia.
\par 20 Dia akan digantikan oleh seorang raja yang akan menyuruh pegawainya yang kejam untuk mengumpulkan pajak dengan paksa untuk menambah kekayaan kerajaannya. Dalam waktu yang singkat raja itu akan dibunuh, tidak secara terang-terangan dan tidak pula dalam peperangan."
\par 21 Malaikat itu berkata lagi, "Raja berikut yang memerintah negeri utara adalah seorang yang hina yang tidak berhak menjadi raja. Tetapi ia akan datang tanpa disangka-sangka dan merebut kedudukan raja dengan tipu daya.
\par 22 Segala tentara yang melawan dia, ya bahkan Imam Agung pun, akan disapu bersih dan dimusnahkan.
\par 23 Ia mengkhianati perjanjian-perjanjian yang baru saja dibuatnya dengan bangsa-bangsa lain. Dan ia akan menjadi semakin kuat, meskipun ia hanya memerintah negara yang kecil.
\par 24 Dengan licik ia akan menyerbu daerah-daerah yang subur pada waktu penduduk merasa aman di situ. Ia akan melakukan apa yang belum pernah dilakukan oleh nenek moyangnya. Lalu ia akan membagi-bagikan kepada para pendukungnya segala barang rampasan dan kekayaan yang didapatnya dalam peperangan. Ia akan membuat siasat untuk menyerang tempat-tempat yang berbenteng, tetapi waktunya segera akan habis.
\par 25 Kemudian dengan berani ia akan membentuk tentara yang besar untuk menyerang raja negeri selatan yang telah bersiap-siap pula hendak melawan dia dengan tentara yang kuat dan besar sekali. Tetapi raja negeri selatan tidak akan dapat bertahan karena ia akan dikhianati.
\par 26 Penasihat-penasihatnya yang paling dekat dengan dia akan menjatuhkan dia. Banyak di antara prajurit-prajuritnya akan terbunuh, dan tentaranya akan disapu bersih.
\par 27 Kemudian kedua raja itu akan duduk pada satu meja dan makan bersama, tetapi mereka bermaksud jahat, dan mereka akan saling membohongi. Tetapi rencana mereka gagal sebab masanya belum tiba.
\par 28 Kemudian raja negeri utara akan pulang ke negerinya dengan segala barang rampasan yang didapatnya. Ia bertekad untuk menghancurkan agama yang dianut oleh umat Allah. Ia akan berbuat semaunya lalu kembali ke negerinya sendiri.
\par 29 Beberapa waktu kemudian ia akan menyerang negeri selatan lagi, tetapi akibatnya lain daripada kali yang pertama.
\par 30 Sebab orang-orang Roma dengan kapal-kapalnya akan datang dari Siprus dan memerangi dia, sehingga ia menjadi ketakutan. Lalu ia akan pulang dengan marah sekali dan berusaha menghancurkan agama umat Allah. Ia akan mengikuti nasihat mereka yang telah murtad dari agama itu.
\par 31 Lalu ia mengirim pasukan-pasukan yang mencemarkan Rumah TUHAN. Mereka akan menghapuskan kurban harian dan menegakkan sesuatu yang mengerikan yang disebut Kejahatan yang menghancurkan.
\par 32 Maka dengan tipu daya raja itu akan membujuk orang-orang yang sudah meninggalkan agama mereka, supaya mau membantunya. Tetapi orang-orang yang taat kepada Allah akan berani melawan dia.
\par 33 Orang-orang bijaksana yang memimpin rakyat akan membuat banyak orang mengerti. Tetapi untuk beberapa waktu lamanya, beberapa di antara mereka akan terbunuh karena perang atau karena api, dan beberapa orang lagi akan dirampok dan ditawan.
\par 34 Sementara pembunuhan itu berlangsung, umat Allah akan mendapat sedikit bantuan, tetapi banyak orang akan bergabung dengan mereka hanya untuk kepentingan mereka sendiri.
\par 35 Sebagian dari pemimpin-pemimpin yang bijaksana itu akan terbunuh, tetapi itu merupakan ujian bagi umat sehingga mereka akan menjadi murni. Hal itu akan berlangsung sampai pada akhir zaman, yaitu zaman yang telah ditetapkan Allah.
\par 36 Raja negeri utara akan berbuat sekehendak hatinya. Dengan sombong ia akan mengatakan bahwa ia lebih besar daripada dewa mana pun, malahan juga lebih besar daripada Allah Yang Mahatinggi. Ia akan dapat berbuat begitu sampai murka Allah berakhir. Sebab apa yang sudah ditetapkan Allah harus terlaksana dahulu.
\par 37 Raja itu tidak akan mengindahkan dewa mana pun yang disembah oleh nenek moyangnya, termasuk dewa pujaan para wanita. Ia tidak mempedulikan dewa-dewa itu karena ia menganggap dirinya lebih besar dari mereka.
\par 38 Sebaliknya, ia akan menghormati dewa yang belum pernah disembah oleh nenek moyangnya, yaitu dewa pelindung benteng-benteng. Ia akan mempersembahkan emas, perak, permata dan persembahan-persembahan mewah lainnya kepada dewa yang belum pernah disembah oleh nenek moyangnya.
\par 39 Ia akan memilih orang-orang yang menyembah dewa asing, untuk menyerang benteng-benteng kuat. Siapa yang mengakui dewa itu akan diberinya kehormatan besar dan kedudukan yang tinggi serta tanah sebagai upah.
\par 40 Bila akhir kekuasaan raja negeri utara sudah hampir tiba, raja negeri selatan akan menyerang dia. Lalu raja negeri utara akan membalas serangan itu dengan sengit. Ia akan mengerahkan banyak kereta perang, kuda dan kapal. Ia akan menyerbu seperti air bah, dan menduduki banyak negeri.
\par 41 Bahkan tanah yang permai pun akan diserbunya juga, dan puluhan ribu orang akan dibunuhnya. Tetapi kerajaan-kerajaan Edom, Moab dan bagian yang terpenting dari kerajaan Amon akan selamat.
\par 42 Pada waktu ia menduduki kerajaan-kerajaan itu, kerajaan Mesir pun tidak akan luput.
\par 43 Ia akan merampas harta Mesir berupa emas dan perak serta harta lain yang sangat berharga. Ia akan dibantu orang Libia dan Sudan.
\par 44 Lalu berita dari timur dan utara akan mengejutkannya, sehingga ia akan berperang dengan sengit dan membunuh banyak orang.
\par 45 Bahkan ia akan memasang kemah kerajaannya di antara laut dan gunung tempat Rumah TUHAN. Tetapi kemudian ia akan mati dan tidak ada seorang pun yang akan menolongnya."

\chapter{12}

\par 1 Malaikat yang memakai pakaian linen itu berkata lagi, "Pada saat itu akan muncul malaikat besar Mikhael pelindung bangsamu. Kemudian akan ada masa kesukaran yang tiada bandingannya sejak ada bangsa-bangsa. Tetapi semua orang dari bangsamu yang namanya tertulis dalam buku Allah, akan diselamatkan.
\par 2 Banyak di antara mereka yang sudah mati akan hidup lagi; sebagian akan menikmati hidup yang kekal, dan sebagian lagi akan mengalami kehinaan dan kengerian yang kekal juga.
\par 3 Para pemimpin yang bijaksana akan bersinar seperti cahaya langit. Dan mereka yang telah mengajar banyak orang untuk melakukan apa yang baik dan adil, akan bersinar seperti bintang-bintang untuk selama-lamanya."
\par 4 Lalu kata malaikat itu lagi kepadaku, "Daniel, tutuplah buku itu sekarang, dan bubuhilah segel supaya tetap tertutup sampai pada akhir zaman. Banyak orang akan mencoba menyelidiki apa yang sedang terjadi. Tetapi usaha mereka itu akan sia-sia belaka."
\par 5 Kemudian kulihat dua orang laki-laki. Yang seorang berdiri di tepi sungai sebelah sini, dan yang lain di tepi sungai sebelah sana.
\par 6 Yang seorang bertanya kepada yang berdiri lebih dekat dengan hulu sungai, "Kapankah kejadian-kejadian yang ajaib ini akan berakhir?"
\par 7 Orang yang lain itu mengangkat kedua tangannya ke langit. Kudengar dia bersumpah demi Allah yang kekal dan berkata, "Kejadian-kejadian itu akan berlangsung selama tiga setengah tahun. Setelah penganiayaan terhadap umat Allah berakhir, maka berakhir pula kejadian-kejadian yang ajaib itu."
\par 8 Semua yang dikatakannya itu memang kudengar, tetapi aku tidak memahaminya. Lalu aku bertanya, "Tuan, bagaimana semua ini akan berakhir?"
\par 9 Ia menjawab, "Pergilah sekarang, Daniel, sebab kata-kata ini harus disembunyikan dan dirahasiakan sampai akhir zaman.
\par 10 Banyak orang akan diuji dan dimurnikan. Mereka yang jahat tidak akan mengerti dan akan terus berlaku jahat; hanya mereka yang bijaksana akan mengerti.
\par 11 Dari waktu kurban harian itu dihentikan dan sesuatu yang mengerikan yang disebut Kejahatan yang menghancurkan itu ditegakkan, akan berlalu 1.290 hari.
\par 12 Berbahagialah mereka yang tetap setia sampai hari ke-1.335.
\par 13 Engkau Daniel, setialah sampai akhir. Engkau akan meninggal, tetapi akan bangkit untuk menerima upahmu pada akhir zaman."


\end{document}