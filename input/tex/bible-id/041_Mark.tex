\begin{document}

\title{Markus}


\chapter{1}

\par 1 Inilah Kabar Baik tentang Yesus Kristus, Anak Allah.
\par 2 Kabar Baik itu mulai seperti yang ditulis oleh Nabi Yesaya, begini, "'Inilah utusan-Ku,' kata Allah; 'Aku mengutus dia lebih dahulu daripada-Mu, supaya ia membuka jalan untuk-Mu.'
\par 3 Ada orang berseru-seru di padang pasir, 'Siapkanlah jalan untuk Tuhan; ratakanlah jalan-jalan yang akan dilewati-Nya.'"
\par 4 Seperti yang tertulis itu, begitulah juga muncul Yohanes di padang gurun. Ia membaptis orang dan menyampaikan berita dari Allah. "Kamu harus bertobat dari dosa-dosamu dan harus dibaptis, supaya Allah mengampuni kamu," begitu kata Yohanes.
\par 5 Semua orang dari negeri Yudea dan kota Yerusalem pergi mendengar Yohanes. Mereka mengaku dosa-dosa mereka, dan Yohanes membaptis mereka di Sungai Yordan.
\par 6 Yohanes memakai pakaian dari bulu unta. Ikat pinggangnya dari kulit, dan makanannya belalang dan madu hutan.
\par 7 Ia mengabarkan berita ini, "Nanti sesudah saya, akan datang orang yang lebih besar daripada saya. Untuk tunduk membuka tali sepatu-Nya pun, saya tidak layak.
\par 8 Saya membaptis kamu dengan air, tetapi Ia akan membaptis kamu dengan Roh Allah."
\par 9 Pada waktu itu Yesus datang dari Nazaret di daerah Galilea, dan Yohanes membaptis Dia di Sungai Yordan.
\par 10 Begitu Yesus keluar dari sungai itu, Ia melihat langit terbuka dan Roh Allah turun seperti burung merpati ke atas-Nya.
\par 11 Kemudian terdengar suara Allah mengatakan, "Engkaulah Anak-Ku yang Kukasihi. Engkau menyenangkan hati-Ku."
\par 12 Langsung sesudah itu Roh Allah membuat Yesus pergi ke padang gurun.
\par 13 Empat puluh hari Ia berada di situ, dicobai oleh Iblis. Binatang-binatang liar ada juga bersama-sama dengan Dia di situ, dan malaikat-malaikat melayani Dia.
\par 14 Setelah Yohanes dipenjarakan, Yesus pergi ke Galilea, dan mengabarkan Kabar Baik dari Allah di sana.
\par 15 Kata-Nya, "Allah segera akan mulai memerintah. Bertobatlah dari dosa-dosamu, dan percayalah akan Kabar Baik yang dari Allah!"
\par 16 Ketika Yesus berjalan di pantai Danau Galilea, Ia melihat dua nelayan, yaitu Simon dan adiknya Andreas. Mereka sedang menangkap ikan di danau itu dengan jala.
\par 17 Yesus berkata kepada mereka, "Ikutlah Aku. Aku akan mengajar kalian menjala orang."
\par 18 Langsung mereka meninggalkan jala mereka lalu mengikuti Yesus.
\par 19 Yesus berjalan terus, lalu melihat pula dua orang bersaudara yang lain, yaitu Yakobus dan Yohanes, anak-anak Zebedeus. Mereka berada di dalam perahu, dan sedang memperbaiki jala mereka.
\par 20 Yesus langsung memanggil mereka. Dan mereka meninggalkan ayah mereka di dalam perahu bersama-sama dengan orang-orang gajiannya. Lalu mereka pergi mengikuti Yesus.
\par 21 Yesus dan pengikut-pengikut-Nya tiba di kota Kapernaum. Pada hari Sabat berikutnya, Yesus masuk ke dalam rumah ibadat dan mulai mengajar.
\par 22 Orang-orang yang mendengar-Nya kagum akan cara-Nya Ia mengajar. Sebab, Ia mengajar dengan wibawa, tidak seperti guru-guru agama.
\par 23 Pada waktu itu seorang yang kemasukan roh jahat masuk ke dalam rumah ibadat, dan berteriak,
\par 24 "Hai Yesus, orang Nazaret, Engkau mau buat apa dengan kami? Engkau datang untuk membinasakan kami? Saya tahu siapa Engkau: Engkaulah utusan yang suci dari Allah!"
\par 25 "Diam!" bentak Yesus kepada roh itu, "keluarlah dari orang ini!"
\par 26 Maka roh jahat itu menggoncang-goncangkan orang itu keras-keras, kemudian keluar dari orang itu sambil berteriak.
\par 27 Semua orang heran sekali, sehingga mereka berkata satu sama lain, "Wah, apa ini? Suatu pengajaran yang baru! Dengan wibawa dan kuasa Ia memerintahkan roh-roh jahat keluar, dan mereka taat kepada-Nya!"
\par 28 Maka berita tentang Yesus tersebar dengan cepat ke seluruh daerah Galilea.
\par 29 Yesus dan pengikut-pengikut-Nya meninggalkan rumah ibadat itu, dan pergi ke rumah Simon dan Andreas. Yakobus dan Yohanes pergi juga bersama mereka.
\par 30 Ibu mertua Simon sedang sakit demam di tempat tidur. Jadi waktu Yesus dan pengikut-pengikut-Nya masuk ke rumah itu, Yesus diberitahukan tentang hal itu.
\par 31 Maka Yesus pergi kepada ibu mertua Simon, dan memegang tangannya, lalu menolong dia bangun. Demamnya hilang, dan ia pun mulai melayani mereka.
\par 32 Petang harinya, sesudah matahari terbenam, orang-orang membawa kepada Yesus semua orang yang sakit dan yang kemasukan roh jahat.
\par 33 Seluruh penduduk kota itu berkerumun di depan rumah itu.
\par 34 Lalu Yesus menyembuhkan banyak orang yang menderita bermacam-macam penyakit, dan mengusir juga banyak roh jahat. Ia tidak mengizinkan roh-roh jahat itu berbicara, sebab mereka tahu siapa Dia.
\par 35 Keesokan harinya, waktu masih subuh, Yesus bangun lalu meninggalkan rumah. Ia pergi ke tempat yang sunyi di luar kota, dan berdoa di sana.
\par 36 Tetapi Simon dan teman-temannya pergi mencari Dia.
\par 37 Dan setelah mereka menemukan-Nya, mereka berkata, "Semua orang sedang mencari Bapak."
\par 38 Tetapi Yesus menjawab, "Mari kita meneruskan perjalanan kita ke kota-kota lain di sekitar sini. Aku harus berkhotbah di sana juga, sebab itulah maksudnya Aku kemari."
\par 39 Karena itu Yesus pergi ke mana-mana di seluruh Galilea, dan berkhotbah di rumah-rumah ibadat serta mengusir roh-roh jahat.
\par 40 Seorang yang berpenyakit kulit yang mengerikan datang kepada Yesus. Orang itu berlutut, dan berkata, "Kalau Bapak mau, Bapak dapat menyembuhkan saya."
\par 41 Yesus kasihan kepada orang itu. Jadi, Ia menjamah orang itu sambil berkata, "Aku mau, sembuhlah!"
\par 42 Saat itu juga penyakitnya hilang dan ia sembuh.
\par 43 Lalu Yesus menyuruh dia pergi dengan peringatan ini,
\par 44 "Awas, jangan ceritakan kepada siapa pun, tetapi pergilah kepada imam, dan minta dia untuk memastikan engkau sudah sembuh. Lalu untuk penyembuhanmu itu, persembahkanlah kurban menurut yang diperintahkan Musa, sebagai bukti kepada orang-orang bahwa engkau sungguh-sungguh sudah sembuh."
\par 45 Tetapi orang itu pergi, dan terus-menerus menceritakan kejadian itu di mana-mana sampai Yesus tidak dapat masuk kota dengan terang-terangan. Ia hanya tinggal di luar kota di tempat-tempat sunyi. Namun orang terus saja datang kepada-Nya dari mana-mana.

\chapter{2}

\par 1 Beberapa hari kemudian Yesus kembali ke Kapernaum. Maka tersebarlah kabar bahwa Ia sedang di rumah.
\par 2 Lalu banyak orang datang. Mereka berkerumun di sana sampai tidak ada lagi tempat. Di pintu pun penuh sesak dengan orang. Lalu Yesus menyampaikan berita dari Allah kepada mereka.
\par 3 Sementara Ia berbicara, empat orang mengusung seorang lumpuh untuk membawanya kepada Yesus.
\par 4 Tetapi karena orang terlalu banyak, mereka tidak dapat sampai ke dekat-Nya. Jadi, mereka membongkar atap tepat di atas tempat Yesus berada. Setelah itu, mereka menurunkan orang lumpuh itu bersama tikarnya.
\par 5 Waktu Yesus melihat betapa besar iman mereka, Ia berkata kepada orang lumpuh itu, "Anak-Ku, dosa-dosamu sudah diampuni."
\par 6 Beberapa guru agama yang sedang duduk di situ mulai bertanya-tanya di dalam hati,
\par 7 "Berani benar orang ini bicara begitu! Ia menghina Allah. Siapa yang boleh mengampuni dosa, selain Allah sendiri?"
\par 8 Pada waktu itu juga Yesus tahu pikiran mereka. Lalu Ia berkata, "Mengapa kalian bertanya-tanya begitu di dalam hatimu?
\par 9 Manakah yang lebih mudah: mengatakan, 'Dosamu sudah diampuni', atau mengatakan 'Bangunlah, angkat tikarmu dan berjalanlah'?
\par 10 Tetapi sekarang Aku akan membuktikan kepadamu bahwa di atas bumi ini Anak Manusia berkuasa mengampuni dosa." Lalu Yesus berkata kepada orang yang lumpuh itu,
\par 11 "Bangunlah, angkat tikarmu dan pulanglah!"
\par 12 Ia bangkit dan segera mengambil tikarnya lalu keluar disaksikan oleh mereka semua. Orang-orang itu kagum lalu memuji Allah. Kata mereka, "Belum pernah kita melihat kejadian seperti ini!"
\par 13 Yesus kembali lagi ke pantai Danau Galilea. Banyak orang datang kepada-Nya, dan Ia mengajar mereka.
\par 14 Sementara Yesus berjalan di situ, Ia melihat seorang penagih pajak bernama Lewi, anak Alfeus, sedang duduk di kantor pajak. "Ikutlah Aku," kata Yesus kepadanya. Maka Lewi berdiri dan mengikuti Yesus.
\par 15 Waktu Yesus sedang makan di rumah Lewi, datanglah banyak penagih pajak dan orang-orang yang dianggap tidak baik oleh masyarakat ikut makan bersama-sama Yesus dan pengikut-pengikut-Nya. Sebab banyak di antara mereka mengikuti Yesus.
\par 16 Beberapa guru agama dari golongan Farisi melihat Yesus makan bersama-sama dengan penagih-penagih pajak dan orang-orang yang dianggap tidak baik itu. Jadi mereka bertanya kepada pengikut-pengikut Yesus, "Mengapa gurumu makan bersama dengan penagih-penagih pajak dan orang-orang yang tidak baik itu?"
\par 17 Yesus mendengar pertanyaan mereka itu, lalu menjawab, "Orang yang sehat tidak memerlukan dokter; hanya orang yang sakit saja. Aku datang bukannya untuk memanggil orang yang menganggap dirinya sudah baik, melainkan orang yang dianggap hina."
\par 18 Pada suatu waktu pengikut-pengikut Yohanes Pembaptis dan orang-orang Farisi sedang berpuasa. Lalu ada orang datang kepada Yesus dan bertanya, "Mengapa pengikut Yohanes Pembaptis dan pengikut orang Farisi berpuasa, sedangkan pengikut Bapak tidak?"
\par 19 Yesus menjawab, "Pada pesta kawin, apakah tamu-tamu tidak makan? Kalau pengantin laki-laki masih bersama-sama mereka, tentu mereka makan.
\par 20 Tetapi akan datang waktunya pengantin laki-laki itu diambil dari mereka. Pada waktu itu barulah mereka tidak makan.
\par 21 Tidak ada orang yang menambal baju tua dengan sepotong kain baru. Sebab kain penambal itu akan menciut dan menyobek baju itu, sehingga mengakibatkan sobekan yang lebih besar.
\par 22 Begitu juga tidak ada orang yang menuang anggur baru ke dalam kantong kulit yang tua. Karena anggur baru itu akan menyebabkan kantong itu pecah. Akhirnya kedua-duanya terbuang. Anggur yang baru harus dituang ke dalam kantong yang baru juga!"
\par 23 Pada suatu hari Sabat, ketika Yesus lewat sebuah ladang gandum, pengikut-pengikut-Nya mulai memetik gandum.
\par 24 Lalu orang-orang Farisi berkata kepada Yesus, "Mengapa pengikut-pengikut-M melanggar hukum agama kita, dengan melakukan yang dilarang pada hari Sabat?"
\par 25 Yesus menjawab, "Belum pernahkah kalian membaca apa yang dilakukan Daud, ketika Abyatar sedang bertugas menjadi imam agung. Waktu Daud dan orang-orangnya lapar, dan tidak punya makanan, ia masuk ke Rumah Tuhan, dan makan roti yang sudah dipersembahkan kepada Allah. Dan roti itu diberikannya juga kepada orang-orangnya. Padahal menurut agama kita, imam-imam saja yang boleh makan roti itu."
\par 26 [2:25]
\par 27 Lalu Yesus berkata lagi, "Hari Sabat dibuat untuk manusia; bukan manusia untuk hari Sabat.
\par 28 Jadi, Anak Manusia berkuasa, bahkan atas hari Sabat."

\chapter{3}

\par 1 Yesus kembali lagi ke rumah ibadat. Di situ ada seorang yang tangannya lumpuh sebelah.
\par 2 Di situ ada orang-orang yang mau mencari kesalahan Yesus, supaya bisa mengadukan Dia. Jadi mereka terus memperhatikan Dia apakah Ia akan menyembuhkan orang pada hari Sabat.
\par 3 Yesus berkata kepada orang yang tangannya lumpuh sebelah itu, "Mari berdiri di sini, di depan."
\par 4 Lalu Ia bertanya kepada orang-orang yang ada di situ, "Menurut agama, kita boleh berbuat apa pada hari Sabat? Berbuat baik atau berbuat jahat? Menyelamatkan orang atau membunuh?" Mereka diam saja.
\par 5 Dengan marah Yesus melihat sekeliling-Nya, tetapi Ia sedih juga, karena mereka terlalu keras kepala. Lalu Ia berkata kepada orang itu, "Ulurkan tanganmu." Orang itu mengulurkan tangannya, dan tangannya pun sembuh.
\par 6 Maka orang-orang Farisi meninggalkan rumah ibadat itu, lalu segera berunding dengan beberapa orang pendukung Herodes, untuk membunuh Yesus.
\par 7 Yesus dan pengikut-pengikut-Nya pergi mengundurkan diri ke Danau Galilea. Maka banyak sekali orang dari Galilea yang pergi mengikuti Yesus. Banyak juga yang datang dari Yudea,
\par 8 dari Yerusalem, dari daerah Idumea, dan dari daerah di sebelah timur Yordan, dan dari sekitar kota-kota di Tirus dan Sidon. Mereka semuanya datang kepada Yesus, sebab mereka mendengar tentang hal-hal yang sudah dilakukan-Nya.
\par 9 Orang-orang itu begitu banyak, sehingga Yesus menyuruh pengikut-pengikut-Ny menyediakan perahu untuk-Nya, sebab jangan-jangan Ia nanti terdesak oleh orang-orang itu.
\par 10 Ia menyembuhkan begitu banyak orang, sehingga semua orang sakit berdesak-desakan berusaha mendekati-Nya agar dapat menjamah Dia.
\par 11 Dan orang-orang yang kemasukan roh jahat, setiap kali melihat Dia, terus saja sujud di hadapan-Nya dan berteriak, "Engkaulah Anak Allah!"
\par 12 Tetapi Yesus melarang keras roh-roh jahat itu memberitahukan siapa Dia.
\par 13 Kemudian Yesus naik ke sebuah bukit, dan memanggil orang-orang yang dikehendaki-Nya. Orang-orang itu datang,
\par 14 lalu Ia memilih dari antara mereka dua belas orang. Kata-Nya, "Aku memilih kalian, supaya kalian menyertai Aku, supaya Kuutus kalian untuk menyebarkan Kabar Baik dari Allah,
\par 15 dan kalian akan mendapat kuasa untuk mengusir roh-roh jahat."
\par 16 Inilah nama kedua belas orang itu: Simon (yang disebut-Nya juga Petrus),
\par 17 Yakobus dan Yohanes saudaranya, yaitu anak-anak Zebedeus (mereka ini diberi-Nya nama Boanerges artinya "anak guntur"),
\par 18 Andreas, Filipus, Bartolomeus, Matius, Tomas, Yakobus anak Alfeus, Tadeus, Simon si Patriot,
\par 19 dan Yudas Iskariot yang mengkhianati Yesus.
\par 20 Kemudian Yesus pulang ke rumah. Tetapi orang banyak datang lagi berkumpul, sampai Yesus dan pengikut-pengikut-Nya tidak sempat makan.
\par 21 Orang-orang berkata, "Dia sudah gila!" Dan ketika keluarga-Nya mendengar hal itu, mereka pergi untuk mengambil Dia.
\par 22 Guru-guru agama yang datang dari Yerusalem berkata, "Ia kemasukan Beelzebul! Kepala roh-roh jahat itulah yang memberikan kuasa kepada-Nya untuk mengusir roh jahat."
\par 23 Maka Yesus memanggil orang banyak itu, dan menceritakan kepada mereka beberapa perumpamaan. "Bagaimana mungkin roh jahat mengusir roh jahat," kata Yesus.
\par 24 "Kalau suatu negara terpecah dalam golongan-golongan yang saling bermusuhan, negara itu tidak akan bertahan.
\par 25 Dan kalau dalam satu keluarga tidak ada persatuan dan anggota-anggotanya saling bermusuhan, keluarga itu akan hancur.
\par 26 Kalau di dalam kerajaan Iblis terjadi perpecahan dan permusuhan, kerajaan itu tidak tahan lama dan pasti akan lenyap.
\par 27 Tidak seorang pun dapat masuk ke dalam rumah seorang yang kuat dan merampas hartanya kalau ia tidak terlebih dahulu mengikat orang kuat itu. Sesudah itu, baru ia dapat merampas hartanya.
\par 28 Ketahuilah! Apabila orang berbuat dosa dan mengucap penghinaan, ia dapat diampuni.
\par 29 Tetapi apabila ia menghina Roh Allah, ia tidak dapat diampuni! Sebab penghinaan itu adalah dosa yang kekal."
\par 30 (Yesus mengatakan begitu sebab ada orang yang berkata bahwa Yesus kemasukan roh jahat.)
\par 31 Setelah itu ibu dan saudara-saudara Yesus datang. Mereka menunggu di luar dan menyuruh orang memanggil Yesus.
\par 32 Ketika itu banyak orang sedang duduk di sekeliling Yesus. Mereka berkata kepada-Nya, "Pak, ibu dan saudara-saudara Bapak ada di luar. Mereka mencari Bapak."
\par 33 Yesus menjawab, "Siapakah ibu-Ku? Siapakah saudara-saudara-Ku?"
\par 34 Kemudian Ia memandang kepada orang-orang yang duduk di sekeliling-Nya lalu berkata, "Inilah ibu dan saudara-saudara-Ku!
\par 35 Orang yang melakukan kehendak Allah, dialah saudara laki-laki, saudara perempuan, dan ibu-Ku."

\chapter{4}

\par 1 Yesus mengajar lagi di pantai Danau Galilea. Banyak sekali orang mengerumuni Dia. Karena itu Ia pergi duduk di dalam sebuah perahu di atas air, dan orang banyak itu berdiri di pinggir danau.
\par 2 Lalu Yesus mengajar banyak hal kepada mereka dengan memakai perumpamaan. Beginilah Ia mengajar mereka.
\par 3 "Dengarlah! Adalah seorang petani pergi menabur benih.
\par 4 Ketika ia sedang menabur, ada benih yang jatuh di jalan. Lalu burung datang dan benih itu dimakan habis.
\par 5 Ada juga yang jatuh di tempat berbatu-batu yang tanahnya sedikit. Benih-benih itu segera tumbuh karena kurang tanah,
\par 6 tetapi waktu matahari naik, tunas-tunas itu mulai layu kemudian kering dan mati karena akarnya tidak masuk cukup dalam.
\par 7 Ada pula benih yang jatuh di tengah semak berduri. Semak berduri itu tumbuh dan menghimpit tunas-tunas itu sehingga tidak berbuah.
\par 8 Tetapi ada juga benih yang jatuh di tanah yang subur. Benih itu tumbuh, lalu menjadi besar lalu berbuah, ada yang tiga puluh, ada yang enam puluh, dan ada yang seratus kali lipat."
\par 9 Sesudah menceritakan perumpamaan itu, Yesus berkata, "Kalian punya telinga, dengarkan!"
\par 10 Ketika Yesus sendirian, orang-orang yang sudah mendengar pengajaran-Nya datang kepada-Nya bersama-sama dengan kedua belas pengikut-Nya. Mereka minta Ia menerangkan arti perumpamaan itu.
\par 11 Maka Yesus berkata kepada mereka, "Kalian sudah diberi anugerah untuk mengetahui rahasia tentang bagaimana Allah memerintah. Tetapi orang-orang luar diajar dengan perumpamaan,
\par 12 supaya 'Mereka akan terus memperhatikan tetapi tidak tahu apa yang terjadi, mereka akan terus mendengar, tetapi tidak mengerti, ini terjadi supaya mereka jangan melihat dan mengerti dan jangan datang kepada Allah dan Allah mengampuni mereka.'"
\par 13 Kemudian Yesus berkata kepada mereka, "Kalau kalian tidak mengerti perumpamaan itu, bagaimana kalian dapat mengerti perumpamaan-perumpama yang lain?
\par 14 Penabur itu adalah orang yang menyiarkan berita dari Allah.
\par 15 Benih yang jatuh di jalan ibarat orang-orang yang mendengar kabar tentang bagaimana Allah memerintah. Begitu mendengar, Iblis datang dan mengambil apa yang sudah ditabur dalam hati mereka.
\par 16 Benih yang jatuh di tempat berbatu-batu ibarat orang-orang yang mendengar kabar itu, dan langsung menerimanya dengan senang hati.
\par 17 Tetapi kabar itu tidak berakar dalam hati mereka, sehingga tidak tahan lama. Begitu mereka menderita kesusahan atau penganiayaan karena kabar itu, langsung mereka murtad.
\par 18 Benih yang jatuh di tengah semak berduri itu ibarat orang-orang yang mendengar kabar itu,
\par 19 tetapi khawatir tentang hidup mereka dan ingin hidup mewah. Nafsu untuk berbagai hal masuk ke dalam hati mereka. Karena itu kabar dari Allah terhimpit di dalam hati mereka, sehingga tidak berbuah.
\par 20 Dan benih yang jatuh di tanah yang subur itu ibarat orang-orang yang mendengar kabar itu dan menerimanya, mereka berbuah banyak, ada yang tiga puluh, ada yang enam puluh, dan ada yang seratus kali lipat hasilnya."
\par 21 Selanjutnya Yesus berkata pula, "Pernahkah orang menyalakan lampu lalu menutupnya dengan tempayan, atau meletakkannya di bawah tempat tidur? Apakah ia tidak menaruh lampu itu pada kaki lampu?
\par 22 Tidak ada yang tersembunyi yang tidak akan kelihatan; dan tidak ada yang dirahasiakan yang tidak akan terbongkar.
\par 23 Sebab itu, kalau punya telinga, dengarkan!"
\par 24 Lalu Yesus berkata lagi, "Perhatikanlah apa yang kalian dengar ini! Ukuran yang kalian pakai untuk orang lain akan dipakai juga oleh Allah untuk kalian--dan bahkan lebih banyak lagi.
\par 25 Sebab orang yang sudah mempunyai, akan diberi lebih banyak lagi; tetapi orang yang tidak mempunyai, sedikit yang masih ada padanya akan diambil juga."
\par 26 Yesus menyambung pembicaraan-Nya lagi, "Bila Allah memerintah sebagai Raja, keadaannya dapat diumpamakan seperti seorang yang menabur benih di ladangnya.
\par 27 Malam hari ia tidur; siang hari ia bangun. Dan sementara itu benih-benih itu terus bertumbuh dan menjadi besar. Tetapi bagaimana caranya benih-benih itu tumbuh dan menjadi besar, orang itu tidak tahu.
\par 28 Tanah itulah yang dengan sendirinya mengeluarkan hasil: mula-mula tangkainya, kemudian bulirnya, lalu buahnya.
\par 29 Dan kalau gandum itu sudah masak, orang itu pun mulailah menyabit karena sudah waktunya untuk menuai."
\par 30 "Apabila Allah memerintah, dengan apa kita dapat membandingkannya?" tanya Yesus pula. "Contoh apakah yang dapat kita pakai untuk menerangkannya?
\par 31 Apabila Allah memerintah, keadaannya seperti perumpamaan ini: Sebuah biji sawi diambil seseorang lalu ditanam di tanah. Biji sawi adalah benih yang terkecil di dunia.
\par 32 Tetapi kalau sudah tumbuh, ia menjadi yang terbesar di antara tanaman-tanaman. Cabang-cabangnya sedemikian rindang sehingga burung-burung dapat datang, dan membuat sarang di bawah naungannya."
\par 33 Begitulah Yesus mengajar orang dengan menggunakan banyak perumpamaan seperti itu, sejauh mereka dapat mengerti.
\par 34 Yesus selalu memakai perumpamaan kalau Ia mengajar orang-orang itu. Tetapi kalau Ia sendiri dengan pengikut-pengikut-Nya, Ia menjelaskan semuanya kepada mereka.
\par 35 Pada sore hari itu juga, Yesus berkata kepada pengikut-pengikut-Nya, "Marilah kita berlayar ke seberang danau."
\par 36 Maka Yesus naik ke perahu, dan pengikut-pengikut-Nya meninggalkan orang banyak di tepi danau, lalu naik ke perahu yang sama. Perahu-perahu lain ada juga di situ. Kemudian Yesus dan pengikut-pengikut-Nya mulai berlayar.
\par 37 Tak lama kemudian datang angin keras. Ombak mulai memukul perahu dan masuk ke dalam sehingga perahu itu hampir penuh dengan air.
\par 38 Di buritan perahu itu, Yesus sedang tidur dengan kepala-Nya di atas bantal. Pengikut-pengikut-Nya membangunkan Dia. Mereka berkata, "Bapak Guru, apakah Bapak tidak peduli, kita celaka?"
\par 39 Yesus bangun, lalu membentak angin itu, dan berkata kepada danau, "Diam, tenanglah!" Angin pun reda, dan danau menjadi sangat tenang.
\par 40 Lalu Yesus berkata kepada pengikut-pengikut-Nya, "Mengapa kalian takut? Mengapa kalian tidak percaya kepada-Ku?"
\par 41 Maka mereka menjadi takut dan berkata satu sama lain, "Siapakah sebenarnya orang ini, sampai angin dan ombak pun taat kepada-Nya."

\chapter{5}

\par 1 Kemudian Yesus dan pengikut-pengikut-Nya sampai di seberang Danau Galilea, di daerah Gerasa.
\par 2 Begitu Yesus turun dari perahu, Ia didatangi seorang laki-laki yang keluar dari gua-gua kuburan.
\par 3 Orang itu dikuasai roh jahat dan tinggal di kuburan-kuburan. Ia sudah tidak dapat diikat lagi; walaupun dengan rantai.
\par 4 Sudah sering kaki dan tangannya dibelenggu, tetapi selalu rantai-rantai itu diputuskannya, dan besi pada kakinya dipatahkannya. Ia begitu kuat sehingga tidak seorang pun dapat menahannya.
\par 5 Siang malam ia berkeliaran di kuburan dan di bukit-bukit, sambil berteriak-teriak dan melukai badannya dengan batu.
\par 6 Ketika dari jauh ia melihat Yesus datang, ia berlari-lari lalu sujud di hadapan Yesus.
\par 7 Dengan suara yang keras ia berteriak, "Yesus, Anak Allah Yang Mahatinggi! Akan Kauapakan saya ini? Demi Allah, saya mohon, janganlah menyiksa saya!"
\par 8 9orang itu berkata begitu sebab Yesus berkata kepadanya, "Roh jahat, keluarlah dari orang ini!")
\par 9 Lalu Yesus bertanya kepadanya, "Siapakah namamu?" Orang itu menjawab, "Nama saya 'Legiun' --sebab kami ini banyak sekali!"
\par 10 Berulang kali ia minta dengan sangat supaya Yesus jangan menyuruh roh-roh jahat itu keluar dari daerah itu.
\par 11 Dekat tempat itu ada banyak sekali babi yang sedang mencari makan di lereng bukit.
\par 12 Roh-roh jahat itu memohon kepada Yesus, "Suruhlah kami masuk ke dalam babi-babi itu."
\par 13 Dan Yesus setuju. Jadi, roh-roh jahat itu keluar dari orang itu lalu masuk ke dalam babi-babi itu. Seluruh kawanan babi itu lari dan terjun dari pinggir jurang ke dalam danau lalu tenggelam--semuanya ada kira-kira dua ribu babi.
\par 14 Penjaga-penjaga babi itu lari, dan menyiarkan kabar itu di kota dan di desa sekitarnya. Lalu orang-orang keluar untuk melihat apa yang telah terjadi.
\par 15 Mereka datang kepada Yesus, lalu melihat orang yang tadinya kemasukan roh jahat itu, sedang duduk di situ. Ia sudah berpakaian, dan pikirannya juga sudah waras. Maka mereka semua menjadi takut.
\par 16 Orang-orang yang telah menyaksikan sendiri kejadian tentang orang itu dan babi-babi itu menceritakan apa yang telah terjadi.
\par 17 Lalu penduduk daerah itu minta supaya Yesus meninggalkan daerah itu.
\par 18 Waktu Yesus naik ke dalam perahu, orang yang tadinya dikuasai roh jahat itu minta kepada Yesus, supaya boleh ikut.
\par 19 Tetapi Yesus tidak setuju, kata-Nya, "Pulanglah dan beritahukan kepada sanak saudaramu apa yang sudah dilakukan Tuhan untukmu dan betapa baiknya Ia terhadapmu!"
\par 20 Orang itu pun pergi, dan mulai menceritakan di daerah Sepuluh Kota apa yang telah diperbuat Yesus kepadanya. Semua orang heran mendengarnya.
\par 21 Yesus kembali lagi ke seberang danau. Di tepi danau itu, banyak orang datang berkerumun di sekeliling Yesus.
\par 22 Datanglah seorang yang bernama Yairus. Ia adalah seorang pemimpin rumah ibadat di kota itu. Ketika ia melihat Yesus, ia sujud di depan-Nya,
\par 23 dan minta dengan sangat, "Pak, anak perempuan saya sakit parah. Sudilah datang untuk menjamahnya, supaya ia sembuh dan jangan mati!"
\par 24 Maka Yesus pun pergi bersama Yairus. Banyak orang mengikuti Dia dan mendesak-Nya dari semua jurusan.
\par 25 Di antaranya ada pula seorang wanita yang telah dua belas tahun sakit pendarahan yang berhubungan dengan haidnya.
\par 26 Semua kekayaannya sudah habis dipakai untuk membayar dokter-dokter, tetapi tidak ada yang dapat menyembuhkannya, malahan penyakitnya terus bertambah parah.
\par 27 Wanita itu sudah mendengar banyak tentang Yesus. Maka di tengah-tengah orang banyak itu, ia mendekati Yesus dari belakang,
\par 28 karena ia berpikir, "Asal saja saya menyentuh jubah-Nya, saya akan sembuh!"
\par 29 Ia menyentuh jubah Yesus, dan seketika itu juga pendarahannya berhenti. Ia merasa bahwa ia sudah sembuh.
\par 30 Pada saat itu juga Yesus merasa bahwa ada kekuatan yang keluar dari diri-Nya. Maka Ia menoleh kepada orang banyak itu dan bertanya, "Siapa yang menyentuh jubah-Ku?"
\par 31 Pengikut-pengikut-Nya berkata, "Bapak lihat sendiri ada begitu banyak orang yang berdesak-desakan. Dan Bapak masih bertanya, siapa yang menyentuh Bapak?"
\par 32 Tetapi Yesus terus saja melihat ke sekeliling-Nya untuk mencari orang yang telah menyentuh-Nya.
\par 33 Wanita itu yang tahu apa yang telah terjadi dengan dirinya, dengan gemetar dan ketakutan sujud di depan Yesus dan mengakui semuanya.
\par 34 Lalu Yesus berkata kepada wanita itu, "Anak-Ku, karena engkau percaya kepada-Ku, engkau sembuh! Pergilah dengan selamat. Engkau sudah sehat sama sekali!"
\par 35 Sementara Yesus masih berbicara, beberapa pesuruh datang dari rumah Yairus. "Putri Tuan sudah meninggal," kata mereka kepada Yairus. "Tak usah Tuan menyusahkan Bapak Guru lagi."
\par 36 Tanpa mempedulikan apa yang dikatakan orang-orang itu, Yesus berkata kepada Yairus, "Jangan takut, percaya saja!"
\par 37 Lalu Yesus berjalan terus, tetapi Dia tidak mengizinkan seorang pun mengikuti-Nya, kecuali Petrus serta Yakobus dan Yohanes bersaudara.
\par 38 Ketika mereka tiba di rumah Yairus, Yesus melihat keadaan hiruk-pikuk, dan mendengar tangisan dan ratapan yang keras.
\par 39 Lalu Yesus masuk dan berkata kepada mereka, "Mengapa ribut-ribut dan menangis? Anak itu tidak mati; ia hanya tidur!"
\par 40 Mereka menertawakan Yesus. Maka Ia menyuruh mereka semua keluar. Lalu Ia membawa ibu bapak anak itu dan ketiga pengikut-Nya masuk ke dalam kamar anak itu.
\par 41 Yesus memegang tangan anak itu lalu berkata kepadanya, "Talita kum," yang berarti, "Anak perempuan, Aku berkata kepadamu: bangun!"
\par 42 Anak gadis itu segera bangun, dan berjalan. (Umurnya sudah dua belas tahun.) Semua yang menyaksikan kejadian itu sangat kagum!
\par 43 Tetapi Yesus dengan keras melarang mereka memberitahukan hal itu kepada siapa pun. Lalu kata-Nya, "Berilah anak itu makan."

\chapter{6}

\par 1 Dari tempat itu, Yesus pulang bersama-sama dengan pengikut-pengikut-Nya ke kampung halaman-Nya.
\par 2 Pada hari Sabat Ia mulai mengajar di rumah ibadat. Ada banyak orang di situ. Waktu orang-orang itu mendengar pengajaran Yesus, mereka heran sekali. Mereka berkata, "Dari mana orang ini mendapat semuanya itu? Kebijaksanaan macam apakah ini yang ada pada-Nya? Bagaimanakah Ia dapat mengadakan keajaiban?
\par 3 Bukankah Ia ini tukang kayu, anak Maria, dan saudara dari Yakobus, Yoses, Yudas dan Simon? Ya, saudara-saudara perempuan-Nya pun ada tinggal di sini juga." Karena itu mereka menolak Dia.
\par 4 Lalu Yesus berkata kepada mereka, "Seorang nabi dihormati di mana-mana, kecuali di kampung halamannya, dan di antara sanak saudaranya dan keluarganya."
\par 5 Di tempat itu Yesus tidak dapat membuat sesuatu keajaiban pun, kecuali menyembuhkan beberapa orang sakit dengan meletakkan tangan-Nya ke atas mereka.
\par 6 Ia heran mereka tidak percaya. Kemudian Yesus pergi ke kampung-kampung di sekitar itu, dan mengajar.
\par 7 Ia memanggil kedua belas pengikut-Nya, lalu mengutus mereka berdua-dua dan memberi kuasa kepada mereka untuk mengusir roh-roh jahat.
\par 8 Ia memberi petunjuk ini kepada mereka, "Janganlah membawa apa-apa untuk perjalananmu, kecuali tongkat. Jangan membawa makanan atau kantong sedekah, ataupun uang.
\par 9 Pakailah sepatu, tetapi jangan membawa dua helai baju."
\par 10 Ia juga berkata, "Kalau kalian masuk ke suatu rumah, tinggallah di situ sampai kalian meninggalkan kota itu.
\par 11 Tetapi kalau kalian sampai di suatu tempat, dan orang-orang di situ tidak mau menerima dan mendengar kalian, tinggalkanlah tempat itu. Dan kebaskanlah debu dari kakimu sebagai peringatan terhadap mereka!"
\par 12 Maka kedua belas pengikut Yesus itu berangkat. Mereka pergi menyiarkan berita bahwa manusia harus bertobat dari dosa-dosanya.
\par 13 Pengikut-pengikut Yesus itu mengusir banyak roh jahat, mengoleskan minyak zaitun pada orang sakit dan menyembuhkan mereka.
\par 14 Berita-berita tentang semua kejadian itu sampai juga pada Raja Herodes, sebab nama Yesus sudah terkenal di mana-mana. Ada orang yang berkata, "Yohanes Pembaptis sudah hidup kembali! Itulah sebabnya Ia mempunyai kuasa melakukan keajaiban itu."
\par 15 Tetapi orang-orang lain berkata, "Dia Elia." Ada pula yang berkata, "Dia nabi, seperti salah seorang nabi zaman dahulu."
\par 16 Ketika Herodes mendengar itu, ia berkata, "Pasti ini Yohanes Pembaptis yang dahulu sudah kusuruh pancung kepalanya. Sekarang ia sudah hidup kembali!"
\par 17 Sebab sebelum itu Herodes telah menyuruh orang menangkap Yohanes, dan memasukkannya ke dalam penjara. Herodes berbuat begitu karena soal Herodias, istri saudaranya sendiri, yaitu Filipus. Sebab Herodes sudah mengawini Herodias,
\par 18 dan mengenai hal itu Yohanes sudah berulang-ulang menegur Herodes begini, "Tidak boleh engkau kawin dengan istri saudaramu itu!"
\par 19 Itulah sebabnya Herodias dendam kepada Yohanes dan ingin membunuh Yohanes, tetapi ia tidak dapat melakukan hal itu, karena dihalang-halangi oleh Herodes.
\par 20 Sebab Herodes telah menyuruh orang menjaga baik-baik keselamatan Yohanes di penjara, karena ia takut kepada Yohanes. Ia tahu Yohanes seorang yang baik yang diutus oleh Allah. Dan memang kalau Yohanes berbicara, Herodes suka juga mendengarkannya, meskipun ia menjadi gelisah sekali karenanya.
\par 21 Akhirnya Herodias mendapat kesempatan pada hari ulang tahun Herodes. Ketika itu Herodes mengadakan pesta untuk semua pejabat tinggi kerajaan, perwira-perwira dan tokoh-tokoh masyarakat Galilea.
\par 22 Di pesta itu anak gadis Herodias menari, dan tariannya itu sangat menyenangkan hati Herodes serta tamu-tamunya. Maka Herodes berkata kepada gadis itu, "Engkau suka apa, minta saja. Aku akan memberikannya kepadamu!"
\par 23 Lalu Herodes berjanji kepada gadis itu dengan sumpah. Herodes berkata, "Apa saja yang engkau minta akan kuberikan, bahkan separuh dari kerajaanku sekalipun!"
\par 24 Maka gadis itu keluar dan bertanya kepada ibunya, "Ibu, apa sebaiknya yang harus saya minta?" Ibunya menjawab, "Mintalah kepala Yohanes Pembaptis."
\par 25 Gadis itu segera kembali kepada Herodes dan berkata, "Saya minta kepala Yohanes Pembaptis diberikan kepada saya sekarang ini juga di atas sebuah baki!"
\par 26 Mendengar permintaan itu Herodes sangat sedih. Tetapi ia tidak dapat menolak permintaan itu karena ia sudah bersumpah di hadapan para tamunya.
\par 27 Jadi ia langsung memerintahkan seorang pengawalnya mengambil kepala Yohanes Pembaptis. Maka prajurit itu pergi ke penjara, lalu memancung kepala Yohanes.
\par 28 Kemudian ia membawa kepala itu di atas baki dan menyerahkannya kepada gadis itu. Dan gadis itu memberikannya pula kepada ibunya.
\par 29 Ketika pengikut-pengikut Yohanes mendengar hal itu, mereka pergi mengambil jenazah Yohanes, lalu menguburkannya.
\par 30 Rasul-rasul yang diutus oleh Yesus itu kemudian kembali lagi, dan berkumpul dengan Yesus. Mereka melaporkan kepada-Nya semua yang telah mereka perbuat dan ajarkan.
\par 31 Banyak sekali orang yang datang dan pergi, sehingga untuk makan pun Yesus dan pengikut-pengikut-Nya tidak sempat. Sebab itu Yesus berkata kepada pengikut-pengikut-Nya, "Marilah kita pergi ke tempat yang sunyi, di mana kita bisa sendirian dan kalian dapat beristirahat sebentar."
\par 32 Maka mereka pun berangkat dengan perahu menuju ke tempat yang sunyi.
\par 33 Tetapi banyak orang sudah melihat mereka meninggalkan tempat itu, dan tahu siapa mereka. Jadi, dari semua kota di wilayah itu, orang-orang berlari-lari melalui jalan darat mendahului Yesus dan pengikut-pengikut-Nya.
\par 34 Ketika Yesus turun dari perahu, Ia melihat orang banyak. Ia kasihan kepada mereka, sebab mereka seperti domba yang tidak punya gembala. Maka Ia pun mulai mengajarkan banyak hal kepada mereka.
\par 35 Ketika sudah petang, pengikut-pengikut Yesus berkata kepada-Nya, "Sudah hampir malam dan tempat ini terpencil.
\par 36 Lebih baik Bapak menyuruh orang-orang ini pergi, supaya mereka dapat membeli makanan di desa-desa dan kampung-kampung di sekitar ini."
\par 37 Tetapi Yesus menjawab, "Kalian saja memberi mereka makan." "Wah, apakah kami harus pergi membeli roti seharga dua ratus uang perak untuk memberi makan orang-orang ini?" begitu kata pengikut-pengikut Yesus itu.
\par 38 Lalu tanya Yesus, "Ada berapa roti pada kalian? Coba pergi lihat." Sesudah mereka pergi melihat, mereka berkata, "Ada lima roti dan ada dua ikan juga."
\par 39 Lalu Yesus menyuruh semua orang itu duduk berkelompok-kelompok di rumput yang hijau.
\par 40 Orang-orang itu pun duduk dengan teratur, berkelompok-kelompok. Ada yang seratus orang sekelompok, dan ada juga yang lima puluh orang sekelompok.
\par 41 Kemudian Yesus mengambil lima roti dan dua ikan itu, lalu menengadah ke langit dan mengucap terima kasih kepada Allah. Sesudah itu, Ia membelah-belah roti itu dengan tangan-Nya dan memberikannya kepada pengikut-pengikut-Nya untuk dibagi-bagikan kepada orang banyak itu. Dan dua ikan itu dibagi-bagikan juga kepada mereka semua.
\par 42 Mereka makan sampai kenyang.
\par 43 Kemudian kelebihan makanan itu dikumpulkan oleh pengikut-pengikut Yesus--semuanya ada dua belas bakul penuh.
\par 44 Orang laki-laki yang makan roti itu ada kira-kira lima ribu.
\par 45 Sesudah itu Yesus segera menyuruh pengikut-pengikut-Nya berangkat dengan perahu mendahului Dia ke Betsaida di seberang danau, sementara Ia menyuruh orang banyak itu pulang.
\par 46 Setelah melepaskan orang banyak itu, Yesus pergi ke bukit untuk berdoa.
\par 47 Ketika sudah malam, perahu pengikut-pengikut Yesus telah berada di tengah-tengah danau, sedangkan Yesus masih berada di darat.
\par 48 Ia melihat mereka bersusah payah mendayung perahu itu karena angin berlawanan arah dengan perahu. Sebab itu, kira-kira antara pukul tiga dan pukul enam pagi, Ia datang kepada mereka berjalan di atas air. Dan Ia berjalan terus seolah-olah akan melewati mereka.
\par 49 Waktu mereka melihat bahwa Ia berjalan di atas air, mereka mengira Dia hantu,
\par 50 sehingga mereka menjerit-jerit ketakutan. Sebab mereka semuanya melihat Dia dan mereka sangat terkejut. Tetapi langsung Yesus berbicara kepada mereka, "Tenanglah! Aku Yesus. Jangan takut!"
\par 51 Lalu Ia naik ke perahu mereka, dan angin pun reda. Pengikut-pengikut Yesus heran sekali.
\par 52 Keajaiban dengan lima buah roti itu belum lagi dipahami oleh mereka. Sukar bagi mereka untuk mengerti.
\par 53 Waktu tiba di seberang danau, mereka berlabuh di pantai Genesaret.
\par 54 Ketika mereka keluar dari perahu, orang-orang melihat bahwa yang datang itu Yesus.
\par 55 Lalu mereka berlari-lari ke mana-mana di seluruh wilayah itu, dan mulai membawa orang-orang sakit di atas tikar kepada Yesus. Kalau mereka mendengar bahwa Yesus berada di suatu tempat, mereka membawa orang-orang sakit ke sana.
\par 56 Di mana saja Yesus datang--baik di kampung, di kota atau di desa--di situ orang selalu datang dan menaruh orang-orang sakit mereka di alun-alun. Lalu mereka minta dengan sangat supaya orang-orang sakit itu diizinkan menyentuh jubah Yesus, biar hanya ujungnya. Semua yang menyentuhnya, menjadi sembuh.

\chapter{7}

\par 1 Sekelompok orang Farisi dan beberapa guru agama dari Yerusalem, datang kepada Yesus.
\par 2 Mereka melihat beberapa pengikut Yesus makan dengan tangan yang tidak bersih secara agama, yaitu tanpa terlebih dahulu mencuci tangan menurut peraturan agama.
\par 3 Orang-orang Farisi, begitu juga semua orang Yahudi, setia sekali mengikuti adat istiadat nenek moyang mereka. Mereka tidak akan makan, sebelum mencuci tangan menurut cara-cara tertentu.
\par 4 Apa yang dibeli di pasar tidak akan dimakan, sebelum dicuci terlebih dahulu. Dan banyak peraturan lain dari nenek moyang mereka yang mereka pegang teguh; seperti misalnya peraturan mencuci gelas, mangkuk, dan perkakas-perkakas tembaga.
\par 5 Sebab itu orang-orang Farisi dan guru-guru agama itu bertanya kepada Yesus, "Mengapa pengikut-pengikut-Mu itu makan dengan tangan yang tidak dicuci? Apa sebab mereka tidak menuruti adat istiadat nenek moyang kita?"
\par 6 Yesus menjawab, "Kalian orang-orang munafik! Tepat sekali apa yang dinubuatkan Yesaya tentang kalian, yaitu, 'Begini kata Allah, Orang-orang itu hanya menyembah Aku dengan kata-kata, tetapi hati mereka jauh dari Aku.
\par 7 Percuma mereka menyembah Aku, sebab peraturan manusia mereka ajarkan seolah-olah itu peraturan-Ku!'
\par 8 Perintah-perintah Allah kalian abaikan, dan peraturan-peraturan manusia kalian pegang kuat-kuat."
\par 9 Lalu Yesus berkata lagi, "Kalian pandai sekali menolak perintah Allah supaya dapat mempertahankan ajaran sendiri.
\par 10 Musa sudah memberi perintah ini, 'Hormatilah ayah dan ibumu,' dan 'Barangsiapa mengata-ngatai ayah ibunya, harus dihukum mati.'
\par 11 Tetapi kalian mengajarkan: Kalau orang berkata kepada orang tuanya, 'Apa yang seharusnya saya berikan kepada ayah dan ibu, sudah saya persembahkan kepada Allah,'
\par 12 maka kalian membebaskan orang itu dari kewajibannya menolong ayah ibunya.
\par 13 Jadi dengan ajaranmu sendiri yang kalian berikan kepada orang-orang, kalian meniadakan perkataan Allah. Masih banyak hal seperti ini yang kalian lakukan."
\par 14 Lalu Yesus memanggil orang banyak itu sekali lagi dan berkata kepada mereka, "Dengarlah supaya mengerti!
\par 15 Tidak ada sesuatu dari luar yang masuk ke dalam orang yang dapat membuat orang itu najis. Sebaliknya, yang keluar dari seseorang, itulah yang membuat dia najis.
\par 16 (Jadi, kalau punya telinga, dengarkan.)"
\par 17 Ketika Yesus meninggalkan orang banyak itu dan masuk rumah, pengikut-pengikut-Nya bertanya kepada-Nya tentang maksud perumpamaan itu.
\par 18 Maka Yesus berkata kepada mereka, "Apakah kalian belum juga mengerti? Apakah kalian tidak bisa mengerti bahwa yang masuk ke dalam seseorang tidak bisa membuat orang itu najis?
\par 19 Sebab yang masuk itu tidak lewat hati, tetapi lewat perut, dan kemudian keluar lagi." Dengan kata-kata itu Yesus menyatakan bahwa semua makanan halal.
\par 20 Lalu Yesus berkata lagi, "Yang keluar dari orang, itulah yang mengotorkan dia.
\par 21 Sebab dari dalam, yaitu dari dalam hati, timbul pikiran-pikiran jahat yang menyebabkan orang berbuat cabul, mencuri, membunuh,
\par 22 berzinah, menipu, memfitnah, serta melakukan segala sesuatu yang jahat, menjadi serakah, tidak sopan, iri hati, sombong, dan susah diajar.
\par 23 Semua yang jahat itu timbul dari dalam, dan itulah yang menjadikan orang najis."
\par 24 Kemudian Yesus meninggalkan tempat itu, dan pergi ke daerah dekat kota Tirus. Ia masuk ke dalam sebuah rumah dan tidak mau bahwa orang tahu Ia berada di situ. Tetapi Ia tidak dapat menyembunyikan diri.
\par 25 Seorang ibu, yang anak perempuannya kemasukan roh jahat, mendengar tentang Yesus. Ia datang kepada Yesus dan sujud di depan-Nya,
\par 26 sambil mohon supaya Yesus mengusir roh jahat dari anak itu. Wanita itu bukan orang Yahudi, lahir di daerah Fenisia di Siria.
\par 27 Yesus berkata kepadanya, "Anak-anak harus diberi makan terlebih dahulu. Tidak baik mengambil makanan anak-anak untuk dilemparkan kepada anjing."
\par 28 "Tuan," jawab wanita itu, "anjing-anjing di bawah meja pun makan sisa-sisa yang dijatuhkan anak-anak!"
\par 29 Lalu Yesus berkata kepadanya, "Karena jawabanmu itu, pulanglah; roh jahat sudah keluar dari anakmu!"
\par 30 Ibu itu pulang. Di rumah, ia mendapati anaknya sedang berbaring di tempat tidur, dan roh jahat benar-benar sudah keluar dari anak itu.
\par 31 Kemudian Yesus meninggalkan daerah Tirus, dan meneruskan perjalanan-Nya melalui Sidon ke Danau Galilea. Ia mengambil jalan lewat daerah Sepuluh Kota.
\par 32 Di situ orang membawa kepada-Nya seorang yang bisu tuli. Mereka minta Yesus meletakkan tangan-Nya ke atas orang itu.
\par 33 Yesus membawa orang itu menyendiri dari orang banyak, lalu meletakkan jari-Nya ke dalam kedua telinga orang itu. Lantas Yesus meludah, dan menjamah lidah orang itu.
\par 34 Sesudah itu Yesus menengadah ke langit, lalu bernapas keras dan berkata kepada orang itu, "Efata," yang berarti, "Terbukalah!"
\par 35 Telinga orang itu terbuka dan lidahnya menjadi lemas kembali, dan ia mulai berbicara dengan mudah.
\par 36 Lalu Yesus melarang mereka sekalian untuk menceritakan hal itu kepada siapa pun. Tetapi semakin Yesus melarang, semakin pula mereka menyebarkannya.
\par 37 Dan semua orang yang mendengar itu heran sekali. Mereka berkata, "Semuanya dibuat-Nya dengan baik! Bahkan Ia membuat orang tuli mendengar dan orang bisu berbicara!"

\chapter{8}

\par 1 Tidak berapa lama kemudian, ada lagi sekelompok orang banyak datang berkumpul. Karena mereka tidak punya makanan, Yesus memanggil pengikut-pengikut-Nya dan berkata,
\par 2 "Aku kasihan kepada orang banyak ini. Sudah tiga hari lamanya mereka bersama-sama-Ku, dan sekarang mereka tidak punya makanan.
\par 3 Kalau Aku menyuruh mereka pulang dengan perut kosong, mereka akan pingsan di tengah jalan. Apalagi di antara mereka ada yang datang dari jauh."
\par 4 Pengikut-pengikut Yesus menjawab, "Di tempat yang terpencil ini, di manakah orang bisa mendapat cukup makanan untuk semua orang ini?"
\par 5 "Ada berapa banyak roti pada kalian?" tanya Yesus. "Tujuh," jawab mereka.
\par 6 Maka Yesus menyuruh orang banyak itu duduk di atas tanah, lalu Ia mengambil ketujuh roti itu dan mengucap syukur kepada Allah. Kemudian Ia membelah-belah roti itu dengan tangan-Nya dan memberikannya kepada pengikut-pengikut-Nya untuk dibagi-bagikan kepada orang-orang. Maka pengikut-pengikut-Nya melakukannya.
\par 7 Mereka mempunyai beberapa ekor ikan kecil juga. Yesus mengucap syukur kepada Allah atas ikan-ikan itu, lalu menyuruh pengikut-pengikut-Nya membagi-bagikan ikan-ikan itu juga.
\par 8 Mereka makan sampai kenyang--ada kira-kira empat ribu orang yang makan. Kemudian pengikut-pengikut Yesus mengumpulkan kelebihan makanan--tujuh bakul penuh. Lalu Yesus menyuruh orang-orang itu pulang,
\par 9 [8:8]
\par 10 dan Ia dengan pengikut-pengikut-Nya langsung naik perahu dan pergi ke daerah Dalmanuta.
\par 11 Beberapa orang Farisi datang kepada Yesus, dan mulai berdebat dengan Dia untuk menjebak-Nya. Mereka minta Yesus membuat keajaiban sebagai tanda bahwa Ia datang dari Allah.
\par 12 Yesus mengeluh lalu menjawab, "Apa sebab orang-orang zaman ini minta Aku membuat keajaiban? Tidak! Aku tidak akan memberikan tanda semacam itu kepada mereka!"
\par 13 Lalu Yesus meninggalkan mereka, dan masuk ke dalam perahu; kemudian berangkat ke seberang danau itu.
\par 14 Pengikut-pengikut Yesus lupa membawa cukup roti. Mereka hanya mempunyai sebuah roti di perahu.
\par 15 "Hati-hatilah terhadap ragi orang-orang Farisi dan ragi Herodes," kata Yesus kepada mereka.
\par 16 Maka pengikut-pengikut Yesus itu mulai mempercakapkan hal itu. Mereka berkata, "Ia berkata begitu, sebab kita tidak punya roti."
\par 17 Yesus tahu apa yang mereka persoalkan. Sebab itu Ia bertanya kepada mereka, "Mengapa kalian persoalkan tentang tidak punya roti? Apakah kalian tidak tahu dan belum mengerti juga? Begitu tumpulkah pikiranmu?
\par 18 Kalian punya mata--mengapa tidak melihat? Kalian punya telinga--mengapa tidak mendengar? Tidakkah kalian ingat
\par 19 akan lima roti itu yang Aku belah-belah untuk lima ribu orang? Berapa bakul penuh kelebihan makanan yang kalian kumpulkan?" "Dua belas," jawab mereka.
\par 20 "Dan waktu Aku membelah-belah tujuh roti untuk empat ribu orang," tanya Yesus lagi, "berapa bakul kelebihan makanan yang kalian kumpulkan?" "Tujuh," jawab mereka.
\par 21 "Nah, belumkah kalian mengerti juga?" kata Yesus lagi.
\par 22 Mereka sampai di Betsaida. Di situ orang membawa seorang buta kepada Yesus, dan minta supaya Ia menjamah orang buta itu untuk menyembuhkannya.
\par 23 Maka Yesus memegang tangan orang buta itu dan menuntun dia ke luar kota itu. Kemudian Yesus meludahi mata orang itu. Ia meletakkan tangan-Nya pada mata orang itu, lalu bertanya kepadanya, "Dapatkah engkau melihat sesuatu sekarang?"
\par 24 Orang itu melihat ke depan, lalu berkata, "Ya. Saya melihat orang berjalan-jalan; tetapi mereka kelihatan seperti pohon."
\par 25 Yesus meletakkan lagi tangan-Nya pada mata orang itu. Kali ini orang itu berusaha melihat dengan sedapat-dapatnya. Matanya sembuh, dan ia melihat semuanya dengan jelas sekali.
\par 26 Lalu Yesus berkata kepadanya, "Pulanglah, dan jangan kembali ke kota itu."
\par 27 Yesus dan pengikut-pengikut-Nya pergi ke desa-desa di sekitar Kaisarea Filipi. Di tengah jalan Yesus bertanya kepada mereka, "Menurut kata orang, siapakah Aku ini?"
\par 28 Mereka menjawab, "Ada yang berkata: Yohanes Pembaptis; ada juga yang berkata Elia, dan yang lain lagi berkata: salah seorang nabi."
\par 29 "Tetapi menurut kalian sendiri, Aku ini siapa?" tanya Yesus. Petrus menjawab, "Bapak adalah Raja Penyelamat!"
\par 30 Lalu Yesus memperingatkan mereka, supaya tidak memberitahukan kepada siapa pun tentang diri-Nya.
\par 31 Setelah itu, Yesus mulai mengajar pengikut-pengikut-Nya bahwa Anak Manusia mesti menderita banyak, dan akan ditentang oleh pemimpin-pemimpin, imam-imam kepala, dan guru-guru agama. Ia akan dibunuh, tetapi pada hari ketiga Ia akan dibangkitkan kembali.
\par 32 Dengan jelas sekali Yesus memberitahukan hal itu kepada pengikut-pengikut-Maka Petrus menarik Yesus ke samping, dan menegur Dia.
\par 33 Tetapi Yesus menoleh dan memandang pengikut-pengikut-Nya, lalu menegur Petrus, "Pergi dari sini, Penggoda! Pikiranmu itu pikiran manusia; bukan pikiran Allah!"
\par 34 Kemudian Yesus memanggil orang banyak yang ada di situ bersama-sama dengan pengikut-pengikut-Nya. Lalu Ia berkata kepada mereka, "Orang yang mau mengikuti Aku, harus melupakan kepentingannya sendiri, kemudian memikul salibnya, dan terus mengikuti Aku.
\par 35 Sebab orang yang mau mempertahankan hidupnya, akan kehilangan hidupnya. Tetapi orang yang mengurbankan hidupnya untuk-Ku dan untuk Kabar Baik dari Allah, akan menyelamatkannya.
\par 36 Apa untungnya bagi seseorang, kalau seluruh dunia ini menjadi miliknya, tetapi ia kehilangan hidupnya?
\par 37 Dapatkah hidup itu ditukar dengan sesuatu?
\par 38 Kalau seseorang malu mengakui Aku dan pengajaran-Ku pada zaman durhaka dan jahat ini, Anak Manusia juga akan malu mengakui orang itu, pada waktu Ia datang nanti dengan kuasa Bapa-Nya, disertai malaikat-malaikat yang suci."

\chapter{9}

\par 1 "Ketahuilah!" kata Yesus. "Dari antara kalian di sini ada yang tidak akan mati, sebelum melihat Allah memerintah dengan kuasa!"
\par 2 Enam hari kemudian Yesus membawa Petrus dengan Yakobus dan saudaranya Yohanes, menyendiri ke sebuah gunung yang tinggi. Di depan mata mereka Yesus berubah rupa.
\par 3 Pakaian-Nya menjadi putih berkilauan. Tidak ada seorang penatu pun di dunia ini yang dapat mencuci seputih itu.
\par 4 Kemudian ketiga orang pengikut-Nya itu melihat Yesus bercakap-cakap dengan Elia dan Musa.
\par 5 Maka Petrus berkata kepada Yesus, "Pak Guru, enak sekali kita di sini. Baiklah kami mendirikan tiga kemah: satu untuk Bapak, satu untuk Musa, dan satu lagi untuk Elia."
\par 6 Sebenarnya Petrus tidak tahu apa yang ia harus katakan, sebab ia dengan kedua temannya sedang ketakutan sekali.
\par 7 Kemudian awan meliputi mereka dan dari awan itu terdengar suara yang berkata, "Inilah Anak-Ku yang Kukasihi. Dengarkan Dia!"
\par 8 Cepat-cepat mereka melihat sekeliling mereka, dan tidak lagi melihat siapa pun di situ bersama mereka, kecuali Yesus saja.
\par 9 Waktu mereka turun dari gunung itu, Yesus memperingatkan mereka, "Jangan memberitahukan kepada siapa pun apa yang kalian lihat tadi sebelum Anak Manusia dibangkitkan dari kematian."
\par 10 Mereka mentaati pesan itu, tetapi di antara mereka sendiri mereka mulai mempercakapkan apa maksud Yesus dengan "hidup kembali dari kematian".
\par 11 Maka mereka bertanya kepada-Nya, "Mengapa guru-guru agama berkata bahwa Elia mesti datang terlebih dahulu?"
\par 12 Yesus menjawab, "Elia memang datang terlebih dahulu untuk membereskan segala sesuatu. Tetapi bagaimanakah halnya dengan Anak Manusia? Apa sebab di dalam Alkitab tertulis bahwa Ia akan banyak menderita dan dihina orang?
\par 13 Tetapi Aku berkata kepadamu: Elia sudah datang, dan orang-orang memperlakukan dia semau mereka. Itu cocok dengan yang sudah tertulis dalam Alkitab tentang dirinya."
\par 14 Ketika Yesus dan ketiga pengikut-Nya sudah berada kembali bersama pengikut-pengikut yang lainnya, mereka melihat banyak orang di situ. Beberapa guru agama sedang berdebat dengan pengikut-pengikut Yesus itu.
\par 15 Begitu orang-orang itu melihat Yesus, mereka tercengang, lalu berlari-lari menyambut Dia.
\par 16 "Apa yang kalian persoalkan dengan guru-guru agama itu?" tanya Yesus kepada pengikut-pengikut-Nya.
\par 17 Seorang dari antara orang banyak itu menjawab, "Bapak Guru, saya membawa anak saya kepada Bapak. Dia bisu karena kemasukan roh jahat.
\par 18 Kalau roh itu menyerang dia, badannya dibanting-banting ke tanah, mulutnya berbusa, giginya mengertak dan seluruh tubuhnya menjadi kaku. Saya minta pengikut-pengikut Bapak mengusir roh jahat itu, tetapi mereka tidak dapat melakukannya."
\par 19 Maka Yesus berkata kepada mereka, "Bukan main kalian ini! Kalian sungguh-sungguh orang yang tidak percaya. Sampai kapan Aku harus tinggal bersama-sama kalian dan bersabar terhadap kalian? Bawa anak itu ke mari!"
\par 20 Mereka pun membawa anak itu kepada Yesus. Begitu roh jahat itu melihat Yesus, ia membuat badan anak itu kejang-kejang sehingga anak itu jatuh terguling-guling ke tanah. Mulutnya berbusa.
\par 21 Lalu Yesus bertanya kepada bapaknya, "Sudah berapa lama ia begini?" "Sejak ia masih kecil!" jawab bapaknya.
\par 22 "Sudah sering roh jahat itu berusaha membunuh dia dengan menjatuhkannya ke dalam api atau ke dalam air. Tetapi kalau Bapak dapat menolong, sudilah Bapak mengasihani kami dan menolong kami!"
\par 23 "Apa katamu? Kalau Bapak dapat?" jawab Yesus. "Segalanya dapat, asal orang percaya!"
\par 24 Langsung ayah itu berteriak, "Tuhan, saya percaya, tetapi iman saya kurang. Tolonglah saya supaya lebih percaya lagi!"
\par 25 Waktu Yesus melihat bahwa orang banyak mulai datang berkerumun, Ia memerintahkan roh jahat itu dengan berkata, "Roh tuli dan bisu, Aku perintahkan kau keluar dari anak ini dan jangan sekali-kali masuk lagi ke dalamnya!"
\par 26 Roh jahat itu berteriak, lalu membuat badan anak itu kejang-kejang, kemudian keluar dari anak itu. Anak itu kelihatan seperti mayat sehingga semua orang berkata, "Ia sudah mati!"
\par 27 Tetapi Yesus memegang tangannya dan menolong dia bangun. Anak itu pun bangun.
\par 28 Setelah Yesus di rumah, pengikut-pengikut-Nya datang secara tersendiri kepada-Nya dan bertanya, "Apa sebab kami tidak dapat mengusir roh jahat itu?"
\par 29 Yesus menjawab, "Roh jahat semacam itu tidak dapat diusir dengan cara apa pun, selain dengan doa."
\par 30 Yesus dan pengikut-pengikut-Nya meninggalkan tempat itu dan meneruskan perjalanan melalui Galilea. Yesus tidak mau orang tahu di mana Ia berada,
\par 31 sebab Ia sedang mengajar pengikut-pengikut-Nya. "Anak Manusia akan diserahkan kepada kuasa manusia," begitu kata Yesus, "dan Ia akan dibunuh, tetapi pada hari ketiga Ia akan bangkit!"
\par 32 Pengikut-pengikut-Nya tidak mengerti yang diajarkan oleh Yesus, tetapi mereka takut bertanya kepada-Nya.
\par 33 Mereka sampai di Kapernaum. Setelah di rumah, Yesus bertanya kepada pengikut-pengikut-Nya, "Kalian mempersoalkan apa di tengah jalan tadi?"
\par 34 Mereka tidak menjawab, sebab di tengah jalan mereka bertengkar mengenai siapa yang terbesar.
\par 35 Yesus duduk, lalu memanggil kedua belas pengikut-Nya itu. Ia berkata kepada mereka, "Orang yang mau menjadi yang nomor satu, ia harus menjadi yang terakhir dan harus menjadi pelayan semua orang."
\par 36 Kemudian Yesus mengambil seorang anak kecil, dan membuat anak itu berdiri di depan mereka semua. Yesus memeluk anak itu dan berkata kepada pengikut-pengikut-Nya,
\par 37 "Orang yang menerima seorang anak seperti ini karena Aku, berarti menerima Aku. Dan orang yang menerima Aku, ia bukan menerima Aku saja, tetapi menerima juga Dia yang mengutus Aku."
\par 38 Lalu Yohanes berkata kepada Yesus, "Pak Guru, kami melihat seseorang mengusir setan atas nama Bapak. Dan kami melarang dia sebab ia bukan dari kita."
\par 39 Tetapi Yesus berkata, "Jangan melarang dia, sebab tidak seorang pun yang membuat keajaiban atas nama-Ku, dapat langsung menjelek-jelekkan Aku.
\par 40 Sebab orang yang tidak melawan kita, berarti berpihak kepada kita.
\par 41 Ingatlah! Orang yang memberi minum kepadamu oleh karena kalian pengikut Raja Penyelamat, ia pasti akan menerima upahnya."
\par 42 "Siapa menyebabkan salah satu dari orang-orang yang kecil ini tidak percaya lagi kepada-Ku, lebih baik kalau batu penggilingan diikatkan pada lehernya, dan ia dibuang ke dalam laut.
\par 43 Kalau tanganmu membuat engkau berdosa, potonglah tangan itu! Lebih baik engkau hidup dengan Allah tanpa sebelah tangan daripada engkau dengan kedua belah tanganmu masuk ke neraka, yaitu api yang abadi.
\par 44 (Di sana api tidak bisa padam, dan ulat tidak bisa mati.)
\par 45 Dan kalau kakimu membuat engkau berdosa, potonglah kaki itu. Lebih baik engkau hidup dengan Allah tanpa sebelah kakimu, daripada engkau dengan kedua belah kakimu dibuang ke dalam neraka.
\par 46 (Di sana api tidak bisa padam dan ulat tidak bisa mati.)
\par 47 Kalau matamu menyebabkan engkau berdosa, cungkillah mata itu! Lebih baik engkau masuk Dunia Baru Allah tanpa satu mata, daripada engkau dengan kedua belah matamu dibuang ke dalam neraka.
\par 48 Di sana api tidak bisa padam dan ulat tidak bisa mati.
\par 49 Setiap orang akan dimurnikan dengan api, seperti kurban disucikan dengan garam.
\par 50 Garam itu baik, tetapi kalau menjadi tawar, mungkinkah diasinkan kembali? Jadi, hendaklah kalian menjadi seperti garam--hiduplah bersama-sama dengan rukun."

\chapter{10}

\par 1 Kemudian Yesus meninggalkan tempat itu, dan pergi ke daerah Yudea dan daerah di seberang Sungai Yordan. Orang banyak datang lagi berkerumun sekeliling Yesus. Dan seperti biasa Yesus mengajar mereka.
\par 2 Beberapa orang Farisi datang juga untuk menjebak Yesus. Mereka bertanya, "Menurut hukum agama kita, apakah boleh orang menceraikan istrinya?"
\par 3 Yesus menjawab, "Musa memberi perintah apa kepada kalian?"
\par 4 "Musa mengizinkan orang menceraikan istrinya, asal menulis surat cerai dahulu," jawab mereka.
\par 5 "Musa menulis perintah itu sebab kalian terlalu sukar diajar," kata Yesus kepada mereka.
\par 6 "Tetapi pada permulaannya, pada waktu Allah menciptakan manusia, dikatakan bahwa 'Allah menjadikan mereka laki-laki dan wanita.
\par 7 Itu sebabnya laki-laki akan meninggalkan ibu bapaknya dan bersatu dengan istrinya,
\par 8 maka keduanya menjadi satu.' Jadi, mereka bukan lagi dua orang, melainkan satu.
\par 9 Itu sebabnya, apa yang sudah disatukan oleh Allah, tidak boleh diceraikan oleh manusia!"
\par 10 Setelah mereka masuk rumah, pengikut-pengikut-Nya bertanya kepada Yesus tentang hal itu.
\par 11 Yesus berkata kepada mereka, "Siapa menceraikan istrinya lalu kawin dengan wanita lain, orang itu berzinah terhadap istrinya yang pertama itu.
\par 12 Begitu juga wanita yang menceraikan suaminya lalu kawin dengan lelaki yang lain, ia pun berzinah."
\par 13 Ada orang-orang membawa anak-anak kepada Yesus supaya Ia menjamah dan memberkati mereka. Tetapi pengikut-pengikut Yesus memarahi orang-orang itu.
\par 14 Melihat hal itu, Yesus marah lalu Ia berkata kepada pengikut-pengikut-Nya "Biarkan anak-anak itu datang kepada-Ku! Jangan melarang mereka, sebab orang-orang seperti inilah yang menjadi anggota umat Allah.
\par 15 Ingatlah ini! Siapa tidak menghadap Allah seperti seorang anak, tidak akan menjadi anggota umat Allah."
\par 16 Sesudah mengatakan demikian, Yesus memeluk anak-anak itu, kemudian Ia meletakkan tangan-Nya ke atas mereka masing-masing dan memberkati mereka.
\par 17 Waktu Yesus meneruskan lagi perjalanan-Nya, seorang datang berlari-lari kepada Yesus. Orang itu sujud di hadapan Yesus dan bertanya, "Guru yang baik, saya harus berbuat apa supaya dapat menerima hidup sejati dan kekal?"
\par 18 "Mengapa engkau mengatakan Aku baik?" tanya Yesus. "Tidak ada yang baik, selain Allah sendiri.
\par 19 Engkau sudah tahu perintah-perintah Allah, 'Jangan membunuh, jangan berzinah, jangan mencuri, jangan bersaksi dusta, jangan menipu, hormatilah ayah dan ibumu.'"
\par 20 "Bapak Guru," kata orang itu, "semua perintah itu sudah saya turuti sejak muda."
\par 21 Yesus memandang orang itu dengan sayang lalu berkata, "Tinggal satu hal lagi yang engkau perlukan. Pergilah jual semua milikmu; berikanlah uangnya kepada orang miskin, dan engkau akan mendapat harta di surga. Sesudah itu datanglah mengikuti Aku."
\par 22 Mendengar Yesus berkata begitu, orang itu kecewa, lalu meninggalkan tempat itu dengan susah hati karena ia kaya sekali.
\par 23 Maka Yesus memandang pengikut-pengikut-Nya lalu berkata kepada mereka, "Sukar sekali untuk orang kaya menjadi anggota umat Allah!"
\par 24 Pengikut-pengikut-Nya heran mendengar perkataan Yesus itu. Tetapi Yesus berkata pula, "Anak-anak-Ku, memang sukar untuk menjadi anggota umat Allah!
\par 25 Lebih mudah seekor unta masuk lubang jarum daripada seorang kaya masuk Dunia Baru Allah."
\par 26 Kata-kata Yesus itu membuat pengikut-pengikut-Nya heran, sehingga mereka bertanya satu sama lain, "Kalau begitu, siapa yang bisa selamat?"
\par 27 Yesus memandang mereka dan menjawab, "Bagi manusia itu mustahil, tetapi tidak mustahil bagi Allah; semua mungkin bagi Allah."
\par 28 Lalu Petrus berkata, "Lihatlah, kami sudah meninggalkan segala-galanya untuk mengikuti Bapak."
\par 29 Kata Yesus, "Percayalah: orang yang sudah meninggalkan rumah tangganya, atau saudaranya yang laki-laki atau perempuan, atau ibunya, atau bapaknya, atau anak-anaknya, ataupun sawah ladangnya karena Aku dan karena Kabar Baik dari Allah,
\par 30 orang itu akan menerima lebih banyak pada masa sekarang ini. Ia akan mendapat seratus kali lebih banyak rumah, saudara laki-laki, saudara perempuan, ibu, anak-anak, sawah ladang, --dan siksaan juga. Dan nanti di zaman yang akan datang, orang itu akan menerima hidup sejati dan kekal.
\par 31 Tetapi banyak orang yang sekarang ini pertama akan menjadi yang terakhir dan banyak yang sekarang ini terakhir akan menjadi yang pertama."
\par 32 Yesus dan pengikut-pengikut-Nya sedang dalam perjalanan ke Yerusalem. Yesus berjalan di depan, dan pengikut-pengikut-Nya cemas semuanya. Dan orang-orang yang mengikuti mereka dari belakang pun merasa takut. Kemudian Yesus memanggil lagi pengikut-pengikut-Nya tersendiri dan memberitahukan kepada mereka apa yang akan terjadi nanti atas diri-Nya.
\par 33 "Dengarlah," kata-Nya, "kita sekarang sedang menuju Yerusalem. Di sana Anak Manusia akan diserahkan kepada imam-imam kepala dan guru-guru agama. Ia akan dihukum mati, kemudian diserahkan kepada orang-orang bukan Yahudi.
\par 34 Mereka akan mengolok-olok Dia, meludahi Dia, menyiksa Dia, dan menyalibkan Dia. Tetapi pada hari ketiga Ia akan bangkit."
\par 35 Lalu Yakobus dan Yohanes, yaitu anak-anak Zebedeus, datang kepada Yesus. "Bapak Guru," kata mereka, "ada suatu hal yang kami ingin Bapak lakukan untuk kami."
\par 36 "Apa yang kalian ingin Aku perbuat bagimu?" tanya Yesus.
\par 37 Mereka menjawab, "Kami ingin duduk di kanan kiri Bapak, apabila Bapak bertakhta dengan mulia."
\par 38 "Kalian tidak tahu apa yang kalian minta," kata Yesus kepada mereka, "Sanggupkah kalian minum dari piala penderitaan yang akan Aku minum dan masuk ke dalam kancah penderitaan yang akan Aku masuki?"
\par 39 "Sanggup," jawab mereka. Maka Yesus berkata lagi kepada mereka, "Memang kalian akan minum dari piala penderitaan yang akan Aku minum, dan masuk ke dalam kancah penderitaan yang akan Aku masuki.
\par 40 Tetapi mengenai siapa yang akan duduk di kanan atau kiri-Ku, itu bukan Aku yang berhak menentukan. Allah yang menentukan siapa-siapa yang akan duduk di tempat-tempat itu."
\par 41 Ketika sepuluh pengikut Yesus yang lainnya itu mendengar hal itu, mereka marah kepada Yakobus dan Yohanes.
\par 42 Jadi Yesus memanggil mereka semuanya, lalu berkata, "Kalian tahu bahwa pemimpin-pemimpin bangsa yang tidak mengenal Allah menindas rakyatnya. Dan pembesar-pembesar mereka menekan mereka.
\par 43 Tetapi kalian tidak boleh begitu! Sebaliknya, orang yang mau menjadi besar di antara kalian, ia harus menjadi pelayanmu.
\par 44 Dan orang yang mau menjadi yang pertama di antara kalian, harus menjadi hamba bagi semua.
\par 45 Sebab Anak Manusia pun tidak datang untuk dilayani. Ia datang untuk melayani dan untuk menyerahkan nyawa-Nya untuk membebaskan banyak orang."
\par 46 Mereka tiba di Yerikho. Dan waktu Yesus dengan pengikut-pengikut-Nya serta orang banyak meninggalkan kota itu, seorang buta sedang duduk minta-minta di pinggir jalan. Namanya Bartimeus, anak dari Timeus.
\par 47 Ketika ia mendengar bahwa yang sedang lewat itu adalah Yesus orang Nazaret, ia berteriak, "Yesus, Anak Daud! Kasihanilah saya!"
\par 48 Ia dimarahi oleh banyak orang dan disuruh diam. Tetapi ia lebih keras lagi berteriak, "Anak Daud, kasihanilah saya!"
\par 49 Maka Yesus berhenti lalu berkata, "Panggillah dia." Jadi mereka memanggil orang buta itu. Mereka berkata kepadanya, "Tenanglah! Kau dipanggil Yesus, bangun!"
\par 50 Orang buta itu pun melemparkan jubahnya, lalu cepat-cepat berdiri dan pergi kepada Yesus.
\par 51 "Apa yang kauingin Aku perbuat untukmu?" tanya Yesus kepadanya. Orang buta itu menjawab, "Pak Guru, saya ingin melihat."
\par 52 "Pergilah," kata Yesus, "karena engkau percaya kepada-Ku, engkau sembuh." Pada saat itu juga orang itu dapat melihat. Lalu ia mengikuti Yesus di jalan.

\chapter{11}

\par 1 Waktu mendekati Yerusalem, mereka sampai ke kota Betfage dan Betania, di lereng Bukit Zaitun. Di situ dua orang pengikut-Nya diutus terlebih dahulu oleh Yesus.
\par 2 "Pergilah ke kampung yang di depan itu," kata Yesus kepada mereka. "Begitu kalian masuk kampung itu, kalian akan melihat seekor anak keledai sedang terikat, yang belum pernah ditunggangi orang. Lepaskanlah keledai itu dan bawa kemari.
\par 3 Dan kalau ada orang bertanya kepadamu apa sebab kalian melepaskan keledai itu, katakanlah, 'Tuhan memerlukannya, dan Ia segera akan mengembalikannya.'"
\par 4 Kedua pengikut Yesus itu pun pergi, dan mendapati seekor anak keledai sedang terikat pada pintu rumah di pinggir jalan. Maka mereka melepaskan keledai itu.
\par 5 Orang-orang yang berdiri di situ bertanya kepada mereka, "Hai, sedang apa kalian? Mengapa melepaskan anak keledai itu?"
\par 6 Mereka menjawab sebagaimana yang sudah dikatakan oleh Yesus kepada mereka. Maka orang-orang itu membiarkan mereka membawa keledai itu.
\par 7 Waktu sampai pada Yesus, punggung keledai itu mereka alasi dengan jubah mereka, lalu Yesus naik ke atasnya.
\par 8 Banyak orang membentangkan jubah mereka di jalan, ada pula yang menyebarkan di tengah jalan ranting-ranting pohon yang mereka ambil dari ladang.
\par 9 Orang-orang yang berjalan di depan dan orang-orang yang mengikuti dari belakang, semuanya berseru-seru, "Pujilah Allah! Diberkatilah Dia yang datang atas nama Tuhan!
\par 10 Hiduplah pemerintahan-Nya yang akan datang--pemerintahan Daud nenek moyang kita! Pujilah Allah Yang Mahatinggi!"
\par 11 Akhirnya Yesus sampai di Yerusalem, lalu masuk ke Rumah Tuhan. Di situ Ia memperhatikan sekeliling-Nya. Tetapi karena sudah hampir gelap, Ia kemudian berangkat ke Betania bersama-sama dengan kedua belas pengikut-Nya.
\par 12 Keesokan harinya, ketika mereka sedang berjalan keluar dari Betania, Yesus lapar.
\par 13 Dari jauh Ia melihat sebatang pohon ara yang daunnya lebat. Jadi Ia pergi ke pohon itu untuk melihat apakah ada buahnya. Tetapi ketika Ia sampai di pohon itu, Ia tidak menemukan apa-apa, kecuali daun-daun saja, sebab pada waktu itu belum musim buah ara.
\par 14 Lalu Yesus berkata kepada pohon ara itu, "Mulai sekarang tidak ada seorang pun yang akan makan buah daripadamu lagi!" Pengikut-pengikut Yesus mendengar ucapan itu.
\par 15 Kemudian mereka sampai di Yerusalem, dan Yesus pergi lagi ke Rumah Tuhan. Di situ Ia mulai mengusir semua orang yang berjual beli di tempat itu. Ia menjungkirbalikkan meja-meja para penukar uang, dan bangku-bangku penjual burung merpati.
\par 16 Dan tidak seorang pun yang diizinkan-Nya membawa apa saja melalui halaman Rumah Tuhan itu.
\par 17 Kemudian Yesus mengajar orang-orang di situ. Ia berkata, "Di dalam Alkitab tertulis begini: Allah berkata, 'Rumah-Ku akan disebut rumah tempat berdoa untuk segala bangsa.' Tetapi kalian menjadikannya sarang penyamun!"
\par 18 Imam-imam kepala dan guru-guru agama mendengar ucapan itu. Maka mereka mulai mencari jalan untuk membunuh Yesus. Mereka takut kepada-Nya, karena semua orang kagum mendengar ajaran-Nya.
\par 19 Menjelang malam, Yesus dan pengikut-pengikut-Nya meninggalkan kota itu.
\par 20 Pagi-pagi keesokan harinya, waktu mereka melewati pohon ara itu, mereka melihat pohon itu sudah mati sampai ke akar-akarnya.
\par 21 Lalu Petrus teringat akan peristiwa sehari sebelumnya. Maka Petrus berkata kepada Yesus, "Bapak Guru, coba lihat! Pohon ara yang Bapak kutuk itu sudah mati!"
\par 22 Yesus menjawab, "Percayalah kepada Allah.
\par 23 Sungguh kalian dapat berkata kepada bukit ini, 'Terangkatlah dan terbuanglah ke dalam laut!' maka hal itu akan dilakukan bagi kalian; asal kalian tidak ragu-ragu, dan kalian percaya bahwa yang kalian katakan itu akan benar-benar terjadi.
\par 24 Sebab itu ingatlah ini: Apabila kalian berdoa dan minta sesuatu, percayalah bahwa Allah sudah memberikan kepadamu apa yang kalian minta, maka kalian akan menerimanya.
\par 25 Dan kalau kalian berdoa, tetapi hatimu tidak senang terhadap seseorang, ampunilah orang itu dahulu, supaya Bapamu di surga juga mengampuni dosa-dosamu.
\par 26 (Kalau kalian tidak mengampuni orang lain, Bapamu yang di surga juga tidak akan mengampuni dosa-dosamu.)"
\par 27 Mereka kembali lagi ke Yerusalem. Dan pada waktu Yesus berjalan berkeliling di dalam Rumah Tuhan, imam-imam kepala, guru-guru agama dan pemimpin-pemimpin Yahudi datang kepada-Nya.
\par 28 Mereka bertanya, "Atas dasar apa Engkau melakukan semuanya ini? Siapa yang memberi hak itu kepada-Mu?"
\par 29 Yesus menjawab, "Aku juga mau bertanya kepada kalian. Jawablah dan Aku akan mengatakan kepadamu dengan hak siapa Aku melakukan hal-hal ini.
\par 30 Yohanes membaptis dengan hak siapa, Allah atau manusia?"
\par 31 Lalu imam-imam kepala dan pemimpin-pemimpin Yahudi mulai berunding di antara mereka. Mereka berkata, "Kalau kita katakan, 'Dengan hak Allah,' Ia akan berkata, 'Mengapa kalian tidak percaya kepada Yohanes?'
\par 32 Tetapi sulit juga untuk berkata, 'Dengan hak manusia.'" Sebab mereka takut akan orang banyak, karena semua orang menganggap Yohanes seorang nabi.
\par 33 Jadi, mereka menjawab, "Kami tidak tahu." Lalu Yesus berkata kepada mereka, "Kalau begitu, Aku pun tidak mau mengatakan kepadamu dengan hak siapa Aku melakukan semuanya ini."

\chapter{12}

\par 1 Kemudian Yesus mulai berbicara dengan perumpamaan kepada imam-imam kepala, guru-guru agama dan pemimpin-pemimpin Yahudi itu. Yesus berkata, "Adalah seorang yang menanami sebidang kebun anggur, lalu memasang pagar di sekelilingnya. Sesudah itu ia menggali lubang untuk alat pemeras anggur, lalu ia mendirikan sebuah menara jaga. Sesudah itu ia menyewakan kebun anggur itu kepada beberapa penggarap lalu berangkat ke negeri lain.
\par 2 Ketika sudah waktunya musim memetik buah anggur, orang itu mengirim seorang pelayannya kepada penggarap-penggarap kebun itu, untuk menerima bagiannya.
\par 3 Tetapi penggarap-penggarap itu menangkap pelayan itu. Kemudian mereka memukulnya, lalu menyuruh dia pulang dengan tangan kosong.
\par 4 Lalu pemilik kebun itu mengirim lagi seorang pelayannya yang lain. Tetapi penggarap-penggarap itu memukul kepala pelayan itu, lalu mengusirnya sambil mencaci maki.
\par 5 Pemilik kebun itu mengirim lagi seorang pelayannya yang lain. Tetapi mereka membunuh pelayan itu. Dan begitulah seterusnya mereka memperlakukan banyak pelayan yang lain pula: ada yang dipukuli dan ada juga yang dibunuh.
\par 6 Siapakah lagi yang dapat dikirim sekarang oleh pemilik kebun itu? Hanya tinggal seorang, yaitu anaknya sendiri yang dikasihinya. Jadi akhirnya ia mengirim anaknya itu kepada penggarap-penggarap itu. 'Pasti anak saya akan dihormati,' pikirnya.
\par 7 Tetapi penggarap-penggarap itu berkata satu sama lain, 'Ini dia ahli warisnya. Mari kita bunuh dia, supaya kita mendapat warisannya!'
\par 8 Maka anak itu ditangkap, lalu dibunuh. Mayatnya mereka buang ke luar kebun itu."
\par 9 Lalu Yesus bertanya, "Apakah yang akan dilakukan oleh pemilik kebun itu? Pasti ia akan datang dan membunuh penggarap-penggarap itu, lalu menyerahkan kebun itu kepada penggarap-penggarap yang lain.
\par 10 Kalian tentunya sudah membaca ayat ini dalam Alkitab, 'Batu yang tidak terpakai oleh tukang-tukang bangunan sudah menjadi batu yang terutama.
\par 11 Inilah perbuatan Tuhan; alangkah indahnya!'"
\par 12 Maka para pemuka bangsa Yahudi yang mendengar perumpamaan itu, berusaha menangkap Yesus, sebab mereka tahu perumpamaan itu ditujukan Yesus kepada mereka. Tetapi mereka takut akan orang banyak. Jadi, mereka pergi meninggalkan Yesus.
\par 13 Beberapa orang Farisi dan beberapa anggota golongan Herodes disuruh menjebak Yesus dengan pertanyaan-pertanyaan.
\par 14 Mereka datang kepada Yesus dan berkata, "Bapak Guru, kami tahu Bapak jujur dan tidak menghiraukan pendapat siapa pun. Bapak mengajar dengan terus terang mengenai kehendak Allah untuk manusia, sebab Bapak tidak pandang orang. Nah, cobalah Bapak katakan kepada kami, 'Menurut peraturan agama kita, bolehkah membayar pajak kepada Kaisar atau tidak? Haruskah kita membayar pajak itu, atau tidak?'"
\par 15 Yesus mengetahui kemunafikan mereka. Ia menjawab, "Apa sebab kalian mau menjebak Aku? Coba perlihatkan kepada-Ku sekeping uang perak."
\par 16 Maka mereka memberikan kepada-Nya sekeping uang perak. Lalu Yesus bertanya, "Gambar dan nama siapakah ini?" "Kaisar," jawab mereka.
\par 17 "Nah, kalau begitu," kata Yesus, "berilah kepada Kaisar apa yang milik Kaisar, dan kepada Allah apa yang milik Allah." Mereka heran mendengar Dia.
\par 18 Beberapa orang dari golongan Saduki datang kepada Yesus. (Mereka adalah golongan yang berpendapat bahwa orang mati tidak akan bangkit kembali.)
\par 19 "Bapak Guru," kata mereka kepada Yesus, "Musa menulis hukum ini untuk kita: 'Kalau seorang laki-laki mati dan ia tidak punya anak, maka saudaranya harus kawin dengan jandanya supaya memberi keturunan kepada orang yang sudah mati itu.'
\par 20 Pernah ada tujuh orang bersaudara. Yang sulung kawin, lalu mati tanpa mempunyai anak.
\par 21 Kemudian yang kedua kawin dengan jandanya, tetapi ia pun mati tanpa mempunyai anak. Hal yang sama terjadi juga dengan saudara yang ketiga,
\par 22 dan seterusnya sampai kepada yang ketujuh. Akhirnya wanita itu sendiri meninggal juga.
\par 23 Pada hari orang mati bangkit kembali, istri siapakah wanita itu? Sebab ketujuh-tujuhnya sudah kawin dengan dia!"
\par 24 Yesus menjawab, "Kalian keliru sekali. Sebab kalian tidak mengerti Alkitab maupun kuasa Allah.
\par 25 Sebab apabila orang-orang mati bangkit kembali, mereka tidak akan kawin lagi, melainkan mereka akan hidup seperti malaikat di surga.
\par 26 Dan tentang orang mati dibangkitkan kembali, belum pernahkah kalian membaca di dalam kitab Musa mengenai belukar yang bernyala itu? Di dalam ayat-ayat itu tertulis bahwa Allah berkata kepada Musa, 'Akulah Allah Abraham, Allah Ishak dan Allah Yakub.'
\par 27 Allah itu bukan Allah orang mati. Ia Allah orang hidup. Kalian keliru sekali!"
\par 28 Lalu datanglah seorang guru agama mendengarkan percakapan antara Yesus dengan orang-orang dari golongan Saduki itu. Guru agama itu melihat bahwa Yesus sudah menjawab orang-orang itu dengan baik. Maka ia bertanya kepada Yesus, "Perintah manakah yang paling penting dari semua perintah?"
\par 29 Yesus menjawab, "Perintah yang pertama, ialah: 'Dengarlah, hai bangsa Israel! Tuhan Allah kita, Tuhan itu esa.
\par 30 Cintailah Tuhan Allahmu dengan sepenuh hatimu, dengan segenap jiwamu, dengan seluruh akalmu dan dengan segala kekuatanmu.'
\par 31 Perintah kedua ialah: 'Cintailah sesamamu, seperti engkau mencintai dirimu sendiri.' Tidak ada lagi perintah lain yang lebih penting dari kedua perintah itu."
\par 32 Lalu guru agama itu berkata kepada Yesus, "Tepat sekali, Bapak Guru! Memang benar apa yang Bapak katakan: Tuhanlah Allah yang esa, dan tidak ada lagi Allah yang lain.
\par 33 Dan manusia harus mencintai Allah dengan sepenuh hatinya, dan dengan seluruh akalnya serta dengan segala kekuatannya. Dan ia juga harus mencintai sesamanya seperti dirinya sendiri. Itu lebih baik daripada mempersembahkan kurban bakaran dan kurban-kurban lainnya kepada Allah."
\par 34 Yesus melihat bahwa guru agama itu sudah menjawab dengan baik sekali. Dan Yesus berkata kepadanya, "Engkau sudah hampir menjadi anggota umat Allah." Sesudah itu tidak seorang pun yang berani lagi mengajukan pertanyaan kepada Yesus.
\par 35 Sementara mengajar di Rumah Tuhan, Yesus bertanya, "Bagaimanakah guru-guru agama dapat mengatakan bahwa Raja Penyelamat itu keturunan Daud?
\par 36 Padahal Daud sendiri--karena diilhami oleh Roh Allah--berkata, 'Tuhan berkata kepada Tuhanku: duduklah di sebelah kanan-Ku sampai Aku membuat musuh-musuh-Mu takluk kepada-Mu.'
\par 37 Jadi kalau Daud menyebut Raja Penyelamat itu 'Tuhan', bagaimana mungkin Dia keturunan Daud?" Orang banyak yang berada di Rumah Tuhan itu senang mendengar Yesus mengajar.
\par 38 Ia berkata kepada mereka, "Hati-hatilah terhadap guru-guru agama. Mereka suka berjalan-jalan dengan jubah yang panjang dan suka dihormati di pasar-pasar.
\par 39 Mereka suka tempat-tempat yang terhormat di dalam rumah ibadat dan di pesta-pesta.
\par 40 Mereka menipu janda-janda dan merampas rumahnya. Dan untuk menutupi kejahatan mereka itu, mereka berdoa panjang-panjang. Hukuman mereka nanti berat!"
\par 41 Waktu duduk bertentangan dengan kotak persembahan di Rumah Tuhan, Yesus memperhatikan orang-orang memasukkan uang mereka ke dalam kotak itu. Banyak orang kaya memasukkan banyak uang;
\par 42 lalu seorang janda yang miskin datang juga. Ia memasukkan dua uang tembaga, yaitu uang receh yang terkecil nilainya.
\par 43 Maka Yesus memanggil pengikut-pengikut-Nya lalu berkata kepada mereka, "Perhatikanlah ini: Janda yang miskin itu memasukkan ke dalam kotak itu lebih banyak daripada yang dimasukkan oleh semua orang-orang lainnya.
\par 44 Sebab mereka semua memberi dari kelebihan hartanya. Tetapi janda itu sekalipun sangat miskin memberikan semua yang ada padanya--justru yang ia perlukan untuk hidup."

\chapter{13}

\par 1 Ketika Yesus meninggalkan Rumah Tuhan, seorang dari pengikut-pengikut-Nya berkata, "Bapak Guru, coba lihat bangunan-bangunan itu. Perhatikan batu-batunya. Bukan main bagusnya!"
\par 2 Yesus menjawab, "Engkau melihat bangunan-bangunan yang besar itu, bukan? Tidak satu batu pun dari bangunan-bangunan itu akan tinggal tersusun pada tempatnya. Semuanya akan dirobohkan."
\par 3 Kemudian Yesus pergi ke Bukit Zaitun, dan duduk di tempat yang berhadapan dengan Rumah Tuhan. Lalu Petrus, Yakobus, Yohanes, dan Andreas datang kepada-Nya untuk berbicara dengan Dia secara pribadi.
\par 4 "Coba Bapak beritahukan kepada kami," kata mereka kepada-Nya, "kapan semuanya itu akan terjadi? Dan tanda-tanda apakah yang menunjukkan bahwa sudah waktunya?"
\par 5 "Waspadalah," jawab Yesus, "jangan sampai kalian tertipu.
\par 6 Banyak orang akan datang dengan memakai nama-Ku dan berkata, 'Akulah Dia!' lalu menipu banyak orang.
\par 7 Kalau kalian mendengar bunyi-bunyi pertempuran dan berita-berita peperangan, jangan takut. Hal-hal itu harus terjadi, tetapi itu tidak berarti bahwa sudah waktunya kiamat.
\par 8 Bangsa yang satu akan berperang melawan bangsa yang lain. Negara yang satu akan menyerang negara yang lain. Di mana-mana akan terjadi gempa bumi dan bahaya kelaparan. Semuanya itu baru permulaan saja, seperti sakit yang dialami seorang wanita yang mau melahirkan.
\par 9 Kalian harus berhati-hati, sebab kalian akan ditangkap dan diseret ke mahkamah-mahkamah. Kalian akan dipukul di rumah-rumah ibadat. Kalian akan dibawa menghadap penguasa-penguasa dan raja-raja karena kalian pengikut-Ku. Dan itulah kesempatan bagimu untuk memberi kesaksian tentang Aku kepada mereka.
\par 10 Kabar Baik dari Allah itu mesti disebarkan dahulu kepada segala bangsa.
\par 11 Dan bila kalian ditangkap dan dibawa ke pengadilan, janganlah khawatir tentang apa yang harus kalian katakan. Kalau sudah sampai waktunya untuk berbicara, katakanlah saja apa yang diberitahukan kepadamu pada waktunya. Karena kata-kata yang kalian ucapkan itu bukan kata-katamu sendiri, melainkan datang dari Roh Allah.
\par 12 Orang akan mengkhianati saudaranya sendiri untuk dibunuh. Itu pun yang akan terjadi antara bapak dengan anaknya. Anak-anak akan melawan ibu-bapaknya, dan menyerahkan mereka untuk dibunuh.
\par 13 Kalian akan dibenci oleh semua orang karena kalian pengikut-Ku. Tetapi orang yang bertahan sampai akhir, akan diselamatkan."
\par 14 "Kalian akan melihat 'Kejahatan yang Menghancurkan' berdiri di tempat yang terlarang baginya. (Catatan kepada pembaca: Perhatikanlah apa artinya!) Pada waktu itu orang yang berada di Yudea harus lari ke pegunungan.
\par 15 Orang yang berada di atas atap rumah jangan turun dan masuk ke dalam rumah untuk mengambil sesuatu.
\par 16 Orang yang berada di ladang jangan kembali untuk mengambil jubahnya.
\par 17 Alangkah ngerinya hari-hari itu bagi wanita yang mengandung dan ibu yang masih menyusui bayi!
\par 18 Berdoalah supaya hal-hal itu jangan terjadi pada musim hujan.
\par 19 Pada hari-hari yang mengerikan itu akan ada suatu kesusahan yang belum pernah terjadi, semenjak Allah menjadikan dunia sampai sekarang, dan tidak pula akan terjadi lagi.
\par 20 Sekiranya Allah tidak memperpendek masa itu, maka tidak ada seorang pun yang selamat. Tetapi karena umat-Nya, Allah memperpendek masa itu.
\par 21 Pada waktu itu kalau seseorang berkata kepada kalian, 'Lihat, Raja Penyelamat itu ada di sini!' atau 'Lihat, Ia ada di situ!' --janganlah percaya kepada orang itu.
\par 22 Sebab penyelamat-penyelamat palsu dan nabi-nabi palsu akan datang. Mereka akan mengerjakan perbuatan-perbuatan luar biasa dan keajaiban-keajaiban untuk menipu kalau mungkin, umat Allah juga.
\par 23 Jadi, waspadalah! Semuanya itu sudah Aku beritahukan kepadamu sebelum hal itu terjadi."
\par 24 "Setelah masa kesusahan itu, matahari akan menjadi gelap, dan bulan tidak lagi bercahaya.
\par 25 Bintang-bintang akan jatuh dari langit, dan para penguasa angkasa raya akan menjadi kacau-balau.
\par 26 Pada waktu itu Anak Manusia akan terlihat datang di dalam awan dengan kuasa besar dan keagungan.
\par 27 Ia akan mengutus malaikat-malaikat untuk mengumpulkan umat pilihan-Nya dari keempat penjuru bumi, dari ujung-ujung bumi ke ujung-ujung langit."
\par 28 "Ambillah pelajaran dari pohon ara. Kalau ranting-rantingnya hijau dan lembut, dan mulai bertunas, kalian tahu bahwa musim panas sudah dekat.
\par 29 Begitu juga kalau kalian melihat hal-hal itu terjadi, kalian tahu bahwa waktunya sudah dekat sekali.
\par 30 Ketahuilah! Semua peristiwa ini akan terjadi sebelum orang-orang yang hidup sekarang ini mati semuanya.
\par 31 Langit dan bumi akan lenyap, tetapi perkataan-Ku tetap selama-lamanya."
\par 32 "Meskipun begitu, tidak seorang pun tahu kapan harinya atau kapan jamnya. Malaikat-malaikat di surga tidak dan Anak pun tidak, hanya Bapa saja yang tahu.
\par 33 Jadi kalian harus berjaga-jaga dan waspada, sebab kalian tidak tahu kapan waktunya.
\par 34 Keadaannya ibarat seorang yang meninggalkan rumahnya lalu pergi ke tempat yang jauh. Ia menyuruh pelayan-pelayannya mengurus rumahnya, dan memberi tugas kepada mereka masing-masing. Kepada penjaga pintu, ia berpesan supaya berjaga baik-baik.
\par 35 Sebab itu kalian harus berjaga-jaga, sebab kalian tidak tahu kapan tuan rumah itu akan kembali--mungkin pada sore hari, mungkin pada tengah malam, mungkin pada waktu subuh, atau mungkin pada waktu matahari terbit.
\par 36 Kalau ia datang tiba-tiba, janganlah sampai ia menemukan kalian sedang tidur.
\par 37 Apa yang Kukatakan ini kepadamu, Kukatakan juga kepada semua orang: berjaga-jagalah!"

\chapter{14}

\par 1 Dua hari lagi Hari Paskah dan Perayaan Roti Tidak Beragi. Imam-imam kepala dan guru-guru agama sedang mencari jalan untuk menangkap Yesus dengan diam-diam, dan untuk membunuh Dia.
\par 2 "Tetapi jangan kita lakukan itu pada waktu perayaan," kata mereka, "sebab nanti timbul huru-hara."
\par 3 Ketika Yesus berada di Betania, di rumah Simon yang dahulu menderita penyakit kulit yang berbahaya, seorang wanita datang kepada-Nya. Ia membawa sebuah botol pualam berisi minyak wangi yang mahal, dibuat dari akar wangi. Waktu Yesus sedang duduk makan, wanita itu memecahkan botol itu dan menuang minyak wangi itu ke atas kepala Yesus.
\par 4 Beberapa orang yang berada di situ menjadi marah dan berkata satu sama lain, "Apa gunanya minyak wangi itu diboroskan?
\par 5 Minyak itu dapat dijual dengan harga lebih dari tiga ratus uang perak, dan uangnya diberikan kepada orang miskin!" Maka mereka marah kepada wanita itu.
\par 6 Tetapi Yesus berkata, "Biarkan dia! Mengapa kalian menyusahkan dia? Ia melakukan sesuatu yang baik dan terpuji terhadap-Ku.
\par 7 Orang miskin selalu ada di antara kalian. Setiap waktu kalau kalian mau, kalian dapat menolong mereka. Tetapi Aku tidak selamanya bersama-sama kalian.
\par 8 Wanita ini telah melakukan apa yang dapat ia lakukan. Ia sudah menyiapkan Aku dengan minyak wangi untuk penguburan-Ku sebelum waktunya.
\par 9 Percayalah! Di seluruh dunia, di mana saja Kabar Baik dari Allah disiarkan, perbuatan wanita ini akan diceritakan juga sebagai kenangan kepadanya."
\par 10 Lalu Yudas Iskariot, seorang dari kedua belas pengikut Yesus, pergi kepada imam-imam kepala dengan maksud mengkhianati Yesus kepada mereka.
\par 11 Mereka senang sekali mendengar tawaran itu, dan berjanji akan memberi uang kepada Yudas. Maka Yudas pun mulai mencari kesempatan untuk mengkhianati Yesus.
\par 12 Pada hari pertama dalam Perayaan Roti Tidak Beragi--pada waktu orang menyembelih domba Paskah--pengikut-pengikut Yesus bertanya kepada-Nya, "Di manakah Bapak ingin kami menyiapkan makanan Paskah untuk Bapak?"
\par 13 Maka Yesus menyuruh dua orang dari mereka, "Pergilah ke kota, di sana seorang laki-laki yang sedang membawa sebuah kendi berisi air, akan bertemu dengan kalian. Ikutilah dia
\par 14 ke rumah yang ia masuki dan katakanlah kepada pemilik rumah itu, 'Bapak Guru bertanya, di mana tempat Dia dan pengikut-pengikut-Nya akan makan makanan Paskah.'
\par 15 Orang itu akan menunjukkan kepadamu sebuah ruangan atas yang besar. Ruangan itu sudah teratur, lengkap dengan perabotannya. Siapkanlah semuanya di sana untuk kita."
\par 16 Lalu kedua orang pengikut Yesus itu pergi ke kota. Di sana mereka mendapati semuanya tepat seperti yang dikatakan oleh Yesus. Lalu mereka pun menyediakan makanan Paskah.
\par 17 Malamnya, Yesus datang dengan kedua belas pengikut-Nya.
\par 18 Dan sementara mereka duduk makan, Yesus berkata, "Dengarkan: seorang dari antara kalian, yang sekarang ini makan bersama-Ku akan mengkhianati Aku."
\par 19 Mendengar itu, pengikut-pengikut Yesus menjadi sangat sedih. Lalu seorang demi seorang mulai bertanya kepada Yesus, "Tentu bukan saya yang Bapak maksudkan?"
\par 20 Yesus menjawab, "Dia salah seorang dari kalian yang dua belas ini, yang makan sepiring dengan Aku.
\par 21 Memang Anak Manusia akan mati seperti yang tertulis dalam Alkitab. Tetapi alangkah celakanya orang yang mengkhianati Anak Manusia itu! Lebih baik untuk orang itu kalau ia tidak pernah dilahirkan sama sekali!"
\par 22 Ketika mereka sedang makan, Yesus mengambil roti dan mengucap doa syukur kepada Allah. Kemudian Ia membelah-belah roti itu dengan tangan-Nya, lalu memberikannya kepada pengikut-pengikut-Nya sambil berkata, "Ambil dan makanlah, ini tubuh-Ku."
\par 23 Sesudah itu Ia mengambil sebuah piala anggur. Ia mengucap doa syukur kepada Allah, lalu memberikan piala itu kepada pengikut-pengikut-Nya Kemudian mereka semua minum anggur itu.
\par 24 Sesudah itu Yesus berkata, "Inilah darah-Ku yang mensahkan perjanjian Allah, darah yang dicurahkan untuk banyak orang.
\par 25 Percayalah: Aku tidak akan minum anggur ini lagi, sampai pada waktu Aku minum anggur yang baru bersama kalian di Dunia Baru Bapa-Ku."
\par 26 Kemudian mereka menyanyikan sebuah nyanyian pujian. Dan sesudah itu mereka pergi ke Bukit Zaitun.
\par 27 Kemudian Yesus berkata kepada mereka, "Kamu semua akan lari meninggalkan Aku. Sebab dalam Alkitab tertulis, 'Allah akan membunuh gembalanya, dan kawanan dombanya akan tercerai-berai.'
\par 28 Tetapi setelah Aku dibangkitkan kembali, Aku akan pergi mendahului kalian ke Galilea."
\par 29 "Tidak," jawab Petrus, "biar mereka semua meninggalkan Bapak, saya sekali-kali tidak."
\par 30 "Ingat," kata Yesus, "malam ini juga, sebelum ayam berkokok dua kali, engkau tiga kali mengingkari Aku."
\par 31 Petrus menjawab dengan tegas, "Biar saya harus mati bersama-sama dengan Bapak, sekali-kali saya tidak akan berkata bahwa saya tidak mengenal Bapak!" Dan pengikut-pengikut yang lain berkata begitu juga.
\par 32 Mereka sampai di suatu tempat yang bernama Getsemani, dan Yesus berkata kepada pengikut-pengikut-Nya, "Duduklah di sini sementara Aku pergi berdoa."
\par 33 Lalu Yesus mengajak Petrus, Yakobus, dan Yohanes pergi bersama-sama dengan Dia. Ia mulai merasa sedih dan gelisah.
\par 34 "Hati-Ku sedih sekali," kata Yesus kepada mereka, "rasanya seperti mau mati saja. Tinggallah di sini dan berjagalah!"
\par 35 Yesus pergi lebih jauh sedikit lalu tersungkur ke tanah dan berdoa. Dalam doa-Nya Ia minta kalau boleh Ia tidak usah mengalami saat penderitaan itu.
\par 36 "Bapa, ya Bapa," kata-Nya, "tidak ada sesuatu pun yang mustahil bagi Bapa. Angkatlah penderitaan ini daripada-Ku. Hanya janganlah mengikuti kemauan-Ku melainkan kemauan Bapa."
\par 37 Sesudah itu Yesus kembali dan mendapati pengikut-pengikut-Nya sedang tidur. Ia berkata kepada Petrus, "Simon, tidurkah engkau? Hanya satu jam saja, engkau tidak dapat berjaga-jaga?"
\par 38 Lalu Yesus berkata kepada mereka, "Berjaga-jagalah, dan berdoalah supaya kalian tidak mengalami cobaan. Memang rohmu mau melakukan yang benar, tetapi kalian tidak sanggup karena tabiat manusia itu lemah."
\par 39 Sekali lagi Yesus pergi berdoa dengan mengucapkan kata-kata yang sama.
\par 40 Sesudah itu Ia kembali lagi kepada pengikut-pengikut-Nya dan mendapati mereka masih juga tidur, karena mereka terlalu mengantuk. Maka mereka tidak tahu apa yang harus mereka katakan kepada Yesus.
\par 41 Ketika Yesus kembali kepada mereka untuk ketiga kalinya, Ia berkata, "Masihkah kalian tidur dan istirahat? Cukuplah! Sudah sampai waktunya Anak Manusia diserahkan ke tangan orang-orang berdosa.
\par 42 Bangunlah, mari kita pergi. Lihat! Orang yang mengkhianati Aku sudah datang!"
\par 43 Sementara Yesus masih berbicara, datanglah Yudas, seorang dari kedua belas pengikut-Nya itu. Bersama-sama dengan dia datang juga banyak orang yang membawa pedang dan pentungan. Mereka diutus oleh imam-imam kepala, guru-guru agama dan pemimpin-pemimpin Yahudi.
\par 44 Si pengkhianat sudah menentukan suatu tanda bagi mereka. "Orang yang saya cium," katanya kepada mereka, "Dialah orangnya. Tangkap Dia dan bawa Dia dengan penjagaan yang ketat."
\par 45 Pada waktu Yudas datang, ia langsung pergi kepada Yesus dan berkata, "Bapak Guru!" kemudian ia mencium Yesus.
\par 46 Lalu orang-orang yang datang bersama-sama dengan Yudas itu menangkap Yesus dan membelenggu Dia.
\par 47 Tetapi salah seorang yang berada di situ mencabut pedangnya dan memarang hamba imam agung sampai putus telinganya.
\par 48 Lalu Yesus berkata kepada mereka, "Apakah Aku ini penjahat, sampai kalian datang dengan pedang dan pentungan untuk menangkap Aku?
\par 49 Setiap hari Aku mengajar di Rumah Tuhan di depan kalian, dan kalian tidak menangkap Aku. Tetapi memang sudah seharusnya begitu, supaya terjadilah apa yang tertulis dalam Alkitab."
\par 50 Semua pengikut-Nya lari meninggalkan Yesus.
\par 51 Seorang muda, yang hanya memakai sehelai kain untuk menutupi badannya, mengikuti Yesus. Orang-orang mau menangkapnya,
\par 52 tetapi ia melepaskan kainnya itu, lalu lari dengan telanjang.
\par 53 Yesus dibawa ke rumah imam agung. Di sana semua imam kepala, pemimpin Yahudi, dan guru agama sedang berkumpul.
\par 54 Petrus mengikuti Yesus dari jauh sampai masuk ke dalam halaman rumah imam agung. Di sana ia duduk menghangatkan badan dekat api bersama-sama dengan pengawal-pengawal.
\par 55 Imam-imam kepala dan segenap Mahkamah Agama berusaha mendapatkan bukti-bukti yang menyalahkan Yesus supaya dapat menjatuhkan hukuman mati ke atas-Nya. Tetapi mereka tidak mendapat satu bukti pun.
\par 56 Banyak saksi yang dipanggil untuk memberi kesaksian palsu terhadap Yesus, tetapi kesaksian mereka bertentangan satu sama lain.
\par 57 Lalu beberapa saksi berdiri dan memberi kesaksian palsu ini tentang Yesus,
\par 58 "Kami mendengar orang ini berkata, 'Aku akan merobohkan Rumah Allah ini yang dibuat oleh manusia, dan setelah tiga hari, Aku akan membangun yang lain yang bukan buatan manusia.'"
\par 59 Tetapi kesaksian orang-orang itu pun bertentangan satu sama lain.
\par 60 Maka imam agung berdiri di hadapan mereka semua dan bertanya kepada Yesus, "Apakah Engkau tidak menjawab tuduhan-tuduhan yang dilontarkan kepada-Mu?"
\par 61 Tetapi Yesus diam saja. Ia tidak menjawab sama sekali. Lalu imam agung itu bertanya sekali lagi kepada-Nya, "Apakah Engkau Raja Penyelamat, Anak Allah Mahakudus?"
\par 62 "Akulah Dia," jawab Yesus, "dan kamu semua akan melihat Anak Manusia duduk di sebelah kanan Allah Yang Mahakuasa dan datang dalam awan dari langit!"
\par 63 Maka imam agung menyobek-nyobek pakaiannya dan berkata, "Tidak perlu lagi saksi!
\par 64 Kalian telah mendengar sendiri kata-kata-Nya yang menghujat Allah. Sekarang apa keputusanmu?" Mereka semuanya memutuskan bahwa Yesus bersalah, dan patut dihukum mati.
\par 65 Lalu beberapa orang mulai meludahi Yesus, dan mereka menutup mata-Nya dan memukul Dia, lalu berkata, "Coba tebak, siapa yang memukul-Mu?" Pengawal-pengawal juga turut menampar Yesus.
\par 66 Sementara Petrus masih berada di halaman, salah seorang pelayan wanita dari imam agung datang ke sana.
\par 67 Ketika melihat Petrus menghangatkan badan di dekat api, ia memperhatikan baik-baik muka Petrus dan berkata, "Bukankah engkau juga bersama-sama Yesus orang Nazaret itu?"
\par 68 Tetapi Petrus menyangkal. "Saya tidak tahu dan tidak mengerti apa maksudmu," katanya kepada pelayan itu. Lalu Petrus pergi ke pintu gerbang rumah imam agung itu. (Pada saat itu, ayam berkokok.)
\par 69 Pelayan wanita itu melihat Petrus lagi, dan berkata pula kepada orang-orang di situ, "Dia memang salah seorang dari mereka!"
\par 70 Tetapi Petrus menyangkal lagi. Tidak lama kemudian, orang-orang di situ berkata lagi kepada Petrus, "Tidak dapat disangkal lagi engkau memang salah seorang dari mereka, sebab engkau dari Galilea!"
\par 71 Lalu Petrus mulai menyumpah-nyumpah dan berkata, "Saya tidak mengenal orang yang kalian maksudkan itu!"
\par 72 Saat itu juga ayam berkokok untuk kedua kalinya. Dan Petrus teringat bahwa Yesus telah berkata kepadanya, "Sebelum ayam berkokok dua kali, engkau tiga kali mengingkari Aku." Maka Petrus pun menangis tersedu-sedu.

\chapter{15}

\par 1 Pagi-pagi sekali, seluruh Mahkamah Agama termasuk imam-imam kepala, pemimpin-pemimpin Yahudi dan guru-guru agama berunding cepat-cepat. Mereka membelenggu Yesus, kemudian membawa Dia dan menyerahkan-Nya kepada Pilatus.
\par 2 Pilatus bertanya kepada Yesus, "Betulkah Engkau raja orang Yahudi?" "Begitulah katamu," jawab Yesus.
\par 3 Imam-imam kepala mengemukakan banyak tuduhan terhadap Yesus.
\par 4 Lalu Pilatus bertanya lagi kepada Yesus, "Tidak maukah Engkau menjawab? Coba lihat berapa banyak tuduhan yang mereka ajukan terhadap-Mu!"
\par 5 Tetapi Yesus sama sekali tidak menjawab lagi, sehingga Pilatus heran.
\par 6 Pada setiap Perayaan Paskah, biasanya Pilatus melepaskan seorang tahanan menurut pilihan orang banyak.
\par 7 Di dalam penjara, di antara pemberontak-pemberontak yang melakukan pembunuhan pada waktu kerusuhan, ada seorang bernama Barabas.
\par 8 Orang banyak berkumpul dan minta kepada Pilatus supaya ia melepaskan seorang tahanan seperti biasa.
\par 9 Pilatus bertanya kepada mereka, "Maukah kalian, saya melepaskan raja orang Yahudi itu untuk kalian?"
\par 10 Sebab Pilatus menyadari bahwa imam-imam kepala menyerahkan Yesus kepadanya karena iri hati.
\par 11 Tetapi imam-imam kepala menghasut orang banyak supaya meminta Pilatus melepaskan Barabas untuk mereka.
\par 12 Maka Pilatus berkata lagi kepada orang banyak itu, "Kalau begitu, saya harus buat apa dengan orang yang kalian sebut raja orang Yahudi itu?"
\par 13 Mereka berteriak, "Salibkan Dia!"
\par 14 "Tetapi apa kejahatan-Nya?" tanya Pilatus. Lalu mereka berteriak lebih kuat lagi, "Salibkan Dia!"
\par 15 Pilatus ingin menyenangkan orang banyak itu, maka ia melepaskan Barabas untuk mereka. Kemudian ia menyuruh orang mencambuk Yesus, lalu menyerahkan-Nya untuk disalibkan.
\par 16 Yesus dibawa oleh prajurit-prajurit ke balai pengadilan di istana gubernur. Kemudian seluruh pasukan dipanggil berkumpul.
\par 17 Lalu mereka mengenakan jubah ungu pada Yesus, dan membuat mahkota dari ranting-ranting berduri, kemudian memasangnya pada kepala Yesus.
\par 18 Setelah itu mereka memberi salam kepada-Nya. "Daulat Raja orang Yahudi!" kata mereka.
\par 19 Mereka memukul kepala Yesus dengan tongkat, lalu mereka meludahi Dia dan bersembah sujud di hadapan-Nya.
\par 20 Sesudah mempermainkan Yesus, mereka membuka jubah ungu itu lalu mengenakan kembali pakaian-Nya sendiri. Kemudian Ia dibawa ke luar untuk disalibkan.
\par 21 Di tengah jalan mereka memaksa seorang memikul salib Yesus. Orang itu kebetulan baru dari desa hendak masuk ke kota. (Namanya Simon, --berasal dari Kirene--ayah dari Aleksander dan Rufus.)
\par 22 Yesus dibawa ke suatu tempat yang bernama Golgota, artinya "Tempat Tengkorak".
\par 23 Di situ mereka mau memberi kepada-Nya anggur yang bercampur mur, tetapi Yesus tidak mau minum anggur itu.
\par 24 Kemudian mereka menyalibkan Dia, dan membagi-bagikan pakaian-Nya dengan undian untuk menentukan bagian masing-masing.
\par 25 Penyaliban-Nya itu terjadi pada pukul sembilan pagi.
\par 26 Di atas salib-Nya dipasang tulisan mengenai tuduhan terhadap-Nya, yaitu: "Raja Orang Yahudi".
\par 27 Bersama-sama dengan Yesus mereka menyalibkan juga dua orang penyamun; seorang di sebelah kanan dan seorang lagi di sebelah kiri-Nya.
\par 28 (Dengan demikian terjadilah yang tertulis dalam Alkitab; yaitu: "Ia dianggap termasuk orang-orang jahat.")
\par 29 Orang-orang yang lewat di situ menggeleng-gelengkan kepala dan menghina Yesus. Mereka berkata, "Hai, Kau yang mau merobohkan Rumah Allah dan membangunnya dalam tiga hari.
\par 30 Coba turun dari salib itu dan selamatkan diri-Mu!"
\par 31 Begitu juga imam-imam kepala dan guru-guru agama mengejek Yesus. Mereka berkata satu sama lain, "Ia menyelamatkan orang lain, tetapi diri-Nya sendiri Ia tidak dapat selamatkan!
\par 32 Kalau Dia raja Israel, Raja Penyelamat, baiklah Ia sekarang turun dari salib itu, supaya kami melihat dan percaya kepada-Nya!" Orang-orang yang disalibkan bersama Yesus itu pun menghina Yesus.
\par 33 Pada tengah hari, selama tiga jam seluruh negeri itu menjadi gelap.
\par 34 Dan pada pukul tiga sore, Yesus berteriak dengan suara yang keras, "Eloi, Eloi, lama sabakhtani?" yang berarti, "Ya Allah-Ku, ya Allah-Ku, mengapakah Engkau meninggalkan Aku?"
\par 35 Beberapa orang di situ mendengar jeritan itu, dan berkata, "Dengarkan, Ia memanggil Elia!"
\par 36 Seorang dari mereka cepat-cepat pergi mengambil bunga karang, lalu mencelupkannya ke dalam anggur asam. Kemudian bunga karang itu dicucukkannya pada ujung sebatang kayu, lalu diulurkannya ke bibir Yesus, sambil berkata, "Tunggu, mari kita lihat apakah Elia datang menurunkan Dia dari salib itu."
\par 37 Lalu Yesus berteriak, dan meninggal.
\par 38 Gorden yang tergantung di dalam Rumah Tuhan sobek menjadi dua dari atas sampai ke bawah.
\par 39 Perwira yang berdiri di depan salib itu, melihat bagaimana Yesus meninggal. Perwira itu berkata, "Memang benar orang ini Anak Allah!"
\par 40 Di situ ada juga beberapa wanita yang sedang melihat semuanya itu dari jauh. Di antaranya ada Salome, Maria Magdalena, dan Maria ibu Yakobus yang muda dan Yoses.
\par 41 Merekalah wanita-wanita yang mengikuti dan menolong Yesus ketika Ia berada di Galilea. Dan ada banyak lagi wanita-wanita lain di situ yang sudah datang ke Yerusalem bersama-sama dengan Yesus.
\par 42 Hari sudah mulai malam ketika Yusuf dari Arimatea datang. Ia anggota Mahkamah Agama yang dihormati. Ia juga sedang menantikan masanya Allah mulai memerintah sebagai Raja. Hari itu hari Persiapan (yaitu hari sebelum hari Sabat). Sebab itu dengan berani Yusuf menghadap Pilatus dan minta jenazah Yesus.
\par 43 [15:42]
\par 44 Pilatus heran mendengar bahwa Yesus sudah meninggal. Jadi ia menyuruh orang memanggil kepala pasukan dan bertanya kepadanya apakah Yesus sudah lama meninggal.
\par 45 Setelah mendengar laporan perwira itu, Pilatus mengizinkan Yusuf mengambil jenazah Yesus.
\par 46 Lalu Yusuf membeli kain kapan dari linen halus dan sesudah menurunkan jenazah Yesus, ia membungkusnya dengan kain itu. Kemudian ia meletakkan jenazah itu di dalam sebuah kuburan yang dibuat di dalam bukit batu. Sesudah itu ia menggulingkan sebuah batu besar menutupi pintu kubur itu.
\par 47 Sementara itu Maria Magdalena dan Maria ibu Yoses memperhatikan di mana Yesus diletakkan.

\chapter{16}

\par 1 Ketika hari Sabat sudah lewat, Maria Magdalena, Maria ibu Yakobus, dan Salome pergi membeli ramuan-ramuan untuk meminyaki jenazah Yesus.
\par 2 Pagi-pagi sekali waktu matahari terbit, pada hari pertama minggu itu, mereka pergi ke kuburan.
\par 3 Di tengah jalan mereka berkata satu sama lain, "Siapakah yang dapat menolong kita menggeserkan batu penutup pada lubang kubur?" Sebab batu itu besar sekali. Tetapi setibanya di situ mereka melihat batu itu sudah terguling.
\par 4 [16:3]
\par 5 Lalu mereka masuk ke dalam kuburan itu. Di dalamnya di sebelah kanan, mereka melihat seorang muda yang memakai jubah putih sedang duduk, dan mereka terkejut.
\par 6 Orang muda itu berkata, "Jangan takut! Saya tahu kalian mencari Yesus orang Nazaret yang sudah disalibkan. Ia tidak ada di sini. Ia sudah bangkit! Lihat saja, ini tempat mereka membaringkan Dia.
\par 7 Sekarang pergilah, sampaikan kabar ini kepada pengikut-pengikut-Nya, termasuk Petrus. Katakan, 'Ia pergi mendahului kalian ke Galilea. Di sana kalian akan melihat Dia, seperti yang sudah dikatakan-Nya kepadamu.'"
\par 8 Maka mereka keluar dari kuburan itu lalu lari karena terkejut dan takut. Mereka tidak mengatakan apa-apa kepada siapa pun, karena takut.
\par 9 Wanita-wanita itu pergi kepada Petrus dan teman-temannya, lalu menceritakan dengan singkat semua yang sudah diberitahukan kepada mereka oleh orang muda itu.
\par 10 Setelah itu, Yesus sendiri melalui pengikut-pengikut-Nya mengabarkan dari timur ke barat berita yang suci dan abadi mengenai keselamatan yang kekal.
\par 11 Tetapi ketika mereka mendengar bahwa Yesus hidup, dan bahwa Maria telah melihat Dia mereka tidak percaya.
\par 12 Setelah itu Yesus memperlihatkan diri kepada dua orang pengikut-Nya dengan cara yang lain. Itu terjadi ketika kedua orang itu sedang berjalan ke sebuah kampung.
\par 13 Jadi mereka kembali dan memberitahukan hal itu kepada yang lain-lainnya. Tetapi mereka tidak percaya.
\par 14 Akhirnya, Yesus memperlihatkan diri kepada kesebelas pengikut-Nya, ketika mereka sedang makan. Ia menegur mereka sebab mereka kurang iman dan terlalu keras kepala sehingga tidak percaya kepada orang-orang yang sudah melihat sendiri bahwa Yesus hidup.
\par 15 Lalu Yesus berkata kepada mereka, "Pergilah ke seluruh dunia dan siarkanlah Kabar Baik dari Allah itu kepada seluruh umat manusia.
\par 16 Orang yang tidak percaya akan dihukum. Tetapi orang yang percaya dan dibaptis, akan selamat.
\par 17 Sebagai bukti bahwa mereka percaya, orang-orang itu akan mengusir roh jahat atas nama-Ku; mereka akan berbicara dalam bahasa-bahasa yang tidak mereka kenal.
\par 18 Kalau mereka memegang ular atau minum racun, mereka tidak akan mendapat celaka. Kalau mereka meletakkan tangan ke atas orang-orang yang sakit, orang-orang itu akan sembuh."
\par 19 Setelah Tuhan Yesus berbicara dengan mereka, Ia diangkat ke surga. Di sana Ia memerintah bersama dengan Allah.
\par 20 Maka pengikut-pengikut-Nya pergi dan menyebarkan berita dari Allah ke mana-mana. Dan Tuhan pun turut bekerja bersama-sama dengan mereka dan membuktikan melalui keajaiban-keajaiban bahwa pemberitaan mereka benar.


\end{document}