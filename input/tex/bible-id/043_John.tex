\begin{document}

\title{John}

Joh 1:1  Pada mulanya, sebelum dunia dijadikan, Sabda sudah ada. Sabda ada bersama Allah dan Sabda sama dengan Allah.
Joh 1:2  Sejak semula Ia bersama Allah.
Joh 1:3  Segalanya dijadikan melalui Dia, dan dari segala yang ada, tak satu pun dijadikan tanpa Dia.
Joh 1:4  Sabda itu sumber hidup, dan hidup memberi terang kepada manusia.
Joh 1:5  Terang itu bercahaya dalam kegelapan, dan kegelapan tak dapat memadamkannya.
Joh 1:6  Datanglah orang yang diutus Allah, Yohanes namanya.
Joh 1:7  Ia datang mewartakan tentang terang itu, supaya semua orang percaya.
Joh 1:8  Ia sendiri bukan terang itu, ia hanya mewartakannya.
Joh 1:9  Terang sejati yang menerangi semua manusia, datang ke dunia.
Joh 1:10  Sabda ada di dunia, dunia dijadikan melalui Dia, tetapi dunia tidak mengenal-Nya.
Joh 1:11  Ia datang ke negeri-Nya sendiri tetapi bangsa-Nya tidak menerima Dia.
Joh 1:12  Namun ada juga orang yang menerima Dia dan percaya kepada-Nya; mereka diberi-Nya hak menjadi anak Allah,
Joh 1:13  yang dilahirkan bukan dari manusia, sebab hidup baru itu dari Allah asalnya.
Joh 1:14  Sabda sudah menjadi manusia, Ia tinggal di antara kita, dan kita sudah melihat keagungan-Nya. Keagungan itu diterima-Nya sebagai Anak tunggal Bapa. Melalui Dia kita melihat Allah dan kasih-Nya kepada kita.
Joh 1:15  Yohanes datang sebagai saksi-Nya, ia mewartakan: "Inilah Dia yang kukatakan: Dia akan datang lebih kemudian dari aku, tetapi lebih besar dari aku, sebab sebelum aku ada, Dia sudah ada."
Joh 1:16  Ia penuh kasih; tiada hentinya, Ia memberkati kita.
Joh 1:17  Hukum Tuhan kita terima melalui Musa. Tetapi kasih dan kesetiaan Allah dinyatakan melalui Yesus Kristus.
Joh 1:18  Tak ada yang pernah melihat Allah, selain anak tunggal Bapa, yang sama dengan Bapa dan erat sekali kepada-Nya. Dialah yang menyatakan Bapa kepada kita.
Joh 1:19  Para penguasa Yahudi di Yerusalem menyuruh imam-imam dan orang-orang Lewi pergi kepada Yohanes dan menanyakan kepadanya, "Engkau ini siapa?"
Joh 1:20  Yohanes mengaku dengan terus terang, "Saya bukan Raja Penyelamat."
Joh 1:21  "Kalau begitu, engkau siapa?" tanya mereka. "Apakah engkau Elia?" "Bukan," jawab Yohanes. "Apakah engkau Sang Nabi?" tanya mereka lagi. "Bukan," jawabnya.
Joh 1:22  "Kalau begitu, katakanlah kepada kami siapa engkau ini," kata mereka, "supaya kami dapat memberi jawaban kepada orang-orang yang menyuruh kami. Apa katamu tentang dirimu sendiri?"
Joh 1:23  Yohanes menjawab, "Sayalah dia yang dikatakan oleh Nabi Yesaya: 'Orang yang berseru di padang pasir: Ratakanlah jalan untuk Tuhan.'"
Joh 1:24  Orang-orang yang diutus oleh orang Farisi
Joh 1:25  bertanya, "Kalau engkau bukan Raja Penyelamat, bukan Elia, bukan juga Sang Nabi, mengapa engkau membaptis?"
Joh 1:26  Yohanes menjawab, "Saya membaptis dengan air. Tetapi di tengah-tengah kalian ada orang yang tidak kalian kenal.
Joh 1:27  Ia datang lebih kemudian dari saya, tetapi untuk membuka tali sepatu-Nya pun saya tidak layak."
Joh 1:28  Semuanya itu terjadi di Betania, sebelah timur Sungai Yordan tempat Yohanes membaptis.
Joh 1:29  Keesokan harinya, Yohanes melihat Yesus datang kepadanya. Lalu Yohanes berkata, "Lihat, itulah Anak Domba Allah yang menghapus dosa dunia.
Joh 1:30  Dialah yang saya katakan akan datang kemudian dari saya, tetapi lebih besar dari saya, sebab sebelum saya lahir, Dia sudah ada.
Joh 1:31  Sebelumnya, saya tidak tahu siapa Dia itu. Padahal saya datang membaptis dengan air supaya bangsa Israel mengenal Dia."
Joh 1:32  Yohanes juga memberi kesaksian ini, "Saya melihat Roh Allah turun seperti merpati dari langit lalu tinggal di atas-Nya.
Joh 1:33  Waktu itu saya belum tahu siapa Dia. Tetapi Allah yang menyuruh saya membaptis dengan air sudah berkata kepada saya, 'Bila engkau melihat Roh Allah turun, lalu tinggal di atas seseorang, Dialah yang akan membaptis dengan Roh Allah.'
Joh 1:34  Saya sudah melihat-Nya sendiri," kata Yohanes, "dan saya memberi kesaksian bahwa Dialah Anak Allah."
Joh 1:35  Keesokan harinya Yohanes ada di tempat itu lagi dengan dua pengikutnya.
Joh 1:36  Ketika ia melihat Yesus lewat, ia berkata, "Lihat! Itulah Anak Domba Allah."
Joh 1:37  Kedua pengikut Yohanes mendengar kata-kata itu, lalu pergi mengikuti Yesus.
Joh 1:38  Yesus menoleh ke belakang, dan melihat mereka sedang mengikuti Dia. Ia bertanya, "Kalian mencari apa?" Jawab mereka, "Rabi, di manakah Rabi tinggal?" (Kata 'Rabi' artinya guru.)
Joh 1:39  "Mari lihat sendiri," kata Yesus. Mereka pergi dengan Dia dan melihat di mana Ia tinggal. Waktu itu pukul empat sore. Hari itu mereka tinggal bersama Dia.
Joh 1:40  Salah satu dari kedua orang yang telah mendengar apa yang dikatakan Yohanes dan kemudian pergi mengikuti Yesus, adalah Andreas, saudara Simon Petrus.
Joh 1:41  Cepat-cepat Andreas mencari Simon, saudaranya, dan berkata kepadanya, "Kami sudah bertemu dengan Mesias!" (Mesias sama dengan Kristus, yaitu: Raja Penyelamat.)
Joh 1:42  Andreas mengantar Simon kepada Yesus. Yesus menatap Simon, lalu berkata, "Engkau Simon, anak Yona. Engkau akan disebut Kefas." (Kefas sama dengan Petrus, artinya: gunung batu.)
Joh 1:43  Keesokan harinya Yesus memutuskan untuk pergi ke Galilea. Ia berjumpa dengan Filipus, dan berkata kepadanya, "Mari ikut Aku!"
Joh 1:44  Filipus berasal dari Betsaida, yaitu tempat tinggal Andreas dan Petrus.
Joh 1:45  Filipus bertemu dengan Natanael dan berkata kepadanya, "Kami sudah menemukan orang yang disebut oleh Musa dalam Buku Hukum Allah, dan yang diwartakan oleh nabi-nabi. Dia itu Yesus dari Nazaret, anak Yusuf."
Joh 1:46  Tetapi Natanael menjawab, "Mungkinkah sesuatu yang baik datang dari Nazaret?" "Mari lihat sendiri," jawab Filipus.
Joh 1:47  Yesus melihat Natanael datang, lalu berkata tentang dia, "Lihat, itu orang Israel sejati. Tak ada kepalsuan padanya."
Joh 1:48  "Bagaimana Bapak mengenal saya?" tanya Natanael kepada Yesus. Yesus menjawab, "Sebelum Filipus memanggilmu, Aku sudah melihat engkau di bawah pohon ara itu."
Joh 1:49  "Bapak Guru," kata Natanael, "Bapak adalah Anak Allah! Bapaklah Raja bangsa Israel!"
Joh 1:50  Yesus berkata, "Engkau percaya, hanya karena Aku mengatakan bahwa Aku sudah melihat engkau di bawah pohon ara itu? Hal-hal yang jauh lebih besar dari itu akan kaulihat!"
Joh 1:51  Kata Yesus pula, "Sungguh, percayalah, engkau akan melihat langit terbuka, dan malaikat-malaikat Allah naik turun pada Anak Manusia."
Joh 2:1  Dua hari kemudian ada pesta kawin di kota Kana di Galilea, dan ibu Yesus ada di sana.
Joh 2:2  Yesus dengan pengikut-pengikut-Nya diundang juga ke pesta itu.
Joh 2:3  Ketika anggur sudah habis, ibu-Nya berkata kepada Yesus, "Mereka kehabisan anggur."
Joh 2:4  Yesus menjawab, "Ibu, jangan menyuruh Aku. Belum sampai waktunya Aku menyatakan diri."
Joh 2:5  Tetapi ibu Yesus berkata kepada pelayan-pelayan, "Lakukan saja apa yang dikatakan-Nya kepadamu."
Joh 2:6  Di situ ada enam tempayan yang disediakan untuk keperluan pembasuhan menurut adat Yahudi. Tempayan itu masing-masing isinya kira-kira seratus liter.
Joh 2:7  Yesus berkata kepada pelayan-pelayan itu, "Isilah tempayan-tempayan itu dengan air." Mereka mengisinya sampai penuh.
Joh 2:8  Lalu Yesus berkata kepada mereka, "Sekarang ambil sedikit air itu dan bawalah kepada pemimpin pesta." Mereka membawa air itu kepada pemimpin pesta,
Joh 2:9  dan ia mencicipi air yang sudah berubah menjadi anggur. (Ia tidak tahu dari mana anggur itu, hanya pelayan-pelayan yang menuang air itu saja yang tahu.) Maka pemimpin pesta itu mendekati pengantin laki-laki,
Joh 2:10  lalu berkata kepadanya, "Biasanya orang menghidangkan anggur yang paling baik lebih dahulu dan kalau para tamu sudah puas minum, barulah anggur yang biasa. Tetapi Saudara menyimpan anggur yang paling baik sampai sekarang!"
Joh 2:11  Itulah keajaiban pertama yang dilakukan Yesus. Ia melakukan itu di Kana di Galilea. Dengan tanda itu Ia menunjukkan keagungan-Nya. Maka pengikut-pengikut-Nya percaya kepada-Nya.
Joh 2:12  Sesudah itu Yesus pergi ke Kapernaum bersama-sama dengan ibu-Nya, saudara-saudara-Nya serta pengikut-pengikut-Nya. Mereka tinggal di sana beberapa hari lamanya.
Joh 2:13  Ketika Hari Raya Paskah Yahudi sudah dekat, Yesus pergi ke Yerusalem.
Joh 2:14  Di Rumah Tuhan di Yerusalem Ia mendapati penjual-penjual sapi, domba, dan burung merpati; juga penukar-penukar uang duduk di situ.
Joh 2:15  Yesus membuat sebuah cambuk dari tali lalu mengusir semua binatang itu, baik domba maupun sapi, dari dalam Rumah Tuhan. Meja-meja para penukar uang dibalikkan-Nya sehingga uang mereka berserakan ke mana-mana.
Joh 2:16  Lalu Ia berkata kepada penjual burung merpati, "Angkat semuanya dari sini. Jangan jadikan Rumah Bapa-Ku tempat berdagang!"
Joh 2:17  Maka pengikut-pengikut-Nya teringat akan ayat Alkitab ini, "Cinta-Ku untuk Rumah-Mu, ya Allah, membakar hati-Ku."
Joh 2:18  Para penguasa Yahudi menantang Yesus, kata mereka, "Coba membuat keajaiban sebagai tanda untuk kami bahwa Engkau berhak bertindak seperti ini."
Joh 2:19  Yesus menjawab, "Runtuhkanlah Rumah ini, dan dalam tiga hari Aku akan membangunnya kembali."
Joh 2:20  Lalu mereka berkata, "Empat puluh enam tahun dibutuhkan untuk membangun Rumah Tuhan ini. Dan Engkau mau membangunnya kembali dalam tiga hari?"
Joh 2:21  Tetapi Rumah Tuhan yang dimaksudkan Yesus adalah tubuh-Nya sendiri.
Joh 2:22  Jadi kemudian, sesudah Yesus dibangkitkan dari mati, teringatlah pengikut-pengikut-Nya bahwa hal itu pernah dikatakan-Nya. Maka percayalah mereka kepada apa yang tertulis dalam Alkitab dan kepada apa yang dikatakan oleh Yesus.
Joh 2:23  Sementara Yesus berada di Yerusalem pada waktu Perayaan Paskah, banyak orang percaya kepada-Nya karena keajaiban-keajaiban yang dibuat-Nya.
Joh 2:24  Tetapi Yesus sendiri tidak percaya mereka, sebab Ia mengenal semua orang.
Joh 2:25  Tidak perlu orang memberi keterangan kepada-Nya tentang siapa pun, sebab Ia sendiri tahu apa yang ada di dalam hati manusia.
Joh 3:1  Ada seorang tokoh agama dari kalangan orang Farisi yang bernama Nikodemus.
Joh 3:2  Pada suatu malam ia datang kepada Yesus dan berkata, "Bapak Guru, kami tahu Bapak diutus Allah. Sebab tak seorang pun dapat membuat keajaiban seperti yang Bapak buat, kalau Allah tidak menyertai dia."
Joh 3:3  Yesus menjawab, "Percayalah, tak seorang pun dapat menjadi anggota umat Allah, kalau ia tidak dilahirkan kembali."
Joh 3:4  "Masakan orang dewasa dapat lahir kembali?" kata Nikodemus kepada Yesus. "Mungkinkah ia masuk kembali ke dalam kandungan ibunya dan dilahirkan lagi?"
Joh 3:5  Yesus menjawab, "Sungguh benar kata-Ku ini: kalau orang tidak dilahirkan dari air dan dari Roh Allah, orang itu tak dapat menjadi anggota umat Allah.
Joh 3:6  Manusia secara jasmani dilahirkan oleh orang tua, tetapi secara rohani dilahirkan oleh Roh Allah.
Joh 3:7  Jangan heran kalau Aku mengatakan: kamu semua harus dilahirkan kembali.
Joh 3:8  Angin bertiup ke mana ia mau; kita mendengar bunyinya, tetapi tidak tahu dari mana datangnya dan ke mana perginya. Begitu juga dengan orang yang dilahirkan oleh Roh Allah."
Joh 3:9  "Bagaimana itu dapat terjadi?" tanya Nikodemus.
Joh 3:10  Yesus menjawab, "Engkau guru di Israel; masakan engkau tidak tahu?
Joh 3:11  Percayalah: kami bicara hanya tentang apa yang kami ketahui, dan kami memberi kesaksian hanya tentang apa yang sudah kami lihat; tetapi kalian tidak mau menerima kesaksian kami.
Joh 3:12  Kalian tidak percaya kalau Aku menceritakan kepadamu mengenai hal-hal dari dunia ini; bagaimana kalian dapat percaya, kalau Aku menceritakan kepadamu hal-hal mengenai surga?
Joh 3:13  Tak seorang pun pernah naik ke surga, selain Dia yang turun ke dunia, yaitu Anak Manusia.
Joh 3:14  Sama seperti Musa menaikkan ular tembaga pada sebatang kayu di padang gurun, begitu juga Anak Manusia harus dinaikkan,
Joh 3:15  supaya semua orang yang percaya kepada-Nya mendapat hidup sejati dan kekal."
Joh 3:16  Karena Allah begitu mengasihi manusia di dunia ini, sehingga Ia memberikan Anak-Nya yang tunggal, supaya setiap orang yang percaya kepada-Nya tidak binasa, melainkan mendapat hidup sejati dan kekal.
Joh 3:17  Sebab Allah mengirim Anak-Nya bukan untuk menghakimi dunia ini, tetapi untuk menyelamatkannya.
Joh 3:18  Orang yang percaya kepada-Nya tidak dihukum. Tetapi orang yang tidak percaya sudah dihukum oleh Allah, karena ia tidak percaya kepada Anak Allah yang tunggal.
Joh 3:19  Ia dituntut berdasarkan hal ini: Terang itu sudah datang ke dunia, tetapi manusia lebih menyukai gelap daripada terang, sebab perbuatan mereka jahat.
Joh 3:20  Setiap orang yang berbuat jahat, benci kepada terang; ia tidak mau datang kepada terang, supaya perbuatannya yang jahat jangan kelihatan.
Joh 3:21  Tetapi orang yang melakukan kehendak Allah, datang kepada terang supaya menjadi nyata bahwa apa yang dilakukannya itu adalah menurut kehendak Allah.
Joh 3:22  Setelah itu Yesus dan pengikut-pengikut-Nya pergi ke Yudea. Ia tinggal di sana beberapa waktu lamanya dengan mereka dan membaptis.
Joh 3:23  Pada waktu itu Yohanes belum masuk penjara. Ia membaptis di Ainon, tak jauh dari Salim, sebab di sana ada banyak air. Orang-orang datang kepadanya, dan ia membaptis mereka.
Joh 3:25  Beberapa pengikut Yohanes mulai bertengkar dengan orang Yahudi tentang peraturan pembersihan.
Joh 3:26  Mereka pergi kepada Yohanes, dan berkata, "Pak Guru, apakah Bapak masih ingat orang yang bersama Bapak di seberang Sungai Yordan itu, yang Bapak tunjukkan kepada kami dahulu? Ia sekarang membaptis juga, dan semua orang pergi kepada-Nya!"
Joh 3:27  Yohanes menjawab, "Manusia tidak dapat mempunyai apa-apa kalau tidak diberikan Allah kepadanya.
Joh 3:28  Kalian sendiri sudah mendengar saya berkata, 'Saya bukan Raja Penyelamat. Saya diutus mendahului Dia.'
Joh 3:29  Pengantin perempuan adalah kepunyaan pengantin laki-laki. Sahabat pengantin laki-laki itu hanya berdiri mendampingi dan mendengarkan, dan ia senang kalau mendengar suara pengantin laki-laki. Begitu juga dengan saya. Sekarang saya senang sekali.
Joh 3:30  Dialah yang harus makin penting, dan saya makin kurang penting."
Joh 3:31  Yang datang dari atas melebihi semuanya. Yang datang dari dunia tergolong orang dunia, dan berbicara tentang hal-hal dunia. Ia yang datang dari atas melebihi semuanya.
Joh 3:32  Ia berbicara mengenai apa yang sudah dilihat dan didengar-Nya, tetapi tidak seorang pun percaya pada kesaksian-Nya.
Joh 3:33  Orang yang percaya pada kesaksian-Nya itu, mengakui bahwa Allah benar.
Joh 3:34  Sebab orang yang diutus Allah menyampaikan kata-kata Allah, karena Roh Allah sudah diberikan kepada-Nya sepenuhnya.
Joh 3:35  Bapa mengasihi Anak-Nya, dan sudah menyerahkan segala kuasa kepada-Nya.
Joh 3:36  Orang yang percaya kepada Anak itu akan mendapat hidup sejati dan kekal. Tetapi orang yang tidak taat kepada Anak itu tidak mendapat hidup. Ia dihukum Allah untuk selama-lamanya.
Joh 4:1  Orang-orang Farisi mendengar bahwa Yesus mendapat dan membaptis lebih banyak pengikut daripada Yohanes.
Joh 4:2  (Sebenarnya Yesus sendiri tidak membaptis melainkan pengikut-pengikut-Nya saja.)
Joh 4:3  Ketika Yesus tahu bahwa orang-orang Farisi sudah mendengar tentang hal itu, Ia pergi dari Yudea kembali ke Galilea.
Joh 4:4  Dalam perjalanannya itu Ia harus lewat Samaria.
Joh 4:5  Maka sampailah Yesus di sebuah kota di Samaria yang bernama Sikhar, tidak jauh dari tanah yang dahulu diberikan Yakub kepada Yusuf anaknya.
Joh 4:6  Di situ terdapat sumur Yakub. Yesus lelah sekali karena perjalanan, sebab itu Ia duduk di pinggir sumur. Waktu itu kira-kira pukul dua belas siang,
Joh 4:7  dan pengikut-pengikut Yesus sudah pergi ke kota untuk membeli makanan. Kemudian seorang wanita Samaria datang menimba air. Yesus berkata kepadanya, "Bu, boleh Aku minta minum?"
Joh 4:9  Wanita Samaria itu menjawab, "Tuan orang Yahudi, saya orang Samaria; mengapa Tuan minta minum dari saya?" (Sebab orang-orang Yahudi tidak ada hubungan dengan orang Samaria.)
Joh 4:10  Yesus menjawab, "Sekiranya engkau tahu pemberian Allah dan siapa yang minta minum kepadamu, pasti engkau sendiri yang minta minum kepada-Nya, dan Ia akan memberi kepadamu air hidup."
Joh 4:11  Kata wanita itu, "Tuan tidak punya timba, dan sumur ini dalam sekali. Dari mana Tuan mendapat air hidup?
Joh 4:12  Yakub, bapak leluhur kami, memberi kami sumur ini. Ia sendiri mengambil air minumnya dari sini; begitu pula anak-anaknya serta binatang ternaknya. Apakah Tuan sangka Tuan lebih besar dari Yakub?"
Joh 4:13  "Orang yang minum air ini akan haus lagi," kata Yesus,
Joh 4:14  "tetapi orang yang minum air yang akan Kuberikan, tidak akan haus lagi selama-lamanya. Sebab air yang akan Kuberikan itu akan menjadi mata air di dalam dirinya yang memancar keluar dan memberikan kepadanya hidup sejati dan kekal."
Joh 4:15  Kata wanita itu, "Tuan, berilah saya air itu, supaya saya tidak haus lagi; dan tidak perlu kembali ke sini untuk mengambil air."
Joh 4:16  "Pergilah, panggil suamimu, lalu kembalilah ke sini," kata Yesus.
Joh 4:17  "Saya tidak punya suami," jawab wanita itu. "Memang benar katamu," kata Yesus.
Joh 4:18  "Sebab engkau sudah kawin lima kali, dan laki-laki yang hidup bersamamu sekarang bukan suamimu."
Joh 4:19  "Sekarang saya tahu Tuan seorang nabi," kata wanita itu.
Joh 4:20  "Nenek moyang kami menyembah Allah di bukit ini, tetapi bangsa Tuan berkata bahwa hanya di Yerusalem saja tempatnya orang menyembah Allah."
Joh 4:21  "Percayalah," kata Yesus kepadanya, "satu waktu orang akan menyembah Bapa, bukan lagi di bukit ini, dan bukan juga di Yerusalem.
Joh 4:22  Kalian orang Samaria menyembah yang tidak kalian kenal, sedangkan kami orang Yahudi menyembah Dia yang kami kenal, sebab keselamatan datang dari orang Yahudi.
Joh 4:23  Tetapi waktunya akan datang, malahan sudah datang, bahwa dengan kuasa Roh Allah orang-orang akan menyembah Bapa sebagai Allah yang benar seperti yang diinginkan Bapa.
Joh 4:24  Sebab Allah itu Roh, dan hanya dengan kuasa Roh Allah orang-orang dapat menyembah Bapa sebagaimana Ia ada."
Joh 4:25  Wanita itu berkata kepada Yesus, "Saya tahu Raja Penyelamat (yang disebut Kristus) akan datang. Kalau Ia datang, Ia akan memberitahukan segala sesuatu kepada kita."
Joh 4:26  "Akulah Dia," kata Yesus, "Aku yang sekarang sedang berbicara dengan engkau."
Joh 4:27  Pada waktu itu pengikut-pengikut Yesus kembali. Mereka heran melihat Yesus berbicara dengan seorang wanita. Tetapi tidak seorang pun dari mereka yang bertanya kepada wanita itu, "Ibu perlu apa?" atau yang bertanya kepada Yesus, "Mengapa Bapak berbicara dengan wanita itu?"
Joh 4:28  Maka wanita itu meninggalkan tempayannya di situ lalu lari ke kota dan berkata kepada orang-orang di sana,
Joh 4:29  "Mari lihat orang yang memberitahukan kepada saya segala sesuatu yang pernah saya lakukan. Mungkinkah Ia itu Raja Penyelamat?"
Joh 4:30  Maka orang-orang itu pun meninggalkan kota lalu pergi kepada Yesus.
Joh 4:31  Sementara itu pengikut-pengikut-Nya mengajak Yesus makan. "Bapak Guru," kata mereka, "silakan makan."
Joh 4:32  Tetapi Yesus menjawab, "Ada makanan pada-Ku, yang tidak kalian tahu."
Joh 4:33  Maka pengikut-pengikut-Nya mulai saling bertanya, "Apakah ada orang membawa makanan untuk Dia?"
Joh 4:34  Lalu Yesus berkata, "Makanan-Ku adalah mengikuti kemauan Dia yang mengutus Aku, dan menyelesaikan pekerjaan yang diserahkan-Nya kepada-Ku.
Joh 4:35  Kalian berkata, 'Empat bulan lagi musim panen.' Tetapi Aku berkata kepadamu: Pandanglah ladang-ladang yang sudah menguning, siap untuk dituai!
Joh 4:36  Orang yang menuai sudah mulai menerima upahnya dan mengumpulkan hasil untuk hidup yang sejati dan kekal. Maka orang yang menabur dan orang yang menuai boleh bersenang bersama-sama.
Joh 4:37  Peribahasa ini benar juga, 'Yang satu menanam, yang lain menuai.'
Joh 4:38  Aku menyuruh kalian pergi menuai di ladang yang tidak kalian usahakan; orang lain sudah bekerja di sana, dan kalian menerima keuntungan dari pekerjaan mereka."
Joh 4:39  Banyak orang Samaria penduduk kota itu percaya kepada Yesus, karena wanita itu berkata, "Ia mengatakan kepada saya segala sesuatu yang pernah saya lakukan."
Joh 4:40  Maka ketika orang-orang Samaria itu bertemu dengan Yesus, mereka minta dengan sangat supaya Ia tinggal dengan mereka. Jadi Yesus tinggal di situ dua hari lamanya.
Joh 4:41  Kemudian lebih banyak lagi orang percaya kepada Yesus karena apa yang diajarkan-Nya sendiri kepada mereka.
Joh 4:42  Mereka berkata kepada wanita itu, "Kami percaya sekarang, bukan lagi karena apa yang engkau katakan kepada kami, tetapi karena kami sendiri sudah mendengar Dia, dan tahu bahwa Ia memang Penyelamat dunia."
Joh 4:43  Sesudah dua hari tinggal di Sikhar, Yesus pergi ke Galilea.
Joh 4:44  Yesus sendiri telah berkata, "Seorang nabi tidak dihormati di negerinya sendiri."
Joh 4:45  Tetapi waktu Ia sampai di Galilea, orang-orang di sana menyambut-Nya dengan senang hati, sebab mereka ada di Yerusalem pada Hari Raya Paskah, dan sudah melihat semua yang dilakukan Yesus.
Joh 4:46  Kemudian Yesus kembali ke Kana di Galilea, di mana Ia pernah mengubah air menjadi anggur. Di kota itu ada seorang pegawai pemerintah, anaknya sedang sakit di Kapernaum.
Joh 4:47  Ketika ia mendengar bahwa Yesus telah datang ke Galilea dari Yudea, ia pergi kepada Yesus dan minta Yesus datang ke Kapernaum untuk menyembuhkan anaknya yang hampir mati.
Joh 4:48  Yesus berkata kepada pegawai pemerintah itu, "Kalau kalian tidak melihat keajaiban-keajaiban, kalian tidak percaya."
Joh 4:49  "Tuan," jawab pegawai pemerintah itu, "cepatlah datang sebelum anak saya mati."
Joh 4:50  Kata Yesus kepadanya, "Pergilah, anakmu sembuh." Orang itu percaya akan perkataan Yesus, lalu ia pergi.
Joh 4:51  Di tengah jalan, pelayan-pelayannya datang kepadanya dan mengabarkan, "Anak Tuan sudah sembuh."
Joh 4:52  Lalu ia bertanya kepada mereka pukul berapa anak itu mulai sembuh. Mereka menjawab, "Kemarin kira-kira pukul satu tengah hari demamnya hilang."
Joh 4:53  Lalu ayah anak itu teringat bahwa pada saat itulah Yesus berkata kepadanya, "Anakmu sembuh." Maka ia dan seluruh keluarganya percaya kepada Yesus.
Joh 4:54  Itulah keajaiban kedua yang dibuat Yesus di Galilea setelah Ia datang dari Yudea.
Joh 5:1  Setelah itu ada perayaan Yahudi, maka Yesus pergi ke Yerusalem.
Joh 5:2  Di Yerusalem dekat "Pintu Domba" ada sebuah kolam, yang dalam bahasa Ibrani dinamakan Betesda. Di situ ada lima serambi.
Joh 5:3  Di serambi-serambi itu banyak orang sakit berbaring; ada yang buta, ada yang timpang, dan ada yang lumpuh. (Mereka semua menunggu air di kolam itu bergoncang.
Joh 5:4  Sebab ada kalanya malaikat Tuhan turun ke dalam kolam itu dan menggoncangkan airnya. Dan orang sakit yang pertama masuk ke dalam kolam itu waktu air bergoncang akan sembuh dari penyakit apa saja yang dideritanya.)
Joh 5:5  Di tempat itu ada seorang laki-laki yang sudah sakit tiga puluh delapan tahun lamanya.
Joh 5:6  Yesus melihat dia berbaring di sana, dan tahu bahwa ia sudah lama sekali sakit; maka Yesus bertanya kepadanya, "Maukah engkau sembuh?"
Joh 5:7  Orang sakit itu menjawab, "Bapak, tidak ada orang di sini untuk memasukkan saya ke dalam kolam waktu airnya bergoncang. Dan sementara saya menuju ke kolam, orang lain sudah masuk lebih dahulu."
Joh 5:8  Maka Yesus berkata kepadanya, "Bangunlah, angkat tikarmu dan berjalanlah."
Joh 5:9  Pada saat itu juga orang itu sembuh. Ia mengangkat tikarnya dan berjalan. Hal itu terjadi pada hari Sabat.
Joh 5:10  Karena itu para penguasa Yahudi berkata kepada orang yang baru sembuh itu, "Hari ini hari Sabat. Engkau tidak boleh mengangkat tikarmu."
Joh 5:11  Tetapi orang itu menjawab, "Orang yang menyembuhkan saya tadi menyuruh saya mengangkat tikar saya dan berjalan."
Joh 5:12  "Siapakah Dia yang menyuruh engkau mengangkat tikarmu dan berjalan?" tanya mereka.
Joh 5:13  Tetapi orang yang sudah sembuh itu tidak tahu siapa orangnya, sebab Yesus telah menghilang di antara orang banyak itu.
Joh 5:14  Kemudian Yesus berjumpa dengan orang itu di dalam Rumah Tuhan, dan berkata kepadanya, "Sekarang engkau sudah sembuh. Janganlah berdosa lagi, supaya tidak mengalami hal yang lebih buruk."
Joh 5:15  Maka pergilah orang itu lalu memberitahukan kepada para penguasa Yahudi bahwa Yesuslah yang menyembuhkannya.
Joh 5:16  Dan karena itu mereka mulai menyusahkan Yesus, sebab Ia menyembuhkan orang pada hari Sabat.
Joh 5:17  Tetapi Yesus berkata kepada mereka, "Bapa-Ku terus bekerja sampai sekarang, dan Aku pun bekerja."
Joh 5:18  Kata-kata itu membuat para penguasa Yahudi semakin berusaha untuk membunuh-Nya. Mereka berbuat itu, bukan hanya karena Ia melanggar peraturan agama mengenai hari Sabat, tetapi juga karena Ia berkata bahwa Allah itu Bapa-Nya; berarti Ia menyamakan diri-Nya dengan Allah.
Joh 5:19  Yesus menjawab orang-orang itu begini, "Percayalah, Anak tidak dapat melakukan apa-apa dengan kuasa sendiri. Ia hanya melakukan apa yang Ia lihat dilakukan oleh Bapa-Nya. Sebab apa yang dilakukan oleh Bapa, itu juga yang dilakukan oleh Anak.
Joh 5:20  Sebab Bapa mengasihi Anak dan menunjukkan kepada-Nya semua yang dilakukan-Nya sendiri. Malah Bapa akan menunjukkan kepada-Nya perbuatan-perbuatan yang lebih besar lagi, sehingga kalian heran.
Joh 5:21  Bapa itu membangkitkan orang mati, dan memberikan mereka hidup sejati dan kekal; begitu juga Anak memberi hidup kekal kepada orang yang mau diberi-Nya hidup.
Joh 5:22  Bapa sendiri tidak menghakimi siapa pun. Semua kekuasaan untuk menghakimi sudah diserahkan kepada Anak-Nya.
Joh 5:23  Bapa melakukan itu supaya semua orang menghormati Anak seperti mereka menghormati Bapa. Orang yang tidak menghormati Anak tidak juga menghormati Bapa yang mengutus Anak.
Joh 5:24  Sungguh benar kata-kata-Ku ini: Orang yang memperhatikan kata-kata-Ku, dan percaya kepada Dia yang mengutus Aku, mempunyai hidup sejati dan kekal. Ia tidak akan dihukum; ia sudah lepas dari kematian dan mendapat kehidupan.
Joh 5:25  Percayalah: Akan datang waktunya--malah sudah sampai waktunya--orang mati akan mendengar suara Anak Allah. Dan orang yang mendengarnya akan hidup.
Joh 5:26  Seperti Bapa sendiri sumber hidup, Ia menjadikan Anak-Nya sumber hidup juga.
Joh 5:27  Ia telah memberikan kepada Anak-Nya hak untuk menghakimi, sebab Ia Anak Manusia.
Joh 5:28  Jangan kalian heran mendengar hal ini, sebab waktunya akan datang bahwa semua orang yang sudah mati mendengar suara-Nya,
Joh 5:29  lalu keluar dari kuburan. Orang yang telah berbuat baik akan bangkit untuk hidup. Tetapi orang yang telah berbuat jahat akan bangkit untuk dihukum."
Joh 5:30  "Aku tak dapat berbuat apa-apa atas kemauan-Ku sendiri. Aku hanya menghakimi sesuai dengan yang diperintahkan Allah. Dan keputusan-Ku adil, sebab Aku tidak mengikuti kemauan sendiri, melainkan kemauan Bapa yang mengutus Aku.
Joh 5:31  Andaikata Aku memberi kesaksian tentang diri sendiri, kesaksian itu tak dapat dipercaya.
Joh 5:32  Tetapi ada orang lain yang memberi kesaksian tentang Aku, dan Aku tahu bahwa kesaksiannya itu benar.
Joh 5:33  Kalian telah mengirim utusan kepada Yohanes, dan ia telah memberi kesaksian yang benar tentang Aku.
Joh 5:34  Hal ini Aku katakan, bukan karena Aku memerlukan kesaksian dari manusia, tetapi supaya kalian diselamatkan.
Joh 5:35  Yohanes itu seperti lampu yang menyala dan memancarkan cahaya. Untuk sementara waktu kalian senang menerima cahayanya itu.
Joh 5:36  Tetapi kesaksian-Ku lebih besar dari kesaksian Yohanes. Apa yang sekarang Kulakukan ini, yaitu pekerjaan yang ditugaskan Bapa kepada-Ku, membuktikan bahwa Bapa telah mengutus Aku.
Joh 5:37  Dan Bapa yang mengutus Aku juga memberi kesaksian tentang Aku. Kalian belum pernah mendengar suara-Nya atau melihat rupa-Nya.
Joh 5:38  Kata-kata-Nya tidak tersimpan di dalam hatimu sebab kalian tidak percaya kepada-Ku yang diutus-Nya.
Joh 5:39  Kalian mempelajari Alkitab sebab menyangka bahwa dengan cara itu kalian mempunyai hidup sejati dan kekal. Dan Alkitab itu sendiri memberi kesaksian tentang Aku.
Joh 5:40  Tetapi kalian tidak mau datang kepada-Ku untuk mendapat hidup kekal.
Joh 5:41  Aku tidak mencari pujian dari manusia.
Joh 5:42  Aku kenal kalian. Aku tahu kalian tidak mengasihi Allah dalam hatimu.
Joh 5:43  Aku datang dengan kuasa Bapa-Ku, namun kalian tidak menerima Aku. Tetapi kalau orang lain datang dengan kuasanya sendiri, kalian mau menerima dia.
Joh 5:44  Bagaimana kalian dapat percaya, kalau kalian mencari pujian dari sesamamu, dan tidak berusaha mencari pujian dari Allah yang Esa?
Joh 5:45  Jangan menyangka Aku akan mempersalahkan kalian di hadapan Bapa. Justru yang akan mempersalahkan kalian adalah Musa, dia yang kalian harapkan.
Joh 5:46  Andaikata kalian percaya kepada Musa, kalian akan percaya kepada-Ku, sebab tentang Akulah dia menulis.
Joh 5:47  Tetapi kalau kalian tidak percaya kepada apa yang ditulis Musa, bagaimana kalian dapat percaya kepada apa yang Kukatakan?"
Joh 6:1  Beberapa waktu kemudian Yesus pergi ke seberang Danau Galilea, yang disebut juga Danau Tiberias.
Joh 6:2  Setibanya di sana, banyak orang mengikuti Dia sebab mereka sudah melihat keajaiban-keajaiban yang dibuat-Nya dengan menyembuhkan orang-orang sakit.
Joh 6:3  Yesus naik ke atas bukit, lalu duduk di situ dengan pengikut-pengikut-Nya.
Joh 6:4  Pada waktu itu sudah dekat Hari Raya Paskah Yahudi.
Joh 6:5  Waktu Yesus melihat ke sekeliling-Nya, Ia melihat orang berduyun-duyun datang kepada-Nya. Maka Ia berkata kepada Filipus, "Di mana kita dapat membeli makanan, supaya semua orang ini bisa makan?"
Joh 6:6  (Yesus sudah tahu apa yang akan dilakukan-Nya, tetapi Ia berkata begitu sebab Ia mau menguji Filipus.)
Joh 6:7  Filipus menjawab, "Roti seharga dua ratus uang perak tidak akan cukup untuk orang-orang ini, sekalipun setiap orang hanya mendapat sedikit saja."
Joh 6:8  Seorang pengikut Yesus yang lain, yaitu Andreas, saudara Simon Petrus, berkata,
Joh 6:9  "Di sini ada anak laki-laki dengan lima roti dan dua ikan. Tetapi apa artinya itu untuk orang sebanyak ini?"
Joh 6:10  "Suruhlah orang-orang itu duduk," kata Yesus. Di tempat itu ada banyak rumput, jadi orang-orang itu duduk di rumput--semuanya ada kira-kira lima ribu orang laki-laki.
Joh 6:11  Kemudian Yesus mengambil roti itu, lalu mengucap syukur kepada Allah. Sesudah itu Ia membagi-bagikan roti itu kepada orang banyak. Kemudian Ia membagi-bagikan ikan itu, dan mereka makan sepuas-puasnya.
Joh 6:12  Setelah semuanya makan sampai kenyang, Yesus berkata kepada pengikut-pengikut-Nya, "Kumpulkanlah kelebihan makanan itu; jangan sampai ada yang terbuang."
Joh 6:13  Lalu mereka mengumpulkan dua belas bakul penuh kelebihan makanan dari lima roti yang dimakan oleh orang banyak itu.
Joh 6:14  Ketika orang banyak melihat keajaiban yang dibuat oleh Yesus, mereka berkata, "Sungguh, inilah Nabi yang diharapkan datang ke dunia!"
Joh 6:15  Yesus tahu mereka mau datang untuk memaksa Dia menjadi raja mereka. Sebab itu pergilah Ia menyingkir ke daerah berbukit.
Joh 6:16  Ketika hari mulai malam, pengikut-pengikut Yesus turun ke danau.
Joh 6:17  Lalu mereka naik perahu menyeberangi danau itu menuju Kapernaum. Hari sudah gelap tetapi Yesus belum juga datang kepada mereka.
Joh 6:18  Sementara itu danau mulai bergelora karena angin keras.
Joh 6:19  Sesudah berlayar kira-kira lima atau enam kilometer, mereka melihat Yesus datang ke perahu dengan berjalan di atas air. Mereka takut sekali.
Joh 6:20  Tetapi Yesus berkata kepada mereka, "Jangan takut, ini Aku!"
Joh 6:21  Lalu dengan senang hati mereka menerima Dia ke dalam perahu, dan saat itu juga perahu mereka itu sampai di tempat tujuan.
Joh 6:22  Keesokan harinya orang banyak yang masih tinggal di seberang danau, menyadari bahwa tadinya hanya ada satu perahu di sana. Mereka tahu bahwa pengikut-pengikut Yesus sudah berangkat dengan perahu itu, sedangkan Yesus tidak ikut.
Joh 6:23  Kemudian beberapa perahu dari kota Tiberias datang dan berlabuh di dekat tempat orang banyak itu makan roti sesudah Tuhan mengucap syukur.
Joh 6:24  Ketika orang banyak itu melihat bahwa baik Yesus maupun pengikut-pengikut-tak ada di situ, mereka juga naik perahu-perahu itu dan pergi ke Kapernaum mencari Yesus.
Joh 6:25  Ketika orang-orang itu bertemu dengan Yesus di seberang danau, mereka bertanya kepada-Nya, "Bapak Guru, kapan Bapak sampai di sini?"
Joh 6:26  Yesus menjawab, "Sungguh, kalian mencari Aku bukan karena kalian sudah mengerti maksud keajaiban-keajaiban yang Kubuat, tetapi karena kalian sudah makan sampai kenyang.
Joh 6:27  Janganlah bekerja untuk mendapat makanan yang bisa habis dan busuk. Bekerjalah untuk mendapat makanan yang tidak bisa busuk dan yang memberi hidup sejati dan kekal. Makanan itu akan diberikan oleh Anak Manusia kepadamu, sebab Ia sudah dilantik oleh Allah Bapa."
Joh 6:28  Lalu mereka bertanya kepada-Nya, "Kami harus berbuat apa untuk melakukan kehendak Allah?"
Joh 6:29  Yesus menjawab, "Inilah yang diinginkan Allah dari kalian: percayalah kepada Dia yang diutus Allah."
Joh 6:30  "Kalau begitu," kata mereka, "bukti apa yang dapat Bapak berikan supaya kami melihat dan percaya kepada Bapak? Apa yang akan Bapak lakukan?
Joh 6:31  Nenek moyang kami makan manna di padang gurun, seperti tertulis di dalam Alkitab, 'Ia memberi mereka makan roti dari surga.'"
Joh 6:32  Lalu Yesus berkata kepada mereka, "Percayalah: Bukan Musa, melainkan Bapa-Kulah yang memberi kepadamu roti yang sesungguhnya dari surga.
Joh 6:33  Sebab roti yang diberi Allah adalah Dia yang turun dari surga dan memberi hidup kepada manusia di dunia."
Joh 6:34  "Bapak," kata mereka, "berilah kepada kami roti itu selalu."
Joh 6:35  "Akulah roti yang memberi hidup," kata Yesus kepada mereka. "Orang yang datang kepada-Ku tidak akan lapar lagi untuk selamanya. Dan orang yang percaya kepada-Ku tidak akan haus lagi untuk selamanya.
Joh 6:36  Tetapi seperti sudah Kukatakan kepadamu, walaupun kalian sudah melihat Aku, kalian tidak percaya.
Joh 6:37  Semua orang yang diberikan Bapa kepada-Ku akan datang kepada-Ku. Aku tidak akan menolak siapa pun yang datang kepada-Ku.
Joh 6:38  Sebab Aku turun dari surga, bukan untuk melakukan kemauan sendiri, melainkan kemauan Dia yang mengutus Aku.
Joh 6:39  Dan inilah kemauan Dia yang mengutus Aku: supaya dari semua orang yang diberikan kepada-Ku, tidak satu pun hilang; tetapi supaya Aku membangkitkan mereka pada Hari Kiamat.
Joh 6:40  Memang inilah keinginan Bapa-Ku: Supaya semua yang melihat Anak dan percaya kepada-Nya mempunyai hidup sejati dan kekal, dan Aku hidupkan kembali pada Hari Kiamat."
Joh 6:41  Orang-orang Yahudi mulai mengomel terhadap Yesus, sebab Ia berkata: "Aku roti yang turun dari surga."
Joh 6:42  Mereka berkata, "Bukankah ini Yesus, anak Yusuf? Kami kenal ibu bapak-Nya! Bagaimana Ia dapat berkata bahwa Ia turun dari surga?"
Joh 6:43  Lalu Yesus berkata kepada mereka, "Jangan mengomel.
Joh 6:44  Tak seorang pun dapat datang kepada-Ku, kalau Bapa yang mengutus Aku, tidak membawa dia kepada-Ku; dan siapa yang datang, akan Kubangkitkan pada Hari Kiamat.
Joh 6:45  Di dalam Buku Nabi-nabi tertulis begini: 'Semua orang akan diajar oleh Allah.' Jadi semua orang yang mendengar Bapa dan belajar dari Dia, datang kepada-Ku.
Joh 6:46  Itu tidak berarti bahwa ada orang yang sudah melihat Bapa. Hanya Dia yang datang dari Allah, sudah melihat Bapa.
Joh 6:47  Ketahuilah: Orang yang percaya, mempunyai hidup sejati dan kekal.
Joh 6:48  Akulah roti yang memberi hidup.
Joh 6:49  Nenek moyangmu makan manna di padang gurun dan mereka mati juga.
Joh 6:50  Tetapi tidak demikian dengan roti yang turun dari surga; orang yang makan roti itu tak akan mati.
Joh 6:51  Akulah roti yang turun dari surga--roti yang memberi hidup. Orang yang makan roti ini akan hidup selamanya. Roti yang akan Kuberikan untuk kehidupan manusia di dunia adalah daging-Ku."
Joh 6:52  Mendengar itu, orang-orang Yahudi bertengkar satu sama lain. "Bagaimana orang ini dapat memberikan daging-Nya kepada kita untuk dimakan?" kata mereka.
Joh 6:53  Lalu Yesus berkata kepada mereka, "Percayalah: Kalau kalian tidak makan daging Anak Manusia dan minum darah-Nya, kalian tidak akan benar-benar hidup.
Joh 6:54  Orang yang makan daging-Ku dan minum darah-Ku mempunyai hidup sejati dan kekal dan Aku akan membangkitkannya pada Hari Kiamat.
Joh 6:55  Sebab daging-Ku sungguh makanan, dan darah-Ku sungguh minuman.
Joh 6:56  Orang yang makan daging-Ku dan minum darah-Ku, tetap bersatu dengan Aku, dan Aku dengan dia.
Joh 6:57  Bapa yang hidup itu, mengutus Aku dan Aku pun hidup dari Bapa. Begitu juga orang yang makan daging-Ku akan hidup dari Aku.
Joh 6:58  Inilah roti yang turun dari surga: bukan roti seperti yang dimakan nenek moyangmu. Karena setelah makan roti itu, mereka mati juga. Tetapi orang yang makan roti ini akan hidup selama-lamanya."
Joh 6:59  Semuanya itu dikatakan oleh Yesus ketika Ia mengajar di rumah ibadat di Kapernaum.
Joh 6:60  Sesudah mendengar kata-kata Yesus itu, banyak di antara pengikut-Nya berkata, "Pengajaran ini terlalu berat. Siapa yang dapat menerimanya!"
Joh 6:61  Yesus sendiri tahu bahwa pengikut-pengikut-Nya mengomel tentang hal itu. Maka Ia berkata, "Apakah kalian tersinggung karena kata-kata itu?
Joh 6:62  Bagaimana jadinya nanti kalau kalian melihat Anak Manusia naik kembali ke tempat-Nya yang semula?
Joh 6:63  Yang membuat manusia hidup ialah Roh Allah. Kekuatan manusia tidak ada gunanya. Kata-kata yang Kukatakan kepadamu ini adalah kata-kata Roh Allah dan kata-kata yang memberi hidup.
Joh 6:64  Namun ada juga di antara kalian yang tidak percaya." (Yesus sudah tahu dari mulanya siapa-siapa yang tidak mau percaya, dan siapa yang akan mengkhianati-Nya.)
Joh 6:65  Lalu Yesus berkata lagi, "Itulah sebabnya Aku memberitahukan kepadamu bahwa tidak seorang pun dapat datang kepada-Ku, kalau Bapa tidak memungkinkannya."
Joh 6:66  Mulai saat itu banyak pengikut-Nya meninggalkan Dia, dan tidak mau mengikuti-Nya lagi.
Joh 6:67  Lalu Yesus bertanya kepada kedua belas pengikut-Nya, "Apakah kalian juga mau meninggalkan Aku?"
Joh 6:68  "Tuhan," kata Simon Petrus kepada-Nya, "kepada siapa kami akan pergi? Perkataan Tuhan memberi hidup sejati dan kekal.
Joh 6:69  Kami sudah percaya dan yakin bahwa Tuhanlah utusan suci dari Allah."
Joh 6:70  Yesus menjawab, "Bukankah Aku yang memilih kalian dua belas orang ini? Namun satu di antara kalian adalah setan!"
Joh 6:71  Yang Yesus maksudkan ialah Yudas anak Simon Iskariot. Sebab meskipun Yudas seorang dari kedua belas pengikut Yesus, ia akan mengkhianati Yesus.
Joh 7:1  Setelah itu Yesus pergi ke mana-mana di Galilea. Ia tidak mau ke daerah Yudea sebab para penguasa Yahudi di sana bermaksud membunuh Dia.
Joh 7:2  Pada waktu itu sudah dekat Hari Raya Pondok Daun.
Joh 7:3  Maka saudara-saudara Yesus berkata kepada-Nya, "Tinggalkanlah tempat ini dan pergilah ke Yudea, supaya pengikut-pengikut-Mu dapat melihat juga pekerjaan-Mu.
Joh 7:4  Tak ada orang yang akan menyembunyikan apa yang ia lakukan, kalau ia ingin menjadi terkenal. Kalau Engkau melakukan hal-hal seperti itu, seluruh dunia harus tahu!"
Joh 7:5  (Sebab saudara-saudara-Nya sendiri juga tidak percaya kepada-Nya.)
Joh 7:6  "Belum waktunya buat Aku," kata Yesus kepada mereka, "tetapi untuk kalian, setiap waktu bisa.
Joh 7:7  Dunia ini tidak mungkin membenci kalian. Tetapi Aku memang dibenci oleh dunia, sebab Aku selalu mengatakan kepada dunia bahwa perbuatannya jahat.
Joh 7:8  Pergilah kalian sendiri ke perayaan itu. Aku tidak pergi sebab belum waktunya buat Aku."
Joh 7:9  Begitulah kata Yesus kepada saudara-saudara-Nya, dan Ia pun tinggal di Galilea.
Joh 7:10  Setelah saudara-saudara-Nya pergi ke perayaan itu, diam-diam Yesus pergi sendirian tanpa diketahui orang.
Joh 7:11  Selama perayaan itu, para penguasa Yahudi mencari Dia dan bertanya-tanya, "Di mana Dia?"
Joh 7:12  Banyak orang mulai berbisik-bisik mengenai Dia. Ada yang berkata: "Ia orang baik." Ada pula yang berkata: "Tidak! Dia menyesatkan orang banyak."
Joh 7:13  Tetapi tidak seorang pun berani berbicara terang-terangan tentang Dia sebab mereka takut kepada para penguasa Yahudi.
Joh 7:14  Tengah-tengah perayaan, Yesus masuk ke dalam Rumah Tuhan, lalu mulai mengajar.
Joh 7:15  Para penguasa Yahudi heran sekali dan berkata, "Bagaimana orang ini bisa tahu begitu banyak, padahal Ia tidak pernah sekolah?"
Joh 7:16  Yesus menjawab, "Yang Aku ajarkan ini bukan ajaran-Ku, tetapi ajaran Dia yang mengutus Aku.
Joh 7:17  Orang yang mau menuruti kemauan Allah, akan tahu apakah ajaran-Ku datangnya dari Allah atau dari Aku sendiri.
Joh 7:18  Orang yang memberi ajarannya sendiri mencari kehormatan untuk dirinya sendiri. Tetapi orang yang mencari kehormatan bagi Dia yang mengutusnya, orang itu jujur, tidak ada penipuan padanya.
Joh 7:19  Bukankah Musa sudah memberikan perintah-perintah Allah kepadamu? Tetapi di antara kalian tak ada yang menuruti perintah-perintah itu. Mengapa kalian mau membunuh Aku?"
Joh 7:20  Orang banyak itu menjawab, "Engkau gila! Siapa mau membunuh Engkau?"
Joh 7:21  Yesus menjawab, "Hanya satu pekerjaan Kulakukan pada hari Sabat, dan kalian heran.
Joh 7:22  Musa memberi kalian peraturan untuk bersunat--walaupun sunat itu sebenarnya tidak berasal dari Musa, tetapi dari bapak-bapak leluhur sebelum Musa. Karena itu, pada hari Sabat pun kalian mau menyunat orang.
Joh 7:23  Kalau kalian melakukan itu supaya jangan melanggar peraturan Musa tentang sunat, mengapa kalian marah kepada-Ku karena menyembuhkan diri orang seluruhnya pada hari Sabat?
Joh 7:24  Jangan menghakimi orang berdasarkan yang kelihatan, tetapi berdasarkan keadilan."
Joh 7:25  Kemudian ada beberapa orang Yerusalem berkata, "Bukankah ini orangnya yang sedang dicari-cari untuk dibunuh?
Joh 7:26  Lihatlah Ia berbicara dengan leluasa di depan umum, dan tidak ada yang berkata apa-apa kepada-Nya! Apakah penguasa-penguasa kita mungkin sudah menyadari bahwa Ia ini Raja Penyelamat?
Joh 7:27  Tetapi kalau Raja Penyelamat itu datang, tidak seorang pun tahu dari mana asal-Nya! Padahal kita tahu dari mana asalnya orang ini."
Joh 7:28  Kemudian, sedang Yesus mengajar di dalam Rumah Tuhan, Ia berseru dengan suara yang keras, "Jadi kalian tahu siapa Aku ini, dan dari mana asal-Ku? Aku tidak datang atas kemauan-Ku sendiri. Aku diutus oleh Dia yang berhak mengutus Aku, dan Ia dapat dipercaya. Tetapi kalian tidak mengenal Dia.
Joh 7:29  Aku mengenal Dia, karena Aku berasal dari Dia, dan Dialah yang mengutus Aku."
Joh 7:30  Pada saat itu mereka ingin menangkap Yesus, tetapi tak ada yang berani memegang Dia, karena belum sampai waktunya.
Joh 7:31  Banyak di antara orang-orang itu mulai percaya kepada-Nya, dan berkata, "Kalau Raja Penyelamat itu datang, apakah Ia dapat melakukan lebih banyak keajaiban daripada orang ini?"
Joh 7:32  Orang-orang Farisi mendengar bagaimana orang banyak itu berbisik-bisik tentang Yesus. Karena itu, bersama-sama dengan imam-imam kepala, mereka menyuruh beberapa pengawal Rumah Tuhan pergi menangkap Yesus.
Joh 7:33  Yesus berkata kepada orang banyak di dalam Rumah Tuhan, "Hanya sebentar saja Aku masih bersama kalian. Setelah itu Aku akan kembali kepada yang mengutus Aku.
Joh 7:34  Kalian akan mencari Aku, tetapi tidak dapat menemukan Aku; sebab kalian tidak dapat datang ke tempat di mana Aku berada."
Joh 7:35  Lalu para penguasa Yahudi berkata satu sama lain, "Orang ini mau pergi ke mana sehingga kita tidak dapat menemukan Dia? Apakah Ia mau pergi kepada orang-orang Yahudi yang tinggal di luar negeri di antara orang Yunani, dan mengajar orang Yunani?
Joh 7:36  Apa maksud-Nya dengan berkata bahwa kita akan mencari Dia tetapi tidak bisa menemukan-Nya dan bahwa kita tidak dapat datang ke tempat di mana Dia berada?"
Joh 7:37  Pada hari terakhir dari perayaan itu, yaitu hari yang paling penting, Yesus berdiri di dalam Rumah Tuhan lalu berseru, "Orang yang haus hendaklah datang kepada-Ku untuk minum.
Joh 7:38  Mengenai orang yang percaya kepada-Ku, tertulis dalam Alkitab: 'Dari dalam hatinya mengalirlah aliran-aliran air yang memberi hidup.'"
Joh 7:39  (Yesus berbicara tentang Roh Allah, yang akan diterima oleh orang-orang yang percaya kepada-Nya. Sebab pada waktu itu Roh Allah belum diberikan; karena Yesus belum dimuliakan dengan kematian-Nya.)
Joh 7:40  Banyak orang mendengar apa yang dikatakan oleh Yesus, dan di antara mereka ada yang berkata, "Orang ini pasti Nabi itu!"
Joh 7:41  Yang lain berkata, "Inilah Raja Penyelamat!" Tetapi ada juga yang berkata, "Ah, masakan Raja Penyelamat datang dari Galilea?
Joh 7:42  Dalam Alkitab tertulis bahwa Raja Penyelamat adalah keturunan Daud dan akan datang dari Betlehem, yaitu kampung halaman Daud."
Joh 7:43  Akhirnya orang-orang mulai bertengkar mengenai Yesus.
Joh 7:44  Ada yang mau menangkap Dia, tetapi tidak seorang pun memegang-Nya.
Joh 7:45  Ketika pengawal-pengawal Rumah Tuhan yang disuruh pergi menangkap Yesus datang kembali, imam-imam kepala dan orang-orang Farisi itu bertanya kepada mereka, "Mengapa kalian tidak membawa Dia kemari?"
Joh 7:46  Pengawal-pengawal itu menjawab, "Wah, belum pernah ada orang berbicara seperti Dia!"
Joh 7:47  "Apakah kalian juga sudah disesatkan oleh Dia?" kata orang-orang Farisi itu.
Joh 7:48  "Adakah dari penguasa-penguasa kita atau orang Farisi yang percaya kepada-Nya?
Joh 7:49  Tetapi orang banyak ini, mereka tidak mengenal hukum Musa, dan bagaimanapun juga mereka sudah terkutuk."
Joh 7:50  Salah seorang di antara orang-orang Farisi itu adalah Nikodemus yang pernah datang kepada Yesus. Nikodemus berkata kepada orang-orang Farisi yang lain,
Joh 7:51  "Menurut Hukum, seseorang tak boleh dihukum sebelum perkaranya didengar dan perbuatannya diperiksa."
Joh 7:52  "Apakah engkau juga dari Galilea?" jawab mereka. "Periksa saja Alkitab! Engkau akan melihat bahwa tak ada nabi yang berasal dari Galilea!"
Joh 7:53  Setelah itu, semua orang pulang ke rumah.
Joh 8:1  Tetapi Yesus pergi ke Bukit Zaitun.
Joh 8:2  Keesokan harinya pagi-pagi Ia pergi lagi ke Rumah Tuhan, dan banyak orang datang kepada-Nya. Yesus duduk, lalu mulai mengajar mereka.
Joh 8:3  Sementara itu, guru-guru agama dan orang-orang Farisi membawa kepada-Nya seorang wanita yang kedapatan berzinah. Mereka menyuruh wanita itu berdiri di tengah-tengah,
Joh 8:4  lalu berkata kepada Yesus, "Bapak Guru, wanita ini kedapatan sedang berbuat zinah.
Joh 8:5  Di dalam Hukum Musa ada peraturan bahwa wanita semacam ini harus dilempari dengan batu sampai mati. Sekarang bagaimana pendapat Bapak?"
Joh 8:6  Mereka bertanya begitu untuk menjebak Dia, supaya mereka dapat menyalahkan-Nya. Tetapi Yesus tunduk saja, dan menulis dengan jari-Nya di tanah.
Joh 8:7  Ketika mereka terus mendesak, Ia mengangkat kepala-Nya dan berkata kepada mereka, "Orang yang tidak punya dosa di antara kalian, biarlah dia yang pertama melemparkan batu kepada wanita itu."
Joh 8:8  Sesudah itu Yesus tunduk kembali dan menulis lagi di tanah.
Joh 8:9  Setelah mendengar Yesus berkata begitu, pergilah mereka meninggalkan tempat itu, satu demi satu mulai dari yang tertua. Akhirnya Yesus tinggal sendirian di situ dengan wanita yang masih berdiri di tempatnya.
Joh 8:10  Lalu Yesus mengangkat kepala-Nya dan berkata kepada wanita itu, "Di mana mereka semuanya? Tidak adakah yang menghukum engkau?"
Joh 8:11  "Tidak, Pak," jawabnya. "Baiklah," kata Yesus, "Aku juga tidak menghukum engkau. Sekarang pergilah, jangan berdosa lagi."
Joh 8:12  Yesus berbicara lagi kepada orang banyak, kata-Nya, "Akulah terang dunia. Orang yang mengikuti Aku tak akan berjalan dalam kegelapan, tetapi mempunyai terang kehidupan."
Joh 8:13  "Sekarang Engkau memberi kesaksian tentang diri sendiri," kata orang-orang Farisi itu kepada-Nya, "kesaksian-Mu tidak benar."
Joh 8:14  Yesus menjawab, "Meskipun Aku memberi kesaksian tentang diri-Ku sendiri, kesaksian-Ku itu benar; sebab Aku tahu dari mana Aku datang dan ke mana Aku pergi. Kalian tidak tahu dari mana Aku datang dan ke mana Aku pergi.
Joh 8:15  Kalian menghakimi orang dengan cara manusia; Aku tidak menghakimi seorang pun.
Joh 8:16  Tetapi sekiranya Aku menghakimi orang, keputusan-Ku itu adil, sebab Aku tidak sendirian; Bapa yang mengutus-Ku ada bersama Aku.
Joh 8:17  Di dalam Hukum Musa tertulis begini: Kesaksian yang benar adalah kesaksian dari dua orang.
Joh 8:18  Yang memberi kesaksian tentang diri-Ku ada dua--Aku dan Bapa yang mengutus Aku."
Joh 8:19  "Bapak-Mu itu di mana?" kata mereka. Yesus menjawab, "Kalian tidak mengenal Aku maupun Bapa-Ku. Andaikata kalian mengenal Aku, pasti kalian mengenal Bapa-Ku juga."
Joh 8:20  Semuanya itu dikatakan Yesus pada waktu Ia sedang mengajar di Rumah Tuhan dekat kotak-kotak uang persembahan. Tetapi tidak seorang pun yang menangkap Dia, sebab belum sampai waktunya.
Joh 8:21  Yesus berkata lagi kepada mereka, "Aku akan pergi, dan kalian akan mencari Aku, tetapi kalian akan mati di dalam dosamu. Ke tempat Aku pergi, kalian tak dapat datang."
Joh 8:22  Maka para penguasa Yahudi berkata, "Barangkali Ia mau bunuh diri, sebab Ia berkata, 'Ke tempat Aku pergi, kalian tidak dapat datang.'"
Joh 8:23  Lalu Yesus berkata kepada mereka, "Kalian datang dari bawah; tetapi Aku datang dari atas. Kalian dari dunia; Aku bukan dari dunia.
Joh 8:24  Itu sebabnya Aku berkata kepadamu, bahwa kalian akan mati di dalam dosa-dosamu. Dan memang kalian akan mati dalam dosa-dosamu, kalau tidak percaya bahwa 'Akulah Dia yang disebut AKU ADA'."
Joh 8:25  "Engkau siapa sebenarnya?" tanya mereka. Yesus menjawab, "Untuk apa bicara lagi dengan kalian!
Joh 8:26  Masih banyak hal tentang kalian yang mau Kukatakan dan hakimi. Tetapi Dia yang mengutus Aku dapat dipercaya. Dan Aku memberitahukan kepada dunia apa yang Aku dengar dari Dia."
Joh 8:27  Mereka tidak mengerti bahwa Ia sedang berbicara kepada mereka tentang Bapa.
Joh 8:28  Karena itu Yesus berkata kepada mereka, "Nanti kalau kalian sudah meninggikan Anak Manusia, kalian akan tahu bahwa 'Akulah Dia yang disebut AKU ADA', dan kalian akan tahu bahwa tidak ada satu pun yang Kulakukan dari diri-Ku sendiri. Aku hanya mengatakan apa yang diajarkan Bapa kepada-Ku.
Joh 8:29  Dan Dia yang mengutus Aku ada bersama-Ku. Ia tidak pernah membiarkan Aku sendirian, sebab Aku selalu melakukan apa yang menyenangkan hati-Nya."
Joh 8:30  Sesudah Yesus mengatakan semuanya itu, banyak orang percaya kepada-Nya.
Joh 8:31  Kemudian Yesus berkata kepada orang-orang Yahudi yang sudah percaya kepada-Nya, "Kalau kalian hidup menurut ajaran-Ku kalian sungguh-sungguh pengikut-Ku,
Joh 8:32  maka kalian akan mengenal Allah yang benar, dan oleh karena itu kalian akan dibebaskan."
Joh 8:33  "Kami ini keturunan Abraham," kata mereka. "Belum pernah kami menjadi hamba siapa pun! Apa maksud-Mu dengan berkata, 'Kalian akan dibebaskan'?"
Joh 8:34  "Sungguh benar kata-Ku ini," kata Yesus kepada mereka. "Orang yang berbuat dosa, adalah hamba dosa.
Joh 8:35  Dan seorang hamba tidak mempunyai tempat yang tetap di dalam keluarga, sedangkan anak untuk selama-lamanya mempunyai tempat dalam keluarga.
Joh 8:36  Karena itulah, kalau Anak membebaskan kalian, kalian sungguh-sungguh bebas.
Joh 8:37  Memang Aku tahu kalian ini keturunan Abraham. Namun kalian mau membunuh Aku, karena kalian tidak mau menerima pengajaran-Ku.
Joh 8:38  Apa yang Kulihat pada Bapa-Ku, itulah yang Kukatakan. Sedangkan kalian melakukan apa yang diajarkan bapakmu kepadamu."
Joh 8:39  Mereka menjawab, "Bapak kami ialah Abraham." "Sekiranya kalian betul-betul anak Abraham," kata Yesus, "pastilah kalian melakukan apa yang dilakukan oleh Abraham.
Joh 8:40  Aku menyampaikan kepadamu kebenaran yang Kudengar dari Allah, tetapi kalian mau membunuh Aku. Abraham tidak berbuat seperti itu!
Joh 8:41  Kalian melakukan apa yang dilakukan oleh bapakmu sendiri." Jawab mereka, "Kami bukan anak-anak haram. Bapa kami hanya satu, yaitu Allah sendiri."
Joh 8:42  Lalu Yesus berkata kepada mereka, "Sekiranya Allah itu Bapamu, kalian akan mengasihi Aku, sebab Aku datang dari Allah. Aku tidak datang dengan kemauan sendiri, tetapi Dialah yang mengutus Aku.
Joh 8:43  Apa sebabnya kalian tidak mengerti apa yang Kukatakan? Sebab kalian tidak tahan mendengar ajaran-Ku.
Joh 8:44  Iblislah bapakmu, dan kalian mau menuruti kemauan bapakmu. Sedari permulaan Iblis itu pembunuh. Ia tidak pernah memihak kebenaran, sebab tidak ada kebenaran padanya. Kalau ia berdusta, itu wajar, karena sudah begitu sifatnya. Ia pendusta dan asal segala dusta.
Joh 8:45  Tetapi Aku mengatakan kebenaran, dan karena itulah kalian tidak percaya kepada-Ku.
Joh 8:46  Siapa di antara kalian dapat membuktikan bahwa ada dosa pada-Ku? Kalau Aku mengatakan kebenaran, mengapa kalian tidak percaya kepada-Ku?
Joh 8:47  Orang yang berasal dari Allah, mendengar perkataan Allah. Tetapi kalian bukan dari Allah, itulah sebabnya kalian tidak mau mendengar."
Joh 8:48  Orang-orang Yahudi itu menjawab Yesus, "Bukankah benar kata kami bahwa Engkau orang Samaria yang kemasukan setan?"
Joh 8:49  Yesus menjawab, "Aku tidak kemasukan setan. Aku menghormati Bapa-Ku, tetapi kalian menghina Aku.
Joh 8:50  Aku tidak mencari kehormatan untuk diri sendiri. Ada satu yang mengusahakan kehormatan untuk-Ku, dan Dialah yang menghakimi.
Joh 8:51  Sungguh benar kata-Ku ini, orang yang menurut perkataan-Ku, selama-lamanya tidak akan mati."
Joh 8:52  Lalu orang-orang Yahudi itu berkata kepada Yesus, "Sekarang kami tahu Engkau ini betul-betul kemasukan setan! Abraham sendiri sudah mati, begitu juga semua nabi. Tetapi Engkau berkata, 'Orang yang menurut perkataan-Ku, selama-lamanya tidak akan mati.'
Joh 8:53  Kalau Abraham sendiri sudah mati, dan nabi-nabi semuanya juga sudah mati, Engkau ini siapa? Masakan Engkau lebih besar dari bapak kami Abraham!"
Joh 8:54  Yesus menjawab, "Sekiranya Aku mencari kehormatan untuk diri-Ku sendiri, kehormatan itu tak ada artinya. Yang menghormati Aku adalah Bapa-Ku yang kalian anggap Allah kalian,
Joh 8:55  padahal kalian tidak mengenal Dia. Tetapi Aku mengenal-Nya. Sekiranya Aku berkata bahwa Aku tidak mengenal Dia, maka Aku seorang pendusta, sama seperti kalian. Aku mengenal Dia, dan mentaati perkataan-Nya.
Joh 8:56  Bapakmu Abraham senang sekali bahwa ia akan melihat hari-Ku. Ia sudah melihatnya dan ia senang!"
Joh 8:57  Lalu orang-orang Yahudi berkata kepada Yesus, "Umur-Mu belum ada lima puluh tahun, dan Engkau sudah melihat Abraham?"
Joh 8:58  Yesus menjawab, "Sungguh Aku berkata kepadamu: sebelum Abraham lahir, Aku sudah ada."
Joh 8:59  Lalu orang-orang Yahudi itu mengambil batu untuk melempari-Nya; tetapi Yesus menyembunyikan diri, lalu pergi meninggalkan Rumah Tuhan.
Joh 9:1  Waktu Yesus berjalan, Ia melihat orang yang buta sejak lahir.
Joh 9:2  Pengikut-pengikut Yesus bertanya kepada Yesus, "Bapak Guru, mengapa orang ini dilahirkan buta? Apakah karena ia sendiri berdosa atau karena ibu bapaknya berdosa?"
Joh 9:3  Yesus menjawab, "Dia buta bukan karena dosanya sendiri atau dosa orang tuanya, tetapi supaya orang bisa melihat kuasa Allah bekerja dalam dirinya.
Joh 9:4  Selama masih siang, kita harus mengerjakan pekerjaan Dia yang mengutus Aku. Malam akan tiba, dan seorang pun tak akan dapat bekerja.
Joh 9:5  Selama Aku di dunia ini, Akulah terang dunia."
Joh 9:6  Setelah berkata begitu Yesus meludah ke tanah, dan mengaduk ludah-Nya itu dengan tanah. Kemudian Ia mengoleskannya pada mata orang itu,
Joh 9:7  lalu berkata kepadanya, "Pergilah bersihkan mukamu di Kolam Siloam." (Siloam berarti 'Diutus'.) Maka orang itu pergi membersihkan mukanya. Waktu ia kembali, ia sudah dapat melihat.
Joh 9:8  Tetangga-tetangganya dan orang-orang yang sebelumnya melihat dia mengemis, semuanya berkata, "Bukankah dia ini orang yang biasanya duduk minta-minta?"
Joh 9:9  Ada yang berkata, "Memang dia." Tetapi ada pula yang berkata, "Bukan, ia hanya mirip orang itu." Tetapi orang itu sendiri berkata, "Sayalah dia."
Joh 9:10  "Bagaimana jadinya sampai engkau bisa melihat?" kata mereka kepadanya.
Joh 9:11  Ia menjawab, "Orang yang bernama Yesus itu membuat sedikit lumpur, lalu mengoleskannya pada mata saya dan berkata, 'Pergilah bersihkan mukamu di Kolam Siloam.' Lalu saya pergi. Dan ketika saya membersihkan muka saya, saya bisa melihat."
Joh 9:12  "Di mana orang itu?" tanya mereka. Ia menjawab, "Tidak tahu."
Joh 9:13  Hari itu adalah hari Sabat waktu Yesus mengaduk tanah dengan ludah-Nya untuk membuat orang buta itu bisa melihat. Maka orang yang tadinya buta itu dibawa kepada orang-orang Farisi.
Joh 9:15  Mereka juga bertanya kepadanya bagaimana ia dapat melihat. Ia menjawab, "Dia menaruh lumpur di mata saya, lalu saya membersihkannya dan saya bisa melihat."
Joh 9:16  Beberapa di antara orang-orang Farisi itu berkata, "Tak mungkin orang yang melakukan ini berasal dari Allah, sebab Ia tidak mengindahkan hari Sabat." Tetapi orang lain berkata, "Mana mungkin orang yang berdosa melakukan keajaiban-keajaiban seperti ini?" Lalu timbullah pertentangan pendapat di antara mereka.
Joh 9:17  Maka orang-orang Farisi itu bertanya lagi kepada orang itu, "Apa pendapatmu tentang Dia yang membuat engkau melihat?" "Dia nabi," jawab orang itu.
Joh 9:18  Tetapi para pemimpin Yahudi itu tidak mau percaya bahwa orang itu memang buta sebelumnya dan sekarang dapat melihat. Karena itu mereka memanggil orang tuanya,
Joh 9:19  dan bertanya, "Benarkah ini anakmu yang katamu lahir buta? Bagaimana ia bisa melihat sekarang?"
Joh 9:20  Ibu bapak orang itu menjawab, "Memang ini anak kami; dan ia memang buta sejak lahir.
Joh 9:21  Tetapi bagaimana ia bisa melihat sekarang, kami tidak tahu. Dan siapa yang membuat dia bisa melihat, itu pun kami tidak tahu. Tanya saja kepadanya, ia sudah dewasa; ia dapat menjawab sendiri."
Joh 9:22  Ibu bapak orang itu berkata begitu sebab mereka takut kepada para pemimpin Yahudi; karena mereka itu sudah sepakat, bahwa orang yang mengakui Yesus sebagai Raja Penyelamat, tidak boleh lagi masuk rumah ibadat.
Joh 9:23  Itu sebabnya ibu bapak orang itu berkata, "Ia sudah dewasa; tanya saja kepadanya."
Joh 9:24  Lalu mereka memanggil lagi orang yang tadinya buta itu, dan berkata kepadanya, "Bersumpahlah bahwa engkau akan berkata yang benar. Kami tahu orang itu orang berdosa."
Joh 9:25  "Apakah Dia berdosa atau tidak," jawab orang itu, "saya tidak tahu. Tetapi satu hal saya tahu; dahulu saya buta, sekarang saya melihat."
Joh 9:26  Lalu mereka berkata lagi kepadanya, "Ia berbuat apa kepadamu? Bagaimana Ia membuat engkau melihat?"
Joh 9:27  Orang itu menjawab, "Sudah saya ceritakan kepadamu, tetapi kalian tidak mau mendengarkan. Mengapa kalian mau mendengarnya lagi? Barangkali kalian mau menjadi pengikut-pengikut-Nya juga?"
Joh 9:28  Lalu mereka memaki dia dan berkata, "Engkaulah pengikut-Nya; kami bukan! Kami pengikut Musa.
Joh 9:29  Kami tahu bahwa Allah sudah berbicara kepada Musa. Tetapi tentang orang itu, kami tidak tahu dari mana asal-Nya."
Joh 9:30  Orang itu menjawab, "Aneh sekali bahwa kalian tidak tahu dari mana asal-Nya, sedangkan Ia sudah membuat saya bisa melihat.
Joh 9:31  Kita tahu bahwa Allah tidak mendengarkan orang berdosa, melainkan orang yang menghormati Allah, dan melakukan kehendak-Nya.
Joh 9:32  Sejak permulaan dunia belum pernah terdengar ada orang membuat orang yang lahir buta bisa melihat.
Joh 9:33  Kalau orang itu bukan dari Allah, Ia tak akan dapat berbuat apa-apa."
Joh 9:34  Jawab mereka, "Apa? Engkau yang penuh dosa sejak lahir, engkau mau mengajar kami?" Maka sejak itu ia dilarang masuk ke rumah ibadat.
Joh 9:35  Yesus mendengar bahwa mereka sudah mengucilkan orang itu dari rumah ibadat. Ia mencari orang itu lalu berkata kepadanya, "Apakah engkau percaya kepada Anak Manusia?"
Joh 9:36  Orang itu menjawab, "Siapa Dia, Tuan? Tolong beritahukan supaya saya percaya kepada-Nya."
Joh 9:37  "Engkau sudah melihat Dia," jawab Yesus. "Dialah yang sekarang ini sedang berbicara dengan engkau."
Joh 9:38  "Saya percaya, Tuhan," kata orang itu, lalu sujud di hadapan Yesus.
Joh 9:39  Yesus berkata, "Aku datang ke dunia ini untuk menghakimi; supaya orang yang buta dapat melihat, dan orang yang dapat melihat, menjadi buta."
Joh 9:40  Beberapa orang Farisi yang ada di situ mendengar Yesus berkata begitu, lalu mereka bertanya kepada-Nya, "Maksud-Mu kami ini buta juga?"
Joh 9:41  Yesus menjawab, "Sekiranya kalian buta, kalian tidak berdosa. Tetapi sebab kalian berkata, 'Kami melihat,' itu berarti kalian masih berdosa."
Joh 10:1  "Sungguh benar kata-Ku ini: Orang yang masuk ke dalam kandang domba lewat pagar, dan tidak melalui pintu, tetapi memanjat lewat jalan lain, orang itu pencuri dan perampok.
Joh 10:2  Tetapi orang yang masuk melalui pintu, dialah gembala domba.
Joh 10:3  Penjaga kandang membuka pintu untuk dia, dan domba-domba mengikuti suaranya pada waktu ia memanggil mereka dengan namanya masing-masing dan menuntun mereka ke luar.
Joh 10:4  Setelah domba-domba itu dibawa ke luar, gembala itu berjalan di depan, dan domba-domba itu mengikuti dia sebab mereka mengenal suaranya.
Joh 10:5  Mereka tidak akan mau mengikuti orang lain, malah akan lari dari orang itu, sebab tidak mengenal suaranya."
Joh 10:6  Yesus menceritakan perumpamaan itu, tetapi mereka tidak mengerti apa yang dimaksudkan-Nya.
Joh 10:7  Maka Yesus berkata sekali lagi, "Sungguh benar kata-Ku ini: Akulah pintu untuk domba.
Joh 10:8  Semua yang datang sebelum Aku adalah pencuri dan perampok, tetapi domba-domba tidak mendengarkan suara mereka.
Joh 10:9  Akulah pintu. Siapa masuk melalui Aku akan selamat; ia keluar masuk dan mendapat makanan.
Joh 10:10  Pencuri datang hanya untuk mencuri, untuk membunuh dan untuk merusak. Tetapi Aku datang supaya manusia mendapat hidup--hidup berlimpah-limpah.
Joh 10:11  Akulah gembala yang baik. Gembala yang baik memberikan nyawanya untuk domba-dombanya.
Joh 10:12  Orang upahan yang bukan gembala dan bukan juga pemilik domba-domba itu, akan lari meninggalkan domba-domba kalau ia melihat serigala datang. Maka domba-domba itu akan diterkam dan diceraiberaikan serigala.
Joh 10:13  Orang upahan itu lari, sebab ia bekerja untuk upah. Ia tidak mempedulikan domba-domba itu.
Joh 10:14  Akulah gembala yang baik. Sama seperti Bapa mengenal Aku dan Aku mengenal Bapa, begitu juga Aku mengenal domba-domba-Ku dan mereka pun mengenal Aku. Aku menyerahkan nyawa-Ku untuk mereka.
Joh 10:16  Masih ada domba-domba lain yang juga milik-Ku, tetapi tidak tergolong dalam kawanan domba ini. Mereka juga harus Kubawa dan mereka akan mendengarkan suara-Ku. Mereka semuanya akan menjadi satu kawanan dengan satu gembala.
Joh 10:17  Bapa mengasihi Aku sebab Aku menyerahkan nyawa-Ku, untuk menerimanya kembali.
Joh 10:18  Tidak seorang pun dapat mengambilnya daripada-Ku. Aku menyerahkannya atas kemauan sendiri. Aku berkuasa untuk menyerahkannya, dan berkuasa mengambilnya kembali. Itulah tugas yang Aku terima dari Bapa-Ku."
Joh 10:19  Karena Yesus berkata begitu, orang-orang Yahudi mulai bertengkar.
Joh 10:20  Banyak yang berkata, "Ia kemasukan setan! Ia gila! Untuk apa kalian dengarkan Dia?"
Joh 10:21  Tetapi ada juga yang berkata, "Orang yang kemasukan setan tidak berbicara begitu! Dapatkah setan membuat orang buta bisa melihat?"
Joh 10:22  Di Yerusalem orang-orang sedang merayakan Hari Raya Peresmian Rumah Tuhan. Pada waktu itu musim dingin.
Joh 10:23  Yesus sedang berjalan di Serambi Salomo di dalam Rumah Tuhan,
Joh 10:24  ketika orang-orang Yahudi datang berkumpul di sekeliling Yesus. Mereka berkata, "Sampai kapan Engkau mau membiarkan kami ragu-ragu? Katakanlah terus terang, kalau Engkau sungguh-sungguh Raja Penyelamat."
Joh 10:25  Yesus menjawab, "Sudah Kukatakan kepadamu, tetapi kalian tidak percaya. Pekerjaan-pekerjaan yang Kulakukan atas nama Bapa-Ku, memberi bukti tentang Aku.
Joh 10:26  Kalian tidak percaya sebab kalian tidak tergolong domba-domba-Ku.
Joh 10:27  Domba-domba-Ku mendengarkan suara-Ku. Aku kenal mereka, dan mereka mengikuti Aku.
Joh 10:28  Aku memberi mereka hidup sejati dan kekal, dan untuk selamanya mereka tak akan binasa. Tak seorang pun dapat merampas mereka dari tangan-Ku.
Joh 10:29  Bapa-Ku, yang sudah memberikan mereka kepada-Ku, melebihi segalanya. Dan tidak seorang pun dapat merampas mereka dari tangan Bapa.
Joh 10:30  Aku dan Bapa adalah satu."
Joh 10:31  Lalu orang-orang Yahudi mengambil lagi batu untuk melempari Yesus.
Joh 10:32  Tetapi Yesus berkata kepada mereka, "Kalian sudah melihat Aku melakukan banyak pekerjaan baik, yang ditugaskan Bapa kepada-Ku. Dari semua pekerjaan itu, manakah yang menyebabkan kalian mau melempari Aku?"
Joh 10:33  Orang-orang Yahudi itu menjawab, "Bukan karena pekerjaan-pekerjaan-Mu yang baik itu kami mau melempari Engkau dengan batu, tetapi karena Engkau menghujat Allah. Engkau seorang manusia, mau menjadikan diri-Mu Allah."
Joh 10:34  Lalu Yesus menjawab, "Bukankah di dalam Buku Hukummu tertulis: Allah berkata, 'Kalian adalah ilah'?
Joh 10:35  Kita tahu bahwa apa yang tertulis dalam Alkitab berlaku untuk selamanya. Jadi, kalau Allah memberi sebutan 'ilah' kepada orang-orang yang menerima perkataan-Nya,
Joh 10:36  mengapa kalian mengatakan Aku menghujat Allah karena berkata Aku Anak Allah? Padahal Aku dipilih oleh Bapa dan diutus ke dunia.
Joh 10:37  Kalau Aku tidak melakukan pekerjaan yang ditugaskan Bapa, jangan percaya kepada-Ku.
Joh 10:38  Tetapi karena Aku melakukannya, percayalah akan apa yang Kulakukan itu, meskipun kalian tidak mau percaya kepada-Ku. Dengan demikian kalian tahu dan mengerti bahwa Bapa tetap bersatu dengan Aku, dan Aku tetap bersatu dengan Bapa."
Joh 10:39  Mereka berusaha lagi menangkap Yesus, tetapi Ia lolos dari mereka.
Joh 10:40  Yesus kembali ke seberang Sungai Yordan, di tempat Yohanes dahulu membaptis, lalu tinggal di sana.
Joh 10:41  Banyak orang datang kepada-Nya. Mereka berkata, "Yohanes tidak melakukan keajaiban-keajaiban, tetapi semua yang dikatakannya tentang orang ini benar."
Joh 10:42  Lalu banyak orang di sana percaya kepada Yesus.
Joh 11:1  Seorang yang bernama Lazarus tinggal di Betania bersama-sama dengan saudaranya Maria dan Marta.
Joh 11:2  Maria ialah wanita yang menuang minyak wangi ke kaki Tuhan, dan menyekanya dengan rambutnya. Suatu ketika Lazarus jatuh sakit.
Joh 11:3  Kedua saudaranya mengabarkan kepada Yesus, "Tuhan, saudara kami yang Tuhan kasihi itu sakit."
Joh 11:4  Ketika Yesus mendengar kabar itu, Ia berkata, "Penyakit ini tidak akan menyebabkan kematian. Ini terjadi supaya Allah diagungkan, dan supaya karenanya Anak Allah juga diagungkan."
Joh 11:5  Yesus mengasihi Marta, Maria dan Lazarus.
Joh 11:6  Tetapi ketika Yesus mendapat berita bahwa Lazarus sakit, Ia sengaja tinggal di tempat-Nya dua hari lagi.
Joh 11:7  Sesudah itu baru Ia berkata kepada pengikut-pengikut-Nya, "Mari kita kembali ke Yudea."
Joh 11:8  Mereka menjawab, "Bapak Guru, baru saja orang-orang Yahudi mau melempari Bapak dengan batu dan sekarang Bapak mau kembali lagi ke sana?"
Joh 11:9  "Bukankah siang hari lamanya dua belas jam?" kata Yesus. "Orang yang berjalan di waktu siang, tidak tersandung sebab ia melihat terang dunia ini.
Joh 11:10  Tetapi orang yang berjalan di waktu malam tersandung, sebab tidak ada terang padanya."
Joh 11:11  Begitulah kata Yesus. Kemudian Ia berkata lagi, "Sahabat kita Lazarus sudah tidur, tetapi Aku akan pergi membangunkan dia."
Joh 11:12  Pengikut-pengikut Yesus berkata, "Tuhan, kalau Lazarus tidur, nanti ia akan sembuh."
Joh 11:13  Maksud Yesus ialah bahwa Lazarus sudah mati. Tetapi mereka menyangka maksud Yesus adalah bahwa Lazarus tidur biasa.
Joh 11:14  Karena itu Yesus berkata kepada mereka dengan terus terang, "Lazarus sudah mati.
Joh 11:15  Tetapi Aku senang juga Aku tidak ada di sana, sebab lebih baik untuk kalian, supaya kalian dapat percaya. Marilah kita pergi sekarang kepada Lazarus."
Joh 11:16  Tomas yang disebut "Si Kembar" berkata kepada teman-temannya pengikut-pengikut Yesus, "Mari kita ikut, biar kita mati bersama Dia!"
Joh 11:17  Ketika Yesus sampai di tempat itu, Lazarus sudah empat hari lamanya dikubur.
Joh 11:18  Betania dekat Yerusalem, kira-kira tiga kilometer jauhnya.
Joh 11:19  Banyak orang Yahudi telah datang mengunjungi Marta dan Maria untuk menghibur mereka karena kematian saudaranya.
Joh 11:20  Ketika Marta mendengar Yesus datang, ia keluar untuk menyambut Yesus, sedangkan Maria tinggal di rumah.
Joh 11:21  Kata Marta kepada Yesus, "Tuhan, sekiranya Tuhan ada di sini waktu itu, pasti saudara saya tidak meninggal.
Joh 11:22  Namun begitu saya tahu bahwa sekarang ini juga Allah akan memberi apa saja yang Tuhan minta kepada-Nya."
Joh 11:23  "Saudaramu akan hidup kembali," kata Yesus kepada Marta.
Joh 11:24  Marta menjawab, "Saya tahu Lazarus akan hidup kembali bila orang mati dibangkitkan pada Hari Kiamat."
Joh 11:25  "Akulah yang memberi hidup dan membangkitkan orang mati," kata Yesus kepada Marta. "Orang yang percaya kepada-Ku akan hidup, walaupun ia sudah mati.
Joh 11:26  Dan orang hidup yang percaya kepada-Ku, selama-lamanya tidak akan mati. Percayakah engkau akan hal itu?"
Joh 11:27  "Tuhan," jawab Marta, "saya percaya Tuhan Anak Allah, Raja Penyelamat yang akan datang ke dunia ini."
Joh 11:28  Setelah Marta berkata begitu, ia pergi memanggil Maria, saudaranya dan berbisik kepadanya, "Bapak Guru ada di sini; Ia menanyakan engkau."
Joh 11:29  Mendengar itu, Maria cepat-cepat bangun, lalu pergi menemui Yesus.
Joh 11:30  Waktu itu, Yesus belum masuk ke desa. Ia masih di tempat Marta menjumpai-Nya.
Joh 11:31  Orang-orang Yahudi yang sedang menghibur Maria di rumah, melihat Maria bangun dan cepat-cepat keluar; jadi mereka pergi mengikuti dia, sebab mereka menyangka ia pergi ke kubur untuk menangis.
Joh 11:32  Waktu Maria sampai di tempat Yesus dan melihat Dia, berlututlah Maria di depan-Nya dan berkata, "Tuhan, sekiranya Tuhan ada di sini waktu itu, pasti saudara saya tidak meninggal."
Joh 11:33  Ketika Yesus melihat Maria menangis, dan orang-orang Yahudi yang datang bersama Maria itu juga menangis, hati-Nya sedih, dan Ia tampak terharu sekali.
Joh 11:34  Maka bertanyalah Ia kepada mereka, "Di mana kalian menguburkan dia?" "Mari lihat, Tuhan," jawab mereka.
Joh 11:35  Lalu Yesus menangis.
Joh 11:36  Maka orang-orang Yahudi itu berkata, "Lihat, bukan main kasih-Nya kepada Lazarus!"
Joh 11:37  Tetapi ada di antara mereka yang berkata, "Ia membuat orang buta melihat, mengapa Ia tidak bisa mencegah supaya Lazarus jangan mati?"
Joh 11:38  Yesus sangat terharu lagi, lalu pergi ke kuburan. Kuburan itu adalah sebuah gua yang ditutup dengan batu besar.
Joh 11:39  "Singkirkan batu itu," kata Yesus. Marta, saudara orang yang meninggal itu, menjawab, "Tetapi, Tuhan, ia sudah empat hari dikubur. Tentu sudah berbau busuk!"
Joh 11:40  Yesus berkata kepada Marta, "Bukankah sudah Kukatakan kepadamu: Kalau engkau percaya, engkau akan melihat betapa besar kuasa Allah!"
Joh 11:41  Maka mereka menyingkirkan batu itu. Kemudian Yesus menengadah ke langit dan berkata, "Terima kasih, Bapa, karena Engkau telah mendengarkan Aku.
Joh 11:42  Aku tahu Engkau selalu mendengarkan Aku, tetapi Aku mengatakan ini, untuk orang-orang yang ada di sini; supaya mereka percaya bahwa Engkaulah yang mengutus Aku."
Joh 11:43  Sesudah berkata begitu, Yesus berseru dengan suara keras, "Lazarus, keluar!"
Joh 11:44  Maka keluarlah orang yang sudah mati itu. Tangan dan kakinya masih terbungkus kain kafan, dan mukanya tertutup dengan kain penutup muka. "Lepaskan kain kafannya supaya ia bebas berjalan," kata Yesus kepada orang-orang di situ.
Joh 11:45  Banyak dari orang-orang Yahudi yang datang mengunjungi Maria, percaya kepada Yesus waktu mereka melihat kejadian itu.
Joh 11:46  Tetapi beberapa di antara mereka pergi kepada orang Farisi dan melaporkan apa yang sudah dilakukan oleh Yesus.
Joh 11:47  Karena itu orang-orang Farisi dan imam-imam kepala mengadakan rapat dengan Mahkamah Agama. Mereka berkata, "Kita harus berbuat apa? Orang ini membuat banyak keajaiban!
Joh 11:48  Kalau kita membiarkan Dia terus begini semua orang akan percaya kepada-Nya. Dan akhirnya penguasa Roma akan datang dan menghancurkan Rumah Tuhan dan seluruh bangsa kita!"
Joh 11:49  Seorang dari mereka yang bernama Kayafas, imam agung pada tahun itu, berkata, "Kalian tidak tahu apa-apa.
Joh 11:50  Apakah kalian tidak menyadari bahwa demi rakyat, lebih baik satu orang mati daripada seluruh bangsa hancur?"
Joh 11:51  Sebenarnya Kayafas mengatakan itu, bukan dari pikirannya sendiri. Tetapi sebagai imam agung tahun itu ia menubuatkan bahwa Yesus akan mati untuk bangsa Yahudi.
Joh 11:52  Dan bukan untuk bangsa Yahudi saja, tetapi juga untuk mengumpulkan dan mempersatukan anak-anak Allah yang tercerai-berai.
Joh 11:53  Mulai hari itu para penguasa Yahudi bersekongkol untuk membunuh Yesus.
Joh 11:54  Karena itu Yesus tidak lagi tampil di muka umum di kalangan orang Yahudi. Ia meninggalkan Yudea, lalu pergi ke kota yang bernama Efraim dekat padang gurun. Di situ Ia tinggal bersama pengikut-pengikut-Nya.
Joh 11:55  Pada waktu itu sudah dekat Hari Raya Paskah Yahudi. Banyak orang dari desa-desa sudah pergi ke Yerusalem untuk menjalankan upacara pembersihan diri sebelum perayaan itu.
Joh 11:56  Mereka mencari Yesus, dan waktu berkumpul di Rumah Tuhan, mereka berkata satu sama lain, "Bagaimana pendapatmu? Mungkin Ia tidak datang ke perayaan ini."
Joh 11:57  Mereka mengatakan itu sebab imam-imam kepala dan orang-orang Farisi sudah mengeluarkan perintah bahwa orang yang tahu di mana Yesus berada, harus melaporkannya supaya Ia dapat ditangkap.
Joh 12:1  Enam hari sebelum Hari Raya Paskah, Yesus pergi ke Betania. Di tempat itu tinggal Lazarus yang sudah dibangkitkan oleh-Nya dari mati.
Joh 12:2  Di sana Yesus dijamu oleh mereka, dan Marta melayani. Lazarus dan tamu-tamu duduk makan bersama-sama dengan Yesus.
Joh 12:3  Kemudian Maria datang dengan kira-kira setengah liter minyak wangi narwastu yang mahal sekali. Ia menuang minyak itu ke kaki Yesus, lalu menyekanya dengan rambutnya. Seluruh rumah itu menjadi harum karena minyak wangi itu.
Joh 12:4  Tetapi Yudas Iskariot, salah seorang pengikut Yesus--yang kemudian mengkhianati-Nya--berkata,
Joh 12:5  "Mengapa minyak wangi itu tidak dijual saja dengan harga tiga ratus uang perak, dan uangnya diberikan kepada orang miskin?"
Joh 12:6  Yudas berkata begitu bukan karena ia memperhatikan orang miskin, tetapi karena ia pencuri. Ia sering mengambil uang dari kas bersama yang disimpan padanya.
Joh 12:7  Tetapi Yesus berkata, "Biarkan wanita itu! Ia melakukan ini untuk hari penguburan-Ku.
Joh 12:8  Orang miskin selalu ada di antara kalian, tetapi Aku tidak."
Joh 12:9  Banyak orang Yahudi mendengar bahwa Yesus ada di Betania, jadi mereka pergi ke sana. Mereka pergi bukan saja karena Yesus, tetapi juga karena mereka mau melihat Lazarus yang sudah dibangkitkan dari mati oleh-Nya.
Joh 12:10  Itu sebabnya imam-imam kepala mau membunuh Lazarus juga;
Joh 12:11  karena ia menyebabkan banyak orang Yahudi meninggalkan mereka dan percaya kepada Yesus.
Joh 12:12  Keesokan harinya orang banyak yang sudah datang untuk merayakan Paskah mendengar bahwa Yesus sedang dalam perjalanan menuju Yerusalem.
Joh 12:13  Maka mereka mengambil daun-daun palem lalu pergi menyambut Dia, sambil bersorak-sorak, "Pujilah Allah! Diberkatilah Dia yang datang atas nama Tuhan. Diberkatilah Raja Israel!"
Joh 12:14  Yesus mendapat seekor keledai muda, dan menungganginya. Maka terjadilah yang tertulis dalam Alkitab:
Joh 12:15  "Jangan takut, putri Sion! Lihatlah Rajamu datang, menunggang seekor keledai muda!"
Joh 12:16  Pada waktu itu pengikut-pengikut Yesus belum mengerti semuanya itu. Tetapi setelah Yesus diagungkan dengan kematian-Nya, barulah mereka teringat bahwa yang dilakukan orang-orang terhadap-Nya sudah tertulis dalam Alkitab mengenai Dia.
Joh 12:17  Orang-orang yang hadir pada waktu Yesus memanggil Lazarus keluar dari kubur dan membangkitkannya dari mati, terus memberi kesaksian tentang hal itu.
Joh 12:18  Itu sebabnya orang banyak itu pergi kepada Yesus sebab mereka mendengar bahwa Dialah yang telah membuat keajaiban itu.
Joh 12:19  Maka orang-orang Farisi berkata satu sama lain, "Kita tidak bisa berbuat apa-apa! Lihat saja, seluruh dunia pergi ikut Dia!"
Joh 12:20  Di antara orang-orang yang pergi ke Yerusalem untuk berbakti pada waktu perayaan itu ada juga beberapa orang Yunani.
Joh 12:21  Mereka datang kepada Filipus dan berkata, "Saudara, kalau dapat, kami ingin bertemu dengan Yesus." (Filipus berasal dari Betsaida di Galilea.)
Joh 12:22  Maka Filipus pergi memberitahukan hal itu kepada Andreas, dan kemudian mereka berdua menyampaikannya kepada Yesus.
Joh 12:23  Yesus berkata kepada mereka, "Sudah waktunya Anak Manusia diagungkan.
Joh 12:24  Sungguh benar kata-Ku ini: Kalau sebutir gandum tidak ditanam ke dalam tanah dan mati, ia akan tetap tinggal sebutir. Tetapi kalau butir gandum itu mati, baru ia akan menghasilkan banyak gandum.
Joh 12:25  Orang yang mencintai hidupnya akan kehilangan hidupnya. Tetapi orang yang membenci hidupnya di dunia ini, akan memeliharanya untuk hidup sejati dan kekal.
Joh 12:26  Orang yang mau melayani Aku harus mengikuti Aku; supaya pelayan-Ku dapat bersama-Ku di mana Aku berada. Orang yang melayani Aku akan dihormati Bapa-Ku."
Joh 12:27  "Hati-Ku cemas; apa yang harus Kukatakan sekarang? Haruskah Aku mengatakan, 'Bapa, luputkanlah Aku dari saat ini'? Tetapi justru untuk mengalami saat penderitaan inilah Aku datang.
Joh 12:28  Bapa, agungkanlah nama-Mu!" Maka terdengar suara dari langit mengatakan, "Aku sudah mengagungkan-Nya, dan Aku akan mengagungkan-Nya lagi."
Joh 12:29  Orang banyak yang berada di situ mendengar suara itu. Mereka berkata, "Itu guntur!" Tetapi ada juga yang berkata, "Bukan! Malaikat berbicara kepada-Nya!"
Joh 12:30  Lalu Yesus berkata kepada mereka, "Suara itu terdengar, bukan untuk kepentingan-Ku, tetapi untuk kepentinganmu.
Joh 12:31  Sekarang sudah waktunya dunia dihakimi; sekarang penguasa dunia ini digulingkan.
Joh 12:32  Tetapi Aku ini, kalau Aku sudah ditinggikan di atas bumi, Aku akan menarik semua orang kepada-Ku."
Joh 12:33  Ia berkata begitu untuk menunjukkan bagaimana caranya Ia akan mati.
Joh 12:34  Orang banyak itu berkata kepada-Nya, "Menurut Buku Hukum kami, Raja Penyelamat akan hidup selama-lamanya. Bagaimana Engkau dapat berkata bahwa Anak Manusia harus ditinggikan di atas bumi? Siapa Anak Manusia itu?"
Joh 12:35  Yesus menjawab, "Hanya untuk sebentar saja terang itu ada di antara kalian. Jadi, berjalanlah selama terang itu masih ada, supaya kalian jangan ditimpa kegelapan. Orang yang berjalan di dalam gelap tidak tahu ke mana ia pergi.
Joh 12:36  Percayalah kepada terang itu, selama terang itu masih ada padamu, supaya kalian menjadi anak-anak terang." Sesudah Yesus berkata begitu, Ia pergi dari sana dan tidak mau menunjukkan diri kepada mereka.
Joh 12:37  Walaupun sudah banyak keajaiban yang dibuat Yesus di depan mereka, mereka tidak percaya kepada-Nya.
Joh 12:38  Maka terjadilah apa yang dikatakan Nabi Yesaya, "Tuhan, siapakah yang percaya pada berita kami? Kepada siapakah kuasa Tuhan diperlihatkan?"
Joh 12:39  Itu sebabnya mereka tidak dapat percaya, sebab Yesaya sudah berkata juga, "Allah berkata,
Joh 12:40  'Aku membutakan mata mereka, membuat mereka tegar hati; supaya mata mereka jangan melihat, pikiran mereka jangan mengerti. Supaya mereka jangan kembali kepada-Ku, lalu Aku menyembuhkan mereka.'"
Joh 12:41  Yesaya berkata begitu sebab ia sudah melihat kebesaran Yesus, dan berbicara mengenai-Nya.
Joh 12:42  Walaupun begitu, banyak orang, bahkan di antara penguasa Yahudi percaya kepada Yesus. Tetapi mereka tidak berani mengakui itu dengan terus terang, sebab mereka takut jangan-jangan orang Farisi tidak memperbolehkan mereka masuk rumah ibadat.
Joh 12:43  Mereka lebih suka mendapat pujian manusia daripada penghargaan Allah.
Joh 12:44  Lalu Yesus berseru, "Orang yang percaya kepada-Ku, bukan kepada Aku ia percaya, tetapi kepada Dia yang mengutus Aku.
Joh 12:45  Dan orang yang melihat Aku, melihat Dia yang mengutus Aku.
Joh 12:46  Aku datang ke dunia ini sebagai terang, supaya semua orang yang percaya kepada-Ku jangan tinggal di dalam gelap.
Joh 12:47  Orang yang mendengar ajaran-Ku, tetapi tidak menurutinya--bukan Aku yang menghukum dia. Sebab Aku datang bukan untuk menghakimi dunia ini, tetapi untuk menyelamatkannya.
Joh 12:48  Orang yang menolak Aku dan tidak mendengarkan perkataan-Ku, sudah ada yang menghakiminya. Perkataan yang Aku sampaikan, itulah yang akan menghakiminya pada Hari Kiamat.
Joh 12:49  Sebab Aku tidak berbicara dari kemauan-Ku sendiri; Bapa yang mengutus Aku, Dialah yang memerintahkan kepada-Ku apa yang harus Kukatakan dan Kusampaikan.
Joh 12:50  Dan Aku tahu bahwa perintah-Nya itu memberi hidup sejati dan kekal. Maka Aku menyampaikan seperti yang diajarkan Bapa kepada-Ku."
Joh 13:1  Sehari sebelum Hari Raya Paskah, Yesus tahu bahwa sudah waktunya Ia meninggalkan dunia ini untuk kembali kepada Bapa-Nya. Ia mengasihi orang-orang yang menjadi milik-Nya di dunia, dan Ia tetap mengasihi mereka sampai penghabisan.
Joh 13:2  Yesus dan pengikut-pengikut-Nya sedang makan malam. Iblis sudah memasukkan niat di dalam hati Yudas anak Simon Iskariot untuk mengkhianati Yesus.
Joh 13:3  Yesus tahu bahwa Bapa sudah menyerahkan seluruh kekuasaan kepada-Nya. Ia tahu juga bahwa Ia datang dari Allah dan akan kembali kepada Allah.
Joh 13:4  Sebab itu Ia berdiri, membuka jubah-Nya, dan mengikat anduk pada pinggang-Nya.
Joh 13:5  Sesudah itu Ia menuang air ke dalam sebuah baskom, lalu mulai membasuh kaki pengikut-pengikut-Nya dan mengeringkannya dengan anduk yang terikat di pinggang-Nya.
Joh 13:6  Sampailah Ia kepada Simon Petrus, yang berkata, "Tuhan, masakan Tuhan yang membasuh kaki saya?"
Joh 13:7  Yesus menjawab, "Sekarang engkau tidak mengerti apa yang Kulakukan ini, tetapi nanti engkau akan mengerti."
Joh 13:8  "Jangan, Tuhan," kata Petrus kepada Yesus, "Jangan sekali-kali Tuhan membasuh kaki saya!" Tetapi Yesus menjawab, "Kalau Aku tidak membasuhmu, engkau tidak ada hubungan dengan Aku."
Joh 13:9  Simon Petrus berkata, "Kalau begitu, Tuhan, jangan hanya kaki saya tetapi tangan dan kepala saya juga!"
Joh 13:10  "Orang yang sudah mandi, sudah bersih seluruhnya," kata Yesus kepada Petrus. "Ia tidak perlu dibersihkan lagi; kecuali kakinya. Kalian ini sudah bersih, tetapi tidak semuanya."
Joh 13:11  (Yesus sudah tahu siapa yang akan mengkhianati-Nya. Itu sebabnya Ia berkata, "Kalian ini sudah bersih, tetapi tidak semuanya.")
Joh 13:12  Sesudah Yesus membasuh kaki mereka, Ia memakai kembali jubah-Nya dan duduk lagi. Lalu Ia berkata kepada mereka, "Mengertikah kalian apa yang baru saja Kulakukan kepadamu?
Joh 13:13  Kalian memanggil Aku Guru dan Tuhan. Dan memang demikian.
Joh 13:14  Kalau Aku sebagai Tuhan dan Gurumu membasuh kakimu, kalian wajib juga saling membasuh kaki.
Joh 13:15  Aku memberi teladan ini kepada kalian, supaya kalian juga melakukan apa yang sudah Kulakukan kepadamu.
Joh 13:16  Sungguh benar kata-Ku ini: Seorang hamba tidak lebih besar dari tuannya, dan seorang utusan tidak lebih besar dari yang mengutusnya.
Joh 13:17  Kalau kalian sudah tahu semuanya ini, bahagialah kalian jika melakukannya.
Joh 13:18  Apa yang Kukatakan ini bukanlah mengenai kalian semua. Aku tahu siapa-siapa yang sudah Kupilih. Tetapi apa yang tertulis dalam Alkitab, harus terjadi, yaitu 'Orang yang makan bersama Aku, akan melawan Aku.'
Joh 13:19  Hal itu Kusampaikan kepadamu sekarang, sebelum terjadi, supaya kalau hal itu terjadi nanti, kalian akan percaya bahwa Akulah Dia yang disebut AKU ADA.
Joh 13:20  Sungguh benar kata-Ku ini: Siapa menerima orang yang Kuutus, menerima Aku. Dan siapa menerima Aku, menerima Dia yang mengutus Aku."
Joh 13:21  Setelah Yesus berkata begitu, Ia menjadi terharu sekali. Lalu Ia mengatakan, "Sungguh benar kata-Ku ini: Salah seorang dari antara kalian akan mengkhianati Aku."
Joh 13:22  Pengikut-pengikut-Nya memandang satu sama lain dengan sangat heran karena tidak tahu siapa yang dimaksud.
Joh 13:23  Pengikut yang dikasihi Yesus duduk di sebelah Yesus.
Joh 13:24  Simon Petrus memberi isyarat kepadanya, supaya ia bertanya kepada Yesus siapa yang dimaksudkan.
Joh 13:25  Maka pengikut itu merapat pada Yesus, dan bertanya, "Siapa dia, Tuhan?"
Joh 13:26  Yesus menjawab, "Orang yang Kuberikan roti yang Kucelupkan ke dalam mangkok, dialah orangnya." Maka Yesus mengambil sepotong roti, mencelupnya ke dalam mangkok; lalu memberikannya kepada Yudas anak Simon Iskariot.
Joh 13:27  Segera setelah Yudas menerima roti itu, Iblis masuk ke dalam hatinya. Lalu Yesus berkata kepadanya, "Lakukanlah cepat apa yang mau kaulakukan."
Joh 13:28  Tidak seorang pun dari mereka yang duduk makan di situ mengerti mengapa Yesus berkata begitu kepada Yudas.
Joh 13:29  Ada yang menyangka Yesus menyuruh Yudas membeli sesuatu yang diperlukan untuk pesta itu, atau memberi sedikit uang kepada orang miskin--sebab Yudas adalah pemegang uang kas mereka.
Joh 13:30  Setelah Yudas menerima roti itu, ia langsung keluar. Hari sudah malam.
Joh 13:31  Sesudah Yudas pergi, Yesus berkata, "Sekarang Anak Manusia diagungkan, dan Allah diagungkan melalui Dia.
Joh 13:32  Kalau Allah diagungkan melalui Dia, Ia pun akan diagungkan Allah melalui diri-Nya sendiri. Malah Allah akan mengagungkan-Nya dengan segera.
Joh 13:33  Anak-anak-Ku, Aku tidak akan tinggal lama lagi dengan kalian. Kalian akan mencari Aku, tetapi seperti yang sudah Kukatakan kepada para penguasa Yahudi, begitu juga Kukatakan kepada kalian; ke tempat Aku pergi, kalian tak dapat datang.
Joh 13:34  Perintah baru Kuberikan kepadamu: Kasihilah satu sama lain. Sama seperti Aku mengasihi kalian, begitu juga kalian harus saling mengasihi.
Joh 13:35  Kalau kalian saling mengasihi, semua orang akan tahu bahwa kalian pengikut-pengikut-Ku."
Joh 13:36  "Tuhan, Tuhan mau pergi ke mana?" tanya Simon Petrus kepada Yesus. Jawab Yesus, "Ke mana Aku pergi, engkau tak dapat ikut sekarang. Engkau akan mengikuti Aku kemudian."
Joh 13:37  "Tuhan, mengapa saya tidak dapat mengikuti Tuhan sekarang?" tanya Petrus lagi. "Saya rela mati untuk Tuhan!"
Joh 13:38  Yesus menjawab, "Sungguhkah engkau rela mati untuk-Ku? Ketahuilah, sebelum ayam berkokok nanti, engkau sudah tiga kali berkata bahwa engkau tidak mengenal Aku!"
Joh 14:1  "Jangan hatimu gelisah," kata Yesus kepada mereka. "Percayalah kepada Allah, dan percayalah kepada-Ku juga.
Joh 14:2  Di rumah Bapa-Ku ada banyak tempat tinggal. Aku pergi ke sana untuk menyediakan tempat bagi kalian. Aku tidak akan berkata begitu kepadamu, sekiranya itu tidak demikian.
Joh 14:3  Sesudah Aku pergi menyediakan tempat untuk kalian, Aku akan kembali dan menjemput kalian, supaya di mana Aku berada, di situ juga kalian berada.
Joh 14:4  Ke tempat Aku pergi kalian tahu jalannya."
Joh 14:5  Lalu Tomas berkata kepada Yesus, "Tuhan, kami tidak tahu ke mana Tuhan pergi, bagaimana kami tahu jalannya?"
Joh 14:6  Yesus menjawab, "Akulah jalan untuk mengenal Allah dan mendapat hidup. Tidak seorang pun dapat datang kepada Bapa, kalau tidak melalui Aku.
Joh 14:7  Sekiranya kalian mengenal Aku, pasti kalian akan mengenal Bapa-Ku juga. Sekarang kalian sudah mengenal Dia, dan sudah melihat Dia."
Joh 14:8  Maka Filipus berkata kepada Yesus, "Tuhan, tunjukkan Bapa kepada kami, supaya kami puas."
Joh 14:9  Tetapi Yesus menjawab, "Sudah begitu lama Aku bersama kalian, dan belum juga engkau mengenal Aku, Filipus? Orang yang sudah melihat Aku, sudah melihat Bapa. Bagaimana engkau dapat mengatakan, 'Tunjukkanlah Bapa kepada kami'?
Joh 14:10  Filipus! Tidakkah engkau percaya, bahwa Aku bersatu dengan Bapa, dan Bapa bersatu dengan Aku? Apa yang Kukatakan kepadamu, tidak Kukatakan dari diri-Ku sendiri. Bapa yang tetap bersatu dengan Aku, Dialah yang mengerjakan semuanya itu.
Joh 14:11  Percayalah kepada-Ku, bahwa Aku bersatu dengan Bapa dan Bapa bersatu dengan Aku. Atau setidak-tidaknya, percayalah karena apa yang sudah Kulakukan.
Joh 14:12  Sungguh benar kata-Ku ini: Orang yang percaya kepada-Ku, akan melakukan apa yang sudah Kulakukan, --malah ia akan melakukan yang lebih besar lagi--sebab Aku pergi kepada Bapa.
Joh 14:13  Dan apa saja yang kalian minta atas nama-Ku, itu akan Kulakukan untuk kalian, supaya Bapa diagungkan melalui Anak.
Joh 14:14  Apa saja yang kalian minta atas nama-Ku, akan Kulakukan."
Joh 14:15  "Kalau kalian mengasihi Aku, kalian akan menjalankan perintah-perintah-Ku
Joh 14:16  Aku akan minta kepada Bapa, dan Ia akan memberikan kepadamu Penolong lain, yang akan tinggal bersama kalian untuk selama-lamanya.
Joh 14:17  Dia itu Roh Allah yang akan menyatakan kebenaran tentang Allah. Dunia tak dapat menerima Dia, karena tidak melihat atau mengenal-Nya. Tetapi kalian mengenal Dia, karena Ia tinggal bersama kalian dan akan bersatu dengan kalian.
Joh 14:18  Kalian tak akan Kutinggalkan sendirian sebagai yatim piatu. Aku akan kembali kepadamu.
Joh 14:19  Tinggal sebentar saja dunia tak akan melihat Aku lagi. Tetapi kalian akan melihat Aku. Dan karena Aku hidup, kalian pun akan hidup.
Joh 14:20  Bila tiba hari itu, kalian akan tahu bahwa Aku bersatu dengan Bapa, kalian bersatu dengan Aku, dan Aku bersatu dengan kalian.
Joh 14:21  Orang yang menerima perintah-perintah-Ku dan melakukannya, dialah yang mengasihi Aku. Bapa-Ku akan mengasihi orang yang mengasihi Aku. Aku pun akan mengasihi orang itu dan menyatakan diri-Ku kepadanya."
Joh 14:22  Yudas (bukan Yudas Iskariot) bertanya kepada Yesus, "Tuhan, mengapa Tuhan mau menyatakan diri kepada kami dan tidak kepada dunia?"
Joh 14:23  Yesus menjawab, "Orang yang mengasihi Aku, akan menuruti ajaran-Ku. Bapa-Ku akan mengasihi dia. Bapa dan Aku akan datang kepadanya dan tinggal bersama dia.
Joh 14:24  Orang yang tidak mengasihi Aku, tidak menuruti ajaran-Ku. Ajaran yang kalian dengar itu, bukan dari Aku, melainkan dari Bapa yang mengutus Aku.
Joh 14:25  Semuanya itu Kukatakan kepadamu selama Aku masih bersama kalian.
Joh 14:26  Tetapi Roh Allah, Penolong yang akan diutus Bapa atas nama-Ku, Dialah yang akan mengajar kalian segalanya dan mengingatkan kalian akan semua yang sudah Kuberitahukan kepadamu.
Joh 14:27  Sejahtera Kutinggalkan kepada kalian. Sejahtera-Ku sendiri yang Kuberikan kepadamu. Yang Kuberikan itu bukan seperti yang diberikan dunia kepadamu. Jangan gelisah, jangan takut.
Joh 14:28  Kalian sudah mendengar Aku berkata, 'Aku akan pergi, tetapi Aku akan datang kembali kepadamu'. Kalau kalian mengasihi Aku, kalian akan senang Aku pergi kepada Bapa, sebab Bapa lebih besar daripada-Ku.
Joh 14:29  Aku memberitahukan itu kepadamu sekarang, sebelum semuanya terjadi, supaya kalau terjadi nanti, kalian akan percaya.
Joh 14:30  Aku tidak akan berbicara lebih banyak lagi dengan kalian, sebab sudah waktunya penguasa dunia ini datang. Tetapi ia tidak berkuasa atas diri-Ku.
Joh 14:31  Namun semua itu harus terjadi supaya dunia menyadari bahwa Aku mengasihi Bapa dan melakukan segala yang diperintahkan Bapa kepada-Ku. Nah, mari kita pergi dari sini."
Joh 15:1  Kata Yesus lagi, "Aku pohon anggur yang sejati, dan Bapa-Ku adalah tukang kebunnya.
Joh 15:2  Setiap cabang pada-Ku yang tidak berbuah, dipotong-Nya, dan setiap cabang yang berbuah, dikurangi daunnya dan dibersihkan-Nya supaya lebih banyak lagi buahnya.
Joh 15:3  Kalian sudah bersih karena ajaran yang Kuberikan kepadamu.
Joh 15:4  Tetaplah bersatu dengan Aku dan Aku pun akan tetap bersatu dengan kalian. Cabang sendiri tak dapat berbuah, kecuali kalau ia tetap pada pohonnya. Demikian juga kalian hanya dapat berbuah, kalau tetap bersatu dengan Aku.
Joh 15:5  Akulah pohon anggur, dan kalian cabang-cabangnya. Orang yang tetap bersatu dengan Aku dan Aku dengan dia, akan berbuah banyak; sebab tanpa Aku, kalian tak dapat berbuat apa-apa.
Joh 15:6  Orang yang tidak tetap bersatu dengan Aku, akan dibuang seperti cabang, lalu menjadi kering. Cabang-cabang yang seperti itu akan dikumpulkan dan dibuang ke dalam api, lalu dibakar.
Joh 15:7  Apabila kalian tetap bersatu dengan Aku dan ajaran-Ku tinggal dalam hatimu, mintalah kepada Bapa apa saja yang kalian mau; permintaanmu itu akan dipenuhi.
Joh 15:8  Kalau kalian berbuah banyak, Bapa-Ku diagungkan; dan dengan demikian kalian betul-betul menjadi pengikut-Ku.
Joh 15:9  Seperti Bapa mengasihi Aku, demikianlah Aku mengasihi kalian. Hendaklah kalian tetap hidup sebagai orang yang Kukasihi.
Joh 15:10  Kalau kalian menjalankan perintah-perintah-Ku, kalian tetap setia kepada kasih-Ku, sama seperti Aku tetap setia kepada kasih Bapa karena menjalankan perintah-perintah-Nya.
Joh 15:11  Semuanya ini Kuberitahukan kepadamu, supaya kegembiraan-Ku ada dalam hatimu, dan kegembiraanmu menjadi sempurna.
Joh 15:12  Inilah perintah-Ku: Kasihilah satu sama lain, sama seperti Aku mengasihi kalian.
Joh 15:13  Orang yang paling mengasihi sahabat-sahabatnya adalah orang yang memberi hidupnya untuk mereka.
Joh 15:14  Kalian adalah sahabat-sahabat-Ku, kalau kalian melakukan apa yang Kuperintahkan kepadamu.
Joh 15:15  Kalian tidak lagi Kupanggil hamba, sebab hamba tidak tahu apa yang sedang dikerjakan tuannya. Kalian Kupanggil sahabat, sebab semua yang Kudengar dari Bapa, sudah Kuberitahukan kepadamu.
Joh 15:16  Bukan kalian yang memilih Aku. Akulah yang memilih kalian, dan menyuruh kalian pergi untuk berbuah banyak--buah-buah yang tak dapat binasa. Maka Bapa akan memberikan kepadamu apa saja yang kalian minta kepada-Nya atas nama-Ku.
Joh 15:17  Inilah perintah-Ku kepadamu: Kasihilah satu sama lain."
Joh 15:18  "Apabila dunia membenci kalian, ingatlah bahwa Aku sudah lebih dahulu dibenci oleh dunia.
Joh 15:19  Sekiranya kalian milik dunia, kalian akan dikasihi oleh dunia sebagai kepunyaannya. Tetapi Aku sudah memilih kalian dari dunia ini, jadi kalian bukan lagi milik dunia. Itu sebabnya dunia membenci kalian.
Joh 15:20  Ingatlah apa yang sudah Kukatakan kepadamu, 'Hamba tidak lebih besar daripada tuannya.' Kalau mereka sudah menganiaya Aku, mereka akan menganiaya kalian juga. Kalau mereka menuruti ajaran-Ku, mereka akan menuruti ajaranmu juga.
Joh 15:21  Semuanya itu akan mereka lakukan terhadap kalian, karena kalian pengikut-Ku, sebab mereka tidak mengenal Dia yang mengutus Aku.
Joh 15:22  Sekiranya Aku tidak datang dan tidak mengatakan semuanya itu kepada mereka, mereka tidak berdosa. Tetapi sekarang mereka tidak punya alasan lagi untuk dosa mereka.
Joh 15:23  Orang yang membenci Aku, membenci Bapa-Ku.
Joh 15:24  Sekiranya di tengah-tengah mereka Aku tidak melakukan hal-hal yang belum pernah dilakukan orang lain, mereka tidak berdosa. Tetapi sekarang mereka sudah melihat apa yang Kulakukan, dan mereka membenci Aku maupun Bapa-Ku.
Joh 15:25  Namun sudah seharusnya demikian, supaya terjadilah apa yang tertulis dalam Buku Hukum mereka, yaitu: 'Mereka membenci Aku tanpa alasan.'
Joh 15:26  Aku akan mengutus kepadamu Penolong yang berasal dari Bapa. Dialah Roh yang akan menyatakan kebenaran tentang Allah. Apabila Ia datang, Ia akan memberi kesaksian tentang Aku.
Joh 15:27  Dan kalian juga harus memberi kesaksian tentang Aku, sebab kalian sudah bersama Aku sejak semula.
Joh 16:1  Semuanya itu Kuberitahukan kepadamu supaya kalian jangan murtad.
Joh 16:2  Kalian akan dikeluarkan dari rumah-rumah ibadat. Dan akan datang waktunya bahwa orang yang membunuh kalian akan menyangka ia mengabdi kepada Allah.
Joh 16:3  Mereka melakukan itu kepadamu sebab mereka belum mengenal Bapa maupun Aku.
Joh 16:4  Tetapi sekarang Kukatakan itu kepadamu, supaya kalau itu terjadi nanti, kalian ingat bahwa Aku sudah memberitahukannya kepadamu." "Hal ini tidak Kuberitahukan kepadamu dari semula, sebab Aku masih bersama-sama dengan kalian.
Joh 16:5  Tetapi sekarang Aku akan pergi kepada Dia yang mengutus Aku; dan tidak seorang pun dari kalian bertanya ke mana Aku pergi.
Joh 16:6  Sekarang malah hatimu menjadi sedih, karena Aku mengatakan hal itu kepadamu.
Joh 16:7  Tetapi Aku mengatakan yang benar kepadamu: Lebih baik untuk kalian, kalau Aku pergi; sebab kalau Aku tidak pergi, Penolong itu tidak akan datang kepadamu. Tetapi kalau Aku pergi, Aku akan mengutus Dia kepadamu.
Joh 16:8  Kalau Ia datang, Ia akan menyatakan kepada dunia arti sebenarnya dari dosa, dari apa yang benar, dan dari hukuman Allah.
Joh 16:9  Ia akan menyatakan bahwa tidak percaya kepada-Ku adalah dosa;
Joh 16:10  bahwa Aku benar, karena Aku pergi kepada Bapa dan kalian tak akan melihat Aku lagi;
Joh 16:11  dan bahwa Allah sudah mulai menghukum, sebab penguasa dunia ini sudah dihukum.
Joh 16:12  Banyak lagi yang mau Kukatakan kepadamu, namun sekarang ini kalian belum sanggup menerimanya.
Joh 16:13  Tetapi kalau Roh itu datang, yaitu Dia yang menyatakan kebenaran tentang Allah, kalian akan dibimbing-Nya untuk mengenal seluruh kebenaran. Ia tidak akan berbicara dari diri-Nya sendiri tetapi mengatakan apa yang sudah didengar-Nya, dan Ia akan memberitahukan kepadamu apa yang akan terjadi di kemudian hari.
Joh 16:14  Ia akan mengagungkan Aku, sebab apa yang disampaikan-Nya kepadamu, diterima-Nya daripada-Ku.
Joh 16:15  Semua yang ada pada Bapa adalah kepunyaan-Ku. Itu sebabnya Aku berkata bahwa apa yang disampaikan Roh kepadamu, diterima-Nya dari Aku."
Joh 16:16  "Tinggal sesaat saja kalian tak akan melihat Aku lagi, dan juga tinggal sesaat lagi kalian akan melihat Aku."
Joh 16:17  Beberapa pengikut Yesus mulai bertanya satu sama lain, "Apa maksudnya Dia berkata kepada kita: 'Tinggal sesaat saja kalian tidak akan melihat Aku lagi, dan juga tinggal sesaat lagi kalian akan melihat Aku'? Apa pula maksudnya dengan: 'Aku pergi kepada Bapa'?"
Joh 16:18  Mereka bertanya terus, "Apa artinya, 'sesaat'? Kita tidak mengerti Ia bicara tentang apa!"
Joh 16:19  Yesus tahu mereka mau bertanya kepada-Nya. Jadi Ia berkata, "Tadi Kukatakan, 'Tinggal sesaat saja, kalian tak akan melihat Aku, dan juga tinggal sesaat lagi kalian akan melihat Aku.' Itukah yang kalian persoalkan di antaramu?
Joh 16:20  Percayalah, kalian akan menangis dan meratap, tetapi dunia akan bergembira. Kalian akan bersusah hati, tetapi kesusahanmu itu akan berubah menjadi kegembiraan.
Joh 16:21  Kalau seorang wanita hampir melahirkan, ia susah, sebab sudah waktunya ia menderita. Tetapi begitu anaknya lahir, wanita itu lupa akan penderitaannya karena gembira bahwa seorang bayi lahir ke dalam dunia.
Joh 16:22  Begitu juga dengan kalian: Sekarang kalian bersusah hati, tetapi Aku akan bertemu lagi dengan kalian, maka hatimu akan bergembira; dan tidak seorang pun dapat mengambil kegembiraan itu dari hatimu.
Joh 16:23  Pada hari itu, kalian tak akan bertanya apa-apa lagi kepada-Ku. Percayalah: Apa saja yang kalian minta kepada Bapa atas nama-Ku, itu akan diberikan Bapa kepadamu.
Joh 16:24  Sampai saat ini kalian belum minta apa-apa atas nama-Ku. Mintalah, maka kalian akan menerima, supaya kegembiraanmu sempurna."
Joh 16:25  "Semuanya ini Kukatakan kepadamu dengan kiasan. Tetapi akan datang waktunya, Aku tidak memakai kiasan lagi, tetapi berbicara terus terang kepadamu tentang Bapa.
Joh 16:26  Pada waktu itu kalian akan minta kepada Bapa atas nama-Ku; ketahuilah, Aku tidak akan minta dari Bapa untuk kalian,
Joh 16:27  karena Bapa sendiri mengasihi kalian. Ia mengasihi kalian karena kalian mengasihi Aku, dan percaya bahwa Aku datang dari Allah.
Joh 16:28  Memang Aku berasal dari Bapa, dan sudah datang ke dalam dunia. Tetapi sekarang Aku meninggalkan dunia untuk kembali kepada Bapa."
Joh 16:29  Lalu pengikut-pengikut Yesus berkata kepada-Nya, "Sekarang Tuhan bicara terus terang dan tidak memakai kiasan,
Joh 16:30  dan kami tahu bahwa Tuhan tahu segalanya. Tak perlu seorang pun menanyakan apa-apa kepada Tuhan. Karena itu kami percaya Tuhan datang dari Allah."
Joh 16:31  Yesus menjawab mereka, "Jadi percayakah kalian sekarang?
Joh 16:32  Ingat! Saatnya akan datang, malah sudah datang, kalian akan diceraiberaikan. Kalian akan pulang ke rumah masing-masing dan meninggalkan Aku sendirian. Tetapi Aku tidak sendirian sebab Bapa ada bersama-Ku.
Joh 16:33  Semuanya ini Kukatakan supaya kalian mendapat sejahtera karena bersatu dengan Aku. Di dunia kalian akan menderita. Tapi tabahkan hatimu! Aku sudah mengalahkan dunia!"
Joh 17:1  Sesudah berkata demikian, Yesus menengadah ke langit dan berkata, "Bapa, sekarang sudah sampai waktunya. Agungkanlah Anak-Mu, supaya Anak-Mu pun mengagungkan Bapa.
Joh 17:2  Bapa sudah memberi kuasa atas seluruh umat manusia kepada Anak, supaya Ia memberi hidup sejati dan kekal kepada semua orang yang Bapa berikan kepada-Nya.
Joh 17:3  Inilah hidup sejati dan kekal; supaya orang mengenal Bapa, satu-satunya Allah yang benar, dan mengenal Yesus Kristus yang diutus oleh Bapa.
Joh 17:4  Aku sudah mengagungkan Bapa di atas bumi ini dengan menyelesaikan pekerjaan yang Bapa tugaskan kepada-Ku.
Joh 17:5  Bapa! Agungkanlah Aku sekarang pada Bapa, dengan keagungan yang Kumiliki bersama Bapa sebelum dunia ini dijadikan.
Joh 17:6  Aku sudah memperkenalkan Bapa kepada orang-orang dari dunia ini yang sudah Bapa berikan kepada-Ku. Mereka adalah kepunyaan Bapa, dan Bapa sudah memberikan mereka kepada-Ku. Mereka sudah menuruti perkataan Bapa.
Joh 17:7  Sekarang mereka tahu bahwa semua yang Bapa berikan kepada-Ku berasal dari Bapa.
Joh 17:8  Sudah Kusampaikan kepada mereka perkataan yang Bapa berikan kepada-Ku; dan mereka sudah menerimanya. Mereka tahu bahwa Aku benar-benar datang dari Bapa dan mereka percaya bahwa Bapalah yang mengutus Aku.
Joh 17:9  Aku berdoa untuk mereka. Aku tidak berdoa untuk dunia melainkan untuk orang-orang yang sudah Bapa berikan kepada-Ku, sebab mereka adalah kepunyaan Bapa.
Joh 17:10  Semua milik-Ku adalah milik Bapa juga; dan semua milik Bapa adalah milik-Ku juga. Aku diagungkan di antara mereka.
Joh 17:11  Sekarang Aku datang kepada Bapa. Aku tidak tinggal lagi di dunia; tetapi mereka ada di dunia. Bapa yang suci! Jagalah mereka dengan kekuasaan nama Bapa, yaitu nama yang sudah Bapa berikan kepada-Ku--supaya mereka menjadi satu, sama seperti Bapa dan Aku juga satu.
Joh 17:12  Sewaktu Aku masih bersama mereka, Aku sudah menjaga mereka dengan kekuasaan nama Bapa--nama yang Bapa berikan kepada-Ku. Aku sudah menjaga mereka dan tak seorang pun dari mereka hilang, kecuali dia yang memang sudah seharusnya hilang; supaya dengan itu terjadilah apa yang tertulis dalam Alkitab.
Joh 17:13  Sekarang Aku datang kepada Bapa. Semuanya ini Kukatakan sementara Aku masih di dunia; supaya mereka dengan sepenuhnya merasakan kegembiraan-Ku.
Joh 17:14  Aku sudah menyampaikan kepada mereka perkataan Bapa, dan dunia membenci mereka, sebab mereka bukan milik dunia, sama seperti Aku juga bukan milik dunia.
Joh 17:15  Aku tidak minta supaya Bapa mengambil mereka dari dunia ini, tetapi Aku minta supaya Bapa menjaga mereka dari si Jahat.
Joh 17:16  Sama halnya seperti Aku bukan milik dunia, mereka pun bukan milik dunia.
Joh 17:17  Jadikanlah mereka milik khusus Bapa melalui kebenaran; perkataan Bapa itulah kebenaran.
Joh 17:18  Seperti Bapa sudah mengutus Aku ke dunia, begitu juga Aku mengutus mereka ke dunia.
Joh 17:19  Untuk kepentingan mereka, Aku menyerahkan diri sebagai milik khusus Bapa, supaya mereka pun menjadi milik khusus Bapa melalui kebenaran.
Joh 17:20  Bukan untuk mereka ini saja Aku berdoa. Aku juga berdoa untuk orang-orang yang akan percaya kepada-Ku oleh kesaksian mereka ini.
Joh 17:21  Aku mohon, Bapa, supaya mereka semua menjadi satu, seperti Bapa bersatu dengan Aku, dan Aku dengan Bapa. Semoga mereka menjadi satu dengan Kita supaya dunia percaya bahwa Bapa yang mengutus Aku.
Joh 17:22  Aku sudah memberikan mereka keagungan yang Bapa berikan kepada-Ku, supaya mereka menjadi satu, sama seperti Kita juga satu;
Joh 17:23  Aku dengan mereka, dan Bapa dengan Aku; supaya mereka benar-benar satu. Maka dunia akan tahu bahwa Bapalah yang mengutus Aku, dan bahwa Bapa mengasihi mereka seperti Bapa mengasihi Aku.
Joh 17:24  Bapa, Aku ingin supaya mereka, yang Bapa berikan kepada-Ku, ada bersama-Ku di tempat Aku berada, supaya mereka melihat keagungan-Ku; yaitu keagungan yang Bapa berikan kepada-Ku, karena Bapa mengasihi Aku sebelum dunia dijadikan.
Joh 17:25  Bapa yang adil! Dunia tidak mengenal Bapa, tetapi Aku mengenal Bapa; dan orang-orang ini tahu bahwa Bapa mengutus Aku.
Joh 17:26  Aku sudah menyatakan nama Bapa kepada mereka; dan Aku akan terus berbuat begitu, supaya kasih Bapa kepada-Ku tetap di dalam hati mereka dan Aku bersatu dengan mereka."
Joh 18:1  Sesudah Yesus berdoa begitu, Ia dengan pengikut-pengikut-Nya pergi ke seberang Sungai Kidron. Di situ ada sebuah taman, dan Yesus dengan pengikut-pengikut-Nya masuk ke taman itu.
Joh 18:2  Yudas pengkhianat itu, tahu tempat itu; sebab Yesus sudah sering berkumpul di situ dengan pengikut-pengikut-Nya.
Joh 18:3  Maka Yudas pergi ke tempat itu dengan membawa sepasukan tentara Romawi dan beberapa pengawal Rumah Tuhan yang disuruh oleh imam-imam kepala dan orang-orang Farisi. Mereka membawa senjata, lentera dan obor.
Joh 18:4  Yesus tahu semua yang akan terjadi pada diri-Nya. Jadi Ia mendekati orang-orang itu dan bertanya, "Kalian mencari siapa?"
Joh 18:5  "Yesus, orang Nazaret," jawab mereka. "Akulah Dia," kata Yesus. Yudas si pengkhianat berdiri di situ dengan mereka.
Joh 18:6  Waktu Yesus berkata kepada mereka, "Akulah Dia," mereka semua mundur lalu jatuh ke tanah.
Joh 18:7  Sekali lagi Yesus bertanya kepada mereka, "Kalian mencari siapa?" "Yesus orang Nazaret," jawab mereka.
Joh 18:8  "Sudah Kukatakan Akulah Dia," kata Yesus. "Dan kalau memang Aku yang kalian cari, biarkan mereka yang lain ini pergi."
Joh 18:9  (Dengan berkata begitu, terjadilah apa yang sudah dikatakan Yesus sebelumnya: "Bapa, dari orang-orang yang Bapa berikan kepada-Ku, tidak seorang pun yang hilang.")
Joh 18:10  Simon Petrus yang membawa sebilah pedang, mencabutnya lalu memarang hamba imam agung sampai putus telinga kanannya. Nama hamba itu Malkus.
Joh 18:11  Maka Yesus berkata kepada Petrus, "Masukkan kembali pedangmu ke dalam tempatnya! Apakah engkau pikir Aku tak akan minum piala penderitaan yang diberikan Bapa kepada-Ku?"
Joh 18:12  Lalu prajurit-prajurit Romawi dengan komandannya dan pengawal-pengawal Yahudi menangkap dan mengikat Yesus.
Joh 18:13  Mula-mula mereka membawa Yesus menghadap Hanas, bapak mertua Kayafas. Kayafas adalah imam agung pada tahun itu.
Joh 18:14  Dan dialah yang sudah menasihati para penguasa Yahudi bahwa lebih baik satu orang mati untuk seluruh bangsa.
Joh 18:15  Simon Petrus dan seorang pengikut lain mengikuti Yesus. Pengikut yang lain ini dikenal oleh imam agung; jadi ia turut masuk bersama-sama dengan Yesus ke halaman rumah imam agung,
Joh 18:16  sedangkan Petrus menunggu di luar, di pintu. Kemudian pengikut yang lain itu pergi ke luar dan berbicara dengan pelayan wanita yang menjaga pintu, lalu membawa Petrus masuk ke dalam.
Joh 18:17  Pelayan wanita penjaga pintu itu berkata kepada Petrus, "Hai, bukankah engkau juga salah seorang pengikut orang itu?" "Bukan," jawab Petrus.
Joh 18:18  Pada waktu itu udara dingin, jadi pelayan-pelayan dan pengawal-pengawal sudah menyalakan api arang dan mereka menghangatkan badan di situ. Petrus pergi ke sana dan berdiri berdiang bersama mereka.
Joh 18:19  Imam agung menanyai Yesus tentang pengikut-pengikut-Nya dan tentang ajaran-Nya.
Joh 18:20  Yesus menjawab, "Aku selalu berbicara dengan terus terang di muka umum. Aku selalu mengajar di rumah-rumah ibadat dan di Rumah Allah, tempat orang Yahudi biasanya berkumpul. Tidak pernah Aku mengatakan apa-apa dengan sembunyi-sembunyi.
Joh 18:21  Jadi mengapa Tuan menanyai Aku? Tanyalah mereka yang sudah mendengar Aku mengajar. Pasti mereka tahu apa yang Kukatakan."
Joh 18:22  Ketika Yesus berkata begitu, salah seorang pengawal di situ menampar-Nya dan berkata, "Berani sekali Engkau bicara seperti itu kepada imam agung!"
Joh 18:23  Yesus menjawab, "Kalau Aku mengatakan sesuatu yang salah, katakanlah di sini apa kesalahannya! Tetapi kalau yang Kukatakan itu memang benar, mengapa engkau menampar Aku?"
Joh 18:24  Kemudian Hanas menyuruh orang membawa Yesus dengan terbelenggu kepada Imam Agung Kayafas.
Joh 18:25  Simon Petrus masih juga berdiri berdiang di situ. Orang-orang berkata kepadanya, "Bukankah engkau juga pengikut orang itu?" Tetapi Petrus menyangkal, katanya, "Bukan!"
Joh 18:26  Seorang hamba imam agung, yaitu keluarga dari orang yang telinganya dipotong Petrus, berkata, "Bukankah saya melihat engkau di taman itu bersama-sama dengan Dia?"
Joh 18:27  Lalu Petrus menyangkalnya lagi, "Tidak," --dan tepat pada saat itu ayam berkokok.
Joh 18:28  Pagi-pagi sekali mereka membawa Yesus dari rumah Kayafas ke istana gubernur. Orang-orang Yahudi sendiri tidak masuk ke dalam istana, supaya mereka tidak menjadi najis secara agama, karena mereka mau ikut makan makanan Paskah.
Joh 18:29  Karena itu Pilatus pergi ke luar pada mereka dan bertanya, "Apa pengaduanmu terhadap orang ini?"
Joh 18:30  Mereka menjawab, "Seandainya Ia tak bersalah, kami tak akan membawa-Nya kepada Bapak Gubernur."
Joh 18:31  Pilatus berkata kepada mereka, "Ambillah Dia dan hakimilah Dia menurut hukummu sendiri!" Tetapi orang-orang Yahudi itu menjawab, "Kami tidak boleh menghukum mati orang."
Joh 18:32  (Ini terjadi supaya terlaksana apa yang dikatakan Yesus mengenai caranya Ia akan mati.)
Joh 18:33  Pilatus masuk kembali ke istana dan memanggil Yesus, lalu bertanya, "Apakah Engkau raja orang Yahudi?"
Joh 18:34  Yesus menjawab, "Apakah pertanyaan ini dari engkau sendiri atau ada orang lain yang sudah memberitahukan kepadamu tentang Aku?"
Joh 18:35  Pilatus menjawab, "Apakah saya ini orang Yahudi? Yang menyerahkan Engkau kepada saya adalah bangsa-Mu sendiri dan imam-imam kepala. Apa yang sudah Kaulakukan?"
Joh 18:36  Yesus berkata, "Kerajaan-Ku bukan dari dunia ini. Andaikata kerajaan-Ku dari dunia ini, orang-orang-Ku akan berjuang supaya Aku jangan diserahkan kepada para penguasa Yahudi. Tetapi memang kerajaan-Ku bukan dari dunia ini!"
Joh 18:37  Maka Pilatus bertanya kepada-Nya, "Kalau begitu, Engkau raja?" Yesus menjawab, "Engkau katakan bahwa Aku ini raja. Aku lahir dan datang ke dunia untuk satu maksud, yaitu memberi kesaksian tentang kebenaran. Orang yang dari kebenaran itu mendengarkan Aku."
Joh 18:38  Pilatus bertanya kepada-Nya, "Apa artinya kebenaran?" Lalu Pilatus keluar lagi dari istana dan berkata kepada orang-orang Yahudi, "Saya tidak mendapat satu kesalahan pun pada-Nya.
Joh 18:39  Tetapi menurut kebiasaanmu, saya selalu melepaskan seorang tahanan pada Hari Raya Paskah. Maukah kalian supaya saya melepaskan raja orang Yahudi untuk kalian?"
Joh 18:40  Mereka menjawab dengan berteriak-teriak, "Tidak, jangan Dia, tapi Barabas!" (Barabas adalah seorang perampok.)
Joh 19:1  Kemudian Pilatus masuk lalu menyuruh orang mencambuk Yesus.
Joh 19:2  Prajurit-prajurit membuat sebuah mahkota dari ranting-ranting berduri, lalu memasangnya di atas kepala Yesus. Sesudah itu mereka memakaikan Dia jubah berwarna ungu,
Joh 19:3  lalu berkali-kali datang kepada-Nya dan berkata, "Hidup raja orang Yahudi!" Kemudian mereka menampari Dia.
Joh 19:4  Sesudah itu Pilatus keluar sekali lagi dan berkata kepada orang banyak itu, "Lihat! Saya membawa Dia ke luar kepadamu, supaya kalian tahu saya tidak menemukan satu kesalahan pun pada-Nya."
Joh 19:5  Maka Yesus keluar dengan memakai mahkota duri dan jubah ungu. Pilatus berkata kepada mereka, "Lihatlah orang itu."
Joh 19:6  Ketika imam-imam kepala dan pengawal-pengawal itu melihat Yesus, mereka berteriak, "Salibkan Dia! Salibkan Dia!" Pilatus berkata kepada mereka, "Ambillah Dia, dan salibkan olehmu sendiri, saya tidak menemukan satu kesalahan pun pada-Nya."
Joh 19:7  Orang-orang Yahudi itu menjawab, "Menurut hukum kami, Ia harus dihukum mati sebab Ia mengaku diri-Nya Anak Allah."
Joh 19:8  Ketika Pilatus mendengar mereka berkata begitu, ia lebih takut lagi.
Joh 19:9  Maka ia masuk kembali ke dalam istana dan setelah Yesus dibawa masuk, Pilatus bertanya kepada-Nya, "Engkau berasal dari mana?" Tetapi Yesus tidak menjawab.
Joh 19:10  Jadi Pilatus berkata lagi, "Engkau tak mau bicara dengan saya? Ketahuilah, saya mempunyai kuasa membebaskan Engkau, dan kuasa menyalibkan Engkau!"
Joh 19:11  Yesus menjawab, "Kalau Allah tidak memberikan kuasa itu kepadamu, engkau sama sekali tidak punya kuasa atas-Ku. Karena itu orang yang menyerahkan Aku kepadamu, lebih besar dosanya daripadamu."
Joh 19:12  Ketika Pilatus mendengar itu, ia berusaha untuk melepaskan Yesus. Tetapi orang-orang Yahudi berteriak-teriak, "Kalau Tuan membebaskan Dia, Tuan bukan kawan Kaisar! Orang yang mengaku dirinya raja, adalah musuh Kaisar!"
Joh 19:13  Ketika Pilatus mendengar kata-kata itu, ia membawa Yesus ke luar lalu duduk di kursi pengadilan di tempat yang bernama Lantai Batu. (Di dalam bahasa Ibrani namanya Gabata.)
Joh 19:14  Waktu itu hampir pukul dua belas siang, hari sebelum Hari Raya Paskah. Pilatus berkata kepada orang-orang itu, "Ini rajamu!"
Joh 19:15  Mereka berteriak-teriak, "Bunuh Dia! Bunuh Dia! Salibkan Dia!" Pilatus bertanya, "Haruskah saya menyalibkan rajamu?" Imam-imam kepala menjawab, "Hanya Kaisar satu-satunya raja kami!"
Joh 19:16  Maka Pilatus menyerahkan Yesus kepada mereka untuk disalibkan. Mereka mengambil Yesus, lalu membawa Dia pergi.
Joh 19:17  Yesus keluar dengan memikul sendiri salib-Nya ke tempat yang bernama "Tempat Tengkorak". (Di dalam bahasa Ibrani disebut Golgota.)
Joh 19:18  Di sana Ia disalibkan. Bersama-sama dengan Dia ada juga dua orang lain yang disalibkan; seorang di sebelah kiri, seorang di sebelah kanan dan Yesus di tengah-tengah.
Joh 19:19  Pada kayu salib Yesus, Pilatus menyuruh memasang tulisan ini: "Yesus dari Nazaret, Raja Orang Yahudi".
Joh 19:20  Banyak orang Yahudi membaca tulisan itu, sebab tempat Yesus disalibkan itu tidak jauh dari kota. Tulisan itu dalam bahasa Ibrani, Latin, dan Yunani.
Joh 19:21  Imam-imam kepala berkata kepada Pilatus, "Jangan menulis 'Raja orang Yahudi', melainkan tulislah, 'Orang ini berkata, Aku Raja orang Yahudi.'"
Joh 19:22  Tetapi Pilatus menjawab, "Yang sudah saya tulis, tetap tertulis."
Joh 19:23  Setelah prajurit-prajurit itu menyalibkan Yesus, mereka mengambil pakaian-Nya. Pakaian itu dibagi empat: masing-masing mendapat satu bagian. Mereka mengambil juga jubah-Nya. Jubah itu tidak ada jahitannya--ditenun dari atas sampai ke bawah.
Joh 19:24  Prajurit-prajurit itu berkata satu sama lain, "Jangan kita potong-potong jubah ini. Mari kita membuang undi untuk menentukan siapa yang boleh mendapatnya." Hal itu terjadi supaya terlaksana apa yang tertulis dalam Alkitab, yaitu: "Mereka membagi-bagi pakaian-Ku, dan membuang undi untuk jubah-Ku." Dan memang prajurit-prajurit itu berbuat begitu.
Joh 19:25  Di dekat salib Yesus berdiri ibu Yesus, saudara perempuan ibu-Nya, Maria istri Klopas, dan Maria Magdalena.
Joh 19:26  Ketika Yesus melihat ibu-Nya dan pengikut yang dikasihi-Nya berdiri di situ, Ia berkata kepada ibu-Nya, "Ibu, itu anak Ibu."
Joh 19:27  Kemudian Yesus berkata kepada pengikut-Nya itu, "Itu ibumu." Semenjak itu pengikut itu menerima ibu Yesus untuk tinggal di rumahnya.
Joh 19:28  Yesus tahu bahwa sekarang semuanya sudah selesai, dan supaya apa yang tertulis dalam Alkitab dapat terjadi, Ia berkata, "Aku haus."
Joh 19:29  Di situ ada sebuah mangkuk penuh dengan air anggur yang asam. Maka sebuah bunga karang dicelupkan ke dalam air anggur itu, dan dicucukkan pada setangkai hisop, lalu diulurkan ke bibir Yesus.
Joh 19:30  Yesus mengecap air anggur itu lalu berkata, "Sudah selesai!" Lalu Ia menundukkan kepala-Nya dan meninggal.
Joh 19:31  Para penguasa Yahudi tidak mau mayat orang-orang yang disalibkan itu tetap tergantung di kayu salib pada hari Sabat, apalagi kali ini Sabat itu Hari Raya yang khusus. Karena hari waktu Yesus disalibkan jatuh sebelum hari Sabat itu, orang-orang Yahudi minta izin kepada Pilatus untuk mematahkan kaki orang-orang yang sudah disalibkan dan menurunkan mayat-mayat itu dari kayu salib.
Joh 19:32  Maka prajurit-prajurit itu pergi dan mematahkan lebih dahulu kaki dari kedua orang yang disalibkan bersama Yesus.
Joh 19:33  Ketika mereka sampai kepada Yesus, mereka melihat Ia sudah meninggal. Jadi mereka tidak mematahkan kaki-Nya.
Joh 19:34  Tetapi lambung Yesus ditusuk dengan tombak oleh seorang dari prajurit-prajurit itu; dan segera keluarlah darah dan air.
Joh 19:35  Orang yang melihat sendiri kejadian itu, dialah yang memberitakan hal itu, supaya kalian juga percaya. Dan kesaksiannya itu benar, dan ia tahu bahwa itu benar.
Joh 19:36  Hal itu terjadi supaya terlaksana apa yang tertulis dalam Alkitab, yaitu "Tidak satu pun dari tulang-Nya akan dipatahkan".
Joh 19:37  Di dalam Alkitab juga tertulis: "Mereka akan memandang Dia yang sudah mereka tikam".
Joh 19:38  Setelah itu Yusuf dari Arimatea minta izin dari Pilatus untuk mengambil jenazah Yesus. (Yusuf adalah pengikut Yesus tetapi secara sembunyi-sembunyi, sebab ia takut kepada para penguasa Yahudi.) Pilatus memberi izin kepadanya, jadi ia pergi mengambil jenazah Yesus.
Joh 19:39  Nikodemus, yang dahulu pernah datang kepada Yesus pada waktu malam, pergi juga bersama Yusuf. Nikodemus membawa ramuan mur dan gaharu--seluruhnya kira-kira tiga puluh kilogram banyaknya.
Joh 19:40  Kedua orang itu mengambil jenazah Yesus lalu membungkusnya dengan kain kafan bersama-sama dengan ramuan wangi itu menurut adat penguburan orang Yahudi.
Joh 19:41  Di tempat Yesus disalibkan ada sebuah taman. Di dalam taman itu ada sebuah kuburan baru, yang belum pernah dipakai untuk penguburan orang.
Joh 19:42  Karena kuburan itu dekat, dan hari Sabat hampir mulai, mereka menguburkan Yesus di sana.
Joh 20:1  Pada hari Minggu pagi, waktu masih gelap, Maria Magdalena pergi ke kuburan. Ia melihat batu penutupnya sudah digeser dari lubang kubur itu.
Joh 20:2  Maka ia lari mencari Simon Petrus dan pengikut yang dikasihi Yesus, dan berkata kepada mereka, "Tuhan sudah diambil dari kubur, dan saya tidak tahu di mana Dia ditaruh."
Joh 20:3  Lalu Petrus dan pengikut lain itu pergi ke kuburan.
Joh 20:4  Keduanya berlari, tetapi pengikut lain itu lebih cepat dari Petrus, dan ia sampai lebih dahulu di kuburan.
Joh 20:5  Ia menengok ke dalam kuburan dan melihat kain kafan terletak di situ, tetapi ia tidak masuk.
Joh 20:6  Simon Petrus menyusul dari belakang, lalu langsung masuk ke dalam kuburan itu. Ia melihat kain kafan terletak di situ,
Joh 20:7  tetapi kain yang diikat pada kepala Yesus tidak ada di dekatnya melainkan tergulung tersendiri.
Joh 20:8  Kemudian pengikut yang lebih dahulu sampai di kuburan, masuk juga. Ia melihat dan percaya.
Joh 20:9  (Sampai pada waktu itu mereka belum mengerti apa yang tertulis dalam Alkitab bahwa Ia harus bangkit dari mati.)
Joh 20:10  Sesudah itu pengikut-pengikut Yesus itu pulang.
Joh 20:11  Maria Magdalena berdiri di depan kuburan sambil menangis. Sementara menangis, ia menjenguk ke dalam kuburan,
Joh 20:12  lalu melihat dua malaikat berpakaian putih. Mereka itu duduk di bekas tempat jenazah Yesus, yang satu di bagian kepala, dan yang lainnya di bagian kaki.
Joh 20:13  Malaikat-malaikat itu bertanya, "Ibu, mengapa menangis?" Maria menjawab, "Tuhan saya sudah diambil, dan saya tidak tahu Ia ditaruh di mana."
Joh 20:14  Setelah berkata begitu, ia menengok ke belakang dan melihat Yesus berdiri di situ. Tetapi ia tidak tahu bahwa itu Yesus.
Joh 20:15  Yesus bertanya kepadanya, "Ibu, mengapa menangis? Ibu mencari siapa?" Maria menyangka itu tukang kebun, jadi ia berkata, "Pak, kalau Bapak yang memindahkan Dia dari sini, tolong katakan kepada saya di mana Bapak menaruh Dia, supaya saya dapat mengambil-Nya."
Joh 20:16  Yesus berkata kepadanya, "Maria!" Maria menoleh kepada Yesus lalu berkata dalam bahasa Ibrani, "Rabuni!" (Berarti "Guru".)
Joh 20:17  "Jangan pegang Aku," kata Yesus kepadanya, "karena Aku belum naik kepada Bapa. Tetapi pergilah kepada saudara-saudara-Ku, dan beritahukanlah kepada mereka bahwa sekarang Aku naik kepada Bapa-Ku dan Bapamu, Allah-Ku dan Allahmu."
Joh 20:18  Maka Maria pergi memberitahukan kepada pengikut-pengikut Yesus bahwa ia sudah melihat Tuhan dan bahwa Tuhan sudah mengatakan semuanya itu kepadanya.
Joh 20:19  Pada hari Minggu itu juga, ketika sudah malam, pengikut-pengikut Yesus berkumpul di sebuah rumah dengan pintu-pintu yang terkunci, sebab mereka takut kepada para penguasa Yahudi. Tiba-tiba Yesus datang dan berdiri di tengah-tengah mereka dan berkata, "Salam sejahtera bagimu."
Joh 20:20  Sesudah berkata begitu, Ia menunjukkan kepada mereka tangan dan lambung-Nya. Pada waktu melihat Tuhan, mereka gembira sekali.
Joh 20:21  Lalu Yesus berkata kepada mereka sekali lagi, "Salam sejahtera bagimu. Seperti Bapa mengutus Aku, begitu juga Aku mengutus kalian."
Joh 20:22  Lalu Ia meniupkan napas-Nya kepada mereka dan berkata, "Terimalah Roh Allah.
Joh 20:23  Kalau kalian mengampuni dosa seseorang, Allah juga mengampuninya. Kalau kalian tidak mengampuni dosa seseorang, Allah juga tidak mengampuninya."
Joh 20:24  Tomas (yang disebut si "Kembar"), seorang dari dua belas pengikut Yesus, tak ada bersama yang lain ketika Yesus datang.
Joh 20:25  Maka pengikut-pengikut Yesus yang lain berkata kepada Tomas, "Kami sudah melihat Tuhan!" Tetapi Tomas menjawab, "Kalau saya belum melihat bekas paku pada tangan-Nya, belum menaruh jari saya pada bekas-bekas luka paku itu dan belum menaruh tangan saya pada lambung-Nya, sekali-kali saya tidak mau percaya."
Joh 20:26  Seminggu kemudian pengikut-pengikut Yesus ada lagi di tempat itu, dan Tomas hadir juga. Semua pintu terkunci. Tetapi Yesus datang dan berdiri di tengah-tengah mereka, lalu berkata, "Salam sejahtera bagimu."
Joh 20:27  Kemudian Yesus berkata kepada Tomas, "Lihatlah tangan-Ku, dan taruhlah jarimu di sini. Ulurkan tanganmu dan taruhlah di lambung-Ku. Jangan ragu-ragu lagi, tetapi percayalah!"
Joh 20:28  Tomas berkata kepada Yesus, "Tuhanku dan Allahku!"
Joh 20:29  Maka Yesus berkata kepadanya, "Engkau percaya karena sudah melihat Aku, bukan? Berbahagialah orang yang percaya meskipun tidak melihat Aku!"
Joh 20:30  Masih banyak lagi keajaiban-keajaiban lain yang dibuat Yesus di depan pengikut-pengikut-Nya, tetapi tidak ditulis di dalam buku ini.
Joh 20:31  Tetapi semuanya ini ditulis, supaya kalian percaya bahwa Yesus adalah Raja Penyelamat, Anak Allah, dan karena percaya kepada-Nya, kalian memperoleh hidup.
Joh 21:1  Setelah itu Yesus memperlihatkan diri sekali lagi di Danau Tiberias kepada pengikut-pengikut-Nya. Beginilah terjadinya:
Joh 21:2  Suatu hari Simon Petrus, Tomas yang disebut si Kembar, Natanael dari Kana di Galilea, anak-anak Zebedeus, dan dua pengikut Yesus yang lainnya, sedang berkumpul.
Joh 21:3  Kata Simon Petrus kepada yang lain, "Saya mau pergi menangkap ikan." "Kami ikut," kata mereka kepadanya. Maka pergilah mereka naik perahu. Tetapi sepanjang malam itu mereka tidak menangkap apa-apa.
Joh 21:4  Ketika matahari mulai terbit, Yesus berdiri di pantai, tetapi mereka tidak tahu bahwa itu Yesus.
Joh 21:5  Yesus berkata kepada mereka, "Anak-anak, apakah kalian punya ikan?" "Tidak," jawab mereka.
Joh 21:6  Yesus berkata kepada mereka, "Lemparkan jalamu ke sebelah kanan perahu, nanti kalian mendapat ikan." Lalu jala itu mereka lemparkan, tetapi tidak sanggup menariknya kembali sebab begitu banyak ikan di dalamnya.
Joh 21:7  Pengikut yang dikasihi Yesus berkata kepada Petrus, "Itu Tuhan!" Ketika Simon Petrus mendengar bahwa itu Tuhan, ia memakai bajunya (sebab ia tidak berbaju) lalu terjun ke dalam air.
Joh 21:8  Pengikut-pengikut yang lain menyusul ke darat dengan perahu, sambil menarik jala yang penuh dengan ikan. Mereka tidak berapa jauh dari darat, kira-kira seratus meter saja.
Joh 21:9  Ketika mereka turun dari perahu, mereka melihat ada bara api di sana dengan ikan di atasnya dan roti.
Joh 21:10  Yesus berkata kepada mereka, "Coba bawa ke mari beberapa ikan yang baru kalian tangkap."
Joh 21:11  Simon Petrus naik ke perahu, lalu menyeret jalanya ke darat. Jala itu penuh dengan ikan yang besar-besar; semuanya ada seratus lima puluh tiga ekor. Dan meskipun sebanyak itu, jalanya tidak sobek.
Joh 21:12  Yesus berkata kepada mereka, "Mari makan." Tidak seorang pun dari pengikut-pengikut-Nya berani bertanya, "Bapak siapa?" Sebab mereka tahu bahwa Ia Tuhan.
Joh 21:13  Kemudian Yesus mendekati mereka, mengambil roti itu, dan memberikannya kepada mereka. Ia berbuat begitu juga dengan ikan itu.
Joh 21:14  Inilah ketiga kalinya Yesus memperlihatkan diri kepada pengikut-pengikut-setelah Ia dibangkitkan dari mati.
Joh 21:15  Sesudah mereka makan, Yesus berkata kepada Simon Petrus, "Simon, anak Yona, apakah engkau lebih mengasihi Aku daripada mereka ini mengasihi Aku?" "Benar, Tuhan," jawab Petrus, "Tuhan tahu saya mencintai Tuhan." Yesus berkata kepadanya, "Peliharalah anak-anak domba-Ku."
Joh 21:16  Untuk kedua kalinya Yesus bertanya kepadanya, "Simon anak Yona, apakah engkau mengasihi Aku?" "Benar, Tuhan," jawab Petrus, "Tuhan tahu saya mencintai Tuhan." Yesus berkata kepadanya, "Peliharalah domba-domba-Ku."
Joh 21:17  Untuk ketiga kalinya Yesus bertanya kepadanya, "Simon anak Yona, apakah engkau mencintai Aku?" Petrus menjadi sedih sebab Yesus bertanya kepadanya sampai tiga kali. Maka Petrus menjawab lagi, "Tuhan, Tuhan tahu segala-galanya. Tuhan tahu saya mencintai Tuhan!" Lalu Yesus berkata kepadanya, "Peliharalah domba-domba-Ku.
Joh 21:18  Sungguh benar kata-Ku ini: Ketika engkau masih muda, engkau sendiri mengikat pinggangmu, dan pergi ke mana saja engkau mau. Tetapi kalau engkau sudah tua nanti, engkau mengulurkan tanganmu, dan orang lain yang mengikat engkau dan membawa engkau ke mana engkau tidak mau pergi."
Joh 21:19  (Dengan kata-kata itu Yesus menunjukkan bagaimana Petrus akan mati nanti untuk mengagungkan Allah.) Sesudah itu Yesus berkata kepada Petrus, "Ikutlah Aku!"
Joh 21:20  Waktu Petrus menoleh, ia melihat di belakangnya pengikut yang dikasihi Yesus. (Dialah yang duduk dekat Yesus pada waktu makan dan yang bertanya kepada-Nya, "Tuhan, siapa yang akan mengkhianati Tuhan?")
Joh 21:21  Melihat dia, Petrus bertanya kepada Yesus, "Tuhan, bagaimana dengan dia ini?"
Joh 21:22  Yesus menjawab, "Andaikata Aku mau ia tinggal hidup sampai Aku datang, itu bukan urusanmu. Tetapi engkau, ikutlah Aku!"
Joh 21:23  Maka tersebarlah kabar di kalangan pengikut Yesus bahwa pengikut itu tidak akan mati. Padahal Yesus tidak mengatakan bahwa pengikut itu tidak akan mati, melainkan: "Andaikata Aku mau ia tinggal hidup sampai Aku datang, itu bukan urusanmu."
Joh 21:24  Pengikut itulah yang memberikan kesaksian tentang kejadian-kejadian ini. Dialah juga yang sudah menulisnya. Dan kita tahu bahwa apa yang dikatakannya itu benar.
Joh 21:25  Masih banyak hal lain yang dilakukan oleh Yesus. Andaikata semuanya itu ditulis satu per satu, saya rasa tak ada cukup tempat di seluruh bumi untuk memuat semua buku yang akan ditulis itu.


\end{document}