\begin{document}

\title{Romans}

Rom 1:1  Saudara-saudara sekalian di Roma yang dikasihi Allah dan yang sudah dipanggil oleh Allah untuk menjadi umat-Nya. Allah sudah memilih dan mengangkat saya khusus untuk memberitakan Kabar Baik dari Allah.
Rom 1:2  Kabar Baik itu sudah dijanjikan oleh Allah pada zaman dahulu melalui nabi-nabi-Nya dan sudah tertulis dalam Alkitab.
Rom 1:3  Kabar Baik itu mengenai Anak Allah, Tuhan kita Yesus Kristus. Secara manusiawi, Ia adalah keturunan Daud,
Rom 1:4  tetapi secara ilahi Ia ternyata adalah Anak Allah. Itu terbukti dengan kuasa yang luar biasa melalui kebangkitan-Nya dari kematian.
Rom 1:5  Melalui Dia juga Allah memberikan kepada saya karunia menjadi rasul, supaya saya--untuk menghormati Kristus--membimbing orang-orang dari segala bangsa supaya percaya dan taat.
Rom 1:6  Yang dimaksudkan dengan segala bangsa adalah kalian juga yang berada di Roma; kalian pun sudah dipanggil untuk menjadi umat Yesus Kristus.
Rom 1:7  Itu sebabnya saya menulis kepadamu. Semoga Allah Bapa kita dan Tuhan Yesus Kristus memberi berkat dan sejahtera kepadamu.
Rom 1:8  Pertama-tama melalui Yesus Kristus, saya berterima kasih kepada Allah karena Saudara-saudara sekalian; sebab seluruh dunia sudah mendengar bahwa kalian percaya sekali kepada Kristus.
Rom 1:9  Saya selalu mengingat kalian kalau saya berdoa. Allah saksinya bahwa apa yang saya katakan itu benar. Dialah Allah yang saya layani sepenuh hati dengan memberitakan Kabar Baik tentang Anak-Nya.
Rom 1:10  Saya mohon dengan sangat kepada Allah, semoga Ia mau mengizinkan saya sekarang mengunjungi kalian.
Rom 1:11  Sebab saya ingin sekali bertemu dengan kalian supaya saya dapat membagi denganmu karunia dari Roh Allah untuk menguatkan kalian.
Rom 1:12  Maksud saya ialah karena kita sama-sama sudah percaya kepada Yesus Kristus, maka kita dapat saling menguatkan.
Rom 1:13  Saudara-saudara! Saya ingin supaya kalian tahu bahwa sudah banyak kali saya bermaksud mengunjungimu tetapi selalu ada saja halangannya. Saya ingin supaya di antaramu pun pekerjaan saya ada hasilnya sebagaimana pekerjaan saya sudah berhasil di antara orang-orang yang bukan Yahudi di tempat-tempat yang lain.
Rom 1:14  Sebab saya mempunyai kewajiban terhadap segala bangsa: yang sudah beradab dan yang biadab, yang berpendidikan maupun yang belum berpendidikan.
Rom 1:15  Itulah sebabnya saya ingin sekali memberitakan Kabar Baik itu kepada kalian yang tinggal di Roma juga.
Rom 1:16  Saya percaya sekali akan Kabar Baik itu, karena kabar itu adalah kekuatan Allah untuk menyelamatkan semua orang yang percaya; pertama-tama orang Yahudi, dan bangsa lain juga.
Rom 1:17  Sebab dengan Kabar Baik itu Allah menunjukkan bagaimana caranya hubungan manusia dengan Allah menjadi baik kembali; caranya ialah dengan percaya kepada Allah, dari mula sampai akhir. Itu sama seperti yang tertulis dalam Alkitab, "Orang yang percaya kepada Allah sehingga hubungannya dengan Allah menjadi baik kembali, orang itu akan hidup!"
Rom 1:18  Dari surga Allah menunjukkan murka-Nya terhadap semua dosa dan kejahatan manusia, sebab kejahatan menghalangi manusia untuk mengenal ajaran benar tentang Allah.
Rom 1:19  Apa yang dapat diketahui manusia tentang Allah sudah jelas di dalam hati nurani manusia, sebab Allah sendiri sudah menyatakan itu kepada manusia.
Rom 1:20  Semenjak Allah menciptakan dunia, sifat-sifat Allah yang tidak kelihatan, yaitu keadaan-Nya sebagai Allah dan kuasa-Nya yang abadi, sudah dapat difahami oleh manusia melalui semua yang telah diciptakan. Jadi manusia sama sekali tidak punya alasan untuk membenarkan diri.
Rom 1:21  Manusia mengenal Allah, tetapi manusia tidak menghormati Dia sebagai Allah dan tidak juga berterima kasih kepada-Nya. Sebaliknya manusia memikirkan yang bukan-bukan; hati mereka sudah menjadi gelap.
Rom 1:22  Mereka merasa diri bijaksana, padahal mereka bodoh.
Rom 1:23  Bukannya Allah yang abadi yang mereka sembah, melainkan patung-patung yang menyerupai makhluk yang bisa mati; yaitu manusia, burung, binatang yang berkaki empat, dan binatang yang melata.
Rom 1:24  Oleh sebab itu Allah membiarkan mereka dikuasai oleh keinginan hati mereka untuk berbuat yang bejat, sehingga mereka melakukan hal-hal yang kotor terhadap sama sendiri.
Rom 1:25  Allah yang benar, mereka ganti dengan sesuatu yang palsu. Bukan Pencipta melainkan yang diciptakan itulah justru yang disembah dan dilayani oleh mereka. Padahal yang menciptakan itulah yang seharusnya dipuji selama-lamanya! Amin.
Rom 1:26  Karena manusia berbuat yang demikian, maka Allah membiarkan mereka menuruti nafsu mereka yang hina. Wanita-wanita mereka tidak lagi tertarik kepada laki-laki seperti yang lazimnya pada manusia, melainkan tertarik kepada sesama wanita.
Rom 1:27  Lelaki pun begitu juga; mereka tidak lagi secara wajar mengadakan hubungan dengan wanita, melainkan berahi terhadap sesama lelaki. Laki-laki melakukan perbuatan yang memalukan terhadap sesama laki-laki, sehingga mereka menerima pembalasan yang setimpal dengan perbuatan mereka yang jahat itu.
Rom 1:28  Oleh sebab manusia tidak merasa perlu mengenal Allah, maka Allah membiarkan pikiran mereka menjadi rusak, sehingga mereka melakukan hal-hal yang mereka tidak boleh lakukan.
Rom 1:29  Hati mereka penuh dengan semua yang jahat, yang tidak benar; penuh dengan keserakahan, kebusukan dan perasaan dengki; penuh dengan keinginan untuk membunuh, berkelahi, menipu dan mendendam. Mereka suka membicarakan orang lain,
Rom 1:30  suka memburuk-burukkan nama orang lain; mereka sombong dan kurang ajar, yang benci kepada Allah dan suka membual. Mereka pandai mencari cara-cara baru untuk melakukan kejahatan. Mereka melawan orang tua;
Rom 1:31  mereka tidak mau mengerti orang lain; mereka tidak setia dan tidak berperikemanusiaan.
Rom 1:32  Mereka tahu, bahwa menurut hukum Allah, orang yang melakukan semuanya itu patut dihukum mati. Walaupun begitu mereka melakukan juga hal-hal itu; dan malah menyetujui pula orang lain melakukannya.
Rom 2:1  Oleh karena itu, Saudara-saudara, siapakah Saudara sehingga Saudara mau menyalahkan orang lain? Saudara tidak punya apa-apa untuk membela diri! Sebab kalau Saudara menyalahkan orang lain, padahal Saudara sendiri melakukan perbuatan yang sama seperti mereka, maka Saudara menjatuhkan hukuman atas diri sendiri juga.
Rom 2:2  Kita tahu bahwa kalau Allah menjatuhkan hukuman ke atas orang-orang yang melakukan perbuatan-perbuatan yang seperti itu, keputusan Allah itu benar.
Rom 2:3  Tetapi kalian, Saudara-saudara, melakukan sendiri juga hal-hal yang kalian tuduhkan kepada orang-orang lain! Dan kalian pikir kalian dapat lolos dari hukuman Allah?
Rom 2:4  Atau kalian pandang enteng kemurahan Allah dan kelapangan hati serta kesabaran-Nya yang begitu besar? Pasti kalian tahu bahwa Allah menunjukkan kebaikan hati-Nya karena Ia mau supaya kalian bertobat dari dosa-dosamu.
Rom 2:5  Tetapi kalian keras kepala dan tidak mau berubah. Oleh sebab itu kalian sendirilah yang membuat hukumanmu menjadi bertambah berat pada Hari Kiamat, bila Allah menyatakan murka-Nya dan menjatuhkan hukuman yang adil.
Rom 2:6  Sebab Allah akan membalas setiap orang setimpal dengan perbuatannya.
Rom 2:7  Allah memberikan hidup sejati dan kekal kepada mereka yang tekun berbuat baik untuk mendapatkan yang mulia, yang terhormat dan yang abadi.
Rom 2:8  Tetapi orang-orang yang mementingkan diri sendiri dan tidak mau taat kepada Allah, melainkan mengikuti yang jahat, orang-orang itu akan sangat dimurkai Allah.
Rom 2:9  Setiap orang yang suka berbuat jahat akan sengsara dan menderita; pertama-tama orang Yahudi, dan juga bangsa-bangsa lain.
Rom 2:10  Tetapi semua orang yang suka berbuat baik akan diberi kemuliaan dan kehormatan serta kesejahteraan oleh Allah; pertama-tama kepada orang Yahudi, begitu juga kepada orang-orang bangsa lain.
Rom 2:11  Sebab Allah memperlakukan semua orang sama.
Rom 2:12  Orang-orang bangsa lain berdosa tanpa mengetahui hukum agama Yahudi. Jadi mereka dihukum di luar hukum itu. Tetapi orang-orang Yahudi berdosa sesudah mengetahui hukum itu; sebab itu mereka akan dituntut juga berdasarkan hukum itu.
Rom 2:13  Sebab orang berbaik kembali dengan Allah, bukan karena orang itu sudah mengetahui hukum agama Yahudi, melainkan karena ia melakukan apa yang tercantum dalam hukum itu.
Rom 2:14  Orang-orang bangsa lain tidak mengenal hukum agama Yahudi. Tetapi kalau mereka atas kemauan sendiri melakukan apa yang diperintahkan oleh hukum itu, hati mereka sendirilah yang menjadi hukum untuk mereka, meskipun mereka tidak mengenal hukum agama Yahudi.
Rom 2:15  Kelakuan mereka menunjukkan bahwa apa yang diperintahkan oleh hukum itu tertulis di hati mereka. Hati nurani mereka pun membuktikan hal itu, sebab mereka sendiri ada kalanya disalahkan dan ada kalanya dibenarkan oleh pikiran mereka.
Rom 2:16  Demikianlah yang akan terjadi nanti pada hari yang sudah ditentukan itu. Pada hari itu--menurut Kabar Baik yang saya beritakan--Allah melalui Yesus Kristus, akan menghakimi segala rahasia hati dan pikiran semua orang.
Rom 2:17  Tetapi sekarang bagaimana dengan Saudara-saudara sendiri? Saudara mengaku diri orang Yahudi. Saudara bergantung kepada hukum agama Yahudi dan Saudara bangga atas hubungan Saudara dengan Allah.
Rom 2:18  Saudara mendapat petunjuk-petunjuk dari hukum itu, sehingga Saudara mengetahui kehendak Allah dan tahu menentukan mana yang baik.
Rom 2:19  Saudara yakin bahwa Saudara adalah pemimpin orang buta dan terang bagi mereka yang berada di dalam kegelapan;
Rom 2:20  Saudara pendidik orang-orang yang berakhlak rendah, dan guru orang-orang yang bodoh. Saudara yakin bahwa di dalam hukum agama Yahudi, Saudara mempunyai segala pengetahuan dan ajaran yang benar.
Rom 2:21  Saudara mengajar orang lain; nah, mengapa Saudara tidak mengajar diri sendiri? Saudara mengajar orang supaya jangan mencuri, padahal Saudara sendiri mencuri!
Rom 2:22  Saudara mengajar orang supaya jangan berzinah, padahal Saudara sendiri berzinah! Saudara benci kepada berhala, padahal Saudara sendiri mengambil barang-barang dari rumah-rumah berhala.
Rom 2:23  Saudara membangga-banggakan bahwa Saudara mempunyai hukum Musa, padahal Saudara menghina Allah dengan tidak menuruti hukum-Nya.
Rom 2:24  Di dalam Alkitab tertulis begini, "Karena kamu, orang-orang Yahudi, maka nama Allah dicemarkan di antara bangsa-bangsa lain."
Rom 2:25  Kalau Saudara taat kepada hukum agama Yahudi, maka sunat Saudara ada gunanya juga. Tetapi kalau Saudara tidak bisa mematuhi hal-hal yang diperintahkan dalam hukum itu, maka sunat Saudara tidak berlaku sama sekali.
Rom 2:26  Kalau seorang bukan Yahudi yang tidak disunat, menuruti hukum agama Yahudi, bukankah Allah menganggapnya sebagai orang yang sudah disunat?
Rom 2:27  Dan orang-orang yang tidak disunat itu akan menyalahkan Saudara orang Yahudi, sebab Saudara mempunyai hukum agama Yahudi dan Saudara disunat, tetapi Saudara melanggar hukum itu. Mereka tidak disunat, tetapi justru merekalah yang mentaati hukum agama Yahudi.
Rom 2:28  Karena orang Yahudi yang sejati bukanlah orang yang hanya namanya saja orang Yahudi; dan orang yang sungguh-sungguh disunat bukanlah orang yang disunat secara lahir saja.
Rom 2:29  Sebaliknya, seorang Yahudi yang sejati adalah orang yang hatinya berjiwa Yahudi; dan sunat yang sejati adalah sunat di hati yang dikerjakan oleh Roh Allah, bukan yang dicatat di dalam buku. Orang semacam itu menerima pujian dari Allah, bukan dari manusia.
Rom 3:1  Kalau begitu, apakah untungnya menjadi orang Yahudi? Dan apakah faedahnya menuruti peraturan sunat?
Rom 3:2  Tentu saja banyak faedahnya! Pertama-tama, karena kepada orang Yahudilah Allah mempercayakan perkataan-perkataan Allah.
Rom 3:3  Tetapi bagaimanakah kalau sebagian orang Yahudi tidak setia? Apakah karena itu Allah menjadi tidak setia?
Rom 3:4  Tentu tidak! Sebab jelaslah Allah selalu benar, walaupun setiap orang berbohong. Dalam Alkitab tertulis, "Hendaknya engkau terbukti benar dalam apa yang engkau ucapkan, dan engkau menang pada waktu engkau dihakimi."
Rom 3:5  Tetapi kalau keadilan Allah menjadi semakin nyata, oleh karena kita berbuat yang tidak benar, apakah yang hendak kita katakan? Bahwa Allah tidak adil kalau Ia menghukum kita? (Memang pertanyaan ini wajar secara manusia.)
Rom 3:6  Sekali-kali tidak! Sebab kalau Allah tidak adil, bagaimanakah Ia dapat menghakimi dunia ini?
Rom 3:7  Tetapi kalau karena perbuatan yang tidak benar, apa yang benar tentang Allah semakin menonjol sehingga Ia dipuji, mengapa orang yang berbuat jahat itu masih disalahkan sebagai orang berdosa?
Rom 3:8  Dan mengapa kita tidak boleh mengatakan, "Baiklah kita berbuat jahat supaya timbul kebaikan?" Memang ada orang-orang yang menghina saya dengan mengatakan bahwa saya sudah berkata begitu. Orang-orang semacam itu sewajarnya dihukum oleh Allah.
Rom 3:9  Nah, apakah kedudukan kita sebagai orang Yahudi lebih baik daripada kedudukan bangsa lain? Sekali-kali tidak! Sudah saya kemukakan bahwa baik orang Yahudi maupun bangsa lain, semuanya sudah dikuasai dosa.
Rom 3:10  Seperti yang tertulis dalam Alkitab, "Tidak seorang pun yang benar,
Rom 3:11  tidak seorang pun yang mengerti dan tidak seorang pun yang menyembah Allah.
Rom 3:12  Semua orang sudah menjauhkan diri dari Allah; semuanya telah sesat. Tidak seorang pun berbuat yang benar; seorang pun tidak!
Rom 3:13  Tenggorokan mereka bagaikan kuburan yang terbuka. Tipu daya mengalir dari lidah mereka, dan bibir mereka menyemburkan fitnah, seperti bisa ular.
Rom 3:14  Mulut mereka penuh dengan kutuk dan kecaman.
Rom 3:15  Langkah mereka cepat kalau hendak menyiksa dan membunuh orang.
Rom 3:16  Kehancuran dan kesusahan, ditabur mereka di mana-mana.
Rom 3:17  Mereka buta terhadap jalan kesejahteraan,
Rom 3:18  dan tidak menghormati Allah."
Rom 3:19  Sekarang kita tahu bahwa semua yang tertulis dalam hukum agama Yahudi, adalah untuk orang-orang yang berada di bawah pemerintahan hukum itu. Dengan demikian tidak seorang pun dapat memberikan alasan apa-apa lagi dan seluruh dunia dapat dituntut oleh Allah.
Rom 3:20  Sebab tidak seorang pun dimungkinkan berbaik dengan Allah oleh karena orang itu melakukan hal-hal yang terdapat dalam hukum agama. Sebaliknya hukum itu cuma menunjukkan kepada manusia bahwa manusia berdosa.
Rom 3:21  Tetapi sekarang Allah sudah menunjukkan jalan bagaimana manusia berbaik dengan Dia; dan caranya itu tidak ada sangkut pautnya dengan hukum agama Yahudi. Buku-buku Musa dan buku-buku nabi-nabi justru menyatakan hal itu,
Rom 3:22  bahwa Allah memungkinkan manusia berbaik dengan Dia, hanya kalau manusia percaya kepada Yesus Kristus. Allah berbuat ini untuk semua orang yang percaya kepada Kristus; sebab tidak ada perbedaannya:
Rom 3:23  Semua orang sudah berdosa dan jauh dari Allah yang hendak menyelamatkan mereka.
Rom 3:24  Hanya karena rahmat Allah saja yang diberikan dengan cuma-cuma, hubungan manusia dengan Allah menjadi baik kembali; caranya ialah: manusia dibebaskan oleh Kristus Yesus.
Rom 3:25  Allah mengurbankan Kristus Yesus supaya dengan kematian-Nya itu manusia dinyatakan bebas dari kesalahan kalau mereka percaya kepada-Nya. Allah berbuat begitu untuk menunjukkan keadilan-Nya. Sebab pada masa yang lampau Allah sudah berlaku sabar terhadap dosa-dosa manusia, sehingga Ia tidak menghukum mereka.
Rom 3:26  Tetapi sekarang Ia bertindak terhadap dosa untuk membuktikan keadilan-Nya. Dengan cara itu Ia menunjukkan bahwa diri-Nya benar; dan setiap orang yang percaya kepada Yesus, dinyatakan-Nya sebagai orang yang sudah berbaik kembali dengan Allah.
Rom 3:27  Oleh karena itu tidak ada lagi alasan bagi kita untuk berbangga-bangga. Mengapa demikian? Apakah karena kita melakukan yang tercantum dalam hukum agama Yahudi? Bukan. Tetapi karena kita percaya.
Rom 3:28  Sebab kesimpulannya adalah begini: Orang dinyatakan berbaik kembali dengan Allah, bukan karena ia melakukan apa yang tercantum dalam hukum agama Yahudi, melainkan karena ia percaya kepada Yesus Kristus.
Rom 3:29  Ataukah Allah itu Allah orang Yahudi saja? Bukankah Ia Allah bangsa lain juga? Ya, memang Ia Allah bangsa lain juga!
Rom 3:30  Sebab Allah hanya satu. Dialah yang memungkinkan orang-orang Yahudi berbaik kembali dengan Allah karena mereka percaya. Dan Dialah pula yang memungkinkan orang-orang bangsa lain berbaik kembali dengan Allah; itu juga karena mereka percaya.
Rom 3:31  Apakah ini berarti bahwa karena kita percaya kepada Kristus, kita membuang hukum agama Yahudi? Sama sekali tidak! Malah justru dengan kepercayaan kita itu, kita menghargai hukum itu.
Rom 4:1  Kalau begitu, apakah yang dapat kita katakan tentang Abraham, nenek moyang bangsa kita? Bagaimanakah pengalamannya?
Rom 4:2  Kalau hal-hal yang dilakukannya menyebabkan Allah menerima dia sebagai orang yang menyenangkan hati Allah, maka ada juga alasan baginya untuk berbangga-bangga. Tetapi ia tidak dapat berbangga di hadapan Allah.
Rom 4:3  Dalam Alkitab tertulis, "Abraham percaya kepada Allah, dan karena kepercayaannya ini ia diterima oleh Allah sebagai orang yang menyenangkan hati Allah."
Rom 4:4  Orang yang bekerja, menerima gaji; dan gajinya tidak dianggap sebagai suatu pemberian, sebab itu adalah haknya.
Rom 4:5  Tetapi ada orang yang tidak bergantung pada usahanya sendiri; ia mempercayakan dirinya kepada Allah yang menyatakan orang berdosa bebas dari kesalahan. Berdasarkan percayanya itulah Allah menerima orang itu sebagai orang yang menyenangkan hati Allah.
Rom 4:6  Begitulah pendapat Daud juga; itu sebabnya ia mengucapkan selamat berbahagia kepada orang yang oleh Allah sudah diterima sebagai orang yang menyenangkan hati-Nya, tanpa Allah memperhatikan perbuatan-perbuatan orang itu. Daud berkata begini,
Rom 4:7  "Berbahagialah orang yang kesalahan-kesalahannya dimaafkan dan dosa-dosanya diampuni Allah!
Rom 4:8  Berbahagialah orang yang dosa-dosanya tidak dituntut oleh Tuhan!"
Rom 4:9  Apakah ucapan berbahagia ini ditujukan hanya kepada orang-orang yang menuruti peraturan sunat saja? Ataukah juga kepada orang-orang yang tidak menuruti peraturan sunat? Sudah kami sebut sebelumnya, bahwa Abraham diterima oleh Allah sebagai orang yang menyenangkan hati-Nya, karena Abraham percaya kepada Allah.
Rom 4:10  Nah, kapankah Allah menyatakan hal itu? Sebelum atau sesudah Abraham disunat? Memang sebelum ia disunat, bukan sesudahnya.
Rom 4:11  Ia disunat kemudian, dan sunatnya itu hanya sebagai tanda bahwa Allah sudah menerimanya sebagai orang yang menyenangkan hati Allah, karena percaya kepada Allah. Ia diterima oleh Allah pada waktu ia masih belum mengikuti peraturan sunat. Dan Abraham menjadi bapak secara rohani bagi semua orang yang percaya kepada Allah, dan yang karenanya sudah diterima oleh Allah sebagai orang yang menyenangkan hati-Nya, walaupun mereka tidak mengikuti peraturan sunat.
Rom 4:12  Abraham juga bapak rohani untuk orang-orang yang mengikuti peraturan sunat. Ia bapak rohani mereka bukan hanya karena mereka mengikuti peraturan sunat, tetapi juga karena mereka hidup dengan percaya kepada Allah, sama seperti Abraham pada waktu ia belum mengikuti peraturan sunat.
Rom 4:13  Allah berjanji kepada Abraham dan keturunannya bahwa dunia ini akan menjadi milik Abraham. Allah berjanji begitu bukan karena Abraham taat kepada hukum agama Yahudi, tetapi karena ia percaya kepada Allah sehingga ia diterima oleh Allah sebagai orang yang menyenangkan hati-Nya.
Rom 4:14  Sebab kalau hanya orang-orang yang taat kepada hukum-hukum agama Yahudi saja yang akan menerima apa yang dijanjikan oleh Allah, maka percaya kepada Allah tidak berguna sama sekali, dan janji Allah pun kosong belaka.
Rom 4:15  Hukum agama Yahudi mendatangkan hukuman Allah. Tetapi kalau hukum tidak ada, maka pelanggaran pun tidak ada.
Rom 4:16  Jadi janji Allah itu berdasarkan percayanya orang kepada Allah. Itu menjadi suatu jaminan kepada semua keturunan Abraham bahwa janji itu diberikan kepada mereka sebagai suatu pemberian yang cuma-cuma dari Allah; bukan hanya kepada mereka yang taat kepada hukum agama Yahudi saja, tetapi juga kepada mereka yang percaya kepada Allah sama seperti Abraham percaya kepada-Nya. Sebab Abraham adalah bapak kita semua secara rohani.
Rom 4:17  Sebab Allah berkata begini kepada Abraham, "Aku sudah menjadikan engkau bapak untuk banyak bangsa." Demikianlah Allah memberikan janji itu kepada Abraham. Dan Abraham percaya kepada-Nya. Dialah Allah yang menghidupkan orang mati; Dialah juga Allah yang dengan berkata saja membuat apa yang tidak pernah ada menjadi ada.
Rom 4:18  Abraham terus saja berharap dan percaya meskipun tidak ada harapan lagi. Karena itu ia menjadi bapak banyak bangsa. Seperti yang tertulis dalam Alkitab, "Keturunanmu akan menjadi banyak sekali."
Rom 4:19  Abraham pada waktu itu tahu bahwa ia sudah tidak mungkin lagi mempunyai keturunan, sebab badannya sudah terlalu tua dan umurnya sudah hampir seratus tahun; lagipula Sara, istrinya itu, mandul. Namun iman Abraham tidak menjadi berkurang.
Rom 4:20  Ia tetap percaya dan tidak ragu-ragu akan janji Allah. Malah imannya menjadikan dia bertambah kuat, sehingga ia memuji-muji Allah.
Rom 4:21  Ia percaya sekali bahwa Allah dapat melakukan apa yang sudah dijanjikan-Nya.
Rom 4:22  Itu sebabnya Abraham diterima oleh Allah sebagai orang yang menyenangkan hati Allah.
Rom 4:23  Perkataan "diterima sebagai orang yang menyenangkan hati Allah" tertulis bukan hanya untuk Abraham sendiri saja,
Rom 4:24  tetapi juga untuk kita. Kita juga akan diterima sebagai orang yang sudah menyenangkan hati Allah, karena kita percaya kepada Allah yang menghidupkan Yesus, Tuhan kita, dari kematian.
Rom 4:25  Yesus itu sudah diserahkan untuk dibunuh karena dosa-dosa kita; lalu Ia dihidupkan kembali oleh Allah untuk memungkinkan kita berbaik kembali dengan Allah.
Rom 5:1  Sekarang kita sudah berbaik kembali dengan Allah, karena kita percaya. Dan oleh sebab itu kita hidup dalam kedamaian dengan Allah melalui Tuhan kita Yesus Kristus.
Rom 5:2  Oleh sebab kita percaya kepada Yesus, maka Ia memungkinkan kita menghayati kasih Allah, dan dengan kasih itulah kita hidup sekarang. Karena itu kita bersuka hati karena kita mempunyai harapan bahwa kita akan merasakan kebahagiaan yang diberikan Allah!
Rom 5:3  Dan lebih dari itu, kita pun gembira di dalam penderitaan, sebab kita tahu bahwa penderitaan membuat orang menjadi tekun,
Rom 5:4  dan ketekunan akan membuat orang tahan uji; inilah yang menimbulkan pengharapan.
Rom 5:5  Harapan yang seperti ini tidak akan mengecewakan kita, sebab hati kita sudah diisi oleh Allah dengan kasih-Nya. Allah melakukan itu dengan perantaraan Roh-Nya, yang sudah diberikan kepada kita.
Rom 5:6  Ketika kita dalam keadaan tidak berdaya, Kristus mati untuk kita pada waktu yang tepat yang ditentukan oleh Allah; padahal kita orang-orang yang jauh dari Allah.
Rom 5:7  Untuk seseorang yang adil pun sukar orang mau mati. Barangkali untuk seseorang yang baik, ada juga orang yang berani mati.
Rom 5:8  Tetapi Allah menyatakan kasih-Nya kepada kita ketika Kristus mati untuk kita pada waktu kita masih orang berdosa.
Rom 5:9  Sekarang kita sudah berbaik kembali dengan Allah melalui kematian Kristus; karena itu pasti kita akan diselamatkan juga dari murka Allah oleh Kristus.
Rom 5:10  Kalau pada masa kita bermusuhan dengan Allah, kita didamaikan dengan-Nya melalui kematian Anak-Nya, apalagi sekarang sesudah hubungan kita dengan Allah baik kembali, tentu kita akan diselamatkan juga melalui hidup Kristus.
Rom 5:11  Dan lebih daripada itu, kita pun bergembira juga atas kebaikan Allah melalui Tuhan kita Yesus Kristus. Sebab melalui Kristus, kita sekarang menikmati hubungan kita yang baik itu dengan Allah.
Rom 5:12  Dosa masuk ke dalam dunia melalui satu orang, dan dari dosa itu timbullah kematian. Akibatnya, kematian menjalar pada seluruh umat manusia, sebab semua orang sudah berdosa.
Rom 5:13  Sebelum hukum agama Yahudi diberikan, dosa sudah ada di dalam dunia. Tetapi dosa tidak dituntut, karena tidak ada hukum yang bisa dilanggar.
Rom 5:14  Namun, dari zaman Adam sampai pada zaman Musa, kematian menguasai seluruh umat manusia. Malah orang-orang yang tidak membuat pelanggaran dengan cara yang sama seperti yang dibuat oleh Adam, orang-orang itu pun turut juga dikuasai oleh kematian. Adam adalah gambaran daripada Dia yang akan datang kemudian.
Rom 5:15  Tetapi keduanya itu tidak sama; sebab pemberian Allah jauh lebih besar kalau dibandingkan dengan pelanggaran yang dibuat oleh Adam. Pelanggaran satu orang menyebabkan banyak orang mati. Betapa lebih besar lagi akibat dari rahmat Allah dan hadiah keselamatan yang diberikan-Nya kepada begitu banyak orang, melalui kebaikan hati satu orang, yaitu Yesus Kristus.
Rom 5:16  Hadiah Allah juga jauh lebih besar daripada dosa orang yang satu itu. Sebab sesudah satu orang melakukan pelanggaran, keluarlah vonis, "Bersalah". Tetapi sesudah banyak orang berbuat dosa datanglah hadiah dari Allah yang menyatakan, "Tidak bersalah".
Rom 5:17  Karena pelanggaran satu orang, kematian menjalar ke mana-mana melalui orang yang satu itu. Betapa lebih besar lagi akibat dari apa yang dilakukan oleh satu orang yang lain, yaitu Yesus Kristus. Melalui Dia, Allah melimpahkan rahmat-Nya kepada begitu banyak orang, dan dengan cuma-cuma memungkinkan mereka berbaik kembali dengan Allah; mereka akan berkuasa di dunia ini melalui Kristus.
Rom 5:18  Jadi, sebagaimana pelanggaran satu orang mengakibatkan seluruh umat manusia dihukum, begitu juga perbuatan satu orang yang mengikuti kehendak Allah, mengakibatkan semua orang dibebaskan dari kesalahan dan diberi hidup.
Rom 5:19  Dan sebagaimana banyak orang menjadi orang berdosa karena satu orang tidak taat, begitu juga banyak orang dimungkinkan berbaik kembali dengan Allah karena satu orang taat kepada Allah.
Rom 5:20  Yang memberikan hukum agama Yahudi adalah Allah sendiri. Tetapi semakin manusia berusaha mentaati hukum itu, semakin mereka melanggarnya dan berdosa terhadap Allah. Namun semakin manusia berbuat dosa, Allah semakin pula mengasihi mereka.
Rom 5:21  Nah, kita harus mati karena kita berbuat dosa. Tetapi Allah mengasihi kita dan memberi jalan kepada kita untuk berbaik kembali dengan Dia. Dan karena hubungan kita dengan Allah sudah baik kembali, Ia memberikan juga hidup sejati dan kekal kepada kita melalui Yesus Kristus, Tuhan kita.
Rom 6:1  Kalau begitu, apakah yang dapat kita katakan? Haruskah kita terus saja berbuat dosa supaya Allah semakin mengasihi kita?
Rom 6:2  Tentu tidak! Dosa tidak lagi berkuasa atas kita, jadi, mana bisa kita terus-menerus hidup dengan berbuat dosa?
Rom 6:3  Tahukah Saudara-saudara bahwa pada waktu kita dibaptis, kita dipersatukan dengan Kristus Yesus? Ini berarti kita dipersatukan dengan kematian-Nya.
Rom 6:4  Dengan baptisan itu, kita dikubur dengan Kristus dan turut mati bersama-sama Dia, supaya sebagaimana Kristus dihidupkan dari kematian oleh kuasa Bapa yang mulia, begitu pun kita dapat menjalani suatu hidup yang baru.
Rom 6:5  Kalau kita sudah menjadi satu dengan Kristus sebab kita turut mati bersama Dia, kita akan menjadi satu dengan Dia juga karena kita turut dihidupkan kembali seperti Dia.
Rom 6:6  Kita mengetahui bahwa tabiat kita yang lama sebagai manusia sudah dimatikan bersama-sama Kristus pada kayu salib supaya kuasa tabiat kita yang berdosa itu dihancurkan; dengan demikian kita tidak lagi diperhamba oleh dosa.
Rom 6:7  Karena kalau seseorang mati, orang itu dibebaskan dari kuasa dosa.
Rom 6:8  Kalau kita sudah mati bersama Kristus, kita percaya bahwa kita pun akan hidup bersama Dia.
Rom 6:9  Sebab kita tahu bahwa Kristus sudah dihidupkan dari kematian dan Ia tidak akan mati lagi; kematian sudah tidak lagi berkuasa atas diri-Nya.
Rom 6:10  Kematian yang dialami Kristus adalah kematian terhadap dosa. Itu terjadi satu kali saja untuk selama-lamanya. Dan hidup yang dijalani-Nya sekarang ini adalah hidup untuk Allah.
Rom 6:11  Kalian harus juga menganggap dirimu mati terhadap dosa, tetapi hidup dalam hubungan yang erat dengan Allah melalui Kristus Yesus.
Rom 6:12  Jangan lagi membiarkan dosa menguasai hidupmu yang fana agar Saudara jangan menuruti keinginanmu yang jahat.
Rom 6:13  Janganlah juga Saudara menyerahkan anggota badanmu kepada kuasa dosa untuk digunakan bagi maksud-maksud yang jahat. Tetapi serahkanlah dirimu kepada Allah sebagai orang yang sudah dipindahkan dari kematian kepada hidup. Serahkanlah dirimu seluruhnya kepada Allah supaya dipakai untuk melakukan kehendak Allah.
Rom 6:14  Dosa tidak boleh menguasai kalian, karena kalian tidak lagi hidup di bawah hukum agama Yahudi tetapi di bawah rahmat Allah.
Rom 6:15  Sekarang, apa kesimpulannya? Bolehkah kita berdosa, sebab kita tidak lagi di bawah kekuasaan hukum agama Yahudi, melainkan di bawah kekuasaan rahmat Allah? Sekali-kali tidak!
Rom 6:16  Tahukah kalian bahwa kalau kalian menyerahkan diri kepada seseorang untuk melakukan kemauannya maka kalian adalah hamba orang yang kalian taati itu--entah hamba dosa yang membawa kalian kepada kematian, atau hamba yang taat kepada Allah, dan dengan demikian berbaik kembali dengan Allah.
Rom 6:17  Tetapi syukur kepada Allah! Sebab dahulu kalian menjadi hamba dosa, tetapi sekarang kalian dengan sepenuh hati mentaati pengajaran benar yang sudah diberikan kepadamu.
Rom 6:18  Kalian sudah dibebaskan dari dosa, dan sekarang menjadi hamba untuk kehendak Allah.
Rom 6:19  Karena daya tangkapmu begitu lemah, saya memakai contoh-contoh perhambaan supaya lebih mudah kalian mengerti. Dahulu kalian menyerahkan dirimu seluruhnya sebagai hamba bagi hal-hal yang kotor dan yang jahat untuk maksud-maksud yang jahat. Begitu juga sekarang, hendaklah kalian menyerahkan diri seluruhnya sebagai hamba bagi kehendak Allah untuk maksud-maksud Allah yang khusus.
Rom 6:20  Waktu kalian diperhamba oleh dosa, kalian tidak dikuasai oleh kehendak Allah.
Rom 6:21  Pada waktu itu keuntungan apakah yang kalian terima dari hal-hal yang sekarang ini kalian malu melakukannya? Perbuatan-perbuatan itu hanya membawa kematian!
Rom 6:22  Tetapi sekarang kalian sudah dibebaskan dari dosa, dan menjadi hamba Allah. Keuntunganmu ialah bahwa Saudara hidup khusus untuk Allah dan hal itu menghasilkan hidup sejati dan kekal.
Rom 6:23  Sebab kematian adalah upah dari dosa; tetapi hidup sejati dan kekal bersama Kristus Yesus Tuhan kita adalah pemberian yang diberikan oleh Allah dengan cuma-cuma.
Rom 7:1  Saudara-saudaraku! Saudara semuanya sudah mengenal hukum. Jadi, kalian tahu bahwa kekuasaan hukum atas seseorang, berlaku hanya selama orang itu masih hidup.
Rom 7:2  Seorang wanita yang bersuami, umpamanya, terikat oleh hukum kepada suaminya hanya selama suaminya masih hidup. Kalau suaminya mati, istri itu bebas dari hukum yang mengikatnya kepada suaminya.
Rom 7:3  Kalau wanita itu hidup dengan laki-laki yang lain pada waktu suaminya masih hidup, maka wanita itu dikatakan berzinah. Tetapi kalau suaminya meninggal, maka secara hukum, wanita itu sudah bebas. Dan kalau ia kawin lagi dengan seorang laki-laki yang lain, maka ia tidak berzinah.
Rom 7:4  Begitu juga halnya dengan kalian, Saudara-saudaraku. Terhadap hukum agama Yahudi kalian sudah mati, karena kalian telah dipersatukan dengan tubuh Kristus dan menjadi kepunyaan Dia yang sudah dihidupkan kembali dari kematian. Ia dihidupkan dari kematian supaya kita menjadi berguna bagi Allah dalam kehidupan kita.
Rom 7:5  Sebab, dahulu kita hidup menurut sifat-sifat manusia. Pada waktu itu keinginan-keinginan yang berdosa, yang timbul karena adanya hukum agama, memegang peranan dalam diri kita. Itu sebabnya kita melakukan hal-hal yang mengakibatkan kematian.
Rom 7:6  Tetapi sekarang kita tidak lagi terikat pada hukum agama Yahudi. Kita sudah mati terhadap hukum yang dahulunya menguasai kita. Kita tidak lagi mengabdi dengan cara yang lama, menurut hukum yang tertulis. Sekarang kita mengabdi menurut cara baru yang ditunjukkan oleh Roh Allah kepada kita.
Rom 7:7  Kalau begitu, apakah yang dapat kita katakan? Bahwa hukum agama Yahudi jahat? Tentu tidak! Tetapi hukum itulah yang mengajar saya tentang dosa. Saya tidak akan tahu tamak itu apa, kalau hukum agama tidak mengatakan, "Janganlah tamak."
Rom 7:8  Melalui hukum agama, dosa mendapat kesempatan untuk menimbulkan segala macam keinginan yang tamak di dalam hati saya; sebab kalau hukum agama tidak ada, maka dosa pun mati.
Rom 7:9  Saya dahulu hidup tanpa hukum agama. Tetapi ketika hukum agama muncul, dosa mulai hidup
Rom 7:10  dan saya mati. Maka hukum agama itu, yang mulanya dimaksudkan untuk memberi hidup, malah mendatangkan kematian kepada saya.
Rom 7:11  Karena melalui hukum agama itu, dosa mengambil kesempatan menipu dan membunuh saya.
Rom 7:12  Hukum agama berasal dari Allah, dan setiap perintah dalam hukum itu datangnya dari Allah, jadi adil dan baik.
Rom 7:13  Nah, apakah ini berarti bahwa yang baik itu mendatangkan kematian bagi saya? Tentu tidak! Dosalah yang mendatangkan kematian itu. Dengan memakai yang baik, dosa mendatangkan kematian kepada saya, supaya sifat-sifat dosa kelihatan dengan jelas. Melalui perintah-perintah Allah, terbuktilah betapa jahatnya dosa.
Rom 7:14  Kita tahu bahwa hukum agama Yahudi berasal dari Roh Allah; tetapi saya ini manusia lemah. Saya sudah dijual untuk menjadi hamba dosa.
Rom 7:15  Sebab saya sendiri tidak mengerti perbuatan saya. Hal-hal yang saya ingin lakukan, itu tidak saya lakukan; tetapi hal-hal yang saya benci, itu malah yang saya lakukan.
Rom 7:16  Nah, kalau saya melakukan hal-hal yang bertentangan dengan keinginan saya, itu berarti saya mengakui bahwa hukum agama Yahudi itu baik.
Rom 7:17  Jadi, bukan lagi saya sebenarnya yang melakukan itu, melainkan dosa yang menguasai diri saya.
Rom 7:18  Saya tahu bahwa tidak ada sesuatu pun yang baik di dalam diri saya; yaitu di dalam tabiat saya sebagai manusia. Sebab ada keinginan pada saya untuk berbuat baik, tetapi saya tidak sanggup menjalankannya.
Rom 7:19  Saya tidak melakukan yang baik yang saya ingin lakukan; sebaliknya saya melakukan hal-hal yang jahat, yang saya tidak mau lakukan.
Rom 7:20  Kalau saya melakukan hal-hal yang saya tidak mau lakukan, itu berarti bukanlah saya yang melakukan hal-hal itu, melainkan dosa yang menguasai diri saya.
Rom 7:21  Jadi, saya mengambil kesimpulan bahwa hukum inilah yang memegang peranan: yaitu bahwa kalau saya mau melakukan yang baik, maka hanya yang jahat saja yang timbul pada saya.
Rom 7:22  Batin saya suka akan hukum Allah,
Rom 7:23  tetapi saya sadar bahwa dalam diri saya ada pula hukum lain yang memegang peranan--yaitu hukum yang berlawanan dengan hukum yang diakui oleh akal budi saya. Itu sebabnya saya terikat pada hukum dosa yang memegang peranan di dalam diri saya.
Rom 7:24  Nah, beginilah keadaan saya: saya mentaati hukum Allah dengan akal budi saya, tetapi dengan tabiat manusia saya, saya takluk pada dosa. Alangkah celakanya saya ini! Siapakah yang mau menyelamatkan saya dari badan ini yang membawa saya kepada kematian? Syukur kepada Allah! Ia mau menyelamatkan saya melalui Yesus Kristus.
Rom 8:1  Sekarang tidak ada lagi penghukuman terhadap mereka yang hidup bersatu dengan Kristus Yesus.
Rom 8:2  Sebab hukum Roh Allah yang membuat kita hidup bersatu dengan Kristus Yesus sudah membebaskan saya dari hukum yang menyebabkan dosa dan kematian.
Rom 8:3  Apa yang tidak dapat dilakukan oleh hukum agama, karena kita manusia lemah, itu sudah dilakukan oleh Allah. Allah mengalahkan kuasa dosa dalam tabiat manusia dengan mengirimkan Anak-Nya sendiri, yang datang dalam keadaan sama dengan manusia yang berdosa, untuk menghapuskan dosa.
Rom 8:4  Allah melakukan itu supaya kehendak-Nya yang dinyatakan dalam hukum agama Yahudi itu dapat dijalankan dalam diri kita yang hidup menurut Roh Allah dan bukan menurut tabiat manusia.
Rom 8:5  Orang-orang yang hidup menurut tabiat manusia, terus memikirkan apa yang diinginkan oleh tabiat manusia. Tetapi orang-orang yang hidup menurut Roh Allah, terus memikirkan apa yang diinginkan oleh Roh Allah.
Rom 8:6  Kalau pikiranmu dikuasai oleh tabiat manusia, maka akibatnya kematian. Tetapi kalau pikiranmu dikuasai oleh Roh Allah, maka akibatnya ialah hidup dan kedamaian dengan Allah.
Rom 8:7  Orang yang pikirannya dikuasai oleh tabiat manusia, orang itu bermusuhan dengan Allah; karena orang itu tidak tunduk kepada hukum Allah; dan memang ia tidak dapat tunduk kepada hukum Allah.
Rom 8:8  Orang-orang yang hidup menurut tabiat manusia, tidak dapat menyenangkan Allah.
Rom 8:9  Tetapi kalian tidak hidup menurut tabiat manusia. Kalian hidup menurut Roh Allah--kalau, tentunya, Roh Allah sungguh-sungguh memegang peranan di dalam dirimu. Orang yang tidak mempunyai Roh Kristus, orang itu bukanlah kepunyaan Kristus.
Rom 8:10  Tetapi kalau Kristus hidup di dalam dirimu, maka meskipun badanmu akan mati karena dosa, namun Roh Allah memberikan hidup kepadamu, sebab hubunganmu dengan Allah sudah baik.
Rom 8:11  Kalau Roh Allah, yang menghidupkan Kristus dari kematian, hidup di dalam dirimu, maka Ia yang menghidupkan Kristus dari kematian itu, akan menghidupkan juga badanmu yang dapat mati itu. Ia melakukan itu dengan Roh-Nya yang hidup di dalammu.
Rom 8:12  Itulah sebabnya, Saudara-saudara, kita mempunyai tanggung jawab; tetapi bukan tanggung jawab kepada tabiat manusia; kita tidak perlu hidup menurut keinginannya.
Rom 8:13  Karena kalau kalian hidup menurut tabiat manusia, maka kalian akan mati; tetapi kalau dengan kuasa Roh Allah, kalian terus saja mematikan perbuatan-perbuatanmu yang berdosa, maka kalian akan hidup.
Rom 8:14  Orang-orang yang dibimbing oleh Roh Allah, adalah anak-anak Allah.
Rom 8:15  Sebab Roh, yang diberikan oleh Allah kepada kalian, tidaklah membuat kalian menjadi hamba sehingga kalian hidup di dalam ketakutan. Sebaliknya Roh Allah itu menjadikan kalian anak-anak Allah. Dan dengan kuasa Roh Allah itu kita memanggil Allah itu, "Bapa, ya Bapaku!"
Rom 8:16  Roh Allah bersama-sama dengan roh kita menyatakan bahwa kita adalah anak-anak Allah.
Rom 8:17  Nah, kalau kita adalah anak-anak-Nya, maka kita pun adalah ahli waris-Nya yang akan menerima berkat-berkat yang disediakan Allah untuk anak-anak-Nya. Kita akan menerima bersama-sama dengan Kristus apa yang sudah disediakan Allah bagi-Nya; sebab kalau kita menderita bersama Kristus, kita akan dimuliakan juga bersama Dia.
Rom 8:18  Semua penderitaan yang kita alami sekarang, menurut pendapat saya, tidak dapat dibandingkan sama sekali dengan kemuliaan yang akan dinyatakan kepada kita.
Rom 8:19  Seluruh alam menunggu dengan sangat rindu akan saatnya Allah menyatakan anak-anak-Nya.
Rom 8:20  Sebab alam sudah dibiarkan untuk menjadi rapuh, bukan karena kemauannya sendiri, tetapi karena Allah membiarkannya demikian. Meskipun begitu ada juga harapan ini:
Rom 8:21  bahwa pada suatu waktu alam akan dibebaskan dari kuasa yang menghancurkannya dan akan turut dimerdekakan dan diagungkan bersama-sama dengan anak-anak Allah.
Rom 8:22  Kita tahu bahwa sampai saat ini seluruh alam mengeluh karena menderita seperti seorang ibu menderita pada waktu melahirkan bayi.
Rom 8:23  Dan bukannya seluruh alam saja yang mengeluh; kita sendiri pun mengeluh di dalam batin kita. Kita sudah menerima Roh Allah sebagai pemberian Allah yang pertama, namun kita masih juga menunggu Allah membebaskan diri kita seluruhnya dan menjadikan kita anak-anak-Nya.
Rom 8:24  Karena dengan berharap, maka kita diselamatkan. Tetapi kalau apa yang kita harapkan itu sudah kita lihat, maka itu bukan lagi harapan. Sebab siapakah masih mengharapkan sesuatu yang sudah dilihatnya?
Rom 8:25  Tetapi kalau kita mengharapkan sesuatu yang belum kita lihat, maka kita menunggunya dengan sabar.
Rom 8:26  Begitu juga Roh Allah datang menolong kita kalau kita lemah. Sebab kita tidak tahu bagaimana seharusnya kita berdoa; Roh itu sendiri menghadap Allah untuk memohonkan bagi kita dengan kerinduan yang sangat dalam sehingga tidak dapat diucapkan.
Rom 8:27  Maka Allah, yang mengetahui isi hati manusia, mengerti kemauan Roh itu; sebab Roh itu memohon kepada Allah untuk umat Allah, dan sesuai dengan kemauan Allah.
Rom 8:28  Kita tahu bahwa Allah mengatur segala hal, sehingga menghasilkan yang baik untuk orang-orang yang mengasihi Dia dan yang dipanggil-Nya sesuai dengan rencana-Nya.
Rom 8:29  Mereka yang telah dipilih oleh Allah, telah juga ditentukan dari semula untuk menjadi serupa dengan Anak-Nya, yaitu Yesus Kristus. Dengan demikian Anak itu menjadi yang pertama di antara banyak saudara-saudara.
Rom 8:30  Begitulah Allah memanggil mereka yang sudah ditentukan-Nya terlebih dahulu; dan mereka yang dipanggil-Nya itu, dimungkinkan-Nya berbaik kembali dengan Dia. Dan mereka yang dimungkinkan-Nya berbaik kembali dengan Allah, mengambil bagian dalam hidup Allah sendiri.
Rom 8:31  Apakah yang dapat dikatakan sekarang tentang semuanya itu? Kalau Allah memihak pada kita, siapakah dapat melawan kita?
Rom 8:32  Anak-Nya sendiri tidak disayangkan-Nya, melainkan diserahkan-Nya untuk kepentingan kita semua; masakan Ia tidak akan memberikan kepada kita segala sesuatu yang lainnya?
Rom 8:33  Siapakah yang dapat menggugat kita umat yang dipilih oleh Allah, kalau Allah sendiri menyatakan bahwa kita tidak bersalah?
Rom 8:34  Apakah ada orang yang mau menyalahkan kita? Kristus Yesus nanti yang membela kita! Dialah yang sudah mati, atau malah Dialah yang sudah dihidupkan kembali dari kematian dan berada pada Allah di tempat yang berkuasa.
Rom 8:35  Apakah ada yang dapat mencegah Kristus mengasihi kita? Dapatkah kesusahan mencegahnya, atau kesukaran, atau penganiayaan, atau kelaparan, atau kemiskinan, atau bahaya, ataupun kematian?
Rom 8:36  Dalam Alkitab tertulis begini, "Sepanjang hari kami hidup di dalam bahaya maut karena Engkau. Kami diperlakukan seperti domba yang mau disembelih."
Rom 8:37  Tidak! Malah di dalam semuanya itu kita mendapat kemenangan yang sempurna oleh Dia yang mengasihi kita!
Rom 8:38  Sebab saya percaya sekali bahwa di seluruh dunia, baik kematian maupun kehidupan, baik malaikat maupun penguasa, baik ancaman-ancaman sekarang ini maupun ancaman-ancaman di masa yang akan datang atau kekuatan-kekuatan lainnya;
Rom 8:39  baik hal-hal yang di langit, maupun hal-hal yang di dalam bumi atau apa saja yang lain, semuanya tidak dapat mencegah Allah mengasihi kita, seperti yang sudah ditunjukkan-Nya melalui Kristus Yesus, Tuhan kita.
Rom 9:1  Apa yang akan saya katakan ini adalah benar karena saya milik Kristus. Saya tidak berdusta. Hati nurani saya yang dibimbing oleh Roh Allah, meyakinkan saya juga bahwa perkataan saya ini benar.
Rom 9:2  Saya sangat sedih dan hati saya menderita
Rom 9:3  karena saudara-saudara saya yang sebangsa dengan saya. Sebab untuk mereka, saya sendiri rela dikutuk oleh Allah dan diceraikan dari Kristus.
Rom 9:4  Mereka adalah umat yang terpilih dan Allah menjadikan mereka anak-anak-Nya sendiri, serta menyatakan kuasa-Nya kepada mereka. Ia mengadakan perjanjian dengan mereka dan memberikan kepada mereka hukum agama. Allah memberitahukan kepada mereka bagaimana mereka harus beribadat dan mereka sudah menerima pula janji-janji-Nya.
Rom 9:5  Mereka adalah keturunan dari nenek moyang kita. Kristus secara manusia berasal dari bangsa mereka. Ia lebih tinggi dari semuanya. Terpujilah Allah untuk selama-lamanya! Amin.
Rom 9:6  Saya tidak bermaksud mengatakan bahwa janji Allah sudah tidak berlaku lagi; tetapi bukan semua orang Israel adalah umat yang dipilih oleh Allah.
Rom 9:7  Tidak semua keturunan Abraham adalah anak-anak Allah. Sebab Allah berkata kepada Abraham, "Hanya keturunan Ishak sajalah yang akan disebut keturunanmu."
Rom 9:8  Itu berarti bahwa keturunan Abraham yang menjadi anak-anak Allah, adalah hanya keturunannya yang lahir karena janji Allah; dan bukan semua keturunannya.
Rom 9:9  Sebab janji Allah adalah sebagai berikut, "Pada waktu yang ditentukan, Aku akan kembali, dan Sara akan mempunyai seorang anak laki-laki."
Rom 9:10  Dan ini juga: Ribka melahirkan dua orang anak laki-laki dari satu ayah, yaitu Ishak nenek moyang kita.
Rom 9:11  Sebelum kedua orang anak itu lahir, Allah sudah menentukan pilihan-Nya untuk selanjutnya. Pilihan Allah itu tidak bergantung kepada apa yang dapat dilakukan oleh orang, tetapi bergantung kepada panggilan Allah sendiri. Sebab pada waktu kedua anak laki-laki Ribka itu belum dapat melakukan sesuatu yang baik atau yang jahat,
Rom 9:12  Allah sudah mengatakan kepada Ribka, "Yang tua akan melayani yang muda."
Rom 9:13  Dalam Alkitab tertulis bahwa Allah berkata begini, "Yakub Aku kasihi, tetapi Esau Aku benci."
Rom 9:14  Apakah kesimpulan kita sekarang? Bahwa Allah itu tidak adilkah? Sudah barang tentu Allah adil!
Rom 9:15  Sebab Allah berkata kepada Musa, "Aku akan menunjukkan rahmat kepada orang yang Aku mau menunjukkan rahmat, dan Aku akan menunjukkan belas kasihan kepada orang yang Aku mau menunjukkan belas kasihan."
Rom 9:16  Jadi, keputusan Allah itu tidaklah bergantung kepada kerelaan manusia atau kepada usaha manusia, melainkan kepada kebaikan hati Allah saja terhadap orang yang dipilih-Nya itu.
Rom 9:17  Sebab dalam Alkitab tertulis, "Allah berkata kepada raja Mesir, 'Aku menjadikan engkau raja untuk satu maksud ini saja, yaitu supaya dengan engkau, Aku menunjukkan kekuasaan-Ku dan membuat nama-Ku termasyhur di seluruh dunia.'"
Rom 9:18  Jadi, Allah berbelaskasihan kepada seseorang, kalau Allah menghendaki begitu. Dan Allah menyebabkan seseorang menjadi keras kepala, kalau Allah menghendaki demikian juga.
Rom 9:19  Nah, Saudara akan berkata kepada saya, "Kalau begitu mengapa Allah masih mau menyalahkan manusia? Bukankah tidak seorang pun dapat mencegah keinginan Allah?"
Rom 9:20  Tetapi, Saudara! Saudara hanya manusia saja. Dan Saudara tidak boleh berani menyahut kepada Allah! Bolehkah pot kembang bertanya kepada orang yang membuatnya, "Mengapa engkau membuat saya begini?"
Rom 9:21  Bukankah orang yang membuat pot kembang itu berhak mengerjakan tanah liat itu sekehendak hatinya? Dari segumpal tanah liat, orang itu berhak membuat dua macam pot kembang: satu yang bagus, dan yang lainnya yang kurang bagus.
Rom 9:22  Begitu jugalah dengan apa yang dibuat oleh Allah. Ia berniat untuk melampiaskan kemarahan-Nya dan memperlihatkan kekuasaan-Nya. Namun Ia sabar terhadap mereka yang harus dihukum karena membuat Ia murka.
Rom 9:23  Allah juga berniat untuk menunjukkan kepada kita kebahagiaan berlimpah-limpah yang dicurahkan-Nya kepada kita yang dikasihani-Nya. Kita sudah disiapkan-Nya untuk menerima kebahagiaan itu.
Rom 9:24  Kitalah yang sudah dipanggil-Nya, bukan hanya dari bangsa Yahudi, tetapi juga dari bangsa-bangsa lain.
Rom 9:25  Sebab dalam buku Nabi Hosea, Allah berkata, "Orang-orang yang bukan umat-Ku, akan Kusebut 'Umat-Ku'. Bangsa yang tidak Aku kasihi, akan Kusebut 'Kekasih-Ku'.
Rom 9:26  Dan di tempat di mana dikatakan kepada orang, 'Kamu bukan umat-Ku,' di situ orang-orang itu akan disebut anak-anak Allah yang hidup."
Rom 9:27  Nabi Yesaya berkata dengan tegas mengenai bangsa Israel, "Sungguh pun jumlah bangsa Israel sebanyak pasir di laut, hanya sedikit saja yang akan selamat;
Rom 9:28  sebab Tuhan akan segera menjatuhkan hukuman terhadap seisi dunia."
Rom 9:29  Yesaya berkata begini juga, "Seandainya Allah Yang Mahakuasa tidak meninggalkan kepada kita keturunan, pasti kita semua sudah menjadi seperti Sodom dan Gomora."
Rom 9:30  Jadi, kesimpulannya ialah ini: Bangsa-bangsa lain yang bukan Yahudi tidak berusaha supaya hubungan mereka dengan Allah menjadi baik kembali. Tetapi karena mereka percaya, maka Allah membuat hubungan mereka dengan Dia menjadi baik kembali.
Rom 9:31  Sebaliknya, orang-orang Yahudi selalu berusaha mentaati hukum supaya hubungan mereka dengan Allah menjadi baik kembali. Tetapi mereka justru tidak berhasil.
Rom 9:32  Mengapa mereka tidak berhasil? Sebab mereka melakukan itu tidak melalui percaya kepada Allah, melainkan melalui usaha mereka sendiri. Maka mereka jatuh tersandung pada "Batu Sandungan".
Rom 9:33  Mengenai batu itu di dalam Alkitab tertulis begini, "Perhatikanlah ini: Di Sion sudah Kuletakkan sebuah batu yang akan mengakibatkan orang tersandung; yaitu sebuah batu besar yang akan mengakibatkan orang jatuh. Tetapi orang yang percaya kepada-Nya tidak akan kecewa."
Rom 10:1  Saudara-saudara! Saya rindu sekali supaya bangsa saya diselamatkan. Dan saya sungguh-sungguh berdoa kepada Allah untuk mereka.
Rom 10:2  Saya berani mengatakan bahwa mereka bersemangat sekali mengabdi kepada Allah. Tetapi semangat mereka itu tidak berdasarkan pengetahuan yang dari Allah.
Rom 10:3  Mereka tidak mengetahui caranya Allah membuat hubungan manusia dengan Dia menjadi baik kembali. Dan karena mereka mau mengikuti cara mereka sendiri, maka mereka tidak tunduk kepada cara yang ditunjuk Allah.
Rom 10:4  Hukum agama sudah dipenuhi oleh Kristus. Jadi setiap orang yang percaya kepada Kristus, hubungannya dengan Allah menjadi baik kembali.
Rom 10:5  Musa menulis bahwa orang yang berbaik dengan Allah berdasarkan hukum agama, orang itu akan hidup karena taat kepada hukum agama itu.
Rom 10:6  Tetapi mengenai cara orang berbaik dengan Allah berdasarkan percaya kepada Allah, Alkitab mengatakan begini, "Tidak usahlah engkau berkata di dalam hatimu, 'Siapakah yang akan naik ke surga?' (artinya membawa Kristus turun),
Rom 10:7  atau, 'Siapakah yang akan turun ke dunia orang mati?' (artinya mengangkat Kristus naik dari kematian)."
Rom 10:8  Maksudnya adalah begini, "Berita dari Allah itu dekat sekali padamu; pada mulutmu dan dalam hatimu." Itulah berita yang kami siarkan; berita bahwa orang harus percaya.
Rom 10:9  Sebab kalau Saudara mengaku dengan mulutmu bahwa "Yesus itu Tuhan", dan Saudara percaya dalam hatimu bahwa Allah sudah menghidupkan Yesus dari kematian, maka Saudara akan selamat.
Rom 10:10  Karena dengan hatinya orang percaya, sehingga Allah menerima dia sebagai orang yang berbaik dengan Allah. Dan dengan mulutnya orang mengaku, sehingga ia diselamatkan.
Rom 10:11  Di dalam Alkitab tertulis, "Orang yang percaya tidak akan dikecewakan."
Rom 10:12  Itu berlaku terhadap semua orang, sebab tidak ada bedanya antara orang Yahudi dengan orang-orang bangsa lain. Allah yang satu itu adalah Tuhan untuk semua orang. Ia memberikan berkat yang berlimpah-limpah kepada semua orang yang meminta tolong kepada-Nya.
Rom 10:13  Dalam Alkitab tertulis, "Semua orang yang berseru kepada Tuhan, akan selamat."
Rom 10:14  Tetapi bagaimanakah orang dapat berseru kepada Tuhan kalau mereka belum percaya? Dan bagaimanakah mereka dapat percaya kepada Tuhan kalau mereka belum mendengar tentang Dia? Juga, bagaimanakah mereka dapat mendengar tentang Tuhan, kalau tidak ada yang memberitakan?
Rom 10:15  Dan bagaimanakah orang dapat membawa berita itu kalau mereka tidak diutus? Di dalam Alkitab tertulis begini, "Alangkah baiknya kedatangan orang-orang yang membawa Kabar Baik dari Allah!"
Rom 10:16  Tetapi tidak semua orang menerima Kabar Baik itu. Sebab Yesaya berkata, "Tuhan, siapakah yang percaya pada berita kami?"
Rom 10:17  Itu sebabnya orang-orang menjadi percaya karena mereka mendengar berita, dan berita didengar karena ada orang yang memberitakan tentang Kristus.
Rom 10:18  Tetapi saya bertanya: Apakah memang mereka belum mendengar berita itu? Pasti sudah! Sebab dalam Alkitab tertulis begini, "Suara mereka sudah berkumandang di seluruh dunia, kata-kata mereka sudah sampai ke ujung bumi."
Rom 10:19  Saya bertanya lagi: Apakah bangsa Israel belum mengetahuinya? Biarlah pertanyaan itu pertama-tama dijawab oleh Musa. Musa berkata, "Beginilah kata Allah, 'Aku akan membuat kamu iri hati terhadap suatu bangsa yang bukan umat dan Aku akan membuat kamu marah terhadap suatu bangsa yang bodoh.'"
Rom 10:20  Yesaya lebih berani lagi. Ia berkata, "Beginilah kata Allah, 'Orang-orang yang tidak mencari Aku, sudah menjumpai Aku; dan Aku memperlihatkan diri-Ku kepada mereka yang tidak menanyakan tentang Aku.'"
Rom 10:21  Tetapi mengenai bangsa Israel Yesaya berkata, "Beginilah kata Allah, 'Sepanjang hari Aku mengulurkan tangan-Ku kepada suatu bangsa yang keras kepala dan tidak taat.'"
Rom 11:1  Saya bertanya: Apakah Allah sudah membuang umat-Nya sendiri? Tentu tidak! Saya sendiri seorang Israel keturunan Abraham, dan dari suku Benyamin.
Rom 11:2  Tidak! Allah tidak membuang umat-Nya yang telah dipilih-Nya sejak semula. Saudara-saudara tahu apa yang tertulis dalam Alkitab mengenai Elia, ketika ia mengadukan soal Israel kepada Allah. Elia berkata,
Rom 11:3  "Tuhan, orang-orang sudah membunuh nabi-nabi-Mu dan menghancurkan tempat-tempat mempersembahkan kurban untuk-Mu. Tinggal saya seorang diri dan mereka mau membunuh saya."
Rom 11:4  Apakah jawaban Allah kepada Elia? Allah menjawab, "Aku sudah meninggalkan tujuh ribu orang untuk diri-Ku sendiri. Mereka belum pernah menyembah Dewa Baal."
Rom 11:5  Begitu juga sekarang ini: Ada sejumlah kecil orang-orang yang telah dipilih oleh Allah karena rahmat-Nya.
Rom 11:6  Ia memilih mereka berdasarkan rahmat-Nya dan bukan berdasarkan perbuatan mereka. Sebab kalau pilihan Allah itu berdasarkan perbuatan manusia, maka rahmat Allah itu bukan lagi rahmat yang sejati.
Rom 11:7  Jadi, bagaimana? Umat Israel tidak mendapat apa yang mereka cari. Yang mendapatnya hanyalah segolongan kecil orang-orang yang telah dipilih oleh Allah. Yang lain semuanya menjadi keras kepala terhadap panggilan Allah.
Rom 11:8  Sebab di dalam Alkitab tertulis begini, "Allah membuat hati dan pikiran mereka menjadi bebal; dan sampai saat ini mata mereka tidak dapat melihat dan telinga mereka tidak dapat mendengar."
Rom 11:9  Daud berkata juga, "Biarlah pesta-pesta mereka menjadi perangkap bagi mereka sendiri, dan menjadi lubang, tempat mereka jatuh dan hancur!
Rom 11:10  Biarlah pandangan mereka menjadi gelap supaya mereka tidak dapat melihat; dan biarlah mereka menjadi bongkok selama-lamanya."
Rom 11:11  Saya bertanya lagi: Ketika orang Yahudi jatuh, apakah itu terjadi supaya mereka hancur? Sekali-kali tidak! Tetapi karena mereka berdosa, maka bangsa lain malah diselamatkan, sehingga menyebabkan orang Yahudi iri hati terhadap bangsa lain itu.
Rom 11:12  Karena bangsa Yahudi bersalah dan tidak menuruti kemauan Allah, maka bangsa-bangsa lain diberkati oleh Allah. Apalagi kalau hubungan bangsa Yahudi dengan Allah menjadi baik kembali; tentu lebih besar lagi berkat yang akan diberikan oleh Allah!
Rom 11:13  Tetapi sekarang baiklah saya berbicara kepada Saudara-saudara yang bukan Yahudi! Selama ini, sebagai rasul untuk bangsa-bangsa yang bukan Yahudi, saya sangat menjunjung tinggi tugas saya.
Rom 11:14  Saya mengharap saya dapat menimbulkan iri hati pada bangsa saya sendiri, supaya dengan jalan itu saya dapat menyelamatkan sebagian dari mereka.
Rom 11:15  Karena mereka ditolak oleh Allah, maka hubungan dunia dengan Allah menjadi baik kembali; apalagi kalau mereka diterima oleh Allah! Tentu itu sama saja seperti orang mati hidup lagi!
Rom 11:16  Kalau sepotong roti yang pertama sudah diberikan kepada Allah, itu berarti seluruh rotinya diberi kepada Allah juga. Dan kalau akar pohon adalah kepunyaan Allah, itu berarti cabang-cabangnya adalah milik-Nya juga.
Rom 11:17  Sebagian dari cabang-cabang pohon zaitun--yaitu orang-orang Yahudi--sudah dikerat. Dan pada bekas keratan itu dicangkokkan cabang pohon zaitun liar, yaitu Saudara-saudara yang bukan Yahudi. Saudara dicangkokkan di situ supaya Saudara menikmati segala yang baik dari kehidupan rohani orang-orang Yahudi.
Rom 11:18  Oleh karena itu janganlah kalian menganggap enteng mereka yang sudah dikerat seperti cabang itu. Kalian harus ingat bahwa kalian hanya cabang saja. Dan bukannya cabang yang memberi makan kepada akar, melainkan akar yang memberi makan kepada cabang.
Rom 11:19  Tetapi Saudara akan berkata, "Ya, tetapi cabang-cabang itu dipotong supaya saya dapat dicangkokkan pada pohonnya!"
Rom 11:20  Itu memang benar. Tetapi mereka dibuang karena mereka tidak percaya, sedangkan Saudara diterima karena Saudara percaya. Jadi janganlah Saudara menjadi sombong karenanya; sebaliknya Saudara harus merasa takut.
Rom 11:21  Sebab kalau Allah tidak merasa sayang untuk membuang orang Yahudi yang seperti cabang-cabang asli itu, jangan menyangka Ia akan merasa sayang untuk membuang Saudara!
Rom 11:22  Jadi di sini kita melihat betapa baiknya Allah dan betapa kerasnya juga Ia. Ia bertindak keras terhadap mereka yang berdosa, tetapi Ia baik hati terhadap Saudara--asal Saudara tetap hidup dari kebaikan-Nya. Kalau tidak, maka Saudara juga akan dibuang.
Rom 11:23  Dan mengenai orang-orang Yahudi itu, kalau mereka berhenti bersikap tidak percaya, maka mereka akan diterima kembali; sebab Allah berkuasa untuk menerima mereka kembali.
Rom 11:24  Saudara yang bukan berasal dari bangsa Yahudi adalah seperti cabang dari pohon zaitun yang liar. Nah, kalau Saudara, bertentangan dengan sifat Saudara, bisa dicangkokkan pada pohon zaitun asli, apalagi orang-orang Yahudi yang diumpamakan dengan cabang-cabang pohon zaitun asli itu. Tentu lebih mudah lagi bagi Allah untuk mengembalikan mereka pada pohon zaitun mereka sendiri.
Rom 11:25  Saudara-saudara! Saya harap kalian jangan merasa sudah tahu segala-galanya. Sebab ada sesuatu hal yang dikehendaki oleh Allah, tetapi yang belum diketahui orang. Dan saya mau kalian mengetahuinya; yaitu ini: Sebagian dari orang Yahudi berkeras kepala, tetapi sikap mereka itu akan berlangsung hanya sampai jumlah orang-orang bukan Yahudi yang datang kepada Allah sudah lengkap.
Rom 11:26  Demikianlah semua orang Yahudi akan selamat. Sebab di dalam Alkitab tertulis begini, "Raja Penyelamat akan datang dari Sion; Ia akan menghapuskan segala kejahatan dari keturunan Yakub.
Rom 11:27  Aku akan mengikat perjanjian ini dengan mereka pada waktu Aku mengampuni dosa-dosa mereka."
Rom 11:28  Karena orang Yahudi tidak mau menerima Kabar Baik dari Allah, maka mereka menjadi musuh Allah; dan itu justru menjadi keuntungan bagi Saudara-saudara yang bukan Yahudi. Tetapi karena pilihan Allah, maka orang-orang Yahudi itu tetap dikasihi oleh Allah demi nenek moyang mereka.
Rom 11:29  Sebab kalau Allah memilih orang dan memberkati orang itu, Allah tidak pernah menarik kembali apa yang telah dibuat-Nya.
Rom 11:30  Dahulu kalian tidak taat kepada Allah. Tetapi sekarang Allah sudah menyatakan bahwa kalian bebas dari kesalahan, karena orang-orang Yahudi tidak taat.
Rom 11:31  Begitu juga dengan orang-orang Yahudi itu. Sekarang ini mereka tidak taat kepada Allah, supaya kalian dinyatakan bebas dari kesalahan. Tetapi nanti mereka juga akan dinyatakan bebas dari kesalahan.
Rom 11:32  Sebab Allah sudah membiarkan seluruh umat manusia dikuasai ketidaktaatan, supaya Ia dapat menunjukkan belas kasihan-Nya kepada mereka semuanya.
Rom 11:33  Sungguh hebat kekayaan Allah! Sungguh besar kebijaksanaan dan pengetahuan-Nya! Siapakah yang dapat menyelidiki keputusan-keputusan-Nya? Siapakah yang dapat mengerti cara-caranya Ia bekerja?
Rom 11:34  Dalam Alkitab tertulis begini, "Siapakah yang mengetahui pikiran Tuhan? Siapakah dapat memberi nasihat kepada-Nya?
Rom 11:35  Siapakah pernah memberi sesuatu kepada Tuhan sehingga bisa menuntut balasan-Nya?"
Rom 11:36  Allah yang menciptakan segala sesuatu. Semuanya berasal dari Allah dan adalah untuk Allah. Terpujilah Allah untuk selama-lamanya! Amin.
Rom 12:1  Saudara-saudara! Allah sangat baik kepada kita. Itu sebabnya saya minta dengan sangat supaya kalian mempersembahkan dirimu sebagai suatu kurban hidup yang khusus untuk Allah dan yang menyenangkan hati-Nya. Ibadatmu kepada Allah seharusnya demikian.
Rom 12:2  Janganlah ikuti norma-norma dunia ini. Biarkan Allah membuat pribadimu menjadi baru, supaya kalian berubah. Dengan demikian kalian sanggup mengetahui kemauan Allah--yaitu apa yang baik dan yang menyenangkan hati-Nya dan yang sempurna.
Rom 12:3  Allah sudah memberi anugerah kepada saya. Itu sebabnya saya menasihati Saudara-saudara semuanya: Janganlah merasa diri lebih tinggi dari yang sebenarnya. Hendaknya kalian menilai keadaan dirimu dengan rendah hati; masing-masing menilai dirinya menurut kemampuan yang diberikan Allah kepadanya oleh karena ia percaya kepada Yesus.
Rom 12:4  Tubuh kita mempunyai banyak anggota. Setiap anggota ada tugasnya sendiri-sendiri.
Rom 12:5  Begitu juga dengan kita. Meskipun kita semuanya banyak, namun kita merupakan satu tubuh karena kita bersatu pada Kristus. Dan kita masing-masing berhubungan satu dengan yang lain sebagai anggota-anggota dari satu tubuh.
Rom 12:6  Kita masing-masing mempunyai karunia-karunia pelayanan yang berlainan. Karunia-karunia itu diberikan oleh Allah kepada kita menurut rahmat-Nya. Sebab itu kita harus memakai karunia-karunia itu. Orang yang mempunyai karunia untuk mengabarkan berita dari Allah, harus mengabarkan berita dari Allah itu menurut kemampuan yang ada padanya.
Rom 12:7  Orang yang mempunyai karunia untuk menolong orang lain, harus sungguh-sungguh menolong orang lain. Orang yang mempunyai karunia untuk mengajar, harus sungguh-sungguh mengajar.
Rom 12:8  Orang yang mempunyai karunia untuk memberi semangat kepada orang lain, harus sungguh-sungguh memberi semangat kepada orang lain. Orang yang mempunyai karunia untuk memberikan kepada orang lain apa yang dipunyainya, harus melakukan itu dengan murah hati secara wajar. Orang yang mempunyai karunia untuk memimpin, harus sungguh-sungguh memimpin. Orang yang mempunyai karunia untuk menunjukkan belas kasihan kepada orang lain, harus melakukannya dengan senang hati.
Rom 12:9  Kasihilah dengan ikhlas. Bencilah yang jahat, dan berpeganglah kepada apa yang baik.
Rom 12:10  Hendaklah Saudara-saudara saling mengasihi satu sama lain dengan mesra seperti orang-orang yang bersaudara dalam satu keluarga, dan hendaknya kalian saling mendahului memberi hormat.
Rom 12:11  Bekerjalah dengan rajin. Jangan malas. Bekerjalah untuk Tuhan dengan semangat dari Roh Allah.
Rom 12:12  Hendaklah Saudara berharap kepada Tuhan dengan gembira, sabarlah di dalam kesusahan, dan tekunlah berdoa.
Rom 12:13  Tolonglah mencukupi kebutuhan orang-orang Kristen lain dan sambutlah saudara-saudara seiman yang tidak Saudara kenal, dengan senang hati di dalam rumahmu.
Rom 12:14  Mintalah kepada Allah supaya Ia memberkati orang-orang yang kejam terhadapmu. Ya, minta Allah memberkati mereka, jangan mengutuk.
Rom 12:15  Turutlah bergembira dengan orang-orang yang bergembira, dan menangislah dengan mereka yang menangis.
Rom 12:16  Hiduplah rukun satu sama lain. Janganlah bersikap tinggi hati, tetapi sesuaikanlah dirimu dengan orang yang rendah kedudukannya. Jangan menganggap diri lebih pandai daripada yang sebenarnya.
Rom 12:17  Kalau orang berbuat jahat kepadamu, janganlah membalasnya dengan kejahatan. Buatlah apa yang dianggap baik oleh semua orang.
Rom 12:18  Dari pihakmu, berusahalah sedapat mungkin untuk hidup damai dengan semua orang.
Rom 12:19  Saudara-saudaraku! Janganlah sekali-kali membalas dendam, biarlah Allah yang menghukum. Sebab di dalam Alkitab tertulis, "Akulah yang membalas. Aku yang akan menghukumnya, kata Tuhan."
Rom 12:20  Sebaliknya, kalau musuhmu lapar, berilah ia makan; dan kalau ia haus, berilah ia minum. Karena dengan berbuat demikian, Saudara akan membuat dia menjadi malu.
Rom 12:21  Janganlah membiarkan dirimu dikalahkan oleh yang jahat, tetapi hendaklah Saudara mengalahkan kejahatan dengan kebaikan.
Rom 13:1  Setiap orang haruslah taat kepada pemerintah, sebab tidak ada pemerintah yang tidak mendapat kekuasaannya dari Allah. Dan pemerintah yang ada sekarang ini, menjalankan kekuasaannya atas perintah dari Allah.
Rom 13:2  Itu sebabnya orang yang menentang pemerintah sama saja dengan menentang apa yang telah ditentukan oleh Allah. Dan orang yang berbuat begitu akan menerima hukuman.
Rom 13:3  Sebab, orang yang berbuat baik tidak usah takut kepada pemerintah. Hanya orang yang berbuat jahat saja yang harus takut. Kalau Saudara ingin supaya Saudara tidak merasa takut terhadap pemerintah, Saudara harus berbuat baik, maka Saudara akan dipuji.
Rom 13:4  Sebab pemerintah adalah hamba Allah yang bekerja untuk kebaikanmu. Tetapi kalau Saudara berbuat jahat, memang Saudara harus takut kepadanya, sebab bukannya sia-sia saja ia berkuasa untuk menghukum orang. Ia adalah hamba Allah, yang menjatuhkan hukuman Allah kepada orang-orang yang berbuat jahat.
Rom 13:5  Itu sebabnya Saudara harus taat kepada pemerintah--bukan hanya karena Saudara tidak mau dihukum, tetapi juga karena suara hati nuranimu.
Rom 13:6  Itulah juga alasannya mengapa Saudara membayar pajak, sebab pemerintah adalah pegawai Allah yang menjalankan tugas yang khusus ini.
Rom 13:7  Jadi bayarlah kepada pemerintah apa yang Saudara harus bayar kepadanya. Bayarlah pajak, kalau Saudara harus membayar pajak; dan bayarlah cukai kalau Saudara harus membayar cukai. Hargailah mereka yang harus dihargai dan hormatilah mereka yang harus dihormati.
Rom 13:8  Janganlah berutang apa pun kepada siapa juga, kecuali berutang kasih terhadap satu sama lain. Sebab orang yang mengasihi sesama manusia, sudah memenuhi semua hukum Musa.
Rom 13:9  Sebab hukum agama Yahudi, yaitu: Jangan berzinah, jangan membunuh, jangan mencuri, jangan ingin mempunyai apa yang orang lain punyai; semuanya itu bersama-sama dengan hukum-hukum yang lain, sudah disimpulkan menjadi satu hukum saja, yaitu, "Kasihilah sesama manusia seperti engkau mengasihi dirimu sendiri."
Rom 13:10  Orang yang mengasihi orang lain, tidak akan berbuat jahat kepada orang itu. Jadi orang yang mengasihi sesamanya adalah orang yang sudah memenuhi semua syarat hukum agama.
Rom 13:11  Selain semuanya itu kalian tahu bahwa saat ini adalah saat bagimu untuk bangun dari tidur. Sebab waktunya untuk kita diselamatkan sudah lebih dekat sekarang ini daripada waktu kita baru mulai percaya.
Rom 13:12  Malam sudah hampir lewat; dan sebentar lagi akan siang. Jadi, baiklah kita berhenti melakukan perbuatan-perbuatan gelap. Kita harus melengkapi diri kita dengan senjata terang.
Rom 13:13  Kita harus melakukan hal-hal terhormat seperti yang biasanya dilakukan orang pada siang hari; jangan berpesta pora melampaui batas, atau mabuk. Jangan cabul, atau berkelakuan tidak sopan. Jangan berkelahi, atau iri hati.
Rom 13:14  Biarlah Tuhan Yesus Kristus yang menentukan apa yang kalian harus lakukan. Dan janganlah menuruti tabiat manusia yang berdosa untuk memuaskan hawa nafsu.
Rom 14:1  Orang yang tidak yakin akan apa yang dipercayainya harus diterima dengan baik di antara Saudara-saudara. Jangan bertengkar dengan dia mengenai pendirian-pendiriannya.
Rom 14:2  Ada orang yang berpendirian bahwa ia boleh makan apa saja. Tetapi ada orang lain yang lemah keyakinannya; ia merasa bahwa ia hanya boleh makan sayur-sayuran saja.
Rom 14:3  Orang yang makan apa saja janganlah menganggap rendah orang yang makan hanya makanan tertentu saja; dan orang yang makan hanya makanan tertentu saja, janganlah pula menyalahkan orang yang makan segala-galanya, sebab Allah sudah menerima dia.
Rom 14:4  Siapakah Saudara sehingga Saudara harus mengadili hamba orang lain? Entah hamba itu jatuh atau bangun, itu adalah urusan tuannya. Dan memang hamba itu akan berdiri tegak, karena Tuhan sanggup membuatnya berdiri tegak.
Rom 14:5  Ada orang yang merasa suatu hari tertentu lebih penting dari hari-hari yang lain, sedangkan orang lain pula menganggap bahwa hari-hari itu sama saja. Biarkan masing-masing orang menentukan pendiriannya sendiri.
Rom 14:6  Orang yang mementingkan hari-hari tertentu, orang itu berbuat begitu untuk menghormati Tuhan. Orang yang makan segala-galanya, berbuat begitu untuk menghormati Tuhan, karena ia bersyukur kepada Allah atas makanan itu. Begitu juga dengan orang yang makan hanya makanan tertentu saja; orang itu juga menghormati Tuhan dan bersyukur kepada Allah.
Rom 14:7  Tidak seorang pun dari kita yang hidup untuk diri sendiri; dan tidak seorang pun dari kita yang mati untuk dirinya sendiri.
Rom 14:8  Kalau kita hidup, kita hidup untuk Tuhan. Dan kalau kita mati, kita pun mati untuk Tuhan. Jadi, hidup atau mati, kita adalah milik Tuhan.
Rom 14:9  Kristus sudah mati dan hidup kembali. Itu sebabnya Ia menjadi Tuhan untuk orang-orang yang hidup dan juga untuk orang-orang yang telah mati.
Rom 14:10  Jadi, Saudara-saudara! Untuk apa menyalahkan saudaramu yang seiman? Dan untuk apa Saudara menganggap dia rendah? Kita semua akan menghadap Allah untuk diadili.
Rom 14:11  Di dalam Alkitab tertulis, "Sesungguhnya," kata Tuhan, "tiap-tiap orang akan bersembah sujud di hadapan-Ku; dan setiap orang akan mengaku bahwa Akulah Allah."
Rom 14:12  Jadi kita masing-masing harus mempertanggungjawabkan segala perbuatan kita kepada Allah.
Rom 14:13  Oleh karena itu janganlah kita saling menyalahkan. Sebaliknya berusahalah supaya kalian tidak berbuat sesuatu pun yang menyebabkan seorang saudara seiman tergoda dan berdosa.
Rom 14:14  Karena saya bersatu dengan Tuhan Yesus, maka saya percaya sekali bahwa tidak ada sesuatu pun yang pada dasarnya najis; tetapi hal itu najis bagi seseorang, kalau orang itu menganggapnya najis.
Rom 14:15  Tetapi kalau dengan apa yang Saudara makan, seorang saudara seiman disakiti hatinya, maka Saudara tidak lagi bertindak berdasarkan kasih. Kalau Kristus sudah mati untuk seseorang, janganlah membiarkan orang itu dirusak oleh apa yang Saudara makan.
Rom 14:16  Itu sebabnya janganlah membiarkan apa yang baik bagi kalian, dianggap tidak baik oleh orang lain.
Rom 14:17  Sebab kalau Allah memerintah hidup seseorang, apa yang ia boleh makan atau minum, tidak lagi penting. Yang penting ialah bahwa orang itu menuruti kemauan Allah, mengalami ketenangan hati dan menerima sukacita yang diberikan oleh Roh Allah.
Rom 14:18  Orang yang melayani Kristus secara demikian, orang itu menyenangkan hati Allah, dan dihargai oleh orang-orang lain.
Rom 14:19  Sebab itu tujuan kita haruslah selalu untuk hal-hal yang menciptakan kerukunan dan saling membangun.
Rom 14:20  Janganlah, karena soal makanan, Saudara merusak apa yang sudah dikerjakan oleh Allah. Segala makanan memang halal untuk dimakan; tetapi kalau apa yang Saudara makan menyebabkan orang lain berdosa, maka Saudara bersalah.
Rom 14:21  Lebih baik tidak usah makan daging atau minum anggur atau melakukan apa saja kalau hal itu menyebabkan seorang saudara seiman menjadi berdosa.
Rom 14:22  Biarlah apa yang Saudara percayai itu, Saudara lakukan di hadapan Allah saja untuk Saudara sendiri. Orang yang tidak mempunyai alasan untuk merasa bersalah atas apa yang dianggapnya benar, orang itu bahagia.
Rom 14:23  Tetapi orang yang merasa ragu-ragu untuk makan sesuatu, kemudian toh makan makanan itu, orang itu disalahkan oleh Allah; sebab orang itu tidak bertindak menurut keyakinannya tentang apa yang benar dan yang salah. Dan apa saja yang dilakukan tanpa keyakinan adalah dosa.
Rom 15:1  Kita yang sungguh-sungguh yakin akan apa yang kita percayai, haruslah bersabar terhadap keberatan-keberatan orang yang lemah keyakinannya. Janganlah kita mau menyenangkan diri kita sendiri saja.
Rom 15:2  Sebaliknya kita masing-masing harus menyenangkan hati sesama saudara kita untuk kebaikannya, supaya keyakinannya bertambah kuat.
Rom 15:3  Sebab Kristus pun tidak memikirkan kesenangan diri-Nya sendiri. Di dalam Alkitab tertulis begini, "Segala celaan yang ditujukan kepada-Mu telah jatuh ke atasku."
Rom 15:4  Semua yang tersurat di dalam Alkitab adalah untuk mengajar kita. Sebab pelajaran yang kita terima dari Alkitab menjadikan kita tabah dan kuat sehingga kita dapat berharap kepada Allah.
Rom 15:5  Semoga Allah, yang memberikan ketabahan dan penghiburan kepada manusia, menolong kalian untuk hidup dengan sehati, menuruti teladan Kristus Yesus.
Rom 15:6  Dengan sehati Saudara semuanya bersama-sama dapat memuji Allah, Bapa Tuhan kita Yesus Kristus.
Rom 15:7  Sebab itu hendaklah Saudara-saudara menerima satu sama lain dengan senang hati, sama seperti Kristus juga menerima kalian untuk memuliakan Allah.
Rom 15:8  Sebab, ingatlah, bahwa Kristus sudah melayani orang Yahudi untuk menunjukkan bahwa Allah setia dalam memenuhi janji-janji-Nya kepada nenek moyang kita.
Rom 15:9  Yesus juga membuat bangsa-bangsa lain memuliakan Allah karena kebaikan hati-Nya kepada mereka. Di dalam Alkitab tertulis begini, "Sebab itu aku akan memuji Engkau di antara bangsa-bangsa, aku akan menyanyikan pujian untuk nama-Mu."
Rom 15:10  Ada tertulis begini pula, "Bergembiralah bersama umat Allah yang terpilih, hai bangsa-bangsa!"
Rom 15:11  Dan ini juga, "Pujilah Allah, hai semua bangsa, hendaklah semua orang memuji Dia!"
Rom 15:12  Dan ini pula yang dikatakan oleh Yesaya, "Dari keturunan Isai akan muncul seseorang yang akan memerintah bangsa-bangsa; kepada Dialah bangsa-bangsa itu akan berharap."
Rom 15:13  Allah adalah tempat kalian berharap. Semoga Ia mengisi hatimu dengan segala sukacita dan sejahtera karena kalian percaya kepada-Nya, supaya dengan kuasa Roh Allah kalian makin berharap kepada Allah.
Rom 15:14  Saudara-saudaraku! Saya yakin bahwa ada banyak sekali hal yang baik padamu. Kalian pun mengetahui segala-galanya dan dapat mengajar satu sama lain juga.
Rom 15:15  Tetapi mengenai hal-hal tertentu, saya memperingatkan kalian dengan tegas dalam surat ini, karena Allah sudah menganugerahi saya tugas itu.
Rom 15:16  Ia sudah menjadikan saya pelayan Kristus Yesus untuk diutus kepada bangsa-bangsa yang bukan Yahudi. Dan saya bertindak sebagai imam, yang mengabarkan Kabar Baik dari Allah, supaya orang-orang yang bukan Yahudi menjadi suatu persembahan kepada Allah yang dapat diterima oleh-Nya. Semoga Roh Allah membuat mereka menjadi suatu persembahan yang khusus untuk Allah.
Rom 15:17  Jadi, karena saya sudah bersatu dengan Kristus Yesus, maka saya boleh merasa bangga atas pekerjaan saya bagi Allah.
Rom 15:18  Saya berani berbicara begitu hanya mengenai apa yang sudah dilakukan Kristus melalui saya untuk menjadikan orang-orang yang bukan Yahudi taat kepada Allah. Saya melakukan itu dengan kata-kata maupun dengan perbuatan,
Rom 15:19  yang disertai keajaiban-keajaiban dan hal-hal luar biasa; semuanya itu oleh kuasa Roh Allah. Kabar Baik tentang Kristus sudah saya beritakan seluruhnya kepada orang-orang, mulai dari Yerusalem sampai ke Ilirikum.
Rom 15:20  Cita-cita saya ialah memberitakan Kabar Baik dari Allah di tempat-tempat di mana orang belum mendengar tentang Kristus. Sebab saya tidak mau membangun pekerjaan pada pondasi yang diletakkan oleh orang lain.
Rom 15:21  Di dalam Alkitab tertulis, "Orang-orang yang belum pernah menerima berita tentang Dia, akan melihat; dan mereka yang belum pernah mendengar, akan mengerti."
Rom 15:22  Itu sebabnya sudah sering saya terhalang untuk mengunjungi kalian.
Rom 15:23  Tetapi sekarang pekerjaan saya di daerah-daerah itu sudah selesai. Bertahun-tahun lamanya saya ingin mengunjungi kalian.
Rom 15:24  Nanti kalau saya lewat kotamu dalam perjalanan saya ke Spanyol, saya harap dapat bertemu dengan kalian dan bergembira sebentar dengan kalian. Sesudah itu saya harap kalian dapat membantu saya untuk perjalanan saya ke Spanyol itu.
Rom 15:25  Tetapi sekarang ini saya mau ke Yerusalem untuk membawa bantuan bagi umat Allah di sana.
Rom 15:26  Sebab jemaat-jemaat di Makedonia dan Akhaya sudah menyetujui untuk memberikan sumbangan kepada saudara-saudara yang miskin di antara umat Allah di Yerusalem.
Rom 15:27  Dengan senang hati mereka memutuskan untuk berbuat itu. Memang sebenarnya mereka patut menolong saudara-saudara yang miskin di Yerusalem itu; sebab dari orang-orang Yahudilah orang-orang bukan Yahudi itu sudah menerima berkat-berkat dari Allah. Jadi, orang-orang bukan Yahudi itu patut juga menolong orang-orang Yahudi dengan berkat-berkat kebendaan.
Rom 15:28  Kalau saya sudah menyelesaikan tugas saya ini, dan menyerahkan kepada mereka yang di Yerusalem itu sumbangan yang sudah dikumpulkan itu, maka saya akan berangkat ke negeri Spanyol melewati kotamu.
Rom 15:29  Kalau saya datang padamu, pasti saya datang dengan banyak berkat dari Kristus untuk kalian.
Rom 15:30  Demi Tuhan kita Yesus Kristus, dan demi kasih yang diberikan oleh Roh Allah, saya minta dengan sangat kepadamu, semoga kalian turut berdoa sungguh-sungguh bersama-sama saya kepada Allah untuk saya.
Rom 15:31  Berdoalah supaya Allah melepaskan saya dari tangan orang-orang yang tidak percaya di Yudea, dan supaya tugas saya untuk membawa sumbangan ke Yerusalem diterima dengan senang hati oleh umat Allah di sana.
Rom 15:32  Dengan demikian, kalau Allah menghendaki, saya dapat datang padamu dengan hati yang senang. Dan saya akan merasa terhibur di tengah-tengah kalian.
Rom 15:33  Mudah-mudahan Allah, yang memberikan sejahtera kepada manusia, menyertai Saudara semuanya. Amin.
Rom 16:1  Saya ingin memperkenalkan saudari kita, Febe. Ia membantu jemaat di Kengkrea.
Rom 16:2  Terimalah dia sebagai salah seorang dari umat Tuhan. Memang demikianlah orang-orang yang percaya kepada Allah harus menerima satu sama lain. Tolonglah dia, kalau ia memerlukan bantuan dari kalian, karena ia sendiri sudah membantu banyak orang--termasuk saya.
Rom 16:3  Sampaikan salam saya kepada Priskila dan Akwila, rekan-rekan saya yang sama-sama bekerja untuk Kristus Yesus.
Rom 16:4  Mereka hampir mati karena mau menyelamatkan saya. Saya berterima kasih kepada mereka--dan bukan saya saja, tetapi semua jemaat bangsa lain yang bukan Yahudi juga berterima kasih kepada mereka.
Rom 16:5  Sampaikanlah juga salam saya kepada jemaat yang berkumpul di rumah mereka. Salam kepada Epenetus yang saya kasihi. Ialah orang yang pertama-tama percaya kepada Kristus di daerah Asia.
Rom 16:6  Salam saya juga kepada Maria. Ia sudah bekerja begitu keras untuk kalian.
Rom 16:7  Sampaikan juga salam saya kepada Andronikus dan Yunias. Mereka sesuku bangsa dengan saya dan pernah masuk penjara bersama-sama saya. Sebelum saya percaya kepada Kristus, mereka sudah lebih dahulu percaya. Mereka terkenal di antara para rasul.
Rom 16:8  Salam kepada Ampliatus, yang saya kasihi karena dia sama-sama dengan saya sudah bersatu dengan Tuhan.
Rom 16:9  Sampaikan hormat saya kepada Urbanus, rekan kita yang bekerja bersama-sama dengan kita untuk Kristus. Juga hormat saya kepada Stakhis yang saya kasihi.
Rom 16:10  Salam kepada Apeles. Kesetiaannya kepada Kristus sudah terbukti. Salam juga kepada keluarga Aristobulus,
Rom 16:11  dan kepada Herodion. Ia sesuku bangsa dengan saya. Juga salam kepada saudara-saudara seiman dalam keluarga Narkisus.
Rom 16:12  Salam saya kepada Trifena dan Trifosa yang bekerja keras melayani Tuhan, dan kepada Persis yang saya kasihi. Dia juga sudah bekerja keras untuk Tuhan.
Rom 16:13  Salam kepada Rufus dan ibunya, yang bagi saya seperti ibu saya sendiri. Rufus adalah orang pilihan Tuhan yang terpuji.
Rom 16:14  Sampaikanlah salam saya kepada Asinkritus, Flegon, Hermes, Patrobas, Hermas, dan semua saudara seiman yang bersama-sama dengan mereka.
Rom 16:15  Salam kepada Filologus dan Yulia; kepada Nereus dan saudara perempuannya; kepada Olimpas dan kepada semua orang percaya yang bersama-sama dengan mereka.
Rom 16:16  Bersalam-salamanlah secara mesra sebagai saudara Kristen. Semua jemaat Kristus menyampaikan salam mereka kepada Saudara-saudara.
Rom 16:17  Saudara-saudara! Saya mohon dengan sangat: Hati-hatilah terhadap orang-orang yang menimbulkan perpecahan dan mengacaukan kepercayaan orang kepada Tuhan. Hal itu bertentangan dengan apa yang sudah diajarkan kepadamu. Jauhilah orang-orang seperti itu.
Rom 16:18  Sebab orang-orang yang berbuat begitu bukannya bekerja untuk Kristus Tuhan kita, melainkan untuk memuaskan keinginan hati mereka sendiri. Dengan bujukan dan kata-kata yang manis, mereka menipu orang-orang yang tidak tahu apa-apa.
Rom 16:19  Semua orang sudah mengetahui bahwa kalian setia kepada pengajaran tentang Kabar Baik yang dari Allah. Dan saya senang atas hal itu mengenai kalian. Saya ingin supaya kalian bijaksana mengenai hal-hal yang baik, dan bodoh mengenai hal-hal yang jahat.
Rom 16:20  Allah yang memberi sejahtera kepada manusia, akan segera menempatkan Iblis di bawah kekuasaanmu dan menghancurkannya. Semoga Saudara-saudara selalu diberkati oleh Yesus Kristus Tuhan kita!
Rom 16:21  Timotius, rekan saya, menyampaikan salamnya kepada kalian. Begitu juga Lukius, Yason dan Sosipater, yang sesuku bangsa dengan saya.
Rom 16:22  Yang menulis surat ini adalah saya, Tertius. Saya mengucapkan salam persaudaraan kepada kalian.
Rom 16:23  Gayus mengirim salam kepadamu. Saya menumpang di rumahnya dan jemaat kami juga berbakti di situ. Erastus, kepala keuangan kota, mengirim salamnya kepadamu. Begitu juga saudara kita Kwartus.
Rom 16:24  (Semoga Tuhan kita Yesus Kristus memberkati Saudara semua. Amin.)
Rom 16:25  Terpujilah Allah! Ia berkuasa menguatkan iman Saudara sesuai dengan Kabar Baik dari Allah yang saya beritakan tentang Yesus Kristus. Saya memberitakan kabar itu sesuai dengan apa yang dinyatakan oleh Allah; yaitu tentang rencana Allah yang sudah berabad-abad lamanya tidak diketahui orang.
Rom 16:26  Tetapi sekarang, atas perintah Allah yang abadi, rencana itu sudah dinyatakan dan diberitahukan kepada semua bangsa melalui tulisan-tulisan para nabi, supaya mereka semuanya percaya dan taat kepada Allah.
Rom 16:27  Dialah satu-satunya Allah, Ia mahabijaksana. Semoga Ia dijunjung tinggi selama-lamanya melalui Yesus Kristus! Amin. Hormat kami, Paulus, hamba Kristus Yesus.


\end{document}