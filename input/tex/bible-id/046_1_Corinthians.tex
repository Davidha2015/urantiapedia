\begin{document}

\title{1 Korintus}


\chapter{1}

\par 1 Saudara-saudara jemaat Allah di Korintus! Saudara sudah dipanggil oleh Allah untuk menjadi umat-Nya karena Saudara milik Kristus Yesus, bersama-sama semua orang di mana saja yang menyembah Tuhan kita Yesus Kristus, yaitu Tuhan mereka dan Tuhan kita juga. Saya, Paulus, dan saudara kita Sostenes, mengharap semoga Allah Bapa kita dan Tuhan Yesus Kristus memberi berkat dan sejahtera kepadamu. Saya menulis surat ini sebagai rasul Kristus Yesus, yang diangkat oleh Allah sendiri dan atas kehendak-Nya juga.
\par 2 [1:1]
\par 3 [1:1]
\par 4 Saya selalu berterima kasih kepada Allah mengenai kalian, sebab kalian sudah menerima rahmat dari Allah melalui Kristus Yesus.
\par 5 Karena kalian sudah menjadi milik Kristus, maka hidupmu kaya dalam segala hal. Pengetahuanmu tentang segala sesuatu sangat dalam, dan kalian pandai mengajar pengetahuan itu.
\par 6 Itu menunjukkan bahwa berita tentang Kristus sudah begitu meresap ke dalam hatimu,
\par 7 sehingga kalian tidak kekurangan satu berkat pun, sementara kalian menunggu Tuhan kita Yesus Kristus datang dan dilihat oleh semua orang.
\par 8 Kristus sendiri akan menjamin kalian sampai pada akhirnya; supaya pada waktu Ia datang kembali, kalian didapati tanpa cela.
\par 9 Allah dapat dipercayai sepenuhnya. Ialah Allah yang sudah memanggil kalian untuk menjadi satu dengan Anak-Nya, yaitu Yesus Kristus, Tuhan kita.
\par 10 Saudara-saudara! Atas nama Tuhan kita Yesus Kristus, saya minta supaya kalian semuanya seia sekata; supaya jangan ada perpecahan di antaramu. Hendaklah kalian bersatu, sehati dan sepikir.
\par 11 Sebab, orang-orang dari keluarga Kloe melaporkan kepada saya bahwa ada pertengkaran di antaramu.
\par 12 Yang saya maksudkan ialah bahwa di antaramu ada yang berkata, "Saya ikut Paulus," ada juga yang berkata, "Saya ikut Apolos," yang lain berkata, "Saya ikut Petrus," sedangkan yang lain lagi berkata, "Saya ikut Kristus."
\par 13 Masakan Kristus terbagi-bagi! Paulus tidak mati disalib untukmu! Kalian pun tidak dibaptis untuk menjadi pengikut-pengikut Paulus, bukan?
\par 14 Syukurlah saya tidak membaptis seorang pun dari antaramu, selain Krispus dan Gayus.
\par 15 Jangan sampai seorang pun berkata bahwa saya sudah membaptis dia untuk menjadi pengikut saya.
\par 16 (Oh ya, Stefanus dan keluarganya, memang saya yang membaptis mereka. Tetapi selain dari itu, seingat saya, tidak ada lagi orang lain yang saya baptis.)
\par 17 Kristus mengutus saya bukan untuk membaptis orang, melainkan untuk memberitakan Kabar Baik dari Allah; dan itu pun harus saya lakukan tanpa memakai kepandaian berbicara secara manusia, agar kuasa dari kematian Kristus pada salib tidak menjadi sia-sia.
\par 18 Sebab bagi orang-orang yang menuju kebinasaan, berita tentang kematian Kristus pada salib merupakan omong kosong. Tetapi, bagi kita yang diselamatkan oleh Allah, berita itu merupakan caranya Allah menunjukkan kuasa-Nya.
\par 19 Sebab dalam Alkitab, Allah berkata, "Kebijaksanaan orang arif akan Kukacaukan, dan pengertian orang-orang berilmu akan Kulenyapkan."
\par 20 Nah, apa gunanya orang-orang arif itu? Apa gunanya mereka yang berilmu? Apa gunanya ahli-ahli pikir dunia ini? Allah sudah menunjukkan bahwa kebijaksanaan dunia ini adalah omong kosong belaka!
\par 21 Karena bagaimanapun pandainya manusia, ia tidak dapat mengenal Allah melalui kepandaiannya sendiri. Tetapi justru karena Allah bijaksana, maka Ia berkenan menyelamatkan orang-orang yang percaya kepada-Nya melalui berita yang kami wartakan yang dianggap omong kosong oleh dunia.
\par 22 Orang Yahudi menuntut keajaiban sebagai bukti, dan orang Yunani mementingkan kebijaksanaan dunia ini.
\par 23 Tetapi kita ini hanya memberitakan Kristus yang mati disalib. Berita itu menyinggung perasaan orang Yahudi, dan dianggap omong kosong oleh orang-orang bukan Yahudi.
\par 24 Tetapi bagi orang-orang yang sudah dipanggil oleh Allah--baik orang Yahudi maupun orang bukan Yahudi--berita itu merupakan caranya Allah menunjukkan kuasa dan kebijaksanaan-Nya.
\par 25 Sebab yang nampaknya bodoh pada Allah, adalah lebih bijaksana daripada kebijaksanaan manusia; dan yang nampaknya lemah pada Allah, adalah lebih kuat daripada kekuatan manusia.
\par 26 Saudara-saudara! Coba ingat bagaimana keadaanmu pada waktu Allah memanggil kalian. Cuma sedikit saja dari antaramu yang bijaksana, atau berkuasa, atau berkedudukan tinggi menurut pandangan manusia.
\par 27 Sebab memang Allah sengaja memilih yang dianggap bodoh oleh dunia ini, supaya orang-orang pandai menjadi malu. Dan Allah memilih juga yang dianggap lemah oleh dunia ini, supaya orang-orang yang gagah perkasa menjadi malu.
\par 28 Allah memilih yang dianggap rendah, hina, dan malah yang dianggap tidak berarti oleh dunia ini, supaya Allah menghancurkan yang dianggap penting oleh dunia.
\par 29 Dengan demikian tidak seorang pun dapat menyombongkan diri di hadapan Allah.
\par 30 Allah sendirilah yang membuat sehingga Saudara bersatu dengan Kristus Yesus. Melalui Kristus, kita dijadikan bijaksana. Dan melalui Dia juga Allah membuat kita berbaik kembali dengan diri-Nya, menjadikan kita umat-Nya yang khusus, dan membebaskan kita.
\par 31 Jadi, seperti yang tertulis dalam Alkitab, "Orang yang mau berbangga-bangga, harus berbangga atas apa yang dilakukan Tuhan."

\chapter{2}

\par 1 Saudara-saudara! Pada waktu saya datang kepadamu dan memberitakan kepadamu tentang rencana Allah yang belum diketahui oleh dunia, saya tidak memakai kebijaksanaan dunia ini atau berbicara dengan kata yang muluk-muluk.
\par 2 Sebab saya sudah bertekad bahwa selama saya ada bersama kalian, saya tidak akan mengemukakan apa-apa, selain mengenai Yesus Kristus; khususnya bahwa Ia sudah mati disalib.
\par 3 Ketika saya berada dengan kalian, saya lemah dan gemetar ketakutan.
\par 4 Berita yang saya sampaikan kepadamu tidak saya sampaikan dengan kata-kata yang memikat menurut kebijaksanaan manusia. Saya menyampaikan itu dengan cara yang membuktikan bahwa Roh Allah berkuasa.
\par 5 Saya berbuat begitu, supaya kepercayaanmu kepada Kristus tidak berdasarkan kebijaksanaan manusia, melainkan berdasarkan kuasa Allah.
\par 6 Walaupun begitu, di antara orang-orang yang kehidupan rohaninya sudah matang, saya memang berbicara mengenai kebijaksanaan. Tetapi kebijaksanaan ini bukanlah kebijaksanaan dunia, atau kebijaksanaan penguasa-penguasa zaman ini, yang kekuasaannya akan lenyap.
\par 7 Kebijaksanaan yang saya kemukakan itu ialah kebijaksanaan dari Allah. Kebijaksanaan itu tidak diketahui oleh dunia, tetapi Allah sudah menyediakannya untuk kebahagiaan kita sebelum dunia ini dijadikan.
\par 8 Tidak seorang pun dari penguasa-penguasa zaman ini yang tahu tentang kebijaksanaan itu. Sebab, seandainya mereka sudah mengetahuinya, tentu mereka tidak akan menyalibkan Tuhan yang mulia itu.
\par 9 Sebaliknya hal itu adalah seperti yang tertulis dalam Alkitab, "Apa yang tidak pernah dilihat atau didengar oleh manusia, dan tidak pernah pula timbul dalam pikiran manusia, itulah yang disediakan Allah untuk orang-orang yang mengasihi-Nya."
\par 10 Tetapi Allah sudah menyatakannya kepada kita dengan perantaraan Roh-Nya. Roh Allah itu menyelidiki segala sesuatu, sampai kepada rencana-rencana Allah yang paling tersembunyi sekalipun.
\par 11 Sebab yang mengetahui isi hati seseorang adalah roh orang itu sendiri, bukan? Begitu juga mengenai Allah. Yang mengetahui isi hati Allah hanyalah Roh Allah sendiri!
\par 12 Dan kita tidak diberi roh dunia ini, melainkan Roh yang dari Allah, supaya kita mengetahui semua yang telah dianugerahkan Allah kepada kita.
\par 13 Sebab itu, bila kami menjelaskan hal-hal mengenai Allah kepada orang-orang yang mempunyai Roh Allah, kami tidak mengemukakannya menurut kebijaksanaan manusia, melainkan menurut ajaran Roh Allah.
\par 14 Orang yang tidak mempunyai Roh Allah, tidak dapat menerima apa yang dinyatakan oleh Roh itu. Sebab bagi orang itu hal-hal tersebut seperti suatu kebodohan saja. Orang itu tidak dapat mengertinya, sebab hal-hal itu hanya dapat dinilai secara rohani.
\par 15 Orang yang mempunyai Roh Allah dapat menilai segala sesuatu, tetapi tidak seorang pun berhak menilai Dia.
\par 16 Dalam Alkitab tertulis, "Siapakah yang tahu pikiran Tuhan? Dan siapakah dapat menasihati-Nya?" Kitalah yang mempunyai pikiran yang sama dengan Kristus!

\chapter{3}

\par 1 Saudara-saudara, sebenarnya saya tidak dapat berbicara dengan Saudara seperti dengan orang yang mempunyai Roh Allah. Saya hanya dapat berbicara denganmu seperti dengan orang yang masih hidup menurut keinginan duniawi; seperti dengan orang yang masih bayi dalam kepercayaannya kepada Kristus.
\par 2 Dahulu saya hanya dapat memberikan kepadamu makanan bayi, bukan makanan orang dewasa; sebab kalian belum cukup kuat untuk itu. Tetapi sekarang pun kalian masih belum kuat untuk itu,
\par 3 karena kalian masih hidup menurut tabiatmu secara manusia. Sebab kalau kalian masih iri hati dan berkelahi satu sama lain, bukankah itu menunjukkan bahwa kalian masih hidup menurut tabiat manusia, seperti orang-orang yang tidak mengenal Allah?
\par 4 Kalau ada yang berkata, "Kami ikut Paulus," dan yang lain berkata, "Saya ikut Apolos," bukankah itu menunjukkan bahwa Saudara berkelakuan seperti orang-orang dunia?
\par 5 Sebenarnya siapakah Apolos itu? Dan siapakah Paulus? Kami hanya pelayan-pelayan Allah, yang sudah membimbing kalian untuk percaya kepada Kristus. Kami hanya menjalankan pekerjaan yang ditugaskan Tuhan kepada kami masing-masing.
\par 6 Saya menanam dan Apolos menyiram, tetapi Allah sendirilah yang membuat tanamannya tumbuh.
\par 7 Jadi yang penting adalah Allah, sebab Dialah yang menumbuhkan. Yang menanam dan yang menyiram tidak penting;
\par 8 keduanya adalah sederajat. Masing-masing akan menerima upah menurut jerih payahnya.
\par 9 Kami adalah orang-orang yang sama-sama bekerja untuk Allah; dan kalian adalah seperti ladang Allah. Saudara-saudara adalah seperti gedung Allah juga.
\par 10 Dengan kepandaian yang diberikan Allah, saya sebagai ahli bangunan sudah meletakkan pondasi untuk gedung tersebut, dan orang lain membangun gedung di atas pondasi itu. Setiap orang harus memperhatikan baik-baik bagaimana ia membangun di atas pondasi itu.
\par 11 Sebab Allah sendiri sudah menempatkan Yesus Kristus sebagai satu-satunya pondasi untuk gedung itu; tidak ada pondasi yang lain.
\par 12 Ada yang membangun di atas pondasi itu dengan memakai emas, ada yang memakai perak, ada yang memakai batu permata, ada pula yang memakai kayu, rumput kering ataupun jerami.
\par 13 Pekerjaan setiap orang akan kelihatan nanti pada saat Kristus datang kembali. Sebab pada hari itu api akan membuat pekerjaan masing-masing orang kelihatan. Api akan menguji dan menentukan mutu dari pekerjaan itu.
\par 14 Kalau gedung yang didirikan orang di atas pondasi itu tahan bakaran api, orang itu akan menerima hadiahnya.
\par 15 Tetapi kalau pekerjaan orang terbakar, ia akan rugi; ia sendiri akan selamat, tetapi seperti orang yang lolos menerusi api.
\par 16 Tahukah Saudara bahwa kalian adalah Rumah Allah? Dan bahwa Roh Allah tinggal di dalam kalian?
\par 17 Kalau ada orang yang merusak Rumah Allah, Allah pun akan merusak orang itu. Sebab Rumah Allah adalah khusus untuk Allah saja, dan kalianlah rumah itu.
\par 18 Janganlah seorang pun menipu dirinya sendiri. Kalau ada orang di antaramu merasa dirinya bijaksana menurut ukuran dunia ini, orang itu harus menjadi bodoh, supaya ia menjadi benar-benar bijaksana.
\par 19 Sebab yang dianggap bijaksana oleh dunia adalah bodoh pada pemandangan Allah. Dalam Alkitab tertulis, "Allah menjebak orang-orang bijaksana dalam kecerdikan mereka sendiri."
\par 20 Ada tertulis begini juga, "Tuhan tahu bahwa pikiran orang-orang bijaksana adalah pikiran yang tidak berguna."
\par 21 Karena itu, janganlah seorang pun menyanjung-nyanjung manusia, sebab segala-galanya sudah diberikan menjadi milikmu:
\par 22 Paulus, Apolos, Petrus, dunia ini, kehidupan dan kematian, zaman ini ataupun zaman yang akan datang, segala-galanya adalah kepunyaanmu.
\par 23 Dan kalian adalah kepunyaan Kristus, dan Kristus kepunyaan Allah.

\chapter{4}

\par 1 Anggaplah kami pelayan-pelayan Kristus, yang bertanggung jawab memberitakan rencana-rencana Allah yang belum diketahui dunia.
\par 2 Yang pertama-tama dituntut dari pelayan yang demikian adalah bahwa ia setia kepada tuannya.
\par 3 Bagi saya, tidak menjadi soal apa yang kalian--atau siapa pun--juga pikirkan tentang diri saya. Malah apa yang saya sendiri pikirkan mengenai diri saya sendiri pun tidak menjadi soal.
\par 4 Saya tidak merasa bersalah dalam hal apa pun juga, tetapi itu bukan bukti bahwa saya memang tidak bersalah. Tuhan sendirilah yang menentukan saya bersalah atau tidak.
\par 5 Sebab itu, karena belum waktunya, janganlah kalian cepat-cepat menentukan bahwa seseorang bersalah atau tidak. Tunggulah sampai Tuhan datang nanti. Ialah yang akan membuka semua rahasia-rahasia yang tersembunyi dalam kegelapan. Ialah yang akan membongkar semua niat yang terpendam di dalam hati manusia. Pada waktu itu barulah setiap orang menerima dari Allah pujian yang patut diterimanya.
\par 6 Saudara-saudara! Untuk kepentinganmu, saya sudah mengenakan semuanya itu kepada Apolos dan kepada diri saya sendiri. Maksudnya supaya dari contoh kami itu kalian dapat belajar apa arti ungkapan, "Berpeganglah pada peraturan yang ada." Dengan demikian tidak seorang pun dari kalian dapat membangga-banggakan seseorang, lalu menghina yang lain.
\par 7 Siapakah yang menjadikan Saudara lebih dari orang lain? Bukankah segala sesuatu Saudara terima dari Allah? Jadi, mengapa mau menyombongkan diri, seolah-olah apa yang ada pada Saudara itu bukan sesuatu yang diberi?
\par 8 Memang kalian tidak memerlukan apa-apa lagi! Kalian sudah kaya! Kalian sudah menjadi raja! Dan kami tidak. Alangkah baiknya kalau kalian betul-betul sudah menjadi raja, supaya kami dapat memerintah bersamamu.
\par 9 Karena menurut pendapat saya, kami rasul-rasul, sudah dijadikan oleh Allah sebagai tontonan di depan manusia dan para malaikat. Kami seperti orang-orang hina yang dijatuhi hukuman mati di depan umum dan disorak-soraki oleh dunia.
\par 10 Karena Kristus, kami adalah orang yang bodoh, dan kalian orang Kristen yang pandai! Kami lemah, kalian kuat! Kami dicela, dan kalian disanjung-sanjung!
\par 11 Sampai saat ini kami mengalami kelaparan dan kehausan; pakaian kami tinggal yang di badan saja; orang menyiksa kami; kami tidak punya tempat untuk menetap;
\par 12 kami membanting tulang untuk mencari nafkah. Apabila kami dikutuk, kami membalas dengan berkat; kalau kami dianiaya, kami sabar;
\par 13 kalau orang memburuk-burukkan kami, kami membalas dengan kata-kata yang manis. Kami tidak lebih dari sampah dunia ini; sampai saat ini kami masih dianggap seperti kotoran bumi.
\par 14 Saya menulis ini kepadamu bukanlah untuk membuat kalian menjadi malu, tetapi untuk menasihati kalian seperti anak-anak saya sendiri.
\par 15 Sebab sayalah yang menjadi bapak kalian, walaupun mungkin sebagai orang Kristen, kalian sudah punya sepuluh ribu guru. Di dalam hidupmu sebagai orang-orang yang bersatu dengan Kristus, saya yang menjadi bapak kepadamu, karena saya yang membawa Kabar Baik tentang Kristus kepadamu.
\par 16 Sebab itu saya minta dengan sangat supaya kalian mengikuti contoh saya.
\par 17 Untuk itu saya sudah mengutus Timotius kepadamu. Sebagai pengikut Kristus, ia sama seperti anak saya sendiri yang saya kasihi. Ia anak yang dapat dipercayai. Nanti ia akan mengingatkan kalian mengenai cara hidup yang saya ikuti sesudah percaya kepada Kristus, yaitu cara hidup yang saya ajarkan di setiap jemaat di mana pun juga.
\par 18 Beberapa orang dari antaramu sudah menjadi sombong, sebab mereka menyangka saya tidak akan datang padamu.
\par 19 Tetapi kalau Tuhan mengizinkan saya akan datang juga kepadamu tidak lama lagi. Nanti pada waktu itu saya akan melihat juga apa yang dapat dilakukan oleh orang-orang yang sombong itu; bukan hanya perkataan mereka saja.
\par 20 Karena kalau Allah memerintah hidup seseorang, hal itu dibuktikan oleh kekuatan hidup orang itu, bukan oleh kata-katanya.
\par 21 Jadi pilih saja apa yang kalian lebih sukai! Apakah kalian mau saya datang untuk mengajar kalian dengan keras atau saya datang mengajar dengan lemah lembut dan kasih sayang?

\chapter{5}

\par 1 Saya sebenarnya sudah menerima laporan bahwa di antaramu ada perbuatan-perbuatan cabul yang keterlaluan: Ada dari antara Saudara yang berzinah dengan ibu tirinya. Orang-orang yang tidak mengenal Tuhan pun tidak melakukan seperti itu!
\par 2 Meskipun begitu, Saudara masih tinggi hati. Sebaliknya Saudara seharusnya sedih, dan mengeluarkan dari antara Saudara orang yang melakukan hal itu.
\par 3 Secara lahir saya jauh dari kalian, tetapi secara batin saya berada di tengah-tengah kalian. Jadi, saya mau bertindak seolah-olah saya berada di antaramu, "Atas nama Tuhan Yesus Kristus, saya nyatakan bahwa orang yang sudah melakukan perbuatan yang sebegitu kotor, harus dihukum!" Kalau kalian berkumpul anggaplah bahwa saya hadir di antaramu, dan dengan kuasa Tuhan kita Yesus,
\par 4 [5:3]
\par 5 kalian harus menyerahkan orang itu kepada Iblis supaya tubuh orang itu binasa dan dengan demikian rohnya diselamatkan pada waktu Tuhan datang kembali.
\par 6 Tidak patut kalian merasa sombong. Barangkali kalian sudah mengenal peribahasa ini, "Ragi yang sedikit membuat seluruh adonan mengembang!"
\par 7 Buanglah dahulu ragi yang lama itu, yaitu ragi dosa, supaya kalian menjadi seperti adonan yang baru, bersih dari ragi dosa yang lama, dan saya tahu bahwa kalian memang demikian. Sebab, perayaan Paskah kita sudah siap, karena Kristus yang menjadi sebagai domba Paskah kita, sudah dikurbankan.
\par 8 Jadi, mari kita merayakan pesta kita itu dengan roti yang tidak beragi, yaitu roti yang melambangkan kemurnian dan semua yang berkenan di hati Allah. Jangan kita merayakan pesta Paskah itu dengan roti yang mengandung ragi yang lama, yaitu ragi dosa dan kejahatan.
\par 9 Di dalam surat saya yang lalu, saya memberitahukan kepadamu supaya jangan bergaul dengan orang cabul.
\par 10 Yang saya maksudkan bukanlah orang cabul atau tamak, atau penipu atau penyembah berhala yang sama sekali tidak mengenal Allah. Bukan! Saya tidak maksudkan mereka; sebab kalau kalian harus menjauhi mereka, maka tentunya kalian perlu keluar sama sekali dari dunia ini.
\par 11 Maksud saya ialah, bahwa kalian jangan bergaul dengan orang yang mengaku dirinya orang Kristen, tetapi orang itu cabul, atau tamak, atau penyembah berhala, atau suka memburuk-burukkan orang lain, atau pemabuk, ataupun pencuri. Duduk makan dengan orang itu pun jangan!
\par 12 Memang mengadili orang-orang bukan Kristen bukanlah urusan saya. Allah sendirilah yang akan mengadili mereka. Tetapi mengenai anggota-anggota jemaatmu, bukankah kalian sendiri yang harus mengadili mereka? Dalam Alkitab tertulis, "Usirlah orang jahat dari antaramu."

\chapter{6}

\par 1 Mengapa ada di antara kalian yang berani pergi kepada hakim yang tidak mengenal Allah, untuk mengadukan saudara seiman, kalau ia bersengketa dengan saudaranya itu? Mengapa ia tidak minta umat Allah menyelesaikan perkara itu?
\par 2 Apakah kalian tidak tahu bahwa umat Allah akan mengadili dunia ini? Jadi, kalau memang kalian akan mengadili dunia ini, apakah kalian tidak sanggup menyelesaikan perkara-perkara pengadilan yang tak berarti?
\par 3 Apakah kalian tidak tahu bahwa kita akan mengadili malaikat-malaikat? Lebih-lebih lagi perkara-perkara kehidupan sehari-hari!
\par 4 Maka kalau timbul perkara-perkara seperti itu, apakah kalian mengajukannya kepada orang-orang yang tidak punya kedudukan sama sekali dalam jemaat?
\par 5 Sungguh memalukan! Tentu di antaramu ada seseorang yang cukup bijaksana untuk menyelesaikan perselisihan antara saudara-saudara yang sama-sama Kristen!
\par 6 Tetapi sebaliknya seorang Kristen pergi kepada orang bukan Kristen untuk mengadukan perkaranya terhadap saudaranya yang Kristen!
\par 7 Jangankan pergi berperkara pada orang bukan Kristen; adanya perselisihan-perselisihan di antaramu pun sudah merupakan suatu kekalahan bagimu. Lebih baik kalian diperlakukan tidak adil, atau dirugikan!
\par 8 Tetapi kalian sendiri malah yang memperlakukan orang lain dengan tidak adil. Kalian sendiri juga yang merugikan orang lain. Dan itu kalian lakukan justru terhadap saudara-saudara seiman!
\par 9 Tahukah kalian bahwa orang-orang yang tidak menuruti kemauan Allah, tidak akan menjadi anggota umat Allah? Jangan tertipu! Orang-orang yang berbuat cabul, orang-orang yang menyembah berhala, yang berzinah, yang melakukan perbuatan yang memalukan terhadap sesama jenisnya,
\par 10 yang mencuri, yang serakah, yang pemabuk, yang suka memburuk-burukkan orang lain, dan yang memeras orang lain--semua orang seperti itu tidak akan menjadi anggota umat Allah.
\par 11 Beberapa di antaramu dahulu berkelakuan seperti itu. Tetapi sekarang kalian dinyatakan bersih dari dosa. Kalian sudah menjadi milik Allah yang khusus. Kalian sudah berbaik kembali dengan Allah, karena kalian percaya kepada Tuhan Yesus Kristus dan karena kuasa Roh dari Allah kita.
\par 12 Ada yang berkata bahwa setiap orang boleh melakukan segala sesuatu. Tetapi bagi saya tidak semuanya berguna. Jadi meskipun saya boleh melakukan apa saja, tetapi saya tidak mau membiarkan diri saya dikuasai oleh apa pun.
\par 13 Ada juga yang berkata bahwa makanan adalah untuk perut, dan perut disediakan untuk makanan. Tetapi kedua-duanya akan ditiadakan oleh Allah. Tubuh manusia tidak boleh dipakai untuk mengadakan hubungan yang cabul, tetapi hanya untuk melayani Tuhan. Dan Tuhan akan memelihara tubuh itu.
\par 14 Allah sudah menghidupkan Tuhan Yesus dari kematian, begitu juga Allah akan menghidupkan kita dengan kuasa-Nya.
\par 15 Saudara-saudara tahu bahwa tubuhmu adalah anggota tubuh Kristus. Jadi, apakah boleh saya mengambil satu anggota tubuh Kristus lalu menjadikannya anggota tubuh seorang pelacur? Sekali-kali tidak!
\par 16 Atau barangkali kalian belum tahu bahwa orang yang bercampur dengan seorang pelacur, menjadi satu dengan pelacur itu! Sebab dalam Alkitab tertulis, "Keduanya menjadi satu."
\par 17 Tetapi orang yang menyatukan dirinya dengan Tuhan, Roh Tuhan dengan roh orang itu menjadi satu.
\par 18 Jauhilah perbuatan-perbuatan yang cabul. Sebab semua dosa lain yang dilakukan orang, terjadi di luar tubuh orang itu. Tetapi orang yang berbuat yang cabul, berarti berbuat dosa terhadap tubuhnya sendiri.
\par 19 Kalian harus tahu bahwa tubuhmu adalah tempat tinggal Roh Allah. Roh itu tinggal di dalam kalian. Dan Allah sendirilah yang memberikan Roh itu kepadamu. Diri Saudara bukan kepunyaanmu. Itu kepunyaan Allah.
\par 20 Allah sudah membeli kalian dan sudah membayarnya dengan lunas. Sebab itu, pakailah tubuhmu sedemikian rupa sehingga Allah dimuliakan.

\chapter{7}

\par 1 Sekarang saya mau bicara mengenai masalah yang kalian sebut dalam suratmu. Kalau seorang laki-laki tidak kawin, itu baik.
\par 2 Tetapi supaya tidak tergoda untuk berbuat hal-hal yang tidak patut, lebih baik setiap orang laki-laki mempunyai istrinya sendiri dan setiap wanita mempunyai suaminya sendiri.
\par 3 Suami harus memenuhi kewajibannya sebagai suami terhadap istrinya, dan istri harus memenuhi kewajibannya sebagai istri terhadap suaminya; masing-masing memenuhi kewajibannya terhadap yang lain.
\par 4 Istri tidak berkuasa atas tubuhnya sendiri; yang berkuasa atas tubuhnya adalah suaminya. Begitu juga suami tidak berkuasa atas tubuhnya sendiri; yang berkuasa atas tubuhnya adalah istrinya.
\par 5 Janganlah menjauhi satu sama lain secara suami istri. Boleh untuk sementara waktu, asal dua-duanya sama-sama sudah setuju. Dengan demikian masing-masing dapat berdoa dengan tidak terganggu. Tetapi kemudian, haruslah kalian kembali saling mendekati secara suami istri. Kalau tidak begitu, nanti kalian bisa menuruti bujukan roh jahat, karena kalian tidak kuat menahan nafsu.
\par 6 Saya mengatakan ini bukan sebagai perintah, tetapi sebagai nasihat.
\par 7 Sebenarnya saya lebih suka kalau semua orang menjadi seperti saya. Tetapi masing-masing sudah menerima karunia yang khusus dari Allah. Seorang mempunyai karunia ini, yang lain mempunyai karunia itu.
\par 8 Kepada orang-orang yang belum kawin dan kepada wanita-wanita yang sudah janda, inilah nasihat saya: Lebih baik Saudara tetap hidup sendiri seperti saya.
\par 9 Tetapi kalau Saudara tidak dapat menahan nafsu, Saudara hendaknya kawin. Sebab lebih baik Saudara kawin daripada nafsu berahimu berkobar-kobar.
\par 10 Terhadap mereka yang sudah kawin, inilah perintah saya: (Sebenarnya bukan saya yang memberi perintah ini, tetapi Tuhan.) Seorang wanita yang sudah kawin janganlah meninggalkan suaminya.
\par 11 Tetapi kalau ia sudah meninggalkannya, ia harus tetap tidak bersuami, atau kembali kepada suaminya. Dan seorang suami tidak boleh menceraikan istrinya.
\par 12 Kepada yang lain-lainnya, nasihat saya ialah: --ini nasihat saya sendiri, bukan Tuhan--kalau seorang Kristen beristrikan seorang wanita yang tidak percaya kepada Kristus, dan istrinya setuju untuk hidup bersama dengan dia, orang itu tidak boleh menceraikan istrinya.
\par 13 Dan kalau seorang wanita Kristen bersuamikan seorang yang tidak percaya kepada Kristus, dan suaminya setuju untuk hidup bersama dengan dia, maka istri itu tidak boleh menceraikan suaminya.
\par 14 Sebab suami yang tidak percaya dilayakkan untuk menjadi anggota umat Allah karena perkawinannya dengan istri yang sudah menjadi milik Allah. Begitu juga istri yang tidak percaya dilayakkan untuk menjadi anggota umat Allah, karena perkawinannya dengan suami yang sudah menjadi milik Allah. Kalau tidak begitu, anak-anak mereka tentunya seperti anak-anak kafir, padahal anak-anak itu dianggap sebagai anggota umat Allah.
\par 15 Tetapi kalau orang yang tidak percaya itu meninggalkan istrinya atau suaminya yang Kristen, jangan menahan dia. Dalam hal ini Saudari atau Saudara itu bebas, sebab Allah mau supaya Saudara hidup dengan rukun.
\par 16 Karena Saudara sebagai istri--yang sudah percaya kepada Tuhan--bagaimanakah Saudara bisa tahu dengan pasti bahwa Saudara tidak dapat menyelamatkan suamimu? Begitu juga Saudara sebagai suami Kristen, bagaimanakah Saudara bisa tahu dengan pasti bahwa Saudara tidak dapat menyelamatkan istrimu?
\par 17 Hendaklah masing-masing mengatur kehidupannya menurut bimbingan Tuhan seperti yang sudah ditentukan oleh Allah baginya pada waktu Allah memanggilnya untuk percaya kepada-Nya. Itulah peraturan yang saya ajarkan di tiap-tiap jemaat.
\par 18 Umpamanya, kalau seseorang sudah disunat pada waktu ia menerima panggilan Allah, maka janganlah orang itu berusaha menghapuskan tanda-tanda sunat itu. Begitu juga kalau seseorang belum disunat pada waktu ia menerima panggilan Allah, janganlah orang itu minta disunat.
\par 19 Sebab mengikuti peraturan sunat atau tidak mengikutinya, kedua-duanya sama-sama tidak berarti apa-apa. Yang penting ialah menuruti perintah-perintah Allah.
\par 20 Hendaklah setiap orang tetap hidup dalam keadaan seperti ketika ia menerima panggilan Allah.
\par 21 Kalau pada waktu Saudara dipanggil, Saudara adalah seorang hamba, tidak usahlah Saudara susah-susah memikirkan hal itu. Tetapi kalau nanti Saudara mendapat kesempatan untuk menjadi bebas, pakailah kesempatan itu.
\par 22 Sebab seorang hamba yang sudah percaya kepada Tuhan, adalah orang Tuhan yang bebas. Dan seorang bebas yang sudah percaya kepada Tuhan, adalah hamba Kristus.
\par 23 Allah sudah membeli Saudara dan sudah lunas membayarnya. Karena itu, janganlah Saudara menyerahkan diri untuk menjadi hamba manusia.
\par 24 Jadi, Saudara-saudara, bagaimanapun keadaanmu pada waktu dipanggil, hendaklah Saudara tetap hidup seperti itu bersama dengan Allah.
\par 25 Sekarang mengenai orang-orang yang belum kawin. Mengenai hal itu, saya tidak menerima perintah apa-apa dari Tuhan. Namun sebagai orang yang karena rahmat Tuhan patut dipercayai, saya mau memberikan nasihat saya.
\par 26 Mengingat segala kesusahan yang mengancam sekarang ini, saya rasa lebih baik kalau orang tetap menjalani hidup seperti keadaannya yang sekarang.
\par 27 Kalau Saudara sudah beristri janganlah berusaha lepas dari istri itu. Kalau Saudara belum beristri tidak usah mencari istri.
\par 28 Tetapi kalau Saudara kawin, itu bukan dosa. Begitu juga kalau seorang gadis kawin, tidak berarti ia berdosa. Hanya, mereka yang kawin itu akan menghadapi banyak kesusahan. Dan saya ingin Saudara terhindar dari kesusahan-kesusahan itu.
\par 29 Maksud saya begini, Saudara-saudara: Kita tidak punya banyak waktu lagi. Mulai dari sekarang, setiap orang yang sudah beristri hendaklah hidup seolah-olah ia tidak beristri;
\par 30 orang yang menangis, hidup seolah-olah ia tidak bersedih hati; orang yang tertawa, seolah-olah ia tidak gembira; orang yang sudah membeli, seolah-olah ia tidak memiliki apa-apa;
\par 31 dan orang yang berkecimpung dalam hal-hal dunia, hendaklah hidup seolah-olah ia tidak disibuki oleh hal-hal itu. Sebab tidak lama lagi dunia ini, dalam keadaannya yang sekarang, akan lenyap!
\par 32 Saya ingin supaya Saudara bebas dari kesusahan. Orang yang tidak beristri akan memusatkan pikirannya pada hal-hal mengenai Tuhan, karena ia ingin menyenangkan Tuhan.
\par 33 Tetapi orang yang sudah beristri akan banyak memikirkan hal-hal dunia ini, sebab ia ingin menyenangkan hati istrinya;
\par 34 akibatnya perhatiannya terbagi-bagi. Seorang wanita yang tidak bersuami, atau seorang anak gadis, akan banyak memikirkan hal-hal mengenai Tuhan, sebab ia ingin supaya jiwa raganya menjadi milik Allah. Tetapi seorang wanita yang sudah bersuami, memusatkan pikirannya pada hal-hal dunia ini, sebab ia ingin menyenangkan hati suaminya.
\par 35 Saya menulis semuanya itu untuk kebaikanmu sendiri, bukan dengan maksud melarang ini dan melarang itu. Yang saya ingini hanyalah supaya Saudara melakukan yang benar dan patut, dan supaya Saudara dapat memusatkan pikiranmu kepada Tuhan.
\par 36 Kalau seseorang merasa tidak menjalankan yang sepatutnya terhadap tunangannya, kalau ia tidak dapat menahan nafsunya, dan ia merasa perlu kawin dengan gadis itu, biarlah ia melakukan apa yang dirasanya baik. Ia tidak berdosa, kalau mereka kawin.
\par 37 Tetapi kalau seseorang sudah membuat keputusan di dalam hatinya untuk tidak kawin dengan tunangannya dan keputusannya itu tidak terpaksa, maka yang dilakukannya itu baik, asal ia kuat melakukannya.
\par 38 Tegasnya, orang yang kawin baik perbuatannya, dan orang yang tidak kawin lebih baik lagi perbuatannya.
\par 39 Seorang wanita yang sudah kawin terikat kepada suaminya hanya selama suaminya hidup. Kalau suaminya sudah meninggal, wanita itu bebas kawin lagi dengan orang yang disukainya; asal perkawinan itu perkawinan Kristen.
\par 40 Namun ia akan lebih beruntung kalau tidak kawin lagi. Itu pendapat saya sendiri, tetapi saya rasa bahwa yang saya ucapkan itu adalah dengan kuasa Roh Allah.

\chapter{8}

\par 1 Sekarang mengenai makanan yang diberikan sebagai persembahan kepada berhala. Memang benar seperti kata orang, "Kita semuanya sudah pandai." Tetapi kepandaian membuat orang menjadi sombong, sedangkan kasih membangun pribadi orang.
\par 2 Orang yang menyangka bahwa ia tahu banyak, sebetulnya belum mengetahui yang sedalam-dalamnya.
\par 3 Tetapi orang yang sungguh-sungguh mengasihi Allah, ia dikenal oleh Allah.
\par 4 Tentang persoalan makan makanan yang sudah dipersembahkan kepada berhala, kita tahu bahwa berhala menggambarkan sesuatu yang sebetulnya tidak ada. Dan kita tahu juga bahwa Allah hanya satu; tidak ada yang lain.
\par 5 Memang banyak juga apa yang dinamakan ilah dan tuhan; baik yang ada di langit maupun yang ada di atas bumi.
\par 6 Tetapi bagi kita, Allah hanya satu. Ia Bapa yang menciptakan segala sesuatu. Untuk Dialah kita hidup. Dan Tuhan hanya satu juga, yaitu Yesus Kristus. Melalui Dia segala sesuatu diciptakan, dan karena Dialah maka kita hidup.
\par 7 Meskipun begitu, tidak semua orang mengetahui hal itu. Ada yang dahulu biasanya menyembah berhala. Jadi, karena itu sampai sekarang pun mereka masih merasa bahwa makanan, yang sudah dipersembahkan kepada berhala, adalah makanan berhala. Maka kalau mereka makan makanan itu, mereka merasa berdosa; karena keyakinan mereka belum kuat.
\par 8 Sebenarnya makanan sendiri tidak membuat hubungan kita dengan Allah menjadi lebih akrab. Kalau kita makan makanan itu, kita tidak mendapat keuntungan apa-apa. Sebaliknya kalau kita tidak makan makanan itu, kita pun tidak rugi apa-apa.
\par 9 Tetapi, hati-hati! Jangan sampai terjadi bahwa orang lain menjadi berdosa--karena keyakinannya belum kuat--oleh sebab Saudara bebas melakukan apa saja.
\par 10 Maksud saya begini: Seandainya Saudara, yang punya keyakinan yang kuat, sedang duduk makan di kuil berhala. Kemudian seseorang melihat Saudara duduk makan di situ. Kalau keyakinan orang itu tidak kuat, bukankah itu akan membuat orang itu berani makan makanan yang sudah diberi kepada berhala itu?
\par 11 Dan karena itu, maka keyakinan Saudara membuat orang yang keyakinannya tidak kuat itu menjadi sesat. Padahal Kristus mati untuk orang itu juga.
\par 12 Dan kalau Saudara melakukan kesalahan seperti itu terhadap saudara-saudara Kristen--yaitu Saudara merusak keyakinan mereka yang lemah--maka Saudara berdosa kepada Kristus.
\par 13 Itu sebabnya, kalau makanan menyebabkan saudara saya berdosa, maka saya sama sekali tidak akan makan daging lagi. Sebab jangan-jangan saudara saya berdosa karena saya.

\chapter{9}

\par 1 Bukankah saya orang bebas? Bukankah saya rasul? Apakah saya belum pernah melihat Yesus, Tuhan kita? Apakah kalian bukan hasil pekerjaan saya untuk Tuhan?
\par 2 Kalaupun orang-orang lain tidak mau mengaku saya sebagai rasul, paling sedikit kalian mengakui itu! Sebab hidupmu sebagai orang Kristen adalah bukti bahwa saya seorang rasul.
\par 3 Kalau orang mengeritik saya, saya akan menjawab begini:
\par 4 Apakah saya tidak berhak menerima makanan dan minuman karena pekerjaan saya?
\par 5 Apakah saya tidak berhak membawa seorang istri Kristen bersama-sama saya dalam perjalanan saya, seperti yang dilakukan oleh saudara-saudara Tuhan Yesus dan oleh rasul-rasul lainnya, termasuk Petrus?
\par 6 Atau cuma Barnabas dan saya saja yang diharuskan mencari nafkah sendiri?
\par 7 Tidak ada anggota tentara yang harus membiayai dirinya sendiri di dalam angkatan perang! Tidak ada petani yang menanam anggur di kebunnya lalu tidak makan hasil anggur dari kebun itu! Tidak ada gembala yang memelihara domba, lalu tidak minum susu dari domba-dombanya itu!
\par 8 Saya mengemukakan itu bukan hanya berdasarkan yang lazim dalam pengalaman sehari-hari, tetapi sebab Alkitab mengemukakan hal itu juga.
\par 9 Di dalam Buku Musa tertulis, "Sapi yang sedang menginjak-injak gandum untuk melepaskan biji gandum dari bulirnya, janganlah diberangus mulutnya." Sapikah yang diperhatikan Allah?
\par 10 Atau itu dimaksudkan untuk kita? Memang itu tertulis untuk kita. Sebab orang yang mengerjakan ladang dan orang yang membersihkan gandumnya sudah selayaknya bekerja dengan harapan untuk menerima sebagian dari hasil pekerjaannya.
\par 11 Kami sudah menabur benih rohani di dalam hidupmu. Maka kalau kami menuai berkat-berkat kebendaan dari kalian, apakah itu berarti kami menuntut terlalu banyak dari kalian?
\par 12 Kalau orang-orang lain berhak mengharapkan ini dari kalian, bukankah kami lebih berhak lagi daripada mereka? Tetapi saya belum menggunakan hak itu. Sebaliknya saya lebih suka menanggung segala sesuatu daripada menimbulkan suatu hal yang menghalangi penyebaran Kabar Baik tentang Kristus.
\par 13 Saudara tentu tahu bahwa orang yang bekerja di dalam Rumah Allah menerima makanan mereka dari Rumah Allah. Dan orang-orang yang mengurus tempat persembahan kurban, mendapat sebagian dari kurban yang dipersembahkan di situ.
\par 14 Begitu juga Tuhan sudah menentukan bahwa orang yang memberitakan Kabar Baik itu harus mendapat nafkahnya dari pemberitaan itu.
\par 15 Namun belum pernah saya memakai satu pun dari hak-hak itu. Dan saya menulis surat ini bukanlah juga dengan maksud supaya hal-hal itu dilakukan sekarang terhadap saya. Lebih baik saya mati daripada kehilangan hal yang saya banggakan itu.
\par 16 Sebab, kalau itu hanya soal memberitakan Kabar Baik dari Allah, maka saya tidak berhak berbangga-bangga, sebab Allah telah memerintahkan saya untuk melakukannya. Celakalah saya, kalau saya tidak memberitakan Kabar Baik itu!
\par 17 Seandainya saya menjalankan pekerjaan itu atas kemauan saya sendiri, maka saya boleh mengharap untuk mendapat upah. Tetapi saya memberitakan Kabar Baik itu justru karena diwajibkan. Saya ditugaskan oleh Allah untuk melakukan itu.
\par 18 Kalau begitu, upah saya apa? Ini upahnya: bahwa saya dapat memberitakan Kabar Baik itu tanpa memberatkan seorang pun untuk membiayai saya, karena saya tidak menuntut hak-hak saya sebagai pemberita Kabar Baik itu.
\par 19 Saya ini bukan hamba siapa pun; saya bebas. Meskipun begitu, saya sudah menjadikan diri saya ini hamba kepada semua orang. Saya lakukan itu supaya saya bisa memenangkan sebanyak mungkin orang untuk Kristus.
\par 20 Terhadap orang Yahudi, saya berlaku sebagai orang Yahudi supaya saya bisa memenangkan orang Yahudi untuk Kristus. Terhadap orang-orang yang hidup menurut hukum Musa, saya berlaku seolah-olah saya terikat pada hukum itu, walaupun saya sebenarnya tidak terikat padanya. Saya lakukan itu supaya saya bisa menarik mereka menjadi pengikut Kristus.
\par 21 Terhadap orang bukan Yahudi, saya berlaku seperti seorang bukan Yahudi, yang hidup di luar hukum Musa. Saya lakukan itu supaya saya bisa menarik mereka menjadi pengikut Kristus. Tetapi itu tidak berarti bahwa saya tidak taat kepada perintah-perintah Allah; saya justru dikuasai oleh perintah-perintah Kristus.
\par 22 Di tengah-tengah orang-orang yang keyakinannya tidak kuat, saya pun berlaku seperti seorang yang keyakinannya tidak kuat supaya saya bisa menarik mereka menjadi pengikut Kristus. Pendeknya, saya menjadi segala-galanya untuk semua orang supaya dengan jalan yang bagaimanapun juga saya bisa menyelamatkan sebagian dari mereka.
\par 23 Semua itu saya lakukan untuk Kabar Baik dari Allah itu, supaya saya juga turut diberkati.
\par 24 Saudara tentu tahu bahwa di dalam perlombaan semua peserta turut lari, tetapi hanya satu orang menerima hadiah. Oleh sebab itu larilah begitu rupa sehingga Saudara menerima hadiahnya.
\par 25 Setiap orang yang sedang dalam latihan, menahan diri dalam segala hal. Ia melakukan itu karena ia ingin dikalungi dengan karangan bunga kejuaraan, yaitu bunga yang segera akan layu. Tetapi kita ini menahan diri dalam segala hal karena kita ingin dikalungi dengan karangan bunga yang tidak akan layu.
\par 26 Itu sebabnya saya berlari dengan tujuan yang tertentu. Seperti dalam pertandingan tinju, saya tidak memukul dengan sembarangan.
\par 27 Saya membiarkan badan saya digembleng dengan keras sampai saya dapat menguasainya. Saya berbuat begitu, sebab saya tidak mau sampai terjadi bahwa setelah saya mengajak orang lain turut dalam perlombaan itu, saya sendiri ditolak.

\chapter{10}

\par 1 Saudara-saudara! Saudara hendaknya mengingat apa yang terjadi kepada nenek moyang kita ketika mereka mengikuti Musa. Mereka semua dilindungi oleh awan, dan dengan selamat menyeberangi Laut Merah.
\par 2 Untuk menjadi pengikut-pengikut Musa, mereka semuanya dibaptis di dalam awan dan di dalam laut itu.
\par 3 Mereka semuanya makan makanan rohani yang sama,
\par 4 dan minum minuman rohani yang sama. Mereka semuanya minum dari gunung batu rohani yang menyertai mereka; gunung batu itu ialah Kristus sendiri.
\par 5 Meskipun begitu, Allah tidak senang terhadap kebanyakan dari mereka, dan itulah sebabnya mayat-mayat mereka bergelimpangan di padang gurun.
\par 6 Semuanya itu menjadi contoh bagi kita, untuk mengingatkan kita supaya jangan menginginkan hal-hal yang jahat seperti mereka.
\par 7 Juga supaya kita jangan menyembah berhala seperti yang dilakukan oleh sebagian dari mereka. Dalam Alkitab tertulis, "Maka bangsa itu mulai makan minum, dan menari untuk menyembah berhala."
\par 8 Kita tidak boleh melakukan hal-hal yang cabul seperti yang dilakukan oleh sebagian dari mereka. Sebab, dua puluh tiga ribu orang dari mereka mati dalam satu hari karena melakukan itu.
\par 9 Kita tidak boleh mencoba-coba Tuhan seperti yang dilakukan oleh sebagian dari mereka dahulu, sehingga mereka mati dipagut ular berbisa.
\par 10 Kita tidak boleh juga menggerutu; seperti yang dilakukan oleh sebagian dari mereka dahulu, sehingga mereka dibunuh oleh Malaikat Kematian.
\par 11 Semua hal itu terjadi kepada mereka untuk menjadi contoh bagi orang-orang lain. Dan semuanya itu tertulis juga untuk menjadi peringatan kepada kita. Sebab kita sekarang hidup di masa akhir zaman.
\par 12 Orang yang menyangka dirinya berdiri teguh, hendaklah berhati-hati; jangan sampai ia jatuh.
\par 13 Setiap cobaan yang Saudara alami adalah cobaan yang lazim dialami manusia. Tetapi Allah setia pada janji-Nya. Ia tidak akan membiarkan Saudara dicoba lebih daripada kesanggupanmu. Pada waktu Saudara ditimpa oleh cobaan, Ia akan memberi jalan kepadamu untuk menjadi kuat supaya Saudara dapat bertahan.
\par 14 Oleh sebab itu, Saudara-saudara yang saya kasihi, hendaklah kalian membenci penyembahan berhala.
\par 15 Saya berbicara kepadamu seperti kepada orang-orang yang bijaksana. Hendaklah kalian menimbang sendiri perkataan saya ini.
\par 16 Pada waktu kita minum anggur dengan mengucap terima kasih kepada Allah, bukankah itu menunjukkan bahwa kita bersatu dengan Kristus dalam kematian-Nya? Dan pada waktu kita membagi-bagikan roti untuk dimakan bersama-sama, bukankah itu menunjukkan bahwa kita bersatu dalam tubuh Kristus?
\par 17 Oleh karena hanya ada satu roti, dan kita semua makan dari roti yang satu itu juga, maka kita yang banyak ini merupakan satu tubuh.
\par 18 Coba perhatikan orang-orang Yahudi. Mereka yang makan makanan yang dipersembahkan di atas tempat persembahan kurban, adalah orang-orang yang bersatu dengan tempat itu.
\par 19 Apa maksud saya dengan itu? Berhala ataupun makanan yang dipersembahkan kepada berhala itu tidak punya arti sama sekali.
\par 20 Apa yang dipersembahkan di tempat kurban untuk berhala tidak dipersembahkan kepada Allah, melainkan kepada roh-roh jahat. Dan saya tidak mau kalian bersatu dengan roh jahat.
\par 21 Kalian tidak boleh minum dari piala anggur Tuhan, dan sekaligus dari piala anggur roh jahat juga. Kalian tidak boleh makan di meja Tuhan, dan di meja roh jahat juga.
\par 22 Ataukah kita mau membuat Tuhan iri hati? Apakah kita lebih kuat dari Tuhan?
\par 23 Kata orang, "Kita boleh berbuat apa saja yang kita mau." Benar! Tetapi tidak semua yang kita mau itu berguna. "Kita boleh berbuat apa saja yang kita mau" --tetapi tidak semua yang kita mau itu membangun kehidupan kita.
\par 24 Janganlah seorang pun berjuang untuk kepentingan dirinya sendiri saja. Setiap orang harus berjuang untuk kepentingan orang lain.
\par 25 Kalian boleh makan apa saja yang dijual di pasar daging. Tidak usah menyelidikinya terlebih dahulu, meskipun ada keberatan-keberatan dalam hati nuranimu.
\par 26 Sebab dalam Alkitab tertulis, "Bumi ini dengan seluruh isinya adalah kepunyaan Tuhan."
\par 27 Kalau kalian diundang makan oleh seorang yang bukan Kristen, dan kalian menerima undangan itu, makanlah apa yang dihidangkan kepadamu. Tidak usah menyelidiki dari mana datangnya makanan itu, supaya tidak ada keberatan-keberatan di dalam hati nuranimu.
\par 28 Tetapi kalau ada yang berkata kepadamu, "Makanan ini sudah dipersembahkan kepada berhala," janganlah makan makanan itu, karena memperhatikan kepentingan orang itu dan karena suara hati nurani.
\par 29 Maksud saya bukan suara hati nuranimu, tetapi suara hati nurani orang itu. Mungkin ada yang bertanya, "Wah, mengapa kebebasan saya harus dibatasi oleh suara hati nurani orang lain?
\par 30 Kalau saya makan makanan dengan berterima kasih kepada Allah, mengapa orang harus mencela saya atas makanan itu, sedangkan saya sudah berterima kasih kepada Allah atasnya?"
\par 31 Apa pun yang Saudara lakukan--Saudara makan atau Saudara minum--lakukanlah semuanya itu untuk memuliakan Allah.
\par 32 Hiduplah sedemikian rupa sehingga Saudara tidak menyebabkan orang lain berbuat dosa; apakah mereka orang Yahudi atau orang bukan Yahudi, ataupun jemaat Allah.
\par 33 Hiduplah seperti saya. Saya berusaha menyenangkan hati semua orang dalam segala sesuatu, tanpa maksud-maksud kepentingan diri sendiri. Tujuan saya adalah hanya supaya mereka semuanya dapat diselamatkan.

\chapter{11}

\par 1 Ikutlah teladan saya, seperti saya pun mengikuti teladan Kristus.
\par 2 Saya memuji kalian sebab kalian selalu mengingat saya dan menuruti pelajaran yang saya berikan kepadamu.
\par 3 Tetapi saya ingin kalian mengetahui satu hal lagi, yaitu bahwa yang menjadi kepala atas setiap orang laki-laki adalah Kristus; yang menjadi kepala atas istri adalah suami, dan yang menjadi kepala atas Kristus adalah Allah.
\par 4 Kalau seorang laki-laki pada waktu berdoa atau pada waktu menyampaikan berita dari Allah di hadapan banyak orang, memakai tutup kepala, maka orang itu menghina Kristus.
\par 5 Dan kalau seorang wanita pada waktu berdoa atau pada waktu menyampaikan berita dari Allah di hadapan banyak orang, tidak memakai tutup kepala, maka wanita itu menghina suaminya yang menjadi kepala atas dirinya. Itu sama saja seolah-olah kepala wanita itu sudah dicukur.
\par 6 Sebab kalau seorang wanita tidak mau memakai tutup kepala lebih baik rambutnya digunting. Tetapi kalau seorang wanita dicukur kepalanya atau digunting rambutnya, maka itu suatu penghinaan bagi dia. Oleh sebab itu lebih baik ia memakai tutup kepala.
\par 7 Laki-laki tidak perlu memakai tutup kepala, sebab ia mencerminkan pribadi dan kebesaran Allah. Tetapi wanita mencerminkan kebesaran laki-laki,
\par 8 sebab laki-laki tidak dijadikan dari wanita, wanitalah yang dijadikan dari laki-laki.
\par 9 Laki-laki tidak pula dijadikan untuk kepentingan wanita, melainkan wanita dijadikan untuk kepentingan laki-laki.
\par 10 Sebab itu, untuk menyenangkan para malaikat, seorang wanita harus memakai tutup kepala sebagai tanda bahwa ia di bawah kekuasaan suaminya.
\par 11 Meskipun begitu dalam kehidupan kita sebagai orang Kristen, wanita tidak berdiri sendiri, lepas dari laki-laki, dan laki-laki pun tidak berdiri sendiri, lepas dari wanita.
\par 12 Karena meskipun wanita pertama dijadikan dari laki-laki, tetapi setelah itu laki-laki lahir dari wanita; dan segala sesuatu berasal dari Allah.
\par 13 Coba Saudara-saudara sendiri menimbang hal ini: Apakah baik seorang wanita berdoa kepada Allah di hadapan orang banyak, tanpa memakai tutup kepala?
\par 14 Dari pengalaman umum, kalian sudah diajar bahwa kalau laki-laki berambut panjang, itu sesuatu yang kurang patut.
\par 15 Tetapi bagi wanita, rambut diberikan kepadanya untuk menutupi kepalanya dan rambut panjang adalah kebanggaannya.
\par 16 Nah, kalau ada yang mau bertengkar tentang masalah ini, satu-satunya yang saya dapat katakan ialah bahwa baik kita maupun jemaat-jemaat Allah lainnya, pada umumnya hanya mempunyai satu kebiasaan itu dalam jemaat; lain daripada itu tidak ada.
\par 17 Mengenai yang berikut ini, saya tidak memuji kalian. Pertemuan ibadatmu bukannya menghasilkan yang baik, melainkan yang tidak baik.
\par 18 Pertama-tama saya mendengar, bahwa di dalam pertemuan-pertemuanmu ada golongan-golongan yang saling bertentangan. Dan saya kira sedikit banyak kabar itu benar.
\par 19 Memang sudah sewajarnya timbul perpecahan di antaramu, supaya kelihatan nanti siapa-siapa orang Kristen yang sejati.
\par 20 Tetapi pada waktu kalian berkumpul, yang kalian adakan itu bukan Perjamuan Tuhan.
\par 21 Sebab pada waktu makan, kalian masing-masing berebut mengambil makanannya sendiri, sampai ada yang tidak mendapat apa-apa, sedangkan yang lainnya menjadi mabuk.
\par 22 Mengapa begitu? Bukankah Saudara punya rumah? Saudara bisa makan dan minum di situ! Ataukah Saudara mau menghina jemaat Allah dan memalukan orang-orang miskin? Apakah yang harus saya katakan kepada kalian? Haruskah saya memuji kalian? Tidak! Sekali-kali saya tidak akan memuji kalian.
\par 23 Sebab yang saya ajarkan kepadamu, itu saya terima dari Tuhan sendiri: bahwa pada malam itu ketika Yesus, Tuhan kita dikhianati, Ia mengambil roti,
\par 24 dan setelah Ia mengucap terima kasih kepada Allah atas roti itu, Ia membelah-belah roti itu dengan tangan-Nya, lalu berkata, "Inilah tubuh-Ku yang diserahkan untuk kalian. Lakukanlah ini untuk mengenang Aku."
\par 25 Begitu pula setelah habis makan, Ia mengambil piala anggur lalu berkata, "Anggur ini adalah perjanjian Allah yang baru, disahkan dengan darah-Ku. Setiap kali kalian minum ini, lakukanlah ini untuk mengenang Aku."
\par 26 Memang setiap kali kalian makan roti dan minum anggur ini, kalian memberitakan kematian Tuhan, sampai Ia datang.
\par 27 Oleh karena itu, orang yang makan roti Tuhan atau minum anggur Tuhan dengan cara yang tidak patut, orang itu berdosa terhadap Tuhan yang sudah mengurbankan tubuh dan darah-Nya.
\par 28 Jadi, setiap orang harus memeriksa dirinya dahulu, baru ia boleh makan roti dan minum anggur itu.
\par 29 Sebab kalau orang makan roti dan minum anggur tanpa mengindahkan bahwa perjamuan itu berkenaan dengan tubuh Tuhan, orang itu makan dan minum untuk menerima hukuman Allah atas dirinya sendiri.
\par 30 Itulah sebabnya banyak dari antara kalian yang sakit dan lemah, dan ada juga yang mati.
\par 31 Tetapi kalau kita memeriksa diri kita terlebih dahulu, kita tidak akan dihukum Allah.
\par 32 Tetapi kalau kita dihukum Allah, Ia akan mencambuk kita, supaya kita jangan kena penghukuman bersama-sama dunia ini.
\par 33 Oleh sebab itu, Saudara-saudaraku, kalau kalian berkumpul untuk makan pada perjamuan Tuhan, kalian harus saling menunggu.
\par 34 Kalau ada yang lapar, ia harus makan dahulu di rumah. Kalau kalian mengindahkan hal ini, pertemuan-pertemuan ibadatmu tidak akan mendatangkan hukuman dari Allah atas dirimu sendiri. Mengenai masalah-masalah yang lain, akan saya terangkan kepadamu, kalau saya datang nanti.

\chapter{12}

\par 1 Sekarang mengenai karunia-karunia yang diberikan Roh Allah. Mengenai itu, saya mau Saudara-saudara mengetahui yang sebenarnya.
\par 2 Ingatlah bahwa pada waktu kalian belum mengenal Tuhan, kalian terpengaruh untuk mengikuti berhala yang bisu.
\par 3 Kalian harus tahu bahwa orang yang dipimpin Roh Allah tidak dapat berkata, "Terkutuklah Yesus!" Begitu juga tidak seorang pun dapat mengatakan, "Yesuslah Tuhan!" kalau orang itu tidak dipimpin Roh Allah.
\par 4 Ada bermacam-macam karunia dari Roh Allah, tetapi semuanya diberi oleh Roh yang satu.
\par 5 Ada bermacam-macam pekerjaan untuk melayani Tuhan, tetapi Tuhan yang dilayani itu, Tuhan yang satu juga!
\par 6 Ada berbagai-bagai cara mengerjakan pekerjaan Tuhan, tetapi yang memberikan kekuatan untuk itu kepada setiap orang adalah Allah yang satu juga.
\par 7 Untuk kebaikan kita semua, Roh Allah bekerja pada setiap orang secara sendiri-sendiri.
\par 8 Kepada yang seorang, Roh itu memberikan kesanggupan untuk berbicara dengan wibawa. Kepada yang lain Roh yang sama itu memberikan kesanggupan untuk menjelaskan tentang Allah.
\par 9 Roh yang satu itu juga memberikan kepada orang yang satu, kemampuan yang luar biasa untuk percaya kepada Kristus; sedangkan kepada yang lain Roh itu memberikan kuasa untuk menyembuhkan orang.
\par 10 Kepada seorang diberikan kuasa untuk mengadakan keajaiban dan kepada yang lain diberikan karunia untuk memberitahukan rencana-rencana Allah. Kepada yang lain lagi Roh itu memberi kesanggupan untuk membeda-bedakan mana karunia yang dari Roh Allah dan mana yang bukan. Ada yang diberikan kesanggupan untuk berbicara dengan berbagai bahasa yang ajaib, dan ada pula yang diberikan kesanggupan untuk menerangkan arti bahasa-bahasa itu.
\par 11 Semuanya itu dikerjakan oleh Roh yang satu itu juga; masing-masing orang diberi karunia yang tersendiri menurut kemauan Roh itu sendiri.
\par 12 Kristus adalah seperti tubuh manusia; tubuh itu satu, tetapi terdiri dari banyak anggota. Semua anggota itu, meskipun banyak, merupakan satu tubuh.
\par 13 Begitu juga kita semua, baik orang Yahudi maupun orang bukan Yahudi, hamba-hamba maupun orang-orang merdeka; kita semua sudah dibaptis oleh Roh yang sama itu, supaya kita dijadikan satu pada tubuh Kristus itu. Kita semua juga mengalami Roh yang satu itu sepenuhnya.
\par 14 Sebab tubuh itu sendiri tidak terdiri dari satu anggota saja tetapi banyak anggotanya.
\par 15 Kalau kaki berkata, "Saya bukan tangan, karena itu saya bukan bagian dari tubuh," itu tidak berarti bahwa kaki itu bukan bagian dari tubuh.
\par 16 Dan kalau telinga berkata, "Sebab saya bukan mata, maka saya bukanlah bagian dari tubuh," itu juga tidak berarti bahwa telinga itu bukan bagian dari tubuh.
\par 17 Seandainya seluruh tubuh itu menjadi mata saja, bagaimana tubuh itu dapat mendengar? Atau kalau seluruh tubuh itu menjadi telinga saja, bagaimanakah tubuh itu mencium?
\par 18 Kita lihat bahwa Allah yang menempatkan anggota-anggota itu pada tubuh. Masing-masing ditempatkan di tempatnya oleh Allah menurut kehendak-Nya.
\par 19 Kalau semuanya hanya satu anggota saja, manakah yang disebut tubuh?
\par 20 Jadi memang ada banyak anggota, tetapi tubuh hanya satu.
\par 21 Oleh sebab itu, mata tidak dapat berkata kepada tangan, "Saya tidak memerlukan engkau!" atau kepala berkata kepada kaki, "Saya tidak memerlukan engkau!"
\par 22 Sebaliknya anggota-anggota tubuh yang dianggap lemah itu, kita perlukan sekali;
\par 23 dan anggota-anggota yang kita anggap tidak begitu berharga, justru adalah anggota-anggota yang kita berikan lebih banyak penghargaan. Anggota-anggota tubuh yang tidak kelihatan cantik, malah lebih kita perhatikan.
\par 24 Anggota-anggota tubuh yang sudah kelihatan bagus, tidak memerlukan perhatian kita. Allah sudah menyusun tubuh kita sebegitu rupa sehingga anggota-anggota yang kurang berharga diberikan lebih banyak penghargaan.
\par 25 Dengan demikian tubuh itu tidak terbagi-bagi; masing-masing anggota memperhatikan satu sama lain.
\par 26 Kalau satu anggota menderita, semua anggota lainnya menderita juga; kalau satu anggota dipuji, semua anggota lainnya turut bergembira.
\par 27 Saudara semuanya bersama-sama adalah tubuh Kristus dan kalian masing-masing pula adalah anggota dari tubuh itu.
\par 28 Begitu juga di dalam jemaat, Allah sudah menentukan tempat untuk berbagai-bagai orang: Mula-mula rasul-rasul; kedua, nabi-nabi, ketiga guru-guru, lalu mereka yang mengadakan keajaiban-keajaiban, kemudian mereka yang diberi karunia untuk menyembuhkan orang, atau untuk menolong orang lain, atau untuk memimpin, ataupun untuk berbicara dengan berbagai bahasa yang ajaib.
\par 29 Mereka bukan semuanya rasul, atau nabi, ataupun guru. Tidak semuanya mempunyai kuasa untuk mengadakan keajaiban,
\par 30 atau untuk menyembuhkan orang, atau untuk berbicara dengan berbagai bahasa yang ajaib, atau untuk menterjemahkan bahasa-bahasa itu.
\par 31 Karena itu, hendaklah kalian berusaha sungguh-sungguh untuk mendapat karunia-karunia yang paling utama. Namun berikut ini saya menunjukkan kepadamu jalan yang terbaik.

\chapter{13}

\par 1 Meskipun saya dapat berbicara dengan berbagai bahasa manusia, bahkan dengan bahasa malaikat sekalipun, tetapi saya tidak mengasihi orang lain, maka ucapan-ucapan saya itu hanya bunyi yang nyaring tanpa arti.
\par 2 Meskipun saya pandai menyampaikan berita dari Allah, dan mengerti semua hal yang dalam-dalam, dan tahu segala sesuatu serta sangat percaya kepada Allah sehingga dapat membuat gunung berpindah, tetapi saya tidak mengasihi orang-orang lain, maka saya tidak berarti apa-apa!
\par 3 Meskipun semua yang saya miliki, saya sedekahkan kepada orang miskin, dan saya menyerahkan diri saya untuk dibakar, tetapi saya tidak mengasihi orang-orang lain, maka semuanya itu tidak ada gunanya sama sekali.
\par 4 Orang yang mengasihi orang-orang lain, sabar dan baik hati. Ia tidak meluap dengan kecemburuan, tidak membual, tidak sombong.
\par 5 Ia tidak angkuh, tidak kasar, ia tidak memaksa orang lain untuk mengikuti kemauannya sendiri, tidak juga cepat tersinggung, dan tidak dendam.
\par 6 Orang yang mengasihi orang-orang lain, tidak senang dengan kejahatan, ia hanya senang dengan kebaikan.
\par 7 Ia tahan menghadapi segala sesuatu dan mau percaya akan yang terbaik pada setiap orang; dalam keadaan yang bagaimanapun juga orang yang mengasihi itu tidak pernah hilang harapannya dan sabar menunggu segala sesuatu.
\par 8 Tidak pernah akan ada saat di mana orang tidak perlu saling mengasihi. Sekarang ini ada orang yang pandai menyampaikan berita dari Allah, tetapi nanti ia akan berhenti menyampaikan berita itu. Sekarang ada yang pandai berbicara dalam berbagai bahasa yang ajaib, tetapi nanti ia akan berhenti berbicara dalam bahasa-bahasa itu. Sekarang ada orang yang mengetahui banyak hal, tetapi nanti apa yang mereka ketahui itu akan dilupakan.
\par 9 Sebab, pengetahuan kita dan kesanggupan kita untuk menyampaikan berita dari Allah, masih kurang sempurna.
\par 10 Nanti akan tiba waktunya Allah membuat semuanya sempurna, dan yang tidak sempurna itu akan hilang.
\par 11 Pada waktu saya masih anak kecil, saya berbicara seperti anak kecil, saya berperasaan seperti anak kecil dan saya berpikir seperti anak kecil. Sekarang saya sudah dewasa, kelakuan saya yang kekanak-kanakan sudah saya buang.
\par 12 Apa yang kita lihat sekarang ini adalah seperti bayangan yang kabur pada cermin. Tetapi nanti kita akan melihat langsung dengan jelas. Sekarang saya belum tahu segalanya, tetapi nanti saya akan tahu segalanya sama seperti Allah tahu segalanya mengenai diri saya.
\par 13 Jadi, untuk saat ini ada tiga hal yang kita harus tetap lakukan: percaya, berharap dan saling mengasihi. Yang paling penting dari ketiganya itu ialah mengasihi orang-orang lain.

\chapter{14}

\par 1 Hendaklah kalian berusaha untuk mengasihi orang-orang lain. Dan berusahalah juga untuk menerima karunia-karunia yang diberikan Roh Allah, terutama sekali kesanggupan untuk menyampaikan rencana-rencana Allah kepada manusia.
\par 2 Orang yang berbicara dalam bahasa yang ajaib, orang itu bukannya berbicara kepada manusia; ia berbicara kepada Allah. Tidak ada orang yang mengerti apa yang ia katakan, sebab Roh Allah yang menyebabkan ia mengucapkan hal-hal yang hanya diketahui Allah.
\par 3 Sebaliknya orang yang menyampaikan berita dari Allah, menyampaikannya kepada manusia; untuk menguatkan mereka, untuk memberi semangat kepada mereka dan untuk menghibur mereka.
\par 4 Orang yang berbicara dalam bahasa yang ajaib hanya menguatkan dirinya sendiri saja, sedangkan orang yang menyampaikan berita dari Allah menolong jemaat menjadi maju.
\par 5 Alangkah baiknya kalau Saudara semua dapat berbicara dengan berbagai bahasa yang ajaib. Tetapi yang paling baik ialah kalau Saudara dapat memberitakan rencana-rencana Allah. Sebab orang yang menyampaikan berita dari Allah, lebih besar daripada orang yang berbicara dengan berbagai bahasa yang ajaib; lain halnya kalau orang yang berbicara dalam berbagai bahasa yang ajaib itu dapat menjelaskan apa yang dikatakannya itu, supaya seluruh jemaat mendapat manfaatnya.
\par 6 Kalau saya seandainya datang kepadamu dan saya berbicara dalam berbagai bahasa yang ajaib, apa gunanya itu untuk kalian? Tidak ada gunanya sedikit pun! Lain halnya, kalau saya menyatakan suatu pengungkapan dari Allah, atau saya menjelaskan sesuatu tentang Allah, atau saya menyampaikan berita dari Allah, atau saya mengajar.
\par 7 Alat-alat musik yang tidak bernyawa sedikit pun, seperti misalnya seruling dan kecapi, kalau nada-nadanya tidak dimainkan dengan jelas, bagaimana orang tahu lagu apa yang dimainkannya?
\par 8 Suatu contoh yang lain lagi: Kalau trompet dibunyikan sembarangan, siapa yang akan bersiap untuk berperang?
\par 9 Begitu juga dengan kesanggupanmu untuk berbicara dengan berbagai bahasa yang ajaib. Kalau Saudara dengan kesanggupan itu mengucapkan kata-kata yang tidak jelas, tidak ada seorang pun yang bisa mengerti apa yang Saudara katakan. Kata-katamu itu akan lenyap tidak menentu.
\par 10 Di dunia ini ada sekian banyak bahasa, tetapi tidak ada satu pun dari bahasa-bahasa itu yang tidak mengandung arti.
\par 11 Tetapi kalau saya tidak mengerti bahasa yang diucapkan seseorang, maka orang yang memakai bahasa itu merupakan orang asing terhadap saya; begitu juga saya terhadap dia.
\par 12 Mengenai kalian sendiri, saya tahu kalian ingin sekali mendapat karunia-karunia dari Roh Allah. Tetapi yang terutama sekali, haruslah kalian berusaha untuk memakai kesanggupan yang menolong jemaat menjadi maju.
\par 13 Itu sebabnya orang yang berbicara dalam bahasa yang ajaib, haruslah memohon dari Allah supaya ia diberi juga kesanggupan untuk menerangkan apa yang dikatakannya itu.
\par 14 Sebab kalau saya berdoa dengan bahasa yang ajaib, roh saya memang berdoa, tetapi pikiran saya tidak bekerja.
\par 15 Jadi, saya harus berbuat apa? Ini yang akan saya buat: Saya akan berdoa dengan roh saya, tetapi saya akan berdoa juga dengan pikiran saya. Saya akan bernyanyi dengan roh saya, tetapi saya mau menyanyi juga dengan pikiran saya.
\par 16 Sebab kalau Saudara mengucap terima kasih kepada Allah dengan rohmu saja, dan ada orang lain yang tidak mengerti bahasa ajaib yang dari Roh Allah itu, maka orang itu tidak dapat berkata, "Aku setuju" terhadap doa syukurmu itu; karena ia tidak tahu apa yang Saudara katakan.
\par 17 Meskipun doa terima kasihmu kepada Tuhan itu sangat baik, namun doa itu tidak ada gunanya sama sekali bagi orang lain.
\par 18 Saya berterima kasih kepada Allah sebab saya sendiri dapat berbicara dalam berbagai-bagai bahasa yang ajaib lebih daripada Saudara semua.
\par 19 Namun, di dalam pertemuan-pertemuan untuk menyembah Tuhan, saya lebih suka memakai lima perkataan yang dapat dimengerti orang daripada memakai beribu-ribu perkataan dalam bahasa yang ajaib. Saya lebih suka begitu supaya saya dapat mengajar orang.
\par 20 Saudara-saudara! Janganlah berpikir seperti anak-anak. Dalam hal kejahatan, hendaklah kalian tetap seperti anak kecil. Tetapi dalam pemikiran, hendaklah kalian menjadi orang yang sudah dewasa.
\par 21 Dalam Alkitab tertulis begini, "Tuhan berkata, 'Dengan perantaraan orang-orang yang berbicara dengan bahasa yang ganjil, Aku akan berbicara kepada umat ini. Bahkan dengan perantaraan orang-orang asing, Aku akan berbicara kepada umat-Ku; namun demikian, mereka tidak mau mendengar perkataan-Ku.'"
\par 22 Jadi karunia untuk berbicara dengan berbagai bahasa yang ajaib adalah sebagai tanda untuk orang yang tidak percaya, bukan untuk orang yang percaya. Dan karunia untuk memberitahukan rencana Allah kepada manusia adalah tanda untuk orang yang percaya, bukan untuk orang yang tidak percaya.
\par 23 Karena itu, kalau seandainya di dalam pertemuan jemaat, seluruh jemaat berbicara dalam berbagai bahasa yang ajaib, lalu datang beberapa orang luar, atau orang-orang yang bukan Kristen, tentu orang-orang itu akan menyangka kalian sudah gila semuanya!
\par 24 Tetapi kalau Saudara semuanya menyampaikan berita dari Allah, lalu datang seorang yang bukan Kristen atau seorang luar, maka hal-hal yang diberitakan oleh Saudara semuanya akan menunjukkan dosa-dosa orang itu dan membuat ia sadar akan dosa-dosanya.
\par 25 Hal-hal yang tersembunyi dalam hatinya akan dinyatakan, sehingga ia akan merendahkan dirinya lalu menyembah Allah. Ia akan mengaku bahwa Allah benar-benar ada di tengah-tengah kalian.
\par 26 Jadi, Saudara-saudara, apa artinya semuanya itu? Kalau kalian berkumpul untuk menyembah Tuhan, ada yang menyanyi, ada yang mengajar, ada yang memberitahukan sesuatu dari Allah, ada yang berbicara dalam bahasa yang ajaib, dan ada yang menjelaskan apa yang dikatakan itu. Tetapi semuanya itu haruslah dilakukan untuk mengajar dan untuk kebaikan semuanya.
\par 27 Kalau ada yang mau berbicara dalam bahasa yang ajaib, haruslah dua atau paling banyak tiga orang saja secara bergilir. Dan harus ada yang menjelaskan apa yang dikatakan oleh orang yang berbicara itu.
\par 28 Kalau tidak ada yang dapat menjelaskannya, maka orang-orang yang berbicara dalam bahasa yang ajaib itu harus diam dalam pertemuan itu. Biarlah mereka berbicara dalam hati saja kepada Allah.
\par 29 Dua atau tiga orang yang mempunyai berita dari Allah harus menyampaikan berita itu sementara yang lain mempertimbangkan apa yang dikatakan itu.
\par 30 Tetapi kalau seandainya berita dari Allah datang pada seorang lain yang duduk pada pertemuan itu, maka orang yang sedang berbicara, harus berhenti.
\par 31 Dengan cara yang demikian, Saudara-saudara semuanya, satu per satu, dapat menyampaikan berita dari Allah; supaya semuanya dapat menerima pelajaran dan menjadi makin percaya.
\par 32 Karunia dari Roh untuk menyampaikan berita dari Allah dapat dikendalikan oleh orang yang menyampaikan berita itu.
\par 33 Allah adalah Allah yang suka akan ketertiban; Ia bukan Allah yang suka pada kekacauan. Seperti yang berlaku di dalam semua jemaat Allah,
\par 34 wanita harus diam pada waktu pertemuan jemaat. Mereka tidak diizinkan berbicara. Mereka tidak boleh memegang pimpinan; itu sesuai dengan hukum agama.
\par 35 Kalau mereka mau mengetahui sesuatu, mereka harus menanyakan itu kepada suami mereka di rumah. Sangat memalukan bila seorang wanita berbicara di dalam pertemuan jemaat.
\par 36 Apakah perkataan Allah datang dari kalian? Atau hanya kepadamu saja perkataan itu disampaikan?
\par 37 Kalau ada orang yang merasa mempunyai karunia untuk menyampaikan berita dari Allah, atau ia mempunyai karunia yang lainnya dari Roh Allah, orang itu harus sadar bahwa apa yang saya tulis ini adalah perintah dari Tuhan.
\par 38 Tetapi kalau ada yang tidak menerima ini, janganlah memperhatikan dia.
\par 39 Oleh sebab itu, Saudara-saudara, berusahalah untuk menyampaikan berita dari Allah, tetapi janganlah melarang orang yang mau berbicara dalam berbagai bahasa yang ajaib.
\par 40 Tetapi semuanya harus dilakukan dengan baik dan teratur.

\chapter{15}

\par 1 Dan sekarang, Saudara-saudara, saya mau kalian mengingat kembali akan Kabar Baik dari Allah yang saya beritakan dahulu kepadamu. Kalian menerimanya, dan percaya kepada Kristus karena Kabar Baik itu.
\par 2 Kalau kalian berpegang teguh pada apa yang saya beritakan itu, maka Kabar Baik itu menyelamatkan kalian; kecuali kalau Saudara percaya tanpa pengertian.
\par 3 Apa yang saya sampaikan kepada Saudara-saudara adalah yang sudah saya terima juga. Yang terpenting, seperti yang tertulis dalam Alkitab, saya menyampaikan kepadamu bahwa Kristus mati karena dosa-dosa kita;
\par 4 bahwa Ia dikubur, tetapi kemudian dihidupkan kembali pada hari yang ketiga. Itu juga tertulis dalam Alkitab.
\par 5 Saya memberitahukan juga kepadamu bahwa Kristus yang dihidupkan kembali itu menunjukkan juga diri-Nya kepada Petrus, dan kemudian kepada kedua belas rasul.
\par 6 Setelah itu Ia menunjukkan diri pula kepada lebih dari lima ratus pengikut-Nya sekaligus. Kebanyakan dari orang-orang itu masih hidup sekarang, hanya beberapa orang yang sudah meninggal.
\par 7 Sesudah itu Kristus menunjukkan diri juga kepada Yakobus dan kemudian kepada semua rasul.
\par 8 Terakhir sekali Ia menunjukkan diri kepada saya juga--saya yang seperti bayi yang lahir tidak pada waktunya!
\par 9 Saya adalah rasul Tuhan yang paling rendah. Saya tidak patut disebut rasul, sebab saya sudah menganiaya jemaat Allah.
\par 10 Tetapi karena rahmat Allah saya menjadi seperti keadaan saya yang sekarang. Dan tidaklah percuma Allah mengampuni saya. Malah justru sayalah yang bekerja lebih keras dari semua rasul yang lainnya. Tetapi itu sebenarnya bukan usaha saya; itu usaha Allah yang mengasihi saya dan yang bekerja bersama-sama saya.
\par 11 Jadi, dari siapa pun kalian menerima Kabar Baik itu--apakah itu dari saya atau dari rasul-rasul yang lain--itu tidak menjadi soal. Yang penting ialah kami memberitakan Kabar Baik itu dan Saudara percaya.
\par 12 Kalau yang kami beritakan itu ialah bahwa Kristus sudah dihidupkan kembali dari kematian, mengapa ada dari antaramu yang berkata bahwa orang mati tidak akan dihidupkan kembali?
\par 13 Kalau betul orang mati tidak akan dihidupkan kembali, itu berarti Kristus juga tidak dihidupkan kembali dari kematian.
\par 14 Dan kalau seandainya Kristus tidak dihidupkan kembali dari kematian, maka tidak ada gunanya kami memberitakan apa-apa dan tidak ada gunanya pula kalian percaya, sebab kepercayaanmu itu tidak mempunyai dasar apa-apa.
\par 15 Lebih dari itu, ternyatalah bahwa kami berbohong mengenai Allah, sebab kami memberitakan bahwa Allah sudah menghidupkan kembali Kristus dari kematian, padahal Allah tidak menghidupkan Dia kembali--kalau memang benar orang mati tidak dihidupkan kembali!
\par 16 Sebab kalau orang mati tidak dihidupkan kembali, maka Kristus pun tidak dihidupkan kembali dari kematian.
\par 17 Dan kalau Kristus tidak dihidupkan kembali maka kepercayaanmu hanyalah impian belaka; itu berarti kalian masih dalam keadaan berdosa dan tidak mempunyai harapan sama sekali.
\par 18 Itu berarti pula bahwa orang-orang Kristen yang sudah meninggal, juga tidak mempunyai harapan.
\par 19 Kalau pengharapan kita kepada Kristus terbatas pada hidup kita di dalam dunia ini saja, maka dari seluruh umat manusia di dalam dunia ini, kitalah yang paling malang!
\par 20 Tetapi nyatanya Kristus sudah dihidupkan kembali dari kematian. Inilah jaminan bahwa orang-orang yang sudah mati akan dihidupkan kembali.
\par 21 Sebab kematian masuk ke dalam dunia dengan perantaraan satu orang, begitu juga hidup kembali dari kematian diberikan kepada manusia dengan perantaraan satu orang pula.
\par 22 Sebagaimana seluruh manusia mati karena tergolong satu dengan Adam, begitu juga semua akan dihidupkan, karena tergolong satu dengan Kristus.
\par 23 Tetapi masing-masing akan dihidupkan menurut gilirannya: pertama-tama Kristus; kemudian nanti pada waktu Ia datang lagi, menyusul giliran orang-orang yang termasuk milik Kristus.
\par 24 Sesudah itu terjadilah kiamat. Pada waktu itu Kristus akan menaklukkan segala pemerintahan, segala kekuasaan dan segala kekuatan; lalu Ia akan menyerahkan kekuasaan-Nya sebagai Raja, kepada Allah, Bapa kita.
\par 25 Kristus harus terus memerintah sampai Allah membuat semua musuh Kristus takluk kepada Kristus.
\par 26 Musuh yang terakhir yang akan ditaklukkan ialah kematian.
\par 27 Dalam Alkitab tertulis begini, "Allah sudah membuat segala sesuatu takluk kepada-Nya." Jelaslah bahwa yang dimaksud dengan "segala sesuatu" itu tidak termasuk Allah sendiri, yang membuat segala sesuatu itu takluk kepada Kristus.
\par 28 Tetapi setelah seluruhnya ditaklukkan ke bawah pemerintahan Kristus, maka Ia sendiri, yaitu Anak Allah, akan menaklukkan diri-Nya kepada Allah yang sudah membuat segala-galanya takluk kepada-Nya. Maka Allah sendiri pun akan memerintah semuanya.
\par 29 Kalau orang mati tidak dihidupkan kembali, mengapa ada orang yang dibaptis untuk orang mati? Untuk apa mereka melakukan hal itu? Kalau memang orang mati sama sekali tidak akan dihidupkan lagi, untuk apa mereka melakukan hal itu?
\par 30 Dan buat apa pula kami mau menghadapi bahaya setiap saat?
\par 31 Saudara-saudara! Setiap hari saya tahan menderita karena saya bangga atas kehidupanmu sebab kalian sudah percaya kepada Kristus Yesus, Tuhan kita.
\par 32 Kalau itu hanya dari segi manusia saja, apa untungnya bagi saya untuk berjuang seolah-olah melawan binatang-binatang buas di kota ini, di Efesus? Kalau memang orang mati tidak dihidupkan kembali, nah, lebih baik kita ikuti peribahasa ini: "Mari kita makan minum dan bersenang-senang, sebab besok kita toh akan mati."
\par 33 Janganlah tertipu! Pergaulan yang buruk merusakkan ahlak yang baik.
\par 34 Sadarlah sekarang, dan jangan lagi berbuat dosa. Sepantasnya kalian harus malu, bahwa di antaramu ada yang belum mengenal Allah.
\par 35 Mungkin ada yang bertanya, "Bagaimanakah orang mati dihidupkan kembali? Tubuh yang bagaimanakah yang diberi kepada mereka, sesudah mereka dihidupkan kembali?"
\par 36 Bodoh sekali! Kalau Saudara menanam benih di dalam tanah, benih itu tidak akan tumbuh, kalau benih itu belum mati dahulu.
\par 37 Dan benih yang Saudara tanam di dalam tanah itu--mungkin benih gandum atau benih lainnya--adalah biji benih itu saja, bukan seluruh tanamannya yang akan tumbuh nanti.
\par 38 Allah sendiri yang akan memberikan kepada benih itu bentuk yang menurut pandangan-Nya baik untuk tanaman itu. Untuk setiap macam benih, Allah memberikan bentuk tanamannya sendiri-sendiri.
\par 39 Tubuh makhluk-makhluk yang bernyawa tidak semuanya sama. Manusia mempunyai sejenis tubuh, binatang-binatang mempunyai jenis tubuh yang lain; burung-burung lain pula jenis tubuhnya, dan ikan-ikan lain lagi jenis tubuhnya.
\par 40 Begitu juga dengan benda-benda yang di langit dan benda-benda yang di bumi. Benda-benda yang di langit masing-masing mempunyai keindahannya sendiri, dan benda-benda yang di bumi pun begitu juga.
\par 41 Keindahan matahari lain daripada keindahan bulan. Bintang-bintang pun mempunyai keindahannya sendiri. Malah bintang-bintang itu masing-masing berlain-lainan pula keindahannya.
\par 42 Begitu halnya nanti dengan orang-orang mati yang dihidupkan kembali. Tubuh yang dikubur itu adalah tubuh yang bisa busuk, tetapi tubuh yang dihidupkan kembali, adalah tubuh yang tidak bisa rusak.
\par 43 Pada waktu tubuh itu dikuburkan, tubuh itu buruk dan lemah; tetapi pada waktu ia dihidupkan kembali, ia adalah tubuh yang bagus dan kuat.
\par 44 Pada waktu dikuburkan, tubuh itu tubuh dari dunia; tetapi setelah hidup kembali, maka tubuh itu tubuh yang diberi oleh Roh Allah. Ada tubuh yang dari dunia, ada juga tubuh yang dari Allah.
\par 45 Dalam Alkitab tertulis begini, "Manusia yang pertama, yakni Adam, menjadi makhluk yang hidup," tetapi Adam yang terakhir adalah Roh yang memberi hidup.
\par 46 Yang datang terlebih dahulu adalah yang jasmani, bukan yang rohani. Yang rohani datang kemudian.
\par 47 Adam yang pertama dijadikan dari tanah, tetapi Adam yang kedua berasal dari surga.
\par 48 Orang-orang dunia ini adalah seperti Adam yang pertama, yang dijadikan dari tanah, tetapi orang-orang surga adalah seperti Dia yang datang dari surga.
\par 49 Sebagaimana kita sekarang adalah seperti Adam yang dijadikan dari tanah, maka nanti kita akan menjadi seperti Dia yang dari surga itu.
\par 50 Maksud saya, Saudara-saudara, ialah: tubuh yang dijadikan dari darah dan daging, tidak dapat masuk Dunia Baru Allah; dan tubuh yang dapat mati tidak dapat menjadi abadi.
\par 51 Perhatikanlah rahasia ini: Tidak semuanya kita akan mati, tetapi kita semuanya akan berubah.
\par 52 Hal itu akan terjadi tiba-tiba dalam sekejap mata, pada waktu trompet dibunyikan untuk terakhir kalinya. Sebab pada waktu terdengar bunyi trompet itu, orang-orang mati akan dihidupkan kembali dengan tubuh yang abadi, dan kita semuanya akan diubah.
\par 53 Tubuh kita yang dapat mati ini harus diganti dengan tubuh yang tidak dapat mati, dan tubuh yang dari dunia harus diganti dengan tubuh yang dari surga.
\par 54 Kalau tubuh yang dapat mati sudah diganti dengan tubuh yang tidak dapat mati, dan tubuh yang dari dunia sudah diganti dengan tubuh yang dari surga, pada waktu itu barulah terjadi apa yang tertulis dalam Alkitab, "Kematian sudah dibasmi; kemenangan sudah tercapai!"
\par 55 "Hai maut, di manakah kemenanganmu? Hai maut, di manakah bisamu?"
\par 56 Bisa maut ialah dosa, dan dosa menjalankan peranannya melalui hukum agama.
\par 57 Tetapi syukur kepada Allah; Ia memberikan kepada kita kemenangan melalui Yesus Kristus Tuhan kita!
\par 58 Oleh sebab itu, Saudara-saudara yang tercinta, hendaklah kalian kuat dan teguh. Bekerjalah terus untuk Tuhan dengan sungguh-sungguh, sebab kalian mengetahui bahwa semua yang kalian kerjakan untuk Tuhan, tidak akan percuma.

\chapter{16}

\par 1 Sekarang mengenai uang yang kalian mau sumbangkan kepada sesama orang Kristen, saya anjurkan supaya kalian lakukan sesuai petunjuk yang saya berikan kepada jemaat-jemaat di Galatia.
\par 2 Pada hari pertama setiap minggu, hendaklah Saudara semuanya menyisihkan uang, masing-masing sesuai dengan pendapatannya. Simpanlah uang itu sampai saya datang, supaya pada waktu itu tidak perlu lagi dikumpulkan.
\par 3 Nanti kalau saya tiba, saya akan mengutus orang-orang yang sudah kalian setujui. Saya akan memberikan kepada mereka surat pengantar, supaya mereka membawa uang sumbangan itu ke Yerusalem.
\par 4 Dan kalau nampaknya baik bahwa saya pergi juga dengan mereka, maka saya akan pergi bersama mereka.
\par 5 Saya akan mengunjungi kalian setelah melewati Makedonia, sebab saya berniat untuk lewat di sana.
\par 6 Boleh jadi saya akan tinggal sebentar dengan kalian, mungkin selama musim dingin. Setelah itu kalian dapat membantu saya meneruskan perjalanan ke tempat berikutnya.
\par 7 Saya tidak mau mengunjungi kalian sekedar singgah saja. Kalau Tuhan mengizinkan, saya ingin tinggal agak lama dengan kalian.
\par 8 Sementara itu saya akan tinggal di kota ini, di Efesus, sampai hari Pentakosta.
\par 9 Banyak kesempatan di sini untuk pekerjaan-pekerjaan yang bermanfaat, meskipun banyak juga orang yang menentang.
\par 10 Kalau Timotius datang, sambutlah dia dengan baik supaya ia merasa betah di antara kalian, sebab ia seperti saya juga bekerja untuk Tuhan.
\par 11 Jangan sampai ada yang meremehkan dia. Bantulah dia supaya ia dapat meneruskan perjalanannya kembali kepada saya dengan selamat, sebab saya menunggu kedatangannya bersama dengan saudara-saudara yang lainnya.
\par 12 Tentang saudara kita Apolos, sudah beberapa kali saya menganjurkan dia supaya ia bersama saudara yang lainnya pergi mengunjungi kalian. Tetapi ia belum merasa yakin bahwa ia harus pergi sekarang. Namun kalau ada kesempatan nanti, tentu ia akan datang.
\par 13 Hendaklah kalian waspada dan teguh dalam hidupmu sebagai orang Kristen. Bertindaklah dengan berani dan jadilah kuat.
\par 14 Semua yang kalian lakukan, lakukanlah dengan kasih.
\par 15 Saudara tentunya mengenal Stefanus dan keluarganya; mereka yang pertama-tama menjadi Kristen di Akhaya. Mereka dengan sepenuh hati bekerja khusus untuk melayani umat Allah.
\par 16 Saya anjurkan dengan sungguh-sungguh supaya kalian mengikuti pimpinan orang-orang yang seperti itu, serta orang-orang lain yang bekerja sama dan melayani bersama mereka.
\par 17 Saya senang atas kedatangan Stefanus, Fortunatus dan Akhaikus. Mereka merupakan pengganti kalian bagi saya.
\par 18 Mereka sudah membuat hati saya menjadi gembira, seperti mereka menggembirakan hatimu. Orang-orang seperti itu harus dihargai.
\par 19 Jemaat-jemaat di Asia menyampaikan salam mereka kepada kalian. Akwila dan Priskila serta jemaat yang berkumpul di rumah mereka pun mengirim salam Kristen yang hangat.
\par 20 Semua saudara di sini menyampaikan salam kepada kalian. Bersalam-salamanl secara mesra sebagai saudara Kristen.
\par 21 Saya tambahkan di sini salam yang saya tulis sendiri: Salam dari saya, Paulus.
\par 22 Orang yang tidak mengasihi Tuhan, biarlah ia terkutuk! Maranatha--Tuhan kami, datanglah!
\par 23 Semoga Tuhan Yesus memberkati Saudara.
\par 24 Teriring kasih kepada Saudara sekalian yang bersatu dengan Kristus Yesus. Hormat kami, Paulus.


\end{document}