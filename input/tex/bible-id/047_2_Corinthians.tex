\begin{document}

\title{2 Korintus}


\chapter{1}

\par 1 Saudara-saudara jemaat Allah di Korintus dan seluruh umat Allah di seluruh Akhaya. Saya, Paulus, rasul Kristus Yesus, yang diangkat atas kehendak Allah, bersama-sama dengan saudara kita Timotius,
\par 2 mengharap semoga Allah Bapa kita dan Tuhan Yesus Kristus memberi berkat dan sejahtera kepada kalian.
\par 3 Terpujilah Allah, Bapa dari Tuhan kita Yesus Kristus. Ia Bapa yang sangat baik hati, dan Ia Allah yang memberikan kekuatan batin kepada manusia.
\par 4 Ia menguatkan batin kami dalam setiap kesukaran yang kami alami, supaya dengan kekuatan yang kami terima dari Allah itu, kami pun dapat menguatkan batin semua orang yang dalam kesusahan.
\par 5 Penderitaan-penderitaan yang dialami oleh Kristus sudah banyak kami alami juga. Dan melalui Kristus pula, batin kami sangat dikuatkan.
\par 6 Kalau kami mengalami kesukaran, itu adalah untuk menguatkan batinmu, demi keselamatanmu. Kalau batin kami dikuatkan, maka kalian juga turut dikuatkan sehingga kalian menjadi tabah dalam kesukaran-kesukaran seperti yang kami pun derita juga.
\par 7 Kami selalu yakin dan tidak pernah ragu-ragu mengenai Saudara, sebab kami tahu bahwa kalian turut menderita dengan kami. Dan oleh karena itu kalian turut juga dikuatkan bersama-sama kami.
\par 8 Saudara-saudara! Kami mau kalian mengetahui tentang kesukaran yang kami alami di wilayah Asia. Penderitaan yang kami tanggung itu terlalu berat menekan kami, sehingga kami sudah tidak mengharap lagi akan hidup;
\par 9 rasanya seperti sudah dijatuhi hukuman mati saja. Tetapi hal itu terjadi supaya kami jangan bersandar pada kekuatan diri sendiri, melainkan pada Allah yang menghidupkan orang mati.
\par 10 Ialah yang sudah menyelamatkan kami dari bahaya kematian yang besar itu. Dan Ialah juga yang akan menyelamatkan kami nanti di hari-hari kemudian, sebab kepada Dialah kami berharap.
\par 11 Kami yakin bahwa dengan doa-doamu untuk kami, Allah akan menyelamatkan kami lagi dari bahaya. Allah berkenan memberkati kami sebagai jawaban atas banyak doa untuk kami, dan karena itu banyak orang akan mengucap terima kasih kepada Allah.
\par 12 Kami bangga, sebab hati nurani kami meyakinkan kami bahwa hidup kami di dunia ini--terutama hubungan kami dengan Saudara--sudah kami jalankan dengan ikhlas dan murni. Kami melakukan itu, bukan dengan kebijaksanaan manusia, tetapi dengan kemampuan yang diberikan Allah.
\par 13 Yang kami tulis kepadamu adalah hanya mengenai hal-hal yang dapat kalian baca dan mengerti. Sekarang kalian tidak mengerti kami sepenuhnya, tetapi saya harap nanti kalian akan betul-betul mengerti kami. Dengan demikian, pada waktu Tuhan Yesus datang nanti, kalian akan bangga atas kami, sebagaimana kami pun bangga atas kalian.
\par 15 Karena keyakinan itulah maka saya pada mulanya merencanakan untuk mengunjungi kalian supaya kalian mendapat berkat dua kali lipat.
\par 16 Saya bermaksud mengunjungi kalian dalam perjalanan saya ke Makedonia dan singgah lagi padamu dalam perjalanan kembali, supaya kalian dapat membantu saya meneruskan perjalanan ke Yudea.
\par 17 Tetapi saya sudah mengubah niat saya itu; nah, apakah itu menunjukkan bahwa pendirian saya tidak teguh? Waktu membuat rencana, apakah saya membuat rencana menurut kemauan saya sendiri, sebentar berkata, "Ya", dan sebentar "Tidak"?
\par 18 Demi Allah yang dapat dipercayai, janji saya kepada Saudara bukanlah "Ya" dan "Tidak".
\par 19 Sebab Yesus Kristus, Anak Allah, yang diberitakan kepadamu oleh Silas, Timotius, dan oleh saya sendiri bukanlah orang "Ya" dan "Tidak". Sebaliknya Ia merupakan jawaban "Ya" dari Allah.
\par 20 Sebab melalui Dia Allah sudah mengatakan "Ya" terhadap semua janji-Nya. Berdasarkan itulah kita mengucapkan "Amin" kepada Allah karena Yesus Kristus. Kami berbuat begitu supaya Allah dimuliakan.
\par 21 Allah sendirilah yang membuat kami dan Saudara teguh bersatu dengan Kristus; Ialah juga yang memilih kita khusus untuk diri-Nya.
\par 22 Untuk itu Allah sudah mensahkan kita menjadi milik-Nya dan memberikan Roh-Nya di dalam hati kita sebagai jaminan bahwa Ia akan memberikan kepada kita semua yang dijanjikan-Nya.
\par 23 Allah saksi saya--sebab Ia tahu isi hati saya--bahwa saya tidak jadi pergi ke Korintus, karena saya tidak mau membuat hatimu sedih.
\par 24 Saya tidak memaksa kalian mengenai apa yang kalian harus percaya, sebab kalian sudah sangat percaya kepada Kristus. Saya hanya bekerjasama dengan kalian supaya kalian makin bahagia.

\chapter{2}

\par 1 Oleh sebab itu saya sudah membuat keputusan untuk tidak datang lagi padamu dengan kunjungan yang membuat kalian sedih.
\par 2 Sebab kalau saya menyedihkan kalian, siapa pula yang dapat menghibur saya, kalau bukan orang-orang itu juga yang sudah saya sedihkan hatinya?
\par 3 Itu sebabnya saya menulis surat itu kepadamu. Saya tidak mau mengunjungimu, lalu disedihkan oleh kalian padahal kalianlah yang seharusnya menggembirakan saya. Sebab saya yakin bahwa kalau saya gembira, maka Saudara semuanya gembira juga.
\par 4 Saya menulis kepadamu dengan hati yang sedih dan berat dan dengan banyak mencucurkan air mata. Maksud saya bukan supaya kalian menjadi sedih, tetapi supaya kalian menyadari bahwa saya sangat mengasihi kalian.
\par 5 Kalau ada orang yang menyebabkan kesedihan, maka hal itu ia lakukan bukan terhadap saya, melainkan terhadap Saudara semuanya atau paling tidak terhadap sebagian dari Saudara. Saya tidak mau terlalu keras terhadap dia.
\par 6 Bagi orang yang semacam itu, hukuman yang dijatuhkan kepadanya oleh kebanyakan di antara Saudara, sudah cukup.
\par 7 Sekarang kalian harus mengampuni dia dan memberi dorongan lagi kepadanya supaya ia jangan terlalu sedih hati sampai putus asa.
\par 8 Sebab itu saya minta, supaya kalian menunjukkan kembali kepadanya bahwa kalian benar-benar mengasihinya.
\par 9 Saya menulis surat itu kepadamu dengan maksud untuk menguji kalian, apakah kalian selalu mau menuruti petunjuk-petunjuk dari saya.
\par 10 Kalau kalian mengampuni orang yang sudah berbuat salah terhadapmu maka saya pun mengampuni orang itu. Sebab apa yang sudah saya ampuni--kalau memang ada yang perlu diampuni--itu saya ampuni di hadapan Kristus untuk kebaikan kalian.
\par 11 Saya lakukan itu supaya Iblis jangan mengambil kesempatan untuk menguasai kita; sebab kita tahu rencana-rencananya.
\par 12 Pada waktu saya sampai di Troas dengan maksud untuk memberitakan Kabar Baik tentang Kristus, Tuhan sudah membuka jalan bagi saya untuk bekerja di tempat itu.
\par 13 Tetapi saya tidak tenang sebab tidak bertemu dengan saudara kita Titus di sana. Maka saya minta diri kepada orang-orang di tempat itu lalu pergi ke Makedonia.
\par 14 Tetapi syukur kepada Allah! Ia selalu memimpin kami untuk ikut dalam pawai kemenangan Kristus karena kami hidup bersatu dengan Dia. Allah memakai kami supaya berita mengenai Kristus tersebar seperti bau harum yang semerbak ke mana-mana.
\par 15 Sebab kami adalah seperti bau kemenyan yang harum, yang dibakar oleh Kristus untuk Allah. Terhadap orang-orang yang sedang menuju kehancuran, kami ini seperti bau kematian yang membunuh; tetapi terhadap orang-orang yang sedang diselamatkan, kami seperti bau harum yang membawa kehidupan. Nah, siapakah yang dapat memenuhi tugas ini?
\par 17 Kami tidak seperti banyak orang yang memperlakukan berita dari Allah seperti barang dagangan saja. Tujuan-tujuan kami adalah murni, sebab kami ditugaskan oleh Allah. Dan Allah sendiri melihat bahwa kami menyampaikan berita itu sebagai hamba-hamba Kristus.

\chapter{3}

\par 1 Apakah ini nampaknya seolah-olah kami memuji-muji diri kami lagi? Atau boleh jadi kami memerlukan surat pujian untuk kalian, atau dari kalian, seperti yang diperlukan oleh orang-orang lain?
\par 2 Saudara sendirilah surat pujian kami, yang tertulis di dalam hati kami dan yang dapat diketahui dan dibaca oleh setiap orang.
\par 3 Mereka sendiri dapat melihat bahwa Saudara merupakan surat yang ditulis Kristus, yang dikirim melalui kami. Surat itu ditulis bukan dengan tinta, tetapi dengan Roh Allah yang hidup; bukan juga di atas batu tulis, tetapi pada hati manusia.
\par 4 Kami berkata begitu karena keyakinan kami terhadap Allah melalui Kristus.
\par 5 Kami tidak punya sesuatu alasan pun untuk menyatakan bahwa kami sanggup melakukan pekerjaan ini, tetapi Allah yang memberi kemampuan itu kepada kami.
\par 6 Ialah yang membuat kami sanggup menjadi pelayan untuk suatu perjanjian yang baru; perjanjian yang bergantung pada Roh Allah, bukan pada hukum yang tertulis. Sebab yang tertulis itu membawa kematian, sedangkan Roh Allah itu memberi hidup.
\par 7 Pada waktu perjanjian yang membawa kematian itu dibuat dan diukir pada batu, cahaya Allah bersinar dengan cemerlang pada muka Musa. Cahayanya begitu gemilang sehingga bangsa Israel tidak sanggup memandang muka Musa, sekalipun cahaya pada mukanya itu sudah mulai pudar pada waktu itu. Nah, kalau pembuatan perjanjian yang membawa kematian itu diresmikan dengan kecemerlangan yang begitu besar,
\par 8 tentu pembuatan perjanjian yang diberikan oleh Roh Allah, diresmikan dengan lebih cemerlang lagi.
\par 9 Kalau perjanjian yang menghukum manusia itu begitu cemerlang, tentu terlebih cemerlang lagi perjanjian yang memungkinkan manusia berbaik dengan Allah.
\par 10 Boleh dikatakan bahwa apa yang dahulu cemerlang, sudah tidak cemerlang lagi karena kecemerlangan yang sekarang ini.
\par 11 Kalau sesuatu yang tahan hanya sementara, begitu cemerlang, tentu sesuatu yang abadi lebih cemerlang lagi.
\par 12 Oleh karena kami mempunyai harapan seperti itu, maka kami berani berbicara begitu.
\par 13 Kami tidak seperti Musa yang menutupi mukanya dengan selubung supaya bangsa Israel tidak melihat cahaya Tuhan sedang padam dan lenyap pada mukanya.
\par 14 Pikiran mereka sudah tertutup. Dan sampai pada hari ini pun pikiran mereka masih tertutup dengan selubung pada waktu mereka membaca buku-buku tentang perjanjian yang lama itu. Selubung itu hanya dapat tersingkap bila orang bersatu dengan Kristus.
\par 15 Sekarang pun, bila mereka membaca buku-buku Musa, selubung itu masih menutupi pikiran mereka.
\par 16 Tetapi kalau seseorang datang menghadap Tuhan, selubung itu pun diangkat dari muka orang itu.
\par 17 Nah, Tuhan yang dimaksudkan di sini adalah Roh. Dan di mana Roh Tuhan ada, di situ juga ada kemerdekaan.
\par 18 Sekarang muka kita semua tidak ditutupi selubung, dan kita memantulkan kecemerlangan Tuhan Yesus. Dan oleh sebab itu kita terus-menerus diubah menjadi seperti Dia; makin lama kita menjadi makin cemerlang. Kecemerlangan itu dari Roh, dan Roh itu adalah Tuhan.

\chapter{4}

\par 1 Kami melakukan pekerjaan ini karena kemurahan hati Allah. Itu sebabnya kami tidak putus asa.
\par 2 Kami tidak memakai cara-cara gelap yang memalukan. Kami tidak mau bekerja dengan licik atau memutarbalikkan perkataan Allah. Kami melayani Allah dengan hati yang murni menurut kehendak-Nya. Sebab itu, kami harapkan semua orang menilai kami dengan baik di dalam hati nuraninya.
\par 3 Kalau Kabar Baik yang kami beritakan itu masih juga belum dipahami, maka hanya orang-orang yang sedang menuju kehancuran sajalah yang tidak memahaminya.
\par 4 Ilah jahat yang menguasai dunia ini menutup pikiran orang-orang yang tidak percaya itu. Ialah yang menghalang-halangi mereka supaya mereka tidak melihat terang dari Kabar Baik itu mengenai kebesaran Kristus, yang merupakan gambaran Allah.
\par 5 Berita yang kami sampaikan itu bukanlah berita tentang kami sendiri. Berita itu adalah berita tentang Yesus Kristus; bahwa Ia adalah Tuhan; dan kami adalah hamba-hambamu karena Dia.
\par 6 Allah yang berkata, "Hendaklah dari dalam gelap terbit terang," Allah itulah juga yang menerbitkan terang itu di dalam hati kita, supaya pikiran kita menjadi terang untuk memahami kecemerlangan Allah yang bersinar pada wajah Kristus.
\par 7 Tetapi harta rohani yang indah itu kami bawa pada diri kami yang tidak berharga ini yang dibuat dari tanah. Dengan demikian nyatalah bahwa kebesaran kuasa itu terletak pada Allah dan bukan pada kami.
\par 8 Kami diserang dari segala pihak, namun kami tidak terjepit. Kami kebingungan, tetapi tidak sampai putus asa.
\par 9 Banyak yang memusuhi kami, tetapi tidak pernah kami tinggal seorang diri. Dan meskipun seringkali kami dipukul sampai jatuh, namun kami tidak mati.
\par 10 Selalu kami merasakan kematian Yesus pada tubuh kami, supaya kehidupan-Nya pun menjadi nyata pada tubuh kami.
\par 11 Selama kami hidup, kami selalu diancam oleh kematian karena Yesus, supaya dengan demikian kehidupan Yesus pun dapat dinyatakan pada tubuh kami yang fana ini.
\par 12 Ini berarti bahwa di dalam diri kami kematian sedang giat menjalankan peranannya, tetapi kami gembira bahwa kehidupanlah yang sedang giat di dalam dirimu.
\par 13 Di dalam Alkitab terdapat ucapan ini, "Saya percaya, itu sebabnya saya berbicara." Nah, dengan semangat percaya yang seperti itu juga, kami berbicara sebab kami percaya.
\par 14 Kami yakin bahwa Allah yang sudah menghidupkan kembali Tuhan Yesus, akan menghidupkan kami juga dengan Yesus, dan membawa kalian dan kami ke hadapan-Nya.
\par 15 Semuanya itu adalah untuk kepentinganmu. Sebab semakin banyak orang mengalami kasih Allah, semakin banyak pula doa syukur yang disampaikan kepada Allah; dengan demikian Allah dimuliakan.
\par 16 Itulah sebabnya kami tidak putus asa. Sekalipun kami secara lahir semakin bertambah rusak, namun secara batin kami dijadikan baru setiap hari.
\par 17 Dan kesusahan yang tidak seberapa ini, yang kami alami untuk sementara, akan menghasilkan bagi kami suatu kebahagiaan yang luar biasa dan abadi. Kebahagiaan itu jauh lebih besar kalau dibandingkan dengan kesusahan itu sendiri.
\par 18 Sebab kami tidak memperhatikan hal-hal yang kelihatan, melainkan yang tidak kelihatan. Yang kelihatan hanya tahan sementara, tetapi yang tidak kelihatan itu kekal sampai selama-lamanya.

\chapter{5}

\par 1 Sebab kami tahu bahwa kalau rumah--yakni tubuh--yang kita diami di dunia ini dibongkar, Allah akan menyediakan bagi kita sebuah rumah di surga, yang dibuat oleh Allah sendiri dan yang tahan selama-lamanya.
\par 2 Di dalam rumah yang sekarang ini, kita mengeluh sebab kita rindu sekali tinggal di dalam rumah kita yang di surga itu.
\par 3 Rumah itulah tubuh kita yang baru.
\par 4 Selama kita tinggal di dalam rumah yang dari dunia ini, kita mengeluh karena beban kita berat. Bukannya karena kita ingin lepas dari tubuh kita yang dari dunia ini, tetapi karena kita ingin mengenakan tubuh yang dari surga itu, supaya tubuh kita yang bisa mati ini dikuasai oleh yang hidup.
\par 5 Yang mempersiapkan kita untuk itu adalah Allah sendiri, dan Ia memberikan Roh-Nya kepada kita sebagai jaminan.
\par 6 Karena itu hati kami selalu merasa kuat. Kami tahu bahwa selama kami masih tinggal di dalam tubuh kami ini, kami jauh dari rumah yang akan kami diami bersama Tuhan.
\par 7 Sebab kami hidup berdasarkan percaya kepada Kristus, bukan berdasarkan apa yang dapat dilihat,
\par 8 itu sebabnya hati kami tabah. Kami lebih suka lepas dari tubuh kami ini, supaya dapat tinggal bersama Tuhan.
\par 9 Karena itu kami berusaha sungguh-sungguh untuk menyenangkan hati-Nya, baik sewaktu kami masih berada di rumah kami di sini, ataupun di sana.
\par 10 Sebab pasti kita semua akan diajukan ke depan pengadilan Kristus, dan masing-masing akan mendapat balasan setimpal dengan perbuatannya di dunia ini--perbuatan baik ataupun jahat.
\par 11 Kami tahu apa artinya takut kepada Tuhan; itu sebabnya kami berusaha meyakinkan orang mengenai diri kami. Allah mengenal kami sedalam-dalamnya, dan saya harap kalian pun mengenal kami di dalam hatimu.
\par 12 Dengan ini kami tidak minta supaya kalian memuji diri kami. Kami hanya mau memberikan suatu alasan yang baik kepadamu untuk menjadi bangga terhadap kami, supaya kalian tahu bagaimana kalian harus menjawab orang-orang yang mementingkan rupa orang dan bukan sifatnya.
\par 13 Kalau kami memang nampaknya sudah gila, itu adalah demi kepentingan Allah. Dan kalau kami nampaknya waras, itu demi kepentinganmu.
\par 14 Kasih Kristuslah yang menguasai kami; dan kami menyadari bahwa kalau satu orang sudah mati untuk semua orang, maka itu berarti bahwa semua orang sudah mati.
\par 15 Kristus mati untuk semua orang, supaya orang-orang yang hidup, tidak lagi hidup untuk diri sendiri, melainkan untuk Kristus yang sudah mati dan dihidupkan kembali demi kepentingan mereka.
\par 16 Oleh karena itu, kami tidak lagi menilai orang menurut ukuran manusia. Memang kami pernah menilai Kristus dari segi manusia, tetapi sekarang tidak lagi.
\par 17 Orang yang sudah bersatu dengan Kristus, menjadi manusia baru sama sekali. Yang lama sudah tidak ada lagi--semuanya sudah menjadi baru.
\par 18 Semuanya itu dikerjakan oleh Allah. Melalui Kristus Allah membuat kita berbaik kembali dengan Dia, lalu menugaskan kita supaya orang-orang lain dimungkinkan berbaik juga dengan Allah.
\par 19 Kami memberitakan bahwa dengan perantaraan Kristus, Allah membuat manusia berbaik kembali dengan diri-Nya. Allah melakukan itu tanpa menuntut kesalahan-kesalahan yang telah dilakukan manusia terhadap diri-Nya. Dan kami sudah ditugaskan Allah untuk memberitakan kabar itu.
\par 20 Jadi kami adalah utusan-utusan Kristus. Melalui kami Allah sendiri yang menyampaikan pesan-Nya. Atas nama Kristus, kami mohon dengan sangat, terimalah uluran tangan Allah yang memungkinkan kalian berbaik dengan Dia.
\par 21 Kristus tidak berdosa, tetapi Allah membuat Dia menanggung dosa kita, supaya kita berbaik kembali dengan Allah karena bersatu dengan Kristus.

\chapter{6}

\par 1 Karena kami bekerja bersama-sama dengan Allah, maka kami mohon dengan sangat janganlah kalian menyia-nyiakan kebaikan hati Allah itu.
\par 2 Dalam Alkitab, Allah berkata, "Pada waktu yang diperkenankan, Aku sudah mendengarkan engkau, dan pada hari keselamatan, Aku telah menolong engkau." Ingatlah baik-baik, sekarang inilah waktu yang diperkenankan itu. Sekarang inilah hari untuk diselamatkan!
\par 3 Kami tidak mau pelayanan kami disalahkan. Karena itu kami berusaha tidak memberi alasan kepada seorang pun untuk melakukan hal itu.
\par 4 Sebaliknya, dalam segala hal, kami menunjukkan bahwa kami adalah hamba-hamba Allah. Sebab, segala macam kesukaran sudah kami derita dengan sabar:
\par 5 kami disiksa, dipenjarakan, dan dikeroyok; kami bekerja keras, sering tidak tidur dan sering pula tidak mempunyai makanan.
\par 6 Dengan berlaku tulus, bijaksana, sabar dan baik hati, kami menunjukkan bahwa kami adalah hamba Allah. Juga dengan bergantung pada pertolongan Roh Allah, dan dengan kasih yang ikhlas,
\par 7 serta dengan memberitakan kabar yang dari Allah dan dengan kuasa Allah, kami menunjukkan bahwa kami ini adalah hamba-hamba Allah. Kami berpegang pada kehendak Allah sebagai senjata kami untuk menyerang maupun untuk membela diri.
\par 8 Kami dihormati, tetapi dihina juga; dipuji, dan difitnah pula. Meskipun kami jujur, kami dituduh pembohong.
\par 9 Kami dianggap tidak terkenal, tetapi kami dikenal oleh semua orang. Kami disangka mati, tetapi nyatanya kami masih hidup. Meskipun kami dianiaya, kami tidak mati.
\par 10 Walaupun hati kami disedihkan, namun kami selalu bergembira juga. Kami nampaknya miskin, tetapi kami sudah membuat banyak orang menjadi kaya. Nampaknya kami tidak mempunyai apa-apa, tetapi sebenarnya kami memiliki segala-galanya.
\par 11 Saudara-saudara yang tercinta di Korintus! Kami sudah berterus terang kepadamu. Semua isi hati kami sudah kami utarakan.
\par 12 Kami tidak menutup hati untuk kalian, hanya kalianlah yang menutup hati terhadap kami.
\par 13 Sekarang, biarlah saya berbicara seperti kepada anak-anak saya sendiri. Bukalah juga hatimu terhadap kami.
\par 14 Janganlah mau menjadi sekutu orang-orang yang tidak percaya kepada Yesus; itu tidak cocok. Mana mungkin kebaikan berpadu dengan kejahatan! Tidak mungkin terang bergabung dengan gelap.
\par 15 Tidak mungkin Kristus sepakat dengan Iblis. Apakah persamaannya antara orang Kristen dengan orang bukan Kristen?
\par 16 Apakah hubungannya antara Rumah Tuhan dengan rumah berhala? Kita ini Rumah Tuhan, yaitu Allah yang hidup. Allah sendiri berkata, "Aku akan tinggal di tengah-tengah mereka, dan hidup bersama-sama mereka. Aku akan menjadi Allah mereka, dan mereka akan menjadi umat-Ku.
\par 17 Sebab itu, tinggalkanlah orang-orang yang tidak mengenal Allah itu, dan pisahkanlah dirimu dari mereka. Janganlah berhubungan sama sekali dengan yang najis, maka Aku akan menerima kamu.
\par 18 Aku akan menjadi Bapamu, dan kamu akan menjadi anak-anak-Ku, demikianlah kata Tuhan Yang Mahakuasa."

\chapter{7}

\par 1 Saudara-saudara yang tercinta! Semua janji itu ditujukan kepada kita. Oleh sebab itu hendaklah kita membersihkan diri dari segala yang mengotori jiwa raga kita. Hendaklah kita takut kepada Allah, supaya kita dapat hidup khusus untuk Dia dengan sempurna.
\par 2 Sambutlah kami di dalam hatimu. Kami tidak bersalah kepada seorang pun, dan tidak pernah ada orang yang kami rugikan. Kami sama sekali tidak mengeruk keuntungan dari siapa pun juga.
\par 3 Saya mengatakan begitu, bukan untuk mempersalahkan kalian. Sebab, seperti yang sudah saya katakan dahulu, kami sangat mengasihi kalian, dan kita adalah kawan-kawan sehidup semati.
\par 4 Saya menaruh kepercayaan penuh kepadamu. Malah saya bangga sekali terhadapmu! Sekalipun kami mengalami banyak kesukaran, hati saya sangat terhibur karena kalian. Hati saya sungguh gembira!
\par 5 Ketika kami tiba di Makedonia pun, kami sama sekali tidak sempat melepaskan lelah. Dari segala segi kami mendapat kesulitan: dari pihak lain pertengkaran, dari dalam hati sendiri ketakutan.
\par 6 Tetapi syukurlah, Allah selalu membesarkan hati orang yang putus asa; Ia menyegarkan hati kami dengan kedatangan Titus.
\par 7 Hati kami terhibur bukan saja karena Titus sudah datang, tetapi juga karena ia melaporkan bagaimana hatinya terhibur oleh kalian. Ia memberitahukan kepada kami bahwa kalian rindu sekali bertemu dengan saya; bahwa kalian menyesal akan perbuatanmu dahulu dan sekarang rela membela saya. Hal itu membuat saya lebih gembira lagi.
\par 8 Meskipun surat saya membuat hatimu menjadi sedih, saya tidak menyesal menulis surat itu. Memang pada waktu saya melihat bahwa surat saya itu menjadikan kalian sedih--meskipun kesedihanmu itu hanya sementara--saya agak menyesal juga.
\par 9 Tetapi saya senang sekarang--bukan karena hatimu menjadi sedih, melainkan karena kesedihanmu itu membuat kelakuanmu berubah. Memang kesedihanmu itu sejalan dengan kehendak Allah. Jadi, kami tidak merugikan kalian.
\par 10 Sebab kesedihan seperti itu menghasilkan perubahan hati yang mendatangkan keselamatan. Dan orang tidak akan menyesal atas hal itu. Sebaliknya, kesedihan yang hanya sejalan dengan kehendak manusia menghasilkan kematian.
\par 11 Coba kalian perhatikan apa hasilnya padamu oleh kesedihan yang sejalan dengan kehendak Allah! Hasilnya ialah kalian sungguh-sungguh berusaha untuk menjernihkan kekeruhan! Kalian menjadi benci terhadap dosa, kalian takut, kalian rindu, kalian menjadi bersemangat, kalian rela menghukum yang bersalah! Dalam seluruh persoalan ini kalian sudah menunjukkan bahwa kalian tidak bersalah.
\par 12 Jadi, meskipun saya sudah menulis surat itu, saya menulis bukan karena orang yang bersalah itu. Bukan juga karena orang yang menderita oleh sebab kesalahan itu. Saya menulis surat itu supaya di hadapan Allah, kalian menyadari sendiri betapa besarnya perhatianmu terhadap kami.
\par 13 Itulah sebabnya hati kami terhibur. Selain terhibur, kami lebih-lebih lagi digembirakan karena melihat kegembiraan hati Titus; hatinya terhibur oleh Saudara sekalian.
\par 14 Saya memang sudah membangga-banggakan kalian kepadanya. Syukurlah, kalian tidak memalukan saya. Semua yang pernah kami katakan kepadamu adalah benar. Begitu juga apa yang kami bangga-banggakan tentang kalian kepada Titus ternyata benar pula.
\par 15 Sekarang ia semakin mengasihi kalian, sebab ia ingat bagaimana Saudara semuanya mau menuruti pimpinannya dan bagaimana kalian menerima dia dengan hormat dan patuh.
\par 16 Saya senang sekali, sebab kalian dapat dipercayai dalam segala hal.

\chapter{8}

\par 1 Saudara-saudara, kami ingin kalian mengetahui juga tentang bagaimana baiknya Allah kepada jemaat-jemaat di Makedonia.
\par 2 Mereka sudah diuji dengan kesukaran-kesukaran yang berat. Tetapi di tengah-tengah kesukaran-kesukaran itu, mereka bergembira dan sangat murah hati dalam memberikan sumbangan untuk menolong orang lain, meskipun mereka miskin sekali.
\par 3 Saya dapat memastikan bahwa mereka memberi semampu mereka, bahkan lebih. Tanpa ada yang menyuruh,
\par 4 mereka minta dengan sangat kepada kami kalau boleh mereka ikut membantu memberi sumbangan kepada umat Allah di Yudea.
\par 5 Mereka memberi jauh lebih dari yang kami harapkan. Mereka mula-mula menyerahkan diri kepada Tuhan, kemudian kepada kami juga, sesuai dengan kehendak Allah.
\par 6 Itu sebabnya kami sangat menganjurkan Titus--yang memulai usaha ini--supaya ia melanjutkan usaha yang baik ini di antara kalian juga.
\par 7 Kalian unggul dalam segala-galanya: Kalian unggul dalam hal percaya, dalam hal menyatakan pendapat, dalam hal pengetahuan, dalam segala macam usaha, dan dalam kasihmu kepada kami. Sebab itu, baiklah kalian juga unggul di dalam usaha yang baik ini.
\par 8 Saya tidak menganjurkan itu sebagai suatu perintah. Tetapi dengan menunjukkan betapa giatnya orang lain menolong sesamanya, saya juga ingin tahu sampai di mana kasihmu.
\par 9 Sebab kalian mengetahui betul bahwa kita sangat dikasihi oleh Yesus Kristus Tuhan kita. Ia kaya, tetapi Ia membuat diri-Nya menjadi miskin untuk kepentinganmu, supaya dengan kemiskinan-Nya itu, kalian menjadi kaya.
\par 10 Menurut pendapat saya, pada tempatnyalah kalian menyelesaikan apa yang sudah kalian mulai tahun lalu. Sebab kalianlah yang pertama-tama memikirkan dan memulai usaha ini.
\par 11 Nah, sekarang hendaklah kalian melanjutkannya menurut kemampuanmu. Hendaknya kalian bersemangat untuk menyelesaikan usaha itu, sebagaimana kalian dahulu pun bersemangat merencanakannya.
\par 12 Kalau kalian rela memberi, maka Allah akan menerima pemberianmu itu berdasarkan apa yang ada padamu, bukan berdasarkan apa yang tidak ada padamu.
\par 13 Saya sama sekali tidak bermaksud untuk membebaskan orang lain dari tanggung jawab, dan memberatkan kalian. Tetapi karena kalian sekarang ini dalam keadaan serba cukup, maka sudah sepatutnyalah kalian mencukupi kekurangan mereka. Nanti kalau kalian berkekurangan, dan mereka dalam keadaan serba cukup, maka mereka akan membantu kalian. Dengan demikian kedua-duanya sama-sama dilayani.
\par 15 Dalam Alkitab tertulis, "Orang yang berpenghasilan banyak, tidak berkelebihan, dan orang yang berpenghasilan sedikit, tidak berkekurangan."
\par 16 Betapa syukurnya kami kepada Allah sebab Ia sudah membuat Titus segiat kami untuk menolong kalian!
\par 17 Sebab Titus bukan saja rela memenuhi permintaan kami, tetapi ia juga begitu giat hendak menolong kalian sampai atas kemauannya sendiri ia memutuskan untuk pergi kepadamu.
\par 18 Bersama dengan dia kami mengirim juga seorang saudara yang sangat dihormati di semua jemaat karena pekerjaannya memberitakan Kabar Baik dari Allah.
\par 19 Di samping itu, orang tersebut sudah dipilih juga dan ditentukan oleh jemaat-jemaat untuk menemani kami dalam perjalanan kami. Sebab kami akan mengadakan perjalanan untuk menyampaikan sumbangan itu, supaya Tuhan dimuliakan dan supaya kalian melihat bahwa kami rela menolong.
\par 20 Kami sangat berhati-hati supaya jangan ada orang yang menyalahkan kami dalam hal menyampaikan sumbangan yang besar ini.
\par 21 Kami mau bertindak jujur bukan hanya di hadapan Tuhan, tetapi di hadapan manusia juga.
\par 22 Bersama-sama dengan Titus dan saudara tersebut, kami juga mengirim seorang saudara yang lain. Kami sudah menguji dia berkali-kali dan ternyata ia selalu suka menolong. Dan sekarang, ia lebih-lebih mau menolong kalian sebab ia menaruh harapan yang besar padamu.
\par 23 Mengenai Titus, ia adalah rekan saya, yang bekerja dengan saya untuk menolong kalian. Dan mengenai kedua saudara itu, yang pergi bersama dia, mereka adalah orang-orang yang diutus oleh jemaat-jemaat dan menjadi kebanggaan bagi Kristus.
\par 24 Kami harap kalian menunjukkan kasihmu kepada mereka, supaya semua jemaat tahu bahwa kalian mengasihi mereka dan bahwa apa yang kami banggakan mengenai kalian adalah benar.

\chapter{9}

\par 1 Sebenarnya tidak perlu lagi saya menulis kepada kalian tentang bantuan yang sedang dikirim kepada umat Allah di Yudea.
\par 2 Saya tahu kalian suka menolong, dan saya sudah membanggakan kalian kepada orang-orang di Makedonia. Saya mengatakan kepada mereka, bahwa sejak tahun yang lalu saudara-saudara di Akhaya sudah siap untuk memberikan sumbangan. Dan semangatmu sudah mengobarkan juga semangat sebagian besar dari mereka.
\par 3 Sekarang saya mengutus Titus dan kedua saudara itu kepadamu supaya apa yang kami bangga-banggakan mengenai kalian jangan merupakan omong kosong belaka. Jadi, kalian hendaknya sudah siap, sebab saya mengatakan kepada mereka bahwa kalian sudah siap untuk itu.
\par 4 Saya khawatir kalau-kalau ada orang-orang Makedonia yang ikut bersama-sama saya nanti pada waktu saya datang kepadamu. Dan kalau mereka mendapati bahwa kalian belum siap, wah, alangkah malunya kami nanti, karena sudah mengatakan bahwa kami percaya sekali kepadamu! Dan kalian sendiri pun akan malu juga.
\par 5 Itulah sebabnya saya merasa perlu mengutus mereka itu kepadamu terlebih dahulu untuk mengurus sumbangan yang kalian janjikan itu. Saya lakukan itu supaya nanti kalau saya datang, pemberian kalian itu sudah siap. Dengan demikian, nyatalah bahwa pemberianmu itu diberikan dengan senang hati, dan bukan karena terpaksa.
\par 6 Ingatlah! Orang yang menabur benih sedikit-sedikit akan memungut hasil yang sedikit juga. Tetapi orang yang menabur benih banyak-banyak akan memungut hasil yang banyak juga.
\par 7 Setiap orang harus memberi menurut kerelaan hatinya. Janganlah ia memberi dengan segan-segan atau karena terpaksa, sebab Allah mengasihi orang yang memberi dengan senang hati.
\par 8 Allah berkuasa memberi kepada kalian berkat yang melimpah ruah, supaya kalian selalu mempunyai apa yang kalian butuhkan; bahkan kalian akan berkelebihan untuk berbuat baik dan beramal.
\par 9 Dalam Alkitab tertulis begini tentang Allah, "Ia menghambur-hamburkan kepada orang miskin; kebaikan hati-Nya kekal selama-lamanya."
\par 10 Allah yang menyediakan benih untuk si penabur dan makanan untuk kita. Ia juga akan menyediakan dan memperbanyak apa yang kalian tabur, supaya hasil kemurahan hatimu bertambah pula.
\par 11 Dengan demikian kalian akan serba cukup dalam segala hal sehingga kalian selalu dapat memberi dengan murah hati. Dan pemberian-pemberianmu yang kami bagi-bagikan, menyebabkan banyak orang mengucap terima kasih kepada Allah.
\par 12 Sebab perbuatan baik yang kalian lakukan ini bukan hanya akan mencukupi kekurangan umat Allah saja, tetapi juga akan menyebabkan banyak orang berterima kasih kepada Allah.
\par 13 Dan dari kebajikan yang kalian tunjukkan itu, banyak orang Kristen di Yudea akan memuji Allah karena mereka melihat kesetiaanmu terhadap Kabar Baik tentang Kristus yang kalian ikuti itu. Juga, mereka memuji Allah, karena kalian sangat murah hati untuk membagi-bagikan kepada mereka dan kepada semua orang lain apa yang ada padamu.
\par 14 Maka mereka akan mendoakan kalian dengan perasaan kasih sayang karena Allah sangat baik hati kepadamu.
\par 15 Hendaklah kita bersyukur kepada Allah atas pemberian-Nya yang luar biasa itu!

\chapter{10}

\par 1 Sekarang saya, Paulus, mau mengajukan sesuatu kepadamu. Saya ini, menurut kata orang, lembut kalau berhadapan dengan kalian, tetapi berani kalau berjauhan. Nah, karena kelembutan dan kebaikan hati Kristus,
\par 2 saya mohon dengan sangat janganlah membuat saya terpaksa berlaku keras terhadap kalian bila saya datang nanti. Sebab saya berniat untuk bertindak tegas terhadap orang-orang yang mengatakan bahwa kami bekerja dengan tujuan-tujuan duniawi.
\par 3 Kami memang masih hidup di dalam dunia, tetapi kami tidak berjuang berdasarkan tujuan duniawi.
\par 4 Senjata-senjata yang kami gunakan di dalam perjuangan kami bukannya senjata dunia ini, tetapi senjata-senjata Allah yang berkuasa. Dengan senjata-senjata itu kami menghancurkan pertahanan-pertahanan; kami menangkis perdebatan-perdebatan
\par 5 dan mendobrak benteng-benteng kesombongan yang dibangun untuk menentang pengetahuan tentang Allah. Kami menawan pikiran orang-orang dan membuat mereka takluk kepada Kristus.
\par 6 Dan kalau kalian sudah taat dengan sepenuhnya, kami siap untuk menghukum semua orang yang tidak taat.
\par 7 Hendaknya kalian menyadari keadaan yang sebenarnya. Kalau seseorang yakin bahwa ia milik Kristus, hendaklah ia mengingatkan dirinya sendiri dan menyadari bahwa kami juga milik Kristus seperti dia.
\par 8 Saya tidak malu kalau saya harus lebih banyak membangga-banggakan kekuasaan yang telah diberikan Tuhan kepada kami, sebab kekuasaan itu diberikan untuk membangun kalian, bukannya untuk meruntuhkan.
\par 9 Saya tidak mau kalian merasa bahwa saya menakut-nakuti kalian dengan surat-surat saya.
\par 10 Orang berkata, "Surat-surat Paulus itu tegas dan berwibawa, tetapi kalau ia sendiri berada di tengah-tengah kita, pribadinya lemah dan kata-katanya tidak berarti apa-apa!"
\par 11 Hendaknya orang yang seperti itu menyadari bahwa kalau kami berada di tengah-tengah kalian, tindakan kami tepat seperti apa yang kami tulis dalam surat-surat kami.
\par 12 Tentu saja kami tidak berani membandingkan atau menempatkan diri kami sederajat dengan orang-orang yang menganggap dirinya tinggi. Alangkah bodohnya mereka! Mereka membuat ukuran sendiri dan menilai diri sendiri dengan ukuran itu.
\par 13 Kami tidak begitu. Kalau kami berbangga-bangga, kami tidak melampaui batas. Kami tinggal di dalam batas-batas daerah pekerjaan yang sudah ditentukan oleh Allah untuk kami. Dan kalian termasuk di dalam daerah itu.
\par 14 Jadi, ketika kami mengunjungi kalian dan membawa Kabar Baik tentang Kristus kepadamu, kami sesungguhnya tidak keluar dari batas-batas daerah pekerjaan kami.
\par 15 Kami tidak memuji diri atas pekerjaan orang lain yang berada di luar batas yang ditentukan Allah untuk kami. Sebaliknya kami tetap di dalam batas-batas yang ditentukan Allah untuk kami dan berharap bahwa kalian semakin percaya kepada Kristus, supaya kami dapat melakukan pekerjaan yang lebih besar di antara kalian.
\par 16 Dengan demikian kami dapat mengabarkan juga Kabar Baik itu di negeri-negeri lain tanpa harus membesar-besarkan diri atas pekerjaan yang telah dilakukan di wilayah pelayanan orang lain.
\par 17 Di dalam Alkitab tertulis, "Orang yang mau membanggakan sesuatu, hendaklah membanggakan apa yang telah dibuat oleh Tuhan."
\par 18 Sebab orang yang terpuji adalah orang yang dipuji oleh Tuhan, bukan orang yang memuji dirinya sendiri.

\chapter{11}

\par 1 Perkenankanlah saya berlaku agak bodoh sedikit. Boleh, bukan?
\par 2 Saya rindu kepadamu, seperti Allah sendiri pun rindu kepadamu. Kalian adalah seperti seorang gadis perawan yang masih suci yang sudah saya janjikan untuk dinikahkan dengan seorang suami, yaitu Kristus.
\par 3 Tetapi saya khawatir pikiranmu akan tergoda untuk tidak setia lagi kepada Kristus, sama seperti Hawa dahulu juga tergoda oleh kelicikan si ular.
\par 4 Sebab rupanya kalian senang saja menerima orang yang datang kepadamu dan mengajar tentang Yesus yang lain--bukan Yesus yang kami perkenalkan kepadamu. Dan kalian mau juga menyambut roh dan "kabar baik" yang sama sekali berlainan dengan Roh Allah dan Kabar Baik yang pernah kalian terima dari kami.
\par 5 Saya sama sekali tidak merasa lebih rendah dari "rasul-rasul" yang luar biasa itu!
\par 6 Mungkin saya kurang pandai berbicara, tetapi mengenai pengetahuan, saya bukan orang yang bodoh. Itu sudah kami buktikan kepadamu dalam segala hal.
\par 7 Pada waktu saya memberitakan Kabar Baik dari Allah kepadamu, saya tidak minta kalian membiayai saya sedikit pun; dan dengan demikian saya membuat diri saya menjadi rendah. Saya lakukan itu untuk meninggikan kalian. Apakah itu suatu kesalahan saya terhadapmu?
\par 8 Pada waktu saya melayani kalian, saya dibiayai oleh jemaat-jemaat lain. Boleh dikatakan saya merugikan jemaat-jemaat itu supaya dapat menolong kalian.
\par 9 Kalau pada waktu itu saya berkekurangan, saya tidak pernah menyusahkan seorang pun dari kalian. Semua yang saya perlukan disediakan oleh saudara-saudara yang datang dari Makedonia. Saya menjaga baik-baik supaya saya tidak menyusahkan kalian dalam hal apa pun, dan saya akan terus menjaga supaya itu tetap demikian.
\par 10 Kebanggaan saya ini tidak dapat dihapuskan di mana pun juga di seluruh negeri Akhaya, karena ajaran Kristus yang benar ada pada saya.
\par 11 Mengapa saya berkata begitu? Apakah oleh sebab saya tidak mengasihi kalian? Allah tahu saya mengasihi!
\par 12 Apa yang saya lakukan sekarang akan terus saya lakukan supaya "rasul-rasul" yang lain itu tidak punya alasan untuk membesar-besarkan diri dan berkata bahwa mereka bekerja seperti kami.
\par 13 Orang-orang seperti itu adalah rasul-rasul palsu. Mereka pekerja-pekerja yang mengelabui orang dengan menyamar sebagai rasul-rasul Kristus.
\par 14 Tidak heran mereka berbuat begitu, sebab Iblis sendiri pun menyamar sebagai malaikat terang!
\par 15 Jadi tidak aneh juga kalau pelayan-pelayan Iblis menyamar sebagai pelayan-pelayan yang melakukan kehendak Allah. Akhirnya mereka juga akan menerima balasan yang sepadan dengan segala perbuatan mereka.
\par 16 Saya ulangi sekali lagi: Jangan sampai ada yang menganggap saya bodoh. Tetapi kalau kalian toh menganggap saya begitu, perkenankanlah saya yang bodoh ini berbangga juga sedikit.
\par 17 Berikut ini akan saya sebutkan kebanggaan saya, tetapi bukan Tuhan yang menyuruh saya mengatakan itu. Dalam hal ini saya sungguh-sungguh berbicara seperti orang bodoh.
\par 18 Memang ada banyak orang yang membanggakan hal-hal keduniaan, jadi saya mau berbangga-bangga juga.
\par 19 Kalian dengan senang hati bersabar terhadap orang yang bodoh, karena kalian merasa diri begitu pandai!
\par 20 Kalian membiarkan saja kalau orang memperbudak kalian, atau memeras dan mengambil keuntungan dari kalian, atau merasa diri lebih tinggi dari kalian dan berani menampar kalian.
\par 21 Saya malu mengakui bahwa kami terlalu lemah untuk berbuat seperti itu. Tetapi apa yang orang lain berani banggakan, saya berani juga! (Saya berbicara seperti orang bodoh.)
\par 22 Orang Ibranikah mereka? Saya juga orang Ibrani. Orang Israelkah mereka? Saya juga orang Israel. Keturunan Abrahamkah mereka? Saya pun begitu!
\par 23 Pelayan-pelayan Kristuskah mereka? Kedengarannya seperti saya sudah hilang akal, tetapi saya memang pelayan yang lebih baik dari mereka semuanya! Saya bekerja lebih keras, saya lebih sering dimasukkan ke dalam penjara, saya lebih banyak disiksa dan sering hampir mati.
\par 24 Sudah lima kali saya disiksa oleh orang Yahudi dengan pukulan cambuk tiga puluh sembilan kali.
\par 25 Tiga kali saya dicambuk oleh orang-orang Roma; pernah pula saya dilempari dengan batu. Tiga kali saya mengalami karam kapal di laut, dan sekali saya terapung-apung di laut selama dua puluh empat jam.
\par 26 Banyak kali saya mengadakan perjalanan yang berbahaya: diancam bahaya banjir, bahaya perampok, bahaya dari pihak Yahudi maupun dari pihak bukan Yahudi, bahaya di dalam kota, bahaya di luar kota, bahaya di laut, dan bahaya dari orang-orang yang mengemukakan diri sebagai saudara Kristen padahal bukan.
\par 27 Saya membanting tulang dan berjuang setengah mati: sering tidak tidur, tidak makan, tidak minum, banyak kali terlantar dalam keadaan lapar, kedinginan karena kurang pakaian dan tidak mempunyai tempat tinggal.
\par 28 Di samping semuanya itu, setiap hari saya cemas juga akan keadaan semua jemaat.
\par 29 Bila ada yang lemah saya pun turut merasa lemah juga. Bila ada yang jatuh ke dalam dosa, hati saya turut hancur.
\par 30 Nah, kalau saya harus membanggakan sesuatu, maka saya membanggakan hal-hal yang menunjukkan kelemahan saya.
\par 31 Allah, Bapa dari Tuhan Yesus tahu bahwa saya tidak berdusta. Terpujilah nama-Nya selama-lamanya.
\par 32 Ketika saya berada di Damsyik, gubernur yang berkuasa di situ di bawah pemerintahan Raja Aretas, menyuruh tentara menjaga pintu kota itu untuk menangkap saya.
\par 33 Tetapi dengan sebuah keranjang saya diulurkan ke bawah melalui suatu lubang pada tembok. Demikianlah saya lolos dari tangan gubernur itu.

\chapter{12}

\par 1 Memang tidak ada untungnya untuk berbangga. Tetapi saya mau juga membanggakan hal-hal yang Allah perlihatkan kepada saya dalam wahyu atau dalam penglihatan.
\par 2 Saya mengenal seorang Kristen yang empat belas tahun yang lalu diangkat ke tempat yang tertinggi di surga. (Saya tidak tahu apakah tubuhnya benar-benar terangkat atau itu hanya suatu penglihatan--Allah sajalah yang tahu.)
\par 3 Saya ulangi sekali lagi: Saya tahu bahwa orang ini diangkat masuk ke Firdaus. (Saya tidak tahu apakah tubuhnya benar-benar terangkat ataukah itu hanya suatu penglihatan--Allah sajalah yang tahu.) Di sana orang itu mendengarkan hal-hal yang tidak dapat diungkapkan oleh manusia dan tidak juga diizinkan kepada manusia untuk mengucapkannya.
\par 5 Tentang orang itulah yang mau saya banggakan, bukan tentang diri saya sendiri. Mengenai diri saya, hanya hal-hal yang menunjukkan kelemahan saya itulah, yang mau saya banggakan.
\par 6 Seandainya saya ingin juga membanggakan sesuatu, saya tidak mau menjadi pembual yang omong kosong; saya akan mengatakan yang benar. Tetapi saya menahan diri, supaya tidak ada orang yang menganggap saya lebih daripada apa yang sudah ia lihat saya lakukan atau yang sudah ia dengar saya katakan.
\par 7 Tetapi supaya saya jangan terlalu sombong karena penglihatan-penglihatan yang luar biasa itu, saya diberikan semacam penyakit pada tubuh saya yang merupakan alat Iblis. Penyakit itu diberikan untuk memukul saya supaya saya tidak menjadi sombong.
\par 8 Tiga kali saya berdoa kepada Tuhan supaya penyakit itu diangkat dari saya.
\par 9 Tetapi Tuhan menjawab, "Aku mengasihi engkau dan itu sudah cukup untukmu; sebab kuasa-Ku justru paling kuat kalau kau dalam keadaan lemah." Itu sebabnya saya lebih senang membanggakan kelemahan-kelemahan saya, sebab apabila saya lemah, maka justru pada waktu itulah saya merasakan Kristus melindungi saya dengan kekuatan-Nya.
\par 10 Jadi saya gembira dengan kelemahan-kelemahan saya. Saya juga gembira kalau oleh karena Kristus saya difitnah, saya mengalami kesulitan, dikejar-kejar dan saya mengalami kesukaran. Sebab kalau saya lemah, maka pada waktu itulah justru saya kuat.
\par 11 Sungguh saya sudah berlaku seperti orang bodoh--tetapi kalianlah yang membuat saya menjadi begitu. Sebenarnya kalianlah yang harus memuji saya. Sebab meskipun saya tidak berarti apa-apa, saya sama sekali tidak kalah terhadap "rasul-rasul" yang luar biasa itu!
\par 12 Keajaiban-keajaiban dan hal-hal luar biasa serta pekerjaan-pekerjaan yang besar-besar sudah diperlihatkan dengan sabar kepadamu untuk membuktikan bahwa saya seorang rasul.
\par 13 Dalam hal apakah kalian kurang diperhatikan, dibanding dengan jemaat-jemaat yang lain? Paling-paling hanya dalam hal ini: bahwa saya tidak menyusahkan kalian untuk membiayai saya. Maafkan saya atas kesalahan itu!
\par 14 Sekarang ini sudah untuk ketiga kalinya saya siap untuk mengunjungi kalian. Dan saya tidak mau menyusahkan kalian, sebab yang saya inginkan bukanlah apa yang kalian miliki, melainkan dirimu. Karena bukanlah anak-anak yang harus mencari nafkah untuk orang tuanya, tetapi orang tualah yang harus mencari nafkah untuk anak-anaknya.
\par 15 Karena itu, dengan senang hati saya rela mengurbankan segala-galanya untukmu, bahkan diri saya sendiri pun. Kalau saya begitu mengasihi kalian, apakah patut kalian kurang mengasihi saya?
\par 16 Nah, kalian setuju bahwa saya tidak pernah menyusahkan kalian. Namun ada yang berkata bahwa saya ini licik; bahwa saya mendapat keuntungan dari kalian karena tipu daya saya.
\par 17 Mana mungkin! Apakah melalui orang-orang yang saya utus kepadamu itu saya mengeruk keuntungan dari kalian?
\par 18 Saya sudah menyuruh Titus pergi mengunjungi kalian, dan aku menyuruh saudara Kristen yang lain itu pergi bersama-sama dengan dia. Apakah Titus mengambil keuntungan dari kalian? Tetapi kami berdua bekerja dengan tujuan yang sama dan bertindak dengan cara-cara yang sama!
\par 19 Boleh jadi kalian mengira kami selama ini sedang berusaha membela diri terhadap kalian? Kalian keliru sekali! Allah mengetahui bahwa semua yang kami katakan itu adalah menurut kehendak Kristus. Dan semua yang kami lakukan adalah untuk membangun kehidupan rohanimu.
\par 20 Saya khawatir bahwa pada waktu saya mengunjungi kalian nanti, saya dapati kalian tidak seperti apa yang saya harapkan, dan kalian pun akan mendapati saya tidak seperti yang kalian harapkan. Jangan-jangan nanti ada yang berkelahi, iri hati, panas hati, mementingkan diri sendiri, fitnah-memfitnah, hasut-menghasut, sombong dan mengacau.
\par 21 Saya takut kalau-kalau pada waktu saya datang nanti Allah akan merendahkan saya di hadapan kalian dan saya akan menangis karena banyak di antara kalian yang berdosa dahulu, tidak berubah dan tidak berhenti melakukan perbuatan-perbuatan mereka yang cabul, kotor dan tidak pantas.

\chapter{13}

\par 1 Ini adalah untuk ketiga kalinya saya datang mengunjungi kalian. Dalam Alkitab tertulis, "Setiap perkara harus didukung oleh kesaksian dua atau tiga orang saksi, baru perkara itu sah."
\par 2 Orang-orang yang di waktu lalu sudah berbuat dosa dan semua orang lainnya, telah saya peringatkan terlebih dahulu ketika saya mengunjungi kalian pada kedua kalinya. Sekarang, sementara saya berjauhan denganmu, saya memperingatkan kembali bahwa kalau saya datang lagi, tidak seorang pun dari mereka yang akan terlepas dari hukuman.
\par 3 Maka kalian akan mendapatkan semua bukti yang kalian inginkan bahwa Kristus benar-benar berbicara melalui saya. Kalau Kristus bertindak terhadap kalian, Ia tidak bertindak lemah-lemah, sebaliknya Ia menunjukkan kuasa-Nya di antaramu.
\par 4 Memang pada waktu Ia disalibkan, Ia lemah, namun sekarang Ia hidup karena kuasa Allah. Kami juga lemah dalam hidup kami yang bersatu dengan Kristus, tetapi bila menghadapi kalian, kami pun kuat bersama Kristus karena kuasa Allah.
\par 5 Coba kalian menguji diri sendiri apakah kalian betul-betul hidup berdasarkan percaya kepada Kristus! Pasti kalian menyadari bahwa Kristus Yesus ada di dalam kalian! --lain halnya kalau memang kalian tidak sungguh-sungguh percaya.
\par 6 Saya harap kalian menyadari bahwa kami bukannya orang yang tidak tahan uji.
\par 7 Kami berdoa kepada Allah semoga Ia mau menolong kalian untuk tidak berbuat salah. Bukannya untuk menunjukkan bahwa kami ini memang cakap menunaikan tugas, tetapi supaya kalian melakukan apa yang Allah kehendaki--biar nampaknya kami gagal, tidak mengapa!
\par 8 Sebab kami tidak dapat berbuat sesuatu pun yang bertentangan dengan yang benar; kami harus menuruti yang benar.
\par 9 Kami senang kalau kami lemah, dan kalian kuat. Kami berdoa juga supaya kalian menjadi sempurna.
\par 10 Itulah sebabnya saya menulis surat ini pada waktu saya belum berada di tengah-tengah kalian. Dengan demikian, kalau saya datang nanti, tidak usah saya bertindak keras terhadapmu dengan menggunakan kekuasaan yang diberikan Tuhan kepada saya. Kekuasaan itu diberikan untuk membangun kalian, bukan untuk menghancurkan.
\par 11 Akhirnya, Saudara-saudara, hendaklah kalian bergembira, dan berusahalah menjadi sempurna. Terimalah segala nasihat saya. Hendaklah kalian sehati dan hidup rukun. Allah Yang Mahakasih dan pendamai itu akan menyertai kalian.
\par 12 Bersalam-salamanlah satu sama lain secara mesra sebagai saudara Kristen.
\par 12 Bersalam-salamanlah satu sama lain secara mesra sebagai saudara Kristen.
\par 13 Tuhan Yesus Kristus memberkati kalian, Allah mengasihi kalian, dan Roh Allah menyertai kalian semuanya! Hormat kami, Paulus.


\end{document}