\begin{document}

\title{Titus}


\chapter{1}

\par 1 Anakku Titus! Saya, Paulus, menulis surat ini sebagai hamba Allah dan rasul Yesus Kristus. Saya dipilih dan diutus untuk menolong orang-orang pilihan Allah supaya menjadi kuat dalam iman mereka. Saya juga harus membimbing mereka supaya mengenal ajaran yang benar yang diajarkan oleh agama kita.
\par 2 Ajaran itu berdasarkan harapan bahwa kita akan menerima hidup sejati dan kekal. Allah sudah menjanjikan hidup itu sebelum permulaan zaman--dan Allah tidak mungkin berdusta.
\par 3 Pada waktu yang tepat, Allah menyatakan janji-Nya itu dalam pesan-Nya. Pesan itu sudah dipercayakan kepada saya, dan saya memberitakannya atas perintah Allah Penyelamat kita.
\par 4 Titus, engkau adalah sungguh-sungguh anak saya, sebab engkau dan saya sama-sama sudah percaya kepada Kristus. Saya mengharap semoga Allah Bapa dan Kristus Yesus Raja Penyelamat kita, memberi berkat dan sejahtera kepadamu.
\par 5 Saya meninggalkan engkau di Kreta supaya engkau dapat mengurus hal-hal yang masih perlu diatur. Juga, supaya engkau mengangkat pemimpin-pemimpin jemaat di setiap kota. Dan ingatlah akan petunjuk-petunjuk saya ini:
\par 6 Seorang pemimpin jemaat hendaklah seorang yang tanpa cela; ia harus mempunyai hanya seorang istri; anak-anaknya harus sudah percaya kepada Kristus dan bukan yang dikenal sebagai anak berandal dan yang tidak bisa diatur.
\par 7 Penilik jemaat adalah orang yang mengurus pekerjaan Allah, ia tidak boleh bercela. Ia tidak boleh sombong atau pemarah, atau pemabuk, atau suka berkelahi, atau mata duitan.
\par 8 Ia harus senang menerima orang di rumahnya, suka akan hal-hal yang baik, dapat menahan diri, jujur, suci dan tertib.
\par 9 Ia harus berpegang teguh pada ajaran yang dapat dipercaya, seperti yang sudah diajarkan kepadanya. Dengan demikian ia sanggup menasihati orang berdasarkan ajaran yang benar, dan menunjukkan kesalahan orang-orang yang menentangnya.
\par 10 Sebab banyak orang yang suka memberontak, terutama orang-orang yang tadinya beragama Yahudi; mereka menipu orang lain dengan omong kosong mereka.
\par 11 Orang-orang seperti itu harus ditutup mulutnya, sebab mereka mengacaukan banyak keluarga dengan ajaran-ajaran yang tidak-tidak. Mereka melakukan itu hanya karena mau mencari untung yang tidak pantas!
\par 12 Pernah seorang nabi mereka sendiri, yang berasal dari Kreta juga, berkata, "Orang-orang Kreta selalu berbohong, dan seperti binatang buas yang rakus dan pemalas."
\par 13 Apa yang dikatakan oleh nabi itu memang benar. Itu sebabnya engkau harus menegur mereka dengan tegas; supaya mereka tetap berpegang pada ajaran yang benar
\par 14 dan tidak lagi berpegang pada dongeng-dongeng Yahudi atau peraturan-peraturan yang dibuat oleh orang-orang yang menolak ajaran yang benar itu.
\par 15 Segala sesuatu adalah suci bagi orang-orang yang suci. Tetapi bagi orang-orang yang pikirannya kotor dan yang tidak beriman, tidak ada sesuatu pun yang suci, sebab pikiran dan hati nurani mereka sudah kotor!
\par 16 Mereka berkata bahwa mereka mengenal Allah, padahal perbuatan mereka menyangkal-Nya. Mereka menjijikkan, dan mereka tidak mau taat; mereka adalah orang-orang yang tidak mampu melakukan sesuatu yang baik.

\chapter{2}

\par 1 Tetapi engkau, Titus, hendaklah engkau mengajarkan ajaran yang tepat.
\par 2 Nasihatilah orang laki-laki yang tua, supaya mereka menahan diri, bijaksana dan hidup sebagai orang yang patut dihormati. Mereka harus juga berpegang pada ajaran yang benar dari Allah, mengasihi dengan sempurna dan menderita dengan tabah.
\par 3 Begitu juga hendaklah engkau menasihati wanita-wanita yang tua supaya kelakuan mereka sesuai dengan apa yang patut bagi orang yang hidup khusus bagi Allah. Mereka tidak boleh memfitnah orang lain, dan tidak boleh ketagihan minuman keras. Mereka harus mengajarkan hal-hal yang baik,
\par 4 supaya dengan itu mereka dapat mendidik wanita-wanita muda untuk mengasihi suami dan anak-anak,
\par 5 menjadi bijaksana, hidup suci dan menjadi ibu rumah tangga yang baik yang tunduk kepada suaminya. Dengan demikian tidak ada orang yang dapat mencela berita dari Allah.
\par 6 Begitu juga, nasihatilah orang-orang muda supaya mereka menjadi bijaksana.
\par 7 Dalam segala hal, hendaklah engkau menjadi teladan kelakuan yang baik. Kalau engkau mengajar, engkau harus jujur dan bersungguh-sungguh.
\par 8 Pakailah kata-kata yang bijaksana, yang tidak dapat dicela orang, supaya lawan-lawanmu menjadi malu karena tidak ada hal-hal buruk yang dapat mereka katakan tentang kita.
\par 9 Hamba-hamba harus tunduk kepada tuannya, dan menyenangkan hatinya dalam segala hal. Mereka tidak boleh membantah,
\par 10 atau mencuri. Hendaklah mereka menunjukkan bahwa mereka selalu bersikap baik dan setia, supaya dengan sikap mereka itu orang memuji ajaran tentang Allah, Penyelamat kita.
\par 11 Sebab Allah sudah menunjukkan rahmat-Nya guna menyelamatkan seluruh umat manusia.
\par 12 Rahmat Allah itu mendidik kita supaya tidak lagi hidup berlawanan dengan kehendak Allah dan tidak menuruti keinginan duniawi. Kita dididik untuk hidup dalam dunia ini dengan tahu menahan diri, tulus dan setia kepada Allah.
\par 13 Sekarang kita sedang menantikan Hari yang kita harap-harapkan itu; pada Hari itu dunia akan melihat keagungan Yesus Kristus, yaitu Allah Mahabesar dan Raja Penyelamat kita.
\par 14 Ia sudah mengurbankan diri-Nya bagi kita untuk membebaskan kita dari segala kejahatan, dan menjadikan kita suatu umat yang bebas dari dosa dan yang menjadi milik-Nya saja, serta yang rajin berbuat baik.
\par 15 Ajarkanlah semuanya itu, dan nasihatilah serta tegurlah para pendengarmu dengan penuh wibawa. Janganlah membiarkan seorang pun memandang engkau rendah.

\chapter{3}

\par 1 Ingatkanlah anggota jemaat-jemaatmu supaya mereka tunduk kepada pemimpin-pemimpin dan penguasa negara, serta taat dan bersedia melakukan setiap hal yang baik.
\par 2 Katakan kepada mereka supaya jangan memfitnah atau bertengkar dengan siapapun juga, melainkan supaya bersikap ramah. Hendaklah mereka selalu bersikap lemah lembut terhadap semua orang.
\par 3 Sebab dahulu kita sendiri juga bodoh, sesat dan tidak taat. Kita diperhamba oleh bermacam-macam nafsu dan keinginan; kita hidup dengan niat-niat jahat dan iri hati, serta saling membenci.
\par 4 Tetapi Allah, Penyelamat kita menunjukkan kasih dan kebaikan hati-Nya kepada kita.
\par 5 Ia menyelamatkan kita, bukan karena kita sudah melakukan sesuatu yang baik, melainkan karena Ia sendiri mengasihani kita. Ia menyelamatkan kita melalui Roh Allah, yang memberikan kita kelahiran baru dan hidup baru dengan jalan membasuh kita.
\par 6 Allah mencurahkan Roh-Nya kepada kita dengan perantaraan Yesus Kristus, Raja Penyelamat kita,
\par 7 supaya oleh rahmat Yesus, kita berbaik kembali dengan Allah dan kita mendapat hidup sejati dan kekal yang kita harap-harapkan.
\par 8 Perkataan ini sungguh benar. Saya mau engkau menekankan hal-hal ini, supaya orang-orang yang percaya kepada Allah sungguh-sungguh berusaha untuk melakukan pekerjaan-pekerjaan yang baik dan berguna untuk semua orang.
\par 9 Jauhilah perdebatan-perdebatan yang tidak berguna, cerita-cerita asal-usul, pertengkaran dan perkelahian mengenai hukum agama. Semuanya itu tidak ada gunanya dan tidak ada untungnya.
\par 10 Orang yang menyebabkan perpecahan dalam jemaat, hendaklah engkau tegur satu dua kali; sesudah itu janganlah lagi bergaul dengan dia.
\par 11 Engkau tahu bahwa orang semacam itu bejat dan dosa-dosanya membuktikan bahwa ia bersalah.
\par 12 Segera sesudah saya mengutus Artemas atau Tikhikus kepadamu, berusahalah secepat mungkin untuk datang kepada saya di Nikopolis, sebab saya berniat tinggal di sana selama musim dingin.
\par 13 Berusahalah sungguh-sungguh untuk membantu Zenas, ahli hukum itu, dan Apolos, supaya mereka bisa berangkat dan tidak kekurangan apa-apa.
\par 14 Orang-orang kita harus belajar melakukan hal-hal yang baik supaya dapat memenuhi kebutuhan-kebutuhan yang sangat diperlukan; jangan sampai mereka hidup tidak berguna.
\par 15 Semua saudara yang ada bersama saya, mengirim salam kepadamu. Sampaikanlah salam kami kepada kawan-kawan kita yang seiman. Semoga Tuhan memberkati Saudara semua. Hormat kami, Paulus.


\end{document}